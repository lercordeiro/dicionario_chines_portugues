%%%
%%% 5画
%%%

\section*{5画}\addcontentsline{toc}{section}{5画}

\begin{entry}{且}{5}{⼀}
  \begin{phonetics}{且}{qie3}
    \definition*{s.}{sobrenome Qie}
    \definition{adv.}{apenas; por enquanto | por um longo tempo}
    \definition{conj.}{mesmo; até; até mesmo | ambos\dots e\dots}
  \end{phonetics}
\end{entry}

\begin{entry}{世代}{5,5}{⼀、⼈}
  \begin{phonetics}{世代}{shi4dai4}
    \definition{adv.}{por muitas gerações, eras}
    \definition{s.}{geração | era}
  \end{phonetics}
\end{entry}

\begin{entry}{世纪}{5,6}{⼀、⽷}
  \begin{phonetics}{世纪}{shi4ji4}[][HSK 3]
    \definition[个]{s.}{século}
  \end{phonetics}
\end{entry}

\begin{entry}{世界}{5,9}{⼀、⽥}
  \begin{phonetics}{世界}{shi4jie4}[][HSK 3]
    \definition[个]{s.}{mundo | a soma da natureza e da sociedade humana | o universo sem limites | situação social}
  \end{phonetics}
\end{entry}

\begin{entry}{世界杯}{5,9,8}{⼀、⽥、⽊}
  \begin{phonetics}{世界杯}{shi4jie4bei1}[][HSK 3]
    \definition*{s.}{Copa do Mundo}
  \end{phonetics}
\end{entry}

\begin{entry}{世锦赛}{5,13,14}{⼀、⾦、⾙}
  \begin{phonetics}{世锦赛}{shi4jin3sai4}
    \definition*{s.}{Campeonato Mundial}
  \end{phonetics}
\end{entry}

\begin{entry}{丘陵}{5,10}{⼀、⾩}
  \begin{phonetics}{丘陵}{qiu1ling2}
    \definition{s.}{colinas}
  \end{phonetics}
\end{entry}

\begin{entry}{业余}{5,7}{⼀、⼈}
  \begin{phonetics}{业余}{ye4yu2}[][HSK 4]
    \definition{adj.}{tempo livre; depois do expediente; fora do horário de trabalho | amador; não profissional}
  \end{phonetics}
\end{entry}

\begin{entry}{东}{5}{⼀}
  \begin{phonetics}{东}{dong1}[][HSK 1]
    \definition*{s.}{sobrenome Dong}
    \definition{s.}{leste}
  \end{phonetics}
\end{entry}

\begin{entry}{东方}{5,4}{⼀、⽅}
  \begin{phonetics}{东方}{dong1 fang1}[][HSK 2]
    \definition*{s.}{sobrenome Dongfang}
    \definition{s.}{leste | oriente}
  \end{phonetics}
\end{entry}

\begin{entry}{东方学院}{5,4,8,9}{⼀、⽅、⼦、⾩}
  \begin{phonetics}{东方学院}{dong1fang1 xue2yuan4}
    \definition*{s.}{Instituto Oriental}
  \end{phonetics}
\end{entry}

\begin{entry}{东北}{5,5}{⼀、⼔}
  \begin{phonetics}{东北}{dong1 bei3}[][HSK 2]
    \definition*{s.}{Nordeste da China | Manchúria}
    \definition{s.}{nordeste}
  \end{phonetics}
\end{entry}

\begin{entry}{东半球}{5,5,11}{⼀、⼗、⽟}
  \begin{phonetics}{东半球}{dong1ban4qiu2}
    \definition*{s.}{Hemisfério Oriental}
  \end{phonetics}
\end{entry}

\begin{entry}{东边}{5,5}{⼀、⾡}
  \begin{phonetics}{东边}{dong1 bian5}[][HSK 1]
    \definition{s.}{este | leste | lado leste | oriente}
  \end{phonetics}
\end{entry}

\begin{entry}{东西}{5,6}{⼀、⾑}
  \begin{phonetics}{东西}{dong1xi1}
    \definition{s.}{leste e oeste}
  \end{phonetics}
  \begin{phonetics}{东西}{dong1xi5}[][HSK 1]
    \definition[个,件]{s.}{coisa | material | pessoa}
  \end{phonetics}
\end{entry}

\begin{entry}{东南}{5,9}{⼀、⼗}
  \begin{phonetics}{东南}{dong1 nan2}[][HSK 2]
    \definition{s.}{sudeste | sudeste da China | o Sudeste}
  \end{phonetics}
\end{entry}

\begin{entry}{东面}{5,9}{⼀、⾯}
  \begin{phonetics}{东面}{dong1mian4}
    \definition{s.}{lado leste (de algo)}
  \end{phonetics}
\end{entry}

\begin{entry}{东部}{5,10}{⼀、⾢}
  \begin{phonetics}{东部}{dong1 bu4}[][HSK 3]
    \definition{s.}{o leste; parte oriental}
  \end{phonetics}
\end{entry}

\begin{entry}{丝}{5}{⼀}
  \begin{phonetics}{丝}{si1}
    \definition{adj.}{filiforme | delgado como um fio | que se assemelha a um fio}
    \definition{clas.}{um traço (de fumaça, etc.) | um pouquinho, etc.}
    \definition{s.}{seda | (cozinha) pedaços ou tiras de julienne, tiras cortadas finas}
  \end{phonetics}
\end{entry}

\begin{entry}{主人}{5,2}{⼂、⼈}
  \begin{phonetics}{主人}{zhu3ren2}[][HSK 2]
    \definition[个,位]{s.}{mestre | anfitrião | proprietário | uma pessoa que tem um certo tipo de bens ou poder}
  \end{phonetics}
\end{entry}

\begin{entry}{主义}{5,3}{⼂、⼂}
  \begin{phonetics}{主义}{zhu3yi4}
    \definition{s.}{ideologia}
    \definition{suf.}{-ismo}
  \end{phonetics}
\end{entry}

\begin{entry}{主任}{5,6}{⼂、⼈}
  \begin{phonetics}{主任}{zhu3ren4}[][HSK 3]
    \definition[个,位]{s.}{chefe; diretor; presidente; o chefe de um departamento ou organização}
  \end{phonetics}
\end{entry}

\begin{entry}{主动}{5,6}{⼂、⼒}
  \begin{phonetics}{主动}{zhu3dong4}[][HSK 3]
    \definition{adj.}{ativo; positivo; agir sem ser pressionado por forças externas (em vez de ser ``被动'') | iniciativo; capaz de impulsionar as coisas por vontade própria; capaz de criar uma situação favorável e fazer as coisas acontecerem de acordo com suas próprias intenções (em oposição a ``被动'')}
  \seealsoref{被动}{bei4dong4}
  \end{phonetics}
\end{entry}

\begin{entry}{主张}{5,7}{⼂、⼸}
  \begin{phonetics}{主张}{zhu3zhang1}[][HSK 3]
    \definition[个]{s.}{vista; posição; proposição}
    \definition{v.}{segurar; advogar; manter; defender}
  \end{phonetics}
\end{entry}

\begin{entry}{主持}{5,9}{⼂、⼿}
  \begin{phonetics}{主持}{zhu3chi2}[][HSK 3]
    \definition{s.}{anfitrião; uma pessoa que é responsável ou lida com uma atividade}
    \definition{v.}{dirigir; gerir; tomar conta de | defender, manter}
  \end{phonetics}
\end{entry}

\begin{entry}{主要}{5,9}{⼂、⾑}
  \begin{phonetics}{主要}{zhu3yao4}[][HSK 2]
    \definition{adj.}{principal}
  \end{phonetics}
\end{entry}

\begin{entry}{主席}{5,10}{⼂、⼱}
  \begin{phonetics}{主席}{zhu3xi2}[][HSK 4]
    \definition*[个,位]{s.}{Presidente (da China)}
    \definition[个,位]{s.}{presidente, \emph{chairman}  (de uma reunião) |
chefe; presidente (de uma organização ou estado)}
  \end{phonetics}
\end{entry}

\begin{entry}{主席台}{5,10,5}{⼂、⼱、⼝}
  \begin{phonetics}{主席台}{zhu3xi2tai2}
    \definition[个]{s.}{plataforma | tribuna}
  \end{phonetics}
\end{entry}

\begin{entry}{主席团}{5,10,6}{⼂、⼱、⼞}
  \begin{phonetics}{主席团}{zhu3xi2tuan2}
    \definition{s.}{presídio}
  \end{phonetics}
\end{entry}

\begin{entry}{主意}{5,13}{⼂、⼼}
  \begin{phonetics}{主意}{zhu3yi5}[][HSK 3]
    \definition[个]{s.}{ideia; plano; decisão}
  \end{phonetics}
\end{entry}

\begin{entry}{主题}{5,15}{⼂、⾴}
  \begin{phonetics}{主题}{zhu3ti2}[][HSK 4]
    \definition[个]{s.}{tema; assunto; motivo; lema; ideias básicas expressas em toda a obra de literatura e arte por meio de imagens artísticas concretas | pontos/conteúdos principais; referência geral ao conteúdo principal de artigos, discursos, conferências, etc.}
  \end{phonetics}
\end{entry}

\begin{entry}{乐}{5}{⼃}
  \begin{phonetics}{乐}{le4}[][HSK 3]
    \definition*{s.}{sobrenome Le}
    \definition{adj.}{feliz; contente; rejubilante; animado; bem disposto}
    \definition{adv.}{alegremente; felizmente; desejosamente}
    \definition{s.}{prazer; diversão}
    \definition{v.}{desfrutar; ficar feliz em; amar; encontrar prazer em
rir; divertir-se}
  \end{phonetics}
  \begin{phonetics}{乐}{yue4}
    \definition*{s.}{sobrenome Yue}
    \definition{s.}{música}
  \end{phonetics}
\end{entry}

\begin{entry}{乐队}{5,4}{⼃、⾩}
  \begin{phonetics}{乐队}{yue4 dui4}[][HSK 3]
    \definition[支]{s.}{orquestra; banda}
  \end{phonetics}
\end{entry}

\begin{entry}{乐观}{5,6}{⼃、⾒}
  \begin{phonetics}{乐观}{le4guan1}[][HSK 3]
    \definition{adj.}{esperançoso; otimista; de um vermelho intenso}
  \end{phonetics}
\end{entry}

\begin{entry}{乐园}{5,7}{⼃、⼞}
  \begin{phonetics}{乐园}{le4yuan2}
    \definition{s.}{paraíso}
  \end{phonetics}
\end{entry}

\begin{entry}{乐高}{5,10}{⼃、⾼}
  \begin{phonetics}{乐高}{le4gao1}
    \definition*{s.}{Lego (brinquedo)}
  \end{phonetics}
\end{entry}

\begin{entry}{乐趣}{5,15}{⼃、⾛}
  \begin{phonetics}{乐趣}{le4qu4}[][HSK 4]
    \definition[个,种,些,点]{s.}{alegria; deleite; prazer; implicação de fazer alguém se sentir feliz; um humor de preferência}
  \end{phonetics}
\end{entry}

\begin{entry}{他}{5}{⼈}
  \begin{phonetics}{他}{ta1}[][HSK 1]
    \definition{pron.}{ele | se, o, lhe | si, consigo, ele}
    \seeref{怹}{tan1}
  \end{phonetics}
\end{entry}

\begin{entry}{他们}{5,5}{⼈、⼈}
  \begin{phonetics}{他们}{ta1men5}[][HSK 1]
    \definition{pron.}{eles | se, os, lhes | si, consigo, eles}
  \end{phonetics}
\end{entry}

\begin{entry}{他们的}{5,5,8}{⼈、⼈、⽩}
  \begin{phonetics}{他们的}{ta1men5 de5}
    \definition{pron.}{deles}
  \end{phonetics}
\end{entry}

\begin{entry}{他妈的}{5,6,8}{⼈、⼥、⽩}
  \begin{phonetics}{他妈的}{ta1ma1de5}
    \definition{interj.}{Dane-se! | Foda-se!}
  \end{phonetics}
\end{entry}

\begin{entry}{他的}{5,8}{⼈、⽩}
  \begin{phonetics}{他的}{ta1 de5}
    \definition{pron.}{dele}
  \end{phonetics}
\end{entry}

\begin{entry}{付}{5}{⼈}
  \begin{phonetics}{付}{fu4}[][HSK 3]
    \definition*{s.}{sobrenome Fu}
    \definition{clas.}{para pares ou conjuntos de coisas | para expressões faciais}
    \definition{v.}{comprometer-se a; entregar (entregar) a; entregar | pagar}
  \end{phonetics}
\end{entry}

\begin{entry}{付出}{5,5}{⼈、⼐}
  \begin{phonetics}{付出}{fu4 chu1}[][HSK 4]
    \definition{v.}{pagar; gastar; entregar (dinheiro, consideração, etc.)}
  \end{phonetics}
\end{entry}

\begin{entry}{付款}{5,12}{⼈、⽋}
  \begin{phonetics}{付款}{fu4kuan3}
    \definition{s.}{pagamento}
    \definition{v.+compl.}{pagar uma quantia em dinheiro}
  \end{phonetics}
\end{entry}

\begin{entry}{仙}{5}{⼈}
  \begin{phonetics}{仙}{xian1}
    \definition{s.}{imortal}
  \end{phonetics}
\end{entry}

\begin{entry}{代}{5}{⼈}
  \begin{phonetics}{代}{dai4}[][HSK 3]
    \definition*{s.}{sobrenome Dai}
    \definition{s.}{período histórico | dinastia | geração | era}
    \definition{v.}{tomar o lugar de; estar no lugar de
agir em nome de; exercer}
  \end{phonetics}
\end{entry}

\begin{entry}{代价}{5,6}{⼈、⼈}
  \begin{phonetics}{代价}{dai4jia4}[][HSK 5]
    \definition[种,个]{s.}{preço; material, energia gasta ou sacrifícios feitos para atingir um objetivo | custo; preço; dinheiro pago para obter algo}
  \end{phonetics}
\end{entry}

\begin{entry}{代言}{5,7}{⼈、⾔}
  \begin{phonetics}{代言}{dai4yan2}
    \definition{v.}{ser um porta-voz | ser um embaixador (para uma marca) | endossar}
  \end{phonetics}
\end{entry}

\begin{entry}{代表}{5,8}{⼈、⾐}
  \begin{phonetics}{代表}{dai4biao3}[][HSK 3]
    \definition[位,个,名]{s.}{deputado; delegado; representante | representante oficial}
    \definition{v.}{representar; defender}
  \end{phonetics}
\end{entry}

\begin{entry}{代表团}{5,8,6}{⼈、⾐、⼞}
  \begin{phonetics}{代表团}{dai4 biao3 tuan2}[][HSK 3]
    \definition[个]{s.}{delegação; contingente}
  \end{phonetics}
\end{entry}

\begin{entry}{代称}{5,10}{⼈、⽲}
  \begin{phonetics}{代称}{dai4cheng1}
    \definition{s.}{nome alternativo | antonomásia}
    \definition{v.}{referir-se a algo ou alguém por outro nome}
  \end{phonetics}
\end{entry}

\begin{entry}{代理}{5,11}{⼈、⽟}
  \begin{phonetics}{代理}{dai4li3}[][HSK 5]
    \definition{v.}{agir em nome de alguém em uma posição de responsabilidade; substituir alguém | agir como procurador; agir como agente; ser encarregado pelas partes de realizar atividades e conduzir assuntos em seu nome dentro do escopo de sua autorização}
  \end{phonetics}
\end{entry}

\begin{entry}{代替}{5,12}{⼈、⽈}
  \begin{phonetics}{代替}{dai4ti4}[][HSK 4]
    \definition{v.}{substituir; substituir por; tomar o lugar de}
  \end{phonetics}
\end{entry}

\begin{entry}{令}{5}{⼈}
  \begin{phonetics}{令}{ling2}
    \definition*{s.}{sobrenome Ling | Antigo nome geográfico, na região atual de Linyi, província de Shanxi.}
  \end{phonetics}
  \begin{phonetics}{令}{ling3}
    \definition{clas.}{resma (de papel); unidade de medida de papel: 500 folhas inteiras de papel original produzidas mecanicamente equivalem a 1 resma}
  \end{phonetics}
  \begin{phonetics}{令}{ling4}[][HSK 5]
    \definition{adj.}{bom; excelente | termos de cortesia usados para se referir aos familiares e parentes da outra pessoa}
    \definition{s.}{ordem; decreto; comando; ordem emitida pela autoridade superior | um título oficial; administradores de certos departamentos governamentais na antiguidade | temporada; estação; clima e fenologia de uma determinada estação | poema-canção; letra curta}
    \definition{v.}{ordenar; comandar | fazer com que alguém; fazer com que; permitir que}
  \end{phonetics}
\end{entry}

\begin{entry}{令人}{5,2}{⼈、⼈}
  \begin{phonetics}{令人}{ling4ren2}
    \definition{v.}{causar alguém (a fazer alguma coisa) | fazer alguém ficar zangado, encantado, etc.}
  \end{phonetics}
\end{entry}

\begin{entry}{仪式}{5,6}{⼈、⼷}
  \begin{phonetics}{仪式}{yi2shi4}
    \definition{s.}{cerimônia}
  \end{phonetics}
\end{entry}

\begin{entry}{们}{5}{⼈}
  \begin{phonetics}{们}{men5}[][HSK 1]
    \definition{part.}{sufixo para plural de pronomes e substantivos referentes a indivíduos}
  \end{phonetics}
\end{entry}

\begin{entry}{兄弟}{5,7}{⼉、⼸}
  \begin{phonetics}{兄弟}{xiong1di4}[][HSK 4]
    \definition{adj.}{fraternal}
    \definition{pron.}{eu, me (termo de uso humilde por homens em discurso público)}
    \definition[个,对]{s.}{irmãos; irmão}
  \end{phonetics}
\end{entry}

\begin{entry}{兰花}{5,7}{⼋、⾋}
  \begin{phonetics}{兰花}{lan2hua1}
    \definition{s.}{orquídea}
  \end{phonetics}
\end{entry}

\begin{entry}{册}{5}{⼌}
  \begin{phonetics}{册}{ce4}[][HSK 5]
    \definition{clas.}{cópias}
    \definition{s.}{volume; livro}
  \end{phonetics}
\end{entry}

\begin{entry}{写}{5}{⼍}
  \begin{phonetics}{写}{xie3}[][HSK 1]
    \definition{v.}{escrever}
  \end{phonetics}
\end{entry}

\begin{entry}{写字}{5,6}{⼍、⼦}
  \begin{phonetics}{写字}{xie3zi4}
    \definition{v.}{escrever (à mão) | praticar caligrafia}
  \end{phonetics}
\end{entry}

\begin{entry}{写字匠}{5,6,6}{⼍、⼦、⼕}
  \begin{phonetics}{写字匠}{xie3zi4 jiang4}
    \definition{s.}{calígrafo}
  \end{phonetics}
\end{entry}

\begin{entry}{写作}{5,7}{⼍、⼈}
  \begin{phonetics}{写作}{xie3zuo4}[][HSK 3]
    \definition{s.}{escrita; redação; composição}
    \definition{v.}{escrever artigos; escrever livros, etc.; também se refere especificamente à criação de obras literárias}
  \end{phonetics}
\end{entry}

\begin{entry}{写真}{5,10}{⼍、⼗}
  \begin{phonetics}{写真}{xie3zhen1}
    \definition{s.}{retrato}
    \definition{v.}{descrever algo com precisão}
  \end{phonetics}
\end{entry}

\begin{entry}{写意}{5,13}{⼍、⼼}
  \begin{phonetics}{写意}{xie3yi4}
    \definition{s.}{estilo de pintura chinesa à mão livre, caracterizado por traços ousados em vez de detalhes precisos}
    \definition{v.}{sugerir (em vez de descrever em detalhes)}
  \end{phonetics}
  \begin{phonetics}{写意}{xie4yi4}
    \definition{adj.}{confortável | agradável | relaxado}
  \end{phonetics}
\end{entry}

\begin{entry}{写照}{5,13}{⼍、⽕}
  \begin{phonetics}{写照}{xie3zhao4}
    \definition{s.}{retrato}
  \end{phonetics}
\end{entry}

\begin{entry}{冬}{5}{⼎}
  \begin{phonetics}{冬}{dong1}
    \definition*{s.}{sobrenome Dong}
    \definition{s.}{inverno}
  \end{phonetics}
\end{entry}

\begin{entry}{冬天}{5,4}{⼎、⼤}
  \begin{phonetics}{冬天}{dong1 tian1}[][HSK 2]
    \definition{s.}{inverno}
  \end{phonetics}
\end{entry}

\begin{entry}{冬瓜}{5,5}{⼎、⽠}
  \begin{phonetics}{冬瓜}{dong1gua1}
    \definition{s.}{melão de inverno}
  \end{phonetics}
\end{entry}

\begin{entry}{冬季}{5,8}{⼎、⼦}
  \begin{phonetics}{冬季}{dong1 ji4}[][HSK 4]
    \definition[个]{s.}{inverno; o quarto trimestre do ano, habitualmente referido na China como o período de três meses entre o início do inverno e o início da primavera, e também referido como ``décimo, décimo primeiro e décimo segundo'' meses do calendário lunar}
  \end{phonetics}
\end{entry}

\begin{entry}{出}{5}{⼐}
  \begin{phonetics}{出}{chu1}[][HSK 1]
    \definition{clas.}{para dramas, peças, óperas, etc.}
    \definition{v.}{sair | ir para fora | vir para fora}
  \end{phonetics}
\end{entry}

\begin{entry}{出于}{5,3}{⼐、⼆}
  \begin{phonetics}{出于}{chu1 yu2}[][HSK 5]
    \definition{prep.}{de; fora de; por causa de; em função de; de um certo ponto de vista}
    \definition{v.}{iniciar a partir de; originar-se de; prosseguir a partir de}
  \end{phonetics}
\end{entry}

\begin{entry}{出口}{5,3}{⼐、⼝}
  \begin{phonetics}{出口}{chu1kou3}[][HSK 2,4]
    \definition[个]{s.}{saída; porta ou passagem para o exterior}
    \definition{v.+compl.}{falar; proferir; manifestar-se | exportar mercadorias do país ou da região para venda no exterior ou em outro lugar | deixar o porto (um navio)}
  \end{phonetics}
\end{entry}

\begin{entry}{出门}{5,3}{⼐、⾨}
  \begin{phonetics}{出门}{chu1 men2}[][HSK 2]
    \definition{v.+compl.}{sair | sair de casa | estar longe de casa | fazer uma viagem | casar}
  \end{phonetics}
\end{entry}

\begin{entry}{出击}{5,5}{⼐、⼐}
  \begin{phonetics}{出击}{chu1ji1}
    \definition{v.}{atacar}
  \end{phonetics}
\end{entry}

\begin{entry}{出去}{5,5}{⼐、⼛}
  \begin{phonetics}{出去}{chu1 qu4}[][HSK 1]
    \definition{v.}{sair | ir para fora (a partir da minha localização)}
  \end{phonetics}
\end{entry}

\begin{entry}{出发}{5,5}{⼐、⼜}
  \begin{phonetics}{出发}{chu1fa1}[][HSK 2]
    \definition{v.}{partir | começar (uma jornada)}
  \end{phonetics}
\end{entry}

\begin{entry}{出生}{5,5}{⼐、⽣}
  \begin{phonetics}{出生}{chu1sheng1}[][HSK 2]
    \definition{v.}{nascer}
  \end{phonetics}
\end{entry}

\begin{entry}{出汗}{5,6}{⼐、⽔}
  \begin{phonetics}{出汗}{chu1 han4}[][HSK 5]
    \definition{v.}{suar; transpirar}
  \end{phonetics}
\end{entry}

\begin{entry}{出色}{5,6}{⼐、⾊}
  \begin{phonetics}{出色}{chu1se4}[][HSK 4]
    \definition{adj.}{esplêndido; extraordinário; notável; excepcionalmente bom; acima da média}
  \end{phonetics}
\end{entry}

\begin{entry}{出行}{5,6}{⼐、⾏}
  \begin{phonetics}{出行}{chu1xing2}
    \definition{v.}{sair para algum lugar (viagem relativamente curta) | partir em uma viagem (viagem mais longa)}
  \end{phonetics}
\end{entry}

\begin{entry}{出来}{5,7}{⼐、⽊}
  \begin{phonetics}{出来}{chu1 lai2}[][HSK 1]
    \definition{v.}{sair | vir para fora (para a minha localização)}
  \end{phonetics}
\end{entry}

\begin{entry}{出国}{5,8}{⼐、⼞}
  \begin{phonetics}{出国}{chu1 guo2}[][HSK 2]
    \definition{v.+compl.}{ir para o exterior | deixar a terra natal}
  \end{phonetics}
\end{entry}

\begin{entry}{出版}{5,8}{⼐、⽚}
  \begin{phonetics}{出版}{chu1ban3}[][HSK 5]
    \definition{v.}{aparecer; publicar; sair; sair da imprensa}
  \end{phonetics}
\end{entry}

\begin{entry}{出版社}{5,8,7}{⼐、⽚、⽰}
  \begin{phonetics}{出版社}{chu1ban3she4}
    \definition{s.}{editora}
  \end{phonetics}
\end{entry}

\begin{entry}{出现}{5,8}{⼐、⾒}
  \begin{phonetics}{出现}{chu1xian4}[][HSK 2]
    \definition{v.}{aparecer | surgir | emergir | crescer}
  \end{phonetics}
\end{entry}

\begin{entry}{出差}{5,9}{⼐、⼯}
  \begin{phonetics}{出差}{chu1chai1}[][HSK 5]
    \definition{v.+compl.}{fazer uma viagem de negócios | assumir tarefas de curto prazo em transporte, construção, etc.}
  \end{phonetics}
\end{entry}

\begin{entry}{出院}{5,9}{⼐、⾩}
  \begin{phonetics}{出院}{chu1 yuan4}[][HSK 2]
    \definition{v.}{deixar o hospital | estar fora do hospital | ter alta do hospital}
  \end{phonetics}
\end{entry}

\begin{entry}{出席}{5,10}{⼐、⼱}
  \begin{phonetics}{出席}{chu1xi2}[][HSK 4]
    \definition{v.}{comparecer; estar presente; participar de reuniões com o direito de falar e votar; juntar-se a uma organização ou atividade}
  \end{phonetics}
\end{entry}

\begin{entry}{出租}{5,10}{⼐、⽲}
  \begin{phonetics}{出租}{chu1 zu1}[][HSK 2]
    \definition{v.}{alugar | arrendar}
  \end{phonetics}
\end{entry}

\begin{entry}{出租车}{5,10,4}{⼐、⽲、⾞}
  \begin{phonetics}{出租车}{chu1zu1che1}[][HSK 2]
    \definition{s.}{táxi}
  \seealsoref{出租汽车}{chu1zu1qi4che1}
  \end{phonetics}
\end{entry}

\begin{entry}{出租司机}{5,10,5,6}{⼐、⽲、⼝、⽊}
  \begin{phonetics}{出租司机}{chu1zu1si1ji1}
    \definition{s.}{motorista de táxi}
  \end{phonetics}
\end{entry}

\begin{entry}{出租汽车}{5,10,7,4}{⼐、⽲、⽔、⾞}
  \begin{phonetics}{出租汽车}{chu1zu1qi4che1}
    \definition[辆]{s.}{táxi}
  \seealsoref{出租车}{chu1zu1che1}
  \end{phonetics}
\end{entry}

\begin{entry}{出站}{5,10}{⼐、⽴}
  \begin{phonetics}{出站}{chu1 zhan4}
    \definition{s.}{saída da estação}
  \end{phonetics}
\end{entry}

\begin{entry}{出售}{5,11}{⼐、⼝}
  \begin{phonetics}{出售}{chu1 shou4}[][HSK 4]
    \definition{v.}{vender; oferecer para venda}
  \end{phonetics}
\end{entry}

\begin{entry}{功夫}{5,4}{⼒、⼤}
  \begin{phonetics}{功夫}{gong1fu5}[][HSK 3]
    \definition*{s.}{Gongfu (Kung Fu), arte marcial}
    \definition[番]{s.}{habilidade; feitura | luta acrobática; habilidade em artes marciais | esforço; tempo e energia}
  \end{phonetics}
\end{entry}

\begin{entry}{功臣}{5,6}{⼒、⾂}
  \begin{phonetics}{功臣}{gong1chen2}
    \definition{s.}{oficial meritório | pessoa que presta serviço excepcional, herói | (fig.) alguém que desempenha um papel vital}
  \end{phonetics}
\end{entry}

\begin{entry}{功能}{5,10}{⼒、⾁}
  \begin{phonetics}{功能}{gong1neng2}[][HSK 3]
    \definition[种,项]{s.}{função}
  \end{phonetics}
\end{entry}

\begin{entry}{功课}{5,10}{⼒、⾔}
  \begin{phonetics}{功课}{gong1 ke4}[][HSK 3]
    \definition[份,门]{s.}{trabalho escolar; dever de casa | tarefa; lições; lição escolar}
  \end{phonetics}
\end{entry}

\begin{entry}{加}{5}{⼒}
  \begin{phonetics}{加}{jia1}[][HSK 2]
    \definition*{s.}{Canadá, abreviação de~加拿大 | sobrenome Jia}
    \seeref{加拿大}{jia1na2da4}
  \end{phonetics}
\end{entry}

\begin{entry}{加入}{5,2}{⼒、⼊}
  \begin{phonetics}{加入}{jia1ru4}[][HSK 4]
    \definition{v.}{juntar-se; unir-se; aderir a; tornar-se um membro de uma organização, grupo | adicionar; colocar em}
  \end{phonetics}
\end{entry}

\begin{entry}{加上}{5,3}{⼒、⼀}
  \begin{phonetics}{加上}{jia1 shang4}[][HSK 5]
    \definition{conj.}{além disso; em adição}
    \definition{v.}{adicionar; acrescentar; dar; aumentar}
  \end{phonetics}
\end{entry}

\begin{entry}{加工}{5,3}{⼒、⼯}
  \begin{phonetics}{加工}{jia1gong1}[][HSK 3]
    \definition{s.}{processo | trabalho (de uma máquina)}
    \definition{v.}{processar | melhorar; polir}
  \end{phonetics}
\end{entry}

\begin{entry}{加以}{5,4}{⼒、⼈}
  \begin{phonetics}{加以}{jia1 yi3}[][HSK 5]
    \definition{conj.}{além disso; em adição; indica outras razões ou condições}
    \definition{v.}{usado na frente de palavras dissilábicas para indicar como um objeto mencionado deve ser tratado ou descartado | usado antes de um verbo polifônico ou de um substantivo formado a partir de um verbo para indicar como tratar ou lidar com o que foi mencionado anteriormente}
  \end{phonetics}
\end{entry}

\begin{entry}{加快}{5,7}{⼒、⼼}
  \begin{phonetics}{加快}{jia1 kuai4}[][HSK 3]
    \definition{v.}{acelerar; aumentar a velocidade}
  \end{phonetics}
\end{entry}

\begin{entry}{加油}{5,8}{⼒、⽔}
  \begin{phonetics}{加油}{jia1you2}[][HSK 2]
    \definition{v.+compl.}{lubrificar | encher o tanque de combustível | fazer um esforço maior | fazer um esforço extra}
  \end{phonetics}
\end{entry}

\begin{entry}{加油站}{5,8,10}{⼒、⽔、⽴}
  \begin{phonetics}{加油站}{jia1you2zhan4}[][HSK 4]
    \definition[个,家]{s.}{posto de gasolina; posto de combustível; postos de abastecimento para venda a varejo de gasolina e óleo para carros e outros veículos motorizados}
  \end{phonetics}
\end{entry}

\begin{entry}{加拿大}{5,10,3}{⼒、⼿、⼤}
  \begin{phonetics}{加拿大}{jia1na2da4}
    \definition{s.}{Canadá}
  \end{phonetics}
\end{entry}

\begin{entry}{加拿大人}{5,10,3,2}{⼒、⼿、⼤、⼈}
  \begin{phonetics}{加拿大人}{jia1na2da4ren2}
    \definition{s.}{canadense | pessoa ou povo do Canadá}
  \end{phonetics}
\end{entry}

\begin{entry}{加热}{5,10}{⼒、⽕}
  \begin{phonetics}{加热}{jia1 re4}[][HSK 5]
    \definition{v.}{aquecer; esquentar; aumentar a temperatura de um objeto}
  \end{phonetics}
\end{entry}

\begin{entry}{加班}{5,10}{⼒、⽟}
  \begin{phonetics}{加班}{jia1ban1}[][HSK 4]
    \definition{v.+compl.}{fazer horas extras; trabalhar horas extras}
  \end{phonetics}
\end{entry}

\begin{entry}{加速}{5,10}{⼒、⾡}
  \begin{phonetics}{加速}{jia1 su4}[][HSK 5]
    \definition{v.}{acelerar; agilizar}
  \end{phonetics}
\end{entry}

\begin{entry}{加速度}{5,10,9}{⼒、⾡、⼴}
  \begin{phonetics}{加速度}{jia1su4du4}
    \definition{s.}{aceleração}
  \end{phonetics}
\end{entry}

\begin{entry}{加强}{5,12}{⼒、⼸}
  \begin{phonetics}{加强}{jia1 qiang2}[][HSK 3]
    \definition{v.}{fortalecer; engrandecer; reforçar}
  \end{phonetics}
\end{entry}

\begin{entry}{务实}{5,8}{⼒、⼧}
  \begin{phonetics}{务实}{wu4shi2}
    \definition{adj.}{pragmático}
    \definition{v.}{lidar com assuntos concretos}
  \end{phonetics}
\end{entry}

\begin{entry}{包}{5}{⼓}
  \begin{phonetics}{包}{bao1}[][HSK 1]
    \definition*{s.}{sobrenome Bao}
    \definition{clas.}{pacotes, sacos, sacolas, embrulhos}
    \definition[个,只]{s.}{bolsa | pacote | recipiente | embrulho}
    \definition{v.}{contratar | cobrir | segurar ou abraçar | incluir | assumir o comando | embrulhar}
  \end{phonetics}
\end{entry}

\begin{entry}{包子}{5,3}{⼓、⼦}
  \begin{phonetics}{包子}{bao1 zi5}[][HSK 1]
    \definition[个]{s.}{pão recheado cozido no vapor}
  \end{phonetics}
\end{entry}

\begin{entry}{包干}{5,3}{⼓、⼲}
  \begin{phonetics}{包干}{bao1gan1}
    \definition{s.}{tarefa alocada}
    \definition{v.}{ter a responsabilidade total sobre um trabalho}
  \end{phonetics}
\end{entry}

\begin{entry}{包办}{5,4}{⼓、⼒}
  \begin{phonetics}{包办}{bao1ban4}
    \definition{v.}{comandar todo o show | comprometer-se a fazer tudo sozinho}
  \end{phonetics}
\end{entry}

\begin{entry}{包含}{5,7}{⼓、⼝}
  \begin{phonetics}{包含}{bao1han2}[][HSK 4]
    \definition{v.}{conter; implicar; incluir; conter dentro, resumir, enfatizar o que está contido dentro, focar em relações internas, muitas vezes coisas abstratas}
  \end{phonetics}
\end{entry}

\begin{entry}{包围}{5,7}{⼓、⼞}
  \begin{phonetics}{包围}{bao1wei2}[][HSK 5]
    \definition{v.}{circundar; cercar; rodear}
  \end{phonetics}
\end{entry}

\begin{entry}{包括}{5,9}{⼓、⼿}
  \begin{phonetics}{包括}{bao1kuo4}[][HSK 4]
    \definition{v.}{incluir; compreender; consistir em; conter, conter dentro, resumir junto, enfatizar a listagem de todas as partes, ou a citação de uma parte delas, que podem ser coisas abstratas ou concretas}
  \end{phonetics}
\end{entry}

\begin{entry}{包容}{5,10}{⼓、⼧}
  \begin{phonetics}{包容}{bao1rong2}
    \definition{adj.}{inclusivo}
    \definition{v.}{perdoar | mostrar tolerância | conter | segurar}
  \end{phonetics}
\end{entry}

\begin{entry}{包租}{5,10}{⼓、⽲}
  \begin{phonetics}{包租}{bao1zu1}
    \definition{s.}{aluguel fixo para terras agrícolas}
    \definition{v.}{fretar | alugar | alugar um terreno ou uma casa para subarrendar}
  \end{phonetics}
\end{entry}

\begin{entry}{包装}{5,12}{⼓、⾐}
  \begin{phonetics}{包装}{bao1zhuang1}[][HSK 5]
    \definition[个,款]{s.}{embalagem; materiais usados para embalar produtos, como papel, sacolas, garrafas ou caixas}
    \definition{v.}{embalar; embrulhar; empacotar |
aumentar a fama e o apelo de alguém ou algo por meio de publicidade | tornar alguém ou algo mais comercialmente viável ou atraente por meio de embelezamento ou publicidade}
  \end{phonetics}
\end{entry}

\begin{entry}{包裹}{5,14}{⼓、⾐}
  \begin{phonetics}{包裹}{bao1guo3}[][HSK 4]
    \definition[个]{s.}{pacote; embrulho}
    \definition{v.}{embrulhar; amarrar; enrolar coisas em pano ou outra coisa}
  \end{phonetics}
\end{entry}

\begin{entry}{匆匆}{5,5}{⼓、⼓}
  \begin{phonetics}{匆匆}{cong1cong1}
    \definition{adv.}{apressadamente}
  \end{phonetics}
\end{entry}

\begin{entry}{北}{5}{⼔}
  \begin{phonetics}{北}{bei3}[][HSK 1]
    \definition{s.}{norte}
    \definition{v.}{(clássico) ser derrotado}
  \end{phonetics}
\end{entry}

\begin{entry}{北大西洋公约组织}{5,3,6,9,4,6,8,8}{⼔、⼤、⾑、⽔、⼋、⽷、⽷、⽷}
  \begin{phonetics}{北大西洋公约组织}{bei3 da4xi1 yang2 gong1 yue1 zu3zhi1}
    \definition*{s.}{Organização do Tratado do Atlântico Norte, OTAN}
  \end{phonetics}
\end{entry}

\begin{entry}{北方}{5,4}{⼔、⽅}
  \begin{phonetics}{北方}{bei3fang1}[][HSK 2]
    \definition{s.}{norte | a parte norte de um país}
  \end{phonetics}
\end{entry}

\begin{entry}{北边}{5,5}{⼔、⾡}
  \begin{phonetics}{北边}{bei3 bian1}[][HSK 1]
    \definition{adv.}{lado norte | ao norte de}
  \end{phonetics}
\end{entry}

\begin{entry}{北约}{5,6}{⼔、⽷}
  \begin{phonetics}{北约}{bei3yue1}
    \definition*{s.}{OTAN (Organização do Tratado do Atlântico Norte), abreviação de 北大西洋公约组织}
    \seeref{北大西洋公约组织}{bei3 da4xi1 yang2 gong1 yue1 zu3zhi1}
  \end{phonetics}
\end{entry}

\begin{entry}{北极}{5,7}{⼔、⽊}
  \begin{phonetics}{北极}{bei3ji2}[][HSK 5]
    \definition*{s.}{Pólo Norte; Pólo Ártico}
    \definition{s.}{pólo norte magnético; o ponto mais setentrional da Terra, também se refere à região mais setentrional da Terra}
  \end{phonetics}
\end{entry}

\begin{entry}{北京}{5,8}{⼔、⼇}
  \begin{phonetics}{北京}{bei3 jing1}[][HSK 1]
    \definition*{s.}{Beijing (Pequim), Capital da República Popular da China | Beijing (Pequim), governo da RPC}
  \end{phonetics}
\end{entry}

\begin{entry}{北面}{5,9}{⼔、⾯}
  \begin{phonetics}{北面}{bei3mian4}
    \definition{s.}{lado norte}
  \end{phonetics}
\end{entry}

\begin{entry}{北部}{5,10}{⼔、⾢}
  \begin{phonetics}{北部}{bei3 bu4}[][HSK 3]
    \definition{s.}{parte norte}
  \end{phonetics}
\end{entry}

\begin{entry}{半}{5}{⼗}
  \begin{phonetics}{半}{ban4}[][HSK 1]
    \definition{adj.}{incompleto}
    \definition{num.}{(depois de um número) ``e meio''}
    \definition{pref.}{semi-}
    \definition{s.}{metade}
  \end{phonetics}
\end{entry}

\begin{entry}{半天}{5,4}{⼗、⼤}
  \begin{phonetics}{半天}{ban4 tian1}[][HSK 1]
    \definition{s.}{metade do dia | muito tempo | bastante tempo}
  \end{phonetics}
\end{entry}

\begin{entry}{半年}{5,6}{⼗、⼲}
  \begin{phonetics}{半年}{ban4 nian2}[][HSK 1]
    \definition{s.}{meio ano}
  \end{phonetics}
\end{entry}

\begin{entry}{半夜}{5,8}{⼗、⼣}
  \begin{phonetics}{半夜}{ban4 ye4}[][HSK 2]
    \definition{adv.}{no meio da noite | metade de uma noite}
    \definition{s.}{meia-noite}
  \end{phonetics}
\end{entry}

\begin{entry}{半音}{5,9}{⼗、⾳}
  \begin{phonetics}{半音}{ban4yin1}
    \definition{s.}{semitom}
  \end{phonetics}
\end{entry}

\begin{entry}{半球}{5,11}{⼗、⽟}
  \begin{phonetics}{半球}{ban4qiu2}
    \definition{s.}{hemisfério}
  \end{phonetics}
\end{entry}

\begin{entry}{占}{5}{⼘}
  \begin{phonetics}{占}{zhan1}
    \definition*{s.}{sobrenome Zhan}
    \definition{v.}{praticar adivinhação | advinhar}
  \end{phonetics}
  \begin{phonetics}{占}{zhan4}[][HSK 2]
    \definition{v.}{ocupar | apreender | tomar | constituir | manter | compor | dar conta de}
  \end{phonetics}
\end{entry}

\begin{entry}{卡}{5}{⼘}
  \begin{phonetics}{卡}{ka3}[][HSK 2]
    \definition{clas.}{para calorias}
    \definition{s.}{cartão}
    \definition{v.}{bloquear | verificar | agarrar}
  \end{phonetics}
  \begin{phonetics}{卡}{qia3}
    \definition[张]{s.}{grampo | prendedor}
    \definition{s.}{posto de controle}
    \definition{v.}{cunhar | ficar preso | encravar}
  \end{phonetics}
\end{entry}

\begin{entry}{卡片}{5,4}{⼘、⽚}
  \begin{phonetics}{卡片}{ka3pian4}
    \definition{s.}{cartão}
  \end{phonetics}
\end{entry}

\begin{entry}{卡片游戏}{5,4,12,6}{⼘、⽚、⽔、⼽}
  \begin{phonetics}{卡片游戏}{ka3pian4 you2xi4}
    \definition{s.}{carta de baralho}
  \end{phonetics}
\end{entry}

\begin{entry}{卡车司机}{5,4,5,6}{⼘、⾞、⼝、⽊}
  \begin{phonetics}{卡车司机}{ka3che1 si1ji1}
    \definition{s.}{motorista de caminhão}
  \end{phonetics}
\end{entry}

\begin{entry}{卡通}{5,10}{⼘、⾡}
  \begin{phonetics}{卡通}{ka3tong1}
    \definition{s.}{(empréstimo linguístico) \emph{cartoon}}
  \end{phonetics}
\end{entry}

\begin{entry}{卢旺达}{5,8,6}{⼘、⽇、⾡}
  \begin{phonetics}{卢旺达}{lu2wang4da2}
    \definition*{s.}{Ruanda}
  \end{phonetics}
\end{entry}

\begin{entry}{印象}{5,11}{⼙、⾗}
  \begin{phonetics}{印象}{yin4xiang4}[][HSK 3]
    \definition[种]{s.}{impressão; marca; ideia; os vestígios deixados por coisas objetivas na mente das pessoas}
  \end{phonetics}
\end{entry}

\begin{entry}{厉害}{5,10}{⼚、⼧}
  \begin{phonetics}{厉害}{li4hai5}[][HSK 5]
    \definition{adj.}{feroz; severo; descreve uma situação como sendo muito grave | severo; duro; descreve uma pessoa que é exigente com os outros, muito severa, muitas vezes deixando os outros um pouco assustados | incrível; talentoso; impressionante; usado para avaliar a capacidade de uma pessoa ou algo que ela fez que é notável | aterrorizante; assustador; descreve animais ferozes e assustadores.}
  \end{phonetics}
\end{entry}

\begin{entry}{厺}{5}{⼤}
  \begin{phonetics}{厺}{qu4}
    \variantof{去}
  \end{phonetics}
\end{entry}

\begin{entry}{去}{5}{⼛}
  \begin{phonetics}{去}{qu4}[][HSK 1]
    \definition{v.}{ir | (eufenismo) morrer}
  \end{phonetics}
\end{entry}

\begin{entry}{去世}{5,5}{⼛、⼀}
  \begin{phonetics}{去世}{qu4shi4}[][HSK 3]
    \definition{v.}{morrer; falecer (um adulto)}
  \end{phonetics}
\end{entry}

\begin{entry}{去年}{5,6}{⼛、⼲}
  \begin{phonetics}{去年}{qu4nian2}[][HSK 1]
    \definition{s.}{ano passado}
  \end{phonetics}
\end{entry}

\begin{entry}{去死}{5,6}{⼛、⽍}
  \begin{phonetics}{去死}{qu4si3}
    \definition{interj.}{Caia morto! | Vá para o Inferno!}
  \end{phonetics}
\end{entry}

\begin{entry}{发}{5}{⼜}
  \begin{phonetics}{发}{fa1}[][HSK 2]
    \definition{clas.}{para tiros (rodadas)}
    \definition{v.}{enviar | mandar}
  \end{phonetics}
  \begin{phonetics}{发}{fa4}
    \definition{s.}{cabelo}
  \end{phonetics}
\end{entry}

\begin{entry}{发出}{5,5}{⼜、⼐}
  \begin{phonetics}{发出}{fa1 chu1}[][HSK 3]
    \definition{v.}{fazer; produzir; deixar sair | emitir; anunciar | enviar; partir | dar; emitir}
  \end{phonetics}
\end{entry}

\begin{entry}{发布}{5,5}{⼜、⼱}
  \begin{phonetics}{发布}{fa1bu4}[][HSK 5]
    \definition{v.}{emitir; publicar; liberar; anunciar; fazer ordens públicas, anúncios, notícias, etc.}
  \end{phonetics}
\end{entry}

\begin{entry}{发生}{5,5}{⼜、⽣}
  \begin{phonetics}{发生}{fa1sheng1}[][HSK 3]
    \definition{v.}{ocorrer; acontecer; tomar lugar}
  \end{phonetics}
\end{entry}

\begin{entry}{发动}{5,6}{⼜、⼒}
  \begin{phonetics}{发动}{fa1dong4}[][HSK 3]
    \definition{v.}{iniciar; lançar; ligar motor; dar a partida (motor de combustão interna) | chamar à ação; mobilizar; estimular; despertar}
  \end{phonetics}
\end{entry}

\begin{entry}{发动机}{5,6,6}{⼜、⼒、⽊}
  \begin{phonetics}{发动机}{fa1dong4ji1}
    \definition[台]{s.}{motor}
  \end{phonetics}
\end{entry}

\begin{entry}{发行}{5,6}{⼜、⾏}
  \begin{phonetics}{发行}{fa1xing2}[][HSK 5]
    \definition{v.}{emitir; liberar; publicar; emitir ou vender de publicações recém-impressas, moeda, selos, etc.}
  \end{phonetics}
\end{entry}

\begin{entry}{发达}{5,6}{⼜、⾡}
  \begin{phonetics}{发达}{fa1da2}[][HSK 3]
    \definition{adj.}{desenvolvido; florescente}
    \definition{v.}{desenvolver; promover; florescer}
  \end{phonetics}
\end{entry}

\begin{entry}{发抖}{5,7}{⼜、⼿}
  \begin{phonetics}{发抖}{fa1dou3}
    \definition{v.}{tremer | sacudir | estremecer}
  \end{phonetics}
\end{entry}

\begin{entry}{发言}{5,7}{⼜、⾔}
  \begin{phonetics}{发言}{fa1yan2}[][HSK 3]
    \definition[个]{s.}{discurso; declaração; palestra}
    \definition{v.+compl.}{falar; fazer uma declaração (discurso)}
  \end{phonetics}
\end{entry}

\begin{entry}{发财}{5,7}{⼜、⾙}
  \begin{phonetics}{发财}{fa1cai2}
    \definition{v.+compl.}{ficar rico | fazer fortuna}
  \end{phonetics}
\end{entry}

\begin{entry}{发明}{5,8}{⼜、⽇}
  \begin{phonetics}{发明}{fa1ming2}[][HSK 3]
    \definition[个]{s.}{invenção}
    \definition{v.}{inventar | expor; explicar}
  \end{phonetics}
\end{entry}

\begin{entry}{发明者}{5,8,8}{⼜、⽇、⽼}
  \begin{phonetics}{发明者}{fa1ming2zhe3}
    \definition{s.}{inventor}
  \end{phonetics}
\end{entry}

\begin{entry}{发现}{5,8}{⼜、⾒}
  \begin{phonetics}{发现}{fa1xian4}[][HSK 2]
    \definition{s.}{descoberta}
    \definition{v.}{perceber, tornar-se ciente de | descobrir, encontrar, detectar}
  \end{phonetics}
\end{entry}

\begin{entry}{发现者}{5,8,8}{⼜、⾒、⽼}
  \begin{phonetics}{发现者}{fa1xian4 zhe3}
    \definition{s.}{descobridor}
  \end{phonetics}
\end{entry}

\begin{entry}{发表}{5,8}{⼜、⾐}
  \begin{phonetics}{发表}{fa1biao3}[][HSK 3]
    \definition{v.}{publicar; entregar; emitir; expressar; anunciar | publicar}
  \end{phonetics}
\end{entry}

\begin{entry}{发型}{5,9}{⼜、⼟}
  \begin{phonetics}{发型}{fa4xing2}
    \definition{s.}{penteado}
  \end{phonetics}
\end{entry}

\begin{entry}{发挥}{5,9}{⼜、⼿}
  \begin{phonetics}{发挥}{fa1hui1}[][HSK 4]
    \definition{v.}{colocar em jogo; dar jogo a; dar espaço a; dar rédea solta a; revelar a natureza ou a capacidade interior | expressar; desenvolver (uma ideia, um tema, etc.); elaborar; fazer valer o ponto ou o motivo}
  \end{phonetics}
\end{entry}

\begin{entry}{发觉}{5,9}{⼜、⾒}
  \begin{phonetics}{发觉}{fa1jue2}[][HSK 5]
    \definition{v.}{vir a saber; estar ciente (de); perceber; tornar-se consciente | encontrar; detectar; perceber; descobrir}
  \end{phonetics}
\end{entry}

\begin{entry}{发送}{5,9}{⼜、⾡}
  \begin{phonetics}{发送}{fa1 song4}[][HSK 3]
    \definition{v.}{enviar; despachar | transmitir; enviar}
  \end{phonetics}
\end{entry}

\begin{entry}{发音}{5,9}{⼜、⾳}
  \begin{phonetics}{发音}{fa1yin1}
    \definition{s.}{pronúncia}
    \definition{v.}{pronunciar}
  \end{phonetics}
\end{entry}

\begin{entry}{发射}{5,10}{⼜、⼨}
  \begin{phonetics}{发射}{fa1she4}[][HSK 5]
    \definition{v.}{subir; disparar; lançar; irradiar; projetar; descarregar; enviar algo (como uma bala, um projétil, um satélite, etc.) de um dispositivo em uma velocidade muito alta}
  \end{phonetics}
\end{entry}

\begin{entry}{发展}{5,10}{⼜、⼫}
  \begin{phonetics}{发展}{fa1zhan3}[][HSK 3]
    \definition{s.}{desenvolvimento}
    \definition{v.}{crescer; expandir; avançar; desenvolver | recrutar; expandir; admitir}
  \end{phonetics}
\end{entry}

\begin{entry}{发烧}{5,10}{⼜、⽕}
  \begin{phonetics}{发烧}{fa1shao1}[][HSK 4]
    \definition{v.}{ter febre; a temperatura corporal normal de uma pessoa é de cerca de 37ºC; se exceder 37,5ºC, é febre}
  \end{phonetics}
\end{entry}

\begin{entry}{发票}{5,11}{⼜、⽰}
  \begin{phonetics}{发票}{fa1piao4}[][HSK 4]
    \definition[张]{s.}{conta; recibo; fatura; recibos emitidos por lojas ou outros escritórios de cobrança}
  \end{phonetics}
\end{entry}

\begin{entry}{发愁}{5,13}{⼜、⼼}
  \begin{phonetics}{发愁}{fa1chou2}
    \definition{v.+compl.}{preocupar-se | ficar ansioso | ficar triste}
  \end{phonetics}
\end{entry}

\begin{entry}{发簪}{5,18}{⼜、⽵}
  \begin{phonetics}{发簪}{fa4zan1}
    \definition{s.}{grampo de cabelo}
  \end{phonetics}
\end{entry}

\begin{entry}{古}{5}{⼝}
  \begin{phonetics}{古}{gu3}[][HSK 3]
    \definition*{s.}{sobrenome Gu}
    \definition{adj.}{arcaico; antigo; antiquíssimo}
    \definition{pref.}{``paleo''; ``arqueo''}
    \definition{s.}{antiguidade; arcaísmo | livros ou ortodoxias de antigos sábios | uma forma de poesia pré-Tang}
  \end{phonetics}
\end{entry}

\begin{entry}{古人}{5,2}{⼝、⼈}
  \begin{phonetics}{古人}{gu3ren2}
    \definition{s.}{pessoas dos tempos antigos | os antigos | espécies humanas extintas, como \emph{Homo erectus} ou \emph{Homo neanderthalensis} | (literário) pessoa falecida}
  \end{phonetics}
\end{entry}

\begin{entry}{古代}{5,5}{⼝、⼈}
  \begin{phonetics}{古代}{gu3dai4}[][HSK 3]
    \definition{s.}{tempos antigos | sociedade antiga; sociedade primitiva | antigamente}
  \end{phonetics}
\end{entry}

\begin{entry}{古老}{5,6}{⼝、⽼}
  \begin{phonetics}{古老}{gu3 lao3}[][HSK 5]
    \definition{adj.}{antigo; antiquado; histórico}
  \end{phonetics}
\end{entry}

\begin{entry}{古城}{5,9}{⼝、⼟}
  \begin{phonetics}{古城}{gu3cheng2}
    \definition{s.}{cidade antiga}
  \end{phonetics}
\end{entry}

\begin{entry}{古铜色}{5,11,6}{⼝、⾦、⾊}
  \begin{phonetics}{古铜色}{gu3tong2 se4}
    \definition{s.}{cor bronze}
  \end{phonetics}
\end{entry}

\begin{entry}{句}{5}{⼝}
  \begin{phonetics}{句}{gou4}
    \variantof{勾}
  \end{phonetics}
  \begin{phonetics}{句}{ju4}[][HSK 2]
    \definition{clas.}{para orações, frases ou linhas de versos}
    \definition{s.}{sentença | cláusula | frase}
  \end{phonetics}
\end{entry}

\begin{entry}{句子}{5,3}{⼝、⼦}
  \begin{phonetics}{句子}{ju4zi5}[][HSK 2]
    \definition[个]{s.}{sentença | frase | oração}
  \end{phonetics}
\end{entry}

\begin{entry}{另一方面}{5,1,4,9}{⼝、⼀、⽅、⾯}
  \begin{phonetics}{另一方面}{ling4 yi4 fang1 mian4}[][HSK 3]
    \definition{adv./conj.}{outro aspecto | por outro lado; por sua vez; em contrapartida}
  \end{phonetics}
\end{entry}

\begin{entry}{另外}{5,5}{⼝、⼣}
  \begin{phonetics}{另外}{ling4wai4}[][HSK 3]
    \definition{adv.}{além disso; em adição; ademais; além do mais; além de que}
    \definition{pron.}{além disso}
  \end{phonetics}
\end{entry}

\begin{entry}{只}{5}{⼝}
  \begin{phonetics}{只}{zhi1}[][HSK 3]
    \definition*{s.}{sobrenome Zhi}
    \definition{adj.}{solteiro; solitário}
    \definition{clas.}{para um de um par | para animais pequenos (pássaros, gatos, cães, etc.) | para certos utensílios, aparelhos | para navios}
  \end{phonetics}
  \begin{phonetics}{只}{zhi3}[][HSK 2]
    \definition{adv.}{só; somente; apenas; simplesmente; meramente}
  \end{phonetics}
\end{entry}

\begin{entry}{只好}{5,6}{⼝、⼥}
  \begin{phonetics}{只好}{zhi3hao3}[][HSK 3]
    \definition{v.}{ter que; ser forçado a; não ter escolha a não ser}
  \end{phonetics}
\end{entry}

\begin{entry}{只有}{5,6}{⼝、⽉}
  \begin{phonetics}{只有}{zhi3 you3}[][HSK 3]
    \definition{adv.}{somente; tem que; forçado a}
    \definition{conj.}{somente se; conecta cláusulas, expressa condições necessárias, geralmente corresponde a ``才'' (apenas) e ``方'' (que significa apenas)}
  \seealsoref{才}{cai2}
  \seealsoref{方}{fang1}
  \end{phonetics}
\end{entry}

\begin{entry}{只有……才……}{5,6,3}{⼝、⽉、⼿}
  \begin{phonetics}{只有……才……}{zhi3you3 cai2}
    \definition{conj.}{só se\dots então\dots}
  \end{phonetics}
\end{entry}

\begin{entry}{只身}{5,7}{⼝、⾝}
  \begin{phonetics}{只身}{zhi1shen1}
    \definition{adv.}{sozinho | por si só}
  \end{phonetics}
\end{entry}

\begin{entry}{只怕}{5,8}{⼝、⼼}
  \begin{phonetics}{只怕}{zhi3pa4}
    \definition{adv.}{receio que\dots | talvez | muito provavelmente}
  \end{phonetics}
\end{entry}

\begin{entry}{只是}{5,9}{⼝、⽇}
  \begin{phonetics}{只是}{zhi3 shi4}[][HSK 3]
    \definition{adv.}{somente; meramente; apenas; mas; para enfatizar que é limitado a uma determinada situação ou escopo}
    \definition{conj.}{somente; mas; exceto que; orações de conexão, indicando uma ligeira transição, equivalente a ``不过''}
  \seealsoref{不过}{bu2guo4}
  \end{phonetics}
\end{entry}

\begin{entry}{只要}{5,9}{⼝、⾑}
  \begin{phonetics}{只要}{zhi3yao4}[][HSK 2]
    \definition{conj.}{se apenas | contanto que}
  \end{phonetics}
\end{entry}

\begin{entry}{只要……就……}{5,9,12}{⼝、⾑、⼪}
  \begin{phonetics}{只要……就……}{zhi3yao4 jiu4}
    \definition{conj.}{contanto que/desde que/se somente\dots, então\dots}
  \end{phonetics}
\end{entry}

\begin{entry}{只消}{5,10}{⼝、⽔}
  \begin{phonetics}{只消}{zhi3xiao1}
    \definition{conj.}{desde que}
  \end{phonetics}
\end{entry}

\begin{entry}{只能}{5,10}{⼝、⾁}
  \begin{phonetics}{只能}{zhi3 neng2}[][HSK 2]
    \definition{adv.}{só pode | obrigado a fazer algo}
  \end{phonetics}
\end{entry}

\begin{entry}{只读}{5,10}{⼝、⾔}
  \begin{phonetics}{只读}{zhi3du2}
    \definition{s.}{somente leitura (computação) | \emph{read-only}}
  \end{phonetics}
\end{entry}

\begin{entry}{只顾}{5,10}{⼝、⾴}
  \begin{phonetics}{只顾}{zhi3gu4}
    \definition{adv.}{exclusivamente preocupado (com uma coisa)}
    \definition{v.}{cuidar de apenas um aspecto}
  \end{phonetics}
\end{entry}

\begin{entry}{只得}{5,11}{⼝、⼻}
  \begin{phonetics}{只得}{zhi3de5}
    \definition{v.}{ser obrigado a | não ter outra alternativa senão}
  \end{phonetics}
\end{entry}

\begin{entry}{叫}{5}{⼝}
  \begin{phonetics}{叫}{jiao4}[][HSK 1,3]
    \definition{adj.}{macho (animal)}
    \definition{prep.}{apresenta a voz ativa na construção passiva.}
    \definition{v.}{chorar; gritar | chamar; cumprimentar | contratar; encomendar | nomear; chamar | pedir; licitar | anunciar-se}
  \end{phonetics}
\end{entry}

\begin{entry}{叫作}{5,7}{⼝、⼈}
  \begin{phonetics}{叫作}{jiao4 zuo4}[][HSK 2]
    \definition{v.}{ser chamado de | ser conhecido como}
  \end{phonetics}
\end{entry}

\begin{entry}{召开}{5,4}{⼝、⼶}
  \begin{phonetics}{召开}{zhao4kai1}[][HSK 4]
    \definition{v.}{convocar; chamar pessoas para uma reunião; realizar (uma reunião)}
  \end{phonetics}
\end{entry}

\begin{entry}{叮嘱}{5,15}{⼝、⼝}
  \begin{phonetics}{叮嘱}{ding1zhu3}
    \definition{v.}{exortar | avisar | insistir de novo e de novo}
  \end{phonetics}
\end{entry}

\begin{entry}{可}{5}{⼝}
  \begin{phonetics}{可}{ke3}[][HSK 5]
    \definition*{s.}{sobrenome Ke}
    \definition{adv.}{indica ênfase |
indica o fortalecimento de perguntas retóricas |
indica um tom de questionamento mais forte |
sobre; a respeito de;}
    \definition{conj.}{mas; ainda}
    \definition{v.}{aprovar; concordar com | poder; permitir; ser capaz de | precisar (fazer); valer a pena (fazer); merecer | ajustar; adequar | estar pronto para; estar disposto a; pretender}
  \end{phonetics}
  \begin{phonetics}{可}{ke4}
    \definition{s.}{governante supremo de uma tribo nômade do norte; Khan (可汗), título do governante supremo dos antigos grupos étnicos xianbei, turco, uigur e mongol}
  \seealsoref{可汗}{ke4han2}
  \end{phonetics}
\end{entry}

\begin{entry}{可口可乐}{5,3,5,5}{⼝、⼝、⼝、⼃}
  \begin{phonetics}{可口可乐}{ke3kou3ke3le4}
    \definition*{s.}{(empréstimo linguístico) Coca-Cola}
  \end{phonetics}
\end{entry}

\begin{entry}{可以}{5,4}{⼝、⼈}
  \begin{phonetics}{可以}{ke3yi3}[][HSK 2]
    \definition{v.}{ser capaz de | poder}
  \end{phonetics}
\end{entry}

\begin{entry}{可见}{5,4}{⼝、⾒}
  \begin{phonetics}{可见}{ke3jian4}[][HSK 4]
    \definition{adj.}{visível; concebível; algo que é óbvio ou evidente}
    \definition{conj.}{isso mostra; isto prova; é, portanto, claro (ou evidente, óbvio) que}
    \definition{v.}{ser ou estar visível ; ser ou estar claro}
  \end{phonetics}
\end{entry}

\begin{entry}{可乐}{5,5}{⼝、⼃}
  \begin{phonetics}{可乐}{ke3 le4}[][HSK 3]
    \definition*{s.}{\emph{coke}; coca; coca-cola}
  \end{phonetics}
\end{entry}

\begin{entry}{可卡因}{5,5,6}{⼝、⼘、⼞}
  \begin{phonetics}{可卡因}{ke3ka3yin1}
    \definition{s.}{(empréstimo linguístico) cocaína}
  \end{phonetics}
\end{entry}

\begin{entry}{可汗}{5,6}{⼝、⽔}
  \begin{phonetics}{可汗}{ke4han2}
    \definition{s.}{khan (empréstimo linguístico); cham}
  \end{phonetics}
\end{entry}

\begin{entry}{可怕}{5,8}{⼝、⼼}
  \begin{phonetics}{可怕}{ke3pa4}[][HSK 2]
    \definition{adj.}{horrível | terrível | formidável | assustador | hediondo}
    \definition{adv.}{terrivelmente}
  \end{phonetics}
\end{entry}

\begin{entry}{可怜}{5,8}{⼝、⼼}
  \begin{phonetics}{可怜}{ke3lian2}[][HSK 5]
    \definition{adj.}{pobre; lamentável; lastimável | miserável (de quantidade ou qualidade); descreve um número pequeno ou um lugar tão pequeno que não vale a pena falar sobre ele}
    \definition{v.}{ter pena; ter piedade de; ter simpatia por pessoas que tiveram coisas muito ruins acontecendo com elas}
  \end{phonetics}
\end{entry}

\begin{entry}{可是}{5,9}{⼝、⽇}
  \begin{phonetics}{可是}{ke3shi4}[][HSK 2]
    \definition{adv.}{(usado para dar ênfase) de fato}
    \definition{conj.}{porém | contudo | mas}
  \end{phonetics}
\end{entry}

\begin{entry}{可爱}{5,10}{⼝、⽖}
  \begin{phonetics}{可爱}{ke3'ai4}[][HSK 2]
    \definition{adj.}{adorável | querido | fofo}
  \end{phonetics}
\end{entry}

\begin{entry}{可能}{5,10}{⼝、⾁}
  \begin{phonetics}{可能}{ke3neng2}[][HSK 2]
    \definition{adj.}{possível | provável}
    \definition{adv.}{possivelmente | provavelmente}
    \definition[个]{s.}{possibilidade | probabilidade}
  \end{phonetics}
\end{entry}

\begin{entry}{可惜}{5,11}{⼝、⼼}
  \begin{phonetics}{可惜}{ke3xi1}[][HSK 5]
    \definition{adj.}{é uma pena; é muito ruim; é lamentável}
    \definition{adv.}{infelizmente}
  \end{phonetics}
\end{entry}

\begin{entry}{可编程}{5,12,12}{⼝、⽷、⽲}
  \begin{phonetics}{可编程}{ke3bian1cheng2}
    \definition{adj.}{programável}
  \end{phonetics}
\end{entry}

\begin{entry}{可靠}{5,15}{⼝、⾮}
  \begin{phonetics}{可靠}{ke3kao4}[][HSK 3]
    \definition{adj.}{confiável | verdadeiro; autêntico}
  \end{phonetics}
\end{entry}

\begin{entry*}{可擦写可编程只读存储器}{5,17,5,5,12,12,5,10,6,12,16}{⼝、⼿、⼍、⼝、⽷、⽲、⼝、⾔、⼦、⼈、⼝}
  \begin{phonetics}{可擦写可编程只读存储器}{ke3ca1xie3ke3bian1cheng2zhi1du2cun2chu3qi4}
    \definition{s.}{EPROM (\emph{erasable programmable read-only memory})}
  \end{phonetics}
\end{entry*}

\begin{entry}{台}{5}{⼝}
  \begin{phonetics}{台}{tai2}[][HSK 3]
    \definition*{s.}{sobrenome Tai}
    \definition{clas.}{para aparelhos e máquinas}
    \definition{s.}{torre | plataforma; palco | suporte; pedestal | qualquer coisa em forma de plataforma ou palco | mesa; escrivaninha | estação de transmissão | um serviço telefônico especial |termo de tratamento respeitoso (nos tempos antigos)}
  \end{phonetics}
\end{entry}

\begin{entry}{台上}{5,3}{⼝、⼀}
  \begin{phonetics}{台上}{tai2 shang4}[][HSK 4]
    \definition{s.}{no palco}
  \end{phonetics}
\end{entry}

\begin{entry}{台下}{5,3}{⼝、⼀}
  \begin{phonetics}{台下}{tai2xia4}
    \definition{s.}{platéia | fora do palco}
  \end{phonetics}
\end{entry}

\begin{entry}{台风}{5,4}{⼝、⾵}
  \begin{phonetics}{台风}{tai2feng1}[][HSK 5]
    \definition[场,阵,级]{s.}{tufão; classificação de um ciclone tropical ocorrido no oeste do Pacífico Norte | postura; presença de palco; comportamento ou estilo que os atores demonstram no palco}
  \end{phonetics}
\end{entry}

\begin{entry}{台阶}{5,6}{⼝、⾩}
  \begin{phonetics}{台阶}{tai2jie1}[][HSK 4]
    \definition[个,级]{s.}{escada; escadaria | passos; metáfora para uma maneira ou oportunidade de evitar constrangimentos causados ​​por um impasse | nova fase; novo nível; novo patamar; metáfora para novas conquistas ou novos patamares alcançados no estudo ou no trabalho}
  \end{phonetics}
\end{entry}

\begin{entry}{右}{5}{⼝}
  \begin{phonetics}{右}{you4}[][HSK 1]
    \definition{s.}{(política) a Direita}
    \definition{s.}{direita}
  \end{phonetics}
\end{entry}

\begin{entry}{右手}{5,4}{⼝、⼿}
  \begin{phonetics}{右手}{you4shou3}
    \definition{s.}{mão direita | lado direito}
  \end{phonetics}
\end{entry}

\begin{entry}{右边}{5,5}{⼝、⾡}
  \begin{phonetics}{右边}{you4bian5}[][HSK 1]
    \definition{adv.}{à direita | ao lado direito}
  \end{phonetics}
\end{entry}

\begin{entry}{右侧}{5,8}{⼝、⼈}
  \begin{phonetics}{右侧}{you4ce4}
    \definition{s.}{lateral direita | lado direito}
  \end{phonetics}
\end{entry}

\begin{entry}{右转}{5,8}{⼝、⾞}
  \begin{phonetics}{右转}{you4zhuan3}
    \definition{v.}{virar à direita}
  \end{phonetics}
\end{entry}

\begin{entry}{右面}{5,9}{⼝、⾯}
  \begin{phonetics}{右面}{you4mian4}
    \definition{s.}{lado direito}
  \end{phonetics}
\end{entry}

\begin{entry}{右倾}{5,10}{⼝、⼈}
  \begin{phonetics}{右倾}{you4qing1}
    \definition{adj.}{conservador | reacionário}
  \end{phonetics}
\end{entry}

\begin{entry}{右袒}{5,10}{⼝、⾐}
  \begin{phonetics}{右袒}{you4tan3}
    \definition{v.}{ser tendencioso | ser parcial | favorecer um lado | tomar partido}
  \end{phonetics}
\end{entry}

\begin{entry}{叶子}{5,3}{⼝、⼦}
  \begin{phonetics}{叶子}{ye4zi5}[][HSK 4]
    \definition[片]{s.}{folha; termo genérico para as folhas de uma planta}
  \end{phonetics}
\end{entry}

\begin{entry}{号}{5}{⼝}
  \begin{phonetics}{号}{hao2}
    \definition[个]{s.}{rugido | choro}
  \end{phonetics}
  \begin{phonetics}{号}{hao4}[][HSK 1]
    \definition{clas.}{para indicar o número de pessoas}
    \definition{num.}{dia do mês | usado para indicar o número de pessoas}
    \definition[个]{s.}{número ordinal | dia de um mês | marca | sinal | estabelecimento comercial | tamanho | buzina (instrumento de sopro) | toque de corneta | nome suposto}
    \definition{suf.}{sufixo de navio}
    \definition{v.}{tomar um pulso}
  \end{phonetics}
\end{entry}

\begin{entry}{号召}{5,5}{⼝、⼝}
  \begin{phonetics}{号召}{hao4zhao4}[][HSK 5]
    \definition{s.}{chamado; apelo; desejo ou pedido solene (de um governo, partido político, organização etc.) para que as massas façam algo}
    \definition{v.}{chamar;  (governo, partido político, organização, etc.) fazer um pedido solene às massas para que façam algo, na esperança de que todos se esforcem para alcançá-lo}
  \end{phonetics}
\end{entry}

\begin{entry}{号角}{5,7}{⼝、⾓}
  \begin{phonetics}{号角}{hao4jiao3}
    \definition{s.}{corneta | trombeta}
  \end{phonetics}
\end{entry}

\begin{entry}{号码}{5,8}{⼝、⽯}
  \begin{phonetics}{号码}{hao4ma3}[][HSK 4]
    \definition[个,组,串]{s.}{número}
  \end{phonetics}
\end{entry}

\begin{entry}{司机}{5,6}{⼝、⽊}
  \begin{phonetics}{司机}{si1ji1}[][HSK 2]
    \definition{s.}{condutor | motorista | chofer}
  \end{phonetics}
\end{entry}

\begin{entry}{囘}{5}{⼞}
  \begin{phonetics}{囘}{hui2}
    \variantof{回}
  \end{phonetics}
\end{entry}

\begin{entry}{四}{5}{⼞}
  \begin{phonetics}{四}{si4}[][HSK 1]
    \definition{num.}{quatro; 4}
  \end{phonetics}
\end{entry}

\begin{entry}{四川}{5,3}{⼞、⼮}
  \begin{phonetics}{四川}{si4chuan1}
    \definition*{s.}{Sichuan}
  \end{phonetics}
\end{entry}

\begin{entry}{四周}{5,8}{⼞、⼝}
  \begin{phonetics}{四周}{si4 zhou1}[][HSK 5]
    \definition{s.}{ao redor; por todos os lados; a parte que circunda o centro}
  \end{phonetics}
\end{entry}

\begin{entry}{四季分明}{5,8,4,8}{⼞、⼦、⼑、⽇}
  \begin{phonetics}{四季分明}{si4ji4-fen1ming2}
    \definition{expr.}{as quatro estações são muito distintas}
  \end{phonetics}
\end{entry}

\begin{entry}{四季如春}{5,8,6,9}{⼞、⼦、⼥、⽇}
  \begin{phonetics}{四季如春}{si4ji4-ru2chun1}
    \definition{expr.}{é primavera todo o ano | clima favorável durante todo o ano | quatro estações como a primavera}
  \end{phonetics}
\end{entry}

\begin{entry}{圣地}{5,6}{⼟、⼟}
  \begin{phonetics}{圣地}{sheng4di4}
    \definition{s.}{terra santa (de uma religião) | lugar sagrado | santuário | cidade santa (como Jerusalém, Meca, etc.) | centro de interesse histórico}
  \end{phonetics}
\end{entry}

\begin{entry}{圣诞节}{5,8,5}{⼟、⾔、⾋}
  \begin{phonetics}{圣诞节}{sheng4dan4jie2}
    \definition*{s.}{Natal}
  \end{phonetics}
\end{entry}

\begin{entry}{处}{5}{⼡}
  \begin{phonetics}{处}{chu3}[][HSK 4]
    \definition*{s.}{sobrenome Chu}
    \definition{v.}{morar; habitar; viver em um lugar | dar-se bem (com alguém); relacionar-se; interagir | estar situado em; estar em uma determinada condição; estar em (um lugar, período ou ocasião) | gerenciar; manejar; lidar com | punir; sentenciar; tomar medidas disciplinares contra (alguém)}
  \end{phonetics}
  \begin{phonetics}{处}{chu4}
    \definition{clas.}{para locais ou itens de danos: lugar, local}
    \definition{s.}{lugar; local; instalação; dependência | parte; ponto; aspecto ou parte de um objeto | escritório; departamento; nomes de determinados órgãos, organizações ou unidades em órgãos por empresa}
  \end{phonetics}
\end{entry}

\begin{entry}{处于}{5,3}{⼡、⼆}
  \begin{phonetics}{处于}{chu3 yu2}[][HSK 4]
    \definition{v.}{estar em (uma condição, estado)}
  \end{phonetics}
\end{entry}

\begin{entry}{处分}{5,4}{⼡、⼑}
  \begin{phonetics}{处分}{chu3fen4}[][HSK 5]
    \definition{s.}{punição; castigo; refere-se a uma decisão de impor uma penalidade ou uma disposição}
    \definition{v.}{punir; tomar medidas disciplinares contra; fornecer algum tratamento ou disposição para aqueles que cometeram erros ou falhas}
  \end{phonetics}
\end{entry}

\begin{entry}{处处}{5,5}{⼡、⼡}
  \begin{phonetics}{处处}{chu4chu4}
    \definition{adv.}{em todos os lugares | em todos os aspectos}
  \end{phonetics}
\end{entry}

\begin{entry}{处在}{5,6}{⼡、⼟}
  \begin{phonetics}{处在}{chu3 zai4}[][HSK 5]
    \definition{v.}{estar situado em; encontrar-se em; estar em (algum estado, posição ou condição)}
  \end{phonetics}
\end{entry}

\begin{entry}{处罚}{5,9}{⼡、⽹}
  \begin{phonetics}{处罚}{chu3 fa2}[][HSK 5]
    \definition[个]{s.}{punição; castigo; penalidade}
    \definition{v.}{punir; disciplinar; castigar; advertir o transgressor ou infrator sobre perdas políticas ou financeiras}
  \end{phonetics}
\end{entry}

\begin{entry}{处理}{5,11}{⼡、⽟}
  \begin{phonetics}{处理}{chu3li3}[][HSK 3]
    \definition{s.}{manuseio; descarte}
    \definition{v.}{lidar com; dispor de | resolver; punir; lidar | vender a preços reduzidos; liquidar | lidar com; processar}
  \end{phonetics}
\end{entry}

\begin{entry}{外}{5}{⼣}
  \begin{phonetics}{外}{wai4}[][HSK 1]
    \definition{s.}{fora | por fora | exterior | estrangeiro}
  \end{phonetics}
\end{entry}

\begin{entry}{外公}{5,4}{⼣、⼋}
  \begin{phonetics}{外公}{wai4gong1}
    \definition{s.}{avô materno}
  \end{phonetics}
\end{entry}

\begin{entry}{外文}{5,4}{⼣、⽂}
  \begin{phonetics}{外文}{wai4 wen2}[][HSK 3]
    \definition{s.}{língua estrangeira (escrita)}
  \end{phonetics}
\end{entry}

\begin{entry}{外水}{5,4}{⼣、⽔}
  \begin{phonetics}{外水}{wai4shui3}
    \definition{s.}{renda extra}
  \end{phonetics}
\end{entry}

\begin{entry}{外号}{5,5}{⼣、⼝}
  \begin{phonetics}{外号}{wai4hao4}
    \definition{s.}{apelido}
  \end{phonetics}
\end{entry}

\begin{entry}{外汇}{5,5}{⼣、⽔}
  \begin{phonetics}{外汇}{wai4 hui4}[][HSK 4]
    \definition{s.}{câmbio estrangeiro; moeda estrangeira; moedas estrangeiras e títulos, como cheques, letras de câmbio, notas promissórias, etc., conversíveis em moedas estrangeiras, usados na compensação do comércio internacional}
  \end{phonetics}
\end{entry}

\begin{entry}{外边}{5,5}{⼣、⾡}
  \begin{phonetics}{外边}{wai4bian5}[][HSK 1]
    \definition{adv.}{fora do país | superfície externa | fora | lugar diferente de sua casa}
  \end{phonetics}
\end{entry}

\begin{entry}{外交}{5,6}{⼣、⼇}
  \begin{phonetics}{外交}{wai4jiao1}[][HSK 3]
    \definition{adj.}{diplomático}
    \definition[个]{s.}{diplomacia; relações exteriores}
  \end{phonetics}
\end{entry}

\begin{entry}{外交官}{5,6,8}{⼣、⼇、⼧}
  \begin{phonetics}{外交官}{wai4 jiao1 guan1}[][HSK 4]
    \definition{s.}{diplomata}
  \end{phonetics}
\end{entry}

\begin{entry}{外协}{5,6}{⼣、⼗}
  \begin{phonetics}{外协}{wai4xie2}
    \definition{s.}{terceirização | pessoas que julgam os outros pela aparência}
  \seealsoref{外貌协会}{wai4mao4xie2hui4}
  \end{phonetics}
\end{entry}

\begin{entry}{外地}{5,6}{⼣、⼟}
  \begin{phonetics}{外地}{wai4 di4}[][HSK 2]
    \definition{s.}{não local | outros lugares}
  \end{phonetics}
\end{entry}

\begin{entry}{外孙}{5,6}{⼣、⼦}
  \begin{phonetics}{外孙}{wai4sun1}
    \definition{s.}{filho da filha}
  \end{phonetics}
\end{entry}

\begin{entry}{外孙女}{5,6,3}{⼣、⼦、⼥}
  \begin{phonetics}{外孙女}{wai4sun1nv3}
    \definition{s.}{filha da filha}
  \end{phonetics}
\end{entry}

\begin{entry}{外衣}{5,6}{⼣、⾐}
  \begin{phonetics}{外衣}{wai4yi1}
    \definition{s.}{aparência | roupa de cima}
  \end{phonetics}
\end{entry}

\begin{entry}{外围}{5,7}{⼣、⼞}
  \begin{phonetics}{外围}{wai4wei2}
    \definition{adv.}{arredores}
  \end{phonetics}
\end{entry}

\begin{entry}{外事}{5,8}{⼣、⼅}
  \begin{phonetics}{外事}{wai4shi4}
    \definition{s.}{assuntos ou relações exteriores}
  \end{phonetics}
\end{entry}

\begin{entry}{外卖}{5,8}{⼣、⼗}
  \begin{phonetics}{外卖}{wai4 mai4}[][HSK 2]
    \definition{s.}{para viagem | para fora}
    \definition{v.}{entregar | oferecer}
  \end{phonetics}
\end{entry}

\begin{entry}{外国}{5,8}{⼣、⼞}
  \begin{phonetics}{外国}{wai4guo2}[][HSK 1]
    \definition[个]{s.}{país estrangeiro}
  \end{phonetics}
\end{entry}

\begin{entry}{外国人}{5,8,2}{⼣、⼞、⼈}
  \begin{phonetics}{外国人}{wai4guo2ren2}
    \definition{s.}{estrangeiro | pessoa de fora do país}
  \end{phonetics}
\end{entry}

\begin{entry}{外界}{5,9}{⼣、⽥}
  \begin{phonetics}{外界}{wai4jie4}[][HSK 5]
    \definition{s.}{o exterior; o mundo externo; área fora de um determinado âmbito; sociedade externa}
  \end{phonetics}
\end{entry}

\begin{entry}{外语}{5,9}{⼣、⾔}
  \begin{phonetics}{外语}{wai4yu3}[][HSK 1]
    \definition[门]{s.}{língua estrangeira}
  \end{phonetics}
\end{entry}

\begin{entry}{外贸}{5,9}{⼣、⾙}
  \begin{phonetics}{外贸}{wai4mao4}
    \definition{s.}{comércio exterior}
  \end{phonetics}
\end{entry}

\begin{entry}{外面}{5,9}{⼣、⾯}
  \begin{phonetics}{外面}{wai4 mian4}[][HSK 3]
    \definition{s.}{o lado de fora | exterior; aparência externa}
  \end{phonetics}
\end{entry}

\begin{entry}{外套}{5,10}{⼣、⼤}
  \begin{phonetics}{外套}{wai4 tao4}[][HSK 4]
    \definition[件,套]{s.}{casaco; jaqueta; paletó; sobretudo}
  \end{phonetics}
\end{entry}

\begin{entry}{外海}{5,10}{⼣、⽔}
  \begin{phonetics}{外海}{wai4hai3}
    \definition{s.}{mar aberto}
  \end{phonetics}
\end{entry}

\begin{entry}{外积}{5,10}{⼣、⽲}
  \begin{phonetics}{外积}{wai4ji1}
    \definition{s.}{produto exterior | (matemática) o produto vetorial de dois vetores}
  \end{phonetics}
\end{entry}

\begin{entry}{外婆}{5,11}{⼣、⼥}
  \begin{phonetics}{外婆}{wai4po2}
    \definition{s.}{avó materna}
  \end{phonetics}
\end{entry}

\begin{entry}{外插}{5,12}{⼣、⼿}
  \begin{phonetics}{外插}{wai4cha1}
    \definition{v.}{extrapolar | (computação) conectar (um dispositivo periférico, etc.)}
  \end{phonetics}
\end{entry}

\begin{entry}{外貌协会}{5,14,6,6}{⼣、⾘、⼗、⼈}
  \begin{phonetics}{外貌协会}{wai4mao4xie2hui4}
    \definition{s.}{``o clube da boa aparência'': pessoas que dão grande importância à aparência de uma pessoa}
  \seealsoref{外协}{wai4xie2}
  \end{phonetics}
\end{entry}

\begin{entry}{失业}{5,5}{⼤、⼀}
  \begin{phonetics}{失业}{shi1ye4}[][HSK 4]
    \definition{v.}{não ter emprego; estar desempregado; estar sem trabalho}
  \end{phonetics}
\end{entry}

\begin{entry}{失去}{5,5}{⼤、⼛}
  \begin{phonetics}{失去}{shi1qu4}[][HSK 3]
    \definition{v.}{perder}
  \end{phonetics}
\end{entry}

\begin{entry}{失败}{5,8}{⼤、⾒}
  \begin{phonetics}{失败}{shi1bai4}[][HSK 4]
    \definition{adj.}{insatisfatório; a maneira como as coisas aconteceram deixou muito a desejar; o resultado final deixou muito a desejar}
    \definition{v.}{perder; ser derrotado; não vencer em uma guerra ou competição | falhar; fracassar; não dar em nada; falhar em atingir um objetivo ou meta desejada (trabalho, carreira, etc.)}
  \end{phonetics}
\end{entry}

\begin{entry}{失误}{5,9}{⼤、⾔}
  \begin{phonetics}{失误}{shi1wu4}[][HSK 5]
    \definition{s.}{erro; engano; equívoco; erros causados por negligência ou medidas inadequadas}
    \definition[个]{v.}{cometer um erro; cometer um equívoco}
  \end{phonetics}
\end{entry}

\begin{entry}{失眠}{5,10}{⼤、⽬}
  \begin{phonetics}{失眠}{shi1mian2}
    \definition{s.}{insônia}
    \definition{v.}{ter insônia}
  \end{phonetics}
\end{entry}

\begin{entry}{失望}{5,11}{⼤、⽉}
  \begin{phonetics}{失望}{shi1wang4}[][HSK 4]
    \definition{adj.}{desapontado; decepcionado}
    \definition{v.}{ficar desapontado; ficar decepcionado; estar desapontado;}
  \end{phonetics}
\end{entry}

\begin{entry}{失落}{5,12}{⼤、⾋}
  \begin{phonetics}{失落}{shi1luo4}
    \definition{s.}{frustração | decepção | perda}
    \definition{v.}{perder (algo) | cair (algo) | sentir uma sensação de perda}
  \end{phonetics}
\end{entry}

\begin{entry}{失意}{5,13}{⼤、⼼}
  \begin{phonetics}{失意}{shi1yi4}
    \definition{adj.}{desapontado | frustrado}
  \end{phonetics}
\end{entry}

\begin{entry}{头}{5}{⼤}
  \begin{phonetics}{头}{tou2}[][HSK 2,3]
    \definition{adj.}{(antes de um numeral) primeiro | (antes de ``年'' ou ``天'') último; anterior}
    \definition{clas.}{para suínos ou gado (animais de criação) | para alho}
    \definition{num.}{primeiro}
    \definition{prep.}{antes de; perto de | (entre dois algarismos, indicando um número aproximado) cerca de}
    \definition[个]{s.}{cabeça | cabelo ou penteado | topo; fim | começo ou fim | fim; remanescente |cabeça; chefe; líder |lado; aspecto}
  \seealsoref{年}{nian2}
  \seealsoref{天}{tian1}
  \end{phonetics}
  \begin{phonetics}{头}{tou5}
    \definition{suf.}{sufixo para nomes}
  \end{phonetics}
\end{entry}

\begin{entry}{头发}{5,5}{⼤、⼜}
  \begin{phonetics}{头发}{tou2fa5}[][HSK 2]
    \definition{s.}{cabelo}
  \end{phonetics}
\end{entry}

\begin{entry}{头号}{5,5}{⼤、⼝}
  \begin{phonetics}{头号}{tou2hao4}
    \definition{adj.}{primeira classe | número um | \emph{top rank}}
  \end{phonetics}
\end{entry}

\begin{entry}{头头}{5,5}{⼤、⼤}
  \begin{phonetics}{头头}{tou2tou2}
    \definition{s.}{chefe | o cabeça}
  \end{phonetics}
\end{entry}

\begin{entry}{头脑}{5,10}{⼤、⾁}
  \begin{phonetics}{头脑}{tou2 nao3}[][HSK 3]
    \definition{s.}{inteligência; mente | pista; tópicos principais | chefe; líder; capitão}
  \end{phonetics}
\end{entry}

\begin{entry}{头脑风暴}{5,10,4,15}{⼤、⾁、⾵、⽇}
  \begin{phonetics}{头脑风暴}{tou2nao3feng1bao4}
    \definition{s.}{\emph{brainstorm}}
  \end{phonetics}
\end{entry}

\begin{entry}{头像}{5,13}{⼤、⼈}
  \begin{phonetics}{头像}{tou2xiang4}
    \definition{s.}{retrato | busto | avatar | imagem de perfil (computação)}
  \end{phonetics}
\end{entry}

\begin{entry}{奶}{5}{⼥}
  \begin{phonetics}{奶}{nai3}[][HSK 1]
    \definition[杯,滴,瓶,只,桶]{s.}{seios | leite}
    \definition{v.}{amamentar}
  \end{phonetics}
\end{entry}

\begin{entry}{奶奶}{5,5}{⼥、⼥}
  \begin{phonetics}{奶奶}{nai3nai5}[][HSK 1]
    \definition[位]{s.}{avó (paterna) | (respeitoso) dona da casa}
  \end{phonetics}
\end{entry}

\begin{entry}{奶茶}{5,9}{⼥、⾋}
  \begin{phonetics}{奶茶}{nai3 cha2}[][HSK 3]
    \definition[杯]{s.}{chá com leite}
  \end{phonetics}
\end{entry}

\begin{entry}{宁}{5}{⼧}
  \begin{phonetics}{宁}{ning2}
    \definition*{s.}{sobrenome Ning}
    \definition{adj.}{calmo, pacífico, sereno | saudável}
  \end{phonetics}
  \begin{phonetics}{宁}{ning4}
    \definition{conj.}{mais\dots do que\dots, melhor\dots do que\dots}
  \end{phonetics}
\end{entry}

\begin{entry}{宁可}{5,5}{⼧、⼝}
  \begin{phonetics}{宁可}{ning4ke3}
    \definition{conj.}{mais\dots do que\dots | melhor\dots do que\dots}
  \end{phonetics}
\end{entry}

\begin{entry}{宁可……也不……}{5,5,3,4}{⼧、⼝、⼄、⼀}
  \begin{phonetics}{宁可……也不……}{ning4ke3 ye3bu4}
    \definition{conj.}{em vez de\dots}
  \end{phonetics}
\end{entry}

\begin{entry}{宁可……也要……}{5,5,3,9}{⼧、⼝、⼄、⾑}
  \begin{phonetics}{宁可……也要……}{ning4ke3 ye3yao4}
    \definition{conj.}{mesmo que tenhamos que\dots nós iremos\dots}
  \end{phonetics}
\end{entry}

\begin{entry}{宁肯}{5,8}{⼧、⾁}
  \begin{phonetics}{宁肯}{ning4ken3}
    \definition{conj.}{mais\dots do que\dots, melhor\dots do que\dots}
  \end{phonetics}
\end{entry}

\begin{entry}{宁愿}{5,14}{⼧、⽕}
  \begin{phonetics}{宁愿}{ning4yuan4}
    \definition{conj.}{mais\dots do que\dots, melhor\dots do que\dots}
  \end{phonetics}
\end{entry}

\begin{entry}{宁静}{5,14}{⼧、⾭}
  \begin{phonetics}{宁静}{ning2 jing4}[][HSK 4]
    \definition{adj.}{calmo; tranquilo; pacífico}
  \end{phonetics}
\end{entry}

\begin{entry}{它}{5}{⼧}
  \begin{phonetics}{它}{ta1}[][HSK 2]
    \definition{pron.}{ele (para objetos inanimados) | se, o, lhe | si, consigo, eles}
  \end{phonetics}
\end{entry}

\begin{entry}{它们}{5,5}{⼧、⼈}
  \begin{phonetics}{它们}{ta1 men5}[][HSK 2]
    \definition{pron.}{eles (para objetos inanimados) | se, os, lhes | si, consigo, eles}
  \end{phonetics}
\end{entry}

\begin{entry}{对}{5}{⼨}
  \begin{phonetics}{对}{dui4}[][HSK 1,2]
    \definition{adj.}{correto | sim}
    \definition{clas.}{para casais}
    \definition{prep.}{com | para | para com}
  \end{phonetics}
\end{entry}

\begin{entry}{对于}{5,3}{⼨、⼆}
  \begin{phonetics}{对于}{dui4yu2}[][HSK 4]
    \definition{prep.}{para; relativo a; no que diz respeito a; a respeito de}
  \end{phonetics}
\end{entry}

\begin{entry}{对不起}{5,4,10}{⼨、⼀、⾛}
  \begin{phonetics}{对不起}{dui4bu5qi3}[][HSK 1]
    \definition{interj.}{Desculpe! | Desculpe-me! | Perdoe-me! | Desculpe? (por favor, repita)}
    \definition{v.}{desculpar | pedir desculpas | perdoar}
  \end{phonetics}
\end{entry}

\begin{entry}{对手}{5,4}{⼨、⼿}
  \begin{phonetics}{对手}{dui4shou3}[][HSK 3]
    \definition[个]{s.}{oponente; adversário}
  \end{phonetics}
\end{entry}

\begin{entry}{对方}{5,4}{⼨、⽅}
  \begin{phonetics}{对方}{dui4fang1}[][HSK 3]
    \definition{s.}{outro lado; lado oposto; outra parte}
  \end{phonetics}
\end{entry}

\begin{entry}{对比}{5,4}{⼨、⽐}
  \begin{phonetics}{对比}{dui4bi3}[][HSK 4]
    \definition{s.}{razão; proporção | contraste; comparação; diferenças ou lacunas encontradas após comparação}
    \definition{v.}{contrastar; comparar}
  \end{phonetics}
\end{entry}

\begin{entry}{对付}{5,5}{⼨、⼈}
  \begin{phonetics}{对付}{dui4fu5}[][HSK 4]
    \definition{adj.}{em bons termos; estar em termos agradáveis ​​(frequentemente usado em negativas); dialeto usado para descrever duas pessoas que têm um bom relacionamento e se dão bem, frequentemente usado para negar}
    \definition{v.}{enfrentar; tratar; lidar com | fazer acontecer; (informal) fazer algo que você não quer fazer; aceitar algo que você não gosta}
  \end{phonetics}
\end{entry}

\begin{entry}{对立}{5,5}{⼨、⽴}
  \begin{phonetics}{对立}{dui4li4}[][HSK 5]
    \definition{v.}{opor-se; contrastar; filosoficamente, refere-se a duas coisas ou dois aspectos da mesma coisa que se contradizem, se excluem ou entram em conflito entre si | opor-se; ser antagônico a}
  \end{phonetics}
\end{entry}

\begin{entry}{对……有兴趣}{5,6,6,15}{⼨、⽉、⼋、⾛}
  \begin{phonetics}{对……有兴趣}{dui4 you3xing4qu4}
    \definition{expr.}{estar interessado em\dots | ter interesse em\dots | interessar-se por\dots}
    \seeref{对……感兴趣}{dui4 gan3xing4qu4}
  \end{phonetics}
\end{entry}

\begin{entry}{对应}{5,7}{⼨、⼴}
  \begin{phonetics}{对应}{dui4ying4}[][HSK 5]
    \definition{adj.}{homólogo; correspondente}
    \definition{v.}{corresponder}
  \end{phonetics}
\end{entry}

\begin{entry}{对话}{5,8}{⼨、⾔}
  \begin{phonetics}{对话}{dui4hua4}[][HSK 2]
    \definition[个]{s.}{diálogo | conversa}
    \definition{v.}{dialogar | conversar}
  \end{phonetics}
\end{entry}

\begin{entry}{对待}{5,9}{⼨、⼻}
  \begin{phonetics}{对待}{dui4dai4}[][HSK 3]
    \definition{v.}{tratar; abordar; manusear; estar em uma posição relacionada ou comparada a outra}
  \end{phonetics}
\end{entry}

\begin{entry}{对……说}{5,9}{⼨、⾔}
  \begin{phonetics}{对……说}{dui4 shuo5}
    \definition{v.}{dizer a alguém}
  \end{phonetics}
\end{entry}

\begin{entry}{对面}{5,9}{⼨、⾯}
  \begin{phonetics}{对面}{dui4mian4}[][HSK 2]
    \definition{s.}{lado oposto}
  \end{phonetics}
\end{entry}

\begin{entry}{对得起}{5,11,10}{⼨、⼻、⾛}
  \begin{phonetics}{对得起}{dui4de5qi3}
    \definition{v.}{não decepcionar alguém | tratar alguém de maneira justa | ser digno de}
  \end{phonetics}
\end{entry}

\begin{entry}{对象}{5,11}{⼨、⾗}
  \begin{phonetics}{对象}{dui4xiang4}[][HSK 3]
    \definition[个]{s.}{alvo; objeto | parceiro; namorado; namorada}
  \end{phonetics}
\end{entry}

\begin{entry}{对……感兴趣}{5,13,6,15}{⼨、⼼、⼋、⾛}
  \begin{phonetics}{对……感兴趣}{dui4 gan3xing4qu4}
    \definition{expr.}{estar interessado em\dots | ter interesse em\dots | interessar-se por\dots}
    \seeref{对……有兴趣}{dui4 you3xing4qu4}
  \end{phonetics}
\end{entry}

\begin{entry}{对……熟悉}{5,15,11}{⼨、⽕、⼼}
  \begin{phonetics}{对……熟悉}{dui4 shu2xi1}
    \definition{expr.}{estar familiarizado com\dots}
  \end{phonetics}
\end{entry}

\begin{entry}{左}{5}{⼯}
  \begin{phonetics}{左}{zuo3}[][HSK 1]
    \definition*{s.}{sobrenome Zuo}
    \definition{s.}{esquerda}
  \end{phonetics}
\end{entry}

\begin{entry}{左右}{5,5}{⼯、⼝}
  \begin{phonetics}{左右}{zuo3you4}[][HSK 3]
    \definition{adv.}{aproximadamente; ou mais ou menos; por aí; usado depois de um número para indicar um número aproximado, o mesmo que ``上下''}
    \definition{s.}{os lados esquerdo e direito; esquerda e direita, também significa circundar
atendentes; pessoas que te seguem}
    \definition{v.}{controlar; manipular; influenciar}
  \end{phonetics}
\end{entry}

\begin{entry}{左边}{5,5}{⼯、⾡}
  \begin{phonetics}{左边}{zuo3bian5}[][HSK 1]
    \definition{s.}{esquerda | lado esquerdo}
  \end{phonetics}
\end{entry}

\begin{entry}{左派}{5,9}{⼯、⽔}
  \begin{phonetics}{左派}{zuo3pai4}
    \definition{s.}{(política) esquerda | esquerdista}
  \end{phonetics}
\end{entry}

\begin{entry}{左面}{5,9}{⼯、⾯}
  \begin{phonetics}{左面}{zuo3mian4}
    \definition{s.}{esquerda | lado esquerdo}
  \end{phonetics}
\end{entry}

\begin{entry}{左倾}{5,10}{⼯、⼈}
  \begin{phonetics}{左倾}{zuo3qing1}
    \definition{s.}{esquerdista | progressivo}
  \end{phonetics}
\end{entry}

\begin{entry}{左袒}{5,10}{⼯、⾐}
  \begin{phonetics}{左袒}{zuo3tan3}
    \definition{v.}{ser tendencioso | ser parcial para | favorecer um lado | tomar partido com}
  \end{phonetics}
\end{entry}

\begin{entry}{左舷}{5,11}{⼯、⾈}
  \begin{phonetics}{左舷}{zuo3xian2}
    \definition{s.}{porto (lado de um navio)}
  \end{phonetics}
\end{entry}

\begin{entry}{左翼}{5,17}{⼯、⽻}
  \begin{phonetics}{左翼}{zuo3yi4}
    \definition{s.}{esquerda (política)}
  \end{phonetics}
\end{entry}

\begin{entry}{巧}{5}{⼯}
  \begin{phonetics}{巧}{qiao3}[][HSK 3]
    \definition{adj.}{habilidoso; engenhoso; esperto | oportuno; coincidente; fortuito | astuto; enganoso; enganador; traiçoeiro; ardiloso}
  \end{phonetics}
\end{entry}

\begin{entry}{巧合}{5,6}{⼯、⼝}
  \begin{phonetics}{巧合}{qiao3he2}
    \definition{s.}{coincidência}
    \definition{v.}{coincidir}
  \end{phonetics}
\end{entry}

\begin{entry}{巧克力}{5,7,2}{⼯、⼗、⼒}
  \begin{phonetics}{巧克力}{qiao3ke4li4}[][HSK 4]
    \definition[块]{s.}{(empréstimo linguístico) chocolate}
  \end{phonetics}
\end{entry}

\begin{entry}{市}{5}{⼱}
  \begin{phonetics}{市}{shi4}[][HSK 2]
    \definition*{s.}{sobrenome Shi}
    \definition{s.}{mercado | cidade | município | referente ao sistema chinês de pesos e medidas}
    \definition{v.}{comprar | vender | negociar}
  \end{phonetics}
\end{entry}

\begin{entry}{市中心}{5,4,4}{⼱、⼁、⼼}
  \begin{phonetics}{市中心}{shi4zhong1xin1}
    \definition{s.}{centro da cidade}
  \end{phonetics}
\end{entry}

\begin{entry}{市区}{5,4}{⼱、⼖}
  \begin{phonetics}{市区}{shi4 qu1}[][HSK 4]
    \definition[个]{s.}{\emph{downtown}; centro da cidade; distrito urbano; áreas que ficam dentro dos limites da cidade e geralmente têm uma alta concentração de população e estoque de moradias.}
  \end{phonetics}
\end{entry}

\begin{entry}{市长}{5,4}{⼱、⾧}
  \begin{phonetics}{市长}{shi4 zhang3}[][HSK 2]
    \definition[个]{s.}{prefeito}
  \end{phonetics}
\end{entry}

\begin{entry}{市场}{5,6}{⼱、⼟}
  \begin{phonetics}{市场}{shi4chang3}[][HSK 3]
    \definition[家]{s.}{mercado (também no abstrato) | área de \emph{marketing} | âmbito de influência (figurado)}
  \end{phonetics}
\end{entry}

\begin{entry}{布}{5}{⼱}
  \begin{phonetics}{布}{bu4}[][HSK 3]
    \definition*{s.}{sobrenome Bu}
    \definition[块,幅,匹]{s.}{pano | tecido | uma moeda de cobre nos tempos antigos}
    \definition{v.}{anunciar | declarar | tornar conhecido | proclamar | publicar | espalhar | disseminar |organizar | implantar | dispor}
  \end{phonetics}
\end{entry}

\begin{entry}{布谷鸟}{5,7,5}{⼱、⾕、⿃}
  \begin{phonetics}{布谷鸟}{bu4gu3niao3}
    \definition{s.}{cuco (pássaro)}
  \seealsoref{杜鹃}{du4juan1}
  \seealsoref{杜鹃鸟}{du4juan1niao3}
  \seealsoref{杜宇}{du4yu3}
  \end{phonetics}
\end{entry}

\begin{entry}{布置}{5,13}{⼱、⽹}
  \begin{phonetics}{布置}{bu4zhi4}[][HSK 4]
    \definition{v.}{arrumar; organizar; decorar; colocar adequadamente objetos ou paisagismo, conforme necessário | designar; tomar providências para; dar instruções sobre; organizar trabalho, atividades, etc.}
  \end{phonetics}
\end{entry}

\begin{entry}{布署}{5,13}{⼱、⽹}
  \begin{phonetics}{布署}{bu4shu3}
    \variantof{部署}
  \end{phonetics}
\end{entry}

\begin{entry}{帅}{5}{⼱}
  \begin{phonetics}{帅}{shuai4}[][HSK 4]
    \definition*{s.}{sobrenome Shuai}
    \definition{adj.}{bonito; arrojado; elegante; inteligente}
    \definition{interj.}{Legal!}
    \definition[位,名]{s.}{comandante em chefe; o mais alto comandante do exército | comandante em chefe, a peça principal no xadrez chinês}
  \end{phonetics}
\end{entry}

\begin{entry}{帅哥}{5,10}{⼱、⼝}
  \begin{phonetics}{帅哥}{shuai4 ge1}[][HSK 4]
    \definition[个,位]{s.}{rapaz bonito; um garoto que é bonito e atraente na aparência}
  \end{phonetics}
\end{entry}

\begin{entry}{平}{5}{⼲}
  \begin{phonetics}{平}{ping2}[][HSK 2]
    \definition*{s.}{sobrenome Ping}
    \definition{adj.}{calmo | pacífico}
    \definition{s.}{plano | nível}
    \definition{v.}{fazer a mesma pontuação | marcar uma pontuação}
  \end{phonetics}
\end{entry}

\begin{entry}{平方}{5,4}{⼲、⽅}
  \begin{phonetics}{平方}{ping2fang1}[][HSK 4]
    \definition{s.}{quadrado}
  \end{phonetics}
\end{entry}

\begin{entry}{平方米}{5,4,6}{⼲、⽅、⽶}
  \begin{phonetics}{平方米}{ping2fang1 mi3}
    \definition{clas.}{unidade de medida de área, 1 metro quadrado equivale a 10.000 centímetros quadrados}
  \end{phonetics}
\end{entry}

\begin{entry}{平台}{5,5}{⼲、⼝}
  \begin{phonetics}{平台}{ping2tai2}
    \definition{s.}{plataforma | terraço | edifício de telhado plano}
  \end{phonetics}
\end{entry}

\begin{entry}{平地}{5,6}{⼲、⼟}
  \begin{phonetics}{平地}{ping2di4}
    \definition{v.}{nivelar a terra | aplanar}
  \end{phonetics}
\end{entry}

\begin{entry}{平安}{5,6}{⼲、⼧}
  \begin{phonetics}{平安}{ping2'an1}[][HSK 2]
    \definition{s.}{seguro | bem | sem contratempos | são e salvo}
  \end{phonetics}
\end{entry}

\begin{entry}{平均}{5,7}{⼲、⼟}
  \begin{phonetics}{平均}{ping2jun1}[][HSK 4]
    \definition{adj.}{igual; médio}
    \definition{s.}{média}
    \definition{v.}{calcular a média de um conjunto de números}
  \end{phonetics}
\end{entry}

\begin{entry}{平时}{5,7}{⼲、⽇}
  \begin{phonetics}{平时}{ping2shi2}[][HSK 2]
    \definition{adv.}{normalmente | em tempos normais | em tempos de paz}
  \end{phonetics}
\end{entry}

\begin{entry}{平坦}{5,8}{⼲、⼟}
  \begin{phonetics}{平坦}{ping2tan3}[][HSK 5]
    \definition{adj.}{plano; uniforme; nivelado; liso; sem elevações ou depressões (referindo-se principalmente ao relevo)}
  \end{phonetics}
\end{entry}

\begin{entry}{平原}{5,10}{⼲、⼚}
  \begin{phonetics}{平原}{ping2yuan2}[][HSK 5]
    \definition[片]{s.}{campo; planície; terreno plano e extenso}
  \end{phonetics}
\end{entry}

\begin{entry}{平常}{5,11}{⼲、⼱}
  \begin{phonetics}{平常}{ping2chang2}[][HSK 2]
    \definition{adj.}{comum | ordinário | usual}
    \definition{adv.}{usualmente | geralmente | ordinariamente | como regra}
  \end{phonetics}
\end{entry}

\begin{entry}{平等}{5,12}{⼲、⽵}
  \begin{phonetics}{平等}{ping2deng3}[][HSK 2]
    \definition{adj.}{igual | igualdade}
  \end{phonetics}
\end{entry}

\begin{entry}{平稳}{5,14}{⼲、⽲}
  \begin{phonetics}{平稳}{ping2 wen3}[][HSK 4]
    \definition{adj.}{firme; estável; suave e constante; sem oscilações ou flutuações}
  \end{phonetics}
\end{entry}

\begin{entry}{平静}{5,14}{⼲、⾭}
  \begin{phonetics}{平静}{ping2jing4}[][HSK 4]
    \definition{adj.}{(humor, ambiente, etc.) calmo; quieto; pacífico; tranquilo}
  \end{phonetics}
\end{entry}

\begin{entry}{幼儿园}{5,2,7}{⼳、⼉、⼞}
  \begin{phonetics}{幼儿园}{you4'er2yuan2}[][HSK 4]
    \definition{s.}{jardim de infância; escola maternal; escola infantil; instituição para a educação de crianças pequenas}
  \end{phonetics}
\end{entry}

\begin{entry}{归}{5}{⼹}
  \begin{phonetics}{归}{gui1}[][HSK 4]
    \definition*{s.}{sobrenome Gui}
    \definition{s.}{divisão no ábaco com divisor de um dígito}
    \definition{v.}{retornar; voltar para; voltar (ou ir) | devolver algo a; dar de volta a | convergir; juntar-se | encarregar alguém de algo | atribuir a; pertencer a | usado entre os mesmos verbos, indicando que a ação não levou ao resultado correspondente}
  \end{phonetics}
\end{entry}

\begin{entry}{必}{5}{⼼}
  \begin{phonetics}{必}{bi4}[][HSK 5]
    \definition{adv.}{certamente; necessariamente; indica que algo é certo ou que alguém acredita que esteja correto}
  \end{phonetics}
\end{entry}

\begin{entry}{必定}{5,8}{⼼、⼧}
  \begin{phonetics}{必定}{bi4ding4}
    \definition{adv.}{sem falta | certamente | com certeza | definitivamente | inevitavelmente | com determinação}
    \definition{v.}{estar vinculado a | ter certeza de}
  \end{phonetics}
\end{entry}

\begin{entry}{必要}{5,9}{⼼、⾑}
  \begin{phonetics}{必要}{bi4yao4}[][HSK 3]
    \definition{adj.}{necessário | essencial | indispensável}
    \definition[个,些]{s.}{necessidade}
  \end{phonetics}
\end{entry}

\begin{entry}{必须}{5,9}{⼼、⾴}
  \begin{phonetics}{必须}{bi4xu1}[][HSK 2]
    \definition{adv.}{necessariamente | obrigatoriamente}
  \end{phonetics}
\end{entry}

\begin{entry}{必然}{5,12}{⼼、⽕}
  \begin{phonetics}{必然}{bi4ran2}[][HSK 3]
    \definition{adj.}{certo | inevitável | necessário}
    \definition{adv.}{inevitavelmente}
    \definition{s.}{necessidade}
  \end{phonetics}
\end{entry}

\begin{entry}{必需}{5,14}{⼼、⾬}
  \begin{phonetics}{必需}{bi4 xu1}[][HSK 5]
    \definition{v.}{ser essencial; ser indispensável}
  \end{phonetics}
\end{entry}

\begin{entry}{扑克}{5,7}{⼿、⼗}
  \begin{phonetics}{扑克}{pu1ke4}
    \definition{s.}{(empréstimo linguístico) (jogo) \emph{poker}  | baralho}
  \end{phonetics}
\end{entry}

\begin{entry}{扒犁}{5,11}{⼿、⽜}
  \begin{phonetics}{扒犁}{pa2li2}
    \definition{s.}{trenó}
    \seeref{爬犁}{pa2li2}
  \end{phonetics}
\end{entry}

\begin{entry}{打}{5}{⼿}
  \begin{phonetics}{打}{da2}
    \definition{clas./s.}{(empréstimo linguístico) dúzia}
  \end{phonetics}
  \begin{phonetics}{打}{da3}[][HSK 1,4,5]
    \definition{prep.}{de; desde; ponto de partida que indica lugar, tempo ou extensão | indica rotas e locais percorridos}
    \definition{v.}{golpear; acertar; bater | quebrar; esmagar | lutar; atacar; espancar | entrar com uma ação judicial; negociar; fazer representações | construir; edificar | fabricar (em uma ferraria); forjar | misturar; mexer; bater | amarrar; embalar | tricotar; tecer | desenhar; pintar; deixar uma marca; imprimir | abrir; perfurar; cavar | içar; levantar
enviar; despachar; projetar | emitir ou receber (um certificado, etc.) | remover; livrar-se de}
  \end{phonetics}
\end{entry}

\begin{entry}{打工}{5,3}{⼿、⼯}
  \begin{phonetics}{打工}{da3gong1}[][HSK 2]
    \definition{v.}{(para alunos) ter um emprego fora do horário de aula ou durante as férias | trabalhar em um emprego temporá rio ou casual}
  \end{phonetics}
\end{entry}

\begin{entry}{打工人}{5,3,2}{⼿、⼯、⼈}
  \begin{phonetics}{打工人}{da3gong1ren2}
    \definition{s.}{trabalhador}
  \end{phonetics}
\end{entry}

\begin{entry}{打开}{5,4}{⼿、⼶}
  \begin{phonetics}{打开}{da3 kai1}[][HSK 1]
    \definition{v.}{abrir | desdobrar | ligar | avançar | espalhar}
  \end{phonetics}
\end{entry}

\begin{entry}{打车}{5,4}{⼿、⾞}
  \begin{phonetics}{打车}{da3 che1}[][HSK 1]
    \definition{v.}{pegar um táxi | chamar um táxi}
  \end{phonetics}
\end{entry}

\begin{entry}{打击}{5,5}{⼿、⼐}
  \begin{phonetics}{打击}{da3ji1}[][HSK 5]
    \definition{v.}{golpear; atacar; reprimir; atacar para frustrar; machucar | bater; bater (em um tambor, etc.); golpear ou bater em algo}
  \end{phonetics}
\end{entry}

\begin{entry}{打包}{5,5}{⼿、⼓}
  \begin{phonetics}{打包}{da3bao1}[][HSK 5]
    \definition{v.}{levar a comida embora; levar para viagem; refere-se especificamente a comer em um restaurante e levar as sobras em uma caixa, sacola ou outro recipiente | embalar; empacotar | desembalar; desempacotar}
  \end{phonetics}
\end{entry}

\begin{entry}{打印}{5,5}{⼿、⼙}
  \begin{phonetics}{打印}{da3yin4}[][HSK 2]
    \definition{v.}{imprimir}
  \end{phonetics}
\end{entry}

\begin{entry}{打电话}{5,5,8}{⼿、⽥、⾔}
  \begin{phonetics}{打电话}{da3 dian4 hua4}[][HSK 1]
    \definition{v.}{telefonar | fazer uma chamada telefônica | dar um telefonema}
  \seealsoref{给……打电话}{gei3 da3 dian4 hua4}
  \end{phonetics}
\end{entry}

\begin{entry}{打压}{5,6}{⼿、⼚}
  \begin{phonetics}{打压}{da3ya1}
    \definition{v.}{reprimir | derrotar}
  \end{phonetics}
\end{entry}

\begin{entry}{打扫}{5,6}{⼿、⼿}
  \begin{phonetics}{打扫}{da3sao3}[][HSK 4]
    \definition{v.}{varrer; limpar; varrer para limpar}
  \end{phonetics}
\end{entry}

\begin{entry}{打听}{5,7}{⼿、⼝}
  \begin{phonetics}{打听}{da3ting5}[][HSK 3]
    \definition{v.}{perguntar sobre; indagar sobre; obter uma linha sobre}
  \end{phonetics}
\end{entry}

\begin{entry}{打屁股}{5,7,8}{⼿、⼫、⾁}
  \begin{phonetics}{打屁股}{da3pi4gu5}
    \definition{v.}{dar um tapa no bumbum de alguém}
  \end{phonetics}
\end{entry}

\begin{entry}{打扮}{5,7}{⼿、⼿}
  \begin{phonetics}{打扮}{da3ban5}[][HSK 5]
    \definition{s.}{estilo de se vestir; o modo de se vestir; as roupas que se usa}
    \definition{v.}{vestir-se bem; maquiar-se; dar uma boa aparência e vestir-se bem; adornar}
  \end{phonetics}
\end{entry}

\begin{entry}{打扰}{5,7}{⼿、⼿}
  \begin{phonetics}{打扰}{da3rao3}[][HSK 5]
    \definition{v.}{perturbar; incomodar; interferir no trabalho normal, na vida ou no que as outras pessoas estão fazendo, etc. | usado para expressar um pedido de desculpas por ajuda; gratidão por ajuda; hospitalidade recebida}
  \end{phonetics}
\end{entry}

\begin{entry}{打折}{5,7}{⼿、⼿}
  \begin{phonetics}{打折}{da3zhe2}[][HSK 4]
    \definition{v.+compl.}{dar desconto; dar um desconto; vender produtos a um preço reduzido em uma determinada porcentagem do preço original; metáfora para não cumprir 100\% do que foi originalmente padronizado ou prometido}
  \end{phonetics}
\end{entry}

\begin{entry}{打针}{5,7}{⼿、⾦}
  \begin{phonetics}{打针}{da3zhen1}[][HSK 4]
    \definition{v.+compl.}{dar ou receber uma injeção; injetar um medicamento líquido em um organismo com uma seringa}
  \end{phonetics}
\end{entry}

\begin{entry}{打的}{5,8}{⼿、⽩}
  \begin{phonetics}{打的}{da3di1}
    \definition{v.+compl.}{(coloquial) pegar um táxi | ir de táxi}
  \end{phonetics}
\end{entry}

\begin{entry}{打败}{5,8}{⼿、⾒}
  \begin{phonetics}{打败}{da3 bai4}[][HSK 4]
    \definition{v.}{derrotar; vencer; piorar | sofrer uma derrota; ser derrotado}
  \end{phonetics}
\end{entry}

\begin{entry}{打架}{5,9}{⼿、⽊}
  \begin{phonetics}{打架}{da3jia4}[][HSK 5]
    \definition{v.+compl.}{brigar; discutir; entrar em conflito | contradizer; conflitar; ser inconsistente}
  \end{phonetics}
\end{entry}

\begin{entry}{打结}{5,9}{⼿、⽷}
  \begin{phonetics}{打结}{da3jie2}
    \definition{v.}{dar um nó | amarrar}
  \end{phonetics}
\end{entry}

\begin{entry}{打骂}{5,9}{⼿、⾺}
  \begin{phonetics}{打骂}{da3ma4}
    \definition{v.}{bater e repreender}
  \end{phonetics}
\end{entry}

\begin{entry}{打破}{5,10}{⼿、⽯}
  \begin{phonetics}{打破}{da3 po4}[][HSK 3]
    \definition{v.}{quebrar; esmagar}
  \end{phonetics}
\end{entry}

\begin{entry}{打猎}{5,11}{⼿、⽝}
  \begin{phonetics}{打猎}{da3lie4}
    \definition{v.}{ir caçar}
  \end{phonetics}
\end{entry}

\begin{entry}{打球}{5,11}{⼿、⽟}
  \begin{phonetics}{打球}{da3 qiu2}[][HSK 1]
    \definition{v.}{jogar bola (com as mãos) | jogar (basquetebol, handbol, etc.)}
  \end{phonetics}
\end{entry}

\begin{entry}{打搅}{5,12}{⼿、⼿}
  \begin{phonetics}{打搅}{da3jiao3}
    \definition{v.}{perturbar | incomodar}
  \end{phonetics}
\end{entry}

\begin{entry}{打雷}{5,13}{⼿、⾬}
  \begin{phonetics}{打雷}{da3 lei2}[][HSK 4]
    \definition{v.}{trovejar; produzir ruídos altos quando as nuvens descarregam eletricidade}
  \end{phonetics}
\end{entry}

\begin{entry}{打算}{5,14}{⼿、⽵}
  \begin{phonetics}{打算}{da3suan4}[][HSK 2]
    \definition[个]{s.}{plano | intenção}
    \definition{v.}{pensar | planejar | pretender}
  \end{phonetics}
\end{entry}

\begin{entry}{打瞌睡}{5,15,13}{⼿、⽬、⽬}
  \begin{phonetics}{打瞌睡}{da3ke1shui4}
    \definition{v.}{cochilar}
  \end{phonetics}
\end{entry}

\begin{entry}{打磨}{5,16}{⼿、⽯}
  \begin{phonetics}{打磨}{da3mo2}
    \definition{v.}{polir | fazer brilhar}
  \end{phonetics}
\end{entry}

\begin{entry}{扔}{5}{⼿}
  \begin{phonetics}{扔}{reng1}[][HSK 5]
    \definition{v.}{arremessar; lançar; atirar; jogar | esquecer; jogar fora; descartar | colocar casualmente; deixar as pessoas ou as coisas de lado, não se importar}
  \end{phonetics}
\end{entry}

\begin{entry}{扔下}{5,3}{⼿、⼀}
  \begin{phonetics}{扔下}{reng1xia4}
    \definition{v.}{lançar (uma bomba) | derrubar}
  \end{phonetics}
\end{entry}

\begin{entry}{扔弃}{5,7}{⼿、⼶}
  \begin{phonetics}{扔弃}{reng1qi4}
    \definition{v.}{abandonar | descartar | jogar fora}
  \end{phonetics}
\end{entry}

\begin{entry}{扔掉}{5,11}{⼿、⼿}
  \begin{phonetics}{扔掉}{reng1diao4}
    \definition{v.}{jogar fora}
  \end{phonetics}
\end{entry}

\begin{entry}{斥骂}{5,9}{⽄、⾺}
  \begin{phonetics}{斥骂}{chi4ma4}
    \definition{v.}{repreender}
  \end{phonetics}
\end{entry}

\begin{entry}{旧}{5}{⽇}
  \begin{phonetics}{旧}{jiu4}[][HSK 3]
    \definition*{s.}{sobrenome Jiu}
    \definition{adj.}{passado; antigo; velho | usado; desgastado; velho}
    \definition{s.}{velha amizade; velho amigo}
  \end{phonetics}
\end{entry}

\begin{entry}{未}{5}{⽊}
  \begin{phonetics}{未}{wei4}
    \definition{adv.}{não ter | ainda não}
  \end{phonetics}
\end{entry}

\begin{entry}{未必}{5,5}{⽊、⼼}
  \begin{phonetics}{未必}{wei4bi4}[][HSK 4]
    \definition{adv.}{não tenho certeza; talvez não; não necessariamente}
  \end{phonetics}
\end{entry}

\begin{entry}{未来}{5,7}{⽊、⽊}
  \begin{phonetics}{未来}{wei4lai2}[][HSK 4]
    \definition{adj.}{próximo (refere-se ao tempo)}
    \definition[个]{s.}{futuro; o amanhã}
  \end{phonetics}
\end{entry}

\begin{entry}{末}{5}{⽊}
  \begin{phonetics}{末}{mo4}[][HSK 4]
    \definition{adj.}{último; final}
    \definition{s.}{ponta; terminal; extremidade; o final de algo | não essenciais; detalhes secundários | fim; final | pó; poeira | um papel na ópera tradicional}
  \end{phonetics}
\end{entry}

\begin{entry}{本}{5}{⽊}
  \begin{phonetics}{本}{ben3}[][HSK 1]
    \definition{adj.}{o atual | original | inerente}
    \definition{adv.}{originalmente}
    \definition{clas.}{para livros, dicionários, periódicos, arquivos, etc.}
    \definition{s.}{raiz | caule | origem | fonte}
  \end{phonetics}
\end{entry}

\begin{entry}{本人}{5,2}{⽊、⼈}
  \begin{phonetics}{本人}{ben3ren2}[][HSK 5]
    \definition{pron.}{eu (mim, mim mesmo); o orador refere-se a si mesmo | a si mesmo; em pessoa; refere-se à própria pessoa ou à pessoa mencionada anteriormente}
  \end{phonetics}
\end{entry}

\begin{entry}{本子}{5,3}{⽊、⼦}
  \begin{phonetics}{本子}{ben3 zi5}[][HSK 1]
    \definition[本]{s.}{caderno}
  \end{phonetics}
\end{entry}

\begin{entry}{本来}{5,7}{⽊、⽊}
  \begin{phonetics}{本来}{ben3lai2}[][HSK 3]
    \definition{adv.}{originalmente | apropriadamente | legalmente}
  \end{phonetics}
\end{entry}

\begin{entry}{本事}{5,8}{⽊、⼅}
  \begin{phonetics}{本事}{ben3shi4}
    \definition{s.}{habilidade | capacidade | \emph{status} | poder | posição | autoridade}
  \end{phonetics}
  \begin{phonetics}{本事}{ben3shi5}[][HSK 3]
    \definition{s.}{habilidade | capacidade |\emph{status} | poder | posição | autoridade}
  \end{phonetics}
\end{entry}

\begin{entry}{本金}{5,8}{⽊、⾦}
  \begin{phonetics}{本金}{ben3 jin1}
    \definition{s.}{capital; capital para a operação do comércio e da indústria; capital para a operação de negócios |
valor principal; dinheiro retirado ao depositar ou tomar emprestado (diferente de ``利息'')}
  \seealsoref{利息}{li4xi1}
  \end{phonetics}
\end{entry}

\begin{entry}{本科}{5,9}{⽊、⽲}
  \begin{phonetics}{本科}{ben3ke1}[][HSK 4]
    \definition{s.}{graduação; bacharelado; o curso básico de uma universidade ou faculdade}
  \end{phonetics}
\end{entry}

\begin{entry}{本领}{5,11}{⽊、⾴}
  \begin{phonetics}{本领}{ben3 ling3}[][HSK 3]
    \definition[项,个]{s.}{capacidade | faculdade | poder | habilidade | talento}
  \end{phonetics}
\end{entry}

\begin{entry}{正}{5}{⽌}
  \begin{phonetics}{正}{zheng1}
    \definition{s.}{o primeiro mês do ano lunar; a primeira lua}
  \end{phonetics}
  \begin{phonetics}{正}{zheng4}[][HSK 1,3]
    \definition*{s.}{sobrenome Zheng}
    \definition{adj.}{reto; ereto; devido; orientação vertical ou padrão | principal; situado no meio; centralizado | anverso; frente | honesto; correto | certo; correto | puro (cor ou sabor); não misturado | padronizado; regular | básico; principal
regular; os comprimentos dos lados e ângulos de uma forma são iguais | positivo; maior que zero; íon positivo; cátion | afiado; exato; usado para tempo, significando naquele ponto ou no meio de um período}
    \definition{adv.}{apenas; exatamente; certo; precisamente | indica o progresso de uma ação ou a continuação de um estado}
    \definition{v.}{definir (colocar) certo; tornar algo reto; não tornar algo torto | retificar; corrigir; ajustar}
  \end{phonetics}
\end{entry}

\begin{entry}{正正}{5,5}{⽌、⽌}
  \begin{phonetics}{正正}{zheng4zheng4}
    \definition{adv.}{na hora certa | ordenadamente}
  \end{phonetics}
\end{entry}

\begin{entry}{正在}{5,6}{⽌、⼟}
  \begin{phonetics}{正在}{zheng4zai4}[][HSK 1]
    \definition{adv.}{no processo de | atualmente | em andamento}
    \definition{v.}{estar a~+~v.inf. | estar~+~v.ger.}
  \end{phonetics}
\end{entry}

\begin{entry}{正好}{5,6}{⽌、⼥}
  \begin{phonetics}{正好}{zheng4hao3}[][HSK 2]
    \definition{adj.}{na medida certa | na hora certa | o suficiente}
    \definition{adv.}{acontecer com | chance de | como acontece}
  \end{phonetics}
\end{entry}

\begin{entry}{正式}{5,6}{⽌、⼷}
  \begin{phonetics}{正式}{zheng4shi4}[][HSK 3]
    \definition{adj.}{formal; oficial; descreve uma atmosfera, atitude ou comportamento sério que não é fácil ou relaxado; descreve o cumprimento de certas formalidades e procedimentos}
  \end{phonetics}
\end{entry}

\begin{entry}{正宗}{5,8}{⽌、⼧}
  \begin{phonetics}{正宗}{zheng4zong1}
    \definition{adj.}{autêntico | genuíno | \emph{old school} | (fig.) tradicional}
  \end{phonetics}
\end{entry}

\begin{entry}{正是}{5,9}{⽌、⽇}
  \begin{phonetics}{正是}{zheng4 shi4}[][HSK 2]
    \definition{adv.}{precisamente | exatamente}
  \end{phonetics}
\end{entry}

\begin{entry}{正常}{5,11}{⽌、⼱}
  \begin{phonetics}{正常}{zheng4chang2}[][HSK 2]
    \definition{adj.}{regular | normal | ordinário}
  \end{phonetics}
\end{entry}

\begin{entry}{正确}{5,12}{⽌、⽯}
  \begin{phonetics}{正确}{zheng4que4}[][HSK 2]
    \definition{adj.}{correto | certo | próprio}
  \end{phonetics}
\end{entry}

\begin{entry}{母亲}{5,9}{⽏、⼇}
  \begin{phonetics}{母亲}{mu3qin1}[][HSK 3]
    \definition[位,个]{s.}{mãe}
  \end{phonetics}
\end{entry}

\begin{entry}{母语}{5,9}{⽏、⾔}
  \begin{phonetics}{母语}{mu3yu3}
    \definition{s.}{língua materna | língua nativa}
  \end{phonetics}
\end{entry}

\begin{entry}{民主}{5,5}{⽒、⼂}
  \begin{phonetics}{民主}{min2zhu3}
    \definition{adj.}{democrático}
    \definition{s.}{democracia}
  \end{phonetics}
\end{entry}

\begin{entry}{民众}{5,6}{⽒、⼈}
  \begin{phonetics}{民众}{min2zhong4}
    \definition{s.}{a população | as massas | as pessoas comuns}
  \end{phonetics}
\end{entry}

\begin{entry}{民间}{5,7}{⽒、⾨}
  \begin{phonetics}{民间}{min2jian1}[][HSK 3]
    \definition{s.}{povo (entre as pessoas) | não governamental; de pessoa para pessoa}
  \end{phonetics}
\end{entry}

\begin{entry}{民族}{5,11}{⽒、⽅}
  \begin{phonetics}{民族}{min2zu2}[][HSK 3]
    \definition[个]{s.}{nação | grupo étnico}
  \end{phonetics}
\end{entry}

\begin{entry}{永不}{5,4}{⽔、⼀}
  \begin{phonetics}{永不}{yong3bu4}
    \definition{adv.}{nunca}
  \end{phonetics}
\end{entry}

\begin{entry}{永远}{5,7}{⽔、⾡}
  \begin{phonetics}{永远}{yong3yuan3}[][HSK 2]
    \definition{adv.}{para sempre, sempre | permanentemente}
  \end{phonetics}
\end{entry}

\begin{entry}{汇}{5}{⽔}
  \begin{phonetics}{汇}{hui4}[][HSK 4]
    \definition{s.}{coisas coletadas; conjunto; coleção}
    \definition{v.}{convergir | reunir-se | remeter; transferir por meio de agências postais e telegráficas, bancos}
  \end{phonetics}
\end{entry}

\begin{entry}{汇报}{5,7}{⽔、⼿}
  \begin{phonetics}{汇报}{hui4bao4}[][HSK 4]
    \definition[份,次]{s.}{relatório; referindo-se ao conteúdo de declarações escritas ou orais feitas a um superior ou pessoa relevante para apresentar uma situação ou refletir um problema}
    \definition{v.}{relatar; fazer um relato de}
  \end{phonetics}
\end{entry}

\begin{entry}{汇率}{5,11}{⽔、⽞}
  \begin{phonetics}{汇率}{hui4lv4}[][HSK 4]
    \definition[个]{s.}{taxa de câmbio; relação entre a moeda de um país e a de outro}
  \end{phonetics}
\end{entry}

\begin{entry}{汇款}{5,12}{⽔、⽋}
  \begin{phonetics}{汇款}{hui4 kuan3}[][HSK 5]
    \definition[笔,个]{s.}{remessa; dinheiro enviado ou recebido}
    \definition{v.+compl.}{remeter dinheiro; fazer uma remessa; enviar dinheiro}
  \end{phonetics}
\end{entry}

\begin{entry}{汉}{5}{⽔}
  \begin{phonetics}{汉}{han4}
    \definition{s.}{grupo étnico Han | chinês (língua) | dinastia Han (206 a.C.-220d.C.) | homem}
  \end{phonetics}
\end{entry}

\begin{entry}{汉字}{5,6}{⽔、⼦}
  \begin{phonetics}{汉字}{han4 zi4}[][HSK 1]
    \definition[个]{s.}{caracter chinês}
  \end{phonetics}
\end{entry}

\begin{entry}{汉服}{5,8}{⽔、⽉}
  \begin{phonetics}{汉服}{han4fu2}
    \definition{s.}{vestido chinês tradicional Han}
  \end{phonetics}
\end{entry}

\begin{entry}{汉语}{5,9}{⽔、⾔}
  \begin{phonetics}{汉语}{han4yu3}[][HSK 1]
    \definition[门]{s.}{língua chinesa, mandarim}
  \end{phonetics}
\end{entry}

\begin{entry}{汉堡王}{5,12,4}{⽔、⼟、⽟}
  \begin{phonetics}{汉堡王}{han4bao3wang2}
    \definition*{s.}{Burguer King (restaurante \emph{fast-food})}
  \end{phonetics}
\end{entry}

\begin{entry}{汉堡包}{5,12,5}{⽔、⼟、⼓}
  \begin{phonetics}{汉堡包}{han4bao3bao1}
    \definition[个]{s.}{hambúrguer}
  \end{phonetics}
\end{entry}

\begin{entry}{汉葡词典}{5,12,7,8}{⽔、⾋、⾔、⼋}
  \begin{phonetics}{汉葡词典}{han4-pu2 ci2dian3}
    \definition[部,本]{s.}{dicionário chinês-português}
  \seealsoref{葡汉词典}{pu2-han4 ci2dian3}
  \end{phonetics}
\end{entry}

\begin{entry}{灭火}{5,4}{⽕、⽕}
  \begin{phonetics}{灭火}{mie4huo3}
    \definition{s.}{combate a incêndios}
    \definition{v.}{extinguir um incêndio}
  \end{phonetics}
\end{entry}

\begin{entry}{犯法}{5,8}{⽝、⽔}
  \begin{phonetics}{犯法}{fan4fa3}
    \definition{v.}{violar (quebrar) a lei}
  \end{phonetics}
\end{entry}

\begin{entry}{犯罪}{5,13}{⽝、⽹}
  \begin{phonetics}{犯罪}{fan4zui4}
    \definition{v.+compl.}{cometer  um crime (uma ofensa)}
  \end{phonetics}
\end{entry}

\begin{entry}{玄学}{5,8}{⽞、⼦}
  \begin{phonetics}{玄学}{xuan2xue2}
    \definition{s.}{Escola Philosófica Wei e Jin amalgamando os ideais daoísta e confucionistas | tradução da metafísica (形而上学)}
    \seeref{形而上学}{xing2'er2shang4xue2}
  \end{phonetics}
\end{entry}

\begin{entry}{玉}{5}{⽟}[Kangxi 96]
  \begin{phonetics}{玉}{yu4}[][HSK 4]
    \definition*{s.}{sobrenome Yu}
    \definition{adj.}{(pessoa, especialmente uma mulher) pura; justa; bonita; bela | cristalino, branco e belo como o jade | (vida) rica; luxuosa}
    \definition{pron.}{seu; um termo de respeito, usado para honrar o corpo, as ações ou as coisas associadas à outra pessoa}
    \definition[块]{s.}{jade}
  \end{phonetics}
\end{entry}

\begin{entry}{玉米}{5,6}{⽟、⽶}
  \begin{phonetics}{玉米}{yu4mi3}[][HSK 4]
    \definition[个,株,粒]{s.}{milho}
  \end{phonetics}
\end{entry}

\begin{entry}{玉米片}{5,6,4}{⽟、⽶、⽚}
  \begin{phonetics}{玉米片}{yu4mi3pian4}
    \definition{s.}{flocos de milho | chips de tortilha}
  \end{phonetics}
\end{entry}

\begin{entry}{玉米花}{5,6,7}{⽟、⽶、⾋}
  \begin{phonetics}{玉米花}{yu4mi3hua1}
    \definition{s.}{pipoca}
  \end{phonetics}
\end{entry}

\begin{entry}{玉米面}{5,6,9}{⽟、⽶、⾯}
  \begin{phonetics}{玉米面}{yu4mi3mian4}
    \definition{s.}{fubá | farinha de milho}
  \end{phonetics}
\end{entry}

\begin{entry}{玉米饼}{5,6,9}{⽟、⽶、⾷}
  \begin{phonetics}{玉米饼}{yu4mi3bing3}
    \definition{s.}{tortilha mexicana | bolo de milho}
  \end{phonetics}
\end{entry}

\begin{entry}{玉米笋}{5,6,10}{⽟、⽶、⽵}
  \begin{phonetics}{玉米笋}{yu4mi3sun3}
    \definition{s.}{broto de milho}
  \end{phonetics}
\end{entry}

\begin{entry}{玉米粉}{5,6,10}{⽟、⽶、⽶}
  \begin{phonetics}{玉米粉}{yu4mi3fen3}
    \definition{s.}{amido de milho | farinha de milho}
  \end{phonetics}
\end{entry}

\begin{entry}{玉米糁}{5,6,14}{⽟、⽶、⽶}
  \begin{phonetics}{玉米糁}{yu4mi3san3}
    \definition{s.}{grãos de milho}
  \end{phonetics}
\end{entry}

\begin{entry}{玉米糕}{5,6,16}{⽟、⽶、⽶}
  \begin{phonetics}{玉米糕}{yu4mi3gao1}
    \definition{s.}{bolo de milho | polenta}
  \end{phonetics}
\end{entry}

\begin{entry}{瓜}{5}{⽠}[Kangxi 97]
  \begin{phonetics}{瓜}{gua1}[][HSK 4]
    \definition*{s.}{sobrenome Gua}
    \definition[个]{s.}{qualquer tipo de melão ou cabaça | companheiro (termo depreciativo para uma pessoa)}
  \end{phonetics}
\end{entry}

\begin{entry}{甘心}{5,4}{⽢、⼼}
  \begin{phonetics}{甘心}{gan1xin1}
    \definition{v.}{estar disposto a | resignar-se a}
  \end{phonetics}
\end{entry}

\begin{entry}{甘薯}{5,16}{⽢、⾋}
  \begin{phonetics}{甘薯}{gan1shu3}
    \definition{s.}{batata doce}
  \end{phonetics}
\end{entry}

\begin{entry}{生}{5}{⽣}[Kangxi 100]
  \begin{phonetics}{生}{sheng1}[][HSK 2,3]
    \definition*{s.}{sobrenome Sheng}
    \definition{adj.}{vivo | imaturo; verde | cru; não cozido | não processado; não refinado; bruto | desconhecido; estranho | rígido; mecânico}
    \definition{adv.}{muito | usado antes de certas palavras que expressam emoções ou sentimentos}
    \definition{s.}{vida | meio de vida; sustento | aluno; estudante; pupilo | erudito | o tipo de personagem masculino na ópera de Pequim, etc.}
    \definition{suf.}{certos sufixos substantivos que se referem a pessoas (学生) | sufixos de certos advérbios (好生)}
    \definition{v.}{dar à luz; suportar | nascer | crescer | viver; existir | obter; ter | acender (um fogo)}
  \seealsoref{好生}{hao3sheng1}
  \seealsoref{学生}{xue2sheng5}
  \end{phonetics}
\end{entry}

\begin{entry}{生日}{5,4}{⽣、⽇}
  \begin{phonetics}{生日}{sheng1ri4}[][HSK 1]
    \definition[个]{s.}{aniversário}
  \end{phonetics}
\end{entry}

\begin{entry}{生气}{5,4}{⽣、⽓}
  \begin{phonetics}{生气}{sheng1 qi4}[][HSK 1]
    \definition{s.}{vitalidade | vigor}
    \definition{v.+compl.}{irritar-se | zangar-se | ofender-se | ficar com raiva}
  \end{phonetics}
\end{entry}

\begin{entry}{生长}{5,4}{⽣、⾧}
  \begin{phonetics}{生长}{sheng1zhang3}[][HSK 3]
    \definition{v.}{cresçer | nascer e crescer}
  \end{phonetics}
\end{entry}

\begin{entry}{生产}{5,6}{⽣、⼇}
  \begin{phonetics}{生产}{sheng1chan3}[][HSK 3]
    \definition{v.}{produzir; fabricar | dar à luz uma criança}
  \end{phonetics}
\end{entry}

\begin{entry}{生动}{5,6}{⽣、⼒}
  \begin{phonetics}{生动}{sheng1dong4}[][HSK 3]
    \definition{adj.}{vívido; animado}
  \end{phonetics}
\end{entry}

\begin{entry}{生存}{5,6}{⽣、⼦}
  \begin{phonetics}{生存}{sheng1cun2}[][HSK 3]
    \definition{v.}{viver; sobreviver; subsistir}
  \end{phonetics}
\end{entry}

\begin{entry}{生成}{5,6}{⽣、⼽}
  \begin{phonetics}{生成}{sheng1 cheng2}[][HSK 5]
    \definition{v.}{formar; gerar; produzir | ter por natureza; nascer com}
  \end{phonetics}
\end{entry}

\begin{entry}{生词}{5,7}{⽣、⾔}
  \begin{phonetics}{生词}{sheng1 ci2}[][HSK 2]
    \definition[个]{s.}{nova palavra}
  \end{phonetics}
\end{entry}

\begin{entry}{生命}{5,8}{⽣、⼝}
  \begin{phonetics}{生命}{sheng1ming4}[][HSK 3]
    \definition{s.}{vida | não envolve apenas a existência e as atividades dos organismos, mas também inclui experiências de vida humana, valores e elementos-chave da sobrevivência e do desenvolvimento de várias coisas}
  \end{phonetics}
\end{entry}

\begin{entry}{生态}{5,8}{⽣、⼼}
  \begin{phonetics}{生态}{sheng1tai4}
    \definition{adj.}{ecológico}
    \definition{s.}{ecologia}
  \end{phonetics}
\end{entry}

\begin{entry}{生物}{5,8}{⽣、⽜}
  \begin{phonetics}{生物}{sheng1wu4}
    \definition{adj.}{biológico}
    \definition{s.}{biologia (disciplina) | organismo | ser vivo}
  \end{phonetics}
\end{entry}

\begin{entry}{生的}{5,8}{⽣、⽩}
  \begin{phonetics}{生的}{sheng1de5}
    \definition{conj.}{para evitar isso | para que\dots não\dots}
  \end{phonetics}
\end{entry}

\begin{entry}{生鱼片}{5,8,4}{⽣、⿂、⽚}
  \begin{phonetics}{生鱼片}{sheng1yu2pian4}
    \definition{s.}{fatias de peixe cru, \emph{sashimi}}
  \end{phonetics}
\end{entry}

\begin{entry}{生活}{5,9}{⽣、⽔}
  \begin{phonetics}{生活}{sheng1huo2}[][HSK 2]
    \definition[道]{s.}{vida | atividade | meios de subsistência}
    \definition{v.}{viver}
  \end{phonetics}
\end{entry}

\begin{entry}{生活垃圾}{5,9,8,6}{⽣、⽔、⼟、⼟}
  \begin{phonetics}{生活垃圾}{sheng1huo2la1ji1}
    \definition{s.}{lixo doméstico}
  \end{phonetics}
\end{entry}

\begin{entry}{生活型}{5,9,9}{⽣、⽔、⼟}
  \begin{phonetics}{生活型}{sheng1huo2 xing2}
    \definition{s.}{forma de vida}
  \end{phonetics}
\end{entry}

\begin{entry}{生病}{5,10}{⽣、⽧}
  \begin{phonetics}{生病}{sheng1 bing4}[][HSK 1]
    \definition{v.}{ficar doente | estar adoecido}
  \end{phonetics}
\end{entry}

\begin{entry}{生理}{5,11}{⽣、⽟}
  \begin{phonetics}{生理}{sheng1li3}
    \definition{adj.}{fisiológico}
    \definition{s.}{fisiologia}
  \end{phonetics}
\end{entry}

\begin{entry}{生菜}{5,11}{⽣、⾋}
  \begin{phonetics}{生菜}{sheng1cai4}
    \definition{s.}{alface}
  \end{phonetics}
\end{entry}

\begin{entry}{生意}{5,13}{⽣、⼼}
  \begin{phonetics}{生意}{sheng1yi4}
    \definition{s.}{tendência a crescer; vida e vitalidade}
  \end{phonetics}
  \begin{phonetics}{生意}{sheng1yi5}[][HSK 3]
    \definition[笔,种,次]{s.}{comércio; negócios}
  \end{phonetics}
\end{entry}

\begin{entry}{用}{5}{⽤}[Kangxi 101]
  \begin{phonetics}{用}{yong4}[][HSK 1]
    \definition{v.}{usar}
  \end{phonetics}
\end{entry}

\begin{entry}{用心}{5,4}{⽤、⼼}
  \begin{phonetics}{用心}{yong4xin1}
    \definition{s.}{motivo | intenção}
    \definition{v.+compl.}{ser diligente ou atencioso}
  \end{phonetics}
\end{entry}

\begin{entry}{用处}{5,5}{⽤、⼡}
  \begin{phonetics}{用处}{yong4chu5}
    \definition[个]{s.}{usabilidade | utilidade}
  \end{phonetics}
\end{entry}

\begin{entry}{用料}{5,10}{⽤、⽃}
  \begin{phonetics}{用料}{yong4liao4}
    \definition{s.}{ingredientes | materiais}
  \end{phonetics}
\end{entry}

\begin{entry}{用途}{5,10}{⽤、⾡}
  \begin{phonetics}{用途}{yong4tu2}[][HSK 4]
    \definition[个,种]{s.}{uso; aplicação; aspectos ou escopo da aplicação}
  \end{phonetics}
\end{entry}

\begin{entry}{田}{5}{⽥}[Kangxi 102]
  \begin{phonetics}{田}{tian2}
    \definition*{s.}{sobrenome Tian}
    \definition[片]{s.}{fazenda | campo}
  \end{phonetics}
\end{entry}

\begin{entry}{田园}{5,7}{⽥、⼞}
  \begin{phonetics}{田园}{tian2yuan2}
    \definition{adj.}{bucólico}
    \definition{s.}{campo | interior | rural}
  \end{phonetics}
\end{entry}

\begin{entry}{由}{5}{⽥}
  \begin{phonetics}{由}{you2}[][HSK 3]
    \definition*{s.}{sobrenome You}
    \definition{prep.}{por causa de; devido a | por; através de | indica que alguém é responsável por fazer algo | indica dependência de | de; indica o ponto de partida}
    \definition[个]{s.}{causa; razão}
    \definition{v.}{atravessar | seguir; obedecer}
  \end{phonetics}
\end{entry}

\begin{entry}{由于}{5,3}{⽥、⼆}
  \begin{phonetics}{由于}{you2yu2}[][HSK 3]
    \definition{conj.}{porque; uma vez que; visto que; porque, usado no início da frase anterior para expressar a causa, e na frase seguinte para expressar o resultado.}
    \definition{prep.}{devido a; graças a; por causa de; em virtude de; como resultado de; apresentar as razões pelas quais um evento ou ação ocorre}
  \end{phonetics}
\end{entry}

\begin{entry}{甲}{5}{⽥}
  \begin{phonetics}{甲}{jia3}[][HSK 5]
    \definition*{s.}{sobrenome Jia}
    \definition{s.}{alfa; primeiro lugar; o primeiro dos caules celestiais, geralmente usado para indicar o primeiro em ordem ou classificação | concha; carapaça; crustáceos | unha; crostas queratinosas nos dedos das mãos e dos pés | armadura; equipamento de proteção feito de metal | unidade de administração civil composta por 10 residências | uma palavra substituta para uma pessoa ou coisa indefinida; usado como pronome |}
    \definition{v.}{ocupar o primeiro lugar; ser melhor do que}
  \end{phonetics}
\end{entry}

\begin{entry}{甲骨文}{5,9,4}{⽥、⾻、⽂}
  \begin{phonetics}{甲骨文}{jia3gu3wen2}
    \definition{s.}{escrituras de oráculos | inscrições em ossos de oráculos (forma original de escritura chinesa)}
  \end{phonetics}
\end{entry}

\begin{entry}{申请}{5,10}{⽥、⾔}
  \begin{phonetics}{申请}{shen1qing3}[][HSK 4]
    \definition[份,批,项]{s.}{pedido de; solicitação de; requerimento para; solicitações escritas para serem mostradas a superiores ou autoridades}
    \definition{v.}{solicitar; apresentar uma solicitação; fazer representações e solicitações a uma autoridade superior ou às autoridades relevantes}
  \end{phonetics}
\end{entry}

\begin{entry}{电}{5}{⽥}
  \begin{phonetics}{电}{dian4}[][HSK 1]
    \definition{s.}{eletricidade | telegrama | cabo}
    \definition{v.}{dar ou receber um choque elétrico | enviar um telegrama | telegrafar}
  \end{phonetics}
\end{entry}

\begin{entry}{电子}{5,3}{⽥、⼦}
  \begin{phonetics}{电子}{dian4zi3}
    \definition{s.}{eletrônico | elétron}
  \end{phonetics}
\end{entry}

\begin{entry}{电子名片}{5,3,6,4}{⽥、⼦、⼝、⽚}
  \begin{phonetics}{电子名片}{dian4zi3 ming2pian4}
    \definition{s.}{cartão de visita eletrônico}
  \end{phonetics}
\end{entry}

\begin{entry}{电子邮件}{5,3,7,6}{⽥、⼦、⾢、⼈}
  \begin{phonetics}{电子邮件}{dian4zi3you2jian4}[][HSK 3]
    \definition[封,份]{s.}{correio eletrônico; \emph{e-mail}}
  \seealsoref{电邮}{dian4you2}
  \end{phonetics}
\end{entry}

\begin{entry}{电子版}{5,3,8}{⽥、⼦、⽚}
  \begin{phonetics}{电子版}{dian4 zi3 ban3}[][HSK 5]
    \definition{s.}{edição eletrônica}
  \end{phonetics}
\end{entry}

\begin{entry}{电车司机}{5,4,5,6}{⽥、⾞、⼝、⽊}
  \begin{phonetics}{电车司机}{dian4che1 si1ji1}
    \definition{s.}{motorista de bonde}
  \end{phonetics}
\end{entry}

\begin{entry}{电台}{5,5}{⽥、⼝}
  \begin{phonetics}{电台}{dian4 tai2}[][HSK 3]
    \definition[个,家]{s.}{transceptor; transmissor-receptor | aparelho de rádio; estação de rádio; estação de transmissão}
  \end{phonetics}
\end{entry}

\begin{entry}{电冰箱}{5,6,15}{⽥、⼎、⾋}
  \begin{phonetics}{电冰箱}{dian4bing1xiang1}
    \definition[台]{s.}{frigorífico | refrigerador}
  \end{phonetics}
\end{entry}

\begin{entry}{电动}{5,6}{⽥、⼒}
  \begin{phonetics}{电动}{dian4dong4}
    \definition{adj.}{movido a eletricidade | elétrico}
  \end{phonetics}
\end{entry}

\begin{entry}{电动车}{5,6,4}{⽥、⼒、⾞}
  \begin{phonetics}{电动车}{dian4 dong4 che1}[][HSK 4]
    \definition{s.}{veículo elétrico (\emph{scooter}, bicicleta, carro, etc.)}
  \end{phonetics}
\end{entry}

\begin{entry}{电池}{5,6}{⽥、⽔}
  \begin{phonetics}{电池}{dian4chi2}[][HSK 5]
    \definition[节,块,组]{s.}{célula; bateria}
  \end{phonetics}
\end{entry}

\begin{entry}{电灯}{5,6}{⽥、⽕}
  \begin{phonetics}{电灯}{dian4 deng1}[][HSK 4]
    \definition[盏,个]{s.}{luz elétrica; lâmpada elétrica; lâmpadas que usam eletricidade como fonte de energia}
  \end{phonetics}
\end{entry}

\begin{entry}{电灯泡}{5,6,8}{⽥、⽕、⽔}
  \begin{phonetics}{电灯泡}{dian4deng1pao4}
    \definition{s.}{lâmpada elétrica | (gíria) terceiro convidado indesejado}
  \end{phonetics}
\end{entry}

\begin{entry}{电邮}{5,7}{⽥、⾢}
  \begin{phonetics}{电邮}{dian4you2}
    \definition{s.}{correio eletrônico, \emph{e-mail} | abreviação de~电子邮件}
  \seealsoref{电子邮件}{dian4zi3you2jian4}
  \end{phonetics}
\end{entry}

\begin{entry}{电饭锅}{5,7,12}{⽥、⾷、⾦}
  \begin{phonetics}{电饭锅}{dian4 fan4 guo1}[][HSK 5]
    \definition{s.}{panela elétrica de arroz}
  \end{phonetics}
\end{entry}

\begin{entry}{电视}{5,8}{⽥、⾒}
  \begin{phonetics}{电视}{dian4shi4}[][HSK 1]
    \definition[台,个]{s.}{televisão | TV | televisor}
  \end{phonetics}
\end{entry}

\begin{entry}{电视台}{5,8,5}{⽥、⾒、⼝}
  \begin{phonetics}{电视台}{dian4 shi4 tai2}[][HSK 3]
    \definition[个]{s.}{canal de TV | estação de televisão}
  \end{phonetics}
\end{entry}

\begin{entry}{电视机}{5,8,6}{⽥、⾒、⽊}
  \begin{phonetics}{电视机}{dian4 shi4 ji1}[][HSK 1]
    \definition[台]{s.}{aparelho de televisão | televisor}
  \end{phonetics}
\end{entry}

\begin{entry}{电视剧}{5,8,10}{⽥、⾒、⼑}
  \begin{phonetics}{电视剧}{dian4 shi4 ju4}[][HSK 3]
    \definition[部]{s.}{série de TV; drama de TV; novela}
  \end{phonetics}
\end{entry}

\begin{entry}{电话}{5,8}{⽥、⾔}
  \begin{phonetics}{电话}{dian4 hua4}[][HSK 1]
    \definition[部]{s.}{telefone}
    \definition[通]{s.}{chamada telefônica}
  \end{phonetics}
\end{entry}

\begin{entry}{电脑}{5,10}{⽥、⾁}
  \begin{phonetics}{电脑}{dian4nao3}[][HSK 1]
    \definition[台]{s.}{computador}
  \end{phonetics}
\end{entry}

\begin{entry}{电脑语言}{5,10,9,7}{⽥、⾁、⾔、⾔}
  \begin{phonetics}{电脑语言}{dian4nao3yu3yan2}
    \definition{s.}{linguagem de programação | linguagem de computador}
  \end{phonetics}
\end{entry}

\begin{entry}{电梯}{5,11}{⽥、⽊}
  \begin{phonetics}{电梯}{dian4ti1}[][HSK 4]
    \definition[部,台,架]{s.}{elevador}
  \end{phonetics}
\end{entry}

\begin{entry}{电梯司机}{5,11,5,6}{⽥、⽊、⼝、⽊}
  \begin{phonetics}{电梯司机}{dian4ti1 si1ji1}
    \definition{s.}{ascensorista}
  \end{phonetics}
\end{entry}

\begin{entry}{电源}{5,13}{⽥、⽔}
  \begin{phonetics}{电源}{dian4yuan2}[][HSK 4]
    \definition{s.}{fonte de alimentação; fonte de energia; fonte de energia elétrica; dispositivo que fornece energia elétrica a um aparelho, como uma bateria, um gerador, etc.}
  \end{phonetics}
\end{entry}

\begin{entry}{电影}{5,15}{⽥、⼺}
  \begin{phonetics}{电影}{dian4ying3}[][HSK 1]
    \definition[部,片,幕,场]{s.}{filme}
  \end{phonetics}
\end{entry}

\begin{entry}{电影艺术}{5,15,4,5}{⽥、⼺、⾋、⽊}
  \begin{phonetics}{电影艺术}{dian4ying3 yi4shu4}
    \definition{s.}{arte cinematográfica}
  \end{phonetics}
\end{entry}

\begin{entry}{电影术}{5,15,5}{⽥、⼺、⽊}
  \begin{phonetics}{电影术}{dian4ying3 shu4}
    \definition{s.}{cinematografia}
  \end{phonetics}
\end{entry}

\begin{entry}{电影节}{5,15,5}{⽥、⼺、⾋}
  \begin{phonetics}{电影节}{dian4ying3jie2}
    \definition{s.}{festival de cinema}
  \end{phonetics}
\end{entry}

\begin{entry}{电影奖}{5,15,9}{⽥、⼺、⼤}
  \begin{phonetics}{电影奖}{dian4ying3jiang3}
    \definition{s.}{premiações de cinema}
  \end{phonetics}
\end{entry}

\begin{entry}{电影界}{5,15,9}{⽥、⼺、⽥}
  \begin{phonetics}{电影界}{dian4ying3jie4}
    \definition{s.}{indústria cinematográfica}
  \end{phonetics}
\end{entry}

\begin{entry}{电影院}{5,15,9}{⽥、⼺、⾩}
  \begin{phonetics}{电影院}{dian4 ying3 yuan4}[][HSK 1]
    \definition[次,家,座]{s.}{sala de cinema}
  \end{phonetics}
\end{entry}

\begin{entry}{电影音乐}{5,15,9,5}{⽥、⼺、⾳、⼃}
  \begin{phonetics}{电影音乐}{dian4ying3 yin1yue4}
    \definition{s.}{música cinematográfica}
  \end{phonetics}
\end{entry}

\begin{entry}{电影票}{5,15,11}{⽥、⼺、⽰}
  \begin{phonetics}{电影票}{dian4ying3piao4}
    \definition{s.}{ingresso de filme}
  \end{phonetics}
\end{entry}

\begin{entry}{电器}{5,16}{⽥、⼝}
  \begin{phonetics}{电器}{dian4qi4}
    \definition{s.}{aparelho elétrico}
  \end{phonetics}
\end{entry}

\begin{entry}{白}{5}{⽩}[Kangxi 106]
  \begin{phonetics}{白}{bai2}[][HSK 1,3]
    \definition*{s.}{sobrenome Bai}
    \definition{adj.}{branco | claro | puro; claro; simples; sem mistura; em branco | branco (como símbolo de reação) | escrito incorretamente ou pronunciado incorretamente | grátis; sem custos}
    \definition{adv.}{em vão; sem propósito; sem resultados}
    \definition{s.}{parte falada em ópera, etc.; frases de peças de teatro, etc. | dialeto local | funeral}
    \definition{v.}{explicar; apresentar; esclarecer; declarar | branquear | olhar para as pessoas com o branco dos olhos (olhar vazio, de desaprovação)}
  \end{phonetics}
\end{entry}

\begin{entry}{白天}{5,4}{⽩、⼤}
  \begin{phonetics}{白天}{bai2 tian1}[][HSK 1]
    \definition{adv.}{dia | de dia}
    \definition[个]{s.}{dia}
  \end{phonetics}
\end{entry}

\begin{entry}{白色}{5,6}{⽩、⾊}
  \begin{phonetics}{白色}{bai2 se4}[][HSK 2]
    \definition{s.}{cor branca}
  \end{phonetics}
\end{entry}

\begin{entry}{白苋}{5,7}{⽩、⾋}
  \begin{phonetics}{白苋}{bai2xian4}
    \definition{s.}{amaranto branco | brotos e folhas tenras de espinafre chinês usados como alimento}
  \end{phonetics}
\end{entry}

\begin{entry}{白拣}{5,8}{⽩、⼿}
  \begin{phonetics}{白拣}{bai2jian3}
    \definition{s.}{uma escolha barata}
    \definition{v.}{escolher algo que não custa nada}
  \end{phonetics}
\end{entry}

\begin{entry}{白酒}{5,10}{⽩、⾣}
  \begin{phonetics}{白酒}{bai2 jiu3}[][HSK 5]
    \definition{s.}{aguardente branca; aguardente (geralmente destilada de sorgo ou milho); bebidas destiladas tradicionais chinesas, feitas de sorgo, milho, etc., transparentes e incolores, com alto teor alcoólico}
  \end{phonetics}
\end{entry}

\begin{entry}{白菜}{5,11}{⽩、⾋}
  \begin{phonetics}{白菜}{bai2 cai4}[][HSK 3]
    \definition[棵,个]{s.}{acelga | repolho chinês}
  \end{phonetics}
\end{entry}

\begin{entry}{白萝卜}{5,11,2}{⽩、⾋、⼘}
  \begin{phonetics}{白萝卜}{bai2luo2bo5}
    \definition{s.}{rabanete branco}
  \end{phonetics}
\end{entry}

\begin{entry}{白蛋白}{5,11,5}{⽩、⾍、⽩}
  \begin{phonetics}{白蛋白}{bai2dan4bai2}
    \definition{s.}{albumina}
  \end{phonetics}
\end{entry}

\begin{entry}{白鹄}{5,12}{⽩、⿃}
  \begin{phonetics}{白鹄}{bai2hu2}
    \definition{s.}{cisne branco}
  \end{phonetics}
\end{entry}

\begin{entry}{白痴}{5,13}{⽩、⽧}
  \begin{phonetics}{白痴}{bai2chi1}
    \definition{adj./s.}{estúpido | imbecil}
  \end{phonetics}
\end{entry}

\begin{entry}{皮}{5}{⽪}[Kangxi 107]
  \begin{phonetics}{皮}{pi2}[][HSK 3]
    \definition*{s.}{sobrenome Pi}
    \definition{adj.}{macios e encharcados; não mais crocantes | danadinho; travesso | endurecido; não se importa mais}
    \definition{pref.}{pico- (um trilhonésimo)}
    \definition[张]{s.}{pele | couro cru; couro | pelagem | capa; envoltório | superfície | uma peça larga e plana (de algum material fino) | borracha}
  \end{phonetics}
\end{entry}

\begin{entry}{皮下}{5,3}{⽪、⼀}
  \begin{phonetics}{皮下}{pi2xia4}
    \definition{adj.}{(injeção) subcutâneo | sob a pele}
  \end{phonetics}
\end{entry}

\begin{entry}{皮包}{5,5}{⽪、⼓}
  \begin{phonetics}{皮包}{pi2 bao1}[][HSK 3]
    \definition[个,只,款]{s.}{bolsa; pasta; portfólio}
  \end{phonetics}
\end{entry}

\begin{entry}{皮卡}{5,5}{⽪、⼘}
  \begin{phonetics}{皮卡}{pi2ka3}
    \definition{s.}{(empréstimo linguístico) \emph{pick-up} | caminhonete}
  \end{phonetics}
\end{entry}

\begin{entry}{皮卡丘}{5,5,5}{⽪、⼘、⼀}
  \begin{phonetics}{皮卡丘}{pi2ka3qiu1}
    \definition*{s.}{\emph{Pikachu} (Pokémon, 口袋妖怪)}
  \seealsoref{口袋妖怪}{kou3dai4 yao1guai4}
  \end{phonetics}
\end{entry}

\begin{entry}{皮肤}{5,8}{⽪、⾁}
  \begin{phonetics}{皮肤}{pi2fu1}[][HSK 5]
    \definition{adj.}{superficial}
    \definition[层,块]{s.}{pele; couro; derme}
  \end{phonetics}
\end{entry}

\begin{entry}{皮鞋}{5,15}{⽪、⾰}
  \begin{phonetics}{皮鞋}{pi2xie2}[][HSK 5]
    \definition[双,只,款]{s.}{sapatos feitos de couro}
  \end{phonetics}
\end{entry}

\begin{entry}{目光}{5,6}{⽬、⼉}
  \begin{phonetics}{目光}{mu4guang1}[][HSK 5]
    \definition[道,束,种]{s.}{olhar fixo; a expressão e atitude reveladas pelos olhos | visão; vista; percepção visual; a linha imaginária formada entre os olhos e o objeto quando se olha para ele | perspicácia (capacidade de observar e reconhecer coisas); conhecimento adquirido através do contato com as coisas, capacidade de observar as coisas}
  \end{phonetics}
\end{entry}

\begin{entry}{目的}{5,8}{⽬、⽩}
  \begin{phonetics}{目的}{mu4di4}[][HSK 2]
    \definition[个]{s.}{objetivo | meta | alvo | propósito}
  \end{phonetics}
\end{entry}

\begin{entry}{目前}{5,9}{⽬、⼑}
  \begin{phonetics}{目前}{mu4qian2}[][HSK 3]
    \definition{adv.}{agora; recentemente; no momento; no presente}
  \end{phonetics}
\end{entry}

\begin{entry}{目标}{5,9}{⽬、⽊}
  \begin{phonetics}{目标}{mu4biao1}[][HSK 3]
    \definition[个]{s.}{alvo; objetivo | objetivo; destino}
  \end{phonetics}
\end{entry}

\begin{entry}{矛}{5}{⽭}[Kangxi 110]
  \begin{phonetics}{矛}{mao2}
    \definition{s.}{lança; lanceta}
  \end{phonetics}
\end{entry}

\begin{entry}{矛盾}{5,9}{⽭、⽬}
  \begin{phonetics}{矛盾}{mao2dun4}[][HSK 5]
    \definition{adj.}{contraditório; descreve pessoas ou coisas que se opõem ou se repelem mutuamente}
    \definition{s.}{problema; contradição; discrepância; inconsistência | disputas e conflitos; relacionamento de oposição entre as duas partes devido a diferenças de opinião ou abordagem}
    \definition{v.}{opor-se; entrar em conflito; contradizer; nesta situação, apenas uma das opções está correta ou é verdadeira; não é possível que ambas estejam corretas ao mesmo tempo}
  \end{phonetics}
\end{entry}

\begin{entry}{石头}{5,5}{⽯、⼤}
  \begin{phonetics}{石头}{shi2tou5}[][HSK 3]
    \definition[块,堆,些]{s.}{rocha; pedra}
  \end{phonetics}
\end{entry}

\begin{entry}{石油}{5,8}{⽯、⽔}
  \begin{phonetics}{石油}{shi2you2}[][HSK 3]
    \definition[桶,吨,升]{s.}{óleo; óleo fóssil; petróleo}
  \end{phonetics}
\end{entry}

\begin{entry}{示范}{5,9}{⽰、⾋}
  \begin{phonetics}{示范}{shi4fan4}[][HSK 5]
    \definition{v.}{demonstrar; dar o exemplo; criar um modelo que todos possam aprender}
  \end{phonetics}
\end{entry}

\begin{entry}{礼}{5}{⽰}
  \begin{phonetics}{礼}{li3}[][HSK 5]
    \definition*{s.}{sobrenome Li}
    \definition[份]{s.}{observâncias cerimoniais em geral; cerimônia; rito | cortesia; etiqueta; boas maneiras | presente; oferta}
  \end{phonetics}
\end{entry}

\begin{entry}{礼节}{5,5}{⽰、⾋}
  \begin{phonetics}{礼节}{li3jie2}
    \definition{s.}{protocolo | cerimônia | etiqueta}
  \end{phonetics}
\end{entry}

\begin{entry}{礼让}{5,5}{⽰、⾔}
  \begin{phonetics}{礼让}{li3rang4}
    \definition{s.}{cortesia}
    \definition{v.}{mostrar consideração por (outros) | ceder a (outro veículo, etc.)}
  \end{phonetics}
\end{entry}

\begin{entry}{礼物}{5,8}{⽰、⽜}
  \begin{phonetics}{礼物}{li3wu4}[][HSK 2]
    \definition[件,个,份]{s.}{prenda | lembrança | presente}
  \end{phonetics}
\end{entry}

\begin{entry}{礼拜}{5,9}{⽰、⼿}
  \begin{phonetics}{礼拜}{li3 bai4}[][HSK 5]
    \definition[个]{s.}{dia da semana; usado em conjunto com ``一、二、三、四、五、六、日(或天)'', indica um dia específico da semana | semana; referência à semana | domingo}
    \definition{v.}{prestar homenagem aos deuses que veneram; rezar; orar}
  \end{phonetics}
\end{entry}

\begin{entry}{礼貌}{5,14}{⽰、⾘}
  \begin{phonetics}{礼貌}{li3mao4}[][HSK 5]
    \definition{adj.}{educado; descreve uma pessoa que fala e age respeitando os outros, sem arrogância, de acordo com as exigências das relações sociais}
    \definition{s.}{cortesia; educação; boas maneiras}
  \end{phonetics}
\end{entry}

\begin{entry}{立}{5}{⽴}
  \begin{phonetics}{立}{li4}[][HSK 5]
    \definition{adj.}{ereto; vertical; na vertical}
    \definition{adv.}{imediatamente; instantaneamente}
    \definition{v.}{ficar em pé, com os pés no chão ou apoiados em algum objeto; o objeto deve estar na vertical | erguer; colocar (ou levantar) algo; colocar em pé | encontrar; criar; elaborar; formular; estabelecer | configurar; fundar; estabelecer | viver; existir | ascender ao trono; antigamente, referia-se à ascensão ao trono de um monarca | nomear; designar; antigamente, significava estabelecer uma determinada posição ou status}
  \end{phonetics}
\end{entry}

\begin{entry}{立场}{5,6}{⽴、⼟}
  \begin{phonetics}{立场}{li4chang3}[][HSK 5]
    \definition[个]{s.}{posição; postura; a posição e a atitude adotadas ao reconhecer e lidar com os problemas | ponto de vista; refere-se especificamente à atitude de reconhecer e lidar com questões a partir dos interesses de uma determinada classe, ou seja, a posição de classe}
  \end{phonetics}
\end{entry}

\begin{entry}{立即}{5,7}{⽴、⼙}
  \begin{phonetics}{立即}{li4ji2}[][HSK 4]
    \definition{adv.}{prontamente; imediatamente; de imediato}
  \end{phonetics}
\end{entry}

\begin{entry}{立刻}{5,8}{⽴、⼑}
  \begin{phonetics}{立刻}{li4ke4}[][HSK 3]
    \definition{adv.}{imediatamente; de ​​uma vez}
  \end{phonetics}
\end{entry}

\begin{entry}{立法}{5,8}{⽴、⽔}
  \begin{phonetics}{立法}{li4fa3}
    \definition{s.}{legislação}
    \definition{v.}{promulgar leis | legislar}
  \end{phonetics}
\end{entry}

\begin{entry}{纠葛}{5,12}{⽷、⾋}
  \begin{phonetics}{纠葛}{jiu1ge2}
    \definition{s.}{emaranhado | disputa}
  \end{phonetics}
\end{entry}

\begin{entry}{节}{5}{⾋}
  \begin{phonetics}{节}{jie2}[][HSK 2]
    \definition*{s.}{sobrenome Jie}
    \definition{clas.}{para nós, seções, comprimentos}
    \definition{s.}{junta | botão | nó | divisão | parte | festival | feriado | item | integridade moral | castidade}
  \end{phonetics}
\end{entry}

\begin{entry}{节日}{5,4}{⾋、⽇}
  \begin{phonetics}{节日}{jie2ri4}[][HSK 2]
    \definition[个]{s.}{festival | feriado}
  \end{phonetics}
\end{entry}

\begin{entry}{节目}{5,5}{⾋、⽬}
  \begin{phonetics}{节目}{jie2mu4}[][HSK 2]
    \definition{s.}{programa | item (em um programa)}
  \end{phonetics}
\end{entry}

\begin{entry}{节约}{5,6}{⾋、⽷}
  \begin{phonetics}{节约}{jie2yue1}[][HSK 3]
    \definition{adj.}{econômico}
    \definition{v.}{guardar; economizar}
  \end{phonetics}
\end{entry}

\begin{entry}{节奏}{5,9}{⾋、⼤}
  \begin{phonetics}{节奏}{jie2zou4}
    \definition{s.}{ritmo | cadência | batida}
  \end{phonetics}
\end{entry}

\begin{entry}{节省}{5,9}{⾋、⽬}
  \begin{phonetics}{节省}{jie2sheng3}[][HSK 4]
    \definition{adj.}{econômico; parcimonioso}
    \definition{v.}{economizar; conservar; usar com moderação; reduzir; eliminar ou minimizar o esgotamento de itens potencialmente esgotáveis}
  \end{phonetics}
\end{entry}

\begin{entry}{讨生活}{5,5,9}{⾔、⽣、⽔}
  \begin{phonetics}{讨生活}{tao3sheng1huo2}
    \definition{v.}{ganhar a vida}
  \end{phonetics}
\end{entry}

\begin{entry}{讨厌}{5,6}{⾔、⼚}
  \begin{phonetics}{讨厌}{tao3yan4}[][HSK 5]
    \definition{adj.}{desagradável; repugnante; repulsivo; irritante; incômodo |}
    \definition{v.}{odiar; não gostar; sentir repulsa por}
  \end{phonetics}
\end{entry}

\begin{entry}{讨论}{5,6}{⾔、⾔}
  \begin{phonetics}{讨论}{tao3lun4}[][HSK 2]
    \definition{v.}{discutir | falar sobre}
  \end{phonetics}
\end{entry}

\begin{entry}{让}{5}{⾔}
  \begin{phonetics}{让}{rang4}[][HSK 2]
    \definition{v.}{deixar alguém fazer alguma coisa |fazer alguém (sentir-se triste, etc.) | permitir | conceder}
  \end{phonetics}
\end{entry}

\begin{entry}{让步}{5,7}{⾔、⽌}
  \begin{phonetics}{让步}{rang4bu4}
    \definition{v.+compl.}{fazer uma concessão | entregar | desistir | comprometer}
  \end{phonetics}
\end{entry}

\begin{entry}{训练}{5,8}{⾔、⽷}
  \begin{phonetics}{训练}{xun4lian4}[][HSK 3]
    \definition{v.}{treinar; exercitar; adquirir certas especialidades ou habilidades de forma planejada e passo a passo}
  \end{phonetics}
\end{entry}

\begin{entry}{议论}{5,6}{⾔、⾔}
  \begin{phonetics}{议论}{yi4lun4}[][HSK 4]
    \definition{s.}{comentário; discussão; opiniões ou pontos de vista sobre o que é bom ou ruim, certo ou errado em relação a pessoas ou coisas}
    \definition{v.}{discutir; comentar; falar sobre; expressar opiniões e trocar pontos de vista sobre o bom, o ruim, o certo e o errado de pessoas ou coisas}
  \end{phonetics}
\end{entry}

\begin{entry}{记}{5}{⾔}
  \begin{phonetics}{记}{ji4}[][HSK 1]
    \definition{clas.}{para tapas, palmadas, bofetadas, etc.}
    \definition{s.}{nota | registro | marca | sinal |marca de nascença}
    \definition{v.}{lembrar | ter em mente | memorizar | escrever (anotar) | registrar}
  \end{phonetics}
\end{entry}

\begin{entry}{记忆}{5,4}{⾔、⼼}
  \begin{phonetics}{记忆}{ji4yi4}[][HSK 5]
    \definition[段]{s.}{memória; manter em sua mente uma imagem do passado}
    \definition{v.}{recordar; lembrar; lembrar-se ou recordar alguém ou algo do passado}
  \end{phonetics}
\end{entry}

\begin{entry}{记住}{5,7}{⾔、⼈}
  \begin{phonetics}{记住}{ji4 zhu5}[][HSK 1]
    \definition{v.}{decorar | memorizar | ter em mente}
  \end{phonetics}
\end{entry}

\begin{entry}{记录}{5,8}{⾔、⼹}
  \begin{phonetics}{记录}{ji4lu4}[][HSK 3]
    \definition[个,位]{s.}{notas; registro | anotador; registrador}
    \definition{v.}{tomar notas; registrar}
  \end{phonetics}
\end{entry}

\begin{entry}{记性}{5,8}{⾔、⼼}
  \begin{phonetics}{记性}{ji4xing5}
    \definition{s.}{memória (habilidade em reter informações)}
  \end{phonetics}
\end{entry}

\begin{entry}{记者}{5,8}{⾔、⽼}
  \begin{phonetics}{记者}{ji4zhe3}[][HSK 3]
    \definition[群,名,位]{s.}{repórter; correspondente; jornalista}
  \end{phonetics}
\end{entry}

\begin{entry}{记载}{5,10}{⾔、⾞}
  \begin{phonetics}{记载}{ji4zai3}[][HSK 4]
    \definition[段,份]{s.}{registro; conta; artigos e materiais que registram eventos}
    \definition{v.}{registrar; colocar por escrito}
  \end{phonetics}
\end{entry}

\begin{entry}{记得}{5,11}{⾔、⼻}
  \begin{phonetics}{记得}{ji4de5}[][HSK 1]
    \definition{v.}{lembrar | lembrar-se}
  \end{phonetics}
\end{entry}

\begin{entry}{边}{5}{⾡}
  \begin{phonetics}{边}{bian1}[][HSK 2]
    \definition{adv.}{simultaneamente}
    \definition[个]{s.}{fronteira | limite | borda | margem | lado}
  \end{phonetics}
  \begin{phonetics}{边}{bian5}
    \definition{suf.}{sufixo de uma palavra de localidade}
  \end{phonetics}
\end{entry}

\begin{entry}{边关}{5,6}{⾡、⼋}
  \begin{phonetics}{边关}{bian1guan1}
    \definition{s.}{posto de fronteira | posição defensiva estratégica na fronteira}
  \end{phonetics}
\end{entry}

\begin{entry}{边防}{5,6}{⾡、⾩}
  \begin{phonetics}{边防}{bian1fang2}
    \definition{s.}{defesa da fronteira}
  \end{phonetics}
\end{entry}

\begin{entry}{边境}{5,14}{⾡、⼟}
  \begin{phonetics}{边境}{bian1jing4}[][HSK 5]
    \definition{s.}{fronteira}
  \end{phonetics}
\end{entry}

\begin{entry}{闪}{5}{⾨}
  \begin{phonetics}{闪}{shan3}[][HSK 4]
    \definition*{s.}{sobrenome Shan}
    \definition{s.}{relâmpago}
    \definition{v.}{esquivar-se; desviar; sair do caminho | torcer; distender | surgir de repente | cintilar; brilhar | deixar para trás; abandonar | (corpo) oscilar dramaticamente}
  \end{phonetics}
\end{entry}

\begin{entry}{闪电}{5,5}{⾨、⽥}
  \begin{phonetics}{闪电}{shan3dian4}[][HSK 4]
    \definition[道]{s.}{relâmpago; descargas elétricas entre nuvens ou entre nuvens e o solo}
  \seealsoref{雷电}{lei2dian4}
  \end{phonetics}
\end{entry}

\begin{entry}{闪存盘}{5,6,11}{⾨、⼦、⽫}
  \begin{phonetics}{闪存盘}{shan3cun2pan2}
    \definition{s.}{unidade de memória \emph{USB} | cartão de memória}
  \seealsoref{优盘}{you1pan2}
  \end{phonetics}
\end{entry}

\begin{entry}{鸟}{5}{⿃}
  \begin{phonetics}{鸟}{diao3}
    \definition{s.}{pênis | órgão genital masculino | aves | aviário}
  \end{phonetics}
  \begin{phonetics}{鸟}{niao3}[][HSK 2]
    \definition[只,群]{s.}{pássaro}
  \end{phonetics}
\end{entry}

\begin{entry}{鸟儿}{5,2}{⿃、⼉}
  \begin{phonetics}{鸟儿}{niao3r5}
    \definition[只]{s.}{pássaro | ave}
  \end{phonetics}
\end{entry}

\begin{entry}{龙}{5}{⿓}[Kangxi 212]
  \begin{phonetics}{龙}{long2}[][HSK 3]
    \definition*{s.}{sobrenome Long}
    \definition{adj.}{imperial}
    \definition[条]{s.}{dragão
um enorme réptil extinto}
  \end{phonetics}
\end{entry}

\begin{entry}{龙山}{5,3}{⿓、⼭}
  \begin{phonetics}{龙山}{long2shan1}
    \definition*{s.}{Longshan}
  \end{phonetics}
\end{entry}

\begin{entry}{龙虾}{5,9}{⿓、⾍}
  \begin{phonetics}{龙虾}{long2xia1}
    \definition{s.}{lagosta}
  \end{phonetics}
\end{entry}

%%%%% EOF %%%%%

