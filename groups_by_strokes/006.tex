%%%
%%% 6画
%%%

\section*{6画}\addcontentsline{toc}{section}{6画}

\begin{entry}{丢}{6}{⼛}
  \begin{phonetics}{丢}{diu1}[][HSK 5]
    \definition{v.}{perder; extraviar; estar ausente | lançar; atirar | colocar (deixar) de lado | deixar (para trás)}
  \end{phonetics}
\end{entry}

\begin{entry}{丢下}{6,3}{⼛、⼀}
  \begin{phonetics}{丢下}{diu1xia4}
    \definition{v.}{abandonar}
  \end{phonetics}
\end{entry}

\begin{entry}{丢开}{6,4}{⼛、⼶}
  \begin{phonetics}{丢开}{diu1kai1}
    \definition{v.}{jogar fora ou deixar de lado | esquecer por um tempo}
  \end{phonetics}
\end{entry}

\begin{entry}{丢失}{6,5}{⼛、⼤}
  \begin{phonetics}{丢失}{diu1shi1}
    \definition{v.}{perder}
  \end{phonetics}
\end{entry}

\begin{entry}{丢弃}{6,7}{⼛、⼶}
  \begin{phonetics}{丢弃}{diu1qi4}
    \definition{v.}{jogar fora | descartar}
  \end{phonetics}
\end{entry}

\begin{entry}{丢官}{6,8}{⼛、⼧}
  \begin{phonetics}{丢官}{diu1guan1}
    \definition{v.}{perder um cargo oficial}
  \end{phonetics}
\end{entry}

\begin{entry}{丢掉}{6,11}{⼛、⼿}
  \begin{phonetics}{丢掉}{diu1diao4}
    \definition{v.}{jogar fora | descartar |perder}
  \end{phonetics}
\end{entry}

\begin{entry}{丢脸}{6,11}{⼛、⾁}
  \begin{phonetics}{丢脸}{diu1lian3}
    \definition{adj.}{vergonhoso}
  \end{phonetics}
\end{entry}

\begin{entry}{乒}{6}{⼃}
  \begin{phonetics}{乒}{ping1}
    \definition{interj.}{(onomatopéia) estalo; estouro; estrondo | (onomatopéia)  ``ping''}
    \definition{s.}{(abreviação) tênis de mesa; pingue-pongue | (abreviação) bola de tênis de mesa; bola de pingue-pongue}
  \end{phonetics}
\end{entry}

\begin{entry}{乒乓球}{6,6,11}{⼃、⼃、⽟}
  \begin{phonetics}{乒乓球}{ping1pang1qiu2}
    \definition[个]{s.}{tênis de mesa |ping-pong}
  \end{phonetics}
\end{entry}

\begin{entry}{乓}{6}{⼃}
  \begin{phonetics}{乓}{pang1}
    \definition{interj.}{(onomatopéia) barulho repentino feito por tiros, uma porta batendo, coisas quebrando, etc.; estrondo; estouro; batida; colisão}
  \end{phonetics}
\end{entry}

\begin{entry}{买}{6}{⼄}
  \begin{phonetics}{买}{mai3}[][HSK 1]
    \definition*{s.}{Sobrenome Mai}
    \definition{v.}{comprar; adquirir | comprar; subornar; usar dinheiro ou outros meios para angariar apoio| pedir; obter; trocar dinheiro por coisas}
  \end{phonetics}
\end{entry}

\begin{entry}{买东西}{6,5,6}{⼄、⼀、⾑}
  \begin{phonetics}{买东西}{mai3dong1xi5}
    \definition{v.}{fazer compras}
  \end{phonetics}
\end{entry}

\begin{entry}{买卖}{6,8}{⼄、⼗}
  \begin{phonetics}{买卖}{mai3 mai4}[][HSK 5]
    \definition[种,笔]{s.}{negócio; compra e venda; transação | (privado) loja; armazém;}
  \end{phonetics}
\end{entry}

\begin{entry}{争}{6}{⼑}
  \begin{phonetics}{争}{zheng1}[][HSK 3]
    \definition*{s.}{Sobrenome Zheng}
    \definition{adv.}{como; por que}
    \definition{v.}{competir; disputar; lutar; esforçar-se para obter ou alcançar | discutir; argumentar; contestar; debater | faltar; estar em falta}
  \end{phonetics}
\end{entry}

\begin{entry}{争风吃醋}{6,4,6,15}{⼑、⾵、⼝、⾣}
  \begin{phonetics}{争风吃醋}{zheng1feng1chi1cu4}
    \definition{v.}{rivalizar com alguém pelo carinho de um homem ou mulher |estar com ciúmes de um rival em um caso de amor}
  \end{phonetics}
\end{entry}

\begin{entry}{争议}{6,5}{⼑、⾔}
  \begin{phonetics}{争议}{zheng1yi4}[][HSK 5]
    \definition{s.}{disputa; controvérsia; situações e questões em que há divergências de opinião}
    \definition{v.}{debater; discutir}
  \end{phonetics}
\end{entry}

\begin{entry}{争先}{6,6}{⼑、⼉}
  \begin{phonetics}{争先}{zheng1xian1}
    \definition{v.}{competir para ser o primeiro |contestar o primeiro lugar}
  \end{phonetics}
\end{entry}

\begin{entry}{争论}{6,6}{⼑、⾔}
  \begin{phonetics}{争论}{zheng1lun4}[][HSK 4]
    \definition{s.}{debate; discussão; argumentação; disputa}
    \definition{v.}{discutir; disputar; debater; argumentar; contestar}
  \end{phonetics}
\end{entry}

\begin{entry}{争取}{6,8}{⼑、⼜}
  \begin{phonetics}{争取}{zheng1qu3}[][HSK 3]
    \definition{v.}{lutar por; conquistar; vencer; se esforçar para conseguir}
  \end{phonetics}
\end{entry}

\begin{entry}{争霸}{6,21}{⼑、⾬}
  \begin{phonetics}{争霸}{zheng1ba4}
    \definition{s.}{hegemonia | uma luta de poder}
    \definition{v.}{disputar a hegemonia}
  \end{phonetics}
\end{entry}

\begin{entry}{亚}{6}{⼆}
  \begin{phonetics}{亚}{ya4}
    \definition*{s.}{Ásia, abreviação de 亚洲 | Sobrenome Ya}
    \definition{adj.}{inferior | abaixo do padrão | (química) de menor valência atômica}
    \definition{pref.}{sub-}
  \seealsoref{亚洲}{ya4zhou1}
  \end{phonetics}
\end{entry}

\begin{entry}{亚军}{6,6}{⼆、⼍}
  \begin{phonetics}{亚军}{ya4jun1}[][HSK 5]
    \definition{s.}{segundo lugar; vice-campeão; medalhista de prata}
  \end{phonetics}
\end{entry}

\begin{entry}{亚运会}{6,7,6}{⼆、⾡、⼈}
  \begin{phonetics}{亚运会}{ya4 yun4 hui4}[][HSK 4]
    \definition*{s.}{Jogos Asiáticos}
  \end{phonetics}
\end{entry}

\begin{entry}{亚细亚洲}{6,8,6,9}{⼆、⽷、⼆、⽔}
  \begin{phonetics}{亚细亚洲}{ya4xi4ya4zhou1}
    \definition*{s.}{Ásia}
  \end{phonetics}
\end{entry}

\begin{entry}{亚洲}{6,9}{⼆、⽔}
  \begin{phonetics}{亚洲}{ya4zhou1}
    \definition*{s.}{Ásia, abreviação de~亚细亚洲}
  \seealsoref{亚细亚洲}{ya4xi4ya4zhou1}
  \end{phonetics}
\end{entry}

\begin{entry}{亚洲人}{6,9,2}{⼆、⽔、⼈}
  \begin{phonetics}{亚洲人}{ya4zhou1ren2}
    \definition{s.}{asiático | pessoa ou povo da Ásia}
  \end{phonetics}
\end{entry}

\begin{entry}{亚热带}{6,10,9}{⼆、⽕、⼱}
  \begin{phonetics}{亚热带}{ya4re4dai4}
    \definition{s.}{zona ou clima subtropical; subtropical; semitropical}
  \end{phonetics}
\end{entry}

\begin{entry}{交}{6}{⼇}
  \begin{phonetics}{交}{jiao1}[][HSK 2]
    \definition*{s.}{Sobrenome Jiao}
    \definition{adv.}{mutuamente; recíprocamente; um ao outro | juntos; simultaneamente}
    \definition{s.}{amigo; conhecido; amizade; relacionamento | transação comercial; negócio; barganha | queda}
    \definition{v.}{entregar | (de lugares ou períodos de tempo) cruzar; encontrar; unir | chegar (a uma determinada hora ou estação); estabelecer-se; vir | cruzar; intersectar | associar-se a | ter relações sexuais | acasalar; reproduzir-se | transferir as coisas para as partes interessadas | unir (lugares ou períodos de tempo)}
  \end{phonetics}
\end{entry}

\begin{entry}{交叉}{6,3}{⼇、⼜}
  \begin{phonetics}{交叉}{jiao1cha1}
    \definition{v.}{cruzar | sobrepor}
  \end{phonetics}
\end{entry}

\begin{entry}{交叉口}{6,3,3}{⼇、⼜、⼝}
  \begin{phonetics}{交叉口}{jiao1cha1kou3}
    \definition{s.}{intersecção (rodovia)}
  \end{phonetics}
\end{entry}

\begin{entry}{交叉点}{6,3,9}{⼇、⼜、⽕}
  \begin{phonetics}{交叉点}{jiao1cha1dian3}
    \definition{s.}{encruzilhada | cruzamento | junção}
  \end{phonetics}
\end{entry}

\begin{entry}{交代}{6,5}{⼇、⼈}
  \begin{phonetics}{交代}{jiao1dai4}[][HSK 5]
    \definition{v.}{contar; entregar | ordenar; insistir; contar aos outros sobre suas intenções, instruções | contar; admitir}
  \end{phonetics}
\end{entry}

\begin{entry}{交运}{6,7}{⼇、⾡}
  \begin{phonetics}{交运}{jiao1yun4}
    \definition{v.}{despachar (bagagem em um aeroporto, etc.) | entregar para transporte}
  \end{phonetics}
\end{entry}

\begin{entry}{交际}{6,7}{⼇、⾩}
  \begin{phonetics}{交际}{jiao1ji4}[][HSK 4]
    \definition{s.}{contato; comunicação; relações sociais; contato interpessoal, socialização}
  \end{phonetics}
\end{entry}

\begin{entry}{交往}{6,8}{⼇、⼻}
  \begin{phonetics}{交往}{jiao1wang3}[][HSK 3]
    \definition{v.}{estar em contato com; associar-se a; interagir}
  \end{phonetics}
\end{entry}

\begin{entry}{交易}{6,8}{⼇、⽇}
  \begin{phonetics}{交易}{jiao1yi4}[][HSK 3]
    \definition[笔,桩,个,场]{s.}{negócio; comércio; transação comercial; transação; atividades de compra e venda de mercadorias}
    \definition{v.}{negociar; comprar e vender mercadorias}
  \end{phonetics}
\end{entry}

\begin{entry}{交朋友}{6,8,4}{⼇、⽉、⼜}
  \begin{phonetics}{交朋友}{jiao1 peng2 you3}[][HSK 2]
    \definition{v.}{fazer amizade com alguém; fazer amigos}
  \end{phonetics}
\end{entry}

\begin{entry}{交杯酒}{6,8,10}{⼇、⽊、⾣}
  \begin{phonetics}{交杯酒}{jiao1bei1jiu3}
    \definition{s.}{copo de vinho nupcial}
  \end{phonetics}
\end{entry}

\begin{entry}{交响}{6,9}{⼇、⼝}
  \begin{phonetics}{交响}{jiao1xiang3}
    \definition{s.}{sinfonia}
  \end{phonetics}
\end{entry}

\begin{entry}{交界}{6,9}{⼇、⽥}
  \begin{phonetics}{交界}{jiao1jie4}
    \definition{s.}{fronteira comum | limite comum | interface}
  \end{phonetics}
\end{entry}

\begin{entry}{交给}{6,9}{⼇、⽷}
  \begin{phonetics}{交给}{jiao1 gei3}[][HSK 2]
    \definition{v.}{entregar para | dar para}
  \end{phonetics}
\end{entry}

\begin{entry}{交费}{6,9}{⼇、⾙}
  \begin{phonetics}{交费}{jiao1 fei4}[][HSK 3]
    \definition{v.}{pagar taxas ou impostos; pagar uma taxa ou imposto}
  \end{phonetics}
\end{entry}

\begin{entry}{交换}{6,10}{⼇、⼿}
  \begin{phonetics}{交换}{jiao1huan4}[][HSK 4]
    \definition{v.}{trocar; permutar; comutar; intercambiar}
  \end{phonetics}
\end{entry}

\begin{entry}{交流}{6,10}{⼇、⽔}
  \begin{phonetics}{交流}{jiao1liu2}[][HSK 3]
    \definition{v.}{trocar; interagir; comunicar-se; compartilhar o que cada um tem com o outro}
  \end{phonetics}
\end{entry}

\begin{entry}{交班}{6,10}{⼇、⽟}
  \begin{phonetics}{交班}{jiao1ban1}
    \definition{v.}{passar para o próximo turno de trabalho}
  \end{phonetics}
\end{entry}

\begin{entry}{交通}{6,10}{⼇、⾡}
  \begin{phonetics}{交通}{jiao1tong1}[][HSK 2]
    \definition{s.}{tráfego | ligação; conexão | transporte; termo genérico para todos os tipos de transporte, como ferroviário e rodoviário}
    \definition{v.}{conspirar; fazer amizades; conchavar | estar conectado; estar ligado; estar vinculado | associar-se a; conspirar com}
  \end{phonetics}
\end{entry}

\begin{entry}{交通警察}{6,10,19,14}{⼇、⾡、⾔、⼧}
  \begin{phonetics}{交通警察}{jiao1tong1jing3cha2}
    \definition{s.}{policial de trânsito}
  \seealsoref{交警}{jiao1 jing3}
  \end{phonetics}
\end{entry}

\begin{entry}{交叠}{6,13}{⼇、⼜}
  \begin{phonetics}{交叠}{jiao1die2}
    \definition{s.}{sobreposição}
  \end{phonetics}
\end{entry}

\begin{entry}{交媾}{6,13}{⼇、⼥}
  \begin{phonetics}{交媾}{jiao1gou4}
    \definition{v.}{copular | ter relações sexuais}
  \end{phonetics}
\end{entry}

\begin{entry}{交警}{6,19}{⼇、⾔}
  \begin{phonetics}{交警}{jiao1 jing3}[][HSK 3]
    \definition{s.}{policial de trânsito, abreviação de 交通警察}
  \seealsoref{交通警察}{jiao1tong1jing3cha2}
  \end{phonetics}
\end{entry}

\begin{entry}{亦}{6}{⼇}
  \begin{phonetics}{亦}{yi4}
    \definition{adv.}{também | igualmente | apenas | embora | já}
  \end{phonetics}
\end{entry}

\begin{entry}{产}{6}{⼇}
  \begin{phonetics}{产}{chan3}
    \definition*{s.}{Sobrenome Chan}
    \definition{s.}{produto | propriedade; espólio | (abreviação) indústria}
    \definition{v.}{dar à luz; ser entregue a | produzir; render | separar um ser humano ou animal de sua mãe}
  \end{phonetics}
\end{entry}

\begin{entry}{产业}{6,5}{⼇、⼀}
  \begin{phonetics}{产业}{chan3ye4}[][HSK 5]
    \definition{s.}{patrimônio; propriedade; bens pessoais, como terrenos, casas, fábricas, etc. | indústria; refere-se especificamente à produção industrial moderna | setor; indústria; indústrias e setores da economia nacional}
  \end{phonetics}
\end{entry}

\begin{entry}{产生}{6,5}{⼇、⽣}
  \begin{phonetics}{产生}{chan3sheng1}[][HSK 3]
    \definition{v.}{produzir; evoluir; emergir; provocar; vir a ser; dar origem a; criar coisas novas e novos fenômenos a partir do que já existe}
  \end{phonetics}
\end{entry}

\begin{entry}{产后}{6,6}{⼇、⼝}
  \begin{phonetics}{产后}{chan3hou4}
    \definition{s.}{pós-parto}
  \end{phonetics}
\end{entry}

\begin{entry}{产品}{6,9}{⼇、⼝}
  \begin{phonetics}{产品}{chan3pin3}[][HSK 4]
    \definition[个,件,种,批,项,类]{s.}{produto; item produzido}
  \end{phonetics}
\end{entry}

\begin{entry}{产量}{6,12}{⼇、⾥}
  \begin{phonetics}{产量}{chan3 liang4}[][HSK 6]
    \definition{v.}{rendimento; produção; a quantidade de produção; a quantidade total de produtos produzidos em um determinado período de tempo}
  \end{phonetics}
\end{entry}

\begin{entry}{仰}{6}{⼈}
  \begin{phonetics}{仰}{yang3}[][HSK 6]
    \definition*{s.}{Sobrenome Yang}
    \definition{v.}{levantar (oposto a 俯) | virar para cima | admirar; respeitar | confiar em; depender de}
  \seealsoref{俯}{fu3}
  \end{phonetics}
\end{entry}

\begin{entry}{件}{6}{⼈}
  \begin{phonetics}{件}{jian4}[][HSK 2]
    \definition*{s.}{Sobrenome Jian}
    \definition{clas.}{item; peça; artigo; usado para coisas individuais}
    \definition{s.}{refere-se a coisas que podem ser contadas uma a uma | papel; carta; documento; correspondência}
  \end{phonetics}
\end{entry}

\begin{entry}{价}{6}{⼈}
  \begin{phonetics}{价}{jia4}[][HSK 5]
    \definition{s.}{preço | valor; (figurativo) valores (éticos, culturais etc.) | (química) valência}
  \end{phonetics}
\end{entry}

\begin{entry}{价值}{6,10}{⼈、⼈}
  \begin{phonetics}{价值}{jia4zhi2}[][HSK 3]
    \definition{s.}{valor; o trabalho social necessário condensado nos produtos | valor; importância; efeitos positivos}
  \end{phonetics}
\end{entry}

\begin{entry}{价格}{6,10}{⼈、⽊}
  \begin{phonetics}{价格}{jia4ge2}[][HSK 3]
    \definition[个,种]{s.}{preço; tarifa; o valor monetário da mercadoria}
  \end{phonetics}
\end{entry}

\begin{entry}{价钱}{6,10}{⼈、⾦}
  \begin{phonetics}{价钱}{jia4 qian2}[][HSK 3]
    \definition[个,种,笔]{s.}{preço}
  \end{phonetics}
\end{entry}

\begin{entry}{任}{6}{⼈}
  \begin{phonetics}{任}{ren4}[][HSK 3]
    \definition{clas.}{usado para o número de mandatos cumpridos em um cargo oficial}
    \definition{conj.}{não importa (como, o que, etc.); orações de conexão, ou usadas antes de pronomes interrogativos, para expressar incondicionalidade, equivalente a 不管 ou 无论}
    \definition{s.}{escritório; posto oficial; cargo | dever; fardo; responsabilidade}
    \definition{v.}{nomear; designar alguém para um cargo | assumir um emprego; assumir um posto; assumir uma posição | deixar; permitir; dar rédea solta a | suportar; empreender | ceder; permitir sem restrições; deixar (alguém) fazer o que quiser}
  \seealsoref{不管}{bu4guan3}
  \seealsoref{无论}{wu2lun4}
  \end{phonetics}
\end{entry}

\begin{entry}{任务}{6,5}{⼈、⼒}
  \begin{phonetics}{任务}{ren4wu5}[][HSK 3]
    \definition[项,个,种,些]{s.}{tarefa; dever; missão; designação; trabalho designado; responsabilidades designadas}
  \end{phonetics}
\end{entry}

\begin{entry}{任何}{6,7}{⼈、⼈}
  \begin{phonetics}{任何}{ren4he2}[][HSK 3]
    \definition{pron.}{qualquer; qualquer que seja; o que for; não importa o que}
  \end{phonetics}
\end{entry}

\begin{entry}{份}{6}{⼈}
  \begin{phonetics}{份}{fen4}
    \definition{clas.}{usado para emparelhar itens em grupos | usado para jornais, documentos, etc. | usado para partes de um todo | usado para aparência, estado, etc.}
    \definition{s.}{porção; parte | a unidade de divisão; usado após 省, 县, 年, 月,  indica a unidade de divisão | grau; extensão de algo}
  \seealsoref{年}{nian2}
  \seealsoref{省}{sheng3}
  \seealsoref{县}{xian4}
  \seealsoref{月}{yue4}
  \end{phonetics}
\end{entry}

\begin{entry}{企}{6}{⼈}
  \begin{phonetics}{企}{qi3}
    \definition{v.}{ficar na ponta dos pés | esperar ansiosamente por algo; ansiar por | planejar um projeto}
  \end{phonetics}
\end{entry}

\begin{entry}{企业}{6,5}{⼈、⼀}
  \begin{phonetics}{企业}{qi3ye4}[][HSK 4]
    \definition[家,个]{s.}{empresa; estabelecimento; empreendimento; negócio; setores envolvidos em atividades econômicas como produção, transporte, comércio, etc., como fábricas, minas, ferrovias, empresas comerciais, etc.}
  \end{phonetics}
\end{entry}

\begin{entry}{伊}{6}{⼈}
  \begin{phonetics}{伊}{yi1}
    \definition*{s.}{Iraque, abreviação de 伊拉克 | Irã,abreviação de  伊朗 | Sobrenome Yi}
    \definition{part.}{(chinês clássico) partícula introdutória sem significado específico}
    \definition{pron.}{(antigo) pronome de terceira pessoa do singular ("ele" ou "ela") | pronome de segunda pessoa do singular ("você") | que (precedendo um substantivo)}
  \seealsoref{伊拉克}{yi1la1ke4}
  \seealsoref{伊朗}{yi1lang3}
  \end{phonetics}
\end{entry}

\begin{entry}{伊马姆}{6,3,8}{⼈、⾺、⼥}
  \begin{phonetics}{伊马姆}{yi1ma3mu3}
    \definition*{s.}{Islã}
  \seealsoref{伊玛目}{yi1ma3mu4}
  \seealsoref{伊曼}{yi1man4}
  \seealsoref{伊斯兰}{yi1si1lan2}
  \end{phonetics}
\end{entry}

\begin{entry}{伊玛目}{6,7,5}{⼈、⽟、⽬}
  \begin{phonetics}{伊玛目}{yi1ma3mu4}
    \definition*{s.}{Islã}
  \seealsoref{伊马姆}{yi1ma3mu3}
  \seealsoref{伊曼}{yi1man4}
  \seealsoref{伊斯兰}{yi1si1lan2}
  \end{phonetics}
\end{entry}

\begin{entry}{伊拉克}{6,8,7}{⼈、⼿、⼗}
  \begin{phonetics}{伊拉克}{yi1la1ke4}
    \definition*{s.}{Iraque}
  \end{phonetics}
\end{entry}

\begin{entry}{伊朗}{6,10}{⼈、⽉}
  \begin{phonetics}{伊朗}{yi1lang3}
    \definition*{s.}{Irã}
  \end{phonetics}
\end{entry}

\begin{entry}{伊曼}{6,11}{⼈、⽈}
  \begin{phonetics}{伊曼}{yi1man4}
    \definition*{s.}{Islã}
  \seealsoref{伊马姆}{yi1ma3mu3}
  \seealsoref{伊玛目}{yi1ma3mu4}
  \seealsoref{伊斯兰}{yi1si1lan2}
  \end{phonetics}
\end{entry}

\begin{entry}{伊斯兰}{6,12,5}{⼈、⽄、⼋}
  \begin{phonetics}{伊斯兰}{yi1si1lan2}
    \definition*{s.}{Islã}
  \seealsoref{伊马姆}{yi1ma3mu3}
  \seealsoref{伊玛目}{yi1ma3mu4}
  \seealsoref{伊曼}{yi1man4}
  \end{phonetics}
\end{entry}

\begin{entry}{休}{6}{⼈}
  \begin{phonetics}{休}{xiu1}
    \definition{adj.}{feliz; alegre; festivo}
    \definition{adv.}{não; indica proibição ou dissuasão, equivalente a 别 ou 不要}
    \definition{s.}{fortuna e infortúnio; bom e mau}
    \definition{v.}{parar; cessar | descansar | abandonar a esposa e mandá-la para casa; antigamente, o marido mandava a esposa de volta para a casa dos pais e rompia o relacionamento conjugal}
  \seealsoref{别}{bie2}
  \seealsoref{不要}{bu2 yao4}
  \end{phonetics}
\end{entry}

\begin{entry}{休兵}{6,7}{⼈、⼋}
  \begin{phonetics}{休兵}{xiu1bing1}
    \definition{s.}{armistício}
    \definition{v.}{cessar fogo}
  \end{phonetics}
\end{entry}

\begin{entry}{休闲}{6,7}{⼈、⾨}
  \begin{phonetics}{休闲}{xiu1xian2}[][HSK 5]
    \definition{s.}{ócio; lazer; tempo livre}
    \definition{v.}{desfrutar do lazer; sair de férias; aproveitar o tempo livre; parar de trabalhar ou estudar, estar em um estado de lazer e descontração | ficar ocioso}
  \end{phonetics}
\end{entry}

\begin{entry}{休息}{6,10}{⼈、⼼}
  \begin{phonetics}{休息}{xiu1xi5}[][HSK 1]
    \definition{s.}{descanço}
    \definition{v.}{descansar; descansar um pouco; fazer uma pausa; interromper o trabalho, os estudos ou as atividades para recuperar as energias | dormir}
  \end{phonetics}
\end{entry}

\begin{entry}{休息室}{6,10,9}{⼈、⼼、⼧}
  \begin{phonetics}{休息室}{xiu1xi1shi4}
    \definition{s.}{saguão | salão}
  \end{phonetics}
\end{entry}

\begin{entry}{休假}{6,11}{⼈、⼈}
  \begin{phonetics}{休假}{xiu1 jia4}[][HSK 2]
    \definition{v.+compl.}{ter um feriado; tirar férias; sair de férias}
  \end{phonetics}
\end{entry}

\begin{entry}{休憩}{6,16}{⼈、⼼}
  \begin{phonetics}{休憩}{xiu1qi4}
    \definition{v.}{relaxar | descansar | dar um tempo}
  \end{phonetics}
\end{entry}

\begin{entry}{休整}{6,16}{⼈、⽁}
  \begin{phonetics}{休整}{xiu1zheng3}
    \definition{v.}{(militar) descansar e reorganizar}
  \end{phonetics}
\end{entry}

\begin{entry}{众}{6}{⼈}
  \begin{phonetics}{众}{zhong4}
    \definition*{s.}{Câmara dos Deputados, abreviação de 众议院}
    \definition{adj.}{numeroso}
    \definition{adv.}{muitos}
    \definition{s.}{multidão}
  \seealsoref{众议院}{zhong4yi4yuan4}
  \end{phonetics}
\end{entry}

\begin{entry}{众议院}{6,5,9}{⼈、⾔、⾩}
  \begin{phonetics}{众议院}{zhong4yi4yuan4}
    \definition*{s.}{Casa baixa da Assembléia Bicameral | Câmara dos Deputados}
  \end{phonetics}
\end{entry}

\begin{entry}{众多}{6,6}{⼈、⼣}
  \begin{phonetics}{众多}{zhong4 duo1}[][HSK 5]
    \definition{adj.}{muitos; numerosos; multitudinários}
  \end{phonetics}
\end{entry}

\begin{entry}{优}{6}{⼈}
  \begin{phonetics}{优}{you1}
    \definition{adj.}{excelente | superior}
  \end{phonetics}
\end{entry}

\begin{entry}{优于}{6,3}{⼈、⼆}
  \begin{phonetics}{优于}{you1yu2}
    \definition{v.}{superar}
  \end{phonetics}
\end{entry}

\begin{entry}{优先}{6,6}{⼈、⼉}
  \begin{phonetics}{优先}{you1xian1}[][HSK 5]
    \definition{adj.}{anterior; sênior; subjacente}
    \definition{v.}{ter prioridade; ter precedência; colocar-se à frente de outras pessoas ou assuntos}
  \end{phonetics}
\end{entry}

\begin{entry}{优伶}{6,7}{⼈、⼈}
  \begin{phonetics}{优伶}{you1ling2}
    \definition{s.}{ator}
  \end{phonetics}
\end{entry}

\begin{entry}{优秀}{6,7}{⼈、⽲}
  \begin{phonetics}{优秀}{you1xiu4}[][HSK 4]
    \definition{adj.}{esplêndido; excelente; extraordinário; excepcional; notável; descreve moral, qualidades, realizações, aprendizado, etc. muito bons.}
  \end{phonetics}
\end{entry}

\begin{entry}{优良}{6,7}{⼈、⾉}
  \begin{phonetics}{优良}{you1 liang2}[][HSK 4]
    \definition{adj.}{ótimo; bom; excelente; (variedade, qualidade, desempenho, estilo, etc.) muito bom}
  \end{phonetics}
\end{entry}

\begin{entry}{优势}{6,8}{⼈、⼒}
  \begin{phonetics}{优势}{you1shi4}[][HSK 3]
    \definition[种,个]{s.}{vantagem; superioridade; preponderância; posição dominante; uma situação favorável que permite superar o adversário}
  \end{phonetics}
\end{entry}

\begin{entry}{优质}{6,8}{⼈、⾙}
  \begin{phonetics}{优质}{you1zhi4}
    \definition{adj.}{excelente qualidade}
  \end{phonetics}
\end{entry}

\begin{entry}{优厚}{6,9}{⼈、⼚}
  \begin{phonetics}{优厚}{you1hou4}
    \definition{adj.}{generoso}
  \end{phonetics}
\end{entry}

\begin{entry}{优点}{6,9}{⼈、⽕}
  \begin{phonetics}{优点}{you1dian3}[][HSK 3]
    \definition[个,项,种,些]{s.}{mérito; virtude; ponto forte; vantagem (em oposição a 缺点)}
  \seealsoref{缺点}{que1dian3}
  \end{phonetics}
\end{entry}

\begin{entry}{优美}{6,9}{⼈、⽺}
  \begin{phonetics}{优美}{you1mei3}[][HSK 4]
    \definition{adj.}{fino; elegante; gracioso; bonito}
  \end{phonetics}
\end{entry}

\begin{entry}{优选}{6,9}{⼈、⾡}
  \begin{phonetics}{优选}{you1xuan3}
    \definition{v.}{otimizar}
  \end{phonetics}
\end{entry}

\begin{entry}{优格}{6,10}{⼈、⽊}
  \begin{phonetics}{优格}{you1ge2}
    \definition{s.}{iogurte}
  \end{phonetics}
\end{entry}

\begin{entry}{优盘}{6,11}{⼈、⽫}
  \begin{phonetics}{优盘}{you1pan2}
    \definition{s.}{unidade de memória USB}
  \seealsoref{闪存盘}{shan3cun2pan2}
  \end{phonetics}
\end{entry}

\begin{entry}{优惠}{6,12}{⼈、⼼}
  \begin{phonetics}{优惠}{you1hui4}[][HSK 5]
    \definition{adj.}{especial; pechincha; reduzido; com desconto | favorável; preferencial; melhores condições ou tratamento do que o normal, permitindo que as pessoas obtenham mais benefícios}
  \end{phonetics}
\end{entry}

\begin{entry}{优等}{6,12}{⼈、⽵}
  \begin{phonetics}{优等}{you1deng3}
    \definition{adj.}{excelente | de primeira linha | alta classe | da mais alta ordem, superior}
  \end{phonetics}
\end{entry}

\begin{entry}{优裕}{6,12}{⼈、⾐}
  \begin{phonetics}{优裕}{you1yu4}
    \definition{adj.}{abundante | bastante}
    \definition{s.}{abundância}
  \end{phonetics}
\end{entry}

\begin{entry}{伙}{6}{⼈}
  \begin{phonetics}{伙}{huo3}[][HSK 4]
    \definition{clas.}{grupo; multidão; banda}
    \definition{s.}{iguaria; alimentação; refeições | parceiro; companheiro | coletivo de colegas}
    \definition{v.}{combinar; unir}
  \end{phonetics}
\end{entry}

\begin{entry}{伙伴}{6,7}{⼈、⼈}
  \begin{phonetics}{伙伴}{huo3ban4}[][HSK 4]
    \definition[个,位,群]{s.}{parceiro; companheiro; antigo sistema militar de dez pessoas para uma fogueira, o chefe da fogueira, uma pessoa encarregada de cozinhar, com a fogueira é chamado de parceiro da fogueira, agora se refere à participação comum em uma determinada organização ou engajada em certas atividades}
  \end{phonetics}
\end{entry}

\begin{entry}{会}{6}{⼈}
  \begin{phonetics}{会}{hui4}[][HSK 1,2]
    \definition{adv.}{um momento}
    \definition{clas.}{momento; um curto período de tempo}
    \definition{s.}{reunião; festa; conferência; reunião com um objetivo específico | reunião; reunião no trabalho | feira do templo; festival religioso | associação; sociedade; sindicato; certas organizações públicas | oportunidade; ocasião; momento oportuno | cidade principal; capital; cidade central}
    \definition{suf.}{união; grupo; associação}
    \definition{v.}{ser provável que; ter certeza de; indica que é possível realizar (é possível responder à pergunta separadamente) |  poder; ser capaz de; significa saber como fazer ou ter a capacidade de fazer (geralmente se refere a coisas que precisam ser aprendidas) | saber; compreender; entender | encontrar; ver | reunir-se; reunir; agregar; juntar | destacar-se em; ser bom em; ser hábil em; indica proficiência | pagar (ou custear) contas}
  \end{phonetics}
  \begin{phonetics}{会}{kuai4}
    \definition[个,场,次]{s.}{contabilidade}
    \definition{v.}{computar; calcular; equilibrar uma conta}
  \end{phonetics}
\end{entry}

\begin{entry}{会计}{6,4}{⼈、⾔}
  \begin{phonetics}{会计}{kuai4ji4}[][HSK 4]
    \definition[个,位,名]{s.}{contabilidade | contador; contabilista; guarda-livros; pessoal que trabalha como contador}
  \end{phonetics}
\end{entry}

\begin{entry}{会议}{6,5}{⼈、⾔}
  \begin{phonetics}{会议}{hui4yi4}[][HSK 3]
    \definition[次,届,个,场]{s.}{reunião; conferência; reunião organizada pela organização relevante para ouvir opiniões, discutir questões e distribuir tarefas | conselho; congresso; um órgão ou organização permanente que discute e trata frequentemente assuntos importantes}
  \end{phonetics}
\end{entry}

\begin{entry}{会员}{6,7}{⼈、⼝}
  \begin{phonetics}{会员}{hui4 yuan2}[][HSK 3]
    \definition[位,名,个,些]{s.}{membro; associado; membros de certos grupos ou organizações}
  \end{phonetics}
\end{entry}

\begin{entry}{会首}{6,9}{⼈、⾸}
  \begin{phonetics}{会首}{hui4shou3}
    \definition{s.}{chefe de uma sociedade | patrocinador de uma organização}
  \end{phonetics}
\end{entry}

\begin{entry}{会谈}{6,10}{⼈、⾔}
  \begin{phonetics}{会谈}{hui4 tan2}[][HSK 5]
    \definition{v.}{manter conversações; comumente usado em assuntos internacionais ou atividades diplomáticas}
  \end{phonetics}
\end{entry}

\begin{entry}{伞}{6}{⼈}
  \begin{phonetics}{伞}{san3}[][HSK 4]
    \definition*{s.}{Sobrenome San}
    \definition[把]{s.}{guarda-chuva; proteção contra chuva ou sol | algo que tem o formato de um guarda-chuva}
  \end{phonetics}
\end{entry}

\begin{entry}{伟}{6}{⼈}
  \begin{phonetics}{伟}{wei3}
    \definition{adj.}{grande | ótimo}
  \end{phonetics}
\end{entry}

\begin{entry}{伟大}{6,3}{⼈、⼤}
  \begin{phonetics}{伟大}{wei3da4}[][HSK 3]
    \definition{adj.}{ótimo; excelente; extrovertido; descreve uma pessoa com moral e qualidades excelentes, habilidades e realizações excepcionais, que inspira grande respeito | ótimo; magnífico; descreve algo de grande importância, com impacto significativo, acima do normal, algo notável}
  \end{phonetics}
\end{entry}

\begin{entry}{传}{6}{⼈}
  \begin{phonetics}{传}{chuan2}[][HSK 3]
    \definition{v.}{passar; passar adiante | passar adiante; legar; passar de\dots para\dots; passar da geração anterior para a seguinte | transmitir (conhecimento, habilidade, etc.); comunicar; ensinar | espalhar; propagar | transmitir; conduzir; transferir | transmitir; expressar | convocar; dar ordem para chamar alguém | infectar; ser contagioso | enviar documentos por e-mail ou fax}
  \end{phonetics}
  \begin{phonetics}{传}{zhuan4}
    \definition{s.}{comentários sobre clássicos; obras que explicam as escrituras| biografia | romances sobre eventos históricos; obras que narram histórias históricas}
  \end{phonetics}
\end{entry}

\begin{entry}{传出}{6,5}{⼈、⼐}
  \begin{phonetics}{传出}{chuan2 chu1}[][HSK 6]
    \definition{adj.}{eferente (nervo)}
    \definition{v.}{disseminar | transmitir para fora}
  \end{phonetics}
\end{entry}

\begin{entry}{传达}{6,6}{⼈、⾡}
  \begin{phonetics}{传达}{chuan2da2}[][HSK 5]
    \definition{s.}{recepção e registro de chamadas em um estabelecimento público | zelador; recepcionista}
    \definition{v.}{passar adiante (informações, etc.); transmitir; retransmitir; comunicar}
  \end{phonetics}
\end{entry}

\begin{entry}{传来}{6,7}{⼈、⽊}
  \begin{phonetics}{传来}{chuan2 lai2}[][HSK 3]
    \definition{v.}{(um som) passar; transmitir de algum lugar para o local onde o locutor se encontra | (notícias) chegar; transmitir mensagens ou informações}
  \end{phonetics}
\end{entry}

\begin{entry}{传言}{6,7}{⼈、⾔}
  \begin{phonetics}{传言}{chuan2 yan2}[][HSK 6]
    \definition[个,种,些]{s.}{boato; rumor}
    \definition{v.}{passar uma mensagem | datado: fazer um discurso | falar; fazer uma declaração}
  \end{phonetics}
\end{entry}

\begin{entry}{传承}{6,8}{⼈、⼿}
  \begin{phonetics}{传承}{chuan2cheng2}
    \definition{s.}{herança | tradição continuada}
    \definition{v.}{transmitir (para as gerações futuras) | passar adiante (desde os tempos antigos)}
  \end{phonetics}
\end{entry}

\begin{entry}{传给}{6,9}{⼈、⽷}
  \begin{phonetics}{传给}{chuan2gei3}
    \definition{v.}{passar para | transferir para | entregar a}
  \end{phonetics}
\end{entry}

\begin{entry}{传统}{6,9}{⼈、⽷}
  \begin{phonetics}{传统}{chuan2tong3}[][HSK 4]
    \definition{adj.}{tradicional; histórico; transmitido de geração em geração | antiquado, conservador e fora de sintonia com os tempos}
    \definition[个]{s.}{tradição; costume; fatores sociais, como costumes, moral, ideias, estilos, artes, instituições etc., que são transmitidos de uma geração para outra e que são característicos da sociedade}
  \end{phonetics}
\end{entry}

\begin{entry}{传说}{6,9}{⼈、⾔}
  \begin{phonetics}{传说}{chuan2shuo1}[][HSK 3]
    \definition[个,种,段]{s.}{lenda; conto popular; folclore; coisas lendárias; especificamente, lendas populares}
    \definition{v.}{dizer que; ser dito; passar de boca em boca; transmitir oralmente, segundo a tradição}
  \end{phonetics}
\end{entry}

\begin{entry}{传真}{6,10}{⼈、⼗}
  \begin{phonetics}{传真}{chuan2zhen1}[][HSK 5]
    \definition[台,部,份]{s.}{\emph{FAX}, facsímile; texto, diagramas, fotografias, etc., transmitidos por aparelho de fax}
    \definition{v.}{enviar um fax}
  \end{phonetics}
\end{entry}

\begin{entry}{传递}{6,10}{⼈、⾡}
  \begin{phonetics}{传递}{chuan2 di4}[][HSK 5]
    \definition{v.}{transmitir; entregar; transferir; passar adiante}
  \end{phonetics}
\end{entry}

\begin{entry}{传媒}{6,12}{⼈、⼥}
  \begin{phonetics}{传媒}{chuan2 mei2}[][HSK 6]
    \definition{s.}{meios de comunicação de massa; mídia; jornais, rádio, televisão, \emph{Internet} e outras ferramentas de notícias | meio; veículo; vetor; o meio ou via de transmissão da doença}
  \end{phonetics}
\end{entry}

\begin{entry}{传输}{6,13}{⼈、⾞}
  \begin{phonetics}{传输}{chuan2 shu1}[][HSK 6]
    \definition{v.}{transmitir, transportar (energia, informação, etc.)}
  \end{phonetics}
\end{entry}

\begin{entry}{传播}{6,15}{⼈、⼿}
  \begin{phonetics}{传播}{chuan2bo1}[][HSK 3]
    \definition{v.}{espalhar; difundir; propagar; disseminar}
  \end{phonetics}
\end{entry}

\begin{entry}{伤}{6}{⼈}
  \begin{phonetics}{伤}{shang1}[][HSK 3]
    \definition*{s.}{Sobrenome Shang}
    \definition[处]{s.}{ferida; ferimento}
    \definition{v.}{ferir; machucar | ter os sentimentos feridos | estar angustiado | enjoar de algo; desenvolver aversão a algo | ser prejudicial a; entravar}
  \end{phonetics}
\end{entry}

\begin{entry}{伤心}{6,4}{⼈、⼼}
  \begin{phonetics}{伤心}{shang1xin1}[][HSK 3]
    \definition{v.+compl.}{estar triste; lamentar; estar com o coração partido; sentir-se triste por causa de infortúnio ou decepção}
  \end{phonetics}
\end{entry}

\begin{entry}{伤害}{6,10}{⼈、⼧}
  \begin{phonetics}{伤害}{shang1hai4}[][HSK 4]
    \definition{v.}{ferir; prejudicar; machucar; magoar; causar danos físicos ou mentais}
  \end{phonetics}
\end{entry}

\begin{entry}{伦}{6}{⼈}
  \begin{phonetics}{伦}{lun2}
    \definition*{s.}{Sobrenome Lun}
    \definition{s.}{relações humanas (especialmente como concebidas pela ética feudal) | lógica; ordem | par; correspondência; (mesma) classe | ética; relações humanas | sequência lógica; ordem | o mesmo tipo; semelhante; igual}
  \end{phonetics}
\end{entry}

\begin{entry}{伦敦}{6,12}{⼈、⽁}
  \begin{phonetics}{伦敦}{lun2dun1}
    \definition*{s.}{Londres}
  \end{phonetics}
\end{entry}

\begin{entry}{伪}{6}{⼈}
  \begin{phonetics}{伪}{wei3}
    \definition{adj.}{falso; falsificado | fantoche; colaboracionista; ilegal | forjado; falso}
    \definition{pref.}{pseudo-; quasi-; quase-}
  \end{phonetics}
\end{entry}

\begin{entry}{似}{6}{⼈}
  \begin{phonetics}{似}{shi4}
    \definition{v.}{ver; parecer}
  \end{phonetics}
  \begin{phonetics}{似}{si4}
    \definition*{s.}{Sobrenome Si}
    \definition{adv.}{parece; como se}
    \definition{v.}{ser semelhante; parecer-se com | parecer; aparecer | exceder}
  \end{phonetics}
\end{entry}

\begin{entry}{似乎}{6,5}{⼈、⼃}
  \begin{phonetics}{似乎}{si4hu1}[][HSK 4]
    \definition{adv.}{como se; aparentemente; se parece como}
  \end{phonetics}
\end{entry}

\begin{entry}{似的}{6,8}{⼈、⽩}
  \begin{phonetics}{似的}{shi4de5}[][HSK 4]
    \definition{part.}{como; como\dots como; como se (embora); usada após uma palavra ou frase para indicar uma semelhança com algo ou uma situação | usada para indicar alto grau}
  \end{phonetics}
\end{entry}

\begin{entry}{似曾相识}{6,12,9,7}{⼈、⽈、⽬、⾔}
  \begin{phonetics}{似曾相识}{si4ceng2xiang1shi2}
    \definition{s.}{\emph{déjà vu} (a experiência de ver exatamente a mesma situação pela segunda vez) | situação aparentemente familiar}
  \end{phonetics}
\end{entry}

\begin{entry}{充}{6}{⼉}
  \begin{phonetics}{充}{chong1}
    \definition*{s.}{Sobrenome Chong}
    \definition{adj.}{suficiente; completo; amplo; cheio}
    \definition{v.}{encher; carregar; atulhar | servir como; agir como | fingir ser; posar como; passar algo como}
  \end{phonetics}
\end{entry}

\begin{entry}{充分}{6,4}{⼉、⼑}
  \begin{phonetics}{充分}{chong1fen4}[][HSK 4]
    \definition{adj.}{cheio; amplo; abundante; suficiente; adequado}
    \definition{adv.}{totalmente; até o fim}
  \end{phonetics}
\end{entry}

\begin{entry}{充电}{6,5}{⼉、⽥}
  \begin{phonetics}{充电}{chong1 dian4}[][HSK 4]
    \definition{v.}{carregar (uma bateria); conectar uma fonte de alimentação CC aos terminais da bateria para recarregar a bateria | relaxar; passar o tempo livre; ``recarregar as baterias''; estudar para adquirir mais conhecimento; reabastecer (ou ampliar) o conhecimento; metaforicamente falando, para reabastecer a força física e a energia por meio do descanso e da recreação; também metaforicamente falando, para reabastecer novos conhecimentos e desenvolver novas habilidades por meio do reaprendizado}
  \end{phonetics}
\end{entry}

\begin{entry}{充电器}{6,5,16}{⼉、⽥、⼝}
  \begin{phonetics}{充电器}{chong1dian4qi4}[][HSK 4]
    \definition{s.}{carregador de bateria; dispositivo para alimentar uma bateria com energia, forçando uma corrente através dela}
  \end{phonetics}
\end{entry}

\begin{entry}{充足}{6,7}{⼉、⾜}
  \begin{phonetics}{充足}{chong1zu2}[][HSK 5]
    \definition{adj.}{bastante; adequado; suficiente}
  \end{phonetics}
\end{entry}

\begin{entry}{充满}{6,13}{⼉、⽔}
  \begin{phonetics}{充满}{chong1man3}[][HSK 3]
    \definition{v.}{preencher; encher; cobrir completamente| estar cheio de; estar repleto de; estar transbordando de; estar impregnado de}
  \end{phonetics}
\end{entry}

\begin{entry}{兆}{6}{⼉}
  \begin{phonetics}{兆}{zhao4}
    \definition{num.}{trilhão}
  \end{phonetics}
\end{entry}

\begin{entry}{先}{6}{⼉}
  \begin{phonetics}{先}{xian1}[][HSK 1]
    \definition*{s.}{Sobrenome Xian}
    \definition{adv.}{primeiro; antes; mais cedo; com antecedência | no momento; por enquanto; em um curto espaço de tempo; temporariamente}
    \definition{s.}{início; começo; em ordem cronológica ou de precedência | ancestral; geração mais velha; antepassado | tardio; falecido; morto (honrar os mortos)}
  \end{phonetics}
\end{entry}

\begin{entry}{先不先}{6,4,6}{⼉、⼀、⼉}
  \begin{phonetics}{先不先}{xian1bu4xian1}
    \definition{adv.}{(dialeto) antes de tudo | em primeiro lugar}
  \end{phonetics}
\end{entry}

\begin{entry}{先天}{6,4}{⼉、⼤}
  \begin{phonetics}{先天}{xian1tian1}
    \definition{adj.}{congênito | inato | natural}
    \definition{s.}{período embrionário}
  \end{phonetics}
\end{entry}

\begin{entry}{先生}{6,5}{⼉、⽣}
  \begin{phonetics}{先生}{xian1sheng5}[][HSK 1]
    \definition[个,位]{s.}{professor; títulos honoríficos para professores, médicos, etc. | marido; antigamente, referia-se ao marido de outra pessoa ou ao próprio marido (ambos com pronomes pessoais como determinantes) | médico; títulos usados para se referir aos médicos no passado | refere-se a pessoas cuja profissão envolve contar histórias, adivinhação, etc.; antigamente, era chamado de contador | senhor; \emph{sir}; título dado aos intelectuais}
  \end{phonetics}
\end{entry}

\begin{entry}{先后}{6,6}{⼉、⼝}
  \begin{phonetics}{先后}{xian1 hou4}[][HSK 5]
    \definition{adv.}{sucessivamente; um após o outro}
    \definition{s.}{prioridade; ordem; cedo ou tarde; primeiro e último}
  \end{phonetics}
\end{entry}

\begin{entry}{先有}{6,6}{⼉、⽉}
  \begin{phonetics}{先有}{xian1you3}
    \definition{adj.}{preexistente | anterior}
  \end{phonetics}
\end{entry}

\begin{entry}{先进}{6,7}{⼉、⾡}
  \begin{phonetics}{先进}{xian1jin4}[][HSK 3]
    \definition{adj.}{avançado; progressos rápidos e nível elevado, podendo servir de exemplo a seguir}
    \definition{s.}{indivíduos ou grupos avançados}
  \end{phonetics}
\end{entry}

\begin{entry}{先到先得}{6,8,6,11}{⼉、⼑、⼉、⼻}
  \begin{phonetics}{先到先得}{xian1dao4xian1de2}
    \definition{expr.}{primeiro a chegar | primeiro a ser servido}
  \end{phonetics}
\end{entry}

\begin{entry}{先前}{6,9}{⼉、⼑}
  \begin{phonetics}{先前}{xian1qian2}[][HSK 5]
    \definition[出]{s.}{antes; anteriormente; geralmente se refere ao passado ou a um certo tempo anterior}
  \end{phonetics}
\end{entry}

\begin{entry}{先烈}{6,10}{⼉、⽕}
  \begin{phonetics}{先烈}{xian1lie4}
    \definition{s.}{mártir}
  \end{phonetics}
\end{entry}

\begin{entry}{先验}{6,10}{⼉、⾺}
  \begin{phonetics}{先验}{xian1yan4}
    \definition{adj.}{(filosofia) a priori}
  \end{phonetics}
\end{entry}

\begin{entry}{先期}{6,12}{⼉、⽉}
  \begin{phonetics}{先期}{xian1qi1}
    \definition{adv.}{antecipadamente}
    \definition{s.}{prematuro | \emph{front-end}}
  \end{phonetics}
\end{entry}

\begin{entry}{光}{6}{⼉}
  \begin{phonetics}{光}{guang1}[][HSK 3]
    \definition*{s.}{Sobrenome Guang}
    \definition{adj.}{suave; liso; brilhante | esgotado; sem nada sobrando | brilhante}
    \definition{adv.}{somente; sozinho; meramente}
    \definition{s.}{luz; raio | cenário; paisagem | honra; glória; brilho | claridade | favor; graça | momento | corpo celeste; referindo-se especificamente a corpos celestes, como o sol, a lua e as estrelas}
    \definition{v.}{glorificar; recuperar; reconquistar | estar nu; expor}
  \end{phonetics}
\end{entry}

\begin{entry}{光污染}{6,6,9}{⼉、⽔、⽊}
  \begin{phonetics}{光污染}{guang1 wu1ran3}
    \definition{s.}{poluição luminosa}
  \end{phonetics}
\end{entry}

\begin{entry}{光明}{6,8}{⼉、⽇}
  \begin{phonetics}{光明}{guang1ming2}[][HSK 3]
    \definition{adj.}{brilhante; luminoso | sincero; ingênuo; metáfora da justiça e da esperança | justo; honesto; franco}
    \definition{s.}{luz}
  \end{phonetics}
\end{entry}

\begin{entry}{光线}{6,8}{⼉、⽷}
  \begin{phonetics}{光线}{guang1 xian4}[][HSK 5]
    \definition[条,道]{s.}{luz; feixe luminoso; raio de luz}
  \end{phonetics}
\end{entry}

\begin{entry}{光临}{6,9}{⼉、⼁}
  \begin{phonetics}{光临}{guang1lin2}[][HSK 4]
    \definition{v.}{honrar com sua presença, uma palavra de honra, usada para dizer que um convidado chegou}
  \end{phonetics}
\end{entry}

\begin{entry}{光荣}{6,9}{⼉、⾋}
  \begin{phonetics}{光荣}{guang1rong2}[][HSK 5]
    \definition{adj.}{honroso; honrado; glorioso; por fazer algo que é benéfico para o país ou para a coletividade e que é considerado por todos como digno de respeito ou elogio}
    \definition{s.}{honra; glória; crédito; sentimento de honra decorrente do fato de ser respeitado ou elogiado por fazer algo importante ou grandioso}
  \end{phonetics}
\end{entry}

\begin{entry}{光盘}{6,11}{⼉、⽫}
  \begin{phonetics}{光盘}{guang1pan2}[][HSK 4]
    \definition[片,张]{s.}{CD; disco compacto; um disco circular feito de plástico rígido composto que usa um laser para registrar e ler informações}
  \end{phonetics}
\end{entry}

\begin{entry}{光槃}{6,14}{⼉、⽊}
  \begin{phonetics}{光槃}{guang1pan2}
    \variantof{光盘}
  \end{phonetics}
\end{entry}

\begin{entry}{全}{6}{⼊}
  \begin{phonetics}{全}{quan2}[][HSK 2]
    \definition*{s.}{Sobrenome Quan}
    \definition{adj.}{completo; total; inteiro}
    \definition{adv.}{inteiramente; totalmente; completamente; significa 100\%; equivalente a 完全 ou 全然}
    \definition{v.}{manter intacto; tornar perfeito ou completo; completar}
  \seealsoref{全然}{quan2ran2}
  \seealsoref{完全}{wan2quan2}
  \end{phonetics}
\end{entry}

\begin{entry}{全世界}{6,5,9}{⼊、⼀、⽥}
  \begin{phonetics}{全世界}{quan2 shi4 jie4}[][HSK 5]
    \definition[种]{s.}{mundo inteiro; mundo todo | em todo o mundo}
  \end{phonetics}
\end{entry}

\begin{entry}{全场}{6,6}{⼊、⼟}
  \begin{phonetics}{全场}{quan2 chang3}[][HSK 3]
    \definition{s.}{toda a audiência; todos os presentes; todo o público}
  \end{phonetics}
\end{entry}

\begin{entry}{全年}{6,6}{⼊、⼲}
  \begin{phonetics}{全年}{quan2 nian2}[][HSK 2]
    \definition{s.}{ano inteiro | anual; todo ano}
  \end{phonetics}
\end{entry}

\begin{entry}{全体}{6,7}{⼊、⼈}
  \begin{phonetics}{全体}{quan2 ti3}[][HSK 2]
    \definition{s.}{(frequentemente referido a pessoas) todos; número total; todos | por todo o corpo | todos; inteiro; a soma de todas as partes; a soma de todos os indivíduos (geralmente se refere a pessoas)}
  \end{phonetics}
\end{entry}

\begin{entry}{全身}{6,7}{⼊、⾝}
  \begin{phonetics}{全身}{quan2 shen1}[][HSK 2]
    \definition{s.}{corpo inteiro; por todo o corpo; todo o corpo}
  \end{phonetics}
\end{entry}

\begin{entry}{全国}{6,8}{⼊、⼞}
  \begin{phonetics}{全国}{quan2 guo2}[][HSK 2]
    \definition{s.}{toda a nação (ou país); em todo o país; em todo o território nacional | toda a nação; todo o país}
  \end{phonetics}
\end{entry}

\begin{entry}{全面}{6,9}{⼊、⾯}
  \begin{phonetics}{全面}{quan2mian4}[][HSK 3]
    \definition{adj.}{geral; completo; abrangente; onipotente}
    \definition{s.}{todos os aspectos; cada aspecto}
  \seealsoref{片面}{pian4mian4}
  \end{phonetics}
\end{entry}

\begin{entry}{全家}{6,10}{⼊、⼧}
  \begin{phonetics}{全家}{quan2 jia1}[][HSK 2]
    \definition{s.}{toda a família; a família inteira}
  \end{phonetics}
\end{entry}

\begin{entry}{全称特命全权大使}{6,10,10,8,6,6,3,8}{⼊、⽲、⽜、⼝、⼊、⽊、⼤、⼈}
  \begin{phonetics}{全称特命全权大使}{quan2cheng1 te4ming4 quan2quan2 da4shi3}
    \definition*{s.}{Embaixador Extraordinário e Plenipotenciário}
  \end{phonetics}
\end{entry}

\begin{entry}{全部}{6,10}{⼊、⾢}
  \begin{phonetics}{全部}{quan2bu4}[][HSK 2]
    \definition{adv.}{tudo; total; inteiro; completo; aplica-se a todos, sem exceção}
    \definition{s.}{totalidade; total; integridade; a soma de todas as partes; o todo}
  \end{phonetics}
\end{entry}

\begin{entry}{全都}{6,10}{⼊、⾢}
  \begin{phonetics}{全都}{quan2 dou1}[][HSK 5]
    \definition{adv.}{tudo; todos; sem exceção}
  \end{phonetics}
\end{entry}

\begin{entry}{全都不}{6,10,4}{⼊、⾢、⼀}
  \begin{phonetics}{全都不}{quan2dou1 bu4}
    \definition{adj.}{nada; nenhum; nenhum deles; nada disso}
  \end{phonetics}
\end{entry}

\begin{entry}{全球}{6,11}{⼊、⽟}
  \begin{phonetics}{全球}{quan2 qiu2}[][HSK 3]
    \definition[门]{s.}{o mundo inteiro; a Terra inteira}
  \end{phonetics}
\end{entry}

\begin{entry}{全职}{6,11}{⼊、⽿}
  \begin{phonetics}{全职}{quan2zhi2}
    \definition{s.}{período integral | tempo inteiro | (trabalho) \emph{full-time}}
  \end{phonetics}
\end{entry}

\begin{entry}{全然}{6,12}{⼊、⽕}
  \begin{phonetics}{全然}{quan2ran2}
    \definition{adv.}{completamente; inteiramente}
  \end{phonetics}
\end{entry}

\begin{entry}{共}{6}{⼋}
  \begin{phonetics}{共}{gong4}[][HSK 4]
    \definition*{s.}{Partido Comunista, abreviação de 共产党 | Sobrenome Gong}
    \definition{adj.}{conjunto; mútuo; geral; comum; o mesmo para todos}
    \definition{adv.}{juntos; juntamente; conjuntamente | em sua totalidade; em todos}
    \definition{v.}{compartilhar com; empreender ou realizar em conjunto}
  \seealsoref{共产党}{gong4chan3dang3}
  \end{phonetics}
\end{entry}

\begin{entry}{共计}{6,4}{⼋、⾔}
  \begin{phonetics}{共计}{gong4ji4}[][HSK 5]
    \definition{s.}{total; total geral; agregado; montante}
    \definition{v.}{contar até; somar até; totalizar}
  \end{phonetics}
\end{entry}

\begin{entry}{共产}{6,6}{⼋、⼇}
  \begin{phonetics}{共产}{gong4chan3}
    \definition{adj.}{comunista}
    \definition{s.}{comunismo}
  \end{phonetics}
\end{entry}

\begin{entry}{共产党}{6,6,10}{⼋、⼇、⼉}
  \begin{phonetics}{共产党}{gong4chan3dang3}
    \definition*{s.}{Partido Comunista}
  \end{phonetics}
\end{entry}

\begin{entry}{共同}{6,6}{⼋、⼝}
  \begin{phonetics}{共同}{gong4tong2}[][HSK 3]
    \definition{adj.}{comum; compartilhado; colaborativo; todos têm}
    \definition{adv.}{juntos; conjuntamente; todos juntos (fazemos)}
  \end{phonetics}
\end{entry}

\begin{entry}{共同体}{6,6,7}{⼋、⼝、⼈}
  \begin{phonetics}{共同体}{gong4tong2ti3}
    \definition{s.}{comunidade}
  \end{phonetics}
\end{entry}

\begin{entry}{共有}{6,6}{⼋、⽉}
  \begin{phonetics}{共有}{gong4 you3}[][HSK 3]
    \definition{v.}{compartilhar; possuir (por todos); possuir ou desfrutar em conjunto}
  \end{phonetics}
\end{entry}

\begin{entry}{共享}{6,8}{⼋、⼇}
  \begin{phonetics}{共享}{gong4 xiang3}[][HSK 5]
    \definition{v.}{compartilhar; desfrutar juntos; aproveitar as coisas boas juntos}
  \end{phonetics}
\end{entry}

\begin{entry}{兲}{6}{⼋}
  \begin{phonetics}{兲}{tian1}
    \variantof{天}
  \end{phonetics}
\end{entry}

\begin{entry}{关}{6}{⼋}
  \begin{phonetics}{关}{guan1}[][HSK 1,4]
    \definition*{s.}{Sobrenome Guan}
    \definition{s.}{passagem; ponto de controle | alfândega; escritórios de cobrança de impostos para exportação e importação de mercadorias | ponto de inflexão ou barreira; ponto de virada ou dificuldade | momento crítico; mecanismo}
    \definition{v.}{fechar; encerrar; amarrar algo | fechar; trancar | encerrar; sair do mercado; falir | conceder ou sacar o pagamento de um salário | desligar | envolver; preocupar-se; conectar-se}
  \end{phonetics}
\end{entry}

\begin{entry}{关上}{6,3}{⼋、⼀}
  \begin{phonetics}{关上}{guan1 shang4}[][HSK 1]
    \definition{v.}{fechar (uma porta); fechar um objeto | desligar (luz, equipamento elétrico etc.); parar ou encerrar (uma atividade, situação, etc.)}
  \end{phonetics}
\end{entry}

\begin{entry}{关于}{6,3}{⼋、⼆}
  \begin{phonetics}{关于}{guan1yu2}[][HSK 4]
    \definition{prep.}{sobre; relativo a; pertencente a; uma questão de; com relação a}
  \end{phonetics}
\end{entry}

\begin{entry}{关心}{6,4}{⼋、⼼}
  \begin{phonetics}{关心}{guan1xin1}[][HSK 2]
    \definition{v.}{cuidar; preocupar-se com; manifestar interesse por; demonstrar solicitude por; (colocar uma pessoa ou coisa) sempre no coração; valorizar e cuidar}
  \end{phonetics}
\end{entry}

\begin{entry}{关机}{6,6}{⼋、⽊}
  \begin{phonetics}{关机}{guan1 ji1}[][HSK 2]
    \definition{v.}{encerrar; terminar; refere-se especificamente à conclusão das filmagens de um filme ou série de TV | desligar; desligar a fonte de alimentação; parar o funcionamento da máquina}
  \end{phonetics}
\end{entry}

\begin{entry}{关闭}{6,6}{⼋、⾨}
  \begin{phonetics}{关闭}{guan1bi4}[][HSK 4]
    \definition{v.}{fechar | (empresa) falir}
  \end{phonetics}
\end{entry}

\begin{entry}{关怀}{6,7}{⼋、⼼}
  \begin{phonetics}{关怀}{guan1huai2}[][HSK 5]
    \definition{v.}{mostrar cuidado amoroso por; mostrar solicitude por; cuidar, amar, apoiar ou ajudar os fracos ou grupos em dificuldade | geralmente usado para superiores para subordinados, anciãos para juniores ou organizações para indivíduos}
  \end{phonetics}
\end{entry}

\begin{entry}{关系}{6,7}{⼋、⽷}
  \begin{phonetics}{关系}{guan1xi5}[][HSK 3]
    \definition[个,种]{s.}{relações; conexões; relacionamento; a interligação entre pessoas ou coisas | consequência; impacto; significado a influência ou importância de algo; algo digno de nota (geralmente usado com 没有, 有). | causa; razão (geralmente usado com 由于 ou 因为); refere-se genericamente a causas, condições, etc. | credenciais que mostram filiação a uma organização; documento que comprova a existência de algum tipo de relação organizacional}
    \definition{v.}{preocupar; afetar; ter influência sobre; ter a ver com}
  \seealsoref{没有}{mei2 you3}
  \seealsoref{因为}{yin1wei4}
  \seealsoref{由于}{you2yu2}
  \seealsoref{有}{you3}
  \end{phonetics}
\end{entry}

\begin{entry}{关注}{6,8}{⼋、⽔}
  \begin{phonetics}{关注}{guan1 zhu4}[][HSK 3]
    \definition{v.}{prestar atenção em; seguir algo de perto; seguir (nas redes sociais)}
  \end{phonetics}
\end{entry}

\begin{entry}{关键}{6,13}{⼋、⾦}
  \begin{phonetics}{关键}{guan1jian4}[][HSK 5]
    \definition{adj.}{crucial; decisivo; importante; que pode determinar o curso e o resultado dos eventos}
    \definition[个]{s.}{chave; ponto crucial; aspectos ou condições mais importantes que determinam o desenvolvimento e o resultado de algo}
  \end{phonetics}
\end{entry}

\begin{entry}{兴}{6}{⼋}
  \begin{phonetics}{兴}{xing1}
    \definition*{s.}{Sobrenome Xing}
    \definition{adv.}{talvez (dialeto)}
    \definition{v.}{subir | florescer | tornar-se popular | começar | encorajar | levantar-se | (frequentemente usado em negativas) permitir (dialeto)}
  \end{phonetics}
  \begin{phonetics}{兴}{xing4}
    \definition{s.}{sentimento ou desejo de fazer algo | interesse em algo | excitação}
  \end{phonetics}
\end{entry}

\begin{entry}{兴奋}{6,8}{⼋、⼤}
  \begin{phonetics}{兴奋}{xing1fen4}[][HSK 4]
    \definition{adj.}{animado; excitante; empolgante;}
    \definition{s.}{excitação; empolgação}
    \definition{v.}{excitar; intoxicar}
  \end{phonetics}
\end{entry}

\begin{entry}{兴趣}{6,15}{⼋、⾛}
  \begin{phonetics}{兴趣}{xing4 qu4}[][HSK 4]
    \definition[个]{s.}{interesse (desejo de conhecer sobre alguma coisa ou coisa no qual está interessado) | \emph{hobby}}
  \end{phonetics}
\end{entry}

\begin{entry}{再}{6}{⼌}
  \begin{phonetics}{再}{zai4}[][HSK 1]
    \definition{adv.}{mais uma vez; além disso; ainda mais; indica a repetição ou continuação de uma mesma ação ou comportamento; refere-se principalmente a ações ou comportamentos não realizados ou contínuos | usado antes do adjetivo, indica intensificação, equivalente a 更 ou 更加 | (para uma ação adiada, precedida por uma expressão de tempo ou condição) então; somente então; depois de algo; indica que a ação ocorrerá após a conclusão de outra ação | além disso; indica um complemento, equivalente a 另外 ou 又 | próxima vez; indica que a ação ocorrerá após um determinado período de tempo | novamente; de novo}
  \seealsoref{更}{geng4}
  \seealsoref{更加}{geng4 jia1}
  \seealsoref{另外}{ling4wai4}
  \seealsoref{又}{you4}
  \end{phonetics}
\end{entry}

\begin{entry}{再三}{6,3}{⼌、⼀}
  \begin{phonetics}{再三}{zai4san1}[][HSK 4]
    \definition{adv.}{repetidamente; repetidas vezes; de novo e de novo}
  \end{phonetics}
\end{entry}

\begin{entry}{再也}{6,3}{⼌、⼄}
  \begin{phonetics}{再也}{zai4 ye3}[][HSK 5]
    \definition{adv.}{não mais; nunca mais; uma determinada situação ou ação nunca mais ocorrerá}
  \end{phonetics}
\end{entry}

\begin{entry}{再不}{6,4}{⼌、⼀}
  \begin{phonetics}{再不}{zai4bu4}
    \definition{adv.}{nunca mais}
  \end{phonetics}
\end{entry}

\begin{entry}{再见}{6,4}{⼌、⾒}
  \begin{phonetics}{再见}{zai4jian4}[][HSK 1]
    \definition{v.}{adeus; tchau; até logo; até mais; até mais tarde}
  \end{phonetics}
\end{entry}

\begin{entry}{再发}{6,5}{⼌、⼜}
  \begin{phonetics}{再发}{zai4fa1}
    \definition{v.}{reenviar}
  \end{phonetics}
\end{entry}

\begin{entry}{再生}{6,5}{⼌、⽣}
  \begin{phonetics}{再生}{zai4sheng1}
    \definition{s.}{reciclagem | regeneração}
    \definition{v.}{reciclar | renascer | regenerar}
  \end{phonetics}
\end{entry}

\begin{entry}{再次}{6,6}{⼌、⽋}
  \begin{phonetics}{再次}{zai4 ci4}[][HSK 5]
    \definition{adv.}{mais uma vez; uma segunda vez; outra vez}
  \end{phonetics}
\end{entry}

\begin{entry}{再审}{6,8}{⼌、⼧}
  \begin{phonetics}{再审}{zai4shen3}
    \definition{s.}{novo julgamento | revisão}
    \definition{v.}{ouvir um caso novamente}
  \end{phonetics}
\end{entry}

\begin{entry}{再者}{6,8}{⼌、⽼}
  \begin{phonetics}{再者}{zai4zhe3}
    \definition{conj.}{além do mais | além disso}
  \end{phonetics}
\end{entry}

\begin{entry}{再育}{6,8}{⼌、⾁}
  \begin{phonetics}{再育}{zai4yu4}
    \definition{v.}{aumentar | multiplicar | proliferar}
  \end{phonetics}
\end{entry}

\begin{entry}{再临}{6,9}{⼌、⼁}
  \begin{phonetics}{再临}{zai4lin2}
    \definition{v.}{vir de novo}
  \end{phonetics}
\end{entry}

\begin{entry}{再度}{6,9}{⼌、⼴}
  \begin{phonetics}{再度}{zai4du4}
    \definition{adv.}{outra vez | mais uma vez}
  \end{phonetics}
\end{entry}

\begin{entry}{再说}{6,9}{⼌、⾔}
  \begin{phonetics}{再说}{zai4shuo1}
    \definition{conj.}{além do mais | além disso | o que mais}
    \definition{v.}{adiar uma discussão para mais tarde | dizer novamente}
  \end{phonetics}
\end{entry}

\begin{entry}{再读}{6,10}{⼌、⾔}
  \begin{phonetics}{再读}{zai4du2}
    \definition{v.}{ler novamente | rever (uma lição, etc.)}
  \end{phonetics}
\end{entry}

\begin{entry}{军}{6}{⼍}
  \begin{phonetics}{军}{jun1}
    \definition*{s.}{Sobrenome Jun}
    \definition{s.}{forças armadas; exército; tropas | exército; contingente; muitas pessoas participando de uma atividade | exército; unidades militares}
  \end{phonetics}
\end{entry}

\begin{entry}{军人}{6,2}{⼍、⼈}
  \begin{phonetics}{军人}{jun1 ren2}[][HSK 5]
    \definition{s.}{soldado; militar; pessoal militar; pessoas com status militar; pessoas servindo nas forças armadas}
  \end{phonetics}
\end{entry}

\begin{entry}{军装}{6,12}{⼍、⾐}
  \begin{phonetics}{军装}{jun1zhuang1}
    \definition{s.}{uniforme militar}
  \end{phonetics}
\end{entry}

\begin{entry}{农}{6}{⼍}
  \begin{phonetics}{农}{nong2}
    \definition*{s.}{Sobrenome Nong}
    \definition{s.}{agricultura; criação de animais | camponês; fazendeiro}
  \end{phonetics}
\end{entry}

\begin{entry}{农业}{6,5}{⼍、⼀}
  \begin{phonetics}{农业}{nong2ye4}[][HSK 3]
    \definition{s.}{agricultura}
  \end{phonetics}
\end{entry}

\begin{entry}{农民}{6,5}{⼍、⽒}
  \begin{phonetics}{农民}{nong2min2}[][HSK 3]
    \definition[个,位,名,些]{s.}{fazendeiro; camponês; campesinato; trabalhadores que participam da produção agrícola há muito tempo}
  \end{phonetics}
\end{entry}

\begin{entry}{农产品}{6,6,9}{⼍、⼇、⼝}
  \begin{phonetics}{农产品}{nong2 chan3 pin3}[][HSK 5]
    \definition{s.}{produtos agrícolas}
  \end{phonetics}
\end{entry}

\begin{entry}{农村}{6,7}{⼍、⽊}
  \begin{phonetics}{农村}{nong2cun1}[][HSK 3]
    \definition{s.}{aldeia; campo; área rural; locais onde vivem os trabalhadores principalmente dedicados à produção agrícola}
  \end{phonetics}
\end{entry}

\begin{entry}{冰}{6}{⼎}
  \begin{phonetics}{冰}{bing1}[][HSK 4]
    \definition[块,层,些]{s.}{gelo; água em estado sólido |  algo parecido com gelo | (gíria) metanfetamina}
    \definition{v.}{colocar gelo; colocar gelo ao redor; colocar no gelo; resfriar objetos com gelo ou água fria | sentir frio}
  \end{phonetics}
\end{entry}

\begin{entry}{冰天雪地}{6,4,11,6}{⼎、⼤、⾬、⼟}
  \begin{phonetics}{冰天雪地}{bing1tian1-xue3di4}
    \definition{expr.}{um mundo de gelo e neve}
  \end{phonetics}
\end{entry}

\begin{entry}{冰球}{6,11}{⼎、⽟}
  \begin{phonetics}{冰球}{bing1qiu2}
    \definition[个]{s.}{hóquei no gelo | disco; a ``bola'' usada no hóquei no gelo}
  \end{phonetics}
\end{entry}

\begin{entry}{冰雪}{6,11}{⼎、⾬}
  \begin{phonetics}{冰雪}{bing1 xue3}[][HSK 4]
    \definition{adj.}{puro como gelo e neve; descreve uma pessoa como pura}
    \definition{s.}{gelo e neve}
  \end{phonetics}
\end{entry}

\begin{entry}{冰棍}{6,12}{⼎、⽊}
  \begin{phonetics}{冰棍}{bing1gun4}
    \definition[根]{s.}{picolé}
  \end{phonetics}
\end{entry}

\begin{entry}{冰箱}{6,15}{⼎、⾋}
  \begin{phonetics}{冰箱}{bing1xiang1}[][HSK 4]
    \definition[台,个]{s.}{geladeira; freezer; refrigerador; aparelhos para congelar alimentos ou medicamentos com gelo para mantê-los frios}
  \end{phonetics}
\end{entry}

\begin{entry}{冰激凌}{6,16,10}{⼎、⽔、⼎}
  \begin{phonetics}{冰激凌}{bing1ji1ling2}
    \definition{s.}{sorvete}
  \end{phonetics}
\end{entry}

\begin{entry}{冰糕}{6,16}{⼎、⽶}
  \begin{phonetics}{冰糕}{bing1gao1}
    \definition{s.}{sorvete | picolé}
  \end{phonetics}
\end{entry}

\begin{entry}{冲}{6}{⼎}
  \begin{phonetics}{冲}{chong1}[][HSK 4,6]
    \definition{s.}{via pública; local importante; via de passagem; via local importante | um trecho de planície em uma área montanhosa | (astronomia) oposição; os planetas externos orbitam até ficarem alinhados com a Terra e o Sol, e a Terra está no meio}
    \definition{v.}{atacar; apressar; correr; passar rapidamente; passar por um obstáculo | colidir; chocar; bater | despejar água fervente sobre | enxaguar; dar descarga; lavar | revelar (filme) | neutralizar a má sorte}
  \end{phonetics}
  \begin{phonetics}{冲}{chong4}
    \definition{adj.}{poderoso; com vigor; com muita força; vigoroso | forte; odor forte e pungente (olfato)}
    \definition{prep.}{de frente; em direção a | na força de; com base em; em virtude de}
    \definition{v.}{estampar (máquina de estamparia)}
  \end{phonetics}
\end{entry}

\begin{entry}{冲击}{6,5}{⼎、⼐}
  \begin{phonetics}{冲击}{chong1ji1}[][HSK 6]
    \definition{v.}{chicotear; bater | correr; voar; atacar; assaltar; atacar bravamente em direção a um alvo predeterminado | chocar; metáfora para interferência ou golpe sério}
  \end{phonetics}
\end{entry}

\begin{entry}{冲动}{6,6}{⼎、⼒}
  \begin{phonetics}{冲动}{chong1dong4}[][HSK 5]
    \definition{adj.}{impulsivo; impetuoso}
    \definition{s.}{impulso; impetuosidade; impulso de movimento; fenômeno psicológico no qual as emoções são particularmente fortes e o controle racional é fraco}
    \definition{v.}{ficar animado; ser impetuoso; agir por impulso}
  \end{phonetics}
\end{entry}

\begin{entry}{冲突}{6,9}{⼎、⽳}
  \begin{phonetics}{冲突}{chong1tu1}[][HSK 5]
    \definition{v.}{chocar-se; entrar em conflito; conflitar | contradizer; duas coisas opostas que interferem uma na outra}
  \end{phonetics}
\end{entry}

\begin{entry}{冲浪}{6,10}{⼎、⽔}
  \begin{phonetics}{冲浪}{chong1lang4}
    \definition{s.}{surfe}
    \definition{v.}{surfar}
  \end{phonetics}
\end{entry}

\begin{entry}{冲锋}{6,12}{⼎、⾦}
  \begin{phonetics}{冲锋}{chong1feng1}
    \definition{v.}{cobrar | tomar de assalto}
  \end{phonetics}
\end{entry}

\begin{entry}{决}{6}{⼎}
  \begin{phonetics}{决}{jue2}
    \definition{v.}{decidir; determinar | executar uma pessoa | (de um dique, etc.) romper; desabar}
  \end{phonetics}
\end{entry}

\begin{entry}{决不}{6,4}{⼎、⼀}
  \begin{phonetics}{决不}{jue2 bu4}[][HSK 5]
    \definition{adv.}{definitivamente não; certamente não; sob nenhuma circunstância; de forma alguma}
  \end{phonetics}
\end{entry}

\begin{entry}{决心}{6,4}{⼎、⼼}
  \begin{phonetics}{决心}{jue2xin1}[][HSK 3]
    \definition{s.}{resolução; determinação; determinação inabalável}
    \definition{v.}{secidir-se; decidir fazer algo e não vacilar nem mudar de ideia}
  \end{phonetics}
\end{entry}

\begin{entry}{决定}{6,8}{⼎、⼧}
  \begin{phonetics}{决定}{jue2ding4}[][HSK 3]
    \definition{adj.}{decisivo; as leis objetivas levam as coisas a se desenvolverem e mudarem em determinada direção}
    \definition[项,个]{s.}{decisão; resolução; assuntos decididos}
    \definition{v.}{decidir; determinar; algo se torna a base ou o pré-requisito para outra coisa; desempenha um papel dominante | decidir; resolver; tomar uma decisão; propor uma forma de agir}
  \end{phonetics}
\end{entry}

\begin{entry}{决赛}{6,14}{⼎、⾙}
  \begin{phonetics}{决赛}{jue2sai4}[][HSK 3]
    \definition[场]{s.}{finais (de uma competição); em competições esportivas, a última partida ou rodada disputada para determinar a classificação}
  \end{phonetics}
\end{entry}

\begin{entry}{划}{6}{⼑}
  \begin{phonetics}{划}{hua2}[][HSK 4]
    \definition{adj.}{rentável; vale (o esforço); compensa (fazer alguma coisa)}
    \definition{v.}{remar | ser vantajoso para alguém; ser uma pechincha | arranhar; cortar a superfície de; cortar em outra coisa com um objeto pontiagudo | arranhar; golpear;  esfregar uma coisa ou varrer sobre outra}
  \end{phonetics}
  \begin{phonetics}{划}{hua4}[][HSK 4]
    \definition*{s.}{Sobrenome Hua}
    \definition{s.}{traço de um caracter chinês}
    \definition{v.}{delimitar; diferenciar; delinear | transferir; ceder | planejar; programar | desenhar; marcar; delinear; fazer linhas ou escrever como marcadores com uma caneta ou objeto semelhante a uma caneta}
  \end{phonetics}
\end{entry}

\begin{entry}{划分}{6,4}{⼑、⼑}
  \begin{phonetics}{划分}{hua4fen1}[][HSK 5]
    \definition{v.}{dividir; particionar; reparticionar | diferenciar; encontrar aspectos diferentes}
  \end{phonetics}
\end{entry}

\begin{entry}{划船}{6,11}{⼑、⾈}
  \begin{phonetics}{划船}{hua2 chuan2}[][HSK 3]
    \definition[次,回]{s.}{remo (ato de remar); passeios de barco; a atividade ou esporte de “remar um barco com remos”}
    \definition{v.}{remar um barco; a ação ou comportamento de mover um barco na água usando remos}
  \end{phonetics}
\end{entry}

\begin{entry}{划艇}{6,12}{⼑、⾈}
  \begin{phonetics}{划艇}{hua2ting3}
    \definition{s.}{barco a remo}
  \end{phonetics}
\end{entry}

\begin{entry}{列}{6}{⼑}
  \begin{phonetics}{列}{lie4}[][HSK 4]
    \definition{v.}{organizar; formar uma linha; alinhar | listar; inserir em uma lista}
  \end{phonetics}
\end{entry}

\begin{entry}{列入}{6,2}{⼑、⼊}
  \begin{phonetics}{列入}{lie4 ru4}[][HSK 4]
    \definition{v.}{incluir em uma lista}
  \end{phonetics}
\end{entry}

\begin{entry}{列为}{6,4}{⼑、⼂}
  \begin{phonetics}{列为}{lie4 wei2}[][HSK 4]
    \definition{v.}{ser classificado como; ser listado como}
  \end{phonetics}
\end{entry}

\begin{entry}{列车}{6,4}{⼑、⾞}
  \begin{phonetics}{列车}{lie4che1}[][HSK 4]
    \definition{s.}{trem; trem em uma composição contínua, puxado por uma locomotiva e equipado com uma tripulação e marcações prescritas; geralmente um trem de passageiros}
  \end{phonetics}
\end{entry}

\begin{entry}{刘}{6}{⼑}
  \begin{phonetics}{刘}{liu2}
    \definition*{s.}{Sobrenome Liu}
    \definition{s.}{Clássico: um tipo de machado de batalha}
    \definition{v.}{matar}
  \end{phonetics}
\end{entry}

\begin{entry}{刚}{6}{⼑}
  \begin{phonetics}{刚}{gang1}[][HSK 2]
    \definition*{s.}{Sobrenome Gang}
    \definition{adj.}{duro; firme; rígido; forte; (personalidade, atitude) forte; (vontade) determinada}
    \definition{adv.}{apenas; exatamente; justamente | apenas; apenas por pouco; significa atingir um certo nível com dificuldade | apenas; há pouco tempo; indica que a ação ou situação ocorreu há pouco tempo | assim que; somente neste momento; aconteceu que; use a palavra 就 para indicar que duas coisas estão intimamente relacionadas}
  \seealsoref{就}{jiu4}
  \end{phonetics}
\end{entry}

\begin{entry}{刚才}{6,3}{⼑、⼿}
  \begin{phonetics}{刚才}{gang1cai2}[][HSK 2]
    \definition{s.}{agora mesmo; há pouco; há pouco tempo; referindo-se ao período recente que acabou de passar}
  \end{phonetics}
\end{entry}

\begin{entry}{刚刚}{6,6}{⼑、⼑}
  \begin{phonetics}{刚刚}{gang1 gang5}[][HSK 2]
    \definition{adv.}{apenas; somente; exatamente; refere-se a algo que é adequado em termos de grau, quantidade, tempo, etc., nem mais nem menos, nem cedo nem tarde, atingindo um estado satisfatório ou que atende exatamente às necessidades | agora mesmo; há pouco; há um momento atrás; referindo-se a um período de tempo muito curto no passado}
  \end{phonetics}
\end{entry}

\begin{entry}{创}{6}{⼑}
  \begin{phonetics}{创}{chuang1}
    \definition{s.}{ferimento; trauma}
  \end{phonetics}
  \begin{phonetics}{创}{chuang4}
    \definition{v.}{começar (fazer algo); alcançar (algo pela primeira vez); estabelecer; fazer pela primeira vez | estabelecer; fundar; criar; perceber algo novo, como um começo | ferir; machucar}
  \end{phonetics}
\end{entry}

\begin{entry}{创办}{6,4}{⼑、⼒}
  \begin{phonetics}{创办}{chuang4 ban4}[][HSK 6]
    \definition{v.}{estabelecer; montar; fundar}
  \end{phonetics}
\end{entry}

\begin{entry}{创业}{6,5}{⼑、⼀}
  \begin{phonetics}{创业}{chuang4ye4}[][HSK 3]
    \definition{s.}{empreendedorismo}
    \definition{v.}{começar um empreendimento; iniciar/fundar um negócio, uma empresa;}
  \end{phonetics}
\end{entry}

\begin{entry}{创立}{6,5}{⼑、⽴}
  \begin{phonetics}{创立}{chuang4li4}[][HSK 5]
    \definition{v.}{fundar; originar; estabelecer}
  \end{phonetics}
\end{entry}

\begin{entry}{创作}{6,7}{⼑、⼈}
  \begin{phonetics}{创作}{chuang4zuo4}[][HSK 3]
    \definition[个]{s.}{criação; trabalho criativo; obras literárias e artísticas}
    \definition{v.}{escrever; criar; produzir; compor; criar obras artísticas}
  \end{phonetics}
\end{entry}

\begin{entry}{创建}{6,8}{⼑、⼵}
  \begin{phonetics}{创建}{chuang4 jian4}[][HSK 6]
    \definition{v.}{fundar; estabelecer; montar}
  \end{phonetics}
\end{entry}

\begin{entry}{创造}{6,10}{⼑、⾡}
  \begin{phonetics}{创造}{chuang4zao4}[][HSK 3]
    \definition{s.}{criação; inovação; primeiro a concluir ou a alcançar resultados}
    \definition{v.}{criar; produzir; trazer à tona; fazer ou estabelecer pela primeira vez; referir-se de maneira geral a fazer ou estabelecer}
  \end{phonetics}
\end{entry}

\begin{entry}{创意}{6,13}{⼑、⼼}
  \begin{phonetics}{创意}{chuang4 yi4}[][HSK 6]
    \definition[个]{s.}{criatividade; originalidade; novidade; uma ideia original, conceito, etc.}
    \definition{v.}{inovar; criar um novo conceito, ideia, etc. | propor designs criativos, ideias, etc.}
  \end{phonetics}
\end{entry}

\begin{entry}{创新}{6,13}{⼑、⽄}
  \begin{phonetics}{创新}{chuang4xin1}[][HSK 3]
    \definition[个,种,次]{s.}{inovação; algo novo ou diferente, uma ideia}
    \definition{v.}{trazer novas ideias; inovar; abrir novos caminhos; criar ou fazer algo novo, diferente do que era antes}
  \end{phonetics}
\end{entry}

\begin{entry}{动}{6}{⼒}
  \begin{phonetics}{动}{dong4}[][HSK 1]
    \definition{adj.}{não estacionário; móvel; variável; mutável}
    \definition{adv.}{facilmente; frequentemente}
    \definition{s.}{ação; movimento}
    \definition{v.}{mover; mexer; (pessoas ou coisas) mudar a posição ou o estado original | agir; começar a agir; entrar em ação | alterar; mudar; alterar a posição ou o estado original | usar; utilizar; tornar ativo | despertar; tocar (o coração de alguém); provocar mudanças emocionais, reações | [geralmente na forma negativa] comer ou beber | emocionar; deixar emocionado}
  \end{phonetics}
\end{entry}

\begin{entry}{动人}{6,2}{⼒、⼈}
  \begin{phonetics}{动人}{dong4 ren2}[][HSK 3]
    \definition{adj.}{comovente; emocionante; tocante}
  \end{phonetics}
\end{entry}

\begin{entry}{动力}{6,2}{⼒、⼒}
  \begin{phonetics}{动力}{dong4li4}
    \definition[种,个]{s.}{poder; a força que faz com que as máquinas funcionem, por exemplo, energia elétrica, eólica, hidráulica, etc. | ímpeto; força motriz (ou propulsora); refere-se, de maneira geral, à força que impulsiona o desenvolvimento das coisas}
  \end{phonetics}
\end{entry}

\begin{entry}{动手}{6,4}{⼒、⼿}
  \begin{phonetics}{动手}{dong4shou3}[][HSK 5]
    \definition{v.+compl.}{iniciar o trabalho; começar a trabalhar | tocar; manusear; manipular | bater; levantar a mão (para bater); espancar}
  \end{phonetics}
\end{entry}

\begin{entry}{动机}{6,6}{⼒、⽊}
  \begin{phonetics}{动机}{dong4ji1}[][HSK 5]
    \definition[部]{s.}{motivo; razão; intenção; ideias que motivam as pessoas a se envolverem em determinados comportamentos}
  \end{phonetics}
\end{entry}

\begin{entry}{动作}{6,7}{⼒、⼈}
  \begin{phonetics}{动作}{dong4zuo4}[][HSK 1]
    \definition[个]{s.}{movimento; ação; atividade de todo o corpo ou parte do corpo}
    \definition{v.}{agir; começar a se mover; entrar em ação}
  \end{phonetics}
\end{entry}

\begin{entry}{动员}{6,7}{⼒、⼝}
  \begin{phonetics}{动员}{dong4yuan2}[][HSK 5]
    \definition{v.}{despertar; mobilizar; iniciar (para fazer algo ou participar de uma atividade) | mobilizar toda a nação; transferir dos setores militar, político e econômico para uma situação de guerra}
  \end{phonetics}
\end{entry}

\begin{entry}{动身}{6,7}{⼒、⾝}
  \begin{phonetics}{动身}{dong4shen1}
    \definition{v.+compl.}{fazer uma jornada | começar uma jornada | partir | partir em uma jornada | sair (para um lugar distante)}
  \end{phonetics}
\end{entry}

\begin{entry}{动态}{6,8}{⼒、⼼}
  \begin{phonetics}{动态}{dong4tai4}[][HSK 5]
    \definition{s.}{tendências; desenvolvimentos; tendência geral dos assuntos; causa provável de ação; curso dos acontecimentos | expressão; comportamento ativo | estado dinâmico; condição dinâmica; de ou em relação a um estado de movimento}
  \end{phonetics}
\end{entry}

\begin{entry}{动物}{6,8}{⼒、⽜}
  \begin{phonetics}{动物}{dong4wu4}[][HSK 2]
    \definition[个,只,群,种]{s.}{animal; uma grande classe de seres vivos, que se alimentam principalmente de matéria orgânica, possuem sistema nervoso, são sensíveis e capazes de se mover; refere-se a todos os tipos de coisas concretas ou abstratas}
  \end{phonetics}
\end{entry}

\begin{entry}{动物园}{6,8,7}{⼒、⽜、⼞}
  \begin{phonetics}{动物园}{dong4 wu4 yuan2}[][HSK 2]
    \definition[个,座,家]{s.}{jardim zoológico; zoo; parque que cria muitos tipos de animais (especialmente animais com valor científico ou raros na região) para exibição ao público}
  \end{phonetics}
\end{entry}

\begin{entry}{动画片}{6,8,4}{⼒、⽥、⽚}
  \begin{phonetics}{动画片}{dong4hua4pian4}[][HSK 4]
    \definition[部]{s.}{desenho animado; animações; filme de animação}
  \end{phonetics}
\end{entry}

\begin{entry}{动感}{6,13}{⼒、⼼}
  \begin{phonetics}{动感}{dong4gan3}
    \definition{adj.}{dinâmica | vívida}
    \definition{adv.}{dinamicamente}
    \definition{s.}{senso de movimento (geralmente em uma obra de arte estática)}
  \end{phonetics}
\end{entry}

\begin{entry}{动摇}{6,13}{⼒、⼿}
  \begin{phonetics}{动摇}{dong4 yao2}[][HSK 4]
    \definition{adj.}{instável}
    \definition{v.}{ondular; pairar; agitar; balançar; sacudir | hesitar; vacilar; esmorecer; abalar}
  \end{phonetics}
\end{entry}

\begin{entry}{动漫}{6,14}{⼒、⽔}
  \begin{phonetics}{动漫}{dong4man4}
    \definition{s.}{desenhos animados | quadrinhos | anime | mangá}
  \end{phonetics}
\end{entry}

\begin{entry}{匈}{6}{⼓}
  \begin{phonetics}{匈}{xiong1}
    \definition*{s.}{Hungria, abreviação de 匈牙利}
    \definition{s.}{peito; seio; tórax}
  \seealsoref{匈牙利}{xiong1ya2li4}
  \end{phonetics}
\end{entry}

\begin{entry}{匈牙利}{6,4,7}{⼓、⽛、⼑}
  \begin{phonetics}{匈牙利}{xiong1ya2li4}
    \definition*{s.}{Hungria}
  \end{phonetics}
\end{entry}

\begin{entry}{匈奴}{6,5}{⼓、⼥}
  \begin{phonetics}{匈奴}{xiong1nu2}
    \definition*{s.}{Xiongnu, um povo da estepe oriental que criou um império que floresceu na época das dinastias Qin e Han}
  \end{phonetics}
\end{entry}

\begin{entry}{匠}{6}{⼕}
  \begin{phonetics}{匠}{jiang4}
    \definition{s.}{artesão}
  \end{phonetics}
\end{entry}

\begin{entry}{华}{6}{⼗}
  \begin{phonetics}{华}{hua2}
    \definition*{s.}{China; refere-se à China (anteriormente conhecida como Huaxia, 华夏, mais tarde chamada de Zhonghua, 中华, ou simplesmente Hua, 华)}
    \definition{adj.}{esplêndido; magnífico | próspero; florescente | chamativo; extravagante; vaidoso | grisalho}
    \definition{s.}{corona; um halo colorido ao redor do sol ou da lua causado pela difração da luz através das nuvens | creme; melhor parte; a melhor parte das coisas | chinês; refere-se à nacionalidade Han (língua e escrita) | vezes; anos; refere-se a (bons) momentos | elixir; essência líquida; substâncias formadas pela sedimentação de minerais na água de nascente | Seu, palavra honorífica, usada para se referir a coisas relacionadas à outra pessoa}
  \seealsoref{华夏}{hua2xia4}
  \seealsoref{中华}{zhong1hua2}
  \end{phonetics}
  \begin{phonetics}{华}{hua4}
    \definition*{s.}{Huashan Mountain (na província de Shaanxi) | Sobrenome Hua}
  \end{phonetics}
\end{entry}

\begin{entry}{华人}{6,2}{⼗、⼈}
  \begin{phonetics}{华人}{hua2 ren2}[][HSK 3]
    \definition[名,位,个]{s.}{Chinês; chinês étnico | chineses no exterior; refere-se a cidadãos estrangeiros de ascendência chinesa que obtiveram a nacionalidade do país em que residem}
  \end{phonetics}
\end{entry}

\begin{entry}{华氏}{6,4}{⼗、⽒}
  \begin{phonetics}{华氏}{hua2shi4}
    \definition{s.}{graus Fahrenheit (°F)}
  \end{phonetics}
\end{entry}

\begin{entry}{华语}{6,9}{⼗、⾔}
  \begin{phonetics}{华语}{hua2 yu3}[][HSK 5]
    \definition*{s.}{Chinês (idioma)}
  \end{phonetics}
\end{entry}

\begin{entry}{华夏}{6,10}{⼗、⼢}
  \begin{phonetics}{华夏}{hua2xia4}
    \definition*{s.}{Huaxia, nome antigo da China | Catai}
  \end{phonetics}
\end{entry}

\begin{entry}{华盛顿}{6,11,10}{⼗、⽫、⾴}
  \begin{phonetics}{华盛顿}{hua2sheng4dun4}
    \definition*{s.}{Washington}
  \end{phonetics}
\end{entry}

\begin{entry}{华裔}{6,13}{⼗、⾐}
  \begin{phonetics}{华裔}{hua2yi4}
    \definition{s.}{descendente de chinês}
  \end{phonetics}
\end{entry}

\begin{entry}{协}{6}{⼗}
  \begin{phonetics}{协}{xie2}
    \definition*{s.}{Sobrenome Xie}
    \definition{adv.}{conjuntamente; coordenadamente; juntos}
    \definition{s.}{harmonioso}
    \definition{v.}{auxiliar; assistir; ajudar}
  \end{phonetics}
\end{entry}

\begin{entry}{协议}{6,5}{⼗、⾔}
  \begin{phonetics}{协议}{xie2yi4}[][HSK 5]
    \definition[项]{s.}{acordo; tratado; decisão conjunta alcançada através de negociação e consulta}
    \definition{v.}{concordar em}
  \end{phonetics}
\end{entry}

\begin{entry}{协议书}{6,5,4}{⼗、⾔、⼄}
  \begin{phonetics}{协议书}{xie2 yi4 shu1}[][HSK 5]
    \definition{s.}{contrato | protocolo}
  \end{phonetics}
\end{entry}

\begin{entry}{危}{6}{⼙}
  \begin{phonetics}{危}{wei1}
    \definition*{s.}{Wei, a décima segunda das vinte e oito constelações em que a esfera celeste foi dividida, consistindo de três estrelas em forma de triângulo obtuso, uma em Aquário e duas em Pégaso | Wei, uma das mansões lunares | Sobrenome Wei}
    \definition{adj.}{arriscado; inseguro; perigoso (oposto a 安) | estar gravemente doente; estar morrendo | alto; íngreme}
    \definition{s.}{perigo | cumeeira (de um telhado)}
    \definition{v.}{pôr em perigo; colocar em perigo; comprometer}
  \seealsoref{安}{an1}
  \end{phonetics}
\end{entry}

\begin{entry}{危急}{6,9}{⼙、⼼}
  \begin{phonetics}{危急}{wei1ji2}
    \definition{adj.}{crítico | desesperadora (situação)}
  \end{phonetics}
\end{entry}

\begin{entry}{危险}{6,9}{⼙、⾩}
  \begin{phonetics}{危险}{wei1xian3}[][HSK 3]
    \definition{adj.}{arriscado; perigoso}
  \end{phonetics}
\end{entry}

\begin{entry}{危害}{6,10}{⼙、⼧}
  \begin{phonetics}{危害}{wei1hai4}[][HSK 3]
    \definition[个,种]{s.}{prejuízo; perigo; dano}
    \definition{v.}{destruir; prejudicar; pôr em perigo; pôr em risco}
  \end{phonetics}
\end{entry}

\begin{entry}{危难}{6,10}{⼙、⾫}
  \begin{phonetics}{危难}{wei1nan4}
    \definition{s.}{calamidade}
  \end{phonetics}
\end{entry}

\begin{entry}{压}{6}{⼚}
  \begin{phonetics}{压}{ya1}[][HSK 3]
    \definition{v.}{pressionar; empurrar para baixo; segurar; pesar | acalmar emoções agitadas ou situações ruins; tranquilizar | intimidar; reprimir; exercer pressão sobre; usar poder, posição ou padrões morais para coagir ou restringir as pessoas, impedindo-as de se expressar, decidir ou se desenvolver livremente | aproximar-se; estar chegando perto | arquivar; deixar de lado | pressionar; metáfora para uma grande carga emocional e psicológica | superar; ultrapassar; voz, capacidade e presença mais fortes do que os outros | apostar em um determinado resultado ao jogar | pressionar; força na superfície de contato do objeto}
  \end{phonetics}
  \begin{phonetics}{压}{ya4}
    \definition{adv.}{fundamentalmente; nunca (usado principalmente em frases negativas)}
  \seealsoref{压根儿}{ya4 gen1r5}
  \end{phonetics}
\end{entry}

\begin{entry}{压力}{6,2}{⼚、⼒}
  \begin{phonetics}{压力}{ya1li4}[][HSK 3]
    \definition[份,个]{s.}{pressão; força atuando perpendicularmente à superfície de um objeto | pressão; força esmagadora; metáfora para a força que coage e intimida as pessoas (principalmente nos aspectos espirituais e psicológicos) | tensão; fardo; os encargos econômicos, psicológicos e espirituais impostos pelo mundo exterior}
  \end{phonetics}
\end{entry}

\begin{entry}{压岁钱}{6,6,10}{⼚、⼭、⾦}
  \begin{phonetics}{压岁钱}{ya1sui4qian2}
    \definition{s.}{dinheiro da sorte | dinheiro dado às crianças como presente no Ano Novo Chinês}
  \end{phonetics}
\end{entry}

\begin{entry}{压根儿}{6,10,2}{⼚、⽊、⼉}
  \begin{phonetics}{压根儿}{ya4 gen1r5}
    \definition{adv.}{fundamentalmente; nunca (usado principalmente em frases negativas)}
  \end{phonetics}
\end{entry}

\begin{entry}{压碎}{6,13}{⼚、⽯}
  \begin{phonetics}{压碎}{ya1sui4}
    \definition{v.}{esmagar em pedaços}
  \end{phonetics}
\end{entry}

\begin{entry}{压韵}{6,13}{⼚、⾳}
  \begin{phonetics}{压韵}{ya1yun4}
    \variantof{押韵}
  \end{phonetics}
\end{entry}

\begin{entry}{吃}{6}{⼝}
  \begin{phonetics}{吃}{chi1}[][HSK 1]
    \definition{s.}{alimentos; necessidades básicas}
    \definition{v.}{comer; pegar; fazer; colocar alimentos na boca, mastigar e engolir (incluindo sugar e beber) | viver; depender de algo para viver | aniquilar; eliminar (usado principalmente em jogos de guerra e jogos de tabuleiro) | esgotar; exaurir; ser um fardo; ser um esforço | absorver | sofrer; incorrer | entender; compreender | entrar um objeto em outro | expressar aceitação psicológica | fazer suas refeições; comer}
  \end{phonetics}
\end{entry}

\begin{entry}{吃力}{6,2}{⼝、⼒}
  \begin{phonetics}{吃力}{chi1li4}[][HSK 5]
    \definition{adj.}{suado; extenuante; trabalhoso; laborioso | cansado; fatigado}
  \end{phonetics}
\end{entry}

\begin{entry}{吃饭}{6,7}{⼝、⾷}
  \begin{phonetics}{吃饭}{chi1 fan4}[][HSK 1]
    \definition{v.+compl.}{comer; ter (comer) uma refeição | manter-se vivo;  ganhar a vida; refere-se à vida ou à sobrevivência em geral}
  \end{phonetics}
\end{entry}

\begin{entry}{吃屎}{6,9}{⼝、⼫}
  \begin{phonetics}{吃屎}{chi1 shi3}
    \definition{expr.}{Coma merda!}
  \end{phonetics}
\end{entry}

\begin{entry}{吃惊}{6,11}{⼝、⼼}
  \begin{phonetics}{吃惊}{chi1jing1}[][HSK 4]
    \definition{v.+compl.}{ficar assustado; ficar chocado; ficar espantado; pegar de surpresa; ficar assustado inesperadamente}
  \end{phonetics}
\end{entry}

\begin{entry}{各}{6}{⼝}
  \begin{phonetics}{各}{ge4}[][HSK 3]
    \definition{adv.}{de várias maneiras; de diversas formas; respectivamente; indica que algo é feito separadamente ou que possui uma determinada característica separadamente}
    \definition{pron.}{todo; todos; cada; refere-se a todos os indivíduos dentro de um determinado intervalo, equivalente a 每个}
  \seealsoref{每个}{mei3ge4}
  \end{phonetics}
\end{entry}

\begin{entry}{各个}{6,3}{⼝、⼈}
  \begin{phonetics}{各个}{ge4 ge4}[][HSK 4]
    \definition{adv./pron.}{cada | um a um; um após o outro}
  \end{phonetics}
\end{entry}

\begin{entry}{各地}{6,6}{⼝、⼟}
  \begin{phonetics}{各地}{ge4 di4}[][HSK 3]
    \definition{s.}{em todos os lugares; em vários locais}
  \end{phonetics}
\end{entry}

\begin{entry}{各自}{6,6}{⼝、⾃}
  \begin{phonetics}{各自}{ge4zi4}[][HSK 3]
    \definition{pron.}{por si mesmo; por conta própria; cada um por si | cada um; indica cada uma das partes envolvidas}
  \end{phonetics}
\end{entry}

\begin{entry}{各位}{6,7}{⼝、⼈}
  \begin{phonetics}{各位}{ge4 wei4}[][HSK 3]
    \definition{pron.}{todos; toda a gente; todo mundo | cada um}
  \end{phonetics}
\end{entry}

\begin{entry}{各种}{6,9}{⼝、⽲}
  \begin{phonetics}{各种}{ge4 zhong3}[][HSK 3]
    \definition{adv.}{todos os tipos; vários tipos}
  \end{phonetics}
\end{entry}

\begin{entry}{合}{6}{⼝}
  \begin{phonetics}{合}{he2}[][HSK 3]
    \definition{adj.}{todo; completo; inteiro}
    \definition{clas.}{usado para rodadas | 100ml | medida para grãos secos igual a um décimo de 升, ou um centésimo de 斗}
    \definition{s.}{casamento; união matrimonial | (astronomia) conjunção | nota da escala em Gongchepu (工尺谱), correspondente ao 5 na notação musical numerada}
    \definition{v.}{fechar | juntar; combinar (oposto de 分) | adequar-se; concordar; conformar-se a | ser igual a; somar | ser adequado}
  \seealsoref{斗}{dou4}
  \seealsoref{分}{fen1}
  \seealsoref{工尺谱}{gong1 che3 pu3}
  \seealsoref{升}{sheng1}
  \end{phonetics}
\end{entry}

\begin{entry}{合同}{6,6}{⼝、⼝}
  \begin{phonetics}{合同}{he2tong5}[][HSK 4]
    \definition[个,份]{s.}{contrato; acordo; uma disposição para observância mútua por duas ou mais partes na condução de um assunto com o objetivo de determinar seus respectivos direitos e obrigações.}
  \end{phonetics}
\end{entry}

\begin{entry}{合并}{6,6}{⼝、⼲}
  \begin{phonetics}{合并}{he2bing4}[][HSK 5]
    \definition{v.}{fundir; amalgamar; combinar várias coisas em uma coisa só | (doença) ser complicada por outra doença; uma doença levar a outra, ataques simultâneos (de várias doenças)}
  \end{phonetics}
\end{entry}

\begin{entry}{合成}{6,6}{⼝、⼽}
  \begin{phonetics}{合成}{he2cheng2}[][HSK 5]
    \definition{s.}{compor; integrar; combinar; misturar | sintetizar, reação química para transformar uma substância com uma composição simples em uma substância com uma composição complexa}
  \end{phonetics}
\end{entry}

\begin{entry}{合作}{6,7}{⼝、⼈}
  \begin{phonetics}{合作}{he2zuo4}[][HSK 3]
    \definition{v.}{cooperar; colaborar; trabalhar em conjunto; trabalhar em conjunto para realizar algo ou concluir uma tarefa}
  \end{phonetics}
\end{entry}

\begin{entry}{合法}{6,8}{⼝、⽔}
  \begin{phonetics}{合法}{he2fa3}[][HSK 3]
    \definition{adj.}{legal; legítimo; lícito;  justo; válido; em conformidade com as disposições legais}
  \end{phonetics}
\end{entry}

\begin{entry}{合宪性}{6,9,8}{⼝、⼧、⼼}
  \begin{phonetics}{合宪性}{he2xian4xing4}
    \definition{s.}{constitucionalismo}
  \end{phonetics}
\end{entry}

\begin{entry}{合适}{6,9}{⼝、⾡}
  \begin{phonetics}{合适}{he2shi4}[][HSK 2]
    \definition{adj.}{correto; adequado; apropriado; conveniente; em conformidade com a realidade ou com os requisitos objetivos}
  \end{phonetics}
\end{entry}

\begin{entry}{合格}{6,10}{⼝、⽊}
  \begin{phonetics}{合格}{he2ge2}[][HSK 3]
    \definition{adj.}{qualificado; dentro dos padrões; em conformidade com os requisitos ou normas}
  \end{phonetics}
\end{entry}

\begin{entry}{合资}{6,10}{⼝、⾙}
  \begin{phonetics}{合资}{he2zi1}
    \definition{s.}{\emph{joint-venture} com capitais mistos}
  \end{phonetics}
\end{entry}

\begin{entry}{合理}{6,11}{⼝、⽟}
  \begin{phonetics}{合理}{he2li3}[][HSK 3]
    \definition{adj.}{racional; razoável; equitativo; razoável ou lógico}
  \end{phonetics}
\end{entry}

\begin{entry}{吉}{6}{⼝}
  \begin{phonetics}{吉}{ji2}
    \definition*{s.}{Província de Jilin, abreviação de 吉林 | Sobrenome Ji}
    \definition{adj.}{sortudo; propício; auspicioso (oposto de 凶)}
  \seealsoref{吉林}{ji2lin2}
  \seealsoref{凶}{xiong1}
  \end{phonetics}
\end{entry}

\begin{entry}{吉他}{6,5}{⼝、⼈}
  \begin{phonetics}{吉他}{ji2ta1}
    \definition[把]{s.}{Empréstimo linguístico: guitarra}
  \end{phonetics}
\end{entry}

\begin{entry}{吉林}{6,8}{⼝、⽊}
  \begin{phonetics}{吉林}{ji2lin2}
    \definition*{s.}{Província de Jilin}
  \end{phonetics}
\end{entry}

\begin{entry}{吊}{6}{⼝}
  \begin{phonetics}{吊}{diao4}[][HSK 6]
    \definition{clas.}{uma sequência de 1.000 em dinheiro; antigamente, uma unidade monetária geralmente era composta por mil pequenas moedas de cobre}
    \definition{s.}{guindaste}
    \definition{v.}{pendurar; suspender | levantar ou abaixar com uma corda, etc. | colocar um forro de pele; adicionar revestimentos ou forros aos barris de couro para fazer roupas | revogar; retirar; recuperar documentos emitidos | lamentar; prestar homenagem os mortos ou oferecer condolências às famílias ou grupos que sofreram uma perda}
  \end{phonetics}
\end{entry}

\begin{entry}{同}{6}{⼝}
  \begin{phonetics}{同}{tong2}[][HSK 6]
    \definition{adj.}{como; igual; parecido; similar; o mesmo; sem diferença}
    \definition{adv.}{juntos; em comum; indica que diferentes atores realizam uma determinada ação juntos ou estão na mesma situação, o que equivale a 一同 ou 一起}
    \definition{v.}{ser o mesmo que}
  \seealsoref{一起}{yi4qi3}
  \seealsoref{一同}{yi4tong2}
  \end{phonetics}
  \begin{phonetics}{同}{tong4}
    \definition[条,处]{s.}{beco; rua estreita}
  \seealsoref{胡同}{hu2tong5}
  \end{phonetics}
\end{entry}

\begin{entry}{同伙}{6,6}{⼝、⼈}
  \begin{phonetics}{同伙}{tong2huo3}
    \definition[个]{s.}{cúmplice | colega}
  \end{phonetics}
\end{entry}

\begin{entry}{同时}{6,7}{⼝、⽇}
  \begin{phonetics}{同时}{tong2shi2}[][HSK 2]
    \definition{conj.}{além disso; além do mais; ainda mais; indica uma relação de equivalência, geralmente com um significado mais profundo}
    \definition{s.}{enquanto isso; ao mesmo tempo}
  \end{phonetics}
\end{entry}

\begin{entry}{同事}{6,8}{⼝、⼅}
  \begin{phonetics}{同事}{tong2shi4}[][HSK 2]
    \definition[个,位,名]{s.}{companheiro; colega; colega de trabalho; pessoas que trabalham juntas}
    \definition{v.}{trabalhar no mesmo lugar; trabalhar juntos; trabalhar na mesma unidade}
  \end{phonetics}
\end{entry}

\begin{entry}{同学}{6,8}{⼝、⼦}
  \begin{phonetics}{同学}{tong2xue2}[][HSK 1]
    \definition[位,个,些]{s.}{colega de escola; colega de turma; colega de estudos; pessoas que estudam na mesma escola}
  \end{phonetics}
\end{entry}

\begin{entry}{同性恋}{6,8,10}{⼝、⼼、⼼}
  \begin{phonetics}{同性恋}{tong2xing4lian4}
    \definition{s.}{homossexualidade | pessoa gay | amor gay}
  \end{phonetics}
\end{entry}

\begin{entry}{同屋}{6,9}{⼝、⼫}
  \begin{phonetics}{同屋}{tong2wu1}
    \definition[个]{s.}{companheiro de quarto | colega de quarto}
  \end{phonetics}
\end{entry}

\begin{entry}{同砚}{6,9}{⼝、⽯}
  \begin{phonetics}{同砚}{tong2yan4}
    \definition[位,个]{s.}{colega de classe | colega estudante}
  \end{phonetics}
\end{entry}

\begin{entry}{同样}{6,10}{⼝、⽊}
  \begin{phonetics}{同样}{tong2 yang4}[][HSK 2]
    \definition{adj.}{igual; semelhante; similar; idêntico; sem diferença}
  \end{phonetics}
\end{entry}

\begin{entry}{同流合污}{6,10,6,6}{⼝、⽔、⼝、⽔}
  \begin{phonetics}{同流合污}{tong2liu2he2wu1}
    \definition{expr.}{chafurdar na lama com alguém | seguir o mau exemplo dos outros}
  \end{phonetics}
\end{entry}

\begin{entry}{同情}{6,11}{⼝、⼼}
  \begin{phonetics}{同情}{tong2qing2}[][HSK 4]
    \definition{s.}{simpatia}
    \definition{v.}{simpatizar com; solidarizar-se; compadecer-se; ter empatia emocional pelo que os outros estão passando}
  \end{phonetics}
\end{entry}

\begin{entry}{同意}{6,13}{⼝、⼼}
  \begin{phonetics}{同意}{tong2yi4}[][HSK 3]
    \definition{v.}{concordar; consentir; aprovar; concordar com; dizer sim}
  \end{phonetics}
\end{entry}

\begin{entry}{名}{6}{⼝}
  \begin{phonetics}{名}{ming2}[][HSK 2]
    \definition*{s.}{Sobrenome Ming}
    \definition{adj.}{notável; famoso; conhecido; renomado}
    \definition{clas.}{usado para pessoas | usado para classificação por ordem}
    \definition{s.}{nome; denominação | desculpa; pretexto | fama; reputação}
    \definition{v.}{nome próprio (é) | expressar; descrever | possuir; tomar; ter}
  \end{phonetics}
\end{entry}

\begin{entry}{名人}{6,2}{⼝、⼈}
  \begin{phonetics}{名人}{ming2 ren2}[][HSK 4]
    \definition{s.}{celebridade; pessoa famosa}
  \end{phonetics}
\end{entry}

\begin{entry}{名片}{6,4}{⼝、⽚}
  \begin{phonetics}{名片}{ming2pian4}[][HSK 4]
    \definition[张,盒,叠]{s.}{cartão de visita; um pedaço de papel retangular com o nome, o cargo, o endereço etc. impressos}
  \end{phonetics}
\end{entry}

\begin{entry}{名字}{6,6}{⼝、⼦}
  \begin{phonetics}{名字}{ming2zi5}[][HSK 1]
    \definition[个]{s.}{nome; nome próprio | nome (de uma coisa)}
  \end{phonetics}
\end{entry}

\begin{entry}{名单}{6,8}{⼝、⼗}
  \begin{phonetics}{名单}{ming2 dan1}[][HSK 2]
    \definition[个,份]{s.}{lista com nomes de pessoas ou nomes de organizações}
  \end{phonetics}
\end{entry}

\begin{entry}{名称}{6,10}{⼝、⽲}
  \begin{phonetics}{名称}{ming2 cheng1}[][HSK 2]
    \definition[个,种]{s.}{nomes, apelidos e formas de se referir a pessoas ou coisas}
  \end{phonetics}
\end{entry}

\begin{entry}{名牌儿}{6,12,2}{⼝、⽚、⼉}
  \begin{phonetics}{名牌儿}{ming2 pai2r5}[][HSK 4]
    \definition{s.}{marca famosa}
  \end{phonetics}
\end{entry}

\begin{entry}{后}{6}{⼝}
  \begin{phonetics}{后}{hou4}[][HSK 1]
    \definition*{s.}{Sobrenome Hou}
    \definition{s.}{atrás; traseiro; a direção oposta àquela para a qual a pessoa está voltada; a direção oposta àquela para a qual a parte de trás de uma casa está voltada (o oposto de 前)  | depois; mais tarde no tempo; futuro (em oposição a 先 ou 前) | último | posteridade; descendência | rainha; imperatriz | governante; soberano; monarca antigo}
  \seealsoref{前}{qian2}
  \seealsoref{先}{xian1}
  \end{phonetics}
\end{entry}

\begin{entry}{后天}{6,4}{⼝、⼤}
  \begin{phonetics}{后天}{hou4 tian1}[][HSK 1]
    \definition{s.}{depois de amanhã; período em que uma pessoa ou animal vive e cresce sozinho após deixar o útero materno (em oposição a 先天)}
  \seealsoref{先天}{xian1tian1}
  \end{phonetics}
\end{entry}

\begin{entry}{后头}{6,5}{⼝、⼤}
  \begin{phonetics}{后头}{hou4 tou5}[][HSK 4]
    \definition{adv.}{posteriormente | atrás | mais tarde}
    \definition{s.}{a parte de trás | a parte traseira}
  \end{phonetics}
\end{entry}

\begin{entry}{后边}{6,5}{⼝、⾡}
  \begin{phonetics}{后边}{hou4 bian5}[][HSK 1]
    \definition{adv.}{costas; traseira; atrás}
  \end{phonetics}
\end{entry}

\begin{entry}{后年}{6,6}{⼝、⼲}
  \begin{phonetics}{后年}{hou4nian2}[][HSK 3]
    \definition{s.}{daqui a dois anos; no ano seguinte ao próximo ano}
  \end{phonetics}
\end{entry}

\begin{entry}{后来}{6,7}{⼝、⽊}
  \begin{phonetics}{后来}{hou4lai2}[][HSK 2]
    \definition{adv.}{mais tarde; depois; refere-se a um período posterior a um determinado momento no passado}
  \end{phonetics}
\end{entry}

\begin{entry}{后果}{6,8}{⼝、⽊}
  \begin{phonetics}{后果}{hou4guo3}[][HSK 3]
    \definition{s.}{consequência; resultado (geralmente negativo)}
  \end{phonetics}
\end{entry}

\begin{entry}{后面}{6,9}{⼝、⾯}
  \begin{phonetics}{后面}{hou4mian4}
    \definition{adv.}{parte de trás; retaguarda; atrás; a parte posterior do espaço ou localização | mais tarde; depois; no futuro; a parte posterior de um artigo ou discurso em relação ao que está sendo narrado no momento}
  \end{phonetics}
\end{entry}

\begin{entry}{后悔}{6,10}{⼝、⼼}
  \begin{phonetics}{后悔}{hou4hui3}[][HSK 5]
    \definition{v.}{lamentar; ter remorso; arrepender-se}
  \end{phonetics}
\end{entry}

\begin{entry}{吐}{6}{⼝}
  \begin{phonetics}{吐}{tu3}[][HSK 5]
    \definition{v.}{cuspir; sair pela boca | surgir ou aparecer pela boca ou por uma fenda | dizer; contar; falar abertamente}
  \end{phonetics}
  \begin{phonetics}{吐}{tu4}[][HSK 5]
    \definition{v.}{vomitar; sair pela boca | vomitar; expelir; metáfora para ser forçado a devolver bens usurpados}
  \end{phonetics}
\end{entry}

\begin{entry}{向}{6}{⼝}
  \begin{phonetics}{向}{xiang4}[][HSK 2]
    \definition*{s.}{Sobrenome Xiang}
    \definition{adv.}{sempre; o tempo todo}
    \definition{prep.}{em direção a; para}
    \definition{s.}{direção | a janela voltada para o norte}
    \definition{v.}{encarar; virar-se para | estar do lado de; ser parcial com; tomar o partido de alguém}
  \end{phonetics}
\end{entry}

\begin{entry}{向上}{6,3}{⼝、⼀}
  \begin{phonetics}{向上}{xiang4 shang4}[][HSK 5]
    \definition{v.}{mover-se; subir; ir para um lugar mais alto; ir para um lugar mais alto em relação a um determinado ponto; ir para um desenvolvimento mais alto que o atual | avançar; continuar se aperfeiçoar; subir na vida; desenvolver-se em direção ao progresso}
  \end{phonetics}
\end{entry}

\begin{entry}{向导}{6,6}{⼝、⼨}
  \begin{phonetics}{向导}{xiang4dao3}[][HSK 5]
    \definition{s.}{guia}
    \definition{v.}{agir como um guia; mostrar a alguém o caminho; levar alguém a algum lugar}
  \end{phonetics}
\end{entry}

\begin{entry}{向汪}{6,7}{⼝、⽔}
  \begin{phonetics}{向汪}{xiang4wang1}
    \definition{v.}{esperar que}
  \end{phonetics}
\end{entry}

\begin{entry}{向往}{6,8}{⼝、⼻}
  \begin{phonetics}{向往}{xiang4wang3}
    \definition{v.}{ansiar por | esperar ansiosamente por}
  \end{phonetics}
\end{entry}

\begin{entry}{向前}{6,9}{⼝、⼑}
  \begin{phonetics}{向前}{xiang4 qian2}[][HSK 5]
    \definition{adv.}{para frente; adiante;}
    \definition{v.}{avançar; ir em direção à frente; mover-se para frente; avançar um pouco mais}
  \end{phonetics}
\end{entry}

\begin{entry}{吓}{6}{⼝}
  \begin{phonetics}{吓}{xia4}[][HSK 5]
    \definition{interj.}{interjeição que demonstra espanto; Interjeição que expressa insatisfação}
    \definition{v.}{ameaçar; intimidar; usar ameaças ou meios coercitivos para intimidar ou assustar}
  \end{phonetics}
\end{entry}

\begin{entry}{吓人}{6,2}{⼝、⼈}
  \begin{phonetics}{吓人}{xia4ren2}
    \definition{adj.}{apavorante | assustador}
    \definition{v.+compl.}{assustar-se | tomar um susto}
  \end{phonetics}
\end{entry}

\begin{entry}{吗}{6}{⼝}
  \begin{phonetics}{吗}{ma2}
    \definition{adv.}{(coloquial) que?}
  \end{phonetics}
  \begin{phonetics}{吗}{ma3}
    \definition{s.}{usada em 吗啡, morfina}
  \seealsoref{吗啡}{ma3fei1}
  \end{phonetics}
  \begin{phonetics}{吗}{ma5}[][HSK 1]
    \definition{part.}{usado no final de uma pergunta | como uma pausa em uma frase antes de introduzir o ponto principal | usado no final de uma pergunta retórica}
  \end{phonetics}
\end{entry}

\begin{entry}{吗啡}{6,11}{⼝、⼝}
  \begin{phonetics}{吗啡}{ma3fei1}
    \definition{s.}{morfina (empréstimo linguístico)}
  \end{phonetics}
\end{entry}

\begin{entry}{吸}{6}{⼝}
  \begin{phonetics}{吸}{xi1}[][HSK 4]
    \definition{v.}{inalar; inspirar; aspirar; itroduzir líquidos, gases, etc. no corpo | absorver; sugar | atrair; atrair para si mesmo; atrair (interesse, investimento etc.)}
  \end{phonetics}
\end{entry}

\begin{entry}{吸引}{6,4}{⼝、⼸}
  \begin{phonetics}{吸引}{xi1yin3}[][HSK 4]
    \definition{v.}{atrair; apelar para; chamar a atenção de outros objetos, forças ou pessoas para si mesmo}
  \end{phonetics}
\end{entry}

\begin{entry}{吸收}{6,6}{⼝、⽁}
  \begin{phonetics}{吸收}{xi1shou1}[][HSK 4]
    \definition{v.}{imbuir; absorver; assimilar; sugar;  chupar; (animais, plantas, etc.) extrair material de fora dos tecidos para o interior dos tecidos | absorver; chupar;  sugar alguma substância de fora para dentro | recrutar; alistar; inscrever-se; matricular-se; admitir; (organizações ou coletivos) aceitar novos membros | absorver; aproveitar e usar a experiência, o conhecimento, o dinheiro e outras coisas valiosas de outras pessoas | absorver; diminuir, atenuar ou eliminar determinados efeitos ou fenômenos}
  \end{phonetics}
\end{entry}

\begin{entry}{吸烟}{6,10}{⼝、⽕}
  \begin{phonetics}{吸烟}{xi1yan1}[][HSK 4]
    \definition{v.+compl.}{fumar}
  \end{phonetics}
\end{entry}

\begin{entry}{吸铁石}{6,10,5}{⼝、⾦、⽯}
  \begin{phonetics}{吸铁石}{xi1tie3shi2}
    \definition{s.}{imã | magneto}
  \seealsoref{磁铁}{ci2tie3}
  \end{phonetics}
\end{entry}

\begin{entry}{吸管}{6,14}{⼝、⽵}
  \begin{phonetics}{吸管}{xi1 guan3}[][HSK 4]
    \definition[根,个]{s.}{tubo de sucção; sugador; canudo (para beber); refere-se ao tubo fino usado para sugar bebidas | conta-gotas; pipeta; cateter para bombeamento de líquidos usando pressão de ar}
  \end{phonetics}
\end{entry}

\begin{entry}{回}{6}{⼞}
  \begin{phonetics}{回}{hui2}[][HSK 1,2]
    \definition*{s.}{Sobrenome Hui}
    \definition*{s.}{Etnia Hui (mulçumanos chineses)}
    \definition{clas.}{usado para coisas, ações, número de vezes |  um trecho de um conto; um capítulo de um romance em capítulos | seção ou capítulo (de um livro clássico)}
    \definition{v.}{circular; enrolar | retornar; voltar; voltar ao lugar de origem | dar meia-volta | responder; contestar | relatar; reportar; responder}
  \end{phonetics}
\end{entry}

\begin{entry}{回忆}{6,4}{⼞、⼼}
  \begin{phonetics}{回忆}{hui2yi4}[][HSK 5]
    \definition[个,段]{s.}{memória; lembrança de eventos ou experiências passadas}
    \definition{v.}{lembrar; recordar}
  \end{phonetics}
\end{entry}

\begin{entry}{回去}{6,5}{⼞、⼛}
  \begin{phonetics}{回去}{hui2 qu4}[][HSK 1]
    \definition{v.}{retornar; voltar; estar de volta ; (a partir da minha localização)}
  \end{phonetics}
\end{entry}

\begin{entry}{回头}{6,5}{⼞、⼤}
  \begin{phonetics}{回头}{hui2 tou2}[][HSK 5]
    \definition{adv.}{mais tarde; depois de um tempo}
    \definition{conj.}{ou então; usado no início da segunda metade de uma frase para indicar o que acontecerá se você não fizer o que fez na primeira metade da frase}
    \definition{v.}{dar a meia-volta; virar a cabeça; virar a cabeça para trás | retornar; voltar | arrepender-se; corrigir seu caminho; reconhecer e corrigir erros}
  \end{phonetics}
\end{entry}

\begin{entry}{回收}{6,6}{⼞、⽁}
  \begin{phonetics}{回收}{hui2shou1}[][HSK 5]
    \definition{v.}{reciclar; reciclar itens (geralmente resíduos ou produtos antigos) para reutilização | recuperar; recolher; recuperar o que foi emitido ou demitido}
  \end{phonetics}
\end{entry}

\begin{entry}{回报}{6,7}{⼞、⼿}
  \begin{phonetics}{回报}{hui2bao4}[][HSK 5]
    \definition{s.}{recompensa; pagamento; benefícios recebidos como resultado de assistência, esforço ou afeto | retornos; benefícios recebidos por meio de investimentos}
    \definition{v.}{pagar de volta; beneficiar pessoas ou organizações que os ajudaram ou cuidaram deles de alguma forma}
  \end{phonetics}
\end{entry}

\begin{entry}{回来}{6,7}{⼞、⽊}
  \begin{phonetics}{回来}{hui2 lai5}[][HSK 1]
    \definition{v.}{voltar; regressar (para a minha localização) | retornar; usado após um verbo, significa ``vir ao lugar original''}
  \end{phonetics}
\end{entry}

\begin{entry}{回到}{6,8}{⼞、⼑}
  \begin{phonetics}{回到}{hui2 dao4}[][HSK 1]
    \definition{v.}{retornar para; voltar e chegar (ao lugar onde estava originalmente); (após uma mudança nas circunstâncias) retornar ao estado original}
  \end{phonetics}
\end{entry}

\begin{entry}{回国}{6,8}{⼞、⼞}
  \begin{phonetics}{回国}{hui2 guo2}[][HSK 2]
    \definition{v.}{regressar ao seu país (terra natal); referindo-se a voltar do exterior}
  \end{phonetics}
\end{entry}

\begin{entry}{回信}{6,9}{⼞、⼈}
  \begin{phonetics}{回信}{hui2 xin4}[][HSK 5]
    \definition[封]{s.}{uma carta em resposta; uma mensagem verbal em resposta}
    \definition{v.+compl.}{escrever em resposta; escrever de volta; responder uma carta; responder verbalmente uma mensagem}
  \end{phonetics}
\end{entry}

\begin{entry}{回复}{6,9}{⼞、⼢}
  \begin{phonetics}{回复}{hui2 fu4}[][HSK 4]
    \definition{v.}{responder (a uma carta) | retornar ao estado normal; restaurar algo ao seu estado original}
  \end{phonetics}
\end{entry}

\begin{entry}{回家}{6,10}{⼞、⼧}
  \begin{phonetics}{回家}{hui2 jia1}[][HSK 1]
    \definition{v.}{ir (voltar) para casa; estar em casa; voltar para casa}
  \end{phonetics}
\end{entry}

\begin{entry}{回顾}{6,10}{⼞、⾴}
  \begin{phonetics}{回顾}{hui2gu4}[][HSK 5]
    \definition{v.}{olhar para trás | revisar; fazer uma retrospectiva; olhar para trás, pensar no passado}
  \end{phonetics}
\end{entry}

\begin{entry}{回旋}{6,11}{⼞、⽅}
  \begin{phonetics}{回旋}{hui2xuan2}
    \definition{v.}{circular | rodar | dar a volta}
  \end{phonetics}
\end{entry}

\begin{entry}{回答}{6,12}{⼞、⽵}
  \begin{phonetics}{回答}{hui2da2}[][HSK 1]
    \definition[个]{s.}{resposta}
    \definition{v.}{responder; explicar a questão; expressar opinião sobre a solicitação}
  \end{phonetics}
\end{entry}

\begin{entry}{回避}{6,16}{⼞、⾌}
  \begin{phonetics}{回避}{hui2bi4}
    \definition{v.}{fugir (de um problema); em direito, refere-se especificamente à não participação nos procedimentos de um caso de um oficial de justiça, etc., que tenha interesse no caso ou nas partes do caso | esquivar-se; evadir-se; evitar (encontrar alguém)}
  \end{phonetics}
\end{entry}

\begin{entry}{因}{6}{⼞}
  \begin{phonetics}{因}{yin1}[][HSK 6]
    \definition*{s.}{Sobrenome Yin}
    \definition{conj.}{porque; orações de conexão, indicando relações de causa e efeito}
    \definition{prep.}{com base em; à luz de; de acordo com; a introdução da ação comportamental equivale a 按照 ou 根据}
    \definition{s.}{causa; motivo; condições em que algo ocorre ou causa um determinado resultado (em oposição a 果)}
    \definition{v.}{seguir; continuar; fazer como sempre fez | estar em conformidade com; estar de acordo com; depender; contar com}
  \seealsoref{按照}{an4zhao4}
  \seealsoref{根据}{gen1ju4}
  \seealsoref{果}{guo3}
  \end{phonetics}
\end{entry}

\begin{entry}{因为}{6,4}{⼞、⼂}
  \begin{phonetics}{因为}{yin1wei4}[][HSK 2]
    \definition{conj.}{porque; indica o motivo e a frase seguinte indica o resultado}
    \definition{prep.}{por causa de; por conta de; indica razão ou justificativa}
  \end{phonetics}
\end{entry}

\begin{entry}{因为……所以……}{6,4,8,4}{⼞、⼂、⼾、⼈}
  \begin{phonetics}{因为……所以……}{yin1wei4 suo3yi3}[][HSK 2]
    \definition{conj.}{porque\dots portanto\dots}
  \end{phonetics}
\end{entry}

\begin{entry}{因此}{6,6}{⼞、⽌}
  \begin{phonetics}{因此}{yin1ci3}[][HSK 3]
    \definition{conj.}{assim; portanto; consequentemente}
  \end{phonetics}
\end{entry}

\begin{entry}{因此就}{6,6,12}{⼞、⽌、⼪}
  \begin{phonetics}{因此就}{yin1ci3 jiu4}
    \definition{conj.}{portanto}
  \end{phonetics}
\end{entry}

\begin{entry}{因而}{6,6}{⼞、⽽}
  \begin{phonetics}{因而}{yin1'er2}[][HSK 5]
    \definition{conj.}{; como resultado; com o resultado que; conecta frases, indicando relação de causa e efeito}
  \end{phonetics}
\end{entry}

\begin{entry}{团}{6}{⼞}
  \begin{phonetics}{团}{tuan2}[][HSK 3]
    \definition*{s.}{Liga da Juventude Comunista da China; Liga}
    \definition{adj.}{redondo; circular}
    \definition{clas.}{usado para algo em forma de bola}
    \definition[个]{s.}{bolinho de massa; comida em forma de bola feita de arroz ou farinha | algo em forma de bola | grupo; corpo; sociedade; organização; um grupo envolvido em um determinado trabalho ou atividade | regimento; unidade organizacional militar, geralmente abaixo do nível de divisão e acima do nível de batalhão}
    \definition{v.}{enrolar algo para formar uma bola; rolar | reunir; unir; conglomerar}
  \end{phonetics}
\end{entry}

\begin{entry}{团长}{6,4}{⼞、⾧}
  \begin{phonetics}{团长}{tuan2 zhang3}[][HSK 5]
    \definition{s.}{comandante do regimento | chefe (ou presidente) de uma delegação, trupe, etc. | líder de uma delegação}
  \end{phonetics}
\end{entry}

\begin{entry}{团队}{6,4}{⼞、⾩}
  \begin{phonetics}{团队}{tuan2dui4}
    \definition{s.}{equipe}
  \end{phonetics}
\end{entry}

\begin{entry}{团体}{6,7}{⼞、⼈}
  \begin{phonetics}{团体}{tuan2ti3}[][HSK 3]
    \definition[种,个]{s.}{equipe; grupo; organização; um grupo de pessoas com objetivos e interesses comuns}
  \end{phonetics}
\end{entry}

\begin{entry}{团结}{6,9}{⼞、⽷}
  \begin{phonetics}{团结}{tuan2jie2}[][HSK 3]
    \definition{adj.}{unido; amigável; harmonioso; relação harmoniosa e coexistência harmoniosa}
    \definition{v.}{unir; reunir}
  \end{phonetics}
\end{entry}

\begin{entry}{在}{6}{⼟}
  \begin{phonetics}{在}{zai4}[][HSK 1]
    \definition{adv.}{em processo de; em curso de}
    \definition{prep.}{em; no (um lugar ou momento); indica tempo, local, âmbito, etc.}
    \definition{v.}{existir; estar vivo | estar em; estar no; estar em (um lugar); indica a localização de pessoas ou coisas | permanecer; ficar | depender de; residir em; repousar com | ingressar ou pertencer a uma organização; ser membro de uma organização}
  \end{phonetics}
\end{entry}

\begin{entry}{在下}{6,3}{⼟、⼀}
  \begin{phonetics}{在下}{zai4xia4}
    \definition{pron.}{eu mesmo (humildemente)}
  \end{phonetics}
\end{entry}

\begin{entry}{在于}{6,3}{⼟、⼆}
  \begin{phonetics}{在于}{zai4yu2}[][HSK 4]
    \definition{v.}{ser responsável por; caber a;  ser da competência de;  apontar a essência das coisas, ou do que elas se tratam | depender de; ser determinado por;  ser devido a (um determinado atributo)/(de um assunto a ser determinado)}
  \end{phonetics}
\end{entry}

\begin{entry}{在内}{6,4}{⼟、⼌}
  \begin{phonetics}{在内}{zai4 nei4}[][HSK 5]
    \definition{adj.}{incluido}
    \definition{adv.}{dentro; internamente; entre eles}
    \definition{v.}{ser incluído}
  \end{phonetics}
\end{entry}

\begin{entry}{在乎}{6,5}{⼟、⼃}
  \begin{phonetics}{在乎}{zai4hu5}[][HSK 4]
    \definition{v.}{preocupar-se; preocupar-se com; levar a sério | ser responsável por; caber ao; ser da competência de}
  \end{phonetics}
\end{entry}

\begin{entry}{在地}{6,6}{⼟、⼟}
  \begin{phonetics}{在地}{zai4di4}
    \definition{s.}{local}
  \end{phonetics}
\end{entry}

\begin{entry}{在场}{6,6}{⼟、⼟}
  \begin{phonetics}{在场}{zai4 chang3}[][HSK 5]
    \definition{v.}{estar presente; estar no local; estar em cena; estar presente onde as coisas estão acontecendo}
  \end{phonetics}
\end{entry}

\begin{entry}{在此}{6,6}{⼟、⽌}
  \begin{phonetics}{在此}{zai4ci3}
    \definition{adv.}{aqui}
  \end{phonetics}
\end{entry}

\begin{entry}{在行}{6,6}{⼟、⾏}
  \begin{phonetics}{在行}{zai4hang2}
    \definition{v.}{ser adepto de algo | ser um especialista em um comércio ou profissão}
  \end{phonetics}
\end{entry}

\begin{entry}{在线}{6,8}{⼟、⽷}
  \begin{phonetics}{在线}{zai4xian4}
    \definition{s.}{\emph{online}}
  \end{phonetics}
\end{entry}

\begin{entry}{在家}{6,10}{⼟、⼧}
  \begin{phonetics}{在家}{zai4 jia1}[][HSK 1]
    \definition{v.}{estar em; estar em casa; estar no local de trabalho ou alojamento; sem sair de casa | continuar sendo um leigo; permanecer leigo; para monges, freiras, taoístas e outros que 出家, as pessoas comuns são consideradas leigas}
  \seealsoref{出家}{chu1 jia1}
  \end{phonetics}
\end{entry}

\begin{entry}{在教}{6,11}{⼟、⽁}
  \begin{phonetics}{在教}{zai4jiao4}
    \definition{v.}{ser um crente (em uma religião)}
  \end{phonetics}
\end{entry}

\begin{entry}{在意}{6,13}{⼟、⼼}
  \begin{phonetics}{在意}{zai4yi4}
    \definition{v.+compl.}{preocupar-se | importar-se | levar a sério}
  \end{phonetics}
\end{entry}

\begin{entry}{地}{6}{⼟}
  \begin{phonetics}{地}{de5}[][HSK 1]
    \definition{part.}{(estrutural) utilizada antes de um verbo ou adjetivo, ligando-o ao adjunto adverbial modificador precedente}
  \end{phonetics}
  \begin{phonetics}{地}{di4}[][HSK 1]
    \definition*{s.}{A Terra | Sobrenome Di}
    \definition[块,片]{s.}{terra; solo | campos | chão; piso | posição; situação | contexto; base | distância percorrida (medida em 里 ou paradas 站) | indicando estado de espírito | território | lugar; local | parte do espaço | distância}
  \end{phonetics}
\end{entry}

\begin{entry}{地上}{6,3}{⼟、⼀}
  \begin{phonetics}{地上}{di4 shang5}[][HSK 1]
    \definition{adv.}{no chão; no solo; em terra}
  \end{phonetics}
\end{entry}

\begin{entry}{地下}{6,3}{⼟、⼀}
  \begin{phonetics}{地下}{di4 xia4}[][HSK 4]
    \definition{s.}{subterrâneo | secreta (atividade) | recursos ocultos}
  \end{phonetics}
\end{entry}

\begin{entry}{地下室}{6,3,9}{⼟、⼀、⼧}
  \begin{phonetics}{地下室}{di4xia4shi4}
    \definition{s.}{subterrâneo | porão}
  \end{phonetics}
\end{entry}

\begin{entry}{地区}{6,4}{⼟、⼖}
  \begin{phonetics}{地区}{di4qu1}[][HSK 3]
    \definition[个,片]{s.}{área; distrito; região; um lugar maior | prefeitura; unidade administrativa | latitudes; localidade; lado | em determinadas circunstâncias, algumas regiões administrativas locais da China, como Hong Kong e Macau, participam individualmente em algumas atividades internacionais}
    \definition{suf.}{como sufixo do nome da cidade, significa prefeitura ou condado}
  \end{phonetics}
\end{entry}

\begin{entry}{地方}{6,4}{⼟、⽅}
  \begin{phonetics}{地方}{di4fang1}
    \definition[个]{s.}{distrito; localidade;  em oposição a 中央, o número total de unidades administrativas em todos os níveis abaixo do centro | governo local e população; refere-se a outros setores que não o militar}
  \seealsoref{中央}{zhong1yang1}
  \end{phonetics}
  \begin{phonetics}{地方}{di4fang5}[][HSK 1,4]
    \definition[个,处,块]{s.}{lugar; cômodo; área; refere-se a um espaço específico | parte}
  \end{phonetics}
\end{entry}

\begin{entry}{地位}{6,7}{⼟、⼈}
  \begin{phonetics}{地位}{di4wei4}[][HSK 4]
    \definition{s.}{lugar; status; posição; posição da pessoa ou do grupo nas relações sociais | lugar; posição (ocupada por uma pessoa ou coisa); espaço ocupado por uma pessoa ou coisa}
  \end{phonetics}
\end{entry}

\begin{entry}{地址}{6,7}{⼟、⼟}
  \begin{phonetics}{地址}{di4zhi3}[][HSK 4]
    \definition[个]{s.}{endereço; local de residência ou correspondência}
  \end{phonetics}
\end{entry}

\begin{entry}{地形}{6,7}{⼟、⼺}
  \begin{phonetics}{地形}{di4 xing2}[][HSK 5]
    \definition{s.}{topografia; forma do terreno; relevo; disposição do terreno; característica do relevo; característica da superfície; terreno}
  \end{phonetics}
\end{entry}

\begin{entry}{地图}{6,8}{⼟、⼞}
  \begin{phonetics}{地图}{di4tu2}[][HSK 1]
    \definition[张,本]{s.}{mapa; mapa que mostra a distribuição de coisas e fenômenos na superfície da Terra, com símbolos e textos, e às vezes também com cores}
  \end{phonetics}
\end{entry}

\begin{entry}{地带}{6,9}{⼟、⼱}
  \begin{phonetics}{地带}{di4 dai4}[][HSK 5]
    \definition[个]{s.}{distrito; região; zona; área de uma determinada natureza ou extensão}
  \end{phonetics}
\end{entry}

\begin{entry}{地点}{6,9}{⼟、⽕}
  \begin{phonetics}{地点}{di4dian3}[][HSK 1]
    \definition[个]{s.}{lugar; local; região; localização}
  \end{phonetics}
\end{entry}

\begin{entry}{地狱}{6,9}{⼟、⽝}
  \begin{phonetics}{地狱}{di4yu4}
    \definition*{s.}{Budismo: Naraka}
    \definition{adj.}{infernal}
    \definition{s.}{inferno | submundo}
  \end{phonetics}
\end{entry}

\begin{entry}{地砖}{6,9}{⼟、⽯}
  \begin{phonetics}{地砖}{di4zhuan1}
    \definition{s.}{ladrilho de piso}
  \end{phonetics}
\end{entry}

\begin{entry}{地面}{6,9}{⼟、⾯}
  \begin{phonetics}{地面}{di4 mian4}[][HSK 4]
    \definition{s.}{a superfície da Terra | térreo; piso; camada de material colocada no chão dentro e ao redor dos edifícios | localidade; chão | região; território; principalmente áreas administrativas}
  \end{phonetics}
\end{entry}

\begin{entry}{地核}{6,10}{⼟、⽊}
  \begin{phonetics}{地核}{di4he2}
    \definition{s.}{(geologia) núcleo da Terra}
  \end{phonetics}
\end{entry}

\begin{entry}{地铁}{6,10}{⼟、⾦}
  \begin{phonetics}{地铁}{di4tie3}[][HSK 2]
    \definition[条,班,列,趟]{s.}{metrô; trem subterrâneo; também se refere ao vagão do metrô}
  \end{phonetics}
\end{entry}

\begin{entry}{地铁站}{6,10,10}{⼟、⾦、⽴}
  \begin{phonetics}{地铁站}{di4 tie3 zhan4}[][HSK 2]
    \definition[个,座]{s.}{estação de metrô}
  \end{phonetics}
\end{entry}

\begin{entry}{地球}{6,11}{⼟、⽟}
  \begin{phonetics}{地球}{di4qiu2}[][HSK 2]
    \definition[个]{s.}{o planeta Terra}
  \end{phonetics}
\end{entry}

\begin{entry}{地理}{6,11}{⼟、⽟}
  \begin{phonetics}{地理}{di4li3}
    \definition{s.}{geografia}
  \end{phonetics}
\end{entry}

\begin{entry}{地震}{6,15}{⼟、⾬}
  \begin{phonetics}{地震}{di4zhen4}[][HSK 5]
    \definition[场,次,级]{s.}{sismo; terremoto; tremor de terra; vibrações na crosta terrestre}
    \definition{v.}{sacudir com vibrações sísmicas}
  \end{phonetics}
\end{entry}

\begin{entry}{场}{6}{⼟}
  \begin{phonetics}{场}{chang2}
    \definition{clas.}{usado para descrever o desenrolar dos acontecimentos}
    \definition{s.}{eira; espaço aberto e plano; um terreno plano, geralmente usado para secar grãos e moer cereais | mercado; feira rural}
  \end{phonetics}
  \begin{phonetics}{场}{chang3}[][HSK 2]
    \definition*{s.}{Sobrenome Chang}
    \definition{clas.}{usado para atividades culturais, recreativas e esportivas | usado para pequenos trechos de uma peça}
    \definition{s.}{um local amplo utilizado para um fim específico | palco; campo | cena | (física) campo (por exemplo: campo manético) | (para atividades recreativas, esportivas ou outras) | um lugar onde as pessoas se reúnem | fazenda; quinta | abertura; encerramento; refere-se ao processo completo de uma apresentação ou competição | local; ponto; o local onde ocorreu o incidente}
  \end{phonetics}
\end{entry}

\begin{entry}{场合}{6,6}{⼟、⼝}
  \begin{phonetics}{场合}{chang3he2}[][HSK 3]
    \definition[个,些,种,类]{s.}{ocasião; situação; um certo tempo, lugar ou situação}
  \end{phonetics}
\end{entry}

\begin{entry}{场地}{6,6}{⼟、⼟}
  \begin{phonetics}{场地}{chang3 di4}[][HSK 6]
    \definition[片,块,个]{s.}{área; pátio; espaço; lugar; quadra; campo; um lugar onde construções ou atividades são realizadas}
  \end{phonetics}
\end{entry}

\begin{entry}{场所}{6,8}{⼟、⼾}
  \begin{phonetics}{场所}{chang3suo3}[][HSK 3]
    \definition{s.}{lugar; sítio; arena; local da atividade}
  \end{phonetics}
\end{entry}

\begin{entry}{场面}{6,9}{⼟、⾯}
  \begin{phonetics}{场面}{chang3mian4}[][HSK 5]
    \definition[个,种,番]{s.}{espetáculo; cena (em teatro, ficção, etc.); uma cena em uma produção teatral, cinematográfica ou televisiva que consiste em um cenário, música e personagens | cena; ocasião; literatura narrativa que consiste em situações da vida em que os personagens se relacionam entre si em determinadas ocasiões | orquestra ou instrumentos de acompanhamento (em ópera); refere-se às pessoas e aos instrumentos musicais que acompanham a apresentação de uma ópera, divididos em dois tipos: música de sopro e cordas é uma cena cultural, e gongos e tambores são uma cena marcial | situação; referência geral a uma situação em um determinado contexto | frente; fachada; aparência; espetáculo superficial}
  \end{phonetics}
\end{entry}

\begin{entry}{场馆}{6,11}{⼟、⾷}
  \begin{phonetics}{场馆}{chang3 guan3}[][HSK 6]
    \definition{s.}{ginásios e estádios | arena | local esportivo}
  \end{phonetics}
\end{entry}

\begin{entry}{场景}{6,12}{⼟、⽇}
  \begin{phonetics}{场景}{chang3 jing3}[][HSK 6]
    \definition[个,幕,种]{s.}{espetáculo; cena (em drama, ficção, etc.); refere-se a cenas de drama, cinema, televisão e obras literárias | cena; visão; circunstâncias; cenas e situações}
  \end{phonetics}
\end{entry}

\begin{entry}{多}{6}{⼣}
  \begin{phonetics}{多}{duo1}[][HSK 1,2]
    \definition*{s.}{Sobrenome Duo}
    \definition{adj.}{grande quantidade (oposto de 少, 寡) | excessivo; desnecessário | excessivo; em demasia; indica um grande grau de diferença | mais do que o número correto ou necessário; em excesso}
    \definition{adv.}{acima de um valor especificado; e mais | em que medida; usado em frases interrogativas para indagar sobre grau ou quantidade, equivalente a 多么 | uma extensão não especificada; usado em frases exclamativas para expressar um alto grau, equivalente a 多么 | quase; significa que a maior parte do intervalo é assim | mais;  sobre; ímpar; usado depois de um quantificador para indicar uma fração}
    \definition{num.}{(após um número) ímpar}
    \definition{pref.}{multi- | poli-}
    \definition{v.}{ter (uma quantidade específica) a mais ou a mais (oposto a 少) | ter algo em abundância  | (em perguntas) até que ponto | (em exclamações) até que ponto | ter mais}
  \seealsoref{多么}{duo1me5}
  \seealsoref{寡}{gua3}
  \seealsoref{少}{shao3}
  \end{phonetics}
\end{entry}

\begin{entry}{多久}{6,3}{⼣、⼃}
  \begin{phonetics}{多久}{duo1 jiu3}[][HSK 2]
    \definition{pron.}{quanto tempo?; quanto tempo; perguntar quanto tempo leva}
  \end{phonetics}
\end{entry}

\begin{entry}{多么}{6,3}{⼣、⼃}
  \begin{phonetics}{多么}{duo1me5}[][HSK 2]
    \definition{adv.}{(em exclamações) como; o quê; em que medida; usado em frases exclamativas, indica um grau muito alto | em grau indeterminado; usado em frases declarativas, indica um grau mais profundo | como (usado em uma frase interrogativa para perguntar sobre grau ou número)}
  \end{phonetics}
\end{entry}

\begin{entry}{多大}{6,3}{⼣、⼤}
  \begin{phonetics}{多大}{duo1da4}
    \definition{adj.}{quantos anos? | que idade? | quão grande?}
  \end{phonetics}
\end{entry}

\begin{entry}{多云}{6,4}{⼣、⼆}
  \begin{phonetics}{多云}{duo1 yun2}[][HSK 2]
    \definition{adj.}{céu nublado; em meteorologia, refere-se a condições atmosféricas em que a cobertura de nuvens médias e baixas ocupa entre 40\% e 70\% da área do céu, ou a cobertura de nuvens altas ocupa entre 60\% e 100\% da área do céu}
  \end{phonetics}
\end{entry}

\begin{entry}{多少}{6,4}{⼣、⼩}
  \begin{phonetics}{多少}{duo1shao3}
    \definition{adv.}{um pouco; mais ou menos; até certo ponto}
    \definition{s.}{número; quantidade; volume}
  \end{phonetics}
  \begin{phonetics}{多少}{duo1shao5}[][HSK 1]
    \definition{adv.}{quantos?; quanto?; usado em perguntas para perguntar sobre quantidade | expressar uma quantidade ou número não especificado; quantidade indefinida}
  \end{phonetics}
\end{entry}

\begin{entry}{多年}{6,6}{⼣、⼲}
  \begin{phonetics}{多年}{duo1 nian2}[][HSK 4]
    \definition{adv.}{por muitos anos; durante muitos anos}
  \end{phonetics}
\end{entry}

\begin{entry}{多次}{6,6}{⼣、⽋}
  \begin{phonetics}{多次}{duo1 ci4}[][HSK 4]
    \definition{adv.}{muitas vezes; de vez em quando; repetidamente; em muitas ocasiões}
  \end{phonetics}
\end{entry}

\begin{entry}{多咱}{6,9}{⼣、⼝}
  \begin{phonetics}{多咱}{duo1 zan5}
    \definition{adv.}{que horas?; quando?}
  \end{phonetics}
\end{entry}

\begin{entry}{多种}{6,9}{⼣、⽲}
  \begin{phonetics}{多种}{duo1 zhong3}[][HSK 4]
    \definition{adj.}{diverso; vários tipos de; múltiplo; diversificado}
  \end{phonetics}
\end{entry}

\begin{entry}{多重}{6,9}{⼣、⾥}
  \begin{phonetics}{多重}{duo1chong2}
    \definition{pref.}{multi (facetado, cultural, étnico, etc.)}
  \end{phonetics}
\end{entry}

\begin{entry}{多样}{6,10}{⼣、⽊}
  \begin{phonetics}{多样}{duo1 yang4}[][HSK 4]
    \definition{adj.}{diversos; variados; diversificado}
    \definition{s.}{diversidade}
  \end{phonetics}
\end{entry}

\begin{entry}{多数}{6,13}{⼣、⽁}
  \begin{phonetics}{多数}{duo1 shu4}[][HSK 2]
    \definition{adj.}{maioria; a maioria; plural}
    \definition{pref.}{pluri-}
  \end{phonetics}
\end{entry}

\begin{entry}{夹}{6}{⼤}
  \begin{phonetics}{夹}{ga1}
    \definition{s.}{axila; sovaco; atualmente, costuma-se escrever 胳肢窝}
  \seealsoref{胳肢窝}{ga1 zhi1 wo1}
  \end{phonetics}
  \begin{phonetics}{夹}{jia1}[][HSK 5]
    \definition{s.}{clipe, grampo, pasta, etc.}
    \definition{v.}{colocar no meio; pressionar de ambos os lados; aplicar força ou ação ao mesmo objeto de ambos os lados ao mesmo tempo | misturar; mesclar; intercalar}
  \end{phonetics}
  \begin{phonetics}{夹}{jia2}
    \definition{adj.}{forrado; com camada dupla; duas camadas (roupas, colchas, etc.) | pinçado; voz deliberadamente engraçada}
  \end{phonetics}
\end{entry}

\begin{entry}{夹肢窝}{6,8,12}{⼤、⾁、⽳}
  \begin{phonetics}{夹肢窝}{jia1 zhi1 wo1}
    \definition{s.}{axila; sovaco; também escrito como 胳肢窝}
  \seealsoref{胳肢窝}{ga1 zhi1 wo1}
  \end{phonetics}
\end{entry}

\begin{entry}{夺}{6}{⼤}
  \begin{phonetics}{夺}{duo2}[][HSK 6]
    \definition{v.}{tomar à força; apreender; arrancar; roubar | forçar a passagem; empurrar para abrir | lutar por; competir por; esforçar-se por; obter primeiro | privar; perder | perder; tirar | decidir; tomar uma decisão | omitir (palavra em um texto)}
  \end{phonetics}
\end{entry}

\begin{entry}{夺冠}{6,9}{⼤、⼍}
  \begin{phonetics}{夺冠}{duo2guan4}
    \definition{v.}{apoderar-se da coroa | (fig.) ganhar um campeonato | ganhar a medalha de ouro}
  \end{phonetics}
\end{entry}

\begin{entry}{奸}{6}{⼥}
  \begin{phonetics}{奸}{jian1}
    \definition{adj.}{perverso; maligno; traiçoeiro; malicioso}
    \definition{s.}{traidor; espião | pessoa perversa; pessoa traiçoeira | relações sexuais ilícitas; comportamento sexual impróprio}
    \definition{v.}{ter relações sexuais ilícitas}
  \end{phonetics}
\end{entry}

\begin{entry}{奸夫}{6,4}{⼥、⼤}
  \begin{phonetics}{奸夫}{jian1fu1}
    \definition{s.}{homem adúltero}
  \end{phonetics}
\end{entry}

\begin{entry}{她}{6}{⼥}
  \begin{phonetics}{她}{ta1}[][HSK 1]
    \definition{pron.}{ela | ela; referir-se a coisas que se ama ou aprecia, como a pátria, a bandeira nacional, etc.}
  \end{phonetics}
\end{entry}

\begin{entry}{她们}{6,5}{⼥、⼈}
  \begin{phonetics}{她们}{ta1men5}[][HSK 1]
    \definition{pron.}{elas; referindo-se a várias mulheres: em textos escritos, use 她们 quando todas as pessoas forem mulheres e 他们 quando houver homens e mulheres}
  \seealsoref{他们}{ta1men5}
  \end{phonetics}
\end{entry}

\begin{entry}{她们的}{6,5,8}{⼥、⼈、⽩}
  \begin{phonetics}{她们的}{ta1men5 de5}
    \definition{pron.}{delas}
  \end{phonetics}
\end{entry}

\begin{entry}{她的}{6,8}{⼥、⽩}
  \begin{phonetics}{她的}{ta1 de5}
    \definition{pron.}{dela}
  \end{phonetics}
\end{entry}

\begin{entry}{好}{6}{⼥}
  \begin{phonetics}{好}{hao3}[][HSK 1,2,4]
    \definition{adj.}{bom; ótimo; agradável; vantajoso; satisfatório | amigável; gentil; amistoso; amável | saudável; bem | pronto; concluído; usado após um verbo para indicar conclusão ou perfeição | fácil (de fazer); conveniente; responsável (por)}
    \definition{adv.}{muito; bastante; tão; usado na frente de uma palavra de quantidade ou uma palavra de tempo para indicar muito ou por muito tempo | em que medida; como; usado antes de adjetivos e verbos para indicar profundidade e com exclamação}
    \definition{interj.}{O.K.; tudo bem; aprovação, acordo ou encerramento | (no início de uma frase ou oração) expressa concordância (ou desaprovação, surpresa, etc.)}
    \definition{prep.}{de modo a; para que}
    \definition{s.}{referindo-se a palavras de elogio ou aplauso | saudações; cumprimentos}
    \definition{suf.}{sufixo que indica conclusão ou prontidão | depois de um pronome significa ``olá''}
    \definition{v.}{deve; precisa; tem que; deveria | apaixonar-se}
  \end{phonetics}
  \begin{phonetics}{好}{hao4}
    \definition*{s.}{Sobrenome Hao}
    \definition{adv.}{algo que acontece com frequência, que é fácil de acontecer}
    \definition{v.}{gostar; amar; ter afeição por}
  \end{phonetics}
\end{entry}

\begin{entry}{好人}{6,2}{⼥、⼈}
  \begin{phonetics}{好人}{hao3 ren2}[][HSK 2]
    \definition[个,位,名]{s.}{pessoa boa (ou excelente) (oposto de 坏人) | pessoa saudável | pessoa gentil que tenta se dar bem com todos (muitas vezes em detrimento dos princípios)}
  \seealsoref{坏人}{huai4 ren2}
  \end{phonetics}
\end{entry}

\begin{entry}{好久}{6,3}{⼥、⼃}
  \begin{phonetics}{好久}{hao3jiu3}[][HSK 2]
    \definition{adv.}{por muito tempo | por eras (no passado)}
  \end{phonetics}
\end{entry}

\begin{entry}{好友}{6,4}{⼥、⼜}
  \begin{phonetics}{好友}{hao3you3}[][HSK 4]
    \definition[位,个]{s.}{bom amigo; amigo próximo}
  \end{phonetics}
\end{entry}

\begin{entry}{好心}{6,4}{⼥、⼼}
  \begin{phonetics}{好心}{hao3xin1}
    \definition{s.}{bondade | boas intenções}
  \end{phonetics}
\end{entry}

\begin{entry}{好处}{6,5}{⼥、⼡}
  \begin{phonetics}{好处}{hao3chu4}[][HSK 2]
    \definition[个]{s.}{bom; benefício; vantagem; fatores favoráveis a pessoas ou coisas | ganho; lucro; algo que não se deveria receber, dado por outra pessoa ou obtido através de uma oportunidade; geralmente tem conotação pejorativa}
  \end{phonetics}
\end{entry}

\begin{entry}{好汉}{6,5}{⼥、⽔}
  \begin{phonetics}{好汉}{hao3han4}
    \definition[条]{s.}{herói | pessoa forte e corajosa}
  \end{phonetics}
\end{entry}

\begin{entry}{好生}{6,5}{⼥、⽣}
  \begin{phonetics}{好生}{hao3sheng1}
    \definition{adv.}{bastante; extremamente | cuidadosamente; apropriadamente}
  \end{phonetics}
\end{entry}

\begin{entry}{好用}{6,5}{⼥、⽤}
  \begin{phonetics}{好用}{hao3yong4}
    \definition{adj.}{fácil de usar | adequado ao uso}
  \end{phonetics}
\end{entry}

\begin{entry}{好吃}{6,6}{⼥、⼝}
  \begin{phonetics}{好吃}{hao3chi1}[][HSK 1]
    \definition{adj.}{bom; saboroso; delicioso; descreve o sabor agradável de algo, que as pessoas gostam de comer}
  \end{phonetics}
  \begin{phonetics}{好吃}{hao4chi1}
    \definition{v.}{ser guloso; gostar de comer boa comida}
  \end{phonetics}
\end{entry}

\begin{entry}{好多}{6,6}{⼥、⼣}
  \begin{phonetics}{好多}{hao3 duo1}[][HSK 2]
    \definition{adj.}{muitos; uma boa quantidade; uma grande quantidade; uma quantidade enorme}
    \definition{pron.}{quantos?; quanto?; frequentemente usado para perguntar sobre quantidade}
  \end{phonetics}
\end{entry}

\begin{entry}{好好}{6,6}{⼥、⼥}
  \begin{phonetics}{好好}{hao3 hao3}[][HSK 3]
    \definition{adj.}{realmente bom/bem; em perfeitas condições; quando tudo está bem}
    \definition{adv.}{diretamente; seriamente; cuidadosamente; com todo o empenho; ao máximo}
  \end{phonetics}
\end{entry}

\begin{entry}{好听}{6,7}{⼥、⼝}
  \begin{phonetics}{好听}{hao3 ting1}[][HSK 1]
    \definition{adj.}{agradável de ouvir (de som ou voz) | bom; palatável; satisfatório (de palavras)  | decente; honrado (de ações, etc.); descreve uma coisa que parece prestigiosa | interessante; descreve palavras, histórias e outras coisas interessantes}
  \end{phonetics}
\end{entry}

\begin{entry}{好运}{6,7}{⼥、⾡}
  \begin{phonetics}{好运}{hao3 yun4}[][HSK 5]
    \definition{s.}{boa sorte, fortuna ou oportunidade}
  \end{phonetics}
\end{entry}

\begin{entry}{好事}{6,8}{⼥、⼅}
  \begin{phonetics}{好事}{hao3 shi4}[][HSK 2]
    \definition[个,件]{s.}{boa ação; gentileza | (antigo) obra de caridade | acontecimento feliz; evento festivo}
  \end{phonetics}
  \begin{phonetics}{好事}{hao4 shi4}
    \definition[个,件]{s.}{intrometido; gostar de se meter na vida dos outros}
  \end{phonetics}
\end{entry}

\begin{entry}{好奇}{6,8}{⼥、⼤}
  \begin{phonetics}{好奇}{hao4qi2}[][HSK 3]
    \definition{adj.}{curioso; curiosidade e interesse por coisas não conhecidas}
    \definition{s.}{curiosidade}
    \definition{v.}{ser ou estar curioso}
  \end{phonetics}
\end{entry}

\begin{entry}{好学}{6,8}{⼥、⼦}
  \begin{phonetics}{好学}{hao3xue2}
    \definition{adj.}{fácil de aprender}
  \end{phonetics}
  \begin{phonetics}{好学}{hao4xue2}
    \definition{s.}{estudioso | erudito}
  \end{phonetics}
\end{entry}

\begin{entry}{好玩儿}{6,8,2}{⼥、⽟、⼉}
  \begin{phonetics}{好玩儿}{hao3 wan2r5}[][HSK 1]
    \definition{adj.}{divertido; interessante; capaz de despertar interesse}
  \end{phonetics}
\end{entry}

\begin{entry}{好看}{6,9}{⼥、⽬}
  \begin{phonetics}{好看}{hao3 kan4}[][HSK 1]
    \definition{adj.}{de boa aparência; agradável; bonito | interessante; descreve o enredo ou conteúdo de filmes, romances, performances, etc., como sendo cativante, agradável ou apreciável}
  \end{phonetics}
\end{entry}

\begin{entry}{好象}{6,11}{⼥、⾗}
  \begin{phonetics}{好象}{hao3xiang4}
    \variantof{好像}
  \end{phonetics}
\end{entry}

\begin{entry}{好像}{6,13}{⼥、⼈}
  \begin{phonetics}{好像}{hao3xiang4}[][HSK 2]
    \definition{adv.}{como se; um pouco parecido; como se fosse}
    \definition{v.}{parecer; ser como; parecer-se com}
  \end{phonetics}
\end{entry}

\begin{entry}{如}{6}{⼥}
  \begin{phonetics}{如}{ru2}[][HSK 6]
    \definition{adv.}{por exemplo; tal como; como}
    \definition{conj.}{se; no caso (de); no caso de; como se; como}
    \definition{prep.}{em conformidade com; de acordo com}
    \definition{v.}{estar em conformidade (ou de acordo) com | (geralmente no negativo) pode ser comparado com; ser comparável a; ser tão bom quanto | superar; exceder | (literário) ir para}
  \end{phonetics}
\end{entry}

\begin{entry}{如下}{6,3}{⼥、⼀}
  \begin{phonetics}{如下}{ru2 xia4}[][HSK 5]
    \definition{adv.}{como descrito ou listado abaixo; conforme segue; conforme abaixo}
  \end{phonetics}
\end{entry}

\begin{entry}{如今}{6,4}{⼥、⼈}
  \begin{phonetics}{如今}{ru2jin1}[][HSK 4]
    \definition{s.}{agora; hoje em dia; atualmente; no presente}
  \end{phonetics}
\end{entry}

\begin{entry}{如同}{6,6}{⼥、⼝}
  \begin{phonetics}{如同}{ru2 tong2}[][HSK 5]
    \definition{v.}{parecer que. usado principalmente em metáforas}
  \end{phonetics}
\end{entry}

\begin{entry}{如此}{6,6}{⼥、⽌}
  \begin{phonetics}{如此}{ru2 ci3}[][HSK 5]
    \definition{adv.}{assim; tal; dessa forma; dessa maneira; refere-se a uma situação mencionada anteriormente, equivalente a 这样}
  \seealsoref{这样}{zhe4 yang4}
  \end{phonetics}
\end{entry}

\begin{entry}{如何}{6,7}{⼥、⼈}
  \begin{phonetics}{如何}{ru2he2}[][HSK 3]
    \definition{pron.}{como?; o que?; usado para perguntar como resolver um problema | como?; o que?; usado para perguntar sobre a situação ou obter a opinião de outras pessoas}
  \end{phonetics}
\end{entry}

\begin{entry}{如果}{6,8}{⼥、⽊}
  \begin{phonetics}{如果}{ru2guo3}[][HSK 2]
    \definition{conj.}{se; no caso de; na eventualidade de; supondo que; para expressar suposições, pode-se usar 要是 na linguagem falada.}
  \seealsoref{要是}{yao4shi5}
  \end{phonetics}
\end{entry}

\begin{entry}{如画}{6,8}{⼥、⽥}
  \begin{phonetics}{如画}{ru2hua4}
    \definition{adj.}{pitoresco}
  \end{phonetics}
\end{entry}

\begin{entry}{妆}{6}{⼥}
  \begin{phonetics}{妆}{zhuang1}
    \definition{s.}{maquiagem | adorno | enxoval | maquiagem e figurino de palco}
    \definition{v.}{maquiar-se | enfeitar-se}
  \end{phonetics}
\end{entry}

\begin{entry}{妆扮}{6,7}{⼥、⼿}
  \begin{phonetics}{妆扮}{zhuang1ban4}
    \variantof{装扮}
  \end{phonetics}
\end{entry}

\begin{entry}{妈}{6}{⼥}
  \begin{phonetics}{妈}{ma1}[][HSK 1]
    \definition[个,位]{s.}{mãe; mamãe | uma forma de tratamento para uma mulher casada uma geração mais velha | (antigo) uma forma de tratamento para uma empregada doméstica de meia-idade ou velha}
  \seealsoref{妈妈}{ma1 ma5}
  \end{phonetics}
\end{entry}

\begin{entry}{妈妈}{6,6}{⼥、⼥}
  \begin{phonetics}{妈妈}{ma1 ma5}[][HSK 1]
    \definition[个,位]{s.}{mamãe; mãe | uma forma de chamar uma mulher de meia-idade; títulos de respeito para mulheres mais velhas}
  \end{phonetics}
\end{entry}

\begin{entry}{字}{6}{⼦}
  \begin{phonetics}{字}{zi4}[][HSK 1]
    \definition[个]{s.}{palavra; caractere; texto | pronúncia (de uma palavra ou caractere); som do caractere | tipo de impressão; estilo de caligrafia; forma de um caractere escrito ou impresso; refere-se às diferentes formas dos caracteres chineses; também se refere às diferentes escolas de caligrafia | escritas; obras de caligrafia | recibo; compromisso por escrito; documento | nome de estilo masculino adotado aos vinte anos de idade | sobrenome | um número indicado num contador elétrico, contador de água, etc.; registrar dos números dos medidores de consumo de água e eletricidade}
    \definition{v.}{ficar noiva (nos tempos antigos)}
  \end{phonetics}
\end{entry}

\begin{entry}{字母}{6,5}{⼦、⽏}
  \begin{phonetics}{字母}{zi4mu3}[][HSK 4]
    \definition[个]{s.}{letra; letras de um alfabeto | caractere que representa uma consoante inicial (em fonologia)}
  \end{phonetics}
\end{entry}

\begin{entry}{字字珠玉}{6,6,10,5}{⼦、⼦、⽟、⽟}
  \begin{phonetics}{字字珠玉}{zi4zi4zhu1yu4}
    \definition{expr.}{cada palavra é uma jóia}
    \definition{s.}{escrita magnífica}
  \end{phonetics}
\end{entry}

\begin{entry}{字典}{6,8}{⼦、⼋}
  \begin{phonetics}{字典}{zi4 dian3}[][HSK 2]
    \definition[本,册,部]{s.}{dicionário de caracteres chineses (contendo verbetes de caracteres únicos, em contraste com 词典 que contém verbetes para palavras com um ou mais caracteres)}
  \seealsoref{词典}{ci2dian3}
  \end{phonetics}
\end{entry}

\begin{entry}{字眼}{6,11}{⼦、⽬}
  \begin{phonetics}{字眼}{zi4yan3}
    \definition[个]{s.}{palavras | redação}
  \end{phonetics}
\end{entry}

\begin{entry}{字脚}{6,11}{⼦、⾁}
  \begin{phonetics}{字脚}{zi4jiao3}
    \definition[典]{s.}{gancho no final da pincelada | serifa}
  \end{phonetics}
\end{entry}

\begin{entry}{存}{6}{⼦}
  \begin{phonetics}{存}{cun2}[][HSK 3]
    \definition{v.}{existir; viver; sobreviver | armazenar; manter | acumular; coletar | depositar | sair com; verificar | reservar; reter | permanecer em equilíbrio; estar em estoque | estimar; abrigar}
  \end{phonetics}
\end{entry}

\begin{entry}{存在}{6,6}{⼦、⼟}
  \begin{phonetics}{存在}{cun2zai4}[][HSK 3]
    \definition{s.}{existência; ser; ente; o mundo objetivo, que não depende da consciência humana para mudar, ou seja, a matéria}
    \definition{v.}{existir; ser; as coisas ocupam continuamente o tempo e o espaço; na verdade, ainda não desapareceram}
  \end{phonetics}
\end{entry}

\begin{entry}{存款}{6,12}{⼦、⽋}
  \begin{phonetics}{存款}{cun2 kuan3}[][HSK 5]
    \definition[笔]{s.}{depósito; poupança bancária}
    \definition{v.}{depositar dinheiro; colocar dinheiro no banco}
  \end{phonetics}
\end{entry}

\begin{entry}{孙}{6}{⼦}
  \begin{phonetics}{孙}{sun1}
    \definition*{s.}{Sobrenome Sun}
    \definition{s.}{neto; neta | gerações abaixo da do neto | parentes pertencentes à geração do neto | segundo crescimento das plantas}
  \end{phonetics}
\end{entry}

\begin{entry}{孙女}{6,3}{⼦、⼥}
  \begin{phonetics}{孙女}{sun1nv3}[][HSK 4]
    \definition{s.}{filha do filho; neta}
  \end{phonetics}
\end{entry}

\begin{entry}{孙子}{6,3}{⼦、⼦}
  \begin{phonetics}{孙子}{sun1zi3}
    \definition*{s.}{Sun Tzu, também conhecido por Sun Wu, 孙武, general, estrategista e filósofo autor do ``Arte da Guerra'', 《孙子兵法》}
  \seealsoref{孙武}{sun1wu3}
  \seealsoref{孙子兵法}{sun1zi3 bing1fa3}
  \end{phonetics}
  \begin{phonetics}{孙子}{sun1zi5}[][HSK 4]
    \definition{s.}{filho do filho; neto}
  \end{phonetics}
\end{entry}

\begin{entry}{孙子兵法}{6,3,7,8}{⼦、⼦、⼋、⽔}
  \begin{phonetics}{孙子兵法}{sun1zi3 bing1fa3}
    \definition*{s.}{``Arte da Guerra'', escrito por Sun Tzu, 孫子}
  \seealsoref{孙武}{sun1wu3}
  \seealsoref{孙子}{sun1zi3}
  \end{phonetics}
\end{entry}

\begin{entry}{孙武}{6,8}{⼦、⽌}
  \begin{phonetics}{孙武}{sun1wu3}
    \definition*{s.}{Sun Wu, também conhecido por Sun Tzu, 孙子, general, estrategista e filósofo autor do ``Arte da Guerra'', 《孙子兵法》}
  \seealsoref{孙子}{sun1zi3}
  \seealsoref{孙子兵法}{sun1zi3 bing1fa3}
  \end{phonetics}
\end{entry}

\begin{entry}{宇}{6}{⼧}
  \begin{phonetics}{宇}{yu3}
    \definition*{s.}{Sobrenome Yu}
    \definition[座,栋]{s.}{beirais; calha; casa | espaço; universo; mundo | postura; porte}
  \end{phonetics}
\end{entry}

\begin{entry}{宇宙}{6,8}{⼧、⼧}
  \begin{phonetics}{宇宙}{yu3zhou4}
    \definition{s.}{universo | cosmos}
  \end{phonetics}
\end{entry}

\begin{entry}{宇航员}{6,10,7}{⼧、⾈、⼝}
  \begin{phonetics}{宇航员}{yu3hang2yuan2}
    \definition{s.}{astronauta}
  \end{phonetics}
\end{entry}

\begin{entry}{守}{6}{⼧}
  \begin{phonetics}{守}{shou3}[][HSK 4]
    \definition*{s.}{Sobrenome Shou}
    \definition{adv.}{próximo; perto de; perto de algum lugar em posição, perto de algum lugar}
    \definition{v.}{guardar; defender; estar presente para cuidar; não ir embora | manter vigilância; defender do ataque do oponente em uma luta ou confronto | observar; cumprir; respeitar; fazer as coisas como elas devem ser feitas | manter, observar a integridade; honrar a palavra de alguém; manter a palavra de alguém}
  \end{phonetics}
\end{entry}

\begin{entry}{守门员}{6,3,7}{⼧、⾨、⼝}
  \begin{phonetics}{守门员}{shou3men2yuan2}
    \definition{s.}{goleiro}
  \end{phonetics}
\end{entry}

\begin{entry}{安}{6}{⼧}
  \begin{phonetics}{安}{an1}[][HSK 4]
    \definition*{s.}{Sobrenome An}
    \definition{adj.}{pacífico; quieto; tranquilo; calmo | seguro; protegido (oposto a 危) | com boa saúde | em paz; bem}
    \definition{adv.}{pacificamente; silenciosamente | com segurança; em segurança | em perguntas retóricas: como?}
    \definition{pron.}{usado como pronome interrogativo, como em 哪里,怎么; 谁,何,如何}
    \definition{s.}{segurança; proteção; paz | ampère; abreviação de ampère, 安培}
    \definition{v.}{tranquilizar (a mente de alguém); acalmar | contentar-se; ficar satisfeito | colocar em uma posição adequada; encontrar um lugar para | instalar; consertar; encaixar; configurar | trazer (uma acusação contra alguém); dar (a alguém um apelido); reivindicar (crédito por algo) | abrigar (uma intenção) | acalmar; estabilizar | sentir-se satisfeito e à vontade}
  \seealsoref{安培}{an1pei2}
  \seealsoref{何}{he2}
  \seealsoref{哪里}{na3 li3}
  \seealsoref{如何}{ru2he2}
  \seealsoref{谁}{shei2}
  \seealsoref{危}{wei1}
  \seealsoref{怎么}{zen3me5}
  \end{phonetics}
\end{entry}

\begin{entry}{安全}{6,6}{⼧、⼊}
  \begin{phonetics}{安全}{an1quan2}[][HSK 2]
    \definition{adj.}{seguro; protegido; sem perigo; sem ameaças; sem acidentes}
    \definition{s.}{segurança; proteção; refere-se a um estado ou conceito, geralmente indicando ausência de ameaças ou perigo}
  \end{phonetics}
\end{entry}

\begin{entry}{安神}{6,9}{⼧、⽰}
  \begin{phonetics}{安神}{an1shen2}
    \definition{v.+compl.}{acalmar os nervos | aliviar a inquietação pela tranquilização da mente e do corpo}
  \end{phonetics}
\end{entry}

\begin{entry}{安家}{6,10}{⼧、⼧}
  \begin{phonetics}{安家}{an1jia1}
    \definition{v.+compl.}{montar uma casa | estabelecer-se}
  \end{phonetics}
\end{entry}

\begin{entry}{安培}{6,11}{⼧、⼟}
  \begin{phonetics}{安培}{an1pei2}
    \definition{clas.}{A; empréstimo linguístico: ampere; física: unidade de corrente elétrica}
  \end{phonetics}
\end{entry}

\begin{entry}{安排}{6,11}{⼧、⼿}
  \begin{phonetics}{安排}{an1pai2}[][HSK 3]
    \definition{s.}{plano; programação; organização; tabela do plano de atividades ou horários}
    \definition{v.}{organizar (assuntos) de acordo com a sequência ou regras; tratar as coisas de acordo com uma determinada ordem ou regras | atribuir tarefas a alguém; colocar as pessoas nos cargos de trabalho determinados, conforme planejado}
  \end{phonetics}
\end{entry}

\begin{entry}{安检}{6,11}{⼧、⽊}
  \begin{phonetics}{安检}{an1 jian3}[][HSK 6]
    \definition{s.}{verificação de segurança}
    \definition{v.}{realizar verificação de segurança}
  \end{phonetics}
\end{entry}

\begin{entry}{安装}{6,12}{⼧、⾐}
  \begin{phonetics}{安装}{an1zhuang1}[][HSK 3]
    \definition{v.}{instalar; consertar; configurar; fixar máquinas ou equipamentos (geralmente conjuntos) em um determinado local, de acordo com métodos e especificações específicos}
  \end{phonetics}
\end{entry}

\begin{entry}{安置}{6,13}{⼧、⽹}
  \begin{phonetics}{安置}{an1zhi4}[][HSK 4]
    \definition{v.}{providenciar; encontrar um lugar para; ajudar a estabelecer-se; colocar pessoas ou coisas em uma determinada posição ou organizá-las adequadamente}
  \end{phonetics}
\end{entry}

\begin{entry}{安静}{6,14}{⼧、⾭}
  \begin{phonetics}{安静}{an1jing4}[][HSK 2]
    \definition{adj.}{silencioso; tranquilo; sem som; sem barulho e sem algazarra}
  \end{phonetics}
\end{entry}

\begin{entry}{安慰}{6,15}{⼧、⼼}
  \begin{phonetics}{安慰}{an1wei4}[][HSK 5]
    \definition{adj.}{confortar; tranquilizar; consolar; apaziguar;}
    \definition[个]{s.}{conforto; consolo; comportamento que alivia a dor de alguém e o acalma com palavras ou gestos}
    \definition{v.}{confortar; consolar; acalmar e confortar; deixar a mente tranquila}
  \end{phonetics}
\end{entry}

\begin{entry}{寺}{6}{⼨}
  \begin{phonetics}{寺}{si4}[][HSK 6]
    \definition*{s.}{Sobrenome Si}
    \definition[座]{s.}{templo | (Islã) mesquita | (datado) ministério; agência governamental na China antiga}
  \end{phonetics}
\end{entry}

\begin{entry}{寺庙}{6,8}{⼨、⼴}
  \begin{phonetics}{寺庙}{si4miao4}
    \definition{s.}{templo | mosteiro | santuário}
  \end{phonetics}
\end{entry}

\begin{entry}{寻}{6}{⼨}
  \begin{phonetics}{寻}{xun2}
    \definition*{s.}{Sobrenome Xun}
    \definition{clas.}{uma unidade antiga de comprimento, igual a 8尺}
    \definition{v.}{procurar; pesquisar; buscar}
  \seealsoref{尺}{chi3}
  \end{phonetics}
\end{entry}

\begin{entry}{寻找}{6,7}{⼨、⼿}
  \begin{phonetics}{寻找}{xun2zhao3}[][HSK 4]
    \definition{v.}{buscar; procurar; pesquisar; encontrar, que pode ser usado tanto para coisas concretas quanto para coisas abstratas}
  \end{phonetics}
\end{entry}

\begin{entry}{寻求}{6,7}{⼨、⽔}
  \begin{phonetics}{寻求}{xun2 qiu2}[][HSK 5]
    \definition{v.}{procurar; perseguir; explorar; ir em busca de}
  \end{phonetics}
\end{entry}

\begin{entry}{导}{6}{⼨}
  \begin{phonetics}{导}{dao3}
    \definition[个,位,名,些]{s.}{guia turístico | diretor}
    \definition{v.}{liderar; guiar | conduzir; transmitir | ensinar; instruir; dar orientação a}
  \end{phonetics}
\end{entry}

\begin{entry}{导致}{6,10}{⼨、⾄}
  \begin{phonetics}{导致}{dao3zhi4}[][HSK 4]
    \definition{v.}{causar; levar a; dar origem a (um resultado ruim)}
  \end{phonetics}
\end{entry}

\begin{entry}{导弹}{6,11}{⼨、⼸}
  \begin{phonetics}{导弹}{dao3dan4}
    \definition[枚]{s.}{míssil (guiado)}
  \end{phonetics}
\end{entry}

\begin{entry}{导游}{6,12}{⼨、⽔}
  \begin{phonetics}{导游}{dao3you2}[][HSK 4]
    \definition[个,位,名]{s.}{guia turístico; pessoas que trabalham como guias turísticos}
    \definition{v.}{guiar; conduzir um passeio turístico}
  \end{phonetics}
\end{entry}

\begin{entry}{导演}{6,14}{⼨、⽔}
  \begin{phonetics}{导演}{dao3yan3}[][HSK 3]
    \definition[位,名,个]{s.}{diretor; pessoa que exerce a função de diretor}
    \definition{v.}{dirigir (um filme, peça, etc.); ensaio de peças teatrais ou filmagem de filmes e séries de TV; organização e orientação do trabalho de produção}
  \end{phonetics}
\end{entry}

\begin{entry}{尖}{6}{⼩}
  \begin{phonetics}{尖}{jian1}[][HSK 6]
    \definition{adj.}{pontiagudo; afilado; agudo | agudo; estridente; penetrante | mesquinho; pão-duro | mordaz; cáustico}
    \definition{s.}{ponto; ponta; topo | o melhor do seu tipo; a melhor escolha; a nata da safra; uma pessoa ou coisa notável}
    \definition{v.}{tornar (a voz, etc.) aguda; estridente}
  \end{phonetics}
\end{entry}

\begin{entry}{尧}{6}{⼪}
  \begin{phonetics}{尧}{yao2}
    \definition*{s.}{Yao, um monarca lendário da China antiga | Sobrenome Yao}
  \end{phonetics}
\end{entry}

\begin{entry}{尽}{6}{⼫}
  \begin{phonetics}{尽}{jin3}
    \definition{adv.}{na maior extensão possível | na extremidade mais distante de | usado antes de palavras que indicam direção, o mesmo que 最 | de agora em diante}
    \definition{prep.}{dentro dos limites de}
    \definition{v.}{dar prioridade a | deixar que certas pessoas ou coisas tenham precedência}
  \seealsoref{最}{zui4}
  \end{phonetics}
  \begin{phonetics}{尽}{jin4}[][HSK 6]
    \definition*{s.}{Sobrenome Jin}
    \definition{adj.}{exausto; acabado | ao máximo; ao limite | tudo; exaustivo}
    \definition{v.}{esgotar | tentar o seu melhor; fazer o melhor uso possível | morrer; falecer | terminar | chegar ao fim ao máximo; alcançar extremos}
  \end{phonetics}
\end{entry}

\begin{entry}{尽力}{6,2}{⼫、⼒}
  \begin{phonetics}{尽力}{jin4li4}[][HSK 4]
    \definition{v.+compl.}{esforçar-se ao máximo; esforçar-se ao máximo; usar toda a sua força; fazer algo com seu melhor esforço}
  \end{phonetics}
\end{entry}

\begin{entry}{尽可能}{6,5,10}{⼫、⼝、⾁}
  \begin{phonetics}{尽可能}{jin3 ke3 neng2}[][HSK 5]
    \definition{adv.}{na medida do possível; com o melhor de sua capacidade; tentar fazer algo, atingir um determinado nível ou extensão}
  \end{phonetics}
\end{entry}

\begin{entry}{尽快}{6,7}{⼫、⼼}
  \begin{phonetics}{尽快}{jin3kuai4}[][HSK 4]
    \definition{adv.}{com toda a velocidade; o mais rápido possível; o mais breve possível}
  \end{phonetics}
\end{entry}

\begin{entry}{尽量}{6,12}{⼫、⾥}
  \begin{phonetics}{尽量}{jin3liang4}[][HSK 3]
    \definition{adv.}{tanto quanto possível; da melhor maneira possível}
  \end{phonetics}
\end{entry}

\begin{entry}{尽管}{6,14}{⼫、⽵}
  \begin{phonetics}{尽管}{jin3guan3}[][HSK 5]
    \definition{adv.}{justo; livremente; faça o que quiser, não se preocupe, não há restrições de movimento ou comportamento}
    \definition{conj.}{no entanto; embora; apesar de ; normalmente usado no início de uma frase anterior para introduzir um fato, seguido de 但是, etc. para introduzir um resultado que o fato não deveria ter; às vezes, também pode ser usado no início de uma frase posterior.}
  \seealsoref{但是}{dan4 shi4}
  \end{phonetics}
\end{entry}

\begin{entry}{岁}{6}{⼭}
  \begin{phonetics}{岁}{sui4}[][HSK 1]
    \definition{clas.}{usado para anos (de idade)}
    \definition{s.}{ano (literário) | colheita do ano (literário) | idade | tempo (literário) | ano (de idade) | ano (para as colheitas)}
  \end{phonetics}
\end{entry}

\begin{entry}{岁月}{6,4}{⼭、⽉}
  \begin{phonetics}{岁月}{sui4yue4}[][HSK 5]
    \definition{s.}{anos; ano e mês; refere-se a tempo em geral}
  \end{phonetics}
\end{entry}

\begin{entry}{岂}{6}{⼭}
  \begin{phonetics}{岂}{qi3}
    \definition*{s.}{Sobrenome Qi}
    \definition{adv.}{expressa uma pergunta retórica, equivalente a 哪里, 怎么 e 难道}
  \seealsoref{哪里}{na3 li3}
  \seealsoref{难道}{nan2dao4}
  \seealsoref{怎么}{zen3me5}
  \end{phonetics}
\end{entry}

\begin{entry}{岂有此理}{6,6,6,11}{⼭、⽉、⽌、⽟}
  \begin{phonetics}{岂有此理}{qi3you3ci3li3}
    \definition{interj.}{Que exorbitante! | Absurdo! | Como isso pode ser assim? | Ridículo!}
  \end{phonetics}
\end{entry}

\begin{entry}{巡}{6}{⾡}
  \begin{phonetics}{巡}{xun2}
    \definition{clas.}{rodada de bebidas | usado para servir vinho a todos}
    \definition{v.}{patrulhar; fazer rondas; fazer uma excursão de inspeção}
  \end{phonetics}
\end{entry}

\begin{entry}{巡逻}{6,11}{⾡、⾡}
  \begin{phonetics}{巡逻}{xun2luo2}
    \definition{s.}{patrulha}
    \definition{v.}{patrulhar (polícia, exército ou marinha)}
  \end{phonetics}
\end{entry}

\begin{entry}{师}{6}{⼱}
  \begin{phonetics}{师}{shi1}
    \definition*{s.}{Sobrenome Shi}
    \definition{s.}{professor | mestre | especialista | modelo | divisão do exército}
    \definition{v.}{despachar tropas}
  \end{phonetics}
\end{entry}

\begin{entry}{师傅}{6,12}{⼱、⼈}
  \begin{phonetics}{师傅}{shi1fu5}[][HSK 5]
    \definition[个,位,名]{s.}{mestre; um trabalhador qualificado; título honorífico para pessoas habilidosas | mestre; professor (em certos ofícios); pessoas que ensinam técnicas em áreas como engenharia, comércio e teatro}
  \end{phonetics}
\end{entry}

\begin{entry}{年}{6}{⼲}
  \begin{phonetics}{年}{nian2}[][HSK 1]
    \definition*{s.}{Sobrenome Nian}
    \definition{clas.}{ano; usado para calcular o número de anos}
    \definition{s.}{ano | idade | um período (época) da história | colheita anual | Ano Novo | artigos para o dia de Ano Novo | um período da vida de uma pessoa; fases da vida humana divididas por idade}
  \end{phonetics}
\end{entry}

\begin{entry}{年代}{6,5}{⼲、⼈}
  \begin{phonetics}{年代}{nian2dai4}[][HSK 3]
    \definition[个]{s.}{idade; anos; tempo; um período de tempo com características distintas na história | uma década de um século; período de dez anos}
  \end{phonetics}
\end{entry}

\begin{entry}{年级}{6,6}{⼲、⽷}
  \begin{phonetics}{年级}{nian2ji2}[][HSK 2]
    \definition[个]{s.}{série; ano; níveis divididos de acordo com o tempo de estudo dos alunos na escola}
  \end{phonetics}
\end{entry}

\begin{entry}{年纪}{6,6}{⼲、⽷}
  \begin{phonetics}{年纪}{nian2ji4}[][HSK 3]
    \definition[把,个]{s.}{idade (de uma pessoa)}
  \end{phonetics}
\end{entry}

\begin{entry}{年初}{6,7}{⼲、⾐}
  \begin{phonetics}{年初}{nian2 chu1}[][HSK 3]
    \definition{s.}{o começo do ano; os primeiros dias do ano}
  \end{phonetics}
\end{entry}

\begin{entry}{年底}{6,8}{⼲、⼴}
  \begin{phonetics}{年底}{nian2 di3}[][HSK 3]
    \definition[个]{s.}{fim de ano; o fim do ano; geralmente os últimos dias de dezembro ou o fim do ano}
  \end{phonetics}
\end{entry}

\begin{entry}{年货}{6,8}{⼲、⾙}
  \begin{phonetics}{年货}{nian2huo4}
    \definition{s.}{mercadorias vendidas no Ano Novo Chinês}
  \end{phonetics}
\end{entry}

\begin{entry}{年前}{6,9}{⼲、⼑}
  \begin{phonetics}{年前}{nian2 qian2}[][HSK 5]
    \definition{s.}{antes do final do ano; antes do ano novo}
  \end{phonetics}
\end{entry}

\begin{entry}{年度}{6,9}{⼲、⼴}
  \begin{phonetics}{年度}{nian2du4}[][HSK 5]
    \definition{s.}{ano; de acordo com a natureza e as necessidades de um negócio, há um prazo de doze meses com data de início e término definidas}
  \end{phonetics}
\end{entry}

\begin{entry}{年轻}{6,9}{⼲、⾞}
  \begin{phonetics}{年轻}{nian2qing1}[][HSK 2]
    \definition{adj.}{jovem; não muito velho (geralmente se refere a pessoas entre 10 e 20 anos)}
  \end{phonetics}
\end{entry}

\begin{entry}{年龄}{6,13}{⼲、⿒}
  \begin{phonetics}{年龄}{nian2ling2}[][HSK 5]
    \definition[个]{s.}{idade; animais, plantas e outros seres vivos vivem e crescem no mundo durante um determinado número de anos}
  \end{phonetics}
\end{entry}

\begin{entry}{并}{6}{⼲}
  \begin{phonetics}{并}{bing4}[][HSK 3,4]
    \definition{adv.}{lado a lado; igualmente; simultaneamente | (usado para reforçar uma negação) na verdade; definitivamente | mesmo assim | (usado para reforçar uma negação) na verdade; de forma alguma | todos; indica o conjunto completo, equivalente a 全部}
    \definition{conj.}{e; além disso}
    \definition{v.}{combinar; fundir; incorporar | ficar (ou colocar) lado a lado | estar paralelo a | anexar; juntar}
  \seealsoref{全部}{quan2bu4}
  \end{phonetics}
\end{entry}

\begin{entry}{并且}{6,5}{⼲、⼀}
  \begin{phonetics}{并且}{bing4qie3}[][HSK 3]
    \definition{conj.}{e; bem como; usado entre verbos, adjetivos ou frases paralelas para indicar que várias ações são realizadas ao mesmo tempo ou que propriedades existem ao mesmo tempo | além disso; além do mais; ademais; usado na segunda metade de uma frase complexa para expressar um significado adicional}
  \end{phonetics}
\end{entry}

\begin{entry}{并排}{6,11}{⼲、⼿}
  \begin{phonetics}{并排}{bing4pai2}
    \definition{adv.}{lado a lado}
  \end{phonetics}
\end{entry}

\begin{entry}{庆}{6}{⼴}
  \begin{phonetics}{庆}{qing4}
    \definition*{s.}{Sobrenome Qing}
    \definition{s.}{celebração | ocasião para celebração; um aniversário que vale a pena comemorar}
    \definition{v.}{celebrar; felicitar; comemorar}
  \end{phonetics}
\end{entry}

\begin{entry}{庆祝}{6,9}{⼴、⽰}
  \begin{phonetics}{庆祝}{qing4zhu4}[][HSK 3]
    \definition{v.}{celebrar; comemorar; festejar; realizar atividades para comemorar ou celebrar festivais comuns e eventos felizes}
  \end{phonetics}
\end{entry}

\begin{entry}{延}{6}{⼵}
  \begin{phonetics}{延}{yan2}
    \definition*{s.}{Sobrenome Yan}
    \definition{v.}{prolongar; estender; alongar | adiar; atrasar | envolver (um professor, conselheiro, etc.); enviar para; convidar}
  \end{phonetics}
\end{entry}

\begin{entry}{延长}{6,4}{⼵、⾧}
  \begin{phonetics}{延长}{yan2chang2}[][HSK 4]
    \definition{v.}{estender; prolongar; alongar; aumentar o tempo, a distância ou a duração de algo específico}
  \end{phonetics}
\end{entry}

\begin{entry}{延伸}{6,7}{⼵、⼈}
  \begin{phonetics}{延伸}{yan2shen1}[][HSK 5]
    \definition{v.}{estender; esticar; alongar; estender-se}
  \end{phonetics}
\end{entry}

\begin{entry}{延续}{6,11}{⼵、⽷}
  \begin{phonetics}{延续}{yan2xu4}[][HSK 4]
    \definition{v.}{durar; continuar; prosseguir; continuar como antes; prolongar}
  \end{phonetics}
\end{entry}

\begin{entry}{延期}{6,12}{⼵、⽉}
  \begin{phonetics}{延期}{yan2qi1}[][HSK 4]
    \definition{v.+compl.}{atrasar; adiar; postergar}
  \end{phonetics}
\end{entry}

\begin{entry}{异}{6}{⼶}
  \begin{phonetics}{异}{yi4}
    \definition{adj.}{diferente | estranho; incomum; extraordinário; especial | outro}
    \definition{v.}{surpreender | separar; divorciar-se}
  \end{phonetics}
\end{entry}

\begin{entry}{异常}{6,11}{⼶、⼱}
  \begin{phonetics}{异常}{yi4chang2}
    \definition{adj.}{extraordinário | anormal}
    \definition{adv.}{extraordinariamente | excepcionalmente}
    \definition{s.}{anormalidade}
  \end{phonetics}
\end{entry}

\begin{entry}{式}{6}{⼷}
  \begin{phonetics}{式}{shi4}[][HSK 5]
    \definition*{s.}{Sobrenome Shi}
    \definition{s.}{tipo; estilo | forma; padrão | ritual; cerimônia | fórmula; conjunto de símbolos que expressam uma lei natural na ciência natural | humor; modo; categoria gramatical que expressa a atitude subjetiva do falante em relação ao que está sendo dito, como narrativa, imperativa e condicional}
  \end{phonetics}
\end{entry}

\begin{entry}{当}{6}{⼹}
  \begin{phonetics}{当}{dang1}[][HSK 2,6]
    \definition*{s.}{Sobrenome Dang}
    \definition{adj.}{igual; adequado; compatível}
    \definition{prep.}{na presença de alguém; na cara de alguém | exatamente em (um momento ou lugar); em algum momento, em algum lugar | na frente de alguém}
    \definition{s.}{topo; cume |uma lacuna no espaço ou no tempo; refere-se a um espaço ou intervalo de tempo}
    \definition{s.}{Onomatopéia: barulho metálico, som de um gongo ou sino}
    \definition{v.}{dever; ter que; dever ser | trabalhar como; servir como; ser; assumir; desempenhar a função de | suportar; aceitar; merecer | dirigir; gerenciar; estar no comando; ser responsável por;  presidir | conter; bloquear; segurar; reter; resistir}
  \end{phonetics}
  \begin{phonetics}{当}{dang4}
    \definition{adj.}{adequado; correto; apropriado | igual; o mesmo}
    \definition{pron.}{naquele mesmo (dia, etc.); refere-se ao momento em que algo aconteceu}
    \definition{s.}{algo penhorado; penhor; garantia; objetos físicos penhorados em casas de penhores}
    \definition{v.}{corresponder; ser igual a; combinar | tratar como; considerar como; tomar como | pensar que; achar que | penhorar; empréstimo com garantia real em uma loja de penhores}
  \end{phonetics}
\end{entry}

\begin{entry}{当中}{6,4}{⼹、⼁}
  \begin{phonetics}{当中}{dang1 zhong1}[][HSK 3]
    \definition{prep.}{no meio; no centro | entre; dentro}
  \end{phonetics}
\end{entry}

\begin{entry}{当天}{6,4}{⼹、⼤}
  \begin{phonetics}{当天}{dan1 tian1}[][HSK 6]
    \definition{s.}{no mesmo dia; naquele mesmo dia; refere-se ao dia em que algo aconteceu no passado}
  \end{phonetics}
\end{entry}

\begin{entry}{当代}{6,5}{⼹、⼈}
  \begin{phonetics}{当代}{dang1dai4}[][HSK 5]
    \definition{s.}{a era atual; a era contemporânea}
  \end{phonetics}
\end{entry}

\begin{entry}{当地}{6,6}{⼹、⼟}
  \begin{phonetics}{当地}{dang1di4}
    \definition{s.}{local; o lugar onde as pessoas e as coisas estão ou onde as coisas acontecem}
  \end{phonetics}
\end{entry}

\begin{entry}{当场}{6,6}{⼹、⼟}
  \begin{phonetics}{当场}{dang1chang3}[][HSK 5]
    \definition{adv.}{na hora; de imediato; na mesma hora}
  \end{phonetics}
\end{entry}

\begin{entry}{当年}{6,6}{⼹、⼲}
  \begin{phonetics}{当年}{dang1 nian2}[][HSK 5]
    \definition{adv.}{durante esse período; durante esse tempo | naquela época; naqueles dias | naqueles anos | nessa ocasião}
  \end{phonetics}
  \begin{phonetics}{当年}{dang4 nian2}
    \definition{s.}{no mesmo ano; naquele mesmo ano}
  \end{phonetics}
\end{entry}

\begin{entry}{当成}{6,6}{⼹、⼽}
  \begin{phonetics}{当成}{dan4 cheng2}[][HSK 6]
    \definition{v.}{considerar como; tratar como; tomar por}
  \end{phonetics}
\end{entry}

\begin{entry}{当作}{6,7}{⼹、⼈}
  \begin{phonetics}{当作}{dang4 zuo4}[][HSK 6]
    \definition{v.}{tratar como; considerar como}
  \end{phonetics}
\end{entry}

\begin{entry}{当初}{6,7}{⼹、⾐}
  \begin{phonetics}{当初}{dang1chu1}[][HSK 3]
    \definition{s.}{no começo; originalmente; no início; em primeiro lugar; refere-se a algo que aconteceu no passado, seja em geral ou especificamente}
  \end{phonetics}
\end{entry}

\begin{entry}{当时}{6,7}{⼹、⽇}
  \begin{phonetics}{当时}{dang1shi2}[][HSK 2]
    \definition{s.}{naquela época; aquela ocasião; aquela vez; refere-se a algo que aconteceu no passado}
    \definition{v.}{ser o momento adequado; acontecer no momento certo}
  \end{phonetics}
  \begin{phonetics}{当时}{dang4shi2}
    \definition{adv.}{(depois de fazer algo ou algo acontecer) imediatamente; de imediato; agora mesmo}
  \end{phonetics}
\end{entry}

\begin{entry}{当前}{6,9}{⼹、⼑}
  \begin{phonetics}{当前}{dang1qian2}[][HSK 5]
    \definition{s.}{presente; atual}
    \definition{v.}{estar diante de alguém; estar frente a frente com alguém; na frente de, geralmente refere-se a uma situação perigosa}
  \end{phonetics}
\end{entry}

\begin{entry}{当选}{6,9}{⼹、⾡}
  \begin{phonetics}{当选}{dang1xuan3}[][HSK 5]
    \definition{v.}{ser eleito}
  \end{phonetics}
\end{entry}

\begin{entry}{当然}{6,12}{⼹、⽕}
  \begin{phonetics}{当然}{dang1ran2}[][HSK 3]
    \definition{adj.}{natural; verdadeiro; espontâneo}
    \definition{adv.}{sem dúvida; certamente; claro}
  \end{phonetics}
\end{entry}

\begin{entry}{忙}{6}{⼼}
  \begin{phonetics}{忙}{mang2}[][HSK 1]
    \definition*{s.}{Sobrenome Mang}
    \definition{adj.}{ocupado; movimentado; totalmente ocupado; muitas coisas para fazer, sem tempo livre (oposto de 闲) | imperativo; ansioso; urgente}
    \definition{v.}{apressar-se; agitar-se; fazer algo com urgência e constantemente | trabalhar; fazer}
  \seealsoref{闲}{xian2}
  \end{phonetics}
\end{entry}

\begin{entry}{戏}{6}{⼽}
  \begin{phonetics}{戏}{xi4}[][HSK 5]
    \definition*{s.}{Sobrenome Xi}
    \definition[场,部,出,台]{s.}{drama; peça; espetáculo; \emph{show}}
    \definition{v.}{brincar; praticar esportes; jogar | zombar; brincar; provocar}
  \end{phonetics}
\end{entry}

\begin{entry}{戏弄}{6,7}{⼽、⼶}
  \begin{phonetics}{戏弄}{xi4nong4}
    \definition{v.}{zombar de | pregar peças | provocar}
  \end{phonetics}
\end{entry}

\begin{entry}{戏法}{6,8}{⼽、⽔}
  \begin{phonetics}{戏法}{xi4fa3}
    \definition{s.}{truque de mágica | prestidigitação}
  \end{phonetics}
\end{entry}

\begin{entry}{戏耍}{6,9}{⼽、⽽}
  \begin{phonetics}{戏耍}{xi4shua3}
    \definition{v.}{divertir-me | brincar com | provocar}
  \end{phonetics}
\end{entry}

\begin{entry}{戏院}{6,9}{⼽、⾩}
  \begin{phonetics}{戏院}{xi4yuan4}
    \definition{s.}{teatro}
  \end{phonetics}
\end{entry}

\begin{entry}{戏剧}{6,10}{⼽、⼑}
  \begin{phonetics}{戏剧}{xi4ju4}[][HSK 5]
    \definition{s.}{drama; peça; teatro | roteiro; peça; cenário}
  \end{phonetics}
\end{entry}

\begin{entry}{戏剧化地}{6,10,4,6}{⼽、⼑、⼔、⼟}
  \begin{phonetics}{戏剧化地}{xi4ju4hua4di4}
    \definition{adv.}{dramaticamente | teatralmente}
  \end{phonetics}
\end{entry}

\begin{entry}{戏剧性}{6,10,8}{⼽、⼑、⼼}
  \begin{phonetics}{戏剧性}{xi4ju4xing4}
    \definition{adj.}{dramático}
  \end{phonetics}
\end{entry}

\begin{entry}{戏剧家}{6,10,10}{⼽、⼑、⼧}
  \begin{phonetics}{戏剧家}{xi4ju4jia1}
    \definition{s.}{dramaturgo}
  \end{phonetics}
\end{entry}

\begin{entry}{戏剧效果}{6,10,10,8}{⼽、⼑、⽁、⽊}
  \begin{phonetics}{戏剧效果}{xi4ju4xiao4guo3}
    \definition{s.}{efeito dramático}
  \end{phonetics}
\end{entry}

\begin{entry}{戏剧般}{6,10,10}{⼽、⼑、⾈}
  \begin{phonetics}{戏剧般}{xi4ju4ban1}
    \definition{adj.}{melodramático}
  \end{phonetics}
\end{entry}

\begin{entry}{戏剧编剧}{6,10,12,10}{⼽、⼑、⽷、⼑}
  \begin{phonetics}{戏剧编剧}{xi4ju4bian1ju4}
    \definition{s.}{dramaturgo}
  \end{phonetics}
\end{entry}

\begin{entry}{戏剧演出}{6,10,14,5}{⼽、⼑、⽔、⼐}
  \begin{phonetics}{戏剧演出}{xi4ju4yan3chu1}
    \definition{s.}{performance dramática}
  \end{phonetics}
\end{entry}

\begin{entry}{戏谑}{6,11}{⼽、⾔}
  \begin{phonetics}{戏谑}{xi4xue4}
    \definition{v.}{brincar | fazer piadas | ridicularizar}
  \end{phonetics}
\end{entry}

\begin{entry}{成}{6}{⼽}
  \begin{phonetics}{成}{cheng2}[][HSK 2,6]
    \definition*{s.}{Sobrenome Cheng}
    \definition{adj.}{capaz; competente | totalmente crescido; totalmente desenvolvido; maduro | estabelecido; Já definido; pronto para uso | em números ou quantidades consideráveis; inteiro; suficiente: enfatiza a quantidade ou a duração}
    \definition{clas.}{um décimo}
    \definition{interj.}{O.K.; tudo bem}
    \definition{s.}{resultado; conquista}
    \definition{v.}{ter sucesso; conseguir; ser bem-sucedido | tornar-se; transformar-se | ajudar a completar; realizar}
  \end{phonetics}
\end{entry}

\begin{entry}{成人}{6,2}{⼽、⼈}
  \begin{phonetics}{成人}{cheng2ren2}[][HSK 4]
    \definition[个]{s.}{adulto; crescido; pessoa adulta}
    \definition{v.}{crescer; tornar-se adulto}
  \end{phonetics}
\end{entry}

\begin{entry}{成为}{6,4}{⼽、⼂}
  \begin{phonetics}{成为}{cheng2wei2}[][HSK 2]
    \definition{v.}{tornar-se; transformar-se; revelar-se; passar de uma situação, identidade ou estado para outro}
  \end{phonetics}
\end{entry}

\begin{entry}{成分}{6,4}{⼽、⼑}
  \begin{phonetics}{成分}{cheng2fen4}[][HSK 6]
    \definition[个,些,种]{s.}{composição; ingrediente; elemento; parte componente; as várias substâncias ou fatores que compõem as coisas | a condição de classe de alguém; a profissão ou a condição econômica de alguém; refere-se à classe à qual uma família pertence; à principal experiência ou ocupação anterior de uma pessoa}
  \end{phonetics}
\end{entry}

\begin{entry}{成长}{6,4}{⼽、⾧}
  \begin{phonetics}{成长}{cheng2zhang3}[][HSK 3]
    \definition{v.}{crescer; amadurecer; tornar-se adulto; o desenvolvimento de seres humanos, animais ou plantas desde a infância até a maturidade}
  \end{phonetics}
\end{entry}

\begin{entry}{成功}{6,5}{⼽、⼒}
  \begin{phonetics}{成功}{cheng2gong1}[][HSK 3]
    \definition{adj.}{bem-sucedido; frutífero}
    \definition[个,次]{s.}{sucesso}
    \definition{v.}{ter sucesso; obter os resultados esperados}
  \end{phonetics}
\end{entry}

\begin{entry}{成本}{6,5}{⼽、⽊}
  \begin{phonetics}{成本}{cheng2ben3}[][HSK 5]
    \definition{s.}{custo principal; custo; custo capitalizado; custo final; primeiro custo; custo próprio; custo de produção de um produto}
  \end{phonetics}
\end{entry}

\begin{entry}{成立}{6,5}{⼽、⽴}
  \begin{phonetics}{成立}{cheng2li4}[][HSK 3]
    \definition{v.}{fundar; estabelecer; criar; (organizações, instituições, etc.) começar a existir e a funcionar | ser válido; ser sustentável; fazer sentido; (teorias, pontos de vista, razões, etc.) fundamentados e válidos}
  \end{phonetics}
\end{entry}

\begin{entry}{成交}{6,6}{⼽、⼇}
  \begin{phonetics}{成交}{cheng2jiao1}[][HSK 5]
    \definition{v.+compl.}{fechar um acordo; fazer uma barganha; concluir uma transação}
  \end{phonetics}
\end{entry}

\begin{entry}{成吉思汗}{6,6,9,6}{⼽、⼝、⼼、⽔}
  \begin{phonetics}{成吉思汗}{cheng2ji2si1han2}
    \definition*{s.}{Genghis Khan (1162-1227), fundador e governante do Império Mongol}
  \end{phonetics}
\end{entry}

\begin{entry}{成色}{6,6}{⼽、⾊}
  \begin{phonetics}{成色}{cheng2se4}
    \definition{v.}{sair-se bem | ser bem sucedido}
  \end{phonetics}
\end{entry}

\begin{entry}{成员}{6,7}{⼽、⼝}
  \begin{phonetics}{成员}{cheng2yuan2}[][HSK 3]
    \definition[个,些,名,位]{s.}{membro; membros de um grupo ou família}
  \end{phonetics}
\end{entry}

\begin{entry}{成批}{6,7}{⼽、⼿}
  \begin{phonetics}{成批}{cheng2pi1}
    \definition{s.}{em lotes | a granel}
  \end{phonetics}
\end{entry}

\begin{entry}{成果}{6,8}{⼽、⽊}
  \begin{phonetics}{成果}{cheng2guo3}[][HSK 3]
    \definition[个]{s.}{realização; resultado; conquista; recompensas no trabalho ou na carreira}
  \end{phonetics}
\end{entry}

\begin{entry}{成品}{6,9}{⼽、⼝}
  \begin{phonetics}{成品}{cheng2 pin3}[][HSK 6]
    \definition[批]{s.}{produto final; produto acabado; produto processado e pronto para ser fornecido}
  \end{phonetics}
\end{entry}

\begin{entry}{成活}{6,9}{⼽、⽔}
  \begin{phonetics}{成活}{cheng2huo2}
    \definition{v.}{sobreviver}
  \end{phonetics}
\end{entry}

\begin{entry}{成语}{6,9}{⼽、⾔}
  \begin{phonetics}{成语}{cheng2yu3}[][HSK 5]
    \definition{s.}{expressão idiomática; frase de conjunto (frases de quatro caracteres em chinês, geralmente com alusões literárias)}
  \end{phonetics}
\end{entry}

\begin{entry}{成家}{6,10}{⼽、⼧}
  \begin{phonetics}{成家}{cheng2jia1}
    \definition{v.}{tornar-se um especialista reconhecido | estabelecer-se e casar-se (de um homem)}
  \end{phonetics}
\end{entry}

\begin{entry}{成效}{6,10}{⼽、⽁}
  \begin{phonetics}{成效}{cheng2xiao4}[][HSK 5]
    \definition{s.}{efeito; resultado}
  \end{phonetics}
\end{entry}

\begin{entry}{成都}{6,10}{⼽、⾢}
  \begin{phonetics}{成都}{cheng2du1}
    \definition*{s.}{Chengdu}
  \end{phonetics}
\end{entry}

\begin{entry}{成婚}{6,11}{⼽、⼥}
  \begin{phonetics}{成婚}{cheng2hun1}
    \definition{v.}{casar-se}
  \end{phonetics}
\end{entry}

\begin{entry}{成绩}{6,11}{⼽、⽷}
  \begin{phonetics}{成绩}{cheng2ji4}[][HSK 2]
    \definition[项,个]{s.}{realização; sucesso; resultado (de trabalho ou estudo); refere-se à pontuação obtida em exames e competições; classificação, também se refere aos resultados alcançados no trabalho}
  \end{phonetics}
\end{entry}

\begin{entry}{成就}{6,12}{⼽、⼪}
  \begin{phonetics}{成就}{cheng2jiu4}[][HSK 3]
    \definition[个,项]{s.}{realização; conquista; sucesso; realizações profissionais}
    \definition{v.}{realizar; alcançar; completar; concluir (carreira)}
  \end{phonetics}
\end{entry}

\begin{entry}{成熟}{6,15}{⼽、⽕}
  \begin{phonetics}{成熟}{cheng2shu2}[][HSK 3]
    \definition{adj.}{maduro; amadurecido; totalmente desenvolvido; descreve que as oportunidades, condições, etc. estão perfeitas e que não haverá nenhum problema}
    \definition{v.}{amadurecer; atingir a maturidade; estar totalmente desenvolvido; frutas e outros frutos totalmente maduros, referindo-se ao desenvolvimento completo de organismos vivos}
  \end{phonetics}
\end{entry}

\begin{entry}{成器}{6,16}{⼽、⼝}
  \begin{phonetics}{成器}{cheng2qi4}
    \definition{v.}{tornar-se uma pessoa digna de respeito | fazer algo de si mesmo}
  \end{phonetics}
\end{entry}

\begin{entry}{托}{6}{⼿}
  \begin{phonetics}{托}{tuo1}[][HSK 6]
    \definition{clas.}{torr, uma unidade de pressão, 1 torr é igual à pressão de 1 mmHg, ou 133,322 Pa}
    \definition{s.}{algo servindo como suporte | fantoche; cúmplice; pessoas que ajudam golpistas a enganar outras pessoas}
    \definition{v.}{segurar na palma; apoiar com a mão ou palma; suportar (um objeto) com um objeto ou com a palma da mão | destacar; servir como contraste | pedir; confiar | implorar; dar como pretexto | dever a; confiar em}
  \end{phonetics}
\end{entry}

\begin{entry}{扛}{6}{⼿}
  \begin{phonetics}{扛}{gang1}
    \definition{v.}{levantar com as duas mãos | carregar alguma coisa juntos (duas ou mais pessoas)}
  \end{phonetics}
  \begin{phonetics}{扛}{kang2}
    \definition{v.}{carregar objetos nos ombros |  suportar; aguentar | lidar; assumir}
  \end{phonetics}
\end{entry}

\begin{entry}{扣}{6}{⼿}
  \begin{phonetics}{扣}{kou4}[][HSK 6]
    \definition*{s.}{Sobrenome Kou}
    \definition{clas.}{giro; volta; uma volta de uma rosca}
    \definition[个,颗,粒]{s.}{nó | fivela; botão | círculo de rosca (em um parafuso)}
    \definition{v.}{fivela; abotoar; amarrar ou prender com um laço ou anel | colocar uma xícara, tigela etc. de cabeça para baixo; cobrir com uma xícara, tigela etc. invertida; colocar a boca do recipiente para baixo | deter; prender; levar sob custódia | cravar; esmagar (a bola); arremessar ou bater (em uma bola) com força de cima para baixo | atracar; deduzir; descontar; subtrair uma parte do valor original | puxar; pressionar | impor; marcar sem fundamento; acusar injustamente; impor ou atribuir (um crime ou má fama) a alguém}
  \end{phonetics}
\end{entry}

\begin{entry}{执}{6}{⼿}
  \begin{phonetics}{执}{zhi2}
    \definition*{s.}{Sobrenome Zhi}
    \definition[期]{s.}{reconhecimento por escrito | (literário) amigo íntimo (ou do peito)}
    \definition{v.}{segurar; agarrar; pegar; capturar | assumir o comando de; dirigir; gerenciar; controlar; administrar; exercer | manter (os próprios pontos de vista, etc.); persistir; persistir em; manter-se em; insistir em | realizar; executar; implementar}
  \end{phonetics}
\end{entry}

\begin{entry}{执行}{6,6}{⼿、⾏}
  \begin{phonetics}{执行}{zhi2xing2}[][HSK 5]
    \definition{v.}{executar; implementar; realizar}
  \end{phonetics}
\end{entry}

\begin{entry}{执着}{6,11}{⼿、⽬}
  \begin{phonetics}{执着}{zhi2zhuo2}
    \definition{s.}{(budismo) apego}
    \definition{v.}{estar fortemente apegado a | ser dedicado | apegar-se a}
  \end{phonetics}
\end{entry}

\begin{entry}{扩}{6}{⼿}
  \begin{phonetics}{扩}{kuo4}
    \definition{v.}{expandir; ampliar; estender; alargar}
  \end{phonetics}
\end{entry}

\begin{entry}{扩大}{6,3}{⼿、⼤}
  \begin{phonetics}{扩大}{kuo4da4}[][HSK 4]
    \definition{v.}{ampliar; expandir; estender; alargar}
  \end{phonetics}
\end{entry}

\begin{entry}{扩展}{6,10}{⼿、⼫}
  \begin{phonetics}{扩展}{kuo4 zhan3}[][HSK 4]
    \definition{v.}{esticar; expandir; estender; espalhar}
  \end{phonetics}
\end{entry}

\begin{entry}{扫}{6}{⼿}
  \begin{phonetics}{扫}{sao3}[][HSK 4]
    \definition{v.}{varrer; limpar | passar rapidamente ao longo ou sobre; varrer | juntar tudo}
  \end{phonetics}
  \begin{phonetics}{扫}{sao4}
  \seealsoref{扫帚}{sao4zhou5}
  \end{phonetics}
\end{entry}

\begin{entry}{扫兴}{6,6}{⼿、⼋}
  \begin{phonetics}{扫兴}{sao3xing4}
    \definition{v.+compl.}{sentir-se decepcionado | entristecer alguém}
  \end{phonetics}
\end{entry}

\begin{entry}{扫帚}{6,8}{⼿、⼱}
  \begin{phonetics}{扫帚}{sao4zhou5}
    \definition[把]{s.}{vassoura; ferramenta de varredura feita de varas de bambu, etc., maior que uma vassora}
  \end{phonetics}
\end{entry}

\begin{entry}{扬}{6}{⼿}
  \begin{phonetics}{扬}{yang2}
    \definition*{s.}{Yangzhou, abreviação de 扬州 | Sobrenome Yang}
    \definition{v.}{levantar | separar e espalhar; peneirar | espalhar; fazer conhecido}
  \seealsoref{扬州}{yang2zhou1}
  \end{phonetics}
\end{entry}

\begin{entry}{扬州}{6,6}{⼿、⼮}
  \begin{phonetics}{扬州}{yang2zhou1}
    \definition*{s.}{Yangzhou, uma cidade na província de Jiangsu}
  \end{phonetics}
\end{entry}

\begin{entry}{扬雄}{6,12}{⼿、⾫}
  \begin{phonetics}{扬雄}{yang2xiong2}
    \definition*{s.}{Yang Xiong (53 AC-18 DC), estudioso, poeta e lexicógrafo, autor do primeiro dicionário de dialeto chinês 方言}
  \seealsoref{方言}{fang1yan2}
  \end{phonetics}
\end{entry}

\begin{entry}{收}{6}{⽁}
  \begin{phonetics}{收}{shou1}[][HSK 2]
    \definition{expr.}{aos cuidados de (usado na linha de endereço após o nome)}
    \definition{v.}{recolocar; juntar; reunir e juntar coisas espalhadas ou dispersas | recolher; cobrar | ganhar; obter (benefícios econômicos) | colher; recolher; colher ou cortar frutas, legumes, cereais maduros, etc. | aceitar; receber; acolher | controlar; restringir; restringir, controlar os sentimentos ou ações, para voltar ao estado normal | finalizar; parar; concluir; encerrar | prender; deter; colocar sob custódia}
  \end{phonetics}
\end{entry}

\begin{entry}{收入}{6,2}{⽁、⼊}
  \begin{phonetics}{收入}{shou1ru4}[][HSK 2]
    \definition[笔,个]{s.}{renda; salário; dinheiro recebido}
    \definition{v.}{receber dinheiro | coletar; receber}
  \end{phonetics}
\end{entry}

\begin{entry}{收买}{6,6}{⽁、⼄}
  \begin{phonetics}{收买}{shou1mai3}
    \definition{v.}{subornar | comprar}
  \end{phonetics}
\end{entry}

\begin{entry}{收回}{6,6}{⽁、⼞}
  \begin{phonetics}{收回}{shou1 hui2}[][HSK 4]
    \definition{v.}{retomar; recuperar; relembrar; recordar; receber de volta o que foi enviado ou emprestado, ou o dinheiro que foi emprestado ou usado | sacar; retirar; recolher; rescindir; cancelar (uma opinião, ordem, etc.)}
  \end{phonetics}
\end{entry}

\begin{entry}{收听}{6,7}{⽁、⼝}
  \begin{phonetics}{收听}{shou1 ting1}[][HSK 3]
    \definition{v.}{ouvir (rádio)}
  \end{phonetics}
\end{entry}

\begin{entry}{收到}{6,8}{⽁、⼑}
  \begin{phonetics}{收到}{shou1 dao4}[][HSK 2]
    \definition{v.}{conseguir; obter; receber; alcançar}
  \end{phonetics}
\end{entry}

\begin{entry}{收购}{6,8}{⽁、⾙}
  \begin{phonetics}{收购}{shou1 gou4}[][HSK 5]
    \definition{v.}{comprar; adquirir; comprar muito em vários lugares | adquirir uma empresa; obter o controle efetivo de uma empresa por meio de dinheiro, transações de ações, etc.}
  \end{phonetics}
\end{entry}

\begin{entry}{收拾}{6,9}{⽁、⼿}
  \begin{phonetics}{收拾}{shou1shi5}[][HSK 5]
    \definition{v.}{arrumar; empacotar; limpar; organizar, policiar, restaurar a normalidade em situações adversas | consertar; reparar; restaurar algo que está danificado ao seu estado ou função original |  punir; punir alguém, geralmente com medidas mais severas | matar}
  \end{phonetics}
\end{entry}

\begin{entry}{收看}{6,9}{⽁、⽬}
  \begin{phonetics}{收看}{shou1 kan4}[][HSK 3]
    \definition{v.}{assistir (a um programa de TV)}
  \end{phonetics}
\end{entry}

\begin{entry}{收费}{6,9}{⽁、⾙}
  \begin{phonetics}{收费}{shou1 fei4}[][HSK 3]
    \definition{v.}{cobrar; cobrar taxas}
  \end{phonetics}
\end{entry}

\begin{entry}{收音机}{6,9,6}{⽁、⾳、⽊}
  \begin{phonetics}{收音机}{shou1yin1ji1}[][HSK 3]
    \definition[部,台]{s.}{rádio; sem fio; um termo geral para receptores de rádio}
  \end{phonetics}
\end{entry}

\begin{entry}{收益}{6,10}{⽁、⽫}
  \begin{phonetics}{收益}{shou1yi4}[][HSK 4]
    \definition{s.}{lucro; renda; benefício; ganhos; vantagens ou benefícios obtidos}
  \end{phonetics}
\end{entry}

\begin{entry}{收获}{6,10}{⽁、⾋}
  \begin{phonetics}{收获}{shou1huo4}[][HSK 4]
    \definition[次,番,份]{s.}{resultados; ganhos; metaforicamente falando, conhecimento, experiência, etc. obtidos em estudo ou trabalho; os resultados obtidos por meio de trabalho árduo | colheita; colheita de safras}
    \definition{v.}{colher; juntar as colheitas;}
  \end{phonetics}
\end{entry}

\begin{entry}{收据}{6,11}{⽁、⼿}
  \begin{phonetics}{收据}{shou1ju4}
    \definition[张]{s.}{recibo | \emph{voucher}}
  \end{phonetics}
\end{entry}

\begin{entry}{收敛}{6,11}{⽁、⽁}
  \begin{phonetics}{收敛}{shou1lian3}
    \definition{v.}{diminuir | desaparecer | fazer desaparecer | exercer restrição | conter (alegria, arrogância, etc.) | constringir | (matemática) convergir}
  \end{phonetics}
\end{entry}

\begin{entry}{收集}{6,12}{⽁、⾫}
  \begin{phonetics}{收集}{shou1 ji2}[][HSK 5]
    \definition{v.}{coletar; reunir; recolher}
  \end{phonetics}
\end{entry}

\begin{entry}{早}{6}{⽇}
  \begin{phonetics}{早}{zao3}[][HSK 1]
    \definition{adj.}{precoce; antes do previsto ou planejado; antes do tempo; antes de um determinado momento |}
    \definition{adv.}{há muito tempo; desde cedo; por muito tempo; há muito tempo atrás}
    \definition{interj.}{bom dia; saudações, usadas para cumprimentar uns aos outros ao se encontrarem pela manhã}
    \definition[个]{s.}{manhã}
  \end{phonetics}
\end{entry}

\begin{entry}{早上}{6,3}{⽇、⼀}
  \begin{phonetics}{早上}{zao3shang5}[][HSK 1]
    \definition[个]{s.}{de manhã cedo; madrugada; o período antes e depois do nascer do sol; geralmente, desde o amanhecer até às 8h ou 9h da manhã; às vezes também se refere ao período entre o amanhecer e o meio-dia}
  \end{phonetics}
\end{entry}

\begin{entry}{早亡}{6,3}{⽇、⼇}
  \begin{phonetics}{早亡}{zao3wang2}
    \definition[个]{s.}{morte prematura}
    \definition{v.}{morrer prematuramente}
  \end{phonetics}
\end{entry}

\begin{entry}{早已}{6,3}{⽇、⼰}
  \begin{phonetics}{早已}{zao3 yi3}[][HSK 3]
    \definition{adv.}{há muito tempo; por muito tempo | (dialeto) no passado}
  \end{phonetics}
\end{entry}

\begin{entry}{早车}{6,4}{⽇、⾞}
  \begin{phonetics}{早车}{zao3che1}
    \definition{s.}{trem matutino | ônibus matutino}
  \end{phonetics}
\end{entry}

\begin{entry}{早安}{6,6}{⽇、⼧}
  \begin{phonetics}{早安}{zao3'an1}
    \definition{interj.}{Bom dia!}
  \end{phonetics}
\end{entry}

\begin{entry}{早早儿}{6,6,2}{⽇、⽇、⼉}
  \begin{phonetics}{早早儿}{zao3zao3r5}
    \definition{adv.}{o mais cedo possível | o mais breve possível}
  \end{phonetics}
\end{entry}

\begin{entry}{早饭}{6,7}{⽇、⾷}
  \begin{phonetics}{早饭}{zao3 fan4}[][HSK 1]
    \definition[份,顿]{s.}{o café da manhã}
  \end{phonetics}
\end{entry}

\begin{entry}{早知}{6,8}{⽇、⽮}
  \begin{phonetics}{早知}{zao3zhi1}
    \definition{v.}{prever | se alguém soubesse antes, \dots}
  \end{phonetics}
\end{entry}

\begin{entry}{早前}{6,9}{⽇、⼑}
  \begin{phonetics}{早前}{zao3qian2}
    \definition{adv.}{previamente}
  \end{phonetics}
\end{entry}

\begin{entry}{早晚}{6,11}{⽇、⽇}
  \begin{phonetics}{早晚}{zao3 wan3}
    \definition{adv./s.}{manhã e noite | mais cedo ou mais tarde; cedo ou tarde | algum tempo no futuro; algum dia; em algum momento no futuro}
  \end{phonetics}
\end{entry}

\begin{entry}{早晨}{6,11}{⽇、⽇}
  \begin{phonetics}{早晨}{zao3 chen2}[][HSK 2]
    \definition[个,段,番]{s.}{manhã cedo; manhãzinha; o período do amanhecer às oito ou nove horas; às vezes, o período da meia-noite ao meio-dia}
  \end{phonetics}
\end{entry}

\begin{entry}{早就}{6,12}{⽇、⼪}
  \begin{phonetics}{早就}{zao3 jiu4}[][HSK 2]
    \definition{adv.}{já; há muito tempo; há muito tempo atrás}
  \end{phonetics}
\end{entry}

\begin{entry}{早期}{6,12}{⽇、⽉}
  \begin{phonetics}{早期}{zao3 qi1}[][HSK 5]
    \definition{s.}{prófase; estágio inicial; fase inicial; a fase inicial de uma determinada época, processo ou vida de uma pessoa}
  \end{phonetics}
\end{entry}

\begin{entry}{早餐}{6,16}{⽇、⾷}
  \begin{phonetics}{早餐}{zao3 can1}[][HSK 2]
    \definition[份,桌,顿]{s.}{café da manhã; desejum}
  \end{phonetics}
\end{entry}

\begin{entry}{曲}{6}{⽈}
  \begin{phonetics}{曲}{qu1}
    \definition*{s.}{Sobrenome Qu}
    \definition{adj.}{dobrado; curvado (oposto a 直) | errado; injustificável | torto}
    \definition{v.}{dobrar | torcer}
  \seealsoref{直}{zhi2}
  \end{phonetics}
\end{entry}

\begin{entry}{曲棍球}{6,12,11}{⽈、⽊、⽟}
  \begin{phonetics}{曲棍球}{qu1gun4qiu2}
    \definition{s.}{hóquei em campo}
  \end{phonetics}
\end{entry}

\begin{entry}{有}{6}{⽉}
  \begin{phonetics}{有}{you3}[][HSK 1]
    \definition*{s.}{Sobrenome You}
    \definition{pref.}{usado antes do nome de certas dinastias ou etnias}
    \definition{v.}{ter; possuir; indica posse ou propriedade | existe; há; indica que certas coisas existem em certos lugares | fazer uma estimativa ou uma comparação; expressar estimativa ou comparação | indicar ação; indica que algo aconteceu ou ocorreu | usado antes de substantivos abstratos, indica quantidade ou grandeza | em termos gerais, semelhante a 某; refere-se de maneira geral a algo semelhante | usado antes de pessoa, hora e lugar, indica a existência parcial | usado antes de certos verbos para formar uma expressão idiomática, indicando cortesia, polidez}
  \seealsoref{某}{mou3}
  \end{phonetics}
\end{entry}

\begin{entry}{有(一)些}{6,1,8}{⽉、⼀、⼆}
  \begin{phonetics}{有(一)些}{you3 (yi4) xie1}[][HSK 1]
    \definition{adv.}{em vez disso; em vez de; de certa forma}
    \definition{pron.}{de certa forma}
  \seealsoref{有些}{you3 xie1}
  \end{phonetics}
\end{entry}

\begin{entry}{有(一)点儿}{6,1,9,2}{⽉、⼀、⽕、⼉}
  \begin{phonetics}{有(一)点儿}{you3 yi4 dian3r5}[][HSK 2]
    \definition{adv.}{um pouco (有点儿 + {s.} ou {v. mental})}
  \seealsoref{有点儿}{you3 dian3r5}
  \end{phonetics}
\end{entry}

\begin{entry}{有人}{6,2}{⽉、⼈}
  \begin{phonetics}{有人}{you3 ren2}[][HSK 2]
    \definition{adj.}{ocupado (como no banheiro)}
    \definition{pron.}{qualquer um; alguém}
    \definition[所]{s.}{pessoas}
    \definition{v.}{ter alguém ali}
  \end{phonetics}
\end{entry}

\begin{entry}{有力}{6,2}{⽉、⼒}
  \begin{phonetics}{有力}{you3 li4}[][HSK 5]
    \definition{adj.}{forte; vigoroso; poderoso; energético}
  \end{phonetics}
\end{entry}

\begin{entry}{有用}{6,5}{⽉、⽤}
  \begin{phonetics}{有用}{you3yong4}[][HSK 1]
    \definition{adj.}{útil; prático; funcional}
  \end{phonetics}
\end{entry}

\begin{entry}{有名}{6,6}{⽉、⼝}
  \begin{phonetics}{有名}{you3ming2}[][HSK 1]
    \definition{adj.}{conhecido; famoso; célebre; nome conhecido por todos}
  \end{phonetics}
\end{entry}

\begin{entry}{有名无实}{6,6,4,8}{⽉、⼝、⽆、⼧}
  \begin{phonetics}{有名无实}{you3ming2wu2shi2}
    \definition{v.}{(literal) tem um nome, mas não tem realidade | existe apenas no nome}
  \end{phonetics}
\end{entry}

\begin{entry}{有利}{6,7}{⽉、⼑}
  \begin{phonetics}{有利}{you3li4}[][HSK 3]
    \definition{adj.}{benéfico; favorável; vantajoso}
  \end{phonetics}
\end{entry}

\begin{entry}{有利于}{6,7,3}{⽉、⼑、⼆}
  \begin{phonetics}{有利于}{you3 li4 yu2}[][HSK 5]
    \definition{prep.}{em favor de; fazer para; em benefício de; aproveitar}
  \end{phonetics}
\end{entry}

\begin{entry}{有劲儿}{6,7,2}{⽉、⼒、⼉}
  \begin{phonetics}{有劲儿}{you3 jin4er5}[][HSK 4]
    \definition{adj.}{forte; enérgico; energético}
  \end{phonetics}
\end{entry}

\begin{entry}{有劳}{6,7}{⽉、⼒}
  \begin{phonetics}{有劳}{you3lao2}
    \definition{v.}{posso incomodá-lo; desculpe incomodá-lo | (educado) obrigado pelo seu trabalho (usado ao pedir um favor ou após ter recebido um)}
  \end{phonetics}
\end{entry}

\begin{entry}{有时}{6,7}{⽉、⽇}
  \begin{phonetics}{有时}{you3 shi2}[][HSK 1]
    \definition{expr.}{às vezes; ocasionalmente; de vez em quando}
  \seealsoref{有的时候}{you3 de5 shi2 hou4}
  \seealsoref{有时候}{you3 shi2 hou5}
  \end{phonetics}
\end{entry}

\begin{entry}{有时……有时……}{6,7,6,7}{⽉、⽇、⽉、⽇}
  \begin{phonetics}{有时……有时……}{you3shi2 you3shi2}
    \definition{adv.}{às vezes\dots às vezes\dots}
  \end{phonetics}
\end{entry}

\begin{entry}{有时候}{6,7,10}{⽉、⽇、⼈}
  \begin{phonetics}{有时候}{you3 shi2 hou5}[][HSK 1]
    \definition{adv.}{às vezes; indica um momento incerto, mas não frequente}
  \seealsoref{有的时候}{you3 de5 shi2 hou4}
  \seealsoref{有时}{you3 shi2}
  \end{phonetics}
\end{entry}

\begin{entry}{有些}{6,8}{⽉、⼆}
  \begin{phonetics}{有些}{you3 xie1}[][HSK 1]
    \definition{adv.}{um pouco; bastante; ligeiramente}
    \definition{pron.}{uma parte; alguns}
    \definition{v.}{usado para indicar que há alguns, mas não muitos;}
  \seealsoref{有(一)些}{you3 (yi4) xie1}
  \end{phonetics}
\end{entry}

\begin{entry}{有的}{6,8}{⽉、⽩}
  \begin{phonetics}{有的}{you3 de5}[][HSK 1]
    \definition{pron.}{algum, alguns}
  \end{phonetics}
\end{entry}

\begin{entry}{有的时候}{6,8,7,10}{⽉、⽩、⽇、⼈}
  \begin{phonetics}{有的时候}{you3 de5 shi2 hou4}
    \definition{adv.}{às vezes; ocasionalmente}
  \seealsoref{有时}{you3 shi2}
  \seealsoref{有时候}{you3 shi2 hou5}
  \end{phonetics}
\end{entry}

\begin{entry}{有的是}{6,8,9}{⽉、⽩、⽇}
  \begin{phonetics}{有的是}{you3 de5 shi4}[][HSK 3]
    \definition{expr.}{ter em abundância; não faltar; enfatizar que há muitos}
  \end{phonetics}
\end{entry}

\begin{entry}{有空儿}{6,8,2}{⽉、⽳、⼉}
  \begin{phonetics}{有空儿}{you3 kong4r5}[][HSK 2]
    \definition{v.}{estar livre; ter tempo livre}
  \end{phonetics}
\end{entry}

\begin{entry}{有限}{6,8}{⽉、⾩}
  \begin{phonetics}{有限}{you3 xian4}[][HSK 4]
    \definition{adj.}{finito; limitado; restrito | indica baixo grau; indica pouco número; número baixo; nível baixo}
  \end{phonetics}
\end{entry}

\begin{entry}{有限公司}{6,8,4,5}{⽉、⾩、⼋、⼝}
  \begin{phonetics}{有限公司}{you3xian4gong1si1}
    \definition{s.}{companhia limitada | corporação}
  \end{phonetics}
\end{entry}

\begin{entry}{有毒}{6,9}{⽉、⽏}
  \begin{phonetics}{有毒}{you3 du2}[][HSK 5]
    \definition{adj.}{venenoso; tóxico; nocivo; geralmente é usada para descrever as propriedades nocivas à saúde de produtos químicos, plantas ou animais.}
  \end{phonetics}
\end{entry}

\begin{entry}{有点儿}{6,9,2}{⽉、⽕、⼉}
  \begin{phonetics}{有点儿}{you3 dian3r5}
    \definition{adv.}{um pouco; indica um grau inferior, equivalente a 稍微 (usado principalmente para coisas que são insatisfatórias)}
    \definition{v.}{há um pouco; tem (ou ser de) algum; existem alguns}
  \seealsoref{稍微}{shao1wei1}
  \seealsoref{有(一)点儿}{you3 yi4 dian3r5}
  \end{phonetics}
\end{entry}

\begin{entry}{有害}{6,10}{⽉、⼧}
  \begin{phonetics}{有害}{you3 hai4}[][HSK 5]
    \definition{adj.}{prejudicial; nocivo; danoso; que pode causar danos ou prejuízos a algo}
  \end{phonetics}
\end{entry}

\begin{entry}{有效}{6,10}{⽉、⽁}
  \begin{phonetics}{有效}{you3 xiao4}[][HSK 3]
    \definition{adj.}{válido; eficiente; eficaz; capaz de alcançar os objetivos esperados}
  \end{phonetics}
\end{entry}

\begin{entry}{有着}{6,11}{⽉、⽬}
  \begin{phonetics}{有着}{you3 zhe5}[][HSK 5]
    \definition{v.}{ter; possuir; haver; existir}
  \end{phonetics}
\end{entry}

\begin{entry}{有道理}{6,12,11}{⽉、⾡、⽟}
  \begin{phonetics}{有道理}{you3dao4li5}
    \definition{adj.}{razoável}
    \definition{v.}{fazer sentido}
  \end{phonetics}
\end{entry}

\begin{entry}{有意思}{6,13,9}{⽉、⼼、⼼}
  \begin{phonetics}{有意思}{you3 yi4 si5}[][HSK 2]
    \definition{adj.}{significativo; significativo e intrigante | interessante; agradável}
    \definition{v.}{ter interesse por; ser atraído sexualmente}
  \end{phonetics}
\end{entry}

\begin{entry}{有趣}{6,15}{⽉、⾛}
  \begin{phonetics}{有趣}{you3qu4}[][HSK 4]
    \definition{adj.}{interessante; fascinante; divertido; excitante; estimulante}
  \end{phonetics}
\end{entry}

\begin{entry}{朵}{6}{⽊}
  \begin{phonetics}{朵}{duo3}[][HSK 5]
    \definition*{s.}{Sobrenome Duo}
    \definition{clas.}{usado para flores, nuvens ou coisas que se assemelham a flores e nuvens}
  \end{phonetics}
\end{entry}

\begin{entry}{机}{6}{⽊}
  \begin{phonetics}{机}{ji1}
    \definition*{s.}{Sobrenome Ji}
    \definition{adj.}{flexível; perspicaz; destreza; agilidade}
    \definition[台]{s.}{máquina; motor | avião; aeronave; aeroplano; refere-se especificamente a aeronaves | ponto crucial; os fatores-chave para a ocorrência e mudança das coisas | chance; ocasião; oportunidade; um momento crítico ou oportuno para o desenvolvimento e mudança das coisas | organismo; funções vitais dos organismos | besta; mecanismo de disparo de flechas de madeira em uma besta antiga | assuntos importantes; assuntos extremamente importantes e confidenciais | ideia; intenção}
  \end{phonetics}
\end{entry}

\begin{entry}{机甲}{6,5}{⽊、⽥}
  \begin{phonetics}{机甲}{ji1jia3}
    \definition{s.}{\emph{mecha} (robôs operados pelo homem em mangá japonês)}
  \end{phonetics}
\end{entry}

\begin{entry}{机会}{6,6}{⽊、⼈}
  \begin{phonetics}{机会}{ji1hui4}[][HSK 2]
    \definition[个,次,种,些]{s.}{chance; oportunidade; momento favorável raro}
  \end{phonetics}
\end{entry}

\begin{entry}{机场}{6,6}{⽊、⼟}
  \begin{phonetics}{机场}{ji1chang3}[][HSK 1]
    \definition[个,家,处,座]{s.}{aeródromo; campo de aviação; aeroporto; campo de voo}
  \end{phonetics}
\end{entry}

\begin{entry}{机制}{6,8}{⽊、⼑}
  \begin{phonetics}{机制}{ji1 zhi4}[][HSK 5]
    \definition{s.}{mecanismo; processado por máquina; feito por máquina}
  \end{phonetics}
\end{entry}

\begin{entry}{机构}{6,8}{⽊、⽊}
  \begin{phonetics}{机构}{ji1gou4}[][HSK 4]
    \definition[所]{s.}{órgão; organização; instituição; instalações; aparelhamento; configuração | mecanismo; funcionamento interno de uma máquina ou unidade | estrutura interna de uma organização}
  \end{phonetics}
\end{entry}

\begin{entry}{机械}{6,11}{⽊、⽊}
  \begin{phonetics}{机械}{ji1xie4}
    \definition{s.}{máquina | maquinaria | mecânica}
  \end{phonetics}
\end{entry}

\begin{entry}{机票}{6,11}{⽊、⽰}
  \begin{phonetics}{机票}{ji1 piao4}[][HSK 1]
    \definition[张]{s.}{passagem aérea; passagem de avião}
  \seealsoref{飞机票}{fei1ji1 piao4}
  \end{phonetics}
\end{entry}

\begin{entry}{机遇}{6,12}{⽊、⾡}
  \begin{phonetics}{机遇}{ji1yu4}[][HSK 4]
    \definition[个]{s.}{chance; oportunidade; circunstâncias favoráveis}
  \end{phonetics}
\end{entry}

\begin{entry}{机器}{6,16}{⽊、⼝}
  \begin{phonetics}{机器}{ji1qi4}[][HSK 3]
    \definition[台,部,个]{s.}{máquina; maquinário; motor; dispositivos e máquinas que são montados a partir de peças, podem funcionar, transformar energia ou produzir trabalho útil podem ser usados como ferramentas de produção, reduzindo a intensidade do trabalho humano e aumentando a produtividade | aparato; sistema político e econômico}
  \end{phonetics}
\end{entry}

\begin{entry}{机器人}{6,16,2}{⽊、⼝、⼈}
  \begin{phonetics}{机器人}{ji1 qi4 ren2}[][HSK 5]
    \definition[个]{s.}{androide; golem | pessoa mecânica | robô}
  \end{phonetics}
\end{entry}

\begin{entry}{杀}{6}{⽊}
  \begin{phonetics}{杀}{sha1}[][HSK 5]
    \definition{adv.}{em extremo; excessivamente; usado após um verbo, indica grau intenso}
    \definition{v.}{matar; abater; esquartejar | lutar; entrar em batalha | enfraquecer; reduzir; diminuir | decolar; neutralizar}
  \end{phonetics}
\end{entry}

\begin{entry}{杀气}{6,4}{⽊、⽓}
  \begin{phonetics}{杀气}{sha1qi4}
    \definition{s.}{espírito assassino | aura de morte}
    \definition{v.}{desabafar a raiva de alguém}
  \end{phonetics}
\end{entry}

\begin{entry}{杀毒}{6,9}{⽊、⽏}
  \begin{phonetics}{杀毒}{sha1 du2}[][HSK 5]
    \definition{s.}{(computação) antivírus}
    \definition{v.}{esterilizar; desinfetar | (computação) eliminar um vírus}
  \end{phonetics}
\end{entry}

\begin{entry}{杂}{6}{⽊}
  \begin{phonetics}{杂}{za2}[][HSK 6]
    \definition{adj.}{diversos; misto; misturados | extra; irregular | variado}
    \definition{v.}{misturar}
  \end{phonetics}
\end{entry}

\begin{entry}{杂志}{6,7}{⽊、⼼}
  \begin{phonetics}{杂志}{za2zhi4}[][HSK 3]
    \definition[本,期,份,种]{s.}{jornal; revista; publicação}
  \end{phonetics}
\end{entry}

\begin{entry}{杂志社}{6,7,7}{⽊、⼼、⽰}
  \begin{phonetics}{杂志社}{za2zhi4she4}
    \definition{s.}{editora de revista}
  \end{phonetics}
\end{entry}

\begin{entry}{杂技}{6,7}{⽊、⼿}
  \begin{phonetics}{杂技}{za2ji4}
    \definition[场]{s.}{acrobacia}
  \end{phonetics}
\end{entry}

\begin{entry}{权}{6}{⽊}
  \begin{phonetics}{权}{quan2}[][HSK 6]
    \definition*{s.}{Sobrenome Quan}
    \definition{adv.}{provisoriamente; por enquanto}
    \definition{s.}{Lliterário: contrapeso; peso deslizante de uma balança romana | poder; autoridade | direito | posição vantajosa | conveniência}
    \definition{v.}{pesar; medir o peso}
  \end{phonetics}
\end{entry}

\begin{entry}{权利}{6,7}{⽊、⼑}
  \begin{phonetics}{权利}{quan2li4}[][HSK 4]
    \definition[项,种,个,条,份]{s.}{direito; interesse; os poderes e benefícios (em oposição a 义务) exercidos por um cidadão ou pessoa jurídica de acordo com a lei}
  \seealsoref{义务}{yi4wu4}
  \end{phonetics}
\end{entry}

\begin{entry}{次}{6}{⽋}
  \begin{phonetics}{次}{ci4}[][HSK 1,4]
    \definition*{s.}{Sobrenome Ci}
    \definition{adj.}{de segunda categoria; de qualidade inferior}
    \definition{clas.}{usado para coisas ou ações que podem ser repetidas}
    \definition{num.}{segundo; próximo}
    \definition{pref.}{(química) hipo-, radical ácido ou composto contendo dois átomos de oxigênio a menos}
    \definition{s.}{ordem; sequência; classificação | local de parada em uma viagem; escala}
  \end{phonetics}
\end{entry}

\begin{entry}{次数}{6,13}{⽋、⽁}
  \begin{phonetics}{次数}{ci4 shu4}[][HSK 6]
    \definition{s.}{frequência; número de vezes; o número de vezes que uma ação ou evento é repetido}
  \end{phonetics}
\end{entry}

\begin{entry}{欢}{6}{⽋}
  \begin{phonetics}{欢}{huan1}
    \definition*{s.}{Sobrenome Huan}
    \definition{adj.}{alegre; feliz; jubilante | vigoroso; energético; em pleno andamento; com grande impulso}
    \definition{s.}{amante; querida; um apelido usado por mulheres nos tempos antigos para se referir aos seus amantes; agora, geralmente se refere a alguém de quem você gosta ou com quem tem um relacionamento romântico}
  \end{phonetics}
\end{entry}

\begin{entry}{欢乐}{6,5}{⽋、⼃}
  \begin{phonetics}{欢乐}{huan1le4}[][HSK 3]
    \definition{adj.}{feliz; alegre; felicidade (geralmente coletiva)}
  \end{phonetics}
\end{entry}

\begin{entry}{欢快}{6,7}{⽋、⼼}
  \begin{phonetics}{欢快}{huan1kuai4}
    \definition{adj.}{feliz e sem ansiedade | vívido}
  \end{phonetics}
\end{entry}

\begin{entry}{欢迎}{6,7}{⽋、⾡}
  \begin{phonetics}{欢迎}{huan1ying2}[][HSK 2]
    \definition{adj.}{bem-vindo}
    \definition{v.}{dar as boas-vindas; cumprimentar; receber com alegria | dar as boas-vindas; receber favoravelmente (bem)}
  \end{phonetics}
\end{entry}

\begin{entry}{此}{6}{⽌}
  \begin{phonetics}{此}{ci3}[][HSK 4]
    \definition*{s.}{Sobrenome Ci}
    \definition{pron.}{esse; essa; isso; este; esta; isto; indica ou se refere a uma pessoa ou coisa que está mais próxima, equivalente a 这 ou 这个 (em oposição a 彼) | aqui e agora; refere-se a um tempo ou lugar recente, equivalente a 这会儿 ou 这里}
  \seealsoref{彼}{bi3}
  \seealsoref{这}{zhe4}
  \seealsoref{这会儿}{zhe4 hui4r5}
  \seealsoref{这里}{zhe4 li3}
  \seealsoref{这个}{zhe4ge5}
  \end{phonetics}
\end{entry}

\begin{entry}{此处}{6,5}{⽌、⼡}
  \begin{phonetics}{此处}{ci3 chu4}[][HSK 6]
    \definition{pron.}{este lugar; aqui (literário)}
  \end{phonetics}
\end{entry}

\begin{entry}{此外}{6,5}{⽌、⼣}
  \begin{phonetics}{此外}{ci3wai4}[][HSK 4]
    \definition{conj.}{além disso; em adição; além das coisas ou situações mencionadas acima}
  \end{phonetics}
\end{entry}

\begin{entry}{此后}{6,6}{⽌、⼝}
  \begin{phonetics}{此后}{ci3 hou4}[][HSK 5]
    \definition{s.}{daqui em diante; doravante; depois disso; após isso; de agora em diante}
  \end{phonetics}
\end{entry}

\begin{entry}{此次}{6,6}{⽌、⽋}
  \begin{phonetics}{此次}{ci3 ci4}[][HSK 6]
    \definition{adv.}{desta vez; refere-se a um ponto específico no tempo ou período de tempo}
  \end{phonetics}
\end{entry}

\begin{entry}{此时}{6,7}{⽌、⽇}
  \begin{phonetics}{此时}{ci3 shi2}[][HSK 5]
    \definition{s.}{agora; no presente; agora mesmo; neste momento; por enquanto}
  \end{phonetics}
\end{entry}

\begin{entry}{此事}{6,8}{⽌、⼅}
  \begin{phonetics}{此事}{ci3 shi4}[][HSK 6]
    \definition{s.}{matéria; assunto}
  \end{phonetics}
\end{entry}

\begin{entry}{此刻}{6,8}{⽌、⼑}
  \begin{phonetics}{此刻}{ci3 ke4}[][HSK 5]
    \definition{s.}{agora; no momento; exatamente agora; neste momento}
  \end{phonetics}
\end{entry}

\begin{entry}{此前}{6,9}{⽌、⼑}
  \begin{phonetics}{此前}{ci3 qian2}[][HSK 6]
    \definition{adv.}{literário: antes; anteriormente | literário: antes disso}
  \end{phonetics}
\end{entry}

\begin{entry}{此致}{6,10}{⽌、⾄}
  \begin{phonetics}{此致}{ci3 zhi4}[][HSK 6]
    \definition{expr.}{Atenciosamente; Sinceramente; Com os melhores votos; usada no final de uma carta ou correspondência oficial}
  \end{phonetics}
\end{entry}

\begin{entry}{死}{6}{⽍}
  \begin{phonetics}{死}{si3}[][HSK 3]
    \definition{adj.}{até a morte | implacável; mortal | fixo; rígido; inflexível | intransitável; fechado | (expressando raiva, reclamação, etc., às vezes jocosamente) maldito}
    \definition{adv.}{(frequentemente no negativo) teimosamente; inflexivelmente}
    \definition{v.}{morrer; estar morto (oposto a 生 e 活)}
  \seealsoref{活}{huo2}
  \seealsoref{生}{sheng1}
  \end{phonetics}
\end{entry}

\begin{entry}{死亡}{6,3}{⽍、⼇}
  \begin{phonetics}{死亡}{si3wang2}
    \definition{s.}{morte}
    \definition{v.}{morrer}
  \end{phonetics}
\end{entry}

\begin{entry}{毕}{6}{⽐}
  \begin{phonetics}{毕}{bi4}
    \definition*{s.}{Bi, uma das mansões lunares; a décima nona das vinte e oito constelações em que a esfera celeste foi dividida, consistindo de oito estrelas, seis em Híades e duas em Touro | Sobrenome Bi}
    \definition{adv.}{tudo; completamente; totalmente}
    \definition{v.}{terminar; realizar; concluir  | completar; terminar}
  \end{phonetics}
\end{entry}

\begin{entry}{毕业}{6,5}{⽐、⼀}
  \begin{phonetics}{毕业}{bi4ye4}[][HSK 4]
    \definition{s.}{formatura}
    \definition{v.+compl.}{formar-se}
  \end{phonetics}
\end{entry}

\begin{entry}{毕业生}{6,5,5}{⽐、⼀、⽣}
  \begin{phonetics}{毕业生}{bi4 ye4 sheng1}[][HSK 4]
    \definition[个]{s.}{diplomado; graduado; bacharel; pessoa que recebeu um diploma, grau ou certificado}
  \end{phonetics}
\end{entry}

\begin{entry}{毕竟}{6,11}{⽐、⾳}
  \begin{phonetics}{毕竟}{bi4jing4}[][HSK 5]
    \definition{adv.}{afinal de contas; quando tudo estiver dito e feito; em última análise; indica um resultado que não pode ser alterado, enfatizando que se trata de uma causa ou fato que precisa ser enfocado para referência | significa 到底, 究竟, 终究, indicando a conclusão final alcançada}
  \seealsoref{到底}{dao4di3}
  \seealsoref{究竟}{jiu1jing4}
  \seealsoref{终究}{zhong1jiu1}
  \end{phonetics}
\end{entry}

\begin{entry}{汗}{6}{⽔}
  \begin{phonetics}{汗}{han2}
    \definition*{s.}{Abreviação de Khan}[他是成吉思汗。___Ele é Genghis Khan.]
  \end{phonetics}
  \begin{phonetics}{汗}{han4}[][HSK 5]
    \definition{s.}{suor; transpiração; perspiração}
  \end{phonetics}
\end{entry}

\begin{entry}{汗水}{6,4}{⽔、⽔}
  \begin{phonetics}{汗水}{han4shui3}
    \definition*{s.}{Rio Han (Hanshui)}
    \definition{s.}{suor | transpiração}
  \end{phonetics}
\end{entry}

\begin{entry}{汗液}{6,11}{⽔、⽔}
  \begin{phonetics}{汗液}{han4ye4}
    \definition{s.}{suor}
  \end{phonetics}
\end{entry}

\begin{entry}{汗腺}{6,13}{⽔、⾁}
  \begin{phonetics}{汗腺}{han4xian4}
    \definition{s.}{glândula sudorípara}
  \end{phonetics}
\end{entry}

\begin{entry}{江}{6}{⽔}
  \begin{phonetics}{江}{jiang1}[][HSK 4]
    \definition*{s.}{Rio Changjiang | Sobrenome Jiang}
    \definition[条,道]{s.}{rio grande}
  \end{phonetics}
\end{entry}

\begin{entry}{江水}{6,4}{⽔、⽔}
  \begin{phonetics}{江水}{jiang1shui3}
    \definition{s.}{água do rio}
  \end{phonetics}
\end{entry}

\begin{entry}{江西}{6,6}{⽔、⾑}
  \begin{phonetics}{江西}{jiang1xi1}
    \definition*{s.}{Jiangxi}
  \end{phonetics}
\end{entry}

\begin{entry}{江苏}{6,7}{⽔、⾋}
  \begin{phonetics}{江苏}{jiang1su1}
    \definition*{s.}{Província de Jiangsu}
  \end{phonetics}
\end{entry}

\begin{entry}{江南水乡}{6,9,4,3}{⽔、⼗、⽔、⼄}
  \begin{phonetics}{江南水乡}{jiang1nan2shui3xiang1}
    \definition*{s.}{Vila Aquática de Jiangnan | Cidades Aquáticas}
  \end{phonetics}
\end{entry}

\begin{entry}{池}{6}{⽔}
  \begin{phonetics}{池}{chi2}
    \definition*{s.}{Sobrenome Chi}
    \definition[个,片]{s.}{piscina; lagoa | qualquer espaço fechado com laterais elevadas | baias (em um teatro); a parte frontal do salão principal do teatro | fosso}
  \end{phonetics}
\end{entry}

\begin{entry}{池子}{6,3}{⽔、⼦}
  \begin{phonetics}{池子}{chi2 zi5}[][HSK 5]
    \definition{s.}{lago; lagoa; viveiro | piscina; piscina do balneário | (antigo) arquibancada (primeiras fileiras em um teatro) | pista de dança de um salão de baile}
  \end{phonetics}
\end{entry}

\begin{entry}{污}{6}{⽔}
  \begin{phonetics}{污}{wu1}
    \definition{adj.}{sujo; imundo; imundo | corrupto}
    \definition{s.}{sujeira; imundície | esgoto; água suja; coisas sujas}
    \definition{v.}{contaminar; sujar | manchar}
  \end{phonetics}
\end{entry}

\begin{entry}{污水}{6,4}{⽔、⽔}
  \begin{phonetics}{污水}{wu1shui3}[][HSK 5]
    \definition{s.}{água suja (ou poluída, residual); esgoto; lodo | efluente; drenagem; água suja; água poluída; água residual}
  \end{phonetics}
\end{entry}

\begin{entry}{污染}{6,9}{⽔、⽊}
  \begin{phonetics}{污染}{wu1ran3}[][HSK 5]
    \definition{s.}{poluição}
    \definition{v.}{poluir; contaminar com substâncias nocivas e prejudiciais; refere-se especificamente à destruição do ambiente natural causada por substâncias nocivas, tais como gases, líquidos e resíduos emitidos por indústrias, minas, veículos, etc. | contaminar; metáfora de que pensamentos prejudiciais causam efeitos negativos nas pessoas}
  \end{phonetics}
\end{entry}

\begin{entry}{污染区}{6,9,4}{⽔、⽊、⼖}
  \begin{phonetics}{污染区}{wu1ran3qu1}
    \definition{s.}{área contaminada}
  \end{phonetics}
\end{entry}

\begin{entry}{污染物}{6,9,8}{⽔、⽊、⽜}
  \begin{phonetics}{污染物}{wu1ran3wu4}
    \definition{s.}{poluente}
  \seealsoref{污染物质}{wu1ran3 wu4zhi4}
  \end{phonetics}
\end{entry}

\begin{entry}{污染物质}{6,9,8,8}{⽔、⽊、⽜、⾙}
  \begin{phonetics}{污染物质}{wu1ran3 wu4zhi4}
    \definition{s.}{poluente}
  \seealsoref{污染物}{wu1ran3wu4}
  \end{phonetics}
\end{entry}

\begin{entry}{汤}{6}{⽔}
  \begin{phonetics}{汤}{shang1}
    \definition{s.}{correnteza forte}
  \end{phonetics}
  \begin{phonetics}{汤}{tang1}[][HSK 3]
    \definition*{s.}{Sobrenome Tang}
    \definition[勺,碗,杯,锅]{s.}{água quente; água fervente | fontes termais | água utilizada para ferver algo| sopa; caldo | uma preparação líquida de ervas medicinais; decocção}
  \end{phonetics}
\end{entry}

\begin{entry}{灯}{6}{⽕}
  \begin{phonetics}{灯}{deng1}[][HSK 2]
    \definition*{s.}{Sobrenome Deng}
    \definition[盏,个]{s.}{lâmpada; luz; lanterna; dispositivo luminoso, usado principalmente para iluminação | queimador; um aparelho que brilha e aquece como uma lâmpada e pode ser usado para aquecer | tubo; válvula; o nome popular dado aos tubos eletrônicos com formato semelhante a lâmpadas encontrados em aparelhos antigos, como rádios}
  \end{phonetics}
\end{entry}

\begin{entry}{灯丝}{6,5}{⽕、⼀}
  \begin{phonetics}{灯丝}{deng1si1}
    \definition{s.}{filamento (de uma lâmpada)}
  \end{phonetics}
\end{entry}

\begin{entry}{灯号}{6,5}{⽕、⼝}
  \begin{phonetics}{灯号}{deng1hao4}
    \definition{s.}{sinal luminoso | luz indicadora}
  \end{phonetics}
\end{entry}

\begin{entry}{灯光}{6,6}{⽕、⼉}
  \begin{phonetics}{灯光}{deng1 guang1}[][HSK 4]
    \definition{s.}{iluminação; luminosidade da lâmpada | luminação (palco); equipamento de iluminação para palco ou estúdio}
  \end{phonetics}
\end{entry}

\begin{entry}{灯泡}{6,8}{⽕、⽔}
  \begin{phonetics}{灯泡}{deng1pao4}
    \definition[个]{s.}{lâmpada | (gíria) terceiro indesejado estragando encontro de casal}
  \seealsoref{电灯泡}{dian4deng1pao4}
  \end{phonetics}
\end{entry}

\begin{entry}{灯标}{6,9}{⽕、⽊}
  \begin{phonetics}{灯标}{deng1biao1}
    \definition{s.}{farol | luz de farol}
  \end{phonetics}
\end{entry}

\begin{entry}{灰}{6}{⽕}
  \begin{phonetics}{灰}{hui1}
    \definition{adj.}{cinza (cor) | desanimado; desencorajado; deprimido}
    \definition[把,堆]{s.}{cinzas; pó que sobra após a queima de um objeto | pó; poeira; substância em pó | cal; argamassa (de cal)}
  \end{phonetics}
\end{entry}

\begin{entry}{灰色}{6,6}{⽕、⾊}
  \begin{phonetics}{灰色}{hui1 se4}[][HSK 5]
    \definition{adj.}{obscuro; ambíguo | sombrio; pessimista}
    \definition[个]{s.}{cor cinza; acinzentado}
  \end{phonetics}
\end{entry}

\begin{entry}{爷}{6}{⽗}
  \begin{phonetics}{爷}{ye2}
    \definition[个,位,名,些]{s.}{(dialeto) pai | (dialeto) avô | (uma forma respeitosa de se dirigir a um homem idoso) tio | (uma forma de se dirigir a um oficial ou homem rico) senhor; mestre; lorde; o antigo nome para burocratas, pessoas ricas, etc. | deus; forma de tratamento de um adorador para um deus}
  \end{phonetics}
\end{entry}

\begin{entry}{爷爷}{6,6}{⽗、⽗}
  \begin{phonetics}{爷爷}{ye2ye5}[][HSK 1]
    \definition[个,位]{s.}{avô (paterno)}
  \end{phonetics}
\end{entry}

\begin{entry}{百}{6}{⽩}
  \begin{phonetics}{百}{bai3}[][HSK 1]
    \definition{adj.}{todos; todos os tipos de; multifacetados; numerosos}
    \definition{adv.}{muito; sempre}
    \definition{num.}{cem; 100}
  \end{phonetics}
\end{entry}

\begin{entry}{百分}{6,4}{⽩、⼑}
  \begin{phonetics}{百分}{bai3fen1}
    \definition{s.}{por cento | nota máxima; pontuação máxima; 100 pontos (em um sistema de classificação de cem pontos) | um jogo específico; um jogo de pôquer}
  \end{phonetics}
\end{entry}

\begin{entry}{百分点}{6,4,9}{⽩、⼑、⽕}
  \begin{phonetics}{百分点}{bai3 fen1 dian3}[][HSK 6]
    \definition[个]{s.}{ponto percentual; em estatística, um por cento é chamado de ponto percentual}
  \end{phonetics}
\end{entry}

\begin{entry}{百货}{6,8}{⽩、⾙}
  \begin{phonetics}{百货}{bai3 huo4}[][HSK 4]
    \definition{s.}{mercadorias em geral; loja de departamentos; um termo geral para bens que incluem principalmente roupas, utensílios e necessidades diárias gerais}
  \end{phonetics}
\end{entry}

\begin{entry}{百般}{6,10}{⽩、⾈}
  \begin{phonetics}{百般}{bai3ban1}
    \definition{adv.}{de todas as maneiras possíveis | por todos os meios}
  \end{phonetics}
\end{entry}

\begin{entry}{竹}{6}{⽵}
  \begin{phonetics}{竹}{zhu2}
    \definition[根]{s.}{bambu | instrumento de sopro | tira de bambu}
  \end{phonetics}
\end{entry}

\begin{entry}{竹子}{6,3}{⽵、⼦}
  \begin{phonetics}{竹子}{zhu2zi5}[][HSK 5]
    \definition[块,株]{s.}{bambu; nome genérico para os tipos de bambu}
  \end{phonetics}
\end{entry}

\begin{entry}{竹马}{6,3}{⽵、⾺}
  \begin{phonetics}{竹马}{zhu2ma3}
    \definition{s.}{cavalo de bambu | vara de bambu usada como cavalo de brinquedo}
  \end{phonetics}
\end{entry}

\begin{entry}{竹排}{6,11}{⽵、⼿}
  \begin{phonetics}{竹排}{zhu2pai2}
    \definition{s.}{jangada de bambu}
  \end{phonetics}
\end{entry}

\begin{entry}{竹编}{6,12}{⽵、⽷}
  \begin{phonetics}{竹编}{zhu2bian1}
    \definition{s.}{vime | tecelagem de bambu}
  \end{phonetics}
\end{entry}

\begin{entry}{米}{6}{⽶}[Kangxi 119]
  \begin{phonetics}{米}{mi3}[][HSK 2,3]
    \definition*{s.}{Sobrenome Mi}
    \definition{clas.}{m, metro; unidade principal de comprimento do sistema métrico}
    \definition[粒,斤]{s.}{arroz | sementes descascadas; refere-se a sementes comestíveis descascadas ou sem casca | qualquer coisa que se assemelhe a um grão de arroz}
  \end{phonetics}
\end{entry}

\begin{entry}{米饭}{6,7}{⽶、⾷}
  \begin{phonetics}{米饭}{mi3fan4}[][HSK 1]
    \definition{s.}{arroz (cozido)}
  \end{phonetics}
\end{entry}

\begin{entry}{红}{6}{⽷}
  \begin{phonetics}{红}{hong2}[][HSK 2]
    \definition*{s.}{Sobrenome Hong}
    \definition{adj.}{vermelho | popular; bem-sucedido; símbolo de sucesso ou valorização | vermelho; revolucionário; símbolo da revolução | festivo; símbolo de alegria}
    \definition{s.}{tecido vermelho, bandeirinhas, etc. usados em ocasiões festivas | bônus; dividendo}
  \end{phonetics}
\end{entry}

\begin{entry}{红包}{6,5}{⽷、⼓}
  \begin{phonetics}{红包}{hong2 bao1}[][HSK 4]
    \definition[个]{s.}{saco de papel vermelho ou envelope contendo dinheiro como presente, gorjeta ou bônus | suborno; propina}
  \end{phonetics}
\end{entry}

\begin{entry}{红色}{6,6}{⽷、⾊}
  \begin{phonetics}{红色}{hong2 se4}[][HSK 2]
    \definition{adj.}{vermelho; revolucionário; símbolo da revolução ou da consciência política elevada}
    \definition{s.}{cor vermelha}
  \end{phonetics}
\end{entry}

\begin{entry}{红宝石}{6,8,5}{⽷、⼧、⽯}
  \begin{phonetics}{红宝石}{hong2bao3shi2}
    \definition{s.}{rubi}
  \end{phonetics}
\end{entry}

\begin{entry}{红线}{6,8}{⽷、⽷}
  \begin{phonetics}{红线}{hong2xian4}
    \definition{s.}{linha vermelha}
  \end{phonetics}
\end{entry}

\begin{entry}{红茶}{6,9}{⽷、⾋}
  \begin{phonetics}{红茶}{hong2 cha2}[][HSK 3]
    \definition[杯,壶,斤,种]{s.}{chá preto; chá acabado produzido através de fermentação completa}
  \end{phonetics}
\end{entry}

\begin{entry}{红烧}{6,10}{⽷、⽕}
  \begin{phonetics}{红烧}{hong2shao1}
    \definition{s.}{guisado em molho de soja (prato)}
  \end{phonetics}
\end{entry}

\begin{entry}{红酒}{6,10}{⽷、⾣}
  \begin{phonetics}{红酒}{hong2 jiu3}[][HSK 3]
    \definition[瓶,杯,壶,斤,箱]{s.}{vinho tinto}
  \end{phonetics}
\end{entry}

\begin{entry}{红绿灯}{6,11,6}{⽷、⽷、⽕}
  \begin{phonetics}{红绿灯}{hong2lv4deng1}
    \definition[个]{s.}{semáforo | sinal de trânsito}
  \end{phonetics}
\end{entry}

\begin{entry}{红薯}{6,16}{⽷、⾋}
  \begin{phonetics}{红薯}{hong2shu3}
    \definition{s.}{batata doce}
  \end{phonetics}
\end{entry}

\begin{entry}{约}{6}{⽷}
  \begin{phonetics}{约}{yao1}
    \definition{adj.}{econômico; frugal | simples; breve | indistinto}
    \definition{adv.}{cerca de; ao redor; aproximadamente}
    \definition{s.}{pacto; acordo; nomeação; coisa prometida}
    \definition{v.}{marcar uma consulta; organizar | perguntar ou convidar com antecedência | restringir; conter | reduzir (fração aproximada)}
  \end{phonetics}
  \begin{phonetics}{约}{yue1}[][HSK 3]
    \definition*{s.}{Sobrenome Yue}
    \definition{adj.}{econômico; frugal | simples; breve; resumido | indistinto; confuso}
    \definition{adv.}{cerca de; ao redor; aproximadamente}
    \definition{s.}{pacto; acordo; nomeação; o que foi combinado}
    \definition{v.}{combinar; propor ou discutir antecipadamente (o que deve ser respeitado por todos) | convidar com antecedência | restringir; conter | reduzir (fração aproximada)}
  \end{phonetics}
\end{entry}

\begin{entry}{约会}{6,6}{⽷、⼈}
  \begin{phonetics}{约会}{yue1hui4}[][HSK 4]
    \definition[个,次]{s.}{data; compromisso; engajamento; reunião pré-agendada}
    \definition{v.}{marcar uma reunião; marcar um encontro;}
  \end{phonetics}
\end{entry}

\begin{entry}{约束}{6,7}{⽷、⽊}
  \begin{phonetics}{约束}{yue1shu4}[][HSK 5]
    \definition{adj.}{amarrado}
    \definition{s.}{restrição; constrangimento; engajamento}
    \definition{v.}{amarrar; prender; reprimir; restringir; manter dentro de si}
  \end{phonetics}
\end{entry}

\begin{entry}{级}{6}{⽷}
  \begin{phonetics}{级}{ji2}[][HSK 2]
    \definition{clas.}{usado para degraus, escadas, pisos de torres, etc.}
    \definition[个,种]{s.}{nível; classificação; grau; classe | série; turma; qualquer uma das divisões anuais de um curso escolar | degrau}
  \end{phonetics}
\end{entry}

\begin{entry}{纪}{6}{⽷}
  \begin{phonetics}{纪}{ji3}
    \definition*{s.}{Sobrenome Ji}
    \definition{s.}{disciplina | um período de doze anos (na China antiga); um período de anos | (geologia) subdivisão de uma era geológica; período}
    \definition{v.}{colocar por escrito; registrar; mesmo significado de 记, usado principalmente em 记录, 纪年, 纪元, 纪传, etc. | classificar (fios de seda)}
  \seealsoref{记}{ji4}
  \seealsoref{纪传}{ji4 zhuan4}
  \seealsoref{记录}{ji4lu4}
  \seealsoref{纪年}{ji4nian2}
  \seealsoref{纪元}{ji4yuan2}
  \end{phonetics}
  \begin{phonetics}{纪}{ji4}
    \definition*{s.}{Sobrenome Ji}
    \definition{s.}{disciplina | idade; época | (geologia) período | um período de doze anos (na China antiga); um período de anos | (geologia) subdivisão de uma era geológica}
    \definition{v.}{colocar por escrito; registrar | registrar, mesmo significado de 记, usado principalmente em 记录, 纪年, 纪元, 纪传, etc. | classificar (fios de seda)}
  \seealsoref{记}{ji4}
  \seealsoref{纪传}{ji4 zhuan4}
  \seealsoref{记录}{ji4lu4}
  \seealsoref{纪年}{ji4nian2}
  \seealsoref{纪元}{ji4yuan2}
  \end{phonetics}
\end{entry}

\begin{entry}{纪元}{6,4}{⽷、⼉}
  \begin{phonetics}{纪元}{ji4yuan2}
    \definition{s.}{o início de uma era (por exemplo, o reinado de um imperador) | época; era}
  \end{phonetics}
\end{entry}

\begin{entry}{纪传}{6,6}{⽷、⼈}
  \begin{phonetics}{纪传}{ji4 zhuan4}
    \definition{s.}{crônica; biografia}
  \end{phonetics}
\end{entry}

\begin{entry}{纪传体}{6,6,7}{⽷、⼈、⼈}
  \begin{phonetics}{纪传体}{ji4 zhuan4 ti3}
    \definition{s.}{história apresentada em uma série de biografias | gênero histórico baseado em biografia}
  \end{phonetics}
\end{entry}

\begin{entry}{纪年}{6,6}{⽷、⼲}
  \begin{phonetics}{纪年}{ji4nian2}
    \definition{s.}{cronologia; uma maneira de numerar os anos | registro cronológico de eventos; anais; um dos gêneros de livros históricos é organizar fatos históricos em ordem cronológica}
  \end{phonetics}
\end{entry}

\begin{entry}{纪录}{6,8}{⽷、⼹}
  \begin{phonetics}{纪录}{ji4lu4}[][HSK 3]
    \definition[项,个]{s.}{recorde (esportes); o número mais alto ou mais baixo registrado em um determinado período de tempo}
  \end{phonetics}
\end{entry}

\begin{entry}{纪念}{6,8}{⽷、⼼}
  \begin{phonetics}{纪念}{ji4nian4}[][HSK 3]
    \definition[个,次]{s.}{lembrança; recordação; usado para representar uma lembrança (objeto)}
    \definition{v.}{comemorar; expressar saudade por pessoas ou coisas através de objetos ou ações}
  \end{phonetics}
\end{entry}

\begin{entry}{纪律}{6,9}{⽷、⼻}
  \begin{phonetics}{纪律}{ji4lv4}[][HSK 4]
    \definition{s.}{disciplina; código de conduta que cada membro da vida coletiva deve observar}
  \end{phonetics}
\end{entry}

\begin{entry}{网}{6}{⽹}[Kangxi 122]
  \begin{phonetics}{网}{wang3}[][HSK 2]
    \definition[张]{s.}{rede; um dispositivo feito de corda ou barbante para capturar peixes e pássaros | algo que parece uma rede | rede; uma rede de organizações; um sistema}
    \definition{v.}{pegar com uma rede | cobrir como com uma rede}
  \end{phonetics}
\end{entry}

\begin{entry}{网上}{6,3}{⽹、⼀}
  \begin{phonetics}{网上}{wang3 shang4}[][HSK 1]
    \definition{s.}{\emph{online}; refere-se a acessar a Internet através de um computador ou celular para pesquisar e consultar informações na rede}
  \end{phonetics}
\end{entry}

\begin{entry}{网上银行}{6,3,11,6}{⽹、⼀、⾦、⾏}
  \begin{phonetics}{网上银行}{wang3shang4yin2hang2}
    \definition[个]{s.}{banco \emph{online} | acesso a operações bancárias via \emph{Internet}}
  \seealsoref{网银}{wang3yin2}
  \end{phonetics}
\end{entry}

\begin{entry}{网友}{6,4}{⽹、⼜}
  \begin{phonetics}{网友}{wang3 you3}[][HSK 1]
    \definition{s.}{internauta; usuário da \emph{Internet}; amigos que se conhecem pela Internet; também usado como forma de tratamento entre internautas}
  \end{phonetics}
\end{entry}

\begin{entry}{网址}{6,7}{⽹、⼟}
  \begin{phonetics}{网址}{wang3 zhi3}[][HSK 4]
    \definition{s.}{\emph{website}; endereço da \emph{web}; endereço de um \emph{site} na \emph{Internet}, que os usuários podem acessar, consultar e obter recursos de informações nesse \emph{site} clicando nele}
  \end{phonetics}
\end{entry}

\begin{entry}{网际网络}{6,7,6,9}{⽹、⾩、⽹、⽷}
  \begin{phonetics}{网际网络}{wang3ji4wang3luo4}
    \definition*{s.}{Internet}
  \seealsoref{互联网}{hu4lian2wang3}
  \seealsoref{网际网路}{wang3ji4wang3lu4}
  \seealsoref{网路}{wang3lu4}
  \end{phonetics}
\end{entry}

\begin{entry}{网际网路}{6,7,6,13}{⽹、⾩、⽹、⾜}
  \begin{phonetics}{网际网路}{wang3ji4wang3lu4}
    \definition*{s.}{Internet}
  \seealsoref{互联网}{hu4lian2wang3}
  \seealsoref{网际网络}{wang3ji4wang3luo4}
  \seealsoref{网路}{wang3lu4}
  \end{phonetics}
\end{entry}

\begin{entry}{网络}{6,9}{⽹、⽷}
  \begin{phonetics}{网络}{wang3luo4}[][HSK 4]
    \definition{s.}{rede; um sistema que consiste em ramificações interconectadas; em um sistema elétrico, um circuito ou parte de um circuito que consiste em vários elementos que permitem a transmissão de sinais elétricos de acordo com determinados requisitos | rede; rede de computadores}
  \end{phonetics}
\end{entry}

\begin{entry}{网站}{6,10}{⽹、⽴}
  \begin{phonetics}{网站}{wang3zhan4}[][HSK 2]
    \definition[个,家]{s.}{\emph{web}; \emph{website}; um site virtual na Internet para uma organização ou indivíduo, geralmente consistindo em uma página inicial e muitas páginas da web}
  \end{phonetics}
\end{entry}

\begin{entry}{网罟}{6,10}{⽹、⽹}
  \begin{phonetics}{网罟}{wang3gu3}
    \definition{s.}{(fig.) a rede da justiça | rede usada para capturar peixes (ou outros animais, como pássaros)}
  \end{phonetics}
\end{entry}

\begin{entry}{网球}{6,11}{⽹、⽟}
  \begin{phonetics}{网球}{wang3qiu2}[][HSK 2]
    \definition[个,颗,些]{s.}{tênis (esporte) | bola de tênis}
  \end{phonetics}
\end{entry}

\begin{entry}{网银}{6,11}{⽹、⾦}
  \begin{phonetics}{网银}{wang3yin2}
    \definition{s.}{banco \emph{online} | acesso a operações bancárias via \emph{Internet}}
  \seealsoref{网上银行}{wang3shang4yin2hang2}
  \end{phonetics}
\end{entry}

\begin{entry}{网路}{6,13}{⽹、⾜}
  \begin{phonetics}{网路}{wang3lu4}
    \definition{s.}{\emph{Internet}}
  \seealsoref{互联网}{hu4lian2wang3}
  \seealsoref{网际网路}{wang3ji4wang3lu4}
  \seealsoref{网际网络}{wang3ji4wang3luo4}
  \end{phonetics}
\end{entry}

\begin{entry}{羊}{6}{⽺}[Kangxi 123]
  \begin{phonetics}{羊}{yang2}[][HSK 3]
    \definition*{s.}{Sobrenome Yang}
    \definition[只,头,群]{s.}{carneiro; ovelha; bode; cabra; antílope}
  \end{phonetics}
\end{entry}

\begin{entry}{羽}{6}{⽻}
  \begin{phonetics}{羽}{yu3}
    \definition*{s.}{Sobrenome Yu}
    \definition{s.}{pena; pluma | asas (de pássaros ou insetos) | uma nota da antiga escala chinesa de cinco tons, correspondente a 6 na notação musical numerada}
  \end{phonetics}
\end{entry}

\begin{entry}{羽毛}{6,4}{⽻、⽑}
  \begin{phonetics}{羽毛}{yu3mao2}
    \definition{s.}{pena | plumagem | pluma}
  \end{phonetics}
\end{entry}

\begin{entry}{羽毛笔}{6,4,10}{⽻、⽑、⽵}
  \begin{phonetics}{羽毛笔}{yu3mao2bi3}
    \definition{s.}{caneta de pena}
  \end{phonetics}
\end{entry}

\begin{entry}{羽毛球}{6,4,11}{⽻、⽑、⽟}
  \begin{phonetics}{羽毛球}{yu3mao2qiu2}[][HSK 5]
    \definition[只]{s.}{\emph{badminton}; esporte com bola, as regras e equipamentos são bastante semelhantes ao tênis | peteca}
  \end{phonetics}
\end{entry}

\begin{entry}{羽林}{6,8}{⽻、⽊}
  \begin{phonetics}{羽林}{yu3lin2}
    \definition{s.}{escolta armada}
  \end{phonetics}
\end{entry}

\begin{entry}{羽冠}{6,9}{⽻、⼍}
  \begin{phonetics}{羽冠}{yu3guan1}
    \definition{s.}{crista emplumada (de pássaro)}
  \end{phonetics}
\end{entry}

\begin{entry}{羽绒服}{6,9,8}{⽻、⽷、⽉}
  \begin{phonetics}{羽绒服}{yu3rong2fu2}[][HSK 5]
    \definition[件]{s.}{jaqueta de plumas; peça de vestuário com enchimento de plumas; casaco cujo interior é preenchido com penas de pato ou ganso}
  \end{phonetics}
\end{entry}

\begin{entry}{羽流}{6,10}{⽻、⽔}
  \begin{phonetics}{羽流}{yu3liu2}
    \definition{s.}{pluma}
  \end{phonetics}
\end{entry}

\begin{entry}{老}{6}{⽼}[Kangxi 125]
  \begin{phonetics}{老}{lao3}[][HSK 1,2]
    \definition*{s.}{Sobrenome Lao}
    \definition{adj.}{velho; envelhecido; idade avançada | antigo; de longa data; existe há muito tempo | antigo; desatualizado; obsoleto; ultrapassado  | antigo; tradicional; original | coberto de vegetação; os vegetais cresceram além do período ideal para serem consumidos | resistente; endurecido; alimentos muito cozidos | escuro; profundo; (sobre cores) | último nascido; o mais novo | veterano; experiente; sofisticado}
    \definition{adv.}{longo; por muito tempo | sempre (fazendo algo) | muito}
    \definition{pref.}{usado para designar pessoas, ordem de classificação, certos nomes de animais e plantas}
    \definition{s.}{idosos; pessoas mais velhas | ancião; sênior; um título respeitoso para pessoas mais velhas}
    \definition{v.}{envelhecer | morrer; referindo-se à morte de um idoso}
  \end{phonetics}
\end{entry}

\begin{entry}{老人}{6,2}{⽼、⼈}
  \begin{phonetics}{老人}{lao3 ren2}[][HSK 1]
    \definition[位]{s.}{homem ou mulher de idade avançada; o idoso; o velho}
  \end{phonetics}
\end{entry}

\begin{entry}{老人家}{6,2,10}{⽼、⼈、⼧}
  \begin{phonetics}{老人家}{lao3 ren2 jia1}
    \definition{s.}{senhor ancião | madame | senhora | termo educado para mulher ou homem velho}
  \end{phonetics}
\end{entry}

\begin{entry}{老公}{6,4}{⽼、⼋}
  \begin{phonetics}{老公}{lao3 gong1}[][HSK 4]
    \definition[个]{s.}{marido; esposo}
  \end{phonetics}
\end{entry}

\begin{entry}{老太太}{6,4,4}{⽼、⼤、⼤}
  \begin{phonetics}{老太太}{lao3 tai4 tai5}[][HSK 3]
    \definition[位,名,个]{s.}{velha senhora; (em tratamento direto)Venerável Senhora; uma maneira respeitosa de chamar uma senhora idosa; título honorífico para mulheres idosas | (forma de tratamento) sua velha mãe; minha velha mãe, avó ou sogra; referindo-se à própria mãe, à mãe do outro ou à mãe de outra pessoa, à sogra ou à sogra política}
  \end{phonetics}
\end{entry}

\begin{entry}{老头儿}{6,5,2}{⽼、⼤、⼉}
  \begin{phonetics}{老头儿}{lao3 tou2r5}[][HSK 3]
    \definition{s.}{(coloquial) (com um tom de intimidade) velho; velho amigo}
  \seealsoref{老头子}{lao3 tou2zi5}
  \end{phonetics}
\end{entry}

\begin{entry}{老头子}{6,5,3}{⽼、⼤、⼦}
  \begin{phonetics}{老头子}{lao3 tou2zi5}
    \definition{s.}{velho antiquado (ou velho rabugento) | (referindo-se ao marido idoso) meu velho | chefe de uma sociedade secreta | (coloquial) velho; velho rabugento}
  \seealsoref{老头儿}{lao3 tou2r5}
  \end{phonetics}
\end{entry}

\begin{entry}{老师}{6,6}{⽼、⼱}
  \begin{phonetics}{老师}{lao3shi1}[][HSK 1]
    \definition[个,位]{s.}{professor; título honorífico para professores; refere-se, de maneira geral, a pessoas que transmitem cultura e tecnologia ou que são dignas de admiração em termos de ideias, moralidade e conhecimentos profissionais}
  \end{phonetics}
\end{entry}

\begin{entry}{老年}{6,6}{⽼、⼲}
  \begin{phonetics}{老年}{lao3 nian2}[][HSK 2]
    \definition[个]{s.}{idoso; velhice; idade acima de 60 ou 70 anos}
  \end{phonetics}
\end{entry}

\begin{entry}{老百姓}{6,6,8}{⽼、⽩、⼥}
  \begin{phonetics}{老百姓}{lao3bai3xing4}[][HSK 3]
    \definition[些]{s.}{povo; civis; pessoas comuns; residentes (em contraste com militares e funcionários públicos)}
  \end{phonetics}
\end{entry}

\begin{entry}{老兵}{6,7}{⽼、⼋}
  \begin{phonetics}{老兵}{lao3bing1}
    \definition{s.}{velho soldado | veterano de guerra | veterano (alguém que tem muita experiência em algum domínio)}
  \end{phonetics}
\end{entry}

\begin{entry}{老实}{6,8}{⽼、⼧}
  \begin{phonetics}{老实}{lao3shi5}[][HSK 4]
    \definition{adj.}{franco; sincero; honesto | bom; bem-comportado | ingênuo; simplório; meio bobo; facilmente enganado; eufemismo para pouco inteligente}
  \end{phonetics}
\end{entry}

\begin{entry}{老朋友}{6,8,4}{⽼、⽉、⼜}
  \begin{phonetics}{老朋友}{lao3 peng2 you3}[][HSK 2]
    \definition[个,位,名]{s.}{velho amigo; refere-se a amigos que conhecemos há muito tempo e com quem temos uma relação íntima}
  \end{phonetics}
\end{entry}

\begin{entry}{老板}{6,8}{⽼、⽊}
  \begin{phonetics}{老板}{lao3ban3}[][HSK 3]
    \definition[个,位]{s.}{chefe; dono; líder; refere-se ao gerente de uma empresa comercial ou industrial | antigo título honorífico dado a atores famosos de ópera ou atores que também eram diretores de companhias de ópera}
  \end{phonetics}
\end{entry}

\begin{entry}{老虎}{6,8}{⽼、⾌}
  \begin{phonetics}{老虎}{lao3hu3}
    \definition[只]{s.}{tigre}
  \seealsoref{虎}{hu3}
  \end{phonetics}
\end{entry}

\begin{entry}{老是}{6,9}{⽼、⽇}
  \begin{phonetics}{老是}{lao3 shi4}[][HSK 2]
    \definition{adv.}{sempre; indica que a ação continua ou que o estado permanece inalterado, equivalente a 一直}
  \seealsoref{一直}{yi4zhi2}
  \end{phonetics}
\end{entry}

\begin{entry}{老家}{6,10}{⽼、⼧}
  \begin{phonetics}{老家}{lao3 jia1}[][HSK 4]
    \definition{s.}{cidade natal; local de origem | lugar nativo; refere-se às gerações anteriores da família ou ao local onde a pessoa nasceu ou viveu}
  \end{phonetics}
\end{entry}

\begin{entry}{老婆}{6,11}{⽼、⼥}
  \begin{phonetics}{老婆}{lao3po2}[][HSK 4]
    \definition[个]{s.}{esposa}
  \end{phonetics}
\end{entry}

\begin{entry}{考}{6}{⽼}
  \begin{phonetics}{考}{kao3}[][HSK 1]
    \definition*{s.}{Sobrenome Kao}
    \definition{adj.}{antigo; velho; com idade avançada}
    \definition{s.}{o pai falecido de alguém}
    \definition{v.}{examinar; dar (fazer) um exame, teste ou questionário | verificar; inspecionar | estudar; verificar; investigar | perguntar; testar; fazer perguntas para que o outro responda, a fim de testar suas habilidades em determinada área}
  \end{phonetics}
\end{entry}

\begin{entry}{考生}{6,5}{⽼、⽣}
  \begin{phonetics}{考生}{kao3 sheng1}[][HSK 2]
    \definition{s.}{candidato a exame; alunos inscritos para o exame de admissão}
  \end{phonetics}
\end{entry}

\begin{entry}{考试}{6,8}{⽼、⾔}
  \begin{phonetics}{考试}{kao3shi4}[][HSK 1]
    \definition[次]{s.}{teste; exame; prova; atividades realizadas para verificar conhecimentos ou habilidades}
    \definition{v.+compl.}{testar; avaliar; avaliar conhecimentos e habilidades por meio de perguntas escritas ou orais.}
  \end{phonetics}
\end{entry}

\begin{entry}{考核}{6,10}{⽼、⽊}
  \begin{phonetics}{考核}{kao3he2}[][HSK 5]
    \definition{v.}{examinar; checar; avaliar; avaliar (a proficiência de alguém)}
  \end{phonetics}
\end{entry}

\begin{entry}{考虑}{6,10}{⽼、⾌}
  \begin{phonetics}{考虑}{kao3lv4}[][HSK 4]
    \definition{v.}{considerar; refletir sobre; levar em conta}
  \end{phonetics}
\end{entry}

\begin{entry}{考验}{6,10}{⽼、⾺}
  \begin{phonetics}{考验}{kao3yan4}[][HSK 3]
    \definition[场,个,种]{s.}{teste; julgamento; atividade realizada para verificar se as habilidades, ideias, moral e qualidades de uma pessoa atendem aos requisitos}
    \definition{v.}{testar; testar as capacidades, ideias, moral e qualidades de uma pessoa através de situações, ações ou ambientes difíceis, para verificar se elas atendem aos requisitos}
  \end{phonetics}
\end{entry}

\begin{entry}{考察}{6,14}{⽼、⼧}
  \begin{phonetics}{考察}{kao3cha2}[][HSK 4]
    \definition{v.}{inspecionar; investigar; observar e estudar}
  \end{phonetics}
\end{entry}

\begin{entry}{而}{6}{⽽}[Kangxi 126]
  \begin{phonetics}{而}{er2}[][HSK 4]
    \definition{conj.}{e (coordenação) | e ainda (restrição) | conexão de componentes com continuidade semântica | conecxão de componentes afirmativos e negativos que se complementam | conexão de componentes com significados opostos para indicar um contraste |  conexão de componentes de causa e efeito no raciocínio | significa “chegar” ou “alcançar” | conexão de componentes que indicam tempo ou modo ao verbo | inserido entre o sujeito e o predicado, significa 如果}
  \seealsoref{如果}{ru2guo3}
  \end{phonetics}
\end{entry}

\begin{entry}{而且}{6,5}{⽽、⼀}
  \begin{phonetics}{而且}{er2 qie3}[][HSK 2]
    \definition{conj.}{e também; indica igualdade | e isso; não só\dots mas (também); indica um passo adiante}
  \end{phonetics}
\end{entry}

\begin{entry}{而况}{6,7}{⽽、⼎}
  \begin{phonetics}{而况}{er2kuang4}
    \definition{conj.}{além disso | além do mais}
  \end{phonetics}
\end{entry}

\begin{entry}{而是}{6,9}{⽽、⽇}
  \begin{phonetics}{而是}{er2 shi4}[][HSK 4]
    \definition{conj.}{mas; em vez disso; geralmente usada em conjunto com 不是 para formar o correlativo 不是……而是, indicando uma relação paralela}
  \seealsoref{不是……而是}{bu4shi4 er2 shi4}
  \end{phonetics}
\end{entry}

\begin{entry}{耳}{6}{⽿}
  \begin{phonetics}{耳}{er3}
    \definition*{s.}{Sobrenome Er}
    \definition{part.}{(clássico) somente; apenas}
    \definition{s.}{orelha | coisa parecida com uma orelha | em ambos os lados; lado | orelha de um utensílio}
  \end{phonetics}
\end{entry}

\begin{entry}{耳朵}{6,6}{⽿、⽊}
  \begin{phonetics}{耳朵}{er3duo5}[][HSK 5]
    \definition[双,只,个,对]{s.}{orelha; ouvido; órgão da audição e do equilíbrio}
  \end{phonetics}
\end{entry}

\begin{entry}{耳机}{6,6}{⽿、⽊}
  \begin{phonetics}{耳机}{er3 ji1}[][HSK 4]
    \definition[副,个,对]{s.}{fone de ouvido; receptor (de telefone); dispositivos que permitem que uma pessoa ouça sons sozinha, como ouvir música, histórias, chamadas telefônicas etc., usados na cabeça ou inseridos nos ouvidos}
  \end{phonetics}
\end{entry}

\begin{entry}{肉}{6}{⾁}[Kangxi 130]
  \begin{phonetics}{肉}{rou4}[][HSK 1]
    \definition{adj.}{não crocante; mole | lento (em movimento); preguiçoso | carnal; erótico}
    \definition[块]{s.}{carne (especialmente carne de porco) | carne | polpa (da fruta)}
  \end{phonetics}
\end{entry}

\begin{entry}{肉桂}{6,10}{⾁、⽊}
  \begin{phonetics}{肉桂}{rou4gui4}
    \definition{s.}{canela}
  \seealsoref{官桂}{guan1gui4}
  \end{phonetics}
\end{entry}

\begin{entry}{肌}{6}{⾁}
  \begin{phonetics}{肌}{ji1}
    \definition[块,片]{s.}{músculo; carne | pele;}
  \end{phonetics}
\end{entry}

\begin{entry}{肌肉}{6,6}{⾁、⾁}
  \begin{phonetics}{肌肉}{ji1rou4}[][HSK 5]
    \definition[身,块]{s.}{músculo; um dos tecidos básicos dos músculos humanos e de alguns animais, composto principalmente de células musculares fibrosas, pode se contrair, é o movimento do corpo e o corpo de digestão, respiração, circulação, excreção e outros processos fisiológicos da fonte de energia; pode ser dividido em três tipos: músculo liso, músculo esquelético e músculo cardíaco}
  \end{phonetics}
\end{entry}

\begin{entry}{自}{6}{⾃}[Kangxi 132]
  \begin{phonetics}{自}{zi4}[][HSK 4]
    \definition*{s.}{Sobrenome Zi}
    \definition{adv.}{certamente; com certeza; é claro; naturalmente}
    \definition{prep.}{de; desde; a partir de; apresenta o ponto de partida, a fonte ou o horário de início do comportamento da ação, equivalente a 从 e 由}
    \definition{pron.}{si mesmo; próprio | próprio; indica que a ação é iniciada por e direcionada a si mesmo | por si mesmo; indica que a ação é autoiniciada e não é causada por uma força externa}
    \definition{v.}{iniciar}
  \seealsoref{从}{cong2}
  \seealsoref{由}{you2}
  \end{phonetics}
\end{entry}

\begin{entry}{自个儿}{6,3,2}{⾃、⼈、⼉}
  \begin{phonetics}{自个儿}{zi4ge3r5}
    \definition{pron.}{(dialeto) a si mesmo, por si mesmo}
  \end{phonetics}
\end{entry}

\begin{entry}{自己}{6,3}{⾃、⼰}
  \begin{phonetics}{自己}{zi4ji3}[][HSK 2]
    \definition{pron.}{a si próprio; a si mesmo; refere-se ao substantivo ou pronome precedente (enfatiza principalmente que não é devido a forças externas)}
  \end{phonetics}
\end{entry}

\begin{entry}{自己动手}{6,3,6,4}{⾃、⼰、⼒、⼿}
  \begin{phonetics}{自己动手}{zi4ji3dong4shou3}
    \definition{v.}{fazer (algo) sozinho | ajudar-se a}
  \end{phonetics}
\end{entry}

\begin{entry}{自从}{6,4}{⾃、⼈}
  \begin{phonetics}{自从}{zi4cong2}[][HSK 3]
    \definition{prep.}{de; desde; a partir de; referir-se a um momento ou evento específico no passado}
  \end{phonetics}
\end{entry}

\begin{entry}{自主}{6,5}{⾃、⼂}
  \begin{phonetics}{自主}{zi4zhu3}[][HSK 3]
    \definition{v.}{agir por conta própria; decidir por si mesmo; manter a iniciativa em suas próprias mãos; tomar suas próprias decisões}
  \end{phonetics}
\end{entry}

\begin{entry}{自由}{6,5}{⾃、⽥}
  \begin{phonetics}{自由}{zi4you2}[][HSK 2]
    \definition{adj.}{livre; irrestrito}
    \definition[个]{s.}{liberdade; o direito de agir de acordo com a própria vontade dentro do âmbito da lei | liberdade; filosoficamente, liberdade é definida como o processo de as pessoas reconhecerem as leis que governam o desenvolvimento das coisas e aplicá-las conscientemente na prática}
  \end{phonetics}
\end{entry}

\begin{entry}{自由泳}{6,5,8}{⾃、⽥、⽔}
  \begin{phonetics}{自由泳}{zi4you2yong3}
    \definition{s.}{natação de estilo livre}
  \end{phonetics}
\end{entry}

\begin{entry}{自动}{6,6}{⾃、⼒}
  \begin{phonetics}{自动}{zi4dong4}[][HSK 3]
    \definition{adj.}{automático; auto-atuante; uso de dispositivos mecânicos, elétricos, etc, para funcionar automaticamente, sem necessidade de controle humano}
    \definition{adv.}{voluntariamente; por vontade própria; por iniciativa própria | automaticamente; espontaneamente; refere-se a movimentos, mudanças, etc., que não são causados pela ação humana, mas sim pelo próprio objeto}
  \end{phonetics}
\end{entry}

\begin{entry}{自动化}{6,6,4}{⾃、⼒、⼔}
  \begin{phonetics}{自动化}{zi4dong4hua4}
    \definition{s.}{automação}
  \end{phonetics}
\end{entry}

\begin{entry}{自杀}{6,6}{⾃、⽊}
  \begin{phonetics}{自杀}{zi4 sha1}[][HSK 5]
    \definition{s.}{suicídio; auto-assassinato; auto-sacrifício}
    \definition{v.}{cometer suicídio; tentar suicídio; matar-se}
  \end{phonetics}
\end{entry}

\begin{entry}{自行车}{6,6,4}{⾃、⾏、⾞}
  \begin{phonetics}{自行车}{zi4xing2che1}[][HSK 2]
    \definition[辆]{s.}{bicicleta; um veículo de duas rodas que é impulsionado para a frente com os pedais}
  \end{phonetics}
\end{entry}

\begin{entry}{自行车架}{6,6,4,9}{⾃、⾏、⾞、⽊}
  \begin{phonetics}{自行车架}{zi4xing2che1jia4}
    \definition{s.}{suporte para bicicleta | bicicletário}
  \end{phonetics}
\end{entry}

\begin{entry}{自行车馆}{6,6,4,11}{⾃、⾏、⾞、⾷}
  \begin{phonetics}{自行车馆}{zi4xing2che1guan3}
    \definition{s.}{estádio de ciclismo | velódromo}
  \end{phonetics}
\end{entry}

\begin{entry}{自行车赛}{6,6,4,14}{⾃、⾏、⾞、⾙}
  \begin{phonetics}{自行车赛}{zi4xing2che1sai4}
    \definition{s.}{corrida de bicicleta}
  \end{phonetics}
\end{entry}

\begin{entry}{自我}{6,7}{⾃、⼽}
  \begin{phonetics}{自我}{zi4wo3}
    \definition{pref.}{auto}
    \definition{pron.}{a si mesmo | eu próprio | (psicologia) ego}
  \end{phonetics}
\end{entry}

\begin{entry}{自我介绍}{6,7,4,8}{⾃、⼽、⼈、⽷}
  \begin{phonetics}{自我介绍}{zi4wo3jie4shao4}
    \definition{s.}{defesa pessoal | auto-defesa}
  \end{phonetics}
\end{entry}

\begin{entry}{自我安慰}{6,7,6,15}{⾃、⼽、⼧、⼼}
  \begin{phonetics}{自我安慰}{zi4wo3'an1wei4}
    \definition{v.}{confortar-se | consolar-se | tranquilizar-se}
  \end{phonetics}
\end{entry}

\begin{entry}{自我防卫}{6,7,6,3}{⾃、⼽、⾩、⼙}
  \begin{phonetics}{自我防卫}{zi4wo3fang2wei4}
    \definition{s.}{defesa pessoal | auto-defesa}
  \end{phonetics}
\end{entry}

\begin{entry}{自我吹嘘}{6,7,7,14}{⾃、⼽、⼝、⼝}
  \begin{phonetics}{自我吹嘘}{zi4wo3chui1xu1}
    \definition{expr.}{tocar a própria buzina}
  \end{phonetics}
\end{entry}

\begin{entry}{自我批评}{6,7,7,7}{⾃、⼽、⼿、⾔}
  \begin{phonetics}{自我批评}{zi4wo3pi1ping2}
    \definition{s.}{autocrítica}
  \end{phonetics}
\end{entry}

\begin{entry}{自我实现}{6,7,8,8}{⾃、⼽、⼧、⾒}
  \begin{phonetics}{自我实现}{zi4wo3shi2xian4}
    \definition{s.}{(psicologia) auto-realização}
  \end{phonetics}
\end{entry}

\begin{entry}{自我的人}{6,7,8,2}{⾃、⼽、⽩、⼈}
  \begin{phonetics}{自我的人}{zi4wo3de5ren2}
    \definition{s.}{(minha, sua) própria pessoa | (afirmar) a própria personalidade}
  \end{phonetics}
\end{entry}

\begin{entry}{自我保存}{6,7,9,6}{⾃、⼽、⼈、⼦}
  \begin{phonetics}{自我保存}{zi4wo3 bao3cun2}
    \definition{v.}{autopreservação}
  \end{phonetics}
\end{entry}

\begin{entry}{自我陶醉}{6,7,10,15}{⾃、⼽、⾩、⾣}
  \begin{phonetics}{自我陶醉}{zi4wo3tao2zui4}
    \definition{s.}{narcisista | auto-imbuído | satisfeito consigo mesmo}
  \end{phonetics}
\end{entry}

\begin{entry}{自我催眠}{6,7,13,10}{⾃、⼽、⼈、⽬}
  \begin{phonetics}{自我催眠}{zi4wo3cui1mian2}
    \definition{v.}{consolar-me | tranquilizar-me}
  \end{phonetics}
\end{entry}

\begin{entry}{自我意识}{6,7,13,7}{⾃、⼽、⼼、⾔}
  \begin{phonetics}{自我意识}{zi4wo3yi4shi2}
    \definition{s.}{autoapresentação}
    \definition{v.}{apresentar-se}
  \end{phonetics}
\end{entry}

\begin{entry}{自我解嘲}{6,7,13,15}{⾃、⼽、⾓、⼝}
  \begin{phonetics}{自我解嘲}{zi4wo3jie3chao2}
    \definition{s.}{referir-se às próprias fraquezas ou falhas com humor autodepreciativo}
  \end{phonetics}
\end{entry}

\begin{entry}{自来水}{6,7,4}{⾃、⽊、⽔}
  \begin{phonetics}{自来水}{zi4lai2shui3}
    \definition{s.}{água corrente | água da torneira}
  \end{phonetics}
\end{entry}

\begin{entry}{自身}{6,7}{⾃、⾝}
  \begin{phonetics}{自身}{zi4 shen1}[][HSK 3]
    \definition{pron.}{eu mesmo (enfatizando que não é outra pessoa ou outra coisa)}
  \end{phonetics}
\end{entry}

\begin{entry}{自责}{6,8}{⾃、⾙}
  \begin{phonetics}{自责}{zi4ze2}
    \definition{v.}{culpar-se}
  \end{phonetics}
\end{entry}

\begin{entry}{自信}{6,9}{⾃、⼈}
  \begin{phonetics}{自信}{zi4xin4}[][HSK 4]
    \definition{adj.}{confiante; descreve a crença em suas próprias habilidades, decisões, etc., tendo confiança em si mesmo}
    \definition[份,种]{s.}{autoconfiança; confiança em si mesmo}
    \definition{v.}{acreditar em si mesmo;}
  \end{phonetics}
\end{entry}

\begin{entry}{自觉}{6,9}{⾃、⾒}
  \begin{phonetics}{自觉}{zi4jue2}[][HSK 3]
    \definition{adj.}{autoconsciente; de ​​livre e espontânea vontade; controlar o próprio comportamento e agir por iniciativa própria}
    \definition{v.}{estar ciente de}
  \end{phonetics}
\end{entry}

\begin{entry}{自救}{6,11}{⾃、⽁}
  \begin{phonetics}{自救}{zi4jiu4}
    \definition{v.}{sair a si mesmo de problemas}
  \end{phonetics}
\end{entry}

\begin{entry}{自然}{6,12}{⾃、⽕}
  \begin{phonetics}{自然}{zi4ran2}[][HSK 3]
    \definition{adj.}{natural; no curso normal dos eventos; formado ou desenvolvido sem intervenção humana; algo que se desenvolve livremente}
    \definition{adv.}{naturalmente; definitivamente; certamente, isso significa que, de acordo com a lógica, deve ser assim}
    \definition{conj.}{usado para ligar duas frases, com a segunda introduzindo informações adicionais ou adversativas; indica explicação complementar ou uma mudança de significado}
    \definition{s.}{natureza; mundo natural; tudo o que não foi criado pelo ser humano}
  \end{phonetics}
\end{entry}

\begin{entry}{自愿}{6,14}{⾃、⽕}
  \begin{phonetics}{自愿}{zi4yuan4}[][HSK 5]
    \definition{adv.}{voluntariamente; por iniciativa própria; por vontade própria}
    \definition{s.}{voluntário}
  \end{phonetics}
\end{entry}

\begin{entry}{自豪}{6,14}{⾃、⾗}
  \begin{phonetics}{自豪}{zi4hao2}[][HSK 5]
    \definition{adj.}{orgulhar-se de; ter orgulho de; sentir-se honrado por possuir qualidades excelentes ou ter alcançado grandes conquistas, seja por si mesmo ou por um grupo ou indivíduo relacionado a si}
  \end{phonetics}
\end{entry}

\begin{entry}{自燃}{6,16}{⾃、⽕}
  \begin{phonetics}{自燃}{zi4ran2}
    \definition{s.}{combustão espontânea}
  \end{phonetics}
\end{entry}

\begin{entry}{至}{6}{⾄}
  \begin{phonetics}{至}{zhi4}[][HSK 5]
    \definition{adv.}{a maior parte; extremamente; indica o grau mais alto, equivalente a 极 ou 最}
    \definition{prep.}{para; até; chegar a um determinado ponto}
    \definition{s.}{extremo, máximo}
    \definition{v.}{chegar; alcançar}
  \seealsoref{极}{ji2}
  \seealsoref{最}{zui4}
  \end{phonetics}
\end{entry}

\begin{entry}{至于}{6,3}{⾄、⼆}
  \begin{phonetics}{至于}{zhi4yu2}
    \definition{conj.}{para | quanto a | a respeiro de}
  \end{phonetics}
\end{entry}

\begin{entry}{至今}{6,4}{⾄、⼈}
  \begin{phonetics}{至今}{zhi4jin1}[][HSK 3]
    \definition{adv.}{até agora; até o momento; até hoje}
  \end{phonetics}
\end{entry}

\begin{entry}{至少}{6,4}{⾄、⼩}
  \begin{phonetics}{至少}{zhi4shao3}[][HSK 3]
    \definition{adv.}{pelo menos; indica o limite mínimo}
  \end{phonetics}
\end{entry}

\begin{entry}{舌}{6}{⾆}[Kangxi 135]
  \begin{phonetics}{舌}{she2}
    \definition*{s.}{Sobrenome She}
    \definition[片,条]{s.}{língua (de um ser humano ou animal); glossa | algo em forma de língua | língua de sino; badalo}
  \end{phonetics}
\end{entry}

\begin{entry}{舌头}{6,5}{⾆、⼤}
  \begin{phonetics}{舌头}{she2tou5}
    \definition[个]{s.}{língua | soldado inimigo capturado com o propósito de extrair informações}
  \end{phonetics}
\end{entry}

\begin{entry}{色}{6}{⾊}[Kangxi 139]
  \begin{phonetics}{色}{se4}[][HSK 4]
    \definition[种]{s.}{cor | aparência; semblante; expressão | tipo; gênero; descrição | cena; cenário;  paisagem | qualidade (de metais preciosos, mercadorias, etc.) | aparência feminina; beleza feminina}
  \end{phonetics}
  \begin{phonetics}{色}{shai3}
    \definition[4]{s.}{cor}
  \end{phonetics}
\end{entry}

\begin{entry}{色狼}{6,10}{⾊、⽝}
  \begin{phonetics}{色狼}{se4lang2}
    \definition*{s.}{Sátiro}
    \definition{adj.}{depravado | tarado}
  \end{phonetics}
\end{entry}

\begin{entry}{色彩}{6,11}{⾊、⼺}
  \begin{phonetics}{色彩}{se4cai3}[][HSK 4]
    \definition[种,丝]{s.}{cor; matiz; tonalidade | cor; sabor; característica; metáfora para um determinado estado de espírito ou tendência de pensamento}
  \end{phonetics}
\end{entry}

\begin{entry}{芋}{6}{⾋}
  \begin{phonetics}{芋}{yu4}
    \definition*{s.}{Sobrenome Yu}
    \definition{s.}{taro; erva perene | tubérculos; geralmente se refere a batatas, etc.}
  \end{phonetics}
\end{entry}

\begin{entry}{芋头}{6,5}{⾋、⼤}
  \begin{phonetics}{芋头}{yu4tou5}
    \definition{s.}{taro, similar ao inhame e batata doce}
  \end{phonetics}
\end{entry}

\begin{entry}{芋头色}{6,5,6}{⾋、⼤、⾊}
  \begin{phonetics}{芋头色}{yu4tou5se4}
    \definition{s.}{cor lilás}
  \end{phonetics}
\end{entry}

\begin{entry}{芝}{6}{⾋}
  \begin{phonetics}{芝}{zhi1}
    \definition*{s.}{Sobrenome Zhi}
    \definition{s.}{Arcaico: fungo mágico, ganoderma brilhante | Arcaico: raiz de angélica dahuriana}
  \end{phonetics}
\end{entry}

\begin{entry}{芝麻}{6,11}{⾋、⿇}
  \begin{phonetics}{芝麻}{zhi1ma5}
    \definition{s.}{semente de gergelim}
  \end{phonetics}
\end{entry}

\begin{entry}{虫}{6}{⾍}[Kangxi 142]
  \begin{phonetics}{虫}{chong2}
    \definition[只,条]{s.}{inseto; verme | (pejorativo) pessoas que se comportam de forma desprezível | fã; viciado | forma inferior de vida animal, incluindo insetos, larvas de insetos, vermes e criaturas semelhantes | pessoa com uma característica indesejável específica}
  \end{phonetics}
\end{entry}

\begin{entry}{虫子}{6,3}{⾍、⼦}
  \begin{phonetics}{虫子}{chong2 zi5}[][HSK 4]
    \definition[条,只,种]{s.}{percevejo; besouro; inseto; verme; criaturas semelhantes a insetos}
  \end{phonetics}
\end{entry}

\begin{entry}{血}{6}{⾎}[Kangxi 143]
  \begin{phonetics}{血}{xie3}
  \end{phonetics}
  \begin{phonetics}{血}{xue4}[][HSK 3]
    \definition[滴,袋,口,毫升]{s.}{sangue | parente consanguíneo; com laços de parentesco | pessoa ativa e animada; metáfora para uma personalidade ou espírito forte e sincero | medicina tradicional chinesa refere-se à menstruação}
  \end{phonetics}
\end{entry}

\begin{entry}{血汗}{6,6}{⾎、⽔}
  \begin{phonetics}{血汗}{xue4han4}
    \definition{s.}{(fig.) suor e labuta, trabalho duro}
  \end{phonetics}
\end{entry}

\begin{entry}{行}{6}{⾏}
  \begin{phonetics}{行}{hang2}[][HSK 3]
    \definition{adj.}{temporário; improvisado | capaz; competente}
    \definition{adv.}{logo; em breve}
    \definition{clas.}{linha; fileira; coisas usadas para formar filas, linhas}
    \definition{s.}{comportamento; conduta | linha; fileira | empresa comercial; certas instituições comerciais | comércio; profissão; ramo de atividade | especialista; conhecedor; refere-se ao conhecimento e experiência em um determinado setor}
    \definition{v.}{ir; caminhar; viajar | estar atualizado; circular | fazer; executar; realizar | (antes de um verbo dissílabo, indicando a realização de alguma ação) | ficar bem; vai dar certo | (remédio) fazer efeito | classificar (entre irmãos e irmãs por ordem de idade)}
  \end{phonetics}
  \begin{phonetics}{行}{heng2}
    \definition{s.}{usado em 道行}
  \seealsoref{道行}{dao4 heng2}
  \end{phonetics}
  \begin{phonetics}{行}{xing2}[][HSK 1]
    \definition*{s.}{Sobrenome Xing}
    \definition{adj.}{de viajar; relacionado a viagens | temporário; improvisado; provisório | capaz; competente}
    \definition{adv.}{em breve}
    \definition{s.}{comportamento; conduta | caligrafia cursiva (na caligrafia chinesa); escrita cursiva}
    \definition{v.}{ir | fazer uma viagem | estar em voga; prevalecer; circular | fazer; executar; realizar; envolver-se em | estar tudo bem; O.K. | indica a realização de uma determinada atividade (usado principalmente antes de verbos dissilábicos) | (em medicina) fazer efeito}
  \end{phonetics}
\end{entry}

\begin{entry}{行人}{6,2}{⾏、⼈}
  \begin{phonetics}{行人}{xing2ren2}[][HSK 2]
    \definition[个]{s.}{pedestre; transeunte; viajante à pé; pessoas caminhando na estrada}
  \end{phonetics}
\end{entry}

\begin{entry}{行为}{6,4}{⾏、⼂}
  \begin{phonetics}{行为}{xing2wei2}[][HSK 2]
    \definition[个,种,类]{s.}{ação; comportamento; conduta; atividades que são controladas por pensamentos e manifestadas externamente}
  \end{phonetics}
\end{entry}

\begin{entry}{行凶}{6,4}{⾏、⼐}
  \begin{phonetics}{行凶}{xing2xiong1}
    \definition{v.+compl.}{cometer agressão física ou assassinato | fazer algo violento}
  \end{phonetics}
\end{entry}

\begin{entry}{行业}{6,5}{⾏、⼀}
  \begin{phonetics}{行业}{hang2ye4}[][HSK 4]
    \definition[种,个]{s.}{comércio; indústria; setor; profissão; categorias em negócios e indústria referem-se a ocupações em geral}
  \end{phonetics}
\end{entry}

\begin{entry}{行礼}{6,5}{⾏、⽰}
  \begin{phonetics}{行礼}{xing2li3}
    \definition{v.}{saudar | fazer saudação}
  \end{phonetics}
\end{entry}

\begin{entry}{行动}{6,6}{⾏、⼒}
  \begin{phonetics}{行动}{xing2dong4}[][HSK 2]
    \definition[次,场,项]{s.}{ação; operação; comportamento;}
    \definition{v.}{circular; mover-se; andar | agir; tomar medidas; atividades para atingir um determinado propósito}
  \end{phonetics}
\end{entry}

\begin{entry}{行李}{6,7}{⾏、⽊}
  \begin{phonetics}{行李}{xing2li5}[][HSK 3]
    \definition[点,个]{s.}{bagagem, malas, cestas de vime, etc. que você leva quando sai de casa}
  \end{phonetics}
\end{entry}

\begin{entry}{行进}{6,7}{⾏、⾡}
  \begin{phonetics}{行进}{xing2jin4}
    \definition{s.}{avançar | movimentar-se para frente}
  \end{phonetics}
\end{entry}

\begin{entry}{行驶}{6,8}{⾏、⾺}
  \begin{phonetics}{行驶}{xing2 shi3}[][HSK 5]
    \definition{v.}{ir; navegar; viajar (utilizando um veículo, navio, etc.);}
  \end{phonetics}
\end{entry}

\begin{entry}{行星}{6,9}{⾏、⽇}
  \begin{phonetics}{行星}{xing2xing1}
    \definition[颗]{s.}{planeta}
  \seealsoref{惑星}{huo4xing1}
  \end{phonetics}
\end{entry}

\begin{entry}{衣}{6}{⾐}[Kangxi 145]
  \begin{phonetics}{衣}{yi1}
    \definition[件]{s.}{roupa}
  \end{phonetics}
  \begin{phonetics}{衣}{yi4}
    \definition{v.}{vestir | vestir-se}
  \end{phonetics}
\end{entry}

\begin{entry}{衣甲}{6,5}{⾐、⽥}
  \begin{phonetics}{衣甲}{yi1jia3}
    \definition{s.}{armadura}
  \end{phonetics}
\end{entry}

\begin{entry}{衣服}{6,8}{⾐、⽉}
  \begin{phonetics}{衣服}{yi1fu5}[][HSK 1]
    \definition[套,件]{s.}{roupas; vestuário; algo que se veste para cobrir o corpo e se proteger do frio}
  \end{phonetics}
\end{entry}

\begin{entry}{衣柜}{6,8}{⾐、⽊}
  \begin{phonetics}{衣柜}{yi1gui4}
    \definition[个]{s.}{armário | guarda-roupa}
  \end{phonetics}
\end{entry}

\begin{entry}{衣架}{6,9}{⾐、⽊}
  \begin{phonetics}{衣架}{yi1 jia4}[][HSK 3]
    \definition[个,副,组]{s.}{cabideiro; móvel para pendurar roupas | estatura; figura; refere-se ao tipo físico de uma pessoa; estrutura corporal}
  \end{phonetics}
\end{entry}

\begin{entry}{西}{6}{⾑}
  \begin{phonetics}{西}{xi1}[][HSK 1]
    \definition*{s.}{Espanha, abreviatura de 西班牙 | Paraíso Ocidental | Sobrenome Xi}
    \definition{s.}{oeste; uma das quatro direções básicas, o lado onde o sol se põe (oposto ao 东) | ocidental; refere-se ao Ocidente (principalmente aos países europeus e americanos) | aqui e ali; em contraposição a 东, significa 到处 ou 零散, 没有次序}
  \seealsoref{到处}{dao4chu4}
  \seealsoref{东}{dong1}
  \seealsoref{零散}{ling2san3}
  \seealsoref{没有次序}{mei2you3 ci4xu4}
  \seealsoref{西班牙}{xi1ban1ya2}
  \end{phonetics}
\end{entry}

\begin{entry}{西文}{6,4}{⾑、⽂}
  \begin{phonetics}{西文}{xi1wen2}
    \definition{s.}{espanhol | língua espanhola}
  \seealsoref{西班牙文}{xi1ban1ya2wen2}
  \end{phonetics}
\end{entry}

\begin{entry}{西方}{6,4}{⾑、⽅}
  \begin{phonetics}{西方}{xi1 fang1}[][HSK 2]
    \definition{s.}{oeste | o Ocidente; o Oeste; países europeus e americanos | Paraíso Ocidental, termo budista}
  \end{phonetics}
\end{entry}

\begin{entry}{西兰花}{6,5,7}{⾑、⼋、⾋}
  \begin{phonetics}{西兰花}{xi1lan2hua1}
    \definition{s.}{brócolis}
  \end{phonetics}
\end{entry}

\begin{entry}{西北}{6,5}{⾑、⼔}
  \begin{phonetics}{西北}{xi1 bei3}[][HSK 2]
    \definition{s.}{noroeste | noroeste da China; o Noroeste}
  \end{phonetics}
\end{entry}

\begin{entry}{西半球}{6,5,11}{⾑、⼗、⽟}
  \begin{phonetics}{西半球}{xi1ban4qiu2}
    \definition{s.}{hemisfério oeste}
  \end{phonetics}
\end{entry}

\begin{entry}{西瓜}{6,5}{⾑、⽠}
  \begin{phonetics}{西瓜}{xi1gua1}[][HSK 4]
    \definition[个,颗,粒]{s.}{melancia; fruto que é uma baga de formato grande, globular ou oval, com muita polpa aguada e doce}
  \end{phonetics}
\end{entry}

\begin{entry}{西边}{6,5}{⾑、⾡}
  \begin{phonetics}{西边}{xi1bian1}[][HSK 1]
    \definition{s.}{lado oeste; (oeste) Uma das quatro direções principais; uma das direções cardeais, oposta ao 东方}
  \seealsoref{东方}{dong1 fang1}
  \end{phonetics}
\end{entry}

\begin{entry}{西安}{6,6}{⾑、⼧}
  \begin{phonetics}{西安}{xi1'an1}
    \definition*{s.}{Xi'an, Capital da Província de Shaanxi}
  \end{phonetics}
\end{entry}

\begin{entry}{西红柿}{6,6,9}{⾑、⽷、⽊}
  \begin{phonetics}{西红柿}{xi1hong2shi4}[][HSK 5]
    \definition[种,只]{s.}{tomate;}
  \end{phonetics}
\end{entry}

\begin{entry}{西西}{6,6}{⾑、⾑}
  \begin{phonetics}{西西}{xi1xi1}
    \definition{num.}{centímetro cúbico}
  \end{phonetics}
\end{entry}

\begin{entry}{西医}{6,7}{⾑、⼖}
  \begin{phonetics}{西医}{xi1 yi1}[][HSK 2]
    \definition[名,位]{s.}{medicina ocidental; medicina introduzida na China a partir da Europa e da América | um médico treinado em medicina ocidental}
  \end{phonetics}
\end{entry}

\begin{entry}{西南}{6,9}{⾑、⼗}
  \begin{phonetics}{西南}{xi1 nan2}[][HSK 2]
    \definition{s.}{sudoeste | o Sudoeste; Sudoeste da China}
  \end{phonetics}
\end{entry}

\begin{entry}{西药}{6,9}{⾑、⾋}
  \begin{phonetics}{西药}{xi1 yao4}
    \definition{s.}{medicina ocidental; refere-se aos medicamentos usados ​​na medicina ocidental, geralmente feitos por métodos sintéticos ou extraídos de produtos naturais, como comprimidos anti-inflamatórios, aspirina, tintura de iodo, penicilina, etc.}
  \end{phonetics}
\end{entry}

\begin{entry}{西语}{6,9}{⾑、⾔}
  \begin{phonetics}{西语}{xi1yu3}
    \definition{s.}{espanhol | língua espanhola}
  \seealsoref{西班牙语}{xi1ban1ya2yu3}
  \end{phonetics}
\end{entry}

\begin{entry}{西面}{6,9}{⾑、⾯}
  \begin{phonetics}{西面}{xi1mian4}
    \definition{s.}{oeste | lado oeste}
  \end{phonetics}
\end{entry}

\begin{entry}{西班牙}{6,10,4}{⾑、⽟、⽛}
  \begin{phonetics}{西班牙}{xi1ban1ya2}
    \definition*{s.}{Espanha}
  \end{phonetics}
\end{entry}

\begin{entry}{西班牙文}{6,10,4,4}{⾑、⽟、⽛、⽂}
  \begin{phonetics}{西班牙文}{xi1ban1ya2wen2}
    \definition{s.}{espanhol, língua espanhola}
  \seealsoref{西文}{xi1wen2}
  \end{phonetics}
\end{entry}

\begin{entry}{西班牙语}{6,10,4,9}{⾑、⽟、⽛、⾔}
  \begin{phonetics}{西班牙语}{xi1ban1ya2yu3}
    \definition{s.}{espanhol | língua espanhola}
  \seealsoref{西语}{xi1yu3}
  \end{phonetics}
\end{entry}

\begin{entry}{西部}{6,10}{⾑、⾢}
  \begin{phonetics}{西部}{xi1 bu4}[][HSK 3]
    \definition{s.}{(EUA) filme de faroeste; filme de \emph{cowboys}; um faroeste | filme da região ocidental (China) | parte ocidental; região oeste da China}
  \end{phonetics}
\end{entry}

\begin{entry}{西装}{6,12}{⾑、⾐}
  \begin{phonetics}{西装}{xi1 zhuang1}[][HSK 5]
    \definition[件,套,个]{s.}{terno; roupas de estilo ocidental; roupas ocidentais, divididas em masculinas e femininas}
  \end{phonetics}
\end{entry}

\begin{entry}{西蓝花}{6,13,7}{⾑、⾋、⾋}
  \begin{phonetics}{西蓝花}{xi1lan2hua1}
    \variantof{西兰花}
  \end{phonetics}
\end{entry}

\begin{entry}{西餐}{6,16}{⾑、⾷}
  \begin{phonetics}{西餐}{xi1 can1}[][HSK 2]
    \definition[份,顿,桌]{s.}{comida ocidental; comida de estilo ocidental, comida com garfo e faca (diferente da 中餐)}
  \seealsoref{中餐}{zhong1 can1}
  \end{phonetics}
\end{entry}

\begin{entry}{西藏}{6,17}{⾑、⾋}
  \begin{phonetics}{西藏}{xi1zang4}
    \definition*{s.}{Xizang; Região Autônoma do Tibete, 西藏自治区}
  \seealsoref{西藏自治区}{xi1zang4 zi4zhi4qu1}
  \end{phonetics}
\end{entry}

\begin{entry}{西藏自治区}{6,17,6,8,4}{⾑、⾋、⾃、⽔、⼖}
  \begin{phonetics}{西藏自治区}{xi1zang4 zi4zhi4qu1}
    \definition*{s.}{Região Autônoma do Tibete}
  \end{phonetics}
\end{entry}

\begin{entry}{观}{6}{⾒}
  \begin{phonetics}{观}{guan1}
    \definition*{s.}{Templo taoísta; ``Koon''}
    \definition{s.}{visão; vista | perspectiva; visão; conceito | aparência; perspectiva | alcance de visão | noção; ideia; conhecimento ou visão das coisas | ponto de vista; postura; uma visão de uma coisa}
    \definition{v.}{olhar para; assistir; observar | contemplar}
  \end{phonetics}
  \begin{phonetics}{观}{guan4}
    \definition*{s.}{Sobrenome Guan}
    \definition{s.}{mosteiro taoísta | torre de vigia do portão do palácio | plataforma}
  \end{phonetics}
\end{entry}

\begin{entry}{观众}{6,6}{⾒、⼈}
  \begin{phonetics}{观众}{guan1zhong4}[][HSK 3]
    \definition[位,名,批,个]{s.}{espectador; público; audiência; pessoas que assistem a espetáculos ou competições}
  \end{phonetics}
\end{entry}

\begin{entry}{观念}{6,8}{⾒、⼼}
  \begin{phonetics}{观念}{guan1nian4}[][HSK 3]
    \definition[种,个]{s.}{ideia; conceito; consciência ideológica}
  \end{phonetics}
\end{entry}

\begin{entry}{观点}{6,9}{⾒、⽕}
  \begin{phonetics}{观点}{guan1dian3}[][HSK 2]
    \definition[个,种]{s.}{ponto de vista; perspectiva; a visão ou atitude que se tem sobre algo a partir de uma determinada posição ou perspectiva | ponto de vista; perspectiva; a posição ou perspectiva adotada ao analisar uma questão}
  \end{phonetics}
\end{entry}

\begin{entry}{观看}{6,9}{⾒、⽬}
  \begin{phonetics}{观看}{guan1 kan4}[][HSK 3]
    \definition{v.}{assistir; ver propositadamente; observar}
  \end{phonetics}
\end{entry}

\begin{entry}{观察}{6,14}{⾒、⼧}
  \begin{phonetics}{观察}{guan1cha2}[][HSK 3]
    \definition{v.}{assistir; pesquisar; observar; examinar cuidadosamente coisas ou fenômenos}
  \end{phonetics}
\end{entry}

\begin{entry}{讲}{6}{⾔}
  \begin{phonetics}{讲}{jiang3}[][HSK 2]
    \definition[种]{s.}{palestra; discurso}
    \definition{v.}{contar; falar | explicar; transmitir oralmente; esclarecer | negociar; barganhar | ser exigente com; valorizar; dar importância}
  \end{phonetics}
\end{entry}

\begin{entry}{讲究}{6,7}{⾔、⽳}
  \begin{phonetics}{讲究}{jiang3jiu5}[][HSK 4]
    \definition{adj.}{requintado; elegante; de bom gosto; exigente com a vida e com outros aspectos, buscando alto nível, qualidade e detalhes}
    \definition{s.}{estudo cuidadoso; algo que merece atenção; elementos e aspectos que merecem atenção especial}
    \definition{v.}{dar ênfase a; ser específico sobre; prestar atenção a}
  \end{phonetics}
\end{entry}

\begin{entry}{讲话}{6,8}{⾔、⾔}
  \begin{phonetics}{讲话}{jiang3 hua4}[][HSK 2]
    \definition[个]{s.}{discurso; palestra | guia; introdução}
    \definition{v.}{falar; conversar; dirigir-se a alguém | criticar}
  \end{phonetics}
\end{entry}

\begin{entry}{讲述}{6,8}{⾔、⾡}
  \begin{phonetics}{讲述}{jiang3shu4}
    \definition{v.}{falar sobre | narrar | descrever}
  \end{phonetics}
\end{entry}

\begin{entry}{讲座}{6,10}{⾔、⼴}
  \begin{phonetics}{讲座}{jiang3zuo4}[][HSK 4]
    \definition[个]{s.}{palestra; um curso de palestras; a forma de instrução usada para ensinar um determinado assunto ou tópico, geralmente por meio de palestras ao vivo, seriados de rádio ou televisão ou seriados de jornal.}
  \end{phonetics}
\end{entry}

\begin{entry}{许}{6}{⾔}
  \begin{phonetics}{许}{xu3}
    \definition*{s.}{Sobrenome Xu}
    \definition{adv.}{um pouco | talvez}
    \definition{v.}{permitir | prometer | elogiar}
  \end{phonetics}
\end{entry}

\begin{entry}{许可}{6,5}{⾔、⼝}
  \begin{phonetics}{许可}{xu3ke3}[][HSK 5]
    \definition{v.}{permitir; autorizar}
  \end{phonetics}
\end{entry}

\begin{entry}{许多}{6,6}{⾔、⼣}
  \begin{phonetics}{许多}{xu3duo1}[][HSK 2]
    \definition{num.}{muitos; muito; numerosos; uma grande quantidade de}
  \end{phonetics}
\end{entry}

\begin{entry}{论}{6}{⾔}
  \begin{phonetics}{论}{lun2}
    \definition*{s.}{Os Analectos de Confúcio, registro dos ditos e feitos de Confúcio e seus discípulos}
  \end{phonetics}
  \begin{phonetics}{论}{lun4}
    \definition*{s.}{Sobrenome Lun}
    \definition{prep.}{por (uma certa unidade de medida) | de acordo com (um certo sistema ou princípio)}
    \definition{s.}{visão; opinião; declaração | (frequentemente em títulos) dissertação; ensaio; tratado | teoria; doutrina | ideia; palavras ou artigos que analisam e explicam coisas}
    \definition{v.}{discutir; falar sobre; discursar sobre; comentar | mencionar; considerar; falar de | decidir sobre; determinar | decidir sobre a natureza da culpa; punir | argumentar; analisar e explicar coisas | considerar; ponderar; medir; avaliar}
  \end{phonetics}
\end{entry}

\begin{entry}{论文}{6,4}{⾔、⽂}
  \begin{phonetics}{论文}{lun4wen2}[][HSK 4]
    \definition[篇]{s.}{tese; redação; artigo; artigo que discute ou examina uma questão}
  \end{phonetics}
\end{entry}

\begin{entry}{设}{6}{⾔}
  \begin{phonetics}{设}{she4}
    \definition*{s.}{Sobrenome She}
    \definition{conj.}{se; no caso | (matemática) dado; suponha; se}
    \definition{v.}{configurar; estabelecer; encontrar; colocar em prática}
  \end{phonetics}
\end{entry}

\begin{entry}{设计}{6,4}{⾔、⾔}
  \begin{phonetics}{设计}{she4ji4}[][HSK 3]
    \definition[份]{s.}{plano; esquema; refere-se a um plano de design ou a um projeto para um plano, etc.}
    \definition{v.}{planejar; projetar; formular métodos, desenhos, etc. com antecedência, de acordo com determinados requisitos de finalidade, antes de iniciar oficialmente um trabalho | arquitetar; idear; tramar; fazer um plano}
  \end{phonetics}
\end{entry}

\begin{entry}{设立}{6,5}{⾔、⽴}
  \begin{phonetics}{设立}{she4li4}[][HSK 3]
    \definition{v.}{fundar; estabelecer; começar}
  \end{phonetics}
\end{entry}

\begin{entry}{设备}{6,8}{⾔、⼡}
  \begin{phonetics}{设备}{she4bei4}[][HSK 3]
    \definition[台,套]{s.}{instalação; equipamento; montagem; um conjunto de edifícios ou equipamentos necessários para executar uma determinada tarefa ou suprir uma determinada necessidade}
  \end{phonetics}
\end{entry}

\begin{entry}{设施}{6,9}{⾔、⽅}
  \begin{phonetics}{设施}{she4shi1}[][HSK 4]
    \definition{s.}{facilidade; instalação; instituições, sistemas, organizações, edifícios, etc., estabelecidos para realizar um trabalho ou atender a uma necessidade}
  \end{phonetics}
\end{entry}

\begin{entry}{设想}{6,13}{⾔、⼼}
  \begin{phonetics}{设想}{she4xiang3}[][HSK 5]
    \definition[个,种]{s.}{plano provisório (ou ideia); (item, tipo) refere-se a algo hipotético ou imaginário}
    \definition{v.}{imaginar; prever; conceber; supor | ter consideração por}
  \end{phonetics}
\end{entry}

\begin{entry}{设置}{6,13}{⾔、⽹}
  \begin{phonetics}{设置}{she4zhi4}[][HSK 4]
    \definition{v.}{estabelecer; colocar em prática; estabelecer ou criar instituições, empregos, profissões ou códigos, etc. | encaixar; ajustar; instalar; configurar; colocar}
  \end{phonetics}
\end{entry}

\begin{entry}{访}{6}{⾔}
  \begin{phonetics}{访}{fang3}
    \definition{v.}{visitar; fazer uma visita; ligar para | procurar por meio de investigação ou busca; tentar obter; obter uma entrevista | entrevistar | investigar; procurar por meio de investigação (pesquisar)}
  \end{phonetics}
\end{entry}

\begin{entry}{访问}{6,6}{⾔、⾨}
  \begin{phonetics}{访问}{fang3wen4}[][HSK 3]
    \definition{v.}{visitar; ligar; entrevistar; visitar e conversar com um objetivo específico | visitar um \emph{site}}
  \end{phonetics}
\end{entry}

\begin{entry}{负}{6}{⾙}
  \begin{phonetics}{负}{fu4}[][HSK 6]
    \definition{adj.}{negativo; menor que zero | negativo; referindo-se ao que recebe elétrons (oposto a 正)}
    \definition{v.}{carregar; transportar nas costas ou nos ombros | suportar; assumir; encarar | confiar em; contar com; depender | sofrer | aproveitar; desfrutar | ter dívidas | trair; violar | perder; ser derrotado}
  \seealsoref{正}{zheng4}
  \end{phonetics}
\end{entry}

\begin{entry}{负担}{6,8}{⾙、⼿}
  \begin{phonetics}{负担}{fu4dan1}[][HSK 4]
    \definition{s.}{carga; fardo; frete; ônus; pressão ou responsabilidade, despesas, etc.}
    \definition{v.}{carregar; carregar (um fardo); assumir (responsabilidade, trabalho, despesas, etc.)}
  \end{phonetics}
\end{entry}

\begin{entry}{负责}{6,8}{⾙、⾙}
  \begin{phonetics}{负责}{fu4ze2}[][HSK 3]
    \definition{adj.}{consciencioso; ser sério e responsável}
    \definition{v.}{ser responsável por; estar encarregado de; assumir responsabilidades}
  \end{phonetics}
\end{entry}

\begin{entry}{负责人}{6,8,2}{⾙、⾙、⼈}
  \begin{phonetics}{负责人}{fu4 ze2 ren2}[][HSK 5]
    \definition{s.}{pessoa responsável; pessoa encarregada; pessoas com responsabilidades de liderança}
  \end{phonetics}
\end{entry}

\begin{entry}{赱}{6}{⼟}
  \begin{phonetics}{赱}{zou3}
    \variantof{走}
  \end{phonetics}
\end{entry}

\begin{entry}{达}{6}{⾡}
  \begin{phonetics}{达}{da2}
    \definition*{s.}{Sobrenome Da}
    \definition{adj.}{eminente; distinto; refere-se a um funcionário distinto; \emph{status} elevado | otimista; de mente aberta}
    \definition{v.}{prolongar | alcançar; atingir; equivaler a | entender completamente; compreender (assuntos) | expressar; comunicar}
  \end{phonetics}
\end{entry}

\begin{entry}{达成}{6,6}{⾡、⼽}
  \begin{phonetics}{达成}{da2cheng2}[][HSK 5]
    \definition{v.}{concluir; chegar (a um acordo); conseguir; obter (principalmente como resultado de uma negociação)}
  \end{phonetics}
\end{entry}

\begin{entry}{达到}{6,8}{⾡、⼑}
  \begin{phonetics}{达到}{da2dao4}[][HSK 3]
    \definition{v.}{alcançar; atender o padrão; atingir (refere-se principalmente a coisas abstratas ou graus); chegar a um determinado ponto ou grau}
  \end{phonetics}
\end{entry}

\begin{entry}{迅}{6}{⾡}
  \begin{phonetics}{迅}{xun4}
    \definition{adj.}{rápido; veloz}
    \definition{adv.}{rapidamente; velozmente}
  \end{phonetics}
\end{entry}

\begin{entry}{迅速}{6,10}{⾡、⾡}
  \begin{phonetics}{迅速}{xun4su4}[][HSK 4]
    \definition{adv.}{rapidamente; velozmente; prontamente}
  \end{phonetics}
\end{entry}

\begin{entry}{过}{6}{⾡}
  \begin{phonetics}{过}{guo1}
    \definition*{s.}{Sobrenome Guo}
  \end{phonetics}
  \begin{phonetics}{过}{guo4}[][HSK 1,2]
    \definition{adv.}{excessivamente; em excesso}
    \definition{clas.}{tempo; número de vezes usado para a ação}
    \definition{s.}{falha; erro; demérito; equívoco; negligência; (oposto a 功)}
    \definition{v.}{cruzar; passar; mudar-se de um lugar para outro; passar por | exceder; ir além; ultrapassar; usado após um adjetivo, significa ``mais do que'' | gastar (tempo); passar (tempo); exceder (um determinado limite ou limite) | celebrar; comemorar | mudar; transferir; transferir de um lado para o outro | passar por um processo; passar por; submeter a (algum tipo de tratamento) | visitar; fazer uma visita | falecer; morrer | infectar; ser contagioso; espalhar | exceder; ir além; usado após o verbo com o sufixo 得, significa ``superar'' ou ``passar'' | viver | revisar; examinar; usar os olhos para ver ou a mente para lembrar}
  \seealsoref{得}{de5}
  \seealsoref{功}{gong1}
  \end{phonetics}
  \begin{phonetics}{过}{guo5}
    \definition{part.}{usado depois de um verbo para indicar conclusão | usado depois de um verbo para indicar que uma ação ou mudança ocorreu | usado depois de um adjetivo para indicar que algo já teve uma certa qualidade ou estado e para compará-lo com o presente}
  \end{phonetics}
\end{entry}

\begin{entry}{过于}{6,3}{⾡、⼆}
  \begin{phonetics}{过于}{guo4yu2}[][HSK 5]
    \definition{adv.}{demais; indevidamente; excessivamente; advérbios de grau ou quantidade excessiva}
  \end{phonetics}
\end{entry}

\begin{entry}{过不惯}{6,4,11}{⾡、⼀、⼼}
  \begin{phonetics}{过不惯}{guo4 bu5 guan4}
    \definition{v.}{não se acostumar | não se habituar}
  \seealsoref{过惯}{guo4guan4}
  \end{phonetics}
\end{entry}

\begin{entry}{过分}{6,4}{⾡、⼑}
  \begin{phonetics}{过分}{guo4fen4}[][HSK 4]
    \definition{adj.}{excessivo; muito longe; demais; falar ou agir além dos limites ou graus adequados}
    \definition{adv.}{excessivamente; indevidamente; muito mesmo}
  \end{phonetics}
\end{entry}

\begin{entry}{过去}{6,5}{⾡、⼛}
  \begin{phonetics}{过去}{guo4 qu4}[][HSK 2,3]
    \definition{adv.}{(no) passado}
    \definition{s.}{o passado; refere-se a um período anterior; também se refere a coisas anteriores}
    \definition{v.}{atravessar; passar; sair do local onde o interlocutor se encontra e deslocar-se para outro local | acabar; passar; ficar para trás; indica que já passou por uma determinada fase | passar; indica que um determinado período ou situação já não existe mais | falecer | ir lá | passar por}
  \end{phonetics}
\end{entry}

\begin{entry}{过节}{6,5}{⾡、⾋}
  \begin{phonetics}{过节}{guo4jie2}
    \definition{v.+compl.}{celebrar festividades | comemorar um festival}
  \end{phonetics}
\end{entry}

\begin{entry}{过关}{6,6}{⾡、⼋}
  \begin{phonetics}{过关}{guo4guan1}
    \definition{v.+compl.}{passar uma barreira | passar por uma provação | passar em um teste | atingir um padrão | passar pela alfândega}
  \end{phonetics}
\end{entry}

\begin{entry}{过年}{6,6}{⾡、⼲}
  \begin{phonetics}{过年}{guo4 nian2}[][HSK 2]
    \definition{v.+compl.}{comemorar o Ano Novo; comemorar o Festival da Primavera; passar o Ano Novo; passar o Festival da Primavera; realizar atividades comemorativas durante o Ano Novo ou o Festival da Primavera}
  \end{phonetics}
\end{entry}

\begin{entry}{过来}{6,7}{⾡、⽊}
  \begin{phonetics}{过来}{guo4 lai2}[][HSK 2]
    \definition{v.}{vir até aqui | ser capaz de cuidar de | lidar com | administrar}
  \end{phonetics}
\end{entry}

\begin{entry}{过度}{6,9}{⾡、⼴}
  \begin{phonetics}{过度}{guo4du4}[][HSK 5]
    \definition{adj.}{excessivo; acima do limite; além do limite; além do que é apropriado}
  \end{phonetics}
\end{entry}

\begin{entry}{过惯}{6,11}{⾡、⼼}
  \begin{phonetics}{过惯}{guo4guan4}
    \definition{v.}{estar acostumado (a um certo estilo de vida, etc.)}
  \seealsoref{过不惯}{guo4 bu5 guan4}
  \end{phonetics}
\end{entry}

\begin{entry}{过敏}{6,11}{⾡、⽁}
  \begin{phonetics}{过敏}{guo4min3}[][HSK 5]
    \definition{adj.}{sensível; excessivamente sensível; resposta acima do normal; ceticismo excessivo}
    \definition{v.}{ser alérgico a}
  \end{phonetics}
\end{entry}

\begin{entry}{过期}{6,12}{⾡、⽉}
  \begin{phonetics}{过期}{guo4qi1}
    \definition{v.+compl.}{exceder a data | passar a data | expirar (passar a data de expiração)}
  \end{phonetics}
\end{entry}

\begin{entry}{过程}{6,12}{⾡、⽲}
  \begin{phonetics}{过程}{guo4cheng2}[][HSK 3]
    \definition[个,段]{s.}{curso dos eventos; processo; o processo pelo qual as coisas acontecem ou se desenvolvem.}
  \end{phonetics}
\end{entry}

\begin{entry}{过瘾}{6,16}{⾡、⽧}
  \begin{phonetics}{过瘾}{guo4yin3}
    \definition{adj.}{gratificante | imensamente agradável | satisfatório}
    \definition{v.+compl.}{satisfazer um desejo | se divertir com algo}
  \end{phonetics}
\end{entry}

\begin{entry}{那}{6}{⾢}
  \begin{phonetics}{那}{na1}
    \definition*{s.}{Sobrenome Na}
  \end{phonetics}
  \begin{phonetics}{那}{na3}
    \definition{adv.}{expressa negação em perguntas retóricas}
    \definition{pron.}{qual? | qualquer que seja; qualquer que; para expressar incerteza em uma declaração | variante de 哪}
  \seealsoref{哪}{na3}
  \end{phonetics}
  \begin{phonetics}{那}{na4}[][HSK 1,2]
    \definition{conj.}{então; nessa situação; nesse caso; o mesmo que 那么}
    \definition{pron.}{aquele; aquilo; indica pessoas ou coisas distantes | aquele; aquilo; expressa muitas coisas, sem se referir especificamente a uma pessoa ou coisa, e é frequentemente usado em conjunto com 这}
  \seealsoref{那么}{na4 me5}
  \seealsoref{这}{zhe4}
  \end{phonetics}
  \begin{phonetics}{那}{ne4}
    \definition{conj.}{então; nesse caso; o mesmo que 那么}
    \definition{pron.}{aquele; aquilo; pronúncia coloquial de 那 (\dpy{na4})}
  \seealsoref{那么}{na4 me5}
  \end{phonetics}
  \begin{phonetics}{那}{nei4}
    \definition{conj.}{então; o mesmo que 那么}
    \definition{pron.}{aquele; aquilo; A pronúncia coloquial de 那 (\dpy{na4})}
  \seealsoref{那么}{na4 me5}
  \end{phonetics}
  \begin{phonetics}{那}{nuo2}
    \definition*{s.}{Sobrenome Nuo}
  \end{phonetics}
\end{entry}

\begin{entry}{那儿}{6,2}{⾢、⼉}
  \begin{phonetics}{那儿}{na4r5}[][HSK 1]
    \definition{pron.}{lá; ali; naquele lugar | então; naquela época (usado após 打, 从 e 由)}
  \seealsoref{从}{cong2}
  \seealsoref{打}{da3}
  \seealsoref{由}{you2}
  \end{phonetics}
\end{entry}

\begin{entry}{那个}{6,3}{⾢、⼈}
  \begin{phonetics}{那个}{na4ge5}
    \definition{pron.}{aquele | usado antes de verbos e adjetivos para indicar exagero | para substituir o discurso direto inconveniente}
  \end{phonetics}
\end{entry}

\begin{entry}{那么}{6,3}{⾢、⼃}
  \begin{phonetics}{那么}{na4 me5}[][HSK 2]
    \definition{conj.}{então; nesse caso; afirmar o resultado esperado ou fazer um julgamento}
    \definition{pron.}{assim; dessa maneira; indica a natureza, o estado, a forma, o grau, etc. | assim; sobre; colocado antes do numeral, indica uma estimativa}
  \end{phonetics}
\end{entry}

\begin{entry}{那边}{6,5}{⾢、⾡}
  \begin{phonetics}{那边}{na4 bian5}[][HSK 1]
    \definition{pron.}{ali; acolá; aquele lado}
  \end{phonetics}
\end{entry}

\begin{entry}{那会儿}{6,6,2}{⾢、⼈、⼉}
  \begin{phonetics}{那会儿}{na4 hui4r5}[][HSK 2]
    \definition{pron.}{então; naquela época; refere-se ao passado ou ao futuro}
  \end{phonetics}
\end{entry}

\begin{entry}{那时}{6,7}{⾢、⽇}
  \begin{phonetics}{那时}{na4 shi2}[][HSK 2]
    \definition{pron.}{então; naquela época; naqueles dias; geralmente se refere a um período de tempo distante do presente}
  \seealsoref{那时候}{na4 shi2 hou5}
  \end{phonetics}
\end{entry}

\begin{entry}{那时候}{6,7,10}{⾢、⽇、⼈}
  \begin{phonetics}{那时候}{na4 shi2 hou5}[][HSK 2]
    \definition{adv.}{naquela hora; em algum momento no passado}
  \seealsoref{那时}{na4 shi2}
  \end{phonetics}
\end{entry}

\begin{entry}{那里}{6,7}{⾢、⾥}
  \begin{phonetics}{那里}{na4 li3}[][HSK 1]
    \definition{pron./s.}{lá; ali; aquele lugar; indica um lugar distante}
  \end{phonetics}
\end{entry}

\begin{entry}{那些}{6,8}{⾢、⼆}
  \begin{phonetics}{那些}{na4 xie1}[][HSK 1]
    \definition{pron.}{aqueles; indica duas ou mais pessoas ou coisas}
  \end{phonetics}
\end{entry}

\begin{entry}{那咱}{6,9}{⾢、⼝}
  \begin{phonetics}{那咱}{na4 zan5}
    \definition{s.}{(informal) naquela época; então | (antigo) naquela época}
  \end{phonetics}
\end{entry}

\begin{entry}{那样}{6,10}{⾢、⽊}
  \begin{phonetics}{那样}{na4 yang4}[][HSK 2]
    \definition{pron.}{assim; tal; desse tipo; desse gênero; dessa natureza; desse tipo; indica a natureza, o estado, a maneira, o grau ou refere-se a uma ação ou situação específica}
  \end{phonetics}
\end{entry}

\begin{entry}{那麽}{6,14}{⾢、⿇}
  \begin{phonetics}{那麽}{na4 me5}
    \variantof{那么}
  \end{phonetics}
\end{entry}

\begin{entry}{闭}{6}{⾨}
  \begin{phonetics}{闭}{bi4}[][HSK 6]
    \definition*{s.}{Sobrenome Bi}
    \definition{v.}{fechar; encerrar | bloquear; obstruir; parar}
  \end{phonetics}
\end{entry}

\begin{entry}{闭幕}{6,13}{⾨、⼱}
  \begin{phonetics}{闭幕}{bi4 mu4}[][HSK 5]
    \definition{v.+compl.}{fechar; concluir; (conferência, exposição, etc.) terminar | cair a cortina; abaixar a cortina; terminar a apresentação e a cortina se fechar em frente ao palco}
  \end{phonetics}
\end{entry}

\begin{entry}{闭幕式}{6,13,6}{⾨、⼱、⼷}
  \begin{phonetics}{闭幕式}{bi4 mu4 shi4}[][HSK 5]
    \definition{s.}{cerimônia de encerramento; cerimônia formal realizada no final de uma conferência ou exposição}
  \end{phonetics}
\end{entry}

\begin{entry}{闭嘴}{6,16}{⾨、⼝}
  \begin{phonetics}{闭嘴}{bi4zui3}
    \definition{expr.}{Cale-se!; Pare de falar!}
  \end{phonetics}
\end{entry}

\begin{entry}{问}{6}{⾨}
  \begin{phonetics}{问}{wen4}[][HSK 1]
    \definition*{s.}{Sobrenome Wen}
    \definition{prep.}{de; introduzir o objeto da ação, equivalente a 向 e 跟}
    \definition{v.}{perguntar; indagar; fazer com que as pessoas respondam ou esclareçam coisas que não sabem ou não têm certeza | perguntar (ou indagar) sobre | examinar; interrogar | intervir; responsabilizar; investigar | cuidar; preocupar-se; gerenciar; interferir}
  \seealsoref{跟}{gen1}
  \seealsoref{向}{xiang4}
  \end{phonetics}
\end{entry}

\begin{entry}{问市}{6,5}{⾨、⼱}
  \begin{phonetics}{问市}{wen4shi4}
    \definition{v.}{chegar ao mercado | bater o mercado | atingir o mercado}
  \end{phonetics}
\end{entry}

\begin{entry}{问安}{6,6}{⾨、⼧}
  \begin{phonetics}{问安}{wen4'an1}
    \definition{s.}{saudações}
    \definition{v.}{dar cumprimentos a | prestar homenagem}
  \end{phonetics}
\end{entry}

\begin{entry}{问卷}{6,8}{⾨、⼙}
  \begin{phonetics}{问卷}{wen4juan4}
    \definition[份]{s.}{questionário}
  \end{phonetics}
\end{entry}

\begin{entry}{问候}{6,10}{⾨、⼈}
  \begin{phonetics}{问候}{wen4hou4}[][HSK 4]
    \definition{s.}{homenagem | saudação}
    \definition{v.}{prestar homenagem; enviar uma saudação;  dar os respeitos (cumprimentos) a alguém | (fig.) (coloquial) fazer referência ofensiva a (alguém querido pela pessoa com quem se está falando)}
  \end{phonetics}
\end{entry}

\begin{entry}{问鼎}{6,12}{⾨、⿍}
  \begin{phonetics}{问鼎}{wen4ding3}
    \definition{v.}{visar (o primeiro lugar, etc.) | aspirar ao trono}
  \end{phonetics}
\end{entry}

\begin{entry}{问路}{6,13}{⾨、⾜}
  \begin{phonetics}{问路}{wen4 lu4}[][HSK 2]
    \definition{v.}{perguntar o caminho; pedir direções}
  \end{phonetics}
\end{entry}

\begin{entry}{问题}{6,15}{⾨、⾴}
  \begin{phonetics}{问题}{wen4ti2}[][HSK 2]
    \definition{adj.}{desqualificado; indesejável; anormal, não atende aos requisitos}
    \definition[个,种,类,串]{s.}{pergunta; problema; perguntas a serem respondidas | problema; questão; contradições que precisam ser estudadas e resolvidas | problema; acidente; incidente | chave; ponto crucial; pontos importantes}
  \end{phonetics}
\end{entry}

\begin{entry}{闯}{6}{⾨}
  \begin{phonetics}{闯}{chuang3}[][HSK 5]
    \definition*{s.}{Sobrenome Chuang}
    \definition{v.}{apressar-se; correr | moderar a si mesmo (lutando contra dificuldades e perigos); aventurar-se no mundo | incorrer; causar (um desastre, etc.)}
  \end{phonetics}
\end{entry}

\begin{entry}{防}{6}{⾩}
  \begin{phonetics}{防}{fang2}[][HSK 3]
    \definition*{s.}{Sobrenome Fang}
    \definition{s.}{defesa | dique; aterro | barragem; represa; estrutura para conter a água}
    \definition{v.}{proteger contra; prevenir contra; tomar precauções contra | defender-se contra}
  \end{phonetics}
\end{entry}

\begin{entry}{防止}{6,4}{⾩、⽌}
  \begin{phonetics}{防止}{fang2zhi3}[][HSK 3]
    \definition{v.}{evitar; prevenir; prevenir; proteger contra; preparar-se com antecedência para evitar que coisas ruins aconteçam}
  \end{phonetics}
\end{entry}

\begin{entry}{防护}{6,7}{⾩、⼿}
  \begin{phonetics}{防护}{fang2hu4}
    \definition{v.}{defender | proteger}
  \end{phonetics}
\end{entry}

\begin{entry}{防治}{6,8}{⾩、⽔}
  \begin{phonetics}{防治}{fang2zhi4}[][HSK 5]
    \definition{s.}{tratamento preventivo; prevenção e cura; profilaxia e tratamento}
  \end{phonetics}
\end{entry}

\begin{entry}{防晒}{6,10}{⾩、⽇}
  \begin{phonetics}{防晒}{fang2shai4}
    \definition{s.}{protetor solar}
  \end{phonetics}
\end{entry}

\begin{entry}{阳}{6}{⾩}
  \begin{phonetics}{阳}{yang2}
    \definition*{s.}{Yang, o princípio positivo de Yin e Yang}
    \definition{s.}{eletricidade: positivo | sol}
  \seealsoref{阴}{yin1}
  \seealsoref{阴阳}{yin1yang2}
  \end{phonetics}
\end{entry}

\begin{entry}{阳台}{6,5}{⾩、⼝}
  \begin{phonetics}{阳台}{yang2tai2}[][HSK 4]
    \definition{s.}{varanda; terraço; sacada; pequeno terraço do edifício com grades para se refrescar, tomar sol ou olhar o horizonte}
  \end{phonetics}
\end{entry}

\begin{entry}{阳光}{6,6}{⾩、⼉}
  \begin{phonetics}{阳光}{yang2guang1}[][HSK 3]
    \definition{adj.}{alegre; otimista; personalidade positiva e alegre; cheio de vitalidade juvenil | aberto; transparente; público; conduzido sob supervisão pública}
    \definition[缕,束,道]{s.}{luz do sol; raio de sol}
  \end{phonetics}
\end{entry}

\begin{entry}{阴}{6}{⾩}
  \begin{phonetics}{阴}{yin1}[][HSK 2]
    \definition*{s.}{Yin, o princípio negativo de Yin e Yang | A Lua; refere-se a Taiyin | Sobrenome Yin}
    \definition{adj.}{nublado; opaco; sombrio | escondido; secreto; não exposto | sinistro | do mundo inferior; dos fantasmas | Física: negativo; cátodo | nublado; mais de 80\% do céu estão cobertos por nuvens | em talhe-doce; rebaixado | (matéria) carregada negativamente}
    \definition[片]{s.}{sombra; lugar sombrio | partes íntimas (especialmente da mulher) | ao norte de uma colina ou ao sul de um rio | verso | entalhe}
  \seealsoref{阳}{yang2}
  \seealsoref{阴阳}{yin1yang2}
  \end{phonetics}
\end{entry}

\begin{entry}{阴天}{6,4}{⾩、⼤}
  \begin{phonetics}{阴天}{yin1 tian1}[][HSK 2]
    \definition[个]{s.}{nublado; céu nublado; dia nublado; uma condição climática em que 80\% do céu está coberto por nuvens e apenas um pouco de sol pode ser visto}
  \end{phonetics}
\end{entry}

\begin{entry}{阴阳}{6,6}{⾩、⾩}
  \begin{phonetics}{阴阳}{yin1yang2}
    \definition*{s.}{Yin e Yang}
  \seealsoref{阳}{yang2}
  \seealsoref{阴}{yin1}
  \end{phonetics}
\end{entry}

\begin{entry}{阵}{6}{⾩}
  \begin{phonetics}{阵}{zhen4}[][HSK 4]
    \definition{clas.}{passagens que expressam a passagem de eventos ou ações}
    \definition{s.}{matriz de batalha (formação); termo tático antigo para as fileiras ou formações de uma equipe de combate | \emph{front}; frente de batalha; posição | um período de tempo}
  \end{phonetics}
\end{entry}

\begin{entry}{阵地}{6,6}{⾩、⼟}
  \begin{phonetics}{阵地}{zhen4di4}
    \definition{s.}{posição (militar) | frente de batalha | \emph{front}}
  \end{phonetics}
\end{entry}

\begin{entry}{阶}{6}{⾩}
  \begin{phonetics}{阶}{jie1}
    \definition{s.}{degrau; escada; escadaria | classificação | escala | ordem | estágio}
  \end{phonetics}
\end{entry}

\begin{entry}{阶段}{6,9}{⾩、⽎}
  \begin{phonetics}{阶段}{jie1duan4}[][HSK 4]
    \definition{s.}{estágio; fase; período; bancada; gradação}
  \end{phonetics}
\end{entry}

\begin{entry}{页}{6}{⾴}[Kangxi 181]
  \begin{phonetics}{页}{ye4}[][HSK 1]
    \definition{clas.}{página; folha de papel; lâmina; antigamente, referia-se a uma folha de um livro encadernado; atualmente, refere-se a uma das faces de um livro impresso em ambos os lados}
    \definition{s.}{página; folha de papel; folhas soltas de um livro}
  \end{phonetics}
\end{entry}

\begin{entry}{齐}{6}{⿑}[Kangxi 210]
  \begin{phonetics}{齐}{qi2}[][HSK 3]
    \definition*{s.}{Qi, um estado da Dinastia Zhou | Dinastia Qi do Sul (479-502), uma das Dinastias do Sul | Dinastia Qi do Norte (550-577), uma das Dinastias do Norte | Sobrenome Qi}
    \definition{adj.}{arrumado; uniforme; regular; comprimento, tamanho, etc. são praticamente iguais; uniformes | semelhante; similar; da mesma forma; de acordo| tudo pronto; todos presentes; completo; perfeito}
    \definition{adv.}{juntos; simultaneamente; ao mesmo tempo}
    \definition{v.}{estar no mesmo nível que; alcançar o mesmo nível | estar nivelado em um ponto ou ao longo de uma linha; tornar consistente; harmonizar}
  \end{phonetics}
\end{entry}

\begin{entry}{齐全}{6,6}{⿑、⼊}
  \begin{phonetics}{齐全}{qi2quan2}[][HSK 5]
    \definition{adj.}{completo; tudo pronto}
  \end{phonetics}
\end{entry}

%%%%% EOF %%%%%

