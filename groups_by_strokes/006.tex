%%%
%%% 6画
%%%

\section*{6画}\addcontentsline{toc}{section}{6画}

\begin{entry}{丢}{6}[Radical ⼛]
  \begin{phonetics}{丢}{diu1}
    \definition{v.}{perder | perder-se}
  \end{phonetics}
\end{entry}

\begin{entry}{丢下}{6,3}[Radicais ⼛、⼀]
  \begin{phonetics}{丢下}{diu1xia4}
    \definition{v.}{abandonar}
  \end{phonetics}
\end{entry}

\begin{entry}{丢开}{6,4}[Radicais ⼛、⼶]
  \begin{phonetics}{丢开}{diu1kai1}
    \definition{v.}{jogar fora ou deixar de lado | esquecer por um tempo}
  \end{phonetics}
\end{entry}

\begin{entry}{丢失}{6,5}[Radicais ⼛、⼤]
  \begin{phonetics}{丢失}{diu1shi1}
    \definition{v.}{perder}
  \end{phonetics}
\end{entry}

\begin{entry}{丢弃}{6,7}[Radicais ⼛、⼶]
  \begin{phonetics}{丢弃}{diu1qi4}
    \definition{v.}{jogar fora | descartar}
  \end{phonetics}
\end{entry}

\begin{entry}{丢官}{6,8}[Radicais ⼛、⼧]
  \begin{phonetics}{丢官}{diu1guan1}
    \definition{v.}{perder um cargo oficial}
  \end{phonetics}
\end{entry}

\begin{entry}{丢掉}{6,11}[Radicais ⼛、⼿]
  \begin{phonetics}{丢掉}{diu1diao4}
    \definition{v.}{jogar fora | descartar |perder}
  \end{phonetics}
\end{entry}

\begin{entry}{丢脸}{6,11}[Radicais ⼛、⾁]
  \begin{phonetics}{丢脸}{diu1lian3}
    \definition{adj.}{vergonhoso}
  \end{phonetics}
\end{entry}

\begin{entry}{乒乓球}{6,6,11}[Radicais ⼃、⼃、⽟]
  \begin{phonetics}{乒乓球}{ping1pang1qiu2}
    \definition[个]{s.}{tênis de mesa |ping-pong}
  \end{phonetics}
\end{entry}

\begin{entry}{买}{6}[Radical ⼄]
  \begin{phonetics}{买}{mai3}[][HSK 1]
    \definition{v.}{comprar}
  \end{phonetics}
\end{entry}

\begin{entry}{买东西}{6,5,6}[Radicais ⼄、⼀、⾑]
  \begin{phonetics}{买东西}{mai3dong1xi5}
    \definition{v.}{fazer compras}
  \end{phonetics}
\end{entry}

\begin{entry}{争}{6}[Radical ⼑]
  \begin{phonetics}{争}{zheng1}[][HSK 3]
    \definition*{s.}{sobrenome Zheng}
    \definition{adv.}{como; por que}
    \definition{v.}{contender; competir; lutar por; esforçar-se | argumentar; disputar; debater | faltar; estar aquém de}
  \end{phonetics}
\end{entry}

\begin{entry}{争风吃醋}{6,4,6,15}[Radicais ⼑、⾵、⼝、⾣]
  \begin{phonetics}{争风吃醋}{zheng1feng1chi1cu4}
    \definition{v.}{rivalizar com alguém pelo carinho de um homem ou mulher |estar com ciúmes de um rival em um caso de amor}
  \end{phonetics}
\end{entry}

\begin{entry}{争先}{6,6}[Radicais ⼑、⼉]
  \begin{phonetics}{争先}{zheng1xian1}
    \definition{v.}{competir para ser o primeiro |contestar o primeiro lugar}
  \end{phonetics}
\end{entry}

\begin{entry}{争取}{6,8}[Radicais ⼑、⼜]
  \begin{phonetics}{争取}{zheng1qu3}[][HSK 3]
    \definition{v.}{esforçar-se por; lutar por; vencer}
  \end{phonetics}
\end{entry}

\begin{entry}{争霸}{6,21}[Radicais ⼑、⾬]
  \begin{phonetics}{争霸}{zheng1ba4}
    \definition{s.}{hegemonia | uma luta de poder}
    \definition{v.}{disputar a hegemonia}
  \end{phonetics}
\end{entry}

\begin{entry}{亚细亚洲}{6,8,6,9}[Radicais ⼆、⽷、⼆、⽔]
  \begin{phonetics}{亚细亚洲}{ya4xi4ya4zhou1}
    \definition*{s.}{Ásia}
  \end{phonetics}
\end{entry}

\begin{entry}{亚洲}{6,9}[Radicais ⼆、⽔]
  \begin{phonetics}{亚洲}{ya4zhou1}
    \definition*{s.}{Ásia, abreviação de~亚细亚洲}
    \seeref{亚细亚洲}{ya4xi4ya4zhou1}
  \end{phonetics}
\end{entry}

\begin{entry}{亚洲人}{6,9,2}[Radicais ⼆、⽔、⼈]
  \begin{phonetics}{亚洲人}{ya4zhou1ren2}
    \definition{s.}{asiático | pessoa ou povo da Ásia}
  \end{phonetics}
\end{entry}

\begin{entry}{交}{6}[Radical ⼇]
  \begin{phonetics}{交}{jiao1}[][HSK 2]
    \definition{v.}{entregar | dar}
  \end{phonetics}
\end{entry}

\begin{entry}{交叉}{6,3}[Radicais ⼇、⼜]
  \begin{phonetics}{交叉}{jiao1cha1}
    \definition{v.}{cruzar | sobrepor}
  \end{phonetics}
\end{entry}

\begin{entry}{交叉口}{6,3,3}[Radicais ⼇、⼜、⼝]
  \begin{phonetics}{交叉口}{jiao1cha1kou3}
    \definition{s.}{intersecção (rodovia)}
  \end{phonetics}
\end{entry}

\begin{entry}{交叉点}{6,3,9}[Radicais ⼇、⼜、⽕]
  \begin{phonetics}{交叉点}{jiao1cha1dian3}
    \definition{s.}{encruzilhada | cruzamento | junção}
  \end{phonetics}
\end{entry}

\begin{entry}{交运}{6,7}[Radicais ⼇、⾡]
  \begin{phonetics}{交运}{jiao1yun4}
    \definition{v.}{despachar (bagagem em um aeroporto, etc.) | entregar para transporte}
  \end{phonetics}
\end{entry}

\begin{entry}{交际}{6,7}[Radicais ⼇、⾩]
  \begin{phonetics}{交际}{jiao1ji4}[][HSK 4]
    \definition{s.}{contato; comunicação; relações sociais; contato interpessoal, socialização}
  \end{phonetics}
\end{entry}

\begin{entry}{交往}{6,8}[Radicais ⼇、⼻]
  \begin{phonetics}{交往}{jiao1wang3}[][HSK 3]
    \definition{v.}{estar em contato com; associar-se a}
  \end{phonetics}
\end{entry}

\begin{entry}{交易}{6,8}[Radicais ⼇、⽇]
  \begin{phonetics}{交易}{jiao1yi4}[][HSK 3]
    \definition[笔,桩,个,场]{s.}{acordo; negócio; transação}
    \definition{v.}{negociar}
  \end{phonetics}
\end{entry}

\begin{entry}{交朋友}{6,8,4}[Radicais ⼇、⽉、⼜]
  \begin{phonetics}{交朋友}{jiao1 peng2 you3}[][HSK 2]
    \definition{v.}{faça amizade com alguém}
  \end{phonetics}
\end{entry}

\begin{entry}{交杯酒}{6,8,10}[Radicais ⼇、⽊、⾣]
  \begin{phonetics}{交杯酒}{jiao1bei1jiu3}
    \definition{s.}{copo de vinho nupcial}
  \end{phonetics}
\end{entry}

\begin{entry}{交响}{6,9}[Radicais ⼇、⼝]
  \begin{phonetics}{交响}{jiao1xiang3}
    \definition{s.}{sinfonia}
  \end{phonetics}
\end{entry}

\begin{entry}{交界}{6,9}[Radicais ⼇、⽥]
  \begin{phonetics}{交界}{jiao1jie4}
    \definition{s.}{fronteira comum | limite comum | interface}
  \end{phonetics}
\end{entry}

\begin{entry}{交给}{6,9}[Radicais ⼇、⽷]
  \begin{phonetics}{交给}{jiao1 gei3}[][HSK 2]
    \definition{v.}{entregar algo | dar algo}
  \end{phonetics}
\end{entry}

\begin{entry}{交费}{6,9}[Radicais ⼇、⾙]
  \begin{phonetics}{交费}{jiao1 fei4}[][HSK 3]
    \definition{v.}{pagar taxas; pagar uma taxa}
  \end{phonetics}
\end{entry}

\begin{entry}{交换}{6,10}[Radicais ⼇、⼿]
  \begin{phonetics}{交换}{jiao1huan4}[][HSK 4]
    \definition{v.}{trocar; permutar; comutar; intercambiar}
  \end{phonetics}
\end{entry}

\begin{entry}{交流}{6,10}[Radicais ⼇、⽔]
  \begin{phonetics}{交流}{jiao1liu2}[][HSK 3]
    \definition{v.}{trocar; permutar; cambiar}
  \end{phonetics}
\end{entry}

\begin{entry}{交班}{6,10}[Radicais ⼇、⽟]
  \begin{phonetics}{交班}{jiao1ban1}
    \definition{v.}{passar para o próximo turno de trabalho}
  \end{phonetics}
\end{entry}

\begin{entry}{交通}{6,10}[Radicais ⼇、⾡]
  \begin{phonetics}{交通}{jiao1tong1}[][HSK 2]
    \definition{s.}{transporte | tráfego | trânsito | comunicação | conexão}
    \definition{v.}{estar conectado | ser conectado}
  \end{phonetics}
\end{entry}

\begin{entry}{交通警察}{6,10,19,14}[Radicais ⼇、⾡、⾔、⼧]
  \begin{phonetics}{交通警察}{jiao1tong1jing3cha2}
    \definition{s.}{policial de trânsito}
  \seealsoref{交警}{jiao1 jing3}
  \end{phonetics}
\end{entry}

\begin{entry}{交叠}{6,13}[Radicais ⼇、⼜]
  \begin{phonetics}{交叠}{jiao1die2}
    \definition{s.}{sobreposição}
  \end{phonetics}
\end{entry}

\begin{entry}{交媾}{6,13}[Radicais ⼇、⼥]
  \begin{phonetics}{交媾}{jiao1gou4}
    \definition{v.}{copular | ter relações sexuais}
  \end{phonetics}
\end{entry}

\begin{entry}{交警}{6,19}[Radicais ⼇、⾔]
  \begin{phonetics}{交警}{jiao1 jing3}[][HSK 3]
    \definition{s.}{policial de trânsito, abreviação de 交通警察}
    \seeref{交通警察}{jiao1tong1jing3cha2}
  \end{phonetics}
\end{entry}

\begin{entry}{亦}{6}[Radical ⼇]
  \begin{phonetics}{亦}{yi4}
    \definition{adv.}{também | igualmente | apenas | embora | já}
  \end{phonetics}
\end{entry}

\begin{entry}{产生}{6,5}[Radicais ⼇、⽣]
  \begin{phonetics}{产生}{chan3sheng1}[][HSK 3]
    \definition{v.}{produzir; evoluir; emergir; provocar; vir a ser; dar origem a}
  \end{phonetics}
\end{entry}

\begin{entry}{产后}{6,6}[Radicais ⼇、⼝]
  \begin{phonetics}{产后}{chan3hou4}
    \definition{s.}{pós-parto}
  \end{phonetics}
\end{entry}

\begin{entry}{产品}{6,9}[Radicais ⼇、⼝]
  \begin{phonetics}{产品}{chan3pin3}[][HSK 4]
    \definition[个,件,种,批,项,类]{s.}{produto; item produzido}
  \end{phonetics}
\end{entry}

\begin{entry}{件}{6}[Radical ⼈]
  \begin{phonetics}{件}{jian4}[][HSK 2]
    \definition{clas.}{para eventos, coisas, roupas etc.}
    \definition{s.}{item | componente}
  \end{phonetics}
\end{entry}

\begin{entry}{价值}{6,10}[Radicais ⼈、⼈]
  \begin{phonetics}{价值}{jia4zhi2}[][HSK 3]
    \definition{s.}{valor}
  \end{phonetics}
\end{entry}

\begin{entry}{价格}{6,10}[Radicais ⼈、⽊]
  \begin{phonetics}{价格}{jia4ge2}[][HSK 3]
    \definition[个]{s.}{preço; tarifa}
  \end{phonetics}
\end{entry}

\begin{entry}{价钱}{6,10}[Radicais ⼈、⾦]
  \begin{phonetics}{价钱}{jia4 qian2}[][HSK 3]
    \definition[些]{s.}{preço}
  \end{phonetics}
\end{entry}

\begin{entry}{任}{6}[Radical ⼈]
  \begin{phonetics}{任}{ren4}[][HSK 3]
    \definition{clas.}{número de vezes que serviu em uma posição}
    \definition{conj.}{não importa (como, o que, etc.)}
    \definition{s.}{correio oficial; escritório}
    \definition{v.}{nomear; designar | assumir um posto; assumir um emprego | deixar; permitir; dar rédea solta a}
  \end{phonetics}
\end{entry}

\begin{entry}{任务}{6,5}[Radicais ⼈、⼒]
  \begin{phonetics}{任务}{ren4wu5}[][HSK 3]
    \definition[项,个]{s.}{tarefa; dever; missão; designação}
  \end{phonetics}
\end{entry}

\begin{entry}{任何}{6,7}[Radicais ⼈、⼈]
  \begin{phonetics}{任何}{ren4he2}[][HSK 3]
    \definition{pron.}{qualquer; qualquer que seja; o que for}
  \end{phonetics}
\end{entry}

\begin{entry}{份}{6}[Radical ⼈]
  \begin{phonetics}{份}{fen4}[][HSK 2]
    \definition{clas.}{para presentes, jornais, revistas, papéis, relatórios, contratos, etc. ou pratos (refeição)}
  \end{phonetics}
\end{entry}

\begin{entry}{企业}{6,5}[Radicais ⼈、⼀]
  \begin{phonetics}{企业}{qi3ye4}[][HSK 4]
    \definition[家,个]{s.}{empresa; estabelecimento; empreendimento; negócio; setores envolvidos em atividades econômicas como produção, transporte, comércio, etc., como fábricas, minas, ferrovias, empresas comerciais, etc.}
  \end{phonetics}
\end{entry}

\begin{entry}{伊马姆}{6,3,8}[Radicais ⼈、⾺、⼥]
  \begin{phonetics}{伊马姆}{yi1ma3mu3}
    \definition*{s.}{Islã}
  \seealsoref{伊玛目}{yi1ma3mu4}
  \seealsoref{伊曼}{yi1man4}
  \seealsoref{伊斯兰}{yi1si1lan2}
  \end{phonetics}
\end{entry}

\begin{entry}{伊玛目}{6,7,5}[Radicais ⼈、⽟、⽬]
  \begin{phonetics}{伊玛目}{yi1ma3mu4}
    \definition*{s.}{Islã}
  \seealsoref{伊马姆}{yi1ma3mu3}
  \seealsoref{伊曼}{yi1man4}
  \seealsoref{伊斯兰}{yi1si1lan2}
  \end{phonetics}
\end{entry}

\begin{entry}{伊朗}{6,10}[Radicais ⼈、⽉]
  \begin{phonetics}{伊朗}{yi1lang3}
    \definition*{s.}{Irã}
  \end{phonetics}
\end{entry}

\begin{entry}{伊曼}{6,11}[Radicais ⼈、⽈]
  \begin{phonetics}{伊曼}{yi1man4}
    \definition*{s.}{Islã}
  \seealsoref{伊马姆}{yi1ma3mu3}
  \seealsoref{伊玛目}{yi1ma3mu4}
  \seealsoref{伊斯兰}{yi1si1lan2}
  \end{phonetics}
\end{entry}

\begin{entry}{伊斯兰}{6,12,5}[Radicais ⼈、⽄、⼋]
  \begin{phonetics}{伊斯兰}{yi1si1lan2}
    \definition*{s.}{Islã}
  \seealsoref{伊马姆}{yi1ma3mu3}
  \seealsoref{伊玛目}{yi1ma3mu4}
  \seealsoref{伊曼}{yi1man4}
  \end{phonetics}
\end{entry}

\begin{entry}{休兵}{6,7}[Radicais ⼈、⼋]
  \begin{phonetics}{休兵}{xiu1bing1}
    \definition{s.}{armistício}
    \definition{v.}{cessar fogo}
  \end{phonetics}
\end{entry}

\begin{entry}{休闲}{6,7}[Radicais ⼈、⾨]
  \begin{phonetics}{休闲}{xiu1xian2}
    \definition{s.}{ócio | lazer}
    \definition{v.}{desfrutar do lazer}
  \end{phonetics}
\end{entry}

\begin{entry}{休息}{6,10}[Radicais ⼈、⼼]
  \begin{phonetics}{休息}{xiu1xi5}[][HSK 1]
    \definition{s.}{descanço}
    \definition{v.}{descansar}
  \end{phonetics}
\end{entry}

\begin{entry}{休息室}{6,10,9}[Radicais ⼈、⼼、⼧]
  \begin{phonetics}{休息室}{xiu1xi1shi4}
    \definition{s.}{saguão | salão}
  \end{phonetics}
\end{entry}

\begin{entry}{休假}{6,11}[Radicais ⼈、⼈]
  \begin{phonetics}{休假}{xiu1 jia4}[][HSK 2]
    \definition{v.+compl.}{ter um feriado | tirar férias | sair de férias}
  \end{phonetics}
\end{entry}

\begin{entry}{休憩}{6,16}[Radicais ⼈、⼼]
  \begin{phonetics}{休憩}{xiu1qi4}
    \definition{v.}{relaxar | descansar | dar um tempo}
  \end{phonetics}
\end{entry}

\begin{entry}{休整}{6,16}[Radicais ⼈、⽁]
  \begin{phonetics}{休整}{xiu1zheng3}
    \definition{v.}{(militar) descansar e reorganizar}
  \end{phonetics}
\end{entry}

\begin{entry}{众}{6}[Radical ⼈]
  \begin{phonetics}{众}{zhong4}
    \definition*{s.}{Câmara dos Deputados, abreviação de 众议院}
    \definition{adj.}{numeroso}
    \definition{adv.}{muitos}
    \definition{s.}{multidão}
    \seeref{众议院}{zhong4yi4yuan4}
  \end{phonetics}
\end{entry}

\begin{entry}{众议院}{6,5,9}[Radicais ⼈、⾔、⾩]
  \begin{phonetics}{众议院}{zhong4yi4yuan4}
    \definition*{s.}{Casa baixa da Assembléia Bicameral | Câmara dos Deputados}
  \end{phonetics}
\end{entry}

\begin{entry}{优}{6}[Radical ⼈]
  \begin{phonetics}{优}{you1}
    \definition{adj.}{excelente | superior}
  \end{phonetics}
\end{entry}

\begin{entry}{优于}{6,3}[Radicais ⼈、⼆]
  \begin{phonetics}{优于}{you1yu2}
    \definition{v.}{superar}
  \end{phonetics}
\end{entry}

\begin{entry}{优先}{6,6}[Radicais ⼈、⼉]
  \begin{phonetics}{优先}{you1xian1}
    \definition{v.}{ter prioridade | ter precedência}
  \end{phonetics}
\end{entry}

\begin{entry}{优伶}{6,7}[Radicais ⼈、⼈]
  \begin{phonetics}{优伶}{you1ling2}
    \definition{s.}{ator}
  \end{phonetics}
\end{entry}

\begin{entry}{优秀}{6,7}[Radicais ⼈、⽲]
  \begin{phonetics}{优秀}{you1xiu4}
    \definition{adj.}{excelente | fora do comum}
  \end{phonetics}
\end{entry}

\begin{entry}{优势}{6,8}[Radicais ⼈、⼒]
  \begin{phonetics}{优势}{you1shi4}[][HSK 3]
    \definition[种,个]{s.}{vantagem; superioridade; preponderância; posição dominante}
  \end{phonetics}
\end{entry}

\begin{entry}{优质}{6,8}[Radicais ⼈、⾙]
  \begin{phonetics}{优质}{you1zhi4}
    \definition{adj.}{excelente qualidade}
  \end{phonetics}
\end{entry}

\begin{entry}{优厚}{6,9}[Radicais ⼈、⼚]
  \begin{phonetics}{优厚}{you1hou4}
    \definition{adj.}{generoso}
  \end{phonetics}
\end{entry}

\begin{entry}{优点}{6,9}[Radicais ⼈、⽕]
  \begin{phonetics}{优点}{you1dian3}[][HSK 3]
    \definition[个,项,种,些]{s.}{mérito; virtude; vantagem; ponto forte}
  \end{phonetics}
\end{entry}

\begin{entry}{优美}{6,9}[Radicais ⼈、⽺]
  \begin{phonetics}{优美}{you1mei3}
    \definition{adj.}{gracioso | fino | elegante}
  \end{phonetics}
\end{entry}

\begin{entry}{优选}{6,9}[Radicais ⼈、⾡]
  \begin{phonetics}{优选}{you1xuan3}
    \definition{v.}{otimizar}
  \end{phonetics}
\end{entry}

\begin{entry}{优格}{6,10}[Radicais ⼈、⽊]
  \begin{phonetics}{优格}{you1ge2}
    \definition{s.}{iogurte}
  \end{phonetics}
\end{entry}

\begin{entry}{优盘}{6,11}[Radicais ⼈、⽫]
  \begin{phonetics}{优盘}{you1pan2}
    \definition{s.}{unidade de memória USB}
  \seealsoref{闪存盘}{shan3cun2pan2}
  \end{phonetics}
\end{entry}

\begin{entry}{优等}{6,12}[Radicais ⼈、⽵]
  \begin{phonetics}{优等}{you1deng3}
    \definition{adj.}{excelente | de primeira linha | alta classe | da mais alta ordem, superior}
  \end{phonetics}
\end{entry}

\begin{entry}{优裕}{6,12}[Radicais ⼈、⾐]
  \begin{phonetics}{优裕}{you1yu4}
    \definition{adj.}{abundante | bastante}
    \definition{s.}{abundância}
  \end{phonetics}
\end{entry}

\begin{entry}{伙}{6}[Radical ⼈]
  \begin{phonetics}{伙}{huo3}[][HSK 4]
    \definition{clas.}{grupo; multidão; banda}
    \definition{s.}{iguaria; alimentação; refeições | parceiro; companheiro | coletivo de colegas}
    \definition{v.}{combinar; unir}
  \end{phonetics}
\end{entry}

\begin{entry}{伙伴}{6,7}[Radicais ⼈、⼈]
  \begin{phonetics}{伙伴}{huo3ban4}[][HSK 4]
    \definition[个,位,群]{s.}{parceiro; companheiro; antigo sistema militar de dez pessoas para uma fogueira, o chefe da fogueira, uma pessoa encarregada de cozinhar, com a fogueira é chamado de parceiro da fogueira, agora se refere à participação comum em uma determinada organização ou engajada em certas atividades}
  \end{phonetics}
\end{entry}

\begin{entry}{会}{6}[Radical ⼈]
  \begin{phonetics}{会}{hui4}[][HSK 1,2]
    \definition{adv.}{um momento}
    \definition{s.}{encontro | reunião}
    \definition{suf.}{união | grupo | associação}
    \definition{v.}{poder (ter a habilidade, saber como fazer) | saber | ter habilidade | saber como fazer | ser provável | ter certeza de | encontrar-se | reunir-se}
  \end{phonetics}
  \begin{phonetics}{会}{kuai4}
    \definition{s.}{contabilidade}
    \definition{v.}{equilibrar uma conta}
  \end{phonetics}
\end{entry}

\begin{entry}{会计}{6,4}[Radicais ⼈、⾔]
  \begin{phonetics}{会计}{kuai4ji4}[][HSK 4]
    \definition[个,位,名]{s.}{contabilidade | contador; contabilista; guarda-livros; pessoal que trabalha como contador}
  \end{phonetics}
\end{entry}

\begin{entry}{会议}{6,5}[Radicais ⼈、⾔]
  \begin{phonetics}{会议}{hui4yi4}[][HSK 3]
    \definition[场,届,个]{s.}{reunião; conferência | conselho; congresso}
  \end{phonetics}
\end{entry}

\begin{entry}{会员}{6,7}[Radicais ⼈、⼝]
  \begin{phonetics}{会员}{hui4 yuan2}[][HSK 3]
    \definition[位]{s.}{membro; associado | filiação}
  \end{phonetics}
\end{entry}

\begin{entry}{会首}{6,9}[Radicais ⼈、⾸]
  \begin{phonetics}{会首}{hui4shou3}
    \definition{s.}{chefe de uma sociedade | patrocinador de uma organização}
  \end{phonetics}
\end{entry}

\begin{entry}{伞}{6}[Radical ⼈]
  \begin{phonetics}{伞}{san3}[][HSK 4]
    \definition*{s.}{sobrenome San}
    \definition[把]{s.}{guarda-chuva; proteção contra chuva ou sol | algo que tem o formato de um guarda-chuva}
  \end{phonetics}
\end{entry}

\begin{entry}{伟}{6}[Radical ⼈]
  \begin{phonetics}{伟}{wei3}
    \definition{adj.}{grande | ótimo}
  \end{phonetics}
\end{entry}

\begin{entry}{伟大}{6,3}[Radicais ⼈、⼤]
  \begin{phonetics}{伟大}{wei3da4}[][HSK 3]
    \definition{adj.}{ótimo; importante (contribuição, etc.) | ótimo; magnífico; digno da maior admiração}
  \end{phonetics}
\end{entry}

\begin{entry}{传}{6}[Radical ⼈]
  \begin{phonetics}{传}{chuan2}[][HSK 3]
    \definition{v.}{passar; passar adiante | passar adiante; legar; passar de \dots para \dots | transmitir (conhecimento, habilidade, etc.); comunicar; ensinar | espalhar; propagar | transmitir; conduzir; transferir | transmitir; expressar |convocar | infectar; ser contagioso}
  \end{phonetics}
  \begin{phonetics}{传}{zhuan4}
    \definition{s.}{comentários sobre clássicos | biografia | romances sobre eventos históricos}
  \end{phonetics}
\end{entry}

\begin{entry}{传来}{6,7}[Radicais ⼈、⽊]
  \begin{phonetics}{传来}{chuan2 lai2}[][HSK 3]
    \definition{v.}{(um som) passar | (notícias) chegar}
  \end{phonetics}
\end{entry}

\begin{entry}{传承}{6,8}[Radicais ⼈、⼿]
  \begin{phonetics}{传承}{chuan2cheng2}
    \definition{s.}{herança | tradição continuada}
    \definition{v.}{transmitir (para as gerações futuras) | passar adiante (desde os tempos antigos)}
  \end{phonetics}
\end{entry}

\begin{entry}{传给}{6,9}[Radicais ⼈、⽷]
  \begin{phonetics}{传给}{chuan2gei3}
    \definition{v.}{passar para | transferir para | entregar a}
  \end{phonetics}
\end{entry}

\begin{entry}{传统}{6,9}[Radicais ⼈、⽷]
  \begin{phonetics}{传统}{chuan2tong3}[][HSK 4]
    \definition{adj.}{tradicional; histórico; transmitido de geração em geração | antiquado, conservador e fora de sintonia com os tempos}
    \definition[个]{s.}{tradição; costume; fatores sociais, como costumes, moral, ideias, estilos, artes, instituições etc., que são transmitidos de uma geração para outra e que são característicos da sociedade}
  \end{phonetics}
\end{entry}

\begin{entry}{传说}{6,9}[Radicais ⼈、⾔]
  \begin{phonetics}{传说}{chuan2shuo1}[][HSK 3]
    \definition{s.}{lenda | conto popular | folclore}
    \definition{v.}{dizer que; ser dito; passar de boca em boca}
  \end{phonetics}
\end{entry}

\begin{entry}{传真}{6,10}[Radicais ⼈、⼗]
  \begin{phonetics}{传真}{chuan2zhen1}
    \definition{s.}{fax, facsímile}
  \end{phonetics}
\end{entry}

\begin{entry}{传播}{6,15}[Radicais ⼈、⼿]
  \begin{phonetics}{传播}{chuan2bo1}[][HSK 3]
    \definition{v.}{espalhar; difundir; propagar; disseminar}
  \end{phonetics}
\end{entry}

\begin{entry}{伤}{6}[Radical ⼈]
  \begin{phonetics}{伤}{shang1}[][HSK 3]
    \definition*{s.}{sobrenome Shang}
    \definition{s.}{ferida; ferimento}
    \definition{v.}{ferir; machucar | estar angustiado | enjoar de algo; desenvolver aversão a algo. |ser prejudicial a; entravar}
  \end{phonetics}
\end{entry}

\begin{entry}{伤心}{6,4}[Radicais ⼈、⼼]
  \begin{phonetics}{伤心}{shang1xin1}[][HSK 3]
    \definition{v.+compl.}{estar triste; lamentar; estar com o coração partido}
  \end{phonetics}
\end{entry}

\begin{entry}{伤害}{6,10}[Radicais ⼈、⼧]
  \begin{phonetics}{伤害}{shang1hai4}[][HSK 4]
    \definition{v.}{ferir; prejudicar; machucar; magoar; causar danos físicos ou mentais}
  \end{phonetics}
\end{entry}

\begin{entry}{伦敦}{6,12}[Radicais ⼈、⽁]
  \begin{phonetics}{伦敦}{lun2dun1}
    \definition*{s.}{Londres}
  \end{phonetics}
\end{entry}

\begin{entry}{似乎}{6,5}[Radicais ⼈、⼃]
  \begin{phonetics}{似乎}{si4hu1}[][HSK 4]
    \definition{adv.}{como se; aparentemente; se parece como}
  \end{phonetics}
\end{entry}

\begin{entry}{似的}{6,8}[Radicais ⼈、⽩]
  \begin{phonetics}{似的}{shi4de5}[][HSK 4]
    \definition{part.}{como; como\dots como; como se (embora); usada após uma palavra ou frase para indicar uma semelhança com algo ou uma situação | usada para indicar alto grau}
  \end{phonetics}
\end{entry}

\begin{entry}{似曾相识}{6,12,9,7}[Radicais ⼈、⽈、⽬、⾔]
  \begin{phonetics}{似曾相识}{si4ceng2xiang1shi2}
    \definition{s.}{\emph{déjà vu} (a experiência de ver exatamente a mesma situação pela segunda vez) | situação aparentemente familiar}
  \end{phonetics}
\end{entry}

\begin{entry}{充分}{6,4}[Radicais ⼉、⼑]
  \begin{phonetics}{充分}{chong1fen4}[][HSK 4]
    \definition{adj.}{cheio; amplo; abundante; suficiente; adequado}
    \definition{adv.}{totalmente; até o fim}
  \end{phonetics}
\end{entry}

\begin{entry}{充电}{6,5}[Radicais ⼉、⽥]
  \begin{phonetics}{充电}{chong1 dian4}[][HSK 4]
    \definition{v.}{carregar (uma bateria); conectar uma fonte de alimentação CC aos terminais da bateria para recarregar a bateria | relaxar; passar o tempo livre; ``recarregar as baterias''; estudar para adquirir mais conhecimento; reabastecer (ou ampliar) o conhecimento; metaforicamente falando, para reabastecer a força física e a energia por meio do descanso e da recreação; também metaforicamente falando, para reabastecer novos conhecimentos e desenvolver novas habilidades por meio do reaprendizado}
  \end{phonetics}
\end{entry}

\begin{entry}{充电器}{6,5,16}[Radicais ⼉、⽥、⼝]
  \begin{phonetics}{充电器}{chong1dian4qi4}[][HSK 4]
    \definition{s.}{carregador de bateria; dispositivo para alimentar uma bateria com energia, forçando uma corrente através dela}
  \end{phonetics}
\end{entry}

\begin{entry}{充满}{6,13}[Radicais ⼉、⽔]
  \begin{phonetics}{充满}{chong1man3}[][HSK 3]
    \definition{v.}{preencher | encher-se de; transbordar de; permear-se de}
  \end{phonetics}
\end{entry}

\begin{entry}{兆}{6}[Radical ⼉]
  \begin{phonetics}{兆}{zhao4}
    \definition{num.}{trilhão}
  \end{phonetics}
\end{entry}

\begin{entry}{先}{6}[Radical ⼉]
  \begin{phonetics}{先}{xian1}[][HSK 1]
    \definition{adv.}{em primeiro lugar | primeiramente | antes do tempo | de antemão}
  \end{phonetics}
\end{entry}

\begin{entry}{先不先}{6,4,6}[Radicais ⼉、⼀、⼉]
  \begin{phonetics}{先不先}{xian1bu4xian1}
    \definition{adv.}{(dialeto) antes de tudo | em primeiro lugar}
  \end{phonetics}
\end{entry}

\begin{entry}{先天}{6,4}[Radicais ⼉、⼤]
  \begin{phonetics}{先天}{xian1tian1}
    \definition{adj.}{congênito | inato | natural}
    \definition{s.}{período embrionário}
  \end{phonetics}
\end{entry}

\begin{entry}{先生}{6,5}[Radicais ⼉、⽣]
  \begin{phonetics}{先生}{xian1sheng5}[][HSK 1]
    \definition[位]{s.}{senhor | marido | professor | (dialeto) doutor}
  \end{phonetics}
\end{entry}

\begin{entry}{先有}{6,6}[Radicais ⼉、⽉]
  \begin{phonetics}{先有}{xian1you3}
    \definition{adj.}{preexistente | anterior}
  \end{phonetics}
\end{entry}

\begin{entry}{先进}{6,7}[Radicais ⼉、⾡]
  \begin{phonetics}{先进}{xian1jin4}[][HSK 3]
    \definition{adj.}{avançado}
    \definition{s.}{indivíduo avançado; grupo avançado}
  \end{phonetics}
\end{entry}

\begin{entry}{先到先得}{6,8,6,11}[Radicais ⼉、⼑、⼉、⼻]
  \begin{phonetics}{先到先得}{xian1dao4xian1de2}
    \definition{expr.}{primeiro a chegar | primeiro a ser servido}
  \end{phonetics}
\end{entry}

\begin{entry}{先烈}{6,10}[Radicais ⼉、⽕]
  \begin{phonetics}{先烈}{xian1lie4}
    \definition{s.}{mártir}
  \end{phonetics}
\end{entry}

\begin{entry}{先验}{6,10}[Radicais ⼉、⾺]
  \begin{phonetics}{先验}{xian1yan4}
    \definition{adj.}{(filosofia) a priori}
  \end{phonetics}
\end{entry}

\begin{entry}{先期}{6,12}[Radicais ⼉、⽉]
  \begin{phonetics}{先期}{xian1qi1}
    \definition{adv.}{antecipadamente}
    \definition{s.}{prematuro | \emph{front-end}}
  \end{phonetics}
\end{entry}

\begin{entry}{光}{6}[Radical ⼉]
  \begin{phonetics}{光}{guang1}[][HSK 3]
    \definition*{s.}{sobrenome Guang}
    \definition{adj.}{suave; brilhante | nu; despido; descoberto | esgotado; sem nada sobrando | glorioso; gracioso | brilhante}
    \definition{adv.}{somente; sozinho; meramente}
    \definition{s.}{luz; raio | cenário | honra; glória; brilho | claridade | favor; graça | momento | corpo celeste}
    \definition{v.}{glorificar; recuperar; reconquistar | estar nu | brilhar}
  \end{phonetics}
\end{entry}

\begin{entry}{光污染}{6,6,9}[Radicais ⼉、⽔、⽊]
  \begin{phonetics}{光污染}{guang1 wu1ran3}
    \definition{s.}{poluição luminosa}
  \end{phonetics}
\end{entry}

\begin{entry}{光明}{6,8}[Radicais ⼉、⽇]
  \begin{phonetics}{光明}{guang1ming2}[][HSK 3]
    \definition{adj.}{brilhante | ingênuo | justo; honesto}
    \definition{s.}{luz}
  \end{phonetics}
\end{entry}

\begin{entry}{光临}{6,9}[Radicais ⼉、⼁]
  \begin{phonetics}{光临}{guang1lin2}[][HSK 4]
    \definition{v.}{honrar com sua presença, uma palavra de honra, usada para dizer que um convidado chegou}
  \end{phonetics}
\end{entry}

\begin{entry}{光盘}{6,11}[Radicais ⼉、⽫]
  \begin{phonetics}{光盘}{guang1pan2}[][HSK 4]
    \definition[片,张]{s.}{CD; disco compacto; um disco circular feito de plástico rígido composto que usa um laser para registrar e ler informações}
  \end{phonetics}
\end{entry}

\begin{entry}{光槃}{6,14}[Radicais ⼉、⽊]
  \begin{phonetics}{光槃}{guang1pan2}
    \variantof{光盘}
  \end{phonetics}
\end{entry}

\begin{entry}{全}{6}[Radical ⼊]
  \begin{phonetics}{全}{quan2}[][HSK 2]
    \definition*{s.}{sobrenome Quan}
    \definition{adv.}{completamente | totalmente}
  \end{phonetics}
\end{entry}

\begin{entry}{全场}{6,6}[Radicais ⼊、⼟]
  \begin{phonetics}{全场}{quan2 chang3}[][HSK 3]
    \definition{s.}{toda a audiência; todos os presentes | corte (de justiça) inteira}
  \end{phonetics}
\end{entry}

\begin{entry}{全年}{6,6}[Radicais ⼊、⼲]
  \begin{phonetics}{全年}{quan2 nian2}[][HSK 2]
    \definition{s.}{anual}
  \end{phonetics}
\end{entry}

\begin{entry}{全体}{6,7}[Radicais ⼊、⼈]
  \begin{phonetics}{全体}{quan2 ti3}[][HSK 2]
    \definition{s.}{tudo | todo | inteiro}
  \end{phonetics}
\end{entry}

\begin{entry}{全身}{6,7}[Radicais ⼊、⾝]
  \begin{phonetics}{全身}{quan2 shen1}[][HSK 2]
    \definition{s.}{corpo inteiro | todo (o corpo)}
  \end{phonetics}
\end{entry}

\begin{entry}{全国}{6,8}[Radicais ⼊、⼞]
  \begin{phonetics}{全国}{quan2 guo2}[][HSK 2]
    \definition{s.}{nação | a nação inteira | o país inteiro | todo o país}
  \end{phonetics}
\end{entry}

\begin{entry}{全面}{6,9}[Radicais ⼊、⾯]
  \begin{phonetics}{全面}{quan2mian4}[][HSK 3]
    \definition{adj.}{geral; completo}
    \definition{s.}{todos os aspectos; cada aspecto}
  \seealsoref{片面}{pian4mian4}
  \end{phonetics}
\end{entry}

\begin{entry}{全家}{6,10}[Radicais ⼊、⼧]
  \begin{phonetics}{全家}{quan2 jia1}[][HSK 2]
    \definition{s.}{a família inteira | toda a família}
  \end{phonetics}
\end{entry}

\begin{entry}{全部}{6,10}[Radicais ⼊、⾢]
  \begin{phonetics}{全部}{quan2bu4}[][HSK 2]
    \definition{adv.}{todo, todos}
  \end{phonetics}
\end{entry}

\begin{entry}{全球}{6,11}[Radicais ⼊、⽟]
  \begin{phonetics}{全球}{quan2 qiu2}[][HSK 3]
    \definition{adj.}{global}
    \definition{s.}{o mundo inteiro}
  \end{phonetics}
\end{entry}

\begin{entry}{全职}{6,11}[Radicais ⼊、⽿]
  \begin{phonetics}{全职}{quan2zhi2}
    \definition{s.}{período integral | tempo inteiro | (trabalho) \emph{full-time}}
  \end{phonetics}
\end{entry}

\begin{entry}{共}{6}[Radical ⼋]
  \begin{phonetics}{共}{gong4}[][HSK 4]
    \definition*{s.}{Abreviação de Partido Comunista | sobrenome Gong}
    \definition{adj.}{conjunto; mútuo; geral; comum; o mesmo para todos}
    \definition{adv.}{juntos; juntamente; conjuntamente | em sua totalidade; em todos}
    \definition{v.}{compartilhar com; empreender ou realizar em conjunto}
  \end{phonetics}
\end{entry}

\begin{entry}{共产}{6,6}[Radicais ⼋、⼇]
  \begin{phonetics}{共产}{gong4chan3}
    \definition{adj.}{comunista}
    \definition{s.}{comunismo}
  \end{phonetics}
\end{entry}

\begin{entry}{共产党}{6,6,10}[Radicais ⼋、⼇、⼉]
  \begin{phonetics}{共产党}{gong4chan3dang3}
    \definition*{s.}{Partido Comunista}
  \end{phonetics}
\end{entry}

\begin{entry}{共同}{6,6}[Radicais ⼋、⼝]
  \begin{phonetics}{共同}{gong4tong2}[][HSK 3]
    \definition{adj.}{comum; compartilhado; colaborativo}
    \definition{adv.}{juntos; conjuntamente}
  \end{phonetics}
\end{entry}

\begin{entry}{共同体}{6,6,7}[Radicais ⼋、⼝、⼈]
  \begin{phonetics}{共同体}{gong4tong2ti3}
    \definition{s.}{comunidade}
  \end{phonetics}
\end{entry}

\begin{entry}{共有}{6,6}[Radicais ⼋、⽉]
  \begin{phonetics}{共有}{gong4 you3}[][HSK 3]
    \definition{v.}{ter completamente; compartilhar; possuir (por todos)}
  \end{phonetics}
\end{entry}

\begin{entry}{兲}{6}[Radical ⼋]
  \begin{phonetics}{兲}{tian1}
    \variantof{天}
  \end{phonetics}
\end{entry}

\begin{entry}{关}{6}[Radical ⼋]
  \begin{phonetics}{关}{guan1}[][HSK 1,4]
    \definition*{s.}{sobrenome Guan}
    \definition{s.}{passagem; ponto de controle | alfândega; escritórios de cobrança de impostos para exportação e importação de mercadorias | ponto de inflexão ou barreira; ponto de virada ou dificuldade | momento crítico; mecanismo}
    \definition{v.}{fechar; encerrar; amarrar algo | fechar; trancar | encerrar; sair do mercado; falir | conceder ou sacar o pagamento de um salário | desligar | envolver; preocupar-se; conectar-se}
  \end{phonetics}
\end{entry}

\begin{entry}{关上}{6,3}[Radicais ⼋、⼀]
  \begin{phonetics}{关上}{guan1 shang5}[][HSK 1]
    \definition{v.}{fechar (uma porta) | fechar | desligar (luz, equipamento elétrico etc.)}
  \end{phonetics}
\end{entry}

\begin{entry}{关于}{6,3}[Radicais ⼋、⼆]
  \begin{phonetics}{关于}{guan1yu2}[][HSK 4]
    \definition{prep.}{sobre; relativo a; pertencente a; uma questão de; com relação a}
  \end{phonetics}
\end{entry}

\begin{entry}{关心}{6,4}[Radicais ⼋、⼼]
  \begin{phonetics}{关心}{guan1xin1}[][HSK 2]
    \definition{v.}{cuidar de | preocupar-se com | expressar interesse em | mostrar solicitude por}
  \end{phonetics}
\end{entry}

\begin{entry}{关机}{6,6}[Radicais ⼋、⽊]
  \begin{phonetics}{关机}{guan1 ji1}[][HSK 2]
    \definition{v.}{encerrar | finalizar | desligar}
  \end{phonetics}
\end{entry}

\begin{entry}{关闭}{6,6}[Radicais ⼋、⾨]
  \begin{phonetics}{关闭}{guan1bi4}[][HSK 4]
    \definition{v.}{fechar | (empresa) falir}
  \end{phonetics}
\end{entry}

\begin{entry}{关系}{6,7}[Radicais ⼋、⽷]
  \begin{phonetics}{关系}{guan1xi5}[][HSK 3]
    \definition[个,种]{s.}{relações; conexões; relacionamento | consequência; impacto; significado | causa; razão (geralmente usado com 由于 ou 因为) | credenciais que mostram filiação a uma organização}
    \definition{v.}{preocupar; afetar; ter influência sobre; ter a ver com}
  \seealsoref{因为}{yin1wei4}
  \seealsoref{由于}{you2yu2}
  \end{phonetics}
\end{entry}

\begin{entry}{关注}{6,8}[Radicais ⼋、⽔]
  \begin{phonetics}{关注}{guan1 zhu4}[][HSK 3]
    \definition{s.}{preocupação; interesse; atenção}
    \definition{v.}{prestar atenção em; seguir algo de perto; seguir (nas redes sociais)}
  \end{phonetics}
\end{entry}

\begin{entry}{关键}{6,13}[Radicais ⼋、⾦]
  \begin{phonetics}{关键}{guan1jian4}
    \definition{adj.}{crucial}
    \definition[个]{s.}{ponto crucial | chave}
  \end{phonetics}
\end{entry}

\begin{entry}{兴}{6}[Radical ⼋]
  \begin{phonetics}{兴}{xing1}
    \definition*{s.}{sobrenome Xing}
    \definition{adv.}{talvez (dialeto)}
    \definition{v.}{subir | florescer | tornar-se popular | começar | encorajar | levantar-se | (frequentemente usado em negativas) permitir (dialeto)}
  \end{phonetics}
  \begin{phonetics}{兴}{xing4}
    \definition{s.}{sentimento ou desejo de fazer algo | interesse em algo | excitação}
  \end{phonetics}
\end{entry}

\begin{entry}{兴趣}{6,15}[Radicais ⼋、⾛]
  \begin{phonetics}{兴趣}{xing4qu4}
    \definition[个]{s.}{interesse (desejo de conhecer sobre alguma coisa ou coisa no qual está interessado) | \emph{hobby}}
  \end{phonetics}
\end{entry}

\begin{entry}{再}{6}[Radical ⼌]
  \begin{phonetics}{再}{zai4}[][HSK 1]
    \definition{adv.}{de novo | outra vez | uma segunda vez | não importa como\dots (seguido por um adjetivo ou verbo, e então (normalmente) 也 ou 都 para dar ênfase)}
  \end{phonetics}
\end{entry}

\begin{entry}{再三}{6,3}[Radicais ⼌、⼀]
  \begin{phonetics}{再三}{zai4san1}
    \definition{adv.}{de novo e de novo | repetidamente}
  \end{phonetics}
\end{entry}

\begin{entry}{再不}{6,4}[Radicais ⼌、⼀]
  \begin{phonetics}{再不}{zai4bu4}
    \definition{adv.}{nunca mais}
  \end{phonetics}
\end{entry}

\begin{entry}{再见}{6,4}[Radicais ⼌、⾒]
  \begin{phonetics}{再见}{zai4jian4}[][HSK 1]
    \definition{v.}{adeus | até à vista | até à próxima | até logo}
  \end{phonetics}
\end{entry}

\begin{entry}{再发}{6,5}[Radicais ⼌、⼜]
  \begin{phonetics}{再发}{zai4fa1}
    \definition{v.}{reenviar}
  \end{phonetics}
\end{entry}

\begin{entry}{再生}{6,5}[Radicais ⼌、⽣]
  \begin{phonetics}{再生}{zai4sheng1}
    \definition{s.}{reciclagem | regeneração}
    \definition{v.}{reciclar | renascer | regenerar}
  \end{phonetics}
\end{entry}

\begin{entry}{再审}{6,8}[Radicais ⼌、⼧]
  \begin{phonetics}{再审}{zai4shen3}
    \definition{s.}{novo julgamento | revisão}
    \definition{v.}{ouvir um caso novamente}
  \end{phonetics}
\end{entry}

\begin{entry}{再者}{6,8}[Radicais ⼌、⽼]
  \begin{phonetics}{再者}{zai4zhe3}
    \definition{conj.}{além do mais | além disso}
  \end{phonetics}
\end{entry}

\begin{entry}{再育}{6,8}[Radicais ⼌、⾁]
  \begin{phonetics}{再育}{zai4yu4}
    \definition{v.}{aumentar | multiplicar | proliferar}
  \end{phonetics}
\end{entry}

\begin{entry}{再临}{6,9}[Radicais ⼌、⼁]
  \begin{phonetics}{再临}{zai4lin2}
    \definition{v.}{vir de novo}
  \end{phonetics}
\end{entry}

\begin{entry}{再度}{6,9}[Radicais ⼌、⼴]
  \begin{phonetics}{再度}{zai4du4}
    \definition{adv.}{outra vez | mais uma vez}
  \end{phonetics}
\end{entry}

\begin{entry}{再说}{6,9}[Radicais ⼌、⾔]
  \begin{phonetics}{再说}{zai4shuo1}
    \definition{conj.}{além do mais | além disso | o que mais}
    \definition{v.}{adiar uma discussão para mais tarde | dizer novamente}
  \end{phonetics}
\end{entry}

\begin{entry}{再读}{6,10}[Radicais ⼌、⾔]
  \begin{phonetics}{再读}{zai4du2}
    \definition{v.}{ler novamente | rever (uma lição, etc.)}
  \end{phonetics}
\end{entry}

\begin{entry}{军人}{6,2}[Radicais ⼍、⼈]
  \begin{phonetics}{军人}{jun1ren2}
    \definition{s.}{soldado | pessoal militar}
  \end{phonetics}
\end{entry}

\begin{entry}{军装}{6,12}[Radicais ⼍、⾐]
  \begin{phonetics}{军装}{jun1zhuang1}
    \definition{s.}{uniforme militar}
  \end{phonetics}
\end{entry}

\begin{entry}{农业}{6,5}[Radicais ⼍、⼀]
  \begin{phonetics}{农业}{nong2ye4}[][HSK 3]
    \definition{s.}{agricultura; lavoura}
  \end{phonetics}
\end{entry}

\begin{entry}{农民}{6,5}[Radicais ⼍、⽒]
  \begin{phonetics}{农民}{nong2min2}[][HSK 3]
    \definition[个,位]{s.}{fazendeiro; camponês; campesinato}
  \end{phonetics}
\end{entry}

\begin{entry}{农村}{6,7}[Radicais ⼍、⽊]
  \begin{phonetics}{农村}{nong2cun1}[][HSK 3]
    \definition[个]{s.}{aldeia; campo; área rural}
  \end{phonetics}
\end{entry}

\begin{entry}{冰}{6}[Radical ⼎]
  \begin{phonetics}{冰}{bing1}[][HSK 4]
    \definition{adj.}{frio (pessoa)| hostil}
    \definition[块]{s.}{gelo; água em estado sólido |  (gíria) metanfetamina}
    \definition{v.}{colocar gelo; colocar gelo ao redor; colocar no gelo; resfriar objetos com gelo ou água fria | sentir frio}
  \end{phonetics}
\end{entry}

\begin{entry}{冰天雪地}{6,4,11,6}[Radicais ⼎、⼤、⾬、⼟]
  \begin{phonetics}{冰天雪地}{bing1tian1-xue3di4}
    \definition{expr.}{um mundo de gelo e neve}
  \end{phonetics}
\end{entry}

\begin{entry}{冰球}{6,11}[Radicais ⼎、⽟]
  \begin{phonetics}{冰球}{bing1qiu2}
    \definition{s.}{hóquei no gelo}
  \end{phonetics}
\end{entry}

\begin{entry}{冰雪}{6,11}[Radicais ⼎、⾬]
  \begin{phonetics}{冰雪}{bing1 xue3}[][HSK 4]
    \definition{adj.}{puro como gelo e neve; descreve uma pessoa como pura}
    \definition{s.}{gelo e neve}
  \end{phonetics}
\end{entry}

\begin{entry}{冰棍}{6,12}[Radicais ⼎、⽊]
  \begin{phonetics}{冰棍}{bing1gun4}
    \definition[根]{s.}{picolé}
  \end{phonetics}
\end{entry}

\begin{entry}{冰箱}{6,15}[Radicais ⼎、⾋]
  \begin{phonetics}{冰箱}{bing1xiang1}[][HSK 4]
    \definition[台,个]{s.}{geladeira; freezer; refrigerador; aparelhos para congelar alimentos ou medicamentos com gelo para mantê-los frios}
  \end{phonetics}
\end{entry}

\begin{entry}{冰激凌}{6,16,10}[Radicais ⼎、⽔、⼎]
  \begin{phonetics}{冰激凌}{bing1ji1ling2}
    \definition{s.}{sorvete}
  \end{phonetics}
\end{entry}

\begin{entry}{冰糕}{6,16}[Radicais ⼎、⽶]
  \begin{phonetics}{冰糕}{bing1gao1}
    \definition{s.}{sorvete | picolé}
  \end{phonetics}
\end{entry}

\begin{entry}{冲}{6}[Radical ⼎]
  \begin{phonetics}{冲}{chong1}[][HSK 4]
    \definition{s.}{via pública; local importante; via de passagem; via local importante | um trecho de planície em uma área montanhosa | oposição; os planetas externos orbitam até ficarem alinhados com a Terra e o Sol, e a Terra está no meio}
    \definition{v.}{atacar; apressar; correr; passar rapidamente; passar por um obstáculo | colidir; chocar; bater | despejar água fervente sobre | enxaguar; dar descarga; lavar | revelar (filme) | neutralizar a má sorte}
  \end{phonetics}
  \begin{phonetics}{冲}{chong4}
    \definition{adj.}{poderoso; com vigor; com muita força; vigoroso | forte; odor forte e pungente (olfato)}
    \definition{adv.}{de frente; em direção a | na força de; com base em; em virtude de}
    \definition{v.}{fazer face a (algo ou alguém) | estampar (máquina de estamparia)}
  \end{phonetics}
\end{entry}

\begin{entry}{冲突}{6,9}[Radicais ⼎、⽳]
  \begin{phonetics}{冲突}{chong1tu1}
    \definition{s.}{conflito | choque de forças opostas | colisão (de interesses)}
  \end{phonetics}
\end{entry}

\begin{entry}{冲浪}{6,10}[Radicais ⼎、⽔]
  \begin{phonetics}{冲浪}{chong1lang4}
    \definition{s.}{surfe}
    \definition{v.}{surfar}
  \end{phonetics}
\end{entry}

\begin{entry}{冲锋}{6,12}[Radicais ⼎、⾦]
  \begin{phonetics}{冲锋}{chong1feng1}
    \definition{v.}{cobrar | tomar de assalto}
  \end{phonetics}
\end{entry}

\begin{entry}{决心}{6,4}[Radicais ⼎、⼼]
  \begin{phonetics}{决心}{jue2xin1}[][HSK 3]
    \definition{s.}{resolução; determinação}
    \definition{v.}{secidir-se; decidir fazer algo e não vacilar nem mudar de ideia}
  \end{phonetics}
\end{entry}

\begin{entry}{决定}{6,8}[Radicais ⼎、⼧]
  \begin{phonetics}{决定}{jue2ding4}[][HSK 3]
    \definition{adj.}{decisivo}
    \definition{s.}{decisão; resolução}
    \definition{v.}{decidir; determinar | decidir; resolver; tomar uma decisão}
  \end{phonetics}
\end{entry}

\begin{entry}{决赛}{6,14}[Radicais ⼎、⾙]
  \begin{phonetics}{决赛}{jue2sai4}[][HSK 3]
    \definition{s.}{finais (de uma competição)}
  \end{phonetics}
\end{entry}

\begin{entry}{划}{6}[Radical ⼑]
  \begin{phonetics}{划}{hua2}[][HSK 4]
    \definition{adj.}{rentável; vale (o esforço); compensa (fazer alguma coisa)}
    \definition{v.}{remar | ser vantajoso para alguém; ser uma pechincha | arranhar; cortar a superfície de; cortar em outra coisa com um objeto pontiagudo | arranhar; golpear;  esfregar uma coisa ou varrer sobre outra}
  \end{phonetics}
  \begin{phonetics}{划}{hua4}[][HSK 4]
    \definition{s.}{traço de um caracter chinês}
    \definition{v.}{delimitar; diferenciar; delinear | transferir; ceder | planejar; programar | desenhar; marcar; delinear; fazer linhas ou escrever como marcadores com uma caneta ou objeto semelhante a uma caneta}
  \end{phonetics}
\end{entry}

\begin{entry}{划船}{6,11}[Radicais ⼑、⾈]
  \begin{phonetics}{划船}{hua2 chuan2}[][HSK 3]
    \definition[次,回]{s.}{remo (ato de remar); passeios de barco}
    \definition{v.}{remar um barco}
  \end{phonetics}
\end{entry}

\begin{entry}{划艇}{6,12}[Radicais ⼑、⾈]
  \begin{phonetics}{划艇}{hua2ting3}
    \definition{s.}{barco a remo}
  \end{phonetics}
\end{entry}

\begin{entry}{列}{6}[Radical ⼑]
  \begin{phonetics}{列}{lie4}[][HSK 4]
    \definition{v.}{organizar; formar uma linha; alinhar | listar; inserir em uma lista}
  \end{phonetics}
\end{entry}

\begin{entry}{列入}{6,2}[Radicais ⼑、⼊]
  \begin{phonetics}{列入}{lie4 ru4}[][HSK 4]
    \definition{v.}{incluir em uma lista}
  \end{phonetics}
\end{entry}

\begin{entry}{列为}{6,4}[Radicais ⼑、⼂]
  \begin{phonetics}{列为}{lie4 wei2}[][HSK 4]
    \definition{v.}{ser classificado como; ser listado como}
  \end{phonetics}
\end{entry}

\begin{entry}{列车}{6,4}[Radicais ⼑、⾞]
  \begin{phonetics}{列车}{lie4che1}[][HSK 4]
    \definition{s.}{trem; trem em uma composição contínua, puxado por uma locomotiva e equipado com uma tripulação e marcações prescritas; geralmente um trem de passageiros}
  \end{phonetics}
\end{entry}

\begin{entry}{刘}{6}[Radical ⼑]
  \begin{phonetics}{刘}{liu2}
    \definition*{s.}{sobrenome Liu}
    \definition{s.}{(clássico) um tipo de machado de batalha}
    \definition{v.}{matar}
  \end{phonetics}
\end{entry}

\begin{entry}{刚}{6}[Radical ⼑]
  \begin{phonetics}{刚}{gang1}[][HSK 2]
    \definition{adj.}{duro (sentido de difícil) | forte}
    \definition{adv.}{apenas | exatamente | há pouco tempo | por muito pouco | assim que}
  \end{phonetics}
\end{entry}

\begin{entry}{刚才}{6,3}[Radicais ⼑、⼿]
  \begin{phonetics}{刚才}{gang1cai2}[][HSK 2]
    \definition{adv.}{ainda agora | há pouco tempo}
  \end{phonetics}
\end{entry}

\begin{entry}{刚刚}{6,6}[Radicais ⼑、⼑]
  \begin{phonetics}{刚刚}{gang1 gang1}[][HSK 2]
    \definition{adv.}{apenas | apenas agora | um momento atrás | por muito pouco}
  \end{phonetics}
\end{entry}

\begin{entry}{创业}{6,5}[Radicais ⼑、⼀]
  \begin{phonetics}{创业}{chuang4ye4}[][HSK 3]
    \definition{s.}{empreendedorismo}
    \definition{v.}{começar um empreendimento; iniciar um negócio, uma empresa | esculpir}
  \end{phonetics}
\end{entry}

\begin{entry}{创作}{6,7}[Radicais ⼑、⼈]
  \begin{phonetics}{创作}{chuang4zuo4}[][HSK 3]
    \definition[个]{s.}{criação; trabalho criativo}
    \definition{v.}{escrever; criar; produzir; compor}
  \end{phonetics}
\end{entry}

\begin{entry}{创造}{6,10}[Radicais ⼑、⾡]
  \begin{phonetics}{创造}{chuang4zao4}[][HSK 3]
    \definition{s.}{criação; inovação}
    \definition{v.}{criar; produzir; trazer à tona}
  \end{phonetics}
\end{entry}

\begin{entry}{创意}{6,13}[Radicais ⼑、⼼]
  \begin{phonetics}{创意}{chuang4yi4}
    \definition{adj.}{criativo}
    \definition{s.}{criatividade}
  \end{phonetics}
\end{entry}

\begin{entry}{创新}{6,13}[Radicais ⼑、⽄]
  \begin{phonetics}{创新}{chuang4xin1}[][HSK 3]
    \definition[个,种,次]{s.}{inovação}
    \definition{v.}{trazer novas ideias; inovar; abrir novos caminhos; criar algo novo}
  \end{phonetics}
\end{entry}

\begin{entry}{动}{6}[Radical ⼒]
  \begin{phonetics}{动}{dong4}[][HSK 1]
    \definition{v.}{mover | movimentar}
  \end{phonetics}
\end{entry}

\begin{entry}{动人}{6,2}[Radicais ⼒、⼈]
  \begin{phonetics}{动人}{dong4 ren2}[][HSK 3]
    \definition{adj.}{em movimento; tocando}
  \end{phonetics}
\end{entry}

\begin{entry}{动力}{6,2}[Radicais ⼒、⼒]
  \begin{phonetics}{动力}{dong4li4}
    \definition{s.}{poder; força motriz | ímpeto; força motriz (ou propulsora)}
  \end{phonetics}
\end{entry}

\begin{entry}{动作}{6,7}[Radicais ⼒、⼈]
  \begin{phonetics}{动作}{dong4zuo4}[][HSK 1]
    \definition[个]{s.}{movimento | ação}
    \definition{v.}{mover | agir}
  \end{phonetics}
\end{entry}

\begin{entry}{动身}{6,7}[Radicais ⼒、⾝]
  \begin{phonetics}{动身}{dong4shen1}
    \definition{v.+compl.}{fazer uma jornada | começar uma jornada | partir | partir em uma jornada | sair (para um lugar distante)}
  \end{phonetics}
\end{entry}

\begin{entry}{动物}{6,8}[Radicais ⼒、⽜]
  \begin{phonetics}{动物}{dong4wu4}[][HSK 2]
    \definition[只,群,个]{s.}{animal}
  \end{phonetics}
\end{entry}

\begin{entry}{动物园}{6,8,7}[Radicais ⼒、⽜、⼞]
  \begin{phonetics}{动物园}{dong4 wu4 yuan2}[][HSK 2]
    \definition[个]{s.}{jardim zoológico | zoo}
  \end{phonetics}
\end{entry}

\begin{entry}{动画片}{6,8,4}[Radicais ⼒、⽥、⽚]
  \begin{phonetics}{动画片}{dong4hua4pian4}[][HSK 4]
    \definition[部]{s.}{desenho animado; animações; filme de animação}
  \end{phonetics}
\end{entry}

\begin{entry}{动感}{6,13}[Radicais ⼒、⼼]
  \begin{phonetics}{动感}{dong4gan3}
    \definition{adj.}{dinâmica | vívida}
    \definition{adv.}{dinamicamente}
    \definition{s.}{senso de movimento (geralmente em uma obra de arte estática)}
  \end{phonetics}
\end{entry}

\begin{entry}{动摇}{6,13}[Radicais ⼒、⼿]
  \begin{phonetics}{动摇}{dong4 yao2}[][HSK 4]
    \definition{adj.}{instável}
    \definition{v.}{ondular; pairar; agitar; balançar; sacudir | hesitar; vacilar; esmorecer; abalar}
  \end{phonetics}
\end{entry}

\begin{entry}{动漫}{6,14}[Radicais ⼒、⽔]
  \begin{phonetics}{动漫}{dong4man4}
    \definition{s.}{desenhos animados | quadrinhos | anime | mangá}
  \end{phonetics}
\end{entry}

\begin{entry}{匈奴}{6,5}[Radicais ⼓、⼥]
  \begin{phonetics}{匈奴}{xiong1nu2}
    \definition*{s.}{Xiongnu, um povo da estepe oriental que criou um império que floresceu na época das dinastias Qin e Han}
  \end{phonetics}
\end{entry}

\begin{entry}{匠}{6}[Radical ⼕]
  \begin{phonetics}{匠}{jiang4}
    \definition{s.}{artesão}
  \end{phonetics}
\end{entry}

\begin{entry}{华人}{6,2}[Radicais ⼗、⼈]
  \begin{phonetics}{华人}{hua2 ren2}[][HSK 3]
    \definition{s.}{Chinês; chinês étnico | cidadãos estrangeiros de ascendência chinesa que adquiriram nacionalidade no seu país de residência}
  \end{phonetics}
\end{entry}

\begin{entry}{华氏}{6,4}[Radicais ⼗、⽒]
  \begin{phonetics}{华氏}{hua2shi4}
    \definition{s.}{graus Fahrenheit (°F)}
  \end{phonetics}
\end{entry}

\begin{entry}{华夏}{6,10}[Radicais ⼗、⼢]
  \begin{phonetics}{华夏}{hua2xia4}
    \definition*{s.}{Huaxia, nome antigo da China | Catai}
  \end{phonetics}
\end{entry}

\begin{entry}{华盛顿}{6,11,10}[Radicais ⼗、⽫、⾴]
  \begin{phonetics}{华盛顿}{hua2sheng4dun4}
    \definition*{s.}{Washington}
  \end{phonetics}
\end{entry}

\begin{entry}{华裔}{6,13}[Radicais ⼗、⾐]
  \begin{phonetics}{华裔}{hua2yi4}
    \definition{s.}{descendente de chinês}
  \end{phonetics}
\end{entry}

\begin{entry}{危急}{6,9}[Radicais ⼙、⼼]
  \begin{phonetics}{危急}{wei1ji2}
    \definition{adj.}{crítico | desesperadora (situação)}
  \end{phonetics}
\end{entry}

\begin{entry}{危险}{6,9}[Radicais ⼙、⾩]
  \begin{phonetics}{危险}{wei1xian3}[][HSK 3]
    \definition{adj.}{arriscado; perigoso}
  \end{phonetics}
\end{entry}

\begin{entry}{危害}{6,10}[Radicais ⼙、⼧]
  \begin{phonetics}{危害}{wei1hai4}[][HSK 3]
    \definition{s.}{prejuízo; perigo; dano}
    \definition{v.}{prejudicar; pôr em perigo; pôr em risco}
  \end{phonetics}
\end{entry}

\begin{entry}{危难}{6,10}[Radicais ⼙、⾫]
  \begin{phonetics}{危难}{wei1nan4}
    \definition{s.}{calamidade}
  \end{phonetics}
\end{entry}

\begin{entry}{压}{6}[Radical ⼚]
  \begin{phonetics}{压}{ya1}[][HSK 3]
    \definition{v.}{pressionar; empurrar para baixo; segurar; pesar | manter sob controle; controlar; manter sob; reprimir |exercer pressão sobre; suprimir; desencorajar; intimidar | aproximar-se; estar chegando perto}
  \end{phonetics}
  \begin{phonetics}{压}{ya4}
    \definition{adv.}{fundamentalmente; nunca (usado principalmente em frases negativas)}
  \seealsoref{压根儿}{ya4 gen1r5}
  \end{phonetics}
\end{entry}

\begin{entry}{压力}{6,2}[Radicais ⼚、⼒]
  \begin{phonetics}{压力}{ya1li4}[][HSK 3]
    \definition[份,个]{s.}{pressão; força atuando perpendicularmente à superfície de um objeto | pressão; força esmagadora; metáfora para a força que coage e intimida as pessoas (principalmente nos aspectos espirituais e psicológicos) | tensão; fardo; fardos econômicos, psicológicos e espirituais que o mundo exterior traz às pessoas}
  \end{phonetics}
\end{entry}

\begin{entry}{压岁钱}{6,6,10}[Radicais ⼚、⼭、⾦]
  \begin{phonetics}{压岁钱}{ya1sui4qian2}
    \definition{s.}{dinheiro da sorte | dinheiro dado às crianças como presente no Ano Novo Chinês}
  \end{phonetics}
\end{entry}

\begin{entry}{压根儿}{6,10,2}[Radicais ⼚、⽊、⼉]
  \begin{phonetics}{压根儿}{ya4 gen1r5}
    \definition{adv.}{fundamentalmente; nunca (usado principalmente em frases negativas)}
  \end{phonetics}
\end{entry}

\begin{entry}{压碎}{6,13}[Radicais ⼚、⽯]
  \begin{phonetics}{压碎}{ya1sui4}
    \definition{v.}{esmagar em pedaços}
  \end{phonetics}
\end{entry}

\begin{entry}{压韵}{6,13}[Radicais ⼚、⾳]
  \begin{phonetics}{压韵}{ya1yun4}
    \variantof{押韵}
  \end{phonetics}
\end{entry}

\begin{entry}{吃}{6}[Radical ⼝]
  \begin{phonetics}{吃}{chi1}[][HSK 1]
    \definition{v.}{comer | consumir | comer em (uma cafeteria, etc.) | erradicar | destruir | absorver}
  \end{phonetics}
\end{entry}

\begin{entry}{吃饭}{6,7}[Radicais ⼝、⾷]
  \begin{phonetics}{吃饭}{chi1 fan4}[][HSK 1]
    \definition{v.+compl.}{comer | ter (comer) uma refeição | manter vivo | ganhar a vida}
  \end{phonetics}
\end{entry}

\begin{entry}{吃屎}{6,9}[Radicais ⼝、⼫]
  \begin{phonetics}{吃屎}{chi1 shi3}
    \definition{expr.}{Coma merda!}
  \end{phonetics}
\end{entry}

\begin{entry}{吃惊}{6,11}[Radicais ⼝、⼼]
  \begin{phonetics}{吃惊}{chi1jing1}[][HSK 4]
    \definition{v.+compl.}{ficar assustado; ficar chocado; ficar espantado; pegar de surpresa; ficar assustado inesperadamente}
  \end{phonetics}
\end{entry}

\begin{entry}{各}{6}[Radical ⼝]
  \begin{phonetics}{各}{ge4}[][HSK 3]
    \definition{adv.}{indica que mais de uma pessoa ou coisa está fazendo algo ou tem um determinado atributo}
    \definition{pron.}{todo; todos; cada | diferentes entre si; vários}
  \end{phonetics}
\end{entry}

\begin{entry}{各个}{6,3}[Radicais ⼝、⼈]
  \begin{phonetics}{各个}{ge4 ge4}[][HSK 4]
    \definition{adv./pron.}{cada | um a um; um após o outro}
  \end{phonetics}
\end{entry}

\begin{entry}{各地}{6,6}[Radicais ⼝、⼟]
  \begin{phonetics}{各地}{ge4 di4}[][HSK 3]
    \definition{s.}{todos os lugares; vários lugares}
  \end{phonetics}
\end{entry}

\begin{entry}{各自}{6,6}[Radicais ⼝、⾃]
  \begin{phonetics}{各自}{ge4zi4}[][HSK 3]
    \definition{pron.}{cada; respectivo; por si mesmo}
  \end{phonetics}
\end{entry}

\begin{entry}{各位}{6,7}[Radicais ⼝、⼈]
  \begin{phonetics}{各位}{ge4 wei4}[][HSK 3]
    \definition{pron.}{todos | cada}
  \end{phonetics}
\end{entry}

\begin{entry}{各种}{6,9}[Radicais ⼝、⽲]
  \begin{phonetics}{各种}{ge4 zhong3}[][HSK 3]
    \definition{adv.}{todos os tipos; vários; cada tipo}
  \end{phonetics}
\end{entry}

\begin{entry}{合}{6}[Radical ⼝]
  \begin{phonetics}{合}{he2}[][HSK 3]
    \definition*{s.}{sobrenome He}
    \definition{adj.}{todo; completo; inteiro}
    \definition{clas.}{para rodadas}
    \definition{s.}{conjunção}
    \definition{v.}{fechar | juntar; combinar | adequar-se; concordar; conformar-se a | ser igual a; somar}
  \end{phonetics}
\end{entry}

\begin{entry}{合同}{6,6}[Radicais ⼝、⼝]
  \begin{phonetics}{合同}{he2tong5}[][HSK 4]
    \definition[个,份]{s.}{contrato; acordo; uma disposição para observância mútua por duas ou mais partes na condução de um assunto com o objetivo de determinar seus respectivos direitos e obrigações.}
  \end{phonetics}
\end{entry}

\begin{entry}{合作}{6,7}[Radicais ⼝、⼈]
  \begin{phonetics}{合作}{he2zuo4}[][HSK 3]
    \definition[个]{s.}{cooperação; colaboração}
    \definition{v.}{cooperar; colaborar; trabalhar em conjunto}
  \end{phonetics}
\end{entry}

\begin{entry}{合法}{6,8}[Radicais ⼝、⽔]
  \begin{phonetics}{合法}{he2fa3}[][HSK 3]
    \definition{adj.}{legal; legítimo; correto}
  \end{phonetics}
\end{entry}

\begin{entry}{合宪性}{6,9,8}[Radicais ⼝、⼧、⼼]
  \begin{phonetics}{合宪性}{he2xian4xing4}
    \definition{s.}{constitucionalismo}
  \end{phonetics}
\end{entry}

\begin{entry}{合适}{6,9}[Radicais ⼝、⾡]
  \begin{phonetics}{合适}{he2shi4}[][HSK 2]
    \definition{adj.}{certo | adequado | apropriado}
  \end{phonetics}
\end{entry}

\begin{entry}{合格}{6,10}[Radicais ⼝、⽊]
  \begin{phonetics}{合格}{he2ge2}[][HSK 3]
    \definition{adj.}{qualificado; de acordo com o padrão}
  \end{phonetics}
\end{entry}

\begin{entry}{合资}{6,10}[Radicais ⼝、⾙]
  \begin{phonetics}{合资}{he2zi1}
    \definition{s.}{\emph{joint-venture} com capitais mistos}
  \end{phonetics}
\end{entry}

\begin{entry}{合理}{6,11}[Radicais ⼝、⽟]
  \begin{phonetics}{合理}{he2li3}[][HSK 3]
    \definition{adj.}{racional; razoável; equitativo}
  \end{phonetics}
\end{entry}

\begin{entry}{吉他}{6,5}[Radicais ⼝、⼈]
  \begin{phonetics}{吉他}{ji2ta1}
    \definition[把]{s.}{(empréstimo linguístico) guitarra}
  \end{phonetics}
\end{entry}

\begin{entry}{同}{6}[Radical ⼝]
  \begin{phonetics}{同}{tong2}
    \definition{adj.}{junto}
    \definition{adv.}{junto com}
  \end{phonetics}
\end{entry}

\begin{entry}{同伙}{6,6}[Radicais ⼝、⼈]
  \begin{phonetics}{同伙}{tong2huo3}
    \definition[个]{s.}{cúmplice | colega}
  \end{phonetics}
\end{entry}

\begin{entry}{同时}{6,7}[Radicais ⼝、⽇]
  \begin{phonetics}{同时}{tong2shi2}[][HSK 2]
    \definition{conj.}{além disso}
    \definition{s.}{enquanto isso | ao mesmo tempo}
  \end{phonetics}
\end{entry}

\begin{entry}{同事}{6,8}[Radicais ⼝、⼅]
  \begin{phonetics}{同事}{tong2shi4}[][HSK 2]
    \definition{s.}{colega | colega de trabalho | companheiro}
  \end{phonetics}
\end{entry}

\begin{entry}{同学}{6,8}[Radicais ⼝、⼦]
  \begin{phonetics}{同学}{tong2xue2}[][HSK 1]
    \definition[位,个]{s.}{colega de classe | colega estudante}
  \end{phonetics}
\end{entry}

\begin{entry}{同性恋}{6,8,10}[Radicais ⼝、⼼、⼼]
  \begin{phonetics}{同性恋}{tong2xing4lian4}
    \definition{s.}{homossexualidade | pessoa gay | amor gay}
  \end{phonetics}
\end{entry}

\begin{entry}{同屋}{6,9}[Radicais ⼝、⼫]
  \begin{phonetics}{同屋}{tong2wu1}
    \definition[个]{s.}{companheiro de quarto | colega de quarto}
  \end{phonetics}
\end{entry}

\begin{entry}{同砚}{6,9}[Radicais ⼝、⽯]
  \begin{phonetics}{同砚}{tong2yan4}
    \definition[位,个]{s.}{colega de classe | colega estudante}
  \end{phonetics}
\end{entry}

\begin{entry}{同样}{6,10}[Radicais ⼝、⽊]
  \begin{phonetics}{同样}{tong2 yang4}[][HSK 2]
    \definition{adj.}{igual | similar}
  \end{phonetics}
\end{entry}

\begin{entry}{同流合污}{6,10,6,6}[Radicais ⼝、⽔、⼝、⽔]
  \begin{phonetics}{同流合污}{tong2liu2he2wu1}
    \definition{expr.}{chafurdar na lama com alguém | seguir o mau exemplo dos outros}
  \end{phonetics}
\end{entry}

\begin{entry}{同情}{6,11}[Radicais ⼝、⼼]
  \begin{phonetics}{同情}{tong2qing2}[][HSK 4]
    \definition{s.}{simpatia}
    \definition{v.}{simpatizar com; solidarizar-se; compadecer-se; ter empatia emocional pelo que os outros estão passando}
  \end{phonetics}
\end{entry}

\begin{entry}{同意}{6,13}[Radicais ⼝、⼼]
  \begin{phonetics}{同意}{tong2yi4}[][HSK 3]
    \definition{v.}{concordar; consentir; aprovar; concordar com; dizer sim}
  \end{phonetics}
\end{entry}

\begin{entry}{名}{6}[Radical ⼝]
  \begin{phonetics}{名}{ming2}[][HSK 2]
    \definition*{s.}{sobrenome Ming}
    \definition{s.}{nome | denominação | fama | reputação}
  \end{phonetics}
\end{entry}

\begin{entry}{名人}{6,2}[Radicais ⼝、⼈]
  \begin{phonetics}{名人}{ming2 ren2}[][HSK 4]
    \definition{s.}{celebridade; pessoa famosa}
  \end{phonetics}
\end{entry}

\begin{entry}{名片}{6,4}[Radicais ⼝、⽚]
  \begin{phonetics}{名片}{ming2pian4}[][HSK 4]
    \definition[张,盒,叠]{s.}{cartão de visita; um pedaço de papel retangular com o nome, o cargo, o endereço etc. impressos}
  \end{phonetics}
\end{entry}

\begin{entry}{名字}{6,6}[Radicais ⼝、⼦]
  \begin{phonetics}{名字}{ming2zi5}[][HSK 1]
    \definition[个]{s.}{nome (de uma pessoa ou coisa)}
  \end{phonetics}
\end{entry}

\begin{entry}{名单}{6,8}[Radicais ⼝、⼗]
  \begin{phonetics}{名单}{ming2 dan1}[][HSK 2]
    \definition[个]{s.}{lista | lista de nomes}
  \end{phonetics}
\end{entry}

\begin{entry}{名称}{6,10}[Radicais ⼝、⽲]
  \begin{phonetics}{名称}{ming2 cheng1}[][HSK 2]
    \definition[个,种]{s.}{nome | designação}
  \end{phonetics}
\end{entry}

\begin{entry}{名牌儿}{6,12,2}[Radicais ⼝、⽚、⼉]
  \begin{phonetics}{名牌儿}{ming2 pai2r5}[][HSK 4]
    \definition*{s.}{Marca famosa}
  \end{phonetics}
\end{entry}

\begin{entry}{后}{6}[Radical ⼝]
  \begin{phonetics}{后}{hou4}[][HSK 1]
    \definition*{s.}{sobrenome Hou}
    \definition{adv.}{atrás | depois | mais tarde}
    \definition{s.}{traseiro | descendência | posteridade |imperatriz | rainha | soberano | governante}
  \end{phonetics}
\end{entry}

\begin{entry}{后天}{6,4}[Radicais ⼝、⼤]
  \begin{phonetics}{后天}{hou4 tian1}[][HSK 1]
    \definition{adv.}{depois de amanhã}
  \end{phonetics}
\end{entry}

\begin{entry}{后头}{6,5}[Radicais ⼝、⼤]
  \begin{phonetics}{后头}{hou4 tou5}[][HSK 4]
    \definition{adv.}{posteriormente | atrás | mais tarde}
    \definition{s.}{a parte de trás | a parte traseira}
  \end{phonetics}
\end{entry}

\begin{entry}{后边}{6,5}[Radicais ⼝、⾡]
  \begin{phonetics}{后边}{hou4 bian5}[][HSK 1]
    \definition{adv.}{atrás | detrás}
  \end{phonetics}
\end{entry}

\begin{entry}{后年}{6,6}[Radicais ⼝、⼲]
  \begin{phonetics}{后年}{hou4nian2}[][HSK 3]
    \definition{s.}{o ano que vem; daqui a dois anos}
  \end{phonetics}
\end{entry}

\begin{entry}{后来}{6,7}[Radicais ⼝、⽊]
  \begin{phonetics}{后来}{hou4lai2}[][HSK 2]
    \definition{adv.}{mais tarde}
  \end{phonetics}
\end{entry}

\begin{entry}{后果}{6,8}[Radicais ⼝、⽊]
  \begin{phonetics}{后果}{hou4guo3}[][HSK 3]
    \definition{s.}{consequência; resultado}
  \end{phonetics}
\end{entry}

\begin{entry}{后面}{6,9}[Radicais ⼝、⾯]
  \begin{phonetics}{后面}{hou4mian4}[][HSK 3]
    \definition{adv.}{parte de trás; retaguarda; atrás | atrás; perto do fim; na parte de trás | mais tarde; depois}
  \end{phonetics}
  \begin{phonetics}{后面}{hou4mian5}[][HSK 3]
    \definition{adv.}{parte de trás; retaguarda; atrás | atrás; perto do fim; na parte de trás | mais tarde; depois}
  \end{phonetics}
\end{entry}

\begin{entry}{吐}{6}[Radical ⼝]
  \begin{phonetics}{吐}{tu3}
    \definition{v.}{cuspir | enviar (seda de um bicho-da-seda, cápsulas de flores de algodão etc.) | dizer | despejar (suas queixas)}
  \end{phonetics}
  \begin{phonetics}{吐}{tu4}
    \definition{v.}{vomitar}
  \end{phonetics}
\end{entry}

\begin{entry}{向}{6}[Radical ⼝]
  \begin{phonetics}{向}{xiang4}[][HSK 2]
    \definition*{s.}{sobrenome Xiang}
    \definition{prep.}{para}
    \definition{v.}{enfrentar | virar para | apoiar}
  \end{phonetics}
\end{entry}

\begin{entry}{向汪}{6,7}[Radicais ⼝、⽔]
  \begin{phonetics}{向汪}{xiang4wang1}
    \definition{v.}{esperar que}
  \end{phonetics}
\end{entry}

\begin{entry}{向往}{6,8}[Radicais ⼝、⼻]
  \begin{phonetics}{向往}{xiang4wang3}
    \definition{v.}{ansiar por | esperar ansiosamente por}
  \end{phonetics}
\end{entry}

\begin{entry}{吓人}{6,2}[Radicais ⼝、⼈]
  \begin{phonetics}{吓人}{xia4ren2}
    \definition{adj.}{apavorante | assustador}
    \definition{v.+compl.}{assustar-se | tomar um susto}
  \end{phonetics}
\end{entry}

\begin{entry}{吗}{6}[Radical ⼝]
  \begin{phonetics}{吗}{ma2}
    \definition{adv.}{(coloquial) que?}
  \end{phonetics}
  \begin{phonetics}{吗}{ma3}
    \definition{s.}{usada em 吗啡}
    \seeref{吗啡}{ma3fei1}
  \end{phonetics}
  \begin{phonetics}{吗}{ma5}[][HSK 1]
    \definition{part.}{partícula interrogativa, usada em perguntas ``sim-não''}
  \end{phonetics}
\end{entry}

\begin{entry}{吗啡}{6,11}[Radicais ⼝、⼝]
  \begin{phonetics}{吗啡}{ma3fei1}
    \definition{s.}{morfina (empréstimo linguístico)}
  \end{phonetics}
\end{entry}

\begin{entry}{吸}{6}[Radical ⼝]
  \begin{phonetics}{吸}{xi1}[][HSK 4]
    \definition{v.}{inalar; inspirar; aspirar; itroduzir líquidos, gases, etc. no corpo | absorver; sugar | atrair; atrair para si mesmo; atrair (interesse, investimento etc.)}
  \end{phonetics}
\end{entry}

\begin{entry}{吸引}{6,4}[Radicais ⼝、⼸]
  \begin{phonetics}{吸引}{xi1yin3}[][HSK 4]
    \definition{v.}{atrair; apelar para; chamar a atenção de outros objetos, forças ou pessoas para si mesmo}
  \end{phonetics}
\end{entry}

\begin{entry}{吸收}{6,6}[Radicais ⼝、⽁]
  \begin{phonetics}{吸收}{xi1shou1}[][HSK 4]
    \definition{v.}{imbuir; absorver; assimilar; sugar;  chupar; (animais, plantas, etc.) extrair material de fora dos tecidos para o interior dos tecidos | absorver; chupar;  sugar alguma substância de fora para dentro | recrutar; alistar; inscrever-se; matricular-se; admitir; (organizações ou coletivos) aceitar novos membros | absorver; aproveitar e usar a experiência, o conhecimento, o dinheiro e outras coisas valiosas de outras pessoas | absorver; diminuir, atenuar ou eliminar determinados efeitos ou fenômenos}
  \end{phonetics}
\end{entry}

\begin{entry}{吸烟}{6,10}[Radicais ⼝、⽕]
  \begin{phonetics}{吸烟}{xi1yan1}[][HSK 4]
    \definition{v.+compl.}{fumar}
  \end{phonetics}
\end{entry}

\begin{entry}{吸铁石}{6,10,5}[Radicais ⼝、⾦、⽯]
  \begin{phonetics}{吸铁石}{xi1tie3shi2}
    \definition{s.}{imã | magneto}
  \seealsoref{磁铁}{ci2tie3}
  \end{phonetics}
\end{entry}

\begin{entry}{吸管}{6,14}[Radicais ⼝、⽵]
  \begin{phonetics}{吸管}{xi1 guan3}[][HSK 4]
    \definition[根,个]{s.}{tubo de sucção; sugador; canudo (para beber); refere-se ao tubo fino usado para sugar bebidas | conta-gotas; pipeta; cateter para bombeamento de líquidos usando pressão de ar}
  \end{phonetics}
\end{entry}

\begin{entry}{回}{6}[Radical ⼞]
  \begin{phonetics}{回}{hui2}[][HSK 1,2]
    \definition{clas.}{de atos de uma peça de teatro}
    \definition{s.}{seção ou capítulo (de um livro clássico) | grupo étnico Hui (mulçumanos chineses)}
    \definition{v.}{regressar | voltar | dar a volta | responder | resolver | circular | curvar}
  \end{phonetics}
\end{entry}

\begin{entry}{回去}{6,5}[Radicais ⼞、⼛]
  \begin{phonetics}{回去}{hui2 qu4}[][HSK 1]
    \definition{v.}{regressar | voltar | estar de volta | (a partir da minha localização)}
  \end{phonetics}
\end{entry}

\begin{entry}{回来}{6,7}[Radicais ⼞、⽊]
  \begin{phonetics}{回来}{hui2 lai5}[][HSK 1]
    \definition{v.}{regressar | voltar | estar de volta | (para a minha localização)}
  \end{phonetics}
\end{entry}

\begin{entry}{回到}{6,8}[Radicais ⼞、⼑]
  \begin{phonetics}{回到}{hui2 dao4}[][HSK 1]
    \definition{v.}{retornar a}
  \end{phonetics}
\end{entry}

\begin{entry}{回国}{6,8}[Radicais ⼞、⼞]
  \begin{phonetics}{回国}{hui2 guo2}[][HSK 2]
    \definition{v.}{retornar ao seu país (terra natal)}
  \end{phonetics}
\end{entry}

\begin{entry}{回信}{6,9}[Radicais ⼞、⼈]
  \begin{phonetics}{回信}{hui2xin4}
    \definition{s.}{uma carta em resposta | uma mensagem verbal em resposta}
    \definition{v.+compl.}{escrever em resposta | escrever de volta | responder uma carta | responder verbalmente uma mensagem}
  \end{phonetics}
\end{entry}

\begin{entry}{回复}{6,9}[Radicais ⼞、⼢]
  \begin{phonetics}{回复}{hui2 fu4}[][HSK 4]
    \definition{v.}{responder (a uma carta) | retornar ao estado normal; restaurar algo ao seu estado original}
  \end{phonetics}
\end{entry}

\begin{entry}{回家}{6,10}[Radicais ⼞、⼧]
  \begin{phonetics}{回家}{hui2 jia1}[][HSK 1]
    \definition{v.}{ir (voltar) para casa | estar em casa | retornar para casa}
  \end{phonetics}
\end{entry}

\begin{entry}{回旋}{6,11}[Radicais ⼞、⽅]
  \begin{phonetics}{回旋}{hui2xuan2}
    \definition{v.}{circular | rodar | dar a volta}
  \end{phonetics}
\end{entry}

\begin{entry}{回答}{6,12}[Radicais ⼞、⽵]
  \begin{phonetics}{回答}{hui2da2}[][HSK 1]
    \definition{v.}{responder}
  \end{phonetics}
\end{entry}

\begin{entry}{因为}{6,4}[Radicais ⼞、⼂]
  \begin{phonetics}{因为}{yin1wei4}[][HSK 2]
    \definition{conj.}{porque | devido a | por conta de}
  \end{phonetics}
\end{entry}

\begin{entry}{因为……所以……}{6,4,8,4}[Radicais ⼞、⼂、⼾、⼈]
  \begin{phonetics}{因为……所以……}{yin1wei4 suo3yi3}[][HSK 2]
    \definition{conj.}{porque\dots portanto\dots}
  \end{phonetics}
\end{entry}

\begin{entry}{因此}{6,6}[Radicais ⼞、⽌]
  \begin{phonetics}{因此}{yin1ci3}[][HSK 3]
    \definition{conj.}{assim; por isso; portanto; consequentemente}
  \end{phonetics}
\end{entry}

\begin{entry}{因而}{6,6}[Radicais ⼞、⽽]
  \begin{phonetics}{因而}{yin1'er2}
    \definition{conj.}{então | portanto | por esta razão | consequentemente}
  \end{phonetics}
\end{entry}

\begin{entry}{团}{6}[Radical ⼞]
  \begin{phonetics}{团}{tuan2}[][HSK 3]
    \definition*{s.}{Liga da Juventude Comunista da China; Liga}
    \definition{adj.}{redondo; circular | coletivo}
    \definition{clas.}{coisas usadas para formar um grupo}
    \definition[个]{s.}{bolinho de massa | algo em forma de bola | grupo; corpo; sociedade; organização | regimento}
    \definition{v.}{enrolar algo para formar uma bola; rolar | reunir; unir; conglomerar}
  \end{phonetics}
\end{entry}

\begin{entry}{团队}{6,4}[Radicais ⼞、⾩]
  \begin{phonetics}{团队}{tuan2dui4}
    \definition{s.}{equipe}
  \end{phonetics}
\end{entry}

\begin{entry}{团体}{6,7}[Radicais ⼞、⼈]
  \begin{phonetics}{团体}{tuan2ti3}[][HSK 3]
    \definition[个]{s.}{equipe; grupo; organização}
  \end{phonetics}
\end{entry}

\begin{entry}{团结}{6,9}[Radicais ⼞、⽷]
  \begin{phonetics}{团结}{tuan2jie2}[][HSK 3]
    \definition{adj.}{unido; amigável; harmonioso}
    \definition{v.}{unir; reunir}
  \end{phonetics}
\end{entry}

\begin{entry}{在}{6}[Radical ⼟]
  \begin{phonetics}{在}{zai4}[][HSK 1]
    \definition{adv.}{para designar ações que estão passando | durante}
    \definition{prep.}{em}
    \definition{v.}{estar | ficar}
  \end{phonetics}
\end{entry}

\begin{entry}{在下}{6,3}[Radicais ⼟、⼀]
  \begin{phonetics}{在下}{zai4xia4}
    \definition{pron.}{eu mesmo (humildemente)}
  \end{phonetics}
\end{entry}

\begin{entry}{在于}{6,3}[Radicais ⼟、⼆]
  \begin{phonetics}{在于}{zai4yu2}
    \definition{v.}{descansar | deitar | ser devido a (um determinado atributo)/(de um assunto) a ser determinado | estar à altura de alguém}
  \end{phonetics}
\end{entry}

\begin{entry}{在乎}{6,5}[Radicais ⼟、⼃]
  \begin{phonetics}{在乎}{zai4hu5}
    \definition{v.}{preocupar-se com}
  \end{phonetics}
\end{entry}

\begin{entry}{在地}{6,6}[Radicais ⼟、⼟]
  \begin{phonetics}{在地}{zai4di4}
    \definition{s.}{local}
  \end{phonetics}
\end{entry}

\begin{entry}{在此}{6,6}[Radicais ⼟、⽌]
  \begin{phonetics}{在此}{zai4ci3}
    \definition{adv.}{aqui}
  \end{phonetics}
\end{entry}

\begin{entry}{在行}{6,6}[Radicais ⼟、⾏]
  \begin{phonetics}{在行}{zai4hang2}
    \definition{v.}{ser adepto de algo | ser um especialista em um comércio ou profissão}
  \end{phonetics}
\end{entry}

\begin{entry}{在线}{6,8}[Radicais ⼟、⽷]
  \begin{phonetics}{在线}{zai4xian4}
    \definition{s.}{\emph{online}}
  \end{phonetics}
\end{entry}

\begin{entry}{在家}{6,10}[Radicais ⼟、⼧]
  \begin{phonetics}{在家}{zai4jia1}[][HSK 1]
    \definition{v.}{estar em casa | permanecer um leigo}
  \end{phonetics}
\end{entry}

\begin{entry}{在教}{6,11}[Radicais ⼟、⽁]
  \begin{phonetics}{在教}{zai4jiao4}
    \definition{v.}{ser um crente (em uma religião)}
  \end{phonetics}
\end{entry}

\begin{entry}{在意}{6,13}[Radicais ⼟、⼼]
  \begin{phonetics}{在意}{zai4yi4}
    \definition{v.+compl.}{preocupar-se | importar-se | levar a sério}
  \end{phonetics}
\end{entry}

\begin{entry}{地}{6}[Radical ⼟]
  \begin{phonetics}{地}{de5}[][HSK 1]
    \definition{part.}{(estrutural) utilizada antes de um verbo ou adjetivo, ligando-o ao adjunto adverbial modificador precedente}
  \end{phonetics}
  \begin{phonetics}{地}{di4}[][HSK 1]
    \definition[个,片]{s.}{mundo | campo | chão | terra | lugar}
  \end{phonetics}
\end{entry}

\begin{entry}{地上}{6,3}[Radicais ⼟、⼀]
  \begin{phonetics}{地上}{di4 shang5}[][HSK 1]
    \definition{adv.}{no chão}
  \end{phonetics}
\end{entry}

\begin{entry}{地下}{6,3}[Radicais ⼟、⼀]
  \begin{phonetics}{地下}{di4 xia4}[][HSK 4]
    \definition{s.}{subterrâneo | secreta (atividade) | recursos ocultos}
  \end{phonetics}
\end{entry}

\begin{entry}{地下室}{6,3,9}[Radicais ⼟、⼀、⼧]
  \begin{phonetics}{地下室}{di4xia4shi4}
    \definition{s.}{subterrâneo | porão}
  \end{phonetics}
\end{entry}

\begin{entry}{地区}{6,4}[Radicais ⼟、⼖]
  \begin{phonetics}{地区}{di4qu1}[][HSK 3]
    \definition{adj.}{regional}
    \definition[个]{s.}{área; distrito; região | prefeitura | latitudes; localidade; lado}
    \definition{suf.}{como sufixo do nome da cidade, significa prefeitura ou condado}
  \end{phonetics}
\end{entry}

\begin{entry}{地方}{6,4}[Radicais ⼟、⽅]
  \begin{phonetics}{地方}{di4fang1}
    \definition[个]{s.}{distrito; localidade;  em oposição a ``中央'', o número total de unidades administrativas em todos os níveis abaixo do centro | governo local e população; refere-se a outros setores que não o militar}
  \seealsoref{中央}{zhong1yang1}
  \end{phonetics}
  \begin{phonetics}{地方}{di4fang5}[][HSK 1,4]
    \definition[个,处,块]{s.}{lugar; cômodo; área; refere-se a um espaço específico
parte}
  \end{phonetics}
\end{entry}

\begin{entry}{地位}{6,7}[Radicais ⼟、⼈]
  \begin{phonetics}{地位}{di4wei4}[][HSK 4]
    \definition{s.}{lugar; status; posição; posição da pessoa ou do grupo nas relações sociais | lugar; posição (ocupada por uma pessoa ou coisa); espaço ocupado por uma pessoa ou coisa}
  \end{phonetics}
\end{entry}

\begin{entry}{地址}{6,7}[Radicais ⼟、⼟]
  \begin{phonetics}{地址}{di4zhi3}[][HSK 4]
    \definition[个]{s.}{endereço; local de residência ou correspondência}
  \end{phonetics}
\end{entry}

\begin{entry}{地图}{6,8}[Radicais ⼟、⼞]
  \begin{phonetics}{地图}{di4tu2}[][HSK 1]
    \definition[张,本]{s.}{mapa}
  \end{phonetics}
\end{entry}

\begin{entry}{地点}{6,9}[Radicais ⼟、⽕]
  \begin{phonetics}{地点}{di4dian3}[][HSK 1]
    \definition[个]{s.}{localização | lugar | local}
  \end{phonetics}
\end{entry}

\begin{entry}{地狱}{6,9}[Radicais ⼟、⽝]
  \begin{phonetics}{地狱}{di4yu4}
    \definition*{s.}{\emph{Naraka} (Budismo)}
    \definition{adj.}{infernal}
    \definition{s.}{inferno | submundo}
  \end{phonetics}
\end{entry}

\begin{entry}{地砖}{6,9}[Radicais ⼟、⽯]
  \begin{phonetics}{地砖}{di4zhuan1}
    \definition{s.}{ladrilho de piso}
  \end{phonetics}
\end{entry}

\begin{entry}{地面}{6,9}[Radicais ⼟、⾯]
  \begin{phonetics}{地面}{di4 mian4}[][HSK 4]
    \definition{s.}{a superfície da Terra | térreo; piso; camada de material colocada no chão dentro e ao redor dos edifícios | localidade; chão | região; território; principalmente áreas administrativas}
  \end{phonetics}
\end{entry}

\begin{entry}{地核}{6,10}[Radicais ⼟、⽊]
  \begin{phonetics}{地核}{di4he2}
    \definition{s.}{(geologia) núcleo da Terra}
  \end{phonetics}
\end{entry}

\begin{entry}{地铁}{6,10}[Radicais ⼟、⾦]
  \begin{phonetics}{地铁}{di4tie3}[][HSK 2]
    \definition{s.}{metrô, metropolitano}
  \end{phonetics}
\end{entry}

\begin{entry}{地铁站}{6,10,10}[Radicais ⼟、⾦、⽴]
  \begin{phonetics}{地铁站}{di4 tie3 zhan4}[][HSK 2]
    \definition{s.}{estação de metrô}
  \end{phonetics}
\end{entry}

\begin{entry}{地球}{6,11}[Radicais ⼟、⽟]
  \begin{phonetics}{地球}{di4qiu2}[][HSK 2]
    \definition{s.}{o planeta terra}
  \end{phonetics}
\end{entry}

\begin{entry}{地理}{6,11}[Radicais ⼟、⽟]
  \begin{phonetics}{地理}{di4li3}
    \definition{s.}{geografia}
  \end{phonetics}
\end{entry}

\begin{entry}{地震}{6,15}[Radicais ⼟、⾬]
  \begin{phonetics}{地震}{di4zhen4}
    \definition{s.}{terremoto | tremor de terra}
  \end{phonetics}
\end{entry}

\begin{entry}{场}{6}[Radical ⼟]
  \begin{phonetics}{场}{chang3}[][HSK 2]
    \definition{clas.}{para número de exames | para atividades esportivas ou recreativas}
    \definition{s.}{local grande usado para um propósito específico | cena (de uma peça) | palco}
  \end{phonetics}
\end{entry}

\begin{entry}{场合}{6,6}[Radicais ⼟、⼝]
  \begin{phonetics}{场合}{chang3he2}[][HSK 3]
    \definition[种]{s.}{ocasião; situação}
  \end{phonetics}
\end{entry}

\begin{entry}{场所}{6,8}[Radicais ⼟、⼾]
  \begin{phonetics}{场所}{chang3suo3}[][HSK 3]
    \definition{s.}{lugar; sítio; arena}
  \end{phonetics}
\end{entry}

\begin{entry}{场面}{6,9}[Radicais ⼟、⾯]
  \begin{phonetics}{场面}{chang3mian4}
    \definition{s.}{cena | espetáculo | ocasião | situação}
  \end{phonetics}
\end{entry}

\begin{entry}{场景}{6,12}[Radicais ⼟、⽇]
  \begin{phonetics}{场景}{chang3jing3}
    \definition{s.}{cena | cenário | situação | contexto}
  \end{phonetics}
\end{entry}

\begin{entry}{多}{6}[Radical ⼣]
  \begin{phonetics}{多}{duo1}[][HSK 1,2]
    \definition{adv.}{muitos | muito | muitas vezes | um monte de | numerosos | mais | em excesso | como (até que ponto)}
    \definition{num.}{(após um número) ímpar}
    \definition{pref.}{multi | poli}
  \end{phonetics}
\end{entry}

\begin{entry}{多久}{6,3}[Radicais ⼣、⼃]
  \begin{phonetics}{多久}{duo1 jiu3}[][HSK 2]
    \definition{pron.}{quanto tempo? | quanto tempo}
  \end{phonetics}
\end{entry}

\begin{entry}{多么}{6,3}[Radicais ⼣、⼃]
  \begin{phonetics}{多么}{duo1me5}[][HSK 2]
    \definition{adv.}{(em exclamações) como |  o que | até que ponto | em uma extensão não especificada | como (usado em uma frase interrogativa para perguntar sobre grau ou número)}
  \end{phonetics}
\end{entry}

\begin{entry}{多大}{6,3}[Radicais ⼣、⼤]
  \begin{phonetics}{多大}{duo1da4}
    \definition{adj.}{quantos anos? | que idade? | quão grande?}
  \end{phonetics}
\end{entry}

\begin{entry}{多云}{6,4}[Radicais ⼣、⼆]
  \begin{phonetics}{多云}{duo1 yun2}[][HSK 2]
    \definition{adj.}{céu nublado}
  \end{phonetics}
\end{entry}

\begin{entry}{多少}{6,4}[Radicais ⼣、⼩]
  \begin{phonetics}{多少}{duo1shao3}
    \definition{num.}{número | quantidade | um pouco}
  \end{phonetics}
  \begin{phonetics}{多少}{duo1shao5}[][HSK 1]
    \definition{adv.}{quanto? | quantos? | (número de telefone, ID de estudante, etc.) qual o número?}
  \end{phonetics}
\end{entry}

\begin{entry}{多年}{6,6}[Radicais ⼣、⼲]
  \begin{phonetics}{多年}{duo1 nian2}[][HSK 4]
    \definition{adv.}{por muitos anos; durante muitos anos}
  \end{phonetics}
\end{entry}

\begin{entry}{多次}{6,6}[Radicais ⼣、⽋]
  \begin{phonetics}{多次}{duo1 ci4}[][HSK 4]
    \definition{adv.}{muitas vezes; de vez em quando; repetidamente; em muitas ocasiões}
  \end{phonetics}
\end{entry}

\begin{entry}{多种}{6,9}[Radicais ⼣、⽲]
  \begin{phonetics}{多种}{duo1 zhong3}[][HSK 4]
    \definition{adj.}{diverso; vários tipos de; múltiplo; diversificado}
  \end{phonetics}
\end{entry}

\begin{entry}{多重}{6,9}[Radicais ⼣、⾥]
  \begin{phonetics}{多重}{duo1chong2}
    \definition{pref.}{multi (facetado, cultural, étnico, etc.)}
  \end{phonetics}
\end{entry}

\begin{entry}{多样}{6,10}[Radicais ⼣、⽊]
  \begin{phonetics}{多样}{duo1 yang4}[][HSK 4]
    \definition{adj.}{diversos; variados; diversificado}
    \definition{s.}{diversidade}
  \end{phonetics}
\end{entry}

\begin{entry}{多数}{6,13}[Radicais ⼣、⽁]
  \begin{phonetics}{多数}{duo1 shu4}[][HSK 2]
    \definition{adj.}{maioria | plural}
    \definition{pref.}{pluri-}
  \end{phonetics}
\end{entry}

\begin{entry}{夺冠}{6,9}[Radicais ⼤、⼍]
  \begin{phonetics}{夺冠}{duo2guan4}
    \definition{v.}{apoderar-se da coroa | (fig.) ganhar um campeonato | ganhar a medalha de ouro}
  \end{phonetics}
\end{entry}

\begin{entry}{奸夫}{6,4}[Radicais ⼥、⼤]
  \begin{phonetics}{奸夫}{jian1fu1}
    \definition{s.}{homem adúltero}
  \end{phonetics}
\end{entry}

\begin{entry}{她}{6}[Radical ⼥]
  \begin{phonetics}{她}{ta1}[][HSK 1]
    \definition{pron.}{ela | se, a, lhe | si, consigo, ela}
  \end{phonetics}
\end{entry}

\begin{entry}{她们}{6,5}[Radicais ⼥、⼈]
  \begin{phonetics}{她们}{ta1men5}[][HSK 1]
    \definition{pron.}{elas | se, as, lhes | si, consigo, elas}
  \end{phonetics}
\end{entry}

\begin{entry}{她们的}{6,5,8}[Radicais ⼥、⼈、⽩]
  \begin{phonetics}{她们的}{ta1men5 de5}
    \definition{pron.}{delas}
  \end{phonetics}
\end{entry}

\begin{entry}{她的}{6,8}[Radicais ⼥、⽩]
  \begin{phonetics}{她的}{ta1 de5}
    \definition{pron.}{dela}
  \end{phonetics}
\end{entry}

\begin{entry}{好}{6}[Radical ⼥]
  \begin{phonetics}{好}{hao3}[][HSK 1,4]
    \definition{adj.}{bom; ótimo; agradável; vantajoso; satisfatório | amigável; gentil; amistoso; amável | saudável; bem | pronto; concluído; usado após um verbo para indicar conclusão ou perfeição | fácil (de fazer); conveniente; responsável (por)}
    \definition{adv.}{muito; bastante; tão; usado na frente de uma palavra de quantidade ou uma palavra de tempo para indicar muito ou por muito tempo | em que medida; como; usado antes de adjetivos e verbos para indicar profundidade e com exclamação}
    \definition{interj.}{O.K.; tudo bem; aprovação, acordo ou encerramento | (no início de uma frase ou oração) expressa concordância (ou desaprovação, surpresa, etc.)}
    \definition{prep.}{de modo a; para que | apaixonar-se}
    \definition{s.}{referindo-se a palavras de elogio ou aplauso | saudações; cumprimentos}
    \definition{suf.}{sufixo que indica conclusão ou prontidão | depois de um pronome significa ``olá''}
    \definition{v.}{dever; precisar; ter que}
  \end{phonetics}
  \begin{phonetics}{好}{hao4}
    \definition*{s.}{sobrenome Hao}
    \definition{adv.}{algo que acontece com frequência, que é fácil de acontecer}
    \definition{v.}{gostar; amar; ter afeição por}
  \end{phonetics}
\end{entry}

\begin{entry}{好人}{6,2}[Radicais ⼥、⼈]
  \begin{phonetics}{好人}{hao3 ren2}[][HSK 2]
    \definition[个]{s.}{boa (legal) pessoa | uma pessoa saudável | uma pessoa que tenta se dar bem com todos}
  \end{phonetics}
\end{entry}

\begin{entry}{好久}{6,3}[Radicais ⼥、⼃]
  \begin{phonetics}{好久}{hao3jiu3}[][HSK 2]
    \definition{adv.}{por muito tempo | por eras (no passado)}
  \end{phonetics}
\end{entry}

\begin{entry}{好友}{6,4}[Radicais ⼥、⼜]
  \begin{phonetics}{好友}{hao3you3}[][HSK 4]
    \definition[位,个]{s.}{bom amigo; amigo próximo}
  \end{phonetics}
\end{entry}

\begin{entry}{好心}{6,4}[Radicais ⼥、⼼]
  \begin{phonetics}{好心}{hao3xin1}
    \definition{s.}{bondade | boas intenções}
  \end{phonetics}
\end{entry}

\begin{entry}{好处}{6,5}[Radicais ⼥、⼡]
  \begin{phonetics}{好处}{hao3chu4}[][HSK 2]
    \definition[个]{s.}{benefício | vantagem | ganho | lucro}
  \end{phonetics}
\end{entry}

\begin{entry}{好汉}{6,5}[Radicais ⼥、⽔]
  \begin{phonetics}{好汉}{hao3han4}
    \definition[条]{s.}{herói | pessoa forte e corajosa}
  \end{phonetics}
\end{entry}

\begin{entry}{好生}{6,5}[Radicais ⼥、⽣]
  \begin{phonetics}{好生}{hao3sheng1}
    \definition{adv.}{bastante; extremamente | cuidadosamente; apropriadamente}
  \end{phonetics}
\end{entry}

\begin{entry}{好用}{6,5}[Radicais ⼥、⽤]
  \begin{phonetics}{好用}{hao3yong4}
    \definition{adj.}{fácil de usar | adequado ao uso}
  \end{phonetics}
\end{entry}

\begin{entry}{好吃}{6,6}[Radicais ⼥、⼝]
  \begin{phonetics}{好吃}{hao3chi1}[][HSK 1]
    \definition{adj.}{delicioso | saboroso}
  \end{phonetics}
  \begin{phonetics}{好吃}{hao4chi1}
    \definition{v.}{gostar de comer | ser guloso}
  \end{phonetics}
\end{entry}

\begin{entry}{好多}{6,6}[Radicais ⼥、⼣]
  \begin{phonetics}{好多}{hao3 duo1}[][HSK 2]
    \definition{adj.}{muito | uma boa quantidade | um bom negócio | uma grande quantidade}
    \definition{pron.}{quanto}
  \end{phonetics}
\end{entry}

\begin{entry}{好好}{6,6}[Radicais ⼥、⼥]
  \begin{phonetics}{好好}{hao3 hao3}[][HSK 3]
    \definition{adj.}{realmente bom/bem; em perfeitas condições; quando tudo está bem}
    \definition{adv.}{diretamente; seriamente; cuidadosamente}
  \end{phonetics}
\end{entry}

\begin{entry}{好听}{6,7}[Radicais ⼥、⼝]
  \begin{phonetics}{好听}{hao3 ting1}[][HSK 1]
    \definition{adj.}{agradável de ouvir}
  \end{phonetics}
\end{entry}

\begin{entry}{好事}{6,8}[Radicais ⼥、⼅]
  \begin{phonetics}{好事}{hao3 shi4}[][HSK 2]
    \definition[个]{s.}{boa ação | um ato de caridade | boas obras | evento feliz}
  \end{phonetics}
  \begin{phonetics}{好事}{hao4 shi4}
    \definition[行]{s.}{bençãos}
  \end{phonetics}
\end{entry}

\begin{entry}{好奇}{6,8}[Radicais ⼥、⼤]
  \begin{phonetics}{好奇}{hao4qi2}[][HSK 3]
    \definition{adj.}{curioso}
    \definition{s.}{curiosidade}
    \definition{v.}{ser ou estar curioso}
  \end{phonetics}
\end{entry}

\begin{entry}{好学}{6,8}[Radicais ⼥、⼦]
  \begin{phonetics}{好学}{hao3xue2}
    \definition{adj.}{fácil de aprender}
  \end{phonetics}
  \begin{phonetics}{好学}{hao4xue2}
    \definition{s.}{estudioso | erudito}
  \end{phonetics}
\end{entry}

\begin{entry}{好玩儿}{6,8,2}[Radicais ⼥、⽟、⼉]
  \begin{phonetics}{好玩儿}{hao3 wan2r5}[][HSK 1]
    \definition{adj.}{divertido | prazeroso | interessante}
  \end{phonetics}
\end{entry}

\begin{entry}{好看}{6,9}[Radicais ⼥、⽬]
  \begin{phonetics}{好看}{hao3 kan4}[][HSK 1]
    \definition{adj.}{boa aparência | bom (um filme, livro, programa de TV, etc.)}
  \end{phonetics}
\end{entry}

\begin{entry}{好象}{6,11}[Radicais ⼥、⾗]
  \begin{phonetics}{好象}{hao3xiang4}
    \variantof{好像}
  \end{phonetics}
\end{entry}

\begin{entry}{好像}{6,13}[Radicais ⼥、⼈]
  \begin{phonetics}{好像}{hao3xiang4}[][HSK 2]
    \definition{adv.}{talvez fosse | parecer | ser como}
  \end{phonetics}
\end{entry}

\begin{entry}{如}{6}[Radical ⼥]
  \begin{phonetics}{如}{ru2}
    \definition{conj.}{por exemplo}
  \end{phonetics}
\end{entry}

\begin{entry}{如今}{6,4}[Radicais ⼥、⼈]
  \begin{phonetics}{如今}{ru2jin1}[][HSK 4]
    \definition{s.}{agora; hoje em dia; atualmente; no presente}
  \end{phonetics}
\end{entry}

\begin{entry}{如此}{6,6}[Radicais ⼥、⽌]
  \begin{phonetics}{如此}{ru2ci3}
    \definition{adv.}{assim | então | tal}
  \end{phonetics}
\end{entry}

\begin{entry}{如何}{6,7}[Radicais ⼥、⼈]
  \begin{phonetics}{如何}{ru2he2}[][HSK 3]
    \definition{adv.}{como?; o que?}
  \end{phonetics}
\end{entry}

\begin{entry}{如果}{6,8}[Radicais ⼥、⽊]
  \begin{phonetics}{如果}{ru2guo3}[][HSK 2]
    \definition{conj.}{se | caso | no caso de | no evento de | supondo que}
  \end{phonetics}
\end{entry}

\begin{entry}{如画}{6,8}[Radicais ⼥、⽥]
  \begin{phonetics}{如画}{ru2hua4}
    \definition{adj.}{pitoresco}
  \end{phonetics}
\end{entry}

\begin{entry}{妆}{6}[Radical ⼥]
  \begin{phonetics}{妆}{zhuang1}
    \definition{s.}{maquiagem | adorno | enxoval | maquiagem e figurino de palco}
    \definition{v.}{maquiar-se | enfeitar-se}
  \end{phonetics}
\end{entry}

\begin{entry}{妆扮}{6,7}[Radicais ⼥、⼿]
  \begin{phonetics}{妆扮}{zhuang1ban4}
    \variantof{装扮}
  \end{phonetics}
\end{entry}

\begin{entry}{妈}{6}[Radical ⼥]
  \begin{phonetics}{妈}{ma1}[][HSK 1]
    \definition[个,位]{s.}{mãe | mamãe | uma forma de tratamento para uma mulher casada uma geração mais velha | uma forma de tratamento para uma empregada doméstica de meia-idade ou velha}
    \seeref{妈妈}{ma1ma5}
  \end{phonetics}
\end{entry}

\begin{entry}{妈妈}{6,6}[Radicais ⼥、⼥]
  \begin{phonetics}{妈妈}{ma1ma5}[][HSK 1]
    \definition[个,位]{s.}{mamãe | mãe}
  \end{phonetics}
\end{entry}

\begin{entry}{字}{6}[Radical ⼦]
  \begin{phonetics}{字}{zi4}[][HSK 1]
    \definition[个]{s.}{carácter | letra | símbolo | palavra}
  \end{phonetics}
\end{entry}

\begin{entry}{字母}{6,5}[Radicais ⼦、⽏]
  \begin{phonetics}{字母}{zi4mu3}
    \definition[个]{s.}{letra (do alfabeto)}
  \end{phonetics}
\end{entry}

\begin{entry}{字字珠玉}{6,6,10,5}[Radicais ⼦、⼦、⽟、⽟]
  \begin{phonetics}{字字珠玉}{zi4zi4zhu1yu4}
    \definition{expr.}{cada palavra é uma jóia}
    \definition{s.}{escrita magnífica}
  \end{phonetics}
\end{entry}

\begin{entry}{字典}{6,8}[Radicais ⼦、⼋]
  \begin{phonetics}{字典}{zi4 dian3}[][HSK 2]
    \definition[本]{s.}{dicionário de caracteres chineses (contendo verbetes de caracteres únicos, em contraste com 词典 que contém verbetes para palavras com um ou mais caracteres)}
    \seeref{词典}{ci2dian3}
  \end{phonetics}
\end{entry}

\begin{entry}{字眼}{6,11}[Radicais ⼦、⽬]
  \begin{phonetics}{字眼}{zi4yan3}
    \definition[个]{s.}{palavras | redação}
  \end{phonetics}
\end{entry}

\begin{entry}{字脚}{6,11}[Radicais ⼦、⾁]
  \begin{phonetics}{字脚}{zi4jiao3}
    \definition[典]{s.}{gancho no final da pincelada | serifa}
  \end{phonetics}
\end{entry}

\begin{entry}{存}{6}[Radical ⼦]
  \begin{phonetics}{存}{cun2}[][HSK 3]
    \definition{v.}{existir; viver; sobreviver | armazenar; manter | acumular; coletar | depositar | sair com; verificar |reservar; reter | permanecer em equilíbrio; estar em estoque | estimar; abrigar}
  \end{phonetics}
\end{entry}

\begin{entry}{存在}{6,6}[Radicais ⼦、⼟]
  \begin{phonetics}{存在}{cun2zai4}[][HSK 3]
    \definition{s.}{existência; ser; ente}
    \definition{v.}{existir; ser}
  \end{phonetics}
\end{entry}

\begin{entry}{孙女}{6,3}[Radicais ⼦、⼥]
  \begin{phonetics}{孙女}{sun1nv3}[][HSK 4]
    \definition{s.}{filha do filho; neta}
  \end{phonetics}
\end{entry}

\begin{entry}{孙子}{6,3}[Radicais ⼦、⼦]
  \begin{phonetics}{孙子}{sun1zi3}
    \definition*{s.}{Sun Tzu, também conhecido por Sun Wu (孙武), general, estrategista e filósofo autor do ``Arte da Guerra'' (孙子兵法)}
    \seeref{孙武}{sun1wu3}
  \seealsoref{孙子兵法}{sun1zi3 bing1fa3}
  \end{phonetics}
  \begin{phonetics}{孙子}{sun1zi5}[][HSK 4]
    \definition{s.}{filho do filho; neto}
  \end{phonetics}
\end{entry}

\begin{entry}{孙子兵法}{6,3,7,8}[Radicais ⼦、⼦、⼋、⽔]
  \begin{phonetics}{孙子兵法}{sun1zi3 bing1fa3}
    \definition*{s.}{``Arte da Guerra'', escrito por Sun Tzu (孫子)}
    \seeref{孙武}{sun1wu3}
    \seeref{孙子}{sun1zi3}
  \end{phonetics}
\end{entry}

\begin{entry}{孙武}{6,8}[Radicais ⼦、⽌]
  \begin{phonetics}{孙武}{sun1wu3}
    \definition*{s.}{Sun Wu, também conhecido por Sun Tzu (孙子), general, estrategista e filósofo autor do ``Arte da Guerra'' (孙子兵法)}
    \seeref{孙子}{sun1zi3}
  \seealsoref{孙子兵法}{sun1zi3 bing1fa3}
  \end{phonetics}
\end{entry}

\begin{entry}{宇宙}{6,8}[Radicais ⼧、⼧]
  \begin{phonetics}{宇宙}{yu3zhou4}
    \definition{s.}{universo | cosmos}
  \end{phonetics}
\end{entry}

\begin{entry}{宇航员}{6,10,7}[Radicais ⼧、⾈、⼝]
  \begin{phonetics}{宇航员}{yu3hang2yuan2}
    \definition{s.}{astronauta}
  \end{phonetics}
\end{entry}

\begin{entry}{守}{6}[Radical ⼧]
  \begin{phonetics}{守}{shou3}[][HSK 4]
    \definition*{s.}{sobrenome Shou}
    \definition{adv.}{próximo; perto de; perto de algum lugar em posição, perto de algum lugar}
    \definition{v.}{guardar; defender; estar presente para cuidar; não ir embora | manter vigilância; defender do ataque do oponente em uma luta ou confronto | observar; cumprir; respeitar; fazer as coisas como elas devem ser feitas | manter, observar a integridade; honrar a palavra de alguém; manter a palavra de alguém}
  \end{phonetics}
\end{entry}

\begin{entry}{守门员}{6,3,7}[Radicais ⼧、⾨、⼝]
  \begin{phonetics}{守门员}{shou3men2yuan2}
    \definition{s.}{goleiro}
  \end{phonetics}
\end{entry}

\begin{entry}{安}{6}[Radical ⼧]
  \begin{phonetics}{安}{an1}[][HSK 4]
    \definition{adj.}{pacífico; quieto; tranquilo; calmo; estáve; sem perturbação | seguro; protegido; com boa saúde; em paz; bem}
    \definition{pron.}{onde; como}
    \definition{s.}{segurança; proteção; paz; conforto | ampère; (eletricidade) abreviação de ampère}
    \definition{v.}{deixar (a mente de alguém) à vontade; acalmar; estabilizar | satisfazer; estar satisfeito; sentir-se satisfeito e à vontade | colocar em uma posição adequada; encontrar um lugar para | instalar; consertar; encaixar; configurar | trazer (uma acusação contra alguém); dar (a alguém um apelido) | abrigar (uma intenção); manter; segurar}
  \end{phonetics}
\end{entry}

\begin{entry}{安全}{6,6}[Radicais ⼧、⼊]
  \begin{phonetics}{安全}{an1quan2}[][HSK 2]
    \definition{adj.}{seguro}
    \definition{s.}{segurança}
  \end{phonetics}
\end{entry}

\begin{entry}{安神}{6,9}[Radicais ⼧、⽰]
  \begin{phonetics}{安神}{an1shen2}
    \definition{v.+compl.}{acalmar os nervos | aliviar a inquietação pela tranquilização da mente e do corpo}
  \end{phonetics}
\end{entry}

\begin{entry}{安家}{6,10}[Radicais ⼧、⼧]
  \begin{phonetics}{安家}{an1jia1}
    \definition{v.+compl.}{montar uma casa | estabelecer-se}
  \end{phonetics}
\end{entry}

\begin{entry}{安排}{6,11}[Radicais ⼧、⼿]
  \begin{phonetics}{安排}{an1pai2}[][HSK 3]
    \definition{s.}{arranjos | planos}
    \definition{v.}{organizar | programar | fazer planos}
  \end{phonetics}
\end{entry}

\begin{entry}{安装}{6,12}[Radicais ⼧、⾐]
  \begin{phonetics}{安装}{an1zhuang1}[][HSK 3]
    \definition{v.}{instalar | consertar | configurar}
  \end{phonetics}
\end{entry}

\begin{entry}{安置}{6,13}[Radicais ⼧、⽹]
  \begin{phonetics}{安置}{an1zhi4}[][HSK 4]
    \definition{v.}{providenciar; encontrar um lugar para; ajudar a estabelecer-se; colocar pessoas ou coisas em uma determinada posição ou organizá-las adequadamente}
  \end{phonetics}
\end{entry}

\begin{entry}{安静}{6,14}[Radicais ⼧、⾭]
  \begin{phonetics}{安静}{an1jing4}[][HSK 2]
    \definition{adj.}{quieto | tranquilo | pacífico | calmo}
  \end{phonetics}
\end{entry}

\begin{entry}{寺}{6}[Radical ⼨]
  \begin{phonetics}{寺}{si4}
    \definition{s.}{Templo Budista | Mesquita}
  \end{phonetics}
\end{entry}

\begin{entry}{寺庙}{6,8}[Radicais ⼨、⼴]
  \begin{phonetics}{寺庙}{si4miao4}
    \definition{s.}{templo | mosteiro | santuário}
  \end{phonetics}
\end{entry}

\begin{entry}{导致}{6,10}[Radicais ⼨、⾄]
  \begin{phonetics}{导致}{dao3zhi4}[][HSK 4]
    \definition{v.}{causar; levar a; dar origem a (um resultado ruim)}
  \end{phonetics}
\end{entry}

\begin{entry}{导弹}{6,11}[Radicais ⼨、⼸]
  \begin{phonetics}{导弹}{dao3dan4}
    \definition[枚]{s.}{míssil (guiado)}
  \end{phonetics}
\end{entry}

\begin{entry}{导游}{6,12}[Radicais ⼨、⽔]
  \begin{phonetics}{导游}{dao3you2}[][HSK 4]
    \definition[个,位,名]{s.}{guia turístico; pessoas que trabalham como guias turísticos}
    \definition{v.}{guiar; conduzir um passeio turístico}
  \end{phonetics}
\end{entry}

\begin{entry}{导演}{6,14}[Radicais ⼨、⽔]
  \begin{phonetics}{导演}{dao3yan3}[][HSK 3]
    \definition[位,名,个]{s.}{diretor}
    \definition{v.}{dirigir (um filme, peça, etc.)}
  \end{phonetics}
\end{entry}

\begin{entry}{尽力}{6,2}[Radicais ⼫、⼒]
  \begin{phonetics}{尽力}{jin4li4}[][HSK 4]
    \definition{v.+compl.}{esforçar-se ao máximo; esforçar-se ao máximo; usar toda a sua força; fazer algo com seu melhor esforço}
  \end{phonetics}
\end{entry}

\begin{entry}{尽快}{6,7}[Radicais ⼫、⼼]
  \begin{phonetics}{尽快}{jin3kuai4}[][HSK 4]
    \definition{adv.}{com toda a velocidade; o mais rápido possível; o mais breve possível}
  \end{phonetics}
\end{entry}

\begin{entry}{尽量}{6,12}[Radicais ⼫、⾥]
  \begin{phonetics}{尽量}{jin3liang4}[][HSK 3]
    \definition{adv.}{tanto quanto possível; da melhor maneira possível}
  \end{phonetics}
\end{entry}

\begin{entry}{尽管}{6,14}[Radicais ⼫、⽵]
  \begin{phonetics}{尽管}{jin3guan3}
    \definition{conj.}{no entanto | embora | apesar de}
  \end{phonetics}
\end{entry}

\begin{entry}{岁}{6}[Radical ⼭]
  \begin{phonetics}{岁}{sui4}[][HSK 1]
    \definition{clas.}{para anos (de idade)}
    \definition{s.}{idade | ano (idade ou colheita)}
  \end{phonetics}
\end{entry}

\begin{entry}{岂}{6}[Radical ⼭]
  \begin{phonetics}{岂}{qi3}
    \definition*{s.}{sobrenome Qi}
    \definition{adv.}{expressa uma pergunta retórica, equivalente a ``哪里'', ``怎么'' e ``难道''}
  \seealsoref{哪里}{na3 li3}
  \seealsoref{难道}{nan2dao4}
  \seealsoref{怎么}{zen3me5}
  \end{phonetics}
\end{entry}

\begin{entry}{岂有此理}{6,6,6,11}[Radicais ⼭、⽉、⽌、⽟]
  \begin{phonetics}{岂有此理}{qi3you3ci3li3}
    \definition{interj.}{Que exorbitante! | Absurdo! | Como isso pode ser assim? | Ridículo!}
  \end{phonetics}
\end{entry}

\begin{entry}{巡逻}{6,11}[Radicais ⾡、⾡]
  \begin{phonetics}{巡逻}{xun2luo2}
    \definition{s.}{patrulha}
    \definition{v.}{patrulhar (polícia, exército ou marinha)}
  \end{phonetics}
\end{entry}

\begin{entry}{师}{6}[Radical ⼱]
  \begin{phonetics}{师}{shi1}
    \definition*{s.}{sobrenome Shi}
    \definition{s.}{professor | mestre | especialista | modelo | divisão do exército}
    \definition{v.}{despachar tropas}
  \end{phonetics}
\end{entry}

\begin{entry}{师傅}{6,12}[Radicais ⼱、⼈]
  \begin{phonetics}{师傅}{shi1fu5}
    \definition[个,位,名]{s.}{técnico | mestre-trabalhador | forma respeitosa de tratamento para homens mais velhos}
  \end{phonetics}
\end{entry}

\begin{entry}{年}{6}[Radical ⼲]
  \begin{phonetics}{年}{nian2}[][HSK 1]
    \definition*{s.}{sobrenome Nian}
    \definition[个]{clas./s.}{ano}
  \end{phonetics}
\end{entry}

\begin{entry}{年代}{6,5}[Radicais ⼲、⼈]
  \begin{phonetics}{年代}{nian2dai4}[][HSK 3]
    \definition[个]{s.}{idade; anos; tempo | uma década de um século}
  \end{phonetics}
\end{entry}

\begin{entry}{年级}{6,6}[Radicais ⼲、⽷]
  \begin{phonetics}{年级}{nian2ji2}[][HSK 2]
    \definition[个]{s.}{classe | ano (escola)}
  \end{phonetics}
\end{entry}

\begin{entry}{年纪}{6,6}[Radicais ⼲、⽷]
  \begin{phonetics}{年纪}{nian2ji4}[][HSK 3]
    \definition{s.}{era; época; idade}
  \end{phonetics}
\end{entry}

\begin{entry}{年初}{6,7}[Radicais ⼲、⾐]
  \begin{phonetics}{年初}{nian2 chu1}[][HSK 3]
    \definition{s.}{o começo do ano}
  \end{phonetics}
\end{entry}

\begin{entry}{年底}{6,8}[Radicais ⼲、⼴]
  \begin{phonetics}{年底}{nian2 di3}[][HSK 3]
    \definition[个]{s.}{fim de ano; o fim do ano}
  \end{phonetics}
\end{entry}

\begin{entry}{年货}{6,8}[Radicais ⼲、⾙]
  \begin{phonetics}{年货}{nian2huo4}
    \definition{s.}{mercadorias vendidas no Ano Novo Chinês}
  \end{phonetics}
\end{entry}

\begin{entry}{年轻}{6,9}[Radicais ⼲、⾞]
  \begin{phonetics}{年轻}{nian2qing1}[][HSK 2]
    \definition{adj.}{jovem}
  \end{phonetics}
\end{entry}

\begin{entry}{并}{6}[Radical ⼲]
  \begin{phonetics}{并}{bing4}[][HSK 3,4]
    \definition{adv.}{igualmente; simultaneamente; lado a lado; coisas diferentes existem ao mesmo tempo; coisas diferentes estão acontecendo ao mesmo tempo | em absoluto (usado antes de uma negativa para dar ênfase);  usado antes de uma palavra negativa para reforçar o tom e refutá-la ligeiramente}
    \definition{conj.}{além de; e}
    \definition{v.}{combinar; fundir; incorporar; anexar; juntar}
  \end{phonetics}
\end{entry}

\begin{entry}{并且}{6,5}[Radicais ⼲、⼀]
  \begin{phonetics}{并且}{bing4qie3}[][HSK 3]
    \definition{conj.}{além disso | o que é mais | e}
  \end{phonetics}
\end{entry}

\begin{entry}{并排}{6,11}[Radicais ⼲、⼿]
  \begin{phonetics}{并排}{bing4pai2}
    \definition{adv.}{lado a lado}
  \end{phonetics}
\end{entry}

\begin{entry}{庆祝}{6,9}[Radicais ⼴、⽰]
  \begin{phonetics}{庆祝}{qing4zhu4}[][HSK 3]
    \definition{v.}{celebrar; comemorar; festejar}
  \end{phonetics}
\end{entry}

\begin{entry}{异常}{6,11}[Radicais ⼶、⼱]
  \begin{phonetics}{异常}{yi4chang2}
    \definition{adj.}{extraordinário | anormal}
    \definition{adv.}{extraordinariamente | excepcionalmente}
    \definition{s.}{anormalidade}
  \end{phonetics}
\end{entry}

\begin{entry}{式}{6}[Radical ⼷]
  \begin{phonetics}{式}{shi4}
    \definition{s.}{tipo | forma | padrão | estilo}
  \end{phonetics}
\end{entry}

\begin{entry}{当}{6}[Radical ⼹]
  \begin{phonetics}{当}{dang1}[][HSK 2]
    \definition*{s.}{sobrenome Dang}
    \definition{adj.}{igual}
    \definition{interj.}{(onomatopéia) barulho metálico, clangor}
    \definition{prep.}{na presença de alguém | na cara de alguém | exatamente em (um momento ou lugar)}
    \definition{v.}{trabalhar como | servir como | ser | suportar | aceitar | merecer | dirigir | gerir | estar encarregado de | deveria | deve}
  \end{phonetics}
  \begin{phonetics}{当}{dang4}
    \definition{adj.}{próprio; certo}
    \definition{pron.}{naquele mesmo (dia, etc.)}
    \definition{s.}{algo penhorado | penhor | promessa}
    \definition{v.}{combinar | igualar a tratar como | considerar como | tomar para pensar | penhorar}
  \end{phonetics}
\end{entry}

\begin{entry}{当中}{6,4}[Radicais ⼹、⼁]
  \begin{phonetics}{当中}{dang1 zhong1}[][HSK 3]
    \definition{prep.}{no meio; no centro | entre}
  \end{phonetics}
\end{entry}

\begin{entry}{当地}{6,6}[Radicais ⼹、⼟]
  \begin{phonetics}{当地}{dang1di4}
    \definition{s.}{local}
  \end{phonetics}
\end{entry}

\begin{entry}{当初}{6,7}[Radicais ⼹、⾐]
  \begin{phonetics}{当初}{dang1chu1}[][HSK 3]
    \definition{s.}{no começo; originalmente; no início; em primeiro lugar}
  \end{phonetics}
\end{entry}

\begin{entry}{当时}{6,7}[Radicais ⼹、⽇]
  \begin{phonetics}{当时}{dang1shi2}[][HSK 2]
    \definition{s.}{aquela época}
    \definition{v.}{ser o momento certo; acontecer na hora certa}
  \end{phonetics}
  \begin{phonetics}{当时}{dang4shi2}
    \definition{adv.}{imediatamente}
  \end{phonetics}
\end{entry}

\begin{entry}{当然}{6,12}[Radicais ⼹、⽕]
  \begin{phonetics}{当然}{dang1ran2}[][HSK 3]
    \definition{adj.}{natural; verdadeiro; espontâneo}
    \definition{adv.}{sem dúvida; certamente; claro}
  \end{phonetics}
\end{entry}

\begin{entry}{忙}{6}[Radical ⼼]
  \begin{phonetics}{忙}{mang2}[][HSK 1]
    \definition{adj.}{ocupado}
    \definition{s.}{apressar}
  \end{phonetics}
\end{entry}

\begin{entry}{戏}{6}[Radical ⼽]
  \begin{phonetics}{戏}{xi4}
    \definition[出,场,台]{s.}{drama | peça de teatro | \emph{show}}
  \end{phonetics}
\end{entry}

\begin{entry}{戏弄}{6,7}[Radicais ⼽、⼶]
  \begin{phonetics}{戏弄}{xi4nong4}
    \definition{v.}{zombar de | pregar peças | provocar}
  \end{phonetics}
\end{entry}

\begin{entry}{戏法}{6,8}[Radicais ⼽、⽔]
  \begin{phonetics}{戏法}{xi4fa3}
    \definition{s.}{truque de mágica | prestidigitação}
  \end{phonetics}
\end{entry}

\begin{entry}{戏耍}{6,9}[Radicais ⼽、⽽]
  \begin{phonetics}{戏耍}{xi4shua3}
    \definition{v.}{divertir-me | brincar com | provocar}
  \end{phonetics}
\end{entry}

\begin{entry}{戏院}{6,9}[Radicais ⼽、⾩]
  \begin{phonetics}{戏院}{xi4yuan4}
    \definition{s.}{teatro}
  \end{phonetics}
\end{entry}

\begin{entry}{戏剧}{6,10}[Radicais ⼽、⼑]
  \begin{phonetics}{戏剧}{xi4ju4}
    \definition{s.}{drama | suspense | teatro}
  \end{phonetics}
\end{entry}

\begin{entry}{戏剧化地}{6,10,4,6}[Radicais ⼽、⼑、⼔、⼟]
  \begin{phonetics}{戏剧化地}{xi4ju4hua4di4}
    \definition{adv.}{dramaticamente | teatralmente}
  \end{phonetics}
\end{entry}

\begin{entry}{戏剧性}{6,10,8}[Radicais ⼽、⼑、⼼]
  \begin{phonetics}{戏剧性}{xi4ju4xing4}
    \definition{adj.}{dramático}
  \end{phonetics}
\end{entry}

\begin{entry}{戏剧家}{6,10,10}[Radicais ⼽、⼑、⼧]
  \begin{phonetics}{戏剧家}{xi4ju4jia1}
    \definition{s.}{dramaturgo}
  \end{phonetics}
\end{entry}

\begin{entry}{戏剧效果}{6,10,10,8}[Radicais ⼽、⼑、⽁、⽊]
  \begin{phonetics}{戏剧效果}{xi4ju4xiao4guo3}
    \definition{s.}{efeito dramático}
  \end{phonetics}
\end{entry}

\begin{entry}{戏剧般}{6,10,10}[Radicais ⼽、⼑、⾈]
  \begin{phonetics}{戏剧般}{xi4ju4ban1}
    \definition{adj.}{melodramático}
  \end{phonetics}
\end{entry}

\begin{entry}{戏剧编剧}{6,10,12,10}[Radicais ⼽、⼑、⽷、⼑]
  \begin{phonetics}{戏剧编剧}{xi4ju4bian1ju4}
    \definition{s.}{dramaturgo}
  \end{phonetics}
\end{entry}

\begin{entry}{戏剧演出}{6,10,14,5}[Radicais ⼽、⼑、⽔、⼐]
  \begin{phonetics}{戏剧演出}{xi4ju4yan3chu1}
    \definition{s.}{performance dramática}
  \end{phonetics}
\end{entry}

\begin{entry}{戏谑}{6,11}[Radicais ⼽、⾔]
  \begin{phonetics}{戏谑}{xi4xue4}
    \definition{v.}{brincar | fazer piadas | ridicularizar}
  \end{phonetics}
\end{entry}

\begin{entry}{成}{6}[Radical ⼽]
  \begin{phonetics}{成}{cheng2}[][HSK 2]
    \definition*{s.}{sobrenome Cheng}
    \definition{v.}{sair-se bem | ser bem sucedido}
  \end{phonetics}
\end{entry}

\begin{entry}{成人}{6,2}[Radicais ⼽、⼈]
  \begin{phonetics}{成人}{cheng2ren2}[][HSK 4]
    \definition[个]{s.}{adulto; crescido; pessoa adulta}
    \definition{v.}{crescer; tornar-se adulto}
  \end{phonetics}
\end{entry}

\begin{entry}{成为}{6,4}[Radicais ⼽、⼂]
  \begin{phonetics}{成为}{cheng2wei2}[][HSK 2]
    \definition{s.}{tornar-se | transformar-se em}
  \end{phonetics}
\end{entry}

\begin{entry}{成长}{6,4}[Radicais ⼽、⾧]
  \begin{phonetics}{成长}{cheng2zhang3}[][HSK 3]
    \definition{v.}{crescer; amadurecer; amadurar}
  \end{phonetics}
\end{entry}

\begin{entry}{成功}{6,5}[Radicais ⼽、⼒]
  \begin{phonetics}{成功}{cheng2gong1}[][HSK 3]
    \definition{adj.}{bem-sucedido | frutífero}
    \definition[个,次]{s.}{sucesso}
    \definition{v.}{ter sucesso}
  \end{phonetics}
\end{entry}

\begin{entry}{成立}{6,5}[Radicais ⼽、⽴]
  \begin{phonetics}{成立}{cheng2li4}[][HSK 3]
    \definition{v.}{fundar; estabelecer; montar | ser válido; ser sustentável; reter água}
  \end{phonetics}
\end{entry}

\begin{entry}{成吉思汗}{6,6,9,6}[Radicais ⼽、⼝、⼼、⽔]
  \begin{phonetics}{成吉思汗}{cheng2ji2si1han2}
    \definition*{s.}{Genghis Khan (1162-1227), fundador e governante do Império Mongol}
  \end{phonetics}
\end{entry}

\begin{entry}{成色}{6,6}[Radicais ⼽、⾊]
  \begin{phonetics}{成色}{cheng2se4}
    \definition{v.}{sair-se bem | ser bem sucedido}
  \end{phonetics}
\end{entry}

\begin{entry}{成员}{6,7}[Radicais ⼽、⼝]
  \begin{phonetics}{成员}{cheng2yuan2}[][HSK 3]
    \definition[个]{s.}{membro}
  \end{phonetics}
\end{entry}

\begin{entry}{成批}{6,7}[Radicais ⼽、⼿]
  \begin{phonetics}{成批}{cheng2pi1}
    \definition{s.}{em lotes | a granel}
  \end{phonetics}
\end{entry}

\begin{entry}{成果}{6,8}[Radicais ⼽、⽊]
  \begin{phonetics}{成果}{cheng2guo3}[][HSK 3]
    \definition{s.}{realização; resultado}
  \end{phonetics}
\end{entry}

\begin{entry}{成活}{6,9}[Radicais ⼽、⽔]
  \begin{phonetics}{成活}{cheng2huo2}
    \definition{v.}{sobreviver}
  \end{phonetics}
\end{entry}

\begin{entry}{成家}{6,10}[Radicais ⼽、⼧]
  \begin{phonetics}{成家}{cheng2jia1}
    \definition{v.}{tornar-se um especialista reconhecido | estabelecer-se e casar-se (de um homem)}
  \end{phonetics}
\end{entry}

\begin{entry}{成都}{6,10}[Radicais ⼽、⾢]
  \begin{phonetics}{成都}{cheng2du1}
    \definition*{s.}{Chengdu}
  \end{phonetics}
\end{entry}

\begin{entry}{成婚}{6,11}[Radicais ⼽、⼥]
  \begin{phonetics}{成婚}{cheng2hun1}
    \definition{v.}{casar-se}
  \end{phonetics}
\end{entry}

\begin{entry}{成绩}{6,11}[Radicais ⼽、⽷]
  \begin{phonetics}{成绩}{cheng2ji4}[][HSK 2]
    \definition[项,个]{s.}{nota | classificação}
  \end{phonetics}
\end{entry}

\begin{entry}{成就}{6,12}[Radicais ⼽、⼪]
  \begin{phonetics}{成就}{cheng2jiu4}[][HSK 3]
    \definition[个]{s.}{realização; sucesso}
    \definition{v.}{realizar; atingir; completar}
  \end{phonetics}
\end{entry}

\begin{entry}{成熟}{6,15}[Radicais ⼽、⽕]
  \begin{phonetics}{成熟}{cheng2shu2}[][HSK 3]
    \definition{adj./s.}{maduro; totalmente crescido}
    \definition{v.}{amadurecer; estar maduro; estar totalmente crescido}
  \end{phonetics}
\end{entry}

\begin{entry}{成器}{6,16}[Radicais ⼽、⼝]
  \begin{phonetics}{成器}{cheng2qi4}
    \definition{v.}{tornar-se uma pessoa digna de respeito | fazer algo de si mesmo}
  \end{phonetics}
\end{entry}

\begin{entry}{扛}{6}[Radical ⼿]
  \begin{phonetics}{扛}{gang1}
    \definition{v.}{levantar com as duas mãos | carregar alguma coisa juntos (duas ou mais pessoas)}
  \end{phonetics}
  \begin{phonetics}{扛}{kang2}
    \definition{v.}{carregar no ombro de alguém |  (fig.) assumir (um fardo, dever, etc.)}
  \end{phonetics}
\end{entry}

\begin{entry}{执着}{6,11}[Radicais ⼿、⽬]
  \begin{phonetics}{执着}{zhi2zhuo2}
    \definition{s.}{(budismo) apego}
    \definition{v.}{estar fortemente apegado a | ser dedicado | apegar-se a}
  \end{phonetics}
\end{entry}

\begin{entry}{扩大}{6,3}[Radicais ⼿、⼤]
  \begin{phonetics}{扩大}{kuo4da4}[][HSK 4]
    \definition{v.}{ampliar; expandir; estender; alargar}
  \end{phonetics}
\end{entry}

\begin{entry}{扩展}{6,10}[Radicais ⼿、⼫]
  \begin{phonetics}{扩展}{kuo4 zhan3}[][HSK 4]
    \definition{v.}{esticar; expandir; estender; espalhar}
  \end{phonetics}
\end{entry}

\begin{entry}{扫}{6}[Radical ⼿]
  \begin{phonetics}{扫}{sao3}[][HSK 4]
    \definition{v.}{varrer; limpar | passar rapidamente ao longo ou sobre; varrer | juntar tudo}
  \end{phonetics}
  \begin{phonetics}{扫}{sao4}
    \seeref{扫帚}{sao4zhou5}
  \end{phonetics}
\end{entry}

\begin{entry}{扫兴}{6,6}[Radicais ⼿、⼋]
  \begin{phonetics}{扫兴}{sao3xing4}
    \definition{v.+compl.}{sentir-se decepcionado | entristecer alguém}
  \end{phonetics}
\end{entry}

\begin{entry}{扫帚}{6,8}[Radicais ⼿、⼱]
  \begin{phonetics}{扫帚}{sao4zhou5}
    \definition[把]{s.}{vassoura; ferramenta de varredura feita de varas de bambu, etc., maior que uma vassora}
  \end{phonetics}
\end{entry}

\begin{entry}{扬雄}{6,12}[Radicais ⼿、⾫]
  \begin{phonetics}{扬雄}{yang2xiong2}
    \definition*{s.}{Yang Xiong (53 AC-18 DC), estudioso, poeta e lexicógrafo, autor do primeiro dicionário de dialeto chinês 方言}
  \seealsoref{方言}{fang1yan2}
  \end{phonetics}
\end{entry}

\begin{entry}{收}{6}[Radical ⽁]
  \begin{phonetics}{收}{shou1}[][HSK 2]
    \definition{expr.}{aos cuidados de (usado na linha de endereço após o nome)}
    \definition{v.}{receber | aceitar | coletar | colher | guardar}
  \end{phonetics}
\end{entry}

\begin{entry}{收入}{6,2}[Radicais ⽁、⼊]
  \begin{phonetics}{收入}{shou1ru4}[][HSK 2]
    \definition[笔,个]{s.}{renda | salário}
    \definition{v.}{receber dinheiro |  coletar | receber}
  \end{phonetics}
\end{entry}

\begin{entry}{收买}{6,6}[Radicais ⽁、⼄]
  \begin{phonetics}{收买}{shou1mai3}
    \definition{v.}{subornar | comprar}
  \end{phonetics}
\end{entry}

\begin{entry}{收回}{6,6}[Radicais ⽁、⼞]
  \begin{phonetics}{收回}{shou1 hui2}[][HSK 4]
    \definition{v.}{retomar; recuperar; relembrar; recordar; receber de volta o que foi enviado ou emprestado, ou o dinheiro que foi emprestado ou usado | sacar; retirar; recolher; rescindir; cancelar (uma opinião, ordem, etc.)}
  \end{phonetics}
\end{entry}

\begin{entry}{收听}{6,7}[Radicais ⽁、⼝]
  \begin{phonetics}{收听}{shou1 ting1}[][HSK 3]
    \definition{v.}{ouvir; escutar}
  \end{phonetics}
\end{entry}

\begin{entry}{收到}{6,8}[Radicais ⽁、⼑]
  \begin{phonetics}{收到}{shou1 dao4}[][HSK 2]
    \definition{v.}{receber}
  \end{phonetics}
\end{entry}

\begin{entry}{收看}{6,9}[Radicais ⽁、⽬]
  \begin{phonetics}{收看}{shou1 kan4}[][HSK 3]
    \definition{v.}{assistir (a um programa de TV)}
  \end{phonetics}
\end{entry}

\begin{entry}{收费}{6,9}[Radicais ⽁、⾙]
  \begin{phonetics}{收费}{shou1 fei4}[][HSK 3]
    \definition{v.}{cobrar; cobrar taxas}
  \end{phonetics}
\end{entry}

\begin{entry}{收音机}{6,9,6}[Radicais ⽁、⾳、⽊]
  \begin{phonetics}{收音机}{shou1yin1ji1}[][HSK 3]
    \definition[部,台]{s.}{rádio; sem fio}
  \end{phonetics}
\end{entry}

\begin{entry}{收益}{6,10}[Radicais ⽁、⽫]
  \begin{phonetics}{收益}{shou1yi4}[][HSK 4]
    \definition{s.}{lucro; renda; benefício; ganhos; vantagens ou benefícios obtidos}
  \end{phonetics}
\end{entry}

\begin{entry}{收获}{6,10}[Radicais ⽁、⾋]
  \begin{phonetics}{收获}{shou1huo4}[][HSK 4]
    \definition[次,番,份]{s.}{resultados; ganhos; metaforicamente falando, conhecimento, experiência, etc. obtidos em estudo ou trabalho; os resultados obtidos por meio de trabalho árduo | colheita; colheita de safras}
    \definition{v.}{colher; juntar as colheitas;}
  \end{phonetics}
\end{entry}

\begin{entry}{收据}{6,11}[Radicais ⽁、⼿]
  \begin{phonetics}{收据}{shou1ju4}
    \definition[张]{s.}{recibo | \emph{voucher}}
  \end{phonetics}
\end{entry}

\begin{entry}{收敛}{6,11}[Radicais ⽁、⽁]
  \begin{phonetics}{收敛}{shou1lian3}
    \definition{v.}{diminuir | desaparecer | fazer desaparecer | exercer restrição | conter (alegria, arrogância, etc.) | constringir | (matemática) convergir}
  \end{phonetics}
\end{entry}

\begin{entry}{早}{6}[Radical ⽇]
  \begin{phonetics}{早}{zao3}[][HSK 1]
    \definition{adj.}{prematuramente}
    \definition{adv.}{cedo | antecipadamente | breve}
    \definition{s.}{manhã}
  \end{phonetics}
\end{entry}

\begin{entry}{早上}{6,3}[Radicais ⽇、⼀]
  \begin{phonetics}{早上}{zao3shang5}[][HSK 1]
    \definition{adv.}{manhã cedo | manhãzinha}
    \definition[个]{s.}{manhã}
  \end{phonetics}
\end{entry}

\begin{entry}{早亡}{6,3}[Radicais ⽇、⼇]
  \begin{phonetics}{早亡}{zao3wang2}
    \definition[个]{s.}{morte prematura}
    \definition{v.}{morrer prematuramente}
  \end{phonetics}
\end{entry}

\begin{entry}{早已}{6,3}[Radicais ⽇、⼰]
  \begin{phonetics}{早已}{zao3 yi3}[][HSK 3]
    \definition{adv.}{há muito tempo; por muito tempo}
  \end{phonetics}
\end{entry}

\begin{entry}{早车}{6,4}[Radicais ⽇、⾞]
  \begin{phonetics}{早车}{zao3che1}
    \definition{s.}{trem matutino | ônibus matutino}
  \end{phonetics}
\end{entry}

\begin{entry}{早安}{6,6}[Radicais ⽇、⼧]
  \begin{phonetics}{早安}{zao3'an1}
    \definition{interj.}{Bom dia!}
  \end{phonetics}
\end{entry}

\begin{entry}{早早儿}{6,6,2}[Radicais ⽇、⽇、⼉]
  \begin{phonetics}{早早儿}{zao3zao3r5}
    \definition{adv.}{o mais cedo possível | o mais breve possível}
  \end{phonetics}
\end{entry}

\begin{entry}{早饭}{6,7}[Radicais ⽇、⾷]
  \begin{phonetics}{早饭}{zao3fan4}[][HSK 1]
    \definition[份,顿,次,餐]{s.}{café da manhã}
  \end{phonetics}
\end{entry}

\begin{entry}{早知}{6,8}[Radicais ⽇、⽮]
  \begin{phonetics}{早知}{zao3zhi1}
    \definition{v.}{prever | se alguém soubesse antes, \dots}
  \end{phonetics}
\end{entry}

\begin{entry}{早前}{6,9}[Radicais ⽇、⼑]
  \begin{phonetics}{早前}{zao3qian2}
    \definition{adv.}{previamente}
  \end{phonetics}
\end{entry}

\begin{entry}{早晨}{6,11}[Radicais ⽇、⽇]
  \begin{phonetics}{早晨}{zao3 chen2}[][HSK 2]
    \definition{adv.}{manhã cedo | manhãzinha}
    \definition[个]{s.}{manhã}
  \end{phonetics}
\end{entry}

\begin{entry}{早就}{6,12}[Radicais ⽇、⼪]
  \begin{phonetics}{早就}{zao3 jiu4}[][HSK 2]
    \definition{adv.}{já em um momento anterior}
  \end{phonetics}
\end{entry}

\begin{entry}{早餐}{6,16}[Radicais ⽇、⾷]
  \begin{phonetics}{早餐}{zao3 can1}[][HSK 2]
    \definition[份,顿,次]{s.}{café da manhã}
  \end{phonetics}
\end{entry}

\begin{entry}{曲棍球}{6,12,11}[Radicais ⽈、⽊、⽟]
  \begin{phonetics}{曲棍球}{qu1gun4qiu2}
    \definition{s.}{hóquei em campo}
  \end{phonetics}
\end{entry}

\begin{entry}{有}{6}[Radical ⽉]
  \begin{phonetics}{有}{you3}[][HSK 1]
    \definition{v.}{ter | haver | existir}
  \end{phonetics}
\end{entry}

\begin{entry}{有(一)点儿}{6,1,9,2}[Radicais ⽉、⼀、⽕、⼉]
  \begin{phonetics}{有(一)点儿}{you3yi4dian3r5}[][HSK 2]
    \definition{adv.}{um pouco (``有点儿+s. ou v. mental'')}
  \end{phonetics}
\end{entry}

\begin{entry}{有人}{6,2}[Radicais ⽉、⼈]
  \begin{phonetics}{有人}{you3 ren2}[][HSK 2]
    \definition{pron.}{qualquer um | alguém}
    \definition[所]{s.}{ocupado (como no banheiro) | pessoas}
  \end{phonetics}
\end{entry}

\begin{entry}{有用}{6,5}[Radicais ⽉、⽤]
  \begin{phonetics}{有用}{you3yong4}[][HSK 1]
    \definition{adj.}{útil}
  \end{phonetics}
\end{entry}

\begin{entry}{有名}{6,6}[Radicais ⽉、⼝]
  \begin{phonetics}{有名}{you3 ming2}[][HSK 1]
    \definition{adj.}{famoso | conhecido}
  \end{phonetics}
\end{entry}

\begin{entry}{有名无实}{6,6,4,8}[Radicais ⽉、⼝、⽆、⼧]
  \begin{phonetics}{有名无实}{you3ming2wu2shi2}
    \definition{v.}{(literal) tem um nome, mas não tem realidade | existe apenas no nome}
  \end{phonetics}
\end{entry}

\begin{entry}{有利}{6,7}[Radicais ⽉、⼑]
  \begin{phonetics}{有利}{you3li4}[][HSK 3]
    \definition{adj.}{benéfico; favorável; vantajoso}
  \end{phonetics}
\end{entry}

\begin{entry}{有时}{6,7}[Radicais ⽉、⽇]
  \begin{phonetics}{有时}{you3shi2}[][HSK 1]
    \definition{expr.}{às vezes | de vez em quando | de quando em quando}
  \seealsoref{有的时候}{you3de5shi2hou4}
  \seealsoref{有时候}{you3shi2hou5}
  \end{phonetics}
\end{entry}

\begin{entry}{有时候}{6,7,10}[Radicais ⽉、⽇、⼈]
  \begin{phonetics}{有时候}{you3shi2hou5}[][HSK 1]
    \definition{adv.}{às vezes | de vez em quando | de quando em quando}
  \seealsoref{有的时候}{you3de5shi2hou4}
  \seealsoref{有时}{you3shi2}
  \end{phonetics}
\end{entry}

\begin{entry}{有些}{6,8}[Radicais ⽉、⼆]
  \begin{phonetics}{有些}{you3 xie1}[][HSK 1]
    \definition{adv.}{um pouco | ao invés}
    \definition{pron.}{parte | algum}
    \definition{v.}{usado para indicar que há alguns, mas não muitos}
  \end{phonetics}
\end{entry}

\begin{entry}{有的}{6,8}[Radicais ⽉、⽩]
  \begin{phonetics}{有的}{you3de5}[][HSK 1]
    \definition{pron.}{algum, alguns}
  \end{phonetics}
\end{entry}

\begin{entry}{有的时候}{6,8,7,10}[Radicais ⽉、⽩、⽇、⼈]
  \begin{phonetics}{有的时候}{you3de5shi2hou4}
    \definition{adv.}{às vezes | de vez em quando | de quando em quando}
  \seealsoref{有时}{you3shi2}
  \seealsoref{有时候}{you3shi2hou5}
  \end{phonetics}
\end{entry}

\begin{entry}{有的是}{6,8,9}[Radicais ⽉、⽩、⽇]
  \begin{phonetics}{有的是}{you3 de5 shi4}[][HSK 3]
    \definition{expr.}{tem bastante; não falta; enfatiza que existem muitos}
  \end{phonetics}
\end{entry}

\begin{entry}{有空儿}{6,8,2}[Radicais ⽉、⽳、⼉]
  \begin{phonetics}{有空儿}{you3 kong4r5}[][HSK 2]
    \definition{v.}{ser livre}
  \end{phonetics}
\end{entry}

\begin{entry}{有限公司}{6,8,4,5}[Radicais ⽉、⾩、⼋、⼝]
  \begin{phonetics}{有限公司}{you3xian4gong1si1}
    \definition{s.}{companhia limitada | corporação}
  \end{phonetics}
\end{entry}

\begin{entry}{有效}{6,10}[Radicais ⽉、⽁]
  \begin{phonetics}{有效}{you3 xiao4}[][HSK 3]
    \definition{adj.}{válido; eficiente; eficaz}
  \end{phonetics}
\end{entry}

\begin{entry}{有道理}{6,12,11}[Radicais ⽉、⾡、⽟]
  \begin{phonetics}{有道理}{you3dao4li5}
    \definition{adj.}{razoável}
    \definition{v.}{fazer sentido}
  \end{phonetics}
\end{entry}

\begin{entry}{有意思}{6,13,9}[Radicais ⽉、⼼、⼼]
  \begin{phonetics}{有意思}{you3 yi4 si5}[][HSK 2]
    \definition{adj.}{interessante | agradável | significativo | divertido}
  \end{phonetics}
\end{entry}

\begin{entry}{机甲}{6,5}[Radicais ⽊、⽥]
  \begin{phonetics}{机甲}{ji1jia3}
    \definition{s.}{\emph{mecha} (robôs operados pelo homem em mangá japonês)}
  \end{phonetics}
\end{entry}

\begin{entry}{机会}{6,6}[Radicais ⽊、⼈]
  \begin{phonetics}{机会}{ji1hui4}[][HSK 2]
    \definition{s.}{chance | oportunidade}
  \end{phonetics}
\end{entry}

\begin{entry}{机场}{6,6}[Radicais ⽊、⼟]
  \begin{phonetics}{机场}{ji1chang3}[][HSK 1]
    \definition[家,处]{s.}{aeroporto | aeródromo}
  \end{phonetics}
\end{entry}

\begin{entry}{机构}{6,8}[Radicais ⽊、⽊]
  \begin{phonetics}{机构}{ji1gou4}[][HSK 4]
    \definition[所]{s.}{órgão; organização; instituição; instalações; aparelhamento; configuração | mecanismo; funcionamento interno de uma máquina ou unidade | estrutura interna de uma organização}
  \end{phonetics}
\end{entry}

\begin{entry}{机械}{6,11}[Radicais ⽊、⽊]
  \begin{phonetics}{机械}{ji1xie4}
    \definition{s.}{máquina | maquinaria | mecânica}
  \end{phonetics}
\end{entry}

\begin{entry}{机票}{6,11}[Radicais ⽊、⽰]
  \begin{phonetics}{机票}{ji1 piao4}[][HSK 1]
    \definition[张]{s.}{bilhete de avião}
  \seealsoref{飞机票}{fei1ji1 piao4}
  \end{phonetics}
\end{entry}

\begin{entry}{机遇}{6,12}[Radicais ⽊、⾡]
  \begin{phonetics}{机遇}{ji1yu4}[][HSK 4]
    \definition[个]{s.}{chance; oportunidade; circunstâncias favoráveis}
  \end{phonetics}
\end{entry}

\begin{entry}{机器}{6,16}[Radicais ⽊、⼝]
  \begin{phonetics}{机器}{ji1qi4}[][HSK 3]
    \definition[台,部,个]{s.}{máquina; maquinário; motor | aparelho; dispositivo}
  \end{phonetics}
\end{entry}

\begin{entry}{机器人}{6,16,2}[Radicais ⽊、⼝、⼈]
  \begin{phonetics}{机器人}{ji1qi4ren2}
    \definition{s.}{robô | androide}
  \end{phonetics}
\end{entry}

\begin{entry}{杀气}{6,4}[Radicais ⽊、⽓]
  \begin{phonetics}{杀气}{sha1qi4}
    \definition{s.}{espírito assassino | aura de morte}
    \definition{v.}{desabafar a raiva de alguém}
  \end{phonetics}
\end{entry}

\begin{entry}{杂志}{6,7}[Radicais ⽊、⼼]
  \begin{phonetics}{杂志}{za2zhi4}[][HSK 3]
    \definition[本,份,期]{s.}{diário; revista}
  \end{phonetics}
\end{entry}

\begin{entry}{杂志社}{6,7,7}[Radicais ⽊、⼼、⽰]
  \begin{phonetics}{杂志社}{za2zhi4she4}
    \definition{s.}{editora de revista}
  \end{phonetics}
\end{entry}

\begin{entry}{杂技}{6,7}[Radicais ⽊、⼿]
  \begin{phonetics}{杂技}{za2ji4}
    \definition[场]{s.}{acrobacia}
  \end{phonetics}
\end{entry}

\begin{entry}{权利}{6,7}[Radicais ⽊、⼑]
  \begin{phonetics}{权利}{quan2li4}[][HSK 4]
    \definition[项,种,个,条,份]{s.}{direito; interesse; os poderes e benefícios (em oposição a “deveres”) exercidos por um cidadão ou pessoa jurídica de acordo com a lei}
  \end{phonetics}
\end{entry}

\begin{entry}{次}{6}[Radical ⽋]
  \begin{phonetics}{次}{ci4}[][HSK 1,4]
    \definition*{s.}{sobrenome Ci}
    \definition{adj.}{de segunda categoria; de qualidade inferior}
    \definition{clas.}{para coisas ou ações que podem se repetir;}
    \definition{num.}{segundo; próximo}
    \definition{pref.}{(química) hipo-, raízes ácidas ou compostos contendo dois átomos de oxigênio a menos}
    \definition{s.}{ordem; sequência; classificação | local de parada em uma viagem; escala}
  \end{phonetics}
\end{entry}

\begin{entry}{欢乐}{6,5}[Radicais ⽋、⼃]
  \begin{phonetics}{欢乐}{huan1le4}[][HSK 3]
    \definition{adj.}{feliz; alegre}
  \end{phonetics}
\end{entry}

\begin{entry}{欢快}{6,7}[Radicais ⽋、⼼]
  \begin{phonetics}{欢快}{huan1kuai4}
    \definition{adj.}{feliz e sem ansiedade | vívido}
  \end{phonetics}
\end{entry}

\begin{entry}{欢迎}{6,7}[Radicais ⽋、⾡]
  \begin{phonetics}{欢迎}{huan1ying2}[][HSK 2]
    \definition{adj.}{bem-vindo}
    \definition{v.}{dar as boas-vindas | ser bem-vindo}
  \end{phonetics}
\end{entry}

\begin{entry}{此}{6}[Radical ⽌]
  \begin{phonetics}{此}{ci3}[][HSK 4]
    \definition{pron.}{esse; essa; isso; este; esta; isto | aqui e agora}
  \end{phonetics}
\end{entry}

\begin{entry}{此外}{6,5}[Radicais ⽌、⼣]
  \begin{phonetics}{此外}{ci3wai4}[][HSK 4]
    \definition{conj.}{além disso; em adição; além das coisas ou situações mencionadas acima}
  \end{phonetics}
\end{entry}

\begin{entry}{死}{6}[Radical ⽍]
  \begin{phonetics}{死}{si3}[][HSK 3]
    \definition{adj.}{até a morte | implacável; mortal | fixo; rígido; inflexível | intransitável; fechado}
    \definition{adv.}{extremamente; até a morte}
    \definition{v.}{morrer; falecer}
  \end{phonetics}
\end{entry}

\begin{entry}{死亡}{6,3}[Radicais ⽍、⼇]
  \begin{phonetics}{死亡}{si3wang2}
    \definition{s.}{morte}
    \definition{v.}{morrer}
  \end{phonetics}
\end{entry}

\begin{entry}{毕业}{6,5}[Radicais ⽐、⼀]
  \begin{phonetics}{毕业}{bi4ye4}[][HSK 4]
    \definition{s.}{formatura}
    \definition{v.+compl.}{formar-se}
  \end{phonetics}
\end{entry}

\begin{entry}{毕业生}{6,5,5}[Radicais ⽐、⼀、⽣]
  \begin{phonetics}{毕业生}{bi4 ye4 sheng1}[][HSK 4]
    \definition[个]{s.}{diplomado; graduado; bacharel; pessoa que recebeu um diploma, grau ou certificado}
  \end{phonetics}
\end{entry}

\begin{entry}{汗水}{6,4}[Radicais ⽔、⽔]
  \begin{phonetics}{汗水}{han4shui3}
    \definition*{s.}{Rio Han (Hanshui)}
    \definition{s.}{suor | transpiração}
  \end{phonetics}
\end{entry}

\begin{entry}{汗液}{6,11}[Radicais ⽔、⽔]
  \begin{phonetics}{汗液}{han4ye4}
    \definition{s.}{suor}
  \end{phonetics}
\end{entry}

\begin{entry}{汗腺}{6,13}[Radicais ⽔、⾁]
  \begin{phonetics}{汗腺}{han4xian4}
    \definition{s.}{glândula sudorípara}
  \end{phonetics}
\end{entry}

\begin{entry}{江}{6}[Radical ⽔]
  \begin{phonetics}{江}{jiang1}[][HSK 4]
    \definition*{s.}{Rio Changjiang | sobrenome Jiang}
    \definition[条,道]{s.}{rio grande}
  \end{phonetics}
\end{entry}

\begin{entry}{江水}{6,4}[Radicais ⽔、⽔]
  \begin{phonetics}{江水}{jiang1shui3}
    \definition{s.}{água do rio}
  \end{phonetics}
\end{entry}

\begin{entry}{江西}{6,6}[Radicais ⽔、⾑]
  \begin{phonetics}{江西}{jiang1xi1}
    \definition*{s.}{Jiangxi}
  \end{phonetics}
\end{entry}

\begin{entry}{江南水乡}{6,9,4,3}[Radicais ⽔、⼗、⽔、⼄]
  \begin{phonetics}{江南水乡}{jiang1nan2shui3xiang1}
    \definition*{s.}{Vila Aquática de Jiangnan | Cidades Aquáticas}
  \end{phonetics}
\end{entry}

\begin{entry}{池}{6}[Radical ⽔]
  \begin{phonetics}{池}{chi2}
    \definition*{s.}{sobrenome Chi}
    \definition{s.}{lagoa | reservatório | fosso}
  \end{phonetics}
\end{entry}

\begin{entry}{污水}{6,4}[Radicais ⽔、⽔]
  \begin{phonetics}{污水}{wu1shui3}
    \definition{s.}{esgoto}
  \end{phonetics}
\end{entry}

\begin{entry}{污染}{6,9}[Radicais ⽔、⽊]
  \begin{phonetics}{污染}{wu1ran3}
    \definition{s.}{poluição}
    \definition{v.}{poluir}
  \end{phonetics}
\end{entry}

\begin{entry}{污染区}{6,9,4}[Radicais ⽔、⽊、⼖]
  \begin{phonetics}{污染区}{wu1ran3qu1}
    \definition{s.}{área contaminada}
  \end{phonetics}
\end{entry}

\begin{entry}{污染物}{6,9,8}[Radicais ⽔、⽊、⽜]
  \begin{phonetics}{污染物}{wu1ran3wu4}
    \definition{s.}{poluente}
  \seealsoref{污染物质}{wu1ran3 wu4zhi4}
  \end{phonetics}
\end{entry}

\begin{entry}{污染物质}{6,9,8,8}[Radicais ⽔、⽊、⽜、⾙]
  \begin{phonetics}{污染物质}{wu1ran3 wu4zhi4}
    \definition{s.}{poluente}
  \seealsoref{污染物}{wu1ran3wu4}
  \end{phonetics}
\end{entry}

\begin{entry}{汤}{6}[Radical ⽔]
  \begin{phonetics}{汤}{shang1}
    \definition{s.}{correnteza forte}
  \end{phonetics}
  \begin{phonetics}{汤}{tang1}[][HSK 3]
    \definition*{s.}{sobrenome Tang}
    \definition{s.}{água quente; água fervente
fontes termais
sopa; caldo
uma preparação líquida de ervas medicinais; decocção}
  \end{phonetics}
\end{entry}

\begin{entry}{灯}{6}[Radical ⽕]
  \begin{phonetics}{灯}{deng1}[][HSK 2]
    \definition[盏]{s.}{lâmpada | lanterna | luz}
  \end{phonetics}
\end{entry}

\begin{entry}{灯丝}{6,5}[Radicais ⽕、⼀]
  \begin{phonetics}{灯丝}{deng1si1}
    \definition{s.}{filamento (de uma lâmpada)}
  \end{phonetics}
\end{entry}

\begin{entry}{灯号}{6,5}[Radicais ⽕、⼝]
  \begin{phonetics}{灯号}{deng1hao4}
    \definition{s.}{sinal luminoso | luz indicadora}
  \end{phonetics}
\end{entry}

\begin{entry}{灯光}{6,6}[Radicais ⽕、⼉]
  \begin{phonetics}{灯光}{deng1 guang1}[][HSK 4]
    \definition{s.}{iluminação; luminosidade da lâmpada | luminação (palco); equipamento de iluminação para palco ou estúdio}
  \end{phonetics}
\end{entry}

\begin{entry}{灯泡}{6,8}[Radicais ⽕、⽔]
  \begin{phonetics}{灯泡}{deng1pao4}
    \definition[个]{s.}{lâmpada | (gíria) terceiro indesejado estragando encontro de casal}
  \seealsoref{电灯泡}{dian4deng1pao4}
  \end{phonetics}
\end{entry}

\begin{entry}{灯标}{6,9}[Radicais ⽕、⽊]
  \begin{phonetics}{灯标}{deng1biao1}
    \definition{s.}{farol | luz de farol}
  \end{phonetics}
\end{entry}

\begin{entry}{灰色}{6,6}[Radicais ⽕、⾊]
  \begin{phonetics}{灰色}{hui1 se4}
    \definition{s.}{cor cinza}
  \end{phonetics}
\end{entry}

\begin{entry}{爷爷}{6,6}[Radicais ⽗、⽗]
  \begin{phonetics}{爷爷}{ye2ye5}[][HSK 1]
    \definition[个]{s.}{avô (paterno)}
  \end{phonetics}
\end{entry}

\begin{entry}{百}{6}[Radical ⽩]
  \begin{phonetics}{百}{bai3}[][HSK 1]
    \definition*{s.}{sobrenome Bai}
    \definition{num.}{cem; 100 | centena | cento}
  \end{phonetics}
\end{entry}

\begin{entry}{百分}{6,4}[Radicais ⽩、⼑]
  \begin{phonetics}{百分}{bai3fen1}
    \definition{num.}{por cento}
    \definition{s.}{porcentagem}
  \end{phonetics}
\end{entry}

\begin{entry}{百货}{6,8}[Radicais ⽩、⾙]
  \begin{phonetics}{百货}{bai3 huo4}[][HSK 4]
    \definition{s.}{mercadorias em geral; loja de departamentos; um termo geral para bens que incluem principalmente roupas, utensílios e necessidades diárias gerais}
  \end{phonetics}
\end{entry}

\begin{entry}{百般}{6,10}[Radicais ⽩、⾈]
  \begin{phonetics}{百般}{bai3ban1}
    \definition{adv.}{de todas as maneiras possíveis | por todos os meios}
  \end{phonetics}
\end{entry}

\begin{entry}{竹子}{6,3}[Radicais ⽵、⼦]
  \begin{phonetics}{竹子}{zhu2zi5}
    \definition[棵,支,根]{s.}{bambu}
  \end{phonetics}
\end{entry}

\begin{entry}{竹马}{6,3}[Radicais ⽵、⾺]
  \begin{phonetics}{竹马}{zhu2ma3}
    \definition{s.}{cavalo de bambu | vara de bambu usada como cavalo de brinquedo}
  \end{phonetics}
\end{entry}

\begin{entry}{竹排}{6,11}[Radicais ⽵、⼿]
  \begin{phonetics}{竹排}{zhu2pai2}
    \definition{s.}{jangada de bambu}
  \end{phonetics}
\end{entry}

\begin{entry}{竹编}{6,12}[Radicais ⽵、⽷]
  \begin{phonetics}{竹编}{zhu2bian1}
    \definition{s.}{vime | tecelagem de bambu}
  \end{phonetics}
\end{entry}

\begin{entry}{米}{6}[Kangxi 119][Radical ⽶]
  \begin{phonetics}{米}{mi3}[][HSK 2,3]
    \definition*{s.}{sobrenome Mi}
    \definition{clas.}{metro (m)}
    \definition{s.}{arroz | semente descascada}
  \end{phonetics}
\end{entry}

\begin{entry}{米饭}{6,7}[Radicais ⽶、⾷]
  \begin{phonetics}{米饭}{mi3fan4}[][HSK 1]
    \definition{s.}{arroz cozido}
  \end{phonetics}
\end{entry}

\begin{entry}{红}{6}[Radical ⽷]
  \begin{phonetics}{红}{hong2}[][HSK 2]
    \definition*{s.}{sobrenome Hong}
    \definition{adj.}{vermelho | popular | revolucionário}
    \definition{s.}{bônus}
  \end{phonetics}
\end{entry}

\begin{entry}{红包}{6,5}[Radicais ⽷、⼓]
  \begin{phonetics}{红包}{hong2 bao1}[][HSK 4]
    \definition[个]{s.}{saco de papel vermelho ou envelope contendo dinheiro como presente, gorjeta ou bônus | suborno; propina}
  \end{phonetics}
\end{entry}

\begin{entry}{红色}{6,6}[Radicais ⽷、⾊]
  \begin{phonetics}{红色}{hong2 se4}[][HSK 2]
    \definition{s.}{cor vermelha}
  \end{phonetics}
\end{entry}

\begin{entry}{红宝石}{6,8,5}[Radicais ⽷、⼧、⽯]
  \begin{phonetics}{红宝石}{hong2bao3shi2}
    \definition{s.}{rubi}
  \end{phonetics}
\end{entry}

\begin{entry}{红线}{6,8}[Radicais ⽷、⽷]
  \begin{phonetics}{红线}{hong2xian4}
    \definition{s.}{linha vermelha}
  \end{phonetics}
\end{entry}

\begin{entry}{红茶}{6,9}[Radicais ⽷、⾋]
  \begin{phonetics}{红茶}{hong2 cha2}[][HSK 3]
    \definition[杯,壶,斤,种]{s.}{chá preto}
  \end{phonetics}
\end{entry}

\begin{entry}{红烧}{6,10}[Radicais ⽷、⽕]
  \begin{phonetics}{红烧}{hong2shao1}
    \definition{s.}{guisado em molho de soja (prato)}
  \end{phonetics}
\end{entry}

\begin{entry}{红酒}{6,10}[Radicais ⽷、⾣]
  \begin{phonetics}{红酒}{hong2 jiu3}[][HSK 3]
    \definition{s.}{vinho tinto}
  \end{phonetics}
\end{entry}

\begin{entry}{红绿灯}{6,11,6}[Radicais ⽷、⽷、⽕]
  \begin{phonetics}{红绿灯}{hong2lv4deng1}
    \definition[个]{s.}{semáforo | sinal de trânsito}
  \end{phonetics}
\end{entry}

\begin{entry}{红薯}{6,16}[Radicais ⽷、⾋]
  \begin{phonetics}{红薯}{hong2shu3}
    \definition{s.}{batata doce}
  \end{phonetics}
\end{entry}

\begin{entry}{约}{6}[Radical ⽷]
  \begin{phonetics}{约}{yao1}
    \definition{adj.}{econômico; frugal | simples; breve | indistinto}
    \definition{adv.}{cerca de; ao redor; aproximadamente}
    \definition{s.}{pacto; acordo; nomeação; coisa prometida}
    \definition{v.}{marcar uma consulta; organizar | perguntar ou convidar com antecedência | restringir; conter | reduzir (fração aproximada)}
  \end{phonetics}
  \begin{phonetics}{约}{yue1}[][HSK 3]
    \definition*{s.}{sobrenome Yue}
    \definition{adj.}{econômico; frugal | simples; breve | indistinto}
    \definition{adv.}{cerca de; ao redor; aproximadamente}
    \definition{s.}{pacto; acordo; nomeação; coisa prometida}
    \definition{v.}{marcar uma consulta; organizar | perguntar ou convidar com antecedência | restringir; conter | reduzir (fração aproximada)}
  \end{phonetics}
\end{entry}

\begin{entry}{约会}{6,6}[Radicais ⽷、⼈]
  \begin{phonetics}{约会}{yue1hui4}
    \definition[次,个]{s.}{compromisso | encontro marcado}
  \end{phonetics}
\end{entry}

\begin{entry}{级}{6}[Radical ⽷]
  \begin{phonetics}{级}{ji2}[][HSK 2]
    \definition{clas.}{para passo, estágio}
    \definition{s.}{nível | classificação | grau | qualquer uma das divisões anuais de um curso escolar: série, classe, etc. | etapa}
  \end{phonetics}
\end{entry}

\begin{entry}{纪录}{6,8}[Radicais ⽷、⼹]
  \begin{phonetics}{纪录}{ji4lu4}[][HSK 3]
    \definition{s.}{recorde (esportes)}
  \end{phonetics}
\end{entry}

\begin{entry}{纪念}{6,8}[Radicais ⽷、⼼]
  \begin{phonetics}{纪念}{ji4nian4}[][HSK 3]
    \definition[个]{s.}{lembrança | aniversário (comemoração)}
    \definition{v.}{comemorar}
  \end{phonetics}
\end{entry}

\begin{entry}{纪律}{6,9}[Radicais ⽷、⼻]
  \begin{phonetics}{纪律}{ji4lv4}[][HSK 4]
    \definition{s.}{disciplina; código de conduta que cada membro da vida coletiva deve observar}
  \end{phonetics}
\end{entry}

\begin{entry}{网}{6}[Kangxi 122][Radical ⽹]
  \begin{phonetics}{网}{wang3}[][HSK 2]
    \definition{s.}{rede}
  \end{phonetics}
\end{entry}

\begin{entry}{网上}{6,3}[Radicais ⽹、⼀]
  \begin{phonetics}{网上}{wang3 shang4}[][HSK 1]
    \definition{s.}{\emph{online}}
  \end{phonetics}
\end{entry}

\begin{entry}{网上银行}{6,3,11,6}[Radicais ⽹、⼀、⾦、⾏]
  \begin{phonetics}{网上银行}{wang3shang4yin2hang2}
    \definition[个]{s.}{banco \emph{online} | acesso a operações bancárias via \emph{Internet}}
  \seealsoref{网银}{wang3yin2}
  \end{phonetics}
\end{entry}

\begin{entry}{网友}{6,4}[Radicais ⽹、⼜]
  \begin{phonetics}{网友}{wang3you3}[][HSK 1]
    \definition{s.}{internauta | usuário da \emph{Internet}}
  \end{phonetics}
\end{entry}

\begin{entry}{网址}{6,7}[Radicais ⽹、⼟]
  \begin{phonetics}{网址}{wang3 zhi3}[][HSK 4]
    \definition{s.}{\emph{website}; endereço da \emph{web}; endereço de um \emph{site} na \emph{Internet}, que os usuários podem acessar, consultar e obter recursos de informações nesse \emph{site} clicando nele}
  \end{phonetics}
\end{entry}

\begin{entry}{网际网络}{6,7,6,9}[Radicais ⽹、⾩、⽹、⽷]
  \begin{phonetics}{网际网络}{wang3ji4wang3luo4}
    \definition*{s.}{\emph{Internet}}
  \seealsoref{互联网}{hu4lian2wang3}
  \seealsoref{网际网路}{wang3ji4wang3lu4}
  \seealsoref{网路}{wang3lu4}
  \end{phonetics}
\end{entry}

\begin{entry}{网际网路}{6,7,6,13}[Radicais ⽹、⾩、⽹、⾜]
  \begin{phonetics}{网际网路}{wang3ji4wang3lu4}
    \definition*{s.}{\emph{Internet}}
  \seealsoref{互联网}{hu4lian2wang3}
  \seealsoref{网际网络}{wang3ji4wang3luo4}
  \seealsoref{网路}{wang3lu4}
  \end{phonetics}
\end{entry}

\begin{entry}{网络}{6,9}[Radicais ⽹、⽷]
  \begin{phonetics}{网络}{wang3luo4}[][HSK 4]
    \definition{s.}{rede; um sistema que consiste em ramificações interconectadas; em um sistema elétrico, um circuito ou parte de um circuito que consiste em vários elementos que permitem a transmissão de sinais elétricos de acordo com determinados requisitos | rede; rede de computadores}
  \end{phonetics}
\end{entry}

\begin{entry}{网站}{6,10}[Radicais ⽹、⽴]
  \begin{phonetics}{网站}{wang3zhan4}[][HSK 2]
    \definition[个,家]{s.}{\emph{website}}
  \end{phonetics}
\end{entry}

\begin{entry}{网罟}{6,10}[Radicais ⽹、⽹]
  \begin{phonetics}{网罟}{wang3gu3}
    \definition{s.}{(fig.) a rede da justiça | rede usada para capturar peixes (ou outros animais, como pássaros)}
  \end{phonetics}
\end{entry}

\begin{entry}{网球}{6,11}[Radicais ⽹、⽟]
  \begin{phonetics}{网球}{wang3qiu2}[][HSK 2]
    \definition{s.}{tênis (esporte)}
    \definition[个]{s.}{bola de tênis}
  \end{phonetics}
\end{entry}

\begin{entry}{网银}{6,11}[Radicais ⽹、⾦]
  \begin{phonetics}{网银}{wang3yin2}
    \definition{s.}{banco \emph{online} | acesso a operações bancárias via \emph{Internet}}
  \seealsoref{网上银行}{wang3shang4yin2hang2}
  \end{phonetics}
\end{entry}

\begin{entry}{网路}{6,13}[Radicais ⽹、⾜]
  \begin{phonetics}{网路}{wang3lu4}
    \definition{s.}{\emph{Internet}}
  \seealsoref{互联网}{hu4lian2wang3}
  \seealsoref{网际网路}{wang3ji4wang3lu4}
  \seealsoref{网际网络}{wang3ji4wang3luo4}
  \end{phonetics}
\end{entry}

\begin{entry}{羊}{6}[Kangxi 123][Radical ⽺]
  \begin{phonetics}{羊}{yang2}[][HSK 3]
    \definition*{s.}{sobrenome Yang}
    \definition{s.}{carneiro; ovelha; bode; cabra}
  \end{phonetics}
\end{entry}

\begin{entry}{羽毛}{6,4}[Radicais ⽻、⽑]
  \begin{phonetics}{羽毛}{yu3mao2}
    \definition{s.}{pena | plumagem | pluma}
  \end{phonetics}
\end{entry}

\begin{entry}{羽毛笔}{6,4,10}[Radicais ⽻、⽑、⽵]
  \begin{phonetics}{羽毛笔}{yu3mao2bi3}
    \definition{s.}{caneta de pena}
  \end{phonetics}
\end{entry}

\begin{entry}{羽毛球}{6,4,11}[Radicais ⽻、⽑、⽟]
  \begin{phonetics}{羽毛球}{yu3mao2qiu2}
    \definition[个]{s.}{\emph{badminton}}
  \end{phonetics}
\end{entry}

\begin{entry}{羽林}{6,8}[Radicais ⽻、⽊]
  \begin{phonetics}{羽林}{yu3lin2}
    \definition{s.}{escolta armada}
  \end{phonetics}
\end{entry}

\begin{entry}{羽冠}{6,9}[Radicais ⽻、⼍]
  \begin{phonetics}{羽冠}{yu3guan1}
    \definition{s.}{crista emplumada (de pássaro)}
  \end{phonetics}
\end{entry}

\begin{entry}{羽流}{6,10}[Radicais ⽻、⽔]
  \begin{phonetics}{羽流}{yu3liu2}
    \definition{s.}{pluma}
  \end{phonetics}
\end{entry}

\begin{entry}{老}{6}[Kangxi 125][Radical ⽼]
  \begin{phonetics}{老}{lao3}[][HSK 1,2]
    \definition{adj.}{velho (pessoa) | venerável (pessoa) | experiente | ultrapassado | duro (carne, etc.)}
    \definition{adv.}{de longa data | sempre | o tempo todo | do passado}
    \definition{pref.}{prefixo de nome: Lao}
  \end{phonetics}
\end{entry}

\begin{entry}{老人}{6,2}[Radicais ⽼、⼈]
  \begin{phonetics}{老人}{lao3 ren2}[][HSK 1]
    \definition[位]{s.}{pessoa idosa | o idoso | o velho}
  \end{phonetics}
\end{entry}

\begin{entry}{老人家}{6,2,10}[Radicais ⽼、⼈、⼧]
  \begin{phonetics}{老人家}{lao3 ren2 jia1}
    \definition{s.}{senhor ancião | madame | senhora | termo educado para mulher ou homem velho}
  \end{phonetics}
\end{entry}

\begin{entry}{老公}{6,4}[Radicais ⽼、⼋]
  \begin{phonetics}{老公}{lao3 gong1}[][HSK 4]
    \definition[个]{s.}{marido; esposo}
  \end{phonetics}
\end{entry}

\begin{entry}{老太太}{6,4,4}[Radicais ⽼、⼤、⼤]
  \begin{phonetics}{老太太}{lao3 tai4 tai5}[][HSK 3]
    \definition[位]{s.}{velha senhora; (em tratamento direto)Venerável Senhora; uma maneira respeitosa de chamar uma senhora idosa | (forma de tratamento) sua velha mãe; minha velha mãe, avó ou sogra}
  \end{phonetics}
\end{entry}

\begin{entry}{老头儿}{6,5,2}[Radicais ⽼、⼤、⼉]
  \begin{phonetics}{老头儿}{lao3 tou2r5}[][HSK 3]
    \definition{s.}{(coloquial) velho; velho rabugento}
  \seealsoref{老头子}{lao3 tou2zi5}
  \end{phonetics}
\end{entry}

\begin{entry}{老头子}{6,5,3}[Radicais ⽼、⼤、⼦]
  \begin{phonetics}{老头子}{lao3 tou2zi5}
    \definition{s.}{(coloquial) velho; velho rabugento}
  \seealsoref{老头儿}{lao3 tou2r5}
  \end{phonetics}
\end{entry}

\begin{entry}{老师}{6,6}[Radicais ⽼、⼱]
  \begin{phonetics}{老师}{lao3shi1}[][HSK 1]
    \definition[个,位]{s.}{professor}
  \end{phonetics}
\end{entry}

\begin{entry}{老年}{6,6}[Radicais ⽼、⼲]
  \begin{phonetics}{老年}{lao3 nian2}[][HSK 2]
    \definition{s.}{idoso | velhice}
  \end{phonetics}
\end{entry}

\begin{entry}{老百姓}{6,6,8}[Radicais ⽼、⽩、⼥]
  \begin{phonetics}{老百姓}{lao3bai3xing4}[][HSK 3]
    \definition[个]{s.}{povo(s); civis; pessoas comuns; pessoas ordinárias}
  \end{phonetics}
\end{entry}

\begin{entry}{老兵}{6,7}[Radicais ⽼、⼋]
  \begin{phonetics}{老兵}{lao3bing1}
    \definition{s.}{velho soldado | veterano de guerra | veterano (alguém que tem muita experiência em algum domínio)}
  \end{phonetics}
\end{entry}

\begin{entry}{老实}{6,8}[Radicais ⽼、⼧]
  \begin{phonetics}{老实}{lao3shi5}[][HSK 4]
    \definition{adj.}{franco; sincero; honesto | bom; bem-comportado | ingênuo; simplório; meio bobo; facilmente enganado; eufemismo para pouco inteligente}
  \end{phonetics}
\end{entry}

\begin{entry}{老朋友}{6,8,4}[Radicais ⽼、⽉、⼜]
  \begin{phonetics}{老朋友}{lao3 peng2 you3}[][HSK 2]
    \definition[个]{s.}{velho amigo}
  \end{phonetics}
\end{entry}

\begin{entry}{老板}{6,8}[Radicais ⽼、⽊]
  \begin{phonetics}{老板}{lao3ban3}[][HSK 3]
    \definition[个,位]{s.}{chefe; dono | tratamento respeitoso a uma estrela de ópera ou a um líder de trupe}
  \end{phonetics}
\end{entry}

\begin{entry}{老虎}{6,8}[Radicais ⽼、⾌]
  \begin{phonetics}{老虎}{lao3hu3}
    \definition[只]{s.}{tigre}
  \seealsoref{虎}{hu3}
  \end{phonetics}
\end{entry}

\begin{entry}{老是}{6,9}[Radicais ⽼、⽇]
  \begin{phonetics}{老是}{lao3 shi4}[][HSK 2]
    \definition{adv.}{sempre | todas as vezes}
  \end{phonetics}
\end{entry}

\begin{entry}{老家}{6,10}[Radicais ⽼、⼧]
  \begin{phonetics}{老家}{lao3 jia1}[][HSK 4]
    \definition{s.}{cidade natal; local de origem | lugar nativo; refere-se às gerações anteriores da família ou ao local onde a pessoa nasceu ou viveu}
  \end{phonetics}
\end{entry}

\begin{entry}{老婆}{6,11}[Radicais ⽼、⼥]
  \begin{phonetics}{老婆}{lao3po2}[][HSK 4]
    \definition[个]{s.}{esposa}
  \end{phonetics}
\end{entry}

\begin{entry}{考}{6}[Radical ⽼]
  \begin{phonetics}{考}{kao3}[][HSK 1]
    \definition*{s.}{sobrenome Kao}
    \definition{s.}{o pai falecido de alguém}
    \definition{v.}{examinar | dar (fazer) um exame, prova ou teste | verificar | inspecionar |estudar | investigar}
  \end{phonetics}
\end{entry}

\begin{entry}{考生}{6,5}[Radicais ⽼、⽣]
  \begin{phonetics}{考生}{kao3 sheng1}[][HSK 2]
    \definition{s.}{candidato a exame}
  \end{phonetics}
\end{entry}

\begin{entry}{考试}{6,8}[Radicais ⽼、⾔]
  \begin{phonetics}{考试}{kao3shi4}[][HSK 1]
    \definition[次]{s.}{teste | prova | exame}
    \definition{v.+compl.}{submeter-se a uma prova | fazer um teste}
  \end{phonetics}
\end{entry}

\begin{entry}{考虑}{6,10}[Radicais ⽼、⾌]
  \begin{phonetics}{考虑}{kao3lv4}[][HSK 4]
    \definition{v.}{considerar; refletir sobre; levar em conta}
  \end{phonetics}
\end{entry}

\begin{entry}{考验}{6,10}[Radicais ⽼、⾺]
  \begin{phonetics}{考验}{kao3yan4}[][HSK 3]
    \definition[场,个,种]{s.}{teste; julgamento}
    \definition{v.}{testar}
  \end{phonetics}
\end{entry}

\begin{entry}{考察}{6,14}[Radicais ⽼、⼧]
  \begin{phonetics}{考察}{kao3cha2}[][HSK 4]
    \definition{v.}{inspecionar; investigar; observar e estudar}
  \end{phonetics}
\end{entry}

\begin{entry}{而}{6}[Kangxi 126][Radical ⽽]
  \begin{phonetics}{而}{er2}[][HSK 4]
    \definition{conj.}{e (coordenação) | e ainda (restrição) | conexão de componentes com continuidade semântica | conecxão de componentes afirmativos e negativos que se complementam | conexão de componentes com significados opostos para indicar um contraste |  conexão de componentes de causa e efeito no raciocínio | significa “chegar” ou “alcançar” | conexão de componentes que indicam tempo ou modo ao verbo | inserido entre o sujeito e o predicado, significa ``如果'' (se)}
  \seealsoref{如果}{ru2guo3}
  \end{phonetics}
\end{entry}

\begin{entry}{而且}{6,5}[Radicais ⽽、⼀]
  \begin{phonetics}{而且}{er2 qie3}[][HSK 2]
    \definition{conj.}{muito menos | além disso | além do mais}
  \end{phonetics}
\end{entry}

\begin{entry}{而况}{6,7}[Radicais ⽽、⼎]
  \begin{phonetics}{而况}{er2kuang4}
    \definition{conj.}{além disso | além do mais}
  \end{phonetics}
\end{entry}

\begin{entry}{而是}{6,9}[Radicais ⽽、⽇]
  \begin{phonetics}{而是}{er2 shi4}[][HSK 4]
    \definition{conj.}{mas; em vez disso; geralmente usada em conjunto com ``不是'' para formar o correlativo ``不是……而是'', indicando uma relação paralela}
  \seealsoref{不是……而是}{bu4shi4 er2 shi4}
  \end{phonetics}
\end{entry}

\begin{entry}{耳朵}{6,6}[Radicais ⽿、⽊]
  \begin{phonetics}{耳朵}{er3duo5}
    \definition[只,个,对]{s.}{orelha}
  \end{phonetics}
\end{entry}

\begin{entry}{耳机}{6,6}[Radicais ⽿、⽊]
  \begin{phonetics}{耳机}{er3 ji1}[][HSK 4]
    \definition[副,个,对]{s.}{fone de ouvido; receptor (de telefone); dispositivos que permitem que uma pessoa ouça sons sozinha, como ouvir música, histórias, chamadas telefônicas etc., usados na cabeça ou inseridos nos ouvidos}
  \end{phonetics}
\end{entry}

\begin{entry}{肉}{6}[Kangxi 130][Radical ⾁]
  \begin{phonetics}{肉}{rou4}[][HSK 1]
    \definition{s.}{carne | polpa de uma fruta}
  \end{phonetics}
\end{entry}

\begin{entry}{肉桂}{6,10}[Radicais ⾁、⽊]
  \begin{phonetics}{肉桂}{rou4gui4}
    \definition{s.}{canela}
  \seealsoref{官桂}{guan1gui4}
  \end{phonetics}
\end{entry}

\begin{entry}{肌肉}{6,6}[Radicais ⾁、⾁]
  \begin{phonetics}{肌肉}{ji1rou4}
    \definition{s.}{músculo | carne}
  \end{phonetics}
\end{entry}

\begin{entry}{自个儿}{6,3,2}[Radicais ⾃、⼈、⼉]
  \begin{phonetics}{自个儿}{zi4ge3r5}
    \definition{pron.}{(dialeto) a si mesmo, por si mesmo}
  \end{phonetics}
\end{entry}

\begin{entry}{自己}{6,3}[Radicais ⾃、⼰]
  \begin{phonetics}{自己}{zi4ji3}[][HSK 2]
    \definition{pron.}{a si próprio | próprio}
  \end{phonetics}
\end{entry}

\begin{entry}{自己动手}{6,3,6,4}[Radicais ⾃、⼰、⼒、⼿]
  \begin{phonetics}{自己动手}{zi4ji3dong4shou3}
    \definition{v.}{fazer (algo) sozinho | ajudar-se a}
  \end{phonetics}
\end{entry}

\begin{entry}{自从}{6,4}[Radicais ⾃、⼈]
  \begin{phonetics}{自从}{zi4cong2}[][HSK 3]
    \definition{prep.}{de; desde; apresentando o ponto de partida de um determinado tempo ou evento no passado}
  \end{phonetics}
\end{entry}

\begin{entry}{自主}{6,5}[Radicais ⾃、⼂]
  \begin{phonetics}{自主}{zi4zhu3}[][HSK 3]
    \definition{v.}{agir por conta própria; decidir por si mesmo; manter a iniciativa em suas próprias mãos}
  \end{phonetics}
\end{entry}

\begin{entry}{自由}{6,5}[Radicais ⾃、⽥]
  \begin{phonetics}{自由}{zi4you2}[][HSK 2]
    \definition{adj.}{livre, irrestrito}
    \definition[种]{s.}{liberdade}
  \end{phonetics}
\end{entry}

\begin{entry}{自由泳}{6,5,8}[Radicais ⾃、⽥、⽔]
  \begin{phonetics}{自由泳}{zi4you2yong3}
    \definition{s.}{natação de estilo livre}
  \end{phonetics}
\end{entry}

\begin{entry}{自动}{6,6}[Radicais ⾃、⼒]
  \begin{phonetics}{自动}{zi4dong4}[][HSK 3]
    \definition{adj.}{automático; auto-atuante; uso de dispositivos mecânicos, elétricos e outros para operar de forma independente, sem controle humano}
    \definition{adv.}{voluntariamente; por vontade própria | automaticamente; espontaneamente; refere-se ao movimento, mudança, etc. que não é causado pelo poder humano, mas pelo próprio objeto}
  \end{phonetics}
\end{entry}

\begin{entry}{自动化}{6,6,4}[Radicais ⾃、⼒、⼔]
  \begin{phonetics}{自动化}{zi4dong4hua4}
    \definition{s.}{automação}
  \end{phonetics}
\end{entry}

\begin{entry}{自行车}{6,6,4}[Radicais ⾃、⾏、⾞]
  \begin{phonetics}{自行车}{zi4xing2che1}[][HSK 2]
    \definition[辆]{s.}{bicicleta}
  \end{phonetics}
\end{entry}

\begin{entry}{自行车架}{6,6,4,9}[Radicais ⾃、⾏、⾞、⽊]
  \begin{phonetics}{自行车架}{zi4xing2che1jia4}
    \definition{s.}{suporte para bicicleta | bicicletário}
  \end{phonetics}
\end{entry}

\begin{entry}{自行车馆}{6,6,4,11}[Radicais ⾃、⾏、⾞、⾷]
  \begin{phonetics}{自行车馆}{zi4xing2che1guan3}
    \definition{s.}{estádio de ciclismo | velódromo}
  \end{phonetics}
\end{entry}

\begin{entry}{自行车赛}{6,6,4,14}[Radicais ⾃、⾏、⾞、⾙]
  \begin{phonetics}{自行车赛}{zi4xing2che1sai4}
    \definition{s.}{corrida de bicicleta}
  \end{phonetics}
\end{entry}

\begin{entry}{自我}{6,7}[Radicais ⾃、⼽]
  \begin{phonetics}{自我}{zi4wo3}
    \definition{pref.}{auto}
    \definition{pron.}{a si mesmo | eu próprio | (psicologia) ego}
  \end{phonetics}
\end{entry}

\begin{entry}{自我介绍}{6,7,4,8}[Radicais ⾃、⼽、⼈、⽷]
  \begin{phonetics}{自我介绍}{zi4wo3jie4shao4}
    \definition{s.}{defesa pessoal | auto-defesa}
  \end{phonetics}
\end{entry}

\begin{entry}{自我安慰}{6,7,6,15}[Radicais ⾃、⼽、⼧、⼼]
  \begin{phonetics}{自我安慰}{zi4wo3'an1wei4}
    \definition{v.}{confortar-se | consolar-se | tranquilizar-se}
  \end{phonetics}
\end{entry}

\begin{entry}{自我防卫}{6,7,6,3}[Radicais ⾃、⼽、⾩、⼙]
  \begin{phonetics}{自我防卫}{zi4wo3fang2wei4}
    \definition{s.}{defesa pessoal | auto-defesa}
  \end{phonetics}
\end{entry}

\begin{entry}{自我吹嘘}{6,7,7,14}[Radicais ⾃、⼽、⼝、⼝]
  \begin{phonetics}{自我吹嘘}{zi4wo3chui1xu1}
    \definition{expr.}{tocar a própria buzina}
  \end{phonetics}
\end{entry}

\begin{entry}{自我批评}{6,7,7,7}[Radicais ⾃、⼽、⼿、⾔]
  \begin{phonetics}{自我批评}{zi4wo3pi1ping2}
    \definition{s.}{autocrítica}
  \end{phonetics}
\end{entry}

\begin{entry}{自我实现}{6,7,8,8}[Radicais ⾃、⼽、⼧、⾒]
  \begin{phonetics}{自我实现}{zi4wo3shi2xian4}
    \definition{s.}{(psicologia) auto-atualização, auto-realização}
  \end{phonetics}
\end{entry}

\begin{entry}{自我的人}{6,7,8,2}[Radicais ⾃、⼽、⽩、⼈]
  \begin{phonetics}{自我的人}{zi4wo3de5ren2}
    \definition{s.}{(minha, sua) própria pessoa | (afirmar) a própria personalidade}
  \end{phonetics}
\end{entry}

\begin{entry}{自我保存}{6,7,9,6}[Radicais ⾃、⼽、⼈、⼦]
  \begin{phonetics}{自我保存}{zi4wo3 bao3cun2}
    \definition{v.}{autopreservação}
  \end{phonetics}
\end{entry}

\begin{entry}{自我陶醉}{6,7,10,15}[Radicais ⾃、⼽、⾩、⾣]
  \begin{phonetics}{自我陶醉}{zi4wo3tao2zui4}
    \definition{s.}{narcisista | auto-imbuído | satisfeito consigo mesmo}
  \end{phonetics}
\end{entry}

\begin{entry}{自我催眠}{6,7,13,10}[Radicais ⾃、⼽、⼈、⽬]
  \begin{phonetics}{自我催眠}{zi4wo3cui1mian2}
    \definition{v.}{consolar-me | tranquilizar-me}
  \end{phonetics}
\end{entry}

\begin{entry}{自我意识}{6,7,13,7}[Radicais ⾃、⼽、⼼、⾔]
  \begin{phonetics}{自我意识}{zi4wo3yi4shi2}
    \definition{s.}{autoapresentação}
    \definition{v.}{apresentar-se}
  \end{phonetics}
\end{entry}

\begin{entry}{自我解嘲}{6,7,13,15}[Radicais ⾃、⼽、⾓、⼝]
  \begin{phonetics}{自我解嘲}{zi4wo3jie3chao2}
    \definition{s.}{referir-se às próprias fraquezas ou falhas com humor autodepreciativo}
  \end{phonetics}
\end{entry}

\begin{entry}{自来水}{6,7,4}[Radicais ⾃、⽊、⽔]
  \begin{phonetics}{自来水}{zi4lai2shui3}
    \definition{s.}{água corrente | água da torneira}
  \end{phonetics}
\end{entry}

\begin{entry}{自身}{6,7}[Radicais ⾃、⾝]
  \begin{phonetics}{自身}{zi4 shen1}[][HSK 3]
    \definition{pron.}{eu mesmo; si mesmo}
  \end{phonetics}
\end{entry}

\begin{entry}{自责}{6,8}[Radicais ⾃、⾙]
  \begin{phonetics}{自责}{zi4ze2}
    \definition{v.}{culpar-se}
  \end{phonetics}
\end{entry}

\begin{entry}{自觉}{6,9}[Radicais ⾃、⾒]
  \begin{phonetics}{自觉}{zi4jue2}[][HSK 3]
    \definition{adj.}{autoconsciente; de ​​livre e espontânea vontade; tomar a iniciativa de fazer as coisas}
    \definition{v.}{estar ciente de}
  \end{phonetics}
\end{entry}

\begin{entry}{自救}{6,11}[Radicais ⾃、⽁]
  \begin{phonetics}{自救}{zi4jiu4}
    \definition{v.}{sair a si mesmo de problemas}
  \end{phonetics}
\end{entry}

\begin{entry}{自然}{6,12}[Radicais ⾃、⽕]
  \begin{phonetics}{自然}{zi4ran2}[][HSK 3]
    \definition{adj.}{natural; no curso normal dos eventos; formado ou desenvolvido sem intervenção humana; algo que se desenvolve livremente}
    \definition{adv.}{naturalmente; certamente; definitivamente}
    \definition{conj.}{usado para ligar duas cláusulas ou frases, com a segunda introduzindo informações adicionais ou adversativas; indica explicação adicional ou uma mudança de significado}
    \definition{s.}{natureza; mundo natural; tudo o que não é criado pelos humanos}
  \end{phonetics}
\end{entry}

\begin{entry}{自燃}{6,16}[Radicais ⾃、⽕]
  \begin{phonetics}{自燃}{zi4ran2}
    \definition{s.}{combustão espontânea}
  \end{phonetics}
\end{entry}

\begin{entry}{至于}{6,3}[Radicais ⾄、⼆]
  \begin{phonetics}{至于}{zhi4yu2}
    \definition{conj.}{para | quanto a | a respeiro de}
  \end{phonetics}
\end{entry}

\begin{entry}{至今}{6,4}[Radicais ⾄、⼈]
  \begin{phonetics}{至今}{zhi4jin1}[][HSK 3]
    \definition{adv.}{até agora; até hoje; até este dia}
  \end{phonetics}
\end{entry}

\begin{entry}{至少}{6,4}[Radicais ⾄、⼩]
  \begin{phonetics}{至少}{zhi4shao3}[][HSK 3]
    \definition{adv.}{pelo menos}
  \end{phonetics}
\end{entry}

\begin{entry}{舌头}{6,5}[Radicais ⾆、⼤]
  \begin{phonetics}{舌头}{she2tou5}
    \definition[个]{s.}{língua | soldado inimigo capturado com o propósito de extrair informações}
  \end{phonetics}
\end{entry}

\begin{entry}{色}{6}[Radical ⾊]
  \begin{phonetics}{色}{se4}[][HSK 4]
    \definition[种]{s.}{cor | aparência; semblante; expressão | tipo; gênero; descrição | cena; cenário;  paisagem | qualidade (de metais preciosos, mercadorias, etc.) | aparência feminina; beleza feminina}
  \end{phonetics}
  \begin{phonetics}{色}{shai3}
    \definition[4]{s.}{cor}
  \end{phonetics}
\end{entry}

\begin{entry}{色狼}{6,10}[Radicais ⾊、⽝]
  \begin{phonetics}{色狼}{se4lang2}
    \definition*{s.}{Sátiro}
    \definition{adj.}{depravado | tarado}
  \end{phonetics}
\end{entry}

\begin{entry}{色彩}{6,11}[Radicais ⾊、⼺]
  \begin{phonetics}{色彩}{se4cai3}[][HSK 4]
    \definition[种,丝]{s.}{cor; matiz; tonalidade | cor; sabor; característica; metáfora para um determinado estado de espírito ou tendência de pensamento}
  \end{phonetics}
\end{entry}

\begin{entry}{芋头}{6,5}[Radicais ⾋、⼤]
  \begin{phonetics}{芋头}{yu4tou5}
    \definition{s.}{taro, similar ao inhame e batata doce}
  \end{phonetics}
\end{entry}

\begin{entry}{芋头色}{6,5,6}[Radicais ⾋、⼤、⾊]
  \begin{phonetics}{芋头色}{yu4tou5se4}
    \definition{s.}{cor lilás}
  \end{phonetics}
\end{entry}

\begin{entry}{芝麻}{6,11}[Radicais ⾋、⿇]
  \begin{phonetics}{芝麻}{zhi1ma5}
    \definition{s.}{semente de gergelim}
  \end{phonetics}
\end{entry}

\begin{entry}{虫子}{6,3}[Radicais ⾍、⼦]
  \begin{phonetics}{虫子}{chong2 zi5}[][HSK 4]
    \definition[条,只,种]{s.}{percevejo; besouro; inseto; verme; criaturas semelhantes a insetos}
  \end{phonetics}
\end{entry}

\begin{entry}{血}{6}[Kangxi 143][Radical ⾎]
  \begin{phonetics}{血}{xie3}
  \end{phonetics}
  \begin{phonetics}{血}{xue4}[][HSK 3]
    \definition{adj.}{relacionado por sangue; parente consanguíneo}
    \definition[片]{s.}{sangue}
  \end{phonetics}
\end{entry}

\begin{entry}{血汗}{6,6}[Radicais ⾎、⽔]
  \begin{phonetics}{血汗}{xue4han4}
    \definition{s.}{(fig.) suor e labuta, trabalho duro}
  \end{phonetics}
\end{entry}

\begin{entry}{行}{6}[Kangxi 144][Radical ⾏]
  \begin{phonetics}{行}{hang2}[][HSK 3]
    \definition{clas.}{para coisas usadas para viajar}
    \definition{s.}{linha; fileira | comércio; profissão; ramo de atividade | empresa de negócios}
  \end{phonetics}
  \begin{phonetics}{行}{xing2}[][HSK 1]
    \definition{adj.}{capaz | competente}
    \definition{expr.}{claro que sim | de acordo | está bem}
    \definition{interj.}{OK!}
    \definition{v.}{caminhar | ir | viajar | atuar}
  \end{phonetics}
\end{entry}

\begin{entry}{行人}{6,2}[Radicais ⾏、⼈]
  \begin{phonetics}{行人}{xing2ren2}[][HSK 2]
    \definition{s.}{transeunte | pedestre | viajante à pé}
  \end{phonetics}
\end{entry}

\begin{entry}{行为}{6,4}[Radicais ⾏、⼂]
  \begin{phonetics}{行为}{xing2wei2}[][HSK 2]
    \definition[个]{s.}{ação | comportamento | conduta}
  \end{phonetics}
\end{entry}

\begin{entry}{行凶}{6,4}[Radicais ⾏、⼐]
  \begin{phonetics}{行凶}{xing2xiong1}
    \definition{v.+compl.}{cometer agressão física ou assassinato | fazer algo violento}
  \end{phonetics}
\end{entry}

\begin{entry}{行业}{6,5}[Radicais ⾏、⼀]
  \begin{phonetics}{行业}{hang2ye4}[][HSK 4]
    \definition[种,个]{s.}{comércio; indústria; setor; profissão; categorias em negócios e indústria referem-se a ocupações em geral}
  \end{phonetics}
\end{entry}

\begin{entry}{行礼}{6,5}[Radicais ⾏、⽰]
  \begin{phonetics}{行礼}{xing2li3}
    \definition{v.}{saudar | fazer saudação}
  \end{phonetics}
\end{entry}

\begin{entry}{行动}{6,6}[Radicais ⾏、⼒]
  \begin{phonetics}{行动}{xing2dong4}[][HSK 2]
    \definition[个]{s.}{ação | operação}
    \definition{v.}{mover}
  \end{phonetics}
\end{entry}

\begin{entry}{行李}{6,7}[Radicais ⾏、⽊]
  \begin{phonetics}{行李}{xing2li5}[][HSK 3]
    \definition[个,件]{s.}{bagagem | pacotes, caixas, cestas, etc. que você carrega quando sai}
  \end{phonetics}
\end{entry}

\begin{entry}{行进}{6,7}[Radicais ⾏、⾡]
  \begin{phonetics}{行进}{xing2jin4}
    \definition{s.}{avançar | movimentar-se para frente}
  \end{phonetics}
\end{entry}

\begin{entry}{行驶}{6,8}[Radicais ⾏、⾺]
  \begin{phonetics}{行驶}{xing2shi3}
    \definition{v.}{viajar ao longo de uma rota (veículos, etc.)}
  \end{phonetics}
\end{entry}

\begin{entry}{行星}{6,9}[Radicais ⾏、⽇]
  \begin{phonetics}{行星}{xing2xing1}
    \definition[颗]{s.}{planeta}
  \seealsoref{惑星}{huo4xing1}
  \end{phonetics}
\end{entry}

\begin{entry}{衣}{6}[Kangxi 145][Radical ⾐]
  \begin{phonetics}{衣}{yi1}
    \definition[件]{s.}{roupa}
  \end{phonetics}
  \begin{phonetics}{衣}{yi4}
    \definition{v.}{vestir | vestir-se}
  \end{phonetics}
\end{entry}

\begin{entry}{衣甲}{6,5}[Radicais ⾐、⽥]
  \begin{phonetics}{衣甲}{yi1jia3}
    \definition{s.}{armadura}
  \end{phonetics}
\end{entry}

\begin{entry}{衣服}{6,8}[Radicais ⾐、⽉]
  \begin{phonetics}{衣服}{yi1fu5}[][HSK 1]
    \definition[件,套]{s.}{roupa | vestuário}
  \end{phonetics}
\end{entry}

\begin{entry}{衣柜}{6,8}[Radicais ⾐、⽊]
  \begin{phonetics}{衣柜}{yi1gui4}
    \definition[个]{s.}{armário | guarda-roupa}
  \end{phonetics}
\end{entry}

\begin{entry}{衣架}{6,9}[Radicais ⾐、⽊]
  \begin{phonetics}{衣架}{yi1 jia4}[][HSK 3]
    \definition[个]{s.}{cabideiro; móvel para pendurar roupas | estatura; figura; refere-se ao formato do corpo de uma pessoa; estrutura corporal}
  \end{phonetics}
\end{entry}

\begin{entry}{西}{6}[Radical ⾑]
  \begin{phonetics}{西}{xi1}[][HSK 1]
    \definition{s.}{oeste}
  \end{phonetics}
\end{entry}

\begin{entry}{西文}{6,4}[Radicais ⾑、⽂]
  \begin{phonetics}{西文}{xi1wen2}
    \definition{s.}{espanhol | língua espanhola}
  \seealsoref{西班牙文}{xi1ban1ya2wen2}
  \end{phonetics}
\end{entry}

\begin{entry}{西方}{6,4}[Radicais ⾑、⽅]
  \begin{phonetics}{西方}{xi1 fang1}[][HSK 2]
    \definition{s.}{países ocidentais | o Ocidente | o Oeste}
  \end{phonetics}
\end{entry}

\begin{entry}{西兰花}{6,5,7}[Radicais ⾑、⼋、⾋]
  \begin{phonetics}{西兰花}{xi1lan2hua1}
    \definition{s.}{brócolis}
  \end{phonetics}
\end{entry}

\begin{entry}{西北}{6,5}[Radicais ⾑、⼔]
  \begin{phonetics}{西北}{xi1 bei3}[][HSK 2]
    \definition{s.}{noroeste | noroeste da China}
  \end{phonetics}
\end{entry}

\begin{entry}{西半球}{6,5,11}[Radicais ⾑、⼗、⽟]
  \begin{phonetics}{西半球}{xi1ban4qiu2}
    \definition{s.}{hemisfério oeste}
  \end{phonetics}
\end{entry}

\begin{entry}{西瓜}{6,5}[Radicais ⾑、⽠]
  \begin{phonetics}{西瓜}{xi1gua1}[][HSK 4]
    \definition[个,颗,粒]{s.}{melancia; fruto que é uma baga de formato grande, globular ou oval, com muita polpa aguada e doce}
  \end{phonetics}
\end{entry}

\begin{entry}{西边}{6,5}[Radicais ⾑、⾡]
  \begin{phonetics}{西边}{xi1bian1}[][HSK 1]
    \definition{adv.}{ao oeste de | oeste | lado oeste | parte ocidental}
  \end{phonetics}
\end{entry}

\begin{entry}{西安}{6,6}[Radicais ⾑、⼧]
  \begin{phonetics}{西安}{xi1'an1}
    \definition*{s.}{Xi'an}
  \end{phonetics}
\end{entry}

\begin{entry}{西西}{6,6}[Radicais ⾑、⾑]
  \begin{phonetics}{西西}{xi1xi1}
    \definition{num.}{centímetro cúbico}
  \end{phonetics}
\end{entry}

\begin{entry}{西医}{6,7}[Radicais ⾑、⼖]
  \begin{phonetics}{西医}{xi1 yi1}[][HSK 2]
    \definition{s.}{medicina ocidental | um médico treinado em medicina ocidental}
  \end{phonetics}
\end{entry}

\begin{entry}{西南}{6,9}[Radicais ⾑、⼗]
  \begin{phonetics}{西南}{xi1 nan2}[][HSK 2]
    \definition{s.}{sudoeste | sudoeste da China}
  \end{phonetics}
\end{entry}

\begin{entry}{西语}{6,9}[Radicais ⾑、⾔]
  \begin{phonetics}{西语}{xi1yu3}
    \definition{s.}{espanhol | língua espanhola}
  \seealsoref{西班牙语}{xi1ban1ya2yu3}
  \end{phonetics}
\end{entry}

\begin{entry}{西面}{6,9}[Radicais ⾑、⾯]
  \begin{phonetics}{西面}{xi1mian4}
    \definition{s.}{oeste | lado oeste}
  \end{phonetics}
\end{entry}

\begin{entry}{西班牙文}{6,10,4,4}[Radicais ⾑、⽟、⽛、⽂]
  \begin{phonetics}{西班牙文}{xi1ban1ya2wen2}
    \definition{s.}{espanhol, língua espanhola}
  \seealsoref{西文}{xi1wen2}
  \end{phonetics}
\end{entry}

\begin{entry}{西班牙语}{6,10,4,9}[Radicais ⾑、⽟、⽛、⾔]
  \begin{phonetics}{西班牙语}{xi1ban1ya2yu3}
    \definition{s.}{espanhol | língua espanhola}
  \seealsoref{西语}{xi1yu3}
  \end{phonetics}
\end{entry}

\begin{entry}{西部}{6,10}[Radicais ⾑、⾢]
  \begin{phonetics}{西部}{xi1 bu4}[][HSK 3]
    \definition{s.}{parte ocidental}
  \end{phonetics}
\end{entry}

\begin{entry}{西蓝花}{6,13,7}[Radicais ⾑、⾋、⾋]
  \begin{phonetics}{西蓝花}{xi1lan2hua1}
    \variantof{西兰花}
  \end{phonetics}
\end{entry}

\begin{entry}{西餐}{6,16}[Radicais ⾑、⾷]
  \begin{phonetics}{西餐}{xi1 can1}[][HSK 2]
    \definition[分,顿]{s.}{comida ocidental}
  \end{phonetics}
\end{entry}

\begin{entry}{观众}{6,6}[Radicais ⾒、⼈]
  \begin{phonetics}{观众}{guan1zhong4}[][HSK 3]
    \definition[位,名,批,个]{s.}{espectador; audiência}
  \end{phonetics}
\end{entry}

\begin{entry}{观念}{6,8}[Radicais ⾒、⼼]
  \begin{phonetics}{观念}{guan1nian4}[][HSK 3]
    \definition[个]{s.}{ideia; conceito}
  \end{phonetics}
\end{entry}

\begin{entry}{观点}{6,9}[Radicais ⾒、⽕]
  \begin{phonetics}{观点}{guan1dian3}[][HSK 2]
    \definition{s.}{ponto de vista | perspectiva}
  \end{phonetics}
\end{entry}

\begin{entry}{观看}{6,9}[Radicais ⾒、⽬]
  \begin{phonetics}{观看}{guan1 kan4}[][HSK 3]
    \definition{v.}{assistir; ver}
  \end{phonetics}
\end{entry}

\begin{entry}{观察}{6,14}[Radicais ⾒、⼧]
  \begin{phonetics}{观察}{guan1cha2}[][HSK 3]
    \definition{v.}{assistir; pesquisar; observar}
  \end{phonetics}
\end{entry}

\begin{entry}{讲}{6}[Radical ⾔]
  \begin{phonetics}{讲}{jiang3}[][HSK 2]
    \definition{v.}{falar (de) | falar (sobre) | relacionar | dizer | contar | explicar | explicitar | elaborar (sobre) | tornar claro |interpretar | discutir | consultar | negociar}
  \end{phonetics}
\end{entry}

\begin{entry}{讲究}{6,7}[Radicais ⾔、⽳]
  \begin{phonetics}{讲究}{jiang3jiu5}[][HSK 4]
    \definition{adj.}{requintado; elegante; de bom gosto; exigente com a vida e com outros aspectos, buscando alto nível, qualidade e detalhes}
    \definition{s.}{estudo cuidadoso; algo que merece atenção; elementos e aspectos que merecem atenção especial}
    \definition{v.}{dar ênfase a; ser específico sobre; prestar atenção a}
  \end{phonetics}
\end{entry}

\begin{entry}{讲话}{6,8}[Radicais ⾔、⾔]
  \begin{phonetics}{讲话}{jiang3 hua4}[][HSK 2]
    \definition{s.}{discurso | guia | introdução}
    \definition{v.+compl.}{falar | conversar | abordar}
  \end{phonetics}
\end{entry}

\begin{entry}{讲述}{6,8}[Radicais ⾔、⾡]
  \begin{phonetics}{讲述}{jiang3shu4}
    \definition{v.}{falar sobre | narrar | descrever}
  \end{phonetics}
\end{entry}

\begin{entry}{讲座}{6,10}[Radicais ⾔、⼴]
  \begin{phonetics}{讲座}{jiang3zuo4}[][HSK 4]
    \definition[个]{s.}{palestra; um curso de palestras; a forma de instrução usada para ensinar um determinado assunto ou tópico, geralmente por meio de palestras ao vivo, seriados de rádio ou televisão ou seriados de jornal.}
  \end{phonetics}
\end{entry}

\begin{entry}{许}{6}[Radical ⾔]
  \begin{phonetics}{许}{xu3}
    \definition*{s.}{sobrenome Xu}
    \definition{adv.}{um pouco | talvez}
    \definition{v.}{permitir | prometer | elogiar}
  \end{phonetics}
\end{entry}

\begin{entry}{许多}{6,6}[Radicais ⾔、⼣]
  \begin{phonetics}{许多}{xu3duo1}[][HSK 2]
    \definition{num.}{muitos | muito | numerosos | uma grande quantidade de}
  \end{phonetics}
\end{entry}

\begin{entry}{论文}{6,4}[Radicais ⾔、⽂]
  \begin{phonetics}{论文}{lun4wen2}[][HSK 4]
    \definition[篇]{s.}{tese; redação; artigo; artigo que discute ou examina uma questão}
  \end{phonetics}
\end{entry}

\begin{entry}{设计}{6,4}[Radicais ⾔、⾔]
  \begin{phonetics}{设计}{she4ji4}[][HSK 3]
    \definition[份]{s.}{plano; esquema}
    \definition{v.}{planejar; projetar | inventar}
  \end{phonetics}
\end{entry}

\begin{entry}{设立}{6,5}[Radicais ⾔、⽴]
  \begin{phonetics}{设立}{she4li4}[][HSK 3]
    \definition{v.}{encontrar; estabelecer; configurar}
  \end{phonetics}
\end{entry}

\begin{entry}{设备}{6,8}[Radicais ⾔、⼡]
  \begin{phonetics}{设备}{she4bei4}[][HSK 3]
    \definition[个]{s.}{facilidade; equipamento; instalação}
  \end{phonetics}
\end{entry}

\begin{entry}{设施}{6,9}[Radicais ⾔、⽅]
  \begin{phonetics}{设施}{she4shi1}[][HSK 4]
    \definition{s.}{facilidade; instalação; instituições, sistemas, organizações, edifícios, etc., estabelecidos para realizar um trabalho ou atender a uma necessidade}
  \end{phonetics}
\end{entry}

\begin{entry}{设置}{6,13}[Radicais ⾔、⽹]
  \begin{phonetics}{设置}{she4zhi4}[][HSK 4]
    \definition{v.}{estabelecer; colocar em prática; estabelecer ou criar instituições, empregos, profissões ou códigos, etc. | encaixar; ajustar; instalar; configurar; colocar}
  \end{phonetics}
\end{entry}

\begin{entry}{访问}{6,6}[Radicais ⾔、⾨]
  \begin{phonetics}{访问}{fang3wen4}[][HSK 3]
    \definition{v.}{visitar; ligar; entrevistar | visitar um \emph{site}}
  \end{phonetics}
\end{entry}

\begin{entry}{负担}{6,8}[Radicais ⾙、⼿]
  \begin{phonetics}{负担}{fu4dan1}[][HSK 4]
    \definition{s.}{carga; fardo; frete; ônus; pressão ou responsabilidade, despesas, etc.}
    \definition{v.}{carregar; carregar (um fardo); assumir (responsabilidade, trabalho, despesas, etc.)}
  \end{phonetics}
\end{entry}

\begin{entry}{负责}{6,8}[Radicais ⾙、⾙]
  \begin{phonetics}{负责}{fu4ze2}[][HSK 3]
    \definition{adj.}{consciencioso}
    \definition{v.}{ser responsável por; estar encarregado de}
  \end{phonetics}
\end{entry}

\begin{entry}{赱}{6}[Radical ⼟]
  \begin{phonetics}{赱}{zou3}
    \variantof{走}
  \end{phonetics}
\end{entry}

\begin{entry}{达到}{6,8}[Radicais ⾡、⼑]
  \begin{phonetics}{达到}{da2dao4}[][HSK 3]
    \definition{v.}{alcançar; atingir; atender o padrão}
  \end{phonetics}
\end{entry}

\begin{entry}{过}{6}[Radical ⾡]
  \begin{phonetics}{过}{guo1}
    \definition*{s.}{sobrenome Guo}
  \end{phonetics}
  \begin{phonetics}{过}{guo4}[][HSK 1,2]
    \definition{part.}{passado}
    \definition{v.}{atravessar | passar (tempo)}
  \end{phonetics}
  \begin{phonetics}{过}{guo5}
    \definition{part.}{(marcador de ação experiente)}
  \end{phonetics}
\end{entry}

\begin{entry}{过不惯}{6,4,11}[Radicais ⾡、⼀、⼼]
  \begin{phonetics}{过不惯}{guo4 bu5 guan4}
    \definition{v.}{não se acostumar | não se habituar}
    \seeref{过惯}{guo4guan4}
  \end{phonetics}
\end{entry}

\begin{entry}{过分}{6,4}[Radicais ⾡、⼑]
  \begin{phonetics}{过分}{guo4fen4}[][HSK 4]
    \definition{adj.}{excessivo; muito longe; demais; falar ou agir além dos limites ou graus adequados}
    \definition{adv.}{excessivamente; indevidamente; muito mesmo}
  \end{phonetics}
\end{entry}

\begin{entry}{过去}{6,5}[Radicais ⾡、⼛]
  \begin{phonetics}{过去}{guo4 qu4}[][HSK 2,3]
    \definition{v.}{atravessar, passar por (a partir da minha localização) | falecer}
  \end{phonetics}
\end{entry}

\begin{entry}{过节}{6,5}[Radicais ⾡、⾋]
  \begin{phonetics}{过节}{guo4jie2}
    \definition{v.+compl.}{celebrar festividades | comemorar um festival}
  \end{phonetics}
\end{entry}

\begin{entry}{过关}{6,6}[Radicais ⾡、⼋]
  \begin{phonetics}{过关}{guo4guan1}
    \definition{v.+compl.}{passar uma barreira | passar por uma provação | passar em um teste | atingir um padrão | passar pela alfândega}
  \end{phonetics}
\end{entry}

\begin{entry}{过年}{6,6}[Radicais ⾡、⼲]
  \begin{phonetics}{过年}{guo4 nian2}[][HSK 2]
    \definition{v.+compl.}{comemorar o Ano Novo | comemorar o Festival da Primavera | passar o Ano Novo | passar o Festival da Primavera}
  \end{phonetics}
\end{entry}

\begin{entry}{过来}{6,7}[Radicais ⾡、⽊]
  \begin{phonetics}{过来}{guo4 lai2}[][HSK 2]
    \definition{v.}{atravessar (para a minha localização) | vir até aqui}
  \end{phonetics}
\end{entry}

\begin{entry}{过惯}{6,11}[Radicais ⾡、⼼]
  \begin{phonetics}{过惯}{guo4guan4}
    \definition{v.}{estar acostumado (a um certo estilo de vida, etc.)}
    \seeref{过不惯}{guo4 bu5 guan4}
  \end{phonetics}
\end{entry}

\begin{entry}{过期}{6,12}[Radicais ⾡、⽉]
  \begin{phonetics}{过期}{guo4qi1}
    \definition{v.+compl.}{exceder a data | passar a data | expirar (passar a data de expiração)}
  \end{phonetics}
\end{entry}

\begin{entry}{过程}{6,12}[Radicais ⾡、⽲]
  \begin{phonetics}{过程}{guo4cheng2}[][HSK 3]
    \definition[个]{s.}{curso dos eventos; processo}
  \end{phonetics}
\end{entry}

\begin{entry}{过瘾}{6,16}[Radicais ⾡、⽧]
  \begin{phonetics}{过瘾}{guo4yin3}
    \definition{adj.}{gratificante | imensamente agradável | satisfatório}
    \definition{v.+compl.}{satisfazer um desejo | se divertir com algo}
  \end{phonetics}
\end{entry}

\begin{entry}{那}{6}[Radical ⾢]
  \begin{phonetics}{那}{na1}
    \definition*{s.}{sobrenome Na}
  \end{phonetics}
  \begin{phonetics}{那}{na3}
    \variantof{哪}
  \end{phonetics}
  \begin{phonetics}{那}{na4}[][HSK 1,2]
    \definition{conj.}{nessa situação | nesse caso}
    \definition{pron.}{aquele | aquilo}
  \end{phonetics}
  \begin{phonetics}{那}{nuo2}
    \definition*{s.}{sobrenome Nuo}
  \end{phonetics}
\end{entry}

\begin{entry}{那儿}{6,2}[Radicais ⾢、⼉]
  \begin{phonetics}{那儿}{na4r5}[][HSK 1]
    \definition{pron.}{lá | ali}
  \end{phonetics}
\end{entry}

\begin{entry}{那么}{6,3}[Radicais ⾢、⼃]
  \begin{phonetics}{那么}{na4 me5}[][HSK 2]
    \definition{adv.}{então | como aquele | dessa maneira}
  \end{phonetics}
\end{entry}

\begin{entry}{那边}{6,5}[Radicais ⾢、⾡]
  \begin{phonetics}{那边}{na4bian5}[][HSK 1]
    \definition{pron.}{ali | acolá}
  \end{phonetics}
\end{entry}

\begin{entry}{那会儿}{6,6,2}[Radicais ⾢、⼈、⼉]
  \begin{phonetics}{那会儿}{na4 hui4r5}[][HSK 2]
    \definition{pron.}{então | naquela época}
  \end{phonetics}
\end{entry}

\begin{entry}{那时}{6,7}[Radicais ⾢、⽇]
  \begin{phonetics}{那时}{na4 shi2}[][HSK 2]
    \definition{pron.}{então | naquela época | naqueles dias}
  \end{phonetics}
\end{entry}

\begin{entry}{那时候}{6,7,10}[Radicais ⾢、⽇、⼈]
  \begin{phonetics}{那时候}{na4 shi2 hou5}[][HSK 2]
    \definition{adv.}{naquela hora}
  \end{phonetics}
\end{entry}

\begin{entry}{那里}{6,7}[Radicais ⾢、⾥]
  \begin{phonetics}{那里}{na4 li3}[][HSK 1]
    \definition{pron.}{lá | ali}
  \end{phonetics}
\end{entry}

\begin{entry}{那些}{6,8}[Radicais ⾢、⼆]
  \begin{phonetics}{那些}{na4xie1}[][HSK 1]
    \definition{pron.}{aqueles}
  \end{phonetics}
\end{entry}

\begin{entry}{那样}{6,10}[Radicais ⾢、⽊]
  \begin{phonetics}{那样}{na4 yang4}[][HSK 2]
    \definition{pron.}{assim | tal | como esse | desse tipo}
  \end{phonetics}
\end{entry}

\begin{entry}{那麽}{6,14}[Radicais ⾢、⿇]
  \begin{phonetics}{那麽}{na4 me5}
    \variantof{那么}
  \end{phonetics}
\end{entry}

\begin{entry}{闭嘴}{6,16}[Radicais ⾨、⼝]
  \begin{phonetics}{闭嘴}{bi4zui3}
    \definition{expr.}{Cale-se!}
  \end{phonetics}
\end{entry}

\begin{entry}{问}{6}[Radical ⾨]
  \begin{phonetics}{问}{wen4}[][HSK 1]
    \definition{v.}{perguntar}
  \end{phonetics}
\end{entry}

\begin{entry}{问市}{6,5}[Radicais ⾨、⼱]
  \begin{phonetics}{问市}{wen4shi4}
    \definition{v.}{chegar ao mercado | bater o mercado | atingir o mercado}
  \end{phonetics}
\end{entry}

\begin{entry}{问安}{6,6}[Radicais ⾨、⼧]
  \begin{phonetics}{问安}{wen4'an1}
    \definition{s.}{saudações}
    \definition{v.}{dar cumprimentos a | prestar homenagem}
  \end{phonetics}
\end{entry}

\begin{entry}{问卷}{6,8}[Radicais ⾨、⼙]
  \begin{phonetics}{问卷}{wen4juan4}
    \definition[份]{s.}{questionário}
  \end{phonetics}
\end{entry}

\begin{entry}{问候}{6,10}[Radicais ⾨、⼈]
  \begin{phonetics}{问候}{wen4hou4}[][HSK 4]
    \definition{s.}{homenagem | saudação}
    \definition{v.}{prestar homenagem; enviar uma saudação;  dar os respeitos (cumprimentos) a alguém | (fig.) (coloquial) fazer referência ofensiva a (alguém querido pela pessoa com quem se está falando)}
  \end{phonetics}
\end{entry}

\begin{entry}{问鼎}{6,12}[Radicais ⾨、⿍]
  \begin{phonetics}{问鼎}{wen4ding3}
    \definition{v.}{visar (o primeiro lugar, etc.) | aspirar ao trono}
  \end{phonetics}
\end{entry}

\begin{entry}{问路}{6,13}[Radicais ⾨、⾜]
  \begin{phonetics}{问路}{wen4 lu4}[][HSK 2]
    \definition{v.}{perguntar sobre o caminho | pedir por direções}
  \end{phonetics}
\end{entry}

\begin{entry}{问题}{6,15}[Radicais ⾨、⾴]
  \begin{phonetics}{问题}{wen4ti2}[][HSK 2]
    \definition[个]{s.}{pergunta | questão | problema}
  \end{phonetics}
\end{entry}

\begin{entry}{防}{6}[Radical ⾩]
  \begin{phonetics}{防}{fang2}[][HSK 3]
    \definition*{s.}{sobrenome Fang}
    \definition{s.}{defesa | barragem; dique; aterro}
    \definition{v.}{prover contra; defender contra; proteger contra}
  \end{phonetics}
\end{entry}

\begin{entry}{防止}{6,4}[Radicais ⾩、⽌]
  \begin{phonetics}{防止}{fang2zhi3}[][HSK 3]
    \definition{v.}{evitar; prevenir; prevenir; proteger contra}
  \end{phonetics}
\end{entry}

\begin{entry}{防护}{6,7}[Radicais ⾩、⼿]
  \begin{phonetics}{防护}{fang2hu4}
    \definition{v.}{defender | proteger}
  \end{phonetics}
\end{entry}

\begin{entry}{防晒}{6,10}[Radicais ⾩、⽇]
  \begin{phonetics}{防晒}{fang2shai4}
    \definition{s.}{protetor solar}
  \end{phonetics}
\end{entry}

\begin{entry}{阳}{6}[Radical ⾩]
  \begin{phonetics}{阳}{yang2}
    \definition*{s.}{Yang (o princípio positivo de Yin e Yang)}
    \definition{s.}{positivo (eletricidade) | sol}
  \seealsoref{阴}{yin1}
  \seealsoref{阴阳}{yin1yang2}
  \end{phonetics}
\end{entry}

\begin{entry}{阳台}{6,5}[Radicais ⾩、⼝]
  \begin{phonetics}{阳台}{yang2tai2}
    \definition{s.}{varanda | sacada}
  \end{phonetics}
\end{entry}

\begin{entry}{阳光}{6,6}[Radicais ⾩、⼉]
  \begin{phonetics}{阳光}{yang2guang1}[][HSK 3]
    \definition{adj.}{alegre; otimista; personalidade positiva e alegre; cheia de vitalidade juvenil | aberto; transparente; público; conduzido sob supervisão pública}
    \definition[缕,束,道]{s.}{luz do sol; brilho do sol; raio de sol}
  \end{phonetics}
\end{entry}

\begin{entry}{阴}{6}[Radical ⾩]
  \begin{phonetics}{阴}{yin1}[][HSK 2]
    \definition*{s.}{Yin (o princípio negativo de Yin e Yang) | sobrenome Yin}
    \definition{adj.}{nublado | sombrio | escondido | implícito}
    \definition{s.}{negativo (eletricidade) | lua}
  \seealsoref{阳}{yang2}
  \seealsoref{阴阳}{yin1yang2}
  \end{phonetics}
\end{entry}

\begin{entry}{阴天}{6,4}[Radicais ⾩、⼤]
  \begin{phonetics}{阴天}{yin1 tian1}[][HSK 2]
    \definition{adj.}{céu nublado | céu cinzento}
  \end{phonetics}
\end{entry}

\begin{entry}{阴阳}{6,6}[Radicais ⾩、⾩]
  \begin{phonetics}{阴阳}{yin1yang2}
    \definition*{s.}{Yin e Yang}
  \seealsoref{阳}{yang2}
  \seealsoref{阴}{yin1}
  \end{phonetics}
\end{entry}

\begin{entry}{阵地}{6,6}[Radicais ⾩、⼟]
  \begin{phonetics}{阵地}{zhen4di4}
    \definition{s.}{posição (militar) | frente de batalha | \emph{front}}
  \end{phonetics}
\end{entry}

\begin{entry}{阶段}{6,9}[Radicais ⾩、⽎]
  \begin{phonetics}{阶段}{jie1duan4}[][HSK 4]
    \definition{s.}{estágio; fase; período; bancada; gradação}
  \end{phonetics}
\end{entry}

\begin{entry}{页}{6}[Radical ⾴]
  \begin{phonetics}{页}{ye4}[][HSK 1]
    \definition{clas.}{páginas}
    \definition{s.}{página | folha}
  \end{phonetics}
\end{entry}

\begin{entry}{齐}{6}[Radical ⿑]
  \begin{phonetics}{齐}{qi2}[][HSK 3]
    \definition*{s.}{sobrenome Qi | Qi, um estado da Dinastia Zhou | Dinastia Qi do Sul (479-502), uma das Dinastias do Sul | Dinastia Qi do Norte (550-577), uma das Dinastias do Norte}
    \definition{adj.}{arrumado; uniforme; regular | semelhante; similar | tudo pronto; todos presentes}
    \definition{adv.}{juntos; simultaneamente}
    \definition{prep.}{ao longo de; junto a; paralelo a}
    \definition{v.}{atingir a altura de; em um nível com; estar nivelado com; no mesmo plano com}
  \end{phonetics}
\end{entry}

%%%%% EOF %%%%%

