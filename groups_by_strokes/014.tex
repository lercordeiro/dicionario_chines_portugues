%%%
%%% 14画
%%%

\section*{14画}\addcontentsline{toc}{section}{14画}

\begin{Entry}{㮸}{14}{⽊}
  \begin{Phonetics}{㮸}{song4}
    \variantof{送}
  \end{Phonetics}
\end{Entry}

\begin{Entry}{僧}{14}{⼈}
  \begin{Phonetics}{僧}{seng1}
    \definition*{s.}{Sobrenome Seng}
    \definition[位,名,个]{s.}{monge Budista, abreviação de 僧伽}
  \seealsoref{僧伽}{seng1qie2}
  \end{Phonetics}
\end{Entry}

\begin{Entry}{僧伽}{14,7}{⼈、⼈}
  \begin{Phonetics}{僧伽}{seng1qie2}
    \definition{s.}{sangha ou sanga (Budismo) | a comunidade monástica | monge}
  \end{Phonetics}
\end{Entry}

\begin{Entry}{僮}{14}{⼈}
  \begin{Phonetics}{僮}{tong2}
    \definition*{s.}{Sobrenome Tong}
  \end{Phonetics}
  \begin{Phonetics}{僮}{zhuang4}
    \variantof{壮}
  \end{Phonetics}
\end{Entry}

\begin{Entry}{嘉}{14}{⼝}
  \begin{Phonetics}{嘉}{jia1}
    \definition*{s.}{Sobrenome Jia}
    \definition{adj.}{bom; ótimo | auspicioso | excelente}
    \definition{v.}{elogiar; recomendar}
    \definition{v.}{elogiar}
  \end{Phonetics}
\end{Entry}

\begin{Entry}{嘉年华}{14,6,6}{⼝、⼲、⼗}
  \begin{Phonetics}{嘉年华}{jia1nian2hua2}
    \definition{s.}{(empréstimo linguístico) carnaval}
  \end{Phonetics}
\end{Entry}

\begin{Entry}{嘉宾}{14,10}{⼝、⼧}
  \begin{Phonetics}{嘉宾}{jia1bin1}[][HSK 6]
    \definition[个,位,名,些]{s.}{convidado}
  \end{Phonetics}
\end{Entry}

\begin{Entry}{嘛}{14}{⼝}
  \begin{Phonetics}{嘛}{ma5}[][HSK 6]
    \definition{part.}{usado no final de uma declaração para expressar que é claro que é verdade que é óbvio | usado no final de uma frase imperativa para expressar expectativa ou dissuasão | usado em uma frase para indicar uma pausa e chamar a atenção da outra pessoa}
  \end{Phonetics}
\end{Entry}

\begin{Entry}{墙}{14}{⼟}
  \begin{Phonetics}{墙}{qiang2}[][HSK 2]
    \definition[面,堵,道]{s.}{parede; barreira ou perímetro construído com tijolos, pedras, etc. | qualquer coisa com a forma ou função de uma parede; a parte de um objeto que funciona como parede ou divisória}
    \definition{v.}{(gíria) bloquear (um website) (usado geralmente na voz passiva: 被墙)}
  \end{Phonetics}
\end{Entry}

\begin{Entry}{墙纸}{14,7}{⼟、⽷}
  \begin{Phonetics}{墙纸}{qiang2zhi3}
    \definition{s.}{papel de parede}
  \end{Phonetics}
\end{Entry}

\begin{Entry}{墙壁}{14,16}{⼟、⼟}
  \begin{Phonetics}{墙壁}{qiang2 bi4}[][HSK 5]
    \definition[面,堵,道]{s.}{parede; barreira ou perímetro construído com tijolos, pedras ou terra}
  \end{Phonetics}
\end{Entry}

\begin{Entry}{墬}{14}{⼟}
  \begin{Phonetics}{墬}{di4}
    \variantof{地}
  \end{Phonetics}
\end{Entry}

\begin{Entry}{寡}{14}{⼧}
  \begin{Phonetics}{寡}{gua3}
    \definition{adj.}{poucos; escassos (oposto a 众, 多)  | insípido; sem sabor | pouco; escasso | insípido; sem graça}
    \definition{pron.}{eu; título autoproclamado de um antigo monarca}
    \definition{s.}{viúva | viuvez; a natureza ou estado de uma mulher viúva que vive sozinha}
  \seealsoref{多}{duo1}
  \seealsoref{众}{zhong4}
  \end{Phonetics}
\end{Entry}

\begin{Entry}{寨}{14}{⼧}
  \begin{Phonetics}{寨}{zhai4}
    \definition{s.}{fortaleza | paliçada | acampamento | vila (paliçada)}
  \end{Phonetics}
\end{Entry}

\begin{Entry}{愿}{14}{⽕}
  \begin{Phonetics}{愿}{yuan4}[][HSK 5]
    \definition{adj.}{honesto e prudente}
    \definition{s.}{esperança; desejo; vontade; a ideia de alcançar algum objetivo no futuro | voto (feito perante o Buda ou um deus); o desejo de retribuição feito ao rezar para os deuses e Buda}
    \definition{v.}{estar disposto; estar pronto; de bom grado, concordar porque está de acordo com seus desejos | ter esperança; desejar; qerer alcançar algum desejo}
  \end{Phonetics}
\end{Entry}

\begin{Entry}{愿望}{14,11}{⽕、⽉}
  \begin{Phonetics}{愿望}{yuan4wang4}[][HSK 3]
    \definition[个,种]{s.}{desejo; aspiração; a ideia de alcançar algum objetivo no futuro.}
  \end{Phonetics}
\end{Entry}

\begin{Entry}{愿意}{14,13}{⽕、⼼}
  \begin{Phonetics}{愿意}{yuan4yi4}[][HSK 2]
    \definition{v.}{estar disposto; estar pronto | desejar; ter esperança}
  \end{Phonetics}
\end{Entry}

\begin{Entry}{慢}{14}{⼼}
  \begin{Phonetics}{慢}{man4}[][HSK 1]
    \definition*{s.}{Sobrenome Man}
    \definition{adj.}{lento; devagar; baixa velocidade; longa duração (em oposição a 快) | rude; arrogante; sem educação com as pessoas | frouxo; lento}
    \definition{adv.}{lentamente}
  \seealsoref{快}{kuai4}
  \end{Phonetics}
\end{Entry}

\begin{Entry}{慢车}{14,4}{⼼、⾞}
  \begin{Phonetics}{慢车}{man4 che1}[][HSK 6]
    \definition{s.}{trem lento com muitas paradas (oposto a 快车) | ônibus ou trem local; parada do trem}
  \seealsoref{快车}{kuai4 che1}
  \end{Phonetics}
\end{Entry}

\begin{Entry}{慢动作}{14,6,7}{⼼、⼒、⼈}
  \begin{Phonetics}{慢动作}{man4dong4zuo4}
    \definition{s.}{(cinema) câmera lenta}
  \end{Phonetics}
\end{Entry}

\begin{Entry}{慢慢}{14,14}{⼼、⼼}
  \begin{Phonetics}{慢慢}{man4 man4}[][HSK 3]
    \definition{adv.}{lentamente; vagarosamente; gradualmente | lentamente; vagarosamente; gradualmente; depois de um longo período de tempo}
  \end{Phonetics}
\end{Entry}

\begin{Entry}{截}{14}{⼽}
  \begin{Phonetics}{截}{jie2}
    \definition{clas.}{seção; pedaço; comprimento}
    \definition{prep.}{por (um tempo especificado); até}
    \definition{v.}{cortar; romper | parar; verificar; interromper; interceptar}
  \end{Phonetics}
\end{Entry}

\begin{Entry}{截止}{14,4}{⼽、⽌}
  \begin{Phonetics}{截止}{jie2zhi3}[][HSK 6]
    \definition{adv.}{até (um certo limite de tempo); por (um tempo especificado)}
  \end{Phonetics}
\end{Entry}

\begin{Entry}{截至}{14,6}{⼽、⾄}
  \begin{Phonetics}{截至}{jie2zhi4}[][HSK 6]
    \definition{adv.}{a partir de; até (um certo limite de tempo); por (um tempo especificado)}
  \end{Phonetics}
\end{Entry}

\begin{Entry}{摔}{14}{⼿}
  \begin{Phonetics}{摔}{shuai1}[][HSK 5]
    \definition{v.}{cair; tropeçar; perder o equilíbrio | mergulhar; precipitar-se; cair de uma altura elevada | quebrar; fazer cair e quebrar | lançar; atirar; arremessar; joguar coisas com força e para baixo | bater; golpear; bater com força para que o que está grudado cair}
  \end{Phonetics}
\end{Entry}

\begin{Entry}{摔倒}{14,10}{⼿、⼈}
  \begin{Phonetics}{摔倒}{shuai1dao3}[][HSK 5]
    \definition{v.}{cair; tropeçar; perder o equilíbrio e cair}
  \end{Phonetics}
\end{Entry}

\begin{Entry}{摘}{14}{⼿}
  \begin{Phonetics}{摘}{zhai1}[][HSK 5]
    \definition{v.}{pegar; arrancar; tirar; colher (flores, frutos, folhas de plantas); retirar (coisas que estão sendo usadas ou penduradas) | selecionar; fazer extrações de | pedir dinheiro emprestado em caso de necessidade urgente | vencer; ganhar; alcançar; obter}
  \end{Phonetics}
\end{Entry}

\begin{Entry}{敲}{14}{⽁}
  \begin{Phonetics}{敲}{qiao1}[][HSK 5]
    \definition{v.}{bater; dar uma pancada; golpear | explorar alguém; cobrar a mais; extorquir; chantagear | lembrar; criticar; alertar; advertir}
  \end{Phonetics}
\end{Entry}

\begin{Entry}{敲门}{14,3}{⽁、⾨}
  \begin{Phonetics}{敲门}{qiao1 men2}[][HSK 5]
    \definition{v.}{bater na porta}
  \end{Phonetics}
\end{Entry}

\begin{Entry}{斡}{14}{⽃}
  \begin{Phonetics}{斡}{wo4}
    \definition{v.}{virar-se}
  \end{Phonetics}
\end{Entry}

\begin{Entry}{斡旋}{14,11}{⽃、⽅}
  \begin{Phonetics}{斡旋}{wo4xuan2}
    \definition{v.}{mediar (um conflito, etc.)}
  \end{Phonetics}
\end{Entry}

\begin{Entry}{旗}{14}{⽅}
  \begin{Phonetics}{旗}{qi2}
    \definition[面]{s.}{bandeira}
  \end{Phonetics}
\end{Entry}

\begin{Entry}{槃}{14}{⽊}
  \begin{Phonetics}{槃}{pan2}
    \variantof{盘}
  \end{Phonetics}
\end{Entry}

\begin{Entry}{模}{14}{⽊}
  \begin{Phonetics}{模}{mo2}
    \definition{s.}{padrão | modelo; exemplo | modelo (pessoa) | exame simulado | módulo}
    \definition{v.}{imitar | copiar; emular}
  \end{Phonetics}
  \begin{Phonetics}{模}{mu2}
    \definition*{s.}{Sobrenome Mu}
    \definition{s.}{molde; padrão; matriz}
  \end{Phonetics}
\end{Entry}

\begin{Entry}{模仿}{14,6}{⽊、⼈}
  \begin{Phonetics}{模仿}{mo2fang3}[][HSK 5]
    \definition{v.}{copiar; imitar; aprender a fazer algo seguindo um modelo pronto}
  \end{Phonetics}
\end{Entry}

\begin{Entry}{模式}{14,6}{⽊、⼷}
  \begin{Phonetics}{模式}{mo2shi4}[][HSK 5]
    \definition{s.}{modelo; modo; padrão; a forma padrão de algo ou o modelo padrão que as pessoas podem seguir}
  \end{Phonetics}
\end{Entry}

\begin{Entry}{模具}{14,8}{⽊、⼋}
  \begin{Phonetics}{模具}{mu2ju4}
    \definition{s.}{molde | matriz | padrão}
  \end{Phonetics}
\end{Entry}

\begin{Entry}{模型}{14,9}{⽊、⼟}
  \begin{Phonetics}{模型}{mo2xing2}[][HSK 4]
    \definition[个]{s.}{modelo; padrão; itens feitos em escala com base em objetos ou desenhos | molde; padrão; molde para fundir máquinas, objetos, etc.}
  \end{Phonetics}
\end{Entry}

\begin{Entry}{模范}{14,9}{⽊、⾋}
  \begin{Phonetics}{模范}{mo2fan4}[][HSK 5]
    \definition{adj.}{exemplar}
    \definition{s.}{modelo; exemplo excelente; pessoa exemplar; coisa exemplar; pessoas ou coisas exemplares que servem de modelo}
  \end{Phonetics}
\end{Entry}

\begin{Entry}{模样}{14,10}{⽊、⽊}
  \begin{Phonetics}{模样}{mu2yang4}[][HSK 5]
    \definition[副,种]{s.}{aparência; a aparência ou o estilo de vestir de uma pessoa | indicando uma estimativa aproximada de tempo ou idade; expressão de estimativas relativas a tempo, idade, etc. | tendência; situação; inclinação}
  \end{Phonetics}
\end{Entry}

\begin{Entry}{模特儿}{14,10,2}{⽊、⽜、⼉}
  \begin{Phonetics}{模特儿}{mo2 te4r5}[][HSK 4]
    \definition[个,名,位]{s.}{modelo (pessoa que posa para um fotógrafo ou pintor ou escultor); objeto de representação ou referência usado por artistas para esboços e esculturas, como o corpo humano, objetos, modelos etc.; também se refere aos arquétipos que os estudiosos da literatura usam para retratar seus personagens | modelo (uma pessoa que usa roupas para exibir modas); pessoa ou manequim usado para exibir estilos de roupas}
  \end{Phonetics}
\end{Entry}

\begin{Entry}{模糊}{14,15}{⽊、⽶}
  \begin{Phonetics}{模糊}{mo2hu5}[][HSK 5]
    \definition{adj.}{vago; confuso; indistinto}
    \definition{v.}{confundir; desorientar}
  \end{Phonetics}
\end{Entry}

\begin{Entry}{歌}{14}{⽋}
  \begin{Phonetics}{歌}{ge1}[][HSK 1]
    \definition[首,支,段]{s.}{canção; poesia cantável}
    \definition{v.}{cantar; entoar | louvar; exaltar; cantar louvores a}
  \end{Phonetics}
\end{Entry}

\begin{Entry}{歌手}{14,4}{⽋、⼿}
  \begin{Phonetics}{歌手}{ge1 shou3}[][HSK 3]
    \definition[个,位,名]{s.}{cantor; vocalista; pessoa com talento para cantar}
  \end{Phonetics}
\end{Entry}

\begin{Entry}{歌曲}{14,6}{⽋、⽈}
  \begin{Phonetics}{歌曲}{ge1 qu3}[][HSK 5]
    \definition[首,支]{s.}{música; obra para as pessoas cantarem, uma combinação de poesia e música}
  \end{Phonetics}
\end{Entry}

\begin{Entry}{歌声}{14,7}{⽋、⼠}
  \begin{Phonetics}{歌声}{ge1 sheng1}[][HSK 3]
    \definition{s.}{canto; voz cantada; som do canto}
  \end{Phonetics}
\end{Entry}

\begin{Entry}{歌词}{14,7}{⽋、⾔}
  \begin{Phonetics}{歌词}{ge1 ci2}[][HSK 6]
    \definition{s.}{letra da música; libreto}
  \end{Phonetics}
\end{Entry}

\begin{Entry}{歌星}{14,9}{⽋、⽇}
  \begin{Phonetics}{歌星}{ge1 xing1}[][HSK 6]
    \definition[位,名]{s.}{cantor famoso; estrela da música}
  \end{Phonetics}
\end{Entry}

\begin{Entry}{歌迷}{14,9}{⽋、⾡}
  \begin{Phonetics}{歌迷}{ge1 mi2}
    \definition{s.}{fã de um cantor; pessoas que gostam de ouvir música ou cantar e ficam fascinadas por isso}
  \end{Phonetics}
\end{Entry}

\begin{Entry}{歌唱}{14,11}{⽋、⼝}
  \begin{Phonetics}{歌唱}{ge1 chang4}[][HSK 6]
    \definition{v.}{cantar | cantar em louvor de; louvor através de cânticos, recitações, etc.}
  \end{Phonetics}
\end{Entry}

\begin{Entry}{滴}{14}{⽔}
  \begin{Phonetics}{滴}{di1}[][HSK 6]
    \definition{clas.}{gota; quantificador para "gotejamento"}
    \definition{s.}{uma gota}
    \definition{v.}{pingar}
  \end{Phonetics}
\end{Entry}

\begin{Entry}{漂}{14}{⽔}
  \begin{Phonetics}{漂}{piao1}
    \definition{v.}{flutuar | estar a deriva}
  \end{Phonetics}
  \begin{Phonetics}{漂}{piao3}
    \definition{v.}{alvejar | branquear}
  \end{Phonetics}
  \begin{Phonetics}{漂}{piao4}
    \definition{adj.}{bonita; usado em 漂亮}
    \definition{v.}{falhar; terminar em fracasso}[这笔投资的钱全都漂了。===Todo o dinheiro desse investimento foi perdido.]
  \seealsoref{漂亮}{piao4liang5}
  \end{Phonetics}
\end{Entry}

\begin{Entry}{漂亮}{14,9}{⽔、⼇}
  \begin{Phonetics}{漂亮}{piao4liang5}[][HSK 2]
    \definition{adj.}{bonito; lindo; atraente; de boa aparência; esteticamente agradável | excelente; notável | não pode ser utilizado para descrever homens}
  \end{Phonetics}
\end{Entry}

\begin{Entry}{漂流}{14,10}{⽔、⽔}
  \begin{Phonetics}{漂流}{piao1liu2}
    \definition{s.}{\emph{rafting}}
    \definition{v.}{ser levado pela correnteza | flutuar ao longo ou sobre}
  \end{Phonetics}
\end{Entry}

\begin{Entry}{漏}{14}{⽔}
  \begin{Phonetics}{漏}{lou4}[][HSK 5]
    \definition{s.}{relógio de água; ampulheta | falha; ponto fraco | gonorreia; a medicina tradicional chinesa refere-se a certas doenças que causam secreção de pus, sangue e muco | unidade de tempo medida por um relógio de água durante a noite}
    \definition{v.}{(líquido, gás, etc.) pingar; vazar; escorrer; cair (de um buraco ou fenda) | vazar; deixar escapar; divulgar | perder; deixar de fora por engano | vazar; o objeto tem poros e pode vazar coisas | há uma fuga de ar}
  \end{Phonetics}
\end{Entry}

\begin{Entry}{漏电}{14,5}{⽔、⽥}
  \begin{Phonetics}{漏电}{lou4dian4}
    \definition{v.}{vazar eletricidade}
  \end{Phonetics}
\end{Entry}

\begin{Entry}{漏洞}{14,9}{⽔、⽔}
  \begin{Phonetics}{漏洞}{lou4 dong4}[][HSK 5]
    \definition[个,点]{s.}{vazamento; rachadura; lacunas ou buracos desnecessários que permitem que coisas vazem | falha; defeito; lacuna; (fala, ação, método, etc.) imperfeições}
  \end{Phonetics}
\end{Entry}

\begin{Entry}{演}{14}{⽔}
  \begin{Phonetics}{演}{yan3}[][HSK 3]
    \definition{v.}{desenvolver; evoluir | deduzir; elaborar | exercitar; praticar | representar; atuar; encenar | desempenhar}
  \end{Phonetics}
\end{Entry}

\begin{Entry}{演出}{14,5}{⽔、⼐}
  \begin{Phonetics}{演出}{yan3chu1}[][HSK 3]
    \definition[场,次]{s.}{show; concerto; performance}
    \definition{v.}{apresentar; representar; fazer um show; apresentar peças teatrais, danças, artes cênicas, acrobacias, etc. para o público apreciar}
  \end{Phonetics}
\end{Entry}

\begin{Entry}{演讲}{14,6}{⽔、⾔}
  \begin{Phonetics}{演讲}{yan3jiang3}[][HSK 4]
    \definition[场,次]{s.}{palestra; discurso; ato ou a atividade de apresentar ou expressar ideias, opiniões ou informações oralmente em público ou diante de um público}
    \definition{v.}{dar uma palestra; fazer um discurso; informar o público sobre uma determinada área de conhecimento ou opinião sobre um determinado assunto}
  \end{Phonetics}
\end{Entry}

\begin{Entry}{演员}{14,7}{⽔、⼝}
  \begin{Phonetics}{演员}{yan3yuan2}[][HSK 3]
    \definition[个,位,名]{s.}{ator; artista; pessoas que participam de apresentações teatrais, cinematográficas, de dança, de artes cênicas, de acrobacias, etc.}
  \end{Phonetics}
\end{Entry}

\begin{Entry}{演奏}{14,9}{⽔、⼤}
  \begin{Phonetics}{演奏}{yan3zou4}[][HSK 6]
    \definition{v.}{tocar um instrumento musical; fazer uma apresentação instrumental}
  \end{Phonetics}
\end{Entry}

\begin{Entry}{演唱}{14,11}{⽔、⼝}
  \begin{Phonetics}{演唱}{yan3 chang4}[][HSK 3]
    \definition{v.}{cantar em uma performance; apresentar canções, óperas, peças teatrais, etc.}
  \end{Phonetics}
\end{Entry}

\begin{Entry}{演唱会}{14,11,6}{⽔、⼝、⼈}
  \begin{Phonetics}{演唱会}{yan3 chang4 hui4}[][HSK 3]
    \definition[个,场]{s.}{recital vocal; concerto vocal; uma forma de apresentação centrada no canto, acompanhada por movimentos de dança simples}
  \end{Phonetics}
\end{Entry}

\begin{Entry}{漫}{14}{⽔}
  \begin{Phonetics}{漫}{man4}
    \definition{adj.}{livre; irrestrito; casual | longo; extenso | em todos os lugares; por toda parte | aleatório; irrestrito; livre}
    \definition{adv.}{não}
    \definition{v.}{transbordar; inundar | estar em todo lugar}
  \end{Phonetics}
\end{Entry}

\begin{Entry}{漫长}{14,4}{⽔、⾧}
  \begin{Phonetics}{漫长}{man4chang2}[][HSK 5]
    \definition{adj.}{muito longo; interminável; (tempo, espaço) dura muito tempo}
  \end{Phonetics}
\end{Entry}

\begin{Entry}{漫画}{14,8}{⽔、⽥}
  \begin{Phonetics}{漫画}{man4hua4}[][HSK 5]
    \definition[幅,本,张,套]{s.}{desenho animado; caricatura; \emph{cartoon}}
  \end{Phonetics}
\end{Entry}

\begin{Entry}{漫骂}{14,9}{⽔、⾺}
  \begin{Phonetics}{漫骂}{man4ma4}
    \variantof{谩骂}
  \end{Phonetics}
\end{Entry}

\begin{Entry}{熊}{14}{⽕}
  \begin{Phonetics}{熊}{xiong2}[][HSK 5]
    \definition*{s.}{Sobrenome Xiong}
    \definition[头,只]{s.}{urso}
    \definition{v.}{repreender; censurar}
  \end{Phonetics}
\end{Entry}

\begin{Entry}{熊猫}{14,11}{⽕、⽝}
  \begin{Phonetics}{熊猫}{xiong2mao1}
    \definition[把,只]{s.}{panda gigante}
  \seealsoref{猫熊}{mao1xiong2}
  \end{Phonetics}
\end{Entry}

\begin{Entry}{熏}{14}{⽕}
  \begin{Phonetics}{熏}{xun1}
    \definition{v.}{expor à fumaça ou vapores; fumigar | tratar (carne, peixe, etc.) com fumaça; defumar | tornar perfumado com incenso, etc. | sufocar (asfixia e envenenamento por gás)}
  \end{Phonetics}
\end{Entry}

\begin{Entry}{熏香}{14,9}{⽕、⾹}
  \begin{Phonetics}{熏香}{xun1xiang1}
    \definition{s.}{incenso}
  \end{Phonetics}
\end{Entry}

\begin{Entry}{疑}{14}{⽦}
  \begin{Phonetics}{疑}{yi2}
    \definition{adj.}{duvidoso; incerto}
    \definition{v.}{duvidar; desacreditar; suspeitar}
  \end{Phonetics}
\end{Entry}

\begin{Entry}{疑问}{14,6}{⽦、⾨}
  \begin{Phonetics}{疑问}{yi2wen4}[][HSK 4]
    \definition[个,些]{s.}{dúvida; consulta; pergunta; questionamento; coisas que não podem ser determinadas ou explicadas}
  \end{Phonetics}
\end{Entry}

\begin{Entry}{瘦}{14}{⽧}
  \begin{Phonetics}{瘦}{shou4}[][HSK 5]
    \definition{adj.}{magro; esquelético (oposto de 胖, 肥) | magro (oposto de 肥) | apertado (oposto de 肥) | infértil; pobre | esquelético; pouca gordura; pouca carne (em oposição a 或 ou 肥) | (roupas, sapatos, meias, etc.) apertado (em oposição a 肥) |magra; (carne comestível) com baixo teor de gordura (em oposição a 肥)}
    \definition{v.}{perder peso}
  \seealsoref{肥}{fei2}
  \seealsoref{或}{huo4}
  \seealsoref{胖}{pang4}
  \end{Phonetics}
\end{Entry}

\begin{Entry}{碳}{14}{⽯}
  \begin{Phonetics}{碳}{tan4}
    \definition{s.}{carbono (elemento químico)}
  \end{Phonetics}
\end{Entry}

\begin{Entry}{磁}{14}{⽯}
  \begin{Phonetics}{磁}{ci2}
    \definition[块]{s.}{porcelana | (física) magnetismo; propriedade de atrair ferro, níquel, etc. | (dialeto)  (de relação) próximo; íntimo}
  \end{Phonetics}
\end{Entry}

\begin{Entry}{磁带}{14,9}{⽯、⼱}
  \begin{Phonetics}{磁带}{ci2dai4}
    \definition[盘,盒]{s.}{cassete | fita magnética}
  \end{Phonetics}
\end{Entry}

\begin{Entry}{磁铁}{14,10}{⽯、⾦}
  \begin{Phonetics}{磁铁}{ci2tie3}
    \definition{s.}{imã | magneto}
  \seealsoref{吸铁石}{xi1tie3shi2}
  \end{Phonetics}
\end{Entry}

\begin{Entry}{磁盘}{14,11}{⽯、⽫}
  \begin{Phonetics}{磁盘}{ci2pan2}
    \definition{s.}{disquete}
  \end{Phonetics}
\end{Entry}

\begin{Entry}{稳}{14}{⽲}
  \begin{Phonetics}{稳}{wen3}[][HSK 4]
    \definition{adj.}{constante; estável; firme | estável; estático; sedado | seguro; confiável; certo}
    \definition{adv.}{certamente; com certeza; seguramente; sem dúvida}
    \definition{v.}{estabilizar, manter estável; acalmar}
  \end{Phonetics}
\end{Entry}

\begin{Entry}{稳定}{14,8}{⽲、⼧}
  \begin{Phonetics}{稳定}{wen3ding4}[][HSK 4]
    \definition{adj.}{estável; firme; descreve uma natureza, um estado, etc. relativamente fixo; não muda significativamente}
    \definition{v.}{manter estável; estabilizar}
  \end{Phonetics}
\end{Entry}

\begin{Entry}{端}{14}{⽴}
  \begin{Phonetics}{端}{duan1}[][HSK 6]
    \definition*{s.}{Sobrenome Duan}
    \definition{adj.}{adequado; próprio | reto; correto}
    \definition{s.}{fim; extremidade | começo | item; ponto; pista, projeto ou aspecto | causa; razão | problema; incidente; coisas (geralmente se refere a coisas ruins, como acidentes, disputas, etc.)}
    \definition{v.}{carregar; segurar algo nivelado com ambas as mãos; segurar algo horizontalmente | erradicar; eliminar; acabar com; remover completamente; varrer | dar ares de superioridade | revelar}
  \end{Phonetics}
\end{Entry}

\begin{Entry}{端午节}{14,4,5}{⽴、⼗、⾋}
  \begin{Phonetics}{端午节}{duan1wu3jie2}[][HSK 6]
    \definition*[个]{s.}{Festa do Duplo Cinco, Festival dos Barcos-Dragão (5º~dia do quinto mês lunar)}
  \end{Phonetics}
\end{Entry}

\begin{Entry}{算}{14}{⽵}
  \begin{Phonetics}{算}{suan4}[][HSK 2]
    \definition{adv.}{finalmente; por fim; no final; significa que, após um longo período de tempo ou muitas dificuldades, finalmente se alcançou o objetivo, equivalente a 总算}
    \definition{v.}{calcular; estimar; computar | contar; incluir | planejar; calcular; projetar | pensar; supor; especular | considerar; considerar como; contar como; reconhecer como | (aritmética) contar; ter peso | deixe estar; deixe passar; seguido por 了: desistir, não se importar mais}
  \seealsoref{了}{le5}
  \seealsoref{总算}{zong3suan4}
  \end{Phonetics}
\end{Entry}

\begin{Entry}{算了}{14,2}{⽵、⼅}
  \begin{Phonetics}{算了}{suan4 le5}[][HSK 6]
    \definition{part.}{deixe estar; deixe passar; usado no final de uma frase para expressar imperativo, término, etc.}
    \definition{v.}{deixar;  deixe estar; deixe passar; esquecer isso; não querer continuar; é usado para persuadir os outros ou para expressar que posso aceitar a situação atual, para encerrar o assunto ou assunto atual, ou para dizer "esqueça"}
  \end{Phonetics}
\end{Entry}

\begin{Entry}{算命}{14,8}{⽵、⼝}
  \begin{Phonetics}{算命}{suan4ming4}
    \definition{s.}{cartomante}
    \definition{v.}{ler a sorte | fazer advinhações}
  \end{Phonetics}
\end{Entry}

\begin{Entry}{算是}{14,9}{⽵、⽇}
  \begin{Phonetics}{算是}{suan4 shi4}[][HSK 6]
    \definition{adv.}{finalmente; por fim; depois de muito tempo, o objetivo foi finalmente alcançado}
    \definition{v.}{contar como; pensar que; ser considerado}
  \end{Phonetics}
\end{Entry}

\begin{Entry}{管}{14}{⽵}
  \begin{Phonetics}{管}{guan3}[][HSK 3]
    \definition*{s.}{Guan, um estado da dinastia Zhou | Sobrenome Guan}
    \definition{adj.}{estreito; restrito; limitado; pequeno}
    \definition{clas.}{usado para objetos cilíndricos longos e finos}
    \definition{conj.}{não importa (quem, o quê, como, etc.)}
    \definition{prep.}{função semelhante a 把, usada especificamente em conjunto com 叫}
    \definition[根,条,排]{s.}{cano; tubo | instrumento musical de sopro | válvula; tubo | duto; canal; vasos}
    \definition{v.}{administrar; dirigir; controlar; cuidar; ser responsável por | ter jurisdição sobre; administrar | disciplinar (crianças ou alunos) | preocupar-se com; importar-se com; incomodar-se com; intervir | fornecer; garantir | supervisionar | governar | submeter alguém a disciplina | assumir; arcar com | incomodar; interferir | assegurar; garantir}
  \seealsoref{把}{ba3}
  \seealsoref{叫}{jiao4}
  \end{Phonetics}
\end{Entry}

\begin{Entry}{管……叫……}{14,5}{⽵、⼝}
  \begin{Phonetics}{管……叫……}{guan3 jiao4}
    \definition{expr.}{chamar alguém (ou algo) de alguém (ou algo)}
  \end{Phonetics}
\end{Entry}

\begin{Entry}{管家}{14,10}{⽵、⼧}
  \begin{Phonetics}{管家}{guan3jia1}
    \definition{s.}{mordomo | governanta}
    \definition{v.}{administrar uma casa}
  \end{Phonetics}
\end{Entry}

\begin{Entry}{管理}{14,11}{⽵、⽟}
  \begin{Phonetics}{管理}{guan3li3}[][HSK 3]
    \definition{v.}{gerenciar; executar; administrar; governar; estar encarregado de; responsável por garantir o bom andamento de uma determinada tarefa | controlar; gerenciar; fazer com que pessoas e animais obedeçam ou se comportem de maneira ordeira | cuidar; zelar por; proteger; cuidar, organizar coisas}
  \end{Phonetics}
\end{Entry}

\begin{Entry}{管道}{14,12}{⽵、⾡}
  \begin{Phonetics}{管道}{guan3 dao4}[][HSK 6]
    \definition[根,千米,公里]{s.}{oleoduto; canal; túnel; tubulação; um tubo feito de metal ou outro material usado para transportar ou descarregar fluidos (como vapor, gás, óleo, água, etc.) | caminho; canal; abordagem}
  \end{Phonetics}
\end{Entry}

\begin{Entry}{精}{14}{⽶}
  \begin{Phonetics}{精}{jing1}[][HSK 6]
    \definition{adv.}{muito; extremamente; antes de certos adjetivos, significa 十分 ou 非常}
    \definition{s.}{refinado; escolhido; escolha; purificado ou selecionado | perfeito; excelente; melhor | fino (em oposição a 粗); preciso; meticuloso | inteligente; astuto; esperto | habilidoso; versado; proficiente | extrato; essência; essência refinada ou selecionada; extraída | energia; espírito | semente; esperma; sêmen | \emph{goblin}; espírito; elfo; demônio}
  \seealsoref{粗}{cu1}
  \seealsoref{非常}{fei1chang2}
  \seealsoref{十分}{shi2fen1}
  \end{Phonetics}
\end{Entry}

\begin{Entry}{精力}{14,2}{⽶、⼒}
  \begin{Phonetics}{精力}{jing1li4}[][HSK 4]
    \definition[些]{s.}{energia; vigor; força mental e física}
  \end{Phonetics}
\end{Entry}

\begin{Entry}{精灵}{14,7}{⽶、⽕}
  \begin{Phonetics}{精灵}{jing1ling2}
    \definition{s.}{espírito | fada | elfo | duende | gênio}
  \end{Phonetics}
\end{Entry}

\begin{Entry}{精品}{14,9}{⽶、⼝}
  \begin{Phonetics}{精品}{jing1pin3}[][HSK 6]
    \definition[个]{s.}{belas obras (de arte); objetos de arte | produtos de qualidade; artigos de excelente qualidade; produto \emph{premium}}
  \end{Phonetics}
\end{Entry}

\begin{Entry}{精神}{14,9}{⽶、⽰}
  \begin{Phonetics}{精神}{jing1shen2}[][HSK 3]
    \definition[种,个,类,股]{s.}{espírito; mente; estado mental; refere-se à consciência, às atividades mentais e ao estado psicológico geral de uma pessoa | substância; espírito; essência; propósito; significado principal}
  \end{Phonetics}
  \begin{Phonetics}{精神}{jing1shen5}[][HSK 3]
    \definition{adj.}{animado; espirituoso; vigoroso; descreve uma pessoa como cheia de energia | muito bonito; boa aparência, bom físico}
    \definition[种,个,类,股]{s.}{impulso; vigor; vitalidade}
  \end{Phonetics}
\end{Entry}

\begin{Entry}{精美}{14,9}{⽶、⽺}
  \begin{Phonetics}{精美}{jing1 mei3}[][HSK 6]
    \definition{adj.}{elegante; requintado}
  \end{Phonetics}
\end{Entry}

\begin{Entry}{精致}{14,10}{⽶、⾄}
  \begin{Phonetics}{精致}{jing1zhi4}
    \definition{adj.}{delicado | exótico | refinado}
  \end{Phonetics}
\end{Entry}

\begin{Entry}{精彩}{14,11}{⽶、⼺}
  \begin{Phonetics}{精彩}{jing1cai3}[][HSK 3]
    \definition{adj.}{brilhante; esplêndido; maravilhoso}
  \end{Phonetics}
\end{Entry}

\begin{Entry}{缩}{14}{⽷}
  \begin{Phonetics}{缩}{suo1}
    \definition*{s.}{Sobrenome Suo}
    \definition{v.}{contrair; encolher | recuar; retirar-se | economizar}
  \end{Phonetics}
\end{Entry}

\begin{Entry}{缩小}{14,3}{⽷、⼩}
  \begin{Phonetics}{缩小}{suo1 xiao3}[][HSK 4]
    \definition{v.}{reduzir, estreitar, encolher;  tornar menor (em oposição a 放大)}
  \seealsoref{放大}{fang4da4}
  \end{Phonetics}
\end{Entry}

\begin{Entry}{缩短}{14,12}{⽷、⽮}
  \begin{Phonetics}{缩短}{suo1duan3}[][HSK 4]
    \definition{v.}{encurtar; reduzir; diminuir}
  \end{Phonetics}
\end{Entry}

\begin{Entry}{缩影卡片}{14,15,5,4}{⽷、⼺、⼘、⽚}
  \begin{Phonetics}{缩影卡片}{suo1ying3 ka3pian4}
    \definition{s.}{cartão em miniatura}
  \end{Phonetics}
\end{Entry}

\begin{Entry}{聚}{14}{⽿}
  \begin{Phonetics}{聚}{ju4}[][HSK 4]
    \definition*{s.}{Sobrenome Ju}
    \definition{v.}{reunir-se; juntar-se}
  \end{Phonetics}
\end{Entry}

\begin{Entry}{聚会}{14,6}{⽿、⼈}
  \begin{Phonetics}{聚会}{ju4hui4}[][HSK 4]
    \definition[个,次]{s.}{reunião; encontro; confraternização; festa}
    \definition{v.}{encontrar-se; reunir-se}
  \end{Phonetics}
\end{Entry}

\begin{Entry}{聚散}{14,12}{⽿、⽁}
  \begin{Phonetics}{聚散}{ju4san4}
    \definition{s.}{juntos e separados | agregação e dissipação}
  \end{Phonetics}
\end{Entry}

\begin{Entry}{膜}{14}{⾁}
  \begin{Phonetics}{膜}{mo2}[][HSK 6]
    \definition[张]{s.}{membrana | filme; revestimento fino}
  \end{Phonetics}
\end{Entry}

\begin{Entry}{膜拜}{14,9}{⾁、⼿}
  \begin{Phonetics}{膜拜}{mo2bai4}
    \definition{v.}{ajoelhar-se e se curvar com as mãos unidas no nível da testa | ter ou mostrar sentimentos fortes de respeito e admiração por um deus}
  \end{Phonetics}
\end{Entry}

\begin{Entry}{舞}{14}{⾇}
  \begin{Phonetics}{舞}{wu3}[][HSK 5]
    \definition[支,段,个]{s.}{dança | palco; metáfora do domínio das atividades sociais}
    \definition{v.}{mover-se como numa dança | dançar com algo nas mãos; brincar com | florescer; empunhar; brandir | esvoaçar | fazer malabarismos; brincar com}
  \end{Phonetics}
\end{Entry}

\begin{Entry}{舞厅}{14,4}{⾇、⼚}
  \begin{Phonetics}{舞厅}{wu3ting1}
    \definition[间]{s.}{salão de dança | salão de baile}
  \end{Phonetics}
\end{Entry}

\begin{Entry}{舞厅舞}{14,4,14}{⾇、⼚、⾇}
  \begin{Phonetics}{舞厅舞}{wu3ting1wu3}
    \definition{s.}{dança de salão}
  \end{Phonetics}
\end{Entry}

\begin{Entry}{舞台}{14,5}{⾇、⼝}
  \begin{Phonetics}{舞台}{wu3 tai2}[][HSK 3]
    \definition[个]{s.}{palco; plataforma elevada usada exclusivamente para apresentações artísticas, geralmente localizada na parte frontal de teatros e auditórios | palco; metáfora do campo das atividades sociais}
  \end{Phonetics}
\end{Entry}

\begin{Entry}{舞会}{14,6}{⾇、⼈}
  \begin{Phonetics}{舞会}{wu3hui4}
    \definition{s.}{baile}
  \end{Phonetics}
\end{Entry}

\begin{Entry}{舞会舞}{14,6,14}{⾇、⼈、⾇}
  \begin{Phonetics}{舞会舞}{wu3hui4wu3}
    \definition{s.}{baile}
  \end{Phonetics}
\end{Entry}

\begin{Entry}{舞抃}{14,7}{⾇、⼿}
  \begin{Phonetics}{舞抃}{wu3bian4}
    \definition{s.}{dançar por prazer}
  \end{Phonetics}
\end{Entry}

\begin{Entry}{舞蹈}{14,17}{⾇、⾜}
  \begin{Phonetics}{舞蹈}{wu3dao3}[][HSK 6]
    \definition[段,支,场,个]{s.}{dança; uma forma de arte que usa movimentos rítmicos como principal meio de expressão, podendo expressar a vida, os pensamentos e os sentimentos das pessoas, geralmente acompanhada de música}
    \definition{v.}{dançar}
  \end{Phonetics}
\end{Entry}

\begin{Entry}{蔓}{14}{⾋}
  \begin{Phonetics}{蔓}{man2}
    \definition{s.}{couve-chinesa | nabo}
  \end{Phonetics}
  \begin{Phonetics}{蔓}{man4}
    \definition{s.}{uma videira com gavinhas; caule fino que não consegue ficar em pé}
    \definition{v.}{rastejar; espalhar; estender}
  \end{Phonetics}
  \begin{Phonetics}{蔓}{wan4}
    \definition*{s.}{Sobrenome Wan}
    \definition{s.}{uma videira com gavinhas; caule fino que não consegue ficar em pé}
  \end{Phonetics}
\end{Entry}

\begin{Entry}{蔓草}{14,9}{⾋、⾋}
  \begin{Phonetics}{蔓草}{man4cao3}
    \definition{s.}{videira | trepadeira}
  \end{Phonetics}
\end{Entry}

\begin{Entry}{蜘}{14}{⾍}
  \begin{Phonetics}{蜘}{zhi1}
    \definition[只]{s.}{aranha}
  \seealsoref{蜘蛛}{zhi1zhu1}
  \end{Phonetics}
\end{Entry}

\begin{Entry}{蜘蛛}{14,12}{⾍、⾍}
  \begin{Phonetics}{蜘蛛}{zhi1zhu1}
    \definition{s.}{aranha}
  \end{Phonetics}
\end{Entry}

\begin{Entry}{蜘蛛网}{14,12,6}{⾍、⾍、⽹}
  \begin{Phonetics}{蜘蛛网}{zhi1zhu1 wang3}
    \definition{s.}{teia de aranha}
  \end{Phonetics}
\end{Entry}

\begin{Entry}{蜜}{14}{⾍}
  \begin{Phonetics}{蜜}{mi4}
    \definition{adj.}{melado; doce}
    \definition{s.}{mel | semelhante ao mel | coisas parecidas com mel; melaço}
  \end{Phonetics}
\end{Entry}

\begin{Entry}{蜜桃}{14,10}{⾍、⽊}
  \begin{Phonetics}{蜜桃}{mi4tao2}
    \definition{s.}{pêssego suculento}
  \end{Phonetics}
\end{Entry}

\begin{Entry}{蜡}{14}{⾍}
  \begin{Phonetics}{蜡}{la4}
    \definition{s.}{cera; óleos produzidos por animais, minerais ou plantas | vela}
  \end{Phonetics}
\end{Entry}

\begin{Entry}{蜡烛}{14,10}{⾍、⽕}
  \begin{Phonetics}{蜡烛}{la4zhu2}
    \definition[根,支]{s.}{vela | círio | peça, geralmente de cera, que possui um pavio e se utiliza para iluminar}
  \end{Phonetics}
\end{Entry}

\begin{Entry}{蜥}{14}{⾍}
  \begin{Phonetics}{蜥}{xi1}
    \definition{s.}{lagarto}
  \end{Phonetics}
\end{Entry}

\begin{Entry}{蜥易}{14,8}{⾍、⽇}
  \begin{Phonetics}{蜥易}{xi1yi4}
    \variantof{蜥蜴}
  \end{Phonetics}
\end{Entry}

\begin{Entry}{蜥蜴}{14,14}{⾍、⾍}
  \begin{Phonetics}{蜥蜴}{xi1yi4}
    \definition{s.}{lagarto}
  \end{Phonetics}
\end{Entry}

\begin{Entry}{蜻}{14}{⾍}
  \begin{Phonetics}{蜻}{qing1}
    \definition[只]{s.}{libélula, 蜻蜓}
  \seealsoref{蜻蜓}{qing1ting2}
  \end{Phonetics}
\end{Entry}

\begin{Entry}{蜻蜓}{14,12}{⾍、⾍}
  \begin{Phonetics}{蜻蜓}{qing1ting2}
    \definition{s.}{libélula}
  \end{Phonetics}
\end{Entry}

\begin{Entry}{蜻蝏}{14,15}{⾍、⾍}
  \begin{Phonetics}{蜻蝏}{qing1ting2}
    \variantof{蜻蜓}
  \end{Phonetics}
\end{Entry}

\begin{Entry}{蝉}{14}{⾍}
  \begin{Phonetics}{蝉}{chan2}
    \definition[只,个]{s.}{cigarra}
  \seealsoref{知了}{zhi1liao3}
  \end{Phonetics}
\end{Entry}

\begin{Entry}{褐}{14}{⾐}
  \begin{Phonetics}{褐}{he4}
    \definition{adj.}{marrom; castanho; pardo}
    \definition{s.}{pano de cânhamo grosso}
  \end{Phonetics}
\end{Entry}

\begin{Entry}{褐色}{14,6}{⾐、⾊}
  \begin{Phonetics}{褐色}{he4 se4}
    \definition{s.}{cor marrom}
  \end{Phonetics}
\end{Entry}

\begin{Entry}{褡}{14}{⾐}
  \begin{Phonetics}{褡}{da1}
    \definition{s.}{bolsa; malote; algibeira | jaqueta sem mangas}
  \end{Phonetics}
\end{Entry}

\begin{Entry}{豪}{14}{⾗}
  \begin{Phonetics}{豪}{hao2}
    \definition*{s.}{Sobrenome Hao}
    \definition{adj.}{direto; irrestrito; ousado | despótico; intimidador | rico e poderoso}
    \definition{s.}{pessoa com poderes ou dons extraordinários}
  \end{Phonetics}
\end{Entry}

\begin{Entry}{豪华}{14,6}{⾗、⼗}
  \begin{Phonetics}{豪华}{hao2hua2}
    \definition{adj.}{luxuoso}
  \end{Phonetics}
\end{Entry}

\begin{Entry}{赚}{14}{⾙}
  \begin{Phonetics}{赚}{zhuan4}[][HSK 6]
    \definition{s.}{lucro}
    \definition{v.}{ganhar (dinheiro); obter lucro com o negócio (em oposição a 赔)}
  \seealsoref{赔}{pei2}
  \end{Phonetics}
\end{Entry}

\begin{Entry}{赚钱}{14,10}{⾙、⾦}
  \begin{Phonetics}{赚钱}{zhuan4 qian2}[][HSK 6]
    \definition{v.}{ganhar dinheiro; obter lucro ou recompensa}
  \end{Phonetics}
\end{Entry}

\begin{Entry}{赛}{14}{⾙}
  \begin{Phonetics}{赛}{sai4}[][HSK 6]
    \definition*{s.}{Sobrenome Sai}
    \definition{s.}{jogo; partida; competição | sacrifício; cerimônia de sacrifício; antigamente, sacrifícios eram feitos para agradecer aos deuses por suas dádivas}
    \definition{v.}{ter uma competição (comparando alto e baixo, forte e fraco) | superar; ser comparável a; comparar com}
  \end{Phonetics}
\end{Entry}

\begin{Entry}{赛车}{14,4}{⾙、⾞}
  \begin{Phonetics}{赛车}{sai4che1}
    \definition{s.}{corrida de automóvel | corrida de bicicleta | carro de corrida}
  \end{Phonetics}
\end{Entry}

\begin{Entry}{赛场}{14,6}{⾙、⼟}
  \begin{Phonetics}{赛场}{sai4 chang3}[][HSK 6]
    \definition{s.}{local de competição; arena; ringue; terreno | campo (para competição de atletismo) | pista de corrida}
  \end{Phonetics}
\end{Entry}

\begin{Entry}{辗}{14}{⾞}
  \begin{Phonetics}{辗}{zhan3}
    \definition{v.}{(arcaico) virar | (arcaico) rolar para o lado | (arcaico) virar a metade}
  \end{Phonetics}
\end{Entry}

\begin{Entry}{辣}{14}{⾟}
  \begin{Phonetics}{辣}{la4}[][HSK 4]
    \definition{adj.}{apimentado; picante; pungente; quente | cruel; implacável; venenoso; vicioso}
    \definition{v.}{queimar; picar; formigar; ter uma irritação picante (boca, nariz ou olhos)}
  \end{Phonetics}
\end{Entry}

\begin{Entry}{遭}{14}{⾡}
  \begin{Phonetics}{遭}{zao1}
    \definition{clas.}{tempo; vez; ocasião | rodadas}
    \definition{v.}{encontrar-se com (desastre, infortúnio, etc.); sofrer}
  \end{Phonetics}
\end{Entry}

\begin{Entry}{遭到}{14,8}{⾡、⼑}
  \begin{Phonetics}{遭到}{zao1 dao4}[][HSK 6]
    \definition{v.}{sofrer; ser rejeitado; receber crítica; significa sofrer infortúnio ou dano}[我们遭到意外事故。===Nós sofremos um acidente.]
  \end{Phonetics}
\end{Entry}

\begin{Entry}{遭受}{14,8}{⾡、⼜}
  \begin{Phonetics}{遭受}{zao1shou4}[][HSK 6]
    \definition{v.}{sofrer; aguentar; ser submetido a; encontrar ou vivenciar coisas dolorosas que você não quer que aconteçam}
  \end{Phonetics}
\end{Entry}

\begin{Entry}{遭遇}{14,12}{⾡、⾡}
  \begin{Phonetics}{遭遇}{zao1yu4}[][HSK 6]
    \definition[场,次,种,段]{s.}{sorte (difícil); experiência (amarga); encontrando coisas ruins}
    \definition{v.}{encontrar; encontrar-se com; esbarrar em; encontros inesperados com pessoas ou coisas que não são boas para você}
  \end{Phonetics}
\end{Entry}

\begin{Entry}{酷}{14}{⾣}
  \begin{Phonetics}{酷}{ku4}[][HSK 6]
    \definition{adj.}{cruel; opressivo | feroz; escaldante | brutal | \emph{cool} (empréstimo linguístico); legal; excelente; moderno; ótimo | elegante e sóbrio; gracioso e severo}
    \definition{adv.}{muito; extremamente}
  \end{Phonetics}
\end{Entry}

\begin{Entry}{酷斯拉}{14,12,8}{⾣、⽄、⼿}
  \begin{Phonetics}{酷斯拉}{ku4si1la1}
    \definition*{s.}{Godzilla. do Japonês Gojira, ゴジラ}
  \seealsoref{哥斯拉}{ge1si1la1}
  \end{Phonetics}
\end{Entry}

\begin{Entry}{酸}{14}{⾣}
  \begin{Phonetics}{酸}{suan1}[][HSK 4]
    \definition{adj.}{azedo; ácido | aflito; angustiado; doente do coração | pedante; descreve uma pessoa que finge ser culta e também descreve uma pessoa que é muito inflexível com suas próprias ideias e não está disposta a mudá-las para atender às exigências da época, é usado principalmente para satirizar intelectuais que fingem ser capazes de escrever poemas e artigos | ciumento; invejoso; sentimentos desconfortáveis porque outra pessoa é melhor do que você e, em geral, também apresenta comportamento hostil}
    \definition{s.}{ácido; produto químico que tem um sabor ácido quando misturado com água}
    \definition{v.}{estar dolorido (devido à fadiga ou doença); descreve a sensação de não ter força muscular e um pouco de dor por estar doente ou muito cansado}
  \end{Phonetics}
\end{Entry}

\begin{Entry}{酸奶}{14,5}{⾣、⼥}
  \begin{Phonetics}{酸奶}{suan1 nai3}[][HSK 4]
    \definition[瓶,杯,盒,袋]{s.}{iogurte; produto lácteo fermentado por bactérias de ácido láctico}
  \end{Phonetics}
\end{Entry}

\begin{Entry}{酸甜苦辣}{14,11,8,14}{⾣、⽢、⾋、⾟}
  \begin{Phonetics}{酸甜苦辣}{suan1 tian2 ku3 la4}[][HSK 5]
    \definition{expr.}{os altos e baixos da vida; as experiências agridoces da vida; os aspectos doces, azedos, amargos e picantes da vida; refere-se a todos os tipos de sabores, como metáfora para experiências diversas, como felicidade, sofrimento, etc. | azedo, doce, amargo, picante --- alegrias e tristezas da vida}
  \end{Phonetics}
\end{Entry}

\begin{Entry}{酸辣汤}{14,14,6}{⾣、⾟、⽔}
  \begin{Phonetics}{酸辣汤}{suan1la4tang1}
    \definition{s.}{sopa avinagrada e picante (prato)}
  \end{Phonetics}
\end{Entry}

\begin{Entry}{锺}{14}{⾦}
  \begin{Phonetics}{锺}{zhong1}
    \variantof{钟}
  \end{Phonetics}
\end{Entry}

\begin{Entry}{锻}{14}{⾦}
  \begin{Phonetics}{锻}{duan4}
    \definition{v.}{forjar; moldar}
  \end{Phonetics}
\end{Entry}

\begin{Entry}{锻炼}{14,9}{⾦、⽕}
  \begin{Phonetics}{锻炼}{duan4lian4}[][HSK 4]
    \definition{v.}{exercitar-se; fazer (ou fazer) exercícios; submeter-se a treinamento físico; fortalecer o corpo por meio do esporte | fortalecer; endurecer; aprimorar as habilidades de trabalho e de vida por meio de trabalho e outras atividades | forjar ou moldar metal para torná-lo mais refinado; refere-se à transformação de materiais metálicos em objetos de determinada forma e tamanho por meio de aquecimento, batimento, prensagem etc.}
  \end{Phonetics}
\end{Entry}

\begin{Entry}{镀}{14}{⾦}
  \begin{Phonetics}{镀}{du4}
    \definition{v.}{cobrir ou revestir (com um metal)}
  \end{Phonetics}
\end{Entry}

\begin{Entry}{镀金}{14,8}{⾦、⾦}
  \begin{Phonetics}{镀金}{du4jin1}
    \definition{v.}{banhar a ouro | dourar | (figurativo) fazer algo muito comum parecer especial}
  \end{Phonetics}
\end{Entry}

\begin{Entry}{隧}{14}{⾩}
  \begin{Phonetics}{隧}{sui4}
    \definition{s.}{túnel; passagem subterrânea | estrada | subúrbios; áreas suburbanas}
    \definition{v.}{virar}
  \end{Phonetics}
\end{Entry}

\begin{Entry}{隧道}{14,12}{⾩、⾡}
  \begin{Phonetics}{隧道}{sui4dao4}
    \definition{s.}{túnel}
  \end{Phonetics}
\end{Entry}

\begin{Entry}{需}{14}{⾬}
  \begin{Phonetics}{需}{xu1}
    \definition*{s.}{Sobrenome Xu}
    \definition{s.}{necessidades; bens de primeira necessidade}
    \definition{v.}{precisar; querer; exigir}
  \end{Phonetics}
\end{Entry}

\begin{Entry}{需求}{14,7}{⾬、⽔}
  \begin{Phonetics}{需求}{xu1qiu2}[][HSK 3]
    \definition[种]{s.}{necessidades; demanda; exigência; solicitações decorrentes de necessidades}
  \end{Phonetics}
\end{Entry}

\begin{Entry}{需要}{14,9}{⾬、⾑}
  \begin{Phonetics}{需要}{xu1yao4}[][HSK 3]
    \definition[种]{s.}{necessidade; desejo ou exigência em relação a algo}
    \definition{v.}{precisar; querer; exigir; demandar; solicitar}
  \end{Phonetics}
\end{Entry}

\begin{Entry}{静}{14}{⾭}
  \begin{Phonetics}{静}{jing4}[][HSK 3]
    \definition*{s.}{Sobrenome Jing}
    \definition{adj.}{tranquilo;  sossegado; calmo; imóvel | silencioso; quieto; sem emitir nenhum som | calmo, sereno; serenidade; (interior) paz}
    \definition{v.}{acalmar-se; aquietar-se; tranquilizar (o coração)}
  \end{Phonetics}
\end{Entry}

\begin{Entry}{颗}{14}{⾴}
  \begin{Phonetics}{颗}{ke1}[][HSK 5]
    \definition{clas.}{usado para grãos, pérolas, dentes, corações, satelites, pequenas esferas, etc.}
    \definition{s.}{grão; partícula; pequenas coisas redondas}
  \end{Phonetics}
\end{Entry}

\begin{Entry}{馒}{14}{⾷}
  \begin{Phonetics}{馒}{man2}
    \definition{s.}{pão cozido no vapor}
  \end{Phonetics}
\end{Entry}

\begin{Entry}{馒头}{14,5}{⾷、⼤}
  \begin{Phonetics}{馒头}{man2tou5}[][HSK 6]
    \definition[个,锅,屉,筐]{s.}{pão cozido no vapor; um alimento cozido no vapor feito de farinha fermentada, geralmente redondo na parte superior e plano na parte inferior, sem recheio}
  \end{Phonetics}
\end{Entry}

\begin{Entry}{魅}{14}{⿁}
  \begin{Phonetics}{魅}{mei4}
    \definition{s.}{espírito maligno; demônio | \emph{goblin}; trasgo; gnomo; duende maléfico}
    \definition{v.}{atormentar; cativar}
  \end{Phonetics}
\end{Entry}

\begin{Entry}{魅力}{14,2}{⿁、⼒}
  \begin{Phonetics}{魅力}{mei4li4}
    \definition{s.}{charme | fascínio | glamour | carisma}
  \end{Phonetics}
\end{Entry}

\begin{Entry}{鲜}{14}{⿂}
  \begin{Phonetics}{鲜}{xian1}[][HSK 4]
    \definition*{s.}{Sobrenome Xian}
    \definition{adj.}{fresco; novo; fresco (experiência, comida etc.) |brilhante; de cores vivas | saboroso; delicioso | exuberante; luxuriante}
    \definition{s.}{aves e animais recém-abatidos; vegetais recém-colhidos; frutas, etc. | alimentos aquáticos; geralmente, peixes vivos, camarões, etc., para alimentação}
  \end{Phonetics}
  \begin{Phonetics}{鲜}{xian3}
    \definition{adj.}{raro; pouco; pequeno}
    \definition{adv.}{raramente}
  \end{Phonetics}
\end{Entry}

\begin{Entry}{鲜花}{14,7}{⿂、⾋}
  \begin{Phonetics}{鲜花}{xian1 hua1}[][HSK 4]
    \definition[朵,束,支]{s.}{flor; flores frescas; flores bonitas e frescas}
  \end{Phonetics}
\end{Entry}

\begin{Entry}{鲜明}{14,8}{⿂、⽇}
  \begin{Phonetics}{鲜明}{xian1ming2}[][HSK 4]
    \definition{adj.}{brilhante (cor) | distinto; bem definido; nítido; claro; característico}
  \end{Phonetics}
\end{Entry}

\begin{Entry}{鲜艳}{14,10}{⿂、⾊}
  \begin{Phonetics}{鲜艳}{xian1yan4}[][HSK 5]
    \definition{adj.}{de cores alegres; de cores brilhantes}
  \end{Phonetics}
\end{Entry}

\begin{Entry}{鼻}{14}{⿐}[Kangxi 209]
  \begin{Phonetics}{鼻}{bi2}
    \definition{s.}{nariz}
  \end{Phonetics}
\end{Entry}

\begin{Entry}{鼻子}{14,3}{⿐、⼦}
  \begin{Phonetics}{鼻子}{bi2zi5}[][HSK 5]
    \definition[个,只]{s.}{nariz; órgão da face, responsável pela respiração e pelo olfato}
  \end{Phonetics}
\end{Entry}

%%%%% EOF %%%%%

