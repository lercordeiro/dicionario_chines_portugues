%%%
%%% 14画
%%%

\section*{14画}\addcontentsline{toc}{section}{14画}

\begin{entry}{㮸}{14}{⽊}
  \begin{phonetics}{㮸}{song4}
    \variantof{送}
  \end{phonetics}
\end{entry}

\begin{entry}{僧}{14}{⼈}
  \begin{phonetics}{僧}{seng1}
    \definition{s.}{monge Budista, abreviação de 僧伽}
  \seealsoref{僧伽}{seng1qie2}
  \end{phonetics}
\end{entry}

\begin{entry}{僧伽}{14,7}{⼈、⼈}
  \begin{phonetics}{僧伽}{seng1qie2}
    \definition{s.}{sangha ou sanga (Budismo) | a comunidade monástica | monge}
  \end{phonetics}
\end{entry}

\begin{entry}{嘉年华}{14,6,6}{⼝、⼲、⼗}
  \begin{phonetics}{嘉年华}{jia1nian2hua2}
    \definition{s.}{(empréstimo linguístico) carnaval}
  \end{phonetics}
\end{entry}

\begin{entry}{墙}{14}{⼟}
  \begin{phonetics}{墙}{qiang2}[][HSK 2]
    \definition[面,堵,道]{s.}{parede; barreira ou perímetro construído com tijolos, pedras, etc. | qualquer coisa com a forma ou função de uma parede; a parte de um objeto que funciona como parede ou divisória}
    \definition{v.}{(gíria) bloquear (um website) (usado geralmente na voz passiva: 被墙)}
  \end{phonetics}
\end{entry}

\begin{entry}{墙纸}{14,7}{⼟、⽷}
  \begin{phonetics}{墙纸}{qiang2zhi3}
    \definition{s.}{papel de parede}
  \end{phonetics}
\end{entry}

\begin{entry}{墙壁}{14,16}{⼟、⼟}
  \begin{phonetics}{墙壁}{qiang2 bi4}[][HSK 5]
    \definition[堵]{s.}{parede; barreira ou perímetro construído com tijolos, pedras ou terra}
  \end{phonetics}
\end{entry}

\begin{entry}{墬}{14}{⼟}
  \begin{phonetics}{墬}{di4}
    \variantof{地}
  \end{phonetics}
\end{entry}

\begin{entry}{寡}{14}{⼧}
  \begin{phonetics}{寡}{gua3}
    \definition{adj.}{poucos; escassos (oposto a 众, 多)  | insípido; sem sabor | pouco; escasso | insípido; sem graça}
  \seealsoref{多}{duo1}
  \seealsoref{众}{zhong4}
  \end{phonetics}
\end{entry}

\begin{entry}{寨}{14}{⼧}
  \begin{phonetics}{寨}{zhai4}
    \definition{s.}{fortaleza | paliçada | acampamento | vila (paliçada)}
  \end{phonetics}
\end{entry}

\begin{entry}{愿}{14}{⽕}
  \begin{phonetics}{愿}{yuan4}[][HSK 5]
    \definition{adj.}{honesto e prudente}
    \definition{s.}{esperança; desejo; vontade; a ideia de alcançar algum objetivo no futuro | voto (feito perante o Buda ou um deus); o desejo de retribuição feito ao rezar para os deuses e Buda}
    \definition{v.}{estar disposto; estar pronto; de bom grado, concordar porque está de acordo com seus desejos | ter esperança; desejar; qerer alcançar algum desejo}
  \end{phonetics}
\end{entry}

\begin{entry}{愿望}{14,11}{⽕、⽉}
  \begin{phonetics}{愿望}{yuan4wang4}[][HSK 3]
    \definition[个]{s.}{desejo; aspiração; a ideia de esperar atingir um determinado objetivo no futuro}
  \end{phonetics}
\end{entry}

\begin{entry}{愿意}{14,13}{⽕、⼼}
  \begin{phonetics}{愿意}{yuan4yi4}[][HSK 2]
    \definition{v.}{estar disposto; estar pronto | desejar; ter esperança}
  \end{phonetics}
\end{entry}

\begin{entry}{慢}{14}{⼼}
  \begin{phonetics}{慢}{man4}[][HSK 1]
    \definition*{s.}{sobrenome Man}
    \definition{adj.}{lento; devagar; baixa velocidade; longa duração (em oposição a 快) | rude; arrogante; sem educação com as pessoas | frouxo; lento}
    \definition{adv.}{lentamente}
  \seealsoref{快}{kuai4}
  \end{phonetics}
\end{entry}

\begin{entry}{慢动作}{14,6,7}{⼼、⼒、⼈}
  \begin{phonetics}{慢动作}{man4dong4zuo4}
    \definition{s.}{(cinema) câmera lenta}
  \end{phonetics}
\end{entry}

\begin{entry}{慢慢}{14,14}{⼼、⼼}
  \begin{phonetics}{慢慢}{man4 man4}[][HSK 3]
    \definition{adv.}{lentamente; vagarosamente; gradualmente | lentamente; vagarosamente; gradualmente; depois de um longo período de tempo}
  \end{phonetics}
\end{entry}

\begin{entry}{摔}{14}{⼿}
  \begin{phonetics}{摔}{shuai1}[][HSK 5]
    \definition{v.}{cair; tropeçar; perder o equilíbrio | mergulhar; precipitar-se; cair de uma altura elevada | quebrar; fazer cair e quebrar | lançar; atirar; arremessar; joguar coisas com força e para baixo | bater; golpear; bater com força para que o que está grudado cair}
  \end{phonetics}
\end{entry}

\begin{entry}{摔倒}{14,10}{⼿、⼈}
  \begin{phonetics}{摔倒}{shuai1dao3}[][HSK 5]
    \definition{v.}{cair; tropeçar; perder o equilíbrio e cair}
  \end{phonetics}
\end{entry}

\begin{entry}{摘}{14}{⼿}
  \begin{phonetics}{摘}{zhai1}[][HSK 5]
    \definition{v.}{pegar; arrancar; tirar; colher (flores, frutos, folhas de plantas); retirar (coisas que estão sendo usadas ou penduradas) | selecionar; fazer extrações de | pedir dinheiro emprestado em caso de necessidade urgente | vencer; ganhar; alcançar; obter}
  \end{phonetics}
\end{entry}

\begin{entry}{敲}{14}{⽁}
  \begin{phonetics}{敲}{qiao1}[][HSK 5]
    \definition{v.}{bater; dar uma pancada; golpear | explorar alguém; cobrar a mais; extorquir; chantagear | lembrar; criticar; alertar; advertir}
  \end{phonetics}
\end{entry}

\begin{entry}{敲门}{14,3}{⽁、⾨}
  \begin{phonetics}{敲门}{qiao1 men2}[][HSK 5]
    \definition{v.}{bater na porta}
  \end{phonetics}
\end{entry}

\begin{entry}{斡旋}{14,11}{⽃、⽅}
  \begin{phonetics}{斡旋}{wo4xuan2}
    \definition{v.}{mediar (um conflito, etc.)}
  \end{phonetics}
\end{entry}

\begin{entry}{旗}{14}{⽅}
  \begin{phonetics}{旗}{qi2}
    \definition[面]{s.}{bandeira}
  \end{phonetics}
\end{entry}

\begin{entry}{槃}{14}{⽊}
  \begin{phonetics}{槃}{pan2}
    \variantof{盘}
  \end{phonetics}
\end{entry}

\begin{entry}{模仿}{14,6}{⽊、⼈}
  \begin{phonetics}{模仿}{mo2fang3}[][HSK 5]
    \definition{v.}{copiar; imitar; aprender a fazer algo seguindo um modelo pronto}
  \end{phonetics}
\end{entry}

\begin{entry}{模式}{14,6}{⽊、⼷}
  \begin{phonetics}{模式}{mo2shi4}[][HSK 5]
    \definition{s.}{modelo; modo; padrão; a forma padrão de algo ou o modelo padrão que as pessoas podem seguir}
  \end{phonetics}
\end{entry}

\begin{entry}{模具}{14,8}{⽊、⼋}
  \begin{phonetics}{模具}{mu2ju4}
    \definition{s.}{molde | matriz | padrão}
  \end{phonetics}
\end{entry}

\begin{entry}{模型}{14,9}{⽊、⼟}
  \begin{phonetics}{模型}{mo2xing2}[][HSK 4]
    \definition[个]{s.}{modelo; padrão; itens feitos em escala com base em objetos ou desenhos | molde; padrão; molde para fundir máquinas, objetos, etc.}
  \end{phonetics}
\end{entry}

\begin{entry}{模范}{14,9}{⽊、⾋}
  \begin{phonetics}{模范}{mo2fan4}[][HSK 5]
    \definition{adj.}{exemplar}
    \definition{s.}{modelo; exemplo excelente; pessoa exemplar; coisa exemplar; pessoas ou coisas exemplares que servem de modelo}
  \end{phonetics}
\end{entry}

\begin{entry}{模样}{14,10}{⽊、⽊}
  \begin{phonetics}{模样}{mu2yang4}[][HSK 5]
    \definition[副,种]{s.}{aparência; a aparência ou o estilo de vestir de uma pessoa | indicando uma estimativa aproximada de tempo ou idade; expressão de estimativas relativas a tempo, idade, etc. | tendência; situação; inclinação}
  \end{phonetics}
\end{entry}

\begin{entry}{模特儿}{14,10,2}{⽊、⽜、⼉}
  \begin{phonetics}{模特儿}{mo2 te4r5}[][HSK 4]
    \definition[个]{s.}{modelo (pessoa que posa para um fotógrafo ou pintor ou escultor); objeto de representação ou referência usado por artistas para esboços e esculturas, como o corpo humano, objetos, modelos etc.; também se refere aos arquétipos que os estudiosos da literatura usam para retratar seus personagens | modelo (uma pessoa que usa roupas para exibir modas); pessoa ou manequim usado para exibir estilos de roupas}
  \end{phonetics}
\end{entry}

\begin{entry}{模糊}{14,15}{⽊、⽶}
  \begin{phonetics}{模糊}{mo2hu5}[][HSK 5]
    \definition{adj.}{vago; confuso; indistinto}
    \definition{v.}{confundir; desorientar}
  \end{phonetics}
\end{entry}

\begin{entry}{歌}{14}{⽋}
  \begin{phonetics}{歌}{ge1}[][HSK 1]
    \definition[首,支,段]{s.}{canção; poesia cantável}
    \definition{v.}{cantar; entoar | louvar; exaltar; cantar louvores a}
  \end{phonetics}
\end{entry}

\begin{entry}{歌手}{14,4}{⽋、⼿}
  \begin{phonetics}{歌手}{ge1 shou3}[][HSK 3]
    \definition[个,位,名]{s.}{cantor; vocalista; pessoa com talento para cantar}
  \end{phonetics}
\end{entry}

\begin{entry}{歌曲}{14,6}{⽋、⽈}
  \begin{phonetics}{歌曲}{ge1 qu3}[][HSK 5]
    \definition{s.}{música; obra para as pessoas cantarem, uma combinação de poesia e música}
  \end{phonetics}
\end{entry}

\begin{entry}{歌声}{14,7}{⽋、⼠}
  \begin{phonetics}{歌声}{ge1 sheng1}[][HSK 3]
    \definition{s.}{canto; voz cantada; som do canto}
  \end{phonetics}
\end{entry}

\begin{entry}{歌迷}{14,9}{⽋、⾡}
  \begin{phonetics}{歌迷}{ge1 mi2}
    \definition{s.}{fã de um cantor; pessoas que gostam de ouvir música ou cantar e ficam fascinadas por isso}
  \end{phonetics}
\end{entry}

\begin{entry}{滴}{14}{⽔}
  \begin{phonetics}{滴}{di1}
    \definition{s.}{uma gota}
    \definition{v.}{pingar}
  \end{phonetics}
\end{entry}

\begin{entry}{漂}{14}{⽔}
  \begin{phonetics}{漂}{piao1}
    \definition{v.}{flutuar | estar a deriva}
  \end{phonetics}
  \begin{phonetics}{漂}{piao3}
    \definition{v.}{alvejar | branquear}
  \end{phonetics}
  \begin{phonetics}{漂}{piao4}
    \definition{adj.}{usado em 漂亮}
  \seealsoref{漂亮}{piao4liang5}
  \end{phonetics}
\end{entry}

\begin{entry}{漂亮}{14,9}{⽔、⼇}
  \begin{phonetics}{漂亮}{piao4liang5}[][HSK 2]
    \definition{adj.}{bonito; lindo; atraente; de boa aparência; esteticamente agradável | excelente; notável | não pode ser utilizado para descrever homens}
  \end{phonetics}
\end{entry}

\begin{entry}{漂流}{14,10}{⽔、⽔}
  \begin{phonetics}{漂流}{piao1liu2}
    \definition{s.}{\emph{rafting}}
    \definition{v.}{ser levado pela correnteza | flutuar ao longo ou sobre}
  \end{phonetics}
\end{entry}

\begin{entry}{漏}{14}{⽔}
  \begin{phonetics}{漏}{lou4}[][HSK 5]
    \definition{s.}{relógio de água; ampulheta | falha; ponto fraco | gonorreia; a medicina tradicional chinesa refere-se a certas doenças que causam secreção de pus, sangue e muco | unidade de tempo medida por um relógio de água durante a noite}
    \definition{v.}{(líquido, gás, etc.) pingar; vazar; escorrer; cair (de um buraco ou fenda) | vazar; deixar escapar; divulgar | perder; deixar de fora por engano | vazar; o objeto tem poros e pode vazar coisas | há uma fuga de ar}
  \end{phonetics}
\end{entry}

\begin{entry}{漏电}{14,5}{⽔、⽥}
  \begin{phonetics}{漏电}{lou4dian4}
    \definition{v.}{vazar eletricidade}
  \end{phonetics}
\end{entry}

\begin{entry}{漏洞}{14,9}{⽔、⽔}
  \begin{phonetics}{漏洞}{lou4 dong4}[][HSK 5]
    \definition[个]{s.}{vazamento; rachadura; lacunas ou buracos desnecessários que permitem que coisas vazem | falha; defeito; lacuna; (fala, ação, método, etc.) imperfeições}
  \end{phonetics}
\end{entry}

\begin{entry}{演}{14}{⽔}
  \begin{phonetics}{演}{yan3}[][HSK 3]
    \definition{v.}{desenvolver; evoluir | deduzir; elaborar | exercitar; praticar | representar; atuar}
  \end{phonetics}
\end{entry}

\begin{entry}{演出}{14,5}{⽔、⼐}
  \begin{phonetics}{演出}{yan3chu1}[][HSK 3]
    \definition[场,次]{s.}{show; concerto; performance}
    \definition{v.}{apresentar; representar; fazer um show; apresentar peças de teatro, dança, arte popular, acrobacias, etc. para o público aproveitar}
  \end{phonetics}
\end{entry}

\begin{entry}{演讲}{14,6}{⽔、⾔}
  \begin{phonetics}{演讲}{yan3jiang3}[][HSK 4]
    \definition[场,次]{s.}{palestra; discurso; ato ou a atividade de apresentar ou expressar ideias, opiniões ou informações oralmente em público ou diante de um público}
    \definition{v.}{dar uma palestra; fazer um discurso; informar o público sobre uma determinada área de conhecimento ou opinião sobre um determinado assunto}
  \end{phonetics}
\end{entry}

\begin{entry}{演员}{14,7}{⽔、⼝}
  \begin{phonetics}{演员}{yan3yuan2}[][HSK 3]
    \definition[个,位,名]{s.}{ator; performer; participantes de teatro, cinema, dança, arte popular, acrobacia e outras apresentações}
  \end{phonetics}
\end{entry}

\begin{entry}{演唱}{14,11}{⽔、⼝}
  \begin{phonetics}{演唱}{yan3 chang4}[][HSK 3]
    \definition{v.}{cantar em uma performance; apresentar canções, óperas, dramas, etc.}
  \end{phonetics}
\end{entry}

\begin{entry}{演唱会}{14,11,6}{⽔、⼝、⼈}
  \begin{phonetics}{演唱会}{yan3 chang4 hui4}[][HSK 3]
    \definition[个,场]{s.}{concerto; recital vocal; concerto vocal}
  \end{phonetics}
\end{entry}

\begin{entry}{漫长}{14,4}{⽔、⾧}
  \begin{phonetics}{漫长}{man4chang2}[][HSK 5]
    \definition{adj.}{muito longo; interminável; (tempo, espaço) dura muito tempo}
  \end{phonetics}
\end{entry}

\begin{entry}{漫画}{14,8}{⽔、⽥}
  \begin{phonetics}{漫画}{man4hua4}[][HSK 5]
    \definition[幅,本,张,套]{s.}{desenho animado; caricatura}
  \end{phonetics}
\end{entry}

\begin{entry}{漫骂}{14,9}{⽔、⾺}
  \begin{phonetics}{漫骂}{man4ma4}
    \variantof{谩骂}
  \end{phonetics}
\end{entry}

\begin{entry}{熊}{14}{⽕}
  \begin{phonetics}{熊}{xiong2}[][HSK 5]
    \definition*{s.}{sobrenome Xiong}
    \definition[把]{s.}{urso}
    \definition{v.}{repreender; censurar;}
  \end{phonetics}
\end{entry}

\begin{entry}{熊猫}{14,11}{⽕、⽝}
  \begin{phonetics}{熊猫}{xiong2mao1}
    \definition[把,只]{s.}{panda gigante}
  \seealsoref{猫熊}{mao1xiong2}
  \end{phonetics}
\end{entry}

\begin{entry}{熏香}{14,9}{⽕、⾹}
  \begin{phonetics}{熏香}{xun1xiang1}
    \definition{s.}{incenso}
  \end{phonetics}
\end{entry}

\begin{entry}{疑问}{14,6}{⽦、⾨}
  \begin{phonetics}{疑问}{yi2wen4}[][HSK 4]
    \definition[个]{s.}{dúvida; consulta; pergunta; questionamento; coisas que não podem ser determinadas ou explicadas}
  \end{phonetics}
\end{entry}

\begin{entry}{瘦}{14}{⽧}
  \begin{phonetics}{瘦}{shou4}[][HSK 5]
    \definition{adj.}{magro; esquelético (oposto de 胖, 肥) | magro (oposto de 肥) | apertado (oposto de 肥) | infértil; pobre | esquelético; pouca gordura; pouca carne (em oposição a 或 ou 肥) | (roupas, sapatos, meias, etc.) apertado (em oposição a 肥) |magra; (carne comestível) com baixo teor de gordura (em oposição a 肥)}
    \definition{v.}{perder peso}
  \seealsoref{肥}{fei2}
  \seealsoref{或}{huo4}
  \seealsoref{胖}{pang4}
  \end{phonetics}
\end{entry}

\begin{entry}{碳}{14}{⽯}
  \begin{phonetics}{碳}{tan4}
    \definition{s.}{carbono (elemento químico)}
  \end{phonetics}
\end{entry}

\begin{entry}{磁带}{14,9}{⽯、⼱}
  \begin{phonetics}{磁带}{ci2dai4}
    \definition[盘,盒]{s.}{cassete | fita magnética}
  \end{phonetics}
\end{entry}

\begin{entry}{磁铁}{14,10}{⽯、⾦}
  \begin{phonetics}{磁铁}{ci2tie3}
    \definition{s.}{imã | magneto}
  \seealsoref{吸铁石}{xi1tie3shi2}
  \end{phonetics}
\end{entry}

\begin{entry}{磁盘}{14,11}{⽯、⽫}
  \begin{phonetics}{磁盘}{ci2pan2}
    \definition{s.}{disquete}
  \end{phonetics}
\end{entry}

\begin{entry}{稳}{14}{⽲}
  \begin{phonetics}{稳}{wen3}[][HSK 4]
    \definition{adj.}{constante; estável; firme | estável; estático; sedado | seguro; confiável; certo}
    \definition{adv.}{certamente; com certeza; seguramente; sem dúvida}
    \definition{v.}{estabilizar, manter estável}
  \end{phonetics}
\end{entry}

\begin{entry}{稳定}{14,8}{⽲、⼧}
  \begin{phonetics}{稳定}{wen3ding4}[][HSK 4]
    \definition{adj.}{estável; firme; descreve uma natureza, um estado, etc. relativamente fixo; não muda significativamente}
    \definition{s.}{estabilidade}
    \definition{v.}{manter estável; estabilizar; liquidar; resolver a situação}
  \end{phonetics}
\end{entry}

\begin{entry}{端午节}{14,4,5}{⽴、⼗、⾋}
  \begin{phonetics}{端午节}{duan1wu3jie2}
    \definition*{s.}{Festa do Duplo Cinco, Festival dos Barcos-Dragão (5º~dia do quinto mês lunar)}
  \end{phonetics}
\end{entry}

\begin{entry}{算}{14}{⽵}
  \begin{phonetics}{算}{suan4}[][HSK 2]
    \definition{adv.}{finalmente; por fim; no final; significa que, após um longo período de tempo ou muitas dificuldades, finalmente se alcançou o objetivo, equivalente a 总算}
    \definition{v.}{calcular; estimar; computar | contar; incluir | planejar; calcular; projetar | pensar; supor; especular | considerar; considerar como; contar como; reconhecer como | (aritmética) contar; ter peso | deixe estar; deixe passar; seguido por 了: desistir, não se importar mais}
  \seealsoref{了}{le5}
  \seealsoref{总算}{zong3suan4}
  \end{phonetics}
\end{entry}

\begin{entry}{算了}{14,2}{⽵、⼅}
  \begin{phonetics}{算了}{suan4le5}
    \definition{v.}{deixar | deixe estar | deixe passar | esqueça isso}
  \end{phonetics}
\end{entry}

\begin{entry}{算命}{14,8}{⽵、⼝}
  \begin{phonetics}{算命}{suan4ming4}
    \definition{s.}{cartomante}
    \definition{v.}{ler a sorte | fazer advinhações}
  \end{phonetics}
\end{entry}

\begin{entry}{管}{14}{⽵}
  \begin{phonetics}{管}{guan3}[][HSK 3]
    \definition*{s.}{Guan, um estado da dinastia Zhou}
    \definition*{s.}{sobrenome Guan}
    \definition{adj.}{estreito; restrito; limitado; pequeno}
    \definition{clas.}{usado para objetos cilíndricos longos e finos}
    \definition{conj.}{não importa (quem, o quê, como, etc.)}
    \definition{prep.}{função semelhante a 把, usada especificamente em conjunto com 叫}
    \definition[根,条,排]{s.}{cano; tubo | instrumento musical de sopro | válvula; tubo | duto; canal; vasos}
    \definition{v.}{administrar; dirigir; controlar; cuidar; ser responsável por | ter jurisdição sobre; administrar | disciplinar (crianças ou alunos) | preocupar-se com; importar-se com; incomodar-se com; intervir | fornecer; garantir | supervisionar | governar | submeter alguém a disciplina | assumir; arcar com | incomodar; interferir | assegurar; garantir}
  \seealsoref{把}{ba3}
  \seealsoref{叫}{jiao4}
  \end{phonetics}
\end{entry}

\begin{entry}{管……叫……}{14,5}{⽵、⼝}
  \begin{phonetics}{管……叫……}{guan3 jiao4}
    \definition{expr.}{chamar alguém (ou algo) de alguém (ou algo)}
  \end{phonetics}
\end{entry}

\begin{entry}{管家}{14,10}{⽵、⼧}
  \begin{phonetics}{管家}{guan3jia1}
    \definition{s.}{mordomo | governanta}
    \definition{v.}{administrar uma casa}
  \end{phonetics}
\end{entry}

\begin{entry}{管理}{14,11}{⽵、⽟}
  \begin{phonetics}{管理}{guan3li3}[][HSK 3]
    \definition{v.}{gerenciar; executar; administrar; governar; estar encarregado de; responsável por garantir o bom andamento de uma determinada tarefa | controlar; gerenciar; fazer com que pessoas e animais obedeçam ou se comportem de maneira ordeira | cuidar; zelar por; proteger; cuidar, organizar coisas}
  \end{phonetics}
\end{entry}

\begin{entry}{精力}{14,2}{⽶、⼒}
  \begin{phonetics}{精力}{jing1li4}[][HSK 4]
    \definition[些]{s.}{energia; vigor; força mental e física}
  \end{phonetics}
\end{entry}

\begin{entry}{精灵}{14,7}{⽶、⽕}
  \begin{phonetics}{精灵}{jing1ling2}
    \definition{s.}{espírito | fada | elfo | duende | gênio}
  \end{phonetics}
\end{entry}

\begin{entry}{精品}{14,9}{⽶、⼝}
  \begin{phonetics}{精品}{jing1pin3}
    \definition{s.}{produtos de qualidade | produto premium | bom trabalho (de arte)}
  \end{phonetics}
\end{entry}

\begin{entry}{精神}{14,9}{⽶、⽰}
  \begin{phonetics}{精神}{jing1shen2}[][HSK 3]
    \definition[种,个,类,股]{s.}{espírito; mente; estado mental; refere-se à consciência, às atividades mentais e ao estado psicológico geral de uma pessoa | substância; espírito; essência; propósito; significado principal}
  \end{phonetics}
  \begin{phonetics}{精神}{jing1shen5}[][HSK 3]
    \definition{adj.}{animado; espirituoso; vigoroso; descreve uma pessoa como cheia de energia | muito bonito; boa aparência, bom físico}
    \definition[种,个,类,股]{s.}{impulso; vigor; vitalidade}
  \end{phonetics}
\end{entry}

\begin{entry}{精致}{14,10}{⽶、⾄}
  \begin{phonetics}{精致}{jing1zhi4}
    \definition{adj.}{delicado | exótico | refinado}
  \end{phonetics}
\end{entry}

\begin{entry}{精彩}{14,11}{⽶、⼺}
  \begin{phonetics}{精彩}{jing1cai3}[][HSK 3]
    \definition{adj.}{brilhante; esplêndido; maravilhoso}
  \end{phonetics}
\end{entry}

\begin{entry}{缩小}{14,3}{⽷、⼩}
  \begin{phonetics}{缩小}{suo1 xiao3}[][HSK 4]
    \definition{v.}{reduzir, estreitar, encolher;  tornar menor (em oposição a 放大)}
  \seealsoref{放大}{fang4da4}
  \end{phonetics}
\end{entry}

\begin{entry}{缩短}{14,12}{⽷、⽮}
  \begin{phonetics}{缩短}{suo1duan3}[][HSK 4]
    \definition{v.}{encurtar; reduzir; diminuir}
  \end{phonetics}
\end{entry}

\begin{entry}{缩影卡片}{14,15,5,4}{⽷、⼺、⼘、⽚}
  \begin{phonetics}{缩影卡片}{suo1ying3 ka3pian4}
    \definition{s.}{cartão em miniatura}
  \end{phonetics}
\end{entry}

\begin{entry}{聚}{14}{⽿}
  \begin{phonetics}{聚}{ju4}[][HSK 4]
    \definition{v.}{reunir-se; juntar-se}
  \end{phonetics}
\end{entry}

\begin{entry}{聚会}{14,6}{⽿、⼈}
  \begin{phonetics}{聚会}{ju4hui4}[][HSK 4]
    \definition[个,次]{s.}{reunião; encontro; confraternização; festa}
    \definition{v.}{encontrar-se; reunir-se}
  \end{phonetics}
\end{entry}

\begin{entry}{聚散}{14,12}{⽿、⽁}
  \begin{phonetics}{聚散}{ju4san4}
    \definition{s.}{juntos e separados | agregação e dissipação}
  \end{phonetics}
\end{entry}

\begin{entry}{膜拜}{14,9}{⾁、⼿}
  \begin{phonetics}{膜拜}{mo2bai4}
    \definition{v.}{ajoelhar-se e se curvar com as mãos unidas no nível da testa | ter ou mostrar sentimentos fortes de respeito e admiração por um deus}
  \end{phonetics}
\end{entry}

\begin{entry}{舞}{14}{⾇}
  \begin{phonetics}{舞}{wu3}[][HSK 5]
    \definition{s.}{dança | palco; metáfora do domínio das atividades sociais}
    \definition{v.}{mover-se como numa dança | dançar com algo nas mãos; brincar com | florescer; empunhar; brandir | esvoaçar | fazer malabarismos; brincar com}
  \end{phonetics}
\end{entry}

\begin{entry}{舞厅}{14,4}{⾇、⼚}
  \begin{phonetics}{舞厅}{wu3ting1}
    \definition[间]{s.}{salão de dança | salão de baile}
  \end{phonetics}
\end{entry}

\begin{entry}{舞厅舞}{14,4,14}{⾇、⼚、⾇}
  \begin{phonetics}{舞厅舞}{wu3ting1wu3}
    \definition{s.}{dança de salão}
  \end{phonetics}
\end{entry}

\begin{entry}{舞台}{14,5}{⾇、⼝}
  \begin{phonetics}{舞台}{wu3 tai2}[][HSK 3]
    \definition[个]{s.}{palco; arena}
  \end{phonetics}
\end{entry}

\begin{entry}{舞会}{14,6}{⾇、⼈}
  \begin{phonetics}{舞会}{wu3hui4}
    \definition{s.}{baile}
  \end{phonetics}
\end{entry}

\begin{entry}{舞会舞}{14,6,14}{⾇、⼈、⾇}
  \begin{phonetics}{舞会舞}{wu3hui4wu3}
    \definition{s.}{baile}
  \end{phonetics}
\end{entry}

\begin{entry}{舞抃}{14,7}{⾇、⼿}
  \begin{phonetics}{舞抃}{wu3bian4}
    \definition{s.}{dançar por prazer}
  \end{phonetics}
\end{entry}

\begin{entry}{舞蹈}{14,17}{⾇、⾜}
  \begin{phonetics}{舞蹈}{wu3dao3}
    \definition{s.}{dança (ato performático)}
  \end{phonetics}
\end{entry}

\begin{entry}{蔓草}{14,9}{⾋、⾋}
  \begin{phonetics}{蔓草}{man4cao3}
    \definition{s.}{videira | trepadeira}
  \end{phonetics}
\end{entry}

\begin{entry}{蜘蛛}{14,12}{⾍、⾍}
  \begin{phonetics}{蜘蛛}{zhi1zhu1}
    \definition{s.}{aranha}
  \end{phonetics}
\end{entry}

\begin{entry}{蜘蛛网}{14,12,6}{⾍、⾍、⽹}
  \begin{phonetics}{蜘蛛网}{zhi1zhu1wang3}
    \definition{s.}{teia de aranha}
  \end{phonetics}
\end{entry}

\begin{entry}{蜜桃}{14,10}{⾍、⽊}
  \begin{phonetics}{蜜桃}{mi4tao2}
    \definition{s.}{pêssego suculento}
  \end{phonetics}
\end{entry}

\begin{entry}{蜡烛}{14,10}{⾍、⽕}
  \begin{phonetics}{蜡烛}{la4zhu2}
    \definition[根,支]{s.}{vela | círio | peça, geralmente de cera, que possui um pavio e se utiliza para iluminar}
  \end{phonetics}
\end{entry}

\begin{entry}{蜥易}{14,8}{⾍、⽇}
  \begin{phonetics}{蜥易}{xi1yi4}
    \variantof{蜥蜴}
  \end{phonetics}
\end{entry}

\begin{entry}{蜥蜴}{14,14}{⾍、⾍}
  \begin{phonetics}{蜥蜴}{xi1yi4}
    \definition{s.}{lagarto}
  \end{phonetics}
\end{entry}

\begin{entry}{蜻蜓}{14,12}{⾍、⾍}
  \begin{phonetics}{蜻蜓}{qing1ting2}
    \definition{s.}{libélula}
  \end{phonetics}
\end{entry}

\begin{entry}{蜻蝏}{14,15}{⾍、⾍}
  \begin{phonetics}{蜻蝏}{qing1ting2}
    \variantof{蜻蜓}
  \end{phonetics}
\end{entry}

\begin{entry}{蝉}{14}{⾍}
  \begin{phonetics}{蝉}{chan2}
    \definition{s.}{cigarra}
  \end{phonetics}
\end{entry}

\begin{entry}{褐色}{14,6}{⾐、⾊}
  \begin{phonetics}{褐色}{he4 se4}
    \definition{s.}{cor marrom}
  \end{phonetics}
\end{entry}

\begin{entry}{豪华}{14,6}{⾗、⼗}
  \begin{phonetics}{豪华}{hao2hua2}
    \definition{adj.}{luxuoso}
  \end{phonetics}
\end{entry}

\begin{entry}{赛}{14}{⾙}
  \begin{phonetics}{赛}{sai4}
    \definition{s.}{competição}
    \definition{v.}{competir | superar | destacar-se}
  \end{phonetics}
\end{entry}

\begin{entry}{赛车}{14,4}{⾙、⾞}
  \begin{phonetics}{赛车}{sai4che1}
    \definition{s.}{corrida de automóvel | corrida de bicicleta | carro de corrida}
  \end{phonetics}
\end{entry}

\begin{entry}{辣}{14}{⾟}
  \begin{phonetics}{辣}{la4}[][HSK 4]
    \definition{adj.}{apimentado; picante; pungente; quente | cruel; implacável; venenoso; vicioso}
    \definition{v.}{queimar; picar; formigar; ter uma irritação picante (boca, nariz ou olhos)}
  \end{phonetics}
\end{entry}

\begin{entry}{遭到}{14,8}{⾡、⼑}
  \begin{phonetics}{遭到}{zao1dao4}
    \definition{v.}{sofrer | encontrar-se com (algo infeliz)}
  \end{phonetics}
\end{entry}

\begin{entry}{遭受}{14,8}{⾡、⼜}
  \begin{phonetics}{遭受}{zao1shou4}
    \definition{v.}{sofrer | suportar (perda, infornúnio)}
  \end{phonetics}
\end{entry}

\begin{entry}{遭遇}{14,12}{⾡、⾡}
  \begin{phonetics}{遭遇}{zao1yu4}
    \definition{s.}{experiência (amarga)}
    \definition{v.}{encontrar-se com (algo infeliz)}
  \end{phonetics}
\end{entry}

\begin{entry}{酷}{14}{⾣}
  \begin{phonetics}{酷}{ku4}
    \definition{adj.}{impiedoso | forte (por exemplo, vinho) | (empréstimo linguístico) legal, \emph{cool}}
  \end{phonetics}
\end{entry}

\begin{entry}{酷斯拉}{14,12,8}{⾣、⽄、⼿}
  \begin{phonetics}{酷斯拉}{ku4si1la1}
    \definition*{s.}{Godzilla (Japonês ゴジラ Gojira)}
  \seealsoref{哥斯拉}{ge1si1la1}
  \end{phonetics}
\end{entry}

\begin{entry}{酸}{14}{⾣}
  \begin{phonetics}{酸}{suan1}[][HSK 4]
    \definition{adj.}{azedo; ácido | aflito; angustiado; doente do coração | pedante; descreve uma pessoa que finge ser culta e também descreve uma pessoa que é muito inflexível com suas próprias ideias e não está disposta a mudá-las para atender às exigências da época, é usado principalmente para satirizar intelectuais que fingem ser capazes de escrever poemas e artigos | ciumento; invejoso; sentimentos desconfortáveis porque outra pessoa é melhor do que você e, em geral, também apresenta comportamento hostil}
    \definition{s.}{ácido; produto químico que tem um sabor ácido quando misturado com água}
    \definition{v.}{estar dolorido (devido à fadiga ou doença); descreve a sensação de não ter força muscular e um pouco de dor por estar doente ou muito cansado}
  \end{phonetics}
\end{entry}

\begin{entry}{酸奶}{14,5}{⾣、⼥}
  \begin{phonetics}{酸奶}{suan1 nai3}[][HSK 4]
    \definition[瓶,杯,盒,袋]{s.}{iogurte; produto lácteo fermentado por bactérias de ácido láctico}
  \end{phonetics}
\end{entry}

\begin{entry}{酸甜苦辣}{14,11,8,14}{⾣、⽢、⾋、⾟}
  \begin{phonetics}{酸甜苦辣}{suan1 tian2 ku3 la4}[][HSK 5]
    \definition{expr.}{os altos e baixos da vida; as experiências agridoces da vida; os aspectos doces, azedos, amargos e picantes da vida; refere-se a todos os tipos de sabores, como metáfora para experiências diversas, como felicidade, sofrimento, etc. | azedo, doce, amargo, picante — alegrias e tristezas da vida}
  \end{phonetics}
\end{entry}

\begin{entry}{酸辣汤}{14,14,6}{⾣、⾟、⽔}
  \begin{phonetics}{酸辣汤}{suan1la4tang1}
    \definition{s.}{sopa avinagrada e picante (prato)}
  \end{phonetics}
\end{entry}

\begin{entry}{锺}{14}{⾦}
  \begin{phonetics}{锺}{zhong1}
    \variantof{钟}
  \end{phonetics}
\end{entry}

\begin{entry}{锻炼}{14,9}{⾦、⽕}
  \begin{phonetics}{锻炼}{duan4lian4}[][HSK 4]
    \definition{v.}{exercitar-se; fazer (ou fazer) exercícios; submeter-se a treinamento físico; fortalecer o corpo por meio do esporte | fortalecer; endurecer; aprimorar as habilidades de trabalho e de vida por meio de trabalho e outras atividades | forjar ou moldar metal para torná-lo mais refinado; refere-se à transformação de materiais metálicos em objetos de determinada forma e tamanho por meio de aquecimento, batimento, prensagem etc.}
  \end{phonetics}
\end{entry}

\begin{entry}{镀金}{14,8}{⾦、⾦}
  \begin{phonetics}{镀金}{du4jin1}
    \definition{v.}{banhar a ouro | dourar | (figurativo) fazer algo muito comum parecer especial}
  \end{phonetics}
\end{entry}

\begin{entry}{隧道}{14,12}{⾩、⾡}
  \begin{phonetics}{隧道}{sui4dao4}
    \definition{s.}{túnel}
  \end{phonetics}
\end{entry}

\begin{entry}{需求}{14,7}{⾬、⽔}
  \begin{phonetics}{需求}{xu1qiu2}[][HSK 3]
    \definition{s.}{necessidades; demanda; requisito; requerimento; exigência | solicitações decorrentes de necessidades}
  \end{phonetics}
\end{entry}

\begin{entry}{需要}{14,9}{⾬、⾑}
  \begin{phonetics}{需要}{xu1yao4}[][HSK 3]
    \definition{s.}{necessidade | desejo ou solicitação de algo}
    \definition{v.}{precisar; querer; requerer; demandar}
  \end{phonetics}
\end{entry}

\begin{entry}{静}{14}{⾭}
  \begin{phonetics}{静}{jing4}[][HSK 3]
    \definition*{s.}{sobrenome Jing}
    \definition{adj.}{tranquilo;  sossegado; calmo; imóvel | silencioso; quieto; sem emitir nenhum som | calmo, sereno; serenidade; (interior) paz}
    \definition{v.}{acalmar-se; aquietar-se; tranquilizar (o coração)}
  \end{phonetics}
\end{entry}

\begin{entry}{颗}{14}{⾴}
  \begin{phonetics}{颗}{ke1}[][HSK 5]
    \definition{clas.}{para grãos, pérolas, dentes, corações, satelites, pequenas esferas, etc.}
    \definition{s.}{grão; partícula; pequenas coisas redondas}
  \end{phonetics}
\end{entry}

\begin{entry}{馒头}{14,5}{⾷、⼤}
  \begin{phonetics}{馒头}{man2tou5}
    \definition{s.}{pão cozido no vapor}
  \end{phonetics}
\end{entry}

\begin{entry}{魅力}{14,2}{⿁、⼒}
  \begin{phonetics}{魅力}{mei4li4}
    \definition{s.}{charme | fascínio | glamour | carisma}
  \end{phonetics}
\end{entry}

\begin{entry}{鲜}{14}{⿂}
  \begin{phonetics}{鲜}{xian1}[][HSK 4]
    \definition*{s.}{sobrenome Xian}
    \definition{adj.}{fresco; novo; fresco (experiência, comida etc.) |brilhante; de cores vivas | saboroso; delicioso | exuberante; luxuriante}
    \definition{s.}{aves e animais recém-abatidos; vegetais recém-colhidos; frutas, etc. | alimentos aquáticos; geralmente, peixes vivos, camarões, etc., para alimentação}
  \end{phonetics}
  \begin{phonetics}{鲜}{xian3}
    \definition{adj.}{raro; pouco; pequeno;}
    \definition{adv.}{raramente}
  \end{phonetics}
\end{entry}

\begin{entry}{鲜花}{14,7}{⿂、⾋}
  \begin{phonetics}{鲜花}{xian1 hua1}[][HSK 4]
    \definition[朵,束,支,捧]{s.}{flor; flores frescas; flores bonitas e frescas}
  \end{phonetics}
\end{entry}

\begin{entry}{鲜明}{14,8}{⿂、⽇}
  \begin{phonetics}{鲜明}{xian1ming2}[][HSK 4]
    \definition{adj.}{brilhante (cor) | distinto; bem definido; nítido; claro; característico}
  \end{phonetics}
\end{entry}

\begin{entry}{鲜艳}{14,10}{⿂、⾊}
  \begin{phonetics}{鲜艳}{xian1yan4}[][HSK 5]
    \definition{adj.}{de cores alegres; de cores brilhantes}
  \end{phonetics}
\end{entry}

\begin{entry}{鼻子}{14,3}{⿐、⼦}
  \begin{phonetics}{鼻子}{bi2zi5}[][HSK 5]
    \definition[个,只]{s.}{nariz; órgão da face, responsável pela respiração e pelo olfato}
  \end{phonetics}
\end{entry}

%%%%% EOF %%%%%

