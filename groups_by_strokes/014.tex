%%%
%%% 14画
%%%

\section*{14画}\addcontentsline{toc}{section}{14画}

\begin{entry}{㮸}{14}[Radical 木]
  \begin{phonetics}{㮸}{song4}
    \variantof{送}
  \end{phonetics}
\end{entry}

\begin{entry}{僧}{14}[Radical 人]
  \begin{phonetics}{僧}{seng1}
    \definition{s.}{monge Budista, abreviação de 僧伽}
    \seeref{僧伽}{seng1qie2}
  \end{phonetics}
\end{entry}

\begin{entry}{僧伽}{14,7}
  \begin{phonetics}{僧伽}{seng1qie2}
    \definition{s.}{sangha ou sanga (Budismo) | a comunidade monástica | monge}
  \end{phonetics}
\end{entry}

\begin{entry}{嘉年华}{14,6,6}
  \begin{phonetics}{嘉年华}{jia1nian2hua2}
    \definition{s.}{(empréstimo linguístico) carnaval}
  \end{phonetics}
\end{entry}

\begin{entry}{墙}{14}[Radical 土]
  \begin{phonetics}{墙}{qiang2}[][HSK 2]
    \definition[面,堵]{s.}{parede}
    \definition{v.}{(gíria) bloquear (um website) (usado geralmente na voz passiva: 被墙)}
  \end{phonetics}
\end{entry}

\begin{entry}{墙纸}{14,7}
  \begin{phonetics}{墙纸}{qiang2zhi3}
    \definition{s.}{papel de parede}
  \end{phonetics}
\end{entry}

\begin{entry}{墬}{14}[Radical 土]
  \begin{phonetics}{墬}{di4}
    \variantof{地}
  \end{phonetics}
\end{entry}

\begin{entry}{寨}{14}[Radical 宀]
  \begin{phonetics}{寨}{zhai4}
    \definition{s.}{fortaleza | paliçada | acampamento | vila (paliçada)}
  \end{phonetics}
\end{entry}

\begin{entry}{愿}{14}[Radical 心]
  \begin{phonetics}{愿}{yuan4}
    \definition{adj.}{honesto e prudente}
  \end{phonetics}
\end{entry}

\begin{entry}{愿意}{14,13}
  \begin{phonetics}{愿意}{yuan4yi4}[][HSK 2]
    \definition{s.}{desejo | esperança}
    \definition{v.}{estar disposto | estar pronto}
  \end{phonetics}
\end{entry}

\begin{entry}{慢}{14}[Radical 心]
  \begin{phonetics}{慢}{man4}[][HSK 1]
    \definition{adj.}{devagar}
  \end{phonetics}
\end{entry}

\begin{entry}{慢动作}{14,6,7}
  \begin{phonetics}{慢动作}{man4dong4zuo4}
    \definition{s.}{(cinema) câmera lenta}
  \end{phonetics}
\end{entry}

\begin{entry}{摔}{14}[Radical 手]
  \begin{phonetics}{摔}{shuai1}
    \definition{v.}{cair | cair e quebrar | partir}
  \end{phonetics}
\end{entry}

\begin{entry}{斡旋}{14,11}
  \begin{phonetics}{斡旋}{wo4xuan2}
    \definition{v.}{mediar (um conflito, etc.)}
  \end{phonetics}
\end{entry}

\begin{entry}{旗}{14}[Radical 方]
  \begin{phonetics}{旗}{qi2}
    \definition[面]{s.}{bandeira}
  \end{phonetics}
\end{entry}

\begin{entry}{槃}{14}[Radical 木]
  \begin{phonetics}{槃}{pan2}
    \variantof{盘}
  \end{phonetics}
\end{entry}

\begin{entry}{模仿}{14,6}
  \begin{phonetics}{模仿}{mo2fang3}
    \definition{v.}{imitar | copiar}
  \end{phonetics}
\end{entry}

\begin{entry}{模式}{14,6}
  \begin{phonetics}{模式}{mo2shi4}
    \definition{s.}{modelo | modo | método | padrão}
  \end{phonetics}
\end{entry}

\begin{entry}{模具}{14,8}
  \begin{phonetics}{模具}{mu2ju4}
    \definition{s.}{molde | matriz | padrão}
  \end{phonetics}
\end{entry}

\begin{entry}{歌}{14}[Radical ⽋]
  \begin{phonetics}{歌}{ge1}[][HSK 1]
    \definition[支,首]{s.}{canção | canto}
  \end{phonetics}
\end{entry}

\begin{entry}{歌迷}{14,9}
  \begin{phonetics}{歌迷}{ge1mi2}
    \definition{s.}{fã de um cantor}
  \end{phonetics}
\end{entry}

\begin{entry}{滴}{14}[Radical 水]
  \begin{phonetics}{滴}{di1}
    \definition{s.}{uma gota}
    \definition{v.}{pingar}
  \end{phonetics}
\end{entry}

\begin{entry}{漂}{14}[Radical 水]
  \begin{phonetics}{漂}{piao1}
    \definition{v.}{flutuar | estar a deriva}
  \end{phonetics}
  \begin{phonetics}{漂}{piao3}
    \definition{v.}{alvejar | branquear}
  \end{phonetics}
  \begin{phonetics}{漂}{piao4}
    \definition{adj.}{usado em 漂亮}
    \seeref{漂亮}{piao4liang5}
  \end{phonetics}
\end{entry}

\begin{entry}{漂亮}{14,9}
  \begin{phonetics}{漂亮}{piao4liang5}[][HSK 2]
    \definition{adj.}{bonita, linda | bonito, lindo (para objetos inanimados)}
  \end{phonetics}
\end{entry}

\begin{entry}{漂流}{14,10}
  \begin{phonetics}{漂流}{piao1liu2}
    \definition{s.}{\emph{rafting}}
    \definition{v.}{ser levado pela correnteza | flutuar ao longo ou sobre}
  \end{phonetics}
\end{entry}

\begin{entry}{漏}{14}[Radical 水]
  \begin{phonetics}{漏}{lou4}
    \definition{s.}{relógio d'água ou ampulheta}
    \definition{v.}{vazar | divulgar | deixar de fora por engano}
  \end{phonetics}
\end{entry}

\begin{entry}{漏电}{14,5}
  \begin{phonetics}{漏电}{lou4dian4}
    \definition{v.}{vazar eletricidade}
  \end{phonetics}
\end{entry}

\begin{entry}{演}{14}[Radical 水]
  \begin{phonetics}{演}{yan3}
    \definition{v.}{atuar, encenar (em uma peça, show, etc.)}
  \end{phonetics}
\end{entry}

\begin{entry}{演员}{14,7}
  \begin{phonetics}{演员}{yan3yuan2}
    \definition{s.}{ator | artista}
  \end{phonetics}
\end{entry}

\begin{entry}{漫骂}{14,9}
  \begin{phonetics}{漫骂}{man4ma4}
    \variantof{谩骂}
  \end{phonetics}
\end{entry}

\begin{entry}{熊}{14}[Radical 火]
  \begin{phonetics}{熊}{xiong2}
    \definition*{s.}{sobrenome Xiong}
    \definition{adj.}{incapaz}
    \definition[把]{s.}{urso}
    \definition{v.}{repreender}
  \end{phonetics}
\end{entry}

\begin{entry}{熊猫}{14,11}
  \begin{phonetics}{熊猫}{xiong2mao1}
    \definition[把,只]{s.}{panda gigante}
  \seealsoref{猫熊}{mao1xiong2}
  \end{phonetics}
\end{entry}

\begin{entry}{熏香}{14,9}
  \begin{phonetics}{熏香}{xun1xiang1}
    \definition{s.}{incenso}
  \end{phonetics}
\end{entry}

\begin{entry}{瘦}{14}[Radical 疒]
  \begin{phonetics}{瘦}{shou4}
    \definition{adj.}{magro | emagrecido | apertado (roupas) | improdutivo (terras) | magro (carne)}
    \definition{v.}{perder peso}
  \end{phonetics}
\end{entry}

\begin{entry}{碳}{14}[Radical 石]
  \begin{phonetics}{碳}{tan4}
    \definition{s.}{carbono (elemento químico)}
  \end{phonetics}
\end{entry}

\begin{entry}{磁带}{14,9}
  \begin{phonetics}{磁带}{ci2dai4}
    \definition[盘,盒]{s.}{cassete | fita magnética}
  \end{phonetics}
\end{entry}

\begin{entry}{磁铁}{14,10}
  \begin{phonetics}{磁铁}{ci2tie3}
    \definition{s.}{imã | magneto}
  \seealsoref{吸铁石}{xi1tie3shi2}
  \end{phonetics}
\end{entry}

\begin{entry}{磁盘}{14,11}
  \begin{phonetics}{磁盘}{ci2pan2}
    \definition{s.}{disquete}
  \end{phonetics}
\end{entry}

\begin{entry}{稳定}{14,8}
  \begin{phonetics}{稳定}{wen3ding4}
    \definition{adj.}{estável}
    \definition{s.}{estabilidade}
    \definition{v.}{estabilizar | pacificar}
  \end{phonetics}
\end{entry}

\begin{entry}{端午节}{14,4,5}
  \begin{phonetics}{端午节}{duan1wu3jie2}
    \definition*{s.}{Festa do Duplo Cinco, Festival dos Barcos-Dragão (5º~dia do quinto mês lunar)}
  \end{phonetics}
\end{entry}

\begin{entry}{算}{14}[Radical 竹]
  \begin{phonetics}{算}{suan4}[][HSK 2]
    \definition{adv.}{finalmente; no fim; finalmente}
    \definition{v.}{calcular | contar | computar | figurar | incluir | planejar | calcular | pensar | supor | considerar | considerar como | contar como | carregar peso | deixar estar | deixar passar}
  \end{phonetics}
\end{entry}

\begin{entry}{算了}{14,2}
  \begin{phonetics}{算了}{suan4le5}
    \definition{v.}{deixar | deixe estar | deixe passar | esqueça isso}
  \end{phonetics}
\end{entry}

\begin{entry}{算命}{14,8}
  \begin{phonetics}{算命}{suan4ming4}
    \definition{s.}{cartomante}
    \definition{v.}{ler a sorte | fazer advinhações}
  \end{phonetics}
\end{entry}

\begin{entry}{管家}{14,10}
  \begin{phonetics}{管家}{guan3jia1}
    \definition{s.}{mordomo | governanta}
    \definition{v.}{administrar uma casa}
  \end{phonetics}
\end{entry}

\begin{entry}{精灵}{14,7}
  \begin{phonetics}{精灵}{jing1ling2}
    \definition{s.}{espírito | fada | elfo | duende | gênio}
  \end{phonetics}
\end{entry}

\begin{entry}{精品}{14,9}
  \begin{phonetics}{精品}{jing1pin3}
    \definition{s.}{produtos de qualidade | produto premium | bom trabalho (de arte)}
  \end{phonetics}
\end{entry}

\begin{entry}{精致}{14,10}
  \begin{phonetics}{精致}{jing1zhi4}
    \definition{adj.}{delicado | exótico | refinado}
  \end{phonetics}
\end{entry}

\begin{entry}{精彩}{14,11}
  \begin{phonetics}{精彩}{jing1cai3}
    \definition{adj.}{espetacular | maravilhoso | brilhante}
  \end{phonetics}
\end{entry}

\begin{entry}{缩影卡片}{14,15,5,4}
  \begin{phonetics}{缩影卡片}{suo1ying3 ka3pian4}
    \definition{s.}{cartão em miniatura}
  \end{phonetics}
\end{entry}

\begin{entry}{聚散}{14,12}
  \begin{phonetics}{聚散}{ju4san4}
    \definition{s.}{juntos e separados | agregação e dissipação}
  \end{phonetics}
\end{entry}

\begin{entry}{膜拜}{14,9}
  \begin{phonetics}{膜拜}{mo2bai4}
    \definition{v.}{ajoelhar-se e se curvar com as mãos unidas no nível da testa | ter ou mostrar sentimentos fortes de respeito e admiração por um deus}
  \end{phonetics}
\end{entry}

\begin{entry}{舞}{14}[Radical 舛]
  \begin{phonetics}{舞}{wu3}
    \definition{s.}{dança}
  \end{phonetics}
\end{entry}

\begin{entry}{舞厅}{14,4}
  \begin{phonetics}{舞厅}{wu3ting1}
    \definition[间]{s.}{salão de dança | salão de baile}
  \end{phonetics}
\end{entry}

\begin{entry}{舞厅舞}{14,4,14}
  \begin{phonetics}{舞厅舞}{wu3ting1wu3}
    \definition{s.}{dança de salão}
  \end{phonetics}
\end{entry}

\begin{entry}{舞会}{14,6}
  \begin{phonetics}{舞会}{wu3hui4}
    \definition{s.}{baile}
  \end{phonetics}
\end{entry}

\begin{entry}{舞会舞}{14,6,14}
  \begin{phonetics}{舞会舞}{wu3hui4wu3}
    \definition{s.}{baile}
  \end{phonetics}
\end{entry}

\begin{entry}{舞抃}{14,7}
  \begin{phonetics}{舞抃}{wu3bian4}
    \definition{s.}{dançar por prazer}
  \end{phonetics}
\end{entry}

\begin{entry}{舞蹈}{14,17}
  \begin{phonetics}{舞蹈}{wu3dao3}
    \definition{s.}{dança (ato performático)}
  \end{phonetics}
\end{entry}

\begin{entry}{蔓草}{14,9}
  \begin{phonetics}{蔓草}{man4cao3}
    \definition{s.}{videira | trepadeira}
  \end{phonetics}
\end{entry}

\begin{entry}{蜘蛛}{14,12}
  \begin{phonetics}{蜘蛛}{zhi1zhu1}
    \definition{s.}{aranha}
  \end{phonetics}
\end{entry}

\begin{entry}{蜘蛛网}{14,12,6}
  \begin{phonetics}{蜘蛛网}{zhi1zhu1wang3}
    \definition{s.}{teia de aranha}
  \end{phonetics}
\end{entry}

\begin{entry}{蜜桃}{14,10}
  \begin{phonetics}{蜜桃}{mi4tao2}
    \definition{s.}{pêssego suculento}
  \end{phonetics}
\end{entry}

\begin{entry}{蜡烛}{14,10}
  \begin{phonetics}{蜡烛}{la4zhu2}
    \definition[根,支]{s.}{vela | círio | peça, geralmente de cera, que possui um pavio e se utiliza para iluminar}
  \end{phonetics}
\end{entry}

\begin{entry}{蜥易}{14,8}
  \begin{phonetics}{蜥易}{xi1yi4}
    \variantof{蜥蜴}
  \end{phonetics}
\end{entry}

\begin{entry}{蜥蜴}{14,14}
  \begin{phonetics}{蜥蜴}{xi1yi4}
    \definition{s.}{lagarto}
  \end{phonetics}
\end{entry}

\begin{entry}{蜻蜓}{14,12}
  \begin{phonetics}{蜻蜓}{qing1ting2}
    \definition{s.}{libélula}
  \end{phonetics}
\end{entry}

\begin{entry}{蜻蝏}{14,15}
  \begin{phonetics}{蜻蝏}{qing1ting2}
    \variantof{蜻蜓}
  \end{phonetics}
\end{entry}

\begin{entry}{蝉}{14}[Radical 虫]
  \begin{phonetics}{蝉}{chan2}
    \definition{s.}{cigarra}
  \end{phonetics}
\end{entry}

\begin{entry}{褐色}{14,6}
  \begin{phonetics}{褐色}{he4se4}
    \definition{s.}{cor marrom}
  \end{phonetics}
\end{entry}

\begin{entry}{豪华}{14,6}
  \begin{phonetics}{豪华}{hao2hua2}
    \definition{adj.}{luxuoso}
  \end{phonetics}
\end{entry}

\begin{entry}{赛}{14}[Radical 貝]
  \begin{phonetics}{赛}{sai4}
    \definition{s.}{competição}
    \definition{v.}{competir | superar | destacar-se}
  \end{phonetics}
\end{entry}

\begin{entry}{赛车}{14,4}
  \begin{phonetics}{赛车}{sai4che1}
    \definition{s.}{corrida de automóvel | corrida de bicicleta | carro de corrida}
  \end{phonetics}
\end{entry}

\begin{entry}{辣}{14}[Radical 辛]
  \begin{phonetics}{辣}{la4}
    \definition{adj.}{picante | pungente}
  \end{phonetics}
\end{entry}

\begin{entry}{遭到}{14,8}
  \begin{phonetics}{遭到}{zao1dao4}
    \definition{v.}{sofrer | encontrar-se com (algo infeliz)}
  \end{phonetics}
\end{entry}

\begin{entry}{遭受}{14,8}
  \begin{phonetics}{遭受}{zao1shou4}
    \definition{v.}{sofrer | suportar (perda, infornúnio)}
  \end{phonetics}
\end{entry}

\begin{entry}{遭遇}{14,12}
  \begin{phonetics}{遭遇}{zao1yu4}
    \definition{s.}{experiência (amarga)}
    \definition{v.}{encontrar-se com (algo infeliz)}
  \end{phonetics}
\end{entry}

\begin{entry}{酷}{14}[Radical 酉]
  \begin{phonetics}{酷}{ku4}
    \definition{adj.}{impiedoso | forte (por exemplo, vinho) | (empréstimo linguístico) legal, \emph{cool}}
  \end{phonetics}
\end{entry}

\begin{entry}{酷斯拉}{14,12,8}
  \begin{phonetics}{酷斯拉}{ku4si1la1}
    \definition*{s.}{Godzilla (Japonês ゴジラ Gojira)}
  \seealsoref{哥斯拉}{ge1si1la1}
  \end{phonetics}
\end{entry}

\begin{entry}{酸}{14}[Radical 酉]
  \begin{phonetics}{酸}{suan1}
    \definition{adj.}{ácido | avinagrado}
  \end{phonetics}
\end{entry}

\begin{entry}{酸辣汤}{14,14,6}
  \begin{phonetics}{酸辣汤}{suan1la4tang1}
    \definition{s.}{sopa avinagrada e picante (prato)}
  \end{phonetics}
\end{entry}

\begin{entry}{锺}{14}[Radical 金]
  \begin{phonetics}{锺}{zhong1}
    \variantof{钟}
  \end{phonetics}
\end{entry}

\begin{entry}{锻炼}{14,9}
  \begin{phonetics}{锻炼}{duan4lian4}
    \definition{v.}{fazer exercício físico | praticar esporte}
  \end{phonetics}
\end{entry}

\begin{entry}{镀金}{14,8}
  \begin{phonetics}{镀金}{du4jin1}
    \definition{v.}{banhar a ouro | dourar | (figurativo) fazer algo muito comum parecer especial}
  \end{phonetics}
\end{entry}

\begin{entry}{隧道}{14,12}
  \begin{phonetics}{隧道}{sui4dao4}
    \definition{s.}{túnel}
  \end{phonetics}
\end{entry}

\begin{entry}{需要}{14,9}
  \begin{phonetics}{需要}{xu1yao4}
    \definition{s.}{necessidade}
    \definition{v.}{precisar | necessitar}
  \end{phonetics}
\end{entry}

\begin{entry}{颗}{14}[Radical 頁]
  \begin{phonetics}{颗}{ke1}
    \definition{clas.}{para grãos, pérolas, dentes, corações, satelites, pequenas esferas, etc.}
  \end{phonetics}
\end{entry}

\begin{entry}{馒头}{14,5}
  \begin{phonetics}{馒头}{man2tou5}
    \definition{s.}{pão cozido no vapor}
  \end{phonetics}
\end{entry}

\begin{entry}{魅力}{14,2}
  \begin{phonetics}{魅力}{mei4li4}
    \definition{s.}{charme | fascínio | glamour | carisma}
  \end{phonetics}
\end{entry}

\begin{entry}{鼻子}{14,3}
  \begin{phonetics}{鼻子}{bi2zi5}
    \definition[个,只]{s.}{nariz}
  \end{phonetics}
\end{entry}

%%%%% EOF %%%%%

