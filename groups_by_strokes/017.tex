%%%
%%% 17画
%%%

\section*{17画}\addcontentsline{toc}{section}{17画}

\begin{entry}{戴}{17}{⼽}
  \begin{phonetics}{戴}{dai4}[][HSK 4]
    \definition*{s.}{sobrenome Dai}
    \definition{v.}{usar/vestir (óculos, gravata, relógio de pulso, luvas); colocar objetos em sua cabeça, rosto, pescoço, peito, braços etc. | honrar; respeitar;}
  \end{phonetics}
\end{entry}

\begin{entry}{擦}{17}{⼿}
  \begin{phonetics}{擦}{ca1}[][HSK 4]
    \definition{v.}{enxugar; esfregar; apagar; limpar; limpar esfregando com um pano, toalha de mão, etc. | espalhar sobre; colocar sobre | passar raspando | ralar (em pedaços); ralar frutas em um ralador para fazer fios finos}
  \end{phonetics}
\end{entry}

\begin{entry}{擦拭}{17,9}{⼿、⼿}
  \begin{phonetics}{擦拭}{ca1shi4}
    \definition{v.}{limpar com um pano}
  \end{phonetics}
\end{entry}

\begin{entry}{瞧}{17}{⽬}
  \begin{phonetics}{瞧}{qiao2}[][HSK 5]
    \definition{v.}{ver; olhar | tratar; diagnosticar e tratar | ver; visitar; fazer uma visita}
  \end{phonetics}
\end{entry}

\begin{entry}{窾}{17}{⽳}
  \begin{phonetics}{窾}{cuan4}
    \definition{adj.}{vazio | seco | destituído; pobre}
    \definition{s.}{buraco | lei}
    \definition{v.}{esconder}
  \end{phonetics}
  \begin{phonetics}{窾}{kuan3}
    \definition{adj.}{oco}
    \definition{s.}{rachadura; cavidade | (onomatopéia) água batendo na rocha}
    \definition{v.}{escavar um buraco}
  \end{phonetics}
\end{entry}

\begin{entry}{糟}{17}{⽶}
  \begin{phonetics}{糟}{zao1}[][HSK 5]
    \definition{adj.}{pobre; apodrecido; deteriorado | estragado; em uma bagunça; em um estado miserável (terrível) | (situação ou circunstância) ruim; desfavorável}
    \definition{s.}{resíduos de destilação de bebidas alcoólicas; resíduos do processo de fermentação do vinho}
    \definition{v.}{marinar alimentos em vinho ou mosto}
  \end{phonetics}
\end{entry}

\begin{entry}{糟糕}{17,16}{⽶、⽶}
  \begin{phonetics}{糟糕}{zao1gao1}[][HSK 5]
    \definition{adj.}{(corpo, situação, etc.) muito ruim, péssimo}
    \definition{interj.}{que terrível; que má sorte; muito ruim}
  \end{phonetics}
\end{entry}

\begin{entry}{繁}{17}{⽷}
  \begin{phonetics}{繁}{fan2}
    \definition{adj.}{em grande número; numerosos; múltiplos (oposto a 简) | em grande número; numerosos; complexos; complicado}
    \definition{v.}{propagar; multiplicar}
  \seealsoref{简}{jian3}
  \end{phonetics}
\end{entry}

\begin{entry}{繁荣}{17,9}{⽷、⾋}
  \begin{phonetics}{繁荣}{fan2rong2}[][HSK 5]
    \definition{adj.}{florescente; próspero}
    \definition{v.}{promover; prosperar}
  \end{phonetics}
\end{entry}

\begin{entry}{螺}{17}{⾍}
  \begin{phonetics}{螺}{luo2}
    \definition{s.}{concha em espiral | caracol | búzio}
  \end{phonetics}
\end{entry}

\begin{entry}{螺丝}{17,5}{⾍、⼀}
  \begin{phonetics}{螺丝}{luo2si1}
    \definition{s.}{parafuso}
  \end{phonetics}
\end{entry}

\begin{entry}{赢}{17}{⾙}
  \begin{phonetics}{赢}{ying2}[][HSK 3]
    \definition{v.}{vencer; derrotar | ganhar; lucrar}
  \end{phonetics}
\end{entry}

\begin{entry}{赢得}{17,11}{⾙、⼻}
  \begin{phonetics}{赢得}{ying2 de2}[][HSK 4]
    \definition{v.}{ganhar; obter; conquistar; assegurar; garantir}
  \end{phonetics}
\end{entry}

\begin{entry}{辫}{17}{⾟}
  \begin{phonetics}{辫}{bian4}
    \definition{s.}{trança; rabo de cavalo | para coisas como uma trança}
  \end{phonetics}
\end{entry}

\begin{entry}{辫子}{17,3}{⾟、⼦}
  \begin{phonetics}{辫子}{bian4zi5}
    \definition[根,条]{s.}{trança | um erro ou falha que pode ser explorado por um oponente | alça}
  \end{phonetics}
\end{entry}

\begin{entry}{邉}{17}{⾡}
  \begin{phonetics}{邉}{bian1}
    \variantof{边}
  \end{phonetics}
\end{entry}

\begin{entry}{霜}{17}{⾬}
  \begin{phonetics}{霜}{shuang1}
    \definition{s.}{geada | pó branco ou creme espalhado por uma superfície | glacê | creme de pele}
  \end{phonetics}
\end{entry}

\begin{entry}{鳄}{17}{⿂}
  \begin{phonetics}{鳄}{e4}
    \definition{s.}{crocodilo;  jacaré}
  \end{phonetics}
\end{entry}

\begin{entry}{鳄鱼}{17,8}{⿂、⿂}
  \begin{phonetics}{鳄鱼}{e4yu2}
    \definition[条]{s.}{jacaré | crocodilo}
  \end{phonetics}
\end{entry}

\begin{entry}{龠}{17}{⿕}[Kangxi 214]
  \begin{phonetics}{龠}{yue4}
    \definition{clas.}{yue, uma unidade de medida seca para grãos (= 0,5 decilitro);}
    \definition{s.}{uma flauta curta antiga}
  \end{phonetics}
\end{entry}

%%%%% EOF %%%%%

