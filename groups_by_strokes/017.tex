%%%
%%% 17画
%%%

\section*{17画}\addcontentsline{toc}{section}{17画}

\begin{Entry}{戴}{17}{⼽}
  \begin{Phonetics}{戴}{dai4}[][HSK 4]
    \definition*{s.}{Sobrenome Dai}
    \definition{v.}{usar/vestir (óculos, gravata, relógio de pulso, luvas); colocar objetos em sua cabeça, rosto, pescoço, peito, braços etc. | honrar; respeitar;}
  \end{Phonetics}
\end{Entry}

\begin{Entry}{擦}{17}{⼿}
  \begin{Phonetics}{擦}{ca1}[][HSK 4]
    \definition{v.}{enxugar; esfregar; apagar; limpar; limpar esfregando com um pano, toalha de mão, etc. | espalhar sobre; colocar sobre | passar raspando | ralar (em pedaços); ralar frutas em um ralador para fazer fios finos}
  \end{Phonetics}
\end{Entry}

\begin{Entry}{擦拭}{17,9}{⼿、⼿}
  \begin{Phonetics}{擦拭}{ca1shi4}
    \definition{v.}{limpar com um pano}
  \end{Phonetics}
\end{Entry}

\begin{Entry}{癌}{17}{⽧}
  \begin{Phonetics}{癌}{ai2}[][HSK 7-9]
    \definition{s.}{câncer; carcinoma; tumor maligno}
  \end{Phonetics}
\end{Entry}

\begin{Entry}{癌症}{17,10}{⽧、⽧}
  \begin{Phonetics}{癌症}{ai2zheng4}[][HSK 7-9]
    \definition[种]{s.}{câncer; tumores malignos no corpo}
  \end{Phonetics}
\end{Entry}

\begin{Entry}{瞧}{17}{⽬}
  \begin{Phonetics}{瞧}{qiao2}[][HSK 5]
    \definition{v.}{ver; olhar | tratar; diagnosticar e tratar | ver; visitar; fazer uma visita}
  \end{Phonetics}
\end{Entry}

\begin{Entry}{瞪}{17}{⽬}
  \begin{Phonetics}{瞪}{deng4}[][HSK 7-9]
    \definition{v.}{abrir bem os olhos; encarar; olhar fixamente com os olhos bem abertos; expressar insatisfação}
  \end{Phonetics}
\end{Entry}

\begin{Entry}{窾}{17}{⽳}
  \begin{Phonetics}{窾}{cuan4}
    \definition{adj.}{vazio | seco | destituído; pobre}
    \definition{s.}{buraco | lei}
    \definition{v.}{esconder}
  \end{Phonetics}
  \begin{Phonetics}{窾}{kuan3}
    \definition{adj.}{oco}
    \definition{s.}{rachadura; cavidade | (onomatopéia) água batendo na rocha}
    \definition{v.}{escavar um buraco}
  \end{Phonetics}
\end{Entry}

\begin{Entry}{簇}{17}{⽵}
  \begin{Phonetics}{簇}{cu4}
    \definition{clas.}{aglomerado; grupo; usado para pessoas ou coisas que se reúnem em grupos ou pilhas}
    \definition{s.}{pilha; aglomerado; buquê}
    \definition{v.}{aglomerar-se; formar um aglomerado; empilhar}
  \end{Phonetics}
\end{Entry}

\begin{Entry}{簇拥}{17,8}{⽵、⼿}
  \begin{Phonetics}{簇拥}{cu4yong1}[][HSK 7-9]
    \definition{v.}{aglomerar-se em volta de  | escoltar}
  \end{Phonetics}
\end{Entry}

\begin{Entry}{糟}{17}{⽶}
  \begin{Phonetics}{糟}{zao1}[][HSK 5]
    \definition{adj.}{pobre; apodrecido; deteriorado | estragado; em uma bagunça; em um estado miserável (terrível) | (situação ou circunstância) ruim; desfavorável}
    \definition{s.}{resíduos de destilação de bebidas alcoólicas; resíduos do processo de fermentação do vinho}
    \definition{v.}{marinar alimentos em vinho ou mosto | desperdiçar; estragar; destruir}
  \end{Phonetics}
\end{Entry}

\begin{Entry}{糟糕}{17,16}{⽶、⽶}
  \begin{Phonetics}{糟糕}{zao1gao1}[][HSK 5]
    \definition{adj.}{(corpo, situação, etc.) muito ruim, péssimo}
    \definition{interj.}{que terrível; que má sorte; muito ruim}
  \end{Phonetics}
\end{Entry}

\begin{Entry}{繁}{17}{⽷}
  \begin{Phonetics}{繁}{fan2}
    \definition{adj.}{em grande número; numerosos; múltiplos (oposto a 简) | em grande número; numerosos; complexos; complicado}
    \definition{v.}{propagar; multiplicar}
  \seealsoref{简}{jian3}
  \end{Phonetics}
\end{Entry}

\begin{Entry}{繁华}{17,6}{⽷、⼗}
  \begin{Phonetics}{繁华}{fan2hua2}[][HSK 7-9]
    \definition{adj.}{ocupado; agitado; próspero; florescente; (cidade, mercado de rua) movimentado e próspero}
  \end{Phonetics}
\end{Entry}

\begin{Entry}{繁忙}{17,6}{⽷、⼼}
  \begin{Phonetics}{繁忙}{fan2mang2}[][HSK 7-9]
    \definition{adj.}{ocupado; agitado; muita coisa para fazer, pouco tempo livre}
  \end{Phonetics}
\end{Entry}

\begin{Entry}{繁体字}{17,7,6}{⽷、⼈、⼦}
  \begin{Phonetics}{繁体字}{fan2ti3zi4}[][HSK 7-9]
    \definition{s.}{forma complexa tradicional de um caractere chinês simplificado; caracteres chineses com mais traços que foram substituídos por caracteres simplificados}
  \end{Phonetics}
\end{Entry}

\begin{Entry}{繁花}{17,7}{⽷、⾋}
  \begin{Phonetics}{繁花}{fan2hua1}
    \definition{s.}{Literário: flores desabrochadas; flores de cores diferentes; flores exuberantes; uma massa de flores}
  \end{Phonetics}
\end{Entry}

\begin{Entry}{繁荣}{17,9}{⽷、⾋}
  \begin{Phonetics}{繁荣}{fan2rong2}[][HSK 5]
    \definition{adj.}{florescente; próspero}
    \definition{v.}{promover; prosperar}
  \end{Phonetics}
\end{Entry}

\begin{Entry}{繁重}{17,9}{⽷、⾥}
  \begin{Phonetics}{繁重}{fan2zhong4}[][HSK 7-9]
    \definition{adj.}{pesado; oneroso; árduo; penoso; (trabalho, tarefas) muitas e pesadas}[我每天都有繁重的工作。===Tenho uma carga de trabalho pesada todos os dias.]
  \end{Phonetics}
\end{Entry}

\begin{Entry}{繁殖}{17,12}{⽷、⽍}
  \begin{Phonetics}{繁殖}{fan2zhi2}[][HSK 6]
    \definition{v.}{criar; reproduzir; propagar; multiplicar; os organismos produzem novos indivíduos}
  \end{Phonetics}
\end{Entry}

\begin{Entry}{藏}{17}{⾋}
  \begin{Phonetics}{藏}{cang2}[][HSK 6]
    \definition*{s.}{Sobrenome Cang}
    \definition{v.}{esconder; ocultar; esconder da vista | armazenar; coletar; colocar de lado}
  \end{Phonetics}
  \begin{Phonetics}{藏}{zang4}
    \definition*{s.}{Escrituras budistas ou taoístas; um termo geral para clássicos budistas ou taoístas | Região Autônoma do Tibete, 西藏}
    \definition{s.}{depósito; local de armazenamento; armazém; local onde grandes quantidades de coisas são armazenadas | os tibetanos, 藏族; grupo étnico Zang (ou tibetano)}
  \seealsoref{西藏}{xi1zang4}
  \seealsoref{藏族}{zang4zu2}
  \end{Phonetics}
\end{Entry}

\begin{Entry}{藏身}{17,7}{⾋、⾝}
  \begin{Phonetics}{藏身}{cang2shen1}[][HSK 7-9]
    \definition{v.}{esconder-se; esconder | abrigar-se; estabelecer-se}
  \end{Phonetics}
\end{Entry}

\begin{Entry}{藏品}{17,9}{⾋、⼝}
  \begin{Phonetics}{藏品}{cang2pin3}[][HSK 7-9]
    \definition[件]{s.}{artigos coletados; coleção | item de colecionador | peça de museu | objeto precioso}
  \end{Phonetics}
\end{Entry}

\begin{Entry}{藏匿}{17,10}{⾋、⼖}
  \begin{Phonetics}{藏匿}{cang2ni4}[][HSK 7-9]
    \definition{v.}{esconder; esconder-se | abrigar}
  \end{Phonetics}
\end{Entry}

\begin{Entry}{藏族}{17,11}{⾋、⽅}
  \begin{Phonetics}{藏族}{zang4zu2}
    \definition*{s.}{Etnia Zang (ou tibetana); Os Zangs (ou tibetanos) , distribuídos pela Região Autônoma do Tibete e pelas províncias de Qinghai, Sichuan, Gansu e Yunnan}
  \end{Phonetics}
\end{Entry}

\begin{Entry}{螺}{17}{⾍}
  \begin{Phonetics}{螺}{luo2}
    \definition{s.}{concha em espiral | caracol | búzio}
  \end{Phonetics}
\end{Entry}

\begin{Entry}{螺丝}{17,5}{⾍、⼀}
  \begin{Phonetics}{螺丝}{luo2si1}
    \definition{s.}{parafuso}
  \end{Phonetics}
\end{Entry}

\begin{Entry}{豁}{17}{⾕}
  \begin{Phonetics}{豁}{hua2}
    \definition{v.}{jogar o jogo de adivinhação de dedos chinês (Huoquan); o mesmo que 划拳}[豁拳规则很简单。===As regras do Huoquan são muito simples.]
  \seealsoref{划拳}{hua2quan2}
  \end{Phonetics}
  \begin{Phonetics}{豁}{huo1}[][HSK 7-9]
    \definition{v.}{cortar; quebrar; rachar; dividir | sacrificar; desistir; pagar o preço cruelmente}
  \end{Phonetics}
  \begin{Phonetics}{豁}{huo4}
    \definition{adj.}{claro; aberto; de mente aberta; generoso}
    \definition{v.}{isentar; remeter}
  \end{Phonetics}
\end{Entry}

\begin{Entry}{豁出去}{17,5,5}{⾕、⼐、⼛}
  \begin{Phonetics}{豁出去}{huo1/chu5qu4}[][HSK 7-9]
    \definition{v.+compl.}{seguir em frente independentemente; pronto para arriscar tudo}[她豁出去所有,去追逐梦想。===Ela arriscou tudo para perseguir seu sonho.]
  \end{Phonetics}
\end{Entry}

\begin{Entry}{豁达}{17,6}{⾕、⾡}
  \begin{Phonetics}{豁达}{huo4da2}[][HSK 7-9]
    \definition{adj.}{otimista; de mente aberta; aberto e claro}
  \end{Phonetics}
\end{Entry}

\begin{Entry}{赢}{17}{⾙}
  \begin{Phonetics}{赢}{ying2}[][HSK 3]
    \definition{v.}{vencer; derrotar | ganhar; lucrar}
  \end{Phonetics}
\end{Entry}

\begin{Entry}{赢得}{17,11}{⾙、⼻}
  \begin{Phonetics}{赢得}{ying2 de2}[][HSK 4]
    \definition{v.}{ganhar; obter; conquistar; assegurar; garantir}
  \end{Phonetics}
\end{Entry}

\begin{Entry}{辫}{17}{⾟}
  \begin{Phonetics}{辫}{bian4}
    \definition{s.}{trança; rabo de cavalo | para coisas como uma trança}
  \end{Phonetics}
\end{Entry}

\begin{Entry}{辫子}{17,3}{⾟、⼦}
  \begin{Phonetics}{辫子}{bian4zi5}[][HSK 7-9]
    \definition[条,根,种]{s.}{trança; rabo de cavalo; uma mecha de cabelo presa reta ou trançada em seções | Metafórico: erro; deficiência; fraqueza | coisas parecidas com tranças}
  \end{Phonetics}
\end{Entry}

\begin{Entry}{邉}{17}{⾡}
  \begin{Phonetics}{邉}{bian1}
    \variantof{边}
  \end{Phonetics}
\end{Entry}

\begin{Entry}{霜}{17}{⾬}
  \begin{Phonetics}{霜}{shuang1}
    \definition{s.}{geada | pó branco ou creme espalhado por uma superfície | glacê | creme de pele}
  \end{Phonetics}
\end{Entry}

\begin{Entry}{鳄}{17}{⿂}
  \begin{Phonetics}{鳄}{e4}
    \definition{s.}{crocodilo;  jacaré}
  \end{Phonetics}
\end{Entry}

\begin{Entry}{鳄鱼}{17,8}{⿂、⿂}
  \begin{Phonetics}{鳄鱼}{e4yu2}[][HSK 7-9]
    \definition[只,群,些]{s.}{crocodilo; jacaré}
  \end{Phonetics}
\end{Entry}

\begin{Entry}{龠}{17}{⿕}[Kangxi 214]
  \begin{Phonetics}{龠}{yue4}
    \definition{clas.}{yue, uma unidade de medida seca para grãos (= 0,5 decilitro);}
    \definition{s.}{uma flauta curta antiga}
  \end{Phonetics}
\end{Entry}

%%%%% EOF %%%%%

