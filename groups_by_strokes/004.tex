%%%
%%% 4画
%%%

\section*{4画}\addcontentsline{toc}{section}{4画}

\begin{entry}{不}{4}[Radical ⼀]
  \begin{phonetics}{不}{bu2}[(antes de quarto tom)][HSK 1]
    \definition{adv.}{não}
    \definition{pref.}{prefixo negativo}
  \end{phonetics}
  \begin{phonetics}{不}{bu4}[][HSK 1]
    \definition{adv.}{não}
    \definition{pref.}{prefixo negativo}
  \end{phonetics}
  \begin{phonetics}{不}{bu5}[][HSK 1]
    \definition{adv.}{não (em expressões ``v.+不+v.'')}
  \end{phonetics}
\end{entry}

\begin{entry}{不一会儿}{4,1,6,2}[Radicais ⼀、⼀、⼈、⼉]
  \begin{phonetics}{不一会儿}{bu4 yi2 hui4r5}[][HSK 2]
    \definition{expr.}{em um momento | em pouco tempo |em breve}
  \end{phonetics}
\end{entry}

\begin{entry}{不一定}{4,1,8}[Radicais ⼀、⼀、⼧]
  \begin{phonetics}{不一定}{bu4 yi2 ding4}[][HSK 2]
    \definition{adv.}{talvez | incerto | não tenho certeza | não necessariamente}
  \end{phonetics}
\end{entry}

\begin{entry}{不久}{4,3}[Radicais ⼀、⼃]
  \begin{phonetics}{不久}{bu4 jiu3}[][HSK 2]
    \definition{adj.}{em breve | futuro próximo | logo depois | não muito depois | não muito tempo (antes ou depois de algo)}
  \end{phonetics}
\end{entry}

\begin{entry}{不大}{4,3}[Radicais ⼀、⼤]
  \begin{phonetics}{不大}{bu2 da4}[][HSK 1]
    \definition{adv.}{não muito | não frequentemente | raramente |dificilmente | escassamente}
  \end{phonetics}
\end{entry}

\begin{entry}{不大离}{4,3,10}[Radicais ⼀、⼤、⼇]
  \begin{phonetics}{不大离}{bu2da4li2}
    \definition{adj.}{bem perto | quase certo | nada mal}
  \end{phonetics}
\end{entry}

\begin{entry}{不仅}{4,4}[Radicais ⼀、⼈]
  \begin{phonetics}{不仅}{bu4jin3}[][HSK 3]
    \definition{adv.}{não apenas (em número, quantidade ou extensão)}
    \definition{conj.}{não somente}
  \end{phonetics}
\end{entry}

\begin{entry}{不公}{4,4}[Radicais ⼀、⼋]
  \begin{phonetics}{不公}{bu4gong1}
    \definition{adj.}{injusto}
  \end{phonetics}
\end{entry}

\begin{entry}{不太}{4,4}[Radicais ⼀、⼤]
  \begin{phonetics}{不太}{bu2 tai4}[][HSK 2]
    \definition{adv.}{não bastante | não muito}
  \end{phonetics}
\end{entry}

\begin{entry}{不少}{4,4}[Radicais ⼀、⼩]
  \begin{phonetics}{不少}{bu4 shao3}[][HSK 2]
    \definition{adj.}{muitos | bastante | não poucos}
  \end{phonetics}
\end{entry}

\begin{entry}{不日}{4,4}[Radicais ⼀、⽇]
  \begin{phonetics}{不日}{bu2ri4}
    \definition{adv.}{em alguns dias}
  \end{phonetics}
\end{entry}

\begin{entry}{不止}{4,4}[Radicais ⼀、⽌]
  \begin{phonetics}{不止}{bu4zhi3}
    \definition{adv.}{incessantemente | sem fim | mais que | não limitado a}
  \end{phonetics}
\end{entry}

\begin{entry}{不可避免}{4,5,16,7}[Radicais ⼀、⼝、⾌、⼉]
  \begin{phonetics}{不可避免}{bu4ke3bi4mian3}
    \definition{adj./adv.}{inevitável}
  \end{phonetics}
\end{entry}

\begin{entry}{不对}{4,5}[Radicais ⼀、⼨]
  \begin{phonetics}{不对}{bu2 dui4}[][HSK 1]
    \definition{adj.}{incorreto | errado | anormal | estranho | estar em desacordo com | ser difícil de conviver}
  \end{phonetics}
\end{entry}

\begin{entry}{不必}{4,5}[Radicais ⼀、⼼]
  \begin{phonetics}{不必}{bu2 bi4}[][HSK 3]
    \definition{adv.}{não precisa | não tem que}
  \end{phonetics}
\end{entry}

\begin{entry}{不用}{4,5}[Radicais ⼀、⽤]
  \begin{phonetics}{不用}{bu2 yong4}[][HSK 1]
    \definition{v.}{não precisar}
    \seeref{甭}{beng2}
  \end{phonetics}
\end{entry}

\begin{entry}{不光}{4,6}[Radicais ⼀、⼉]
  \begin{phonetics}{不光}{bu4 guang1}[][HSK 3]
    \definition{adv.}{não é o único}
    \definition{conj.}{não somente}
  \end{phonetics}
\end{entry}

\begin{entry}{不同}{4,6}[Radicais ⼀、⼝]
  \begin{phonetics}{不同}{bu4 tong2}[][HSK 2]
    \definition{adj.}{diferente | distinto}
  \end{phonetics}
\end{entry}

\begin{entry}{不在乎}{4,6,5}[Radicais ⼀、⼟、⼃]
  \begin{phonetics}{不在乎}{bu2 zai4 hu1}[][HSK 4]
    \definition{v.}{não se importar; não dar a mínima; não dar atenção}
  \end{phonetics}
\end{entry}

\begin{entry}{不好意思}{4,6,13,9}[Radicais ⼀、⼥、⼼、⼼]
  \begin{phonetics}{不好意思}{bu4 hao3 yi4 si5}[][HSK 2]
    \definition{adj.}{pedir desculpas (por incomodar alguém) | sentir-se envergonhado | achar isso embaraçoso}
  \end{phonetics}
\end{entry}

\begin{entry}{不如}{4,6}[Radicais ⼀、⼥]
  \begin{phonetics}{不如}{bu4ru2}[][HSK 2]
    \definition{conj.}{em vez de | melhor que | seria melhor}
    \definition{v.}{ser inferior a | não ser igual a | não ser tão bom quanto | não poder fazer melhor que}
  \end{phonetics}
\end{entry}

\begin{entry}{不安}{4,6}[Radicais ⼀、⼧]
  \begin{phonetics}{不安}{bu4'an1}[][HSK 3]
    \definition{adj.}{inquieto | instável | intranquilo | pesaroso}
  \end{phonetics}
\end{entry}

\begin{entry}{不成话}{4,6,8}[Radicais ⼀、⼽、⾔]
  \begin{phonetics}{不成话}{bu4cheng2hua4}
    \definition{expr.}{sem razão | demasiado irracionável}
    \seeref{不是话}{bu2shi4hua4}
    \seeref{不像话}{bu2xiang4hua4}
  \end{phonetics}
\end{entry}

\begin{entry}{不行}{4,6}[Radicais ⼀、⾏]
  \begin{phonetics}{不行}{bu4 xing2}[][HSK 2]
    \definition{adj.}{não funciona | não é bom}
    \definition{adv.}{profundamente | terrivelmente | extremamente}
    \definition{v.}{não fazer | não ser permitido | estar fora de questão | estar à beira da morte}
  \end{phonetics}
\end{entry}

\begin{entry}{不论}{4,6}[Radicais ⼀、⾔]
  \begin{phonetics}{不论}{bu2 lun4}[][HSK 3]
    \definition{conj.}{não importa (o que, quem, como, etc.) | se \dots ou \dots}
  \end{phonetics}
\end{entry}

\begin{entry}{不论……也……}{4,6,3}[Radicais ⼀、⾔、⼄]
  \begin{phonetics}{不论……也……}{bu2lun4 ye3}
    \definition{conj.}{não apenas\dots, (o que, quem, como, etc.), \dots}
  \end{phonetics}
\end{entry}

\begin{entry}{不论……都……}{4,6,10}[Radicais ⼀、⾔、⾢]
  \begin{phonetics}{不论……都……}{bu2lun4 dou1}
    \definition{conj.}{não apenas\dots, (o que, quem, como, etc.), \dots}
  \end{phonetics}
\end{entry}

\begin{entry}{不过}{4,6}[Radicais ⼀、⾡]
  \begin{phonetics}{不过}{bu2guo4}[][HSK 2]
    \definition{conj.}{mas | contudo | no entanto}
  \end{phonetics}
\end{entry}

\begin{entry}{不但}{4,7}[Radicais ⼀、⼈]
  \begin{phonetics}{不但}{bu2 dan4}[][HSK 2]
    \definition{conj.}{não somente}
  \end{phonetics}
\end{entry}

\begin{entry}{不但……而且……}{4,7,6,5}[Radicais ⼀、⼈、⽽、⼀]
  \begin{phonetics}{不但……而且……}{bu2 dan4 er2qie3}[][HSK 2]
    \definition{conj.}{não só\dots mas também\dots}
  \end{phonetics}
\end{entry}

\begin{entry}{不到}{4,8}[Radicais ⼀、⼑]
  \begin{phonetics}{不到}{bu2dao4}
    \definition{adj.}{insuficiente}
    \definition{adv.}{menos que}
    \definition{v.}{não chegar}
  \end{phonetics}
\end{entry}

\begin{entry}{不注意}{4,8,13}[Radicais ⼀、⽔、⼼]
  \begin{phonetics}{不注意}{bu2zhu4yi4}
    \definition{adj.}{impensado | distraído}
    \definition{s.}{descuido | distração}
  \end{phonetics}
\end{entry}

\begin{entry}{不客气}{4,9,4}[Radicais ⼀、⼧、⽓]
  \begin{phonetics}{不客气}{bu2 ke4 qi5}[][HSK 1]
    \definition{adj.}{indelicado | rude | brusco}
    \definition{expr.}{de nada | não há de que | não mencione isso}
  \end{phonetics}
\end{entry}

\begin{entry}{不是……而是}{4,9,6,9}[Radicais ⼀、⽇、⽽、⽇]
  \begin{phonetics}{不是……而是}{bu4shi4 er2 shi4}
    \definition{conj.}{não somente\dots mas também\dots, expressam um relacionamento mais profundo e avançado em significado, mas as orações antes e depois são consistentes em expressar significados negativos e afirmativos, entretanto, a primeira metade da frase expressa negação, e a segunda metade expressa afirmação, e o significado das orações anteriores e seguintes não pode ser de um nível mais alto}
  \end{phonetics}
\end{entry}

\begin{entry}{不是话}{4,9,8}[Radicais ⼀、⽇、⾔]
  \begin{phonetics}{不是话}{bu2shi4hua4}
    \definition{expr.}{sem razão | demasiado irracionável}
    \seeref{不像话}{bu2xiang4hua4}
    \seeref{不成话}{bu4cheng2hua4}
  \end{phonetics}
\end{entry}

\begin{entry}{不要}{4,9}[Radicais ⼀、⾑]
  \begin{phonetics}{不要}{bu2 yao4}[][HSK 2]
    \definition{adv.}{nada de (pedir a alguém para não fazer) | não}
  \end{phonetics}
\end{entry}

\begin{entry}{不要紧}{4,9,10}[Radicais ⼀、⾑、⽷]
  \begin{phonetics}{不要紧}{bu2yao4jin3}[][HSK 4]
    \definition{adj.}{sem importância; sem seriedade; não problemático | não importa; não é um obstáculo | parece estar tudo bem, mas | à primeira vista, isso não parece atrapalhar}
  \end{phonetics}
\end{entry}

\begin{entry}{不够}{4,11}[Radicais ⼀、⼣]
  \begin{phonetics}{不够}{bu2 gou4}[][HSK 2]
    \definition{adv.}{insuficiente}
    \definition{v.}{não ser suficiente}
  \end{phonetics}
\end{entry}

\begin{entry}{不得不}{4,11,4}[Radicais ⼀、⼻、⼀]
  \begin{phonetics}{不得不}{bu4de2bu4}[][HSK 3]
    \definition{adv.}{tem que | não tem escolha a não ser}
  \end{phonetics}
\end{entry}

\begin{entry}{不断}{4,11}[Radicais ⼀、⽄]
  \begin{phonetics}{不断}{bu2duan4}[][HSK 3]
    \definition{adv.}{continuamente | sem fim}
  \end{phonetics}
\end{entry}

\begin{entry}{不然}{4,12}[Radicais ⼀、⽕]
  \begin{phonetics}{不然}{bu4ran2}[][HSK 4]
    \definition{adj.}{não é assim; não é o caso}
    \definition{conj.}{se não; caso contrário; indica outra consequência ou circunstância que teria ocorrido se não fosse}
  \end{phonetics}
\end{entry}

\begin{entry}{不像话}{4,13,8}[Radicais ⼀、⼈、⾔]
  \begin{phonetics}{不像话}{bu2xiang4hua4}
    \definition{expr.}{sem razão | demasiado irracionável}
    \seeref{不是话}{bu2shi4hua4}
    \seeref{不成话}{bu4cheng2hua4}
  \end{phonetics}
\end{entry}

\begin{entry}{不满}{4,13}[Radicais ⼀、⽔]
  \begin{phonetics}{不满}{bu4 man3}[][HSK 2]
    \definition{adj.}{ressentido | insatisfeito | descontente}
    \definition{v.}{estar descontente com |ser menor que}
  \end{phonetics}
\end{entry}

\begin{entry}{不错}{4,13}[Radicais ⼀、⾦]
  \begin{phonetics}{不错}{bu2 cuo4}[][HSK 2]
    \definition{adj.}{correto | não (é) mau | bastante bom | certo}
  \end{phonetics}
\end{entry}

\begin{entry}{不管}{4,14}[Radicais ⼀、⽵]
  \begin{phonetics}{不管}{bu4guan3}[][HSK 4]
    \definition{conj.}{não importa (o que, como, etc.); independentemente de; indica que, embora as condições ou circunstâncias tenham mudado, o resultado permanece o mesmo}
    \seeref{不管……都……}{bu4guan3 dou1}
    \seeref{不管……也……}{bu4guan3 ye3}
  \end{phonetics}
\end{entry}

\begin{entry}{不管……也……}{4,14,3}[Radicais ⼀、⽵、⼄]
  \begin{phonetics}{不管……也……}{bu4guan3 ye3}
    \definition{conj.}{não apenas\dots, (o que, quem, como, etc.), \dots}
  \end{phonetics}
\end{entry}

\begin{entry}{不管……都……}{4,14,10}[Radicais ⼀、⽵、⾢]
  \begin{phonetics}{不管……都……}{bu4guan3 dou1}
    \definition{conj.}{não apenas\dots, (o que, quem, como, etc.), \dots}
  \end{phonetics}
\end{entry}

\begin{entry}{专门}{4,3}[Radicais ⼀、⾨]
  \begin{phonetics}{专门}{zhuan1men2}[][HSK 3]
    \definition{adj.}{especializado}
    \definition{adv.}{especialmente}
  \end{phonetics}
\end{entry}

\begin{entry}{专业}{4,5}[Radicais ⼀、⼀]
  \begin{phonetics}{专业}{zhuan1ye4}[][HSK 3]
    \definition{adj.}{profissional; descreve uma pessoa que tem um alto nível ou rico conhecimento em uma determinada área}
    \definition[个,门]{s.}{profissão; linha especial; comércio especializado; unidades de negócios no departamento de produção | especialidade; disciplina; assunto especializado; campo especial de estudo; um departamento em uma faculdade ou escola profissional secundária}
  \end{phonetics}
\end{entry}

\begin{entry}{专业人士}{4,5,2,3}[Radicais ⼀、⼀、⼈、⼠]
  \begin{phonetics}{专业人士}{zhuan1ye4ren2shi4}
    \definition{s.}{profissional}
  \end{phonetics}
\end{entry}

\begin{entry}{专业人才}{4,5,2,3}[Radicais ⼀、⼀、⼈、⼿]
  \begin{phonetics}{专业人才}{zhuan1ye4ren2cai2}
    \definition{s.}{especialista (em uma área)}
  \end{phonetics}
\end{entry}

\begin{entry}{专业化}{4,5,4}[Radicais ⼀、⼀、⼔]
  \begin{phonetics}{专业化}{zhuan1ye4hua4}
    \definition{s.}{especialização}
  \end{phonetics}
\end{entry}

\begin{entry}{专业户}{4,5,4}[Radicais ⼀、⼀、⼾]
  \begin{phonetics}{专业户}{zhuan1ye4hu4}
    \definition{s.}{indústria caseira | empresa familiar produzindo um produto especial}
  \end{phonetics}
\end{entry}

\begin{entry}{专业性}{4,5,8}[Radicais ⼀、⼀、⼼]
  \begin{phonetics}{专业性}{zhuan1ye4xing4}
    \definition{s.}{profissionalismo | expertise}
  \end{phonetics}
\end{entry}

\begin{entry}{专业教育}{4,5,11,8}[Radicais ⼀、⼀、⽁、⾁]
  \begin{phonetics}{专业教育}{zhuan1ye4jiao4yu4}
    \definition{s.}{educação especializada | escola técnica}
  \end{phonetics}
\end{entry}

\begin{entry}{专家}{4,10}[Radicais ⼀、⼧]
  \begin{phonetics}{专家}{zhuan1jia1}[][HSK 3]
    \definition[个]{s.}{perito; especialista; proficiente; uma pessoa que é especialista em um determinado assunto; uma pessoa que é boa em uma determinada tecnologia}
  \end{phonetics}
\end{entry}

\begin{entry}{专题}{4,15}[Radicais ⼀、⾴]
  \begin{phonetics}{专题}{zhuan1ti2}[][HSK 3]
    \definition{s.}{assunto especial; tópico especial}
  \end{phonetics}
\end{entry}

\begin{entry}{中}{4}[Radical ⼁]
  \begin{phonetics}{中}{zhong1}[][HSK 1]
    \definition*{s.}{China}
    \definition*{s.}{sobrenome Zhong}
    \definition{s.}{centro | meio | médio | intermediário | média | meio caminho entre dois extremos | intermediador}
  \seealsoref{中国}{zhong1guo2}
  \end{phonetics}
  \begin{phonetics}{中}{zhong4}
    \definition{v.}{acertar | encaixar exatamente |ser atingido por | cair em | ser afetado por | sofrer | sustentar}
  \end{phonetics}
\end{entry}

\begin{entry}{中小学}{4,3,8}[Radicais ⼁、⼩、⼦]
  \begin{phonetics}{中小学}{zhong1 xiao3 xue2}[][HSK 2]
    \definition{s.}{escolas primárias e secundárias}
  \end{phonetics}
\end{entry}

\begin{entry}{中午}{4,4}[Radicais ⼁、⼗]
  \begin{phonetics}{中午}{zhong1wu3}[][HSK 1]
    \definition[个]{s.}{meio-dia}
  \end{phonetics}
\end{entry}

\begin{entry}{中心}{4,4}[Radicais ⼁、⼼]
  \begin{phonetics}{中心}{zhong1xin1}[][HSK 2]
    \definition[个]{s.}{núcleo | coração | meio | centro |chave}
  \end{phonetics}
\end{entry}

\begin{entry}{中文}{4,4}[Radicais ⼁、⽂]
  \begin{phonetics}{中文}{zhong1wen2}[][HSK 1]
    \definition{s.}{chinês, língua chinesa}
  \end{phonetics}
\end{entry}

\begin{entry}{中东}{4,5}[Radicais ⼁、⼀]
  \begin{phonetics}{中东}{zhong1dong1}
    \definition*{s.}{Oriente Médio}
  \end{phonetics}
\end{entry}

\begin{entry}{中央}{4,5}[Radicais ⼁、⼤]
  \begin{phonetics}{中央}{zhong1yang1}
    \definition{s.}{centro; meio | autoridades centrais; o mais alto órgão de governo de um país ou partido, etc.}
  \end{phonetics}
\end{entry}

\begin{entry}{中央情报局}{4,5,11,7,7}[Radicais ⼁、⼤、⼼、⼿、⼫]
  \begin{phonetics}{中央情报局}{zhong1yang1 qing2bao4ju2}
    \definition*{s.}{Agência Central de Inteligência dos EUA, CIA}
  \end{phonetics}
\end{entry}

\begin{entry}{中华民族}{4,6,5,11}[Radicais ⼁、⼗、⽒、⽅]
  \begin{phonetics}{中华民族}{zhong1 hua2 min2 zu2}[][HSK 3]
    \definition*{s.}{O Povo Chinês; termo geral para todos os grupos étnicos na China, incluindo 56 grupos étnicos, com uma longa história, esplêndida herança cultural e gloriosas tradições revolucionárias}
  \end{phonetics}
\end{entry}

\begin{entry}{中年}{4,6}[Radicais ⼁、⼲]
  \begin{phonetics}{中年}{zhong1 nian2}[][HSK 2]
    \definition{s.}{meia-idade}
  \end{phonetics}
\end{entry}

\begin{entry}{中级}{4,6}[Radicais ⼁、⽷]
  \begin{phonetics}{中级}{zhong1 ji2}[][HSK 2]
    \definition{adj.}{nível médio | nível intermediário}
  \end{phonetics}
\end{entry}

\begin{entry}{中医}{4,7}[Radicais ⼁、⼖]
  \begin{phonetics}{中医}{zhong1 yi1}[][HSK 2]
    \definition{s.}{ciência médica tradicional chinesa | médico de medicina tradicional chinesa | praticante de medicina chinesa}
  \end{phonetics}
\end{entry}

\begin{entry}{中间}{4,7}[Radicais ⼁、⾨]
  \begin{phonetics}{中间}{zhong1jian1}[][HSK 1]
    \definition{adv.}{central | centro | no meio}
  \end{phonetics}
\end{entry}

\begin{entry}{中国}{4,8}[Radicais ⼁、⼞]
  \begin{phonetics}{中国}{zhong1guo2}[][HSK 1]
    \definition*{s.}{China}
  \end{phonetics}
\end{entry}

\begin{entry}{中国人}{4,8,2}[Radicais ⼁、⼞、⼈]
  \begin{phonetics}{中国人}{zhong1guo2ren2}
    \definition{s.}{chinês | pessoa ou povo da China}
  \end{phonetics}
\end{entry}

\begin{entry}{中国城}{4,8,9}[Radicais ⼁、⼞、⼟]
  \begin{phonetics}{中国城}{zhong1guo2cheng2}
    \definition*{s.}{Bairro Chinês, \emph{Chinatown}}
    \seeref{唐人街}{tang2ren2 jie1}
  \end{phonetics}
\end{entry}

\begin{entry}{中国科学院}{4,8,9,8,9}[Radicais ⼁、⼞、⽲、⼦、⾩]
  \begin{phonetics}{中国科学院}{zhong1guo2 ke1xue2yuan4}
    \definition*{s.}{Academia Chinesa de Ciências}
  \end{phonetics}
\end{entry}

\begin{entry}{中国通}{4,8,10}[Radicais ⼁、⼞、⾡]
  \begin{phonetics}{中国通}{zhong1guo2tong1}
    \definition*{s.}{Conhecedor da China, especialista em tudo sobre a China}
  \end{phonetics}
\end{entry}

\begin{entry}{中学}{4,8}[Radicais ⼁、⼦]
  \begin{phonetics}{中学}{zhong1xue2}[][HSK 1]
    \definition[个]{s.}{escola ensino médio}
  \end{phonetics}
\end{entry}

\begin{entry}{中学生}{4,8,5}[Radicais ⼁、⼦、⽣]
  \begin{phonetics}{中学生}{zhong1xue2sheng1}[][HSK 1]
    \definition{s.}{aluno, estudante de escola ensino médio}
  \end{phonetics}
\end{entry}

\begin{entry}{中性}{4,8}[Radicais ⼁、⼼]
  \begin{phonetics}{中性}{zhong1xing4}
    \definition{adj.}{neutro}
  \end{phonetics}
\end{entry}

\begin{entry}{中询}{4,8}[Radicais ⼁、⾔]
  \begin{phonetics}{中询}{zhong1 xun2}
    \definition{adv.}{segunda dezena do mês | meio do mês | em meados do mês}
  \end{phonetics}
\end{entry}

\begin{entry}{中秋节}{4,9,5}[Radicais ⼁、⽲、⾋]
  \begin{phonetics}{中秋节}{zhong1qiu1jie2}
    \definition*{s.}{Festival do Meio-Outono | Festival do Bolo Lunar (15º dia do oitavo mês lunar)}
  \end{phonetics}
\end{entry}

\begin{entry}{中药}{4,9}[Radicais ⼁、⾋]
  \begin{phonetics}{中药}{zhong1yao4}
    \definition[服,种]{s.}{medicina tradicional chinesa}
  \end{phonetics}
\end{entry}

\begin{entry}{中部}{4,10}[Radicais ⼁、⾢]
  \begin{phonetics}{中部}{zhong1 bu4}[][HSK 3]
    \definition{s.}{parte do meio; região central; seção central; a área ou parte do meio | parte do meio; parte central; refere-se à parte central de um romance, filme ou obra televisiva}
  \end{phonetics}
\end{entry}

\begin{entry}{中情局}{4,11,7}[Radicais ⼁、⼼、⼫]
  \begin{phonetics}{中情局}{zhong1qing2ju2}
    \definition*{s.}{Agência Central de Inteligência dos EUA, CIA (abreviação de 中央情报局)}
    \seeref{中央情报局}{zhong1yang1 qing2bao4ju2}
  \end{phonetics}
\end{entry}

\begin{entry}{中意}{4,13}[Radicais ⼁、⼼]
  \begin{phonetics}{中意}{zhong4yi4}
    \definition{s.}{ser do seu agrado | começar a gostar muito de algo ou de alguém}
  \end{phonetics}
\end{entry}

\begin{entry}{中餐}{4,16}[Radicais ⼁、⾷]
  \begin{phonetics}{中餐}{zhong1 can1}[][HSK 2]
    \definition[分,顿]{s.}{comida chinesa | almoço}
  \end{phonetics}
\end{entry}

\begin{entry}{丰收}{4,6}[Radicais ⼁、⽁]
  \begin{phonetics}{丰收}{feng1shou1}
    \definition{s.}{colheita abundante}
  \end{phonetics}
\end{entry}

\begin{entry}{丰富}{4,12}[Radicais ⼁、⼧]
  \begin{phonetics}{丰富}{feng1fu4}[][HSK 3]
    \definition{adj.}{rico; abundante; pleno}
    \definition{v.}{enriquecer}
  \end{phonetics}
\end{entry}

\begin{entry}{为}{4}[Radical ⼂]
  \begin{phonetics}{为}{wei2}[][HSK 3]
    \definition*{s.}{sobrenome Wei}
    \definition{part.}{frequentemente usado com “何” para expressar dúvida}
    \definition{prep.}{como (na capacidade de) | por (na voz passiva)}
    \definition{suf.}{anexado a certos adjetivos monossilábicos, indicando grau ou alcançe | anexado a certos advérbios de grau para fortalecer o tom}
    \definition{v.}{fazer; agir | servir como; agir como; desempenhar o papel de | tornar-se; transformar-se em | ser; significar}
  \seealsoref{何}{he2}
  \end{phonetics}
  \begin{phonetics}{为}{wei4}[][HSK 2,3]
    \definition{prep.}{objeto da ação | indicando propósito | indicando razões | para; em direção a}
    \definition{v.}{apoiar; defender}
  \end{phonetics}
\end{entry}

\begin{entry}{为了}{4,2}[Radicais ⼂、⼅]
  \begin{phonetics}{为了}{wei4le5}[][HSK 3]
    \definition{conj.}{para; por causa de; a fim de}
  \end{phonetics}
\end{entry}

\begin{entry}{为什么}{4,4,3}[Radicais ⼂、⼈、⼃]
  \begin{phonetics}{为什么}{wei4shen2me5}[][HSK 2]
    \definition{adv.}{por que?}
  \end{phonetics}
\end{entry}

\begin{entry}{乌克兰}{4,7,5}[Radicais ⼃、⼗、⼋]
  \begin{phonetics}{乌克兰}{wu1ke4lan2}
    \definition*{s.}{Ucrânia}
  \end{phonetics}
\end{entry}

\begin{entry}{乌龟}{4,7}[Radicais ⼃、⿔]
  \begin{phonetics}{乌龟}{wu1gui1}
    \definition{s.}{tartaruga}
  \end{phonetics}
\end{entry}

\begin{entry}{书}{4}[Radical ⼄]
  \begin{phonetics}{书}{shu1}[][HSK 1]
    \definition[本,册,部]{s.}{livro | carta | documento}
  \end{phonetics}
\end{entry}

\begin{entry}{书包}{4,5}[Radicais ⼄、⼓]
  \begin{phonetics}{书包}{shu1bao1}[][HSK 1]
    \definition[个,款]{s.}{mochila escolar}
  \end{phonetics}
\end{entry}

\begin{entry}{书记}{4,5}[Radicais ⼄、⾔]
  \begin{phonetics}{书记}{shu1ji5}
    \definition{s.}{secretário (chefe de um ramo de um partido socialista ou comunista) | atendente | balconista | escriturário}
  \end{phonetics}
\end{entry}

\begin{entry}{书店}{4,8}[Radicais ⼄、⼴]
  \begin{phonetics}{书店}{shu1dian4}[][HSK 1]
    \definition[家]{s.}{livraria}
  \end{phonetics}
\end{entry}

\begin{entry}{书架}{4,9}[Radicais ⼄、⽊]
  \begin{phonetics}{书架}{shu1jia4}[][HSK 3]
    \definition[个]{s.}{estante de livros}
  \end{phonetics}
\end{entry}

\begin{entry}{云}{4}[Radical ⼆]
  \begin{phonetics}{云}{yun2}[][HSK 2]
    \definition*{s.}{sobrenome Yun}
    \definition[朵]{s.}{nuvem}
  \end{phonetics}
\end{entry}

\begin{entry}{云云}{4,4}[Radicais ⼆、⼆]
  \begin{phonetics}{云云}{yun2yun2}
    \definition{adv.}{e assim por diante | assim e assim}
  \end{phonetics}
\end{entry}

\begin{entry}{云南}{4,9}[Radicais ⼆、⼗]
  \begin{phonetics}{云南}{yun2nan2}
    \definition*{s.}{Yunnan}
  \end{phonetics}
\end{entry}

\begin{entry}{云端}{4,14}[Radicais ⼆、⽴]
  \begin{phonetics}{云端}{yun2duan1}
    \definition{s.}{alto nas nuvens | (computação) a nuvem}
  \end{phonetics}
\end{entry}

\begin{entry}{互}{4}[Radical ⼆]
  \begin{phonetics}{互}{hu4}
    \definition{adj.}{mútuo | recíproco}
  \end{phonetics}
\end{entry}

\begin{entry}{互动}{4,6}[Radicais ⼆、⼒]
  \begin{phonetics}{互动}{hu4dong4}
    \definition{s.}{interativo}
    \definition{v.}{interagir}
  \end{phonetics}
\end{entry}

\begin{entry}{互利}{4,7}[Radicais ⼆、⼑]
  \begin{phonetics}{互利}{hu4li4}
    \definition{s.}{benefício mútuo}
  \end{phonetics}
\end{entry}

\begin{entry}{互相}{4,9}[Radicais ⼆、⽬]
  \begin{phonetics}{互相}{hu4xiang1}[][HSK 3]
    \definition{adv.}{mutuamente; um ao outro}
  \end{phonetics}
\end{entry}

\begin{entry}{互联网}{4,12,6}[Radicais ⼆、⽿、⽹]
  \begin{phonetics}{互联网}{hu4lian2wang3}[][HSK 3]
    \definition{s.}{\emph{Internet}}
  \seealsoref{网际网路}{wang3ji4wang3lu4}
  \seealsoref{网际网络}{wang3ji4wang3luo4}
  \seealsoref{网路}{wang3lu4}
  \end{phonetics}
\end{entry}

\begin{entry}{五}{4}[Radical ⼆]
  \begin{phonetics}{五}{wu3}[][HSK 1]
    \definition{num.}{cinco; 5}
  \end{phonetics}
\end{entry}

\begin{entry}{五五}{4,4}[Radicais ⼆、⼆]
  \begin{phonetics}{五五}{wu3wu3}
    \definition{num.}{50-50}
    \definition{s.}{igual (partilha, parceria, etc.)}
  \end{phonetics}
\end{entry}

\begin{entry}{五体投地}{4,7,7,6}[Radicais ⼆、⼈、⼿、⼟]
  \begin{phonetics}{五体投地}{wu3ti3tou2di4}
    \definition{expr.}{prostrar-se em admiração | adular alguém}
  \end{phonetics}
\end{entry}

\begin{entry}{五颜六色}{4,15,4,6}[Radicais ⼆、⾴、⼋、⾊]
  \begin{phonetics}{五颜六色}{wu3 yan2 liu4 se4}[][HSK 4]
    \definition{adj.}{todas as cores sob o sol; multicolorido; colorido}
  \end{phonetics}
\end{entry}

\begin{entry}{井}{4}[Radical ⼆]
  \begin{phonetics}{井}{jing3}
    \definition{adj.}{puro | ordenado}
    \definition[口]{s.}{poço}
  \end{phonetics}
\end{entry}

\begin{entry}{什么}{4,3}[Radicais ⼈、⼃]
  \begin{phonetics}{什么}{shen2me5}[][HSK 1]
    \definition{pron.}{que? | o que?}
    \definition{pron.}{algo | qualquer coisa}
  \end{phonetics}
\end{entry}

\begin{entry}{什么时候}{4,3,7,10}[Radicais ⼈、⼃、⽇、⼈]
  \begin{phonetics}{什么时候}{shen2me5shi2hou5}
    \definition{adv.}{quando? | a que horas?}
  \end{phonetics}
\end{entry}

\begin{entry}{什么样}{4,3,10}[Radicais ⼈、⼃、⽊]
  \begin{phonetics}{什么样}{shen2 me5 yang4}[][HSK 2]
    \definition{pron.}{que tipo? | o quê? | que tipo?}
  \end{phonetics}
\end{entry}

\begin{entry}{仅}{4}[Radical ⼈]
  \begin{phonetics}{仅}{jin3}[][HSK 3]
    \definition{adv.}{somente; meramente; por muito pouco}
  \end{phonetics}
\end{entry}

\begin{entry}{仅仅}{4,4}[Radicais ⼈、⼈]
  \begin{phonetics}{仅仅}{jin3 jin3}[][HSK 3]
    \definition{adv.}{somente; meramente; por muito pouco}
  \end{phonetics}
\end{entry}

\begin{entry}{今天}{4,4}[Radicais ⼈、⼤]
  \begin{phonetics}{今天}{jin1tian1}[][HSK 1]
    \definition{adv.}{hoje | no presente | agora}
  \end{phonetics}
\end{entry}

\begin{entry}{今后}{4,6}[Radicais ⼈、⼝]
  \begin{phonetics}{今后}{jin1 hou4}[][HSK 2]
    \definition{s.}{de agora em diante | daqui em diante | no futuro}
  \end{phonetics}
\end{entry}

\begin{entry}{今年}{4,6}[Radicais ⼈、⼲]
  \begin{phonetics}{今年}{jin1 nian2}[][HSK 1]
    \definition{adv.}{este ano}
  \end{phonetics}
\end{entry}

\begin{entry}{介绍}{4,8}[Radicais ⼈、⽷]
  \begin{phonetics}{介绍}{jie4shao4}[][HSK 1]
    \definition{s.}{introdução | apresentação}
    \definition{v.}{fazer uma apresentação | apresentar (alguém para alguém) | apresentar (alguém para um emprego, etc.)}
  \end{phonetics}
\end{entry}

\begin{entry}{仍}{4}[Radical ⼈]
  \begin{phonetics}{仍}{reng2}[][HSK 3]
    \definition*{s.}{sobrenome Reng}
    \definition{adv.}{ainda}
    \definition{conj.}{por isso}
    \definition{v.}{permanecer}
  \end{phonetics}
\end{entry}

\begin{entry}{仍然}{4,12}[Radicais ⼈、⽕]
  \begin{phonetics}{仍然}{reng2ran2}[][HSK 3]
    \definition{adv.}{ainda; como antes}
  \end{phonetics}
\end{entry}

\begin{entry}{从}{4}[Radical ⼈]
  \begin{phonetics}{从}{cong2}[][HSK 1]
    \definition*{s.}{sobrenome Cong}
    \definition{prep.}{de | desde | a partir de}
  \end{phonetics}
\end{entry}

\begin{entry}{从小}{4,3}[Radicais ⼈、⼩]
  \begin{phonetics}{从小}{cong2 xiao3}[][HSK 2]
    \definition{adv.}{desde a infância | desde muito jovem | quando criança}
  \end{phonetics}
\end{entry}

\begin{entry}{从不}{4,4}[Radicais ⼈、⼀]
  \begin{phonetics}{从不}{cong2bu4}
    \definition{adv.}{nunca}
  \end{phonetics}
\end{entry}

\begin{entry}{从未}{4,5}[Radicais ⼈、⽊]
  \begin{phonetics}{从未}{cong2wei4}
    \definition{adv.}{nunca}
  \end{phonetics}
\end{entry}

\begin{entry}{从此}{4,6}[Radicais ⼈、⽌]
  \begin{phonetics}{从此}{cong2ci3}[][HSK 4]
    \definition{conj.}{doravante; portanto; a partir deste momento; de agora em diante; a partir de então}
  \end{phonetics}
\end{entry}

\begin{entry}{从而}{4,6}[Radicais ⼈、⽽]
  \begin{phonetics}{从而}{cong2'er2}
    \definition{conj.}{assim | desse modo}
  \end{phonetics}
\end{entry}

\begin{entry}{从来}{4,7}[Radicais ⼈、⽊]
  \begin{phonetics}{从来}{cong2lai2}[][HSK 3]
    \definition{adv.}{sempre; o tempo todo; em todos os momentos}
  \end{phonetics}
\end{entry}

\begin{entry}{从事}{4,8}[Radicais ⼈、⼅]
  \begin{phonetics}{从事}{cong2shi4}[][HSK 3]
    \definition{v.}{trabalhar; empreender; empenhar-se em; envolver-se em | lidar com; manusear}
  \end{phonetics}
\end{entry}

\begin{entry}{从前}{4,9}[Radicais ⼈、⼑]
  \begin{phonetics}{从前}{cong2qian2}[][HSK 3]
    \definition{s.}{antes; antigamente; no passado | era uma vez; há muito tempo atrás}
  \end{phonetics}
\end{entry}

\begin{entry}{以上}{4,3}[Radicais ⼈、⼀]
  \begin{phonetics}{以上}{yi3 shang4}[][HSK 2]
    \definition{s.}{mais que | sobre | acima | o acima | o precedente | o acima mencionado}
  \end{phonetics}
\end{entry}

\begin{entry}{以下}{4,3}[Radicais ⼈、⼀]
  \begin{phonetics}{以下}{yi3 xia4}[][HSK 2]
    \definition[所]{s.}{abaixo | sob | seguinte}
  \end{phonetics}
\end{entry}

\begin{entry}{以及}{4,3}[Radicais ⼈、⼃]
  \begin{phonetics}{以及}{yi3ji2}
    \definition{conj.}{assim como | juntamente como}
  \end{phonetics}
\end{entry}

\begin{entry}{以为}{4,4}[Radicais ⼈、⼂]
  \begin{phonetics}{以为}{yi3wei2}[][HSK 2]
    \definition{v.}{pensar, ou seja, considerar que\dots (geralmente há uma implicação de que a noção está errada --- exceto ao expressar a própria opnião atual)}
  \end{phonetics}
\end{entry}

\begin{entry}{以外}{4,5}[Radicais ⼈、⼣]
  \begin{phonetics}{以外}{yi3 wai4}[][HSK 2]
    \definition{s.}{além | exceto | fora | diferente de}
  \end{phonetics}
\end{entry}

\begin{entry}{以后}{4,6}[Radicais ⼈、⼝]
  \begin{phonetics}{以后}{yi3 hou4}[][HSK 2]
    \definition{adv.}{depois de | depois | após}
  \end{phonetics}
\end{entry}

\begin{entry}{以此}{4,6}[Radicais ⼈、⽌]
  \begin{phonetics}{以此}{yi3ci3}
    \definition{adv.}{devido a esta | deste modo | por isso | com isso}
  \end{phonetics}
\end{entry}

\begin{entry}{以至}{4,6}[Radicais ⼈、⾄]
  \begin{phonetics}{以至}{yi3zhi4}
    \definition{adv.}{até}
    \definition{conj.}{a tal ponto que\dots}
  \seealsoref{以至于}{yi3zhi4yu2}
  \end{phonetics}
\end{entry}

\begin{entry}{以至于}{4,6,3}[Radicais ⼈、⾄、⼆]
  \begin{phonetics}{以至于}{yi3zhi4yu2}
    \definition{adv.}{até}
    \definition{conj.}{na medida em que\dots}
  \seealsoref{以至}{yi3zhi4}
  \end{phonetics}
\end{entry}

\begin{entry}{以色列}{4,6,6}[Radicais ⼈、⾊、⼑]
  \begin{phonetics}{以色列}{yi3se4lie4}
    \definition*{s.}{Israel}
  \end{phonetics}
\end{entry}

\begin{entry}{以免}{4,7}[Radicais ⼈、⼉]
  \begin{phonetics}{以免}{yi3mian3}
    \definition{conj.}{para evitar isso}
  \end{phonetics}
\end{entry}

\begin{entry}{以来}{4,7}[Radicais ⼈、⽊]
  \begin{phonetics}{以来}{yi3lai2}[][HSK 3]
    \definition{prep.}{desde (um evento anterior); indica um período de um certo tempo no passado até o presente}
  \end{phonetics}
\end{entry}

\begin{entry}{以求}{4,7}[Radicais ⼈、⽔]
  \begin{phonetics}{以求}{yi3qiu2}
    \definition{conj.}{a fim de}
  \end{phonetics}
\end{entry}

\begin{entry}{以便}{4,9}[Radicais ⼈、⼈]
  \begin{phonetics}{以便}{yi3bian4}
    \definition{conj.}{a fim de | para que | assim como}
  \end{phonetics}
\end{entry}

\begin{entry}{以前}{4,9}[Radicais ⼈、⼑]
  \begin{phonetics}{以前}{yi3qian2}[][HSK 2]
    \definition{adv.}{antes de | antes}
  \end{phonetics}
\end{entry}

\begin{entry}{以期}{4,12}[Radicais ⼈、⽉]
  \begin{phonetics}{以期}{yi3qi1}
    \definition{v.}{tentando | esperando | esperando por}
  \end{phonetics}
\end{entry}

\begin{entry}{元}{4}[Radical ⼉]
  \begin{phonetics}{元}{yuan2}[][HSK 1]
    \definition*{s.}{sobrenome Yuan | Dinastia Yuan (1279-1368)}
    \definition{clas.}{unidade monetária da China}
  \end{phonetics}
\end{entry}

\begin{entry}{元气}{4,4}[Radicais ⼉、⽓]
  \begin{phonetics}{元气}{yuan2qi4}
    \definition{s.}{força | vigor | vitalidade | energial vital}
  \end{phonetics}
\end{entry}

\begin{entry}{元旦}{4,5}[Radicais ⼉、⽇]
  \begin{phonetics}{元旦}{yuan2dan4}
    \definition*{s.}{Dia de Ano Novo (1 de janeiro)}
  \end{phonetics}
\end{entry}

\begin{entry}{元来}{4,7}[Radicais ⼉、⽊]
  \begin{phonetics}{元来}{yuan2lai2}
    \variantof{原来}
  \end{phonetics}
\end{entry}

\begin{entry}{元夜}{4,8}[Radicais ⼉、⼣]
  \begin{phonetics}{元夜}{yuan2ye4}
    \definition*{s.}{Festival das Lanternas}
  \seealsoref{元宵}{yuan2xiao1}
  \seealsoref{元宵节}{yuan2xiao1jie2}
  \end{phonetics}
\end{entry}

\begin{entry}{元宵}{4,10}[Radicais ⼉、⼧]
  \begin{phonetics}{元宵}{yuan2xiao1}
    \definition*{s.}{Festival das Lanternas}
  \seealsoref{元宵节}{yuan2xiao1jie2}
  \seealsoref{元夜}{yuan2ye4}
  \end{phonetics}
\end{entry}

\begin{entry}{元宵节}{4,10,5}[Radicais ⼉、⼧、⾋]
  \begin{phonetics}{元宵节}{yuan2xiao1jie2}
    \definition*{s.}{Festival das Lanternas (15º~dia do primeiro mês lunar)}
  \seealsoref{元宵}{yuan2xiao1}
  \seealsoref{元夜}{yuan2ye4}
  \end{phonetics}
\end{entry}

\begin{entry}{公元}{4,4}[Radicais ⼋、⼉]
  \begin{phonetics}{公元}{gong1yuan2}[][HSK 4]
    \definition{s.}{D.C. (Depois de~Cristo); a era cristã; um método internacionalmente aceito de registro de datas, o ano lendário do nascimento de Jesus é 1 d.C., também conhecido como o primeiro ano da Era Comum, e é denotado por D.C.}
  \seealsoref{前}{qian2}
  \end{phonetics}
\end{entry}

\begin{entry}{公开}{4,4}[Radicais ⼋、⼶]
  \begin{phonetics}{公开}{gong1kai1}[][HSK 3]
    \definition{adj.}{aberto; público}
    \definition{v.}{tornar público}
  \end{phonetics}
\end{entry}

\begin{entry}{公斤}{4,4}[Radicais ⼋、⽄]
  \begin{phonetics}{公斤}{gong1jin1}[][HSK 2]
    \definition{clas.}{quilograma (kg)}
  \end{phonetics}
\end{entry}

\begin{entry}{公车}{4,4}[Radicais ⼋、⾞]
  \begin{phonetics}{公车}{gong1che1}
    \definition{s.}{abreviação de~公共汽车, ônibus}
    \seeref{公共汽车}{gong1gong4qi4che1}
  \seealsoref{公共}{gong1 gong4}
  \end{phonetics}
\end{entry}

\begin{entry}{公务员}{4,5,7}[Radicais ⼋、⼒、⼝]
  \begin{phonetics}{公务员}{gong1 wu4 yuan2}[][HSK 3]
    \definition[个,名]{s.}{funcionário público}
  \end{phonetics}
\end{entry}

\begin{entry}{公司}{4,5}[Radicais ⼋、⼝]
  \begin{phonetics}{公司}{gong1si1}[][HSK 2]
    \definition[家]{s.}{empresa | companhia | corporação | firma}
  \end{phonetics}
\end{entry}

\begin{entry}{公司治理}{4,5,8,11}[Radicais ⼋、⼝、⽔、⽟]
  \begin{phonetics}{公司治理}{gong1si1zhi4li3}
    \definition{s.}{governança corporativa}
  \end{phonetics}
\end{entry}

\begin{entry}{公布}{4,5}[Radicais ⼋、⼱]
  \begin{phonetics}{公布}{gong1bu4}[][HSK 3]
    \definition{v.}{promulgar; anunciar; publicar; tornar público}
  \end{phonetics}
\end{entry}

\begin{entry}{公平}{4,5}[Radicais ⼋、⼲]
  \begin{phonetics}{公平}{gong1ping2}[][HSK 2]
    \definition{adj.}{justo | imparcial | equitativo}
  \end{phonetics}
\end{entry}

\begin{entry}{公民}{4,5}[Radicais ⼋、⽒]
  \begin{phonetics}{公民}{gong1min2}[][HSK 3]
    \definition{s.}{cidadão; civil}
  \end{phonetics}
\end{entry}

\begin{entry}{公用电话}{4,5,5,8}[Radicais ⼋、⽤、⽥、⾔]
  \begin{phonetics}{公用电话}{gong1yong4dian4hua4}
    \definition[部]{s.}{telefone público}
  \end{phonetics}
\end{entry}

\begin{entry}{公交车}{4,6,4}[Radicais ⼋、⼇、⾞]
  \begin{phonetics}{公交车}{gong1 jiao1 che1}[][HSK 2]
    \definition[辆]{s.}{ônibus urbano | veículo de transporte público}
  \end{phonetics}
\end{entry}

\begin{entry}{公共}{4,6}[Radicais ⼋、⼋]
  \begin{phonetics}{公共}{gong1 gong4}[][HSK 3]
    \definition{adj.}{público; comum; comunal}
    \definition{s.}{ônibus}
  \seealsoref{公车}{gong1che1}
  \seealsoref{公共汽车}{gong1gong4qi4che1}
  \end{phonetics}
\end{entry}

\begin{entry}{公共汽车}{4,6,7,4}[Radicais ⼋、⼋、⽔、⾞]
  \begin{phonetics}{公共汽车}{gong1gong4qi4che1}[][HSK 2]
    \definition[辆,班]{s.}{ônibus}
    \seeref{公车}{gong1che1}
  \seealsoref{公共}{gong1 gong4}
  \end{phonetics}
\end{entry}

\begin{entry}{公克}{4,7}[Radicais ⼋、⼗]
  \begin{phonetics}{公克}{gong1ke4}
    \definition{s.}{grama (medida de peso)}
  \end{phonetics}
\end{entry}

\begin{entry}{公园}{4,7}[Radicais ⼋、⼞]
  \begin{phonetics}{公园}{gong1yuan2}[][HSK 2]
    \definition[座]{s.}{parque (para recreação pública)}
  \end{phonetics}
\end{entry}

\begin{entry}{公里}{4,7}[Radicais ⼋、⾥]
  \begin{phonetics}{公里}{gong1li3}[][HSK 2]
    \definition{s.}{quilômetro}
  \end{phonetics}
\end{entry}

\begin{entry}{公寓}{4,12}[Radicais ⼋、⼧]
  \begin{phonetics}{公寓}{gong1yu4}
    \definition[套]{s.}{prédio de apartamentos | pensão}
  \end{phonetics}
\end{entry}

\begin{entry}{公路}{4,13}[Radicais ⼋、⾜]
  \begin{phonetics}{公路}{gong1 lu4}[][HSK 2]
    \definition[条]{s.}{rodovia | via de trânsito | estrada | auto-estrada}
  \end{phonetics}
\end{entry}

\begin{entry}{六}{4}[Radical ⼋]
  \begin{phonetics}{六}{liu4}[][HSK 1]
    \definition{num.}{seis; 6}
  \end{phonetics}
\end{entry}

\begin{entry}{内}{4}[Radical ⼌]
  \begin{phonetics}{内}{nei4}[][HSK 3]
    \definition*{s.}{sobrenome Nei}
    \definition{adj.}{interno; interior}
    \definition{prep.}{dentro}
    \definition{s.}{interior; lado de dentro; parte de dentro | a esposa ou parentes dela}
  \end{phonetics}
\end{entry}

\begin{entry}{内心}{4,4}[Radicais ⼌、⼼]
  \begin{phonetics}{内心}{nei4 xin1}[][HSK 3]
    \definition{s.}{coração; interior; íntimo do ser}
  \end{phonetics}
\end{entry}

\begin{entry}{内存}{4,6}[Radicais ⼌、⼦]
  \begin{phonetics}{内存}{nei4cun2}
    \definition{s.}{armazenamento interno | memória do computador | RAM (\emph{random access memory})}
  \seealsoref{随机存取存储器}{sui2ji1cun2qu3cun2chu3qi4}
  \seealsoref{随机存取记忆体}{sui2ji1cun2qu3ji4yi4ti3}
  \end{phonetics}
\end{entry}

\begin{entry}{内省}{4,9}[Radicais ⼌、⽬]
  \begin{phonetics}{内省}{nei4xing3}
    \definition{s.}{introspecção}
    \definition{v.}{refletir sobre si mesmo}
  \end{phonetics}
\end{entry}

\begin{entry}{内科}{4,9}[Radicais ⼌、⽲]
  \begin{phonetics}{内科}{nei4ke1}[][HSK 4]
    \definition{s.}{medicina geral; clínica geral; clínica médica}
  \end{phonetics}
\end{entry}

\begin{entry}{内容}{4,10}[Radicais ⼌、⼧]
  \begin{phonetics}{内容}{nei4rong2}[][HSK 3]
    \definition[个]{s.}{conteúdo; substância}
  \end{phonetics}
\end{entry}

\begin{entry}{内部}{4,10}[Radicais ⼌、⾢]
  \begin{phonetics}{内部}{nei4bu4}[][HSK 4]
    \definition{s.}{interior; dentro; interno; dentro de um determinado intervalo}
  \end{phonetics}
\end{entry}

\begin{entry}{内燃机}{4,16,6}[Radicais ⼌、⽕、⽊]
  \begin{phonetics}{内燃机}{nei4ran2ji1}
    \definition{s.}{motor de combustão interna}
  \end{phonetics}
\end{entry}

\begin{entry}{凤凰}{4,11}[Radicais ⼏、⼏]
  \begin{phonetics}{凤凰}{feng4huang2}
    \definition{s.}{fênix}
  \end{phonetics}
\end{entry}

\begin{entry}{分}{4}[Radical ⼑]
  \begin{phonetics}{分}{fen1}[][HSK 1,2]
    \definition{s.}{parte ou subdivisão | fração | um décimo (de certas unidades) | unidade de comprimento equivalente a 0,33cm | minuto (unidade de tempo) | minuto (unidade de medida angular) | um ponto (em esportes e jogos) | 0,01 yuan (unidade de dinheiro)}
    \definition{v.}{dividir | separar | distribuir | atribuir | distinguir (bom e mau)}
  \end{phonetics}
  \begin{phonetics}{分}{fen4}[][HSK 2]
    \definition{s.}{parte | ingrediente | componente}
  \end{phonetics}
\end{entry}

\begin{entry}{分之}{4,3}[Radicais ⼑、⼂]
  \begin{phonetics}{分之}{fen1 zhi1}[][HSK 4]
    \definition{expr.}{indicando uma fração; formatação e leitura de frações, ou seja, partes de um total}[顾客减少了三分之一。 (O número de clientes caiu em um terço.)]
  \end{phonetics}
\end{entry}

\begin{entry}{分子}{4,3}[Radicais ⼑、⼦]
  \begin{phonetics}{分子}{fen1zi3}
    \definition{s.}{molécula | (matemática) numerador de uma fração}
  \end{phonetics}
  \begin{phonetics}{分子}{fen4zi3}
    \definition{s.}{membros de uma classe ou grupo | elementos políticos (como intelectuais ou extremistas)}
  \end{phonetics}
\end{entry}

\begin{entry}{分为}{4,4}[Radicais ⼑、⼂]
  \begin{phonetics}{分为}{fen1 wei2}[][HSK 4]
    \definition{v.}{subdividir; dividir algo em}
  \end{phonetics}
\end{entry}

\begin{entry}{分公司}{4,4,5}[Radicais ⼑、⼋、⼝]
  \begin{phonetics}{分公司}{fen1gong1si1}
    \definition{s.}{sucursal | filial de companhia}
  \end{phonetics}
\end{entry}

\begin{entry}{分开}{4,4}[Radicais ⼑、⼶]
  \begin{phonetics}{分开}{fen1 kai1}[][HSK 2]
    \definition{v.+compl.}{separar | dividir | desacoplar | desempacotar | quebrar | desmembrar | romper | desfazer | desvincular | distribuir | separar de (em) | dividir\dots de\dots | separar de}
  \end{phonetics}
\end{entry}

\begin{entry}{分手}{4,4}[Radicais ⼑、⼿]
  \begin{phonetics}{分手}{fen1shou3}[][HSK 4]
    \definition{v.+compl.}{separar; romper; terminar um relacionamento ou um casal | separar-se (de uma empresa); dizer adeus; despedir-se da família, dos amigos, etc.}
  \end{phonetics}
\end{entry}

\begin{entry}{分布}{4,5}[Radicais ⼑、⼱]
  \begin{phonetics}{分布}{fen1bu4}[][HSK 4]
    \definition{v.}{espalhar; distribuir; dispersar (em uma determinada área)}
  \end{phonetics}
\end{entry}

\begin{entry}{分别}{4,7}[Radicais ⼑、⼑]
  \begin{phonetics}{分别}{fen1bie2}[][HSK 3]
    \definition{adv.}{diferentemente; de maneiras diferentes}
    \definition{s.}{diferença}
    \definition{v.}{partir; deixar um ao outro | distinguir; diferenciar}
  \end{phonetics}
\end{entry}

\begin{entry}{分组}{4,8}[Radicais ⼑、⽷]
  \begin{phonetics}{分组}{fen1 zu3}[][HSK 3]
    \definition{v.}{agrupar; dividir em grupos}
  \end{phonetics}
\end{entry}

\begin{entry}{分钟}{4,9}[Radicais ⼑、⾦]
  \begin{phonetics}{分钟}{fen1zhong1}[][HSK 2]
    \definition{s.}{minuto (usado em intervalos de tempo)}
  \end{phonetics}
\end{entry}

\begin{entry}{分配}{4,10}[Radicais ⼑、⾣]
  \begin{phonetics}{分配}{fen1pei4}[][HSK 3]
    \definition{v.}{atribuir; dispor | atribuir; compartilhar; distribuir}
  \end{phonetics}
\end{entry}

\begin{entry}{分散}{4,12}[Radicais ⼑、⽁]
  \begin{phonetics}{分散}{fen1san4}[][HSK 4]
    \definition{adj.}{espalhado; disperso; desviado; fragmentado; sem foco}
    \definition{v.}{dispersar; espalhar; descentralizar | separar-se; desunir-se}
  \end{phonetics}
\end{entry}

\begin{entry}{分量}{4,12}[Radicais ⼑、⾥]
  \begin{phonetics}{分量}{fen1liang4}
    \definition{s.}{componente vetorial}
  \end{phonetics}
  \begin{phonetics}{分量}{fen4liang4}
    \definition{s.}{tamanho da porção (comida)}
  \end{phonetics}
  \begin{phonetics}{分量}{fen4liang5}
    \definition{s.}{quantidade | peso | medida}
  \end{phonetics}
\end{entry}

\begin{entry}{分数}{4,13}[Radicais ⼑、⽁]
  \begin{phonetics}{分数}{fen1 shu4}[][HSK 2]
    \definition{s.}{fração | número fracionário | marca | nota | ponto}
  \end{phonetics}
\end{entry}

\begin{entry}{切}{4}[Radical ⼑]
  \begin{phonetics}{切}{qie1}[][HSK 4]
    \definition{v.}{cortar; fatiar; separar itens com uma faca | cortar ou romper; truncar | (geometria) geometria, refere-se a quando uma linha, círculo ou superfície intercepta um círculo, arco ou esfera em apenas um ponto}
  \end{phonetics}
  \begin{phonetics}{切}{qie4}
    \definition{adj.}{ansioso; sério | duro; severo; rude; áspero}
    \definition{adv.}{com certeza; certamente}
    \definition{s.}{limiar; degrau}
    \definition{v.}{ser prático ou realista | ajustar-se ou corresponder | ser próximo ou íntimo | cortar algo em pedaços com uma faca | (medicina tradicional chinesa) tomar o pulso}
  \end{phonetics}
\end{entry}

\begin{entry}{切割}{4,12}[Radicais ⼑、⼑]
  \begin{phonetics}{切割}{qie1ge1}
    \definition{v.}{cortar}
  \end{phonetics}
\end{entry}

\begin{entry}{办}{4}[Radical ⼒]
  \begin{phonetics}{办}{ban4}[][HSK 2]
    \definition{v.}{lidar com | lidar | gerenciar | configurar}
  \end{phonetics}
\end{entry}

\begin{entry}{办公}{4,4}[Radicais ⼒、⼋]
  \begin{phonetics}{办公}{ban4gong1}
    \definition{v.+compl.}{lidar com negócios oficiais | trabalhar (especialmente em um escritório)}
  \end{phonetics}
\end{entry}

\begin{entry}{办公室}{4,4,9}[Radicais ⼒、⼋、⼧]
  \begin{phonetics}{办公室}{ban4gong1shi4}[][HSK 2]
    \definition[间]{s.}{gabinete | escritório}
  \end{phonetics}
\end{entry}

\begin{entry}{办事}{4,8}[Radicais ⼒、⼅]
  \begin{phonetics}{办事}{ban4 shi4}[][HSK 4]
    \definition{v.}{trabalhar | lidar com assuntos; manipular transações}
  \end{phonetics}
\end{entry}

\begin{entry}{办法}{4,8}[Radicais ⼒、⽔]
  \begin{phonetics}{办法}{ban4fa3}[][HSK 2]
    \definition[条,个]{s.}{meio (de se fazer alguma coisa) | método | medida}
  \end{phonetics}
\end{entry}

\begin{entry}{办理}{4,11}[Radicais ⼒、⽟]
  \begin{phonetics}{办理}{ban4li3}[][HSK 3]
    \definition{v.}{conduzir | manusear | transacionar}
  \end{phonetics}
\end{entry}

\begin{entry}{勾}{4}[Radical ⼓]
  \begin{phonetics}{勾}{gou1}
    \definition*{s.}{sobrenome Gou}
    \definition{v.}{atrair | excitar | marcar | atacar | delinear | conspirar}
    \variantof{钩}
  \end{phonetics}
  \begin{phonetics}{勾}{gou4}
    \definition{s.}{usado em 勾当}
    \seeref{勾当}{gou4dang4}
  \end{phonetics}
\end{entry}

\begin{entry}{勾当}{4,6}[Radicais ⼓、⼹]
  \begin{phonetics}{勾当}{gou4dang4}
    \definition{s.}{negócio obscuro}
  \end{phonetics}
\end{entry}

\begin{entry}{化}{4}[Radical ⼔]
  \begin{phonetics}{化}{hua1}[][HSK 3]
    \definition*{s.}{sobrenome Hua}
    \definition{s.}{químico}
    \definition{suf.}{modernizar; modernização}
    \definition{v.}{mudar; converter; transformar | converter; influenciar | derreter; dissolver | digerir | queimar | morrer | pedir esmola}
    \variantof{花}
  \end{phonetics}
\end{entry}

\begin{entry}{化学}{4,8}[Radicais ⼔、⼦]
  \begin{phonetics}{化学}{hua4xue2}
    \definition{s.}{química (disciplina)}
  \end{phonetics}
\end{entry}

\begin{entry}{区}{4}[Radical ⼖]
  \begin{phonetics}{区}{ou1}
    \definition*{s.}{sobrenome Ou}
  \end{phonetics}
  \begin{phonetics}{区}{qu1}[][HSK 3]
    \definition[个]{s.}{área; distrito; região; zona | uma divisão administrativa}
    \definition{v.}{distinguir; classificar; subdividir}
  \end{phonetics}
\end{entry}

\begin{entry}{区别}{4,7}[Radicais ⼖、⼑]
  \begin{phonetics}{区别}{qu1bie2}[][HSK 3]
    \definition[个]{s.}{diferença; distinção; discriminação}
    \definition{v.}{distinguir; diferenciar; fazer distinção entre}
  \end{phonetics}
\end{entry}

\begin{entry}{区域}{4,11}[Radicais ⼖、⼟]
  \begin{phonetics}{区域}{qu1yu4}
    \definition{s.}{área | região | distrito}
  \end{phonetics}
\end{entry}

\begin{entry}{升}{4}[Radical ⼗]
  \begin{phonetics}{升}{sheng1}[][HSK 3]
    \definition*{s.}{sobrenome Sheng}
    \definition{clas.}{litro (l)}
    \definition{s.}{sheng, uma unidade de medida seca para grãos (= 1 litro)}
    \definition{v.}{elevar; içar; subir; ascender | promover}
  \end{phonetics}
\end{entry}

\begin{entry}{升起}{4,10}[Radicais ⼗、⾛]
  \begin{phonetics}{升起}{sheng1qi3}
    \definition{v.}{levantar | içar | subir}
  \end{phonetics}
\end{entry}

\begin{entry}{午}{4}[Radical ⼗]
  \begin{phonetics}{午}{wu3}
    \definition{s.}{período entre 11h00 e 13h00, meio-dia}
  \end{phonetics}
\end{entry}

\begin{entry}{午休}{4,6}[Radicais ⼗、⼈]
  \begin{phonetics}{午休}{wu3xiu1}
    \definition{s.}{pausa para almoço | cochilo na hora do almoço | intervalo do meio-dia}
  \end{phonetics}
\end{entry}

\begin{entry}{午后}{4,6}[Radicais ⼗、⼝]
  \begin{phonetics}{午后}{wu3hou4}
    \definition{s.}{tarde | período da tarde}
  \end{phonetics}
\end{entry}

\begin{entry}{午饭}{4,7}[Radicais ⼗、⾷]
  \begin{phonetics}{午饭}{wu3fan4}[][HSK 1]
    \definition[份,顿,次,餐]{s.}{almoço}
  \seealsoref{午餐}{wu3 can1}
  \end{phonetics}
\end{entry}

\begin{entry}{午夜}{4,8}[Radicais ⼗、⼣]
  \begin{phonetics}{午夜}{wu3ye4}
    \definition{s.}{meia-noite}
  \end{phonetics}
\end{entry}

\begin{entry}{午前}{4,9}[Radicais ⼗、⼑]
  \begin{phonetics}{午前}{wu3qian2}
    \definition{s.}{\emph{A.M.} | manhã | período da manhã}
  \end{phonetics}
\end{entry}

\begin{entry}{午宴}{4,10}[Radicais ⼗、⼧]
  \begin{phonetics}{午宴}{wu3yan4}
    \definition{s.}{banquete de almoço}
  \end{phonetics}
\end{entry}

\begin{entry}{午睡}{4,13}[Radicais ⼗、⽬]
  \begin{phonetics}{午睡}{wu3 shui4}[][HSK 2]
    \definition{s.}{siesta}
    \definition{v.}{tirar uma soneca}
  \end{phonetics}
\end{entry}

\begin{entry}{午餐}{4,16}[Radicais ⼗、⾷]
  \begin{phonetics}{午餐}{wu3 can1}[][HSK 2]
    \definition[份,顿,次]{s.}{almoço}
  \seealsoref{午饭}{wu3fan4}
  \end{phonetics}
\end{entry}

\begin{entry}{历史}{4,5}[Radicais ⼚、⼝]
  \begin{phonetics}{历史}{li4shi3}[][HSK 4]
    \definition[个,门,段]{s.}{histórico; registro do passado; processo de desenvolvimento da natureza e da sociedade humana; processo de desenvolvimento de uma coisa ou pessoa | história; eventos passados; experiência | história; refere-se ao tema da história}
  \end{phonetics}
\end{entry}

\begin{entry}{友好}{4,6}[Radicais ⼜、⼥]
  \begin{phonetics}{友好}{you3hao3}[][HSK 2]
    \definition{adj.}{amigável}
    \definition{s.}{amigo próximo, íntimo}
  \end{phonetics}
\end{entry}

\begin{entry}{双}{4}[Radical ⼜]
  \begin{phonetics}{双}{shuang1}[][HSK 3]
    \definition*{s.}{sobrenome Shuang}
    \definition{adj.}{dois; gêmeo; par; dual | par (número) | duplo; dublê; duplicata; cópia; sósia}
    \definition{clas.}{par}
  \end{phonetics}
\end{entry}

\begin{entry}{双方}{4,4}[Radicais ⼜、⽅]
  \begin{phonetics}{双方}{shuang1fang1}[][HSK 3]
    \definition{s.}{ambos os lados; as duas partes}
  \end{phonetics}
\end{entry}

\begin{entry}{双方同意}{4,4,6,13}[Radicais ⼜、⽅、⼝、⼼]
  \begin{phonetics}{双方同意}{shuang1fang1tong2yi4}
    \definition{s.}{acordo bilateral}
  \end{phonetics}
\end{entry}

\begin{entry}{双打}{4,5}[Radicais ⼜、⼿]
  \begin{phonetics}{双打}{shuang1da3}
    \definition[场]{s.}{duplas (em esportes)}
  \end{phonetics}
\end{entry}

\begin{entry}{双层床}{4,7,7}[Radicais ⼜、⼫、⼴]
  \begin{phonetics}{双层床}{shuang1ceng2chuang2}
    \definition{s.}{beliche}
  \end{phonetics}
\end{entry}

\begin{entry}{反}{4}[Radical ⼜]
  \begin{phonetics}{反}{fan3}[][HSK 4]
    \definition{adj.}{oposto; contrário; invertido | contrarrevolucionário, reacionário}
    \definition{adv.}{pelo contrário; inversamente}
    \definition{v.}{inverter o lado; de cabeça para baixo; na direção oposta | virar; converter | retornar | opor-se; combater; voltar-se contra | rebelar-se; revoltar-se | inferir; deduzir; raciocinar por analogia}
  \end{phonetics}
\end{entry}

\begin{entry}{反对}{4,5}[Radicais ⼜、⼨]
  \begin{phonetics}{反对}{fan3dui4}[][HSK 3]
    \definition{v.}{lutar; opor-se; objetar a; ser contra}
  \end{phonetics}
\end{entry}

\begin{entry}{反对派}{4,5,9}[Radicais ⼜、⼨、⽔]
  \begin{phonetics}{反对派}{fan3dui4pai4}
    \definition{s.}{facção de oposição}
  \end{phonetics}
\end{entry}

\begin{entry}{反对党}{4,5,10}[Radicais ⼜、⼨、⼉]
  \begin{phonetics}{反对党}{fan3dui4dang3}
    \definition{s.}{partido de oposição}
  \end{phonetics}
\end{entry}

\begin{entry}{反对票}{4,5,11}[Radicais ⼜、⼨、⽰]
  \begin{phonetics}{反对票}{fan3dui4piao4}
    \definition{s.}{voto dissidente}
  \end{phonetics}
\end{entry}

\begin{entry}{反正}{4,5}[Radicais ⼜、⽌]
  \begin{phonetics}{反正}{fan3zheng4}[][HSK 3]
    \definition{adv.}{de qualquer forma | tudo igual; em qualquer caso}
  \end{phonetics}
\end{entry}

\begin{entry}{反而}{4,6}[Radicais ⼜、⽽]
  \begin{phonetics}{反而}{fan3'er2}[][HSK 4]
    \definition{adv.}{em vez disso; ao contrário de; contrário ao significado da frase anterior ou inesperado, desempenha o papel de uma reviravolta em uma frase}
  \end{phonetics}
\end{entry}

\begin{entry}{反应}{4,7}[Radicais ⼜、⼴]
  \begin{phonetics}{反应}{fan3ying4}[][HSK 3]
    \definition[个]{s.}{reação; resposta}
    \definition{v.}{reagir; responder}
  \end{phonetics}
\end{entry}

\begin{entry}{反复}{4,9}[Radicais ⼜、⼢]
  \begin{phonetics}{反复}{fan3fu4}[][HSK 3]
    \definition{adv.}{repetidamente; de ​​novo e de novo}
    \definition{s.}{reversão; recaída}
    \definition{v.}{recuar; cortar e mudar}
  \end{phonetics}
\end{entry}

\begin{entry}{反映}{4,9}[Radicais ⼜、⽇]
  \begin{phonetics}{反映}{fan3ying4}[][HSK 4]
    \definition{s.}{reflexão; opiniões sobre pessoas ou situações}
    \definition{v.}{refletir; espelhar; figurativamente, trazer à tona a essência de uma questão objetiva (expressão idiomática); expressar a essência de algo objetivamente | relatar; tornar conhecido; informar às autoridades superiores | refletir; espelhar; a imagem de um objeto aparece invertida em outro objeto}
  \end{phonetics}
\end{entry}

\begin{entry}{反省}{4,9}[Radicais ⼜、⽬]
  \begin{phonetics}{反省}{fan3xing3}
    \definition{v.}{examinar a consciência | questionar-se | refletir sobre si mesmo | sondar a alma}
  \end{phonetics}
\end{entry}

\begin{entry}{天}{4}[Radical ⼤]
  \begin{phonetics}{天}{tian1}[][HSK 1]
    \definition{s.}{dia | céu | paraíso}
  \end{phonetics}
\end{entry}

\begin{entry}{天上}{4,3}[Radicais ⼤、⼀]
  \begin{phonetics}{天上}{tian1 shang4}[][HSK 2]
    \definition{s.}{o céu | paraíso}
  \end{phonetics}
\end{entry}

\begin{entry}{天下}{4,3}[Radicais ⼤、⼀]
  \begin{phonetics}{天下}{tian1xia4}
    \definition{s.}{terra sob o céu | o mundo todo | toda a China | reino}
  \end{phonetics}
\end{entry}

\begin{entry}{天才}{4,3}[Radicais ⼤、⼿]
  \begin{phonetics}{天才}{tian1cai2}
    \definition{adj.}{talentoso | superdotado | genial}
    \definition{s.}{talento | dom | gênio}
  \end{phonetics}
\end{entry}

\begin{entry}{天公}{4,4}[Radicais ⼤、⼋]
  \begin{phonetics}{天公}{tian1gong1}
    \definition{s.}{céu, paraíso | senhor do céu}
  \end{phonetics}
\end{entry}

\begin{entry}{天天}{4,4}[Radicais ⼤、⼤]
  \begin{phonetics}{天天}{tian1tian1}
    \definition{adv.}{todo dia}
  \end{phonetics}
\end{entry}

\begin{entry}{天气}{4,4}[Radicais ⼤、⽓]
  \begin{phonetics}{天气}{tian1qi4}[][HSK 1]
    \definition{s.}{clima, tempo}
  \end{phonetics}
\end{entry}

\begin{entry}{天花板}{4,7,8}[Radicais ⼤、⾋、⽊]
  \begin{phonetics}{天花板}{tian1hua1ban3}
    \definition{s.}{teto}
  \end{phonetics}
\end{entry}

\begin{entry}{天使}{4,8}[Radicais ⼤、⼈]
  \begin{phonetics}{天使}{tian1shi3}
    \definition{s.}{anjo}
  \end{phonetics}
\end{entry}

\begin{entry}{天择}{4,8}[Radicais ⼤、⼿]
  \begin{phonetics}{天择}{tian1ze2}
    \definition{s.}{seleção natural}
  \end{phonetics}
\end{entry}

\begin{entry}{天空}{4,8}[Radicais ⼤、⽳]
  \begin{phonetics}{天空}{tian1kong1}[][HSK 3]
    \definition{s.}{o céu; o firmamento}
  \end{phonetics}
\end{entry}

\begin{entry}{天柱}{4,9}[Radicais ⼤、⽊]
  \begin{phonetics}{天柱}{tian1zhu4}
    \definition{s.}{pilar celestial, que sustenta o céu}
  \end{phonetics}
\end{entry}

\begin{entry}{天真}{4,10}[Radicais ⼤、⼗]
  \begin{phonetics}{天真}{tian1zhen1}[][HSK 4]
    \definition{adj.}{ingênuo; inocente; ignorante; simples de coração, direto por natureza, livre de fingimento e hipocrisia}
  \end{phonetics}
\end{entry}

\begin{entry}{天堂}{4,11}[Radicais ⼤、⼟]
  \begin{phonetics}{天堂}{tian1tang2}
    \definition{s.}{paraíso, céu}
  \end{phonetics}
\end{entry}

\begin{entry}{天然}{4,12}[Radicais ⼤、⽕]
  \begin{phonetics}{天然}{tian1ran2}
    \definition{adj.}{natural}
  \end{phonetics}
\end{entry}

\begin{entry}{天鹅}{4,12}[Radicais ⼤、⿃]
  \begin{phonetics}{天鹅}{tian1'e2}
    \definition{s.}{cisne}
  \end{phonetics}
\end{entry}

\begin{entry}{太}{4}[Radical ⼤]
  \begin{phonetics}{太}{tai4}[][HSK 1]
    \definition{adv.}{excessivamente | demais | muito}
  \end{phonetics}
\end{entry}

\begin{entry}{太太}{4,4}[Radicais ⼤、⼤]
  \begin{phonetics}{太太}{tai4tai5}[][HSK 2]
    \definition[个,位]{s.}{esposa | madame| mulher casada}
  \end{phonetics}
\end{entry}

\begin{entry}{太平洋}{4,5,9}[Radicais ⼤、⼲、⽔]
  \begin{phonetics}{太平洋}{tai4ping2 yang2}
    \definition*{s.}{Oceano Pacífico}
  \end{phonetics}
\end{entry}

\begin{entry}{太阳}{4,6}[Radicais ⼤、⾩]
  \begin{phonetics}{太阳}{tai4yang5}[][HSK 2]
    \definition[个]{s.}{sol | abreviação de 太阳穴}
    \seeref{太阳穴}{tai4yang2xue2}
  \end{phonetics}
\end{entry}

\begin{entry}{太阳日}{4,6,4}[Radicais ⼤、⾩、⽇]
  \begin{phonetics}{太阳日}{tai4yang2ri4}
    \definition{s.}{dia solar}
  \end{phonetics}
\end{entry}

\begin{entry}{太阳风}{4,6,4}[Radicais ⼤、⾩、⾵]
  \begin{phonetics}{太阳风}{tai4yang2feng1}
    \definition{s.}{vento solar}
  \end{phonetics}
\end{entry}

\begin{entry}{太阳穴}{4,6,5}[Radicais ⼤、⾩、⽳]
  \begin{phonetics}{太阳穴}{tai4yang2xue2}
    \definition{s.}{têmpora (nas laterais da cabeça humana)}
  \end{phonetics}
\end{entry}

\begin{entry}{太阳灯}{4,6,6}[Radicais ⼤、⾩、⽕]
  \begin{phonetics}{太阳灯}{tai4yang2deng1}
    \definition{s.}{lâmpada solar (com células fotovoltaicas)}
  \end{phonetics}
\end{entry}

\begin{entry}{太阳雨}{4,6,8}[Radicais ⼤、⾩、⾬]
  \begin{phonetics}{太阳雨}{tai4yang2yu3}
    \definition{s.}{banho de sol}
  \end{phonetics}
\end{entry}

\begin{entry}{太阳窗}{4,6,12}[Radicais ⼤、⾩、⽳]
  \begin{phonetics}{太阳窗}{tai4yang2chuang1}
    \definition{s.}{teto solar (de veículos)}
  \end{phonetics}
\end{entry}

\begin{entry}{太阳镜}{4,6,16}[Radicais ⼤、⾩、⾦]
  \begin{phonetics}{太阳镜}{tai4yang2jing4}
    \definition{s.}{óculos de sol}
  \end{phonetics}
\end{entry}

\begin{entry}{太阳翼}{4,6,17}[Radicais ⼤、⾩、⽻]
  \begin{phonetics}{太阳翼}{tai4yang2yi4}
    \definition{s.}{painel solar}
  \end{phonetics}
\end{entry}

\begin{entry}{太极拳}{4,7,10}[Radicais ⼤、⽊、⼿]
  \begin{phonetics}{太极拳}{tai4ji2quan2}
    \definition*{s.}{Tai Chi Chuan, Taiji, T'aichi ou T'aichichuan; forma tradicional de exercício físico ou relaxamento}
  \end{phonetics}
\end{entry}

\begin{entry}{太空}{4,8}[Radicais ⼤、⽳]
  \begin{phonetics}{太空}{tai4kong1}
    \definition{s.}{espaço sideral | espaço exterior}
  \end{phonetics}
\end{entry}

\begin{entry}{夫人}{4,2}[Radicais ⼤、⼈]
  \begin{phonetics}{夫人}{fu1ren2}[][HSK 4]
    \definition[位]{s.}{senhora; \emph{lady}; madame; na antiguidade, as esposas dos senhores feudais eram chamadas de ``madame'' e, nas dinastias Ming e Qing, as esposas dos oficiais de primeiro e segundo escalão eram chamadas de ``madame'', que mais tarde foi usada para homenagear as esposas das pessoas em geral e agora é usada principalmente em ocasiões diplomáticas}
  \end{phonetics}
\end{entry}

\begin{entry}{夫妇}{4,6}[Radicais ⼤、⼥]
  \begin{phonetics}{夫妇}{fu1fu4}[][HSK 4]
    \definition[对]{s.}{casal; marido e mulher}
  \end{phonetics}
\end{entry}

\begin{entry}{夫妻}{4,8}[Radicais ⼤、⼥]
  \begin{phonetics}{夫妻}{fu1qi1}[][HSK 4]
    \definition[对]{s.}{casal; marido e mulher}
  \end{phonetics}
\end{entry}

\begin{entry}{孔}{4}[Radical ⼦]
  \begin{phonetics}{孔}{kong3}
    \definition*{s.}{sobrenome Kong}
    \definition{clas.}{para habitações em cavernas}
    \definition[个]{s.}{buraco}
  \end{phonetics}
\end{entry}

\begin{entry}{孔子}{4,3}[Radicais ⼦、⼦]
  \begin{phonetics}{孔子}{kong3zi3}
    \definition*{s.}{Confúcio (551-479 aC), pensador e filósofo social chinês}
  \seealsoref{孔夫子}{kong3fu1zi3}
  \end{phonetics}
\end{entry}

\begin{entry}{孔子学院}{4,3,8,9}[Radicais ⼦、⼦、⼦、⾩]
  \begin{phonetics}{孔子学院}{kong3zi3 xue2yuan4}
    \definition*{s.}{Instituto Confúcio, organização estabelecida internacionalmente pela República Popular da China, que promove a língua e a cultura chinesas}
  \end{phonetics}
\end{entry}

\begin{entry}{孔夫子}{4,4,3}[Radicais ⼦、⼤、⼦]
  \begin{phonetics}{孔夫子}{kong3fu1zi3}
    \definition*{s.}{Confúcio (551-479 aC), pensador e filósofo social chinês}
  \seealsoref{孔子}{kong3zi3}
  \end{phonetics}
\end{entry}

\begin{entry}{孔雀}{4,11}[Radicais ⼦、⾫]
  \begin{phonetics}{孔雀}{kong3que4}
    \definition{s.}{pavão}
  \end{phonetics}
\end{entry}

\begin{entry}{少}{4}[Radical ⼩]
  \begin{phonetics}{少}{shao3}[][HSK 1]
    \definition{adj.}{pouco, poucos}
    \definition{v.}{sentir falta | faltar | parar (de fazer algo)}
  \end{phonetics}
  \begin{phonetics}{少}{shao4}
    \definition{s.}{jovem}
  \end{phonetics}
\end{entry}

\begin{entry}{少年}{4,6}[Radicais ⼩、⼲]
  \begin{phonetics}{少年}{shao4 nian2}[][HSK 2]
    \definition[个]{s.}{adolescente; juventude; atualmente, a faixa etária geralmente referida é de 10 anos ou mais a 18 anos ou mais | menor; jovem; juvenil; refere-se a menores na faixa etária anterior | jovem; adolescente; rapaz}
  \end{phonetics}
\end{entry}

\begin{entry}{少数}{4,13}[Radicais ⼩、⽁]
  \begin{phonetics}{少数}{shao3 shu4}[][HSK 2]
    \definition{s.}{pequeno número | poucos | minoria}
  \end{phonetics}
\end{entry}

\begin{entry}{尤其}{4,8}[Radicais ⼪、⼋]
  \begin{phonetics}{尤其}{you2qi2}
    \definition{adv.}{especialmente | particularmente}
  \end{phonetics}
\end{entry}

\begin{entry}{尺}{4}[Radical ⼫]
  \begin{phonetics}{尺}{che3}
    \definition{s.}{(tom) uma nota da escala em gongchepu (工尺谱), correspondente a 2 na notação musical numerada}
  \seealsoref{工尺谱}{gong1 che3 pu3}
  \end{phonetics}
  \begin{phonetics}{尺}{chi3}[][HSK 4]
    \definition{clas.}{chi, uma unidade de comprimento (=13 metros)}
    \definition[把]{s.}{régua; instrumentos de medição | um instrumento no formato de uma régua}
  \end{phonetics}
\end{entry}

\begin{entry}{尺子}{4,3}[Radicais ⼫、⼦]
  \begin{phonetics}{尺子}{chi3zi5}[][HSK 4]
    \definition[把]{s.}{régua de madeira ou metal para orientar a caneta ou o lápis para desenhar linhas ou fazer medições}
  \end{phonetics}
\end{entry}

\begin{entry}{尺寸}{4,3}[Radicais ⼫、⼨]
  \begin{phonetics}{尺寸}{chi3 cun4}[][HSK 4]
    \definition{s.}{tamanho; medida; dimensão}
  \end{phonetics}
\end{entry}

\begin{entry}{巨大}{4,3}[Radicais ⼯、⼤]
  \begin{phonetics}{巨大}{ju4da4}[][HSK 4]
    \definition{adj.}{enorme; tremendo; enorme; gigantesco; imenso}
  \end{phonetics}
\end{entry}

\begin{entry}{巴士}{4,3}[Radicais ⼰、⼠]
  \begin{phonetics}{巴士}{ba1 shi4}[][HSK 4]
    \definition[辆]{s.}{ônibus; transliteração da palavra inglesa ``\emph{bus}''}
  \end{phonetics}
\end{entry}

\begin{entry}{巴西}{4,6}[Radicais ⼰、⾑]
  \begin{phonetics}{巴西}{ba1xi1}
    \definition*{s.}{Brasil}
  \end{phonetics}
\end{entry}

\begin{entry}{巴西人}{4,6,2}[Radicais ⼰、⾑、⼈]
  \begin{phonetics}{巴西人}{ba1xi1ren2}
    \definition[个,位]{s.}{brasileiro | pessoa ou povo do Brasil}[他是巴西人。(Ele é brasileiro.)]
  \end{phonetics}
\end{entry}

\begin{entry}{巴西战舞}{4,6,9,14}[Radicais ⼰、⾑、⼽、⾇]
  \begin{phonetics}{巴西战舞}{ba1xi1zhan4wu3}
    \definition{s.}{capoeira}
  \end{phonetics}
\end{entry}

\begin{entry}{巴勒斯坦}{4,11,12,8}[Radicais ⼰、⼒、⽄、⼟]
  \begin{phonetics}{巴勒斯坦}{ba1le4si1tan3}
    \definition*{s.}{Palestina}
  \end{phonetics}
\end{entry}

\begin{entry}{幻觉}{4,9}[Radicais ⼳、⾒]
  \begin{phonetics}{幻觉}{huan4jue2}
    \definition{s.}{ilusão | alucinação}
  \end{phonetics}
\end{entry}

\begin{entry}{开}{4}[Radical ⼶]
  \begin{phonetics}{开}{kai1}[][HSK 1]
    \definition{clas.}{quilate (ouro)}
    \definition{v.}{abrir | ligar | dirigir | iniciar (alguma coisa) | começar | ferver | escrever  (uma receita, cheque, fatura, etc.) | operar (um veículo) | abreviação de Kelvin 开尔文}
    \seeref{开尔文}{kai1'er3wen2}
  \end{phonetics}
\end{entry}

\begin{entry}{开口}{4,3}[Radicais ⼶、⼝]
  \begin{phonetics}{开口}{kai1kou3}
    \definition{v.}{abrir a boca de alguém | começar a falar}
  \end{phonetics}
\end{entry}

\begin{entry}{开心}{4,4}[Radicais ⼶、⼼]
  \begin{phonetics}{开心}{kai1xin1}[][HSK 2]
    \definition{v.}{sentir-se feliz | regozijar-se | divertir-se | tirar sarro de alguém}
  \end{phonetics}
\end{entry}

\begin{entry}{开水}{4,4}[Radicais ⼶、⽔]
  \begin{phonetics}{开水}{kai1shui3}[][HSK 4]
    \definition[杯,瓶]{s.}{água fervida; água fervente}
  \end{phonetics}
\end{entry}

\begin{entry}{开车}{4,4}[Radicais ⼶、⾞]
  \begin{phonetics}{开车}{kai1 che1}[][HSK 1]
    \definition{v.+compl.}{conduzir | dirigir}
  \end{phonetics}
\end{entry}

\begin{entry}{开业}{4,5}[Radicais ⼶、⼀]
  \begin{phonetics}{开业}{kai1 ye4}[][HSK 3]
    \definition{v.}{iniciar um negócio; abrir para negócios | abrir um consultório particular}
  \end{phonetics}
\end{entry}

\begin{entry}{开发}{4,5}[Radicais ⼶、⼜]
  \begin{phonetics}{开发}{kai1fa1}[][HSK 3]
    \definition{v.}{explorar | tornar acessível}
  \end{phonetics}
\end{entry}

\begin{entry}{开发区}{4,5,4}[Radicais ⼶、⼜、⼖]
  \begin{phonetics}{开发区}{kai1fa1qu1}
    \definition{s.}{zona de desenvolvimento}
  \end{phonetics}
\end{entry}

\begin{entry}{开头}{4,5}[Radicais ⼶、⼤]
  \begin{phonetics}{开头}{kai1tou2}
    \definition{s.}{início | começo}
    \definition{v.+compl.}{iniciar | começar | fazer um começo}
  \end{phonetics}
\end{entry}

\begin{entry}{开尔文}{4,5,4}[Radicais ⼶、⼩、⽂]
  \begin{phonetics}{开尔文}{kai1'er3wen2}
    \definition{s.}{Kelvin, temperatura absoluta | K, escala de temperatura}
  \end{phonetics}
\end{entry}

\begin{entry}{开会}{4,6}[Radicais ⼶、⼈]
  \begin{phonetics}{开会}{kai1 hui4}[][HSK 1]
    \definition{v.+compl.}{realizar uma reunião | ter uma reunião | participar de uma reunião (conferência)}
  \end{phonetics}
\end{entry}

\begin{entry}{开机}{4,6}[Radicais ⼶、⽊]
  \begin{phonetics}{开机}{kai1 ji1}[][HSK 2]
    \definition{v.}{começar a filmar um filme ou programa de TV | iniciar uma máquina}
  \end{phonetics}
\end{entry}

\begin{entry}{开启}{4,7}[Radicais ⼶、⼝]
  \begin{phonetics}{开启}{kai1qi3}
    \definition{v.}{abrir | iniciar | (computação) ativar}
  \end{phonetics}
\end{entry}

\begin{entry}{开花}{4,7}[Radicais ⼶、⾋]
  \begin{phonetics}{开花}{kai1hua1}[][HSK 4]
    \definition{v.}{florescer; desabrochar; estar em flor; entrar em flor;  metáfora para um coração feliz ou um rosto sorridente | explodir}
  \end{phonetics}
\end{entry}

\begin{entry}{开夜车}{4,8,4}[Radicais ⼶、⼣、⾞]
  \begin{phonetics}{开夜车}{kai1ye4che1}
    \definition{expr.}{trabalho noturno | (literalmente) ``conduzir carro à noite''}
  \end{phonetics}
\end{entry}

\begin{entry}{开始}{4,8}[Radicais ⼶、⼥]
  \begin{phonetics}{开始}{kai1shi3}[][HSK 3]
    \definition{adv.}{inicial}
    \definition[个]{s.}{começo; início; estágio inicial}
    \definition{v.}{começar; iniciar}
  \end{phonetics}
\end{entry}

\begin{entry}{开学}{4,8}[Radicais ⼶、⼦]
  \begin{phonetics}{开学}{kai1 xue2}[][HSK 2]
    \definition{v.}{iniciar as aulas | iniciar o semestre | começar as aulas}
  \end{phonetics}
\end{entry}

\begin{entry}{开放}{4,8}[Radicais ⼶、⽅]
  \begin{phonetics}{开放}{kai1fang4}[][HSK 3]
    \definition{adj.}{de mente aberta; sem restrições por convenções}
    \definition{v.}{florescer | abrir (para o público) | diminuir uma proibição, restrição, etc. (de política)}
  \end{phonetics}
\end{entry}

\begin{entry}{开玩笑}{4,8,10}[Radicais ⼶、⽟、⽵]
  \begin{phonetics}{开玩笑}{kai1 wan2xiao4}[][HSK 1]
    \definition{v.}{contar uma piada | brincar | fazer piada de | pregar uma peça | provocar}
  \end{phonetics}
\end{entry}

\begin{entry}{开展}{4,10}[Radicais ⼶、⼫]
  \begin{phonetics}{开展}{kai1zhan3}[][HSK 3]
    \definition{v.}{lançar; desenvolver | abrir; inaugurar}
  \end{phonetics}
\end{entry}

\begin{entry}{开锁}{4,12}[Radicais ⼶、⾦]
  \begin{phonetics}{开锁}{kai1suo3}
    \definition{v.}{desbloquear | destravar}
  \end{phonetics}
\end{entry}

\begin{entry}{引擎}{4,16}[Radicais ⼸、⼿]
  \begin{phonetics}{引擎}{yin3qing2}
    \definition[台]{s.}{motor | (empréstimo linguístico) \emph{engine}}
  \end{phonetics}
\end{entry}

\begin{entry}{心}{4}[Kangxi 61][Radical ⼼]
  \begin{phonetics}{心}{xin1}[][HSK 3]
    \definition*{s.}{Xin, uma das mansões lunares}
    \definition[颗]{s.}{o coração | coração; mente; sentimento; intenção | centro; núcleo}
  \end{phonetics}
\end{entry}

\begin{entry}{心中}{4,4}[Radicais ⼼、⼁]
  \begin{phonetics}{心中}{xin1zhong1}[][HSK 2]
    \definition{adv.}{nos pensamentos | no coração}
    \definition{s.}{ponto central}
  \end{phonetics}
\end{entry}

\begin{entry}{心机}{4,6}[Radicais ⼼、⽊]
  \begin{phonetics}{心机}{xin1ji1}
    \definition{s.}{pensamento | esquema}
  \end{phonetics}
\end{entry}

\begin{entry}{心声}{4,7}[Radicais ⼼、⼠]
  \begin{phonetics}{心声}{xin1sheng1}
    \definition{s.}{desejo sincero | voz interior | aspiração}
  \end{phonetics}
\end{entry}

\begin{entry}{心里}{4,7}[Radicais ⼼、⾥]
  \begin{phonetics}{心里}{xin1 li3}[][HSK 2]
    \definition[把]{s.}{no coração | no coração de alguém | na mente}
  \end{phonetics}
\end{entry}

\begin{entry}{心疼}{4,10}[Radicais ⼼、⽧]
  \begin{phonetics}{心疼}{xin1teng2}
    \definition{adj.}{angustiado}
    \definition{v.}{sentir pena de alguém | arrepender-se | ressentir-se | ficar angustiado}
  \end{phonetics}
\end{entry}

\begin{entry}{心情}{4,11}[Radicais ⼼、⼼]
  \begin{phonetics}{心情}{xin1qing2}[][HSK 2]
    \definition{s.}{humor | sentimento | estado de espírito}
  \end{phonetics}
\end{entry}

\begin{entry}{户}{4}[Radical ⼾]
  \begin{phonetics}{户}{hu4}[][HSK 4]
    \definition*{s.}{sobrenome Hu}
    \definition[个]{s.}{porta com um painel; porta | domicílio; residência; família | status familiar | conta (banco)}
  \end{phonetics}
\end{entry}

\begin{entry}{手}{4}[Kangxi 64][Radical ⼿]
  \begin{phonetics}{手}{shou3}[][HSK 1]
    \definition{adj.}{conveniente}
    \definition{clas.}{de habilidade}
    \definition[双,只]{s.}{mão | pessoa envolvida em certos tipos de trabalho | pessoa qualificada para certos tipos de trabalho}
    \definition{v.}{segurar (formal)}
  \end{phonetics}
\end{entry}

\begin{entry}{手工}{4,3}[Radicais ⼿、⼯]
  \begin{phonetics}{手工}{shou3gong1}[][HSK 4]
    \definition{s.}{trabalho manual; trabalho feito à mão | método de operação manual; método manual, sem máquina | remuneração por trabalho manual, braçal; custo de mão de obra braçal}
  \end{phonetics}
\end{entry}

\begin{entry}{手工艺人}{4,3,4,2}[Radicais ⼿、⼯、⾋、⼈]
  \begin{phonetics}{手工艺人}{shou3gong1 yi4ren2}
    \definition{s.}{artesão}
  \end{phonetics}
\end{entry}

\begin{entry}{手术}{4,5}[Radicais ⼿、⽊]
  \begin{phonetics}{手术}{shou3shu4}[][HSK 4]
    \definition[个]{s.}{cirurgia; operação (cirúrgica); método de tratamento no qual o médico usa uma faca, tesoura etc. para fazer uma incisão em uma parte do corpo do paciente}
    \definition{v.}{realizar uma cirurgia}
  \end{phonetics}
\end{entry}

\begin{entry}{手边}{4,5}[Radicais ⼿、⾡]
  \begin{phonetics}{手边}{shou3bian1}
    \definition{adv.}{à mão | na mão}
  \end{phonetics}
\end{entry}

\begin{entry}{手机}{4,6}[Radicais ⼿、⽊]
  \begin{phonetics}{手机}{shou3ji1}[][HSK 1]
    \definition[部,支]{s.}{telefone celular ou móvel}
  \end{phonetics}
\end{entry}

\begin{entry}{手里}{4,7}[Radicais ⼿、⾥]
  \begin{phonetics}{手里}{shou3 li3}[][HSK 4]
    \definition[个]{s.}{(uma situação está) nas mãos de alguém | em mãos}
  \end{phonetics}
\end{entry}

\begin{entry}{手刹}{4,8}[Radicais ⼿、⼑]
  \begin{phonetics}{手刹}{shou3sha1}
    \definition{s.}{freio de mão}
  \end{phonetics}
\end{entry}

\begin{entry}{手表}{4,8}[Radicais ⼿、⾐]
  \begin{phonetics}{手表}{shou3biao3}[][HSK 2]
    \definition[块,只,个]{s.}{relógio de pulso}
  \end{phonetics}
\end{entry}

\begin{entry}{手指}{4,9}[Radicais ⼿、⼿]
  \begin{phonetics}{手指}{shou3zhi3}[][HSK 3]
    \definition[根,个]{s.}{dedo da mão}
  \end{phonetics}
\end{entry}

\begin{entry}{手套}{4,10}[Radicais ⼿、⼤]
  \begin{phonetics}{手套}{shou3tao4}[][HSK 4]
    \definition[副,套,双,种]{s.}{luvas; itens usados ​​nas mãos, feitos de algodão, lã, couro, etc., para proteger as mãos ou manter o frio longe}
  \end{phonetics}
\end{entry}

\begin{entry}{手续}{4,11}[Radicais ⼿、⽷]
  \begin{phonetics}{手续}{shou3xu4}[][HSK 3]
    \definition[个]{s.}{processo; formalidade; procedimento}
  \end{phonetics}
\end{entry}

\begin{entry}{手臂}{4,17}[Radicais ⼿、⾁]
  \begin{phonetics}{手臂}{shou3bi4}
    \definition{s.}{braço}
  \end{phonetics}
\end{entry}

\begin{entry}{支}{4}[Kangxi 65][Radical ⽀]
  \begin{phonetics}{支}{zhi1}[][HSK 3]
    \definition*{s.}{sobrenome Zhi}
    \definition{clas.}{para equipes, etc. | para canções ou peças musicais | intensidade luminosa para luzes elétricas (velas, watts) | para objetos finos (canetas, armas, etc.) | (fiação) unidade de medida para espessura do fio}
    \definition{s.}{ramo; ramificação; tribo | os doze ramos terrestres}
    \definition{v.}{apoiar; sustentar | esticar-se; levantar | ordenar; mandar embora; colocar alguém para fora; despachar | pagar; desembolsar | receber; obter pagamento; sacar (dinheiro)}
  \end{phonetics}
\end{entry}

\begin{entry}{支支吾吾}{4,4,7,7}[Radicais ⽀、⽀、⼝、⼝]
  \begin{phonetics}{支支吾吾}{zhi1zhi1wu2wu2}
    \definition{v.}{falhar | murmurar | paralisar | gaguejar}
  \end{phonetics}
\end{entry}

\begin{entry}{支付}{4,5}[Radicais ⽀、⼈]
  \begin{phonetics}{支付}{zhi1 fu4}[][HSK 3]
    \definition{v.}{pagar (dinheiro); custear}
  \end{phonetics}
\end{entry}

\begin{entry}{支应}{4,7}[Radicais ⽀、⼴]
  \begin{phonetics}{支应}{zhi1ying4}
    \definition{v.}{lidar com | fornecer}
  \end{phonetics}
\end{entry}

\begin{entry}{支承}{4,8}[Radicais ⽀、⼿]
  \begin{phonetics}{支承}{zhi1cheng2}
    \definition{v.}{suportar o peso de (um edifício) | suportar}
  \end{phonetics}
\end{entry}

\begin{entry}{支持}{4,9}[Radicais ⽀、⼿]
  \begin{phonetics}{支持}{zhi1chi2}[][HSK 3]
    \definition[个]{s.}{apoio | suporte}
    \definition{v.}{suportar; sustentar; resistir | patrocinar; apoiar; incentivar}
  \end{phonetics}
\end{entry}

\begin{entry}{支根}{4,10}[Radicais ⽀、⽊]
  \begin{phonetics}{支根}{zhi1gen1}
    \definition{s.}{raiz ramificada | raízes de apoio | radícula}
  \end{phonetics}
\end{entry}

\begin{entry}{支票}{4,11}[Radicais ⽀、⽰]
  \begin{phonetics}{支票}{zhi1piao4}
    \definition[本]{s.}{cheque (banco)}
  \end{phonetics}
\end{entry}

\begin{entry}{文化}{4,4}[Radicais ⽂、⼔]
  \begin{phonetics}{文化}{wen2hua4}[][HSK 3]
    \definition[个,种]{s.}{cultura; civilização | cultura; alfabetização; escolaridade; educação}
  \end{phonetics}
\end{entry}

\begin{entry}{文化水平}{4,4,4,5}[Radicais ⽂、⼔、⽔、⼲]
  \begin{phonetics}{文化水平}{wen2hua4 shui3ping2}
    \definition{s.}{nível educacional}
  \end{phonetics}
\end{entry}

\begin{entry}{文化史}{4,4,5}[Radicais ⽂、⼔、⼝]
  \begin{phonetics}{文化史}{wen2hua4shi3}
    \definition*{s.}{História Cultural}
  \end{phonetics}
\end{entry}

\begin{entry}{文化层}{4,4,7}[Radicais ⽂、⼔、⼫]
  \begin{phonetics}{文化层}{wen2hua4ceng2}
    \definition{s.}{nível de cultura (em sítio arqueológico)}
  \end{phonetics}
\end{entry}

\begin{entry}{文化宫}{4,4,9}[Radicais ⽂、⼔、⼧]
  \begin{phonetics}{文化宫}{wen2hua4gong1}
    \definition{s.}{palácio cultural}
  \end{phonetics}
\end{entry}

\begin{entry}{文化热}{4,4,10}[Radicais ⽂、⼔、⽕]
  \begin{phonetics}{文化热}{wen2hua4re4}
    \definition{s.}{mania cultural | febre cultural}
  \end{phonetics}
\end{entry}

\begin{entry}{文化圈}{4,4,11}[Radicais ⽂、⼔、⼞]
  \begin{phonetics}{文化圈}{wen2hua4quan1}
    \definition{s.}{esfera de influência cultural}
  \end{phonetics}
\end{entry}

\begin{entry}{文化障碍}{4,4,13,13}[Radicais ⽂、⼔、⾩、⽯]
  \begin{phonetics}{文化障碍}{wen2hua4zhang4'ai4}
    \definition{s.}{barreira cultural}
  \end{phonetics}
\end{entry}

\begin{entry}{文件}{4,6}[Radicais ⽂、⼈]
  \begin{phonetics}{文件}{wen2jian4}[][HSK 3]
    \definition[份,分]{s.}{documentos oficiais; papéis; instrumentos | os arquivos no computador | artigos ou trabalhos sobre teorias políticas, atualidades, pesquisas acadêmicas, etc.}
  \end{phonetics}
\end{entry}

\begin{entry}{文字}{4,6}[Radicais ⽂、⼦]
  \begin{phonetics}{文字}{wen2zi4}[][HSK 3]
    \definition[种,类,段,行,篇]{s.}{personagens; roteiro; escrita
linguagem escrita}
  \end{phonetics}
\end{entry}

\begin{entry}{文学}{4,8}[Radicais ⽂、⼦]
  \begin{phonetics}{文学}{wen2xue2}[][HSK 3]
    \definition[个,种]{s.}{literatura}
  \end{phonetics}
\end{entry}

\begin{entry}{文学系}{4,8,7}[Radicais ⽂、⼦、⽷]
  \begin{phonetics}{文学系}{wen2xue2 xi4}
    \definition*{s.}{Faculdade de Letras}
  \end{phonetics}
\end{entry}

\begin{entry}{文明}{4,8}[Radicais ⽂、⽇]
  \begin{phonetics}{文明}{wen2ming2}[][HSK 3]
    \definition{adj.}{civilizado}
    \definition[个]{s.}{cultura; civilização}
  \end{phonetics}
\end{entry}

\begin{entry}{文章}{4,11}[Radicais ⽂、⾳]
  \begin{phonetics}{文章}{wen2zhang1}[][HSK 3]
    \definition[篇,段,页]{s.}{ensaio; dissertação; artigo | significado oculto; significado implícito | trabalho (coisas para fazer)}
  \end{phonetics}
\end{entry}

\begin{entry}{斤}{4}[Kangxi 69][Radical ⽄]
  \begin{phonetics}{斤}{jin1}[][HSK 2]
    \definition{clas.}{peso igual a 500 g}
  \end{phonetics}
\end{entry}

\begin{entry}{方}{4}[Radical ⽅]
  \begin{phonetics}{方}{fang1}[][HSK 4]
    \definition*{s.}{sobrenome Fang | Alquimia ``方术''}
    \definition{adj.}{reto; honesto; imparcial}
    \definition{adv.}{exatamente quando; no momento em que}
    \definition{clas.}{para coisas quadradas | quadrado ou cúbico (geralmente metro quadrado ou cúbico)}
    \definition{s.}{quadrado; um quadrado ou sólido com seis faces quadradas | (matemática) potência; o número de vezes que uma quantidade deve ser multiplicada por si mesma | direção | lado; festa | lugar; região; localidade | maneira; método; solução | prescrição | lei; regra}
  \seealsoref{方术}{fang1 shu4}
  \end{phonetics}
\end{entry}

\begin{entry}{方术}{4,5}[Radicais ⽅、⽊]
  \begin{phonetics}{方术}{fang1 shu4}
    \definition{s.}{artes de cura, adivinhação, horóscopo etc. | artes sobrenaturais (antigo)}
  \end{phonetics}
\end{entry}

\begin{entry}{方向}{4,6}[Radicais ⽅、⼝]
  \begin{phonetics}{方向}{fang1xiang4}[][HSK 2]
    \definition[个]{s.}{direção | orientação | alvo | meta | objetivo}
  \end{phonetics}
\end{entry}

\begin{entry}{方式}{4,6}[Radicais ⽅、⼷]
  \begin{phonetics}{方式}{fang1shi4}[][HSK 3]
    \definition[种,个]{s.}{maneira; método}
  \end{phonetics}
\end{entry}

\begin{entry}{方言}{4,7}[Radicais ⽅、⾔]
  \begin{phonetics}{方言}{fang1yan2}
    \definition*{s.}{o primeiro dicionário de dialeto chinês, editado por Yang Xiong 扬雄 no século I, contendo mais de 9.000 caracteres}
    \definition{s.}{dialeto}
  \seealsoref{扬雄}{yang2xiong2}
  \end{phonetics}
\end{entry}

\begin{entry}{方针}{4,7}[Radicais ⽅、⾦]
  \begin{phonetics}{方针}{fang1zhen1}[][HSK 4]
    \definition[个]{s.}{política; diretriz; princípio orientador; orientação da direção e das metas de um empreendimento}
  \end{phonetics}
\end{entry}

\begin{entry}{方法}{4,8}[Radicais ⽅、⽔]
  \begin{phonetics}{方法}{fang1fa3}[][HSK 2]
    \definition[个]{s.}{método | meio}
  \end{phonetics}
\end{entry}

\begin{entry}{方便}{4,9}[Radicais ⽅、⼈]
  \begin{phonetics}{方便}{fang1bian4}[][HSK 2]
    \definition{adj.}{conveniente | adequado}
    \definition{v.}{facilitar, facilitar as coisas | ter dinheiro de sobra | (eufemismo) aliviar-se}
  \end{phonetics}
\end{entry}

\begin{entry}{方便面}{4,9,9}[Radicais ⽅、⼈、⾯]
  \begin{phonetics}{方便面}{fang1 bian4 mian4}[][HSK 2]
    \definition{s.}{macarrão instantâneo}
  \end{phonetics}
\end{entry}

\begin{entry}{方面}{4,9}[Radicais ⽅、⾯]
  \begin{phonetics}{方面}{fang1mian4}[][HSK 2]
    \definition[个]{s.}{lado | campo | aspecto}
  \end{phonetics}
\end{entry}

\begin{entry}{方案}{4,10}[Radicais ⽅、⽊]
  \begin{phonetics}{方案}{fang1'an4}[][HSK 4]
    \definition[个,套]{s.}{plano; esquema; programa; planos específicos para tratar de um determinado problema | o esquema criado pelo governo; medidas ou regulamentações formuladas e implementadas pelo governo ou autoridades relevantes}
  \end{phonetics}
\end{entry}

\begin{entry}{无}{4}[Kangxi 71][Radical ⽆]
  \begin{phonetics}{无}{wu2}[][HSK 4]
    \definition{adv.}{não; não ter algo; não há\dots}
    \definition{conj.}{independentemente de; não importa se, o que, etc.}
    \definition{v.}{não ter; estar sem; não existir;}
  \end{phonetics}
\end{entry}

\begin{entry}{无人}{4,2}[Radicais ⽆、⼈]
  \begin{phonetics}{无人}{wu2ren2}
    \definition{adj.}{não tripulado | desabitado}
  \end{phonetics}
\end{entry}

\begin{entry}{无人机}{4,2,6}[Radicais ⽆、⼈、⽊]
  \begin{phonetics}{无人机}{wu2ren2ji1}
    \definition{s.}{\emph{drone} | veículo aéreo não tripulado}
  \end{phonetics}
\end{entry}

\begin{entry}{无论}{4,6}[Radicais ⽆、⾔]
  \begin{phonetics}{无论}{wu2lun4}[][HSK 4]
    \definition{conj.}{não importa o quê; não importa como; independentemente de; indica que as condições são diferentes, mas resultado é o mesmo |}
  \seealsoref{无论……也……}{wu2lun4 ye3}
  \end{phonetics}
\end{entry}

\begin{entry}{无论……也……}{4,6,3}[Radicais ⽆、⾔、⼄]
  \begin{phonetics}{无论……也……}{wu2lun4 ye3}
    \definition{conj.}{não apenas\dots, (o que, quem, como, etc.), \dots}
  \end{phonetics}
\end{entry}

\begin{entry}{无所谓}{4,8,11}[Radicais ⽆、⼾、⾔]
  \begin{phonetics}{无所谓}{wu2suo3wei4}[][HSK 4]
    \definition{v.}{não pode ser designado como; não merece o nome de; ser incapaz de dizer ou contar | não ter importância; ser indiferente;}
  \end{phonetics}
\end{entry}

\begin{entry}{无法}{4,8}[Radicais ⽆、⽔]
  \begin{phonetics}{无法}{wu2 fa3}[][HSK 4]
    \definition{adj.}{incapaz; incapacitado}
    \definition{v.}{não há nada a ser feito}
  \end{phonetics}
\end{entry}

\begin{entry}{无视}{4,8}[Radicais ⽆、⾒]
  \begin{phonetics}{无视}{wu2shi4}
    \definition{v.}{ignorar | desconsiderar}
  \end{phonetics}
\end{entry}

\begin{entry}{无限}{4,8}[Radicais ⽆、⾩]
  \begin{phonetics}{无限}{wu2 xian4}[][HSK 4]
    \definition{adj.}{infinito; ilimitado; sem limites; sem fim à vista}
  \end{phonetics}
\end{entry}

\begin{entry}{无故}{4,9}[Radicais ⽆、⽁]
  \begin{phonetics}{无故}{wu2gu4}
    \definition{adv.}{sem causa ou razão | sem motivo}
  \end{phonetics}
\end{entry}

\begin{entry}{无骨}{4,9}[Radicais ⽆、⾻]
  \begin{phonetics}{无骨}{wu2 gu3}
    \definition{adj.}{desossado}
  \end{phonetics}
\end{entry}

\begin{entry}{无敌}{4,10}[Radicais ⽆、⾆]
  \begin{phonetics}{无敌}{wu2di2}
    \definition{adj.}{invencível | inigualável}
  \end{phonetics}
\end{entry}

\begin{entry}{无氧}{4,10}[Radicais ⽆、⽓]
  \begin{phonetics}{无氧}{wu2yang3}
    \definition{adj.}{anaeróbico}
  \end{phonetics}
\end{entry}

\begin{entry}{无聊}{4,11}[Radicais ⽆、⽿]
  \begin{phonetics}{无聊}{wu2liao2}[][HSK 4]
    \definition{adj.}{entediado; aborrecido; sentir-se desinteressado porque não há nada para fazer | tolo; bobo; sem sentido; descreve palavras ou coisas ditas ou feitas como sem sentido e irritantes; descreve pessoas ou coisas como sem sentido e pouco atraentes}
  \end{phonetics}
\end{entry}

\begin{entry}{无数}{4,13}[Radicais ⽆、⽁]
  \begin{phonetics}{无数}{wu2shu4}[][HSK 4]
    \definition{adj.}{incontável; inumerável | inseguro; incerto; não conhecer a história ou os detalhes internos; não ter certeza}
  \end{phonetics}
\end{entry}

\begin{entry}{日}{4}[Kangxi 72][Radical ⽇]
  \begin{phonetics}{日}{ri4}[][HSK 1]
    \definition*{s.}{Japão, abreviação de~日本}
    \definition{clas.}{dia (mais usado em escrita) | data, dia do mês}
    \seeref{日本}{ri4ben3}
  \end{phonetics}
\end{entry}

\begin{entry}{日子}{4,3}[Radicais ⽇、⼦]
  \begin{phonetics}{日子}{ri4zi5}[][HSK 2]
    \definition{s.}{dia | uma data (calendário) | dias de vida de alguém}
  \end{phonetics}
\end{entry}

\begin{entry}{日历}{4,4}[Radicais ⽇、⼚]
  \begin{phonetics}{日历}{ri4li4}[][HSK 4]
    \definition[张,本]{s.}{caledário; livro com o ano, mês, dia, semana, termo solar, aniversário, etc. registrados, um livro por ano, uma página por dia, aberto diariamente}
  \end{phonetics}
\end{entry}

\begin{entry}{日出}{4,5}[Radicais ⽇、⼐]
  \begin{phonetics}{日出}{ri4chu1}
    \definition{s.}{nascer do sol}
  \seealsoref{夕阳}{xi1yang2}
  \end{phonetics}
\end{entry}

\begin{entry}{日本}{4,5}[Radicais ⽇、⽊]
  \begin{phonetics}{日本}{ri4ben3}
    \definition*{s.}{Japão}
  \end{phonetics}
\end{entry}

\begin{entry}{日本人}{4,5,2}[Radicais ⽇、⽊、⼈]
  \begin{phonetics}{日本人}{ri4ben3ren2}
    \definition{s.}{japonês | pessoa ou povo do Japão}
  \end{phonetics}
\end{entry}

\begin{entry}{日记}{4,5}[Radicais ⽇、⾔]
  \begin{phonetics}{日记}{ri4ji4}[][HSK 4]
    \definition[本,篇,册]{s.}{diário; artigo que registra eventos e pensamentos diários}
  \end{phonetics}
\end{entry}

\begin{entry}{日光灯}{4,6,6}[Radicais ⽇、⼉、⽕]
  \begin{phonetics}{日光灯}{ri4guang1deng1}
    \definition{s.}{lâmpada fluorescente}
  \end{phonetics}
\end{entry}

\begin{entry}{日报}{4,7}[Radicais ⽇、⼿]
  \begin{phonetics}{日报}{ri4 bao4}[][HSK 2]
    \definition[张]{s.}{diário | jornal diários}
  \end{phonetics}
\end{entry}

\begin{entry}{日常}{4,11}[Radicais ⽇、⼱]
  \begin{phonetics}{日常}{ri4chang2}[][HSK 3]
    \definition{adv.}{usual; diário; cotidiano; dia a dia}
  \end{phonetics}
\end{entry}

\begin{entry}{日期}{4,12}[Radicais ⽇、⽉]
  \begin{phonetics}{日期}{ri4qi1}[][HSK 1]
    \definition{s.}{data}
  \end{phonetics}
\end{entry}

\begin{entry}{月}{4}[Kangxi 74][Radical ⽉]
  \begin{phonetics}{月}{yue4}[][HSK 1]
    \definition[个,轮]{s.}{mês}
  \end{phonetics}
\end{entry}

\begin{entry}{月月}{4,4}[Radicais ⽉、⽉]
  \begin{phonetics}{月月}{yue4yue4}
    \definition{adv.}{todo mês}
  \end{phonetics}
\end{entry}

\begin{entry}{月份}{4,6}[Radicais ⽉、⼈]
  \begin{phonetics}{月份}{yue4 fen4}[][HSK 2]
    \definition{s.}{mês}
  \end{phonetics}
\end{entry}

\begin{entry}{月径}{4,8}[Radicais ⽉、⼻]
  \begin{phonetics}{月径}{yue4jing4}
    \definition{s.}{diâmetro da lua | diâmetro da órbita da lua | caminho iluminado pela lua}
  \end{phonetics}
\end{entry}

\begin{entry}{月亮}{4,9}[Radicais ⽉、⼇]
  \begin{phonetics}{月亮}{yue4liang5}[][HSK 2]
    \definition{s.}{lua}
  \end{phonetics}
\end{entry}

\begin{entry}{月相}{4,9}[Radicais ⽉、⽬]
  \begin{phonetics}{月相}{yue4xiang4}
    \definition{s.}{fases da lua, a saber: lua nova 朔, lua crescente 上弦, lua cheia 望 e lua minguante 下弦}
  \end{phonetics}
\end{entry}

\begin{entry}{月饼}{4,9}[Radicais ⽉、⾷]
  \begin{phonetics}{月饼}{yue4bing3}
    \definition[张]{s.}{bolo da lua}
  \end{phonetics}
\end{entry}

\begin{entry}{月球}{4,11}[Radicais ⽉、⽟]
  \begin{phonetics}{月球}{yue4qiu2}
    \definition{s.}{a lua}
  \end{phonetics}
\end{entry}

\begin{entry}{月壤}{4,20}[Radicais ⽉、⼟]
  \begin{phonetics}{月壤}{yue4rang3}
    \definition{s.}{solo lunar}
  \end{phonetics}
\end{entry}

\begin{entry}{木头}{4,5}[Radicais ⽊、⼤]
  \begin{phonetics}{木头}{mu4tou5}[][HSK 3]
    \definition{adj.}{estúpido; cabeça-dura}
    \definition[块,根]{s.}{tronco; madeira; viga; prancha}
  \end{phonetics}
\end{entry}

\begin{entry}{木偶}{4,11}[Radicais ⽊、⼈]
  \begin{phonetics}{木偶}{mu4'ou3}
    \definition{s.}{fantoche, marionete}
  \end{phonetics}
\end{entry}

\begin{entry}{歹徒}{4,10}[Radicais ⽍、⼻]
  \begin{phonetics}{歹徒}{dai3tu2}
    \definition{s.}{malfeitor | gangster | bandido}
  \end{phonetics}
\end{entry}

\begin{entry}{比}{4}[Kangxi 81][Radical ⽐]
  \begin{phonetics}{比}{bi3}[][HSK 1]
    \definition*{s.}{Bélgica, abreviação de 比利时}
    \definition{part.}{partícula usada para comparação (superioridade)}
    \definition{prep.}{que | do que | (seguido por um substantivo e adjetivo) mais \{adj.\} do que \{s.\}}
    \definition{s.}{razão (taxa)}
    \definition{v.}{comparar | contrastar | gesticular (com as mãos)}
    \seeref{比利时}{bi3li4shi2}
  \end{phonetics}
\end{entry}

\begin{entry}{比分}{4,4}[Radicais ⽐、⼑]
  \begin{phonetics}{比分}{bi3 fen1}[][HSK 4]
    \definition{s.}{pontuação; comparação de pontuações entre as duas equipes em uma partida}
  \end{phonetics}
\end{entry}

\begin{entry}{比亚迪}{4,6,8}[Radicais ⽐、⼆、⾡]
  \begin{phonetics}{比亚迪}{bi3ya4di2}
    \definition*{s.}{Montadora BYD}
  \end{phonetics}
\end{entry}

\begin{entry}{比如}{4,6}[Radicais ⽐、⼥]
  \begin{phonetics}{比如}{bi3ru2}[][HSK 2]
    \definition{conj.}{por exemplo | como}
  \end{phonetics}
\end{entry}

\begin{entry}{比如说}{4,6,9}[Radicais ⽐、⼥、⾔]
  \begin{phonetics}{比如说}{bi3 ru2 shuo1}[][HSK 2]
    \definition{adv.}{por exemplo}
  \end{phonetics}
\end{entry}

\begin{entry}{比利时}{4,7,7}[Radicais ⽐、⼑、⽇]
  \begin{phonetics}{比利时}{bi3li4shi2}
    \definition*{s.}{Bélgica}
  \end{phonetics}
\end{entry}

\begin{entry}{比例}{4,8}[Radicais ⽐、⼈]
  \begin{phonetics}{比例}{bi3li4}[][HSK 3]
    \definition{s.}{escala | razão | proporção}
  \end{phonetics}
\end{entry}

\begin{entry}{比拼}{4,9}[Radicais ⽐、⼿]
  \begin{phonetics}{比拼}{bi3pin1}
    \definition{s.}{concurso}
    \definition{v.}{competir ferozmente}
  \end{phonetics}
\end{entry}

\begin{entry}{比较}{4,10}[Radicais ⽐、⾞]
  \begin{phonetics}{比较}{bi3jiao4}[][HSK 3]
    \definition{adv.}{comparativamente | relativamente}
    \definition{s.}{comparação}
    \definition{v.}{comparar}
  \end{phonetics}
\end{entry}

\begin{entry}{比萨饼}{4,11,9}[Radicais ⽐、⾋、⾷]
  \begin{phonetics}{比萨饼}{bi3sa4bing3}
    \definition[张]{s.}{pizza}
  \end{phonetics}
\end{entry}

\begin{entry}{比赛}{4,14}[Radicais ⽐、⾙]
  \begin{phonetics}{比赛}{bi3sai4}[][HSK 3]
    \definition[场,次]{s.}{competição | concurso}
    \definition{v.}{competir}
  \end{phonetics}
\end{entry}

\begin{entry}{毛}{4}[Kangxi 82][Radical ⽑]
  \begin{phonetics}{毛}{mao2}[][HSK 1,3]
    \definition*{s.}{sobrenome Mao}
    \definition{adj.}{bronco; bruto; semi-acabado | grosseiro | pequeno | descuidado; precipitado; impetuoso | agitado; assustado; amedrontado}
    \definition{clas.}{1 mao corresponde a 10 centavos}
    \definition[根]{s.}{cabelo; penugem; pena; pele | mofo; bolor | mao, uma unidade fracionária de dinheiro na China; uma moeda de dez centavos | lã}
    \definition{v.}{depreciar; desvalorizar; rebaixar
ficar bravo; explodir}
  \end{phonetics}
\end{entry}

\begin{entry}{毛巾}{4,3}[Radicais ⽑、⼱]
  \begin{phonetics}{毛巾}{mao2jin1}[][HSK 4]
    \definition[条]{s.}{toalha; toalha de banho}
  \end{phonetics}
\end{entry}

\begin{entry}{毛衣}{4,6}[Radicais ⽑、⾐]
  \begin{phonetics}{毛衣}{mao2 yi1}[][HSK 4]
    \definition[件]{s.}{suéter; blusa feita de lã}
  \end{phonetics}
\end{entry}

\begin{entry}{毛病}{4,10}[Radicais ⽑、⽧]
  \begin{phonetics}{毛病}{mao2bing4}[][HSK 3]
    \definition[个]{s.}{doença ou deficiência | problema; fracasso | mau hábito; deficiência}
  \end{phonetics}
\end{entry}

\begin{entry}{气}{4}[Kangxi 84][Radical ⽓]
  \begin{phonetics}{气}{qi4}[][HSK 2]
    \definition[口]{s.}{gás | ar | respiração | clima | cheiro | odor | espírito | moral | ares | maneira | estilo | insulto | intimidação | energia vital | energia da vida}
    \definition{v.}{ficar bravo | ficar enfurecido | irritar | enfurecer}
  \end{phonetics}
\end{entry}

\begin{entry}{气质}{4,8}[Radicais ⽓、⾙]
  \begin{phonetics}{气质}{qi4zhi4}
    \definition{s.}{traços de personalidade, temperamento, disposição | aura, ar, sentimento, \emph{vibe} | refinamento, sofisticação, classe}
  \end{phonetics}
\end{entry}

\begin{entry}{气候}{4,10}[Radicais ⽓、⼈]
  \begin{phonetics}{气候}{qi4hou4}[][HSK 3]
    \definition[种]{s.}{clima; tempo
tendência; situação
resultado; efeito; conquista}
  \end{phonetics}
\end{entry}

\begin{entry}{气球}{4,11}[Radicais ⽓、⽟]
  \begin{phonetics}{气球}{qi4qiu2}[][HSK 4]
    \definition{s.}{balão; bolas feitas de borracha, plástico, etc., que podem ser aumentadas soprando ar nelas e podem ser usadas como brinquedos, decorações ou meios de transporte}
  \end{phonetics}
\end{entry}

\begin{entry}{气温}{4,12}[Radicais ⽓、⽔]
  \begin{phonetics}{气温}{qi4 wen1}[][HSK 2]
    \definition[个]{s.}{temperatura do ar}
  \end{phonetics}
\end{entry}

\begin{entry}{水}{4}[Kangxi 85][Radical ⽔]
  \begin{phonetics}{水}{shui3}[][HSK 1]
    \definition*{s.}{sobrenome Shui}
    \definition{clas.}{para número de lavagens}
    \definition{s.}{água | líquido | encargos ou receitas adicionais}
  \end{phonetics}
\end{entry}

\begin{entry}{水平}{4,5}[Radicais ⽔、⼲]
  \begin{phonetics}{水平}{shui3ping2}[][HSK 2]
    \definition{s.}{nível (de realização, etc.) | padrão | nível horizontal}
  \end{phonetics}
\end{entry}

\begin{entry}{水平以下}{4,5,4,3}[Radicais ⽔、⼲、⼈、⼀]
  \begin{phonetics}{水平以下}{shui3ping2 yi3xia4}
    \definition{s.}{sub-nível}
  \end{phonetics}
\end{entry}

\begin{entry}{水平尺}{4,5,4}[Radicais ⽔、⼲、⼫]
  \begin{phonetics}{水平尺}{shui3ping2chi3}
    \definition{s.}{nível espiritual}
  \end{phonetics}
\end{entry}

\begin{entry}{水平仪}{4,5,5}[Radicais ⽔、⼲、⼈]
  \begin{phonetics}{水平仪}{shui3ping2yi2}
    \definition{s.}{nível (dispositivo para determinar horizontal) | nível espiritual | nível de topógrafo}
  \end{phonetics}
\end{entry}

\begin{entry}{水平视差}{4,5,8,9}[Radicais ⽔、⼲、⾒、⼯]
  \begin{phonetics}{水平视差}{shui3ping2 shi4cha1}
    \definition{s.}{paralaxe horizontal}
  \end{phonetics}
\end{entry}

\begin{entry}{水平度}{4,5,9}[Radicais ⽔、⼲、⼴]
  \begin{phonetics}{水平度}{shui3ping2 du4}
    \definition{s.}{nivelamento}
  \end{phonetics}
\end{entry}

\begin{entry}{水平轴}{4,5,9}[Radicais ⽔、⼲、⾞]
  \begin{phonetics}{水平轴}{shui3ping2zhou2}
    \definition{s.}{eixo horizontal}
  \end{phonetics}
\end{entry}

\begin{entry}{水平面}{4,5,9}[Radicais ⽔、⼲、⾯]
  \begin{phonetics}{水平面}{shui3ping2mian4}
    \definition{s.}{plano horizontal | nível-da-água | superfície horizontal}
  \end{phonetics}
\end{entry}

\begin{entry}{水边}{4,5}[Radicais ⽔、⾡]
  \begin{phonetics}{水边}{shui3bian1}
    \definition{s.}{beira d'água | beira-mar | costa (de mar, lago ou rio)}
  \end{phonetics}
\end{entry}

\begin{entry}{水污染}{4,6,9}[Radicais ⽔、⽔、⽊]
  \begin{phonetics}{水污染}{shui3wu1ran3}
    \definition{s.}{poluição da água}
  \end{phonetics}
\end{entry}

\begin{entry}{水灵}{4,7}[Radicais ⽔、⽕]
  \begin{phonetics}{水灵}{shui3ling2}
    \definition{adj.}{cheio de vida (sobre uma pessoa, etc.) | úmido e brilhante (sobre os olhos) | fresco (sobre frutas, etc.) | brilhante | aparência saudável}
  \end{phonetics}
\end{entry}

\begin{entry}{水果}{4,8}[Radicais ⽔、⽊]
  \begin{phonetics}{水果}{shui3guo3}[][HSK 1]
    \definition[个]{s.}{fruta}
  \end{phonetics}
\end{entry}

\begin{entry}{水波}{4,8}[Radicais ⽔、⽔]
  \begin{phonetics}{水波}{shui3bo1}
    \definition{s.}{ondulação (na água) | onda}
  \end{phonetics}
\end{entry}

\begin{entry}{水饺}{4,9}[Radicais ⽔、⾷]
  \begin{phonetics}{水饺}{shui3jiao3}
    \definition{s.}{\emph{dumplings} | pastéis chineses cozidos}
  \end{phonetics}
\end{entry}

\begin{entry}{水瓶}{4,10}[Radicais ⽔、⽡]
  \begin{phonetics}{水瓶}{shui3 ping2}
    \definition{s.}{garrada de água}
  \end{phonetics}
\end{entry}

\begin{entry}{水培}{4,11}[Radicais ⽔、⼟]
  \begin{phonetics}{水培}{shui3pei2}
    \definition{v.}{cultivar plantas hidroponicamente}
  \end{phonetics}
\end{entry}

\begin{entry}{水豚}{4,11}[Radicais ⽔、⾗]
  \begin{phonetics}{水豚}{shui3tun2}
    \definition{s.}{capivara}
  \end{phonetics}
\end{entry}

\begin{entry}{水路}{4,13}[Radicais ⽔、⾜]
  \begin{phonetics}{水路}{shui3lu4}
    \definition{s.}{hidrovia}
  \end{phonetics}
\end{entry}

\begin{entry}{水槽}{4,15}[Radicais ⽔、⽊]
  \begin{phonetics}{水槽}{shui3cao2}
    \definition{s.}{pia (de cozinha)}
  \end{phonetics}
\end{entry}

\begin{entry}{火}{4}[Kangxi 86][Radical ⽕]
  \begin{phonetics}{火}{huo3}[][HSK 3,4]
    \definition*{s.}{sobrenome Huo}
    \definition{adj.}{ardente; flamejante; vermelho como fogo | efervescente; próspero}
    \definition{adv.}{urgentemente}
    \definition{clas.}{para unidades militares (antigo)}
    \definition{s.}{fogo | armas de fogo; munições | calor interno (uma das seis causas de doenças) | a ação de lutar}
    \definition{v.}{ficar com raiva; perder a paciência}
  \end{phonetics}
\end{entry}

\begin{entry}{火车}{4,4}[Radicais ⽕、⾞]
  \begin{phonetics}{火车}{huo3 che1}[][HSK 1]
    \definition[列,节,班,趟]{s.}{trem | comboio}
  \end{phonetics}
\end{entry}

\begin{entry}{火车司机}{4,4,5,6}[Radicais ⽕、⾞、⼝、⽊]
  \begin{phonetics}{火车司机}{huo3che1 si1ji1}
    \definition{s.}{maquinista de trem}
  \end{phonetics}
\end{entry}

\begin{entry}{火柴}{4,10}[Radicais ⽕、⽊]
  \begin{phonetics}{火柴}{huo3chai2}
    \definition[根,盒]{s.}{fósforo (palito de fósforo)}
  \end{phonetics}
\end{entry}

\begin{entry}{火海}{4,10}[Radicais ⽕、⽔]
  \begin{phonetics}{火海}{huo3hai3}
    \definition{s.}{um mar de chamas}
  \end{phonetics}
\end{entry}

\begin{entry}{父母}{4,5}[Radicais ⽗、⽏]
  \begin{phonetics}{父母}{fu4 mu3}[][HSK 3]
    \definition{s.}{pai e mãe; pais}
  \end{phonetics}
\end{entry}

\begin{entry}{父母亲}{4,5,9}[Radicais ⽗、⽏、⼇]
  \begin{phonetics}{父母亲}{fu4mu3qin1}
    \definition{s.}{pais}
  \end{phonetics}
\end{entry}

\begin{entry}{父亲}{4,9}[Radicais ⽗、⼇]
  \begin{phonetics}{父亲}{fu4qin1}[][HSK 3]
    \definition[个,位]{s.}{pai}
  \end{phonetics}
\end{entry}

\begin{entry}{片}{4}[Kangxi 91][Radical ⽚]
  \begin{phonetics}{片}{pian4}[][HSK 2]
    \definition{adj.}{parcial | incompleto | que só tem um lado}
    \definition{clas.}{para CDs, filmes, DVDs, etc. | para fatias, comprimidos, extensão de terra, área de água | usado com numeral~一:~para  cenário, cena, sentimento, atmosfera, som etc.}
    \definition{s.}{uma fatia | floco | filme | pedaço fino}
    \definition{v.}{fatiar | esculpir fino}
  \end{phonetics}
\end{entry}

\begin{entry}{片面}{4,9}[Radicais ⽚、⾯]
  \begin{phonetics}{片面}{pian4mian4}[][HSK 4]
    \definition{adj.}{unilateral (em oposição a ``全面'')}
  \seealsoref{全面}{quan2mian4}
  \end{phonetics}
\end{entry}

\begin{entry}{牙}{4}[Kangxi 92][Radical ⽛]
  \begin{phonetics}{牙}{ya2}
    \definition[颗]{s.}{dente | marfim}
  \end{phonetics}
\end{entry}

\begin{entry}{牙行}{4,6}[Radicais ⽛、⾏]
  \begin{phonetics}{牙行}{ya2hang2}
    \definition{s.}{corretor | \emph{broker}}
  \end{phonetics}
\end{entry}

\begin{entry}{牙医}{4,7}[Radicais ⽛、⼖]
  \begin{phonetics}{牙医}{ya2yi1}
    \definition{s.}{dentista}
  \end{phonetics}
\end{entry}

\begin{entry}{牙刷}{4,8}[Radicais ⽛、⼑]
  \begin{phonetics}{牙刷}{ya2shua1}
    \definition[把]{s.}{escova de dentes}
  \end{phonetics}
\end{entry}

\begin{entry}{牙线}{4,8}[Radicais ⽛、⽷]
  \begin{phonetics}{牙线}{ya2xian4}
    \definition[条]{s.}{fio dental}
  \end{phonetics}
\end{entry}

\begin{entry}{牙齿}{4,8}[Radicais ⽛、⿒]
  \begin{phonetics}{牙齿}{ya2chi3}
    \definition{adv.}{dental}
    \definition[颗]{s.}{dente}
  \end{phonetics}
\end{entry}

\begin{entry}{牙膏}{4,14}[Radicais ⽛、⾁]
  \begin{phonetics}{牙膏}{ya2gao1}
    \definition[管]{s.}{pasta de dente}
  \end{phonetics}
\end{entry}

\begin{entry}{牛}{4}[Kangxi 93][Radical ⽜]
  \begin{phonetics}{牛}{niu2}[][HSK 3]
    \definition*{s.}{sobrenome Niu}
    \definition{adj.}{muito capaz ou bom
teimoso; arrogante}
    \definition{clas.}{Newton (medida física de força)}
    \definition[头]{s.}{gado; boi | niu (nona das vinte e oito constelações em que a esfera celeste foi dividida, consistindo de seis estrelas, três em Áries e três em Sagitário)}
  \end{phonetics}
\end{entry}

\begin{entry}{牛人}{4,2}[Radicais ⽜、⼈]
  \begin{phonetics}{牛人}{niu2ren2}
    \definition{s.}{(coloquial) o cara | verdadeiro especialista | \emph{badass}}
  \end{phonetics}
\end{entry}

\begin{entry}{牛仔裤}{4,5,12}[Radicais ⽜、⼈、⾐]
  \begin{phonetics}{牛仔裤}{niu2zai3ku4}
    \definition[条]{s.}{calça de ganga, jeans}
  \end{phonetics}
\end{entry}

\begin{entry}{牛奶}{4,5}[Radicais ⽜、⼥]
  \begin{phonetics}{牛奶}{niu2nai3}[][HSK 1]
    \definition[瓶,杯]{s.}{leite de vaca}
  \end{phonetics}
\end{entry}

\begin{entry}{牛肉}{4,6}[Radicais ⽜、⾁]
  \begin{phonetics}{牛肉}{niu2rou4}
    \definition{s.}{carne de vaca | bife}
  \end{phonetics}
\end{entry}

\begin{entry}{牛郎织女}{4,8,8,3}[Radicais ⽜、⾢、⽷、⼥]
  \begin{phonetics}{牛郎织女}{niu2lang2zhi1nv3}
    \definition*{s.}{Vaqueiro e Tecelã (personagens de contos folclóricos) | amantes separados | Altair e Vega (estrelas)}
  \end{phonetics}
\end{entry}

\begin{entry}{牛顿}{4,10}[Radicais ⽜、⾴]
  \begin{phonetics}{牛顿}{niu2dun4}
    \definition*{s.}{Newton (nome) | newton (N, unidade de força do SI)}
  \end{phonetics}
\end{entry}

\begin{entry}{犬}{4}[Kangxi 94][Radical ⽝]
  \begin{phonetics}{犬}{quan3}
    \definition{s.}{cachorro}
  \end{phonetics}
\end{entry}

\begin{entry}{王}{4}[Radical ⽟]
  \begin{phonetics}{王}{wang2}[][HSK 4]
    \definition*{s.}{sobrenome Wang}
    \definition{adj.}{grande; ótimo; honoríficos antigos para avós}
    \definition{s.}{rei; monarca; imperador; governante supremo de uma monarquia | cabeça; chefe; líder | o primeiro, maior ou mais forte de seu tipo | duque; príncipe; o título mais alto da sociedade feudal após a dinastia Han}
  \end{phonetics}
  \begin{phonetics}{王}{wang4}
    \definition{v.}{reger; governar; reinar; dominar}
  \end{phonetics}
\end{entry}

\begin{entry}{王五}{4,4}[Radicais ⽟、⼆]
  \begin{phonetics}{王五}{wang2wu3}
    \definition{s.}{Wang Wu | Zé Ninguém | nome para uma pessoa não especificada, 3 de 3}
  \seealsoref{李四}{li3si4}
  \seealsoref{张三}{zhang1san1}
  \end{phonetics}
\end{entry}

\begin{entry}{王朝}{4,12}[Radicais ⽟、⽉]
  \begin{phonetics}{王朝}{wang2chao2}
    \definition{s.}{dinastia}
  \end{phonetics}
\end{entry}

\begin{entry}{瓦}{4}[Kangxi 98][Radical ⽡]
  \begin{phonetics}{瓦}{wa3}
    \definition{s.}{telha | abreviação de 瓦特}
    \seeref{瓦特}{wa3te4}
  \end{phonetics}
\end{entry}

\begin{entry}{瓦努阿图}{4,7,7,8}[Radicais ⽡、⼒、⾩、⼞]
  \begin{phonetics}{瓦努阿图}{wa3nu3'a1tu2}
    \definition*{s.}{Vanuatu, país do sudoeste do Oceano Pacífico}
  \end{phonetics}
\end{entry}

\begin{entry}{瓦特}{4,10}[Radicais ⽡、⽜]
  \begin{phonetics}{瓦特}{wa3te4}
    \definition{s.}{(empréstimo linguístico) watt | medida de potência}
  \end{phonetics}
\end{entry}

\begin{entry}{艺人}{4,2}[Radicais ⾋、⼈]
  \begin{phonetics}{艺人}{yi4ren2}
    \definition{s.}{artista | ator}
  \end{phonetics}
\end{entry}

\begin{entry}{艺术}{4,5}[Radicais ⾋、⽊]
  \begin{phonetics}{艺术}{yi4shu4}[][HSK 3]
    \definition{adj.}{artístico,único e lindo}
    \definition[个]{s.}{arte; literatura e arte | habilidade; arte; ofício; abordagem criativa}
  \end{phonetics}
\end{entry}

\begin{entry}{见}{4}[Radical ⾒]
  \begin{phonetics}{见}{jian4}[][HSK 1]
    \definition{s.}{opinião, visão}
    \definition{v.}{ver | entrevistar | encontrar alguém | parecer (ser alguma coisa)}
  \end{phonetics}
  \begin{phonetics}{见}{xian4}
    \definition{v.}{aparecer | também escrito como 现}
    \seeref{现}{xian4}
  \end{phonetics}
\end{entry}

\begin{entry}{见过}{4,6}[Radicais ⾒、⾡]
  \begin{phonetics}{见过}{jian4 guo4}[][HSK 2]
    \definition{s.}{visto (ver)}
  \end{phonetics}
\end{entry}

\begin{entry}{见到}{4,8}[Radicais ⾒、⼑]
  \begin{phonetics}{见到}{jian4 dao4}[][HSK 2]
    \definition{v.}{ver | esbarrar em | encontrar-se com}
  \end{phonetics}
\end{entry}

\begin{entry}{见面}{4,9}[Radicais ⾒、⾯]
  \begin{phonetics}{见面}{jian4 mian4}[][HSK 1]
    \definition{v.+compl.}{encontrar-se com alguém | ver alguém face-a-face}
  \end{phonetics}
\end{entry}

\begin{entry}{计划}{4,6}[Radicais ⾔、⼑]
  \begin{phonetics}{计划}{ji4hua4}[][HSK 2]
    \definition[个,项]{s.}{plano | projeto | programa}
    \definition{v.}{planejar | mapear}
  \end{phonetics}
\end{entry}

\begin{entry}{计算}{4,14}[Radicais ⾔、⽵]
  \begin{phonetics}{计算}{ji4suan4}[][HSK 3]
    \definition{v.}{contar; calcular; computar; enumerar | planejar; considerar | conspirar secretamente contra os outros}
  \end{phonetics}
\end{entry}

\begin{entry}{计算机}{4,14,6}[Radicais ⾔、⽵、⽊]
  \begin{phonetics}{计算机}{ji4 suan4 ji1}[][HSK 2]
    \definition[部,台]{s.}{computador | calculadora}
  \end{phonetics}
\end{entry}

\begin{entry}{订}{4}[Radical ⾔]
  \begin{phonetics}{订}{ding4}[][HSK 3]
    \definition{v.}{concluir; elaborar; concordar com |assinar (um jornal, etc.); reservar (assentos, ingressos, etc.); encomendar (mercadorias, etc.) |fazer correções; revisar | grampear junto; unir}
  \end{phonetics}
\end{entry}

\begin{entry}{认为}{4,4}[Radicais ⾔、⼂]
  \begin{phonetics}{认为}{ren4wei2}[][HSK 2]
    \definition{v.}{pensar | considerar | segurar | julgar}
  \end{phonetics}
\end{entry}

\begin{entry}{认出}{4,5}[Radicais ⾔、⼐]
  \begin{phonetics}{认出}{ren4 chu1}[][HSK 3]
    \definition{v.}{reconhecer; decifrar; identificar}
  \end{phonetics}
\end{entry}

\begin{entry}{认可}{4,5}[Radicais ⾔、⼝]
  \begin{phonetics}{认可}{ren4ke3}[][HSK 3]
    \definition{v.}{aceitar; aprovar; confirmar; dar força legal a | permitir; concordar}
  \end{phonetics}
\end{entry}

\begin{entry}{认识}{4,7}[Radicais ⾔、⾔]
  \begin{phonetics}{认识}{ren4shi5}[][HSK 1]
    \definition{s.}{conhecimento | saber | entendimento}
    \definition{v.}{estar familiarizado com | conhecer alguém | saber | reconhecer}
  \end{phonetics}
\end{entry}

\begin{entry}{认真}{4,10}[Radicais ⾔、⼗]
  \begin{phonetics}{认真}{ren4zhen1}[][HSK 1]
    \definition{adj.}{sério | consciencioso}
    \definition{adv.}{seriamente}
    \definition{v.}{levar a sério}
  \end{phonetics}
\end{entry}

\begin{entry}{认得}{4,11}[Radicais ⾔、⼻]
  \begin{phonetics}{认得}{ren4 de5}[][HSK 3]
    \definition{v.}{saber; reconhecer}
  \end{phonetics}
\end{entry}

\begin{entry}{车}{4}[Kangxi 159][Radical ⾞]
  \begin{phonetics}{车}{che1}[][HSK 1]
    \definition*{s.}{sobrenome Che}
    \definition[辆]{s.}{carro | veículo | viatura}
  \end{phonetics}
  \begin{phonetics}{车}{ju1}
    \definition{s.}{(arcaico) carruagem de guerra | torre (no xadrez)}
  \end{phonetics}
\end{entry}

\begin{entry}{车上}{4,3}[Radicais ⾞、⼀]
  \begin{phonetics}{车上}{che1 shang4}[][HSK 1]
    \definition{adv.}{no carro | dentro do veículo}
  \end{phonetics}
\end{entry}

\begin{entry}{车子}{4,3}[Radicais ⾞、⼦]
  \begin{phonetics}{车子}{che1zi5}
    \definition{s.}{qualquer veículo (carro, bicicleta, caminhão, etc)}
  \end{phonetics}
\end{entry}

\begin{entry}{车水马龙}{4,4,3,5}[Radicais ⾞、⽔、⾺、⿓]
  \begin{phonetics}{车水马龙}{che1shui3-ma3long2}
    \definition{expr.}{tráfego engarrafado | engarrafamento | (literalmente) ``fluxo interminável de cavalos e carruagens''}
  \end{phonetics}
\end{entry}

\begin{entry}{车主}{4,5}[Radicais ⾞、⼂]
  \begin{phonetics}{车主}{che1zhu3}
    \definition{s.}{proprietário do carro}
  \end{phonetics}
\end{entry}

\begin{entry}{车次}{4,6}[Radicais ⾞、⽋]
  \begin{phonetics}{车次}{che1ci4}
    \definition{s.}{número do trem}
  \end{phonetics}
\end{entry}

\begin{entry}{车库}{4,7}[Radicais ⾞、⼴]
  \begin{phonetics}{车库}{che1ku4}
    \definition{s.}{garagem}
  \end{phonetics}
\end{entry}

\begin{entry}{车站}{4,10}[Radicais ⾞、⽴]
  \begin{phonetics}{车站}{che1 zhan4}[][HSK 1]
    \definition[处,个]{s.}{estação | ponto de ônibus}
  \end{phonetics}
\end{entry}

\begin{entry}{车票}{4,11}[Radicais ⾞、⽰]
  \begin{phonetics}{车票}{che1 piao4}[][HSK 1]
    \definition{s.}{bilhete (de ônibus, trem, metrô)}
  \end{phonetics}
\end{entry}

\begin{entry}{车辆}{4,11}[Radicais ⾞、⾞]
  \begin{phonetics}{车辆}{che1 liang4}[][HSK 2]
    \definition{s.}{veículo | carro}
  \end{phonetics}
\end{entry}

\begin{entry}{车牌}{4,12}[Radicais ⾞、⽚]
  \begin{phonetics}{车牌}{che1pai2}
    \definition{s.}{matrícula | placa de carro}
  \end{phonetics}
\end{entry}

\begin{entry}{长}{4}[Radical ⾧]
  \begin{phonetics}{长}{chang2}[][HSK 2]
    \definition{adj.}{comprido | longo}
  \end{phonetics}
  \begin{phonetics}{长}{zhang3}[][HSK 2]
    \definition{s.}{chefe | ancião}
    \definition{v.}{crescer | desenvolver | aumentar | melhorar}
  \end{phonetics}
\end{entry}

\begin{entry}{长大}{4,3}[Radicais ⾧、⼤]
  \begin{phonetics}{长大}{zhang3 da4}[][HSK 2]
    \definition{v.}{crescer | ser criado}
  \end{phonetics}
\end{entry}

\begin{entry}{长处}{4,5}[Radicais ⾧、⼡]
  \begin{phonetics}{长处}{chang2 chu4}[][HSK 3]
    \definition{s.}{força; boas qualidades; pontos fortes}
  \end{phonetics}
\end{entry}

\begin{entry}{长城}{4,9}[Radicais ⾧、⼟]
  \begin{phonetics}{长城}{chang2cheng2}[][HSK 3]
    \definition*{s.}{A Grande Muralha}
  \end{phonetics}
\end{entry}

\begin{entry}{长途}{4,10}[Radicais ⾧、⾡]
  \begin{phonetics}{长途}{chang2tu2}[][HSK 4]
    \definition{adj.}{de longa distância; longe}
    \definition[段,次,程]{s.}{longa-distância; referindo-se especificamente a chamadas telefônicas de longa distância ou ônibus de longa distância}
  \end{phonetics}
\end{entry}

\begin{entry}{长颈鹿}{4,11,11}[Radicais ⾧、⾴、⿅]
  \begin{phonetics}{长颈鹿}{chang2jing3lu4}
    \definition[只]{s.}{girafa}
  \end{phonetics}
\end{entry}

\begin{entry}{长期}{4,12}[Radicais ⾧、⽉]
  \begin{phonetics}{长期}{chang2 qi1}[][HSK 3]
    \definition{adj.}{secular; longo prazo; longo alcance; durante um longo período de tempo}
    \definition{s.}{longo prazo}
  \end{phonetics}
\end{entry}

\begin{entry}{队}{4}[Radical ⾩]
  \begin{phonetics}{队}{dui4}[][HSK 2]
    \definition[个]{s.}{esquadrão | equipe | grupo}
  \end{phonetics}
\end{entry}

\begin{entry}{队友}{4,4}[Radicais ⾩、⼜]
  \begin{phonetics}{队友}{dui4you3}
    \definition{s.}{companheiro de equipe}
  \end{phonetics}
\end{entry}

\begin{entry}{队长}{4,4}[Radicais ⾩、⾧]
  \begin{phonetics}{队长}{dui4 zhang3}[][HSK 2]
    \definition{s.}{capitão (de equipe) | líder da equipe}
  \end{phonetics}
\end{entry}

\begin{entry}{队员}{4,7}[Radicais ⾩、⼝]
  \begin{phonetics}{队员}{dui4 yuan2}[][HSK 3]
    \definition{s.}{membro da equipe}
  \end{phonetics}
\end{entry}

\begin{entry}{风}{4}[Kangxi 182][Radical ⾵]
  \begin{phonetics}{风}{feng1}[][HSK 1]
    \definition[阵,丝]{s.}{vento}
  \end{phonetics}
\end{entry}

\begin{entry}{风俗}{4,9}[Radicais ⾵、⼈]
  \begin{phonetics}{风俗}{feng1su2}[][HSK 4]
    \definition[种,个,些]{s.}{costumes; a soma de costumes sociais, maneiras, hábitos, etc., desenvolvidos ao longo do tempo.}
  \end{phonetics}
\end{entry}

\begin{entry}{风险}{4,9}[Radicais ⾵、⾩]
  \begin{phonetics}{风险}{feng1xian3}[][HSK 3]
    \definition[个,种,项,类]{s.}{risco; perigo}
  \end{phonetics}
\end{entry}

\begin{entry}{风扇}{4,10}[Radicais ⾵、⼾]
  \begin{phonetics}{风扇}{feng1shan4}
    \definition{s.}{ventilador elétrico}
  \end{phonetics}
\end{entry}

\begin{entry}{风格}{4,10}[Radicais ⾵、⽊]
  \begin{phonetics}{风格}{feng1ge2}[][HSK 4]
    \definition{s.}{modo; estilo; maneira; caráter | características das criações literárias de diferentes épocas, povos, escolas ou indivíduos em termos de conteúdo ideológico e técnicas artísticas}
  \end{phonetics}
\end{entry}

\begin{entry}{风景}{4,12}[Radicais ⾵、⽇]
  \begin{phonetics}{风景}{feng1jing3}[][HSK 4]
    \definition[种,处,道]{s.}{cenário; paisagem; cenários e vistas que podem ser apreciados, inclui paisagens, flores, árvores, edifícios e determinados fenômenos naturais}
  \end{phonetics}
\end{entry}

\begin{entry}{风筝}{4,12}[Radicais ⾵、⽵]
  \begin{phonetics}{风筝}{feng1zheng5}
    \definition{s.}{pipa | papagaio | pandorga}
  \end{phonetics}
\end{entry}

%%%%% EOF %%%%%

