%%%
%%% 4画
%%%

\section*{4画}\addcontentsline{toc}{section}{4画}

\begin{Entry}{不}{4}{⼀}
  \begin{Phonetics}{不}{bu2}[(antes de quarto tom)][HSK 1]
  \end{Phonetics}
  \begin{Phonetics}{不}{bu4}[][HSK 1]
    \definition{adv.}{(antes de verbos, adjetivos e outros advérbios; nunca antes do verbo 有) não; não vai; não quer | em algumas expressões educadas, significa que não é necessário fazer isso, o que equivale a 不用 ou 不要 | | (entre um verbo e seu complemento) não pode | usado com 就 para indicar escolha}
    \definition{part.}{no final da frase para indicar uma pergunta; (usar sozinho ou com uma partícula nas respostas) não}
    \definition{pref.}{(antes de certos substantivos para formar um adjetivo) un-; in-}
  \seealsoref{不要}{bu2 yao4}
  \seealsoref{不用}{bu2 yong4}
  \seealsoref{就}{jiu4}
  \seealsoref{有}{you3}
  \end{Phonetics}
  \begin{Phonetics}{不}{bu5}[][HSK 1]
    \definition{adv.}{não (em expressões \{v.\} + 不 + \{v.\})}
  \end{Phonetics}
\end{Entry}

\begin{Entry}{不一会儿}{4,1,6,2}{⼀、⼀、⼈、⼉}
  \begin{Phonetics}{不一会儿}{bu4 yi2 hui4r5}[][HSK 2]
    \definition{expr.}{em um momento; em pouco tempo; em breve; depois de algum tempo}
  \end{Phonetics}
\end{Entry}

\begin{Entry}{不一定}{4,1,8}{⼀、⼀、⼧}
  \begin{Phonetics}{不一定}{bu4 yi2 ding4}[][HSK 2]
    \definition{adv.}{talvez; incerto; não tenho certeza; não necessariamente assim; refere-se a algo que não pode ser determinado}
  \end{Phonetics}
\end{Entry}

\begin{Entry}{不了了之}{4,2,2,3}{⼀、⼅、⼅、⼂}
  \begin{Phonetics}{不了了之}{bu4liao3-liao3zhi1}[][HSK 7-9]
    \definition{expr.}{deixar um assunto sem solução; acabar sem nada definido; pontas soltas | resolver um assunto deixando-o sem solução; abafar; deixando-o sem solução; concluir sem uma conclusão; acabar sem nada definido; concluir sem resultado concreto (decisão); deixá-lo sem solução; deixar (um assunto) seguir seu próprio curso | deixe-o inquieto}
  \end{Phonetics}
\end{Entry}

\begin{Entry}{不久}{4,3}{⼀、⼃}
  \begin{Phonetics}{不久}{bu4 jiu3}[][HSK 2]
    \definition{adv.}{em breve; dentro em breve; num futuro próximo | logo depois; pouco tempo depois | não muito tempo (antes ou depois de algo)}
  \end{Phonetics}
\end{Entry}

\begin{Entry}{不大}{4,3}{⼀、⼤}
  \begin{Phonetics}{不大}{bu2 da4}[][HSK 1]
    \definition{adv.}{não muito (indicando um grau baixo); não demasiado | não com frequência; raramente; dificilmente}
  \end{Phonetics}
\end{Entry}

\begin{Entry}{不大离}{4,3,10}{⼀、⼤、⼇}
  \begin{Phonetics}{不大离}{bu2da4li2}
    \definition{adj.}{bem perto | quase certo | nada mal}
  \end{Phonetics}
\end{Entry}

\begin{Entry}{不已}{4,3}{⼀、⼰}
  \begin{Phonetics}{不已}{bu4yi3}[][HSK 7-9]
    \definition{v.aux.}{ser infinito; usado depois de um verbo para indicar que ele continua sem parar}
  \end{Phonetics}
\end{Entry}

\begin{Entry}{不为人知}{4,4,2,8}{⼀、⼂、⼈、⽮}
  \begin{Phonetics}{不为人知}{bu4wei2ren2zhi1}[][HSK 7-9]
    \definition{expr.}{não conhecido por ninguém | segredo | desconhecido}
  \end{Phonetics}
\end{Entry}

\begin{Entry}{不予}{4,4}{⼀、⼅}
  \begin{Phonetics}{不予}{bu4yu3}[][HSK 7-9]
    \definition{v.}{Literário: não dar; negar; recusar; não conceder}
  \end{Phonetics}
\end{Entry}

\begin{Entry}{不仅}{4,4}{⼀、⼈}
  \begin{Phonetics}{不仅}{bu4jin3}[][HSK 3]
    \definition{adv.}{não apenas (em número, quantidade ou extensão); costuma-se dizer 不仅仅}
    \definition{conj.}{não somente}
  \seealsoref{不仅仅}{bu4 jin3 jin3}
  \end{Phonetics}
\end{Entry}

\begin{Entry}{不仅仅}{4,4,4}{⼀、⼈、⼈}
  \begin{Phonetics}{不仅仅}{bu4 jin3 jin3}[][HSK 6]
    \definition{adv.}{não só; não apenas}
  \end{Phonetics}
\end{Entry}

\begin{Entry}{不以为然}{4,4,4,12}{⼀、⼈、⼂、⽕}
  \begin{Phonetics}{不以为然}{bu4yi3wei2ran2}[][HSK 7-9]
    \definition{expr.}{não aceitar como correto; objetar | desaprovar | ter exceção a}
  \end{Phonetics}
\end{Entry}

\begin{Entry}{不公}{4,4}{⼀、⼋}
  \begin{Phonetics}{不公}{bu4gong1}
    \definition{adj.}{injusto}
  \end{Phonetics}
\end{Entry}

\begin{Entry}{不太}{4,4}{⼀、⼤}
  \begin{Phonetics}{不太}{bu2 tai4}[][HSK 2]
    \definition{adv.}{não exatamente | não muito bom}
  \end{Phonetics}
\end{Entry}

\begin{Entry}{不少}{4,4}{⼀、⼩}
  \begin{Phonetics}{不少}{bu4 shao3}[][HSK 2]
    \definition{adj.}{muitos; bastante; não poucos; indica uma quantidade considerável, equivalente a muitos ou bastante}
  \end{Phonetics}
\end{Entry}

\begin{Entry}{不日}{4,4}{⼀、⽇}
  \begin{Phonetics}{不日}{bu2ri4}
    \definition{adv.}{em alguns dias}
  \end{Phonetics}
\end{Entry}

\begin{Entry}{不止}{4,4}{⼀、⽌}
  \begin{Phonetics}{不止}{bu4zhi3}[][HSK 5]
    \definition{adv.}{mais do que; não limitado a; indica mais do que esse valor ou intervalo}
    \definition{v.}{exceder; superar; não ser possível interromper a ação}
  \end{Phonetics}
\end{Entry}

\begin{Entry}{不见}{4,4}{⼀、⾒}
  \begin{Phonetics}{不见}{bu2 jian4}[][HSK 6]
    \definition{v.}{não ver; não conhecer; não encontrar | estar desaparecido; desaparecer; não consiguir encontrar algo}
  \end{Phonetics}
\end{Entry}

\begin{Entry}{不见得}{4,4,11}{⼀、⾒、⼻}
  \begin{Phonetics}{不见得}{bu2jian4de2}[][HSK 7-9]
    \definition{adv.}{não pode; não é provável; não necessariamente; é improvável que}
  \end{Phonetics}
\end{Entry}

\begin{Entry}{不计其数}{4,4,8,13}{⼀、⾔、⼋、⽁}
  \begin{Phonetics}{不计其数}{bu2 ji4 qi2 shu4}
    \definition{expr.}{seu número não pode ser contado; incontáveis; inumeráveis}
  \end{Phonetics}
\end{Entry}

\begin{Entry}{不可思议}{4,5,9,5}{⼀、⼝、⼼、⾔}
  \begin{Phonetics}{不可思议}{bu4ke3-si1yi4}[][HSK 7-9]
    \definition{expr.}{incrível; inacreditável; inimaginável; inconcebível; algo que é difícil de entender ou imaginar.}
  \end{Phonetics}
\end{Entry}

\begin{Entry}{不可避免}{4,5,16,7}{⼀、⼝、⾌、⼉}
  \begin{Phonetics}{不可避免}{bu4ke3-bi4mian3}[][HSK 7-9]
    \definition{expr.}{inevitavelmente; inescapabilidade | inevitável}
  \end{Phonetics}
\end{Entry}

\begin{Entry}{不对}{4,5}{⼀、⼨}
  \begin{Phonetics}{不对}{bu2 dui4}[][HSK 1]
    \definition{adj.}{incorreto; errado | anormal; anômalo; estranho | desarmonia; incompatibilidade; discórdia}
  \end{Phonetics}
\end{Entry}

\begin{Entry}{不平}{4,5}{⼀、⼲}
  \begin{Phonetics}{不平}{bu4ping2}[][HSK 7-9]
    \definition{adj.}{irregular; não nivelado | injusto}
    \definition[些,点]{s.}{injustiça; deslealdade | ressentimento; queixa}
    \definition{v.}{estar indignado; estar ressentido | estar irregular; não estar nivelado; não estar liso | ser injusto}
  \end{Phonetics}
\end{Entry}

\begin{Entry}{不必}{4,5}{⼀、⼼}
  \begin{Phonetics}{不必}{bu2 bi4}[][HSK 3]
    \definition{adv.}{não precisa; não tem que; indica que não é necessário em termos de razão ou emoção}
  \end{Phonetics}
\end{Entry}

\begin{Entry}{不正之风}{4,5,3,4}{⼀、⽌、⼂、⾵}
  \begin{Phonetics}{不正之风}{bu2zheng4zhi1feng1}[][HSK 7-9]
    \definition[种]{expr.}{tendências prejudiciais; tendência doentia; práticas inadequadas; maus estilos de trabalho; ventos malignos; tendências doentias | tendência doentia; negligência médica}
  \end{Phonetics}
\end{Entry}

\begin{Entry}{不用}{4,5}{⼀、⽤}
  \begin{Phonetics}{不用}{bu2 yong4}[][HSK 1]
    \definition{v.}{não precisar; não ter necessidade; indicar que, na verdade, não é necessário}
  \end{Phonetics}
\end{Entry}

\begin{Entry}{不用说}{4,5,9}{⼀、⽤、⾔}
  \begin{Phonetics}{不用说}{bu2yong4shuo1}[][HSK 7-9]
    \definition{v.}{desnecessário dizer; ser evidente}
  \end{Phonetics}
\end{Entry}

\begin{Entry}{不由自主}{4,5,6,5}{⼀、⽥、⾃、⼂}
  \begin{Phonetics}{不由自主}{bu4you2zi4zhu3}[][HSK 7-9]
    \definition{expr.}{não pode ajudar; involuntariamente | além do controle de alguém; apesar de si mesmo; incapaz de se controlar | não posso evitar}
  \end{Phonetics}
\end{Entry}

\begin{Entry}{不由得}{4,5,11}{⼀、⽥、⼻}
  \begin{Phonetics}{不由得}{bu4you2de5}[][HSK 7-9]
    \definition{adv.}{involuntariamente; como uma consequência necessária}
    \definition{v.}{não pode deixar de; não pode se abster de; não pode evitar (fazer algo); não permitir que certos resultados ocorram em determinadas circunstâncias}
  \end{Phonetics}
\end{Entry}

\begin{Entry}{不亚于}{4,6,3}{⼀、⼆、⼆}
  \begin{Phonetics}{不亚于}{bu2ya4yu2}[][HSK 7-9]
    \definition{v.}{ser tão bom quanto; não ser inferior a}
  \end{Phonetics}
\end{Entry}

\begin{Entry}{不亦乐乎}{4,6,5,5}{⼀、⼇、⼃、⼃}
  \begin{Phonetics}{不亦乐乎}{bu2yi4le4hu1}[][HSK 7-9]
    \definition{expr.}{terrivelmente; extremamente; Não é um grande prazer?; Que delícia seria se\dots; o significado original é "Não é também muito feliz?" (visto em "Os Analectos de Confúcio·Aprendizado"), e agora é frequentemente usado para expressar atingir o extremo | (como um complemento após 忙得) extremamente; terrivelmente | não é um grande prazer}
  \seealsoref{忙得}{mang2de2}
  \end{Phonetics}
\end{Entry}

\begin{Entry}{不光}{4,6}{⼀、⼉}
  \begin{Phonetics}{不光}{bu4 guang1}[][HSK 3]
    \definition{adv.}{não é o único; não apenas; não só; indica que excede uma determinada quantidade ou faixa}
    \definition{conj.}{não somente; não só}
  \end{Phonetics}
\end{Entry}

\begin{Entry}{不再}{4,6}{⼀、⼌}
  \begin{Phonetics}{不再}{bu2 zai4}[][HSK 6]
    \definition{adv.}{não mais; não repita uma segunda vez}
    \definition{v.}{ter ido embora; não retornar; não aparecer; não existir mais}
  \end{Phonetics}
\end{Entry}

\begin{Entry}{不同}{4,6}{⼀、⼝}
  \begin{Phonetics}{不同}{bu4 tong2}[][HSK 2]
    \definition{adj.}{diferente; distinto; não semelhante;}
  \end{Phonetics}
\end{Entry}

\begin{Entry}{不同寻常}{4,6,6,11}{⼀、⼝、⼨、⼱}
  \begin{Phonetics}{不同寻常}{bu4tong2-xun2chang2}[][HSK 7-9]
    \definition{adj.}{extraordinário; incomum}
  \end{Phonetics}
\end{Entry}

\begin{Entry}{不在乎}{4,6,5}{⼀、⼟、⼃}
  \begin{Phonetics}{不在乎}{bu2 zai4 hu1}[][HSK 4]
    \definition{v.}{não se importar; não dar a mínima; não dar atenção}
  \end{Phonetics}
\end{Entry}

\begin{Entry}{不好意思}{4,6,13,9}{⼀、⼥、⼼、⼼}
  \begin{Phonetics}{不好意思}{bu4 hao3 yi4 si5}[][HSK 2]
    \definition{adj.}{envergonhado; desconfortável; constrangido; sem jeito}
    \definition{interj.}{com licença; peço desculpas; desculpe-me}
    \definition{v.}{achar constrangedor (fazer algo) | pedir desculpas (por incomodar alguém) | sentir-se envergonhado | achar algo embaraçoso}
  \end{Phonetics}
\end{Entry}

\begin{Entry}{不如}{4,6}{⼀、⼥}
  \begin{Phonetics}{不如}{bu4ru2}[][HSK 2]
    \definition{conj.}{em vez de; melhor do que; seria melhor; preferiria; seria melhor; usado no início da segunda parte da frase, indica uma escolha feita após comparação (geralmente em correspondência com o termo 与其 no texto anterior)}
    \definition{v.}{ser inferior a; não ser igual a; não ser tão bom quanto;  não poder fazer melhor que}
  \seealsoref{与其}{yu3qi2}
  \end{Phonetics}
\end{Entry}

\begin{Entry}{不如说}{4,6,9}{⼀、⼥、⾔}
  \begin{Phonetics}{不如说}{bu4ru2 shuo1}[][HSK 7-9]
    \definition{expr.}{é melhor dizer\dots;  usada para indicar que uma afirmação é mais apropriada, precisa ou preferível a outra; é frequentemente usada em situações de comparação e escolha para enfatizar uma afirmação ou ação superior}
  \end{Phonetics}
\end{Entry}

\begin{Entry}{不安}{4,6}{⼀、⼧}
  \begin{Phonetics}{不安}{bu4'an1}[][HSK 3]
    \definition{adj.}{(humor) inquieto; (ambiente, etc.)  instável; intranquilo; perturbado; sem paz | desculpe; frases de cortesia, expressões de desculpas, equivalentes a 不好意思}
  \seealsoref{不好意思}{bu4 hao3 yi4 si5}
  \end{Phonetics}
\end{Entry}

\begin{Entry}{不成}{4,6}{⼀、⼽}
  \begin{Phonetics}{不成}{bu4 cheng2}[][HSK 6]
    \definition{adj.}{não é bom; não funciona; impraticável}
    \definition{part.}{usada no final de uma frase para expressar especulação ou tom contraintuitivo, geralmente precedido por palavras como 嘛 ou 莫非}
    \definition{v.}{não ser permitido; não ser permissível; ser impossível}
  \seealsoref{莫非}{mo4fei1}
  \seealsoref{难道}{nan2dao4}
  \end{Phonetics}
\end{Entry}

\begin{Entry}{不成话}{4,6,8}{⼀、⼽、⾔}
  \begin{Phonetics}{不成话}{bu4cheng2hua4}
    \definition{expr.}{irracional | chocante; ultrajante; inapropriado}
  \seealsoref{不是话}{bu2shi4hua4}
  \seealsoref{不像话}{bu2xiang4hua4}
  \end{Phonetics}
\end{Entry}

\begin{Entry}{不约而同}{4,6,6,6}{⼀、⽷、⽽、⼝}
  \begin{Phonetics}{不约而同}{bu4yue1'er2tong2}[][HSK 7-9]
    \definition{expr.}{por coincidência; involuntariamente o mesmo; sem acordo prévio, mas de acordo; acontecer de coincidir; opiniões ou ações unânimes sem consulta prévia | fazer ou pensar a mesma coisa sem consulta prévia; acontecer de coincidir}
  \end{Phonetics}
\end{Entry}

\begin{Entry}{不至于}{4,6,3}{⼀、⾄、⼆}
  \begin{Phonetics}{不至于}{bu2 zhi4 yu2}[][HSK 6]
    \definition{adv.}{não pode ir tão longe a ponto de; não tanto\dots. a ponto de\dots; não a ponto de}
  \end{Phonetics}
\end{Entry}

\begin{Entry}{不行}{4,6}{⼀、⾏}
  \begin{Phonetics}{不行}{bu4 xing2}[][HSK 2]
    \definition{adj.}{não funciona; não é bom; falta de capacidade e habilidade; nível baixo}
    \definition{adv.}{profundamente; terrivelmente; extremamente; expressa um grau muito profundo; incrível (usado após o caractere 得 como complemento)}
    \definition{v.}{não servir; não ser permitido; estar fora de questão | estar à beira da morte}
  \seealsoref{得}{de5}
  \end{Phonetics}
\end{Entry}

\begin{Entry}{不许}{4,6}{⼀、⾔}
  \begin{Phonetics}{不许}{bu4 xu3}[][HSK 5]
    \definition{v.}{não permitir; ser proibido; proibir firmemente | não pode (usado em perguntas retóricas)}
  \end{Phonetics}
\end{Entry}

\begin{Entry}{不论}{4,6}{⼀、⾔}
  \begin{Phonetics}{不论}{bu2 lun4}[][HSK 3]
    \definition{conj.}{não importa (o que, quem, como, etc.); se \dots ou \dots; significa que as condições ou situações são diferentes, mas os resultados permanecem os mesmos; geralmente é seguido por palavras paralelas ou pronomes interrogativos; geralmente é seguido por advérbios como 都 e 总}
    \definition{v.}{não discutir nem argumentar; não discutir; não debater}
  \seealsoref{都}{dou1}
  \seealsoref{总}{zong3}
  \end{Phonetics}
\end{Entry}

\begin{Entry}{不论……也……}{4,6,3}{⼀、⾔、⼄}
  \begin{Phonetics}{不论……也……}{bu2lun4 ye3}
    \definition{conj.}{não apenas\dots, (o que, quem, como, etc.), \dots}
  \end{Phonetics}
\end{Entry}

\begin{Entry}{不论……都……}{4,6,10}{⼀、⾔、⾢}
  \begin{Phonetics}{不论……都……}{bu2lun4 dou1}
    \definition{conj.}{não apenas\dots, (o que, quem, como, etc.), \dots}
  \end{Phonetics}
\end{Entry}

\begin{Entry}{不过}{4,6}{⼀、⾡}
  \begin{Phonetics}{不过}{bu2guo4}[][HSK 2]
    \definition{adv.}{apenas; meramente; nada mais do que; indica que não excede um determinado limite, equivalente a 仅 ou 只 | como intensificador após certos adjetivos}
    \definition{conj.}{mas; no entanto; apenas; usado no início da segunda parte da frase, indica o contrário do sentido anterior e modifica ou complementa o significado anterior}
  \end{Phonetics}
\end{Entry}

\begin{Entry}{不但}{4,7}{⼀、⼈}
  \begin{Phonetics}{不但}{bu2 dan4}[][HSK 2]
    \definition{conj.}{não só\dots mas também; usado na primeira parte de uma frase composta que expressa progressão, a segunda parte geralmente contém conjunções como 而且,  并且 ou advérbios como 也, 还 que correspondem à primeira parte}
  \seealsoref{并且}{bing4qie3}
  \seealsoref{而且}{er2 qie3}
  \seealsoref{还}{hai2}
  \seealsoref{也}{ye3}
  \end{Phonetics}
\end{Entry}

\begin{Entry}{不但……而且……}{4,7,6,5}{⼀、⼈、⽽、⼀}
  \begin{Phonetics}{不但……而且……}{bu2 dan4 er2qie3}[][HSK 2]
    \definition{conj.}{não só\dots mas também\dots}
  \end{Phonetics}
\end{Entry}

\begin{Entry}{不免}{4,7}{⼀、⼉}
  \begin{Phonetics}{不免}{bu4mian3}[][HSK 5]
    \definition{adv.}{inevitavelmente; inexoravelmente}
  \end{Phonetics}
\end{Entry}

\begin{Entry}{不利}{4,7}{⼀、⼑}
  \begin{Phonetics}{不利}{bu2 li4}[][HSK 5]
    \definition{adj.}{desfavorável; desvantajoso; nocivo; prejudicial | malsucedido}
  \end{Phonetics}
\end{Entry}

\begin{Entry}{不利于}{4,7,3}{⼀、⼑、⼆}
  \begin{Phonetics}{不利于}{bu2li4 yu2}[][HSK 7-9]
    \definition{v.}{ser prejudicial a}
  \end{Phonetics}
\end{Entry}

\begin{Entry}{不妨}{4,7}{⼀、⼥}
  \begin{Phonetics}{不妨}{bu4fang2}[][HSK 7-9]
    \definition{adv.}{pode muito bem; não há mal nenhum em; significa que você pode fazer isso, não há problema}
  \end{Phonetics}
\end{Entry}

\begin{Entry}{不时}{4,7}{⼀、⽇}
  \begin{Phonetics}{不时}{bu4shi2}[][HSK 5]
    \definition{adv.}{frequentemente; de tempos em tempos | a qualquer momento}
  \end{Phonetics}
\end{Entry}

\begin{Entry}{不良}{4,7}{⼀、⾉}
  \begin{Phonetics}{不良}{bu4 liang2}[][HSK 5]
    \definition{adj.}{ruim; prejudicial; nocivo; insalubre}
  \end{Phonetics}
\end{Entry}

\begin{Entry}{不足}{4,7}{⼀、⾜}
  \begin{Phonetics}{不足}{bu4zu2}[][HSK 5]
    \definition{adj.}{não o bastante; inadequado; insuficiente}
    \definition{s.}{deficiência; inadequação; desvantagens, não é bom o suficiente}
    \definition{v.}{não exceder um determinado número | não valer a pena; ser inferior; não merecer | não pode; não deveria}
  \end{Phonetics}
\end{Entry}

\begin{Entry}{不到}{4,8}{⼀、⼑}
  \begin{Phonetics}{不到}{bu2dao4}
    \definition{adj.}{insuficiente}
    \definition{adv.}{menos que}
    \definition{v.}{não chegar}
  \end{Phonetics}
\end{Entry}

\begin{Entry}{不定}{4,8}{⼀、⼧}
  \begin{Phonetics}{不定}{bu2ding4}[][HSK 7-9]
    \definition{adj.}{indeterminado; indica pensamentos ou ações instáveis; incerteza; às vezes de uma maneira, às vezes de outra}
    \definition{adv.}{não é certo; não necessariamente; não sei, não tenho certeza do que vai acontecer}
  \end{Phonetics}
\end{Entry}

\begin{Entry}{不宜}{4,8}{⼀、⼧}
  \begin{Phonetics}{不宜}{bu4yi2}[][HSK 7-9]
    \definition{adv.}{não adequado; desaconselhável; inapropriado}
  \end{Phonetics}
\end{Entry}

\begin{Entry}{不幸}{4,8}{⼀、⼲}
  \begin{Phonetics}{不幸}{bu2 xing4}[][HSK 5]
    \definition{adj.}{triste; infeliz; lamentável; azarado | infeliz; indica o mais indesejável (que aconteceu)}
    \definition[些]{s.}{morte; desastre; infortúnio; adversidade; calamidade}
  \end{Phonetics}
\end{Entry}

\begin{Entry}{不易}{4,8}{⼀、⽇}
  \begin{Phonetics}{不易}{bu2 yi4}[][HSK 5]
    \definition{adj.}{não é fácil; difícil | imutável}
    \definition{v.}{não é fácil fazer algo | não pode ser alterado}
  \end{Phonetics}
\end{Entry}

\begin{Entry}{不服}{4,8}{⼀、⽉}
  \begin{Phonetics}{不服}{bu4fu2}[][HSK 7-9]
    \definition{v.}{recusar-se a obedecer (ou cumprir); recusar-se a aceitar como final; permanecer não convencido por; não ceder a | não estar acostumado a | recusar-se a obedecer; recalcitrar}
  \end{Phonetics}
\end{Entry}

\begin{Entry}{不服气}{4,8,4}{⼀、⽉、⽓}
  \begin{Phonetics}{不服气}{bu4 fu2qi4}[][HSK 7-9]
    \definition{v.}{recusar-se a ser convencido; não confiar}
  \end{Phonetics}
\end{Entry}

\begin{Entry}{不注意}{4,8,13}{⼀、⽔、⼼}
  \begin{Phonetics}{不注意}{bu2zhu4yi4}
    \definition{adj.}{impensado | distraído}
    \definition{s.}{descuido | distração}
  \end{Phonetics}
\end{Entry}

\begin{Entry}{不知}{4,8}{⼀、⽮}
  \begin{Phonetics}{不知}{bu4zhi1}[][HSK 7-9]
    \definition{v.}{não saber; ser ignorante de; não saber nada sobre; não ter ideia de; não estar ciente de; não ouvir falar de; não ter a mínima ideia}
  \end{Phonetics}
\end{Entry}

\begin{Entry}{不知不觉}{4,8,4,9}{⼀、⽮、⼀、⾒}
  \begin{Phonetics}{不知不觉}{bu4zhi1-bu4jue2}[][HSK 7-9]
    \definition{expr.}{imperceptivelmente; inconscientemente; involuntariamente; sem saber; sem querer}
  \end{Phonetics}
\end{Entry}

\begin{Entry}{不经意}{4,8,13}{⼀、⽷、⼼}
  \begin{Phonetics}{不经意}{bu4jing1yi4}[][HSK 7-9]
    \definition{v.}{não tomar cuidado; ser descuidado; ser desatento}
  \end{Phonetics}
\end{Entry}

\begin{Entry}{不肯}{4,8}{⼀、⾁}
  \begin{Phonetics}{不肯}{bu4 ken3}[][HSK 7-9]
    \definition{v.aux.}{não irá; não iria; usado como verbo auxiliar negativo para expressar recusa}
  \end{Phonetics}
\end{Entry}

\begin{Entry}{不便}{4,9}{⼀、⼈}
  \begin{Phonetics}{不便}{bu2 bian4}[][HSK 6]
    \definition{adj.}{inconveniente; inapropriado | não ter dinheiro em mãos; estar com pouco dinheiro}
    \definition{v.}{inadequado fazer algo; indica que fazer algo é inapropriado ou inconveniente}
  \end{Phonetics}
\end{Entry}

\begin{Entry}{不客气}{4,9,4}{⼀、⼧、⽓}
  \begin{Phonetics}{不客气}{bu2 ke4 qi5}[][HSK 1]
    \definition{adj.}{rude; indelicado; duro | franco; sincero; direto}
    \definition{expr.}{de modo algum; não mencione isso; de nada}
    \definition{v.}{dizer palavras ou fazer gestos indelicados}
  \end{Phonetics}
\end{Entry}

\begin{Entry}{不怎么}{4,9,3}{⼀、⼼、⼃}
  \begin{Phonetics}{不怎么}{bu4 zen3 me5}[][HSK 6]
    \definition{adv.}{não muito; não particularmente; não exatamente}
  \end{Phonetics}
\end{Entry}

\begin{Entry}{不怎么样}{4,9,3,10}{⼀、⼼、⼃、⽊}
  \begin{Phonetics}{不怎么样}{bu4 zen3 me5 yang4}[][HSK 6]
    \definition{adj.}{não muito bom; não particularmente bom | muito indiferente; mais ou menos}
  \end{Phonetics}
\end{Entry}

\begin{Entry}{不是……而是}{4,9,6,9}{⼀、⽇、⽽、⽇}
  \begin{Phonetics}{不是……而是}{bu4shi4 er2 shi4}
    \definition{conj.}{não somente\dots mas também\dots, expressam um relacionamento mais profundo e avançado em significado, mas as orações antes e depois são consistentes em expressar significados negativos e afirmativos, entretanto, a primeira metade da frase expressa negação, e a segunda metade expressa afirmação, e o significado das orações anteriores e seguintes não pode ser de um nível mais alto}
  \end{Phonetics}
\end{Entry}

\begin{Entry}{不是话}{4,9,8}{⼀、⽇、⾔}
  \begin{Phonetics}{不是话}{bu2shi4hua4}
    \definition{expr.}{inacreditável; além das palavras; (palavras) não fazem sentido}
  \seealsoref{不像话}{bu2xiang4hua4}
  \seealsoref{不成话}{bu4cheng2hua4}
  \end{Phonetics}
\end{Entry}

\begin{Entry}{不相上下}{4,9,3,3}{⼀、⽬、⼀、⼀}
  \begin{Phonetics}{不相上下}{bu4xiang1-shang4xia4}[][HSK 7-9]
    \definition{expr.}{igualmente semelhante; quase igual; quase no mesmo nível | ser aproximadamente o mesmo; quase igual; ser igual a\dots; estar no mesmo nível; estar um pouco no mesmo nível de\dots; mais ou menos de força igual; aumentar o tamanho para; há pouca (não há muito) para escolher entre os dois; sem muita diferença}
  \end{Phonetics}
\end{Entry}

\begin{Entry}{不耐烦}{4,9,10}{⼀、⽽、⽕}
  \begin{Phonetics}{不耐烦}{bu2nai4fan2}[][HSK 5]
    \definition{adj.}{impaciente; significa não ser capaz de suportar coisas tediosas ou que causam distração}
  \end{Phonetics}
\end{Entry}

\begin{Entry}{不要}{4,9}{⼀、⾑}
  \begin{Phonetics}{不要}{bu2 yao4}[][HSK 2]
    \definition{adv.}{nada de (pedir a alguém para não fazer) | não; expressa proibição e dissuasão}
  \end{Phonetics}
\end{Entry}

\begin{Entry}{不要紧}{4,9,10}{⼀、⾑、⽷}
  \begin{Phonetics}{不要紧}{bu2yao4jin3}[][HSK 4]
    \definition{adj.}{não é sério; não importa; não importa; não é um problema; nenhum obstáculo; nenhum problema; parece estar tudo bem; à primeira vista, não parece haver nenhum obstáculo}
  \end{Phonetics}
\end{Entry}

\begin{Entry}{不适}{4,9}{⼀、⾡}
  \begin{Phonetics}{不适}{bu2shi4}[][HSK 7-9]
    \definition{adj.}{Literário: desconfortável; indisposto; mal-estar; sentir-se mal | Literário: não adequado; inadequado}
  \end{Phonetics}
\end{Entry}

\begin{Entry}{不值}{4,10}{⼀、⼈}
  \begin{Phonetics}{不值}{bu4 zhi2}[][HSK 6]
    \definition{v.}{não valer a pena}
  \end{Phonetics}
\end{Entry}

\begin{Entry}{不准}{4,10}{⼀、⼎}
  \begin{Phonetics}{不准}{bu4 zhun3}[][HSK 7-9]
    \definition{v.}{proibir; impedir; vedar; não permitir}
  \end{Phonetics}
\end{Entry}

\begin{Entry}{不容}{4,10}{⼀、⼧}
  \begin{Phonetics}{不容}{bu4rong2}[][HSK 7-9]
    \definition{s.}{pontos de acupuntura}
    \definition{v.}{não tolerar; não permitir; não poder deixar de}
  \end{Phonetics}
\end{Entry}

\begin{Entry}{不屑}{4,10}{⼀、⼫}
  \begin{Phonetics}{不屑}{bu2xie4}[][HSK 7-9]
    \definition{adj.}{desdenhoso; zombador}
    \definition{v.}{pensar que algo não vale a pena fazer; sentir que está abaixo da dignidade de alguém fazer algo; ignorar}
  \end{Phonetics}
\end{Entry}

\begin{Entry}{不料}{4,10}{⼀、⽃}
  \begin{Phonetics}{不料}{bu2liao4}[][HSK 6]
    \definition{conj.}{inesperadamente; para surpresa de alguém}
  \end{Phonetics}
\end{Entry}

\begin{Entry}{不耻下问}{4,10,3,6}{⼀、⽿、⼀、⾨}
  \begin{Phonetics}{不耻下问}{bu4chi3-xia4wen4}[][HSK 7-9]
    \definition{expr.}{não ter vergonha de perguntar e aprender com os subordinados; Os Analectos de Confúcio: Gongye Chang: "Seja rápido para aprender e não tenha vergonha de fazer perguntas" significa não ter vergonha de pedir conselhos a pessoas de status inferior ou menos informadas do que você}
  \end{Phonetics}
\end{Entry}

\begin{Entry}{不能不}{4,10,4}{⼀、⾁、⼀}
  \begin{Phonetics}{不能不}{bu4 neng2 bu4}[][HSK 5]
    \definition{adv.}{tem que; não pode, mas; necessariamente; definitivamente}
  \end{Phonetics}
\end{Entry}

\begin{Entry}{不起眼}{4,10,11}{⼀、⾛、⽬}
  \begin{Phonetics}{不起眼}{bu4qi3yan3}[][HSK 7-9]
    \definition{adj.}{imperceptível; discreto; não chamativo; despercebido; não apreciado}
  \end{Phonetics}
\end{Entry}

\begin{Entry}{不通}{4,10}{⼀、⾡}
  \begin{Phonetics}{不通}{bu4 tong1}[][HSK 6]
    \definition{adj.}{sem sentido; ilógico; agramatical | usado para se referir a coisas abstratas}
    \definition{v.}{obstruir; bloquear; estar obstruído; estar bloqueado; ser intransitável | não saber; não entender; não poder aceitar}
  \end{Phonetics}
\end{Entry}

\begin{Entry}{不难}{4,10}{⼀、⾫}
  \begin{Phonetics}{不难}{bu4 nan2}[][HSK 7-9]
    \definition{v.}{não ser difícil}[这台电脑修理起来不难。===Este computador não é difícil de consertar.]
  \end{Phonetics}
\end{Entry}

\begin{Entry}{不顾}{4,10}{⼀、⾴}
  \begin{Phonetics}{不顾}{bu2gu4}[][HSK 5]
    \definition{v.}{não considerar; desconsiderar | desconsiderar; não levar em consideração; ignorar; não se preocupar com}
  \end{Phonetics}
\end{Entry}

\begin{Entry}{不假思索}{4,11,9,10}{⼀、⼈、⼼、⽷}
  \begin{Phonetics}{不假思索}{bu4jia3-si1suo3}[][HSK 7-9]
    \definition{expr.}{(agir, responder, etc.) sem pensar; sem hesitação; prontamente; de ​​improviso; reagir instantaneamente}
  \end{Phonetics}
\end{Entry}

\begin{Entry}{不停}{4,11}{⼀、⼈}
  \begin{Phonetics}{不停}{bu4 ting2}[][HSK 5]
    \definition{adv.}{sem parar; sem interrupção; continuamente}
  \end{Phonetics}
\end{Entry}

\begin{Entry}{不够}{4,11}{⼀、⼣}
  \begin{Phonetics}{不够}{bu2 gou4}[][HSK 2]
    \definition{adv.}{insuficiente; indica que não atingiu o nível esperado}
    \definition{v.}{não ser suficiente; indica que é inferior ao exigido em quantidade ou grau}
  \end{Phonetics}
\end{Entry}

\begin{Entry}{不得了}{4,11,2}{⼀、⼻、⼅}
  \begin{Phonetics}{不得了}{bu4de2liao3}[][HSK 5]
    \definition{adj.}{terrível; horrível; extremamente sério; indica uma situação grave}
    \definition{adv.}{muito; extremamente; excessivamente; indica um grau profundo}
  \end{Phonetics}
\end{Entry}

\begin{Entry}{不得已}{4,11,3}{⼀、⼻、⼰}
  \begin{Phonetics}{不得已}{bu4de2yi3}[][HSK 7-9]
    \definition{adj.}{agir contra a própria vontade; não ter alternativa senão; não há alternativa; tem que ser assim}
  \end{Phonetics}
\end{Entry}

\begin{Entry}{不得不}{4,11,4}{⼀、⼻、⼀}
  \begin{Phonetics}{不得不}{bu4de2bu4}[][HSK 3]
    \definition{adv.}{ter que; não ter outra escolha a não ser; como obrigação ou necessidade}
  \end{Phonetics}
\end{Entry}

\begin{Entry}{不得而知}{4,11,6,8}{⼀、⼻、⽽、⽮}
  \begin{Phonetics}{不得而知}{bu4de2'er2zhi1}[][HSK 7-9]
    \definition{expr.}{desconhecido; incapaz de descobrir}
  \end{Phonetics}
\end{Entry}

\begin{Entry}{不惜}{4,11}{⼀、⼼}
  \begin{Phonetics}{不惜}{bu4xi1}[][HSK 7-9]
    \definition{v.}{não hesitar (em fazer algo)}
  \end{Phonetics}
\end{Entry}

\begin{Entry}{不敢当}{4,11,6}{⼀、⽁、⼹}
  \begin{Phonetics}{不敢当}{bu4gan3dang1}[][HSK 5]
    \definition{expr.}{Eu realmente não mereço isso.; Eu não sou digno de tais elogios.; Não estou à altura da honra.; Você me lisonjeia.; palavra de humildade, para mostrar que você não pode pagar (hospitalidade, elogios, etc.)}
  \end{Phonetics}
\end{Entry}

\begin{Entry}{不断}{4,11}{⼀、⽄}
  \begin{Phonetics}{不断}{bu2duan4}[][HSK 3]
    \definition{adv.}{incessantemente; ininterruptamente; continuamente; constantemente}
    \definition{v.}{continuar; enfatiza a continuação da ação}
  \end{Phonetics}
\end{Entry}

\begin{Entry}{不理}{4,11}{⼀、⽟}
  \begin{Phonetics}{不理}{bu4 li3}[][HSK 7-9]
    \definition{v.}{ignorar; desconsiderar; recusar-se a reconhecer; não prestar atenção a; não tomar conhecimento de; não responder | não fazer; não lidar com}
  \end{Phonetics}
\end{Entry}

\begin{Entry}{不堪}{4,12}{⼀、⼟}
  \begin{Phonetics}{不堪}{bu4kan1}[][HSK 7-9]
    \definition{adj.}{extremamente indesejável; usado após adjetivos negativos | (após palavras com conotação negativa) insuportável;muito ruim}
    \definition{v.}{não poder suportar; não poder ficar de pé; ser incapaz de suportar | (geralmente referindo-se a algo indesejável) não poder; ser incapaz de}
  \end{Phonetics}
\end{Entry}

\begin{Entry}{不景气}{4,12,4}{⼀、⽇、⽓}
  \begin{Phonetics}{不景气}{bu4jing3qi4}[][HSK 7-9]
    \definition{adj.}{frouxo; lento; em estado de depressão | Economia: em depressão; em recessão; em crise; estagnada | estado depressivo; não próspero}
  \end{Phonetics}
\end{Entry}

\begin{Entry}{不曾}{4,12}{⼀、⽈}
  \begin{Phonetics}{不曾}{bu4 ceng2}[][HSK 5]
    \definition{adv.}{nunca (ter feito algo); indica que não aconteceu (negação de 曾经)}
  \seealsoref{曾经}{ceng2jing1}
  \end{Phonetics}
\end{Entry}

\begin{Entry}{不然}{4,12}{⼀、⽕}
  \begin{Phonetics}{不然}{bu4ran2}[][HSK 4]
    \definition{adj.}{não é assim; não é o caso}
    \definition{conj.}{se não; caso contrário; indica outra consequência ou circunstância que teria ocorrido se não fosse}
  \end{Phonetics}
\end{Entry}

\begin{Entry}{不像话}{4,13,8}{⼀、⼈、⾔}
  \begin{Phonetics}{不像话}{bu2xiang4hua4}[][HSK 7-9]
    \definition{expr.}{absurdo; sem sentido; irracional; uma determinada prática ou afirmação não está de acordo com o senso comum ou a razão e parece irracional | chocante; ultrajante; uma ação, palavra ou situação tão extrema que não pode ser aceita ou tolerada}
  \seealsoref{不是话}{bu2shi4hua4}
  \seealsoref{不成话}{bu4cheng2hua4}
  \end{Phonetics}
\end{Entry}

\begin{Entry}{不慎}{4,13}{⼀、⼼}
  \begin{Phonetics}{不慎}{bu2shen4}[][HSK 7-9]
    \definition{adj.}{descuidado; desavisado}
  \end{Phonetics}
\end{Entry}

\begin{Entry}{不满}{4,13}{⼀、⽔}
  \begin{Phonetics}{不满}{bu4 man3}[][HSK 2]
    \definition{adj.}{ressentido; insatisfeito; descontente}
    \definition{v.}{estar descontente com; insatisfação ou descontentamento com alguém ou alguma coisa |ser menor que; quantidade ou tempo insuficientes ou inadequados}
  \end{Phonetics}
\end{Entry}

\begin{Entry}{不禁}{4,13}{⼀、⽰}
  \begin{Phonetics}{不禁}{bu4jin1}[][HSK 6]
    \definition{adv.}{não pode evitar (fazer algo); não pode se abster de; incapaz de conter (produzir certas emoções, realizar certas ações)}
  \end{Phonetics}
\end{Entry}

\begin{Entry}{不解}{4,13}{⼀、⾓}
  \begin{Phonetics}{不解}{bu4jie3}[][HSK 7-9]
    \definition{adj.}{inexplicável; inseparável}
    \definition{v.}{não entender; deixar de compreender}
  \end{Phonetics}
\end{Entry}

\begin{Entry}{不辞而别}{4,13,6,7}{⼀、⾟、⽽、⼑}
  \begin{Phonetics}{不辞而别}{bu4ci2'er2bie2}[][HSK 7-9]
    \definition{expr.}{ir embora sem se despedir; sair sem se despedir}
    \definition{v.}{ir embora sem dizer adeus}
  \end{Phonetics}
\end{Entry}

\begin{Entry}{不错}{4,13}{⼀、⾦}
  \begin{Phonetics}{不错}{bu2 cuo4}[][HSK 2]
    \definition{adj.}{certo; correto | nada mal; muito bom}
  \end{Phonetics}
\end{Entry}

\begin{Entry}{不算}{4,14}{⼀、⽵}
  \begin{Phonetics}{不算}{bu2 suan4}[][HSK 7-9]
    \definition{adv.}{não realmente; não em vão; não é grande coisa}
  \end{Phonetics}
\end{Entry}

\begin{Entry}{不管}{4,14}{⼀、⽵}
  \begin{Phonetics}{不管}{bu4guan3}[][HSK 4]
    \definition{conj.}{não importa (o que, como, etc.); independentemente de; indica que, embora as condições ou circunstâncias tenham mudado, o resultado permanece o mesmo; 不管 deve ser seguido por algo incerto}
  \seealsoref{不管……都……}{bu4guan3 dou1}
  \seealsoref{不管……也……}{bu4guan3 ye3}
  \end{Phonetics}
\end{Entry}

\begin{Entry}{不管……也……}{4,14,3}{⼀、⽵、⼄}
  \begin{Phonetics}{不管……也……}{bu4guan3 ye3}
    \definition{conj.}{não apenas\dots, (o que, quem, como, etc.), \dots}
  \end{Phonetics}
\end{Entry}

\begin{Entry}{不管……都……}{4,14,10}{⼀、⽵、⾢}
  \begin{Phonetics}{不管……都……}{bu4guan3 dou1}
    \definition{conj.}{não apenas\dots, (o que, quem, como, etc.), \dots}
  \end{Phonetics}
\end{Entry}

\begin{Entry}{不懈}{4,16}{⼀、⼼}
  \begin{Phonetics}{不懈}{bu2xie4}[][HSK 7-9]
    \definition{adj.}{incansável; incessante; infatigável}
  \end{Phonetics}
\end{Entry}

\begin{Entry}{不翼而飞}{4,17,6,3}{⼀、⽻、⽽、⾶}
  \begin{Phonetics}{不翼而飞}{bu2yi4'er2fei1}[][HSK 7-9]
    \definition{expr.}{(um objeto) desaparecer sem deixar vestígios; desaparecer no ar; “voar sem asas” -- desaparecer sem deixar rastros; ganhar asas; desaparecer de repente; desaparecer repentinamente | espalhar-se rapidamente como se tivesse asas; espalhar-se como fogo}
  \end{Phonetics}
\end{Entry}

\begin{Entry}{丑}{4}{⼀}
  \begin{Phonetics}{丑}{chou3}[][HSK 5]
    \definition*{s.}{Sobrenome Chou}
    \definition{adj.}{feio, sem atrativos; em oposição a 美 | vergonhoso; desavergonhado; escandaloso; censurável; questionável}
    \definition{s.}{palhaço na ópera de Pequim, etc. | o segundo dos Doze Ramos Terrestres}
  \seealsoref{美}{mei3}
  \end{Phonetics}
\end{Entry}

\begin{Entry}{丑陋}{4,8}{⼀、⾩}
  \begin{Phonetics}{丑陋}{chou3lou4}[][HSK 7-9]
    \definition{adj.}{feio; de aparência muito feia | desgraçado; inglório; refere-se a pensamentos e comportamentos desprezíveis}
  \end{Phonetics}
\end{Entry}

\begin{Entry}{丑闻}{4,9}{⼀、⾨}
  \begin{Phonetics}{丑闻}{chou3wen2}[][HSK 7-9]
    \definition[个,件,些]{s.}{escândalo; rumores ou notícias sobre escândalos}
  \end{Phonetics}
\end{Entry}

\begin{Entry}{丑恶}{4,10}{⼀、⼼}
  \begin{Phonetics}{丑恶}{chou3'e4}[][HSK 7-9]
    \definition{adj./s.}{feio; repulsivo; hediondo}
  \end{Phonetics}
\end{Entry}

\begin{Entry}{专}{4}{⼀}
  \begin{Phonetics}{专}{zhuan1}
    \definition{adj.}{específico para; dedicado a um uso específico; dedicado a; especial}
    \definition{adv.}{especialmente; especificamente}
    \definition{v.}{monopolizar}
  \end{Phonetics}
\end{Entry}

\begin{Entry}{专门}{4,3}{⼀、⾨}
  \begin{Phonetics}{专门}{zhuan1men2}[][HSK 3]
    \definition{adj.}{especializado; dedicar-se exclusivamente a uma determinada tarefa; expressa ênfase em fazer frequentemente um determinado tipo de coisa}
    \definition{adv.}{especialmente}
  \end{Phonetics}
\end{Entry}

\begin{Entry}{专心}{4,4}{⼀、⼼}
  \begin{Phonetics}{专心}{zhuan1xin1}[][HSK 4]
    \definition{adj.}{absorto; concentrado}
  \end{Phonetics}
\end{Entry}

\begin{Entry}{专业}{4,5}{⼀、⼀}
  \begin{Phonetics}{专业}{zhuan1ye4}[][HSK 3]
    \definition{adj.}{profissional; descreve uma pessoa que possui um alto nível ou conhecimento profundo em determinada área}
    \definition[个,门]{s.}{profissão; área específica; comércio especializado; departamentos operacionais da divisão de produção | especialidade; disciplina; matéria especializada; área de estudo específica; em um departamento de uma instituição de ensino superior ou em uma escola profissionalizante de nível médio}
  \end{Phonetics}
\end{Entry}

\begin{Entry}{专业人士}{4,5,2,3}{⼀、⼀、⼈、⼠}
  \begin{Phonetics}{专业人士}{zhuan1ye4ren2shi4}
    \definition{s.}{profissional}
  \end{Phonetics}
\end{Entry}

\begin{Entry}{专业人才}{4,5,2,3}{⼀、⼀、⼈、⼿}
  \begin{Phonetics}{专业人才}{zhuan1ye4ren2cai2}
    \definition{s.}{especialista (em uma área)}
  \end{Phonetics}
\end{Entry}

\begin{Entry}{专业化}{4,5,4}{⼀、⼀、⼔}
  \begin{Phonetics}{专业化}{zhuan1ye4hua4}
    \definition{s.}{especialização}
  \end{Phonetics}
\end{Entry}

\begin{Entry}{专业户}{4,5,4}{⼀、⼀、⼾}
  \begin{Phonetics}{专业户}{zhuan1ye4hu4}
    \definition{s.}{indústria caseira | empresa familiar produzindo um produto especial}
  \end{Phonetics}
\end{Entry}

\begin{Entry}{专业性}{4,5,8}{⼀、⼀、⼼}
  \begin{Phonetics}{专业性}{zhuan1ye4xing4}
    \definition{s.}{profissionalismo | expertise}
  \end{Phonetics}
\end{Entry}

\begin{Entry}{专业教育}{4,5,11,8}{⼀、⼀、⽁、⾁}
  \begin{Phonetics}{专业教育}{zhuan1ye4jiao4yu4}
    \definition{s.}{educação especializada | escola técnica}
  \end{Phonetics}
\end{Entry}

\begin{Entry}{专用}{4,5}{⼀、⽤}
  \begin{Phonetics}{专用}{zhuan1 yong4}[][HSK 6]
    \definition{adj.}{(reservado para) uso especial; para um propósito especial; dedicado a uma determinada necessidade ou a uma determinada pessoa}[他需要一个专用的工作空间。===Ele precisava de um espaço de trabalho dedicado.]
  \end{Phonetics}
\end{Entry}

\begin{Entry}{专利}{4,7}{⼀、⼑}
  \begin{Phonetics}{专利}{zhuan1li4}[][HSK 5]
    \definition[个,项,些]{s.}{patente; a garantia de que os criadores e inventores desfrutem exclusivamente dos benefícios decorrentes de suas criações e invenções durante um determinado período | direitos de patente; referência à patente}
  \end{Phonetics}
\end{Entry}

\begin{Entry}{专家}{4,10}{⼀、⼧}
  \begin{Phonetics}{专家}{zhuan1jia1}[][HSK 3]
    \definition[个,位]{s.}{perito; especialista; profissional; pessoa que se dedica ao estudo aprofundado de uma determinada disciplina; pessoa especializada em uma determinada técnica}
  \end{Phonetics}
\end{Entry}

\begin{Entry}{专辑}{4,13}{⼀、⾞}
  \begin{Phonetics}{专辑}{zhuan1 ji2}[][HSK 5]
    \definition[张]{s.}{álbum (música) | registro (música) | coleção especial de material impresso ou transmitido}
  \end{Phonetics}
\end{Entry}

\begin{Entry}{专题}{4,15}{⼀、⾴}
  \begin{Phonetics}{专题}{zhuan1ti2}[][HSK 3]
    \definition[个,些,种]{s.}{assunto especial; tópico especial; questões específicas}
  \end{Phonetics}
\end{Entry}

\begin{Entry}{中}{4}{⼁}
  \begin{Phonetics}{中}{zhong1}[][HSK 1]
    \definition*{s.}{China; referindo-se à China | Sobrenome Zhong}
    \definition{adj.}{\emph{O.K.}; tudo bem, ótimo, adequado}
    \definition{s.}{centro; meio; a parte que está à mesma distância de todos os lados, acima e abaixo ou nas duas extremidades | em; entre (usado para indicar coisas dentro de um determinado intervalo) | meio; centro; localizado entre os dois extremos | médio; intermediário; classificação entre os dois extremos | médio; a meio caminho entre dois extremos; imparcial | intermediário | enquanto; durante (use após um verbo para mostrar que a ação está em andamento)}
    \definition{v.}{ser adequado para; ser compatível com}
  \seealsoref{中国}{zhong1guo2}
  \end{Phonetics}
  \begin{Phonetics}{中}{zhong4}
    \definition{v.}{acertar | encaixar exatamente |ser atingido por | cair em | ser afetado por | sofrer | sustentar}
  \end{Phonetics}
\end{Entry}

\begin{Entry}{中土}{4,3}{⼁、⼟}
  \begin{Phonetics}{中土}{zhong1 tu3}
    \definition*{s.}{Datado: China (Terra Média) | Sino-turco}
  \end{Phonetics}
\end{Entry}

\begin{Entry}{中小学}{4,3,8}{⼁、⼩、⼦}
  \begin{Phonetics}{中小学}{zhong1 xiao3 xue2}[][HSK 2]
    \definition{s.}{escolas primárias e secundárias}
  \end{Phonetics}
\end{Entry}

\begin{Entry}{中介}{4,4}{⼁、⼈}
  \begin{Phonetics}{中介}{zhong1jie4}[][HSK 4]
    \definition[个]{s.}{agente; intermediário; o meio de conexão}
  \end{Phonetics}
\end{Entry}

\begin{Entry}{中午}{4,4}{⼁、⼗}
  \begin{Phonetics}{中午}{zhong1wu3}[][HSK 1]
    \definition[个]{s.}{meio-dia; refere-se ao período entre 12 e 13 horas}
  \end{Phonetics}
\end{Entry}

\begin{Entry}{中心}{4,4}{⼁、⼼}
  \begin{Phonetics}{中心}{zhong1xin1}[][HSK 2]
    \definition[个]{s.}{núcleo; coração; meio; centro; posições com distâncias iguais de todas as áreas circundantes | chave; coração; a parte principal de algo; a pessoa ou coisa que desempenha um papel importante | centro; uma cidade ou lugar que tem um impacto significativo ou desempenha um papel importante em uma determinada área | centro; uma instituição com equipamentos e tecnologia relativamente completos e avançados em uma determinada área}
  \end{Phonetics}
\end{Entry}

\begin{Entry}{中文}{4,4}{⼁、⽂}
  \begin{Phonetics}{中文}{zhong1wen2}[][HSK 1]
    \definition{s.}{a língua chinesa; chinês; a língua e a escrita da China; especificamente, o chinês e os caracteres chineses}
  \end{Phonetics}
\end{Entry}

\begin{Entry}{中东}{4,5}{⼁、⼀}
  \begin{Phonetics}{中东}{zhong1dong1}
    \definition*{s.}{Oriente Médio}
  \end{Phonetics}
\end{Entry}

\begin{Entry}{中外}{4,5}{⼁、⼣}
  \begin{Phonetics}{中外}{zhong1 wai4}[][HSK 6]
    \definition{s.}{China e países estrangeiros}[这里吸引了许多中外游客。===Este lugar atrai muitos turistas chineses e estrangeiros.]
  \end{Phonetics}
\end{Entry}

\begin{Entry}{中央}{4,5}{⼁、⼤}
  \begin{Phonetics}{中央}{zhong1yang1}[][HSK 5]
    \definition{s.}{centro; meio; localização central | autoridades centrais; refere-se especificamente ao órgão máximo de liderança de um país ou partido político}
  \end{Phonetics}
\end{Entry}

\begin{Entry}{中央情报局}{4,5,11,7,7}{⼁、⼤、⼼、⼿、⼫}
  \begin{Phonetics}{中央情报局}{zhong1yang1 qing2bao4ju2}
    \definition*{s.}{Agência Central de Inteligência dos EUA, CIA}
  \end{Phonetics}
\end{Entry}

\begin{Entry}{中华}{4,6}{⼁、⼗}
  \begin{Phonetics}{中华}{zhong1 hua2}[][HSK 6]
    \definition{s.}{China; na antiguidade, a região do rio Amarelo era chamada de Zhonghua, sendo o local onde a etnia Han surgiu inicialmente. Posteriormente, passou a designar a China}
  \seealsoref{中国}{zhong1guo2}
  \end{Phonetics}
\end{Entry}

\begin{Entry}{中华民族}{4,6,5,11}{⼁、⼗、⽒、⽅}
  \begin{Phonetics}{中华民族}{zhong1 hua2 min2 zu2}[][HSK 3]
    \definition*{s.}{O Povo Chinês; o nome genérico para todas as etnias da China, incluindo 56 etnias, com uma longa história, um rico patrimônio cultural e uma gloriosa tradição revolucionária | A Nação Chinesa}
  \end{Phonetics}
\end{Entry}

\begin{Entry}{中州}{4,6}{⼁、⼮}
  \begin{Phonetics}{中州}{zhong1zhou1}
    \definition*{s.}{Região Central, ou seja, província de Henan, devido à sua localização central no país}
  \end{Phonetics}
\end{Entry}

\begin{Entry}{中年}{4,6}{⼁、⼲}
  \begin{Phonetics}{中年}{zhong1 nian2}[][HSK 2]
    \definition{s.}{meia-idade; na casa dos quarenta ou cinquenta anos}
  \end{Phonetics}
\end{Entry}

\begin{Entry}{中级}{4,6}{⼁、⽷}
  \begin{Phonetics}{中级}{zhong1 ji2}[][HSK 2]
    \definition{adj.}{nível médio; nível intermediário; entre avançado e iniciante}
  \end{Phonetics}
\end{Entry}

\begin{Entry}{中医}{4,7}{⼁、⼖}
  \begin{Phonetics}{中医}{zhong1 yi1}[][HSK 2]
    \definition[位,个,名,些,群]{s.}{ciência médica tradicional chinesa; medicina chinesa | médico de medicina tradicional chinesa; praticante de medicina chinesa}
  \end{Phonetics}
\end{Entry}

\begin{Entry}{中间}{4,7}{⼁、⾨}
  \begin{Phonetics}{中间}{zhong1jian1}[][HSK 1]
    \definition[本]{s.}{em meio a; entre; dentro de um determinado intervalo | meio; centro; posição entre os dois extremos de uma coisa ou entre duas coisas | posição intermediária; espaço entre duas extremidades; posição entre os dois extremos de uma coisa, dois momentos, duas coisas, etc.}
  \end{Phonetics}
\end{Entry}

\begin{Entry}{中国}{4,8}{⼁、⼞}
  \begin{Phonetics}{中国}{zhong1guo2}[][HSK 1]
    \definition*{s.}{China; os povos Huaxia e Han estabeleceram suas capitais principalmente ao sul e ao norte do rio Huang He, e por isso chamaram essa região de 中国, com o mesmo significado de 中土, 中原l, 中州 e 中华}
  \seealsoref{中华}{zhong1 hua2}
  \seealsoref{中土}{zhong1 tu3}
  \seealsoref{中原}{zhong1yuan2}
  \seealsoref{中州}{zhong1zhou1}
  \end{Phonetics}
\end{Entry}

\begin{Entry}{中国人}{4,8,2}{⼁、⼞、⼈}
  \begin{Phonetics}{中国人}{zhong1guo2ren2}
    \definition{s.}{chinês | pessoa ou povo da China}
  \end{Phonetics}
\end{Entry}

\begin{Entry}{中国城}{4,8,9}{⼁、⼞、⼟}
  \begin{Phonetics}{中国城}{zhong1guo2cheng2}
    \definition*[座]{s.}{Bairro Chinês, Chinatown}
  \seealsoref{唐人街}{tang2ren2 jie1}
  \end{Phonetics}
\end{Entry}

\begin{Entry}{中国科学院}{4,8,9,8,9}{⼁、⼞、⽲、⼦、⾩}
  \begin{Phonetics}{中国科学院}{zhong1guo2 ke1xue2yuan4}
    \definition*{s.}{Academia Chinesa de Ciências}
  \end{Phonetics}
\end{Entry}

\begin{Entry}{中国通}{4,8,10}{⼁、⼞、⾡}
  \begin{Phonetics}{中国通}{zhong1guo2tong1}
    \definition*{s.}{Sinólogo | Conhecedor da China, especialista em tudo sobre a China; refere-se a estrangeiros que estão familiarizados com a China}
  \end{Phonetics}
\end{Entry}

\begin{Entry}{中学}{4,8}{⼁、⼦}
  \begin{Phonetics}{中学}{zhong1 xue2}[][HSK 1]
    \definition[个]{s.}{escola ensino médio; escolas onde os jovens recebem educação secundária geralmente incluem o ensino fundamental II e o ensino médio | aprendizagem chinesa (um termo da dinastia Qing tardia para a aprendizagem tradicional chinesa); antigamente, referia-se à academia tradicional da China, como filosofia, linguística, medicina tradicional chinesa, etc.}
  \end{Phonetics}
\end{Entry}

\begin{Entry}{中学生}{4,8,5}{⼁、⼦、⽣}
  \begin{Phonetics}{中学生}{zhong1 xue2 sheng1}[][HSK 1]
    \definition{s.}{aluno, estudante do ensino médio; alunos matriculados no ensino médio. Inclui alunos do ensino fundamental II e do ensino médio}
  \end{Phonetics}
\end{Entry}

\begin{Entry}{中性}{4,8}{⼁、⼼}
  \begin{Phonetics}{中性}{zhong1 xing4}
    \definition{adj.}{neutro | (penteado, roupa, etc.) unissex}
  \end{Phonetics}
\end{Entry}

\begin{Entry}{中询}{4,8}{⼁、⾔}
  \begin{Phonetics}{中询}{zhong1 xun2}
    \definition{adv.}{segunda dezena do mês | meio do mês | em meados do mês}
  \end{Phonetics}
\end{Entry}

\begin{Entry}{中奖}{4,9}{⼁、⼤}
  \begin{Phonetics}{中奖}{zhong4 jiang3}[][HSK 4]
    \definition{v.}{ganhar um prêmio (em uma loteria, etc.)}
  \end{Phonetics}
\end{Entry}

\begin{Entry}{中毒}{4,9}{⼁、⽏}
  \begin{Phonetics}{中毒}{zhong4 du2}[][HSK 5]
    \definition{v.}{envenenar; intoxicar | ser envenenado; ideia de que o pensamento foi contaminado}
  \end{Phonetics}
\end{Entry}

\begin{Entry}{中秋节}{4,9,5}{⼁、⽲、⾋}
  \begin{Phonetics}{中秋节}{zhong1 qiu1 jie2}[][HSK 5]
    \definition*[个]{s.}{Festival do Meio-Outono | Festival do Bolo Lunar (15º dia do oitavo mês lunar)}
  \end{Phonetics}
\end{Entry}

\begin{Entry}{中药}{4,9}{⼁、⾋}
  \begin{Phonetics}{中药}{zhong1 yao4}[][HSK 5]
    \definition[些,种]{s.}{medicina herbal; medicina tradicional chinesa; fitoterapia; medicamentos utilizados na medicina tradicional chinesa; incluem medicamentos naturais e seus produtos processados (em contraste com 西药)}
  \seealsoref{西药}{xi1 yao4}
  \end{Phonetics}
\end{Entry}

\begin{Entry}{中原}{4,10}{⼁、⼚}
  \begin{Phonetics}{中原}{zhong1yuan2}
    \definition*{s.}{Planícies Centrais (compreendendo os trechos médio e inferior do rio Huang He) | Planície Central, regiões média e baixa do rio Amarelo, incluindo Henan, oeste de Shandong, sul de Shanxi e Hebei}
  \end{Phonetics}
\end{Entry}

\begin{Entry}{中部}{4,10}{⼁、⾢}
  \begin{Phonetics}{中部}{zhong1 bu4}[][HSK 3]
    \definition{s.}{parte do meio; região central; seção central; região ou parte intermediária | parte do meio; parte central; refere-se à parte intermediária de uma série de três partes, como romances, obras cinematográficas e televisivas}
  \end{Phonetics}
\end{Entry}

\begin{Entry}{中情局}{4,11,7}{⼁、⼼、⼫}
  \begin{Phonetics}{中情局}{zhong1qing2ju2}
    \definition*{s.}{Agência Central de Inteligência dos EUA, CIA, abreviação de 中央情报局}
  \seealsoref{中央情报局}{zhong1yang1 qing2bao4ju2}
  \end{Phonetics}
\end{Entry}

\begin{Entry}{中断}{4,11}{⼁、⽄}
  \begin{Phonetics}{中断}{zhong1duan4}[][HSK 5]
    \definition{v.}{suspender; romper; descontinuar; interromper; quebrar | dividir; quebrar; ser quebrado}
  \end{Phonetics}
\end{Entry}

\begin{Entry}{中期}{4,12}{⼁、⽉}
  \begin{Phonetics}{中期}{zhong1 qi1}[][HSK 6]
    \definition{s.}{período intermediário | médio prazo (plano, previsão etc.); meio-termo | meio (de um período de tempo)}
  \end{Phonetics}
\end{Entry}

\begin{Entry}{中等}{4,12}{⼁、⽵}
  \begin{Phonetics}{中等}{zhong1 deng3}[][HSK 6]
    \definition{adj.}{médio; a nota está entre superior e inferior ou entre avançado e elementar | de altura média; nem baixo nem alto}
  \end{Phonetics}
\end{Entry}

\begin{Entry}{中意}{4,13}{⼁、⼼}
  \begin{Phonetics}{中意}{zhong4yi4}
    \definition{s.}{ser do seu agrado | começar a gostar muito de algo ou de alguém}
  \end{Phonetics}
\end{Entry}

\begin{Entry}{中餐}{4,16}{⼁、⾷}
  \begin{Phonetics}{中餐}{zhong1 can1}[][HSK 2]
    \definition[份,顿,桌]{s.}{refeição chinesa; comida chinesa; comida de estilo chinês (diferente de 西餐) | almoço}
  \seealsoref{西餐}{xi1 can1}
  \end{Phonetics}
\end{Entry}

\begin{Entry}{丰}{4}{⼁}
  \begin{Phonetics}{丰}{feng1}
    \definition*{s.}{Sobrenome Feng}
    \definition[阵,丝]{adj.}{cheio; rico; abundante | ótimo | bonito; de boa aparência; cheio e redondo}
  \end{Phonetics}
\end{Entry}

\begin{Entry}{丰收}{4,6}{⼁、⽁}
  \begin{Phonetics}{丰收}{feng1shou1}[][HSK 5]
    \definition{v.}{ter uma boa colheita; obter uma colheita boa e abundante; obter bons resultados}
  \end{Phonetics}
\end{Entry}

\begin{Entry}{丰厚}{4,9}{⼁、⼚}
  \begin{Phonetics}{丰厚}{feng1hou4}[][HSK 7-9]
    \definition{adj.}{rico e generoso; (renda, presentes, etc.) grande em quantidade e alto em valor | grosso; abundante e espesso}
  \end{Phonetics}
\end{Entry}

\begin{Entry}{丰盛}{4,11}{⼁、⽫}
  \begin{Phonetics}{丰盛}{feng1sheng4}[][HSK 7-9]
    \definition{adj.}{rico; abundante; substancial; suntuoso; descreve comida que é abundante e boa}[他吃了一顿丰盛的早餐。===Ele tomou um café da manhã substancial.]
  \end{Phonetics}
\end{Entry}

\begin{Entry}{丰硕}{4,11}{⼁、⽯}
  \begin{Phonetics}{丰硕}{feng1shuo4}[][HSK 7-9]
    \definition{adj.}{rico; frutífero; abundante e substancial; usado principalmente para coisas abstratas}[她的研究成果非常丰硕。===Os resultados de sua pesquisa são muito frutíferos.]
  \end{Phonetics}
\end{Entry}

\begin{Entry}{丰富}{4,12}{⼁、⼧}
  \begin{Phonetics}{丰富}{feng1fu4}[][HSK 3]
    \definition{adj.}{rico; abundante; pleno; (riqueza material, conhecimento, experiência, etc.) variedade ou quantidade}
    \definition{v.}{enriquecer}
  \end{Phonetics}
\end{Entry}

\begin{Entry}{丰富多彩}{4,12,6,11}{⼁、⼧、⼣、⼺}
  \begin{Phonetics}{丰富多彩}{feng1fu4-duo1cai3}[][HSK 7-9]
    \definition{expr.}{rica e colorida; variada e colorida; abundante e diversa; descreve uma grande variedade de tipos e cores}
  \end{Phonetics}
\end{Entry}

\begin{Entry}{丰满}{4,13}{⼁、⽔}
  \begin{Phonetics}{丰满}{feng1man3}[][HSK 7-9]
    \definition{adj.}{(corpo ou partes do corpo) roliço; adulto; cheio e redondo; bem desenvolvido; roliço e bem proporcionado | abundante; adequado | completo; cheio}
  \end{Phonetics}
\end{Entry}

\begin{Entry}{为}{4}{⼂}
  \begin{Phonetics}{为}{wei2}[][HSK 3]
    \definition*{s.}{Sobrenome Wei}
    \definition{part.}{frequentemente usado com 何 em uma pergunta retórica}
    \definition{prep.}{por; usado em frases passivas para introduzir o agente da ação, equivalente a 被 (frequentemente usado com 所)}
    \definition{suf.}{é anexado a alguns adjetivos ou advérbios monossilábicos para formar advérbios dissilábicos que expressam grau ou amplitude, geralmente modificando adjetivos ou verbos dissilábicos}
    \definition{v.}{fazer; agir | tornar-se; transformar-se em | ser; significar | servir como; agir como; desempenhar o papel de | fazer; trabalhar; indica certas ações e comportamentos, incluindo os significados de governança, engajamento, cenário e pesquisa}
  \seealsoref{被}{bei4}
  \seealsoref{何}{he2}
  \seealsoref{所}{suo3}
  \end{Phonetics}
  \begin{Phonetics}{为}{wei4}[][HSK 2,3]
    \definition*{s.}{Sobrenome Wei}
    \definition{part.}{com 何 em uma pergunta retórica para expressar dúvida}
    \definition{prep.}{por; usado em frases passivas para introduzir o agente da ação, equivalente a 被 (frequentemente usado com 所)}
    \definition{suf.}{é anexado a alguns adjetivos ou advérbios monossilábicos para formar advérbios dissilábicos que expressam grau ou amplitude, geralmente modificando adjetivos ou verbos dissilábicos}
    \definition{v.}{fazer; agir | tornar-se; transformar-se em | ser;  significar | servir como; agir como; desempenhar o papel de | fazer; trabalhar; indica certas ações e comportamentos, incluindo os significados de governança, engajamento, cenário e pesquisa}
  \seealsoref{被}{bei4}
  \seealsoref{何}{he2}
  \seealsoref{所}{suo3}
  \end{Phonetics}
\end{Entry}

\begin{Entry}{为了}{4,2}{⼂、⼅}
  \begin{Phonetics}{为了}{wei4le5}[][HSK 3]
    \definition{prep.}{para; por causa de; a fim de; o objetivo da introdução de ações comportamentais}
  \end{Phonetics}
\end{Entry}

\begin{Entry}{为什么}{4,4,3}{⼂、⼈、⼃}
  \begin{Phonetics}{为什么}{wei4shen2me5}[][HSK 2]
    \definition{adv.}{por que?; por que é que?; como é que?;  nota: 为什么不 geralmente tem o significado de conselho, o mesmo que 何不}
  \seealsoref{何不}{he2bu4}
  \end{Phonetics}
\end{Entry}

\begin{Entry}{为止}{4,4}{⼂、⽌}
  \begin{Phonetics}{为止}{wei2 zhi3}[][HSK 5]
    \definition{adv.}{até; até um determinado momento}
  \end{Phonetics}
\end{Entry}

\begin{Entry}{为主}{4,5}{⼂、⼂}
  \begin{Phonetics}{为主}{wei2 zhu3}[][HSK 5]
    \definition{v.}{dar prioridade a; dar preferência a; dar importância a}
  \end{Phonetics}
\end{Entry}

\begin{Entry}{为此}{4,6}{⼂、⽌}
  \begin{Phonetics}{为此}{wei4 ci3}[][HSK 6]
    \definition{conj.}{portanto; para este fim; por esta razão; para este propósito; nesta conexão; contexto de conexão, indicando que o comportamento descrito é devido aos motivos mencionados anteriormente}
  \end{Phonetics}
\end{Entry}

\begin{Entry}{为何}{4,7}{⼂、⼈}
  \begin{Phonetics}{为何}{wei4 he2}[][HSK 6]
    \definition{adv.}{por que?; por qual razão?}
  \seealsoref{为什么}{wei4shen2me5}
  \end{Phonetics}
\end{Entry}

\begin{Entry}{为难}{4,10}{⼂、⾫}
  \begin{Phonetics}{为难}{wei2nan2}[][HSK 5]
    \definition{adj.}{envergonhado; sentir-se constrangido; sentir-se sobrecarregado; sentir-se incapaz de lidar com algo}
    \definition{v.}{dificultar as coisas para; dificultar; contrariar}
  \end{Phonetics}
\end{Entry}

\begin{Entry}{为期}{4,12}{⼂、⽉}
  \begin{Phonetics}{为期}{wei2qi1}[][HSK 5]
    \definition{s.}{Literário: tempo restante}
    \definition{v.}{Literário: a ser concluído (até uma data definida, por um determinado período de tempo)}
  \end{Phonetics}
\end{Entry}

\begin{Entry}{乌}{4}{⼃}
  \begin{Phonetics}{乌}{wu1}
    \definition*{s.}{Sobrenome Wu}
    \definition{adj.}{preto; escuro}
    \definition{pron.}{como; o que é}
    \definition{s.}{corvo; gralha}
  \end{Phonetics}
\end{Entry}

\begin{Entry}{乌云}{4,4}{⼃、⼆}
  \begin{Phonetics}{乌云}{wu1 yun2}[][HSK 6]
    \definition[片]{s.}{nuvens negras; nuvens escuras | cabelo preto (de mulher); uma metáfora para o cabelo preto brilhante de uma mulher}
  \end{Phonetics}
\end{Entry}

\begin{Entry}{乌克兰}{4,7,5}{⼃、⼗、⼋}
  \begin{Phonetics}{乌克兰}{wu1ke4lan2}
    \definition*{s.}{Ucrânia}
  \end{Phonetics}
\end{Entry}

\begin{Entry}{乌龟}{4,7}{⼃、⿔}
  \begin{Phonetics}{乌龟}{wu1gui1}
    \definition[头,只]{s.}{tartaruga}
  \end{Phonetics}
\end{Entry}

\begin{Entry}{书}{4}{⼄}
  \begin{Phonetics}{书}{shu1}[][HSK 1]
    \definition*{s.}{Sobrenome Shu}
    \definition[本,册,部,套,卷]{s.}{livro; obras encadernadas | carta; carta especial | documento | estilo de caligrafia; escrita}
    \definition{v.}{escrever; registrar}
  \end{Phonetics}
\end{Entry}

\begin{Entry}{书包}{4,5}{⼄、⼓}
  \begin{Phonetics}{书包}{shu1 bao1}[][HSK 1]
    \definition[个,款]{s.}{mochila para guardar livros e materiais escolares}
  \end{Phonetics}
\end{Entry}

\begin{Entry}{书记}{4,5}{⼄、⾔}
  \begin{Phonetics}{书记}{shu1ji5}
    \definition{s.}{secretário (chefe de um ramo de um partido socialista ou comunista) | atendente | balconista | escriturário}
  \end{Phonetics}
\end{Entry}

\begin{Entry}{书店}{4,8}{⼄、⼴}
  \begin{Phonetics}{书店}{shu1 dian4}[][HSK 1]
    \definition[个,家]{s.}{livraria; lojas que vendem livros}
  \end{Phonetics}
\end{Entry}

\begin{Entry}{书房}{4,8}{⼄、⼾}
  \begin{Phonetics}{书房}{shu1 fang2}[][HSK 6]
    \definition[间]{s.}{uma biblioteca (em uma residência privada); espaço para leitura e escrita}
  \end{Phonetics}
\end{Entry}

\begin{Entry}{书柜}{4,8}{⼄、⽊}
  \begin{Phonetics}{书柜}{shu1 gui4}[][HSK 5]
    \definition{s.}{estante; armário de livros}
  \end{Phonetics}
\end{Entry}

\begin{Entry}{书法}{4,8}{⼄、⽔}
  \begin{Phonetics}{书法}{shu1fa3}[][HSK 5]
    \definition[幅,卷,种,派]{s.}{caligrafia; arte de escrever caracteres, especialmente arte de escrever caracteres chineses com um pincel}
  \end{Phonetics}
\end{Entry}

\begin{Entry}{书架}{4,9}{⼄、⽊}
  \begin{Phonetics}{书架}{shu1jia4}[][HSK 3]
    \definition[个,种,套]{s.}{estante de livros}
  \end{Phonetics}
\end{Entry}

\begin{Entry}{书桌}{4,10}{⼄、⽊}
  \begin{Phonetics}{书桌}{shu1 zhuo1}[][HSK 5]
    \definition[个,张]{s.}{escrivaninha; mesa para ler e escrever}
  \end{Phonetics}
\end{Entry}

\begin{Entry}{云}{4}{⼆}
  \begin{Phonetics}{云}{yun2}[][HSK 2]
    \definition*{s.}{Província de Yunnan, abreviação de 云南 | Sobrenome Yun}
    \definition[片,朵]{s.}{nuvem}
    \definition{v.}{dizer}
  \seealsoref{云南}{yun2nan2}
  \end{Phonetics}
\end{Entry}

\begin{Entry}{云云}{4,4}{⼆、⼆}
  \begin{Phonetics}{云云}{yun2yun2}
    \definition{adv.}{e assim por diante | assim e assim}
  \end{Phonetics}
\end{Entry}

\begin{Entry}{云南}{4,9}{⼆、⼗}
  \begin{Phonetics}{云南}{yun2nan2}
    \definition*{s.}{Província de Yunnan}
  \end{Phonetics}
\end{Entry}

\begin{Entry}{云端}{4,14}{⼆、⽴}
  \begin{Phonetics}{云端}{yun2duan1}
    \definition{s.}{alto nas nuvens | (computação) a nuvem}
  \end{Phonetics}
\end{Entry}

\begin{Entry}{互}{4}{⼆}
  \begin{Phonetics}{互}{hu4}
    \definition{adv.}{mutuamente; um ao outro}
    \definition{pron.}{um ao outro; mútuo}
  \end{Phonetics}
\end{Entry}

\begin{Entry}{互动}{4,6}{⼆、⼒}
  \begin{Phonetics}{互动}{hu4 dong4}[][HSK 6]
    \definition{v.}{interagir; participar juntos; promover uns aos outros}
  \end{Phonetics}
\end{Entry}

\begin{Entry}{互访}{4,6}{⼆、⾔}
  \begin{Phonetics}{互访}{hu4fang3}[][HSK 7-9]
    \definition{s.}{visitas mútuas}
    \definition{v.}{trocar visitas}
  \end{Phonetics}
\end{Entry}

\begin{Entry}{互利}{4,7}{⼆、⼑}
  \begin{Phonetics}{互利}{hu4li4}
    \definition{s.}{benefício mútuo}
  \end{Phonetics}
\end{Entry}

\begin{Entry}{互助}{4,7}{⼆、⼒}
  \begin{Phonetics}{互助}{hu4zhu4}[][HSK 7-9]
    \definition{v.}{ajudar uns aos outros; ajudar-se mutualmente}[我们应该互助合作。===Devemos ajudar e cooperar uns com os outros.]
  \end{Phonetics}
\end{Entry}

\begin{Entry}{互补}{4,7}{⼆、⾐}
  \begin{Phonetics}{互补}{hu4bu3}[][HSK 7-9]
    \definition{v.}{complementar; complementar-se}
  \end{Phonetics}
\end{Entry}

\begin{Entry}{互信}{4,9}{⼆、⼈}
  \begin{Phonetics}{互信}{hu4xin4}[][HSK 7-9]
    \definition{s.}{confiança mútua}
    \definition{v.}{confiar um no outro}
  \end{Phonetics}
\end{Entry}

\begin{Entry}{互相}{4,9}{⼆、⽬}
  \begin{Phonetics}{互相}{hu4xiang1}[][HSK 3]
    \definition{adv.}{mutuamente; um ao outro; expressa uma relação de igualdade entre as partes}
  \end{Phonetics}
\end{Entry}

\begin{Entry}{互联网}{4,12,6}{⼆、⽿、⽹}
  \begin{Phonetics}{互联网}{hu4 lian2 wang3}[][HSK 3]
    \definition{s.}{\emph{Internet}; uma enorme rede conectando computadores e redes de computadores}
  \seealsoref{网际网路}{wang3 ji4 wang3 lu4}
  \seealsoref{网际网络}{wang3 ji4 wang3 luo4}
  \seealsoref{网路}{wang3 lu4}
  \end{Phonetics}
\end{Entry}

\begin{Entry}{五}{4}{⼆}
  \begin{Phonetics}{五}{wu3}[][HSK 1]
    \definition*{s.}{Sobrenome Wu}
    \definition{num.}{cinco; 5}
    \definition{s.}{uma nota da escala em Gongchepu (工尺谱), correspondente a 6 na notação musical numerada}
  \seealsoref{工尺谱}{gong1 che3 pu3}
  \end{Phonetics}
\end{Entry}

\begin{Entry}{五五}{4,4}{⼆、⼆}
  \begin{Phonetics}{五五}{wu3wu3}
    \definition{num.}{50-50}
    \definition{s.}{igual (partilha, parceria, etc.)}
  \end{Phonetics}
\end{Entry}

\begin{Entry}{五体投地}{4,7,7,6}{⼆、⼈、⼿、⼟}
  \begin{Phonetics}{五体投地}{wu3ti3tou2di4}
    \definition{expr.}{prostrar-se em admiração | adular alguém}
  \end{Phonetics}
\end{Entry}

\begin{Entry}{五颜六色}{4,15,4,6}{⼆、⾴、⼋、⾊}
  \begin{Phonetics}{五颜六色}{wu3 yan2 liu4 se4}[][HSK 4]
    \definition{adj.}{de várias (ou todas) cores; multicolorido; colorido}
  \end{Phonetics}
\end{Entry}

\begin{Entry}{井}{4}{⼆}
  \begin{Phonetics}{井}{jing3}[][HSK 6]
    \definition*{s.}{Jing, uma das mansões lunares | Sobrenome Jing}
    \definition{adj.}{limpo; organizado}
    \definition[口]{s.}{poço; um buraco profundo cavado no chão para tirar água | algo em forma de poço | vila natal ou cidade natal}
  \end{Phonetics}
\end{Entry}

\begin{Entry}{什}{4}{⼈}
  \begin{Phonetics}{什}{shen2}
    \definition{pron.}{o que; qualquer coisa}
  \seealsoref{什么}{shen2me5}
  \end{Phonetics}
  \begin{Phonetics}{什}{shi2}
    \definition*{s.}{Sobrenome Shi}
    \definition{adj.}{variado; sortido; diverso; vários; misturados}
    \definition{num.}{(em frações ou múltiplos) dez}
    \definition{s.}{várias coisas; artigos diversos}
  \end{Phonetics}
\end{Entry}

\begin{Entry}{什么}{4,3}{⼈、⼃}
  \begin{Phonetics}{什么}{shen2me5}[][HSK 1]
    \definition{pron.}{o que?; expressar dúvida, perguntar sobre o mundo, locais, pessoas ou coisas | usado para se referir a algo indefinido; expressar incerteza | qualquer; todos; refere-se a todas as pessoas ou coisas | dois 什么 são usados juntos, indicando que o primeiro determina o segundo | usado para expressar surpresa ou insatisfação | usado para expressar discordância com o que foi dito; expressar negação | usado antes de elementos paralelos para indicar que a lista é infinita}
  \end{Phonetics}
\end{Entry}

\begin{Entry}{什么时候}{4,3,7,10}{⼈、⼃、⽇、⼈}
  \begin{Phonetics}{什么时候}{shen2me5shi2hou5}
    \definition{adv.}{quando? | a que horas?}
  \end{Phonetics}
\end{Entry}

\begin{Entry}{什么样}{4,3,10}{⼈、⼃、⽊}
  \begin{Phonetics}{什么样}{shen2 me5 yang4}[][HSK 2]
    \definition{pron.}{que tipo?; usado para perguntar sobre a natureza, características ou aparência de algo |  o quê?; de que tipo?; usado para perguntar sobre a situação ou o estado de alguém ou algo}
  \end{Phonetics}
\end{Entry}

\begin{Entry}{仅}{4}{⼈}
  \begin{Phonetics}{仅}{jin3}[][HSK 3]
    \definition{adv.}{somente; meramente; por muito pouco}
  \end{Phonetics}
\end{Entry}

\begin{Entry}{仅仅}{4,4}{⼈、⼈}
  \begin{Phonetics}{仅仅}{jin3 jin3}[][HSK 3]
    \definition{adv.}{somente; meramente; por muito pouco; indica que está limitado a um determinado âmbito}
  \end{Phonetics}
\end{Entry}

\begin{Entry}{仅此而已}{4,6,6,3}{⼈、⽌、⽽、⼰}
  \begin{Phonetics}{仅此而已}{jin3ci3'er2yi3}
    \definition{adv.}{apenas isso e nada mais | isso é tudo}
  \end{Phonetics}
\end{Entry}

\begin{Entry}{仇}{4}{⼈}
  \begin{Phonetics}{仇}{chou2}[][HSK 7-9]
    \definition{s.}{inimigo; adversário | ódio; inimizade}
    \definition{v.}{odiar; detestar}
  \end{Phonetics}
  \begin{Phonetics}{仇}{qiu2}
    \definition*{s.}{Sobrenome Qiu}
    \definition{s.}{Literário: cônjuge; esposa; companheira}
  \end{Phonetics}
\end{Entry}

\begin{Entry}{仇人}{4,2}{⼈、⼈}
  \begin{Phonetics}{仇人}{chou2ren2}[][HSK 7-9]
    \definition[个]{s.}{inimigo; inimigo pessoal; pessoas que são hostis por causa do ódio}
  \end{Phonetics}
\end{Entry}

\begin{Entry}{仇外心理}{4,5,4,11}{⼈、⼣、⼼、⽟}
  \begin{Phonetics}{仇外心理}{chou2wai4xin1li3}
    \definition{s.}{xenofobia}[他的仇外心理很严重。===Ele tem uma mentalidade xenófoba séria.]
  \end{Phonetics}
\end{Entry}

\begin{Entry}{仇恨}{4,9}{⼈、⼼}
  \begin{Phonetics}{仇恨}{chou2hen4}[][HSK 7-9]
    \definition{s.}{ódio; inimizade; hostilidade; velho rancor}[他对她充满仇恨。===Ele estava cheio de ódio por ela.]
    \definition{v.}{odiar}
  \end{Phonetics}
\end{Entry}

\begin{Entry}{今}{4}{⼈}
  \begin{Phonetics}{今}{jin1}
    \definition*{s.}{Sobrenome Jin}
    \definition{s.}{agora; o presente | moderno (em oposição a 古) | de hoje; deste ano | isso; isto}
  \seealsoref{古}{gu3}
  \end{Phonetics}
\end{Entry}

\begin{Entry}{今人}{4,2}{⼈、⼈}
  \begin{Phonetics}{今人}{jin1ren2}
    \definition{s.}{contemporâneos; pessoas da nossa era; pessoas de hoje (oposto a古人) | pessoas modernas; pessoas contemporâneas}
  \seealsoref{古人}{gu3ren2}
  \end{Phonetics}
\end{Entry}

\begin{Entry}{今天}{4,4}{⼈、⼤}
  \begin{Phonetics}{今天}{jin1tian1}[][HSK 1]
    \definition{adv.}{hoje; neste dia | agora; o momento ou a época atual}
  \end{Phonetics}
\end{Entry}

\begin{Entry}{今日}{4,4}{⼈、⽇}
  \begin{Phonetics}{今日}{jin1 ri4}[][HSK 5]
    \definition{s.}{hoje}
  \end{Phonetics}
\end{Entry}

\begin{Entry}{今后}{4,6}{⼈、⼝}
  \begin{Phonetics}{今后}{jin1 hou4}[][HSK 2]
    \definition{s.}{a partir de agora; doravante; no futuro; desde o momento em que falamos}
  \end{Phonetics}
\end{Entry}

\begin{Entry}{今年}{4,6}{⼈、⼲}
  \begin{Phonetics}{今年}{jin1 nian2}[][HSK 1]
    \definition{adv.}{este ano}
  \end{Phonetics}
\end{Entry}

\begin{Entry}{介}{4}{⼈}
  \begin{Phonetics}{介}{jie4}
    \definition*{s.}{Sobrenome Jie}
    \definition{adj.}{direto; honesto e franco; correto}
    \definition{s.}{armadura | concha (crustáceos e criaturas aquáticas) | preposição}
    \definition{v.}{estar situado entre; interpor | levar a sério; levar em conta; ter em mente}
  \end{Phonetics}
\end{Entry}

\begin{Entry}{介绍}{4,8}{⼈、⽷}
  \begin{Phonetics}{介绍}{jie4shao4}[][HSK 1]
    \definition{s.}{introdução; apresentação}
    \definition{v.}{introduzir; apresentar | recomendar; sugerir | dar a conhecer; informar}
  \end{Phonetics}
\end{Entry}

\begin{Entry}{仍}{4}{⼈}
  \begin{Phonetics}{仍}{reng2}[][HSK 3]
    \definition{adv.}{ainda; repetidamente; frequentemente; continuamente}
    \definition{v.}{permanecer}
  \end{Phonetics}
\end{Entry}

\begin{Entry}{仍旧}{4,5}{⼈、⽇}
  \begin{Phonetics}{仍旧}{reng2jiu4}[][HSK 5]
    \definition{adv.}{ainda; ainda assim; contudo}
    \definition{v.}{permanecer igual; continuar sendo}
  \end{Phonetics}
\end{Entry}

\begin{Entry}{仍然}{4,12}{⼈、⽕}
  \begin{Phonetics}{仍然}{reng2ran2}[][HSK 3]
    \definition{adv.}{ainda; contudo; como antes; indica que a situação continua inalterada ou retorna ao seu estado original}
  \end{Phonetics}
\end{Entry}

\begin{Entry}{从}{4}{⼈}
  \begin{Phonetics}{从}{cong2}[][HSK 1]
    \definition*{s.}{Sobrenome Cong}
    \definition{adj.}{secundário; acessório | relacionamento entre primos, etc., do mesmo avô paterno, bisavô ou de um ancestral comum ainda mais antigo; do mesmo clã}
    \definition{adv.}{(seguido de uma negativa) jamais | jamais; usado antes de palavras negativas, indica que algo nunca aconteceu desde o passado, equivalente a 从来}
    \definition{prep.}{de (um tempo, um lugar ou um ponto de vista) | via, através ou após (um local) | de; via; através de; passado (um lugar); introdução das ações, trajetórias e locais| (de comportamento) de; introdução de ações e comportamentos com base em referências e fundamentos}
    \definition{s.}{seguidor; acompanhante}
    \definition{v.}{seguir; cumprir; obedecer | participar; estar envolvido em | seguir o princípio de; empregar o método de | estar envolvido em | seguir; adotar (um determinado princípio ou atitude)}
  \seealsoref{从来}{cong2lai2}
  \end{Phonetics}
\end{Entry}

\begin{Entry}{从小}{4,3}{⼈、⼩}
  \begin{Phonetics}{从小}{cong2 xiao3}[][HSK 2]
    \definition{adv.}{desde a infância; desde muito jovem; quando criança}
  \end{Phonetics}
\end{Entry}

\begin{Entry}{从不}{4,4}{⼈、⼀}
  \begin{Phonetics}{从不}{cong2bu4}[][HSK 6]
    \definition{adv.}{nunca; impossível (para expressar surpresa ou choque)}
  \end{Phonetics}
\end{Entry}

\begin{Entry}{从中}{4,4}{⼈、⼁}
  \begin{Phonetics}{从中}{cong2 zhong1}[][HSK 5]
    \definition{adv.}{de; dentre; daí}
  \end{Phonetics}
\end{Entry}

\begin{Entry}{从今以后}{4,4,4,6}{⼈、⼈、⼈、⼝}
  \begin{Phonetics}{从今以后}{cong2 jin1 yi3hou4}[][HSK 7-9]
    \definition{conj.}{de agora em diante; doravante}
  \end{Phonetics}
\end{Entry}

\begin{Entry}{从业}{4,5}{⼈、⼀}
  \begin{Phonetics}{从业}{cong2ye4}[][HSK 7-9]
    \definition{v.}{obter emprego | praticar (um ofício)}
  \end{Phonetics}
\end{Entry}

\begin{Entry}{从头}{4,5}{⼈、⼤}
  \begin{Phonetics}{从头}{cong2tou2}[][HSK 7-9]
    \definition{adv.}{desde o começo | de novo; mais uma vez}
  \end{Phonetics}
\end{Entry}

\begin{Entry}{从未}{4,5}{⼈、⽊}
  \begin{Phonetics}{从未}{cong2wei4}[][HSK 7-9]
    \definition{adv.}{nunca}
  \end{Phonetics}
\end{Entry}

\begin{Entry}{从早到晚}{4,6,8,11}{⼈、⽇、⼑、⽇}
  \begin{Phonetics}{从早到晚}{cong2zao3-dao4wan3}[][HSK 7-9]
    \definition{adv.}{do amanhecer ao anoitecer; da manhã à noite}
  \end{Phonetics}
\end{Entry}

\begin{Entry}{从此}{4,6}{⼈、⽌}
  \begin{Phonetics}{从此}{cong2ci3}[][HSK 4]
    \definition{conj.}{doravante; portanto; a partir deste momento; de agora em diante; a partir de então}
  \end{Phonetics}
\end{Entry}

\begin{Entry}{从而}{4,6}{⼈、⽽}
  \begin{Phonetics}{从而}{cong2'er2}[][HSK 5]
    \definition{conj.}{assim; por isso; portanto; desse modo; por esse motivo; conjunção usada no início do texto seguinte para expressar o resultado, propósito ou ação posterior, o que é equivalente a 因此就}
  \seealsoref{因此就}{yin1ci3 jiu4}
  \end{Phonetics}
\end{Entry}

\begin{Entry}{从来}{4,7}{⼈、⽊}
  \begin{Phonetics}{从来}{cong2lai2}[][HSK 3]
    \definition{adv.}{sempre; o tempo todo; em todos os momentos; do passado até o presente}
  \end{Phonetics}
\end{Entry}

\begin{Entry}{从来不}{4,7,4}{⼈、⽊、⼀}
  \begin{Phonetics}{从来不}{cong2lai2 bu4}[][HSK 7-9]
    \definition{adv.}{nunca; indica que algo não foi feito no passado ou no presente}
  \end{Phonetics}
\end{Entry}

\begin{Entry}{从没}{4,7}{⼈、⽔}
  \begin{Phonetics}{从没}{cong2 mei2}[][HSK 6]
    \definition{adv.}{nunca (no passado)}
  \end{Phonetics}
\end{Entry}

\begin{Entry}{从事}{4,8}{⼈、⼅}
  \begin{Phonetics}{从事}{cong2shi4}[][HSK 3]
    \definition{v.}{trabalhar; empreender; empenhar-se em; envolver-se em; dedicar-se ou comprometer-se (a uma causa); participar (de algo) | lidar com; deve ter um advérbio antes e não pode ter um objeto depois}
  \end{Phonetics}
\end{Entry}

\begin{Entry}{从前}{4,9}{⼈、⼑}
  \begin{Phonetics}{从前}{cong2qian2}[][HSK 3]
    \definition{s.}{antes; antigamente; no passado | era uma vez; há muito tempo atrás}
  \end{Phonetics}
\end{Entry}

\begin{Entry}{从容}{4,10}{⼈、⼧}
  \begin{Phonetics}{从容}{cong2rong2}[][HSK 7-9]
    \definition{adj.}{calmo; sem pressa; ao se deparar com situações especiais, sua expressão e atitude são calmas, sem nervosismo ou agitação | abundante (tempo, dinheiro); tempo ou dinheiro suficiente}
  \end{Phonetics}
\end{Entry}

\begin{Entry}{从容不迫}{4,10,4,8}{⼈、⼧、⼀、⾡}
  \begin{Phonetics}{从容不迫}{cong2rong2-bu2po4}[][HSK 7-9]
    \definition{expr.}{ir com calma e sem pressa; em etapas fáceis; calmo e sem pressa; levar o seu tempo; calmamente; vagarosamente; confiantemente e sem pressa; ir com calma; autoconfiante; de ​​maneira vagarosa}
  \end{Phonetics}
\end{Entry}

\begin{Entry}{仓}{4}{⼈}
  \begin{Phonetics}{仓}{cang1}
    \definition*{s.}{Sobrenome Cang}
    \definition{s.}{armazém; depósito}
  \end{Phonetics}
\end{Entry}

\begin{Entry}{仓库}{4,7}{⼈、⼴}
  \begin{Phonetics}{仓库}{cang1ku4}[][HSK 6]
    \definition[间,座,个]{s.}{armazém; depósito; um edifício usado para armazenar grandes quantidades de alimentos ou outros suprimentos}
  \end{Phonetics}
\end{Entry}

\begin{Entry}{以}{4}{⼈}
  \begin{Phonetics}{以}{yi3}
    \definition*{s.}{Sobrenome Yi}
    \definition{conj.}{e; bem como; o mesmo que 而 | de modo a; a fim de; usado no início da frase seguinte para expressar o propósito de atingir um determinado objetivo}
    \definition{prep.}{por; com | de acordo com | por causa de; porque | em (uma data fixa); em (um certo momento) | colocado antes de palavras posicionais simples, indica os limites de tempo, posição e quantidade}
  \seealsoref{而}{er2}
  \end{Phonetics}
\end{Entry}

\begin{Entry}{以上}{4,3}{⼈、⼀}
  \begin{Phonetics}{以上}{yi3 shang4}[][HSK 2]
    \definition[本]{s.}{mais do que; sobre; acima; indica posição, ordem ou número acima de um determinado ponto | o acima; o precedente; o acima mencionado; refere-se às palavras anteriores}
  \end{Phonetics}
\end{Entry}

\begin{Entry}{以下}{4,3}{⼈、⼀}
  \begin{Phonetics}{以下}{yi3 xia4}[][HSK 2]
    \definition[所]{s.}{abaixo; sob; indica posição, ordem ou número abaixo de um certo ponto | seguinte; refere-se às seguintes palavras}
  \end{Phonetics}
\end{Entry}

\begin{Entry}{以及}{4,3}{⼈、⼃}
  \begin{Phonetics}{以及}{yi3ji2}[][HSK 4]
    \definition{conj.}{assim como; juntamente como; bem como; também}
  \end{Phonetics}
\end{Entry}

\begin{Entry}{以为}{4,4}{⼈、⼂}
  \begin{Phonetics}{以为}{yi3wei2}[][HSK 2]
    \definition{v.}{pensar; acreditar; considerar (geralmente erroneamente); expressa opiniões e atitudes em relação às coisas, geralmente erradas}
  \end{Phonetics}
\end{Entry}

\begin{Entry}{以内}{4,4}{⼈、⼌}
  \begin{Phonetics}{以内}{yi3 nei4}[][HSK 4]
    \definition{adv.}{dentro de; menos que; não mais que; dentro de certos limites de tempo, premissas, quantidade e escopo}
  \end{Phonetics}
\end{Entry}

\begin{Entry}{以外}{4,5}{⼈、⼣}
  \begin{Phonetics}{以外}{yi3 wai4}[][HSK 2]
    \definition{s.}{além; exceto; fora; diferente de; fora dos limites de um determinado tempo, quantidade ou lugar}
  \end{Phonetics}
\end{Entry}

\begin{Entry}{以后}{4,6}{⼈、⼝}
  \begin{Phonetics}{以后}{yi3 hou4}[][HSK 2]
    \definition{s.}{depois; mais tarde; após; daqui em diante}
  \end{Phonetics}
\end{Entry}

\begin{Entry}{以此}{4,6}{⼈、⽌}
  \begin{Phonetics}{以此}{yi3ci3}
    \definition{adv.}{devido a esta | deste modo | por isso | com isso}
  \end{Phonetics}
\end{Entry}

\begin{Entry}{以至}{4,6}{⼈、⾄}
  \begin{Phonetics}{以至}{yi3zhi4}
    \definition{adv.}{até}
    \definition{conj.}{a tal ponto que\dots}
  \seealsoref{以至于}{yi3zhi4yu2}
  \end{Phonetics}
\end{Entry}

\begin{Entry}{以至于}{4,6,3}{⼈、⾄、⼆}
  \begin{Phonetics}{以至于}{yi3zhi4yu2}
    \definition{adv.}{até}
    \definition{conj.}{na medida em que\dots}
  \seealsoref{以至}{yi3zhi4}
  \end{Phonetics}
\end{Entry}

\begin{Entry}{以色列}{4,6,6}{⼈、⾊、⼑}
  \begin{Phonetics}{以色列}{yi3se4lie4}
    \definition*{s.}{Israel}
  \end{Phonetics}
\end{Entry}

\begin{Entry}{以免}{4,7}{⼈、⼉}
  \begin{Phonetics}{以免}{yi3mian3}
    \definition{conj.}{para evitar isso}
  \end{Phonetics}
\end{Entry}

\begin{Entry}{以来}{4,7}{⼈、⽊}
  \begin{Phonetics}{以来}{yi3lai2}[][HSK 3]
    \definition{s.}{desde (em termos de tempo); indica um período de tempo desde um determinado momento no passado até o presente}
  \end{Phonetics}
\end{Entry}

\begin{Entry}{以求}{4,7}{⼈、⽔}
  \begin{Phonetics}{以求}{yi3qiu2}
    \definition{conj.}{a fim de}
  \end{Phonetics}
\end{Entry}

\begin{Entry}{以往}{4,8}{⼈、⼻}
  \begin{Phonetics}{以往}{yi3wang3}[][HSK 5]
    \definition{s.}{antes; anterior; no passado}
  \end{Phonetics}
\end{Entry}

\begin{Entry}{以便}{4,9}{⼈、⼈}
  \begin{Phonetics}{以便}{yi3bian4}[][HSK 5]
    \definition{conj.}{para que; de modo que; a fim de; com o objetivo de; para o propósito de; usado no início da segunda parte da frase, indica que o objetivo mencionado na segunda parte será facilmente alcançado}
  \end{Phonetics}
\end{Entry}

\begin{Entry}{以前}{4,9}{⼈、⼑}
  \begin{Phonetics}{以前}{yi3qian2}[][HSK 2]
    \definition{s.}{antes; antigamente; anteriormente (no tempo); agora ou o período anterior ao tempo indicado}
  \end{Phonetics}
\end{Entry}

\begin{Entry}{以期}{4,12}{⼈、⽉}
  \begin{Phonetics}{以期}{yi3qi1}
    \definition{v.}{tentando | esperando | esperando por}
  \end{Phonetics}
\end{Entry}

\begin{Entry}{允}{4}{⼉}
  \begin{Phonetics}{允}{yun3}
    \definition*{s.}{Sobrenome Yun}
    \definition{adj.}{justo; imparcial}
    \definition{v.}{permitir; deixar; consentir}
  \end{Phonetics}
\end{Entry}

\begin{Entry}{允许}{4,6}{⼉、⾔}
  \begin{Phonetics}{允许}{yun3xu3}[][HSK 6]
    \definition{s.}{permitido; permissão}
    \definition{v.}{permitir; deixar; concordar com alguém para fazer algo}
  \end{Phonetics}
\end{Entry}

\begin{Entry}{元}{4}{⼉}
  \begin{Phonetics}{元}{yuan2}[][HSK 1]
    \definition*{s.}{Dinastia Yuan (1271-1368) | Sobrenome Yuan}
    \definition{adj.}{primeiro; principal; primário | chefe; diretor; líder | básico; fundamental; principal | unidade; componente; formando um todo}
    \definition{clas.}{yuan, a unidade monetária da China}
    \definition{s.}{moeda com valor e peso fixos | origem; elemento}
  \end{Phonetics}
\end{Entry}

\begin{Entry}{元气}{4,4}{⼉、⽓}
  \begin{Phonetics}{元气}{yuan2qi4}
    \definition{s.}{força | vigor | vitalidade | energial vital}
  \end{Phonetics}
\end{Entry}

\begin{Entry}{元旦}{4,5}{⼉、⽇}
  \begin{Phonetics}{元旦}{yuan2dan4}[][HSK 5]
    \definition*{s.}{Dia de Ano Novo (1 de janeiro)}
  \end{Phonetics}
\end{Entry}

\begin{Entry}{元来}{4,7}{⼉、⽊}
  \begin{Phonetics}{元来}{yuan2lai2}
    \variantof{原来}
  \end{Phonetics}
\end{Entry}

\begin{Entry}{元夜}{4,8}{⼉、⼣}
  \begin{Phonetics}{元夜}{yuan2ye4}
    \definition*{s.}{Festival das Lanternas}
  \seealsoref{元宵}{yuan2xiao1}
  \seealsoref{元宵节}{yuan2xiao1jie2}
  \end{Phonetics}
\end{Entry}

\begin{Entry}{元宵}{4,10}{⼉、⼧}
  \begin{Phonetics}{元宵}{yuan2xiao1}
    \definition*{s.}{Festival das Lanternas}
  \seealsoref{元宵节}{yuan2xiao1jie2}
  \seealsoref{元夜}{yuan2ye4}
  \end{Phonetics}
\end{Entry}

\begin{Entry}{元宵节}{4,10,5}{⼉、⼧、⾋}
  \begin{Phonetics}{元宵节}{yuan2xiao1jie2}
    \definition*{s.}{Festival das Lanternas (15º~dia do primeiro mês lunar)}
  \seealsoref{元宵}{yuan2xiao1}
  \seealsoref{元夜}{yuan2ye4}
  \end{Phonetics}
\end{Entry}

\begin{Entry}{元素}{4,10}{⼉、⽷}
  \begin{Phonetics}{元素}{yuan2su4}[][HSK 6]
    \definition{s.}{elemento; fator essencial | elemento químico | Geometria: as partes que compõem uma figura, como os lados e ângulos de um triângulo | Algebra: os elementos algébricos incluem números e símbolos}
  \end{Phonetics}
\end{Entry}

\begin{Entry}{公}{4}{⼋}
  \begin{Phonetics}{公}{gong1}[][HSK 6]
    \definition*{s.}{Sobrenome Gong}
    \definition{adj.}{público; estatal; coletivo (oposto a 私) | comum; geral | do mundo; internacional; universal; métrico | imparcial; justo; equitativo |  (de um animal) masculino (oposto a 母)}
    \definition{s.}{assuntos públicos; negócios oficiais (ou deveres) (oposto a 私) | autoridade; coletivo | duque | títulos respeitosos para homens idosos; uma saudação respeitosa | marido}
    \definition{v.}{tornar público; divulgar; abrir a todos; exibir}
  \seealsoref{母}{mu3}
  \seealsoref{私}{si1}
  \end{Phonetics}
\end{Entry}

\begin{Entry}{公仆}{4,4}{⼋、⼈}
  \begin{Phonetics}{公仆}{gong1pu2}[][HSK 7-9]
    \definition[个,位,名,些]{s.}{servidor público; oficial | funcionário público; pessoas que servem o público}
  \end{Phonetics}
\end{Entry}

\begin{Entry}{公元}{4,4}{⼋、⼉}
  \begin{Phonetics}{公元}{gong1yuan2}[][HSK 4]
    \definition{s.}{D.C. (Depois de~Cristo); a era cristã; um método internacionalmente aceito de registro de datas, o ano lendário do nascimento de Jesus é 1 d.C., também conhecido como o primeiro ano da Era Comum, e é denotado por D.C.}
  \seealsoref{前}{qian2}
  \end{Phonetics}
\end{Entry}

\begin{Entry}{公开}{4,4}{⼋、⼶}
  \begin{Phonetics}{公开}{gong1kai1}[][HSK 3]
    \definition{adj.}{aberto; público; não oculto; exposto ao público}
    \definition{v.}{tornar público}
  \end{Phonetics}
\end{Entry}

\begin{Entry}{公开信}{4,4,9}{⼋、⼶、⼈}
  \begin{Phonetics}{公开信}{gong1kai1xin4}[][HSK 7-9]
    \definition{s.}{carta aberta; cartas endereçadas a indivíduos ou grupos que o autor acredita serem necessárias para divulgação pública}
  \end{Phonetics}
\end{Entry}

\begin{Entry}{公斤}{4,4}{⼋、⽄}
  \begin{Phonetics}{公斤}{gong1jin1}[][HSK 2]
    \definition{clas.}{quilograma (kg)}
  \end{Phonetics}
\end{Entry}

\begin{Entry}{公认}{4,4}{⼋、⾔}
  \begin{Phonetics}{公认}{gong1ren4}[][HSK 5]
    \definition{v.}{(geralmente) reconhecer; (universalmente) aceitar}
  \end{Phonetics}
\end{Entry}

\begin{Entry}{公车}{4,4}{⼋、⾞}
  \begin{Phonetics}{公车}{gong1che1}[][HSK 7-9]
    \definition[辆]{s.}{ônibus, abreviação de 公共汽车 | carro pertencente a uma organização e usado por seus membros (carro do governo, carro de polícia, carro da empresa etc.), abreviação de 公务用车}
  \seealsoref{公共}{gong1 gong4}
  \seealsoref{公共汽车}{gong1gong4qi4che1}
  \seealsoref{公务用车}{gong1wu4yong4che1}
  \end{Phonetics}
\end{Entry}

\begin{Entry}{公主}{4,5}{⼋、⼂}
  \begin{Phonetics}{公主}{gong1zhu3}[][HSK 6]
    \definition[个,位,名,些]{s.}{princesa; a filha do monarca}
  \end{Phonetics}
\end{Entry}

\begin{Entry}{公务}{4,5}{⼋、⼒}
  \begin{Phonetics}{公务}{gong1wu4}[][HSK 7-9]
    \definition{s.}{assuntos públicos; negócios oficiais; em relação a assuntos nacionais ou coletivos}
  \end{Phonetics}
\end{Entry}

\begin{Entry}{公务用车}{4,5,5,4}{⼋、⼒、⽤、⾞}
  \begin{Phonetics}{公务用车}{gong1wu4yong4che1}
    \definition{s.}{veículos oficiais}
  \end{Phonetics}
\end{Entry}

\begin{Entry}{公务员}{4,5,7}{⼋、⼒、⼝}
  \begin{Phonetics}{公务员}{gong1 wu4 yuan2}[][HSK 3]
    \definition[个,位,名,些]{s.}{funcionário público; funcionário de órgãos governamentais}
  \end{Phonetics}
\end{Entry}

\begin{Entry}{公司}{4,5}{⼋、⼝}
  \begin{Phonetics}{公司}{gong1si1}[][HSK 2]
    \definition[个,家]{s.}{empresa; companhia; corporação; uma organização industrial e comercial que opera na produção de produtos, circulação de mercadorias ou certos empreendimentos de construção, etc.}
  \end{Phonetics}
\end{Entry}

\begin{Entry}{公司治理}{4,5,8,11}{⼋、⼝、⽔、⽟}
  \begin{Phonetics}{公司治理}{gong1si1zhi4li3}
    \definition{s.}{governança corporativa}
  \end{Phonetics}
\end{Entry}

\begin{Entry}{公布}{4,5}{⼋、⼱}
  \begin{Phonetics}{公布}{gong1bu4}[][HSK 3]
    \definition{v.}{(leis, decretos, comunicados e avisos de órgãos governamentais) promulgar; anunciar; publicar; tornar público; divulgar publicamente}
  \end{Phonetics}
\end{Entry}

\begin{Entry}{公平}{4,5}{⼋、⼲}
  \begin{Phonetics}{公平}{gong1ping2}[][HSK 2]
    \definition{adj.}{justo; imparcial; equitativo; equidade}
  \end{Phonetics}
\end{Entry}

\begin{Entry}{公正}{4,5}{⼋、⽌}
  \begin{Phonetics}{公正}{gong1zheng4}[][HSK 5]
    \definition{adj.}{justo; equitativo; imparcial; de mente justa; equidade e integridade sem favoritismo}
  \end{Phonetics}
\end{Entry}

\begin{Entry}{公民}{4,5}{⼋、⽒}
  \begin{Phonetics}{公民}{gong1min2}[][HSK 3]
    \definition[个,位]{s.}{cidadão; civil; pessoa que possui a nacionalidade de um país, goza dos direitos e cumpre as obrigações previstos na Constituição e nas demais leis desse país}
  \end{Phonetics}
\end{Entry}

\begin{Entry}{公用}{4,5}{⼋、⽤}
  \begin{Phonetics}{公用}{gong1yong4}[][HSK 7-9]
    \definition{adj.}{público; comunitário; para uso público, uso comum}
  \end{Phonetics}
\end{Entry}

\begin{Entry}{公用电话}{4,5,5,8}{⼋、⽤、⽥、⾔}
  \begin{Phonetics}{公用电话}{gong1yong4dian4hua4}
    \definition[部]{s.}{telefone público}
  \end{Phonetics}
\end{Entry}

\begin{Entry}{公示}{4,5}{⼋、⽰}
  \begin{Phonetics}{公示}{gong1shi4}[][HSK 7-9]
    \definition{v.}{dar a conhecer ao público e pedir opiniões}
  \end{Phonetics}
\end{Entry}

\begin{Entry}{公立}{4,5}{⼋、⽴}
  \begin{Phonetics}{公立}{gong1li4}[][HSK 7-9]
    \definition{adj.}{estabelecido e mantido pelo governo; público (oposto a 私立) | financiado publicamente e administrado pelo governo; estabelecido pelo governo e operado com fundos governamentais para fornecer serviços ao público}
  \seealsoref{私立}{si1li4}
  \end{Phonetics}
\end{Entry}

\begin{Entry}{公交车}{4,6,4}{⼋、⼇、⾞}
  \begin{Phonetics}{公交车}{gong1 jiao1 che1}[][HSK 2]
    \definition[辆]{s.}{ônibus urbano; veículo de transporte público}
  \end{Phonetics}
\end{Entry}

\begin{Entry}{公众}{4,6}{⼋、⼈}
  \begin{Phonetics}{公众}{gong1 zhong4}[][HSK 6]
    \definition[对]{s.}{o público; as massas; refere-se à maioria das pessoas na sociedade}
  \end{Phonetics}
\end{Entry}

\begin{Entry}{公共}{4,6}{⼋、⼋}
  \begin{Phonetics}{公共}{gong1 gong4}[][HSK 3]
    \definition{adj.}{público; comum; comunal; comunitário; pertencente à sociedade}
    \definition[辆]{s.}{ônibus}
  \seealsoref{公车}{gong1che1}
  \seealsoref{公共汽车}{gong1gong4qi4che1}
  \end{Phonetics}
\end{Entry}

\begin{Entry}{公共关系}{4,6,6,7}{⼋、⼋、⼋、⽷}
  \begin{Phonetics}{公共关系}{gong1gong4 guan1xi4}
    \definition{s.}{relações públicas; refere-se à relação entre grupos, empresas ou indivíduos em atividades sociais, denominada relações públicas}
  \end{Phonetics}
\end{Entry}

\begin{Entry}{公共场所}{4,6,6,8}{⼋、⼋、⼟、⼾}
  \begin{Phonetics}{公共场所}{gong1gong4 chang3suo3}[][HSK 7-9]
    \definition{s.}{lugares públicos; locais onde o público pode ir}
  \end{Phonetics}
\end{Entry}

\begin{Entry}{公共汽车}{4,6,7,4}{⼋、⼋、⽔、⾞}
  \begin{Phonetics}{公共汽车}{gong1gong4qi4che1}[][HSK 2]
    \definition[辆,个]{s.}{ônibus}
  \seealsoref{公共}{gong1 gong4}
  \seealsoref{公车}{gong1che1}
  \end{Phonetics}
\end{Entry}

\begin{Entry}{公关}{4,6}{⼋、⼋}
  \begin{Phonetics}{公关}{gong1guan1}[][HSK 7-9]
    \definition{s.}{relações públicas, abreviação de 公共关系 | pessoa que trabalha em relações públicas}
  \seealsoref{公共关系}{gong1gong4 guan1xi4}
  \end{Phonetics}
\end{Entry}

\begin{Entry}{公安}{4,6}{⼋、⼧}
  \begin{Phonetics}{公安}{gong1 an1}[][HSK 6]
    \definition[名,位]{s.}{segurança pública; a segurança e estabilidade dos direitos dos cidadãos, da propriedade da segurança pública e da ordem social | agente de segurança pública; pessoal que mantém a segurança pública}
  \end{Phonetics}
\end{Entry}

\begin{Entry}{公安局}{4,6,7}{⼋、⼧、⼫}
  \begin{Phonetics}{公安局}{gong1'an1ju2}[][HSK 7-9]
    \definition{s.}{departamento de polícia; departamento de segurança pública; o departamento responsável pelo trabalho de segurança pública do Governo Popular}
  \end{Phonetics}
\end{Entry}

\begin{Entry}{公式}{4,6}{⼋、⼷}
  \begin{Phonetics}{公式}{gong1shi4}[][HSK 5]
    \definition[个,些,种]{s.}{fórmula; expressão}
  \end{Phonetics}
\end{Entry}

\begin{Entry}{公约}{4,6}{⼋、⽷}
  \begin{Phonetics}{公约}{gong1yue1}[][HSK 7-9]
    \definition[项,条]{s.}{convenção; pacto; aliança; tratado | compromisso conjunto; regulamentos acordados coletivamente dentro de uma organização | regulamentos acordados coletivamente dentro de uma unidade de trabalho}
  \end{Phonetics}
\end{Entry}

\begin{Entry}{公克}{4,7}{⼋、⼗}
  \begin{Phonetics}{公克}{gong1ke4}
    \definition{s.}{grama (medida de peso)}
  \end{Phonetics}
\end{Entry}

\begin{Entry}{公告}{4,7}{⼋、⼝}
  \begin{Phonetics}{公告}{gong1gao4}[][HSK 5]
    \definition[张,份,项]{s.}{anúncio; notificação de assuntos importantes ao público em geral pelo governo ou por um órgão importante}
    \definition{v.}{anunciar; o governo ou órgão governamental informa publicamente às pessoas algo importante}
  \end{Phonetics}
\end{Entry}

\begin{Entry}{公园}{4,7}{⼋、⼞}
  \begin{Phonetics}{公园}{gong1yuan2}[][HSK 2]
    \definition[个,座]{s.}{parque; jardim público; os jardins abertos ao público para passeios e descanso geralmente ficam nas cidades, têm muitas flores, árvores e, em alguns casos, lagos}
  \end{Phonetics}
\end{Entry}

\begin{Entry}{公证}{4,7}{⼋、⾔}
  \begin{Phonetics}{公证}{gong1zheng4}[][HSK 7-9]
    \definition{adj.}{autenticado em cartório}
    \definition{s.}{reconhecimento de firma}
    \definition{v.}{autenticar}
  \end{Phonetics}
\end{Entry}

\begin{Entry}{公里}{4,7}{⼋、⾥}
  \begin{Phonetics}{公里}{gong1li3}[][HSK 2]
    \definition{s.}{quilômetro (km)}
  \end{Phonetics}
\end{Entry}

\begin{Entry}{公鸡}{4,7}{⼋、⿃}
  \begin{Phonetics}{公鸡}{gong1 ji1}[][HSK 6]
    \definition{s.}{galo; frango macho}
  \end{Phonetics}
\end{Entry}

\begin{Entry}{公事}{4,8}{⼋、⼅}
  \begin{Phonetics}{公事}{gong1shi4}[][HSK 7-9]
    \definition{s.}{assuntos públicos; negócios oficiais (ou deveres) (oposto a 私事) | Coloquial: documento oficial}
  \seealsoref{私事}{si1shi4}
  \end{Phonetics}
\end{Entry}

\begin{Entry}{公函}{4,8}{⼋、⼐}
  \begin{Phonetics}{公函}{gong1han2}[][HSK 7-9]
    \definition{s.}{carta oficial; correspondência oficial entre departamentos paralelos e não relacionados (oposto a 便函, 私函)}
  \seealsoref{便函}{bian4han2}
  \seealsoref{私函}{si1han2}
  \end{Phonetics}
\end{Entry}

\begin{Entry}{公顷}{4,8}{⼋、⾴}
  \begin{Phonetics}{公顷}{gong1qing3}[][HSK 7-9]
    \definition{clas.}{hectare; é uma unidade de área terrestre no sistema métrico;  equivale a 10.000 metros quadrados, ou 15 市亩}
  \seealsoref{市亩}{shi4mu3}
  \end{Phonetics}
\end{Entry}

\begin{Entry}{公费}{4,9}{⼋、⾙}
  \begin{Phonetics}{公费}{gong1fei4}[][HSK 7-9]
    \definition{s.}{despesa pública; despesas fornecidas pelo estado ou grupo}
  \end{Phonetics}
\end{Entry}

\begin{Entry}{公益}{4,10}{⼋、⽫}
  \begin{Phonetics}{公益}{gong1yi4}[][HSK 7-9]
    \definition{s.}{bem público; comunitário; bem-estar; interesse público (referindo-se principalmente ao bem-estar público, como saúde e assistência)}
  \end{Phonetics}
\end{Entry}

\begin{Entry}{公益性}{4,10,8}{⼋、⽫、⼼}
  \begin{Phonetics}{公益性}{gong1yi4xing4}[][HSK 7-9]
    \definition{s.}{bem-estar público}
  \end{Phonetics}
\end{Entry}

\begin{Entry}{公积金}{4,10,8}{⼋、⽲、⾦}
  \begin{Phonetics}{公积金}{gong1ji1jin1}[][HSK 7-9]
    \definition{s.}{fundo de acumulação (comum); fundo de reserva pública; fundo de reserva comum | fundo acumulado | reservas oficiais}
  \end{Phonetics}
\end{Entry}

\begin{Entry}{公职}{4,11}{⼋、⽿}
  \begin{Phonetics}{公职}{gong1zhi2}[][HSK 7-9]
    \definition{s.}{cargo público; emprego público; cargos ou patentes oficiais}
  \end{Phonetics}
\end{Entry}

\begin{Entry}{公募}{4,12}{⼋、⼒}
  \begin{Phonetics}{公募}{gong1mu4}[][HSK 7-9]
    \definition{s.}{financiamento público; arrecadação pública de fundos (investimento)}
  \end{Phonetics}
\end{Entry}

\begin{Entry}{公寓}{4,12}{⼋、⼧}
  \begin{Phonetics}{公寓}{gong1yu4}[][HSK 7-9]
    \definition[套,座,栋,间]{s.}{apartamentos; moradias; colônias de férias; prédio de apartamentos; construções que podem acomodar várias famílias}
  \end{Phonetics}
\end{Entry}

\begin{Entry}{公款}{4,12}{⼋、⽋}
  \begin{Phonetics}{公款}{gong1kuan3}[][HSK 7-9]
    \definition{s.}{dinheiro público (ou fundo); despesa do governo | fundos públicos}
  \end{Phonetics}
\end{Entry}

\begin{Entry}{公然}{4,12}{⼋、⽕}
  \begin{Phonetics}{公然}{gong1ran2}[][HSK 7-9]
    \definition{adv.}{Pejorativo: abertamente; publicamente; sem disfarces; descaradamente; flagrantemente (oposto a 暗自)}
  \seealsoref{暗自}{an4zi4}
  \end{Phonetics}
\end{Entry}

\begin{Entry}{公道}{4,12}{⼋、⾡}
  \begin{Phonetics}{公道}{gong1dao4}
    \definition{s.}{justiça; o princípio da justiça}
  \end{Phonetics}
  \begin{Phonetics}{公道}{gong1dao5}[][HSK 7-9]
    \definition{adj.}{equitativo | justo}
  \end{Phonetics}
\end{Entry}

\begin{Entry}{公路}{4,13}{⼋、⾜}
  \begin{Phonetics}{公路}{gong1 lu4}[][HSK 2]
    \definition[条,段]{s.}{rodovia; via de acesso; via de tráfego; estrada; estrada principal;}
  \end{Phonetics}
\end{Entry}

\begin{Entry}{六}{4}{⼋}
  \begin{Phonetics}{六}{liu4}[][HSK 1]
    \definition*{s.}{Sobrenome Liu}
    \definition{num.}{seis; 6}
    \definition{s.}{símbolo musical utilizado na partitura da música tradicional chinesa, representando o primeiro grau da escala musical, equivalente ao ``5'' da notação musical simplificada}
  \end{Phonetics}
\end{Entry}

\begin{Entry}{内}{4}{⼌}
  \begin{Phonetics}{内}{nei4}[][HSK 3]
    \definition*{s.}{Sobrenome Nei}
    \definition{s.}{dentro; interior; parte interna ou lateral (oposto de 外) |  (antes de um substantivo ou verbo na formação de uma palavra composta) interno | (depois de um substantivo para indicar lugar, tempo, escopo ou limites) dentro; em | coração; mente | esposa ou parentes da esposa}
  \seealsoref{外}{wai4}
  \end{Phonetics}
\end{Entry}

\begin{Entry}{内心}{4,4}{⼌、⼼}
  \begin{Phonetics}{内心}{nei4 xin1}[][HSK 3]
    \definition{s.}{coração; interior; íntimo do ser}
  \end{Phonetics}
\end{Entry}

\begin{Entry}{内外}{4,5}{⼌、⼣}
  \begin{Phonetics}{内外}{nei4 wai4}[][HSK 6]
    \definition{s.}{dentro e fora; nacional e estrangeiro; interno e externo | ao redor; aproximadamente; número aproximado de exibição}
  \end{Phonetics}
\end{Entry}

\begin{Entry}{内在}{4,6}{⼌、⼟}
  \begin{Phonetics}{内在}{nei4zai4}[][HSK 5]
    \definition{adj.}{intrínseco; algo que existe em si mesmo, mas que não pode ser descoberto através da observação direta | interno; imanente; difícil de perceber}
  \end{Phonetics}
\end{Entry}

\begin{Entry}{内地}{4,6}{⼌、⼟}
  \begin{Phonetics}{内地}{nei4 di4}[][HSK 6]
    \definition{s.}{interior; sertão | China continental (RPC excluindo Hong Kong e Macau, mas incluindo ilhas como Hainan) | Japão (usado em Taiwan durante a colonização japonesa)}
  \end{Phonetics}
\end{Entry}

\begin{Entry}{内存}{4,6}{⼌、⼦}
  \begin{Phonetics}{内存}{nei4cun2}
    \definition{s.}{armazenamento interno | memória do computador | RAM (\emph{random access memory})}
  \seealsoref{随机存取存储器}{sui2ji1cun2qu3cun2chu3qi4}
  \seealsoref{随机存取记忆体}{sui2ji1cun2qu3ji4yi4ti3}
  \end{Phonetics}
\end{Entry}

\begin{Entry}{内衣}{4,6}{⼌、⾐}
  \begin{Phonetics}{内衣}{nei4 yi1}[][HSK 6]
    \definition[件,个]{s.}{roupa íntima}
  \end{Phonetics}
\end{Entry}

\begin{Entry}{内省}{4,9}{⼌、⽬}
  \begin{Phonetics}{内省}{nei4xing3}
    \definition{s.}{introspecção}
    \definition{v.}{refletir sobre si mesmo}
  \end{Phonetics}
\end{Entry}

\begin{Entry}{内科}{4,9}{⼌、⽲}
  \begin{Phonetics}{内科}{nei4ke1}[][HSK 4]
    \definition{s.}{medicina geral; clínica geral; clínica médica}
  \end{Phonetics}
\end{Entry}

\begin{Entry}{内容}{4,10}{⼌、⼧}
  \begin{Phonetics}{内容}{nei4rong2}[][HSK 3]
    \definition[份,个,项]{s.}{conteúdo; substância; a substância ou significado contido em algo}
  \end{Phonetics}
\end{Entry}

\begin{Entry}{内资}{4,10}{⼌、⾙}
  \begin{Phonetics}{内资}{nei4 zi1}
    \definition{s.}{capital nacional; financiamento interno; investimento de fontes nacionais (em oposição a 外资)}
  \seealsoref{外资}{wai4 zi1}
  \end{Phonetics}
\end{Entry}

\begin{Entry}{内部}{4,10}{⼌、⾢}
  \begin{Phonetics}{内部}{nei4bu4}[][HSK 4]
    \definition{s.}{interior; dentro; interno; dentro de um determinado intervalo}
  \end{Phonetics}
\end{Entry}

\begin{Entry}{内燃机}{4,16,6}{⼌、⽕、⽊}
  \begin{Phonetics}{内燃机}{nei4ran2ji1}
    \definition{s.}{motor de combustão interna}
  \end{Phonetics}
\end{Entry}

\begin{Entry}{凤}{4}{⼏}
  \begin{Phonetics}{凤}{feng4}
    \definition[只]{s.}{Mitologia: fênix}
    \definition{s.}{Sobrenome Feng}
  \end{Phonetics}
\end{Entry}

\begin{Entry}{凤凰}{4,11}{⼏、⼏}
  \begin{Phonetics}{凤凰}{feng4huang2}[][HSK 7-9]
    \definition[只,个,对]{s.}{fênix; o rei dos pássaros nas lendas antigas, com belas penas, o macho é chamado de 凤 e a fêmea é chamada de 凰, frequentemente usado para simbolizar auspiciosidade}
  \seealsoref{凤}{feng4}
  \seealsoref{凰}{huang2}
  \end{Phonetics}
\end{Entry}

\begin{Entry}{凶}{4}{⼐}
  \begin{Phonetics}{凶}{xiong1}[][HSK 6]
    \definition{adj.}{sinistro; desfavorável; azarado (oposto de "吉") | ruim para as colheitas; improdutivo; ameaçado pela fome | feroz; vicioso; cruel | medroso; terrível}
    \definition[个]{s.}{mal; assassinato; ato de violência; atos de matar ou ferir pessoas | assassino; malfeitor; criminoso; pessoa má; pessoa violenta}
    \definition{v.}{ser feroz; tratar cruelmente}
  \seealsoref{吉}{ji2}
  \end{Phonetics}
\end{Entry}

\begin{Entry}{凶手}{4,4}{⼐、⼿}
  \begin{Phonetics}{凶手}{xiong1shou3}[][HSK 6]
    \definition[名]{s.}{assassino; homicida | agressor (que causou ferimentos a alguém)}
  \end{Phonetics}
\end{Entry}

\begin{Entry}{分}{4}{⼑}
  \begin{Phonetics}{分}{fen1}[][HSK 1,2]
    \definition{adj.}{filial (de uma organização)}
    \definition{clas.}{parte ou subdivisão | fração | um décimo (de certas unidades) | unidade de comprimento equivalente a 0,33cm | unidade de área (=66,666 metros quadrados) | unidade de peso (=1/2 grama) | minuto (unidade de tempo) | minuto (unidade de medida angular) | 0,01 yuan (unidade de dinheiro) | taxa de juros | marca; ponto; unidade de contagem para avaliação de notas, etc.}
    \definition{s.}{fração}
    \definition{v.}{separar; dividir; partir; dividir algo inteiro em várias partes ou separar coisas que estão ligadas entre si | atribuir; designar; distribuir | distinguir; diferenciar; diferenciar um do outro}
  \end{Phonetics}
  \begin{Phonetics}{分}{fen4}[][HSK 2]
    \definition{s.}{componente | o que está dentro dos deveres ou direitos de alguém; limites das responsabilidades e direitos | afeto; sentimento de amizade}
    \definition{v.}{pensar; esperar; estimar}
  \end{Phonetics}
\end{Entry}

\begin{Entry}{分之}{4,3}{⼑、⼂}
  \begin{Phonetics}{分之}{fen1 zhi1}[][HSK 4]
    \definition{expr.}{indicando uma fração; formatação e leitura de frações, ou seja, partes de um total}[顾客减少了三分之一。===O número de clientes caiu em um terço.]
  \end{Phonetics}
\end{Entry}

\begin{Entry}{分子}{4,3}{⼑、⼦}
  \begin{Phonetics}{分子}{fen1zi3}
    \definition{s.}{molécula | (matemática) numerador de uma fração}
  \end{Phonetics}
  \begin{Phonetics}{分子}{fen4zi3}
    \definition{s.}{membros de uma classe ou grupo | elementos políticos (como intelectuais ou extremistas)}
  \end{Phonetics}
\end{Entry}

\begin{Entry}{分寸}{4,3}{⼑、⼨}
  \begin{Phonetics}{分寸}{fen1cun5}[][HSK 7-9]
    \definition{s.}{limites adequados para a fala ou ação; senso de propriedade (ou proporção)}
  \end{Phonetics}
\end{Entry}

\begin{Entry}{分工}{4,3}{⼑、⼯}
  \begin{Phonetics}{分工}{fen1 gong1}[][HSK 6]
    \definition[种]{s.}{divisão do trabalho}
    \definition{v.}{dividir o trabalho; envolver-se em várias tarefas diferentes, mas complementares}
  \end{Phonetics}
\end{Entry}

\begin{Entry}{分为}{4,4}{⼑、⼂}
  \begin{Phonetics}{分为}{fen1 wei2}[][HSK 4]
    \definition{v.}{subdividir; dividir algo em}
  \end{Phonetics}
\end{Entry}

\begin{Entry}{分公司}{4,4,5}{⼑、⼋、⼝}
  \begin{Phonetics}{分公司}{fen1gong1si1}
    \definition{s.}{sucursal | filial de companhia}
  \end{Phonetics}
\end{Entry}

\begin{Entry}{分化}{4,4}{⼑、⼔}
  \begin{Phonetics}{分化}{fen1hua4}[][HSK 7-9]
    \definition{s.}{diferenciação}
    \definition{v.}{dividir-se; fragmentar; romper; separar-se; separar | Biologia: (células ou tecidos) diferenciar}
  \end{Phonetics}
\end{Entry}

\begin{Entry}{分开}{4,4}{⼑、⼶}
  \begin{Phonetics}{分开}{fen1/kai1}[][HSK 2]
    \definition{v.+compl.}{separar; dividir; desacoplar; desembalar; romper; desfolhar; decolar; romper; distribuir; separar de (em); dividir\dots de\dots | separar; fazer com que uma pessoa ou algo deixe de estar junto com outra pessoa ou coisa}
  \end{Phonetics}
\end{Entry}

\begin{Entry}{分手}{4,4}{⼑、⼿}
  \begin{Phonetics}{分手}{fen1/shou3}[][HSK 4]
    \definition{v.+compl.}{separar; romper; terminar um relacionamento ou um casal | separar-se (de uma empresa); dizer adeus; despedir-se da família, dos amigos, etc.}
  \end{Phonetics}
\end{Entry}

\begin{Entry}{分支}{4,4}{⼑、⽀}
  \begin{Phonetics}{分支}{fen1zhi1}[][HSK 7-9]
    \definition{s.}{subdivisão; filial; afiliada; uma parte separada de um sistema ou corpo}
  \end{Phonetics}
\end{Entry}

\begin{Entry}{分外}{4,5}{⼑、⼣}
  \begin{Phonetics}{分外}{fen4wai4}[][HSK 7-9]
    \definition{adj.}{não é tarefa (dever) de alguém; além do dever de alguém; fora do escopo do dever de alguém}[他对分外的工作总是抢着干。===Ele está sempre disposto e ansioso para fazer trabalho extra.]
    \definition{adv.}{particularmente; especialmente; excepcionalmente}[雨后初晴的天空分外明朗。===O céu depois da chuva está excepcionalmente claro.]
  \end{Phonetics}
\end{Entry}

\begin{Entry}{分布}{4,5}{⼑、⼱}
  \begin{Phonetics}{分布}{fen1bu4}[][HSK 4]
    \definition{v.}{espalhar; distribuir; dispersar (em uma determinada área)}
  \end{Phonetics}
\end{Entry}

\begin{Entry}{分成}{4,6}{⼑、⼽}
  \begin{Phonetics}{分成}{fen1 cheng2}[][HSK 5]
    \definition{v.}{dividir em; separar em; dividir dinheiro, bens, etc. de acordo com a porcentagem}
  \end{Phonetics}
\end{Entry}

\begin{Entry}{分红}{4,6}{⼑、⽷}
  \begin{Phonetics}{分红}{fen1/hong2}[][HSK 7-9]
    \definition{v.+compl.}{receber dividendos; distribuir bônus; obter dividendos extras (lucros)}
  \end{Phonetics}
\end{Entry}

\begin{Entry}{分别}{4,7}{⼑、⼑}
  \begin{Phonetics}{分别}{fen1bie2}[][HSK 3]
    \definition{adv.}{diferentemente; de maneiras diferentes; expressar de maneiras diferentes | separadamente; individualmente; respectivamente}
    \definition{s.}{diferença; pontos diferentes}
    \definition{v.}{partir; deixar um ao outro; não estar mais junto | distinguir; diferenciar}
  \end{Phonetics}
\end{Entry}

\begin{Entry}{分享}{4,8}{⼑、⼇}
  \begin{Phonetics}{分享}{fen1 xiang3}[][HSK 5]
    \definition{v.}{compartilhar; partilhar}
  \end{Phonetics}
\end{Entry}

\begin{Entry}{分担}{4,8}{⼑、⼿}
  \begin{Phonetics}{分担}{fen1dan1}[][HSK 7-9]
    \definition{v.}{contribuir; compartilhar a responsabilidade por; participar}
  \end{Phonetics}
\end{Entry}

\begin{Entry}{分明}{4,8}{⼑、⽇}
  \begin{Phonetics}{分明}{fen1ming2}[][HSK 7-9]
    \definition{adj.}{claro; distinto; óbvio}
    \definition{adv.}{claramente; evidentemente; obviamente; os fatos são claros, óbvios e inquestionáveis}
  \end{Phonetics}
\end{Entry}

\begin{Entry}{分析}{4,8}{⼑、⽊}
  \begin{Phonetics}{分析}{fen1xi1}[][HSK 5]
    \definition{v.}{analisar; dividir uma coisa, um fenômeno, um conceito em componentes mais simples e descobrir as propriedades essenciais desses componentes e a relação entre eles (em oposição à 综合)}
  \seealsoref{综合}{zong1he2}
  \end{Phonetics}
\end{Entry}

\begin{Entry}{分歧}{4,8}{⼑、⽌}
  \begin{Phonetics}{分歧}{fen1qi2}[][HSK 7-9]
    \definition[点,个]{s.}{diferença; divergência; diferenças de pontos de vista, opiniões, etc.}
  \end{Phonetics}
\end{Entry}

\begin{Entry}{分泌}{4,8}{⼑、⽔}
  \begin{Phonetics}{分泌}{fen1mi4}[][HSK 7-9]
    \definition{v.}{secretar; excretar}
  \end{Phonetics}
\end{Entry}

\begin{Entry}{分组}{4,8}{⼑、⽷}
  \begin{Phonetics}{分组}{fen1 zu3}[][HSK 3]
    \definition{v.}{dividir em grupos}
  \end{Phonetics}
\end{Entry}

\begin{Entry}{分类}{4,9}{⼑、⽶}
  \begin{Phonetics}{分类}{fen1/lei4}[][HSK 5]
    \definition{v.+compl.}{ordenar; classificar; categorizar; classificar as coisas de acordo com sua natureza e características}
  \end{Phonetics}
\end{Entry}

\begin{Entry}{分钟}{4,9}{⼑、⾦}
  \begin{Phonetics}{分钟}{fen1zhong1}[][HSK 2]
    \definition{clas.}{minuto (usado em intervalos de tempo); 60 segundos}
  \end{Phonetics}
\end{Entry}

\begin{Entry}{分离}{4,10}{⼑、⼇}
  \begin{Phonetics}{分离}{fen1 li2}[][HSK 5]
    \definition{v.}{cortar; separar (de coisas) | separar; sair; separar (de pessoas); partir (em uma longa viagem)}
  \end{Phonetics}
\end{Entry}

\begin{Entry}{分赃}{4,10}{⼑、⾙}
  \begin{Phonetics}{分赃}{fen1/zang1}[][HSK 7-9]
    \definition{v.+compl.}{dividir ganhos ilícitos; compartilhar o saque}
  \end{Phonetics}
\end{Entry}

\begin{Entry}{分配}{4,10}{⼑、⾣}
  \begin{Phonetics}{分配}{fen1pei4}[][HSK 3]
    \definition{v.}{atribuir; dispor; organizar o trabalho, as tarefas, os recursos, o tempo, etc. | atribuir; compartilhar; distribuir dinheiro ou bens às pessoas envolvidas de acordo com um determinado plano, padrão ou regulamento}
  \end{Phonetics}
\end{Entry}

\begin{Entry}{分割}{4,12}{⼑、⼑}
  \begin{Phonetics}{分割}{fen1ge1}[][HSK 7-9]
    \definition{v.}{cortar; separar; quebrar; separar uma coisa inteira ou conectada}
  \end{Phonetics}
\end{Entry}

\begin{Entry}{分散}{4,12}{⼑、⽁}
  \begin{Phonetics}{分散}{fen1san4}[][HSK 4]
    \definition{adj.}{espalhado; disperso; desviado; fragmentado; sem foco}
    \definition{v.}{dispersar; espalhar; descentralizar | separar-se; desunir-se}
  \end{Phonetics}
\end{Entry}

\begin{Entry}{分裂}{4,12}{⼑、⾐}
  \begin{Phonetics}{分裂}{fen1lie4}[][HSK 6]
    \definition{s.}{fissão; divisão}
    \definition{v.}{dividir; separar; romper}
  \end{Phonetics}
\end{Entry}

\begin{Entry}{分量}{4,12}{⼑、⾥}
  \begin{Phonetics}{分量}{fen1liang4}
    \definition{s.}{componente vetorial}
  \end{Phonetics}
  \begin{Phonetics}{分量}{fen4liang4}[][HSK 7-9]
    \definition{s.}{tamanho da porção (comida)}[这道菜的分量太少了。===A porção deste prato era muito pequena.]
  \end{Phonetics}
  \begin{Phonetics}{分量}{fen4liang5}
    \definition{s.}{quantidade; peso; medida | Figurativo: peso (importância, prestígio, autoridade etc.); (de material escrito) densidade}
  \end{Phonetics}
\end{Entry}

\begin{Entry}{分数}{4,13}{⼑、⽁}
  \begin{Phonetics}{分数}{fen1 shu4}[][HSK 2]
    \definition[个]{s.}{fração; número fracionário | nota; classificação; ponto; pontuação registrada ao avaliar o resultado ou a vitória/derrota}
  \end{Phonetics}
\end{Entry}

\begin{Entry}{分解}{4,13}{⼑、⾓}
  \begin{Phonetics}{分解}{fen1jie3}[][HSK 5]
    \definition{v.}{quebrar; separar em partes; dividir um todo em seus componentes | resolver; decompor | Química: transformar uma substância em duas ou mais substâncias por meio de uma reação química | desintegrar-se; dividir-se; desunir uma organização | mediar; fazer a paz; resolver conflitos e disputas | explicar; defender-se}
  \end{Phonetics}
\end{Entry}

\begin{Entry}{分辨}{4,16}{⼑、⾟}
  \begin{Phonetics}{分辨}{fen1bian4}[][HSK 7-9]
    \definition{s.}{resolução; descreve o grau em que imagens, exibições, sensores, etc. podem distinguir claramente os detalhes}
    \definition{v.}{distinguir; diferenciar}
  \end{Phonetics}
\end{Entry}

\begin{Entry}{切}{4}{⼑}
  \begin{Phonetics}{切}{qie1}[][HSK 4]
    \definition{v.}{cortar; fatiar; separar itens com uma faca | cortar ou romper; truncar | Geometria: refere-se a quando uma linha, círculo ou superfície intercepta um círculo, arco ou esfera em apenas um ponto}
  \end{Phonetics}
  \begin{Phonetics}{切}{qie4}
    \definition{adj.}{ansioso; sério | duro; severo; rude; áspero}
    \definition{adv.}{com certeza; certamente}
    \definition{s.}{limiar; degrau}
    \definition{v.}{ser prático ou realista | ajustar-se ou corresponder | ser próximo ou íntimo | cortar algo em pedaços com uma faca | tomar o pulso (medicina tradicional chinesa)}
  \end{Phonetics}
\end{Entry}

\begin{Entry}{切实}{4,8}{⼑、⼧}
  \begin{Phonetics}{切实}{qie4shi2}[][HSK 6]
    \definition{adj.}{prático; viável; realista}
  \end{Phonetics}
\end{Entry}

\begin{Entry}{切割}{4,12}{⼑、⼑}
  \begin{Phonetics}{切割}{qie1ge1}
    \definition{v.}{esculpir; cortar algo com uma faca | cortar algo com máquina, fogo, arco voltaico; refere-se especificamente ao corte de materiais metálicos com máquinas-ferramentas ou à queima deles com arco elétrico, laser, etc.}
  \end{Phonetics}
\end{Entry}

\begin{Entry}{劝}{4}{⼒}
  \begin{Phonetics}{劝}{quan4}[][HSK 5]
    \definition*{s.}{Sobrenome Quan}
    \definition{v.}{insistir; aconselhar; tentar persuadir; persuadir, argumentar para que as pessoas obedeçam | incentivar; encorajar}
  \end{Phonetics}
\end{Entry}

\begin{Entry}{办}{4}{⼒}
  \begin{Phonetics}{办}{ban4}[][HSK 2]
    \definition{v.}{fazer; lidar com; gerenciar; cuidar de | executar; configurar | preparar algo; comprar uma quantidade razoável de | punir; levar à justiça; punir com medidas}
  \end{Phonetics}
\end{Entry}

\begin{Entry}{办不到}{4,4,8}{⼒、⼀、⼑}
  \begin{Phonetics}{办不到}{ban4 bu5 dao4}[][HSK 7-9]
    \definition{v.}{ser incapaz de fazer algo; ser incapaz de realizar; não poder ser feito; não poder fazer}
  \end{Phonetics}
\end{Entry}

\begin{Entry}{办公}{4,4}{⼒、⼋}
  \begin{Phonetics}{办公}{ban4 gong1}[][HSK 6]
    \definition{v.}{trabalhar; fazer trabalho de escritório; lidar com negócios oficiais; tratar de assuntos oficiais}
  \end{Phonetics}
\end{Entry}

\begin{Entry}{办公室}{4,4,9}{⼒、⼋、⼧}
  \begin{Phonetics}{办公室}{ban4gong1shi4}[][HSK 2]
    \definition[个,间]{s.}{órgãos, escolas, grupos, empresas e outras entidades que lidam com assuntos administrativos cotidianos | escritório; sala de escritório}
  \end{Phonetics}
\end{Entry}

\begin{Entry}{办事}{4,8}{⼒、⼅}
  \begin{Phonetics}{办事}{ban4 shi4}[][HSK 4]
    \definition{v.}{trabalhar | lidar com assuntos; manipular transações}
  \end{Phonetics}
\end{Entry}

\begin{Entry}{办事处}{4,8,5}{⼒、⼅、⼡}
  \begin{Phonetics}{办事处}{ban4 shi4 chu4}[][HSK 6]
    \definition[个,家]{s.}{escritório; agência; agências enviadas pelo governo, militares, grupos, etc.}
  \end{Phonetics}
\end{Entry}

\begin{Entry}{办学}{4,8}{⼒、⼦}
  \begin{Phonetics}{办学}{ban4/xue2}[][HSK 6]
    \definition{v.+compl.}{administrar uma escola}
  \end{Phonetics}
\end{Entry}

\begin{Entry}{办法}{4,8}{⼒、⽔}
  \begin{Phonetics}{办法}{ban4fa3}[][HSK 2]
    \definition[个,种]{s.}{método; meio; medida; caminho; maneira; método de lidar com situações ou resolver problemas}
  \end{Phonetics}
\end{Entry}

\begin{Entry}{办理}{4,11}{⼒、⽟}
  \begin{Phonetics}{办理}{ban4li3}[][HSK 3]
    \definition{v.}{conduzir; lidar; transacionar; negociar; solicitar um documento ou realizar um procedimento específico}
  \end{Phonetics}
\end{Entry}

\begin{Entry}{勾}{4}{⼓}
  \begin{Phonetics}{勾}{gou1}[][HSK 7-9]
    \definition*{s.}{Sobrenome Gou}
    \definition{s.}{nos tempos antigos, referia-se ao lado mais curto de um triângulo retângulo isósceles}
    \definition{v.}{cancelar; riscar; marcar | delinear; desenhar | preencher as juntas da alvenaria com argamassa ou cimento; apontar | adicionar algo para engrossar; engrossar | induzir; evocar; trazer à mente | conspirar com; unir-se a | excluir; apagar | seduzir; atrair}
    \variantof{钩}
  \end{Phonetics}
  \begin{Phonetics}{勾}{gou4}
    \definition*{s.}{Sobrenome Gou}
    \definition{s.}{usado em 勾当}
  \seealsoref{勾当}{gou4dang4}
  \end{Phonetics}
\end{Entry}

\begin{Entry}{勾当}{4,6}{⼓、⼹}
  \begin{Phonetics}{勾当}{gou4dang4}
    \definition{s.}{(depreciativo, obscuro) negócio; acordo; esquema; atividade}
  \end{Phonetics}
\end{Entry}

\begin{Entry}{勾画}{4,8}{⼓、⽥}
  \begin{Phonetics}{勾画}{gou1hua4}[][HSK 7-9]
    \definition{v.}{esboçar; delinear; desenhar o contorno de; descrever em palavras curtas}
  \end{Phonetics}
\end{Entry}

\begin{Entry}{勾结}{4,9}{⼓、⽷}
  \begin{Phonetics}{勾结}{gou1jie2}[][HSK 7-9]
    \definition{v.}{conspirar com; estar em aliança com; ser de mão e luva com; colaborar com; unir-se a; conspirar secretamente e combinar-se entre si para realizar atividades impróprias}
  \end{Phonetics}
\end{Entry}

\begin{Entry}{化}{4}{⼔}
  \begin{Phonetics}{化}{hua1}
    \variantof{花}
  \end{Phonetics}
  \begin{Phonetics}{化}{hua4}[][HSK 3]
    \definition*{s.}{Sobrenome Hua}
    \definition{s.}{química | cultura; costumes e tradições}
    \definition{suf.}{modernizar; modernização; anexado a componentes nominais ou adjetivos para formar verbos, indicando a transformação em algum estado ou qualidade}
    \definition{v.}{mudar; converter; transformar; causasr mudanças | converter; influenciar; influenciar e induzir as pessoas com palavras e ações, levando-as a mudar | derreter; dissolver; fundir | digerir | queimar; reduzir a cinzas | (monge, taoísta) morrer | (de monges budistas ou sacerdotes taoístas) pedir esmolas; arrecadar bens, alimentos, etc.}
  \end{Phonetics}
\end{Entry}

\begin{Entry}{化石}{4,5}{⼔、⽯}
  \begin{Phonetics}{化石}{hua4shi2}[][HSK 5]
    \definition{s.}{fóssil; restos, relíquias ou vestígios de organismos antigos enterrados no solo e transformados em objetos semelhantes a pedras}
  \end{Phonetics}
\end{Entry}

\begin{Entry}{化合}{4,6}{⼔、⼝}
  \begin{Phonetics}{化合}{hua4he2}
    \definition{s.}{combinação química}
    \definition{s.}{Química: combinar; sintentizar}
  \end{Phonetics}
\end{Entry}

\begin{Entry}{化妆}{4,6}{⼔、⼥}
  \begin{Phonetics}{化妆}{hua4/zhuang1}[][HSK 7-9]
    \definition{v.+compl.}{maquiar-se; colocar maquiagem; usar algo para deixar o rosto mais bonito}
  \end{Phonetics}
\end{Entry}

\begin{Entry}{化纤}{4,6}{⼔、⽷}
  \begin{Phonetics}{化纤}{hua4xian1}[][HSK 7-9]
    \definition[吨]{s.}{fibra sintética; abreviação de 化学纤维}[这件衣服是化纤材质。===Este vestido é feito de fibra sintética.]
  \seealsoref{化学纤维}{hua4 xue2 xian1 wei2}
  \end{Phonetics}
\end{Entry}

\begin{Entry}{化身}{4,7}{⼔、⾝}
  \begin{Phonetics}{化身}{hua4shen1}[][HSK 7-9]
    \definition{s.}{encarnação; corporificação; personificação}
  \end{Phonetics}
\end{Entry}

\begin{Entry}{化学}{4,8}{⼔、⼦}
  \begin{Phonetics}{化学}{hua4xue2}
    \definition[门]{s.}{química; a ciência que estuda a composição, estrutura, propriedades e leis de mudança da matéria | celuloide}
  \end{Phonetics}
\end{Entry}

\begin{Entry}{化学纤维}{4,8,6,11}{⼔、⼦、⽷、⽷}
  \begin{Phonetics}{化学纤维}{hua4 xue2 xian1 wei2}
    \definition{s.}{fibra química | fibra sintética}
  \end{Phonetics}
\end{Entry}

\begin{Entry}{化肥}{4,8}{⼔、⾁}
  \begin{Phonetics}{化肥}{hua4fei2}[][HSK 7-9]
    \definition[袋,吨,种]{s.}{fertilizante químico}
  \end{Phonetics}
\end{Entry}

\begin{Entry}{化险为夷}{4,9,4,6}{⼔、⾩、⼂、⼤}
  \begin{Phonetics}{化险为夷}{hua4xian3wei2yi2}[][HSK 7-9]
    \definition{expr.}{``Evite o perigo.''; transformar perigo em segurança; evitar um desastre}
  \end{Phonetics}
\end{Entry}

\begin{Entry}{化验}{4,10}{⼔、⾺}
  \begin{Phonetics}{化验}{hua4yan4}[][HSK 7-9]
    \definition{v.}{fazer um teste de laboratório; conduzir um exame químico; usar métodos físicos ou químicos para examinar a composição e as propriedades das substâncias}
  \end{Phonetics}
\end{Entry}

\begin{Entry}{化解}{4,13}{⼔、⾓}
  \begin{Phonetics}{化解}{hua4 jie3}[][HSK 6]
    \definition{v.}{resolver; eliminar; dissolver; neutralizar}
  \end{Phonetics}
\end{Entry}

\begin{Entry}{匹}{4}{⼖}
  \begin{Phonetics}{匹}{pi3}[][HSK 5]
    \definition{adj.}{solitário}
    \definition{clas.}{usado para cavalos, mulas, etc. | usado para rolos inteiros de seda ou tecido}
    \definition{v.}{ser igual a; ser compatível com}
  \end{Phonetics}
\end{Entry}

\begin{Entry}{区}{4}{⼖}
  \begin{Phonetics}{区}{ou1}
    \definition*{s.}{Sobrenome Ou}
  \end{Phonetics}
  \begin{Phonetics}{区}{qu1}[][HSK 3]
    \definition{s.}{área; distrito; região; zona; uma determinada área em terra, água ou ar | uma divisão administrativa; as divisões administrativas incluem regiões autônomas étnicas de nível provincial, distritos municipais e de condado e distritos de condado; grandes regiões administrativas, regiões, zonas especiais e regiões administrativas especiais}
    \definition{v.}{classificar; subdividir; distinguir}
  \end{Phonetics}
\end{Entry}

\begin{Entry}{区分}{4,4}{⼖、⼑}
  \begin{Phonetics}{区分}{qu1fen1}[][HSK 6]
    \definition{v.}{discriminar; diferenciar; distinguir; comparar dois ou mais objetos; reconhecer suas diferenças}
  \end{Phonetics}
\end{Entry}

\begin{Entry}{区别}{4,7}{⼖、⼑}
  \begin{Phonetics}{区别}{qu1bie2}[][HSK 3]
    \definition[种,个]{s.}{diferença; distinção; discriminação}
    \definition{v.}{distinguir; diferenciar; fazer distinção entre}
  \end{Phonetics}
\end{Entry}

\begin{Entry}{区域}{4,11}{⼖、⼟}
  \begin{Phonetics}{区域}{qu1yu4}[][HSK 5]
    \definition[片,块,个]{s.}{área; setor; região; faixa; inclui áreas regionais com condições naturais, culturais, administrativas, etc.}
  \end{Phonetics}
\end{Entry}

\begin{Entry}{升}{4}{⼗}
  \begin{Phonetics}{升}{sheng1}[][HSK 3]
    \definition*{s.}{Sobrenome Sheng}
    \definition{clas.}{litro (l)}
    \definition{s.}{sheng, uma unidade de medida seca para grãos (= 1 litro), um décimo de 斗}
    \definition{v.}{elevar; içar; subir; ascender; subir ou subir mais alto (oposto de 降) | promover; melhorar (nível)}
  \seealsoref{斗}{dou4}
  \seealsoref{降}{jiang4}
  \end{Phonetics}
\end{Entry}

\begin{Entry}{升级}{4,6}{⼗、⽷}
  \begin{Phonetics}{升级}{sheng1/ji2}[][HSK 6]
    \definition{v.+compl.}{atualizar (software) | (guerra) escalar; (tensão) aprofundar | subir um ou mais níveis; passar de uma série ou classe inferior para uma série ou classe superior}
  \end{Phonetics}
\end{Entry}

\begin{Entry}{升学}{4,8}{⼗、⼦}
  \begin{Phonetics}{升学}{sheng1 xue2}[][HSK 6]
    \definition{v.}{ir para uma universidade, faculdade; entrar em uma universidade, faculdade}
  \end{Phonetics}
\end{Entry}

\begin{Entry}{升值}{4,10}{⼗、⼈}
  \begin{Phonetics}{升值}{sheng1 zhi2}[][HSK 6]
    \definition{v.}{Economia: reavaliar; apreciar | Figurativo: aumento de valor | valorização; apreciação; aumentar o valor; aumentar os preços}
  \end{Phonetics}
\end{Entry}

\begin{Entry}{升起}{4,10}{⼗、⾛}
  \begin{Phonetics}{升起}{sheng1qi3}
    \definition{v.}{levantar | içar | subir}
  \end{Phonetics}
\end{Entry}

\begin{Entry}{升高}{4,10}{⼗、⾼}
  \begin{Phonetics}{升高}{sheng1 gao1}[][HSK 5]
    \definition{v.}{subir; ascender | promover; elevar; intensificar; potencializar; melhorar}
  \end{Phonetics}
\end{Entry}

\begin{Entry}{午}{4}{⼗}
  \begin{Phonetics}{午}{wu3}
    \definition{s.}{meio-dia; período entre 11h00 e 13h00 | wu (sétimo dos doze Ramos Terrestres)}
  \end{Phonetics}
\end{Entry}

\begin{Entry}{午休}{4,6}{⼗、⼈}
  \begin{Phonetics}{午休}{wu3xiu1}
    \definition{s.}{pausa para almoço | cochilo na hora do almoço | intervalo do meio-dia}
  \end{Phonetics}
\end{Entry}

\begin{Entry}{午后}{4,6}{⼗、⼝}
  \begin{Phonetics}{午后}{wu3hou4}
    \definition{s.}{tarde | período da tarde}
  \end{Phonetics}
\end{Entry}

\begin{Entry}{午饭}{4,7}{⼗、⾷}
  \begin{Phonetics}{午饭}{wu3 fan4}[][HSK 1]
    \definition[顿]{s.}{almoço}
  \seealsoref{午餐}{wu3 can1}
  \end{Phonetics}
\end{Entry}

\begin{Entry}{午夜}{4,8}{⼗、⼣}
  \begin{Phonetics}{午夜}{wu3ye4}
    \definition{s.}{meia-noite}
  \end{Phonetics}
\end{Entry}

\begin{Entry}{午前}{4,9}{⼗、⼑}
  \begin{Phonetics}{午前}{wu3qian2}
    \definition{s.}{\emph{A.M.} | manhã | período da manhã}
  \end{Phonetics}
\end{Entry}

\begin{Entry}{午宴}{4,10}{⼗、⼧}
  \begin{Phonetics}{午宴}{wu3yan4}
    \definition{s.}{banquete de almoço}
  \end{Phonetics}
\end{Entry}

\begin{Entry}{午睡}{4,13}{⼗、⽬}
  \begin{Phonetics}{午睡}{wu3 shui4}[][HSK 2]
    \definition{s.}{\emph{siesta}; cochilo da tarde; soneca do meio-dia}
    \definition{v.}{tirar uma soneca depois do almoço}
  \end{Phonetics}
\end{Entry}

\begin{Entry}{午餐}{4,16}{⼗、⾷}
  \begin{Phonetics}{午餐}{wu3 can1}[][HSK 2]
    \definition[份,顿,次]{s.}{almoço}
  \seealsoref{午饭}{wu3 fan4}
  \end{Phonetics}
\end{Entry}

\begin{Entry}{厄}{4}{⼚}
  \begin{Phonetics}{厄}{e4}
    \definition[个]{s.}{ponto estratégico; lugares perigosos | adversidade; desastre; dificuldade}
    \definition{v.}{estar em perigo; estar abandonado; estar; encurralado}
  \end{Phonetics}
\end{Entry}

\begin{Entry}{厄运}{4,7}{⼚、⾡}
  \begin{Phonetics}{厄运}{e4yun4}[][HSK 7-9]
    \definition{s.}{adversidade; infortúnio; experiência infeliz}
  \end{Phonetics}
\end{Entry}

\begin{Entry}{厅}{4}{⼚}
  \begin{Phonetics}{厅}{ting1}[][HSK 5]
    \definition{s.}{salão; sala grande para reuniões ou receber convidados | escritório; nome de um departamento administrativo de uma grande organização | departamento governamental a nível provincial; nomes de alguns órgãos estaduais}
  \end{Phonetics}
\end{Entry}

\begin{Entry}{历}{4}{⼚}
  \begin{Phonetics}{历}{li4}
    \definition{adj.}{todas as anteriores (ocasiões, sessões, etc.)}
    \definition{adv.}{por toda parte; um por um}
    \definition{s.}{experiência; registro | almanaque; anuário; calendário}
    \definition{v.}{passar por; sofrer; experimentar | passar através; atravessar}
  \end{Phonetics}
\end{Entry}

\begin{Entry}{历史}{4,5}{⼚、⼝}
  \begin{Phonetics}{历史}{li4shi3}[][HSK 4]
    \definition[段]{s.}{história; registro do passado; processo de desenvolvimento da natureza e da sociedade humana; processo de desenvolvimento de uma coisa ou pessoa | história; eventos passados; experiência | história; refere-se ao tema da história}
  \end{Phonetics}
\end{Entry}

\begin{Entry}{友}{4}{⼜}
  \begin{Phonetics}{友}{you3}
    \definition{adj.}{amigável; bom relacionamento; próximo | de relações amigáveis}
    \definition{s.}{amigo; pessoas intimamente relacionadas}
  \end{Phonetics}
\end{Entry}

\begin{Entry}{友好}{4,6}{⼜、⼥}
  \begin{Phonetics}{友好}{you3hao3}[][HSK 2]
    \definition{adj.}{amigável; amistoso; muito próximo, relacionamento muito bom; como bons amigos}
    \definition{s.}{amigo próximo, íntimo; em ocasiões formais referem-se a bons amigos}
  \end{Phonetics}
\end{Entry}

\begin{Entry}{友谊}{4,10}{⼜、⾔}
  \begin{Phonetics}{友谊}{you3yi4}[][HSK 5]
    \definition[段,份]{s.}{amizade; amizade entre amigos}
  \end{Phonetics}
\end{Entry}

\begin{Entry}{双}{4}{⼜}
  \begin{Phonetics}{双}{shuang1}[][HSK 3]
    \definition*{s.}{Sobrenome Shuang}
    \definition{adj.}{dois; gêmeo; par; dual; em oposição a 单 | números pares | duplo; dobro}
    \definition{clas.}{usado para certos membros, órgãos ou coisas pareadas que são bilateralmente simétricas, por exemplo, sapatos, meias, pauzinhos, etc.}
  \seealsoref{单}{dan1}
  \end{Phonetics}
\end{Entry}

\begin{Entry}{双手}{4,4}{⼜、⼿}
  \begin{Phonetics}{双手}{shuang1 shou3}[][HSK 5]
    \definition{s.}{com as duas mãos; ambas as mãos; par de mãos}
  \end{Phonetics}
\end{Entry}

\begin{Entry}{双方}{4,4}{⼜、⽅}
  \begin{Phonetics}{双方}{shuang1fang1}[][HSK 3]
    \definition{s.}{ambos os lados; as duas partes; duas pessoas ou dois grupos frente a frente em um determinado relacionamento ou situação}
  \end{Phonetics}
\end{Entry}

\begin{Entry}{双方同意}{4,4,6,13}{⼜、⽅、⼝、⼼}
  \begin{Phonetics}{双方同意}{shuang1fang1tong2yi4}
    \definition{s.}{acordo bilateral}
  \end{Phonetics}
\end{Entry}

\begin{Entry}{双打}{4,5}{⼜、⼿}
  \begin{Phonetics}{双打}{shuang1 da3}[][HSK 6]
    \definition[场,局,次]{s.}{duplas (em esportes)}
  \end{Phonetics}
\end{Entry}

\begin{Entry}{双层床}{4,7,7}{⼜、⼫、⼴}
  \begin{Phonetics}{双层床}{shuang1ceng2chuang2}
    \definition{s.}{beliche}
  \end{Phonetics}
\end{Entry}

\begin{Entry}{反}{4}{⼜}
  \begin{Phonetics}{反}{fan3}[][HSK 4]
    \definition{adj.}{oposto; contrário; invertido}
    \definition{adv.}{pelo contrário; inversamente}
    \definition{v.}{inverter o lado; de cabeça para baixo; na direção oposta | virar; converter | retornar | opor-se; combater; voltar-se contra | rebelar-se; revoltar-se | inferir; deduzir; raciocinar por analogia}
  \end{Phonetics}
\end{Entry}

\begin{Entry}{反击}{4,5}{⼜、⼐}
  \begin{Phonetics}{反击}{fan3ji1}[][HSK 7-9]
    \definition{v.}{revidar; responder fogo; contra-atacar}
  \end{Phonetics}
\end{Entry}

\begin{Entry}{反对}{4,5}{⼜、⼨}
  \begin{Phonetics}{反对}{fan3dui4}[][HSK 3]
    \definition{v.}{lutar; opor-se; objetar a; ser contra; discordar}
  \end{Phonetics}
\end{Entry}

\begin{Entry}{反对派}{4,5,9}{⼜、⼨、⽔}
  \begin{Phonetics}{反对派}{fan3dui4pai4}
    \definition{s.}{facção de oposição}
  \end{Phonetics}
\end{Entry}

\begin{Entry}{反对党}{4,5,10}{⼜、⼨、⼉}
  \begin{Phonetics}{反对党}{fan3dui4dang3}
    \definition{s.}{partido de oposição}
  \end{Phonetics}
\end{Entry}

\begin{Entry}{反对票}{4,5,11}{⼜、⼨、⽰}
  \begin{Phonetics}{反对票}{fan3dui4piao4}
    \definition{s.}{voto dissidente}
  \end{Phonetics}
\end{Entry}

\begin{Entry}{反正}{4,5}{⼜、⽌}
  \begin{Phonetics}{反正}{fan3zheng4}[][HSK 3]
    \definition{adv.}{de qualquer forma; de qualquer maneira; embora as circunstâncias sejam diferentes, o resultado é o mesmo | tudo igual; em qualquer caso; tom de voz que expressa afirmação categórica}
  \end{Phonetics}
\end{Entry}

\begin{Entry}{反而}{4,6}{⼜、⽽}
  \begin{Phonetics}{反而}{fan3'er2}[][HSK 4]
    \definition{adv.}{em vez disso; ao contrário de; contrário ao significado da frase anterior ou inesperado, desempenha o papel de uma reviravolta em uma frase}
  \end{Phonetics}
\end{Entry}

\begin{Entry}{反过来}{4,6,7}{⼜、⾡、⽊}
  \begin{Phonetics}{反过来}{fan3 guo4lai2}[][HSK 7-9]
    \definition{adv.}{inversamente; na ordem inversa; em direção oposta; indica uma reversão ou mudança de uma ação ou estado}
    \definition{conj.}{vice-versa; inversamente; na ordem inversa; ao contrário; usado para orientar relacionamentos de transição, como condições e causas}
    \definition{v.}{virar; virar para frente e para trás}
  \end{Phonetics}
\end{Entry}

\begin{Entry}{反问}{4,6}{⼜、⾨}
  \begin{Phonetics}{反问}{fan3wen4}[][HSK 6]
    \definition{v.}{fazer uma pergunta em resposta; responder a uma pergunta com outra pergunta | fazer uma pergunta retórica (uma pergunta com significado negativo)}
  \end{Phonetics}
\end{Entry}

\begin{Entry}{反应}{4,7}{⼜、⼴}
  \begin{Phonetics}{反应}{fan3ying4}[][HSK 3]
    \definition[个]{s.}{reação; resposta; opiniões, atitudes ou ações causadas pelo acontecimento}
    \definition{v.}{reagir; responder; atividade correspondente causada pela estimulação do organismo}
  \end{Phonetics}
\end{Entry}

\begin{Entry}{反抗}{4,7}{⼜、⼿}
  \begin{Phonetics}{反抗}{fan3kang4}[][HSK 6]
    \definition{s.}{resistência}
    \definition{v.}{revoltar-se; resistir; opor-se com ação}
  \end{Phonetics}
\end{Entry}

\begin{Entry}{反驳}{4,7}{⼜、⾺}
  \begin{Phonetics}{反驳}{fan3bo2}[][HSK 7-9]
    \definition{v.}{refutar; replicar; apresentar suas próprias razões para refutar teorias ou opiniões de outras pessoas que diferem das suas}
  \end{Phonetics}
\end{Entry}

\begin{Entry}{反响}{4,9}{⼜、⼝}
  \begin{Phonetics}{反响}{fan3 xiang3}[][HSK 6]
    \definition{s.}{eco; reverberação; repercusão}
  \end{Phonetics}
\end{Entry}

\begin{Entry}{反复}{4,9}{⼜、⼢}
  \begin{Phonetics}{反复}{fan3fu4}[][HSK 3]
    \definition{adv.}{repetidamente; de ​​novo e de novo; várias vezes}
    \definition{s.}{reversão; recaída; a situação anterior se repetiu}
    \definition{v.}{recuar; cortar e mudar; virar de cabeça para baixo; arrepender-se; aparecer várias vezes (usado principalmente em situações ruins)}
  \end{Phonetics}
\end{Entry}

\begin{Entry}{反差}{4,9}{⼜、⼯}
  \begin{Phonetics}{反差}{fan3cha1}[][HSK 7-9]
    \definition{s.}{contraste; o contraste de cores da foto ou do cenário é muito diferente, como o contraste entre o preto e o branco | discrepância; o contraste entre o bem e o mal, o alto e o baixo, etc. de pessoas ou coisas}
  \end{Phonetics}
\end{Entry}

\begin{Entry}{反思}{4,9}{⼜、⼼}
  \begin{Phonetics}{反思}{fan3si1}[][HSK 7-9]
    \definition{v.}{refletir; introspectar; refletir sobre o passado e tirar lições dele}
  \end{Phonetics}
\end{Entry}

\begin{Entry}{反映}{4,9}{⼜、⽇}
  \begin{Phonetics}{反映}{fan3ying4}[][HSK 4]
    \definition{s.}{reflexão; opiniões sobre pessoas ou situações}
    \definition{v.}{refletir; espelhar; figurativamente, trazer à tona a essência de uma questão objetiva (expressão idiomática); expressar a essência de algo objetivamente | relatar; tornar conhecido; informar às autoridades superiores | refletir; espelhar; a imagem de um objeto aparece invertida em outro objeto}
  \end{Phonetics}
\end{Entry}

\begin{Entry}{反省}{4,9}{⼜、⽬}
  \begin{Phonetics}{反省}{fan3xing3}[][HSK 7-9]
    \definition{v.}{refletir sobre si mesmo; envolver-se em introspecção e autoexame; refletir sobre seus pensamentos e ações e examinar quaisquer erros; examinar a consciência; questionar-se; sondar a alma}
  \end{Phonetics}
\end{Entry}

\begin{Entry}{反面}{4,9}{⼜、⾯}
  \begin{Phonetics}{反面}{fan3mian4}[][HSK 7-9]
    \definition{adj.}{oposto; negativo; ruim}
    \definition{s.}{costas; lado reverso; lado avesso; o lado de um objeto oposto à frente | o reverso de um estado de coisas, um problema, etc.; o outro lado de uma questão, problema, etc.}
  \end{Phonetics}
\end{Entry}

\begin{Entry}{反倒}{4,10}{⼜、⼈}
  \begin{Phonetics}{反倒}{fan3dao4}[][HSK 7-9]
    \definition{adv.}{em vez disso; pelo contrário}
    \definition{conj.}{em vez disso; pelo contrário; frequentemente acompanhadas por várias palavras que expressam negação}
  \end{Phonetics}
\end{Entry}

\begin{Entry}{反常}{4,11}{⼜、⼱}
  \begin{Phonetics}{反常}{fan3chang2}[][HSK 7-9]
    \definition{adj.}{incomum; anormal; diferente da situação normal}
  \end{Phonetics}
\end{Entry}

\begin{Entry}{反弹}{4,11}{⼜、⼸}
  \begin{Phonetics}{反弹}{fan3tan2}[][HSK 7-9]
    \definition{s.}{rebote}
    \definition{v.}{recuperar; um objeto elástico retorna à sua forma original após ser deformado por uma força externa | rebater; saltar de volta; ressurgir; metáfora para recuperação de preço ou mercado | rebotar; quicar; ricochetear; um objeto em movimento salta na direção oposta quando encontra um obstáculo}
  \end{Phonetics}
\end{Entry}

\begin{Entry}{反馈}{4,12}{⼜、⾶}
  \begin{Phonetics}{反馈}{fan3kui4}[][HSK 7-9]
    \definition{s.}{\emph{feedback}; resposta; uma resposta ou reação a algo, informação, etc.}
    \definition{v.}{dar \emph{feedback}; enviar informações de volta; (informações, \emph{feedback}, etc.) retornar ao local de onde foi enviado}
  \end{Phonetics}
\end{Entry}

\begin{Entry}{反感}{4,13}{⼜、⼼}
  \begin{Phonetics}{反感}{fan3gan3}[][HSK 7-9]
    \definition{adj.}{avesso; enojado; desgostoso; insatisfeito}
    \definition{s.}{antipatia; aversão; oposição ou insatisfação}
  \end{Phonetics}
\end{Entry}

\begin{Entry}{天}{4}{⼤}
  \begin{Phonetics}{天}{tian1}[][HSK 1]
    \definition*{s.}{Sobrenome Tian}
    \definition{adj.}{localizado no topo; suspenso no ar | inato; natural}
    \definition{clas.}{usado para contar dias}
    \definition{s.}{céu; paraíso; espaço onde se encontram o sol, a lua e as estrelas | dia; as 24 horas do dia, às vezes referindo-se especificamente ao período diurno | um período de tempo em um dia; em algum momento do dia | temporada; estação do ano | clima | natureza | Deus; céu; o criador | paraíso; refere-se ao local onde residem os deuses, budas e imortais}
  \end{Phonetics}
\end{Entry}

\begin{Entry}{天上}{4,3}{⼤、⼀}
  \begin{Phonetics}{天上}{tian1 shang4}[][HSK 2]
    \definition[期]{s.}{o céu; o paraíso}
  \end{Phonetics}
\end{Entry}

\begin{Entry}{天下}{4,3}{⼤、⼀}
  \begin{Phonetics}{天下}{tian1 xia4}[][HSK 6]
    \definition[期]{s.}{China ou o mundo; refere-se à China ou ao mundo | dominação; o poder dominante de um país | situação; um determinado campo; uma metáfora para uma determinada área ou situação}
  \end{Phonetics}
\end{Entry}

\begin{Entry}{天才}{4,3}{⼤、⼿}
  \begin{Phonetics}{天才}{tian1cai2}[][HSK 5]
    \definition{adj.}{talentoso | superdotado | genial}
    \definition[个,位,名]{s.}{dom; genialidade; talento natural; inteligência e sabedoria acima da média}
  \end{Phonetics}
\end{Entry}

\begin{Entry}{天公}{4,4}{⼤、⼋}
  \begin{Phonetics}{天公}{tian1gong1}
    \definition{s.}{céu, paraíso | senhor do céu}
  \end{Phonetics}
\end{Entry}

\begin{Entry}{天天}{4,4}{⼤、⼤}
  \begin{Phonetics}{天天}{tian1tian1}
    \definition{adv.}{todo dia}
  \end{Phonetics}
\end{Entry}

\begin{Entry}{天文}{4,4}{⼤、⽂}
  \begin{Phonetics}{天文}{tian1wen2}[][HSK 5]
    \definition[对]{s.}{astronomia; a distribuição e o movimento dos corpos celestes, como o sol, a lua e as estrelas, no universo}
  \end{Phonetics}
\end{Entry}

\begin{Entry}{天气}{4,4}{⼤、⽓}
  \begin{Phonetics}{天气}{tian1qi4}[][HSK 1]
    \definition{s.}{clima, tempo; mudanças meteorológicas que ocorrem na atmosfera em uma determinada área e durante um determinado período de tempo, tais como temperatura, umidade, pressão atmosférica, precipitação, vento, nuvens, etc.}
  \end{Phonetics}
\end{Entry}

\begin{Entry}{天花板}{4,7,8}{⼤、⾋、⽊}
  \begin{Phonetics}{天花板}{tian1hua1ban3}
    \definition{s.}{teto}
  \end{Phonetics}
\end{Entry}

\begin{Entry}{天使}{4,8}{⼤、⼈}
  \begin{Phonetics}{天使}{tian1shi3}
    \definition{s.}{anjo}
  \end{Phonetics}
\end{Entry}

\begin{Entry}{天择}{4,8}{⼤、⼿}
  \begin{Phonetics}{天择}{tian1ze2}
    \definition{s.}{seleção natural}
  \end{Phonetics}
\end{Entry}

\begin{Entry}{天空}{4,8}{⼤、⽳}
  \begin{Phonetics}{天空}{tian1kong1}[][HSK 3]
    \definition{s.}{o céu; o firmamento}
  \end{Phonetics}
\end{Entry}

\begin{Entry}{天柱}{4,9}{⼤、⽊}
  \begin{Phonetics}{天柱}{tian1zhu4}
    \definition{s.}{pilar celestial, que sustenta o céu}
  \end{Phonetics}
\end{Entry}

\begin{Entry}{天真}{4,10}{⼤、⼗}
  \begin{Phonetics}{天真}{tian1zhen1}[][HSK 4]
    \definition{adj.}{ingênuo; inocente; ignorante; simples de coração, direto por natureza, livre de fingimento e hipocrisia}
  \end{Phonetics}
\end{Entry}

\begin{Entry}{天堂}{4,11}{⼤、⼟}
  \begin{Phonetics}{天堂}{tian1tang2}[][HSK 6]
    \definition[间]{s.}{paraíso, céu; em algumas religiões, refere-se ao paraíso para onde as almas das pessoas boas retornam após a morte (diferente do 地狱) | lugar perfeito; ambiente de vida extremamente feliz e bonito; uma metáfora para um ambiente de vida feliz e bonito}
  \seealsoref{地狱}{di4yu4}
  \end{Phonetics}
\end{Entry}

\begin{Entry}{天然}{4,12}{⼤、⽕}
  \begin{Phonetics}{天然}{tian1 ran2}[][HSK 6]
    \definition{adj.}{natural; produzido ou ocorrido narturalmente}
  \end{Phonetics}
\end{Entry}

\begin{Entry}{天然气}{4,12,4}{⼤、⽕、⽓}
  \begin{Phonetics}{天然气}{tian1ran2qi4}[][HSK 5]
    \definition{s.}{gás; gás natural; gás combustível produzido em campos petrolíferos, carboníferos e pântanos}
  \end{Phonetics}
\end{Entry}

\begin{Entry}{天鹅}{4,12}{⼤、⿃}
  \begin{Phonetics}{天鹅}{tian1'e2}
    \definition{s.}{cisne}
  \end{Phonetics}
\end{Entry}

\begin{Entry}{太}{4}{⼤}
  \begin{Phonetics}{太}{tai4}[][HSK 1]
    \definition*{s.}{Sobrenome Tai}
    \definition{adj.}{mais alto; maior; mais distante | maior; extremo | bisavô; mais velho ou mais antigo; o de maior posição social ou hierarquia}
    \definition{adv.}{demais; expressa um grau excessivo (usado principalmente para coisas indesejáveis) | muito; extremamente; excessivamente; indica um grau extremamente elevado | muito; usado após o advérbio negativo 不, enfraquece o grau de negação e contém um tom diplomático}
  \end{Phonetics}
\end{Entry}

\begin{Entry}{太太}{4,4}{⼤、⼤}
  \begin{Phonetics}{太太}{tai4tai5}[][HSK 2]
    \definition[位,名,个,些]{s.}{senhora; madame; títulos para mulheres casadas | esposa; senhora; madame; referir-se à própria esposa ou à esposa de outra pessoa}
  \end{Phonetics}
\end{Entry}

\begin{Entry}{太平洋}{4,5,9}{⼤、⼲、⽔}
  \begin{Phonetics}{太平洋}{tai4ping2 yang2}
    \definition*{s.}{Oceano Pacífico}
  \end{Phonetics}
\end{Entry}

\begin{Entry}{太阳}{4,6}{⼤、⾩}
  \begin{Phonetics}{太阳}{tai4yang5}[][HSK 2]
    \definition[个,轮,枚,颗,盏]{s.}{o Sol | luz do sol; luz solar}
  \end{Phonetics}
\end{Entry}

\begin{Entry}{太阳日}{4,6,4}{⼤、⾩、⽇}
  \begin{Phonetics}{太阳日}{tai4yang2ri4}
    \definition{s.}{dia solar}
  \end{Phonetics}
\end{Entry}

\begin{Entry}{太阳风}{4,6,4}{⼤、⾩、⾵}
  \begin{Phonetics}{太阳风}{tai4yang2feng1}
    \definition{s.}{vento solar}
  \end{Phonetics}
\end{Entry}

\begin{Entry}{太阳穴}{4,6,5}{⼤、⾩、⽳}
  \begin{Phonetics}{太阳穴}{tai4yang2xue2}
    \definition{s.}{têmpora (nas laterais da cabeça humana)}
  \end{Phonetics}
\end{Entry}

\begin{Entry}{太阳灯}{4,6,6}{⼤、⾩、⽕}
  \begin{Phonetics}{太阳灯}{tai4yang2deng1}
    \definition{s.}{lâmpada solar (com células fotovoltaicas)}
  \end{Phonetics}
\end{Entry}

\begin{Entry}{太阳雨}{4,6,8}{⼤、⾩、⾬}
  \begin{Phonetics}{太阳雨}{tai4yang2yu3}
    \definition{s.}{banho de sol}
  \end{Phonetics}
\end{Entry}

\begin{Entry}{太阳能}{4,6,10}{⼤、⾩、⾁}
  \begin{Phonetics}{太阳能}{tai4 yang2 neng2}[][HSK 6]
    \definition{s.}{energia solar; a energia de radiação emitida pelo Sol é a energia solar produzida pela reação de fusão dos núcleos de hidrogênio no Sol, é a fonte de luz e calor na Terra}
  \end{Phonetics}
\end{Entry}

\begin{Entry}{太阳窗}{4,6,12}{⼤、⾩、⽳}
  \begin{Phonetics}{太阳窗}{tai4yang2chuang1}
    \definition{s.}{teto solar (de veículos)}
  \end{Phonetics}
\end{Entry}

\begin{Entry}{太阳镜}{4,6,16}{⼤、⾩、⾦}
  \begin{Phonetics}{太阳镜}{tai4yang2jing4}
    \definition{s.}{óculos de sol}
  \end{Phonetics}
\end{Entry}

\begin{Entry}{太阳翼}{4,6,17}{⼤、⾩、⽻}
  \begin{Phonetics}{太阳翼}{tai4yang2yi4}
    \definition{s.}{painel solar}
  \end{Phonetics}
\end{Entry}

\begin{Entry}{太极拳}{4,7,10}{⼤、⽊、⼿}
  \begin{Phonetics}{太极拳}{tai4ji2quan2}
    \definition*{s.}{Tai Chi Chuan, Taiji, T'aichi ou T'aichichuan; forma tradicional de exercício físico ou relaxamento}
  \end{Phonetics}
\end{Entry}

\begin{Entry}{太空}{4,8}{⼤、⽳}
  \begin{Phonetics}{太空}{tai4kong1}[][HSK 5]
    \definition[把]{s.}{firmamento; espaço sideral; espaço além da atmosfera terrestre; o céu vasto e infinito}
  \end{Phonetics}
\end{Entry}

\begin{Entry}{夫}{4}{⼤}
  \begin{Phonetics}{夫}{fu1}
    \definition{s.}{marido | homem | (velho) alguém que faz algum tipo de trabalho manual | (velho) uma pessoa que serviu em trabalho forçado}
  \end{Phonetics}
  \begin{Phonetics}{夫}{fu2}
    \definition{part.}{usado no início de uma frase | usado no final de uma frase ou em uma pausa no meio de uma frase para expressar uma exclamação}
    \definition{pron.}{isto; isso; aqueles; estes | ele}
  \end{Phonetics}
\end{Entry}

\begin{Entry}{夫人}{4,2}{⼤、⼈}
  \begin{Phonetics}{夫人}{fu1ren2}[][HSK 4]
    \definition[位,名,个]{s.}{senhora; \emph{lady}; madame; na antiguidade, as esposas dos senhores feudais eram chamadas de ``madame'' e, nas dinastias Ming e Qing, as esposas dos oficiais de primeiro e segundo escalão eram chamadas de ``madame'', que mais tarde foi usada para homenagear as esposas das pessoas em geral e agora é usada principalmente em ocasiões diplomáticas}
  \end{Phonetics}
\end{Entry}

\begin{Entry}{夫妇}{4,6}{⼤、⼥}
  \begin{Phonetics}{夫妇}{fu1fu4}[][HSK 4]
    \definition[对]{s.}{casal; marido e mulher}
  \end{Phonetics}
\end{Entry}

\begin{Entry}{夫妻}{4,8}{⼤、⼥}
  \begin{Phonetics}{夫妻}{fu1qi1}[][HSK 4]
    \definition[对]{s.}{casal; marido e mulher}
  \end{Phonetics}
\end{Entry}

\begin{Entry}{孔}{4}{⼦}
  \begin{Phonetics}{孔}{kong3}
    \definition*{s.}{Abreviação de Confúcio, 孔子 | Sobrenome Kong}
    \definition{adj.}{Clássico: muito; bastante; razoavelmente}
    \definition{adj.}{aberto; desimpedido; claro; desobstruído}
    \definition{clas.}{usado para habitações em cavernas}
    \definition[个,排]{s.}{buraco; abertura; poro}
  \seealsoref{孔子}{kong3zi3}
  \end{Phonetics}
\end{Entry}

\begin{Entry}{孔子}{4,3}{⼦、⼦}
  \begin{Phonetics}{孔子}{kong3zi3}
    \definition*{s.}{Confúcio (551-479 aC), pensador e filósofo social chinês}
  \seealsoref{孔夫子}{kong3fu1zi3}
  \end{Phonetics}
\end{Entry}

\begin{Entry}{孔子学院}{4,3,8,9}{⼦、⼦、⼦、⾩}
  \begin{Phonetics}{孔子学院}{kong3zi3 xue2yuan4}
    \definition*{s.}{Instituto Confúcio, organização estabelecida internacionalmente pela República Popular da China, que promove a língua e a cultura chinesas}
  \end{Phonetics}
\end{Entry}

\begin{Entry}{孔夫子}{4,4,3}{⼦、⼤、⼦}
  \begin{Phonetics}{孔夫子}{kong3fu1zi3}
    \definition*{s.}{Confúcio (551-479 aC), pensador e filósofo social chinês}
  \seealsoref{孔子}{kong3zi3}
  \end{Phonetics}
\end{Entry}

\begin{Entry}{孔雀}{4,11}{⼦、⾫}
  \begin{Phonetics}{孔雀}{kong3que4}
    \definition{s.}{pavão}
  \end{Phonetics}
\end{Entry}

\begin{Entry}{少}{4}{⼩}
  \begin{Phonetics}{少}{shao3}[][HSK 1]
    \definition{adj.}{menos; pouco (oposto a 多); escasso; não atingir a quantidade original ou esperada}
    \definition{adv.}{um momento; um instante; provisoriamente; ligeiramente}
    \definition{v.}{faltar; ser insuficiente | dever | perder; desaparecer; extraviar | parar; desistir}
  \seealsoref{多}{duo1}
  \end{Phonetics}
  \begin{Phonetics}{少}{shao4}
    \definition*{s.}{Sobrenome Shao}
    \definition{s.}{jovem (em oposição a 老)}
    \definition{s.}{jovem mestre; filho de uma família rica}
  \seealsoref{老}{lao3}
  \end{Phonetics}
\end{Entry}

\begin{Entry}{少儿}{4,2}{⼩、⼉}
  \begin{Phonetics}{少儿}{shao4 er2}[][HSK 6]
    \definition{s.}{criança}
  \end{Phonetics}
\end{Entry}

\begin{Entry}{少见}{4,4}{⼩、⾒}
  \begin{Phonetics}{少见}{shao3jian4}
    \definition{adj.}{raramente visto; infrequente; raro}
    \definition{v.}{(forma de saudação) ``Não te vejo há muito tempo.''; ou ``Tenho te visto muito pouco ultimamente.''--``Estou muito feliz em te ver novamente.'' | difícil de ver | não familiar (para o falante) | ser raro (algo)}
  \end{Phonetics}
\end{Entry}

\begin{Entry}{少年}{4,6}{⼩、⼲}
  \begin{Phonetics}{少年}{shao4 nian2}[][HSK 2]
    \definition[个,名,位]{s.}{adolescente; juventude; atualmente, a faixa etária geralmente referida é de 10 anos ou mais a 18 anos ou mais | menor; jovem; juvenil; refere-se a menores na faixa etária anterior | jovem; adolescente; rapaz}
  \end{Phonetics}
\end{Entry}

\begin{Entry}{少数}{4,13}{⼩、⽁}
  \begin{Phonetics}{少数}{shao3 shu4}[][HSK 2]
    \definition{s.}{número pequeno; poucos; minoria}
  \end{Phonetics}
\end{Entry}

\begin{Entry}{尤}{4}{⼪}
  \begin{Phonetics}{尤}{you2}
    \definition*{s.}{Sobrenome You}
    \definition{adj.}{excelente; peculiar; notável}
    \definition{adv.}{particularmente; especialmente}
    \definition{s.}{falha; erro | irregularidade}
    \definition{v.}{ter rancor contra; culpar}
  \end{Phonetics}
\end{Entry}

\begin{Entry}{尤其}{4,8}{⼪、⼋}
  \begin{Phonetics}{尤其}{you2qi2}[][HSK 5]
    \definition{adv.}{especialmente; particularmente; indica um grau mais avançado, equivalente a 更加}
  \seealsoref{更加}{geng4 jia1}
  \end{Phonetics}
\end{Entry}

\begin{Entry}{尺}{4}{⼫}
  \begin{Phonetics}{尺}{che3}
    \definition{s.}{(tom) uma nota da escala em Gongchepu (工尺谱), correspondente a 2 na notação musical numerada}
  \seealsoref{工尺谱}{gong1 che3 pu3}
  \end{Phonetics}
  \begin{Phonetics}{尺}{chi3}[][HSK 4]
    \definition{clas.}{chi, uma unidade de comprimento (=13 metros)}
    \definition[支,把]{s.}{régua; instrumentos de medição | um instrumento no formato de uma régua}
  \end{Phonetics}
\end{Entry}

\begin{Entry}{尺子}{4,3}{⼫、⼦}
  \begin{Phonetics}{尺子}{chi3zi5}[][HSK 4]
    \definition[把,根]{s.}{régua de madeira ou metal para orientar a caneta ou o lápis para desenhar linhas ou fazer medições}
  \end{Phonetics}
\end{Entry}

\begin{Entry}{尺寸}{4,3}{⼫、⼨}
  \begin{Phonetics}{尺寸}{chi3 cun4}[][HSK 4]
    \definition{s.}{tamanho; medida; dimensão}
  \end{Phonetics}
\end{Entry}

\begin{Entry}{尺度}{4,9}{⼫、⼴}
  \begin{Phonetics}{尺度}{chi3du4}[][HSK 7-9]
    \definition{s.}{padrão; critério; medida}
  \end{Phonetics}
\end{Entry}

\begin{Entry}{屯}{4}{⼬}
  \begin{Phonetics}{屯}{tun2}
    \definition*{s.}{Sobrenome Tun}
    \definition{s.}{vila (geralmente usado em nomes de vilas); vilarejos; aldeias; povoados}
    \definition{v.}{coletar; estocar; armazenar; acumular | estacionar (tropas); aquartelar}
  \end{Phonetics}
  \begin{Phonetics}{屯}{zhun1}
    \definition*{s.}{Sobrenome Zhun}
    \definition{adj.}{difícil; árduo | lento; obtuso}
  \end{Phonetics}
\end{Entry}

\begin{Entry}{巨}{4}{⼯}
  \begin{Phonetics}{巨}{ju4}
    \definition*{s.}{Sobrenome Ju}
    \definition{adj.}{enorme; tremendo; gigantesco}
  \end{Phonetics}
\end{Entry}

\begin{Entry}{巨大}{4,3}{⼯、⼤}
  \begin{Phonetics}{巨大}{ju4da4}[][HSK 4]
    \definition{adj.}{enorme; tremendo; enorme; gigantesco; imenso}
  \end{Phonetics}
\end{Entry}

\begin{Entry}{巴}{4}{⼰}
  \begin{Phonetics}{巴}{ba1}
    \definition*{s.}{Ba, um estado da Dinastia Zhou | Nome antigo de Sichuan Oriental | Sobrenome Ba}
    \definition{s.}{píton (cobra) | crosta; formação semelhante a uma crosta | parte de órgãos biológicos que denotam ponta, extremidade, cauda | bar, unidade física de pressão, 1 atm = 1,01325 bar}
    \definition{v.}{esperar sinceramente; esperar ansiosamente por; aguardar ansiosamente | agarrar-se a; grudar em; manter-se fiel a | escalar física e socialmente | estar perto de; estar ao lado de | abrir; dividir; rachar; quebrar}
  \end{Phonetics}
\end{Entry}

\begin{Entry}{巴士}{4,3}{⼰、⼠}
  \begin{Phonetics}{巴士}{ba1 shi4}[][HSK 4]
    \definition[辆]{s.}{ônibus, transliteração da palavra inglesa \emph{bus}}
  \end{Phonetics}
\end{Entry}

\begin{Entry}{巴不得}{4,4,11}{⼰、⼀、⼻}
  \begin{Phonetics}{巴不得}{ba1bu5de2}[][HSK 7-9]
    \definition{v.}{desejar sinceramente; ansiar ansiosamente por; estar muito ansioso (para fazer algo); ansiosamente aguardar; usado na linguagem falada}
  \end{Phonetics}
\end{Entry}

\begin{Entry}{巴西}{4,6}{⼰、⾑}
  \begin{Phonetics}{巴西}{ba1xi1}
    \definition*{s.}{Brasil}
  \end{Phonetics}
\end{Entry}

\begin{Entry}{巴西人}{4,6,2}{⼰、⾑、⼈}
  \begin{Phonetics}{巴西人}{ba1xi1 ren2}
    \definition[个,位]{s.}{brasileiro | pessoa ou povo do Brasil}[他是巴西人。===Ele é brasileiro.]
  \end{Phonetics}
\end{Entry}

\begin{Entry}{巴西战舞}{4,6,9,14}{⼰、⾑、⼽、⾇}
  \begin{Phonetics}{巴西战舞}{ba1xi1zhan4wu3}
    \definition{s.}{capoeira}
  \end{Phonetics}
\end{Entry}

\begin{Entry}{巴勒斯坦}{4,11,12,8}{⼰、⼒、⽄、⼟}
  \begin{Phonetics}{巴勒斯坦}{ba1le4si1tan3}
    \definition*{s.}{Palestina}
  \end{Phonetics}
\end{Entry}

\begin{Entry}{幻}{4}{⼳}
  \begin{Phonetics}{幻}{huan4}
    \definition{adj.}{irreal; imaginário; ilusório | mágico; mutável}
    \definition{v.}{mudar magicamente}
  \end{Phonetics}
\end{Entry}

\begin{Entry}{幻觉}{4,9}{⼳、⾒}
  \begin{Phonetics}{幻觉}{huan4jue2}[][HSK 7-9]
    \definition{s.}{uma ilusão; uma alucinação; sensações falsas na visão, audição, tato, etc., que ocorrem sem estímulo externo}
  \end{Phonetics}
\end{Entry}

\begin{Entry}{幻想}{4,13}{⼳、⼼}
  \begin{Phonetics}{幻想}{huan4xiang3}[][HSK 6]
    \definition[个,种]{s.}{fantasia; visão; arco-íris; ilusão; fruto da imaginação de alguém; imaginar algo que é difícil ou impossível de alcançar}
    \definition{v.}{imaginar; fantasiar; imaginar coisas que ainda não foram realizadas com base em ideais e desejos sociais ou pessoais}
  \end{Phonetics}
\end{Entry}

\begin{Entry}{幻影}{4,15}{⼳、⼺}
  \begin{Phonetics}{幻影}{huan4ying3}[][HSK 7-9]
    \definition{s.}{fantasma; imagem irreal | miragem}
  \end{Phonetics}
\end{Entry}

\begin{Entry}{廿}{4}{⼶}
  \begin{Phonetics}{廿}{nian4}
    \definition{num.}{(dialeto) vinte; 20}
  \end{Phonetics}
\end{Entry}

\begin{Entry}{开}{4}{⼶}
  \begin{Phonetics}{开}{kai1}[][HSK 1]
    \definition*{s.}{Sobrenome Kai}
    \definition{clas.}{divisão do papel de impressão de tamanho padrão (uma parte da folha inteira) | quilate; unidade de cálculo da quantidade de ouro puro contida no ouro}
    \definition{s.}{porcentagem; percentual}
    \definition{v.}{abrir; estar ligado; ligar | recuperar; abrir; fazer uma abertura; escavar; abrir caminho; desbravar | abrir para fora; soltar-se | descongelar (rios); tornar-se navegável | levantar; libertar | iniciar; operar; manobrar | mover; estabelecer | executar; configurar | começar; iniciar | manter | escrever; fazer uma lista de | pagamento (salários, tarifas, etc.) | ferver}
    \definition{v.aux.}{usado após um verbo, indica ampliação ou expansão | usado após um verbo, indica o início e a continuidade}
  \seealsoref{开尔文}{kai1'er3wen2}
  \end{Phonetics}
\end{Entry}

\begin{Entry}{开口}{4,3}{⼶、⼝}
  \begin{Phonetics}{开口}{kai1kou3}
    \definition{v.}{abrir a boca de alguém | começar a falar}
  \end{Phonetics}
\end{Entry}

\begin{Entry}{开心}{4,4}{⼶、⼼}
  \begin{Phonetics}{开心}{kai1xin1}[][HSK 2]
    \definition{adj.}{feliz; alegre; exultante; encantado}
    \definition{v.}{provocar; brincar; tirar sarro de alguém; zombar; divertir-se}
  \end{Phonetics}
\end{Entry}

\begin{Entry}{开水}{4,4}{⼶、⽔}
  \begin{Phonetics}{开水}{kai1shui3}[][HSK 4]
    \definition[杯,瓶]{s.}{água fervida; água fervente}
  \end{Phonetics}
\end{Entry}

\begin{Entry}{开车}{4,4}{⼶、⾞}
  \begin{Phonetics}{开车}{kai1/che1}[][HSK 1]
    \definition{v.+compl.}{dirigir um carro, trem, etc. | colocar uma máquina em funcionamento | (de um trem, etc.) partida | dirigir veículos motorizados}
  \end{Phonetics}
\end{Entry}

\begin{Entry}{开业}{4,5}{⼶、⼀}
  \begin{Phonetics}{开业}{kai1 ye4}[][HSK 3]
    \definition[场]{v.}{iniciar um negócio; abrir para negócios | abrir um consultório particular}
  \end{Phonetics}
\end{Entry}

\begin{Entry}{开发}{4,5}{⼶、⼜}
  \begin{Phonetics}{开发}{kai1fa1}[][HSK 3]
    \definition{v.}{explorar; trabalhar com recursos naturais como terras baldias, minas, florestas e energia hidráulica para fins de aproveitamento | tornar acessível; descobrir ou explorar talentos, tecnologias, etc. para aproveitamento}
  \end{Phonetics}
\end{Entry}

\begin{Entry}{开发区}{4,5,4}{⼶、⼜、⼖}
  \begin{Phonetics}{开发区}{kai1fa1qu1}
    \definition{s.}{zona de desenvolvimento}
  \end{Phonetics}
\end{Entry}

\begin{Entry}{开头}{4,5}{⼶、⼤}
  \begin{Phonetics}{开头}{kai1/tou2}[][HSK 6]
    \definition{s.}{início; começo; o momento ou estágio do início; antecedente no tempo}
    \definition{v.+compl.}{começar, iniciar; a primeira ocorrência de um evento, ação, fenômeno, etc. | pôr-se a pé; começar}
  \end{Phonetics}
\end{Entry}

\begin{Entry}{开尔文}{4,5,4}{⼶、⼩、⽂}
  \begin{Phonetics}{开尔文}{kai1'er3wen2}
    \definition{s.}{Kelvin, temperatura absoluta | K, escala de temperatura}
  \end{Phonetics}
\end{Entry}

\begin{Entry}{开会}{4,6}{⼶、⼈}
  \begin{Phonetics}{开会}{kai1/hui4}[][HSK 1]
    \definition{v.+compl.}{realizar uma reunião; ter uma reunião; participar de uma reunião (conferência)}
  \end{Phonetics}
\end{Entry}

\begin{Entry}{开关}{4,6}{⼶、⼋}
  \begin{Phonetics}{开关}{kai1 guan1}[][HSK 6]
    \definition[个,种,些]{s.}{interruptor; um dispositivo que conecta e desconecta o circuito de um dispositivo elétrico | registro; um dispositivo instalado em uma tubulação de fluido para controlar o fluxo}
  \end{Phonetics}
\end{Entry}

\begin{Entry}{开创}{4,6}{⼶、⼑}
  \begin{Phonetics}{开创}{kai1 chuang4}[][HSK 6]
    \definition{v.}{começar; iniciar; fundar; ser pioneiro; estabelecer; criar}
  \end{Phonetics}
\end{Entry}

\begin{Entry}{开机}{4,6}{⼶、⽊}
  \begin{Phonetics}{开机}{kai1 ji1}[][HSK 2]
    \definition{v.}{começar a filmar um filme ou programa de TV; refere-se ao início das filmagens (de filmes, séries de TV, etc.) | ligar uma máquina}
  \end{Phonetics}
\end{Entry}

\begin{Entry}{开设}{4,6}{⼶、⾔}
  \begin{Phonetics}{开设}{kai1 she4}[][HSK 6]
    \definition{v.}{montar; estabelecer; abrir (uma loja, fábrica, etc.); estabelecer novas instituições ou campos | oferecer (um curso na faculdade, etc.)}
  \end{Phonetics}
\end{Entry}

\begin{Entry}{开启}{4,7}{⼶、⼝}
  \begin{Phonetics}{开启}{kai1qi3}
    \definition{v.}{abrir | iniciar | (computação) ativar}
  \end{Phonetics}
\end{Entry}

\begin{Entry}{开花}{4,7}{⼶、⾋}
  \begin{Phonetics}{开花}{kai1hua1}[][HSK 4]
    \definition{v.}{florescer; desabrochar; estar em flor; entrar em flor;  metáfora para um coração feliz ou um rosto sorridente | explodir; quebrar; dividir | sentir-se feliz ou sorrir alegremente | (experiência) espalhar-se; (empreendimento) surgir; surgir | (cabeça) ser ferido e sangrar profusamente}
  \end{Phonetics}
\end{Entry}

\begin{Entry}{开夜车}{4,8,4}{⼶、⼣、⾞}
  \begin{Phonetics}{开夜车}{kai1/ye4che1}[][HSK 6]
    \definition{v.+compl.}{trabalhar até tarde da noite; ficar acordado até tarde da noite estudando ou trabalhando para cumprir prazos | (literalmente) ``conduzir carro à noite''}
  \end{Phonetics}
\end{Entry}

\begin{Entry}{开始}{4,8}{⼶、⼥}
  \begin{Phonetics}{开始}{kai1shi3}[][HSK 3]
    \definition[个]{s.}{começo; início; estágio inicial}
    \definition{v.}{começar; iniciar; começar a fazer algo}
  \end{Phonetics}
\end{Entry}

\begin{Entry}{开学}{4,8}{⼶、⼦}
  \begin{Phonetics}{开学}{kai1 xue2}[][HSK 2]
    \definition{v.}{iniciar as aulas; iniciar o semestre; começar as aulas}
  \end{Phonetics}
\end{Entry}

\begin{Entry}{开放}{4,8}{⼶、⽅}
  \begin{Phonetics}{开放}{kai1fang4}[][HSK 3]
    \definition{adj.}{de mente aberta; sem restrições por convenções; pensamento e ambiente não conservadores, disposto a aceitar coisas novas e novas ideias; personalidade alegre}
    \definition{v.}{florescer | abrir (para o público); levantar bloqueios, proibições, restrições, etc. | diminuir uma proibição, restrição, etc. (de política); (economia) reduzir as restrições políticas, com justificativas específicas}
  \end{Phonetics}
\end{Entry}

\begin{Entry}{开玩笑}{4,8,10}{⼶、⽟、⽵}
  \begin{Phonetics}{开玩笑}{kai1wan2xiao4}[][HSK 1]
    \definition{v.}{fazer (ou brincar, fazer) uma piada; gracejar; zombar de; provocar; fazer uma brincadeira; zombar de alguém | tratar casualmente; dar pouca importância a; considerar como um assunto insignificante; insignificante | fazer uma brincadeira; pregar uma peça; brincar; em tom de brincadeira}
  \end{Phonetics}
\end{Entry}

\begin{Entry}{开展}{4,10}{⼶、⼫}
  \begin{Phonetics}{开展}{kai1zhan3}[][HSK 3]
    \definition{v.}{lançar; desenvolver | abrir; inaugurar}
  \end{Phonetics}
\end{Entry}

\begin{Entry}{开通}{4,10}{⼶、⾡}
  \begin{Phonetics}{开通}{kai1 tong1}[][HSK 6]
    \definition{v.}{limpar; dragar; remover obstáculos de; abrir o canal; desbloquear}
  \end{Phonetics}
  \begin{Phonetics}{开通}{kai1 tong5}
    \definition{adj.}{liberal; mente aberta; mente moderna; mente liberal; sábio e sensato; não conservador ou teimoso}
  \end{Phonetics}
\end{Entry}

\begin{Entry}{开锁}{4,12}{⼶、⾦}
  \begin{Phonetics}{开锁}{kai1suo3}
    \definition{v.}{desbloquear | destravar}
  \end{Phonetics}
\end{Entry}

\begin{Entry}{开幕}{4,13}{⼶、⼱}
  \begin{Phonetics}{开幕}{kai1 mu4}[][HSK 5]
    \definition{v.}{começar a apresentação; iniciar o espetáculo; levantar das cortinas | abrir; inaugurar; iniciar (uma conferência, exposição, etc.)}
  \end{Phonetics}
\end{Entry}

\begin{Entry}{开幕式}{4,13,6}{⼶、⼱、⼷}
  \begin{Phonetics}{开幕式}{kai1mu4shi4}[][HSK 5]
    \definition[场,次,届]{s.}{cerimônia de abertura; cerimônias e apresentações antes de eventos esportivos ou grandes eventos}
  \end{Phonetics}
\end{Entry}

\begin{Entry}{引}{4}{⼸}
  \begin{Phonetics}{引}{yin3}[][HSK 4]
    \definition*{s.}{Sobrenome Yin}
    \definition{clas.}{uma unidade de comprimento (=33,333\dots metros)}
    \definition{v.}{puxar; esticar | liderar; conduzir; guiar | sair; deixar | sobressair | atrair; provocar; trazer à existência | causar; provocar | citar; ser usado como evidência ou justificativa}
  \end{Phonetics}
\end{Entry}

\begin{Entry}{引导}{4,6}{⼸、⼨}
  \begin{Phonetics}{引导}{yin3dao3}[][HSK 4]
    \definition{v.}{conduzir; guiar; liderar; andar na frente e deixar que os outros sigam atrás para ver ou andar; usar imagens ou sinais para mostrar às pessoas para onde ir | esclarecer; fornecer orientação em termos de ideias, métodos, conceitos, etc.}
  \end{Phonetics}
\end{Entry}

\begin{Entry}{引进}{4,7}{⼸、⾡}
  \begin{Phonetics}{引进}{yin3 jin4}[][HSK 4]
    \definition{v.}{importar; trazer de fora | recomendar; dar uma indicação}
  \end{Phonetics}
\end{Entry}

\begin{Entry}{引起}{4,10}{⼸、⾛}
  \begin{Phonetics}{引起}{yin3qi3}[][HSK 4]
    \definition{v.}{causar; despertar; levar a; desencadear; dar origem a}
  \end{Phonetics}
\end{Entry}

\begin{Entry}{引擎}{4,16}{⼸、⼿}
  \begin{Phonetics}{引擎}{yin3qing2}
    \definition[台]{s.}{motor | (empréstimo linguístico) \emph{engine}}
  \end{Phonetics}
\end{Entry}

\begin{Entry}{心}{4}{⼼}[Kangxi 61]
  \begin{Phonetics}{心}{xin1}[][HSK 3]
    \definition*{s.}{Xin, uma das mansões lunares; uma das vinte e oito constelações}
    \definition[颗,个]{s.}{o coraçã; órgão que impulsiona a circulação sanguínea no corpo humano e nos vertebrados| coração; mente; sentimento; intenção; refere-se aos órgãos do pensamento e ao pensamento, sentimentos, etc. | centro; núcleo; parte central}
  \end{Phonetics}
\end{Entry}

\begin{Entry}{心中}{4,4}{⼼、⼁}
  \begin{Phonetics}{心中}{xin1zhong1}[][HSK 2]
    \definition{s.}{no coração; na mente}
  \end{Phonetics}
\end{Entry}

\begin{Entry}{心机}{4,6}{⼼、⽊}
  \begin{Phonetics}{心机}{xin1ji1}
    \definition{s.}{pensamento | esquema}
  \end{Phonetics}
\end{Entry}

\begin{Entry}{心声}{4,7}{⼼、⼠}
  \begin{Phonetics}{心声}{xin1sheng1}
    \definition{s.}{desejo sincero | voz interior | aspiração}
  \end{Phonetics}
\end{Entry}

\begin{Entry}{心灵}{4,7}{⼼、⽕}
  \begin{Phonetics}{心灵}{xin1ling2}[][HSK 6]
    \definition[个,颗]{s.}{alma; coração; espírito; refere-se ao coração, espírito, pensamentos, etc.}
  \end{Phonetics}
\end{Entry}

\begin{Entry}{心里}{4,7}{⼼、⾥}
  \begin{Phonetics}{心里}{xin1 li3}[][HSK 2]
    \definition[个]{s.}{no coração; no coração de alguém | no coração; na mente; na cabeça e no peito}
  \end{Phonetics}
\end{Entry}

\begin{Entry}{心态}{4,8}{⼼、⼼}
  \begin{Phonetics}{心态}{xin1tai4}[][HSK 5]
    \definition[种,个]{s.}{mentalidade; psicologia; estado mental}
  \end{Phonetics}
\end{Entry}

\begin{Entry}{心疼}{4,10}{⼼、⽧}
  \begin{Phonetics}{心疼}{xin1teng2}[][HSK 5]
    \definition{v.}{amar profundamente; sentir pena porque coisas valiosas foram destruídas ou perdidas; não querer se separar delas | sentir pena; ficar angustiado; preocupar-se e sofrer pelo sofrimento dos outros; estar disposto a cuidar mais por causa da preocupação}
  \end{Phonetics}
\end{Entry}

\begin{Entry}{心脏}{4,10}{⼼、⾁}
  \begin{Phonetics}{心脏}{xin1zang4}[][HSK 6]
    \definition[颗,个]{s.}{coração; um órgão importante no corpo de humanos ou animais superiores que faz o sangue circular | coração; o centro ou a parte mais importante de uma metáfora}
  \end{Phonetics}
\end{Entry}

\begin{Entry}{心脏病}{4,10,10}{⼼、⾁、⽧}
  \begin{Phonetics}{心脏病}{xin1 zang4 bing4}[][HSK 6]
    \definition{s.}{doença cardíaca; cardiopatia um termo geral para anormalidades ou doenças na estrutura e função do coração humano}
  \end{Phonetics}
\end{Entry}

\begin{Entry}{心情}{4,11}{⼼、⼼}
  \begin{Phonetics}{心情}{xin1qing2}[][HSK 2]
    \definition{s.}{humor; tom de sentimento; estado de espírito; estado emocional interior}
  \end{Phonetics}
\end{Entry}

\begin{Entry}{心理}{4,11}{⼼、⽟}
  \begin{Phonetics}{心理}{xin1li3}[][HSK 4]
    \definition[个]{s.}{mentalidade; refere-se à reflexão da mente humana sobre coisas objetivas, incluindo sensação, percepção, memória, pensamento e emoções | psicologia}
  \end{Phonetics}
\end{Entry}

\begin{Entry}{心愿}{4,14}{⼼、⽕}
  \begin{Phonetics}{心愿}{xin1 yuan4}[][HSK 6]
    \definition[桩]{s.}{desejo acalentado; aspiração; desejo; sonho | o desejo do coração}
  \end{Phonetics}
\end{Entry}

\begin{Entry}{戈}{4}{⼽}[Kangxi 62]
  \begin{Phonetics}{戈}{ge1}
    \definition*{s.}{Sobrenome Ge}
    \definition{s.}{Arcaico: machado-adaga (com cabo longo e lâmina horizontal) | punhal-machado; arma antiga, lâmina cruzada, feita de bronze ou ferro, com cabo longo}
  \end{Phonetics}
\end{Entry}

\begin{Entry}{戈壁}{4,16}{⼽、⼟}
  \begin{Phonetics}{戈壁}{ge1bi4}[][HSK 7-9]
    \definition*{s.}{Deserto de Gobi}
    \definition{s.}{deserto; refere-se a uma área desértica onde o solo é quase coberto por areia grossa e cascalho e há poucas plantas}
  \end{Phonetics}
\end{Entry}

\begin{Entry}{户}{4}{⼾}[Kangxi 63]
  \begin{Phonetics}{户}{hu4}[][HSK 4]
    \definition*{s.}{Sobrenome Hu}
    \definition[个]{s.}{porta com um painel; porta | domicílio; residência; família | status familiar | conta (banco)}
  \end{Phonetics}
\end{Entry}

\begin{Entry}{户外}{4,5}{⼾、⼣}
  \begin{Phonetics}{户外}{hu4 wai4}[][HSK 6]
    \definition{s.}{ao ar livre; espaço aberto ao ar livre}
  \end{Phonetics}
\end{Entry}

\begin{Entry}{手}{4}{⼿}[Kangxi 64]
  \begin{Phonetics}{手}{shou3}[][HSK 1]
    \definition{adj.}{prático; conveniente}
    \definition{adv.}{pessoalmente | para habilidade ou destreza}
    \definition{clas.}{usado para habilidades e competências | usado para indicar o número de vezes em que algo foi feito}
    \definition[双,只]{s.}{mão | pessoa proficiente em determinada atividade | habilidade; meios; referência a habilidades, técnicas ou meios | uma pessoa que faz ou é boa em determinado trabalho}
    \definition{v.}{ter na mão; segurar}
  \end{Phonetics}
\end{Entry}

\begin{Entry}{手工}{4,3}{⼿、⼯}
  \begin{Phonetics}{手工}{shou3gong1}[][HSK 4]
    \definition{s.}{trabalho manual; trabalho feito à mão | método de operação manual; método manual, sem máquina | remuneração por trabalho manual, braçal; custo de mão de obra braçal}
  \end{Phonetics}
\end{Entry}

\begin{Entry}{手工艺人}{4,3,4,2}{⼿、⼯、⾋、⼈}
  \begin{Phonetics}{手工艺人}{shou3gong1 yi4ren2}
    \definition{s.}{artesão}
  \end{Phonetics}
\end{Entry}

\begin{Entry}{手术}{4,5}{⼿、⽊}
  \begin{Phonetics}{手术}{shou3shu4}[][HSK 4]
    \definition[个,次]{s.}{cirurgia; operação (cirúrgica); método de tratamento no qual o médico usa uma faca, tesoura etc. para fazer uma incisão em uma parte do corpo do paciente}
    \definition{v.}{realizar uma cirurgia}
  \end{Phonetics}
\end{Entry}

\begin{Entry}{手边}{4,5}{⼿、⾡}
  \begin{Phonetics}{手边}{shou3bian1}
    \definition{adv.}{à mão | na mão}
  \end{Phonetics}
\end{Entry}

\begin{Entry}{手机}{4,6}{⼿、⽊}
  \begin{Phonetics}{手机}{shou3ji1}[][HSK 1]
    \definition[部,台,个]{s.}{celular; telefone celular; telefone móvel}
  \end{Phonetics}
\end{Entry}

\begin{Entry}{手里}{4,7}{⼿、⾥}
  \begin{Phonetics}{手里}{shou3 li3}[][HSK 4]
    \definition[个]{s.}{(uma situação está) nas mãos de alguém | em mãos}
  \end{Phonetics}
\end{Entry}

\begin{Entry}{手刹}{4,8}{⼿、⼑}
  \begin{Phonetics}{手刹}{shou3sha1}
    \definition{s.}{freio de mão}
  \end{Phonetics}
\end{Entry}

\begin{Entry}{手法}{4,8}{⼿、⽔}
  \begin{Phonetics}{手法}{shou3fa3}[][HSK 5]
    \definition[种,个]{s.}{habilidade; técnica; técnicas de criação (de obras literárias e artísticas) | truque; artifício; artimanha; refere-se a métodos inadequados usados para lidar com as pessoas}
  \end{Phonetics}
\end{Entry}

\begin{Entry}{手表}{4,8}{⼿、⾐}
  \begin{Phonetics}{手表}{shou3biao3}[][HSK 2]
    \definition[块,只,个]{s.}{relógio de pulso}
  \end{Phonetics}
\end{Entry}

\begin{Entry}{手指}{4,9}{⼿、⼿}
  \begin{Phonetics}{手指}{shou3zhi3}[][HSK 3]
    \definition[个,根,只]{s.}{dedo da mão}
  \end{Phonetics}
\end{Entry}

\begin{Entry}{手段}{4,9}{⼿、⽎}
  \begin{Phonetics}{手段}{shou3 duan4}[][HSK 5]
    \definition[种,个]{s.}{meios; meio; medida; método; métodos e técnicas utilizados para atingir um determinado objetivo | truque; artifício; métodos inadequados de lidar com as pessoas | habilidade; capacidade; delicadeza; sutileza; técnica}
  \end{Phonetics}
\end{Entry}

\begin{Entry}{手套}{4,10}{⼿、⼤}
  \begin{Phonetics}{手套}{shou3tao4}[][HSK 4]
    \definition[副,套,双,种]{s.}{luvas; itens usados ​​nas mãos, feitos de algodão, lã, couro, etc., para proteger as mãos ou manter o frio longe}
  \end{Phonetics}
\end{Entry}

\begin{Entry}{手续}{4,11}{⼿、⽷}
  \begin{Phonetics}{手续}{shou3xu4}[][HSK 3]
    \definition[项]{s.}{processo; formalidade; procedimento; procedimentos realizados de acordo com os regulamentos}
  \end{Phonetics}
\end{Entry}

\begin{Entry}{手续费}{4,11,9}{⼿、⽷、⾙}
  \begin{Phonetics}{手续费}{shou3 xu4 fei4}[][HSK 6]
    \definition{s.}{comissão; corretagem; taxa de serviço; taxas a pagar pelos procedimentos de manuseio}
  \end{Phonetics}
\end{Entry}

\begin{Entry}{手臂}{4,17}{⼿、⾁}
  \begin{Phonetics}{手臂}{shou3bi4}
    \definition{s.}{braço}
  \end{Phonetics}
\end{Entry}

\begin{Entry}{扎}{4}{⼿}
  \begin{Phonetics}{扎}{za1}
    \definition{clas.}{usado para pacotes, feixes, maços, etc.}
    \definition{v.}{atar; amarrar; embrulhar}
  \end{Phonetics}
  \begin{Phonetics}{扎}{zha1}[][HSK 6]
    \definition{s.}{chope; cerveja de pressão | caneca para chope}
    \definition{v.}{furar; esfaquear; enfiar (uma agulha, etc.) em | estacionar; aquartelar | entrar em; mergulhar em | trapacear; defraudar}
  \end{Phonetics}
  \begin{Phonetics}{扎}{zha2}
    \definition{v.}{lutar}
  \end{Phonetics}
\end{Entry}

\begin{Entry}{扎实}{4,8}{⼿、⼧}
  \begin{Phonetics}{扎实}{zha1shi5}[][HSK 6]
    \definition{adj.}{robusto; forte; sólido; confiável | sólido; pé no chão; prático}
  \end{Phonetics}
\end{Entry}

\begin{Entry}{支}{4}{⽀}[Kangxi 65]
  \begin{Phonetics}{支}{zhi1}[][HSK 3,4]
    \definition*{s.}{Sobrenome Zhi}
    \definition{clas.}{usado para equipes, etc. | usado em canções ou peças musicais | intensidade luminosa utilizada para luzes elétricas | usado para objetos longos e finos}
    \definition{s.}{ramo; ramificação; tribo | os doze ramos terrestres}
    \definition{v.}{segurar; apoiar; sustentar | sobressair; esticar; erguer; estender | ordenar; enviar | receber; pagar; obter pagamento (dinheiro)}
  \end{Phonetics}
\end{Entry}

\begin{Entry}{支支吾吾}{4,4,7,7}{⽀、⽀、⼝、⼝}
  \begin{Phonetics}{支支吾吾}{zhi1zhi1wu2wu2}
    \definition{v.}{falhar | murmurar | paralisar | gaguejar}
  \end{Phonetics}
\end{Entry}

\begin{Entry}{支付}{4,5}{⽀、⼈}
  \begin{Phonetics}{支付}{zhi1 fu4}[][HSK 3]
    \definition{v.}{pagar (dinheiro); custear; efetuar pagamento}
  \end{Phonetics}
\end{Entry}

\begin{Entry}{支出}{4,5}{⽀、⼐}
  \begin{Phonetics}{支出}{zhi1chu1}[][HSK 5]
    \definition[笔,项]{s.}{despesas; gastos}
    \definition{v.}{pagar; gastar; desembolsar; efetuar o pagamento}
  \end{Phonetics}
\end{Entry}

\begin{Entry}{支应}{4,7}{⽀、⼴}
  \begin{Phonetics}{支应}{zhi1ying4}
    \definition{v.}{lidar com | fornecer}
  \end{Phonetics}
\end{Entry}

\begin{Entry}{支承}{4,8}{⽀、⼿}
  \begin{Phonetics}{支承}{zhi1cheng2}
    \definition{v.}{suportar o peso de (um edifício) | suportar}
  \end{Phonetics}
\end{Entry}

\begin{Entry}{支持}{4,9}{⽀、⼿}
  \begin{Phonetics}{支持}{zhi1chi2}[][HSK 3]
    \definition{v.}{suportar; aguentar; resistir; manter-se com dificuldade | apoiar; dar incentivo ou patrocínio}
  \end{Phonetics}
\end{Entry}

\begin{Entry}{支根}{4,10}{⽀、⽊}
  \begin{Phonetics}{支根}{zhi1gen1}
    \definition{s.}{raiz ramificada | raízes de apoio | radícula}
  \end{Phonetics}
\end{Entry}

\begin{Entry}{支配}{4,10}{⽀、⾣}
  \begin{Phonetics}{支配}{zhi1pei4}[][HSK 5]
    \definition{v.}{organizar; alocar; orçar; distribuir | controlar; dominar; governar; exercer influência e controle sobre pessoas ou coisas}
  \end{Phonetics}
\end{Entry}

\begin{Entry}{支票}{4,11}{⽀、⽰}
  \begin{Phonetics}{支票}{zhi1piao4}
    \definition[本]{s.}{cheque (banco)}
  \end{Phonetics}
\end{Entry}

\begin{Entry}{支援}{4,12}{⽀、⼿}
  \begin{Phonetics}{支援}{zhi1yuan2}[][HSK 6]
    \definition{v.}{ajudar; auxiliar; apoiar; utilizar ações humanas, materiais, financeiras ou outras ações práticas para apoiar e auxiliar}
  \end{Phonetics}
\end{Entry}

\begin{Entry}{支撑}{4,15}{⽀、⼿}
  \begin{Phonetics}{支撑}{zhi1cheng1}[][HSK 6]
    \definition{v.}{sustentar; apoiar}[柱子支撑着整个建筑。===Os pilares sustentam todo o edifício. | 一家的生活由他支撑。===Ele sustenta a vida da família.]
  \end{Phonetics}
\end{Entry}

\begin{Entry}{文}{4}{⽂}[Kangxi 67]
  \begin{Phonetics}{文}{wen2}
    \definition*{s.}{Sobrenome Wen}
    \definition{adj.}{civil; não militar, não violento, oposto a 武 | suave; refinado | refinado; literário ; descreve que o conteúdo de um artigo ou discurso é difícil de entender}
    \definition{clas.}{usado para moedas de cobre antigas (moedas de cobre com palavras gravadas em um lado)}
    \definition{s.}{roteiro; escrita; personagem | escrita; composição literária; artigo | linguagem literária; redação | cultura; refere-se ao estado manifestado quando a sociedade atinge um estágio superior de desenvolvimento | ritual; refere-se ao antigo sistema ritual e musical | certos fenômenos naturais; refere-se a certos fenômenos na natureza ou na sociedade humana | literatos; coisas não militares (ao contrário de "武") | arte liberal; refere-se às ciências humanas e sociais | documento; refere-se a documentos oficiais | padrão; textura}
    \definition{v.}{tatuar padrões ou palavras no corpo ou no rosto | cobrir; pintar por cima}
  \seealsoref{武}{wu3}
  \end{Phonetics}
\end{Entry}

\begin{Entry}{文化}{4,4}{⽂、⼔}
  \begin{Phonetics}{文化}{wen2hua4}[][HSK 3]
    \definition[个,种]{s.}{cultura; civilização; tudo o que os seres humanos criaram em termos materiais e espirituais ao longo da história social | cultura; alfabetização; escolaridade; educação; o nível de conhecimento das pessoas e a capacidade de usar a linguagem escrita}
  \end{Phonetics}
\end{Entry}

\begin{Entry}{文化水平}{4,4,4,5}{⽂、⼔、⽔、⼲}
  \begin{Phonetics}{文化水平}{wen2hua4 shui3ping2}
    \definition{s.}{nível educacional}
  \end{Phonetics}
\end{Entry}

\begin{Entry}{文化史}{4,4,5}{⽂、⼔、⼝}
  \begin{Phonetics}{文化史}{wen2hua4shi3}
    \definition*{s.}{História Cultural}
  \end{Phonetics}
\end{Entry}

\begin{Entry}{文化层}{4,4,7}{⽂、⼔、⼫}
  \begin{Phonetics}{文化层}{wen2hua4ceng2}
    \definition{s.}{nível de cultura (em sítio arqueológico)}
  \end{Phonetics}
\end{Entry}

\begin{Entry}{文化宫}{4,4,9}{⽂、⼔、⼧}
  \begin{Phonetics}{文化宫}{wen2hua4gong1}
    \definition{s.}{palácio cultural}
  \end{Phonetics}
\end{Entry}

\begin{Entry}{文化热}{4,4,10}{⽂、⼔、⽕}
  \begin{Phonetics}{文化热}{wen2hua4re4}
    \definition{s.}{mania cultural | febre cultural}
  \end{Phonetics}
\end{Entry}

\begin{Entry}{文化圈}{4,4,11}{⽂、⼔、⼞}
  \begin{Phonetics}{文化圈}{wen2hua4quan1}
    \definition{s.}{esfera de influência cultural}
  \end{Phonetics}
\end{Entry}

\begin{Entry}{文化障碍}{4,4,13,13}{⽂、⼔、⾩、⽯}
  \begin{Phonetics}{文化障碍}{wen2hua4 zhang4'ai4}
    \definition{s.}{barreira cultural}
  \end{Phonetics}
\end{Entry}

\begin{Entry}{文艺}{4,4}{⽂、⾋}
  \begin{Phonetics}{文艺}{wen2yi4}[][HSK 5]
    \definition{s.}{termo genérico para literatura e arte | performance (arte); refere-se especificamente às artes performativas, como música e dança}
  \end{Phonetics}
\end{Entry}

\begin{Entry}{文艺界}{4,4,9}{⽂、⾋、⽥}
  \begin{Phonetics}{文艺界}{wen2 yi4 jie4}[][HSK 6]
    \definition{s.}{círculos literários e artísticos; o mundo da literatura e da arte}
  \end{Phonetics}
\end{Entry}

\begin{Entry}{文件}{4,6}{⽂、⼈}
  \begin{Phonetics}{文件}{wen2jian4}[][HSK 3]
    \definition[份,堆,叠]{s.}{documentos oficiais; papéis; instrumentos; termo genérico para documentos oficiais, cartas, etc. | os arquivos no computador; informações registradas no celular ou computador | artigos ou trabalhos sobre teorias políticas, atualidades, pesquisas acadêmicas, etc.; textos ou artigos sobre teoria política, políticas, etc.}
  \end{Phonetics}
\end{Entry}

\begin{Entry}{文字}{4,6}{⽂、⼦}
  \begin{Phonetics}{文字}{wen2zi4}[][HSK 3]
    \definition[种,类,段,行,篇]{s.}{caracteres; caligrafia; escrita; símbolos escritos para registrar a linguagem| linguagem escrita; a forma escrita da língua}
  \end{Phonetics}
\end{Entry}

\begin{Entry}{文学}{4,8}{⽂、⼦}
  \begin{Phonetics}{文学}{wen2xue2}[][HSK 3]
    \definition[个,种]{s.}{literatura; a arte de moldar imagens e refletir a vida social através da linguagem e da escrita, incluindo romances, poesia, prosa, teatro, etc.}
  \end{Phonetics}
\end{Entry}

\begin{Entry}{文学系}{4,8,7}{⽂、⼦、⽷}
  \begin{Phonetics}{文学系}{wen2xue2 xi4}
    \definition*{s.}{Departamento de Literatura}
  \end{Phonetics}
\end{Entry}

\begin{Entry}{文明}{4,8}{⽂、⽇}
  \begin{Phonetics}{文明}{wen2ming2}[][HSK 3]
    \definition{adj.}{civilizado; sociedade desenvolvida e com alto nível cultural}
    \definition[个,种]{s.}{cultura; civilização}
  \end{Phonetics}
\end{Entry}

\begin{Entry}{文娱}{4,10}{⽂、⼥}
  \begin{Phonetics}{文娱}{wen2 yu2}[][HSK 6]
    \definition{s.}{recreação cultural; entretenimento}
  \end{Phonetics}
\end{Entry}

\begin{Entry}{文章}{4,11}{⽂、⾳}
  \begin{Phonetics}{文章}{wen2zhang1}[][HSK 3]
    \definition[篇,段,页,系列]{s.}{ensaio; artigo; texto independente; também se refere a obras literárias em geral | significado oculto; significado implícito | trabalho; coisas que podem ser feitas}
  \end{Phonetics}
\end{Entry}

\begin{Entry}{斗}{4}{⽃}[Kangxi 68]
  \begin{Phonetics}{斗}{dou3}
    \definition*{s.}{Ursa Maior; Abreviação de Estrela Polar | Dou, uma das mansões lunares; uma das 28 constelações | Sobrenome Dou}
    \definition{clas.}{dou, unidade de medida seca para grãos (=1 decalitro)}
    \definition{s.}{dou, medida para grãos; instrumento para medir grãos | um objeto com a forma de um copo ou concha; algo que se parece com um balde | espiral (de uma impressão digital)}
  \end{Phonetics}
  \begin{Phonetics}{斗}{dou4}[][HSK 7-9]
    \definition{adj.}{Dialeto: engraçado}
    \definition{v.}{brigar; lutar | lutar contra; denunciar | competir com; disputar com | fazer animais ou insetos lutarem (como um jogo) | encaixar-se | brincar com; provocar | provocar (risos, etc.); divertir | ser remendado; juntar; unir; combinar}
  \end{Phonetics}
\end{Entry}

\begin{Entry}{斗争}{4,6}{⽃、⼑}
  \begin{Phonetics}{斗争}{dou4zheng1}[][HSK 6]
    \definition[场,番]{s.}{luta; conflito; batalha; os dois lados entram em conflito entre si}
    \definition{v.}{lutar; combater; batalhar; os dois lados estão em conflito, lutando para derrotar um ao outro | esforçar-se por; lutar por; trabalhar duro por; expor e criticar; atacar | lutar; batalhar}
  \end{Phonetics}
\end{Entry}

\begin{Entry}{斗志}{4,7}{⽃、⼼}
  \begin{Phonetics}{斗志}{dou4zhi4}[][HSK 7-9]
    \definition{s.}{vontade de lutar; espírito combativo}
  \end{Phonetics}
\end{Entry}

\begin{Entry}{斤}{4}{⽄}[Kangxi 69]
  \begin{Phonetics}{斤}{jin1}[][HSK 2]
    \definition{clas.}{uma unidade de peso (=500 gramas)}
    \definition{s.}{machado; cutelo; ferramentas antigas para cortar árvores}
  \end{Phonetics}
\end{Entry}

\begin{Entry}{方}{4}{⽅}[Kangxi 70]
  \begin{Phonetics}{方}{fang1}[][HSK 4]
    \definition*{s.}{Alquimia, 方术 | Sobrenome Fang}
    \definition{adj.}{reto; honesto; imparcial}
    \definition{adv.}{exatamente quando; no momento em que}
    \definition{clas.}{usado para coisas quadradas | quadrado ou cúbico (geralmente metro quadrado ou cúbico)}
    \definition[个,张]{s.}{quadrado; um quadrado ou sólido com seis faces quadradas | matemática: potência; o número de vezes que uma quantidade deve ser multiplicada por si mesma | direção | lado; festa | lugar; região; localidade | maneira; método; solução | prescrição | lei; regra}
  \seealsoref{方术}{fang1 shu4}
  \end{Phonetics}
\end{Entry}

\begin{Entry}{方方面面}{4,4,9,9}{⽅、⽅、⾯、⾯}
  \begin{Phonetics}{方方面面}{fang1fang1mian4mian4}[][HSK 7-9]
    \definition{expr.}{todos os aspectos; todos os lados | multifacetado}
  \end{Phonetics}
\end{Entry}

\begin{Entry}{方片}{4,4}{⽅、⽚}
  \begin{Phonetics}{方片}{fang1 pian4}
    \definition{s.}{ouros ♦ (em jogos de cartas)}
  \seealsoref{黑桃}{hei1 tao2}
  \seealsoref{红心}{hong2 xin1}
  \seealsoref{梅花}{mei2 hua1}
  \end{Phonetics}
\end{Entry}

\begin{Entry}{方术}{4,5}{⽅、⽊}
  \begin{Phonetics}{方术}{fang1 shu4}
    \definition{s.}{artes de cura, adivinhação, horóscopo etc. | Arcaico: artes sobrenaturais}
  \end{Phonetics}
\end{Entry}

\begin{Entry}{方向}{4,6}{⽅、⼝}
  \begin{Phonetics}{方向}{fang1xiang4}[][HSK 2]
    \definition[个,种]{s.}{direção; orientação; referindo-se a leste, sul, oeste, norte, sudeste, sudoeste, nordeste, noroeste, etc. | objetivo; meta; finalidade}
  \end{Phonetics}
\end{Entry}

\begin{Entry}{方向盘}{4,6,11}{⽅、⼝、⽫}
  \begin{Phonetics}{方向盘}{fang1xiang4pan2}[][HSK 7-9]
    \definition[个]{s.}{volante; um dispositivo em forma de roda para controlar a direção de viagem de navios, carros, etc.}
  \end{Phonetics}
\end{Entry}

\begin{Entry}{方式}{4,6}{⽅、⼷}
  \begin{Phonetics}{方式}{fang1shi4}[][HSK 3]
    \definition[种,个]{s.}{maneira; método}
  \end{Phonetics}
\end{Entry}

\begin{Entry}{方言}{4,7}{⽅、⾔}
  \begin{Phonetics}{方言}{fang1yan2}[][HSK 7-9]
    \definition*{s.}{O primeiro Dicionário de Dialeto Chinês, editado por Yang Xiong, 扬雄, no século I, contendo mais de 9.000 caracteres}
    \definition[口]{s.}{dialeto; um ramo regional de uma língua, formado durante sua evolução, que difere da língua padrão e é usado apenas em uma determinada área}
  \seealsoref{扬雄}{yang2xiong2}
  \end{Phonetics}
\end{Entry}

\begin{Entry}{方针}{4,7}{⽅、⾦}
  \begin{Phonetics}{方针}{fang1zhen1}[][HSK 4]
    \definition[个,项]{s.}{política; diretriz; princípio orientador; orientação da direção e das metas de um empreendimento}
  \end{Phonetics}
\end{Entry}

\begin{Entry}{方法}{4,8}{⽅、⽔}
  \begin{Phonetics}{方法}{fang1fa3}[][HSK 2]
    \definition[种,个,套,类]{s.}{método; meio; maneira; sobre os meios e procedimentos para resolver questões relacionadas com o pensamento, a fala e as ações, etc.}
  \end{Phonetics}
\end{Entry}

\begin{Entry}{方便}{4,9}{⽅、⼈}
  \begin{Phonetics}{方便}{fang1bian4}[][HSK 2]
    \definition{adj.}{conveniente; sem complicações; sem dificuldades; muito fácil| adequado; condições ou circunstâncias adequadas}
    \definition{s.}{conveniência}
    \definition{v.}{ir ao banheiro; uma maneira delicada de dizer ``ir ao banheiro'' | facilitar; tornar algo conveniente para alguém; facilitar a realização de tarefas ou o alcance de objetivos | ter dinheiro sobrando}
  \end{Phonetics}
\end{Entry}

\begin{Entry}{方便面}{4,9,9}{⽅、⼈、⾯}
  \begin{Phonetics}{方便面}{fang1 bian4 mian4}[][HSK 2]
    \definition[袋,包,碗,桶]{s.}{macarrão instantâneo}
  \end{Phonetics}
\end{Entry}

\begin{Entry}{方面}{4,9}{⽅、⾯}
  \begin{Phonetics}{方面}{fang1mian4}[][HSK 2]
    \definition[个,种]{s.}{lado; campo; aspecto; respeito}
  \end{Phonetics}
\end{Entry}

\begin{Entry}{方案}{4,10}{⽅、⽊}
  \begin{Phonetics}{方案}{fang1'an4}[][HSK 4]
    \definition[个,些,种]{s.}{plano; esquema; programa; planos específicos para tratar de um determinado problema | o esquema criado pelo governo; medidas ou regulamentações formuladas e implementadas pelo governo ou autoridades relevantes}
  \end{Phonetics}
\end{Entry}

\begin{Entry}{无}{4}{⽆}[Kangxi 71]
  \begin{Phonetics}{无}{wu2}[][HSK 4]
    \definition{adv.}{não (em posição a 有); não ter algo; não há\dots}
    \definition{conj.}{independentemente de; não importa se, o que, etc.}
    \definition{v.}{não ter; estar sem; não existir}
  \seealsoref{有}{you3}
  \end{Phonetics}
\end{Entry}

\begin{Entry}{无人}{4,2}{⽆、⼈}
  \begin{Phonetics}{无人}{wu2ren2}
    \definition{adj.}{não tripulado | desabitado}
  \end{Phonetics}
\end{Entry}

\begin{Entry}{无人机}{4,2,6}{⽆、⼈、⽊}
  \begin{Phonetics}{无人机}{wu2ren2ji1}
    \definition{s.}{\emph{drone} | veículo aéreo não tripulado}
  \end{Phonetics}
\end{Entry}

\begin{Entry}{无边}{4,5}{⽆、⾡}
  \begin{Phonetics}{无边}{wu2 bian1}[][HSK 6]
    \definition{adj.}{ilimitado; vasto; sem limites; sem abas; sem bordas}
  \end{Phonetics}
\end{Entry}

\begin{Entry}{无关}{4,6}{⽆、⼋}
  \begin{Phonetics}{无关}{wu2 guan1}[][HSK 6]
    \definition{v.}{não ter nada a ver com; nada a fazer | não envolver; ser irrelevante; não ter efeito sobre}
  \end{Phonetics}
\end{Entry}

\begin{Entry}{无论}{4,6}{⽆、⾔}
  \begin{Phonetics}{无论}{wu2lun4}[][HSK 4]
    \definition{conj.}{não importa o quê; não importa como; independentemente de; indica que as condições são diferentes, mas resultado é o mesmo}
  \seealsoref{无论……也……}{wu2lun4 ye3}
  \end{Phonetics}
\end{Entry}

\begin{Entry}{无论……也……}{4,6,3}{⽆、⾔、⼄}
  \begin{Phonetics}{无论……也……}{wu2lun4 ye3}
    \definition{conj.}{não apenas\dots, (o que, quem, como, etc.), \dots}
  \end{Phonetics}
\end{Entry}

\begin{Entry}{无奈}{4,8}{⽆、⼤}
  \begin{Phonetics}{无奈}{wu2nai4}[][HSK 5]
    \definition{conj.}{mas (infelizmente); no entanto}
    \definition{v.}{não poder evitar; não ter alternativa; não ter escolha; não haver nada a fazer}
  \end{Phonetics}
\end{Entry}

\begin{Entry}{无所谓}{4,8,11}{⽆、⼾、⾔}
  \begin{Phonetics}{无所谓}{wu2suo3wei4}[][HSK 4]
    \definition{v.}{não pode ser designado como; não merece o nome de; ser incapaz de dizer ou contar | não ter importância; ser indiferente}
  \end{Phonetics}
\end{Entry}

\begin{Entry}{无法}{4,8}{⽆、⽔}
  \begin{Phonetics}{无法}{wu2 fa3}[][HSK 4]
    \definition{adj.}{incapaz; incapacitado; não tem jeito}
  \end{Phonetics}
\end{Entry}

\begin{Entry}{无视}{4,8}{⽆、⾒}
  \begin{Phonetics}{无视}{wu2shi4}
    \definition{v.}{ignorar | desconsiderar}
  \end{Phonetics}
\end{Entry}

\begin{Entry}{无限}{4,8}{⽆、⾩}
  \begin{Phonetics}{无限}{wu2 xian4}[][HSK 4]
    \definition{adj.}{infinito; ilimitado; sem limites; sem fim à vista}
  \end{Phonetics}
\end{Entry}

\begin{Entry}{无非}{4,8}{⽆、⾮}
  \begin{Phonetics}{无非}{wu2fei1}
    \definition{adv.}{somente; simplesmente; nada além de; não mais que; nada mais do que; significando tudo dentro de um certo intervalo}
  \end{Phonetics}
\end{Entry}

\begin{Entry}{无故}{4,9}{⽆、⽁}
  \begin{Phonetics}{无故}{wu2gu4}
    \definition{adv.}{sem causa (ou razão); sem razão alguma}
  \end{Phonetics}
\end{Entry}

\begin{Entry}{无骨}{4,9}{⽆、⾻}
  \begin{Phonetics}{无骨}{wu2 gu3}
    \definition{adj.}{desossado}
  \end{Phonetics}
\end{Entry}

\begin{Entry}{无效}{4,10}{⽆、⽁}
  \begin{Phonetics}{无效}{wu2 xiao4}[][HSK 6]
    \definition{adj.}{sem efeito; de (ou sem) utilidade; em vão; (uma certa abordagem) é inútil e não pode resolver o problema ou atingir o objetivo | inválido; nulo e sem efeito; (um determinado comportamento) não é reconhecido por lei e os direitos relevantes não serão protegidos}
  \end{Phonetics}
\end{Entry}

\begin{Entry}{无敌}{4,10}{⽆、⾆}
  \begin{Phonetics}{无敌}{wu2di2}
    \definition{adj.}{invencível | inigualável}
  \end{Phonetics}
\end{Entry}

\begin{Entry}{无氧}{4,10}{⽆、⽓}
  \begin{Phonetics}{无氧}{wu2yang3}
    \definition{adj.}{anaeróbico}
  \end{Phonetics}
\end{Entry}

\begin{Entry}{无聊}{4,11}{⽆、⽿}
  \begin{Phonetics}{无聊}{wu2liao2}[][HSK 4]
    \definition{adj.}{entediado; aborrecido; sentir-se desinteressado porque não há nada para fazer | tolo; bobo; sem sentido; descreve palavras ou coisas ditas ou feitas como sem sentido e irritantes; descreve pessoas ou coisas como sem sentido e pouco atraentes}
  \end{Phonetics}
\end{Entry}

\begin{Entry}{无数}{4,13}{⽆、⽁}
  \begin{Phonetics}{无数}{wu2shu4}[][HSK 4]
    \definition{adj.}{incontável; inumerável | inseguro; incerto; não conhecer a história ou os detalhes internos; não ter certeza}
  \end{Phonetics}
\end{Entry}

\begin{Entry}{无疑}{4,14}{⽆、⽦}
  \begin{Phonetics}{无疑}{wu2 yi2}[][HSK 5]
    \definition{adv.}{indubitavelmente; sem dúvida; sem sombra de dúvida}
  \end{Phonetics}
\end{Entry}

\begin{Entry}{日}{4}{⽇}[Kangxi 72]
  \begin{Phonetics}{日}{ri4}[][HSK 1]
    \definition*{s.}{Japão, abreviação de 日本}
    \definition{clas.}{usado para contar o número de dias}
    \definition{s.}{sol | dia (em oposição a 夜); período diurno | diariamente; todos os dias; a cada dia que passa | um dia específico; dia especial | tempo; refere-se a um período de tempo | dia; uma rotação da Terra}
  \seealsoref{日本}{ri4ben3}
  \seealsoref{夜}{ye4}
  \end{Phonetics}
\end{Entry}

\begin{Entry}{日子}{4,3}{⽇、⼦}
  \begin{Phonetics}{日子}{ri4zi5}[][HSK 2]
    \definition[个,段,些,番]{s.}{dia; data; referência a uma data específica | dias; tempo; referência ao número de dias e horas | vida; subsistência; refere-se à vida ou ao sustento}
  \end{Phonetics}
\end{Entry}

\begin{Entry}{日历}{4,4}{⽇、⼚}
  \begin{Phonetics}{日历}{ri4li4}[][HSK 4]
    \definition[张,本]{s.}{caledário; livro com o ano, mês, dia, semana, termo solar, aniversário, etc. registrados, um livro por ano, uma página por dia, aberto diariamente}
  \end{Phonetics}
\end{Entry}

\begin{Entry}{日心说}{4,4,9}{⽇、⼼、⾔}
  \begin{Phonetics}{日心说}{ri4 xin1 shuo1}
    \definition{s.}{teoria heliocêntrica | a teoria de que o sol está no centro do universo}
  \end{Phonetics}
\end{Entry}

\begin{Entry}{日出}{4,5}{⽇、⼐}
  \begin{Phonetics}{日出}{ri4chu1}
    \definition{s.}{nascer do sol}
  \seealsoref{夕阳}{xi1yang2}
  \end{Phonetics}
\end{Entry}

\begin{Entry}{日本}{4,5}{⽇、⽊}
  \begin{Phonetics}{日本}{ri4ben3}
    \definition*{s.}{Japão}
  \end{Phonetics}
\end{Entry}

\begin{Entry}{日本人}{4,5,2}{⽇、⽊、⼈}
  \begin{Phonetics}{日本人}{ri4ben3ren2}
    \definition{s.}{japonês | pessoa ou povo do Japão}
  \end{Phonetics}
\end{Entry}

\begin{Entry}{日记}{4,5}{⽇、⾔}
  \begin{Phonetics}{日记}{ri4ji4}[][HSK 4]
    \definition[本,篇,册]{s.}{diário; artigo que registra eventos e pensamentos diários}
  \end{Phonetics}
\end{Entry}

\begin{Entry}{日光灯}{4,6,6}{⽇、⼉、⽕}
  \begin{Phonetics}{日光灯}{ri4guang1deng1}
    \definition{s.}{lâmpada fluorescente}
  \end{Phonetics}
\end{Entry}

\begin{Entry}{日报}{4,7}{⽇、⼿}
  \begin{Phonetics}{日报}{ri4 bao4}[][HSK 2]
    \definition[份,种]{s.}{diário; jornais diários; jornal publicado todas as manhãs}
  \end{Phonetics}
\end{Entry}

\begin{Entry}{日夜}{4,8}{⽇、⼣}
  \begin{Phonetics}{日夜}{ri4 ye4}[][HSK 6]
    \definition{s.}{dia e noite; noite e dia; 24 horas por dia}
  \end{Phonetics}
\end{Entry}

\begin{Entry}{日语}{4,9}{⽇、⾔}
  \begin{Phonetics}{日语}{ri4 yu3}[][HSK 6]
    \definition{s.}{japonês; língua japonesa}
  \end{Phonetics}
\end{Entry}

\begin{Entry}{日常}{4,11}{⽇、⼱}
  \begin{Phonetics}{日常}{ri4chang2}[][HSK 3]
    \definition{adj.}{usual; diário; cotidiano; dia a dia; pertencem ao habitual}
  \end{Phonetics}
\end{Entry}

\begin{Entry}{日期}{4,12}{⽇、⽉}
  \begin{Phonetics}{日期}{ri4qi1}[][HSK 1]
    \definition[个,段]{s.}{data; a data ou período específico em que algo aconteceu}
  \end{Phonetics}
\end{Entry}

\begin{Entry}{月}{4}{⽉}[Kangxi 74]
  \begin{Phonetics}{月}{yue4}[][HSK 1]
    \definition*{s.}{Sobrenome Yue}
    \definition[个]{s.}{mês | por mês | lua | redondo; em forma de lua cheia}
  \end{Phonetics}
\end{Entry}

\begin{Entry}{月月}{4,4}{⽉、⽉}
  \begin{Phonetics}{月月}{yue4yue4}
    \definition{adv.}{todo mês}
  \end{Phonetics}
\end{Entry}

\begin{Entry}{月份}{4,6}{⽉、⼈}
  \begin{Phonetics}{月份}{yue4 fen4}[][HSK 2]
    \definition[个]{s.}{mês; refere-se a um determinado mês}
  \end{Phonetics}
\end{Entry}

\begin{Entry}{月底}{4,8}{⽉、⼴}
  \begin{Phonetics}{月底}{yue4 di3}[][HSK 4]
    \definition[个]{s.}{final do mês; últimos dias do mês}
  \end{Phonetics}
\end{Entry}

\begin{Entry}{月径}{4,8}{⽉、⼻}
  \begin{Phonetics}{月径}{yue4jing4}
    \definition{s.}{diâmetro da lua | diâmetro da órbita da lua | caminho iluminado pela lua}
  \end{Phonetics}
\end{Entry}

\begin{Entry}{月亮}{4,9}{⽉、⼇}
  \begin{Phonetics}{月亮}{yue4liang5}[][HSK 2]
    \definition[个,轮,挂,快,页]{s.}{Lua; Lua é o nome comum do satélite da Terra}
  \end{Phonetics}
\end{Entry}

\begin{Entry}{月相}{4,9}{⽉、⽬}
  \begin{Phonetics}{月相}{yue4xiang4}
    \definition{s.}{fases da lua, a saber: lua nova 朔, lua crescente 上弦, lua cheia 望 e lua minguante 下弦}
  \end{Phonetics}
\end{Entry}

\begin{Entry}{月饼}{4,9}{⽉、⾷}
  \begin{Phonetics}{月饼}{yue4 bing3}[][HSK 5]
    \definition[个,块,盒,筒]{s.}{bolinho da lua; comida típica do Festival do Meio Outono; redonda e recheada; simboliza a reunião familiar}
  \end{Phonetics}
\end{Entry}

\begin{Entry}{月球}{4,11}{⽉、⽟}
  \begin{Phonetics}{月球}{yue4 qiu2}[][HSK 5]
    \definition*[颗,个]{s.}{Lua}
    \definition{pref.}{seleno-; seleni-}
  \end{Phonetics}
\end{Entry}

\begin{Entry}{月壤}{4,20}{⽉、⼟}
  \begin{Phonetics}{月壤}{yue4rang3}
    \definition{s.}{solo lunar}
  \end{Phonetics}
\end{Entry}

\begin{Entry}{木}{4}{⽊}[Kangxi 75]
  \begin{Phonetics}{木}{mu4}
    \definition{adj.}{de madeira; feito de madeira | estúpido; de raciocínio lento; atordoado; lento para reagir | simplório; chato | entorpecido; de madeira; dormência localizada ou perda de sensibilidade}
    \definition{s.}{árvore | madeira; madeiramento | caixão}
  \end{Phonetics}
\end{Entry}

\begin{Entry}{木头}{4,5}{⽊、⼤}
  \begin{Phonetics}{木头}{mu4tou5}[][HSK 3]
    \definition[根,块,堆,截]{s.}{tronco; madeira; lenha; denominação genérica para madeira e materiais de madeira}
  \end{Phonetics}
\end{Entry}

\begin{Entry}{木偶}{4,11}{⽊、⼈}
  \begin{Phonetics}{木偶}{mu4'ou3}
    \definition{s.}{fantoche, marionete}
  \end{Phonetics}
\end{Entry}

\begin{Entry}{欠}{4}{⽋}[Kangxi 76]
  \begin{Phonetics}{欠}{qian4}[][HSK 5]
    \definition{v.}{bocejar | levantar ligeiramente (uma parte do corpo) | estar em dívida; estar atrasado com; não devolver o que pediu emprestado a outra pessoa, ou não dar o que deveria ter dado a outra pessoa | faltar; não ser suficiente}
  \end{Phonetics}
\end{Entry}

\begin{Entry}{止}{4}{⽌}[Kangxi 77]
  \begin{Phonetics}{止}{zhi3}[][HSK 6]
    \definition{adv.}{somente; apenas}
    \definition{v.}{parar; cortar; bloquear | finalizar; fechar (até o prazo final) | não ser permitido}
  \end{Phonetics}
\end{Entry}

\begin{Entry}{歹}{4}{⽍}[Kangxi 78]
  \begin{Phonetics}{歹}{dai3}
    \definition{adj.}{maligno; cruel; ruim, refere-se principalmente a pessoas e coisas}
  \end{Phonetics}
\end{Entry}

\begin{Entry}{歹徒}{4,10}{⽍、⼻}
  \begin{Phonetics}{歹徒}{dai3tu2}[][HSK 7-9]
    \definition[个,伙,群]{s.}{rufião; malfeitor; arruaceiro; canalha}
  \end{Phonetics}
\end{Entry}

\begin{Entry}{比}{4}{⽐}[Kangxi 81]
  \begin{Phonetics}{比}{bi3}[][HSK 1]
    \definition*{s.}{Abreviação de Bélgica, 比利时}
    \definition{adj.}{específico}
    \definition{adv.}{recentemente}
    \definition{part.}{partícula usada para comparação (superioridade)}
    \definition{prep.}{que; do que | (seguido por um substantivo e adjetivo) mais \{adj.\} do que \{s.\}}
    \definition{s.}{razão; proporção | contraste; comparação | metáfora em poesia; técnica de composição poética}
    \definition{v.}{estar ao lado de; estar próximo a | igualar; comparar; competir; contrastar; emular; comparar superioridade, inferioridade, comprimento, distância, qualidade, etc. | assemelhar-se a; comparar com; fazer uma analogia | gesticular; fazer gestos | ser treinado em; ser direcionado a | copiar; imitar | poder ser comparado | apegar-se a; depender}
  \seealsoref{比利时}{bi3li4shi2}
  \end{Phonetics}
\end{Entry}

\begin{Entry}{比不上}{4,4,3}{⽐、⼀、⼀}
  \begin{Phonetics}{比不上}{bi3 bu5 shang4}[][HSK 7-9]
    \definition{expr.}{não se compara a; uma coisa não pode ser comparada com outra, não pode atingir o mesmo nível}
  \end{Phonetics}
\end{Entry}

\begin{Entry}{比分}{4,4}{⽐、⼑}
  \begin{Phonetics}{比分}{bi3 fen1}[][HSK 4]
    \definition{s.}{pontuação; comparação de pontuações entre as duas equipes em uma partida}
  \end{Phonetics}
\end{Entry}

\begin{Entry}{比方}{4,4}{⽐、⽅}
  \begin{Phonetics}{比方}{bi3fang1}[][HSK 5]
    \definition{conj.}{se; suponha que; expressa uma hipótese, equivalente a 如果 (com eufemismos)}
    \definition{s.}{analogia; exemplo; instância; expressão que usa uma coisa para descrever outra (expressão idiomática); (figurativo) usar uma coisa para descrever outra}
    \definition{v.}{ilustrar; exemplificar; fazer uma analogia; usar uma coisa para descrever outra (expressão idiomática)}
  \seealsoref{如果}{ru2guo3}
  \end{Phonetics}
\end{Entry}

\begin{Entry}{比比皆是}{4,4,9,9}{⽐、⽐、⽩、⽇}
  \begin{Phonetics}{比比皆是}{bi3bi3-jie1shi4}[][HSK 7-9]
    \definition{suf.}{pode ser encontrado em todos os lugares; ao redor | encontrar o olhar em todos os lugares; ser grande em número; ser visto em todos os lugares -- em abundância; um após o outro}
  \end{Phonetics}
\end{Entry}

\begin{Entry}{比亚迪}{4,6,8}{⽐、⼆、⾡}
  \begin{Phonetics}{比亚迪}{bi3ya4di2}
    \definition*{s.}{Montadora BYD}
  \end{Phonetics}
\end{Entry}

\begin{Entry}{比如}{4,6}{⽐、⼥}
  \begin{Phonetics}{比如}{bi3ru2}[][HSK 2]
    \definition{conj.}{por exemplo; tal como; suponha; digamos; a seguir, apresentamos alguns exemplos; na linguagem coloquial, também se pode dizer 比如说}
  \seealsoref{比如说}{bi3 ru2 shuo1}
  \end{Phonetics}
\end{Entry}

\begin{Entry}{比如说}{4,6,9}{⽐、⼥、⾔}
  \begin{Phonetics}{比如说}{bi3 ru2 shuo1}[][HSK 2]
    \definition{adv.}{por exemplo}
  \seealsoref{比如}{bi3ru2}
  \end{Phonetics}
\end{Entry}

\begin{Entry}{比利时}{4,7,7}{⽐、⼑、⽇}
  \begin{Phonetics}{比利时}{bi3li4shi2}
    \definition*{s.}{Bélgica}
  \end{Phonetics}
\end{Entry}

\begin{Entry}{比例}{4,8}{⽐、⼈}
  \begin{Phonetics}{比例}{bi3li4}[][HSK 3]
    \definition{s.}{escala; razão; relação de múltiplos entre dois números | porporção; a quantidade que uma parte representa no todo | proporção; na descrição do grau de coordenação das coisas}
  \end{Phonetics}
\end{Entry}

\begin{Entry}{比试}{4,8}{⽐、⾔}
  \begin{Phonetics}{比试}{bi3shi4}[][HSK 7-9]
    \definition{v.}{competir; ter uma competição; contestar | medir com a mão ou o braço; fazer um gesto de medição}
  \end{Phonetics}
\end{Entry}

\begin{Entry}{比拼}{4,9}{⽐、⼿}
  \begin{Phonetics}{比拼}{bi3pin1}
    \definition{s.}{concurso}
    \definition{v.}{competir ferozmente}
  \end{Phonetics}
\end{Entry}

\begin{Entry}{比重}{4,9}{⽐、⾥}
  \begin{Phonetics}{比重}{bi3zhong4}[][HSK 5]
    \definition{s.}{proporção; o peso da parte em relação ao todo | Física: densidade específica; a relação entre o peso de um objeto e seu volume}
  \end{Phonetics}
\end{Entry}

\begin{Entry}{比起}{4,10}{⽐、⾛}
  \begin{Phonetics}{比起}{bi3qi3}[][HSK 7-9]
    \definition{adv./prep.}{comparado com}
  \end{Phonetics}
\end{Entry}

\begin{Entry}{比较}{4,10}{⽐、⾞}
  \begin{Phonetics}{比较}{bi3jiao4}[][HSK 3]
    \definition{adv.}{razoavelmente; relativamente; bastante; um pouco; comparativamente; indica um certo grau, com o significado de 相当}
    \definition{prep.}{usado para comparar uma diferença de grau. para distinguir as diferenças ou superioridades entre duas ou mais coisas semelhantes}
    \definition{v.}{comparar; contrastar; usado para comparar diferenças em propriedades e graus; para distinguir semelhanças, diferenças ou superioridade entre duas ou mais coisas semelhantes}
  \seealsoref{相当}{xiang1dang1}
  \end{Phonetics}
\end{Entry}

\begin{Entry}{比萨饼}{4,11,9}{⽐、⾋、⾷}
  \begin{Phonetics}{比萨饼}{bi3sa4bing3}
    \definition[张]{s.}{pizza}
  \end{Phonetics}
\end{Entry}

\begin{Entry}{比喻}{4,12}{⽐、⼝}
  \begin{Phonetics}{比喻}{bi3yu4}[][HSK 7-9]
    \definition[个,种]{s.}{analogia; metáfora; um método ou exemplo de uso de uma imagem ou coisa concreta para ilustrar outra coisa}
    \definition{v.}{fazer uma analogia; usar uma metáfora; usar algo semelhante para comparar com algo que você quer dizer}
  \end{Phonetics}
\end{Entry}

\begin{Entry}{比赛}{4,14}{⽐、⾙}
  \begin{Phonetics}{比赛}{bi3sai4}[][HSK 3]
    \definition[场,次,轮,站,个]{s.}{competição; atividades da competição}
    \definition{v.}{competir; disputar; comparar o nível e a qualidade das habilidades e competências}
  \end{Phonetics}
\end{Entry}

\begin{Entry}{毛}{4}{⽑}[Kangxi 82]
  \begin{Phonetics}{毛}{mao2}[][HSK 1,3]
    \definition*{s.}{Sobrenome Mao}
    \definition{adj.}{bruto; semiacabado | grosseiro | pequeno | fino | descuidado; rude; precipitado | assustado; nervoso; em pânico | impetuoso | rústico; sem acabamento | impuro | (de moeda) que não vale mais seu valor nominal; depreciado}
    \definition{clas.}{mao, uma unidade fracionária de dinheiro na China; dez centavos; uma peça de dez centavos}
    \definition[根]{s.}{(de um animal, planta, etc.) cabelo; pena; penugem | (de humanos) cabelo; barba | planta; colheita | lã | mofo; bolor}
    \definition{v.}{depreciar; desvalorizar; refere-se à desvalorização da moeda | (de cavalos, gado, etc.) assustar-se; sentir medo}
  \end{Phonetics}
\end{Entry}

\begin{Entry}{毛巾}{4,3}{⽑、⼱}
  \begin{Phonetics}{毛巾}{mao2jin1}[][HSK 4]
    \definition[条,块]{s.}{toalha; toalha de banho}
  \end{Phonetics}
\end{Entry}

\begin{Entry}{毛衣}{4,6}{⽑、⾐}
  \begin{Phonetics}{毛衣}{mao2 yi1}[][HSK 4]
    \definition[件,个]{s.}{suéter; blusa feita de lã}
  \end{Phonetics}
\end{Entry}

\begin{Entry}{毛病}{4,10}{⽑、⽧}
  \begin{Phonetics}{毛病}{mao2bing4}[][HSK 3]
    \definition[个,点,种,些]{s.}{doença ou deficiência; condição de saúde precária ou deficiência física | problema; fracasso; onde o produto está com defeito ou não funciona corretamente | mau hábito; deficiência; falhas no comportamento humano}
  \end{Phonetics}
\end{Entry}

\begin{Entry}{毛笔}{4,10}{⽑、⽵}
  \begin{Phonetics}{毛笔}{mao2 bi3}[][HSK 5]
    \definition[支,枝,根,管]{s.}{pincel para escrever; pincel chinês; canetas feitas com pelos de coelho, carneiro, doninha, etc., são materiais tradicionais utilizados para escrever caracteres chineses e pintar pinturas tradicionais chinesas}
  \end{Phonetics}
\end{Entry}

\begin{Entry}{气}{4}{⽓}[Kangxi 84]
  \begin{Phonetics}{气}{qi4}[][HSK 2]
    \definition*{s.}{Sobrenome Qi}
    \definition[口]{s.}{gás; gás em geral | ar; especificamente, o ar | respiração | clima; refere-se a fenômenos naturais como sol, chuva, frio e calor | cheiro; odor; o cheiro que o nariz sente | ânimo; moral; estado mental | ares; estilo; maneiras; refere-se ao estilo e aos hábitos de uma pessoa | raiva; irritação; aborrecimento; sentimento de irritação | energia vital; energia da vida; na medicina tradicional chinesa refere-se às substâncias sutis que circulam no corpo humano e permitem que os vários órgãos funcionem normalmente | certos sintomas (de doenças); na medicina tradicional chinesa refere-se a um determinado quadro clínico}
    \definition{v.}{ficar com raiva; ficar furioso; ficar irritado | irritar; enfurecer; deixar com raiva | ser intimidado; sofrer injustiça; intimidar}
  \end{Phonetics}
\end{Entry}

\begin{Entry}{气体}{4,7}{⽓、⼈}
  \begin{Phonetics}{气体}{qi4 ti3}[][HSK 5]
    \definition[种,瓶,升]{s.}{gás; não têm forma nem volume definidos e podem fluir.; o ar, o oxigênio, o gás metano e outros são gases}
  \end{Phonetics}
\end{Entry}

\begin{Entry}{气氛}{4,8}{⽓、⽓}
  \begin{Phonetics}{气氛}{qi4fen1}[][HSK 6]
    \definition{s.}{atmosfera; sensação circundante; uma certa emoção ou cena que existe em um determinado ambiente e pode fazer as pessoas sentirem}
  \end{Phonetics}
\end{Entry}

\begin{Entry}{气质}{4,8}{⽓、⾙}
  \begin{Phonetics}{气质}{qi4zhi4}
    \definition{s.}{traços de personalidade, temperamento, disposição | aura, ar, sentimento, \emph{vibe} | refinamento, sofisticação, classe}
  \end{Phonetics}
\end{Entry}

\begin{Entry}{气候}{4,10}{⽓、⼈}
  \begin{Phonetics}{气候}{qi4hou4}[][HSK 3]
    \definition[种]{s.}{clima; tempo; condições meteorológicas gerais obtidas após muitos anos de observação em uma determinada região, estão relacionadas com correntes de ar, latitude, altitude acima do nível do mar, relevo, etc. | tendência; situação; metáfora do ambiente social, de uma determinada tendência | resultado; influência; conquista; realização; metáfora para algum tipo de resultado, conquista, influência significativa ou potencial de desenvolvimento}
  \end{Phonetics}
\end{Entry}

\begin{Entry}{气球}{4,11}{⽓、⽟}
  \begin{Phonetics}{气球}{qi4qiu2}[][HSK 4]
    \definition[个,只]{s.}{balão; bolas feitas de borracha, plástico, etc., que podem ser aumentadas soprando ar nelas e podem ser usadas como brinquedos, decorações ou meios de transporte}
  \end{Phonetics}
\end{Entry}

\begin{Entry}{气象}{4,11}{⽓、⾗}
  \begin{Phonetics}{气象}{qi4xiang4}[][HSK 5]
    \definition[种,派]{s.}{fenômenos meteorológicos; condições e fenômenos atmosféricos, como vento, relâmpagos, trovões, geadas, neve, etc. | meteorologia | situação; atmosfera; cena; circunstância | maneira imponente}
  \end{Phonetics}
\end{Entry}

\begin{Entry}{气温}{4,12}{⽓、⽔}
  \begin{Phonetics}{气温}{qi4 wen1}[][HSK 2]
    \definition[个]{s.}{temperatura do ar}
  \end{Phonetics}
\end{Entry}

\begin{Entry}{水}{4}{⽔}[Kangxi 85]
  \begin{Phonetics}{水}{shui3}[][HSK 1]
    \definition*{s.}{Etnia Shui, que vive principalmente em Guizhou | Sobrenome Shui}
    \definition{adj.}{de má qualidade; mal feito; de baixa qualidade e conteúdo}
    \definition{clas.}{usado para número de lavagens}
    \definition[条,杯]{s.}{água | rio | termo geral para rios, lagos, mares, etc.; água | corrente; fluxo de água | um líquido; suco ralo | teor de prata nas moedas | encargos adicionais ou receitas | água, um dos cinco elementos}
  \end{Phonetics}
\end{Entry}

\begin{Entry}{水分}{4,4}{⽔、⼑}
  \begin{Phonetics}{水分}{shui3 fen4}[][HSK 5]
    \definition{s.}{teor de umidade; água contida em um objeto | exagero; metáfora de algo falso}
  \end{Phonetics}
\end{Entry}

\begin{Entry}{水平}{4,5}{⽔、⼲}
  \begin{Phonetics}{水平}{shui3ping2}[][HSK 2]
    \definition{adj.}{horizontal; nivelado; paralelo à superfície da água}
    \definition{s.}{padrão; nível; o nível alcançado em determinado aspecto}
  \end{Phonetics}
\end{Entry}

\begin{Entry}{水平以下}{4,5,4,3}{⽔、⼲、⼈、⼀}
  \begin{Phonetics}{水平以下}{shui3ping2 yi3xia4}
    \definition{s.}{sub-nível}
  \end{Phonetics}
\end{Entry}

\begin{Entry}{水平尺}{4,5,4}{⽔、⼲、⼫}
  \begin{Phonetics}{水平尺}{shui3ping2chi3}
    \definition{s.}{nível espiritual}
  \end{Phonetics}
\end{Entry}

\begin{Entry}{水平仪}{4,5,5}{⽔、⼲、⼈}
  \begin{Phonetics}{水平仪}{shui3ping2yi2}
    \definition{s.}{nível (dispositivo para determinar horizontal) | nível espiritual | nível de topógrafo}
  \end{Phonetics}
\end{Entry}

\begin{Entry}{水平视差}{4,5,8,9}{⽔、⼲、⾒、⼯}
  \begin{Phonetics}{水平视差}{shui3ping2 shi4cha1}
    \definition{s.}{paralaxe horizontal}
  \end{Phonetics}
\end{Entry}

\begin{Entry}{水平度}{4,5,9}{⽔、⼲、⼴}
  \begin{Phonetics}{水平度}{shui3ping2 du4}
    \definition{s.}{nivelamento}
  \end{Phonetics}
\end{Entry}

\begin{Entry}{水平轴}{4,5,9}{⽔、⼲、⾞}
  \begin{Phonetics}{水平轴}{shui3ping2zhou2}
    \definition{s.}{eixo horizontal}
  \end{Phonetics}
\end{Entry}

\begin{Entry}{水平面}{4,5,9}{⽔、⼲、⾯}
  \begin{Phonetics}{水平面}{shui3ping2mian4}
    \definition{s.}{plano horizontal | nível-da-água | superfície horizontal}
  \end{Phonetics}
\end{Entry}

\begin{Entry}{水边}{4,5}{⽔、⾡}
  \begin{Phonetics}{水边}{shui3bian1}
    \definition{s.}{beira d'água | beira-mar | costa (de mar, lago ou rio)}
  \end{Phonetics}
\end{Entry}

\begin{Entry}{水产品}{4,6,9}{⽔、⼇、⼝}
  \begin{Phonetics}{水产品}{shui3 chan3 pin3}[][HSK 5]
    \definition{s.}{produto aquático (peixes, camarões, etc.)}
  \end{Phonetics}
\end{Entry}

\begin{Entry}{水污染}{4,6,9}{⽔、⽔、⽊}
  \begin{Phonetics}{水污染}{shui3wu1ran3}
    \definition{s.}{poluição da água}
  \end{Phonetics}
\end{Entry}

\begin{Entry}{水库}{4,7}{⽔、⼴}
  \begin{Phonetics}{水库}{shui3 ku4}[][HSK 5]
    \definition[座]{s.}{reservatório; lago artificial construído pelo homem, que utiliza barragens e outras estruturas para represar a água e regular o fluxo, podendo ser utilizado para armazenamento de água, geração de energia e piscicultura, entre outros fins}
  \end{Phonetics}
\end{Entry}

\begin{Entry}{水灵}{4,7}{⽔、⽕}
  \begin{Phonetics}{水灵}{shui3ling2}
    \definition{adj.}{cheio de vida (sobre uma pessoa, etc.) | úmido e brilhante (sobre os olhos) | fresco (sobre frutas, etc.) | brilhante | aparência saudável}
  \end{Phonetics}
\end{Entry}

\begin{Entry}{水灾}{4,7}{⽔、⽕}
  \begin{Phonetics}{水灾}{shui3zai1}[][HSK 5]
    \definition[场,次]{s.}{inundação; desastres causados por excesso de chuvas, entre outros motivos}
  \end{Phonetics}
\end{Entry}

\begin{Entry}{水果}{4,8}{⽔、⽊}
  \begin{Phonetics}{水果}{shui3guo3}[][HSK 1]
    \definition[个]{s.}{fruta; um nome genérico para frutas com alto teor de água que podem ser consumidas, como peras, pêssegos, maçãs, etc.}
  \end{Phonetics}
\end{Entry}

\begin{Entry}{水波}{4,8}{⽔、⽔}
  \begin{Phonetics}{水波}{shui3bo1}
    \definition{s.}{ondulação (na água) | onda}
  \end{Phonetics}
\end{Entry}

\begin{Entry}{水泥}{4,8}{⽔、⽔}
  \begin{Phonetics}{水泥}{shui3ni2}[][HSK 6]
    \definition[袋,层]{s.}{cimento; um tipo de material mineral em pó que pode endurecer gradualmente no ar e na água após a mistura com água}
  \end{Phonetics}
\end{Entry}

\begin{Entry}{水饺}{4,9}{⽔、⾷}
  \begin{Phonetics}{水饺}{shui3jiao3}
    \definition{s.}{\emph{dumplings} | pastéis chineses cozidos}
  \end{Phonetics}
\end{Entry}

\begin{Entry}{水瓶}{4,10}{⽔、⽡}
  \begin{Phonetics}{水瓶}{shui3 ping2}
    \definition{s.}{garrada de água}
  \end{Phonetics}
\end{Entry}

\begin{Entry}{水培}{4,11}{⽔、⼟}
  \begin{Phonetics}{水培}{shui3pei2}
    \definition{v.}{cultivar plantas hidroponicamente}
  \end{Phonetics}
\end{Entry}

\begin{Entry}{水豚}{4,11}{⽔、⾗}
  \begin{Phonetics}{水豚}{shui3tun2}
    \definition{s.}{capivara}
  \end{Phonetics}
\end{Entry}

\begin{Entry}{水路}{4,13}{⽔、⾜}
  \begin{Phonetics}{水路}{shui3lu4}
    \definition{s.}{hidrovia}
  \end{Phonetics}
\end{Entry}

\begin{Entry}{水槽}{4,15}{⽔、⽊}
  \begin{Phonetics}{水槽}{shui3cao2}
    \definition{s.}{pia (de cozinha)}
  \end{Phonetics}
\end{Entry}

\begin{Entry}{火}{4}{⽕}[Kangxi 86]
  \begin{Phonetics}{火}{huo3}[][HSK 3,4]
    \definition*{s.}{Sobrenome Huo}
    \definition{adj.}{ardente; flamejante; vermelho como fogo | efervescente; próspero}
    \definition{adv.}{urgentemente}
    \definition{clas.}{usado para unidades militares (antigo)}
    \definition[场,把,团,堆]{s.}{fogo; a luz e as chamas emitidas pela combustão de um objeto | fúria; metáfora para emoções agitadas, irritadas ou raivosas | calor interno (uma das seis causas de doenças) | armas de fogo e munições | a ação de lutar}
    \definition{v.}{ficar com raiva; perder a paciência}
  \end{Phonetics}
\end{Entry}

\begin{Entry}{火山}{4,3}{⽕、⼭}
  \begin{Phonetics}{火山}{huo3shan1}[][HSK 7-9]
    \definition[座]{s.}{vulcão}
  \end{Phonetics}
\end{Entry}

\begin{Entry}{火车}{4,4}{⽕、⾞}
  \begin{Phonetics}{火车}{huo3 che1}[][HSK 1]
    \definition[个,列,节,班,趟]{s.}{trem; comboio}
  \end{Phonetics}
\end{Entry}

\begin{Entry}{火车司机}{4,4,5,6}{⽕、⾞、⼝、⽊}
  \begin{Phonetics}{火车司机}{huo3che1 si1ji1}
    \definition{s.}{maquinista de trem}
  \end{Phonetics}
\end{Entry}

\begin{Entry}{火灾}{4,7}{⽕、⽕}
  \begin{Phonetics}{火灾}{huo3 zai1}[][HSK 5]
    \definition[起,场]{s.}{fogo (como um desastre); conflagração; desastres causados por incêndios}
  \end{Phonetics}
\end{Entry}

\begin{Entry}{火花}{4,7}{⽕、⾋}
  \begin{Phonetics}{火花}{huo3hua1}[][HSK 7-9]
    \definition[簇]{s.}{faísca; explosão de chamas | um padrão brilhante}
  \end{Phonetics}
\end{Entry}

\begin{Entry}{火炬}{4,8}{⽕、⽕}
  \begin{Phonetics}{火炬}{huo3ju4}[][HSK 7-9]
    \definition[把]{s.}{tocha}
  \end{Phonetics}
\end{Entry}

\begin{Entry}{火药}{4,9}{⽕、⾋}
  \begin{Phonetics}{火药}{huo3yao4}[][HSK 7-9]
    \definition[桶,克]{s.}{pólvora; um tipo de explosivo que explode com fumaça, como pólvora preta, ou sem fumaça, como nitrato de celulose}
  \end{Phonetics}
\end{Entry}

\begin{Entry}{火候}{4,10}{⽕、⼈}
  \begin{Phonetics}{火候}{huo3hou5}[][HSK 7-9]
    \definition{s.}{duração e grau de aquecimento, cozimento, fundição, etc.  |Coloquial: nível de realização | Coloquial: momento crucial, crítico; emergência}
  \end{Phonetics}
\end{Entry}

\begin{Entry}{火柴}{4,10}{⽕、⽊}
  \begin{Phonetics}{火柴}{huo3chai2}[][HSK 5]
    \definition[根,盒,包]{s.}{fósforo (palito de fósforo); fósforo de segurança; iniciador de fogo feito de uma tira fina de madeira mergulhada em um composto de fósforo ou enxofre}
  \end{Phonetics}
\end{Entry}

\begin{Entry}{火海}{4,10}{⽕、⽔}
  \begin{Phonetics}{火海}{huo3hai3}
    \definition{s.}{um mar de chamas}
  \end{Phonetics}
\end{Entry}

\begin{Entry}{火热}{4,10}{⽕、⽕}
  \begin{Phonetics}{火热}{huo3re4}[][HSK 7-9]
    \definition{adj.}{ardente; fervente; fervoroso; apaixonado; escaldante; abrasador (opp. 冰冷)}
  \seealsoref{冰冷}{bing1leng3}
  \end{Phonetics}
\end{Entry}

\begin{Entry}{火速}{4,10}{⽕、⾡}
  \begin{Phonetics}{火速}{huo3su4}[][HSK 7-9]
    \definition{adv.}{em alta velocidade; com pressa; usar a velocidade mais rápida (para fazer coisas urgentes)}
  \end{Phonetics}
\end{Entry}

\begin{Entry}{火焰}{4,12}{⽕、⽕}
  \begin{Phonetics}{火焰}{huo3yan4}[][HSK 7-9]
    \definition[团,缕,股,道]{s.}{chama; labareda; flama}
  \end{Phonetics}
\end{Entry}

\begin{Entry}{火锅}{4,12}{⽕、⾦}
  \begin{Phonetics}{火锅}{huo3guo1}[][HSK 7-9]
    \definition[顿]{s.}{\emph{hot pot}; uma panela feita de metal ou outro material, que pode ser usada para aquecer a sopa continuamente com eletricidade, álcool, etc., e depois adicionar carne, vegetais, etc. à sopa e comê-la enquanto cozinha; atualmente, refere-se principalmente a alimentos cozidos dessa maneira}
  \end{Phonetics}
\end{Entry}

\begin{Entry}{火腿}{4,13}{⽕、⾁}
  \begin{Phonetics}{火腿}{huo3 tui3}[][HSK 5]
    \definition[道,个]{s.}{presunto; as pernas de porco marinadas mais famosas são produzidas em Jinhua, na província de Zhejiang, e em Xuanwei, na província de Yunnan.}
  \end{Phonetics}
\end{Entry}

\begin{Entry}{火辣辣}{4,14,14}{⽕、⾟、⾟}
  \begin{Phonetics}{火辣辣}{huo3la4la4}[][HSK 7-9]
    \definition{adj.}{ardente; escaldante; abrasador}
  \end{Phonetics}
\end{Entry}

\begin{Entry}{火暴}{4,15}{⽕、⽇}
  \begin{Phonetics}{火暴}{huo3bao4}[][HSK 7-9]
    \definition{adj.}{impetuoso; impetuoso; irritável; impaciente | próspero; florescente; vigoroso; animado}
  \end{Phonetics}
\end{Entry}

\begin{Entry}{火箭}{4,15}{⽕、⽵}
  \begin{Phonetics}{火箭}{huo3jian4}[][HSK 6]
    \definition[个,艘,发,枚]{s.}{foguete; uma aeronave de alta velocidade que utiliza força de reação para se impulsionar para a frente; é usado para lançar satélites, naves espaciais, etc.; também pode ser equipado com uma ogiva para fabricar um míssil}
  \end{Phonetics}
\end{Entry}

\begin{Entry}{父}{4}{⽗}[Kangxi 88]
  \begin{Phonetics}{父}{fu4}
    \definition{s.}{pai | um homem mais velho na família ou parentes | fundador; uma pessoa que inventa ou inicia algo}
  \end{Phonetics}
\end{Entry}

\begin{Entry}{父女}{4,3}{⽗、⼥}
  \begin{Phonetics}{父女}{fu4 nv3}[][HSK 6]
    \definition{s.}{pai e filha}
  \end{Phonetics}
\end{Entry}

\begin{Entry}{父子}{4,3}{⽗、⼦}
  \begin{Phonetics}{父子}{fu4 zi3}[][HSK 6]
    \definition{s.}{pai e filho}
  \end{Phonetics}
\end{Entry}

\begin{Entry}{父母}{4,5}{⽗、⽏}
  \begin{Phonetics}{父母}{fu4 mu3}[][HSK 3]
    \definition{s.}{pai e mãe; pais}
  \end{Phonetics}
\end{Entry}

\begin{Entry}{父母亲}{4,5,9}{⽗、⽏、⼇}
  \begin{Phonetics}{父母亲}{fu4mu3qin1}
    \definition{s.}{os pais; pai e mãe}
  \end{Phonetics}
\end{Entry}

\begin{Entry}{父亲}{4,9}{⽗、⼇}
  \begin{Phonetics}{父亲}{fu4qin1}[][HSK 3]
    \definition[个,位,名]{s.}{pai; homem com filhos; pai dos filhos}
  \end{Phonetics}
\end{Entry}

\begin{Entry}{爿}{4}{⽙}[Kangxi 90]
  \begin{Phonetics}{爿}{pan2}
    \definition{clas.}{usado para faixas de terra ou bambu, lojas, fábricas etc.}
  \end{Phonetics}
\end{Entry}

\begin{Entry}{片}{4}{⽚}[Kangxi 91]
  \begin{Phonetics}{片}{pian1}
    \definition{s.}{película; filme; refere-se a filmes com imagens, paisagens ou imagens gravadas com som}
  \end{Phonetics}
  \begin{Phonetics}{片}{pian4}[][HSK 2]
    \definition*{s.}{Sobrenome Pian}
    \definition{adj.}{breve; parcial; incompleto; fragmentário; esporádico; breve | unilateral}
    \definition{clas.}{usado para coisas em forma de lâminas | usado para terrenos ou superfícies aquáticas com a mesma paisagem e que estão conectados entre si | usado para paisagens, clima, sons, linguagem, intenções, etc. (usado em conjunto com o numeral 一)}
    \definition{s.}{plano, fatia; floco; pedaço fino; algo plano e fino | seção; parte de uma grande área; uma pequena parte do todo ou uma área menor dividida dentro de uma área maior | filme; peça de TV; referência ao filme}
    \definition{v.}{fatiar; cortar em fatias; cortar em fatias finas com uma faca | abrir; cortar; separar}
  \seealsoref{一}{yi1}
  \end{Phonetics}
\end{Entry}

\begin{Entry}{片儿}{4,2}{⽚、⼉}
  \begin{Phonetics}{片儿}{pian1r5}
    \definition{s.}{folha | película; filme}
  \end{Phonetics}
\end{Entry}

\begin{Entry}{片面}{4,9}{⽚、⾯}
  \begin{Phonetics}{片面}{pian4mian4}[][HSK 4]
    \definition{adj.}{unilateral; tendencioso para um lado (em oposição a 全面)}
  \seealsoref{全面}{quan2mian4}
  \end{Phonetics}
\end{Entry}

\begin{Entry}{牙}{4}{⽛}[Kangxi 92]
  \begin{Phonetics}{牙}{ya2}[][HSK 4]
    \definition*{s.}{Sobrenome Ya}
    \definition[颗,排]{s.}{dente | marfim | algo semelhante a um dente}
  \end{Phonetics}
\end{Entry}

\begin{Entry}{牙行}{4,6}{⽛、⾏}
  \begin{Phonetics}{牙行}{ya2hang2}
    \definition{s.}{corretor | \emph{broker}}
  \end{Phonetics}
\end{Entry}

\begin{Entry}{牙医}{4,7}{⽛、⼖}
  \begin{Phonetics}{牙医}{ya2yi1}
    \definition{s.}{dentista}
  \end{Phonetics}
\end{Entry}

\begin{Entry}{牙刷}{4,8}{⽛、⼑}
  \begin{Phonetics}{牙刷}{ya2 shua1}[][HSK 4]
    \definition[把,个,支]{s.}{escova de dentes}
  \end{Phonetics}
\end{Entry}

\begin{Entry}{牙线}{4,8}{⽛、⽷}
  \begin{Phonetics}{牙线}{ya2xian4}
    \definition[条]{s.}{fio dental}
  \end{Phonetics}
\end{Entry}

\begin{Entry}{牙齿}{4,8}{⽛、⿒}
  \begin{Phonetics}{牙齿}{ya2chi3}
    \definition{adv.}{dental}
    \definition[颗]{s.}{dente}
  \end{Phonetics}
\end{Entry}

\begin{Entry}{牙膏}{4,14}{⽛、⾁}
  \begin{Phonetics}{牙膏}{ya2gao1}
    \definition[管]{s.}{pasta de dente}
  \end{Phonetics}
\end{Entry}

\begin{Entry}{牛}{4}{⽜}[Kangxi 93]
  \begin{Phonetics}{牛}{niu2}[][HSK 3,5]
    \definition*{s.}{Sobrenome Niu}
    \definition{adj.}{muito capaz ou bom; descreve pessoas ou coisas como sendo muito capazes, muito competentes | teimoso; arrogante; descreve uma pessoa que é muito orgulhosa ou muito insistente em suas opiniões, difícil de mudar}
    \definition{clas.}{Newton (medida física de força)}
    \definition[头]{s.}{gado; boi | niu (nona das vinte e oito constelações em que a esfera celeste foi dividida, consistindo de seis estrelas, três em Áries e três em Sagitário)}
  \end{Phonetics}
\end{Entry}

\begin{Entry}{牛人}{4,2}{⽜、⼈}
  \begin{Phonetics}{牛人}{niu2ren2}
    \definition{s.}{(coloquial) o cara | verdadeiro especialista | \emph{badass}}
  \end{Phonetics}
\end{Entry}

\begin{Entry}{牛仔裤}{4,5,12}{⽜、⼈、⾐}
  \begin{Phonetics}{牛仔裤}{niu2zai3ku4}[][HSK 5]
    \definition[条]{s.}{calças jeans; calças geralmente feitas de tecido jeans azul grosso}
  \end{Phonetics}
\end{Entry}

\begin{Entry}{牛奶}{4,5}{⽜、⼥}
  \begin{Phonetics}{牛奶}{niu2nai3}[][HSK 1]
    \definition[杯,袋,瓶,盒,箱,桶]{s.}{leite}
  \end{Phonetics}
\end{Entry}

\begin{Entry}{牛肉}{4,6}{⽜、⾁}
  \begin{Phonetics}{牛肉}{niu2rou4}
    \definition{s.}{carne de vaca | bife}
  \end{Phonetics}
\end{Entry}

\begin{Entry}{牛郎织女}{4,8,8,3}{⽜、⾢、⽷、⼥}
  \begin{Phonetics}{牛郎织女}{niu2 lang2 zhi1nv3}
    \definition*{s.}{Vaqueiro e Tecelã (personagens de contos folclóricos) | Altair e Vega (estrelas)}[我们这些牛郎织女都恨透了那条无情的“天河”。===Nós, o Vaqueiro e a Tecelã, odiamos a implacável ``Via Láctea''.]
    \definition{s.}{marido e mulher que vivem longe um do outro}
  \end{Phonetics}
\end{Entry}

\begin{Entry}{牛顿}{4,10}{⽜、⾴}
  \begin{Phonetics}{牛顿}{niu2dun4}
    \definition*{s.}{Newton (nome) | N; Newton, unidade de força do SI}
  \end{Phonetics}
\end{Entry}

\begin{Entry}{犬}{4}{⽝}[Kangxi 94]
  \begin{Phonetics}{犬}{quan3}
    \definition{s.}{cachorro}
  \end{Phonetics}
\end{Entry}

\begin{Entry}{王}{4}{⽟}[Kangxi 96]
  \begin{Phonetics}{王}{wang2}[][HSK 4]
    \definition*{s.}{Sobrenome Wang}
    \definition{adj.}{grande; ótimo; honoríficos antigos para avós}
    \definition{s.}{rei; monarca; imperador; governante supremo de uma monarquia | cabeça; chefe; líder | o primeiro, maior ou mais forte de seu tipo | duque; príncipe; o título mais alto da sociedade feudal após a dinastia Han}
  \end{Phonetics}
  \begin{Phonetics}{王}{wang4}
    \definition{v.}{reger; governar; reinar; dominar}
  \end{Phonetics}
\end{Entry}

\begin{Entry}{王子}{4,3}{⽟、⼦}
  \begin{Phonetics}{王子}{wang2zi3}[][HSK 6]
    \definition[位]{s.}{príncipe; filho do rei}
  \end{Phonetics}
\end{Entry}

\begin{Entry}{王五}{4,4}{⽟、⼆}
  \begin{Phonetics}{王五}{wang2wu3}
    \definition{s.}{Wang Wu | Zé Ninguém | nome para uma pessoa não especificada, 3 de 3}
  \seealsoref{李四}{li3si4}
  \seealsoref{张三}{zhang1san1}
  \end{Phonetics}
\end{Entry}

\begin{Entry}{王后}{4,6}{⽟、⼝}
  \begin{Phonetics}{王后}{wang2 hou4}[][HSK 6]
    \definition[个,位,名,些]{s.}{rainha consorte; rainha}
  \end{Phonetics}
\end{Entry}

\begin{Entry}{王朝}{4,12}{⽟、⽉}
  \begin{Phonetics}{王朝}{wang2chao2}
    \definition{s.}{dinastia}
  \end{Phonetics}
\end{Entry}

\begin{Entry}{瓦}{4}{⽡}[Kangxi 98]
  \begin{Phonetics}{瓦}{wa3}
    \definition*{s.}{Sobrenome Wa}
    \definition{adj.}{de barro; feito de barro}
    \definition{clas.}{W, watt; medida de potência elétrica}
    \definition[片,块]{s.}{Literário: fuso primordial | telha |  algo feito de barro cozido | abreviação de 瓦特}
  \seealsoref{瓦特}{wa3te4}
  \end{Phonetics}
\end{Entry}

\begin{Entry}{瓦努阿图}{4,7,7,8}{⽡、⼒、⾩、⼞}
  \begin{Phonetics}{瓦努阿图}{wa3nu3'a1tu2}
    \definition*{s.}{Vanuatu, país do sudoeste do Oceano Pacífico}
  \end{Phonetics}
\end{Entry}

\begin{Entry}{瓦特}{4,10}{⽡、⽜}
  \begin{Phonetics}{瓦特}{wa3te4}
    \definition{s.}{(empréstimo linguístico) watt | medida de potência}
  \end{Phonetics}
\end{Entry}

\begin{Entry}{艺}{4}{⾋}
  \begin{Phonetics}{艺}{yi4}
    \definition*{s.}{Sobrenome Yi}
    \definition[个,种]{s.}{habilidade | arte | regra; norma | padrão; critério; diretrizes | limite}
    \definition{v.}{plantar; crescer}
  \end{Phonetics}
\end{Entry}

\begin{Entry}{艺人}{4,2}{⾋、⼈}
  \begin{Phonetics}{艺人}{yi4 ren2}[][HSK 6]
    \definition[位]{s.}{artista performático; ator profissional; ator ou artista (em teatro local, narradores, acrobacia ou outro show business); atores de ópera, arte popular, acrobacia, cinema e televisão, etc. | artesão; artífice}
  \end{Phonetics}
\end{Entry}

\begin{Entry}{艺术}{4,5}{⾋、⽊}
  \begin{Phonetics}{艺术}{yi4shu4}[][HSK 3]
    \definition{adj.}{artístico,único; elegante}
    \definition[个,种,门,项,类]{s.}{arte; literatura e arte | habilidade; arte; ofício; métodos criativos}
  \end{Phonetics}
\end{Entry}

\begin{Entry}{见}{4}{⾒}[Kangxi 147]
  \begin{Phonetics}{见}{jian4}[][HSK 1]
    \definition*{s.}{Sobrenome Jian}
    \definition{part.}{usado antes de um verbo para indicar voz passiva ou para expressar como isso me afeta}
    \definition{s.}{visão; ideia; opinião sobre algo; ponto de vista}
    \definition{v.}{ver; avistar | encontrar-se com; ser exposto a | parecer ser; mostrar evidência de | ver; referir-se a; indicar a fonte ou o local onde deve ser consultado | ver; encontrar; convocar}
  \end{Phonetics}
  \begin{Phonetics}{见}{xian4}
    \definition{v.}{aparecer; também escrito como 现}
  \seealsoref{现}{xian4}
  \end{Phonetics}
\end{Entry}

\begin{Entry}{见过}{4,6}{⾒、⾡}
  \begin{Phonetics}{见过}{jian4 guo4}[][HSK 2]
    \definition{s.}{visto (ver); já viu alguém ou algo; indica um momento no passado; alguém já viu ou encontrou um determinado objeto}
  \end{Phonetics}
\end{Entry}

\begin{Entry}{见到}{4,8}{⾒、⼑}
  \begin{Phonetics}{见到}{jian4 dao4}[][HSK 2]
    \definition{v.}{ver | encontrar; esbarrar; deparar-se com}
  \end{Phonetics}
\end{Entry}

\begin{Entry}{见面}{4,9}{⾒、⾯}
  \begin{Phonetics}{见面}{jian4/mian4}[][HSK 1]
    \definition{v.+compl.}{encontrar-se com alguém;  ver um ao outro; ver alguém face-a-face}
  \end{Phonetics}
\end{Entry}

\begin{Entry}{计}{4}{⾔}
  \begin{Phonetics}{计}{ji4}[][HSK 7-9]
    \definition*{s.}{Sobrenome Ji}
    \definition{s.}{medidor; aferidor; indicador; um instrumento para medir ou calcular graus, tempo, etc. | ideia; ardil; estratagema; plano}
    \definition{v.}{contar; calcular; numerar | planejar; traçar; imaginar}
  \end{Phonetics}
\end{Entry}

\begin{Entry}{计划}{4,6}{⾔、⼑}
  \begin{Phonetics}{计划}{ji4hua4}[][HSK 2]
    \definition[个,项]{s.}{plano; projeto; programa; trabalho, ações, conteúdo e etapas previamente definidos}
    \definition{v.}{planejar; traçar um plano}
  \end{Phonetics}
\end{Entry}

\begin{Entry}{计时}{4,7}{⾔、⽇}
  \begin{Phonetics}{计时}{ji4shi2}[][HSK 7-9]
    \definition{v.}{contar pelo tempo; calcular com base em horas de trabalho e proficiência técnica}
  \end{Phonetics}
\end{Entry}

\begin{Entry}{计较}{4,10}{⾔、⾞}
  \begin{Phonetics}{计较}{ji4jiao4}[][HSK 7-9]
    \definition{v.}{pechinchar; discutir sobre | argumentar; disputar | planejar; pensar sobre}
  \end{Phonetics}
\end{Entry}

\begin{Entry}{计策}{4,12}{⾔、⽵}
  \begin{Phonetics}{计策}{ji4ce4}[][HSK 7-9]
    \definition{s.}{estratagema; plano; manobra}
  \end{Phonetics}
\end{Entry}

\begin{Entry}{计算}{4,14}{⾔、⽵}
  \begin{Phonetics}{计算}{ji4suan4}[][HSK 3]
    \definition{v.}{contar; calcular; computar; enumerar; encontrar a variável desconhecida | planejar; considerar | conspirar secretamente contra os outros; planejar secretamente prejudicar os outros}
  \end{Phonetics}
\end{Entry}

\begin{Entry}{计算机}{4,14,6}{⾔、⽵、⽊}
  \begin{Phonetics}{计算机}{ji4 suan4 ji1}[][HSK 2]
    \definition[部,台]{s.}{computador; calculadora; máquinas capazes de realizar cálculos matemáticos são feitas com dispositivos mecânicos, como calculadoras manuais, outras são feitas com componentes eletrônicos, como computadores eletrônicos}
  \end{Phonetics}
\end{Entry}

\begin{Entry}{计算机程序}{4,14,6,12,7}{⾔、⽵、⽊、⽲、⼴}
  \begin{Phonetics}{计算机程序}{ji4suan4ji1 cheng2xu4}
    \definition{s.}{programa de computador}
  \end{Phonetics}
\end{Entry}

\begin{Entry}{订}{4}{⾔}
  \begin{Phonetics}{订}{ding4}[][HSK 3]
    \definition{v.}{concluir; elaborar; concordar com |assinar (um jornal, etc.); reservar (assentos, ingressos, etc.); encomendar (mercadorias, etc.) | fazer correções; revisar | grampear junto; unir; usar linha ou arame para encadernar páginas soltas ou folhas de papel | julgar; determinar}
  \end{Phonetics}
\end{Entry}

\begin{Entry}{订立}{4,5}{⾔、⽴}
  \begin{Phonetics}{订立}{ding4li4}[][HSK 7-9]
    \definition{v.}{concluir (um tratado, acordo, etc.); fazer (um contrato, etc.)}
  \end{Phonetics}
\end{Entry}

\begin{Entry}{订单}{4,8}{⾔、⼗}
  \begin{Phonetics}{订单}{ding4dan1}[][HSK 7-9]
    \definition{s.}{pedido de mercadorias; formulário de pedido; contratos e documentos para encomenda de mercadorias}
  \end{Phonetics}
\end{Entry}

\begin{Entry}{订购}{4,8}{⾔、⾙}
  \begin{Phonetics}{订购}{ding4gou4}[][HSK 7-9]
    \definition{v.}{encomendar (mercadorias); enviar para; fazer um pedido de algo | concordar em comprar (bens, etc.)}
  \end{Phonetics}
\end{Entry}

\begin{Entry}{订金}{4,8}{⾔、⾦}
  \begin{Phonetics}{订金}{ding4jin1}
    \definition{s.}{depósito; sinal; entrada; o pagamento antecipado de parte do pagamento das mercadorias tem um certo significado de compromisso, mas não garante legalmente a execução do contrato}
  \end{Phonetics}
\end{Entry}

\begin{Entry}{订婚}{4,11}{⾔、⼥}
  \begin{Phonetics}{订婚}{ding4/hun1}[][HSK 7-9]
    \definition{v.+compl.}{estar noivo(a); estar prometido(a) a}
  \end{Phonetics}
\end{Entry}

\begin{Entry}{讣}{4}{⾔}
  \begin{Phonetics}{讣}{fu4}
    \definition[个,则]{s.}{obituário}
    \definition{v.}{anunciar a morte de alguém | relatar um luto}
  \end{Phonetics}
\end{Entry}

\begin{Entry}{认}{4}{⾔}
  \begin{Phonetics}{认}{ren4}[][HSK 5]
    \definition{v.}{reconhecer; saber; distinguir; identificar | estabelecer uma determinada relação com; adotar | admitir; reconhecer; assumir | comprometer-se a fazer algo | (frequentemente seguido por 了) aceitar como inevitável; resignar-se}
  \seealsoref{了}{le5}
  \end{Phonetics}
\end{Entry}

\begin{Entry}{认为}{4,4}{⾔、⼂}
  \begin{Phonetics}{认为}{ren4wei2}[][HSK 2]
    \definition{v.}{pensar; considerar; manter; julgar; formar uma opinião sobre uma pessoa ou coisa, fazer um julgamento}
  \end{Phonetics}
\end{Entry}

\begin{Entry}{认出}{4,5}{⾔、⼐}
  \begin{Phonetics}{认出}{ren4 chu1}[][HSK 3]
    \definition{v.}{reconhecer; identificar; reconhecer alguém ou algo pela observação ou memória}
  \end{Phonetics}
\end{Entry}

\begin{Entry}{认可}{4,5}{⾔、⼝}
  \begin{Phonetics}{认可}{ren4ke3}[][HSK 3]
    \definition{v.}{aceitar; aprovar; confirmar; dar força legal a | permitir; concordar}
  \end{Phonetics}
\end{Entry}

\begin{Entry}{认同}{4,6}{⾔、⼝}
  \begin{Phonetics}{认同}{ren4 tong2}[][HSK 6]
    \definition{v.}{identificar; pensar que a outra pessoa tem algo em comum com você | aprovar; reconhecer}
  \end{Phonetics}
\end{Entry}

\begin{Entry}{认识}{4,7}{⾔、⾔}
  \begin{Phonetics}{认识}{ren4shi5}[][HSK 1]
    \definition[份]{s.}{cognição; conhecimento; compreensão; refere-se à reflexão da mente humana sobre o mundo objetivo}
    \definition{v.}{saber; compreender; reconhecer}
  \end{Phonetics}
\end{Entry}

\begin{Entry}{认定}{4,8}{⾔、⼧}
  \begin{Phonetics}{认定}{ren4ding4}[][HSK 5]
    \definition{v.}{afirmar; manter; acreditar firmemente; considerar com certeza | decidir-se por algo; confirmar; chegar a uma conclusão afirmativa}
  \end{Phonetics}
\end{Entry}

\begin{Entry}{认真}{4,10}{⾔、⼗}
  \begin{Phonetics}{认真}{ren4zhen1}[][HSK 1]
    \definition{adj.}{sério; sério e meticuloso}
    \definition{adv.}{seriamente}
    \definition{v.}{levar algo a sério; considerar como verdadeiro; levar a sério}
  \end{Phonetics}
\end{Entry}

\begin{Entry}{认得}{4,11}{⾔、⼻}
  \begin{Phonetics}{认得}{ren4 de5}[][HSK 3]
    \definition{v.}{saber; reconhecer; capacidade de confirmar a pessoa ou coisa que você vê}
  \end{Phonetics}
\end{Entry}

\begin{Entry}{讥}{4}{⾔}
  \begin{Phonetics}{讥}{ji1}
    \definition{v.}{ridicularizar; zombar; satirizar}
  \end{Phonetics}
\end{Entry}

\begin{Entry}{讥笑}{4,10}{⾔、⽵}
  \begin{Phonetics}{讥笑}{ji1xiao4}[][HSK 7-9]
    \definition{v.}{ridicularizar; zombar; zombar de; tirar sarro de}
  \end{Phonetics}
\end{Entry}

\begin{Entry}{贝}{4}{⾙}[Kangxi 154]
  \begin{Phonetics}{贝}{bei4}
    \definition*{s.}{Sobrenome Bei}
    \definition{clas.}{bel (= dez decibéis)}
    \definition{s.}{marisco; crustáceo; um termo geral para moluscos com concha | moeda antiga feita de conchas}
  \end{Phonetics}
\end{Entry}

\begin{Entry}{贝壳}{4,7}{⾙、⼠}
  \begin{Phonetics}{贝壳}{bei4ke2}[][HSK 7-9]
    \definition[个,种]{s.}{concha; concha do mar; concha de animais de concha}
  \end{Phonetics}
\end{Entry}

\begin{Entry}{车}{4}{⾞}[Kangxi 159]
  \begin{Phonetics}{车}{che1}[][HSK 1]
    \definition*{s.}{Sobrenome Che}
    \definition[辆]{s.}{veículo; meios de transporte terrestres sobre rodas | máquina ou instrumento com rodas; ferramentas com eixo giratório | máquina}
    \definition{v.}{tornear; usinar com torno mecânico | elevar água por meio de uma roda d'água; usar caminhão-pipa para coletar água | girar, geralmente se refere ao corpo}
  \end{Phonetics}
  \begin{Phonetics}{车}{ju1}
    \definition{s.}{torre; castelo; carruagem, uma das peças do xadrez chinês}
  \end{Phonetics}
\end{Entry}

\begin{Entry}{车上}{4,3}{⾞、⼀}
  \begin{Phonetics}{车上}{che1 shang4}[][HSK 1]
    \definition{adv.}{no carro; no interior do veículo}
  \end{Phonetics}
\end{Entry}

\begin{Entry}{车子}{4,3}{⾞、⼦}
  \begin{Phonetics}{车子}{che1zi5}
    \definition{s.}{qualquer veículo (carro, bicicleta, caminhão, etc)}
  \end{Phonetics}
\end{Entry}

\begin{Entry}{车水马龙}{4,4,3,5}{⾞、⽔、⾺、⿓}
  \begin{Phonetics}{车水马龙}{che1shui3-ma3long2}
    \definition{expr.}{tráfego engarrafado | engarrafamento | (literalmente) ``fluxo interminável de cavalos e carruagens''}
  \end{Phonetics}
\end{Entry}

\begin{Entry}{车主}{4,5}{⾞、⼂}
  \begin{Phonetics}{车主}{che1 zhu3}[][HSK 5]
    \definition[位,个]{s.}{proprietário do carro; uma pessoa física ou família que possui um veículo motorizado}
  \end{Phonetics}
\end{Entry}

\begin{Entry}{车号}{4,5}{⾞、⼝}
  \begin{Phonetics}{车号}{che1 hao4}[][HSK 6]
    \definition[个]{s.}{número da licença (de um veículo) | número do carro}
  \end{Phonetics}
\end{Entry}

\begin{Entry}{车次}{4,6}{⾞、⽋}
  \begin{Phonetics}{车次}{che1ci4}
    \definition{s.}{número do trem}
  \end{Phonetics}
\end{Entry}

\begin{Entry}{车位}{4,7}{⾞、⼈}
  \begin{Phonetics}{车位}{che1wei4}[][HSK 7-9]
    \definition[个]{s.}{vaga de estacionamento | vaga de garagem | ponto de táxi | ponto de descarga}
  \end{Phonetics}
\end{Entry}

\begin{Entry}{车库}{4,7}{⾞、⼴}
  \begin{Phonetics}{车库}{che1ku4}
    \definition{s.}{garagem}
  \end{Phonetics}
\end{Entry}

\begin{Entry}{车间}{4,7}{⾞、⾨}
  \begin{Phonetics}{车间}{che1jian1}[][HSK 7-9]
    \definition[个,栋]{s.}{fábrica; oficina; uma unidade dentro de uma empresa que conclui certos processos no processo de produção ou produz certos produtos de forma independente}
  \end{Phonetics}
\end{Entry}

\begin{Entry}{车轮}{4,8}{⾞、⾞}
  \begin{Phonetics}{车轮}{che1lun2}[][HSK 7-9]
    \definition[个]{s.}{roda (de um veículo); a parte redonda e giratória de um carro; também chamada de 车轮子}
  \seealsoref{车轮子}{che1lun2zi5}
  \end{Phonetics}
\end{Entry}

\begin{Entry}{车轮子}{4,8,3}{⾞、⾞、⼦}
  \begin{Phonetics}{车轮子}{che1lun2zi5}
    \definition{s.}{roda}
  \end{Phonetics}
\end{Entry}

\begin{Entry}{车型}{4,9}{⾞、⼟}
  \begin{Phonetics}{车型}{che1xing2}[][HSK 7-9]
    \definition{s.}{marca e modelo do carro; tipo de motocicleta | modelo (de um carro)}
  \end{Phonetics}
\end{Entry}

\begin{Entry}{车轴}{4,9}{⾞、⾞}
  \begin{Phonetics}{车轴}{che1zhou2}[][HSK 7-9]
    \definition[根,条,个]{s.}{eixo (de um veículo)}[这辆货车一共有六个轮子,所以有三条车轴。===Este caminhão tem seis rodas, então ele tem três eixos.]
  \end{Phonetics}
\end{Entry}

\begin{Entry}{车展}{4,10}{⾞、⼫}
  \begin{Phonetics}{车展}{che1 zhan3}[][HSK 6]
    \definition{s.}{salão do automóvel; exposição de carros}
  \end{Phonetics}
\end{Entry}

\begin{Entry}{车站}{4,10}{⾞、⽴}
  \begin{Phonetics}{车站}{che1 zhan4}[][HSK 1]
    \definition[个,处]{s.}{estação; estação ferroviária; parada; pontos de parada estabelecidos nas linhas de transporte rodoviário são locais para embarque e desembarque de passageiros ou carga e descarga de mercadorias}
  \end{Phonetics}
\end{Entry}

\begin{Entry}{车速}{4,10}{⾞、⾡}
  \begin{Phonetics}{车速}{che1su4}[][HSK 7-9]
    \definition{s.}{velocidade de um veículo | velocidade de um torno}
  \end{Phonetics}
\end{Entry}

\begin{Entry}{车厢}{4,11}{⾞、⼚}
  \begin{Phonetics}{车厢}{che1xiang1}[][HSK 7-9]
    \definition[节,个,号,辆]{s.}{compartimento; vagão ferroviário; trem de vagões; a parte de um carro ou outro veículo usado para transportar pessoas ou coisas}
  \end{Phonetics}
\end{Entry}

\begin{Entry}{车票}{4,11}{⾞、⽰}
  \begin{Phonetics}{车票}{che1 piao4}[][HSK 1]
    \definition{s.}{passagem de trem ou ônibus; bilhete; bilhete de transporte público}
  \end{Phonetics}
\end{Entry}

\begin{Entry}{车祸}{4,11}{⾞、⽰}
  \begin{Phonetics}{车祸}{che1huo4}[][HSK 7-9]
    \definition[场,起]{s.}{acidente de trânsito; acidente automobilístico; acidentes envolvendo vítimas durante a condução (principalmente carros)}
  \end{Phonetics}
\end{Entry}

\begin{Entry}{车辆}{4,11}{⾞、⾞}
  \begin{Phonetics}{车辆}{che1 liang4}[][HSK 2]
    \definition{s.}{veículo; carro; termo genérico para todos os tipos de veículos}
  \end{Phonetics}
\end{Entry}

\begin{Entry}{车牌}{4,12}{⾞、⽚}
  \begin{Phonetics}{车牌}{che1 pai2}[][HSK 6]
    \definition[个,块]{s.}{placa de licença; placa instalada no veículo}
  \end{Phonetics}
\end{Entry}

\begin{Entry}{车道}{4,12}{⾞、⾡}
  \begin{Phonetics}{车道}{che1dao4}[][HSK 7-9]
    \definition{s.}{(tráfego) pista; faixa; estrada}[三车道公路。===Rodovia de três pistas.]
  \end{Phonetics}
\end{Entry}

\begin{Entry}{长}{4}{⾧}[Kangxi 168]
  \begin{Phonetics}{长}{chang2}[][HSK 2]
    \definition*{s.}{Sobrenome Chang}
    \definition{adj.}{longo; comprido; a distância entre uma extremidade e outra é grande (em oposição a 短 ) | para sempre; duradouro; a distância entre o ponto inicial e o ponto final de um determinado período é grande | excesso; o que sobra; o que é desnecessário}
    \definition{s.}{comprimento; na direção horizontal, a distância percorrida por um objeto de uma extremidade à outra | ponto forte; especialidade; especialização em determinada área; vantagem}
    \definition{v.}{ser bom em; ser forte em; ser muito bom em algo; ser especialista em algo}
  \seealsoref{短}{duan3}
  \end{Phonetics}
  \begin{Phonetics}{长}{zhang3}[][HSK 2,6]
    \definition{adj.}{mais velho; ancião; sênior; mais antigo; o primeiro da fila}
    \definition{s.}{ancião; pessoas mais velhas ou de posição social mais elevada | chefe; líder; dirigente; responsável}
    \definition{v.}{crescer; desenvolver-se | surgir; começar a crescer; formar-se; nascer em pessoas, animais, plantas ou objetos (algo) | adquirir; melhorar; aumentar; aumentar conhecimento, capacidade, etc.; tornar-se cada vez mais ou cada vez melhor}
  \end{Phonetics}
\end{Entry}

\begin{Entry}{长久}{4,3}{⾧、⼃}
  \begin{Phonetics}{长久}{chang2 jiu3}[][HSK 6]
    \definition{adj.}{longo; permanente; duradouro; por muito tempo; já faz muito tempo}
  \end{Phonetics}
\end{Entry}

\begin{Entry}{长大}{4,3}{⾧、⼤}
  \begin{Phonetics}{长大}{zhang3 da4}[][HSK 2]
    \definition{v.}{crescer; ser criado; um estado de maturidade física ou mental completa}
  \end{Phonetics}
\end{Entry}

\begin{Entry}{长处}{4,5}{⾧、⼡}
  \begin{Phonetics}{长处}{chang2 chu4}[][HSK 3]
    \definition[个]{s.}{forte; boas qualidades; pontos fortes; especialização em determinada área}
  \end{Phonetics}
\end{Entry}

\begin{Entry}{长达}{4,6}{⾧、⾡}
  \begin{Phonetics}{长达}{chang2 da2}[][HSK 7-9]
    \definition{v.}{estender-se tanto quanto | alongar-se para}
  \end{Phonetics}
\end{Entry}

\begin{Entry}{长寿}{4,7}{⾧、⼨}
  \begin{Phonetics}{长寿}{chang2 shou4}[][HSK 5]
    \definition{adj.}{vida longa; longevidade}
  \end{Phonetics}
\end{Entry}

\begin{Entry}{长足}{4,7}{⾧、⾜}
  \begin{Phonetics}{长足}{chang2zu2}[][HSK 7-9]
    \definition{adj.}{Literário: aos trancos e barrancos; substancial; progredindo rapidamente}
  \end{Phonetics}
\end{Entry}

\begin{Entry}{长远}{4,7}{⾧、⾡}
  \begin{Phonetics}{长远}{chang2 yuan3}[][HSK 6]
    \definition{adj.}{longo prazo; longo alcance; duradouro; há muito tempo (referindo-se ao futuro); crônico}
  \end{Phonetics}
\end{Entry}

\begin{Entry}{长征}{4,8}{⾧、⼻}
  \begin{Phonetics}{长征}{chang2zheng1}[][HSK 7-9]
    \definition*{s.}{A Longa Marcha (1934-1935, um importante movimento estratégico do Exército Vermelho dos Trabalhadores e Camponeses da China que conseguiu alcançar a base revolucionária no norte de Shaanxi depois de atravessar onze províncias e cobrir 25.000 li, ou 12.500 quilômetros)}
    \definition{s.}{expedição; longa marcha; viagem de longa distância; expedição de longa distância}
  \end{Phonetics}
\end{Entry}

\begin{Entry}{长城}{4,9}{⾧、⼟}
  \begin{Phonetics}{长城}{chang2cheng2}[][HSK 3]
    \definition*[段,座,条]{s.}{A Grande Muralha da China; também é usado como metáfora para uma força poderosa e inabalável, um obstáculo intransponível, etc.}
  \end{Phonetics}
\end{Entry}

\begin{Entry}{长度}{4,9}{⾧、⼴}
  \begin{Phonetics}{长度}{chang2 du4}[][HSK 5]
    \definition{s.}{comprimento; extensão; distância entre dois pontos}
  \end{Phonetics}
\end{Entry}

\begin{Entry}{长效}{4,10}{⾧、⽁}
  \begin{Phonetics}{长效}{chang2xiao4}[][HSK 7-9]
    \definition{adj.}{duradouro}
    \definition{s.}{efeito duradouro}
    \definition{v.}{ser eficaz por um período prolongado}
  \end{Phonetics}
\end{Entry}

\begin{Entry}{长途}{4,10}{⾧、⾡}
  \begin{Phonetics}{长途}{chang2tu2}[][HSK 4]
    \definition{adj.}{de longa distância; longe}
    \definition[段,次,程]{s.}{longa-distância; referindo-se especificamente a chamadas telefônicas de longa distância ou ônibus de longa distância}
  \end{Phonetics}
\end{Entry}

\begin{Entry}{长假}{4,11}{⾧、⼈}
  \begin{Phonetics}{长假}{chang2 jia4}[][HSK 6]
    \definition{s.}{longa licença de ausência | feriado prolongado | (datado) renúncia | férias longas | refere-se a um feriado nacional de uma semana na China, começando em 1º de maio e 1º de outubro}
  \end{Phonetics}
\end{Entry}

\begin{Entry}{长颈鹿}{4,11,11}{⾧、⾴、⿅}
  \begin{Phonetics}{长颈鹿}{chang2jing3lu4}
    \definition[只]{s.}{girafa}
  \end{Phonetics}
\end{Entry}

\begin{Entry}{长期}{4,12}{⾧、⽉}
  \begin{Phonetics}{长期}{chang2 qi1}[][HSK 3]
    \definition{adj.}{secular; longo prazo; longo alcance; durante um longo período de tempo}
    \definition{s.}{longo prazo; por muito tempo}
  \end{Phonetics}
\end{Entry}

\begin{Entry}{长期以来}{4,12,4,7}{⾧、⽉、⼈、⽊}
  \begin{Phonetics}{长期以来}{chang2qi1 yi3lai2}[][HSK 7-9]
    \definition{adv.}{por muito tempo passado | desde muito tempo atrás}
  \end{Phonetics}
\end{Entry}

\begin{Entry}{长短}{4,12}{⾧、⽮}
  \begin{Phonetics}{长短}{chang2 duan3}[][HSK 6]
    \definition{s.}{comprimento | acidente; infortúnio | certo e errado; pontos fortes e fracos}
  \end{Phonetics}
\end{Entry}

\begin{Entry}{长跑}{4,12}{⾧、⾜}
  \begin{Phonetics}{长跑}{chang2 pao3}[][HSK 6]
    \definition{s.}{corrida de longa distância (oposto a 短跑)}
  \seealsoref{短跑}{duan3 pao3}
  \end{Phonetics}
\end{Entry}

\begin{Entry}{队}{4}{⾩}
  \begin{Phonetics}{队}{dui4}[][HSK 2]
    \definition*{s.}{Jovens Pioneiros, refere-se especificamente à Patrulha de Jovens Pioneiros}
    \definition[条,个,支]{s.}{fila de pessoas | equipe; grupo;}
  \end{Phonetics}
\end{Entry}

\begin{Entry}{队友}{4,4}{⾩、⼜}
  \begin{Phonetics}{队友}{dui4you3}
    \definition{s.}{companheiro de equipe}
  \end{Phonetics}
\end{Entry}

\begin{Entry}{队长}{4,4}{⾩、⾧}
  \begin{Phonetics}{队长}{dui4 zhang3}[][HSK 2]
    \definition[个,位,名]{s.}{capitão (de equipe); capitães | líder de equipe}
  \end{Phonetics}
\end{Entry}

\begin{Entry}{队伍}{4,6}{⾩、⼈}
  \begin{Phonetics}{队伍}{dui4wu3}[][HSK 6]
    \definition[条,行,列,支,个]{s.}{tropas; exército | fileiras; contingente | desfile}
  \end{Phonetics}
\end{Entry}

\begin{Entry}{队员}{4,7}{⾩、⼝}
  \begin{Phonetics}{队员}{dui4 yuan2}[][HSK 3]
    \definition[名,位,个]{s.}{membro da equipe; a composição de uma equipe}
  \end{Phonetics}
\end{Entry}

\begin{Entry}{队形}{4,7}{⾩、⼺}
  \begin{Phonetics}{队形}{dui4xing2}[][HSK 7-9]
    \definition{s.}{formação; disposição}
  \end{Phonetics}
\end{Entry}

\begin{Entry}{风}{4}{⾵}[Kangxi 182]
  \begin{Phonetics}{风}{feng1}[][HSK 1]
    \definition*{s.}{Sobrenome Feng}
    \definition{adj.}{lendário; sem fundamento concreto | rápido; veloz | promíscuo; libertino; sedutor}
    \definition[阵,丝]{s.}{vento; fluxo de ar | prática; ambiente; costume | cena; vista | notícias; fofocas; rumores | comportamento; maneira; estilo | canção folclórica | certas doenças}
    \definition{v.}{colocar para secar ou arejar; secar ao vento}
  \end{Phonetics}
\end{Entry}

\begin{Entry}{风力}{4,2}{⾵、⼒}
  \begin{Phonetics}{风力}{feng1li4}[][HSK 7-9]
    \definition{s.}{força do vento; o poder do vento | energia eólica; alimentado pelo vento}
  \end{Phonetics}
\end{Entry}

\begin{Entry}{风云}{4,4}{⾵、⼆}
  \begin{Phonetics}{风云}{feng1yun2}[][HSK 7-9]
    \definition{s.}{vento e nuvem; uma situação tempestuosa ou instável}
  \end{Phonetics}
\end{Entry}

\begin{Entry}{风气}{4,4}{⾵、⽓}
  \begin{Phonetics}{风气}{feng1qi4}[][HSK 7-9]
    \definition[种]{s.}{ethos; atmosfera; humor geral; prática comum; um \emph{hobby} ou hábito popular na sociedade ou em um grupo}
  \seealsoref{风尚}{feng1shang4}
  \end{Phonetics}
\end{Entry}

\begin{Entry}{风水}{4,4}{⾵、⽔}
  \begin{Phonetics}{风水}{feng1shui3}[][HSK 7-9]
    \definition*{s.}{Feng Shui}
    \definition{s.}{geomancia; refere-se à localização geográfica de locais residenciais, cemitérios, etc., como a direção dos veios de terra, montanhas e rios}
  \end{Phonetics}
\end{Entry}

\begin{Entry}{风风雨雨}{4,4,8,8}{⾵、⾵、⾬、⾬}
  \begin{Phonetics}{风风雨雨}{feng1feng1yu3yu3}[][HSK 7-9]
    \definition{expr.}{altos e baixos | dificuldades e sofrimentos | fofocas infundadas}
  \end{Phonetics}
\end{Entry}

\begin{Entry}{风光}{4,6}{⾵、⼉}
  \begin{Phonetics}{风光}{feng1guang1}[][HSK 5]
    \definition{s.}{cena; vista; paisagens naturais e humanas}
  \end{Phonetics}
\end{Entry}

\begin{Entry}{风沙}{4,7}{⾵、⽔}
  \begin{Phonetics}{风沙}{feng1sha1}[][HSK 7-9]
    \definition{s.}{areia soprada pelo vento; areia levada pelo vento}
  \end{Phonetics}
\end{Entry}

\begin{Entry}{风味}{4,8}{⾵、⼝}
  \begin{Phonetics}{风味}{feng1wei4}[][HSK 7-9]
    \definition{s.}{sabor especial; cor (ou sabor) local; características das coisas (referindo-se principalmente às características locais)}
  \end{Phonetics}
\end{Entry}

\begin{Entry}{风和日丽}{4,8,4,7}{⾵、⼝、⽇、⼀}
  \begin{Phonetics}{风和日丽}{feng1he2-ri4li4}[][HSK 7-9]
    \definition{expr.}{vento moderado, sol bonito; tempo bom e ensolarado; sol brilhante e uma brisa suave; clima quente e ensolarado; descreve o clima ensolarado e ameno (usado principalmente na primavera)}
  \end{Phonetics}
\end{Entry}

\begin{Entry}{风尚}{4,8}{⾵、⼩}
  \begin{Phonetics}{风尚}{feng1shang4}[][HSK 7-9]
    \definition{s.}{costume (ou prática, hábito) predominante; as tendências e costumes sociais predominantes durante um determinado período}
  \seealsoref{风气}{feng1qi4}
  \end{Phonetics}
\end{Entry}

\begin{Entry}{风波}{4,8}{⾵、⽔}
  \begin{Phonetics}{风波}{feng1bo1}[][HSK 7-9]
    \definition[场]{s.}{Meteorologia: vento e ondas --- perturbação | Literário: tempestade | perturbação; onda de vento; uma tempestade em copo d'água | tempestade; crise; perturbação}
  \end{Phonetics}
\end{Entry}

\begin{Entry}{风采}{4,8}{⾵、⾤}
  \begin{Phonetics}{风采}{feng1cai3}[][HSK 7-9]
    \definition{s.}{graça; elegância; comportamento; carisma | estilo; graça; talento literário}
  \end{Phonetics}
\end{Entry}

\begin{Entry}{风雨}{4,8}{⾵、⾬}
  \begin{Phonetics}{风雨}{feng1yu3}[][HSK 7-9]
    \definition{s.}{o clima; vento e chuva | tribulações; estresse e tempestade; provações e dificuldades; uma metáfora para uma situação difícil e dura}
  \end{Phonetics}
\end{Entry}

\begin{Entry}{风俗}{4,9}{⾵、⼈}
  \begin{Phonetics}{风俗}{feng1su2}[][HSK 4]
    \definition[种,个,些]{s.}{costumes; a soma de costumes sociais, maneiras, hábitos, etc., desenvolvidos ao longo do tempo}
  \end{Phonetics}
\end{Entry}

\begin{Entry}{风度}{4,9}{⾵、⼴}
  \begin{Phonetics}{风度}{feng1du4}[][HSK 5]
    \definition{s.}{postura; comportamento; porte; conduta; atitude}
  \end{Phonetics}
\end{Entry}

\begin{Entry}{风范}{4,9}{⾵、⾋}
  \begin{Phonetics}{风范}{feng1fan4}[][HSK 7-9]
    \definition{s.}{comportamento; porte; postura | estilo; maneira; ar | modelo; protótipo}
  \end{Phonetics}
\end{Entry}

\begin{Entry}{风险}{4,9}{⾵、⾩}
  \begin{Phonetics}{风险}{feng1xian3}[][HSK 3]
    \definition[个,种]{s.}{risco; perigo; ameaça; riscos possíveis}
  \end{Phonetics}
\end{Entry}

\begin{Entry}{风扇}{4,10}{⾵、⼾}
  \begin{Phonetics}{风扇}{feng1shan4}
    \definition{s.}{ventilador elétrico}
  \end{Phonetics}
\end{Entry}

\begin{Entry}{风格}{4,10}{⾵、⽊}
  \begin{Phonetics}{风格}{feng1ge2}[][HSK 4]
    \definition{s.}{modo; estilo; maneira; caráter | características das criações literárias de diferentes épocas, povos, escolas ou indivíduos em termos de conteúdo ideológico e técnicas artísticas}
  \end{Phonetics}
\end{Entry}

\begin{Entry}{风流}{4,10}{⾵、⽔}
  \begin{Phonetics}{风流}{feng1liu2}[][HSK 7-9]
    \definition{adj.}{refinado e saboroso | descontrolado em espírito e comportamento | romântico; amoroso; licencioso | distinto e admirável | meritório e talentoso | talentoso e romântico; talentoso em letras e não convencional no estilo de vida | dissoluto; solto}
  \end{Phonetics}
\end{Entry}

\begin{Entry}{风浪}{4,10}{⾵、⽔}
  \begin{Phonetics}{风浪}{feng1lang4}[][HSK 7-9]
    \definition{s.}{tempestade; ondas tempestuosas; vento e ondas na água | dificuldades; sofrimento; uma metáfora para uma experiência difícil ou perigosa}
  \end{Phonetics}
\end{Entry}

\begin{Entry}{风情}{4,11}{⾵、⼼}
  \begin{Phonetics}{风情}{feng1qing2}[][HSK 7-9]
    \definition{s.}{charme; bom temperamento; graça; expressão | charme; panorama; gosto elegante | charme; sentimentos amorosos; pornografia | costumes e práticas locais | a condição do vento; informações sobre direção e velocidade do vento | sensação; a sensação do ambiente circundante em um determinado ambiente}
  \end{Phonetics}
\end{Entry}

\begin{Entry}{风景}{4,12}{⾵、⽇}
  \begin{Phonetics}{风景}{feng1jing3}[][HSK 4]
    \definition[种,处,道]{s.}{cenário; paisagem; cenários e vistas que podem ser apreciados, inclui paisagens, flores, árvores, edifícios e determinados fenômenos naturais}
  \end{Phonetics}
\end{Entry}

\begin{Entry}{风筝}{4,12}{⾵、⽵}
  \begin{Phonetics}{风筝}{feng1zheng5}[][HSK 7-9]
    \definition[个,只]{s.}{pipa; papagaio; pandorga; é feito de tiras de bambu amarradas em um esqueleto em forma de pássaros, insetos, peixes, dragões, etc., coberto com papel ou seda, e flutua no ar com a ajuda do vento, as pessoas puxam a longa linha amarrada a ele para controlá-lo}
  \end{Phonetics}
\end{Entry}

\begin{Entry}{风貌}{4,14}{⾵、⾘}
  \begin{Phonetics}{风貌}{feng1mao4}[][HSK 7-9]
    \definition{s.}{estilo e características; o estilo e a aparência das coisas | vista; cena; cenário | aparência e porte elegantes; o comportamento e a aparência de uma pessoa}
  \end{Phonetics}
\end{Entry}

\begin{Entry}{风暴}{4,15}{⾵、⽇}
  \begin{Phonetics}{风暴}{feng1bao4}[][HSK 6]
    \definition{s.}{tempestade; vendaval; um termo geral para perturbações violentas na atmosfera e mudanças drásticas no clima, como tempestades de areia, tornados, ciclones tropicais, etc. | tempestade; comoção violenta; uma metáfora para um evento tão poderoso que abala toda a sociedade}
  \end{Phonetics}
\end{Entry}

\begin{Entry}{风趣}{4,15}{⾵、⾛}
  \begin{Phonetics}{风趣}{feng1qu4}[][HSK 7-9]
    \definition{adj.}{espirituoso; humorístico; (fala, texto, etc.) humorístico e interessante}
    \definition{s.}{sagacidade; humor; refere-se ao humor e ao gosto humorísticos e interessantes}
  \end{Phonetics}
\end{Entry}

\begin{Entry}{风餐露宿}{4,16,21,11}{⾵、⾷、⾬、⼧}
  \begin{Phonetics}{风餐露宿}{feng1can1-lu4su4}[][HSK 7-9]
    \definition{expr.}{comer ao vento e dormir no orvalho --- suportar as dificuldades de uma jornada árdua | o sol brilha no chão; para descrever as dificuldades de viajar ou viver ao ar livre, também é dito ``dormir ao ar livre''}
  \end{Phonetics}
\end{Entry}

%%%%% EOF %%%%%

