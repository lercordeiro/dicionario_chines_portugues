%%%
%%% 7画
%%%

\section*{7画}\addcontentsline{toc}{section}{7画}

\begin{Entry}{两}{7}{⼀}
  \begin{Phonetics}{两}{liang3}[][HSK 1,2]
    \definition*{s.}{Sobrenome Liang}
    \definition{clas.}{liang, uma unidade de peso (=50 gramas)}
    \definition{num.}{dois (sempre usado antes de classificadores) | poucos; alguns; indica um número indeterminado}
    \definition{s.}{ambos (lados); qualquer (lado)}
  \end{Phonetics}
\end{Entry}

\begin{Entry}{两手}{7,4}{⼀、⼿}
  \begin{Phonetics}{两手}{liang3 shou3}[][HSK 6]
    \definition{s.}{ambas as mãos | ambos os aspectos; táticas duplas | Coloquial: habilidade; capacidade}
  \end{Phonetics}
\end{Entry}

\begin{Entry}{两边}{7,5}{⼀、⾡}
  \begin{Phonetics}{两边}{liang3 bian1}[][HSK 4]
    \definition{s.}{ambos os lados; ambas as direções; ambos os lugares | ambas as partes; ambos os lados}
  \end{Phonetics}
\end{Entry}

\begin{Entry}{两侧}{7,8}{⼀、⼈}
  \begin{Phonetics}{两侧}{liang3 ce4}[][HSK 6]
    \definition{s.}{dois flancos; dois (ambos) lados; ambos}
  \end{Phonetics}
\end{Entry}

\begin{Entry}{两岸}{7,8}{⼀、⼭}
  \begin{Phonetics}{两岸}{liang3 an4}[][HSK 5]
    \definition{s.}{ambos os lados; ambas as margens; ambas as costas; entre os dois lados do estreito; bilateral}
  \end{Phonetics}
\end{Entry}

\begin{Entry}{两码事}{7,8,8}{⼀、⽯、⼅}
  \begin{Phonetics}{两码事}{liang3ma3shi4}
    \definition{expr.}{duas coisas completamente diferentes; dois assuntos diferentes}
  \end{Phonetics}
\end{Entry}

\begin{Entry}{严}{7}{⼀}
  \begin{Phonetics}{严}{yan2}[][HSK 4]
    \definition*{s.}{Sobrenome Yan}
    \definition{adj.}{apertado; próximo | rigoroso; severo; duro; áspero; rigoroso; austero | severo; extremo; difícil}
    \definition{s.}{pai; refere-se ao pai}
  \end{Phonetics}
\end{Entry}

\begin{Entry}{严厉}{7,5}{⼀、⼚}
  \begin{Phonetics}{严厉}{yan2li4}[][HSK 5]
    \definition{adj.}{severo; rigoroso; as palavras e atitudes de crítica ou punição são muito sérias e severas}
  \end{Phonetics}
\end{Entry}

\begin{Entry}{严肃}{7,8}{⼀、⾀}
  \begin{Phonetics}{严肃}{yan2su4}[][HSK 5]
    \definition{adj.}{sério; solene; sincero; (expressão, atmosfera, etc.) faz as pessoas se sentirem admiradas e desconfortáveis | sóbrio; grave; sério; sincero}
    \definition{v.}{aplicar rigorosamente; fazer algo sério}
  \end{Phonetics}
\end{Entry}

\begin{Entry}{严重}{7,9}{⼀、⾥}
  \begin{Phonetics}{严重}{yan2zhong4}[][HSK 4]
    \definition{adj.}{sério; grave; crítico; severo}
  \end{Phonetics}
\end{Entry}

\begin{Entry}{严重打伤}{7,9,5,6}{⼀、⾥、⼿、⼈}
  \begin{Phonetics}{严重打伤}{yan2zhong4 da3 shang1}
    \definition{s.}{gravemente ferido}
  \end{Phonetics}
\end{Entry}

\begin{Entry}{严重伤害}{7,9,6,10}{⼀、⾥、⼈、⼧}
  \begin{Phonetics}{严重伤害}{yan2zhong4 shang1hai4}
    \definition{s.}{ferimento grave; lesão grave}
  \end{Phonetics}
\end{Entry}

\begin{Entry}{严重关切}{7,9,6,4}{⼀、⾥、⼋、⼑}
  \begin{Phonetics}{严重关切}{yan2zhong4guan1qie4}
    \definition{s.}{preocupação séria}
  \end{Phonetics}
\end{Entry}

\begin{Entry}{严重危害}{7,9,6,10}{⼀、⾥、⼙、⼧}
  \begin{Phonetics}{严重危害}{yan2zhong4 wei1hai4}
    \definition{s.}{perigo crítico | dano grave}
  \end{Phonetics}
\end{Entry}

\begin{Entry}{严重后果}{7,9,6,8}{⼀、⾥、⼝、⽊}
  \begin{Phonetics}{严重后果}{yan2zhong4hou4guo3}
    \definition{s.}{consequências sérias | repercursões graves}
  \end{Phonetics}
\end{Entry}

\begin{Entry}{严重地}{7,9,6}{⼀、⾥、⼟}
  \begin{Phonetics}{严重地}{yan2zhong4 di4}
    \definition{adv.}{seriamente | gravemente}
  \end{Phonetics}
\end{Entry}

\begin{Entry}{严重问题}{7,9,6,15}{⼀、⾥、⾨、⾴}
  \begin{Phonetics}{严重问题}{yan2zhong4 wen4ti2}
    \definition{s.}{problema sério}
  \end{Phonetics}
\end{Entry}

\begin{Entry}{严重性}{7,9,8}{⼀、⾥、⼼}
  \begin{Phonetics}{严重性}{yan2zhong4xing4}
    \definition{s.}{seriedade | gravidade}
  \end{Phonetics}
\end{Entry}

\begin{Entry}{严重破坏}{7,9,10,7}{⼀、⾥、⽯、⼟}
  \begin{Phonetics}{严重破坏}{yan2zhong4 po4huai4}
    \definition{s.}{destruição grave}
  \end{Phonetics}
\end{Entry}

\begin{Entry}{严格}{7,10}{⼀、⽊}
  \begin{Phonetics}{严格}{yan2ge2}[][HSK 4]
    \definition{adj.}{rígido; estrito; rigoroso; muito consciente e meticuloso na implementação de sistemas e no domínio de padrões}
    \definition{v.}{tornar (sistemas, provisões, etc.) rigorosos}
  \end{Phonetics}
\end{Entry}

\begin{Entry}{串}{7}{⼁}
  \begin{Phonetics}{串}{chuan4}[][HSK 6]
    \definition{clas.}{corda; grupo; aglomerado; usado para amarrar coisas}
    \definition{s.}{espeto}
    \definition{v.}{atuar; desempenhar um papel (em uma peça) | misturar as coisas (de maneira caótica) | vagar; correr; ir de um lugar para outro | conspirar (conluio, depreciativo) | encadear juntos; conectar as coisas uma a uma para formar um todo | misturar; refere-se à mistura de coisas diferentes e à alteração de suas características originais}
  \end{Phonetics}
\end{Entry}

\begin{Entry}{串门}{7,3}{⼁、⾨}
  \begin{Phonetics}{串门}{chuan4/men2}[][HSK 7-9]
    \definition{v.+compl.}{aparecer; ligar para a casa de alguém; visitar a casa de um vizinho, amigo ou parente para conversar e cumprimentá-lo}
  \end{Phonetics}
\end{Entry}

\begin{Entry}{乱}{7}{⼄}
  \begin{Phonetics}{乱}{luan4}[][HSK 3]
    \definition{adj.}{em desordem; em confusão; em desarrumação; sem ordem nem organização | em um estado mental confuso | (de uma sociedade) turbulento; agitado | (de relações sexuais) impróprio; promíscuo}
    \definition{adv.}{aleatoriamente; arbitrariamente; indiscriminadamente; sem restrições; à vontade}
    \definition{s.}{motim; agitação; tumulto; revolta; guerra; calamidade}
    \definition{v.}{confundir; embaralhar; misturar; causar desordem}
  \end{Phonetics}
\end{Entry}

\begin{Entry}{亩}{7}{⼇}
  \begin{Phonetics}{亩}{mu3}
    \definition{clas.}{usado para campos | unidade de área igual a um décimo quinto de um hectare}
  \end{Phonetics}
\end{Entry}

\begin{Entry}{伯}{7}{⼈}
  \begin{Phonetics}{伯}{bai3}
  \seealsoref{大伯子}{da4 bai3zi5}
  \end{Phonetics}
  \begin{Phonetics}{伯}{bo2}
    \definition*{s.}{Conde; o terceiro dos cinco graus de nobreza | Sobrenome Bo}
    \definition{s.}{tio; o primeiro (mais velho) dos irmãos; o mais velho entre irmãos}
  \end{Phonetics}
\end{Entry}

\begin{Entry}{伯父}{7,4}{⼈、⽗}
  \begin{Phonetics}{伯父}{bo2fu4}[][HSK 7-9]
    \definition[个,位]{s.}{tio; irmão mais velho do pai | tio; termo usado para se referir a um homem um pouco mais velho que o pai; um título respeitoso para o pai de um colega ou colega de classe}
  \end{Phonetics}
\end{Entry}

\begin{Entry}{伯母}{7,5}{⼈、⽏}
  \begin{Phonetics}{伯母}{bo2mu3}[][HSK 7-9]
    \definition[个,位]{s.}{tia; esposa do irmão mais velho do pai; esposa do tio | forma educada de se dirigir a uma mulher que tem aproximadamente a mesma idade da mãe; um título respeitoso para a mãe de um colega ou colega de classe}
  \end{Phonetics}
\end{Entry}

\begin{Entry}{伯伯}{7,7}{⼈、⼈}
  \begin{Phonetics}{伯伯}{bo2bo5}[][HSK 7-9]
    \definition[个,位]{s.}{tio; irmão mais velho do pai | termo de tratamento para um homem da mesma geração do seu pai, mas mais velho que ele}
  \end{Phonetics}
\end{Entry}

\begin{Entry}{估}{7}{⼈}
  \begin{Phonetics}{估}{gu1}
    \definition{v.}{estimar; avaliar; aferir}
  \end{Phonetics}
  \begin{Phonetics}{估}{gu4}
    \definition{adj.}{velho | roupas de segunda mão}
  \end{Phonetics}
\end{Entry}

\begin{Entry}{估计}{7,4}{⼈、⾔}
  \begin{Phonetics}{估计}{gu1ji4}[][HSK 5]
    \definition{v.}{fazer contas; estimar; calcular; julgar a natureza, quantidade, mudança, etc. de uma coisa em uma determinada situação | parecer; parecer como se; aparentar; fazer inferências aproximadas sobre a natureza, a quantidade e a mudança das coisas com base em determinadas circunstâncias}
  \end{Phonetics}
\end{Entry}

\begin{Entry}{估算}{7,14}{⼈、⽵}
  \begin{Phonetics}{估算}{gu1suan4}[][HSK 7-9]
    \definition{v.}{calcular; estimar}
  \end{Phonetics}
\end{Entry}

\begin{Entry}{伲}{7}{⼈}
  \begin{Phonetics}{伲}{ni4}
    \definition{pron.}{Dialeto: eu; nós; meu; nosso}
  \seealsoref{你}{ni3}
  \end{Phonetics}
\end{Entry}

\begin{Entry}{伴}{7}{⼈}
  \begin{Phonetics}{伴}{ban4}[][HSK 7-9]
    \definition[个,位]{s.}{companheiro; parceiro}
    \definition{v.}{acompanhar; estar perto}[伴君如伴虎。===Acompanhar o rei é como acompanhar um tigre.]
  \end{Phonetics}
\end{Entry}

\begin{Entry}{伴侣}{7,8}{⼈、⼈}
  \begin{Phonetics}{伴侣}{ban4lv3}[][HSK 7-9]
    \definition[个,对]{s.}{companheiro; parceiro | parceiro; companheiro; refere-se a um casal ou a um dos casais}
  \end{Phonetics}
\end{Entry}

\begin{Entry}{伴奏}{7,9}{⼈、⼤}
  \begin{Phonetics}{伴奏}{ban4zou4}[][HSK 7-9]
    \definition{v.}{acompanhar (com instrumentos musicais); tocar (um instrumento musical) em conjunto com canto, dança ou apresentação solo, etc.}
  \end{Phonetics}
\end{Entry}

\begin{Entry}{伴随}{7,11}{⼈、⾩}
  \begin{Phonetics}{伴随}{ban4sui2}[][HSK 7-9]
    \definition{v.}{seguir; acompanhar}
  \end{Phonetics}
\end{Entry}

\begin{Entry}{伸}{7}{⼈}
  \begin{Phonetics}{伸}{shen1}[][HSK 5]
    \definition{v.}{alongar; esticar; estender}
  \end{Phonetics}
\end{Entry}

\begin{Entry}{伺}{7}{⼈}
  \begin{Phonetics}{伺}{ci4}
    \definition{v.}{servir}
  \seealsoref{伺候}{ci4hou5}
  \end{Phonetics}
  \begin{Phonetics}{伺}{si4}
    \definition{v.}{aguardar; observar; esperar por}
  \end{Phonetics}
\end{Entry}

\begin{Entry}{伺候}{7,10}{⼈、⼈}
  \begin{Phonetics}{伺候}{ci4hou5}[][HSK 7-9]
    \definition{v.}{servir; esperar; estar à disposição de alguém; cuidar de}
  \end{Phonetics}
\end{Entry}

\begin{Entry}{但}{7}{⼈}
  \begin{Phonetics}{但}{dan4}[][HSK 2]
    \definition*{s.}{Sobrenome Dan}
    \definition{adv.}{apenas; meramente; indica uma restrição ao âmbito da ação, equivalente a 只 ou 仅}
    \definition{conj.}{mas; ainda assim; mesmo assim; no entanto; contudo; usado na última oração, conecta duas orações, expressando uma relação de transição, equivalente a 可是 ou 不过}
  \seealsoref{不过}{bu2guo4}
  \seealsoref{仅}{jin3}
  \seealsoref{可是}{ke3shi4}
  \seealsoref{只}{zhi3}
  \end{Phonetics}
\end{Entry}

\begin{Entry}{但是}{7,9}{⼈、⽇}
  \begin{Phonetics}{但是}{dan4 shi4}[][HSK 2]
    \definition{conj.}{mas; contudo; no entanto; mesmo assim; usado na segunda parte da frase para indicar uma mudança, geralmente acompanhada de expressões como 虽然 ou 尽管}
  \seealsoref{尽管}{jin3guan3}
  \seealsoref{虽然}{sui1 ran2}
  \end{Phonetics}
\end{Entry}

\begin{Entry}{但愿}{7,14}{⼈、⽕}
  \begin{Phonetics}{但愿}{dan4yuan4}[][HSK 7-9]
    \definition{v.}{se ao menos (algo fosse possível); desejaria (que); espero (que)}
  \end{Phonetics}
\end{Entry}

\begin{Entry}{位}{7}{⼈}
  \begin{Phonetics}{位}{wei4}[][HSK 2]
    \definition*{s.}{Sobrenome Wei}
    \definition{clas.}{usado para pessoas (com cortesia, respeito) | usado para bits binários}[十六位===16 bits]
    \definition{s.}{lugar; localização; o lugar onde ou onde alguém está localizado | posto; \emph{status}; posição; a posição de uma pessoa em uma determinada área da vida social | trono; refere-se especificamente ao status do imperador | lugar; dígito; a posição de cada dígito em um número}
  \end{Phonetics}
\end{Entry}

\begin{Entry}{位于}{7,3}{⼈、⼆}
  \begin{Phonetics}{位于}{wei4yu2}[][HSK 4]
    \definition{v.}{estar localizado; estar situado}
  \end{Phonetics}
\end{Entry}

\begin{Entry}{位子}{7,3}{⼈、⼦}
  \begin{Phonetics}{位子}{wei4zi5}
    \definition{s.}{lugar | assento}
  \end{Phonetics}
\end{Entry}

\begin{Entry}{位居}{7,8}{⼈、⼫}
  \begin{Phonetics}{位居}{wei4ju1}
    \definition{v.}{estar localizado em}
  \end{Phonetics}
\end{Entry}

\begin{Entry}{位置}{7,13}{⼈、⽹}
  \begin{Phonetics}{位置}{wei4zhi4}[][HSK 4]
    \definition[个,处]{s.}{assento; lugar; localização | lugar; posição; \emph{status} | posição (por exemplo: cargo no escritório)}
  \end{Phonetics}
\end{Entry}

\begin{Entry}{低}{7}{⼈}
  \begin{Phonetics}{低}{di1}[][HSK 2]
    \definition*{s.}{Sobrenome Di}
    \definition{adj.}{baixo; distância pequena de baixo para cima; próximo ao solo | abaixo da média; abaixo do padrão geral | inferior (em grau); de nível inferior}
    \definition{v.}{deixar cair; pendurar; abaixar (a cabeça)}
  \end{Phonetics}
\end{Entry}

\begin{Entry}{低下}{7,3}{⼈、⼀}
  \begin{Phonetics}{低下}{di1xia4}[][HSK 7-9]
    \definition{adj.}{(\emph{status} ou padrões de vida) baixo; modesto; humilde | (gosto, etc.) vulgar; baixo}
  \end{Phonetics}
\end{Entry}

\begin{Entry}{低于}{7,3}{⼈、⼆}
  \begin{Phonetics}{低于}{di1 yu2}[][HSK 5]
    \definition{v.}{ser inferior a; algo ou fenômeno é, de alguma forma, inferior ou pior do que outra coisa}
  \end{Phonetics}
\end{Entry}

\begin{Entry}{低头}{7,5}{⼈、⼤}
  \begin{Phonetics}{低头}{di1 tou2}[][HSK 6]
    \definition{v.}{abaixar a cabeça; curvar a cabeça; pendurar a cabeça | ceder; submeter-se; refere-se à rendição e à admissão da derrota}
  \end{Phonetics}
\end{Entry}

\begin{Entry}{低价}{7,6}{⼈、⼈}
  \begin{Phonetics}{低价}{di1jia4}[][HSK 7-9]
    \definition{s.}{preço baixo}
  \end{Phonetics}
\end{Entry}

\begin{Entry}{低压}{7,6}{⼈、⼚}
  \begin{Phonetics}{低压}{di1ya1}
    \definition{s.}{Física: baixa pressão | Eletricidade: baixa tensão; baixa voltagem | Meteorologia: baixa pressão; depressão | Medicina: pressão diastólica, pressão mínima}
  \end{Phonetics}
\end{Entry}

\begin{Entry}{低估}{7,7}{⼈、⼈}
  \begin{Phonetics}{低估}{di1gu1}[][HSK 7-9]
    \definition{v.}{subestimar; menosprezar}
  \end{Phonetics}
\end{Entry}

\begin{Entry}{低谷}{7,7}{⼈、⾕}
  \begin{Phonetics}{低谷}{di1gu3}[][HSK 7-9]
    \definition{s.}{vale | maré baixa; uma metáfora para um estado de impróprio; o ponto mais baixo}[雾气笼罩着整个低谷。===A neblina cobria todo o vale.]
  \end{Phonetics}
\end{Entry}

\begin{Entry}{低空}{7,8}{⼈、⽳}
  \begin{Phonetics}{低空}{di1kong1}
    \definition{s.}{baixa altitude; baixo nível (oposto a高空 )}
  \seealsoref{高空}{gao1kong1}
  \end{Phonetics}
\end{Entry}

\begin{Entry}{低迷}{7,9}{⼈、⾡}
  \begin{Phonetics}{低迷}{di1mi2}[][HSK 7-9]
    \definition{adj.}{baixo; estagnado; deprimente; abatido | escuro; nebuloso; borrado}
  \end{Phonetics}
\end{Entry}

\begin{Entry}{低调}{7,10}{⼈、⾔}
  \begin{Phonetics}{低调}{di1diao4}[][HSK 7-9]
    \definition{adj.}{moderado; discreto; metaforicamente, abordagem discreta ou gentil}
    \definition{s.}{um perfil/tom/chave baixo; uma metáfora para pensamentos ou observações pessimistas e negativas}
  \end{Phonetics}
\end{Entry}

\begin{Entry}{低温}{7,12}{⼈、⽔}
  \begin{Phonetics}{低温}{di1 wen1}[][HSK 6]
    \definition{s.}{baixa temperatura | Meteorologia: microtermia  | Medicina: hipotermia}
  \end{Phonetics}
\end{Entry}

\begin{Entry}{低等}{7,12}{⼈、⽵}
  \begin{Phonetics}{低等}{di1 deng3}
    \definition{adj.}{inferior; classe baixa (oposto a 高等) | inferior}
  \seealsoref{高等}{gao1 deng3}
  \end{Phonetics}
\end{Entry}

\begin{Entry}{低落}{7,12}{⼈、⾋}
  \begin{Phonetics}{低落}{di1luo4}
    \definition{adj.}{desanimado; deprimido; abatido; muitas vezes se refere a mau humor}
    \definition{v.}{cair; declinar; diminuir; (preços, sentimento, etc.) cair rapidamente (em oposição a 高涨)}
  \seealsoref{高涨}{gao1zhang3}
  \end{Phonetics}
\end{Entry}

\begin{Entry}{低碳}{7,14}{⼈、⽯}
  \begin{Phonetics}{低碳}{di1tan4}[][HSK 7-9]
    \definition{adj.}{baixo carbono; nos últimos anos, termos como 低碳 e 碳足迹 tornaram-se termos populares tanto no país quanto no exterior; o ``carbono'' aqui se refere principalmente ao gás dióxido de carbono}
  \seealsoref{碳足迹}{tan4 zu2ji4}
  \end{Phonetics}
\end{Entry}

\begin{Entry}{低潮}{7,15}{⼈、⽔}
  \begin{Phonetics}{低潮}{di1chao2}
    \definition{s.}{maré baixa/vazante; o nível mais baixo da maré durante um ciclo de maré (distinto da 高潮) | vazante baixa; o ponto mais baixo; uma metáfora para o baixo estágio de desenvolvimento das coisas}
  \seealsoref{高潮}{gao1chao2}
  \end{Phonetics}
\end{Entry}

\begin{Entry}{住}{7}{⼈}
  \begin{Phonetics}{住}{zhu4}[][HSK 1]
    \definition{adv.}{firmemente; indica estabilidade ou firmeza}
    \definition{v.}{viver; residir; morar; ficar | parar; cessar | (após um verbo) com firmeza; até parar | hospedar; acomodar | parar; interromper | ser competente; ser qualificado; estar à altura; usado com 得 ou 不, indica que a força é suficiente (ou insuficiente)}
  \seealsoref{不}{bu4}
  \seealsoref{得}{de5}
  \end{Phonetics}
\end{Entry}

\begin{Entry}{住处}{7,5}{⼈、⼡}
  \begin{Phonetics}{住处}{zhu4chu4}
    \definition{s.}{morada | habitação | residência}
  \end{Phonetics}
\end{Entry}

\begin{Entry}{住宅}{7,6}{⼈、⼧}
  \begin{Phonetics}{住宅}{zhu4zhai2}[][HSK 6]
    \definition[套,处,栋,座]{s.}{habitação; residência}
  \end{Phonetics}
\end{Entry}

\begin{Entry}{住房}{7,8}{⼈、⼾}
  \begin{Phonetics}{住房}{zhu4fang2}[][HSK 2]
    \definition[套,处]{s.}{habitação; alojamento; casas para as pessoas morarem}
  \end{Phonetics}
\end{Entry}

\begin{Entry}{住所}{7,8}{⼈、⼾}
  \begin{Phonetics}{住所}{zhu4suo3}
    \definition[处]{s.}{morada | habitação | residência}
  \end{Phonetics}
\end{Entry}

\begin{Entry}{住院}{7,9}{⼈、⾩}
  \begin{Phonetics}{住院}{zhu4 yuan4}[][HSK 2]
    \definition{v.}{estar hospitalizado; estar no hospital; ser internado no hospital para tratamento}
  \end{Phonetics}
\end{Entry}

\begin{Entry}{住嘴}{7,16}{⼈、⼝}
  \begin{Phonetics}{住嘴}{zhu4zui3}
    \definition{interj.}{Cale-se!}
    \definition{v.}{calar | calar-se}
  \end{Phonetics}
\end{Entry}

\begin{Entry}{体}{7}{⼈}
  \begin{Phonetics}{体}{ti3}
    \definition{s.}{corpo; parte do corpo | substância; objeto; estado de uma substância | estilo; forma | sistema | estilo de caligrafia | tipo de letra; fonte | (linguística) aspecto (de um verbo) | estrutura; a forma escrita do texto; o gênero da obra}
    \definition{v.}{fazer ou vivenciar algo pessoalmente | colocar-se na posição de outro; colocar-se mentalmente na posição do outro; colocar-se no lugar do outro}
  \end{Phonetics}
\end{Entry}

\begin{Entry}{体力}{7,2}{⼈、⼒}
  \begin{Phonetics}{体力}{ti3 li4}[][HSK 5]
    \definition{s.}{força física; vigor físico (ou corporal); a força do corpo humano para sustentar suas próprias atividades}
  \end{Phonetics}
\end{Entry}

\begin{Entry}{体内}{7,4}{⼈、⼌}
  \begin{Phonetics}{体内}{ti3nei4}
    \definition{adj.}{dentro do corpo | \emph{in vivo} (versus \emph{in vitro} | interno a}
  \end{Phonetics}
\end{Entry}

\begin{Entry}{体会}{7,6}{⼈、⼈}
  \begin{Phonetics}{体会}{ti3hui4}[][HSK 3]
    \definition[个,些,种]{s.}{conhecimento; compreensão; experiência pessoal}
    \definition{v.}{perceber; saber (ou aprender) com a experiência}
  \end{Phonetics}
\end{Entry}

\begin{Entry}{体现}{7,8}{⼈、⾒}
  \begin{Phonetics}{体现}{ti3xian4}[][HSK 3]
    \definition{v.}{refletir; incorporar; encarnar; uma certa qualidade ou fenômeno se manifesta especificamente em uma determinada coisa}
  \end{Phonetics}
\end{Entry}

\begin{Entry}{体育}{7,8}{⼈、⾁}
  \begin{Phonetics}{体育}{ti3yu4}[][HSK 2]
    \definition{s.}{cultura física; treinamento físico; educação cuja principal tarefa é desenvolver a capacidade física e fortalecer a constituição física, alcançada através da participação em várias atividades esportivas | esportes; atividades esportivas; refere-se a esportes}
  \end{Phonetics}
\end{Entry}

\begin{Entry}{体育场}{7,8,6}{⼈、⾁、⼟}
  \begin{Phonetics}{体育场}{ti3 yu4 chang3}[][HSK 2]
    \definition[个,座]{s.}{estádio; campo esportivo; espaço ao ar livre para a prática de exercícios físicos ou competições esportivas}
  \end{Phonetics}
\end{Entry}

\begin{Entry}{体育馆}{7,8,11}{⼈、⾁、⾷}
  \begin{Phonetics}{体育馆}{ti3 yu4 guan3}[][HSK 2]
    \definition[个,座,家]{s.}{ginásio; locais esportivos ou competições em ambientes fechados geralmente têm arquibancadas fixas}
  \end{Phonetics}
\end{Entry}

\begin{Entry}{体重}{7,9}{⼈、⾥}
  \begin{Phonetics}{体重}{ti3 zhong4}[][HSK 4]
    \definition{s.}{peso corporal}
  \end{Phonetics}
\end{Entry}

\begin{Entry}{体积}{7,10}{⼈、⽲}
  \begin{Phonetics}{体积}{ti3ji1}[][HSK 5]
    \definition[个]{s.}{volume; quantidade; o tamanho do espaço ocupado pelo objeto}
  \end{Phonetics}
\end{Entry}

\begin{Entry}{体验}{7,10}{⼈、⾺}
  \begin{Phonetics}{体验}{ti3yan4}[][HSK 3]
    \definition[种]{s.}{experiência; a sensação adquirida pela experiência pessoal}
    \definition{v.}{aprender através da prática; aprender através da experiência pessoal; entender as coisas através da experiência pessoal}
  \end{Phonetics}
\end{Entry}

\begin{Entry}{体检}{7,11}{⼈、⽊}
  \begin{Phonetics}{体检}{ti3 jian3}[][HSK 4]
    \definition{v.}{fazer um exame médico}
  \end{Phonetics}
\end{Entry}

\begin{Entry}{体操}{7,16}{⼈、⼿}
  \begin{Phonetics}{体操}{ti3 cao1}[][HSK 4]
    \definition{s.}{ginástica; esportes, exercícios ou performances de vários movimentos, sem armas ou com o auxílio de determinados equipamentos}
  \end{Phonetics}
\end{Entry}

\begin{Entry}{何}{7}{⼈}
  \begin{Phonetics}{何}{he2}
    \definition*{s.}{Sobrenome He}
    \definition{adv.}{enfatiza um alto grau de intensidade, equivalente a 多么}
    \definition{pron.}{O que?; Qual?; em nome de pessoas ou coisas, equivalente a 什么 | Onde?; em nome do lugar, equivalente a 哪里 | Por que?; Como?; a razão, é equivalente a 为什么 e 怎么}
  \seealsoref{多么}{duo1me5}
  \seealsoref{哪里}{na3 li3}
  \seealsoref{岂}{qi3}
  \seealsoref{什么}{shen2me5}
  \seealsoref{为什么}{wei4shen2me5}
  \seealsoref{怎}{zen3}
  \seealsoref{怎么}{zen3me5}
  \end{Phonetics}
\end{Entry}

\begin{Entry}{何不}{7,4}{⼈、⼀}
  \begin{Phonetics}{何不}{he2bu4}
    \definition{adv.}{por que não?; use o tom interrogativo para expressar "deveria" ou "pode"}
  \end{Phonetics}
\end{Entry}

\begin{Entry}{何处}{7,5}{⼈、⼡}
  \begin{Phonetics}{何处}{he2chu4}[][HSK 7-9]
    \definition{pron.}{onde; que lugar}[你要去何处?===Onde você está indo?]
  \end{Phonetics}
\end{Entry}

\begin{Entry}{何必}{7,5}{⼈、⼼}
  \begin{Phonetics}{何必}{he2bi4}[][HSK 7-9]
    \definition{adv.}{por que?; não há necessidade; use um tom interrogativo para expressar que não é necessário}[你何必如此匆忙?===Por que você está com tanta pressa?]
  \end{Phonetics}
\end{Entry}

\begin{Entry}{何况}{7,7}{⼈、⼎}
  \begin{Phonetics}{何况}{he2kuang4}[][HSK 7-9]
    \definition{conj.}{muito menos; use um tom interrogativo para expressar que é mais óbvio e razoável em comparação | além disso; além do mais; indique outras razões ou razões adicionais}
  \end{Phonetics}
\end{Entry}

\begin{Entry}{何时}{7,7}{⼈、⽇}
  \begin{Phonetics}{何时}{he2shi2}[][HSK 7-9]
    \definition{adv.}{quando?; que horas?}[我何时能再见到你?===Quando poderei vê-lo novamente?]
  \end{Phonetics}
\end{Entry}

\begin{Entry}{何典}{7,8}{⼈、⼋}
  \begin{Phonetics}{何典}{he2 dian3}
    \definition*{s.}{He Dian; este é um romance clássico chinês único, com uma arte inesquecível e um estilo humorístico único; o romance satiriza o submundo}
  \end{Phonetics}
\end{Entry}

\begin{Entry}{何苦}{7,8}{⼈、⾋}
  \begin{Phonetics}{何苦}{he2ku3}[][HSK 7-9]
    \definition{adv.}{por que se preocupar?; vale a pena o esforço?; use uma pergunta retórica para expressar que não vale a pena e por que se preocupar}[为了这点事,何苦生气呢?===Por que ficar bravo com uma coisa dessas?]
  \end{Phonetics}
\end{Entry}

\begin{Entry}{何故}{7,9}{⼈、⽁}
  \begin{Phonetics}{何故}{he2gu4}
    \definition{adv.}{qual razão?; por que? | para quê? | qual é o motivo?}
  \end{Phonetics}
\end{Entry}

\begin{Entry}{佛}{7}{⼈}
  \begin{Phonetics}{佛}{fo2}[][HSK 6]
    \definition*{s.}{Buda, abreviação de 佛陀 | Budismo}
    \definition{s.}{imagem de Buda | budista | nome de Buda; escritura budista | uma pessoa que alcançou a perfeição na prática espiritual; budista real | estátua do Buda}
  \seealsoref{佛陀}{fo2tuo2}
  \end{Phonetics}
  \begin{Phonetics}{佛}{fu2}
    \definition{adv.}{aparentemente}
    \definition{s.}{ornamento da cabeça (feminino)}
  \end{Phonetics}
\end{Entry}

\begin{Entry}{佛陀}{7,7}{⼈、⾩}
  \begin{Phonetics}{佛陀}{fo2tuo2}
    \definition{s.}{Buda, um título para Sakyamuni ou uma pessoa que atingiu a iluminação | Buda, uma pessoa que atingiu a Budeidade, ou especificamente Siddhartha Gautama}
  \end{Phonetics}
\end{Entry}

\begin{Entry}{佛教}{7,11}{⼈、⽁}
  \begin{Phonetics}{佛教}{fo2 jiao4}[][HSK 6]
    \definition*{s.}{Budismo; uma das principais religiões do mundo, diz-se que foi fundada por Sakyamuni, um príncipe do antigo reino indiano de Kapilavastu (no atual Nepal), no século VI ou V a.C.; foi amplamente difundida em muitos países asiáticos e introduzida na China no final da Dinastia Han Ocidental}
  \end{Phonetics}
\end{Entry}

\begin{Entry}{作}{7}{⼈}
  \begin{Phonetics}{作}{zuo1}
    \definition{adj.}{(gíria) incômodo}
    \definition{s.}{trabalhador | oficina | (pessoa) de alta manutenção}
  \end{Phonetics}
  \begin{Phonetics}{作}{zuo4}[][HSK 6]
    \definition*{s.}{Sobrenome Zuo}
    \definition{s.}{trabalho; escritos; obras}
    \definition{v.}{subir; aumentar; crescer; aparecer | escrever; compor; criar | afetar; fingir; assumir; fingir deliberadamente ser uma determinada pessoa | considerar como; tomar algo ou alguém por | ter; sentir | ser; agir como; tornar-se; servir como | envolver-se em uma atividade; realizar alguma atividade | fazer; criar}
  \end{Phonetics}
\end{Entry}

\begin{Entry}{作为}{7,4}{⼈、⼂}
  \begin{Phonetics}{作为}{zuo4wei2}[][HSK 4]
    \definition{prep.}{como; na capacidade de; no caráter de; no papel de; em termos de uma certa identidade de uma pessoa ou de uma certa natureza de uma coisa}
    \definition{s.}{ato; ação; conduta; feito; comportamento | conquista; realização; especificamente, uma boa ação}
    \definition{v.}{considerar como; tomar por; olhar como; tratar como | realizar; fazer conquistas; deixar uma marca}
  \end{Phonetics}
\end{Entry}

\begin{Entry}{作文}{7,4}{⼈、⽂}
  \begin{Phonetics}{作文}{zuo4/wen2}[][HSK 2]
    \definition[篇]{s.}{ensaio; composição; redação}
    \definition{v.+compl.}{(de alunos) escrever uma redação, artigo ou ensaio}
  \end{Phonetics}
\end{Entry}

\begin{Entry}{作业}{7,5}{⼈、⼀}
  \begin{Phonetics}{作业}{zuo4ye4}[][HSK 2]
    \definition[份,个]{s.}{tarefa escolar; tarefa de casa atribuída pelos professores aos alunos}
    \definition{v.}{trabalhar; executar tarefa}
  \end{Phonetics}
\end{Entry}

\begin{Entry}{作出}{7,5}{⼈、⼐}
  \begin{Phonetics}{作出}{zuo4 chu1}[][HSK 4]
    \definition{v.}{mostrar; tomar (decisões, conclusões, etc. por meio de consideração ou discussão); formar (uma conclusão, decisão, etc.) por meio de consideração ou discussão}
  \end{Phonetics}
\end{Entry}

\begin{Entry}{作用}{7,5}{⼈、⽤}
  \begin{Phonetics}{作用}{zuo4yong4}[][HSK 2]
    \definition[副]{s.}{efeito; ação; função; a influência sobre as coisas; o efeito; a utilidade}
    \definition{v.}{afetar; agir sobre; realizar atividades que têm algum impacto nas coisas}
  \end{Phonetics}
\end{Entry}

\begin{Entry}{作废}{7,8}{⼈、⼴}
  \begin{Phonetics}{作废}{zuo4fei4}[][HSK 6]
    \definition{v.}{anular; tornar inválido; abandonar devido a falha}
  \end{Phonetics}
\end{Entry}

\begin{Entry}{作者}{7,8}{⼈、⽼}
  \begin{Phonetics}{作者}{zuo4zhe3}[][HSK 3]
    \definition[位,名,个]{s.}{autor; escritor; pessoas que escrevem artigos ou criam obras de arte}
  \end{Phonetics}
\end{Entry}

\begin{Entry}{作品}{7,9}{⼈、⼝}
  \begin{Phonetics}{作品}{zuo4pin3}[][HSK 3]
    \definition[个,部,篇,幅]{s.}{obra de arte; obras literárias e artísticas}
  \end{Phonetics}
\end{Entry}

\begin{Entry}{作战}{7,9}{⼈、⼽}
  \begin{Phonetics}{作战}{zuo4 zhan4}[][HSK 6]
    \definition{s.}{lutar; combater; batalhar}
  \end{Phonetics}
\end{Entry}

\begin{Entry}{作家}{7,10}{⼈、⼧}
  \begin{Phonetics}{作家}{zuo4jia1}[][HSK 2]
    \definition[位,名,个,些]{s.}{escritor; autor; pessoas que alcançaram sucesso na criação literária}
  \end{Phonetics}
\end{Entry}

\begin{Entry}{你}{7}{⼈}
  \begin{Phonetics}{你}{ni3}[][HSK 1]
    \definition{pron.}{você (segunda pessoa do singular); refere-se à pessoa com quem se está conversando | (referindo-se a qualquer pessoa) você; um; qualquer um | com 我 ou 你 em estruturas paralelas para indicar várias ou muitas pessoas se comportando da mesma maneira}
  \seealsoref{您}{nin2}
  \seealsoref{我}{wo3}
  \end{Phonetics}
\end{Entry}

\begin{Entry}{你们}{7,5}{⼈、⼈}
  \begin{Phonetics}{你们}{ni3men5}[][HSK 1]
    \definition{pron.}{você (segunda pessoa do plural); refere-se a mais de uma pessoa ou a várias pessoas, incluindo a outra parte}
  \end{Phonetics}
\end{Entry}

\begin{Entry}{你们的}{7,5,8}{⼈、⼈、⽩}
  \begin{Phonetics}{你们的}{ni3men5 de5}
    \definition{pron.}{vossos}
  \end{Phonetics}
\end{Entry}

\begin{Entry}{你好}{7,6}{⼈、⼥}
  \begin{Phonetics}{你好}{ni3hao3}
    \definition{interj.}{Olá! | Oi!}
  \end{Phonetics}
\end{Entry}

\begin{Entry}{你的}{7,8}{⼈、⽩}
  \begin{Phonetics}{你的}{ni3 de5}
    \definition{pron.}{seu}
  \end{Phonetics}
\end{Entry}

\begin{Entry}{克}{7}{⼗}
  \begin{Phonetics}{克}{ke4}[][HSK 2]
    \definition*{s.}{Sobrenome Ke}
    \definition{clas.}{g, grama, unidade de peso | unidade tibetana de volume ou medida seca (com capacidade para cerca de 25 斤, de cevada) | unidade tibetana de área de terra equivalente a cerca de 1 亩}
    \definition{v.}{poder; ser capaz de | tolerar; conter; restringir; suprimir| subjugar; capturar; conquistar (uma cidade, etc.) | digerir (alimentos) | reduzir; diminuir | definir um limite de tempo}
  \seealsoref{斤}{jin1}
  \seealsoref{亩}{mu3}
  \end{Phonetics}
\end{Entry}

\begin{Entry}{克服}{7,8}{⼗、⽉}
  \begin{Phonetics}{克服}{ke4fu2}[][HSK 3]
    \definition{v.}{sobrepujar; superar; conquistar; vencer com força de vontade e determinação (deficiências, erros, fenômenos negativos, condições desfavoráveis, etc.) | aguentar; suportar (dificuldades, inconveniências, etc.)}
  \end{Phonetics}
\end{Entry}

\begin{Entry}{免}{7}{⼉}
  \begin{Phonetics}{免}{mian3}
    \definition*{s.}{Sobrenome Mian}
    \definition{v.}{desculpar alguém de algo; isentar; dispensar; renunciar | remover do cargo; demitir | evitar; desviar; escapar | não deveria ser permitido; não precisar fazer algo | remover; livrar-se de | isentar; dispensar | não permitir}
  \end{Phonetics}
\end{Entry}

\begin{Entry}{免费}{7,9}{⼉、⾙}
  \begin{Phonetics}{免费}{mian3/fei4}[][HSK 4]
    \definition{v.+compl.}{isentar de taxas; tonar grátis}
  \end{Phonetics}
\end{Entry}

\begin{Entry}{免得}{7,11}{⼉、⼻}
  \begin{Phonetics}{免得}{mian3de5}[][HSK 6]
    \definition{conj.}{de modo a não; para evitar; para que não; indica evitar uma situação que não é desejável e é frequentemente usado no início da oração seguinte}
  \end{Phonetics}
\end{Entry}

\begin{Entry}{免税}{7,12}{⼉、⽲}
  \begin{Phonetics}{免税}{mian3/shui4}
    \definition{adj.}{isento de impostos (tributação)}
    \definition{s.}{livre de impostos | isenção de impostos}
    \definition{v.+compl.}{isentar impostos}
  \end{Phonetics}
\end{Entry}

\begin{Entry}{兑}{7}{⼉}
  \begin{Phonetics}{兑}{dui4}
    \definition*{s.}{Sobrenome Dui}
    \definition{s.}{dui, um dos Oito Trigramas que representa pântano}
    \definition{v.}{trocar; converter | adicionar (água, etc.) | sacar; pagar ou receber dinheiro por fatura}
  \end{Phonetics}
\end{Entry}

\begin{Entry}{兑现}{7,8}{⼉、⾒}
  \begin{Phonetics}{兑现}{dui4xian4}[][HSK 7-9]
    \definition{v.}{sacar (dinheiro, cheques, etc.) | cumprir; fazer o bem; honrar (um compromisso, etc.); metáfora para cumprir uma promessa}
  \end{Phonetics}
\end{Entry}

\begin{Entry}{兑换}{7,10}{⼉、⼿}
  \begin{Phonetics}{兑换}{dui4huan4}[][HSK 7-9]
    \definition{v.}{converter; trocar; trocar uma moeda por outra; trocar um cheque, etc., por dinheiro}
  \end{Phonetics}
\end{Entry}

\begin{Entry}{兵}{7}{⼋}
  \begin{Phonetics}{兵}{bing1}[][HSK 4]
    \definition[个,种]{s.}{armas; armamentos | soldado; pessoal militar | exército; tropas | soldado raso; membro mais jovem do exército | assuntos militares (estratégia) | peão, uma das peças do xadrez chinês}
  \end{Phonetics}
\end{Entry}

\begin{Entry}{兵器}{7,16}{⼋、⼝}
  \begin{Phonetics}{兵器}{bing1qi4}
    \definition{s.}{armas | armamento}
  \end{Phonetics}
\end{Entry}

\begin{Entry}{况}{7}{⼎}
  \begin{Phonetics}{况}{kuang4}
    \definition*{s.}{Sobrenome Kuang}
    \definition{conj.}{além disso | mesmo; muito menos; sem mencionar}
    \definition{s.}{condição; situação}
    \definition{v.}{comparar}
  \end{Phonetics}
\end{Entry}

\begin{Entry}{况且}{7,5}{⼎、⼀}
  \begin{Phonetics}{况且}{kuang4qie3}
    \definition{conj.}{além disso; além do mais; orações de conexão para expressar uma relação progressiva}
  \end{Phonetics}
\end{Entry}

\begin{Entry}{冷}{7}{⼎}
  \begin{Phonetics}{冷}{leng3}[][HSK 1]
    \definition*{s.}{Sobrenome Leng}
    \definition{adj.}{frio; baixa temperatura; sensação de frio | gelado; frio por natureza; sem entusiasmo; sem gentileza | desolado; pouco frequentado; quieto; sem agitação | negligenciado; indesejável; ignorado por todos | raro; estranho; incomum | feito em segredo; filmado de forma escondida; lançado secretamente}
    \definition{v.}{esfriar; resfriar | esfriar; congelar; tornar-se indiferente, apático | ignorar}
  \end{Phonetics}
\end{Entry}

\begin{Entry}{冷门}{7,3}{⼎、⾨}
  \begin{Phonetics}{冷门}{leng3men2}
    \definition{s.}{uma profissão, ofício ou ramo de aprendizagem que recebe pouca atenção | um vencedor inesperado; azarão}
  \end{Phonetics}
\end{Entry}

\begin{Entry}{冷气}{7,4}{⼎、⽓}
  \begin{Phonetics}{冷气}{leng3 qi4}[][HSK 6]
    \definition[股,阵]{s.}{ar frio (ou fresco); correntes de ar frio | ar condicionado; ar resfriado por equipamento de refrigeração | ar condicionado; equipamentos de ar condicionado}
  \end{Phonetics}
\end{Entry}

\begin{Entry}{冷水}{7,4}{⼎、⽔}
  \begin{Phonetics}{冷水}{leng3 shui3}[][HSK 6]
    \definition[杯,瓶]{s.}{água fria | água não fervida}
  \end{Phonetics}
\end{Entry}

\begin{Entry}{冷静}{7,14}{⼎、⾭}
  \begin{Phonetics}{冷静}{leng3jing4}[][HSK 4]
    \definition{adj.}{calmo; descreve uma pessoa que consegue ficar atenta em uma situação importante ou de emergência e não toma decisões aleatórias por causa de seus sentimentos no momento | (lugar) tranquilo; quieto; deserto}
  \end{Phonetics}
\end{Entry}

\begin{Entry}{冻}{7}{⼎}
  \begin{Phonetics}{冻}{dong4}[][HSK 5]
    \definition*{s.}{Sobrenome Dong}
    \definition{s.}{geleia; gelatina}
    \definition{v.}{congelar; ser congelado | ficar com frio ou sentir frio}
  \end{Phonetics}
\end{Entry}

\begin{Entry}{冻结}{7,9}{⼎、⽷}
  \begin{Phonetics}{冻结}{dong4jie2}[][HSK 7-9]
    \definition{adj.}{congelado}
    \definition{s.}{(salários, preços, etc.) congelamento; bloqueio de fluxo ou mudança}
    \definition{v.}{congelar}
  \end{Phonetics}
\end{Entry}

\begin{Entry}{初}{7}{⾐}
  \begin{Phonetics}{初}{chu1}[][HSK 3]
    \definition*{s.}{Sobrenome Chu}
    \definition{adj.}{primeiro (em ordem) | elementar; rudimentar | original}
    \definition{adv.}{pela primeira vez; apenas começando; indica que a ação está ocorrendo pela primeira vez ou acabou de começar}
    \definition{pref.}{anexado a alguns substantivos ou números cardinais, indicando o primeiro}
    \definition{s.}{no início de; na primeira parte de | o estágio júnior (pleno; sênior)}
  \end{Phonetics}
\end{Entry}

\begin{Entry}{初中}{7,4}{⾐、⼁}
  \begin{Phonetics}{初中}{chu1 zhong1}[][HSK 3]
    \definition[所,个]{s.}{ensino médio; ensino fundamental}
  \end{Phonetics}
\end{Entry}

\begin{Entry}{初心}{7,4}{⾐、⼼}
  \begin{Phonetics}{初心}{chu1xin1}
    \definition{s.}{intenção original de alguém, aspiração, etc. | (budismo) ``mente do iniciante'' (ter a mente aberta quando estudando um assunto como um iniciante no assunto teria)}
  \end{Phonetics}
\end{Entry}

\begin{Entry}{初次}{7,6}{⾐、⽋}
  \begin{Phonetics}{初次}{chu1ci4}[][HSK 7-9]
    \definition{adv.}{a primeira vez}
  \end{Phonetics}
\end{Entry}

\begin{Entry}{初级}{7,6}{⾐、⽷}
  \begin{Phonetics}{初级}{chu1ji2}[][HSK 3]
    \definition{adj.}{elementar; primário; júnior; inicial; o nível mais baixo; de baixa qualidade}
  \end{Phonetics}
\end{Entry}

\begin{Entry}{初步}{7,7}{⾐、⽌}
  \begin{Phonetics}{初步}{chu1bu4}[][HSK 3]
    \definition{adj.}{inicial; preliminar; imaturo, incompleto}
  \end{Phonetics}
\end{Entry}

\begin{Entry}{初衷}{7,10}{⾐、⾐}
  \begin{Phonetics}{初衷}{chu1zhong1}[][HSK 7-9]
    \definition{s.}{intenção original}
  \end{Phonetics}
\end{Entry}

\begin{Entry}{初期}{7,12}{⾐、⽉}
  \begin{Phonetics}{初期}{chu1 qi1}[][HSK 5]
    \definition{s.}{primórdio; estágio inicial; primeiros dias; estágio preliminar; período inicial}
  \end{Phonetics}
\end{Entry}

\begin{Entry}{初等}{7,12}{⾐、⽵}
  \begin{Phonetics}{初等}{chu1 deng3}[][HSK 6]
    \definition{adj.}{elementar; primário; rudimentar| elementar (ou seja, fácil)}
  \end{Phonetics}
\end{Entry}

\begin{Entry}{判}{7}{⼑}
  \begin{Phonetics}{判}{pan4}[][HSK 6]
    \definition{adv.}{obviamente há uma diferença}
    \definition{v.}{distinguir; discriminar; separar | julgar; decidir; avaliar | sentenciar; condenar}
  \end{Phonetics}
\end{Entry}

\begin{Entry}{判断}{7,11}{⼑、⽄}
  \begin{Phonetics}{判断}{pan4duan4}[][HSK 3]
    \definition[个,项]{s.}{julgamento; conclusões tiradas após reflexão e análise}
    \definition{v.}{julgar; decidir}
  \end{Phonetics}
\end{Entry}

\begin{Entry}{利}{7}{⼑}
  \begin{Phonetics}{利}{li4}[][HSK 6]
    \definition*{s.}{Sobrenome Li}
    \definition{adj.}{afiado; cortante | favorável; conveniente; sem dificuldades; sem ou com poucas dificuldades}
    \definition{s.}{benefício; vantagem | lucro; ganhos; juros}
    \definition{v.}{beneficiar; tornar vantajoso}
  \end{Phonetics}
\end{Entry}

\begin{Entry}{利用}{7,5}{⼑、⽤}
  \begin{Phonetics}{利用}{li4yong4}[][HSK 3]
    \definition{v.}{usar; utilizar; fazer uso de; fazer com que algo ou alguém funcione bem| explorar; tirar vantagem de; usar meios para fazer com que pessoas ou coisas sirvam aos seus interesses}
  \end{Phonetics}
\end{Entry}

\begin{Entry}{利息}{7,10}{⼑、⼼}
  \begin{Phonetics}{利息}{li4xi1}[][HSK 4]
    \definition{s.}{acréscimo; juros; dinheiro recebido além do valor principal como resultado de depósitos ou empréstimos (diferenciado de 本金)}
  \seealsoref{本金}{ben3 jin1}
  \end{Phonetics}
\end{Entry}

\begin{Entry}{利润}{7,10}{⼑、⽔}
  \begin{Phonetics}{利润}{li4run4}[][HSK 5]
    \definition[笔,份]{s.}{lucro; o dinheiro ganho com atividades comerciais e industriais}
  \end{Phonetics}
\end{Entry}

\begin{Entry}{利益}{7,10}{⼑、⽫}
  \begin{Phonetics}{利益}{li4yi4}[][HSK 4]
    \definition[个,种]{s.}{ganho; lucro; juros; benefício}
  \end{Phonetics}
\end{Entry}

\begin{Entry}{别}{7}{⼑}
  \begin{Phonetics}{别}{bie2}[][HSK 1,4]
    \definition*{s.}{Sobrenome Bie}
    \definition{adv.}{não; nada de (pedir a alguém para não fazer); é melhor não | talvez, usado em conjunto com a palavra 是 para indicar especulação}
    \definition{pron.}{outro; algum outro}
    \definition{s.}{distinção; diferença | classificação}
    \definition{v.}{deixar; partir; separar | diferenciar; distinguir; encontrar aspectos diferentes | fixar objetos com pinos | girar; virar | aderir; colar; preder}
  \seealsoref{是}{shi4}
  \end{Phonetics}
  \begin{Phonetics}{别}{bie4}
    \definition{v.}{fazer com que alguém mude seus hábitos, opiniões, etc. | mudar a opinião de alguém (usado principalmente em 别不过)}
  \seealsoref{别不过}{bie2 bu2guo4}
  \end{Phonetics}
\end{Entry}

\begin{Entry}{别人}{7,2}{⼑、⼈}
  \begin{Phonetics}{别人}{bie2 ren2}[][HSK 1]
    \definition{pron.}{outros; outras pessoas}
    \definition{s.}{outros; pessoas; outras pessoas; refere-se a alguém diferente de si mesmo}
  \end{Phonetics}
\end{Entry}

\begin{Entry}{别不过}{7,4,6}{⼑、⼀、⾡}
  \begin{Phonetics}{别不过}{bie2 bu2guo4}
    \definition{expr.}{Não, mas}
  \end{Phonetics}
\end{Entry}

\begin{Entry}{别扭}{7,7}{⼑、⼿}
  \begin{Phonetics}{别扭}{bie4niu5}[][HSK 7-9]
    \definition{adj.}{estranho; desconfortável | desajeitado (ao falar ou escrever); não suave; não fluente | nervoso e desconfortável; muito nervoso; muito antinatural}
    \definition{v.}{ser difícil com alguém; relacionamentos ruins, opiniões diferentes, conflitos frequentes; falta de coordenação}
  \end{Phonetics}
\end{Entry}

\begin{Entry}{别具匠心}{7,8,6,4}{⼑、⼋、⼕、⼼}
  \begin{Phonetics}{别具匠心}{bie2ju4-jiang4xin1}[][HSK 7-9]
    \definition{expr.}{único e engenhoso | mostrar engenhosidade; ter originalidade}
  \end{Phonetics}
\end{Entry}

\begin{Entry}{别的}{7,8}{⼑、⽩}
  \begin{Phonetics}{别的}{bie2 de5}[][HSK 1]
    \definition{pron.}{outro; o resto}
  \end{Phonetics}
\end{Entry}

\begin{Entry}{别看}{7,9}{⼑、⽬}
  \begin{Phonetics}{别看}{bie2kan4}[][HSK 7-9]
    \definition{conj.}{apesar de}
  \end{Phonetics}
\end{Entry}

\begin{Entry}{别说}{7,9}{⼑、⾔}
  \begin{Phonetics}{别说}{bie2shuo1}[][HSK 7-9]
    \definition{v.}{Coloquial: não dizer nada de; não mencionar; deixar sozinho, usado no início de uma frase para reconhecer a declaração seguinte}[别说在下雨, 你现在出去也太晚了。===Não se preocupe com a chuva, já é tarde demais para você sair.]
  \end{Phonetics}
\end{Entry}

\begin{Entry}{别致}{7,10}{⼑、⾄}
  \begin{Phonetics}{别致}{bie2zhi4}[][HSK 7-9]
    \definition{adj.}{único; não convencional; novo, diferente do comum}
  \end{Phonetics}
\end{Entry}

\begin{Entry}{别提了}{7,12,2}{⼑、⼿、⼅}
  \begin{Phonetics}{别提了}{bie2ti2 le5}[][HSK 7-9]
    \definition{v.}{não mencionar mais isso, não falar mais sobre isso}
  \end{Phonetics}
\end{Entry}

\begin{Entry}{别墅}{7,14}{⼑、⼟}
  \begin{Phonetics}{别墅}{bie2shu4}[][HSK 7-9]
    \definition[栋,幢,座,套,个]{s.}{vila; mansão; residência de campo}
  \end{Phonetics}
\end{Entry}

\begin{Entry}{助}{7}{⼒}
  \begin{Phonetics}{助}{zhu4}
    \definition{v.}{ajudar; auxiliar; acudir; apoiar}
  \end{Phonetics}
\end{Entry}

\begin{Entry}{助手}{7,4}{⼒、⼿}
  \begin{Phonetics}{助手}{zhu4shou3}[][HSK 5]
    \definition[个,位,种,名]{s.}{ajudante; auxiliar; assistente; alguém que ajuda os outros com seu trabalho}
  \end{Phonetics}
\end{Entry}

\begin{Entry}{助兴}{7,6}{⼒、⼋}
  \begin{Phonetics}{助兴}{zhu4/xing4}
    \definition{v.+compl.}{animar as coisas | juntar-se à diversão}
  \end{Phonetics}
\end{Entry}

\begin{Entry}{助理}{7,11}{⼒、⽟}
  \begin{Phonetics}{助理}{zhu4li3}[][HSK 5]
    \definition[个,位,名,些]{s.}{deputado; assistente; auxiliar do diretor responsável (geralmente usado em cargos) | ajudante; assistente; pessoa que auxilia o responsável a fazer as coisas}
  \end{Phonetics}
\end{Entry}

\begin{Entry}{努}{7}{⼒}
  \begin{Phonetics}{努}{nu3}
    \definition{v.}{(coloquial) aplicar (a força de alguém); exercer (o esforço de alguém) | (dialeto) machucar-se por esforço excessivo | projetar-se; inchar | aplicar (força); exercer (esforço); usar}
  \seealsoref{呶}{nao2}
  \end{Phonetics}
\end{Entry}

\begin{Entry}{努力}{7,2}{⼒、⼒}
  \begin{Phonetics}{努力}{nu3li4}[][HSK 2]
    \definition{adj.}{extenuante; árduo | diligente; trabalhador; quem faz as coisas com o máximo de capacidade ou esforço possível}
    \definition{s.}{esforço; tentativa; fazer o melhor possível}
    \definition{v.}{fazer grandes esforços; esforçar-se; empenhar-se | esforçar-se; usar toda a força possível}
  \end{Phonetics}
\end{Entry}

\begin{Entry}{劳}{7}{⼒}
  \begin{Phonetics}{劳}{lao2}
    \definition*{s.}{Sobrenome Lao}
    \definition{adj.}{difícil; cansativo; cansado}
    \definition{s.}{fadiga; trabalho árduo | ação meritória; serviço; conquistas | trabalhador | mérito | trabalhador braçal}
    \definition{v.}{trabalho; labor | esforço; exercício intenso | (pedir um favor a alguém, também 有劳) colocar alguém no trabalho de | expressar apreço (ao executor de uma tarefa); recompensar | colocar alguém no trabalho de; incomodar alguém com algo | trazer presentes para}
  \seealsoref{有劳}{you3lao2}
  \end{Phonetics}
\end{Entry}

\begin{Entry}{劳工同事}{7,3,6,8}{⼒、⼯、⼝、⼅}
  \begin{Phonetics}{劳工同事}{lao2gong1 tong2shi4}
    \definition{s.}{colaborador | colega de trabalho}
  \end{Phonetics}
\end{Entry}

\begin{Entry}{劳动}{7,6}{⼒、⼒}
  \begin{Phonetics}{劳动}{lao2dong4}[][HSK 5]
    \definition[次]{s.}{trabalho; mão de obra; atividades intelectuais ou físicas que podem criar valor | trabalho físico; trabalho manual; referindo-se especificamente ao trabalho físico}
    \definition{v.}{realizar trabalho físico}
  \end{Phonetics}
\end{Entry}

\begin{Entry}{医}{7}{⼖}
  \begin{Phonetics}{医}{yi1}
    \definition*{s.}{Sobrenome Yi}
    \definition{s.}{médico | medicina; ciência médica}
    \definition{v.}{curar; tratar}
  \end{Phonetics}
\end{Entry}

\begin{Entry}{医生}{7,5}{⼖、⽣}
  \begin{Phonetics}{医生}{yi1sheng1}[][HSK 1]
    \definition[位,个,名]{s.}{médico; clínico; pessoa que possui conhecimentos médicos e cuja profissão é tratar doenças}
  \end{Phonetics}
\end{Entry}

\begin{Entry}{医疗}{7,7}{⼖、⽧}
  \begin{Phonetics}{医疗}{yi1 liao2}[][HSK 4]
    \definition{s.}{tratamento médico; tratamento de doenças}
  \end{Phonetics}
\end{Entry}

\begin{Entry}{医学}{7,8}{⼖、⼦}
  \begin{Phonetics}{医学}{yi1 xue2}[][HSK 4]
    \definition{s.}{medicina; iatrologia; ciência médica; ciência da prevenção e do tratamento de doenças e da proteção e promoção da saúde humana}
  \end{Phonetics}
\end{Entry}

\begin{Entry}{医药}{7,9}{⼖、⾋}
  \begin{Phonetics}{医药}{yi1 yao4}[][HSK 6]
    \definition{s.}{medicina | médico | cuidados médicos e medicamentos | medicamento (droga) | farmacêutica}
  \end{Phonetics}
\end{Entry}

\begin{Entry}{医院}{7,9}{⼖、⾩}
  \begin{Phonetics}{医院}{yi1yuan4}[][HSK 1]
    \definition[家,所,个]{s.}{hospital; instituições que tratam e cuidam de pacientes, e também realizam exames de saúde, prevenção de doenças, etc.}
  \end{Phonetics}
\end{Entry}

\begin{Entry}{即}{7}{⼙}
  \begin{Phonetics}{即}{ji2}
    \definition{adv.}{no presente; no futuro imediato | prontamente; imediatamente}
    \definition{conj.}{e | mesmo; mesmo que}[她瞟了一眼睡着的孩子,随即匆匆离开了。===Ela olhou para a criança adormecida e então saiu correndo. | 即使下雨我也去。===Eu irei mesmo que chova.]
    \definition{v.}{aproximar-se; alcançar; estar perto | assumir; ascender a; aceitar; começar a se envolver em | ser motivado pela ocasião | estar perto | Literário: ser; significar}
  \end{Phonetics}
\end{Entry}

\begin{Entry}{即使}{7,8}{⼙、⼈}
  \begin{Phonetics}{即使}{ji2shi3}[][HSK 5]
    \definition{conj.}{mesmo; mesmo que; mesmo se; apesar de; expressando uma concessão hipotética}
  \end{Phonetics}
\end{Entry}

\begin{Entry}{即或}{7,8}{⼙、⼽}
  \begin{Phonetics}{即或}{ji2huo4}
    \definition{conj.}{mesmo se/embora}
  \end{Phonetics}
\end{Entry}

\begin{Entry}{即若}{7,8}{⼙、⾋}
  \begin{Phonetics}{即若}{ji2ruo4}
    \definition{conj.}{mesmo se/embora}
  \end{Phonetics}
\end{Entry}

\begin{Entry}{即便}{7,9}{⼙、⼈}
  \begin{Phonetics}{即便}{ji2bian4}
    \definition{conj.}{mesmo se/embora}
  \end{Phonetics}
\end{Entry}

\begin{Entry}{即便是}{7,9,9}{⼙、⼈、⽇}
  \begin{Phonetics}{即便是}{ji2bian4 shi4}
    \definition{conj.}{mesmo que seja}
  \end{Phonetics}
\end{Entry}

\begin{Entry}{即将}{7,9}{⼙、⼨}
  \begin{Phonetics}{即将}{ji2jiang1}[][HSK 4]
    \definition{adv.}{em breve; estar prestes a; estar a ponto de}
  \end{Phonetics}
\end{Entry}

\begin{Entry}{即是}{7,9}{⼙、⽇}
  \begin{Phonetics}{即是}{ji2shi4}
    \definition{conj.}{aquilo é}
  \end{Phonetics}
\end{Entry}

\begin{Entry}{却}{7}{⼙}
  \begin{Phonetics}{却}{que4}[][HSK 4]
    \definition{adv.}{mas; contudo; no entanto; enquanto; indica um ponto de virada}
    \definition{v.}{recuar; retroceder | afastar; repelir; desencorajar | declinar; recusar; rejeitar}
    \definition{v.aux.}{usado depois de certos verbos para indicar a conclusão de uma ação, resultado, equivalente a 去 ou 掉}
  \seealsoref{掉}{diao4}
  \seealsoref{去}{qu4}
  \end{Phonetics}
\end{Entry}

\begin{Entry}{却是}{7,9}{⼙、⽇}
  \begin{Phonetics}{却是}{que4 shi4}[][HSK 6]
    \definition{conj.}{na verdade; no entanto; o fato é\dots; indica um ponto de virada, contrário às suas expectativas anteriores}
  \end{Phonetics}
\end{Entry}

\begin{Entry}{县}{7}{⼛}
  \begin{Phonetics}{县}{xian4}[][HSK 4]
    \definition[个]{s.}{condado; unidade de divisão administrativa}
  \end{Phonetics}
\end{Entry}

\begin{Entry}{君}{7}{⼝}
  \begin{Phonetics}{君}{jun1}
    \definition*{s.}{Sobrenome Jun}
    \definition[个,位,名,些]{s.}{monarca; soberano; governante supremo | (como título) Senhor; Sr. | (literário) (em trato direto) você; senhor | cavalheiro | governante}
  \end{Phonetics}
\end{Entry}

\begin{Entry}{君主立宪制}{7,5,5,9,8}{⼝、⼂、⽴、⼧、⼑}
  \begin{Phonetics}{君主立宪制}{jun1zhu3li4xian4zhi4}
    \definition{s.}{monarquia constitucional}
  \end{Phonetics}
\end{Entry}

\begin{Entry}{吞}{7}{⼝}
  \begin{Phonetics}{吞}{tun1}[][HSK 6]
    \definition*{s.}{Sobrenome Tun}
    \definition{v.}{engolir; engolir em seco | tomar posse de; anexar | engolir; tragar; devorar; engolir inteiro ou em pedaços | absorver; engolir; engolfar}
  \end{Phonetics}
\end{Entry}

\begin{Entry}{吟}{7}{⼝}
  \begin{Phonetics}{吟}{yin2}
    \definition*{s.}{Sobrenome Yin}
    \definition{s.}{canção (como um tipo de poesia clássica) | grito de certos animais ou insetos}
    \definition{v.}{cantar; recitar | gemer; lamentar}
  \end{Phonetics}
\end{Entry}

\begin{Entry}{吟诗}{7,8}{⼝、⾔}
  \begin{Phonetics}{吟诗}{yin2shi1}
    \definition{v.}{recitar poesia}
  \end{Phonetics}
\end{Entry}

\begin{Entry}{否}{7}{⼝}
  \begin{Phonetics}{否}{fou3}
    \definition{adv.}{não; expressa discordância, equivalente à palavra falada 不 | usado no final de uma pergunta para indicar investigação | 是否, 能否 e 可否 que significa respectivamente 是不是, 能不能 e 可不可}
    \definition{v.}{negar}
  \seealsoref{不}{bu4}
  \seealsoref{可}{ke3}
  \seealsoref{能}{neng2}
  \seealsoref{是}{shi4}
  \end{Phonetics}
  \begin{Phonetics}{否}{pi3}
    \definition{adj.}{ruim; maligno; perverso}
    \definition{v.}{censurar}
  \end{Phonetics}
\end{Entry}

\begin{Entry}{否认}{7,4}{⼝、⾔}
  \begin{Phonetics}{否认}{fou3ren4}[][HSK 3]
    \definition{v.}{negar; repudiar; não reconhecer}
  \end{Phonetics}
\end{Entry}

\begin{Entry}{否决}{7,6}{⼝、⼎}
  \begin{Phonetics}{否决}{fou3jue2}[][HSK 7-9]
    \definition{v.}{rejeitar; votar contra; vetar; anular}
  \end{Phonetics}
\end{Entry}

\begin{Entry}{否则}{7,6}{⼝、⼑}
  \begin{Phonetics}{否则}{fou3ze2}[][HSK 4]
    \definition{conj.}{senão; se não; ou então; se não for isso}
  \end{Phonetics}
\end{Entry}

\begin{Entry}{否定}{7,8}{⼝、⼧}
  \begin{Phonetics}{否定}{fou3ding4}[][HSK 3]
    \definition{adj.}{negativo; contrário}
    \definition{v.}{rejeitar; negar a existência ou a autenticidade de algo}
  \end{Phonetics}
\end{Entry}

\begin{Entry}{吧}{7}{⼝}
  \begin{Phonetics}{吧}{ba1}
    \definition{s.}{som de estalo, som crepitante |  abreviação de bar, 酒吧 | cibercafé; um local público que fornece computadores e serviços de \emph{Internet} onde as pessoas podem navegar, jogar, etc.}
    \definition{v.}{fumar; dar uma tragada (puxar) no cachimbo}
  \seealsoref{酒吧}{jiu3ba1}
  \end{Phonetics}
  \begin{Phonetics}{吧}{ba5}[][HSK 1]
    \definition{part.}{indica discussão, sugestão, solicitação ou comando no final de uma frase | indica concordância ou aprovação no final de uma frase | indica uma pergunta ou especulação no final de uma frase | indica incerteza no final de uma frase | em uma frase, indica uma pausa, carrega um tom hipotético, frequentemente apresenta um contraste e implica um dilema}
  \end{Phonetics}
\end{Entry}

\begin{Entry}{吨}{7}{⼝}
  \begin{Phonetics}{吨}{dun1}[][HSK 5]
    \definition{clas.}{tonelada}
  \end{Phonetics}
\end{Entry}

\begin{Entry}{吩}{7}{⼝}
  \begin{Phonetics}{吩}{fen1}
    \definition{v.}{deixar instruções; instruir | ordenar; mandar}
  \end{Phonetics}
\end{Entry}

\begin{Entry}{吩咐}{7,8}{⼝、⼝}
  \begin{Phonetics}{吩咐}{fen1fu4}[][HSK 7-9]
    \definition{v.}{dizer; instruir; comandar; dizer a alguém para lembrar o que deve ou não ser feito}
  \end{Phonetics}
\end{Entry}

\begin{Entry}{含}{7}{⼝}
  \begin{Phonetics}{含}{han2}[][HSK 4]
    \definition{v.}{manter na boca (sem engolir ou cuspir) | conter; incluir | cuidar; acalentar; abrigar}
  \end{Phonetics}
\end{Entry}

\begin{Entry}{含义}{7,3}{⼝、⼂}
  \begin{Phonetics}{含义}{han2yi4}[][HSK 4]
    \definition[个,种,层]{s.}{sentido; mensagem; significado; implicação; o significado contido em (palavras, frases, sentenças e discursos)}
  \end{Phonetics}
\end{Entry}

\begin{Entry}{含有}{7,6}{⼝、⽉}
  \begin{Phonetics}{含有}{han2 you3}[][HSK 4]
    \definition{v.}{conter; ter; incluir}
  \end{Phonetics}
\end{Entry}

\begin{Entry}{含金量}{7,8,12}{⼝、⾦、⾥}
  \begin{Phonetics}{含金量}{han2jin1liang4}
    \definition{adj.}{conteúdo de ouro | (fig.) valioso}
  \end{Phonetics}
\end{Entry}

\begin{Entry}{含量}{7,12}{⼝、⾥}
  \begin{Phonetics}{含量}{han2 liang4}[][HSK 4]
    \definition{s.}{conteúdo; a quantidade de um componente contido em uma substância}
  \end{Phonetics}
\end{Entry}

\begin{Entry}{含蓄}{7,13}{⼝、⾋}
  \begin{Phonetics}{含蓄}{han2xu4}[][HSK 7-9]
    \definition{v.}{conter; incorporar}
  \end{Phonetics}
\end{Entry}

\begin{Entry}{含糊}{7,15}{⼝、⽶}
  \begin{Phonetics}{含糊}{han2hu5}[][HSK 7-9]
    \definition{adj.}{(atitude, palavras, etc.) vago; ambíguo; pouco claro | (falar, fazer coisas, etc.) descuidado; desleixado (usado principalmente em termos negativos) | covarde; demonstrando fraqueza (usado principalmente em sentido negativo)}
  \end{Phonetics}
\end{Entry}

\begin{Entry}{听}{7}{⼝}
  \begin{Phonetics}{听}{ting1}[][HSK 1]
    \definition{clas.}{latas; usado para bebidas e alimentos para levar consigo}
    \definition{s.}{lata; embalagem metálica; recipiente cilíndrico utilizado para armazenar bebidas, alimentos, etc.}
    \definition{v.}{ouvir; escutar | obedecer; dar ouvidos; aceitar | supervisionar; administrar; tratar (assuntos políticos); julgar (casos) | permitir; deixar ser; deixar fazer}
  \end{Phonetics}
  \begin{Phonetics}{听}{yin3}
    \definition[个]{s.}{lata; embalagem metálica}
  \end{Phonetics}
\end{Entry}

\begin{Entry}{听力}{7,2}{⼝、⼒}
  \begin{Phonetics}{听力}{ting1li4}[][HSK 3]
    \definition{s.}{audição; capacidade auditiva | compreensão auditiva (na aprendizagem de línguas)}
  \end{Phonetics}
\end{Entry}

\begin{Entry}{听力理解}{7,2,11,13}{⼝、⼒、⽟、⾓}
  \begin{Phonetics}{听力理解}{ting1li4li3jie3}
    \definition{s.}{compreensão auditiva}
  \end{Phonetics}
\end{Entry}

\begin{Entry}{听小骨}{7,3,9}{⼝、⼩、⾻}
  \begin{Phonetics}{听小骨}{ting1xiao3gu3}
    \definition{s.}{ossículos (do ouvido médio)}
  \seealsoref{听骨}{ting1gu3}
  \end{Phonetics}
\end{Entry}

\begin{Entry}{听见}{7,4}{⼝、⾒}
  \begin{Phonetics}{听见}{ting1 jian4}[][HSK 1]
    \definition{v.}{ouvir; conseguir ouvir}
  \end{Phonetics}
\end{Entry}

\begin{Entry}{听写}{7,5}{⼝、⼍}
  \begin{Phonetics}{听写}{ting1 xie3}[][HSK 1]
    \definition{s.}{ditado}
    \definition{v.}{ouvir e escrever}
  \end{Phonetics}
\end{Entry}

\begin{Entry}{听众}{7,6}{⼝、⼈}
  \begin{Phonetics}{听众}{ting1 zhong4}[][HSK 3]
    \definition[位,名,个]{s.}{audiência; ouvintes; pessoas que ouvem palestras, música ou transmissões}
  \end{Phonetics}
\end{Entry}

\begin{Entry}{听会}{7,6}{⼝、⼈}
  \begin{Phonetics}{听会}{ting1hui4}
    \definition{v.}{participar de uma reunião (e ouvir o que é discutido)}
  \end{Phonetics}
\end{Entry}

\begin{Entry}{听戏}{7,6}{⼝、⼽}
  \begin{Phonetics}{听戏}{ting1xi4}
    \definition{v.}{assistir a uma ópera | ver uma ópera}
  \end{Phonetics}
\end{Entry}

\begin{Entry}{听讲}{7,6}{⼝、⾔}
  \begin{Phonetics}{听讲}{ting1/jiang3}[][HSK 2]
    \definition{v.+compl.}{assistir a uma palestra; ouvir palestras ou discursos}
  \end{Phonetics}
\end{Entry}

\begin{Entry}{听来}{7,7}{⼝、⽊}
  \begin{Phonetics}{听来}{ting1lai2}
    \definition{v.}{ouvir de algum lugar | soar (antigo, estrangeiro, excitante, certo, etc.) | soar como se (ou seja, dar uma impressão ao ouvinte)}
  \end{Phonetics}
\end{Entry}

\begin{Entry}{听凭}{7,8}{⼝、⼏}
  \begin{Phonetics}{听凭}{ting1ping2}
    \definition{v.}{permitir (alguém a fazer o que desejar)}
  \end{Phonetics}
\end{Entry}

\begin{Entry}{听到}{7,8}{⼝、⼑}
  \begin{Phonetics}{听到}{ting1 dao4}[][HSK 1]
    \definition{v.}{ouvir, escutar; ouvir atentamente, escutar atentamente}
  \end{Phonetics}
\end{Entry}

\begin{Entry}{听取}{7,8}{⼝、⼜}
  \begin{Phonetics}{听取}{ting1 qu3}[][HSK 6]
    \definition{v.}{ouvir (opiniões, reflexões, relatórios, etc.)}
  \end{Phonetics}
\end{Entry}

\begin{Entry}{听命}{7,8}{⼝、⼝}
  \begin{Phonetics}{听命}{ting1ming4}
    \definition{v.}{obedecer ordens | receber ordens}
  \end{Phonetics}
\end{Entry}

\begin{Entry}{听说}{7,9}{⼝、⾔}
  \begin{Phonetics}{听说}{ting1 shuo1}[][HSK 2]
    \definition{v.}{ser informado; ouvir falar de; ouvir dizer | ouvir e falar}
  \end{Phonetics}
\end{Entry}

\begin{Entry}{听骨}{7,9}{⼝、⾻}
  \begin{Phonetics}{听骨}{ting1gu3}
    \definition{s.}{ossículos (do ouvido médio)}
  \seealsoref{听小骨}{ting1xiao3gu3}
  \end{Phonetics}
\end{Entry}

\begin{Entry}{听断}{7,11}{⼝、⽄}
  \begin{Phonetics}{听断}{ting1duan4}
    \definition{v.}{ouvir e decidir | julgar (ou seja, ouvir e julgar em um tribunal)}
  \end{Phonetics}
\end{Entry}

\begin{Entry}{听随}{7,11}{⼝、⾩}
  \begin{Phonetics}{听随}{ting1sui2}
    \definition{v.}{permitir | obedecer}
  \end{Phonetics}
\end{Entry}

\begin{Entry}{启}{7}{⼝}
  \begin{Phonetics}{启}{qi3}
    \definition*{s.}{Sobrenome Qi}
    \definition{s.}{nota; carta; um dos antigos estilos literários, uma carta relativamente curta}
    \definition{v.}{abrir | despertar; iluminar | começar; iniciar | declarar; informar}
  \end{Phonetics}
\end{Entry}

\begin{Entry}{启发}{7,5}{⼝、⼜}
  \begin{Phonetics}{启发}{qi3fa1}[][HSK 5]
    \definition{s.}{iluminação; esclarecimento; fenômenos e princípios que levam as pessoas a refletir e a abrir suas mentes}
    \definition{v.}{despertar; inspirar; esclarecer; orientar, fazer com que compreendam}
  \end{Phonetics}
\end{Entry}

\begin{Entry}{启动}{7,6}{⼝、⼒}
  \begin{Phonetics}{启动}{qi3 dong4}[][HSK 5]
    \definition{v.}{ligar (uma máquina); acionar; ligar máquinas, equipamentos elétricos, etc., para começar a trabalhar | entrar em vigor; começar a vigorar e a ser implementados planos, projetos, documentos jurídicos, etc.}
  \end{Phonetics}
\end{Entry}

\begin{Entry}{启事}{7,8}{⼝、⼅}
  \begin{Phonetics}{启事}{qi3shi4}[][HSK 5]
    \definition[个,则,份,张,条]{s.}{aviso; anúncio; texto publicado em jornais ou afixado em paredes com o objetivo de divulgar publicamente algo}
  \end{Phonetics}
\end{Entry}

\begin{Entry}{吵}{7}{⼝}
  \begin{Phonetics}{吵}{chao3}[][HSK 3]
    \definition{adj.}{barulhento; ruidoso e perturbador}
    \definition{v.}{brigar; discutir; disputar}
  \end{Phonetics}
\end{Entry}

\begin{Entry}{吵架}{7,9}{⼝、⽊}
  \begin{Phonetics}{吵架}{chao3/jia4}[][HSK 3]
    \definition{v.+compl.}{brigar; discutir; ter uma discussão acalorada}
  \end{Phonetics}
\end{Entry}

\begin{Entry}{吵嘴}{7,16}{⼝、⼝}
  \begin{Phonetics}{吵嘴}{chao3/zui3}[][HSK 7-9]
    \definition{v.}{brigar; discutir}[我们之间从不吵嘴。===Nós nunca brigamos.]
  \end{Phonetics}
\end{Entry}

\begin{Entry}{吹}{7}{⼝}
  \begin{Phonetics}{吹}{chui1}[][HSK 2]
    \definition{v.}{soprar; baforar | tocar (instrumentos de sopro) | (do vento) soprar | gabar-se; vangloriar-se | elogiar; louvar aos céus; adular | (relacionamento) romper; separar-se; (assunto) fracassar}
  \end{Phonetics}
\end{Entry}

\begin{Entry}{吹了}{7,2}{⼝、⼅}
  \begin{Phonetics}{吹了}{chui1 le5}[][HSK 7-9]
    \definition{v.}{quebrar; falhar | ter esfriado (um relacionamento) | ter morrido | não ter tido sucesso | ter se separado}
  \end{Phonetics}
\end{Entry}

\begin{Entry}{吹牛}{7,4}{⼝、⽜}
  \begin{Phonetics}{吹牛}{chui1/niu2}[][HSK 7-9]
    \definition{v.+compl.}{gabar-se; vangloriar-se; falar alto; falar com arrogância; falar sem parar}
  \end{Phonetics}
\end{Entry}

\begin{Entry}{吹捧}{7,11}{⼝、⼿}
  \begin{Phonetics}{吹捧}{chui1peng3}[][HSK 7-9]
    \definition{v.}{bajular; elogiar até os céus; elogiar abundantemente}
  \end{Phonetics}
\end{Entry}

\begin{Entry}{吾}{7}{⼝}
  \begin{Phonetics}{吾}{wu2}
    \definition*{s.}{Sobrenome Wu}
    \definition{pron.}{eu; nós}
  \end{Phonetics}
\end{Entry}

\begin{Entry}{呀}{7}{⼝}
  \begin{Phonetics}{呀}{ya5}[][HSK 4]
    \definition{part.}{usado no lugar de 啊 quando a palavra anterior termina com o som a, e, i, o ou ü}
  \seealsoref{啊}{a5}
  \end{Phonetics}
\end{Entry}

\begin{Entry}{呆}{7}{⼝}
  \begin{Phonetics}{呆}{dai1}[][HSK 5]
    \definition*{s.}{Sobrenome Dai}
    \definition{adj.}{maçante; de raciocínio lento | em branco; de madeira; rígido; inflexível}
    \definition{v.}{ficar; permanecer}
  \end{Phonetics}
\end{Entry}

\begin{Entry}{呈}{7}{⼝}
  \begin{Phonetics}{呈}{cheng2}
    \definition*{s.}{Sobrenome Cheng}
    \definition{s.}{documento submetido a um superior; petição; memorial}
    \definition{v.}{apresentar; assumir (forma, cor, etc.) | submeter; apresentar; enviar respeitosamente}
  \end{Phonetics}
\end{Entry}

\begin{Entry}{呈现}{7,8}{⼝、⾒}
  \begin{Phonetics}{呈现}{cheng2xian4}[][HSK 7-9]
    \definition{v.}{apresentar (uma certa aparência); aparecer. mostrar uma determinada forma, cor ou tendência, etc., para que as pessoas possam ver}[大海呈现出碧蓝的颜色。===O mar apresenta uma cor azul vibrante.]
  \end{Phonetics}
\end{Entry}

\begin{Entry}{告}{7}{⼝}
  \begin{Phonetics}{告}{gao4}[][HSK 7-9]
    \definition*{s.}{Sobrenome Gao}
    \definition{s.}{anúncio; declaração; notificação}
    \definition{v.}{informar; contar; notificar; explicar aos outros | acusar; processar; relatar | pedir; requisitar; solicitar | dar a conhecer; mostrar | anunciar; declarar; proclamar}
  \end{Phonetics}
\end{Entry}

\begin{Entry}{告示}{7,5}{⼝、⽰}
  \begin{Phonetics}{告示}{gao4shi5}[][HSK 7-9]
    \definition[张,条,篇]{s.}{nota oficial; boletim; cartaz | Datado: slogan; cartaz}
    \definition{v.}{notificar; anunciar}
  \end{Phonetics}
\end{Entry}

\begin{Entry}{告别}{7,7}{⼝、⼑}
  \begin{Phonetics}{告别}{gao4/bie2}[][HSK 3]
    \definition{v.+compl.}{dizer adeus a; expressar a outros, por meio de palavras, que está prestes a partir | deixar; sair; partir de | prestar as últimas homenagens ao falecido}
  \end{Phonetics}
\end{Entry}

\begin{Entry}{告状}{7,7}{⼝、⽝}
  \begin{Phonetics}{告状}{gao4/zhuang4}[][HSK 7-9]
    \definition{v.+compl.}{ir à justiça contra alguém; apresentar uma queixa ao tribunal e solicitar que o caso seja aberto para julgamento | apresentar uma acusação ou queixa; informar aos pais ou superiores que você ou outras pessoas foram vítimas de \emph{bullying} ou injustiças}
  \end{Phonetics}
\end{Entry}

\begin{Entry}{告诉}{7,7}{⼝、⾔}
  \begin{Phonetics}{告诉}{gao4su4}
    \definition{v.}{dizer; informar (dar a conhecer); dizer aos outros, para que todos saibam}
  \end{Phonetics}
  \begin{Phonetics}{告诉}{gao4su5}[][HSK 1]
    \definition{v.}{dizer; informar (dar a conhecer)}
  \end{Phonetics}
\end{Entry}

\begin{Entry}{告知}{7,8}{⼝、⽮}
  \begin{Phonetics}{告知}{gao4zhi1}[][HSK 7-9]
    \definition{v.}{contar; informar; transmitir; familiarizar}
  \end{Phonetics}
\end{Entry}

\begin{Entry}{告急}{7,9}{⼝、⼼}
  \begin{Phonetics}{告急}{gao4ji2}
    \definition{v.}{estar em estado de emergência; ser crítico | relatar uma emergência; pedir ajuda de emergência | estar em uma emergência; fazer uma solicitação urgente de ajuda em uma emergência}
  \end{Phonetics}
\end{Entry}

\begin{Entry}{告诫}{7,9}{⼝、⾔}
  \begin{Phonetics}{告诫}{gao4jie4}[][HSK 7-9]
    \definition{v.}{avisar; advertir; exortar; admoestar}
  \end{Phonetics}
\end{Entry}

\begin{Entry}{告辞}{7,13}{⼝、⾟}
  \begin{Phonetics}{告辞}{gao4ci2}[][HSK 7-9]
    \definition{v.}{despedir-se (do anfitrião) | despedir-se}
  \end{Phonetics}
\end{Entry}

\begin{Entry}{员}{7}{⼝}
  \begin{Phonetics}{员}{yuan2}[][HSK 3]
    \definition{clas.}{para comandantes militares}
    \definition{s.}{uma pessoa envolvida em algum campo de atividade; refere-se a pessoas que trabalham ou estudam | membro; refere-se aos membros de um grupo ou organização | vizinhança}
  \end{Phonetics}
\end{Entry}

\begin{Entry}{员工}{7,3}{⼝、⼯}
  \begin{Phonetics}{员工}{yuan2gong1}[][HSK 3]
    \definition[位,名,个]{s.}{equipe; funcionário; trabalhador; pessoal}
  \end{Phonetics}
\end{Entry}

\begin{Entry}{园}{7}{⼞}
  \begin{Phonetics}{园}{yuan2}[][HSK 6]
    \definition*{s.}{Sobrenome Yuan}
    \definition[个]{s.}{jardim; terreno; plantação; terra para cultivar plantas | local para recreação pública; locais para passeios turísticos e entretenimento | área para fins especiais | uma área de terra para o cultivo de plantas; um lugar onde vegetais, flores, frutas e árvores são cultivados}
  \end{Phonetics}
\end{Entry}

\begin{Entry}{园地}{7,6}{⼞、⼟}
  \begin{Phonetics}{园地}{yuan2 di4}[][HSK 6]
    \definition{s.}{jardim; campo | Figurativo: campo; escopo}
  \end{Phonetics}
\end{Entry}

\begin{Entry}{园林}{7,8}{⼞、⽊}
  \begin{Phonetics}{园林}{yuan2lin2}[][HSK 5]
    \definition[处,座,个]{s.}{parque; jardim; área paisagística com plantas e árvores para as pessoas apreciarem e descansarem.}
  \end{Phonetics}
\end{Entry}

\begin{Entry}{囯}{7}{⼞}
  \begin{Phonetics}{囯}{guo2}
    \definition*{s.}{Sobrenome Guo}
    \definition{adj.}{do estado; nacional | do nosso país; Chinês | do país}
    \definition{s.}{país; nação; estado | o melhor da nação | o melhor; o mais bonito do país}
    \variantof{国}
  \end{Phonetics}
\end{Entry}

\begin{Entry}{困}{7}{⼞}
  \begin{Phonetics}{困}{kun4}[][HSK 3]
    \definition{adj.}{cansado; exausto; fatigado | difícil; complicado; difícil e penoso; pobre e miserável | sonolento; com sono; cansado, com vontade de dormir}
    \definition{v.}{ficar encalhado; estar em apuros; preso em dificuldades e sofrimentos ou limitado por circunstâncias e condições que não pode escapar | cercar; envolver; imobilizar; controlar dentro de um determinado limite | dormir}
  \end{Phonetics}
\end{Entry}

\begin{Entry}{困扰}{7,7}{⼞、⼿}
  \begin{Phonetics}{困扰}{kun4 rao3}[][HSK 5]
    \definition{v.}{perturbar; deixar perplexo; perseguir}
  \end{Phonetics}
\end{Entry}

\begin{Entry}{困难}{7,10}{⼞、⾫}
  \begin{Phonetics}{困难}{kun4nan5}[][HSK 3]
    \definition{adj.}{dificuldades financeiras; circunstâncias difíceis | complicado; complexo; difícil; árduo; a situação é complexa e há muitos obstáculos}
    \definition[种]{s.}{dificuldade; situação difícil; problemas ou situações difíceis de resolver no trabalho e na vida}
  \end{Phonetics}
\end{Entry}

\begin{Entry}{围}{7}{⼞}
  \begin{Phonetics}{围}{wei2}[][HSK 3]
    \definition*{s.}{Sobrenome Wei}
    \definition{clas.}{o comprimento das duas mãos com os polegares e os dedos indicadores juntos ou dos dois braços juntos}
    \definition{s.}{em volta de tudo; ao redor}
    \definition{v.}{cercar; rodear; circundar; encurralar; cercar tudo, impedindo a passagem entre o interior e o exterior | envolver; contornar}
  \end{Phonetics}
\end{Entry}

\begin{Entry}{围巾}{7,3}{⼞、⼱}
  \begin{Phonetics}{围巾}{wei2jin1}[][HSK 4]
    \definition[条]{s.}{lenço; cachecol; echarpe; gravata; tiras longas de malha ou tecido usadas ao redor do pescoço para aquecimento, proteção do colarinho ou decoração}
  \end{Phonetics}
\end{Entry}

\begin{Entry}{围绕}{7,9}{⼞、⽷}
  \begin{Phonetics}{围绕}{wei2rao4}[][HSK 5]
    \definition{v.}{girar; circundar; dar voltas; girar em torno de algo; cercar | concentrar-se em; centrar-se em; centrar-se em uma questão ou evento (para realizar atividades)}
  \end{Phonetics}
\end{Entry}

\begin{Entry}{坏}{7}{⼟}
  \begin{Phonetics}{坏}{huai4}[][HSK 1]
    \definition{adj.}{ruim; prejudicial; insatisfatório; péssimo | mal; extremamente; indica um grau profundo, geralmente usado após verbos ou adjetivos que expressam estado psicológico | podre; estragado; impróprio; prejudicial ao uso}
    \definition[种]{s.}{ideia maligna; truque sujo; péssima ideia}
    \definition{v.}{estragar; destruir; corromper}
  \end{Phonetics}
\end{Entry}

\begin{Entry}{坏人}{7,2}{⼟、⼈}
  \begin{Phonetics}{坏人}{huai4 ren2}[][HSK 2]
    \definition[个,种]{s.}{malfeitor; canalha; pessoa má; pessoa de má qualidade; pessoa que faz coisas ruins}
  \end{Phonetics}
\end{Entry}

\begin{Entry}{坏处}{7,5}{⼟、⼡}
  \begin{Phonetics}{坏处}{huai4 chu4}[][HSK 2]
    \definition[个]{s.}{dano; prejuízo; desvantagem; fatores prejudiciais a pessoas ou coisas}
  \end{Phonetics}
\end{Entry}

\begin{Entry}{坏蛋}{7,11}{⼟、⾍}
  \begin{Phonetics}{坏蛋}{huai4dan4}
    \definition{s.}{bastardo | canalha | pessoa má}
  \end{Phonetics}
\end{Entry}

\begin{Entry}{坐}{7}{⼟}
  \begin{Phonetics}{坐}{zuo4}[][HSK 1]
    \definition*{s.}{Sobrenome Zuo}
    \definition{adv.}{sem motivo algum; sem causa ou razão; sem motivo aparente}
    \definition{prep.}{porque; pelo fato de que; pela razão de que; pelo motivo de que}
    \definition{s.}{assento; lugar; posição}
    \definition{v.}{sentar; sentar-se; ocupar um lugar; colocar os glúteos sobre um objeto para apoiar o peso corporal | pegar; viajar de; pegar carona | ter as costas voltadas para | colocar (uma panela, chaleira, etc.) no fogo | recuo; coice (de rifles, armas, etc.)  | produzir frutos; formar sementes | ser punido; ser acusado de crime | contrair (ou ter) uma doença; sofrer de uma doença | (um edifício) afundar; ceder}
  \end{Phonetics}
\end{Entry}

\begin{Entry}{坐下}{7,3}{⼟、⼀}
  \begin{Phonetics}{坐下}{zuo4 xia5}[][HSK 1]
    \definition{v.}{sentar-se; tomar um assento; passar da posição em pé para a posição sentada}
  \end{Phonetics}
\end{Entry}

\begin{Entry}{坐车}{7,4}{⼟、⾞}
  \begin{Phonetics}{坐车}{zuo4che1}
    \definition{v.}{andar de carro, ônibus, trem, etc.}
  \end{Phonetics}
\end{Entry}

\begin{Entry}{坐好}{7,6}{⼟、⼥}
  \begin{Phonetics}{坐好}{zuo4hao3}
    \definition{v.}{sentar-se corretamente | sentar direito}
  \end{Phonetics}
\end{Entry}

\begin{Entry}{坐享}{7,8}{⼟、⼇}
  \begin{Phonetics}{坐享}{zuo4xiang3}
    \definition{v.}{curtir algo sem levantar um dedo}
  \end{Phonetics}
\end{Entry}

\begin{Entry}{坐垫}{7,9}{⼟、⼟}
  \begin{Phonetics}{坐垫}{zuo4dian4}
    \definition[块]{s.}{assento (motocicleta) | almofada}
  \end{Phonetics}
\end{Entry}

\begin{Entry}{坐标}{7,9}{⼟、⽊}
  \begin{Phonetics}{坐标}{zuo4biao1}
    \definition{s.}{Geometria: coordenada; um número ou conjunto de números que pode determinar a posição de um ponto no espaço}
  \end{Phonetics}
\end{Entry}

\begin{Entry}{坑}{7}{⼟}
  \begin{Phonetics}{坑}{keng1}
    \definition[个]{s.}{poço; buraco; cavidade | poço; túnel; caverna subterrânea}
    \definition{v.}{enredar; enganar; trapacear | nos tempos antigos, significava enterrar as pessoas vivas}
  \end{Phonetics}
\end{Entry}

\begin{Entry}{坑人}{7,2}{⼟、⼈}
  \begin{Phonetics}{坑人}{keng1/ren2}
    \definition{v.+compl.}{trapacear alguém}
  \end{Phonetics}
\end{Entry}

\begin{Entry}{块}{7}{⼟}
  \begin{Phonetics}{块}{kuai4}[][HSK 1]
    \definition{clas.}{usado para coisas em pedaços | usado para coisas em pedaços ou em algumas formas de folhas | usado para moedas de prata ou notas de papel equivalentes a 圆}
    \definition{s.}{pedaço; pedaço (de terra); peça; algo que forma um pedaço ou massa}
  \seealsoref{圆}{yuan2}
  \end{Phonetics}
\end{Entry}

\begin{Entry}{坚}{7}{⼟}
  \begin{Phonetics}{坚}{jian1}
    \definition*{s.}{Sobrenome Jian}
    \definition{adj.}{duro; firme; sólido; forte | firme; resoluto; constante}
    \definition{s.}{fortaleza; fortificação; um ponto fortemente fortificado; coisas sólidas, principalmente referindo-se a posições | armadura}
  \end{Phonetics}
\end{Entry}

\begin{Entry}{坚决}{7,6}{⼟、⼎}
  \begin{Phonetics}{坚决}{jian1jue2}[][HSK 3]
    \definition{adj.}{firme; resoluto; (atitude, opinião, ação, etc.) determinado e inabalável}
  \end{Phonetics}
\end{Entry}

\begin{Entry}{坚守}{7,6}{⼟、⼧}
  \begin{Phonetics}{坚守}{jian1shou3}
    \definition{v.}{agarrar-se}
  \end{Phonetics}
\end{Entry}

\begin{Entry}{坚固}{7,8}{⼟、⼞}
  \begin{Phonetics}{坚固}{jian1gu4}[][HSK 4]
    \definition{adj.}{firme; sólido; robusto; forte; durável; firmemente unidos e inquebráveis}
  \end{Phonetics}
\end{Entry}

\begin{Entry}{坚定}{7,8}{⼟、⼧}
  \begin{Phonetics}{坚定}{jian1ding4}[][HSK 5]
    \definition{adj.}{firme; inabalável; inamovível; (posição, opinião, vontade, etc.) firme e estável, inabalável}
    \definition{v.}{fortalecer}
  \end{Phonetics}
\end{Entry}

\begin{Entry}{坚持}{7,9}{⼟、⼿}
  \begin{Phonetics}{坚持}{jian1chi2}[][HSK 3]
    \definition{v.}{persistir em; perseverar em; defender; insistir em; manter-se fiel a; aderir a; persistir com determinação e não desistir quando se depara com dificuldades | aderir a; insistir em; não alterar (os princípios, opiniões, pontos de vista originais, etc.)}
  \end{Phonetics}
\end{Entry}

\begin{Entry}{坚强}{7,12}{⼟、⼸}
  \begin{Phonetics}{坚强}{jian1qiang2}[][HSK 3]
    \definition{adj.}{forte; firme; convicto; (qualidades humanas, personalidade, determinação, etc.) firme e forte, não vacila diante das dificuldades}
    \definition{v.}{fortalecer; tornar forte; é a qualidade, a determinação, etc., que não vacilam}
  \end{Phonetics}
\end{Entry}

\begin{Entry}{坝}{7}{⼟}
  \begin{Phonetics}{坝}{ba4}[][HSK 7-9]
    \definition[座,道,个,条]{s.}{barragem | dique; aterro | Dialeto: banco de areia | (usualmente em nomes de lugares) planície}
  \end{Phonetics}
\end{Entry}

\begin{Entry}{坟}{7}{⼟}
  \begin{Phonetics}{坟}{fen2}[][HSK 7-9]
    \definition[座,片,个]{s.}{sepultura; túmulo; cova}
  \end{Phonetics}
\end{Entry}

\begin{Entry}{坟墓}{7,13}{⼟、⼟}
  \begin{Phonetics}{坟墓}{fen2mu4}[][HSK 7-9]
    \definition[座,片,个]{s.}{sepultura; túmulo; a cova funerária e a sepultura acima dela}
  \end{Phonetics}
\end{Entry}

\begin{Entry}{坠}{7}{⼟}
  \begin{Phonetics}{坠}{zhui4}
    \definition{s.}{peso; objeto pendurado | um objeto pendurado; um pingente}
    \definition{v.}{cair; derrubar | curvar para baixo}
  \end{Phonetics}
\end{Entry}

\begin{Entry}{坠落}{7,12}{⼟、⾋}
  \begin{Phonetics}{坠落}{zhui4luo4}
    \definition{v.}{cair}
  \end{Phonetics}
\end{Entry}

\begin{Entry}{声}{7}{⼠}
  \begin{Phonetics}{声}{sheng1}[][HSK 5]
    \definition{clas.}{indica o número de vezes que um som é emitido}
    \definition{s.}{som; voz | reputação | consoante inicial (de uma sílaba chinesa) | tom; tom de voz | informação; notícia}
    \definition{v.}{declarar; anunciar; emitir um som}
  \end{Phonetics}
\end{Entry}

\begin{Entry}{声明}{7,8}{⼠、⽇}
  \begin{Phonetics}{声明}{sheng1ming2}[][HSK 3]
    \definition[项,份]{s.}{declaração}
    \definition{v.}{declarar; anunciar; expressar publicamente a sua atitude ou dizer a verdade}
  \end{Phonetics}
\end{Entry}

\begin{Entry}{声音}{7,9}{⼠、⾳}
  \begin{Phonetics}{声音}{sheng1yin1}[][HSK 2]
    \definition[个,种]{s.}{som; voz; a percepção auditiva das ondas sonoras}
  \end{Phonetics}
\end{Entry}

\begin{Entry}{壳}{7}{⼠}
  \begin{Phonetics}{壳}{ke2}
    \definition[层,个]{s.}{casca, concha; significa o mesmo que 壳 | concha; revestimento externo | empresa de fachada}
  \end{Phonetics}
  \begin{Phonetics}{壳}{qiao4}
    \definition[层,个]{s.}{Coloquial: concha | invólucro; caixa; carapaça | empresa de fachada (ou corporação) | superfície dura}
  \end{Phonetics}
\end{Entry}

\begin{Entry}{妒}{7}{⼥}
  \begin{Phonetics}{妒}{du4}
    \definition{v.}{ter ciúmes (inveja) de}
  \end{Phonetics}
\end{Entry}

\begin{Entry}{妒忌}{7,7}{⼥、⼼}
  \begin{Phonetics}{妒忌}{du4ji4}[][HSK 7-9]
    \definition{v.}{invejar; ter ciúmes (ou inveja) de}
  \end{Phonetics}
\end{Entry}

\begin{Entry}{妖}{7}{⼥}
  \begin{Phonetics}{妖}{yao1}
    \definition{adj.}{maligno e fraudulento | sedutor; encantador | paquerador}
    \definition[个,只]{s.}{\emph{goblin}; demônio; espírito maligno}
  \end{Phonetics}
\end{Entry}

\begin{Entry}{妙}{7}{⼥}
  \begin{Phonetics}{妙}{miao4}[][HSK 6]
    \definition*{s.}{Sobrenome Miao}
    \definition{adj.}{maravilhoso; excelente; bom | engenhoso; esperto; sutil | extraordinário | requintado; mágico; engenhoso; misterioso}
  \end{Phonetics}
\end{Entry}

\begin{Entry}{妙招}{7,8}{⼥、⼿}
  \begin{Phonetics}{妙招}{miao4zhao1}
    \definition{adj.}{escorregadio}
    \definition{s.}{movimento inteligente | maneira inteligente de fazer algo}
  \end{Phonetics}
\end{Entry}

\begin{Entry}{妨}{7}{⼥}
  \begin{Phonetics}{妨}{fang2}
    \definition{v.}{dificultar; entravar; impedir; obstruir | (no negativo ou interrogativo) prejudicar | interferir com}
  \end{Phonetics}
\end{Entry}

\begin{Entry}{妨害}{7,10}{⼥、⼧}
  \begin{Phonetics}{妨害}{fang2hai4}[][HSK 7-9]
    \definition{v.}{prejudicar; pôr em risco; ser prejudicial a; causar dano; ferir}
  \end{Phonetics}
\end{Entry}

\begin{Entry}{妨碍}{7,13}{⼥、⽯}
  \begin{Phonetics}{妨碍}{fang2'ai4}[][HSK 7-9]
    \definition{v.}{dificultar; entravar; impedir; obstruir; impedir que as coisas corram bem}
  \end{Phonetics}
\end{Entry}

\begin{Entry}{宋}{7}{⼧}
  \begin{Phonetics}{宋}{song4}
    \definition*{s.}{Dinastia Song (960-1279) | Song das dinastias do sul (420-479) | Sobrenome Song}
    \definition{clas.}{sone; unidade de intensidade sonora}
  \end{Phonetics}
\end{Entry}

\begin{Entry}{完}{7}{⼧}
  \begin{Phonetics}{完}{wan2}[][HSK 2]
    \definition*{s.}{Sobrenome Wan}
    \definition{adj.}{inteiro; intacto; completo}
    \definition{v.}{acabar; terminar; completar | pagar | estar terminado; estar pronto para | esgotar; ser usado}
  \end{Phonetics}
\end{Entry}

\begin{Entry}{完了}{7,2}{⼧、⼅}
  \begin{Phonetics}{完了}{wan2 le5}[][HSK 5]
    \definition{v.}{acabar; terminar; concluir; chegar ao fim}
  \end{Phonetics}
\end{Entry}

\begin{Entry}{完人}{7,2}{⼧、⼈}
  \begin{Phonetics}{完人}{wan2ren2}
    \definition{s.}{pessoa perfeita}
  \end{Phonetics}
\end{Entry}

\begin{Entry}{完全}{7,6}{⼧、⼊}
  \begin{Phonetics}{完全}{wan2quan2}[][HSK 2]
    \definition{adj.}{inteiro; completo; não falta nada, está tudo completo}
    \definition{adv.}{completamente; representa tudo}
  \end{Phonetics}
\end{Entry}

\begin{Entry}{完成}{7,6}{⼧、⼽}
  \begin{Phonetics}{完成}{wan2cheng2}[][HSK 2]
    \definition{v.}{realizar; completar; terminar; cumprir; levar ao sucesso}
  \end{Phonetics}
\end{Entry}

\begin{Entry}{完毕}{7,6}{⼧、⽐}
  \begin{Phonetics}{完毕}{wan2bi4}
    \definition{v.}{completar | terminar | acabar}
  \end{Phonetics}
\end{Entry}

\begin{Entry}{完完全全}{7,7,6,6}{⼧、⼧、⼊、⼊}
  \begin{Phonetics}{完完全全}{wan2wan2quan2quan2}
    \definition{adv.}{completamente}
  \end{Phonetics}
\end{Entry}

\begin{Entry}{完备}{7,8}{⼧、⼡}
  \begin{Phonetics}{完备}{wan2bei4}
    \definition{adj.}{completo | impecável | perfeito}
    \definition{v.}{não deixar nada a desejar}
  \end{Phonetics}
\end{Entry}

\begin{Entry}{完美}{7,9}{⼧、⽺}
  \begin{Phonetics}{完美}{wan2mei3}[][HSK 3]
    \definition{adj.}{perfeito; impecável; consumado}
  \end{Phonetics}
\end{Entry}

\begin{Entry}{完善}{7,12}{⼧、⼝}
  \begin{Phonetics}{完善}{wan2shan4}[][HSK 3]
    \definition{adj.}{perfeito; consumado}
    \definition{v.}{refinar; melhorar; tornar perfeito}
  \end{Phonetics}
\end{Entry}

\begin{Entry}{完税}{7,12}{⼧、⽲}
  \begin{Phonetics}{完税}{wan2shui4}
    \definition{v.}{pagar imposto}
  \end{Phonetics}
\end{Entry}

\begin{Entry}{完满}{7,13}{⼧、⽔}
  \begin{Phonetics}{完满}{wan2man3}
    \definition{adj.}{satisfatório | bem-sucedido}
  \end{Phonetics}
\end{Entry}

\begin{Entry}{完整}{7,16}{⼧、⽁}
  \begin{Phonetics}{完整}{wan2zheng3}[][HSK 3]
    \definition{adj.}{intacto; inteiro; completo; integrado; nenhum dano ou mutilação}
  \end{Phonetics}
\end{Entry}

\begin{Entry}{宏}{7}{⼧}
  \begin{Phonetics}{宏}{hong2}
    \definition*{s.}{Sobrenome Hong}
    \definition{adj.}{grande; grandioso; magnífico}
    \definition{v.}{divulgar algo; promover algo; atualmente, geralmente é escrito como 弘}
  \seealsoref{弘}{hong2}
  \end{Phonetics}
\end{Entry}

\begin{Entry}{宏大}{7,3}{⼧、⼤}
  \begin{Phonetics}{宏大}{hong2 da4}[][HSK 6]
    \definition{adj.}{grande; ótimo | imenso; vasto}
  \end{Phonetics}
\end{Entry}

\begin{Entry}{寿}{7}{⼨}
  \begin{Phonetics}{寿}{shou4}
    \definition[个,份]{s.}{vida longa; velhice | vida; idade | aniversário | (eufenismo) funerário; preparado antes da morte | longevidade}
  \end{Phonetics}
\end{Entry}

\begin{Entry}{寿司}{7,5}{⼨、⼝}
  \begin{Phonetics}{寿司}{shou4 si1}[][HSK 5]
    \definition[份]{s.}{\emph{sushi}; iguaria tradicional japonesa}
  \end{Phonetics}
\end{Entry}

\begin{Entry}{尾}{7}{⼫}
  \begin{Phonetics}{尾}{wei3}
    \definition*{s.}{Wei, sexta das vinte e oito constelações nas quais a esfera celeste foi dividida, consistindo de nove estrelas em forma de gancho em Escorpião| Wei, uma das mansões lunares | Sobrenome Wei}
    \definition{clas.}{usado para peixes}
    \definition{s.}{cauda; traseira | parte semelhante a uma cauda | fim | parte restante (ou inacabada); remanescente; a parte fora da parte principal; negócio inacabado}
  \end{Phonetics}
  \begin{Phonetics}{尾}{yi3}
    \definition{s.}{rabo do cavalo | parte posterior pontiaguda de um gafanhoto etc.}
  \end{Phonetics}
\end{Entry}

\begin{Entry}{尾巴}{7,4}{⼫、⼰}
  \begin{Phonetics}{尾巴}{wei3ba5}[][HSK 4]
    \definition[条,根]{s.}{cauda; projeções na extremidade do corpo de certos animais | parte semelhante a uma cauda; refere-se, em geral, ao final de algo | apêndice; anexo; adepto servil; pessoa que segue ou concorda com outra pessoa | (figura de linguagem) alguém que faz sombra a outro | fim; remanescente; parte restante (ou inacabada)}
  \end{Phonetics}
\end{Entry}

\begin{Entry}{尿}{7}{⼫}
  \begin{Phonetics}{尿}{niao4}
    \definition[泡]{s.}{urina}
    \definition{v.}{urinar}
  \end{Phonetics}
  \begin{Phonetics}{尿}{sui1}
    \definition{s.}{(coloquial) urina}
  \end{Phonetics}
\end{Entry}

\begin{Entry}{局}{7}{⼫}
  \begin{Phonetics}{局}{ju2}[][HSK 4,6]
    \definition{adj.}{limitado; confinado}
    \definition{clas.}{\emph{set}; jogo; turno}
    \definition{s.}{tabuleiro de xadrez | situação; estado de coisas | generosidade de espírito; extensão da tolerância de alguém | festa; reunião; refere-se a certas reuniões | ardil; armadilha | parte; porção; papel | escritório; agência; agências governamentais divididas por negócios | significa ``loja'' em nomes de lojas | departamento; agência; nomes de certas entidades empresariais | escritório; usado como nome de uma instituição ou outro local de negócios}
  \end{Phonetics}
\end{Entry}

\begin{Entry}{局长}{7,4}{⼫、⾧}
  \begin{Phonetics}{局长}{ju2 zhang3}[][HSK 5]
    \definition[位,名,个,些]{s.}{comissário; diretor; principais chefes de gabinete do governo}
  \end{Phonetics}
\end{Entry}

\begin{Entry}{局面}{7,9}{⼫、⾯}
  \begin{Phonetics}{局面}{ju2mian4}[][HSK 5]
    \definition[种]{s.}{aspecto; fase; situação; o estado das coisas em um período de tempo, em sua maior parte abstraído | escopo; escala}
  \end{Phonetics}
\end{Entry}

\begin{Entry}{屁}{7}{⼫}
  \begin{Phonetics}{屁}{pi4}
    \definition{s.}{vento (ou gás) (dos intestinos); peido | (vulgar) bobagem; merda; lixo | quadril; bunda}
  \end{Phonetics}
\end{Entry}

\begin{Entry}{屁股}{7,8}{⼫、⾁}
  \begin{Phonetics}{屁股}{pi4gu5}
    \definition{s.}{nádega | quadris}
  \end{Phonetics}
\end{Entry}

\begin{Entry}{屁话}{7,8}{⼫、⾔}
  \begin{Phonetics}{屁话}{pi4hua4}
    \definition{s.}{absurdo | tolice | besteira}
  \end{Phonetics}
\end{Entry}

\begin{Entry}{层}{7}{⼫}
  \begin{Phonetics}{层}{ceng2}[][HSK 2]
    \definition{clas.}{usado para coisas que se sobrepõem e se acumulam, como andares, camadas e estratos | usado para coisas que podem ser divididas em itens e etapas | usado para coisas que podem ser removidas ou apagadas da superfície de um objeto}
    \definition{s.}{camada; nível; coisas que se sobrepõem | nível; classificação; camada}
    \definition{v.}{sobrepor; empilhar camada sobre camada}
  \end{Phonetics}
\end{Entry}

\begin{Entry}{层出不穷}{7,5,4,7}{⼫、⼐、⼀、⽳}
  \begin{Phonetics}{层出不穷}{ceng2chu1-bu4qiong2}[][HSK 7-9]
    \definition{expr.}{surgem um após o outro; surgem em um fluxo sem fim; aparecem em sucessão; surgem continuamente; sem fim}
  \end{Phonetics}
\end{Entry}

\begin{Entry}{层次}{7,6}{⼫、⽋}
  \begin{Phonetics}{层次}{ceng2ci4}[][HSK 5]
    \definition[个]{s.}{disposição ordenada do conteúdo (de um discurso ou texto) | nível ou estrutura administrativa; distinções entre a mesma coisa devido a diferenças de tamanho, altura, etc. | nível; níveis de afiliação}
  \end{Phonetics}
\end{Entry}

\begin{Entry}{层层}{7,7}{⼫、⼫}
  \begin{Phonetics}{层层}{ceng2ceng2}
    \definition{s.}{camada sobre camada}
  \end{Phonetics}
\end{Entry}

\begin{Entry}{层面}{7,9}{⼫、⾯}
  \begin{Phonetics}{层面}{ceng2 mian4}[][HSK 6]
    \definition[个]{s.}{escopo; alcance | aspecto; campo}
  \end{Phonetics}
\end{Entry}

\begin{Entry}{岗}{7}{⼭}
  \begin{Phonetics}{岗}{gang3}
    \definition{s.}{outeiro; monte | crista; vergão (no rosto, pele, etc.) | sentinela; posto | trabalho | batida policial}
  \end{Phonetics}
\end{Entry}

\begin{Entry}{岗位}{7,7}{⼭、⼈}
  \begin{Phonetics}{岗位}{gang3 wei4}[][HSK 6]
    \definition[个,类]{s.}{posto; estação; originalmente se refere ao local guardado pelos militares e pela polícia, agora se refere a uma posição geral}
  \end{Phonetics}
\end{Entry}

\begin{Entry}{岛}{7}{⼭}
  \begin{Phonetics}{岛}{dao3}[][HSK 6]
    \definition[个,座]{s.}{ilha; uma massa de terra menor que um continente cercada por água}
  \end{Phonetics}
\end{Entry}

\begin{Entry}{岛屿}{7,6}{⼭、⼭}
  \begin{Phonetics}{岛屿}{dao3yu3}[][HSK 7-9]
    \definition[座,些,群]{s.}{ilha; ilhota}
  \end{Phonetics}
\end{Entry}

\begin{Entry}{希}{7}{⼱}
  \begin{Phonetics}{希}{xi1}
    \definition*{s.}{Sobrenome Xi}
    \definition{v.}{ter esperança}
  \end{Phonetics}
\end{Entry}

\begin{Entry}{希望}{7,11}{⼱、⽉}
  \begin{Phonetics}{希望}{xi1wang4}[][HSK 3]
    \definition[个,丝,点]{s.}{esperança; desejo; expectativa; a possibilidade de alcançar um determinado objetivo ou de ocorrer uma determinada situação ideal no futuro | aquilo em que a esperança é depositada; o objeto da esperança}
    \definition{v.}{ter esperança; desejar; esperar; pensar em alcançar algum objetivo ou que alguma situação ocorra}
  \end{Phonetics}
\end{Entry}

\begin{Entry}{床}{7}{⼴}
  \begin{Phonetics}{床}{chuang2}[][HSK 1]
    \definition{clas.}{usado para colchas, roupas de cama, etc.}
    \definition[张]{s.}{cama; sofá; móveis para dormir | algo com o formato de uma cama}
  \end{Phonetics}
\end{Entry}

\begin{Entry}{床位}{7,7}{⼴、⼈}
  \begin{Phonetics}{床位}{chuang2wei4}[][HSK 7-9]
    \definition{s.}{beliche; cama; camas para pacientes, viajantes e hóspedes em hospitais, navios e dormitórios}
  \end{Phonetics}
\end{Entry}

\begin{Entry}{库}{7}{⼴}
  \begin{Phonetics}{库}{ku4}[][HSK 5]
    \definition{s.}{depósito; tesouraria; armazém; almoxarifado; edifícios e equipamentos para armazenamento de mercadorias | Computação: banco de dados}
  \end{Phonetics}
\end{Entry}

\begin{Entry}{应}{7}{⼴}
  \begin{Phonetics}{应}{ying1}[][HSK 4,5]
    \definition{v.}{ecoar; responder; responder a; responder às chamadas, saudações, perguntas, etc. de outras pessoas | conceder; cumprir | adequar; adaptar; responder a | lidar com; enfrentar; abordar | tornar-se realidade; ser cumprido}
  \end{Phonetics}
\end{Entry}

\begin{Entry}{应对}{7,5}{⼴、⼨}
  \begin{Phonetics}{应对}{ying4 dui4}[][HSK 6]
    \definition{v.}{reagir; responder; lidar com; dar uma resposta; tomar medidas e contramedidas para lidar com a situação}
  \end{Phonetics}
\end{Entry}

\begin{Entry}{应用}{7,5}{⼴、⽤}
  \begin{Phonetics}{应用}{ying4yong4}[][HSK 3]
    \definition{adj.}{aplicado (na vida ou na produção); usado diretamente na vida ou na produção}
    \definition{v.}{usar; aplicar}
  \end{Phonetics}
\end{Entry}

\begin{Entry}{应用程序}{7,5,12,7}{⼴、⽤、⽲、⼴}
  \begin{Phonetics}{应用程序}{ying4yong4 cheng2xu4}
    \definition{s.}{programa aplicativo; principais categorias de \emph{software}}
  \end{Phonetics}
\end{Entry}

\begin{Entry}{应用程序接口}{7,5,12,7,11,3}{⼴、⽤、⽲、⼴、⼿、⼝}
  \begin{Phonetics}{应用程序接口}{ying4yong4 cheng2xu4 jie1kou3}
    \definition{s.}{API (\emph{application programming interface})}
  \seealsoref{应用程序编程接口}{ying4yong4 cheng2xu4 bian1cheng2 jie1kou3}
  \end{Phonetics}
\end{Entry}

\begin{Entry*}{应用程序编程接口}{7,5,12,7,12,12,11,3}{⼴、⽤、⽲、⼴、⽷、⽲、⼿、⼝}
  \begin{Phonetics}{应用程序编程接口}{ying4yong4 cheng2xu4 bian1cheng2 jie1kou3}
    \definition{s.}{API (\emph{application programming interface})}
  \seealsoref{应用程序接口}{ying4yong4 cheng2xu4 jie1kou3}
  \end{Phonetics}
\end{Entry*}

\begin{Entry}{应当}{7,6}{⼴、⼹}
  \begin{Phonetics}{应当}{ying1 dang1}[][HSK 3]
    \definition{v.}{dever}[学生们应当努力学习。===Os alunos devem se esforçar nos estudos.]
  \end{Phonetics}
\end{Entry}

\begin{Entry}{应该}{7,8}{⼴、⾔}
  \begin{Phonetics}{应该}{ying1gai1}[][HSK 2]
    \definition{v.}{deveria; deve ser assim | deveria; acho que deve ser esse o caso}
  \end{Phonetics}
\end{Entry}

\begin{Entry}{应急}{7,9}{⼴、⼼}
  \begin{Phonetics}{应急}{ying4 ji2}[][HSK 6]
    \definition{v.}{atender a uma necessidade urgente (emergência, contingência, etc.)}
  \end{Phonetics}
\end{Entry}

\begin{Entry}{弄}{7}{⼶}
  \begin{Phonetics}{弄}{long4}
    \definition{s.}{rua estreita; beco; viela; travessa}
  \end{Phonetics}
  \begin{Phonetics}{弄}{nong4}[][HSK 2]
    \definition{v.}{fazer, realizar; tratar; organizar | obter; buscar; tentar conseguir; encontrar uma maneira de conseguir | brincar com; enganar | pregar uma peça; brincar; manipular | mexer com; perturbar}
  \end{Phonetics}
\end{Entry}

\begin{Entry}{弟}{7}{⼸}
  \begin{Phonetics}{弟}{di4}[][HSK 1]
    \definition*{s.}{Sobrenome Di}
    \definition[个]{s.}{irmão mais novo | (entre amigos homens) eu | geralmente se refere a colegas do sexo masculino mais jovens na família ou entre parentes | forma humilde que os amigos usam para se referir uns aos outros, usada principalmente em correspondência}
  \end{Phonetics}
\end{Entry}

\begin{Entry}{弟子}{7,3}{⼸、⼦}
  \begin{Phonetics}{弟子}{di4zi3}[][HSK 7-9]
    \definition{s.}{discípulo; pupilo; seguidor; nome antigo para estudante, aprendiz}
  \end{Phonetics}
\end{Entry}

\begin{Entry}{弟弟}{7,7}{⼸、⼸}
  \begin{Phonetics}{弟弟}{di4 di5}[][HSK 1]
    \definition[个,位]{s.}{irmão mais novo | primo}
  \end{Phonetics}
\end{Entry}

\begin{Entry}{弟妹}{7,8}{⼸、⼥}
  \begin{Phonetics}{弟妹}{di4mei4}
    \definition{s.}{esposa do irmão mais novo}
  \end{Phonetics}
\end{Entry}

\begin{Entry}{张}{7}{⼸}
  \begin{Phonetics}{张}{zhang1}[][HSK 3]
    \definition*{s.}{Zhang, uma das vinte e oito constelações | Zhang, uma das mansões lunares | Sobrenome Zhang}
    \definition{adj.}{indulgente; desenfreado; devasso; libertino}
    \definition{clas.}{usado para papel, couro, etc. | usado para cama, mesa, etc. | usado para o rosto, boca, etc. | usado para arco}
    \definition{v.}{abrir; espalhar; esticar | expor; exibir | (de uma loja) iniciar atividades comerciais; abrir | olhar; contemplar | expandir; estender; ampliar; exagerar | fixar (uma corda de arco); encordoar (um instrumento musical); puxar a corda do arco}
  \end{Phonetics}
\end{Entry}

\begin{Entry}{张三}{7,3}{⼸、⼀}
  \begin{Phonetics}{张三}{zhang1san1}
    \definition*{s.}{Zhang San | Zé Ninguém | Nome para uma pessoa não especificada, 1 de 3}
  \seealsoref{李四}{li3si4}
  \seealsoref{王五}{wang2wu3}
  \end{Phonetics}
\end{Entry}

\begin{Entry}{张狂}{7,7}{⼸、⽝}
  \begin{Phonetics}{张狂}{zhang1kuang2}
    \definition{adj.}{impetuoso | frenético | insolente}
  \end{Phonetics}
\end{Entry}

\begin{Entry}{张南庄}{7,9,6}{⼸、⼗、⼴}
  \begin{Phonetics}{张南庄}{zhang1 nan2zhuang1}
    \definition*{s.}{Zhang Nanzhuang (anos de nascimento e morte desconhecidos), que viveu durante os períodos Qianlong e Jiaqing da Dinastia Qing, era conhecido como ``Passante'' 过路人; ele era bom em caligrafia no estilo de Ouyang Xun e imitava Fan Chengda e Lu You na escrita de poesia; ele gastava milhares de moedas de ouro todos os anos para colecionar livros raros, e sua coleção era a maior de 上海 naquela época}
  \seealsoref{范成大}{fan4 cheng2da4}
  \seealsoref{过路人}{guo4lu4 ren2}
  \seealsoref{陆游}{lu4 you2}
  \seealsoref{欧阳询}{ou1yang2 xun2}
  \seealsoref{上海}{shang4hai3}
  \end{Phonetics}
\end{Entry}

\begin{Entry}{形}{7}{⼺}
  \begin{Phonetics}{形}{xing2}[][HSK 6]
    \definition{s.}{forma; formato | corpo; entidade}
    \definition{v.}{aparecer; revelar; mostrar | comparar; contrastar}
  \end{Phonetics}
\end{Entry}

\begin{Entry}{形式}{7,6}{⼺、⼷}
  \begin{Phonetics}{形式}{xing2shi4}[][HSK 3]
    \definition[种,个]{s.}{forma; formato; modalidade; a aparência, estrutura ou estado das coisas, etc.}
  \end{Phonetics}
\end{Entry}

\begin{Entry}{形成}{7,6}{⼺、⼽}
  \begin{Phonetics}{形成}{xing2cheng2}[][HSK 3]
    \definition{v.}{moldar; formar; tomar forma; tornar-se algo ou surgir uma situação após mudanças e desenvolvimentos}
  \end{Phonetics}
\end{Entry}

\begin{Entry}{形而上学}{7,6,3,8}{⼺、⽽、⼀、⼦}
  \begin{Phonetics}{形而上学}{xing2'er2shang4xue2}
    \definition{s.}{metafísica}
  \end{Phonetics}
\end{Entry}

\begin{Entry}{形状}{7,7}{⼺、⽝}
  \begin{Phonetics}{形状}{xing2zhuang4}[][HSK 3]
    \definition[个,种]{s.}{forma; aparência ; aspecto; a aparência de um objeto ou figura, representada pela combinação de superfícies ou linhas externas}
  \end{Phonetics}
\end{Entry}

\begin{Entry}{形势}{7,8}{⼺、⼒}
  \begin{Phonetics}{形势}{xing2shi4}[][HSK 4]
    \definition[个,种]{s.}{terreno; características topográficas; situação geográfica, principalmente de uma perspectiva militar | situação; circunstâncias; a situação geral, a tendência de como as coisas estão se desenvolvendo e mudando | geralmente não é usado em situações pessoais}
  \end{Phonetics}
\end{Entry}

\begin{Entry}{形态}{7,8}{⼺、⼼}
  \begin{Phonetics}{形态}{xing2tai4}[][HSK 5]
    \definition[种]{s.}{forma; forma como as coisas se apresentam | forma; padrão; postura | morfologia; forma; Gramática: refere-se às formas internas de mudança das palavras, incluindo a formação de palavras e as mudanças morfológicas}
  \end{Phonetics}
\end{Entry}

\begin{Entry}{形容}{7,10}{⼺、⼧}
  \begin{Phonetics}{形容}{xing2rong2}[][HSK 4]
    \definition{s.}{aparência; semblante}
    \definition{v.}{descrever}
  \end{Phonetics}
\end{Entry}

\begin{Entry}{形象}{7,11}{⼺、⾗}
  \begin{Phonetics}{形象}{xing2xiang4}[][HSK 3]
    \definition{adj.}{vívido; expressão concreta e vívida}
    \definition[个,种]{s.}{imagem; forma; figura; formas ou posturas específicas que podem despertar pensamentos ou emoções nas pessoas | imagem literária; imagem artística; pessoas ou coisas com características diferentes criadas na literatura, no cinema e em outras artes}
  \end{Phonetics}
\end{Entry}

\begin{Entry}{彻}{7}{⼻}
  \begin{Phonetics}{彻}{che4}
    \definition{adj.}{minucioso; completo; penetrante}
    \definition{adv.}{minuciosamente; profundamente}
  \end{Phonetics}
\end{Entry}

\begin{Entry}{彻夜}{7,8}{⼻、⼣}
  \begin{Phonetics}{彻夜}{che4ye4}[][HSK 7-9]
    \definition{adv.}{a noite toda; durante toda a noite; do anoitecer ao amanhecer}
  \end{Phonetics}
\end{Entry}

\begin{Entry}{彻底}{7,8}{⼻、⼴}
  \begin{Phonetics}{彻底}{che4di3}[][HSK 4]
    \definition{adj.}{minucioso; completo; exaustivo; profundo e completo; nada é deixado de fora}
  \end{Phonetics}
\end{Entry}

\begin{Entry}{忍}{7}{⼼}
  \begin{Phonetics}{忍}{ren3}[][HSK 5]
    \definition{v.}{suportar; aguentar; tolerar; aturar | ter coragem para; ser insensível o suficiente para; ser capaz de endurecer o coração e fazer coisas que não se devem fazer por uma questão de razão}
  \end{Phonetics}
\end{Entry}

\begin{Entry}{忍不住}{7,4,7}{⼼、⼀、⼈}
  \begin{Phonetics}{忍不住}{ren3bu5zhu4}[][HSK 5]
    \definition{v.}{incapaz de suportar; não conseguir evitar fazer algo; não conseguir se controlar}
  \end{Phonetics}
\end{Entry}

\begin{Entry}{忍受}{7,8}{⼼、⼜}
  \begin{Phonetics}{忍受}{ren3shou4}[][HSK 5]
    \definition{v.}{suportar; sofrer; aguentar; tolerar; suportar com dificuldade o sofrimento, as dificuldades e as adversidades da vida}
  \end{Phonetics}
\end{Entry}

\begin{Entry}{忍耐}{7,9}{⼼、⽽}
  \begin{Phonetics}{忍耐}{ren3nai4}
    \definition{s.}{paciência | resistência}
    \definition{v.}{suportar | resistir | exercer paciência}
  \end{Phonetics}
\end{Entry}

\begin{Entry}{志}{7}{⼼}
  \begin{Phonetics}{志}{zhi4}
    \definition{s.}{vontade; ideal; aspiração; ambição | anais; registros; transcrição | marca; sinal}
    \definition{v.}{ter em mente; lembrar | pesar; medir comprimento e quantidade}
  \end{Phonetics}
\end{Entry}

\begin{Entry}{志愿}{7,14}{⼼、⽕}
  \begin{Phonetics}{志愿}{zhi4 yuan4}[][HSK 3]
    \definition{s.}{desejo; ideal; aspiração; os ideais, desejos ou objetivos que se deseja realizar no coração}
    \definition{v.}{ser voluntário; ser proativo e disposto a realizar trabalhos sem remuneração ou com remuneração baixa, mas que possam ajudar outras pessoas}
  \end{Phonetics}
\end{Entry}

\begin{Entry}{志愿书}{7,14,4}{⼼、⽕、⼄}
  \begin{Phonetics}{志愿书}{zhi4yuan4shu1}
    \definition{s.}{formulário de inscrição; formulário de adesão; carta de intenções}
  \end{Phonetics}
\end{Entry}

\begin{Entry}{志愿者}{7,14,8}{⼼、⽕、⽼}
  \begin{Phonetics}{志愿者}{zhi4yuan4zhe3}[][HSK 3]
    \definition[名,位,个]{s.}{voluntário; pessoas que se voluntariam para prestar serviços em atividades sociais, grandes eventos esportivos, conferências, etc.}
  \end{Phonetics}
\end{Entry}

\begin{Entry}{忘}{7}{⼼}
  \begin{Phonetics}{忘}{wang4}[][HSK 1]
    \definition{v.}{esquecer | ignorar; negligenciar}
  \end{Phonetics}
\end{Entry}

\begin{Entry}{忘本}{7,5}{⼼、⽊}
  \begin{Phonetics}{忘本}{wang4ben3}
    \definition{v.}{esquecer as próprias raízes}
  \end{Phonetics}
\end{Entry}

\begin{Entry}{忘记}{7,5}{⼼、⾔}
  \begin{Phonetics}{忘记}{wang4ji4}[][HSK 1]
    \definition{v.}{esquecer | ignorar; negligenciar | sair da memória de alguém; não ser lembrado | descartar da mente; ignorar}
  \end{Phonetics}
\end{Entry}

\begin{Entry}{忘却}{7,7}{⼼、⼙}
  \begin{Phonetics}{忘却}{wang4que4}
    \definition{v.}{esquecer}
  \end{Phonetics}
\end{Entry}

\begin{Entry}{忘怀}{7,7}{⼼、⼼}
  \begin{Phonetics}{忘怀}{wang4huai2}
    \definition{v.}{esquecer}
  \end{Phonetics}
\end{Entry}

\begin{Entry}{忘恩}{7,10}{⼼、⼼}
  \begin{Phonetics}{忘恩}{wang4'en1}
    \definition{v.}{ser ingrato}
  \end{Phonetics}
\end{Entry}

\begin{Entry}{忘掉}{7,11}{⼼、⼿}
  \begin{Phonetics}{忘掉}{wang4diao4}
    \definition{v.}{esquecer}
  \end{Phonetics}
\end{Entry}

\begin{Entry}{忘餐}{7,16}{⼼、⾷}
  \begin{Phonetics}{忘餐}{wang4can1}
    \definition{v.}{esquecer as refeições}
  \end{Phonetics}
\end{Entry}

\begin{Entry}{忧}{7}{⼼}
  \begin{Phonetics}{忧}{you1}
    \definition{s.}{tristeza; ansiedade; preocupação; cuidado; coisas que causam tristeza}
    \definition{v.}{preocupar-se; estar preocupado; estar ansioso; estar triste}
  \end{Phonetics}
\end{Entry}

\begin{Entry}{忧郁}{7,8}{⼼、⾢}
  \begin{Phonetics}{忧郁}{you1yu4}
    \definition{adj.}{deprimido | melancólico | desanimado}
    \definition{s.}{depressão | melancolia}
  \end{Phonetics}
\end{Entry}

\begin{Entry}{快}{7}{⼼}
  \begin{Phonetics}{快}{kuai4}[][HSK 1]
    \definition*{s.}{Sobrenome Kuai}
    \definition{adj.}{rápido; veloz (oposto a 慢) | apressado | perspicaz; ágil; inteligente; de ​​mente rápida | (de uma faca, espada, etc.) afiado (oposto a 钝) | direto; franco; sem rodeios | satisfeito; feliz; gratificado | rápido; veloz; alta velocidade; tempo de execução curto | satisfeito; feliz; contente | engenhoso; ágil | afiado; facas, tesouras, machados e outros objetos afiados | sincero}
    \definition{adv.}{em breve; antes de muito tempo; estar prestes a | rapidamente}
    \definition{s.}{policial; polícia | (antigo) oficial encarregado de efetuar prisões}
  \seealsoref{钝}{dun4}
  \seealsoref{慢}{man4}
  \end{Phonetics}
\end{Entry}

\begin{Entry}{快车}{7,4}{⼼、⾞}
  \begin{Phonetics}{快车}{kuai4 che1}[][HSK 6]
    \definition{s.}{trem ou ônibus expresso (em oposição a 慢车); um trem ou ônibus com menos paradas e tempos de viagem mais curtos (usado principalmente para transporte de passageiros)}
  \seealsoref{慢车}{man4 che1}
  \end{Phonetics}
\end{Entry}

\begin{Entry}{快乐}{7,5}{⼼、⼃}
  \begin{Phonetics}{快乐}{kuai4le4}[][HSK 2]
    \definition{adj.}{feliz; alegre; animado; prazeiroso}
    \definition{s.}{felicidade | alegria}
  \end{Phonetics}
\end{Entry}

\begin{Entry}{快活}{7,9}{⼼、⽔}
  \begin{Phonetics}{快活}{kuai4huo5}[][HSK 5]
    \definition{adj.}{feliz; alegre; contente; animado}
  \end{Phonetics}
\end{Entry}

\begin{Entry}{快点儿}{7,9,2}{⼼、⽕、⼉}
  \begin{Phonetics}{快点儿}{kuai4 dian3r5}[][HSK 2]
    \definition{v.}{apressar-se}
  \end{Phonetics}
\end{Entry}

\begin{Entry}{快要}{7,9}{⼼、⾑}
  \begin{Phonetics}{快要}{kuai4 yao4}[][HSK 2]
    \definition{adv.}{estar prestes a; estar indo para; estar à beira de; em breve; em pouco tempo; indica que a situação está prestes a ocorrer}
  \end{Phonetics}
\end{Entry}

\begin{Entry}{快递}{7,10}{⼼、⾡}
  \begin{Phonetics}{快递}{kuai4 di4}[][HSK 4]
    \definition[个,件,批]{s.}{correio rápido; entrega expressa; entrega rápida}
    \definition{v.}{entregar (serviço de entrega rápida por transportadoras especializadas)}
  \end{Phonetics}
\end{Entry}

\begin{Entry}{快速}{7,10}{⼼、⾡}
  \begin{Phonetics}{快速}{kuai4 su4}[][HSK 3]
    \definition{adj.}{rápido; veloz; de alta velocidade; descreve o tempo curto gasto para caminhar, fazer algo, etc.}
  \end{Phonetics}
\end{Entry}

\begin{Entry}{快餐}{7,16}{⼼、⾷}
  \begin{Phonetics}{快餐}{kuai4 can1}[][HSK 2]
    \definition[份,顿]{s.}{pedido (comida) rápido; \emph{fast food}; refere-se a refeições simples preparadas com antecedência e que podem ser servidas rapidamente}
  \end{Phonetics}
\end{Entry}

\begin{Entry}{怀}{7}{⼼}
  \begin{Phonetics}{怀}{huai2}
    \definition*{s.}{Sobrenome Huai}
    \definition{s.}{seio; peito | mente}
    \definition{v.}{manter em mente; estimar; abrigar | sentir falta; pensar em; ansiar por | conceber (uma criança)}
  \end{Phonetics}
\end{Entry}

\begin{Entry}{怀旧}{7,5}{⼼、⽇}
  \begin{Phonetics}{怀旧}{huai2jiu4}
    \definition{s.}{nostalgia}
    \definition{v.}{sentir-se nostálgico}
  \end{Phonetics}
\end{Entry}

\begin{Entry}{怀念}{7,8}{⼼、⼼}
  \begin{Phonetics}{怀念}{huai2nian4}[][HSK 4]
    \definition{v.}{pensar em; valorizar a memória de}
  \end{Phonetics}
\end{Entry}

\begin{Entry}{怀疑}{7,14}{⼼、⽦}
  \begin{Phonetics}{怀疑}{huai2yi2}[][HSK 4]
    \definition{v.}{duvidar; suspeitar | supor}
  \end{Phonetics}
\end{Entry}

\begin{Entry}{我}{7}{⼽}
  \begin{Phonetics}{我}{wo3}[][HSK 1]
    \definition{pron.}{eu; mim | um; qualquer um; usado para contrastar 他 e 我; refere-se a muitas pessoas em geral}
  \seealsoref{他}{ta1}
  \end{Phonetics}
\end{Entry}

\begin{Entry}{我们}{7,5}{⼽、⼈}
  \begin{Phonetics}{我们}{wo3men5}[][HSK 1]
    \definition{pron.}{nós; nos}
  \end{Phonetics}
\end{Entry}

\begin{Entry}{我们的}{7,5,8}{⼽、⼈、⽩}
  \begin{Phonetics}{我们的}{wo3men5 de5}
    \definition{pron.}{nosso, nossos}
  \end{Phonetics}
\end{Entry}

\begin{Entry}{我去}{7,5}{⼽、⼛}
  \begin{Phonetics}{我去}{wo3qu4}
    \definition{interj.}{(gíria) O que\dots!! | Oh meu Deus! | Isso é insano!}
  \end{Phonetics}
\end{Entry}

\begin{Entry}{我的}{7,8}{⼽、⽩}
  \begin{Phonetics}{我的}{wo3 de5}
    \definition{pron.}{meu, meus}
  \end{Phonetics}
\end{Entry}

\begin{Entry}{戒}{7}{⼽}
  \begin{Phonetics}{戒}{jie4}[][HSK 5]
    \definition[个,枚]{s.}{advertência; exortação | disciplina monástica budista; preceitos budistas | anel (dedo)}
    \definition{v.}{proteger-se contra; estar preparado; estar atento | advertir; exortar; admoestar | abandonar; parar; desistir; desistir (de um hábito ruim)}
  \end{Phonetics}
\end{Entry}

\begin{Entry}{扭}{7}{⼿}
  \begin{Phonetics}{扭}{niu3}[][HSK 6]
    \definition{v.}{virar-se; girar | torcer; girar | torcer; luxar | rolar; balançar (ao caminhar) | agarrar; pegar;  lutar com}
  \end{Phonetics}
\end{Entry}

\begin{Entry}{扮}{7}{⼿}
  \begin{Phonetics}{扮}{ban4}[][HSK 7-9]
    \definition{v.}{vestir-se como; desempenhar o papel de | maquiar-se; disfarçar-se como | (expressão facial) fazer cara de}
  \end{Phonetics}
\end{Entry}

\begin{Entry}{扮演}{7,14}{⼿、⽔}
  \begin{Phonetics}{扮演}{ban4yan3}[][HSK 5]
    \definition{v.}{desempenhar o papel de; ter um papel (em uma peça, etc.); atuar}
  \end{Phonetics}
\end{Entry}

\begin{Entry}{扯}{7}{⼿}
  \begin{Phonetics}{扯}{che3}[][HSK 7-9]
    \definition{v.}{puxar | rasgar; arrancar | comprar (tecido, linha, etc.) | conversar; fofocar; bater papo}
  \end{Phonetics}
\end{Entry}

\begin{Entry}{扳}{7}{⼿}
  \begin{Phonetics}{扳}{ban1}[][HSK 7-9]
    \definition{v.}{puxar; virar | reconquistar; compensar | reconquistar; virar o jogo; virar-se (uma situação de perda)}
  \end{Phonetics}
  \begin{Phonetics}{扳}{pan1}
    \definition{v.}{segurar; agarrar; puxar; escalar | confiar em; buscar ajuda; associar-se a pessoas de status superior; refere-se a formar um relacionamento ou estabelecer um relacionamento com alguém de alto \emph{status} | envolver; relacionar-se com}
  \end{Phonetics}
\end{Entry}

\begin{Entry}{扶}{7}{⼿}
  \begin{Phonetics}{扶}{fu2}[][HSK 5]
    \definition*{s.}{Sobrenome Fu}
    \definition{v.}{segurar; apoiar com a mão; segurar algo com o apoio das mãos para que ninguém, objeto ou pessoa caia | dar apoio a; ajudar uma pessoa deitada ou caída a se levantar com as mãos; endireitar um objeto caído com as mãos | ajudar; tirar de baixo}
  \end{Phonetics}
\end{Entry}

\begin{Entry}{扶持}{7,9}{⼿、⼿}
  \begin{Phonetics}{扶持}{fu2chi2}[][HSK 7-9]
    \definition{v.}{apoiar com a mão; colocar uma mão em alguém para apoio; apoiar | apoiar; dar ajuda a; ajudar a sustentar}
  \end{Phonetics}
\end{Entry}

\begin{Entry}{扶梯}{7,11}{⼿、⽊}
  \begin{Phonetics}{扶梯}{fu2ti1}
    \definition{s.}{escada rolante}
  \end{Phonetics}
\end{Entry}

\begin{Entry}{批}{7}{⼿}
  \begin{Phonetics}{批}{pi1}[][HSK 4]
    \definition{adj.}{(compra ou venda) atacado; a granel; em grandes quantidades}
    \definition{clas.}{usado para mercadorias a granel, grande número de pessoas}
    \definition{s.}{fibras de algodão, linho, etc., prontas para serem estiradas e torcidas | anotação; comentário}
    \definition{v.}{escrever comentários ou críticas sobre documentos subordinados, textos de outras pessoas, tarefas etc. | refutar; criticar | dar um tapa}
  \end{Phonetics}
\end{Entry}

\begin{Entry}{批评}{7,7}{⼿、⾔}
  \begin{Phonetics}{批评}{pi1ping2}[][HSK 3]
    \definition{v.}{criticar; comentar sobre deficiências e erros | criticar; apontar vantagens e desvantagens; comentar sobre o que é bom e o que é ruim}
  \end{Phonetics}
\end{Entry}

\begin{Entry}{批准}{7,10}{⼿、⼎}
  \begin{Phonetics}{批准}{pi1zhun3}[][HSK 3]
    \definition{v.}{aprovar}
  \end{Phonetics}
\end{Entry}

\begin{Entry}{找}{7}{⼿}
  \begin{Phonetics}{找}{zhao3}[][HSK 1]
    \definition{v.}{procurar; tentar encontrar; buscar | querer ver; visitar; abordar; solicitar | dar troco | descobrir; esforçar-se para ver ou obter a pessoa ou coisa desejada | examinar; investigar; completar as partes que faltam | causar intencionalmente (um resultado indesejável, negativo)}
  \end{Phonetics}
\end{Entry}

\begin{Entry}{找见}{7,4}{⼿、⾒}
  \begin{Phonetics}{找见}{zhao3jian4}
    \definition{v.}{encontrar (algo que está procurando)}
  \end{Phonetics}
\end{Entry}

\begin{Entry}{找出}{7,5}{⼿、⼐}
  \begin{Phonetics}{找出}{zhao3 chu1}[][HSK 2]
    \definition{v.}{encontrar | procurar}
  \end{Phonetics}
\end{Entry}

\begin{Entry}{找回}{7,6}{⼿、⼞}
  \begin{Phonetics}{找回}{zhao3hui2}
    \definition{v.}{recuperar algo}
  \end{Phonetics}
\end{Entry}

\begin{Entry}{找寻}{7,6}{⼿、⼨}
  \begin{Phonetics}{找寻}{zhao3xun2}
    \definition{v.}{encontrar falhas | procurar | buscar}
  \end{Phonetics}
\end{Entry}

\begin{Entry}{找事}{7,8}{⼿、⼅}
  \begin{Phonetics}{找事}{zhao3shi4}
    \definition{v.}{procurar emprego | começar uma briga}
  \end{Phonetics}
\end{Entry}

\begin{Entry}{找到}{7,8}{⼿、⼑}
  \begin{Phonetics}{找到}{zhao3 dao4}[][HSK 1]
    \definition{v.}{encontrar; procurar; achar; encontar através de pesquisa, exploração, etc.;  ver ou encontrar coisas ou padrões que os antepassados não viram}
  \end{Phonetics}
\end{Entry}

\begin{Entry}{找钱}{7,10}{⼿、⾦}
  \begin{Phonetics}{找钱}{zhao3qian2}
    \definition{v.}{dar troco}
  \end{Phonetics}
\end{Entry}

\begin{Entry}{找着}{7,11}{⼿、⽬}
  \begin{Phonetics}{找着}{zhao3zhao2}
    \definition{v.}{encontrar}
  \end{Phonetics}
\end{Entry}

\begin{Entry}{找遍}{7,12}{⼿、⾡}
  \begin{Phonetics}{找遍}{zhao3bian4}
    \definition{v.}{pentear | pesquisar em todos os lugares}
  \end{Phonetics}
\end{Entry}

\begin{Entry}{找零}{7,13}{⼿、⾬}
  \begin{Phonetics}{找零}{zhao3ling2}
    \definition{v.}{trocar dinheiro | dar troco}
  \end{Phonetics}
\end{Entry}

\begin{Entry}{找辙}{7,16}{⼿、⾞}
  \begin{Phonetics}{找辙}{zhao3zhe2}
    \definition{v.}{procurar um pretexto}
  \end{Phonetics}
\end{Entry}

\begin{Entry}{技}{7}{⼿}
  \begin{Phonetics}{技}{ji4}
    \definition[门,项]{s.}{destreza; habilidade; estratagema | técnica; tecnologia}
  \end{Phonetics}
\end{Entry}

\begin{Entry}{技巧}{7,5}{⼿、⼯}
  \begin{Phonetics}{技巧}{ji4qiao3}[][HSK 4]
    \definition[个,些]{s.}{habilidade; técnica; habilidades engenhosas expressas em artes, artesanato, esportes, etc.}
  \end{Phonetics}
\end{Entry}

\begin{Entry}{技术}{7,5}{⼿、⽊}
  \begin{Phonetics}{技术}{ji4shu4}[][HSK 3]
    \definition[种,门,项]{s.}{habilidade; técnica; tecnologia; a experiência e o conhecimento acumulados pelo ser humano no processo de utilização e transformação da natureza, e refletidos no trabalho produtivo, também se referem, de maneira geral, a outras habilidades operacionais}
  \end{Phonetics}
\end{Entry}

\begin{Entry}{技俩}{7,9}{⼿、⼈}
  \begin{Phonetics}{技俩}{ji4liang3}
    \definition{s.}{truque | estratagema | ardil | esquema | estratégia | tática}
  \end{Phonetics}
\end{Entry}

\begin{Entry}{技能}{7,10}{⼿、⾁}
  \begin{Phonetics}{技能}{ji4 neng2}[][HSK 5]
    \definition[种,项]{s.}{habilidade técnica; domínio de uma habilidade ou técnica; capacidade de adquirir e aplicar conhecimento}
  \end{Phonetics}
\end{Entry}

\begin{Entry}{抄}{7}{⼿}
  \begin{Phonetics}{抄}{chao1}[][HSK 4]
    \definition*{s.}{Sobrenome Chao}
    \definition{v.}{copiar; transcrever | plagiar | registrar as leituras de um medidor | revistar e confiscar; fazer uma incursão em  | pegar um atalho | dobrar (os braços) | agarrar; pegar | ir (andar) embora com}
  \end{Phonetics}
\end{Entry}

\begin{Entry}{抄写}{7,5}{⼿、⼍}
  \begin{Phonetics}{抄写}{chao1 xie3}[][HSK 4]
    \definition{v.}{copiar; transcrever}
  \end{Phonetics}
\end{Entry}

\begin{Entry}{抄表}{7,8}{⼿、⾐}
  \begin{Phonetics}{抄表}{chao1 biao3}
    \definition{s.}{leitura do medidor}
  \end{Phonetics}
\end{Entry}

\begin{Entry}{抄袭}{7,11}{⼿、⾐}
  \begin{Phonetics}{抄袭}{chao1xi2}[][HSK 7-9]
    \definition{v.}{plagiar | emular indiscriminadamente; copiar | Militar: lançar um ataque surpresa ao inimigo fazendo um desvio; atacar a retaguarda ou os flancos do inimigo | Militar: tomar emprestado indiscriminadamente da experiência de outras pessoas; copiar a experiência e os métodos de outras pessoas}
  \end{Phonetics}
\end{Entry}

\begin{Entry}{把}{7}{⼿}
  \begin{Phonetics}{把}{ba3}[][HSK 3]
    \definition{adj.}{referindo-se à relação de irmandade}
    \definition{clas.}{usado antes de objetos com alças ou coisas para segurar | um punhado de; a quantidade que se pode pegar com uma mão | usado antes de coisas abstratas | usado em coisas feitas com as mãos | número de ações, coisas}
    \definition{part.}{adicionado após quantificadores como 百, 千, 万 e 里, 斤, 个, indica que a quantidade é próxima dessa unidade (não pode ser adicionado outro quantificador antes)}
    \definition{prep.}{fazer uma determinada alteração em um objeto; causar uma determinada mudança em um objeto | fazer com que os outros façam/sintam algo}
    \definition{s.}{alça; punho; a parte que se segura | feixe; molho; algo que se segura com as mãos ou se amarra em pequenos feixes}
    \definition{v.}{agarrar; segurar | segurar (um bebê enquanto ele urina) | controlar; dominar; monopolizar | encostar-se; apoiar-se | vigiar (locais importantes); observar; guardar | dar | usar algo como; considerar como; tratar como; conter o significado de 拿 | acorrentar; trancar}
  \seealsoref{百}{bai3}
  \seealsoref{个}{ge4}
  \seealsoref{斤}{jin1}
  \seealsoref{里}{li3}
  \seealsoref{拿}{na2}
  \seealsoref{千}{qian1}
  \seealsoref{万}{wan4}
  \end{Phonetics}
  \begin{Phonetics}{把}{ba4}
    \definition{s.}{punho; alça; empunhadura; parte do utensílio que é fácil de segurar com a mão |haste (de uma folha, flor ou fruto) | motivo de ridículo; alvo; comportamentos e declarações que servem de assunto para piadas}
  \end{Phonetics}
\end{Entry}

\begin{Entry}{把手}{7,4}{⼿、⼿}
  \begin{Phonetics}{把手}{ba3shou5}[][HSK 7-9]
    \definition[个]{s.}{pega; botão; alça; as partes de portas, janelas, móveis, etc. que são fáceis de segurar com as mãos}
  \end{Phonetics}
\end{Entry}

\begin{Entry}{把风}{7,4}{⼿、⾵}
  \begin{Phonetics}{把风}{ba3feng1}
    \definition{v.}{vigiar (em uma atividade clandestina) | estar atento}
  \end{Phonetics}
\end{Entry}

\begin{Entry}{把关}{7,6}{⼿、⼋}
  \begin{Phonetics}{把关}{ba3/guan1}[][HSK 7-9]
    \definition{v.+compl.}{verificar rigorosamente; examinar cuidadosamente para ver se algo está sendo feito de acordo com o padrão fixo; fazer a verificação final | proteger uma fronteira, passagem, portões, etc.}
  \end{Phonetics}
\end{Entry}

\begin{Entry}{把守}{7,6}{⼿、⼧}
  \begin{Phonetics}{把守}{ba3shou3}
    \definition{v.}{vigiar | guardar}
  \end{Phonetics}
\end{Entry}

\begin{Entry}{把式}{7,6}{⼿、⼷}
  \begin{Phonetics}{把式}{ba3shi4}
    \definition{s.}{pessoa qualificada em um comércio}
  \end{Phonetics}
\end{Entry}

\begin{Entry}{把戏}{7,6}{⼿、⼽}
  \begin{Phonetics}{把戏}{ba3xi4}
    \definition{s.}{acrobacia | malabarismo | truque barato}
  \end{Phonetics}
\end{Entry}

\begin{Entry}{把玩}{7,8}{⼿、⽟}
  \begin{Phonetics}{把玩}{ba3wan2}
    \definition{v.}{brincar com | mexer com | girar nas mãos}
  \end{Phonetics}
\end{Entry}

\begin{Entry}{把持}{7,9}{⼿、⼿}
  \begin{Phonetics}{把持}{ba3chi2}
    \definition{v.}{(frequentemente pejorativo) dominar; monopolizar | controlar (os próprios sentimentos, etc.) | manter sob controle}
  \end{Phonetics}
\end{Entry}

\begin{Entry}{把柄}{7,9}{⼿、⽊}
  \begin{Phonetics}{把柄}{ba3bing3}[][HSK 7-9]
    \definition[个]{s.}{alça; a parte de um objeto que é fácil de segurar com as mãos | evidências que podem ser obtidas em ações judiciais ou argumentos; uma metáfora para um erro ou falha que pode ser usada para chantagear alguém}
  \end{Phonetics}
\end{Entry}

\begin{Entry}{把脉}{7,9}{⼿、⾁}
  \begin{Phonetics}{把脉}{ba3mai4}
    \definition{v.}{resolver problemas por meio de investigação e estudo | sentir o pulso | para tomar o pulso de alguém}
  \end{Phonetics}
\end{Entry}

\begin{Entry}{把握}{7,12}{⼿、⼿}
  \begin{Phonetics}{把握}{ba3wo4}[][HSK 3]
    \definition[的]{s.}{seguro; garantia; certeza; confiabilidade do sucesso}
    \definition{v.}{agarrar; segurar; apreender |  (algo abstrato) agarrar; segurar}
  \end{Phonetics}
\end{Entry}

\begin{Entry}{把稳}{7,14}{⼿、⽲}
  \begin{Phonetics}{把稳}{ba3wen3}
    \definition{adj.}{confiável}
    \definition{v.}{ter certeza de; ser firme; manter-se firme}
  \end{Phonetics}
\end{Entry}

\begin{Entry}{抓}{7}{⼿}
  \begin{Phonetics}{抓}{zhua1}[][HSK 3]
    \definition{v.}{agarrar; segurar; obter; apreender; juntar os dedos para segurar o objeto na mão | riscar; arranhar; usar as unhas, objetos com dentes ou garras de animais para riscar a superfície de um objeto | apanhar; capturar; controlar pessoas ou animais; fazer com que pessoas ou animais caiam nas mãos de alguém | compreender; saber onde está o ponto principal ou a chave de uma questão ou problema | concentrar-se em algo; reforçar a força para fazer (alguma coisa), controlar (algum aspecto) | chamar a atenção de alguém; atrair a atenção}
  \end{Phonetics}
\end{Entry}

\begin{Entry}{抓住}{7,7}{⼿、⼈}
  \begin{Phonetics}{抓住}{zhua1 zhu4}[][HSK 3]
    \definition{v.}{prender; deter; capturar (pessoas ou animais) e ter sucesso | segurar; agarrar; apreender; agarrar algo para que não se mova}
  \end{Phonetics}
\end{Entry}

\begin{Entry}{抓紧}{7,10}{⼿、⽷}
  \begin{Phonetics}{抓紧}{zhua1jin3}[][HSK 4]
    \definition{v.}{agarrar com firmeza; segurar firme e não soltar | prestar muita atenção a}
  \end{Phonetics}
\end{Entry}

\begin{Entry}{投}{7}{⼿}
  \begin{Phonetics}{投}{tou2}[][HSK 4]
    \definition*{s.}{Sobrenome Tou}
    \definition{pron.}{para; indica tempo, equivalente a 到, 临 | para; em direção a; indica orientação, direção, equivalente a 朝 ou 向}
    \definition{s.}{um jogo durante uma festa em que o vencedor era decidido pelo número de flechas lançadas em um pote distante | jogo de dados}
    \definition{v.}{lançar; arremessar; atirar | deixar cair; colocar em; lançar | mergulhar em; lançar-se em; pular dentro | lançar; projetar; sombrear | entregar; postar; enviar | ir até; ir para; buscar; juntar-se | sentir-se atraído por; adaptar-se a; concordar com; atender a}
  \seealsoref{朝}{chao2}
  \seealsoref{到}{dao4}
  \seealsoref{临}{lin2}
  \seealsoref{向}{xiang4}
  \end{Phonetics}
\end{Entry}

\begin{Entry}{投入}{7,2}{⼿、⼊}
  \begin{Phonetics}{投入}{tou2ru4}[][HSK 4]
    \definition{adj.}{sisudo; dedicado; devotado; absorto}
    \definition{s.}{investimento; insumo; refere-se à aplicação de recursos}
    \definition{v.}{lançar em; colocar em; jogar em; por em | entrar em uma situação; participar de | aplicar; investir; colocar fundos em}
  \end{Phonetics}
\end{Entry}

\begin{Entry}{投诉}{7,7}{⼿、⾔}
  \begin{Phonetics}{投诉}{tou2su4}[][HSK 4]
    \definition{v.}{reclamar; queixar-se; reclamar às autoridades ou pessoas envolvidas}
  \end{Phonetics}
\end{Entry}

\begin{Entry}{投资}{7,10}{⼿、⾙}
  \begin{Phonetics}{投资}{tou2zi1}[][HSK 4]
    \definition[笔]{s.}{investimento}
    \definition{v.}{investir; aplicar dinheiro; investir dinheiro em negócios}
  \end{Phonetics}
\end{Entry}

\begin{Entry}{投资人}{7,10,2}{⼿、⾙、⼈}
  \begin{Phonetics}{投资人}{tou2zi1ren2}
    \definition{s.}{investidor}
  \seealsoref{投资家}{tou2zi1jia1}
  \seealsoref{投资者}{tou2zi1zhe3}
  \end{Phonetics}
\end{Entry}

\begin{Entry}{投资风险}{7,10,4,9}{⼿、⾙、⾵、⾩}
  \begin{Phonetics}{投资风险}{tou2zi1 feng1xian3}
    \definition*{s.}{risco de investimento}
  \end{Phonetics}
\end{Entry}

\begin{Entry}{投资回报率}{7,10,6,7,11}{⼿、⾙、⼞、⼿、⽞}
  \begin{Phonetics}{投资回报率}{tou2zi1 hui2bao4 lv4}
    \definition{s.}{retorno sobre o investimento (ROI)}
  \end{Phonetics}
\end{Entry}

\begin{Entry}{投资者}{7,10,8}{⼿、⾙、⽼}
  \begin{Phonetics}{投资者}{tou2zi1zhe3}
    \definition{s.}{investidor}
  \seealsoref{投资家}{tou2zi1jia1}
  \seealsoref{投资人}{tou2zi1ren2}
  \end{Phonetics}
\end{Entry}

\begin{Entry}{投资家}{7,10,10}{⼿、⾙、⼧}
  \begin{Phonetics}{投资家}{tou2zi1jia1}
    \definition{s.}{investidor}
  \seealsoref{投资人}{tou2zi1ren2}
  \seealsoref{投资者}{tou2zi1zhe3}
  \end{Phonetics}
\end{Entry}

\begin{Entry}{投递}{7,10}{⼿、⾡}
  \begin{Phonetics}{投递}{tou2di4}
    \definition{v.}{despachar | enviar}
  \end{Phonetics}
\end{Entry}

\begin{Entry}{投票}{7,11}{⼿、⽰}
  \begin{Phonetics}{投票}{tou2/piao4}[][HSK 6]
    \definition{v.+compl.}{votar; dar um voto; um método de eleição no qual os eleitores escrevem o nome da pessoa que querem eleger na cédula, ou marcam a cédula com o nome do candidato impresso e depois a colocam na urna para votar na resolução}
  \end{Phonetics}
\end{Entry}

\begin{Entry}{抖}{7}{⼿}
  \begin{Phonetics}{抖}{dou3}[][HSK 7-9]
    \definition{v.}{tremer; estremecer; arrepiar | sacudir | despertar; agitar | Coloquial geralmente sarcástico: (usualmente com 起来) progredir no mundo; ganhar fama (ou fortuna)}
  \seealsoref{起来}{qi3lai5}
  \end{Phonetics}
\end{Entry}

\begin{Entry}{抗}{7}{⼿}
  \begin{Phonetics}{抗}{kang4}[][HSK 6]
    \definition*{s.}{Sobrenome Kang}
    \definition{pref.}{anti-}
    \definition{v.}{resistir; combater; lutar | recusar; desafiar}
  \end{Phonetics}
\end{Entry}

\begin{Entry}{抗议}{7,5}{⼿、⾔}
  \begin{Phonetics}{抗议}{kang4yi4}[][HSK 6]
    \definition{v.}{protestar; reconsiderar; levantar objeções fortes}
  \end{Phonetics}
\end{Entry}

\begin{Entry}{折}{7}{⼿}
  \begin{Phonetics}{折}{she2}
    \definition*{s.}{Sobrenome Zhe}
    \definition{clas.}{um ato de zaju | um parágrafo em um drama da Dinastia Yuan, aproximadamente equivalente a uma cena ou ato em uma ópera moderna}
    \definition[张,个,些]{s.}{abatimento; desconto | os traços dos caracteres chineses têm o formato de 𠃍 e 乚 | pasta; livreto}
    \definition{v.}{estalar; quebrar; fazer quebrar | perder; sofrer a perda de | dobrar; torcer; curvar-se | voltar; mudar de direção; retornar | estar convencido; estar cheio de admiração | equivaler a; converter em}
  \end{Phonetics}
  \begin{Phonetics}{折}{zhe1}
    \definition{v.}{rolar; virar | despejar algo de um recipiente em outro; ficar despejando algo entre dois recipientes}
  \end{Phonetics}
  \begin{Phonetics}{折}{zhe2}[][HSK 4]
    \definition*{s.}{Sobrenome Zhe}
    \definition{clas.}{uma passagem em um roteiro de ópera miscelânea de Yuan, aproximadamente equivalente a uma cena ou ato em uma ópera moderna}
    \definition[张,个,些]{s.}{fratura; quebra | abatimento; desconto | traços dos caracteres chineses que têm o formato de "𠃍" e "乚", etc. | pasta; livreto; \emph{folder}}
    \definition{v.}{estalar; quebrar; fazer quebrar | perder; sofrer a perda de | voltar para trás; mudar de direção; retornar |ser convencido; estar cheio de admiração | equivaler a; converter em | dobrar}
  \end{Phonetics}
\end{Entry}

\begin{Entry}{折转}{7,8}{⼿、⾞}
  \begin{Phonetics}{折转}{zhe2zhuan3}
    \definition{s.}{reflexo (ângulo)}
    \definition{v.}{voltar atrás}
  \end{Phonetics}
\end{Entry}

\begin{Entry}{抚}{7}{⼿}
  \begin{Phonetics}{抚}{fu3}
    \definition{v.}{confortar; consolar | nutrir; fomentar | Literário: acariciar | proteger; promover; criar | o mesmo que 拊}
  \seealsoref{拊}{fu3}
  \end{Phonetics}
\end{Entry}

\begin{Entry}{抚养}{7,9}{⼿、⼋}
  \begin{Phonetics}{抚养}{fu3yang3}[][HSK 7-9]
    \definition{v.}{criar; cuidar; proporcionar às crianças as condições de vida necessárias para que possam crescer com saúde}
  \end{Phonetics}
\end{Entry}

\begin{Entry}{抚养费}{7,9,9}{⼿、⼋、⾙}
  \begin{Phonetics}{抚养费}{fu3yang3fei4}[][HSK 7-9]
    \definition{s.}{pensão alimentícia (após o divórcio) | pagamento pela educação dos filhos (como após o divórcio)}
  \end{Phonetics}
\end{Entry}

\begin{Entry}{抚恤}{7,9}{⼿、⼼}
  \begin{Phonetics}{抚恤}{fu3xu4}[][HSK 7-9]
    \definition{v.}{(estado ou organização) fornecer conforto e assistência material às famílias de pessoal ferido ou incapacitado no cumprimento do dever, ou daqueles que morreram de doença ou morreram no cumprimento do dever}
  \end{Phonetics}
\end{Entry}

\begin{Entry}{抚摸}{7,13}{⼿、⼿}
  \begin{Phonetics}{抚摸}{fu3mo1}[][HSK 7-9]
    \definition{v.}{acariciar; afagar; amimar}
  \end{Phonetics}
\end{Entry}

\begin{Entry}{抢}{7}{⼿}
  \begin{Phonetics}{抢}{qiang1}
    \definition{prep.}{contra; direção relativa inversa}
    \definition{v.}{bater; tocar}
  \end{Phonetics}
  \begin{Phonetics}{抢}{qiang3}[][HSK 5]
    \definition{v.}{roubar; saquear | agarrar; apanhar; arrebatar | disputar; lutar por; ser o primeiro; competir para ser o primeiro | correr; apressar-se; fazer uma incursão | raspar; arranhar; raspar ou esfregar uma camada da superfície de um objeto}
  \end{Phonetics}
\end{Entry}

\begin{Entry}{抢掠}{7,11}{⼿、⼿}
  \begin{Phonetics}{抢掠}{qiang3lve4}
    \definition{s.}{saque | pilhagem}
    \definition{v.}{saquear | pilhar}
  \end{Phonetics}
\end{Entry}

\begin{Entry}{抢救}{7,11}{⼿、⽁}
  \begin{Phonetics}{抢救}{qiang3jiu4}[][HSK 5]
    \definition{v.}{salvar; resgatar; prestar de socorro ou assistência rápidos em situações de emergência | salvar; tomar medidas rápidas para evitar ou minimizar perdas iminentes.}
  \end{Phonetics}
\end{Entry}

\begin{Entry}{护}{7}{⼿}
  \begin{Phonetics}{护}{hu4}[][HSK 6]
    \definition{v.}{proteger; defender | blindar; ser parcial; proteger-se da censura}
  \end{Phonetics}
\end{Entry}

\begin{Entry}{护士}{7,3}{⼿、⼠}
  \begin{Phonetics}{护士}{hu4shi5}[][HSK 4]
    \definition[名,位]{s.}{enfermeiro; pessoas especializadas em enfermagem em hospitais ou instituições epidemiológicas}
  \end{Phonetics}
\end{Entry}

\begin{Entry}{护照}{7,13}{⼿、⽕}
  \begin{Phonetics}{护照}{hu4zhao4}[][HSK 2]
    \definition[本,个]{s.}{passaporte; documento emitido pela autoridade competente do país para comprovar a nacionalidade e a identidade dos cidadãos que viajam para o exterior}
  \end{Phonetics}
\end{Entry}

\begin{Entry}{报}{7}{⼿}
  \begin{Phonetics}{报}{bao4}[][HSK 3,7-9]
    \definition[份,张]{s.}{jornal | revista; periódico; referência a uma publicação específica | relatório; boletim; algo que transmite alguma informação | telegrama | julgamento; retribuição}
    \definition{v.}{relatar; declarar; anunciar; informar; comunicar | responder; retribuir; revidar | retribuir; recompensar | vingar-se; retaliar | relatar; condenar de acordo com a lei e reportar às autoridades superiores | enviar; submeter; especificamente, relatar ao superior}
  \end{Phonetics}
\end{Entry}

\begin{Entry}{报仇}{7,4}{⼿、⼈}
  \begin{Phonetics}{报仇}{bao4/chou2}[][HSK 7-9]
    \definition{v.+compl.}{vingar; vingar-se}
  \end{Phonetics}
\end{Entry}

\begin{Entry}{报刊}{7,5}{⼿、⼑}
  \begin{Phonetics}{报刊}{bao4 kan1}[][HSK 6]
    \definition{s.}{a imprensa; jornais e periódicos}
  \end{Phonetics}
\end{Entry}

\begin{Entry}{报名}{7,6}{⼿、⼝}
  \begin{Phonetics}{报名}{bao4/ming2}[][HSK 2]
    \definition{v.+compl.}{inscrever-se; alistar-se; registrar seu nome; cadastrar-se; matricular-se; informar seu nome à pessoa responsável, órgão, grupo etc., indicando que você deseja participar de alguma atividade ou organização}
  \end{Phonetics}
\end{Entry}

\begin{Entry}{报考}{7,6}{⼿、⽼}
  \begin{Phonetics}{报考}{bao4 kao3}[][HSK 6]
    \definition{v.}{inscrever-se para um exame}
  \end{Phonetics}
\end{Entry}

\begin{Entry}{报告}{7,7}{⼿、⼝}
  \begin{Phonetics}{报告}{bao4gao4}[][HSK 3]
    \definition[份,篇]{s.}{relatório; discurso; palestra; consultivo; declaração formal feita a superiores ou ao público}
    \definition{v.}{relatar; divulgar; informar; informar formalmente sobre um assunto ou opinião aos superiores ou ao público em geral}
  \end{Phonetics}
\end{Entry}

\begin{Entry}{报社}{7,7}{⼿、⽰}
  \begin{Phonetics}{报社}{bao4she4}[][HSK 7-9]
    \definition[家,个]{s.}{escritório de jornal; escritório geral de um jornal; uma organização que edita e publica jornais}
  \end{Phonetics}
\end{Entry}

\begin{Entry}{报纸}{7,7}{⼿、⽷}
  \begin{Phonetics}{报纸}{bao4zhi3}[][HSK 2]
    \definition[分,期,张]{s.}{jornal; publicações periódicas cujo conteúdo principal é notícias, geralmente referem-se a jornais diários | papel jornal; um tipo de papel usado para imprimir jornais ou publicações em geral}
  \end{Phonetics}
\end{Entry}

\begin{Entry}{报到}{7,8}{⼿、⼑}
  \begin{Phonetics}{报到}{bao4/dao4}[][HSK 3]
    \definition{v.+compl.}{apresentar-se ao serviço; fazer o check-in; registrar-se; assinar o livro de presença; informar à organização que você já chegou}
  \end{Phonetics}
\end{Entry}

\begin{Entry}{报废}{7,8}{⼿、⼴}
  \begin{Phonetics}{报废}{bao4/fei4}[][HSK 7-9]
    \definition{v.+compl.}{sucatear; rejeitar; descartar como inútil}
  \end{Phonetics}
\end{Entry}

\begin{Entry}{报复}{7,9}{⼿、⼢}
  \begin{Phonetics}{报复}{bao4fu4}[][HSK 7-9]
    \definition{v.}{retaliar; fazer represálias; agir de forma muito cruel com alguém que criticou ou prejudicou seus interesses}
  \end{Phonetics}
\end{Entry}

\begin{Entry}{报答}{7,12}{⼿、⽵}
  \begin{Phonetics}{报答}{bao4da2}[][HSK 5]
    \definition{v.}{reembolsar; devolver; retribuir; pagar de volta; mostrar seu apreço de forma tangível}
  \end{Phonetics}
\end{Entry}

\begin{Entry}{报道}{7,12}{⼿、⾡}
  \begin{Phonetics}{报道}{bao4dao4}[][HSK 3]
    \definition[个,篇,分]{s.}{história; reportagem; comunicado de imprensa publicado por escrito ou transmitido pela rádio}
    \definition{v.}{cobrir; reportar (notícias); divulgar notícias ao público através de jornais, rádio, etc.}
  \end{Phonetics}
\end{Entry}

\begin{Entry}{报销}{7,12}{⼿、⾦}
  \begin{Phonetics}{报销}{bao4xiao1}[][HSK 7-9]
    \definition{v.}{reembolsar; enviar uma fatura; elaborar a relação dos valores recebidos ou das contas de receitas e despesas e comunicar ao superior para verificação | anular; acabar; consumir; matar alguém; fazer algo perder sua utilidade; comer toda a comida}
  \end{Phonetics}
\end{Entry}

\begin{Entry}{报酬}{7,13}{⼿、⾣}
  \begin{Phonetics}{报酬}{bao4chou5}[][HSK 7-9]
    \definition[笔,个]{s.}{pagamento; recompensa; remuneração; dinheiro ou bens pagos a outros pelo uso de seu trabalho, objetos, etc.}
  \end{Phonetics}
\end{Entry}

\begin{Entry}{报警}{7,19}{⼿、⾔}
  \begin{Phonetics}{报警}{bao4jing3}[][HSK 5]
    \definition{v.}{relatar (um incidente) à polícia; relatar uma situação crítica ou sinalizar uma emergência às autoridades competentes}
  \end{Phonetics}
\end{Entry}

\begin{Entry}{拒}{7}{⼿}
  \begin{Phonetics}{拒}{ju4}
    \definition{v.}{resistir; repelir | recusar; rejeitar}
  \end{Phonetics}
\end{Entry}

\begin{Entry}{拒绝}{7,9}{⼿、⽷}
  \begin{Phonetics}{拒绝}{ju4jue2}[][HSK 5]
    \definition{v.}{recusar; rejeitar; declinar; não aceitar (pedidos, sugestões ou presentes)}
  \end{Phonetics}
\end{Entry}

\begin{Entry}{改}{7}{⽁}
  \begin{Phonetics}{改}{gai3}[][HSK 2]
    \definition{v.}{mudar; converter; transformar; alterar; substituir | alterar; revisar; aperfeiçoar; modificar | corrigir; retificar; remediar; consertar}
  \end{Phonetics}
\end{Entry}

\begin{Entry}{改为}{7,4}{⽁、⼂}
  \begin{Phonetics}{改为}{gai3wei2}[][HSK 7-9]
    \definition{v.}{mudar para}[原计划改为明天开始。===O plano original foi mudado para começar amanhã.]
  \end{Phonetics}
\end{Entry}

\begin{Entry}{改日}{7,4}{⽁、⽇}
  \begin{Phonetics}{改日}{gai3ri4}[][HSK 7-9]
    \definition{adv.}{algum outro dia; outro dia}
  \end{Phonetics}
\end{Entry}

\begin{Entry}{改正}{7,5}{⽁、⽌}
  \begin{Phonetics}{改正}{gai3 zheng4}[][HSK 4]
    \definition{v.}{corrigir; emendar; mudar o errado para o correto}
  \end{Phonetics}
\end{Entry}

\begin{Entry}{改动}{7,6}{⽁、⼒}
  \begin{Phonetics}{改动}{gai3dong4}[][HSK 7-9]
    \definition{v.}{mudar; alterar; modificar; polir; melhorar | alterar (texto, itens, ordem, etc.)}
  \end{Phonetics}
\end{Entry}

\begin{Entry}{改名}{7,6}{⽁、⼝}
  \begin{Phonetics}{改名}{gai3ming2}[][HSK 7-9]
    \definition{v.}{mudar o próprio nome; alterar nome}
  \end{Phonetics}
\end{Entry}

\begin{Entry}{改邪归正}{7,6,5,5}{⽁、⾢、⼹、⽌}
  \begin{Phonetics}{改邪归正}{gai3xie2-gui1zheng4}[][HSK 7-9]
    \definition{expr.}{abandonar os maus caminhos e retornar ao caminho certo; abandonar o vício e voltar-se para a virtude; virar uma nova página; retornar a um modo de vida cumpridor da lei}
  \end{Phonetics}
\end{Entry}

\begin{Entry}{改良}{7,7}{⽁、⾉}
  \begin{Phonetics}{改良}{gai3liang2}[][HSK 7-9]
    \definition{v.}{melhorar; amenizar; remover as deficiências individuais das coisas para torná-las mais adequadas às necessidades | reformar; melhorar | Metalurgia: modificar}
  \end{Phonetics}
\end{Entry}

\begin{Entry}{改进}{7,7}{⽁、⾡}
  \begin{Phonetics}{改进}{gai3jin4}[][HSK 3]
    \definition[个,些]{s.}{melhoria}
    \definition{v.}{aprimorar; aperfeiçoar; melhorar; tornar melhor; mudar a situação antiga para melhorar | modificar (mudança mecânica)}
  \end{Phonetics}
\end{Entry}

\begin{Entry}{改变}{7,8}{⽁、⼜}
  \begin{Phonetics}{改变}{gai3bian4}[][HSK 2]
    \definition{v.}{mudar; alterar; transformar; converter; moldar; modificar | causar mudanças; alterar}
  \end{Phonetics}
\end{Entry}

\begin{Entry}{改版}{7,8}{⽁、⽚}
  \begin{Phonetics}{改版}{gai3/ban3}[][HSK 7-9]
    \definition{s.}{(programas de rádio ou TV) reformulação; ajuste | edição revisada}
    \definition{v.+compl.}{alterar o layout de uma folha impressa | alterar ou corrigir uma página definida | revisar a edição atual}
  \end{Phonetics}
\end{Entry}

\begin{Entry}{改革}{7,9}{⽁、⾰}
  \begin{Phonetics}{改革}{gai3ge2}[][HSK 5]
    \definition[项,次,种]{s.}{reforma; reformação; iniciativas para aprimorar a inovação}
    \definition{v.}{reformar; transformar as antigas partes irracionais das coisas em novas que possam ser adaptadas à situação objetiva}
  \end{Phonetics}
\end{Entry}

\begin{Entry}{改革开放}{7,9,4,8}{⽁、⾰、⼶、⽅}
  \begin{Phonetics}{改革开放}{gai3ge2 kai1fang4}[][HSK 7-9]
    \definition{v.}{reformar e abrir-se ao mundo exterior (refere-se às políticas de Deng Xiaoping por volta de 1980)}
  \end{Phonetics}
\end{Entry}

\begin{Entry}{改造}{7,10}{⽁、⾡}
  \begin{Phonetics}{改造}{gai3 zao4}[][HSK 3]
    \definition{v.}{transformar; renovar; modificar o original para melhor se adequar às necessidades; usado principalmente para coisas específicas | remodelar; mudar radicalmente o que é velho e ruim; criar algo novo e bom, para se adaptar às novas circunstâncias e necessidades; usado principalmente para coisas abstratas}
  \end{Phonetics}
\end{Entry}

\begin{Entry}{改善}{7,12}{⽁、⼝}
  \begin{Phonetics}{改善}{gai3shan4}[][HSK 4]
    \definition{v.}{melhorar; amenizar; mudar a situação original para torná-la melhor}
  \end{Phonetics}
\end{Entry}

\begin{Entry}{改善关系}{7,12,6,7}{⽁、⼝、⼋、⽷}
  \begin{Phonetics}{改善关系}{gai3shan4guan1xi5}
    \definition{v.}{melhorar a relação}
  \end{Phonetics}
\end{Entry}

\begin{Entry}{改善通讯}{7,12,10,5}{⽁、⼝、⾡、⾔}
  \begin{Phonetics}{改善通讯}{gai3shan4tong1xun4}
    \definition{v.}{melhorar a comunicação}
  \end{Phonetics}
\end{Entry}

\begin{Entry}{改编}{7,12}{⽁、⽷}
  \begin{Phonetics}{改编}{gai3bian1}[][HSK 7-9]
    \definition{v.}{adaptar; revisar; converter; reorganizar; transcrever; reescrever com base no trabalho original (geralmente em um gênero diferente) | reorganizar; redesignar; alterar a organização original (referindo-se principalmente ao exército)}
  \end{Phonetics}
\end{Entry}

\begin{Entry}{改装}{7,12}{⽁、⾐}
  \begin{Phonetics}{改装}{gai3 zhuang1}[][HSK 6]
    \definition{v.}{mudar de traje ou vestido | reembalar | reequipar; reaparelhar | modificar; alterar o dispositivo original}
  \end{Phonetics}
\end{Entry}

\begin{Entry}{攻}{7}{⽁}
  \begin{Phonetics}{攻}{gong1}[][HSK 7-9]
    \definition*{s.}{Sobrenome Gong}
    \definition{v.}{atacar; assaltar; tomar a ofensiva | acusar; cobrar | estudar; trabalhar em; especializar-se em}
  \end{Phonetics}
\end{Entry}

\begin{Entry}{攻击}{7,5}{⽁、⼐}
  \begin{Phonetics}{攻击}{gong1ji1}[][HSK 6]
    \definition{v.}{atacar; assaltar; lançar uma ofensiva | difamar; caluniar; acusar; atacar (verbalmente)}
  \end{Phonetics}
\end{Entry}

\begin{Entry}{攻关}{7,6}{⽁、⼋}
  \begin{Phonetics}{攻关}{gong1guan1}[][HSK 7-9]
    \definition{v.}{ter que encarar; superar esseobstáculo; começar essa jornada; resolver esse problema; abordar os principais problemas}
  \end{Phonetics}
\end{Entry}

\begin{Entry}{攻读}{7,10}{⽁、⾔}
  \begin{Phonetics}{攻读}{gong1du2}[][HSK 7-9]
    \definition{v.}{1. especializar-se em; trabalhar arduamente em uma matéria para obter um diploma ou certificado nessa matéria; estudar bastante; estudar ou aprofundar-se em um assunto}
  \end{Phonetics}
\end{Entry}

\begin{Entry}{旱}{7}{⽇}
  \begin{Phonetics}{旱}{han4}[][HSK 7-9]
    \definition{adj.}{atingido pela seca (em oposição a 涝) | seco; árido | em terra; terrestre}
    \definition{s.}{período de seca; nenhuma precipitação ou precipitação muito baixa; seca (oposto a 涝) | rota terrestre (ou comunicação) | terra firme}
  \seealsoref{涝}{lao4}
  \end{Phonetics}
\end{Entry}

\begin{Entry}{旱灾}{7,7}{⽇、⽕}
  \begin{Phonetics}{旱灾}{han4zai1}[][HSK 7-9]
    \definition[场]{s.}{seca (oposto a 水灾); desastres causados ​​por secas prolongadas e escassez de água que resultam na morte de colheitas ou redução significativa da produção}
  \seealsoref{水灾}{shui3zai1}
  \end{Phonetics}
\end{Entry}

\begin{Entry}{时}{7}{⽇}
  \begin{Phonetics}{时}{shi2}[][HSK 3]
    \definition*{s.}{Sobrenome Shi}
    \definition{adj.}{atual; presente | temporário; oportuno}
    \definition{adv.}{de vez em quando; ocasionalmente; de ​​tempos em tempos; equivalente a 常常 ou 经常 | às vezes\dots às vezes\dots; dois caracteres 时 usados juntos são equivalentes a ``有时……有时……'' e ``一会儿……一会儿……''}
    \definition{clas.}{hora, cada uma das 24 partes iguais de um dia e uma noite; também usada como unidade legal de tempo}
    \definition{s.}{dias; tempos; longo período de tempo; refere-se a um período de tempo | tempo; tempo fixo; refere-se ao tempo especificado | hora; hora do dia | temporada | chance; oportunidade; momento oportuno | atual; presente | tempo verbal; uma categoria gramatical que utiliza certas formas gramaticais para indicar o momento em que uma ação ocorre; geralmente é dividida em presente, pretérito e futuro}
  \seealsoref{常常}{chang2 chang2}
  \seealsoref{经常}{jing1chang2}
  \seealsoref{一会儿……一会儿……}{yi1hui4r5 yi1hui4r5}
  \seealsoref{有时……有时……}{you3shi2 you3shi2}
  \end{Phonetics}
\end{Entry}

\begin{Entry}{时代}{7,5}{⽇、⼈}
  \begin{Phonetics}{时代}{shi2dai4}[][HSK 3]
    \definition[个]{s.}{idade; era; tempos; época; períodos e fases históricas divididas de acordo com condições econômicas, políticas, culturais e outras | um período na vida de alguém; uma fase na vida de uma pessoa}
  \end{Phonetics}
\end{Entry}

\begin{Entry}{时节}{7,5}{⽇、⾋}
  \begin{Phonetics}{时节}{shi2 jie2}[][HSK 6]
    \definition{s.}{temporada; um período de tempo em um ano com certas características, geralmente relacionadas à estação ou ao termo solar | época; tempo}
  \end{Phonetics}
\end{Entry}

\begin{Entry}{时光}{7,6}{⽇、⼉}
  \begin{Phonetics}{时光}{shi2guang1}[][HSK 5]
    \definition[台]{s.}{tempo; passagem do tempo | dias; horas; anos; épocas; períodos}
  \end{Phonetics}
\end{Entry}

\begin{Entry}{时机}{7,6}{⽇、⽊}
  \begin{Phonetics}{时机}{shi2ji1}[][HSK 5]
    \definition[个]{s.}{oportunidade; momento oportuno}
  \end{Phonetics}
\end{Entry}

\begin{Entry}{时而}{7,6}{⽇、⽽}
  \begin{Phonetics}{时而}{shi2'er2}[][HSK 6]
    \definition{adv.}{às vezes; de tempos em tempos; indica que algo acontece repetidamente em intervalos irregulares}
  \end{Phonetics}
\end{Entry}

\begin{Entry}{时而……,时而……}{7,6,7,6}{⽇、⽽、⽇、⽽}
  \begin{Phonetics}{时而……,时而……}{shi2'er2 shi2'er2}[][HSK 6]
    \definition{adv.}{agora\dots, agora\dots; às vezes\dots, às vezes\dots; usado antes e depois; indica que diferentes fenômenos ou coisas ocorrem alternadamente ou mudam continuamente dentro de um determinado período de tempo}[\underline{时而}下雨,\underline{时而}晴天。===Às vezes chove, às vezes faz sol. | 这个地方\underline{时而}热,\underline{时而}冷。===Este lugar às vezes é quente e às vezes frio.]
  \end{Phonetics}
\end{Entry}

\begin{Entry}{时时}{7,7}{⽇、⽇}
  \begin{Phonetics}{时时}{shi2 shi2}[][HSK 6]
    \definition{adv.}{frequentemente; sempre; constantemente; indica que algo acontece várias vezes dentro de um determinado período de tempo}
  \end{Phonetics}
\end{Entry}

\begin{Entry}{时间}{7,7}{⽇、⾨}
  \begin{Phonetics}{时间}{shi2jian1}[][HSK 1]
    \definition[段]{s.}{tempo; refere-se à forma de existência do movimento da matéria, um sistema contínuo composto pelo passado, presente e futuro | tempo; período (duração); um período de tempo com início e fim | tempo (um ponto); em algum momento do tempo}
  \end{Phonetics}
\end{Entry}

\begin{Entry}{时事}{7,8}{⽇、⼅}
  \begin{Phonetics}{时事}{shi2shi4}[][HSK 5]
    \definition{s.}{acontecimentos atuais; assuntos atuais; eventos atuais | tendências atuais | como as coisas estão indo | a situação atual}
  \end{Phonetics}
\end{Entry}

\begin{Entry}{时刻}{7,8}{⽇、⼑}
  \begin{Phonetics}{时刻}{shi2ke4}[][HSK 3]
    \definition{adv.}{constantemente; sempre; a cada momento; frequentemente}
    \definition[个,段]{s.}{tempo; hora; momento; conjuntura; um ponto no tempo}
  \end{Phonetics}
\end{Entry}

\begin{Entry}{时差}{7,9}{⽇、⼯}
  \begin{Phonetics}{时差}{shi2cha1}
    \definition{s.}{diferença de tempo | \emph{jet lag}}
  \end{Phonetics}
\end{Entry}

\begin{Entry}{时候}{7,10}{⽇、⼈}
  \begin{Phonetics}{时候}{shi2hou5}[][HSK 1]
    \definition[个]{s.}{(um ponto no) tempo; momento; um determinado momento no tempo | (a duração do) tempo; um período de tempo com início e fim}
  \end{Phonetics}
\end{Entry}

\begin{Entry}{时常}{7,11}{⽇、⼱}
  \begin{Phonetics}{时常}{shi2chang2}[][HSK 5]
    \definition{adv.}{frequentemente; com frequência}
  \end{Phonetics}
\end{Entry}

\begin{Entry}{时期}{7,12}{⽇、⽉}
  \begin{Phonetics}{时期}{shi2qi1}[][HSK 6]
    \definition[个,段]{s.}{um período específico; um período de tempo com uma certa característica}
  \end{Phonetics}
\end{Entry}

\begin{Entry}{时装}{7,12}{⽇、⾐}
  \begin{Phonetics}{时装}{shi2 zhuang1}[][HSK 6]
    \definition{s.}{vestido da moda; a última moda; os últimos estilos de roupas | roupas contemporâneas (em oposição ao 古装)}
  \seealsoref{古装}{gu3 zhuang1}
  \end{Phonetics}
\end{Entry}

\begin{Entry}{旷}{7}{⽇}
  \begin{Phonetics}{旷}{kuang4}
    \definition*{s.}{Sobrenome Kuang}
    \definition{adj.}{vasto; espaçoso | livre de preocupações e ideias mesquinhas | folgado}
    \definition{v.}{negligenciar ou desperdiçar | estar ausente de | desperdiçar; abandonar; negligenciar}
  \end{Phonetics}
\end{Entry}

\begin{Entry}{旷野}{7,11}{⽇、⾥}
  \begin{Phonetics}{旷野}{kuang4ye3}
    \definition{s.}{região selvagem}
  \end{Phonetics}
\end{Entry}

\begin{Entry}{更}{7}{⽈}
  \begin{Phonetics}{更}{geng1}
    \definition*{s.}{Sobrenome Geng}
    \definition{clas.}{um dos cinco períodos de duas horas em que a noite era anteriormente dividida; vigília; antigamente, a noite era dividida em cinco turnos, cada um com aproximadamente duas horas de duração}
    \definition{v.}{alterar; substituir | experimentar}
  \end{Phonetics}
  \begin{Phonetics}{更}{geng4}[][HSK 2]
    \definition{adv.}{mais; ainda mais | além disso; além do mais; ainda mais}
  \end{Phonetics}
\end{Entry}

\begin{Entry}{更加}{7,5}{⽈、⼒}
  \begin{Phonetics}{更加}{geng4 jia1}[][HSK 3]
    \definition{adv.}{mais; ainda mais; em maior grau; indica um nível mais profundo ou um aumento ou diminuição quantitativa adicional}
  \end{Phonetics}
\end{Entry}

\begin{Entry}{更衣室}{7,6,9}{⽈、⾐、⼧}
  \begin{Phonetics}{更衣室}{geng1yi1shi4}[][HSK 7-9]
    \definition{s.}{vestiário | camarim | \emph{toilet}; toucador}
  \end{Phonetics}
\end{Entry}

\begin{Entry}{更改}{7,7}{⽈、⽁}
  \begin{Phonetics}{更改}{geng1gai3}[][HSK 7-9]
    \definition{v.}{alterar; mudar}
  \end{Phonetics}
\end{Entry}

\begin{Entry}{更是}{7,9}{⽈、⽇}
  \begin{Phonetics}{更是}{geng4 shi4}[][HSK 6]
    \definition{adv.}{ainda mais (assim)}
  \end{Phonetics}
\end{Entry}

\begin{Entry}{更换}{7,10}{⽈、⼿}
  \begin{Phonetics}{更换}{geng1 huan4}[][HSK 5]
    \definition{v.}{alterar; mudar; substituir; comutar}
  \end{Phonetics}
\end{Entry}

\begin{Entry}{更新}{7,13}{⽈、⽄}
  \begin{Phonetics}{更新}{geng1xin1}[][HSK 5]
    \definition{v.}{renovar; atualizar; substituir; remover o antigo e substituir pelo novo}
  \end{Phonetics}
\end{Entry}

\begin{Entry}{杆}{7}{⽊}
  \begin{Phonetics}{杆}{gan1}[][HSK 6]
    \definition{s.}{poste; pólo; mastro}
  \end{Phonetics}
  \begin{Phonetics}{杆}{gan3}
    \definition{clas.}{usado para objetos semelhantes a hastes}
    \definition{s.}{eixo; braço | haste; barra; poste; a parte longa e fina de um objeto, semelhante a um bastão}
  \end{Phonetics}
\end{Entry}

\begin{Entry}{李}{7}{⽊}
  \begin{Phonetics}{李}{li3}
    \definition*{s.}{Sobrenome Li}
    \definition[棵]{s.}{ameixa | ameixeira}
  \end{Phonetics}
\end{Entry}

\begin{Entry}{李子}{7,3}{⽊、⼦}
  \begin{Phonetics}{李子}{li3zi5}
    \definition[个]{s.}{ameixa}
  \end{Phonetics}
\end{Entry}

\begin{Entry}{李四}{7,5}{⽊、⼞}
  \begin{Phonetics}{李四}{li3si4}
    \definition*{s.}{Li Si | Zé Ninguém | Nome para uma pessoa não especificada, 2 de 3}
  \seealsoref{王五}{wang2wu3}
  \seealsoref{张三}{zhang1san1}
  \end{Phonetics}
\end{Entry}

\begin{Entry}{材}{7}{⽊}
  \begin{Phonetics}{材}{cai2}
    \definition[份]{s.}{madeira | material; geralmente se refere a coisas que podem ser transformadas diretamente em produtos acabados | material; materiais para escrita ou referência | pessoa capaz; pessoas talentosas | habilidade; talento; aptidão | caixão}
  \end{Phonetics}
\end{Entry}

\begin{Entry}{材料}{7,10}{⽊、⽃}
  \begin{Phonetics}{材料}{cai2liao4}[][HSK 4]
    \definition[份,个,种]{s.}{material; algo para fazer um produto acabado | material (figura de linguagem) | dados; material para estudo, pesquisa, etc.; conteúdo de uma obra}
  \end{Phonetics}
\end{Entry}

\begin{Entry}{村}{7}{⽊}
  \begin{Phonetics}{村}{cun1}[][HSK 3]
    \definition{adj.}{rústico; grosseiro}
    \definition[个,座]{s.}{aldeia; vila | área povoada de certo tipo}
  \end{Phonetics}
\end{Entry}

\begin{Entry}{村儿}{7,2}{⽊、⼉}
  \begin{Phonetics}{村儿}{cun1r5}
    \definition{s.}{vila; aldeia}
  \end{Phonetics}
\end{Entry}

\begin{Entry}{村庄}{7,6}{⽊、⼴}
  \begin{Phonetics}{村庄}{cun1 zhuang1}[][HSK 6]
    \definition[个,座,片]{s.}{aldeia; vila; onde vivem os agricultores}
  \end{Phonetics}
\end{Entry}

\begin{Entry}{杜}{7}{⽊}
  \begin{Phonetics}{杜}{du4}
    \definition*{s.}{Sobrenome Du}
    \definition{s.}{pêra de folha de bétula}
    \definition{v.}{excluir; parar; impedir; bloquear}
  \end{Phonetics}
\end{Entry}

\begin{Entry}{杜宇}{7,6}{⽊、⼧}
  \begin{Phonetics}{杜宇}{du4yu3}
    \definition{s.}{cuco (pássaro)}
  \seealsoref{布谷鸟}{bu4gu3niao3}
  \seealsoref{杜鹃}{du4juan1}
  \seealsoref{杜鹃鸟}{du4juan1niao3}
  \end{Phonetics}
\end{Entry}

\begin{Entry}{杜绝}{7,9}{⽊、⽷}
  \begin{Phonetics}{杜绝}{du4jue2}[][HSK 7-9]
    \definition{v.}{parar; pôr fim a; bloquear a fonte e parar (coisas ruins)}
  \end{Phonetics}
\end{Entry}

\begin{Entry}{杜鹃}{7,12}{⽊、⿃}
  \begin{Phonetics}{杜鹃}{du4juan1}
    \definition{s.}{cuco (pássaro)}
  \seealsoref{布谷鸟}{bu4gu3niao3}
  \seealsoref{杜鹃鸟}{du4juan1niao3}
  \seealsoref{杜宇}{du4yu3}
  \end{Phonetics}
\end{Entry}

\begin{Entry}{杜鹃鸟}{7,12,5}{⽊、⿃、⿃}
  \begin{Phonetics}{杜鹃鸟}{du4juan1niao3}
    \definition{s.}{cuco (pássaro)}
  \seealsoref{布谷鸟}{bu4gu3niao3}
  \seealsoref{杜鹃}{du4juan1}
  \seealsoref{杜宇}{du4yu3}
  \end{Phonetics}
\end{Entry}

\begin{Entry}{束}{7}{⽊}
  \begin{Phonetics}{束}{shu4}[][HSK 3]
    \definition*{s.}{Sobrenome Shu}
    \definition{clas.}{usado para cachos, molhos, feixes, feixes de luz, etc.}
    \definition{s.}{monte; pacote; maço; feixe; cacho; coisas agrupadas ou reunidas em tiras}
    \definition{v.}{atar; amarrar; vincular | controlar; restringir}
  \end{Phonetics}
\end{Entry}

\begin{Entry}{束腰}{7,13}{⽊、⾁}
  \begin{Phonetics}{束腰}{shu4yao1}
    \definition{s.}{cinto | cinta | cinturão}
  \end{Phonetics}
\end{Entry}

\begin{Entry}{杠}{7}{⽊}
  \begin{Phonetics}{杠}{gang1}
    \definition{s.}{pequena ponte | mastro de bandeira}
  \end{Phonetics}
  \begin{Phonetics}{杠}{gang4}
    \definition{s.}{vara grossa | (esportes) barra | peça sobressalente em forma de haste; peça sobressalente em forma de haste usada para máquinas-ferramentas | varas robustas usadas para carregar um caixão | (em um texto) linha grossa desenhada ao lado ou abaixo das palavras como uma marca | (coloquial) padrão; critério}
    \definition{v.}{marcar com uma linha grossa | afiar (faca, navalha, etc.)}
  \end{Phonetics}
\end{Entry}

\begin{Entry}{杠铃}{7,10}{⽊、⾦}
  \begin{Phonetics}{杠铃}{gang4ling2}[][HSK 7-9]
    \definition{s.}{barra; levantamento de peso; equipamento de levantamento de peso, com placas metálicas em forma de disco instaladas em ambas as extremidades da barra horizontal}
  \end{Phonetics}
\end{Entry}

\begin{Entry}{条}{7}{⽊}
  \begin{Phonetics}{条}{tiao2}[][HSK 2]
    \definition*{s.}{Sobrenome Tiao}
    \definition{clas.}{usado para objetos longos e finos; usado para sintetizar certas coisas longas e retangulares em quantidades fixas | usado para itemização | aplicado ao corpo humano}
    \definition{s.}{galho; galhos finos e longos | tira; faixa | item; artigo | ordem; método | nota; anotação em papel}
  \end{Phonetics}
\end{Entry}

\begin{Entry}{条目}{7,5}{⽊、⽬}
  \begin{Phonetics}{条目}{tiao2mu4}
    \definition{s.}{cláusulas e subcláusulas (em documento formal) | verbete (em um dicionário, enciclopédia, etc.)}
  \end{Phonetics}
\end{Entry}

\begin{Entry}{条件}{7,6}{⽊、⼈}
  \begin{Phonetics}{条件}{tiao2jian4}[][HSK 2]
    \definition[个,项,些]{s.}{condição; termo; fator; fatores que restringem a ocorrência, existência ou desenvolvimento das coisas | requisito; pré-requisito; qualificação; requisitos ou padrões estabelecidos para determinadas coisas | situação; estado; condição}
  \end{Phonetics}
\end{Entry}

\begin{Entry}{条例}{7,8}{⽊、⼈}
  \begin{Phonetics}{条例}{tiao2li4}
    \definition{s.}{código de conduta | ordenanças | regulamentos | regras | estatutos}
  \end{Phonetics}
\end{Entry}

\begin{Entry}{条贯}{7,8}{⽊、⾙}
  \begin{Phonetics}{条贯}{tiao2guan4}
    \definition{s.}{ordem | procedimentos | sequência | sistema}
  \end{Phonetics}
\end{Entry}

\begin{Entry}{条幅}{7,12}{⽊、⼱}
  \begin{Phonetics}{条幅}{tiao2fu2}
    \definition{s.}{faixa | banner | pergaminho de parede (para pintura ou caligrafia)}
  \end{Phonetics}
\end{Entry}

\begin{Entry}{来}{7}{⽊}
  \begin{Phonetics}{来}{lai2}[][HSK 1]
    \definition*{s.}{Sobrenome Lai}
    \definition{part.}{usado após uma palavra numérica ou de quantidade; indica uma quantidade aproximada | usado depois de numerais como 一, 二, 三; para listar razões ou fatos, etc.}
    \definition{s.}{usado após uma expressão de tempo para indicar uma duração que vai do passado ao presente}
    \definition{v.}{vir; chegar; de outro lugar para o lugar onde o interlocutor se encontra | aparecer; acontecer; vir; (problemas, coisas, etc.) ocorrerem; surgirem | substitui um verbo com significado específico, indicando a realização de uma ação específica | estar indo para; usado antes de outro verbo, indica que algo será feito | vir para fazer algo; usado após outro verbo, indica que se vai fazer algo | usado para indicar um propósito; expressar o objetivo, fazer algo usando o método, a atitude ou a direção anteriores | usado com 得 ou 不 para indicar possibilidade, capacidade ou hábito}
  \seealsoref{不}{bu4}
  \seealsoref{得}{de5}
  \end{Phonetics}
\end{Entry}

\begin{Entry}{来不及}{7,4,3}{⽊、⼀、⼃}
  \begin{Phonetics}{来不及}{lai2bu5ji2}[][HSK 4]
    \definition{v.}{ser tarde demais; não ter tempo; não ter tempo suficiente (para fazer algo); não ser possível participar ou se atualizar devido a restrições de tempo}
  \end{Phonetics}
\end{Entry}

\begin{Entry}{来自}{7,6}{⽊、⾃}
  \begin{Phonetics}{来自}{lai2zi4}[][HSK 2]
    \definition{v.}{vir de (um local) | \emph{From:} (cabeçalho de \emph{e -mail})}
  \end{Phonetics}
\end{Entry}

\begin{Entry}{来到}{7,8}{⽊、⼑}
  \begin{Phonetics}{来到}{lai2 dao4}[][HSK 1]
    \definition{v.}{chegar; vir}
  \end{Phonetics}
\end{Entry}

\begin{Entry}{来往}{7,8}{⽊、⼻}
  \begin{Phonetics}{来往}{lai2 wang3}[][HSK 6]
    \definition{s.}{negociação; contato com alguém; interações sociais}
    \definition{v.}{ir e vir | ter negócios com alguém}
  \end{Phonetics}
\end{Entry}

\begin{Entry}{来信}{7,9}{⽊、⼈}
  \begin{Phonetics}{来信}{lai2 xin4}[][HSK 5]
    \definition[封]{s.}{sua carta; carta recebida; carta ao interlocutor}
    \definition{v.}{enviar uma carta para aqui; enviar uma carta para o remetente}
  \end{Phonetics}
\end{Entry}

\begin{Entry}{来得及}{7,11,3}{⽊、⼻、⼃}
  \begin{Phonetics}{来得及}{lai2de5ji2}[][HSK 4]
    \definition{v.}{ainda ter tempo; ser capaz de fazer isso; ser capaz de fazer algo a tempo; ainda ter tempo de cuidar ou de colocar em dia}
  \end{Phonetics}
\end{Entry}

\begin{Entry}{来源}{7,13}{⽊、⽔}
  \begin{Phonetics}{来源}{lai2yuan2}[][HSK 4]
    \definition{s.}{origem; causa; fonte; tabula rasa (ou seja, o lugar de onde as coisas vêm)}
    \definition{v.}{originar-se; surgir; ter origem; (algo) originar (seguido de 于)}
  \seealsoref{于}{yu2}
  \end{Phonetics}
\end{Entry}

\begin{Entry}{极}{7}{⽊}
  \begin{Phonetics}{极}{ji2}[][HSK 4]
    \definition*{s.}{Sobrenome Ji}
    \definition{adj.}{máximo; extremo; final; supremo}
    \definition{adv.}{extremamente; excessivamente}
    \definition{s.}{o ponto máximo, mais alto; extremo; ápice; ponto culminante | pólo; as extremidades norte e sul da Terra; as extremidades de um ímã; a extremidade de uma fonte de alimentação ou de um aparelho elétrico onde a corrente entra ou sai do aparelho}
    \definition{v.}{chegar ao fim de; levar a extremos | Literário: fazer o máximo possível}
  \end{Phonetics}
\end{Entry}

\begin{Entry}{……极了}{7,2}{⽊、⼅}
  \begin{Phonetics}{……极了}{ji2le5}[][HSK 3]
    \definition{expr.}{extremamente; alto grau de expressão}
  \end{Phonetics}
\end{Entry}

\begin{Entry}{极其}{7,8}{⽊、⼋}
  \begin{Phonetics}{极其}{ji2qi2}[][HSK 4]
    \definition{adv.}{mais; extremamente; excessivamente}
  \end{Phonetics}
\end{Entry}

\begin{Entry}{极端}{7,14}{⽊、⽴}
  \begin{Phonetics}{极端}{ji2duan1}[][HSK 6]
    \definition{adj.}{extremo; absoluto; sem quaisquer restrições}
    \definition{adv.}{excessivamente; extremamente; alto grau de expressão}
    \definition{s.}{extremo; extremidade; o auge do desenvolvimento}
  \end{Phonetics}
\end{Entry}

\begin{Entry}{步}{7}{⽌}
  \begin{Phonetics}{步}{bu4}[][HSK 3]
    \definition*{s.}{Geralmente em nomes de lugares | Sobrenome Bu}[盐步===Yanbu, na província de Guangdong]
    \definition{clas.}{uma unidade antiga para medida de comprimento, equivalente a cinco 尺}
    \definition{s.}{passo; ritmo | etapa; passo | condição; situação; estado | cais; píer | porto; cidade portuária | (geralmente em nomes de lugares)}
    \definition{v.}{caminhar; ir a pé | seguir os passos de alguém | (dialeto) medir com passos | seguir; acompanhar | medir a distância com os passos}
  \seealsoref{尺}{chi3}
  \end{Phonetics}
\end{Entry}

\begin{Entry}{步入}{7,2}{⽌、⼊}
  \begin{Phonetics}{步入}{bu4ru4}[][HSK 7-9]
    \definition{v.}{entrar (em)}[所有人梳妆打扮以步入上流社会。===Todo mundo se veste bem para se encaixar na alta sociedade.]
  \end{Phonetics}
\end{Entry}

\begin{Entry}{步伐}{7,6}{⽌、⼈}
  \begin{Phonetics}{步伐}{bu4fa2}[][HSK 7-9]
    \definition{s.}{passo; ritmo; os passos de uma pessoa ao caminhar; o ritmo | passo; ritmo; metáfora para a velocidade com que as coisas acontecem}
  \end{Phonetics}
\end{Entry}

\begin{Entry}{步行}{7,6}{⽌、⾏}
  \begin{Phonetics}{步行}{bu4 xing2}[][HSK 4]
    \definition{v.}{caminhar; ir a pé; andar a pé (diferente de andar de carro, a cavalo, etc.)}
  \end{Phonetics}
\end{Entry}

\begin{Entry}{步骤}{7,17}{⽌、⾺}
  \begin{Phonetics}{步骤}{bu4zhou4}[][HSK 7-9]
    \definition[个]{s.}{passo; movimento; procedimento; medida; o procedimento para que as coisas aconteçam}
  \end{Phonetics}
\end{Entry}

\begin{Entry}{每}{7}{⽏}
  \begin{Phonetics}{每}{mei3}[][HSK 3]
    \definition{adv.}{cada um; cada qual; indica qualquer uma das repetições ou um conjunto de repetições de um movimento}
    \definition{pron.}{cada; cada um; cada qual; refere-se a qualquer indivíduo do grupo, enfatizando as semelhanças entre os indivíduos}
  \end{Phonetics}
\end{Entry}

\begin{Entry}{每个}{7,3}{⽏、⼈}
  \begin{Phonetics}{每个}{mei3ge4}
    \definition{pron.}{cada; cada um}
  \end{Phonetics}
\end{Entry}

\begin{Entry}{每个人}{7,3,2}{⽏、⼈、⼈}
  \begin{Phonetics}{每个人}{mei3ge5ren2}
    \definition{pron.}{todo mundo | todos}
  \end{Phonetics}
\end{Entry}

\begin{Entry}{每天}{7,4}{⽏、⼤}
  \begin{Phonetics}{每天}{mei3tian1}
    \definition{adv.}{todo dia | cada dia}
  \end{Phonetics}
\end{Entry}

\begin{Entry}{每次}{7,6}{⽏、⽋}
  \begin{Phonetics}{每次}{mei3ci4}
    \definition{adv.}{toda vez | cada vez}
  \end{Phonetics}
\end{Entry}

\begin{Entry}{求}{7}{⽔}
  \begin{Phonetics}{求}{qiu2}[][HSK 2]
    \definition*{s.}{Sobrenome Qiu}
    \definition{v.}{implorar; solicitar; suplicar; rogar | lutar por; buscar; investigar | tentar; procurar; tentar obter | demandar}
  \end{Phonetics}
\end{Entry}

\begin{Entry}{求职}{7,11}{⽔、⽿}
  \begin{Phonetics}{求职}{qiu2 zhi2}[][HSK 6]
    \definition{v.}{procurar emprego; candidatar-se a um emprego; encontrar um emprego}
  \end{Phonetics}
\end{Entry}

\begin{Entry}{汹}{7}{⽔}
  \begin{Phonetics}{汹}{xiong1}
    \definition{adj.}{turbulento; tempestuoso | rugindo; estrondoso | tumultuado}
  \end{Phonetics}
\end{Entry}

\begin{Entry}{汹涌}{7,10}{⽔、⽔}
  \begin{Phonetics}{汹涌}{xiong1yong3}
    \definition{adj.}{turbulento}
    \definition{v.}{aumentar ou emergir violentamente (oceano, rio, lago, etc.)}
  \end{Phonetics}
\end{Entry}

\begin{Entry}{汽}{7}{⽔}
  \begin{Phonetics}{汽}{qi4}
    \definition{s.}{vapor | vaporizador}
  \end{Phonetics}
\end{Entry}

\begin{Entry}{汽水}{7,4}{⽔、⽔}
  \begin{Phonetics}{汽水}{qi4 shui3}[][HSK 4]
    \definition[罐,杯,瓶,听,口]{s.}{refrigerante; refrigerante gaseificado; bebida refrescante, feita com a pressão de dióxido de carbono para dissolver na água e adicionar açúcar, suco de frutas, especiarias etc.}
  \end{Phonetics}
\end{Entry}

\begin{Entry}{汽车}{7,4}{⽔、⾞}
  \begin{Phonetics}{汽车}{qi4 che1}[][HSK 1]
    \definition[辆,种,款]{s.}{automóvel; carro; veículo motorizado; veículo movido a motor de combustão interna, que circula principalmente em rodovias ou ruas, geralmente com quatro ou mais pneus de borracha, usado para transportar pessoas ou mercadorias}
  \end{Phonetics}
\end{Entry}

\begin{Entry}{汽油}{7,8}{⽔、⽔}
  \begin{Phonetics}{汽油}{qi4you2}[][HSK 4]
    \definition[桶,升,吨]{s.}{gasolina; mistura líquida de hidrocarbonetos com volatilidade e combustibilidade, que é usada como combustível a partir do fracionamento ou craqueamento do petróleo}
  \end{Phonetics}
\end{Entry}

\begin{Entry}{沉}{7}{⽔}
  \begin{Phonetics}{沉}{chen2}[][HSK 4]
    \definition{adj.}{profundo | pesado | pesado (sentir-se pesado)}
    \definition{v.}{afundar; submergir; imergir | manter baixo; abaixar | descansar; parar}
  \end{Phonetics}
\end{Entry}

\begin{Entry}{沉甸甸}{7,7,7}{⽔、⽥、⽥}
  \begin{Phonetics}{沉甸甸}{chen2dian4dian4}[][HSK 7-9]
    \definition{adj.}{pesado; pesado e difícil de manejar}
  \end{Phonetics}
\end{Entry}

\begin{Entry}{沉闷}{7,7}{⽔、⾨}
  \begin{Phonetics}{沉闷}{chen2men4}[][HSK 7-9]
    \definition{adj.}{triste; opressivo; deprimente; o clima e a atmosfera fazem as pessoas se sentirem pesadas e deprimidas | deprimido; desanimado; (humor) baixo; (caráter) nada alegre | (som e voz) baixo}
  \end{Phonetics}
\end{Entry}

\begin{Entry}{沉思}{7,9}{⽔、⼼}
  \begin{Phonetics}{沉思}{chen2si1}[][HSK 7-9]
    \definition{v.}{ponderar; meditar; contemplar; perder-se em pensamentos}
  \end{Phonetics}
\end{Entry}

\begin{Entry}{沉迷}{7,9}{⽔、⾡}
  \begin{Phonetics}{沉迷}{chen2mi2}[][HSK 7-9]
    \definition{v.}{entregar-se; chafurdar; remoer}[不要沉迷在回忆中。===Não fique remoendo memórias.]
  \end{Phonetics}
\end{Entry}

\begin{Entry}{沉重}{7,9}{⽔、⾥}
  \begin{Phonetics}{沉重}{chen2zhong4}[][HSK 4]
    \definition{adj.}{(pressão, fardo, etc.) muito pesado; profundo | sério; pesado; humor pouco animador; fardo pesado de pensamentos}
  \end{Phonetics}
\end{Entry}

\begin{Entry}{沉浸}{7,10}{⽔、⽔}
  \begin{Phonetics}{沉浸}{chen2jin4}[][HSK 7-9]
    \definition{v.}{estar imerso em; estar submerso em; estar permeado com; invasão de água, frequentemente usada como uma metáfora para estar em um determinado estado ou atividade de pensamento}
  \end{Phonetics}
\end{Entry}

\begin{Entry}{沉淀}{7,11}{⽔、⽔}
  \begin{Phonetics}{沉淀}{chen2dian4}[][HSK 7-9]
    \definition{s.}{acúmulo; precipitado; sedimento; matéria sólida que afunda no fundo de líquidos como água ou óleo}
    \definition{v.}{assentar; baixar | acumular; reunir}
  \end{Phonetics}
\end{Entry}

\begin{Entry}{沉着}{7,11}{⽔、⽬}
  \begin{Phonetics}{沉着}{chen2zhuo2}[][HSK 7-9]
    \definition{adj.}{calmo; estável; composto; cabeça fria | (diante de problemas) calmo; sem pressa}
  \end{Phonetics}
\end{Entry}

\begin{Entry}{沉稳}{7,14}{⽔、⽲}
  \begin{Phonetics}{沉稳}{chen2wen3}[][HSK 7-9]
    \definition{adj.}{estável; sóbrio; calmo; sereno | calmo; imperturbável}
  \end{Phonetics}
\end{Entry}

\begin{Entry}{沉默}{7,16}{⽔、⿊}
  \begin{Phonetics}{沉默}{chen2mo4}[][HSK 4]
    \definition{adj.}{silencioso; reticente; taciturno; não comunicativo}
    \definition{v.}{silenciar; não falar por causa de alguma coisa}
  \end{Phonetics}
\end{Entry}

\begin{Entry}{沙}{7}{⽔}
  \begin{Phonetics}{沙}{sha1}
    \definition*{s.}{Sobrenome Sha}
    \definition{adj.}{granulado; em pó | rouco}[我今天感冒了,嗓音有点沙哑。===Estou resfriado hoje e minha voz está um pouco rouca.]
    \definition[车,把,袋,吨]{s.}{areia; cascalho; grânulo; pó}
  \end{Phonetics}
\end{Entry}

\begin{Entry}{沙子}{7,3}{⽔、⼦}
  \begin{Phonetics}{沙子}{sha1 zi5}[][HSK 3]
    \definition[粒,把,堆,袋,车]{s.}{areia; grão; pequenas pedras | \emph{pellets}; grãos pequenos; coisas parecidas com areia}
  \end{Phonetics}
\end{Entry}

\begin{Entry}{沙发}{7,5}{⽔、⼜}
  \begin{Phonetics}{沙发}{sha1fa1}[][HSK 3]
    \definition[套,组,个,张]{s.}{sofá; assentos com molas ou espuma plástica espessa, etc., com apoios de braços em ambos os lados}
  \end{Phonetics}
\end{Entry}

\begin{Entry}{沙鱼}{7,8}{⽔、⿂}
  \begin{Phonetics}{沙鱼}{sha1yu2}
    \variantof{鲨鱼}
  \end{Phonetics}
\end{Entry}

\begin{Entry}{沙特}{7,10}{⽔、⽜}
  \begin{Phonetics}{沙特}{sha1te4}
    \definition*{s.}{Saudita | Arábia Saudita, abreviação de 沙特阿拉伯}
  \seealsoref{沙特阿拉伯}{sha1te4 a1la1bo2}
  \end{Phonetics}
\end{Entry}

\begin{Entry}{沙特阿拉伯}{7,10,7,8,7}{⽔、⽜、⾩、⼿、⼈}
  \begin{Phonetics}{沙特阿拉伯}{sha1te4 a1la1bo2}
    \definition*{s.}{Arábia Saudita}
  \end{Phonetics}
\end{Entry}

\begin{Entry}{沙漠}{7,13}{⽔、⽔}
  \begin{Phonetics}{沙漠}{sha1mo4}[][HSK 5]
    \definition[个,片]{s.}{deserto; superfície totalmente coberta por areia, sem água corrente, clima seco e vegetação escassa}
  \end{Phonetics}
\end{Entry}

\begin{Entry}{沟}{7}{⽔}
  \begin{Phonetics}{沟}{gou1}[][HSK 5]
    \definition[条,道,段]{s.}{canal; vala; sarjeta; trincheira; cursos d'água ou fortificações escavados | ranhura; sulco raso; uma depressão que se assemelha a uma vala | ravina; barranco; cursos d'água}
  \end{Phonetics}
\end{Entry}

\begin{Entry}{沟通}{7,10}{⽔、⾡}
  \begin{Phonetics}{沟通}{gou1tong1}[][HSK 5]
    \definition{v.}{comunicar; comunicar-se para entender as ideias, opiniões, etc. | conectar; ligar; estabelecer um paralelo entre os dois}
  \end{Phonetics}
\end{Entry}

\begin{Entry}{没}{7}{⽔}
  \begin{Phonetics}{没}{mei2}[][HSK 1]
    \definition{adv.}{não; nunca; negar que uma ação ou situação tenha ocorrido, com o significado de 不曾}
    \definition{pref.}{não (prefixo negativo para verbos, traduzido para outras línguas com verbos no pretérito)}
    \definition{v.}{não possuir; não ter | não existe; não há | ninguém; usado antes de 谁, 什么, 哪个, significa 全都不 | não ser tão bom quanto; ser inferior a; não chega a; não é tão bom quanto | menor que; insuficiente}
  \seealsoref{不曾}{bu4 ceng2}
  \seealsoref{哪个}{na3ge5}
  \seealsoref{全都不}{quan2dou1 bu4}
  \seealsoref{谁}{shei2}
  \seealsoref{什么}{shen2me5}
  \end{Phonetics}
  \begin{Phonetics}{没}{mo4}
    \definition{adj.}{último; final}
    \definition{v.}{afundar na água; submergir | transbordar; subir além; exceder ou ultrapassar | esconder-se; desaparecer; sumir; ocultar-se | confiscar; expropriar | morrer}
    \variantof{没}
  \end{Phonetics}
\end{Entry}

\begin{Entry}{没了}{7,2}{⽔、⼅}
  \begin{Phonetics}{没了}{mei2le5}
    \definition{v.}{estar morto | deixar de existir}
  \end{Phonetics}
\end{Entry}

\begin{Entry}{没什么}{7,4,3}{⽔、⼈、⼃}
  \begin{Phonetics}{没什么}{mei2 shen2 me5}[][HSK 1]
    \definition{expr.}{não é nada; está tudo bem; não importa}
  \end{Phonetics}
\end{Entry}

\begin{Entry}{没用}{7,5}{⽔、⽤}
  \begin{Phonetics}{没用}{mei2 yong4}[][HSK 3]
    \definition{adj.}{inútil; imprestável; sem valor; sem préstimo; vão; que não serve para nada}
  \end{Phonetics}
\end{Entry}

\begin{Entry}{没关系}{7,6,7}{⽔、⼋、⽷}
  \begin{Phonetics}{没关系}{mei2guan1xi5}[][HSK 1]
    \definition{v.}{está tudo bem; não é nada; não importa; não se preocupe}
  \seealsoref{没有关系}{mei2you3guan1xi5}
  \end{Phonetics}
\end{Entry}

\begin{Entry}{没收}{7,6}{⽔、⽁}
  \begin{Phonetics}{没收}{mo4 shou1}[][HSK 6]
    \definition{v.}{confiscar; expropriar; os bens e pertences de pessoas ou grupos que violem leis ou proibições serão tornados propriedade pública, de acordo com a lei}
  \end{Phonetics}
\end{Entry}

\begin{Entry}{没有}{7,6}{⽔、⽉}
  \begin{Phonetics}{没有}{mei2 you3}[][HSK 1]
    \definition{adv.}{ainda não; (usado com o pretérito) não; ação ou estado negativo ocorreu}
    \definition{v.}{não há; não tem; não existe}
  \end{Phonetics}
\end{Entry}

\begin{Entry}{没有关系}{7,6,6,7}{⽔、⽉、⼋、⽷}
  \begin{Phonetics}{没有关系}{mei2you3guan1xi5}
    \definition{expr.}{Está tudo bem; sem problemas}
  \seealsoref{没关系}{mei2guan1xi5}
  \end{Phonetics}
\end{Entry}

\begin{Entry}{没有次序}{7,6,6,7}{⽔、⽉、⽋、⼴}
  \begin{Phonetics}{没有次序}{mei2you3 ci4xu4}
    \definition{adj.}{sem ordem; nenhuma ordem}
  \end{Phonetics}
\end{Entry}

\begin{Entry}{没有哪一种东西}{7,6,9,1,9,5,6}{⽔、⽉、⼝、⼀、⽲、⼀、⾑}
  \begin{Phonetics}{没有哪一种东西}{mei2you3 na3 yi4 zhong3 dong1xi1}
    \definition{pron.}{nada; não existe tal coisa}
  \end{Phonetics}
\end{Entry}

\begin{Entry}{没有谁}{7,6,10}{⽔、⽉、⾔}
  \begin{Phonetics}{没有谁}{mei2you3 shei2}
    \definition{pron.}{ninguém}
  \end{Phonetics}
\end{Entry}

\begin{Entry}{没有意思}{7,6,13,9}{⽔、⽉、⼼、⼼}
  \begin{Phonetics}{没有意思}{mei2you3yi4si5}
    \definition{adj.}{tedioso | chato | sem interesse}
  \end{Phonetics}
\end{Entry}

\begin{Entry}{没事儿}{7,8,2}{⽔、⼅、⼉}
  \begin{Phonetics}{没事儿}{mei2 shi4r5}[][HSK 1]
    \definition{expr.}{fora de perigo; nada sério | não importa; não é nada; está tudo bem; não importa | está tudo bem; sem problemas; não se preocupe com isso; não é grande coisa; não há nada errado}
    \definition{v.}{não ter nada para fazer; ser livre; estar perdido | estar desempregado; estar sem trabalho | não ter responsabilidade}
  \end{Phonetics}
\end{Entry}

\begin{Entry}{没法儿}{7,8,2}{⽔、⽔、⼉}
  \begin{Phonetics}{没法儿}{mei2 fa3r5}[][HSK 4]
    \definition{adv.}{não pode; sem chance}
  \end{Phonetics}
\end{Entry}

\begin{Entry}{没想到}{7,13,8}{⽔、⼼、⼑}
  \begin{Phonetics}{没想到}{mei2 xiang3 dao4}[][HSK 4]
    \definition{expr.}{não esperava; inesperado}
  \end{Phonetics}
\end{Entry}

\begin{Entry}{没错}{7,13}{⽔、⾦}
  \begin{Phonetics}{没错}{mei2 cuo4}[][HSK 4]
    \definition{adv.}{está certo; é isso mesmo; não há como errar}
  \end{Phonetics}
\end{Entry}

\begin{Entry}{沧}{7}{⽔}
  \begin{Phonetics}{沧}{cang1}
    \definition{adj.}{(mar) azul profundo | azul-esverdeado ou azul-celeste (água) | frio | vasto (água)}
  \end{Phonetics}
\end{Entry}

\begin{Entry}{沧桑}{7,10}{⽔、⽊}
  \begin{Phonetics}{沧桑}{cang1sang1}[][HSK 7-9]
    \definition{adj.}{vicissitudes; grandes mudanças; altos e baixos; abreviação de 沧海桑田}
  \seealsoref{沧海桑田}{cang1 hai3 sang1 tian2}
  \end{Phonetics}
\end{Entry}

\begin{Entry}{沧海桑田}{7,10,10,5}{⽔、⽔、⽊、⽥}
  \begin{Phonetics}{沧海桑田}{cang1 hai3 sang1 tian2}
    \definition{expr.}{mudança dos mares para campos de amoreiras e dos campos de amoreiras para mares --- o tempo traz grandes mudanças; vicissitudes | Figurativo: as transformações do mundo | Literário: o mar azul se transformou em campos de amoreiras}
  \end{Phonetics}
\end{Entry}

\begin{Entry}{泛}{7}{⽔}
  \begin{Phonetics}{泛}{fan4}
    \definition{adj.}{extenso; amplo; geral; inespecífico | superficial; raso | amarelo (ficar amarelo)}
    \definition{v.}{Literário: flutuar; derivar | afastar-se; espalhar-se; transbordar | transbordar; inundar | emergir; sair}
  \end{Phonetics}
\end{Entry}

\begin{Entry}{泛滥}{7,13}{⽔、⽔}
  \begin{Phonetics}{泛滥}{fan4lan4}[][HSK 7-9]
    \definition{v.}{fluir; transbordar; inundar | espalhar sem controle; metáfora para coisas ruins se tornarem populares sem restrições}
  \end{Phonetics}
\end{Entry}

\begin{Entry}{灵}{7}{⽕}
  \begin{Phonetics}{灵}{ling2}
    \definition*{s.}{Sobrenome Ling}
    \definition{adj.}{rápido; inteligente; afiado | eficaz; efetivo | flexível; hábil}
    \definition{s.}{espírito; alma | inteligência; mente | fada; duende; elfo | restos mortais do falecido; esquife | carro funerário; caixão ou algo relacionado aos mortos}
  \end{Phonetics}
\end{Entry}

\begin{Entry}{灵活}{7,9}{⽕、⽔}
  \begin{Phonetics}{灵活}{ling2huo2}[][HSK 6]
    \definition[种,点,些]{adj.}{ágil; rápido; ligeiro; descreve a capacidade de fazer rapidamente mudanças apropriadas com base na situação ao lidar com as coisas | flexível; elástico; descreve reações rápidas, como movimentos e funções cerebrais}
  \end{Phonetics}
\end{Entry}

\begin{Entry}{灵感}{7,13}{⽕、⼼}
  \begin{Phonetics}{灵感}{ling2gan3}
    \definition{s.}{inspiração | explosão de criatividade em empreendimento científico ou artístico}
  \end{Phonetics}
\end{Entry}

\begin{Entry}{灵魂}{7,13}{⽕、⿁}
  \begin{Phonetics}{灵魂}{ling2hun2}
    \definition{s.}{alma | espírito}
  \end{Phonetics}
\end{Entry}

\begin{Entry}{灶}{7}{⽕}
  \begin{Phonetics}{灶}{zao4}
    \definition[口,个]{s.}{fogão de cozinha; fogão de cozinha | cozinha; bagunça; cantina}
  \end{Phonetics}
\end{Entry}

\begin{Entry}{灶台}{7,5}{⽕、⼝}
  \begin{Phonetics}{灶台}{zao4tai2}
    \definition{s.}{fogão}
  \end{Phonetics}
\end{Entry}

\begin{Entry}{灾}{7}{⽕}
  \begin{Phonetics}{灾}{zai1}[][HSK 5]
    \definition[个,场]{s.}{calamidade; desastre | infortúnio pessoal; adversidade | azar}
  \end{Phonetics}
\end{Entry}

\begin{Entry}{灾区}{7,4}{⽕、⼖}
  \begin{Phonetics}{灾区}{zai1 qu1}[][HSK 5]
    \definition{s.}{área de desastre; área afetada por catástrofes}
  \end{Phonetics}
\end{Entry}

\begin{Entry}{灾害}{7,10}{⽕、⼧}
  \begin{Phonetics}{灾害}{zai1hai4}[][HSK 5]
    \definition[场,次,个]{s.}{desastre; calamidade; danos causados pela seca, inundações, pragas, granizo, guerras, etc.}
  \end{Phonetics}
\end{Entry}

\begin{Entry}{灾难}{7,10}{⽕、⾫}
  \begin{Phonetics}{灾难}{zai1nan4}[][HSK 5]
    \definition[场,次,个,种]{s.}{desastre; sofrimento; calamidade; catástrofe; danos e sofrimentos causados por desastres naturais ou guerras}
  \end{Phonetics}
\end{Entry}

\begin{Entry}{灿}{7}{⽕}
  \begin{Phonetics}{灿}{can4}
    \definition{adj.}{brilhante; luminoso; resplandecente; flamejante; deslumbrante}
  \end{Phonetics}
\end{Entry}

\begin{Entry}{灿烂}{7,9}{⽕、⽕}
  \begin{Phonetics}{灿烂}{can4lan4}[][HSK 7-9]
    \definition{adj.}{brilhante; glorioso; esplêndido; resplandecente; magnífico; deslumbrante}
  \end{Phonetics}
\end{Entry}

\begin{Entry}{牢}{7}{⼧}
  \begin{Phonetics}{牢}{lao2}[][HSK 6]
    \definition*{s.}{Sobrenome Lao}
    \definition{adj.}{firme; durável}
    \definition{s.}{prisão; cadeia | (cercado para animais) curral; baia; galinheiro; estábulo; estrebaria; cocheira | (arcaico) animal de sacrifício}
  \end{Phonetics}
\end{Entry}

\begin{Entry}{状}{7}{⽝}
  \begin{Phonetics}{状}{zhuang4}
    \definition{s.}{forma | estado; condição | conta; registro | reclamação escrita; queixa; reclamação legal | certificado | situação; circunstâncias | documento oficial; documentos de elogio, nomeação, etc.}
    \definition{v.}{descrever | narrar}
  \end{Phonetics}
\end{Entry}

\begin{Entry}{状况}{7,7}{⽝、⼎}
  \begin{Phonetics}{状况}{zhuang4kuang4}[][HSK 3]
    \definition[个,种]{s.}{estado; \emph{status}; situação; condição; estado de coisas; a aparência ou o estado em que as coisas se apresentam}
  \end{Phonetics}
\end{Entry}

\begin{Entry}{状态}{7,8}{⽝、⼼}
  \begin{Phonetics}{状态}{zhuang4tai4}[][HSK 3]
    \definition[种,个]{s.}{\emph{status}; estado; condição; situação; estado de coisas; a forma manifestada por pessoas ou coisas}
  \end{Phonetics}
\end{Entry}

\begin{Entry}{犹}{7}{⽝}
  \begin{Phonetics}{犹}{you2}
    \definition*{s.}{Sobrenome You}
    \definition{adv.}{ainda | assim como; exatamente como; como se}
    \definition{v.}{ser exatamente como; ser como}
  \end{Phonetics}
\end{Entry}

\begin{Entry}{犹豫}{7,15}{⽝、⾗}
  \begin{Phonetics}{犹豫}{you2yu4}[][HSK 5]
    \definition{adj.}{hesitante; indeciso, incapaz de decidir ou agir}
    \definition{v.}{hesitar; ser indeciso}
  \end{Phonetics}
\end{Entry}

\begin{Entry}{狂}{7}{⽝}
  \begin{Phonetics}{狂}{kuang2}[][HSK 5]
    \definition*{s.}{Sobrenome Kuang}
    \definition{adj.}{louco; maluco | violento; selvagem | selvagem; delirante; furioso; desenfreado; desinibido; sem restrições | arrogante; autoritário}
  \end{Phonetics}
\end{Entry}

\begin{Entry}{狂欢节}{7,6,5}{⽝、⽋、⾋}
  \begin{Phonetics}{狂欢节}{kuang2huan1 jie2}
    \definition*{s.}{Carnaval}
  \end{Phonetics}
\end{Entry}

\begin{Entry}{男}{7}{⽥}
  \begin{Phonetics}{男}{nan2}[][HSK 1]
    \definition{adj.}{homem; macho; masculino (em oposição a 女)}
    \definition[个,位]{s.}{filho; menino | homem | barão (o mais baixo de cinco ordens de nobreza)}
  \seealsoref{女}{nv3}
  \end{Phonetics}
\end{Entry}

\begin{Entry}{男人}{7,2}{⽥、⼈}
  \begin{Phonetics}{男人}{nan2 ren2}[][HSK 1]
    \definition[个]{s.}{homem adulto; macho; cavalheiro | marido}
  \end{Phonetics}
\end{Entry}

\begin{Entry}{男士}{7,3}{⽥、⼠}
  \begin{Phonetics}{男士}{nan2 shi4}[][HSK 4]
    \definition{s.}{cavalheiro; \emph{gentleman}}
  \end{Phonetics}
\end{Entry}

\begin{Entry}{男女}{7,3}{⽥、⼥}
  \begin{Phonetics}{男女}{nan2 nv3}[][HSK 4]
    \definition{s.}{homens e mulheres; masculino e feminino}
  \end{Phonetics}
\end{Entry}

\begin{Entry}{男子}{7,3}{⽥、⼦}
  \begin{Phonetics}{男子}{nan2zi3}[][HSK 3]
    \definition[个,位]{s.}{uma pessoa do sexo masculino; um homem}
  \end{Phonetics}
\end{Entry}

\begin{Entry}{男生}{7,5}{⽥、⽣}
  \begin{Phonetics}{男生}{nan2 sheng1}[][HSK 1]
    \definition[个]{s.}{menino; estudante; estudante do sexo masculino; aluno do sexo masculino}
  \end{Phonetics}
\end{Entry}

\begin{Entry}{男性}{7,8}{⽥、⼼}
  \begin{Phonetics}{男性}{nan2 xing4}[][HSK 5]
    \definition{s.}{masculino; homem; masculinidade; em oposição a 女性}
  \seealsoref{女性}{nv3 xing4}
  \end{Phonetics}
\end{Entry}

\begin{Entry}{男朋友}{7,8,4}{⽥、⽉、⼜}
  \begin{Phonetics}{男朋友}{nan2 peng2 you5}[][HSK 1]
    \definition{s.}{namorado}
  \end{Phonetics}
\end{Entry}

\begin{Entry}{男孩儿}{7,9,2}{⽥、⼦、⼉}
  \begin{Phonetics}{男孩儿}{nan2hai2r5}[][HSK 1]
    \definition{s.}{menino; rapaz}
  \end{Phonetics}
\end{Entry}

\begin{Entry}{疗}{7}{⽧}
  \begin{Phonetics}{疗}{liao2}
    \definition{v.}{tratar; curar | recuperar}
  \end{Phonetics}
\end{Entry}

\begin{Entry}{疗养}{7,9}{⽧、⼋}
  \begin{Phonetics}{疗养}{liao2 yang3}[][HSK 4]
    \definition{v.}{recuperar; convalescer; tratar pessoas com doenças crônicas ou debilitantes em instituições médicas especializadas com foco na recuperação}
  \end{Phonetics}
\end{Entry}

\begin{Entry}{盯}{7}{⽬}
  \begin{Phonetics}{盯}{ding1}[][HSK 7-9]
    \definition{v.}{olhar; fixar os olhos em; olhar fixamente para; encarar; concentrar seu olhar em um ponto}
  \end{Phonetics}
\end{Entry}

\begin{Entry}{社}{7}{⽰}
  \begin{Phonetics}{社}{she4}[][HSK 5]
    \definition[个,家]{s.}{agência; sociedade; órgão organizado; organização; comunidade | comuna popular | o deus da terra, sacrifícios a ele ou altares para tais sacrifícios; na antiguidade, o deus da terra, o local onde ele era venerado, o dia da veneração e o ritual eram chamados de 社 | agência de notícias |  imprensa}
  \end{Phonetics}
\end{Entry}

\begin{Entry}{社区}{7,4}{⽰、⼖}
  \begin{Phonetics}{社区}{she4qu1}[][HSK 5]
    \definition[个]{s.}{bairro; comunidade residencial; bairros da cidade, divididos de acordo com a localização geográfica | distrito; comunidade (para pessoas da mesma classe social, etc.) ; lugar onde pessoas com características comuns, como classe social, vivem juntas}
  \end{Phonetics}
\end{Entry}

\begin{Entry}{社会}{7,6}{⽰、⼈}
  \begin{Phonetics}{社会}{she4hui4}[][HSK 3]
    \definition[个,种]{s.}{sociedade; em um determinado estágio do desenvolvimento histórico, a relação geral entre as pessoas nas atividades de produção | comunidade; geralmente se refere a um grupo de pessoas que estão conectadas por atividades comuns}
  \end{Phonetics}
\end{Entry}

\begin{Entry}{私}{7}{⽲}
  \begin{Phonetics}{私}{si1}
    \definition*{s.}{Sobrenome Si}
    \definition{adj.}{pessoal; privado (oposição a 公) | egoísta (oposto a 公) | secreto; privado | ilícito; ilegal}
    \definition{s.}{interesse privado (ou egoísta); motivo (ou ideia) egoísta (oposição a 公) | contrabando; mercadorias contrabandeadas | propriedade privada | interesses privados; ganho pessoal}
  \seealsoref{公}{gong1}
  \end{Phonetics}
\end{Entry}

\begin{Entry}{私人}{7,2}{⽲、⼈}
  \begin{Phonetics}{私人}{si1ren2}[][HSK 5]
    \definition{adj.}{privado; pertencente a um indivíduo ou exercido a título individual; não público | pessoal; entre indivíduos}
    \definition[个]{s.}{algo privado; pessoas que se aproximam de você por motivos pessoais ou interesses próprios}
  \end{Phonetics}
\end{Entry}

\begin{Entry}{私人诊所}{7,2,7,8}{⽲、⼈、⾔、⼾}
  \begin{Phonetics}{私人诊所}{si1ren2 zhen3suo3}
    \definition[些]{s.}{clínica privada}
  \end{Phonetics}
\end{Entry}

\begin{Entry}{私人信件}{7,2,9,6}{⽲、⼈、⼈、⼈}
  \begin{Phonetics}{私人信件}{si1ren2 xin4jian4}
    \definition{s.}{carta pessoal}
  \end{Phonetics}
\end{Entry}

\begin{Entry}{私人钥匙}{7,2,9,11}{⽲、⼈、⾦、⼔}
  \begin{Phonetics}{私人钥匙}{si1ren2yao4shi5}
    \definition{s.}{(criptografia) chave privada}
  \end{Phonetics}
\end{Entry}

\begin{Entry}{私生活}{7,5,9}{⽲、⽣、⽔}
  \begin{Phonetics}{私生活}{si1sheng1huo2}
    \definition{s.}{vida privada}
  \end{Phonetics}
\end{Entry}

\begin{Entry}{私立}{7,5}{⽲、⽴}
  \begin{Phonetics}{私立}{si1li4}
    \definition{s.}{privado; estabelecido privadamente}[这是一所私立学校。===Esta é uma escola particular.]
    \definition{v.}{estabelecer-se ilegalmente}
  \end{Phonetics}
\end{Entry}

\begin{Entry}{私自}{7,6}{⽲、⾃}
  \begin{Phonetics}{私自}{si1zi4}
    \definition{adj.}{privado | pessoal}
    \definition{adv.}{secretamente | sem aprovação explícita}
  \end{Phonetics}
\end{Entry}

\begin{Entry}{私事}{7,8}{⽲、⼅}
  \begin{Phonetics}{私事}{si1shi4}
    \definition[件,桩]{s.}{privacidade; assuntos privados; assuntos pessoais (oposto a 公事)}
  \seealsoref{公事}{gong1shi4}
  \end{Phonetics}
\end{Entry}

\begin{Entry}{私函}{7,8}{⽲、⼐}
  \begin{Phonetics}{私函}{si1han2}
    \definition{s.}{carta privada}
  \end{Phonetics}
\end{Entry}

\begin{Entry}{究}{7}{⽳}
  \begin{Phonetics}{究}{jiu1}
    \definition{adv.}{na verdade; realmente; afinal}
    \definition{v.}{estudar cuidadosamente; aprofundar; investigar; rastrear}
  \end{Phonetics}
\end{Entry}

\begin{Entry}{究竟}{7,11}{⽳、⾳}
  \begin{Phonetics}{究竟}{jiu1jing4}[][HSK 4]
    \definition{adv.}{de fato; exatamente; usado em frases interrogativas para buscar | afinal de contas, no final; ênfase em fatos ou motivos}
    \definition{s.}{resultado; desfecho; a causa, o efeito ou a história completa do que aconteceu}
  \end{Phonetics}
\end{Entry}

\begin{Entry}{穷}{7}{⽳}
  \begin{Phonetics}{穷}{qiong2}[][HSK 4]
    \definition{adj.}{remoto; isolado; de difícil acesso | pobre; atingido pela pobreza | situação difícil, sem saída}
    \definition{adv.}{completamente | extremamente}
    \definition{v.}{exaurir; esgotar; consmir | ir até o fim; perseguir completamente perseguido; sondar profundamente | gastar}
  \end{Phonetics}
\end{Entry}

\begin{Entry}{穷人}{7,2}{⽳、⼈}
  \begin{Phonetics}{穷人}{qiong2 ren2}[][HSK 4]
    \definition[个]{s.}{os pobres; pessoas pobres}
  \end{Phonetics}
\end{Entry}

\begin{Entry}{系}{7}{⽷}
  \begin{Phonetics}{系}{ji4}
    \definition{v.}{amarrar; prender; abotoar; dar um nó}
  \end{Phonetics}
  \begin{Phonetics}{系}{xi4}[][HSK 3,4]
    \definition*{s.}{Sobrenome Xi}
    \definition{s.}{sistema; série | departamento; faculdade; unidades administrativas de ensino divididas por disciplina nas instituições de ensino superior}
    \definition{v.}{relacionar-se com; suportar; depender de | sentir-se ansioso; estar preocupado | amarrar; prender | ser; expressa julgamento, equivalente a 是}
  \seealsoref{是}{shi4}
  \end{Phonetics}
\end{Entry}

\begin{Entry}{系囚}{7,5}{⽷、⼞}
  \begin{Phonetics}{系囚}{xi4qiu2}
    \definition{s.}{prisioneiro}
  \end{Phonetics}
\end{Entry}

\begin{Entry}{系列}{7,6}{⽷、⼑}
  \begin{Phonetics}{系列}{xi4lie4}[][HSK 4]
    \definition{s.}{série; conjunto; conjunto de coisas relacionadas (matemática)}
  \end{Phonetics}
\end{Entry}

\begin{Entry}{系统}{7,9}{⽷、⽷}
  \begin{Phonetics}{系统}{xi4tong3}[][HSK 4]
    \definition{adj.}{sistemático; organizado}
    \definition[套,个]{s.}{sistema; relação de tipos semelhantes (ou seja, grupo de coisas semelhantes)}
  \end{Phonetics}
\end{Entry}

\begin{Entry}{纬}{7}{⽷}
  \begin{Phonetics}{纬}{wei3}
    \definition*{s.}{Sobrenome Wei}
    \definition[本]{s.}{trama; o fio ou linha horizontal no tecido (oposto a 经) | latitude (oposto a 经)}
  \seealsoref{经}{jing1}
  \end{Phonetics}
\end{Entry}

\begin{Entry}{纯}{7}{⽷}
  \begin{Phonetics}{纯}{chun2}[][HSK 4]
    \definition{adj.}{puro; não misturado; livre de impurezas | simples; puro e simples | habilidoso; proficiente; bem versado}
    \definition{adv.}{puramente; completamente; totalmente | genuinamente}
  \end{Phonetics}
\end{Entry}

\begin{Entry}{纯朴}{7,6}{⽷、⽊}
  \begin{Phonetics}{纯朴}{chun2pu3}[][HSK 7-9]
    \definition{adj.}{honesto; simples; pouco sofisticado}
  \end{Phonetics}
\end{Entry}

\begin{Entry}{纯净水}{7,8,4}{⽷、⼎、⽔}
  \begin{Phonetics}{纯净水}{chun2 jing4 shui3}[][HSK 4]
    \definition{s.}{água pura | água purificada}
  \end{Phonetics}
\end{Entry}

\begin{Entry}{纯洁}{7,9}{⽷、⽔}
  \begin{Phonetics}{纯洁}{chun2jie2}[][HSK 7-9]
    \definition{adj.}{puro; limpo e honesto; sem impurezas e manchas; metáfora para pensamentos puros, sem pensamentos egoístas}
    \definition{v.}{purificar}
  \end{Phonetics}
\end{Entry}

\begin{Entry}{纯真}{7,10}{⽷、⼗}
  \begin{Phonetics}{纯真}{chun2zhen1}
    \definition{adj.}{inocente e não afetado | puro e não adulterado}
    \definition{s.}{inocência}
  \end{Phonetics}
\end{Entry}

\begin{Entry}{纯粹}{7,14}{⽷、⽶}
  \begin{Phonetics}{纯粹}{chun2cui4}[][HSK 7-9]
    \definition{adj.}{puro; sem mistura; não adulterado; sem outros ingredientes}
    \definition{adv.}{unicamente; puramente; somente; simplesmente, apenas}
  \end{Phonetics}
\end{Entry}

\begin{Entry}{纲}{7}{⽷}
  \begin{Phonetics}{纲}{gang1}
    \definition[条,个]{s.}{a corda principal para içar uma rede, frequentemente usada como metáfora | elo-chave; princípio orientador; esboço; programa | Biologia: classe | Arcaico: transporte de mercadorias em comboio (na China feudal)}
    \definition{v.}{amarrar; comprender | corrigir; esclarecer o contorno do problema}
  \end{Phonetics}
\end{Entry}

\begin{Entry}{纲要}{7,9}{⽷、⾑}
  \begin{Phonetics}{纲要}{gang1yao4}[][HSK 7-9]
    \definition{s.}{esboço; contorno | (usualmente em títulos de livros) essenciais; compêndio | fundamentos; princípios básicos}
  \end{Phonetics}
\end{Entry}

\begin{Entry}{纲领}{7,11}{⽷、⾴}
  \begin{Phonetics}{纲领}{gang1ling3}[][HSK 7-9]
    \definition{s.}{programa | princípios de orientação; normas diretivas; diretrizes | credo | esboço}
  \end{Phonetics}
\end{Entry}

\begin{Entry}{纵}{7}{⽷}
  \begin{Phonetics}{纵}{zong4}
    \definition{adj.}{de norte a sul; geograficamente norte-sul | longitudinal | vertical; horizontal; paralelo ao lado longo do objeto | amassado; com rugas}
    \definition{conj.}{embora; mesmo que}
    \definition{v.}{libertar; deixar ir | entregar-se a; deixar-se levar | pular; saltar}
  \end{Phonetics}
\end{Entry}

\begin{Entry}{纷}{7}{⽷}
  \begin{Phonetics}{纷}{fen1}
    \definition[场]{adj.}{confuso; emaranhado; desordenado | muitos e variados; profusos; numerosos}
  \end{Phonetics}
\end{Entry}

\begin{Entry}{纷纷}{7,7}{⽷、⽷}
  \begin{Phonetics}{纷纷}{fen1fen1}[][HSK 4]
    \definition{adj.}{numeroso e confuso; muitos e desordenados}
    \definition{adv.}{um após o outro; em sucessão; em rápida sucessão}
  \end{Phonetics}
\end{Entry}

\begin{Entry}{纸}{7}{⽷}
  \begin{Phonetics}{纸}{zhi3}[][HSK 2]
    \definition{clas.}{usado para documentos, cartas, etc.}
    \definition[张,沓]{s.}{papel; uma folha fina de material usada para escrever, pintar, imprimir, embalar, etc., feita principalmente de fibras vegetais | papel joss; papel de incenso; refere-se especificamente a itens supersticiosos, como papel-moeda}
  \end{Phonetics}
\end{Entry}

\begin{Entry}{纸巾}{7,3}{⽷、⼱}
  \begin{Phonetics}{纸巾}{zhi3jin1}
    \definition[张,包]{s.}{lenço | guardanapo | papel toalha}
  \end{Phonetics}
\end{Entry}

\begin{Entry}{纸币}{7,4}{⽷、⼱}
  \begin{Phonetics}{纸币}{zhi3bi4}
    \definition[张]{s.}{nota (dinheiro) | cédula}
  \end{Phonetics}
\end{Entry}

\begin{Entry}{纸尿裤}{7,7,12}{⽷、⼫、⾐}
  \begin{Phonetics}{纸尿裤}{zhi3niao4ku4}
    \definition{s.}{fralda descartável}
  \end{Phonetics}
\end{Entry}

\begin{Entry}{纸张}{7,7}{⽷、⼸}
  \begin{Phonetics}{纸张}{zhi3zhang1}
    \definition{s.}{papel}
  \end{Phonetics}
\end{Entry}

\begin{Entry}{纸烟}{7,10}{⽷、⽕}
  \begin{Phonetics}{纸烟}{zhi3yan1}
    \definition{s.}{cigarro}
  \end{Phonetics}
\end{Entry}

\begin{Entry}{纹}{7}{⽷}
  \begin{Phonetics}{纹}{wen2}
    \definition[个]{s.}{linhas; veios; grãos; rugas na pele | padrão; desenho em tecido de seda; listras ou padrões em tecidos de seda; geralmente se refere a padrões lineares na superfície de um objeto}
  \end{Phonetics}
\end{Entry}

\begin{Entry}{纹路}{7,13}{⽷、⾜}
  \begin{Phonetics}{纹路}{wen2lu4}
    \definition{s.}{padrão de linhas | rugas | veias | veias (em mármore ou impressão digital) | grãos (em madeira, etc.)}
  \end{Phonetics}
\end{Entry}

\begin{Entry}{纺}{7}{⽷}
  \begin{Phonetics}{纺}{fang3}
    \definition{s.}{tecido de seda fina | um pano de seda fino}
    \definition{v.}{girar | enrolar}
  \end{Phonetics}
\end{Entry}

\begin{Entry}{纺织}{7,8}{⽷、⽷}
  \begin{Phonetics}{纺织}{fang3zhi1}[][HSK 7-9]
    \definition{v.}{tecer; fiar; fiar algodão, linho, seda, lã e outras fibras em fios ou linhas e tecê-los em tecidos, cetim, lã, etc.}
  \end{Phonetics}
\end{Entry}

\begin{Entry}{罕}{7}{⽹}
  \begin{Phonetics}{罕}{han3}
    \definition*{s.}{Sobrenome Han}
    \definition{adj.}{raro; escasso | raro; incomum}
  \end{Phonetics}
\end{Entry}

\begin{Entry}{罕见}{7,4}{⽹、⾒}
  \begin{Phonetics}{罕见}{han3jian4}[][HSK 7-9]
    \definition{adj.}{raro; raramente visto}
  \end{Phonetics}
\end{Entry}

\begin{Entry}{肚}{7}{⾁}
  \begin{Phonetics}{肚}{du3}
    \definition{s.}{tripas; entranhas}
  \end{Phonetics}
  \begin{Phonetics}{肚}{du4}
    \definition{s.}{barriga; abdômen; estômago | tolerância}
  \end{Phonetics}
\end{Entry}

\begin{Entry}{肚子}{7,3}{⾁、⼦}
  \begin{Phonetics}{肚子}{du4zi5}[][HSK 4]
    \definition[个,只]{s.}{abdômen; barriguinha; ventre; barriga}
  \end{Phonetics}
\end{Entry}

\begin{Entry}{肝}{7}{⾁}
  \begin{Phonetics}{肝}{gan1}[][HSK 6]
    \definition[个]{s.}{fígado; um dos órgãos digestivos dos humanos e dos animais superiores}
  \end{Phonetics}
\end{Entry}

\begin{Entry}{肝脏}{7,10}{⾁、⾁}
  \begin{Phonetics}{肝脏}{gan1zang4}[][HSK 7-9]
    \definition{s.}{fígado}
  \end{Phonetics}
\end{Entry}

\begin{Entry}{肠}{7}{⾁}
  \begin{Phonetics}{肠}{chang2}[][HSK 5]
    \definition[根,段,片]{s.}{intestinos; parte do sistema digestivo | salsicha; linguiça; alimentos de tripas recheadas com carne, peixe, etc. | sentimentos; emoções; humor}
  \end{Phonetics}
\end{Entry}

\begin{Entry}{良}{7}{⾉}
  \begin{Phonetics}{良}{liang2}
    \definition*{s.}{Sobrenome Liang}
    \definition{adj.}{bom; ótimo; agradável}
    \definition{adv.}{muito; muito mesmo; de fato}
    \definition{s.}{boas pessoas; pessoas gentis; talentos excepcionais}
  \end{Phonetics}
\end{Entry}

\begin{Entry}{良心}{7,4}{⾉、⼼}
  \begin{Phonetics}{良心}{liang2xin1}
    \definition{s.}{consciência}
  \end{Phonetics}
\end{Entry}

\begin{Entry}{良田}{7,5}{⾉、⽥}
  \begin{Phonetics}{良田}{liang2tian2}
    \definition{s.}{terra agrícola boa | terra fértil}
  \end{Phonetics}
\end{Entry}

\begin{Entry}{良好}{7,6}{⾉、⼥}
  \begin{Phonetics}{良好}{liang2hao3}[][HSK 4]
    \definition{adj.}{bom; ótimo; bem; satisfatório}
  \end{Phonetics}
\end{Entry}

\begin{Entry}{良性}{7,8}{⾉、⼼}
  \begin{Phonetics}{良性}{liang2xing4}
    \definition{adj.}{favorável; benéfico | Medicina: benigno (tumor, etc.) | levando a boas consequências | positivo (em seu efeito) | virtuoso}
  \end{Phonetics}
\end{Entry}

\begin{Entry}{芥}{7}{⾋}
  \begin{Phonetics}{芥}{gai4}
    \definition{s.}{mostarda}
  \seealsoref{芥蓝}{gai4lan2}
  \end{Phonetics}
  \begin{Phonetics}{芥}{jie4}
    \definition{s.}{mostarda}
  \end{Phonetics}
\end{Entry}

\begin{Entry}{芥兰}{7,5}{⾋、⼋}
  \begin{Phonetics}{芥兰}{gai4lan2}
    \variantof{芥蓝}
  \end{Phonetics}
  \begin{Phonetics}{芥兰}{jie4lan2}
    \definition{s.}{couve}
  \end{Phonetics}
\end{Entry}

\begin{Entry}{芥蓝}{7,13}{⾋、⾋}
  \begin{Phonetics}{芥蓝}{gai4lan2}
    \definition{s.}{brócolis chinês; couve chinesa; mostarda}
  \seealsoref{格兰菜}{ge2lan2cai4}
  \end{Phonetics}
\end{Entry}

\begin{Entry}{芦}{7}{⾋}
  \begin{Phonetics}{芦}{lu2}
    \definition*{s.}{Sobrenome Lu}
    \definition{s.}{junco}
  \end{Phonetics}
\end{Entry}

\begin{Entry}{芦笋}{7,10}{⾋、⽵}
  \begin{Phonetics}{芦笋}{lu2sun3}
    \definition{s.}{aspargos}
  \end{Phonetics}
\end{Entry}

\begin{Entry}{芬}{7}{⾋}
  \begin{Phonetics}{芬}{fen1}
    \definition*{s.}{Sobrenome Fen}
    \definition{adj.}{perfumado}
    \definition[阵,股]{s.}{cheiro doce; fragrância; aroma}
  \end{Phonetics}
\end{Entry}

\begin{Entry}{芬芳}{7,7}{⾋、⾋}
  \begin{Phonetics}{芬芳}{fen1fang1}[][HSK 7-9]
    \definition{adj.}{perfumado; com cheiro doce}
    \definition{s.}{fragrância; aroma; cheiro doce}
  \end{Phonetics}
\end{Entry}

\begin{Entry}{芭}{7}{⾋}
  \begin{Phonetics}{芭}{ba1}
    \definition{s.}{Arcaico: uma planta perfumada | junco; um tipo de erva | flor}
  \end{Phonetics}
\end{Entry}

\begin{Entry}{芭蕾}{7,16}{⾋、⾋}
  \begin{Phonetics}{芭蕾}{ba1lei3}[][HSK 7-9]
    \definition[场]{s.}{Empréstimo linguístico: balé; um tipo de dança popular na Europa, em que as dançarinas costumam tocar o chão com os dedos dos pés}
  \seealsoref{芭蕾舞}{ba1lei3wu3}
  \end{Phonetics}
\end{Entry}

\begin{Entry}{芭蕾舞}{7,16,14}{⾋、⾋、⾇}
  \begin{Phonetics}{芭蕾舞}{ba1lei3wu3}
    \definition{s.}{Empréstimo linguístico: balé; um tipo de dança clássica europeia; também conhecido como balé clássico europeu}
  \end{Phonetics}
\end{Entry}

\begin{Entry}{芯}{7}{⾋}
  \begin{Phonetics}{芯}{xin1}
    \definition{s.}{medula de junco | pavio}
  \end{Phonetics}
  \begin{Phonetics}{芯}{xin4}
    \definition{s.}{núcleo; a parte central de um objeto | língua de cobra}
  \end{Phonetics}
\end{Entry}

\begin{Entry}{芯片}{7,4}{⾋、⽚}
  \begin{Phonetics}{芯片}{xin1pian4}
    \definition{s.}{\emph{chip} de computador; \emph{microchip}; um substrato (geralmente uma pastilha de silício) que contém um circuito integrado completo}
  \end{Phonetics}
\end{Entry}

\begin{Entry}{花}{7}{⾋}
  \begin{Phonetics}{花}{hua1}[][HSK 1,2,4]
    \definition*{s.}{Sobrenome Hua}
    \definition{adj.}{multicolorido; colorido | embaçado; obscuro; deslumbrado e confuso | extravagante; florido; vistoso}
    \definition[朵,支,束,把,盆,簇]{s.}{flor; órgãos de reprodução sexual de plantas com sementes | flor; planta ornamental |  qualquer coisa que se assemelhe a uma flor | fogos de artifício | padrão; design; design decorativo | flor; metáfora para a essência de uma causa | prostituta; cortesã; referindo-se a prostitutas ou a assuntos relacionados a prostitutas | algodão | varíola | ferimento; ferida; lesões traumáticas sofridas em combate}
    \definition{v.}{gastar; despender; consumir}
  \end{Phonetics}
\end{Entry}

\begin{Entry}{花儿}{7,2}{⾋、⼉}
  \begin{Phonetics}{花儿}{hua1r5}
    \definition[朵,支,束,把,盆,簇]{s.}{flor}
  \end{Phonetics}
\end{Entry}

\begin{Entry}{花生}{7,5}{⾋、⽣}
  \begin{Phonetics}{花生}{hua1sheng1}[][HSK 6]
    \definition[把,颗,粒,袋]{s.}{amendoim}
  \seealsoref{落花生}{luo4 hua1 sheng1}
  \end{Phonetics}
\end{Entry}

\begin{Entry}{花园}{7,7}{⾋、⼞}
  \begin{Phonetics}{花园}{hua1 yuan2}[][HSK 2]
    \definition[个,座]{s.}{jardim; um local onde se plantam flores e árvores para passear e descansar}
  \end{Phonetics}
\end{Entry}

\begin{Entry}{花店}{7,8}{⾋、⼴}
  \begin{Phonetics}{花店}{hua1dian4}
    \definition{s.}{floricultura}
  \end{Phonetics}
\end{Entry}

\begin{Entry}{花茶}{7,9}{⾋、⾋}
  \begin{Phonetics}{花茶}{hua1cha2}
    \definition[杯,壶]{s.}{chá perfumado}
  \end{Phonetics}
\end{Entry}

\begin{Entry}{花费}{7,9}{⾋、⾙}
  \begin{Phonetics}{花费}{hua1 fei4}[][HSK 6]
    \definition[笔]{s.}{dinheiro gasto; despesas | custo; gastos; desembolso | despesa}
    \definition{v.}{gastar (tempo ou dinheiro)}
  \end{Phonetics}
\end{Entry}

\begin{Entry}{花样游泳}{7,10,12,8}{⾋、⽊、⽔、⽔}
  \begin{Phonetics}{花样游泳}{hua1yang4you2yong3}
    \definition{s.}{nado sincronizado}
  \end{Phonetics}
\end{Entry}

\begin{Entry}{花瓶}{7,10}{⾋、⽡}
  \begin{Phonetics}{花瓶}{hua1 ping2}[][HSK 6]
    \definition[个,对]{s.}{vaso de flores; vaso usado para arranjos florais colocado em ambientes internos como decoração | Figurativo: um ornamento; mulher empregada não por sua habilidade, mas por sua aparência; uma metáfora para uma pessoa ou coisa que é usada apenas para exibição e não tem uso prático}
  \end{Phonetics}
\end{Entry}

\begin{Entry}{花脸}{7,11}{⾋、⾁}
  \begin{Phonetics}{花脸}{hua1lian3}
    \definition*{s.}{Hualian, personagem do rosto florido, um nome popular para 净 (assim chamado devido à elaborada pintura facial)}
  \seealsoref{净}{jing4}
  \end{Phonetics}
\end{Entry}

\begin{Entry}{花椰菜}{7,12,11}{⾋、⽊、⾋}
  \begin{Phonetics}{花椰菜}{hua1ye1cai4}
    \definition{s.}{couve-flor}
  \end{Phonetics}
\end{Entry}

\begin{Entry}{芹}{7}{⾋}
  \begin{Phonetics}{芹}{qin2}
    \definition[把,棵]{s.}{aipo | aipo chinês}
  \end{Phonetics}
\end{Entry}

\begin{Entry}{芹菜}{7,11}{⾋、⾋}
  \begin{Phonetics}{芹菜}{qin2cai4}
    \definition{s.}{salsão}
  \end{Phonetics}
\end{Entry}

\begin{Entry}{苍}{7}{⾋}
  \begin{Phonetics}{苍}{cang1}
    \definition*{s.}{Sobrenome Cang}
    \definition{adj.}{verde escuro (ou azul); ciano (inclui azul e verde) | cinza; acinzentado}
    \definition{s.}{o céu azul; o céu acima}
  \end{Phonetics}
\end{Entry}

\begin{Entry}{苍蝇}{7,14}{⾋、⾍}
  \begin{Phonetics}{苍蝇}{cang1ying5}[][HSK 7-9]
    \definition[只,个,群]{s.}{mosca;mosca doméstica}
  \end{Phonetics}
\end{Entry}

\begin{Entry}{苏}{7}{⾋}
  \begin{Phonetics}{苏}{su1}
    \definition*{s.}{Suzhou, abreviação de 苏州 | Província de Jiangsu, abreviação de 江苏 | União Soviética, abreviação de 苏联 | Sobrenome Su}
    \definition{s.}{perilla planta da família das mentas}
    \definition{v.}{reviver; vir a; acordar}
  \seealsoref{江苏}{jiang1su1}
  \seealsoref{苏联}{su1lian2}
  \seealsoref{苏州}{su1zhou1}
  \end{Phonetics}
\end{Entry}

\begin{Entry}{苏州}{7,6}{⾋、⼮}
  \begin{Phonetics}{苏州}{su1zhou1}
    \definition*{s.}{Suzhou, cidade na Província de Jiangsu}
  \end{Phonetics}
\end{Entry}

\begin{Entry}{苏格兰}{7,10,5}{⾋、⽊、⼋}
  \begin{Phonetics}{苏格兰}{su1ge2lan2}
    \definition*{s.}{Escócia}
  \end{Phonetics}
\end{Entry}

\begin{Entry}{苏联}{7,12}{⾋、⽿}
  \begin{Phonetics}{苏联}{su1lian2}
    \definition*{s.}{União das Repúblicas Socialistas Soviéticas (1922-1991)}
  \end{Phonetics}
\end{Entry}

\begin{Entry}{补}{7}{⾐}
  \begin{Phonetics}{补}{bu3}[][HSK 3]
    \definition*{s.}{Sobrenome Bu}
    \definition{s.}{ajuda; uso; benefício; utilidade}
    \definition{v.}{reparar; consertar; remendar; adicionar materiais, consertar coisas quebradas | abastecer; encher; repor; adicionar suplemento; complementar; completar; preencher | nutrir}
  \end{Phonetics}
\end{Entry}

\begin{Entry}{补习}{7,3}{⾐、⼄}
  \begin{Phonetics}{补习}{bu3 xi2}[][HSK 6]
    \definition{clas.}{treinar; ter aulas depois da escola ou do trabalho; estudar depois da aula ou no seu tempo livre para compensar a falta de conhecimento}
  \end{Phonetics}
\end{Entry}

\begin{Entry}{补充}{7,6}{⾐、⼉}
  \begin{Phonetics}{补充}{bu3chong1}[][HSK 3]
    \definition{adj.}{adicional | suplementar}
    \definition{v.}{reabastecer; suplementar; complementar; aumentar uma parte quando houver insuficiência ou perda}
  \end{Phonetics}
\end{Entry}

\begin{Entry}{补考}{7,6}{⾐、⽼}
  \begin{Phonetics}{补考}{bu3 kao3}[][HSK 6]
    \definition{v.}{repetir ou refazer um exame}
  \end{Phonetics}
\end{Entry}

\begin{Entry}{补助}{7,7}{⾐、⼒}
  \begin{Phonetics}{补助}{bu3 zhu4}[][HSK 6]
    \definition{s.}{subsídio; mesada}
    \definition{v.}{ajudar financeiramente; subsidiar}
  \end{Phonetics}
\end{Entry}

\begin{Entry}{补给}{7,9}{⾐、⽷}
  \begin{Phonetics}{补给}{bu3ji3}[][HSK 7-9]
    \definition{v.}{fornecer; prover; equipar; reabastecer; alimentar; recarregar}
  \end{Phonetics}
\end{Entry}

\begin{Entry}{补贴}{7,9}{⾐、⾙}
  \begin{Phonetics}{补贴}{bu3tie1}[][HSK 5]
    \definition[笔,项,种,份]{s.}{subsídio; ajuda de custo; custos de indenização ou assistência concedida a empresas ou indivíduos pelo estado ou governo}
    \definition{v.}{subsidiar; compensar a falta de dinheiro ou coisas; refere-se principalmente à compensação financeira ou ajuda dada pelo estado ou governo a empresas ou indivíduos}
  \end{Phonetics}
\end{Entry}

\begin{Entry}{补课}{7,10}{⾐、⾔}
  \begin{Phonetics}{补课}{bu3/ke4}[][HSK 6]
    \definition{v.+compl.}{compensar uma aula perdida; compensar cursos perdidos | refazer; fazer algo de novo; metáfora para refazer algo que não foi bem feito}
  \end{Phonetics}
\end{Entry}

\begin{Entry}{补偿}{7,11}{⾐、⼈}
  \begin{Phonetics}{补偿}{bu3chang2}[][HSK 5]
    \definition{v.}{compensar (perda, consumo); compensar (deficiências, diferenças)}
  \end{Phonetics}
\end{Entry}

\begin{Entry}{补救}{7,11}{⾐、⽁}
  \begin{Phonetics}{补救}{bu3jiu4}[][HSK 7-9]
    \definition{v.}{remediar; depois que algo dá errado, tomar medidas para compensar e salvar a situação}
  \end{Phonetics}
\end{Entry}

\begin{Entry}{角}{7}{⾓}[Kangxi 148]
  \begin{Phonetics}{角}{jiao3}[][HSK 2]
    \definition*{s.}{Jiao, uma das mansões lunares}
    \definition{clas.}{uma unidade monetária fracionária na China (=1/10 de um yuan ou 10 fen)}
    \definition[个,只,对]{s.}{chifre; o objeto duro que cresce na cabeça de bovinos, ovinos, veados, etc. | buzina; corneta; instrumentos musicais tocados no exército antigo | algo com a forma de um chifre | cabo; promontório; península | esquina; canto; a junção entre duas arestas de um objeto | ângulo}
  \end{Phonetics}
  \begin{Phonetics}{角}{jue2}
    \definition*{s.}{Sobrenome Jue}
    \definition[个,只,对]{s.}{papel; parte; personagem | tipo de papel (no drama tradicional chinês); categorias de divisão profissional do trabalho entre atores de ópera | ator ou atriz | uma antiga taça de vinho com três pernas | uma nota da antiga escala chinesa de cinco tons, correspondente a 3 na notação musical numerada}
    \definition{v.}{competir; contender; lutar}
  \end{Phonetics}
\end{Entry}

\begin{Entry}{角色}{7,6}{⾓、⾊}
  \begin{Phonetics}{角色}{jue2se4}[][HSK 4]
    \definition{s.}{papel; personagem em uma peça; personagem representado por um ator | papel; função; parte; uma metáfora para um certo tipo de pessoas na vida social}
  \end{Phonetics}
\end{Entry}

\begin{Entry}{角度}{7,9}{⾓、⼴}
  \begin{Phonetics}{角度}{jiao3du4}[][HSK 2]
    \definition[个,种]{s.}{perspectiva; ponto de vista; o ponto de partida para ver as coisas | ângulo; o tamanho do ângulo; normalmente expresso em graus ou radianos}
  \end{Phonetics}
\end{Entry}

\begin{Entry}{言}{7}{⾔}[Kangxi 149]
  \begin{Phonetics}{言}{yan2}
    \definition*{s.}{Sobrenome Yan}
    \definition{s.}{palavra; discurso; o que foi dito | palavra; caracter; uma frase ou palavra chinesa}
    \definition{v.}{dizer; falar}
  \end{Phonetics}
\end{Entry}

\begin{Entry}{言论}{7,6}{⾔、⾔}
  \begin{Phonetics}{言论}{yan2lun4}
    \definition{s.}{expressão de opinião |  visualizações | comentários | argumentos}
  \end{Phonetics}
\end{Entry}

\begin{Entry}{言语}{7,9}{⾔、⾔}
  \begin{Phonetics}{言语}{yan2 yu3}[][HSK 5]
    \definition{s.}{verbal; fala; linguagem falada; conversa; palavras}
  \end{Phonetics}
\end{Entry}

\begin{Entry}{证}{7}{⾔}
  \begin{Phonetics}{证}{zheng4}[][HSK 3]
    \definition{s.}{evidência; prova; testemunho | certificado; cartão | evidência; testemunha | doença; enfermidade}
    \definition{v.}{provar; demonstrar | verificar}
  \end{Phonetics}
\end{Entry}

\begin{Entry}{证书}{7,4}{⾔、⼄}
  \begin{Phonetics}{证书}{zheng4shu1}[][HSK 5]
    \definition[张,份,些]{s.}{certificado; documentos emitidos por instituições, grupos, etc., que comprovem experiência, nível, honras, poderes, etc.}
  \end{Phonetics}
\end{Entry}

\begin{Entry}{证件}{7,6}{⾔、⼈}
  \begin{Phonetics}{证件}{zheng4jian4}[][HSK 3]
    \definition[个,本,张,份]{s.}{documentos; credenciais; certificado; documentos que comprovem a identidade, experiência, etc., tais como carteira de estudante, carteira de trabalho, diploma de graduação, etc.}
  \end{Phonetics}
\end{Entry}

\begin{Entry}{证实}{7,8}{⾔、⼧}
  \begin{Phonetics}{证实}{zheng4shi2}[][HSK 5]
    \definition{v.}{verificar; afirmar; confirmar; corroborar; demonstrar; autenticar; provar que é verdadeiro}
  \end{Phonetics}
\end{Entry}

\begin{Entry}{证明}{7,8}{⾔、⽇}
  \begin{Phonetics}{证明}{zheng4ming2}[][HSK 3]
    \definition[个,份]{s.}{certificado; atestado; identificação; certificado ou carta de referência; documentos que comprovem identidade, experiência, etc., tais como carteira de estudante, carteira de trabalho, diploma de graduação, etc.}
    \definition{v.}{provar; testemunhar; sustentar; usar materiais confiáveis para demonstrar ou determinar a autenticidade de pessoas ou coisas}
  \end{Phonetics}
\end{Entry}

\begin{Entry}{证据}{7,11}{⾔、⼿}
  \begin{Phonetics}{证据}{zheng4ju4}[][HSK 3]
    \definition{s.}{prova; evidência; testemunho; fatos ou materiais que comprovam a veracidade de algo}
  \end{Phonetics}
\end{Entry}

\begin{Entry}{评}{7}{⾔}
  \begin{Phonetics}{评}{ping2}[][HSK 6]
    \definition*{s.}{Sobrenome Ping}
    \definition{v.}{comentar; criticar; revisar | julgar; avaliar}
  \end{Phonetics}
\end{Entry}

\begin{Entry}{评价}{7,6}{⾔、⼈}
  \begin{Phonetics}{评价}{ping2jia4}[][HSK 3]
    \definition[个,项,条,份]{s.}{avaliação; apreciação; comentários ou opiniões de pessoas sobre alguém ou algo}
    \definition{v.}{estimar valor; avaliar valor}
  \end{Phonetics}
\end{Entry}

\begin{Entry}{评论}{7,6}{⾔、⾔}
  \begin{Phonetics}{评论}{ping2lun4}[][HSK 5]
    \definition[篇,些]{s.}{revisão; comentário; artigos ou comentários críticos}
    \definition{v.}{discutir; comentar sobre algo ou alguém}
  \end{Phonetics}
\end{Entry}

\begin{Entry}{评估}{7,7}{⾔、⼈}
  \begin{Phonetics}{评估}{ping2gu1}[][HSK 5]
    \definition{v.}{estimar; avaliar; apreciar; avaliar e estimar (coisas abstratas)}
  \end{Phonetics}
\end{Entry}

\begin{Entry}{评选}{7,9}{⾔、⾡}
  \begin{Phonetics}{评选}{ping2 xuan3}[][HSK 6]
    \definition{v.}{escolher por meio de avaliação pública; avaliar e eleger}
  \end{Phonetics}
\end{Entry}

\begin{Entry}{诅}{7}{⾔}
  \begin{Phonetics}{诅}{zu3}
    \definition{v.}{amaldiçoar; xingar; imprecar; abusar; desejar algo maldoso | jurar; fazer um voto; prestar juramento}
  \end{Phonetics}
\end{Entry}

\begin{Entry}{诅咒}{7,8}{⾔、⼝}
  \begin{Phonetics}{诅咒}{zu3zhou4}
    \definition{v.}{amaldiçoar}
  \end{Phonetics}
\end{Entry}

\begin{Entry}{识}{7}{⾔}
  \begin{Phonetics}{识}{shi2}[][HSK 6]
    \definition{s.}{percepção; conhecimento}
    \definition{v.}{saber; reconhecer | saber; entender}
  \end{Phonetics}
  \begin{Phonetics}{识}{zhi4}
    \definition{s.}{marca; sinal; símbolo}
    \definition{v.}{lembrar; memorizar | anotar; registrar}
  \end{Phonetics}
\end{Entry}

\begin{Entry}{识字}{7,6}{⾔、⼦}
  \begin{Phonetics}{识字}{shi2 zi4}[][HSK 6]
    \definition{v.}{aprender a ler; tornar-se alfabetizado; reconhecer caracteres}
  \end{Phonetics}
\end{Entry}

\begin{Entry}{诊}{7}{⾔}
  \begin{Phonetics}{诊}{zhen3}
    \definition{v.}{examinar (um paciente)}
  \end{Phonetics}
\end{Entry}

\begin{Entry}{诊断}{7,11}{⾔、⽄}
  \begin{Phonetics}{诊断}{zhen3duan4}[][HSK 5]
    \definition{s.}{diagnóstico; diacrisis}
    \definition{v.}{diagnosticar; após examinar os sintomas do paciente, determinar a doença e seu desenvolvimento}
  \end{Phonetics}
\end{Entry}

\begin{Entry}{词}{7}{⾔}
  \begin{Phonetics}{词}{ci2}[][HSK 2]
    \definition[个,组,句,段,首]{s.}{palavra; termo; antigamente, referia-se a palavras vazias; atualmente, refere-se a palavras com forma fonética fixa e significado específico na língua; a menor unidade que pode ser usada de forma independente | discurso; declaração; linguagem; texto | ci (um tipo de poesia clássica chinesa, originária da dinastia Tang e plenamente desenvolvida na dinastia Song); gênero poético escrito de acordo com uma estrutura fixa, com versos de comprimentos variados | palavras; redação; refere-se genericamente ao teatro; a parte da letra cantada em harmonia com a melodia em canções e certas artes vocais}
  \end{Phonetics}
\end{Entry}

\begin{Entry}{词汇}{7,5}{⾔、⽔}
  \begin{Phonetics}{词汇}{ci2hui4}[][HSK 4]
    \definition[个,组,批,串,堆]{s.}{vocabulário; termo geral para palavras usadas em um idioma}
  \end{Phonetics}
\end{Entry}

\begin{Entry}{词典}{7,8}{⾔、⼋}
  \begin{Phonetics}{词典}{ci2 dian3}[][HSK 2]
    \definition[本,部]{s.}{dicionário, livro de referência que reúne palavras e explicações para consulta}
  \seealsoref{字典}{zi4 dian3}
  \end{Phonetics}
\end{Entry}

\begin{Entry}{词语}{7,9}{⾔、⾔}
  \begin{Phonetics}{词语}{ci2yu3}[][HSK 2]
    \definition[个,租]{s.}{termo; palavra; expressão; conjunto de palavras e frases}
  \end{Phonetics}
\end{Entry}

\begin{Entry}{谷}{7}{⾕}[Kangxi 150]
  \begin{Phonetics}{谷}{gu3}
    \definition*{s.}{Sobrenome Gu}
    \definition{adj.}{bom; gentil}
    \definition{s.}{vale; ravina; desfiladeiro; garganta; faixa estreita de terra com uma saída no meio de duas colinas ou dois platôs | arroz não descascado | salário de funcionário (na época feudal) | calha; cocho; canal | fossa sob o cerebelo (anatomia); valécula | dificuldade; dilema}
    \definition{v.}{criar (filhos) | crescer}
  \end{Phonetics}
\end{Entry}

\begin{Entry}{豆}{7}{⾖}[Kangxi 151]
  \begin{Phonetics}{豆}{dou4}
    \definition*{s.}{Sobrenome Dou}
    \definition{s.}{planta que produz vagens ou suas sementes | coisa em forma de feijão | leguminosas ou sementes de leguminosas; feijões; ervilhas | uma xícara ou tigela antiga com haste}
  \end{Phonetics}
\end{Entry}

\begin{Entry}{豆子}{7,3}{⾖、⼦}
  \begin{Phonetics}{豆子}{dou4zi5}[][HSK 7-9]
    \definition[颗,粒,把,袋]{s.}{planta que produz vagens ou suas sementes | algo em forma de feijão | leguminosas; feijões}
  \end{Phonetics}
\end{Entry}

\begin{Entry}{豆角}{7,7}{⾖、⾓}
  \begin{Phonetics}{豆角}{dou4jiao3}
    \definition{s.}{feijão verde}
  \end{Phonetics}
\end{Entry}

\begin{Entry}{豆制品}{7,8,9}{⾖、⼑、⼝}
  \begin{Phonetics}{豆制品}{dou4 zhi4 pin3}[][HSK 5]
    \definition{s.}{produtos de soja}
  \end{Phonetics}
\end{Entry}

\begin{Entry}{豆荚}{7,9}{⾖、⾋}
  \begin{Phonetics}{豆荚}{dou4jia2}
    \definition{s.}{vagem (de legumes)}
  \end{Phonetics}
\end{Entry}

\begin{Entry}{豆浆}{7,10}{⾖、⽔}
  \begin{Phonetics}{豆浆}{dou4jiang1}[][HSK 7-9]
    \definition[杯,碗]{s.}{leite de soja}
  \end{Phonetics}
\end{Entry}

\begin{Entry}{豆腐}{7,14}{⾖、⾁}
  \begin{Phonetics}{豆腐}{dou4fu5}[][HSK 4]
    \definition[块,盒,斤,盘]{s.}{\emph{tofu}}
  \end{Phonetics}
\end{Entry}

\begin{Entry}{贡}{7}{⾙}
  \begin{Phonetics}{贡}{gong4}
    \definition*{s.}{Sobrenome Gong}
    \definition[个,批]{s.}{tributo}
    \definition{v.}{recomendar uma pessoa adequada à corte imperial; recomendar talentos à corte na era feudal | prestar homenagem; pagar tributos (à corte imperial)}
  \end{Phonetics}
\end{Entry}

\begin{Entry}{贡献}{7,13}{⾙、⽝}
  \begin{Phonetics}{贡献}{gong4xian4}[][HSK 6]
    \definition[份]{s.}{contribuição; boas ações feitas para o país ou para o público}
    \definition{v.}{dedicar; contribuir; contribuir com materiais, força, experiência, etc. para o país ou para o público}
  \end{Phonetics}
\end{Entry}

\begin{Entry}{财}{7}{⾙}
  \begin{Phonetics}{财}{cai2}
    \definition[笔]{s.}{riqueza; dinheiro; fortuna | propriedade; objetos de valor; um termo geral para dinheiro e materiais}
  \end{Phonetics}
\end{Entry}

\begin{Entry}{财力}{7,2}{⾙、⼒}
  \begin{Phonetics}{财力}{cai2li4}[][HSK 7-9]
    \definition{s.}{capacidade financeira; recursos financeiros; poder econômico (refere-se principalmente ao capital)}
  \end{Phonetics}
\end{Entry}

\begin{Entry}{财务}{7,5}{⾙、⼒}
  \begin{Phonetics}{财务}{cai2wu4}[][HSK 7-9]
    \definition{s.}{finanças; assuntos financeiros; assuntos relacionados à administração de bens ou retirada de dinheiro, guarda e cálculo}
  \end{Phonetics}
\end{Entry}

\begin{Entry}{财产}{7,6}{⾙、⼇}
  \begin{Phonetics}{财产}{cai2chan3}[][HSK 4]
    \definition[笔]{s.}{ativos; propriedade; pertences; refere-se à posse de riqueza material, como dinheiro, bens, casas, terras, etc.}
  \end{Phonetics}
\end{Entry}

\begin{Entry}{财物}{7,8}{⾙、⽜}
  \begin{Phonetics}{财物}{cai2wu4}[][HSK 7-9]
    \definition{s.}{dinheiro e bens; propriedade (espólio não incluído) | propriedade; pertences; dinheiro e suprimentos}
  \end{Phonetics}
\end{Entry}

\begin{Entry}{财经}{7,8}{⾙、⽷}
  \begin{Phonetics}{财经}{cai2jing1}[][HSK 7-9]
    \definition{s.}{finanças e economia}
  \end{Phonetics}
\end{Entry}

\begin{Entry}{财政}{7,9}{⾙、⽁}
  \begin{Phonetics}{财政}{cai2zheng4}[][HSK 7-9]
    \definition{s.}{finanças; a gestão estatal das receitas e despesas financeiras}
  \end{Phonetics}
\end{Entry}

\begin{Entry}{财富}{7,12}{⾙、⼧}
  \begin{Phonetics}{财富}{cai2fu4}[][HSK 4]
    \definition{s.}{riqueza; fortuna; algo de valor}
  \end{Phonetics}
\end{Entry}

\begin{Entry}{赤}{7}{⾚}[Kangxi 155]
  \begin{Phonetics}{赤}{chi4}
    \definition*{s.}{Sobrenome Chi}
    \definition{adj.}{vermelho | um tipo de vermelho um pouco mais claro que o vermelhão | (história)  revolucionário; comunista | leal; sincero | nú; exposto}
    \definition{s.}{ouro puro}
  \end{Phonetics}
\end{Entry}

\begin{Entry}{赤字}{7,6}{⾚、⼦}
  \begin{Phonetics}{赤字}{chi4zi4}[][HSK 7-9]
    \definition{s.}{déficit; refere-se à diferença entre despesas e receitas nas atividades econômicas}
  \end{Phonetics}
\end{Entry}

\begin{Entry}{走}{7}{⾛}[Kangxi 156]
  \begin{Phonetics}{走}{zou3}[][HSK 1]
    \definition{v.}{andar; caminhar | correr | mover; movimentar; deslocar | sair; partir; ir embora | visitar; fazer uma visita; (entre amigos e familiares) troca de visitas | passar por; atravessar; ultrapassar | vazar; revelar; divulgar | afastar-se do original; alterar ou perder a forma, o sabor, a cor, etc. originais}
  \end{Phonetics}
\end{Entry}

\begin{Entry}{走开}{7,4}{⾛、⼶}
  \begin{Phonetics}{走开}{zou3 kai1}[][HSK 2]
    \definition{v.}{ir embora; fugir; ir para outro lugar}
  \end{Phonetics}
\end{Entry}

\begin{Entry}{走去}{7,5}{⾛、⼛}
  \begin{Phonetics}{走去}{zou3qu4}
    \definition{v.}{caminhar até (para)}
  \end{Phonetics}
\end{Entry}

\begin{Entry}{走过}{7,6}{⾛、⾡}
  \begin{Phonetics}{走过}{zou3 guo4}[][HSK 2]
    \definition{v.}{passar por; perambular}
  \end{Phonetics}
\end{Entry}

\begin{Entry}{走秀}{7,7}{⾛、⽲}
  \begin{Phonetics}{走秀}{zou3xiu4}
    \definition{s.}{desfile de moda}
    \definition{v.}{andar na passarela (em um desfile de moda)}
  \end{Phonetics}
\end{Entry}

\begin{Entry}{走私}{7,7}{⾛、⽲}
  \begin{Phonetics}{走私}{zou3si1}[][HSK 6]
    \definition{v.}{contrabandear; ter um caso ilícito; violar regulamentos alfandegários; evadir inspeções alfandegárias; transportar mercadorias ilegalmente para dentro e para fora do país}
  \end{Phonetics}
\end{Entry}

\begin{Entry}{走进}{7,7}{⾛、⾡}
  \begin{Phonetics}{走进}{zou3 jin4}[][HSK 2]
    \definition{v.}{entrar}
  \end{Phonetics}
\end{Entry}

\begin{Entry}{走势}{7,8}{⾛、⼒}
  \begin{Phonetics}{走势}{zou3shi4}
    \definition{s.}{caminho | tendência}
  \end{Phonetics}
\end{Entry}

\begin{Entry}{走卒}{7,8}{⾛、⼗}
  \begin{Phonetics}{走卒}{zou3zu2}
    \definition{s.}{lacaio (masculino) | peão (isto é, soldado de infantaria) | servo}
  \end{Phonetics}
\end{Entry}

\begin{Entry}{走鬼}{7,9}{⾛、⿁}
  \begin{Phonetics}{走鬼}{zou3gui3}
    \definition{s.}{vendedor ambulante sem licença}
  \end{Phonetics}
\end{Entry}

\begin{Entry}{走索}{7,10}{⾛、⽷}
  \begin{Phonetics}{走索}{zou3suo3}
    \definition{v.}{Acrobacia: andar na corda bamba; dança da corda; caminhada na corda}
  \seealsoref{走绳}{zou3sheng2}
  \end{Phonetics}
\end{Entry}

\begin{Entry}{走绳}{7,11}{⾛、⽷}
  \begin{Phonetics}{走绳}{zou3sheng2}
    \definition{v.}{andar na corda bamba; dança da corda; caminhada na corda; um tipo de acrobacia em que atores andam para frente e para trás em uma corda suspensa e realizam diversas ações}
  \seealsoref{走索}{zou3suo3}
  \end{Phonetics}
\end{Entry}

\begin{Entry}{走路}{7,13}{⾛、⾜}
  \begin{Phonetics}{走路}{zou3 lu4}[][HSK 1]
    \definition{v.}{caminhar; ir a pé; andar em pé sobre a terra | sair; ir embora; partir}
  \end{Phonetics}
\end{Entry}

\begin{Entry}{足}{7}{⾜}[Kangxi 157]
  \begin{Phonetics}{足}{ju4}
    \definition{adj.}{excessivo}
  \end{Phonetics}
  \begin{Phonetics}{足}{zu2}[][HSK 6]
    \definition{adj.}{amplo; bastante; suficiente}
    \definition{adv.}{totalmente; tanto quanto; indica suficiente até certo ponto ou grau | (geralmente no negativo) o suficiente; suficientemente; é totalmente possível; vale a pena}
    \definition{s.}{pé; um termo geral para os membros inferiores do corpo humano; especificamente a parte abaixo do tornozelo | futebol; refere-se ao futebol ou a um time de futebol | pé (de utensílios); a parte inferior do objeto tem o formato de uma pé e serve de suporte}
  \end{Phonetics}
\end{Entry}

\begin{Entry}{足以}{7,4}{⾜、⼈}
  \begin{Phonetics}{足以}{zu2yi3}[][HSK 6]
    \definition{adj.}{suficiente; totalmente capaz de; o suficiente para fazer algo}
    \definition{v.}{bastar}
  \end{Phonetics}
\end{Entry}

\begin{Entry}{足月}{7,4}{⾜、⽉}
  \begin{Phonetics}{足月}{zu2yue4}
    \definition{s.}{gestação completa}
  \end{Phonetics}
\end{Entry}

\begin{Entry}{足足}{7,7}{⾜、⾜}
  \begin{Phonetics}{足足}{zu2zu2}
    \definition{adv.}{tanto quanto | extremamente | completamente | não menos que}
  \end{Phonetics}
\end{Entry}

\begin{Entry}{足够}{7,11}{⾜、⼣}
  \begin{Phonetics}{足够}{zu2 gou4}[][HSK 3]
    \definition{adj.}{bastante; amplo; suficiente; atingir o nível adequado ou capaz de satisfazer as necessidades}
    \definition{v.}{satisfazer; ser suficiente; estar a contento}
  \end{Phonetics}
\end{Entry}

\begin{Entry}{足球}{7,11}{⾜、⽟}
  \begin{Phonetics}{足球}{zu2qiu2}[][HSK 3]
    \definition[个,只,颗,袋]{s.}{futebol | bola de futebol}
  \end{Phonetics}
\end{Entry}

\begin{Entry}{足球队}{7,11,4}{⾜、⽟、⾩}
  \begin{Phonetics}{足球队}{zu2qiu2 dui4}
    \definition[个,支]{s.}{time de futebol}
  \end{Phonetics}
\end{Entry}

\begin{Entry}{足球协会}{7,11,6,6}{⾜、⽟、⼗、⼈}
  \begin{Phonetics}{足球协会}{zu2qiu2xie2hui4}
    \definition*{s.}{Associação de Futebol}
  \end{Phonetics}
\end{Entry}

\begin{Entry}{足球场}{7,11,6}{⾜、⽟、⼟}
  \begin{Phonetics}{足球场}{zu2qiu2chang3}
    \definition{s.}{campo de futebol}
  \end{Phonetics}
\end{Entry}

\begin{Entry}{足球迷}{7,11,9}{⾜、⽟、⾡}
  \begin{Phonetics}{足球迷}{zu2qiu2mi2}
    \definition{s.}{fã (ou entusiasta) de futebol}
  \end{Phonetics}
\end{Entry}

\begin{Entry}{足球赛}{7,11,14}{⾜、⽟、⾙}
  \begin{Phonetics}{足球赛}{zu2qiu2sai4}
    \definition{s.}{competição de futebol | partida de futebol}
  \end{Phonetics}
\end{Entry}

\begin{Entry}{身}{7}{⾝}[Kangxi 158]
  \begin{Phonetics}{身}{shen1}
    \definition*{s.}{Sobrenome Shen}
    \definition{adv.}{eu mesmo; a si mesmo; pessoalmente}
    \definition{s.}{corpo humano ou animal | vida | o caráter moral e a conduta de alguém; cultivo moral | corpo; a parte principal de uma estrutura; o corpo principal ou tronco de um objeto |  uma vida inteira; a vida inteira de alguém | \emph{status} social; identidade}
  \end{Phonetics}
\end{Entry}

\begin{Entry}{身上}{7,3}{⾝、⼀}
  \begin{Phonetics}{身上}{shen1 shang5}[][HSK 1]
    \definition{s.}{no corpo de alguém | em um;  com um}
  \end{Phonetics}
\end{Entry}

\begin{Entry}{身亡}{7,3}{⾝、⼇}
  \begin{Phonetics}{身亡}{shen1wang2}
    \definition{v.}{morrer}
  \end{Phonetics}
\end{Entry}

\begin{Entry}{身边}{7,5}{⾝、⾡}
  \begin{Phonetics}{身边}{shen1 bian1}[][HSK 2]
    \definition{adv.}{ao redor; ao lado de alguém; perto do corpo | carregar consigo (transportar); à mão}
  \end{Phonetics}
\end{Entry}

\begin{Entry}{身份}{7,6}{⾝、⼈}
  \begin{Phonetics}{身份}{shen1fen4}[][HSK 4]
    \definition[种]{s.}{status; capacidade; identidade; refere-se à origem, ao status e às qualificações de uma pessoa | dignidade; posição honrada; referência especial ao status respeitável}
  \end{Phonetics}
\end{Entry}

\begin{Entry}{身份证}{7,6,7}{⾝、⼈、⾔}
  \begin{Phonetics}{身份证}{shen1 fen4 zheng4}[][HSK 3]
    \definition[张]{s.}{ID; bilhete de identidade; carteira de identidade}
  \end{Phonetics}
\end{Entry}

\begin{Entry}{身体}{7,7}{⾝、⼈}
  \begin{Phonetics}{身体}{shen1ti3}[][HSK 1]
    \definition[具,个]{s.}{corpo | saúde; saúde das pessoas}
  \end{Phonetics}
\end{Entry}

\begin{Entry}{身体乳}{7,7,8}{⾝、⼈、⼄}
  \begin{Phonetics}{身体乳}{shen1ti3 ru3}
    \definition{s.}{loção corporal}
  \end{Phonetics}
\end{Entry}

\begin{Entry}{身体能力}{7,7,10,2}{⾝、⼈、⾁、⼒}
  \begin{Phonetics}{身体能力}{shen1ti3 neng2li4}
    \definition{s.}{habilidade física}
  \end{Phonetics}
\end{Entry}

\begin{Entry}{身材}{7,7}{⾝、⽊}
  \begin{Phonetics}{身材}{shen1cai2}[][HSK 4]
    \definition[种,个,具]{s.}{figura; estatura; altura e peso corporal}
  \end{Phonetics}
\end{Entry}

\begin{Entry}{身高}{7,10}{⾝、⾼}
  \begin{Phonetics}{身高}{shen1 gao1}[][HSK 4]
    \definition[个,种,段]{s.}{estatura; altura (de uma pessoa)}
  \end{Phonetics}
\end{Entry}

\begin{Entry}{辛}{7}{⾟}[Kangxi 160]
  \begin{Phonetics}{辛}{xin1}
    \definition*{s.}{Sobrenome Xin}
    \definition{adj.}{quente (no sabor, etc.); pungente | difícil; trabalhoso | ponto da bússola chinesa antiga: 285° | oitavo na ordem}
    \definition{pref.}{octa-}
    \definition{s.}{sofrimento}
    \definition{s.}{oitavo dos dez caules celestiais}
  \end{Phonetics}
\end{Entry}

\begin{Entry}{辛苦}{7,8}{⾟、⾋}
  \begin{Phonetics}{辛苦}{xin1ku3}[][HSK 5]
    \definition{adj.}{difícil; trabalhoso; árduo; descreve muito trabalho, alta intensidade e pouco descanso}
    \definition{s.}{dificuldades}
    \definition{v.}{trabalhar duro; passar por grandes dificuldades; passar por dificuldades}
  \end{Phonetics}
\end{Entry}

\begin{Entry}{迎}{7}{⾡}
  \begin{Phonetics}{迎}{ying2}
    \definition{v.}{ir ao encontro; cumprimentar; acolher; receber | mover-se em direção a; encontrar-se cara a cara}
  \end{Phonetics}
\end{Entry}

\begin{Entry}{迎来}{7,7}{⾡、⽊}
  \begin{Phonetics}{迎来}{ying2 lai2}[][HSK 6]
    \definition{v.}{dar boas-vindas; cumprimentar | introduzir}
  \end{Phonetics}
\end{Entry}

\begin{Entry}{迎接}{7,11}{⾡、⼿}
  \begin{Phonetics}{迎接}{ying2jie1}[][HSK 3]
    \definition{v.}{conhecer; cumprimentar; felicitar; dar as boas-vindas | cumprimentar; felicitar; dar as boas-vindas; preparar-se; aguardar a chegada de um determinado momento ou evento}
  \end{Phonetics}
\end{Entry}

\begin{Entry}{运}{7}{⾡}
  \begin{Phonetics}{运}{yun4}[][HSK 5]
    \definition*{s.}{Sobrenome Yun}
    \definition{s.}{sorte; destino; fortuna}
    \definition{v.}{mover; deslocar | transportar; levar | usar; empunhar; utilizar}
  \end{Phonetics}
\end{Entry}

\begin{Entry}{运气}{7,4}{⾡、⽓}
  \begin{Phonetics}{运气}{yun4/qi4}
    \definition{v.+compl.}{tentar a sorte | concentrar a energia em uma parte do corpo}[他们地运一口气。===Eles respiraram fundo.]
  \end{Phonetics}
  \begin{Phonetics}{运气}{yun4qi5}[][HSK 4]
    \definition{adj.}{sortudo; afortunado}
    \definition{s.}{sorte; fortuna}
  \end{Phonetics}
\end{Entry}

\begin{Entry}{运用}{7,5}{⾡、⽤}
  \begin{Phonetics}{运用}{yun4yong4}[][HSK 4]
    \definition{v.}{usar; utilizar; manejar; aplicar; explorar as coisas de acordo com suas características}
  \end{Phonetics}
\end{Entry}

\begin{Entry}{运动}{7,6}{⾡、⼒}
  \begin{Phonetics}{运动}{yun4dong4}[][HSK 2]
    \definition[项,种,场,次]{s.}{esportes; atletismo; exercício; atividades esportivas | movimento; campanha (política); atividades de massa organizadas, intencionais e de alto nível na política, cultura, produção, etc. | movimento; refere-se a todas as mudanças}
    \definition{v.}{exercitar; fazer atividade física | mover-se; refere-se à mudança na posição de um objeto}
  \end{Phonetics}
\end{Entry}

\begin{Entry}{运动会}{7,6,6}{⾡、⼒、⼈}
  \begin{Phonetics}{运动会}{yun4 dong4 hui4}[][HSK 4]
    \definition[届,场,次,个]{s.}{jogos; encontro esportivo; dia de esportes; encontro atlético; competição esportiva abrangente}
  \end{Phonetics}
\end{Entry}

\begin{Entry}{运动场}{7,6,6}{⾡、⼒、⼟}
  \begin{Phonetics}{运动场}{yun4dong4chang3}
    \definition{s.}{campo desportivo | campo de jogos}
  \end{Phonetics}
\end{Entry}

\begin{Entry}{运动员}{7,6,7}{⾡、⼒、⼝}
  \begin{Phonetics}{运动员}{yun4 dong4 yuan2}[][HSK 4]
    \definition[名,个,班]{s.}{jogador; atleta; esportista; pessoas que participam de competições esportivas}
  \end{Phonetics}
\end{Entry}

\begin{Entry}{运动学}{7,6,8}{⾡、⼒、⼦}
  \begin{Phonetics}{运动学}{yun4dong4xue2}
    \definition{s.}{cinemática; um ramo da ciência do esporte que usa a anatomia e a mecânica humanas para explicar várias atividades esportivas}
  \end{Phonetics}
\end{Entry}

\begin{Entry}{运动服}{7,6,8}{⾡、⼒、⽉}
  \begin{Phonetics}{运动服}{yun4dong4fu2}
    \definition{s.}{roupa para prática de esporte}
  \end{Phonetics}
\end{Entry}

\begin{Entry}{运动衫}{7,6,8}{⾡、⼒、⾐}
  \begin{Phonetics}{运动衫}{yun4dong4shan1}
    \definition[件]{s.}{moletom | camisa esportiva}
  \end{Phonetics}
\end{Entry}

\begin{Entry}{运动家}{7,6,10}{⾡、⼒、⼧}
  \begin{Phonetics}{运动家}{yun4dong4jia1}
    \definition{s.}{ativista | atleta | esportista}
  \end{Phonetics}
\end{Entry}

\begin{Entry}{运动病}{7,6,10}{⾡、⼒、⽧}
  \begin{Phonetics}{运动病}{yun4dong4bing4}
    \definition{s.}{enjôo (movimento, carro, etc.)}
  \end{Phonetics}
\end{Entry}

\begin{Entry}{运动鞋}{7,6,15}{⾡、⼒、⾰}
  \begin{Phonetics}{运动鞋}{yun4dong4xie2}
    \definition{s.}{tênis | sapatos esportivos}
  \end{Phonetics}
\end{Entry}

\begin{Entry}{运行}{7,6}{⾡、⾏}
  \begin{Phonetics}{运行}{yun4xing2}[][HSK 5]
    \definition{v.}{correr; mover; trabalhar; estar em movimento; (veículo, nave, planeta, etc.) mover-se em um ciclo repetitivo; avançar de maneira regular e direcional}
  \end{Phonetics}
\end{Entry}

\begin{Entry}{运作}{7,7}{⾡、⼈}
  \begin{Phonetics}{运作}{yun4 zuo4}[][HSK 6]
    \definition{v.}{trabalhar; operar; (uma instituição, organização, etc.) realizar trabalho; realizar atividades}
  \end{Phonetics}
\end{Entry}

\begin{Entry}{运河}{7,8}{⾡、⽔}
  \begin{Phonetics}{运河}{yun4he2}
    \definition{s.}{canal (em um rio)}
  \end{Phonetics}
\end{Entry}

\begin{Entry}{运输}{7,13}{⾡、⾞}
  \begin{Phonetics}{运输}{yun4shu1}[][HSK 3]
    \definition{v.}{enviar; transportar; transportar pessoas ou coisas de um lugar para outro usando carros, barcos, aviões, etc.}
  \end{Phonetics}
\end{Entry}

\begin{Entry}{近}{7}{⾡}
  \begin{Phonetics}{近}{jin4}[][HSK 2]
    \definition{adj.}{próximo; perto; distância espacial ou temporal curta (oposto de 远) | íntimo; intimamente relacionado; relação estreita | fácil de entender}
  \seealsoref{远}{yuan3}
  \end{Phonetics}
\end{Entry}

\begin{Entry}{近日}{7,4}{⾡、⽇}
  \begin{Phonetics}{近日}{jin4 ri4}[][HSK 6]
    \definition{s.}{recentemente; nos últimos dias; apontando para o passado | nos próximos dias; refere-se ao futuro}
  \end{Phonetics}
\end{Entry}

\begin{Entry}{近代}{7,5}{⾡、⼈}
  \begin{Phonetics}{近代}{jin4dai4}[][HSK 4]
    \definition{s.}{tempos modernos; era passada relativamente próxima à era moderna, geralmente referida na história chinesa como 1840 a 1919 | o tempo ou era do capitalismo}
  \end{Phonetics}
\end{Entry}

\begin{Entry}{近来}{7,7}{⾡、⽊}
  \begin{Phonetics}{近来}{jin4lai2}[][HSK 5]
    \definition{adv.}{ultimamente; recentemente; de ​​tarde; refere-se a um período de tempo entre o passado imediato e o presente}
  \end{Phonetics}
\end{Entry}

\begin{Entry}{近视}{7,8}{⾡、⾒}
  \begin{Phonetics}{近视}{jin4 shi4}[][HSK 6]
    \definition{adj.}{miopia; uma deficiência visual em que a visão próxima é clara, mas a visão distante é turva | míope (figurativo); metáfora para miopia}
  \end{Phonetics}
\end{Entry}

\begin{Entry}{近期}{7,12}{⾡、⽉}
  \begin{Phonetics}{近期}{jin4 qi1}[][HSK 3]
    \definition{adv.}{num futuro próximo; brevemente}
  \end{Phonetics}
\end{Entry}

\begin{Entry}{返}{7}{⾡}
  \begin{Phonetics}{返}{fan3}
    \definition{v.}{retornar; vir ou voltar}
  \end{Phonetics}
\end{Entry}

\begin{Entry}{返回}{7,6}{⾡、⼞}
  \begin{Phonetics}{返回}{fan3 hui2}[][HSK 5]
    \definition{v.}{retornar; ir (voltar); reverter; recorrer; retroceder; voltar para (o lugar original)}
  \end{Phonetics}
\end{Entry}

\begin{Entry}{返还}{7,7}{⾡、⾡}
  \begin{Phonetics}{返还}{fan3huan2}[][HSK 7-9]
    \definition{s.}{remessa | restituição | devolução de algo ao seu dono original}
  \end{Phonetics}
\end{Entry}

\begin{Entry}{还}{7}{⾡}
  \begin{Phonetics}{还}{hai2}[][HSK 1]
    \definition{adv.}{ainda; indica que a ação ou estado permanece inalterado, equivalente a 仍然 | também; além disso; em adição; indica que há um aumento ou suplemento além do escopo já indicado | ainda mais; usado com 比 para indicar que as características e o grau das coisas comparadas aumentaram, o que é equivalente a 更加 razoavelmente; medianamente; usado antes de um adjetivo, indica que algo atinge apenas o nível mínimo exigido | mesmo; usado na primeira parte da frase como complemento, e na segunda parte como conclusão, equivalente a 尚且 | que expressa realização ou descoberta; expressa surpresa por algo que não se esperava, mas que acabou acontecendo | tão cedo quanto; por um curto período de tempo; indica que já era assim há muito tempo | para dar ênfase; para reforçar o tom}
  \seealsoref{比}{bi3}
  \seealsoref{更加}{geng4 jia1}
  \seealsoref{仍然}{reng2ran2}
  \seealsoref{尚且}{shang4 qie3}
  \end{Phonetics}
  \begin{Phonetics}{还}{huan2}[][HSK 1]
    \definition*{s.}{Sobrenome Huan}
    \definition{v.}{voltar; retornar; voltar ao lugar original ou restaurar o estado original | retribuir; devolver; reembolsar; devolver o dinheiro ou os bens emprestados ao seu proprietário | dar ou fazer algo em troca; retribuir as ações dos outros}
  \end{Phonetics}
\end{Entry}

\begin{Entry}{还有}{7,6}{⾡、⽉}
  \begin{Phonetics}{还有}{hai2 you3}[][HSK 1]
    \definition{adv.}{também; ainda; além disso; então novamente; enfatizar as partes complementares, excedentes ou não mencionadas além do que já é conhecido}
  \end{Phonetics}
\end{Entry}

\begin{Entry}{还是}{7,9}{⾡、⽇}
  \begin{Phonetics}{还是}{hai2shi5}[][HSK 1]
    \definition{adv.}{ainda; ainda assim; não é a continuação de um determinado estado, fenômeno ou ação; o resultado é o mesmo de antes, sem mudanças  |que expressa uma preferência por uma alternativa; expressa comparação ou escolha feita após consideração cuidadosa, frequentemente usado para fazer sugestões | que expressa realização ou descoberta; indica que o resultado final foi inesperado}
    \definition{conj.}{ou (somente para frases interrogativas); indica várias opções, geralmente usado em perguntas | tudo; se; não importa; independentemente de; significa que, independentemente das mudanças que ocorram, o resultado permanecerá o mesmo}
  \end{Phonetics}
\end{Entry}

\begin{Entry}{这}{7}{⾡}
  \begin{Phonetics}{这}{zhe4}[][HSK 1]
    \definition{pron.}{este, isto; substitui pessoas ou coisas que estão mais próximas | agora; em vez de 这时候, tem o efeito de reforçar a ênfase}
  \seealsoref{这时候}{zhe4 shi2 hou5}
  \end{Phonetics}
  \begin{Phonetics}{这}{zhei4}
    \definition{pron.}{(coloquial) este}
  \end{Phonetics}
\end{Entry}

\begin{Entry}{这儿}{7,2}{⾡、⼉}
  \begin{Phonetics}{这儿}{zhe4r5}[][HSK 1]
    \definition{pron.}{aqui | agora; neste momento (utilizado apenas após 打, 从, 由)}
  \seealsoref{从}{cong2}
  \seealsoref{打}{da3}
  \seealsoref{由}{you2}
  \end{Phonetics}
\end{Entry}

\begin{Entry}{这个}{7,3}{⾡、⼈}
  \begin{Phonetics}{这个}{zhe4ge5}
    \definition{pron.}{isto; este | isso; em vez das coisas mencionadas anteriormente | assim; tal; usado antes de verbos e adjetivos, indica um grau muito profundo, com um sentido exagerado | usado junto com 那个 para indicar pessoas ou objetos indefinidos}
  \seealsoref{那个}{na4ge5}
  \end{Phonetics}
\end{Entry}

\begin{Entry}{这么}{7,3}{⾡、⼃}
  \begin{Phonetics}{这么}{zhe4 me5}[][HSK 2]
    \definition{pron.}{tal (usado para mostrar o grau) | então (usado para mostrar exagero e exclamação) | desta forma; assim; formas de expressar ações | tal; indica quantidade}
  \end{Phonetics}
\end{Entry}

\begin{Entry}{这边}{7,5}{⾡、⾡}
  \begin{Phonetics}{这边}{zhe4 bian1}[][HSK 1]
    \definition{pron.}{aqui; deste lado; refere-se a um lugar próximo}
  \end{Phonetics}
\end{Entry}

\begin{Entry}{这会儿}{7,6,2}{⾡、⼈、⼉}
  \begin{Phonetics}{这会儿}{zhe4 hui4r5}
    \definition{adv./pron./s.}{agora; no momento; no presente}
  \end{Phonetics}
\end{Entry}

\begin{Entry}{这时}{7,7}{⾡、⽇}
  \begin{Phonetics}{这时}{zhe4 shi2}[][HSK 2]
    \definition{adv.}{neste momento}
  \end{Phonetics}
\end{Entry}

\begin{Entry}{这时候}{7,7,10}{⾡、⽇、⼈}
  \begin{Phonetics}{这时候}{zhe4 shi2 hou5}[][HSK 2]
    \definition{adv.}{neste momento}
  \end{Phonetics}
\end{Entry}

\begin{Entry}{这里}{7,7}{⾡、⾥}
  \begin{Phonetics}{这里}{zhe4 li3}[][HSK 1]
    \definition{pron.}{aqui; pronomes demonstrativo, indicando locais próximos}
  \end{Phonetics}
\end{Entry}

\begin{Entry}{这些}{7,8}{⾡、⼆}
  \begin{Phonetics}{这些}{zhe4 xie1}[][HSK 1]
    \definition{pron.}{estes; pronome demonstrativo, que indicam duas ou mais pessoas ou coisas que estão próximas}
  \end{Phonetics}
\end{Entry}

\begin{Entry}{这咱}{7,9}{⾡、⼝}
  \begin{Phonetics}{这咱}{zhe4 zan5}
    \definition{s.}{agora; no momento; no presente | neste momento}
  \end{Phonetics}
\end{Entry}

\begin{Entry}{这样}{7,10}{⾡、⽊}
  \begin{Phonetics}{这样}{zhe4 yang4}[][HSK 2]
    \definition{pron.}{assim; tal; assim; deste jeito; pronome demonstrativo, que indica a natureza, estado, maneira, grau, etc.}
  \end{Phonetics}
\end{Entry}

\begin{Entry}{这就是说}{7,12,9,9}{⾡、⼪、⽇、⾔}
  \begin{Phonetics}{这就是说}{zhe4 jiu4 shi4 shuo1}[][HSK 6]
    \definition{expr.}{isto significa que; isto é dizer}
  \end{Phonetics}
\end{Entry}

\begin{Entry}{这麽}{7,14}{⾡、⿇}
  \begin{Phonetics}{这麽}{zhe4 me5}
    \variantof{这么}
  \end{Phonetics}
\end{Entry}

\begin{Entry}{进}{7}{⾡}
  \begin{Phonetics}{进}{jin4}[][HSK 1]
    \definition*{s.}{Sobrenome Jin}
    \definition{clas.}{para seções em um edifício ou complexo residencial; qualquer uma das várias fileiras de casas em um complexo residencial de estilo antigo}
    \definition{s.}{(matemática) base de um sistema numérico}
    \definition{v.}{avançar; ir adiante; seguir em frente; (oposto a 退) | entrar; entrar em; entrar ou sair; (oposto a 出) | receber | comer; tomar; beber | submeter; apresentar | marcar um gol}
    \definition{v.aux.}{usado após um verbo, significa ``para dentro''}
  \seealsoref{出}{chu1}
  \seealsoref{退}{tui4}
  \end{Phonetics}
\end{Entry}

\begin{Entry}{进一步}{7,1,7}{⾡、⼀、⽌}
  \begin{Phonetics}{进一步}{jin4 yi2 bu4}[][HSK 3]
    \definition{adv.}{mais; dar um passo adiante; avançar um passo; indica que as coisas estão progredindo em um nível mais alto do que antes}
  \end{Phonetics}
\end{Entry}

\begin{Entry}{进入}{7,2}{⾡、⼊}
  \begin{Phonetics}{进入}{jin4 ru4}[][HSK 2]
    \definition{v.}{entrar; entrar em}
  \end{Phonetics}
\end{Entry}

\begin{Entry}{进口}{7,3}{⾡、⼝}
  \begin{Phonetics}{进口}{jin4/kou3}[][HSK 4]
    \definition{adj.}{importado}
    \definition{s.}{importação; entrada de um edifício ou local, também chamada de 入口}
    \definition{v.+compl.}{importar; comprar ou transportar mercadorias de outro país ou região | entrar no porto; navegar em direção a um porto}
  \seealsoref{入口}{ru4kou3}
  \end{Phonetics}
\end{Entry}

\begin{Entry}{进化}{7,4}{⾡、⼔}
  \begin{Phonetics}{进化}{jin4hua4}[][HSK 5]
    \definition[个]{s.}{evolução; os organismos se desenvolvem e evoluem do simples para o complexo e de níveis baixos para altos}
    \definition{v.}{evoluir; um termo geral usado para descrever uma mudança gradual para melhor}
  \end{Phonetics}
\end{Entry}

\begin{Entry}{进出口}{7,5,3}{⾡、⼐、⼝}
  \begin{Phonetics}{进出口}{jin4chu1kou3}
    \definition{s.}{importação e exportação}
    \definition{v.}{importar e exportar}
  \end{Phonetics}
\end{Entry}

\begin{Entry}{进去}{7,5}{⾡、⼛}
  \begin{Phonetics}{进去}{jin4 qu4}[][HSK 1]
    \definition{v.}{entrar (a partir da minha localização)}
    \definition{v.aux.}{usado depois de um verbo, significa ``ir para dentro''; para um determinado intervalo ou período de tempo}
  \end{Phonetics}
\end{Entry}

\begin{Entry}{进行}{7,6}{⾡、⾏}
  \begin{Phonetics}{进行}{jin4xing2}[][HSK 2]
    \definition{v.}{continuar; estar em andamento; estar em progresso | fazer; conduzir; realizar; executar | marchar; avançar; prosseguir; estar em marcha}
  \end{Phonetics}
\end{Entry}

\begin{Entry}{进行编程}{7,6,12,12}{⾡、⾏、⽷、⽲}
  \begin{Phonetics}{进行编程}{jin4xing2bian1cheng2}
    \definition{s.}{programa de computador executável}
  \end{Phonetics}
\end{Entry}

\begin{Entry}{进攻}{7,7}{⾡、⽁}
  \begin{Phonetics}{进攻}{jin4gong1}[][HSK 6]
    \definition{s.}{ofensiva}
    \definition{v.}{atacar; assaltar; tomar a ofensiva (oposto à 防守)}
  \seealsoref{防守}{fang2shou3}
  \end{Phonetics}
\end{Entry}

\begin{Entry}{进来}{7,7}{⾡、⽊}
  \begin{Phonetics}{进来}{jin4 lai2}[][HSK 1]
    \definition{v.}{entrar (para a minha localização)}
  \end{Phonetics}
\end{Entry}

\begin{Entry}{进步}{7,7}{⾡、⽌}
  \begin{Phonetics}{进步}{jin4bu4}[][HSK 3]
    \definition{adj.}{progressivo; adequado às tendências da época; que impulsiona o desenvolvimento social (em oposição a 落后)}
    \definition{v.}{avançar; progredir; melhorar}
  \seealsoref{落后}{luo4hou4}
  \end{Phonetics}
\end{Entry}

\begin{Entry}{进展}{7,10}{⾡、⼫}
  \begin{Phonetics}{进展}{jin4zhan3}[][HSK 3]
    \definition{v.}{fazer progresso; progredir; avançar no desenvolvimento}
  \end{Phonetics}
\end{Entry}

\begin{Entry}{远}{7}{⾡}
  \begin{Phonetics}{远}{yuan3}[][HSK 1]
    \definition*{s.}{Sobrenome Yuan}
    \definition{adj.}{distante (no tempo ou no espaço); longe; remoto; Longa distância espacial ou temporal (em oposição a 近) | (relações de parentesco) distante | com grande diferença}
    \definition{v.}{manter-se afastado de; não se aproximar}
  \seealsoref{近}{jin4}
  \end{Phonetics}
\end{Entry}

\begin{Entry}{远天}{7,4}{⾡、⼤}
  \begin{Phonetics}{远天}{yuan3tian1}
    \definition{s.}{paraíso | o céu distante}
  \end{Phonetics}
\end{Entry}

\begin{Entry}{远方}{7,4}{⾡、⽅}
  \begin{Phonetics}{远方}{yuan3 fang1}[][HSK 6]
    \definition{s.}{distância; de longe; lugar distante}
  \end{Phonetics}
\end{Entry}

\begin{Entry}{远处}{7,5}{⾡、⼡}
  \begin{Phonetics}{远处}{yuan3 chu4}[][HSK 5]
    \definition{s.}{distância; lugar distante}
  \end{Phonetics}
\end{Entry}

\begin{Entry}{远远}{7,7}{⾡、⾡}
  \begin{Phonetics}{远远}{yuan3 yuan3}[][HSK 6]
    \definition{adv.}{de longe; em grande medida; para descrever um alto grau ou uma grande quantidade}
  \end{Phonetics}
\end{Entry}

\begin{Entry}{远征}{7,8}{⾡、⼻}
  \begin{Phonetics}{远征}{yuan3zheng1}
    \definition{s.}{uma expedição militar | marcha para regiões remotas}
  \end{Phonetics}
\end{Entry}

\begin{Entry}{远离}{7,10}{⾡、⼇}
  \begin{Phonetics}{远离}{yuan3 li2}[][HSK 6]
    \definition{adj.}{afastado; distante}
    \definition{v.}{partir para; ficar longe}
  \end{Phonetics}
\end{Entry}

\begin{Entry}{违}{7}{⾡}
  \begin{Phonetics}{违}{wei2}
    \definition{v.}{desobedecer; violar | ser separado; separar-se de | desafiar; não cumprir; não obedecer}
  \end{Phonetics}
\end{Entry}

\begin{Entry}{违反}{7,4}{⾡、⼜}
  \begin{Phonetics}{违反}{wei2fan3}[][HSK 5]
    \definition{v.}{violar; transgredir; contrariar; não estar em conformidade (com as regras, regulamentos, etc.)}
  \end{Phonetics}
\end{Entry}

\begin{Entry}{违法}{7,8}{⾡、⽔}
  \begin{Phonetics}{违法}{wei2 fa3}[][HSK 5]
    \definition{v.}{ser ilegal; infringir a lei; violar a lei ou os regulamentos}
  \end{Phonetics}
\end{Entry}

\begin{Entry}{违规}{7,8}{⾡、⾒}
  \begin{Phonetics}{违规}{wei2 gui1}[][HSK 5]
    \definition{v.}{violar (regras); infringir as regras e regulamentos}
  \end{Phonetics}
\end{Entry}

\begin{Entry}{违宪}{7,9}{⾡、⼧}
  \begin{Phonetics}{违宪}{wei2xian4}
    \definition{adj.}{inconstitucional}
  \end{Phonetics}
\end{Entry}

\begin{Entry}{连}{7}{⾡}
  \begin{Phonetics}{连}{lian2}[][HSK 3]
    \definition*{s.}{Sobrenome Lian}
    \definition{adv.}{em sucessão; um após o outro; repetidamente}
    \definition{prep.}{incluindo; incluido | até mesmo}
    \definition[个]{s.}{companhia; unidades organizacionais das forças armadas}
    \definition{v.}{ligar; juntar; conectar | envolver-se (em problemas); implicar; incriminar | costurar; coser}
  \end{Phonetics}
\end{Entry}

\begin{Entry}{连忙}{7,6}{⾡、⼼}
  \begin{Phonetics}{连忙}{lian2mang2}[][HSK 3]
    \definition{adv.}{imediatamente; de imediato; com pressa; apressadamente}
  \end{Phonetics}
\end{Entry}

\begin{Entry}{连接}{7,11}{⾡、⼿}
  \begin{Phonetics}{连接}{lian2 jie1}[][HSK 5]
    \definition[条]{s.}{conexão}
    \definition{v.}{ligar; unir; relacionar, conectar; anexar}
  \end{Phonetics}
\end{Entry}

\begin{Entry}{连续}{7,11}{⾡、⽷}
  \begin{Phonetics}{连续}{lian2xu4}[][HSK 3]
    \definition{adv.}{continuamente; sucessivamente; em uma fileira; um após o outro}
  \end{Phonetics}
\end{Entry}

\begin{Entry}{连续剧}{7,11,10}{⾡、⽷、⼑}
  \begin{Phonetics}{连续剧}{lian2 xu4 ju4}[][HSK 3]
    \definition[部,集]{s.}{série; novela; drama dividido em vários episódios, transmitido continuamente pela rádio ou televisão, com enredo contínuo}
  \end{Phonetics}
\end{Entry}

\begin{Entry}{连锁反应}{7,12,4,7}{⾡、⾦、⼜、⼴}
  \begin{Phonetics}{连锁反应}{lian2suo3fan3ying4}
    \definition{s.}{reação em cadeia}
  \end{Phonetics}
\end{Entry}

\begin{Entry}{迟}{7}{⾡}
  \begin{Phonetics}{迟}{chi2}[][HSK 5]
    \definition*{s.}{Sobrenome Chi}
    \definition{adj.}{lento; tardio; demorado | atrasado | lento; obtuso}
  \end{Phonetics}
\end{Entry}

\begin{Entry}{迟早}{7,6}{⾡、⽇}
  \begin{Phonetics}{迟早}{chi2zao3}[][HSK 7-9]
    \definition{adv.}{mais cedo ou mais tarde; cedo ou tarde}[我们迟早会成功的。===Teremos sucesso mais cedo ou mais tarde.]
  \end{Phonetics}
\end{Entry}

\begin{Entry}{迟迟}{7,7}{⾡、⾡}
  \begin{Phonetics}{迟迟}{chi2chi2}[][HSK 7-9]
    \definition{adv.}{lentamente; tardiamente}
  \end{Phonetics}
\end{Entry}

\begin{Entry}{迟到}{7,8}{⾡、⼑}
  \begin{Phonetics}{迟到}{chi2dao4}[][HSK 4]
    \definition{v.}{chegar atrasado; atrasar-se; chegar depois do horário estipulado, geralmente usado para aulas, reuniões ou encontros em horário combinado, etc.}
  \end{Phonetics}
\end{Entry}

\begin{Entry}{迟疑}{7,14}{⾡、⽦}
  \begin{Phonetics}{迟疑}{chi2yi2}[][HSK 7-9]
    \definition{v.}{hesitar; titubear}
  \end{Phonetics}
\end{Entry}

\begin{Entry}{邮}{7}{⾢}
  \begin{Phonetics}{邮}{you2}
    \definition*{s.}{Sobrenome You}
    \definition{s.}{postal; correio; refere-se a serviços postais | agência dos correios}
    \definition{v.}{postar; enviar pelo correio}
  \end{Phonetics}
\end{Entry}

\begin{Entry}{邮包}{7,5}{⾢、⼓}
  \begin{Phonetics}{邮包}{you2bao1}
    \definition{s.}{encomenda postal}
  \end{Phonetics}
\end{Entry}

\begin{Entry}{邮市}{7,5}{⾢、⼱}
  \begin{Phonetics}{邮市}{you2shi4}
    \definition{s.}{mercado postal}
  \end{Phonetics}
\end{Entry}

\begin{Entry}{邮电}{7,5}{⾢、⽥}
  \begin{Phonetics}{邮电}{you2dian4}
    \definition*{s.}{Correios e Telecomunicações}
  \end{Phonetics}
\end{Entry}

\begin{Entry}{邮件}{7,6}{⾢、⼈}
  \begin{Phonetics}{邮件}{you2 jian4}[][HSK 3]
    \definition[封,份,个,条]{s.}{correspondência; correio; assunto postal; termo que se refere a cartas, encomendas, etc., recebidos, transportados e entregues pelos correios | \emph{e-mail}; refere-se a e-mails, informações recebidas e enviadas através de caixas de correio eletrônico na \emph{Internet}, etc.}
  \end{Phonetics}
\end{Entry}

\begin{Entry}{邮局}{7,7}{⾢、⼫}
  \begin{Phonetics}{邮局}{you2ju2}[][HSK 4]
    \definition[家,个]{s.}{correio; agência dos correios; organizações que lidam com serviços postais}
  \end{Phonetics}
\end{Entry}

\begin{Entry}{邮费}{7,9}{⾢、⾙}
  \begin{Phonetics}{邮费}{you2fei4}
    \definition{s.}{postagem}
    \definition{v.}{postar}
  \end{Phonetics}
\end{Entry}

\begin{Entry}{邮迷}{7,9}{⾢、⾡}
  \begin{Phonetics}{邮迷}{you2mi2}
    \definition{s.}{filatelista | colecionador de selos}
  \end{Phonetics}
\end{Entry}

\begin{Entry}{邮资}{7,10}{⾢、⾙}
  \begin{Phonetics}{邮资}{you2zi1}
    \definition{s.}{postagem}
  \end{Phonetics}
\end{Entry}

\begin{Entry}{邮递}{7,10}{⾢、⾡}
  \begin{Phonetics}{邮递}{you2di4}
    \definition{v.}{enviar por correio}
  \end{Phonetics}
\end{Entry}

\begin{Entry}{邮票}{7,11}{⾢、⽰}
  \begin{Phonetics}{邮票}{you2 piao4}[][HSK 3]
    \definition[枚,张,套,版]{s.}{selo; selo postal; comprovante vendido pelos correios, usado para colar nas correspondências para indicar que o porte foi pago}
  \end{Phonetics}
\end{Entry}

\begin{Entry}{邮箱}{7,15}{⾢、⾋}
  \begin{Phonetics}{邮箱}{you2 xiang1}[][HSK 3]
    \definition{s.}{caixa de correio | \emph{mailbox}; refere-se ao endereço de \emph{e-mail}}
  \end{Phonetics}
\end{Entry}

\begin{Entry}{邻}{7}{⾢}
  \begin{Phonetics}{邻}{lin2}
    \definition{adj.}{vizinho; perto; adjacente; perto; próximo}
    \definition{s.}{vizinho | bairro; vizinhança}
  \end{Phonetics}
\end{Entry}

\begin{Entry}{邻居}{7,8}{⾢、⼫}
  \begin{Phonetics}{邻居}{lin2ju1}[][HSK 5]
    \definition[个,位,名,家]{s.}{vizinho; pessoas ou famílias que moram muito perto}
  \end{Phonetics}
\end{Entry}

\begin{Entry}{里}{7}{⾥}[Kangxi 166]
  \begin{Phonetics}{里}{li3}[][HSK 1]
    \definition*{s.}{Sobrenome Li}
    \definition{clas.}{li, uma unidade chinesa de comprimento (= 1/2 quilômetro)}
    \definition{s.}{forro; revestimento; interior; parte de trás do tecido | interno; dentro; no interior | vizinhança; vizinhos | cidade natal; local de origem}
  \end{Phonetics}
\end{Entry}

\begin{Entry}{里头}{7,5}{⾥、⼤}
  \begin{Phonetics}{里头}{li3 tou5}[][HSK 2]
    \definition{s.}{dentro}
  \end{Phonetics}
\end{Entry}

\begin{Entry}{里边}{7,5}{⾥、⾡}
  \begin{Phonetics}{里边}{li3 bian5}[][HSK 1]
    \definition{s.}{em; dentro; no interior}
  \end{Phonetics}
\end{Entry}

\begin{Entry}{里面}{7,9}{⾥、⾯}
  \begin{Phonetics}{里面}{li3 mian4}[][HSK 3]
    \definition{s.}{dentro; interior}
  \end{Phonetics}
\end{Entry}

\begin{Entry}{里斯本}{7,12,5}{⾥、⽄、⽊}
  \begin{Phonetics}{里斯本}{li3si1ben3}
    \definition*{s.}{Lisboa}
  \end{Phonetics}
\end{Entry}

\begin{Entry}{里斯本大学}{7,12,5,3,8}{⾥、⽄、⽊、⼤、⼦}
  \begin{Phonetics}{里斯本大学}{li3si1ben3 da4xue2}
    \definition*{s.}{Universidade de Lisboa}
  \end{Phonetics}
\end{Entry}

\begin{Entry}{针}{7}{⾦}
  \begin{Phonetics}{针}{zhen1}[][HSK 4]
    \definition*{s.}{Sobrenome Zhen}
    \definition[根,枚,套,支]{s.}{agulha; ferramentas para costura de roupas | objetos semelhantes a agulhas; algo longo e fino como uma agulha | injeção | ponto de costura | pontos de acupuntura na medicina chinesa}
  \end{Phonetics}
\end{Entry}

\begin{Entry}{针对}{7,5}{⾦、⼨}
  \begin{Phonetics}{针对}{zhen1dui4}[][HSK 4]
    \definition{prep.}{em conexão com; de acordo com; à luz de; introdução de objetos de comportamento com uma finalidade clara}
    \definition{v.}{contrariar; apontar para; ter como objetivo; ser direcionado contra; fazer algo especificamente sobre um problema ou uma pessoa}
  \end{Phonetics}
\end{Entry}

\begin{Entry}{钉}{7}{⾦}
  \begin{Phonetics}{钉}{ding1}
    \definition[根,颗,枚]{s.}{prego; tacha}
    \definition{v.}{seguir de perto | instar; pressionar | olhar fixamente; fixar os olhos em; o mesmo que 盯}
  \seealsoref{盯}{ding1}
  \end{Phonetics}
  \begin{Phonetics}{钉}{ding4}[][HSK 7-9]
    \definition{v.}{pregar; prender com pregos | costurar; costurar com agulha e linha}
  \end{Phonetics}
\end{Entry}

\begin{Entry}{钉子}{7,3}{⾦、⼦}
  \begin{Phonetics}{钉子}{ding1zi5}[][HSK 7-9]
    \definition[颗,根,枚]{s.}{prego; prego de ferro | obstáculo; dificuldade; inconveniente | sabotador}
    \definition{v.}{bater em um obstáculo; encontrar um obstáculo; encontrar uma rejeição (de); correr contra um obstáculo}
  \end{Phonetics}
\end{Entry}

\begin{Entry}{闲}{7}{⾨}
  \begin{Phonetics}{闲}{xian2}[][HSK 5]
    \definition{adj.}{ocioso; não ocupado; desocupado; sem coisas para fazer; sem atividades; tempo livre | desocupado; (casa, objeto, etc.) não em uso; ocioso | não oficial; não sério; não relacionado ao negócio}
    \definition{s.}{lazer; tempo livre}
  \end{Phonetics}
\end{Entry}

\begin{Entry}{间}{7}{⾨}
  \begin{Phonetics}{间}{jian1}[][HSK 1]
    \definition{clas.}{a menor unidade de uma casa; a menor unidade habitacional; cômodo}
    \definition{s.}{espaço entre duas partes  | (em um) tempo ou espaço definido | sala; quarto | uma seção de uma sala ou o espaço lateral entre dois pares de pilares | com um tempo ou espaço definido}
  \end{Phonetics}
  \begin{Phonetics}{间}{jian4}
    \definition{s.}{espaço entre as duas partes; abertura; lacuna}
    \definition{v.}{separar | semear a discórdia | desbastar (mudas); podar; remover ou arrancar as mudas em excesso}
  \end{Phonetics}
\end{Entry}

\begin{Entry}{间或}{7,8}{⾨、⼽}
  \begin{Phonetics}{间或}{jian4huo4}
    \definition{adv.}{às vezes | ocasionalmente | de vez em quando}
  \end{Phonetics}
\end{Entry}

\begin{Entry}{间接}{7,11}{⾨、⼿}
  \begin{Phonetics}{间接}{jian4jie1}[][HSK 5]
    \definition{adj.}{indireto; de segunda mão; em oposição a 直接}
  \seealsoref{直接}{zhi2jie1}
  \end{Phonetics}
\end{Entry}

\begin{Entry}{闷}{7}{⾨}
  \begin{Phonetics}{闷}{men1}
    \definition{adj.}{abafado; fechado; sufocante; baixa pressão de ar ou má circulação de ar | abafado; som baixo ou opaco}
    \definition{v.}{cubrir bem; fazer algo hermético | ficar sem fala; parar de falar | fechar a si mesmo ou alguém dentro de casa; ficar em casa e não sair}
  \end{Phonetics}
  \begin{Phonetics}{闷}{men4}
    \definition{adj.}{entediado; deprimido; irritado; desanimado | hermeticamente fechado; selado | triste e silencioso; chateado | hermético}
    \definition{s.}{desânimo}
  \end{Phonetics}
\end{Entry}

\begin{Entry}{闷热}{7,10}{⾨、⽕}
  \begin{Phonetics}{闷热}{men1re4}
    \definition{adj.}{abafado | quente e abafado | sufocantemente quente | quente e sensual}
  \end{Phonetics}
\end{Entry}

\begin{Entry}{阻}{7}{⾩}
  \begin{Phonetics}{阻}{zu3}
    \definition{v.}{impedir; bloquear; obstruir; atrapalhar}
  \end{Phonetics}
\end{Entry}

\begin{Entry}{阻止}{7,4}{⾩、⽌}
  \begin{Phonetics}{阻止}{zu3zhi3}[][HSK 4]
    \definition{v.}{parar; reter; conter; interromper; impedir o avanço; impedir o movimento; obstruir}
  \end{Phonetics}
\end{Entry}

\begin{Entry}{阻击}{7,5}{⾩、⼐}
  \begin{Phonetics}{阻击}{zu3ji1}
    \definition{v.}{verificar | parar}
  \end{Phonetics}
\end{Entry}

\begin{Entry}{阻碍}{7,13}{⾩、⽯}
  \begin{Phonetics}{阻碍}{zu3'ai4}[][HSK 5]
    \definition{s.}{obstáculo; impedimento; barreira}
    \definition{v.}{bloquear; impedir; obstruir; impedir o bom andamento ou desenvolvimento}
  \end{Phonetics}
\end{Entry}

\begin{Entry}{阿}{7}{⾩}
  \begin{Phonetics}{阿}{a1}
    \definition{pref.}{em dialetos do sul para formar termos carinhosos, antes de nomes de animais de estimação, sobrenomes monossilábicos ou números que denotam ordem de antiguidade em uma; anexado a 大, 二, 三,\dots\ para indicar classificação (e, às vezes, intimidade) | antes dos termos de parentesco; na frente de um sobrenome, de um nome próprio ou de um determinado título, com uma conotação de intimidade | em alguns contextos, pode soar infantil ou muito informal (por exemplo, chamar um colega de trabalho por ``阿 + Nome'' sem intimidade)}[阿妈===mamãe | 阿明 ===forma carinhosa de chamar uma pessoa chamada Ming]
  \end{Phonetics}
  \begin{Phonetics}{阿}{e1}
    \definition*{s.}{Dong'e, um condado na província de Shandong | Sobrenome E}
    \definition{s.}{grande monte (ou colina) | um lugar sinuoso (montanha, água, etc.)}
    \definition{v.}{bajular; satisfazer}
  \end{Phonetics}
\end{Entry}

\begin{Entry}{阿拉伯语}{7,8,7,9}{⾩、⼿、⼈、⾔}
  \begin{Phonetics}{阿拉伯语}{a1la1bo2 yu3}[][HSK 7-9]
    \definition{s.}{árabe; idioma árabe}
  \end{Phonetics}
\end{Entry}

\begin{Entry}{阿姨}{7,9}{⾩、⼥}
  \begin{Phonetics}{阿姨}{a1yi2}[][HSK 4]
    \definition[个,位,名,些]{s.}{tia; uma forma de tratamento para uma mulher da geração dos pais; dirigir-se a uma mulher que tem aproximadamente a mesma idade da sua mãe, geralmente não é parente | babá em uma família; professora em um jardim de infância | tia; irmã da mãe (mais comum no sul da China)}[阿姨,生日快乐!===Tia, feliz aniversário! | 阿姨,这个苹果多少钱一斤?===Tia/Senhora, quanto custa o quilo dessas maçãs? | 阿姨,我想喝水。===Tia/Babá, eu quero beber água.]
  \end{Phonetics}
\end{Entry}

\begin{Entry}{阿哥}{7,10}{⾩、⼝}
  \begin{Phonetics}{阿哥}{a1ge1}
    \definition{s.}{irmão mais velho (afetivo) | forma afetuosa de tratamento entre homens da mesma idade}[阿哥,帮我拿一下书包!===Irmão, ajude-me com minha mochila escolar!]
  \end{Phonetics}
\end{Entry}

\begin{Entry}{附}{7}{⾩}
  \begin{Phonetics}{附}{fu4}[][HSK 7-9]
    \definition*{s.}{Sobrenome Fu}
    \definition{v.}{adicionar; anexar; incluir | chegar perto de; estar perto de | depender de; confiar em; cumprir com | concordar com; anexar a; aderir a; cumprir com; depender de}
  \end{Phonetics}
\end{Entry}

\begin{Entry}{附加}{7,5}{⾩、⼒}
  \begin{Phonetics}{附加}{fu4jia1}[][HSK 7-9]
    \definition{adj.}{adicional; anexado; extra}
    \definition{v.}{adicionar; anexar; exceder a quantidade ou intervalo prescrito}
  \end{Phonetics}
\end{Entry}

\begin{Entry}{附件}{7,6}{⾩、⼈}
  \begin{Phonetics}{附件}{fu4jian4}[][HSK 5]
    \definition*{s.}{\emph{Adnexa Uteri}, refere-se à genitália interna feminina que não seja o útero, as trompas de falópio e os ovários}
    \definition{s.}{apêndice; documentos que acompanham o documento principal | acessório; anexo; peças ou sobressalentes que não sejam peças principais de máquinas e equipamentos | anexo; documentos ou itens relevantes emitidos com o documento}
  \end{Phonetics}
\end{Entry}

\begin{Entry}{附近}{7,7}{⾩、⾡}
  \begin{Phonetics}{附近}{fu4jin4}[][HSK 4]
    \definition{adj.}{perto; vizinho}
    \definition{s.}{vizinhança; bairro}
  \end{Phonetics}
\end{Entry}

\begin{Entry}{附和}{7,8}{⾩、⼝}
  \begin{Phonetics}{附和}{fu4he4}[][HSK 7-9]
    \definition{v.}{ecoar; entrar na conversa com; repetir o que os outros dizem (alguém disse)}
  \end{Phonetics}
\end{Entry}

\begin{Entry}{附带}{7,9}{⾩、⼱}
  \begin{Phonetics}{附带}{fu4dai4}[][HSK 7-9]
    \definition{adj.}{subsidiário; suplementar; incidental}
    \definition{adv.}{de passagem; a propósito; incidentalmente}
    \definition{v.}{anexar}
  \end{Phonetics}
\end{Entry}

\begin{Entry}{附属}{7,12}{⾩、⼫}
  \begin{Phonetics}{附属}{fu4shu3}[][HSK 7-9]
    \definition{adj.}{anexado; afiliado; dependente ou pertencente a uma instituição}
    \definition{v.}{afiliar-se; filiar-se; inscrever-se; agregar-se}
  \end{Phonetics}
\end{Entry}

\begin{Entry}{陆}{7}{⾩}
  \begin{Phonetics}{陆}{liu4}
    \definition{num.}{seis, usado para o numeral 六 em cheques, etc. para evitar erros ou alterações}
  \seealsoref{六}{liu4}
  \end{Phonetics}
  \begin{Phonetics}{陆}{lu4}
    \definition*{s.}{Sobrenome Lu}
    \definition[个]{s.}{terra; terreno | rota terrestre; por terra}
  \end{Phonetics}
\end{Entry}

\begin{Entry}{陆军}{7,6}{⾩、⼍}
  \begin{Phonetics}{陆军}{lu4 jun1}[][HSK 6]
    \definition{s.}{força terrestre; exército}
  \end{Phonetics}
\end{Entry}

\begin{Entry}{陆地}{7,6}{⾩、⼟}
  \begin{Phonetics}{陆地}{lu4di4}[][HSK 4]
    \definition[块,片]{s.}{terra; terra seca (em oposição ao mar); superfície da Terra, excluindo os oceanos (e, às vezes, rios e lagos)}
  \end{Phonetics}
\end{Entry}

\begin{Entry}{陆续}{7,11}{⾩、⽷}
  \begin{Phonetics}{陆续}{lu4xu4}[][HSK 4]
    \definition{adv.}{sucessivamente; um após o outro; intermitentemente}
  \end{Phonetics}
\end{Entry}

\begin{Entry}{陆游}{7,12}{⾩、⽔}
  \begin{Phonetics}{陆游}{lu4 you2}
    \definition*{s.}{Lu You (1125-1210), nome de cortesia Wuguan 务观 e pseudônimo Fangweng 放翁, era natural de Shanyin 山阴, Yuezhou 越州 (atual Shaoxing 绍兴, Zhejiang 浙江); foi poeta e letrista da Dinastia Song do Sul; gerações posteriores costumam considerar Lu You o maior poeta da Dinastia Song do Sul; o próprio Lu You afirmou ter escrito ``10.000 poemas em 60 anos''; ele é o poeta com o maior número de poemas autoescritos sobreviventes na história chinesa}
  \seealsoref{山阴}{shan1yin1}
  \seealsoref{绍兴}{shao4xing1}
  \seealsoref{浙江}{zhe4jiang1}
  \end{Phonetics}
\end{Entry}

\begin{Entry}{陆路}{7,13}{⾩、⾜}
  \begin{Phonetics}{陆路}{lu4lu4}
    \definition{s.}{rota terrestre}
  \end{Phonetics}
\end{Entry}

\begin{Entry}{陈}{7}{⾩}
  \begin{Phonetics}{陈}{chen2}
    \definition*{s.}{Chen, um estado vassalo da Dinastia Zhou (1046-256 a.C.) | Dinastia Chen (557-589), uma das dinastias do sul | Sobrenome Chen}
    \definition{adj.}{velho; obsoleto; desatualizado}
    \definition{v.}{expor; colocar em exposição; organizar; colocar para fora | declarar; explicar; expressar pensamentos, opiniões, etc. de maneira organizada}
  \end{Phonetics}
\end{Entry}

\begin{Entry}{陈旧}{7,5}{⾩、⽇}
  \begin{Phonetics}{陈旧}{chen2jiu4}[][HSK 7-9]
    \definition{adj.}{ultrapassado; fora de moda; antiquado; velho; desatualizado}
  \end{Phonetics}
\end{Entry}

\begin{Entry}{陈列}{7,6}{⾩、⼑}
  \begin{Phonetics}{陈列}{chen2lie4}[][HSK 7-9]
    \definition{v.}{exibir; expor; organizar itens de uma determinada maneira}[博物馆陈列了古代的陶瓷器。===O museu exibe cerâmicas antigas.]
  \end{Phonetics}
\end{Entry}

\begin{Entry}{陈述}{7,8}{⾩、⾡}
  \begin{Phonetics}{陈述}{chen2shu4}[][HSK 7-9]
    \definition{v.}{declarar; alegar; afirmar; falar de forma ordenada}
  \end{Phonetics}
\end{Entry}

\begin{Entry}{韧}{7}{⾱}
  \begin{Phonetics}{韧}{ren4}
    \definition{adj.}{flexível, mas forte; tenaz; resistente (oposto a 脆) | resistente; macio e forte, não quebra facilmente (ao contrário de 脆)}
  \seealsoref{脆}{cui4}
  \end{Phonetics}
\end{Entry}

\begin{Entry}{饭}{7}{⾷}
  \begin{Phonetics}{饭}{fan4}[][HSK 1]
    \definition{s.}{(empréstimo linguístico) fã, devoto}
    \definition[顿,份,碗,口,锅]{s.}{cereais cozidos; grãos cozidos | refeição; alimentos consumidos diariamente em horários regulares | trabalho; meio de subsistência; meio de vida}
  \end{Phonetics}
\end{Entry}

\begin{Entry}{饭店}{7,8}{⾷、⼴}
  \begin{Phonetics}{饭店}{fan4dian4}[][HSK 1]
    \definition[家,个]{s.}{restaurante | hotel; hotel grande e bem equipado}
  \end{Phonetics}
\end{Entry}

\begin{Entry}{饭馆}{7,11}{⾷、⾷}
  \begin{Phonetics}{饭馆}{fan4 guan3}[][HSK 2]
    \definition[家,个]{s.}{restaurante; lanchonete}
  \end{Phonetics}
\end{Entry}

\begin{Entry}{饭碗}{7,13}{⾷、⽯}
  \begin{Phonetics}{饭碗}{fan4wan3}[][HSK 7-9]
    \definition[个,只]{s.}{tigela de arroz | Coloquial: emprego; meio de subsistência | Figurativo: meio de subsistência; maneira de ganhar a vida}
  \end{Phonetics}
\end{Entry}

\begin{Entry}{饮}{7}{⾷}
  \begin{Phonetics}{饮}{yin3}
    \definition{s.}{bebidas; \emph{drinks}; algo para beber | uma decocção da medicina chinesa para ser tomada fria | fluido retido}
    \definition{v.}{beber | cuidar; engolir a pílula amarga}
  \end{Phonetics}
  \begin{Phonetics}{饮}{yin4}
    \definition{v.}{dar (aos animais) água para beber}
  \end{Phonetics}
\end{Entry}

\begin{Entry}{饮食}{7,9}{⾷、⾷}
  \begin{Phonetics}{饮食}{yin3shi2}[][HSK 5]
    \definition{s.}{dieta; alimentos e bebidas}
    \definition{v.}{comer; beber}
  \end{Phonetics}
\end{Entry}

\begin{Entry}{饮料}{7,10}{⾷、⽃}
  \begin{Phonetics}{饮料}{yin3liao4}[][HSK 5]
    \definition[杯,瓶,种]{s.}{bebida; drinque; líquidos processados e fabricados para consumo, como vinho, chá, refrigerantes, suco de laranja, etc.}
  \end{Phonetics}
\end{Entry}

\begin{Entry}{驱}{7}{⾺}
  \begin{Phonetics}{驱}{qu1}
    \definition{v.}{dirigir (um cavalo, um carro, etc.) | expulsar; dispersar | correr rápido}
  \end{Phonetics}
\end{Entry}

\begin{Entry}{驳}{7}{⾺}
  \begin{Phonetics}{驳}{bo2}
    \definition{adj.}{Literário: misturado; heterogêneo; de cores diferentes; originalmente se refere à cor impura do cabelo do cavalo, estendida aos ingredientes impuros; cores misturadas}
    \definition{s.}{barcaça; fragata}
    \definition{v.}{refutar; contradizer; contestar; argumentar; distinguir o certo do errado; dar razões para refutar as opiniões erradas dos outros | transporte por barcaça | Dialeto: estender ou alargar (um banco, um dique ou um aterro)}
  \end{Phonetics}
\end{Entry}

\begin{Entry}{驳回}{7,6}{⾺、⼞}
  \begin{Phonetics}{驳回}{bo2hui2}[][HSK 7-9]
    \definition{v.}{rejeitar; refutar; anular; recusar; não concordar (solicitação)}
  \end{Phonetics}
\end{Entry}

\begin{Entry}{驴}{7}{⾺}
  \begin{Phonetics}{驴}{lv2}
    \definition[头,只]{s.}{burro; asno; jumento; jegue}
  \end{Phonetics}
\end{Entry}

\begin{Entry}{鸡}{7}{⿃}
  \begin{Phonetics}{鸡}{ji1}[][HSK 2]
    \definition*{s.}{Sobrenome Ji}
    \definition[只]{s.}{galo, galinha, frango | palavra ofensiva para uma mulher que ganha dinheiro fazendo sexo com homens}
  \end{Phonetics}
\end{Entry}

\begin{Entry}{鸡蛋}{7,11}{⿃、⾍}
  \begin{Phonetics}{鸡蛋}{ji1dan4}[][HSK 1]
    \definition[个,枚,筐,箱,打]{s.}{ovo de galinha}
  \end{Phonetics}
\end{Entry}

\begin{Entry}{麦}{7}{⿆}[Kangxi 199]
  \begin{Phonetics}{麦}{mai4}
    \definition*{s.}{Sobrenome Mai}
    \definition[袋,筐,车]{s.}{um termo geral para trigo, cevada, etc.}
  \end{Phonetics}
\end{Entry}

\begin{Entry}{麦当劳}{7,6,7}{⿆、⼹、⼒}
  \begin{Phonetics}{麦当劳}{mai4dang1lao2}
    \definition*{s.}{McDonald's, restaurante de \emph{fast-food}}
  \seealsoref{麦当劳叔叔}{mai4dang1lao2 shu1shu5}
  \end{Phonetics}
\end{Entry}

\begin{Entry}{麦当劳叔叔}{7,6,7,8,8}{⿆、⼹、⼒、⼜、⼜}
  \begin{Phonetics}{麦当劳叔叔}{mai4dang1lao2 shu1shu5}
    \definition*{s.}{Ronald McDonald}
  \seealsoref{麦当劳}{mai4dang1lao2}
  \end{Phonetics}
\end{Entry}

\begin{Entry}{麦淇淋}{7,11,11}{⿆、⽔、⽔}
  \begin{Phonetics}{麦淇淋}{mai4qi2lin2}
    \definition{s.}{Empréstimo linguístico: margarina}
  \end{Phonetics}
\end{Entry}

\begin{Entry}{龟}{7}{⿔}[Kangxi 213]
  \begin{Phonetics}{龟}{gui1}[][HSK 7-9]
    \definition[只]{s.}{tartaruga; cágado}
  \end{Phonetics}
\end{Entry}

\begin{Entry}{龟速}{7,10}{⿔、⾡}
  \begin{Phonetics}{龟速}{gui1su4}
    \definition{adv.}{tão lento quanto uma tartaruga}
  \end{Phonetics}
\end{Entry}

%%%%% EOF %%%%%

