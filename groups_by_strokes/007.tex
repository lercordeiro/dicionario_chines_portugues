%%%
%%% 7画
%%%

\section*{7画}\addcontentsline{toc}{section}{7画}

\begin{entry}{两}{7}[Radical ⼀]
  \begin{phonetics}{两}{liang3}[][HSK 1,2]
    \definition{adv.}{ambos (lados) | cada (lado)}
    \definition{clas.}{liang, uma unidade de peso (=50 gramas)}
    \definition{num.}{dois (sempre usado antes de classificadores) | poucos; alguns}
  \end{phonetics}
\end{entry}

\begin{entry}{两边}{7,5}[Radicais ⼀、⾡]
  \begin{phonetics}{两边}{liang3 bian1}[][HSK 4]
    \definition{s.}{ambos os lados; ambas as direções; ambos os lugares | ambas as partes; ambos os lados}
  \end{phonetics}
\end{entry}

\begin{entry}{两码事}{7,8,8}[Radicais ⼀、⽯、⼅]
  \begin{phonetics}{两码事}{liang3ma3shi4}
    \definition{expr.}{duas coisas completamente diferentes}
  \end{phonetics}
\end{entry}

\begin{entry}{严重}{7,9}[Radicais ⼀、⾥]
  \begin{phonetics}{严重}{yan2zhong4}
    \definition{adj.}{crítico | grave | sério | severo}
  \end{phonetics}
\end{entry}

\begin{entry}{严重打伤}{7,9,5,6}[Radicais ⼀、⾥、⼿、⼈]
  \begin{phonetics}{严重打伤}{yan2zhong4 da3 shang1}
    \definition{s.}{gravemente ferido}
  \end{phonetics}
\end{entry}

\begin{entry}{严重伤害}{7,9,6,10}[Radicais ⼀、⾥、⼈、⼧]
  \begin{phonetics}{严重伤害}{yan2zhong4 shang1hai4}
    \definition{s.}{ferimento grave}
  \end{phonetics}
\end{entry}

\begin{entry}{严重关切}{7,9,6,4}[Radicais ⼀、⾥、⼋、⼑]
  \begin{phonetics}{严重关切}{yan2zhong4guan1qie4}
    \definition{s.}{preocupação séria}
  \end{phonetics}
\end{entry}

\begin{entry}{严重危害}{7,9,6,10}[Radicais ⼀、⾥、⼙、⼧]
  \begin{phonetics}{严重危害}{yan2zhong4wei1hai4}
    \definition{s.}{danos graves}
  \end{phonetics}
\end{entry}

\begin{entry}{严重后果}{7,9,6,8}[Radicais ⼀、⾥、⼝、⽊]
  \begin{phonetics}{严重后果}{yan2zhong4hou4guo3}
    \definition{s.}{consequências sérias | repercursões graves}
  \end{phonetics}
\end{entry}

\begin{entry}{严重地}{7,9,6}[Radicais ⼀、⾥、⼟]
  \begin{phonetics}{严重地}{yan2zhong4 di4}
    \definition{adv.}{seriamente | gravemente}
  \end{phonetics}
\end{entry}

\begin{entry}{严重问题}{7,9,6,15}[Radicais ⼀、⾥、⾨、⾴]
  \begin{phonetics}{严重问题}{yan2zhong4wen4ti2}
    \definition{s.}{problema sério}
  \end{phonetics}
\end{entry}

\begin{entry}{严重性}{7,9,8}[Radicais ⼀、⾥、⼼]
  \begin{phonetics}{严重性}{yan2zhong4xing4}
    \definition{s.}{seriedade | gravidade}
  \end{phonetics}
\end{entry}

\begin{entry}{严重破坏}{7,9,10,7}[Radicais ⼀、⾥、⽯、⼟]
  \begin{phonetics}{严重破坏}{yan2zhong4 po4huai4}
    \definition{s.}{destruição grave}
  \end{phonetics}
\end{entry}

\begin{entry}{乱}{7}[Radical ⼄]
  \begin{phonetics}{乱}{luan4}[][HSK 3]
    \definition{adj.}{bagunçado; confuso; desordenado | turbulento; perturbado (estado de espírito) | arbitrário; aleatório}
    \definition{adv.}{em confusão ou desordem; em um estado de espírito confuso}
    \definition{s.}{caos; tumulto; agitação; turbilhão | comportamento sexual promíscuo; promiscuidade}
    \definition{v.}{confundir; embaralhar; misturar}
  \end{phonetics}
\end{entry}

\begin{entry}{估计}{7,4}[Radicais ⼈、⾔]
  \begin{phonetics}{估计}{gu1ji4}
    \definition{v.}{estimar | avaliar | calcular}
  \end{phonetics}
\end{entry}

\begin{entry}{伲}{7}[Radical ⼈]
  \begin{phonetics}{伲}{ni4}
    \definition{pron.}{(dialeto) eu | meu | nosso | nós}
    \seeref{你}{ni3}
  \end{phonetics}
\end{entry}

\begin{entry}{伴侣}{7,8}[Radicais ⼈、⼈]
  \begin{phonetics}{伴侣}{ban4lv3}
    \definition{s.}{companheiro | parceiro}
  \end{phonetics}
\end{entry}

\begin{entry}{但}{7}[Radical ⼈]
  \begin{phonetics}{但}{dan4}[][HSK 2]
    \definition{conj.}{mas | ainda | no entanto | apenas}
  \end{phonetics}
\end{entry}

\begin{entry}{但是}{7,9}[Radicais ⼈、⽇]
  \begin{phonetics}{但是}{dan4 shi4}[][HSK 2]
    \definition{conj.}{mas | ainda | no entanto}
  \end{phonetics}
\end{entry}

\begin{entry}{位}{7}[Radical ⼈]
  \begin{phonetics}{位}{wei4}[][HSK 2]
    \definition{clas.}{para pessoas (com cortesia) | para bits binários}
    \definition{s.}{(física) potencial | localização | lugar | posição | assento}
    \example{十六位}[16 bits]
  \end{phonetics}
\end{entry}

\begin{entry}{位子}{7,3}[Radicais ⼈、⼦]
  \begin{phonetics}{位子}{wei4zi5}
    \definition{s.}{lugar | assento}
  \end{phonetics}
\end{entry}

\begin{entry}{位居}{7,8}[Radicais ⼈、⼫]
  \begin{phonetics}{位居}{wei4ju1}
    \definition{v.}{estar localizado em}
  \end{phonetics}
\end{entry}

\begin{entry}{位置}{7,13}[Radicais ⼈、⽹]
  \begin{phonetics}{位置}{wei4zhi5}
    \definition[个]{s.}{lugar | posição | assento}
  \end{phonetics}
\end{entry}

\begin{entry}{低}{7}[Radical ⼈]
  \begin{phonetics}{低}{di1}[][HSK 2]
    \definition{adj.}{baixo}
    \definition{adv.}{abaixo}
    \definition{v.}{abaixar (a cabeça) | deixar cair | pendurar | inclinar}
  \end{phonetics}
\end{entry}

\begin{entry}{低潮}{7,15}[Radicais ⼈、⽔]
  \begin{phonetics}{低潮}{di1chao2}
    \definition{s.}{maré baixa/vazante; o nível mais baixo da maré durante um ciclo de maré (distinto da ``高潮'') | vazante baixa; o ponto mais baixo; uma metáfora para o baixo estágio de desenvolvimento das coisas}
  \seealsoref{高潮}{gao1chao2}
  \end{phonetics}
\end{entry}

\begin{entry}{住}{7}[Radical ⼈]
  \begin{phonetics}{住}{zhu4}[][HSK 1]
    \definition{v.}{habitar | residir | morar | alojar-se}
  \end{phonetics}
\end{entry}

\begin{entry}{住处}{7,5}[Radicais ⼈、⼡]
  \begin{phonetics}{住处}{zhu4chu4}
    \definition{s.}{morada | habitação | residência}
  \end{phonetics}
\end{entry}

\begin{entry}{住宅}{7,6}[Radicais ⼈、⼧]
  \begin{phonetics}{住宅}{zhu4zhai2}
    \definition{s.}{residência}
  \end{phonetics}
\end{entry}

\begin{entry}{住房}{7,8}[Radicais ⼈、⼾]
  \begin{phonetics}{住房}{zhu4fang2}[][HSK 2]
    \definition{s.}{habitação}
  \end{phonetics}
\end{entry}

\begin{entry}{住所}{7,8}[Radicais ⼈、⼾]
  \begin{phonetics}{住所}{zhu4suo3}
    \definition[处]{s.}{morada | habitação | residência}
  \end{phonetics}
\end{entry}

\begin{entry}{住院}{7,9}[Radicais ⼈、⾩]
  \begin{phonetics}{住院}{zhu4 yuan4}[][HSK 2]
    \definition{v.}{estar hospitalizado | estar no hospital}
  \end{phonetics}
\end{entry}

\begin{entry}{住嘴}{7,16}[Radicais ⼈、⼝]
  \begin{phonetics}{住嘴}{zhu4zui3}
    \definition{interj.}{Cale-se!}
    \definition{v.}{calar | calar-se}
  \end{phonetics}
\end{entry}

\begin{entry}{体内}{7,4}[Radicais ⼈、⼌]
  \begin{phonetics}{体内}{ti3nei4}
    \definition{adj.}{dentro do corpo | \emph{in vivo} (versus \emph{in vitro} | interno a}
  \end{phonetics}
\end{entry}

\begin{entry}{体会}{7,6}[Radicais ⼈、⼈]
  \begin{phonetics}{体会}{ti3hui4}[][HSK 3]
    \definition{s.}{conhecimento; compreensão; experiência pessoal}
    \definition{v.}{perceber; saber (ou aprender) com a experiência}
  \end{phonetics}
\end{entry}

\begin{entry}{体现}{7,8}[Radicais ⼈、⾒]
  \begin{phonetics}{体现}{ti3xian4}[][HSK 3]
    \definition{v.}{refletir; incorporar; encarnar}
  \end{phonetics}
\end{entry}

\begin{entry}{体育}{7,8}[Radicais ⼈、⾁]
  \begin{phonetics}{体育}{ti3yu4}[][HSK 2]
    \definition{s.}{treinamento físico | esportes | atividades esportivas}
  \end{phonetics}
\end{entry}

\begin{entry}{体育场}{7,8,6}[Radicais ⼈、⾁、⼟]
  \begin{phonetics}{体育场}{ti3 yu4 chang3}[][HSK 2]
    \definition[个,座]{s.}{estádio | campo de esportes}
  \end{phonetics}
\end{entry}

\begin{entry}{体育馆}{7,8,11}[Radicais ⼈、⾁、⾷]
  \begin{phonetics}{体育馆}{ti3 yu4 guan3}[][HSK 2]
    \definition[个]{s.}{ginásio | estádio}
  \end{phonetics}
\end{entry}

\begin{entry}{体验}{7,10}[Radicais ⼈、⾺]
  \begin{phonetics}{体验}{ti3yan4}[][HSK 3]
    \definition[种]{s.}{experiência}
    \definition{v.}{aprender através da prática; aprender através da experiência pessoal}
  \end{phonetics}
\end{entry}

\begin{entry}{何}{7}[Radical ⼈]
  \begin{phonetics}{何}{he2}
    \definition*{s.}{sobrenome He}
    \definition{adv.}{expressa exclamação, equivalente a "多么".}
    \definition{pron.}{O que?; Onde?; Por que? | expressa uma pergunta retórica, equivalente a ``岂'', ``怎''}
  \seealsoref{多么}{duo1me5}
  \seealsoref{岂}{qi3}
  \seealsoref{怎}{zen3}
  \end{phonetics}
\end{entry}

\begin{entry}{何况}{7,7}[Radicais ⼈、⼎]
  \begin{phonetics}{何况}{he2kuang4}
    \definition{conj.}{além disso | muito menos}
  \end{phonetics}
\end{entry}

\begin{entry}{佛}{7}[Radical ⼈]
  \begin{phonetics}{佛}{fo2}
    \definition*{s.}{Buda, abreviação de 佛陀 | Budismo}
  \seealsoref{佛陀}{fo2tuo2}
  \end{phonetics}
  \begin{phonetics}{佛}{fu2}
    \definition{adv.}{aparentemente}
    \definition{s.}{ornamento da cabeça (feminino)}
  \end{phonetics}
\end{entry}

\begin{entry}{佛陀}{7,7}[Radicais ⼈、⾩]
  \begin{phonetics}{佛陀}{fo2tuo2}
    \definition{s.}{Buda (uma pessoa que atingiu a Budeidade, ou especificamente Siddhartha Gautama)}
  \end{phonetics}
\end{entry}

\begin{entry}{作}{7}[Radical ⼈]
  \begin{phonetics}{作}{zuo1}
    \definition{adj.}{(gíria) incômodo}
    \definition{s.}{trabalhador | oficina | (pessoa) de alta manutenção}
  \end{phonetics}
  \begin{phonetics}{作}{zuo4}
    \definition{s.}{escritos ou obras}
    \definition{v.}{fazer | crescer | escrever ou compor | fingir | considerar como | sentir}
  \end{phonetics}
\end{entry}

\begin{entry}{作文}{7,4}[Radicais ⼈、⽂]
  \begin{phonetics}{作文}{zuo4wen2}[][HSK 2]
    \definition[篇]{s.}{ensaio |  composição | redação}
    \definition{v.+compl.}{(de alunos) para escrever uma redação}
  \end{phonetics}
\end{entry}

\begin{entry}{作业}{7,5}[Radicais ⼈、⼀]
  \begin{phonetics}{作业}{zuo4ye4}[][HSK 2]
    \definition[份,个]{s.}{tarefa escolar | trabalho | tarefa | operação}
  \end{phonetics}
\end{entry}

\begin{entry}{作用}{7,5}[Radicais ⼈、⽤]
  \begin{phonetics}{作用}{zuo4yong4}[][HSK 2]
    \definition{s.}{efeito | ação | função}
    \definition{v.}{afetar | agir em}
  \end{phonetics}
\end{entry}

\begin{entry}{作者}{7,8}[Radicais ⼈、⽼]
  \begin{phonetics}{作者}{zuo4zhe3}[][HSK 3]
    \definition[位,名,个]{s.}{autor; escritor; uma pessoa que escreve artigos ou cria obras de arte}
  \end{phonetics}
\end{entry}

\begin{entry}{作品}{7,9}[Radicais ⼈、⼝]
  \begin{phonetics}{作品}{zuo4pin3}[][HSK 3]
    \definition[个,部,篇,幅]{s.}{obra de arte; obras concluídas de literatura e arte}
  \end{phonetics}
\end{entry}

\begin{entry}{作家}{7,10}[Radicais ⼈、⼧]
  \begin{phonetics}{作家}{zuo4jia1}[][HSK 2]
    \definition[位,个]{s.}{autor | escritor}
  \end{phonetics}
\end{entry}

\begin{entry}{你}{7}[Radical ⼈]
  \begin{phonetics}{你}{ni3}[][HSK 1]
    \definition{pron.}{você (informal) | tu | te | ti | contigo}
    \seeref{您}{nin2}
  \end{phonetics}
\end{entry}

\begin{entry}{你们}{7,5}[Radicais ⼈、⼈]
  \begin{phonetics}{你们}{ni3men5}[][HSK 1]
    \definition{pron.}{vocês (informal) | vós | vos | convosco}
  \end{phonetics}
\end{entry}

\begin{entry}{你们的}{7,5,8}[Radicais ⼈、⼈、⽩]
  \begin{phonetics}{你们的}{ni3men5 de5}
    \definition{pron.}{vossos}
  \end{phonetics}
\end{entry}

\begin{entry}{你好}{7,6}[Radicais ⼈、⼥]
  \begin{phonetics}{你好}{ni3hao3}
    \definition{interj.}{Olá! | Oi!}
  \end{phonetics}
\end{entry}

\begin{entry}{你的}{7,8}[Radicais ⼈、⽩]
  \begin{phonetics}{你的}{ni3 de5}
    \definition{pron.}{seu}
  \end{phonetics}
\end{entry}

\begin{entry}{克}{7}[Radical ⼗]
  \begin{phonetics}{克}{ke4}[][HSK 2]
    \definition{clas.}{grama (g)}
    \definition{v.}{pode | ser capaz de | restringir | controlar | superar | subjugar | capturar (uma cidade, etc.) | digerir | cortar | reduzir | definir um limite de tempo}
  \end{phonetics}
\end{entry}

\begin{entry}{克服}{7,8}[Radicais ⼗、⽉]
  \begin{phonetics}{克服}{ke4fu2}[][HSK 3]
    \definition{v.}{sobrepujar; superar; conquistar | suportar (dificuldades, inconveniências, etc.)}
  \end{phonetics}
\end{entry}

\begin{entry}{免费}{7,9}[Radicais ⼉、⾙]
  \begin{phonetics}{免费}{mian3fei4}[][HSK 4]
    \definition{s.}{gratuito; sem custo}
    \definition{v.+compl.}{isentar de taxas; tonar grátis}
  \end{phonetics}
\end{entry}

\begin{entry}{免得}{7,11}[Radicais ⼉、⼻]
  \begin{phonetics}{免得}{mian3de5}
    \definition{conj.}{de modo a não | para evitar | para que não}
  \end{phonetics}
\end{entry}

\begin{entry}{免税}{7,12}[Radicais ⼉、⽲]
  \begin{phonetics}{免税}{mian3shui4}
    \definition{adj.}{isento de impostos (tributação)}
    \definition{s.}{livre de impostos | isenção de impostos}
    \definition{v.+compl.}{isentar impostos}
  \end{phonetics}
\end{entry}

\begin{entry}{兵}{7}[Radical ⼋]
  \begin{phonetics}{兵}{bing1}[][HSK 4]
    \definition[名]{s.}{armas; armamentos | soldado; pessoal militar | exército; tropas | soldado raso; membro mais jovem do exército | assuntos militares (estratégia) | peão, uma das peças do xadrez chinês}
  \end{phonetics}
\end{entry}

\begin{entry}{兵器}{7,16}[Radicais ⼋、⼝]
  \begin{phonetics}{兵器}{bing1qi4}
    \definition{s.}{armas | armamento}
  \end{phonetics}
\end{entry}

\begin{entry}{况且}{7,5}[Radicais ⼎、⼀]
  \begin{phonetics}{况且}{kuang4qie3}
    \definition{conj.}{além disso | além do mais}
  \end{phonetics}
\end{entry}

\begin{entry}{冷}{7}[Radical ⼎]
  \begin{phonetics}{冷}{leng3}[][HSK 1]
    \definition*{s.}{sobrenome Leng}
    \definition{adj.}{frio}
  \end{phonetics}
\end{entry}

\begin{entry}{冷静}{7,14}[Radicais ⼎、⾭]
  \begin{phonetics}{冷静}{leng3jing4}[][HSK 4]
    \definition{adj.}{calmo; descreve uma pessoa que consegue ficar atenta em uma situação importante ou de emergência e não toma decisões aleatórias por causa de seus sentimentos no momento | (lugar) tranquilo; quieto; deserto}
  \end{phonetics}
\end{entry}

\begin{entry}{初}{7}[Radical ⾐]
  \begin{phonetics}{初}{chu1}[][HSK 3]
    \definition*{s.}{sobrenome Chu}
    \definition{adj.}{primeiro (em ordem) | elementar; rudimentar | original}
    \definition{adv.}{pela primeira vez}
    \definition{pref.}{anexado aos numerais de um a dez para indicar ordem (primeiro ao décimo)}
    \definition{s.}{no início de; na primeira parte de | o estágio júnior (pleno; sênior)}
  \end{phonetics}
\end{entry}

\begin{entry}{初中}{7,4}[Radicais ⾐、⼁]
  \begin{phonetics}{初中}{chu1 zhong1}[][HSK 3]
    \definition[所,个]{s.}{ensino médio; ensino fundamental}
  \end{phonetics}
\end{entry}

\begin{entry}{初心}{7,4}[Radicais ⾐、⼼]
  \begin{phonetics}{初心}{chu1xin1}
    \definition{s.}{intenção original de alguém, aspiração, etc. | (budismo) ``mente do iniciante'' (ter a mente aberta quando estudando um assunto como um iniciante no assunto teria)}
  \end{phonetics}
\end{entry}

\begin{entry}{初级}{7,6}[Radicais ⾐、⽷]
  \begin{phonetics}{初级}{chu1ji2}[][HSK 3]
    \definition{adj.}{elementar; primário; júnior; inicial}
  \end{phonetics}
\end{entry}

\begin{entry}{初步}{7,7}[Radicais ⾐、⽌]
  \begin{phonetics}{初步}{chu1bu4}[][HSK 3]
    \definition{adj.}{inicial; preliminar}
  \end{phonetics}
\end{entry}

\begin{entry}{判断}{7,11}[Radicais ⼑、⽄]
  \begin{phonetics}{判断}{pan4duan4}[][HSK 3]
    \definition[个]{s.}{julgamento}
    \definition{v.}{julgar; decidir}
  \end{phonetics}
\end{entry}

\begin{entry}{利用}{7,5}[Radicais ⼑、⽤]
  \begin{phonetics}{利用}{li4yong4}[][HSK 3]
    \definition{v.}{usar; utilizar; fazer uso de | explorar; tirar vantagem de}
  \end{phonetics}
\end{entry}

\begin{entry}{利息}{7,10}[Radicais ⼑、⼼]
  \begin{phonetics}{利息}{li4xi1}[][HSK 4]
    \definition{s.}{acréscimo; juros; dinheiro recebido além do valor principal como resultado de depósitos ou empréstimos (diferenciado de ``本金'')}
  \seealsoref{本金}{ben3 jin1}
  \end{phonetics}
\end{entry}

\begin{entry}{利益}{7,10}[Radicais ⼑、⽫]
  \begin{phonetics}{利益}{li4yi4}[][HSK 4]
    \definition[个,种]{s.}{ganho; lucro; juros; benefício}
  \end{phonetics}
\end{entry}

\begin{entry}{别}{7}[Radical ⼑]
  \begin{phonetics}{别}{bie2}[][HSK 1,4]
    \definition*{s.}{sobrenome Bie}
    \definition{adv.}{não; nada de (pedir a alguém para não fazer); é melhor não | talvez, usado em conjunto com a palavra ``是'' para indicar especulação.}
    \definition{pron.}{outro; algum outro}
    \definition{s.}{distinção; diferença | classificação}
    \definition{v.}{deixar; partir; separar | diferenciar; distinguir; encontrar aspectos diferentes | fixar objetos com pinos | girar; virar | aderir; colar; preder}
  \seealsoref{是}{shi4}
  \end{phonetics}
  \begin{phonetics}{别}{bie4}
    \definition{v.}{fazer com que alguém mude seus hábitos, opiniões, etc.}
  \end{phonetics}
\end{entry}

\begin{entry}{别人}{7,2}[Radicais ⼑、⼈]
  \begin{phonetics}{别人}{bie2ren5}
    \definition{pron.}{outra pessoa | outro povo | outros}
  \end{phonetics}
\end{entry}

\begin{entry}{别的}{7,8}[Radicais ⼑、⽩]
  \begin{phonetics}{别的}{bie2 de5}[][HSK 1]
    \definition{pron.}{outro}
  \end{phonetics}
\end{entry}

\begin{entry}{别说}{7,9}[Radicais ⼑、⾔]
  \begin{phonetics}{别说}{bie2shuo1}
    \definition{v.}{não falar de | não mencionar}
  \end{phonetics}
\end{entry}

\begin{entry}{助兴}{7,6}[Radicais ⼒、⼋]
  \begin{phonetics}{助兴}{zhu4xing4}
    \definition{v.+compl.}{animar as coisas | juntar-se à diversão}
  \end{phonetics}
\end{entry}

\begin{entry}{努力}{7,2}[Radicais ⼒、⼒]
  \begin{phonetics}{努力}{nu3li4}[][HSK 2]
    \definition{adj.}{diligente | aplicado}
    \definition{s.}{esforçar-se | se esforçar}
  \end{phonetics}
\end{entry}

\begin{entry}{劳工同事}{7,3,6,8}[Radicais ⼒、⼯、⼝、⼅]
  \begin{phonetics}{劳工同事}{lao2gong1 tong2shi4}
    \definition{s.}{colaborador | colega de trabalho}
  \end{phonetics}
\end{entry}

\begin{entry}{医}{7}[Radical ⼖]
  \begin{phonetics}{医}{yi1}
    \definition{s.}{médico | medicina}
    \definition{v.}{curar | tratar}
  \end{phonetics}
\end{entry}

\begin{entry}{医生}{7,5}[Radicais ⼖、⽣]
  \begin{phonetics}{医生}{yi1sheng1}[][HSK 1]
    \definition[个,位,名]{s.}{médico | clínico}
  \end{phonetics}
\end{entry}

\begin{entry}{医院}{7,9}[Radicais ⼖、⾩]
  \begin{phonetics}{医院}{yi1yuan4}[][HSK 1]
    \definition[所,家,座]{s.}{hospital}
  \end{phonetics}
\end{entry}

\begin{entry}{即}{7}[Radical ⼙]
  \begin{phonetics}{即}{ji2}
    \definition{conj.}{e | até | mesmo se/embora}
  \end{phonetics}
\end{entry}

\begin{entry}{即使}{7,8}[Radicais ⼙、⼈]
  \begin{phonetics}{即使}{ji2shi3}
    \definition{conj.}{mesmo se/embora}
  \end{phonetics}
\end{entry}

\begin{entry}{即或}{7,8}[Radicais ⼙、⼽]
  \begin{phonetics}{即或}{ji2huo4}
    \definition{conj.}{mesmo se/embora}
  \end{phonetics}
\end{entry}

\begin{entry}{即若}{7,8}[Radicais ⼙、⾋]
  \begin{phonetics}{即若}{ji2ruo4}
    \definition{conj.}{mesmo se/embora}
  \end{phonetics}
\end{entry}

\begin{entry}{即便}{7,9}[Radicais ⼙、⼈]
  \begin{phonetics}{即便}{ji2bian4}
    \definition{conj.}{mesmo se/embora}
  \end{phonetics}
\end{entry}

\begin{entry}{即将}{7,9}[Radicais ⼙、⼨]
  \begin{phonetics}{即将}{ji2jiang1}[][HSK 4]
    \definition{adv.}{em breve; estar prestes a; estar a ponto de}
  \end{phonetics}
\end{entry}

\begin{entry}{即是}{7,9}[Radicais ⼙、⽇]
  \begin{phonetics}{即是}{ji2shi4}
    \definition{conj.}{aquilo é}
  \end{phonetics}
\end{entry}

\begin{entry}{却}{7}[Radical ⼙]
  \begin{phonetics}{却}{que4}[][HSK 4]
    \definition{adv.}{mas; contudo; no entanto; enquanto; indica um ponto de virada}
    \definition{v.}{recuar; retroceder | afastar; repelir; desencorajar | declinar; recusar; rejeitar | usado depois de certos verbos para indicar a conclusão de uma ação}
  \end{phonetics}
\end{entry}

\begin{entry}{却是}{7,9}[Radicais ⼙、⽇]
  \begin{phonetics}{却是}{que4shi4}
    \definition{conj.}{no entanto | realmente | o fato é\dots | mas isso é\dots}
  \end{phonetics}
\end{entry}

\begin{entry}{君主立宪制}{7,5,5,9,8}[Radicais ⼝、⼂、⽴、⼧、⼑]
  \begin{phonetics}{君主立宪制}{jun1zhu3li4xian4zhi4}
    \definition{s.}{monarquia constitucional}
  \end{phonetics}
\end{entry}

\begin{entry}{吟诗}{7,8}[Radicais ⼝、⾔]
  \begin{phonetics}{吟诗}{yin2shi1}
    \definition{v.}{recitar poesia}
  \end{phonetics}
\end{entry}

\begin{entry}{否认}{7,4}[Radicais ⼝、⾔]
  \begin{phonetics}{否认}{fou3ren4}[][HSK 3]
    \definition{v.}{negar; repudiar}
  \end{phonetics}
\end{entry}

\begin{entry}{否则}{7,6}[Radicais ⼝、⼑]
  \begin{phonetics}{否则}{fou3ze2}[][HSK 4]
    \definition{conj.}{senão; se não; ou então; se não for isso}
  \end{phonetics}
\end{entry}

\begin{entry}{否定}{7,8}[Radicais ⼝、⼧]
  \begin{phonetics}{否定}{fou3ding4}[][HSK 3]
    \definition{adj.}{negativo}
    \definition{s.}{negativo (resposta); negação}
    \definition{v.}{rejeitar; negar}
  \end{phonetics}
\end{entry}

\begin{entry}{吧}{7}[Radical ⼝]
  \begin{phonetics}{吧}{ba1}
    \definition{s.}{som de estalo, som crepitante}
    \definition{v.}{puxar o cachimbo; fumar | abreviação de ``bar''}
  \end{phonetics}
  \begin{phonetics}{吧}{ba5}[][HSK 1]
    \definition{part.}{indica discussão, sugestão, solicitação ou comando no final de uma frase | indica concordância ou aprovação no final de uma frase | indica uma pergunta ou especulação no final de uma frase | indica incerteza no final de uma frase | em uma frase, indica uma pausa, carrega um tom hipotético, frequentemente apresenta um contraste e implica um dilema}
  \end{phonetics}
\end{entry}

\begin{entry}{含}{7}[Radical ⼝]
  \begin{phonetics}{含}{han2}[][HSK 4]
    \definition{v.}{manter na boca (sem engolir ou cuspir) | conter; incluir | cuidar; acalentar; abrigar}
  \end{phonetics}
\end{entry}

\begin{entry}{含义}{7,3}[Radicais ⼝、⼂]
  \begin{phonetics}{含义}{han2yi4}[][HSK 4]
    \definition[个,种,层]{s.}{sentido; mensagem; significado; implicação}
  \end{phonetics}
\end{entry}

\begin{entry}{含有}{7,6}[Radicais ⼝、⽉]
  \begin{phonetics}{含有}{han2 you3}[][HSK 4]
    \definition{v.}{conter; ter; incluir}
  \end{phonetics}
\end{entry}

\begin{entry}{含金量}{7,8,12}[Radicais ⼝、⾦、⾥]
  \begin{phonetics}{含金量}{han2jin1liang4}
    \definition{adj.}{conteúdo de ouro | (fig.) valioso}
  \end{phonetics}
\end{entry}

\begin{entry}{含量}{7,12}[Radicais ⼝、⾥]
  \begin{phonetics}{含量}{han2 liang4}[][HSK 4]
    \definition{s.}{conteúdo; a quantidade de um componente contido em uma substância}
  \end{phonetics}
\end{entry}

\begin{entry}{听}{7}[Radical ⼝]
  \begin{phonetics}{听}{ting1}[][HSK 1]
    \definition{clas.}{para bebidas enlatadas}
    \definition{s.}{lata de bebida (empréstimo linguístico, do inglês ``\emph{tin}'')}
    \definition{v.}{ouvir | escutar | obedecer}
  \end{phonetics}
\end{entry}

\begin{entry}{听力}{7,2}[Radicais ⼝、⼒]
  \begin{phonetics}{听力}{ting1li4}[][HSK 3]
    \definition{s.}{audição; capacidade auditiva | compreensão auditiva (na aprendizagem de línguas)}
  \end{phonetics}
\end{entry}

\begin{entry}{听力理解}{7,2,11,13}[Radicais ⼝、⼒、⽟、⾓]
  \begin{phonetics}{听力理解}{ting1li4li3jie3}
    \definition{s.}{compreensão auditiva}
  \end{phonetics}
\end{entry}

\begin{entry}{听小骨}{7,3,9}[Radicais ⼝、⼩、⾻]
  \begin{phonetics}{听小骨}{ting1xiao3gu3}
    \definition{s.}{ossículos (do ouvido médio)}
  \seealsoref{听骨}{ting1gu3}
  \end{phonetics}
\end{entry}

\begin{entry}{听见}{7,4}[Radicais ⼝、⾒]
  \begin{phonetics}{听见}{ting1 jian4}[][HSK 1]
    \definition{v.}{ouvir}
  \end{phonetics}
\end{entry}

\begin{entry}{听写}{7,5}[Radicais ⼝、⼍]
  \begin{phonetics}{听写}{ting1xie3}[][HSK 1]
    \definition{s.}{ditado}
    \definition{v.}{transcrever música de ouvido | escrever (em um exercício de ditado)}
  \end{phonetics}
\end{entry}

\begin{entry}{听众}{7,6}[Radicais ⼝、⼈]
  \begin{phonetics}{听众}{ting1 zhong4}[][HSK 3]
    \definition{s.}{audiência; ouvintes}
  \end{phonetics}
\end{entry}

\begin{entry}{听会}{7,6}[Radicais ⼝、⼈]
  \begin{phonetics}{听会}{ting1hui4}
    \definition{v.}{participar de uma reunião (e ouvir o que é discutido)}
  \end{phonetics}
\end{entry}

\begin{entry}{听戏}{7,6}[Radicais ⼝、⼽]
  \begin{phonetics}{听戏}{ting1xi4}
    \definition{v.}{assistir a uma ópera | ver uma ópera}
  \end{phonetics}
\end{entry}

\begin{entry}{听讲}{7,6}[Radicais ⼝、⾔]
  \begin{phonetics}{听讲}{ting1 jiang3}[][HSK 2]
    \definition{v.+compl.}{assistir a uma palestra; ouvir uma conversa}
  \end{phonetics}
\end{entry}

\begin{entry}{听来}{7,7}[Radicais ⼝、⽊]
  \begin{phonetics}{听来}{ting1lai2}
    \definition{v.}{ouvir de algum lugar | soar (antigo, estrangeiro, excitante, certo, etc.) | soar como se (ou seja, dar uma impressão ao ouvinte)}
  \end{phonetics}
\end{entry}

\begin{entry}{听凭}{7,8}[Radicais ⼝、⼏]
  \begin{phonetics}{听凭}{ting1ping2}
    \definition{v.}{permitir (alguém a fazer o que desejar)}
  \end{phonetics}
\end{entry}

\begin{entry}{听到}{7,8}[Radicais ⼝、⼑]
  \begin{phonetics}{听到}{ting1dao4}[][HSK 1]
    \definition{v.}{ouvir | notar}
  \end{phonetics}
\end{entry}

\begin{entry}{听命}{7,8}[Radicais ⼝、⼝]
  \begin{phonetics}{听命}{ting1ming4}
    \definition{v.}{obedecer ordens | receber ordens}
  \end{phonetics}
\end{entry}

\begin{entry}{听说}{7,9}[Radicais ⼝、⾔]
  \begin{phonetics}{听说}{ting1 shuo1}[][HSK 2]
    \definition{v.}{ouvir dizer}
  \end{phonetics}
\end{entry}

\begin{entry}{听骨}{7,9}[Radicais ⼝、⾻]
  \begin{phonetics}{听骨}{ting1gu3}
    \definition{s.}{ossículos (do ouvido médio)}
  \seealsoref{听小骨}{ting1xiao3gu3}
  \end{phonetics}
\end{entry}

\begin{entry}{听断}{7,11}[Radicais ⼝、⽄]
  \begin{phonetics}{听断}{ting1duan4}
    \definition{v.}{ouvir e decidir | julgar (ou seja, ouvir e julgar em um tribunal)}
  \end{phonetics}
\end{entry}

\begin{entry}{听随}{7,11}[Radicais ⼝、⾩]
  \begin{phonetics}{听随}{ting1sui2}
    \definition{v.}{permitir | obedecer}
  \end{phonetics}
\end{entry}

\begin{entry}{吵}{7}[Radical ⼝]
  \begin{phonetics}{吵}{chao3}[][HSK 3]
    \definition{adj.}{barulhento; ruidoso}
    \definition{v.}{perturbar fazendo barulho; fazer barulho | discutir; brigar; disputar}
  \end{phonetics}
\end{entry}

\begin{entry}{吵架}{7,9}[Radicais ⼝、⽊]
  \begin{phonetics}{吵架}{chao3jia4}[][HSK 3]
    \definition{v.+compl.}{brigar; discutir; ter uma briga}
  \end{phonetics}
\end{entry}

\begin{entry}{吹}{7}[Radical ⼝]
  \begin{phonetics}{吹}{chui1}[][HSK 2]
    \definition{v.}{soprar | tocar (instrumentos de sopro) | bajular |  louvar aos céus | separar (casal)  | fracassar}
  \end{phonetics}
\end{entry}

\begin{entry}{吹牛}{7,4}[Radicais ⼝、⽜]
  \begin{phonetics}{吹牛}{chui1niu2}
    \definition{v.+compl.}{ogulhar-se | gabar-se | destacar-se}
  \end{phonetics}
\end{entry}

\begin{entry}{吾}{7}[Radical ⼝]
  \begin{phonetics}{吾}{wu2}
    \definition{pron.}{eu | (antigo) meu}
    \definition{s.}{sobrenome Wu}
  \end{phonetics}
\end{entry}

\begin{entry}{告别}{7,7}[Radicais ⼝、⼑]
  \begin{phonetics}{告别}{gao4bie2}[][HSK 3]
    \definition{v.+compl.}{dizer adeus a | deixar; partir de | prestar as últimas homenagens ao falecido}
  \end{phonetics}
\end{entry}

\begin{entry}{告诉}{7,7}[Radicais ⼝、⾔]
  \begin{phonetics}{告诉}{gao4su4}
    \definition{v.}{apresentar queixa | registar uma reclamação}
  \end{phonetics}
  \begin{phonetics}{告诉}{gao4su5}[][HSK 1]
    \definition{v.}{contar | dar a conhecer | informar}
  \end{phonetics}
\end{entry}

\begin{entry}{告急}{7,9}[Radicais ⼝、⼼]
  \begin{phonetics}{告急}{gao4ji2}
    \definition{v.}{estar em estado de emergência | relatar uma emergência | solicitar assistência de emergência}
  \end{phonetics}
\end{entry}

\begin{entry}{员}{7}[Radical ⼝]
  \begin{phonetics}{员}{yuan2}[][HSK 3]
    \definition{clas.}{para comandantes militares}
    \definition{s.}{uma pessoa envolvida em algum campo de atividade; refere-se a pessoas que trabalham ou estudam | membro; refere-se aos membros de um grupo ou organização}
  \end{phonetics}
\end{entry}

\begin{entry}{员工}{7,3}[Radicais ⼝、⼯]
  \begin{phonetics}{员工}{yuan2gong1}[][HSK 3]
    \definition[位,名,个]{s.}{funcionário; atendente; balconista; empregado; trabalhador; pessoal}
  \end{phonetics}
\end{entry}

\begin{entry}{囯}{7}[Radical ⼞]
  \begin{phonetics}{囯}{guo2}
    \variantof{国}
  \end{phonetics}
\end{entry}

\begin{entry}{困}{7}[Radical ⼞]
  \begin{phonetics}{困}{kun4}[][HSK 3]
    \definition{adj.}{cansado | sonolento}
    \definition{v.}{estar encalhado; estar em grande pressão | cercar; prender; sitiar; cercar; rodear}
  \end{phonetics}
\end{entry}

\begin{entry}{困难}{7,10}[Radicais ⼞、⾫]
  \begin{phonetics}{困难}{kun4nan5}[][HSK 3]
    \definition{adj.}{dificuldades financeiras; circunstâncias difíceis | complicado; nodoso; difícil; duro;}
    \definition[种]{s.}{dificuldade; situação difícil}
  \end{phonetics}
\end{entry}

\begin{entry}{围}{7}[Radical ⼞]
  \begin{phonetics}{围}{wei2}[][HSK 3]
    \definition*{s.}{sobrenome Wei}
    \definition{clas.}{o comprimento dos dois polegares e indicadores ou o comprimento de ambos os braços quando unidos}
    \definition{s.}{em volta de tudo; ao redor}
    \definition{v.}{cercar; rodear; circundar; encurralar | enrolar; envolver}
  \end{phonetics}
\end{entry}

\begin{entry}{坏}{7}[Radical ⼟]
  \begin{phonetics}{坏}{huai4}[][HSK 1]
    \definition{adj.}{avariado | mau}
    \definition{v.}{perder o controle}
  \end{phonetics}
\end{entry}

\begin{entry}{坏人}{7,2}[Radicais ⼟、⼈]
  \begin{phonetics}{坏人}{huai4 ren2}[][HSK 2]
    \definition[个]{s.}{malfeitor | canalha | pessoa má}
  \end{phonetics}
\end{entry}

\begin{entry}{坏处}{7,5}[Radicais ⼟、⼡]
  \begin{phonetics}{坏处}{huai4 chu4}[][HSK 2]
    \definition[个]{s.}{dano | problema}
  \end{phonetics}
\end{entry}

\begin{entry}{坏蛋}{7,11}[Radicais ⼟、⾍]
  \begin{phonetics}{坏蛋}{huai4dan4}
    \definition{s.}{bastardo | canalha | pessoa má}
  \end{phonetics}
\end{entry}

\begin{entry}{坐}{7}[Radical ⼟]
  \begin{phonetics}{坐}{zuo4}[][HSK 1]
    \definition*{s.}{sobrenome Zuo}
    \definition{v.}{sentar-se | andar de carro, ônibus, trem, avião, etc.}
  \end{phonetics}
\end{entry}

\begin{entry}{坐下}{7,3}[Radicais ⼟、⼀]
  \begin{phonetics}{坐下}{zuo4xia5}[][HSK 1]
    \definition{v.}{sentar-se | tomar um assento}
  \end{phonetics}
\end{entry}

\begin{entry}{坐车}{7,4}[Radicais ⼟、⾞]
  \begin{phonetics}{坐车}{zuo4che1}
    \definition{v.}{andar de carro, ônibus, trem, etc.}
  \end{phonetics}
\end{entry}

\begin{entry}{坐好}{7,6}[Radicais ⼟、⼥]
  \begin{phonetics}{坐好}{zuo4hao3}
    \definition{v.}{sentar-se corretamente | sentar direito}
  \end{phonetics}
\end{entry}

\begin{entry}{坐享}{7,8}[Radicais ⼟、⼇]
  \begin{phonetics}{坐享}{zuo4xiang3}
    \definition{v.}{curtir algo sem levantar um dedo}
  \end{phonetics}
\end{entry}

\begin{entry}{坐垫}{7,9}[Radicais ⼟、⼟]
  \begin{phonetics}{坐垫}{zuo4dian4}
    \definition[块]{s.}{assento (motocicleta) | almofada}
  \end{phonetics}
\end{entry}

\begin{entry}{坐标}{7,9}[Radicais ⼟、⽊]
  \begin{phonetics}{坐标}{zuo4biao1}
    \definition{s.}{coordenada (geometria)}
  \end{phonetics}
\end{entry}

\begin{entry}{坑}{7}[Radical ⼟]
  \begin{phonetics}{坑}{keng1}
    \definition{s.}{poço | depressão | túnel | buraco no chão}
    \definition{v.}{enganar | trapacear}
  \end{phonetics}
\end{entry}

\begin{entry}{坑人}{7,2}[Radicais ⼟、⼈]
  \begin{phonetics}{坑人}{keng1ren2}
    \definition{v.+compl.}{trapacear alguém}
  \end{phonetics}
\end{entry}

\begin{entry}{块}{7}[Radical ⼟]
  \begin{phonetics}{块}{kuai4}[][HSK 1]
    \definition{clas.}{(coloquial) para dinheiro e unidades monetárias | para peças ou pedaços de roupa, bolos, sabão, etc.}
    \definition{s.}{pedaço | pedaço (de terra) | peça}
  \end{phonetics}
\end{entry}

\begin{entry}{坚决}{7,6}[Radicais ⼟、⼎]
  \begin{phonetics}{坚决}{jian1jue2}[][HSK 3]
    \definition{adj.}{firme; resoluto}
  \end{phonetics}
\end{entry}

\begin{entry}{坚守}{7,6}[Radicais ⼟、⼧]
  \begin{phonetics}{坚守}{jian1shou3}
    \definition{v.}{agarrar-se}
  \end{phonetics}
\end{entry}

\begin{entry}{坚固}{7,8}[Radicais ⼟、⼞]
  \begin{phonetics}{坚固}{jian1gu4}[][HSK 4]
    \definition{adj.}{firme; sólido; robusto; forte; durável; firmemente unidos e inquebráveis}
  \end{phonetics}
\end{entry}

\begin{entry}{坚持}{7,9}[Radicais ⼟、⼿]
  \begin{phonetics}{坚持}{jian1chi2}[][HSK 3]
    \definition{v.}{persistir e; perseverar em; sustentar; insistir em; manter-se fiel a; aderir a}
  \end{phonetics}
\end{entry}

\begin{entry}{坚强}{7,12}[Radicais ⼟、⼸]
  \begin{phonetics}{坚强}{jian1qiang2}[][HSK 3]
    \definition{adj.}{forte; firme; convicto}
    \definition{v.}{fortalecer; tornar forte}
  \end{phonetics}
\end{entry}

\begin{entry}{坠}{7}[Radical ⼟]
  \begin{phonetics}{坠}{zhui4}
    \definition{v.}{cair | pesar | fazer vergar com o peso}
  \end{phonetics}
\end{entry}

\begin{entry}{坠落}{7,12}[Radicais ⼟、⾋]
  \begin{phonetics}{坠落}{zhui4luo4}
    \definition{v.}{cair}
  \end{phonetics}
\end{entry}

\begin{entry}{声明}{7,8}[Radicais ⼠、⽇]
  \begin{phonetics}{声明}{sheng1ming2}[][HSK 3]
    \definition[项,份]{s.}{declaração}
    \definition{v.}{declarar}
    \definition{v.}{declarar; anunciar}
  \end{phonetics}
\end{entry}

\begin{entry}{声音}{7,9}[Radicais ⼠、⾳]
  \begin{phonetics}{声音}{sheng1yin1}[][HSK 2]
    \definition[个,种]{s.}{som | voz}
  \end{phonetics}
\end{entry}

\begin{entry}{壳}{7}[Radical ⼠]
  \begin{phonetics}{壳}{ke2}
    \definition{s.}{casca (de ovo, noz, caranguejo, etc.) | caixa | invólucro | alojamento (de uma máquina ou dispositivo)}
  \end{phonetics}
\end{entry}

\begin{entry}{妖}{7}[Radical ⼥]
  \begin{phonetics}{妖}{yao1}
    \definition{adj.}{enfeitiçante | encantador}
    \definition{s.}{\emph{goblin} | bruxa | diabo | monstro | fantasma | demônio}
  \end{phonetics}
\end{entry}

\begin{entry}{妙招}{7,8}[Radicais ⼥、⼿]
  \begin{phonetics}{妙招}{miao4zhao1}
    \definition{adj.}{escorregadio}
    \definition{s.}{movimento inteligente | maneira inteligente de fazer algo}
  \end{phonetics}
\end{entry}

\begin{entry}{宋}{7}[Radical ⼧]
  \begin{phonetics}{宋}{song4}
    \definition*{s.}{sobrenome Song}
    \definition{s.}{Dinastia Song (960-1279) | Song das dinastias do sul (420-479)}
  \end{phonetics}
\end{entry}

\begin{entry}{完}{7}[Radical ⼧]
  \begin{phonetics}{完}{wan2}[][HSK 2]
    \definition{adj.}{completo | inteiro}
    \definition{adv.}{todo}
    \definition{v.}{acabar | completar | terminar}
  \end{phonetics}
\end{entry}

\begin{entry}{完人}{7,2}[Radicais ⼧、⼈]
  \begin{phonetics}{完人}{wan2ren2}
    \definition{s.}{pessoa perfeita}
  \end{phonetics}
\end{entry}

\begin{entry}{完全}{7,6}[Radicais ⼧、⼊]
  \begin{phonetics}{完全}{wan2quan2}[][HSK 2]
    \definition{adj.}{completo | todo}
    \definition{adv.}{inteiramente | totalmente}
  \end{phonetics}
\end{entry}

\begin{entry}{完成}{7,6}[Radicais ⼧、⼽]
  \begin{phonetics}{完成}{wan2cheng2}[][HSK 2]
    \definition{v.}{realizar | completar}
  \end{phonetics}
\end{entry}

\begin{entry}{完毕}{7,6}[Radicais ⼧、⽐]
  \begin{phonetics}{完毕}{wan2bi4}
    \definition{v.}{completar | terminar | acabar}
  \end{phonetics}
\end{entry}

\begin{entry}{完完全全}{7,7,6,6}[Radicais ⼧、⼧、⼊、⼊]
  \begin{phonetics}{完完全全}{wan2wan2quan2quan2}
    \definition{adv.}{completamente}
  \end{phonetics}
\end{entry}

\begin{entry}{完备}{7,8}[Radicais ⼧、⼡]
  \begin{phonetics}{完备}{wan2bei4}
    \definition{adj.}{completo | impecável | perfeito}
    \definition{v.}{não deixar nada a desejar}
  \end{phonetics}
\end{entry}

\begin{entry}{完美}{7,9}[Radicais ⼧、⽺]
  \begin{phonetics}{完美}{wan2mei3}[][HSK 3]
    \definition{adj.}{perfeito; impecável; consumado}
    \definition{adv.}{perfeitamente}
    \definition{s.}{perfeição}
  \end{phonetics}
\end{entry}

\begin{entry}{完善}{7,12}[Radicais ⼧、⼝]
  \begin{phonetics}{完善}{wan2shan4}[][HSK 3]
    \definition{adj.}{perfeito; consumado}
    \definition{v.}{refinar; melhorar; tornar perfeito}
  \end{phonetics}
\end{entry}

\begin{entry}{完税}{7,12}[Radicais ⼧、⽲]
  \begin{phonetics}{完税}{wan2shui4}
    \definition{v.}{pagar imposto}
  \end{phonetics}
\end{entry}

\begin{entry}{完满}{7,13}[Radicais ⼧、⽔]
  \begin{phonetics}{完满}{wan2man3}
    \definition{adj.}{satisfatório | bem-sucedido}
  \end{phonetics}
\end{entry}

\begin{entry}{完整}{7,16}[Radicais ⼧、⽁]
  \begin{phonetics}{完整}{wan2zheng3}[][HSK 3]
    \definition{adj.}{intacto; inteiro; completo; integrado}
  \end{phonetics}
\end{entry}

\begin{entry}{尾巴}{7,4}[Radicais ⼫、⼰]
  \begin{phonetics}{尾巴}{wei3ba5}
    \definition{s.}{cauda}
  \end{phonetics}
\end{entry}

\begin{entry}{尿}{7}[Radical ⼫]
  \begin{phonetics}{尿}{niao4}
    \definition[泡]{s.}{urina}
    \definition{v.}{urinar}
  \end{phonetics}
  \begin{phonetics}{尿}{sui1}
    \definition{s.}{(coloquial) urina}
  \end{phonetics}
\end{entry}

\begin{entry}{局}{7}[Radical ⼫]
  \begin{phonetics}{局}{ju2}[][HSK 4]
    \definition{s.}{tabuleiro de xadrez | jogo; turno; \emph{set} | situação; estado das coisas | tolerância; grandeza ou pequenez da mente; grau de tolerância de uma pessoa em relação às outras | reunião de pessoas em festas | ardil; artidício; estratagema; armadilha | parte; porção; parcela | nome de determinadas lojas}
  \end{phonetics}
\end{entry}

\begin{entry}{屁股}{7,8}[Radicais ⼫、⾁]
  \begin{phonetics}{屁股}{pi4gu5}
    \definition{s.}{nádega | quadris}
  \end{phonetics}
\end{entry}

\begin{entry}{屁话}{7,8}[Radicais ⼫、⾔]
  \begin{phonetics}{屁话}{pi4hua4}
    \definition{s.}{absurdo | tolice | besteira}
  \end{phonetics}
\end{entry}

\begin{entry}{层}{7}[Radical ⼫]
  \begin{phonetics}{层}{ceng2}[][HSK 2]
    \definition{clas.}{para andar, piso}
  \end{phonetics}
\end{entry}

\begin{entry}{层次}{7,6}[Radicais ⼫、⽋]
  \begin{phonetics}{层次}{ceng2ci4}
    \definition{s.}{camada | nível | graduação | arranjo de ideias}
  \end{phonetics}
\end{entry}

\begin{entry}{层层}{7,7}[Radicais ⼫、⼫]
  \begin{phonetics}{层层}{ceng2ceng2}
    \definition{s.}{camada sobre camada}
  \end{phonetics}
\end{entry}

\begin{entry}{希望}{7,11}[Radicais ⼱、⽉]
  \begin{phonetics}{希望}{xi1wang4}[][HSK 3]
    \definition[个]{s.}{esperança; desejo; expectativa | aquilo em que a esperança é depositada}
    \definition{v.}{ter esperança; desejar; esperar}
  \end{phonetics}
\end{entry}

\begin{entry}{床}{7}[Radical ⼴]
  \begin{phonetics}{床}{chuang2}[][HSK 1]
    \definition{clas.}{para camas}
    \definition[张]{s.}{cama}
  \end{phonetics}
\end{entry}

\begin{entry}{应对}{7,5}[Radicais ⼴、⼨]
  \begin{phonetics}{应对}{ying4dui4}
    \definition{v.}{responder | manusear | lidar}
  \end{phonetics}
\end{entry}

\begin{entry}{应用}{7,5}[Radicais ⼴、⽤]
  \begin{phonetics}{应用}{ying4yong4}[][HSK 3]
    \definition{adj.}{aplicado (na vida ou na produção); usado diretamente na vida ou na produção}
    \definition{s.}{aplicativo}
    \definition{v.}{usar; aplicar}
  \end{phonetics}
\end{entry}

\begin{entry}{应用程序}{7,5,12,7}[Radicais ⼴、⽤、⽲、⼴]
  \begin{phonetics}{应用程序}{ying4yong4cheng2xu4}
    \definition{s.}{aplicativo | programa de computador}
  \end{phonetics}
\end{entry}

\begin{entry}{应用程序接口}{7,5,12,7,11,3}[Radicais ⼴、⽤、⽲、⼴、⼿、⼝]
  \begin{phonetics}{应用程序接口}{ying4yong4cheng2xu4jie1kou3}
    \definition{s.}{API (\emph{application programming interface})}
  \seealsoref{应用程序编程接口}{ying4yong4cheng2xu4bian1cheng2jie1kou3}
  \end{phonetics}
\end{entry}

\begin{entry*}{应用程序编程接口}{7,5,12,7,12,12,11,3}[Radicais ⼴、⽤、⽲、⼴、⽷、⽲、⼿、⼝]
  \begin{phonetics}{应用程序编程接口}{ying4yong4cheng2xu4bian1cheng2jie1kou3}
    \definition{s.}{API (\emph{application programming interface})}
  \seealsoref{应用程序接口}{ying4yong4cheng2xu4jie1kou3}
  \end{phonetics}
\end{entry*}

\begin{entry}{应当}{7,6}[Radicais ⼴、⼹]
  \begin{phonetics}{应当}{ying1 dang1}[][HSK 3]
    \definition{v.}{dever}
  \end{phonetics}
\end{entry}

\begin{entry}{应该}{7,8}[Radicais ⼴、⾔]
  \begin{phonetics}{应该}{ying1gai1}[][HSK 2]
    \definition{v.}{dever | ter de}
  \end{phonetics}
\end{entry}

\begin{entry}{弄}{7}[Radical ⼶]
  \begin{phonetics}{弄}{long4}
    \definition{s.}{beco | viela | travessa}
  \end{phonetics}
  \begin{phonetics}{弄}{nong4}[][HSK 2]
    \definition{s.}{beco | viela | travessa}
  \end{phonetics}
\end{entry}

\begin{entry}{弟}{7}[Radical ⼸]
  \begin{phonetics}{弟}{di4}[][HSK 1]
    \definition{s.}{irmão mais novo | júnior}
  \end{phonetics}
\end{entry}

\begin{entry}{弟弟}{7,7}[Radicais ⼸、⼸]
  \begin{phonetics}{弟弟}{di4 di5}[][HSK 1]
    \definition[个,位]{s.}{irmão mais novo}
  \end{phonetics}
\end{entry}

\begin{entry}{弟妹}{7,8}[Radicais ⼸、⼥]
  \begin{phonetics}{弟妹}{di4mei4}
    \definition{s.}{esposa do irmão mais novo}
  \end{phonetics}
\end{entry}

\begin{entry}{张}{7}[Radical ⼸]
  \begin{phonetics}{张}{zhang1}[][HSK 3]
    \definition*{s.}{Zhang, uma das mansões lunares | sobrenome Zhang}
    \definition{adj.}{nervoso; tenso}
    \definition{clas.}{para papel, couro, etc. | para camas, mesas, etc. | para a boca e o rosto | para arcos}
    \definition{s.}{folha de papel}
    \definition{v.}{consertar (uma corda de arco); encordoar (um instrumento musical ou um arco) | abrir; espalhar; esticar | expor; exibir |expandir; estender | ampliar; exagerar | olhar | dar rédea solta a; satisfazer | iniciar um negócio; abrir uma loja | colocar em bom uso; dar liberdade para | pegar com uma rede; montar armadilhas para capturar pássaros e animais}
  \end{phonetics}
\end{entry}

\begin{entry}{张三}{7,3}[Radicais ⼸、⼀]
  \begin{phonetics}{张三}{zhang1san1}
    \definition*{s.}{Zhang San | Zé Ninguém | nome para uma pessoa não especificada, 1 de 3}
  \seealsoref{李四}{li3si4}
  \seealsoref{王五}{wang2wu3}
  \end{phonetics}
\end{entry}

\begin{entry}{张狂}{7,7}[Radicais ⼸、⽝]
  \begin{phonetics}{张狂}{zhang1kuang2}
    \definition{adj.}{impetuoso | frenético | insolente}
  \end{phonetics}
\end{entry}

\begin{entry}{形式}{7,6}[Radicais ⼺、⼷]
  \begin{phonetics}{形式}{xing2shi4}[][HSK 3]
    \definition[种,个]{s.}{forma; formato; modalidade | aparência, estrutura ou estado de algo}
  \end{phonetics}
\end{entry}

\begin{entry}{形成}{7,6}[Radicais ⼺、⼽]
  \begin{phonetics}{形成}{xing2cheng2}[][HSK 3]
    \definition{v.}{moldar; formar; tomar forma | tornar-se algo ou algo através do desenvolvimento e da mudança}
  \end{phonetics}
\end{entry}

\begin{entry}{形而上学}{7,6,3,8}[Radicais ⼺、⽽、⼀、⼦]
  \begin{phonetics}{形而上学}{xing2'er2shang4xue2}
    \definition{s.}{metafísica}
  \end{phonetics}
\end{entry}

\begin{entry}{形状}{7,7}[Radicais ⼺、⽝]
  \begin{phonetics}{形状}{xing2zhuang4}[][HSK 3]
    \definition[个]{s.}{forma; aparência | a aparência de um objeto ou figura formada pela combinação de superfícies ou linhas externas}
  \end{phonetics}
\end{entry}

\begin{entry}{形容}{7,10}[Radicais ⼺、⼧]
  \begin{phonetics}{形容}{xing2rong2}
    \definition{s.}{descrever}
    \definition{s.}{semblante (literário) | aparência}
  \end{phonetics}
\end{entry}

\begin{entry}{形象}{7,11}[Radicais ⼺、⾗]
  \begin{phonetics}{形象}{xing2xiang4}[][HSK 3]
    \definition{adj.}{vívido}
    \definition[个]{s.}{imagem; forma; figura | uma forma ou gesto específico que pode despertar os pensamentos ou emoções das pessoas | imagem literária; imagem artística | pessoas ou coisas com características diferentes criadas na literatura, no cinema e em outras artes}
  \end{phonetics}
\end{entry}

\begin{entry}{彻底}{7,8}[Radicais ⼻、⼴]
  \begin{phonetics}{彻底}{che4di3}[][HSK 4]
    \definition{adj.}{minucioso; completo; exaustivo; profundo e completo; nada é deixado de fora}
  \end{phonetics}
\end{entry}

\begin{entry}{忍耐}{7,9}[Radicais ⼼、⽽]
  \begin{phonetics}{忍耐}{ren3nai4}
    \definition{s.}{paciência | resistência}
    \definition{v.}{suportar | resistir | exercer paciência}
  \end{phonetics}
\end{entry}

\begin{entry}{志愿}{7,14}[Radicais ⼼、⽕]
  \begin{phonetics}{志愿}{zhi4 yuan4}[][HSK 3]
    \definition{s.}{desejo; ideal; aspiração; meta que se espera alcançar}
    \definition{v.}{ser voluntário; tomar a iniciativa e esteja disposto a fazer um trabalho que não gere renda ou que tenha renda muito baixa, mas que possa ajudar outras pessoas}
  \end{phonetics}
\end{entry}

\begin{entry}{志愿者}{7,14,8}[Radicais ⼼、⽕、⽼]
  \begin{phonetics}{志愿者}{zhi4yuan4zhe3}[][HSK 3]
    \definition{s.}{voluntário; pessoa que se voluntaria para servir em atividades de assistência social, eventos de grande porte, conferências, etc.}
  \end{phonetics}
\end{entry}

\begin{entry}{忘}{7}[Radical ⼼]
  \begin{phonetics}{忘}{wang4}[][HSK 1]
    \definition{v.}{esquecer | negligenciar | ignorar}
  \end{phonetics}
\end{entry}

\begin{entry}{忘本}{7,5}[Radicais ⼼、⽊]
  \begin{phonetics}{忘本}{wang4ben3}
    \definition{v.}{esquecer as próprias raízes}
  \end{phonetics}
\end{entry}

\begin{entry}{忘记}{7,5}[Radicais ⼼、⾔]
  \begin{phonetics}{忘记}{wang4ji4}[][HSK 1]
    \definition{v.}{esquecer}
  \end{phonetics}
\end{entry}

\begin{entry}{忘却}{7,7}[Radicais ⼼、⼙]
  \begin{phonetics}{忘却}{wang4que4}
    \definition{v.}{esquecer}
  \end{phonetics}
\end{entry}

\begin{entry}{忘怀}{7,7}[Radicais ⼼、⼼]
  \begin{phonetics}{忘怀}{wang4huai2}
    \definition{v.}{esquecer}
  \end{phonetics}
\end{entry}

\begin{entry}{忘恩}{7,10}[Radicais ⼼、⼼]
  \begin{phonetics}{忘恩}{wang4'en1}
    \definition{v.}{ser ingrato}
  \end{phonetics}
\end{entry}

\begin{entry}{忘掉}{7,11}[Radicais ⼼、⼿]
  \begin{phonetics}{忘掉}{wang4diao4}
    \definition{v.}{esquecer}
  \end{phonetics}
\end{entry}

\begin{entry}{忘餐}{7,16}[Radicais ⼼、⾷]
  \begin{phonetics}{忘餐}{wang4can1}
    \definition{v.}{esquecer as refeições}
  \end{phonetics}
\end{entry}

\begin{entry}{忧郁}{7,8}[Radicais ⼼、⾢]
  \begin{phonetics}{忧郁}{you1yu4}
    \definition{adj.}{deprimido | melancólico | desanimado}
    \definition{s.}{depressão | melancolia}
  \end{phonetics}
\end{entry}

\begin{entry}{快}{7}[Radical ⼼]
  \begin{phonetics}{快}{kuai4}[][HSK 1]
    \definition{adj.}{quase | rápido | depressa}
    \definition{v.}{apressar-se}
  \end{phonetics}
\end{entry}

\begin{entry}{快乐}{7,5}[Radicais ⼼、⼃]
  \begin{phonetics}{快乐}{kuai4le4}[][HSK 2]
    \definition{adj.}{feliz | alegre}
    \definition{s.}{felicidade | alegria}
  \end{phonetics}
\end{entry}

\begin{entry}{快点儿}{7,9,2}[Radicais ⼼、⽕、⼉]
  \begin{phonetics}{快点儿}{kuai4 dian3r5}[][HSK 2]
    \definition{v.}{apressar-se}
  \end{phonetics}
\end{entry}

\begin{entry}{快要}{7,9}[Radicais ⼼、⾑]
  \begin{phonetics}{快要}{kuai4 yao4}[][HSK 2]
    \definition{adv.}{estar prestes a | estar indo para | estar à beira de | em breve | em nenhum momento}
  \end{phonetics}
\end{entry}

\begin{entry}{快递}{7,10}[Radicais ⼼、⾡]
  \begin{phonetics}{快递}{kuai4 di4}[][HSK 4]
    \definition[个]{s.}{correio rápido; entrega expressa; entrega rápida}
    \definition{v.}{entregar (serviço de entrega rápida por transportadoras especializadas)}
  \end{phonetics}
\end{entry}

\begin{entry}{快速}{7,10}[Radicais ⼼、⾡]
  \begin{phonetics}{快速}{kuai4 su4}[][HSK 3]
    \definition{adj.}{rápido; veloz; de alta velocidade}
  \end{phonetics}
\end{entry}

\begin{entry}{快餐}{7,16}[Radicais ⼼、⾷]
  \begin{phonetics}{快餐}{kuai4 can1}[][HSK 2]
    \definition[份,顿]{s.}{comida rápida | \emph{fast food}}
  \end{phonetics}
\end{entry}

\begin{entry}{怀旧}{7,5}[Radicais ⼼、⽇]
  \begin{phonetics}{怀旧}{huai2jiu4}
    \definition{s.}{nostalgia}
    \definition{v.}{sentir-se nostálgico}
  \end{phonetics}
\end{entry}

\begin{entry}{怀念}{7,8}[Radicais ⼼、⼼]
  \begin{phonetics}{怀念}{huai2nian4}[][HSK 4]
    \definition{v.}{pensar em; valorizar a memória de}
  \end{phonetics}
\end{entry}

\begin{entry}{怀疑}{7,14}[Radicais ⼼、⽦]
  \begin{phonetics}{怀疑}{huai2yi2}[][HSK 4]
    \definition{v.}{duvidar; suspeitar | supor}
  \end{phonetics}
\end{entry}

\begin{entry}{我}{7}[Radical ⼽]
  \begin{phonetics}{我}{wo3}[][HSK 1]
    \definition{pron.}{eu | me | mim | comigo}
  \end{phonetics}
\end{entry}

\begin{entry}{我们}{7,5}[Radicais ⼽、⼈]
  \begin{phonetics}{我们}{wo3men5}[][HSK 1]
    \definition{pron.}{nós | nos | conosco}
  \end{phonetics}
\end{entry}

\begin{entry}{我们的}{7,5,8}[Radicais ⼽、⼈、⽩]
  \begin{phonetics}{我们的}{wo3men5 de5}
    \definition{pron.}{nosso, nossos}
  \end{phonetics}
\end{entry}

\begin{entry}{我去}{7,5}[Radicais ⼽、⼛]
  \begin{phonetics}{我去}{wo3qu4}
    \definition{interj.}{(gíria) O que\dots!! | Oh meu Deus! | Isso é insano!}
  \end{phonetics}
\end{entry}

\begin{entry}{我的}{7,8}[Radicais ⼽、⽩]
  \begin{phonetics}{我的}{wo3 de5}
    \definition{pron.}{meu, meus}
  \end{phonetics}
\end{entry}

\begin{entry}{扶梯}{7,11}[Radicais ⼿、⽊]
  \begin{phonetics}{扶梯}{fu2ti1}
    \definition{s.}{escada rolante}
  \end{phonetics}
\end{entry}

\begin{entry}{批}{7}[Radical ⼿]
  \begin{phonetics}{批}{pi1}[][HSK 4]
    \definition{adj.}{(compra ou venda) atacado; a granel; em grandes quantidades}
    \definition{clas.}{para mercadorias a granel, grande número de pessoas}
    \definition{s.}{fibras de algodão, linho, etc., prontas para serem estiradas e torcidas | anotação; comentário}
    \definition{v.}{escrever comentários ou críticas sobre documentos subordinados, textos de outras pessoas, tarefas etc. | refutar; criticar | dar um tapa}
  \end{phonetics}
\end{entry}

\begin{entry}{批评}{7,7}[Radicais ⼿、⾔]
  \begin{phonetics}{批评}{pi1ping2}[][HSK 3]
    \definition{s.}{crítica}
    \definition{v.}{criticar; comentar sobre}
  \end{phonetics}
\end{entry}

\begin{entry}{批准}{7,10}[Radicais ⼿、⼎]
  \begin{phonetics}{批准}{pi1zhun3}[][HSK 3]
    \definition{v.}{aprovar}
  \end{phonetics}
\end{entry}

\begin{entry}{找}{7}[Radical ⼿]
  \begin{phonetics}{找}{zhao3}[][HSK 1]
    \definition{v.}{andar à procura de | procurar | tentar procurar | dar troco | retornar algo}
  \end{phonetics}
\end{entry}

\begin{entry}{找见}{7,4}[Radicais ⼿、⾒]
  \begin{phonetics}{找见}{zhao3jian4}
    \definition{v.}{encontrar (algo que está procurando)}
  \end{phonetics}
\end{entry}

\begin{entry}{找出}{7,5}[Radicais ⼿、⼐]
  \begin{phonetics}{找出}{zhao3 chu1}[][HSK 2]
    \definition{v.}{encontrar | procurar}
  \end{phonetics}
\end{entry}

\begin{entry}{找回}{7,6}[Radicais ⼿、⼞]
  \begin{phonetics}{找回}{zhao3hui2}
    \definition{v.}{recuperar algo}
  \end{phonetics}
\end{entry}

\begin{entry}{找寻}{7,6}[Radicais ⼿、⼨]
  \begin{phonetics}{找寻}{zhao3xun2}
    \definition{v.}{encontrar falhas | procurar | buscar}
  \end{phonetics}
\end{entry}

\begin{entry}{找事}{7,8}[Radicais ⼿、⼅]
  \begin{phonetics}{找事}{zhao3shi4}
    \definition{v.}{procurar emprego | começar uma briga}
  \end{phonetics}
\end{entry}

\begin{entry}{找到}{7,8}[Radicais ⼿、⼑]
  \begin{phonetics}{找到}{zhao3dao4}[][HSK 1]
    \definition{v.}{encontrar}
  \end{phonetics}
\end{entry}

\begin{entry}{找钱}{7,10}[Radicais ⼿、⾦]
  \begin{phonetics}{找钱}{zhao3qian2}
    \definition{v.}{dar troco}
  \end{phonetics}
\end{entry}

\begin{entry}{找着}{7,11}[Radicais ⼿、⽬]
  \begin{phonetics}{找着}{zhao3zhao2}
    \definition{v.}{encontrar}
  \end{phonetics}
\end{entry}

\begin{entry}{找遍}{7,12}[Radicais ⼿、⾡]
  \begin{phonetics}{找遍}{zhao3bian4}
    \definition{v.}{pentear | pesquisar em todos os lugares}
  \end{phonetics}
\end{entry}

\begin{entry}{找零}{7,13}[Radicais ⼿、⾬]
  \begin{phonetics}{找零}{zhao3ling2}
    \definition{v.}{trocar dinheiro | dar troco}
  \end{phonetics}
\end{entry}

\begin{entry}{找辙}{7,16}[Radicais ⼿、⾞]
  \begin{phonetics}{找辙}{zhao3zhe2}
    \definition{v.}{procurar um pretexto}
  \end{phonetics}
\end{entry}

\begin{entry}{技巧}{7,5}[Radicais ⼿、⼯]
  \begin{phonetics}{技巧}{ji4qiao3}[][HSK 4]
    \definition{s.}{habilidade; técnica; habilidades engenhosas expressas em artes, artesanato, esportes, etc.}
  \end{phonetics}
\end{entry}

\begin{entry}{技术}{7,5}[Radicais ⼿、⽊]
  \begin{phonetics}{技术}{ji4shu4}[][HSK 3]
    \definition[种,门,项]{s.}{habilidade; técnica; tecnologia}
  \end{phonetics}
\end{entry}

\begin{entry}{技俩}{7,9}[Radicais ⼿、⼈]
  \begin{phonetics}{技俩}{ji4liang3}
    \definition{s.}{truque | estratagema | ardil | esquema | estratégia | tática}
  \end{phonetics}
\end{entry}

\begin{entry}{抄}{7}[Radical ⼿]
  \begin{phonetics}{抄}{chao1}[][HSK 4]
    \definition*{s.}{sobrenome Chao}
    \definition{v.}{copiar; transcrever | plagiar | revistar e confiscar; fazer uma batida | pegar um atalho | dobrar (os braços) | agarrar; pegar}
  \end{phonetics}
\end{entry}

\begin{entry}{抄写}{7,5}[Radicais ⼿、⼍]
  \begin{phonetics}{抄写}{chao1 xie3}[][HSK 4]
    \definition{v.}{copiar; transcrever}
  \end{phonetics}
\end{entry}

\begin{entry}{把}{7}[Radical ⼿]
  \begin{phonetics}{把}{ba3}[][HSK 3]
    \definition{clas.}{para objetos com alça | para objetos pequenos:~punhado}
    \definition{part.}{partícula tornando o substantivo seguinte um objeto direto}
    \definition{v.}{conter | alcançar | segurar}
  \end{phonetics}
  \begin{phonetics}{把}{ba4}
    \definition{v.}{lidar}
  \end{phonetics}
\end{entry}

\begin{entry}{把风}{7,4}[Radicais ⼿、⾵]
  \begin{phonetics}{把风}{ba3feng1}
    \definition{v.}{estar atento | vigiar (durante uma atividade clandestina)}
  \end{phonetics}
\end{entry}

\begin{entry}{把关}{7,6}[Radicais ⼿、⼋]
  \begin{phonetics}{把关}{ba3guan1}
    \definition{v.}{verificar estritamente | examinar cuidadosamente para ver se algo é feito de acordo com um padrão fixo | fazer a verificação final | guardar uma passagem, fronteira}
  \end{phonetics}
\end{entry}

\begin{entry}{把守}{7,6}[Radicais ⼿、⼧]
  \begin{phonetics}{把守}{ba3shou3}
    \definition{v.}{vigiar | guardar}
  \end{phonetics}
\end{entry}

\begin{entry}{把式}{7,6}[Radicais ⼿、⼷]
  \begin{phonetics}{把式}{ba3shi4}
    \definition{s.}{pessoa qualificada em um comércio}
  \end{phonetics}
\end{entry}

\begin{entry}{把戏}{7,6}[Radicais ⼿、⼽]
  \begin{phonetics}{把戏}{ba3xi4}
    \definition{s.}{acrobacia | malabarismo | truque barato}
  \end{phonetics}
\end{entry}

\begin{entry}{把玩}{7,8}[Radicais ⼿、⽟]
  \begin{phonetics}{把玩}{ba3wan2}
    \definition{v.}{brincar com | mexer com}
  \end{phonetics}
\end{entry}

\begin{entry}{把持}{7,9}[Radicais ⼿、⼿]
  \begin{phonetics}{把持}{ba3chi2}
    \definition{v.}{controlar | dominar | monopolizar}
  \end{phonetics}
\end{entry}

\begin{entry}{把柄}{7,9}[Radicais ⼿、⽊]
  \begin{phonetics}{把柄}{ba3bing3}
    \definition{s.}{(figurativo) informações que podem ser usadas contra alguém}
  \end{phonetics}
\end{entry}

\begin{entry}{把脉}{7,9}[Radicais ⼿、⾁]
  \begin{phonetics}{把脉}{ba3mai4}
    \definition{v.}{sentir ou tomar o pulso de alguém}
  \end{phonetics}
\end{entry}

\begin{entry}{把握}{7,12}[Radicais ⼿、⼿]
  \begin{phonetics}{把握}{ba3wo4}[][HSK 3]
    \definition{s.}{seguro | garantia | certeza}
    \definition{v.}{agarrar | segurar | aproveitar}
  \end{phonetics}
\end{entry}

\begin{entry}{把稳}{7,14}[Radicais ⼿、⽲]
  \begin{phonetics}{把稳}{ba3wen3}
    \definition{adj.}{confiável}
  \end{phonetics}
\end{entry}

\begin{entry}{抓}{7}[Radical ⼿]
  \begin{phonetics}{抓}{zhua1}[][HSK 3]
    \definition{v.}{agarrar | arranhar | capturar | compreender; conhecer a chave ou a chave das coisas ou problemas | focar em algo; fortalecer o poder de fazer (algo) ou administrar (algum aspecto) | atrair a atenção de alguém}
  \end{phonetics}
\end{entry}

\begin{entry}{抓住}{7,7}[Radicais ⼿、⼈]
  \begin{phonetics}{抓住}{zhua1 zhu4}[][HSK 3]
    \definition{v.}{apanhar; prender; capturar (uma pessoa ou animal) e ter sucesso | segurar; agarrar; segurar algo e deixá-lo imóvel}
  \end{phonetics}
\end{entry}

\begin{entry}{投资}{7,10}[Radicais ⼿、⾙]
  \begin{phonetics}{投资}{tou2zi1}
    \definition{s.}{investimento}
    \definition{v.}{investir}
  \end{phonetics}
\end{entry}

\begin{entry}{投资人}{7,10,2}[Radicais ⼿、⾙、⼈]
  \begin{phonetics}{投资人}{tou2zi1ren2}
    \definition{s.}{investidor}
  \seealsoref{投资家}{tou2zi1jia1}
  \seealsoref{投资者}{tou2zi1zhe3}
  \end{phonetics}
\end{entry}

\begin{entry}{投资风险}{7,10,4,9}[Radicais ⼿、⾙、⾵、⾩]
  \begin{phonetics}{投资风险}{tou2zi1feng1xian3}
    \definition{s.}{risco de investimento}
  \end{phonetics}
\end{entry}

\begin{entry}{投资回报率}{7,10,6,7,11}[Radicais ⼿、⾙、⼞、⼿、⽞]
  \begin{phonetics}{投资回报率}{tou2zi1hui2bao4lv4}
    \definition{s.}{retorno sobre o investimento (ROI)}
  \end{phonetics}
\end{entry}

\begin{entry}{投资者}{7,10,8}[Radicais ⼿、⾙、⽼]
  \begin{phonetics}{投资者}{tou2zi1zhe3}
    \definition{s.}{investidor}
  \seealsoref{投资家}{tou2zi1jia1}
  \seealsoref{投资人}{tou2zi1ren2}
  \end{phonetics}
\end{entry}

\begin{entry}{投资家}{7,10,10}[Radicais ⼿、⾙、⼧]
  \begin{phonetics}{投资家}{tou2zi1jia1}
    \definition{s.}{investidor}
  \seealsoref{投资人}{tou2zi1ren2}
  \seealsoref{投资者}{tou2zi1zhe3}
  \end{phonetics}
\end{entry}

\begin{entry}{投递}{7,10}[Radicais ⼿、⾡]
  \begin{phonetics}{投递}{tou2di4}
    \definition{v.}{despachar | enviar}
  \end{phonetics}
\end{entry}

\begin{entry}{投票}{7,11}[Radicais ⼿、⽰]
  \begin{phonetics}{投票}{tou2piao4}
    \definition{v.+compl.}{votar | depositar um voto}
  \end{phonetics}
\end{entry}

\begin{entry}{折转}{7,8}[Radicais ⼿、⾞]
  \begin{phonetics}{折转}{zhe2zhuan3}
    \definition{s.}{reflexo (ângulo)}
    \definition{v.}{voltar atrás}
  \end{phonetics}
\end{entry}

\begin{entry}{抢掠}{7,11}[Radicais ⼿、⼿]
  \begin{phonetics}{抢掠}{qiang3lve4}
    \definition{s.}{saque | pilhagem}
    \definition{v.}{saquear | pilhar}
  \end{phonetics}
\end{entry}

\begin{entry}{护士}{7,3}[Radicais ⼿、⼠]
  \begin{phonetics}{护士}{hu4shi5}[][HSK 4]
    \definition[名,位]{s.}{enfermeiro; pessoas especializadas em enfermagem em hospitais ou instituições epidemiológicas}
  \end{phonetics}
\end{entry}

\begin{entry}{护照}{7,13}[Radicais ⼿、⽕]
  \begin{phonetics}{护照}{hu4zhao4}[][HSK 2]
    \definition[本,个]{s.}{passaporte}
  \end{phonetics}
\end{entry}

\begin{entry}{报}{7}[Radical ⼿]
  \begin{phonetics}{报}{bao4}[][HSK 3]
    \definition[份,张]{s.}{jornal | recompensa | relatório | vingança}
    \definition{v.}{anunciar | informar}
  \end{phonetics}
\end{entry}

\begin{entry}{报名}{7,6}[Radicais ⼿、⼝]
  \begin{phonetics}{报名}{bao4ming2}[][HSK 2]
    \definition{v.+compl.}{matricular-se | alistar-se | inscrever-se | inserir o nome de alguém}
  \end{phonetics}
\end{entry}

\begin{entry}{报告}{7,7}[Radicais ⼿、⼝]
  \begin{phonetics}{报告}{bao4gao4}[][HSK 3]
    \definition[份,篇,分,个,通]{s.}{relatório | discurso | palestra | aconselhamento}
    \definition{v.}{relatar | dar a conhecer | informar}
  \end{phonetics}
\end{entry}

\begin{entry}{报纸}{7,7}[Radicais ⼿、⽷]
  \begin{phonetics}{报纸}{bao4zhi3}[][HSK 2]
    \definition[张]{s.}{jornal | diário}
  \end{phonetics}
\end{entry}

\begin{entry}{报到}{7,8}[Radicais ⼿、⼑]
  \begin{phonetics}{报到}{bao4dao4}[][HSK 3]
    \definition{v.+compl.}{apresentar-se para o serviço | fazer check-in | registrar-se | assinar}
  \end{phonetics}
\end{entry}

\begin{entry}{报道}{7,12}[Radicais ⼿、⾡]
  \begin{phonetics}{报道}{bao4dao4}[][HSK 3]
    \definition[个,篇,分]{s.}{história | reportagem}
    \definition{v.}{cobrir | relatar (notícias)}
  \end{phonetics}
\end{entry}

\begin{entry}{报酬}{7,13}[Radicais ⼿、⾣]
  \begin{phonetics}{报酬}{bao4chou5}
    \definition{s.}{recompensa | remuneração}
  \end{phonetics}
\end{entry}

\begin{entry}{改}{7}[Radical ⽁]
  \begin{phonetics}{改}{gai3}[][HSK 2]
    \definition*{s.}{sobrenome Gai}
    \definition{v.}{mudar | transformar | revisar | alterar | modificar | retificar | corrigir | mudar para (fazer outra coisa)}
  \end{phonetics}
\end{entry}

\begin{entry}{改正}{7,5}[Radicais ⽁、⽌]
  \begin{phonetics}{改正}{gai3 zheng4}[][HSK 4]
    \definition{v.}{corrigir; emendar; mudar o errado para o correto}
  \end{phonetics}
\end{entry}

\begin{entry}{改良}{7,7}[Radicais ⽁、⾉]
  \begin{phonetics}{改良}{gai3liang2}
    \definition{v.}{melhorar (algo) | reformar (um sistema)}
  \end{phonetics}
\end{entry}

\begin{entry}{改进}{7,7}[Radicais ⽁、⾡]
  \begin{phonetics}{改进}{gai3jin4}[][HSK 3]
    \definition[个]{s.}{melhoria}
    \definition{v.}{aprimorar; aperfeiçoar; melhorar; tornar melhor
modificar}
  \end{phonetics}
\end{entry}

\begin{entry}{改变}{7,8}[Radicais ⽁、⼜]
  \begin{phonetics}{改变}{gai3bian4}[][HSK 2]
    \definition{v.}{mudar | alterar | transformar | virar | converter | moldar | modificar}
  \end{phonetics}
\end{entry}

\begin{entry}{改造}{7,10}[Radicais ⽁、⾡]
  \begin{phonetics}{改造}{gai3 zao4}[][HSK 3]
    \definition{v.}{transformar; renovar | remodelar}
  \end{phonetics}
\end{entry}

\begin{entry}{改善}{7,12}[Radicais ⽁、⼝]
  \begin{phonetics}{改善}{gai3shan4}[][HSK 4]
    \definition{v.}{melhorar; amenizar; mudar a situação original para torná-la melhor}
  \end{phonetics}
\end{entry}

\begin{entry}{改善关系}{7,12,6,7}[Radicais ⽁、⼝、⼋、⽷]
  \begin{phonetics}{改善关系}{gai3shan4guan1xi5}
    \definition{v.}{melhorar a relação}
  \end{phonetics}
\end{entry}

\begin{entry}{改善通讯}{7,12,10,5}[Radicais ⽁、⼝、⾡、⾔]
  \begin{phonetics}{改善通讯}{gai3shan4tong1xun4}
    \definition{v.}{melhorar a comunicação}
  \end{phonetics}
\end{entry}

\begin{entry}{时}{7}[Radical ⽇]
  \begin{phonetics}{时}{shi2}[][HSK 3]
    \definition*{s.}{sobrenome Shi}
    \definition{adj.}{atual; presente | a tempo; feito a tempo}
    \definition{adv.}{de vez em quando; ocasionalmente; de ​​tempos em tempos | às vezes\dots às vezes\dots}
    \definition{clas.}{hora; horas}
    \definition{s.}{dias; tempos; longo período de tempo | tempo; tempo fixo | hora; hora do dia | temporada | chance; oportunidade | atualidade; presente | tempo verbal}
  \end{phonetics}
\end{entry}

\begin{entry}{时代}{7,5}[Radicais ⽇、⼈]
  \begin{phonetics}{时代}{shi2dai4}[][HSK 3]
    \definition[个]{s.}{idade; era; tempos; época | um período na vida de alguém}
  \end{phonetics}
\end{entry}

\begin{entry}{时光}{7,6}[Radicais ⽇、⼉]
  \begin{phonetics}{时光}{shi2guang1}
    \definition{s.}{tempo | época | período de tempo}
  \end{phonetics}
\end{entry}

\begin{entry}{时时}{7,7}[Radicais ⽇、⽇]
  \begin{phonetics}{时时}{shi2shi2}
    \definition{adv.}{muitas vezes | constantemente}
  \end{phonetics}
\end{entry}

\begin{entry}{时间}{7,7}[Radicais ⽇、⾨]
  \begin{phonetics}{时间}{shi2jian1}[][HSK 1]
    \definition{s.}{(conceito de, duração de, um ponto no) tempo}
  \end{phonetics}
\end{entry}

\begin{entry}{时刻}{7,8}[Radicais ⽇、⼑]
  \begin{phonetics}{时刻}{shi2ke4}[][HSK 3]
    \definition{adv.}{constantemente; sempre}
    \definition[个,段]{s.}{tempo; hora; momento; conjuntura}
  \end{phonetics}
\end{entry}

\begin{entry}{时差}{7,9}[Radicais ⽇、⼯]
  \begin{phonetics}{时差}{shi2cha1}
    \definition{s.}{diferença de tempo | \emph{jet lag}}
  \end{phonetics}
\end{entry}

\begin{entry}{时候}{7,10}[Radicais ⽇、⼈]
  \begin{phonetics}{时候}{shi2hou5}[][HSK 1]
    \definition{adv.}{quando?}
    \definition{s.}{duração de tempo | momento | período | tempo}
  \end{phonetics}
\end{entry}

\begin{entry}{旷野}{7,11}[Radicais ⽇、⾥]
  \begin{phonetics}{旷野}{kuang4ye3}
    \definition{s.}{região selvagem}
  \end{phonetics}
\end{entry}

\begin{entry}{更}{7}[Radical ⽈]
  \begin{phonetics}{更}{geng1}
    \definition{s.}{vigia (por exemplo, de sentinela ou guarda)}
    \definition{v.}{alterar ou substituir | experimentar}
  \end{phonetics}
  \begin{phonetics}{更}{geng4}[][HSK 2]
    \definition{adv.}{mais | ainda mais}
  \end{phonetics}
\end{entry}

\begin{entry}{更加}{7,5}[Radicais ⽈、⼒]
  \begin{phonetics}{更加}{geng4 jia1}[][HSK 3]
    \definition{adv.}{mais; ainda mais; em maior grau}
  \end{phonetics}
\end{entry}

\begin{entry}{李}{7}[Radical ⽊]
  \begin{phonetics}{李}{li3}
    \definition*{s.}{sobrenome Li}
    \definition{s.}{ameixa}
  \end{phonetics}
\end{entry}

\begin{entry}{李子}{7,3}[Radicais ⽊、⼦]
  \begin{phonetics}{李子}{li3zi5}
    \definition[个]{s.}{ameixa}
  \end{phonetics}
\end{entry}

\begin{entry}{李四}{7,5}[Radicais ⽊、⼞]
  \begin{phonetics}{李四}{li3si4}
    \definition*{s.}{Li Si | Zé Ninguém | nome para uma pessoa não especificada, 2 de 3}
  \seealsoref{王五}{wang2wu3}
  \seealsoref{张三}{zhang1san1}
  \end{phonetics}
\end{entry}

\begin{entry}{材料}{7,10}[Radicais ⽊、⽃]
  \begin{phonetics}{材料}{cai2liao4}[][HSK 4]
    \definition[份,个,种]{s.}{material; algo para fazer um produto acabado | material (figura de linguagem) | dados; material para estudo, pesquisa, etc.; conteúdo de uma obra}
  \end{phonetics}
\end{entry}

\begin{entry}{村}{7}[Radical ⽊]
  \begin{phonetics}{村}{cun1}[][HSK 3]
    \definition{adj.}{rústico; grosseiro}
    \definition{s.}{aldeia; vila}
  \end{phonetics}
\end{entry}

\begin{entry}{杜宇}{7,6}[Radicais ⽊、⼧]
  \begin{phonetics}{杜宇}{du4yu3}
    \definition{s.}{cuco (pássaro)}
  \seealsoref{布谷鸟}{bu4gu3niao3}
  \seealsoref{杜鹃}{du4juan1}
  \seealsoref{杜鹃鸟}{du4juan1niao3}
  \end{phonetics}
\end{entry}

\begin{entry}{杜鹃}{7,12}[Radicais ⽊、⿃]
  \begin{phonetics}{杜鹃}{du4juan1}
    \definition{s.}{cuco (pássaro)}
  \seealsoref{布谷鸟}{bu4gu3niao3}
  \seealsoref{杜鹃鸟}{du4juan1niao3}
  \seealsoref{杜宇}{du4yu3}
  \end{phonetics}
\end{entry}

\begin{entry}{杜鹃鸟}{7,12,5}[Radicais ⽊、⿃、⿃]
  \begin{phonetics}{杜鹃鸟}{du4juan1niao3}
    \definition{s.}{cuco (pássaro)}
  \seealsoref{布谷鸟}{bu4gu3niao3}
  \seealsoref{杜鹃}{du4juan1}
  \seealsoref{杜宇}{du4yu3}
  \end{phonetics}
\end{entry}

\begin{entry}{束}{7}[Radical ⽊]
  \begin{phonetics}{束}{shu4}[][HSK 3]
    \definition*{s.}{sobrenome Shu}
    \definition{clas.}{para cachos, molhos, feixes, feixes de luz, etc.}
    \definition{s.}{monte; pacote; maço; feixe; cacho}
    \definition{v.}{atar; amarrar; vincular | controlar; restringir}
  \end{phonetics}
\end{entry}

\begin{entry}{束腰}{7,13}[Radicais ⽊、⾁]
  \begin{phonetics}{束腰}{shu4yao1}
    \definition{s.}{cinto | cinta | cinturão}
  \end{phonetics}
\end{entry}

\begin{entry}{杠}{7}[Radical ⽊]
  \begin{phonetics}{杠}{gang1}
    \definition{s.}{mastro de bandeira | poste | passarela}
  \end{phonetics}
  \begin{phonetics}{杠}{gang4}
    \definition{s.}{vara grossa | barra | linha grossa | padrão, critério | hífen, traço}
    \definition{v.}{marcar com uma linha grossa | afiar (faca, navalha, etc.)}
  \end{phonetics}
\end{entry}

\begin{entry}{条}{7}[Radical ⽊]
  \begin{phonetics}{条}{tiao2}[][HSK 2]
    \definition{clas.}{para coisas longas e finas (fita, rio, estrada, calças, etc.)}
    \definition{s.}{artigo | cláusula (de lei ou tratado) | item | faixa}
  \end{phonetics}
\end{entry}

\begin{entry}{条目}{7,5}[Radicais ⽊、⽬]
  \begin{phonetics}{条目}{tiao2mu4}
    \definition{s.}{cláusulas e subcláusulas (em documento formal) | verbete (em um dicionário, enciclopédia, etc.)}
  \end{phonetics}
\end{entry}

\begin{entry}{条件}{7,6}[Radicais ⽊、⼈]
  \begin{phonetics}{条件}{tiao2jian4}[][HSK 2]
    \definition[个]{s.}{circunstâncias | condição | fator | pré-requisito | qualificação | requisito}
  \end{phonetics}
\end{entry}

\begin{entry}{条例}{7,8}[Radicais ⽊、⼈]
  \begin{phonetics}{条例}{tiao2li4}
    \definition{s.}{código de conduta | ordenanças | regulamentos | regras | estatutos}
  \end{phonetics}
\end{entry}

\begin{entry}{条贯}{7,8}[Radicais ⽊、⾙]
  \begin{phonetics}{条贯}{tiao2guan4}
    \definition{s.}{ordem | procedimentos | sequência | sistema}
  \end{phonetics}
\end{entry}

\begin{entry}{条幅}{7,12}[Radicais ⽊、⼱]
  \begin{phonetics}{条幅}{tiao2fu2}
    \definition{s.}{faixa | banner | pergaminho de parede (para pintura ou caligrafia)}
  \end{phonetics}
\end{entry}

\begin{entry}{来}{7}[Radical ⽊]
  \begin{phonetics}{来}{lai2}[][HSK 1]
    \definition{v.}{vir | chegar | trazer}
  \end{phonetics}
\end{entry}

\begin{entry}{来不及}{7,4,3}[Radicais ⽊、⼀、⼃]
  \begin{phonetics}{来不及}{lai2bu5ji2}[][HSK 4]
    \definition{v.}{ser tarde demais; não ter tempo; não ter tempo suficiente (para fazer algo); não ser possível participar ou se atualizar devido a restrições de tempo}
  \end{phonetics}
\end{entry}

\begin{entry}{来自}{7,6}[Radicais ⽊、⾃]
  \begin{phonetics}{来自}{lai2zi4}[][HSK 2]
    \definition{v.}{vir de (um local) | \emph{From:} (cabeçalho de \emph{e -mail})}
  \end{phonetics}
\end{entry}

\begin{entry}{来到}{7,8}[Radicais ⽊、⼑]
  \begin{phonetics}{来到}{lai2 dao4}[][HSK 1]
    \definition{v.}{chegar | vir}
  \end{phonetics}
\end{entry}

\begin{entry}{来得及}{7,11,3}[Radicais ⽊、⼻、⼃]
  \begin{phonetics}{来得及}{lai2de5ji2}[][HSK 4]
    \definition{v.}{ainda ter tempo; ser capaz de fazê-lo; ser capaz de fazer algo a tempo; ainda ter tempo de chegar lá ou de se atualizar}
  \end{phonetics}
\end{entry}

\begin{entry}{来源}{7,13}[Radicais ⽊、⽔]
  \begin{phonetics}{来源}{lai2yuan2}[][HSK 4]
    \definition{s.}{origem; causa; fonte; tabula rasa (ou seja, o lugar de onde as coisas vêm)}
    \definition{v.}{originar-se; surgir; ter origem; (algo) originar (seguido de ``于'')}
  \seealsoref{于}{yu2}
  \end{phonetics}
\end{entry}

\begin{entry}{极}{7}[Radical ⽊]
  \begin{phonetics}{极}{ji2}[][HSK 4]
    \definition*{s.}{sobrenome Ji}
    \definition{adj.}{máximo; extremo; final; supremo}
    \definition{adv.}{extremamente; excessivamente}
    \definition{s.}{o ponto máximo, mais alto; extremo; ápice; ponto culminante |
pólo; as extremidades norte e sul da Terra; as extremidades de um ímã; a extremidade de uma fonte de alimentação ou de um aparelho elétrico onde a corrente entra ou sai do aparelho}
    \definition{v.}{chegar ao fim de; levar a extremos}
  \end{phonetics}
\end{entry}

\begin{entry}{……极了}{7,2}[Radicais ⽊、⼅]
  \begin{phonetics}{……极了}{ji2le5}[][HSK 3]
    \definition{expr.}{extremamente}
  \end{phonetics}
\end{entry}

\begin{entry}{极其}{7,8}[Radicais ⽊、⼋]
  \begin{phonetics}{极其}{ji2qi2}[][HSK 4]
    \definition{adv.}{mais; extremamente; excessivamente}
  \end{phonetics}
\end{entry}

\begin{entry}{步}{7}[Radical ⽌]
  \begin{phonetics}{步}{bu4}[][HSK 3]
    \definition*{s.}{sobrenome Bu}
    \definition{clas.}{uma unidade antiga para medida de comprimento, equivalente a cinco chi}
    \definition{s.}{ritmo | passo | estágio | passo | condição | situação | estado}
    \definition{v.}{ir a pé | andar | pisar | contar passos}
  \end{phonetics}
\end{entry}

\begin{entry}{步行}{7,6}[Radicais ⽌、⾏]
  \begin{phonetics}{步行}{bu4 xing2}[][HSK 4]
    \definition{v.}{caminhar; ir a pé; andar a pé (diferente de andar de carro, a cavalo, etc.)}
  \end{phonetics}
\end{entry}

\begin{entry}{每}{7}[Radical ⽏]
  \begin{phonetics}{每}{mei3}[][HSK 3]
    \definition{adv.}{frequentemente; todo}
    \definition{pron.}{cada; cada um; cada qual;  todo}
    \definition{s.}{sobrenome Mei}
  \end{phonetics}
\end{entry}

\begin{entry}{每个人}{7,3,2}[Radicais ⽏、⼈、⼈]
  \begin{phonetics}{每个人}{mei3ge5ren2}
    \definition{pron.}{todo mundo | todos}
  \end{phonetics}
\end{entry}

\begin{entry}{每天}{7,4}[Radicais ⽏、⼤]
  \begin{phonetics}{每天}{mei3tian1}
    \definition{adv.}{todo dia | cada dia}
  \end{phonetics}
\end{entry}

\begin{entry}{每次}{7,6}[Radicais ⽏、⽋]
  \begin{phonetics}{每次}{mei3ci4}
    \definition{adv.}{toda vez | cada vez}
  \end{phonetics}
\end{entry}

\begin{entry}{求}{7}[Radical ⽔]
  \begin{phonetics}{求}{qiu2}[][HSK 2]
    \definition*{s.}{sobrenome Qiu}
    \definition{s.}{demanda}
    \definition{v.}{pedir | implorar | solicitar | suplicar | esforçar-se por | procurar | tentar}
  \end{phonetics}
\end{entry}

\begin{entry}{汹涌}{7,10}[Radicais ⽔、⽔]
  \begin{phonetics}{汹涌}{xiong1yong3}
    \definition{adj.}{turbulento}
    \definition{v.}{aumentar ou emergir violentamente (oceano, rio, lago, etc.)}
  \end{phonetics}
\end{entry}

\begin{entry}{汽水}{7,4}[Radicais ⽔、⽔]
  \begin{phonetics}{汽水}{qi4 shui3}[][HSK 4]
    \definition[罐,瓶]{s.}{refrigerante; refrigerante gaseificado; bebida refrescante, feita com a pressão de dióxido de carbono para dissolver na água e adicionar açúcar, suco de frutas, especiarias etc.}
  \end{phonetics}
\end{entry}

\begin{entry}{汽车}{7,4}[Radicais ⽔、⾞]
  \begin{phonetics}{汽车}{qi4che1}[][HSK 1]
    \definition[辆]{s.}{automóvel | carro | veículo motorizado}
  \end{phonetics}
\end{entry}

\begin{entry}{汽油}{7,8}[Radicais ⽔、⽔]
  \begin{phonetics}{汽油}{qi4you2}[][HSK 4]
    \definition{s.}{gasolina; mistura líquida de hidrocarbonetos com volatilidade e combustibilidade, que é usada como combustível a partir do fracionamento ou craqueamento do petróleo}
  \end{phonetics}
\end{entry}

\begin{entry}{沉}{7}[Radical ⽔]
  \begin{phonetics}{沉}{chen2}[][HSK 4]
    \definition{adj.}{profundo | pesado | pesado (sentir-se pesado)}
    \definition{v.}{afundar; submergir; imergir | manter baixo; abaixar | descansar; parar}
  \end{phonetics}
\end{entry}

\begin{entry}{沉重}{7,9}[Radicais ⽔、⾥]
  \begin{phonetics}{沉重}{chen2zhong4}[][HSK 4]
    \definition{adj.}{(pressão, fardo, etc.) muito pesado; profundo | sério; pesado; humor pouco animador; fardo pesado de pensamentos}
  \end{phonetics}
\end{entry}

\begin{entry}{沉默}{7,16}[Radicais ⽔、⿊]
  \begin{phonetics}{沉默}{chen2mo4}[][HSK 4]
    \definition{adj.}{silencioso; reticente; taciturno; não comunicativo}
    \definition{v.}{silenciar; não falar por causa de alguma coisa}
  \end{phonetics}
\end{entry}

\begin{entry}{沙}{7}[Radical ⽔]
  \begin{phonetics}{沙}{sha1}
    \definition*{s.}{sobrenome Sha}
    \definition[粒]{s.}{areia | cascalho | grânulo | pó}
  \end{phonetics}
\end{entry}

\begin{entry}{沙子}{7,3}[Radicais ⽔、⼦]
  \begin{phonetics}{沙子}{sha1 zi5}[][HSK 3]
    \definition[粒,把]{s.}{areia; grão | \emph{pellets}; grãos pequenos}
  \end{phonetics}
\end{entry}

\begin{entry}{沙发}{7,5}[Radicais ⽔、⼜]
  \begin{phonetics}{沙发}{sha1fa1}[][HSK 3]
    \definition[套,组,个,张]{s.}{sofá; divã}
  \end{phonetics}
\end{entry}

\begin{entry}{沙鱼}{7,8}[Radicais ⽔、⿂]
  \begin{phonetics}{沙鱼}{sha1yu2}
    \variantof{鲨鱼}
  \end{phonetics}
\end{entry}

\begin{entry}{沙特}{7,10}[Radicais ⽔、⽜]
  \begin{phonetics}{沙特}{sha1te4}
    \definition*{s.}{Saudita | abreviação de 沙特阿拉伯}
    \seeref{沙特阿拉伯}{sha1te4 a1la1bo2}
  \end{phonetics}
\end{entry}

\begin{entry}{沙特阿拉伯}{7,10,7,8,7}[Radicais ⽔、⽜、⾩、⼿、⼈]
  \begin{phonetics}{沙特阿拉伯}{sha1te4 a1la1bo2}
    \definition*{s.}{Arábia Saudita}
  \end{phonetics}
\end{entry}

\begin{entry}{沙漠}{7,13}[Radicais ⽔、⽔]
  \begin{phonetics}{沙漠}{sha1mo4}
    \definition[个]{s.}{deserto}
  \end{phonetics}
\end{entry}

\begin{entry}{没}{7}[Radical ⽔]
  \begin{phonetics}{没}{mei2}[][HSK 1]
    \definition{adv.}{não ter | não há | ficar sem}
    \definition{pref.}{não (prefixo negativo para verbos, traduzido para outras línguas com verbos no pretérito)}
  \end{phonetics}
  \begin{phonetics}{没}{mo4}
    \definition{adj.}{afogado}
    \definition{v.}{acabar | morrer | inundar}
    \variantof{没}
  \end{phonetics}
\end{entry}

\begin{entry}{没了}{7,2}[Radicais ⽔、⼅]
  \begin{phonetics}{没了}{mei2le5}
    \definition{v.}{estar morto | deixar de existir}
  \end{phonetics}
\end{entry}

\begin{entry}{没什么}{7,4,3}[Radicais ⽔、⼈、⼃]
  \begin{phonetics}{没什么}{mei2 shen2me5}[][HSK 1]
    \definition{expr.}{não é nada | está tudo bem | não importa | não importa}
  \end{phonetics}
\end{entry}

\begin{entry}{没用}{7,5}[Radicais ⽔、⽤]
  \begin{phonetics}{没用}{mei2 yong4}[][HSK 3]
    \definition{adj.}{inútil; imprestável; sem valor; sem préstimo; vão; que não serve para nada}
  \end{phonetics}
\end{entry}

\begin{entry}{没关系}{7,6,7}[Radicais ⽔、⼋、⽷]
  \begin{phonetics}{没关系}{mei2 guan1xi5}[][HSK 1]
    \definition{v.}{não ter problema | não ter importância | não fazer mal}
    \seeref{没有关系}{mei2you3guan1xi5}
  \end{phonetics}
\end{entry}

\begin{entry}{没有}{7,6}[Radicais ⽔、⽉]
  \begin{phonetics}{没有}{mei2you3}[][HSK 1]
    \definition{v.}{não há | não tem | não existe}
  \end{phonetics}
\end{entry}

\begin{entry}{没有关系}{7,6,6,7}[Radicais ⽔、⽉、⼋、⽷]
  \begin{phonetics}{没有关系}{mei2you3guan1xi5}
    \definition{v.}{não ter problema | não ter importância | não fazer mal}
    \seeref{没关系}{mei2 guan1xi5}
  \end{phonetics}
\end{entry}

\begin{entry}{没有意思}{7,6,13,9}[Radicais ⽔、⽉、⼼、⼼]
  \begin{phonetics}{没有意思}{mei2you3yi4si5}
    \definition{adj.}{tedioso | chato | sem interesse}
  \end{phonetics}
\end{entry}

\begin{entry}{没事儿}{7,8,2}[Radicais ⽔、⼅、⼉]
  \begin{phonetics}{没事儿}{mei2 shi4r5}[][HSK 1]
    \definition{expr.}{livre de trabalho | sem problemas | não é importante |não é nada |deixa para lá}
    \definition{v.}{ter tempo livre}
  \end{phonetics}
\end{entry}

\begin{entry}{没法儿}{7,8,2}[Radicais ⽔、⽔、⼉]
  \begin{phonetics}{没法儿}{mei2 fa3r5}[][HSK 4]
    \definition{adv.}{não pode; sem chance}
  \end{phonetics}
\end{entry}

\begin{entry}{没想到}{7,13,8}[Radicais ⽔、⼼、⼑]
  \begin{phonetics}{没想到}{mei2 xiang3 dao4}[][HSK 4]
    \definition{expr.}{não esperava; inesperado}
  \end{phonetics}
\end{entry}

\begin{entry}{没错}{7,13}[Radicais ⽔、⾦]
  \begin{phonetics}{没错}{mei2 cuo4}[][HSK 4]
    \definition{adv.}{está certo; é isso mesmo; não há como errar}
  \end{phonetics}
\end{entry}

\begin{entry}{灵感}{7,13}[Radicais ⽕、⼼]
  \begin{phonetics}{灵感}{ling2gan3}
    \definition{s.}{inspiração | explosão de criatividade em empreendimento científico ou artístico}
  \end{phonetics}
\end{entry}

\begin{entry}{灵魂}{7,13}[Radicais ⽕、⿁]
  \begin{phonetics}{灵魂}{ling2hun2}
    \definition{s.}{alma | espírito}
  \end{phonetics}
\end{entry}

\begin{entry}{灶台}{7,5}[Radicais ⽕、⼝]
  \begin{phonetics}{灶台}{zao4tai2}
    \definition{s.}{fogão}
  \end{phonetics}
\end{entry}

\begin{entry}{状况}{7,7}[Radicais ⽝、⼎]
  \begin{phonetics}{状况}{zhuang4kuang4}[][HSK 3]
    \definition[个,种]{s.}{estado; \emph{status}; condição; estado de coisas}
  \end{phonetics}
\end{entry}

\begin{entry}{状态}{7,8}[Radicais ⽝、⼼]
  \begin{phonetics}{状态}{zhuang4tai4}[][HSK 3]
    \definition[种,个]{s.}{\emph{status}; estado; condição; estado de coisas; a forma em que uma pessoa ou coisa aparece}
  \end{phonetics}
\end{entry}

\begin{entry}{狂欢节}{7,6,5}[Radicais ⽝、⽋、⾋]
  \begin{phonetics}{狂欢节}{kuang2huan1jie2}
    \definition*{s.}{Carnaval}
  \end{phonetics}
\end{entry}

\begin{entry}{男}{7}[Radical ⽥]
  \begin{phonetics}{男}{nan2}[][HSK 1]
    \definition{adj.}{masculino}
    \definition{s.}{Barão, o mais baixo de cinco ordens de nobreza}
  \end{phonetics}
\end{entry}

\begin{entry}{男人}{7,2}[Radicais ⽥、⼈]
  \begin{phonetics}{男人}{nan2ren2}[][HSK 1]
    \definition[个]{s.}{um homem | um macho | cavalheiro | marido}
  \end{phonetics}
\end{entry}

\begin{entry}{男士}{7,3}[Radicais ⽥、⼠]
  \begin{phonetics}{男士}{nan2 shi4}[][HSK 4]
    \definition{s.}{cavalheiro; \emph{gentleman}}
  \end{phonetics}
\end{entry}

\begin{entry}{男女}{7,3}[Radicais ⽥、⼥]
  \begin{phonetics}{男女}{nan2 nv3}[][HSK 4]
    \definition{s.}{homens e mulheres; masculino e feminino}
  \end{phonetics}
\end{entry}

\begin{entry}{男子}{7,3}[Radicais ⽥、⼦]
  \begin{phonetics}{男子}{nan2zi3}[][HSK 3]
    \definition[名]{s.}{homem; macho}
  \end{phonetics}
\end{entry}

\begin{entry}{男生}{7,5}[Radicais ⽥、⽣]
  \begin{phonetics}{男生}{nan2sheng1}[][HSK 1]
    \definition[个]{s.}{aluno | estudante do sexo masculino}
  \end{phonetics}
\end{entry}

\begin{entry}{男朋友}{7,8,4}[Radicais ⽥、⽉、⼜]
  \begin{phonetics}{男朋友}{nan2peng2you5}[][HSK 1]
    \definition{s.}{namorado}
  \end{phonetics}
\end{entry}

\begin{entry}{男孩儿}{7,9,2}[Radicais ⽥、⼦、⼉]
  \begin{phonetics}{男孩儿}{nan2hai2r5}[][HSK 1]
    \definition{s.}{menino | rapaz}
  \end{phonetics}
\end{entry}

\begin{entry}{疗养}{7,9}[Radicais ⽧、⼋]
  \begin{phonetics}{疗养}{liao2 yang3}[][HSK 4]
    \definition{v.}{recuperar; convalescer; tratar pessoas com doenças crônicas ou debilitantes em instituições médicas especializadas com foco na recuperação}
  \end{phonetics}
\end{entry}

\begin{entry}{社会}{7,6}[Radicais ⽰、⼈]
  \begin{phonetics}{社会}{she4hui4}[][HSK 3]
    \definition[个,种]{s.}{sociedade | comunidade}
  \end{phonetics}
\end{entry}

\begin{entry}{私人}{7,2}[Radicais ⽲、⼈]
  \begin{phonetics}{私人}{si1ren2}
    \definition{adj.}{privado | interpessoal}
    \definition[些]{s.}{alguém com quem se tem um relacionamento pessoal próximo}
  \end{phonetics}
\end{entry}

\begin{entry}{私人诊所}{7,2,7,8}[Radicais ⽲、⼈、⾔、⼾]
  \begin{phonetics}{私人诊所}{si1ren2 zhen3suo3}
    \definition[些]{s.}{clínica privada}
  \end{phonetics}
\end{entry}

\begin{entry}{私人信件}{7,2,9,6}[Radicais ⽲、⼈、⼈、⼈]
  \begin{phonetics}{私人信件}{si1ren2 xin4jian4}
    \definition{s.}{carta pessoal}
  \end{phonetics}
\end{entry}

\begin{entry}{私人钥匙}{7,2,9,11}[Radicais ⽲、⼈、⾦、⼔]
  \begin{phonetics}{私人钥匙}{si1ren2yao4shi5}
    \definition{s.}{(criptografia) chave privada}
  \end{phonetics}
\end{entry}

\begin{entry}{私生活}{7,5,9}[Radicais ⽲、⽣、⽔]
  \begin{phonetics}{私生活}{si1sheng1huo2}
    \definition{s.}{vida privada}
  \end{phonetics}
\end{entry}

\begin{entry}{私自}{7,6}[Radicais ⽲、⾃]
  \begin{phonetics}{私自}{si1zi4}
    \definition{adj.}{privado | pessoal}
    \definition{adv.}{secretamente | sem aprovação explícita}
  \end{phonetics}
\end{entry}

\begin{entry}{究竟}{7,11}[Radicais ⽳、⾳]
  \begin{phonetics}{究竟}{jiu1jing4}[][HSK 4]
    \definition{adv.}{de fato; exatamente; usado em frases interrogativas para buscar | afinal de contas, no final; ênfase em fatos ou motivos}
    \definition{s.}{resultado; desfecho; a causa, o efeito ou a história completa do que aconteceu}
  \end{phonetics}
\end{entry}

\begin{entry}{穷}{7}[Radical ⽳]
  \begin{phonetics}{穷}{qiong2}[][HSK 4]
    \definition{adj.}{remoto; isolado; de difícil acesso | pobre; atingido pela pobreza | situação difícil, sem saída}
    \definition{adv.}{completamente | extremamente}
    \definition{v.}{exaurir; esgotar; consmir | ir até o fim; perseguir completamente perseguido; sondar profundamente | gastar}
  \end{phonetics}
\end{entry}

\begin{entry}{穷人}{7,2}[Radicais ⽳、⼈]
  \begin{phonetics}{穷人}{qiong2 ren2}[][HSK 4]
    \definition{s.}{os pobres; pessoas pobres}
  \end{phonetics}
\end{entry}

\begin{entry}{系}{7}[Radical ⽷]
  \begin{phonetics}{系}{ji4}
    \definition{v.}{amarrar; prender; abotoar}
  \end{phonetics}
  \begin{phonetics}{系}{xi4}[][HSK 3,4]
    \definition*{s.}{sobrenome Xi}
    \definition{s.}{faculdade (da universidade) | departamento}
    \definition{v.}{sistema; série | departamento; faculdade}
    \definition{v.}{relacionar-se com; suportar; depender de | sentir-se ansioso; estar preocupado | amarrar; prender | ser}
  \end{phonetics}
\end{entry}

\begin{entry}{系囚}{7,5}[Radicais ⽷、⼞]
  \begin{phonetics}{系囚}{xi4qiu2}
    \definition{s.}{prisioneiro}
  \end{phonetics}
\end{entry}

\begin{entry}{系列}{7,6}[Radicais ⽷、⼑]
  \begin{phonetics}{系列}{xi4lie4}
    \definition{s.}{série | conjunto}
  \end{phonetics}
\end{entry}

\begin{entry}{系统}{7,9}[Radicais ⽷、⽷]
  \begin{phonetics}{系统}{xi4tong3}
    \definition[个]{s.}{sistema}
  \end{phonetics}
\end{entry}

\begin{entry}{纯}{7}[Radical ⽷]
  \begin{phonetics}{纯}{chun2}[][HSK 4]
    \definition{adj.}{puro; não misturado; livre de impurezas | simples; puro e simples | habilidoso; proficiente; bem versado}
  \end{phonetics}
\end{entry}

\begin{entry}{纯净水}{7,8,4}[Radicais ⽷、⼎、⽔]
  \begin{phonetics}{纯净水}{chun2 jing4 shui3}[][HSK 4]
    \definition{s.}{água purificada}
  \end{phonetics}
\end{entry}

\begin{entry}{纯真}{7,10}[Radicais ⽷、⼗]
  \begin{phonetics}{纯真}{chun2zhen1}
    \definition{adj.}{inocente e não afetado | puro e não adulterado}
    \definition{s.}{inocência}
  \end{phonetics}
\end{entry}

\begin{entry}{纷纷}{7,7}[Radicais ⽷、⽷]
  \begin{phonetics}{纷纷}{fen1fen1}[][HSK 4]
    \definition{adj.}{numeroso e confuso; muitos e desordenados}
    \definition{adv.}{um após o outro; em sucessão; em rápida sucessão}
  \end{phonetics}
\end{entry}

\begin{entry}{纸}{7}[Radical ⽷]
  \begin{phonetics}{纸}{zhi3}[][HSK 2]
    \definition{clas.}{para documentos, cartas, etc.}
    \definition[张,沓]{s.}{papel}
  \end{phonetics}
\end{entry}

\begin{entry}{纸巾}{7,3}[Radicais ⽷、⼱]
  \begin{phonetics}{纸巾}{zhi3jin1}
    \definition[张,包]{s.}{lenço | guardanapo | papel toalha}
  \end{phonetics}
\end{entry}

\begin{entry}{纸币}{7,4}[Radicais ⽷、⼱]
  \begin{phonetics}{纸币}{zhi3bi4}
    \definition[张]{s.}{nota (dinheiro) | cédula}
  \end{phonetics}
\end{entry}

\begin{entry}{纸尿裤}{7,7,12}[Radicais ⽷、⼫、⾐]
  \begin{phonetics}{纸尿裤}{zhi3niao4ku4}
    \definition{s.}{fralda descartável}
  \end{phonetics}
\end{entry}

\begin{entry}{纸张}{7,7}[Radicais ⽷、⼸]
  \begin{phonetics}{纸张}{zhi3zhang1}
    \definition{s.}{papel}
  \end{phonetics}
\end{entry}

\begin{entry}{纸烟}{7,10}[Radicais ⽷、⽕]
  \begin{phonetics}{纸烟}{zhi3yan1}
    \definition{s.}{cigarro}
  \end{phonetics}
\end{entry}

\begin{entry}{纹路}{7,13}[Radicais ⽷、⾜]
  \begin{phonetics}{纹路}{wen2lu4}
    \definition{s.}{padrão de linhas | rugas | veias | veias (em mármore ou impressão digital) | grãos (em madeira, etc.)}
  \end{phonetics}
\end{entry}

\begin{entry}{肚}{7}[Radical ⾁]
  \begin{phonetics}{肚}{du3}
    \definition{s.}{tripas | entranhas}
  \end{phonetics}
  \begin{phonetics}{肚}{du4}
    \definition{s.}{barriga}
  \end{phonetics}
\end{entry}

\begin{entry}{肚子}{7,3}[Radicais ⾁、⼦]
  \begin{phonetics}{肚子}{du4zi5}[][HSK 4]
    \definition[个,只]{s.}{abdômen; barriguinha; ventre; barriga}
  \end{phonetics}
\end{entry}

\begin{entry}{良心}{7,4}[Radicais ⾉、⼼]
  \begin{phonetics}{良心}{liang2xin1}
    \definition{s.}{consciência}
  \end{phonetics}
\end{entry}

\begin{entry}{良田}{7,5}[Radicais ⾉、⽥]
  \begin{phonetics}{良田}{liang2tian2}
    \definition{s.}{terra agrícola boa | terra fértil}
  \end{phonetics}
\end{entry}

\begin{entry}{良好}{7,6}[Radicais ⾉、⼥]
  \begin{phonetics}{良好}{liang2hao3}[][HSK 4]
    \definition{adj.}{bom; ótimo; bem}
  \end{phonetics}
\end{entry}

\begin{entry}{芥}{7}[Radical ⾋]
  \begin{phonetics}{芥}{gai4}
    \definition{s.}{usado em 芥蓝 \dpy{gai4lan2}}
    \seeref{芥蓝}{gai4lan2}
  \end{phonetics}
  \begin{phonetics}{芥}{jie4}
    \definition{s.}{mostarda}
  \end{phonetics}
\end{entry}

\begin{entry}{芥兰}{7,5}[Radicais ⾋、⼋]
  \begin{phonetics}{芥兰}{gai4lan2}
    \variantof{芥蓝}
  \end{phonetics}
  \begin{phonetics}{芥兰}{jie4lan2}
    \definition{s.}{couve}
  \end{phonetics}
\end{entry}

\begin{entry}{芥蓝}{7,13}[Radicais ⾋、⾋]
  \begin{phonetics}{芥蓝}{gai4lan2}
    \definition{s.}{brócolis chinês | couve chinesa | mostarda}
    \seeref{格兰菜}{ge2lan2cai4}
  \end{phonetics}
\end{entry}

\begin{entry}{芦笋}{7,10}[Radicais ⾋、⽵]
  \begin{phonetics}{芦笋}{lu2sun3}
    \definition{s.}{aspargos}
  \end{phonetics}
\end{entry}

\begin{entry}{芯片}{7,4}[Radicais ⾋、⽚]
  \begin{phonetics}{芯片}{xin1pian4}
    \definition{s.}{chip de computador | microchip}
  \end{phonetics}
\end{entry}

\begin{entry}{花}{7}[Radical ⾋]
  \begin{phonetics}{花}{hua1}[][HSK 1,2,4]
    \definition*{s.}{sobrenome Hua}
    \definition{adj.}{multicolorido; colorido | embaçado; obscuro; deslumbrado e confuso | extravagante; florido; vistoso}
    \definition[朵,支,束,把,盆,簇]{s.}{flor; órgãos de reprodução sexual de plantas com sementes | flor; planta ornamental |  qualquer coisa que se assemelhe a uma flor | fogos de artifício | padrão; design; design decorativo | flor; metáfora para a essência de uma causa | prostituta; cortesã; referindo-se a prostitutas ou a assuntos relacionados a prostitutas | algodão | varíola | ferimento; ferida; lesões traumáticas sofridas em combate}
    \definition{v.}{gastar; despender; consumir}
  \end{phonetics}
\end{entry}

\begin{entry}{花儿}{7,2}[Radicais ⾋、⼉]
  \begin{phonetics}{花儿}{hua1r5}
    \definition[朵,支,束,把,盆,簇]{s.}{flor}
  \end{phonetics}
\end{entry}

\begin{entry}{花生}{7,5}[Radicais ⾋、⽣]
  \begin{phonetics}{花生}{hua1sheng1}
    \definition[粒]{s.}{amendoim}
  \end{phonetics}
\end{entry}

\begin{entry}{花园}{7,7}[Radicais ⾋、⼞]
  \begin{phonetics}{花园}{hua1 yuan2}[][HSK 2]
    \definition[个,座]{s.}{jardim}
  \end{phonetics}
\end{entry}

\begin{entry}{花店}{7,8}[Radicais ⾋、⼴]
  \begin{phonetics}{花店}{hua1dian4}
    \definition{s.}{floricultura}
  \end{phonetics}
\end{entry}

\begin{entry}{花茶}{7,9}[Radicais ⾋、⾋]
  \begin{phonetics}{花茶}{hua1cha2}
    \definition[杯,壶]{s.}{chá perfumado}
  \end{phonetics}
\end{entry}

\begin{entry}{花样游泳}{7,10,12,8}[Radicais ⾋、⽊、⽔、⽔]
  \begin{phonetics}{花样游泳}{hua1yang4you2yong3}
    \definition{s.}{nado sincronizado}
  \end{phonetics}
\end{entry}

\begin{entry}{花椰菜}{7,12,11}[Radicais ⾋、⽊、⾋]
  \begin{phonetics}{花椰菜}{hua1ye1cai4}
    \definition{s.}{couve-flor}
  \end{phonetics}
\end{entry}

\begin{entry}{芹菜}{7,11}[Radicais ⾋、⾋]
  \begin{phonetics}{芹菜}{qin2cai4}
    \definition{s.}{salsão}
  \end{phonetics}
\end{entry}

\begin{entry}{苏格兰}{7,10,5}[Radicais ⾋、⽊、⼋]
  \begin{phonetics}{苏格兰}{su1ge2lan2}
    \definition*{s.}{Escócia}
  \end{phonetics}
\end{entry}

\begin{entry}{补}{7}[Radical ⾐]
  \begin{phonetics}{补}{bu3}[][HSK 3]
    \definition*{s.}{sobrenome Bu}
    \definition{s.}{benefício | ajuda | uso}
    \definition{v.}{consertar | remendar | preencher | adicionar suplemento | suprir | compensar |nutrir}
  \end{phonetics}
\end{entry}

\begin{entry}{补充}{7,6}[Radicais ⾐、⼉]
  \begin{phonetics}{补充}{bu3chong1}[][HSK 3]
    \definition{adj.}{adicional | suplementar}
    \definition[个]{s.}{aditivo | suplemento}
    \definition{v.}{reabastecer | suplementar | complementar}
  \end{phonetics}
\end{entry}

\begin{entry}{角}{7}[Radical ⾓]
  \begin{phonetics}{角}{jiao3}[][HSK 2]
    \definition{clas.}{1 jiao = 10 centavos}
    \definition[个]{s.}{ângulo | esquina | chifre | em forma de chifre}
  \end{phonetics}
  \begin{phonetics}{角}{jue2}
    \definition*{s.}{sobrenome Jue}
    \definition{s.}{papel (teatro)}
    \definition{v.}{competir}
  \end{phonetics}
\end{entry}

\begin{entry}{角色}{7,6}[Radicais ⾓、⾊]
  \begin{phonetics}{角色}{jue2se4}[][HSK 4]
    \definition{s.}{papel; personagem em uma peça; personagem representado por um ator | papel; função; parte}
  \end{phonetics}
\end{entry}

\begin{entry}{角度}{7,9}[Radicais ⾓、⼴]
  \begin{phonetics}{角度}{jiao3du4}[][HSK 2]
    \definition{s.}{ângulo | ponto de vista}
  \end{phonetics}
\end{entry}

\begin{entry}{言论}{7,6}[Radicais ⾔、⾔]
  \begin{phonetics}{言论}{yan2lun4}
    \definition{s.}{expressão de opinião |  visualizações | comentários | argumentos}
  \end{phonetics}
\end{entry}

\begin{entry}{证}{7}[Radical ⾔]
  \begin{phonetics}{证}{zheng4}[][HSK 3]
    \definition{s.}{evidência; prova; testemunho; testemunha | certificado; cartão | doença; enfermidade}
    \definition{v.}{provar; verificar; demonstrar}
  \end{phonetics}
\end{entry}

\begin{entry}{证件}{7,6}[Radicais ⾔、⼈]
  \begin{phonetics}{证件}{zheng4jian4}[][HSK 3]
    \definition[个,本,张]{s.}{documentos; credenciais; certificado}
  \end{phonetics}
\end{entry}

\begin{entry}{证实}{7,8}[Radicais ⾔、⼧]
  \begin{phonetics}{证实}{zheng4shi2}
    \definition{v.}{confirmar (algo como verdadeiro) | verificar}
  \end{phonetics}
\end{entry}

\begin{entry}{证明}{7,8}[Radicais ⾔、⽇]
  \begin{phonetics}{证明}{zheng4ming2}[][HSK 3]
    \definition[个,份]{s.}{certificado; testemunho; identificação; certificado ou carta de certificação; documentos que comprovem identidade, experiência, etc., como carteira de estudante, carteira de trabalho, certificado de graduação, etc.}
    \definition{v.}{provar; testemunhar; sustentar; usar materiais confiáveis ​​para mostrar ou determinar a autenticidade de uma pessoa ou coisa}
  \end{phonetics}
\end{entry}

\begin{entry}{证据}{7,11}[Radicais ⾔、⼿]
  \begin{phonetics}{证据}{zheng4ju4}[][HSK 3]
    \definition{s.}{prova; evidência; testemunho; fatos ou materiais relevantes que podem provar a autenticidade de algo}
  \end{phonetics}
\end{entry}

\begin{entry}{评价}{7,6}[Radicais ⾔、⼈]
  \begin{phonetics}{评价}{ping2jia4}[][HSK 3]
    \definition[个,项,条,份]{s.}{avaliação; apreciação}
    \definition{v.}{estimar; avaliar}
  \end{phonetics}
\end{entry}

\begin{entry}{评论}{7,6}[Radicais ⾔、⾔]
  \begin{phonetics}{评论}{ping2lun4}
    \definition[篇]{s.}{comentário}
    \definition{v.}{comentar | discutir}
  \end{phonetics}
\end{entry}

\begin{entry}{诅咒}{7,8}[Radicais ⾔、⼝]
  \begin{phonetics}{诅咒}{zu3zhou4}
    \definition{v.}{amaldiçoar}
  \end{phonetics}
\end{entry}

\begin{entry}{词}{7}[Radical ⾔]
  \begin{phonetics}{词}{ci2}[][HSK 2]
    \definition[个,组]{s.}{discurso | declaração | linhas de jogo | um tipo de poesia clássica chinesa, originária da Dinastia Tang e totalmente desenvolvida na Dinastia Song | palavra  | termo}
  \end{phonetics}
\end{entry}

\begin{entry}{词汇}{7,5}[Radicais ⾔、⽔]
  \begin{phonetics}{词汇}{ci2hui4}[][HSK 4]
    \definition[个,组,批,串,堆]{s.}{vocabulário; termo geral para palavras usadas em um idioma}
  \end{phonetics}
\end{entry}

\begin{entry}{词典}{7,8}[Radicais ⾔、⼋]
  \begin{phonetics}{词典}{ci2dian3}[][HSK 2]
    \definition[部,本]{s.}{dicionário}
  \seealsoref{字典}{zi4 dian3}
  \end{phonetics}
\end{entry}

\begin{entry}{词语}{7,9}[Radicais ⾔、⾔]
  \begin{phonetics}{词语}{ci2yu3}[][HSK 2]
    \definition{s.}{palavra (termo geral, incluindo desdemonossilábicas até frases curtas) | termo (por exemplo, termo técnico) | expressão}
  \end{phonetics}
\end{entry}

\begin{entry}{豆角}{7,7}[Radicais ⾖、⾓]
  \begin{phonetics}{豆角}{dou4jiao3}
    \definition{s.}{feijão verde}
  \end{phonetics}
\end{entry}

\begin{entry}{豆荚}{7,9}[Radicais ⾖、⾋]
  \begin{phonetics}{豆荚}{dou4jia2}
    \definition{s.}{vagem (de legumes)}
  \end{phonetics}
\end{entry}

\begin{entry}{豆腐}{7,14}[Radicais ⾖、⾁]
  \begin{phonetics}{豆腐}{dou4fu5}[][HSK 4]
    \definition[块,盒,斤,盘,锅]{s.}{\emph{tofu}}
  \end{phonetics}
\end{entry}

\begin{entry}{财产}{7,6}[Radicais ⾙、⼇]
  \begin{phonetics}{财产}{cai2chan3}[][HSK 4]
    \definition{s.}{ativos; propriedade; pertences; refere-se à posse de riqueza material, como dinheiro, bens, casas, terras, etc.}
  \end{phonetics}
\end{entry}

\begin{entry}{财富}{7,12}[Radicais ⾙、⼧]
  \begin{phonetics}{财富}{cai2fu4}[][HSK 4]
    \definition{s.}{riqueza; fortuna}
  \end{phonetics}
\end{entry}

\begin{entry}{走}{7}[Kangxi 156][Radical ⾛]
  \begin{phonetics}{走}{zou3}[][HSK 1]
    \definition{v.}{andar | caminhar}
  \end{phonetics}
\end{entry}

\begin{entry}{走开}{7,4}[Radicais ⾛、⼶]
  \begin{phonetics}{走开}{zou3 kai1}[][HSK 2]
    \definition{v.}{ir embora | fugir | ir para outro lugar}
  \end{phonetics}
\end{entry}

\begin{entry}{走去}{7,5}[Radicais ⾛、⼛]
  \begin{phonetics}{走去}{zou3qu4}
    \definition{v.}{caminhar até (para)}
  \end{phonetics}
\end{entry}

\begin{entry}{走过}{7,6}[Radicais ⾛、⾡]
  \begin{phonetics}{走过}{zou3 guo4}[][HSK 2]
    \definition{v.}{passar}
  \end{phonetics}
\end{entry}

\begin{entry}{走秀}{7,7}[Radicais ⾛、⽲]
  \begin{phonetics}{走秀}{zou3xiu4}
    \definition{s.}{desfile de moda}
    \definition{v.}{andar na passarela (em um desfile de moda)}
  \end{phonetics}
\end{entry}

\begin{entry}{走进}{7,7}[Radicais ⾛、⾡]
  \begin{phonetics}{走进}{zou3 jin4}[][HSK 2]
    \definition{v.}{entrar}
  \end{phonetics}
\end{entry}

\begin{entry}{走势}{7,8}[Radicais ⾛、⼒]
  \begin{phonetics}{走势}{zou3shi4}
    \definition{s.}{caminho | tendência}
  \end{phonetics}
\end{entry}

\begin{entry}{走卒}{7,8}[Radicais ⾛、⼗]
  \begin{phonetics}{走卒}{zou3zu2}
    \definition{s.}{lacaio (masculino) | peão (isto é, soldado de infantaria) | servo}
  \end{phonetics}
\end{entry}

\begin{entry}{走鬼}{7,9}[Radicais ⾛、⿁]
  \begin{phonetics}{走鬼}{zou3gui3}
    \definition{s.}{vendedor ambulante sem licença}
  \end{phonetics}
\end{entry}

\begin{entry}{走索}{7,10}[Radicais ⾛、⽷]
  \begin{phonetics}{走索}{zou3suo3}
    \definition{v.}{andar na corda bamba}
    \seeref{走绳}{zou3sheng2}
  \end{phonetics}
\end{entry}

\begin{entry}{走绳}{7,11}[Radicais ⾛、⽷]
  \begin{phonetics}{走绳}{zou3sheng2}
    \definition{v.}{andar na corda bamba}
    \seeref{走索}{zou3suo3}
  \end{phonetics}
\end{entry}

\begin{entry}{走路}{7,13}[Radicais ⾛、⾜]
  \begin{phonetics}{走路}{zou3lu4}[][HSK 1]
    \definition{v.}{andar | ir a pé | sair | ir embora}
  \end{phonetics}
\end{entry}

\begin{entry}{足}{7}[Kangxi 157][Radical ⾜]
  \begin{phonetics}{足}{ju4}
    \definition{adj.}{excessivo}
  \end{phonetics}
  \begin{phonetics}{足}{zu2}
    \definition{adj.}{amplo}
    \definition{s.}{pé}
    \definition{v.}{ser suficiente}
  \end{phonetics}
\end{entry}

\begin{entry}{足月}{7,4}[Radicais ⾜、⽉]
  \begin{phonetics}{足月}{zu2yue4}
    \definition{s.}{gestação completa}
  \end{phonetics}
\end{entry}

\begin{entry}{足足}{7,7}[Radicais ⾜、⾜]
  \begin{phonetics}{足足}{zu2zu2}
    \definition{adv.}{tanto quanto | extremamente | completamente | não menos que}
  \end{phonetics}
\end{entry}

\begin{entry}{足够}{7,11}[Radicais ⾜、⼣]
  \begin{phonetics}{足够}{zu2 gou4}[][HSK 3]
    \definition{adj.}{bastante; amplo; suficiente; na medida em que deve ser ou pode atender às necessidades}
    \definition{v.}{satisfazer; ser suficiente; estar a contento}
  \end{phonetics}
\end{entry}

\begin{entry}{足球}{7,11}[Radicais ⾜、⽟]
  \begin{phonetics}{足球}{zu2qiu2}[][HSK 3]
    \definition[个,只,颗,袋]{s.}{futebol | bola de futebol}
  \end{phonetics}
\end{entry}

\begin{entry}{足球队}{7,11,4}[Radicais ⾜、⽟、⾩]
  \begin{phonetics}{足球队}{zu2qiu2dui4}
    \definition{s.}{time de futebol}
  \end{phonetics}
\end{entry}

\begin{entry}{足球协会}{7,11,6,6}[Radicais ⾜、⽟、⼗、⼈]
  \begin{phonetics}{足球协会}{zu2qiu2xie2hui4}
    \definition*{s.}{Associação de Futebol}
  \end{phonetics}
\end{entry}

\begin{entry}{足球场}{7,11,6}[Radicais ⾜、⽟、⼟]
  \begin{phonetics}{足球场}{zu2qiu2chang3}
    \definition{s.}{campo de futebol}
  \end{phonetics}
\end{entry}

\begin{entry}{足球迷}{7,11,9}[Radicais ⾜、⽟、⾡]
  \begin{phonetics}{足球迷}{zu2qiu2mi2}
    \definition{s.}{fã de futebol}
  \end{phonetics}
\end{entry}

\begin{entry}{足球赛}{7,11,14}[Radicais ⾜、⽟、⾙]
  \begin{phonetics}{足球赛}{zu2qiu2sai4}
    \definition{s.}{competição de futebol | partida de futebol}
  \end{phonetics}
\end{entry}

\begin{entry}{身上}{7,3}[Radicais ⾝、⼀]
  \begin{phonetics}{身上}{shen1shang5}[][HSK 1]
    \definition{adv.}{no corpo de alguém | em um | com um}
  \end{phonetics}
\end{entry}

\begin{entry}{身亡}{7,3}[Radicais ⾝、⼇]
  \begin{phonetics}{身亡}{shen1wang2}
    \definition{v.}{morrer}
  \end{phonetics}
\end{entry}

\begin{entry}{身边}{7,5}[Radicais ⾝、⾡]
  \begin{phonetics}{身边}{shen1 bian1}[][HSK 2]
    \definition{adv.}{ao redor | ao lado de alguém | em mãos}
  \end{phonetics}
\end{entry}

\begin{entry}{身份证}{7,6,7}[Radicais ⾝、⼈、⾔]
  \begin{phonetics}{身份证}{shen1 fen4 zheng4}[][HSK 3]
    \definition[张]{s.}{ID; bilhete de identidade; carteira de identidade}
  \end{phonetics}
\end{entry}

\begin{entry}{身体}{7,7}[Radicais ⾝、⼈]
  \begin{phonetics}{身体}{shen1ti3}[][HSK 1]
    \definition[具,个]{s.}{em pessoa | saúde de alguém | o corpo}
  \end{phonetics}
\end{entry}

\begin{entry}{身体乳}{7,7,8}[Radicais ⾝、⼈、⼄]
  \begin{phonetics}{身体乳}{shen1ti3 ru3}
    \definition{s.}{loção corporal}
  \end{phonetics}
\end{entry}

\begin{entry}{身体能力}{7,7,10,2}[Radicais ⾝、⼈、⾁、⼒]
  \begin{phonetics}{身体能力}{shen1ti3 neng2li4}
    \definition{s.}{habilidade física}
  \end{phonetics}
\end{entry}

\begin{entry}{辛苦}{7,8}[Radicais ⾟、⾋]
  \begin{phonetics}{辛苦}{xin1ku3}
    \definition{adj.}{exaustivo | duro | árduo}
    \definition{s.}{dificuldades}
    \definition{v.}{trabalhar duro | ter muitos problemas}
  \end{phonetics}
\end{entry}

\begin{entry}{迎接}{7,11}[Radicais ⾡、⼿]
  \begin{phonetics}{迎接}{ying2jie1}[][HSK 3]
    \definition{v.}{conhecer; cumprimentar; dar as boas-vindas}
  \end{phonetics}
\end{entry}

\begin{entry}{运气}{7,4}[Radicais ⾡、⽓]
  \begin{phonetics}{运气}{yun4qi5}
    \definition{s.}{sorte (boa ou má)}
  \end{phonetics}
\end{entry}

\begin{entry}{运动}{7,6}[Radicais ⾡、⼒]
  \begin{phonetics}{运动}{yun4dong4}[][HSK 2]
    \definition[场]{s.}{esporte | desporto}
    \definition{v.}{exercitar | mover-se}
  \end{phonetics}
\end{entry}

\begin{entry}{运动会}{7,6,6}[Radicais ⾡、⼒、⼈]
  \begin{phonetics}{运动会}{yun4dong4hui4}
    \definition[个]{s.}{competição esportiva}
  \end{phonetics}
\end{entry}

\begin{entry}{运动场}{7,6,6}[Radicais ⾡、⼒、⼟]
  \begin{phonetics}{运动场}{yun4dong4chang3}
    \definition{s.}{campo desportivo | campo de jogos}
  \end{phonetics}
\end{entry}

\begin{entry}{运动员}{7,6,7}[Radicais ⾡、⼒、⼝]
  \begin{phonetics}{运动员}{yun4dong4yuan2}
    \definition[名,个]{s.}{jogador | atleta}
  \end{phonetics}
\end{entry}

\begin{entry}{运动学}{7,6,8}[Radicais ⾡、⼒、⼦]
  \begin{phonetics}{运动学}{yun4dong4xue2}
    \definition{s.}{cinemática}
  \end{phonetics}
\end{entry}

\begin{entry}{运动服}{7,6,8}[Radicais ⾡、⼒、⽉]
  \begin{phonetics}{运动服}{yun4dong4fu2}
    \definition{s.}{roupa para prática de esporte}
  \end{phonetics}
\end{entry}

\begin{entry}{运动衫}{7,6,8}[Radicais ⾡、⼒、⾐]
  \begin{phonetics}{运动衫}{yun4dong4shan1}
    \definition[件]{s.}{moletom | camisa esportiva}
  \end{phonetics}
\end{entry}

\begin{entry}{运动家}{7,6,10}[Radicais ⾡、⼒、⼧]
  \begin{phonetics}{运动家}{yun4dong4jia1}
    \definition{s.}{ativista | atleta | esportista}
  \end{phonetics}
\end{entry}

\begin{entry}{运动病}{7,6,10}[Radicais ⾡、⼒、⽧]
  \begin{phonetics}{运动病}{yun4dong4bing4}
    \definition{s.}{enjôo (movimento, carro, etc.)}
  \end{phonetics}
\end{entry}

\begin{entry}{运动鞋}{7,6,15}[Radicais ⾡、⼒、⾰]
  \begin{phonetics}{运动鞋}{yun4dong4xie2}
    \definition{s.}{tênis | sapatos esportivos}
  \end{phonetics}
\end{entry}

\begin{entry}{运行}{7,6}[Radicais ⾡、⾏]
  \begin{phonetics}{运行}{yun4xing2}
    \definition{v.}{(corpos celestes, etc.) mover-se ao longo do curso | (figurativo) funcionar, estar em operação | (serviço de trem, etc.) operar | (computador) executar um programa}
  \end{phonetics}
\end{entry}

\begin{entry}{运河}{7,8}[Radicais ⾡、⽔]
  \begin{phonetics}{运河}{yun4he2}
    \definition{s.}{canal (em um rio)}
  \end{phonetics}
\end{entry}

\begin{entry}{运输}{7,13}[Radicais ⾡、⾞]
  \begin{phonetics}{运输}{yun4shu1}[][HSK 3]
    \definition{v.}{enviar; transportar; usar um carro, navio, avião, etc. para transportar pessoas ou coisas de um lugar para outro}
  \end{phonetics}
\end{entry}

\begin{entry}{近}{7}[Radical ⾡]
  \begin{phonetics}{近}{jin4}[][HSK 2]
    \definition{adj.}{perto | próximo}
  \end{phonetics}
\end{entry}

\begin{entry}{近代}{7,5}[Radicais ⾡、⼈]
  \begin{phonetics}{近代}{jin4dai4}[][HSK 4]
    \definition{s.}{tempos modernos; era passada relativamente próxima à era moderna, geralmente referida na história chinesa como 1840 a 1919 | na história mundial, geralmente se refere à era capitalista}
  \end{phonetics}
\end{entry}

\begin{entry}{近期}{7,12}[Radicais ⾡、⽉]
  \begin{phonetics}{近期}{jin4 qi1}[][HSK 3]
    \definition{adv.}{num futuro próximo; brevemente}
  \end{phonetics}
\end{entry}

\begin{entry}{还}{7}[Radical ⾡]
  \begin{phonetics}{还}{hai2}[][HSK 1]
    \definition{adv.}{ainda | também | ainda mais | razoavelmente | bastante}
  \end{phonetics}
  \begin{phonetics}{还}{huan2}[][HSK 1]
    \definition*{s.}{sobrenome Huan}
    \definition{v.}{devolver | restituir | pagar de volta}
  \end{phonetics}
\end{entry}

\begin{entry}{还有}{7,6}[Radicais ⾡、⽉]
  \begin{phonetics}{还有}{hai2 you3}[][HSK 1]
    \definition{adv.}{além do mais | além disso | ainda permanece | ainda há}
  \end{phonetics}
\end{entry}

\begin{entry}{还是}{7,9}[Radicais ⾡、⽇]
  \begin{phonetics}{还是}{hai2shi5}[][HSK 1]
    \definition{adv.}{ainda (como antes) | inesperadamente | teve melhor}
    \definition{conj.}{ou (somente para frases interrogativas)}
  \end{phonetics}
\end{entry}

\begin{entry}{这}{7}[Radical ⾡]
  \begin{phonetics}{这}{zhe4}[][HSK 1]
    \definition{pron.}{este, isto}
  \end{phonetics}
  \begin{phonetics}{这}{zhei4}
    \definition{pron.}{(coloquial) este}
  \end{phonetics}
\end{entry}

\begin{entry}{这儿}{7,2}[Radicais ⾡、⼉]
  \begin{phonetics}{这儿}{zhe4r5}[][HSK 1]
    \definition{pron.}{aqui}
  \end{phonetics}
\end{entry}

\begin{entry}{这么}{7,3}[Radicais ⾡、⼃]
  \begin{phonetics}{这么}{zhe4 me5}[][HSK 2]
    \definition{adv.}{como este | desta maneira}
  \end{phonetics}
\end{entry}

\begin{entry}{这边}{7,5}[Radicais ⾡、⾡]
  \begin{phonetics}{这边}{zhe4bian5}[][HSK 1]
    \definition{pron.}{aqui | este lado}
  \end{phonetics}
\end{entry}

\begin{entry}{这时}{7,7}[Radicais ⾡、⽇]
  \begin{phonetics}{这时}{zhe4 shi2}[][HSK 2]
    \definition{adv.}{neste momento}
  \end{phonetics}
\end{entry}

\begin{entry}{这时候}{7,7,10}[Radicais ⾡、⽇、⼈]
  \begin{phonetics}{这时候}{zhe4 shi2 hou5}[][HSK 2]
    \definition{adv.}{neste momento}
  \end{phonetics}
\end{entry}

\begin{entry}{这里}{7,7}[Radicais ⾡、⾥]
  \begin{phonetics}{这里}{zhe4li3}[][HSK 1]
    \definition{pron.}{aqui}
  \end{phonetics}
\end{entry}

\begin{entry}{这些}{7,8}[Radicais ⾡、⼆]
  \begin{phonetics}{这些}{zhe4xie1}[][HSK 1]
    \definition{pron.}{estes}
  \end{phonetics}
\end{entry}

\begin{entry}{这样}{7,10}[Radicais ⾡、⽊]
  \begin{phonetics}{这样}{zhe4 yang4}[][HSK 2]
    \definition{adv.}{assim | dessa maneira | deste modo}
  \end{phonetics}
\end{entry}

\begin{entry}{这麽}{7,14}[Radicais ⾡、⿇]
  \begin{phonetics}{这麽}{zhe4 me5}
    \variantof{这么}
  \end{phonetics}
\end{entry}

\begin{entry}{进}{7}[Radical ⾡]
  \begin{phonetics}{进}{jin4}[][HSK 1]
    \definition{clas.}{para seções em um edifício ou complexo residencial}
    \definition{s.}{(matemática) base de um sistema numérico}
    \definition{v.}{entrar}
  \end{phonetics}
\end{entry}

\begin{entry}{进一步}{7,1,7}[Radicais ⾡、⼀、⽌]
  \begin{phonetics}{进一步}{jin4 yi2 bu4}[][HSK 3]
    \definition{adv.}{mais; dar um passo adiante; avançar um passo}
  \end{phonetics}
\end{entry}

\begin{entry}{进入}{7,2}[Radicais ⾡、⼊]
  \begin{phonetics}{进入}{jin4 ru4}[][HSK 2]
    \definition{v.}{entrar | juntar-se}
  \end{phonetics}
\end{entry}

\begin{entry}{进口}{7,3}[Radicais ⾡、⼝]
  \begin{phonetics}{进口}{jin4kou3}[][HSK 4]
    \definition{adj.}{importado}
    \definition{s.}{importação; entrada de um edifício ou local, também chamada de `` 入口''}
    \definition{v.+compl.}{importar; comprar ou transportar mercadorias de outro país ou região | entrar no porto; navegar em direção a um porto}
  \seealsoref{入口}{ru4kou3}
  \end{phonetics}
\end{entry}

\begin{entry}{进出口}{7,5,3}[Radicais ⾡、⼐、⼝]
  \begin{phonetics}{进出口}{jin4chu1kou3}
    \definition{s.}{importação e exportação}
    \definition{v.}{importar e exportar}
  \end{phonetics}
\end{entry}

\begin{entry}{进去}{7,5}[Radicais ⾡、⼛]
  \begin{phonetics}{进去}{jin4 qu4}[][HSK 1]
    \definition{v.}{entrar (a partir da minha localização)}
  \end{phonetics}
\end{entry}

\begin{entry}{进行}{7,6}[Radicais ⾡、⾏]
  \begin{phonetics}{进行}{jin4xing2}[][HSK 2]
    \definition{v.}{continuar | estar em andamento | fazer | conduzir | continuar | executar | marchar | avançar | prosseguir | estar em marcha}
  \end{phonetics}
\end{entry}

\begin{entry}{进行编程}{7,6,12,12}[Radicais ⾡、⾏、⽷、⽲]
  \begin{phonetics}{进行编程}{jin4xing2bian1cheng2}
    \definition{s.}{programa de computador executável}
  \end{phonetics}
\end{entry}

\begin{entry}{进来}{7,7}[Radicais ⾡、⽊]
  \begin{phonetics}{进来}{jin4 lai2}[][HSK 1]
    \definition{v.}{entrar (para a minha localização)}
  \end{phonetics}
\end{entry}

\begin{entry}{进步}{7,7}[Radicais ⾡、⽌]
  \begin{phonetics}{进步}{jin4bu4}[][HSK 3]
    \definition{adj.}{progressivo}
    \definition[个]{s.}{avanço; progresso; melhora}
    \definition{v.}{avançar; progredir; melhorar}
  \end{phonetics}
\end{entry}

\begin{entry}{进展}{7,10}[Radicais ⾡、⼫]
  \begin{phonetics}{进展}{jin4zhan3}[][HSK 3]
    \definition{v.}{fazer progresso; progredir}
  \end{phonetics}
\end{entry}

\begin{entry}{远}{7}[Radical ⾡]
  \begin{phonetics}{远}{yuan3}[][HSK 1]
    \definition{adj.}{longe | distante | remoto}
  \end{phonetics}
\end{entry}

\begin{entry}{远天}{7,4}[Radicais ⾡、⼤]
  \begin{phonetics}{远天}{yuan3tian1}
    \definition{s.}{paraíso | o céu distante}
  \end{phonetics}
\end{entry}

\begin{entry}{远方}{7,4}[Radicais ⾡、⽅]
  \begin{phonetics}{远方}{yuan3fang1}
    \definition{s.}{longe | um local distante}
  \end{phonetics}
\end{entry}

\begin{entry}{远远}{7,7}[Radicais ⾡、⾡]
  \begin{phonetics}{远远}{yuan3yuan3}
    \definition{adv.}{de longe}
  \end{phonetics}
\end{entry}

\begin{entry}{远征}{7,8}[Radicais ⾡、⼻]
  \begin{phonetics}{远征}{yuan3zheng1}
    \definition{s.}{uma expedição militar | marcha para regiões remotas}
  \end{phonetics}
\end{entry}

\begin{entry}{违规}{7,8}[Radicais ⾡、⾒]
  \begin{phonetics}{违规}{wei2gui1}
    \definition{v.}{violar as regras}
  \end{phonetics}
\end{entry}

\begin{entry}{违宪}{7,9}[Radicais ⾡、⼧]
  \begin{phonetics}{违宪}{wei2xian4}
    \definition{adj.}{inconstitucional}
  \end{phonetics}
\end{entry}

\begin{entry}{连}{7}[Radical ⾡]
  \begin{phonetics}{连}{lian2}[][HSK 3]
    \definition*{s.}{sobrenome Lian}
    \definition{adv.}{em sucessão; um após o outro; repetidamente | até}
    \definition{prep.}{incluindo}
    \definition{s.}{companhia | conjunção}
    \definition{v.}{ligar; juntar; conectar | envolver (em problemas); implicar | costurar; coser}
  \end{phonetics}
\end{entry}

\begin{entry}{连忙}{7,6}[Radicais ⾡、⼼]
  \begin{phonetics}{连忙}{lian2mang2}[][HSK 3]
    \definition{adv.}{prontamente; imediatamente; apressadamente}
  \end{phonetics}
\end{entry}

\begin{entry}{连续}{7,11}[Radicais ⾡、⽷]
  \begin{phonetics}{连续}{lian2xu4}[][HSK 3]
    \definition{adv.}{continuamente; sucessivamente; em uma fileira}
  \end{phonetics}
\end{entry}

\begin{entry}{连续剧}{7,11,10}[Radicais ⾡、⽷、⼑]
  \begin{phonetics}{连续剧}{lian2 xu4 ju4}[][HSK 3]
    \definition{s.}{série; novela}
  \end{phonetics}
\end{entry}

\begin{entry}{连锁反应}{7,12,4,7}[Radicais ⾡、⾦、⼜、⼴]
  \begin{phonetics}{连锁反应}{lian2suo3fan3ying4}
    \definition{s.}{reação em cadeia}
  \end{phonetics}
\end{entry}

\begin{entry}{迟到}{7,8}[Radicais ⾡、⼑]
  \begin{phonetics}{迟到}{chi2dao4}[][HSK 4]
    \definition{v.}{chegar atrasado; atrasar-se}
  \end{phonetics}
\end{entry}

\begin{entry}{邮包}{7,5}[Radicais ⾢、⼓]
  \begin{phonetics}{邮包}{you2bao1}
    \definition{s.}{encomenda postal}
  \end{phonetics}
\end{entry}

\begin{entry}{邮市}{7,5}[Radicais ⾢、⼱]
  \begin{phonetics}{邮市}{you2shi4}
    \definition{s.}{mercado postal}
  \end{phonetics}
\end{entry}

\begin{entry}{邮电}{7,5}[Radicais ⾢、⽥]
  \begin{phonetics}{邮电}{you2dian4}
    \definition*{s.}{Correios e Telecomunicações}
  \end{phonetics}
\end{entry}

\begin{entry}{邮件}{7,6}[Radicais ⾢、⼈]
  \begin{phonetics}{邮件}{you2 jian4}[][HSK 3]
    \definition[封,个]{s.}{correspondência; correio; assunto postal; um termo geral para cartas, encomendas, etc. recebidas, transportadas e entregues pelos correios | \emph{e-mail}; mensagens enviadas e recebidas por meio eletrônico}
  \end{phonetics}
\end{entry}

\begin{entry}{邮局}{7,7}[Radicais ⾢、⼫]
  \begin{phonetics}{邮局}{you2ju2}
    \definition[家,个]{s.}{correio | agência dos correios}
  \end{phonetics}
\end{entry}

\begin{entry}{邮费}{7,9}[Radicais ⾢、⾙]
  \begin{phonetics}{邮费}{you2fei4}
    \definition{s.}{postagem}
    \definition{v.}{postar}
  \end{phonetics}
\end{entry}

\begin{entry}{邮迷}{7,9}[Radicais ⾢、⾡]
  \begin{phonetics}{邮迷}{you2mi2}
    \definition{s.}{filatelista | colecionador de selos}
  \end{phonetics}
\end{entry}

\begin{entry}{邮资}{7,10}[Radicais ⾢、⾙]
  \begin{phonetics}{邮资}{you2zi1}
    \definition{s.}{postagem}
  \end{phonetics}
\end{entry}

\begin{entry}{邮递}{7,10}[Radicais ⾢、⾡]
  \begin{phonetics}{邮递}{you2di4}
    \definition{v.}{enviar por correio}
  \end{phonetics}
\end{entry}

\begin{entry}{邮票}{7,11}[Radicais ⾢、⽰]
  \begin{phonetics}{邮票}{you2 piao4}[][HSK 3]
    \definition[枚,张,套,版]{s.}{selo; selo postal; um \emph{voucher} vendido pelos correios e afixado na correspondência para indicar que a postagem foi paga}
  \end{phonetics}
\end{entry}

\begin{entry}{邮箱}{7,15}[Radicais ⾢、⾋]
  \begin{phonetics}{邮箱}{you2 xiang1}[][HSK 3]
    \definition{s.}{caixa de correio | \emph{mailbox}; refere-se ao endereço de \emph{e-mail}}
  \end{phonetics}
\end{entry}

\begin{entry}{邻居}{7,8}[Radicais ⾢、⼫]
  \begin{phonetics}{邻居}{lin2ju1}
    \definition[个]{s.}{vizinho}
  \end{phonetics}
\end{entry}

\begin{entry}{里}{7}[Kangxi 166][Radical ⾥]
  \begin{phonetics}{里}{li3}[][HSK 1]
    \definition*{s.}{sobrenome Li}
    \definition{adv.}{em | dentro | interior | interno}
    \definition{s.}{vizinhança | bairro | li, medida antiga de comprimento, aproximadamente 500m | unidade administrativa antiga de 25 famílias}
  \end{phonetics}
\end{entry}

\begin{entry}{里头}{7,5}[Radicais ⾥、⼤]
  \begin{phonetics}{里头}{li3 tou5}[][HSK 2]
    \definition{s.}{dentro}
  \end{phonetics}
\end{entry}

\begin{entry}{里边}{7,5}[Radicais ⾥、⾡]
  \begin{phonetics}{里边}{li3 bian5}[][HSK 1]
    \definition{prep.}{em | dentro}
  \end{phonetics}
\end{entry}

\begin{entry}{里面}{7,9}[Radicais ⾥、⾯]
  \begin{phonetics}{里面}{li3 mian4}[][HSK 3]
    \definition{s.}{dentro; interior}
  \end{phonetics}
\end{entry}

\begin{entry}{里斯本}{7,12,5}[Radicais ⾥、⽄、⽊]
  \begin{phonetics}{里斯本}{li3si1ben3}
    \definition*{s.}{Lisboa}
  \end{phonetics}
\end{entry}

\begin{entry}{里斯本大学}{7,12,5,3,8}[Radicais ⾥、⽄、⽊、⼤、⼦]
  \begin{phonetics}{里斯本大学}{li3si1ben3 da4xue2}
    \definition*{s.}{Universidade de Lisboa}
  \end{phonetics}
\end{entry}

\begin{entry}{间}{7}[Radical ⾨]
  \begin{phonetics}{间}{jian1}
    \definition{adv.}{entre | dentro de um tempo ou espaço definidos}
    \definition{clas.}{para salas}
    \definition{s.}{sala | seção de uma sala ou espaço lateral entre dois pares de pilares}
  \end{phonetics}
  \begin{phonetics}{间}{jian4}[][HSK 1]
    \definition{s.}{lacuna}
    \definition{v.}{separar | podar (mudas) | semear descontentamento}
  \end{phonetics}
\end{entry}

\begin{entry}{间或}{7,8}[Radicais ⾨、⼽]
  \begin{phonetics}{间或}{jian4huo4}
    \definition{adv.}{às vezes | ocasionalmente | de vez em quando}
  \end{phonetics}
\end{entry}

\begin{entry}{间接}{7,11}[Radicais ⾨、⼿]
  \begin{phonetics}{间接}{jian4jie1}
    \definition{adj.}{indireto}
  \seealsoref{直接}{zhi2jie1}
  \end{phonetics}
\end{entry}

\begin{entry}{闷热}{7,10}[Radicais ⾨、⽕]
  \begin{phonetics}{闷热}{men1re4}
    \definition{adj.}{abafado | quente e abafado | sufocantemente quente | quente e sensual}
  \end{phonetics}
\end{entry}

\begin{entry}{阻击}{7,5}[Radicais ⾩、⼐]
  \begin{phonetics}{阻击}{zu3ji1}
    \definition{v.}{verificar | parar}
  \end{phonetics}
\end{entry}

\begin{entry}{阿}{7}[Radical ⾩]
  \begin{phonetics}{阿}{a1}
    \definition{pref.}{utilizado para indicar familiaridade antes de nomes monossilábicos, termos de parentesco, etc.}
  \end{phonetics}
  \begin{phonetics}{阿}{e1}
    \definition{adj.}{gracioso}
    \definition{pron.}{monte grande | canto, esquina}
    \definition{v.}{jogar | agradar | atender | ser injustamente parcial com | ser dobrado}
  \end{phonetics}
\end{entry}

\begin{entry}{阿姨}{7,9}[Radicais ⾩、⼥]
  \begin{phonetics}{阿姨}{a1yi2}[][HSK 4]
    \definition[个,位]{s.}{tia; uma forma de tratamento para uma mulher da geração dos pais; dirigir-se a uma mulher que tem aproximadamente a mesma idade da sua mãe, geralmente não é parente | babá em uma família; professora em um jardim de infância | tia; irmã da mãe (mais comum no sul da China)}
  \end{phonetics}
\end{entry}

\begin{entry}{阿哥}{7,10}[Radicais ⾩、⼝]
  \begin{phonetics}{阿哥}{a1ge1}
    \definition{s.}{irmão mais velho (familiar)}
  \end{phonetics}
\end{entry}

\begin{entry}{附近}{7,7}[Radicais ⾩、⾡]
  \begin{phonetics}{附近}{fu4jin4}[][HSK 4]
    \definition{adj.}{perto; vizinho}
    \definition{s.}{vizinhança; bairro}
  \end{phonetics}
\end{entry}

\begin{entry}{陆地}{7,6}[Radicais ⾩、⼟]
  \begin{phonetics}{陆地}{lu4di4}[][HSK 4]
    \definition[块,片]{s.}{terra; terra seca (em oposição ao mar); superfície da Terra, excluindo os oceanos (e, às vezes, rios e lagos)}
  \end{phonetics}
\end{entry}

\begin{entry}{陆续}{7,11}[Radicais ⾩、⽷]
  \begin{phonetics}{陆续}{lu4xu4}[][HSK 4]
    \definition{adv.}{sucessivamente; um após o outro; intermitentemente}
  \end{phonetics}
\end{entry}

\begin{entry}{陆路}{7,13}[Radicais ⾩、⾜]
  \begin{phonetics}{陆路}{lu4lu4}
    \definition{s.}{rota terrestre}
  \end{phonetics}
\end{entry}

\begin{entry}{饭}{7}[Radical ⾷]
  \begin{phonetics}{饭}{fan4}[][HSK 1]
    \definition[碗]{s.}{arroz cozido}
    \definition[顿]{s.}{refeição}
    \definition{s.}{(empréstimo linguístico) fã, devoto}
  \end{phonetics}
\end{entry}

\begin{entry}{饭店}{7,8}[Radicais ⾷、⼴]
  \begin{phonetics}{饭店}{fan4dian4}[][HSK 1]
    \definition[家,个]{s.}{restaurante | hotel}
  \end{phonetics}
\end{entry}

\begin{entry}{饭馆}{7,11}[Radicais ⾷、⾷]
  \begin{phonetics}{饭馆}{fan4 guan3}[][HSK 2]
    \definition[家,个]{s.}{restaurante | lanchonete}
  \end{phonetics}
\end{entry}

\begin{entry}{饮料}{7,10}[Radicais ⾷、⽃]
  \begin{phonetics}{饮料}{yin3liao4}
    \definition{s.}{bebida}
  \end{phonetics}
\end{entry}

\begin{entry}{驱}{7}[Radical ⾺]
  \begin{phonetics}{驱}{qu1}
    \definition{v.}{expulsar | repelir}
  \end{phonetics}
\end{entry}

\begin{entry}{驴}{7}[Radical ⾺]
  \begin{phonetics}{驴}{lv2}
    \definition[头]{s.}{burro | asno | jumento | jegue}
  \end{phonetics}
\end{entry}

\begin{entry}{鸡}{7}[Radical ⿃]
  \begin{phonetics}{鸡}{ji1}[][HSK 2]
    \definition[只]{s.}{galo, galinha | (gíria) prostituta}
  \end{phonetics}
\end{entry}

\begin{entry}{鸡蛋}{7,11}[Radicais ⿃、⾍]
  \begin{phonetics}{鸡蛋}{ji1dan4}[][HSK 1]
    \definition[个,打]{s.}{ovo de galinha}
  \end{phonetics}
\end{entry}

\begin{entry}{麦当劳}{7,6,7}[Radicais ⿆、⼹、⼒]
  \begin{phonetics}{麦当劳}{mai4dang1lao2}
    \definition*{s.}{McDonald's (empresa de \emph{fast-food})}
  \seealsoref{麦当劳叔叔}{mai4dang1lao2 shu1shu5}
  \end{phonetics}
\end{entry}

\begin{entry}{麦当劳叔叔}{7,6,7,8,8}[Radicais ⿆、⼹、⼒、⼜、⼜]
  \begin{phonetics}{麦当劳叔叔}{mai4dang1lao2 shu1shu5}
    \definition*{s.}{Ronald McDonald}
  \seealsoref{麦当劳}{mai4dang1lao2}
  \end{phonetics}
\end{entry}

\begin{entry}{麦淇淋}{7,11,11}[Radicais ⿆、⽔、⽔]
  \begin{phonetics}{麦淇淋}{mai4qi2lin2}
    \definition{s.}{(empréstimo linguístico) margarina}
  \end{phonetics}
\end{entry}

\begin{entry}{龟速}{7,10}[Radicais ⿔、⾡]
  \begin{phonetics}{龟速}{gui1su4}
    \definition{adv.}{tão lento quanto uma tartaruga}
  \end{phonetics}
\end{entry}

%%%%% EOF %%%%%

