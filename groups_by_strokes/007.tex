%%%
%%% 7画
%%%

\section*{7画}\addcontentsline{toc}{section}{7画}

\begin{entry}{两}{7}[Radical 一]
  \begin{phonetics}{两}{liang3}[][HSK 1]
    \definition{adv.}{ambos (lados) | cada (lado)}
    \definition{clas.}{liang, uma unidade de peso (=50 gramas)}
    \definition{num.}{dois (sempre usado antes de classificadores) | poucos; alguns}
  \end{phonetics}
\end{entry}

\begin{entry}{两码事}{7,8,8}
  \begin{phonetics}{两码事}{liang3ma3shi4}
    \definition{expr.}{duas coisas completamente diferentes}
  \end{phonetics}
\end{entry}

\begin{entry}{严重}{7,9}
  \begin{phonetics}{严重}{yan2zhong4}
    \definition{adj.}{crítico | grave | sério | severo}
  \end{phonetics}
\end{entry}

\begin{entry}{严重打伤}{7,9,5,6}
  \begin{phonetics}{严重打伤}{yan2zhong4 da3 shang1}
    \definition{s.}{gravemente ferido}
  \end{phonetics}
\end{entry}

\begin{entry}{严重伤害}{7,9,6,10}
  \begin{phonetics}{严重伤害}{yan2zhong4 shang1hai4}
    \definition{s.}{ferimento grave}
  \end{phonetics}
\end{entry}

\begin{entry}{严重关切}{7,9,6,4}
  \begin{phonetics}{严重关切}{yan2zhong4guan1qie4}
    \definition{s.}{preocupação séria}
  \end{phonetics}
\end{entry}

\begin{entry}{严重危害}{7,9,6,10}
  \begin{phonetics}{严重危害}{yan2zhong4wei1hai4}
    \definition{s.}{danos graves}
  \end{phonetics}
\end{entry}

\begin{entry}{严重后果}{7,9,6,8}
  \begin{phonetics}{严重后果}{yan2zhong4hou4guo3}
    \definition{s.}{consequências sérias | repercursões graves}
  \end{phonetics}
\end{entry}

\begin{entry}{严重地}{7,9,6}
  \begin{phonetics}{严重地}{yan2zhong4 di4}
    \definition{adv.}{seriamente | gravemente}
  \end{phonetics}
\end{entry}

\begin{entry}{严重问题}{7,9,6,15}
  \begin{phonetics}{严重问题}{yan2zhong4wen4ti2}
    \definition{s.}{problema sério}
  \end{phonetics}
\end{entry}

\begin{entry}{严重性}{7,9,8}
  \begin{phonetics}{严重性}{yan2zhong4xing4}
    \definition{s.}{seriedade | gravidade}
  \end{phonetics}
\end{entry}

\begin{entry}{严重破坏}{7,9,10,7}
  \begin{phonetics}{严重破坏}{yan2zhong4 po4huai4}
    \definition{s.}{destruição grave}
  \end{phonetics}
\end{entry}

\begin{entry}{乱}{7}[Radical 乙]
  \begin{phonetics}{乱}{luan4}
    \definition{adj.}{indiscriminado | aleatório | arbitrário}
    \definition{adv.}{em confusão ou desordem | em um estado de espírito confuso}
    \definition{s.}{desordem | revolta | rebelião | relações sexuais ilícitas}
    \definition{v.}{jogar em desordem | misturar}
  \end{phonetics}
\end{entry}

\begin{entry}{估计}{7,4}
  \begin{phonetics}{估计}{gu1ji4}
    \definition{v.}{estimar | avaliar | calcular}
  \end{phonetics}
\end{entry}

\begin{entry}{伲}{7}[Radical 人]
  \begin{phonetics}{伲}{ni3}
    \variantof{你}
  \end{phonetics}
\end{entry}

\begin{entry}{伴侣}{7,8}
  \begin{phonetics}{伴侣}{ban4lv3}
    \definition{s.}{companheiro | parceiro}
  \end{phonetics}
\end{entry}

\begin{entry}{但}{7}[Radical 人]
  \begin{phonetics}{但}{dan4}[][HSK 2]
    \definition{conj.}{mas | ainda | no entanto | apenas}
  \end{phonetics}
\end{entry}

\begin{entry}{但是}{7,9}
  \begin{phonetics}{但是}{dan4 shi4}[][HSK 2]
    \definition{conj.}{mas | ainda | no entanto}
  \end{phonetics}
\end{entry}

\begin{entry}{位}{7}[Radical 人]
  \begin{phonetics}{位}{wei4}[][HSK 2]
    \definition{clas.}{para pessoas (com cortesia) | para bits binários}
    \definition{s.}{(física) potencial | localização | lugar | posição | assento}
    \example{十六位}[16 bits]
  \end{phonetics}
\end{entry}

\begin{entry}{位子}{7,3}
  \begin{phonetics}{位子}{wei4zi5}
    \definition{s.}{lugar | assento}
  \end{phonetics}
\end{entry}

\begin{entry}{位居}{7,8}
  \begin{phonetics}{位居}{wei4ju1}
    \definition{v.}{estar localizado em}
  \end{phonetics}
\end{entry}

\begin{entry}{位置}{7,13}
  \begin{phonetics}{位置}{wei4zhi5}
    \definition[个]{s.}{lugar | posição | assento}
  \end{phonetics}
\end{entry}

\begin{entry}{低}{7}[Radical 人]
  \begin{phonetics}{低}{di1}[][HSK 2]
    \definition{adj.}{baixo}
    \definition{adv.}{abaixo}
    \definition{v.}{abaixar (a cabeça) | deixar cair | pendurar | inclinar}
  \end{phonetics}
\end{entry}

\begin{entry}{住}{7}[Radical 人]
  \begin{phonetics}{住}{zhu4}[][HSK 1]
    \definition{v.}{habitar | residir | morar | alojar-se}
  \end{phonetics}
\end{entry}

\begin{entry}{住处}{7,5}
  \begin{phonetics}{住处}{zhu4chu4}
    \definition{s.}{morada | habitação | residência}
  \end{phonetics}
\end{entry}

\begin{entry}{住宅}{7,6}
  \begin{phonetics}{住宅}{zhu4zhai2}
    \definition{s.}{residência}
  \end{phonetics}
\end{entry}

\begin{entry}{住房}{7,8}
  \begin{phonetics}{住房}{zhu4fang2}[][HSK 2]
    \definition{s.}{habitação}
  \end{phonetics}
\end{entry}

\begin{entry}{住所}{7,8}
  \begin{phonetics}{住所}{zhu4suo3}
    \definition[处]{s.}{morada | habitação | residência}
  \end{phonetics}
\end{entry}

\begin{entry}{住院}{7,9}
  \begin{phonetics}{住院}{zhu4 yuan4}[][HSK 2]
    \definition{v.}{estar hospitalizado | estar no hospital}
  \end{phonetics}
\end{entry}

\begin{entry}{住嘴}{7,16}
  \begin{phonetics}{住嘴}{zhu4zui3}
    \definition{interj.}{Cale-se!}
    \definition{v.}{calar | calar-se}
  \end{phonetics}
\end{entry}

\begin{entry}{体内}{7,4}
  \begin{phonetics}{体内}{ti3nei4}
    \definition{adj.}{dentro do corpo | \emph{in vivo} (versus \emph{in vitro} | interno a}
  \end{phonetics}
\end{entry}

\begin{entry}{体育}{7,8}
  \begin{phonetics}{体育}{ti3yu4}[][HSK 2]
    \definition{s.}{treinamento físico | esportes | atividades esportivas}
  \end{phonetics}
\end{entry}

\begin{entry}{体育场}{7,8,6}
  \begin{phonetics}{体育场}{ti3 yu4 chang3}[][HSK 2]
    \definition[个,座]{s.}{estádio | campo de esportes}
  \end{phonetics}
\end{entry}

\begin{entry}{体育馆}{7,8,11}
  \begin{phonetics}{体育馆}{ti3 yu4 guan3}[][HSK 2]
    \definition[个]{s.}{ginásio | estádio}
  \end{phonetics}
\end{entry}

\begin{entry}{体验}{7,10}
  \begin{phonetics}{体验}{ti3yan4}
    \definition{v.}{vivenciar | experimentar por si mesmo}
  \end{phonetics}
\end{entry}

\begin{entry}{何况}{7,7}
  \begin{phonetics}{何况}{he2kuang4}
    \definition{conj.}{além disso | muito menos}
  \end{phonetics}
\end{entry}

\begin{entry}{佛}{7}[Radical 人]
  \begin{phonetics}{佛}{fo2}
    \definition*{s.}{Buda, abreviação de 佛陀 | Budismo}
  \seealsoref{佛陀}{fo2tuo2}
  \end{phonetics}
  \begin{phonetics}{佛}{fu2}
    \definition{adv.}{aparentemente}
    \definition{s.}{ornamento da cabeça (feminino)}
  \end{phonetics}
\end{entry}

\begin{entry}{佛陀}{7,7}
  \begin{phonetics}{佛陀}{fo2tuo2}
    \definition{s.}{Buda (uma pessoa que atingiu a Budeidade, ou especificamente Siddhartha Gautama)}
  \end{phonetics}
\end{entry}

\begin{entry}{作}{7}[Radical 人]
  \begin{phonetics}{作}{zuo1}
    \definition{adj.}{(gíria) incômodo}
    \definition{s.}{trabalhador | oficina | (pessoa) de alta manutenção}
  \end{phonetics}
  \begin{phonetics}{作}{zuo4}
    \definition{s.}{escritos ou obras}
    \definition{v.}{fazer | crescer | escrever ou compor | fingir | considerar como | sentir}
  \end{phonetics}
\end{entry}

\begin{entry}{作文}{7,4}
  \begin{phonetics}{作文}{zuo4wen2}[][HSK 2]
    \definition[篇]{s.}{ensaio |  composição | redação}
    \definition{v.+compl.}{(de alunos) para escrever uma redação}
  \end{phonetics}
\end{entry}

\begin{entry}{作业}{7,5}
  \begin{phonetics}{作业}{zuo4ye4}[][HSK 2]
    \definition[份,个]{s.}{tarefa escolar | trabalho | tarefa | operação}
  \end{phonetics}
\end{entry}

\begin{entry}{作用}{7,5}
  \begin{phonetics}{作用}{zuo4yong4}[][HSK 2]
    \definition{s.}{efeito | ação | função}
    \definition{v.}{afetar | agir em}
  \end{phonetics}
\end{entry}

\begin{entry}{作家}{7,10}
  \begin{phonetics}{作家}{zuo4jia1}[][HSK 2]
    \definition[位,个]{s.}{autor | escritor}
  \end{phonetics}
\end{entry}

\begin{entry}{你}{7}[Radical 人]
  \begin{phonetics}{你}{ni3}[][HSK 1]
    \definition{pron.}{você (informal) | tu | te | ti | contigo}
    \seeref{您}{nin2}
  \end{phonetics}
\end{entry}

\begin{entry}{你们}{7,5}
  \begin{phonetics}{你们}{ni3men5}[][HSK 1]
    \definition{pron.}{vocês (informal) | vós | vos | convosco}
  \end{phonetics}
\end{entry}

\begin{entry}{你们的}{7,5,8}
  \begin{phonetics}{你们的}{ni3men5 de5}
    \definition{pron.}{vossos}
  \end{phonetics}
\end{entry}

\begin{entry}{你好}{7,6}
  \begin{phonetics}{你好}{ni3hao3}
    \definition{interj.}{Olá! | Oi!}
  \end{phonetics}
\end{entry}

\begin{entry}{你的}{7,8}
  \begin{phonetics}{你的}{ni3 de5}
    \definition{pron.}{seu}
  \end{phonetics}
\end{entry}

\begin{entry}{克}{7}[Radical 十]
  \begin{phonetics}{克}{ke4}[][HSK 2]
    \definition{clas.}{grama (g)}
    \definition{v.}{pode | ser capaz de | restringir | controlar | superar | subjugar | capturar (uma cidade, etc.) | digerir | cortar | reduzir | definir um limite de tempo}
  \end{phonetics}
  \begin{phonetics}{克}{kei1}[][HSK 2]
    \definition{clas.}{grama (g)}
    \definition{v.}{pode | ser capaz de | restringir | controlar | superar | subjugar | capturar (uma cidade, etc.) | digerir | cortar | reduzir | definir um limite de tempo}
  \end{phonetics}
\end{entry}

\begin{entry}{免得}{7,11}
  \begin{phonetics}{免得}{mian3de5}
    \definition{conj.}{de modo a não | para evitar | para que não}
  \end{phonetics}
\end{entry}

\begin{entry}{免税}{7,12}
  \begin{phonetics}{免税}{mian3shui4}
    \definition{adj.}{isento de impostos (tributação)}
    \definition{s.}{livre de impostos | isenção de impostos}
    \definition{v.+compl.}{isentar impostos}
  \end{phonetics}
\end{entry}

\begin{entry}{兵}{7}[Radical 八]
  \begin{phonetics}{兵}{bing1}
    \definition[个]{s.}{soldado | militar | exército}
  \end{phonetics}
\end{entry}

\begin{entry}{兵器}{7,16}
  \begin{phonetics}{兵器}{bing1qi4}
    \definition{s.}{armas | armamento}
  \end{phonetics}
\end{entry}

\begin{entry}{况且}{7,5}
  \begin{phonetics}{况且}{kuang4qie3}
    \definition{conj.}{além disso | além do mais}
  \end{phonetics}
\end{entry}

\begin{entry}{冷}{7}[Radical 冫]
  \begin{phonetics}{冷}{leng3}[][HSK 1]
    \definition*{s.}{sobrenome Leng}
    \definition{adj.}{frio}
  \end{phonetics}
\end{entry}

\begin{entry}{初}{7}[Radical 衣]
  \begin{phonetics}{初}{chu1}[][HSK 3]
    \definition*{s.}{sobrenome Chu}
    \definition{adj.}{primeiro (em ordem) | elementar; rudimentar | original}
    \definition{adv.}{pela primeira vez}
    \definition{pref.}{anexado aos numerais de um a dez para indicar ordem (primeiro ao décimo)}
    \definition{s.}{no início de; na primeira parte de | o estágio júnior (pleno; sênior)}
  \end{phonetics}
\end{entry}

\begin{entry}{初中}{7,4}
  \begin{phonetics}{初中}{chu1 zhong1}[][HSK 3]
    \definition[所,个]{s.}{ensino médio; ensino fundamental}
  \end{phonetics}
\end{entry}

\begin{entry}{初心}{7,4}
  \begin{phonetics}{初心}{chu1xin1}
    \definition{s.}{intenção original de alguém, aspiração, etc. | (budismo) ``mente do iniciante'' (ter a mente aberta quando estudando um assunto como um iniciante no assunto teria)}
  \end{phonetics}
\end{entry}

\begin{entry}{初级}{7,6}
  \begin{phonetics}{初级}{chu1ji2}[][HSK 3]
    \definition{adj.}{elementar; primário; júnior; inicial}
  \end{phonetics}
\end{entry}

\begin{entry}{初步}{7,7}
  \begin{phonetics}{初步}{chu1bu4}[][HSK 3]
    \definition{adj.}{inicial; preliminar}
  \end{phonetics}
\end{entry}

\begin{entry}{别}{7}[Radical 刀]
  \begin{phonetics}{别}{bie2}[][HSK 1]
    \definition*{s.}{sobrenome Bie}
    \definition{adv.}{nada de (pedir a alguém para não fazer) | é melhor não | não}
    \definition{pron.}{outro}
    \definition{v.}{classificar | separar | distinguir | partir | deixar | fixar | colar alguma coisa em}
  \end{phonetics}
  \begin{phonetics}{别}{bie4}[][HSK 0]
    \definition{v.}{fazer com que alguém mude seus hábitos, opiniões, etc.}
  \end{phonetics}
\end{entry}

\begin{entry}{别人}{7,2}
  \begin{phonetics}{别人}{bie2ren5}[][HSK 1]
    \definition{pron.}{outra pessoa | outro povo | outros}
  \end{phonetics}
\end{entry}

\begin{entry}{别的}{7,8}
  \begin{phonetics}{别的}{bie2de5}[][HSK 1]
    \definition{pron.}{outro}
  \end{phonetics}
\end{entry}

\begin{entry}{别说}{7,9}
  \begin{phonetics}{别说}{bie2shuo1}
    \definition{v.}{não falar de | não mencionar}
  \end{phonetics}
\end{entry}

\begin{entry}{助兴}{7,6}
  \begin{phonetics}{助兴}{zhu4xing4}
    \definition{v.+compl.}{animar as coisas | juntar-se à diversão}
  \end{phonetics}
\end{entry}

\begin{entry}{努力}{7,2}
  \begin{phonetics}{努力}{nu3li4}[][HSK 2]
    \definition{adj.}{diligente | aplicado}
    \definition{s.}{esforçar-se | se esforçar}
  \end{phonetics}
\end{entry}

\begin{entry}{劳工同事}{7,3,6,8}
  \begin{phonetics}{劳工同事}{lao2gong1 tong2shi4}
    \definition{s.}{colaborador | colega de trabalho}
  \end{phonetics}
\end{entry}

\begin{entry}{医}{7}[Radical 匸]
  \begin{phonetics}{医}{yi1}
    \definition{s.}{médico | medicina}
    \definition{v.}{curar | tratar}
  \end{phonetics}
\end{entry}

\begin{entry}{医生}{7,5}
  \begin{phonetics}{医生}{yi1sheng1}[][HSK 1]
    \definition[个,位,名]{s.}{médico | clínico}
  \end{phonetics}
\end{entry}

\begin{entry}{医院}{7,9}
  \begin{phonetics}{医院}{yi1yuan4}[][HSK 1]
    \definition[所,家,座]{s.}{hospital}
  \end{phonetics}
\end{entry}

\begin{entry}{即}{7}[Radical 卩]
  \begin{phonetics}{即}{ji2}
    \definition{conj.}{e | até | mesmo se/embora}
  \end{phonetics}
\end{entry}

\begin{entry}{即使}{7,8}
  \begin{phonetics}{即使}{ji2shi3}
    \definition{conj.}{mesmo se/embora}
  \end{phonetics}
\end{entry}

\begin{entry}{即或}{7,8}
  \begin{phonetics}{即或}{ji2huo4}
    \definition{conj.}{mesmo se/embora}
  \end{phonetics}
\end{entry}

\begin{entry}{即若}{7,8}
  \begin{phonetics}{即若}{ji2ruo4}
    \definition{conj.}{mesmo se/embora}
  \end{phonetics}
\end{entry}

\begin{entry}{即便}{7,9}
  \begin{phonetics}{即便}{ji2bian4}
    \definition{conj.}{mesmo se/embora}
  \end{phonetics}
\end{entry}

\begin{entry}{即是}{7,9}
  \begin{phonetics}{即是}{ji2shi4}
    \definition{conj.}{aquilo é}
  \end{phonetics}
\end{entry}

\begin{entry}{却}{7}[Radical 卩]
  \begin{phonetics}{却}{que4}
    \definition{adv.}{mas | contudo | entretanto}
  \end{phonetics}
\end{entry}

\begin{entry}{却是}{7,9}
  \begin{phonetics}{却是}{que4shi4}
    \definition{conj.}{no entanto | realmente | o fato é\dots | mas isso é\dots}
  \end{phonetics}
\end{entry}

\begin{entry}{君主立宪制}{7,5,5,9,8}
  \begin{phonetics}{君主立宪制}{jun1zhu3li4xian4zhi4}
    \definition{s.}{monarquia constitucional}
  \end{phonetics}
\end{entry}

\begin{entry}{吟诗}{7,8}
  \begin{phonetics}{吟诗}{yin2shi1}
    \definition{v.}{recitar poesia}
  \end{phonetics}
\end{entry}

\begin{entry}{否认}{7,4}
  \begin{phonetics}{否认}{fou3ren4}[][HSK 3]
    \definition{v.}{negar; repudiar}
  \end{phonetics}
\end{entry}

\begin{entry}{否则}{7,6}
  \begin{phonetics}{否则}{fou3ze2}
    \definition{conj.}{caso contrário | ou}
  \end{phonetics}
\end{entry}

\begin{entry}{否定}{7,8}
  \begin{phonetics}{否定}{fou3ding4}[][HSK 3]
    \definition{adj.}{negativo}
    \definition{s.}{negativo (resposta); negação}
    \definition{v.}{rejeitar; negar}
  \end{phonetics}
\end{entry}

\begin{entry}{吧}{7}[Radical 口]
  \begin{phonetics}{吧}{ba1}
    \definition{s.}{bar (servindo bebidas ou fornecendo acesso à \emph{Internet}) | (onomatopéia) \emph{Bang!}}
    \definition{v.}{soprar (em um cachimbo, etc.)}
  \end{phonetics}
  \begin{phonetics}{吧}{ba5}
    \definition{part.}{partícula modal indicando sugestão ou suposição | \dots eu presumo. | \dots OK? | \dots certo?}
  \end{phonetics}
  \begin{phonetics}{吧}{bia1}
    \definition{s.}{bar (servindo bebidas ou fornecendo acesso à \emph{Internet}) | (onomatopéia) \emph{Smack!} (beijo)}
    \definition{v.}{soprar (em um cachimbo, etc.)}
  \end{phonetics}
\end{entry}

\begin{entry}{含金量}{7,8,12}
  \begin{phonetics}{含金量}{han2jin1liang4}
    \definition{adj.}{conteúdo de ouro | (fig.) valioso}
  \end{phonetics}
\end{entry}

\begin{entry}{听}{7}[Radical 口]
  \begin{phonetics}{听}{ting1}[][HSK 1]
    \definition{clas.}{para bebidas enlatadas}
    \definition{s.}{lata de bebida (empréstimo linguístico, do inglês ``\emph{tin}'')}
    \definition{v.}{ouvir | escutar | obedecer}
  \end{phonetics}
\end{entry}

\begin{entry}{听力}{7,2}
  \begin{phonetics}{听力}{ting1li4}
    \definition{s.}{audição | capacidade de compreensão oral}
  \end{phonetics}
\end{entry}

\begin{entry}{听力理解}{7,2,11,13}
  \begin{phonetics}{听力理解}{ting1li4li3jie3}
    \definition{s.}{compreensão auditiva}
  \end{phonetics}
\end{entry}

\begin{entry}{听小骨}{7,3,9}
  \begin{phonetics}{听小骨}{ting1xiao3gu3}
    \definition{s.}{ossículos (do ouvido médio)}
  \seealsoref{听骨}{ting1gu3}
  \end{phonetics}
\end{entry}

\begin{entry}{听见}{7,4}
  \begin{phonetics}{听见}{ting1 jian4}[][HSK 1]
    \definition{v.}{ouvir}
  \end{phonetics}
\end{entry}

\begin{entry}{听写}{7,5}
  \begin{phonetics}{听写}{ting1xie3}[][HSK 1]
    \definition{s.}{ditado}
    \definition{v.}{transcrever música de ouvido | escrever (em um exercício de ditado)}
  \end{phonetics}
\end{entry}

\begin{entry}{听会}{7,6}
  \begin{phonetics}{听会}{ting1hui4}
    \definition{v.}{participar de uma reunião (e ouvir o que é discutido)}
  \end{phonetics}
\end{entry}

\begin{entry}{听戏}{7,6}
  \begin{phonetics}{听戏}{ting1xi4}
    \definition{v.}{assistir a uma ópera | ver uma ópera}
  \end{phonetics}
\end{entry}

\begin{entry}{听讲}{7,6}
  \begin{phonetics}{听讲}{ting1 jiang3}[][HSK 2]
    \definition{v.+compl.}{assistir a uma palestra; ouvir uma conversa}
  \end{phonetics}
\end{entry}

\begin{entry}{听来}{7,7}
  \begin{phonetics}{听来}{ting1lai2}
    \definition{v.}{ouvir de algum lugar | soar (antigo, estrangeiro, excitante, certo, etc.) | soar como se (ou seja, dar uma impressão ao ouvinte)}
  \end{phonetics}
\end{entry}

\begin{entry}{听凭}{7,8}
  \begin{phonetics}{听凭}{ting1ping2}
    \definition{v.}{permitir (alguém a fazer o que desejar)}
  \end{phonetics}
\end{entry}

\begin{entry}{听到}{7,8}
  \begin{phonetics}{听到}{ting1dao4}[][HSK 1]
    \definition{v.}{ouvir | notar}
  \end{phonetics}
\end{entry}

\begin{entry}{听命}{7,8}
  \begin{phonetics}{听命}{ting1ming4}
    \definition{v.}{obedecer ordens | receber ordens}
  \end{phonetics}
\end{entry}

\begin{entry}{听说}{7,9}
  \begin{phonetics}{听说}{ting1 shuo1}[][HSK 2]
    \definition{v.}{ouvir dizer}
  \end{phonetics}
\end{entry}

\begin{entry}{听骨}{7,9}
  \begin{phonetics}{听骨}{ting1gu3}
    \definition{s.}{ossículos (do ouvido médio)}
  \seealsoref{听小骨}{ting1xiao3gu3}
  \end{phonetics}
\end{entry}

\begin{entry}{听断}{7,11}
  \begin{phonetics}{听断}{ting1duan4}
    \definition{v.}{ouvir e decidir | julgar (ou seja, ouvir e julgar em um tribunal)}
  \end{phonetics}
\end{entry}

\begin{entry}{听随}{7,11}
  \begin{phonetics}{听随}{ting1sui2}
    \definition{v.}{permitir | obedecer}
  \end{phonetics}
\end{entry}

\begin{entry}{吵}{7}[Radical ⼝]
  \begin{phonetics}{吵}{chao3}[][HSK 3]
    \definition{adj.}{barulhento; ruidoso}
    \definition{v.}{perturbar fazendo barulho; fazer barulho | discutir; brigar; disputar}
  \end{phonetics}
\end{entry}

\begin{entry}{吵架}{7,9}
  \begin{phonetics}{吵架}{chao3jia4}[][HSK 3]
    \definition{v.+compl.}{brigar; discutir; ter uma briga}
  \end{phonetics}
\end{entry}

\begin{entry}{吹}{7}[Radical 口]
  \begin{phonetics}{吹}{chui1}[][HSK 2]
    \definition{v.}{soprar | tocar (instrumentos de sopro) | bajular |  louvar aos céus | separar (casal)  | fracassar}
  \end{phonetics}
\end{entry}

\begin{entry}{吹牛}{7,4}
  \begin{phonetics}{吹牛}{chui1niu2}
    \definition{v.+compl.}{ogulhar-se | gabar-se | destacar-se}
  \end{phonetics}
\end{entry}

\begin{entry}{吾}{7}[Radical 口]
  \begin{phonetics}{吾}{wu2}
    \definition{pron.}{eu | (antigo) meu}
    \definition{s.}{sobrenome Wu}
  \end{phonetics}
\end{entry}

\begin{entry}{告别}{7,7}
  \begin{phonetics}{告别}{gao4bie2}[][HSK 3]
    \definition{v.+compl.}{dizer adeus a | deixar; partir de | prestar as últimas homenagens ao falecido}
  \end{phonetics}
\end{entry}

\begin{entry}{告诉}{7,7}
  \begin{phonetics}{告诉}{gao4su4}[][HSK 0]
    \definition{v.}{apresentar queixa | registar uma reclamação}
  \end{phonetics}
  \begin{phonetics}{告诉}{gao4su5}[][HSK 1]
    \definition{v.}{contar | dar a conhecer | informar}
  \end{phonetics}
\end{entry}

\begin{entry}{告急}{7,9}
  \begin{phonetics}{告急}{gao4ji2}
    \definition{v.}{estar em estado de emergência | relatar uma emergência | solicitar assistência de emergência}
  \end{phonetics}
\end{entry}

\begin{entry}{囯}{7}[Radical 囗]
  \begin{phonetics}{囯}{guo2}
    \variantof{国}
  \end{phonetics}
\end{entry}

\begin{entry}{困}{7}[Radical ⼞]
  \begin{phonetics}{困}{kun4}
    \definition{adj.}{pressionado | preso}
    \definition{v.}{prender | cercar}
  \end{phonetics}
\end{entry}

\begin{entry}{困难}{7,10}
  \begin{phonetics}{困难}{kun4nan5}
    \definition{adj.}{difícil | desafiante}
    \definition{s.}{situação difícil}
  \end{phonetics}
\end{entry}

\begin{entry}{坏}{7}[Radical 土]
  \begin{phonetics}{坏}{huai4}[][HSK 1]
    \definition{adj.}{avariado | mau}
    \definition{v.}{perder o controle}
  \end{phonetics}
\end{entry}

\begin{entry}{坏人}{7,2}
  \begin{phonetics}{坏人}{huai4 ren2}[][HSK 2]
    \definition[个]{s.}{malfeitor | canalha | pessoa má}
  \end{phonetics}
\end{entry}

\begin{entry}{坏处}{7,5}
  \begin{phonetics}{坏处}{huai4 chu4}[][HSK 2]
    \definition[个]{s.}{dano | problema}
  \end{phonetics}
\end{entry}

\begin{entry}{坏蛋}{7,11}
  \begin{phonetics}{坏蛋}{huai4dan4}
    \definition{s.}{bastardo | canalha | pessoa má}
  \end{phonetics}
\end{entry}

\begin{entry}{坐}{7}[Radical 土]
  \begin{phonetics}{坐}{zuo4}[][HSK 1]
    \definition*{s.}{sobrenome Zuo}
    \definition{v.}{sentar-se | andar de carro, ônibus, trem, avião, etc.}
  \end{phonetics}
\end{entry}

\begin{entry}{坐下}{7,3}
  \begin{phonetics}{坐下}{zuo4xia5}[][HSK 1]
    \definition{v.}{sentar-se | tomar um assento}
  \end{phonetics}
\end{entry}

\begin{entry}{坐车}{7,4}
  \begin{phonetics}{坐车}{zuo4che1}
    \definition{v.}{andar de carro, ônibus, trem, etc.}
  \end{phonetics}
\end{entry}

\begin{entry}{坐好}{7,6}
  \begin{phonetics}{坐好}{zuo4hao3}
    \definition{v.}{sentar-se corretamente | sentar direito}
  \end{phonetics}
\end{entry}

\begin{entry}{坐享}{7,8}
  \begin{phonetics}{坐享}{zuo4xiang3}
    \definition{v.}{curtir algo sem levantar um dedo}
  \end{phonetics}
\end{entry}

\begin{entry}{坐垫}{7,9}
  \begin{phonetics}{坐垫}{zuo4dian4}
    \definition[块]{s.}{assento (motocicleta) | almofada}
  \end{phonetics}
\end{entry}

\begin{entry}{坐标}{7,9}
  \begin{phonetics}{坐标}{zuo4biao1}
    \definition{s.}{coordenada (geometria)}
  \end{phonetics}
\end{entry}

\begin{entry}{坑}{7}[Radical 土]
  \begin{phonetics}{坑}{keng1}
    \definition{s.}{poço | depressão | túnel | buraco no chão}
    \definition{v.}{enganar | trapacear}
  \end{phonetics}
\end{entry}

\begin{entry}{坑人}{7,2}
  \begin{phonetics}{坑人}{keng1ren2}
    \definition{v.+compl.}{trapacear alguém}
  \end{phonetics}
\end{entry}

\begin{entry}{块}{7}[Radical 土]
  \begin{phonetics}{块}{kuai4}[][HSK 1]
    \definition{clas.}{(coloquial) para dinheiro e unidades monetárias | para peças ou pedaços de roupa, bolos, sabão, etc.}
    \definition{s.}{pedaço | pedaço (de terra) | peça}
  \end{phonetics}
\end{entry}

\begin{entry}{坚决}{7,6}
  \begin{phonetics}{坚决}{jian1jue2}[][HSK 3]
    \definition{adj.}{firme; resoluto}
  \end{phonetics}
\end{entry}

\begin{entry}{坚守}{7,6}
  \begin{phonetics}{坚守}{jian1shou3}
    \definition{v.}{agarrar-se}
  \end{phonetics}
\end{entry}

\begin{entry}{坚持}{7,9}
  \begin{phonetics}{坚持}{jian1chi2}[][HSK 3]
    \definition{v.}{persistir e; perseverar em; sustentar; insistir em; manter-se fiel a; aderir a}
  \end{phonetics}
\end{entry}

\begin{entry}{坚强}{7,12}
  \begin{phonetics}{坚强}{jian1qiang2}[][HSK 3]
    \definition{adj.}{forte; firme; convicto}
    \definition{v.}{fortalecer; tornar forte}
  \end{phonetics}
\end{entry}

\begin{entry}{坠}{7}[Radical 土]
  \begin{phonetics}{坠}{zhui4}
    \definition{v.}{cair | pesar | fazer vergar com o peso}
  \end{phonetics}
\end{entry}

\begin{entry}{坠落}{7,12}
  \begin{phonetics}{坠落}{zhui4luo4}
    \definition{v.}{cair}
  \end{phonetics}
\end{entry}

\begin{entry}{声明}{7,8}
  \begin{phonetics}{声明}{sheng1ming2}
    \definition[项,份]{s.}{declaração}
    \definition{v.}{declarar}
  \end{phonetics}
\end{entry}

\begin{entry}{声音}{7,9}
  \begin{phonetics}{声音}{sheng1yin1}[][HSK 2]
    \definition[个,种]{s.}{som | voz}
  \end{phonetics}
\end{entry}

\begin{entry}{壳}{7}[Radical 士]
  \begin{phonetics}{壳}{ke2}
    \definition{s.}{casca (de ovo, noz, caranguejo, etc.) | caixa | invólucro | alojamento (de uma máquina ou dispositivo)}
  \end{phonetics}
\end{entry}

\begin{entry}{妖}{7}[Radical 女]
  \begin{phonetics}{妖}{yao1}
    \definition{adj.}{enfeitiçante | encantador}
    \definition{s.}{\emph{goblin} | bruxa | diabo | monstro | fantasma | demônio}
  \end{phonetics}
\end{entry}

\begin{entry}{妙招}{7,8}
  \begin{phonetics}{妙招}{miao4zhao1}
    \definition{adj.}{escorregadio}
    \definition{s.}{movimento inteligente | maneira inteligente de fazer algo}
  \end{phonetics}
\end{entry}

\begin{entry}{宋}{7}[Radical 宀]
  \begin{phonetics}{宋}{song4}
    \definition*{s.}{sobrenome Song}
    \definition{s.}{Dinastia Song (960-1279) | Song das dinastias do sul (420-479)}
  \end{phonetics}
\end{entry}

\begin{entry}{完}{7}[Radical 宀]
  \begin{phonetics}{完}{wan2}[][HSK 2]
    \definition{adj.}{completo | inteiro}
    \definition{adv.}{todo}
    \definition{v.}{acabar | completar | terminar}
  \end{phonetics}
\end{entry}

\begin{entry}{完人}{7,2}
  \begin{phonetics}{完人}{wan2ren2}
    \definition{s.}{pessoa perfeita}
  \end{phonetics}
\end{entry}

\begin{entry}{完全}{7,6}
  \begin{phonetics}{完全}{wan2quan2}[][HSK 2]
    \definition{adj.}{completo | todo}
    \definition{adv.}{inteiramente | totalmente}
  \end{phonetics}
\end{entry}

\begin{entry}{完成}{7,6}
  \begin{phonetics}{完成}{wan2cheng2}[][HSK 2]
    \definition{v.}{realizar | completar}
  \end{phonetics}
\end{entry}

\begin{entry}{完毕}{7,6}
  \begin{phonetics}{完毕}{wan2bi4}
    \definition{v.}{completar | terminar | acabar}
  \end{phonetics}
\end{entry}

\begin{entry}{完完全全}{7,7,6,6}
  \begin{phonetics}{完完全全}{wan2wan2quan2quan2}
    \definition{adv.}{completamente}
  \end{phonetics}
\end{entry}

\begin{entry}{完备}{7,8}
  \begin{phonetics}{完备}{wan2bei4}
    \definition{adj.}{completo | impecável | perfeito}
    \definition{v.}{não deixar nada a desejar}
  \end{phonetics}
\end{entry}

\begin{entry}{完美}{7,9}
  \begin{phonetics}{完美}{wan2mei3}
    \definition{adj.}{perfeito}
    \definition{adv.}{perfeitamente}
    \definition{s.}{perfeição}
  \end{phonetics}
\end{entry}

\begin{entry}{完税}{7,12}
  \begin{phonetics}{完税}{wan2shui4}
    \definition{v.}{pagar imposto}
  \end{phonetics}
\end{entry}

\begin{entry}{完满}{7,13}
  \begin{phonetics}{完满}{wan2man3}
    \definition{adj.}{satisfatório | bem-sucedido}
  \end{phonetics}
\end{entry}

\begin{entry}{尾巴}{7,4}
  \begin{phonetics}{尾巴}{wei3ba5}
    \definition{s.}{cauda}
  \end{phonetics}
\end{entry}

\begin{entry}{尿}{7}[Radical 尸]
  \begin{phonetics}{尿}{niao4}
    \definition[泡]{s.}{urina}
    \definition{v.}{urinar}
  \end{phonetics}
  \begin{phonetics}{尿}{sui1}
    \definition{s.}{(coloquial) urina}
  \end{phonetics}
\end{entry}

\begin{entry}{屁股}{7,8}
  \begin{phonetics}{屁股}{pi4gu5}
    \definition{s.}{nádega | quadris}
  \end{phonetics}
\end{entry}

\begin{entry}{屁话}{7,8}
  \begin{phonetics}{屁话}{pi4hua4}
    \definition{s.}{absurdo | tolice | besteira}
  \end{phonetics}
\end{entry}

\begin{entry}{层}{7}[Radical ⼫]
  \begin{phonetics}{层}{ceng2}[][HSK 2]
    \definition{clas.}{para andar, piso}
  \end{phonetics}
\end{entry}

\begin{entry}{层次}{7,6}
  \begin{phonetics}{层次}{ceng2ci4}
    \definition{s.}{camada | nível | graduação | arranjo de ideias}
  \end{phonetics}
\end{entry}

\begin{entry}{层层}{7,7}
  \begin{phonetics}{层层}{ceng2ceng2}
    \definition{s.}{camada sobre camada}
  \end{phonetics}
\end{entry}

\begin{entry}{希望}{7,11}
  \begin{phonetics}{希望}{xi1wang4}
    \definition[个]{s.}{desejo}
    \definition{v.}{desejar}
  \end{phonetics}
\end{entry}

\begin{entry}{床}{7}[Radical ⼴]
  \begin{phonetics}{床}{chuang2}[][HSK 1]
    \definition{clas.}{para camas}
    \definition[张]{s.}{cama}
  \end{phonetics}
\end{entry}

\begin{entry}{应对}{7,5}
  \begin{phonetics}{应对}{ying4dui4}
    \definition{v.}{responder | manusear | lidar}
  \end{phonetics}
\end{entry}

\begin{entry}{应用程序}{7,5,12,7}
  \begin{phonetics}{应用程序}{ying4yong4cheng2xu4}
    \definition{s.}{aplicativo | programa de computador}
  \end{phonetics}
\end{entry}

\begin{entry}{应用程序接口}{7,5,12,7,11,3}
  \begin{phonetics}{应用程序接口}{ying4yong4cheng2xu4jie1kou3}
    \definition{s.}{API (\emph{application programming interface})}
  \seealsoref{应用程序编程接口}{ying4yong4cheng2xu4bian1cheng2jie1kou3}
  \end{phonetics}
\end{entry}

\begin{entry}{应用程序编程接口}{7,5,12,7,12,12,11,3}
  \begin{phonetics}{应用程序编程接口}{ying4yong4cheng2xu4bian1cheng2jie1kou3}
    \definition{s.}{API (\emph{application programming interface})}
  \seealsoref{应用程序接口}{ying4yong4cheng2xu4jie1kou3}
  \end{phonetics}
\end{entry}

\begin{entry}{应该}{7,8}
  \begin{phonetics}{应该}{ying1gai1}[][HSK 2]
    \definition{v.}{dever | ter de}
  \end{phonetics}
\end{entry}

\begin{entry}{弄}{7}[Radical 廾]
  \begin{phonetics}{弄}{long4}
    \definition{s.}{beco | viela | travessa}
  \end{phonetics}
  \begin{phonetics}{弄}{nong4}
    \definition{s.}{beco | viela | travessa}
  \end{phonetics}
\end{entry}

\begin{entry}{弟}{7}[Radical 弓]
  \begin{phonetics}{弟}{di4}[][HSK 1]
    \definition{s.}{irmão mais novo | júnior}
  \end{phonetics}
\end{entry}

\begin{entry}{弟弟}{7,7}
  \begin{phonetics}{弟弟}{di4di5}[][HSK 1]
    \definition[个,位]{s.}{irmão mais novo}
  \end{phonetics}
\end{entry}

\begin{entry}{弟妹}{7,8}
  \begin{phonetics}{弟妹}{di4mei4}
    \definition{s.}{esposa do irmão mais novo}
  \end{phonetics}
\end{entry}

\begin{entry}{张}{7}[Radical 弓]
  \begin{phonetics}{张}{zhang1}
    \definition*{s.}{sobrenome Zhang}
    \definition{clas.}{para folha de papéis, mapas, etc. | para votos}
    \definition{s.}{folha de papel}
    \definition{v.}{abrir | espalhar}
  \end{phonetics}
\end{entry}

\begin{entry}{张三}{7,3}
  \begin{phonetics}{张三}{zhang1san1}
    \definition*{s.}{Zhang San | Zé Ninguém | nome para uma pessoa não especificada, 1 de 3}
  \seealsoref{李四}{li3si4}
  \seealsoref{王五}{wang2wu3}
  \end{phonetics}
\end{entry}

\begin{entry}{张狂}{7,7}
  \begin{phonetics}{张狂}{zhang1kuang2}
    \definition{adj.}{impetuoso | frenético | insolente}
  \end{phonetics}
\end{entry}

\begin{entry}{形而上学}{7,6,3,8}
  \begin{phonetics}{形而上学}{xing2'er2shang4xue2}
    \definition{s.}{metafísica}
  \end{phonetics}
\end{entry}

\begin{entry}{形容}{7,10}
  \begin{phonetics}{形容}{xing2rong2}
    \definition{s.}{descrever}
    \definition{s.}{semblante (literário) | aparência}
  \end{phonetics}
\end{entry}

\begin{entry}{形象}{7,11}
  \begin{phonetics}{形象}{xing2xiang4}
    \definition[个]{s.}{imagem | forma | figura | vializuação}
  \end{phonetics}
\end{entry}

\begin{entry}{忍耐}{7,9}
  \begin{phonetics}{忍耐}{ren3nai4}
    \definition{s.}{paciência | resistência}
    \definition{v.}{suportar | resistir | exercer paciência}
  \end{phonetics}
\end{entry}

\begin{entry}{志愿}{7,14}
  \begin{phonetics}{志愿}{zhi4yuan4}
    \definition{s.}{aspiração | ambição}
    \definition{v.}{ser voluntário}
  \end{phonetics}
\end{entry}

\begin{entry}{忘}{7}[Radical 心]
  \begin{phonetics}{忘}{wang4}[][HSK 1]
    \definition{v.}{esquecer | negligenciar | ignorar}
  \end{phonetics}
\end{entry}

\begin{entry}{忘本}{7,5}
  \begin{phonetics}{忘本}{wang4ben3}
    \definition{v.}{esquecer as próprias raízes}
  \end{phonetics}
\end{entry}

\begin{entry}{忘记}{7,5}
  \begin{phonetics}{忘记}{wang4ji4}[][HSK 1]
    \definition{v.}{esquecer}
  \end{phonetics}
\end{entry}

\begin{entry}{忘却}{7,7}
  \begin{phonetics}{忘却}{wang4que4}
    \definition{v.}{esquecer}
  \end{phonetics}
\end{entry}

\begin{entry}{忘怀}{7,7}
  \begin{phonetics}{忘怀}{wang4huai2}
    \definition{v.}{esquecer}
  \end{phonetics}
\end{entry}

\begin{entry}{忘恩}{7,10}
  \begin{phonetics}{忘恩}{wang4'en1}
    \definition{v.}{ser ingrato}
  \end{phonetics}
\end{entry}

\begin{entry}{忘掉}{7,11}
  \begin{phonetics}{忘掉}{wang4diao4}
    \definition{v.}{esquecer}
  \end{phonetics}
\end{entry}

\begin{entry}{忘餐}{7,16}
  \begin{phonetics}{忘餐}{wang4can1}
    \definition{v.}{esquecer as refeições}
  \end{phonetics}
\end{entry}

\begin{entry}{忧郁}{7,8}
  \begin{phonetics}{忧郁}{you1yu4}
    \definition{adj.}{deprimido | melancólico | desanimado}
    \definition{s.}{depressão | melancolia}
  \end{phonetics}
\end{entry}

\begin{entry}{快}{7}[Radical 心]
  \begin{phonetics}{快}{kuai4}[][HSK 1]
    \definition{adj.}{quase | rápido | depressa}
    \definition{v.}{apressar-se}
  \end{phonetics}
\end{entry}

\begin{entry}{快乐}{7,5}
  \begin{phonetics}{快乐}{kuai4le4}[][HSK 2]
    \definition{adj.}{feliz | alegre}
    \definition{s.}{felicidade | alegria}
  \end{phonetics}
\end{entry}

\begin{entry}{快点儿}{7,9,2}
  \begin{phonetics}{快点儿}{kuai4 dian3r5}[][HSK 2]
    \definition{v.}{apressar-se}
  \end{phonetics}
\end{entry}

\begin{entry}{快要}{7,9}
  \begin{phonetics}{快要}{kuai4 yao4}[][HSK 2]
    \definition{adv.}{estar prestes a | estar indo para | estar à beira de | em breve | em nenhum momento}
  \end{phonetics}
\end{entry}

\begin{entry}{快递}{7,10}
  \begin{phonetics}{快递}{kuai4di4}
    \definition{s.}{entrega expressa}
  \end{phonetics}
\end{entry}

\begin{entry}{快速}{7,10}
  \begin{phonetics}{快速}{kuai4su4}
    \definition{adj.}{veloz | de alta velocidade | rápido}
  \end{phonetics}
\end{entry}

\begin{entry}{快餐}{7,16}
  \begin{phonetics}{快餐}{kuai4 can1}[][HSK 2]
    \definition[份,顿]{s.}{comida rápida | \emph{fast food}}
  \end{phonetics}
\end{entry}

\begin{entry}{怀旧}{7,5}
  \begin{phonetics}{怀旧}{huai2jiu4}
    \definition{s.}{nostalgia}
    \definition{v.}{sentir-se nostálgico}
  \end{phonetics}
\end{entry}

\begin{entry}{我}{7}[Radical 戈]
  \begin{phonetics}{我}{wo3}[][HSK 1]
    \definition{pron.}{eu | me | mim | comigo}
  \end{phonetics}
\end{entry}

\begin{entry}{我们}{7,5}
  \begin{phonetics}{我们}{wo3men5}[][HSK 1]
    \definition{pron.}{nós | nos | conosco}
  \end{phonetics}
\end{entry}

\begin{entry}{我们的}{7,5,8}
  \begin{phonetics}{我们的}{wo3men5 de5}
    \definition{pron.}{nosso, nossos}
  \end{phonetics}
\end{entry}

\begin{entry}{我去}{7,5}
  \begin{phonetics}{我去}{wo3qu4}
    \definition{interj.}{(gíria) O que\dots!! | Oh meu Deus! | Isso é insano!}
  \end{phonetics}
\end{entry}

\begin{entry}{我的}{7,8}
  \begin{phonetics}{我的}{wo3 de5}
    \definition{pron.}{meu, meus}
  \end{phonetics}
\end{entry}

\begin{entry}{扶梯}{7,11}
  \begin{phonetics}{扶梯}{fu2ti1}
    \definition{s.}{escada rolante}
  \end{phonetics}
\end{entry}

\begin{entry}{找}{7}[Radical 手]
  \begin{phonetics}{找}{zhao3}[][HSK 1]
    \definition{v.}{andar à procura de | procurar | tentar procurar | dar troco | retornar algo}
  \end{phonetics}
\end{entry}

\begin{entry}{找见}{7,4}
  \begin{phonetics}{找见}{zhao3jian4}
    \definition{v.}{encontrar (algo que está procurando)}
  \end{phonetics}
\end{entry}

\begin{entry}{找出}{7,5}
  \begin{phonetics}{找出}{zhao3 chu1}[][HSK 2]
    \definition{v.}{encontrar | procurar}
  \end{phonetics}
\end{entry}

\begin{entry}{找回}{7,6}
  \begin{phonetics}{找回}{zhao3hui2}
    \definition{v.}{recuperar algo}
  \end{phonetics}
\end{entry}

\begin{entry}{找寻}{7,6}
  \begin{phonetics}{找寻}{zhao3xun2}
    \definition{v.}{encontrar falhas | procurar | buscar}
  \end{phonetics}
\end{entry}

\begin{entry}{找事}{7,8}
  \begin{phonetics}{找事}{zhao3shi4}
    \definition{v.}{procurar emprego | começar uma briga}
  \end{phonetics}
\end{entry}

\begin{entry}{找到}{7,8}
  \begin{phonetics}{找到}{zhao3dao4}[][HSK 1]
    \definition{v.}{encontrar}
  \end{phonetics}
\end{entry}

\begin{entry}{找钱}{7,10}
  \begin{phonetics}{找钱}{zhao3qian2}
    \definition{v.}{dar troco}
  \end{phonetics}
\end{entry}

\begin{entry}{找着}{7,11}
  \begin{phonetics}{找着}{zhao3zhao2}
    \definition{v.}{encontrar}
  \end{phonetics}
\end{entry}

\begin{entry}{找遍}{7,12}
  \begin{phonetics}{找遍}{zhao3bian4}
    \definition{v.}{pentear | pesquisar em todos os lugares}
  \end{phonetics}
\end{entry}

\begin{entry}{找零}{7,13}
  \begin{phonetics}{找零}{zhao3ling2}
    \definition{v.}{trocar dinheiro | dar troco}
  \end{phonetics}
\end{entry}

\begin{entry}{找辙}{7,16}
  \begin{phonetics}{找辙}{zhao3zhe2}
    \definition{v.}{procurar um pretexto}
  \end{phonetics}
\end{entry}

\begin{entry}{技术}{7,5}
  \begin{phonetics}{技术}{ji4shu4}[][HSK 3]
    \definition[种,门,项]{s.}{habilidade; técnica; tecnologia}
  \end{phonetics}
\end{entry}

\begin{entry}{技俩}{7,9}
  \begin{phonetics}{技俩}{ji4liang3}
    \definition{s.}{truque | estratagema | ardil | esquema | estratégia | tática}
  \end{phonetics}
\end{entry}

\begin{entry}{把}{7}[Radical 手]
  \begin{phonetics}{把}{ba3}
    \definition{clas.}{para objetos com alça | para objetos pequenos:~punhado}
    \definition{part.}{partícula tornando o substantivo seguinte um objeto direto}
    \definition{v.}{conter | alcançar | segurar}
  \end{phonetics}
  \begin{phonetics}{把}{ba4}
    \definition{v.}{lidar}
  \end{phonetics}
\end{entry}

\begin{entry}{把风}{7,4}
  \begin{phonetics}{把风}{ba3feng1}
    \definition{v.}{estar atento | vigiar (durante uma atividade clandestina)}
  \end{phonetics}
\end{entry}

\begin{entry}{把关}{7,6}
  \begin{phonetics}{把关}{ba3guan1}
    \definition{v.}{verificar estritamente | examinar cuidadosamente para ver se algo é feito de acordo com um padrão fixo | fazer a verificação final | guardar uma passagem, fronteira}
  \end{phonetics}
\end{entry}

\begin{entry}{把守}{7,6}
  \begin{phonetics}{把守}{ba3shou3}
    \definition{v.}{vigiar | guardar}
  \end{phonetics}
\end{entry}

\begin{entry}{把式}{7,6}
  \begin{phonetics}{把式}{ba3shi4}
    \definition{s.}{pessoa qualificada em um comércio}
  \end{phonetics}
\end{entry}

\begin{entry}{把戏}{7,6}
  \begin{phonetics}{把戏}{ba3xi4}
    \definition{s.}{acrobacia | malabarismo | truque barato}
  \end{phonetics}
\end{entry}

\begin{entry}{把玩}{7,8}
  \begin{phonetics}{把玩}{ba3wan2}
    \definition{v.}{brincar com | mexer com}
  \end{phonetics}
\end{entry}

\begin{entry}{把持}{7,9}
  \begin{phonetics}{把持}{ba3chi2}
    \definition{v.}{controlar | dominar | monopolizar}
  \end{phonetics}
\end{entry}

\begin{entry}{把柄}{7,9}
  \begin{phonetics}{把柄}{ba3bing3}
    \definition{s.}{(figurativo) informações que podem ser usadas contra alguém}
  \end{phonetics}
\end{entry}

\begin{entry}{把脉}{7,9}
  \begin{phonetics}{把脉}{ba3mai4}
    \definition{v.}{sentir ou tomar o pulso de alguém}
  \end{phonetics}
\end{entry}

\begin{entry}{把握}{7,12}
  \begin{phonetics}{把握}{ba3wo4}[][HSK 3]
    \definition{s.}{seguro | garantia | certeza}
    \definition{v.}{agarrar | segurar | aproveitar}
  \end{phonetics}
\end{entry}

\begin{entry}{把稳}{7,14}
  \begin{phonetics}{把稳}{ba3wen3}
    \definition{adj.}{confiável}
  \end{phonetics}
\end{entry}

\begin{entry}{投资}{7,10}
  \begin{phonetics}{投资}{tou2zi1}
    \definition{s.}{investimento}
    \definition{v.}{investir}
  \end{phonetics}
\end{entry}

\begin{entry}{投资人}{7,10,2}
  \begin{phonetics}{投资人}{tou2zi1ren2}
    \definition{s.}{investidor}
  \seealsoref{投资家}{tou2zi1jia1}
  \seealsoref{投资者}{tou2zi1zhe3}
  \end{phonetics}
\end{entry}

\begin{entry}{投资风险}{7,10,4,9}
  \begin{phonetics}{投资风险}{tou2zi1feng1xian3}
    \definition{s.}{risco de investimento}
  \end{phonetics}
\end{entry}

\begin{entry}{投资回报率}{7,10,6,7,11}
  \begin{phonetics}{投资回报率}{tou2zi1hui2bao4lv4}
    \definition{s.}{retorno sobre o investimento (ROI)}
  \end{phonetics}
\end{entry}

\begin{entry}{投资者}{7,10,8}
  \begin{phonetics}{投资者}{tou2zi1zhe3}
    \definition{s.}{investidor}
  \seealsoref{投资家}{tou2zi1jia1}
  \seealsoref{投资人}{tou2zi1ren2}
  \end{phonetics}
\end{entry}

\begin{entry}{投资家}{7,10,10}
  \begin{phonetics}{投资家}{tou2zi1jia1}
    \definition{s.}{investidor}
  \seealsoref{投资人}{tou2zi1ren2}
  \seealsoref{投资者}{tou2zi1zhe3}
  \end{phonetics}
\end{entry}

\begin{entry}{投递}{7,10}
  \begin{phonetics}{投递}{tou2di4}
    \definition{v.}{despachar | enviar}
  \end{phonetics}
\end{entry}

\begin{entry}{投票}{7,11}
  \begin{phonetics}{投票}{tou2piao4}
    \definition{v.+compl.}{votar | depositar um voto}
  \end{phonetics}
\end{entry}

\begin{entry}{折转}{7,8}
  \begin{phonetics}{折转}{zhe2zhuan3}
    \definition{s.}{reflexo (ângulo)}
    \definition{v.}{voltar atrás}
  \end{phonetics}
\end{entry}

\begin{entry}{抢掠}{7,11}
  \begin{phonetics}{抢掠}{qiang3lve4}
    \definition{s.}{saque | pilhagem}
    \definition{v.}{saquear | pilhar}
  \end{phonetics}
\end{entry}

\begin{entry}{护照}{7,13}
  \begin{phonetics}{护照}{hu4zhao4}[][HSK 2]
    \definition[本,个]{s.}{passaporte}
  \end{phonetics}
\end{entry}

\begin{entry}{报}{7}[Radical 手]
  \begin{phonetics}{报}{bao4}[][HSK 3]
    \definition[份,张]{s.}{jornal | recompensa | relatório | vingança}
    \definition{v.}{anunciar | informar}
  \end{phonetics}
\end{entry}

\begin{entry}{报名}{7,6}
  \begin{phonetics}{报名}{bao4ming2}[][HSK 2]
    \definition{v.+compl.}{matricular-se | alistar-se | inscrever-se | inserir o nome de alguém}
  \end{phonetics}
\end{entry}

\begin{entry}{报告}{7,7}
  \begin{phonetics}{报告}{bao4gao4}[][HSK 3]
    \definition[份,篇,分,个,通]{s.}{relatório | discurso | palestra | aconselhamento}
    \definition{v.}{relatar | dar a conhecer | informar}
  \end{phonetics}
\end{entry}

\begin{entry}{报纸}{7,7}
  \begin{phonetics}{报纸}{bao4zhi3}[][HSK 2]
    \definition[张]{s.}{jornal | diário}
  \end{phonetics}
\end{entry}

\begin{entry}{报到}{7,8}
  \begin{phonetics}{报到}{bao4dao4}[][HSK 3]
    \definition{v.+compl.}{apresentar-se para o serviço | fazer check-in | registrar-se | assinar}
  \end{phonetics}
\end{entry}

\begin{entry}{报道}{7,12}
  \begin{phonetics}{报道}{bao4dao4}[][HSK 3]
    \definition[个,篇,分]{s.}{história | reportagem}
    \definition{v.}{cobrir | relatar (notícias)}
  \end{phonetics}
\end{entry}

\begin{entry}{报酬}{7,13}
  \begin{phonetics}{报酬}{bao4chou5}
    \definition{s.}{recompensa | remuneração}
  \end{phonetics}
\end{entry}

\begin{entry}{改}{7}[Radical 攴]
  \begin{phonetics}{改}{gai3}[][HSK 2]
    \definition*{s.}{sobrenome Gai}
    \definition{v.}{mudar | transformar | revisar | alterar | modificar | retificar | corrigir | mudar para (fazer outra coisa)}
  \end{phonetics}
\end{entry}

\begin{entry}{改良}{7,7}
  \begin{phonetics}{改良}{gai3liang2}
    \definition{v.}{melhorar (algo) | reformar (um sistema)}
  \end{phonetics}
\end{entry}

\begin{entry}{改进}{7,7}
  \begin{phonetics}{改进}{gai4jin4}[][HSK 3]
    \definition[个]{s.}{melhoria}
    \definition{v.}{aprimorar; aperfeiçoar; melhorar; tornar melhor
modificar}
  \end{phonetics}
\end{entry}

\begin{entry}{改变}{7,8}
  \begin{phonetics}{改变}{gai3bian4}[][HSK 2]
    \definition{v.}{mudar | alterar | transformar | virar | converter | moldar | modificar}
  \end{phonetics}
\end{entry}

\begin{entry}{改造}{7,10}
  \begin{phonetics}{改造}{gai3 zao4}[][HSK 3]
    \definition{v.}{transformar; renovar | remodelar}
  \end{phonetics}
\end{entry}

\begin{entry}{改善}{7,12}
  \begin{phonetics}{改善}{gai3shan4}
    \definition{v.}{aperfeiçoar | melhorar}
  \end{phonetics}
\end{entry}

\begin{entry}{改善关系}{7,12,6,7}
  \begin{phonetics}{改善关系}{gai3shan4guan1xi5}
    \definition{v.}{melhorar a relação}
  \end{phonetics}
\end{entry}

\begin{entry}{改善通讯}{7,12,10,5}
  \begin{phonetics}{改善通讯}{gai3shan4tong1xun4}
    \definition{v.}{melhorar a comunicação}
  \end{phonetics}
\end{entry}

\begin{entry}{时光}{7,6}
  \begin{phonetics}{时光}{shi2guang1}
    \definition{s.}{tempo | época | período de tempo}
  \end{phonetics}
\end{entry}

\begin{entry}{时时}{7,7}
  \begin{phonetics}{时时}{shi2shi2}
    \definition{adv.}{muitas vezes | constantemente}
  \end{phonetics}
\end{entry}

\begin{entry}{时间}{7,7}
  \begin{phonetics}{时间}{shi2jian1}[][HSK 1]
    \definition{s.}{(conceito de, duração de, um ponto no) tempo}
  \end{phonetics}
\end{entry}

\begin{entry}{时刻}{7,8}
  \begin{phonetics}{时刻}{shi2ke4}
    \definition{adv.}{constantemente | sempre}
    \definition[个,段]{s.}{tempo | conjuntura | momento | período de tempo}
  \end{phonetics}
\end{entry}

\begin{entry}{时差}{7,9}
  \begin{phonetics}{时差}{shi2cha1}
    \definition{s.}{diferença de tempo | \emph{jet lag}}
  \end{phonetics}
\end{entry}

\begin{entry}{时候}{7,10}
  \begin{phonetics}{时候}{shi2hou5}[][HSK 1]
    \definition{adv.}{quando?}
    \definition{s.}{duração de tempo | momento | período | tempo}
  \end{phonetics}
\end{entry}

\begin{entry}{旷野}{7,11}
  \begin{phonetics}{旷野}{kuang4ye3}
    \definition{s.}{região selvagem}
  \end{phonetics}
\end{entry}

\begin{entry}{更}{7}[Radical ⽈]
  \begin{phonetics}{更}{geng1}
    \definition{s.}{vigia (por exemplo, de sentinela ou guarda)}
    \definition{v.}{alterar ou substituir | experimentar}
  \end{phonetics}
  \begin{phonetics}{更}{geng4}
    \definition{adv.}{mais | ainda mais}
  \end{phonetics}
\end{entry}

\begin{entry}{更加}{7,5}
  \begin{phonetics}{更加}{geng4 jia1}[][HSK 3]
    \definition{adv.}{mais; ainda mais; em maior grau}
  \end{phonetics}
\end{entry}

\begin{entry}{李}{7}[Radical 木]
  \begin{phonetics}{李}{li3}
    \definition*{s.}{sobrenome Li}
    \definition{s.}{ameixa}
  \end{phonetics}
\end{entry}

\begin{entry}{李子}{7,3}
  \begin{phonetics}{李子}{li3zi5}
    \definition[个]{s.}{ameixa}
  \end{phonetics}
\end{entry}

\begin{entry}{李四}{7,5}
  \begin{phonetics}{李四}{li3si4}
    \definition*{s.}{Li Si | Zé Ninguém | nome para uma pessoa não especificada, 2 de 3}
  \seealsoref{王五}{wang2wu3}
  \seealsoref{张三}{zhang1san1}
  \end{phonetics}
\end{entry}

\begin{entry}{村}{7}[Radical ⽊]
  \begin{phonetics}{村}{cun1}[][HSK 3]
    \definition{adj.}{rústico; grosseiro}
    \definition{s.}{aldeia; vila}
  \end{phonetics}
\end{entry}

\begin{entry}{杜宇}{7,6}
  \begin{phonetics}{杜宇}{du4yu3}
    \definition{s.}{cuco (pássaro)}
  \seealsoref{布谷鸟}{bu4gu3niao3}
  \seealsoref{杜鹃}{du4juan1}
  \seealsoref{杜鹃鸟}{du4juan1niao3}
  \end{phonetics}
\end{entry}

\begin{entry}{杜鹃}{7,12}
  \begin{phonetics}{杜鹃}{du4juan1}
    \definition{s.}{cuco (pássaro)}
  \seealsoref{布谷鸟}{bu4gu3niao3}
  \seealsoref{杜鹃鸟}{du4juan1niao3}
  \seealsoref{杜宇}{du4yu3}
  \end{phonetics}
\end{entry}

\begin{entry}{杜鹃鸟}{7,12,5}
  \begin{phonetics}{杜鹃鸟}{du4juan1niao3}
    \definition{s.}{cuco (pássaro)}
  \seealsoref{布谷鸟}{bu4gu3niao3}
  \seealsoref{杜鹃}{du4juan1}
  \seealsoref{杜宇}{du4yu3}
  \end{phonetics}
\end{entry}

\begin{entry}{束}{7}[Radical 木]
  \begin{phonetics}{束}{shu4}
    \definition*{s.}{sobrenome Shu}
    \definition{clas.}{para cachos, feixes, feixes de luz, etc.}
    \definition{s.}{monte | pacote | maço | feixe | cacho}
    \definition{v.}{vincular | controlar}
  \end{phonetics}
\end{entry}

\begin{entry}{束腰}{7,13}
  \begin{phonetics}{束腰}{shu4yao1}
    \definition{s.}{cinto | cinta | cinturão}
  \end{phonetics}
\end{entry}

\begin{entry}{杠}{7}[Radical ⽊]
  \begin{phonetics}{杠}{gang1}
    \definition{s.}{mastro de bandeira | poste | passarela}
  \end{phonetics}
  \begin{phonetics}{杠}{gang4}
    \definition{s.}{vara grossa | barra | linha grossa | padrão, critério | hífen, traço}
    \definition{v.}{marcar com uma linha grossa | afiar (faca, navalha, etc.)}
  \end{phonetics}
\end{entry}

\begin{entry}{条}{7}[Radical 木]
  \begin{phonetics}{条}{tiao2}[][HSK 2]
    \definition{clas.}{para coisas longas e finas (fita, rio, estrada, calças, etc.)}
    \definition{s.}{artigo | cláusula (de lei ou tratado) | item | faixa}
  \end{phonetics}
\end{entry}

\begin{entry}{条目}{7,5}
  \begin{phonetics}{条目}{tiao2mu4}
    \definition{s.}{cláusulas e subcláusulas (em documento formal) | verbete (em um dicionário, enciclopédia, etc.)}
  \end{phonetics}
\end{entry}

\begin{entry}{条件}{7,6}
  \begin{phonetics}{条件}{tiao2jian4}[][HSK 2]
    \definition[个]{s.}{circunstâncias | condição | fator | pré-requisito | qualificação | requisito}
  \end{phonetics}
\end{entry}

\begin{entry}{条例}{7,8}
  \begin{phonetics}{条例}{tiao2li4}
    \definition{s.}{código de conduta | ordenanças | regulamentos | regras | estatutos}
  \end{phonetics}
\end{entry}

\begin{entry}{条贯}{7,8}
  \begin{phonetics}{条贯}{tiao2guan4}
    \definition{s.}{ordem | procedimentos | sequência | sistema}
  \end{phonetics}
\end{entry}

\begin{entry}{条幅}{7,12}
  \begin{phonetics}{条幅}{tiao2fu2}
    \definition{s.}{faixa | banner | pergaminho de parede (para pintura ou caligrafia)}
  \end{phonetics}
\end{entry}

\begin{entry}{来}{7}[Radical 木]
  \begin{phonetics}{来}{lai2}[][HSK 1]
    \definition{v.}{vir | chegar | trazer}
  \end{phonetics}
\end{entry}

\begin{entry}{来自}{7,6}
  \begin{phonetics}{来自}{lai2zi4}[][HSK 2]
    \definition{v.}{vir de (um local) | \emph{From:} (cabeçalho de \emph{e -mail})}
  \end{phonetics}
\end{entry}

\begin{entry}{来到}{7,8}
  \begin{phonetics}{来到}{lai2dao4}[][HSK 1]
    \definition{v.}{chegar | vir}
  \end{phonetics}
\end{entry}

\begin{entry}{极}{7}[Radical 木]
  \begin{phonetics}{极}{ji2}
    \definition*{s.}{sobrenome Ji}
    \definition{s.}{o ponto mais alto | extremo | pólo}
  \end{phonetics}
\end{entry}

\begin{entry}{……极了}{7,2}
  \begin{phonetics}{……极了}{ji2le5}[][HSK 3]
    \definition{expr.}{extremamente}
  \end{phonetics}
\end{entry}

\begin{entry}{极其}{7,8}
  \begin{phonetics}{极其}{ji2qi2}
    \definition{adv.}{extremamente | muito}
  \end{phonetics}
\end{entry}

\begin{entry}{步}{7}[Radical 止]
  \begin{phonetics}{步}{bu4}[][HSK 3]
    \definition*{s.}{sobrenome Bu}
    \definition{clas.}{uma unidade antiga para medida de comprimento, equivalente a cinco chi}
    \definition{s.}{ritmo | passo | estágio | passo | condição | situação | estado}
    \definition{v.}{ir a pé | andar | pisar | contar passos}
  \end{phonetics}
\end{entry}

\begin{entry}{每}{7}[Radical 毋]
  \begin{phonetics}{每}{mei3}
    \definition{pron.}{cada}
  \end{phonetics}
\end{entry}

\begin{entry}{每个人}{7,3,2}
  \begin{phonetics}{每个人}{mei3ge5ren2}
    \definition{pron.}{todo mundo | todos}
  \end{phonetics}
\end{entry}

\begin{entry}{每天}{7,4}
  \begin{phonetics}{每天}{mei3tian1}
    \definition{adv.}{todo dia | cada dia}
  \end{phonetics}
\end{entry}

\begin{entry}{每次}{7,6}
  \begin{phonetics}{每次}{mei3ci4}
    \definition{adv.}{toda vez | cada vez}
  \end{phonetics}
\end{entry}

\begin{entry}{求}{7}[Radical 水]
  \begin{phonetics}{求}{qiu2}[][HSK 2]
    \definition*{s.}{sobrenome Qiu}
    \definition{s.}{demanda}
    \definition{v.}{pedir | implorar | solicitar | suplicar | esforçar-se por | procurar | tentar}
  \end{phonetics}
\end{entry}

\begin{entry}{汹涌}{7,10}
  \begin{phonetics}{汹涌}{xiong1yong3}
    \definition{adj.}{turbulento}
    \definition{v.}{aumentar ou emergir violentamente (oceano, rio, lago, etc.)}
  \end{phonetics}
\end{entry}

\begin{entry}{汽车}{7,4}
  \begin{phonetics}{汽车}{qi4che1}[][HSK 1]
    \definition[辆]{s.}{automóvel | carro | veículo motorizado}
  \end{phonetics}
\end{entry}

\begin{entry}{沉}{7}[Radical 水]
  \begin{phonetics}{沉}{chen2}
    \definition{adj.}{profundo}
    \definition{v.}{submergir | imergir | mergulhar | afundar}
  \end{phonetics}
\end{entry}

\begin{entry}{沉默}{7,16}
  \begin{phonetics}{沉默}{chen2mo4}
    \definition{adj.}{taciturno | não comunicativo | silencioso}
  \end{phonetics}
\end{entry}

\begin{entry}{沙}{7}[Radical 水]
  \begin{phonetics}{沙}{sha1}
    \definition*{s.}{sobrenome Sha}
    \definition[粒]{s.}{areia | cascalho | grânulo | pó}
  \end{phonetics}
\end{entry}

\begin{entry}{沙鱼}{7,8}
  \begin{phonetics}{沙鱼}{sha1yu2}
    \variantof{鲨鱼}
  \end{phonetics}
\end{entry}

\begin{entry}{沙特}{7,10}
  \begin{phonetics}{沙特}{sha1te4}
    \definition*{s.}{Saudita | abreviação de 沙特阿拉伯}
    \seeref{沙特阿拉伯}{sha1te4 a1la1bo2}
  \end{phonetics}
\end{entry}

\begin{entry}{沙特阿拉伯}{7,10,7,8,7}
  \begin{phonetics}{沙特阿拉伯}{sha1te4 a1la1bo2}
    \definition*{s.}{Arábia Saudita}
  \end{phonetics}
\end{entry}

\begin{entry}{沙漠}{7,13}
  \begin{phonetics}{沙漠}{sha1mo4}
    \definition[个]{s.}{deserto}
  \end{phonetics}
\end{entry}

\begin{entry}{没}{7}[Radical 水]
  \begin{phonetics}{没}{mei2}[][HSK 1]
    \definition{adv.}{não ter | não há | ficar sem}
    \definition{pref.}{não (prefixo negativo para verbos, traduzido para outras línguas com verbos no pretérito)}
  \end{phonetics}
  \begin{phonetics}{没}{mo4}[][HSK 0]
    \definition{adj.}{afogado}
    \definition{v.}{acabar | morrer | inundar}
    \variantof{没}
  \end{phonetics}
\end{entry}

\begin{entry}{没了}{7,2}
  \begin{phonetics}{没了}{mei2le5}
    \definition{v.}{estar morto | deixar de existir}
  \end{phonetics}
\end{entry}

\begin{entry}{没什么}{7,4,3}
  \begin{phonetics}{没什么}{mei2 shen2me5}[][HSK 1]
    \definition{expr.}{não é nada | está tudo bem | não importa | não importa}
  \end{phonetics}
\end{entry}

\begin{entry}{没用}{7,5}
  \begin{phonetics}{没用}{mei2yong4}
    \definition{adj.}{inútil}
  \end{phonetics}
\end{entry}

\begin{entry}{没关系}{7,6,7}
  \begin{phonetics}{没关系}{mei2 guan1xi5}[][HSK 1]
    \definition{v.}{não ter problema | não ter importância | não fazer mal}
    \seeref{没有关系}{mei2you3guan1xi5}
  \end{phonetics}
\end{entry}

\begin{entry}{没有}{7,6}
  \begin{phonetics}{没有}{mei2you3}[][HSK 1]
    \definition{v.}{não há | não tem | não existe}
  \end{phonetics}
\end{entry}

\begin{entry}{没有关系}{7,6,6,7}
  \begin{phonetics}{没有关系}{mei2you3guan1xi5}
    \definition{v.}{não ter problema | não ter importância | não fazer mal}
    \seeref{没关系}{mei2guan1xi5}
  \end{phonetics}
\end{entry}

\begin{entry}{没有意思}{7,6,13,9}
  \begin{phonetics}{没有意思}{mei2you3yi4si5}
    \definition{adj.}{tedioso | chato | sem interesse}
  \end{phonetics}
\end{entry}

\begin{entry}{没事儿}{7,8,2}
  \begin{phonetics}{没事儿}{mei2 shi4r5}[][HSK 1]
    \definition{expr.}{livre de trabalho | sem problemas | não é importante |não é nada |deixa para lá}
    \definition{v.}{ter tempo livre}
  \end{phonetics}
\end{entry}

\begin{entry}{灵感}{7,13}
  \begin{phonetics}{灵感}{ling2gan3}
    \definition{s.}{inspiração | explosão de criatividade em empreendimento científico ou artístico}
  \end{phonetics}
\end{entry}

\begin{entry}{灵魂}{7,13}
  \begin{phonetics}{灵魂}{ling2hun2}
    \definition{s.}{alma | espírito}
  \end{phonetics}
\end{entry}

\begin{entry}{灶台}{7,5}
  \begin{phonetics}{灶台}{zao4tai2}
    \definition{s.}{fogão}
  \end{phonetics}
\end{entry}

\begin{entry}{狂欢节}{7,6,5}
  \begin{phonetics}{狂欢节}{kuang2huan1jie2}
    \definition*{s.}{Carnaval}
  \end{phonetics}
\end{entry}

\begin{entry}{男}{7}[Radical 田]
  \begin{phonetics}{男}{nan2}[][HSK 1]
    \definition{adj.}{masculino}
    \definition{s.}{Barão, o mais baixo de cinco ordens de nobreza}
  \end{phonetics}
\end{entry}

\begin{entry}{男人}{7,2}
  \begin{phonetics}{男人}{nan2ren2}[][HSK 1]
    \definition[个]{s.}{um homem | um macho | cavalheiro | marido}
  \end{phonetics}
\end{entry}

\begin{entry}{男生}{7,5}
  \begin{phonetics}{男生}{nan2sheng1}[][HSK 1]
    \definition[个]{s.}{aluno | estudante do sexo masculino}
  \end{phonetics}
\end{entry}

\begin{entry}{男朋友}{7,8,4}
  \begin{phonetics}{男朋友}{nan2peng2you5}
    \definition{s.}{namorado}
  \end{phonetics}
\end{entry}

\begin{entry}{男孩儿}{7,9,2}
  \begin{phonetics}{男孩儿}{nan2hai2r5}[][HSK 1]
    \definition{s.}{menino | rapaz}
  \end{phonetics}
\end{entry}

\begin{entry}{私人}{7,2}
  \begin{phonetics}{私人}{si1ren2}
    \definition{adj.}{privado | interpessoal}
    \definition[些]{s.}{alguém com quem se tem um relacionamento pessoal próximo}
  \end{phonetics}
\end{entry}

\begin{entry}{私人诊所}{7,2,7,8}
  \begin{phonetics}{私人诊所}{si1ren2 zhen3suo3}
    \definition[些]{s.}{clínica privada}
  \end{phonetics}
\end{entry}

\begin{entry}{私人信件}{7,2,9,6}
  \begin{phonetics}{私人信件}{si1ren2 xin4jian4}
    \definition{s.}{carta pessoal}
  \end{phonetics}
\end{entry}

\begin{entry}{私人钥匙}{7,2,9,11}
  \begin{phonetics}{私人钥匙}{si1ren2yao4shi5}
    \definition{s.}{(criptografia) chave privada}
  \end{phonetics}
\end{entry}

\begin{entry}{私生活}{7,5,9}
  \begin{phonetics}{私生活}{si1sheng1huo2}
    \definition{s.}{vida privada}
  \end{phonetics}
\end{entry}

\begin{entry}{私自}{7,6}
  \begin{phonetics}{私自}{si1zi4}
    \definition{adj.}{privado | pessoal}
    \definition{adv.}{secretamente | sem aprovação explícita}
  \end{phonetics}
\end{entry}

\begin{entry}{究竟}{7,11}
  \begin{phonetics}{究竟}{jiu1jing4}
    \definition{adv.}{afinal | no final | no final das contas | na verdade | exatamente | são ou não são}
  \end{phonetics}
\end{entry}

\begin{entry}{系}{7}[Radical 糸]
  \begin{phonetics}{系}{xi4}
    \definition{s.}{faculdade (da universidade) | departamento}
    \definition{v.}{prender | vincular | conectar | relacionar com | amarrar | se preocupar}
  \end{phonetics}
\end{entry}

\begin{entry}{系囚}{7,5}
  \begin{phonetics}{系囚}{xi4qiu2}
    \definition{s.}{prisioneiro}
  \end{phonetics}
\end{entry}

\begin{entry}{系列}{7,6}
  \begin{phonetics}{系列}{xi4lie4}
    \definition{s.}{série | conjunto}
  \end{phonetics}
\end{entry}

\begin{entry}{系统}{7,9}
  \begin{phonetics}{系统}{xi4tong3}
    \definition[个]{s.}{sistema}
  \end{phonetics}
\end{entry}

\begin{entry}{纯真}{7,10}
  \begin{phonetics}{纯真}{chun2zhen1}
    \definition{adj.}{inocente e não afetado | puro e não adulterado}
    \definition{s.}{inocência}
  \end{phonetics}
\end{entry}

\begin{entry}{纸}{7}[Radical 糸]
  \begin{phonetics}{纸}{zhi3}[][HSK 2]
    \definition{clas.}{para documentos, cartas, etc.}
    \definition[张,沓]{s.}{papel}
  \end{phonetics}
\end{entry}

\begin{entry}{纸巾}{7,3}
  \begin{phonetics}{纸巾}{zhi3jin1}
    \definition[张,包]{s.}{lenço | guardanapo | papel toalha}
  \end{phonetics}
\end{entry}

\begin{entry}{纸币}{7,4}
  \begin{phonetics}{纸币}{zhi3bi4}
    \definition[张]{s.}{nota (dinheiro) | cédula}
  \end{phonetics}
\end{entry}

\begin{entry}{纸尿裤}{7,7,12}
  \begin{phonetics}{纸尿裤}{zhi3niao4ku4}
    \definition{s.}{fralda descartável}
  \end{phonetics}
\end{entry}

\begin{entry}{纸张}{7,7}
  \begin{phonetics}{纸张}{zhi3zhang1}
    \definition{s.}{papel}
  \end{phonetics}
\end{entry}

\begin{entry}{纸烟}{7,10}
  \begin{phonetics}{纸烟}{zhi3yan1}
    \definition{s.}{cigarro}
  \end{phonetics}
\end{entry}

\begin{entry}{纹路}{7,13}
  \begin{phonetics}{纹路}{wen2lu4}
    \definition{s.}{padrão de linhas | rugas | veias | veias (em mármore ou impressão digital) | grãos (em madeira, etc.)}
  \end{phonetics}
\end{entry}

\begin{entry}{肚}{7}[Radical 肉]
  \begin{phonetics}{肚}{du3}
    \definition{s.}{tripas | entranhas}
  \end{phonetics}
  \begin{phonetics}{肚}{du4}
    \definition{s.}{barriga}
  \end{phonetics}
\end{entry}

\begin{entry}{肚子}{7,3}
  \begin{phonetics}{肚子}{du4zi5}
    \definition[个]{s.}{abdómen | barriga}
  \end{phonetics}
\end{entry}

\begin{entry}{良心}{7,4}
  \begin{phonetics}{良心}{liang2xin1}
    \definition{s.}{consciência}
  \end{phonetics}
\end{entry}

\begin{entry}{良田}{7,5}
  \begin{phonetics}{良田}{liang2tian2}
    \definition{s.}{terra agrícola boa | terra fértil}
  \end{phonetics}
\end{entry}

\begin{entry}{芥}{7}[Radical 艸]
  \begin{phonetics}{芥}{gai4}
    \definition{s.}{usado em 芥蓝 \dpy{gai4lan2}}
    \seeref{芥蓝}{gai4lan2}
  \end{phonetics}
  \begin{phonetics}{芥}{jie4}
    \definition{s.}{mostarda}
  \end{phonetics}
\end{entry}

\begin{entry}{芥兰}{7,5}
  \begin{phonetics}{芥兰}{gai4lan2}
    \variantof{芥蓝}
  \end{phonetics}
  \begin{phonetics}{芥兰}{jie4lan2}
    \definition{s.}{couve}
  \end{phonetics}
\end{entry}

\begin{entry}{芥蓝}{7,13}
  \begin{phonetics}{芥蓝}{gai4lan2}
    \definition{s.}{brócolis chinês | couve chinesa | mostarda}
    \seeref{格兰菜}{ge2lan2cai4}
  \end{phonetics}
\end{entry}

\begin{entry}{芦笋}{7,10}
  \begin{phonetics}{芦笋}{lu2sun3}
    \definition{s.}{aspargos}
  \end{phonetics}
\end{entry}

\begin{entry}{芯片}{7,4}
  \begin{phonetics}{芯片}{xin1pian4}
    \definition{s.}{chip de computador | microchip}
  \end{phonetics}
\end{entry}

\begin{entry}{花}{7}[Radical 艸]
  \begin{phonetics}{花}{hua1}[][HSK 1]
    \definition*{s.}{sobrenome Hua}
    \definition[朵,支,束,把,盆,簇]{s.}{flor}
    \definition{v.}{gastar}
  \end{phonetics}
\end{entry}

\begin{entry}{花儿}{7,2}
  \begin{phonetics}{花儿}{hua1r5}
    \definition[朵,支,束,把,盆,簇]{s.}{flor}
  \end{phonetics}
\end{entry}

\begin{entry}{花生}{7,5}
  \begin{phonetics}{花生}{hua1sheng1}
    \definition[粒]{s.}{amendoim}
  \end{phonetics}
\end{entry}

\begin{entry}{花园}{7,7}
  \begin{phonetics}{花园}{hua1 yuan2}[][HSK 2]
    \definition[个,座]{s.}{jardim}
  \end{phonetics}
\end{entry}

\begin{entry}{花店}{7,8}
  \begin{phonetics}{花店}{hua1dian4}
    \definition{s.}{floricultura}
  \end{phonetics}
\end{entry}

\begin{entry}{花茶}{7,9}
  \begin{phonetics}{花茶}{hua1cha2}
    \definition[杯,壶]{s.}{chá perfumado}
  \end{phonetics}
\end{entry}

\begin{entry}{花样游泳}{7,10,12,8}
  \begin{phonetics}{花样游泳}{hua1yang4you2yong3}
    \definition{s.}{nado sincronizado}
  \end{phonetics}
\end{entry}

\begin{entry}{花椰菜}{7,12,11}
  \begin{phonetics}{花椰菜}{hua1ye1cai4}
    \definition{s.}{couve-flor}
  \end{phonetics}
\end{entry}

\begin{entry}{芹菜}{7,11}
  \begin{phonetics}{芹菜}{qin2cai4}
    \definition{s.}{salsão}
  \end{phonetics}
\end{entry}

\begin{entry}{苏格兰}{7,10,5}
  \begin{phonetics}{苏格兰}{su1ge2lan2}
    \definition*{s.}{Escócia}
  \end{phonetics}
\end{entry}

\begin{entry}{补}{7}[Radical 衣]
  \begin{phonetics}{补}{bu3}[][HSK 3]
    \definition*{s.}{sobrenome Bu}
    \definition{s.}{benefício | ajuda | uso}
    \definition{v.}{consertar | remendar | preencher | adicionar suplemento | suprir | compensar |nutrir}
  \end{phonetics}
\end{entry}

\begin{entry}{补充}{7,6}
  \begin{phonetics}{补充}{bu3chong1}[][HSK 3]
    \definition{adj.}{adicional | suplementar}
    \definition[个]{s.}{aditivo | suplemento}
    \definition{v.}{reabastecer | suplementar | complementar}
  \end{phonetics}
\end{entry}

\begin{entry}{角}{7}[Radical 角]
  \begin{phonetics}{角}{jiao3}
    \definition{clas.}{1 jiao = 10 centavos}
    \definition[个]{s.}{ângulo | esquina | chifre | em forma de chifre}
  \end{phonetics}
  \begin{phonetics}{角}{jue2}
    \definition*{s.}{sobrenome Jue}
    \definition{s.}{papel (teatro)}
    \definition{v.}{competir}
  \end{phonetics}
\end{entry}

\begin{entry}{角度}{7,9}
  \begin{phonetics}{角度}{jiao3du4}[][HSK 2]
    \definition{s.}{ângulo | ponto de vista}
  \end{phonetics}
\end{entry}

\begin{entry}{言论}{7,6}
  \begin{phonetics}{言论}{yan2lun4}
    \definition{s.}{expressão de opinião |  visualizações | comentários | argumentos}
  \end{phonetics}
\end{entry}

\begin{entry}{证件}{7,6}
  \begin{phonetics}{证件}{zheng4jian4}
    \definition{s.}{documento de identificação | credencial | certificado | comprovante}
  \end{phonetics}
\end{entry}

\begin{entry}{证实}{7,8}
  \begin{phonetics}{证实}{zheng4shi2}
    \definition{v.}{confirmar (algo como verdadeiro) | verificar}
  \end{phonetics}
\end{entry}

\begin{entry}{证据}{7,11}
  \begin{phonetics}{证据}{zheng4ju4}
    \definition{s.}{evidência | prova | testemunho}
  \end{phonetics}
\end{entry}

\begin{entry}{评论}{7,6}
  \begin{phonetics}{评论}{ping2lun4}
    \definition[篇]{s.}{comentário}
    \definition{v.}{comentar | discutir}
  \end{phonetics}
\end{entry}

\begin{entry}{诅咒}{7,8}
  \begin{phonetics}{诅咒}{zu3zhou4}
    \definition{v.}{amaldiçoar}
  \end{phonetics}
\end{entry}

\begin{entry}{词}{7}[Radical 言]
  \begin{phonetics}{词}{ci2}[][HSK 2]
    \definition[个,组]{s.}{discurso | declaração | linhas de jogo | um tipo de poesia clássica chinesa, originária da Dinastia Tang e totalmente desenvolvida na Dinastia Song | palavra  | termo}
  \end{phonetics}
\end{entry}

\begin{entry}{词典}{7,8}
  \begin{phonetics}{词典}{ci2dian3}[][HSK 2]
    \definition[部,本]{s.}{dicionário}
  \seealsoref{字典}{zi4dian3}
  \end{phonetics}
\end{entry}

\begin{entry}{词语}{7,9}
  \begin{phonetics}{词语}{ci2yu3}[][HSK 2]
    \definition{s.}{palavra (termo geral, incluindo desdemonossilábicas até frases curtas) | termo (por exemplo, termo técnico) | expressão}
  \end{phonetics}
\end{entry}

\begin{entry}{豆角}{7,7}
  \begin{phonetics}{豆角}{dou4jiao3}
    \definition{s.}{feijão verde}
  \end{phonetics}
\end{entry}

\begin{entry}{豆荚}{7,9}
  \begin{phonetics}{豆荚}{dou4jia2}
    \definition{s.}{vagem (de legumes)}
  \end{phonetics}
\end{entry}

\begin{entry}{走}{7}[Radical 走][Kangxi 156]
  \begin{phonetics}{走}{zou3}[][HSK 1]
    \definition{v.}{andar | caminhar}
  \end{phonetics}
\end{entry}

\begin{entry}{走开}{7,4}
  \begin{phonetics}{走开}{zou3 kai1}[][HSK 2]
    \definition{v.}{ir embora | fugir | ir para outro lugar}
  \end{phonetics}
\end{entry}

\begin{entry}{走去}{7,5}
  \begin{phonetics}{走去}{zou3qu4}
    \definition{v.}{caminhar até (para)}
  \end{phonetics}
\end{entry}

\begin{entry}{走过}{7,6}
  \begin{phonetics}{走过}{zou3 guo4}[][HSK 2]
    \definition{v.}{passar}
  \end{phonetics}
\end{entry}

\begin{entry}{走秀}{7,7}
  \begin{phonetics}{走秀}{zou3xiu4}
    \definition{s.}{desfile de moda}
    \definition{v.}{andar na passarela (em um desfile de moda)}
  \end{phonetics}
\end{entry}

\begin{entry}{走进}{7,7}
  \begin{phonetics}{走进}{zou3 jin4}[][HSK 2]
    \definition{v.}{entrar}
  \end{phonetics}
\end{entry}

\begin{entry}{走势}{7,8}
  \begin{phonetics}{走势}{zou3shi4}
    \definition{s.}{caminho | tendência}
  \end{phonetics}
\end{entry}

\begin{entry}{走卒}{7,8}
  \begin{phonetics}{走卒}{zou3zu2}
    \definition{s.}{lacaio (masculino) | peão (isto é, soldado de infantaria) | servo}
  \end{phonetics}
\end{entry}

\begin{entry}{走鬼}{7,9}
  \begin{phonetics}{走鬼}{zou3gui3}
    \definition{s.}{vendedor ambulante sem licença}
  \end{phonetics}
\end{entry}

\begin{entry}{走索}{7,10}
  \begin{phonetics}{走索}{zou3suo3}
    \definition{v.}{andar na corda bamba}
    \seeref{走绳}{zou3sheng2}
  \end{phonetics}
\end{entry}

\begin{entry}{走绳}{7,11}
  \begin{phonetics}{走绳}{zou3sheng2}
    \definition{v.}{andar na corda bamba}
    \seeref{走索}{zou3suo3}
  \end{phonetics}
\end{entry}

\begin{entry}{走路}{7,13}
  \begin{phonetics}{走路}{zou3lu4}[][HSK 1]
    \definition{v.}{andar | ir a pé | sair | ir embora}
  \end{phonetics}
\end{entry}

\begin{entry}{足}{7}[Radical 足][Kangxi 157]
  \begin{phonetics}{足}{ju4}
    \definition{adj.}{excessivo}
  \end{phonetics}
  \begin{phonetics}{足}{zu2}
    \definition{adj.}{amplo}
    \definition{s.}{pé}
    \definition{v.}{ser suficiente}
  \end{phonetics}
\end{entry}

\begin{entry}{足月}{7,4}
  \begin{phonetics}{足月}{zu2yue4}
    \definition{s.}{gestação completa}
  \end{phonetics}
\end{entry}

\begin{entry}{足足}{7,7}
  \begin{phonetics}{足足}{zu2zu2}
    \definition{adv.}{tanto quanto | extremamente | completamente | não menos que}
  \end{phonetics}
\end{entry}

\begin{entry}{足球}{7,11}
  \begin{phonetics}{足球}{zu2qiu2}
    \definition[个]{s.}{futebol | bola de futebol}
  \end{phonetics}
\end{entry}

\begin{entry}{足球队}{7,11,4}
  \begin{phonetics}{足球队}{zu2qiu2dui4}
    \definition{s.}{time de futebol}
  \end{phonetics}
\end{entry}

\begin{entry}{足球协会}{7,11,6,6}
  \begin{phonetics}{足球协会}{zu2qiu2xie2hui4}
    \definition*{s.}{Associação de Futebol}
  \end{phonetics}
\end{entry}

\begin{entry}{足球场}{7,11,6}
  \begin{phonetics}{足球场}{zu2qiu2chang3}
    \definition{s.}{campo de futebol}
  \end{phonetics}
\end{entry}

\begin{entry}{足球迷}{7,11,9}
  \begin{phonetics}{足球迷}{zu2qiu2mi2}
    \definition{s.}{fã de futebol}
  \end{phonetics}
\end{entry}

\begin{entry}{足球赛}{7,11,14}
  \begin{phonetics}{足球赛}{zu2qiu2sai4}
    \definition{s.}{competição de futebol | partida de futebol}
  \end{phonetics}
\end{entry}

\begin{entry}{身上}{7,3}
  \begin{phonetics}{身上}{shen1shang5}[][HSK 1]
    \definition{adv.}{no corpo de alguém | em um | com um}
  \end{phonetics}
\end{entry}

\begin{entry}{身亡}{7,3}
  \begin{phonetics}{身亡}{shen1wang2}
    \definition{v.}{morrer}
  \end{phonetics}
\end{entry}

\begin{entry}{身边}{7,5}
  \begin{phonetics}{身边}{shen1 bian1}[][HSK 2]
    \definition{adv.}{ao redor | ao lado de alguém | em mãos}
  \end{phonetics}
\end{entry}

\begin{entry}{身体}{7,7}
  \begin{phonetics}{身体}{shen1ti3}[][HSK 1]
    \definition[具,个]{s.}{em pessoa | saúde de alguém | o corpo}
  \end{phonetics}
\end{entry}

\begin{entry}{身体乳}{7,7,8}
  \begin{phonetics}{身体乳}{shen1ti3 ru3}
    \definition{s.}{loção corporal}
  \end{phonetics}
\end{entry}

\begin{entry}{身体能力}{7,7,10,2}
  \begin{phonetics}{身体能力}{shen1ti3 neng2li4}
    \definition{s.}{habilidade física}
  \end{phonetics}
\end{entry}

\begin{entry}{辛苦}{7,8}
  \begin{phonetics}{辛苦}{xin1ku3}
    \definition{adj.}{exaustivo | duro | árduo}
    \definition{s.}{dificuldades}
    \definition{v.}{trabalhar duro | ter muitos problemas}
  \end{phonetics}
\end{entry}

\begin{entry}{运气}{7,4}
  \begin{phonetics}{运气}{yun4qi5}
    \definition{s.}{sorte (boa ou má)}
  \end{phonetics}
\end{entry}

\begin{entry}{运动}{7,6}
  \begin{phonetics}{运动}{yun4dong4}[][HSK 2]
    \definition[场]{s.}{esporte | desporto}
    \definition{v.}{exercitar | mover-se}
  \end{phonetics}
\end{entry}

\begin{entry}{运动会}{7,6,6}
  \begin{phonetics}{运动会}{yun4dong4hui4}
    \definition[个]{s.}{competição esportiva}
  \end{phonetics}
\end{entry}

\begin{entry}{运动场}{7,6,6}
  \begin{phonetics}{运动场}{yun4dong4chang3}
    \definition{s.}{campo desportivo | campo de jogos}
  \end{phonetics}
\end{entry}

\begin{entry}{运动员}{7,6,7}
  \begin{phonetics}{运动员}{yun4dong4yuan2}
    \definition[名,个]{s.}{jogador | atleta}
  \end{phonetics}
\end{entry}

\begin{entry}{运动学}{7,6,8}
  \begin{phonetics}{运动学}{yun4dong4xue2}
    \definition{s.}{cinemática}
  \end{phonetics}
\end{entry}

\begin{entry}{运动服}{7,6,8}
  \begin{phonetics}{运动服}{yun4dong4fu2}
    \definition{s.}{roupa para prática de esporte}
  \end{phonetics}
\end{entry}

\begin{entry}{运动衫}{7,6,8}
  \begin{phonetics}{运动衫}{yun4dong4shan1}
    \definition[件]{s.}{moletom | camisa esportiva}
  \end{phonetics}
\end{entry}

\begin{entry}{运动家}{7,6,10}
  \begin{phonetics}{运动家}{yun4dong4jia1}
    \definition{s.}{ativista | atleta | esportista}
  \end{phonetics}
\end{entry}

\begin{entry}{运动病}{7,6,10}
  \begin{phonetics}{运动病}{yun4dong4bing4}
    \definition{s.}{enjôo (movimento, carro, etc.)}
  \end{phonetics}
\end{entry}

\begin{entry}{运动鞋}{7,6,15}
  \begin{phonetics}{运动鞋}{yun4dong4xie2}
    \definition{s.}{tênis | sapatos esportivos}
  \end{phonetics}
\end{entry}

\begin{entry}{运行}{7,6}
  \begin{phonetics}{运行}{yun4xing2}
    \definition{v.}{(corpos celestes, etc.) mover-se ao longo do curso | (figurativo) funcionar, estar em operação | (serviço de trem, etc.) operar | (computador) executar um programa}
  \end{phonetics}
\end{entry}

\begin{entry}{运河}{7,8}
  \begin{phonetics}{运河}{yun4he2}
    \definition{s.}{canal (em um rio)}
  \end{phonetics}
\end{entry}

\begin{entry}{近}{7}[Radical 辵]
  \begin{phonetics}{近}{jin4}[][HSK 2]
    \definition{adj.}{perto | próximo}
  \end{phonetics}
\end{entry}

\begin{entry}{还}{7}[Radical 辵]
  \begin{phonetics}{还}{hai2}[][HSK 1]
    \definition{adv.}{ainda | também | ainda mais | razoavelmente | bastante}
  \end{phonetics}
  \begin{phonetics}{还}{huan2}[][HSK 1]
    \definition*{s.}{sobrenome Huan}
    \definition{v.}{devolver | restituir | pagar de volta}
  \end{phonetics}
\end{entry}

\begin{entry}{还有}{7,6}
  \begin{phonetics}{还有}{hai2 you3}[][HSK 1]
    \definition{adv.}{além do mais | além disso | ainda permanece | ainda há}
  \end{phonetics}
\end{entry}

\begin{entry}{还是}{7,9}
  \begin{phonetics}{还是}{hai2shi5}[][HSK 1]
    \definition{adv.}{ainda (como antes) | inesperadamente | teve melhor}
    \definition{conj.}{ou (somente para frases interrogativas)}
  \end{phonetics}
\end{entry}

\begin{entry}{这}{7}[Radical 辵]
  \begin{phonetics}{这}{zhe4}
    \definition{pron.}{este, isto}
  \end{phonetics}
  \begin{phonetics}{这}{zhei4}
    \definition{pron.}{(coloquial) este}
  \end{phonetics}
\end{entry}

\begin{entry}{这儿}{7,2}
  \begin{phonetics}{这儿}{zhe4r5}[][HSK 1]
    \definition{pron.}{aqui}
  \end{phonetics}
\end{entry}

\begin{entry}{这么}{7,3}
  \begin{phonetics}{这么}{zhe4 me5}[][HSK 2]
    \definition{adv.}{como este | desta maneira}
  \end{phonetics}
\end{entry}

\begin{entry}{这末}{7,5}
  \begin{phonetics}{这末}{zhe4me5}
    \variantof{这么}
  \end{phonetics}
\end{entry}

\begin{entry}{这边}{7,5}
  \begin{phonetics}{这边}{zhe4bian5}[][HSK 1]
    \definition{pron.}{aqui | este lado}
  \end{phonetics}
\end{entry}

\begin{entry}{这时}{7,7}
  \begin{phonetics}{这时}{zhe4 shi2}[][HSK 2]
    \definition{adv.}{neste momento}
  \end{phonetics}
\end{entry}

\begin{entry}{这时候}{7,7,10}
  \begin{phonetics}{这时候}{zhe4 shi2 hou5}
    \definition{adv.}{neste momento}
  \end{phonetics}
\end{entry}

\begin{entry}{这里}{7,7}
  \begin{phonetics}{这里}{zhe4li3}[][HSK 1]
    \definition{pron.}{aqui}
  \end{phonetics}
\end{entry}

\begin{entry}{这些}{7,8}
  \begin{phonetics}{这些}{zhe4xie1}[][HSK 1]
    \definition{pron.}{estes}
  \end{phonetics}
\end{entry}

\begin{entry}{这样}{7,10}
  \begin{phonetics}{这样}{zhe4 yang4}[][HSK 2]
    \definition{adv.}{assim | dessa maneira | deste modo}
  \end{phonetics}
\end{entry}

\begin{entry}{这麽}{7,14}
  \begin{phonetics}{这麽}{zhe4me5}
    \variantof{这么}
  \end{phonetics}
\end{entry}

\begin{entry}{进}{7}[Radical 辵]
  \begin{phonetics}{进}{jin4}[][HSK 1]
    \definition{clas.}{para seções em um edifício ou complexo residencial}
    \definition{s.}{(matemática) base de um sistema numérico}
    \definition{v.}{entrar}
  \end{phonetics}
\end{entry}

\begin{entry}{进入}{7,2}
  \begin{phonetics}{进入}{jin4 ru4}[][HSK 2]
    \definition{v.}{entrar | juntar-se}
  \end{phonetics}
\end{entry}

\begin{entry}{进口}{7,3}
  \begin{phonetics}{进口}{jin4kou3}
    \definition{adj.}{importado}
    \definition{s.}{importação | entrada | entrada (para entrada de ar, água, etc.)}
    \definition{v.}{importar}
  \end{phonetics}
\end{entry}

\begin{entry}{进出口}{7,5,3}
  \begin{phonetics}{进出口}{jin4chu1kou3}
    \definition{s.}{importação e exportação}
    \definition{v.}{importar e exportar}
  \end{phonetics}
\end{entry}

\begin{entry}{进去}{7,5}
  \begin{phonetics}{进去}{jin4 qu4}[][HSK 1]
    \definition{v.}{entrar (a partir da minha localização)}
  \end{phonetics}
\end{entry}

\begin{entry}{进行}{7,6}
  \begin{phonetics}{进行}{jin4xing2}[][HSK 2]
    \definition{v.}{continuar | estar em andamento | fazer | conduzir | continuar | executar | marchar | avançar | prosseguir | estar em marcha}
  \end{phonetics}
\end{entry}

\begin{entry}{进行编程}{7,6,12,12}
  \begin{phonetics}{进行编程}{jin4xing2bian1cheng2}
    \definition{s.}{programa de computador executável}
  \end{phonetics}
\end{entry}

\begin{entry}{进来}{7,7}
  \begin{phonetics}{进来}{jin4 lai2}[][HSK 1]
    \definition{v.}{entrar (para a minha localização)}
  \end{phonetics}
\end{entry}

\begin{entry}{进步}{7,7}
  \begin{phonetics}{进步}{jin4bu4}
    \definition[个]{s.}{progresso | melhora}
    \definition{v.}{progredir | melhorar}
  \end{phonetics}
\end{entry}

\begin{entry}{远}{7}[Radical 辵]
  \begin{phonetics}{远}{yuan3}[][HSK 1]
    \definition{adj.}{longe | distante | remoto}
  \end{phonetics}
  \begin{phonetics}{远}{yuan4}[][HSK 0]
    \definition{v.}{distanciar-se de (clássico)}
  \end{phonetics}
\end{entry}

\begin{entry}{远天}{7,4}
  \begin{phonetics}{远天}{yuan3tian1}
    \definition{s.}{paraíso | o céu distante}
  \end{phonetics}
\end{entry}

\begin{entry}{远方}{7,4}
  \begin{phonetics}{远方}{yuan3fang1}
    \definition{s.}{longe | um local distante}
  \end{phonetics}
\end{entry}

\begin{entry}{远远}{7,7}
  \begin{phonetics}{远远}{yuan3yuan3}
    \definition{adv.}{de longe}
  \end{phonetics}
\end{entry}

\begin{entry}{远征}{7,8}
  \begin{phonetics}{远征}{yuan3zheng1}
    \definition{s.}{uma expedição militar | marcha para regiões remotas}
  \end{phonetics}
\end{entry}

\begin{entry}{违规}{7,8}
  \begin{phonetics}{违规}{wei2gui1}
    \definition{v.}{violar as regras}
  \end{phonetics}
\end{entry}

\begin{entry}{违宪}{7,9}
  \begin{phonetics}{违宪}{wei2xian4}
    \definition{adj.}{inconstitucional}
  \end{phonetics}
\end{entry}

\begin{entry}{连锁反应}{7,12,4,7}
  \begin{phonetics}{连锁反应}{lian2suo3fan3ying4}
    \definition{s.}{reação em cadeia}
  \end{phonetics}
\end{entry}

\begin{entry}{迟到}{7,8}
  \begin{phonetics}{迟到}{chi2dao4}
    \definition{v.}{chegar atrasado | tardar}
  \end{phonetics}
\end{entry}

\begin{entry}{邮包}{7,5}
  \begin{phonetics}{邮包}{you2bao1}
    \definition{s.}{encomenda postal}
  \end{phonetics}
\end{entry}

\begin{entry}{邮市}{7,5}
  \begin{phonetics}{邮市}{you2shi4}
    \definition{s.}{mercado postal}
  \end{phonetics}
\end{entry}

\begin{entry}{邮电}{7,5}
  \begin{phonetics}{邮电}{you2dian4}
    \definition*{s.}{Correios e Telecomunicações}
  \end{phonetics}
\end{entry}

\begin{entry}{邮件}{7,6}
  \begin{phonetics}{邮件}{you2jian4}
    \definition{s.}{correspondência | \emph{e-mail}}
  \end{phonetics}
\end{entry}

\begin{entry}{邮局}{7,7}
  \begin{phonetics}{邮局}{you2ju2}
    \definition[家,个]{s.}{correio | agência dos correios}
  \end{phonetics}
\end{entry}

\begin{entry}{邮费}{7,9}
  \begin{phonetics}{邮费}{you2fei4}
    \definition{s.}{postagem}
    \definition{v.}{postar}
  \end{phonetics}
\end{entry}

\begin{entry}{邮迷}{7,9}
  \begin{phonetics}{邮迷}{you2mi2}
    \definition{s.}{filatelista | colecionador de selos}
  \end{phonetics}
\end{entry}

\begin{entry}{邮资}{7,10}
  \begin{phonetics}{邮资}{you2zi1}
    \definition{s.}{postagem}
  \end{phonetics}
\end{entry}

\begin{entry}{邮递}{7,10}
  \begin{phonetics}{邮递}{you2di4}
    \definition{v.}{enviar por correio}
  \end{phonetics}
\end{entry}

\begin{entry}{邮票}{7,11}
  \begin{phonetics}{邮票}{you2piao4}
    \definition[枚,张]{s.}{selo postal}
  \end{phonetics}
\end{entry}

\begin{entry}{邻居}{7,8}
  \begin{phonetics}{邻居}{lin2ju1}
    \definition[个]{s.}{vizinho}
  \end{phonetics}
\end{entry}

\begin{entry}{里}{7}[Radical 里][Kangxi 166]
  \begin{phonetics}{里}{li3}[][HSK 1]
    \definition*{s.}{sobrenome Li}
    \definition{adv.}{em | dentro | interior | interno}
    \definition{s.}{vizinhança | bairro | li, medida antiga de comprimento, aproximadamente 500m | unidade administrativa antiga de 25 famílias}
  \end{phonetics}
\end{entry}

\begin{entry}{里头}{7,5}
  \begin{phonetics}{里头}{li3 tou5}[][HSK 2]
    \definition{s.}{dentro}
  \end{phonetics}
\end{entry}

\begin{entry}{里边}{7,5}
  \begin{phonetics}{里边}{li3bian5}[][HSK 1]
    \definition{prep.}{em | dentro}
  \end{phonetics}
\end{entry}

\begin{entry}{里斯本}{7,12,5}
  \begin{phonetics}{里斯本}{li3si1ben3}
    \definition*{s.}{Lisboa}
  \end{phonetics}
\end{entry}

\begin{entry}{里斯本大学}{7,12,5,3,8}
  \begin{phonetics}{里斯本大学}{li3si1ben3 da4xue2}
    \definition*{s.}{Universidade de Lisboa}
  \end{phonetics}
\end{entry}

\begin{entry}{间}{7}[Radical 門]
  \begin{phonetics}{间}{jian1}
    \definition{adv.}{entre | dentro de um tempo ou espaço definidos}
    \definition{clas.}{para salas}
    \definition{s.}{sala | seção de uma sala ou espaço lateral entre dois pares de pilares}
  \end{phonetics}
  \begin{phonetics}{间}{jian4}
    \definition{s.}{lacuna}
    \definition{v.}{separar | podar (mudas) | semear descontentamento}
  \end{phonetics}
\end{entry}

\begin{entry}{间或}{7,8}
  \begin{phonetics}{间或}{jian4huo4}
    \definition{adv.}{às vezes | ocasionalmente | de vez em quando}
  \end{phonetics}
\end{entry}

\begin{entry}{间接}{7,11}
  \begin{phonetics}{间接}{jian4jie1}
    \definition{adj.}{indireto}
  \seealsoref{直接}{zhi2jie1}
  \end{phonetics}
\end{entry}

\begin{entry}{闷热}{7,10}
  \begin{phonetics}{闷热}{men1re4}
    \definition{adj.}{abafado | quente e abafado | sufocantemente quente | quente e sensual}
  \end{phonetics}
\end{entry}

\begin{entry}{阻击}{7,5}
  \begin{phonetics}{阻击}{zu3ji1}
    \definition{v.}{verificar | parar}
  \end{phonetics}
\end{entry}

\begin{entry}{阿}{7}[Radical 阜]
  \begin{phonetics}{阿}{a1}
    \definition{pref.}{utilizado para indicar familiaridade antes de nomes monossilábicos, termos de parentesco, etc.}
  \end{phonetics}
  \begin{phonetics}{阿}{e1}
    \definition{adj.}{gracioso}
    \definition{pron.}{monte grande | canto, esquina}
    \definition{v.}{jogar | agradar | atender | ser injustamente parcial com | ser dobrado}
  \end{phonetics}
\end{entry}

\begin{entry}{阿姨}{7,9}
  \begin{phonetics}{阿姨}{a1yi2}
    \definition[个]{s.}{tia materna | madrasta | cuidadora de crianças | babá | mulher da mesma idade dos pais (termo de tratamento usado pela criança)}
  \end{phonetics}
\end{entry}

\begin{entry}{阿哥}{7,10}
  \begin{phonetics}{阿哥}{a1ge1}
    \definition{s.}{irmão mais velho (familiar)}
  \end{phonetics}
\end{entry}

\begin{entry}{附近}{7,7}
  \begin{phonetics}{附近}{fu4jin4}
    \definition{adv.}{aqui perto | perto daqui}
  \end{phonetics}
\end{entry}

\begin{entry}{陆路}{7,13}
  \begin{phonetics}{陆路}{lu4lu4}
    \definition{s.}{rota terrestre}
  \end{phonetics}
\end{entry}

\begin{entry}{饭}{7}[Radical 食]
  \begin{phonetics}{饭}{fan4}[][HSK 1]
    \definition[碗]{s.}{arroz cozido}
    \definition[顿]{s.}{refeição}
    \definition{s.}{(empréstimo linguístico) fã, devoto}
  \end{phonetics}
\end{entry}

\begin{entry}{饭店}{7,8}
  \begin{phonetics}{饭店}{fan4dian4}[][HSK 1]
    \definition[家,个]{s.}{restaurante | hotel}
  \end{phonetics}
\end{entry}

\begin{entry}{饭馆}{7,11}
  \begin{phonetics}{饭馆}{fan4 guan3}[][HSK 2]
    \definition[家,个]{s.}{restaurante | lanchonete}
  \end{phonetics}
\end{entry}

\begin{entry}{饮料}{7,10}
  \begin{phonetics}{饮料}{yin3liao4}
    \definition{s.}{bebida}
  \end{phonetics}
\end{entry}

\begin{entry}{驱}{7}[Radical 馬]
  \begin{phonetics}{驱}{qu1}
    \definition{v.}{expulsar | repelir}
  \end{phonetics}
\end{entry}

\begin{entry}{驴}{7}[Radical 馬]
  \begin{phonetics}{驴}{lv2}
    \definition[头]{s.}{burro | asno | jumento | jegue}
  \end{phonetics}
\end{entry}

\begin{entry}{鸡}{7}[Radical 鳥]
  \begin{phonetics}{鸡}{ji1}[][HSK 2]
    \definition[只]{s.}{galo, galinha | (gíria) prostituta}
  \end{phonetics}
\end{entry}

\begin{entry}{鸡蛋}{7,11}
  \begin{phonetics}{鸡蛋}{ji1dan4}[][HSK 1]
    \definition[个,打]{s.}{ovo de galinha}
  \end{phonetics}
\end{entry}

\begin{entry}{麦当劳}{7,6,7}
  \begin{phonetics}{麦当劳}{mai4dang1lao2}
    \definition*{s.}{McDonald's (empresa de \emph{fast-food})}
  \seealsoref{麦当劳叔叔}{mai4dang1lao2 shu1shu5}
  \end{phonetics}
\end{entry}

\begin{entry}{麦当劳叔叔}{7,6,7,8,8}
  \begin{phonetics}{麦当劳叔叔}{mai4dang1lao2 shu1shu5}
    \definition*{s.}{Ronald McDonald}
  \seealsoref{麦当劳}{mai4dang1lao2}
  \end{phonetics}
\end{entry}

\begin{entry}{麦淇淋}{7,11,11}
  \begin{phonetics}{麦淇淋}{mai4qi2lin2}
    \definition{s.}{(empréstimo linguístico) margarina}
  \end{phonetics}
\end{entry}

\begin{entry}{龟速}{7,10}
  \begin{phonetics}{龟速}{gui1su4}
    \definition{adv.}{tão lento quanto uma tartaruga}
  \end{phonetics}
\end{entry}

%%%%% EOF %%%%%

