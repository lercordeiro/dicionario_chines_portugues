%%%
%%% 7画
%%%

\section*{7画}\addcontentsline{toc}{section}{7画}

\begin{entry}{两}{7}{⼀}
  \begin{phonetics}{两}{liang3}[][HSK 1,2]
    \definition*{s.}{sobrenome Liang}
    \definition{clas.}{liang, uma unidade de peso (=50 gramas)}
    \definition{num.}{dois (sempre usado antes de classificadores) | poucos; alguns; indica um número indeterminado}
    \definition{s.}{ambos (lados); qualquer (lado)}
  \end{phonetics}
\end{entry}

\begin{entry}{两边}{7,5}{⼀、⾡}
  \begin{phonetics}{两边}{liang3 bian1}[][HSK 4]
    \definition{s.}{ambos os lados; ambas as direções; ambos os lugares | ambas as partes; ambos os lados}
  \end{phonetics}
\end{entry}

\begin{entry}{两岸}{7,8}{⼀、⼭}
  \begin{phonetics}{两岸}{liang3 an4}[][HSK 5]
    \definition{s.}{ambos os lados; ambas as margens; ambas as costas; entre os dois lados do estreito; bilateral}
  \end{phonetics}
\end{entry}

\begin{entry}{两码事}{7,8,8}{⼀、⽯、⼅}
  \begin{phonetics}{两码事}{liang3ma3shi4}
    \definition{expr.}{duas coisas completamente diferentes}
  \end{phonetics}
\end{entry}

\begin{entry}{严}{7}{⼀}
  \begin{phonetics}{严}{yan2}[][HSK 4]
    \definition*{s.}{sobrenome Yan}
    \definition{adj.}{rígido; rigoroso; estrito; severo}
    \definition{s.}{pai; refere-se ao pai}
  \end{phonetics}
\end{entry}

\begin{entry}{严厉}{7,5}{⼀、⼚}
  \begin{phonetics}{严厉}{yan2li4}[][HSK 5]
    \definition{adj.}{severo; rigoroso}
  \end{phonetics}
\end{entry}

\begin{entry}{严肃}{7,8}{⼀、⾀}
  \begin{phonetics}{严肃}{yan2su4}[][HSK 5]
    \definition{adj.}{sério; solene; sincero; (expressão, atmosfera, etc.) faz as pessoas se sentirem admiradas e desconfortáveis | sóbrio; grave; sério; sincero}
    \definition{v.}{aplicar rigorosamente; fazer algo sério}
  \end{phonetics}
\end{entry}

\begin{entry}{严重}{7,9}{⼀、⾥}
  \begin{phonetics}{严重}{yan2zhong4}[][HSK 4]
    \definition{adj.}{sério; grave; crítico; severo}
  \end{phonetics}
\end{entry}

\begin{entry}{严重打伤}{7,9,5,6}{⼀、⾥、⼿、⼈}
  \begin{phonetics}{严重打伤}{yan2zhong4 da3 shang1}
    \definition{s.}{gravemente ferido}
  \end{phonetics}
\end{entry}

\begin{entry}{严重伤害}{7,9,6,10}{⼀、⾥、⼈、⼧}
  \begin{phonetics}{严重伤害}{yan2zhong4 shang1hai4}
    \definition{s.}{ferimento grave}
  \end{phonetics}
\end{entry}

\begin{entry}{严重关切}{7,9,6,4}{⼀、⾥、⼋、⼑}
  \begin{phonetics}{严重关切}{yan2zhong4guan1qie4}
    \definition{s.}{preocupação séria}
  \end{phonetics}
\end{entry}

\begin{entry}{严重危害}{7,9,6,10}{⼀、⾥、⼙、⼧}
  \begin{phonetics}{严重危害}{yan2zhong4wei1hai4}
    \definition{s.}{danos graves}
  \end{phonetics}
\end{entry}

\begin{entry}{严重后果}{7,9,6,8}{⼀、⾥、⼝、⽊}
  \begin{phonetics}{严重后果}{yan2zhong4hou4guo3}
    \definition{s.}{consequências sérias | repercursões graves}
  \end{phonetics}
\end{entry}

\begin{entry}{严重地}{7,9,6}{⼀、⾥、⼟}
  \begin{phonetics}{严重地}{yan2zhong4 di4}
    \definition{adv.}{seriamente | gravemente}
  \end{phonetics}
\end{entry}

\begin{entry}{严重问题}{7,9,6,15}{⼀、⾥、⾨、⾴}
  \begin{phonetics}{严重问题}{yan2zhong4wen4ti2}
    \definition{s.}{problema sério}
  \end{phonetics}
\end{entry}

\begin{entry}{严重性}{7,9,8}{⼀、⾥、⼼}
  \begin{phonetics}{严重性}{yan2zhong4xing4}
    \definition{s.}{seriedade | gravidade}
  \end{phonetics}
\end{entry}

\begin{entry}{严重破坏}{7,9,10,7}{⼀、⾥、⽯、⼟}
  \begin{phonetics}{严重破坏}{yan2zhong4 po4huai4}
    \definition{s.}{destruição grave}
  \end{phonetics}
\end{entry}

\begin{entry}{严格}{7,10}{⼀、⽊}
  \begin{phonetics}{严格}{yan2ge2}[][HSK 4]
    \definition{adj.}{rígido; estrito; rigoroso; muito consciente e meticuloso na implementação de sistemas e no domínio de padrões}
    \definition{v.}{tornar (sistemas, provisões, etc.) rigorosos;}
  \end{phonetics}
\end{entry}

\begin{entry}{乱}{7}{⼄}
  \begin{phonetics}{乱}{luan4}[][HSK 3]
    \definition{adj.}{em desordem; em confusão; em desarrumação; sem ordem nem organização | em um estado mental confuso | (de uma sociedade) turbulento; agitado | (de relações sexuais) impróprio; promíscuo}
    \definition{adv.}{aleatoriamente; arbitrariamente; indiscriminadamente; sem restrições; à vontade}
    \definition{s.}{motim; agitação; tumulto; revolta; guerra; calamidade}
    \definition{v.}{confundir; embaralhar; misturar; causar desordem}
  \end{phonetics}
\end{entry}

\begin{entry}{亩}{7}{⼇}
  \begin{phonetics}{亩}{mu3}
    \definition{clas.}{usado para campos | unidade de área igual a um décimo quinto de um hectare}
  \end{phonetics}
\end{entry}

\begin{entry}{估计}{7,4}{⼈、⾔}
  \begin{phonetics}{估计}{gu1ji4}[][HSK 5]
    \definition{v.}{fazer contas; estimar; calcular; julgar a natureza, quantidade, mudança, etc. de uma coisa em uma determinada situação | parecer; parecer como se; aparentar; fazer inferências aproximadas sobre a natureza, a quantidade e a mudança das coisas com base em determinadas circunstâncias}
  \end{phonetics}
\end{entry}

\begin{entry}{伲}{7}{⼈}
  \begin{phonetics}{伲}{ni4}
    \definition{pron.}{(dialeto) eu | meu | nosso | nós}
  \seealsoref{你}{ni3}
  \end{phonetics}
\end{entry}

\begin{entry}{伴侣}{7,8}{⼈、⼈}
  \begin{phonetics}{伴侣}{ban4lv3}
    \definition{s.}{companheiro | parceiro}
  \end{phonetics}
\end{entry}

\begin{entry}{伸}{7}{⼈}
  \begin{phonetics}{伸}{shen1}[][HSK 5]
    \definition{v.}{alongar; esticar; estender}
  \end{phonetics}
\end{entry}

\begin{entry}{但}{7}{⼈}
  \begin{phonetics}{但}{dan4}[][HSK 2]
    \definition*{s.}{sobrenome Dan}
    \definition{adv.}{apenas; meramente; indica uma restrição ao âmbito da ação, equivalente a 只 ou 仅}
    \definition{conj.}{mas; ainda assim; mesmo assim; no entanto; contudo; usado na última oração, conecta duas orações, expressando uma relação de transição, equivalente a 可是 ou 不过}
  \seealsoref{不过}{bu2guo4}
  \seealsoref{仅}{jin3}
  \seealsoref{可是}{ke3shi4}
  \seealsoref{只}{zhi3}
  \end{phonetics}
\end{entry}

\begin{entry}{但是}{7,9}{⼈、⽇}
  \begin{phonetics}{但是}{dan4 shi4}[][HSK 2]
    \definition{conj.}{mas; contudo; no entanto; mesmo assim; usado na segunda parte da frase para indicar uma mudança, geralmente acompanhada de expressões como 虽然 ou 尽管}
  \seealsoref{尽管}{jin3guan3}
  \seealsoref{虽然}{sui1 ran2}
  \end{phonetics}
\end{entry}

\begin{entry}{位}{7}{⼈}
  \begin{phonetics}{位}{wei4}[][HSK 2]
    \definition*{s.}{sobrenome Wei}
    \definition{clas.}{usado para pessoas (com cortesia, respeito) | usado para bits binários}[十六位___16 bits]
    \definition{s.}{lugar; localização; o lugar onde ou onde alguém está localizado | posto; \emph{status}; posição; a posição de uma pessoa em uma determinada área da vida social | trono; refere-se especificamente ao status do imperador | lugar; dígito; a posição de cada dígito em um número}
  \end{phonetics}
\end{entry}

\begin{entry}{位于}{7,3}{⼈、⼆}
  \begin{phonetics}{位于}{wei4yu2}[][HSK 4]
    \definition{v.}{estar localizado; estar situado}
  \end{phonetics}
\end{entry}

\begin{entry}{位子}{7,3}{⼈、⼦}
  \begin{phonetics}{位子}{wei4zi5}
    \definition{s.}{lugar | assento}
  \end{phonetics}
\end{entry}

\begin{entry}{位居}{7,8}{⼈、⼫}
  \begin{phonetics}{位居}{wei4ju1}
    \definition{v.}{estar localizado em}
  \end{phonetics}
\end{entry}

\begin{entry}{位置}{7,13}{⼈、⽹}
  \begin{phonetics}{位置}{wei4zhi4}[][HSK 4]
    \definition[通,个]{s.}{assento; lugar; localização | lugar; posição; \emph{status} | posição (por exemplo: cargo no escritório)}
  \end{phonetics}
\end{entry}

\begin{entry}{低}{7}{⼈}
  \begin{phonetics}{低}{di1}[][HSK 2]
    \definition*{s.}{sobrenome Di}
    \definition{adj.}{baixo; distância pequena de baixo para cima; próximo ao solo | abaixo da média; abaixo do padrão geral | inferior (em grau); de nível inferior}
    \definition{v.}{deixar cair; pendurar; abaixar (a cabeça)}
  \end{phonetics}
\end{entry}

\begin{entry}{低于}{7,3}{⼈、⼆}
  \begin{phonetics}{低于}{di1 yu2}[][HSK 5]
    \definition{v.}{ser inferior a; algo ou fenômeno é, de alguma forma, inferior ou pior do que outra coisa}
  \end{phonetics}
\end{entry}

\begin{entry}{低潮}{7,15}{⼈、⽔}
  \begin{phonetics}{低潮}{di1chao2}
    \definition{s.}{maré baixa/vazante; o nível mais baixo da maré durante um ciclo de maré (distinto da 高潮) | vazante baixa; o ponto mais baixo; uma metáfora para o baixo estágio de desenvolvimento das coisas}
  \seealsoref{高潮}{gao1chao2}
  \end{phonetics}
\end{entry}

\begin{entry}{住}{7}{⼈}
  \begin{phonetics}{住}{zhu4}[][HSK 1]
    \definition{adv.}{firmemente; indica estabilidade ou firmeza}
    \definition{v.}{viver; residir; morar; ficar | parar; cessar | (após um verbo) com firmeza; até parar | hospedar; acomodar | parar; interromper | ser competente; ser qualificado; estar à altura; usado com 得 ou 不, indica que a força é suficiente (ou insuficiente)}
  \seealsoref{不}{bu4}
  \seealsoref{得}{de5}
  \end{phonetics}
\end{entry}

\begin{entry}{住处}{7,5}{⼈、⼡}
  \begin{phonetics}{住处}{zhu4chu4}
    \definition{s.}{morada | habitação | residência}
  \end{phonetics}
\end{entry}

\begin{entry}{住宅}{7,6}{⼈、⼧}
  \begin{phonetics}{住宅}{zhu4zhai2}
    \definition{s.}{residência}
  \end{phonetics}
\end{entry}

\begin{entry}{住房}{7,8}{⼈、⼾}
  \begin{phonetics}{住房}{zhu4fang2}[][HSK 2]
    \definition[套,处]{s.}{habitação; alojamento; casas para as pessoas morarem}
  \end{phonetics}
\end{entry}

\begin{entry}{住所}{7,8}{⼈、⼾}
  \begin{phonetics}{住所}{zhu4suo3}
    \definition[处]{s.}{morada | habitação | residência}
  \end{phonetics}
\end{entry}

\begin{entry}{住院}{7,9}{⼈、⾩}
  \begin{phonetics}{住院}{zhu4 yuan4}[][HSK 2]
    \definition{v.}{estar hospitalizado; estar no hospital; ser internado no hospital para tratamento}
  \end{phonetics}
\end{entry}

\begin{entry}{住嘴}{7,16}{⼈、⼝}
  \begin{phonetics}{住嘴}{zhu4zui3}
    \definition{interj.}{Cale-se!}
    \definition{v.}{calar | calar-se}
  \end{phonetics}
\end{entry}

\begin{entry}{体力}{7,2}{⼈、⼒}
  \begin{phonetics}{体力}{ti3 li4}[][HSK 5]
    \definition{s.}{força física; vigor físico (ou corporal); a força do corpo humano para sustentar suas próprias atividades}
  \end{phonetics}
\end{entry}

\begin{entry}{体内}{7,4}{⼈、⼌}
  \begin{phonetics}{体内}{ti3nei4}
    \definition{adj.}{dentro do corpo | \emph{in vivo} (versus \emph{in vitro} | interno a}
  \end{phonetics}
\end{entry}

\begin{entry}{体会}{7,6}{⼈、⼈}
  \begin{phonetics}{体会}{ti3hui4}[][HSK 3]
    \definition{s.}{conhecimento; compreensão; experiência pessoal}
    \definition{v.}{perceber; saber (ou aprender) com a experiência}
  \end{phonetics}
\end{entry}

\begin{entry}{体现}{7,8}{⼈、⾒}
  \begin{phonetics}{体现}{ti3xian4}[][HSK 3]
    \definition{v.}{refletir; incorporar; encarnar}
  \end{phonetics}
\end{entry}

\begin{entry}{体育}{7,8}{⼈、⾁}
  \begin{phonetics}{体育}{ti3yu4}[][HSK 2]
    \definition{s.}{cultura física; treinamento físico; educação cuja principal tarefa é desenvolver a capacidade física e fortalecer a constituição física, alcançada através da participação em várias atividades esportivas | esportes; atividades esportivas; refere-se a esportes}
  \end{phonetics}
\end{entry}

\begin{entry}{体育场}{7,8,6}{⼈、⾁、⼟}
  \begin{phonetics}{体育场}{ti3 yu4 chang3}[][HSK 2]
    \definition[个,座]{s.}{estádio; campo esportivo; espaço ao ar livre para a prática de exercícios físicos ou competições esportivas}
  \end{phonetics}
\end{entry}

\begin{entry}{体育馆}{7,8,11}{⼈、⾁、⾷}
  \begin{phonetics}{体育馆}{ti3 yu4 guan3}[][HSK 2]
    \definition[个,座,家]{s.}{ginásio; locais esportivos ou competições em ambientes fechados geralmente têm arquibancadas fixas}
  \end{phonetics}
\end{entry}

\begin{entry}{体重}{7,9}{⼈、⾥}
  \begin{phonetics}{体重}{ti3 zhong4}[][HSK 4]
    \definition{s.}{peso corporal}
  \end{phonetics}
\end{entry}

\begin{entry}{体积}{7,10}{⼈、⽲}
  \begin{phonetics}{体积}{ti3ji1}[][HSK 5]
    \definition[个]{s.}{volume; quantidade; o tamanho do espaço ocupado pelo objeto}
  \end{phonetics}
\end{entry}

\begin{entry}{体验}{7,10}{⼈、⾺}
  \begin{phonetics}{体验}{ti3yan4}[][HSK 3]
    \definition[种]{s.}{experiência}
    \definition{v.}{aprender através da prática; aprender através da experiência pessoal}
  \end{phonetics}
\end{entry}

\begin{entry}{体检}{7,11}{⼈、⽊}
  \begin{phonetics}{体检}{ti3 jian3}[][HSK 4]
    \definition{s.}{exame clínico}
    \definition{v.}{fazer um exame médico}
  \end{phonetics}
\end{entry}

\begin{entry}{体操}{7,16}{⼈、⼿}
  \begin{phonetics}{体操}{ti3 cao1}[][HSK 4]
    \definition{s.}{ginástica; esportes, exercícios ou performances de vários movimentos, sem armas ou com o auxílio de determinados equipamentos}
  \end{phonetics}
\end{entry}

\begin{entry}{何}{7}{⼈}
  \begin{phonetics}{何}{he2}
    \definition*{s.}{sobrenome He}
    \definition{adv.}{expressa exclamação, equivalente a 多么}
    \definition{pron.}{O que?; Onde?; Por que? | expressa uma pergunta retórica, equivalente a 岂, 怎}
  \seealsoref{多么}{duo1me5}
  \seealsoref{岂}{qi3}
  \seealsoref{怎}{zen3}
  \end{phonetics}
\end{entry}

\begin{entry}{何不}{7,4}{⼈、⼀}
  \begin{phonetics}{何不}{he2bu4}
    \definition{adv.}{por que não?; use o tom interrogativo para expressar "deveria" ou "pode"}
  \end{phonetics}
\end{entry}

\begin{entry}{何况}{7,7}{⼈、⼎}
  \begin{phonetics}{何况}{he2kuang4}
    \definition{conj.}{além disso | muito menos}
  \end{phonetics}
\end{entry}

\begin{entry}{佛}{7}{⼈}
  \begin{phonetics}{佛}{fo2}
    \definition*{s.}{Buda, abreviação de 佛陀 | Budismo}
  \seealsoref{佛陀}{fo2tuo2}
  \end{phonetics}
  \begin{phonetics}{佛}{fu2}
    \definition{adv.}{aparentemente}
    \definition{s.}{ornamento da cabeça (feminino)}
  \end{phonetics}
\end{entry}

\begin{entry}{佛陀}{7,7}{⼈、⾩}
  \begin{phonetics}{佛陀}{fo2tuo2}
    \definition{s.}{Buda (uma pessoa que atingiu a Budeidade, ou especificamente Siddhartha Gautama)}
  \end{phonetics}
\end{entry}

\begin{entry}{作}{7}{⼈}
  \begin{phonetics}{作}{zuo1}
    \definition{adj.}{(gíria) incômodo}
    \definition{s.}{trabalhador | oficina | (pessoa) de alta manutenção}
  \end{phonetics}
  \begin{phonetics}{作}{zuo4}
    \definition{s.}{escritos ou obras}
    \definition{v.}{fazer | crescer | escrever ou compor | fingir | considerar como | sentir}
  \end{phonetics}
\end{entry}

\begin{entry}{作为}{7,4}{⼈、⼂}
  \begin{phonetics}{作为}{zuo4wei2}[][HSK 4]
    \definition{prep.}{como; na capacidade de; no caráter de; no papel de; em termos de uma certa identidade de uma pessoa ou de uma certa natureza de uma coisa}
    \definition{s.}{ato; ação; conduta; feito; comportamento | conquista; realização; especificamente, uma boa ação}
    \definition{v.}{considerar como; tomar por; olhar como; tratar como | realizar; fazer conquistas; deixar uma marca}
  \end{phonetics}
\end{entry}

\begin{entry}{作文}{7,4}{⼈、⽂}
  \begin{phonetics}{作文}{zuo4wen2}[][HSK 2]
    \definition[篇]{s.}{ensaio; composição; redação}
    \definition{v.+compl.}{(de alunos) escrever uma redação, artigo ou ensaio}
  \end{phonetics}
\end{entry}

\begin{entry}{作业}{7,5}{⼈、⼀}
  \begin{phonetics}{作业}{zuo4ye4}[][HSK 2]
    \definition[份,个]{s.}{tarefa escolar; tarefa de casa atribuída pelos professores aos alunos}
    \definition{v.}{trabalhar; executar tarefa}
  \end{phonetics}
\end{entry}

\begin{entry}{作出}{7,5}{⼈、⼐}
  \begin{phonetics}{作出}{zuo4 chu1}[][HSK 4]
    \definition{v.}{mostrar; tomar (decisões, conclusões, etc. por meio de consideração ou discussão); formar (uma conclusão, decisão, etc.) por meio de consideração ou discussão}
  \end{phonetics}
\end{entry}

\begin{entry}{作用}{7,5}{⼈、⽤}
  \begin{phonetics}{作用}{zuo4yong4}[][HSK 2]
    \definition[副]{s.}{efeito; ação; função; a influência sobre as coisas; o efeito; a utilidade}
    \definition{v.}{afetar; agir sobre; realizar atividades que têm algum impacto nas coisas}
  \end{phonetics}
\end{entry}

\begin{entry}{作者}{7,8}{⼈、⽼}
  \begin{phonetics}{作者}{zuo4zhe3}[][HSK 3]
    \definition[位,名,个]{s.}{autor; escritor; uma pessoa que escreve artigos ou cria obras de arte}
  \end{phonetics}
\end{entry}

\begin{entry}{作品}{7,9}{⼈、⼝}
  \begin{phonetics}{作品}{zuo4pin3}[][HSK 3]
    \definition[个,部,篇,幅]{s.}{obra de arte; obras concluídas de literatura e arte}
  \end{phonetics}
\end{entry}

\begin{entry}{作家}{7,10}{⼈、⼧}
  \begin{phonetics}{作家}{zuo4jia1}[][HSK 2]
    \definition[位,名,个,些]{s.}{escritor; autor; pessoas que alcançaram sucesso na criação literária}
  \end{phonetics}
\end{entry}

\begin{entry}{你}{7}{⼈}
  \begin{phonetics}{你}{ni3}[][HSK 1]
    \definition{pron.}{você (segunda pessoa do singular); refere-se à pessoa com quem se está conversando | (referindo-se a qualquer pessoa) você; um; qualquer um | com 我 ou 你 em estruturas paralelas para indicar várias ou muitas pessoas se comportando da mesma maneira}
  \seealsoref{您}{nin2}
  \seealsoref{我}{wo3}
  \end{phonetics}
\end{entry}

\begin{entry}{你们}{7,5}{⼈、⼈}
  \begin{phonetics}{你们}{ni3men5}[][HSK 1]
    \definition{pron.}{você (segunda pessoa do plural); refere-se a mais de uma pessoa ou a várias pessoas, incluindo a outra parte}
  \end{phonetics}
\end{entry}

\begin{entry}{你们的}{7,5,8}{⼈、⼈、⽩}
  \begin{phonetics}{你们的}{ni3men5 de5}
    \definition{pron.}{vossos}
  \end{phonetics}
\end{entry}

\begin{entry}{你好}{7,6}{⼈、⼥}
  \begin{phonetics}{你好}{ni3hao3}
    \definition{interj.}{Olá! | Oi!}
  \end{phonetics}
\end{entry}

\begin{entry}{你的}{7,8}{⼈、⽩}
  \begin{phonetics}{你的}{ni3 de5}
    \definition{pron.}{seu}
  \end{phonetics}
\end{entry}

\begin{entry}{克}{7}{⼗}
  \begin{phonetics}{克}{ke4}[][HSK 2]
    \definition*{s.}{sobrenome Ke}
    \definition{clas.}{grama (g) | unidade tibetana de volume ou medida seca (com capacidade para cerca de 25 jin,  斤, de cevada) | unidade tibetana de área de terra equivalente a cerca de 1 mu, 亩}
    \definition{v.}{poder; ser capaz de | tolerar; conter; restringir; suprimir| subjugar; capturar; conquistar (uma cidade, etc.) | digerir (alimentos) | reduzir; diminuir | definir um limite de tempo}
  \seealsoref{斤}{jin1}
  \seealsoref{亩}{mu3}
  \end{phonetics}
\end{entry}

\begin{entry}{克服}{7,8}{⼗、⽉}
  \begin{phonetics}{克服}{ke4fu2}[][HSK 3]
    \definition{v.}{sobrepujar; superar; conquistar; vencer com força de vontade e determinação (deficiências, erros, fenômenos negativos, condições desfavoráveis, etc.) | aguentar; suportar (dificuldades, inconveniências, etc.)}
  \end{phonetics}
\end{entry}

\begin{entry}{免费}{7,9}{⼉、⾙}
  \begin{phonetics}{免费}{mian3fei4}[][HSK 4]
    \definition{s.}{gratuito; sem custo}
    \definition{v.+compl.}{isentar de taxas; tonar grátis}
  \end{phonetics}
\end{entry}

\begin{entry}{免得}{7,11}{⼉、⼻}
  \begin{phonetics}{免得}{mian3de5}
    \definition{conj.}{de modo a não | para evitar | para que não}
  \end{phonetics}
\end{entry}

\begin{entry}{免税}{7,12}{⼉、⽲}
  \begin{phonetics}{免税}{mian3shui4}
    \definition{adj.}{isento de impostos (tributação)}
    \definition{s.}{livre de impostos | isenção de impostos}
    \definition{v.+compl.}{isentar impostos}
  \end{phonetics}
\end{entry}

\begin{entry}{兵}{7}{⼋}
  \begin{phonetics}{兵}{bing1}[][HSK 4]
    \definition[名]{s.}{armas; armamentos | soldado; pessoal militar | exército; tropas | soldado raso; membro mais jovem do exército | assuntos militares (estratégia) | peão, uma das peças do xadrez chinês}
  \end{phonetics}
\end{entry}

\begin{entry}{兵器}{7,16}{⼋、⼝}
  \begin{phonetics}{兵器}{bing1qi4}
    \definition{s.}{armas | armamento}
  \end{phonetics}
\end{entry}

\begin{entry}{况且}{7,5}{⼎、⼀}
  \begin{phonetics}{况且}{kuang4qie3}
    \definition{conj.}{além disso | além do mais}
  \end{phonetics}
\end{entry}

\begin{entry}{冷}{7}{⼎}
  \begin{phonetics}{冷}{leng3}[][HSK 1]
    \definition*{s.}{sobrenome Leng}
    \definition{adj.}{frio; baixa temperatura; sensação de frio | gelado; frio por natureza; sem entusiasmo; sem gentileza | desolado; pouco frequentado; quieto; sem agitação | negligenciado; indesejável; ignorado por todos | raro; estranho; incomum | feito em segredo; filmado de forma escondida; lançado secretamente}
    \definition{v.}{esfriar; resfriar | esfriar; congelar; tornar-se indiferente, apático | ignorar}
  \end{phonetics}
\end{entry}

\begin{entry}{冷门}{7,3}{⼎、⾨}
  \begin{phonetics}{冷门}{leng3men2}
    \definition{s.}{uma profissão, ofício ou ramo de aprendizagem que recebe pouca atenção | um vencedor inesperado; azarão}
  \end{phonetics}
\end{entry}

\begin{entry}{冷静}{7,14}{⼎、⾭}
  \begin{phonetics}{冷静}{leng3jing4}[][HSK 4]
    \definition{adj.}{calmo; descreve uma pessoa que consegue ficar atenta em uma situação importante ou de emergência e não toma decisões aleatórias por causa de seus sentimentos no momento | (lugar) tranquilo; quieto; deserto}
  \end{phonetics}
\end{entry}

\begin{entry}{冻}{7}{⼎}
  \begin{phonetics}{冻}{dong4}[][HSK 5]
    \definition*{s.}{sobrenome Dong}
    \definition{s.}{geleia; gelatina;}
    \definition{v.}{congelar; ser congelado | ficar com frio ou sentir frio}
  \end{phonetics}
\end{entry}

\begin{entry}{初}{7}{⾐}
  \begin{phonetics}{初}{chu1}[][HSK 3]
    \definition*{s.}{sobrenome Chu}
    \definition{adj.}{primeiro (em ordem) | elementar; rudimentar | original}
    \definition{adv.}{pela primeira vez; apenas começando; indica que a ação está ocorrendo pela primeira vez ou acabou de começar}
    \definition{pref.}{anexado aos numerais de um a dez para indicar ordem (primeiro ao décimo)}
    \definition{s.}{no início de; na primeira parte de | o estágio júnior (pleno; sênior)}
  \end{phonetics}
\end{entry}

\begin{entry}{初中}{7,4}{⾐、⼁}
  \begin{phonetics}{初中}{chu1 zhong1}[][HSK 3]
    \definition[所,个]{s.}{ensino médio; ensino fundamental}
  \end{phonetics}
\end{entry}

\begin{entry}{初心}{7,4}{⾐、⼼}
  \begin{phonetics}{初心}{chu1xin1}
    \definition{s.}{intenção original de alguém, aspiração, etc. | (budismo) ``mente do iniciante'' (ter a mente aberta quando estudando um assunto como um iniciante no assunto teria)}
  \end{phonetics}
\end{entry}

\begin{entry}{初级}{7,6}{⾐、⽷}
  \begin{phonetics}{初级}{chu1ji2}[][HSK 3]
    \definition{adj.}{elementar; primário; júnior; inicial; o nível mais baixo; de baixa qualidade}
  \end{phonetics}
\end{entry}

\begin{entry}{初步}{7,7}{⾐、⽌}
  \begin{phonetics}{初步}{chu1bu4}[][HSK 3]
    \definition{adj.}{inicial; preliminar; imaturo, incompleto}
  \end{phonetics}
\end{entry}

\begin{entry}{初期}{7,12}{⾐、⽉}
  \begin{phonetics}{初期}{chu1 qi1}[][HSK 5]
    \definition{s.}{primórdio; estágio inicial; primeiros dias; estágio preliminar; período inicial}
  \end{phonetics}
\end{entry}

\begin{entry}{判断}{7,11}{⼑、⽄}
  \begin{phonetics}{判断}{pan4duan4}[][HSK 3]
    \definition[个]{s.}{julgamento}
    \definition{v.}{julgar; decidir}
  \end{phonetics}
\end{entry}

\begin{entry}{利}{7}{⼑}
  \begin{phonetics}{利}{li4}
    \definition*{s.}{sobrenome Li}
    \definition{adj.}{afiado; cortante | favorável; conveniente; sem dificuldades; sem ou com poucas dificuldades}
    \definition{s.}{benefício; vantagem | lucro; ganhos; juros}
    \definition{v.}{beneficiar; tornar vantajoso}
  \end{phonetics}
\end{entry}

\begin{entry}{利用}{7,5}{⼑、⽤}
  \begin{phonetics}{利用}{li4yong4}[][HSK 3]
    \definition{v.}{usar; utilizar; fazer uso de; fazer com que algo ou alguém funcione bem| explorar; tirar vantagem de; usar meios para fazer com que pessoas ou coisas sirvam aos seus interesses}
  \end{phonetics}
\end{entry}

\begin{entry}{利息}{7,10}{⼑、⼼}
  \begin{phonetics}{利息}{li4xi1}[][HSK 4]
    \definition{s.}{acréscimo; juros; dinheiro recebido além do valor principal como resultado de depósitos ou empréstimos (diferenciado de 本金)}
  \seealsoref{本金}{ben3 jin1}
  \end{phonetics}
\end{entry}

\begin{entry}{利润}{7,10}{⼑、⽔}
  \begin{phonetics}{利润}{li4run4}[][HSK 5]
    \definition[笔]{s.}{lucro; o dinheiro ganho com atividades comerciais e industriais}
  \end{phonetics}
\end{entry}

\begin{entry}{利益}{7,10}{⼑、⽫}
  \begin{phonetics}{利益}{li4yi4}[][HSK 4]
    \definition[个,种]{s.}{ganho; lucro; juros; benefício}
  \end{phonetics}
\end{entry}

\begin{entry}{别}{7}{⼑}
  \begin{phonetics}{别}{bie2}[][HSK 1,4]
    \definition*{s.}{sobrenome Bie}
    \definition{adv.}{não; nada de (pedir a alguém para não fazer); é melhor não | talvez, usado em conjunto com a palavra 是 para indicar especulação.}
    \definition{pron.}{outro; algum outro}
    \definition{s.}{distinção; diferença | classificação}
    \definition{v.}{deixar; partir; separar | diferenciar; distinguir; encontrar aspectos diferentes | fixar objetos com pinos | girar; virar | aderir; colar; preder}
  \seealsoref{是}{shi4}
  \end{phonetics}
  \begin{phonetics}{别}{bie4}
    \definition{v.}{fazer com que alguém mude seus hábitos, opiniões, etc.}
  \end{phonetics}
\end{entry}

\begin{entry}{别人}{7,2}{⼑、⼈}
  \begin{phonetics}{别人}{bie2 ren2}[][HSK 1]
    \definition{pron.}{outros; outras pessoas}
    \definition{s.}{outros; pessoas; outras pessoas; refere-se a alguém diferente de si mesmo}
  \end{phonetics}
\end{entry}

\begin{entry}{别的}{7,8}{⼑、⽩}
  \begin{phonetics}{别的}{bie2 de5}[][HSK 1]
    \definition{pron.}{outro; o resto}
  \end{phonetics}
\end{entry}

\begin{entry}{别说}{7,9}{⼑、⾔}
  \begin{phonetics}{别说}{bie2shuo1}
    \definition{v.}{não falar de | não mencionar}
  \end{phonetics}
\end{entry}

\begin{entry}{助手}{7,4}{⼒、⼿}
  \begin{phonetics}{助手}{zhu4shou3}[][HSK 5]
    \definition[个]{s.}{ajudante; auxiliar; assistente; alguém que ajuda os outros com seu trabalho}
  \end{phonetics}
\end{entry}

\begin{entry}{助兴}{7,6}{⼒、⼋}
  \begin{phonetics}{助兴}{zhu4xing4}
    \definition{v.+compl.}{animar as coisas | juntar-se à diversão}
  \end{phonetics}
\end{entry}

\begin{entry}{助理}{7,11}{⼒、⽟}
  \begin{phonetics}{助理}{zhu4li3}[][HSK 5]
    \definition[个,名,位]{s.}{deputado; assistente; auxiliar do diretor responsável (geralmente usado em cargos) | ajudante; assistente; pessoa que auxilia o responsável a fazer as coisas}
  \end{phonetics}
\end{entry}

\begin{entry}{努力}{7,2}{⼒、⼒}
  \begin{phonetics}{努力}{nu3li4}[][HSK 2]
    \definition{adj.}{extenuante; árduo | diligente; trabalhador; quem faz as coisas com o máximo de capacidade ou esforço possível}
    \definition{s.}{esforço; tentativa; fazer o melhor possível}
    \definition{v.}{fazer grandes esforços; esforçar-se; empenhar-se | esforçar-se; usar toda a força possível}
  \end{phonetics}
\end{entry}

\begin{entry}{劳工同事}{7,3,6,8}{⼒、⼯、⼝、⼅}
  \begin{phonetics}{劳工同事}{lao2gong1 tong2shi4}
    \definition{s.}{colaborador | colega de trabalho}
  \end{phonetics}
\end{entry}

\begin{entry}{劳动}{7,6}{⼒、⼒}
  \begin{phonetics}{劳动}{lao2dong4}[][HSK 5]
    \definition[次]{s.}{trabalho; mão de obra; atividades intelectuais ou físicas que podem criar valor | trabalho físico; trabalho manual; referindo-se especificamente ao trabalho físico}
    \definition{v.}{realizar trabalho físico}
  \end{phonetics}
\end{entry}

\begin{entry}{医}{7}{⼖}
  \begin{phonetics}{医}{yi1}
    \definition{s.}{médico | medicina}
    \definition{v.}{curar | tratar}
  \end{phonetics}
\end{entry}

\begin{entry}{医生}{7,5}{⼖、⽣}
  \begin{phonetics}{医生}{yi1sheng1}[][HSK 1]
    \definition[位,个,名]{s.}{médico; clínico; pessoa que possui conhecimentos médicos e cuja profissão é tratar doenças}
  \end{phonetics}
\end{entry}

\begin{entry}{医疗}{7,7}{⼖、⽧}
  \begin{phonetics}{医疗}{yi1 liao2}[][HSK 4]
    \definition{s.}{tratamento médico; tratamento de doenças}
  \end{phonetics}
\end{entry}

\begin{entry}{医学}{7,8}{⼖、⼦}
  \begin{phonetics}{医学}{yi1 xue2}[][HSK 4]
    \definition{s.}{medicina; iatrologia; ciência médica; ciência da prevenção e do tratamento de doenças e da proteção e promoção da saúde humana}
  \end{phonetics}
\end{entry}

\begin{entry}{医院}{7,9}{⼖、⾩}
  \begin{phonetics}{医院}{yi1yuan4}[][HSK 1]
    \definition[家,所,个]{s.}{hospital; instituições que tratam e cuidam de pacientes, e também realizam exames de saúde, prevenção de doenças, etc.}
  \end{phonetics}
\end{entry}

\begin{entry}{即}{7}{⼙}
  \begin{phonetics}{即}{ji2}
    \definition{conj.}{e | até | mesmo se/embora}
  \end{phonetics}
\end{entry}

\begin{entry}{即使}{7,8}{⼙、⼈}
  \begin{phonetics}{即使}{ji2shi3}[][HSK 5]
    \definition{conj.}{mesmo; mesmo que; mesmo se; apesar de; expressando uma concessão hipotética}
  \end{phonetics}
\end{entry}

\begin{entry}{即或}{7,8}{⼙、⼽}
  \begin{phonetics}{即或}{ji2huo4}
    \definition{conj.}{mesmo se/embora}
  \end{phonetics}
\end{entry}

\begin{entry}{即若}{7,8}{⼙、⾋}
  \begin{phonetics}{即若}{ji2ruo4}
    \definition{conj.}{mesmo se/embora}
  \end{phonetics}
\end{entry}

\begin{entry}{即便}{7,9}{⼙、⼈}
  \begin{phonetics}{即便}{ji2bian4}
    \definition{conj.}{mesmo se/embora}
  \end{phonetics}
\end{entry}

\begin{entry}{即将}{7,9}{⼙、⼨}
  \begin{phonetics}{即将}{ji2jiang1}[][HSK 4]
    \definition{adv.}{em breve; estar prestes a; estar a ponto de}
  \end{phonetics}
\end{entry}

\begin{entry}{即是}{7,9}{⼙、⽇}
  \begin{phonetics}{即是}{ji2shi4}
    \definition{conj.}{aquilo é}
  \end{phonetics}
\end{entry}

\begin{entry}{却}{7}{⼙}
  \begin{phonetics}{却}{que4}[][HSK 4]
    \definition{adv.}{mas; contudo; no entanto; enquanto; indica um ponto de virada}
    \definition{v.}{recuar; retroceder | afastar; repelir; desencorajar | declinar; recusar; rejeitar}
    \definition{v.aux.}{usado depois de certos verbos para indicar a conclusão de uma ação}
  \end{phonetics}
\end{entry}

\begin{entry}{却是}{7,9}{⼙、⽇}
  \begin{phonetics}{却是}{que4shi4}
    \definition{conj.}{no entanto | realmente | o fato é\dots | mas isso é\dots}
  \end{phonetics}
\end{entry}

\begin{entry}{县}{7}{⼛}
  \begin{phonetics}{县}{xian4}[][HSK 4]
    \definition[个]{s.}{condado; unidade de divisão administrativa}
  \end{phonetics}
\end{entry}

\begin{entry}{君主立宪制}{7,5,5,9,8}{⼝、⼂、⽴、⼧、⼑}
  \begin{phonetics}{君主立宪制}{jun1zhu3li4xian4zhi4}
    \definition{s.}{monarquia constitucional}
  \end{phonetics}
\end{entry}

\begin{entry}{吟诗}{7,8}{⼝、⾔}
  \begin{phonetics}{吟诗}{yin2shi1}
    \definition{v.}{recitar poesia}
  \end{phonetics}
\end{entry}

\begin{entry}{否认}{7,4}{⼝、⾔}
  \begin{phonetics}{否认}{fou3ren4}[][HSK 3]
    \definition{v.}{negar; repudiar; não reconhecer}
  \end{phonetics}
\end{entry}

\begin{entry}{否则}{7,6}{⼝、⼑}
  \begin{phonetics}{否则}{fou3ze2}[][HSK 4]
    \definition{conj.}{senão; se não; ou então; se não for isso}
  \end{phonetics}
\end{entry}

\begin{entry}{否定}{7,8}{⼝、⼧}
  \begin{phonetics}{否定}{fou3ding4}[][HSK 3]
    \definition{adj.}{negativo; contrário}
    \definition{v.}{rejeitar; negar a existência ou a autenticidade de algo}
  \end{phonetics}
\end{entry}

\begin{entry}{吧}{7}{⼝}
  \begin{phonetics}{吧}{ba1}
    \definition{s.}{som de estalo, som crepitante}
    \definition{v.}{puxar o cachimbo; fumar | abreviação de ``bar''}
  \end{phonetics}
  \begin{phonetics}{吧}{ba5}[][HSK 1]
    \definition{part.}{indica discussão, sugestão, solicitação ou comando no final de uma frase | indica concordância ou aprovação no final de uma frase | indica uma pergunta ou especulação no final de uma frase | indica incerteza no final de uma frase | em uma frase, indica uma pausa, carrega um tom hipotético, frequentemente apresenta um contraste e implica um dilema}
  \end{phonetics}
\end{entry}

\begin{entry}{吨}{7}{⼝}
  \begin{phonetics}{吨}{dun1}[][HSK 5]
    \definition{clas.}{tonelada}
  \end{phonetics}
\end{entry}

\begin{entry}{含}{7}{⼝}
  \begin{phonetics}{含}{han2}[][HSK 4]
    \definition{v.}{manter na boca (sem engolir ou cuspir) | conter; incluir | cuidar; acalentar; abrigar}
  \end{phonetics}
\end{entry}

\begin{entry}{含义}{7,3}{⼝、⼂}
  \begin{phonetics}{含义}{han2yi4}[][HSK 4]
    \definition[个,种,层]{s.}{sentido; mensagem; significado; implicação}
  \end{phonetics}
\end{entry}

\begin{entry}{含有}{7,6}{⼝、⽉}
  \begin{phonetics}{含有}{han2 you3}[][HSK 4]
    \definition{v.}{conter; ter; incluir}
  \end{phonetics}
\end{entry}

\begin{entry}{含金量}{7,8,12}{⼝、⾦、⾥}
  \begin{phonetics}{含金量}{han2jin1liang4}
    \definition{adj.}{conteúdo de ouro | (fig.) valioso}
  \end{phonetics}
\end{entry}

\begin{entry}{含量}{7,12}{⼝、⾥}
  \begin{phonetics}{含量}{han2 liang4}[][HSK 4]
    \definition{s.}{conteúdo; a quantidade de um componente contido em uma substância}
  \end{phonetics}
\end{entry}

\begin{entry}{听}{7}{⼝}
  \begin{phonetics}{听}{ting1}[][HSK 1]
    \definition{clas.}{latas; usado para bebidas e alimentos para levar consigo}
    \definition{s.}{lata; embalagem metálica; recipiente cilíndrico utilizado para armazenar bebidas, alimentos, etc.}
    \definition{v.}{ouvir; escutar | obedecer; dar ouvidos; aceitar | supervisionar; administrar; tratar (assuntos políticos); julgar (casos) | permitir; deixar ser; deixar fazer}
  \end{phonetics}
  \begin{phonetics}{听}{yin3}
    \definition[个]{s.}{lata; embalagem metálica}
  \end{phonetics}
\end{entry}

\begin{entry}{听力}{7,2}{⼝、⼒}
  \begin{phonetics}{听力}{ting1li4}[][HSK 3]
    \definition{s.}{audição; capacidade auditiva | compreensão auditiva (na aprendizagem de línguas)}
  \end{phonetics}
\end{entry}

\begin{entry}{听力理解}{7,2,11,13}{⼝、⼒、⽟、⾓}
  \begin{phonetics}{听力理解}{ting1li4li3jie3}
    \definition{s.}{compreensão auditiva}
  \end{phonetics}
\end{entry}

\begin{entry}{听小骨}{7,3,9}{⼝、⼩、⾻}
  \begin{phonetics}{听小骨}{ting1xiao3gu3}
    \definition{s.}{ossículos (do ouvido médio)}
  \seealsoref{听骨}{ting1gu3}
  \end{phonetics}
\end{entry}

\begin{entry}{听见}{7,4}{⼝、⾒}
  \begin{phonetics}{听见}{ting1 jian4}[][HSK 1]
    \definition{v.}{ouvir; conseguir ouvir}
  \end{phonetics}
\end{entry}

\begin{entry}{听写}{7,5}{⼝、⼍}
  \begin{phonetics}{听写}{ting1 xie3}[][HSK 1]
    \definition{s.}{ditado}
    \definition{v.}{ouvir e escrever}
  \end{phonetics}
\end{entry}

\begin{entry}{听众}{7,6}{⼝、⼈}
  \begin{phonetics}{听众}{ting1 zhong4}[][HSK 3]
    \definition{s.}{audiência; ouvintes}
  \end{phonetics}
\end{entry}

\begin{entry}{听会}{7,6}{⼝、⼈}
  \begin{phonetics}{听会}{ting1hui4}
    \definition{v.}{participar de uma reunião (e ouvir o que é discutido)}
  \end{phonetics}
\end{entry}

\begin{entry}{听戏}{7,6}{⼝、⼽}
  \begin{phonetics}{听戏}{ting1xi4}
    \definition{v.}{assistir a uma ópera | ver uma ópera}
  \end{phonetics}
\end{entry}

\begin{entry}{听讲}{7,6}{⼝、⾔}
  \begin{phonetics}{听讲}{ting1 jiang3}[][HSK 2]
    \definition{v.+compl.}{assistir a uma palestra; ouvir palestras ou discursos}
  \end{phonetics}
\end{entry}

\begin{entry}{听来}{7,7}{⼝、⽊}
  \begin{phonetics}{听来}{ting1lai2}
    \definition{v.}{ouvir de algum lugar | soar (antigo, estrangeiro, excitante, certo, etc.) | soar como se (ou seja, dar uma impressão ao ouvinte)}
  \end{phonetics}
\end{entry}

\begin{entry}{听凭}{7,8}{⼝、⼏}
  \begin{phonetics}{听凭}{ting1ping2}
    \definition{v.}{permitir (alguém a fazer o que desejar)}
  \end{phonetics}
\end{entry}

\begin{entry}{听到}{7,8}{⼝、⼑}
  \begin{phonetics}{听到}{ting1 dao4}[][HSK 1]
    \definition{v.}{ouvir, escutar; ouvir atentamente, escutar atentamente}
  \end{phonetics}
\end{entry}

\begin{entry}{听命}{7,8}{⼝、⼝}
  \begin{phonetics}{听命}{ting1ming4}
    \definition{v.}{obedecer ordens | receber ordens}
  \end{phonetics}
\end{entry}

\begin{entry}{听说}{7,9}{⼝、⾔}
  \begin{phonetics}{听说}{ting1 shuo1}[][HSK 2]
    \definition{v.}{ser informado; ouvir falar de; ouvir dizer | ouvir e falar}
  \end{phonetics}
\end{entry}

\begin{entry}{听骨}{7,9}{⼝、⾻}
  \begin{phonetics}{听骨}{ting1gu3}
    \definition{s.}{ossículos (do ouvido médio)}
  \seealsoref{听小骨}{ting1xiao3gu3}
  \end{phonetics}
\end{entry}

\begin{entry}{听断}{7,11}{⼝、⽄}
  \begin{phonetics}{听断}{ting1duan4}
    \definition{v.}{ouvir e decidir | julgar (ou seja, ouvir e julgar em um tribunal)}
  \end{phonetics}
\end{entry}

\begin{entry}{听随}{7,11}{⼝、⾩}
  \begin{phonetics}{听随}{ting1sui2}
    \definition{v.}{permitir | obedecer}
  \end{phonetics}
\end{entry}

\begin{entry}{启发}{7,5}{⼝、⼜}
  \begin{phonetics}{启发}{qi3fa1}[][HSK 5]
    \definition{s.}{iluminação; esclarecimento; fenômenos e princípios que levam as pessoas a refletir e a abrir suas mentes}
    \definition{v.}{despertar; inspirar; esclarecer; orientar, fazer com que compreendam}
  \end{phonetics}
\end{entry}

\begin{entry}{启动}{7,6}{⼝、⼒}
  \begin{phonetics}{启动}{qi3 dong4}[][HSK 5]
    \definition{v.}{ligar (uma máquina); acionar; ligar máquinas, equipamentos elétricos, etc., para começar a trabalhar | entrar em vigor; começar a vigorar e a ser implementados planos, projetos, documentos jurídicos, etc.}
  \end{phonetics}
\end{entry}

\begin{entry}{启事}{7,8}{⼝、⼅}
  \begin{phonetics}{启事}{qi3shi4}[][HSK 5]
    \definition{s.}{aviso; anúncio; texto publicado em jornais ou afixado em paredes com o objetivo de divulgar publicamente algo}
  \end{phonetics}
\end{entry}

\begin{entry}{吵}{7}{⼝}
  \begin{phonetics}{吵}{chao3}[][HSK 3]
    \definition{v.}{brigar; discutir; disputar}
  \end{phonetics}
\end{entry}

\begin{entry}{吵架}{7,9}{⼝、⽊}
  \begin{phonetics}{吵架}{chao3jia4}[][HSK 3]
    \definition{v.+compl.}{brigar; discutir; ter uma discussão acalorada}
  \end{phonetics}
\end{entry}

\begin{entry}{吹}{7}{⼝}
  \begin{phonetics}{吹}{chui1}[][HSK 2]
    \definition{v.}{soprar; baforar | tocar (instrumentos de sopro) | (do vento) soprar | gabar-se; vangloriar-se | elogiar; louvar aos céus; adular | (relacionamento) romper; separar-se; (assunto) fracassar}
  \end{phonetics}
\end{entry}

\begin{entry}{吹牛}{7,4}{⼝、⽜}
  \begin{phonetics}{吹牛}{chui1niu2}
    \definition{v.+compl.}{ogulhar-se | gabar-se | destacar-se}
  \end{phonetics}
\end{entry}

\begin{entry}{吾}{7}{⼝}
  \begin{phonetics}{吾}{wu2}
    \definition*{s.}{sobrenome Wu}
    \definition{pron.}{eu | (antigo) meu}
  \end{phonetics}
\end{entry}

\begin{entry}{呀}{7}{⼝}
  \begin{phonetics}{呀}{ya5}[][HSK 4]
    \definition{part.}{usado no lugar de 啊 quando a palavra anterior termina com o som a, e, i, o ou ü}
  \seealsoref{啊}{a5}
  \end{phonetics}
\end{entry}

\begin{entry}{呆}{7}{⼝}
  \begin{phonetics}{呆}{dai1}[][HSK 5]
    \definition*{s.}{sobrenome Dai}
    \definition{adj.}{maçante; de raciocínio lento | em branco; de madeira; rígido; inflexível}
    \definition{v.}{ficar; permanecer;}
  \end{phonetics}
\end{entry}

\begin{entry}{告别}{7,7}{⼝、⼑}
  \begin{phonetics}{告别}{gao4bie2}[][HSK 3]
    \definition{v.+compl.}{dizer adeus a; expressar a outros, por meio de palavras, que está prestes a partir | deixar; sair; partir de | prestar as últimas homenagens ao falecido}
  \end{phonetics}
\end{entry}

\begin{entry}{告诉}{7,7}{⼝、⾔}
  \begin{phonetics}{告诉}{gao4su4}
    \definition{v.}{dizer; informar (dar a conhecer); dizer aos outros, para que todos saibam}
  \end{phonetics}
  \begin{phonetics}{告诉}{gao4su5}[][HSK 1]
    \definition{v.}{dizer; informar (dar a conhecer)}
  \end{phonetics}
\end{entry}

\begin{entry}{告急}{7,9}{⼝、⼼}
  \begin{phonetics}{告急}{gao4ji2}
    \definition{v.}{estar em estado de emergência | relatar uma emergência | solicitar assistência de emergência}
  \end{phonetics}
\end{entry}

\begin{entry}{员}{7}{⼝}
  \begin{phonetics}{员}{yuan2}[][HSK 3]
    \definition{clas.}{para comandantes militares}
    \definition{s.}{uma pessoa envolvida em algum campo de atividade; refere-se a pessoas que trabalham ou estudam | membro; refere-se aos membros de um grupo ou organização}
  \end{phonetics}
\end{entry}

\begin{entry}{员工}{7,3}{⼝、⼯}
  \begin{phonetics}{员工}{yuan2gong1}[][HSK 3]
    \definition[位,名,个]{s.}{funcionário; atendente; balconista; empregado; trabalhador; pessoal}
  \end{phonetics}
\end{entry}

\begin{entry}{园林}{7,8}{⼞、⽊}
  \begin{phonetics}{园林}{yuan2lin2}[][HSK 5]
    \definition{s.}{parque; jardim; área paisagística com plantas e árvores para as pessoas apreciarem e descansarem.}
  \end{phonetics}
\end{entry}

\begin{entry}{囯}{7}{⼞}
  \begin{phonetics}{囯}{guo2}
    \variantof{国}
  \end{phonetics}
\end{entry}

\begin{entry}{困}{7}{⼞}
  \begin{phonetics}{困}{kun4}[][HSK 3]
    \definition{adj.}{cansado; exausto; fatigado | difícil; complicado; difícil e penoso; pobre e miserável | sonolento; com sono; cansado, com vontade de dormir}
    \definition{v.}{ficar encalhado; estar em apuros; preso em dificuldades e sofrimentos ou limitado por circunstâncias e condições que não pode escapar | cercar; envolver; imobilizar; controlar dentro de um determinado limite | dormir}
  \end{phonetics}
\end{entry}

\begin{entry}{困扰}{7,7}{⼞、⼿}
  \begin{phonetics}{困扰}{kun4 rao3}[][HSK 5]
    \definition{v.}{perturbar; deixar perplexo; perseguir}
  \end{phonetics}
\end{entry}

\begin{entry}{困难}{7,10}{⼞、⾫}
  \begin{phonetics}{困难}{kun4nan5}[][HSK 3]
    \definition{adj.}{dificuldades financeiras; circunstâncias difíceis | complicado; complexo; difícil; árduo; a situação é complexa e há muitos obstáculos}
    \definition[种]{s.}{dificuldade; situação difícil; problemas ou situações difíceis de resolver no trabalho e na vida}
  \end{phonetics}
\end{entry}

\begin{entry}{围}{7}{⼞}
  \begin{phonetics}{围}{wei2}[][HSK 3]
    \definition*{s.}{sobrenome Wei}
    \definition{clas.}{o comprimento dos dois polegares e indicadores ou o comprimento de ambos os braços quando unidos}
    \definition{s.}{em volta de tudo; ao redor}
    \definition{v.}{cercar; rodear; circundar; encurralar | enrolar; envolver}
  \end{phonetics}
\end{entry}

\begin{entry}{围巾}{7,3}{⼞、⼱}
  \begin{phonetics}{围巾}{wei2jin1}[][HSK 4]
    \definition[条]{s.}{lenço; cachecol; echarpe; gravata; tiras longas de malha ou tecido usadas ao redor do pescoço para aquecimento, proteção do colarinho ou decoração}
  \end{phonetics}
\end{entry}

\begin{entry}{围绕}{7,9}{⼞、⽷}
  \begin{phonetics}{围绕}{wei2rao4}[][HSK 5]
    \definition{v.}{girar; circundar; dar voltas; girar em torno de algo; cercar | concentrar-se em; centrar-se em; centrar-se em uma questão ou evento (para realizar atividades)}
  \end{phonetics}
\end{entry}

\begin{entry}{坏}{7}{⼟}
  \begin{phonetics}{坏}{huai4}[][HSK 1]
    \definition{adj.}{ruim; prejudicial; insatisfatório; péssimo | mal; extremamente; indica um grau profundo, geralmente usado após verbos ou adjetivos que expressam estado psicológico | podre; estragado; impróprio; prejudicial ao uso}
    \definition[种]{s.}{ideia maligna; truque sujo; péssima ideia}
    \definition{v.}{estragar; destruir; corromper}
  \end{phonetics}
\end{entry}

\begin{entry}{坏人}{7,2}{⼟、⼈}
  \begin{phonetics}{坏人}{huai4 ren2}[][HSK 2]
    \definition[个,种]{s.}{malfeitor; canalha; pessoa má; pessoa de má qualidade; pessoa que faz coisas ruins}
  \end{phonetics}
\end{entry}

\begin{entry}{坏处}{7,5}{⼟、⼡}
  \begin{phonetics}{坏处}{huai4 chu4}[][HSK 2]
    \definition[个]{s.}{dano; prejuízo; desvantagem; fatores prejudiciais a pessoas ou coisas}
  \end{phonetics}
\end{entry}

\begin{entry}{坏蛋}{7,11}{⼟、⾍}
  \begin{phonetics}{坏蛋}{huai4dan4}
    \definition{s.}{bastardo | canalha | pessoa má}
  \end{phonetics}
\end{entry}

\begin{entry}{坐}{7}{⼟}
  \begin{phonetics}{坐}{zuo4}[][HSK 1]
    \definition*{s.}{sobrenome Zuo}
    \definition{adv.}{sem motivo algum; sem causa ou razão; sem motivo aparente}
    \definition{prep.}{porque; pelo fato de que; pela razão de que; pelo motivo de que}
    \definition{s.}{assento; lugar; posição}
    \definition{v.}{sentar; sentar-se; ocupar um lugar; colocar os glúteos sobre um objeto para apoiar o peso corporal | pegar; viajar de; pegar carona | ter as costas voltadas para | colocar (uma panela, chaleira, etc.) no fogo | recuo; coice (de rifles, armas, etc.)  | produzir frutos; formar sementes | ser punido; ser acusado de crime | contrair (ou ter) uma doença; sofrer de uma doença | (um edifício) afundar; ceder}
  \end{phonetics}
\end{entry}

\begin{entry}{坐下}{7,3}{⼟、⼀}
  \begin{phonetics}{坐下}{zuo4 xia5}[][HSK 1]
    \definition{v.}{sentar-se; tomar um assento; passar da posição em pé para a posição sentada}
  \end{phonetics}
\end{entry}

\begin{entry}{坐车}{7,4}{⼟、⾞}
  \begin{phonetics}{坐车}{zuo4che1}
    \definition{v.}{andar de carro, ônibus, trem, etc.}
  \end{phonetics}
\end{entry}

\begin{entry}{坐好}{7,6}{⼟、⼥}
  \begin{phonetics}{坐好}{zuo4hao3}
    \definition{v.}{sentar-se corretamente | sentar direito}
  \end{phonetics}
\end{entry}

\begin{entry}{坐享}{7,8}{⼟、⼇}
  \begin{phonetics}{坐享}{zuo4xiang3}
    \definition{v.}{curtir algo sem levantar um dedo}
  \end{phonetics}
\end{entry}

\begin{entry}{坐垫}{7,9}{⼟、⼟}
  \begin{phonetics}{坐垫}{zuo4dian4}
    \definition[块]{s.}{assento (motocicleta) | almofada}
  \end{phonetics}
\end{entry}

\begin{entry}{坐标}{7,9}{⼟、⽊}
  \begin{phonetics}{坐标}{zuo4biao1}
    \definition{s.}{coordenada (geometria)}
  \end{phonetics}
\end{entry}

\begin{entry}{坑}{7}{⼟}
  \begin{phonetics}{坑}{keng1}
    \definition{s.}{poço | depressão | túnel | buraco no chão}
    \definition{v.}{enganar | trapacear}
  \end{phonetics}
\end{entry}

\begin{entry}{坑人}{7,2}{⼟、⼈}
  \begin{phonetics}{坑人}{keng1ren2}
    \definition{v.+compl.}{trapacear alguém}
  \end{phonetics}
\end{entry}

\begin{entry}{块}{7}{⼟}
  \begin{phonetics}{块}{kuai4}[][HSK 1]
    \definition{clas.}{usado para coisas em pedaços | usado para coisas em pedaços ou em algumas formas de folhas | usado para moedas de prata ou notas de papel equivalentes a 圆}
    \definition{s.}{pedaço; pedaço (de terra); peça; algo que forma um pedaço ou massa}
  \seealsoref{圆}{yuan2}
  \end{phonetics}
\end{entry}

\begin{entry}{坚决}{7,6}{⼟、⼎}
  \begin{phonetics}{坚决}{jian1jue2}[][HSK 3]
    \definition{adj.}{firme; resoluto; (atitude, opinião, ação, etc.) determinado e inabalável}
  \end{phonetics}
\end{entry}

\begin{entry}{坚守}{7,6}{⼟、⼧}
  \begin{phonetics}{坚守}{jian1shou3}
    \definition{v.}{agarrar-se}
  \end{phonetics}
\end{entry}

\begin{entry}{坚固}{7,8}{⼟、⼞}
  \begin{phonetics}{坚固}{jian1gu4}[][HSK 4]
    \definition{adj.}{firme; sólido; robusto; forte; durável; firmemente unidos e inquebráveis}
  \end{phonetics}
\end{entry}

\begin{entry}{坚定}{7,8}{⼟、⼧}
  \begin{phonetics}{坚定}{jian1ding4}[][HSK 5]
    \definition{adj.}{firme; inabalável; inamovível; (posição, opinião, vontade, etc.) firme e estável, inabalável}
    \definition{v.}{fortalecer}
  \end{phonetics}
\end{entry}

\begin{entry}{坚持}{7,9}{⼟、⼿}
  \begin{phonetics}{坚持}{jian1chi2}[][HSK 3]
    \definition{v.}{persistir em; perseverar em; defender; insistir em; manter-se fiel a; aderir a; persistir com determinação e não desistir quando se depara com dificuldades | aderir a; insistir em; não alterar (os princípios, opiniões, pontos de vista originais, etc.)}
  \end{phonetics}
\end{entry}

\begin{entry}{坚强}{7,12}{⼟、⼸}
  \begin{phonetics}{坚强}{jian1qiang2}[][HSK 3]
    \definition{adj.}{forte; firme; convicto; (qualidades humanas, personalidade, determinação, etc.) firme e forte, não vacila diante das dificuldades}
    \definition{v.}{fortalecer; tornar forte; é a qualidade, a determinação, etc., que não vacilam}
  \end{phonetics}
\end{entry}

\begin{entry}{坠}{7}{⼟}
  \begin{phonetics}{坠}{zhui4}
    \definition{v.}{cair | pesar | fazer vergar com o peso}
  \end{phonetics}
\end{entry}

\begin{entry}{坠落}{7,12}{⼟、⾋}
  \begin{phonetics}{坠落}{zhui4luo4}
    \definition{v.}{cair}
  \end{phonetics}
\end{entry}

\begin{entry}{声}{7}{⼠}
  \begin{phonetics}{声}{sheng1}[][HSK 5]
    \definition{clas.}{indica o número de vezes que um som é emitido}
    \definition{s.}{som; voz | reputação | consoante inicial (de uma sílaba chinesa) | tom; tom de voz | informação; notícia}
    \definition{v.}{declarar; anunciar; emitir um som}
  \end{phonetics}
\end{entry}

\begin{entry}{声明}{7,8}{⼠、⽇}
  \begin{phonetics}{声明}{sheng1ming2}[][HSK 3]
    \definition[项,份]{s.}{declaração}
    \definition{v.}{declarar}
    \definition{v.}{declarar; anunciar}
  \end{phonetics}
\end{entry}

\begin{entry}{声音}{7,9}{⼠、⾳}
  \begin{phonetics}{声音}{sheng1yin1}[][HSK 2]
    \definition[个,种]{s.}{som; voz; a percepção auditiva das ondas sonoras}
  \end{phonetics}
\end{entry}

\begin{entry}{壳}{7}{⼠}
  \begin{phonetics}{壳}{ke2}
    \definition{s.}{casca (de ovo, noz, caranguejo, etc.) | caixa | invólucro | alojamento (de uma máquina ou dispositivo)}
  \end{phonetics}
\end{entry}

\begin{entry}{妖}{7}{⼥}
  \begin{phonetics}{妖}{yao1}
    \definition{adj.}{enfeitiçante | encantador}
    \definition{s.}{\emph{goblin} | bruxa | diabo | monstro | fantasma | demônio}
  \end{phonetics}
\end{entry}

\begin{entry}{妙招}{7,8}{⼥、⼿}
  \begin{phonetics}{妙招}{miao4zhao1}
    \definition{adj.}{escorregadio}
    \definition{s.}{movimento inteligente | maneira inteligente de fazer algo}
  \end{phonetics}
\end{entry}

\begin{entry}{宋}{7}{⼧}
  \begin{phonetics}{宋}{song4}
    \definition*{s.}{sobrenome Song}
    \definition{s.}{Dinastia Song (960-1279) | Song das dinastias do sul (420-479)}
  \end{phonetics}
\end{entry}

\begin{entry}{完}{7}{⼧}
  \begin{phonetics}{完}{wan2}[][HSK 2]
    \definition*{s.}{sobrenome Wan}
    \definition{adj.}{inteiro; intacto; completo}
    \definition{v.}{acabar; terminar; completar | pagar | estar terminado; estar pronto para | esgotar; ser usado}
  \end{phonetics}
\end{entry}

\begin{entry}{完了}{7,2}{⼧、⼅}
  \begin{phonetics}{完了}{wan2 le5}[][HSK 5]
    \definition{v.}{acabar; terminar; concluir; chegar ao fim}
  \end{phonetics}
\end{entry}

\begin{entry}{完人}{7,2}{⼧、⼈}
  \begin{phonetics}{完人}{wan2ren2}
    \definition{s.}{pessoa perfeita}
  \end{phonetics}
\end{entry}

\begin{entry}{完全}{7,6}{⼧、⼊}
  \begin{phonetics}{完全}{wan2quan2}[][HSK 2]
    \definition{adj.}{inteiro; completo; não falta nada, está tudo completo}
    \definition{adv.}{completamente; representa tudo}
  \end{phonetics}
\end{entry}

\begin{entry}{完成}{7,6}{⼧、⼽}
  \begin{phonetics}{完成}{wan2cheng2}[][HSK 2]
    \definition{v.}{realizar; completar; terminar; cumprir; levar ao sucesso}
  \end{phonetics}
\end{entry}

\begin{entry}{完毕}{7,6}{⼧、⽐}
  \begin{phonetics}{完毕}{wan2bi4}
    \definition{v.}{completar | terminar | acabar}
  \end{phonetics}
\end{entry}

\begin{entry}{完完全全}{7,7,6,6}{⼧、⼧、⼊、⼊}
  \begin{phonetics}{完完全全}{wan2wan2quan2quan2}
    \definition{adv.}{completamente}
  \end{phonetics}
\end{entry}

\begin{entry}{完备}{7,8}{⼧、⼡}
  \begin{phonetics}{完备}{wan2bei4}
    \definition{adj.}{completo | impecável | perfeito}
    \definition{v.}{não deixar nada a desejar}
  \end{phonetics}
\end{entry}

\begin{entry}{完美}{7,9}{⼧、⽺}
  \begin{phonetics}{完美}{wan2mei3}[][HSK 3]
    \definition{adj.}{perfeito; impecável; consumado}
    \definition{adv.}{perfeitamente}
    \definition{s.}{perfeição}
  \end{phonetics}
\end{entry}

\begin{entry}{完善}{7,12}{⼧、⼝}
  \begin{phonetics}{完善}{wan2shan4}[][HSK 3]
    \definition{adj.}{perfeito; consumado}
    \definition{v.}{refinar; melhorar; tornar perfeito}
  \end{phonetics}
\end{entry}

\begin{entry}{完税}{7,12}{⼧、⽲}
  \begin{phonetics}{完税}{wan2shui4}
    \definition{v.}{pagar imposto}
  \end{phonetics}
\end{entry}

\begin{entry}{完满}{7,13}{⼧、⽔}
  \begin{phonetics}{完满}{wan2man3}
    \definition{adj.}{satisfatório | bem-sucedido}
  \end{phonetics}
\end{entry}

\begin{entry}{完整}{7,16}{⼧、⽁}
  \begin{phonetics}{完整}{wan2zheng3}[][HSK 3]
    \definition{adj.}{intacto; inteiro; completo; integrado}
  \end{phonetics}
\end{entry}

\begin{entry}{寿司}{7,5}{⼨、⼝}
  \begin{phonetics}{寿司}{shou4 si1}[][HSK 5]
    \definition[份]{s.}{\emph{sushi}; iguaria tradicional japonesa}
  \end{phonetics}
\end{entry}

\begin{entry}{尾巴}{7,4}{⼫、⼰}
  \begin{phonetics}{尾巴}{wei3ba5}[][HSK 4]
    \definition{s.}{cauda; projeções na extremidade do corpo de certos animais | parte semelhante a uma cauda; refere-se, em geral, ao final de algo | apêndice; anexo; adepto servil; pessoa que segue ou concorda com outra pessoa | (figura de linguagem) alguém que faz sombra a outro | fim; remanescente; parte restante (ou inacabada)}
  \end{phonetics}
\end{entry}

\begin{entry}{尿}{7}{⼫}
  \begin{phonetics}{尿}{niao4}
    \definition[泡]{s.}{urina}
    \definition{v.}{urinar}
  \end{phonetics}
  \begin{phonetics}{尿}{sui1}
    \definition{s.}{(coloquial) urina}
  \end{phonetics}
\end{entry}

\begin{entry}{局}{7}{⼫}
  \begin{phonetics}{局}{ju2}[][HSK 4]
    \definition{s.}{tabuleiro de xadrez | jogo; turno; \emph{set} | situação; estado das coisas | tolerância; grandeza ou pequenez da mente; grau de tolerância de uma pessoa em relação às outras | reunião de pessoas em festas | ardil; artidício; estratagema; armadilha | parte; porção; parcela | nome de determinadas lojas}
  \end{phonetics}
\end{entry}

\begin{entry}{局长}{7,4}{⼫、⾧}
  \begin{phonetics}{局长}{ju2 zhang3}[][HSK 5]
    \definition[位,个]{s.}{comissário; diretor; principais chefes de gabinete do governo}
  \end{phonetics}
\end{entry}

\begin{entry}{局面}{7,9}{⼫、⾯}
  \begin{phonetics}{局面}{ju2mian4}[][HSK 5]
    \definition[种]{s.}{aspecto; fase; situação; o estado das coisas em um período de tempo, em sua maior parte abstraído | escopo; escala}
  \end{phonetics}
\end{entry}

\begin{entry}{屁股}{7,8}{⼫、⾁}
  \begin{phonetics}{屁股}{pi4gu5}
    \definition{s.}{nádega | quadris}
  \end{phonetics}
\end{entry}

\begin{entry}{屁话}{7,8}{⼫、⾔}
  \begin{phonetics}{屁话}{pi4hua4}
    \definition{s.}{absurdo | tolice | besteira}
  \end{phonetics}
\end{entry}

\begin{entry}{层}{7}{⼫}
  \begin{phonetics}{层}{ceng2}[][HSK 2]
    \definition{clas.}{usado para coisas que se sobrepõem e se acumulam, como andares, camadas e estratos | usado para coisas que podem ser divididas em itens e etapas | usado para coisas que podem ser removidas ou apagadas da superfície de um objeto}
    \definition{s.}{camada; nível; coisas que se sobrepõem | nível; classificação; camada}
    \definition{v.}{sobrepor; empilhar camada sobre camada}
  \end{phonetics}
\end{entry}

\begin{entry}{层次}{7,6}{⼫、⽋}
  \begin{phonetics}{层次}{ceng2ci4}[][HSK 5]
    \definition{s.}{disposição ordenada do conteúdo (de um discurso ou texto) | nível ou estrutura administrativa; distinções entre a mesma coisa devido a diferenças de tamanho, altura, etc. | nível; níveis de afiliação}
  \end{phonetics}
\end{entry}

\begin{entry}{层层}{7,7}{⼫、⼫}
  \begin{phonetics}{层层}{ceng2ceng2}
    \definition{s.}{camada sobre camada}
  \end{phonetics}
\end{entry}

\begin{entry}{希望}{7,11}{⼱、⽉}
  \begin{phonetics}{希望}{xi1wang4}[][HSK 3]
    \definition[个]{s.}{esperança; desejo; expectativa | aquilo em que a esperança é depositada}
    \definition{v.}{ter esperança; desejar; esperar}
  \end{phonetics}
\end{entry}

\begin{entry}{床}{7}{⼴}
  \begin{phonetics}{床}{chuang2}[][HSK 1]
    \definition{clas.}{para colchas, roupas de cama, etc.}
    \definition[张]{s.}{cama; sofá; móveis para dormir | algo com o formato de uma cama}
  \end{phonetics}
\end{entry}

\begin{entry}{库}{7}{⼴}
  \begin{phonetics}{库}{ku4}[][HSK 5]
    \definition{s.}{depósito; tesouraria; armazém; almoxarifado; edifícios e equipamentos para armazenamento de mercadorias | banco de dados}
  \end{phonetics}
\end{entry}

\begin{entry}{应}{7}{⼴}
  \begin{phonetics}{应}{ying1}[][HSK 4,5]
    \definition{v.}{ecoar; responder; responder a; responder às chamadas, saudações, perguntas, etc. de outras pessoas | conceder; cumprir | adequar; adaptar; responder a | lidar com; enfrentar; abordar | tornar-se realidade; ser cumprido}
  \end{phonetics}
\end{entry}

\begin{entry}{应对}{7,5}{⼴、⼨}
  \begin{phonetics}{应对}{ying4dui4}
    \definition{v.}{responder | manusear | lidar}
  \end{phonetics}
\end{entry}

\begin{entry}{应用}{7,5}{⼴、⽤}
  \begin{phonetics}{应用}{ying4yong4}[][HSK 3]
    \definition{adj.}{aplicado (na vida ou na produção); usado diretamente na vida ou na produção}
    \definition{s.}{aplicativo}
    \definition{v.}{usar; aplicar}
  \end{phonetics}
\end{entry}

\begin{entry}{应用程序}{7,5,12,7}{⼴、⽤、⽲、⼴}
  \begin{phonetics}{应用程序}{ying4yong4cheng2xu4}
    \definition{s.}{aplicativo | programa de computador}
  \end{phonetics}
\end{entry}

\begin{entry}{应用程序接口}{7,5,12,7,11,3}{⼴、⽤、⽲、⼴、⼿、⼝}
  \begin{phonetics}{应用程序接口}{ying4yong4cheng2xu4jie1kou3}
    \definition{s.}{API (\emph{application programming interface})}
  \seealsoref{应用程序编程接口}{ying4yong4cheng2xu4bian1cheng2jie1kou3}
  \end{phonetics}
\end{entry}

\begin{entry*}{应用程序编程接口}{7,5,12,7,12,12,11,3}{⼴、⽤、⽲、⼴、⽷、⽲、⼿、⼝}
  \begin{phonetics}{应用程序编程接口}{ying4yong4cheng2xu4bian1cheng2jie1kou3}
    \definition{s.}{API (\emph{application programming interface})}
  \seealsoref{应用程序接口}{ying4yong4cheng2xu4jie1kou3}
  \end{phonetics}
\end{entry*}

\begin{entry}{应当}{7,6}{⼴、⼹}
  \begin{phonetics}{应当}{ying1 dang1}[][HSK 3]
    \definition{v.}{dever}
  \end{phonetics}
\end{entry}

\begin{entry}{应该}{7,8}{⼴、⾔}
  \begin{phonetics}{应该}{ying1gai1}[][HSK 2]
    \definition{v.}{deveria; deve ser assim | deveria; acho que deve ser esse o caso}
  \end{phonetics}
\end{entry}

\begin{entry}{弄}{7}{⼶}
  \begin{phonetics}{弄}{long4}
    \definition{s.}{rua estreita; beco; viela; travessa}
  \end{phonetics}
  \begin{phonetics}{弄}{nong4}[][HSK 2]
    \definition{v.}{fazer, realizar; tratar; organizar | obter; buscar; tentar conseguir; encontrar uma maneira de conseguir | brincar com; enganar | pregar uma peça; brincar; manipular | mexer com; perturbar}
  \end{phonetics}
\end{entry}

\begin{entry}{弟}{7}{⼸}
  \begin{phonetics}{弟}{di4}[][HSK 1]
    \definition*{s.}{sobrenome Di}
    \definition[个]{s.}{irmão mais novo | (entre amigos homens) eu | geralmente se refere a colegas do sexo masculino mais jovens na família ou entre parentes | forma humilde que os amigos usam para se referir uns aos outros, usada principalmente em correspondência}
  \end{phonetics}
\end{entry}

\begin{entry}{弟弟}{7,7}{⼸、⼸}
  \begin{phonetics}{弟弟}{di4 di5}[][HSK 1]
    \definition[个,位]{s.}{irmão mais novo | primo}
  \end{phonetics}
\end{entry}

\begin{entry}{弟妹}{7,8}{⼸、⼥}
  \begin{phonetics}{弟妹}{di4mei4}
    \definition{s.}{esposa do irmão mais novo}
  \end{phonetics}
\end{entry}

\begin{entry}{张}{7}{⼸}
  \begin{phonetics}{张}{zhang1}[][HSK 3]
    \definition*{s.}{sobrenome Zhang}
    \definition*{s.}{Zhang, uma das mansões lunares}
    \definition{adj.}{nervoso; tenso}
    \definition{clas.}{para papel, couro, etc. | para camas, mesas, etc. | para a boca e o rosto | para arcos}
    \definition{s.}{folha de papel}
    \definition{v.}{consertar (uma corda de arco); encordoar (um instrumento musical ou um arco) | abrir; espalhar; esticar | expor; exibir |expandir; estender | ampliar; exagerar | olhar | dar rédea solta a; satisfazer | iniciar um negócio; abrir uma loja | colocar em bom uso; dar liberdade para | pegar com uma rede; montar armadilhas para capturar pássaros e animais}
  \end{phonetics}
\end{entry}

\begin{entry}{张三}{7,3}{⼸、⼀}
  \begin{phonetics}{张三}{zhang1san1}
    \definition*{s.}{Zhang San | Zé Ninguém | nome para uma pessoa não especificada, 1 de 3}
  \seealsoref{李四}{li3si4}
  \seealsoref{王五}{wang2wu3}
  \end{phonetics}
\end{entry}

\begin{entry}{张狂}{7,7}{⼸、⽝}
  \begin{phonetics}{张狂}{zhang1kuang2}
    \definition{adj.}{impetuoso | frenético | insolente}
  \end{phonetics}
\end{entry}

\begin{entry}{形式}{7,6}{⼺、⼷}
  \begin{phonetics}{形式}{xing2shi4}[][HSK 3]
    \definition[种,个]{s.}{forma; formato; modalidade | aparência, estrutura ou estado de algo}
  \end{phonetics}
\end{entry}

\begin{entry}{形成}{7,6}{⼺、⼽}
  \begin{phonetics}{形成}{xing2cheng2}[][HSK 3]
    \definition{v.}{moldar; formar; tomar forma | tornar-se algo ou algo através do desenvolvimento e da mudança}
  \end{phonetics}
\end{entry}

\begin{entry}{形而上学}{7,6,3,8}{⼺、⽽、⼀、⼦}
  \begin{phonetics}{形而上学}{xing2'er2shang4xue2}
    \definition{s.}{metafísica}
  \end{phonetics}
\end{entry}

\begin{entry}{形状}{7,7}{⼺、⽝}
  \begin{phonetics}{形状}{xing2zhuang4}[][HSK 3]
    \definition[个]{s.}{forma; aparência | a aparência de um objeto ou figura formada pela combinação de superfícies ou linhas externas}
  \end{phonetics}
\end{entry}

\begin{entry}{形势}{7,8}{⼺、⼒}
  \begin{phonetics}{形势}{xing2shi4}[][HSK 4]
    \definition[个]{s.}{terreno; características topográficas; situação geográfica, principalmente de uma perspectiva militar | situação; circunstâncias; a situação geral, a tendência de como as coisas estão se desenvolvendo e mudando | geralmente não é usado em situações pessoais}
  \end{phonetics}
\end{entry}

\begin{entry}{形态}{7,8}{⼺、⼼}
  \begin{phonetics}{形态}{xing2tai4}[][HSK 5]
    \definition{s.}{forma; forma como as coisas se apresentam | forma; padrão; postura | morfologia; forma; (gramática) refere-se às formas internas de mudança das palavras, incluindo a formação de palavras e as mudanças morfológicas}
  \end{phonetics}
\end{entry}

\begin{entry}{形容}{7,10}{⼺、⼧}
  \begin{phonetics}{形容}{xing2rong2}[][HSK 4]
    \definition{s.}{aparência; semblante}
    \definition{v.}{descrever}
  \end{phonetics}
\end{entry}

\begin{entry}{形象}{7,11}{⼺、⾗}
  \begin{phonetics}{形象}{xing2xiang4}[][HSK 3]
    \definition{adj.}{vívido}
    \definition[个]{s.}{imagem; forma; figura | uma forma ou gesto específico que pode despertar os pensamentos ou emoções das pessoas | imagem literária; imagem artística | pessoas ou coisas com características diferentes criadas na literatura, no cinema e em outras artes}
  \end{phonetics}
\end{entry}

\begin{entry}{彻底}{7,8}{⼻、⼴}
  \begin{phonetics}{彻底}{che4di3}[][HSK 4]
    \definition{adj.}{minucioso; completo; exaustivo; profundo e completo; nada é deixado de fora}
  \end{phonetics}
\end{entry}

\begin{entry}{忍}{7}{⼼}
  \begin{phonetics}{忍}{ren3}[][HSK 5]
    \definition{v.}{suportar; aguentar; tolerar; aturar | ter coragem para; ser insensível o suficiente para; ser capaz de endurecer o coração e fazer coisas que não se devem fazer por uma questão de razão}
  \end{phonetics}
\end{entry}

\begin{entry}{忍不住}{7,4,7}{⼼、⼀、⼈}
  \begin{phonetics}{忍不住}{ren3bu5zhu4}[][HSK 5]
    \definition{v.}{incapaz de suportar; não conseguir evitar fazer algo; não conseguir se controlar}
  \end{phonetics}
\end{entry}

\begin{entry}{忍受}{7,8}{⼼、⼜}
  \begin{phonetics}{忍受}{ren3shou4}[][HSK 5]
    \definition{v.}{suportar; sofrer; aguentar; tolerar; suportar com dificuldade o sofrimento, as dificuldades e as adversidades da vida}
  \end{phonetics}
\end{entry}

\begin{entry}{忍耐}{7,9}{⼼、⽽}
  \begin{phonetics}{忍耐}{ren3nai4}
    \definition{s.}{paciência | resistência}
    \definition{v.}{suportar | resistir | exercer paciência}
  \end{phonetics}
\end{entry}

\begin{entry}{志愿}{7,14}{⼼、⽕}
  \begin{phonetics}{志愿}{zhi4 yuan4}[][HSK 3]
    \definition{s.}{desejo; ideal; aspiração; meta que se espera alcançar}
    \definition{v.}{ser voluntário; tomar a iniciativa e esteja disposto a fazer um trabalho que não gere renda ou que tenha renda muito baixa, mas que possa ajudar outras pessoas}
  \end{phonetics}
\end{entry}

\begin{entry}{志愿者}{7,14,8}{⼼、⽕、⽼}
  \begin{phonetics}{志愿者}{zhi4yuan4zhe3}[][HSK 3]
    \definition{s.}{voluntário; pessoa que se voluntaria para servir em atividades de assistência social, eventos de grande porte, conferências, etc.}
  \end{phonetics}
\end{entry}

\begin{entry}{忘}{7}{⼼}
  \begin{phonetics}{忘}{wang4}[][HSK 1]
    \definition{v.}{esquecer | ignorar; negligenciar}
  \end{phonetics}
\end{entry}

\begin{entry}{忘本}{7,5}{⼼、⽊}
  \begin{phonetics}{忘本}{wang4ben3}
    \definition{v.}{esquecer as próprias raízes}
  \end{phonetics}
\end{entry}

\begin{entry}{忘记}{7,5}{⼼、⾔}
  \begin{phonetics}{忘记}{wang4ji4}[][HSK 1]
    \definition{v.}{esquecer | ignorar; negligenciar | sair da memória de alguém; não ser lembrado | descartar da mente; ignorar}
  \end{phonetics}
\end{entry}

\begin{entry}{忘却}{7,7}{⼼、⼙}
  \begin{phonetics}{忘却}{wang4que4}
    \definition{v.}{esquecer}
  \end{phonetics}
\end{entry}

\begin{entry}{忘怀}{7,7}{⼼、⼼}
  \begin{phonetics}{忘怀}{wang4huai2}
    \definition{v.}{esquecer}
  \end{phonetics}
\end{entry}

\begin{entry}{忘恩}{7,10}{⼼、⼼}
  \begin{phonetics}{忘恩}{wang4'en1}
    \definition{v.}{ser ingrato}
  \end{phonetics}
\end{entry}

\begin{entry}{忘掉}{7,11}{⼼、⼿}
  \begin{phonetics}{忘掉}{wang4diao4}
    \definition{v.}{esquecer}
  \end{phonetics}
\end{entry}

\begin{entry}{忘餐}{7,16}{⼼、⾷}
  \begin{phonetics}{忘餐}{wang4can1}
    \definition{v.}{esquecer as refeições}
  \end{phonetics}
\end{entry}

\begin{entry}{忧郁}{7,8}{⼼、⾢}
  \begin{phonetics}{忧郁}{you1yu4}
    \definition{adj.}{deprimido | melancólico | desanimado}
    \definition{s.}{depressão | melancolia}
  \end{phonetics}
\end{entry}

\begin{entry}{快}{7}{⼼}
  \begin{phonetics}{快}{kuai4}[][HSK 1]
    \definition*{s.}{sobrenome Kuai}
    \definition{adj.}{rápido; veloz (oposto a 慢) | apressado | perspicaz; ágil; inteligente; de ​​mente rápida | (de uma faca, espada, etc.) afiado (oposto a 钝) | direto; franco; sem rodeios | satisfeito; feliz; gratificado | rápido; veloz; alta velocidade; tempo de execução curto | satisfeito; feliz; contente | engenhoso; ágil | afiado; facas, tesouras, machados e outros objetos afiados | sincero}
    \definition{adv.}{em breve; antes de muito tempo; estar prestes a | rapidamente}
    \definition{s.}{policial; polícia | (antigo) oficial encarregado de efetuar prisões}
  \seealsoref{钝}{dun4}
  \seealsoref{慢}{man4}
  \end{phonetics}
\end{entry}

\begin{entry}{快乐}{7,5}{⼼、⼃}
  \begin{phonetics}{快乐}{kuai4le4}[][HSK 2]
    \definition{adj.}{feliz; alegre; animado; prazeiroso}
    \definition{s.}{felicidade | alegria}
  \end{phonetics}
\end{entry}

\begin{entry}{快活}{7,9}{⼼、⽔}
  \begin{phonetics}{快活}{kuai4huo5}[][HSK 5]
    \definition{adj.}{feliz; alegre; contente; animado}
  \end{phonetics}
\end{entry}

\begin{entry}{快点儿}{7,9,2}{⼼、⽕、⼉}
  \begin{phonetics}{快点儿}{kuai4 dian3r5}[][HSK 2]
    \definition{v.}{apressar-se}
  \end{phonetics}
\end{entry}

\begin{entry}{快要}{7,9}{⼼、⾑}
  \begin{phonetics}{快要}{kuai4 yao4}[][HSK 2]
    \definition{adv.}{estar prestes a; estar indo para; estar à beira de; em breve; em pouco tempo; indica que a situação está prestes a ocorrer}
  \end{phonetics}
\end{entry}

\begin{entry}{快递}{7,10}{⼼、⾡}
  \begin{phonetics}{快递}{kuai4 di4}[][HSK 4]
    \definition[个]{s.}{correio rápido; entrega expressa; entrega rápida}
    \definition{v.}{entregar (serviço de entrega rápida por transportadoras especializadas)}
  \end{phonetics}
\end{entry}

\begin{entry}{快速}{7,10}{⼼、⾡}
  \begin{phonetics}{快速}{kuai4 su4}[][HSK 3]
    \definition{adj.}{rápido; veloz; de alta velocidade; descreve o tempo curto gasto para caminhar, fazer algo, etc.}
  \end{phonetics}
\end{entry}

\begin{entry}{快餐}{7,16}{⼼、⾷}
  \begin{phonetics}{快餐}{kuai4 can1}[][HSK 2]
    \definition[份,顿]{s.}{pedido (comida) rápido; \emph{fast food}; refere-se a refeições simples preparadas com antecedência e que podem ser servidas rapidamente}
  \end{phonetics}
\end{entry}

\begin{entry}{怀旧}{7,5}{⼼、⽇}
  \begin{phonetics}{怀旧}{huai2jiu4}
    \definition{s.}{nostalgia}
    \definition{v.}{sentir-se nostálgico}
  \end{phonetics}
\end{entry}

\begin{entry}{怀念}{7,8}{⼼、⼼}
  \begin{phonetics}{怀念}{huai2nian4}[][HSK 4]
    \definition{v.}{pensar em; valorizar a memória de}
  \end{phonetics}
\end{entry}

\begin{entry}{怀疑}{7,14}{⼼、⽦}
  \begin{phonetics}{怀疑}{huai2yi2}[][HSK 4]
    \definition{v.}{duvidar; suspeitar | supor}
  \end{phonetics}
\end{entry}

\begin{entry}{我}{7}{⼽}
  \begin{phonetics}{我}{wo3}[][HSK 1]
    \definition{pron.}{eu; mim | um; qualquer um; usado para contrastar 他 e 我; refere-se a muitas pessoas em geral}
  \seealsoref{他}{ta1}
  \end{phonetics}
\end{entry}

\begin{entry}{我们}{7,5}{⼽、⼈}
  \begin{phonetics}{我们}{wo3men5}[][HSK 1]
    \definition{pron.}{nós; nos}
  \end{phonetics}
\end{entry}

\begin{entry}{我们的}{7,5,8}{⼽、⼈、⽩}
  \begin{phonetics}{我们的}{wo3men5 de5}
    \definition{pron.}{nosso, nossos}
  \end{phonetics}
\end{entry}

\begin{entry}{我去}{7,5}{⼽、⼛}
  \begin{phonetics}{我去}{wo3qu4}
    \definition{interj.}{(gíria) O que\dots!! | Oh meu Deus! | Isso é insano!}
  \end{phonetics}
\end{entry}

\begin{entry}{我的}{7,8}{⼽、⽩}
  \begin{phonetics}{我的}{wo3 de5}
    \definition{pron.}{meu, meus}
  \end{phonetics}
\end{entry}

\begin{entry}{戒}{7}{⼽}
  \begin{phonetics}{戒}{jie4}[][HSK 5]
    \definition[个]{s.}{advertência; exortação | disciplina monástica budista; preceitos budistas | anel (dedo)}
    \definition{v.}{proteger-se contra; estar preparado; estar atento | advertir; exortar; admoestar | abandonar; parar; desistir; desistir (de um hábito ruim)}
  \end{phonetics}
\end{entry}

\begin{entry}{扮演}{7,14}{⼿、⽔}
  \begin{phonetics}{扮演}{ban4yan3}[][HSK 5]
    \definition{v.}{desempenhar o papel de; ter um papel (em uma peça, etc.); atuar}
  \end{phonetics}
\end{entry}

\begin{entry}{扶}{7}{⼿}
  \begin{phonetics}{扶}{fu2}[][HSK 5]
    \definition*{s.}{sobrenome Fu}
    \definition{v.}{segurar; apoiar com a mão; segurar algo com o apoio das mãos para que ninguém, objeto ou pessoa caia | dar apoio a; ajudar uma pessoa deitada ou caída a se levantar com as mãos; endireitar um objeto caído com as mãos | ajudar; tirar de baixo}
  \end{phonetics}
\end{entry}

\begin{entry}{扶梯}{7,11}{⼿、⽊}
  \begin{phonetics}{扶梯}{fu2ti1}
    \definition{s.}{escada rolante}
  \end{phonetics}
\end{entry}

\begin{entry}{批}{7}{⼿}
  \begin{phonetics}{批}{pi1}[][HSK 4]
    \definition{adj.}{(compra ou venda) atacado; a granel; em grandes quantidades}
    \definition{clas.}{para mercadorias a granel, grande número de pessoas}
    \definition{s.}{fibras de algodão, linho, etc., prontas para serem estiradas e torcidas | anotação; comentário}
    \definition{v.}{escrever comentários ou críticas sobre documentos subordinados, textos de outras pessoas, tarefas etc. | refutar; criticar | dar um tapa}
  \end{phonetics}
\end{entry}

\begin{entry}{批评}{7,7}{⼿、⾔}
  \begin{phonetics}{批评}{pi1ping2}[][HSK 3]
    \definition{s.}{crítica}
    \definition{v.}{criticar; comentar sobre}
  \end{phonetics}
\end{entry}

\begin{entry}{批准}{7,10}{⼿、⼎}
  \begin{phonetics}{批准}{pi1zhun3}[][HSK 3]
    \definition{v.}{aprovar}
  \end{phonetics}
\end{entry}

\begin{entry}{找}{7}{⼿}
  \begin{phonetics}{找}{zhao3}[][HSK 1]
    \definition{v.}{procurar; tentar encontrar; buscar | querer ver; visitar; abordar; solicitar | dar troco | descobrir; esforçar-se para ver ou obter a pessoa ou coisa desejada | examinar; investigar; completar as partes que faltam | causar intencionalmente (um resultado indesejável, negativo)}
  \end{phonetics}
\end{entry}

\begin{entry}{找见}{7,4}{⼿、⾒}
  \begin{phonetics}{找见}{zhao3jian4}
    \definition{v.}{encontrar (algo que está procurando)}
  \end{phonetics}
\end{entry}

\begin{entry}{找出}{7,5}{⼿、⼐}
  \begin{phonetics}{找出}{zhao3 chu1}[][HSK 2]
    \definition{v.}{encontrar | procurar}
  \end{phonetics}
\end{entry}

\begin{entry}{找回}{7,6}{⼿、⼞}
  \begin{phonetics}{找回}{zhao3hui2}
    \definition{v.}{recuperar algo}
  \end{phonetics}
\end{entry}

\begin{entry}{找寻}{7,6}{⼿、⼨}
  \begin{phonetics}{找寻}{zhao3xun2}
    \definition{v.}{encontrar falhas | procurar | buscar}
  \end{phonetics}
\end{entry}

\begin{entry}{找事}{7,8}{⼿、⼅}
  \begin{phonetics}{找事}{zhao3shi4}
    \definition{v.}{procurar emprego | começar uma briga}
  \end{phonetics}
\end{entry}

\begin{entry}{找到}{7,8}{⼿、⼑}
  \begin{phonetics}{找到}{zhao3 dao4}[][HSK 1]
    \definition{v.}{encontrar; procurar; achar; encontar através de pesquisa, exploração, etc.;  ver ou encontrar coisas ou padrões que os antepassados não viram}
  \end{phonetics}
\end{entry}

\begin{entry}{找钱}{7,10}{⼿、⾦}
  \begin{phonetics}{找钱}{zhao3qian2}
    \definition{v.}{dar troco}
  \end{phonetics}
\end{entry}

\begin{entry}{找着}{7,11}{⼿、⽬}
  \begin{phonetics}{找着}{zhao3zhao2}
    \definition{v.}{encontrar}
  \end{phonetics}
\end{entry}

\begin{entry}{找遍}{7,12}{⼿、⾡}
  \begin{phonetics}{找遍}{zhao3bian4}
    \definition{v.}{pentear | pesquisar em todos os lugares}
  \end{phonetics}
\end{entry}

\begin{entry}{找零}{7,13}{⼿、⾬}
  \begin{phonetics}{找零}{zhao3ling2}
    \definition{v.}{trocar dinheiro | dar troco}
  \end{phonetics}
\end{entry}

\begin{entry}{找辙}{7,16}{⼿、⾞}
  \begin{phonetics}{找辙}{zhao3zhe2}
    \definition{v.}{procurar um pretexto}
  \end{phonetics}
\end{entry}

\begin{entry}{技巧}{7,5}{⼿、⼯}
  \begin{phonetics}{技巧}{ji4qiao3}[][HSK 4]
    \definition{s.}{habilidade; técnica; habilidades engenhosas expressas em artes, artesanato, esportes, etc.}
  \end{phonetics}
\end{entry}

\begin{entry}{技术}{7,5}{⼿、⽊}
  \begin{phonetics}{技术}{ji4shu4}[][HSK 3]
    \definition[种,门,项]{s.}{habilidade; técnica; tecnologia; a experiência e o conhecimento acumulados pelo ser humano no processo de utilização e transformação da natureza, e refletidos no trabalho produtivo, também se referem, de maneira geral, a outras habilidades operacionais}
  \end{phonetics}
\end{entry}

\begin{entry}{技俩}{7,9}{⼿、⼈}
  \begin{phonetics}{技俩}{ji4liang3}
    \definition{s.}{truque | estratagema | ardil | esquema | estratégia | tática}
  \end{phonetics}
\end{entry}

\begin{entry}{技能}{7,10}{⼿、⾁}
  \begin{phonetics}{技能}{ji4 neng2}[][HSK 5]
    \definition[种,项]{s.}{habilidade técnica; domínio de uma habilidade ou técnica; capacidade de adquirir e aplicar conhecimento}
  \end{phonetics}
\end{entry}

\begin{entry}{抄}{7}{⼿}
  \begin{phonetics}{抄}{chao1}[][HSK 4]
    \definition*{s.}{sobrenome Chao}
    \definition{v.}{copiar; transcrever | plagiar | revistar e confiscar; fazer uma batida | pegar um atalho | dobrar (os braços) | agarrar; pegar}
  \end{phonetics}
\end{entry}

\begin{entry}{抄写}{7,5}{⼿、⼍}
  \begin{phonetics}{抄写}{chao1 xie3}[][HSK 4]
    \definition{v.}{copiar; transcrever}
  \end{phonetics}
\end{entry}

\begin{entry}{把}{7}{⼿}
  \begin{phonetics}{把}{ba3}[][HSK 3]
    \definition{adj.}{referindo-se à relação de irmandade}
    \definition{clas.}{usado antes de objetos com alças ou coisas para segurar | um punhado de; a quantidade que se pode pegar com uma mão | usado antes de coisas abstratas | usado em coisas feitas com as mãos | número de ações, coisas}
    \definition{part.}{adicionado após quantificadores como 百, 千, 万 e 里, 斤, 个, indica que a quantidade é próxima dessa unidade (não pode ser adicionado outro quantificador antes)}
    \definition{prep.}{fazer uma determinada alteração em um objeto; causar uma determinada mudança em um objeto | fazer com que os outros façam/sintam algo}
    \definition{s.}{alça; punho; a parte que se segura | feixe; molho; algo que se segura com as mãos ou se amarra em pequenos feixes}
    \definition{v.}{agarrar; segurar | segurar (um bebê enquanto ele urina) | controlar; dominar; monopolizar | encostar-se; apoiar-se | vigiar (locais importantes); observar; guardar | dar | usar algo como; considerar como; tratar como; conter o significado de 拿 | acorrentar; trancar}
  \seealsoref{百}{bai3}
  \seealsoref{个}{ge4}
  \seealsoref{斤}{jin1}
  \seealsoref{里}{li3}
  \seealsoref{拿}{na2}
  \seealsoref{千}{qian1}
  \seealsoref{万}{wan4}
  \end{phonetics}
  \begin{phonetics}{把}{ba4}
    \definition{s.}{punho; alça; empunhadura; parte do utensílio que é fácil de segurar com a mão |haste (de uma folha, flor ou fruto) | motivo de ridículo; alvo; comportamentos e declarações que servem de assunto para piadas}
  \end{phonetics}
\end{entry}

\begin{entry}{把风}{7,4}{⼿、⾵}
  \begin{phonetics}{把风}{ba3feng1}
    \definition{v.}{estar atento | vigiar (durante uma atividade clandestina)}
  \end{phonetics}
\end{entry}

\begin{entry}{把关}{7,6}{⼿、⼋}
  \begin{phonetics}{把关}{ba3guan1}
    \definition{v.}{verificar estritamente | examinar cuidadosamente para ver se algo é feito de acordo com um padrão fixo | fazer a verificação final | guardar uma passagem, fronteira}
  \end{phonetics}
\end{entry}

\begin{entry}{把守}{7,6}{⼿、⼧}
  \begin{phonetics}{把守}{ba3shou3}
    \definition{v.}{vigiar | guardar}
  \end{phonetics}
\end{entry}

\begin{entry}{把式}{7,6}{⼿、⼷}
  \begin{phonetics}{把式}{ba3shi4}
    \definition{s.}{pessoa qualificada em um comércio}
  \end{phonetics}
\end{entry}

\begin{entry}{把戏}{7,6}{⼿、⼽}
  \begin{phonetics}{把戏}{ba3xi4}
    \definition{s.}{acrobacia | malabarismo | truque barato}
  \end{phonetics}
\end{entry}

\begin{entry}{把玩}{7,8}{⼿、⽟}
  \begin{phonetics}{把玩}{ba3wan2}
    \definition{v.}{brincar com | mexer com}
  \end{phonetics}
\end{entry}

\begin{entry}{把持}{7,9}{⼿、⼿}
  \begin{phonetics}{把持}{ba3chi2}
    \definition{v.}{controlar | dominar | monopolizar}
  \end{phonetics}
\end{entry}

\begin{entry}{把柄}{7,9}{⼿、⽊}
  \begin{phonetics}{把柄}{ba3bing3}
    \definition{s.}{(figurativo) informações que podem ser usadas contra alguém}
  \end{phonetics}
\end{entry}

\begin{entry}{把脉}{7,9}{⼿、⾁}
  \begin{phonetics}{把脉}{ba3mai4}
    \definition{v.}{sentir ou tomar o pulso de alguém}
  \end{phonetics}
\end{entry}

\begin{entry}{把握}{7,12}{⼿、⼿}
  \begin{phonetics}{把握}{ba3wo4}[][HSK 3]
    \definition[的]{s.}{seguro; garantia; certeza; confiabilidade do sucesso}
    \definition{v.}{agarrar; segurar; apreender |  (algo abstrato) agarrar; segurar}
  \end{phonetics}
\end{entry}

\begin{entry}{把稳}{7,14}{⼿、⽲}
  \begin{phonetics}{把稳}{ba3wen3}
    \definition{adj.}{confiável}
  \end{phonetics}
\end{entry}

\begin{entry}{抓}{7}{⼿}
  \begin{phonetics}{抓}{zhua1}[][HSK 3]
    \definition{v.}{agarrar | arranhar | capturar | compreender; conhecer a chave ou a chave das coisas ou problemas | focar em algo; fortalecer o poder de fazer (algo) ou administrar (algum aspecto) | atrair a atenção de alguém}
  \end{phonetics}
\end{entry}

\begin{entry}{抓住}{7,7}{⼿、⼈}
  \begin{phonetics}{抓住}{zhua1 zhu4}[][HSK 3]
    \definition{v.}{apanhar; prender; capturar (uma pessoa ou animal) e ter sucesso | segurar; agarrar; segurar algo e deixá-lo imóvel}
  \end{phonetics}
\end{entry}

\begin{entry}{抓紧}{7,10}{⼿、⽷}
  \begin{phonetics}{抓紧}{zhua1jin3}[][HSK 4]
    \definition{v.}{agarrar com firmeza; segurar firme e não soltar | prestar muita atenção a}
  \end{phonetics}
\end{entry}

\begin{entry}{投}{7}{⼿}
  \begin{phonetics}{投}{tou2}[][HSK 4]
    \definition*{s.}{sobrenome Tou}
    \definition{pron.}{para; indica tempo, equivalente a 到, 临 | para; em direção a; indica orientação, direção, equivalente a 朝 ou 向}
    \definition{s.}{um jogo durante uma festa em que o vencedor era decidido pelo número de flechas lançadas em um pote distante | jogo de dados}
    \definition{v.}{lançar; arremessar; atirar | deixar cair; colocar em; lançar | mergulhar em; lançar-se em; pular dentro | lançar; projetar; sombrear | entregar; postar; enviar | ir até; ir para; buscar; juntar-se | sentir-se atraído por; adaptar-se a; concordar com; atender a}
  \seealsoref{朝}{chao2}
  \seealsoref{到}{dao4}
  \seealsoref{临}{lin2}
  \seealsoref{向}{xiang4}
  \end{phonetics}
\end{entry}

\begin{entry}{投入}{7,2}{⼿、⼊}
  \begin{phonetics}{投入}{tou2ru4}[][HSK 4]
    \definition{adj.}{sisudo; dedicado; devotado; absorto}
    \definition{s.}{investimento; insumo; refere-se à aplicação de recursos}
    \definition{v.}{lançar em; colocar em; jogar em; por em | entrar em uma situação; participar de | aplicar; investir; colocar fundos em}
  \end{phonetics}
\end{entry}

\begin{entry}{投诉}{7,7}{⼿、⾔}
  \begin{phonetics}{投诉}{tou2su4}[][HSK 4]
    \definition{v.}{reclamar; queixar-se; reclamar às autoridades ou pessoas envolvidas}
  \end{phonetics}
\end{entry}

\begin{entry}{投资}{7,10}{⼿、⾙}
  \begin{phonetics}{投资}{tou2zi1}[][HSK 4]
    \definition[次]{s.}{investimento}
    \definition{v.}{investir; aplicar dinheiro; investir dinheiro em negócios}
  \end{phonetics}
\end{entry}

\begin{entry}{投资人}{7,10,2}{⼿、⾙、⼈}
  \begin{phonetics}{投资人}{tou2zi1ren2}
    \definition{s.}{investidor}
  \seealsoref{投资家}{tou2zi1jia1}
  \seealsoref{投资者}{tou2zi1zhe3}
  \end{phonetics}
\end{entry}

\begin{entry}{投资风险}{7,10,4,9}{⼿、⾙、⾵、⾩}
  \begin{phonetics}{投资风险}{tou2zi1feng1xian3}
    \definition{s.}{risco de investimento}
  \end{phonetics}
\end{entry}

\begin{entry}{投资回报率}{7,10,6,7,11}{⼿、⾙、⼞、⼿、⽞}
  \begin{phonetics}{投资回报率}{tou2zi1hui2bao4lv4}
    \definition{s.}{retorno sobre o investimento (ROI)}
  \end{phonetics}
\end{entry}

\begin{entry}{投资者}{7,10,8}{⼿、⾙、⽼}
  \begin{phonetics}{投资者}{tou2zi1zhe3}
    \definition{s.}{investidor}
  \seealsoref{投资家}{tou2zi1jia1}
  \seealsoref{投资人}{tou2zi1ren2}
  \end{phonetics}
\end{entry}

\begin{entry}{投资家}{7,10,10}{⼿、⾙、⼧}
  \begin{phonetics}{投资家}{tou2zi1jia1}
    \definition{s.}{investidor}
  \seealsoref{投资人}{tou2zi1ren2}
  \seealsoref{投资者}{tou2zi1zhe3}
  \end{phonetics}
\end{entry}

\begin{entry}{投递}{7,10}{⼿、⾡}
  \begin{phonetics}{投递}{tou2di4}
    \definition{v.}{despachar | enviar}
  \end{phonetics}
\end{entry}

\begin{entry}{投票}{7,11}{⼿、⽰}
  \begin{phonetics}{投票}{tou2piao4}
    \definition{v.+compl.}{votar | depositar um voto}
  \end{phonetics}
\end{entry}

\begin{entry}{折}{7}{⼿}
  \begin{phonetics}{折}{she2}
    \definition{v.}{estalar; quebrar | perder dinheiro em um negócio}
  \end{phonetics}
  \begin{phonetics}{折}{zhe1}
    \definition{v.}{rolar; virar | despejar algo de um recipiente em outro; ficar despejando algo entre dois recipientes}
  \end{phonetics}
  \begin{phonetics}{折}{zhe2}[][HSK 4]
    \definition*{s.}{sobrenome Zhe}
    \definition{clas.}{uma passagem em um roteiro de ópera miscelânea de Yuan, aproximadamente equivalente a uma cena ou ato em uma ópera moderna}
    \definition[张,个,些]{s.}{fratura; quebra | abatimento; desconto | traços dos caracteres chineses que têm o formato de "𠃍" e "乚", etc. | pasta; livreto; \emph{folder}}
    \definition{v.}{estalar; quebrar; fazer quebrar | perder; sofrer a perda de | voltar para trás; mudar de direção; retornar |ser convencido; estar cheio de admiração | equivaler a; converter em | dobrar}
  \end{phonetics}
\end{entry}

\begin{entry}{折转}{7,8}{⼿、⾞}
  \begin{phonetics}{折转}{zhe2zhuan3}
    \definition{s.}{reflexo (ângulo)}
    \definition{v.}{voltar atrás}
  \end{phonetics}
\end{entry}

\begin{entry}{抢}{7}{⼿}
  \begin{phonetics}{抢}{qiang1}
    \definition{prep.}{contra; direção relativa inversa}
    \definition{v.}{bater; tocar}
  \end{phonetics}
  \begin{phonetics}{抢}{qiang3}[][HSK 5]
    \definition{v.}{roubar; saquear | agarrar; apanhar; arrebatar | disputar; lutar por; ser o primeiro; competir para ser o primeiro | correr; apressar-se; fazer uma incursão | raspar; arranhar; raspar ou esfregar uma camada da superfície de um objeto}
  \end{phonetics}
\end{entry}

\begin{entry}{抢掠}{7,11}{⼿、⼿}
  \begin{phonetics}{抢掠}{qiang3lve4}
    \definition{s.}{saque | pilhagem}
    \definition{v.}{saquear | pilhar}
  \end{phonetics}
\end{entry}

\begin{entry}{抢救}{7,11}{⼿、⽁}
  \begin{phonetics}{抢救}{qiang3jiu4}[][HSK 5]
    \definition{v.}{salvar; resgatar; prestar de socorro ou assistência rápidos em situações de emergência | salvar; tomar medidas rápidas para evitar ou minimizar perdas iminentes.}
  \end{phonetics}
\end{entry}

\begin{entry}{护士}{7,3}{⼿、⼠}
  \begin{phonetics}{护士}{hu4shi5}[][HSK 4]
    \definition[名,位]{s.}{enfermeiro; pessoas especializadas em enfermagem em hospitais ou instituições epidemiológicas}
  \end{phonetics}
\end{entry}

\begin{entry}{护照}{7,13}{⼿、⽕}
  \begin{phonetics}{护照}{hu4zhao4}[][HSK 2]
    \definition[本,个]{s.}{passaporte; documento emitido pela autoridade competente do país para comprovar a nacionalidade e a identidade dos cidadãos que viajam para o exterior}
  \end{phonetics}
\end{entry}

\begin{entry}{报}{7}{⼿}
  \begin{phonetics}{报}{bao4}[][HSK 3]
    \definition[份,张]{s.}{jornal | revista; periódico; referência a uma publicação específica | relatório; boletim; algo que transmite alguma informação | telegrama | julgamento; retribuição}
    \definition{v.}{relatar; declarar; anunciar; informar; comunicar | responder; retribuir; revidar | retribuir; recompensar | vingar-se; retaliar | relatar; condenar de acordo com a lei e reportar às autoridades superiores | enviar; submeter; especificamente, relatar ao superior}
  \end{phonetics}
\end{entry}

\begin{entry}{报名}{7,6}{⼿、⼝}
  \begin{phonetics}{报名}{bao4ming2}[][HSK 2]
    \definition{v.+compl.}{inscrever-se; alistar-se; registrar seu nome; cadastrar-se; matricular-se; informar seu nome à pessoa responsável, órgão, grupo etc., indicando que você deseja participar de alguma atividade ou organização}
  \end{phonetics}
\end{entry}

\begin{entry}{报告}{7,7}{⼿、⼝}
  \begin{phonetics}{报告}{bao4gao4}[][HSK 3]
    \definition[份,篇]{s.}{relatório; discurso; palestra; consultivo; declaração formal feita a superiores ou ao público}
    \definition{v.}{relatar; divulgar; informar; informar formalmente sobre um assunto ou opinião aos superiores ou ao público em geral}
  \end{phonetics}
\end{entry}

\begin{entry}{报纸}{7,7}{⼿、⽷}
  \begin{phonetics}{报纸}{bao4zhi3}[][HSK 2]
    \definition[分,期,张]{s.}{jornal; publicações periódicas cujo conteúdo principal é notícias, geralmente referem-se a jornais diários | papel jornal; um tipo de papel usado para imprimir jornais ou publicações em geral}
  \end{phonetics}
\end{entry}

\begin{entry}{报到}{7,8}{⼿、⼑}
  \begin{phonetics}{报到}{bao4dao4}[][HSK 3]
    \definition{v.+compl.}{apresentar-se ao serviço; fazer o check-in; registrar-se; assinar o livro de presença; informar à organização que você já chegou}
  \end{phonetics}
\end{entry}

\begin{entry}{报答}{7,12}{⼿、⽵}
  \begin{phonetics}{报答}{bao4da2}[][HSK 5]
    \definition{v.}{reembolsar; devolver; retribuir; pagar de volta; mostrar seu apreço de forma tangível}
  \end{phonetics}
\end{entry}

\begin{entry}{报道}{7,12}{⼿、⾡}
  \begin{phonetics}{报道}{bao4dao4}[][HSK 3]
    \definition[个,篇,分]{s.}{história; reportagem; comunicado de imprensa publicado por escrito ou transmitido pela rádio}
    \definition{v.}{cobrir; reportar (notícias); divulgar notícias ao público através de jornais, rádio, etc.}
  \end{phonetics}
\end{entry}

\begin{entry}{报酬}{7,13}{⼿、⾣}
  \begin{phonetics}{报酬}{bao4chou5}
    \definition{s.}{recompensa | remuneração}
  \end{phonetics}
\end{entry}

\begin{entry}{报警}{7,19}{⼿、⾔}
  \begin{phonetics}{报警}{bao4jing3}[][HSK 5]
    \definition{v.}{relatar (um incidente) à polícia; relatar uma situação crítica ou sinalizar uma emergência às autoridades competentes}
  \end{phonetics}
\end{entry}

\begin{entry}{拒绝}{7,9}{⼿、⽷}
  \begin{phonetics}{拒绝}{ju4jue2}[][HSK 5]
    \definition{v.}{recusar; rejeitar; declinar; não aceitar (pedidos, sugestões ou presentes)}
  \end{phonetics}
\end{entry}

\begin{entry}{改}{7}{⽁}
  \begin{phonetics}{改}{gai3}[][HSK 2]
    \definition*{s.}{sobrenome Gai}
    \definition{v.}{mudar; converter; transformar; alterar; substituir | alterar; revisar; aperfeiçoar; modificar | corrigir; retificar; remediar; consertar}
  \end{phonetics}
\end{entry}

\begin{entry}{改正}{7,5}{⽁、⽌}
  \begin{phonetics}{改正}{gai3 zheng4}[][HSK 4]
    \definition{v.}{corrigir; emendar; mudar o errado para o correto}
  \end{phonetics}
\end{entry}

\begin{entry}{改良}{7,7}{⽁、⾉}
  \begin{phonetics}{改良}{gai3liang2}
    \definition{v.}{melhorar (algo) | reformar (um sistema)}
  \end{phonetics}
\end{entry}

\begin{entry}{改进}{7,7}{⽁、⾡}
  \begin{phonetics}{改进}{gai3jin4}[][HSK 3]
    \definition[个,些]{s.}{melhoria}
    \definition{v.}{aprimorar; aperfeiçoar; melhorar; tornar melhor; mudar a situação antiga para melhorar | modificar (mudança mecânica)}
  \end{phonetics}
\end{entry}

\begin{entry}{改变}{7,8}{⽁、⼜}
  \begin{phonetics}{改变}{gai3bian4}[][HSK 2]
    \definition{v.}{mudar; alterar; transformar; converter; moldar; modificar | causar mudanças; alterar}
  \end{phonetics}
\end{entry}

\begin{entry}{改革}{7,9}{⽁、⾰}
  \begin{phonetics}{改革}{gai3ge2}[][HSK 5]
    \definition[项,次,种]{s.}{reforma; reformação; iniciativas para aprimorar a inovação}
    \definition{v.}{reformar; transformar as antigas partes irracionais das coisas em novas que possam ser adaptadas à situação objetiva}
  \end{phonetics}
\end{entry}

\begin{entry}{改造}{7,10}{⽁、⾡}
  \begin{phonetics}{改造}{gai3 zao4}[][HSK 3]
    \definition{v.}{transformar; renovar; modificar o original para melhor se adequar às necessidades; usado principalmente para coisas específicas | remodelar; mudar radicalmente o que é velho e ruim; criar algo novo e bom, para se adaptar às novas circunstâncias e necessidades; usado principalmente para coisas abstratas}
  \end{phonetics}
\end{entry}

\begin{entry}{改善}{7,12}{⽁、⼝}
  \begin{phonetics}{改善}{gai3shan4}[][HSK 4]
    \definition{v.}{melhorar; amenizar; mudar a situação original para torná-la melhor}
  \end{phonetics}
\end{entry}

\begin{entry}{改善关系}{7,12,6,7}{⽁、⼝、⼋、⽷}
  \begin{phonetics}{改善关系}{gai3shan4guan1xi5}
    \definition{v.}{melhorar a relação}
  \end{phonetics}
\end{entry}

\begin{entry}{改善通讯}{7,12,10,5}{⽁、⼝、⾡、⾔}
  \begin{phonetics}{改善通讯}{gai3shan4tong1xun4}
    \definition{v.}{melhorar a comunicação}
  \end{phonetics}
\end{entry}

\begin{entry}{时}{7}{⽇}
  \begin{phonetics}{时}{shi2}[][HSK 3]
    \definition*{s.}{sobrenome Shi}
    \definition{adj.}{atual; presente | a tempo; feito a tempo}
    \definition{adv.}{de vez em quando; ocasionalmente; de ​​tempos em tempos | às vezes\dots às vezes\dots}
    \definition{clas.}{hora; horas}
    \definition{s.}{dias; tempos; longo período de tempo | tempo; tempo fixo | hora; hora do dia | temporada | chance; oportunidade | atualidade; presente | tempo verbal}
  \end{phonetics}
\end{entry}

\begin{entry}{时代}{7,5}{⽇、⼈}
  \begin{phonetics}{时代}{shi2dai4}[][HSK 3]
    \definition[个]{s.}{idade; era; tempos; época | um período na vida de alguém}
  \end{phonetics}
\end{entry}

\begin{entry}{时光}{7,6}{⽇、⼉}
  \begin{phonetics}{时光}{shi2guang1}[][HSK 5]
    \definition[台]{s.}{tempo; passagem do tempo | dias; horas; anos; épocas; períodos}
  \end{phonetics}
\end{entry}

\begin{entry}{时机}{7,6}{⽇、⽊}
  \begin{phonetics}{时机}{shi2ji1}[][HSK 5]
    \definition{s.}{oportunidade; momento oportuno}
  \end{phonetics}
\end{entry}

\begin{entry}{时时}{7,7}{⽇、⽇}
  \begin{phonetics}{时时}{shi2shi2}
    \definition{adv.}{muitas vezes | constantemente}
  \end{phonetics}
\end{entry}

\begin{entry}{时间}{7,7}{⽇、⾨}
  \begin{phonetics}{时间}{shi2jian1}[][HSK 1]
    \definition[段]{s.}{tempo; refere-se à forma de existência do movimento da matéria, um sistema contínuo composto pelo passado, presente e futuro | tempo; período (duração); um período de tempo com início e fim | tempo (um ponto); em algum momento do tempo}
  \end{phonetics}
\end{entry}

\begin{entry}{时事}{7,8}{⽇、⼅}
  \begin{phonetics}{时事}{shi2shi4}[][HSK 5]
    \definition{s.}{acontecimentos atuais; assuntos atuais; eventos atuais | tendências atuais | como as coisas estão indo | a situação atual}
  \end{phonetics}
\end{entry}

\begin{entry}{时刻}{7,8}{⽇、⼑}
  \begin{phonetics}{时刻}{shi2ke4}[][HSK 3]
    \definition{adv.}{constantemente; sempre}
    \definition[个,段]{s.}{tempo; hora; momento; conjuntura}
  \end{phonetics}
\end{entry}

\begin{entry}{时差}{7,9}{⽇、⼯}
  \begin{phonetics}{时差}{shi2cha1}
    \definition{s.}{diferença de tempo | \emph{jet lag}}
  \end{phonetics}
\end{entry}

\begin{entry}{时候}{7,10}{⽇、⼈}
  \begin{phonetics}{时候}{shi2hou5}[][HSK 1]
    \definition[个]{s.}{(um ponto no) tempo; momento; um determinado momento no tempo | (a duração do) tempo; um período de tempo com início e fim}
  \end{phonetics}
\end{entry}

\begin{entry}{时常}{7,11}{⽇、⼱}
  \begin{phonetics}{时常}{shi2chang2}[][HSK 5]
    \definition{adv.}{frequentemente; com frequência}
  \end{phonetics}
\end{entry}

\begin{entry}{旷野}{7,11}{⽇、⾥}
  \begin{phonetics}{旷野}{kuang4ye3}
    \definition{s.}{região selvagem}
  \end{phonetics}
\end{entry}

\begin{entry}{更}{7}{⽈}
  \begin{phonetics}{更}{geng1}
    \definition*{s.}{sobrenome Geng}
    \definition{clas.}{um dos cinco períodos de duas horas em que a noite era anteriormente dividida; vigília; antigamente, a noite era dividida em cinco turnos, cada um com aproximadamente duas horas de duração}
    \definition{v.}{alterar; substituir | experimentar}
  \end{phonetics}
  \begin{phonetics}{更}{geng4}[][HSK 2]
    \definition{adv.}{mais; ainda mais | além disso; além do mais; ainda mais}
  \end{phonetics}
\end{entry}

\begin{entry}{更加}{7,5}{⽈、⼒}
  \begin{phonetics}{更加}{geng4 jia1}[][HSK 3]
    \definition{adv.}{mais; ainda mais; em maior grau; indica um nível mais profundo ou um aumento ou diminuição quantitativa adicional}
  \end{phonetics}
\end{entry}

\begin{entry}{更换}{7,10}{⽈、⼿}
  \begin{phonetics}{更换}{geng1 huan4}[][HSK 5]
    \definition{v.}{alterar; mudar; substituir; comutar}
  \end{phonetics}
\end{entry}

\begin{entry}{更新}{7,13}{⽈、⽄}
  \begin{phonetics}{更新}{geng1xin1}[][HSK 5]
    \definition{v.}{renovar; atualizar; substituir; remover o antigo e substituir pelo novo}
  \end{phonetics}
\end{entry}

\begin{entry}{李}{7}{⽊}
  \begin{phonetics}{李}{li3}
    \definition*{s.}{sobrenome Li}
    \definition{s.}{ameixa}
  \end{phonetics}
\end{entry}

\begin{entry}{李子}{7,3}{⽊、⼦}
  \begin{phonetics}{李子}{li3zi5}
    \definition[个]{s.}{ameixa}
  \end{phonetics}
\end{entry}

\begin{entry}{李四}{7,5}{⽊、⼞}
  \begin{phonetics}{李四}{li3si4}
    \definition*{s.}{Li Si | Zé Ninguém | nome para uma pessoa não especificada, 2 de 3}
  \seealsoref{王五}{wang2wu3}
  \seealsoref{张三}{zhang1san1}
  \end{phonetics}
\end{entry}

\begin{entry}{材料}{7,10}{⽊、⽃}
  \begin{phonetics}{材料}{cai2liao4}[][HSK 4]
    \definition[份,个,种]{s.}{material; algo para fazer um produto acabado | material (figura de linguagem) | dados; material para estudo, pesquisa, etc.; conteúdo de uma obra}
  \end{phonetics}
\end{entry}

\begin{entry}{村}{7}{⽊}
  \begin{phonetics}{村}{cun1}[][HSK 3]
    \definition{adj.}{rústico; grosseiro}
    \definition{s.}{aldeia; vila | área povoada de certo tipo}
  \end{phonetics}
\end{entry}

\begin{entry}{杜宇}{7,6}{⽊、⼧}
  \begin{phonetics}{杜宇}{du4yu3}
    \definition{s.}{cuco (pássaro)}
  \seealsoref{布谷鸟}{bu4gu3niao3}
  \seealsoref{杜鹃}{du4juan1}
  \seealsoref{杜鹃鸟}{du4juan1niao3}
  \end{phonetics}
\end{entry}

\begin{entry}{杜鹃}{7,12}{⽊、⿃}
  \begin{phonetics}{杜鹃}{du4juan1}
    \definition{s.}{cuco (pássaro)}
  \seealsoref{布谷鸟}{bu4gu3niao3}
  \seealsoref{杜鹃鸟}{du4juan1niao3}
  \seealsoref{杜宇}{du4yu3}
  \end{phonetics}
\end{entry}

\begin{entry}{杜鹃鸟}{7,12,5}{⽊、⿃、⿃}
  \begin{phonetics}{杜鹃鸟}{du4juan1niao3}
    \definition{s.}{cuco (pássaro)}
  \seealsoref{布谷鸟}{bu4gu3niao3}
  \seealsoref{杜鹃}{du4juan1}
  \seealsoref{杜宇}{du4yu3}
  \end{phonetics}
\end{entry}

\begin{entry}{束}{7}{⽊}
  \begin{phonetics}{束}{shu4}[][HSK 3]
    \definition*{s.}{sobrenome Shu}
    \definition{clas.}{para cachos, molhos, feixes, feixes de luz, etc.}
    \definition{s.}{monte; pacote; maço; feixe; cacho}
    \definition{v.}{atar; amarrar; vincular | controlar; restringir}
  \end{phonetics}
\end{entry}

\begin{entry}{束腰}{7,13}{⽊、⾁}
  \begin{phonetics}{束腰}{shu4yao1}
    \definition{s.}{cinto | cinta | cinturão}
  \end{phonetics}
\end{entry}

\begin{entry}{杠}{7}{⽊}
  \begin{phonetics}{杠}{gang1}
    \definition{s.}{mastro de bandeira | poste | passarela}
  \end{phonetics}
  \begin{phonetics}{杠}{gang4}
    \definition{s.}{vara grossa | barra | linha grossa | padrão, critério | hífen, traço}
    \definition{v.}{marcar com uma linha grossa | afiar (faca, navalha, etc.)}
  \end{phonetics}
\end{entry}

\begin{entry}{条}{7}{⽊}
  \begin{phonetics}{条}{tiao2}[][HSK 2]
    \definition*{s.}{sobrenome Tiao}
    \definition{clas.}{usado para objetos longos e finos; usado para sintetizar certas coisas longas e retangulares em quantidades fixas | usado para itemização | aplicado ao corpo humano}
    \definition{s.}{galho; galhos finos e longos | tira; faixa | item; artigo | ordem; método | nota; anotação em papel}
  \end{phonetics}
\end{entry}

\begin{entry}{条目}{7,5}{⽊、⽬}
  \begin{phonetics}{条目}{tiao2mu4}
    \definition{s.}{cláusulas e subcláusulas (em documento formal) | verbete (em um dicionário, enciclopédia, etc.)}
  \end{phonetics}
\end{entry}

\begin{entry}{条件}{7,6}{⽊、⼈}
  \begin{phonetics}{条件}{tiao2jian4}[][HSK 2]
    \definition[个,项,些]{s.}{condição; termo; fator; fatores que restringem a ocorrência, existência ou desenvolvimento das coisas | requisito; pré-requisito; qualificação; requisitos ou padrões estabelecidos para determinadas coisas | situação; estado; condição}
  \end{phonetics}
\end{entry}

\begin{entry}{条例}{7,8}{⽊、⼈}
  \begin{phonetics}{条例}{tiao2li4}
    \definition{s.}{código de conduta | ordenanças | regulamentos | regras | estatutos}
  \end{phonetics}
\end{entry}

\begin{entry}{条贯}{7,8}{⽊、⾙}
  \begin{phonetics}{条贯}{tiao2guan4}
    \definition{s.}{ordem | procedimentos | sequência | sistema}
  \end{phonetics}
\end{entry}

\begin{entry}{条幅}{7,12}{⽊、⼱}
  \begin{phonetics}{条幅}{tiao2fu2}
    \definition{s.}{faixa | banner | pergaminho de parede (para pintura ou caligrafia)}
  \end{phonetics}
\end{entry}

\begin{entry}{来}{7}{⽊}
  \begin{phonetics}{来}{lai2}[][HSK 1]
    \definition*{s.}{sobrenome Lai}
    \definition{part.}{usado após uma palavra numérica ou de quantidade; indica uma quantidade aproximada | usado depois de numerais como 一, 二, 三; para listar razões ou fatos, etc.}
    \definition{s.}{usado após uma expressão de tempo para indicar uma duração que vai do passado ao presente}
    \definition{v.}{vir; chegar; de outro lugar para o lugar onde o interlocutor se encontra | aparecer; acontecer; vir; (problemas, coisas, etc.) ocorrerem; surgirem | substitui um verbo com significado específico, indicando a realização de uma ação específica | estar indo para; usado antes de outro verbo, indica que algo será feito | vir para fazer algo; usado após outro verbo, indica que se vai fazer algo | usado para indicar um propósito; expressar o objetivo, fazer algo usando o método, a atitude ou a direção anteriores | usado com 得 ou 不 para indicar possibilidade, capacidade ou hábito}
  \seealsoref{不}{bu4}
  \seealsoref{得}{de5}
  \end{phonetics}
\end{entry}

\begin{entry}{来不及}{7,4,3}{⽊、⼀、⼃}
  \begin{phonetics}{来不及}{lai2bu5ji2}[][HSK 4]
    \definition{v.}{ser tarde demais; não ter tempo; não ter tempo suficiente (para fazer algo); não ser possível participar ou se atualizar devido a restrições de tempo}
  \end{phonetics}
\end{entry}

\begin{entry}{来自}{7,6}{⽊、⾃}
  \begin{phonetics}{来自}{lai2zi4}[][HSK 2]
    \definition{v.}{vir de (um local) | \emph{From:} (cabeçalho de \emph{e -mail})}
  \end{phonetics}
\end{entry}

\begin{entry}{来到}{7,8}{⽊、⼑}
  \begin{phonetics}{来到}{lai2 dao4}[][HSK 1]
    \definition{v.}{chegar; vir}
  \end{phonetics}
\end{entry}

\begin{entry}{来信}{7,9}{⽊、⼈}
  \begin{phonetics}{来信}{lai2 xin4}[][HSK 5]
    \definition{s.}{sua carta; carta recebida; carta ao interlocutor}
    \definition{v.}{enviar uma carta para aqui; enviar uma carta para o remetente}
  \end{phonetics}
\end{entry}

\begin{entry}{来得及}{7,11,3}{⽊、⼻、⼃}
  \begin{phonetics}{来得及}{lai2de5ji2}[][HSK 4]
    \definition{v.}{ainda ter tempo; ser capaz de fazê-lo; ser capaz de fazer algo a tempo; ainda ter tempo de chegar lá ou de se atualizar}
  \end{phonetics}
\end{entry}

\begin{entry}{来源}{7,13}{⽊、⽔}
  \begin{phonetics}{来源}{lai2yuan2}[][HSK 4]
    \definition{s.}{origem; causa; fonte; tabula rasa (ou seja, o lugar de onde as coisas vêm)}
    \definition{v.}{originar-se; surgir; ter origem; (algo) originar (seguido de 于)}
  \seealsoref{于}{yu2}
  \end{phonetics}
\end{entry}

\begin{entry}{极}{7}{⽊}
  \begin{phonetics}{极}{ji2}[][HSK 4]
    \definition*{s.}{sobrenome Ji}
    \definition{adj.}{máximo; extremo; final; supremo}
    \definition{adv.}{extremamente; excessivamente}
    \definition{s.}{o ponto máximo, mais alto; extremo; ápice; ponto culminante | pólo; as extremidades norte e sul da Terra; as extremidades de um ímã; a extremidade de uma fonte de alimentação ou de um aparelho elétrico onde a corrente entra ou sai do aparelho}
    \definition{v.}{chegar ao fim de; levar a extremos}
  \end{phonetics}
\end{entry}

\begin{entry}{……极了}{7,2}{⽊、⼅}
  \begin{phonetics}{……极了}{ji2le5}[][HSK 3]
    \definition{expr.}{extremamente; alto grau de expressão}
  \end{phonetics}
\end{entry}

\begin{entry}{极其}{7,8}{⽊、⼋}
  \begin{phonetics}{极其}{ji2qi2}[][HSK 4]
    \definition{adv.}{mais; extremamente; excessivamente}
  \end{phonetics}
\end{entry}

\begin{entry}{步}{7}{⽌}
  \begin{phonetics}{步}{bu4}[][HSK 3]
    \definition*{s.}{sobrenome Bu}
    \definition*{s.}{geralmente em nomes de lugares}[盐步___Yanbu, na província de Guangdong]
    \definition{clas.}{uma unidade antiga para medida de comprimento, equivalente a cinco 尺}
    \definition{s.}{passo; ritmo | etapa; passo | condição; situação; estado | cais; píer | porto; cidade portuária | (geralmente em nomes de lugares) |}
    \definition{v.}{caminhar; ir a pé | seguir os passos de alguém | (dialeto) medir com passos | seguir; acompanhar | medir a distância com os passos}
  \seealsoref{尺}{chi3}
  \end{phonetics}
\end{entry}

\begin{entry}{步行}{7,6}{⽌、⾏}
  \begin{phonetics}{步行}{bu4 xing2}[][HSK 4]
    \definition{v.}{caminhar; ir a pé; andar a pé (diferente de andar de carro, a cavalo, etc.)}
  \end{phonetics}
\end{entry}

\begin{entry}{每}{7}{⽏}
  \begin{phonetics}{每}{mei3}[][HSK 3]
    \definition{adv.}{cada um; cada qual; indica qualquer uma das repetições ou um conjunto de repetições de um movimento}
    \definition{pron.}{cada; cada um; cada qual; refere-se a qualquer indivíduo do grupo, enfatizando as semelhanças entre os indivíduos}
  \end{phonetics}
\end{entry}

\begin{entry}{每个}{7,3}{⽏、⼈}
  \begin{phonetics}{每个}{mei3ge4}
    \definition{pron.}{cada; cada um}
  \end{phonetics}
\end{entry}

\begin{entry}{每个人}{7,3,2}{⽏、⼈、⼈}
  \begin{phonetics}{每个人}{mei3ge5ren2}
    \definition{pron.}{todo mundo | todos}
  \end{phonetics}
\end{entry}

\begin{entry}{每天}{7,4}{⽏、⼤}
  \begin{phonetics}{每天}{mei3tian1}
    \definition{adv.}{todo dia | cada dia}
  \end{phonetics}
\end{entry}

\begin{entry}{每次}{7,6}{⽏、⽋}
  \begin{phonetics}{每次}{mei3ci4}
    \definition{adv.}{toda vez | cada vez}
  \end{phonetics}
\end{entry}

\begin{entry}{求}{7}{⽔}
  \begin{phonetics}{求}{qiu2}[][HSK 2]
    \definition*{s.}{sobrenome Qiu}
    \definition{v.}{implorar; solicitar; suplicar; rogar | lutar por; buscar; investigar | tentar; procurar; tentar obter | demandar}
  \end{phonetics}
\end{entry}

\begin{entry}{汹涌}{7,10}{⽔、⽔}
  \begin{phonetics}{汹涌}{xiong1yong3}
    \definition{adj.}{turbulento}
    \definition{v.}{aumentar ou emergir violentamente (oceano, rio, lago, etc.)}
  \end{phonetics}
\end{entry}

\begin{entry}{汽水}{7,4}{⽔、⽔}
  \begin{phonetics}{汽水}{qi4 shui3}[][HSK 4]
    \definition[罐,瓶]{s.}{refrigerante; refrigerante gaseificado; bebida refrescante, feita com a pressão de dióxido de carbono para dissolver na água e adicionar açúcar, suco de frutas, especiarias etc.}
  \end{phonetics}
\end{entry}

\begin{entry}{汽车}{7,4}{⽔、⾞}
  \begin{phonetics}{汽车}{qi4 che1}[][HSK 1]
    \definition[辆,种,款]{s.}{automóvel; carro; veículo motorizado; veículo movido a motor de combustão interna, que circula principalmente em rodovias ou ruas, geralmente com quatro ou mais pneus de borracha, usado para transportar pessoas ou mercadorias}
  \end{phonetics}
\end{entry}

\begin{entry}{汽油}{7,8}{⽔、⽔}
  \begin{phonetics}{汽油}{qi4you2}[][HSK 4]
    \definition{s.}{gasolina; mistura líquida de hidrocarbonetos com volatilidade e combustibilidade, que é usada como combustível a partir do fracionamento ou craqueamento do petróleo}
  \end{phonetics}
\end{entry}

\begin{entry}{沉}{7}{⽔}
  \begin{phonetics}{沉}{chen2}[][HSK 4]
    \definition{adj.}{profundo | pesado | pesado (sentir-se pesado)}
    \definition{v.}{afundar; submergir; imergir | manter baixo; abaixar | descansar; parar}
  \end{phonetics}
\end{entry}

\begin{entry}{沉重}{7,9}{⽔、⾥}
  \begin{phonetics}{沉重}{chen2zhong4}[][HSK 4]
    \definition{adj.}{(pressão, fardo, etc.) muito pesado; profundo | sério; pesado; humor pouco animador; fardo pesado de pensamentos}
  \end{phonetics}
\end{entry}

\begin{entry}{沉默}{7,16}{⽔、⿊}
  \begin{phonetics}{沉默}{chen2mo4}[][HSK 4]
    \definition{adj.}{silencioso; reticente; taciturno; não comunicativo}
    \definition{v.}{silenciar; não falar por causa de alguma coisa}
  \end{phonetics}
\end{entry}

\begin{entry}{沙}{7}{⽔}
  \begin{phonetics}{沙}{sha1}
    \definition*{s.}{sobrenome Sha}
    \definition[粒]{s.}{areia | cascalho | grânulo | pó}
  \end{phonetics}
\end{entry}

\begin{entry}{沙子}{7,3}{⽔、⼦}
  \begin{phonetics}{沙子}{sha1 zi5}[][HSK 3]
    \definition[粒,把]{s.}{areia; grão | \emph{pellets}; grãos pequenos}
  \end{phonetics}
\end{entry}

\begin{entry}{沙发}{7,5}{⽔、⼜}
  \begin{phonetics}{沙发}{sha1fa1}[][HSK 3]
    \definition[套,组,个,张]{s.}{sofá; divã}
  \end{phonetics}
\end{entry}

\begin{entry}{沙鱼}{7,8}{⽔、⿂}
  \begin{phonetics}{沙鱼}{sha1yu2}
    \variantof{鲨鱼}
  \end{phonetics}
\end{entry}

\begin{entry}{沙特}{7,10}{⽔、⽜}
  \begin{phonetics}{沙特}{sha1te4}
    \definition*{s.}{Saudita | abreviação de 沙特阿拉伯}
  \seealsoref{沙特阿拉伯}{sha1te4 a1la1bo2}
  \end{phonetics}
\end{entry}

\begin{entry}{沙特阿拉伯}{7,10,7,8,7}{⽔、⽜、⾩、⼿、⼈}
  \begin{phonetics}{沙特阿拉伯}{sha1te4 a1la1bo2}
    \definition*{s.}{Arábia Saudita}
  \end{phonetics}
\end{entry}

\begin{entry}{沙漠}{7,13}{⽔、⽔}
  \begin{phonetics}{沙漠}{sha1mo4}[][HSK 5]
    \definition[个]{s.}{deserto; superfície totalmente coberta por areia, sem água corrente, clima seco e vegetação escassa}
  \end{phonetics}
\end{entry}

\begin{entry}{沟}{7}{⽔}
  \begin{phonetics}{沟}{gou1}[][HSK 5]
    \definition[条]{s.}{canal; vala; sarjeta; trincheira; cursos d'água ou fortificações escavados | ranhura; sulco raso; uma depressão que se assemelha a uma vala | ravina; barranco; cursos d'água}
  \end{phonetics}
\end{entry}

\begin{entry}{沟通}{7,10}{⽔、⾡}
  \begin{phonetics}{沟通}{gou1tong1}[][HSK 5]
    \definition{v.}{comunicar; comunicar-se para entender as ideias, opiniões, etc. | conectar; ligar; estabelecer um paralelo entre os dois}
  \end{phonetics}
\end{entry}

\begin{entry}{没}{7}{⽔}
  \begin{phonetics}{没}{mei2}[][HSK 1]
    \definition{adv.}{não; nunca; negar que uma ação ou situação tenha ocorrido, com o significado de 不曾}
    \definition{pref.}{não (prefixo negativo para verbos, traduzido para outras línguas com verbos no pretérito)}
    \definition{v.}{não possuir; não ter | não existe; não há | ninguém; usado antes de 谁, 什么, 哪个, significa 全都不 | não ser tão bom quanto; ser inferior a; não chega a; não é tão bom quanto | menor que; insuficiente}
  \seealsoref{不曾}{bu4 ceng2}
  \seealsoref{哪个}{na3ge5}
  \seealsoref{全都不}{quan2dou1 bu4}
  \seealsoref{谁}{shei2}
  \seealsoref{什么}{shen2me5}
  \end{phonetics}
  \begin{phonetics}{没}{mo4}
    \definition{adj.}{último; final}
    \definition{v.}{afundar na água; submergir | transbordar; subir além; exceder ou ultrapassar | esconder-se; desaparecer; sumir; ocultar-se | confiscar; expropriar | morrer}
    \variantof{没}
  \end{phonetics}
\end{entry}

\begin{entry}{没了}{7,2}{⽔、⼅}
  \begin{phonetics}{没了}{mei2le5}
    \definition{v.}{estar morto | deixar de existir}
  \end{phonetics}
\end{entry}

\begin{entry}{没什么}{7,4,3}{⽔、⼈、⼃}
  \begin{phonetics}{没什么}{mei2 shen2 me5}[][HSK 1]
    \definition{expr.}{não é nada; está tudo bem; não importa}
  \end{phonetics}
\end{entry}

\begin{entry}{没用}{7,5}{⽔、⽤}
  \begin{phonetics}{没用}{mei2 yong4}[][HSK 3]
    \definition{adj.}{inútil; imprestável; sem valor; sem préstimo; vão; que não serve para nada}
  \end{phonetics}
\end{entry}

\begin{entry}{没关系}{7,6,7}{⽔、⼋、⽷}
  \begin{phonetics}{没关系}{mei2guan1xi5}[][HSK 1]
    \definition{v.}{está tudo bem; não é nada; não importa; não se preocupe}
  \seealsoref{没有关系}{mei2you3guan1xi5}
  \end{phonetics}
\end{entry}

\begin{entry}{没有}{7,6}{⽔、⽉}
  \begin{phonetics}{没有}{mei2 you3}[][HSK 1]
    \definition{adv.}{ainda não; (usado com o pretérito) não; ação ou estado negativo ocorreu}
    \definition{v.}{não há; não tem; não existe}
  \end{phonetics}
\end{entry}

\begin{entry}{没有关系}{7,6,6,7}{⽔、⽉、⼋、⽷}
  \begin{phonetics}{没有关系}{mei2you3guan1xi5}
    \definition{v.}{não ter problema | não ter importância | não fazer mal}
  \seealsoref{没关系}{mei2guan1xi5}
  \end{phonetics}
\end{entry}

\begin{entry}{没有次序}{7,6,6,7}{⽔、⽉、⽋、⼴}
  \begin{phonetics}{没有次序}{mei2you3 ci4xu4}
    \definition{adj.}{sem ordem; nenhuma ordem}
  \end{phonetics}
\end{entry}

\begin{entry}{没有意思}{7,6,13,9}{⽔、⽉、⼼、⼼}
  \begin{phonetics}{没有意思}{mei2you3yi4si5}
    \definition{adj.}{tedioso | chato | sem interesse}
  \end{phonetics}
\end{entry}

\begin{entry}{没事儿}{7,8,2}{⽔、⼅、⼉}
  \begin{phonetics}{没事儿}{mei2 shi4r5}[][HSK 1]
    \definition{expr.}{fora de perigo; nada sério | não importa; não é nada; está tudo bem; não importa | está tudo bem; sem problemas; não se preocupe com isso; não é grande coisa; não há nada errado}
    \definition{v.}{não ter nada para fazer; ser livre; estar perdido | estar desempregado; estar sem trabalho | não ter responsabilidade}
  \end{phonetics}
\end{entry}

\begin{entry}{没法儿}{7,8,2}{⽔、⽔、⼉}
  \begin{phonetics}{没法儿}{mei2 fa3r5}[][HSK 4]
    \definition{adv.}{não pode; sem chance}
  \end{phonetics}
\end{entry}

\begin{entry}{没想到}{7,13,8}{⽔、⼼、⼑}
  \begin{phonetics}{没想到}{mei2 xiang3 dao4}[][HSK 4]
    \definition{expr.}{não esperava; inesperado}
  \end{phonetics}
\end{entry}

\begin{entry}{没错}{7,13}{⽔、⾦}
  \begin{phonetics}{没错}{mei2 cuo4}[][HSK 4]
    \definition{adv.}{está certo; é isso mesmo; não há como errar}
  \end{phonetics}
\end{entry}

\begin{entry}{灵感}{7,13}{⽕、⼼}
  \begin{phonetics}{灵感}{ling2gan3}
    \definition{s.}{inspiração | explosão de criatividade em empreendimento científico ou artístico}
  \end{phonetics}
\end{entry}

\begin{entry}{灵魂}{7,13}{⽕、⿁}
  \begin{phonetics}{灵魂}{ling2hun2}
    \definition{s.}{alma | espírito}
  \end{phonetics}
\end{entry}

\begin{entry}{灶台}{7,5}{⽕、⼝}
  \begin{phonetics}{灶台}{zao4tai2}
    \definition{s.}{fogão}
  \end{phonetics}
\end{entry}

\begin{entry}{灾}{7}{⽕}
  \begin{phonetics}{灾}{zai1}[][HSK 5]
    \definition[个,场]{s.}{calamidade; desastre | infortúnio pessoal; adversidade | azar}
  \end{phonetics}
\end{entry}

\begin{entry}{灾区}{7,4}{⽕、⼖}
  \begin{phonetics}{灾区}{zai1 qu1}[][HSK 5]
    \definition{s.}{área de desastre; área afetada por catástrofes}
  \end{phonetics}
\end{entry}

\begin{entry}{灾害}{7,10}{⽕、⼧}
  \begin{phonetics}{灾害}{zai1hai4}[][HSK 5]
    \definition[个]{s.}{desastre; calamidade; danos causados pela seca, inundações, pragas, granizo, guerras, etc.}
  \end{phonetics}
\end{entry}

\begin{entry}{灾难}{7,10}{⽕、⾫}
  \begin{phonetics}{灾难}{zai1nan4}[][HSK 5]
    \definition[场,次]{s.}{desastre; sofrimento; calamidade; catástrofe; danos e sofrimentos causados por desastres naturais ou guerras}
  \end{phonetics}
\end{entry}

\begin{entry}{状况}{7,7}{⽝、⼎}
  \begin{phonetics}{状况}{zhuang4kuang4}[][HSK 3]
    \definition[个,种]{s.}{estado; \emph{status}; condição; estado de coisas}
  \end{phonetics}
\end{entry}

\begin{entry}{状态}{7,8}{⽝、⼼}
  \begin{phonetics}{状态}{zhuang4tai4}[][HSK 3]
    \definition[种,个]{s.}{\emph{status}; estado; condição; estado de coisas; a forma em que uma pessoa ou coisa aparece}
  \end{phonetics}
\end{entry}

\begin{entry}{犹豫}{7,15}{⽝、⾗}
  \begin{phonetics}{犹豫}{you2yu4}[][HSK 5]
    \definition{adj.}{hesitante; indeciso, incapaz de decidir ou agir}
    \definition{v.}{hesitar; ser indeciso}
  \end{phonetics}
\end{entry}

\begin{entry}{狂}{7}{⽝}
  \begin{phonetics}{狂}{kuang2}[][HSK 5]
    \definition*{s.}{sobrenome Kuang}
    \definition{adj.}{louco; maluco | violento; selvagem | selvagem; delirante; furioso; desenfreado; desinibido; sem restrições | arrogante; autoritário}
  \end{phonetics}
\end{entry}

\begin{entry}{狂欢节}{7,6,5}{⽝、⽋、⾋}
  \begin{phonetics}{狂欢节}{kuang2huan1jie2}
    \definition*{s.}{Carnaval}
  \end{phonetics}
\end{entry}

\begin{entry}{男}{7}{⽥}
  \begin{phonetics}{男}{nan2}[][HSK 1]
    \definition{adj.}{homem; macho; masculino (em oposição a 女)}
    \definition[个,位]{s.}{filho; menino | homem | barão (o mais baixo de cinco ordens de nobreza)}
  \seealsoref{女}{nv3}
  \end{phonetics}
\end{entry}

\begin{entry}{男人}{7,2}{⽥、⼈}
  \begin{phonetics}{男人}{nan2 ren2}[][HSK 1]
    \definition[个]{s.}{homem adulto; macho; cavalheiro | marido}
  \end{phonetics}
\end{entry}

\begin{entry}{男士}{7,3}{⽥、⼠}
  \begin{phonetics}{男士}{nan2 shi4}[][HSK 4]
    \definition{s.}{cavalheiro; \emph{gentleman}}
  \end{phonetics}
\end{entry}

\begin{entry}{男女}{7,3}{⽥、⼥}
  \begin{phonetics}{男女}{nan2 nv3}[][HSK 4]
    \definition{s.}{homens e mulheres; masculino e feminino}
  \end{phonetics}
\end{entry}

\begin{entry}{男子}{7,3}{⽥、⼦}
  \begin{phonetics}{男子}{nan2zi3}[][HSK 3]
    \definition[名]{s.}{homem; macho}
  \end{phonetics}
\end{entry}

\begin{entry}{男生}{7,5}{⽥、⽣}
  \begin{phonetics}{男生}{nan2 sheng1}[][HSK 1]
    \definition[个]{s.}{menino; estudante; estudante do sexo masculino; aluno do sexo masculino}
  \end{phonetics}
\end{entry}

\begin{entry}{男性}{7,8}{⽥、⼼}
  \begin{phonetics}{男性}{nan2 xing4}[][HSK 5]
    \definition{s.}{masculino; homem; masculinidade}
  \end{phonetics}
\end{entry}

\begin{entry}{男朋友}{7,8,4}{⽥、⽉、⼜}
  \begin{phonetics}{男朋友}{nan2 peng2 you5}[][HSK 1]
    \definition{s.}{namorado}
  \end{phonetics}
\end{entry}

\begin{entry}{男孩儿}{7,9,2}{⽥、⼦、⼉}
  \begin{phonetics}{男孩儿}{nan2hai2r5}[][HSK 1]
    \definition{s.}{menino; rapaz}
  \end{phonetics}
\end{entry}

\begin{entry}{疗养}{7,9}{⽧、⼋}
  \begin{phonetics}{疗养}{liao2 yang3}[][HSK 4]
    \definition{v.}{recuperar; convalescer; tratar pessoas com doenças crônicas ou debilitantes em instituições médicas especializadas com foco na recuperação}
  \end{phonetics}
\end{entry}

\begin{entry}{社}{7}{⽰}
  \begin{phonetics}{社}{she4}[][HSK 5]
    \definition[个]{s.}{agência; sociedade; órgão organizado; organização; comunidade | comuna popular | o deus da terra, sacrifícios a ele ou altares para tais sacrifícios; na antiguidade, o deus da terra, o local onde ele era venerado, o dia da veneração e o ritual eram chamados de 社 | agência de notícias |  imprensa}
  \end{phonetics}
\end{entry}

\begin{entry}{社区}{7,4}{⽰、⼖}
  \begin{phonetics}{社区}{she4qu1}[][HSK 5]
    \definition{s.}{bairro; comunidade residencial; bairros da cidade, divididos de acordo com a localização geográfica | distrito; comunidade (para pessoas da mesma classe social, etc.) ; lugar onde pessoas com características comuns, como classe social, vivem juntas}
  \end{phonetics}
\end{entry}

\begin{entry}{社会}{7,6}{⽰、⼈}
  \begin{phonetics}{社会}{she4hui4}[][HSK 3]
    \definition[个,种]{s.}{sociedade | comunidade}
  \end{phonetics}
\end{entry}

\begin{entry}{私人}{7,2}{⽲、⼈}
  \begin{phonetics}{私人}{si1ren2}[][HSK 5]
    \definition{adj.}{privado; pertencente a um indivíduo ou exercido a título individual; não público | interpessoal}
    \definition[个]{s.}{algo privado; pessoas que se aproximam de você por motivos pessoais ou interesses próprios}
  \end{phonetics}
\end{entry}

\begin{entry}{私人诊所}{7,2,7,8}{⽲、⼈、⾔、⼾}
  \begin{phonetics}{私人诊所}{si1ren2 zhen3suo3}
    \definition[些]{s.}{clínica privada}
  \end{phonetics}
\end{entry}

\begin{entry}{私人信件}{7,2,9,6}{⽲、⼈、⼈、⼈}
  \begin{phonetics}{私人信件}{si1ren2 xin4jian4}
    \definition{s.}{carta pessoal}
  \end{phonetics}
\end{entry}

\begin{entry}{私人钥匙}{7,2,9,11}{⽲、⼈、⾦、⼔}
  \begin{phonetics}{私人钥匙}{si1ren2yao4shi5}
    \definition{s.}{(criptografia) chave privada}
  \end{phonetics}
\end{entry}

\begin{entry}{私生活}{7,5,9}{⽲、⽣、⽔}
  \begin{phonetics}{私生活}{si1sheng1huo2}
    \definition{s.}{vida privada}
  \end{phonetics}
\end{entry}

\begin{entry}{私自}{7,6}{⽲、⾃}
  \begin{phonetics}{私自}{si1zi4}
    \definition{adj.}{privado | pessoal}
    \definition{adv.}{secretamente | sem aprovação explícita}
  \end{phonetics}
\end{entry}

\begin{entry}{究竟}{7,11}{⽳、⾳}
  \begin{phonetics}{究竟}{jiu1jing4}[][HSK 4]
    \definition{adv.}{de fato; exatamente; usado em frases interrogativas para buscar | afinal de contas, no final; ênfase em fatos ou motivos}
    \definition{s.}{resultado; desfecho; a causa, o efeito ou a história completa do que aconteceu}
  \end{phonetics}
\end{entry}

\begin{entry}{穷}{7}{⽳}
  \begin{phonetics}{穷}{qiong2}[][HSK 4]
    \definition{adj.}{remoto; isolado; de difícil acesso | pobre; atingido pela pobreza | situação difícil, sem saída}
    \definition{adv.}{completamente | extremamente}
    \definition{v.}{exaurir; esgotar; consmir | ir até o fim; perseguir completamente perseguido; sondar profundamente | gastar}
  \end{phonetics}
\end{entry}

\begin{entry}{穷人}{7,2}{⽳、⼈}
  \begin{phonetics}{穷人}{qiong2 ren2}[][HSK 4]
    \definition{s.}{os pobres; pessoas pobres}
  \end{phonetics}
\end{entry}

\begin{entry}{系}{7}{⽷}
  \begin{phonetics}{系}{ji4}
    \definition{v.}{amarrar; prender; abotoar}
  \end{phonetics}
  \begin{phonetics}{系}{xi4}[][HSK 3,4]
    \definition*{s.}{sobrenome Xi}
    \definition{s.}{faculdade (da universidade) | departamento}
    \definition{v.}{sistema; série | departamento; faculdade}
    \definition{v.}{relacionar-se com; suportar; depender de | sentir-se ansioso; estar preocupado | amarrar; prender | ser}
  \end{phonetics}
\end{entry}

\begin{entry}{系囚}{7,5}{⽷、⼞}
  \begin{phonetics}{系囚}{xi4qiu2}
    \definition{s.}{prisioneiro}
  \end{phonetics}
\end{entry}

\begin{entry}{系列}{7,6}{⽷、⼑}
  \begin{phonetics}{系列}{xi4lie4}[][HSK 4]
    \definition{s.}{série; conjunto; conjunto de coisas relacionadas (matemática)}
  \end{phonetics}
\end{entry}

\begin{entry}{系统}{7,9}{⽷、⽷}
  \begin{phonetics}{系统}{xi4tong3}[][HSK 4]
    \definition{adj.}{sistemático; organizado}
    \definition[个]{s.}{sistema; relação de tipos semelhantes (ou seja, grupo de coisas semelhantes)}
  \end{phonetics}
\end{entry}

\begin{entry}{纯}{7}{⽷}
  \begin{phonetics}{纯}{chun2}[][HSK 4]
    \definition{adj.}{puro; não misturado; livre de impurezas | simples; puro e simples | habilidoso; proficiente; bem versado}
  \end{phonetics}
\end{entry}

\begin{entry}{纯净水}{7,8,4}{⽷、⼎、⽔}
  \begin{phonetics}{纯净水}{chun2 jing4 shui3}[][HSK 4]
    \definition{s.}{água purificada}
  \end{phonetics}
\end{entry}

\begin{entry}{纯真}{7,10}{⽷、⼗}
  \begin{phonetics}{纯真}{chun2zhen1}
    \definition{adj.}{inocente e não afetado | puro e não adulterado}
    \definition{s.}{inocência}
  \end{phonetics}
\end{entry}

\begin{entry}{纷纷}{7,7}{⽷、⽷}
  \begin{phonetics}{纷纷}{fen1fen1}[][HSK 4]
    \definition{adj.}{numeroso e confuso; muitos e desordenados}
    \definition{adv.}{um após o outro; em sucessão; em rápida sucessão}
  \end{phonetics}
\end{entry}

\begin{entry}{纸}{7}{⽷}
  \begin{phonetics}{纸}{zhi3}[][HSK 2]
    \definition{clas.}{usado para documentos, cartas, etc.}
    \definition[张,沓]{s.}{papel; uma folha fina de material usada para escrever, pintar, imprimir, embalar, etc., feita principalmente de fibras vegetais | papel joss; papel de incenso; refere-se especificamente a itens supersticiosos, como papel-moeda}
  \end{phonetics}
\end{entry}

\begin{entry}{纸巾}{7,3}{⽷、⼱}
  \begin{phonetics}{纸巾}{zhi3jin1}
    \definition[张,包]{s.}{lenço | guardanapo | papel toalha}
  \end{phonetics}
\end{entry}

\begin{entry}{纸币}{7,4}{⽷、⼱}
  \begin{phonetics}{纸币}{zhi3bi4}
    \definition[张]{s.}{nota (dinheiro) | cédula}
  \end{phonetics}
\end{entry}

\begin{entry}{纸尿裤}{7,7,12}{⽷、⼫、⾐}
  \begin{phonetics}{纸尿裤}{zhi3niao4ku4}
    \definition{s.}{fralda descartável}
  \end{phonetics}
\end{entry}

\begin{entry}{纸张}{7,7}{⽷、⼸}
  \begin{phonetics}{纸张}{zhi3zhang1}
    \definition{s.}{papel}
  \end{phonetics}
\end{entry}

\begin{entry}{纸烟}{7,10}{⽷、⽕}
  \begin{phonetics}{纸烟}{zhi3yan1}
    \definition{s.}{cigarro}
  \end{phonetics}
\end{entry}

\begin{entry}{纹路}{7,13}{⽷、⾜}
  \begin{phonetics}{纹路}{wen2lu4}
    \definition{s.}{padrão de linhas | rugas | veias | veias (em mármore ou impressão digital) | grãos (em madeira, etc.)}
  \end{phonetics}
\end{entry}

\begin{entry}{肚}{7}{⾁}
  \begin{phonetics}{肚}{du3}
    \definition{s.}{tripas | entranhas}
  \end{phonetics}
  \begin{phonetics}{肚}{du4}
    \definition{s.}{barriga}
  \end{phonetics}
\end{entry}

\begin{entry}{肚子}{7,3}{⾁、⼦}
  \begin{phonetics}{肚子}{du4zi5}[][HSK 4]
    \definition[个,只]{s.}{abdômen; barriguinha; ventre; barriga}
  \end{phonetics}
\end{entry}

\begin{entry}{肠}{7}{⾁}
  \begin{phonetics}{肠}{chang2}[][HSK 5]
    \definition{s.}{intestinos | salsicha; linguiça | coração; sentimentos; emoções}
  \end{phonetics}
\end{entry}

\begin{entry}{良心}{7,4}{⾉、⼼}
  \begin{phonetics}{良心}{liang2xin1}
    \definition{s.}{consciência}
  \end{phonetics}
\end{entry}

\begin{entry}{良田}{7,5}{⾉、⽥}
  \begin{phonetics}{良田}{liang2tian2}
    \definition{s.}{terra agrícola boa | terra fértil}
  \end{phonetics}
\end{entry}

\begin{entry}{良好}{7,6}{⾉、⼥}
  \begin{phonetics}{良好}{liang2hao3}[][HSK 4]
    \definition{adj.}{bom; ótimo; bem}
  \end{phonetics}
\end{entry}

\begin{entry}{芥}{7}{⾋}
  \begin{phonetics}{芥}{gai4}
    \definition{s.}{usado em 芥蓝 \dpy{gai4lan2}}
  \seealsoref{芥蓝}{gai4lan2}
  \end{phonetics}
  \begin{phonetics}{芥}{jie4}
    \definition{s.}{mostarda}
  \end{phonetics}
\end{entry}

\begin{entry}{芥兰}{7,5}{⾋、⼋}
  \begin{phonetics}{芥兰}{gai4lan2}
    \variantof{芥蓝}
  \end{phonetics}
  \begin{phonetics}{芥兰}{jie4lan2}
    \definition{s.}{couve}
  \end{phonetics}
\end{entry}

\begin{entry}{芥蓝}{7,13}{⾋、⾋}
  \begin{phonetics}{芥蓝}{gai4lan2}
    \definition{s.}{brócolis chinês | couve chinesa | mostarda}
  \seealsoref{格兰菜}{ge2lan2cai4}
  \end{phonetics}
\end{entry}

\begin{entry}{芦笋}{7,10}{⾋、⽵}
  \begin{phonetics}{芦笋}{lu2sun3}
    \definition{s.}{aspargos}
  \end{phonetics}
\end{entry}

\begin{entry}{芯片}{7,4}{⾋、⽚}
  \begin{phonetics}{芯片}{xin1pian4}
    \definition{s.}{chip de computador | microchip}
  \end{phonetics}
\end{entry}

\begin{entry}{花}{7}{⾋}
  \begin{phonetics}{花}{hua1}[][HSK 1,2,4]
    \definition*{s.}{sobrenome Hua}
    \definition{adj.}{multicolorido; colorido | embaçado; obscuro; deslumbrado e confuso | extravagante; florido; vistoso}
    \definition[朵,支,束,把,盆,簇]{s.}{flor; órgãos de reprodução sexual de plantas com sementes | flor; planta ornamental |  qualquer coisa que se assemelhe a uma flor | fogos de artifício | padrão; design; design decorativo | flor; metáfora para a essência de uma causa | prostituta; cortesã; referindo-se a prostitutas ou a assuntos relacionados a prostitutas | algodão | varíola | ferimento; ferida; lesões traumáticas sofridas em combate}
    \definition{v.}{gastar; despender; consumir}
  \end{phonetics}
\end{entry}

\begin{entry}{花儿}{7,2}{⾋、⼉}
  \begin{phonetics}{花儿}{hua1r5}
    \definition[朵,支,束,把,盆,簇]{s.}{flor}
  \end{phonetics}
\end{entry}

\begin{entry}{花生}{7,5}{⾋、⽣}
  \begin{phonetics}{花生}{hua1sheng1}
    \definition[粒]{s.}{amendoim}
  \end{phonetics}
\end{entry}

\begin{entry}{花园}{7,7}{⾋、⼞}
  \begin{phonetics}{花园}{hua1 yuan2}[][HSK 2]
    \definition[个,座]{s.}{jardim; um local onde se plantam flores e árvores para passear e descansar}
  \end{phonetics}
\end{entry}

\begin{entry}{花店}{7,8}{⾋、⼴}
  \begin{phonetics}{花店}{hua1dian4}
    \definition{s.}{floricultura}
  \end{phonetics}
\end{entry}

\begin{entry}{花茶}{7,9}{⾋、⾋}
  \begin{phonetics}{花茶}{hua1cha2}
    \definition[杯,壶]{s.}{chá perfumado}
  \end{phonetics}
\end{entry}

\begin{entry}{花样游泳}{7,10,12,8}{⾋、⽊、⽔、⽔}
  \begin{phonetics}{花样游泳}{hua1yang4you2yong3}
    \definition{s.}{nado sincronizado}
  \end{phonetics}
\end{entry}

\begin{entry}{花椰菜}{7,12,11}{⾋、⽊、⾋}
  \begin{phonetics}{花椰菜}{hua1ye1cai4}
    \definition{s.}{couve-flor}
  \end{phonetics}
\end{entry}

\begin{entry}{芹菜}{7,11}{⾋、⾋}
  \begin{phonetics}{芹菜}{qin2cai4}
    \definition{s.}{salsão}
  \end{phonetics}
\end{entry}

\begin{entry}{苏格兰}{7,10,5}{⾋、⽊、⼋}
  \begin{phonetics}{苏格兰}{su1ge2lan2}
    \definition*{s.}{Escócia}
  \end{phonetics}
\end{entry}

\begin{entry}{补}{7}{⾐}
  \begin{phonetics}{补}{bu3}[][HSK 3]
    \definition*{s.}{sobrenome Bu}
    \definition{s.}{ajuda; uso; benefício; utilidade}
    \definition{v.}{reparar; consertar; remendar; adicionar materiais, consertar coisas quebradas | abastecer; encher; repor; adicionar suplemento; complementar; completar; preencher | nutrir}
  \end{phonetics}
\end{entry}

\begin{entry}{补充}{7,6}{⾐、⼉}
  \begin{phonetics}{补充}{bu3chong1}[][HSK 3]
    \definition{adj.}{adicional | suplementar}
    \definition{v.}{reabastecer; suplementar; complementar; aumentar uma parte quando houver insuficiência ou perda}
  \end{phonetics}
\end{entry}

\begin{entry}{补贴}{7,9}{⾐、⾙}
  \begin{phonetics}{补贴}{bu3tie1}[][HSK 5]
    \definition[笔,项,种,份]{s.}{subsídio; ajuda de custo; custos de indenização ou assistência concedida a empresas ou indivíduos pelo estado ou governo}
    \definition{v.}{subsidiar; compensar a falta de dinheiro ou coisas; refere-se principalmente à compensação financeira ou ajuda dada pelo estado ou governo a empresas ou indivíduos}
  \end{phonetics}
\end{entry}

\begin{entry}{补偿}{7,11}{⾐、⼈}
  \begin{phonetics}{补偿}{bu3chang2}[][HSK 5]
    \definition{v.}{compensar (perda, consumo); compensar (deficiências, diferenças)}
  \end{phonetics}
\end{entry}

\begin{entry}{角}{7}{⾓}
  \begin{phonetics}{角}{jiao3}[][HSK 2]
    \definition*{s.}{Jiao, uma das mansões lunares}
    \definition{clas.}{uma unidade monetária fracionária na China (=1/10 de um yuan ou 10 fen)}
    \definition[个,只,对]{s.}{chifre; o objeto duro que cresce na cabeça de bovinos, ovinos, veados, etc. | buzina; corneta; instrumentos musicais tocados no exército antigo | algo com a forma de um chifre | cabo; promontório; península | esquina; canto; a junção entre duas arestas de um objeto | ângulo}
  \end{phonetics}
  \begin{phonetics}{角}{jue2}
    \definition*{s.}{sobrenome Jue}
    \definition{s.}{papel (teatro)}
    \definition{v.}{competir}
  \end{phonetics}
\end{entry}

\begin{entry}{角色}{7,6}{⾓、⾊}
  \begin{phonetics}{角色}{jue2se4}[][HSK 4]
    \definition{s.}{papel; personagem em uma peça; personagem representado por um ator | papel; função; parte}
  \end{phonetics}
\end{entry}

\begin{entry}{角度}{7,9}{⾓、⼴}
  \begin{phonetics}{角度}{jiao3du4}[][HSK 2]
    \definition[个,种]{s.}{perspectiva; ponto de vista; o ponto de partida para ver as coisas | ângulo; o tamanho do ângulo; normalmente expresso em graus ou radianos}
  \end{phonetics}
\end{entry}

\begin{entry}{言论}{7,6}{⾔、⾔}
  \begin{phonetics}{言论}{yan2lun4}
    \definition{s.}{expressão de opinião |  visualizações | comentários | argumentos}
  \end{phonetics}
\end{entry}

\begin{entry}{言语}{7,9}{⾔、⾔}
  \begin{phonetics}{言语}{yan2 yu3}[][HSK 5]
    \definition{s.}{verbal; fala; linguagem falada; conversa; palavras}
  \end{phonetics}
\end{entry}

\begin{entry}{证}{7}{⾔}
  \begin{phonetics}{证}{zheng4}[][HSK 3]
    \definition{s.}{evidência; prova; testemunho; testemunha | certificado; cartão | doença; enfermidade}
    \definition{v.}{provar; verificar; demonstrar}
  \end{phonetics}
\end{entry}

\begin{entry}{证书}{7,4}{⾔、⼄}
  \begin{phonetics}{证书}{zheng4shu1}[][HSK 5]
    \definition[张,份,些]{s.}{certificado; documentos emitidos por instituições, grupos, etc., que comprovem experiência, nível, honras, poderes, etc.}
  \end{phonetics}
\end{entry}

\begin{entry}{证件}{7,6}{⾔、⼈}
  \begin{phonetics}{证件}{zheng4jian4}[][HSK 3]
    \definition[个,本,张]{s.}{documentos; credenciais; certificado}
  \end{phonetics}
\end{entry}

\begin{entry}{证实}{7,8}{⾔、⼧}
  \begin{phonetics}{证实}{zheng4shi2}[][HSK 5]
    \definition{v.}{verificar; afirmar; confirmar; corroborar; demonstrar; autenticar; provar que é verdadeiro}
  \end{phonetics}
\end{entry}

\begin{entry}{证明}{7,8}{⾔、⽇}
  \begin{phonetics}{证明}{zheng4ming2}[][HSK 3]
    \definition[个,份]{s.}{certificado; testemunho; identificação; certificado ou carta de certificação; documentos que comprovem identidade, experiência, etc., como carteira de estudante, carteira de trabalho, certificado de graduação, etc.}
    \definition{v.}{provar; testemunhar; sustentar; usar materiais confiáveis ​​para mostrar ou determinar a autenticidade de uma pessoa ou coisa}
  \end{phonetics}
\end{entry}

\begin{entry}{证据}{7,11}{⾔、⼿}
  \begin{phonetics}{证据}{zheng4ju4}[][HSK 3]
    \definition{s.}{prova; evidência; testemunho; fatos ou materiais relevantes que podem provar a autenticidade de algo}
  \end{phonetics}
\end{entry}

\begin{entry}{评价}{7,6}{⾔、⼈}
  \begin{phonetics}{评价}{ping2jia4}[][HSK 3]
    \definition[个,项,条,份]{s.}{avaliação; apreciação}
    \definition{v.}{estimar; avaliar}
  \end{phonetics}
\end{entry}

\begin{entry}{评论}{7,6}{⾔、⾔}
  \begin{phonetics}{评论}{ping2lun4}[][HSK 5]
    \definition[篇]{s.}{revisão; comentário; artigos ou comentários críticos}
    \definition{v.}{discutir; comentar sobre algo ou alguém}
  \end{phonetics}
\end{entry}

\begin{entry}{评估}{7,7}{⾔、⼈}
  \begin{phonetics}{评估}{ping2gu1}[][HSK 5]
    \definition{v.}{estimar; avaliar; apreciar; avaliar e estimar (coisas abstratas)}
  \end{phonetics}
\end{entry}

\begin{entry}{诅咒}{7,8}{⾔、⼝}
  \begin{phonetics}{诅咒}{zu3zhou4}
    \definition{v.}{amaldiçoar}
  \end{phonetics}
\end{entry}

\begin{entry}{诊断}{7,11}{⾔、⽄}
  \begin{phonetics}{诊断}{zhen3duan4}[][HSK 5]
    \definition{s.}{diagnóstico; diacrisis}
    \definition{v.}{diagnosticar; após examinar os sintomas do paciente, determinar a doença e seu desenvolvimento}
  \end{phonetics}
\end{entry}

\begin{entry}{词}{7}{⾔}
  \begin{phonetics}{词}{ci2}[][HSK 2]
    \definition[个,组,句,段,首]{s.}{palavra; termo; antigamente, referia-se a palavras vazias; atualmente, refere-se a palavras com forma fonética fixa e significado específico na língua; a menor unidade que pode ser usada de forma independente | discurso; declaração; linguagem; texto | ci (um tipo de poesia clássica chinesa, originária da dinastia Tang e plenamente desenvolvida na dinastia Song); gênero poético escrito de acordo com uma estrutura fixa, com versos de comprimentos variados | palavras; redação; refere-se genericamente ao teatro; a parte da letra cantada em harmonia com a melodia em canções e certas artes vocais}
  \end{phonetics}
\end{entry}

\begin{entry}{词汇}{7,5}{⾔、⽔}
  \begin{phonetics}{词汇}{ci2hui4}[][HSK 4]
    \definition[个,组,批,串,堆]{s.}{vocabulário; termo geral para palavras usadas em um idioma}
  \end{phonetics}
\end{entry}

\begin{entry}{词典}{7,8}{⾔、⼋}
  \begin{phonetics}{词典}{ci2dian3}[][HSK 2]
    \definition[本,部]{s.}{dicionário, livro de referência que reúne palavras e explicações para consulta}
  \seealsoref{字典}{zi4 dian3}
  \end{phonetics}
\end{entry}

\begin{entry}{词语}{7,9}{⾔、⾔}
  \begin{phonetics}{词语}{ci2yu3}[][HSK 2]
    \definition[个,租]{s.}{termo; palavra; expressão; conjunto de palavras e frases}
  \end{phonetics}
\end{entry}

\begin{entry}{谷}{7}{⾕}[Kangxi 150]
  \begin{phonetics}{谷}{gu3}
    \definition{adj.}{bom; gentil;}
    \definition{s.}{vale; ravina; desfiladeiro; garganta; faixa estreita de terra com uma saída no meio de duas colinas ou dois platôs | arroz não descascado | salário de funcionário (na época feudal) |calha; cocho; canal | fossa sob o cerebelo (anatomia); valécula | dificuldade; dilema}
    \definition{v.}{criar (filhos) | crescer}
  \end{phonetics}
\end{entry}

\begin{entry}{豆角}{7,7}{⾖、⾓}
  \begin{phonetics}{豆角}{dou4jiao3}
    \definition{s.}{feijão verde}
  \end{phonetics}
\end{entry}

\begin{entry}{豆制品}{7,8,9}{⾖、⼑、⼝}
  \begin{phonetics}{豆制品}{dou4 zhi4 pin3}[][HSK 5]
    \definition{s.}{produtos de soja}
  \end{phonetics}
\end{entry}

\begin{entry}{豆荚}{7,9}{⾖、⾋}
  \begin{phonetics}{豆荚}{dou4jia2}
    \definition{s.}{vagem (de legumes)}
  \end{phonetics}
\end{entry}

\begin{entry}{豆腐}{7,14}{⾖、⾁}
  \begin{phonetics}{豆腐}{dou4fu5}[][HSK 4]
    \definition[块,盒,斤,盘,锅]{s.}{\emph{tofu}}
  \end{phonetics}
\end{entry}

\begin{entry}{财产}{7,6}{⾙、⼇}
  \begin{phonetics}{财产}{cai2chan3}[][HSK 4]
    \definition{s.}{ativos; propriedade; pertences; refere-se à posse de riqueza material, como dinheiro, bens, casas, terras, etc.}
  \end{phonetics}
\end{entry}

\begin{entry}{财富}{7,12}{⾙、⼧}
  \begin{phonetics}{财富}{cai2fu4}[][HSK 4]
    \definition{s.}{riqueza; fortuna}
  \end{phonetics}
\end{entry}

\begin{entry}{赤}{7}{⾚}[Kangxi 155]
  \begin{phonetics}{赤}{chi4}
    \definition*{s.}{sobrenome Chi}
    \definition{adj.}{vermelho; de cor vermelha | leal; sincero; de coração único | nu; sem roupa}
  \end{phonetics}
\end{entry}

\begin{entry}{走}{7}{⾛}[Kangxi 156]
  \begin{phonetics}{走}{zou3}[][HSK 1]
    \definition{v.}{andar; caminhar | correr | mover; movimentar; deslocar | sair; partir; ir embora | visitar; fazer uma visita; (entre amigos e familiares) troca de visitas | passar por; atravessar; ultrapassar | vazar; revelar; divulgar | afastar-se do original; alterar ou perder a forma, o sabor, a cor, etc. originais}
  \end{phonetics}
\end{entry}

\begin{entry}{走开}{7,4}{⾛、⼶}
  \begin{phonetics}{走开}{zou3 kai1}[][HSK 2]
    \definition{v.}{ir embora; fugir; ir para outro lugar}
  \end{phonetics}
\end{entry}

\begin{entry}{走去}{7,5}{⾛、⼛}
  \begin{phonetics}{走去}{zou3qu4}
    \definition{v.}{caminhar até (para)}
  \end{phonetics}
\end{entry}

\begin{entry}{走过}{7,6}{⾛、⾡}
  \begin{phonetics}{走过}{zou3 guo4}[][HSK 2]
    \definition{v.}{passar por; perambular}
  \end{phonetics}
\end{entry}

\begin{entry}{走秀}{7,7}{⾛、⽲}
  \begin{phonetics}{走秀}{zou3xiu4}
    \definition{s.}{desfile de moda}
    \definition{v.}{andar na passarela (em um desfile de moda)}
  \end{phonetics}
\end{entry}

\begin{entry}{走进}{7,7}{⾛、⾡}
  \begin{phonetics}{走进}{zou3 jin4}[][HSK 2]
    \definition{v.}{entrar}
  \end{phonetics}
\end{entry}

\begin{entry}{走势}{7,8}{⾛、⼒}
  \begin{phonetics}{走势}{zou3shi4}
    \definition{s.}{caminho | tendência}
  \end{phonetics}
\end{entry}

\begin{entry}{走卒}{7,8}{⾛、⼗}
  \begin{phonetics}{走卒}{zou3zu2}
    \definition{s.}{lacaio (masculino) | peão (isto é, soldado de infantaria) | servo}
  \end{phonetics}
\end{entry}

\begin{entry}{走鬼}{7,9}{⾛、⿁}
  \begin{phonetics}{走鬼}{zou3gui3}
    \definition{s.}{vendedor ambulante sem licença}
  \end{phonetics}
\end{entry}

\begin{entry}{走索}{7,10}{⾛、⽷}
  \begin{phonetics}{走索}{zou3suo3}
    \definition{v.}{andar na corda bamba}
  \seealsoref{走绳}{zou3sheng2}
  \end{phonetics}
\end{entry}

\begin{entry}{走绳}{7,11}{⾛、⽷}
  \begin{phonetics}{走绳}{zou3sheng2}
    \definition{v.}{andar na corda bamba}
  \seealsoref{走索}{zou3suo3}
  \end{phonetics}
\end{entry}

\begin{entry}{走路}{7,13}{⾛、⾜}
  \begin{phonetics}{走路}{zou3 lu4}[][HSK 1]
    \definition{v.}{caminhar; ir a pé; andar em pé sobre a terra | sair; ir embora; partir}
  \end{phonetics}
\end{entry}

\begin{entry}{足}{7}{⾜}[Kangxi 157]
  \begin{phonetics}{足}{ju4}
    \definition{adj.}{excessivo}
  \end{phonetics}
  \begin{phonetics}{足}{zu2}
    \definition{adj.}{amplo}
    \definition{s.}{pé}
    \definition{v.}{ser suficiente}
  \end{phonetics}
\end{entry}

\begin{entry}{足月}{7,4}{⾜、⽉}
  \begin{phonetics}{足月}{zu2yue4}
    \definition{s.}{gestação completa}
  \end{phonetics}
\end{entry}

\begin{entry}{足足}{7,7}{⾜、⾜}
  \begin{phonetics}{足足}{zu2zu2}
    \definition{adv.}{tanto quanto | extremamente | completamente | não menos que}
  \end{phonetics}
\end{entry}

\begin{entry}{足够}{7,11}{⾜、⼣}
  \begin{phonetics}{足够}{zu2 gou4}[][HSK 3]
    \definition{adj.}{bastante; amplo; suficiente; na medida em que deve ser ou pode atender às necessidades}
    \definition{v.}{satisfazer; ser suficiente; estar a contento}
  \end{phonetics}
\end{entry}

\begin{entry}{足球}{7,11}{⾜、⽟}
  \begin{phonetics}{足球}{zu2qiu2}[][HSK 3]
    \definition[个,只,颗,袋]{s.}{futebol | bola de futebol}
  \end{phonetics}
\end{entry}

\begin{entry}{足球队}{7,11,4}{⾜、⽟、⾩}
  \begin{phonetics}{足球队}{zu2qiu2dui4}
    \definition{s.}{time de futebol}
  \end{phonetics}
\end{entry}

\begin{entry}{足球协会}{7,11,6,6}{⾜、⽟、⼗、⼈}
  \begin{phonetics}{足球协会}{zu2qiu2xie2hui4}
    \definition*{s.}{Associação de Futebol}
  \end{phonetics}
\end{entry}

\begin{entry}{足球场}{7,11,6}{⾜、⽟、⼟}
  \begin{phonetics}{足球场}{zu2qiu2chang3}
    \definition{s.}{campo de futebol}
  \end{phonetics}
\end{entry}

\begin{entry}{足球迷}{7,11,9}{⾜、⽟、⾡}
  \begin{phonetics}{足球迷}{zu2qiu2mi2}
    \definition{s.}{fã de futebol}
  \end{phonetics}
\end{entry}

\begin{entry}{足球赛}{7,11,14}{⾜、⽟、⾙}
  \begin{phonetics}{足球赛}{zu2qiu2sai4}
    \definition{s.}{competição de futebol | partida de futebol}
  \end{phonetics}
\end{entry}

\begin{entry}{身上}{7,3}{⾝、⼀}
  \begin{phonetics}{身上}{shen1 shang5}[][HSK 1]
    \definition{s.}{no corpo de alguém | em um;  com um}
  \end{phonetics}
\end{entry}

\begin{entry}{身亡}{7,3}{⾝、⼇}
  \begin{phonetics}{身亡}{shen1wang2}
    \definition{v.}{morrer}
  \end{phonetics}
\end{entry}

\begin{entry}{身边}{7,5}{⾝、⾡}
  \begin{phonetics}{身边}{shen1 bian1}[][HSK 2]
    \definition{adv.}{ao redor; ao lado de alguém; perto do corpo | carregar consigo (transportar); à mão}
  \end{phonetics}
\end{entry}

\begin{entry}{身份}{7,6}{⾝、⼈}
  \begin{phonetics}{身份}{shen1fen4}[][HSK 4]
    \definition[种]{s.}{status; capacidade; identidade; refere-se à origem, ao status e às qualificações de uma pessoa | dignidade; posição honrada; referência especial ao status respeitável}
  \end{phonetics}
\end{entry}

\begin{entry}{身份证}{7,6,7}{⾝、⼈、⾔}
  \begin{phonetics}{身份证}{shen1 fen4 zheng4}[][HSK 3]
    \definition[张]{s.}{ID; bilhete de identidade; carteira de identidade}
  \end{phonetics}
\end{entry}

\begin{entry}{身体}{7,7}{⾝、⼈}
  \begin{phonetics}{身体}{shen1ti3}[][HSK 1]
    \definition[具,个]{s.}{corpo | saúde; saúde das pessoas}
  \end{phonetics}
\end{entry}

\begin{entry}{身体乳}{7,7,8}{⾝、⼈、⼄}
  \begin{phonetics}{身体乳}{shen1ti3 ru3}
    \definition{s.}{loção corporal}
  \end{phonetics}
\end{entry}

\begin{entry}{身体能力}{7,7,10,2}{⾝、⼈、⾁、⼒}
  \begin{phonetics}{身体能力}{shen1ti3 neng2li4}
    \definition{s.}{habilidade física}
  \end{phonetics}
\end{entry}

\begin{entry}{身材}{7,7}{⾝、⽊}
  \begin{phonetics}{身材}{shen1cai2}[][HSK 4]
    \definition[副,种,个,具]{s.}{figura; estatura; altura e peso corporal}
  \end{phonetics}
\end{entry}

\begin{entry}{身高}{7,10}{⾝、⾼}
  \begin{phonetics}{身高}{shen1 gao1}[][HSK 4]
    \definition[个,种,段]{s.}{estatura; altura (de uma pessoa);}
  \end{phonetics}
\end{entry}

\begin{entry}{辛苦}{7,8}{⾟、⾋}
  \begin{phonetics}{辛苦}{xin1ku3}[][HSK 5]
    \definition{adj.}{difícil; trabalhoso; árduo; descreve muito trabalho, alta intensidade e pouco descanso}
    \definition{s.}{dificuldades}
    \definition{v.}{trabalhar duro; passar por grandes dificuldades; passar por dificuldades}
  \end{phonetics}
\end{entry}

\begin{entry}{迎接}{7,11}{⾡、⼿}
  \begin{phonetics}{迎接}{ying2jie1}[][HSK 3]
    \definition{v.}{conhecer; cumprimentar; dar as boas-vindas}
  \end{phonetics}
\end{entry}

\begin{entry}{运}{7}{⾡}
  \begin{phonetics}{运}{yun4}[][HSK 5]
    \definition*{s.}{sobrenome Yun}
    \definition{s.}{sorte; destino; fortuna}
    \definition{v.}{mover; deslocar | transportar; levar | usar; empunhar; utilizar}
  \end{phonetics}
\end{entry}

\begin{entry}{运气}{7,4}{⾡、⽓}
  \begin{phonetics}{运气}{yun4qi5}[][HSK 4]
    \definition{adj.}{sortudo; afortunado}
    \definition{s.}{sorte; fortuna}
  \end{phonetics}
\end{entry}

\begin{entry}{运用}{7,5}{⾡、⽤}
  \begin{phonetics}{运用}{yun4yong4}[][HSK 4]
    \definition{v.}{usar; utilizar; manejar; aplicar; explorar as coisas de acordo com suas características}
  \end{phonetics}
\end{entry}

\begin{entry}{运动}{7,6}{⾡、⼒}
  \begin{phonetics}{运动}{yun4dong4}[][HSK 2]
    \definition[项,种,场,次]{s.}{esportes; atletismo; exercício; atividades esportivas | movimento; campanha (política); atividades de massa organizadas, intencionais e de alto nível na política, cultura, produção, etc. | movimento; refere-se a todas as mudanças}
    \definition{v.}{exercitar; fazer atividade física | mover-se; refere-se à mudança na posição de um objeto}
  \end{phonetics}
\end{entry}

\begin{entry}{运动会}{7,6,6}{⾡、⼒、⼈}
  \begin{phonetics}{运动会}{yun4 dong4 hui4}[][HSK 4]
    \definition[个]{s.}{jogos; encontro esportivo; dia de esportes; reunião atlética}
  \end{phonetics}
\end{entry}

\begin{entry}{运动场}{7,6,6}{⾡、⼒、⼟}
  \begin{phonetics}{运动场}{yun4dong4chang3}
    \definition{s.}{campo desportivo | campo de jogos}
  \end{phonetics}
\end{entry}

\begin{entry}{运动员}{7,6,7}{⾡、⼒、⼝}
  \begin{phonetics}{运动员}{yun4 dong4 yuan2}[][HSK 4]
    \definition[名,个]{s.}{jogador; atleta; esportista; pessoas que participam de competições esportivas}
  \end{phonetics}
\end{entry}

\begin{entry}{运动学}{7,6,8}{⾡、⼒、⼦}
  \begin{phonetics}{运动学}{yun4dong4xue2}
    \definition{s.}{cinemática}
  \end{phonetics}
\end{entry}

\begin{entry}{运动服}{7,6,8}{⾡、⼒、⽉}
  \begin{phonetics}{运动服}{yun4dong4fu2}
    \definition{s.}{roupa para prática de esporte}
  \end{phonetics}
\end{entry}

\begin{entry}{运动衫}{7,6,8}{⾡、⼒、⾐}
  \begin{phonetics}{运动衫}{yun4dong4shan1}
    \definition[件]{s.}{moletom | camisa esportiva}
  \end{phonetics}
\end{entry}

\begin{entry}{运动家}{7,6,10}{⾡、⼒、⼧}
  \begin{phonetics}{运动家}{yun4dong4jia1}
    \definition{s.}{ativista | atleta | esportista}
  \end{phonetics}
\end{entry}

\begin{entry}{运动病}{7,6,10}{⾡、⼒、⽧}
  \begin{phonetics}{运动病}{yun4dong4bing4}
    \definition{s.}{enjôo (movimento, carro, etc.)}
  \end{phonetics}
\end{entry}

\begin{entry}{运动鞋}{7,6,15}{⾡、⼒、⾰}
  \begin{phonetics}{运动鞋}{yun4dong4xie2}
    \definition{s.}{tênis | sapatos esportivos}
  \end{phonetics}
\end{entry}

\begin{entry}{运行}{7,6}{⾡、⾏}
  \begin{phonetics}{运行}{yun4xing2}[][HSK 5]
    \definition{v.}{(corpos celestes, etc.) mover-se ao longo do curso | (figurativo) funcionar, estar em operação | (serviço de trem, etc.) operar | (computador) executar um programa}
  \end{phonetics}
\end{entry}

\begin{entry}{运河}{7,8}{⾡、⽔}
  \begin{phonetics}{运河}{yun4he2}
    \definition{s.}{canal (em um rio)}
  \end{phonetics}
\end{entry}

\begin{entry}{运输}{7,13}{⾡、⾞}
  \begin{phonetics}{运输}{yun4shu1}[][HSK 3]
    \definition{v.}{enviar; transportar; usar um carro, navio, avião, etc. para transportar pessoas ou coisas de um lugar para outro}
  \end{phonetics}
\end{entry}

\begin{entry}{近}{7}{⾡}
  \begin{phonetics}{近}{jin4}[][HSK 2]
    \definition{adj.}{próximo; perto; distância espacial ou temporal curta (oposto de 远) | íntimo; intimamente relacionado; relação estreita | fácil de entender}
  \seealsoref{远}{yuan3}
  \end{phonetics}
\end{entry}

\begin{entry}{近代}{7,5}{⾡、⼈}
  \begin{phonetics}{近代}{jin4dai4}[][HSK 4]
    \definition{s.}{tempos modernos; era passada relativamente próxima à era moderna, geralmente referida na história chinesa como 1840 a 1919 | na história mundial, geralmente se refere à era capitalista}
  \end{phonetics}
\end{entry}

\begin{entry}{近来}{7,7}{⾡、⽊}
  \begin{phonetics}{近来}{jin4lai2}[][HSK 5]
    \definition{adv.}{ultimamente; recentemente; de ​​tarde; refere-se a um período de tempo entre o passado imediato e o presente}
  \end{phonetics}
\end{entry}

\begin{entry}{近期}{7,12}{⾡、⽉}
  \begin{phonetics}{近期}{jin4 qi1}[][HSK 3]
    \definition{adv.}{num futuro próximo; brevemente}
  \end{phonetics}
\end{entry}

\begin{entry}{返回}{7,6}{⾡、⼞}
  \begin{phonetics}{返回}{fan3 hui2}[][HSK 5]
    \definition{v.}{retornar; ir (voltar); reverter; recorrer; retroceder; voltar para (o lugar original)}
  \end{phonetics}
\end{entry}

\begin{entry}{还}{7}{⾡}
  \begin{phonetics}{还}{hai2}[][HSK 1]
    \definition{adv.}{ainda; indica que a ação ou estado permanece inalterado, equivalente a 仍然 | também; além disso; em adição; indica que há um aumento ou suplemento além do escopo já indicado | ainda mais; usado com 比 para indicar que as características e o grau das coisas comparadas aumentaram, o que é equivalente a 更加 razoavelmente; medianamente; usado antes de um adjetivo, indica que algo atinge apenas o nível mínimo exigido | mesmo; usado na primeira parte da frase como complemento, e na segunda parte como conclusão, equivalente a 尚且 | que expressa realização ou descoberta; expressa surpresa por algo que não se esperava, mas que acabou acontecendo | tão cedo quanto; por um curto período de tempo; indica que já era assim há muito tempo | para dar ênfase; para reforçar o tom}
  \seealsoref{比}{bi3}
  \seealsoref{更加}{geng4 jia1}
  \seealsoref{仍然}{reng2ran2}
  \seealsoref{尚且}{shang4qie3}
  \end{phonetics}
  \begin{phonetics}{还}{huan2}[][HSK 1]
    \definition*{s.}{sobrenome Huan}
    \definition{v.}{voltar; retornar; voltar ao lugar original ou restaurar o estado original | retribuir; devolver; reembolsar; devolver o dinheiro ou os bens emprestados ao seu proprietário | dar ou fazer algo em troca; retribuir as ações dos outros}
  \end{phonetics}
\end{entry}

\begin{entry}{还有}{7,6}{⾡、⽉}
  \begin{phonetics}{还有}{hai2 you3}[][HSK 1]
    \definition{adv.}{também; ainda; além disso; então novamente; enfatizar as partes complementares, excedentes ou não mencionadas além do que já é conhecido}
  \end{phonetics}
\end{entry}

\begin{entry}{还是}{7,9}{⾡、⽇}
  \begin{phonetics}{还是}{hai2shi5}[][HSK 1]
    \definition{adv.}{ainda; ainda assim; não é a continuação de um determinado estado, fenômeno ou ação; o resultado é o mesmo de antes, sem mudanças  |que expressa uma preferência por uma alternativa; expressa comparação ou escolha feita após consideração cuidadosa, frequentemente usado para fazer sugestões | que expressa realização ou descoberta; indica que o resultado final foi inesperado}
    \definition{conj.}{ou (somente para frases interrogativas); indica várias opções, geralmente usado em perguntas | tudo; se; não importa; independentemente de; significa que, independentemente das mudanças que ocorram, o resultado permanecerá o mesmo}
  \end{phonetics}
\end{entry}

\begin{entry}{这}{7}{⾡}
  \begin{phonetics}{这}{zhe4}[][HSK 1]
    \definition{pron.}{este, isto; substitui pessoas ou coisas que estão mais próximas | agora; em vez de 这时候, tem o efeito de reforçar a ênfase}
  \seealsoref{这时候}{zhe4 shi2 hou5}
  \end{phonetics}
  \begin{phonetics}{这}{zhei4}
    \definition{pron.}{(coloquial) este}
  \end{phonetics}
\end{entry}

\begin{entry}{这儿}{7,2}{⾡、⼉}
  \begin{phonetics}{这儿}{zhe4r5}[][HSK 1]
    \definition{pron.}{aqui | agora; neste momento (utilizado apenas após 打, 从, 由)}
  \seealsoref{从}{cong2}
  \seealsoref{打}{da3}
  \seealsoref{由}{you2}
  \end{phonetics}
\end{entry}

\begin{entry}{这个}{7,3}{⾡、⼈}
  \begin{phonetics}{这个}{zhe4ge5}
    \definition{pron.}{isto; este | isso; em vez das coisas mencionadas anteriormente | assim; tal; usado antes de verbos e adjetivos, indica um grau muito profundo, com um sentido exagerado | usado junto com 那个 para indicar pessoas ou objetos indefinidos}
  \seealsoref{那个}{na4ge5}
  \end{phonetics}
\end{entry}

\begin{entry}{这么}{7,3}{⾡、⼃}
  \begin{phonetics}{这么}{zhe4 me5}[][HSK 2]
    \definition{pron.}{tal (usado para mostrar o grau) | então (usado para mostrar exagero e exclamação) | desta forma; assim; formas de expressar ações | tal; indica quantidade}
  \end{phonetics}
\end{entry}

\begin{entry}{这边}{7,5}{⾡、⾡}
  \begin{phonetics}{这边}{zhe4 bian1}[][HSK 1]
    \definition{pron.}{aqui; deste lado; refere-se a um lugar próximo}
  \end{phonetics}
\end{entry}

\begin{entry}{这时}{7,7}{⾡、⽇}
  \begin{phonetics}{这时}{zhe4 shi2}[][HSK 2]
    \definition{adv.}{neste momento}
  \end{phonetics}
\end{entry}

\begin{entry}{这时候}{7,7,10}{⾡、⽇、⼈}
  \begin{phonetics}{这时候}{zhe4 shi2 hou5}[][HSK 2]
    \definition{adv.}{neste momento}
  \end{phonetics}
\end{entry}

\begin{entry}{这里}{7,7}{⾡、⾥}
  \begin{phonetics}{这里}{zhe4 li3}[][HSK 1]
    \definition{pron.}{aqui; pronomes demonstrativo, indicando locais próximos}
  \end{phonetics}
\end{entry}

\begin{entry}{这些}{7,8}{⾡、⼆}
  \begin{phonetics}{这些}{zhe4 xie1}[][HSK 1]
    \definition{pron.}{estes; pronome demonstrativo, que indicam duas ou mais pessoas ou coisas que estão próximas}
  \end{phonetics}
\end{entry}

\begin{entry}{这咱}{7,9}{⾡、⼝}
  \begin{phonetics}{这咱}{zhe4 zan5}
    \definition{s.}{agora; no momento; no presente | neste momento}
  \end{phonetics}
\end{entry}

\begin{entry}{这样}{7,10}{⾡、⽊}
  \begin{phonetics}{这样}{zhe4 yang4}[][HSK 2]
    \definition{pron.}{assim; tal; assim; deste jeito; pronome demonstrativo, que indica a natureza, estado, maneira, grau, etc.}
  \end{phonetics}
\end{entry}

\begin{entry}{这麽}{7,14}{⾡、⿇}
  \begin{phonetics}{这麽}{zhe4 me5}
    \variantof{这么}
  \end{phonetics}
\end{entry}

\begin{entry}{进}{7}{⾡}
  \begin{phonetics}{进}{jin4}[][HSK 1]
    \definition*{s.}{sobrenome Jin}
    \definition{clas.}{para seções em um edifício ou complexo residencial; qualquer uma das várias fileiras de casas em um complexo residencial de estilo antigo}
    \definition{s.}{(matemática) base de um sistema numérico}
    \definition{v.}{avançar; ir adiante; seguir em frente; (oposto a 退) | entrar; entrar em; entrar ou sair; (oposto a 出) | receber | comer; tomar; beber | submeter; apresentar | marcar um gol}
    \definition{v.aux.}{usado após um verbo, significa ``para dentro''}
  \seealsoref{出}{chu1}
  \seealsoref{退}{tui4}
  \end{phonetics}
\end{entry}

\begin{entry}{进一步}{7,1,7}{⾡、⼀、⽌}
  \begin{phonetics}{进一步}{jin4 yi2 bu4}[][HSK 3]
    \definition{adv.}{mais; dar um passo adiante; avançar um passo; indica que as coisas estão progredindo em um nível mais alto do que antes}
  \end{phonetics}
\end{entry}

\begin{entry}{进入}{7,2}{⾡、⼊}
  \begin{phonetics}{进入}{jin4 ru4}[][HSK 2]
    \definition{v.}{entrar; entrar em}
  \end{phonetics}
\end{entry}

\begin{entry}{进口}{7,3}{⾡、⼝}
  \begin{phonetics}{进口}{jin4kou3}[][HSK 4]
    \definition{adj.}{importado}
    \definition{s.}{importação; entrada de um edifício ou local, também chamada de 入口}
    \definition{v.+compl.}{importar; comprar ou transportar mercadorias de outro país ou região | entrar no porto; navegar em direção a um porto}
  \seealsoref{入口}{ru4kou3}
  \end{phonetics}
\end{entry}

\begin{entry}{进化}{7,4}{⾡、⼔}
  \begin{phonetics}{进化}{jin4hua4}[][HSK 5]
    \definition[个]{s.}{evolução; os organismos se desenvolvem e evoluem do simples para o complexo e de níveis baixos para altos}
    \definition{v.}{evoluir; um termo geral usado para descrever uma mudança gradual para melhor}
  \end{phonetics}
\end{entry}

\begin{entry}{进出口}{7,5,3}{⾡、⼐、⼝}
  \begin{phonetics}{进出口}{jin4chu1kou3}
    \definition{s.}{importação e exportação}
    \definition{v.}{importar e exportar}
  \end{phonetics}
\end{entry}

\begin{entry}{进去}{7,5}{⾡、⼛}
  \begin{phonetics}{进去}{jin4 qu4}[][HSK 1]
    \definition{v.}{entrar (a partir da minha localização)}
    \definition{v.aux.}{usado depois de um verbo, significa ``ir para dentro''; para um determinado intervalo ou período de tempo}
  \end{phonetics}
\end{entry}

\begin{entry}{进行}{7,6}{⾡、⾏}
  \begin{phonetics}{进行}{jin4xing2}[][HSK 2]
    \definition{v.}{continuar; estar em andamento; estar em progresso | fazer; conduzir; realizar; executar | marchar; avançar; prosseguir; estar em marcha}
  \end{phonetics}
\end{entry}

\begin{entry}{进行编程}{7,6,12,12}{⾡、⾏、⽷、⽲}
  \begin{phonetics}{进行编程}{jin4xing2bian1cheng2}
    \definition{s.}{programa de computador executável}
  \end{phonetics}
\end{entry}

\begin{entry}{进来}{7,7}{⾡、⽊}
  \begin{phonetics}{进来}{jin4 lai2}[][HSK 1]
    \definition{v.}{entrar (para a minha localização)}
  \end{phonetics}
\end{entry}

\begin{entry}{进步}{7,7}{⾡、⽌}
  \begin{phonetics}{进步}{jin4bu4}[][HSK 3]
    \definition{adj.}{progressivo; adequado às tendências da época; que impulsiona o desenvolvimento social (em oposição a 落后)}
    \definition{v.}{avançar; progredir; melhorar}
  \seealsoref{落后}{luo4hou4}
  \end{phonetics}
\end{entry}

\begin{entry}{进展}{7,10}{⾡、⼫}
  \begin{phonetics}{进展}{jin4zhan3}[][HSK 3]
    \definition{v.}{fazer progresso; progredir; avançar no desenvolvimento}
  \end{phonetics}
\end{entry}

\begin{entry}{远}{7}{⾡}
  \begin{phonetics}{远}{yuan3}[][HSK 1]
    \definition*{s.}{sobrenome Yuan}
    \definition{adj.}{distante (no tempo ou no espaço); longe; remoto; Longa distância espacial ou temporal (em oposição a 近) | (relações de parentesco) distante | com grande diferença}
    \definition{v.}{manter-se afastado de; não se aproximar}
  \seealsoref{近}{jin4}
  \end{phonetics}
\end{entry}

\begin{entry}{远天}{7,4}{⾡、⼤}
  \begin{phonetics}{远天}{yuan3tian1}
    \definition{s.}{paraíso | o céu distante}
  \end{phonetics}
\end{entry}

\begin{entry}{远方}{7,4}{⾡、⽅}
  \begin{phonetics}{远方}{yuan3fang1}
    \definition{s.}{longe | um local distante}
  \end{phonetics}
\end{entry}

\begin{entry}{远处}{7,5}{⾡、⼡}
  \begin{phonetics}{远处}{yuan3 chu4}[][HSK 5]
    \definition{s.}{distância; lugar distante}
  \end{phonetics}
\end{entry}

\begin{entry}{远远}{7,7}{⾡、⾡}
  \begin{phonetics}{远远}{yuan3yuan3}
    \definition{adv.}{de longe}
  \end{phonetics}
\end{entry}

\begin{entry}{远征}{7,8}{⾡、⼻}
  \begin{phonetics}{远征}{yuan3zheng1}
    \definition{s.}{uma expedição militar | marcha para regiões remotas}
  \end{phonetics}
\end{entry}

\begin{entry}{违反}{7,4}{⾡、⼜}
  \begin{phonetics}{违反}{wei2fan3}[][HSK 5]
    \definition{v.}{violar; transgredir; contrariar; não estar em conformidade (com as regras, regulamentos, etc.)}
  \end{phonetics}
\end{entry}

\begin{entry}{违法}{7,8}{⾡、⽔}
  \begin{phonetics}{违法}{wei2 fa3}[][HSK 5]
    \definition{v.}{ser ilegal; infringir a lei; violar a lei ou os regulamentos}
  \end{phonetics}
\end{entry}

\begin{entry}{违规}{7,8}{⾡、⾒}
  \begin{phonetics}{违规}{wei2 gui1}[][HSK 5]
    \definition{v.}{violar (regras); infringir as regras e regulamentos}
  \end{phonetics}
\end{entry}

\begin{entry}{违宪}{7,9}{⾡、⼧}
  \begin{phonetics}{违宪}{wei2xian4}
    \definition{adj.}{inconstitucional}
  \end{phonetics}
\end{entry}

\begin{entry}{连}{7}{⾡}
  \begin{phonetics}{连}{lian2}[][HSK 3]
    \definition*{s.}{sobrenome Lian}
    \definition{adv.}{em sucessão; um após o outro; repetidamente}
    \definition{prep.}{incluindo; incluido | até mesmo}
    \definition[个]{s.}{companhia; unidades organizacionais das forças armadas}
    \definition{v.}{ligar; juntar; conectar | envolver-se (em problemas); implicar; incriminar | costurar; coser}
  \end{phonetics}
\end{entry}

\begin{entry}{连忙}{7,6}{⾡、⼼}
  \begin{phonetics}{连忙}{lian2mang2}[][HSK 3]
    \definition{adv.}{imediatamente; de imediato; com pressa; apressadamente}
  \end{phonetics}
\end{entry}

\begin{entry}{连接}{7,11}{⾡、⼿}
  \begin{phonetics}{连接}{lian2 jie1}[][HSK 5]
    \definition[条]{s.}{conexão}
    \definition{v.}{ligar; unir; relacionar, conectar; anexar}
  \end{phonetics}
\end{entry}

\begin{entry}{连续}{7,11}{⾡、⽷}
  \begin{phonetics}{连续}{lian2xu4}[][HSK 3]
    \definition{adv.}{continuamente; sucessivamente; em uma fileira; um após o outro}
  \end{phonetics}
\end{entry}

\begin{entry}{连续剧}{7,11,10}{⾡、⽷、⼑}
  \begin{phonetics}{连续剧}{lian2 xu4 ju4}[][HSK 3]
    \definition[部,集]{s.}{série; novela; drama dividido em vários episódios, transmitido continuamente pela rádio ou televisão, com enredo contínuo}
  \end{phonetics}
\end{entry}

\begin{entry}{连锁反应}{7,12,4,7}{⾡、⾦、⼜、⼴}
  \begin{phonetics}{连锁反应}{lian2suo3fan3ying4}
    \definition{s.}{reação em cadeia}
  \end{phonetics}
\end{entry}

\begin{entry}{迟}{7}{⾡}
  \begin{phonetics}{迟}{chi2}[][HSK 5]
    \definition*{s.}{sobrenome Chi}
    \definition{adj.}{lento; tardio; demorado | atrasado | lento; obtuso}
  \end{phonetics}
\end{entry}

\begin{entry}{迟到}{7,8}{⾡、⼑}
  \begin{phonetics}{迟到}{chi2dao4}[][HSK 4]
    \definition{v.}{chegar atrasado; atrasar-se}
  \end{phonetics}
\end{entry}

\begin{entry}{邮包}{7,5}{⾢、⼓}
  \begin{phonetics}{邮包}{you2bao1}
    \definition{s.}{encomenda postal}
  \end{phonetics}
\end{entry}

\begin{entry}{邮市}{7,5}{⾢、⼱}
  \begin{phonetics}{邮市}{you2shi4}
    \definition{s.}{mercado postal}
  \end{phonetics}
\end{entry}

\begin{entry}{邮电}{7,5}{⾢、⽥}
  \begin{phonetics}{邮电}{you2dian4}
    \definition*{s.}{Correios e Telecomunicações}
  \end{phonetics}
\end{entry}

\begin{entry}{邮件}{7,6}{⾢、⼈}
  \begin{phonetics}{邮件}{you2 jian4}[][HSK 3]
    \definition[封,个]{s.}{correspondência; correio; assunto postal; um termo geral para cartas, encomendas, etc. recebidas, transportadas e entregues pelos correios | \emph{e-mail}; mensagens enviadas e recebidas por meio eletrônico}
  \end{phonetics}
\end{entry}

\begin{entry}{邮局}{7,7}{⾢、⼫}
  \begin{phonetics}{邮局}{you2ju2}[][HSK 4]
    \definition[家]{s.}{correio; agência dos correios; organizações que lidam com serviços postais}
  \end{phonetics}
\end{entry}

\begin{entry}{邮费}{7,9}{⾢、⾙}
  \begin{phonetics}{邮费}{you2fei4}
    \definition{s.}{postagem}
    \definition{v.}{postar}
  \end{phonetics}
\end{entry}

\begin{entry}{邮迷}{7,9}{⾢、⾡}
  \begin{phonetics}{邮迷}{you2mi2}
    \definition{s.}{filatelista | colecionador de selos}
  \end{phonetics}
\end{entry}

\begin{entry}{邮资}{7,10}{⾢、⾙}
  \begin{phonetics}{邮资}{you2zi1}
    \definition{s.}{postagem}
  \end{phonetics}
\end{entry}

\begin{entry}{邮递}{7,10}{⾢、⾡}
  \begin{phonetics}{邮递}{you2di4}
    \definition{v.}{enviar por correio}
  \end{phonetics}
\end{entry}

\begin{entry}{邮票}{7,11}{⾢、⽰}
  \begin{phonetics}{邮票}{you2 piao4}[][HSK 3]
    \definition[枚,张,套,版]{s.}{selo; selo postal; um \emph{voucher} vendido pelos correios e afixado na correspondência para indicar que a postagem foi paga}
  \end{phonetics}
\end{entry}

\begin{entry}{邮箱}{7,15}{⾢、⾋}
  \begin{phonetics}{邮箱}{you2 xiang1}[][HSK 3]
    \definition{s.}{caixa de correio | \emph{mailbox}; refere-se ao endereço de \emph{e-mail}}
  \end{phonetics}
\end{entry}

\begin{entry}{邻居}{7,8}{⾢、⼫}
  \begin{phonetics}{邻居}{lin2ju1}[][HSK 5]
    \definition[个,位,家]{s.}{vizinho; pessoas ou famílias que moram muito perto}
  \end{phonetics}
\end{entry}

\begin{entry}{里}{7}{⾥}[Kangxi 166]
  \begin{phonetics}{里}{li3}[][HSK 1]
    \definition*{s.}{sobrenome Li}
    \definition{clas.}{li, uma unidade chinesa de comprimento (= 1/2 quilômetro)}
    \definition{s.}{forro; revestimento; interior; parte de trás do tecido | interno; dentro; no interior | vizinhança; vizinhos | cidade natal; local de origem}
  \end{phonetics}
\end{entry}

\begin{entry}{里头}{7,5}{⾥、⼤}
  \begin{phonetics}{里头}{li3 tou5}[][HSK 2]
    \definition{s.}{dentro}
  \end{phonetics}
\end{entry}

\begin{entry}{里边}{7,5}{⾥、⾡}
  \begin{phonetics}{里边}{li3 bian5}[][HSK 1]
    \definition{s.}{em; dentro; no interior}
  \end{phonetics}
\end{entry}

\begin{entry}{里面}{7,9}{⾥、⾯}
  \begin{phonetics}{里面}{li3 mian4}[][HSK 3]
    \definition{s.}{dentro; interior}
  \end{phonetics}
\end{entry}

\begin{entry}{里斯本}{7,12,5}{⾥、⽄、⽊}
  \begin{phonetics}{里斯本}{li3si1ben3}
    \definition*{s.}{Lisboa}
  \end{phonetics}
\end{entry}

\begin{entry}{里斯本大学}{7,12,5,3,8}{⾥、⽄、⽊、⼤、⼦}
  \begin{phonetics}{里斯本大学}{li3si1ben3 da4xue2}
    \definition*{s.}{Universidade de Lisboa}
  \end{phonetics}
\end{entry}

\begin{entry}{针}{7}{⾦}
  \begin{phonetics}{针}{zhen1}[][HSK 4]
    \definition*{s.}{sobrenome Zhen}
    \definition[根]{s.}{agulha; ferramentas para costura de roupas | objetos semelhantes a agulhas; algo longo e fino como uma agulha | injeção | ponto de costura | pontos de acupuntura na medicina chinesa}
  \end{phonetics}
\end{entry}

\begin{entry}{针对}{7,5}{⾦、⼨}
  \begin{phonetics}{针对}{zhen1dui4}[][HSK 4]
    \definition{prep.}{em conexão com; de acordo com; à luz de; introdução de objetos de comportamento com uma finalidade clara}
    \definition{v.}{contrariar; apontar para; ter como objetivo; ser direcionado contra; fazer algo especificamente sobre um problema ou uma pessoa}
  \end{phonetics}
\end{entry}

\begin{entry}{闲}{7}{⾨}
  \begin{phonetics}{闲}{xian2}[][HSK 5]
    \definition{adj.}{ocioso; não ocupado; desocupado; sem coisas para fazer; sem atividades; tempo livre | desocupado; (casa, objeto, etc.) não em uso; ocioso | não oficial; não sério; não relacionado ao negócio}
    \definition{s.}{lazer; tempo livre}
  \end{phonetics}
\end{entry}

\begin{entry}{间}{7}{⾨}
  \begin{phonetics}{间}{jian1}[][HSK 1]
    \definition{clas.}{a menor unidade de uma casa; a menor unidade habitacional; cômodo}
    \definition{s.}{espaço entre duas partes  | (em um) tempo ou espaço definido | sala; quarto | uma seção de uma sala ou o espaço lateral entre dois pares de pilares | com um tempo ou espaço definido}
  \end{phonetics}
  \begin{phonetics}{间}{jian4}
    \definition{s.}{espaço entre as duas partes; abertura; lacuna}
    \definition{v.}{separar | semear a discórdia | desbastar (mudas); podar; remover ou arrancar as mudas em excesso}
  \end{phonetics}
\end{entry}

\begin{entry}{间或}{7,8}{⾨、⼽}
  \begin{phonetics}{间或}{jian4huo4}
    \definition{adv.}{às vezes | ocasionalmente | de vez em quando}
  \end{phonetics}
\end{entry}

\begin{entry}{间接}{7,11}{⾨、⼿}
  \begin{phonetics}{间接}{jian4jie1}[][HSK 5]
    \definition{adj.}{indireto; de segunda mão; em oposição a 直接}
  \seealsoref{直接}{zhi2jie1}
  \end{phonetics}
\end{entry}

\begin{entry}{闷热}{7,10}{⾨、⽕}
  \begin{phonetics}{闷热}{men1re4}
    \definition{adj.}{abafado | quente e abafado | sufocantemente quente | quente e sensual}
  \end{phonetics}
\end{entry}

\begin{entry}{阻止}{7,4}{⾩、⽌}
  \begin{phonetics}{阻止}{zu3zhi3}[][HSK 4]
    \definition{v.}{parar; reter; conter; interromper; impedir o avanço; impedir o movimento; obstruir}
  \end{phonetics}
\end{entry}

\begin{entry}{阻击}{7,5}{⾩、⼐}
  \begin{phonetics}{阻击}{zu3ji1}
    \definition{v.}{verificar | parar}
  \end{phonetics}
\end{entry}

\begin{entry}{阻碍}{7,13}{⾩、⽯}
  \begin{phonetics}{阻碍}{zu3'ai4}[][HSK 5]
    \definition{s.}{obstáculo; impedimento; barreira}
    \definition{v.}{bloquear; impedir; obstruir; impedir o bom andamento ou desenvolvimento}
  \end{phonetics}
\end{entry}

\begin{entry}{阿}{7}{⾩}
  \begin{phonetics}{阿}{a1}
    \definition{pref.}{em dialetos do sul para formar termos carinhosos, antes de nomes de animais de estimação, sobrenomes monossilábicos ou números que denotam ordem de antiguidade em uma; anexado a 大, 二, 三,\dots\ para indicar classificação (e, às vezes, intimidade) | antes dos termos de parentesco; na frente de um sobrenome, de um nome próprio ou de um determinado título, com uma conotação de intimidade | em alguns contextos, pode soar infantil ou muito informal (por exemplo, chamar um colega de trabalho por ``阿 + Nome'' sem intimidade)}[阿妈___mamãe | 阿明 ___forma carinhosa de chamar alguém chamado Ming]
  \end{phonetics}
  \begin{phonetics}{阿}{e1}
    \definition*{s.}{sobrenome E}
    \definition*{s.}{Dong'e (um condado na província de Shandong)}
    \definition{s.}{grande monte (ou colina) | um lugar sinuoso (montanha, água, etc.)}
    \definition{v.}{bajular; satisfazer}
  \end{phonetics}
\end{entry}

\begin{entry}{阿姨}{7,9}{⾩、⼥}
  \begin{phonetics}{阿姨}{a1yi2}[][HSK 4]
    \definition[个,位]{s.}{tia; uma forma de tratamento para uma mulher da geração dos pais; dirigir-se a uma mulher que tem aproximadamente a mesma idade da sua mãe, geralmente não é parente | babá em uma família; professora em um jardim de infância | tia; irmã da mãe (mais comum no sul da China)}[阿姨,生日快乐!___Tia, feliz aniversário! | 阿姨,这个苹果多少钱一斤?___Tia/Senhora, quanto custa o quilo dessas maçãs? | 阿姨,我想喝水。___Tia/Babá, eu quero beber água.]
  \end{phonetics}
\end{entry}

\begin{entry}{阿哥}{7,10}{⾩、⼝}
  \begin{phonetics}{阿哥}{a1ge1}
    \definition{s.}{irmão mais velho (afetivo)}[阿哥,帮我拿一下书包___Irmão, ajude-me com minha mochila escolar!]
  \end{phonetics}
\end{entry}

\begin{entry}{附件}{7,6}{⾩、⼈}
  \begin{phonetics}{附件}{fu4jian4}[][HSK 5]
    \definition*{s.}{\emph{Adnexa Uteri} ; refere-se à genitália interna feminina que não seja o útero, as trompas de falópio e os ovários}
    \definition{s.}{apêndice; documentos que acompanham o documento principal | acessório; anexo; peças ou sobressalentes que não sejam peças principais de máquinas e equipamentos | anexo; documentos ou itens relevantes emitidos com o documento}
  \end{phonetics}
\end{entry}

\begin{entry}{附近}{7,7}{⾩、⾡}
  \begin{phonetics}{附近}{fu4jin4}[][HSK 4]
    \definition{adj.}{perto; vizinho}
    \definition{s.}{vizinhança; bairro}
  \end{phonetics}
\end{entry}

\begin{entry}{陆地}{7,6}{⾩、⼟}
  \begin{phonetics}{陆地}{lu4di4}[][HSK 4]
    \definition[块,片]{s.}{terra; terra seca (em oposição ao mar); superfície da Terra, excluindo os oceanos (e, às vezes, rios e lagos)}
  \end{phonetics}
\end{entry}

\begin{entry}{陆续}{7,11}{⾩、⽷}
  \begin{phonetics}{陆续}{lu4xu4}[][HSK 4]
    \definition{adv.}{sucessivamente; um após o outro; intermitentemente}
  \end{phonetics}
\end{entry}

\begin{entry}{陆路}{7,13}{⾩、⾜}
  \begin{phonetics}{陆路}{lu4lu4}
    \definition{s.}{rota terrestre}
  \end{phonetics}
\end{entry}

\begin{entry}{饭}{7}{⾷}
  \begin{phonetics}{饭}{fan4}[][HSK 1]
    \definition{s.}{(empréstimo linguístico) fã, devoto}
    \definition[顿,碗]{s.}{cereais cozidos; grãos cozidos | refeição; alimentos consumidos diariamente em horários regulares | trabalho; meio de subsistência; meio de vida}
  \end{phonetics}
\end{entry}

\begin{entry}{饭店}{7,8}{⾷、⼴}
  \begin{phonetics}{饭店}{fan4dian4}[][HSK 1]
    \definition[家,个]{s.}{restaurante | hotel; hotel grande e bem equipado}
  \end{phonetics}
\end{entry}

\begin{entry}{饭馆}{7,11}{⾷、⾷}
  \begin{phonetics}{饭馆}{fan4 guan3}[][HSK 2]
    \definition[家,个]{s.}{restaurante; lanchonete}
  \end{phonetics}
\end{entry}

\begin{entry}{饮食}{7,9}{⾷、⾷}
  \begin{phonetics}{饮食}{yin3shi2}[][HSK 5]
    \definition{s.}{dieta; alimentos e bebidas}
    \definition{v.}{comer; beber}
  \end{phonetics}
\end{entry}

\begin{entry}{饮料}{7,10}{⾷、⽃}
  \begin{phonetics}{饮料}{yin3liao4}[][HSK 5]
    \definition[杯,瓶,种]{s.}{bebida; drinque; líquidos processados e fabricados para consumo, como vinho, chá, refrigerantes, suco de laranja, etc.}
  \end{phonetics}
\end{entry}

\begin{entry}{驱}{7}{⾺}
  \begin{phonetics}{驱}{qu1}
    \definition{v.}{expulsar | repelir}
  \end{phonetics}
\end{entry}

\begin{entry}{驴}{7}{⾺}
  \begin{phonetics}{驴}{lv2}
    \definition[头]{s.}{burro | asno | jumento | jegue}
  \end{phonetics}
\end{entry}

\begin{entry}{鸡}{7}{⿃}
  \begin{phonetics}{鸡}{ji1}[][HSK 2]
    \definition*{s.}{sobrenome Ji}
    \definition[只]{s.}{galo, galinha, frango | palavra ofensiva para uma mulher que ganha dinheiro fazendo sexo com um homem}
  \end{phonetics}
\end{entry}

\begin{entry}{鸡蛋}{7,11}{⿃、⾍}
  \begin{phonetics}{鸡蛋}{ji1dan4}[][HSK 1]
    \definition[个,枚,筐,箱,打]{s.}{ovo de galinha}
  \end{phonetics}
\end{entry}

\begin{entry}{麦当劳}{7,6,7}{⿆、⼹、⼒}
  \begin{phonetics}{麦当劳}{mai4dang1lao2}
    \definition*{s.}{McDonald's (empresa de \emph{fast-food})}
  \seealsoref{麦当劳叔叔}{mai4dang1lao2 shu1shu5}
  \end{phonetics}
\end{entry}

\begin{entry}{麦当劳叔叔}{7,6,7,8,8}{⿆、⼹、⼒、⼜、⼜}
  \begin{phonetics}{麦当劳叔叔}{mai4dang1lao2 shu1shu5}
    \definition*{s.}{Ronald McDonald}
  \seealsoref{麦当劳}{mai4dang1lao2}
  \end{phonetics}
\end{entry}

\begin{entry}{麦淇淋}{7,11,11}{⿆、⽔、⽔}
  \begin{phonetics}{麦淇淋}{mai4qi2lin2}
    \definition{s.}{(empréstimo linguístico) margarina}
  \end{phonetics}
\end{entry}

\begin{entry}{龟速}{7,10}{⿔、⾡}
  \begin{phonetics}{龟速}{gui1su4}
    \definition{adv.}{tão lento quanto uma tartaruga}
  \end{phonetics}
\end{entry}

%%%%% EOF %%%%%

