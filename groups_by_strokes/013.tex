%%%
%%% 13画
%%%

\section*{13画}\addcontentsline{toc}{section}{13画}

\begin{entry}{傻}{13}{⼈}
  \begin{phonetics}{傻}{sha3}[][HSK 5]
    \definition{adj.}{estúpido; confuso; burro; idiota; inflexível}
  \end{phonetics}
\end{entry}

\begin{entry}{傻瓜}{13,5}{⼈、⽠}
  \begin{phonetics}{傻瓜}{sha3gua1}
    \definition{adj.}{tolo | burro | simplório | idiota}
    \definition{v.}{enganar | iludir | lograr}
  \end{phonetics}
\end{entry}

\begin{entry}{傻眼}{13,11}{⼈、⽬}
  \begin{phonetics}{傻眼}{sha3yan3}
    \definition{adj.}{estupefato | atordoado}
  \end{phonetics}
\end{entry}

\begin{entry}{像}{13}{⼈}
  \begin{phonetics}{像}{xiang4}[][HSK 2]
    \definition{s.}{imagem | retrato | aparência}
    \definition{v.}{assemelhar-se | ser como}
  \end{phonetics}
\end{entry}

\begin{entry}{勤奋}{13,8}{⼒、⼤}
  \begin{phonetics}{勤奋}{qin2fen4}[][HSK 5]
    \definition{adj.}{diligente; assíduo; trabalhador; descreve alguém que se esforça continuamente nos estudos ou no trabalho}
  \end{phonetics}
\end{entry}

\begin{entry}{嗄}{13}{⼝}
  \begin{phonetics}{嗄}{a2}
    \definition{adj.}{rouco}
    \variantof{啊}
  \end{phonetics}
\end{entry}

\begin{entry}{嗅}{13}{⼝}
  \begin{phonetics}{嗅}{xiu4}
    \definition{v.}{cheirar; farejar; identificar odores pelo nariz}
  \end{phonetics}
\end{entry}

\begin{entry}{嗡嗡}{13,13}{⼝、⼝}
  \begin{phonetics}{嗡嗡}{weng1weng1}
    \definition{s.}{zumbido}
    \definition{v.}{zumbir}
  \end{phonetics}
\end{entry}

\begin{entry}{嘟}{13}{⼝}
  \begin{phonetics}{嘟}{du1}
    \definition{s.}{buzina | bip}
    \definition{v.}{fazer beicinho}
  \end{phonetics}
\end{entry}

\begin{entry}{塑料}{13,10}{⼟、⽃}
  \begin{phonetics}{塑料}{su4 liao4}[][HSK 4]
    \definition[块,种]{s.}{plástico; compostos de polímeros feitos de resinas naturais ou sintéticas como componente principal}
  \end{phonetics}
\end{entry}

\begin{entry}{塑料袋}{13,10,11}{⼟、⽃、⾐}
  \begin{phonetics}{塑料袋}{su4liao4dai4}[][HSK 4]
    \definition{s.}{saco plástico; sacola de plástico}
  \end{phonetics}
\end{entry}

\begin{entry}{填}{13}{⼟}
  \begin{phonetics}{填}{tian2}
    \definition{v.}{encher; rechear | reabastecer; suplementar; complementar | preencher; escrever dados em uma caixa (em um questionário ou formulário da \emph{Web})}
  \end{phonetics}
\end{entry}

\begin{entry}{填空}{13,8}{⼟、⽳}
  \begin{phonetics}{填空}{tian2kong4}[][HSK 4]
    \definition{v.}{preencher o espaço em branco (por exemplo, em um teste)}
  \end{phonetics}
\end{entry}

\begin{entry}{嫉妒}{13,7}{⼥、⼥}
  \begin{phonetics}{嫉妒}{ji2du4}
    \definition{v.}{estar com ciúmes de | invejar}
  \end{phonetics}
\end{entry}

\begin{entry}{幕}{13}{⼱}
  \begin{phonetics}{幕}{mu4}
    \definition{s.}{cortina ou tela | dossel ou tenda | quartel de um general | ato (de uma peça)}
  \end{phonetics}
\end{entry}

\begin{entry}{彀}{13}{⼸}
  \begin{phonetics}{彀}{gou4}
    \definition{s.}{calcance de um arco e flecha}
    \definition{v.}{puxar um arco ao máximo}
  \end{phonetics}
\end{entry}

\begin{entry}{微风}{13,4}{⼻、⾵}
  \begin{phonetics}{微风}{wei1feng1}
    \definition{s.}{brisa | vento leve}
  \end{phonetics}
\end{entry}

\begin{entry}{微软}{13,8}{⼻、⾞}
  \begin{phonetics}{微软}{wei1ruan3}
    \definition*{s.}{\emph{Microsoft Corporation}}
  \end{phonetics}
\end{entry}

\begin{entry}{微信}{13,9}{⼻、⼈}
  \begin{phonetics}{微信}{wei1 xin4}[][HSK 4]
    \definition*{s.}{WeChat; aplicativo gratuito lançado pela Tencent em 21 de janeiro de 2011 para fornecer serviços de mensagens instantâneas para terminais inteligentes}
  \end{phonetics}
\end{entry}

\begin{entry}{微型}{13,9}{⼻、⼟}
  \begin{phonetics}{微型}{wei1xing2}
    \definition{pref.}{micro-}
    \definition{s.}{miniatura}
  \end{phonetics}
\end{entry}

\begin{entry}{微笑}{13,10}{⼻、⽵}
  \begin{phonetics}{微笑}{wei1xiao4}[][HSK 4]
    \definition[个,丝]{s.}{sorriso;}
    \definition{v.}{sorrir}
  \end{phonetics}
\end{entry}

\begin{entry}{微博}{13,12}{⼻、⼗}
  \begin{phonetics}{微博}{wei1 bo2}[][HSK 5]
    \definition*{s.}{Weibo (um aplicativo de mídia social chinês)}
    \definition[条]{s.}{\emph{microblog}}
  \end{phonetics}
\end{entry}

\begin{entry}{想}{13}{⼼}
  \begin{phonetics}{想}{xiang3}[][HSK 1]
    \definition{v.}{pensar; ponderar; refletir | supor; contar; considerar; pensar; estimar | querer; gostaria de; sentir vontade (de fazer algo) | lembrar com saudade; sentir falta}
  \end{phonetics}
\end{entry}

\begin{entry}{想到}{13,8}{⼼、⼑}
  \begin{phonetics}{想到}{xiang3 dao4}[][HSK 2]
    \definition{v.}{pensar em | trazer à mente | ter no coração}
  \end{phonetics}
\end{entry}

\begin{entry}{想念}{13,8}{⼼、⼼}
  \begin{phonetics}{想念}{xiang3nian4}[][HSK 4]
    \definition{v.}{sentir falta; pensar em; lembrar com carinho; ficar doente por; desejar ver novamente; lembrar com saudade}
  \end{phonetics}
\end{entry}

\begin{entry}{想法}{13,8}{⼼、⽔}
  \begin{phonetics}{想法}{xiang3 fa3}[][HSK 2]
    \definition[个]{s.}{noção | opinião | jeito de pensar}
    \definition{s.}{maneira de pensar | opinião | noção}
    \definition{v.}{pensar em uma maneira (de fazer algo)}
  \end{phonetics}
\end{entry}

\begin{entry}{想起}{13,10}{⼼、⾛}
  \begin{phonetics}{想起}{xiang3 qi3}[][HSK 2]
    \definition{v.}{recordar | lembrar | pensar em | trazer à mente | cruzar pelos pensamentos de alguém | passar pelo pensamento de alguém}
  \end{phonetics}
\end{entry}

\begin{entry}{想象}{13,11}{⼼、⾗}
  \begin{phonetics}{想象}{xiang3xiang4}[][HSK 4]
    \definition[个]{s.}{imaginação; refere-se ao processo mental de processamento e transformação de representações armazenadas na mente para formar novas imagens}
    \definition{v.}{imaginar; vislumbrar; visualizar; refere-se a ter uma imagem concreta de algo que não está na frente dos olhos}
  \end{phonetics}
\end{entry}

\begin{entry}{想想看}{13,13,9}{⼼、⼼、⽬}
  \begin{phonetics}{想想看}{xiang3xiang3kan4}
    \definition{v.}{pensar sobre isso}
  \end{phonetics}
\end{entry}

\begin{entry}{愁}{13}{⼼}
  \begin{phonetics}{愁}{chou2}[][HSK 5]
    \definition{adj.}{triste; pesaroso; angustiado; desconsolado}
    \definition{s.}{pesar; sofrimento; dor; tristeza}
    \definition{v.}{preocupar-se; estar preocuoado; ficar ansioso; sentir ansiedade}
  \end{phonetics}
\end{entry}

\begin{entry}{愈}{13}{⼼}
  \begin{phonetics}{愈}{yu4}
    \definition{adv.}{mais e mais | ainda mais}
    \definition{v.}{recuperar | curar}
  \end{phonetics}
\end{entry}

\begin{entry}{意义}{13,3}{⼼、⼂}
  \begin{phonetics}{意义}{yi4yi4}[][HSK 3]
    \definition[个]{s.}{sentido; significado; significado expresso por palavras ou outros sinais; significado indicado por comportamento ou aquisição |valor; efeito; significância; impacto}
  \end{phonetics}
\end{entry}

\begin{entry}{意见}{13,4}{⼼、⾒}
  \begin{phonetics}{意见}{yi4jian4}[][HSK 2]
    \definition[点,条]{s.}{reclamação | ideia | objeção | opinião | sugestão}
  \end{phonetics}
\end{entry}

\begin{entry}{意外}{13,5}{⼼、⼣}
  \begin{phonetics}{意外}{yi4wai4}[][HSK 3]
    \definition{adj.}{inesperado; imprevisto}
    \definition{adv.}{acidentalmente}
    \definition[个]{s.}{acidente; infortúnio; infortúnio inesperado}
  \end{phonetics}
\end{entry}

\begin{entry}{意志}{13,7}{⼼、⼼}
  \begin{phonetics}{意志}{yi4zhi4}[][HSK 5]
    \definition[个,股]{s.}{vontade; determinação; desejo; força de vontade}
  \end{phonetics}
\end{entry}

\begin{entry}{意识}{13,7}{⼼、⾔}
  \begin{phonetics}{意识}{yi4shi2}[][HSK 5]
    \definition{s.}{consciência}
    \definition{s.}{consciência; percepção; grau de reconhecimento e importância atribuído a uma determinada questão}
    \definition{v.}{perceber; despertar para; estar ciente de; sentir, descobrir o que antes não se sentia ou não se descobria; geralmente é usado junto com 到}
  \seealsoref{到}{dao4}
  \end{phonetics}
\end{entry}

\begin{entry}{意译}{13,7}{⼼、⾔}
  \begin{phonetics}{意译}{yi4yi4}
    \definition{s.}{tradução livre | significado (de expressão estrangeira) | paráfrase | tradução do significado (em oposição à tradução literal)}
  \seealsoref{直译}{zhi2yi4}
  \end{phonetics}
\end{entry}

\begin{entry}{意味着}{13,8,11}{⼼、⼝、⽬}
  \begin{phonetics}{意味着}{yi4wei4zhe5}[][HSK 5]
    \definition{v.}{significar; subentender}
  \end{phonetics}
\end{entry}

\begin{entry}{意思}{13,9}{⼼、⼼}
  \begin{phonetics}{意思}{yi4si5}[][HSK 2]
    \definition[个]{s.}{interesse}
  \end{phonetics}
\end{entry}

\begin{entry}{意指}{13,9}{⼼、⼿}
  \begin{phonetics}{意指}{yi4zhi3}
    \definition{v.}{implicar | significar}
  \end{phonetics}
\end{entry}

\begin{entry}{感兴趣}{13,6,15}{⼼、⼋、⾛}
  \begin{phonetics}{感兴趣}{gan3xing4qu4}[][HSK 4]
    \definition{v.}{estar interessado}
  \seealsoref{对……感兴趣}{dui4 gan3xing4qu4}
  \end{phonetics}
\end{entry}

\begin{entry}{感动}{13,6}{⼼、⼒}
  \begin{phonetics}{感动}{gan3dong4}[][HSK 2]
    \definition{v.}{mover (alguém) | tocar (alguém emocionalmente)}
  \end{phonetics}
\end{entry}

\begin{entry}{感到}{13,8}{⼼、⼑}
  \begin{phonetics}{感到}{gan3 dao4}[][HSK 2]
    \definition{v.}{sentir; achar; perceber}
  \end{phonetics}
\end{entry}

\begin{entry}{感受}{13,8}{⼼、⼜}
  \begin{phonetics}{感受}{gan3shou4}[][HSK 3]
    \definition{s.}{percepção ; sentimento; experiência}
    \definition{v.}{sentir; sentir (através dos sentidos); experimentar}
  \end{phonetics}
\end{entry}

\begin{entry}{感冒}{13,9}{⼼、⽇}
  \begin{phonetics}{感冒}{gan3mao4}[][HSK 3]
    \definition{adj.}{interessado}
    \definition[场,次]{s.}{resfriado; resfriado comum; gripe}
    \definition{v.}{pegar (ter) um resfriado}
  \end{phonetics}
\end{entry}

\begin{entry}{感染}{13,9}{⼼、⽊}
  \begin{phonetics}{感染}{gan3ran3}
    \definition{s.}{infecção}
    \definition{v.}{infectar | (figurativo) influenciar}
  \end{phonetics}
\end{entry}

\begin{entry}{感觉}{13,9}{⼼、⾒}
  \begin{phonetics}{感觉}{gan3jue2}[][HSK 2]
    \definition[个]{s.}{sentimento; sensação; percepção sensorial;}
    \definition{v.}{sentir; perceber; tomar consciência; sentir no coração, acreditar}
  \end{phonetics}
\end{entry}

\begin{entry}{感情}{13,11}{⼼、⼼}
  \begin{phonetics}{感情}{gan3qing2}[][HSK 3]
    \definition[份,个,种]{s.}{emoção; sentimento | amor; afeição; apego}
  \end{phonetics}
\end{entry}

\begin{entry}{感谢}{13,12}{⼼、⾔}
  \begin{phonetics}{感谢}{gan3xie4}[][HSK 2]
    \definition{v.}{agradecer; ser grato; expressar gratidão com palavras ou ações}
  \end{phonetics}
\end{entry}

\begin{entry}{感想}{13,13}{⼼、⼼}
  \begin{phonetics}{感想}{gan3xiang3}[][HSK 5]
    \definition[个,条]{s.}{pensamentos; impressões; reflexões; resposta do pensamento decorrente da exposição ao mundo exterior}
  \end{phonetics}
\end{entry}

\begin{entry}{搞}{13}{⼿}
  \begin{phonetics}{搞}{gao3}[][HSK 5]
    \definition{v.}{fazer; realizar; estar envolvido em; engajar-se em um estudo, fazer algo em relação a, etc. | fazer; produzir; gerar; trabalhar | iniciar; estabelecer; organizar; configurar | consertar (mudar) alguém; fazer alguém sofrer | obter; assegurar; agarrar |  (seguido de um complemento) fazer com que se torne; produzir um determinado efeito ou resultado}
  \end{phonetics}
\end{entry}

\begin{entry}{搞好}{13,6}{⼿、⼥}
  \begin{phonetics}{搞好}{gao3 hao3}[][HSK 5]
    \definition{v.}{fazer um bom trabalho; fazer bem; suar; tornar submisso, tornar útil, por meio de solicitações e presentes amigáveis; amolecer}
  \end{phonetics}
\end{entry}

\begin{entry}{搞乱}{13,7}{⼿、⼄}
  \begin{phonetics}{搞乱}{gao3luan4}
    \definition{v.}{estragar | confundir | bagunçar}
  \end{phonetics}
\end{entry}

\begin{entry}{搞定}{13,8}{⼿、⼧}
  \begin{phonetics}{搞定}{gao3ding4}
    \definition{v.}{consertar | resolver}
  \end{phonetics}
\end{entry}

\begin{entry}{搞鬼}{13,9}{⼿、⿁}
  \begin{phonetics}{搞鬼}{gao3gui3}
    \definition{v.}{fazer travessuras | fazer truques}
  \end{phonetics}
\end{entry}

\begin{entry}{搞笑}{13,10}{⼿、⽵}
  \begin{phonetics}{搞笑}{gao3xiao4}
    \definition{adj.}{engraçado | hilário}
    \definition{v.}{fazer as pessoas rirem}
  \end{phonetics}
\end{entry}

\begin{entry}{搞通}{13,10}{⼿、⾡}
  \begin{phonetics}{搞通}{gao3tong1}
    \definition{v.}{entender algo}
  \end{phonetics}
\end{entry}

\begin{entry}{搞钱}{13,10}{⼿、⾦}
  \begin{phonetics}{搞钱}{gao3qian2}
    \definition{v.}{fazer dinheiro | acumular dinheiro}
  \end{phonetics}
\end{entry}

\begin{entry}{搞混}{13,11}{⼿、⽔}
  \begin{phonetics}{搞混}{gao3hun4}
    \definition{v.}{confundir}
  \end{phonetics}
\end{entry}

\begin{entry}{搞错}{13,13}{⼿、⾦}
  \begin{phonetics}{搞错}{gao3cuo4}
    \definition{v.}{cometer um erro}
  \end{phonetics}
\end{entry}

\begin{entry}{搬}{13}{⼿}
  \begin{phonetics}{搬}{ban1}[][HSK 3]
    \definition{v.}{copiar indiscriminadamente | mover-se (ou seja, mudar-se) | mover-se (algo relativamente pesado ou volumoso) | mudar | mudar-se}
  \end{phonetics}
\end{entry}

\begin{entry}{搬口}{13,3}{⼿、⼝}
  \begin{phonetics}{搬口}{ban1kou3}
    \definition{v.}{tagarelar | (idioma) transmitir histórias | semear dissensão | contar histórias}
  \end{phonetics}
\end{entry}

\begin{entry}{搬动}{13,6}{⼿、⼒}
  \begin{phonetics}{搬动}{ban1dong4}
    \definition{v.}{mover-se (alguma coisa) | se mudar}
  \end{phonetics}
\end{entry}

\begin{entry}{搬弄}{13,7}{⼿、⼶}
  \begin{phonetics}{搬弄}{ban1nong4}
    \definition{v.}{causar problemas | mexer com alguém | mostrar (o que se pode fazer)}
  \end{phonetics}
\end{entry}

\begin{entry}{搬走}{13,7}{⼿、⾛}
  \begin{phonetics}{搬走}{ban1zou3}
    \definition{v.}{carregar}
  \end{phonetics}
\end{entry}

\begin{entry}{搬运}{13,7}{⼿、⾡}
  \begin{phonetics}{搬运}{ban1yun4}
    \definition{s.}{frete | transporte}
    \definition{v.}{carregar | transportar}
  \end{phonetics}
\end{entry}

\begin{entry}{搬家}{13,10}{⼿、⼧}
  \begin{phonetics}{搬家}{ban1jia1}[][HSK 3]
    \definition{s.}{mudança}
    \definition{v.+compl.}{mudar-se de casa}
  \end{phonetics}
\end{entry}

\begin{entry}{摄氏}{13,4}{⼿、⽒}
  \begin{phonetics}{摄氏}{she4shi4}
    \definition{s.}{graus Celsius (°C), centígrado}
  \end{phonetics}
\end{entry}

\begin{entry}{摄像}{13,13}{⼿、⼈}
  \begin{phonetics}{摄像}{she4 xiang4}[][HSK 5]
    \definition{v.}{gravar; filmar; filmar com câmera; fazer uma gravação de vídeo (com uma câmera de vídeo ou TV)}
  \end{phonetics}
\end{entry}

\begin{entry}{摄像机}{13,13,6}{⼿、⼈、⽊}
  \begin{phonetics}{摄像机}{she4 xiang4 ji1}[][HSK 5]
    \definition[个,部]{s.}{câmera de vídeo; dispositivo que pode ser usado para converter imagens captadas em sinais de imagem de televisão}
  \end{phonetics}
\end{entry}

\begin{entry}{摄影}{13,15}{⼿、⼺}
  \begin{phonetics}{摄影}{she4ying3}[][HSK 5]
    \definition{s.}{fotografia}
    \definition{v.}{fotografar; tirar uma foto; tirar fotos ou filmar}
  \end{phonetics}
\end{entry}

\begin{entry}{摄影师}{13,15,6}{⼿、⼺、⼱}
  \begin{phonetics}{摄影师}{she4 ying3 shi1}[][HSK 5]
    \definition[个]{s.}{fotógrafo; cinegrafista; operador de câmera; técnico de fotografia em estúdio fotográfico}
  \end{phonetics}
\end{entry}

\begin{entry}{摆}{13}{⼿}
  \begin{phonetics}{摆}{bai3}[][HSK 4]
    \definition*{s.}{sobrenome Bai}
    \definition*{s.}{Festival de Ganbai; uma reunião realizada nas áreas Dai durante festivais religiosos, para celebrar uma boa colheita ou para trocar materiais; geralmente se refere a uma reunião em massa}
    \definition{s.}{pêndulo; dispositivo mecânico que controla a frequência de oscilação em relógios e instrumentos |  a bainha inferior de um vestido, jaqueta ou saia}
    \definition{v.}{colocar; posicionar; organizar | assumir; mostrar intencionalmente | balançar; ondular; balançar para frente e para trás | revelar; listar; afirmar claramente | dizer; falar; declarar | libertar-se; livrar-se}
  \end{phonetics}
\end{entry}

\begin{entry}{摆手}{13,4}{⼿、⼿}
  \begin{phonetics}{摆手}{bai3shou3}
    \definition{v.+compl.}{gesticular com a mão (acenando, acenando adeus, etc.) | balançar os braços | acenar com as mãos}
  \end{phonetics}
\end{entry}

\begin{entry}{摆动}{13,6}{⼿、⼒}
  \begin{phonetics}{摆动}{bai3 dong4}[][HSK 4]
    \definition{v.}{balançar; balançar para frente e para trás; oscilar; vibrar}
  \end{phonetics}
\end{entry}

\begin{entry}{摆烂}{13,9}{⼿、⽕}
  \begin{phonetics}{摆烂}{bai3lan4}
    \definition{v.}{(neologismo, gíria) parar de lutar (especialmente quando se sabe que não pode ter sucesso) | deixar tudo ir para o inferno}
  \end{phonetics}
\end{entry}

\begin{entry}{摆脱}{13,11}{⼿、⾁}
  \begin{phonetics}{摆脱}{bai3tuo1}[][HSK 4]
    \definition{v.}{sacudir; rejeitar; romper com; libertar-se (ou desembaraçar-se) de; livrar-se de dificuldades, escravidão, controle, etc.}
  \end{phonetics}
\end{entry}

\begin{entry}{摇}{13}{⼿}
  \begin{phonetics}{摇}{yao2}[][HSK 4]
    \definition{v.}{chacoalhar; ondular; balançar; fazer com que um objeto se mova para frente e para trás | agitar algo | sacudir; chacoalhar; agitar algo para que se mova}
  \end{phonetics}
\end{entry}

\begin{entry}{摇头}{13,5}{⼿、⼤}
  \begin{phonetics}{摇头}{yao2tou2}[][HSK 5]
    \definition{v.+compl.}{sacudir; balançar a cabeça; balançar a cabeça para a esquerda e para a direita, indicando negação, desacordo ou impedimento}
  \end{phonetics}
\end{entry}

\begin{entry}{摇晃}{13,10}{⼿、⽇}
  \begin{phonetics}{摇晃}{yao2huang4}
    \definition{v.}{sacudir | agitar | balançar | chacoalhar}
  \end{phonetics}
\end{entry}

\begin{entry}{摸}{13}{⼿}
  \begin{phonetics}{摸}{mo1}[][HSK 4]
    \definition{v.}{sentir; acariciar; tocar; tocar (um objeto) levemente com a mão e depois removê-lo ou mover a mão suavemente sobre a superfície do objeto | sentir para; tatear para; sentir algo com as mãos | descobrir; sentir; sondar; explorar; tentar fazer ou entender | sentir o caminho; tatear no escuro; andar por estradas que você não consegue reconhecer | furtar; roubar}
  \end{phonetics}
\end{entry}

\begin{entry}{数}{13}{⽁}
  \begin{phonetics}{数}{shu3}[][HSK 2]
    \definition{v.}{contar (número); contar (número) um a um | ser considerado excepcionalmente (bom, ruim, etc.) | enumerar; listar}
  \end{phonetics}
  \begin{phonetics}{数}{shu4}
    \definition{num.}{vários; alguns}
    \definition{s.}{número; cifra; figura | número (conceito matemático básico que representa a quantidade de coisas) | número; indica a quantidade de coisas a que se referem os substantivos ou pronomes | destino; sorte}
  \end{phonetics}
  \begin{phonetics}{数}{shuo4}
    \definition{adv.}{com frequência; repetidamente; indica uma ação frequente, equivalente a 屡次}
  \seealsoref{屡次}{lv3ci4}
  \end{phonetics}
\end{entry}

\begin{entry}{数目}{13,5}{⽁、⽬}
  \begin{phonetics}{数目}{shu4 mu4}[][HSK 5]
    \definition{s.}{número; quantidade; quantidade de algo expressa em uma determinada medida padrão (como unidades de medida, etc.)}
  \end{phonetics}
\end{entry}

\begin{entry}{数字}{13,6}{⽁、⼦}
  \begin{phonetics}{数字}{shu4zi4}[][HSK 2]
    \definition{adj.}{digital}
    \definition[个]{s.}{dígito | figura | número | numeral | quantidade | montante}
  \end{phonetics}
\end{entry}

\begin{entry}{数学}{13,8}{⽁、⼦}
  \begin{phonetics}{数学}{shu4xue2}
    \definition{s.}{matemática (disciplina)}
  \end{phonetics}
\end{entry}

\begin{entry}{数码}{13,8}{⽁、⽯}
  \begin{phonetics}{数码}{shu4ma3}[][HSK 4]
    \definition{s.}{dígito; numeral; algarismo | número; quantidade (usado principalmente na linguagem falada)}
    \definition{v.}{digitalizar}
  \end{phonetics}
\end{entry}

\begin{entry}{数据}{13,11}{⽁、⼿}
  \begin{phonetics}{数据}{shu4ju4}[][HSK 4]
    \definition[些,个]{s.}{dados; valores com base nos quais são realizadas estatísticas, cálculos, pesquisas científicas ou projetos técnicos}
  \end{phonetics}
\end{entry}

\begin{entry}{数量}{13,12}{⽁、⾥}
  \begin{phonetics}{数量}{shu4liang4}[][HSK 3]
    \definition[个]{s.}{quantidade; quantum; quantia; magnitude; número}
  \end{phonetics}
\end{entry}

\begin{entry}{新}{13}{⽄}
  \begin{phonetics}{新}{xin1}[][HSK 1]
    \definition*{s.}{sobrenome Xin}
    \definition*{s.}{abreviação de Xinjiang (新疆)}
    \definition*{s.}{abreviação de Singapura (新加坡)}
    \definition{adj.}{novo; fresco; inovador; atualizado; aparecer ou ser experimentado pela primeira vez | nunca utilizado; novo; não foi usado ou foi usado por pouco tempo | recém-casado}
    \definition{adv.}{recém; recentemente; há pouco tempo}
    \definition{pref.}{(química) meso-}
    \definition{v.}{atualizar; renovar}
  \seealsoref{新加坡}{xin1jia1po1}
  \seealsoref{新疆}{xin1jiang1}
  \end{phonetics}
\end{entry}

\begin{entry}{新加坡}{13,5,8}{⽄、⼒、⼟}
  \begin{phonetics}{新加坡}{xin1jia1po1}
    \definition*{s.}{Singapura}
  \end{phonetics}
\end{entry}

\begin{entry}{新年}{13,6}{⽄、⼲}
  \begin{phonetics}{新年}{xin1 nian2}[][HSK 1]
    \definition*[个]{s.}{Ano Novo}
  \end{phonetics}
\end{entry}

\begin{entry}{新郎}{13,8}{⽄、⾢}
  \begin{phonetics}{新郎}{xin1lang2}[][HSK 4]
    \definition[位,个]{s.}{noivo; homens no momento do casamento}
  \end{phonetics}
\end{entry}

\begin{entry}{新型}{13,9}{⽄、⼟}
  \begin{phonetics}{新型}{xin1 xing2}[][HSK 4]
    \definition[种]{s.}{ultimo modelo; novo tipo; novo padrão; novo estilo}
  \end{phonetics}
\end{entry}

\begin{entry}{新闻}{13,9}{⽄、⾨}
  \begin{phonetics}{新闻}{xin1wen2}[][HSK 2]
    \definition[条,个]{s.}{notícia}
  \end{phonetics}
\end{entry}

\begin{entry}{新娘}{13,10}{⽄、⼥}
  \begin{phonetics}{新娘}{xin1niang2}[][HSK 4]
    \definition[位,个]{s.}{noiva; a mulher no momento do casamento}
  \seealsoref{新娘子}{xin1niang2zi5}
  \end{phonetics}
\end{entry}

\begin{entry}{新娘子}{13,10,3}{⽄、⼥、⼦}
  \begin{phonetics}{新娘子}{xin1niang2zi5}
    \definition{s.}{noiva}
  \seealsoref{新娘}{xin1niang2}
  \end{phonetics}
\end{entry}

\begin{entry}{新娘服装}{13,10,8,12}{⽄、⼥、⽉、⾐}
  \begin{phonetics}{新娘服装}{xin1niang2 fu2zhuang1}
    \definition{s.}{roupas de noiva}
  \end{phonetics}
\end{entry}

\begin{entry}{新鲜}{13,14}{⽄、⿂}
  \begin{phonetics}{新鲜}{xin1xian1}
    \definition{adj.}{fresco (experiência, alimento, etc.)}
    \definition{s.}{frescor}
  \end{phonetics}
\end{entry}

\begin{entry}{新疆}{13,19}{⽄、⼸}
  \begin{phonetics}{新疆}{xin1jiang1}
    \definition*{s.}{Xinjiang}
  \end{phonetics}
\end{entry}

\begin{entry}{新疆维吾尔自治区}{13,19,11,7,5,6,8,4}{⽄、⼸、⽷、⼝、⼩、⾃、⽔、⼖}
  \begin{phonetics}{新疆维吾尔自治区}{xin1jiang1 wei2wu2'er3 zi4zhi4qu1}
    \definition*{s.}{Região Autônoma Uigur de Xinjiang}
  \end{phonetics}
\end{entry}

\begin{entry}{暖}{13}{⽇}
  \begin{phonetics}{暖}{nuan3}[][HSK 5]
    \definition{adj.}{caloroso; cordial}
    \definition{v.}{aquecer; esquentar; aquecer algo ou aquecer o corpo}
  \end{phonetics}
\end{entry}

\begin{entry}{暖气}{13,4}{⽇、⽓}
  \begin{phonetics}{暖气}{nuan3qi4}[][HSK 4]
    \definition[个]{s.}{aquecedor; aquecimento; aquecimento central}
  \end{phonetics}
\end{entry}

\begin{entry}{暖和}{13,8}{⽇、⼝}
  \begin{phonetics}{暖和}{nuan3huo5}[][HSK 3]
    \definition{adj.}{morno; agradável e quente}
    \definition{v.}{aquecer}
  \end{phonetics}
\end{entry}

\begin{entry}{暗}{13}{⽇}
  \begin{phonetics}{暗}{an4}[][HSK 4]
    \definition{adj.}{escuro; opaco; sem graça; pouca luz | escondido; secreto; não revelado | pouco claro; nebuloso; vago; confuso | subterrâneo}
    \definition{adv.}{secretamente | no escuro}
  \end{phonetics}
\end{entry}

\begin{entry}{暗示}{13,5}{⽇、⽰}
  \begin{phonetics}{暗示}{an4shi4}[][HSK 4]
    \definition[个]{s.}{sugestão; insinuação; intimação; (psicologia) refere-se ao uso de palavras, gestos, expressões, etc. para fazer as pessoas aceitarem involuntariamente uma determinada opinião ou fazerem algo}
    \definition{v.}{dar uma dica; sugerir secretamente; indicar algo a alguém usando outras palavras, expressões faciais ou gestos sem dizer em voz alta}
  \end{phonetics}
\end{entry}

\begin{entry}{暗香}{13,9}{⽇、⾹}
  \begin{phonetics}{暗香}{an4xiang1}
    \definition{s.}{fragrância sutil}
  \end{phonetics}
\end{entry}

\begin{entry}{暗恋}{13,10}{⽇、⼼}
  \begin{phonetics}{暗恋}{an4lian4}
    \definition{s.}{amor secreto}
    \definition{v.}{estar secretamente apaixonado por}
  \end{phonetics}
\end{entry}

\begin{entry}{楼}{13}{⽊}
  \begin{phonetics}{楼}{lou2}[][HSK 1]
    \definition*{s.}{sobrenome Lou}
    \definition{clas.}{andar, piso}
    \definition[层,座,栋]{s.}{um prédio com muitos andares | piso; andar | superestrutura; uma estrutura com um convés superior; um andar adicional construído sobre uma casa ou outro edifício | nome usado para certas lojas ou locais de entretenimento | arco ornamental; certas construções decorativas altas com passagens por baixo}
  \end{phonetics}
\end{entry}

\begin{entry}{楼上}{13,3}{⽊、⼀}
  \begin{phonetics}{楼上}{lou2 shang4}[][HSK 1]
    \definition{s.}{no andar de cima | autor anterior em um tópico do fórum; em plataformas como fóruns na internet, refere-se à pessoa que se manifesta antes de você.}
  \end{phonetics}
\end{entry}

\begin{entry}{楼下}{13,3}{⽊、⼀}
  \begin{phonetics}{楼下}{lou2 xia4}[][HSK 1]
    \definition{s.}{no andar de baixo}
  \end{phonetics}
\end{entry}

\begin{entry}{楼梯}{13,11}{⽊、⽊}
  \begin{phonetics}{楼梯}{lou2 ti1}[][HSK 4]
    \definition[个]{s.}{escada; escadaria; degraus no meio de dois andares para permitir que as pessoas subam ou desçam as escadas}
  \end{phonetics}
\end{entry}

\begin{entry}{概念}{13,8}{⽊、⼼}
  \begin{phonetics}{概念}{gai4nian4}[][HSK 3]
    \definition[个]{s.}{ideia; noção; conceito; concepção}
  \end{phonetics}
\end{entry}

\begin{entry}{概括}{13,9}{⽊、⼿}
  \begin{phonetics}{概括}{gai4kuo4}[][HSK 4]
    \definition{adj.}{genérico; simples e claro, captando o conteúdo principal}
    \definition{s.}{generalização}
    \definition{v.}{generalizar; resumir}
  \end{phonetics}
\end{entry}

\begin{entry}{歇}{13}{⽋}
  \begin{phonetics}{歇}{xie1}[][HSK 5]
    \definition*{s.}{sobrenome Xie}
    \definition{s.}{um pouco de tempo}
    \definition{v.}{descansar; fazer uma pausa | parar (o trabalho); encerrar o expediente | dormir; ir para a cama}
  \end{phonetics}
\end{entry}

\begin{entry}{源}{13}{⽔}
  \begin{phonetics}{源}{yuan2}
    \definition*{s.}{sobrenome Yuan}
    \definition{s.}{nascente (de um rio); fonte | fonte; origem; causa}
    \definition{v.}{originar-se; provir de}
  \end{phonetics}
\end{entry}

\begin{entry}{滔天}{13,4}{⽔、⼤}
  \begin{phonetics}{滔天}{tao1tian1}
    \definition{adj.}{(ondas, raiva, desastres, crimes, etc.) imponente, avassalador, imenso}
  \end{phonetics}
\end{entry}

\begin{entry}{滚}{13}{⽔}
  \begin{phonetics}{滚}{gun3}[][HSK 5]
    \definition*{s.}{sobrenome Gun}
    \definition{adj.}{rolante | fervente | precipitado; torrencial}
    \definition{adv.}{muito; em um grau elevado}
    \definition{v.}{rolar; girar; virar | escapar; fugir; ir embora | ferver | amarrar; aparar; fazer bainha}
  \end{phonetics}
\end{entry}

\begin{entry}{滚轮}{13,8}{⽔、⾞}
  \begin{phonetics}{滚轮}{gun3lun2}
    \definition{s.}{pneu | dial rotativo | roda de rolagem (\emph{scroll})  (mouse de computador)}
  \end{phonetics}
\end{entry}

\begin{entry}{滚滚}{13,13}{⽔、⽔}
  \begin{phonetics}{滚滚}{gun3gun3}
    \definition*{s.}{Apelido para um panda}
    \definition{v.}{mover-se | rolar | fluir continuamente}
  \end{phonetics}
\end{entry}

\begin{entry}{满}{13}{⽔}
  \begin{phonetics}{满}{man3}[][HSK 2]
    \definition*{s.}{sobrenome Man}
    \definition*{s.}{etnia Manchu}
    \definition{adj.}{cheio; repleto; lotado; totalmente cheio; atingindo o limite da capacidade | tudo; inteiro; completo | presunçoso; complacente; orgulhoso}
    \definition{adv.}{muito; um tanto; bastante | completamente; inteiramente; perfeitamente}
    \definition{v.}{encher | sentir-se satisfeito; sentir que já é o suficiente | expirar; atingir o limite; atingir um determinado prazo ou limite}
  \end{phonetics}
\end{entry}

\begin{entry}{满分}{13,4}{⽔、⼑}
  \begin{phonetics}{满分}{man3fen1}
    \definition{s.}{pontuação completa}
  \end{phonetics}
\end{entry}

\begin{entry}{满足}{13,7}{⽔、⾜}
  \begin{phonetics}{满足}{man3zu2}[][HSK 3]
    \definition{v.}{estar satisfeito; contentar-se | satisfazer; causar satisfação; contentar}
  \end{phonetics}
\end{entry}

\begin{entry}{满意}{13,13}{⽔、⼼}
  \begin{phonetics}{满意}{man3yi4}[][HSK 2]
    \definition{adj.}{satisfeito; contente; gratificado}
    \definition{v.}{estar satisfeito; sentir-se contente; satisfazer os seus desejos; estar de acordo com os seus desejos}
  \end{phonetics}
\end{entry}

\begin{entry}{满满}{13,13}{⽔、⽔}
  \begin{phonetics}{满满}{man3man3}
    \definition{adj.}{completo | densamente empacotado}
  \end{phonetics}
\end{entry}

\begin{entry}{煎}{13}{⽕}
  \begin{phonetics}{煎}{jian1}
    \definition{v.}{fritar | refogar}
  \end{phonetics}
\end{entry}

\begin{entry}{煎饼}{13,9}{⽕、⾷}
  \begin{phonetics}{煎饼}{jian1bing3}
    \definition[张]{s.}{jianbing, crepe chinês | panqueca}
  \end{phonetics}
\end{entry}

\begin{entry}{煎蛋}{13,11}{⽕、⾍}
  \begin{phonetics}{煎蛋}{jian1dan4}
    \definition{s.}{ovos fritos}
  \end{phonetics}
\end{entry}

\begin{entry}{煤}{13}{⽕}
  \begin{phonetics}{煤}{mei2}[][HSK 5]
    \definition[吨,堆,块]{s.}{carvão; carvão vegetal; minério sólido preto}
  \end{phonetics}
\end{entry}

\begin{entry}{煤气}{13,4}{⽕、⽓}
  \begin{phonetics}{煤气}{mei2 qi4}[][HSK 5]
    \definition[把]{s.}{gás; gás de carvão; gás obtido a partir do processamento do carvão não tem cor nem odor, é tóxico e pode ser queimado ou utilizado como matéria-prima na indústria química | envenenamento por monóxido de carbono}
  \end{phonetics}
\end{entry}

\begin{entry}{照}{13}{⽕}
  \begin{phonetics}{照}{zhao4}[][HSK 3]
    \definition{adv.}{de acordo com; indica agir de acordo com o original ou um certo padrão}
    \definition{prep.}{em direção a; na direção de | de acordo com; conforme}
    \definition{s.}{imagem; fotografia | permissão; licença | brilho; iluminação}
    \definition{v.}{brilhar; acender; iluminar | refletir; espelhar; olhar para sua própria imagem em um espelho, etc. | filmar; fotografar; tirar uma foto (fotografia) | cuidar de; tomar conta de | notificar | contrastar | entender}
  \end{phonetics}
\end{entry}

\begin{entry}{照片}{13,4}{⽕、⽚}
  \begin{phonetics}{照片}{zhao4pian4}[][HSK 2]
    \definition[张,套,幅]{s.}{fotografia | foto}
  \end{phonetics}
\end{entry}

\begin{entry}{照片子}{13,4,3}{⽕、⽚、⼦}
  \begin{phonetics}{照片子}{zhao4pian4zi5}
    \definition{v.}{tirar um raio X}
  \end{phonetics}
\end{entry}

\begin{entry}{照片底版}{13,4,8,8}{⽕、⽚、⼴、⽚}
  \begin{phonetics}{照片底版}{zhao4pian4di3ban3}
    \definition{s.}{placa fotográfica}
  \end{phonetics}
\end{entry}

\begin{entry}{照亮}{13,9}{⽕、⼇}
  \begin{phonetics}{照亮}{zhao4liang4}
    \definition{s.}{iluminação}
    \definition{v.}{iluminar}
  \end{phonetics}
\end{entry}

\begin{entry}{照相}{13,9}{⽕、⽬}
  \begin{phonetics}{照相}{zhao4 xiang4}[][HSK 2]
    \definition{v.+compl.}{tirar fotografia}
  \end{phonetics}
\end{entry}

\begin{entry}{照相机}{13,9,6}{⽕、⽬、⽊}
  \begin{phonetics}{照相机}{zhao4xiang4ji1}
    \definition[个,架,部,台,只]{s.}{câmera/máquina fotográfica}
  \end{phonetics}
\end{entry}

\begin{entry}{照准}{13,10}{⽕、⼎}
  \begin{phonetics}{照准}{zhao4zhun3}
    \definition{s.}{solicitação concedida (uso formal em documento antigo)}
    \definition{v.}{mirar (arma)}
  \end{phonetics}
\end{entry}

\begin{entry}{照顾}{13,10}{⽕、⾴}
  \begin{phonetics}{照顾}{zhao4gu4}[][HSK 2]
    \definition{v.}{cuidar de | atender a | oferecer tratamento preferencial | (de um cliente) patrocinar | fazer compras em | dar consideração a | mostrar consideração por | levar em conta | fazer concessões para}
  \end{phonetics}
\end{entry}

\begin{entry}{照骗}{13,12}{⽕、⾺}
  \begin{phonetics}{照骗}{zhao4pian4}
    \definition{s.}{imagem alterada digitalmente; ``photoshopada''}
  \end{phonetics}
\end{entry}

\begin{entry}{照像}{13,13}{⽕、⼈}
  \begin{phonetics}{照像}{zhao4 xiang4}
    \variantof{照相}
  \end{phonetics}
\end{entry}

\begin{entry}{照像机}{13,13,6}{⽕、⼈、⽊}
  \begin{phonetics}{照像机}{zhao4xiang4ji1}
    \variantof{照相机}
  \end{phonetics}
\end{entry}

\begin{entry}{献}{13}{⽝}
  \begin{phonetics}{献}{xian4}[][HSK 5]
    \definition{v.}{oferecer; apresentar; dedicar; doar | mostrar; apresentar; exibir | exibir-se; mostrar-se para que os outros vejam}
  \end{phonetics}
\end{entry}

\begin{entry}{瑜伽}{13,7}{⽟、⼈}
  \begin{phonetics}{瑜伽}{yu2jia1}
    \definition*{s.}{Ioga}
  \end{phonetics}
\end{entry}

\begin{entry}{瑜珈}{13,9}{⽟、⽟}
  \begin{phonetics}{瑜珈}{yu2jia1}
    \variantof{瑜伽}
  \end{phonetics}
\end{entry}

\begin{entry}{睡}{13}{⽬}
  \begin{phonetics}{睡}{shui4}[][HSK 1]
    \definition{v.}{dormir | deitar-se}
  \end{phonetics}
\end{entry}

\begin{entry}{睡衣}{13,6}{⽬、⾐}
  \begin{phonetics}{睡衣}{shui4yi1}
    \definition{s.}{pijamas | roupas de dormir}
  \end{phonetics}
\end{entry}

\begin{entry}{睡觉}{13,9}{⽬、⾒}
  \begin{phonetics}{睡觉}{shui4jiao4}[][HSK 1]
    \definition{v.+compl.}{dormir; ir para a cama; entrar em estado de sono}
  \end{phonetics}
\end{entry}

\begin{entry}{睡眠}{13,10}{⽬、⽬}
  \begin{phonetics}{睡眠}{shui4 mian2}[][HSK 5]
    \definition{s.}{sono; \emph{somnus}; sonolência}
  \end{phonetics}
\end{entry}

\begin{entry}{睡着}{13,11}{⽬、⽬}
  \begin{phonetics}{睡着}{shui4 zhao2}[][HSK 4]
    \definition{v.}{dormir; adormecer; cair no sono}
  \end{phonetics}
\end{entry}

\begin{entry}{睡懒觉}{13,16,9}{⽬、⼼、⾒}
  \begin{phonetics}{睡懒觉}{shui4lan3jiao4}
    \definition{v.}{levantar-se tarde | passar o tempo a dormir}
  \end{phonetics}
\end{entry}

\begin{entry}{矮}{13}{⽮}
  \begin{phonetics}{矮}{ai3}[][HSK 4]
    \definition{adj.}{baixo em estatura, dimensão, grau ou ranque | curto (em comprimento)}[他比我矮。(Ele é mais baixo que eu.) | 这栋楼很矮,只有三层。(Esse prédio é baixo, tem só três andares.) | 她虽然矮,但是跑得很快!(Ela pode ser baixinha, mas corre muito rápido!)]
  \end{phonetics}
\end{entry}

\begin{entry}{矮人}{13,2}{⽮、⼈}
  \begin{phonetics}{矮人}{ai3ren2}
    \definition{s.}{anão; pessoa de baixa estatura (indivíduo) | homúnculo; figuras criadas artificialmente pelos alquimistas em frascos de destilação | nanismo}[他虽然是矮人,但很有力气。(Embora ele seja baixo, é muito forte.) | 北欧神话中的矮人是技艺高超的工匠。(Na mitologia nórdica, os anões são artesãos habilidosos.) | 他因为身高被嘲笑为‘矮人’,这让他很伤心。(Ele foi zombado por ser chamado de ‘anão’ devido à sua altura, o que o magoou.)]
  \end{phonetics}
\end{entry}

\begin{entry}{矮子}{13,3}{⽮、⼦}
  \begin{phonetics}{矮子}{ai3zi5}
    \definition{s.}{pessoa baixa; anão; baixinho}[白雪公主和七个小矮子住在森林里。(Branca de Neve e os sete anões vivem na floresta.) | 用`矮子'称呼他人是不礼貌的。(Chamar alguém de ``baixinho'' é falta de educação.)]
  \end{phonetics}
\end{entry}

\begin{entry}{矮小}{13,3}{⽮、⼩}
  \begin{phonetics}{矮小}{ai3 xiao3}[][HSK 4]
    \definition{adj.}{subdimensionado; curto e pequeno; baixo e pequeno | quando usado para pessoas, pode soar depreciativo se não for em contexto neutro ou afetuoso}[这位矮小的老人是村里的智者。(Este idoso baixinho é o sábio da vila.) | 这种矮小的灌木适合盆栽。(Este tipo de arbusto pequeno é ideal para vasos.) | 山脚下有一片矮小的房屋,显得格外宁静。(Ao pé da montanha, havia casas baixas que transmitiam uma tranquilidade única.)]
  \end{phonetics}
\end{entry}

\begin{entry}{矮林}{13,8}{⽮、⽊}
  \begin{phonetics}{矮林}{ai3lin2}
    \definition{s.}{mata rasteira | bosque baixo}[这片矮林里有很多野兔和鸟类。(Neste bosque baixo há muitos coelhos selvagens e pássaros.) | 山坡上长满了矮林,远看像绿色的地毯。(A encosta está coberta de mata rasteira, que de longe parece um tapete verde.)]
  \end{phonetics}
\end{entry}

\begin{entry}{矮星}{13,9}{⽮、⽇}
  \begin{phonetics}{矮星}{ai3xing1}
    \definition{s.}{estrela anã}[白矮星是恒星演化的最终阶段之一。(Anãs brancas são um dos estágios finais da evolução estelar.)]
  \end{phonetics}
\end{entry}

\begin{entry}{矮树}{13,9}{⽮、⽊}
  \begin{phonetics}{矮树}{ai3shu4}
    \definition{s.}{arbusto | árvore pequena, baixa}[矮树比高树更容易修剪。(Árvores baixas são mais fáceis de podar do que árvores altas.) | 我们种了些矮树作为花园的边界。(Plantamos alguns arbustos como cerca natural do jardim.)]
  \end{phonetics}
\end{entry}

\begin{entry}{矮胖}{13,9}{⽮、⾁}
  \begin{phonetics}{矮胖}{ai3pang4}
    \definition{adj.}{atarracado; gorducho; rechonchudo; roliço; baixo e robusto | chamar alguém diretamente de 矮胖 pode ser ofensivo}[我家猫矮胖矮胖的,像个毛球。(Meu gato é baixinho e gordinho, parece uma bolinha de pelo.)]
  \end{phonetics}
\end{entry}

\begin{entry}{矮凳}{13,14}{⽮、⼏}
  \begin{phonetics}{矮凳}{ai3deng4}
    \definition{s.}{banquinho baixo | banqueta}[这个矮凳是木制的,很结实。(Este banquinho é de madeira e bem resistente.)]
  \end{phonetics}
\end{entry}

\begin{entry}{碍事}{13,8}{⽯、⼅}
  \begin{phonetics}{碍事}{ai4shi4}
    \definition{s.}{(usualmente em frases negativas) sem consequência, não importa}
    \definition{v.+compl.}{estar no caminho | ser um obstáculo}
  \end{phonetics}
\end{entry}

\begin{entry}{碎}{13}{⽯}
  \begin{phonetics}{碎}{sui4}[][HSK 5]
    \definition*{s.}{sobrenome Sui}
    \definition{adj.}{quebrado; fragmentado | tagarela; falante}
    \definition{v.}{(transitivo ou intransitivo) quebrar em pedaços; esmagar}
  \end{phonetics}
\end{entry}

\begin{entry}{碗}{13}{⽯}
  \begin{phonetics}{碗}{wan3}[][HSK 2]
    \definition*{s.}{sobrenome Wan}
    \definition{clas.}{usado para medição de alimentos e bebidas}
    \definition[只,个]{s.}{tigela | objeto em forma de tigela |}
  \end{phonetics}
\end{entry}

\begin{entry}{碗子}{13,3}{⽯、⼦}
  \begin{phonetics}{碗子}{wan3zi5}
    \definition{s.}{tigela}
  \end{phonetics}
\end{entry}

\begin{entry}{碗柜}{13,8}{⽯、⽊}
  \begin{phonetics}{碗柜}{wan3gui4}
    \definition{s.}{armário}
  \end{phonetics}
\end{entry}

\begin{entry}{碰}{13}{⽯}
  \begin{phonetics}{碰}{peng4}[][HSK 2]
    \definition{v.}{tocar; bater; esbarrar | encontrar; esbarrar | arriscar; tentar | tentar a sorte | reunir-se para discutir; ter uma reunião curta}
  \end{phonetics}
\end{entry}

\begin{entry}{碰见}{13,4}{⽯、⾒}
  \begin{phonetics}{碰见}{peng4 jian4}[][HSK 2]
    \definition{v.}{encontrar; encontrar-se; sem combinar, encontrar-se por acaso}
  \end{phonetics}
\end{entry}

\begin{entry}{碰头}{13,5}{⽯、⼤}
  \begin{phonetics}{碰头}{peng4tou2}
    \definition{s.}{colisão | conflito}
    \definition{v.}{colidir}
    \definition{v.+compl.}{conhecer e discutir | juntar ideias | ver-se}
  \end{phonetics}
\end{entry}

\begin{entry}{碰运气}{13,7,4}{⽯、⾡、⽓}
  \begin{phonetics}{碰运气}{peng4yun4qi5}
    \definition{v.}{deixar algo ao acaso | tentar a sorte}
  \end{phonetics}
\end{entry}

\begin{entry}{碰到}{13,8}{⽯、⼑}
  \begin{phonetics}{碰到}{peng4 dao4}[][HSK 2]
    \definition{v.}{encontrar (com); esbarrar; cruzar}
  \end{phonetics}
\end{entry}

\begin{entry}{禁止}{13,4}{⽰、⽌}
  \begin{phonetics}{禁止}{jin4zhi3}[][HSK 4]
    \definition{v.}{banir; proibir; interditar}
  \end{phonetics}
\end{entry}

\begin{entry}{福}{13}{⽰}
  \begin{phonetics}{福}{fu2}[][HSK 3]
    \definition*{s.}{sobrenome Fu}
    \definition{s.}{benção; felicidade; boa sorte; boa fortuna}
    \definition{v.}{curvar-se; reverenciar}
  \end{phonetics}
\end{entry}

\begin{entry}{福克斯}{13,7,12}{⽰、⼗、⽄}
  \begin{phonetics}{福克斯}{fu2ke4si1}
    \definition*{s.}{Fox (empresa de mídia) | Focus (automóvel fabricado pela Ford)}
  \end{phonetics}
\end{entry}

\begin{entry}{福利}{13,7}{⽰、⼑}
  \begin{phonetics}{福利}{fu2li4}[][HSK 5]
    \definition{s.}{bem-estar; benefícios materiais}
    \definition{v.}{melhorar suas condições de vida; facilitar a vida}
  \end{phonetics}
\end{entry}

\begin{entry}{福泽}{13,8}{⽰、⽔}
  \begin{phonetics}{福泽}{fu2ze2}
    \definition{s.}{boa sorte}
  \end{phonetics}
\end{entry}

\begin{entry}{筷子}{13,3}{⽵、⼦}
  \begin{phonetics}{筷子}{kuai4zi5}[][HSK 2]
    \definition[根,双,副,把,对]{s.}{pauzinhos; \emph{chopsticks}; dois bastôes finos feitos de bambu, madeira, metal ou outro material, usados para segurar comida ou outros objetos}
  \end{phonetics}
\end{entry}

\begin{entry}{签}{13}{⽵}
  \begin{phonetics}{签}{qian1}[][HSK 5]
    \definition{s.}{tiras de bambu usadas para adivinhação ou sorteio; pPequenas tiras de bambu ou varas finas com caracteres e símbolos gravados, usadas para adivinhação, jogos de azar ou como fichas para contagem, etc. | etiqueta; adesivo; pequena tira usada como marca | um pedaço fino e pontiagudo de bambu ou madeira; pequeno bastão pontiagudo}
    \definition{v.}{assinar; autografar; escrever o nome, palavras ou fazer marcas em documentos ou recibos | fazer comentários breves em um documento; escrever brevemente (pontos principais ou opiniões) | (em costura) alinhavar; costura grosseira}
  \end{phonetics}
\end{entry}

\begin{entry}{签订}{13,4}{⽵、⾔}
  \begin{phonetics}{签订}{qian1 ding4}[][HSK 5]
    \definition{v.}{concluir e assinar (um tratado, etc.)}
  \end{phonetics}
\end{entry}

\begin{entry}{签名}{13,6}{⽵、⼝}
  \begin{phonetics}{签名}{qian1 ming2}[][HSK 5]
    \definition[个,次]{s.}{assinatura; autógrafo}
    \definition{v.+compl.}{assinar o próprio nome; autografar; escrever seu nome para indicar concordância, apoio ou homenagem, etc.}
  \end{phonetics}
\end{entry}

\begin{entry}{签字}{13,6}{⽵、⼦}
  \begin{phonetics}{签字}{qian1 zi4}[][HSK 5]
    \definition{v.}{assinar; colocar a assinatura; escrever seu nome à mão em documentos, recibos, etc., para demonstrar responsabilidade}
  \end{phonetics}
\end{entry}

\begin{entry}{签约}{13,6}{⽵、⽷}
  \begin{phonetics}{签约}{qian1 yue1}[][HSK 5]
    \definition{v.}{assinar um contrato; assinar contratos e tratados, frequentemente utilizado no trabalho e em cooperações comerciais}
  \end{phonetics}
\end{entry}

\begin{entry}{签证}{13,7}{⽵、⾔}
  \begin{phonetics}{签证}{qian1zheng4}[][HSK 5]
    \definition[张,个]{s.}{visto; visto de entrada em um país}
  \end{phonetics}
\end{entry}

\begin{entry}{简历}{13,4}{⽵、⼚}
  \begin{phonetics}{简历}{jian3li4}[][HSK 4]
    \definition[个,份]{s.}{currículo; \emph{curriculum vitae} (CV); notas biográficas}
  \end{phonetics}
\end{entry}

\begin{entry}{简单}{13,8}{⽵、⼗}
  \begin{phonetics}{简单}{jian3dan1}[][HSK 3]
    \definition{adj.}{simples; descomplicado | comum; lugar-comum | casual; simplificado}
  \end{phonetics}
\end{entry}

\begin{entry}{简直}{13,8}{⽵、⽬}
  \begin{phonetics}{简直}{jian3zhi2}[][HSK 3]
    \definition{adv.}{simplesmente; em tudo; virtualmente}
  \end{phonetics}
\end{entry}

\begin{entry}{粮食}{13,9}{⽶、⾷}
  \begin{phonetics}{粮食}{liang2shi5}[][HSK 4]
    \definition[种,斤,吨,袋]{s.}{alimentos; grãos; termo geral para os vários tipos de arroz, feijão, etc. que podem ser consumidos}
  \end{phonetics}
\end{entry}

\begin{entry}{缝纫}{13,6}{⽷、⽷}
  \begin{phonetics}{缝纫}{feng2ren4}
    \definition{v.}{costurar}
  \end{phonetics}
\end{entry}

\begin{entry}{缝纫机}{13,6,6}{⽷、⽷、⽊}
  \begin{phonetics}{缝纫机}{feng2ren4ji1}
    \definition[架]{s.}{máquina de costura}
  \end{phonetics}
\end{entry}

\begin{entry}{罪犯}{13,5}{⽹、⽝}
  \begin{phonetics}{罪犯}{zui4fan4}
    \definition{s.}{criminoso}
  \end{phonetics}
\end{entry}

\begin{entry}{罪行}{13,6}{⽹、⾏}
  \begin{phonetics}{罪行}{zui4xing2}
    \definition{s.}{crime | ofensa}
  \end{phonetics}
\end{entry}

\begin{entry}{置疑}{13,14}{⽹、⽦}
  \begin{phonetics}{置疑}{zhi4yi2}
    \definition{v.}{duvidar}
  \end{phonetics}
\end{entry}

\begin{entry}{群}{13}{⽺}
  \begin{phonetics}{群}{qun2}[][HSK 3]
    \definition{clas.}{grupo; rebanho; manada}
    \definition{s.}{multidão; grupo}
  \end{phonetics}
\end{entry}

\begin{entry}{群山}{13,3}{⽺、⼭}
  \begin{phonetics}{群山}{qun2shan1}
    \definition{s.}{montanhas | uma cadeia de colinas}
  \end{phonetics}
\end{entry}

\begin{entry}{群众}{13,6}{⽺、⼈}
  \begin{phonetics}{群众}{qun2zhong4}[][HSK 5]
    \definition[个,名,位]{s.}{as massas; refere-se ao povo em geral | não filiado; apartidário; refere-se a pessoas que não são membros do Partido Comunista Chinês nem da Liga da Juventude Comunista |
alguém que não ocupa uma posição de liderança}
  \end{phonetics}
\end{entry}

\begin{entry}{群体}{13,7}{⽺、⼈}
  \begin{phonetics}{群体}{qun2 ti3}[][HSK 5]
    \definition{s.}{colônia; um conjunto composto por muitos indivíduos da mesma espécie que estão fisicamente conectados, exemplos incluem corais entre os animais e certas algas entre as plantas | grupos; refere-se, de maneira geral, ao conjunto formado por muitos indivíduos interligados que compartilham características essenciais em comum}
  \end{phonetics}
\end{entry}

\begin{entry}{肆}{13}{⾀}
  \begin{phonetics}{肆}{si4}
    \definition*{s.}{sobrenome Si}
    \definition{adj.}{arbitrário; desenfreado; sem limites; descuidado; imprudente}
    \definition{num.}{quatro (usado para o numeral 四 em cheques, etc., para evitar erros ou alterações)}
    \definition{s.}{loja}
  \seealsoref{四}{si4}
  \end{phonetics}
\end{entry}

\begin{entry}{腰}{13}{⾁}
  \begin{phonetics}{腰}{yao1}[][HSK 4]
    \definition*{s.}{sobrenome Yao}
    \definition[个]{s.}{cintura; região lombar | cós | bolso | parte do meio das coisas | lombo}
  \end{phonetics}
\end{entry}

\begin{entry}{腰包}{13,5}{⾁、⼓}
  \begin{phonetics}{腰包}{yao1bao1}
    \definition{s.}{pochete | bolso}
  \end{phonetics}
\end{entry}

\begin{entry}{腰椎}{13,12}{⾁、⽊}
  \begin{phonetics}{腰椎}{yao1zhui1}
    \definition{s.}{vértebra lombar (espinha dorsal inferior)}
  \end{phonetics}
\end{entry}

\begin{entry}{腿}{13}{⾁}
  \begin{phonetics}{腿}{tui3}[][HSK 2]
    \definition[条,双]{s.}{perna; as partes dos humanos e dos animais que sustentam o corpo e permitem caminhar | um suporte em forma de perna; a parte inferior de um objeto que atua como uma perna e serve de suporte | presunto}
  \end{phonetics}
\end{entry}

\begin{entry}{腿号}{13,5}{⾁、⼝}
  \begin{phonetics}{腿号}{tui3hao4}
    \definition{s.}{anilha numerada (por exemplo, usada para identificar pássaros)}
  \seealsoref{腿号箍}{tui3hao4gu1}
  \end{phonetics}
\end{entry}

\begin{entry}{腿号箍}{13,5,14}{⾁、⼝、⽵}
  \begin{phonetics}{腿号箍}{tui3hao4gu1}
    \definition{s.}{anilha numerada (por exemplo, usada para identificar pássaros)}
  \seealsoref{腿号}{tui3hao4}
  \end{phonetics}
\end{entry}

\begin{entry}{艁}{13}{⾈}
  \begin{phonetics}{艁}{zao4}
    \variantof{造}
  \end{phonetics}
\end{entry}

\begin{entry}{蒙面}{13,9}{⾋、⾯}
  \begin{phonetics}{蒙面}{meng2mian4}
    \definition{adj.}{descarado | desavergonhado | mascarado}
    \definition{v.}{cobrir o rosto | usar uma máscara}
  \end{phonetics}
\end{entry}

\begin{entry}{蓝}{13}{⾋}
  \begin{phonetics}{蓝}{lan2}[][HSK 2]
    \definition*{s.}{sobrenome Lan}
    \definition{adj.}{azul}
    \definition{s.}{planta índigo; anil | plantas azuis; refere-se a certas plantas que podem ser usadas como corante azul ou certas plantas cujas folhas são azul-esverdeadas}
  \end{phonetics}
\end{entry}

\begin{entry}{蓝色}{13,6}{⾋、⾊}
  \begin{phonetics}{蓝色}{lan2 se4}[][HSK 2]
    \definition[抹,片,缕,团,块]{s.}{cor azul}
  \end{phonetics}
\end{entry}

\begin{entry}{解开}{13,4}{⾓、⼶}
  \begin{phonetics}{解开}{jie3 kai1}[][HSK 3]
    \definition{v.}{desatar; desfazer; desamarrar; desabotoar}
  \end{phonetics}
\end{entry}

\begin{entry}{解决}{13,6}{⾓、⼎}
  \begin{phonetics}{解决}{jie3jue2}[][HSK 3]
    \definition{v.}{solucionar; resolver; liquidar | acabar com; descartar}
  \end{phonetics}
\end{entry}

\begin{entry}{解压}{13,6}{⾓、⼚}
  \begin{phonetics}{解压}{jie3ya1}
    \definition{v.}{aliviar o estresse | (computação) descomprimir}
  \end{phonetics}
\end{entry}

\begin{entry}{解放}{13,8}{⾓、⽅}
  \begin{phonetics}{解放}{jie3fang4}[][HSK 5]
    \definition{v.}{libertar; emancipar; eliminar as restrições para permitir o desenvolvimento da liberdade}
  \end{phonetics}
\end{entry}

\begin{entry}{解除}{13,9}{⾓、⾩}
  \begin{phonetics}{解除}{jie3chu2}[][HSK 5]
    \definition{v.}{remover; aliviar; livrar-se de; eliminar}
  \end{phonetics}
\end{entry}

\begin{entry}{解救}{13,11}{⾓、⽁}
  \begin{phonetics}{解救}{jie3jiu4}
    \definition{v.}{resgatar | ajudar a sair de dificuldades | salvar a situação}
  \end{phonetics}
\end{entry}

\begin{entry}{解释}{13,12}{⾓、⾤}
  \begin{phonetics}{解释}{jie3shi4}[][HSK 4]
    \definition{v.}{explicar; expor; interpretar | analisar; explicaro significado, razões, justificativas, etc.}
  \end{phonetics}
\end{entry}

\begin{entry}{解雇}{13,12}{⾓、⾫}
  \begin{phonetics}{解雇}{jie3gu4}
    \definition{v.}{demitir}
  \end{phonetics}
\end{entry}

\begin{entry}{谩骂}{13,9}{⾔、⾺}
  \begin{phonetics}{谩骂}{man4ma4}
    \definition{v.}{ridicularizar | abusar}
  \end{phonetics}
\end{entry}

\begin{entry}{赖}{13}{⾙}
  \begin{phonetics}{赖}{lai4}
    \definition*{s.}{sobrenome Lai}
    \definition{v.}{depender | aguentar em um lugar | renegar (promessa) | isolar-se | culpar | colocar a culpa em}
  \end{phonetics}
\end{entry}

\begin{entry}{跟}{13}{⾜}
  \begin{phonetics}{跟}{gen1}[][HSK 1]
    \definition{conj.}{e; expressa uma relação de união; 和}
    \definition{prep.}{com; Introduzir objetos relacionados à mesma ação, equivalente a 同 | para; em direção a | de; introduzir o objeto de comparação; equivalente a 从, 由 | como; objetos que causam comparações e semelhanças}
    \definition[个]{s.}{calcanhar; parte posterior do pé ou parte posterior do sapato ou meia |
base (de um objeto)}
    \definition{v.}{seguir; acompanhar; seguir imediatamente na mesma direção | (de uma mulher) estar casada com; casar-se com alguém}
  \seealsoref{从}{cong2}
  \seealsoref{和}{he2}
  \seealsoref{同}{tong2}
  \seealsoref{由}{you2}
  \end{phonetics}
\end{entry}

\begin{entry}{跟前}{13,9}{⾜、⼑}
  \begin{phonetics}{跟前}{gen1qian2}[][HSK 5]
    \definition{s.}{próximo; perto de; na frente de; (na ou para) a presença de alguém | o tempo imediatamente anterior a algum evento; tempo que se aproxima}
  \end{phonetics}
  \begin{phonetics}{跟前}{gen1qian5}
    \definition{v.}{(dos filhos de alguém) viver com alguém (exclusivamente com relação à presença ou ausência de crianças)}
  \end{phonetics}
\end{entry}

\begin{entry}{跟随}{13,11}{⾜、⾩}
  \begin{phonetics}{跟随}{gen1sui2}[][HSK 5]
    \definition{s.}{seguidor; usado para se referir a alguém que seguiu}
    \definition{v.}{seguir; ir atrás; acompanhar}
  \end{phonetics}
\end{entry}

\begin{entry}{跪拜}{13,9}{⾜、⼿}
  \begin{phonetics}{跪拜}{gui4bai4}
    \definition{v.}{prostrar-se | ajoelhar-se e adorar}
  \end{phonetics}
\end{entry}

\begin{entry}{路}{13}{⾜}
  \begin{phonetics}{路}{lu4}[][HSK 1]
    \definition*{s.}{sobrenome Lu}
    \definition{clas.}{tipo; classe | linha; coluna; usado para um grupo de pessoas ou uma equipe; para organizar em ordem}
    \definition[条]{s.}{estrada; caminho; via | viagem; jornada; distância | maneira; meios | sequência; linha; lógica | região; distrito | rota | classe; classificação; grau | linha; fileira}
  \end{phonetics}
\end{entry}

\begin{entry}{路上}{13,3}{⾜、⼀}
  \begin{phonetics}{路上}{lu4 shang5}[][HSK 1]
    \definition{s.}{na estrada | a caminho; na rota; em processo de mudança de um lugar para outro}
  \end{phonetics}
\end{entry}

\begin{entry}{路口}{13,3}{⾜、⼝}
  \begin{phonetics}{路口}{lu4 kou3}[][HSK 1]
    \definition[个]{s.}{cruzamento; intersecção; onde as estradas se encontram}
  \end{phonetics}
\end{entry}

\begin{entry}{路边}{13,5}{⾜、⾡}
  \begin{phonetics}{路边}{lu4 bian1}[][HSK 2]
    \definition{s.}{calçada; beira da estrada; margem da rua}
  \end{phonetics}
\end{entry}

\begin{entry}{路线}{13,8}{⾜、⽷}
  \begin{phonetics}{路线}{lu4 xian4}[][HSK 3]
    \definition[条]{s.}{rota; caminho; linha | linha; diretriz (de política, ideologia, campo de trabalho)}
  \end{phonetics}
\end{entry}

\begin{entry}{跳}{13}{⾜}
  \begin{phonetics}{跳}{tiao4}[][HSK 3]
    \definition{v.}{pular; saltar; quicar | mover para cima e para baixo; pulsar; palpitar; contrair-se | pular; saltar por cima}
  \end{phonetics}
\end{entry}

\begin{entry}{跳水}{13,4}{⾜、⽔}
  \begin{phonetics}{跳水}{tiao4shui3}
    \definition{s.}{mergulho esportivo}
    \definition{v.}{mergulhar (na água) | cometer suicídio pulando na água | (figurativo, preços das ações, etc.) cair dramaticamente}
  \end{phonetics}
\end{entry}

\begin{entry}{跳电}{13,5}{⾜、⽥}
  \begin{phonetics}{跳电}{tiao4dian4}
    \definition{v.}{desarmar (um disjuntor ou interruptor)}
  \end{phonetics}
\end{entry}

\begin{entry}{跳伞}{13,6}{⾜、⼈}
  \begin{phonetics}{跳伞}{tiao4san3}
    \definition{s.}{paraquedas}
    \definition{v.}{saltar de paraquedas}
  \end{phonetics}
\end{entry}

\begin{entry}{跳远}{13,7}{⾜、⾡}
  \begin{phonetics}{跳远}{tiao4 yuan3}[][HSK 3]
    \definition{s.}{salto em distância (atletismo)}
  \end{phonetics}
\end{entry}

\begin{entry}{跳挡}{13,9}{⾜、⼿}
  \begin{phonetics}{跳挡}{tiao4dang3}
    \definition{v.}{pular marcha (de um carro) | perder a marcha}
  \end{phonetics}
\end{entry}

\begin{entry}{跳蚤}{13,9}{⾜、⾍}
  \begin{phonetics}{跳蚤}{tiao4zao5}
    \definition{s.}{pulga}
  \end{phonetics}
\end{entry}

\begin{entry}{跳高}{13,10}{⾜、⾼}
  \begin{phonetics}{跳高}{tiao4 gao1}[][HSK 3]
    \definition{s.}{salto em altura (atletismo)}
  \end{phonetics}
\end{entry}

\begin{entry}{跳绳}{13,11}{⾜、⽷}
  \begin{phonetics}{跳绳}{tiao4sheng2}
    \definition{v.}{pular corda}
  \end{phonetics}
\end{entry}

\begin{entry}{跳跳糖}{13,13,16}{⾜、⾜、⽶}
  \begin{phonetics}{跳跳糖}{tiao4tiao4tang2}
    \definition{s.}{\emph{Pop Rocks}, \emph{popping candy}}
  \end{phonetics}
\end{entry}

\begin{entry}{跳频}{13,13}{⾜、⾴}
  \begin{phonetics}{跳频}{tiao4pin2}
    \definition{s.}{FHSS, \emph{Frequency-Hopping Spread Spectrum}, método de transmissão de sinais de rádio}
  \end{phonetics}
\end{entry}

\begin{entry}{跳舞}{13,14}{⾜、⾇}
  \begin{phonetics}{跳舞}{tiao4wu3}[][HSK 3]
    \definition{v.+compl.}{dançar (como performance)}
  \end{phonetics}
\end{entry}

\begin{entry}{躲}{13}{⾝}
  \begin{phonetics}{躲}{duo3}[][HSK 5]
    \definition{v.}{esconder (a si mesmo); ocultar (a si mesmo); esconder-se | evitar; esquivar-se}
  \end{phonetics}
\end{entry}

\begin{entry}{躲闪}{13,5}{⾝、⾨}
  \begin{phonetics}{躲闪}{duo3shan3}
    \definition{v.}{desviar | evadir | esquivar (para fora do caminho)}
  \end{phonetics}
\end{entry}

\begin{entry}{输}{13}{⾞}
  \begin{phonetics}{输}{shu1}[][HSK 3]
    \definition{v.}{transportar; transmitir | contribuir com dinheiro; doar | perder; ser batido; ser derrotado}
  \end{phonetics}
\end{entry}

\begin{entry}{输入}{13,2}{⾞、⼊}
  \begin{phonetics}{输入}{shu1ru4}[][HSK 3]
    \definition{v.}{introduzir; importar  (de fora para dentro) | inserir informações, programas, dados, sinais, etc. em uma máquina}
  \end{phonetics}
\end{entry}

\begin{entry}{输出}{13,5}{⾞、⼐}
  \begin{phonetics}{输出}{shu1 chu1}[][HSK 5]
    \definition{v.}{exportar (de dentro para fora); transportar (de dentro) para fora | exportar; vender ou distribuir no exterior ou fora do país | emitir informações, programas, dados, sinais, etc. a partir de uma máquina; enviar por uma determinada instituição ou dispositivo (energia, sinal, etc.)}
  \end{phonetics}
\end{entry}

\begin{entry}{辞典}{13,8}{⾟、⼋}
  \begin{phonetics}{辞典}{ci2 dian3}[][HSK 5]
    \definition[本,部]{s.}{dicionário; coleção de termos especializados ou enciclopédicos, organizados em uma determinada ordem e explicados, para fins de referência}
    \variantof{词典}
  \end{phonetics}
\end{entry}

\begin{entry}{辞职}{13,11}{⾟、⽿}
  \begin{phonetics}{辞职}{ci2zhi2}[][HSK 5]
    \definition{v.+compl.}{renunciar; deixar o cargo; entregar a renúncia; pedir para ser dispensado de suas funções}
  \end{phonetics}
\end{entry}

\begin{entry}{遛狗}{13,8}{⾡、⽝}
  \begin{phonetics}{遛狗}{liu4gou3}
    \definition{v.+compl.}{passear com um cachorro}
  \end{phonetics}
\end{entry}

\begin{entry}{遥控}{13,11}{⾡、⼿}
  \begin{phonetics}{遥控}{yao2kong4}
    \definition{s.}{controle remoto}
    \definition{v.}{dirigir operações de um local remoto | controlar remotamente}
  \end{phonetics}
\end{entry}

\begin{entry}{酬劳}{13,7}{⾣、⼒}
  \begin{phonetics}{酬劳}{chou2lao2}
    \definition{s.}{recompensa}
  \end{phonetics}
\end{entry}

\begin{entry}{酱}{13}{⾣}
  \begin{phonetics}{酱}{jiang4}
    \definition{s.}{pasta grossa de soja fermentada | marinada em pasta de soja | pasta | geléia}
  \end{phonetics}
\end{entry}

\begin{entry}{错}{13}{⾦}
  \begin{phonetics}{错}{cuo4}[][HSK 1]
    \definition*{s.}{sobrenome Cuo}
    \definition{adj.}{errado; equivocado; errôneo | (na negativa) nada ruim; muito bom | entrelaçado e recortado; intrincado; complexo | ruim; pobre; péssimo (usado apenas em negativas)}
    \definition{s.}{falha; demérito | erro; engano | (arcaico) pedra de amolar para polir jade}
    \definition{v.}{estar entrelaçado e serrilhado; ser intrincado | moer; esfregar | abrir caminho; sair do caminho | alternar; escalonar | estar fora de alinhamento | deslocar | evitar; fazer com que não se encontre ou não entre em conflito | polir; polir pedras preciosas | (literário) incrustar ou revestir com ouro, prata, etc. | interseccionar; cruzar; entrecruzar}
  \end{phonetics}
\end{entry}

\begin{entry}{错误}{13,9}{⾦、⾔}
  \begin{phonetics}{错误}{cuo4wu4}[][HSK 3]
    \definition{adj.}{equivocado; errado; errôneo}
    \definition[个,次]{s.}{engano; erro; erro grosseiro; falha}
  \end{phonetics}
\end{entry}

\begin{entry}{锤}{13}{⾦}
  \begin{phonetics}{锤}{chui2}
    \definition{s.}{martelo | marreta}
    \definition{s.}{pesos (por exemplo, de uma balança)}
    \definition{v.}{marterlar para dar forma | atacar com um martelo}
  \end{phonetics}
\end{entry}

\begin{entry}{锦上添花}{13,3,11,7}{⾦、⼀、⽔、⾋}
  \begin{phonetics}{锦上添花}{jin3shang4tian1hua1}
    \definition{expr.}{A cereja do bolo | (literalmente) adicione flores ao brocato}
    \definition{v.}{dar a alguém esplendor adicional | fornecer o toque final}
  \end{phonetics}
\end{entry}

\begin{entry}{键}{13}{⾦}
  \begin{phonetics}{键}{jian4}[][HSK 5]
    \definition[个]{s.}{pino (para máquinas) | tecla (de uma máquina de escrever, piano, etc.) | chave | etapa crucial}
  \end{phonetics}
\end{entry}

\begin{entry}{键盘}{13,11}{⾦、⽫}
  \begin{phonetics}{键盘}{jian4pan2}[][HSK 5]
    \definition[个]{s.}{braço; teclado; cravo; painel de teclas; porta-chaves}
  \end{phonetics}
\end{entry}

\begin{entry}{零下}{13,3}{⾬、⼀}
  \begin{phonetics}{零下}{ling2 xia4}[][HSK 2]
    \definition{s.}{abaixo de zero; negativo}
  \end{phonetics}
\end{entry}

\begin{entry}{零食}{13,9}{⾬、⾷}
  \begin{phonetics}{零食}{ling2shi2}[][HSK 4]
    \definition[包,袋,盒,箱,堆]{s.}{lanches; refrescos; petiscos entre as refeições; alimentação esporádica, além das refeições normais}
  \end{phonetics}
\end{entry}

\begin{entry}{零散}{13,12}{⾬、⽁}
  \begin{phonetics}{零散}{ling2san3}
    \definition{adj.}{espalhado; disperso}
  \end{phonetics}
\end{entry}

\begin{entry}{零/〇}{13,13}{⾬、⾬}
  \begin{phonetics}{零/〇}{ling2 ling2}[][HSK 1]
    \definition*{s.}{sobrenome Ling}
    \definition{adj.}{ímpar; dispersos; fragmentados (em oposição a 整)}
    \definition{num.}{zero; 0; representa um número menor que qualquer número positivo e maior que qualquer número negativo; representa a ausência de quantidade | zero grau no termômetro | usado para indicar qualidade, comprimento, tempo, idade, etc. Entre dois dígitos, indica que a quantidade da unidade mais alta é acompanhada pela quantidade da unidade mais baixa | sinal de zero (0); nulo; espaço em branco para indicar números em caracteres chineses maiúsculos}
    \definition{s.}{fragmento; fração; lote ímpar; um número fracionário que não é suficiente para uma determinada unidade; um ponto decimal diferente de um inteiro}
    \definition{v.}{(de chuva, lágrimas, etc.) cair | murchar e cair}
  \seealsoref{整}{zheng3}
  \end{phonetics}
\end{entry}

\begin{entry}{雷电}{13,5}{⾬、⽥}
  \begin{phonetics}{雷电}{lei2dian4}
    \definition{s.}{trovão e relâmpago; raio}
  \end{phonetics}
\end{entry}

\begin{entry}{雷亚尔}{13,6,5}{⾬、⼆、⼩}
  \begin{phonetics}{雷亚尔}{lei2ya4'er3}
    \definition*{s.}{Real Brasileiro}
  \end{phonetics}
\end{entry}

\begin{entry}{雾气}{13,4}{⾬、⽓}
  \begin{phonetics}{雾气}{wu4qi4}
    \definition{s.}{nevoeiro | névoa | vapor}
  \end{phonetics}
\end{entry}

\begin{entry}{颐和园}{13,8,7}{⾴、⼝、⼞}
  \begin{phonetics}{颐和园}{yi2he2yuan2}
    \definition*{s.}{Palácio de Verão}
  \end{phonetics}
\end{entry}

\begin{entry}{频道}{13,12}{⾴、⾡}
  \begin{phonetics}{频道}{pin2dao4}[][HSK 5]
    \definition[个]{s.}{canal; canal de frequência; televisão e rádio, os sinais de som e imagem ocupam um determinado canal de frequência}
  \end{phonetics}
\end{entry}

\begin{entry}{频繁}{13,17}{⾴、⽷}
  \begin{phonetics}{频繁}{pin2fan2}[][HSK 5]
    \definition{adj.}{frequentemente}
    \definition{adj.}{frequente}
  \end{phonetics}
\end{entry}

\begin{entry}{魂}{13}{⿁}
  \begin{phonetics}{魂}{hun2}
    \definition{s.}{alma | espírito | alma imortal (que pode ser separada do corpo)}
  \end{phonetics}
\end{entry}

\begin{entry}{鼓}{13}{⿎}
  \begin{phonetics}{鼓}{gu3}[][HSK 5]
    \definition*{s.}{sobrenome Gu}
    \definition{adj.}{abaulado; inchado; saliente; protuberante}
    \definition{clas.}{unidades antigas de cronometragem noturna; vigílias da noite}
    \definition{s.}{tambor; instrumento de percussão |
coisas semelhantes a tambores; formato, som e função semelhantes aos de um tambor |}
    \definition{v.}{soar; bater; golpear; fazer um objeto soar | ventilar; soprar com fole | agitar; despertar; ativar; incitar; revigorar | bater asas | aumentar; fazer beicinho}
  \end{phonetics}
\end{entry}

\begin{entry}{鼓励}{13,7}{⿎、⼒}
  \begin{phonetics}{鼓励}{gu3li4}[][HSK 5]
    \definition{v.}{incitar; encorajar; provocar e incentivar}
  \end{phonetics}
\end{entry}

\begin{entry}{鼓掌}{13,12}{⿎、⼿}
  \begin{phonetics}{鼓掌}{gu3zhang3}[][HSK 5]
    \definition{v.+compl.}{aplaudir; bater palmas, principalmente para expressar felicidade, aprovação ou boas-vindas}
  \end{phonetics}
\end{entry}

\begin{entry}{鼠}{13}{⿏}[Kangxi 208]
  \begin{phonetics}{鼠}{shu3}[][HSK 5]
    \definition[只]{s.}{rato; camundongo}
  \end{phonetics}
\end{entry}

\begin{entry}{鼠标}{13,9}{⿏、⽊}
  \begin{phonetics}{鼠标}{shu3biao1}[][HSK 5]
    \definition[个]{s.}{\emph{mouse} (de computador); dispositivo de entrada externo para computadores, usado para controlar o movimento do cursor na tela do computador, selecionar objetos de operação, executar vários comandos, etc.}
  \end{phonetics}
\end{entry}

%%%%% EOF %%%%%

