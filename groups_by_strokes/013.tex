%%%
%%% 13画
%%%

\section*{13画}\addcontentsline{toc}{section}{13画}

\begin{entry}{傻瓜}{13,5}
  \begin{phonetics}{傻瓜}{sha3gua1}
    \definition{adj.}{tolo | burro | simplório | idiota}
    \definition{v.}{enganar | iludir | lograr}
  \end{phonetics}
\end{entry}

\begin{entry}{傻眼}{13,11}
  \begin{phonetics}{傻眼}{sha3yan3}
    \definition{adj.}{estupefato | atordoado}
  \end{phonetics}
\end{entry}

\begin{entry}{像}{13}[Radical 人]
  \begin{phonetics}{像}{xiang4}[][HSK 2]
    \definition{s.}{imagem | retrato | aparência}
    \definition{v.}{assemelhar-se | ser como}
  \end{phonetics}
\end{entry}

\begin{entry}{嗄}{13}[Radical 口]
  \begin{phonetics}{嗄}{a2}
    \definition{adj.}{rouco}
    \variantof{啊}
  \end{phonetics}
\end{entry}

\begin{entry}{嗡嗡}{13,13}
  \begin{phonetics}{嗡嗡}{weng1weng1}
    \definition{s.}{zumbido}
    \definition{v.}{zumbir}
  \end{phonetics}
\end{entry}

\begin{entry}{嘟}{13}[Radical ⼝]
  \begin{phonetics}{嘟}{du1}
    \definition{s.}{buzina | bip}
    \definition{v.}{fazer beicinho}
  \end{phonetics}
\end{entry}

\begin{entry}{嫉妒}{13,7}
  \begin{phonetics}{嫉妒}{ji2du4}
    \definition{v.}{estar com ciúmes de | invejar}
  \end{phonetics}
\end{entry}

\begin{entry}{幕}{13}[Radical 巾]
  \begin{phonetics}{幕}{mu4}
    \definition{s.}{cortina ou tela | dossel ou tenda | quartel de um general | ato (de uma peça)}
  \end{phonetics}
\end{entry}

\begin{entry}{彀}{13}[Radical ⼸]
  \begin{phonetics}{彀}{gou4}
    \definition{s.}{calcance de um arco e flecha}
    \definition{v.}{puxar um arco ao máximo}
  \end{phonetics}
\end{entry}

\begin{entry}{微风}{13,4}
  \begin{phonetics}{微风}{wei1feng1}
    \definition{s.}{brisa | vento leve}
  \end{phonetics}
\end{entry}

\begin{entry}{微软}{13,8}
  \begin{phonetics}{微软}{wei1ruan3}
    \definition*{s.}{\emph{Microsoft Corporation}}
  \end{phonetics}
\end{entry}

\begin{entry}{微型}{13,9}
  \begin{phonetics}{微型}{wei1xing2}
    \definition{pref.}{``micro''}
    \definition{s.}{miniatura}
  \end{phonetics}
\end{entry}

\begin{entry}{想}{13}[Radical 心]
  \begin{phonetics}{想}{xiang3}[][HSK 1]
    \definition{v.}{acreditar | sentir falta (sentir-se melancólico com a ausência de alguém ou algo) | supor | pensar | querer | desejar}
  \end{phonetics}
\end{entry}

\begin{entry}{想到}{13,8}
  \begin{phonetics}{想到}{xiang3 dao4}[][HSK 2]
    \definition{v.}{pensar em | trazer à mente | ter no coração}
  \end{phonetics}
\end{entry}

\begin{entry}{想念}{13,8}
  \begin{phonetics}{想念}{xiang3nian4}
    \definition{v.}{perder | sentir falta | lembrar com saudade}
  \end{phonetics}
\end{entry}

\begin{entry}{想法}{13,8}
  \begin{phonetics}{想法}{xiang3 fa3}[][HSK 2]
    \definition[个]{s.}{noção | opinião | jeito de pensar}
    \definition{s.}{maneira de pensar | opinião | noção}
    \definition{v.}{pensar em uma maneira (de fazer algo)}
  \end{phonetics}
\end{entry}

\begin{entry}{想起}{13,10}
  \begin{phonetics}{想起}{xiang3 qi3}[][HSK 2]
    \definition{v.}{recordar | lembrar | pensar em | trazer à mente | cruzar pelos pensamentos de alguém | passar pelo pensamento de alguém}
  \end{phonetics}
\end{entry}

\begin{entry}{想象}{13,11}
  \begin{phonetics}{想象}{xiang3xiang4}
    \definition{v.}{imaginar}
  \end{phonetics}
\end{entry}

\begin{entry}{想想看}{13,13,9}
  \begin{phonetics}{想想看}{xiang3xiang3kan4}
    \definition{v.}{pensar sobre isso}
  \end{phonetics}
\end{entry}

\begin{entry}{愈}{13}[Radical 心]
  \begin{phonetics}{愈}{yu4}
    \definition{adv.}{mais e mais | ainda mais}
    \definition{v.}{recuperar | curar}
  \end{phonetics}
\end{entry}

\begin{entry}{意义}{13,3}
  \begin{phonetics}{意义}{yi4yi4}
    \definition[个]{s.}{importância | significado | senso | desejo | força de vontade}
  \end{phonetics}
\end{entry}

\begin{entry}{意见}{13,4}
  \begin{phonetics}{意见}{yi4jian4}[][HSK 2]
    \definition[点,条]{s.}{reclamação | ideia | objeção | opinião | sugestão}
  \end{phonetics}
\end{entry}

\begin{entry}{意外}{13,5}
  \begin{phonetics}{意外}{yi4wai4}
    \definition{adj.}{inesperado}
    \definition{adv.}{acidentalmente}
    \definition[个]{s.}{acidente}
  \end{phonetics}
\end{entry}

\begin{entry}{意志}{13,7}
  \begin{phonetics}{意志}{yi4zhi4}
    \definition[个]{s.}{determinação | desejo | força de vontade}
  \end{phonetics}
\end{entry}

\begin{entry}{意识}{13,7}
  \begin{phonetics}{意识}{yi4shi2}
    \definition{s.}{consciência}
    \definition{v.}{(usualmente seguido de 到) estar ciente, constatar}
  \end{phonetics}
\end{entry}

\begin{entry}{意译}{13,7}
  \begin{phonetics}{意译}{yi4yi4}
    \definition{s.}{tradução livre | significado (de expressão estrangeira) | paráfrase | tradução do significado (em oposição à tradução literal)}
  \seealsoref{直译}{zhi2yi4}
  \end{phonetics}
\end{entry}

\begin{entry}{意思}{13,9}
  \begin{phonetics}{意思}{yi4si5}[][HSK 2]
    \definition[个]{s.}{interesse}
  \end{phonetics}
\end{entry}

\begin{entry}{意指}{13,9}
  \begin{phonetics}{意指}{yi4zhi3}
    \definition{v.}{implicar | significar}
  \end{phonetics}
\end{entry}

\begin{entry}{感动}{13,6}
  \begin{phonetics}{感动}{gan3dong4}[][HSK 2]
    \definition{v.}{mover (alguém) | tocar (alguém emocionalmente)}
  \end{phonetics}
\end{entry}

\begin{entry}{感到}{13,8}
  \begin{phonetics}{感到}{gan3 dao4}[][HSK 2]
    \definition{v.}{sentir | perceber}
  \end{phonetics}
\end{entry}

\begin{entry}{感受}{13,8}
  \begin{phonetics}{感受}{gan3shou4}
    \definition{s.}{percepção | um sentimento | uma percepção | uma experiência}
    \definition{v.}{sentir | sentir (através dos sentidos) | experimentar}
  \end{phonetics}
\end{entry}

\begin{entry}{感冒}{13,9}
  \begin{phonetics}{感冒}{gan3mao4}
    \definition{v.}{ficar resfriado | estar com resfriado}
  \end{phonetics}
\end{entry}

\begin{entry}{感染}{13,9}
  \begin{phonetics}{感染}{gan3ran3}
    \definition{s.}{infecção}
    \definition{v.}{infectar | (figurativo) influenciar}
  \end{phonetics}
\end{entry}

\begin{entry}{感觉}{13,9}
  \begin{phonetics}{感觉}{gan3jue2}[][HSK 2]
    \definition{s.}{sentimento | impressão | sensação}
    \definition{v.}{sentir | perceber}
  \end{phonetics}
\end{entry}

\begin{entry}{感情}{13,11}
  \begin{phonetics}{感情}{gan3qing2}
    \definition{s.}{afeição | emoção | sentimento | sentimento amoroso}
  \end{phonetics}
\end{entry}

\begin{entry}{感谢}{13,12}
  \begin{phonetics}{感谢}{gan3xie4}[][HSK 2]
    \definition{s.}{gratidão | agradecimento}
  \end{phonetics}
\end{entry}

\begin{entry}{搞}{13}[Radical 手]
  \begin{phonetics}{搞}{gao3}
    \definition{v.}{fazer}
  \end{phonetics}
\end{entry}

\begin{entry}{搞好}{13,6}
  \begin{phonetics}{搞好}{gao3hao3}
    \definition{v.}{fazer um ótimo trabalho}
  \end{phonetics}
\end{entry}

\begin{entry}{搞乱}{13,7}
  \begin{phonetics}{搞乱}{gao3luan4}
    \definition{v.}{estragar | confundir | bagunçar}
  \end{phonetics}
\end{entry}

\begin{entry}{搞定}{13,8}
  \begin{phonetics}{搞定}{gao3ding4}
    \definition{v.}{consertar | resolver}
  \end{phonetics}
\end{entry}

\begin{entry}{搞鬼}{13,9}
  \begin{phonetics}{搞鬼}{gao3gui3}
    \definition{v.}{fazer travessuras | fazer truques}
  \end{phonetics}
\end{entry}

\begin{entry}{搞笑}{13,10}
  \begin{phonetics}{搞笑}{gao3xiao4}
    \definition{adj.}{engraçado | hilário}
    \definition{v.}{fazer as pessoas rirem}
  \end{phonetics}
\end{entry}

\begin{entry}{搞通}{13,10}
  \begin{phonetics}{搞通}{gao3tong1}
    \definition{v.}{entender algo}
  \end{phonetics}
\end{entry}

\begin{entry}{搞钱}{13,10}
  \begin{phonetics}{搞钱}{gao3qian2}
    \definition{v.}{fazer dinheiro | acumular dinheiro}
  \end{phonetics}
\end{entry}

\begin{entry}{搞混}{13,11}
  \begin{phonetics}{搞混}{gao3hun4}
    \definition{v.}{confundir}
  \end{phonetics}
\end{entry}

\begin{entry}{搞错}{13,13}
  \begin{phonetics}{搞错}{gao3cuo4}
    \definition{v.}{cometer um erro}
  \end{phonetics}
\end{entry}

\begin{entry}{搬}{13}[Radical 手]
  \begin{phonetics}{搬}{ban1}
    \definition{v.}{copiar indiscriminadamente | mover-se (ou seja, mudar-se) | mover-se (algo relativamente pesado ou volumoso) | mudar | mudar-se}
  \end{phonetics}
\end{entry}

\begin{entry}{搬口}{13,3}
  \begin{phonetics}{搬口}{ban1kou3}
    \definition{v.}{tagarelar | (idioma) transmitir histórias | semear dissensão | contar histórias}
  \end{phonetics}
\end{entry}

\begin{entry}{搬动}{13,6}
  \begin{phonetics}{搬动}{ban1dong4}
    \definition{v.}{mover-se (alguma coisa) | se mudar}
  \end{phonetics}
\end{entry}

\begin{entry}{搬弄}{13,7}
  \begin{phonetics}{搬弄}{ban1nong4}
    \definition{v.}{causar problemas | mexer com alguém | mostrar (o que se pode fazer)}
  \end{phonetics}
\end{entry}

\begin{entry}{搬走}{13,7}
  \begin{phonetics}{搬走}{ban1zou3}
    \definition{v.}{carregar}
  \end{phonetics}
\end{entry}

\begin{entry}{搬运}{13,7}
  \begin{phonetics}{搬运}{ban1yun4}
    \definition{s.}{frete | transporte}
    \definition{v.}{carregar | transportar}
  \end{phonetics}
\end{entry}

\begin{entry}{搬家}{13,10}
  \begin{phonetics}{搬家}{ban1jia1}
    \definition{s.}{mudança}
    \definition{v.+compl.}{mudar-se de casa}
  \end{phonetics}
\end{entry}

\begin{entry}{摄氏}{13,4}
  \begin{phonetics}{摄氏}{she4shi4}
    \definition{s.}{graus Celsius (°C), centígrado}
  \end{phonetics}
\end{entry}

\begin{entry}{摆手}{13,4}
  \begin{phonetics}{摆手}{bai3shou3}
    \definition{v.+compl.}{gesticular com a mão (acenando, acenando adeus, etc.) | balançar os braços | acenar com as mãos}
  \end{phonetics}
\end{entry}

\begin{entry}{摆烂}{13,9}
  \begin{phonetics}{摆烂}{bai3lan4}
    \definition{v.}{(neologismo, gíria) parar de lutar (especialmente quando se sabe que não pode ter sucesso) | deixar tudo ir para o inferno}
  \end{phonetics}
\end{entry}

\begin{entry}{摇头}{13,5}
  \begin{phonetics}{摇头}{yao2tou2}
    \definition{v.+compl.}{balançar a cabeça de alguém}
  \end{phonetics}
\end{entry}

\begin{entry}{摇晃}{13,10}
  \begin{phonetics}{摇晃}{yao2huang4}
    \definition{v.}{sacudir | agitar | balançar | chacoalhar}
  \end{phonetics}
\end{entry}

\begin{entry}{数}{13}[Radical 攴]
  \begin{phonetics}{数}{shu3}
    \definition{v.}{contar
ser considerado excepcionalmente (bom, ruim, etc.)
enumerar; listar}
  \end{phonetics}
  \begin{phonetics}{数}{shu4}
    \definition{num.}{vários | alguns}
    \definition{s.}{número | figura | destino}
  \end{phonetics}
  \begin{phonetics}{数}{shuo4}
    \definition{adv.}{frequentemente | repetidamente}
  \end{phonetics}
\end{entry}

\begin{entry}{数字}{13,6}
  \begin{phonetics}{数字}{shu4zi4}[][HSK 2]
    \definition{adj.}{digital}
    \definition[个]{s.}{dígito | figura | número | numeral | quantidade | montante}
  \end{phonetics}
\end{entry}

\begin{entry}{数学}{13,8}
  \begin{phonetics}{数学}{shu4xue2}
    \definition{s.}{matemática (disciplina)}
  \end{phonetics}
\end{entry}

\begin{entry}{新}{13}[Radical 斤]
  \begin{phonetics}{新}{xin1}[][HSK 1]
    \definition*{s.}{sobrenome Xin | abreviação de Xinjiang (新疆) | abreviação de Singapura (新加坡)}
    \definition{adj.}{novo}
    \definition{adv.}{recentemente}
    \definition{pref.}{(química) ``meso''}
  \seealsoref{新加坡}{xin1jia1po1}
  \seealsoref{新疆}{xin1jiang1}
  \end{phonetics}
\end{entry}

\begin{entry}{新加坡}{13,5,8}
  \begin{phonetics}{新加坡}{xin1jia1po1}
    \definition*{s.}{Singapura}
  \end{phonetics}
\end{entry}

\begin{entry}{新年}{13,6}
  \begin{phonetics}{新年}{xin1nian2}[][HSK 1]
    \definition*[个]{s.}{Ano Novo}
  \end{phonetics}
\end{entry}

\begin{entry}{新闻}{13,9}
  \begin{phonetics}{新闻}{xin1wen2}[][HSK 2]
    \definition[条,个]{s.}{notícia}
  \end{phonetics}
\end{entry}

\begin{entry}{新娘}{13,10}
  \begin{phonetics}{新娘}{xin1niang2}
    \definition{s.}{noiva}
  \seealsoref{新娘子}{xin1niang2zi5}
  \end{phonetics}
\end{entry}

\begin{entry}{新娘子}{13,10,3}
  \begin{phonetics}{新娘子}{xin1niang2zi5}
    \definition{s.}{noiva}
  \seealsoref{新娘}{xin1niang2}
  \end{phonetics}
\end{entry}

\begin{entry}{新娘服装}{13,10,8,12}
  \begin{phonetics}{新娘服装}{xin1niang2 fu2zhuang1}
    \definition{s.}{roupas de noiva}
  \end{phonetics}
\end{entry}

\begin{entry}{新鲜}{13,14}
  \begin{phonetics}{新鲜}{xin1xian1}
    \definition{adj.}{fresco (experiência, alimento, etc.)}
    \definition{s.}{frescor}
  \end{phonetics}
\end{entry}

\begin{entry}{新疆}{13,19}
  \begin{phonetics}{新疆}{xin1jiang1}
    \definition*{s.}{Xinjiang}
  \end{phonetics}
\end{entry}

\begin{entry}{新疆维吾尔自治区}{13,19,11,7,5,6,8,4}
  \begin{phonetics}{新疆维吾尔自治区}{xin1jiang1 wei2wu2'er3 zi4zhi4qu1}
    \definition*{s.}{Região Autônoma Uigur de Xinjiang}
  \end{phonetics}
\end{entry}

\begin{entry}{暖}{13}[Radical 日]
  \begin{phonetics}{暖}{nuan3}
    \definition{adj.}{quente}
    \definition{v.}{esquentar}
  \end{phonetics}
\end{entry}

\begin{entry}{暖气}{13,4}
  \begin{phonetics}{暖气}{nuan3qi4}
    \definition{s.}{aquecimento central | aquecedor | ar quente}
  \end{phonetics}
\end{entry}

\begin{entry}{暖和}{13,8}
  \begin{phonetics}{暖和}{nuan3huo5}
    \definition{adj.}{morno; agradável e quente}
  \end{phonetics}
\end{entry}

\begin{entry}{暗香}{13,9}
  \begin{phonetics}{暗香}{an4xiang1}
    \definition{s.}{fragrância sutil}
  \end{phonetics}
\end{entry}

\begin{entry}{暗恋}{13,10}
  \begin{phonetics}{暗恋}{an4lian4}
    \definition{s.}{amor secreto}
    \definition{v.}{estar secretamente apaixonado por}
  \end{phonetics}
\end{entry}

\begin{entry}{楼}{13}[Radical 木]
  \begin{phonetics}{楼}{lou2}[][HSK 1]
    \definition*{s.}{sobrenome Lou}
    \definition{clas.}{andar, piso}
    \definition[层,座,栋]{s.}{edifício | prédio | sobrado | casa com 2 ou mais andares}
  \end{phonetics}
\end{entry}

\begin{entry}{楼上}{13,3}
  \begin{phonetics}{楼上}{lou2 shang4}[][HSK 1]
    \definition{adv.}{no andar de cima | (gíria da Internet) post anterior em um fio de um fórum}
  \end{phonetics}
\end{entry}

\begin{entry}{楼下}{13,3}
  \begin{phonetics}{楼下}{lou2 xia4}[][HSK 1]
    \definition{adv.}{no andar de baixo}
  \end{phonetics}
\end{entry}

\begin{entry}{楼梯}{13,11}
  \begin{phonetics}{楼梯}{lou2ti1}
    \definition[个]{s.}{escada | escadaria}
  \end{phonetics}
\end{entry}

\begin{entry}{概念}{13,8}
  \begin{phonetics}{概念}{gai4nian4}
    \definition[个]{s.}{conceito | ideia}
  \end{phonetics}
\end{entry}

\begin{entry}{滔天}{13,4}
  \begin{phonetics}{滔天}{tao1tian1}
    \definition{adj.}{(ondas, raiva, desastres, crimes, etc.) imponente, avassalador, imenso}
  \end{phonetics}
\end{entry}

\begin{entry}{滚轮}{13,8}
  \begin{phonetics}{滚轮}{gun3lun2}
    \definition{s.}{pneu | dial rotativo | roda de rolagem (\emph{scroll})  (mouse de computador)}
  \end{phonetics}
\end{entry}

\begin{entry}{滚滚}{13,13}
  \begin{phonetics}{滚滚}{gun3gun3}
    \definition*{s.}{Apelido para um panda}
    \definition{v.}{mover-se | rolar | fluir continuamente}
  \end{phonetics}
\end{entry}

\begin{entry}{满}{13}[Radical 水]
  \begin{phonetics}{满}{man3}[][HSK 2]
    \definition{adj.}{completo | preenchido | embalado | satisfeito | contente}
    \definition{adv.}{completamente | bastante}
    \definition{v.}{preencher | atingir o limite | satisfazer}
  \end{phonetics}
\end{entry}

\begin{entry}{满分}{13,4}
  \begin{phonetics}{满分}{man3fen1}
    \definition{s.}{pontuação completa}
  \end{phonetics}
\end{entry}

\begin{entry}{满意}{13,13}
  \begin{phonetics}{满意}{man3yi4}[][HSK 2]
    \definition{adj.}{satisfatório}
  \end{phonetics}
\end{entry}

\begin{entry}{满满}{13,13}
  \begin{phonetics}{满满}{man3man3}
    \definition{adj.}{completo | densamente empacotado}
  \end{phonetics}
\end{entry}

\begin{entry}{煎}{13}[Radical 火]
  \begin{phonetics}{煎}{jian1}
    \definition{v.}{fritar | refogar}
  \end{phonetics}
\end{entry}

\begin{entry}{煎饼}{13,9}
  \begin{phonetics}{煎饼}{jian1bing3}
    \definition[张]{s.}{jianbing, crepe chinês | panqueca}
  \end{phonetics}
\end{entry}

\begin{entry}{煎蛋}{13,11}
  \begin{phonetics}{煎蛋}{jian1dan4}
    \definition{s.}{ovos fritos}
  \end{phonetics}
\end{entry}

\begin{entry}{照}{13}[Radical 火]
  \begin{phonetics}{照}{zhao4}
    \definition{adv.}{de acordo com | como antes | como pedido | conforme}
    \definition{s.}{foto}
    \definition{v.}{iluminar | olhar (o reflexo de alguém) | refletir | brilhar | tirar uma foto}
  \end{phonetics}
\end{entry}

\begin{entry}{照片}{13,4}
  \begin{phonetics}{照片}{zhao4pian4}[][HSK 2]
    \definition[张,套,幅]{s.}{fotografia | foto}
  \end{phonetics}
\end{entry}

\begin{entry}{照片子}{13,4,3}
  \begin{phonetics}{照片子}{zhao4pian4zi5}
    \definition{v.}{tirar um raio X}
  \end{phonetics}
\end{entry}

\begin{entry}{照片底版}{13,4,8,8}
  \begin{phonetics}{照片底版}{zhao4pian4di3ban3}
    \definition{s.}{placa fotográfica}
  \end{phonetics}
\end{entry}

\begin{entry}{照亮}{13,9}
  \begin{phonetics}{照亮}{zhao4liang4}
    \definition{s.}{iluminação}
    \definition{v.}{iluminar}
  \end{phonetics}
\end{entry}

\begin{entry}{照相}{13,9}
  \begin{phonetics}{照相}{zhao4 xiang4}[][HSK 2]
    \definition{v.+compl.}{tirar fotografia}
  \end{phonetics}
\end{entry}

\begin{entry}{照相机}{13,9,6}
  \begin{phonetics}{照相机}{zhao4xiang4ji1}
    \definition[个,架,部,台,只]{s.}{câmera/máquina fotográfica}
  \end{phonetics}
\end{entry}

\begin{entry}{照准}{13,10}
  \begin{phonetics}{照准}{zhao4zhun3}
    \definition{s.}{solicitação concedida (uso formal em documento antigo)}
    \definition{v.}{mirar (arma)}
  \end{phonetics}
\end{entry}

\begin{entry}{照顾}{13,10}
  \begin{phonetics}{照顾}{zhao4gu4}[][HSK 2]
    \definition{v.}{cuidar de | atender a | oferecer tratamento preferencial | (de um cliente) patrocinar | fazer compras em | dar consideração a | mostrar consideração por | levar em conta | fazer concessões para}
  \end{phonetics}
\end{entry}

\begin{entry}{照骗}{13,12}
  \begin{phonetics}{照骗}{zhao4pian4}
    \definition{s.}{imagem ``photoshopada''}
  \end{phonetics}
\end{entry}

\begin{entry}{照像}{13,13}
  \begin{phonetics}{照像}{zhao4xiang4}
    \variantof{照相}
  \end{phonetics}
\end{entry}

\begin{entry}{照像机}{13,13,6}
  \begin{phonetics}{照像机}{zhao4xiang4ji1}
    \variantof{照相机}
  \end{phonetics}
\end{entry}

\begin{entry}{瑜伽}{13,7}
  \begin{phonetics}{瑜伽}{yu2jia1}
    \definition*{s.}{Ioga}
  \end{phonetics}
\end{entry}

\begin{entry}{瑜珈}{13,9}
  \begin{phonetics}{瑜珈}{yu2jia1}
    \variantof{瑜伽}
  \end{phonetics}
\end{entry}

\begin{entry}{睡}{13}[Radical 目]
  \begin{phonetics}{睡}{shui4}[][HSK 1]
    \definition{v.}{dormir}
  \end{phonetics}
\end{entry}

\begin{entry}{睡衣}{13,6}
  \begin{phonetics}{睡衣}{shui4yi1}
    \definition{s.}{pijamas | roupas de dormir}
  \end{phonetics}
\end{entry}

\begin{entry}{睡觉}{13,9}
  \begin{phonetics}{睡觉}{shui4jiao4}[][HSK 1]
    \definition{v.+compl.}{ir para a cama | dormir | deitar-se}
  \end{phonetics}
\end{entry}

\begin{entry}{睡懒觉}{13,16,9}
  \begin{phonetics}{睡懒觉}{shui4lan3jiao4}
    \definition{v.}{levantar-se tarde | passar o tempo a dormir}
  \end{phonetics}
\end{entry}

\begin{entry}{矮}{13}[Radical 矢]
  \begin{phonetics}{矮}{ai3}
    \definition{adj.}{baixo em estatura, dimensão, grau ou ranque | curto (em comprimento)}
  \end{phonetics}
\end{entry}

\begin{entry}{矮人}{13,2}
  \begin{phonetics}{矮人}{ai3ren2}
    \definition{s.}{anão | homúnculo | nanismo}
  \end{phonetics}
\end{entry}

\begin{entry}{矮子}{13,3}
  \begin{phonetics}{矮子}{ai3zi5}
    \definition{s.}{pessoa baixa | anão}
  \end{phonetics}
\end{entry}

\begin{entry}{矮小}{13,3}
  \begin{phonetics}{矮小}{ai3xiao3}
    \definition{adj.}{baixo e pequeno | curto e pequeno | subdimensionado}
  \end{phonetics}
\end{entry}

\begin{entry}{矮林}{13,8}
  \begin{phonetics}{矮林}{ai3lin2}
    \definition{s.}{mato | mata}
  \end{phonetics}
\end{entry}

\begin{entry}{矮星}{13,9}
  \begin{phonetics}{矮星}{ai3xing1}
    \definition{s.}{estrela anã}
  \end{phonetics}
\end{entry}

\begin{entry}{矮树}{13,9}
  \begin{phonetics}{矮树}{ai3shu4}
    \definition{s.}{arbusto | árvore pequena}
  \end{phonetics}
\end{entry}

\begin{entry}{矮胖}{13,9}
  \begin{phonetics}{矮胖}{ai3pang4}
    \definition{adj.}{atarracado |  gorducho | rechonchudo | roliço | curto e robusto}
  \end{phonetics}
\end{entry}

\begin{entry}{矮凳}{13,14}
  \begin{phonetics}{矮凳}{ai3deng4}
    \definition{s.}{banquinho baixo | banqueta}
  \end{phonetics}
\end{entry}

\begin{entry}{碍事}{13,8}
  \begin{phonetics}{碍事}{ai4shi4}
    \definition{s.}{(usualmente em frases negativas) sem consequência, não importa}
    \definition{v.+compl.}{estar no caminho | ser um obstáculo}
  \end{phonetics}
\end{entry}

\begin{entry}{碎}{13}[Radical 石]
  \begin{phonetics}{碎}{sui4}
    \definition{adj.}{quebrado | fragmentado | espalhado | tagarela}
    \definition{v.}{(transitivo ou intransitivo) quebrar em pedaços, quebrar, desmoronar}
  \end{phonetics}
\end{entry}

\begin{entry}{碗}{13}[Radical 石]
  \begin{phonetics}{碗}{wan3}[][HSK 2]
    \definition{clas.}{tigelas}
    \definition[只,个]{s.}{tigela}
  \end{phonetics}
\end{entry}

\begin{entry}{碗子}{13,3}
  \begin{phonetics}{碗子}{wan3zi5}
    \definition{s.}{tigela}
  \end{phonetics}
\end{entry}

\begin{entry}{碗柜}{13,8}
  \begin{phonetics}{碗柜}{wan3gui4}
    \definition{s.}{armário}
  \end{phonetics}
\end{entry}

\begin{entry}{碰}{13}[Radical 石]
  \begin{phonetics}{碰}{peng4}[][HSK 2]
    \definition{v.}{tocar | bater | encontrar | correr para | tentar a sorte | arriscar | encontrar para discutir}
  \end{phonetics}
\end{entry}

\begin{entry}{碰见}{13,4}
  \begin{phonetics}{碰见}{peng4 jian4}[][HSK 2]
    \definition{v.}{reunir-se | encontrar}
  \end{phonetics}
\end{entry}

\begin{entry}{碰头}{13,5}
  \begin{phonetics}{碰头}{peng4tou2}
    \definition{s.}{colisão | conflito}
    \definition{v.}{colidir}
    \definition{v.+compl.}{conhecer e discutir | juntar ideias | ver-se}
  \end{phonetics}
\end{entry}

\begin{entry}{碰运气}{13,7,4}
  \begin{phonetics}{碰运气}{peng4yun4qi5}
    \definition{v.}{deixar algo ao acaso | tentar a sorte}
  \end{phonetics}
\end{entry}

\begin{entry}{碰到}{13,8}
  \begin{phonetics}{碰到}{peng4 dao4}[][HSK 2]
    \definition{v.}{encontrar (com) | esbarrar em | deparar-se com}
  \end{phonetics}
\end{entry}

\begin{entry}{福克斯}{13,7,12}
  \begin{phonetics}{福克斯}{fu2ke4si1}
    \definition*{s.}{Fox (empresa de mídia) | Focus (automóvel fabricado pela Ford)}
  \end{phonetics}
\end{entry}

\begin{entry}{福泽}{13,8}
  \begin{phonetics}{福泽}{fu2ze2}
    \definition{s.}{boa sorte}
  \end{phonetics}
\end{entry}

\begin{entry}{筷子}{13,3}
  \begin{phonetics}{筷子}{kuai4zi5}[][HSK 2]
    \definition[对,根,把,双]{s.}{pauzinhos | \emph{chopsticks}}
  \end{phonetics}
\end{entry}

\begin{entry}{签}{13}[Radical 竹]
  \begin{phonetics}{签}{qian1}
    \definition{s.}{vara de bambu com inscrição (usada em adivinhação, jogos de azar, sorteios, etc.) | rótulo | pequena lasca de madeira | etiqueta}
    \definition{v.}{assinar}
  \end{phonetics}
\end{entry}

\begin{entry}{签名}{13,6}
  \begin{phonetics}{签名}{qian1ming2}
    \definition{s.}{assinatura}
    \definition{v.+compl.}{autografar | assinar}
  \end{phonetics}
\end{entry}

\begin{entry}{简单}{13,8}
  \begin{phonetics}{简单}{jian3dan1}
    \definition{adj.}{simples | sem complicações}
  \end{phonetics}
\end{entry}

\begin{entry}{简直}{13,8}
  \begin{phonetics}{简直}{jian3zhi2}
    \definition{adv.}{simplesmente | realmente | absolutamente | em tudo}
  \end{phonetics}
\end{entry}

\begin{entry}{缝纫}{13,6}
  \begin{phonetics}{缝纫}{feng2ren4}
    \definition{v.}{costurar}
  \end{phonetics}
\end{entry}

\begin{entry}{缝纫机}{13,6,6}
  \begin{phonetics}{缝纫机}{feng2ren4ji1}
    \definition[架]{s.}{máquina de costura}
  \end{phonetics}
\end{entry}

\begin{entry}{罪犯}{13,5}
  \begin{phonetics}{罪犯}{zui4fan4}
    \definition{s.}{criminoso}
  \end{phonetics}
\end{entry}

\begin{entry}{罪行}{13,6}
  \begin{phonetics}{罪行}{zui4xing2}
    \definition{s.}{crime | ofensa}
  \end{phonetics}
\end{entry}

\begin{entry}{置疑}{13,14}
  \begin{phonetics}{置疑}{zhi4yi2}
    \definition{v.}{duvidar}
  \end{phonetics}
\end{entry}

\begin{entry}{群山}{13,3}
  \begin{phonetics}{群山}{qun2shan1}
    \definition{s.}{montanhas | uma cadeia de colinas}
  \end{phonetics}
\end{entry}

\begin{entry}{腰}{13}[Radical 肉]
  \begin{phonetics}{腰}{yao1}
    \definition{s.}{cintura}
  \end{phonetics}
\end{entry}

\begin{entry}{腰包}{13,5}
  \begin{phonetics}{腰包}{yao1bao1}
    \definition{s.}{pochete | bolso}
  \end{phonetics}
\end{entry}

\begin{entry}{腰椎}{13,12}
  \begin{phonetics}{腰椎}{yao1zhui1}
    \definition{s.}{vértebra lombar (espinha dorsal inferior)}
  \end{phonetics}
\end{entry}

\begin{entry}{腿}{13}[Radical 肉]
  \begin{phonetics}{腿}{tui3}[][HSK 2]
    \definition[条]{s.}{perna | osso do quadril}
  \end{phonetics}
\end{entry}

\begin{entry}{腿号}{13,5}
  \begin{phonetics}{腿号}{tui3hao4}
    \definition{s.}{anilha numerada (por exemplo, usada para identificar pássaros)}
  \seealsoref{腿号箍}{tui3hao4gu1}
  \end{phonetics}
\end{entry}

\begin{entry}{腿号箍}{13,5,14}
  \begin{phonetics}{腿号箍}{tui3hao4gu1}
    \definition{s.}{anilha numerada (por exemplo, usada para identificar pássaros)}
  \seealsoref{腿号}{tui3hao4}
  \end{phonetics}
\end{entry}

\begin{entry}{艁}{13}[Radical 舟]
  \begin{phonetics}{艁}{zao4}
    \variantof{造}
  \end{phonetics}
\end{entry}

\begin{entry}{蒙面}{13,9}
  \begin{phonetics}{蒙面}{meng2mian4}
    \definition{adj.}{descarado | desavergonhado | mascarado}
    \definition{v.}{cobrir o rosto | usar uma máscara}
  \end{phonetics}
\end{entry}

\begin{entry}{蓝}{13}[Radical 艸]
  \begin{phonetics}{蓝}{lan2}[][HSK 2]
    \definition*{s.}{sobrenome Lan}
    \definition{adj.}{azul}
  \end{phonetics}
\end{entry}

\begin{entry}{蓝色}{13,6}
  \begin{phonetics}{蓝色}{lan2 se4}[][HSK 2]
    \definition{s.}{cor azul}
  \end{phonetics}
\end{entry}

\begin{entry}{解决}{13,6}
  \begin{phonetics}{解决}{jie3jue2}
    \definition{v.}{resolver (uma disputa) | resolver | solucionar}
  \end{phonetics}
\end{entry}

\begin{entry}{解压}{13,6}
  \begin{phonetics}{解压}{jie3ya1}
    \definition{v.}{aliviar o estresse | (computação) descomprimir}
  \end{phonetics}
\end{entry}

\begin{entry}{解救}{13,11}
  \begin{phonetics}{解救}{jie3jiu4}
    \definition{v.}{resgatar | ajudar a sair de dificuldades | salvar a situação}
  \end{phonetics}
\end{entry}

\begin{entry}{解释}{13,12}
  \begin{phonetics}{解释}{jie3shi4}
    \definition[个]{s.}{explicação}
    \definition{v.}{explicar | interpretar | resolver}
  \end{phonetics}
\end{entry}

\begin{entry}{解雇}{13,12}
  \begin{phonetics}{解雇}{jie3gu4}
    \definition{v.}{demitir}
  \end{phonetics}
\end{entry}

\begin{entry}{谩骂}{13,9}
  \begin{phonetics}{谩骂}{man4ma4}
    \definition{v.}{ridicularizar | abusar}
  \end{phonetics}
\end{entry}

\begin{entry}{赖}{13}[Radical 貝]
  \begin{phonetics}{赖}{lai4}
    \definition*{s.}{sobrenome Lai}
    \definition{v.}{depender | aguentar em um lugar | renegar (promessa) | isolar-se | culpar | colocar a culpa em}
  \end{phonetics}
\end{entry}

\begin{entry}{跟}{13}[Radical ⾜]
  \begin{phonetics}{跟}{gen1}[][HSK 1]
    \definition{conj.}{e; com}
    \definition{prep.}{com}
    \definition{v.}{acompanhar junto | seguir de perto | ir com}
  \end{phonetics}
\end{entry}

\begin{entry}{跪拜}{13,9}
  \begin{phonetics}{跪拜}{gui4bai4}
    \definition{v.}{prostrar-se | ajoelhar-se e adorar}
  \end{phonetics}
\end{entry}

\begin{entry}{路}{13}[Radical 足]
  \begin{phonetics}{路}{lu4}[][HSK 1]
    \definition*{s.}{sobrenome Lu}
    \definition[条]{s.}{caminho | estrada | via | jornada | linha (ônibus, etc.) | rota}
  \end{phonetics}
\end{entry}

\begin{entry}{路上}{13,3}
  \begin{phonetics}{路上}{lu4shang5}[][HSK 1]
    \definition{adv.}{na estrada | no caminho | a caminho}
  \end{phonetics}
\end{entry}

\begin{entry}{路口}{13,3}
  \begin{phonetics}{路口}{lu4kou3}[][HSK 1]
    \definition{s.}{cruzamento | interseção (de estradas)}
  \end{phonetics}
\end{entry}

\begin{entry}{路边}{13,5}
  \begin{phonetics}{路边}{lu4 bian1}[][HSK 2]
    \definition{s.}{meio-fio | acostamento}
  \end{phonetics}
\end{entry}

\begin{entry}{跳}{13}[Radical 足]
  \begin{phonetics}{跳}{tiao4}
    \definition{v.}{pular | saltar}
  \end{phonetics}
\end{entry}

\begin{entry}{跳水}{13,4}
  \begin{phonetics}{跳水}{tiao4shui3}
    \definition{s.}{mergulho esportivo}
    \definition{v.}{mergulhar (na água) | cometer suicídio pulando na água | (figurativo, preços das ações, etc.) cair dramaticamente}
  \end{phonetics}
\end{entry}

\begin{entry}{跳电}{13,5}
  \begin{phonetics}{跳电}{tiao4dian4}
    \definition{v.}{desarmar (um disjuntor ou interruptor)}
  \end{phonetics}
\end{entry}

\begin{entry}{跳伞}{13,6}
  \begin{phonetics}{跳伞}{tiao4san3}
    \definition{s.}{paraquedas}
    \definition{v.}{saltar de paraquedas}
  \end{phonetics}
\end{entry}

\begin{entry}{跳远}{13,7}
  \begin{phonetics}{跳远}{tiao4yuan3}
    \definition{v.+compl.}{salto em distância (atletismo)}
  \end{phonetics}
\end{entry}

\begin{entry}{跳挡}{13,9}
  \begin{phonetics}{跳挡}{tiao4dang3}
    \definition{v.}{pular marcha (de um carro) | perder a marcha}
  \end{phonetics}
\end{entry}

\begin{entry}{跳蚤}{13,9}
  \begin{phonetics}{跳蚤}{tiao4zao5}
    \definition{s.}{pulga}
  \end{phonetics}
\end{entry}

\begin{entry}{跳绳}{13,11}
  \begin{phonetics}{跳绳}{tiao4sheng2}
    \definition{v.}{pular corda}
  \end{phonetics}
\end{entry}

\begin{entry}{跳跳糖}{13,13,16}
  \begin{phonetics}{跳跳糖}{tiao4tiao4tang2}
    \definition{s.}{\emph{Pop Rocks}, \emph{popping candy}}
  \end{phonetics}
\end{entry}

\begin{entry}{跳频}{13,13}
  \begin{phonetics}{跳频}{tiao4pin2}
    \definition{s.}{FHSS, \emph{Frequency-Hopping Spread Spectrum}, método de transmissão de sinais de rádio}
  \end{phonetics}
\end{entry}

\begin{entry}{跳舞}{13,14}
  \begin{phonetics}{跳舞}{tiao4wu3}
    \definition{v.+compl.}{dançar}
  \end{phonetics}
\end{entry}

\begin{entry}{躲}{13}[Radical ⾝]
  \begin{phonetics}{躲}{duo3}
    \definition{v.}{esconder | esquivar | evitar}
  \end{phonetics}
\end{entry}

\begin{entry}{躲闪}{13,5}
  \begin{phonetics}{躲闪}{duo3shan3}
    \definition{v.}{desviar | evadir | esquivar (para fora do caminho)}
  \end{phonetics}
\end{entry}

\begin{entry}{辞典}{13,8}
  \begin{phonetics}{辞典}{ci2dian3}
    \variantof{词典}
  \end{phonetics}
\end{entry}

\begin{entry}{遛狗}{13,8}
  \begin{phonetics}{遛狗}{liu4gou3}
    \definition{v.+compl.}{passear com um cachorro}
  \end{phonetics}
\end{entry}

\begin{entry}{遥控}{13,11}
  \begin{phonetics}{遥控}{yao2kong4}
    \definition{s.}{controle remoto}
    \definition{v.}{dirigir operações de um local remoto | controlar remotamente}
  \end{phonetics}
\end{entry}

\begin{entry}{酬劳}{13,7}
  \begin{phonetics}{酬劳}{chou2lao2}
    \definition{s.}{recompensa}
  \end{phonetics}
\end{entry}

\begin{entry}{酱}{13}[Radical 酉]
  \begin{phonetics}{酱}{jiang4}
    \definition{s.}{pasta grossa de soja fermentada | marinada em pasta de soja | pasta | geléia}
  \end{phonetics}
\end{entry}

\begin{entry}{错}{13}[Radical 金]
  \begin{phonetics}{错}{cuo4}[][HSK 1]
    \definition*{s.}{sobrenome Cuo}
    \definition{adj.}{errado | enganado}
  \end{phonetics}
\end{entry}

\begin{entry}{锤}{13}[Radical 金]
  \begin{phonetics}{锤}{chui2}
    \definition{s.}{martelo | marreta}
    \definition{s.}{pesos (por exemplo, de uma balança)}
    \definition{v.}{marterlar para dar forma | atacar com um martelo}
  \end{phonetics}
\end{entry}

\begin{entry}{锦上添花}{13,3,11,7}
  \begin{phonetics}{锦上添花}{jin3shang4tian1hua1}
    \definition{expr.}{A cereja do bolo | (literalmente) adicione flores ao brocato}
    \definition{v.}{dar a alguém esplendor adicional | fornecer o toque final}
  \end{phonetics}
\end{entry}

\begin{entry}{键}{13}[Radical 金]
  \begin{phonetics}{键}{jian4}
    \definition{s.}{tecla (em um teclado de piano ou computador) | botão (em um mouse ou outro dispositivo) | ligação química | cavilha de roda | chaveta}
  \end{phonetics}
\end{entry}

\begin{entry}{零/〇}{13}
  \begin{phonetics}{零/〇}{ling2}[][HSK 1]
    \definition{adj.}{extra}
    \definition{num.}{zero; 0}
    \definition{s.}{(matemática) resto (após a divisão) | fração | nada}
  \end{phonetics}
\end{entry}

\begin{entry}{零下}{13,3}
  \begin{phonetics}{零下}{ling2 xia4}[][HSK 2]
    \definition{s.}{abaixo de zero}
  \end{phonetics}
\end{entry}

\begin{entry}{雷亚尔}{13,6,5}
  \begin{phonetics}{雷亚尔}{lei2ya4'er3}
    \definition*{s.}{Real Brasileiro}
  \end{phonetics}
\end{entry}

\begin{entry}{雾气}{13,4}
  \begin{phonetics}{雾气}{wu4qi4}
    \definition{s.}{nevoeiro | névoa | vapor}
  \end{phonetics}
\end{entry}

\begin{entry}{颐和园}{13,8,7}
  \begin{phonetics}{颐和园}{yi2he2yuan2}
    \definition*{s.}{Palácio de Verão}
  \end{phonetics}
\end{entry}

\begin{entry}{频道}{13,12}
  \begin{phonetics}{频道}{pin2dao4}
    \definition{s.}{frequência | (televisão) canal}
  \end{phonetics}
\end{entry}

\begin{entry}{魂}{13}[Radical 鬼]
  \begin{phonetics}{魂}{hun2}
    \definition{s.}{alma | espírito | alma imortal (que pode ser separada do corpo)}
  \end{phonetics}
\end{entry}

\begin{entry}{鼓掌}{13,12}
  \begin{phonetics}{鼓掌}{gu3zhang3}
    \definition{v.+compl.}{aplaudir | bater palmas}
  \end{phonetics}
\end{entry}

%%%%% EOF %%%%%

