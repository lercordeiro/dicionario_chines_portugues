%%%
%%% 13画
%%%

\section*{13画}\addcontentsline{toc}{section}{13画}

\begin{Entry}{催}{13}{⼈}
  \begin{Phonetics}{催}{cui1}[][HSK 7-9]
    \definition*{s.}{Sobrenome Cui}
    \definition{v.}{instar; apressar; pressionar | apressar; agilizar; acelerar}
  \end{Phonetics}
\end{Entry}

\begin{Entry}{催促}{13,9}{⼈、⼈}
  \begin{Phonetics}{催促}{cui1cu4}[][HSK 7-9]
    \definition{v.}{instar; apressar; pressionar; urgir}
  \end{Phonetics}
\end{Entry}

\begin{Entry}{催眠}{13,10}{⼈、⽬}
  \begin{Phonetics}{催眠}{cui1mian2}[][HSK 7-9]
    \definition{adj.}{hipnótico}
    \definition{v.}{hipnotizar; mesmerizar; embalar (para dormir); usar drogas ou sons, movimentos, etc. para induzir o sono}
  \end{Phonetics}
\end{Entry}

\begin{Entry}{傻}{13}{⼈}
  \begin{Phonetics}{傻}{sha3}[][HSK 5]
    \definition{adj.}{estúpido; confuso; burro; idiota; inflexível; (ação ou pensamento) mecânico}
  \end{Phonetics}
\end{Entry}

\begin{Entry}{傻瓜}{13,5}{⼈、⽠}
  \begin{Phonetics}{傻瓜}{sha3gua1}
    \definition{adj.}{tolo | burro | simplório | idiota}
    \definition{v.}{enganar | iludir | lograr}
  \end{Phonetics}
\end{Entry}

\begin{Entry}{傻眼}{13,11}{⼈、⽬}
  \begin{Phonetics}{傻眼}{sha3yan3}
    \definition{adj.}{estupefato | atordoado}
  \end{Phonetics}
\end{Entry}

\begin{Entry}{像}{13}{⼈}
  \begin{Phonetics}{像}{xiang4}[][HSK 2]
    \definition{adv.}{parecer; parecer como se}
    \definition{s.}{imagem; retrato; semelhança a alguém | imagem}
    \definition{v.}{assemelhar-se; ser como; parecer-se com | ser como; ser tal como}
  \end{Phonetics}
\end{Entry}

\begin{Entry}{勤}{13}{⼒}
  \begin{Phonetics}{勤}{qin2}
    \definition*{s.}{Sobrenome Qin}
    \definition{adj.}{diligente; industrial; trabalhador}
    \definition{adv.}{frequentemente}
    \definition{s.}{dever; serviço | presença; trabalhadores que chegam ao trabalho no horário especificado}
  \end{Phonetics}
\end{Entry}

\begin{Entry}{勤奋}{13,8}{⼒、⼤}
  \begin{Phonetics}{勤奋}{qin2fen4}[][HSK 5]
    \definition{adj.}{diligente; assíduo; trabalhador; descreve alguém que se esforça continuamente nos estudos ou no trabalho}
  \end{Phonetics}
\end{Entry}

\begin{Entry}{叠}{13}{⼜}
  \begin{Phonetics}{叠}{die2}[][HSK 7-9]
    \definition{clas.}{maço; pacote; pilha}
    \definition{v.}{acumular; empilhar | dobrar}
  \end{Phonetics}
\end{Entry}

\begin{Entry}{嗄}{13}{⼝}
  \begin{Phonetics}{嗄}{a2}
    \variantof{啊}
  \end{Phonetics}
  \begin{Phonetics}{嗄}{sha4}
    \definition{adj.}{rouco}
  \end{Phonetics}
\end{Entry}

\begin{Entry}{嗅}{13}{⼝}
  \begin{Phonetics}{嗅}{xiu4}
    \definition{v.}{cheirar; farejar; identificar odores pelo nariz}
  \end{Phonetics}
\end{Entry}

\begin{Entry}{嗡}{13}{⼝}
  \begin{Phonetics}{嗡}{weng1}
    \definition[出]{part.}{(onomatopéia) zumbido; zunido; zunzum; descreve o som do bater de asas de um inseto}
  \end{Phonetics}
\end{Entry}

\begin{Entry}{嗡嗡}{13,13}{⼝、⼝}
  \begin{Phonetics}{嗡嗡}{weng1weng1}
    \definition{s.}{zumbido}
    \definition{v.}{zumbir}
  \end{Phonetics}
\end{Entry}

\begin{Entry}{嘟}{13}{⼝}
  \begin{Phonetics}{嘟}{du1}
    \definition{part.}{(onomatopéia) buzina}
    \definition{v.}{fazer beicinho}
  \end{Phonetics}
\end{Entry}

\begin{Entry}{塑}{13}{⼟}
  \begin{Phonetics}{塑}{su4}
    \definition{s.}{plástico; material plástico}
    \definition{v.}{modelo; molde; forma}
  \end{Phonetics}
\end{Entry}

\begin{Entry}{塑料}{13,10}{⼟、⽃}
  \begin{Phonetics}{塑料}{su4 liao4}[][HSK 4]
    \definition[块,种]{s.}{plástico; compostos de polímeros feitos de resinas naturais ou sintéticas como componente principal}
  \end{Phonetics}
\end{Entry}

\begin{Entry}{塑料袋}{13,10,11}{⼟、⽃、⾐}
  \begin{Phonetics}{塑料袋}{su4liao4dai4}[][HSK 4]
    \definition[个,只]{s.}{saco plástico; sacola de plástico}
  \end{Phonetics}
\end{Entry}

\begin{Entry}{塞}{13}{⼟}
  \begin{Phonetics}{塞}{sai1}[][HSK 6]
    \definition{s.}{rolha; plugue}
    \definition{v.}{encher; conectar; preencher; espremer; bloquear | superar (para comparação)}
  \end{Phonetics}
\end{Entry}

\begin{Entry}{填}{13}{⼟}
  \begin{Phonetics}{填}{tian2}
    \definition{v.}{encher; rechear | reabastecer; suplementar; complementar | preencher; escrever dados em uma caixa (em um questionário ou formulário da \emph{Web})}
  \end{Phonetics}
\end{Entry}

\begin{Entry}{填空}{13,8}{⼟、⽳}
  \begin{Phonetics}{填空}{tian2kong4}[][HSK 4]
    \definition{v.}{preencher o espaço em branco (por exemplo, em um teste)}
  \end{Phonetics}
\end{Entry}

\begin{Entry}{墓}{13}{⼟}
  \begin{Phonetics}{墓}{mu4}
    \definition[座,个,号]{s.}{sepultura; túmulo; mausoléu}
  \end{Phonetics}
\end{Entry}

\begin{Entry}{嫁}{13}{⼥}
  \begin{Phonetics}{嫁}{jia4}[][HSK 7-9]
    \definition{v.}{(uma mulher) casar (oposto a 娶) | casar uma filha | transferir (uma culpa, perda, fardo, etc.)}
  \seealsoref{娶}{qu3}
  \end{Phonetics}
\end{Entry}

\begin{Entry}{嫁妆}{13,6}{⼥、⼥}
  \begin{Phonetics}{嫁妆}{jia4zhuang5}[][HSK 7-9]
    \definition{s.}{dote; enxoval}
  \end{Phonetics}
\end{Entry}

\begin{Entry}{嫉}{13}{⼥}
  \begin{Phonetics}{嫉}{ji2}
    \definition{v.}{invejar | odiar | ter ciúmes; ter inveja}
  \end{Phonetics}
\end{Entry}

\begin{Entry}{嫉妒}{13,7}{⼥、⼥}
  \begin{Phonetics}{嫉妒}{ji2du4}[][HSK 7-9]
    \definition{v.}{invejar; ter ciúmes de}
  \end{Phonetics}
\end{Entry}

\begin{Entry}{嫌}{13}{⼥}
  \begin{Phonetics}{嫌}{xian2}[][HSK 6]
    \definition{s.}{suspeita; suspeição; cisma | inimizade; rancor; má vontade; ressentimento}
    \definition{v.}{importar-se com; não gostar e evitar; reclamar de}
  \end{Phonetics}
\end{Entry}

\begin{Entry}{尴}{13}{⼪}
  \begin{Phonetics}{尴}{gan1}
    \definition{adj.}{envergonhado; uma situação ou assunto difícil de lidar | pouco à vontade; expressão não natural; envergonhado}
  \end{Phonetics}
\end{Entry}

\begin{Entry}{尴尬}{13,7}{⼪、⼪}
  \begin{Phonetics}{尴尬}{gan1ga4}[][HSK 7-9]
    \definition{adj.}{estranho; envergonhado; quando você se depara com algo difícil de lidar ou algo que o deixa envergonhado}
  \end{Phonetics}
\end{Entry}

\begin{Entry}{幕}{13}{⼱}
  \begin{Phonetics}{幕}{mu4}
    \definition{s.}{cortina ou tela | dossel ou tenda | quartel de um general | ato (de uma peça)}
  \end{Phonetics}
\end{Entry}

\begin{Entry}{彀}{13}{⼸}
  \begin{Phonetics}{彀}{gou4}
    \definition{adj.}{suficiente; adequado}
    \definition{v.}{puxar um arco ao máximo}
  \end{Phonetics}
\end{Entry}

\begin{Entry}{微}{13}{⼻}
  \begin{Phonetics}{微}{wei1}
    \definition{adj.}{minúsculo; leve | profundo; abstruso | humilde; tendo pouca influência; baixo \emph{status}}
    \definition{adv.}{pouco; ligeiramente; indica um grau menor, equivalente a 稍 ou 略}
    \definition{num.}{um milionésimo de uma determinada unidade de medida}
    \definition{suf.}{micro-}
  \seealsoref{略}{lve4}
  \seealsoref{稍}{shao1}
  \end{Phonetics}
\end{Entry}

\begin{Entry}{微风}{13,4}{⼻、⾵}
  \begin{Phonetics}{微风}{wei1feng1}
    \definition{s.}{brisa | vento leve}
  \end{Phonetics}
\end{Entry}

\begin{Entry}{微波炉}{13,8,8}{⼻、⽔、⽕}
  \begin{Phonetics}{微波炉}{wei1 bo1 lu2}[][HSK 6]
    \definition[台,个]{s.}{forno de micro-ondas}
  \end{Phonetics}
\end{Entry}

\begin{Entry}{微软}{13,8}{⼻、⾞}
  \begin{Phonetics}{微软}{wei1ruan3}
    \definition*{s.}{Microsoft Corporation}
  \end{Phonetics}
\end{Entry}

\begin{Entry}{微信}{13,9}{⼻、⼈}
  \begin{Phonetics}{微信}{wei1 xin4}[][HSK 4]
    \definition*[个,条]{s.}{WeChat, aplicativo gratuito lançado pela Tencent em 21 de janeiro de 2011 para fornecer serviços de mensagens instantâneas para terminais inteligentes}
  \end{Phonetics}
\end{Entry}

\begin{Entry}{微型}{13,9}{⼻、⼟}
  \begin{Phonetics}{微型}{wei1xing2}
    \definition{adj.}{minúsculo}
    \definition{pref.}{micro-; mini-}
    \definition{s.}{miniatura; microescala}
  \end{Phonetics}
\end{Entry}

\begin{Entry}{微型博客}{13,9,12,9}{⼻、⼟、⼗、⼧}
  \begin{Phonetics}{微型博客}{wei1xing2 bo2ke4}
    \definition{s.}{\emph{microblog}}
  \end{Phonetics}
\end{Entry}

\begin{Entry}{微笑}{13,10}{⼻、⽵}
  \begin{Phonetics}{微笑}{wei1xiao4}[][HSK 4]
    \definition[个,丝]{s.}{sorriso; sorriso sutil}
    \definition{v.}{sorrir; rir baixinho e sutilmente}
  \end{Phonetics}
\end{Entry}

\begin{Entry}{微博}{13,12}{⼻、⼗}
  \begin{Phonetics}{微博}{wei1 bo2}[][HSK 5]
    \definition*{s.}{Weibo (um aplicativo de mídia social chinês)}
    \definition[条]{s.}{\emph{microblog}; abreviação de 微型博客}
  \seealsoref{微型博客}{wei1xing2 bo2ke4}
  \end{Phonetics}
\end{Entry}

\begin{Entry}{想}{13}{⼼}
  \begin{Phonetics}{想}{xiang3}[][HSK 1]
    \definition{v.}{pensar; ponderar; refletir | supor; contar; considerar; pensar; estimar | querer; gostaria de; sentir vontade (de fazer algo) | lembrar com saudade; sentir falta}
  \end{Phonetics}
\end{Entry}

\begin{Entry}{想不到}{13,4,8}{⼼、⼀、⼑}
  \begin{Phonetics}{想不到}{xiang3 bu2 dao4}[][HSK 6]
    \definition{adj.}{inesperado; imprevisto}
  \end{Phonetics}
\end{Entry}

\begin{Entry}{想到}{13,8}{⼼、⼑}
  \begin{Phonetics}{想到}{xiang3 dao4}[][HSK 2]
    \definition{v.}{pensar em; trazer à mente; ter no coração; ter uma ideia (na mente); ter uma ideia (no coração)}
  \end{Phonetics}
\end{Entry}

\begin{Entry}{想念}{13,8}{⼼、⼼}
  \begin{Phonetics}{想念}{xiang3nian4}[][HSK 4]
    \definition{v.}{sentir falta; pensar em; lembrar com carinho; ficar doente por; desejar ver novamente; lembrar com saudade}
  \end{Phonetics}
\end{Entry}

\begin{Entry}{想法}{13,8}{⼼、⽔}
  \begin{Phonetics}{想法}{xiang3 fa3}[][HSK 2]
    \definition[种]{s.}{ideia; opinião; pensamento; noção; o que alguém tem em mente; visões e opiniões sobre alguém ou algo obtidas através do pensamento}
    \definition{s.}{maneira de pensar | opinião | noção}
    \definition{v.}{tentar; pensar em uma maneira (de fazer algo); fazer o que puder; encontrar um jeito}
  \end{Phonetics}
\end{Entry}

\begin{Entry}{想起}{13,10}{⼼、⾛}
  \begin{Phonetics}{想起}{xiang3 qi3}[][HSK 2]
    \definition{v.}{recordar; lembrar; pensar em; trazer à mente; cruzar pelos pensamentos de alguém; passar pelo pensamento de alguém}
  \end{Phonetics}
\end{Entry}

\begin{Entry}{想象}{13,11}{⼼、⾗}
  \begin{Phonetics}{想象}{xiang3xiang4}[][HSK 4]
    \definition[个,种,面]{s.}{imaginação; refere-se ao processo mental de processamento e transformação de representações armazenadas na mente para formar novas imagens}
    \definition{v.}{imaginar; vislumbrar; visualizar; refere-se a ter uma imagem concreta de algo que não está na frente dos olhos}
  \end{Phonetics}
\end{Entry}

\begin{Entry}{想想看}{13,13,9}{⼼、⼼、⽬}
  \begin{Phonetics}{想想看}{xiang3xiang3kan4}
    \definition{v.}{pensar sobre isso}
  \end{Phonetics}
\end{Entry}

\begin{Entry}{愁}{13}{⼼}
  \begin{Phonetics}{愁}{chou2}[][HSK 5]
    \definition{adj.}{triste; pesaroso; angustiado; desconsolado; preocupado; deprimido}
    \definition{s.}{pesar; sofrimento; dor; tristeza}
    \definition{v.}{preocupar-se; estar preocupado; ficar ansioso; sentir ansiedade}
  \end{Phonetics}
\end{Entry}

\begin{Entry}{愁眉苦脸}{13,9,8,11}{⼼、⽬、⾋、⾁}
  \begin{Phonetics}{愁眉苦脸}{chou2mei2-ku3lian3}[][HSK 7-9]
    \definition{expr.}{com uma expressão preocupada e angustiada; parecendo perturbado; usar uma expressão preocupada; fazer uma cara feia}
  \end{Phonetics}
\end{Entry}

\begin{Entry}{愈}{13}{⼼}
  \begin{Phonetics}{愈}{yu4}
    \definition{adv.}{mais e mais | ainda mais}
    \definition{v.}{recuperar | curar}
  \end{Phonetics}
\end{Entry}

\begin{Entry}{意}{13}{⼼}
  \begin{Phonetics}{意}{yi4}
    \definition*{s.}{Itália, abreviação de 意大利}
    \definition{s.}{ideia; significado; pensamento | desejo; vontade; intenção | significância}
  \seealsoref{意大利}{yi4da4li4}
  \end{Phonetics}
\end{Entry}

\begin{Entry}{意义}{13,3}{⼼、⼂}
  \begin{Phonetics}{意义}{yi4yi4}[][HSK 3]
    \definition[个,种,层,重,点]{s.}{sentido; significado; o significado expresso por meio de linguagem escrita ou outros sinais; o significado identificado por meio de ações ou obtenção | valor; efeito; significado; influência; impacto}
  \end{Phonetics}
\end{Entry}

\begin{Entry}{意大利}{13,3,7}{⼼、⼤、⼑}
  \begin{Phonetics}{意大利}{yi4da4li4}
    \definition*{s.}{Itália}
  \end{Phonetics}
\end{Entry}

\begin{Entry}{意见}{13,4}{⼼、⾒}
  \begin{Phonetics}{意见}{yi4jian4}[][HSK 2]
    \definition[种,点,条]{s.}{ideia; visão; opinião; sugestão; uma certa visão ou ideia sobre algo | objeção; reclamação; opinião divergente; (em relação a uma pessoa ou coisa) o sentimento de estar insatisfeito com algo porque está errado}
  \end{Phonetics}
\end{Entry}

\begin{Entry}{意外}{13,5}{⼼、⼣}
  \begin{Phonetics}{意外}{yi4wai4}[][HSK 3]
    \definition{adj.}{inesperado; imprevisto}
    \definition{adv.}{acidentalmente}
    \definition[个,种]{s.}{acidente; infortúnio; um infortúnio inesperado}
  \end{Phonetics}
\end{Entry}

\begin{Entry}{意志}{13,7}{⼼、⼼}
  \begin{Phonetics}{意志}{yi4zhi4}[][HSK 5]
    \definition[个,股]{s.}{vontade; determinação; desejo; força de vontade; o estado psicológico produzido pela determinação de atingir um determinado objetivo, muitas vezes expresso por meio de linguagem e ações}
  \end{Phonetics}
\end{Entry}

\begin{Entry}{意识}{13,7}{⼼、⾔}
  \begin{Phonetics}{意识}{yi4shi2}[][HSK 5]
    \definition{s.}{consciência; percepção; grau de reconhecimento e importância atribuído a uma determinada questão}
    \definition{s.}{consciência; percepção; o reflexo da mente humana no mundo material objetivo é a soma de vários processos psicológicos, como sensação e pensamento | consciência; conscientização; o grau de conscientização e atenção dada a um problema}
    \definition{v.}{perceber; despertar para; estar ciente de; sentir, descobrir o que antes não se sentia ou não se descobria; geralmente é usado junto com 到}
  \seealsoref{到}{dao4}
  \end{Phonetics}
\end{Entry}

\begin{Entry}{意译}{13,7}{⼼、⾔}
  \begin{Phonetics}{意译}{yi4yi4}
    \definition{s.}{tradução livre | significado (de expressão estrangeira) | paráfrase | tradução do significado (em oposição à tradução literal)}
  \seealsoref{直译}{zhi2yi4}
  \end{Phonetics}
\end{Entry}

\begin{Entry}{意味着}{13,8,11}{⼼、⼝、⽬}
  \begin{Phonetics}{意味着}{yi4wei4zhe5}[][HSK 5]
    \definition{v.}{significar; subentender; implicar; entender que tem vários significados}
  \end{Phonetics}
\end{Entry}

\begin{Entry}{意思}{13,9}{⼼、⼼}
  \begin{Phonetics}{意思}{yi4si5}[][HSK 2]
    \definition[个]{s.}{ideia; significado; o significado da linguagem e das palavras; conteúdo ideológico | desejo; vontade; opiniões | um símbolo de afeto, apreciação, gratidão, etc. | dica; traço; sugestão; refere-se principalmente ao afeto entre homens e mulheres | diversão; interesse}
    \definition{v.}{dar uma dica; demonstrar sua gratidão com presentes ou outros meios}
  \end{Phonetics}
\end{Entry}

\begin{Entry}{意指}{13,9}{⼼、⼿}
  \begin{Phonetics}{意指}{yi4zhi3}
    \definition{v.}{implicar | significar}
  \end{Phonetics}
\end{Entry}

\begin{Entry}{意想不到}{13,13,4,8}{⼼、⼼、⼀、⼑}
  \begin{Phonetics}{意想不到}{yi4 xiang3 bu2 dao4}[][HSK 6]
    \definition{expr.}{anteriormente inimaginável | inesperado}
  \end{Phonetics}
\end{Entry}

\begin{Entry}{意愿}{13,14}{⼼、⽕}
  \begin{Phonetics}{意愿}{yi4 yuan4}[][HSK 6]
    \definition{s.}{desejo; aspiração; vontade}
  \end{Phonetics}
\end{Entry}

\begin{Entry}{感}{13}{⼼}
  \begin{Phonetics}{感}{gan3}[][HSK 7-9]
    \definition{s.}{sentido; sensação; sentimento; impressão | emoção; sentimento}
    \definition{v.}{sentir; perceber; estar ciente | mover; tocar; afetar | ser grato; ser agradecido | ser afetado (pelo frio); pegar um resfriado | (fotografia) sensibilizar | ser grato; apreciar | ser afetado}
  \end{Phonetics}
\end{Entry}

\begin{Entry}{感人}{13,2}{⼼、⼈}
  \begin{Phonetics}{感人}{gan3 ren2}[][HSK 6]
    \definition{adj.}{comovente; tocante}
  \end{Phonetics}
\end{Entry}

\begin{Entry}{感叹}{13,5}{⼼、⼝}
  \begin{Phonetics}{感叹}{gan3tan4}[][HSK 7-9]
    \definition{v.}{suspirar com sentimento; suspirar por causa de um sentimento}
  \end{Phonetics}
\end{Entry}

\begin{Entry}{感兴趣}{13,6,15}{⼼、⼋、⾛}
  \begin{Phonetics}{感兴趣}{gan3xing4qu4}[][HSK 4]
    \definition{v.}{estar interessado}
  \seealsoref{对……感兴趣}{dui4 gan3xing4qu4}
  \end{Phonetics}
\end{Entry}

\begin{Entry}{感动}{13,6}{⼼、⼒}
  \begin{Phonetics}{感动}{gan3dong4}[][HSK 2]
    \definition{v.}{mover (alguém) | tocar (alguém emocionalmente)}
  \end{Phonetics}
\end{Entry}

\begin{Entry}{感到}{13,8}{⼼、⼑}
  \begin{Phonetics}{感到}{gan3 dao4}[][HSK 2]
    \definition{v.}{sentir; achar; perceber}
  \end{Phonetics}
\end{Entry}

\begin{Entry}{感受}{13,8}{⼼、⼜}
  \begin{Phonetics}{感受}{gan3shou4}[][HSK 3]
    \definition{s.}{percepção ; compreenção; sentimento; experiência; influência do contato com o mundo exterior}
    \definition{v.}{sentir; sentir (através dos sentidos); experimentar; ser afetado}
  \end{Phonetics}
\end{Entry}

\begin{Entry}{感性}{13,8}{⼼、⼼}
  \begin{Phonetics}{感性}{gan3xing4}[][HSK 7-9]
    \definition{adj.}{perceptivo; sentimental; emocional; pertencente a formas intuitivas como sensação, percepção e representação}
    \definition{s.}{percepção; sensibilidade; refere-se a pessoas que são emocionalmente ricas, sentimentais, capazes de ter empatia pelos outros, que têm grande sensibilidade e conseguem entender as mudanças emocionais de qualquer coisa}
  \end{Phonetics}
\end{Entry}

\begin{Entry}{感冒}{13,9}{⼼、⽇}
  \begin{Phonetics}{感冒}{gan3mao4}[][HSK 3]
    \definition{adj.}{interessado em}
    \definition[场,次]{s.}{resfriado; gripe comum; \emph{influenza}; doença infecciosa causada por um vírus, que tende a causar sintomas como garganta seca, congestão nasal, tosse, espirros, dor de cabeça e febre quando o corpo está excessivamente cansado, resfriado ou com a imunidade enfraquecida}
    \definition{v.}{pegar (ter) um resfriado}
  \end{Phonetics}
\end{Entry}

\begin{Entry}{感染}{13,9}{⼼、⽊}
  \begin{Phonetics}{感染}{gan3ran3}[][HSK 7-9]
    \definition{v.}{infectar; ser infectado com | infectar; afetar; influenciar; evocar os mesmos pensamentos e sentimentos}
  \end{Phonetics}
\end{Entry}

\begin{Entry}{感染力}{13,9,2}{⼼、⽊、⼒}
  \begin{Phonetics}{感染力}{gan3ran3li4}[][HSK 7-9]
    \definition{s.}{poder de mover os sentimentos; apelo | infeccioso (entusiasmo) | inspiração}
  \end{Phonetics}
\end{Entry}

\begin{Entry}{感觉}{13,9}{⼼、⾒}
  \begin{Phonetics}{感觉}{gan3jue2}[][HSK 2]
    \definition[个]{s.}{sentimento; sensação; percepção sensorial;}
    \definition{v.}{sentir; perceber; tomar consciência; sentir no coração, acreditar}
  \end{Phonetics}
\end{Entry}

\begin{Entry}{感恩}{13,10}{⼼、⼼}
  \begin{Phonetics}{感恩}{gan3/en1}[][HSK 7-9]
    \definition{v.+compl.}{sentir-se grato; ser grato; expressar gratidão pela ajuda dada por outros}
  \end{Phonetics}
\end{Entry}

\begin{Entry}{感情}{13,11}{⼼、⼼}
  \begin{Phonetics}{感情}{gan3qing2}[][HSK 3]
    \definition[份,个,种]{s.}{emoção; sentimento; reações psicológicas como amor, ódio, alegria, raiva, tristeza e felicidade, geradas por estímulos externos | amor; afeto; apego; preocupação e afeição por pessoas ou coisas}
  \end{Phonetics}
\end{Entry}

\begin{Entry}{感慨}{13,12}{⼼、⼼}
  \begin{Phonetics}{感慨}{gan3kai3}[][HSK 7-9]
    \definition{v.}{suspirar de emoção; estar cheio de emoções; ficar profundamente comovido;  geralmente expresso em palavras}
  \end{Phonetics}
\end{Entry}

\begin{Entry}{感谢}{13,12}{⼼、⾔}
  \begin{Phonetics}{感谢}{gan3xie4}[][HSK 2]
    \definition{v.}{agradecer; ser grato; expressar gratidão com palavras ou ações}
  \end{Phonetics}
\end{Entry}

\begin{Entry}{感想}{13,13}{⼼、⼼}
  \begin{Phonetics}{感想}{gan3xiang3}[][HSK 5]
    \definition[个,条]{s.}{pensamentos; impressões; reflexões; resposta do pensamento decorrente da exposição ao mundo exterior}
  \end{Phonetics}
\end{Entry}

\begin{Entry}{感触}{13,13}{⼼、⾓}
  \begin{Phonetics}{感触}{gan3chu4}[][HSK 7-9]
    \definition{s.}{sentimento; pensamentos e sentimentos; pensamentos e emoções causados ​​por fatores externos}
  \end{Phonetics}
\end{Entry}

\begin{Entry}{感激}{13,16}{⼼、⽔}
  \begin{Phonetics}{感激}{gan3ji1}[][HSK 7-9]
    \definition{v.}{apreciar; ser grato; sentir-se grato; sentir-se em dívida; desenvolver uma impressão favorável de alguém por causa de sua gentileza ou ajuda}
  \end{Phonetics}
\end{Entry}

\begin{Entry}{慈}{13}{⼼}
  \begin{Phonetics}{慈}{ci2}
    \definition*{s.}{Sobrenome Ci}
    \definition{adj.}{gentil; amoroso}
    \definition{s.}{mãe; refere-se à mãe}
    \definition{v.}{Literário: amar (amor de cima para baixo)}
  \end{Phonetics}
\end{Entry}

\begin{Entry}{慈祥}{13,10}{⼼、⽰}
  \begin{Phonetics}{慈祥}{ci2xiang2}[][HSK 7-9]
    \definition{adj.}{gentil; benigno; amável; descreve a aparência ou atitude de uma pessoa idosa como sendo gentil, amável e acessível}
  \end{Phonetics}
\end{Entry}

\begin{Entry}{慈善}{13,12}{⼼、⼝}
  \begin{Phonetics}{慈善}{ci2shan4}[][HSK 7-9]
    \definition{adj.}{caridoso; benevolente; filantrópico; gentil e caridoso}
  \end{Phonetics}
\end{Entry}

\begin{Entry}{搏}{13}{⼿}
  \begin{Phonetics}{搏}{bo2}
    \definition{v.}{brigar; lutar; combater | atacar | bater; pulsar (coração)}
  \end{Phonetics}
\end{Entry}

\begin{Entry}{搏斗}{13,4}{⼿、⽃}
  \begin{Phonetics}{搏斗}{bo2dou4}[][HSK 7-9]
    \definition{v.}{brigar; lutar; combater | envolver-se em combate corpo a corpo}
  \end{Phonetics}
\end{Entry}

\begin{Entry}{搞}{13}{⼿}
  \begin{Phonetics}{搞}{gao3}[][HSK 5]
    \definition{v.}{fazer; realizar; estar envolvido em; engajar-se em um estudo, fazer algo em relação a, etc. | fazer; produzir; gerar; trabalhar | iniciar; estabelecer; organizar; configurar | consertar (mudar) alguém; fazer alguém sofrer | obter; assegurar; agarrar |  (seguido de um complemento) fazer com que se torne; produzir um determinado efeito ou resultado}
  \end{Phonetics}
\end{Entry}

\begin{Entry}{搞好}{13,6}{⼿、⼥}
  \begin{Phonetics}{搞好}{gao3 hao3}[][HSK 5]
    \definition{v.}{fazer um bom trabalho; fazer bem; suar; tornar submisso, tornar útil, por meio de solicitações e presentes amigáveis; amolecer}
  \end{Phonetics}
\end{Entry}

\begin{Entry}{搞乱}{13,7}{⼿、⼄}
  \begin{Phonetics}{搞乱}{gao3luan4}
    \definition{v.}{estragar | confundir | bagunçar}
  \end{Phonetics}
\end{Entry}

\begin{Entry}{搞定}{13,8}{⼿、⼧}
  \begin{Phonetics}{搞定}{gao3ding4}
    \definition{v.}{consertar | resolver}
  \end{Phonetics}
\end{Entry}

\begin{Entry}{搞鬼}{13,9}{⼿、⿁}
  \begin{Phonetics}{搞鬼}{gao3/gui3}[][HSK 7-9]
    \definition{v.+compl.}{Coloquial: pregar peças; planejar em segredo; fazer alguma travessura}
  \end{Phonetics}
\end{Entry}

\begin{Entry}{搞笑}{13,10}{⼿、⽵}
  \begin{Phonetics}{搞笑}{gao3xiao4}[][HSK 7-9]
    \definition{adj.}{engraçado; divertido; descreve um estado ou qualidade que é interessante, engraçado ou faz as pessoas rirem}
    \definition{v.}{fazer palhaçadas para provocar risos; fazer as pessoas rirem deliberadamente; criar piadas e fazer as pessoas rirem}
  \end{Phonetics}
\end{Entry}

\begin{Entry}{搞通}{13,10}{⼿、⾡}
  \begin{Phonetics}{搞通}{gao3tong1}
    \definition{v.}{entender algo}
  \end{Phonetics}
\end{Entry}

\begin{Entry}{搞钱}{13,10}{⼿、⾦}
  \begin{Phonetics}{搞钱}{gao3qian2}
    \definition{v.}{fazer dinheiro | acumular dinheiro}
  \end{Phonetics}
\end{Entry}

\begin{Entry}{搞混}{13,11}{⼿、⽔}
  \begin{Phonetics}{搞混}{gao3hun4}
    \definition{v.}{confundir; embaralhar}
  \end{Phonetics}
\end{Entry}

\begin{Entry}{搞错}{13,13}{⼿、⾦}
  \begin{Phonetics}{搞错}{gao3cuo4}
    \definition{v.}{cometer um erro}
  \end{Phonetics}
\end{Entry}

\begin{Entry}{搬}{13}{⼿}
  \begin{Phonetics}{搬}{ban1}[][HSK 3]
    \definition{v.}{tirar; mover; remover | mudar-se (de casa) | aplicar indiscriminadamente; copiar mecanicamente}
  \end{Phonetics}
\end{Entry}

\begin{Entry}{搬口}{13,3}{⼿、⼝}
  \begin{Phonetics}{搬口}{ban1kou3}
    \definition{v.}{tagarelar | (idioma) transmitir histórias;  semear dissensão | contar histórias}
  \end{Phonetics}
\end{Entry}

\begin{Entry}{搬动}{13,6}{⼿、⼒}
  \begin{Phonetics}{搬动}{ban1dong4}
    \definition{v.}{mover (algo ao redor) | mudar de casa}
  \end{Phonetics}
\end{Entry}

\begin{Entry}{搬迁}{13,6}{⼿、⾡}
  \begin{Phonetics}{搬迁}{ban1qian1}[][HSK 7-9]
    \definition{v.}{mover; transferir; realocar}
  \end{Phonetics}
\end{Entry}

\begin{Entry}{搬弄}{13,7}{⼿、⼶}
  \begin{Phonetics}{搬弄}{ban1nong4}
    \definition{v.}{causar problemas | mexer com alguém | mostrar (o que se pode fazer)}
  \end{Phonetics}
\end{Entry}

\begin{Entry}{搬走}{13,7}{⼿、⾛}
  \begin{Phonetics}{搬走}{ban1zou3}
    \definition{v.}{carregar}
  \end{Phonetics}
\end{Entry}

\begin{Entry}{搬运}{13,7}{⼿、⾡}
  \begin{Phonetics}{搬运}{ban1yun4}
    \definition{v.}{carregar; transportar}
  \end{Phonetics}
\end{Entry}

\begin{Entry}{搬家}{13,10}{⼿、⼧}
  \begin{Phonetics}{搬家}{ban1/jia1}[][HSK 3]
    \definition{v.+compl.}{mudar de casa; mudar-se para outro lugar}
  \end{Phonetics}
\end{Entry}

\begin{Entry}{摄}{13}{⼿}
  \begin{Phonetics}{摄}{she4}
    \definition*{s.}{Sobrenome She}
    \definition{v.}{absorver; assimilar | tirar uma fotografia de; fotografar | conservar (a saúde) | atuar}
  \end{Phonetics}
\end{Entry}

\begin{Entry}{摄氏}{13,4}{⼿、⽒}
  \begin{Phonetics}{摄氏}{she4shi4}
    \definition{s.}{graus Celsius (°C), centígrado}
  \end{Phonetics}
\end{Entry}

\begin{Entry}{摄像}{13,13}{⼿、⼈}
  \begin{Phonetics}{摄像}{she4 xiang4}[][HSK 5]
    \definition{v.}{gravar; filmar; filmar com câmera; fazer uma gravação de vídeo (com uma câmera de vídeo ou TV)}
  \end{Phonetics}
\end{Entry}

\begin{Entry}{摄像机}{13,13,6}{⼿、⼈、⽊}
  \begin{Phonetics}{摄像机}{she4 xiang4 ji1}[][HSK 5]
    \definition[个,部,台]{s.}{câmera de vídeo; dispositivo que pode ser usado para converter imagens captadas em sinais de imagem de televisão}
  \end{Phonetics}
\end{Entry}

\begin{Entry}{摄影}{13,15}{⼿、⼺}
  \begin{Phonetics}{摄影}{she4ying3}[][HSK 5]
    \definition{v.}{fotografar; tirar uma foto; tirar fotos ou filmar}
  \end{Phonetics}
\end{Entry}

\begin{Entry}{摄影师}{13,15,6}{⼿、⼺、⼱}
  \begin{Phonetics}{摄影师}{she4 ying3 shi1}[][HSK 5]
    \definition[个,名,位]{s.}{fotógrafo; cinegrafista; operador de câmera; técnico de fotografia em estúdio fotográfico}
  \end{Phonetics}
\end{Entry}

\begin{Entry}{摆}{13}{⼿}
  \begin{Phonetics}{摆}{bai3}[][HSK 4]
    \definition*{s.}{Festival de Ganbai; uma reunião realizada nas áreas Dai durante festivais religiosos, para celebrar uma boa colheita ou para trocar materiais; geralmente se refere a uma reunião em massa | Sobrenome Bai}
    \definition{s.}{pêndulo; dispositivo mecânico que controla a frequência de oscilação em relógios e instrumentos |  a bainha inferior de um vestido, jaqueta ou saia}
    \definition{v.}{colocar; posicionar; organizar | assumir; mostrar intencionalmente | balançar; ondular; balançar para frente e para trás | revelar; listar; afirmar claramente | dizer; falar; declarar | libertar-se; livrar-se}
  \end{Phonetics}
\end{Entry}

\begin{Entry}{摆手}{13,4}{⼿、⼿}
  \begin{Phonetics}{摆手}{bai3/shou3}
    \definition{v.+compl.}{gesticular com a mão (acenando, acenando adeus, etc.) | balançar os braços | acenar com as mãos}
  \end{Phonetics}
\end{Entry}

\begin{Entry}{摆平}{13,5}{⼿、⼲}
  \begin{Phonetics}{摆平}{bai3/ping2}[][HSK 7-9]
    \definition{v.+compl.}{ser justo com; ser imparcial com; tratar com justiça | Dialeto: punir}
  \end{Phonetics}
\end{Entry}

\begin{Entry}{摆动}{13,6}{⼿、⼒}
  \begin{Phonetics}{摆动}{bai3 dong4}[][HSK 4]
    \definition{v.}{balançar; balançar para frente e para trás; oscilar; vibrar}
  \end{Phonetics}
\end{Entry}

\begin{Entry}{摆设}{13,6}{⼿、⾔}
  \begin{Phonetics}{摆设}{bai3she4}[][HSK 7-9]
    \definition{v.}{mobiliar e decorar (um cômodo)}
  \end{Phonetics}
\end{Entry}

\begin{Entry}{摆放}{13,8}{⼿、⽅}
  \begin{Phonetics}{摆放}{bai3fang4}[][HSK 7-9]
    \definition{v.}{colocar; posicionar; arranjar; organizar}
  \end{Phonetics}
\end{Entry}

\begin{Entry}{摆烂}{13,9}{⼿、⽕}
  \begin{Phonetics}{摆烂}{bai3lan4}
    \definition{v.}{(neologismo, gíria) parar de lutar (especialmente quando se sabe que não pode ter sucesso) | deixar tudo ir para o inferno}
  \end{Phonetics}
\end{Entry}

\begin{Entry}{摆脱}{13,11}{⼿、⾁}
  \begin{Phonetics}{摆脱}{bai3tuo1}[][HSK 4]
    \definition{v.}{sacudir; rejeitar; romper com; libertar-se (ou desembaraçar-se) de; livrar-se de dificuldades, escravidão, controle, etc.}
  \end{Phonetics}
\end{Entry}

\begin{Entry}{摇}{13}{⼿}
  \begin{Phonetics}{摇}{yao2}[][HSK 4]
    \definition{v.}{chacoalhar; ondular; balançar; fazer com que um objeto se mova para frente e para trás | agitar algo | sacudir; chacoalhar; agitar algo para que se mova}
  \end{Phonetics}
\end{Entry}

\begin{Entry}{摇头}{13,5}{⼿、⼤}
  \begin{Phonetics}{摇头}{yao2/tou2}[][HSK 5]
    \definition{v.+compl.}{sacudir; balançar a cabeça; balançar a cabeça para a esquerda e para a direita, indicando negação, desacordo ou impedimento}
  \end{Phonetics}
\end{Entry}

\begin{Entry}{摇晃}{13,10}{⼿、⽇}
  \begin{Phonetics}{摇晃}{yao2huang4}
    \definition{v.}{sacudir | agitar | balançar | chacoalhar}
  \end{Phonetics}
\end{Entry}

\begin{Entry}{摸}{13}{⼿}
  \begin{Phonetics}{摸}{mo1}[][HSK 4]
    \definition{v.}{sentir; acariciar; tocar; tocar (um objeto) levemente com a mão e depois removê-lo ou mover a mão suavemente sobre a superfície do objeto | sentir para; tatear para; sentir algo com as mãos | descobrir; sentir; sondar; explorar; tentar fazer ou entender | sentir o caminho; tatear no escuro; andar por estradas que você não consegue reconhecer | furtar; roubar}
  \end{Phonetics}
\end{Entry}

\begin{Entry}{数}{13}{⽁}
  \begin{Phonetics}{数}{shu3}[][HSK 2]
    \definition{v.}{contar (número); contar (número) um a um | ser considerado excepcionalmente (bom, ruim, etc.) | enumerar; listar}
  \end{Phonetics}
  \begin{Phonetics}{数}{shu4}
    \definition{num.}{vários; alguns}
    \definition{s.}{número; cifra; figura | número (conceito matemático básico que representa a quantidade de coisas) | número; indica a quantidade de coisas a que se referem os substantivos ou pronomes | destino; sorte}
  \end{Phonetics}
  \begin{Phonetics}{数}{shuo4}
    \definition{adv.}{com frequência; repetidamente; indica uma ação frequente, equivalente a 屡次}
  \seealsoref{屡次}{lv3ci4}
  \end{Phonetics}
\end{Entry}

\begin{Entry}{数目}{13,5}{⽁、⽬}
  \begin{Phonetics}{数目}{shu4 mu4}[][HSK 5]
    \definition{s.}{número; quantidade; quantidade de algo expressa em uma determinada medida padrão (como unidades de medida, etc.)}
  \end{Phonetics}
\end{Entry}

\begin{Entry}{数字}{13,6}{⽁、⼦}
  \begin{Phonetics}{数字}{shu4zi4}[][HSK 2]
    \definition{adj.}{digital; usando tecnologia digital}
    \definition[个,串]{s.}{dígito; número; um caractere que representa um número | numeral; símbolos que representam números, como algarismos arábicos, algarismos romanos, etc. | quantidade; montante}
  \end{Phonetics}
\end{Entry}

\begin{Entry}{数学}{13,8}{⽁、⼦}
  \begin{Phonetics}{数学}{shu4xue2}
    \definition{s.}{matemática; a ciência que estuda as formas espaciais e as relações quantitativas do mundo real, incluindo matemática elementar e matemática superior}
  \end{Phonetics}
\end{Entry}

\begin{Entry}{数码}{13,8}{⽁、⽯}
  \begin{Phonetics}{数码}{shu4ma3}[][HSK 4]
    \definition{s.}{dígito; numeral; algarismo | número; quantidade (usado principalmente na linguagem falada)}
    \definition{v.}{digitalizar}
  \end{Phonetics}
\end{Entry}

\begin{Entry}{数据}{13,11}{⽁、⼿}
  \begin{Phonetics}{数据}{shu4ju4}[][HSK 4]
    \definition[组,个,条]{s.}{dados; valores com base nos quais são realizadas estatísticas, cálculos, pesquisas científicas ou projetos técnicos}
  \end{Phonetics}
\end{Entry}

\begin{Entry}{数量}{13,12}{⽁、⾥}
  \begin{Phonetics}{数量}{shu4liang4}[][HSK 3]
    \definition[个,种]{s.}{quantidade; quantum; quantia; magnitude; número}
  \end{Phonetics}
\end{Entry}

\begin{Entry}{新}{13}{⽄}
  \begin{Phonetics}{新}{xin1}[][HSK 1]
    \definition*{s.}{Xinjiang, abreviação de 新疆 | Singapura, abreviação de 新加坡 | Sobrenome Xin}
    \definition{adj.}{novo; fresco; inovador; atualizado; aparecer ou ser experimentado pela primeira vez | nunca utilizado; novo; não foi usado ou foi usado por pouco tempo | recém-casado}
    \definition{adv.}{recém; recentemente; há pouco tempo}
    \definition{pref.}{(química) meso-}
    \definition{v.}{atualizar; renovar}
  \seealsoref{新加坡}{xin1jia1po1}
  \seealsoref{新疆}{xin1jiang1}
  \end{Phonetics}
\end{Entry}

\begin{Entry}{新人}{13,2}{⽄、⼈}
  \begin{Phonetics}{新人}{xin1 ren2}[][HSK 6]
    \definition[位]{s.}{pessoas de um novo tipo; nova pessoa;  pessoa que virou uma nova página | nova personalidade; novo talento | recém-chegado; novo membro | noiva ou noivo; recém-casado | \emph{neoanthropus}; \emph{homo sapiens}}
  \end{Phonetics}
\end{Entry}

\begin{Entry}{新加坡}{13,5,8}{⽄、⼒、⼟}
  \begin{Phonetics}{新加坡}{xin1jia1po1}
    \definition*{s.}{Singapura}
  \end{Phonetics}
\end{Entry}

\begin{Entry}{新兴}{13,6}{⽄、⼋}
  \begin{Phonetics}{新兴}{xin1 xing1}[][HSK 6]
    \definition[个]{adj.}{recém-desenvolvido; crescente; florescente; emergente; descreve algo que está apenas começando a se tornar popular ou se desenvolver}
  \end{Phonetics}
\end{Entry}

\begin{Entry}{新年}{13,6}{⽄、⼲}
  \begin{Phonetics}{新年}{xin1 nian2}[][HSK 1]
    \definition*[个]{s.}{Ano Novo}
  \end{Phonetics}
\end{Entry}

\begin{Entry}{新郎}{13,8}{⽄、⾢}
  \begin{Phonetics}{新郎}{xin1lang2}[][HSK 4]
    \definition[位,名,个,些]{s.}{noivo; homens no momento do casamento}
  \end{Phonetics}
\end{Entry}

\begin{Entry}{新型}{13,9}{⽄、⼟}
  \begin{Phonetics}{新型}{xin1 xing2}[][HSK 4]
    \definition[种]{s.}{ultimo modelo; novo tipo; novo padrão; novo estilo}
  \end{Phonetics}
\end{Entry}

\begin{Entry}{新闻}{13,9}{⽄、⾨}
  \begin{Phonetics}{新闻}{xin1wen2}[][HSK 2]
    \definition[个,条,则,版]{s.}{notícias; notícias nacionais e internacionais reportadas em jornais, estações de rádio, etc. | notícias; refere-se a coisas importantes ou novas que aconteceram recentemente na sociedade}
  \end{Phonetics}
\end{Entry}

\begin{Entry}{新娘}{13,10}{⽄、⼥}
  \begin{Phonetics}{新娘}{xin1niang2}[][HSK 4]
    \definition[位,个]{s.}{noiva; a mulher no momento do casamento}
  \seealsoref{新娘子}{xin1niang2zi5}
  \end{Phonetics}
\end{Entry}

\begin{Entry}{新娘子}{13,10,3}{⽄、⼥、⼦}
  \begin{Phonetics}{新娘子}{xin1niang2zi5}
    \definition{s.}{noiva}
  \seealsoref{新娘}{xin1niang2}
  \end{Phonetics}
\end{Entry}

\begin{Entry}{新娘服装}{13,10,8,12}{⽄、⼥、⽉、⾐}
  \begin{Phonetics}{新娘服装}{xin1niang2 fu2zhuang1}
    \definition{s.}{roupas de noiva}
  \end{Phonetics}
\end{Entry}

\begin{Entry}{新鲜}{13,14}{⽄、⿂}
  \begin{Phonetics}{新鲜}{xin1xian1}
    \definition{adj.}{fresco (experiência, alimento, etc.)}
    \definition{s.}{frescor}
  \end{Phonetics}
\end{Entry}

\begin{Entry}{新疆}{13,19}{⽄、⼸}
  \begin{Phonetics}{新疆}{xin1jiang1}
    \definition*{s.}{Região Autônoma Uigur de Xinjiang}
  \end{Phonetics}
\end{Entry}

\begin{Entry}{新疆维吾尔自治区}{13,19,11,7,5,6,8,4}{⽄、⼸、⽷、⼝、⼩、⾃、⽔、⼖}
  \begin{Phonetics}{新疆维吾尔自治区}{xin1jiang1 wei2wu2'er3 zi4zhi4qu1}
    \definition*{s.}{Região Autônoma Uigur de Xinjiang}
  \end{Phonetics}
\end{Entry}

\begin{Entry}{暖}{13}{⽇}
  \begin{Phonetics}{暖}{nuan3}[][HSK 5]
    \definition{adj.}{caloroso; cordial}
    \definition{v.}{aquecer; esquentar; aquecer algo ou aquecer o corpo}
  \end{Phonetics}
\end{Entry}

\begin{Entry}{暖气}{13,4}{⽇、⽓}
  \begin{Phonetics}{暖气}{nuan3qi4}[][HSK 4]
    \definition[个,种]{s.}{aquecedor; aquecimento; aquecimento central}
  \end{Phonetics}
\end{Entry}

\begin{Entry}{暖和}{13,8}{⽇、⼝}
  \begin{Phonetics}{暖和}{nuan3huo5}[][HSK 3]
    \definition{adj.}{morno; nem frio nem quente}
    \definition{v.}{aquecer; esquentar}
  \end{Phonetics}
\end{Entry}

\begin{Entry}{暗}{13}{⽇}
  \begin{Phonetics}{暗}{an4}[][HSK 4]
    \definition{adj.}{escuro; opaco; sem graça; pouca luz | escondido; secreto; não revelado | pouco claro; nebuloso; vago; confuso | subterrâneo}
    \definition{adv.}{secretamente | no escuro}
  \end{Phonetics}
\end{Entry}

\begin{Entry}{暗中}{13,4}{⽇、⼁}
  \begin{Phonetics}{暗中}{an4zhong1}[][HSK 7-9]
    \definition{adv.}{no escuro; na escuridão | em segredo; às escondidas; sorrateiramente | furtivamente; nos bastidores}
  \end{Phonetics}
\end{Entry}

\begin{Entry}{暗示}{13,5}{⽇、⽰}
  \begin{Phonetics}{暗示}{an4shi4}[][HSK 4]
    \definition[种]{s.}{sugestão; insinuação; intimação; (psicologia) refere-se ao uso de palavras, gestos, expressões, etc. para fazer as pessoas aceitarem involuntariamente uma determinada opinião ou fazerem algo}
    \definition{v.}{dar uma dica; sugerir secretamente; indicar algo a alguém usando outras palavras, expressões faciais ou gestos sem dizer em voz alta}
  \end{Phonetics}
\end{Entry}

\begin{Entry}{暗地里}{13,6,7}{⽇、⼟、⾥}
  \begin{Phonetics}{暗地里}{an4di4li3}[][HSK 7-9]
    \definition{adv.}{secretamente; interiormente; às escondidas}
  \end{Phonetics}
\end{Entry}

\begin{Entry}{暗杀}{13,6}{⽇、⽊}
  \begin{Phonetics}{暗杀}{an4sha1}[][HSK 7-9]
    \definition{s.}{assassinato}
    \definition{v.}{assassinar | matar secretamente}
  \end{Phonetics}
\end{Entry}

\begin{Entry}{暗自}{13,6}{⽇、⾃}
  \begin{Phonetics}{暗自}{an4zi4}
    \definition{adv.}{interiormente; secretamente; para si mesmo}
  \end{Phonetics}
\end{Entry}

\begin{Entry}{暗香}{13,9}{⽇、⾹}
  \begin{Phonetics}{暗香}{an4xiang1}
    \definition{s.}{fragrância sutil}
  \end{Phonetics}
\end{Entry}

\begin{Entry}{暗恋}{13,10}{⽇、⼼}
  \begin{Phonetics}{暗恋}{an4lian4}
    \definition{s.}{amor secreto}
    \definition{v.}{estar secretamente apaixonado por}
  \end{Phonetics}
\end{Entry}

\begin{Entry}{楔}{13}{⽊}
  \begin{Phonetics}{楔}{xie1}
    \definition[个]{s.}{cunha | pino; pregos de madeira; pregos de bambu}
    \definition{v.}{cunhar}
  \end{Phonetics}
\end{Entry}

\begin{Entry}{楔子}{13,3}{⽊、⼦}
  \begin{Phonetics}{楔子}{xie1zi5}
    \definition{s.}{cunha | pino | prólogo ou interlúdio no drama da Dinastia Yuan | prólogo em alguns romances modernos; introduções a óperas e romances | calço; chuteira; lascas de madeira inseridas nas juntas de encaixe e espiga, etc. | estaca de madeira; estaca de bambu; pregos de madeira; pregos de bambu}
  \end{Phonetics}
\end{Entry}

\begin{Entry}{楼}{13}{⽊}
  \begin{Phonetics}{楼}{lou2}[][HSK 1]
    \definition*{s.}{Sobrenome Lou}
    \definition{clas.}{andar, piso}
    \definition[层,座,栋]{s.}{um prédio com muitos andares | piso; andar | superestrutura; uma estrutura com um convés superior; um andar adicional construído sobre uma casa ou outro edifício | nome usado para certas lojas ou locais de entretenimento | arco ornamental; certas construções decorativas altas com passagens por baixo}
  \end{Phonetics}
\end{Entry}

\begin{Entry}{楼上}{13,3}{⽊、⼀}
  \begin{Phonetics}{楼上}{lou2 shang4}[][HSK 1]
    \definition{s.}{no andar de cima | autor anterior em um tópico do fórum; em plataformas como fóruns na internet, refere-se à pessoa que se manifesta antes de você.}
  \end{Phonetics}
\end{Entry}

\begin{Entry}{楼下}{13,3}{⽊、⼀}
  \begin{Phonetics}{楼下}{lou2 xia4}[][HSK 1]
    \definition{s.}{no andar de baixo}
  \end{Phonetics}
\end{Entry}

\begin{Entry}{楼房}{13,8}{⽊、⼾}
  \begin{Phonetics}{楼房}{lou2 fang2}[][HSK 6]
    \definition[栋,幢,座,套,层]{s.}{um edifício de dois ou mais andares}
  \end{Phonetics}
\end{Entry}

\begin{Entry}{楼梯}{13,11}{⽊、⽊}
  \begin{Phonetics}{楼梯}{lou2 ti1}[][HSK 4]
    \definition[个,层,段,阶]{s.}{escada; escadaria; degraus no meio de dois andares para permitir que as pessoas subam ou desçam as escadas}
  \end{Phonetics}
\end{Entry}

\begin{Entry}{楼道}{13,12}{⽊、⾡}
  \begin{Phonetics}{楼道}{lou2 dao4}[][HSK 6]
    \definition[个]{s.}{corredor; passagem | passagem (em edifício de vários andares)}
  \end{Phonetics}
\end{Entry}

\begin{Entry}{概}{13}{⽊}
  \begin{Phonetics}{概}{gai4}
    \definition{adj.}{geral; aproximado}
    \definition{adv.}{sem exceção; categoricamente}
    \definition{s.}{ideia principal; esboço geral | maneira de se portar e conduzir; comportamento}
    \definition{v.}{generalizar; exemplificar; tipificar}
  \end{Phonetics}
\end{Entry}

\begin{Entry}{概论}{13,6}{⽊、⾔}
  \begin{Phonetics}{概论}{gai4lun4}[][HSK 7-9]
    \definition{s.}{esboço; introdução; enquete; frequentemente usado em títulos de livros: um resumo da discussão}[«艺术历史概论»===«Introdução à História da Arte»]
  \end{Phonetics}
\end{Entry}

\begin{Entry}{概况}{13,7}{⽊、⼎}
  \begin{Phonetics}{概况}{gai4kuang4}[][HSK 7-9]
    \definition{s.}{situação geral; levantamento; breve relato (de algo); fatos básicos}[个人概况==Perfil Pessoal]
  \end{Phonetics}
\end{Entry}

\begin{Entry}{概念}{13,8}{⽊、⼼}
  \begin{Phonetics}{概念}{gai4nian4}[][HSK 3]
    \definition[个,种,项]{s.}{ideia; noção; conceito; concepção; uma forma de pensamento que resume as características comuns de algo em uma palavra}
  \end{Phonetics}
\end{Entry}

\begin{Entry}{概括}{13,9}{⽊、⼿}
  \begin{Phonetics}{概括}{gai4kuo4}[][HSK 4]
    \definition{adj.}{genérico; simples e claro, captando o conteúdo principal}
    \definition{s.}{generalização}
    \definition{v.}{generalizar; resumir}
  \end{Phonetics}
\end{Entry}

\begin{Entry}{概率}{13,11}{⽊、⽞}
  \begin{Phonetics}{概率}{gai4lv4}[][HSK 7-9]
    \definition{s.}{acaso; probabilidade; a probabilidade de um certo tipo de evento ocorrer nas mesmas condições}[成功的概率只有8\%。===A probabilidade de sucesso é de apenas 8\%.]
  \end{Phonetics}
\end{Entry}

\begin{Entry}{槐}{13}{⽊}
  \begin{Phonetics}{槐}{huai2}
    \definition*{s.}{Sobrenome Huai}
    \definition{s.}{sophora japonica; alfarrobeira; acácia}
  \end{Phonetics}
\end{Entry}

\begin{Entry}{槐树}{13,9}{⽊、⽊}
  \begin{Phonetics}{槐树}{huai2shu4}[][HSK 7-9]
    \definition[棵,株]{s.}{acácia; árvore de alfarroba; árvore de pagode}
  \end{Phonetics}
\end{Entry}

\begin{Entry}{歇}{13}{⽋}
  \begin{Phonetics}{歇}{xie1}[][HSK 5]
    \definition*{s.}{Sobrenome Xie}
    \definition{s.}{um pouco de tempo}
    \definition{v.}{descansar; fazer uma pausa | parar (o trabalho); encerrar o expediente | dormir; ir para a cama}
  \end{Phonetics}
\end{Entry}

\begin{Entry}{殿}{13}{⽎}
  \begin{Phonetics}{殿}{dian4}
    \definition*{s.}{Sobrenome Dian}
    \definition[座]{s.}{salão; palácio; templo}
    \definition{v.}{fechar a retaguarda (em uma marcha)}
  \end{Phonetics}
\end{Entry}

\begin{Entry}{殿堂}{13,11}{⽎、⼟}
  \begin{Phonetics}{殿堂}{dian4tang2}[][HSK 7-9]
    \definition{s.}{palácio; templo; santuário | salão do palácio; salão do templo}
  \end{Phonetics}
\end{Entry}

\begin{Entry}{毁}{13}{⽎}
  \begin{Phonetics}{毁}{hui3}[][HSK 6]
    \definition{v.}{destruir; arruinar; danificar | (dialeto)  transformar, remodelar um item antigo em outra coisa, geralmente roupas | queimar | difamar; caluniar}
  \end{Phonetics}
\end{Entry}

\begin{Entry}{毁灭}{13,5}{⽎、⽕}
  \begin{Phonetics}{毁灭}{hui3mie4}[][HSK 7-9]
    \definition{v.}{arruinar; destruir; exterminar; destruir ou eliminar completamente}
  \end{Phonetics}
\end{Entry}

\begin{Entry}{毁坏}{13,7}{⽎、⼟}
  \begin{Phonetics}{毁坏}{hui3huai4}[][HSK 7-9]
    \definition{v.}{danificar; devastar; degradar; destroçar; estilhaçar; vandalizar}
  \end{Phonetics}
\end{Entry}

\begin{Entry}{源}{13}{⽔}
  \begin{Phonetics}{源}{yuan2}
    \definition*{s.}{Sobrenome Yuan}
    \definition{s.}{nascente (de um rio); fonte | fonte; origem; causa}
    \definition{v.}{originar-se; provir de}
  \end{Phonetics}
\end{Entry}

\begin{Entry}{滔}{13}{⽔}
  \begin{Phonetics}{滔}{tao1}
    \definition{adj.}{(de água) transbordando | arrogante | turbulento | largo e longo; grande}
    \definition{v.}{inundar; alagar}
  \end{Phonetics}
\end{Entry}

\begin{Entry}{滔天}{13,4}{⽔、⼤}
  \begin{Phonetics}{滔天}{tao1tian1}
    \definition{adj.}{(ondas, raiva, desastres, crimes, etc.) imponente, avassalador, imenso}
  \end{Phonetics}
\end{Entry}

\begin{Entry}{滚}{13}{⽔}
  \begin{Phonetics}{滚}{gun3}[][HSK 5]
    \definition*{s.}{Sobrenome Gun}
    \definition{adj.}{rolante | fervente | precipitado; torrencial}
    \definition{adv.}{muito; em um grau elevado}
    \definition{v.}{rolar; girar; virar | escapar; fugir; ir embora | ferver | amarrar; aparar; fazer bainha}
  \end{Phonetics}
\end{Entry}

\begin{Entry}{滚动}{13,6}{⽔、⼒}
  \begin{Phonetics}{滚动}{gun3dong4}[][HSK 7-9]
    \definition{adv.}{(fazer algo) em intervalos regulares | (fazer algo) em um loop; onduladamente | expandir progressivamente (economia)}
    \definition{v.}{rolar; girar; fazer rodízio | (um trovão) fazer barulho; ribombar; estrondear | Computação: rolar}
  \end{Phonetics}
\end{Entry}

\begin{Entry}{滚轮}{13,8}{⽔、⾞}
  \begin{Phonetics}{滚轮}{gun3lun2}
    \definition{s.}{pneu | dial rotativo | roda de rolagem (\emph{scroll})  (mouse de computador)}
  \end{Phonetics}
\end{Entry}

\begin{Entry}{滚滚}{13,13}{⽔、⽔}
  \begin{Phonetics}{滚滚}{gun3 gun3}
    \definition{adj.}{ondulante | rolando continuamente}
    \definition{v.}{rolar; ondular; fluir}
  \end{Phonetics}
\end{Entry}

\begin{Entry}{满}{13}{⽔}
  \begin{Phonetics}{满}{man3}[][HSK 2]
    \definition*{s.}{Etnia Manchu | Sobrenome Man}
    \definition{adj.}{cheio; repleto; lotado; totalmente cheio; atingindo o limite da capacidade | tudo; inteiro; completo | presunçoso; complacente; orgulhoso}
    \definition{adv.}{muito; um tanto; bastante | completamente; inteiramente; perfeitamente}
    \definition{v.}{encher | sentir-se satisfeito; sentir que já é o suficiente | expirar; atingir o limite; atingir um determinado prazo ou limite}
  \end{Phonetics}
\end{Entry}

\begin{Entry}{满分}{13,4}{⽔、⼑}
  \begin{Phonetics}{满分}{man3fen1}
    \definition{s.}{pontuação completa}
  \end{Phonetics}
\end{Entry}

\begin{Entry}{满足}{13,7}{⽔、⾜}
  \begin{Phonetics}{满足}{man3zu2}[][HSK 3]
    \definition{v.}{estar satisfeito; contentar-se; sentir-se satisfeito | satisfazer}
  \end{Phonetics}
\end{Entry}

\begin{Entry}{满意}{13,13}{⽔、⼼}
  \begin{Phonetics}{满意}{man3yi4}[][HSK 2]
    \definition{adj.}{satisfeito; contente; gratificado}
    \definition{v.}{estar satisfeito; sentir-se contente; satisfazer os seus desejos; estar de acordo com os seus desejos}
  \end{Phonetics}
\end{Entry}

\begin{Entry}{满满}{13,13}{⽔、⽔}
  \begin{Phonetics}{满满}{man3man3}
    \definition{adj.}{completo | densamente empacotado}
  \end{Phonetics}
\end{Entry}

\begin{Entry}{滨}{13}{⽔}
  \begin{Phonetics}{滨}{bin1}
    \definition[个,片]{s.}{banco; beira; costa | praia; margem (beira) do rio; beira da água; perto da água}
    \definition{v.}{estar perto (do mar, de um rio, etc.)}
  \end{Phonetics}
\end{Entry}

\begin{Entry}{滨海}{13,10}{⽔、⽔}
  \begin{Phonetics}{滨海}{bin1 hai3}[][HSK 7-9]
    \definition*{s.}{Condado de Binhai em Yancheng, Jiangsu | A cidade fictícia de Binhai na sátira política}
    \definition{adj.}{costeiro}
    \definition{v.}{estar situado (ou localizado) perto do mar}
  \end{Phonetics}
\end{Entry}

\begin{Entry}{煎}{13}{⽕}
  \begin{Phonetics}{煎}{jian1}
    \definition{v.}{fritar | refogar}
  \end{Phonetics}
\end{Entry}

\begin{Entry}{煎饼}{13,9}{⽕、⾷}
  \begin{Phonetics}{煎饼}{jian1bing3}
    \definition[张]{s.}{jianbing, crepe chinês | panqueca}
  \end{Phonetics}
\end{Entry}

\begin{Entry}{煎蛋}{13,11}{⽕、⾍}
  \begin{Phonetics}{煎蛋}{jian1dan4}
    \definition{s.}{ovos fritos}
  \end{Phonetics}
\end{Entry}

\begin{Entry}{煤}{13}{⽕}
  \begin{Phonetics}{煤}{mei2}[][HSK 5]
    \definition[块,吨,斤,堆]{s.}{carvão; carvão vegetal; minério sólido preto}
  \end{Phonetics}
\end{Entry}

\begin{Entry}{煤气}{13,4}{⽕、⽓}
  \begin{Phonetics}{煤气}{mei2 qi4}[][HSK 5]
    \definition[罐,瓶]{s.}{gás; gás de carvão; gás obtido a partir do processamento do carvão não tem cor nem odor, é tóxico e pode ser queimado ou utilizado como matéria-prima na indústria química | envenenamento por monóxido de carbono}
  \end{Phonetics}
\end{Entry}

\begin{Entry}{照}{13}{⽕}
  \begin{Phonetics}{照}{zhao4}[][HSK 3]
    \definition{adv.}{de acordo com; significa agir de acordo com o original ou com determinados padrões}
    \definition{prep.}{em direção a; na direção de | de acordo com; em conformidade com}
    \definition{s.}{imagem; fotografia | permissão; licença; autorização | brilho; iluminação}
    \definition{v.}{brilhar; iluminar | refletir; espelhar; olhar para sua própria imagem em um espelho, etc. | filmar; fotografar; tirar uma foto (fotografia) | cuidar de; tomar conta de; zelar por | notificar; informar | contrastar; comparar; verificar | entender; compreender}
  \end{Phonetics}
\end{Entry}

\begin{Entry}{照片}{13,4}{⽕、⽚}
  \begin{Phonetics}{照片}{zhao4pian4}[][HSK 2]
    \definition[张,套,幅]{s.}{fotografia, foto, imagem}
  \end{Phonetics}
\end{Entry}

\begin{Entry}{照片子}{13,4,3}{⽕、⽚、⼦}
  \begin{Phonetics}{照片子}{zhao4pian4zi5}
    \definition{v.}{tirar um raio X}
  \end{Phonetics}
\end{Entry}

\begin{Entry}{照片底版}{13,4,8,8}{⽕、⽚、⼴、⽚}
  \begin{Phonetics}{照片底版}{zhao4pian4 di3ban3}
    \definition{s.}{placa fotográfica; negativo fotográfico}
  \end{Phonetics}
\end{Entry}

\begin{Entry}{照亮}{13,9}{⽕、⼇}
  \begin{Phonetics}{照亮}{zhao4liang4}
    \definition{s.}{iluminação}
    \definition{v.}{iluminar}
  \end{Phonetics}
\end{Entry}

\begin{Entry}{照相}{13,9}{⽕、⽬}
  \begin{Phonetics}{照相}{zhao4/xiang4}[][HSK 2]
    \definition{v.+compl.}{fotografar; tirar fotos; tirar uma foto; tirar uma fotografia}
  \end{Phonetics}
\end{Entry}

\begin{Entry}{照相机}{13,9,6}{⽕、⽬、⽊}
  \begin{Phonetics}{照相机}{zhao4xiang4ji1}
    \definition[个,架,部,台,只]{s.}{câmera/máquina fotográfica}
  \end{Phonetics}
\end{Entry}

\begin{Entry}{照准}{13,10}{⽕、⼎}
  \begin{Phonetics}{照准}{zhao4zhun3}
    \definition{s.}{solicitação concedida (uso formal em documento antigo)}
    \definition{v.}{mirar (arma)}
  \end{Phonetics}
\end{Entry}

\begin{Entry}{照样}{13,10}{⽕、⽊}
  \begin{Phonetics}{照样}{zhao4yang4}[][HSK 6]
    \definition{adv.}{como antes; da mesma maneira; isso significa que, embora as condições externas tenham mudado ou sido afetadas, uma determinada situação ou estado permanece inalterado | após um padrão; de um modelo}
    \definition{v.}{fazer algo de uma certa maneira}
  \end{Phonetics}
\end{Entry}

\begin{Entry}{照顾}{13,10}{⽕、⾴}
  \begin{Phonetics}{照顾}{zhao4gu4}[][HSK 2]
    \definition{v.}{cuidar; cuidar de; atender | oferecer tratamento preferencial; prestar atenção especial e dar tratamento preferencial | (de um cliente) patrocinar; comprar em; cuidar de clientes que vêm comprar coisas ou solicitar serviços em lojas ou indústrias de serviços | dar consideração a; mostrar consideração por; levar em conta; fazer concessões a}
  \end{Phonetics}
\end{Entry}

\begin{Entry}{照骗}{13,12}{⽕、⾺}
  \begin{Phonetics}{照骗}{zhao4pian4}
    \definition{s.}{Gíria da \emph{Internet}:  foto lisonjeira (trocadilho com 照片) | imagem alterada digitalmente; ``photoshopada''}
  \seealsoref{照片}{zhao4pian4}
  \end{Phonetics}
\end{Entry}

\begin{Entry}{照像}{13,13}{⽕、⼈}
  \begin{Phonetics}{照像}{zhao4/xiang4}
    \variantof{照相}
  \end{Phonetics}
\end{Entry}

\begin{Entry}{照像机}{13,13,6}{⽕、⼈、⽊}
  \begin{Phonetics}{照像机}{zhao4xiang4ji1}
    \variantof{照相机}
  \end{Phonetics}
\end{Entry}

\begin{Entry}{照耀}{13,20}{⽕、⽻}
  \begin{Phonetics}{照耀}{zhao4yao4}[][HSK 6]
    \definition{v.}{brilhar; iluminar | esclarecer}
  \end{Phonetics}
\end{Entry}

\begin{Entry}{煲}{13}{⽕}
  \begin{Phonetics}{煲}{bao1}[][HSK 7-9]
    \definition{s.}{panela; caldeira}
    \definition{v.}{cozinhar; ensopar; ferver}
  \end{Phonetics}
\end{Entry}

\begin{Entry}{献}{13}{⽝}
  \begin{Phonetics}{献}{xian4}[][HSK 5]
    \definition{v.}{oferecer; apresentar; dedicar; doar | mostrar; apresentar; exibir | exibir-se; mostrar-se para que os outros vejam}
  \end{Phonetics}
\end{Entry}

\begin{Entry}{瑜}{13}{⽟}
  \begin{Phonetics}{瑜}{yu2}
    \definition{s.}{(arcaico) jade fino; gema | (literário) brilho das gemas — virtudes; pontos positivos | excelência}
  \end{Phonetics}
\end{Entry}

\begin{Entry}{瑜伽}{13,7}{⽟、⼈}
  \begin{Phonetics}{瑜伽}{yu2jia1}
    \definition*{s.}{Ioga}
  \end{Phonetics}
\end{Entry}

\begin{Entry}{瑜珈}{13,9}{⽟、⽟}
  \begin{Phonetics}{瑜珈}{yu2jia1}
    \variantof{瑜伽}
  \end{Phonetics}
\end{Entry}

\begin{Entry}{瑰}{13}{⽟}
  \begin{Phonetics}{瑰}{gui1}
    \definition{adj.}{Literário: raro; maravilhoso; fabuloso}
    \definition[朵]{s.}{jaspe fino | Arcaico: uma espécie de pedra semelhante ao jade}
  \end{Phonetics}
\end{Entry}

\begin{Entry}{瑰宝}{13,8}{⽟、⼧}
  \begin{Phonetics}{瑰宝}{gui1bao3}[][HSK 7-9]
    \definition{s.}{raridade; tesouro; joia; coisas muito preciosas}
  \end{Phonetics}
\end{Entry}

\begin{Entry}{畸}{13}{⽥}
  \begin{Phonetics}{畸}{ji1}
    \definition{adj.}{assimétrico; desequilibrado | irregular; anormal}
    \definition{s.}{Literário: uma quantidade fracionária (acima daquela mencionada em um número redondo); lotes ímpares | deformidade}
  \end{Phonetics}
\end{Entry}

\begin{Entry}{畸形}{13,7}{⽥、⼺}
  \begin{Phonetics}{畸形}{ji1xing2}[][HSK 7-9]
    \definition{adj.}{Medicina: deformado; malformado | desequilibrado; torto; desigual; anormal | deformidade; dismorfia; dismorfose; malformação; disgenesia; monstruosidade; aberração; desenvolvimento anormal de uma parte}
  \end{Phonetics}
\end{Entry}

\begin{Entry}{痴}{13}{⽧}
  \begin{Phonetics}{痴}{chi1}
    \definition{adj.}{bobo; idiota; estúpido | louco por alguém (ou algo); extremamente obcecado por alguém ou alguma coisa | Dialeto: louco; insano; mentalmente perturbado}
  \end{Phonetics}
\end{Entry}

\begin{Entry}{痴心}{13,4}{⽧、⼼}
  \begin{Phonetics}{痴心}{chi1xin1}[][HSK 7-9]
    \definition{adj.}{apaixonado; obcecado por alguém ou alguma coisa}
    \definition{s.}{desejo tolo; amor cego; paixão cega}
  \end{Phonetics}
\end{Entry}

\begin{Entry}{痴呆}{13,7}{⽧、⼝}
  \begin{Phonetics}{痴呆}{chi1dai1}[][HSK 7-9]
    \definition{adj.}{estúpido; tolo; expressão ou comportamento enfadonho}
    \definition{s.}{demência}
  \end{Phonetics}
\end{Entry}

\begin{Entry}{痴迷}{13,9}{⽧、⾡}
  \begin{Phonetics}{痴迷}{chi1mi2}[][HSK 7-9]
    \definition{v.}{ser ou estar apaixonado; ser ou estar obcecado; ser ou estar louco por}
  \end{Phonetics}
\end{Entry}

\begin{Entry}{睡}{13}{⽬}
  \begin{Phonetics}{睡}{shui4}[][HSK 1]
    \definition{v.}{dormir | deitar-se}
  \end{Phonetics}
\end{Entry}

\begin{Entry}{睡衣}{13,6}{⽬、⾐}
  \begin{Phonetics}{睡衣}{shui4yi1}
    \definition{s.}{pijamas | roupas de dormir}
  \end{Phonetics}
\end{Entry}

\begin{Entry}{睡觉}{13,9}{⽬、⾒}
  \begin{Phonetics}{睡觉}{shui4/jiao4}[][HSK 1]
    \definition{v.+compl.}{dormir; ir para a cama; entrar em estado de sono}
  \end{Phonetics}
\end{Entry}

\begin{Entry}{睡眠}{13,10}{⽬、⽬}
  \begin{Phonetics}{睡眠}{shui4 mian2}[][HSK 5]
    \definition{s.}{sono; \emph{somnus}; sonolência}
  \end{Phonetics}
\end{Entry}

\begin{Entry}{睡着}{13,11}{⽬、⽬}
  \begin{Phonetics}{睡着}{shui4 zhao2}[][HSK 4]
    \definition{v.}{dormir; adormecer; cair no sono}
  \end{Phonetics}
\end{Entry}

\begin{Entry}{睡懒觉}{13,16,9}{⽬、⼼、⾒}
  \begin{Phonetics}{睡懒觉}{shui4lan3jiao4}
    \definition{v.}{levantar-se tarde | passar o tempo a dormir}
  \end{Phonetics}
\end{Entry}

\begin{Entry}{督}{13}{⽬}
  \begin{Phonetics}{督}{du1}
    \definition*{s.}{Sobrenome Du}
    \definition{v.}{supervisionar e comandar}
  \end{Phonetics}
\end{Entry}

\begin{Entry}{督促}{13,9}{⽬、⼈}
  \begin{Phonetics}{督促}{du1cu4}[][HSK 7-9]
    \definition{v.}{supervisionar e incentivar}
  \end{Phonetics}
\end{Entry}

\begin{Entry}{矮}{13}{⽮}
  \begin{Phonetics}{矮}{ai3}[][HSK 4]
    \definition{adj.}{baixo; baixa estatura; pequeno em altura | baixo; refere-se a grau, classificação, nível, etc.}[他比我矮。===Ele é mais baixo que eu. | 这栋楼很矮,只有三层。===Esse prédio é baixo, tem só três andares. | 她虽然矮,但是跑得很快!===Ela pode ser baixinha, mas corre muito rápido!]
  \end{Phonetics}
\end{Entry}

\begin{Entry}{矮人}{13,2}{⽮、⼈}
  \begin{Phonetics}{矮人}{ai3ren2}
    \definition{s.}{anão; pessoa de baixa estatura (indivíduo) | homúnculo; figuras criadas artificialmente pelos alquimistas em frascos de destilação | nanismo}[他虽然是矮人,但很有力气。===Embora ele seja baixo, é muito forte. | 北欧神话中的矮人是技艺高超的工匠。===Na mitologia nórdica, os anões são artesãos habilidosos. | 他因为身高被嘲笑为‘矮人’,这让他很伤心。===Ele foi zombado por ser chamado de ‘anão’ devido à sua altura, o que o magoou.]
  \end{Phonetics}
\end{Entry}

\begin{Entry}{矮子}{13,3}{⽮、⼦}
  \begin{Phonetics}{矮子}{ai3zi5}
    \definition{s.}{pessoa baixa; anão; baixinho}[白雪公主和七个小矮子住在森林里。===Branca de Neve e os sete anões vivem na floresta. | 用`矮子'称呼他人是不礼貌的。===Chamar alguém de `baixinho' é falta de educação.]
  \end{Phonetics}
\end{Entry}

\begin{Entry}{矮小}{13,3}{⽮、⼩}
  \begin{Phonetics}{矮小}{ai3 xiao3}[][HSK 4]
    \definition{adj.}{subdimensionado; curto e pequeno; baixo e pequeno | quando usado para pessoas, pode soar depreciativo se não for em contexto neutro ou afetuoso}[这位矮小的老人是村里的智者。===Este idoso baixinho é o sábio da vila. | 这种矮小的灌木适合盆栽。===Este tipo de arbusto pequeno é ideal para vasos. | 山脚下有一片矮小的房屋,显得格外宁静。===Ao pé da montanha, havia casas baixas que transmitiam uma tranquilidade única.]
  \end{Phonetics}
\end{Entry}

\begin{Entry}{矮林}{13,8}{⽮、⽊}
  \begin{Phonetics}{矮林}{ai3lin2}
    \definition{s.}{mata rasteira | bosque baixo}[这片矮林里有很多野兔和鸟类。===Neste bosque baixo há muitos coelhos selvagens e pássaros. | 山坡上长满了矮林,远看像绿色的地毯。===A encosta está coberta de mata rasteira, que de longe parece um tapete verde.]
  \end{Phonetics}
\end{Entry}

\begin{Entry}{矮星}{13,9}{⽮、⽇}
  \begin{Phonetics}{矮星}{ai3xing1}
    \definition{s.}{estrela anã}[白矮星是恒星演化的最终阶段之一。===Anãs brancas são um dos estágios finais da evolução estelar.]
  \end{Phonetics}
\end{Entry}

\begin{Entry}{矮树}{13,9}{⽮、⽊}
  \begin{Phonetics}{矮树}{ai3shu4}
    \definition[棵]{s.}{arbusto | árvore pequena, baixa}[矮树比高树更容易修剪。===Árvores baixas são mais fáceis de podar do que árvores altas. | 我们种了些矮树作为花园的边界。===Plantamos alguns arbustos como cerca natural do jardim.]
  \end{Phonetics}
\end{Entry}

\begin{Entry}{矮胖}{13,9}{⽮、⾁}
  \begin{Phonetics}{矮胖}{ai3pang4}
    \definition{adj.}{atarracado; gorducho; rechonchudo; roliço; baixo e robusto | chamar alguém diretamente de 矮胖 pode ser ofensivo}[我家猫矮胖矮胖的,像个毛球。===Meu gato é baixinho e gordinho, parece uma bolinha de pelo.]
  \end{Phonetics}
\end{Entry}

\begin{Entry}{矮凳}{13,14}{⽮、⼏}
  \begin{Phonetics}{矮凳}{ai3deng4}
    \definition{s.}{banquinho baixo | banqueta}[这个矮凳是木制的,很结实。===Este banquinho é de madeira e bem resistente.]
  \end{Phonetics}
\end{Entry}

\begin{Entry}{碍}{13}{⽯}
  \begin{Phonetics}{碍}{ai4}
    \definition{v.}{atrapalhar; dificultar; obstruir; estar no caminho de | levar em consideração}
  \end{Phonetics}
\end{Entry}

\begin{Entry}{碍事}{13,8}{⽯、⼅}
  \begin{Phonetics}{碍事}{ai4/shi4}[][HSK 7-9]
    \definition{s.}{importa; tem consequências | (usualmente em frases negativas) sem consequência, não importa}[这决定不碍什么事。===Esta decisão não importa.]
    \definition{v.+compl.}{ser um obstáculo; estar no caminho; manter-se sob os pés de alguém; afetar o trabalho; causar inconveniência}
  \end{Phonetics}
\end{Entry}

\begin{Entry}{碎}{13}{⽯}
  \begin{Phonetics}{碎}{sui4}[][HSK 5]
    \definition*{s.}{Sobrenome Sui}
    \definition{adj.}{quebrado; fragmentado | tagarela; falante}
    \definition{v.}{quebrar em pedaços; esmagar}
  \end{Phonetics}
\end{Entry}

\begin{Entry}{碑}{13}{⽯}
  \begin{Phonetics}{碑}{bei1}[][HSK 7-9]
    \definition[块,座,个,面]{s.}{estela; uma tábua de pedra; uma pedra gravada com palavras e erguida como um memorial ou marca}
  \end{Phonetics}
\end{Entry}

\begin{Entry}{碗}{13}{⽯}
  \begin{Phonetics}{碗}{wan3}[][HSK 2]
    \definition*{s.}{Sobrenome Wan}
    \definition{clas.}{usado para medição de alimentos e bebidas}
    \definition[只,个]{s.}{tigela | objeto em forma de tigela |}
  \end{Phonetics}
\end{Entry}

\begin{Entry}{碗子}{13,3}{⽯、⼦}
  \begin{Phonetics}{碗子}{wan3zi5}
    \definition{s.}{tigela}
  \end{Phonetics}
\end{Entry}

\begin{Entry}{碗柜}{13,8}{⽯、⽊}
  \begin{Phonetics}{碗柜}{wan3gui4}
    \definition{s.}{armário}
  \end{Phonetics}
\end{Entry}

\begin{Entry}{碰}{13}{⽯}
  \begin{Phonetics}{碰}{peng4}[][HSK 2]
    \definition{v.}{tocar; bater; esbarrar | encontrar; esbarrar | arriscar; tentar | tentar a sorte | reunir-se para discutir; ter uma reunião curta}
  \end{Phonetics}
\end{Entry}

\begin{Entry}{碰见}{13,4}{⽯、⾒}
  \begin{Phonetics}{碰见}{peng4 jian4}[][HSK 2]
    \definition{v.}{encontrar; encontrar-se; sem combinar, encontrar-se por acaso}
  \end{Phonetics}
\end{Entry}

\begin{Entry}{碰头}{13,5}{⽯、⼤}
  \begin{Phonetics}{碰头}{peng4/tou2}
    \definition{s.}{colisão | conflito}
    \definition{v.}{colidir}
    \definition{v.+compl.}{conhecer e discutir | juntar ideias | ver-se}
  \end{Phonetics}
\end{Entry}

\begin{Entry}{碰运气}{13,7,4}{⽯、⾡、⽓}
  \begin{Phonetics}{碰运气}{peng4yun4qi5}
    \definition{v.}{deixar algo ao acaso | tentar a sorte}
  \end{Phonetics}
\end{Entry}

\begin{Entry}{碰到}{13,8}{⽯、⼑}
  \begin{Phonetics}{碰到}{peng4 dao4}[][HSK 2]
    \definition{v.}{encontrar (com); esbarrar; cruzar}
  \end{Phonetics}
\end{Entry}

\begin{Entry}{禁}{13}{⽰}
  \begin{Phonetics}{禁}{jin4}
    \definition*{s.}{Sobrenome Jin}
    \definition{s.}{um tabu; assuntos não permitidos por lei ou costume | área proibida | residência real; o lugar onde o imperador viveu nos tempos antigos}
    \definition{v.}{proibir; banir | aprisionar; deter}
  \end{Phonetics}
\end{Entry}

\begin{Entry}{禁止}{13,4}{⽰、⽌}
  \begin{Phonetics}{禁止}{jin4zhi3}[][HSK 4]
    \definition{v.}{banir; proibir; interditar}
  \end{Phonetics}
\end{Entry}

\begin{Entry}{福}{13}{⽰}
  \begin{Phonetics}{福}{fu2}[][HSK 3]
    \definition*{s.}{Província de Fujian | Sobrenome Fu}
    \definition{s.}{benção; felicidade; boa sorte; boa fortuna; sorte (oposto de 祸)}
    \definition{v.}{(de uma mulher) fazer uma reverência; antigamente, as mulheres faziam a reverência 万福 (colocando as duas mãos na cintura do mesmo lado e dobrando ligeiramente os joelhos)}
  \seealsoref{祸}{huo4}
  \seealsoref{万福}{wan4fu2}
  \end{Phonetics}
\end{Entry}

\begin{Entry}{福气}{13,4}{⽰、⽓}
  \begin{Phonetics}{福气}{fu2qi5}[][HSK 7-9]
    \definition{s.}{bênção; boa sorte; refere-se ao destino de desfrutar de uma vida feliz}
  \end{Phonetics}
\end{Entry}

\begin{Entry}{福克斯}{13,7,12}{⽰、⼗、⽄}
  \begin{Phonetics}{福克斯}{fu2ke4si1}
    \definition*{s.}{Fox (empresa de mídia) | Focus (automóvel fabricado pela Ford)}
  \end{Phonetics}
\end{Entry}

\begin{Entry}{福利}{13,7}{⽰、⼑}
  \begin{Phonetics}{福利}{fu2li4}[][HSK 5]
    \definition[项,种]{s.}{bem-estar; benefícios materiais}
    \definition{v.}{melhorar suas condições de vida; facilitar a vida}
  \end{Phonetics}
\end{Entry}

\begin{Entry}{福泽}{13,8}{⽰、⽔}
  \begin{Phonetics}{福泽}{fu2ze2}
    \definition{s.}{boa sorte}
  \end{Phonetics}
\end{Entry}

\begin{Entry}{稠}{13}{⽲}
  \begin{Phonetics}{稠}{chou2}[][HSK 7-9]
    \definition*{s.}{Sobrenome Chou}
    \definition{adj.}{(sopa, etc.) grosso | denso}[人烟稠密。===Densamente povoado.]
  \end{Phonetics}
\end{Entry}

\begin{Entry}{稠密}{13,11}{⽲、⼧}
  \begin{Phonetics}{稠密}{chou2mi4}[][HSK 7-9]
    \definition{adj.}{denso; muitos e densos}[这个地区的森林非常稠密。===As florestas nesta área são muito densas.]
  \end{Phonetics}
\end{Entry}

\begin{Entry}{筷}{13}{⽵}
  \begin{Phonetics}{筷}{kuai4}
    \definition[双,根,个]{s.}{pauzinhos para comer}
  \end{Phonetics}
\end{Entry}

\begin{Entry}{筷子}{13,3}{⽵、⼦}
  \begin{Phonetics}{筷子}{kuai4zi5}[][HSK 2]
    \definition[根,双,副,把,对]{s.}{pauzinhos; \emph{chopsticks}; dois bastôes finos feitos de bambu, madeira, metal ou outro material, usados para segurar comida ou outros objetos}
  \end{Phonetics}
\end{Entry}

\begin{Entry}{筹}{13}{⽵}
  \begin{Phonetics}{筹}{chou2}[][HSK 7-9]
    \definition{clas.}{usado para pessoas, visto principalmente no vernáculo antigo}
    \definition{s.}{estratégia; estratagema; meios; método | ficha; contador; um pequeno pedaço de bambu ou madeira; frequentemente usado para contagem ou como um voucher}
    \definition{v.}{preparar; planejar; levantar}
  \end{Phonetics}
\end{Entry}

\begin{Entry}{筹办}{13,4}{⽵、⼒}
  \begin{Phonetics}{筹办}{chou2ban4}[][HSK 7-9]
    \definition{v.}{fazer preparativos; fazer arranjos}
  \end{Phonetics}
\end{Entry}

\begin{Entry}{筹划}{13,6}{⽵、⼑}
  \begin{Phonetics}{筹划}{chou2hua4}[][HSK 7-9]
    \definition{v.}{planejar e preparar; encontrar um caminho; fazer um plano}
  \end{Phonetics}
\end{Entry}

\begin{Entry}{筹备}{13,8}{⽵、⼡}
  \begin{Phonetics}{筹备}{chou2bei4}[][HSK 7-9]
    \definition{v.}{preparar; organizar; planejar}
  \end{Phonetics}
\end{Entry}

\begin{Entry}{筹码}{13,8}{⽵、⽯}
  \begin{Phonetics}{筹码}{chou2ma3}[][HSK 7-9]
    \definition{s.}{ficha; contador (usado para cálculos, frequentemente em jogos de azar como substituto de moeda)}
  \end{Phonetics}
\end{Entry}

\begin{Entry}{筹措}{13,11}{⽵、⼿}
  \begin{Phonetics}{筹措}{chou2cuo4}[][HSK 7-9]
    \definition{v.}{arrecadar (dinheiro) | coletar (fundos)}
  \end{Phonetics}
\end{Entry}

\begin{Entry}{筹集}{13,12}{⽵、⾫}
  \begin{Phonetics}{筹集}{chou2ji2}[][HSK 7-9]
    \definition{v.}{arrecadar (dinheiro)}
  \end{Phonetics}
\end{Entry}

\begin{Entry}{签}{13}{⽵}
  \begin{Phonetics}{签}{qian1}[][HSK 5]
    \definition[个,根,支]{s.}{tiras de bambu usadas para adivinhação ou sorteio; pPequenas tiras de bambu ou varas finas com caracteres e símbolos gravados, usadas para adivinhação, jogos de azar ou como fichas para contagem, etc. | etiqueta; adesivo; pequena tira usada como marca | um pedaço fino e pontiagudo de bambu ou madeira; pequeno bastão pontiagudo}
    \definition{v.}{assinar; autografar; escrever o nome, palavras ou fazer marcas em documentos ou recibos | fazer comentários breves em um documento; escrever brevemente (pontos principais ou opiniões) | (em costura) alinhavar; costura grosseira}
  \end{Phonetics}
\end{Entry}

\begin{Entry}{签订}{13,4}{⽵、⾔}
  \begin{Phonetics}{签订}{qian1 ding4}[][HSK 5]
    \definition{v.}{concluir e assinar (um tratado, etc.)}
  \end{Phonetics}
\end{Entry}

\begin{Entry}{签名}{13,6}{⽵、⼝}
  \begin{Phonetics}{签名}{qian1/ming2}[][HSK 5]
    \definition[个,次]{s.}{assinatura; autógrafo}
    \definition{v.+compl.}{assinar o próprio nome; autografar; escrever seu nome para indicar concordância, apoio ou homenagem, etc.}
  \end{Phonetics}
\end{Entry}

\begin{Entry}{签字}{13,6}{⽵、⼦}
  \begin{Phonetics}{签字}{qian1 zi4}[][HSK 5]
    \definition{v.}{assinar; colocar a assinatura; escrever seu nome à mão em documentos, recibos, etc., para demonstrar responsabilidade}
  \end{Phonetics}
\end{Entry}

\begin{Entry}{签约}{13,6}{⽵、⽷}
  \begin{Phonetics}{签约}{qian1 yue1}[][HSK 5]
    \definition{v.}{assinar um contrato; assinar contratos e tratados, frequentemente utilizado no trabalho e em cooperações comerciais}
  \end{Phonetics}
\end{Entry}

\begin{Entry}{签证}{13,7}{⽵、⾔}
  \begin{Phonetics}{签证}{qian1zheng4}[][HSK 5]
    \definition[张,个,份]{s.}{visto; visto de entrada em um país}
  \end{Phonetics}
\end{Entry}

\begin{Entry}{简}{13}{⽵}
  \begin{Phonetics}{简}{jian3}
    \definition*{s.}{Sobrenome Jian}
    \definition{adj.}{simples; simplificado; breve (oposto a 繁) | breve; em resumo; em poucas palavras}
    \definition{s.}{Arcaico: tiras de bambu (para escrever) | carta; correspondência}
    \definition{v.}{simplificar | (literário) selecionar; escolher}
  \seealsoref{繁}{fan2}
  \end{Phonetics}
\end{Entry}

\begin{Entry}{简介}{13,4}{⽵、⼈}
  \begin{Phonetics}{简介}{jian3 jie4}[][HSK 6]
    \definition{s.}{breve introdução; sinopse; relato resumido}
    \definition{v.}{fazer um breve relato de (algo)}
  \end{Phonetics}
\end{Entry}

\begin{Entry}{简历}{13,4}{⽵、⼚}
  \begin{Phonetics}{简历}{jian3li4}[][HSK 4]
    \definition[个,份]{s.}{currículo; \emph{curriculum vitae} (CV); notas biográficas}
  \end{Phonetics}
\end{Entry}

\begin{Entry}{简单}{13,8}{⽵、⼗}
  \begin{Phonetics}{简单}{jian3dan1}[][HSK 3]
    \definition{adj.}{simples; descomplicado; estrutura simples; poucas complicações; fácil de entender, usar ou lidar | comum; lugar-comum; (experiência, capacidade, etc.) comum (usado principalmente em frases negativas) | casual; simplificado; precipitado; pouco cuidadoso}
  \end{Phonetics}
\end{Entry}

\begin{Entry}{简直}{13,8}{⽵、⽬}
  \begin{Phonetics}{简直}{jian3zhi2}[][HSK 3]
    \definition{adv.}{simplesmente; de forma alguma; praticamente; significa “exatamente assim” (tom exagerado)}
  \end{Phonetics}
\end{Entry}

\begin{Entry}{粮}{13}{⽶}
  \begin{Phonetics}{粮}{liang2}
    \definition[斤,粒]{s.}{grãos; alimentos; provisões | imposto sobre grãos | nutrição | imposto agrícola; grãos como imposto agrícola}
  \end{Phonetics}
\end{Entry}

\begin{Entry}{粮食}{13,9}{⽶、⾷}
  \begin{Phonetics}{粮食}{liang2shi5}[][HSK 4]
    \definition[种,吨,袋,颗,粒]{s.}{alimentos; grãos; termo geral para os vários tipos de arroz, feijão, etc. que podem ser consumidos}
  \end{Phonetics}
\end{Entry}

\begin{Entry}{缝}{13}{⽷}
  \begin{Phonetics}{缝}{feng2}[][HSK 7-9]
    \definition{v.}{costurar; dar um ponto}
  \end{Phonetics}
  \begin{Phonetics}{缝}{feng4}[][HSK 7-9]
    \definition[道]{s.}{costura | fenda; rachadura; fissura; brecha; fresta}
  \end{Phonetics}
\end{Entry}

\begin{Entry}{缝合}{13,6}{⽷、⼝}
  \begin{Phonetics}{缝合}{feng2he2}[][HSK 7-9]
    \definition{v.}{suturar; costurar uma ferida com agulhas e linhas especiais}
  \end{Phonetics}
\end{Entry}

\begin{Entry}{缝纫}{13,6}{⽷、⽷}
  \begin{Phonetics}{缝纫}{feng2ren4}
    \definition{v.}{costurar}
  \end{Phonetics}
\end{Entry}

\begin{Entry}{缝纫机}{13,6,6}{⽷、⽷、⽊}
  \begin{Phonetics}{缝纫机}{feng2ren4ji1}
    \definition[架]{s.}{máquina de costura}
  \end{Phonetics}
\end{Entry}

\begin{Entry}{缠}{13}{⽷}
  \begin{Phonetics}{缠}{chan2}[][HSK 7-9]
    \definition*{s.}{Sobrenome Chan}
    \definition{v.}{enrolar; entrelaçar; bobinar | emaranhar | amarrar; importunar; perturbar | Dialeto: lidar com}
  \end{Phonetics}
\end{Entry}

\begin{Entry}{缤}{13}{⽷}
  \begin{Phonetics}{缤}{bin1}
    \definition{adj.}{Arcaico: vários; numerosos; profusos | Arcaico: em confusão | Arcaico: cores misturadas}
  \end{Phonetics}
\end{Entry}

\begin{Entry}{缤纷}{13,7}{⽷、⽷}
  \begin{Phonetics}{缤纷}{bin1fen1}[][HSK 7-9]
    \definition{adj.}{em profusão desenfreada; numerosos e confusos}
  \end{Phonetics}
\end{Entry}

\begin{Entry}{罪}{13}{⽹}
  \begin{Phonetics}{罪}{zui4}[][HSK 6]
    \definition{s.}{crime; culpa | falha; culpa | sofrimento; dor; dificuldade | má conduta; transgressão; negligência | agonia; dor; sofrimento}
    \definition{v.}{colocar a culpa em alguém; culpar}
  \end{Phonetics}
\end{Entry}

\begin{Entry}{罪人}{13,2}{⽹、⼈}
  \begin{Phonetics}{罪人}{zui4ren2}
    \definition{s.}{culpado; infrator; pecador (oposição a 功臣)}
  \seealsoref{功臣}{gong1chen2}
  \end{Phonetics}
\end{Entry}

\begin{Entry}{罪犯}{13,5}{⽹、⽝}
  \begin{Phonetics}{罪犯}{zui4fan4}
    \definition[名,个]{s.}{culpado; criminoso; infrator; pessoas que cometem crime}
  \end{Phonetics}
\end{Entry}

\begin{Entry}{罪行}{13,6}{⽹、⾏}
  \begin{Phonetics}{罪行}{zui4xing2}
    \definition{s.}{crime | ofensa}
  \end{Phonetics}
\end{Entry}

\begin{Entry}{罪恶}{13,10}{⽹、⼼}
  \begin{Phonetics}{罪恶}{zui4'e4}[][HSK 6]
    \definition{s.}{pecado; mal; crime; comportamento criminoso grave}
  \end{Phonetics}
\end{Entry}

\begin{Entry}{置}{13}{⽹}
  \begin{Phonetics}{置}{zhi4}
    \definition{v.}{colocar | configurar; estabelecer; instalar | comprar | organizar; consertar}
  \end{Phonetics}
\end{Entry}

\begin{Entry}{置疑}{13,14}{⽹、⽦}
  \begin{Phonetics}{置疑}{zhi4yi2}
    \definition{v.}{duvidar}
  \end{Phonetics}
\end{Entry}

\begin{Entry}{群}{13}{⽺}
  \begin{Phonetics}{群}{qun2}[][HSK 3]
    \definition*{s.}{Sobrenome Qun}
    \definition{adj.}{em grupos; numerosos}
    \definition{clas.}{usado para grupos de pessoas ou coisas; grupo; rebanho; manada}
    \definition{s.}{multidão; grupo; muitas pessoas ou coisas reunidas | as massas; um grupo de pessoas; refere-se a um grande número de pessoas}
  \end{Phonetics}
\end{Entry}

\begin{Entry}{群山}{13,3}{⽺、⼭}
  \begin{Phonetics}{群山}{qun2shan1}
    \definition{s.}{montanhas | uma cadeia de colinas}
  \end{Phonetics}
\end{Entry}

\begin{Entry}{群众}{13,6}{⽺、⼈}
  \begin{Phonetics}{群众}{qun2zhong4}[][HSK 5]
    \definition[个,名,位]{s.}{as massas; refere-se ao povo em geral | não filiado; apartidário; refere-se a pessoas que não são membros do Partido Comunista Chinês nem da Liga da Juventude Comunista | alguém que não ocupa uma posição de liderança}
  \end{Phonetics}
\end{Entry}

\begin{Entry}{群体}{13,7}{⽺、⼈}
  \begin{Phonetics}{群体}{qun2 ti3}[][HSK 5]
    \definition[个]{s.}{colônia; um conjunto composto por muitos indivíduos da mesma espécie que estão fisicamente conectados, exemplos incluem corais entre os animais e certas algas entre as plantas | grupos; refere-se, de maneira geral, ao conjunto formado por muitos indivíduos interligados que compartilham características essenciais em comum}
  \end{Phonetics}
\end{Entry}

\begin{Entry}{聘}{13}{⽿}
  \begin{Phonetics}{聘}{pin4}
    \definition{v.}{contratar | noivar | (de uma menina) casar ou ser casada}
  \end{Phonetics}
\end{Entry}

\begin{Entry}{聘请}{13,10}{⽿、⾔}
  \begin{Phonetics}{聘请}{pin4 qing3}[][HSK 6]
    \definition{v.}{convidar; empregar; envolver; chamar; contratar alguém para assumir uma posição}
  \end{Phonetics}
\end{Entry}

\begin{Entry}{肆}{13}{⾀}
  \begin{Phonetics}{肆}{si4}
    \definition*{s.}{Sobrenome Si}
    \definition{adj.}{desenfreado; sem limites; descuidado; imprudente}
    \definition{num.}{quatro (usado para o numeral 四 em cheques, etc., para evitar erros ou alterações)}
    \definition{s.}{Literário: loja; armazém}
  \seealsoref{四}{si4}
  \end{Phonetics}
\end{Entry}

\begin{Entry}{腰}{13}{⾁}
  \begin{Phonetics}{腰}{yao1}[][HSK 4]
    \definition*{s.}{Sobrenome Yao}
    \definition[个,尺]{s.}{cintura; região lombar | cós | bolso | parte do meio das coisas | lombo}
  \end{Phonetics}
\end{Entry}

\begin{Entry}{腰包}{13,5}{⾁、⼓}
  \begin{Phonetics}{腰包}{yao1bao1}
    \definition{s.}{pochete | bolso}
  \end{Phonetics}
\end{Entry}

\begin{Entry}{腰椎}{13,12}{⾁、⽊}
  \begin{Phonetics}{腰椎}{yao1zhui1}
    \definition{s.}{vértebra lombar (espinha dorsal inferior)}
  \end{Phonetics}
\end{Entry}

\begin{Entry}{腹}{13}{⾁}
  \begin{Phonetics}{腹}{fu4}
    \definition*{s.}{Sobrenome Fu}
    \definition[个]{s.}{barriga (do corpo); abdômen; estômago | barriga (de uma garrafa, etc.) | coração; mente | parte vazia e saliente no meio de um recipiente ou vaso}
  \end{Phonetics}
\end{Entry}

\begin{Entry}{腹泻}{13,8}{⾁、⽔}
  \begin{Phonetics}{腹泻}{fu4xie4}[][HSK 7-9]
    \definition{s.}{diarreia; refere-se ao aumento da frequência de fezes aquosas, com pus ou com sangue, acompanhadas de dor abdominal, causadas por infecção intestinal ou disfunção digestiva}
  \end{Phonetics}
\end{Entry}

\begin{Entry}{腹部}{13,10}{⾁、⾢}
  \begin{Phonetics}{腹部}{fu4bu4}[][HSK 7-9]
    \definition{s.}{abdômen; estômago; barriga}
  \end{Phonetics}
\end{Entry}

\begin{Entry}{腿}{13}{⾁}
  \begin{Phonetics}{腿}{tui3}[][HSK 2]
    \definition[条,双]{s.}{perna; as partes dos humanos e dos animais que sustentam o corpo e permitem caminhar | um suporte em forma de perna; a parte inferior de um objeto que atua como uma perna e serve de suporte | presunto}
  \end{Phonetics}
\end{Entry}

\begin{Entry}{腿号}{13,5}{⾁、⼝}
  \begin{Phonetics}{腿号}{tui3hao4}
    \definition{s.}{anilha numerada (por exemplo, usada para identificar pássaros)}
  \seealsoref{腿号箍}{tui3hao4gu1}
  \end{Phonetics}
\end{Entry}

\begin{Entry}{腿号箍}{13,5,14}{⾁、⼝、⽵}
  \begin{Phonetics}{腿号箍}{tui3hao4gu1}
    \definition{s.}{anilha numerada (por exemplo, usada para identificar pássaros)}
  \seealsoref{腿号}{tui3hao4}
  \end{Phonetics}
\end{Entry}

\begin{Entry}{艁}{13}{⾈}
  \begin{Phonetics}{艁}{zao4}
    \variantof{造}
  \end{Phonetics}
\end{Entry}

\begin{Entry}{蒙}{13}{⾋}
  \begin{Phonetics}{蒙}{meng1}[][HSK 6]
    \definition{adj.}{inconsciente; sem sentido;  em coma | confuso; perplexo}
    \definition{v.}{enganar; enganar; trapacear; iludir; trair | fazer um palpite ousado; dar um palpite ousado; arriscar-se}
  \end{Phonetics}
  \begin{Phonetics}{蒙}{meng2}[][HSK 6]
    \definition*{s.}{Sobrenome Meng}
    \definition{adj.}{ignorância; analfabetismo; falta de instrução | nebuloso; aparência pequena e pouco clara, como chuva ou neblina}
    \definition{s.}{aberto; inicial}
    \definition{v.}{cobrir; espalhar | receber apoio | receber; encontrar-se com; encontrar-se; palavras respeitosas; expressam os benefícios recebidos de outros | sofrer; incorrer}
  \end{Phonetics}
  \begin{Phonetics}{蒙}{meng3}
    \definition{s.}{grupo étnico mongol; mongol}
  \end{Phonetics}
\end{Entry}

\begin{Entry}{蒙面}{13,9}{⾋、⾯}
  \begin{Phonetics}{蒙面}{meng2mian4}
    \definition{adj.}{descarado | desavergonhado | mascarado}
    \definition{v.}{cobrir o rosto | usar uma máscara}
  \end{Phonetics}
\end{Entry}

\begin{Entry}{蓝}{13}{⾋}
  \begin{Phonetics}{蓝}{lan2}[][HSK 2]
    \definition*{s.}{Sobrenome Lan}
    \definition{adj.}{azul}
    \definition{s.}{planta índigo; anil | plantas azuis; refere-se a certas plantas que podem ser usadas como corante azul ou certas plantas cujas folhas são azul-esverdeadas}
  \end{Phonetics}
\end{Entry}

\begin{Entry}{蓝色}{13,6}{⾋、⾊}
  \begin{Phonetics}{蓝色}{lan2 se4}[][HSK 2]
    \definition[抹,片,缕,团,块]{s.}{cor azul}
  \end{Phonetics}
\end{Entry}

\begin{Entry}{蓝领}{13,11}{⾋、⾴}
  \begin{Phonetics}{蓝领}{lan2 ling3}[][HSK 6]
    \definition[名,位,个]{s.}{trabalhador braçal}
  \end{Phonetics}
\end{Entry}

\begin{Entry}{虞}{13}{⾌}
  \begin{Phonetics}{虞}{yu2}
    \definition*{s.}{Reino Yu, uma dinastia lendária fundada por Shun 舜 |Yu (um estado da Dinastia Zhou 周) | Sobrenome Yu}
    \definition{s.}{Literário: suposição; previsão | Literário: ansiedade; preocupação}
    \definition{v.}{Literário: enganar; trapacear; fazer de bobo}
  \seealsoref{舜}{shun4}
  \seealsoref{周}{zhou1}
  \end{Phonetics}
\end{Entry}

\begin{Entry}{虞世南}{13,5,9}{⾌、⼀、⼗}
  \begin{Phonetics}{虞世南}{yu2 shi4'nan2}
    \definition*{s.}{Yu Shi'nan (558-638), político dos períodos Sui e Tang inicial, poeta e calígrafo, um dos Quatro Grandes Calígrafos do início da Dinastia Tang, 唐初四大家}
  \seealsoref{唐初四大家}{tang2 chu1 si4 da4jia1}
  \end{Phonetics}
\end{Entry}

\begin{Entry}{蜂}{13}{⾍}
  \begin{Phonetics}{蜂}{feng1}
    \definition{adv.}{em enxames}
    \definition[只,群,窝]{s.}{vespa | abelha}
  \end{Phonetics}
\end{Entry}

\begin{Entry}{蜂蜜}{13,14}{⾍、⾍}
  \begin{Phonetics}{蜂蜜}{feng1mi4}[][HSK 7-9]
    \definition[杯,瓶,罐,斤,碗,勺]{s.}{mel (de abelhas)}
  \end{Phonetics}
\end{Entry}

\begin{Entry}{褚}{13}{⾐}
  \begin{Phonetics}{褚}{chu3}
    \definition*{s.}{Sobrenome Chu}
  \end{Phonetics}
  \begin{Phonetics}{褚}{zhu3}
    \definition{s.}{acolchoamento (na vestimenta) | bolso}
    \definition{v.}{armazenar}
  \end{Phonetics}
\end{Entry}

\begin{Entry}{褚遂良}{13,12,7}{⾐、⾡、⾉}
  \begin{Phonetics}{褚遂良}{chu3 sui4liang2}
    \definition*{s.}{Chu Suiliang (596-659), um dos quatro grandes calígrafos do início da dinastia Tang, 唐初四大家}
  \seealsoref{唐初四大家}{tang2 chu1 si4 da4jia1}
  \end{Phonetics}
\end{Entry}

\begin{Entry}{解}{13}{⾓}
  \begin{Phonetics}{解}{jie3}[][HSK 6]
    \definition{s.}{solução; o valor de uma variável desconhecida em uma equação algébrica}
    \definition{v.}{dividir; separar | desfazer; desatar; abrir algo que esteja amarrado ou encadernado | acalmar; dissipar; dispensar; eliminar | resolver; explicar; interpretar | entender; compreender | aliviar-se (excreção de urina e fezes) | dissolver; desintegrar | (cálculo analítico) resolver; solucionar}
  \end{Phonetics}
\end{Entry}

\begin{Entry}{解开}{13,4}{⾓、⼶}
  \begin{Phonetics}{解开}{jie3 kai1}[][HSK 3]
    \definition{v.}{desatar; desamarrar; desabotoar; desamarrar ou desfazer nós}
  \end{Phonetics}
\end{Entry}

\begin{Entry}{解决}{13,6}{⾓、⼎}
  \begin{Phonetics}{解决}{jie3jue2}[][HSK 3]
    \definition{v.}{solucionar; resolver; liquidar; resolver problemas com resultados | acabar com; descartar; eliminar (o inimigo)}
  \end{Phonetics}
\end{Entry}

\begin{Entry}{解压}{13,6}{⾓、⼚}
  \begin{Phonetics}{解压}{jie3ya1}
    \definition{v.}{aliviar o estresse | (computação) descomprimir}
  \end{Phonetics}
\end{Entry}

\begin{Entry}{解放}{13,8}{⾓、⽅}
  \begin{Phonetics}{解放}{jie3fang4}[][HSK 5]
    \definition*{s.}{Libertação (que significou o fim do domínio do regime reacionário Kuomintang em 1949 e ao estabelecimento da República Popular da China)}
    \definition{v.}{libertar; emancipar; eliminar as restrições para permitir o desenvolvimento da liberdade}
  \end{Phonetics}
\end{Entry}

\begin{Entry}{解说}{13,9}{⾓、⾔}
  \begin{Phonetics}{解说}{jie3 shuo1}[][HSK 6]
    \definition{v.}{narrar; comentar; fazer um comentário; explicar oralmente}
  \end{Phonetics}
\end{Entry}

\begin{Entry}{解除}{13,9}{⾓、⾩}
  \begin{Phonetics}{解除}{jie3chu2}[][HSK 5]
    \definition{v.}{remover; aliviar; livrar-se de; eliminar}
  \end{Phonetics}
\end{Entry}

\begin{Entry}{解救}{13,11}{⾓、⽁}
  \begin{Phonetics}{解救}{jie3jiu4}
    \definition{v.}{resgatar | ajudar a sair de dificuldades | salvar a situação}
  \end{Phonetics}
\end{Entry}

\begin{Entry}{解释}{13,12}{⾓、⾤}
  \begin{Phonetics}{解释}{jie3shi4}[][HSK 4]
    \definition{v.}{explicar; expor; interpretar | analisar; explicaro significado, razões, justificativas, etc.}
  \end{Phonetics}
\end{Entry}

\begin{Entry}{解雇}{13,12}{⾓、⾫}
  \begin{Phonetics}{解雇}{jie3gu4}
    \definition{v.}{demitir; dispensar; exonerar}
  \end{Phonetics}
\end{Entry}

\begin{Entry}{触}{13}{⾓}
  \begin{Phonetics}{触}{chu4}
    \definition{v.}{tocar; contatar | atacar; dar um toque | tocar/mover alguém emocionalmente; agitar os sentimentos de alguém | levar um choque elétrico; ser eletrocutado}
  \end{Phonetics}
\end{Entry}

\begin{Entry}{触犯}{13,5}{⾓、⽝}
  \begin{Phonetics}{触犯}{chu4fan4}[][HSK 7-9]
    \definition{v.}{ofender; violar; ir contra; infringir}
  \end{Phonetics}
\end{Entry}

\begin{Entry}{触目惊心}{13,5,11,4}{⾓、⽬、⼼、⼼}
  \begin{Phonetics}{触目惊心}{chu4mu4-jing1xin1}[][HSK 7-9]
    \definition{expr.}{ver a cena que é terrível para a mente; atingir os olhos e despertar a mente; uma visão medonha; horripilante; chocante (para as pessoas); assustador | (uma cena) surpreendente; chocante}
  \end{Phonetics}
\end{Entry}

\begin{Entry}{触动}{13,6}{⾓、⼒}
  \begin{Phonetics}{触动}{chu4dong4}[][HSK 7-9]
    \definition{v.}{tocar em algo; tocar (um interruptor, uma tela, etc.) | comover alguém; despertar os sentimentos de alguém; ter uma mudança emocional causada por algum estímulo; ser movido | afetar; um comportamento que afeta, perturba ou prejudica outras pessoas}
  \end{Phonetics}
\end{Entry}

\begin{Entry}{触觉}{13,9}{⾓、⾒}
  \begin{Phonetics}{触觉}{chu4jue2}[][HSK 7-9]
    \definition{s.}{sensação tátil; sentido do tato | tato; recepção do toque; tigmestesia; pselaphesia; pselaphesis; tactilidade}
  \end{Phonetics}
\end{Entry}

\begin{Entry}{触摸}{13,13}{⾓、⼿}
  \begin{Phonetics}{触摸}{chu4mo1}[][HSK 7-9]
    \definition{s.}{tocar; acariciar brevemente uma parte do corpo}
  \end{Phonetics}
\end{Entry}

\begin{Entry}{谩}{13}{⾔}
  \begin{Phonetics}{谩}{man2}
    \definition{v.}{enganar; ludibriar; iludir}
  \end{Phonetics}
  \begin{Phonetics}{谩}{man4}
    \definition{v.}{ser desrespeitoso | caluniar | desconsiderar}
  \end{Phonetics}
\end{Entry}

\begin{Entry}{谩骂}{13,9}{⾔、⾺}
  \begin{Phonetics}{谩骂}{man4ma4}
    \definition{v.}{ridicularizar | abusar}
  \end{Phonetics}
\end{Entry}

\begin{Entry}{赖}{13}{⾙}
  \begin{Phonetics}{赖}{lai4}[][HSK 6]
    \definition*{s.}{Sobrenome Lai}
    \definition{adj.}{ruim; pobre; não é bom}
    \definition{v.}{confiar em; depender de | permanecer em um lugar; prolongar a permanência de alguém em um lugar; ficar e recusar-se a sair | negar o próprio erro ou responsabilidade; voltar atrás na palavra; repudiar; negar; não admitir culpa; não assumir responsabilidade | colocar a culpa nos outros; incriminar falsamente (acusar); acusar alguém de algo errado; acusar alguém falsamente | culpar}
  \end{Phonetics}
\end{Entry}

\begin{Entry}{跟}{13}{⾜}
  \begin{Phonetics}{跟}{gen1}[][HSK 1]
    \definition{conj.}{e; expressa uma relação de união; 和}
    \definition{prep.}{com; Introduzir objetos relacionados à mesma ação, equivalente a 同 | para; em direção a | de; introduzir o objeto de comparação; equivalente a 从, 由 | como; objetos que causam comparações e semelhanças}
    \definition[个]{s.}{calcanhar; parte posterior do pé ou parte posterior do sapato ou meia | base (de um objeto)}
    \definition{v.}{seguir; acompanhar; seguir imediatamente na mesma direção | (de uma mulher) estar casada com; casar-se com alguém}
  \seealsoref{从}{cong2}
  \seealsoref{和}{he2}
  \seealsoref{同}{tong2}
  \seealsoref{由}{you2}
  \end{Phonetics}
\end{Entry}

\begin{Entry}{跟上}{13,3}{⾜、⼀}
  \begin{Phonetics}{跟上}{gen1shang5}[][HSK 7-9]
    \definition{v.}{acompanhar; alcançar; manter-se a par de}
  \end{Phonetics}
\end{Entry}

\begin{Entry}{跟不上}{13,4,3}{⾜、⼀、⼀}
  \begin{Phonetics}{跟不上}{gen1 bu5 shang4}[][HSK 7-9]
    \definition{v.}{não é capaz de acompanhar; não conseguir alcançar}
  \end{Phonetics}
\end{Entry}

\begin{Entry}{跟前}{13,9}{⾜、⼑}
  \begin{Phonetics}{跟前}{gen1qian2}[][HSK 5]
    \definition{s.}{próximo; perto de; na frente de; (na ou para) a presença de alguém | o tempo imediatamente anterior a algum evento; tempo que se aproxima}
  \end{Phonetics}
  \begin{Phonetics}{跟前}{gen1qian5}
    \definition{v.}{(filhos de alguém) viver com alguém (exclusivamente com relação à presença ou ausência de crianças)}
  \end{Phonetics}
\end{Entry}

\begin{Entry}{跟随}{13,11}{⾜、⾩}
  \begin{Phonetics}{跟随}{gen1sui2}[][HSK 5]
    \definition{s.}{seguidor; usado para se referir a alguém que seguiu}
    \definition{v.}{seguir; ir atrás; acompanhar}
  \end{Phonetics}
\end{Entry}

\begin{Entry}{跟踪}{13,15}{⾜、⾜}
  \begin{Phonetics}{跟踪}{gen1zong1}[][HSK 7-9]
    \definition{v.}{rastrear; alcançar; seguir; seguir atrás; seguir alguém; seguir os rastros de; seguir de perto}
  \end{Phonetics}
\end{Entry}

\begin{Entry}{跨}{13}{⾜}
  \begin{Phonetics}{跨}{kua4}[][HSK 6]
    \definition{adj.}{localizado ao lado de; anexo a}
    \definition{v.}{dar um passo; andar a passos largos | disputar; ficar de pernas abertas | atravessar; ir além (dos limites de uma certa quantidade, tempo, região, etc.)}
  \end{Phonetics}
\end{Entry}

\begin{Entry}{跪}{13}{⾜}
  \begin{Phonetics}{跪}{gui4}[][HSK 6]
    \definition{v.}{ajoelhar-se; dobrar os joelhos de modo que um ou ambos os joelhos toquem o chão}
  \end{Phonetics}
\end{Entry}

\begin{Entry}{跪拜}{13,9}{⾜、⼿}
  \begin{Phonetics}{跪拜}{gui4bai4}
    \definition{v.}{prostrar-se | ajoelhar-se e adorar}
  \end{Phonetics}
\end{Entry}

\begin{Entry}{路}{13}{⾜}
  \begin{Phonetics}{路}{lu4}[][HSK 1]
    \definition*{s.}{Sobrenome Lu}
    \definition{clas.}{tipo; classe | linha; coluna; usado para um grupo de pessoas ou uma equipe; para organizar em ordem}
    \definition[条]{s.}{estrada; caminho; via | viagem; jornada; distância | maneira; meios | sequência; linha; lógica | região; distrito | rota | classe; classificação; grau | linha; fileira}
  \end{Phonetics}
\end{Entry}

\begin{Entry}{路上}{13,3}{⾜、⼀}
  \begin{Phonetics}{路上}{lu4 shang5}[][HSK 1]
    \definition{s.}{na estrada | a caminho; na rota; em processo de mudança de um lugar para outro}
  \end{Phonetics}
\end{Entry}

\begin{Entry}{路口}{13,3}{⾜、⼝}
  \begin{Phonetics}{路口}{lu4 kou3}[][HSK 1]
    \definition[个]{s.}{cruzamento; intersecção; onde as estradas se encontram}
  \end{Phonetics}
\end{Entry}

\begin{Entry}{路边}{13,5}{⾜、⾡}
  \begin{Phonetics}{路边}{lu4 bian1}[][HSK 2]
    \definition{s.}{calçada; beira da estrada; margem da rua}
  \end{Phonetics}
\end{Entry}

\begin{Entry}{路过}{13,6}{⾜、⾡}
  \begin{Phonetics}{路过}{lu4 guo4}[][HSK 6]
    \definition{v.}{passar por (algum lugar); atravessar}
  \end{Phonetics}
\end{Entry}

\begin{Entry}{路线}{13,8}{⾜、⽷}
  \begin{Phonetics}{路线}{lu4 xian4}[][HSK 3]
    \definition[条]{s.}{rota; caminho; linha; a estrada percorrida de um lugar a outro | linha; diretriz (de política, ideologia, campo de trabalho); a via fundamental a seguir em termos ideológicos, políticos ou profissionais}
  \end{Phonetics}
\end{Entry}

\begin{Entry}{跳}{13}{⾜}
  \begin{Phonetics}{跳}{tiao4}[][HSK 3]
    \definition{v.}{pular; saltar | mover para cima e para baixo | pular (por cima); fazer omissões | quicar; a força elástica faz com que o objeto se mova repentinamente para cima | pulsar; palpitar; contrair-se | pular sobre;  saltar sobre; cruzar}
  \end{Phonetics}
\end{Entry}

\begin{Entry}{跳水}{13,4}{⾜、⽔}
  \begin{Phonetics}{跳水}{tiao4 shui3}[][HSK 6]
    \definition{s.}{Esporte: mergulho}
    \definition{v.}{mergulhar | Figurativo: (preços, lucros, etc.) cair drasticamente; cair repentinamente; mergulhar; despencar | cometer suicídio pulando na água | mergulhar (na água)}
  \end{Phonetics}
\end{Entry}

\begin{Entry}{跳电}{13,5}{⾜、⽥}
  \begin{Phonetics}{跳电}{tiao4dian4}
    \definition{v.}{desarmar (um disjuntor ou interruptor)}
  \end{Phonetics}
\end{Entry}

\begin{Entry}{跳伞}{13,6}{⾜、⼈}
  \begin{Phonetics}{跳伞}{tiao4san3}
    \definition{s.}{paraquedas}
    \definition{v.}{saltar de paraquedas}
  \end{Phonetics}
\end{Entry}

\begin{Entry}{跳远}{13,7}{⾜、⾡}
  \begin{Phonetics}{跳远}{tiao4 yuan3}[][HSK 3]
    \definition{s.}{salto em distância (atletismo)}
  \end{Phonetics}
\end{Entry}

\begin{Entry}{跳挡}{13,9}{⾜、⼿}
  \begin{Phonetics}{跳挡}{tiao4dang3}
    \definition{v.}{pular marcha (de um carro) | perder a marcha}
  \end{Phonetics}
\end{Entry}

\begin{Entry}{跳蚤}{13,9}{⾜、⾍}
  \begin{Phonetics}{跳蚤}{tiao4zao5}
    \definition{s.}{pulga}
  \end{Phonetics}
\end{Entry}

\begin{Entry}{跳高}{13,10}{⾜、⾼}
  \begin{Phonetics}{跳高}{tiao4 gao1}[][HSK 3]
    \definition{s.}{salto em altura (atletismo)}
    \definition{v.}{saltar em altura}
  \end{Phonetics}
\end{Entry}

\begin{Entry}{跳绳}{13,11}{⾜、⽷}
  \begin{Phonetics}{跳绳}{tiao4sheng2}
    \definition{v.}{pular corda}
  \end{Phonetics}
\end{Entry}

\begin{Entry}{跳跳糖}{13,13,16}{⾜、⾜、⽶}
  \begin{Phonetics}{跳跳糖}{tiao4tiao4tang2}
    \definition{s.}{\emph{Pop Rocks}, \emph{popping candy}}
  \end{Phonetics}
\end{Entry}

\begin{Entry}{跳频}{13,13}{⾜、⾴}
  \begin{Phonetics}{跳频}{tiao4pin2}
    \definition{s.}{FHSS, \emph{Frequency-Hopping Spread Spectrum}, método de transmissão de sinais de rádio}
  \end{Phonetics}
\end{Entry}

\begin{Entry}{跳舞}{13,14}{⾜、⾇}
  \begin{Phonetics}{跳舞}{tiao4/wu3}[][HSK 3]
    \definition{v.+compl.}{dançar (como performance); executar dança, especialmente dança de salão}
  \end{Phonetics}
\end{Entry}

\begin{Entry}{躲}{13}{⾝}
  \begin{Phonetics}{躲}{duo3}[][HSK 5]
    \definition{v.}{esconder (a si mesmo); ocultar (a si mesmo); esconder-se | evitar; esquivar-se}
  \end{Phonetics}
\end{Entry}

\begin{Entry}{躲闪}{13,5}{⾝、⾨}
  \begin{Phonetics}{躲闪}{duo3shan3}
    \definition{v.}{desviar | evadir | esquivar (para fora do caminho)}
  \end{Phonetics}
\end{Entry}

\begin{Entry}{躲避}{13,16}{⾝、⾌}
  \begin{Phonetics}{躲避}{duo3bi4}[][HSK 7-9]
    \definition{v.}{esquivar; evitar; fugir; sair ou se esconder deliberadamente para que as pessoas não possam vê-lo | evitar; esquivar-se; fugir; deixar para trás as coisas que não são boas para você}
  \end{Phonetics}
\end{Entry}

\begin{Entry}{躲藏}{13,17}{⾝、⾋}
  \begin{Phonetics}{躲藏}{duo3cang2}[][HSK 7-9]
    \definition{v.}{esconder-se; esconder seu corpo da vista}
  \end{Phonetics}
\end{Entry}

\begin{Entry}{辐}{13}{⾞}
  \begin{Phonetics}{辐}{fu2}
    \definition{s.}{raio (de uma roda); a conexão entre o cubo e o aro de uma roda}
  \end{Phonetics}
\end{Entry}

\begin{Entry}{辐射}{13,10}{⾞、⼨}
  \begin{Phonetics}{辐射}{fu2she4}[][HSK 7-9]
    \definition{s.}{radiação;  uma forma de propagação de calor que se irradia da fonte de calor em linha reta para a área circundante; a propagação de ondas eletromagnéticas, como luz e ondas de rádio}
    \definition{v.}{irradiar}
  \end{Phonetics}
\end{Entry}

\begin{Entry}{输}{13}{⾞}
  \begin{Phonetics}{输}{shu1}[][HSK 3]
    \definition{v.}{transportar; entregar | contribuir com dinheiro; doar | perder; falhar; ser batido; ser derrotado}
  \end{Phonetics}
\end{Entry}

\begin{Entry}{输入}{13,2}{⾞、⼊}
  \begin{Phonetics}{输入}{shu1ru4}[][HSK 3]
    \definition{v.}{introduzir; importar; comprar bens, introduzir tecnologia, contratar mão de obra, introduzir capital, etc. | inserir informações, programas, dados, sinais, etc. em uma máquina}
  \end{Phonetics}
\end{Entry}

\begin{Entry}{输出}{13,5}{⾞、⼐}
  \begin{Phonetics}{输出}{shu1 chu1}[][HSK 5]
    \definition{v.}{exportar (de dentro para fora); transportar (de dentro) para fora | exportar; vender ou distribuir no exterior ou fora do país | emitir informações, programas, dados, sinais, etc. a partir de uma máquina; enviar por uma determinada instituição ou dispositivo (energia, sinal, etc.)}
  \end{Phonetics}
\end{Entry}

\begin{Entry}{辞}{13}{⾟}
  \begin{Phonetics}{辞}{ci2}[][HSK 7-9]
    \definition[首]{s.}{dicção; fraseologia | um tipo de literatura clássica chinesa; um gênero da literatura clássica | uma forma de poesia clássica}
    \definition{v.}{despedir-se | declinar | renunciar | dispensar; demitir | fugir; evitar}
  \end{Phonetics}
\end{Entry}

\begin{Entry}{辞去}{13,5}{⾟、⼛}
  \begin{Phonetics}{辞去}{ci2qu4}[][HSK 7-9]
    \definition{v.}{desistir | renunciar}
  \end{Phonetics}
\end{Entry}

\begin{Entry}{辞呈}{13,7}{⾟、⼝}
  \begin{Phonetics}{辞呈}{ci2cheng2}[][HSK 7-9]
    \definition{s.}{renúncia (por escrito)}
  \end{Phonetics}
\end{Entry}

\begin{Entry}{辞典}{13,8}{⾟、⼋}
  \begin{Phonetics}{辞典}{ci2 dian3}[][HSK 5]
    \definition[本,部]{s.}{dicionário; coleção de termos especializados ou enciclopédicos, organizados em uma determinada ordem e explicados, para fins de referência}
    \variantof{词典}
  \end{Phonetics}
\end{Entry}

\begin{Entry}{辞退}{13,9}{⾟、⾡}
  \begin{Phonetics}{辞退}{ci2tui4}[][HSK 7-9]
    \definition{v.}{dispensar; demitir; a unidade ou instituição toma a iniciativa de encerrar a relação de trabalho com o funcionário | recusar; retornar educadamente}
  \end{Phonetics}
\end{Entry}

\begin{Entry}{辞职}{13,11}{⾟、⽿}
  \begin{Phonetics}{辞职}{ci2/zhi2}[][HSK 5]
    \definition{v.+compl.}{renunciar; deixar o cargo; entregar a renúncia; pedir para ser dispensado de suas funções}
  \end{Phonetics}
\end{Entry}

\begin{Entry}{遛}{13}{⾡}
  \begin{Phonetics}{遛}{liu4}
    \definition{v.}{passear | andar (um animal) | caminhar conduzindo um animal doméstico}
  \end{Phonetics}
\end{Entry}

\begin{Entry}{遛狗}{13,8}{⾡、⽝}
  \begin{Phonetics}{遛狗}{liu4/gou3}
    \definition{v.+compl.}{passear com um cachorro}
  \end{Phonetics}
\end{Entry}

\begin{Entry}{遥}{13}{⾡}
  \begin{Phonetics}{遥}{yao2}
    \definition{adj.}{distante; remoto; longe}
  \end{Phonetics}
\end{Entry}

\begin{Entry}{遥控}{13,11}{⾡、⼿}
  \begin{Phonetics}{遥控}{yao2kong4}
    \definition{s.}{controle remoto}
    \definition{v.}{dirigir operações de um local remoto | controlar remotamente}
  \end{Phonetics}
\end{Entry}

\begin{Entry}{鄙}{13}{⾢}
  \begin{Phonetics}{鄙}{bi3}
    \definition*{s.}{Sobrenome Bi}
    \definition{adj.}{baixo; mesquinho; vulgar | rústico; básico; desprezível}
    \definition{pron.}{(auto-depreciativo) meu}
    \definition{s.}{Literário: um lugar remoto; cidade fronteiriça; cidade pequena}
    \definition{v.}{Literário: desprezar; desdenhar; menosprezar; olhar de cima para baixo}
  \end{Phonetics}
\end{Entry}

\begin{Entry}{鄙视}{13,8}{⾢、⾒}
  \begin{Phonetics}{鄙视}{bi3shi4}[][HSK 7-9]
    \definition{v.}{desprezar; desdenhar; menosprezar}
  \end{Phonetics}
\end{Entry}

\begin{Entry}{酬}{13}{⾣}
  \begin{Phonetics}{酬}{chou2}
    \definition[份]{s.}{recompensa; pagamento}
    \definition{v.}{trocar amigavelmente | cumprir; perceber | (literário) propor um brinde; brindar | pagar; reembolsar | completar; concluir}
  \end{Phonetics}
\end{Entry}

\begin{Entry}{酬劳}{13,7}{⾣、⼒}
  \begin{Phonetics}{酬劳}{chou2lao2}
    \definition{s.}{recompensa}
  \end{Phonetics}
\end{Entry}

\begin{Entry}{酱}{13}{⾣}
  \begin{Phonetics}{酱}{jiang4}[][HSK 6]
    \definition{adj.}{marinado em molho de soja; cozido em molho de soja}
    \definition{s.}{molho espesso feito de soja, farinha, etc. | molho; pasta; geleia | um condimento pastoso feito de feijão, trigo fermentados e sal}
    \definition{v.}{cozinhar ou conservar em molho de soja}
  \end{Phonetics}
\end{Entry}

\begin{Entry}{酱油}{13,8}{⾣、⽔}
  \begin{Phonetics}{酱油}{jiang4you2}[][HSK 6]
    \definition[袋,瓶,壶,桶]{s.}{molho de soja}
  \end{Phonetics}
\end{Entry}

\begin{Entry}{鉴}{13}{⾦}
  \begin{Phonetics}{鉴}{jian4}[][HSK 6]
    \definition*{s.}{Sobrenome Jian}
    \definition{expr.}{uma expressão idiomática antiga usada para escrever cartas, depois da saudação inicial para pedir que alguém leia a carta}
    \definition{s.}{espelho (feito de bronze ou latão); espelho de bronze antigo | advertência; lição objetiva}
    \definition{v.}{Literário: refletir; espelhar | inspecionar; examinar; escrutinar; olhar cuidadosamente}
  \end{Phonetics}
\end{Entry}

\begin{Entry}{鉴定}{13,8}{⾦、⼧}
  \begin{Phonetics}{鉴定}{jian4ding4}[][HSK 6]
    \definition{s.}{avaliação dos pontos fortes e fracos de uma pessoa; avaliação de pessoas ou coisas}
    \definition{v.}{avaliar; identificar; autenticar; determinar; identificar e determinar (a autenticidade e a qualidade das coisas) | conduzir uma avaliação; avaliar o desempenho de uma pessoa ao longo de um determinado período de tempo}
  \end{Phonetics}
\end{Entry}

\begin{Entry}{错}{13}{⾦}
  \begin{Phonetics}{错}{cuo4}[][HSK 1]
    \definition{adj.}{errado; equivocado; errôneo | (na negativa) nada ruim; muito bom | entrelaçado e recortado; intrincado; complexo | ruim; pobre; péssimo (usado apenas em negativas)}
    \definition{s.}{falha; demérito | erro; engano | (arcaico) pedra de amolar para polir jade}
    \definition{v.}{estar entrelaçado e serrilhado; ser intrincado | moer; esfregar | abrir caminho; sair do caminho | alternar; escalonar | estar fora de alinhamento | deslocar | evitar; fazer com que não se encontre ou não entre em conflito | polir; polir pedras preciosas | (literário) incrustar ou revestir com ouro, prata, etc. | interseccionar; cruzar; entrecruzar}
  \end{Phonetics}
\end{Entry}

\begin{Entry}{错过}{13,6}{⾦、⾡}
  \begin{Phonetics}{错过}{cuo4 guo4}[][HSK 6]
    \definition{v.}{perder (oportunidade); deixar escapar}
  \end{Phonetics}
\end{Entry}

\begin{Entry}{错位}{13,7}{⾦、⼈}
  \begin{Phonetics}{错位}{cuo4/wei4}[][HSK 7-9]
    \definition{s.}{deslocamento (por exemplo, de ossos quebrados) | julgamento errôneo | contato defeituoso | inversão (médica, por exemplo, parto prematuro) | fora de alinhamento}
    \definition{v.+compl.}{Medicina: deslocar | deslocar; inverter | extraviar}
  \end{Phonetics}
\end{Entry}

\begin{Entry}{错别字}{13,7,6}{⾦、⼑、⼦}
  \begin{Phonetics}{错别字}{cuo4bie2zi4}[][HSK 7-9]
    \definition{s.}{caracteres escritos incorretamente ou mal pronunciados; erros de digitação e ortografia}[文章里有一些错别字。===Há alguns erros de digitação no artigo.]
  \end{Phonetics}
\end{Entry}

\begin{Entry}{错觉}{13,9}{⾦、⾒}
  \begin{Phonetics}{错觉}{cuo4jue2}[][HSK 7-9]
    \definition{s.}{ilusão; concepção errônea; impressão errada; percepção incorreta de coisas objetivas devido a alguns motivos}
  \end{Phonetics}
\end{Entry}

\begin{Entry}{错误}{13,9}{⾦、⾔}
  \begin{Phonetics}{错误}{cuo4wu4}[][HSK 3]
    \definition{adj.}{equivocado; errado; errôneo; incorreto; não condizente com a realidade objetiva}
    \definition[个,次]{s.}{engano; erro; erro grosseiro; falha; coisas, comportamentos, etc. incorretos}
  \end{Phonetics}
\end{Entry}

\begin{Entry}{错综复杂}{13,11,9,6}{⾦、⽷、⼢、⽊}
  \begin{Phonetics}{错综复杂}{cuo4zong1-fu4za2}[][HSK 7-9]
    \definition{expr.}{desconcertante; complicado e confuso; complexo e misturado; muito complicado; intrincado; complexo}
  \end{Phonetics}
\end{Entry}

\begin{Entry}{锤}{13}{⾦}
  \begin{Phonetics}{锤}{chui2}
    \definition[把,个]{s.}{uma bola de metal com uma alça ou corrente, usada como arma; maça | algo como um martelo | martelo}
    \definition{v.}{martelar para dar forma; bater com um martelo}
  \end{Phonetics}
\end{Entry}

\begin{Entry}{锦}{13}{⾦}
  \begin{Phonetics}{锦}{jin3}
    \definition*{s.}{Sobrenome Jin}
    \definition{adj.}{brilhante e bonito (cores brilhantes e lindas)}
    \definition[块]{s.}{brocado; tecidos de seda com padrões coloridos}
  \end{Phonetics}
\end{Entry}

\begin{Entry}{锦上添花}{13,3,11,7}{⾦、⼀、⽔、⾋}
  \begin{Phonetics}{锦上添花}{jin3 shang4 tian1 hua1}
    \definition{expr.}{adicionar flores ao brocado --- tornar o que é bom ainda melhor; melhorar | dourando o lírio}
  \end{Phonetics}
\end{Entry}

\begin{Entry}{键}{13}{⾦}
  \begin{Phonetics}{键}{jian4}[][HSK 5]
    \definition[个]{s.}{chave | tecla (de uma máquina de escrever, piano, etc.) | Química: ligação | Literário: ferrolho (de uma porta) | pino (para máquinas)  | etapa crucial}
  \end{Phonetics}
\end{Entry}

\begin{Entry}{键盘}{13,11}{⾦、⽫}
  \begin{Phonetics}{键盘}{jian4pan2}[][HSK 5]
    \definition[台,个]{s.}{teclado; cravo; painel de teclas}
  \end{Phonetics}
\end{Entry}

\begin{Entry}{障}{13}{⾩}
  \begin{Phonetics}{障}{zhang4}
    \definition{s.}{barreira; bloco; obstáculo; obstruções}
    \definition{v.}{atrapalhar; obstruir; bloquear; cobrir}
  \end{Phonetics}
\end{Entry}

\begin{Entry}{障碍}{13,13}{⾩、⽯}
  \begin{Phonetics}{障碍}{zhang4'ai4}[][HSK 6]
    \definition[个,种]{s.}{barreira; obstáculo; bloqueio; obstrução; impedimento}
  \end{Phonetics}
\end{Entry}

\begin{Entry}{零}{13}{⾬}
  \begin{Phonetics}{零}{ling2}[][HSK 1]
    \definition*{s.}{Sobrenome Ling}
    \definition{adj.}{ímpar; dispersos; fragmentados (em oposição a 整)}
    \definition{num.}{zero; 0; também grafado como 〇; representa um número menor que qualquer número positivo e maior que qualquer número negativo; representa a ausência de quantidade | zero grau no termômetro | usado para indicar qualidade, comprimento, tempo, idade, etc. Entre dois dígitos, indica que a quantidade da unidade mais alta é acompanhada pela quantidade da unidade mais baixa | sinal de zero (0); nulo; espaço em branco para indicar números em caracteres chineses maiúsculos}
    \definition{s.}{fragmento; fração; lote ímpar; um número fracionário que não é suficiente para uma determinada unidade; um ponto decimal diferente de um inteiro}
    \definition{v.}{(de chuva, lágrimas, etc.) cair | murchar e cair}
  \seealsoref{整}{zheng3}
  \end{Phonetics}
\end{Entry}

\begin{Entry}{零下}{13,3}{⾬、⼀}
  \begin{Phonetics}{零下}{ling2 xia4}[][HSK 2]
    \definition{s.}{abaixo de zero; negativo}
  \end{Phonetics}
\end{Entry}

\begin{Entry}{零食}{13,9}{⾬、⾷}
  \begin{Phonetics}{零食}{ling2shi2}[][HSK 4]
    \definition[包,袋,盒,箱,堆]{s.}{lanches; refrescos; petiscos entre as refeições; alimentação esporádica, além das refeições normais}
  \end{Phonetics}
\end{Entry}

\begin{Entry}{零散}{13,12}{⾬、⽁}
  \begin{Phonetics}{零散}{ling2san3}
    \definition{adj.}{espalhado; disperso}
  \end{Phonetics}
\end{Entry}

\begin{Entry}{雷}{13}{⾬}
  \begin{Phonetics}{雷}{lei2}
    \definition*{s.}{Sobrenome Lei}
    \definition[声,个,颗]{s.}{trovão | (militar) mina}
  \end{Phonetics}
\end{Entry}

\begin{Entry}{雷电}{13,5}{⾬、⽥}
  \begin{Phonetics}{雷电}{lei2dian4}
    \definition{s.}{trovão e relâmpago; raio}
  \end{Phonetics}
\end{Entry}

\begin{Entry}{雷亚尔}{13,6,5}{⾬、⼆、⼩}
  \begin{Phonetics}{雷亚尔}{lei2ya4'er3}
    \definition*{s.}{Real Brasileiro}
  \end{Phonetics}
\end{Entry}

\begin{Entry}{雾}{13}{⾬}
  \begin{Phonetics}{雾}{wu4}
    \definition[层,场,阵]{s.}{neblina; pequenas gotas de água condensadas do vapor de água | pulverização fina; como muitas pequenas gotas de água na neblina}
  \end{Phonetics}
\end{Entry}

\begin{Entry}{雾气}{13,4}{⾬、⽓}
  \begin{Phonetics}{雾气}{wu4qi4}
    \definition{s.}{nevoeiro | névoa | vapor}
  \end{Phonetics}
\end{Entry}

\begin{Entry}{靶}{13}{⾰}
  \begin{Phonetics}{靶}{ba3}
    \definition{s.}{alvo; um alvo para prática de tiro}
  \end{Phonetics}
\end{Entry}

\begin{Entry}{靶子}{13,3}{⾰、⼦}
  \begin{Phonetics}{靶子}{ba3zi5}[][HSK 7-9]
    \definition[个]{s.}{alvo; alvos para prática de tiro ou arco e flecha}
  \end{Phonetics}
\end{Entry}

\begin{Entry}{颐}{13}{⾴}
  \begin{Phonetics}{颐}{yi2}
    \definition{s.}{bochecha}
    \definition{v.}{manter-se em forma; cuidar de si mesmo}
  \end{Phonetics}
\end{Entry}

\begin{Entry}{颐和园}{13,8,7}{⾴、⼝、⼞}
  \begin{Phonetics}{颐和园}{yi2he2yuan2}
    \definition*{s.}{Palácio de Verão}
  \end{Phonetics}
\end{Entry}

\begin{Entry}{频}{13}{⾴}
  \begin{Phonetics}{频}{pin2}
    \definition*{s.}{Sobrenome Pin}
    \definition{adj.}{frequente}
    \definition{adv.}{frequentemente; repetidamente}
    \definition{s.}{Física: frequência; o número de vezes que um objeto vibra por segundo}
  \end{Phonetics}
\end{Entry}

\begin{Entry}{频道}{13,12}{⾴、⾡}
  \begin{Phonetics}{频道}{pin2dao4}[][HSK 5]
    \definition[个]{s.}{canal; canal de frequência; televisão e rádio, os sinais de som e imagem ocupam um determinado canal de frequência}
  \end{Phonetics}
\end{Entry}

\begin{Entry}{频繁}{13,17}{⾴、⽷}
  \begin{Phonetics}{频繁}{pin2fan2}[][HSK 5]
    \definition{adj.}{frequentemente}
    \definition{adj.}{frequente}
  \end{Phonetics}
\end{Entry}

\begin{Entry}{魂}{13}{⿁}
  \begin{Phonetics}{魂}{hun2}[][HSK 7-9]
    \definition[个]{s.}{alma | humor; espírito | espírito elevado de uma nação}
  \end{Phonetics}
\end{Entry}

\begin{Entry}{鼓}{13}{⿎}[Kangxi 207]
  \begin{Phonetics}{鼓}{gu3}[][HSK 5]
    \definition*{s.}{Sobrenome Gu}
    \definition{adj.}{abaulado; inchado; saliente; protuberante}
    \definition{clas.}{unidades antigas de cronometragem noturna; vigílias da noite}
    \definition[个,架,面,张]{s.}{tambor; instrumento de percussão | coisas semelhantes a tambores; formato, som e função semelhantes aos de um tambor}
    \definition{v.}{soar; bater; golpear; fazer um objeto soar | ventilar; soprar com fole | agitar; despertar; ativar; incitar; revigorar | bater asas | aumentar; fazer beicinho}
  \end{Phonetics}
\end{Entry}

\begin{Entry}{鼓动}{13,6}{⿎、⼒}
  \begin{Phonetics}{鼓动}{gu3dong4}[][HSK 7-9]
    \definition{v.}{promover; ativar; agitar; despertar; inspirar as pessoas a agir | instigar; incitar}
  \end{Phonetics}
\end{Entry}

\begin{Entry}{鼓励}{13,7}{⿎、⼒}
  \begin{Phonetics}{鼓励}{gu3li4}[][HSK 5]
    \definition{v.}{incitar; encorajar; provocar e incentivar}
  \end{Phonetics}
\end{Entry}

\begin{Entry}{鼓吹}{13,7}{⿎、⼝}
  \begin{Phonetics}{鼓吹}{gu3chui1}
    \definition{v.}{defender (um réu) | Pejorativo: pregar; anunciar; exagerar gabar-se}
  \end{Phonetics}
\end{Entry}

\begin{Entry}{鼓掌}{13,12}{⿎、⼿}
  \begin{Phonetics}{鼓掌}{gu3/zhang3}[][HSK 5]
    \definition{v.+compl.}{aplaudir; bater palmas, principalmente para expressar felicidade, aprovação ou boas-vindas}
  \end{Phonetics}
\end{Entry}

\begin{Entry}{鼓舞}{13,14}{⿎、⾇}
  \begin{Phonetics}{鼓舞}{gu3wu3}[][HSK 7-9]
    \definition{adj.}{animado; elevado; inspirado; encorajado; motivado}
    \definition{v.}{animar; elevar; inspirar; encorajar; motivar}
  \end{Phonetics}
\end{Entry}

\begin{Entry}{鼠}{13}{⿏}[Kangxi 208]
  \begin{Phonetics}{鼠}{shu3}[][HSK 5]
    \definition[只]{s.}{rato; camundongo}
  \end{Phonetics}
\end{Entry}

\begin{Entry}{鼠标}{13,9}{⿏、⽊}
  \begin{Phonetics}{鼠标}{shu3biao1}[][HSK 5]
    \definition[个,只]{s.}{\emph{mouse} (de computador); dispositivo de entrada externo para computadores, usado para controlar o movimento do cursor na tela do computador, selecionar objetos de operação, executar vários comandos, etc.}
  \end{Phonetics}
\end{Entry}

%%%%% EOF %%%%%

