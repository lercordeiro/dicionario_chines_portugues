%%%
%%% 2画
%%%

\section*{2画}\addcontentsline{toc}{section}{2画}

\begin{entry}{七}{2}{⼀}
  \begin{phonetics}{七}{qi1}[][HSK 1]
    \definition{num.}{sete; 7}
  \end{phonetics}
\end{entry}

\begin{entry}{七夕}{2,3}{⼀、⼣}
  \begin{phonetics}{七夕}{qi1xi1}
    \definition*{s.}{Dia dos Namorados Chinês, quando o vaqueiro e a tecelã (牛郎织女) têm permissão para se reunirem anualmente | Festival das Meninas | Festival Duplo Sete, noite do sétimo mês lunar}
    \seeref{牛郎织女}{niu2lang2zhi1nv3}
  \end{phonetics}
\end{entry}

\begin{entry}{九}{2}{⼄}
  \begin{phonetics}{九}{jiu3}[][HSK 1]
    \definition{num.}{nove; 9}
  \end{phonetics}
\end{entry}

\begin{entry}{了}{2}{⼅}
  \begin{phonetics}{了}{le5}[][HSK 1]
    \definition{part.}{usada depois de verbos ou adjetivos para indicar que uma ação ou mudança foi concluída | usada no final de uma frase ou em uma pausa na frase, indica uma mudança, significa o surgimento de uma nova situação e expressa uma insistência ou um conselho contra algo}
  \end{phonetics}
  \begin{phonetics}{了}{liao3}
    \definition{v.}{terminar | alcançar | entender claramente}
  \end{phonetics}
  \begin{phonetics}{了}{liao4}
    \definition{adj.}{brilhantes (olhos)}
    \definition{v.}{observar | olhar para fora | olhar para baixo de um lugar mais alto | compreender claramente}
  \end{phonetics}
\end{entry}

\begin{entry}{了不起}{2,4,10}{⼅、⼀、⾛}
  \begin{phonetics}{了不起}{liao3bu5qi3}[][HSK 4]
    \definition{adj.}{incrível; fantástico; extraordinário | sério; grave}
  \end{phonetics}
\end{entry}

\begin{entry}{了解}{2,13}{⼅、⾓}
  \begin{phonetics}{了解}{liao3jie3}[][HSK 4]
    \definition{v.}{entender; compreender | investigar; indagar sobre}
  \end{phonetics}
\end{entry}

\begin{entry}{二}{2}{⼆}[Kangxi 7]
  \begin{phonetics}{二}{er4}[][HSK 1]
    \definition{num.}{dois; 2 | (dialeto de Pequim) estúpido}
  \end{phonetics}
\end{entry}

\begin{entry}{二手}{2,4}{⼆、⼿}
  \begin{phonetics}{二手}{er4 shou3}[][HSK 4]
    \definition{adj.}{usado; de segunda mão; refere-se especificamente a usados e revendidos}
  \end{phonetics}
\end{entry}

\begin{entry}{二战}{2,9}{⼆、⼽}
  \begin{phonetics}{二战}{er4zhan4}
    \definition*{s.}{Segunda Guerra Mundial}
  \end{phonetics}
\end{entry}

\begin{entry}{人}{2}{⼈}[Kangxi 9]
  \begin{phonetics}{人}{ren2}[][HSK 1]
    \definition[个,位]{s.}{pessoa | gente}
  \end{phonetics}
\end{entry}

\begin{entry}{人口}{2,3}{⼈、⼝}
  \begin{phonetics}{人口}{ren2kou3}[][HSK 2]
    \definition{s.}{pessoas | população}
  \end{phonetics}
\end{entry}

\begin{entry}{人工}{2,3}{⼈、⼯}
  \begin{phonetics}{人工}{ren2gong1}[][HSK 3]
    \definition{adj.}{feito pelo homem; artificial}
    \definition[个]{s.}{trabalho manual; trabalho feito à mão | mão de obra; homem-dia; uma unidade de cálculo da quantidade de trabalho realizado}
  \end{phonetics}
\end{entry}

\begin{entry}{人才}{2,3}{⼈、⼿}
  \begin{phonetics}{人才}{ren2cai2}[][HSK 3]
    \definition{adj.}{aparência bonita, elegante}
    \definition[个]{s.}{talento; pessoal qualificado; pessoa com capacidade}
  \end{phonetics}
\end{entry}

\begin{entry}{人们}{2,5}{⼈、⼈}
  \begin{phonetics}{人们}{ren2 men5}[][HSK 2]
    \definition{s.}{homens |  pessoas | o público}
  \end{phonetics}
\end{entry}

\begin{entry}{人民}{2,5}{⼈、⽒}
  \begin{phonetics}{人民}{ren2 min2}[][HSK 3]
    \definition[群,批,个]{s.}{o povo}
  \end{phonetics}
\end{entry}

\begin{entry}{人民币}{2,5,4}{⼈、⽒、⼱}
  \begin{phonetics}{人民币}{ren2min2bi4}[][HSK 3]
    \definition*[块,张,元]{s.}{Renminbi (RMB); Yuan Chinês (CYN); nome da moeda chinesa}
  \end{phonetics}
\end{entry}

\begin{entry}{人生}{2,5}{⼈、⽣}
  \begin{phonetics}{人生}{ren2sheng1}[][HSK 3]
    \definition{s.}{vida (tempo de alguém na Terra)}
  \end{phonetics}
\end{entry}

\begin{entry}{人权}{2,6}{⼈、⽊}
  \begin{phonetics}{人权}{ren2quan2}
    \definition*{s.}{Direitos Humanos}
  \seealsoref{人权法}{ren2quan2fa3}
  \end{phonetics}
\end{entry}

\begin{entry}{人权法}{2,6,8}{⼈、⽊、⽔}
  \begin{phonetics}{人权法}{ren2quan2fa3}
    \definition*{s.}{Direitos Humanos}
  \seealsoref{人权}{ren2quan2}
  \end{phonetics}
\end{entry}

\begin{entry}{人行道}{2,6,12}{⼈、⾏、⾡}
  \begin{phonetics}{人行道}{ren2xing2dao4}
    \definition{s.}{calçada}
  \end{phonetics}
\end{entry}

\begin{entry}{人员}{2,7}{⼈、⼝}
  \begin{phonetics}{人员}{ren2yuan2}[][HSK 3]
    \definition[个,位,名]{s.}{funcionários | pessoal}
  \end{phonetics}
\end{entry}

\begin{entry}{人材}{2,7}{⼈、⽊}
  \begin{phonetics}{人材}{ren2cai2}
    \variantof{人才}
  \end{phonetics}
\end{entry}

\begin{entry}{人间}{2,7}{⼈、⾨}
  \begin{phonetics}{人间}{ren2jian1}
    \definition{s.}{o mundo humano | a Terra}
  \end{phonetics}
\end{entry}

\begin{entry}{人鱼}{2,8}{⼈、⿂}
  \begin{phonetics}{人鱼}{ren2yu2}
    \definition{s.}{sereia | peixe-boi | salamandra gigante}
  \end{phonetics}
\end{entry}

\begin{entry}{人类}{2,9}{⼈、⽶}
  \begin{phonetics}{人类}{ren2lei4}[][HSK 3]
    \definition[种]{s.}{humano; humanidade; raça humana}
  \end{phonetics}
\end{entry}

\begin{entry}{人家}{2,10}{⼈、⼧}
  \begin{phonetics}{人家}{ren2jia1}[][HSK 4]
    \definition[对]{s.}{lar; família; família do noivo; casa do futuro marido}
  \end{phonetics}
  \begin{phonetics}{人家}{ren2jia5}
    \definition{pron.}{outros; uma pessoa ou pessoas diferentes do falante ou ouvinte; refere-se a alguém diferente de si mesmo ou de outra pessoa | certa pessoa ou pessoas (a pessoa ou pessoas mencionadas em um contexto próximo, aproximadamente equivalente ao pronome de terceira pessoa);  refere-se a uma pessoa ou algumas pessoas, com significado semelhante a ``他'' | eu; mim (usado retoricamente no lugar do primeiro pronome pessoal, muitas vezes expressando descontentamento de forma jocosa; geralmente usado quando se fala com pessoas próximas, para significar ``自己'', usado principamente por meninas)}
  \seealsoref{他}{ta1}
  \seealsoref{自己}{zi4ji3}
  \end{phonetics}
\end{entry}

\begin{entry}{人海}{2,10}{⼈、⽔}
  \begin{phonetics}{人海}{ren2hai3}
    \definition{s.}{uma multidão | um mar de pessoas}
  \end{phonetics}
\end{entry}

\begin{entry}{人道}{2,12}{⼈、⾡}
  \begin{phonetics}{人道}{ren2dao4}
    \definition{s.}{solidariedade humana | humanitarismo | humano | a ``maneira humana'', um dos estágios do ciclo de reencarnação (budismo) | relação sexual}
  \end{phonetics}
\end{entry}

\begin{entry}{人像}{2,13}{⼈、⼈}
  \begin{phonetics}{人像}{ren2xiang4}
    \definition{s.}{``retrato'' de uma pessoa (esboço, foto, escultura, etc.)}
  \end{phonetics}
\end{entry}

\begin{entry}{人数}{2,13}{⼈、⽁}
  \begin{phonetics}{人数}{ren2 shu4}[][HSK 2]
    \definition{s.}{número de pessoas}
  \end{phonetics}
\end{entry}

\begin{entry}{人群}{2,13}{⼈、⽺}
  \begin{phonetics}{人群}{ren2 qun2}[][HSK 3]
    \definition{s.}{multidão; ajuntamento; torpel; aglomeração; um grupo de pessoas}
  \end{phonetics}
\end{entry}

\begin{entry}{儿}{2}{⼉}
  \begin{phonetics}{儿}{er2}
    \definition{s.}{criança | filho}
  \end{phonetics}
  \begin{phonetics}{儿}{r5}
    \definition{suf.}{sufixo diminutivo não silábico | final retroflexo}
  \end{phonetics}
  \begin{phonetics}{儿}{ren2}
    \definition{s.}{pessoa, radical em caracteres chineses}
    \variantof{人}
  \end{phonetics}
\end{entry}

\begin{entry}{儿子}{2,3}{⼉、⼦}
  \begin{phonetics}{儿子}{er2zi5}
    \definition{s.}{filho}
  \seealsoref{女儿}{nv3'er2}
  \end{phonetics}
\end{entry}

\begin{entry}{儿童}{2,12}{⼉、⽴}
  \begin{phonetics}{儿童}{er2tong2}[][HSK 4]
    \definition[个,群]{s.}{criança; menor de idade (mais jovem do que ``少年'')}
  \seealsoref{少年}{shao4 nian2}
  \end{phonetics}
\end{entry}

\begin{entry}{儿媳}{2,13}{⼉、⼥}
  \begin{phonetics}{儿媳}{er2xi2}
    \definition{s.}{esposa do filho}
  \end{phonetics}
\end{entry}

\begin{entry}{入乡随俗}{2,3,11,9}{⼊、⼄、⾩、⼈}
  \begin{phonetics}{入乡随俗}{ru4xiang1-sui2su2}
    \definition{expr.}{Em roma, faça como os romanos!}
  \end{phonetics}
\end{entry}

\begin{entry}{入口}{2,3}{⼊、⼝}
  \begin{phonetics}{入口}{ru4kou3}[][HSK 2]
    \definition[个]{s.}{entrada | enseada}
    \definition{v.}{entra na boca | importar}
  \end{phonetics}
\end{entry}

\begin{entry}{入门}{2,3}{⼊、⾨}
  \begin{phonetics}{入门}{ru4men2}
    \definition{s.}{curso elementar | ABC | guia}
    \definition{v.+compl.}{atravessar o limiar | aprender o ABC de | introduzir um assunto | aprender os rudimentos de um assunto}
  \end{phonetics}
\end{entry}

\begin{entry}{入党}{2,10}{⼊、⼉}
  \begin{phonetics}{入党}{ru4dang3}
    \definition{v.}{ingressar em um partido político (especialmente o partido comunista)}
  \end{phonetics}
\end{entry}

\begin{entry}{入境}{2,14}{⼊、⼟}
  \begin{phonetics}{入境}{ru4jing4}
    \definition{s.}{imigração}
    \definition{v.+compl.}{entrar em um país | imigrar}
  \end{phonetics}
\end{entry}

\begin{entry}{八}{2}{⼋}[Kangxi 12]
  \begin{phonetics}{八}{ba1}[][HSK 1]
    \definition{num.}{oito; 8}
  \end{phonetics}
\end{entry}

\begin{entry}{八八六}{2,2,4}{⼋、⼋、⼋}
  \begin{phonetics}{八八六}{ba1 ba1 liu4}
    \definition{expr.}{\emph{Bye bye!} (em salas de bate-papo e mensagens de texto)}
  \end{phonetics}
\end{entry}

\begin{entry}{几}{2}{⼏}
  \begin{phonetics}{几}{ji1}
    \definition{adv.}{quase}
    \definition{s.}{mesa pequena}
  \end{phonetics}
  \begin{phonetics}{几}{ji3}[][HSK 1]
    \definition{adv.}{quantos?, (até 10 itens) | vários | alguns}
  \end{phonetics}
\end{entry}

\begin{entry}{几乎}{2,5}{⼏、⼃}
  \begin{phonetics}{几乎}{ji1hu1}[][HSK 4]
    \definition{adv.}{quase; praticamente; próximo a | perto de; quase; à beira de}
  \end{phonetics}
\end{entry}

\begin{entry}{几何}{2,7}{⼏、⼈}
  \begin{phonetics}{几何}{ji3he2}
    \definition{s.}{geometria}
  \end{phonetics}
\end{entry}

\begin{entry}{刀}{2}{⼑}[Kangxi 18]
  \begin{phonetics}{刀}{dao1}[][HSK 3]
    \definition*{s.}{sobrenome Dao}
    \definition{clas.}{para cortes de faca ou facadas | para cem folhas (de papel)}
    \definition[把]{s.}{faca; espada | algo em forma de faca | moeda antiga em forma de faca}
  \end{phonetics}
\end{entry}

\begin{entry}{力}{2}{⼒}[Kangxi 19]
  \begin{phonetics}{力}{li4}[][HSK 3]
    \definition*{s.}{sobrenome Li}
    \definition{adv.}{energicamente; arduamente; vigorosamente}
    \definition{s.}{poder; força; habilidade; capacidade | força; energia; poder | força física}
    \definition{v.}{fazer tudo o que puder; fazer todo o esforço}
  \end{phonetics}
\end{entry}

\begin{entry}{力气}{2,4}{⼒、⽓}
  \begin{phonetics}{力气}{li4qi5}[][HSK 4]
    \definition[点,把]{s.}{força física | esforço}
  \end{phonetics}
\end{entry}

\begin{entry}{力量}{2,12}{⼒、⾥}
  \begin{phonetics}{力量}{li4liang5}[][HSK 3]
    \definition[出]{s.}{força física; força espiritual | habilidade; capacidade | eficácia; efeito | força (pessoa ou grupo que tem muito poder ou influência)}
  \end{phonetics}
\end{entry}

\begin{entry}{十}{2}{⼗}[Kangxi 24]
  \begin{phonetics}{十}{shi2}[][HSK 1]
    \definition{num.}{dez; 10 | dezena}
  \end{phonetics}
\end{entry}

\begin{entry}{十分}{2,4}{⼗、⼑}
  \begin{phonetics}{十分}{shi2fen1}[][HSK 2]
    \definition{adv.}{muito | extremamente | totalmente | absolutamente}
  \end{phonetics}
\end{entry}

\begin{entry}{十足}{2,7}{⼗、⾜}
  \begin{phonetics}{十足}{shi2zu2}
    \definition{adj.}{amplo | completo | cento por cento | tom puro (de alguma cor)}
  \end{phonetics}
\end{entry}

\begin{entry}{厂}{2}{⼚}[Kangxi 27]
  \begin{phonetics}{厂}{chang3}[][HSK 3]
    \definition[家]{s.}{fábrica; moinho; planta; obra | pátio; depósito}
  \end{phonetics}
  \begin{phonetics}{厂}{han3}
    \definition{s.}{radical ``penhasco'' em caracteres chineses (radical Kangxi 27)}
  \end{phonetics}
\end{entry}

\begin{entry}{又}{2}{⼜}[Kangxi 29]
  \begin{phonetics}{又}{you4}[][HSK 2]
    \definition{adv.}{mais uma vez | (usado para dar ênfase) de qualquer maneira | e ainda | e também}
  \end{phonetics}
\end{entry}

\begin{entry}{又一次}{2,1,6}{⼜、⼀、⽋}
  \begin{phonetics}{又一次}{you4yi2ci4}
    \definition{adv.}{outra vez | mais uma vez | de novo}
  \end{phonetics}
\end{entry}

\begin{entry}{又及}{2,3}{⼜、⼃}
  \begin{phonetics}{又及}{you4ji2}
    \definition{s.}{P.S., \emph{postscript}}
  \end{phonetics}
\end{entry}

\begin{entry}{又名}{2,6}{⼜、⼝}
  \begin{phonetics}{又名}{you4ming2}
    \definition{s.}{também conhecido como | nome alternativo}
  \end{phonetics}
\end{entry}

\begin{entry}{又称}{2,10}{⼜、⽲}
  \begin{phonetics}{又称}{you4cheng1}
    \definition{s.}{também conhecido como}
  \end{phonetics}
\end{entry}

%%%%% EOF %%%%%

