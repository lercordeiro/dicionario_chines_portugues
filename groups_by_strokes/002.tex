%%%
%%% 2画
%%%

\section*{2画}\addcontentsline{toc}{section}{2画}

\begin{Entry}{七}{2}{⼀}
  \begin{Phonetics}{七}{qi1}[][HSK 1]
    \definition*{s.}{Sobrenome Qi}
    \definition{num.}{sete; 7}
    \definition{s.}{antigamente, os mortos eram homenageados a cada sete dias, chamados de 七, até o quadragésimo nono dia, num total de sete 七}
  \end{Phonetics}
\end{Entry}

\begin{Entry}{七夕}{2,3}{⼀、⼣}
  \begin{Phonetics}{七夕}{qi1xi1}
    \definition*{s.}{Dia dos Namorados Chinês, quando o vaqueiro e a tecelã (牛郎织女) têm permissão para se reunirem anualmente | Festival das Meninas | Festival Duplo Sete, noite do sétimo mês lunar}
  \seealsoref{牛郎织女}{niu2 lang2 zhi1nv3}
  \end{Phonetics}
\end{Entry}

\begin{Entry}{九}{2}{⼄}
  \begin{Phonetics}{九}{jiu3}[][HSK 1]
    \definition*{s.}{Sobrenome Jiu}
    \definition{adj.}{muitos; numerosos; indica várias vezes ou a maioria das vezes}
    \definition{num.}{nove; 9}
    \definition{s.}{cada um dos nove períodos de nove dias começando no dia seguinte ao solstício de inverno}
  \end{Phonetics}
\end{Entry}

\begin{Entry}{了}{2}{⼅}
  \begin{Phonetics}{了}{le5}[][HSK 1,3]
    \definition{part.}{usada após verbos ou adjetivos para indicar a conclusão de uma ação, em um momento no passado ou antes do início de outra ação, ou uma ação esperada ou presumida | usada para indicar uma mudança de situação ou estado, seja real ou prevista | comandos ou solicitações em resposta a uma situação alterada; usada para xpressar urgência ou dissuadir | usada para indicar que algo chegou ao extremo; usada no final da frase ou em pausas no meio da frase, para expressar um tom de exclamação}
  \end{Phonetics}
  \begin{Phonetics}{了}{liao3}
    \definition*{s.}{Sobrenome Liao}
    \definition{adv.}{inteiramente; um pouco; totalmente (mais usado em negativas)}
    \definition{v.}{terminar; concluir; encerrar; cumprir; eliminar; resolver | compreender; saber; perceber; saber claramente | expressar possibilidade ou impossibilidade; usado com 得 ou 不 após o verbo, indica possibilidade ou impossibilidade}
  \seealsoref{不}{bu4}
  \seealsoref{得}{de5}
  \end{Phonetics}
\end{Entry}

\begin{Entry}{了不起}{2,4,10}{⼅、⼀、⾛}
  \begin{Phonetics}{了不起}{liao3bu5qi3}[][HSK 4]
    \definition{adj.}{incrível; fantástico; extraordinário | sério; grave}
  \end{Phonetics}
\end{Entry}

\begin{Entry}{了解}{2,13}{⼅、⾓}
  \begin{Phonetics}{了解}{liao3jie3}[][HSK 4]
    \definition{v.}{entender; compreender | investigar; indagar sobre}
  \end{Phonetics}
\end{Entry}

\begin{Entry}{二}{2}{⼆}[Kangxi 7]
  \begin{Phonetics}{二}{er4}[][HSK 1]
    \definition{adj.}{diferente; refere-se a duas coisas ou coisas diferentes | bobo; pateta; tolo; sem inteligência | desleal; infiel; indiferente; sem determinação}
    \definition{num.}{dois; 2}
  \end{Phonetics}
\end{Entry}

\begin{Entry}{二手}{2,4}{⼆、⼿}
  \begin{Phonetics}{二手}{er4 shou3}[][HSK 4]
    \definition{adj.}{usado; de segunda mão; refere-se especificamente a usados e revendidos}
  \end{Phonetics}
\end{Entry}

\begin{Entry}{二战}{2,9}{⼆、⼽}
  \begin{Phonetics}{二战}{er4zhan4}
    \definition*{s.}{Segunda Guerra Mundial}
  \end{Phonetics}
\end{Entry}

\begin{Entry}{二胡}{2,9}{⼆、⾁}
  \begin{Phonetics}{二胡}{er4hu2}
    \definition{s.}{erhu; um instrumento de arco de duas cordas com um registro mais baixo que o 京胡; um tipo de 胡琴, a caixa de som é feita de bambu, madeira, etc., coberta com pele de cobra, etc., tem duas cordas e o tom é baixo e suave}
  \seealsoref{胡琴}{hu2qin2}
  \seealsoref{京胡}{jing1hu2}
  \end{Phonetics}
\end{Entry}

\begin{Entry}{二维码}{2,11,8}{⼆、⽷、⽯}
  \begin{Phonetics}{二维码}{er4 wei2 ma3}[][HSK 5]
    \definition[个]{s.}{\emph{QR code}; um identificador gráfico que distribui formas geométricas específicas em um plano ou direção bidimensional de acordo com certas regras para expressar um conjunto de informações}
  \end{Phonetics}
\end{Entry}

\begin{Entry}{人}{2}{⼈}[Kangxi 9]
  \begin{Phonetics}{人}{ren2}[][HSK 1]
    \definition*{s.}{Sobrenome Ren}
    \definition[个,名,位]{s.}{homem; pessoa; pessoas; ser humano | todos; cada um; todo mundo | adulto; crescido | uma pessoa envolvida em uma atividade específica | pessoas; outras pessoas | caráter; personalidade; qualidade, caráter ou reputação de uma pessoa | como alguém se sente; estado de saúde de alguém | mão de obra; força de trabalho}
  \end{Phonetics}
\end{Entry}

\begin{Entry}{人力}{2,2}{⼈、⼒}
  \begin{Phonetics}{人力}{ren2 li4}[][HSK 5]
    \definition{s.}{mão de obra; trabalho manual; força de trabalho}
  \end{Phonetics}
\end{Entry}

\begin{Entry}{人力车}{2,2,4}{⼈、⼒、⾞}
  \begin{Phonetics}{人力车}{ren2 li4 che1}
    \definition{s.}{veículo de duas rodas puxado ou empurrado por um homem (oposto a 兽力车 e 机动车) | Datado: riquixá | uma carroça puxada ou empurrada por humanos}
  \seealsoref{机动车}{ji1 dong4 che1}
  \seealsoref{兽力车}{shou4 li4 che1}
  \end{Phonetics}
\end{Entry}

\begin{Entry}{人口}{2,3}{⼈、⼝}
  \begin{Phonetics}{人口}{ren2kou3}[][HSK 2]
    \definition[个,群]{s.}{população; o número total de pessoas que vivem em uma determinada região durante um determinado período de tempo | número de membros da família; o número total de pessoas em uma família | pessoas; público; população; referência geral a pessoas | rumores do povo; referindo-se à opinião pública}
  \end{Phonetics}
\end{Entry}

\begin{Entry}{人士}{2,3}{⼈、⼠}
  \begin{Phonetics}{人士}{ren2shi4}[][HSK 5]
    \definition{s.}{pessoa; figura; personalidade; figura pública; pessoas com certa influência social}
  \end{Phonetics}
\end{Entry}

\begin{Entry}{人工}{2,3}{⼈、⼯}
  \begin{Phonetics}{人工}{ren2gong1}[][HSK 3]
    \definition{adj.}{feito pelo homem; artificial (oposto a 天然)}
    \definition[个]{s.}{trabalho manual; trabalho feito à mão | mão de obra; homem-dia; uma unidade de cálculo da quantidade de trabalho realizado}
  \seealsoref{天然}{tian1 ran2}
  \end{Phonetics}
\end{Entry}

\begin{Entry}{人工智能}{2,3,12,10}{⼈、⼯、⽇、⾁}
  \begin{Phonetics}{人工智能}{ren2 gong1 zhi4 neng2}
    \definition*{s.}{Inteligência Artificial (IA)}
  \end{Phonetics}
\end{Entry}

\begin{Entry}{人才}{2,3}{⼈、⼿}
  \begin{Phonetics}{人才}{ren2cai2}[][HSK 3]
    \definition{adj.}{aparência bonita, elegante}
    \definition[个,些,位]{s.}{talento; pessoal qualificado; pessoa com capacidade; uma pessoa com capacidade e integridade política; uma pessoa com talentos especiais | aparência bonita; refere-se à aparência; especialmente à aparência bonita}
  \end{Phonetics}
\end{Entry}

\begin{Entry}{人们}{2,5}{⼈、⼈}
  \begin{Phonetics}{人们}{ren2 men5}[][HSK 2]
    \definition{s.}{homens; pessoas; o público; referindo-se a muitas pessoas; todos}
  \end{Phonetics}
\end{Entry}

\begin{Entry}{人民}{2,5}{⼈、⽒}
  \begin{Phonetics}{人民}{ren2 min2}[][HSK 3]
    \definition[群,批,个,国]{s.}{o povo; refere-se a um certo tipo de pessoas; membros básicos da sociedade com as massas trabalhadoras como o corpo principal}
  \end{Phonetics}
\end{Entry}

\begin{Entry}{人民币}{2,5,4}{⼈、⽒、⼱}
  \begin{Phonetics}{人民币}{ren2min2bi4}[][HSK 3]
    \definition*[块,张,元]{s.}{Renminbi (RMB); Yuan Chinês (CYN); nome da moeda chinesa}
  \end{Phonetics}
\end{Entry}

\begin{Entry}{人生}{2,5}{⼈、⽣}
  \begin{Phonetics}{人生}{ren2sheng1}[][HSK 3]
    \definition{s.}{vida; sobrevivência e vida humana}
  \end{Phonetics}
\end{Entry}

\begin{Entry}{人权}{2,6}{⼈、⽊}
  \begin{Phonetics}{人权}{ren2quan2}[][HSK 6]
    \definition{s.}{direitos humanos}[最基本的人权是生存权。===O direito humano mais básico é o direito à vida.]
  \seealsoref{人权法}{ren2quan2fa3}
  \end{Phonetics}
\end{Entry}

\begin{Entry}{人权法}{2,6,8}{⼈、⽊、⽔}
  \begin{Phonetics}{人权法}{ren2quan2fa3}
    \definition*{s.}{Direitos Humanos}
  \seealsoref{人权}{ren2quan2}
  \end{Phonetics}
\end{Entry}

\begin{Entry}{人行道}{2,6,12}{⼈、⾏、⾡}
  \begin{Phonetics}{人行道}{ren2xing2dao4}
    \definition{s.}{calçada}
  \end{Phonetics}
\end{Entry}

\begin{Entry}{人员}{2,7}{⼈、⼝}
  \begin{Phonetics}{人员}{ren2yuan2}[][HSK 3]
    \definition[个,位,名]{s.}{funcionários ; uma pessoa que ocupa uma determinada posição| pessoal; membros de um grupo}
  \end{Phonetics}
\end{Entry}

\begin{Entry}{人材}{2,7}{⼈、⽊}
  \begin{Phonetics}{人材}{ren2cai2}
    \variantof{人才}
  \end{Phonetics}
\end{Entry}

\begin{Entry}{人间}{2,7}{⼈、⾨}
  \begin{Phonetics}{人间}{ren2jian1}[][HSK 5]
    \definition{s.}{o mundo humano; o Mundo; a Terra}
  \end{Phonetics}
\end{Entry}

\begin{Entry}{人物}{2,8}{⼈、⽜}
  \begin{Phonetics}{人物}{ren2wu4}[][HSK 5]
    \definition[个,位,名]{s.}{personagem; personagens criados em obras literárias e artísticas | figura; personalidade; homem influente; refere-se a pessoas com grande talento e status; também se refere a pessoas com certas características ou que são representativas em algum aspecto | pintura figurativa; um tipo de pintura tradicional chinesa com personagens como tema}
  \end{Phonetics}
\end{Entry}

\begin{Entry}{人鱼}{2,8}{⼈、⿂}
  \begin{Phonetics}{人鱼}{ren2yu2}
    \definition{s.}{sereia | peixe-boi | salamandra gigante}
  \end{Phonetics}
\end{Entry}

\begin{Entry}{人类}{2,9}{⼈、⽶}
  \begin{Phonetics}{人类}{ren2lei4}[][HSK 3]
    \definition[种]{s.}{humano; humanidade; raça humana; um termo geral para pessoas}
  \end{Phonetics}
\end{Entry}

\begin{Entry}{人家}{2,10}{⼈、⼧}
  \begin{Phonetics}{人家}{ren2jia1}[][HSK 4]
    \definition[户,个]{s.}{lar; família; família do noivo; casa do futuro marido}
  \end{Phonetics}
  \begin{Phonetics}{人家}{ren2jia5}
    \definition{pron.}{outros; uma pessoa ou pessoas diferentes do falante ou ouvinte; refere-se a alguém diferente de si mesmo ou de outra pessoa | certa pessoa ou pessoas (a pessoa ou pessoas mencionadas em um contexto próximo, aproximadamente equivalente ao pronome de terceira pessoa);  refere-se a uma pessoa ou algumas pessoas, com significado semelhante a 他 | eu; mim (usado retoricamente no lugar do primeiro pronome pessoal, muitas vezes expressando descontentamento de forma jocosa; geralmente usado quando se fala com pessoas próximas, para significar 自己, usado principamente por meninas)}
  \seealsoref{他}{ta1}
  \seealsoref{自己}{zi4ji3}
  \end{Phonetics}
\end{Entry}

\begin{Entry}{人海}{2,10}{⼈、⽔}
  \begin{Phonetics}{人海}{ren2hai3}
    \definition{s.}{uma multidão | um mar de pessoas}
  \end{Phonetics}
\end{Entry}

\begin{Entry}{人道}{2,12}{⼈、⾡}
  \begin{Phonetics}{人道}{ren2dao4}
    \definition{s.}{solidariedade humana | humanitarismo | humano | a ``maneira humana'', um dos estágios do ciclo de reencarnação (budismo) | relação sexual}
  \end{Phonetics}
\end{Entry}

\begin{Entry}{人像}{2,13}{⼈、⼈}
  \begin{Phonetics}{人像}{ren2xiang4}
    \definition{s.}{``retrato'' de uma pessoa (esboço, foto, escultura, etc.)}
  \end{Phonetics}
\end{Entry}

\begin{Entry}{人数}{2,13}{⼈、⽁}
  \begin{Phonetics}{人数}{ren2 shu4}[][HSK 2]
    \definition{s.}{número de pessoas; significa o número total de pessoas, uma quantidade de pessoas; normalmente, usa-se números para fazer estatísticas específicas, mas às vezes também se usa um intervalo aproximado para fazer estimativas}
  \end{Phonetics}
\end{Entry}

\begin{Entry}{人群}{2,13}{⼈、⽺}
  \begin{Phonetics}{人群}{ren2 qun2}[][HSK 3]
    \definition[个,类]{s.}{multidão; ajuntamento; torpel; aglomeração; um grupo de pessoas}
  \end{Phonetics}
\end{Entry}

\begin{Entry}{儿}{2}{⼉}[Kangxi 10]
  \begin{Phonetics}{儿}{er2}
    \definition{adj.}{macho}
    \definition{s.}{criança | jovem; juventude | filho}
    \definition{suf.}{adicionado a substantivos para expressar pequenez  | adicionado a verbos, adjetivos e classificadores para formar substantivos | adicionado a substantivos para formar substantivos com significados diferentes | sufixos de alguns verbos | anexado após adjetivos duplicados}
  \end{Phonetics}
  \begin{Phonetics}{儿}{r5}
    \definition{suf.}{sufixo diminutivo não silábico | final retroflexo, pronunciado como ``r'' | adicionado a substantivos para expressar pequenez  | adicionado a verbos, adjetivos e classificadores para formar substantivos | adicionado a substantivos para formar substantivos com significados diferentes | sufixos de alguns verbos | anexado após adjetivos duplicados}
  \end{Phonetics}
\end{Entry}

\begin{Entry}{儿女}{2,3}{⼉、⼥}
  \begin{Phonetics}{儿女}{er2 nv3}[][HSK 5]
    \definition{s.}{crianças; filhos e filhas | homem e mulher jovens (apaixonados)}
  \end{Phonetics}
\end{Entry}

\begin{Entry}{儿子}{2,3}{⼉、⼦}
  \begin{Phonetics}{儿子}{er2zi5}[][HSK 1]
    \definition[个]{s.}{filho}
  \seealsoref{女儿}{nv3'er2}
  \end{Phonetics}
\end{Entry}

\begin{Entry}{儿科}{2,9}{⼉、⽲}
  \begin{Phonetics}{儿科}{er2 ke1}[][HSK 6]
    \definition{s.}{(departamento de) pediatria | pediatria; o ramo da medicina que trata do desenvolvimento, cuidado e doença das crianças}
  \end{Phonetics}
\end{Entry}

\begin{Entry}{儿童}{2,12}{⼉、⽴}
  \begin{Phonetics}{儿童}{er2tong2}[][HSK 4]
    \definition[个,群]{s.}{criança; menor de idade (mais jovem do que 少年)}
  \seealsoref{少年}{shao4 nian2}
  \end{Phonetics}
\end{Entry}

\begin{Entry}{儿媳}{2,13}{⼉、⼥}
  \begin{Phonetics}{儿媳}{er2xi2}
    \definition{s.}{esposa do filho}
  \end{Phonetics}
\end{Entry}

\begin{Entry}{入}{2}{⼊}[Kangxi 11]
  \begin{Phonetics}{入}{ru4}[][HSK 6]
    \definition{s.}{renda | tom de entrada}
    \definition{v.}{entrar; entrar (oposto a 出) | juntar-se; ser admitido em; tornar-se membro de | conformar-se com; concordar com | alcançar; atingir; entrar em (um certo nível ou estado) | fazer entrar; fazer algo entrar; fazer entrada}
  \seealsoref{出}{chu1}
  \end{Phonetics}
\end{Entry}

\begin{Entry}{入乡随俗}{2,3,11,9}{⼊、⼄、⾩、⼈}
  \begin{Phonetics}{入乡随俗}{ru4xiang1-sui2su2}
    \definition{expr.}{Em roma, faça como os romanos!}
  \end{Phonetics}
\end{Entry}

\begin{Entry}{入口}{2,3}{⼊、⼝}
  \begin{Phonetics}{入口}{ru4kou3}[][HSK 2]
    \definition[个]{s.}{entrada; entrada em locais, edifícios, estradas, etc., através de portões ou portas}
    \definition{v.}{entrar na boca | importar; mercadorias estrangeiras importadas, às vezes também se refere a mercadorias de outras regiões importadas para esta região}
  \end{Phonetics}
\end{Entry}

\begin{Entry}{入门}{2,3}{⼊、⾨}
  \begin{Phonetics}{入门}{ru4/men2}[][HSK 5]
    \definition{s.}{(geralmente em títulos de livros) curso básico; manual introdutório | ABC; guia; refere-se a leituras básicas; conhecimentos básicos}
    \definition{v.+compl.}{ultrapassar o limiar; aprender os rudimentos de um assunto | aprender o ABC de; ser introduzido a um assunto; aprender o básico}
  \end{Phonetics}
\end{Entry}

\begin{Entry}{入学}{2,8}{⼊、⼦}
  \begin{Phonetics}{入学}{ru4/xue2}[][HSK 6]
    \definition{v.+compl.}{(de uma criança) começar a escola; começar a escola primária | entrar em uma escola; matricular-se em uma escola}
  \end{Phonetics}
\end{Entry}

\begin{Entry}{入党}{2,10}{⼊、⼉}
  \begin{Phonetics}{入党}{ru4dang3}
    \definition{v.}{ingressar em um partido político (especialmente o partido comunista)}
  \end{Phonetics}
\end{Entry}

\begin{Entry}{入境}{2,14}{⼊、⼟}
  \begin{Phonetics}{入境}{ru4/jing4}
    \definition{s.}{imigração}
    \definition{v.+compl.}{entrar em um país | imigrar}
  \end{Phonetics}
\end{Entry}

\begin{Entry}{八}{2}{⼋}[Kangxi 12]
  \begin{Phonetics}{八}{ba1}[][HSK 1]
    \definition{num.}{oito; 8}
  \end{Phonetics}
\end{Entry}

\begin{Entry}{八八六}{2,2,4}{⼋、⼋、⼋}
  \begin{Phonetics}{八八六}{ba1 ba1 liu4}
    \definition{expr.}{\emph{Bye bye!}, em salas de bate-papo e mensagens de texto}
  \end{Phonetics}
\end{Entry}

\begin{Entry}{八卦}{2,8}{⼋、⼘}
  \begin{Phonetics}{八卦}{ba1gua4}[][HSK 7-9]
    \definition*{s.}{Oito Trigramas; na China antiga, havia um conjunto de símbolos com significados simbólicos. "—" representa Yang, "--" representa Yin, e oito grupos desses símbolos são chamados de Bagua; cada hexagrama representa uma coisa específica e, mais tarde, foi usado para prever sucesso ou fracasso, sorte ou infortúnio, etc.}
    \definition[个,条]{adj.}{fofoca}
    \definition{adj.}{fofoqueiro}
  \end{Phonetics}
\end{Entry}

\begin{Entry}{几}{2}{⼏}[Kangxi 16]
  \begin{Phonetics}{几}{ji1}
    \definition{adv.}{quase; praticamente}
    \definition{s.}{uma mesa pequena}
  \end{Phonetics}
  \begin{Phonetics}{几}{ji3}[][HSK 1]
    \definition{adv.}{quanto?, usado para perguntar sobre quantidade e tempo}
    \definition{num.}{alguns; vários; poucos; indica um número indeterminado maior que um e menor que dez}
  \end{Phonetics}
\end{Entry}

\begin{Entry}{几乎}{2,5}{⼏、⼃}
  \begin{Phonetics}{几乎}{ji1hu1}[][HSK 4]
    \definition{adv.}{quase; praticamente; próximo a | perto de; quase; à beira de}
  \end{Phonetics}
\end{Entry}

\begin{Entry}{几何}{2,7}{⼏、⼈}
  \begin{Phonetics}{几何}{ji3he2}
    \definition{s.}{geometria}
  \end{Phonetics}
\end{Entry}

\begin{Entry}{几率}{2,11}{⼏、⽞}
  \begin{Phonetics}{几率}{ji1lv4}
    \definition{s.}{probabilidade; um evento pode ou não ocorrer sob as mesmas condições, a grandeza que indica a possibilidade de ocorrência é chamada de probabilidade}
  \end{Phonetics}
\end{Entry}

\begin{Entry}{刀}{2}{⼑}[Kangxi 18]
  \begin{Phonetics}{刀}{dao1}[][HSK 3]
    \definition*{s.}{Sobrenome Dao}
    \definition{clas.}{unidade de medida para papel, geralmente cem folhas por pacote}
    \definition[把,口]{s.}{faca; espada; armas antigas, referindo-se a ferramentas para cortar, retalhar, raspar, golpear e fatiar, geralmente feitas de ferro e aço | ferramenta; ferramenta de corte; lâminas para tornos; fresas (ferramentas; ferramentas de ferro para máquinas) | algo com a forma de uma faca}
  \end{Phonetics}
\end{Entry}

\begin{Entry}{力}{2}{⼒}[Kangxi 19]
  \begin{Phonetics}{力}{li4}[][HSK 3]
    \definition*{s.}{Sobrenome Li}
    \definition{adj.}{forte; eficiente; capaz | forte; poderoso; referência geral à função das coisas}
    \definition{adv.}{energicamente; arduamente; vigorosamente; com todo o esforço; com toda a dedicação}
    \definition{s.}{força; energia; poder; (física) refere-se à ação de alterar o estado de movimento ou a forma de um objeto |poder; força; habilidade; capacidade; funções dos órgãos do corpo humano | força física; resistência física}
  \end{Phonetics}
\end{Entry}

\begin{Entry}{力气}{2,4}{⼒、⽓}
  \begin{Phonetics}{力气}{li4qi5}[][HSK 4]
    \definition[把]{s.}{força física; eficiência muscular; força | esforço}
  \end{Phonetics}
\end{Entry}

\begin{Entry}{力量}{2,12}{⼒、⾥}
  \begin{Phonetics}{力量}{li4liang5}[][HSK 3]
    \definition[出]{s.}{força física; força espiritual | habilidade; capacidade | eficácia; efeito | força (pessoa ou grupo que tem muito poder ou influência); referência a uma pessoa ou grupo que pode desempenhar um papel importante}
  \end{Phonetics}
\end{Entry}

\begin{Entry}{十}{2}{⼗}[Kangxi 24]
  \begin{Phonetics}{十}{shi2}[][HSK 1]
    \definition*{s.}{Sobrenome Shi}
    \definition{num.}{dez; 10 | dezena | completo; no topo; máximo; referindo-se a algo que atingiu o ápice da perfeição ou plenitude | um monte de; indica que há muitos}
  \end{Phonetics}
\end{Entry}

\begin{Entry}{十分}{2,4}{⼗、⼑}
  \begin{Phonetics}{十分}{shi2fen1}[][HSK 2]
    \definition{adv.}{muito; totalmente; completamente; extremamente; indica um nível muito alto}
  \end{Phonetics}
\end{Entry}

\begin{Entry}{十足}{2,7}{⼗、⾜}
  \begin{Phonetics}{十足}{shi2zu2}[][HSK 5]
    \definition{adj.}{puro e simples; apenas este componente ou esta característica é muito evidente | 100\%; completo; total; muito satisfatório; muito adequado}
  \end{Phonetics}
\end{Entry}

\begin{Entry}{厂}{2}{⼚}[Kangxi 27]
  \begin{Phonetics}{厂}{an1}
    \definition{s.}{usado principalmente em nomes pessoais}[他名中有个厂字。===O nome dele contém a palavra `An'.]
  \end{Phonetics}
  \begin{Phonetics}{厂}{chang3}[][HSK 3]
    \definition[家,间]{s.}{fábrica; moinho; planta; obra | pátio; depósito; refere-se a um estabelecimento comercial com um amplo espaço para armazenamento de mercadorias e processamento}
  \end{Phonetics}
  \begin{Phonetics}{厂}{han3}
    \definition[家,间]{s.}{radical ``penhasco'' em caracteres chineses (radical Kangxi 27)}
  \end{Phonetics}
\end{Entry}

\begin{Entry}{厂长}{2,4}{⼚、⾧}
  \begin{Phonetics}{厂长}{chang3 zhang3}[][HSK 5]
    \definition[位,个,名]{s.}{diretor de fábrica; gerente de fábrica; líder responsável pela produção, pela vida e por todos os outros assuntos de toda a fábrica}
  \end{Phonetics}
\end{Entry}

\begin{Entry}{厂家}{2,10}{⼚、⼧}
  \begin{Phonetics}{厂家}{chang3jia1}[][HSK 7-9]
    \definition[个]{s.}{fábrica; refere-se aos aspectos de fábrica ou fábrica}
  \end{Phonetics}
\end{Entry}

\begin{Entry}{厂商}{2,11}{⼚、⼝}
  \begin{Phonetics}{厂商}{chang3 shang1}[][HSK 6]
    \definition[家,个]{s.}{empresa; fornecedor; fábrica; negócio; fabricante; uma unidade que produz e vende produtos; uma pessoa que administra uma fábrica}
  \end{Phonetics}
\end{Entry}

\begin{Entry}{又}{2}{⼜}[Kangxi 29]
  \begin{Phonetics}{又}{you4}[][HSK 2]
    \definition{adv.}{indica repetição ou continuação | indica que várias situações ou propriedades existem simultaneamente | indica um nível mais profundo de significado | indica adicionar zero a números inteiros | indica duas coisas contraditórias | indica um ponto de virada, significando 可是 | usado em frases negativas ou perguntas retóricas para fortalecer o tom | além disso; indica informações adicionais ou suplementares}
  \seealsoref{可是}{ke3shi4}
  \end{Phonetics}
\end{Entry}

\begin{Entry}{又一次}{2,1,6}{⼜、⼀、⽋}
  \begin{Phonetics}{又一次}{you4yi2ci4}
    \definition{adv.}{outra vez | mais uma vez | de novo}
  \end{Phonetics}
\end{Entry}

\begin{Entry}{又……又……}{2,2}{⼜、⼜}
  \begin{Phonetics}{又……又……}{you4 you4}
    \definition{conj.}{\dots e\dots; tanto\dots como\dots; as duas palavras usadas depois de 又 não devem ter nenhuma conotação de contraste, ambas devem ser positivas ou negativas}[这件毛衣挺不错的,\underline{又}便宜\underline{又}漂亮。===Este suéter é muito bom, barato e bonito.]
  \end{Phonetics}
\end{Entry}

\begin{Entry}{又及}{2,3}{⼜、⼃}
  \begin{Phonetics}{又及}{you4ji2}
    \definition{s.}{P.S., \emph{postscript}}
  \end{Phonetics}
\end{Entry}

\begin{Entry}{又名}{2,6}{⼜、⼝}
  \begin{Phonetics}{又名}{you4ming2}
    \definition{s.}{também conhecido como | nome alternativo}
  \end{Phonetics}
\end{Entry}

\begin{Entry}{又称}{2,10}{⼜、⽲}
  \begin{Phonetics}{又称}{you4cheng1}
    \definition{s.}{também conhecido como}
  \end{Phonetics}
\end{Entry}

%%%%% EOF %%%%%

