%%%
%%% 2画
%%%

\section*{2画}\addcontentsline{toc}{section}{2画}

\begin{entry}{七}{2}[Radical 一]
  \begin{phonetics}{七}{qi1}
    \definition{num.}{sete; 7}
  \end{phonetics}
\end{entry}

\begin{entry}{七夕}{2,3}
  \begin{phonetics}{七夕}{qi1xi1}
    \definition*{s.}{Dia dos Namorados Chinês, quando o vaqueiro e a tecelã (牛郎织女) têm permissão para se reunirem anualmente | Festival das Meninas | Festival Duplo Sete, noite do sétimo mês lunar}
    \seeref{牛郎织女}{niu2lang2zhi1nv3}
  \end{phonetics}
\end{entry}

\begin{entry}{九}{2}[Radical 乙]
  \begin{phonetics}{九}{jiu3}
    \definition{num.}{nove; 9}
  \end{phonetics}
\end{entry}

\begin{entry}{了}{2}[Radical 亅]
  \begin{phonetics}{了}{le5}
    \definition{part.}{marcador de ação concluída | partícula modal indicando mudança de estado, situação | partícula modal intensificando a cláusula anterior}
  \end{phonetics}
  \begin{phonetics}{了}{liao3}
    \definition{v.}{terminar | alcançar | entender claramente}
  \end{phonetics}
  \begin{phonetics}{了}{liao4}
    \definition{adj.}{brilhantes (olhos)}
    \definition{v.}{observar | olhar para fora | olhar para baixo de um lugar mais alto | compreender claramente}
  \end{phonetics}
\end{entry}

\begin{entry}{了解}{2,13}
  \begin{phonetics}{了解}{liao3jie3}
    \definition{v.}{entender | perceber | descobrir}
  \end{phonetics}
\end{entry}

\begin{entry}{二}{2}[Radical ⼆][Kangxi 7]
  \begin{phonetics}{二}{er4}
    \definition{num.}{dois; 2 | (dialeto de Pequim) estúpido}
  \end{phonetics}
\end{entry}

\begin{entry}{二战}{2,9}
  \begin{phonetics}{二战}{er4zhan4}
    \definition*{s.}{Segunda Guerra Mundial}
  \end{phonetics}
\end{entry}

\begin{entry}{人}{2}[Radical 人][Kangxi 9]
  \begin{phonetics}{人}{ren2}
    \definition[个,位]{s.}{pessoa | gente}
  \end{phonetics}
\end{entry}

\begin{entry}{人口}{2,3}
  \begin{phonetics}{人口}{ren2kou3}
    \definition{s.}{pessoas | população}
  \end{phonetics}
\end{entry}

\begin{entry}{人才}{2,3}
  \begin{phonetics}{人才}{ren2cai2}
    \definition{s.}{talento | pessoa talentosa}
  \end{phonetics}
\end{entry}

\begin{entry}{人民}{2,5}
  \begin{phonetics}{人民}{ren2min2}
    \definition[个]{s.}{povo | população}
  \end{phonetics}
\end{entry}

\begin{entry}{人民币}{2,5,4}
  \begin{phonetics}{人民币}{ren2min2bi4}
    \definition*{s.}{Renminbi (RMB) | Yuan Chinês (CYN) | nome da moeda chinesa}
  \end{phonetics}
\end{entry}

\begin{entry}{人生}{2,5}
  \begin{phonetics}{人生}{ren2sheng1}
    \definition{s.}{vida (tempo de alguém na Terra)}
  \end{phonetics}
\end{entry}

\begin{entry}{人权}{2,6}
  \begin{phonetics}{人权}{ren2quan2}
    \definition*{s.}{Direitos Humanos}
  \seealsoref{人权法}{ren2quan2fa3}
  \end{phonetics}
\end{entry}

\begin{entry}{人权法}{2,6,8}
  \begin{phonetics}{人权法}{ren2quan2fa3}
    \definition*{s.}{Direitos Humanos}
  \seealsoref{人权}{ren2quan2}
  \end{phonetics}
\end{entry}

\begin{entry}{人行道}{2,6,12}
  \begin{phonetics}{人行道}{ren2xing2dao4}
    \definition{s.}{calçada}
  \end{phonetics}
\end{entry}

\begin{entry}{人材}{2,7}
  \begin{phonetics}{人材}{ren2cai2}
    \variantof{人才}
  \end{phonetics}
\end{entry}

\begin{entry}{人间}{2,7}
  \begin{phonetics}{人间}{ren2jian1}
    \definition{s.}{o mundo humano | a Terra}
  \end{phonetics}
\end{entry}

\begin{entry}{人鱼}{2,8}
  \begin{phonetics}{人鱼}{ren2yu2}
    \definition{s.}{sereia | peixe-boi | salamandra gigante}
  \end{phonetics}
\end{entry}

\begin{entry}{人类}{2,9}
  \begin{phonetics}{人类}{ren2lei4}
    \definition{s.}{humanidade | raça humana}
  \end{phonetics}
\end{entry}

\begin{entry}{人海}{2,10}
  \begin{phonetics}{人海}{ren2hai3}
    \definition{s.}{uma multidão | um mar de pessoas}
  \end{phonetics}
\end{entry}

\begin{entry}{人道}{2,12}
  \begin{phonetics}{人道}{ren2dao4}
    \definition{s.}{solidariedade humana | humanitarismo | humano | a ``maneira humana'', um dos estágios do ciclo de reencarnação (budismo) | relação sexual}
  \end{phonetics}
\end{entry}

\begin{entry}{人像}{2,13}
  \begin{phonetics}{人像}{ren2xiang4}
    \definition{s.}{``retrato'' de uma pessoa (esboço, foto, escultura, etc.)}
  \end{phonetics}
\end{entry}

\begin{entry}{儿}{2}[Radical 儿]
  \begin{phonetics}{儿}{er2}
    \definition{s.}{criança | filho}
  \end{phonetics}
  \begin{phonetics}{儿}{r5}
    \definition{suf.}{sufixo diminutivo não silábico | final retroflexo}
  \end{phonetics}
  \begin{phonetics}{儿}{ren2}
    \definition{s.}{pessoa, radical em caracteres chineses}
    \variantof{人}
  \end{phonetics}
\end{entry}

\begin{entry}{儿子}{2,3}
  \begin{phonetics}{儿子}{er2zi5}
    \definition{s.}{filho}
  \seealsoref{女儿}{nv3'er2}
  \end{phonetics}
\end{entry}

\begin{entry}{儿媳}{2,13}
  \begin{phonetics}{儿媳}{er2xi2}
    \definition{s.}{esposa do filho}
  \end{phonetics}
\end{entry}

\begin{entry}{入乡随俗}{2,3,11,9}
  \begin{phonetics}{入乡随俗}{ru4xiang1-sui2su2}
    \definition{expr.}{Em roma, faça como os romanos!}
  \end{phonetics}
\end{entry}

\begin{entry}{入门}{2,3}
  \begin{phonetics}{入门}{ru4men2}
    \definition{s.}{curso elementar | ABC | guia}
    \definition{v.+compl.}{atravessar o limiar | aprender o ABC de | introduzir um assunto | aprender os rudimentos de um assunto}
  \end{phonetics}
\end{entry}

\begin{entry}{入党}{2,10}
  \begin{phonetics}{入党}{ru4dang3}
    \definition{v.}{ingressar em um partido político (especialmente o partido comunista)}
  \end{phonetics}
\end{entry}

\begin{entry}{入境}{2,14}
  \begin{phonetics}{入境}{ru4jing4}
    \definition{s.}{imigração}
    \definition{v.+compl.}{entrar em um país | imigrar}
  \end{phonetics}
\end{entry}

\begin{entry}{八}{2}[Radical 八][Kangxi 12]
  \begin{phonetics}{八}{ba1}
    \definition{num.}{oito; 8}
  \end{phonetics}
\end{entry}

\begin{entry}{八八六}{2,2,4}
  \begin{phonetics}{八八六}{ba1ba1liu4}
    \definition{expr.}{\emph{Bye bye!} (em salas de bate-papo e mensagens de texto)}
  \end{phonetics}
\end{entry}

\begin{entry}{几}{2}[Radical 几]
  \begin{phonetics}{几}{ji1}
    \definition{adv.}{quase}
    \definition{s.}{mesa pequena}
  \end{phonetics}
  \begin{phonetics}{几}{ji3}
    \definition{adv.}{quantos?, (até 10 itens) | vários | alguns}
  \end{phonetics}
\end{entry}

\begin{entry}{几乎}{2,5}
  \begin{phonetics}{几乎}{ji1hu1}
    \definition{adv.}{quase}
  \end{phonetics}
\end{entry}

\begin{entry}{几何}{2,7}
  \begin{phonetics}{几何}{ji3he2}
    \definition{s.}{geometria}
  \end{phonetics}
\end{entry}

\begin{entry}{刀}{2}[Radical ⼑][Kangxi 18]
  \begin{phonetics}{刀}{dao1}
    \definition*{s.}{sobrenome Dao}
    \definition{clas.}{para cortes de faca ou facadas}
    \definition[把]{s.}{faca | lâmina | espada de fio único | cutelo | (empréstimo linguístico) (gíria) dólar}
  \end{phonetics}
\end{entry}

\begin{entry}{力量}{2,12}
  \begin{phonetics}{力量}{li4liang5}
    \definition{s.}{poder | vigor | força}
  \end{phonetics}
\end{entry}

\begin{entry}{十}{2}[Radical 十][Kangxi 24]
  \begin{phonetics}{十}{shi2}
    \definition{num.}{dez; 10 | dezena}
  \end{phonetics}
\end{entry}

\begin{entry}{十分}{2,4}
  \begin{phonetics}{十分}{shi2fen1}
    \definition{adv.}{muito | extremamente | totalmente | absolutamente}
  \end{phonetics}
\end{entry}

\begin{entry}{十足}{2,7}
  \begin{phonetics}{十足}{shi2zu2}
    \definition{adj.}{amplo | completo | cento por cento | tom puro (de alguma cor)}
  \end{phonetics}
\end{entry}

\begin{entry}{又}{2}[Radical 又][Kangxi 29]
  \begin{phonetics}{又}{you4}
    \definition{adv.}{mais uma vez | (usado para dar ênfase) de qualquer maneira | e ainda | e também}
  \end{phonetics}
\end{entry}

\begin{entry}{又一次}{2,1,6}
  \begin{phonetics}{又一次}{you4yi2ci4}
    \definition{adv.}{outra vez | mais uma vez | de novo}
  \end{phonetics}
\end{entry}

\begin{entry}{又及}{2,3}
  \begin{phonetics}{又及}{you4ji2}
    \definition{s.}{P.S., \emph{postscript}}
  \end{phonetics}
\end{entry}

\begin{entry}{又名}{2,6}
  \begin{phonetics}{又名}{you4ming2}
    \definition{s.}{também conhecido como | nome alternativo}
  \end{phonetics}
\end{entry}

\begin{entry}{又称}{2,10}
  \begin{phonetics}{又称}{you4cheng1}
    \definition{s.}{também conhecido como}
  \end{phonetics}
\end{entry}

%%%%% EOF %%%%%

