%%%
%%% 1画
%%%

\section*{1画}\addcontentsline{toc}{section}{1画}

\begin{entry}{一}{1}{⼀}[Kangxi 1]
  \begin{phonetics}{一}{yi1}[(quando usado sozinho)][HSK 1]
    \definition{num.}{um; 1 | pronunciado como \dpy{yao1} quando dito número a número}
  \end{phonetics}
  \begin{phonetics}{一}{yi2}[(antes de quarto tom)][HSK 1]
    \definition{num.}{um; 1 | um (artigo)}
  \end{phonetics}
  \begin{phonetics}{一}{yi4}[][HSK 1]
    \definition{adv.}{uma vez | assim que | ao}
    \definition{num.}{um; 1 | um (artigo)}
  \end{phonetics}
\end{entry}

\begin{entry}{一下}{1,3}{⼀、⼀}
  \begin{phonetics}{一下}{yi2xia4}
    \definition{adv.}{em um curto tempo | rapidamente}
  \end{phonetics}
\end{entry}

\begin{entry}{一下儿}{1,3,2}{⼀、⼀、⼉}
  \begin{phonetics}{一下儿}{yi2xia4r5}[][HSK 1]
    \definition{adv.}{um pouco}
  \end{phonetics}
\end{entry}

\begin{entry}{一个样}{1,3,10}{⼀、⼈、⽊}
  \begin{phonetics}{一个样}{yi2ge5yang4}
    \definition{adj.}{igual | mesmo}
    \seeref{一样}{yi2yang4}
  \end{phonetics}
\end{entry}

\begin{entry}{一切}{1,4}{⼀、⼑}
  \begin{phonetics}{一切}{yi2qie4}[][HSK 3]
    \definition{pron.}{tudo; todo; todas as coisas}
  \end{phonetics}
\end{entry}

\begin{entry}{一方面}{1,4,9}{⼀、⽅、⾯}
  \begin{phonetics}{一方面}{yi4 fang1 mian4}[][HSK 3]
    \definition{s.}{um lado; um dos dois aspectos opostos ou um lado de algo que está relacionado a outro}
  \end{phonetics}
\end{entry}

\begin{entry}{一方面……,一方面……}{1,4,9,1,4,9}{⼀、⽅、⾯、⼀、⽅、⾯}
  \begin{phonetics}{一方面……,一方面……}{yi4 fang1 mian4 yi4 fang1 mian4}[][HSK 3]
    \definition{conj.}{por um lado\dots, por outro lado\dots; conecta duas orações paralelas (devem ser usadas juntas)}
  \end{phonetics}
\end{entry}

\begin{entry}{一半}{1,5}{⼀、⼗}
  \begin{phonetics}{一半}{yi2ban4}[][HSK 1]
    \definition{num.}{meio | metade}
  \end{phonetics}
\end{entry}

\begin{entry}{一生}{1,5}{⼀、⽣}
  \begin{phonetics}{一生}{yi4 sheng1}[][HSK 2]
    \definition{s.}{toda a vida | ao longo da vida | a vida de alguém}
  \end{phonetics}
\end{entry}

\begin{entry}{一边}{1,5}{⼀、⾡}
  \begin{phonetics}{一边}{yi4bian1}[][HSK 1]
    \definition{adj.}{mesmo | igual}
    \definition{adv.}{enquanto | como | ao mesmo tempo | simultaneamente}
    \definition{s.}{lado | um lado | cada lado | ao lado de}
  \end{phonetics}
\end{entry}

\begin{entry}{一会儿}{1,6,2}{⼀、⼈、⼉}
  \begin{phonetics}{一会儿}{yi2 hui4r5}[][HSK 1,2]
    \definition{adv.}{daqui a pouco tempo | pouco tempo}
  \end{phonetics}
\end{entry}

\begin{entry}{一共}{1,6}{⼀、⼋}
  \begin{phonetics}{一共}{yi2gong4}[][HSK 2]
    \definition{adv.}{completamente | no total | no todo | em suma}
  \end{phonetics}
\end{entry}

\begin{entry}{一再}{1,6}{⼀、⼌}
  \begin{phonetics}{一再}{yi2zai4}[][HSK 4]
    \definition{adv.}{repetidamente; de novo e de novo; repetidas vezes}
  \end{phonetics}
\end{entry}

\begin{entry}{一同}{1,6}{⼀、⼝}
  \begin{phonetics}{一同}{yi4tong2}
    \definition{adv.}{juntos, ao mesmo tempo}
  \end{phonetics}
\end{entry}

\begin{entry}{一行}{1,6}{⼀、⾏}
  \begin{phonetics}{一行}{yi1xing2}
    \definition{s.}{festa | delegação}
  \end{phonetics}
\end{entry}

\begin{entry}{一齐}{1,6}{⼀、⿑}
  \begin{phonetics}{一齐}{yi4qi2}
    \definition{adv.}{tudo ao mesmo tempo | em uníssono | junto}
  \end{phonetics}
\end{entry}

\begin{entry}{一块}{1,7}{⼀、⼟}
  \begin{phonetics}{一块}{yi2kuai4}
    \definition{adv.}{(principalmente mandarim) juntos}
  \end{phonetics}
\end{entry}

\begin{entry}{一块儿}{1,7,2}{⼀、⼟、⼉}
  \begin{phonetics}{一块儿}{yi2kuai4r5}[][HSK 1]
    \definition{adv.}{juntos}
  \end{phonetics}
\end{entry}

\begin{entry}{一时}{1,7}{⼀、⽇}
  \begin{phonetics}{一时}{yi4shi2}
    \definition{adv.}{por pouco tempo | por um tempo | temporariamente | momentaneamente | uma vez | de tempos em tempos | ocasionalmente}
  \end{phonetics}
\end{entry}

\begin{entry}{一些}{1,8}{⼀、⼆}
  \begin{phonetics}{一些}{yi4xie1}[][HSK 1]
    \definition{pron.}{uns | alguns}
  \end{phonetics}
\end{entry}

\begin{entry}{一定}{1,8}{⼀、⼧}
  \begin{phonetics}{一定}{yi2ding4}[][HSK 2]
    \definition{adv.}{certamente | definitivamente}
  \end{phonetics}
\end{entry}

\begin{entry}{一直}{1,8}{⼀、⽬}
  \begin{phonetics}{一直}{yi4zhi2}[][HSK 2]
    \definition{adv.}{diretamente | sempre em frente | o tempo todo | sempre | constantemente}
  \end{phonetics}
\end{entry}

\begin{entry}{一律}{1,9}{⼀、⼻}
  \begin{phonetics}{一律}{yi2lv4}[][HSK 4]
    \definition{adj.}{igual; semelhante; uniforme; parecido; idêntico}
    \definition{adv.}{todos; tudo; sem exceção; enfatiza que todos devem ser assim, sem exceção, e é usado principalmente em regulamentos ou requisitos}
  \end{phonetics}
\end{entry}

\begin{entry}{一战}{1,9}{⼀、⼽}
  \begin{phonetics}{一战}{yi2zhan4}
    \definition*{s.}{Primeira Guerra Mundial}
  \end{phonetics}
\end{entry}

\begin{entry}{一点儿}{1,9,2}{⼀、⽕、⼉}
  \begin{phonetics}{一点儿}{yi4dian3r5}[][HSK 1]
    \definition{adv.}{um pouco (``{adj.}+一点儿'' ou ``一点儿+{s.}'') | um ponto}
  \end{phonetics}
\end{entry}

\begin{entry}{一点点}{1,9,9}{⼀、⽕、⽕}
  \begin{phonetics}{一点点}{yi4 dian3 dian3}[][HSK 2]
    \definition{adj.}{um pouco}
  \end{phonetics}
\end{entry}

\begin{entry}{一样}{1,10}{⼀、⽊}
  \begin{phonetics}{一样}{yi2yang4}[][HSK 1]
    \definition{adj.}{igual | mesmo}
  \end{phonetics}
\end{entry}

\begin{entry}{一致}{1,10}{⼀、⾄}
  \begin{phonetics}{一致}{yi2zhi4}[][HSK 4]
    \definition{adj.}{equado; idêntico; uniforme; unânime; nenhuma diferença (de opinião ou ação)}
    \definition{adv.}{juntos; em conjunto}
  \end{phonetics}
\end{entry}

\begin{entry}{一般}{1,10}{⼀、⾈}
  \begin{phonetics}{一般}{yi4ban1}[][HSK 2]
    \definition{adj.}{geral | comum | normal}
    \definition{adv.}{normalmente}
  \end{phonetics}
\end{entry}

\begin{entry}{一般来说}{1,10,7,9}{⼀、⾈、⽊、⾔}
  \begin{phonetics}{一般来说}{yi4 ban1 lai2 shuo1}[][HSK 4]
    \definition{expr.}{de modo geral; na média; no caso usual; a declaração usual}
  \end{phonetics}
\end{entry}

\begin{entry}{一起}{1,10}{⼀、⾛}
  \begin{phonetics}{一起}{yi4qi3}[][HSK 1]
    \definition{adv.}{juntamente | em conjunto | no mesmo lugar | completamente | em todos}
  \end{phonetics}
\end{entry}

\begin{entry}{一部分}{1,10,4}{⼀、⾢、⼑}
  \begin{phonetics}{一部分}{yi2 bu4 fen4}[][HSK 2]
    \definition{adv.}{parcialmente}
    \definition{num.}{parte | porção | seção | fração}
    \definition[把]{s.}{parcial}
  \end{phonetics}
\end{entry}

\begin{entry}{一……就……}{1,12}{⼀、⼪}
  \begin{phonetics}{一……就……}{yi1 jiu4}
    \definition{expr.}{logo que |  uma vez que}
  \end{phonetics}
\end{entry}

\begin{entry}{一道}{1,12}{⼀、⾡}
  \begin{phonetics}{一道}{yi2dao4}
    \definition{adv.}{juntos | ao lado}
  \end{phonetics}
\end{entry}

\begin{entry}{一路平安}{1,13,5,6}{⼀、⾜、⼲、⼧}
  \begin{phonetics}{一路平安}{yi2 lu4 ping2 an1}[][HSK 2]
    \definition{expr.}{Boa viagem!}
    \definition{v.}{ter uma viagem agradável}
  \end{phonetics}
\end{entry}

\begin{entry}{一路顺风}{1,13,9,4}{⼀、⾜、⾴、⾵}
  \begin{phonetics}{一路顺风}{yi2 lu4 shun4 feng1}[][HSK 2]
    \definition{expr.}{ter uma viagem agradável}
  \end{phonetics}
\end{entry}

%%%%% EOF %%%%%

