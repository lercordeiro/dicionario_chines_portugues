%%%
%%% 1画
%%%

\section*{1画}\addcontentsline{toc}{section}{1画}

\begin{Entry}{一}{1}{⼀}[Kangxi 1]
  \begin{Phonetics}{一}{yi1}[(quando usado sozinho)][HSK 1]
    \definition{adv.}{uma vez; assim que; indica que duas ações ocorreram em um intervalo de tempo muito curto, uma terminando e a outra começando imediatamente em seguida | indica que primeiro se realiza uma ação e, em seguida, o resultado dessa ação  | indica uma ação única, indicando que a ação é muito curta ou apenas uma tentativa}
    \definition{num.}{um; 1 | pronunciado como \dpy{yao1} quando dito número a número | igual; refere-se ao mesmo ou igual | inteiro; todo; por toda parte | exclusivo ou único | refere-se a algo específico | também; caso contrário; referindo-se a outro ou mais um}
    \definition{part.}{antes de certas palavras para dar ênfase}
    \definition{prep.}{cada; por; toda vez}
    \definition{s.}{uma nota da escala em Gongchepu (工尺谱), correspondente ao 17 na notação musical numerada}
  \seealsoref{工尺谱}{gong1 che3 pu3}
  \end{Phonetics}
  \begin{Phonetics}{一}{yi2}[(antes de quarto tom)][HSK 1]
    \definition{num.}{um; 1 | um (artigo)}
  \end{Phonetics}
  \begin{Phonetics}{一}{yi4}[][HSK 1]
    \definition{adv.}{uma vez | assim que | ao}
    \definition{num.}{um; 1 | um (artigo)}
  \end{Phonetics}
\end{Entry}

\begin{Entry}{一下}{1,3}{⼀、⼀}
  \begin{Phonetics}{一下}{yi2xia4}
    \definition{adv.}{em um curto tempo | rapidamente}
  \end{Phonetics}
\end{Entry}

\begin{Entry}{一下儿}{1,3,2}{⼀、⼀、⼉}
  \begin{Phonetics}{一下儿}{yi2 xia4r5}[][HSK 1,5]
    \definition{s.}{um tempo; um momento}
  \end{Phonetics}
\end{Entry}

\begin{Entry}{一下子}{1,3,3}{⼀、⼀、⼦}
  \begin{Phonetics}{一下子}{yi2 xia4 zi5}[][HSK 5]
    \definition{adv.}{tudo de uma vez; de repente; em pouco tempo; em um curto espaço de tempo}
  \end{Phonetics}
\end{Entry}

\begin{Entry}{一个样}{1,3,10}{⼀、⼈、⽊}
  \begin{Phonetics}{一个样}{yi2ge5yang4}
    \definition{s.}{o mesmo}
  \seealsoref{一样}{yi2yang4}
  \end{Phonetics}
\end{Entry}

\begin{Entry}{一口气}{1,3,4}{⼀、⼝、⽓}
  \begin{Phonetics}{一口气}{yi4 kou3 qi4}[][HSK 5]
    \definition{adv.}{em um só fôlego; sem pausa; fazer algo continuamente}
  \end{Phonetics}
\end{Entry}

\begin{Entry}{一切}{1,4}{⼀、⼑}
  \begin{Phonetics}{一切}{yi2qie4}[][HSK 3]
    \definition{pron.}{tudo; todo; todas as coisas}
  \end{Phonetics}
\end{Entry}

\begin{Entry}{一方面}{1,4,9}{⼀、⽅、⾯}
  \begin{Phonetics}{一方面}{yi4 fang1 mian4}[][HSK 3]
    \definition{s.}{um lado; um dos dois aspectos opostos ou um lado de algo que está relacionado a outro}
  \seealsoref{一方面……,一方面……}{yi4 fang1 mian4 yi4 fang1 mian4}
  \end{Phonetics}
\end{Entry}

\begin{Entry}{一方面……,一方面……}{1,4,9,1,4,9}{⼀、⽅、⾯、⼀、⽅、⾯}
  \begin{Phonetics}{一方面……,一方面……}{yi4 fang1 mian4 yi4 fang1 mian4}[][HSK 3]
    \definition{conj.}{por um lado\dots, por outro lado\dots; conecta duas orações paralelas (devem ser usadas juntas)}[\underline{一方面}觉得兴奋,\underline{一方面}又害怕。===Por um lado, sinto-me entusiasmado, mas, por outro, também sinto medo.]
  \end{Phonetics}
\end{Entry}

\begin{Entry}{一代}{1,5}{⼀、⼈}
  \begin{Phonetics}{一代}{yi2 dai4}[][HSK 6]
    \definition{s.}{uma dinastia | era; época atual | vida; geração; toda a vida de uma pessoa}
  \end{Phonetics}
\end{Entry}

\begin{Entry}{一半}{1,5}{⼀、⼗}
  \begin{Phonetics}{一半}{yi2ban4}[][HSK 1]
    \definition{num.}{metade; em parte; uma metade}
  \end{Phonetics}
\end{Entry}

\begin{Entry}{一句话}{1,5,8}{⼀、⼝、⾔}
  \begin{Phonetics}{一句话}{yi2 ju4 hua4}[][HSK 5]
    \definition{s.}{em resumo; em uma palavra; expressar um conteúdo complexo de forma sucinta | trabalho fácil; fácil de fazer; descrever uma tarefa ou trabalho como muito simples e fácil de realizar}
  \end{Phonetics}
\end{Entry}

\begin{Entry}{一旦}{1,5}{⼀、⽇}
  \begin{Phonetics}{一旦}{yi2dan4}[][HSK 5]
    \definition{adv.}{uma vez; no caso; agora que | de repente; uma vez}
    \definition{s.}{em um único dia; em um tempo muito curto;}
  \end{Phonetics}
\end{Entry}

\begin{Entry}{一生}{1,5}{⼀、⽣}
  \begin{Phonetics}{一生}{yi4 sheng1}[][HSK 2]
    \definition{s.}{vida inteira; toda a vida; ao longo da vida; todo o tempo desde o nascimento até a morte; às vezes exagerado para indicar um longo período de tempo no curso da vida}
  \end{Phonetics}
\end{Entry}

\begin{Entry}{一边}{1,5}{⼀、⾡}
  \begin{Phonetics}{一边}{yi4bian1}[][HSK 1]
    \definition{adj.}{igual; idêntico; da mesma forma}
    \definition{adv.}{enquanto; ao mesmo tempo; simultaneamente; indica que uma ação ocorre simultaneamente a outra ação}
    \definition{s.}{lado; um lado; um aspecto | ambos os lados; ao lado de}
  \end{Phonetics}
\end{Entry}

\begin{Entry}{一会儿}{1,6,2}{⼀、⼈、⼉}
  \begin{Phonetics}{一会儿}{yi2 hui4r5}[][HSK 1,2]
    \definition{adv.}{agora\dots agora\dots; um momento\dots o próximo\dots; usado antes de dois antônimos, indica a alternância de situações}
    \definition{s.}{um pouquinho de tempo; muito pouco tempo}
  \end{Phonetics}
\end{Entry}

\begin{Entry}{一会儿……一会儿……}{1,6,2,1,6,2}{⼀、⼈、⼉、⼀、⼈、⼉}
  \begin{Phonetics}{一会儿……一会儿……}{yi1hui4r5 yi1hui4r5}
    \definition{adv.}{um tempo\dots um tempo\dots}
  \end{Phonetics}
\end{Entry}

\begin{Entry}{一共}{1,6}{⼀、⼋}
  \begin{Phonetics}{一共}{yi2gong4}[][HSK 2]
    \definition{adv.}{completamente; em tudo; no todo}
  \end{Phonetics}
\end{Entry}

\begin{Entry}{一再}{1,6}{⼀、⼌}
  \begin{Phonetics}{一再}{yi2zai4}[][HSK 4]
    \definition{adv.}{repetidamente; de novo e de novo; repetidas vezes; uma e outra vez}
  \end{Phonetics}
\end{Entry}

\begin{Entry}{一同}{1,6}{⼀、⼝}
  \begin{Phonetics}{一同}{yi4tong2}[][HSK 6]
    \definition{adv.}{juntos; ao mesmo tempo e lugar}
  \end{Phonetics}
\end{Entry}

\begin{Entry}{一向}{1,6}{⼀、⼝}
  \begin{Phonetics}{一向}{yi2xiang4}[][HSK 5]
    \definition{adv.}{desde o início; indica do passado até o presente}
  \end{Phonetics}
\end{Entry}

\begin{Entry}{一次性}{1,6,8}{⼀、⽋、⼼}
  \begin{Phonetics}{一次性}{yi2 ci4 xing4}[][HSK 6]
    \definition{adj.}{único; uso único; descartável (produtos); apenas uma vez, sem necessidade ou necessidade de fazer novamente}
  \end{Phonetics}
\end{Entry}

\begin{Entry}{一行}{1,6}{⼀、⾏}
  \begin{Phonetics}{一行}{yi1 xing2}[][HSK 6]
    \definition{s.}{delegação; um grupo viajando junto; festa}
  \end{Phonetics}
\end{Entry}

\begin{Entry}{一齐}{1,6}{⼀、⿑}
  \begin{Phonetics}{一齐}{yi4 qi2}[][HSK 6]
    \definition{adv.}{juntos; em uníssono; simultaneamente; ao mesmo tempo; indica que diferentes sujeitos emitem simultaneamente o mesmo comportamento ou o mesmo sujeito emite vários comportamentos diferentes ao mesmo tempo}
  \end{Phonetics}
\end{Entry}

\begin{Entry}{一块}{1,7}{⼀、⼟}
  \begin{Phonetics}{一块}{yi2kuai4}
    \definition{adv.}{(principalmente mandarim) juntos}
  \end{Phonetics}
\end{Entry}

\begin{Entry}{一块儿}{1,7,2}{⼀、⼟、⼉}
  \begin{Phonetics}{一块儿}{yi2 kuai4r5}[][HSK 1]
    \definition{adv.}{juntos; em conjunto}
    \definition{s.}{no mesmo lugar; no mesmo local}
  \end{Phonetics}
\end{Entry}

\begin{Entry}{一时}{1,7}{⼀、⽇}
  \begin{Phonetics}{一时}{yi4 shi2}[][HSK 6]
    \definition{adv.}{por um curto período; temporário | (usado em pares) agora\dots, agora\dots; este momento\dots, e o próximo\dots; o mesmo que 时而}
    \definition{s.}{um período de tempo | um momento; um breve momento; um tempo muito curto}
  \seealsoref{时而}{shi2'er2}
  \seealsoref{一时……,一时……}{yi4 shi2 yi4 shi2}
  \end{Phonetics}
\end{Entry}

\begin{Entry}{一时……,一时……}{1,7,1,7}{⼀、⽇、⼀、⽇}
  \begin{Phonetics}{一时……,一时……}{yi4 shi2 yi4 shi2}[][HSK 6]
    \definition{adv.}{por um tempo\dots, por um tempo\dots}
  \seealsoref{一时}{yi4 shi2}
  \end{Phonetics}
\end{Entry}

\begin{Entry}{一身}{1,7}{⼀、⾝}
  \begin{Phonetics}{一身}{yi4 shen1}[][HSK 5]
    \definition{s.}{o corpo inteiro; em todo o corpo | um terno; (um conjunto completo de) roupas | sozinho; uma única pessoa; relativo a uma única pessoa}
  \end{Phonetics}
\end{Entry}

\begin{Entry}{一些}{1,8}{⼀、⼆}
  \begin{Phonetics}{一些}{yi4 xie1}[][HSK 1]
    \definition{clas.}{alguns; um número de; quantidade indeterminada | um pouco; uma pequena quantidade | mais de um; mais de uma vez; indica mais de um ou mais de uma vez, etc. | uma ligeira mudança no grau, intensidade; usado após certos verbos, adjetivos, etc., para indicar uma quantidade muito pequena}
    \definition{pron.}{uns; alguns}
  \end{Phonetics}
\end{Entry}

\begin{Entry}{一定}{1,8}{⼀、⼧}
  \begin{Phonetics}{一定}{yi2ding4}[][HSK 2]
    \definition{adj.}{certo; particular; tendo um certo nível de especificidade; (objeto, situação) determinado em um ou mais | devido; certo; sempre foi assim, não vai mudar | fixo; especificado; há requisitos claros quanto à maneira, método, quantidade, etc.}
    \definition{adv.}{certamente; necessariamente; expressando determinação ou certeza | certamente; indica especulação ou avaliação de que um evento ou situação definitivamente acontecerá ou realmente existirá}
  \end{Phonetics}
\end{Entry}

\begin{Entry}{一直}{1,8}{⼀、⽬}
  \begin{Phonetics}{一直}{yi4zhi2}[][HSK 2]
    \definition{adv.}{direto; indica que permanece inalterado em uma direção | sempre; continuamente; o tempo todo; o tempo todo; indica que a ação é sempre ininterrupta ou o estado é sempre inalterado | de um ponto a outro sem enfatizar nenhuma exceção}
  \end{Phonetics}
\end{Entry}

\begin{Entry}{一贯}{1,8}{⼀、⾙}
  \begin{Phonetics}{一贯}{yi2guan4}[][HSK 6]
    \definition{adj./adv.}{do começo ao fim; inabalável; consistente; persistente; o tempo todo}
  \end{Phonetics}
\end{Entry}

\begin{Entry}{一带}{1,9}{⼀、⼱}
  \begin{Phonetics}{一带}{yi2 dai4}[][HSK 5]
    \definition{s.}{a área em torno de um determinado local; refere-se a um determinado local e suas proximidades}
  \end{Phonetics}
\end{Entry}

\begin{Entry}{一律}{1,9}{⼀、⼻}
  \begin{Phonetics}{一律}{yi2lv4}[][HSK 4]
    \definition{adj.}{igual; semelhante; uniforme; parecido; idêntico}
    \definition{adv.}{todos; tudo; sem exceção; enfatiza que todos devem ser assim, sem exceção, e é usado principalmente em regulamentos ou requisitos}
  \end{Phonetics}
\end{Entry}

\begin{Entry}{一战}{1,9}{⼀、⼽}
  \begin{Phonetics}{一战}{yi2zhan4}
    \definition*{s.}{Primeira Guerra Mundial}
  \end{Phonetics}
\end{Entry}

\begin{Entry}{一点儿}{1,9,2}{⼀、⽕、⼉}
  \begin{Phonetics}{一点儿}{yi4dian3r5}[][HSK 1]
    \definition{adv.}{um pouco; uma pitada; uma gota; uma amostra; uma pequena quantidade; ({adj.} + (一)点儿, 一点儿 + {s.} ou 有 + (一)点儿 + {s.})}
  \end{Phonetics}
\end{Entry}

\begin{Entry}{一点点}{1,9,9}{⼀、⽕、⽕}
  \begin{Phonetics}{一点点}{yi4 dian3 dian3}[][HSK 2]
    \definition{adj.}{um pouco; muito pouco ou um pouquinho}
  \end{Phonetics}
\end{Entry}

\begin{Entry}{一样}{1,10}{⼀、⽊}
  \begin{Phonetics}{一样}{yi2yang4}[][HSK 1]
    \definition{adj.}{o mesmo; igualmente; semelhante; tão\dots quanto\dots}
    \definition{part.}{na mesma medida; anexado a verbos ou palavras nominais, indica uma comparação ou semelhança, equivalente a 似的}
  \seealsoref{似的}{shi4de5}
  \end{Phonetics}
\end{Entry}

\begin{Entry}{一流}{1,10}{⼀、⽔}
  \begin{Phonetics}{一流}{yi4liu2}[][HSK 5]
    \definition{adj.}{clássico; de primeira linha; de primeira classe; o melhor}
    \definition[些]{s.}{tipo; mesmo tipo; da mesma classe; da mesma categoria; uma categoria}
  \end{Phonetics}
\end{Entry}

\begin{Entry}{一致}{1,10}{⼀、⾄}
  \begin{Phonetics}{一致}{yi2zhi4}[][HSK 4]
    \definition{adj.}{equado; idêntico; uniforme; unânime; nenhuma diferença (de opinião ou ação)}
    \definition{adv.}{juntos; em conjunto}
  \end{Phonetics}
\end{Entry}

\begin{Entry}{一般}{1,10}{⼀、⾈}
  \begin{Phonetics}{一般}{yi4ban1}[][HSK 2]
    \definition{adj.}{o mesmo que; exatamente como | geral; ordinário; comum | médio; medíocre; o grau ou nível não é muito alto}
    \definition{adv.}{frequentemente; geralmente}
  \end{Phonetics}
\end{Entry}

\begin{Entry}{一般来说}{1,10,7,9}{⼀、⾈、⽊、⾔}
  \begin{Phonetics}{一般来说}{yi4 ban1 lai2 shuo1}[][HSK 4]
    \definition{expr.}{de modo geral; na média; no caso usual; a declaração usual}
  \end{Phonetics}
\end{Entry}

\begin{Entry}{一起}{1,10}{⼀、⾛}
  \begin{Phonetics}{一起}{yi4qi3}[][HSK 1]
    \definition{adv.}{juntos; em companhia; indica o mesmo local, ao mesmo tempo que se faz algo | no total; em todos; no conjunto}
    \definition{s.}{no mesmo lugar}
  \end{Phonetics}
\end{Entry}

\begin{Entry}{一部分}{1,10,4}{⼀、⾢、⼑}
  \begin{Phonetics}{一部分}{yi2 bu4 fen4}[][HSK 2]
    \definition{adj.}{parcial}
    \definition{adv.}{parcialmente}
    \definition{num.}{parte; porção; seção; fração}
  \end{Phonetics}
\end{Entry}

\begin{Entry}{一……就……}{1,12}{⼀、⼪}
  \begin{Phonetics}{一……就……}{yi1 jiu4}
    \definition{expr.}{logo que |  uma vez que}
  \end{Phonetics}
\end{Entry}

\begin{Entry}{一番}{1,12}{⼀、⽥}
  \begin{Phonetics}{一番}{yi4 fan1}[][HSK 6]
    \definition{adv.}{uma demonstração de, uma dose de, um pedaço de (conversa, investigação, pensamento)}
  \end{Phonetics}
\end{Entry}

\begin{Entry}{一辈子}{1,12,3}{⼀、⾞、⼦}
  \begin{Phonetics}{一辈子}{yi2bei4zi5}[][HSK 5]
    \definition{s.}{uma vida inteira; vida inteira; toda a vida; durante toda a vida; enquanto se vive; todo o tempo entre o nascimento e a morte}
  \end{Phonetics}
\end{Entry}

\begin{Entry}{一道}{1,12}{⼀、⾡}
  \begin{Phonetics}{一道}{yi2 dao4}[][HSK 6]
    \definition{adv.}{juntos; lado a lado; junto com}
  \end{Phonetics}
\end{Entry}

\begin{Entry}{一路}{1,13}{⼀、⾜}
  \begin{Phonetics}{一路}{yi2 lu4}[][HSK 5]
    \definition{adv.}{o tempo todo; persistentemente; continuamente | juntos; sem parar; continuamente}
    \definition{s.}{o mesmo caminho; a mesma rota; ao longo de toda a viagem, ao longo do caminho | do mesmo tipo; da mesma categoria}
  \end{Phonetics}
\end{Entry}

\begin{Entry}{一路上}{1,13,3}{⼀、⾜、⼀}
  \begin{Phonetics}{一路上}{yi2 lu4 shang4}[][HSK 6]
    \definition{s.}{ao longo do caminho; todo o caminho}
  \end{Phonetics}
\end{Entry}

\begin{Entry}{一路平安}{1,13,5,6}{⼀、⾜、⼲、⼧}
  \begin{Phonetics}{一路平安}{yi2 lu4 ping2 an1}[][HSK 2]
    \definition{expr.}{Boa viagem!; Tenha uma boa viagem!}
    \definition{v.}{ter uma viagem agradável}
  \end{Phonetics}
\end{Entry}

\begin{Entry}{一路顺风}{1,13,9,4}{⼀、⾜、⾴、⾵}
  \begin{Phonetics}{一路顺风}{yi2 lu4 shun4 feng1}[][HSK 2]
    \definition{expr.}{ter uma viagem agradável; toda a viagem foi segura e tranquila; é uma metáfora para cada etapa do processo de lidar com algo que ocorre sem problemas | Tenha uma boa viagem!; Boa viagem!}
  \end{Phonetics}
\end{Entry}

\begin{Entry}{一模一样}{1,14,1,10}{⼀、⽊、⼀、⽊}
  \begin{Phonetics}{一模一样}{yi4 mu2 yi2 yang4}[][HSK 6]
    \definition{expr.}{tão parecidos quanto duas ervilhas; ser exatamente iguais; muito parecido, a mesma aparência}
  \end{Phonetics}
\end{Entry}

\begin{Entry}{乙}{1}{⼄}[Kangxi 5]
  \begin{Phonetics}{乙}{yi3}[][HSK 5]
    \definition*{s.}{Sobrenome Yi}
    \definition{num.}{segundo}
    \definition{s.}{o segundo lugar do Tian Gan | uma nota da escala em Gongchepu (工尺谱); nível superior na música tradicional chinesa}
  \seealsoref{工尺谱}{gong1 che3 pu3}
  \end{Phonetics}
\end{Entry}

%%%%% EOF %%%%%

