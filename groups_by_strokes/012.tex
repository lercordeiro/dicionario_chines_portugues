%%%
%%% 12画
%%%

\section*{12画}\addcontentsline{toc}{section}{12画}

\begin{Entry}{傍}{12}{⼈}
  \begin{Phonetics}{傍}{bang4}
    \definition*{s.}{Sobrenome Bang}
    \definition{v.}{estar perto de (à distância); aproximar-se | estar perto de (no tempo) | depender de; confiar em}
  \end{Phonetics}
\end{Entry}

\begin{Entry}{傍晚}{12,11}{⼈、⽇}
  \begin{Phonetics}{傍晚}{bang4wan3}[][HSK 6]
    \definition[个]{s.}{ao entardecer; ao cair da noite; (tarde) refere-se ao momento em que se aproxima o anoitecer, frequentemente usado na linguagem escrita}
  \end{Phonetics}
\end{Entry}

\begin{Entry}{傢}{12}{⼈}
  \begin{Phonetics}{傢}{jia1}
    \definition{s.}{usado em 家伙  e 家俱}
    \variantof{家}
  \seealsoref{傢伙}{jia1huo5}
  \seealsoref{家俱}{jia1ju4}
  \end{Phonetics}
\end{Entry}

\begin{Entry}{傢伙}{12,6}{⼈、⼈}
  \begin{Phonetics}{傢伙}{jia1huo5}
    \variantof{家伙}
  \end{Phonetics}
\end{Entry}

\begin{Entry}{傢俱}{12,10}{⼈、⼈}
  \begin{Phonetics}{傢俱}{jia1ju4}
    \variantof{家俱}
  \end{Phonetics}
\end{Entry}

\begin{Entry}{储}{12}{⼈}
  \begin{Phonetics}{储}{chu3}
    \definition*{s.}{Sobrenome Chu}
    \definition{s.}{herdeiro de um trono | herdeiro}
    \definition{v.}{armazenar | guardar; manter (ter) em reserva}
  \end{Phonetics}
\end{Entry}

\begin{Entry}{储存}{12,6}{⼈、⼦}
  \begin{Phonetics}{储存}{chu3cun2}[][HSK 6]
    \definition{v.}{armazenar; depositar; colocar em; economizar dinheiro ou coisas que você não precisará em um futuro próximo}
  \end{Phonetics}
\end{Entry}

\begin{Entry}{储备}{12,8}{⼈、⼡}
  \begin{Phonetics}{储备}{chu3bei4}[][HSK 7-9]
    \definition{v.}{guardar; armazenar para uso futuro}
  \end{Phonetics}
\end{Entry}

\begin{Entry}{储蓄}{12,13}{⼈、⾋}
  \begin{Phonetics}{储蓄}{chu3xu4}[][HSK 7-9]
    \definition[笔,份]{s.}{depósitos; poupanças; refere-se a dinheiro ou coisas acumuladas}
    \definition{v.}{poupar; depositar; guardar dinheiro ou coisas que são guardadas ou não são usadas temporariamente, geralmente significa depositar dinheiro em um banco}
  \end{Phonetics}
\end{Entry}

\begin{Entry}{傲}{12}{⼈}
  \begin{Phonetics}{傲}{ao4}[][HSK 7-9]
    \definition{adj.}{orgulhoso; altivo | arrogante}
    \definition{v.}{recusar-se a ceder; desafiar}
  \end{Phonetics}
\end{Entry}

\begin{Entry}{傲慢}{12,14}{⼈、⼼}
  \begin{Phonetics}{傲慢}{ao4man4}[][HSK 7-9]
    \definition{adj.}{altivo; arrogante; autoritário}
  \end{Phonetics}
\end{Entry}

\begin{Entry}{剩}{12}{⼑}
  \begin{Phonetics}{剩}{sheng4}[][HSK 5]
    \definition*{s.}{Sobrenome Sheng}
    \definition{v.}{permanecer; ser deixado (para trás)}
  \end{Phonetics}
\end{Entry}

\begin{Entry}{剩下}{12,3}{⼑、⼀}
  \begin{Phonetics}{剩下}{sheng4 xia4}[][HSK 5]
    \definition{v.}{permanecer; ser deixado (para trás); consumir e utilizar, restando apenas os resíduos}
  \end{Phonetics}
\end{Entry}

\begin{Entry}{割}{12}{⼑}
  \begin{Phonetics}{割}{ge1}[][HSK 7-9]
    \definition{v.}{cortar; ceifar | dividir; cortar}
  \end{Phonetics}
\end{Entry}

\begin{Entry}{博}{12}{⼗}
  \begin{Phonetics}{博}{bo2}
    \definition*{s.}{Sobrenome Bo}
    \definition{adj.}{rico; abundante | erudito; bem informado | solto; grande | grande}
    \definition{s.}{doutor em filosofia; doutorado}
    \definition{v.}{ter um amplo conhecimento de; ser bem lido | ganhar; vencer | jogar}
  \end{Phonetics}
\end{Entry}

\begin{Entry}{博士}{12,3}{⼗、⼠}
  \begin{Phonetics}{博士}{bo2shi4}[][HSK 5]
    \definition[位,名,个,些]{s.}{doutorado; grau de doutor; nível mais alto de um diploma; também, uma pessoa que obteve esse diploma | doutor; antigo título honorífico para uma pessoa que é habilidosa em um determinado ofício ou especializada em uma determinada ocupação | doutor; autoridades que ensinavam as escrituras na China nos tempos antigos}
  \end{Phonetics}
\end{Entry}

\begin{Entry}{博文}{12,4}{⼗、⽂}
  \begin{Phonetics}{博文}{bo2wen2}
    \definition{s.}{artigo em um blog}
    \definition{v.}{escrever um artigo em um blog}
  \end{Phonetics}
\end{Entry}

\begin{Entry}{博主}{12,5}{⼗、⼂}
  \begin{Phonetics}{博主}{bo2zhu3}
    \definition{s.}{blogueiro}
  \end{Phonetics}
\end{Entry}

\begin{Entry}{博物馆}{12,8,11}{⼗、⽜、⾷}
  \begin{Phonetics}{博物馆}{bo2wu4guan3}[][HSK 5]
    \definition[座,个]{s.}{museu; locais para coleta, armazenamento, pesquisa, exibição e exposição de relíquias culturais ou espécimes relacionados à história, cultura, arte, ciências naturais, ciência e tecnologia, etc.}
  \end{Phonetics}
\end{Entry}

\begin{Entry}{博客}{12,9}{⼗、⼧}
  \begin{Phonetics}{博客}{bo2 ke4}[][HSK 5]
    \definition[个]{s.}{\emph{blog}; página da Web ou site gerenciado por um indivíduo, geralmente composto por postagens organizadas da mais recente para a mais antiga | blogueiro; \emph{blogger}; pessoas que possuem ou escrevem \emph{blogs}}
  \end{Phonetics}
\end{Entry}

\begin{Entry}{博览会}{12,9,6}{⼗、⾒、⼈}
  \begin{Phonetics}{博览会}{bo2lan3hui4}[][HSK 5]
    \definition[次,届]{s.}{exposição; feira internacional; exposições de produtos em grande escala}
  \end{Phonetics}
\end{Entry}

\begin{Entry}{厨}{12}{⼚}
  \begin{Phonetics}{厨}{chu2}
    \definition[个]{s.}{cozinha}
  \end{Phonetics}
\end{Entry}

\begin{Entry}{厨师}{12,6}{⼚、⼱}
  \begin{Phonetics}{厨师}{chu2 shi1}[][HSK 6]
    \definition[名,位,个]{s.}{chefe de cozinha; cozinheiro; alguém que é bom em cozinhar e faz disso uma profissão}
  \end{Phonetics}
\end{Entry}

\begin{Entry}{厨房}{12,8}{⼚、⼾}
  \begin{Phonetics}{厨房}{chu2fang2}[][HSK 5]
    \definition[间,个]{s.}{cozinha}
  \end{Phonetics}
\end{Entry}

\begin{Entry}{喂}{12}{⼝}
  \begin{Phonetics}{喂}{wei4}[][HSK 2,4]
    \definition{interj.}{Ei!, Olá!, para chamar atenção | Alô? (quando respondendo uma chamada telefônica, pronuncia-se como \dpy{wei2})}
    \definition{v.}{criar; alimentar (animais); dar comida a um animal | alimentar (pessoas); colocar alimentos, medicamentos, etc. na boca de alguém}
  \end{Phonetics}
\end{Entry}

\begin{Entry}{喂奶}{12,5}{⼝、⼥}
  \begin{Phonetics}{喂奶}{wei4nai3}
    \definition{v.}{amamentar}
  \end{Phonetics}
\end{Entry}

\begin{Entry}{喂母乳}{12,5,8}{⼝、⽏、⼄}
  \begin{Phonetics}{喂母乳}{wei4mu3ru3}
    \definition{s.}{amamentação}
  \end{Phonetics}
\end{Entry}

\begin{Entry}{喂养}{12,9}{⼝、⼋}
  \begin{Phonetics}{喂养}{wei4yang3}
    \definition{v.}{alimentar (uma criança, animal doméstico, etc.) | manter | criar (um animal)}
  \end{Phonetics}
\end{Entry}

\begin{Entry}{喂食}{12,9}{⼝、⾷}
  \begin{Phonetics}{喂食}{wei4shi2}
    \definition{v.}{alimentar}
  \end{Phonetics}
\end{Entry}

\begin{Entry}{喂哺}{12,10}{⼝、⼝}
  \begin{Phonetics}{喂哺}{wei4bu3}
    \definition{v.}{alimentar (um bebê)}
  \end{Phonetics}
\end{Entry}

\begin{Entry}{喂料}{12,10}{⼝、⽃}
  \begin{Phonetics}{喂料}{wei4liao4}
    \definition{v.}{alimentar (também no sentido figurativo)}
  \end{Phonetics}
\end{Entry}

\begin{Entry}{善}{12}{⼝}
  \begin{Phonetics}{善}{shan4}
    \definition*{s.}{Sobrenome Shan}
    \definition{adj.}{bom; bem | bom; satisfatório | gentil; amigável | familiar}
    \definition{adv.}{bom; bem}
    \definition{s.}{boa ação; ato benevolente; coisas boas (em oposição a 恶)}
    \definition{v.}{fazer sucesso; fazer bem; fazer acontecer | ser bom em; ser especialista (versado) em | ser apto a}
  \seealsoref{恶}{e4}
  \end{Phonetics}
\end{Entry}

\begin{Entry}{善于}{12,3}{⼝、⼆}
  \begin{Phonetics}{善于}{shan4yu2}[][HSK 4]
    \definition{adv./v.}{ser bom em; ser hábil em}
  \end{Phonetics}
\end{Entry}

\begin{Entry}{善良}{12,7}{⼝、⾉}
  \begin{Phonetics}{善良}{shan4liang2}[][HSK 4]
    \definition{adj.}{de bom coração; bom e honesto; de bom coração e cheio de boa vontade}
  \end{Phonetics}
\end{Entry}

\begin{Entry}{善意}{12,13}{⼝、⼼}
  \begin{Phonetics}{善意}{shan4yi4}
    \definition{s.}{boa vontade | benevolência | bondade}
  \end{Phonetics}
\end{Entry}

\begin{Entry}{喉}{12}{⼝}
  \begin{Phonetics}{喉}{hou2}
    \definition{s.}{laringe; garganta; a parte do órgão respiratório de humanos e vertebrados terrestres, localizada entre a faringe e a traqueia, tem as funções de ventilação e pronúncia; a faringe e a laringe são geralmente misturadas e chamadas de garganta ou caixa vocal}
  \end{Phonetics}
\end{Entry}

\begin{Entry}{喉咙}{12,8}{⼝、⼝}
  \begin{Phonetics}{喉咙}{hou2long2}[][HSK 7-9]
    \definition{s.}{garganta; laringe}
  \end{Phonetics}
\end{Entry}

\begin{Entry}{喊}{12}{⼝}
  \begin{Phonetics}{喊}{han3}[][HSK 2]
    \definition{v.}{gritar; clamar; berrar | chamar (uma pessoa) | chamar; dirigir-se a}
  \end{Phonetics}
\end{Entry}

\begin{Entry}{喔}{12}{⼝}
  \begin{Phonetics}{喔}{o1}
    \definition{interj.}{Oh!, Entendi!, usado para indicar realização, compreensão}
  \end{Phonetics}
\end{Entry}

\begin{Entry}{喘}{12}{⼝}
  \begin{Phonetics}{喘}{chuan3}[][HSK 7-9]
    \definition{s.}{Medicina: asma}
    \definition{v.}{respirar pesadamente; ofegar por ar; ofegar | sopro; respiração rápida}
  \end{Phonetics}
\end{Entry}

\begin{Entry}{喘息}{12,10}{⼝、⼼}
  \begin{Phonetics}{喘息}{chuan3xi1}[][HSK 7-9]
    \definition{s.}{sopro; respiração rápida |respirador; pausas curtas durante atividades intensas | síndrome caracterizada por dispneia | (ponto de acupuntura) chuanxi}
    \definition{v.}{ofegar; ofegar por ar | fazer uma pausa para respirar; fazer uma pausa}
  \end{Phonetics}
\end{Entry}

\begin{Entry}{喜}{12}{⼝}
  \begin{Phonetics}{喜}{xi3}
    \definition{adj.}{feliz; satisfeito; encantado}
    \definition[桩,件]{s.}{evento feliz (especialmente casamento); ocasião para celebração; algo para comemorar | gravidez | casamento ou coisas relacionadas a ele}
    \definition{v.}{gostar; fonte de; ter inclinação para | precisa; requer; combina melhor com; (um certo organismo) precisa ou é adequado para (um certo ambiente ou algo)}
  \end{Phonetics}
\end{Entry}

\begin{Entry}{喜欢}{12,6}{⼝、⽋}
  \begin{Phonetics}{喜欢}{xi3huan5}[][HSK 1]
    \definition{adj.}{feliz; encantado; exultante; cheio de alegria}
    \definition{v.}{gostar; amar; ter afeição por; estar interessado em; ter uma boa impressão ou interesse por alguém ou algo}
  \end{Phonetics}
\end{Entry}

\begin{Entry}{喜剧}{12,10}{⼝、⼑}
  \begin{Phonetics}{喜剧}{xi3 ju4}[][HSK 5]
    \definition[部,出]{s.}{comédia (oposto de 悲剧) | comédia; uma das principais categorias do teatro; usa o exagero para satirizar e ridicularizar o feio; fenômenos retrógrados; destaca as contradições inerentes a esses fenômenos e seu conflito com coisas saudáveis; costuma provocar risadas; o final geralmente é feliz}
  \seealsoref{悲剧}{bei1 ju4}
  \end{Phonetics}
\end{Entry}

\begin{Entry}{喜爱}{12,10}{⼝、⽖}
  \begin{Phonetics}{喜爱}{xi3 ai4}[][HSK 4]
    \definition{v.}{gostar; amar; ter afeição por; estar interessado em; ter uma queda ou sentir interesse por pessoas ou coisas}
  \end{Phonetics}
\end{Entry}

\begin{Entry}{喝}{12}{⼝}
  \begin{Phonetics}{喝}{he1}[][HSK 1]
    \definition{interj.}{Meu Deus!; Oh!; Ah!; Uau!}
    \definition{s.}{bebida; especificamente, vinho}
    \definition{v.}{beber; engolir líquidos ou alimentos líquidos | beber bebida alcoólica; referência específica ao consumo de álcool}
  \end{Phonetics}
  \begin{Phonetics}{喝}{he4}
    \definition{v.}{gritar bem alto}
  \end{Phonetics}
\end{Entry}

\begin{Entry}{喝采}{12,8}{⼝、⾤}
  \begin{Phonetics}{喝采}{he4/cai3}[][HSK 7-9]
    \definition{v.+compl.}{aclamar; aplaudir}
  \end{Phonetics}
\end{Entry}

\begin{Entry}{喝彩}{12,11}{⼝、⼺}
  \begin{Phonetics}{喝彩}{he4cai3}
    \definition{s.}{aclamar | torcer}
  \end{Phonetics}
\end{Entry}

\begin{Entry}{喝醉}{12,15}{⼝、⾣}
  \begin{Phonetics}{喝醉}{he1zui4}
    \definition{v.}{ficar bêbado}
  \end{Phonetics}
\end{Entry}

\begin{Entry}{喷}{12}{⼝}
  \begin{Phonetics}{喷}{pen1}[][HSK 5]
    \definition{v.}{jorrar; esguichar; expelir sob pressão | borrifar; espalhar; pulverizar}
  \end{Phonetics}
  \begin{Phonetics}{喷}{pen4}
    \definition{s.}{na época; tempo no mercado; época em que frutas, peixes e camarões são comercializados em grande quantidade | colheita; número de vezes que floresceu e frutificou; número de vezes que foi colhido na maturação}
  \end{Phonetics}
\end{Entry}

\begin{Entry}{喻}{12}{⼝}
  \begin{Phonetics}{喻}{yu4}
    \definition{s.}{analogia | símile | metáfora | alegoria}
    \definition{v.}{descrever algo como}
  \end{Phonetics}
\end{Entry}

\begin{Entry}{堡}{12}{⼟}
  \begin{Phonetics}{堡}{bao3}
    \definition{s.}{forte; fortaleza | uma terraplenagem | castelo | posição de defesa | usado em nomes de lugares}
  \end{Phonetics}
  \begin{Phonetics}{堡}{bu3}
    \definition{s.}{forte; vila; cidade; assentamento fortificado (uma vila ou cidade cercada por muros de terra, frequentemente usada em nomes de lugares) | cidade; frequentemente usado em nomes de lugares}
  \end{Phonetics}
  \begin{Phonetics}{堡}{pu4}
    \definition{s.}{cidade ou rua (frequentemente usado em nomes de lugares)}
  \end{Phonetics}
\end{Entry}

\begin{Entry}{堡垒}{12,9}{⼟、⼟}
  \begin{Phonetics}{堡垒}{bao3lei3}[][HSK 7-9]
    \definition[处,座,个]{s.}{forte; fortaleza; casamata | bastião | fortificação}
  \end{Phonetics}
\end{Entry}

\begin{Entry}{堤}{12}{⼟}
  \begin{Phonetics}{堤}{di1}[][HSK 7-9]
    \definition[道,条]{s.}{dique; aterro}
  \end{Phonetics}
\end{Entry}

\begin{Entry}{堤坝}{12,7}{⼟、⼟}
  \begin{Phonetics}{堤坝}{di1ba4}[][HSK 7-9]
    \definition[座,道,个]{s.}{diques; barragem; represa}
  \end{Phonetics}
\end{Entry}

\begin{Entry}{塔}{12}{⼟}
  \begin{Phonetics}{塔}{ta3}[][HSK 6]
    \definition*{s.}{Sobrenome Ta}
    \definition[个,座]{s.}{pagode budista; pagode | torre | (química) coluna; torre}[蒸馏塔===torre de destilação]
  \end{Phonetics}
\end{Entry}

\begin{Entry}{奠}{12}{⼤}
  \begin{Phonetics}{奠}{dian4}
    \definition{v.}{estabelecer; construir; fundar; lançar a pedra fundamental | fazer oferendas aos espíritos dos mortos | consertar}
  \end{Phonetics}
\end{Entry}

\begin{Entry}{奠定}{12,8}{⼤、⼧}
  \begin{Phonetics}{奠定}{dian4ding4}[][HSK 7-9]
    \definition{v.}{estabelecer; estabelecer de forma estável; tornar estável; estabelecer a base}
  \end{Phonetics}
\end{Entry}

\begin{Entry}{奥}{12}{⼤}
  \begin{Phonetics}{奥}{ao4}
    \definition*{s.}{Oersted, a unidade eletromagnética de intensidade do campo magnético; abreviação de 奥斯特 | Sobrenome Ao}
    \definition{adj.}{profundo e difícil de entender; abstruso | significado profundo, não é fácil de entender}
    \definition{s.}{canto secreto da casa; antigamente, referia-se ao canto sudoeste de uma casa e também, de modo geral, à profundidade de uma casa}
  \seealsoref{奥斯特}{ao4 si1 te4}
  \end{Phonetics}
\end{Entry}

\begin{Entry}{奥运}{12,7}{⼤、⾡}
  \begin{Phonetics}{奥运}{ao4yun4}
    \definition*{s.}{Jogos Olímpicos, Olimpíadas; Abreviação de 奥林匹克运动会}
  \seealsoref{奥林匹克运动会}{ao4lin2pi3ke4 yun4dong4hui4}
  \end{Phonetics}
\end{Entry}

\begin{Entry}{奥运会}{12,7,6}{⼤、⾡、⼈}
  \begin{Phonetics}{奥运会}{ao4yun4hui4}[][HSK 7-9]
    \definition*[届,次]{s.}{Jogos Olímpicos, Olimpíadas; Abreviação de 奥林匹克运动会}
  \seealsoref{奥林匹克运动会}{ao4lin2pi3ke4 yun4dong4hui4}
  \end{Phonetics}
\end{Entry}

\begin{Entry}{奥林匹克运动会}{12,8,4,7,7,6,6}{⼤、⽊、⼖、⼗、⾡、⼒、⼈}
  \begin{Phonetics}{奥林匹克运动会}{ao4lin2pi3ke4 yun4dong4hui4}
    \definition*{s.}{Jogos Olímpicos, Olimpíadas}
  \end{Phonetics}
\end{Entry}

\begin{Entry}{奥特曼}{12,10,11}{⼤、⽜、⽈}
  \begin{Phonetics}{奥特曼}{ao4te4man4}
    \definition*{s.}{Ultraman,  super-herói de ficção científica japonesa}
  \end{Phonetics}
\end{Entry}

\begin{Entry}{奥秘}{12,10}{⼤、⽲}
  \begin{Phonetics}{奥秘}{ao4mi4}[][HSK 7-9]
    \definition[个]{s.}{enigma; mistério profundo; fenômenos ou princípios profundos e misteriosos}
  \end{Phonetics}
\end{Entry}

\begin{Entry}{奥斯特}{12,12,10}{⼤、⽄、⽜}
  \begin{Phonetics}{奥斯特}{ao4 si1 te4}
    \definition{s.}{Oersted}
  \end{Phonetics}
\end{Entry}

\begin{Entry}{媒}{12}{⼥}
  \begin{Phonetics}{媒}{mei2}
    \definition{s.}{casamenteiro; intermediário | intermediário; médio}
    \definition{v.}{fazer uma combinação}
  \end{Phonetics}
\end{Entry}

\begin{Entry}{媒体}{12,7}{⼥、⼈}
  \begin{Phonetics}{媒体}{mei2ti3}[][HSK 3]
    \definition[家,个,种]{s.}{mídia; mídia de massa; vários meios de comunicação e transmissão de informações, como televisão, rádio, jornais, etc.}
  \end{Phonetics}
\end{Entry}

\begin{Entry}{嫂}{12}{⼥}
  \begin{Phonetics}{嫂}{sao3}
    \definition[个,位,名,些]{s.}{esposa do irmão mais velho; cunhada | irmã (uma forma de tratamento para uma mulher casada, mais ou menos da mesma idade)}
  \end{Phonetics}
\end{Entry}

\begin{Entry}{嫂子}{12,3}{⼥、⼦}
  \begin{Phonetics}{嫂子}{sao3zi5}
    \definition{s.}{esposa do irmão mais velho}
  \end{Phonetics}
\end{Entry}

\begin{Entry}{富}{12}{⼧}
  \begin{Phonetics}{富}{fu4}[][HSK 3]
    \definition*{s.}{Sobrenome Fu}
    \definition{adj.}{rico; abastado; abundante; refere-se a ter muito dinheiro (oposto de 贫) | rico; abundante}
    \definition{v.}{tornar-se rico; enriquecer}
  \seealsoref{贫}{pin2}
  \end{Phonetics}
\end{Entry}

\begin{Entry}{富人}{12,2}{⼧、⼈}
  \begin{Phonetics}{富人}{fu4 ren2}[][HSK 6]
    \definition{s.}{os ricos; os abastados}
  \end{Phonetics}
\end{Entry}

\begin{Entry}{富有}{12,6}{⼧、⽉}
  \begin{Phonetics}{富有}{fu4 you3}[][HSK 6]
    \definition{adj.}{rico; abastado; possuir uma grande quantidade de propriedades | rico em espírito; metáfora para uma vida espiritual rica}
    \definition{v.}{ser rico ou abundante em; principalmente referindo-se a coisas abstratas com significados positivos que são suficientes}
  \end{Phonetics}
\end{Entry}

\begin{Entry}{富含}{12,7}{⼧、⼝}
  \begin{Phonetics}{富含}{fu4han2}[][HSK 7-9]
    \definition{v.}{ser rico em}[橘子富含维生素。===As laranjas são ricas em vitaminas.]
  \end{Phonetics}
\end{Entry}

\begin{Entry}{富足}{12,7}{⼧、⾜}
  \begin{Phonetics}{富足}{fu4zu2}[][HSK 7-9]
    \definition{adj.}{rico; pleno; abundante}
  \end{Phonetics}
\end{Entry}

\begin{Entry}{富翁}{12,10}{⼧、⽺}
  \begin{Phonetics}{富翁}{fu4weng1}[][HSK 7-9]
    \definition[个,名,位]{s.}{homem rico; homem de riqueza; pessoas que possuem muitas propriedades}
  \end{Phonetics}
\end{Entry}

\begin{Entry}{富强}{12,12}{⼧、⼸}
  \begin{Phonetics}{富强}{fu4qiang2}[][HSK 7-9]
    \definition{adj.}{próspero e forte; próspero e poderoso; rico e poderoso; (país) rico e poderoso}
  \end{Phonetics}
\end{Entry}

\begin{Entry}{富裕}{12,12}{⼧、⾐}
  \begin{Phonetics}{富裕}{fu4yu4}[][HSK 7-9]
    \definition{adj.}{rico; próspero; abastado; em boas condições econômicas e com dinheiro suficiente}
  \end{Phonetics}
\end{Entry}

\begin{Entry}{富豪}{12,14}{⼧、⾗}
  \begin{Phonetics}{富豪}{fu4hao2}[][HSK 7-9]
    \definition{s.}{rico; pessoa com muito dinheiro e grande poder}
  \end{Phonetics}
\end{Entry}

\begin{Entry}{寒}{12}{⼧}
  \begin{Phonetics}{寒}{han2}
    \definition*{s.}{Sobrenome Han}
    \definition{adj.}{frio | pobre; necessitado | (autodepreciativo) meu/minha humilde\dots | assustado; medroso | com medo; tremendo (de medo) | humilde}
    \definition{s.}{estação fria; inverno (oposto a 暑) | (medicina chinesa) sintomas causados por fatores frios}
  \seealsoref{暑}{shu3}
  \end{Phonetics}
\end{Entry}

\begin{Entry}{寒冷}{12,7}{⼧、⼎}
  \begin{Phonetics}{寒冷}{han2 leng3}[][HSK 4]
    \definition[度,阵,股]{adj.}{frio; frígido; gélido; gelado}
  \end{Phonetics}
\end{Entry}

\begin{Entry}{寒假}{12,11}{⼧、⼈}
  \begin{Phonetics}{寒假}{han2jia4}[][HSK 4]
    \definition[个,段]{s.}{férias de inverno (feriados); férias escolares no meio do inverno, em janeiro e fevereiro (na China)}
  \end{Phonetics}
\end{Entry}

\begin{Entry}{寓}{12}{⼧}
  \begin{Phonetics}{寓}{yu4}
    \definition[座,间,栋]{s.}{residência; morada}
    \definition{v.}{(literário) residir; viver | implicar; conter}
  \end{Phonetics}
\end{Entry}

\begin{Entry}{寓意}{12,13}{⼧、⼼}
  \begin{Phonetics}{寓意}{yu4yi4}
    \definition{s.}{moral (de uma história),  lição a ser aprendida, implicação, mensagem, significado metafórico}
  \end{Phonetics}
\end{Entry}

\begin{Entry}{尊}{12}{⼨}
  \begin{Phonetics}{尊}{zun1}
    \definition*{s.}{Sobrenome Zun}
    \definition{adj.}{sênior; de uma geração sênior; alto status ou antiguidade}
    \definition{clas.}{usado para estátuas, canhões, etc.}
    \definition{pron.}{seu; vossa; antigamente, referia-se a pessoas ou coisas relacionadas entre si}
    \definition{s.}{um tipo de recipiente para vinho usado nos tempos antigos}
    \definition{v.}{respeitar; reverenciar; venerar; honrar}
  \end{Phonetics}
\end{Entry}

\begin{Entry}{尊重}{12,9}{⼨、⾥}
  \begin{Phonetics}{尊重}{zun1zhong4}[][HSK 5]
    \definition{adj.}{sério; adequado; correto; (linguagem, comportamento) não ser descuidado; não ser leviano}
    \definition{v.}{respeitar; valorizar; estimar; tratar com educação; valorizar | tratar com seriedade; levar a sério e tratar com seriedade}
  \end{Phonetics}
\end{Entry}

\begin{Entry}{尊敬}{12,12}{⼨、⽁}
  \begin{Phonetics}{尊敬}{zun1jing4}[][HSK 5]
    \definition{adj.}{respeitoso; respeitável}
    \definition{v.}{respeitar; honrar; estimar}
  \end{Phonetics}
\end{Entry}

\begin{Entry}{就}{12}{⼪}
  \begin{Phonetics}{就}{jiu4}[][HSK 1]
    \definition{adv.}{de imediato; imediatamente; indica que algo ocorrerá em breve | tão cedo quanto; já; há muito tempo; indica que a ação ocorreu há muito tempo | assim que; logo depois; indica que os eventos se sucedem imediatamente | nesse caso; então; indica que, sob determinadas condições, ocorre naturalmente um determinado resultado | exatamente; precisamente; indica que é exatamente assim | apenas; meramente; somente | tantos quanto; enfatiza a quantidade | apenas; simplesmente; reforço da afirmação | colocado entre dois componentes idênticos, significa tolerância ou indiferença}
    \definition{prep.}{tirar proveito de alguém (algo); expressa condições, oportunidades, etc., equivalente a 趁 | quando se trata de alguém (algo); relativo a; com relação a; sobre; objeto ou escopo da introdução da ação |no local; introduz o local próximo ao qual a ação ocorreu}
    \definition{v.}{ser comido com; ir com; pratos, frutas, etc., acompanhados de alimentos básicos ou bebidas alcoólicas | aproximar-se; mover-se em direção a | ir para; assumir; empreender; envolver-se em; entrar em | realizar; fazer | tirar proveito de; acomodar-se a; adequar-se; encaixar-se | assumir; começar a entrar ou a exercer | seguir; acompanhar}
  \seealsoref{趁}{chen4}
  \end{Phonetics}
\end{Entry}

\begin{Entry}{就业}{12,5}{⼪、⼀}
  \begin{Phonetics}{就业}{jiu4/ye4}[][HSK 3]
    \definition{v.+compl.}{conseguir um emprego; obter emprego; assumir uma ocupação; começar a trabalhar}
  \end{Phonetics}
\end{Entry}

\begin{Entry}{就可以了}{12,5,4,2}{⼪、⼝、⼈、⼅}
  \begin{Phonetics}{就可以了}{jiu4 ke3yi3le5}
    \definition{expr.}{é isso; é o suficiente}
  \end{Phonetics}
\end{Entry}

\begin{Entry}{就是}{12,9}{⼪、⽇}
  \begin{Phonetics}{就是}{jiu4 shi4}[][HSK 3]
    \definition{adv.}{exatamente; precisamente; expressar concordância com a afirmação da outra pessoa ou confirmar que a afirmação da outra pessoa está correta | apenas; simplesmente; expressa afirmação, determinação ou ênfase, o significado específico deve ser determinado com base no contexto anterior ou posterior | usado para indicar escolha}
    \definition{conj.}{ainda que; mesmo que se reconheça que essa situação é verdadeira, a situação posterior não mudará}
    \definition{part.}{usado no final de uma frase para expressar afirmação}
  \end{Phonetics}
\end{Entry}

\begin{Entry}{就是说}{12,9,9}{⼪、⽇、⾔}
  \begin{Phonetics}{就是说}{jiu4 shi4 shuo1}[][HSK 6]
    \definition{interj.}{ou seja; isto é; em outras palavras; é frequentemente usado como uma interjeição em uma frase para indicar que as palavras seguintes são uma explicação ou esclarecimento das anteriores}
  \end{Phonetics}
\end{Entry}

\begin{Entry}{就要}{12,9}{⼪、⾑}
  \begin{Phonetics}{就要}{jiu4 yao4}[][HSK 2]
    \definition{adv.}{estar prestes a; estar indo para; estar no ponto de}
  \end{Phonetics}
\end{Entry}

\begin{Entry}{就职}{12,11}{⼪、⽿}
  \begin{Phonetics}{就职}{jiu4zhi2}
    \definition{v.}{assumir o cargo | assumir um posto}
  \end{Phonetics}
\end{Entry}

\begin{Entry}{就算}{12,14}{⼪、⽵}
  \begin{Phonetics}{就算}{jiu4 suan4}[][HSK 6]
    \definition{conj.}{mesmo que; concedido que; expressam uma relação hipotética e concessiva, frequentemente usadas com 也, equivalente a 即使}
  \seealsoref{即使}{ji2shi3}
  \seealsoref{也}{ye3}
  \end{Phonetics}
\end{Entry}

\begin{Entry}{属}{12}{⼫}
  \begin{Phonetics}{属}{shu3}[][HSK 3]
    \definition{s.}{categoria | gênero | membros da família; dependentes; familiares; parentes}
    \definition{v.}{estar sob; subordinado a | pertencer a | nascer no ano de (um dos doze animais do zodíaco)}
  \end{Phonetics}
  \begin{Phonetics}{属}{zhu3}
    \definition{v.}{juntar; combinar | fixar (a mente) em; centrar (a atenção, etc.) em}
  \end{Phonetics}
\end{Entry}

\begin{Entry}{属于}{12,3}{⼫、⼆}
  \begin{Phonetics}{属于}{shu3yu2}[][HSK 3]
    \definition{v.}{pertencer a; fazer parte de; pertencer ou ser propriedade de uma determinada parte}
  \end{Phonetics}
\end{Entry}

\begin{Entry}{屡}{12}{⼫}
  \begin{Phonetics}{屡}{lv3}
    \definition{adv.}{uma e outra vez; repetidamente | frequentemente}
  \end{Phonetics}
\end{Entry}

\begin{Entry}{屡次}{12,6}{⼫、⽋}
  \begin{Phonetics}{屡次}{lv3ci4}
    \definition{adv.}{repetidamente | uma e outra vez | muitas vezes}
  \end{Phonetics}
\end{Entry}

\begin{Entry}{帽}{12}{⼱}
  \begin{Phonetics}{帽}{mao4}
    \definition[个,顶]{s.}{chapéu; boné | capa; uma coisa que cobre um objeto e tem a função ou formato de um chapéu | elmo; capacete}
  \end{Phonetics}
\end{Entry}

\begin{Entry}{帽子}{12,3}{⼱、⼦}
  \begin{Phonetics}{帽子}{mao4zi5}[][HSK 4]
    \definition[顶,个,种]{s.}{boné; chapéu; capacete | etiqueta; rótulo; marca}
  \end{Phonetics}
\end{Entry}

\begin{Entry}{幅}{12}{⼱}
  \begin{Phonetics}{幅}{fu2}[][HSK 5]
    \definition{clas.}{usado para tecidos, telas de lã, pinturas, etc.}
    \definition{s.}{largura do tecido, seda, tweed, etc. | tamanho; largura; geralmente se refere à largura}
  \end{Phonetics}
\end{Entry}

\begin{Entry}{幅度}{12,9}{⼱、⼴}
  \begin{Phonetics}{幅度}{fu2du4}[][HSK 5]
    \definition{s.}{alcance; escopo; extensão; largura; largura da propagação de um objeto que vibra ou balança, uma metáfora para a magnitude de uma mudança em algo}
  \end{Phonetics}
\end{Entry}

\begin{Entry}{强}{12}{⼸}
  \begin{Phonetics}{强}{jiang4}
    \definition{adj.}{teimoso; inflexível}
  \end{Phonetics}
  \begin{Phonetics}{强}{qiang2}[][HSK 3]
    \definition*{s.}{Sobrenome Qiang}
    \definition{adj.}{forte; poderoso  (em oposição a 弱) | melhor; superior | mais; extra; adicional; um pouco mais que; usado após uma fração ou decimal para indicar que é um pouco maior que o número | resoluto; firme | violento | alto padrão}
    \definition{v.}{fortalecer; tornar forte; tornar poderoso}
  \seealsoref{弱}{ruo4}
  \end{Phonetics}
  \begin{Phonetics}{强}{qiang3}
    \definition{v.}{fazer um esforço; esforçar-se}
  \end{Phonetics}
\end{Entry}

\begin{Entry}{强大}{12,3}{⼸、⼤}
  \begin{Phonetics}{强大}{qiang2 da4}[][HSK 3]
    \definition{adj.}{forte; poderoso; potente; possante; descreve força forte e grande poder}
  \end{Phonetics}
\end{Entry}

\begin{Entry}{强化}{12,4}{⼸、⼔}
  \begin{Phonetics}{强化}{qiang2 hua4}[][HSK 6]
    \definition{v.}{intensificar; fortalecer; consolidar; tornar mais forte, melhorar sua habilidade e nível}
  \end{Phonetics}
\end{Entry}

\begin{Entry}{强壮}{12,6}{⼸、⼠}
  \begin{Phonetics}{强壮}{qiang2 zhuang4}[][HSK 6]
    \definition{s.}{(corpo) forte, poderoso, robusto, resistente}
    \definition{v.}{fortalecer; construir}
  \end{Phonetics}
\end{Entry}

\begin{Entry}{强势}{12,8}{⼸、⼒}
  \begin{Phonetics}{强势}{qiang2 shi4}[][HSK 6]
    \definition*{adj.}{forte; poderoso; dominante}
    \definition{s.}{momento; ímpeto; grande impulso; forte impulso | força; influência dominante; forças poderosas}
  \end{Phonetics}
\end{Entry}

\begin{Entry}{强迫}{12,8}{⼸、⾡}
  \begin{Phonetics}{强迫}{qiang3po4}[][HSK 5]
    \definition{v.}{impelir; forçar; impor; compelir; aplicar pessão para obedecer}
  \end{Phonetics}
\end{Entry}

\begin{Entry}{强度}{12,9}{⼸、⼴}
  \begin{Phonetics}{强度}{qiang2 du4}[][HSK 5]
    \definition[个,种]{s.}{intensidade; força | magnitude; rigor; avidez}
  \end{Phonetics}
\end{Entry}

\begin{Entry}{强烈}{12,10}{⼸、⽕}
  \begin{Phonetics}{强烈}{qiang2lie4}[][HSK 3]
    \definition{adj.}{muito forte; intenso; poderoso | violento; impetuoso; nível muito alto; atitude muito firme, sem espaço para mudanças | afiado; marcante; mostrado em contraste; muito claro}
  \end{Phonetics}
\end{Entry}

\begin{Entry}{强调}{12,10}{⼸、⾔}
  \begin{Phonetics}{强调}{qiang2diao4}[][HSK 3]
    \definition{v.}{salientar; sublinhar; enfatizar; dar ênfase a; vincar}
  \end{Phonetics}
\end{Entry}

\begin{Entry}{强盗}{12,11}{⼸、⽫}
  \begin{Phonetics}{强盗}{qiang2 dao4}[][HSK 6]
    \definition[个,群,伙,帮]{s.}{ladrão; bandido; uma pessoa que usa violência para confiscar a propriedade de outros; também se refere a uma pessoa ou força que se envolve em comportamento semelhante}
  \end{Phonetics}
\end{Entry}

\begin{Entry}{循}{12}{⼻}
  \begin{Phonetics}{循}{xun2}
    \definition{v.}{seguir; cumprir; cumprir com}
  \end{Phonetics}
\end{Entry}

\begin{Entry}{循环}{12,8}{⼻、⽟}
  \begin{Phonetics}{循环}{xun2huan2}[][HSK 6]
    \definition{s.}{ciclo; circulação}
    \definition{v.}{circular; as coisas se movem ou mudam em um ciclo}
  \end{Phonetics}
\end{Entry}

\begin{Entry}{悲}{12}{⽕}
  \begin{Phonetics}{悲}{bei1}
    \definition{adj.}{triste; pesaroso; melancólico | compassivo; misericordioso}
  \end{Phonetics}
\end{Entry}

\begin{Entry}{悲伤}{12,6}{⽕、⼈}
  \begin{Phonetics}{悲伤}{bei1 shang1}[][HSK 5]
    \definition{adj.}{triste; pesaroso}
  \end{Phonetics}
\end{Entry}

\begin{Entry}{悲欢离合}{12,6,10,6}{⽕、⽋、⼇、⼝}
  \begin{Phonetics}{悲欢离合}{bei1huan1-li2he2}[][HSK 7-9]
    \definition{expr.}{alegrias e tristezas | separações e reencontros | as vicissitudes da vida}
  \end{Phonetics}
\end{Entry}

\begin{Entry}{悲观}{12,6}{⽕、⾒}
  \begin{Phonetics}{悲观}{bei1guan1}[][HSK 7-9]
    \definition{adj.}{pessimista; negativismo, falta de confiança no futuro (oposto a 乐观)}
  \seealsoref{乐观}{le4guan1}
  \end{Phonetics}
\end{Entry}

\begin{Entry}{悲哀}{12,9}{⽕、⼝}
  \begin{Phonetics}{悲哀}{bei1'ai1}[][HSK 7-9]
    \definition{adj.}{triste}
    \definition{s.}{tristeza; refere-se a coisas tristes e dolorosas}
  \end{Phonetics}
\end{Entry}

\begin{Entry}{悲剧}{12,10}{⽕、⼑}
  \begin{Phonetics}{悲剧}{bei1 ju4}[][HSK 5]
    \definition[出,部]{s.}{tragédia; drama trágico; uma das principais categorias de teatro, caracterizada basicamente pela representação do conflito irreconciliável entre o protagonista e a realidade e seu final trágico | tragédia; evento triste; metáfora para encontro infeliz}
  \end{Phonetics}
\end{Entry}

\begin{Entry}{悲惨}{12,11}{⽕、⽕}
  \begin{Phonetics}{悲惨}{bei1can3}[][HSK 6]
    \definition{adj.}{trágico; miserável; extremamente doloroso e triste}
  \end{Phonetics}
\end{Entry}

\begin{Entry}{悲痛}{12,12}{⽕、⽧}
  \begin{Phonetics}{悲痛}{bei1tong4}[][HSK 7-9]
    \definition{adj.}{triste; aflito}
    \definition{s.}{tristeza; sofrimento}
  \end{Phonetics}
\end{Entry}

\begin{Entry}{惑}{12}{⼼}
  \begin{Phonetics}{惑}{huo4}
    \definition{v.}{ficar confuso; ficar perplexo | iludir; enganar; confundir}
  \end{Phonetics}
\end{Entry}

\begin{Entry}{惑星}{12,9}{⼼、⽇}
  \begin{Phonetics}{惑星}{huo4xing1}
    \definition{s.}{planeta}
  \seealsoref{行星}{xing2xing1}
  \end{Phonetics}
\end{Entry}

\begin{Entry}{惩}{12}{⼼}
  \begin{Phonetics}{惩}{cheng2}
    \definition{v.}{receber ou dar aviso | punir; penalizar}
  \end{Phonetics}
\end{Entry}

\begin{Entry}{惩处}{12,5}{⼼、⼡}
  \begin{Phonetics}{惩处}{cheng2chu3}[][HSK 7-9]
    \definition{v.}{penalizar; punir | punir; administrar justiça}
  \end{Phonetics}
\end{Entry}

\begin{Entry}{惩罚}{12,9}{⼼、⽹}
  \begin{Phonetics}{惩罚}{cheng2fa2}[][HSK 7-9]
    \definition[次,种]{s.}{punição; o ato ou método de punição}
    \definition{v.}{punir (severamente); penalizar}
  \end{Phonetics}
\end{Entry}

\begin{Entry}{惶}{12}{⼼}
  \begin{Phonetics}{惶}{huang2}
    \definition{adj.}{cheio de medo; assustado}
    \definition{s.}{medo; pânico}
    \definition{v.}{temer}
  \end{Phonetics}
\end{Entry}

\begin{Entry}{惶恐}{12,10}{⼼、⼼}
  \begin{Phonetics}{惶恐}{huang2kong3}
    \definition{adj.}{aterrorizado; em pânico; petrificado | inquieto; apreensivo}
  \end{Phonetics}
\end{Entry}

\begin{Entry}{愉}{12}{⼼}
  \begin{Phonetics}{愉}{yu2}
    \definition{adj.}{satisfeito; feliz; alegre}
  \end{Phonetics}
\end{Entry}

\begin{Entry}{愉快}{12,7}{⼼、⼼}
  \begin{Phonetics}{愉快}{yu2kuai4}[][HSK 6]
    \definition{adj.}{feliz; alegre; de bom humor, muito feliz}
  \end{Phonetics}
\end{Entry}

\begin{Entry}{愤}{12}{⼼}
  \begin{Phonetics}{愤}{fen4}
    \definition{s.}{raiva; indignação; ressentimento; exasperação}
    \definition{v.}{ressentir-se; ficar indignado; ficar com raiva}
  \end{Phonetics}
\end{Entry}

\begin{Entry}{愤世嫉俗}{12,5,13,9}{⼼、⼀、⼥、⼈}
  \begin{Phonetics}{愤世嫉俗}{fen4shi4ji2su2}
    \definition{v.}{ser cínico | ser amargurado}
  \end{Phonetics}
\end{Entry}

\begin{Entry}{愤怒}{12,9}{⼼、⼼}
  \begin{Phonetics}{愤怒}{fen4nu4}[][HSK 6]
    \definition{adj.}{zangado; enraivecido; iracundo; furioso; emocionalmente agitado por extrema insatisfação}
  \end{Phonetics}
\end{Entry}

\begin{Entry}{慌}{12}{⼼}
  \begin{Phonetics}{慌}{huang1}[][HSK 5]
    \definition{adj.}{agitado; perturbado; confuso; que inspira terror}
    \definition{v.}{estar em estado de pânico; ficar com medo; ficar nervoso | estar com pressa}
  \end{Phonetics}
\end{Entry}

\begin{Entry}{慌忙}{12,6}{⼼、⼼}
  \begin{Phonetics}{慌忙}{huang1 mang2}[][HSK 5]
    \definition{adj.}{apressado; afobado; com muita pressa}
    \definition{adv.}{apressadamente}
  \end{Phonetics}
\end{Entry}

\begin{Entry}{慌乱}{12,7}{⼼、⼄}
  \begin{Phonetics}{慌乱}{huang1luan4}[][HSK 7-9]
    \definition{adj.}{agitado; alarmado e confuso; em pânico e ocupado}
  \end{Phonetics}
\end{Entry}

\begin{Entry}{慌张}{12,7}{⼼、⼸}
  \begin{Phonetics}{慌张}{huang1zhang1}[][HSK 7-9]
    \definition{adj.}{em pânico; agitado; perturbado; confuso}
  \end{Phonetics}
\end{Entry}

\begin{Entry}{掌}{12}{⼿}
  \begin{Phonetics}{掌}{zhang3}
    \definition{s.}{palma da mão | sola do pé | pata | ferradura}
    \definition{v.}{dar um tapa | segurar na mão | empunhar}
  \end{Phonetics}
\end{Entry}

\begin{Entry}{掌声}{12,7}{⼿、⼠}
  \begin{Phonetics}{掌声}{zhang3 sheng1}[][HSK 6]
    \definition[阵]{s.}{aplausos; palmas; o som dos aplausos}
  \end{Phonetics}
\end{Entry}

\begin{Entry}{掌握}{12,12}{⼿、⼿}
  \begin{Phonetics}{掌握}{zhang3wo4}[][HSK 5]
    \definition{v.}{compreender; dominar; conhecer bem; compreender as coisas; ser capaz de dominar ou utilizar plenamente | segurar; controlar; ter em mãos; tomar nas mãos}
  \end{Phonetics}
\end{Entry}

\begin{Entry}{掰}{12}{⼿}
  \begin{Phonetics}{掰}{bai1}[][HSK 7-9]
    \definition{v.}{separar ou quebrar coisas com as mãos | Dialeto: romper (relacionamento); cortar | Dialeto: analisar; estudar; examinar}
  \end{Phonetics}
\end{Entry}

\begin{Entry}{掱}{12}{⼿}
  \begin{Phonetics}{掱}{shou3}
    \variantof{手}
  \end{Phonetics}
\end{Entry}

\begin{Entry}{揉}{12}{⼿}
  \begin{Phonetics}{揉}{rou2}
    \definition{v.}{amassar | massagear | esfregar}
  \end{Phonetics}
\end{Entry}

\begin{Entry}{揉碎}{12,13}{⼿、⽯}
  \begin{Phonetics}{揉碎}{rou2sui4}
    \definition{v.}{esmagar | desintegrar-se em pedaços}
  \end{Phonetics}
\end{Entry}

\begin{Entry}{提}{12}{⼿}
  \begin{Phonetics}{提}{ti2}[][HSK 2]
    \definition*{s.}{Sobrenome Ti}
    \definition{s.}{concha; utensílio para servir óleo ou vinho | traço ascendente (em caracteres chineses)}
    \definition{v.}{carregar (na mão, com o braço para baixo) ; segurar com as mãos para baixo | elevar; levantar; promover | avançar; antecipar uma data; mudar para uma data anterior; adiar o prazo previsto | levantar; apresentar; indicar ou citar | extrair; retirar (tirar) | (prisioneiros) trazer; entregar | mencionar; referir-se a; abordar}
  \end{Phonetics}
\end{Entry}

\begin{Entry}{提及}{12,3}{⼿、⼃}
  \begin{Phonetics}{提及}{ti2ji2}
    \definition{v.}{mencionar | levantar (um assunto) | chamar a atenção de alguém}
  \end{Phonetics}
\end{Entry}

\begin{Entry}{提升}{12,4}{⼿、⼗}
  \begin{Phonetics}{提升}{ti2 sheng1}[][HSK 6]
    \definition{v.}{promover; avançar; melhorar (posição, grau, qualidade, etc.) | içar; elevar; transportar (minerais, materiais, etc.) para um local mais alto usando um guincho, etc.}
  \end{Phonetics}
\end{Entry}

\begin{Entry}{提出}{12,5}{⼿、⼐}
  \begin{Phonetics}{提出}{ti2 chu1}[][HSK 2]
    \definition{v.}{levantar; propor; apresentar; expressar seus desejos, ideias, sugestões, etc. por meio de palavras ou textos}
  \end{Phonetics}
\end{Entry}

\begin{Entry}{提示}{12,5}{⼿、⽰}
  \begin{Phonetics}{提示}{ti2shi4}[][HSK 5]
    \definition[个]{s.}{dica; lembrete; pistas ou informações fornecidas para chamar a atenção, fazer com que a outra pessoa pense ou compreenda}
    \definition{v.}{solicitar; lembrar; indicar; alertar; levantar questões que o outro não tenha pensado ou não tenha imaginado, para chamar a atenção dele}
  \end{Phonetics}
\end{Entry}

\begin{Entry}{提交}{12,6}{⼿、⼇}
  \begin{Phonetics}{提交}{ti2 jiao1}[][HSK 6]
    \definition{v.}{referir-se a; submeter (um problema, etc.) a; enviar questões que precisam ser discutidas, decididas ou tratadas para agências ou reuniões relevantes}
  \end{Phonetics}
\end{Entry}

\begin{Entry}{提问}{12,6}{⼿、⾨}
  \begin{Phonetics}{提问}{ti2wen4}[][HSK 3]
    \definition{v.}{\emph{quiz}; fazer uma pergunta; colocar questões para}
  \end{Phonetics}
\end{Entry}

\begin{Entry}{提防}{12,6}{⼿、⾩}
  \begin{Phonetics}{提防}{di1fang5}[][HSK 7-9]
    \definition{v.}{proteger-se contra; ter cuidado com; tomar precauções contra; tomar cuidado}
  \end{Phonetics}
\end{Entry}

\begin{Entry}{提供}{12,8}{⼿、⼈}
  \begin{Phonetics}{提供}{ti2gong1}[][HSK 4]
    \definition{v.}{oferecer; fornecer; suprir; prover; proporcionar}
  \end{Phonetics}
\end{Entry}

\begin{Entry}{提到}{12,8}{⼿、⼑}
  \begin{Phonetics}{提到}{ti2 dao4}[][HSK 2]
    \definition{v.}{mencionar; referir-se a; levantar (assunto)}
  \end{Phonetics}
\end{Entry}

\begin{Entry}{提前}{12,9}{⼿、⼑}
  \begin{Phonetics}{提前}{ti2qian2}[][HSK 3]
    \definition{adv.}{antecipadamente; faça uma coisa antes de fazer outra}
    \definition{v.}{avançar; adiantar; mudar para uma data anterior; trazer para frente}
  \end{Phonetics}
\end{Entry}

\begin{Entry}{提倡}{12,10}{⼿、⼈}
  \begin{Phonetics}{提倡}{ti2chang4}[][HSK 5]
    \definition{v.}{promover; incentivar; recomendar; apresentar as vantagens de algo para incentivar as pessoas a usá-lo ou implementá-lo}
  \end{Phonetics}
\end{Entry}

\begin{Entry}{提起}{12,10}{⼿、⾛}
  \begin{Phonetics}{提起}{ti2 qi3}[][HSK 5]
    \definition{v.}{mencionar; falar sobre; abordar | levantar; despertar; estimular; revigorar; alegrar/animar | iniciar; instituir; propor | levantar; pegar}
  \end{Phonetics}
\end{Entry}

\begin{Entry}{提高}{12,10}{⼿、⾼}
  \begin{Phonetics}{提高}{ti2gao1}[][HSK 2]
    \definition{v.}{elevar; aprimorar; aumentar; melhorar a posição, o nível, a quantidade, a qualidade e outros aspectos em relação ao estado original}
  \end{Phonetics}
\end{Entry}

\begin{Entry}{提醒}{12,16}{⼿、⾣}
  \begin{Phonetics}{提醒}{ti2/xing3}[][HSK 4]
    \definition{v.+compl.}{alertar; avisar; advertir; lembrar; apontar para ou chamar a atenção para}
  \end{Phonetics}
\end{Entry}

\begin{Entry}{插}{12}{⼿}
  \begin{Phonetics}{插}{cha1}[][HSK 5]
    \definition{v.}{perfurar; inserir | interpor; inserir; colocar no meio}
  \end{Phonetics}
\end{Entry}

\begin{Entry}{插手}{12,4}{⼿、⼿}
  \begin{Phonetics}{插手}{cha1/shou3}[][HSK 7-9]
    \definition{v.+compl.}{participar; dar uma mão | meter a mão em; meter o nariz em; intrometer-se | ter (tomar) uma mão em}
  \end{Phonetics}
\end{Entry}

\begin{Entry}{插图}{12,8}{⼿、⼞}
  \begin{Phonetics}{插图}{cha1tu2}[][HSK 7-9]
    \definition[张,幅]{s.}{ilustração (artística ou científica) | ilustração; figura; mapa; demonstração; inserção}
  \end{Phonetics}
\end{Entry}

\begin{Entry}{插话}{12,8}{⼿、⾔}
  \begin{Phonetics}{插话}{cha1/hua4}
    \definition{s.}{interrupção | digressão}
    \definition{v.+compl.}{interromper (a fala de alguém)}
  \end{Phonetics}
\end{Entry}

\begin{Entry}{插嘴}{12,16}{⼿、⼝}
  \begin{Phonetics}{插嘴}{cha1/zui3}[][HSK 7-9]
    \definition{v.+compl.}{interromper; intrometer-se; participar da conversa (geralmente de forma inadequada)}
  \end{Phonetics}
\end{Entry}

\begin{Entry}{握}{12}{⼿}
  \begin{Phonetics}{握}{wo4}[][HSK 5]
    \definition{v.}{segurar; agarrar | agarrar; segurar; empunhar; controlar | pegar pela mão}
  \end{Phonetics}
\end{Entry}

\begin{Entry}{握手}{12,4}{⼿、⼿}
  \begin{Phonetics}{握手}{wo4/shou3}[][HSK 3]
    \definition{v.+compl.}{apertar as mãos; dar um aperto de mão; estender a mão e apertar a mão do outro é uma forma de saudação ao se encontrar ou se despedir, e também é usado para expressar felicitações ou condolências}
  \end{Phonetics}
\end{Entry}

\begin{Entry}{揣}{12}{⼿}
  \begin{Phonetics}{揣}{chuai1}[][HSK 7-9]
    \definition{v.}{esconder (ou carregar) nas roupas | Dialeto: encher-se de comida; comer demais; encher alguém de comida; alimentar em excesso}
  \end{Phonetics}
  \begin{Phonetics}{揣}{chuai3}
    \definition*{s.}{Sobrenome Chuai}
    \definition{v.}{contar; calcular; medir | estimar; palpitar; conjecturar}
  \end{Phonetics}
\end{Entry}

\begin{Entry}{揣测}{12,9}{⼿、⽔}
  \begin{Phonetics}{揣测}{chuai3ce4}[][HSK 7-9]
    \definition{v.}{adivinhar; conjecturar | supor; calcular; especular}
  \end{Phonetics}
\end{Entry}

\begin{Entry}{揣摩}{12,15}{⼿、⼿}
  \begin{Phonetics}{揣摩}{chuai3mo2}[][HSK 7-9]
    \definition{v.}{tentar compreender; tentar descobrir; obter algo por meio de estudo cuidadoso; pesar e considerar}
  \end{Phonetics}
\end{Entry}

\begin{Entry}{揭}{12}{⼿}
  \begin{Phonetics}{揭}{jie1}[][HSK 6]
    \definition*{s.}{Sobrenome Jie}
    \definition{v.}{rasgar; arrancar; tirar | descobrir; levantar (a tampa, etc.) | expor; mostrar; trazer à luz | (literário) levantar; içar}
  \end{Phonetics}
\end{Entry}

\begin{Entry}{援}{12}{⼿}
  \begin{Phonetics}{援}{yuan2}
    \definition*{s.}{Sobrenome Yuan}
    \definition{v.}{puxar com a mão; segurar | citar; referenciar | ajudar; auxiliar; resgatar}
  \end{Phonetics}
\end{Entry}

\begin{Entry}{援助}{12,7}{⼿、⼒}
  \begin{Phonetics}{援助}{yuan2 zhu4}[][HSK 6]
    \definition{s.}{ajuda; assistência; auxílio}
    \definition{v.}{ajudar; apoiar; auxiliar}
  \end{Phonetics}
\end{Entry}

\begin{Entry}{搀}{12}{⼿}
  \begin{Phonetics}{搀}{chan1}[][HSK 7-9]
    \definition{v.}{apoiar alguém pelo braço; apoiar alguém com a mão; apoiar | misturar}
  \end{Phonetics}
\end{Entry}

\begin{Entry}{搁}{12}{⼿}
  \begin{Phonetics}{搁}{ge1}[][HSK 7-9]
    \definition{v.}{pôr; colocar | colocar à parte; deixar para trás; deixar para mais tarde| deixar de lado}
  \end{Phonetics}
  \begin{Phonetics}{搁}{ge2}
    \definition{v.}{suportar; resistir}
  \end{Phonetics}
\end{Entry}

\begin{Entry}{搁浅}{12,8}{⼿、⽔}
  \begin{Phonetics}{搁浅}{ge1/qian3}[][HSK 7-9]
    \definition{v.+compl.}{ficar encalhado (navio); encalhar | ser retido; chegar a um impasse; metaforicamente, algo está bloqueado e não pode prosseguir}
  \end{Phonetics}
\end{Entry}

\begin{Entry}{搁置}{12,13}{⼿、⽹}
  \begin{Phonetics}{搁置}{ge1zhi4}[][HSK 7-9]
    \definition{v.}{arquivar; deixar de lado; suspender; classificar; deitar; adiar; colocar na prateleira}
  \end{Phonetics}
\end{Entry}

\begin{Entry}{搓}{12}{⼿}
  \begin{Phonetics}{搓}{cuo1}[][HSK 7-9]
    \definition{s.}{torção}
    \definition{v.}{esfregar ou rolar entre as mãos ou dedos |  (no tênis, tênis de mesa, críquete, etc.) cortar | (roupa, etc.) torcer}
  \end{Phonetics}
\end{Entry}

\begin{Entry}{搜}{12}{⼿}
  \begin{Phonetics}{搜}{sou1}[][HSK 5]
    \definition{v.}{procurar | pesquisar | coletar; reunir | procurar ou revistar um lugar de forma completa e desordenada}
  \end{Phonetics}
\end{Entry}

\begin{Entry}{搜索}{12,10}{⼿、⽷}
  \begin{Phonetics}{搜索}{sou1suo3}[][HSK 5]
    \definition{v.}{procurar; caçar; explorar; pesquisar cuidadosamente; refere-se especificamente à busca militar para identificar situações suspeitas em determinada região, área marítima ou aérea}
  \end{Phonetics}
\end{Entry}

\begin{Entry}{搭}{12}{⼿}
  \begin{Phonetics}{搭}{da1}[][HSK 6]
    \definition{v.}{colocar em prática; construir | ficar pendurado; colocar para cima | entrar em contato; juntar-se | adicionar (mais pessoas, dinheiro, etc.) | levantar algo junto |
pegar (um navio, avião, etc.); viajar (ou ir) por}
    \variantof{褡}
  \end{Phonetics}
\end{Entry}

\begin{Entry}{搭讪}{12,5}{⼿、⾔}
  \begin{Phonetics}{搭讪}{da1shan4}
    \definition{v.}{bater em alguém | incitar uma conversa | começar a conversar para acabar com um silêncio constrangedor ou uma situação embaraçosa}
  \end{Phonetics}
\end{Entry}

\begin{Entry}{搭建}{12,8}{⼿、⼵}
  \begin{Phonetics}{搭建}{da1jian4}[][HSK 7-9]
    \definition{v.}{montar (um galpão, abrigo temporário, etc.) | criar (uma organização) | construir (especialmente com materiais simples) | juntar (um galpão temporário) | armar}
  \end{Phonetics}
\end{Entry}

\begin{Entry}{搭乘}{12,10}{⼿、⽲}
  \begin{Phonetics}{搭乘}{da1cheng2}[][HSK 7-9]
    \definition{v.}{viajar de (carro, barco, avião etc.)}
  \end{Phonetics}
\end{Entry}

\begin{Entry}{搭档}{12,10}{⼿、⽊}
  \begin{Phonetics}{搭档}{da1dang4}[][HSK 6]
    \definition[个,名,位]{s.}{parceiro; colega de trabalho}
    \definition{v.}{cooperar; trabalhar em conjunto; formar pares; colaborar; formar uma parceria}
  \end{Phonetics}
\end{Entry}

\begin{Entry}{搭配}{12,10}{⼿、⾣}
  \begin{Phonetics}{搭配}{da1pei4}[][HSK 6]
    \definition{v.}{emparelhar; organizar em pares ou grupos; organizar a distribuição de acordo com certos requisitos | encaixar; combinar}
  \end{Phonetics}
\end{Entry}

\begin{Entry}{敞}{12}{⽁}
  \begin{Phonetics}{敞}{chang3}
    \definition{adj.}{espaçoso; aberto; desobstruído | Dialeto: (casa, pátio, etc.) espaçoso}
    \definition{v.}{abrir; descobrir}
  \end{Phonetics}
\end{Entry}

\begin{Entry}{敞开}{12,4}{⽁、⼶}
  \begin{Phonetics}{敞开}{chang3kai1}[][HSK 7-9]
    \definition{adj.}{aberto; irrestrito}
    \definition{adv.}{livremente; sem reservas; ilimitadamente; irrestritamente; totalmente aberto; infinitamente; sem limite}
    \definition{v.}{abrir o máximo possível}
  \end{Phonetics}
\end{Entry}

\begin{Entry}{散}{12}{⽁}
  \begin{Phonetics}{散}{san3}[][HSK 5]
    \definition{adj.}{disperso; fragmentado; não integrado}
    \definition{s.}{medicamento em forma de pó}
    \definition{v.}{divergir; espalhar-se; separar-se; soltar-se; não se manter unido;  desintegrar}
  \end{Phonetics}
  \begin{Phonetics}{散}{san4}
    \definition{v.}{quebrar; fragmentar; dispersar | dar; distribuir; disseminar; divulgar | dissipar; deixar sai  | terminar um acordo ou contrato; demitir}
  \end{Phonetics}
\end{Entry}

\begin{Entry}{散心}{12,4}{⽁、⼼}
  \begin{Phonetics}{散心}{san4/xin1}
    \definition{v.+compl.}{aliviar o tédio | desfrutar de uma diversão | estar despreocupado}
  \end{Phonetics}
\end{Entry}

\begin{Entry}{散文}{12,4}{⽁、⽂}
  \begin{Phonetics}{散文}{san3wen2}[][HSK 5]
    \definition[篇,种]{s.}{ensaio; prosa; gênero literário, na antiguidade, referia-se a textos em prosa, em oposição à poesia e à prosa paralela; atualmente, refere-se a obras literárias que não sejam poesia, teatro ou romance, incluindo ensaios, contos, crônicas, relatos de viagem, etc.}
  \end{Phonetics}
\end{Entry}

\begin{Entry}{散步}{12,7}{⽁、⽌}
  \begin{Phonetics}{散步}{san4/bu4}[][HSK 3]
    \definition{v.+compl.}{dar uma volta; dar um passeio; dar uma caminhada}
  \end{Phonetics}
\end{Entry}

\begin{Entry}{敦}{12}{⽁}
  \begin{Phonetics}{敦}{dui4}
    \definition{s.}{um recipiente tradicional para armazenar arroz ou grãos; utensílios antigos para guardar painço}
  \end{Phonetics}
  \begin{Phonetics}{敦}{dun1}
    \definition*{s.}{Sobrenome Dun}
    \definition{adj.}{honesto; sincero}
  \end{Phonetics}
\end{Entry}

\begin{Entry}{敦促}{12,9}{⽁、⼈}
  \begin{Phonetics}{敦促}{dun1cu4}[][HSK 7-9]
    \definition{v.}{instar; pressionar; urgir}
  \end{Phonetics}
\end{Entry}

\begin{Entry}{敦厚}{12,9}{⽁、⼚}
  \begin{Phonetics}{敦厚}{dun1hou4}[][HSK 7-9]
    \definition{adj.}{genuíno | honesto e sincero}
  \end{Phonetics}
\end{Entry}

\begin{Entry}{敬}{12}{⽁}
  \begin{Phonetics}{敬}{jing4}
    \definition*{s.}{Sobrenome Jing}
    \definition{adj.}{respeitoso; reverente}
    \definition{adv.}{respeitosamente}
    \definition{v.}{respeitar; honrar; estimar | oferecer educadamente | envolver-se em; dedicar-se a}
  \end{Phonetics}
\end{Entry}

\begin{Entry}{敬礼}{12,5}{⽁、⽰}
  \begin{Phonetics}{敬礼}{jing4li3}
    \definition{s.}{saudação}
    \definition{v.}{saudar}
  \end{Phonetics}
\end{Entry}

\begin{Entry}{斑}{12}{⽂}
  \begin{Phonetics}{斑}{ban1}
    \definition{adj.}{manchado; listrado; de cor variegada}
    \definition[块,片,个]{s.}{mancha; pinta; salpico; listra | nódoa; estria; mácula; imperfeição}
  \end{Phonetics}
\end{Entry}

\begin{Entry}{斑点}{12,9}{⽂、⽕}
  \begin{Phonetics}{斑点}{ban1dian3}[][HSK 7-9]
    \definition[块,片,个]{s.}{salpico; cisco; ponto; pinta; sardas | mancha; marca}
  \end{Phonetics}
\end{Entry}

\begin{Entry}{斯}{12}{⽄}
  \begin{Phonetics}{斯}{si1}
    \definition*{s.}{Sobrenome Si}
    \definition{adv.}{então; assim}
    \definition{pron.}{isto; aqui}
  \end{Phonetics}
\end{Entry}

\begin{Entry}{斯巴达}{12,4,6}{⽄、⼰、⾡}
  \begin{Phonetics}{斯巴达}{si1ba1da2}
    \definition*{s.}{Esparta}
  \end{Phonetics}
\end{Entry}

\begin{Entry}{普}{12}{⽇}
  \begin{Phonetics}{普}{pu3}
    \definition*{s.}{Sobrenome Pu}
    \definition{adj.}{geral; universal}
  \end{Phonetics}
\end{Entry}

\begin{Entry}{普及}{12,3}{⽇、⼃}
  \begin{Phonetics}{普及}{pu3ji2}[][HSK 3]
    \definition{adj.}{popular; universal; onipresente; amplamente compreendido, aceito ou utilizado}
    \definition[种]{v.}{popularizar; disseminar; espalhar entre as pessoas; promover amplamente o conhecimento, a educação, a tecnologia, etc. para popularizá-los}
  \end{Phonetics}
\end{Entry}

\begin{Entry}{普通}{12,10}{⽇、⾡}
  \begin{Phonetics}{普通}{pu3 tong1}[][HSK 2]
    \definition{adj.}{comum; normal; geral; médio; em geral, nada de especial, como a maioria das pessoas ou coisas}
  \end{Phonetics}
\end{Entry}

\begin{Entry}{普通话}{12,10,8}{⽇、⾡、⾔}
  \begin{Phonetics}{普通话}{pu3tong1hua4}[][HSK 2]
    \definition*{s.}{Mandarim (literalmente "linguagem comum") | Putonghua (fala comum da língua chinesa) | Língua oficial da China}
  \end{Phonetics}
\end{Entry}

\begin{Entry}{普遍}{12,12}{⽇、⾡}
  \begin{Phonetics}{普遍}{pu3bian4}[][HSK 3]
    \definition{adj.}{geral; comum; universal; difundido; a existência é muito ampla; tem semelhança}
  \end{Phonetics}
\end{Entry}

\begin{Entry}{景}{12}{⽇}
  \begin{Phonetics}{景}{jing3}[][HSK 6]
    \definition*{s.}{Sobrenome Jing}
    \definition{adj.}{grandioso; elevado; grande}
    \definition{s.}{vista; cenário; cena | situação; condição | cenário (de uma peça ou filme) | cena (de uma peça)}
    \definition{v.}{admirar; reverenciar; respeitar}
  \end{Phonetics}
\end{Entry}

\begin{Entry}{景色}{12,6}{⽇、⾊}
  \begin{Phonetics}{景色}{jing3se4}[][HSK 3]
    \definition[片,幅,道,处]{s.}{vista; cena; cenário; paisagem}
  \end{Phonetics}
\end{Entry}

\begin{Entry}{景点}{12,9}{⽇、⽕}
  \begin{Phonetics}{景点}{jing3 dian3}[][HSK 6]
    \definition[个,处]{s.}{local cênico; atração turística; um lugar onde se concentram as atrações turísticas, incluindo atrações naturais e culturais}
  \end{Phonetics}
\end{Entry}

\begin{Entry}{景象}{12,11}{⽇、⾗}
  \begin{Phonetics}{景象}{jing3 xiang4}[][HSK 5]
    \definition[个,种]{s.}{cena; visão; vista; quadro}
  \end{Phonetics}
\end{Entry}

\begin{Entry}{晴}{12}{⽇}
  \begin{Phonetics}{晴}{qing2}[][HSK 2]
    \definition{adj.}{ensolarado; bom; claro; não há nuvens no céu ou há poucas nuvens}
  \end{Phonetics}
\end{Entry}

\begin{Entry}{晴天}{12,4}{⽇、⼤}
  \begin{Phonetics}{晴天}{qing2 tian1}[][HSK 2]
    \definition[个]{s.}{dia ensolarado; tempo sem nuvens ou com poucas nuvens; em meteorologia, refere-se a um tempo em que a cobertura de nuvens no céu é inferior a 10\%}
  \end{Phonetics}
\end{Entry}

\begin{Entry}{晴朗}{12,10}{⽇、⽉}
  \begin{Phonetics}{晴朗}{qing2lang3}[][HSK 5]
    \definition{adj.}{bom; claro; ensolarado; céu limpo e sem nuvens}
  \end{Phonetics}
\end{Entry}

\begin{Entry}{智}{12}{⽇}
  \begin{Phonetics}{智}{zhi4}
    \definition*{s.}{Sobrenome Zhi}
    \definition{adj.}{engenhoso; sábio; inteligente; astuto}
    \definition{s.}{discernimento; engenhosidade; sagacidade | inteligência; conhecimento; sabedoria; percepção}
  \end{Phonetics}
\end{Entry}

\begin{Entry}{智力}{12,2}{⽇、⼒}
  \begin{Phonetics}{智力}{zhi4li4}[][HSK 4]
    \definition{s.}{inteligência; refere-se à capacidade de uma pessoa de conhecer e entender coisas objetivas e aplicar o conhecimento e a experiência para resolver problemas, incluindo memória, observação, imaginação, pensamento e julgamento}
  \end{Phonetics}
\end{Entry}

\begin{Entry}{智能}{12,10}{⽇、⾁}
  \begin{Phonetics}{智能}{zhi4neng2}[][HSK 4]
    \definition{adj.}{inteligente (telefone, sistema, etc.); descreve máquinas, equipamentos, tecnologia, etc. que foram processados com alta tecnologia e têm a capacidade de falar, pensar, calcular, resolver problemas, etc., como um ser humano}
    \definition{s.}{intelecto; a capacidade de aprender, agir, pensar, inventar, criar, resolver problemas, etc.}
  \end{Phonetics}
\end{Entry}

\begin{Entry}{智商}{12,11}{⽇、⼝}
  \begin{Phonetics}{智商}{zhi4shang1}
    \definition{s.}{quociente de inteligência, QI}
  \end{Phonetics}
\end{Entry}

\begin{Entry}{智障}{12,13}{⽇、⾩}
  \begin{Phonetics}{智障}{zhi4zhang4}
    \definition{adj./s.}{retardado}
  \end{Phonetics}
\end{Entry}

\begin{Entry}{智慧}{12,15}{⽇、⼼}
  \begin{Phonetics}{智慧}{zhi4hui4}[][HSK 6]
    \definition[种]{s.}{sagacidade; sabedoria; inteligência; capacidade de analisar, julgar, inventar, criar e resolver problemas}
  \end{Phonetics}
\end{Entry}

\begin{Entry}{暂}{12}{⽇}
  \begin{Phonetics}{暂}{zan4}
    \definition{adj.}{de curta duração (oposto a 久) | curto; momentâneo; pouco tempo}
    \definition{adv.}{temporariamente; por enquanto}
  \seealsoref{久}{jiu3}
  \end{Phonetics}
\end{Entry}

\begin{Entry}{暂时}{12,7}{⽇、⽇}
  \begin{Phonetics}{暂时}{zan4shi2}[][HSK 5]
    \definition{adj.}{transitório; temporário}
    \definition{adv.}{por enquanto; em pouco tempo}
  \end{Phonetics}
\end{Entry}

\begin{Entry}{暂停}{12,11}{⽇、⼈}
  \begin{Phonetics}{暂停}{zan4 ting2}[][HSK 5]
    \definition{s.}{suspensão temporária; refere-se especificamente à suspensão temporária de certas competições desportivas de acordo com as regras}
    \definition{v.}{pausar; suspender; esgotar o tempo}
  \end{Phonetics}
\end{Entry}

\begin{Entry}{暑}{12}{⽇}
  \begin{Phonetics}{暑}{shu3}
    \definition{adj.}{calor; clima quente; quente (em oposição a 寒)}
    \definition{s.}{verão}
  \seealsoref{寒}{han2}
  \end{Phonetics}
\end{Entry}

\begin{Entry}{暑假}{12,11}{⽇、⼈}
  \begin{Phonetics}{暑假}{shu3 jia4}[][HSK 4]
    \definition[个]{s.}{férias de verão; feriado de verão; férias escolares de verão, na China, durante o sétimo e o oitavo meses do calendário gregoriano}
  \end{Phonetics}
\end{Entry}

\begin{Entry}{曾}{12}{⽈}
  \begin{Phonetics}{曾}{ceng2}[][HSK 4]
    \definition{adv.}{indica que uma ação já aconteceu ou um estado já existiu}
  \end{Phonetics}
  \begin{Phonetics}{曾}{zeng1}
    \definition*{s.}{Sobrenome Zeng}
    \definition{s.}{relacionamento entre bisnetos e bisavós; (parentesco) duas gerações de diferença}
  \end{Phonetics}
\end{Entry}

\begin{Entry}{曾经}{12,8}{⽈、⽷}
  \begin{Phonetics}{曾经}{ceng2jing1}[][HSK 3]
    \definition{adv.}{uma vez; indica que houve algum comportamento ou situação}
  \end{Phonetics}
\end{Entry}

\begin{Entry}{替}{12}{⽈}
  \begin{Phonetics}{替}{ti4}[][HSK 4]
    \definition{prep.}{para; em nome de}
    \definition{s.}{decadência; declínio; enfraquecimento}
    \definition{v.}{substituir; substituir por; tomar o lugar de}
  \end{Phonetics}
\end{Entry}

\begin{Entry}{替代}{12,5}{⽈、⼈}
  \begin{Phonetics}{替代}{ti4 dai4}[][HSK 4]
    \definition{v.}{substituir; suplantar}
  \end{Phonetics}
\end{Entry}

\begin{Entry}{最}{12}{⽈}
  \begin{Phonetics}{最}{zui4}[][HSK 1]
    \definition{adv.}{(diante de um adjetivo ou verbo) o mais | (colocado antes de um substantivo de localidade ou de uma palavra que indica um lugar)  mais distante ou mais próximo de (um lugar) | mais; melhor; pior; primeiro; muito; menos; acima de tudo; indica que uma determinada característica excede todas as outras pessoas ou coisas do mesmo tipo}
    \definition{s.}{o máximo; o melhor (ou o mais alto, o maior, etc.)}
  \end{Phonetics}
\end{Entry}

\begin{Entry}{最少}{12,4}{⽈、⼩}
  \begin{Phonetics}{最少}{zui4shao3}
    \definition{adv.}{finalmente}
  \end{Phonetics}
\end{Entry}

\begin{Entry}{最优}{12,6}{⽈、⼈}
  \begin{Phonetics}{最优}{zui4you1}
    \definition{adj.}{ótimo}
  \end{Phonetics}
\end{Entry}

\begin{Entry}{最先}{12,6}{⽈、⼉}
  \begin{Phonetics}{最先}{zui4xian1}
    \definition{adv.}{o primeiro}
  \end{Phonetics}
\end{Entry}

\begin{Entry}{最后}{12,6}{⽈、⼝}
  \begin{Phonetics}{最后}{zui4hou4}[][HSK 1]
    \definition{s.}{último; final; definitivo; refere-se ao tempo, local, etc. que vem depois de outros tempos, locais, etc. na ordem sequencial}
  \end{Phonetics}
\end{Entry}

\begin{Entry}{最多}{12,6}{⽈、⼣}
  \begin{Phonetics}{最多}{zui4duo1}
    \definition{adv.}{no máximo | máximo}
  \end{Phonetics}
\end{Entry}

\begin{Entry}{最好}{12,6}{⽈、⼥}
  \begin{Phonetics}{最好}{zui4hao3}[][HSK 1]
    \definition{adj.}{melhor; de primeira qualidade; excelente}
    \definition{adv.}{seria melhor; seria o ideal; indica a escolha mais adequada entre várias possibilidades}
  \end{Phonetics}
\end{Entry}

\begin{Entry}{最初}{12,7}{⽈、⾐}
  \begin{Phonetics}{最初}{zui4chu1}[][HSK 4]
    \definition{adj.}{primordial; inicial; primeiro}
    \definition{adv.}{inicialmente; originalmente}
    \definition{s.}{o período mais antigo; início; começo}
  \end{Phonetics}
\end{Entry}

\begin{Entry}{最近}{12,7}{⽈、⾡}
  \begin{Phonetics}{最近}{zui4jin4}[][HSK 2]
    \definition{adj.}{mais próximo}
    \definition{s.}{recentemente; ultimamente; de tarde; refere-se aos dias antes ou logo depois de um discurso | em breve; no futuro próximo; o futuro próximo}
  \end{Phonetics}
\end{Entry}

\begin{Entry}{最远}{12,7}{⽈、⾡}
  \begin{Phonetics}{最远}{zui4yuan3}
    \definition{adv.}{mais distante | mais longe}
  \end{Phonetics}
\end{Entry}

\begin{Entry}{最佳}{12,8}{⽈、⼈}
  \begin{Phonetics}{最佳}{zui4 jia1}[][HSK 6]
    \definition{adj.}{o melhor (atleta, filme etc); ótimo}
  \end{Phonetics}
\end{Entry}

\begin{Entry}{最终}{12,8}{⽈、⽷}
  \begin{Phonetics}{最终}{zui4 zhong1}[][HSK 6]
    \definition{adv.}{finalmente; por fim}
    \definition{s.}{final; definitivo}
  \end{Phonetics}
\end{Entry}

\begin{Entry}{最高}{12,10}{⽈、⾼}
  \begin{Phonetics}{最高}{zui4gao1}
    \definition{adj.}{altíssimo | supremo | mais alto}
  \end{Phonetics}
\end{Entry}

\begin{Entry}{最善}{12,12}{⽈、⼝}
  \begin{Phonetics}{最善}{zui4shan4}
    \definition{adj.}{ótimo | o melhor}
  \end{Phonetics}
\end{Entry}

\begin{Entry}{最新}{12,13}{⽈、⽄}
  \begin{Phonetics}{最新}{zui4xin1}
    \definition{adv.}{mais recente | mais novo}
  \end{Phonetics}
\end{Entry}

\begin{Entry}{朝}{12}{⽉}
  \begin{Phonetics}{朝}{chao2}[][HSK 3]
    \definition*{s.}{Sobrenome Chao}
    \definition{prep.}{para; em direção a; a direção ou o objeto da ação introduzida, equivalente a 向 ou 对}
    \definition[个]{s.}{corte real; governo; assembleia realizada por um soberano; também se refere à posição no poder, em oposição ao 野 | dinastia, todo o período de governo transmitido de geração em geração por um determinado sobrenome imperial | reinado (de um soberano); o período de reinado de um determinado monarca}
    \definition{v.}{fazer uma peregrinação para; ter uma audiência com (um rei, um imperador, etc.) | estar voltado para; estar em frente a}
  \seealsoref{对}{dui4}
  \seealsoref{向}{xiang4}
  \seealsoref{野}{ye3}
  \end{Phonetics}
  \begin{Phonetics}{朝}{zhao1}
    \definition{s.}{manhã cedo; manhã | dia}
  \end{Phonetics}
\end{Entry}

\begin{Entry}{朝代}{12,5}{⽉、⼈}
  \begin{Phonetics}{朝代}{chao2dai4}[][HSK 7-9]
    \definition[个]{s.}{dinastia; refere-se a um período ou era histórica}
  \end{Phonetics}
\end{Entry}

\begin{Entry}{朝廷}{12,6}{⽉、⼵}
  \begin{Phonetics}{朝廷}{chao2ting2}
    \definition{s.}{corte imperial | dinastia}
  \end{Phonetics}
\end{Entry}

\begin{Entry}{朝着}{12,11}{⽉、⽬}
  \begin{Phonetics}{朝着}{chao2zhe5}[][HSK 7-9]
    \definition{prep.}{voltado para; em direção a; pessoas ou coisas voltadas para uma direção}
  \end{Phonetics}
\end{Entry}

\begin{Entry}{朝鲜}{12,14}{⽉、⿂}
  \begin{Phonetics}{朝鲜}{chao2xian3}
    \definition*{s.}{Coréia do Norte}
  \end{Phonetics}
\end{Entry}

\begin{Entry}{期}{12}{⽉}
  \begin{Phonetics}{期}{qi1}[][HSK 3]
    \definition{clas.}{questão; número; termo; coisas usadas para parcelamento}
    \definition{s.}{um período de tempo; fase; estágio | horário agendado; data agendada | tempo designado (programado)}
    \definition{v.}{marcar uma consulta | esperar; aguardar | esperar; ter esperança}
  \end{Phonetics}
\end{Entry}

\begin{Entry}{期中}{12,4}{⽉、⼁}
  \begin{Phonetics}{期中}{qi1 zhong1}[][HSK 4]
    \definition{adj.}{provisório; interino; intermediário}
  \end{Phonetics}
\end{Entry}

\begin{Entry}{期末}{12,5}{⽉、⽊}
  \begin{Phonetics}{期末}{qi1 mo4}[][HSK 4]
    \definition{s.}{terminal; final do prazo; fim do período}
  \end{Phonetics}
\end{Entry}

\begin{Entry}{期间}{12,7}{⽉、⾨}
  \begin{Phonetics}{期间}{qi1jian1}[][HSK 4]
    \definition{s.}{prazo; tempo; período}
  \end{Phonetics}
\end{Entry}

\begin{Entry}{期限}{12,8}{⽉、⾩}
  \begin{Phonetics}{期限}{qi1xian4}[][HSK 4]
    \definition{s.}{prazo; limite de tempo; tempo alocado; período de tempo limitado, também o limite final do limite de tempo; \emph{deadline}}
  \end{Phonetics}
\end{Entry}

\begin{Entry}{期待}{12,9}{⽉、⼻}
  \begin{Phonetics}{期待}{qi1dai4}[][HSK 4]
    \definition{v.}{aguardar; esperar; aguardar ansiosamente; ter em mente a realização de um determinado fim ou a ocorrência de uma determinada situação}
  \end{Phonetics}
\end{Entry}

\begin{Entry}{期望}{12,11}{⽉、⽉}
  \begin{Phonetics}{期望}{qi1wang4}[][HSK 5]
    \definition{s.}{esperança; expectativa}
    \definition{v.}{esperar; ter esperança}
  \end{Phonetics}
\end{Entry}

\begin{Entry}{棉}{12}{⽊}
  \begin{Phonetics}{棉}{mian2}
    \definition{adj.}{almofadado com algodão; acolchoado}
    \definition[些,种,类]{s.}{termo genérico para algodão ou paina | algodão | material semelhante ao algodão | acolchoado ou estofado de algodão}
  \end{Phonetics}
\end{Entry}

\begin{Entry}{棍}{12}{⽊}
  \begin{Phonetics}{棍}{gun4}[][HSK 7-9]
    \definition[根]{s.}{vara; bastão; porrete | canalha; patife; ladino; bandido}
  \end{Phonetics}
\end{Entry}

\begin{Entry}{棍子}{12,3}{⽊、⼦}
  \begin{Phonetics}{棍子}{gun4zi5}[][HSK 7-9]
    \definition[根]{s.}{vara; bastão; um objeto longo e redondo feito de madeira, bambu ou metal}
  \end{Phonetics}
\end{Entry}

\begin{Entry}{棒}{12}{⽊}
  \begin{Phonetics}{棒}{bang4}[][HSK 5]
    \definition{adj.}{bom; forte; excelente}
    \definition[根]{s.}{porrete; bastão; cajado; clava}
  \end{Phonetics}
\end{Entry}

\begin{Entry}{棒冰}{12,6}{⽊、⼎}
  \begin{Phonetics}{棒冰}{bang4bing1}
    \definition{s.}{picolé}
  \end{Phonetics}
\end{Entry}

\begin{Entry}{棒球}{12,11}{⽊、⽟}
  \begin{Phonetics}{棒球}{bang4qiu2}[][HSK 7-9]
    \definition[个,只]{s.}{beisebol}
  \end{Phonetics}
\end{Entry}

\begin{Entry}{棒棒糖}{12,12,16}{⽊、⽊、⽶}
  \begin{Phonetics}{棒棒糖}{bang4bang4tang2}
    \definition[根]{s.}{pirulito}
  \end{Phonetics}
\end{Entry}

\begin{Entry}{棕}{12}{⽊}
  \begin{Phonetics}{棕}{zong1}
    \definition{adj.}{marrom}
    \definition[个]{s.}{palmeira | fibra de palmeira; fibra de coco}
  \end{Phonetics}
\end{Entry}

\begin{Entry}{棕褐色}{12,14,6}{⽊、⾐、⾊}
  \begin{Phonetics}{棕褐色}{zong1he4 se4}
    \definition{s.}{cor sépia | bronzeado}
  \end{Phonetics}
\end{Entry}

\begin{Entry}{棘}{12}{⽊}
  \begin{Phonetics}{棘}{ji2}
    \definition*{s.}{Sobrenome Ji}
    \definition{s.}{árvore de jujuba | arbustos espinhosos; silvas | espinho}
  \end{Phonetics}
\end{Entry}

\begin{Entry}{棘手}{12,4}{⽊、⼿}
  \begin{Phonetics}{棘手}{ji2shou3}[][HSK 7-9]
    \definition{adj.}{complicado; difícil; espinhoso; difícil de manusear}
  \end{Phonetics}
\end{Entry}

\begin{Entry}{森}{12}{⽊}
  \begin{Phonetics}{森}{sen1}
    \definition{adj.}{cheio de árvores | multitudinário; em multidões | escuro; sombrio}
  \end{Phonetics}
\end{Entry}

\begin{Entry}{森林}{12,8}{⽊、⽊}
  \begin{Phonetics}{森林}{sen1lin2}[][HSK 4]
    \definition[片,座,处]{s.}{floresta; bosque; normalmente, refere-se a uma grande área de árvores em crescimento; na silvicultura, refere-se a um grande número de árvores que crescem em uma área razoavelmente grande de terra, juntamente com os animais e outras plantas}
  \end{Phonetics}
\end{Entry}

\begin{Entry}{棵}{12}{⽊}
  \begin{Phonetics}{棵}{ke1}[][HSK 4]
    \definition{clas.}{usado para plantas, árvores}
  \end{Phonetics}
\end{Entry}

\begin{Entry}{棹}{12}{⽊}
  \begin{Phonetics}{棹}{zhuo1}
    \variantof{桌}
  \end{Phonetics}
\end{Entry}

\begin{Entry}{棺}{12}{⽊}
  \begin{Phonetics}{棺}{guan1}
    \definition[副]{s.}{caixão; esquife; ataúde}
  \end{Phonetics}
\end{Entry}

\begin{Entry}{棺材}{12,7}{⽊、⽊}
  \begin{Phonetics}{棺材}{guan1cai5}[][HSK 7-9]
    \definition[具,口]{s.}{caixão; esquife; ataúde; féretro; urna funerária usada para enterrar os mortos, geralmente feito de madeira}
  \end{Phonetics}
\end{Entry}

\begin{Entry}{椅}{12}{⽊}
  \begin{Phonetics}{椅}{yi3}
    \definition*{s.}{Sobrenome Yi}
    \definition{s.}{cadeira}
  \end{Phonetics}
\end{Entry}

\begin{Entry}{椅子}{12,3}{⽊、⼦}
  \begin{Phonetics}{椅子}{yi3zi5}[][HSK 2]
    \definition[把,套,排]{s.}{cadeira; assentos com encosto, feitos principalmente de madeira, bambu, rattan, etc.; móveis com pernas, mas sem encosto para as pessoas se sentarem}
  \end{Phonetics}
\end{Entry}

\begin{Entry}{植}{12}{⽊}
  \begin{Phonetics}{植}{zhi2}
    \definition*{s.}{Sobrenome Zhi}
    \definition{s.}{flora; planta; vegetação}
    \definition{v.}{plantar; crescer; cultivar | configurar; estabelecer}
  \end{Phonetics}
\end{Entry}

\begin{Entry}{植物}{12,8}{⽊、⽜}
  \begin{Phonetics}{植物}{zhi2wu4}[][HSK 4]
    \definition[种,株,棵,盆]{s.}{planta; vegetação; flora}
  \end{Phonetics}
\end{Entry}

\begin{Entry}{椰}{12}{⽊}
  \begin{Phonetics}{椰}{ye1}
    \definition[只,棵]{s.}{coqueiro; coco}
  \end{Phonetics}
\end{Entry}

\begin{Entry}{椰汁}{12,5}{⽊、⽔}
  \begin{Phonetics}{椰汁}{ye1zhi1}
    \definition{s.}{água de coco}
  \end{Phonetics}
\end{Entry}

\begin{Entry}{欺}{12}{⽋}
  \begin{Phonetics}{欺}{qi1}
    \definition{v.}{enganar; trapacear | intimidar; tirar vantagem de alguém; tirar vantagem da fraqueza de (alguém, etc.)}
  \end{Phonetics}
\end{Entry}

\begin{Entry}{欺负}{12,6}{⽋、⾙}
  \begin{Phonetics}{欺负}{qi1fu5}[][HSK 6]
    \definition{v.}{violar, oprimir ou insultar com meios irracionais; \emph{bully}}
  \end{Phonetics}
\end{Entry}

\begin{Entry}{款}{12}{⽋}
  \begin{Phonetics}{款}{kuan3}
    \definition{clas.}{para versões ou modelos (de um produto)}
    \definition[笔,个]{s.}{montante de dinheiro | fundos | parágrafo | seção}
  \end{Phonetics}
\end{Entry}

\begin{Entry}{殖}{12}{⽍}
  \begin{Phonetics}{殖}{zhi2}
    \definition{v.}{crescer | reproduzir}
  \end{Phonetics}
\end{Entry}

\begin{Entry}{渡}{12}{⽔}
  \begin{Phonetics}{渡}{du4}[][HSK 6]
    \definition{s.}{(usualmente em nomes de lugares) travessia de balsa}
    \definition{v.}{atravessar (um rio, o mar, etc.) | superar; sobreviver | transportar (pessoas, mercadorias, etc.) através}
  \end{Phonetics}
\end{Entry}

\begin{Entry}{渡过}{12,6}{⽔、⾡}
  \begin{Phonetics}{渡过}{du4guo4}[][HSK 7-9]
    \definition{v.}{viajar; atravessar; cruzar}[渡过最困难的时期。===Atravessar os momentos mais difíceis.]
  \end{Phonetics}
\end{Entry}

\begin{Entry}{温}{12}{⽔}
  \begin{Phonetics}{温}{wen1}
    \definition{adj.}{morno; quente; suave}
    \definition{s.}{temperatura | doenças transmissíveis agudas; praga}
    \definition{v.}{aquecer; reaquecer; aquecer ligeiramente | revisar; repassar}
  \end{Phonetics}
\end{Entry}

\begin{Entry}{温和}{12,8}{⽔、⼝}
  \begin{Phonetics}{温和}{wen1he2}[][HSK 5]
    \definition{adj.}{gentil; suave; moderado}
  \end{Phonetics}
\end{Entry}

\begin{Entry}{温度}{12,9}{⽔、⼴}
  \begin{Phonetics}{温度}{wen1du4}[][HSK 2]
    \definition[度,级,档,个]{s.}{temperatura}
  \end{Phonetics}
\end{Entry}

\begin{Entry}{温度计}{12,9,4}{⽔、⼴、⾔}
  \begin{Phonetics}{温度计}{wen1du4ji4}
    \definition{s.}{termógrafo | termômetro}
  \end{Phonetics}
\end{Entry}

\begin{Entry}{温度表}{12,9,8}{⽔、⼴、⾐}
  \begin{Phonetics}{温度表}{wen1du4biao3}
    \definition{s.}{termômetro}
  \end{Phonetics}
\end{Entry}

\begin{Entry}{温度梯度}{12,9,11,9}{⽔、⼴、⽊、⼴}
  \begin{Phonetics}{温度梯度}{wen1du4ti1du4}
    \definition{s.}{gradiente de temperatura}
  \end{Phonetics}
\end{Entry}

\begin{Entry}{温柔}{12,9}{⽔、⽊}
  \begin{Phonetics}{温柔}{wen1rou2}
    \definition{adj.}{gentil e suave | terno | doce (comumente usado para descrever uma menina ou mulher)}
  \end{Phonetics}
\end{Entry}

\begin{Entry}{温暖}{12,13}{⽔、⽇}
  \begin{Phonetics}{温暖}{wen1nuan3}[][HSK 3]
    \definition{adj.}{caloroso; gentil; amigável | caloroso; quente}
    \definition{v.}{aquecer; fazer com que se sinta calor}
  \end{Phonetics}
\end{Entry}

\begin{Entry}{港}{12}{⽔}
  \begin{Phonetics}{港}{gang3}[][HSK 7-9]
    \definition*{s.}{Hong Kong, abreviação de 香港 | Sobrenome Gang}
    \definition{s.}{porto; ancoradouro}
  \seealsoref{香港}{xiang1gang3}
  \end{Phonetics}
\end{Entry}

\begin{Entry}{港口}{12,3}{⽔、⼝}
  \begin{Phonetics}{港口}{gang3kou3}[][HSK 6]
    \definition[个,座]{s.}{porto; locais com certas condições naturais e instalações portuárias para atracação de navios, embarque e desembarque de passageiros e coleta e distribuição de cargas}
  \end{Phonetics}
\end{Entry}

\begin{Entry}{渴}{12}{⽔}
  \begin{Phonetics}{渴}{ke3}[][HSK 1]
    \definition{adj.}{sedento}
    \definition{adv.}{ansiosamente; metáfora de urgência}
    \definition{v.}{desejar; ansiar por}
  \end{Phonetics}
\end{Entry}

\begin{Entry}{渴望}{12,11}{⽔、⽉}
  \begin{Phonetics}{渴望}{ke3wang4}[][HSK 5]
    \definition{v.}{aspirar; (ter sede, ansiar, desejar) por}
  \end{Phonetics}
\end{Entry}

\begin{Entry}{游}{12}{⽔}
  \begin{Phonetics}{游}{you2}[][HSK 3]
    \definition*{s.}{Sobrenome You}
    \definition{adj.}{itinerante; não fixo; que se move frequentemente}
    \definition{s.}{parte de um rio; uma seção do rio; trecho; bacia; curso}
    \definition{v.}{nadar | vagar por aí; caminhar; viajar; fazer turismo | associar com (comunicação) | vagar; passear; andar tranquilamente por todos os lugares}
  \end{Phonetics}
\end{Entry}

\begin{Entry}{游人}{12,2}{⽔、⼈}
  \begin{Phonetics}{游人}{you2 ren2}[][HSK 6]
    \definition[个,名,位,批]{s.}{visitante (de um parque, etc.); turista}
  \end{Phonetics}
\end{Entry}

\begin{Entry}{游戏}{12,6}{⽔、⼽}
  \begin{Phonetics}{游戏}{you2xi4}[][HSK 3]
    \definition[场]{s.}{jogo; recreação; atividades recreativas, como esconde-esconde, adivinhar charadas, etc.; certas atividades esportivas não competitivas; jogos recreativos}
    \definition{v.}{jogar; fazer atividades divertidas e agradáveis, sozinho ou com outras pessoas}
  \end{Phonetics}
\end{Entry}

\begin{Entry}{游戏机}{12,6,6}{⽔、⼽、⽊}
  \begin{Phonetics}{游戏机}{you2 xi4 ji1}[][HSK 6]
    \definition[台]{s.}{jogador de videogame | console | videogame}
  \end{Phonetics}
\end{Entry}

\begin{Entry}{游行}{12,6}{⽔、⾏}
  \begin{Phonetics}{游行}{you2 xing2}[][HSK 6]
    \definition{s.}{desfilar; marchar; manifestar-se; marchar em grupos nas ruas para celebrar, comemorar, manifestar-se, etc.}
  \end{Phonetics}
\end{Entry}

\begin{Entry}{游泳}{12,8}{⽔、⽔}
  \begin{Phonetics}{游泳}{you2/yong3}[][HSK 3]
    \definition[次]{s.}{natação; refere-se ao esporte ou atividade de natação}
    \definition{v.+compl.}{nadar; pessoas ou animais nadando na água}
  \end{Phonetics}
\end{Entry}

\begin{Entry}{游泳池}{12,8,6}{⽔、⽔、⽔}
  \begin{Phonetics}{游泳池}{you2 yong3 chi2}[][HSK 5]
    \definition[场,个]{s.}{piscina; piscinas artificiais para natação, divididas em duas categorias: internas e externas}
  \seealsoref{泳池}{yong3chi2}
  \seealsoref{游泳馆}{you2yong3guan3}
  \end{Phonetics}
\end{Entry}

\begin{Entry}{游泳衣}{12,8,6}{⽔、⽔、⾐}
  \begin{Phonetics}{游泳衣}{you2yong3yi1}
    \definition{s.}{roupa de banho}
  \seealsoref{泳衣}{yong3yi1}
  \end{Phonetics}
\end{Entry}

\begin{Entry}{游泳馆}{12,8,11}{⽔、⽔、⾷}
  \begin{Phonetics}{游泳馆}{you2yong3guan3}
    \definition{s.}{natatório; piscina coberta; edifícios esportivos usados ​​principalmente para esportes aquáticos, como natação, mergulho e polo aquático}
  \seealsoref{泳池}{yong3chi2}
  \seealsoref{游泳池}{you2 yong3 chi2}
  \end{Phonetics}
\end{Entry}

\begin{Entry}{游泳镜}{12,8,16}{⽔、⽔、⾦}
  \begin{Phonetics}{游泳镜}{you2yong3jing4}
    \definition{s.}{óculos de natação}
  \end{Phonetics}
\end{Entry}

\begin{Entry}{游玩}{12,8}{⽔、⽟}
  \begin{Phonetics}{游玩}{you2 wan2}[][HSK 6]
    \definition{v.}{brincar; jogar; divertir-se | passear; vagar; fazer turismo}
  \end{Phonetics}
\end{Entry}

\begin{Entry}{游客}{12,9}{⽔、⼧}
  \begin{Phonetics}{游客}{you2 ke4}[][HSK 2]
    \definition[个,位,名,群]{s.}{visitante; turista | (jogo online) jogador convidado}
  \end{Phonetics}
\end{Entry}

\begin{Entry}{游艇}{12,12}{⽔、⾈}
  \begin{Phonetics}{游艇}{you2ting3}
    \definition[只]{s.}{barcaça | iate}
  \end{Phonetics}
\end{Entry}

\begin{Entry}{湖}{12}{⽔}
  \begin{Phonetics}{湖}{hu2}[][HSK 2]
    \definition*{s.}{Huzhou, abreviação de 湖州 | Um nome que se refere às províncias de Hunan, 湖南,  e Hubei, 湖北}
    \definition[个,片]{s.}{lago}
  \seealsoref{湖北}{hu2bei3}
  \seealsoref{湖南}{hu2nan2}
  \seealsoref{湖州}{hu2zhou1}
  \end{Phonetics}
\end{Entry}

\begin{Entry}{湖北}{12,5}{⽔、⼔}
  \begin{Phonetics}{湖北}{hu2bei3}
    \definition*{s.}{Província de Hubei (Hupeh), na China central}
  \end{Phonetics}
\end{Entry}

\begin{Entry}{湖州}{12,6}{⽔、⼮}
  \begin{Phonetics}{湖州}{hu2zhou1}
    \definition*{s.}{Cidade de Huzhou, em Zhejiang}
  \end{Phonetics}
\end{Entry}

\begin{Entry}{湖泊}{12,8}{⽔、⽔}
  \begin{Phonetics}{湖泊}{hu2po1}[][HSK 7-9]
    \definition[个,片,些]{s.}{lago; nome geral para lagos}[湖泊中有丰富的鱼类。===O lago é abundante em peixes.]
  \end{Phonetics}
\end{Entry}

\begin{Entry}{湖南}{12,9}{⽔、⼗}
  \begin{Phonetics}{湖南}{hu2nan2}
    \definition*{s.}{Província de Hunan}
  \end{Phonetics}
\end{Entry}

\begin{Entry}{湿}{12}{⽔}
  \begin{Phonetics}{湿}{shi1}[][HSK 4]
    \definition{adj.}{molhado; úmido; algo com água ou com muita água dentro}
  \end{Phonetics}
\end{Entry}

\begin{Entry}{滑}{12}{⽔}
  \begin{Phonetics}{滑}{hua2}[][HSK 5]
    \definition*{s.}{Sobrenome Hua}
    \definition{adj.}{escorregadio; liso; objetos com superfícies lisas e baixo atrito | astuto; ardiloso; escorregadio}
    \definition{v.}{escorregar; deslizar | se atrapalhar; se safar de algo}
  \end{Phonetics}
\end{Entry}

\begin{Entry}{滑冰}{12,6}{⽔、⼎}
  \begin{Phonetics}{滑冰}{hua2bing1}[][HSK 7-9]
    \definition{s.}{patinação no gelo; um evento esportivo em que os atletas usam patins especiais para patinar no gelo, competindo em velocidade ou realizando manobras}
    \definition{v.}{patinar; patinar no gelo; deslizar no gelo}
  \end{Phonetics}
\end{Entry}

\begin{Entry}{滑梯}{12,11}{⽔、⽊}
  \begin{Phonetics}{滑梯}{hua2ti1}[][HSK 7-9]
    \definition{s.}{escorregador infantil}
  \end{Phonetics}
\end{Entry}

\begin{Entry}{滑雪}{12,11}{⽔、⾬}
  \begin{Phonetics}{滑雪}{hua2/xue3}[][HSK 7-9]
    \definition{v.+compl.}{esquiar; praticar esqui; usar pranchas especiais nos pés para deslizar na neve}
  \end{Phonetics}
\end{Entry}

\begin{Entry}{滑稽}{12,15}{⽔、⽲}
  \begin{Phonetics}{滑稽}{hua2ji1}[][HSK 7-9]
    \definition{adj.}{engraçado; divertido; cômico; (palavras, ações ou gestos) que fazem as pessoas rirem}
    \definition{s.}{conversa cômica; um tipo de arte popular, popular nas áreas de Xangai, Jiangsu e Zhejiang, semelhante ao \emph{crosstalk}}
  \end{Phonetics}
\end{Entry}

\begin{Entry}{焚}{12}{⽕}
  \begin{Phonetics}{焚}{fen2}
    \definition{v.}{queimar}
  \end{Phonetics}
\end{Entry}

\begin{Entry}{焚香}{12,9}{⽕、⾹}
  \begin{Phonetics}{焚香}{fen2xiang1}
    \definition{v.}{queimar incenso}
  \end{Phonetics}
\end{Entry}

\begin{Entry}{焚烧}{12,10}{⽕、⽕}
  \begin{Phonetics}{焚烧}{fen2shao1}[][HSK 7-9]
    \definition{v.}{queimar; incendiar; colocar fogo}
  \end{Phonetics}
\end{Entry}

\begin{Entry}{焦}{12}{⽕}
  \begin{Phonetics}{焦}{jiao1}
    \definition*{s.}{Sobrenome Jiao}
    \definition{adj.}{queimado; chamuscado; carbonizado | preocupado; ansioso}
    \definition{clas.}{J; Joule, abreviação}
    \definition{pref.}{(química) piro-}
    \definition{s.}{Metalurgia: coque}
  \end{Phonetics}
\end{Entry}

\begin{Entry}{焦点}{12,9}{⽕、⽕}
  \begin{Phonetics}{焦点}{jiao1dian3}[][HSK 6]
    \definition{s.}{foco; ponto focal; Matemática: refere-se a um ponto que tem uma relação especial com uma elipse, hipérbole, parábola, etc. | foco; ponto focal; Óptica: refere-se à intersecção de feixes de luz paralelos após serem refratados por uma lente ou refletidos por um espelho curvo | foco; questão central; metaforicamente, uma coisa ou princípio que chama a atenção para o foco}
  \end{Phonetics}
\end{Entry}

\begin{Entry}{焦虑}{12,10}{⽕、⾌}
  \begin{Phonetics}{焦虑}{jiao1lv4}
    \definition{adj.}{ansioso | preocupado | apreensivo}
  \end{Phonetics}
\end{Entry}

\begin{Entry}{然}{12}{⽕}
  \begin{Phonetics}{然}{ran2}
    \definition{conj.}{mas | no entanto}
  \end{Phonetics}
\end{Entry}

\begin{Entry}{然后}{12,6}{⽕、⼝}
  \begin{Phonetics}{然后}{ran2hou4}[][HSK 2]
    \definition{conj.}{então; depois disso; posteriormente; indica que algo segue após uma ação ou situação}
  \end{Phonetics}
\end{Entry}

\begin{Entry}{然而}{12,6}{⽕、⽽}
  \begin{Phonetics}{然而}{ran2'er2}[][HSK 4]
    \definition{conj.}{ainda; mas; contudo; todavia; usado no início de uma frase para indicar uma transição; para indicar uma transição, geralmente é precedido por uma conjunção como 虽然 para indicar concessão}
  \seealsoref{虽然}{sui1 ran2}
  \end{Phonetics}
\end{Entry}

\begin{Entry}{煮}{12}{⽕}
  \begin{Phonetics}{煮}{zhu3}[][HSK 6]
    \definition*{s.}{Sobrenome Zhu}
    \definition{v.}{ferver; cozinhar; aquecer alimentos ou outros itens em água}
  \end{Phonetics}
\end{Entry}

\begin{Entry}{牌}{12}{⽚}
  \begin{Phonetics}{牌}{pai2}[][HSK 4]
    \definition[块,副,张,个,种]{s.}{placa; tabuleta; quadro; placar | marca; marca registrada; marca comercial; \emph{trademark} | cartas, dominó, etc. | a tonalidade de uma música}
  \end{Phonetics}
\end{Entry}

\begin{Entry}{牌子}{12,3}{⽚、⼦}
  \begin{Phonetics}{牌子}{pai2 zi5}[][HSK 3]
    \definition[个,种,块]{s.}{sinal; placa; placas feitas de madeira ou outros materiais, geralmente com texto nelas | marca; marca registrada; um nome especial dado por uma empresa ao seu próprio produto}
  \end{Phonetics}
\end{Entry}

\begin{Entry}{猩}{12}{⽝}
  \begin{Phonetics}{猩}{xing1}
    \definition[只]{s.}{orangotango}
  \end{Phonetics}
\end{Entry}

\begin{Entry}{猩猩}{12,12}{⽝、⽝}
  \begin{Phonetics}{猩猩}{xing1xing5}
    \definition{s.}{orangotango}
  \end{Phonetics}
\end{Entry}

\begin{Entry}{猴}{12}{⽝}
  \begin{Phonetics}{猴}{hou2}[][HSK 5]
    \definition{adj.}{esperto; inteligente; perspicaz | travesso (menino)}
    \definition[只,群]{s.}{macaco}
  \end{Phonetics}
\end{Entry}

\begin{Entry}{猴子}{12,3}{⽝、⼦}
  \begin{Phonetics}{猴子}{hou2zi5}
    \definition[只]{s.}{macaco}
  \end{Phonetics}
\end{Entry}

\begin{Entry}{琴}{12}{⽟}
  \begin{Phonetics}{琴}{qin2}[][HSK 5]
    \definition*{s.}{Sobrenome Qin}
    \definition[架,台]{s.}{cítara; qin; guqin (um instrumento de cordas dedilhadas com sete cordas, em alguns aspectos semelhante à cítara)  | nome genérico para certos instrumentos musicais}
  \end{Phonetics}
\end{Entry}

\begin{Entry}{琴键}{12,13}{⽟、⾦}
  \begin{Phonetics}{琴键}{qin2jian4}
    \definition{s.}{tecla de piano}
  \end{Phonetics}
\end{Entry}

\begin{Entry}{甁}{12}{⽡}
  \begin{Phonetics}{甁}{ping2}
    \variantof{瓶}
  \end{Phonetics}
\end{Entry}

\begin{Entry}{番}{12}{⽥}
  \begin{Phonetics}{番}{fan1}[][HSK 6]
    \definition{adj.}{estrangeiro; de tribos estrangeiras; estrangeiro ou alienígena}
    \definition{clas.}{usado para o número de vezes que uma ação é executada, equivalente a 回 ou 次 | usado para o tipo de coisas, equivalente a 种}
    \definition{s.}{estrangeiro; de tribos estrangeiras; (velho) refere-se a países estrangeiros ou raças estrangeiras | tomate; batata-doce | aborígenes; nativos; povos indígenas}
    \definition{v.}{revezar; rotacionar; substituir}
  \seealsoref{次}{ci4}
  \seealsoref{回}{hui2}
  \seealsoref{种}{zhong3}
  \end{Phonetics}
\end{Entry}

\begin{Entry}{番茄}{12,8}{⽥、⾋}
  \begin{Phonetics}{番茄}{fan1 qie2}[][HSK 6]
    \definition[个,斤,磅,公斤]{s.}{tomate | tomateiro}
  \end{Phonetics}
\end{Entry}

\begin{Entry}{疏}{12}{⽦}
  \begin{Phonetics}{疏}{shu1}
    \definition*{s.}{Sobrenome Shu}
    \definition{adj.}{fino; esparso; disperso (oposto a 密) | espalhado; disperso; difuso; a distância entre as coisas é grande; as lacunas entre as partes das coisas são grandes | distante; relacionamento distante; não próximo (de relações familiares ou sociais) | não familiarizado com; desconhecido | escasso; vazio}
    \definition{s.}{memorial; memorial ao trono; um texto em que um ministro na era feudal apresentava seus assuntos ao monarca em detalhes | comentário; anotações mais detalhadas de livros antigos do que 注}
    \definition{v.}{dragar (um rio, etc.) | negligenciar | dispersar; espalhar}
  \seealsoref{密}{mi4}
  \seealsoref{注}{zhu4}
  \end{Phonetics}
\end{Entry}

\begin{Entry}{痛}{12}{⽧}
  \begin{Phonetics}{痛}{tong4}[][HSK 3]
    \definition{adv.}{extremamente; profundamente; amargamente}
    \definition{s.}{dor; sofrimento | tristeza; pesar}
  \end{Phonetics}
\end{Entry}

\begin{Entry}{痛快}{12,7}{⽧、⼼}
  \begin{Phonetics}{痛快}{tong4kuai4}[][HSK 4]
    \definition{adj.}{encantado; alegre; muito feliz; confortável | franco; direto; simples e direto}
  \end{Phonetics}
\end{Entry}

\begin{Entry}{痛苦}{12,8}{⽧、⾋}
  \begin{Phonetics}{痛苦}{tong4ku3}[][HSK 3]
    \definition{adj.}{doloroso; angustiado; sentindo-se muito desconfortável física ou mentalmente}
    \definition[降,种]{s.}{dor; agonia; sofrimento; refere-se a um estado ou sentimento de extremo desconforto físico ou mental}
  \end{Phonetics}
\end{Entry}

\begin{Entry}{痛骂}{12,9}{⽧、⾺}
  \begin{Phonetics}{痛骂}{tong4ma4}
    \definition{v.}{repreender severamente}
  \end{Phonetics}
\end{Entry}

\begin{Entry}{痠}{12}{⽧}
  \begin{Phonetics}{痠}{suan1}
    \definition{v.}{doer | estar dolorido}
    \variantof{酸}
  \end{Phonetics}
\end{Entry}

\begin{Entry}{登}{12}{⽨}
  \begin{Phonetics}{登}{deng1}[][HSK 4]
    \definition{v.}{subir; montar; escalar (uma altura) | publicar; registrar; inserir | recolher e levar para a eira | pisar em; pisar | calçar (calçados ou calças) | partir; começar uma jornada; embarcar em uma jornada}
  \end{Phonetics}
\end{Entry}

\begin{Entry}{登山}{12,3}{⽨、⼭}
  \begin{Phonetics}{登山}{deng1 shan1}[][HSK 4]
    \definition{s.}{escalar; fazer alpinismo; subir uma montanha}
  \end{Phonetics}
\end{Entry}

\begin{Entry}{登记}{12,5}{⽨、⾔}
  \begin{Phonetics}{登记}{deng1/ji4}[][HSK 4]
    \definition{v.+compl.}{registrar-se; fazer o \emph{check-in} | registrar; reportar; informar; relatar por escrito a um superior ou autoridade relevante (usado principalmente para documentos legais)}
  \end{Phonetics}
\end{Entry}

\begin{Entry}{登机}{12,6}{⽨、⽊}
  \begin{Phonetics}{登机}{deng1ji1}[][HSK 7-9]
    \definition{v.}{embarcar; embarcar em um avião}
  \end{Phonetics}
\end{Entry}

\begin{Entry}{登陆}{12,7}{⽨、⾩}
  \begin{Phonetics}{登陆}{deng1/lu4}[][HSK 7-9]
    \definition{v.+compl.}{desembarcar; chegar à costa | entrar em um mercado; Metáfora: mercadorias entram em um determinado mercado e começam a ser vendidas; comerciantes entram em um determinado mercado e começam a fazer negócios}
  \end{Phonetics}
\end{Entry}

\begin{Entry}{登录}{12,8}{⽨、⼹}
  \begin{Phonetics}{登录}{deng1lu4}[][HSK 4]
    \definition{v.}{fazer \emph{logon}; fazer \emph{login} | gravar; registrar; computadores eletrônicos e sua terminologia de rede, referindo-se ao acesso ao sistema operacional ou ao site a ser visitado}
  \end{Phonetics}
\end{Entry}

\begin{Entry}{短}{12}{⽮}
  \begin{Phonetics}{短}{duan3}[][HSK 2]
    \definition{adj.}{curto; comprimento pequeno de uma extremidade à outra (em oposição a 长) | curto; breve; a distância entre o ponto inicial e o ponto final de um determinado período é pequena | raso; superficial}
    \definition{s.}{falha; defeito; ponto fraco; desvantagens | tonelada curta (EUA)}
    \definition{v.}{dever; carecer}
  \seealsoref{长}{zhang3}
  \end{Phonetics}
\end{Entry}

\begin{Entry}{短少}{12,4}{⽮、⼩}
  \begin{Phonetics}{短少}{duan3shao3}
    \definition{v.}{estar aquém do valor total}
  \end{Phonetics}
\end{Entry}

\begin{Entry}{短片}{12,4}{⽮、⽚}
  \begin{Phonetics}{短片}{duan3 pian4}[][HSK 6]
    \definition{s.}{curta-metragem; curtas-metragens documentais ou educativos exibidos individualmente ou em série}
  \end{Phonetics}
\end{Entry}

\begin{Entry}{短处}{12,5}{⽮、⼡}
  \begin{Phonetics}{短处}{duan3 chu4}[][HSK 3]
    \definition[个]{s.}{deficiência; ponto fraco; defeito; fraqueza}
  \end{Phonetics}
\end{Entry}

\begin{Entry}{短视}{12,8}{⽮、⾒}
  \begin{Phonetics}{短视}{duan3shi4}
    \definition{adj.}{míope}
  \end{Phonetics}
\end{Entry}

\begin{Entry}{短促}{12,9}{⽮、⼈}
  \begin{Phonetics}{短促}{duan3cu4}
    \definition{adj.}{curto (tom de voz) | fugaz | ofegante (respiração) | curto no tempo}
  \end{Phonetics}
\end{Entry}

\begin{Entry}{短信}{12,9}{⽮、⼈}
  \begin{Phonetics}{短信}{duan3xin4}[][HSK 2]
    \definition[条,个,封]{s.}{mensagem de texto; refere-se especificamente a mensagens de texto curtas, imagens, etc., enviadas ou recebidas por celular}
  \end{Phonetics}
\end{Entry}

\begin{Entry}{短缺}{12,10}{⽮、⽸}
  \begin{Phonetics}{短缺}{duan3que1}[][HSK 7-9]
    \definition{s.}{falta; déficit; escassez; insuficiência}
  \end{Phonetics}
\end{Entry}

\begin{Entry}{短暂}{12,12}{⽮、⽇}
  \begin{Phonetics}{短暂}{duan3zan4}[][HSK 7-9]
    \definition{adj.}{breve; transitório; momentâneo; de curta duração}
  \end{Phonetics}
\end{Entry}

\begin{Entry}{短期}{12,12}{⽮、⽉}
  \begin{Phonetics}{短期}{duan3 qi1}[][HSK 3]
    \definition{adj.}{de curta duração; de prazo curto}
    \definition[个]{s.}{curto prazo}
  \end{Phonetics}
\end{Entry}

\begin{Entry}{短裤}{12,12}{⽮、⾐}
  \begin{Phonetics}{短裤}{duan3 ku4}[][HSK 3]
    \definition[条]{s.}{calças curtas; calção; \emph{shorts}; calças com bainha acima do joelho}
  \end{Phonetics}
\end{Entry}

\begin{Entry}{短跑}{12,12}{⽮、⾜}
  \begin{Phonetics}{短跑}{duan3 pao3}
    \definition{s.}{corrida de curta distância; corrida rápida (oposto a 长跑)}
  \seealsoref{长跑}{chang2 pao3}
  \end{Phonetics}
\end{Entry}

\begin{Entry}{硬}{12}{⽯}
  \begin{Phonetics}{硬}{ying4}[][HSK 4,5]
    \definition{adj.}{duro; rígido; resistente;  objeto resistente e não se deforma facilmente quando submetido a forças externas (em oposição a 软) | firme; forte; resistente; obstinado; (vontade, atitude, etc.) inabalável, forte e poderoso | capaz (pessoa); boa (qualidade) | rígido; severo; sem flexibilidade | duro; rígido; rigoroso; imutável}
    \definition{adv.}{conseguir fazer algo com dificuldade; indica fazer algo à força, independentemente das circunstâncias}
  \seealsoref{软}{ruan3}
  \end{Phonetics}
\end{Entry}

\begin{Entry}{硬件}{12,6}{⽯、⼈}
  \begin{Phonetics}{硬件}{ying4jian4}[][HSK 5]
    \definition[种]{s.}{\emph{hardware}; nome genérico dado aos vários elementos, componentes e dispositivos que constituem um computador | máquina, materiais; equipamento; referência a máquinas, equipamentos, materiais físicos, etc., utilizados nos processos de produção, pesquisa científica, gestão, etc.}
  \end{Phonetics}
\end{Entry}

\begin{Entry}{确}{12}{⽯}
  \begin{Phonetics}{确}{que4}
    \definition{adj.}{autenticado | sólido | firme | real | verdadeiro}
  \end{Phonetics}
\end{Entry}

\begin{Entry}{确认}{12,4}{⽯、⾔}
  \begin{Phonetics}{确认}{que4ren4}[][HSK 4]
    \definition{v.}{afirmar; confirmar; reconhecer; confirmar explicitamente (fatos, princípios, etc.)}
  \end{Phonetics}
\end{Entry}

\begin{Entry}{确立}{12,5}{⽯、⽴}
  \begin{Phonetics}{确立}{que4li4}[][HSK 5]
    \definition{v.}{estabelecer; criar; construir; estabelecer ou consolidar firmemente}
  \end{Phonetics}
\end{Entry}

\begin{Entry}{确定}{12,8}{⽯、⼧}
  \begin{Phonetics}{确定}{que4ding4}[][HSK 3]
    \definition{adj.}{definido; certo; claro}
    \definition{v.}{firmar; definir; determinar; tomar uma decisão clara e não mudar}
  \end{Phonetics}
\end{Entry}

\begin{Entry}{确实}{12,8}{⽯、⼧}
  \begin{Phonetics}{确实}{que4shi2}[][HSK 3]
    \definition{adj.}{verdadeiro; confiável; autêntico}
    \definition{adv.}{verdadeiramente; realmente; de ​​fato; afirmar a autenticidade de fatos objetivos}
  \end{Phonetics}
\end{Entry}

\begin{Entry}{确保}{12,9}{⽯、⼈}
  \begin{Phonetics}{确保}{que4bao3}[][HSK 3]
    \definition{v.}{assegurar; garantir; manter ou garantir com certeza}
  \end{Phonetics}
\end{Entry}

\begin{Entry}{禅}{12}{⽰}
  \begin{Phonetics}{禅}{chan2}
    \definition{s.}{Budismo: contemplação prolongada e intensa; meditação profunda | budista; refere-se geralmente a coisas relacionadas ao budismo}
  \end{Phonetics}
  \begin{Phonetics}{禅}{shan4}
    \definition{v.}{abdicar e entregar a coroa a outra pessoa}
  \end{Phonetics}
\end{Entry}

\begin{Entry}{禅杖}{12,7}{⽰、⽊}
  \begin{Phonetics}{禅杖}{chan2zhang4}[][HSK 7-9]
    \definition[根,支]{s.}{cajado (bastão) do monge budista | uma bengala com cabeça acolchoada para bater na cabeça de quem adormece}
  \end{Phonetics}
\end{Entry}

\begin{Entry}{禽}{12}{⽱}
  \begin{Phonetics}{禽}{qin2}
    \definition*{s.}{Sobrenome Qin}
    \definition[只]{s.}{aves; pássaros | termo genérico para aves e animais}
  \end{Phonetics}
\end{Entry}

\begin{Entry}{程}{12}{⽲}
  \begin{Phonetics}{程}{cheng2}
    \definition{s.}{regra; regulamento; lei | ordem; procedimento | jornada; etapa de uma jornada; estrada; um trecho de estrada | distância percorrida ou movida por um objeto | programação | medição; termo geral para pesos e medidas}
  \end{Phonetics}
\end{Entry}

\begin{Entry}{程序}{12,7}{⽲、⼴}
  \begin{Phonetics}{程序}{cheng2xu4}[][HSK 4]
    \definition[个,套,种]{s.}{ordem; curso; sequência; procedimento; ordem em que algo é feito; também, um determinado número de etapas em um trabalho | programa; conjunto de instruções de computador projetado em sequência para permitir que um computador execute uma ou mais operações}
  \end{Phonetics}
\end{Entry}

\begin{Entry}{程序设计}{12,7,6,4}{⽲、⼴、⾔、⾔}
  \begin{Phonetics}{程序设计}{cheng2xu4she4ji4}
    \definition{s.}{programação de computadores}
  \end{Phonetics}
\end{Entry}

\begin{Entry}{程序库}{12,7,7}{⽲、⼴、⼴}
  \begin{Phonetics}{程序库}{cheng2xu4ku4}
    \definition{s.}{biblioteca de funções e procedimentos para programas de computador}
  \end{Phonetics}
\end{Entry}

\begin{Entry}{程度}{12,9}{⽲、⼴}
  \begin{Phonetics}{程度}{cheng2du4}[][HSK 3]
    \definition[种]{s.}{nível; grau (de cultura, educação, aprendizagem, etc.) | extensão; grau; a situação, o nível ou o estágio em que as coisas mudam}
  \end{Phonetics}
\end{Entry}

\begin{Entry}{程控}{12,11}{⽲、⼿}
  \begin{Phonetics}{程控}{cheng2kong4}
    \definition{s.}{programado | sob controle automático}
  \end{Phonetics}
\end{Entry}

\begin{Entry}{稍}{12}{⽲}
  \begin{Phonetics}{稍}{shao1}[][HSK 5]
    \definition{adv.}{ligeiramente; um pouco; um pouquinho}
  \end{Phonetics}
\end{Entry}

\begin{Entry}{稍微}{12,13}{⽲、⼻}
  \begin{Phonetics}{稍微}{shao1wei1}[][HSK 5]
    \definition{adv.}{um pouco; um pouquinho; uma ninharia; indica que a quantidade é pequena ou o grau é superficial}
  \end{Phonetics}
\end{Entry}

\begin{Entry}{税}{12}{⽲}
  \begin{Phonetics}{税}{shui4}[][HSK 6]
    \definition*{s.}{Sobrenome Shui}
    \definition{s.}{imposto; taxa; tarifa}
  \end{Phonetics}
\end{Entry}

\begin{Entry}{窗}{12}{⽳}
  \begin{Phonetics}{窗}{chuang1}
    \definition[扇,个]{s.}{janela}
  \end{Phonetics}
\end{Entry}

\begin{Entry}{窗口}{12,3}{⽳、⼝}
  \begin{Phonetics}{窗口}{chuang1 kou3}[][HSK 6]
    \definition[个,号]{s.}{janela; em frente à janela; perto da janela | janela; postigo; refere-se a uma abertura especial em forma de janela | janela; meio; intermediário; peça de exibição; campo de testes; uma metáfora para um lugar com muitas interações com o mundo exterior e através do qual o entendimento mútuo é alcançado |  janela; uma metáfora para um lugar que pode refletir ou exibir a totalidade ou parte de algo |  caixa de diálogo; uma caixa de operação quadrada para aplicativos ou arquivos que aparece na tela do computador}
  \end{Phonetics}
\end{Entry}

\begin{Entry}{窗子}{12,3}{⽳、⼦}
  \begin{Phonetics}{窗子}{chuang1 zi5}[][HSK 4]
    \definition[扇,个]{s.}{janela}
  \end{Phonetics}
\end{Entry}

\begin{Entry}{窗户}{12,4}{⽳、⼾}
  \begin{Phonetics}{窗户}{chuang1hu5}[][HSK 4]
    \definition[个,扇,面,排]{s.}{janela; dispositivo de ventilação e transmissão de luz nas paredes}
  \end{Phonetics}
\end{Entry}

\begin{Entry}{窗台}{12,5}{⽳、⼝}
  \begin{Phonetics}{窗台}{chuang1 tai2}[][HSK 4]
    \definition{s.}{parapeito da janela; peitoril; parte plana de uma janela que segura a moldura}
  \end{Phonetics}
\end{Entry}

\begin{Entry}{窗帘}{12,8}{⽳、⼱}
  \begin{Phonetics}{窗帘}{chuang1lian2}[][HSK 5]
    \definition[个,套,片,对]{s.}{cortinas para janelas}
  \end{Phonetics}
\end{Entry}

\begin{Entry}{窜}{12}{⽳}
  \begin{Phonetics}{窜}{cuan4}[][HSK 7-9]
    \definition{v.}{fugir; correr (usado para bandidos, tropas inimigas e animais) | Literário: exilar; expulsar | Datado: mudar (a redação de um texto, manuscrito, etc.); alterar}
  \end{Phonetics}
\end{Entry}

\begin{Entry}{童}{12}{⽴}
  \begin{Phonetics}{童}{tong2}
    \definition*{s.}{Sobrenome Tong}
    \definition{adj.}{virgem; solteira | nu; careca | árido; estéril}
    \definition{s.}{criança | jovem servo; antigamente, referia-se a um servo menor de idade.}
  \end{Phonetics}
\end{Entry}

\begin{Entry}{童年}{12,6}{⽴、⼲}
  \begin{Phonetics}{童年}{tong2 nian2}[][HSK 4]
    \definition[对]{s.}{infância; primeiros anos de vida}
  \end{Phonetics}
\end{Entry}

\begin{Entry}{童话}{12,8}{⽴、⾔}
  \begin{Phonetics}{童话}{tong2hua4}[][HSK 4]
    \definition[个,部]{s.}{conto de fadas; gênero de literatura infantil no qual as histórias adequadas para a diversão das crianças são escritas com muita imaginação, fantasia e exagero}
  \end{Phonetics}
\end{Entry}

\begin{Entry}{等}{12}{⽵}
  \begin{Phonetics}{等}{deng3}[][HSK 1,2]
    \definition*{s.}{Sobrenome Deng}
    \definition{adj.}{igual; na mesma medida ou quantidade}
    \definition{clas.}{usado para classe, grau, classificação | usado para tipo}
    \definition{part.}{e assim por diante; etc.; indica que a enumeração não está completa (pode ser usada repetidamente) | indica o fim de uma enumeração; após a enumeração, é usado para encerrar; geralmente é seguido pelo total dos itens anteriores}
    \definition{pron.}{usado após pronomes pessoais ou substantivos que se referem a pessoas; indica plural}
    \definition{s.}{classe; série; posição | equilíbrio; balança para pesar pequenas quantidades de objetos valiosos e ervas medicinais; atualmente, geralmente escrita como 戥}
    \definition{v.}{esperar; aguardar | esperar até}
  \end{Phonetics}
\end{Entry}

\begin{Entry}{等于}{12,3}{⽵、⼆}
  \begin{Phonetics}{等于}{deng3yu2}[][HSK 2]
    \definition{adv.}{igual a | equivalente a}
    \definition{v.}{equivaler a; ser equivalente a; ser quase igual a; não ter diferença}
  \end{Phonetics}
\end{Entry}

\begin{Entry}{等级}{12,6}{⽵、⽷}
  \begin{Phonetics}{等级}{deng3ji2}[][HSK 5]
    \definition[个]{s.}{grau; classificação; posição; distinções por qualidade, grau, status, etc. | estado social; estrato social; ordem e grau; grupos sociais desiguais em termos de status social e legal}
  \end{Phonetics}
\end{Entry}

\begin{Entry}{等到}{12,8}{⽵、⼑}
  \begin{Phonetics}{等到}{deng3 dao4}[][HSK 2]
    \definition{prep.}{na hora; quando; expressão de condições temporais | esperar até; aguardar até}
  \end{Phonetics}
\end{Entry}

\begin{Entry}{等待}{12,9}{⽵、⼻}
  \begin{Phonetics}{等待}{deng3dai4}[][HSK 3]
    \definition{v.}{esperar; aguardar; não agir até que a pessoa, coisa ou situação desejada apareça}
  \end{Phonetics}
\end{Entry}

\begin{Entry}{等候}{12,10}{⽵、⼈}
  \begin{Phonetics}{等候}{deng3hou4}[][HSK 5]
    \definition{v.}{esperar; aguardar; expectar; usado principalmente para objetos específicos}
  \end{Phonetics}
\end{Entry}

\begin{Entry}{等等}{12,12}{⽵、⽵}
  \begin{Phonetics}{等等}{deng3 deng3}
    \definition{part.}{etc.; e assim por diante; usada depois de duas ou mais palavras paralelas para indicar que a lista não está completa}
  \end{Phonetics}
\end{Entry}

\begin{Entry}{筏}{12}{⽵}
  \begin{Phonetics}{筏}{fa2}
    \definition[条]{s.}{jangada (de troncos, bambus, etc.)}
  \end{Phonetics}
\end{Entry}

\begin{Entry}{筒}{12}{⽵}
  \begin{Phonetics}{筒}{tong3}
    \definition[个]{s.}{seção de bambu grosso; tubo grosso de bambu | objeto em forma de tubo largo | a parte em forma de tubo das roupas etc.}
  \end{Phonetics}
\end{Entry}

\begin{Entry}{答}{12}{⽵}
  \begin{Phonetics}{答}{da1}[][HSK 5]
    \definition{v.}{concordar; responder | responder; prestar atenção}
  \end{Phonetics}
  \begin{Phonetics}{答}{da2}[][HSK 5]
    \definition{v.}{responder; dar resposta a; responder a | retribuir; devolver (uma visita, etc.); retribuir um favor feito a alguém por outro; fazer o bem}
  \end{Phonetics}
\end{Entry}

\begin{Entry}{答应}{12,7}{⽵、⼴}
  \begin{Phonetics}{答应}{da1ying5}[][HSK 2]
    \definition{v.}{responder; retribuir; reagir; retrucar | concordar; prometer; cumprir}
  \end{Phonetics}
\end{Entry}

\begin{Entry}{答复}{12,9}{⽵、⼢}
  \begin{Phonetics}{答复}{da2fu4}[][HSK 5]
    \definition[个]{s.}{resposta; respostas a perguntas ou solicitações}
    \definition{v.}{responder; dar uma resposta}
  \end{Phonetics}
\end{Entry}

\begin{Entry}{答案}{12,10}{⽵、⽊}
  \begin{Phonetics}{答案}{da2'an4}[][HSK 4]
    \definition[个,条,种,些]{s.}{chave; resposta; solução}
  \end{Phonetics}
\end{Entry}

\begin{Entry}{答辩}{12,16}{⽵、⾟}
  \begin{Phonetics}{答辩}{da2bian4}[][HSK 7-9]
    \definition{v.}{responder a perguntas, acusações, etc. de outras pessoas; defender as próprias opiniões ou ações}
  \end{Phonetics}
\end{Entry}

\begin{Entry}{策}{12}{⽵}
  \begin{Phonetics}{策}{ce4}
    \definition*{s.}{Sobrenome Ce}
    \definition[个,项,根]{s.}{plano; esquema | tiras de bambu ou madeira usadas para escrever na China antiga | questões sobre atualidades definidas para os exames imperiais | chicote de montaria antigo | um tipo de ensaio na China antiga; um estilo de escrita para exames antigos | estratégia; método}
    \definition{v.}{chicotear (um cavalo) com um chicote de montaria | incitar com um chicote de cavalo, espora}
  \end{Phonetics}
\end{Entry}

\begin{Entry}{策划}{12,6}{⽵、⼑}
  \begin{Phonetics}{策划}{ce4hua4}[][HSK 6]
    \definition{v.}{planejar; traçar; esquematizar; pensar repetidamente para elaborar um plano}
  \end{Phonetics}
\end{Entry}

\begin{Entry}{策略}{12,11}{⽵、⽥}
  \begin{Phonetics}{策略}{ce4lve4}[][HSK 6]
    \definition{adj.}{diplomático; (métodos) flexíveis sem sacrificar princípios}
    \definition[种,个,条,套]{s.}{tática; estratégia; política; para atingir determinadas tarefas estratégicas, o curso de ação e os métodos de luta são formulados de acordo com o desenvolvimento da situação}
  \end{Phonetics}
\end{Entry}

\begin{Entry}{粤}{12}{⾔}
  \begin{Phonetics}{粤}{yue4}
    \definition*{s.}{Outro nome para a Província de Guangdong, 广东}
  \seealsoref{广东}{guang3dong1}
  \end{Phonetics}
\end{Entry}

\begin{Entry}{粤语}{12,9}{⾔、⾔}
  \begin{Phonetics}{粤语}{yue4yu3}
    \definition{s.}{cantonês | língua cantonesa}
  \end{Phonetics}
\end{Entry}

\begin{Entry}{粥}{12}{⽶}
  \begin{Phonetics}{粥}{yu4}
    \definition{v.}{dar a luz; ter filhos}
  \end{Phonetics}
  \begin{Phonetics}{粥}{zhou1}[][HSK 6]
    \definition[碗,锅,口]{s.}{mingau; mingau de aveia; alimentos semilíquidos feitos de grãos ou grãos misturados com outras coisas}
  \end{Phonetics}
\end{Entry}

\begin{Entry}{粪}{12}{⽶}
  \begin{Phonetics}{粪}{fen4}[][HSK 7-9]
    \definition[缸,桶]{s.}{excremento; fezes; esterco}
    \definition{v.}{Literário: aplicar esterco; fertilizar | Literário: limpar; remover; eliminar; acabar com}
  \end{Phonetics}
\end{Entry}

\begin{Entry}{粪便}{12,9}{⽶、⼈}
  \begin{Phonetics}{粪便}{fen4bian4}[][HSK 7-9]
    \definition{s.}{excremento; fezes; esterco; excrementos e urina}
  \end{Phonetics}
\end{Entry}

\begin{Entry}{紫}{12}{⽷}
  \begin{Phonetics}{紫}{zi3}[][HSK 5]
    \definition*{s.}{Sobrenome Zi}
    \definition{adj.}{roxo; púrpura; violeta; cor resultante da combinação do vermelho e do azul}
  \end{Phonetics}
\end{Entry}

\begin{Entry}{紫色}{12,6}{⽷、⾊}
  \begin{Phonetics}{紫色}{zi3 se4}
    \definition{s.}{cor púrpura | cor violeta}
  \end{Phonetics}
\end{Entry}

\begin{Entry}{絫}{12}{⽷}
  \begin{Phonetics}{絫}{lei3}
    \variantof{累}
  \end{Phonetics}
\end{Entry}

\begin{Entry}{缓}{12}{⽶}
  \begin{Phonetics}{缓}{huan3}[][HSK 7-9]
    \definition{adj.}{lento; sem pressa | sem tensão; relaxado}
    \definition{v.}{atrasar; adiar; protelar | recuperar; reviver; voltar a si}
  \end{Phonetics}
\end{Entry}

\begin{Entry}{缓和}{12,8}{⽶、⼝}
  \begin{Phonetics}{缓和}{huan3he2}[][HSK 7-9]
    \definition{adj.}{relaxado; moderado; suave; pacífico e relaxante; não tenso ou intenso}
    \definition{v.}{relaxar; aliviar; atenuar; facilitar}
  \end{Phonetics}
\end{Entry}

\begin{Entry}{缓缓}{12,12}{⽶、⽶}
  \begin{Phonetics}{缓缓}{huan3huan3}[][HSK 7-9]
    \definition{adv.}{lentamente; vagarosamente; gradualmente}
  \end{Phonetics}
\end{Entry}

\begin{Entry}{缓解}{12,13}{⽶、⾓}
  \begin{Phonetics}{缓解}{huan3jie3}[][HSK 4]
    \definition{v.}{facilitar; aliviar; atenuar; amenizar; reduzir}
  \end{Phonetics}
\end{Entry}

\begin{Entry}{缓慢}{12,14}{⽶、⼼}
  \begin{Phonetics}{缓慢}{huan3man4}[][HSK 7-9]
    \definition{adj.}{lento; vagaroso}
    \definition{adv.}{lentamente; vagarosamente}
  \end{Phonetics}
\end{Entry}

\begin{Entry}{编}{12}{⽷}
  \begin{Phonetics}{编}{bian1}[][HSK 4]
    \definition*{s.}{Sobrenome Bian}
    \definition{s.}{livro; volume; parte de um livro | organização e pessoal; estabelecimento}
    \definition{v.}{tecer; trançar; entrançar | fazer uma lista; organizar em uma lista; organizar; agrupar | editar; compilar | compor; escrever | fabricar; inventar; fazer; preparar}
  \end{Phonetics}
\end{Entry}

\begin{Entry}{编写}{12,5}{⽷、⼍}
  \begin{Phonetics}{编写}{bian1xie3}[][HSK 7-9]
    \definition{v.}{compilar; organizar materiais existentes em um livro ou artigo | escrever; compor; criar | oletar informações e organizá-las ou criar algo}
  \end{Phonetics}
\end{Entry}

\begin{Entry}{编号}{12,5}{⽷、⼝}
  \begin{Phonetics}{编号}{bian1hao4}[][HSK 7-9]
    \definition{s.}{número de série; números dados em sequência}
    \definition{v.}{numerar; dar números às pessoas ou coisas em ordem}
  \end{Phonetics}
\end{Entry}

\begin{Entry}{编制}{12,8}{⽷、⼑}
  \begin{Phonetics}{编制}{bian1 zhi4}[][HSK 6]
    \definition{s.}{estabelecimento; organização e pessoal; refere-se à estrutura organizacional de uma unidade, cotas de pessoal, alocação de tarefas, etc.}
    \definition{v.}{tecer; trançar; entrelaçar tiras de vime, salgueiro, bambu, etc. para fazer objetos | resolver; realizar; elaborar; fazer de acordo com os dados (procedimentos, planos, etc.)}
  \end{Phonetics}
\end{Entry}

\begin{Entry}{编剧}{12,10}{⽷、⼑}
  \begin{Phonetics}{编剧}{bian1ju4}[][HSK 7-9]
    \definition[个,位,名]{s.}{dramaturgo | roteirista}
    \definition{v.}{escrever uma peça, um roteiro, etc.}
  \end{Phonetics}
\end{Entry}

\begin{Entry}{编造}{12,10}{⽷、⾡}
  \begin{Phonetics}{编造}{bian1zao4}[][HSK 7-9]
    \definition{v.}{compilar; compor; preparar | fabricar; inventar; criar; cozinhar | criar a partir da imaginação}
  \end{Phonetics}
\end{Entry}

\begin{Entry}{编排}{12,11}{⽷、⼿}
  \begin{Phonetics}{编排}{bian1pai2}[][HSK 7-9]
    \definition{v.}{dispor; organizar em uma determinada ordem | escrever uma peça e ensaiá-la}
  \end{Phonetics}
\end{Entry}

\begin{Entry}{编程}{12,12}{⽷、⽲}
  \begin{Phonetics}{编程}{bian1cheng2}
    \definition{v.}{programar computador}
  \end{Phonetics}
\end{Entry}

\begin{Entry}{编辑}{12,13}{⽷、⾞}
  \begin{Phonetics}{编辑}{bian1ji2}[][HSK 5]
    \definition[名,位,个]{s.}{editor; compilador; uma pessoa que organiza e processa dados ou trabalhos existentes}
    \definition{v.}{editar; compilar; organizar e processar dados ou trabalhos existentes}
  \end{Phonetics}
  \begin{Phonetics}{编辑}{bian1ji5}[][HSK 5]
    \definition{s.}{editor; compilador; pessoa que organiza e processa dados ou trabalhos existentes}
  \end{Phonetics}
\end{Entry}

\begin{Entry}{缘}{12}{⽷}
  \begin{Phonetics}{缘}{yuan2}
    \definition{s.}{causa | razão | karma | destino | predestinação}
  \end{Phonetics}
\end{Entry}

\begin{Entry}{缘分}{12,4}{⽷、⼑}
  \begin{Phonetics}{缘分}{yuan2fen4}
    \definition{s.}{destino ou acaso que une as pessoas | afinidade ou relacionamento predestinado | destino (Budismo)}
  \end{Phonetics}
\end{Entry}

\begin{Entry}{缘故}{12,9}{⽷、⽁}
  \begin{Phonetics}{缘故}{yuan2gu4}[][HSK 6]
    \definition{s.}{causa; razão}
  \end{Phonetics}
\end{Entry}

\begin{Entry}{羡}{12}{⽺}
  \begin{Phonetics}{羡}{xian4}
    \definition{v.}{admirar; invejar}
  \end{Phonetics}
\end{Entry}

\begin{Entry}{羡慕}{12,14}{⽺、⼼}
  \begin{Phonetics}{羡慕}{xian4mu4}
    \definition{v.}{invejar; admirar; ver os outros terem certos pontos fortes ou vantagens e desejar tê-los também}
  \end{Phonetics}
\end{Entry}

\begin{Entry}{联}{12}{⽿}
  \begin{Phonetics}{联}{lian2}
    \definition{s.}{dísticos (antitéticos)}
    \definition{v.}{aliar-se a; unir-se; juntar-se a}
  \end{Phonetics}
\end{Entry}

\begin{Entry}{联手}{12,4}{⽿、⼿}
  \begin{Phonetics}{联手}{lian2 shou3}[][HSK 6]
    \definition{v.}{dar as mãos; cooperar | Literário: dar as mãos | agir em conjunto}
  \end{Phonetics}
\end{Entry}

\begin{Entry}{联合}{12,6}{⽿、⼝}
  \begin{Phonetics}{联合}{lian2he2}[][HSK 3]
    \definition{adj.}{conjunto; unido; federal; combinado}
    \definition{s.}{aliado; união; aliança; conectar-se ou unir-se para agir em conjunto}
  \end{Phonetics}
\end{Entry}

\begin{Entry}{联合会}{12,6,6}{⽿、⼝、⼈}
  \begin{Phonetics}{联合会}{lian2he2hui4}
    \definition{s.}{federação}
  \end{Phonetics}
\end{Entry}

\begin{Entry}{联合国}{12,6,8}{⽿、⼝、⼞}
  \begin{Phonetics}{联合国}{lian2 he2 guo2}[][HSK 3]
    \definition*{s.}{Nações Unidas; Organização internacional fundada em 1945, após o fim da Segunda Guerra Mundial, com sede em Nova Iorque, Estados Unidos ; as suas principais instituições são a Assembleia Geral, o Conselho de Segurança, o Conselho Econômico e Social e o Secretariado; de acordo com a Carta das Nações Unidas, os seus principais objetivos são manter a paz e a segurança internacionais, desenvolver relações amigáveis entre os países e promover a cooperação internacional nas áreas econômica e cultural}
  \end{Phonetics}
\end{Entry}

\begin{Entry}{联系}{12,7}{⽿、⽷}
  \begin{Phonetics}{联系}{lian2xi4}[][HSK 3]
    \definition[个,种,层]{s.}{relacionamento; relacionamento entre duas coisas}
    \definition{v.}{entrar em contato; contatar; comunicar-se com alguém por telefone, e-mail ou carta | agendar; entrar em contato com; estabelecer algum tipo de relação com a outra parte | relacionar; combinar; integrar}
  \end{Phonetics}
\end{Entry}

\begin{Entry}{联络}{12,9}{⽿、⽷}
  \begin{Phonetics}{联络}{lian2luo4}[][HSK 5]
    \definition{v.}{entrar em contato; comunicar-se; entrar em contato com}
  \end{Phonetics}
\end{Entry}

\begin{Entry}{联想}{12,13}{⽿、⼼}
  \begin{Phonetics}{联想}{lian2xiang3}[][HSK 5]
    \definition*{s.}{Lenovo (empresa)}
    \definition{v.}{associar-se a; estabelecer uma conexão mental; lembrar-se de algo; lembrar-se de outras pessoas ou coisas relacionadas devido a alguém ou algo; evocar outros conceitos relacionados devido a um determinado conceito}
  \end{Phonetics}
\end{Entry}

\begin{Entry}{联盟}{12,13}{⽿、⽫}
  \begin{Phonetics}{联盟}{lian2meng2}[][HSK 6]
    \definition{s.}{aliança; coalizão; liga; união}
  \end{Phonetics}
\end{Entry}

\begin{Entry}{联赛}{12,14}{⽿、⾙}
  \begin{Phonetics}{联赛}{lian2 sai4}[][HSK 6]
    \definition{s.}{jogos da liga | liga (esportiva) | torneio da liga}
  \end{Phonetics}
\end{Entry}

\begin{Entry}{脾}{12}{⾁}
  \begin{Phonetics}{脾}{pi2}
    \definition{s.}{baço}
  \end{Phonetics}
\end{Entry}

\begin{Entry}{脾气}{12,4}{⾁、⽓}
  \begin{Phonetics}{脾气}{pi2qi5}[][HSK 5]
    \definition[股]{s.}{temperamento; disposição; referindo-se ao caráter de uma pessoa | mau humor; temperamento irascível}
  \end{Phonetics}
\end{Entry}

\begin{Entry}{舒}{12}{⾆}
  \begin{Phonetics}{舒}{shu1}
    \definition*{s.}{Sobrenome Shu}
    \definition{adj.}{lento; vagaroso; sem pressa | confortável; relaxado e feliz}
    \definition{v.}{esticar; desdobrar | alongar; relaxar}
  \end{Phonetics}
\end{Entry}

\begin{Entry}{舒服}{12,8}{⾆、⽉}
  \begin{Phonetics}{舒服}{shu1fu5}[][HSK 2]
    \definition{adj.}{confortável; sentir-se relaxado e feliz, tanto física quanto mentalmente}
  \end{Phonetics}
\end{Entry}

\begin{Entry}{舒适}{12,9}{⾆、⾡}
  \begin{Phonetics}{舒适}{shu1shi4}[][HSK 4]
    \definition{adj.}{aconchegante; confortável; acolhedor; cômodo}
  \end{Phonetics}
\end{Entry}

\begin{Entry}{舜}{12}{⾇}
  \begin{Phonetics}{舜}{shun4}
    \definition*{s.}{Shun, o nome de um monarca lendário da China antiga | Sobrenome Shun}
  \end{Phonetics}
\end{Entry}

\begin{Entry}{落}{12}{⾋}
  \begin{Phonetics}{落}{la4}[][HSK 5]
    \definition{v.}{deixar de fora; estar ausente | deixar para trás; esquecer de trazer; deixar algo em algum lugar e esquecer de levar| ficar para trás (ou cair); não conseguir acompanhar}
  \end{Phonetics}
  \begin{Phonetics}{落}{lao4}
    \definition{v.}{cair; cair de uma altura elevada | se abaixar; descer; ir para baixo | permanecer; fazer uma parada; deixar para trás | obter; ter; receber}
  \end{Phonetics}
  \begin{Phonetics}{落}{luo4}[][HSK 4]
    \definition*{s.}{Sobrenome Luo}
    \definition{s.}{paradeiro; lugar para ficar; local de descanso | assentamento; local de reunião | parte curta; área pequena; refere-se a um pequeno lugar ou área}
    \definition{v.}{cair; cair de uma altura elevada | se abaixar; descer; ir para baixo | abaixar; deixar cair (ou descer); fazer descer | afundar; declinar; descer | ficar para trás; ficar para trás ou ficar de fora | permanecer; fazer uma parada; deixar para trás | cair sobre; repousar com | obter; ter; receber | anotar; escrever no papel | cair em; entrar em; ficar preso}
  \end{Phonetics}
\end{Entry}

\begin{Entry}{落日}{12,4}{⾋、⽇}
  \begin{Phonetics}{落日}{luo4ri4}
    \definition{s.}{pôr do sol}
  \end{Phonetics}
\end{Entry}

\begin{Entry}{落后}{12,6}{⾋、⼝}
  \begin{Phonetics}{落后}{luo4hou4}[][HSK 3]
    \definition{adj.}{atrasado; trabalho em atraso, nível de desenvolvimento ou grau de reconhecimento (em oposição a 进步)}
    \definition{v.}{ficar para trás; ficar atrasado; ficar para trás em relação aos outros durante o avanço ou o progresso do trabalho}
  \seealsoref{进步}{jin4bu4}
  \end{Phonetics}
\end{Entry}

\begin{Entry}{落汤鸡}{12,6,7}{⾋、⽔、⿃}
  \begin{Phonetics}{落汤鸡}{luo4tang1ji1}
    \definition{s.}{uma pessoa que parece encharcada e acamada| sofrimento profundo}
  \end{Phonetics}
\end{Entry}

\begin{Entry}{落花生}{12,7,5}{⾋、⾋、⽣}
  \begin{Phonetics}{落花生}{luo4 hua1 sheng1}
    \definition{s.}{amendoim | noz de macaco}
  \end{Phonetics}
\end{Entry}

\begin{Entry}{落实}{12,8}{⾋、⼧}
  \begin{Phonetics}{落实}{luo4shi2}[][HSK 5]
    \definition{adj.}{sentimento de tranquilidade; (humor) estável; seguro}
    \definition{v.}{implementar; ser praticável; tornar os planos, políticas, medidas, etc. específicos e compreensíveis, de modo a que possam ser realizados | implementar; colocar em prática; pôr em prática significa que os planos, políticas e medidas são específicos e claros, e podem ser realizados}
  \end{Phonetics}
\end{Entry}

\begin{Entry}{葡}{12}{⾋}
  \begin{Phonetics}{葡}{pu2}
    \definition*{s.}{Portugal, abreviação de 葡萄牙}
  \seealsoref{葡萄牙}{pu2tao2ya2}
  \end{Phonetics}
\end{Entry}

\begin{Entry}{葡文}{12,4}{⾋、⽂}
  \begin{Phonetics}{葡文}{pu2wen2}
    \definition{s.}{português, língua portuguesa}
  \seealsoref{葡萄牙文}{pu2tao2ya2wen2}
  \end{Phonetics}
\end{Entry}

\begin{Entry}{葡汉词典}{12,5,7,8}{⾋、⽔、⾔、⼋}
  \begin{Phonetics}{葡汉词典}{pu2-han4 ci2dian3}
    \definition{s.}{dicionário português-chinês}
  \seealsoref{汉葡词典}{han4-pu2 ci2dian3}
  \end{Phonetics}
\end{Entry}

\begin{Entry}{葡语}{12,9}{⾋、⾔}
  \begin{Phonetics}{葡语}{pu2yu3}
    \definition{s.}{português, língua portuguesa}
  \seealsoref{葡萄牙语}{pu2tao2ya2yu3}
  \end{Phonetics}
\end{Entry}

\begin{Entry}{葡萄}{12,11}{⾋、⾋}
  \begin{Phonetics}{葡萄}{pu2tao5}[][HSK 5]
    \definition[串,颗,粒,棵,种]{s.}{parreira | uva}
  \end{Phonetics}
\end{Entry}

\begin{Entry}{葡萄牙}{12,11,4}{⾋、⾋、⽛}
  \begin{Phonetics}{葡萄牙}{pu2tao2ya2}
    \definition{s.}{Portugal}
  \end{Phonetics}
\end{Entry}

\begin{Entry}{葡萄牙文}{12,11,4,4}{⾋、⾋、⽛、⽂}
  \begin{Phonetics}{葡萄牙文}{pu2tao2ya2wen2}
    \definition{s.}{português, língua portuguesa}
  \seealsoref{葡文}{pu2wen2}
  \end{Phonetics}
\end{Entry}

\begin{Entry}{葡萄牙语}{12,11,4,9}{⾋、⾋、⽛、⾔}
  \begin{Phonetics}{葡萄牙语}{pu2tao2ya2yu3}
    \definition{s.}{português, língua portuguesa}
  \seealsoref{葡语}{pu2yu3}
  \end{Phonetics}
\end{Entry}

\begin{Entry}{葡萄酒}{12,11,10}{⾋、⾋、⾣}
  \begin{Phonetics}{葡萄酒}{pu2 tao2 jiu3}[][HSK 5]
    \definition[瓶,杯,口,桶]{s.}{vinho (de uva)}
  \end{Phonetics}
\end{Entry}

\begin{Entry}{董}{12}{⾋}
  \begin{Phonetics}{董}{dong3}
    \definition*{s.}{Sobrenome Dong}
    \definition{s.}{diretor; administrador}
    \definition{v.}{Literário: dirigir; supervisionar; supervisionar}
  \end{Phonetics}
\end{Entry}

\begin{Entry}{董事}{12,8}{⾋、⼅}
  \begin{Phonetics}{董事}{dong3shi4}[][HSK 7-9]
    \definition[个,位,名]{s.}{administrador; diretor}
  \end{Phonetics}
\end{Entry}

\begin{Entry}{董事长}{12,8,4}{⾋、⼅、⾧}
  \begin{Phonetics}{董事长}{dong3shi4zhang3}[][HSK 7-9]
    \definition[个,位,名]{s.}{presidente; presidente do conselho; \emph{chairman}; o principal responsável pelo conselho de administração}
  \end{Phonetics}
\end{Entry}

\begin{Entry}{董事会}{12,8,6}{⾋、⼅、⼈}
  \begin{Phonetics}{董事会}{dong3shi4hui4}[][HSK 7-9]
    \definition[届]{s.}{conselho de administração (numa empresa); conselho de curadores (numa instituição de ensino); órgão decisório de uma sociedade anônima, escola ou grupo}
  \end{Phonetics}
\end{Entry}

\begin{Entry}{葫}{12}{⾋}
  \begin{Phonetics}{葫}{hu2}
    \definition{s.}{cabaça}
  \end{Phonetics}
\end{Entry}

\begin{Entry}{葫芦}{12,7}{⾋、⾋}
  \begin{Phonetics}{葫芦}{hu2lu5}
    \definition{adj.}{confuso}
    \definition{s.}{cabaça | termo genérico para bloco e equipamento (ou partes dele)}
  \end{Phonetics}
\end{Entry}

\begin{Entry}{葬}{12}{⾋}
  \begin{Phonetics}{葬}{zang4}
    \definition{v.}{enterrar (os mortos) | sepultar}
  \end{Phonetics}
\end{Entry}

\begin{Entry}{葱}{12}{⾋}
  \begin{Phonetics}{葱}{cong1}[][HSK 7-9]
    \definition{adj.}{verde; turquesa}
    \definition[根,把,捆]{s.}{cebola; cebolinha}
  \end{Phonetics}
\end{Entry}

\begin{Entry}{葵}{12}{⾋}
  \begin{Phonetics}{葵}{kui2}
    \definition*{s.}{Sobrenome Kui}
    \definition[朵]{s.}{certas ervas de flores grandes}
  \end{Phonetics}
\end{Entry}

\begin{Entry}{葵花}{12,7}{⾋、⾋}
  \begin{Phonetics}{葵花}{kui2hua1}
    \definition{s.}{girassol (flor)}
  \end{Phonetics}
\end{Entry}

\begin{Entry}{街}{12}{⾏}
  \begin{Phonetics}{街}{jie1}[][HSK 2]
    \definition[条]{s.}{rua; avenida com prédios dos dois lados | mercado; feira rural}
  \end{Phonetics}
\end{Entry}

\begin{Entry}{街头}{12,5}{⾏、⼤}
  \begin{Phonetics}{街头}{jie1 tou2}[][HSK 6]
    \definition{s.}{rua; esquina da rua}
  \end{Phonetics}
\end{Entry}

\begin{Entry}{街道}{12,12}{⾏、⾡}
  \begin{Phonetics}{街道}{jie1dao4}[][HSK 4]
    \definition[条]{s.}{caminho; rua; estrada; via pública com casas em ambos os lados, relativamente larga | escritório do subdistrito; tipo de organização responsável por gerenciar determinados aspectos da rua}
  \end{Phonetics}
\end{Entry}

\begin{Entry}{街舞}{12,14}{⾏、⾇}
  \begin{Phonetics}{街舞}{jie1wu3}
    \definition{s.}{dança de rua, \emph{street dance} (por exemplo, \emph{breakdance})}
  \end{Phonetics}
\end{Entry}

\begin{Entry}{裁}{12}{⾐}
  \begin{Phonetics}{裁}{cai2}[][HSK 7-9]
    \definition{clas.}{divisão de papel de impressão de tamanho padrão}
    \definition{s.}{planejamento | tipo de escrita | planejamento mental; arranjo e seleção, usados principalmente na literatura e na arte | sanção; restrição | estilo; forma | (impressão) tamanho do corte de papel}
    \definition{v.}{cortar (papel, tecido, etc.) em partes | reduzir; cortar; dispensar | julgar; decidir | verificar; sancionar | cortar; eliminar; remover coisas desnecessárias ou redundantes | discernir; medir; julgar}
  \end{Phonetics}
\end{Entry}

\begin{Entry}{裁决}{12,6}{⾐、⼎}
  \begin{Phonetics}{裁决}{cai2jue2}[][HSK 7-9]
    \definition[项]{s.}{sentença arbitral; decisão; adjudicação}
    \definition{v.}{fazer uma decisão; julgar; decidir; adjudicar veredicto}
  \end{Phonetics}
\end{Entry}

\begin{Entry}{裁判}{12,7}{⾐、⼑}
  \begin{Phonetics}{裁判}{cai2pan4}[][HSK 5]
    \definition[个,位,名]{s.}{árbitro; juiz; pessoa que desempenha funções de arbitragem em esportes e outras competições}
    \definition{v.}{arbitrar; atuar como árbitro; em esportes e outras atividades competitivas, julgar o desempenho dos atletas, vitórias e derrotas, classificações e problemas que ocorrem durante a competição de acordo com as regras da competição | julgar; refere-se a um terceiro que faz um julgamento quando surge uma disputa entre duas partes}
  \end{Phonetics}
\end{Entry}

\begin{Entry}{裁定}{12,8}{⾐、⼧}
  \begin{Phonetics}{裁定}{cai2ding4}[][HSK 7-9]
    \definition{v.}{decidir (ou declarar) judicialmente; governar}
  \end{Phonetics}
\end{Entry}

\begin{Entry}{裂}{12}{⾐}
  \begin{Phonetics}{裂}{lie4}[][HSK 6]
    \definition{s.}{entalhe; incisão; entalhes grandes e profundos nas bordas das folhas ou corolas | brecha; lacuna; rachadura; refere-se à rachadura ou divisão que aparece na superfície ou no interior de um objeto}
    \definition{v.}{dividir; rachar; rasgar | (figurativo) quebrar; esmagar; arruinar}
  \end{Phonetics}
\end{Entry}

\begin{Entry}{装}{12}{⾐}
  \begin{Phonetics}{装}{zhuang1}[][HSK 2]
    \definition*{s.}{Sobrenome Zhuang}
    \definition{s.}{vestido; traje; vestimenta; roupa | maquiagem e figurino de palco; maquiagem de ator}
    \definition{v.}{enfeitar; adornar; vestir; decorar; vestir-se; vestir-se bem | fingir; fazer de conta | segurar; embalar; carregar; colocar as coisas em recipientes; colocar as coisas no transporte | encaixar; instalar; equipar; aparelhar; montar | embalar; encaixotar; embrulhar produtos ou colocá-los em caixas, garrafas, etc.}
  \end{Phonetics}
\end{Entry}

\begin{Entry}{装扮}{12,7}{⾐、⼿}
  \begin{Phonetics}{装扮}{zhuang1ban4}
    \definition{v.}{enfeitar | decorar | disfarçar-me | vestir-se}
  \end{Phonetics}
\end{Entry}

\begin{Entry}{装备}{12,8}{⾐、⼡}
  \begin{Phonetics}{装备}{zhuang1bei4}[][HSK 6]
    \definition[套]{s.}{equipamento; equipagem; traje}
    \definition{v.}{equipar}
  \end{Phonetics}
\end{Entry}

\begin{Entry}{装饰}{12,8}{⾐、⾷}
  \begin{Phonetics}{装饰}{zhuang1shi4}[][HSK 5]
    \definition[件,个]{s.}{decoração; acessórios decorativos}
    \definition{v.}{enfeitar; adornar; decorar; ornamentar; embelezar; destacar}
  \end{Phonetics}
\end{Entry}

\begin{Entry}{装修}{12,9}{⾐、⼈}
  \begin{Phonetics}{装修}{zhuang1 xiu1}[][HSK 4]
    \definition{v.}{equipar; renovar; decorar (equipar uma sala ou prédio com equipamentos ou decorações); reboco, pintura e instalação de portas, janelas, encanamentos e outros equipamentos em projetos habitacionais}
  \end{Phonetics}
\end{Entry}

\begin{Entry}{装配}{12,10}{⾐、⾣}
  \begin{Phonetics}{装配}{zhuang1pei4}
    \definition{v.}{montar | encaixar}
  \end{Phonetics}
\end{Entry}

\begin{Entry}{装置}{12,13}{⾐、⽹}
  \begin{Phonetics}{装置}{zhuang1 zhi4}[][HSK 4]
    \definition[个,台,种,些]{s.}{dispositivo; equipamento; máquinas, instrumentos ou outros equipamentos de construção mais complexa e com alguma função independente}
    \definition{v.}{instalar; ajustar; configurar; equipar; montar}
  \end{Phonetics}
\end{Entry}

\begin{Entry}{裙}{12}{⾐}
  \begin{Phonetics}{裙}{qun2}
    \definition[条]{s.}{saia | avental | algo como uma saia}
  \end{Phonetics}
\end{Entry}

\begin{Entry}{裙子}{12,3}{⾐、⼦}
  \begin{Phonetics}{裙子}{qun2zi5}[][HSK 3]
    \definition[条,件]{s.}{saia (peça de vestuário); uma vestimenta usada abaixo da cintura}
  \end{Phonetics}
\end{Entry}

\begin{Entry}{裤}{12}{⾐}
  \begin{Phonetics}{裤}{ku4}
    \definition[条]{s.}{calças}
  \end{Phonetics}
\end{Entry}

\begin{Entry}{裤子}{12,3}{⾐、⼦}
  \begin{Phonetics}{裤子}{ku4zi5}[][HSK 3]
    \definition[条]{s.}{calças; calções; roupas usadas abaixo da cintura, com cós, virilha e duas pernas}
  \end{Phonetics}
\end{Entry}

\begin{Entry}{詈}{12}{⾔}
  \begin{Phonetics}{詈}{li4}
    \definition{v.}{xingar; usar linguagem severa}
  \end{Phonetics}
\end{Entry}

\begin{Entry}{詈骂}{12,9}{⾔、⾺}
  \begin{Phonetics}{詈骂}{li4ma4}
    \definition{v.}{xingar | abusar}
  \end{Phonetics}
\end{Entry}

\begin{Entry}{谢}{12}{⾔}
  \begin{Phonetics}{谢}{xie4}
    \definition*{s.}{Sobrenome Xie}
    \definition{v.}{agradecer | desculpar-se; pedir desculpas; admitir a própria culpa | recusar; declinar; renunciar | murchar; perder de flores ou folhas}
  \end{Phonetics}
\end{Entry}

\begin{Entry}{谢天谢地}{12,4,12,6}{⾔、⼤、⾔、⼟}
  \begin{Phonetics}{谢天谢地}{xie4tian1xie4di4}
    \definition{expr.}{agradecer a Deus | agradecer aos céus}
  \end{Phonetics}
\end{Entry}

\begin{Entry}{谢世}{12,5}{⾔、⼀}
  \begin{Phonetics}{谢世}{xie4shi4}
    \definition{v.}{morrer | falecer}
  \end{Phonetics}
\end{Entry}

\begin{Entry}{谢恩}{12,10}{⾔、⼼}
  \begin{Phonetics}{谢恩}{xie4'en1}
    \definition{v.}{agradecer a alguém pelo favor (especialmente imperador ou oficial superior)}
  \end{Phonetics}
\end{Entry}

\begin{Entry}{谢病}{12,10}{⾔、⽧}
  \begin{Phonetics}{谢病}{xie4bing4}
    \definition{v.}{desculpar-se por causa de doença}
  \end{Phonetics}
\end{Entry}

\begin{Entry}{谢媒}{12,12}{⾔、⼥}
  \begin{Phonetics}{谢媒}{xie4mei2}
    \definition{v.}{agradecer ao casamenteiro}
  \end{Phonetics}
\end{Entry}

\begin{Entry}{谢谢}{12,12}{⾔、⾔}
  \begin{Phonetics}{谢谢}{xie4xie5}[][HSK 1]
    \definition{interj.}{Obrigado!}
    \definition{v.}{agradecer; agradecer a gentileza dos outros}
  \end{Phonetics}
\end{Entry}

\begin{Entry}{谢意}{12,13}{⾔、⼼}
  \begin{Phonetics}{谢意}{xie4yi4}
    \definition{s.}{gratidão}
  \end{Phonetics}
\end{Entry}

\begin{Entry}{谦}{12}{⾔}
  \begin{Phonetics}{谦}{qian1}
    \definition*{s.}{Sobrenome Qian}
    \definition{adj.}{modesto}
    \definition{s.}{modéstia}
  \end{Phonetics}
\end{Entry}

\begin{Entry}{谦虚}{12,11}{⾔、⾌}
  \begin{Phonetics}{谦虚}{qian1xu1}[][HSK 6]
    \definition{adj.}{modesto; não se orgulhe de suas próprias conquistas e esteja disposto a aceitar críticas e opiniões de outras pessoas}
    \definition{v.}{falar modestamente; quando recebo elogios e cumprimentos de outras pessoas, sinto que não sou tão bom}
  \end{Phonetics}
\end{Entry}

\begin{Entry}{貂}{12}{⾘}
  \begin{Phonetics}{貂}{diao1}
    \definition*{s.}{Sobrenome Diao}
    \definition[只]{s.}{marta; fuinha; arminho}
  \end{Phonetics}
\end{Entry}

\begin{Entry}{赋}{12}{⾙}
  \begin{Phonetics}{赋}{fu4}
    \definition{s.}{dotação (natural) | Datado: imposto territorial | fu, estilo antigo, uma forma literária complexa que combina elementos de poesia e prosa, muito cultivada desde a época Han até o período das Seis Dinastias}
    \definition{v.}{compor (versos); escrever poemas e letras | Literário: conceder; entregar; dotar com}
  \end{Phonetics}
\end{Entry}

\begin{Entry}{赋予}{12,4}{⾙、⼅}
  \begin{Phonetics}{赋予}{fu4yu3}[][HSK 7-9]
    \definition{v.}{conceder; dotar (uma tarefa importante, missão, etc.); dar a alguém uma tarefa, responsabilidade, direito, autoridade, etc. | investir com (significado, característica, etc.); dar a algo uma cor, significado, importância, etc.}
  \end{Phonetics}
\end{Entry}

\begin{Entry}{赌}{12}{⾙}
  \begin{Phonetics}{赌}{du3}[][HSK 6]
    \definition{v.}{jogar | apostar; geralmente se refere à luta pela vitória ou derrota}
  \end{Phonetics}
\end{Entry}

\begin{Entry}{赌博}{12,12}{⾙、⼗}
  \begin{Phonetics}{赌博}{du3bo2}[][HSK 6]
    \definition{v.}{apostar; jogar; usar jogos de cartas, rolagem de dados, etc., para apostar dinheiro}
  \end{Phonetics}
\end{Entry}

\begin{Entry}{赏}{12}{⾙}
  \begin{Phonetics}{赏}{shang3}[][HSK 4]
    \definition*{s.}{Sobrenome Shang}
    \definition{s.}{recompensa; prêmio}
    \definition{v.}{conceder (outorgar) uma recompensa; recompensar; premiar | admirar; desfrutar; apreciar; valorizar}
  \end{Phonetics}
\end{Entry}

\begin{Entry}{赏心悦目}{12,4,10,5}{⾙、⼼、⼼、⽬}
  \begin{Phonetics}{赏心悦目}{shang3xin1yue4mu4}
    \definition{expr.}{``Aquece o coração e encanta os olhos.''; achar a paisagem agradável tanto aos olhos quanto à mente}
  \end{Phonetics}
\end{Entry}

\begin{Entry}{赏赐}{12,12}{⾙、⾙}
  \begin{Phonetics}{赏赐}{shang3ci4}
    \definition{s.}{recompensa | prêmio}
    \definition{v.}{recompensar | premiar}
  \end{Phonetics}
\end{Entry}

\begin{Entry}{赐}{12}{⾙}
  \begin{Phonetics}{赐}{ci4}[][HSK 7-9]
    \definition{s.}{favor; bênção; presente; uma recompensa, um benefício concedido}
    \definition{v.}{conceder; conferir; favorecer; presentear; dar, antigamente, se referia ao superior dando ao subordinado ou ao mais velho dando ao mais novo | responder; dar conselho; instruir; uma palavra usada para demonstrar respeito; quando alguém lhe dá instruções, responde ou lhe entrega algo}
  \end{Phonetics}
\end{Entry}

\begin{Entry}{赐教}{12,11}{⾙、⽁}
  \begin{Phonetics}{赐教}{ci4jiao4}[][HSK 7-9]
    \definition{v.}{condescender em ensinar; conceder instrução | importa-se em me esclarecer com suas instruções}
  \end{Phonetics}
\end{Entry}

\begin{Entry}{赔}{12}{⾙}
  \begin{Phonetics}{赔}{pei2}[][HSK 5]
    \definition{v.}{compensar; pagar por; indenizar | sofrer uma perda; fazer negócios e perder dinheiro | desculpar-se | suportar uma perda}
  \end{Phonetics}
\end{Entry}

\begin{Entry}{赔钱}{12,10}{⾙、⾦}
  \begin{Phonetics}{赔钱}{pei2/qian2}
    \definition{v.+compl.}{perder dinheiro | compensar; compensar com dinheiro os prejuízos causados a terceiros}
  \end{Phonetics}
\end{Entry}

\begin{Entry}{赔偿}{12,11}{⾙、⼈}
  \begin{Phonetics}{赔偿}{pei2chang2}[][HSK 5]
    \definition{v.}{indenizar; compensar; pagar por; indenizar outras pessoas ou grupos por perdas causadas por suas próprias ações}
  \end{Phonetics}
\end{Entry}

\begin{Entry}{趁}{12}{⾛}
  \begin{Phonetics}{趁}{chen4}[][HSK 7-9]
    \definition{prep.}{aproveitar-se de; tirar vantagem de (tempo, oportunidade, etc.); indica o tempo e as condições de uso}
    \definition{v.}{ser rico em; possuir}
  \end{Phonetics}
\end{Entry}

\begin{Entry}{趁早}{12,6}{⾛、⽇}
  \begin{Phonetics}{趁早}{chen4zao3}[][HSK 7-9]
    \definition{adv.}{o mais cedo possível; antes que seja tarde demais; na primeira oportunidade}
  \end{Phonetics}
\end{Entry}

\begin{Entry}{趁机}{12,6}{⾛、⽊}
  \begin{Phonetics}{趁机}{chen4ji1}[][HSK 7-9]
    \definition{adv./adv.}{aproveitar a ocasião; aproveitar a oportunidade}
  \end{Phonetics}
\end{Entry}

\begin{Entry}{趁着}{12,11}{⾛、⽬}
  \begin{Phonetics}{趁着}{chen4zhe5}[][HSK 7-9]
    \definition{v.}{aproveitar-se de; tirar vantagem de}
  \end{Phonetics}
\end{Entry}

\begin{Entry}{超}{12}{⾛}
  \begin{Phonetics}{超}{chao1}[][HSK 6]
    \definition{adj.}{super; extremamente; maior (ou menor) que o padrão geral}
    \definition{v.}{exceder; ultrapassar; vir para a frente por trás; prevalecer | transcender; ir além; não ser sujeito a certas restrições; ir além de um certo intervalo | exceder; superar; exceder o limite prescrito}
  \end{Phonetics}
\end{Entry}

\begin{Entry}{超车}{12,4}{⾛、⾞}
  \begin{Phonetics}{超车}{chao1/che1}[][HSK 7-9]
    \definition{v.+compl.}{ultrapassar (um veículo); passar por um veículo que trafega na mesma direção}
  \end{Phonetics}
\end{Entry}

\begin{Entry}{超出}{12,5}{⾛、⼐}
  \begin{Phonetics}{超出}{chao1 chu1}[][HSK 6]
    \definition{v.}{exceder; ultrapassar; ir além (de uma certa quantidade ou intervalo)}
  \end{Phonetics}
\end{Entry}

\begin{Entry}{超市}{12,5}{⾛、⼱}
  \begin{Phonetics}{超市}{chao1shi4}[][HSK 2]
    \definition[家]{s.}{supermercado; abreviação de 超级市场}
  \seealsoref{超级市场}{chao1 ji2 shi4 chang3}
  \end{Phonetics}
\end{Entry}

\begin{Entry}{超级}{12,6}{⾛、⽷}
  \begin{Phonetics}{超级}{chao1ji2}[][HSK 3]
    \definition{adj.}{super; além do nível geral}
    \definition{pref.}{super-; ultra-; hiper-}
  \end{Phonetics}
\end{Entry}

\begin{Entry}{超级市场}{12,6,5,6}{⾛、⽷、⼱、⼟}
  \begin{Phonetics}{超级市场}{chao1 ji2 shi4 chang3}
    \definition[个,间,所,家]{s.}{supermercado; hipermercado}
  \end{Phonetics}
\end{Entry}

\begin{Entry}{超过}{12,6}{⾛、⾡}
  \begin{Phonetics}{超过}{chao1guo4}[][HSK 2]
    \definition{v.}{ultrapassar; superar (algo ou alguém); passar de trás para a frente de alguém ou algo | exceder; ser mais do que; ultrapassar (um padrão)}
  \end{Phonetics}
\end{Entry}

\begin{Entry}{超声}{12,7}{⾛、⼠}
  \begin{Phonetics}{超声}{chao1sheng1}
    \definition{adj.}{ultrasônico}
    \definition{s.}{ultrasom}
  \end{Phonetics}
\end{Entry}

\begin{Entry}{超前}{12,9}{⾛、⼑}
  \begin{Phonetics}{超前}{chao1qian2}[][HSK 7-9]
    \definition{v.}{transcender os tempos; estar à frente dos tempos; estar à frente do seu tempo | superar os antecessores; liderar; assumir a liderança}
  \end{Phonetics}
\end{Entry}

\begin{Entry}{超标}{12,9}{⾛、⽊}
  \begin{Phonetics}{超标}{chao1/biao1}[][HSK 7-9]
    \definition{v.+compl.}{exceder uma cota (ou padrão); exceder o limite prescrito}
  \end{Phonetics}
\end{Entry}

\begin{Entry}{超速}{12,10}{⾛、⾡}
  \begin{Phonetics}{超速}{chao1su4}[][HSK 7-9]
    \definition{s.}{Física: hipervelocidade | excesso de velocidade}
    \definition{v.}{exceder o limite de velocidade}
  \end{Phonetics}
\end{Entry}

\begin{Entry}{超越}{12,12}{⾛、⾛}
  \begin{Phonetics}{超越}{chao1yue4}[][HSK 5]
    \definition{v.}{ultrapassar; superar; passar por cima; transcender}
  \end{Phonetics}
\end{Entry}

\begin{Entry}{越}{12}{⾛}
  \begin{Phonetics}{越}{yue4}[][HSK 2]
    \definition{adj.}{superior; excede ou ultrapassa o ordinário}
    \definition{adv.}{quanto mais\dots mais; sados juntos, eles formam o formato de "越……越……" para indicar que o grau de uma situação se torna mais sério à medida que se desenvolve; "成年……" para indicar que o grau de uma situação se torna mais sério à medida que o tempo passa}
    \definition{v.}{passar por cima; pular; cruzar | exceder; ultrapassar | estar em um tom alto; estar animado | saquear; pilhar; expoliar; apreender; roubar | passar; passar através; atravessar}
  \seealsoref{越来越……}{yue4 lai2 yue4}
  \seealsoref{越……越……}{yue4 yue4}
  \end{Phonetics}
\end{Entry}

\begin{Entry}{越来越……}{12,7,12}{⾛、⽊、⾛}
  \begin{Phonetics}{越来越……}{yue4 lai2 yue4}[][HSK 2]
    \definition{adv.}{cada vez mais\dots; isso significa que o grau de algo se aprofunda à medida que o tempo passa}
  \end{Phonetics}
\end{Entry}

\begin{Entry}{越……越……}{12,12}{⾛、⾛}
  \begin{Phonetics}{越……越……}{yue4 yue4}[][HSK 2]
    \definition{expr.}{quanto mais\dots tanto mais\dots}
  \end{Phonetics}
\end{Entry}

\begin{Entry}{越障}{12,13}{⾛、⾩}
  \begin{Phonetics}{越障}{yue4zhang4}
    \definition{s.}{curso com obstáculos para treinamento de tropas}
    \definition{v.}{superar obstáculos}
  \end{Phonetics}
\end{Entry}

\begin{Entry}{越境}{12,14}{⾛、⼟}
  \begin{Phonetics}{越境}{yue4jing4}
    \definition{v.}{cruzar uma fronteira (geralmente ilegalmente) | entrar ou sair furtivamente de um país}
  \end{Phonetics}
\end{Entry}

\begin{Entry}{趋}{12}{⾛}
  \begin{Phonetics}{趋}{qu1}
    \definition{v.}{apressar-se | tender para; tender a se tornar | (de um ganso, cobra, etc.) estalar a cabeça e morder as pessoas}
  \end{Phonetics}
\end{Entry}

\begin{Entry}{趋势}{12,8}{⾛、⼒}
  \begin{Phonetics}{趋势}{qu1shi4}[][HSK 4]
    \definition{s.}{tendência; tendência; direção; impulso das coisas que se movem em uma direção ou outra}
  \end{Phonetics}
\end{Entry}

\begin{Entry}{跌}{12}{⾜}
  \begin{Phonetics}{跌}{die1}[][HSK 6]
    \definition{s.}{(de um objeto, etc.) queda; tombo | (de preços, etc.) queda}
    \definition{v.}{cair; tombar; perder o equilíbrio e cair | cair (objetos caindo); descer | cair (queda de preços)}
  \end{Phonetics}
\end{Entry}

\begin{Entry}{跑}{12}{⾜}
  \begin{Phonetics}{跑}{pao2}
    \definition{v.}{(de animais) bater com a pata (no chão); (de animais) escavar o solo com suas garras ou cascos}
  \end{Phonetics}
  \begin{Phonetics}{跑}{pao3}[][HSK 1]
    \definition{v.}{correr; pessoas ou animais que se movem rapidamente para a frente com as pernas e os pés | caminhar; passear | fugir; escapar | correr de um lado para outro; fazer rondas; correr atrás de algo | de um líquido ou gás) vazar; evaporar | (como complemento de um verbo) fora; longe | participar de uma corrida}
  \end{Phonetics}
\end{Entry}

\begin{Entry}{跑马}{12,3}{⾜、⾺}
  \begin{Phonetics}{跑马}{pao3ma3}
    \definition{s.}{corrida de cavalos}
    \definition{v.}{andar a cavalo em ritmo acelerado}
  \end{Phonetics}
\end{Entry}

\begin{Entry}{跑步}{12,7}{⾜、⽌}
  \begin{Phonetics}{跑步}{pao3/bu4}[][HSK 3]
    \definition{v.+compl.}{correr; trotar}
  \end{Phonetics}
\end{Entry}

\begin{Entry}{跑肚}{12,7}{⾜、⾁}
  \begin{Phonetics}{跑肚}{pao3du4}
    \definition{v.}{(coloquial) ter diarréia}
  \end{Phonetics}
\end{Entry}

\begin{Entry}{跑调}{12,10}{⾜、⾔}
  \begin{Phonetics}{跑调}{pao3diao4}
    \definition{v.}{(coloquial) estar fora do tom ou desafinado (enquanto canta)}
  \end{Phonetics}
\end{Entry}

\begin{Entry}{跑掉}{12,11}{⾜、⼿}
  \begin{Phonetics}{跑掉}{pao3diao4}
    \definition{v.}{fugir}
  \end{Phonetics}
\end{Entry}

\begin{Entry}{跑腿}{12,13}{⾜、⾁}
  \begin{Phonetics}{跑腿}{pao3tui3}
    \definition{v.}{realizar tarefas}
  \end{Phonetics}
\end{Entry}

\begin{Entry}{跑酷}{12,14}{⾜、⾣}
  \begin{Phonetics}{跑酷}{pao3ku4}
    \definition*{s.}{Eempréstimo linguístico: Parkour}
  \end{Phonetics}
\end{Entry}

\begin{Entry}{跑题}{12,15}{⾜、⾴}
  \begin{Phonetics}{跑题}{pao3ti2}
    \definition{v.}{divagar | fugir do assunto | tergiversar}
  \end{Phonetics}
\end{Entry}

\begin{Entry}{辈}{12}{⾞}
  \begin{Phonetics}{辈}{bei4}[][HSK 5]
    \definition*{s.}{Sobrenome Bei}
    \definition{s.}{pessoas de um certo tipo; semelhantes | geração; geração na família | duração da vida | círculo familiar}
  \end{Phonetics}
\end{Entry}

\begin{Entry}{辉}{12}{⾞}
  \begin{Phonetics}{辉}{hui1}
    \definition{s.}{brilho; esplendor; fulgor}
    \definition{v.}{brilhar}
  \end{Phonetics}
\end{Entry}

\begin{Entry}{辉煌}{12,13}{⾞、⽕}
  \begin{Phonetics}{辉煌}{hui1huang2}[][HSK 7-9]
    \definition{adj.}{brilhante; esplêndido; deslumbrante  | brilhante; glorioso; descreve conquistas notáveis}
  \end{Phonetics}
\end{Entry}

\begin{Entry}{辜}{12}{⾟}
  \begin{Phonetics}{辜}{gu1}
    \definition*{s.}{Sobrenome Gu}
    \definition{s.}{culpa; crime}
  \end{Phonetics}
\end{Entry}

\begin{Entry}{辜负}{12,6}{⾟、⾙}
  \begin{Phonetics}{辜负}{gu1fu4}[][HSK 7-9]
    \definition{v.}{desapontar; decepcionar; ser indigno de; não corresponder a}
  \end{Phonetics}
\end{Entry}

\begin{Entry}{逼}{12}{⾡}
  \begin{Phonetics}{逼}{bi1}[][HSK 6]
    \definition{adj.}{estreito}
    \definition{v.}{forçar; pressionar; compelir | extorquir; pressionar por | fechar em; pressionar em direção a; aproximar-se}
  \end{Phonetics}
\end{Entry}

\begin{Entry}{逼近}{12,7}{⾡、⾡}
  \begin{Phonetics}{逼近}{bi1jin4}[][HSK 7-9]
    \definition{adj.}{aproximado (valor númerico)}
    \definition{s.}{aproximação (função matemática mais simples)}
    \definition{v.}{avançar em direção a; aproximar-se de; aproximar-se; aproximar-se | ganhar em (sobre); aglomerar-se em}
  \end{Phonetics}
\end{Entry}

\begin{Entry}{逼迫}{12,8}{⾡、⾡}
  \begin{Phonetics}{逼迫}{bi1po4}[][HSK 7-9]
    \definition{v.}{forçar; compelir; coagir | restringir; exercer pressão para induzir; forçar}
  \end{Phonetics}
\end{Entry}

\begin{Entry}{逼真}{12,10}{⾡、⼗}
  \begin{Phonetics}{逼真}{bi1zhen1}[][HSK 7-9]
    \definition{adj.}{fiel à realidade; realista; muito semelhante à coisa real | claro; distinto; verdadeiro}
  \end{Phonetics}
\end{Entry}

\begin{Entry}{遇}{12}{⾡}
  \begin{Phonetics}{遇}{yu4}[][HSK 4]
    \definition*{s.}{Sobrenome Yu}
    \definition{s.}{chance; oportunidade}
    \definition{v.}{encontrar; deparar-se com; encontrar-se | tratar; receber}
  \end{Phonetics}
\end{Entry}

\begin{Entry}{遇见}{12,4}{⾡、⾒}
  \begin{Phonetics}{遇见}{yu4 jian4}[][HSK 4]
    \definition{v.}{encontrar; deparar-se com}
  \end{Phonetics}
\end{Entry}

\begin{Entry}{遇到}{12,8}{⾡、⼑}
  \begin{Phonetics}{遇到}{yu4dao4}[][HSK 4]
    \definition{v.}{esbarrar em; encontrar; deparar-se com; conhecer alguém ou algo (inesperado)}
  \end{Phonetics}
\end{Entry}

\begin{Entry}{遍}{12}{⾡}
  \begin{Phonetics}{遍}{bian4}[][HSK 2]
    \definition{adv.}{por toda parte; em toda parte; em todos os lugares}
    \definition{clas.}{usado para a repetição de ações de leitura, fala ou escrita}
  \end{Phonetics}
\end{Entry}

\begin{Entry}{遍布}{12,5}{⾡、⼱}
  \begin{Phonetics}{遍布}{bian4bu4}[][HSK 7-9]
    \definition{v.}{encontrar em todos os lugares; espalhar por toda parte; distribuir em todos os lugares}
  \end{Phonetics}
\end{Entry}

\begin{Entry}{遍地}{12,6}{⾡、⼟}
  \begin{Phonetics}{遍地}{bian4 di4}[][HSK 6]
    \definition{adv.}{em todos os lugares; em toda parte; por toda parte}
  \end{Phonetics}
\end{Entry}

\begin{Entry}{遏}{12}{⾡}
  \begin{Phonetics}{遏}{e4}
    \definition{v.}{reprimir; restringir; reter; impedir; proibir}
  \end{Phonetics}
\end{Entry}

\begin{Entry}{遏制}{12,8}{⾡、⼑}
  \begin{Phonetics}{遏制}{e4zhi4}[][HSK 7-9]
    \definition{v.}{conter; restringir; controlar e prevenir ativamente o desenvolvimento de coisas que possam trazer perigo; usado principalmente para discutir tópicos formais}
  \end{Phonetics}
\end{Entry}

\begin{Entry}{道}{12}{⾡}
  \begin{Phonetics}{道}{dao4}[][HSK 2]
    \definition*{s.}{Taoismo;  Taoista | Sobrenome Dao}
    \definition{clas.}{usado para pratos em refeições, etapas em um procedimento, etc. | usado para certos objetos longos e estreitos; tira | usado para portas, paredes, etc.; pesado | usado para comandos, títulos, etc.}
    \definition[条]{s.}{estrada; caminho; trilha | curso; canal; o caminho percorrido pelo fluxo da água | maneira; método; princípio; raciocínio | moral; moralidade | habilidade; técnica | doutrina; princípio; sistema de pensamento acadêmico ou religioso; origem de todas as coisas no universo | taoísta; taoísmo; pertencente ao taoísmo | seita supersticiosa; certas organizações reacionárias e supersticiosas | linha; traços finos e alongados | trato; os canais dentro do corpo}
    \definition{v.}{dizer; falar; expressar-se | pensar; supor; considerar; acreditar que}
  \end{Phonetics}
\end{Entry}

\begin{Entry}{道行}{12,6}{⾡、⾏}
  \begin{Phonetics}{道行}{dao4 heng2}
    \definition{s.}{realizações de um monge budista ou sacerdote taoísta | habilidades; capacidades; aptidões | (figurativo) habilidade | habilidades adquiridas através da prática religiosa}
  \end{Phonetics}
\end{Entry}

\begin{Entry}{道具}{12,8}{⾡、⼋}
  \begin{Phonetics}{道具}{dao4ju4}[][HSK 7-9]
    \definition{s.}{adereços; objetos de cena; artigos de palco; objetos usados ​​em apresentações, como mesas e cadeiras, são chamados de grandes adereços, enquanto cigarros e xícaras de chá são chamados de pequenos adereços}
  \end{Phonetics}
\end{Entry}

\begin{Entry}{道教}{12,11}{⾡、⽁}
  \begin{Phonetics}{道教}{dao4 jiao4}[][HSK 6]
    \definition*{s.}{Taoísmo (sistema de crenças chinês)}
    \definition{s.}{a religião taoísta; taoísmo}
  \end{Phonetics}
\end{Entry}

\begin{Entry}{道理}{12,11}{⾡、⽟}
  \begin{Phonetics}{道理}{dao4li5}[][HSK 2]
    \definition[个,种]{s.}{verdade; princípio; a lei das coisas | sentido; razão}
  \end{Phonetics}
\end{Entry}

\begin{Entry}{道路}{12,13}{⾡、⾜}
  \begin{Phonetics}{道路}{dao4 lu4}[][HSK 2]
    \definition[条,段]{s.}{estrada; caminho; os canais de comunicação entre os dois lugares, incluindo terrestres e aquáticos | caminho; processo; refere-se à vida, à existência (significado abstrato)}
  \end{Phonetics}
\end{Entry}

\begin{Entry}{道歉}{12,14}{⾡、⽋}
  \begin{Phonetics}{道歉}{dao4/qian4}[][HSK 6]
    \definition{v.+compl.}{pedir desculpas; fazer um pedido de desculpas; dizer aos outros que você estava errado e pedir perdão}
  \end{Phonetics}
\end{Entry}

\begin{Entry}{道德}{12,15}{⾡、⼻}
  \begin{Phonetics}{道德}{dao4de2}[][HSK 5]
    \definition{adj.}{moral; descreve uma pessoa ou comportamento que atende aos requisitos morais; mais usado em situações negativas}
    \definition[种]{s.}{moral; ética; moralidade; regras e normas para que as pessoas vivam juntas e se comportem em comum}
  \end{Phonetics}
\end{Entry}

\begin{Entry}{遗}{12}{⾡}
  \begin{Phonetics}{遗}{yi2}
    \definition*{s.}{Sobrenome Yi}
    \definition{s.}{descarga involuntária de urina, etc. | algo perdido}
    \definition{v.}{perder | omitir | deixar para trás; guardar; não dar | deixar para trás após a morte; legar; transmitir}
  \end{Phonetics}
\end{Entry}

\begin{Entry}{遗产}{12,6}{⾡、⼇}
  \begin{Phonetics}{遗产}{yi2chan3}[][HSK 4]
    \definition[笔,份]{s.}{legado; herança; patrimônio; propriedade deixada pelo falecido | patrimônio; riqueza cultural ou riqueza material transmitida pela história}
  \end{Phonetics}
\end{Entry}

\begin{Entry}{遗传}{12,6}{⾡、⼈}
  \begin{Phonetics}{遗传}{yi2chuan2}[][HSK 4]
    \definition{v.}{herdar, descender, transmitir, passar adiante}
  \end{Phonetics}
\end{Entry}

\begin{Entry}{遗男}{12,7}{⾡、⽥}
  \begin{Phonetics}{遗男}{yi2nan2}
    \definition{s.}{órfão | filho póstumo}
  \end{Phonetics}
\end{Entry}

\begin{Entry}{遗迹}{12,9}{⾡、⾡}
  \begin{Phonetics}{遗迹}{yi2ji4}
    \definition{s.}{vestígio histórico; sítio; vestígio; traço; ruína; vestígios deixados por tempos antigos ou eras passadas}
  \end{Phonetics}
\end{Entry}

\begin{Entry}{遗案}{12,10}{⾡、⽊}
  \begin{Phonetics}{遗案}{yi2'an4}
    \definition{s.}{(lei) caso não resolvido}
  \end{Phonetics}
\end{Entry}

\begin{Entry}{遗落}{12,12}{⾡、⾋}
  \begin{Phonetics}{遗落}{yi2luo4}
    \definition{v.}{esquecer | deixar para trás (inadvertidamente) | deixar de fora | omitir}
  \end{Phonetics}
\end{Entry}

\begin{Entry}{遗嘱}{12,15}{⾡、⼝}
  \begin{Phonetics}{遗嘱}{yi2zhu3}
    \definition{s.}{testamento}
  \end{Phonetics}
\end{Entry}

\begin{Entry}{遗骸}{12,15}{⾡、⾻}
  \begin{Phonetics}{遗骸}{yi2hai2}
    \definition{v.}{restos mortais}
  \end{Phonetics}
\end{Entry}

\begin{Entry}{遗憾}{12,16}{⾡、⼼}
  \begin{Phonetics}{遗憾}{yi2han4}[][HSK 6]
    \definition{adj.}{triste; arrependido; contrito; sentir pena de situações que estão fora de controle ou são insatisfatórias}
    \definition{s.}{pena; arrependimento; sentindo pena que os desejos não se realizaram}
  \end{Phonetics}
\end{Entry}

\begin{Entry}{酢}{12}{⾣}
  \begin{Phonetics}{酢}{cu4}
    \definition{s.}{vinagre | (figurativo) ciúme (como em um caso de amor)}
    \variantof{醋}
  \end{Phonetics}
  \begin{Phonetics}{酢}{zuo4}
    \definition{s.}{brinde ao anfitrião feito pelo convidado}
  \end{Phonetics}
\end{Entry}

\begin{Entry}{酣}{12}{⾣}
  \begin{Phonetics}{酣}{han1}
    \definition{adj.}{intoxicado}
  \end{Phonetics}
\end{Entry}

\begin{Entry}{酣畅}{12,8}{⾣、⽥}
  \begin{Phonetics}{酣畅}{han1chang4}[][HSK 7-9]
    \definition{adj.}{alegre e animado (com bebida) | profundo (sono profundo)}
    \definition{adv.}{com facilidade e entusiasmo; totalmente; refere-se a obras literárias e artísticas}
  \end{Phonetics}
\end{Entry}

\begin{Entry}{酣睡}{12,13}{⾣、⽬}
  \begin{Phonetics}{酣睡}{han1shui4}[][HSK 7-9]
    \definition{v.}{dormir profundamente; estar em sono profundo | estar profundamente adormecido; cair em sono profundo}
  \end{Phonetics}
\end{Entry}

\begin{Entry}{量}{12}{⾥}
  \begin{Phonetics}{量}{liang2}[][HSK 4]
    \definition{v.}{medir | estimar; dimensionar}
  \end{Phonetics}
  \begin{Phonetics}{量}{liang4}
    \definition{s.}{instrumento de medida; antigamente, o termo se referia a objetos como baldes e litros, que medem o volume | capacidade (para tolerância ou ingestão de alimentos ou bebidas); refere-se ao limite do que pode ser acomodado | quantidade; valor; volume; número}
    \definition{v.}{estimar; medir; pesar}
  \end{Phonetics}
\end{Entry}

\begin{Entry}{铺}{12}{⾦}
  \begin{Phonetics}{铺}{pu1}[][HSK 6]
    \definition{clas.}{usado para kang, etc.; kang, uma plataforma de alvenaria ou de barro em uma extremidade de um cômodo, aquecida no inverno por fogueiras embaixo e coberta com esteiras para dormir}
    \definition{v.}{espalhar; estender; desdobrar | colocar; pavimentar}
  \end{Phonetics}
  \begin{Phonetics}{铺}{pu4}
    \definition{s.}{pequena loja; depósito | uma cama feita de tábuas de madeira; geralmente se refere a uma cama | estação de correios; antiga estação de correios (usada principalmente em nomes de lugares)}
  \end{Phonetics}
\end{Entry}

\begin{Entry}{铺垫}{12,9}{⾦、⼟}
  \begin{Phonetics}{铺垫}{pu1dian4}
    \definition{s.}{cobre leito | colcha | roupa de cama}
    \definition{v.}{pavimentar}
  \end{Phonetics}
\end{Entry}

\begin{Entry}{销}{12}{⾦}
  \begin{Phonetics}{销}{xiao1}
    \definition*{s.}{Sobrenome Xiao}
    \definition{s.}{gasto; despesa | pino}
    \definition{v.}{derreter (metal) | cancelar; anular | vender; comercializar | aferrolhar; fixar; prender; pregar | fixar com um parafuso; parafusar | gastar (consumo) | inserir um pino}
  \end{Phonetics}
\end{Entry}

\begin{Entry}{销售}{12,11}{⾦、⼝}
  \begin{Phonetics}{销售}{xiao1shou4}[][HSK 4]
    \definition{v.}{vender; comercializar}
  \end{Phonetics}
\end{Entry}

\begin{Entry}{锁}{12}{⾦}
  \begin{Phonetics}{锁}{suo3}[][HSK 5]
    \definition[把]{s.}{fechadura; dispositivo que impede que as pessoas abram facilmente a parte que se abre e fecha | correntes; cadeado e correntes | qualquer coisa com a forma de um cadeado antigo}
    \definition{v.}{trancar; trancar com chave | costurar com ponto fixo | tricotar}
  \end{Phonetics}
\end{Entry}

\begin{Entry}{锅}{12}{⾦}
  \begin{Phonetics}{锅}{guo1}[][HSK 5]
    \definition[口,个,只]{s.}{panela; frigideira; utensílios de cozinha, redondos e côncavos, feitos principalmente de ferro, alumínio, etc. | parte que se parece com um pote em alguns objetos}
  \end{Phonetics}
\end{Entry}

\begin{Entry}{锐}{12}{⾦}
  \begin{Phonetics}{锐}{rui4}
    \definition*{s.}{Sobrenome Rui}
    \definition{adj.}{afiado; aguçado (oposto a 钝) | agudo; perspicaz | rápido; ágil; veloz}
    \definition{adv.}{rapidamente; de ​​repente}
    \definition{s.}{vigor; espírito de luta | armas afiadas}
  \seealsoref{钝}{dun4}
  \end{Phonetics}
\end{Entry}

\begin{Entry}{阔}{12}{⾨}
  \begin{Phonetics}{阔}{kuo4}[][HSK 6]
    \definition{adj.}{amplo; amplo; vasto | rico | longo, no sentido de ``há muito tempo'' | vazio; impraticável}
  \end{Phonetics}
\end{Entry}

\begin{Entry}{隔}{12}{⾩}
  \begin{Phonetics}{隔}{ge2}[][HSK 4]
    \definition{adj.}{seguinte; vizinho}
    \definition{v.}{separar; cortar; dividir; particionar | estar a uma distância de, após ou em um intervalo de | ficar de pé ou deitar entre}
  \end{Phonetics}
\end{Entry}

\begin{Entry}{隔开}{12,4}{⾩、⼶}
  \begin{Phonetics}{隔开}{ge2 kai1}[][HSK 4]
    \definition{v.}{separar; manter separado; barricar; separar completamente duas pessoas (ou coisas) ou duas partes de uma coisa que estão intimamente unidas}
  \end{Phonetics}
\end{Entry}

\begin{Entry}{隔阂}{12,9}{⾩、⾨}
  \begin{Phonetics}{隔阂}{ge2he2}[][HSK 7-9]
    \definition[层,种,点]{s.}{estranhamento; mal-entendido; há uma falta de conexão emocional e uma distância de pensamento entre eles}
  \end{Phonetics}
\end{Entry}

\begin{Entry}{隔离}{12,10}{⾩、⼇}
  \begin{Phonetics}{隔离}{ge2li2}[][HSK 7-9]
    \definition{v.}{segregar; não permitir que as pessoas se reúnam, cortar o contato | isolar; colocar em quarentena; separar pessoas e animais com doenças infecciosas de pessoas e animais saudáveis ​​para evitar o contato}
  \end{Phonetics}
\end{Entry}

\begin{Entry}{隔壁}{12,16}{⾩、⼟}
  \begin{Phonetics}{隔壁}{ge2bi4}[][HSK 5]
    \definition{s.}{vizinho; casas ou pessoas vizinhas | septo; distante (socialmente distante) | anteparo; partição}
  \end{Phonetics}
\end{Entry}

\begin{Entry}{雄}{12}{⾫}
  \begin{Phonetics}{雄}{xiong2}
    \definition*{s.}{Sobrenome Xiong}
    \definition{adj.}{masculino | grandioso; imponente; audacioso | poderoso}
    \definition{s.}{uma pessoa ou país com grande poder e influência}
  \end{Phonetics}
\end{Entry}

\begin{Entry}{雄伟}{12,6}{⾫、⼈}
  \begin{Phonetics}{雄伟}{xiong2wei3}[][HSK 5]
    \definition{adj.}{magnífico; magnificente | imponente; magnífico}
  \end{Phonetics}
\end{Entry}

\begin{Entry}{集}{12}{⾫}
  \begin{Phonetics}{集}{ji2}[][HSK 6]
    \definition*{s.}{Sobrenome Ji}
    \definition{clas.}{parte; volume}
    \definition[个,本]{s.}{mercado; feira rural | coleção; conjunto; antologia | (matemática) conjunto}
    \definition{v.}{reunir; coletar; montar}
  \end{Phonetics}
\end{Entry}

\begin{Entry}{集中}{12,4}{⾫、⼁}
  \begin{Phonetics}{集中}{ji2zhong1}[][HSK 3]
    \definition{adj.}{centralizado; concentrado}
    \definition{v.}{concentrar; centralizar; focar; acumular; reunir (oposto de 分散) | reunir pessoas, coisas, forças, etc. dispersas; resumir opiniões, experiências, etc.}
  \seealsoref{分散}{fen1san4}
  \end{Phonetics}
\end{Entry}

\begin{Entry}{集会}{12,6}{⾫、⼈}
  \begin{Phonetics}{集会}{ji2hui4}[][HSK 7-9]
    \definition[个,次]{s.}{assembleia; reunião}
    \definition{v.}{reunir; reunir-se}
  \end{Phonetics}
\end{Entry}

\begin{Entry}{集合}{12,6}{⾫、⼝}
  \begin{Phonetics}{集合}{ji2he2}[][HSK 4]
    \definition{s.}{conjunto; montagem; coleção; agregação}
    \definition{v.}{reunir-se; juntar-se | reunir, juntar, convocar}
  \end{Phonetics}
\end{Entry}

\begin{Entry}{集团}{12,6}{⾫、⼞}
  \begin{Phonetics}{集团}{ji2tuan2}[][HSK 5]
    \definition[个,家,些]{s.}{anel; bloco; grupo; panelinha; círculo; grupo organizado para agir em conjunto com um determinado objetivo | grupo; entidade econômica com uma direção de negócios especializada, liderada por uma grande empresa com forte poder econômico e alta visibilidade, e formada pela combinação ou fusão de empresas relacionadas}
  \end{Phonetics}
\end{Entry}

\begin{Entry}{集体}{12,7}{⾫、⼈}
  \begin{Phonetics}{集体}{ji2ti3}[][HSK 3]
    \definition{s.}{coletivo; comunidade; grupo; equipe; organizações ou grupos em que muitas pessoas trabalham, estudam e vivem juntas}
  \end{Phonetics}
\end{Entry}

\begin{Entry}{集邮}{12,7}{⾫、⾢}
  \begin{Phonetics}{集邮}{ji2/you2}[][HSK 7-9]
    \definition[本]{s.}{filatelia; coleção de selos}
    \definition{v.+compl.}{colecionar selos}
  \end{Phonetics}
\end{Entry}

\begin{Entry}{集结}{12,9}{⾫、⽷}
  \begin{Phonetics}{集结}{ji2jie2}[][HSK 7-9]
    \definition{v.}{(especialmente tropas) reunir; concentrar; estabelecer; fortalecer}
  \end{Phonetics}
\end{Entry}

\begin{Entry}{集资}{12,10}{⾫、⾙}
  \begin{Phonetics}{集资}{ji2zi1}[][HSK 7-9]
    \definition{v.}{angariar fundos; recolher dinheiro; reunir recursos | arrecadar (reunir) dinheiro; concentrar fundos; retirar dinheiro (de muitas fontes); arrecadar fundos; solicitar fundos}
  \end{Phonetics}
\end{Entry}

\begin{Entry}{集装箱}{12,12,15}{⾫、⾐、⾋}
  \begin{Phonetics}{集装箱}{ji2zhuang1xiang1}[][HSK 7-9]
    \definition{s.}{\emph{container}}
  \end{Phonetics}
\end{Entry}

\begin{Entry}{雇}{12}{⾫}
  \begin{Phonetics}{雇}{gu4}[][HSK 7-9]
    \definition{v.}{contratar; empregar; pagar pessoas para fazerem coisas por você | contratar (transporte de aluguel)}
  \end{Phonetics}
\end{Entry}

\begin{Entry}{雇主}{12,5}{⾫、⼂}
  \begin{Phonetics}{雇主}{gu4zhu3}[][HSK 7-9]
    \definition[名]{s.}{empregador; uma pessoa que contrata trabalhadores, veículos ou barcos}
  \end{Phonetics}
\end{Entry}

\begin{Entry}{雇佣}{12,7}{⾫、⼈}
  \begin{Phonetics}{雇佣}{gu4yong1}[][HSK 7-9]
    \definition{v.}{contratar; empregar; comprar mão de obra com dinheiro}
  \end{Phonetics}
\end{Entry}

\begin{Entry}{雇员}{12,7}{⾫、⼝}
  \begin{Phonetics}{雇员}{gu4yuan2}[][HSK 7-9]
    \definition[名,位,个]{s.}{empregado; servo; pessoal contratado ou temporário fora do estabelecimento}
  \end{Phonetics}
\end{Entry}

\begin{Entry}{韩}{12}{⾱}
  \begin{Phonetics}{韩}{han2}
    \definition*{s.}{Um estado durante o Período dos Estados Combatentes nas atuais províncias centrais de Henan e sudeste de Shanxi | O nome de um estado feudal durante a dinastia Zhou, localizado no que hoje é o nordeste de Hejin, província de Shanxi | Coreia do Sul, abreviação de 韩国; República da Coreia (RC) | Sobrenome Han}
  \seealsoref{韩国}{han2guo2}
  \end{Phonetics}
\end{Entry}

\begin{Entry}{韩国}{12,8}{⾱、⼞}
  \begin{Phonetics}{韩国}{han2guo2}
    \definition*{s.}{Coréia do Sul; República da Coreia}
  \end{Phonetics}
\end{Entry}

\begin{Entry}{韩国人}{12,8,2}{⾱、⼞、⼈}
  \begin{Phonetics}{韩国人}{han2guo2ren2}
    \definition{s.}{coreano | pessoa ou povo da Coréia}
  \end{Phonetics}
\end{Entry}

\begin{Entry}{馋}{12}{⾷}
  \begin{Phonetics}{馋}{chan2}[][HSK 7-9]
    \definition{adj.}{cobiçoso; invejoso | guloso; comilão; glutão}
    \definition{v.}{desejar comida; querer comer (alguma coisa)}
  \end{Phonetics}
\end{Entry}

\begin{Entry}{骗}{12}{⾺}
  \begin{Phonetics}{骗}{pian4}[][HSK 5]
    \definition{v.}{enganar; trapacear; iludir; ludibriar; usar mentiras ou meios fraudulentos para fazer alguém acreditar ou ser enganado | enganar; fraudar | montar (um cavalo); balançar (ou saltar) para a sela}
  \end{Phonetics}
\end{Entry}

\begin{Entry}{骗子}{12,3}{⾺、⼦}
  \begin{Phonetics}{骗子}{pian4 zi5}[][HSK 5]
    \definition[个]{s.}{trapaceiro; vigarista; fraudador; impostor; golpista; pessoa que obtém bens de forma fraudulenta}
  \end{Phonetics}
\end{Entry}

\begin{Entry}{骚}{12}{⾺}
  \begin{Phonetics}{骚}{sao1}
    \definition*{s.}{Abreviação de Li Sao (Encontrando a Tristeza), um poema do poeta e estadista do século IV a.C. Qu Yuan (屈原)}
    \definition{adj.}{coquete; (de uma mulher) lasciva | masculino (de alguns animais domésticos)}
    \definition{s.}{escritos literários; geralmente se refere à poesia | o cheiro de urina; mau cheiro}
    \definition{v.}{perturbar}
  \seealsoref{屈原}{qu1yuan2}
  \end{Phonetics}
\end{Entry}

\begin{Entry}{骚乱}{12,7}{⾺、⼄}
  \begin{Phonetics}{骚乱}{sao1luan4}
    \definition{s.}{rebelião | perturbação | tumulto}
    \definition{v.}{criar um distúrbio}
  \end{Phonetics}
\end{Entry}

\begin{Entry}{鹅}{12}{⿃}
  \begin{Phonetics}{鹅}{e2}[][HSK 7-9]
    \definition[只,群]{s.}{ganso}
  \end{Phonetics}
\end{Entry}

\begin{Entry}{黍}{12}{⿉}[Kangxi 202]
  \begin{Phonetics}{黍}{shu3}
    \definition{s.}{painço}
  \end{Phonetics}
\end{Entry}

\begin{Entry}{黑}{12}{⿊}[Kangxi 203]
  \begin{Phonetics}{黑}{hei1}[][HSK 2]
    \definition*{s.}{Província de Heilongjiang, abreviação de 黑龙江 | Sobrenome Hei}
    \definition{adj.}{preto; cor semelhante à do carvão | escuro | obscuro; secreto | perverso; sinistro; ruim; cruel | reacionário}
    \definition{s.}{noite}
    \definition{v.}{fazer algo ilegalmente ou de forma desonesta; enganar; desviar dinheiro ilegalmente | invadir (uma rede, sites, computador, etc.)}
  \seealsoref{黑龙江}{hei1long2jiang1}
  \end{Phonetics}
\end{Entry}

\begin{Entry}{黑马}{12,3}{⿊、⾺}
  \begin{Phonetics}{黑马}{hei1ma3}[][HSK 7-9]
    \definition[匹,群]{s.}{azarão (cavalo preto) | Figurativo: pessoa pouco conhecida que alcança sucesso inesperado}
  \end{Phonetics}
\end{Entry}

\begin{Entry}{黑心}{12,4}{⿊、⼼}
  \begin{Phonetics}{黑心}{hei1xin1}[][HSK 7-9]
    \definition{adj.}{malvado; perverso | ganancioso; avarento | (certos bens) de má qualidade | implacável e sem consciência | de mente viciosa cheia de ódio e ciúme}
    \definition{s.}{coração negro; mente maligna | núcleo preto (falha na cerâmica)}
  \end{Phonetics}
\end{Entry}

\begin{Entry}{黑手}{12,4}{⿊、⼿}
  \begin{Phonetics}{黑手}{hei1shou3}[][HSK 7-9]
    \definition{s.}{mão negra; manipulador maligno dos bastidores | uma pessoa cruel manipulando alguém ou algo nos bastidores; uma metáfora para pessoas ou forças que secretamente realizam atividades de conspiração}
  \end{Phonetics}
\end{Entry}

\begin{Entry}{黑白}{12,5}{⿊、⽩}
  \begin{Phonetics}{黑白}{hei1bai2}[][HSK 7-9]
    \definition[只]{s.}{preto e branco | certo e errado; metáfora para o certo e o errado, o bem e o mal}
  \end{Phonetics}
\end{Entry}

\begin{Entry}{黑龙江}{12,5,6}{⿊、⿓、⽔}
  \begin{Phonetics}{黑龙江}{hei1long2jiang1}
    \definition*{s.}{Província de Heilongjiang | Rio Heilong Jiang;  Rio Amur (na Rússia)}
  \end{Phonetics}
\end{Entry}

\begin{Entry}{黑色}{12,6}{⿊、⾊}
  \begin{Phonetics}{黑色}{hei1 se4}[][HSK 2]
    \definition{adj.}{metafórico: suspeito, ilegal}
    \definition{s.}{cor preta}
  \end{Phonetics}
\end{Entry}

\begin{Entry}{黑夜}{12,8}{⿊、⼣}
  \begin{Phonetics}{黑夜}{hei1 ye4}[][HSK 6]
    \definition[个]{s.}{noite ; uma noite muito escura sem luz}
  \end{Phonetics}
\end{Entry}

\begin{Entry}{黑板}{12,8}{⿊、⽊}
  \begin{Phonetics}{黑板}{hei1ban3}[][HSK 2]
    \definition[块,个]{s.}{quadro negro; quadro de giz; uma placa, na qual se pode escrever com giz}
  \end{Phonetics}
\end{Entry}

\begin{Entry}{黑客}{12,9}{⿊、⼧}
  \begin{Phonetics}{黑客}{hei1ke4}[][HSK 7-9]
    \definition[个,些,位,名]{s.}{Empréstimo linguístico: \emph{hacker}; \emph{cracker}; intruso cibernético; gênio da computação; originalmente se refere a pessoas que não são profissionais de informática, mas são muito proficientes em tecnologia de computadores; agora se refere especificamente a pessoas que podem escrever programas de descriptografia para invadir ilegalmente redes de computadores de outras pessoas para interferir ou destruí-las}
  \end{Phonetics}
\end{Entry}

\begin{Entry}{黑桃}{12,10}{⿊、⽊}
  \begin{Phonetics}{黑桃}{hei1 tao2}
    \definition{s.}{espadas ♠ (em jogos de cartas)}
  \seealsoref{方片}{fang1 pian4}
  \seealsoref{红心}{hong2 xin1}
  \seealsoref{梅花}{mei2 hua1}
  \end{Phonetics}
\end{Entry}

\begin{Entry}{黑暗}{12,13}{⿊、⽇}
  \begin{Phonetics}{黑暗}{hei1 an4}[][HSK 4]
    \definition{adj.}{escuro; sombrio; sem luz | maligno; corrupto; reacionário}
  \end{Phonetics}
\end{Entry}

\begin{Entry}{黹}{12}{⿋}[Kangxi 204]
  \begin{Phonetics}{黹}{zhi3}
    \definition{v.}{costurar; bordar}
  \end{Phonetics}
\end{Entry}

\begin{Entry}{鼎}{12}{⿍}[Kangxi 206]
  \begin{Phonetics}{鼎}{ding3}
    \definition{adj.}{grande; generoso | importante; grandioso}
    \definition{adv.}{exatamente quando; exatamente o momento para}
    \definition[尊]{s.}{um antigo recipiente de cozinha com duas alças e três ou quatro pernas | pote; caldeirão | poder do estado; o trono | como símbolo de dinastia; nos tempos antigos, era considerada uma ferramenta importante para estabelecer um país}
  \end{Phonetics}
\end{Entry}

%%%%% EOF %%%%%

