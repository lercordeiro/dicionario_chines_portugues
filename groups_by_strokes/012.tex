%%%
%%% 12画
%%%

\section*{12画}\addcontentsline{toc}{section}{12画}

\begin{entry}{傢具}{12,8}
  \begin{phonetics}{傢具}{jia1ju4}
    \variantof{家具}
  \end{phonetics}
\end{entry}

\begin{entry}{博文}{12,4}
  \begin{phonetics}{博文}{bo2wen2}
    \definition{s.}{artigo em um blog}
    \definition{v.}{escrever um artigo em um blog}
  \end{phonetics}
\end{entry}

\begin{entry}{博主}{12,5}
  \begin{phonetics}{博主}{bo2zhu3}
    \definition{s.}{blogueiro}
  \end{phonetics}
\end{entry}

\begin{entry}{博物馆}{12,8,11}
  \begin{phonetics}{博物馆}{bo2wu4guan3}
    \definition{s.}{museu}
  \end{phonetics}
\end{entry}

\begin{entry}{厨房}{12,8}
  \begin{phonetics}{厨房}{chu2fang2}
    \definition[间]{s.}{cozinha}
  \end{phonetics}
\end{entry}

\begin{entry}{喂}{12}[Radical 口]
  \begin{phonetics}{喂}{wei2}
    \definition{interj.}{Alô!,  Olá! quando respondendo a um telefonema}
  \end{phonetics}
  \begin{phonetics}{喂}{wei4}
    \definition{interj.}{Ei!, para chamar atenção}
    \definition{v.}{alimentar | alimentar (um animal, bebê, inválido, etc.)}
  \end{phonetics}
\end{entry}

\begin{entry}{喂奶}{12,5}
  \begin{phonetics}{喂奶}{wei4nai3}
    \definition{v.}{amamentar}
  \end{phonetics}
\end{entry}

\begin{entry}{喂母乳}{12,5,8}
  \begin{phonetics}{喂母乳}{wei4mu3ru3}
    \definition{s.}{amamentação}
  \end{phonetics}
\end{entry}

\begin{entry}{喂养}{12,9}
  \begin{phonetics}{喂养}{wei4yang3}
    \definition{v.}{alimentar (uma criança, animal doméstico, etc.) | manter | criar (um animal)}
  \end{phonetics}
\end{entry}

\begin{entry}{喂食}{12,9}
  \begin{phonetics}{喂食}{wei4shi2}
    \definition{v.}{alimentar}
  \end{phonetics}
\end{entry}

\begin{entry}{喂哺}{12,10}
  \begin{phonetics}{喂哺}{wei4bu3}
    \definition{v.}{alimentar (um bebê)}
  \end{phonetics}
\end{entry}

\begin{entry}{喂料}{12,10}
  \begin{phonetics}{喂料}{wei4liao4}
    \definition{v.}{alimentar (também no sentido figurativo)}
  \end{phonetics}
\end{entry}

\begin{entry}{善意}{12,13}
  \begin{phonetics}{善意}{shan4yi4}
    \definition{s.}{boa vontade | benevolência | bondade}
  \end{phonetics}
\end{entry}

\begin{entry}{喔}{12}[Radical 口]
  \begin{phonetics}{喔}{o1}
    \definition{interj.}{Oh!, Entendi!, usado para indicar realização, compreensão}
  \end{phonetics}
\end{entry}

\begin{entry}{喜欢}{12,6}
  \begin{phonetics}{喜欢}{xi3huan5}
    \definition{v.}{gostar}
  \end{phonetics}
\end{entry}

\begin{entry}{喜剧}{12,10}
  \begin{phonetics}{喜剧}{xi3ju4}
    \definition[部,出]{s.}{uma comédia}
  \end{phonetics}
\end{entry}

\begin{entry}{喝}{12}[Radical 口]
  \begin{phonetics}{喝}{he1}
    \definition{interj.}{Meu Deus!}
    \definition{v.}{beber}
  \end{phonetics}
  \begin{phonetics}{喝}{he4}
    \definition{v.}{gritar bem alto}
  \end{phonetics}
\end{entry}

\begin{entry}{喝彩}{12,11}
  \begin{phonetics}{喝彩}{he4cai3}
    \definition{s.}{aclamar | torcer}
  \end{phonetics}
\end{entry}

\begin{entry}{喝醉}{12,15}
  \begin{phonetics}{喝醉}{he1zui4}
    \definition{v.}{ficar bêbado}
  \end{phonetics}
\end{entry}

\begin{entry}{喻}{12}[Radical 口]
  \begin{phonetics}{喻}{yu4}
    \definition{s.}{analogia | símile | metáfora | alegoria}
    \definition{v.}{descrever algo como}
  \end{phonetics}
\end{entry}

\begin{entry}{堤坝}{12,7}
  \begin{phonetics}{堤坝}{di1ba4}
    \definition{s.}{represa | dique | barragem}
  \end{phonetics}
\end{entry}

\begin{entry}{奥}{12}[Radical 大]
  \begin{phonetics}{奥}{ao4}
    \definition{adj.}{obscuro | misterioso}
  \end{phonetics}
\end{entry}

\begin{entry}{奥运}{12,7}
  \begin{phonetics}{奥运}{ao4yun4}
    \definition*{s.}{Jogos Olímpicos, Olimpíadas, abreviação de 奥林匹克运动会}
  \seealsoref{奥林匹克运动会}{ao4lin2pi3ke4 yun4dong4hui4}
  \end{phonetics}
\end{entry}

\begin{entry}{奥运会}{12,7,6}
  \begin{phonetics}{奥运会}{ao4yun4hui4}
    \definition*{s.}{Jogos Olímpicos, Olimpíadas, abreviação de 奥林匹克运动会}
  \seealsoref{奥林匹克运动会}{ao4lin2pi3ke4 yun4dong4hui4}
  \end{phonetics}
\end{entry}

\begin{entry}{奥林匹克运动会}{12,8,4,7,7,6,6}
  \begin{phonetics}{奥林匹克运动会}{ao4lin2pi3ke4 yun4dong4hui4}
    \definition*{s.}{Jogos Olímpicos, Olimpíadas}
  \end{phonetics}
\end{entry}

\begin{entry}{奥特曼}{12,10,11}
  \begin{phonetics}{奥特曼}{ao4te4man4}
    \definition*{s.}{\emph{Ultraman},  super-herói de ficção científica japonesa}
  \end{phonetics}
\end{entry}

\begin{entry}{嫂子}{12,3}
  \begin{phonetics}{嫂子}{sao3zi5}
    \definition{s.}{esposa do irmão mais velho}
  \end{phonetics}
\end{entry}

\begin{entry}{寓意}{12,13}
  \begin{phonetics}{寓意}{yu4yi4}
    \definition{s.}{moral (de uma história),  lição a ser aprendida, implicação, mensagem, significado metafórico}
  \end{phonetics}
\end{entry}

\begin{entry}{就}{12}[Radical 尢]
  \begin{phonetics}{就}{jiu4}
    \definition{adv.}{exatamente | justamente}
    \definition{v.}{realizar | se envolver em | acompanhar (em alimentos) | aproveitar | avançar | empreender}
  \end{phonetics}
\end{entry}

\begin{entry}{就业}{12,5}
  \begin{phonetics}{就业}{jiu4ye4}
    \definition{v.+compl.}{obter emprego | assumir uma ocupação | conseguir um emprego}
  \end{phonetics}
\end{entry}

\begin{entry}{就是}{12,9}
  \begin{phonetics}{就是}{jiu4shi4}
    \definition{adv.}{exatamente | precisamente | apenas | simplesmente | (usado correlativamente com 也) mesmo, mesmo se}
  \end{phonetics}
\end{entry}

\begin{entry}{就职}{12,11}
  \begin{phonetics}{就职}{jiu4zhi2}
    \definition{v.}{assumir o cargo | assumir um posto}
  \end{phonetics}
\end{entry}

\begin{entry}{属}{12}[Radical 尸]
  \begin{phonetics}{属}{shu3}
    \definition{s.}{categoria | gênero (taxonomia) | familiares | dependentes}
    \definition{v.}{pertencer | subordinar | nascer no ano do signo de (um dos doze animais zodiacais) | provar ser | constituir}
  \end{phonetics}
  \begin{phonetics}{属}{zhu3}
    \definition{v.}{juntar-se | fixar a atenção em | concentrar-se em}
  \end{phonetics}
\end{entry}

\begin{entry}{属于}{12,3}
  \begin{phonetics}{属于}{shu3yu2}
    \definition{v.}{ser classificado como | pertencer a | fazer parte de}
  \end{phonetics}
\end{entry}

\begin{entry}{屡次}{12,6}
  \begin{phonetics}{屡次}{lv3ci4}
    \definition{adv.}{repetidamente | uma e outra vez | muitas vezes}
  \end{phonetics}
\end{entry}

\begin{entry}{帽子}{12,3}
  \begin{phonetics}{帽子}{mao4zi5}
    \definition[顶]{s.}{chapéu | boné | (figurativo) rótulo}
  \end{phonetics}
\end{entry}

\begin{entry}{强}{12}[Radical 弓]
  \begin{phonetics}{强}{jiang4}
    \definition{adj.}{teimoso | inflexível}
  \end{phonetics}
  \begin{phonetics}{强}{qiang2}
    \definition*{s.}{sobrenome Qiang}
    \definition{adj.}{melhor em sua categoria | melhor | poderoso | forte | vigoroso | violento}
  \end{phonetics}
  \begin{phonetics}{强}{qiang3}
    \definition{v.}{obrigar | forçar | fazer um esforço | esforçar-se}
  \end{phonetics}
\end{entry}

\begin{entry}{惑星}{12,9}
  \begin{phonetics}{惑星}{huo4xing1}
    \definition{s.}{planeta}
  \seealsoref{行星}{xing2xing1}
  \end{phonetics}
\end{entry}

\begin{entry}{惩处}{12,5}
  \begin{phonetics}{惩处}{cheng2chu3}
    \definition{v.}{administrar justiça | punir}
  \end{phonetics}
\end{entry}

\begin{entry}{惩罚}{12,9}
  \begin{phonetics}{惩罚}{cheng2fa2}
    \definition{v.}{punir | penalizar}
  \end{phonetics}
\end{entry}

\begin{entry}{愉快}{12,7}
  \begin{phonetics}{愉快}{yu2kuai4}
    \definition{adj.}{alegre | delicioso | prazeroso | agradável | feliz | encantado}
    \definition{adv.}{alegremente | agradavelmente}
  \end{phonetics}
\end{entry}

\begin{entry}{愤世嫉俗}{12,5,13,9}
  \begin{phonetics}{愤世嫉俗}{fen4shi4ji2su2}
    \definition{v.}{ser cínico | ser amargurado}
  \end{phonetics}
\end{entry}

\begin{entry}{愤怒}{12,9}
  \begin{phonetics}{愤怒}{fen4nu4}
    \definition{adj.}{zangado | indignado}
    \definition{s.}{ira}
  \end{phonetics}
\end{entry}

\begin{entry}{掌}{12}[Radical 手]
  \begin{phonetics}{掌}{zhang3}
    \definition{s.}{palma da mão | sola do pé | pata | ferradura}
    \definition{v.}{dar um tapa | segurar na mão | empunhar}
  \end{phonetics}
\end{entry}

\begin{entry}{掱}{12}[Radical 手]
  \begin{phonetics}{掱}{shou3}
    \variantof{手}
  \end{phonetics}
\end{entry}

\begin{entry}{揉}{12}[Radical 手]
  \begin{phonetics}{揉}{rou2}
    \definition{v.}{amassar | massagear | esfregar}
  \end{phonetics}
\end{entry}

\begin{entry}{揉碎}{12,13}
  \begin{phonetics}{揉碎}{rou2sui4}
    \definition{v.}{esmagar | desintegrar-se em pedaços}
  \end{phonetics}
\end{entry}

\begin{entry}{提及}{12,3}
  \begin{phonetics}{提及}{ti2ji2}
    \definition{v.}{mencionar | levantar (um assunto) | chamar a atenção de alguém}
  \end{phonetics}
\end{entry}

\begin{entry}{提升}{12,4}
  \begin{phonetics}{提升}{ti2sheng1}
    \definition{v.}{promover (para uma posição de classificação mais alta) | levantar | içar | (figurativo) elevar, levantar, melhorar}
  \end{phonetics}
\end{entry}

\begin{entry}{提高}{12,10}
  \begin{phonetics}{提高}{ti2gao1}
    \definition{v.}{melhorar | aumentar | elevar}
  \end{phonetics}
\end{entry}

\begin{entry}{插手}{12,4}
  \begin{phonetics}{插手}{cha1shou3}
    \definition{v.+compl.}{envolver-se em | dar uma mão | ter (tomar) uma mão | cutucar o nariz de alguém | intrometer-se}
  \end{phonetics}
\end{entry}

\begin{entry}{插话}{12,8}
  \begin{phonetics}{插话}{cha1hua4}
    \definition{s.}{interrupção | digressão}
    \definition{v.+compl.}{interromper (a fala de alguém)}
  \end{phonetics}
\end{entry}

\begin{entry}{握手}{12,4}
  \begin{phonetics}{握手}{wo4shou3}
    \definition{v.+compl.}{apertar as mãos}
  \end{phonetics}
\end{entry}

\begin{entry}{援助}{12,7}
  \begin{phonetics}{援助}{yuan2zhu4}
    \definition{s.}{assistência}
    \definition{v.}{ajudar | apoiar | assistir}
  \end{phonetics}
\end{entry}

\begin{entry}{搁浅}{12,8}
  \begin{phonetics}{搁浅}{ge1qian3}
    \definition{v.}{ficar encalhado (navio) | encalhar | (figurativo) encontrar dificuldades e parar}
  \end{phonetics}
\end{entry}

\begin{entry}{搓}{12}[Radical 手]
  \begin{phonetics}{搓}{cuo1}
    \definition{s.}{torção}
    \definition{v.}{esfregar ou rolar entre as mãos ou dedos | torcer}
  \end{phonetics}
\end{entry}

\begin{entry}{搭讪}{12,5}
  \begin{phonetics}{搭讪}{da1shan4}
    \definition{v.}{bater em alguém | incitar uma conversa | começar a conversar para acabar com um silêncio constrangedor ou uma situação embaraçosa}
  \end{phonetics}
\end{entry}

\begin{entry}{搭配}{12,10}
  \begin{phonetics}{搭配}{da1pei4}
    \definition{v.}{emparelhar | combinar | organizar em pares | adicionar alguém em um grupo}
  \end{phonetics}
\end{entry}

\begin{entry}{散}{12}[Radical 攴]
  \begin{phonetics}{散}{san3}
    \definition{adj.}{disseminado | dispersado | solto}
    \definition{s.}{medicamento em pó}
    \definition{v.}{soltar-se | desmoronar}
  \end{phonetics}
  \begin{phonetics}{散}{san4}
    \definition{v.}{terminar (uma reunião, etc.) | dispersar | disseminar | dissipar}
  \end{phonetics}
\end{entry}

\begin{entry}{散心}{12,4}
  \begin{phonetics}{散心}{san4xin1}
    \definition{v.+compl.}{aliviar o tédio | desfrutar de uma diversão | estar despreocupado}
  \end{phonetics}
\end{entry}

\begin{entry}{散步}{12,7}
  \begin{phonetics}{散步}{san4bu4}
    \definition{v.+compl.}{dar um passeio | passear | dar uma caminhada}
  \end{phonetics}
\end{entry}

\begin{entry}{敬礼}{12,5}
  \begin{phonetics}{敬礼}{jing4li3}
    \definition{s.}{saudação}
    \definition{v.}{saudar}
  \end{phonetics}
\end{entry}

\begin{entry}{斯巴达}{12,4,6}
  \begin{phonetics}{斯巴达}{si1ba1da2}
    \definition*{s.}{Esparta}
  \end{phonetics}
\end{entry}

\begin{entry}{普通话}{12,10,8}
  \begin{phonetics}{普通话}{pu3tong1hua4}
    \definition*{s.}{Mandarim (literalmente ``linguagem comum'') | Putonghua (fala comum da língua chinesa) | discurso comum}
  \end{phonetics}
\end{entry}

\begin{entry}{景色}{12,6}
  \begin{phonetics}{景色}{jing3se4}
    \definition{s.}{paisagem | panorama | vista}
  \end{phonetics}
\end{entry}

\begin{entry}{智商}{12,11}
  \begin{phonetics}{智商}{zhi4shang1}
    \definition{s.}{quociente de inteligência, QI}
  \end{phonetics}
\end{entry}

\begin{entry}{智障}{12,13}
  \begin{phonetics}{智障}{zhi4zhang4}
    \definition{adj./s.}{retardado}
  \end{phonetics}
\end{entry}

\begin{entry}{智慧}{12,15}
  \begin{phonetics}{智慧}{zhi4hui4}
    \definition{s.}{sabedoria | inteligência}
  \end{phonetics}
\end{entry}

\begin{entry}{暑假}{12,11}
  \begin{phonetics}{暑假}{shu3jia4}
    \definition[个]{s.}{férias de verão}
  \end{phonetics}
\end{entry}

\begin{entry}{曾经}{12,8}
  \begin{phonetics}{曾经}{ceng2jing1}
    \definition{adv.}{uma vez | antes | costumava | no passado}
  \end{phonetics}
\end{entry}

\begin{entry}{最}{12}[Radical 冂]
  \begin{phonetics}{最}{zui4}
    \definition{adv.}{o mais | o melhor | a coisa mais\dots | grau superlativo relativo de superioridade}
  \end{phonetics}
\end{entry}

\begin{entry}{最少}{12,4}
  \begin{phonetics}{最少}{zui4shao3}
    \definition{adv.}{finalmente}
  \end{phonetics}
\end{entry}

\begin{entry}{最优}{12,6}
  \begin{phonetics}{最优}{zui4you1}
    \definition{adj.}{ótimo}
  \end{phonetics}
\end{entry}

\begin{entry}{最先}{12,6}
  \begin{phonetics}{最先}{zui4xian1}
    \definition{adv.}{o primeiro}
  \end{phonetics}
\end{entry}

\begin{entry}{最后}{12,6}
  \begin{phonetics}{最后}{zui4hou4}
    \definition{adj.}{final | último}
    \definition{adv.}{finalmente}
  \end{phonetics}
\end{entry}

\begin{entry}{最多}{12,6}
  \begin{phonetics}{最多}{zui4duo1}
    \definition{adv.}{no máximo | máximo}
  \end{phonetics}
\end{entry}

\begin{entry}{最好}{12,6}
  \begin{phonetics}{最好}{zui4hao3}
    \definition{adv.}{ser melhor que}
    \definition{v.}{(você) estar melhor (faça o que sugerimos) | querer ser o melhor}
  \end{phonetics}
\end{entry}

\begin{entry}{最初}{12,7}
  \begin{phonetics}{最初}{zui4chu1}
    \definition{adj.}{inicial | original | primário}
    \definition{adv.}{inicialmente | originalmente}
  \end{phonetics}
\end{entry}

\begin{entry}{最近}{12,7}
  \begin{phonetics}{最近}{zui4jin4}
    \definition{adv.}{ultimamente | recentemente}
  \end{phonetics}
\end{entry}

\begin{entry}{最远}{12,7}
  \begin{phonetics}{最远}{zui4yuan3}
    \definition{adv.}{mais distante | mais longe}
  \end{phonetics}
\end{entry}

\begin{entry}{最佳}{12,8}
  \begin{phonetics}{最佳}{zui4jia1}
    \definition{adj.}{melhor (atleta, filme etc) | ótimo}
  \end{phonetics}
\end{entry}

\begin{entry}{最终}{12,8}
  \begin{phonetics}{最终}{zui4zhong1}
    \definition{adv.}{pelo menos | finalmente}
    \definition{s.}{final | ultimato}
  \end{phonetics}
\end{entry}

\begin{entry}{最高}{12,10}
  \begin{phonetics}{最高}{zui4gao1}
    \definition{adj.}{altíssimo | supremo | mais alto}
  \end{phonetics}
\end{entry}

\begin{entry}{最善}{12,12}
  \begin{phonetics}{最善}{zui4shan4}
    \definition{adj.}{ótimo | o melhor}
  \end{phonetics}
\end{entry}

\begin{entry}{最新}{12,13}
  \begin{phonetics}{最新}{zui4xin1}
    \definition{adv.}{mais recente | mais novo}
  \end{phonetics}
\end{entry}

\begin{entry}{朝}{12}[Radical 月]
  \begin{phonetics}{朝}{chao2}
    \definition{s.}{corte imperial ou real | governo | dinastia | reinado de um soberano ou imperador | tribunal ou assembléia mantida por um soberano ou imperador}
    \definition{v.}{fazer uma peregrinação a}
  \end{phonetics}
  \begin{phonetics}{朝}{zhao1}
    \definition{s.}{manhã}
  \end{phonetics}
\end{entry}

\begin{entry}{朝廷}{12,6}
  \begin{phonetics}{朝廷}{chao2ting2}
    \definition{s.}{corte imperial | dinastia}
  \end{phonetics}
\end{entry}

\begin{entry}{朝鲜}{12,14}
  \begin{phonetics}{朝鲜}{chao2xian3}
    \definition*{s.}{Coréia do Norte}
  \end{phonetics}
\end{entry}

\begin{entry}{棉}{12}[Radical 木]
  \begin{phonetics}{棉}{mian2}
    \definition{s.}{termo genérico para algodão ou paina | algodão | acolchoado ou estofado com algodão}
  \end{phonetics}
\end{entry}

\begin{entry}{棒冰}{12,6}
  \begin{phonetics}{棒冰}{bang4bing1}
    \definition{s.}{picolé}
  \end{phonetics}
\end{entry}

\begin{entry}{棒棒糖}{12,12,16}
  \begin{phonetics}{棒棒糖}{bang4bang4tang2}
    \definition[根]{s.}{pirulito}
  \end{phonetics}
\end{entry}

\begin{entry}{棕褐色}{12,14,6}
  \begin{phonetics}{棕褐色}{zong1he4se4}
    \definition{s.}{cor sépia | bronzeado}
  \end{phonetics}
\end{entry}

\begin{entry}{森林}{12,8}
  \begin{phonetics}{森林}{sen1lin2}
    \definition{s.}{floresta}
  \end{phonetics}
\end{entry}

\begin{entry}{棹}{12}[Radical 木]
  \begin{phonetics}{棹}{zhuo1}
    \variantof{桌}
  \end{phonetics}
\end{entry}

\begin{entry}{棺}{12}[Radical ⽊]
  \begin{phonetics}{棺}{guan1}
    \definition{s.}{caixão | esquife | ataúde}
  \end{phonetics}
\end{entry}

\begin{entry}{椰汁}{12,5}
  \begin{phonetics}{椰汁}{ye1zhi1}
    \definition{s.}{água de coco}
  \end{phonetics}
\end{entry}

\begin{entry}{款}{12}[Radical 欠]
  \begin{phonetics}{款}{kuan3}
    \definition{clas.}{para versões ou modelos (de um produto)}
    \definition[笔,个]{s.}{montante de dinheiro | fundos | parágrafo | seção}
  \end{phonetics}
\end{entry}

\begin{entry}{殖}{12}[Radical 歹]
  \begin{phonetics}{殖}{zhi2}
    \definition{v.}{crescer | reproduzir}
  \end{phonetics}
\end{entry}

\begin{entry}{渡过}{12,6}
  \begin{phonetics}{渡过}{du4guo4}
    \definition{v.}{atravessar | passar por}
  \end{phonetics}
\end{entry}

\begin{entry}{温度}{12,9}
  \begin{phonetics}{温度}{wen1du4}
    \definition[个]{s.}{temperatura}
  \end{phonetics}
\end{entry}

\begin{entry}{温度计}{12,9,4}
  \begin{phonetics}{温度计}{wen1du4ji4}
    \definition{s.}{termógrafo | termômetro}
  \end{phonetics}
\end{entry}

\begin{entry}{温度表}{12,9,8}
  \begin{phonetics}{温度表}{wen1du4biao3}
    \definition{s.}{termômetro}
  \end{phonetics}
\end{entry}

\begin{entry}{温度梯度}{12,9,11,9}
  \begin{phonetics}{温度梯度}{wen1du4ti1du4}
    \definition{s.}{gradiente de temperatura}
  \end{phonetics}
\end{entry}

\begin{entry}{温柔}{12,9}
  \begin{phonetics}{温柔}{wen1rou2}
    \definition{adj.}{gentil e suave | terno | doce (comumente usado para descrever uma menina ou mulher)}
  \end{phonetics}
\end{entry}

\begin{entry}{温暖}{12,13}
  \begin{phonetics}{温暖}{wen1nuan3}
    \definition{s.}{cordialidade}
  \end{phonetics}
\end{entry}

\begin{entry}{渴}{12}[Radical 水]
  \begin{phonetics}{渴}{ke3}
    \definition{adj.}{sedento}
  \end{phonetics}
\end{entry}

\begin{entry}{游戏}{12,6}
  \begin{phonetics}{游戏}{you2xi4}
    \definition[场]{s.}{jogo}
    \definition{v.}{jogar}
  \end{phonetics}
\end{entry}

\begin{entry}{游泳}{12,8}
  \begin{phonetics}{游泳}{you2yong3}
    \definition{s.}{natação}
    \definition{v.+compl.}{nadar}
  \end{phonetics}
\end{entry}

\begin{entry}{游泳池}{12,8,6}
  \begin{phonetics}{游泳池}{you2yong3chi2}
    \definition[场]{s.}{piscina}
  \seealsoref{泳池}{yong3chi2}
  \seealsoref{游泳馆}{you2yong3guan3}
  \end{phonetics}
\end{entry}

\begin{entry}{游泳衣}{12,8,6}
  \begin{phonetics}{游泳衣}{you2yong3yi1}
    \definition{s.}{roupa de banho}
  \seealsoref{泳衣}{yong3yi1}
  \end{phonetics}
\end{entry}

\begin{entry}{游泳馆}{12,8,11}
  \begin{phonetics}{游泳馆}{you2yong3guan3}
    \definition[场]{s.}{piscina}
  \seealsoref{泳池}{yong3chi2}
  \seealsoref{游泳池}{you2yong3chi2}
  \end{phonetics}
\end{entry}

\begin{entry}{游泳镜}{12,8,16}
  \begin{phonetics}{游泳镜}{you2yong3jing4}
    \definition{s.}{óculos de natação}
  \end{phonetics}
\end{entry}

\begin{entry}{游客}{12,9}
  \begin{phonetics}{游客}{you2ke4}
    \definition{s.}{viajante | turista | (jogo online) jogador convidado}
  \end{phonetics}
\end{entry}

\begin{entry}{游艇}{12,12}
  \begin{phonetics}{游艇}{you2ting3}
    \definition[只]{s.}{barcaça | iate}
  \end{phonetics}
\end{entry}

\begin{entry}{湖}{12}[Radical 水]
  \begin{phonetics}{湖}{hu2}
    \definition[个,片]{s.}{lago}
  \end{phonetics}
\end{entry}

\begin{entry}{湖南}{12,9}
  \begin{phonetics}{湖南}{hu2nan2}
    \definition*{s.}{Hunan}
  \end{phonetics}
\end{entry}

\begin{entry}{滑}{12}[Radical 水]
  \begin{phonetics}{滑}{hua2}
    \definition*{s.}{sobrenome Hua}
    \definition{adj.}{deslizado}
    \definition{v.}{deslizar}
  \end{phonetics}
\end{entry}

\begin{entry}{滑雪}{12,11}
  \begin{phonetics}{滑雪}{hua2xue3}
    \definition{v.+compl.}{esquiar | praticar esqui}
  \end{phonetics}
\end{entry}

\begin{entry}{焚香}{12,9}
  \begin{phonetics}{焚香}{fen2xiang1}
    \definition{v.}{queimar incenso}
  \end{phonetics}
\end{entry}

\begin{entry}{焦虑}{12,10}
  \begin{phonetics}{焦虑}{jiao1lv4}
    \definition{adj.}{ansioso | preocupado | apreensivo}
  \end{phonetics}
\end{entry}

\begin{entry}{然}{12}[Radical 火]
  \begin{phonetics}{然}{ran2}
    \definition{conj.}{mas | no entanto}
  \end{phonetics}
\end{entry}

\begin{entry}{然后}{12,6}
  \begin{phonetics}{然后}{ran2hou4}
    \definition{conj.}{depois | logo | portanto}
  \end{phonetics}
\end{entry}

\begin{entry}{然而}{12,6}
  \begin{phonetics}{然而}{ran2'er2}
    \definition{conj.}{mas | no entanto}
  \end{phonetics}
\end{entry}

\begin{entry}{猩猩}{12,12}
  \begin{phonetics}{猩猩}{xing1xing5}
    \definition{s.}{orangotango}
  \end{phonetics}
\end{entry}

\begin{entry}{猴子}{12,3}
  \begin{phonetics}{猴子}{hou2zi5}
    \definition[只]{s.}{macaco}
  \end{phonetics}
\end{entry}

\begin{entry}{琴键}{12,13}
  \begin{phonetics}{琴键}{qin2jian4}
    \definition{s.}{tecla de piano}
  \end{phonetics}
\end{entry}

\begin{entry}{甁}{12}
  \begin{phonetics}{甁}{ping2}
    \variantof{瓶}
  \end{phonetics}
\end{entry}

\begin{entry}{番茄}{12,8}
  \begin{phonetics}{番茄}{fan1qie2}
    \definition{s.}{tomate}
  \end{phonetics}
\end{entry}

\begin{entry}{痛骂}{12,9}
  \begin{phonetics}{痛骂}{tong4ma4}
    \definition{v.}{repreender severamente}
  \end{phonetics}
\end{entry}

\begin{entry}{痠}{12}
  \begin{phonetics}{痠}{suan1}
    \definition{v.}{doer | estar dolorido}
    \variantof{酸}
  \end{phonetics}
\end{entry}

\begin{entry}{登}{12}[Radical 癶]
  \begin{phonetics}{登}{deng1}
    \definition{v.}{subir (montanha, cume)}
  \end{phonetics}
\end{entry}

\begin{entry}{短}{12}[Radical ⽮]
  \begin{phonetics}{短}{duan3}
    \definition{adj.}{curto | breve}
  \end{phonetics}
\end{entry}

\begin{entry}{短少}{12,4}
  \begin{phonetics}{短少}{duan3shao3}
    \definition{v.}{estar aquém do valor total}
  \end{phonetics}
\end{entry}

\begin{entry}{短处}{12,5}
  \begin{phonetics}{短处}{duan3chu4}
    \definition{s.}{defeito | falta | pontos fracos de alguém | deficiência}
  \end{phonetics}
\end{entry}

\begin{entry}{短视}{12,8}
  \begin{phonetics}{短视}{duan3shi4}
    \definition{adj.}{míope}
  \end{phonetics}
\end{entry}

\begin{entry}{短促}{12,9}
  \begin{phonetics}{短促}{duan3cu4}
    \definition{adj.}{curto (tom de voz) | fugaz | ofegante (respiração) | curto no tempo}
  \end{phonetics}
\end{entry}

\begin{entry}{短缺}{12,10}
  \begin{phonetics}{短缺}{duan3que1}
    \definition{s.}{escassez}
  \end{phonetics}
\end{entry}

\begin{entry}{短暂}{12,12}
  \begin{phonetics}{短暂}{duan3zan4}
    \definition{adj.}{momentâneo | de curta duração}
  \end{phonetics}
\end{entry}

\begin{entry}{短期}{12,12}
  \begin{phonetics}{短期}{duan3qi1}
    \definition{s.}{curto prazo}
  \end{phonetics}
\end{entry}

\begin{entry}{短裤}{12,12}
  \begin{phonetics}{短裤}{duan3ku4}
    \definition{s.}{calção | shorts}
  \end{phonetics}
\end{entry}

\begin{entry}{短跑}{12,12}
  \begin{phonetics}{短跑}{duan3pao3}
    \definition{s.}{corrida}
  \end{phonetics}
\end{entry}

\begin{entry}{硬件}{12,6}
  \begin{phonetics}{硬件}{ying4jian4}
    \definition{s.}{\emph{hardware}}
  \end{phonetics}
\end{entry}

\begin{entry}{确}{12}[Radical 石]
  \begin{phonetics}{确}{que4}
    \definition{adj.}{autenticado | sólido | firme | real | verdadeiro}
  \end{phonetics}
\end{entry}

\begin{entry}{确实}{12,8}
  \begin{phonetics}{确实}{que4shi2}
    \definition{adj.}{real | verdadeiro | confiável}
    \definition{adv.}{realmente}
  \end{phonetics}
\end{entry}

\begin{entry}{禅}{12}[Radical 示]
  \begin{phonetics}{禅}{chan2}
    \definition*{s.}{Zen}
    \definition{s.}{meditação (Budismo)}
  \end{phonetics}
  \begin{phonetics}{禅}{shan4}
    \definition{v.}{abdicar}
  \end{phonetics}
\end{entry}

\begin{entry}{程序}{12,7}
  \begin{phonetics}{程序}{cheng2xu4}
    \definition{s.}{procedimento | sequência | ordem | programa de computador}
  \end{phonetics}
\end{entry}

\begin{entry}{程序设计}{12,7,6,4}
  \begin{phonetics}{程序设计}{cheng2xu4she4ji4}
    \definition{s.}{programação de computadores}
  \end{phonetics}
\end{entry}

\begin{entry}{程序库}{12,7,7}
  \begin{phonetics}{程序库}{cheng2xu4ku4}
    \definition{s.}{biblioteca de funções e procedimentos para programas de computador}
  \end{phonetics}
\end{entry}

\begin{entry}{程控}{12,11}
  \begin{phonetics}{程控}{cheng2kong4}
    \definition{s.}{programado | sob controle automático}
  \end{phonetics}
\end{entry}

\begin{entry}{稍}{12}[Radical 禾]
  \begin{phonetics}{稍}{shao1}
    \definition{adv.}{um pouco | ligeiramente | em vez de}
  \end{phonetics}
\end{entry}

\begin{entry}{稍微}{12,13}
  \begin{phonetics}{稍微}{shao1wei1}
    \definition{adv.}{um pouco}
  \end{phonetics}
\end{entry}

\begin{entry}{税}{12}[Radical 禾]
  \begin{phonetics}{税}{shui4}
    \definition{s.}{taxas | impostos}
  \end{phonetics}
\end{entry}

\begin{entry}{窗帘}{12,8}
  \begin{phonetics}{窗帘}{chuang1lian2}
    \definition{s.}{cortina}
  \end{phonetics}
\end{entry}

\begin{entry}{童年}{12,6}
  \begin{phonetics}{童年}{tong2nian2}
    \definition{s.}{infância}
  \end{phonetics}
\end{entry}

\begin{entry}{等}{12}[Radical 竹]
  \begin{phonetics}{等}{deng3}
    \definition{v.}{esperar | esperar por}
  \end{phonetics}
\end{entry}

\begin{entry}{等待}{12,9}
  \begin{phonetics}{等待}{deng3dai4}
    \definition{v.}{esperar | esperar por}
  \end{phonetics}
\end{entry}

\begin{entry}{筏}{12}[Radical 竹]
  \begin{phonetics}{筏}{fa2}
    \definition{s.}{jangada (de troncos, bambus, etc.)}
  \end{phonetics}
\end{entry}

\begin{entry}{答案}{12,10}
  \begin{phonetics}{答案}{da2'an4}
    \definition[个]{s.}{resposta | solução}
  \end{phonetics}
\end{entry}

\begin{entry}{策划}{12,6}
  \begin{phonetics}{策划}{ce4hua4}
    \definition{s.}{planejador | produtor | plano}
    \definition{v.}{esquematizar | engenhar | planejar}
  \end{phonetics}
\end{entry}

\begin{entry}{粤语}{12,9}
  \begin{phonetics}{粤语}{yue4yu3}
    \definition{s.}{cantonês | língua cantonesa}
  \end{phonetics}
\end{entry}

\begin{entry}{紫}{12}[Radical 糸]
  \begin{phonetics}{紫}{zi3}
    \definition{adj.}{púrpura | violeta}
  \end{phonetics}
\end{entry}

\begin{entry}{紫色}{12,6}
  \begin{phonetics}{紫色}{zi3se4}
    \definition{s.}{cor púrpura | cor violeta}
  \end{phonetics}
\end{entry}

\begin{entry}{絫}{12}
  \begin{phonetics}{絫}{lei3}
    \variantof{累}
  \end{phonetics}
\end{entry}

\begin{entry}{编程}{12,12}
  \begin{phonetics}{编程}{bian1cheng2}
    \definition{s.}{programa de computador}
    \definition{v.}{programar computador}
  \end{phonetics}
\end{entry}

\begin{entry}{缘}{12}[Radical 糸]
  \begin{phonetics}{缘}{yuan2}
    \definition{s.}{causa | razão | karma | destino | predestinação}
  \end{phonetics}
\end{entry}

\begin{entry}{缘分}{12,4}
  \begin{phonetics}{缘分}{yuan2fen4}
    \definition{s.}{destino ou acaso que une as pessoas | afinidade ou relacionamento predestinado | destino (Budismo)}
  \end{phonetics}
\end{entry}

\begin{entry}{羡慕}{12,14}
  \begin{phonetics}{羡慕}{xian4mu4}
    \definition{v.}{invejar | admirar}
  \end{phonetics}
\end{entry}

\begin{entry}{联合}{12,6}
  \begin{phonetics}{联合}{lian2he2}
    \definition{s.}{aliança | articulação}
    \definition{v.}{combinar | unir | articular}
  \end{phonetics}
\end{entry}

\begin{entry}{联合会}{12,6,6}
  \begin{phonetics}{联合会}{lian2he2hui4}
    \definition{s.}{federação}
  \end{phonetics}
\end{entry}

\begin{entry}{脾气}{12,4}
  \begin{phonetics}{脾气}{pi2qi5}
    \definition{s.}{temperamento | humor | disposição | caráter}
  \end{phonetics}
\end{entry}

\begin{entry}{舒服}{12,8}
  \begin{phonetics}{舒服}{shu1fu5}
    \definition{adj.}{estar confortável | bem disposto | sentir-se bem}
  \end{phonetics}
\end{entry}

\begin{entry}{落日}{12,4}
  \begin{phonetics}{落日}{luo4ri4}
    \definition{s.}{pôr do sol}
  \end{phonetics}
\end{entry}

\begin{entry}{落后}{12,6}
  \begin{phonetics}{落后}{luo4hou4}
    \definition{s.}{atraso}
    \definition{v.+compl.}{ficar para trás | atrasar}
  \end{phonetics}
\end{entry}

\begin{entry}{落汤鸡}{12,6,7}
  \begin{phonetics}{落汤鸡}{luo4tang1ji1}
    \definition{s.}{uma pessoa que parece encharcada e acamada| sofrimento profundo}
  \end{phonetics}
\end{entry}

\begin{entry}{葡}{12}[Radical 艸]
  \begin{phonetics}{葡}{pu2}
    \definition*{s.}{Portugal, abreviação de 葡萄牙}
    \seeref{葡萄牙}{pu2tao2ya2}
  \end{phonetics}
\end{entry}

\begin{entry}{葡文}{12,4}
  \begin{phonetics}{葡文}{pu2wen2}
    \definition{s.}{português, língua portuguesa}
    \seeref{葡萄牙文}{pu2tao2ya2wen2}
  \end{phonetics}
\end{entry}

\begin{entry}{葡汉词典}{12,5,7,8}
  \begin{phonetics}{葡汉词典}{pu2-han4 ci2dian3}
    \definition{s.}{dicionário português-chinês}
  \seealsoref{汉葡词典}{han4-pu2 ci2dian3}
  \end{phonetics}
\end{entry}

\begin{entry}{葡语}{12,9}
  \begin{phonetics}{葡语}{pu2yu3}
    \definition{s.}{português, língua portuguesa}
    \seeref{葡萄牙语}{pu2tao2ya2yu3}
  \end{phonetics}
\end{entry}

\begin{entry}{葡萄}{12,11}
  \begin{phonetics}{葡萄}{pu2tao5}
    \definition{s.}{uva}
  \end{phonetics}
\end{entry}

\begin{entry}{葡萄牙}{12,11,4}
  \begin{phonetics}{葡萄牙}{pu2tao2ya2}
    \definition{s.}{Portugal}
    \seeref{葡}{pu2}
  \end{phonetics}
\end{entry}

\begin{entry}{葡萄牙文}{12,11,4,4}
  \begin{phonetics}{葡萄牙文}{pu2tao2ya2wen2}
    \definition{s.}{português, língua portuguesa}
    \seeref{葡文}{pu2wen2}
  \end{phonetics}
\end{entry}

\begin{entry}{葡萄牙语}{12,11,4,9}
  \begin{phonetics}{葡萄牙语}{pu2tao2ya2yu3}
    \definition{s.}{português, língua portuguesa}
    \seeref{葡语}{pu2yu3}
  \end{phonetics}
\end{entry}

\begin{entry}{葫芦}{12,7}
  \begin{phonetics}{葫芦}{hu2lu5}
    \definition{adj.}{confuso}
    \definition{s.}{cabaça | termo genérico para bloco e equipamento (ou partes dele)}
  \end{phonetics}
\end{entry}

\begin{entry}{葬}{12}[Radical 艸]
  \begin{phonetics}{葬}{zang4}
    \definition{v.}{enterrar (os mortos) | sepultar}
  \end{phonetics}
\end{entry}

\begin{entry}{葱}{12}[Radical 艸]
  \begin{phonetics}{葱}{cong1}
    \definition{s.}{cebolinha}
  \end{phonetics}
\end{entry}

\begin{entry}{葵花}{12,7}
  \begin{phonetics}{葵花}{kui2hua1}
    \definition{s.}{girassol (flor)}
  \end{phonetics}
\end{entry}

\begin{entry}{街}{12}[Radical 行]
  \begin{phonetics}{街}{jie1}
    \definition[条]{s.}{rua}
  \end{phonetics}
\end{entry}

\begin{entry}{街舞}{12,14}
  \begin{phonetics}{街舞}{jie1wu3}
    \definition{s.}{dança de rua, \emph{street dance} (por exemplo, \emph{breakdance})}
  \end{phonetics}
\end{entry}

\begin{entry}{裁}{12}[Radical 衣]
  \begin{phonetics}{裁}{cai2}
    \definition{s.}{decisão | julgamento}
    \definition{v.}{recortar (tecido de uma roupa) | cortar | aparar | reduzir | diminuir | cortar pessoal de uma equipe}
  \end{phonetics}
\end{entry}

\begin{entry}{装}{12}[Radical 衣]
  \begin{phonetics}{装}{zhuang1}
    \definition{s.}{adorno | roupa | traje (de um ator em uma peça)}
    \definition{v.}{adornar | vestir | desepenhar um papel | fingir | instalar | consertar | embrulhar (algo em um saco) | empacotar}
  \end{phonetics}
\end{entry}

\begin{entry}{装扮}{12,7}
  \begin{phonetics}{装扮}{zhuang1ban4}
    \definition{v.}{enfeitar | decorar | disfarçar-me | vestir-se}
  \end{phonetics}
\end{entry}

\begin{entry}{装备}{12,8}
  \begin{phonetics}{装备}{zhuang1bei4}
    \definition{s.}{equipamento}
    \definition{v.}{equipar}
  \end{phonetics}
\end{entry}

\begin{entry}{装配}{12,10}
  \begin{phonetics}{装配}{zhuang1pei4}
    \definition{v.}{montar | encaixar}
  \end{phonetics}
\end{entry}

\begin{entry}{裙子}{12,3}
  \begin{phonetics}{裙子}{qun2zi5}
    \definition[条]{s.}{saia | vestido}
  \end{phonetics}
\end{entry}

\begin{entry}{裤子}{12,3}
  \begin{phonetics}{裤子}{ku4zi5}
    \definition[条]{s.}{calças}
  \end{phonetics}
\end{entry}

\begin{entry}{詈骂}{12,9}
  \begin{phonetics}{詈骂}{li4ma4}
    \definition{v.}{xingar | abusar}
  \end{phonetics}
\end{entry}

\begin{entry}{谢天谢地}{12,4,12,6}
  \begin{phonetics}{谢天谢地}{xie4tian1xie4di4}
    \definition{expr.}{agradecer a Deus | agradecer aos céus}
  \end{phonetics}
\end{entry}

\begin{entry}{谢世}{12,5}
  \begin{phonetics}{谢世}{xie4shi4}
    \definition{v.}{morrer | falecer}
  \end{phonetics}
\end{entry}

\begin{entry}{谢恩}{12,10}
  \begin{phonetics}{谢恩}{xie4'en1}
    \definition{v.}{agradecer a alguém pelo favor (especialmente imperador ou oficial superior)}
  \end{phonetics}
\end{entry}

\begin{entry}{谢病}{12,10}
  \begin{phonetics}{谢病}{xie4bing4}
    \definition{v.}{desculpar-se por causa de doença}
  \end{phonetics}
\end{entry}

\begin{entry}{谢媒}{12,12}
  \begin{phonetics}{谢媒}{xie4mei2}
    \definition{v.}{agradecer ao casamenteiro}
  \end{phonetics}
\end{entry}

\begin{entry}{谢谢}{12,12}
  \begin{phonetics}{谢谢}{xie4xie5}
    \definition{interj.}{Obrigado!}
    \definition{v.}{agradecer}
  \end{phonetics}
\end{entry}

\begin{entry}{谢意}{12,13}
  \begin{phonetics}{谢意}{xie4yi4}
    \definition{s.}{gratidão}
  \end{phonetics}
\end{entry}

\begin{entry}{貂}{12}[Radical 豸]
  \begin{phonetics}{貂}{diao1}
    \definition{s.}{marta | fuinha}
  \end{phonetics}
\end{entry}

\begin{entry}{赏心悦目}{12,4,10,5}
  \begin{phonetics}{赏心悦目}{shang3xin1yue4mu4}
    \definition{expr.}{"Aquece o coração e encanta os olhos."}
  \end{phonetics}
\end{entry}

\begin{entry}{赏赐}{12,12}
  \begin{phonetics}{赏赐}{shang3ci4}
    \definition{s.}{recompensa | prêmio}
    \definition{v.}{recompensar | premiar}
  \end{phonetics}
\end{entry}

\begin{entry}{赔钱}{12,10}
  \begin{phonetics}{赔钱}{pei2qian2}
    \definition{v.+compl.}{perder dinheiro | pagar pelos danos}
  \end{phonetics}
\end{entry}

\begin{entry}{超市}{12,5}
  \begin{phonetics}{超市}{chao1shi4}
    \definition[家]{s.}{supermercado}
  \end{phonetics}
\end{entry}

\begin{entry}{超级}{12,6}
  \begin{phonetics}{超级}{chao1ji2}
    \definition{pref.}{``super'' | ``ultra'' | ``hiper''}
  \end{phonetics}
\end{entry}

\begin{entry}{超声}{12,7}
  \begin{phonetics}{超声}{chao1sheng1}
    \definition{adj.}{ultrasônico}
    \definition{s.}{ultrasom}
  \end{phonetics}
\end{entry}

\begin{entry}{越}{12}[Radical 走]
  \begin{phonetics}{越}{yue4}
    \definition{adv.}{quanto mais\dots mais}
    \definition{v.}{subir | exceder | superar}
  \end{phonetics}
\end{entry}

\begin{entry}{越来越……}{12,7,12}
  \begin{phonetics}{越来越……}{yue4lai2yue4}
    \definition{adv.}{cada vez mais\dots}
  \end{phonetics}
\end{entry}

\begin{entry}{越……越……}{12,12}
  \begin{phonetics}{越……越……}{yue4 yue4}
    \definition{expr.}{quanto mais\dots tanto mais\dots}
  \end{phonetics}
\end{entry}

\begin{entry}{越障}{12,13}
  \begin{phonetics}{越障}{yue4zhang4}
    \definition{s.}{curso com obstáculos para treinamento de tropas}
    \definition{v.}{superar obstáculos}
  \end{phonetics}
\end{entry}

\begin{entry}{越境}{12,14}
  \begin{phonetics}{越境}{yue4jing4}
    \definition{v.}{cruzar uma fronteira (geralmente ilegalmente) | entrar ou sair furtivamente de um país}
  \end{phonetics}
\end{entry}

\begin{entry}{趋势}{12,8}
  \begin{phonetics}{趋势}{qu1shi4}
    \definition{s.}{tendência}
  \end{phonetics}
\end{entry}

\begin{entry}{跑}{12}[Radical 足]
  \begin{phonetics}{跑}{pao2}
    \definition{v.}{(de um animal) dar patadas (no chão)}
  \end{phonetics}
  \begin{phonetics}{跑}{pao3}
    \definition{v.}{vazar ou evaporar (sobre um gás ou líquido) | escapar | correr | correr (em tarefas, etc.) | fugir}
  \end{phonetics}
\end{entry}

\begin{entry}{跑马}{12,3}
  \begin{phonetics}{跑马}{pao3ma3}
    \definition{s.}{corrida de cavalos}
    \definition{v.}{andar a cavalo em ritmo acelerado}
  \end{phonetics}
\end{entry}

\begin{entry}{跑步}{12,7}
  \begin{phonetics}{跑步}{pao3bu4}
    \definition{s.}{corrida}
    \definition{v.+compl.}{correr | (militar) marchar em dupla}
  \end{phonetics}
\end{entry}

\begin{entry}{跑肚}{12,7}
  \begin{phonetics}{跑肚}{pao3du4}
    \definition{v.}{(coloquial) ter diarréia}
  \end{phonetics}
\end{entry}

\begin{entry}{跑调}{12,10}
  \begin{phonetics}{跑调}{pao3diao4}
    \definition{v.}{(coloquial) estar fora do tom ou desafinado (enquanto canta)}
  \end{phonetics}
\end{entry}

\begin{entry}{跑掉}{12,11}
  \begin{phonetics}{跑掉}{pao3diao4}
    \definition{v.}{fugir}
  \end{phonetics}
\end{entry}

\begin{entry}{跑腿}{12,13}
  \begin{phonetics}{跑腿}{pao3tui3}
    \definition{v.}{realizar tarefas}
  \end{phonetics}
\end{entry}

\begin{entry}{跑酷}{12,14}
  \begin{phonetics}{跑酷}{pao3ku4}
    \definition*{s.}{(empréstimo linguístico) \emph{Parkour}}
  \end{phonetics}
\end{entry}

\begin{entry}{跑题}{12,15}
  \begin{phonetics}{跑题}{pao3ti2}
    \definition{v.}{divagar | fugir do assunto | tergiversar}
  \end{phonetics}
\end{entry}

\begin{entry}{遍}{12}[Radical 辵]
  \begin{phonetics}{遍}{bian4}
    \definition{adv.}{em todos os lugares | por toda parte}
    \definition{clas.}{para a repetição de ações de leitura, fala ou escrita}
  \end{phonetics}
\end{entry}

\begin{entry}{道理}{12,11}
  \begin{phonetics}{道理}{dao4li5}
    \definition[个]{s.}{razão | argumento | sentido | princípio | base | justificativa}
  \end{phonetics}
\end{entry}

\begin{entry}{道歉}{12,14}
  \begin{phonetics}{道歉}{dao4qian4}
    \definition{v.+compl.}{desculpar-se | fazer um pedido de desculpas}
  \end{phonetics}
\end{entry}

\begin{entry}{遗产}{12,6}
  \begin{phonetics}{遗产}{yi2chan3}
    \definition[笔]{s.}{legado | herança}
  \end{phonetics}
\end{entry}

\begin{entry}{遗男}{12,7}
  \begin{phonetics}{遗男}{yi2nan2}
    \definition{s.}{órfão | filho póstumo}
  \end{phonetics}
\end{entry}

\begin{entry}{遗迹}{12,9}
  \begin{phonetics}{遗迹}{yi2ji4}
    \definition{s.}{vestígios históricos | remanescente | vestígio}
  \end{phonetics}
\end{entry}

\begin{entry}{遗案}{12,10}
  \begin{phonetics}{遗案}{yi2'an4}
    \definition{s.}{(lei) caso não resolvido}
  \end{phonetics}
\end{entry}

\begin{entry}{遗落}{12,12}
  \begin{phonetics}{遗落}{yi2luo4}
    \definition{v.}{esquecer | deixar para trás (inadvertidamente) | deixar de fora | omitir}
  \end{phonetics}
\end{entry}

\begin{entry}{遗嘱}{12,15}
  \begin{phonetics}{遗嘱}{yi2zhu3}
    \definition{s.}{testamento}
  \end{phonetics}
\end{entry}

\begin{entry}{遗骸}{12,15}
  \begin{phonetics}{遗骸}{yi2hai2}
    \definition{v.}{restos mortais}
  \end{phonetics}
\end{entry}

\begin{entry}{遗憾}{12,16}
  \begin{phonetics}{遗憾}{yi2han4}
    \definition{v.}{ter pena de | lamentar}
  \end{phonetics}
\end{entry}

\begin{entry}{酢}{12}[Radical 酉]
  \begin{phonetics}{酢}{cu4}
    \variantof{醋}
  \end{phonetics}
  \begin{phonetics}{酢}{zuo4}
    \definition{v.}{brindar o anfitrião com vinho}
  \end{phonetics}
\end{entry}

\begin{entry}{铺}{12}[Radical 金]
  \begin{phonetics}{铺}{pu1}
    \definition{v.}{espalhar | exibir | montar}
  \end{phonetics}
  \begin{phonetics}{铺}{pu4}
    \definition{s.}{cama de tábua | lugar para dormir | loja | depósito}
  \end{phonetics}
\end{entry}

\begin{entry}{铺垫}{12,9}
  \begin{phonetics}{铺垫}{pu1dian4}
    \definition{s.}{cobre leito | colcha | roupa de cama}
    \definition{v.}{pavimentar}
  \end{phonetics}
\end{entry}

\begin{entry}{锅}{12}[Radical 金]
  \begin{phonetics}{锅}{guo1}
    \definition[口,只]{s.}{panela | frigideira | \emph{wok} | caldeirão | coisa em forma de pote}
  \end{phonetics}
\end{entry}

\begin{entry}{集团}{12,6}
  \begin{phonetics}{集团}{ji2tuan2}
    \definition{s.}{grupo | bloco | corporação | conglomerado}
  \end{phonetics}
\end{entry}

\begin{entry}{集体}{12,7}
  \begin{phonetics}{集体}{ji2ti3}
    \definition{s.}{coletivo (decisão) | esforço (conjunto) | um grupo | uma equipe}
  \end{phonetics}
\end{entry}

\begin{entry}{韩国}{12,8}
  \begin{phonetics}{韩国}{han2guo2}
    \definition*{s.}{Coréia do Sul}
  \end{phonetics}
\end{entry}

\begin{entry}{韩国人}{12,8,2}
  \begin{phonetics}{韩国人}{han2guo2ren2}
    \definition{s.}{coreano | pessoa ou povo da Coréia}
  \end{phonetics}
\end{entry}

\begin{entry}{骚乱}{12,7}
  \begin{phonetics}{骚乱}{sao1luan4}
    \definition{s.}{rebelião | perturbação | tumulto}
    \definition{v.}{criar um distúrbio}
  \end{phonetics}
\end{entry}

\begin{entry}{黑}{12}[Radical ⿊][Kangxi 203]
  \begin{phonetics}{黑}{hei1}
    \definition{adj.}{preto | escuro | ilegal | secreto | sombrio | sinistro}
    \definition{v.}{esconder (algo) | difamar | (empréstimo linguístico) (computador) hackear}
  \end{phonetics}
\end{entry}

\begin{entry}{黑色}{12,6}
  \begin{phonetics}{黑色}{hei1se4}
    \definition{s.}{cor preta}
  \end{phonetics}
\end{entry}

\begin{entry}{黑板}{12,8}
  \begin{phonetics}{黑板}{hei1ban3}
    \definition[块,个]{s.}{quadro negro}
  \end{phonetics}
\end{entry}

\begin{entry}{黑客}{12,9}
  \begin{phonetics}{黑客}{hei1ke4}
    \definition{s.}{(empréstimo linguístico) (computação) \emph{hacker}}
  \end{phonetics}
\end{entry}

%%%%% EOF %%%%%

