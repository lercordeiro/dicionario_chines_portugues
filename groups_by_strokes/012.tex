%%%
%%% 12画
%%%

\section*{12画}\addcontentsline{toc}{section}{12画}

\begin{entry}{傢具}{12,8}{⼈、⼋}
  \begin{phonetics}{傢具}{jia1ju4}
    \variantof{家具}
  \end{phonetics}
\end{entry}

\begin{entry}{博士}{12,3}{⼗、⼠}
  \begin{phonetics}{博士}{bo2shi4}[][HSK 5]
    \definition{s.}{doutorado; grau de doutor; nível mais alto de um diploma; também, uma pessoa que obteve esse diploma | doutor; antigo título honorífico para uma pessoa que é habilidosa em um determinado ofício ou especializada em uma determinada ocupação | doutor; autoridades que ensinavam as escrituras na China nos tempos antigos}
  \end{phonetics}
\end{entry}

\begin{entry}{博文}{12,4}{⼗、⽂}
  \begin{phonetics}{博文}{bo2wen2}
    \definition{s.}{artigo em um blog}
    \definition{v.}{escrever um artigo em um blog}
  \end{phonetics}
\end{entry}

\begin{entry}{博主}{12,5}{⼗、⼂}
  \begin{phonetics}{博主}{bo2zhu3}
    \definition{s.}{blogueiro}
  \end{phonetics}
\end{entry}

\begin{entry}{博物馆}{12,8,11}{⼗、⽜、⾷}
  \begin{phonetics}{博物馆}{bo2wu4guan3}[][HSK 5]
    \definition[个]{s.}{museu; locais para coleta, armazenamento, pesquisa, exibição e exposição de relíquias culturais ou espécimes relacionados à história, cultura, arte, ciências naturais, ciência e tecnologia, etc.}
  \end{phonetics}
\end{entry}

\begin{entry}{博客}{12,9}{⼗、⼧}
  \begin{phonetics}{博客}{bo2 ke4}[][HSK 5]
    \definition{s.}{\emph{blog}; página da Web ou site gerenciado por um indivíduo, geralmente composto por postagens organizadas da mais recente para a mais antiga | blogueiro; \emph{blogger}; pessoas que possuem ou escrevem \emph{blogs}}
  \end{phonetics}
\end{entry}

\begin{entry}{博览会}{12,9,6}{⼗、⾒、⼈}
  \begin{phonetics}{博览会}{bo2lan3hui4}[][HSK 5]
    \definition[次]{s.}{exposição; feira internacional; exposições de produtos em grande escala}
  \end{phonetics}
\end{entry}

\begin{entry}{厨房}{12,8}{⼚、⼾}
  \begin{phonetics}{厨房}{chu2fang2}[][HSK 5]
    \definition[间,个]{s.}{cozinha}
  \end{phonetics}
\end{entry}

\begin{entry}{喂}{12}{⼝}
  \begin{phonetics}{喂}{wei4}[][HSK 2,4]
    \definition{interj.}{Ei!, Olá!, para chamar atenção | Alô? (quando respondendo uma chamada telefônica, pronuncia-se como \dpy{wei2})}
    \definition{v.}{criar; alimentar (animais); dar comida a um animal |
alimentar (pessoas); colocar alimentos, medicamentos, etc. na boca de alguém}
  \end{phonetics}
\end{entry}

\begin{entry}{喂奶}{12,5}{⼝、⼥}
  \begin{phonetics}{喂奶}{wei4nai3}
    \definition{v.}{amamentar}
  \end{phonetics}
\end{entry}

\begin{entry}{喂母乳}{12,5,8}{⼝、⽏、⼄}
  \begin{phonetics}{喂母乳}{wei4mu3ru3}
    \definition{s.}{amamentação}
  \end{phonetics}
\end{entry}

\begin{entry}{喂养}{12,9}{⼝、⼋}
  \begin{phonetics}{喂养}{wei4yang3}
    \definition{v.}{alimentar (uma criança, animal doméstico, etc.) | manter | criar (um animal)}
  \end{phonetics}
\end{entry}

\begin{entry}{喂食}{12,9}{⼝、⾷}
  \begin{phonetics}{喂食}{wei4shi2}
    \definition{v.}{alimentar}
  \end{phonetics}
\end{entry}

\begin{entry}{喂哺}{12,10}{⼝、⼝}
  \begin{phonetics}{喂哺}{wei4bu3}
    \definition{v.}{alimentar (um bebê)}
  \end{phonetics}
\end{entry}

\begin{entry}{喂料}{12,10}{⼝、⽃}
  \begin{phonetics}{喂料}{wei4liao4}
    \definition{v.}{alimentar (também no sentido figurativo)}
  \end{phonetics}
\end{entry}

\begin{entry}{善于}{12,3}{⼝、⼆}
  \begin{phonetics}{善于}{shan4yu2}[][HSK 4]
    \definition{adv./v.}{ser bom em; ser hábil em}
  \end{phonetics}
\end{entry}

\begin{entry}{善良}{12,7}{⼝、⾉}
  \begin{phonetics}{善良}{shan4liang2}[][HSK 4]
    \definition{adj.}{de bom coração; bom e honesto; de bom coração e cheio de boa vontade}
  \end{phonetics}
\end{entry}

\begin{entry}{善意}{12,13}{⼝、⼼}
  \begin{phonetics}{善意}{shan4yi4}
    \definition{s.}{boa vontade | benevolência | bondade}
  \end{phonetics}
\end{entry}

\begin{entry}{喊}{12}{⼝}
  \begin{phonetics}{喊}{han3}[][HSK 2]
    \definition{clas.}{gritar | berrar | chamar (uma pessoa)}
  \end{phonetics}
\end{entry}

\begin{entry}{喔}{12}{⼝}
  \begin{phonetics}{喔}{o1}
    \definition{interj.}{Oh!, Entendi!, usado para indicar realização, compreensão}
  \end{phonetics}
\end{entry}

\begin{entry}{喜欢}{12,6}{⼝、⽋}
  \begin{phonetics}{喜欢}{xi3huan5}[][HSK 1]
    \definition{v.}{gostar}
  \end{phonetics}
\end{entry}

\begin{entry}{喜剧}{12,10}{⼝、⼑}
  \begin{phonetics}{喜剧}{xi3ju4}
    \definition[部,出]{s.}{uma comédia}
  \end{phonetics}
\end{entry}

\begin{entry}{喜爱}{12,10}{⼝、⽖}
  \begin{phonetics}{喜爱}{xi3 ai4}[][HSK 4]
    \definition{v.}{gostar; amar; ter afeição por; estar interessado em; ter uma queda ou sentir interesse por pessoas ou coisas}
  \end{phonetics}
\end{entry}

\begin{entry}{喝}{12}{⼝}
  \begin{phonetics}{喝}{he1}[][HSK 1]
    \definition{interj.}{Meu Deus!}
    \definition{v.}{beber}
  \end{phonetics}
  \begin{phonetics}{喝}{he4}
    \definition{v.}{gritar bem alto}
  \end{phonetics}
\end{entry}

\begin{entry}{喝彩}{12,11}{⼝、⼺}
  \begin{phonetics}{喝彩}{he4cai3}
    \definition{s.}{aclamar | torcer}
  \end{phonetics}
\end{entry}

\begin{entry}{喝醉}{12,15}{⼝、⾣}
  \begin{phonetics}{喝醉}{he1zui4}
    \definition{v.}{ficar bêbado}
  \end{phonetics}
\end{entry}

\begin{entry}{喻}{12}{⼝}
  \begin{phonetics}{喻}{yu4}
    \definition{s.}{analogia | símile | metáfora | alegoria}
    \definition{v.}{descrever algo como}
  \end{phonetics}
\end{entry}

\begin{entry}{堤坝}{12,7}{⼟、⼟}
  \begin{phonetics}{堤坝}{di1ba4}
    \definition{s.}{represa | dique | barragem}
  \end{phonetics}
\end{entry}

\begin{entry}{奥}{12}{⼤}
  \begin{phonetics}{奥}{ao4}
    \definition{adj.}{obscuro | misterioso}
  \end{phonetics}
\end{entry}

\begin{entry}{奥运}{12,7}{⼤、⾡}
  \begin{phonetics}{奥运}{ao4yun4}
    \definition*{s.}{Jogos Olímpicos, Olimpíadas, abreviação de 奥林匹克运动会}
  \seealsoref{奥林匹克运动会}{ao4lin2pi3ke4 yun4dong4hui4}
  \end{phonetics}
\end{entry}

\begin{entry}{奥运会}{12,7,6}{⼤、⾡、⼈}
  \begin{phonetics}{奥运会}{ao4yun4hui4}
    \definition*{s.}{Jogos Olímpicos, Olimpíadas, abreviação de 奥林匹克运动会}
  \seealsoref{奥林匹克运动会}{ao4lin2pi3ke4 yun4dong4hui4}
  \end{phonetics}
\end{entry}

\begin{entry}{奥林匹克运动会}{12,8,4,7,7,6,6}{⼤、⽊、⼖、⼗、⾡、⼒、⼈}
  \begin{phonetics}{奥林匹克运动会}{ao4lin2pi3ke4 yun4dong4hui4}
    \definition*{s.}{Jogos Olímpicos, Olimpíadas}
  \end{phonetics}
\end{entry}

\begin{entry}{奥特曼}{12,10,11}{⼤、⽜、⽈}
  \begin{phonetics}{奥特曼}{ao4te4man4}
    \definition*{s.}{\emph{Ultraman},  super-herói de ficção científica japonesa}
  \end{phonetics}
\end{entry}

\begin{entry}{媒体}{12,7}{⼥、⼈}
  \begin{phonetics}{媒体}{mei2ti3}[][HSK 3]
    \definition[家,个,种]{s.}{mídia; mídia de massa}
  \end{phonetics}
\end{entry}

\begin{entry}{嫂子}{12,3}{⼥、⼦}
  \begin{phonetics}{嫂子}{sao3zi5}
    \definition{s.}{esposa do irmão mais velho}
  \end{phonetics}
\end{entry}

\begin{entry}{富}{12}{⼧}
  \begin{phonetics}{富}{fu4}[][HSK 3]
    \definition*{s.}{sobrenome Fu}
    \definition{adj.}{rico; póspero | rico; abundante}
    \definition{s.}{fortuna; riqueza}
  \end{phonetics}
\end{entry}

\begin{entry}{寒冷}{12,7}{⼧、⼎}
  \begin{phonetics}{寒冷}{han2 leng3}[][HSK 4]
    \definition{adj.}{frio; frígido; gélido; gelado}
  \end{phonetics}
\end{entry}

\begin{entry}{寒假}{12,11}{⼧、⼈}
  \begin{phonetics}{寒假}{han2jia4}[][HSK 4]
    \definition[个]{s.}{férias de inverno (feriados); férias escolares no meio do inverno, em janeiro e fevereiro (na China)}
  \end{phonetics}
\end{entry}

\begin{entry}{寓意}{12,13}{⼧、⼼}
  \begin{phonetics}{寓意}{yu4yi4}
    \definition{s.}{moral (de uma história),  lição a ser aprendida, implicação, mensagem, significado metafórico}
  \end{phonetics}
\end{entry}

\begin{entry}{就}{12}{⼪}
  \begin{phonetics}{就}{jiu4}[][HSK 1]
    \definition{adv.}{exatamente | justamente}
    \definition{v.}{realizar | se envolver em | acompanhar (em alimentos) | aproveitar | avançar | empreender}
  \end{phonetics}
\end{entry}

\begin{entry}{就业}{12,5}{⼪、⼀}
  \begin{phonetics}{就业}{jiu4ye4}[][HSK 3]
    \definition{v.+compl.}{conseguir um emprego; obter emprego; assumir uma ocupação}
  \end{phonetics}
\end{entry}

\begin{entry}{就是}{12,9}{⼪、⽇}
  \begin{phonetics}{就是}{jiu4 shi4}[][HSK 3]
    \definition{adv.}{exatamente; precisamente | apenas; simplesmente | usado para indicar escolha}
    \definition{conj.}{ainda que}
    \definition{part.}{usado no final de uma frase para expressar afirmação}
  \end{phonetics}
\end{entry}

\begin{entry}{就要}{12,9}{⼪、⾑}
  \begin{phonetics}{就要}{jiu4 yao4}[][HSK 2]
    \definition{adv.}{estar prestes a | estar indo para | estar a ponto de}
  \end{phonetics}
\end{entry}

\begin{entry}{就职}{12,11}{⼪、⽿}
  \begin{phonetics}{就职}{jiu4zhi2}
    \definition{v.}{assumir o cargo | assumir um posto}
  \end{phonetics}
\end{entry}

\begin{entry}{属}{12}{⼫}
  \begin{phonetics}{属}{shu3}[][HSK 3]
    \definition{s.}{categoria
gênero
membros da família; dependentes}
    \definition{v.}{estar sob; subordinado a | pertencer a | nascer no ano de (um dos doze animais do zodíaco)}
  \end{phonetics}
  \begin{phonetics}{属}{zhu3}
    \definition{v.}{juntar; combinar | fixar (a mente) em; centrar (a atenção, etc.) em}
  \end{phonetics}
\end{entry}

\begin{entry}{属于}{12,3}{⼫、⼆}
  \begin{phonetics}{属于}{shu3yu2}[][HSK 3]
    \definition{v.}{pertencer a; fazer parte de; ser classificado como}
  \end{phonetics}
\end{entry}

\begin{entry}{屡次}{12,6}{⼫、⽋}
  \begin{phonetics}{屡次}{lv3ci4}
    \definition{adv.}{repetidamente | uma e outra vez | muitas vezes}
  \end{phonetics}
\end{entry}

\begin{entry}{帽子}{12,3}{⼱、⼦}
  \begin{phonetics}{帽子}{mao4zi5}[][HSK 4]
    \definition[顶,个,种]{s.}{boné; chapéu; capacete | etiqueta; rótulo; marca}
  \end{phonetics}
\end{entry}

\begin{entry}{幅}{12}{⼱}
  \begin{phonetics}{幅}{fu2}[][HSK 5]
    \definition{clas.}{para tecidos, telas de lã, pinturas, etc.}
    \definition{s.}{largura do tecido, seda, tweed, etc. | tamanho; largura; largura em termos gerais}
  \end{phonetics}
\end{entry}

\begin{entry}{幅度}{12,9}{⼱、⼴}
  \begin{phonetics}{幅度}{fu2du4}[][HSK 5]
    \definition{s.}{alcance; escopo; extensão; largura; largura da propagação de um objeto que vibra ou balança, uma metáfora para a magnitude de uma mudança em algo}
  \end{phonetics}
\end{entry}

\begin{entry}{强}{12}{⼸}
  \begin{phonetics}{强}{jiang4}
    \definition{adj.}{teimoso; inflexível}
  \end{phonetics}
  \begin{phonetics}{强}{qiang2}[][HSK 3]
    \definition*{s.}{sobrenome Qiang}
    \definition{adj.}{forte; poderoso | melhor; superior | mais; extra; adicional; um pouco mais que | resoluto; firme | decidido; resolvido | violento; impetuoso | alto padrão}
    \definition{v.}{fortalecer; tornar forte}
  \end{phonetics}
  \begin{phonetics}{强}{qiang3}
    \definition{v.}{fazer um esforço; esforçar-se}
  \end{phonetics}
\end{entry}

\begin{entry}{强大}{12,3}{⼸、⼤}
  \begin{phonetics}{强大}{qiang2 da4}[][HSK 3]
    \definition{adj.}{forte; poderoso; potente; possante}
  \end{phonetics}
\end{entry}

\begin{entry}{强烈}{12,10}{⼸、⽕}
  \begin{phonetics}{强烈}{qiang2lie4}[][HSK 3]
    \definition{adj.}{forte; intenso | violento; impetuoso | afiado; marcante}
  \end{phonetics}
\end{entry}

\begin{entry}{强调}{12,10}{⼸、⾔}
  \begin{phonetics}{强调}{qiang2diao4}[][HSK 3]
    \definition{v.}{salientar; sublinhar; enfatizar; dar ênfase a; vincar}
  \end{phonetics}
\end{entry}

\begin{entry}{悲伤}{12,6}{⽕、⼈}
  \begin{phonetics}{悲伤}{bei1 shang1}[][HSK 5]
    \definition{adj.}{triste; pesaroso}
  \end{phonetics}
\end{entry}

\begin{entry}{悲剧}{12,10}{⽕、⼑}
  \begin{phonetics}{悲剧}{bei1 ju4}[][HSK 5]
    \definition[部,出]{s.}{tragédia; drama trágico; uma das principais categorias de teatro, caracterizada basicamente pela representação do conflito irreconciliável entre o protagonista e a realidade e seu final trágico | tragédia; evento triste; metáfora para encontro infeliz}
  \end{phonetics}
\end{entry}

\begin{entry}{惑星}{12,9}{⼼、⽇}
  \begin{phonetics}{惑星}{huo4xing1}
    \definition{s.}{planeta}
  \seealsoref{行星}{xing2xing1}
  \end{phonetics}
\end{entry}

\begin{entry}{惩处}{12,5}{⼼、⼡}
  \begin{phonetics}{惩处}{cheng2chu3}
    \definition{v.}{administrar justiça | punir}
  \end{phonetics}
\end{entry}

\begin{entry}{惩罚}{12,9}{⼼、⽹}
  \begin{phonetics}{惩罚}{cheng2fa2}
    \definition{v.}{punir | penalizar}
  \end{phonetics}
\end{entry}

\begin{entry}{愉快}{12,7}{⼼、⼼}
  \begin{phonetics}{愉快}{yu2kuai4}
    \definition{adj.}{alegre | delicioso | prazeroso | agradável | feliz | encantado}
    \definition{adv.}{alegremente | agradavelmente}
  \end{phonetics}
\end{entry}

\begin{entry}{愤世嫉俗}{12,5,13,9}{⼼、⼀、⼥、⼈}
  \begin{phonetics}{愤世嫉俗}{fen4shi4ji2su2}
    \definition{v.}{ser cínico | ser amargurado}
  \end{phonetics}
\end{entry}

\begin{entry}{愤怒}{12,9}{⼼、⼼}
  \begin{phonetics}{愤怒}{fen4nu4}
    \definition{adj.}{zangado | indignado}
    \definition{s.}{ira}
  \end{phonetics}
\end{entry}

\begin{entry}{掌}{12}{⼿}
  \begin{phonetics}{掌}{zhang3}
    \definition{s.}{palma da mão | sola do pé | pata | ferradura}
    \definition{v.}{dar um tapa | segurar na mão | empunhar}
  \end{phonetics}
\end{entry}

\begin{entry}{掱}{12}{⼿}
  \begin{phonetics}{掱}{shou3}
    \variantof{手}
  \end{phonetics}
\end{entry}

\begin{entry}{揉}{12}{⼿}
  \begin{phonetics}{揉}{rou2}
    \definition{v.}{amassar | massagear | esfregar}
  \end{phonetics}
\end{entry}

\begin{entry}{揉碎}{12,13}{⼿、⽯}
  \begin{phonetics}{揉碎}{rou2sui4}
    \definition{v.}{esmagar | desintegrar-se em pedaços}
  \end{phonetics}
\end{entry}

\begin{entry}{提}{12}{⼿}
  \begin{phonetics}{提}{ti2}[][HSK 2]
    \definition*{s.}{sobrenome Ti}
    \definition{s.}{concha | traço ascendente (em caracteres chineses)}
    \definition{v.}{carregar (na mão com o braço para baixo) | levantar | elevar | promover | avançar | mudar para um momento anterior | mover uma data para a frente | trazer à tona | apresentar | extrair | tirar | trazer | entregar | mencionar | referir-se a}
  \end{phonetics}
\end{entry}

\begin{entry}{提及}{12,3}{⼿、⼃}
  \begin{phonetics}{提及}{ti2ji2}
    \definition{v.}{mencionar | levantar (um assunto) | chamar a atenção de alguém}
  \end{phonetics}
\end{entry}

\begin{entry}{提升}{12,4}{⼿、⼗}
  \begin{phonetics}{提升}{ti2sheng1}
    \definition{v.}{promover (para uma posição de classificação mais alta) | levantar | içar | (figurativo) elevar, levantar, melhorar}
  \end{phonetics}
\end{entry}

\begin{entry}{提出}{12,5}{⼿、⼐}
  \begin{phonetics}{提出}{ti2 chu1}[][HSK 2]
    \definition{v.}{levantar | propor | expor | apresentar}
  \end{phonetics}
\end{entry}

\begin{entry}{提问}{12,6}{⼿、⾨}
  \begin{phonetics}{提问}{ti2wen4}[][HSK 3]
    \definition{v.}{\emph{quiz}; fazer uma pergunta; colocar questões para}
  \end{phonetics}
\end{entry}

\begin{entry}{提供}{12,8}{⼿、⼈}
  \begin{phonetics}{提供}{ti2gong1}[][HSK 4]
    \definition{v.}{oferecer; fornecer; suprir; prover; proporcionar}
  \end{phonetics}
\end{entry}

\begin{entry}{提到}{12,8}{⼿、⼑}
  \begin{phonetics}{提到}{ti2 dao4}[][HSK 2]
    \definition{v.}{mencionar | referir-se a | levantar (assunto)}
  \end{phonetics}
\end{entry}

\begin{entry}{提前}{12,9}{⼿、⼑}
  \begin{phonetics}{提前}{ti2qian2}[][HSK 3]
    \definition{adv.}{antecipadamente}
    \definition{v.}{avançar; adiantar; mudar para uma data anterior; mover para a frente (uma data)}
  \end{phonetics}
\end{entry}

\begin{entry}{提高}{12,10}{⼿、⾼}
  \begin{phonetics}{提高}{ti2gao1}[][HSK 2]
    \definition{v.}{melhorar | aumentar | elevar}
  \end{phonetics}
\end{entry}

\begin{entry}{提醒}{12,16}{⼿、⾣}
  \begin{phonetics}{提醒}{ti2xing3}[][HSK 4]
    \definition{v.+compl.}{alertar; avisar; advertir; lembrar; apontar para ou chamar a atenção para}
  \end{phonetics}
\end{entry}

\begin{entry}{插}{12}{⼿}
  \begin{phonetics}{插}{cha1}[][HSK 5]
    \definition{v.}{enfiar; inserir; colocar, apertar, empurrar ou perfurar uma coisa fina ou delgada; mergulhar |interpor; inserir; colocar no meio}
  \end{phonetics}
\end{entry}

\begin{entry}{插手}{12,4}{⼿、⼿}
  \begin{phonetics}{插手}{cha1shou3}
    \definition{v.+compl.}{envolver-se em | dar uma mão | ter (tomar) uma mão | cutucar o nariz de alguém | intrometer-se}
  \end{phonetics}
\end{entry}

\begin{entry}{插话}{12,8}{⼿、⾔}
  \begin{phonetics}{插话}{cha1hua4}
    \definition{s.}{interrupção | digressão}
    \definition{v.+compl.}{interromper (a fala de alguém)}
  \end{phonetics}
\end{entry}

\begin{entry}{握手}{12,4}{⼿、⼿}
  \begin{phonetics}{握手}{wo4shou3}[][HSK 3]
    \definition{v.+compl.}{apertar as mãos}
  \end{phonetics}
\end{entry}

\begin{entry}{援助}{12,7}{⼿、⼒}
  \begin{phonetics}{援助}{yuan2zhu4}
    \definition{s.}{assistência}
    \definition{v.}{ajudar | apoiar | assistir}
  \end{phonetics}
\end{entry}

\begin{entry}{搁浅}{12,8}{⼿、⽔}
  \begin{phonetics}{搁浅}{ge1qian3}
    \definition{v.}{ficar encalhado (navio) | encalhar | (figurativo) encontrar dificuldades e parar}
  \end{phonetics}
\end{entry}

\begin{entry}{搓}{12}{⼿}
  \begin{phonetics}{搓}{cuo1}
    \definition{s.}{torção}
    \definition{v.}{esfregar ou rolar entre as mãos ou dedos | torcer}
  \end{phonetics}
\end{entry}

\begin{entry}{搭讪}{12,5}{⼿、⾔}
  \begin{phonetics}{搭讪}{da1shan4}
    \definition{v.}{bater em alguém | incitar uma conversa | começar a conversar para acabar com um silêncio constrangedor ou uma situação embaraçosa}
  \end{phonetics}
\end{entry}

\begin{entry}{搭配}{12,10}{⼿、⾣}
  \begin{phonetics}{搭配}{da1pei4}
    \definition{v.}{emparelhar | combinar | organizar em pares | adicionar alguém em um grupo}
  \end{phonetics}
\end{entry}

\begin{entry}{散}{12}{⽁}
  \begin{phonetics}{散}{san3}
    \definition{adj.}{disperso; fragmentado; não integrado}
    \definition{s.}{medicamento em forma de pó}
    \definition{v.}{divergir; espalhar-se; separar-se; soltar-se; não se manter unido;  desintegrar}
  \end{phonetics}
  \begin{phonetics}{散}{san4}
    \definition{v.}{quebrar; fragmentar; dispersar | dar; distribuir; disseminar; divulgar | dissipar; deixar sai  | terminar um acordo ou contrato; demitir}
  \end{phonetics}
\end{entry}

\begin{entry}{散心}{12,4}{⽁、⼼}
  \begin{phonetics}{散心}{san4xin1}
    \definition{v.+compl.}{aliviar o tédio | desfrutar de uma diversão | estar despreocupado}
  \end{phonetics}
\end{entry}

\begin{entry}{散步}{12,7}{⽁、⽌}
  \begin{phonetics}{散步}{san4bu4}[][HSK 3]
    \definition{v.+compl.}{dar uma volta; passear; dar uma caminhada}
  \end{phonetics}
\end{entry}

\begin{entry}{敬礼}{12,5}{⽁、⽰}
  \begin{phonetics}{敬礼}{jing4li3}
    \definition{s.}{saudação}
    \definition{v.}{saudar}
  \end{phonetics}
\end{entry}

\begin{entry}{斯巴达}{12,4,6}{⽄、⼰、⾡}
  \begin{phonetics}{斯巴达}{si1ba1da2}
    \definition*{s.}{Esparta}
  \end{phonetics}
\end{entry}

\begin{entry}{普及}{12,3}{⽇、⼃}
  \begin{phonetics}{普及}{pu3ji2}[][HSK 3]
    \definition{adj.}{popular; universal; onipresente}
    \definition{v.}{popularizar; disseminar; espalhar entre o povo}
  \end{phonetics}
\end{entry}

\begin{entry}{普通}{12,10}{⽇、⾡}
  \begin{phonetics}{普通}{pu3 tong1}[][HSK 2]
    \definition{adj.}{ordinário | comum | geral | médio}
  \end{phonetics}
\end{entry}

\begin{entry}{普通话}{12,10,8}{⽇、⾡、⾔}
  \begin{phonetics}{普通话}{pu3tong1hua4}[][HSK 2]
    \definition*{s.}{Mandarim (literalmente ``linguagem comum'') | Putonghua (fala comum da língua chinesa) | discurso comum}
  \end{phonetics}
\end{entry}

\begin{entry}{普遍}{12,12}{⽇、⾡}
  \begin{phonetics}{普遍}{pu3bian4}[][HSK 3]
    \definition{adj.}{geral; comum; universal; difundido}
  \end{phonetics}
\end{entry}

\begin{entry}{景色}{12,6}{⽇、⾊}
  \begin{phonetics}{景色}{jing3se4}[][HSK 3]
    \definition[片,幅,道,处]{s.}{vista; cena; cenário; paisagem}
  \end{phonetics}
\end{entry}

\begin{entry}{晴}{12}{⽇}
  \begin{phonetics}{晴}{qing2}[][HSK 2]
    \definition{adj.}{ensolarado | claro}
  \end{phonetics}
\end{entry}

\begin{entry}{晴天}{12,4}{⽇、⼤}
  \begin{phonetics}{晴天}{qing2 tian1}[][HSK 2]
    \definition[个]{s.}{dia ensolarado}
  \end{phonetics}
\end{entry}

\begin{entry}{智力}{12,2}{⽇、⼒}
  \begin{phonetics}{智力}{zhi4li4}[][HSK 4]
    \definition{s.}{inteligência; refere-se à capacidade de uma pessoa de conhecer e entender coisas objetivas e aplicar o conhecimento e a experiência para resolver problemas, incluindo memória, observação, imaginação, pensamento e julgamento}
  \end{phonetics}
\end{entry}

\begin{entry}{智能}{12,10}{⽇、⾁}
  \begin{phonetics}{智能}{zhi4neng2}[][HSK 4]
    \definition{adj.}{inteligente (telefone, sistema, etc.); descreve máquinas, equipamentos, tecnologia, etc. que foram processados com alta tecnologia e têm a capacidade de falar, pensar, calcular, resolver problemas, etc., como um ser humano}
    \definition{s.}{intelecto; a capacidade de aprender, agir, pensar, inventar, criar, resolver problemas, etc.}
  \end{phonetics}
\end{entry}

\begin{entry}{智商}{12,11}{⽇、⼝}
  \begin{phonetics}{智商}{zhi4shang1}
    \definition{s.}{quociente de inteligência, QI}
  \end{phonetics}
\end{entry}

\begin{entry}{智障}{12,13}{⽇、⾩}
  \begin{phonetics}{智障}{zhi4zhang4}
    \definition{adj./s.}{retardado}
  \end{phonetics}
\end{entry}

\begin{entry}{智慧}{12,15}{⽇、⼼}
  \begin{phonetics}{智慧}{zhi4hui4}
    \definition{s.}{sabedoria | inteligência}
  \end{phonetics}
\end{entry}

\begin{entry}{暑假}{12,11}{⽇、⼈}
  \begin{phonetics}{暑假}{shu3 jia4}[][HSK 4]
    \definition[个]{s.}{férias de verão; feriado de verão; férias escolares de verão, na China, durante o sétimo e o oitavo meses do calendário gregoriano}
  \end{phonetics}
\end{entry}

\begin{entry}{曾}{12}{⽈}
  \begin{phonetics}{曾}{ceng2}[][HSK 4]
    \definition{adv.}{uma vez; antigamente; há algum tempo; usado para indicar ação ou estado passado}
  \end{phonetics}
  \begin{phonetics}{曾}{zeng1}
    \definition*{s.}{sobrenome Zeng}
    \definition{s.}{relacionamento entre bisnetos e bisavós}
  \end{phonetics}
\end{entry}

\begin{entry}{曾经}{12,8}{⽈、⽷}
  \begin{phonetics}{曾经}{ceng2jing1}[][HSK 3]
    \definition{adv.}{uma vez; indica certos comportamentos ou situações}
  \end{phonetics}
\end{entry}

\begin{entry}{替}{12}{⽈}
  \begin{phonetics}{替}{ti4}[][HSK 4]
    \definition{prep.}{para; em nome de}
    \definition{s.}{decadência; declínio; enfraquecimento}
    \definition{v.}{substituir; substituir por; tomar o lugar de}
  \end{phonetics}
\end{entry}

\begin{entry}{替代}{12,5}{⽈、⼈}
  \begin{phonetics}{替代}{ti4 dai4}[][HSK 4]
    \definition{v.}{substituir; suplantar}
  \end{phonetics}
\end{entry}

\begin{entry}{最}{12}{⽈}
  \begin{phonetics}{最}{zui4}[][HSK 1]
    \definition{adv.}{o mais | o melhor | a coisa mais\dots | grau superlativo relativo de superioridade}
  \end{phonetics}
\end{entry}

\begin{entry}{最少}{12,4}{⽈、⼩}
  \begin{phonetics}{最少}{zui4shao3}
    \definition{adv.}{finalmente}
  \end{phonetics}
\end{entry}

\begin{entry}{最优}{12,6}{⽈、⼈}
  \begin{phonetics}{最优}{zui4you1}
    \definition{adj.}{ótimo}
  \end{phonetics}
\end{entry}

\begin{entry}{最先}{12,6}{⽈、⼉}
  \begin{phonetics}{最先}{zui4xian1}
    \definition{adv.}{o primeiro}
  \end{phonetics}
\end{entry}

\begin{entry}{最后}{12,6}{⽈、⼝}
  \begin{phonetics}{最后}{zui4hou4}[][HSK 1]
    \definition{adj.}{final | último}
    \definition{adv.}{finalmente}
  \end{phonetics}
\end{entry}

\begin{entry}{最多}{12,6}{⽈、⼣}
  \begin{phonetics}{最多}{zui4duo1}
    \definition{adv.}{no máximo | máximo}
  \end{phonetics}
\end{entry}

\begin{entry}{最好}{12,6}{⽈、⼥}
  \begin{phonetics}{最好}{zui4hao3}[][HSK 1]
    \definition{adv.}{ser melhor que}
    \definition{v.}{(você) estar melhor (faça o que sugerimos) | querer ser o melhor}
  \end{phonetics}
\end{entry}

\begin{entry}{最初}{12,7}{⽈、⾐}
  \begin{phonetics}{最初}{zui4chu1}[][HSK 4]
    \definition{adj.}{primordial; inicial; primeiro}
    \definition{adv.}{inicialmente; originalmente}
    \definition{s.}{o período mais antigo; início; começo}
  \end{phonetics}
\end{entry}

\begin{entry}{最近}{12,7}{⽈、⾡}
  \begin{phonetics}{最近}{zui4jin4}[][HSK 2]
    \definition{adv.}{ultimamente | recentemente}
  \end{phonetics}
\end{entry}

\begin{entry}{最远}{12,7}{⽈、⾡}
  \begin{phonetics}{最远}{zui4yuan3}
    \definition{adv.}{mais distante | mais longe}
  \end{phonetics}
\end{entry}

\begin{entry}{最佳}{12,8}{⽈、⼈}
  \begin{phonetics}{最佳}{zui4jia1}
    \definition{adj.}{melhor (atleta, filme etc) | ótimo}
  \end{phonetics}
\end{entry}

\begin{entry}{最终}{12,8}{⽈、⽷}
  \begin{phonetics}{最终}{zui4zhong1}
    \definition{adv.}{pelo menos | finalmente}
    \definition{s.}{final | ultimato}
  \end{phonetics}
\end{entry}

\begin{entry}{最高}{12,10}{⽈、⾼}
  \begin{phonetics}{最高}{zui4gao1}
    \definition{adj.}{altíssimo | supremo | mais alto}
  \end{phonetics}
\end{entry}

\begin{entry}{最善}{12,12}{⽈、⼝}
  \begin{phonetics}{最善}{zui4shan4}
    \definition{adj.}{ótimo | o melhor}
  \end{phonetics}
\end{entry}

\begin{entry}{最新}{12,13}{⽈、⽄}
  \begin{phonetics}{最新}{zui4xin1}
    \definition{adv.}{mais recente | mais novo}
  \end{phonetics}
\end{entry}

\begin{entry}{朝}{12}{⽉}
  \begin{phonetics}{朝}{chao2}[][HSK 3]
    \definition*{s.}{sobrenome Chao}
    \definition{prep.}{para; em direção a}
    \definition{s.}{tribunal; governo | dinastia | o reino de um imperador}
    \definition{v.}{ter uma audiência com (um rei, um imperador, etc.); fazer uma peregrinação a | encarar; olhar}
  \end{phonetics}
  \begin{phonetics}{朝}{zhao1}
    \definition{s.}{manhã cedo; manhã | dia}
  \end{phonetics}
\end{entry}

\begin{entry}{朝廷}{12,6}{⽉、⼵}
  \begin{phonetics}{朝廷}{chao2ting2}
    \definition{s.}{corte imperial | dinastia}
  \end{phonetics}
\end{entry}

\begin{entry}{朝鲜}{12,14}{⽉、⿂}
  \begin{phonetics}{朝鲜}{chao2xian3}
    \definition*{s.}{Coréia do Norte}
  \end{phonetics}
\end{entry}

\begin{entry}{期}{12}{⽉}
  \begin{phonetics}{期}{qi1}[][HSK 3]
    \definition{clas.}{questão; número; termo}
    \definition{s.}{tempo designado (programado) | um período de tempo; fase; estágio}
    \definition{v.}{marcar uma consulta | esperar; supor; imaginar}
  \end{phonetics}
\end{entry}

\begin{entry}{期中}{12,4}{⽉、⼁}
  \begin{phonetics}{期中}{qi1 zhong1}[][HSK 4]
    \definition{adj.}{provisório; interino; intermediário}
  \end{phonetics}
\end{entry}

\begin{entry}{期末}{12,5}{⽉、⽊}
  \begin{phonetics}{期末}{qi1 mo4}[][HSK 4]
    \definition{s.}{terminal; final do prazo; fim do período}
  \end{phonetics}
\end{entry}

\begin{entry}{期间}{12,7}{⽉、⾨}
  \begin{phonetics}{期间}{qi1jian1}[][HSK 4]
    \definition{s.}{prazo; tempo; período}
  \end{phonetics}
\end{entry}

\begin{entry}{期限}{12,8}{⽉、⾩}
  \begin{phonetics}{期限}{qi1xian4}[][HSK 4]
    \definition{s.}{prazo; limite de tempo; tempo alocado; período de tempo limitado, também o limite final do limite de tempo}
  \end{phonetics}
\end{entry}

\begin{entry}{期待}{12,9}{⽉、⼻}
  \begin{phonetics}{期待}{qi1dai4}[][HSK 4]
    \definition{v.}{aguardar; esperar; aguardar ansiosamente; ter em mente a realização de um determinado fim ou a ocorrência de uma determinada situação}
  \end{phonetics}
\end{entry}

\begin{entry}{棉}{12}{⽊}
  \begin{phonetics}{棉}{mian2}
    \definition{s.}{termo genérico para algodão ou paina | algodão | acolchoado ou estofado com algodão}
  \end{phonetics}
\end{entry}

\begin{entry}{棒}{12}{⽊}
  \begin{phonetics}{棒}{bang4}[][HSK 5]
    \definition{adj.}{bom; forte; excelente}
    \definition[根]{s.}{porrete; vara; bastão; cacete; haste}
  \end{phonetics}
\end{entry}

\begin{entry}{棒冰}{12,6}{⽊、⼎}
  \begin{phonetics}{棒冰}{bang4bing1}
    \definition{s.}{picolé}
  \end{phonetics}
\end{entry}

\begin{entry}{棒棒糖}{12,12,16}{⽊、⽊、⽶}
  \begin{phonetics}{棒棒糖}{bang4bang4tang2}
    \definition[根]{s.}{pirulito}
  \end{phonetics}
\end{entry}

\begin{entry}{棕褐色}{12,14,6}{⽊、⾐、⾊}
  \begin{phonetics}{棕褐色}{zong1he4 se4}
    \definition{s.}{cor sépia | bronzeado}
  \end{phonetics}
\end{entry}

\begin{entry}{森林}{12,8}{⽊、⽊}
  \begin{phonetics}{森林}{sen1lin2}[][HSK 4]
    \definition[片,座,处]{s.}{floresta; bosque; normalmente, refere-se a uma grande área de árvores em crescimento; na silvicultura, refere-se a um grande número de árvores que crescem em uma área razoavelmente grande de terra, juntamente com os animais e outras plantas}
  \end{phonetics}
\end{entry}

\begin{entry}{棵}{12}{⽊}
  \begin{phonetics}{棵}{ke1}[][HSK 4]
    \definition{clas.}{para plantas, árvores}
  \end{phonetics}
\end{entry}

\begin{entry}{棹}{12}{⽊}
  \begin{phonetics}{棹}{zhuo1}
    \variantof{桌}
  \end{phonetics}
\end{entry}

\begin{entry}{棺}{12}{⽊}
  \begin{phonetics}{棺}{guan1}
    \definition{s.}{caixão | esquife | ataúde}
  \end{phonetics}
\end{entry}

\begin{entry}{椅子}{12,3}{⽊、⼦}
  \begin{phonetics}{椅子}{yi3zi5}[][HSK 2]
    \definition[把,套]{s.}{cadeira}
  \end{phonetics}
\end{entry}

\begin{entry}{植物}{12,8}{⽊、⽜}
  \begin{phonetics}{植物}{zhi2wu4}[][HSK 4]
    \definition[种,株,盆,棵]{s.}{planta; vegetação; flora}
  \end{phonetics}
\end{entry}

\begin{entry}{椰汁}{12,5}{⽊、⽔}
  \begin{phonetics}{椰汁}{ye1zhi1}
    \definition{s.}{água de coco}
  \end{phonetics}
\end{entry}

\begin{entry}{款}{12}{⽋}
  \begin{phonetics}{款}{kuan3}
    \definition{clas.}{para versões ou modelos (de um produto)}
    \definition[笔,个]{s.}{montante de dinheiro | fundos | parágrafo | seção}
  \end{phonetics}
\end{entry}

\begin{entry}{殖}{12}{⽍}
  \begin{phonetics}{殖}{zhi2}
    \definition{v.}{crescer | reproduzir}
  \end{phonetics}
\end{entry}

\begin{entry}{渡过}{12,6}{⽔、⾡}
  \begin{phonetics}{渡过}{du4guo4}
    \definition{v.}{atravessar | passar por}
  \end{phonetics}
\end{entry}

\begin{entry}{温度}{12,9}{⽔、⼴}
  \begin{phonetics}{温度}{wen1du4}[][HSK 2]
    \definition[个]{s.}{temperatura}
  \end{phonetics}
\end{entry}

\begin{entry}{温度计}{12,9,4}{⽔、⼴、⾔}
  \begin{phonetics}{温度计}{wen1du4ji4}
    \definition{s.}{termógrafo | termômetro}
  \end{phonetics}
\end{entry}

\begin{entry}{温度表}{12,9,8}{⽔、⼴、⾐}
  \begin{phonetics}{温度表}{wen1du4biao3}
    \definition{s.}{termômetro}
  \end{phonetics}
\end{entry}

\begin{entry}{温度梯度}{12,9,11,9}{⽔、⼴、⽊、⼴}
  \begin{phonetics}{温度梯度}{wen1du4ti1du4}
    \definition{s.}{gradiente de temperatura}
  \end{phonetics}
\end{entry}

\begin{entry}{温柔}{12,9}{⽔、⽊}
  \begin{phonetics}{温柔}{wen1rou2}
    \definition{adj.}{gentil e suave | terno | doce (comumente usado para descrever uma menina ou mulher)}
  \end{phonetics}
\end{entry}

\begin{entry}{温暖}{12,13}{⽔、⽇}
  \begin{phonetics}{温暖}{wen1nuan3}[][HSK 3]
    \definition{adj.}{caloroso; gentil}
    \definition{v.}{aquecer (fazer você se sentir aquecido)}
  \end{phonetics}
\end{entry}

\begin{entry}{渴}{12}{⽔}
  \begin{phonetics}{渴}{ke3}[][HSK 1]
    \definition{adj.}{sedento}
  \end{phonetics}
\end{entry}

\begin{entry}{游}{12}{⽔}
  \begin{phonetics}{游}{you2}[][HSK 3]
    \definition*{s.}{sobrenome You}
    \definition{adj.}{itinerante; errante; não fixo; frequentemente em movimento}
    \definition{s.}{parte de um rio; uma seção do rio}
    \definition{v.}{nadar; pessoas ou animais se movendo na água | vagar por aí; vagar; viajar; passear | associar com (comunicação)}
  \end{phonetics}
\end{entry}

\begin{entry}{游戏}{12,6}{⽔、⼽}
  \begin{phonetics}{游戏}{you2xi4}[][HSK 3]
    \definition[场]{s.}{jogo; recreação; atividades recreativas, como esconde-esconde, adivinhação de enigmas de lanternas e algumas atividades esportivas informais, como bola recreativa, também são chamadas de jogos}
    \definition{v.}{jogar; fazer atividades relaxantes e prazerosas sozinho ou com outras pessoas}
  \end{phonetics}
\end{entry}

\begin{entry}{游泳}{12,8}{⽔、⽔}
  \begin{phonetics}{游泳}{you2yong3}[][HSK 3]
    \definition[次]{s.}{natação; refere-se ao esporte ou atividade de natação}
    \definition{v.+compl.}{nadar; pessoas ou animais nadando na água}
  \end{phonetics}
\end{entry}

\begin{entry}{游泳池}{12,8,6}{⽔、⽔、⽔}
  \begin{phonetics}{游泳池}{you2yong3chi2}
    \definition[场]{s.}{piscina}
  \seealsoref{泳池}{yong3chi2}
  \seealsoref{游泳馆}{you2yong3guan3}
  \end{phonetics}
\end{entry}

\begin{entry}{游泳衣}{12,8,6}{⽔、⽔、⾐}
  \begin{phonetics}{游泳衣}{you2yong3yi1}
    \definition{s.}{roupa de banho}
  \seealsoref{泳衣}{yong3yi1}
  \end{phonetics}
\end{entry}

\begin{entry}{游泳馆}{12,8,11}{⽔、⽔、⾷}
  \begin{phonetics}{游泳馆}{you2yong3guan3}
    \definition[场]{s.}{piscina}
  \seealsoref{泳池}{yong3chi2}
  \seealsoref{游泳池}{you2yong3chi2}
  \end{phonetics}
\end{entry}

\begin{entry}{游泳镜}{12,8,16}{⽔、⽔、⾦}
  \begin{phonetics}{游泳镜}{you2yong3jing4}
    \definition{s.}{óculos de natação}
  \end{phonetics}
\end{entry}

\begin{entry}{游客}{12,9}{⽔、⼧}
  \begin{phonetics}{游客}{you2 ke4}[][HSK 2]
    \definition{s.}{viajante | turista | (jogo online) jogador convidado}
  \end{phonetics}
\end{entry}

\begin{entry}{游艇}{12,12}{⽔、⾈}
  \begin{phonetics}{游艇}{you2ting3}
    \definition[只]{s.}{barcaça | iate}
  \end{phonetics}
\end{entry}

\begin{entry}{湖}{12}{⽔}
  \begin{phonetics}{湖}{hu2}[][HSK 2]
    \definition[个,片]{s.}{lago}
  \end{phonetics}
\end{entry}

\begin{entry}{湖南}{12,9}{⽔、⼗}
  \begin{phonetics}{湖南}{hu2nan2}
    \definition*{s.}{Hunan}
  \end{phonetics}
\end{entry}

\begin{entry}{湿}{12}{⽔}
  \begin{phonetics}{湿}{shi1}[][HSK 4]
    \definition{adj.}{molhado; úmido; algo com água ou com muita água dentro}
  \end{phonetics}
\end{entry}

\begin{entry}{滑}{12}{⽔}
  \begin{phonetics}{滑}{hua2}
    \definition*{s.}{sobrenome Hua}
    \definition{adj.}{deslizado}
    \definition{v.}{deslizar}
  \end{phonetics}
\end{entry}

\begin{entry}{滑雪}{12,11}{⽔、⾬}
  \begin{phonetics}{滑雪}{hua2xue3}
    \definition{v.+compl.}{esquiar | praticar esqui}
  \end{phonetics}
\end{entry}

\begin{entry}{焚香}{12,9}{⽕、⾹}
  \begin{phonetics}{焚香}{fen2xiang1}
    \definition{v.}{queimar incenso}
  \end{phonetics}
\end{entry}

\begin{entry}{焦虑}{12,10}{⽕、⾌}
  \begin{phonetics}{焦虑}{jiao1lv4}
    \definition{adj.}{ansioso | preocupado | apreensivo}
  \end{phonetics}
\end{entry}

\begin{entry}{然}{12}{⽕}
  \begin{phonetics}{然}{ran2}
    \definition{conj.}{mas | no entanto}
  \end{phonetics}
\end{entry}

\begin{entry}{然后}{12,6}{⽕、⼝}
  \begin{phonetics}{然后}{ran2hou4}[][HSK 2]
    \definition{conj.}{depois | logo | portanto}
  \end{phonetics}
\end{entry}

\begin{entry}{然而}{12,6}{⽕、⽽}
  \begin{phonetics}{然而}{ran2'er2}[][HSK 4]
    \definition{conj.}{ainda; mas; contudo; todavia; usado no início de uma frase para indicar uma transição; para indicar uma transição, geralmente é precedido por uma conjunção como ``虽然'' para indicar concessão}
  \seealsoref{虽然}{sui1 ran2}
  \end{phonetics}
\end{entry}

\begin{entry}{牌}{12}{⽚}
  \begin{phonetics}{牌}{pai2}[][HSK 4]
    \definition[块]{s.}{placa; tabuleta; quadro; placar | marca; marca registrada; marca comercial | cartas, dominó, etc. | a tonalidade de uma música}
  \end{phonetics}
\end{entry}

\begin{entry}{牌子}{12,3}{⽚、⼦}
  \begin{phonetics}{牌子}{pai2 zi5}[][HSK 3]
    \definition[个,种,块]{s.}{sinal; placa | marca; marca registrada}
  \end{phonetics}
\end{entry}

\begin{entry}{猩猩}{12,12}{⽝、⽝}
  \begin{phonetics}{猩猩}{xing1xing5}
    \definition{s.}{orangotango}
  \end{phonetics}
\end{entry}

\begin{entry}{猴子}{12,3}{⽝、⼦}
  \begin{phonetics}{猴子}{hou2zi5}
    \definition[只]{s.}{macaco}
  \end{phonetics}
\end{entry}

\begin{entry}{琴键}{12,13}{⽟、⾦}
  \begin{phonetics}{琴键}{qin2jian4}
    \definition{s.}{tecla de piano}
  \end{phonetics}
\end{entry}

\begin{entry}{甁}{12}{⽡}
  \begin{phonetics}{甁}{ping2}
    \variantof{瓶}
  \end{phonetics}
\end{entry}

\begin{entry}{番茄}{12,8}{⽥、⾋}
  \begin{phonetics}{番茄}{fan1qie2}
    \definition{s.}{tomate}
  \end{phonetics}
\end{entry}

\begin{entry}{痛}{12}{⽧}
  \begin{phonetics}{痛}{tong4}[][HSK 3]
    \definition*{s.}{sobrenome Tong}
    \definition{s.}{pesar; angústia; aflição; tristeza}
    \definition{v.}{doer; causar dor}
  \end{phonetics}
\end{entry}

\begin{entry}{痛快}{12,7}{⽧、⼼}
  \begin{phonetics}{痛快}{tong4kuai4}[][HSK 4]
    \definition{adj.}{encantado; alegre; muito feliz; confortável | franco; direto; simples e direto}
  \end{phonetics}
\end{entry}

\begin{entry}{痛苦}{12,8}{⽧、⾋}
  \begin{phonetics}{痛苦}{tong4ku3}[][HSK 3]
    \definition{adj.}{doloroso; angustiado}
    \definition[降,种]{s.}{dor; agonia; sofrimento}
  \end{phonetics}
\end{entry}

\begin{entry}{痛骂}{12,9}{⽧、⾺}
  \begin{phonetics}{痛骂}{tong4ma4}
    \definition{v.}{repreender severamente}
  \end{phonetics}
\end{entry}

\begin{entry}{痠}{12}{⽧}
  \begin{phonetics}{痠}{suan1}
    \definition{v.}{doer | estar dolorido}
    \variantof{酸}
  \end{phonetics}
\end{entry}

\begin{entry}{登}{12}{⽨}
  \begin{phonetics}{登}{deng1}[][HSK 4]
    \definition*{s.}{sobrenome Deng}
    \definition{v.}{subir; montar; escalar (uma altura) | publicar; registrar; inserir | ser colhidas e levadas para a eira | pressionar com o pé; pedalar; pisar | pisar em; pisar | calçar (calçados, etc.)}
  \end{phonetics}
\end{entry}

\begin{entry}{登山}{12,3}{⽨、⼭}
  \begin{phonetics}{登山}{deng1 shan1}[][HSK 4]
    \definition{s.}{escalar; fazer alpinismo; subir uma montanha}
  \end{phonetics}
\end{entry}

\begin{entry}{登记}{12,5}{⽨、⾔}
  \begin{phonetics}{登记}{deng1ji4}[][HSK 4]
    \definition{v.+compl.}{registrar-se; fazer o \emph{check-in} | registrar; reportar; informar; relatar por escrito a um superior ou autoridade relevante (usado principalmente para documentos legais)}
  \end{phonetics}
\end{entry}

\begin{entry}{登录}{12,8}{⽨、⼹}
  \begin{phonetics}{登录}{deng1lu4}[][HSK 4]
    \definition{v.}{fazer \emph{logon}; fazer \emph{login} | gravar; registrar; computadores eletrônicos e sua terminologia de rede, referindo-se ao acesso ao sistema operacional ou ao site a ser visitado}
  \end{phonetics}
\end{entry}

\begin{entry}{短}{12}{⽮}
  \begin{phonetics}{短}{duan3}[][HSK 2]
    \definition{adj.}{curto | breve}
  \end{phonetics}
\end{entry}

\begin{entry}{短少}{12,4}{⽮、⼩}
  \begin{phonetics}{短少}{duan3shao3}
    \definition{v.}{estar aquém do valor total}
  \end{phonetics}
\end{entry}

\begin{entry}{短处}{12,5}{⽮、⼡}
  \begin{phonetics}{短处}{duan3 chu4}[][HSK 3]
    \definition{s.}{deficiência; ponto fraco; defeito; fraqueza}
  \end{phonetics}
\end{entry}

\begin{entry}{短视}{12,8}{⽮、⾒}
  \begin{phonetics}{短视}{duan3shi4}
    \definition{adj.}{míope}
  \end{phonetics}
\end{entry}

\begin{entry}{短促}{12,9}{⽮、⼈}
  \begin{phonetics}{短促}{duan3cu4}
    \definition{adj.}{curto (tom de voz) | fugaz | ofegante (respiração) | curto no tempo}
  \end{phonetics}
\end{entry}

\begin{entry}{短信}{12,9}{⽮、⼈}
  \begin{phonetics}{短信}{duan3xin4}[][HSK 2]
    \definition{s.}{mensagem de texto}
  \end{phonetics}
\end{entry}

\begin{entry}{短缺}{12,10}{⽮、⽸}
  \begin{phonetics}{短缺}{duan3que1}
    \definition{s.}{escassez}
  \end{phonetics}
\end{entry}

\begin{entry}{短暂}{12,12}{⽮、⽇}
  \begin{phonetics}{短暂}{duan3zan4}
    \definition{adj.}{momentâneo | de curta duração}
  \end{phonetics}
\end{entry}

\begin{entry}{短期}{12,12}{⽮、⽉}
  \begin{phonetics}{短期}{duan3 qi1}[][HSK 3]
    \definition{adj.}{curto prazo}
    \definition[个]{s.}{curto período}
  \end{phonetics}
\end{entry}

\begin{entry}{短裤}{12,12}{⽮、⾐}
  \begin{phonetics}{短裤}{duan3 ku4}[][HSK 3]
    \definition[条]{s.}{calças curtas; calção; \emph{shorts}}
  \end{phonetics}
\end{entry}

\begin{entry}{短跑}{12,12}{⽮、⾜}
  \begin{phonetics}{短跑}{duan3pao3}
    \definition{s.}{corrida}
  \end{phonetics}
\end{entry}

\begin{entry}{硬件}{12,6}{⽯、⼈}
  \begin{phonetics}{硬件}{ying4jian4}
    \definition{s.}{\emph{hardware}}
  \end{phonetics}
\end{entry}

\begin{entry}{确}{12}{⽯}
  \begin{phonetics}{确}{que4}
    \definition{adj.}{autenticado | sólido | firme | real | verdadeiro}
  \end{phonetics}
\end{entry}

\begin{entry}{确认}{12,4}{⽯、⾔}
  \begin{phonetics}{确认}{que4ren4}[][HSK 4]
    \definition{v.}{afirmar; confirmar; reconhecer; confirmar explicitamente (fatos, princípios, etc.)}
  \end{phonetics}
\end{entry}

\begin{entry}{确定}{12,8}{⽯、⼧}
  \begin{phonetics}{确定}{que4ding4}[][HSK 3]
    \definition{adj.}{definido; certo}
    \definition{v.}{consertar; definir; determinar}
  \end{phonetics}
\end{entry}

\begin{entry}{确实}{12,8}{⽯、⼧}
  \begin{phonetics}{确实}{que4shi2}[][HSK 3]
    \definition{adj.}{verdadeiro; confiável}
    \definition{adv.}{verdadeiramente; realmente; de ​​fato}
  \end{phonetics}
\end{entry}

\begin{entry}{确保}{12,9}{⽯、⼈}
  \begin{phonetics}{确保}{que4bao3}[][HSK 3]
    \definition{v.}{assegurar; garantir}
  \end{phonetics}
\end{entry}

\begin{entry}{禅}{12}{⽰}
  \begin{phonetics}{禅}{chan2}
    \definition*{s.}{Zen}
    \definition{s.}{meditação (Budismo)}
  \end{phonetics}
  \begin{phonetics}{禅}{shan4}
    \definition{v.}{abdicar}
  \end{phonetics}
\end{entry}

\begin{entry}{禽}{12}{⽱}
  \begin{phonetics}{禽}{qin2}
    \definition*{s.}{sobrenome Qin}
    \definition[只]{s.}{aves; pássaros | termo genérico para aves e animais}
  \end{phonetics}
\end{entry}

\begin{entry}{程序}{12,7}{⽲、⼴}
  \begin{phonetics}{程序}{cheng2xu4}[][HSK 4]
    \definition[个,套,种]{s.}{ordem; curso; sequência; procedimento; ordem em que algo é feito; também, um determinado número de etapas em um trabalho | programa; conjunto de instruções de computador projetado em sequência para permitir que um computador execute uma ou mais operações}
  \end{phonetics}
\end{entry}

\begin{entry}{程序设计}{12,7,6,4}{⽲、⼴、⾔、⾔}
  \begin{phonetics}{程序设计}{cheng2xu4she4ji4}
    \definition{s.}{programação de computadores}
  \end{phonetics}
\end{entry}

\begin{entry}{程序库}{12,7,7}{⽲、⼴、⼴}
  \begin{phonetics}{程序库}{cheng2xu4ku4}
    \definition{s.}{biblioteca de funções e procedimentos para programas de computador}
  \end{phonetics}
\end{entry}

\begin{entry}{程度}{12,9}{⽲、⼴}
  \begin{phonetics}{程度}{cheng2du4}[][HSK 3]
    \definition[种]{s.}{nível; grau (de cultura, educação, aprendizagem, etc.) | extensão; grau}
  \end{phonetics}
\end{entry}

\begin{entry}{程控}{12,11}{⽲、⼿}
  \begin{phonetics}{程控}{cheng2kong4}
    \definition{s.}{programado | sob controle automático}
  \end{phonetics}
\end{entry}

\begin{entry}{稍}{12}{⽲}
  \begin{phonetics}{稍}{shao1}
    \definition{adv.}{um pouco | ligeiramente | em vez de}
  \end{phonetics}
\end{entry}

\begin{entry}{稍微}{12,13}{⽲、⼻}
  \begin{phonetics}{稍微}{shao1wei1}
    \definition{adv.}{um pouco}
  \end{phonetics}
\end{entry}

\begin{entry}{税}{12}{⽲}
  \begin{phonetics}{税}{shui4}
    \definition{s.}{taxas | impostos}
  \end{phonetics}
\end{entry}

\begin{entry}{窗子}{12,3}{⽳、⼦}
  \begin{phonetics}{窗子}{chuang1 zi5}[][HSK 4]
    \definition{s.}{janela}
  \end{phonetics}
\end{entry}

\begin{entry}{窗户}{12,4}{⽳、⼾}
  \begin{phonetics}{窗户}{chuang1hu5}[][HSK 4]
    \definition[个,扇,面,排]{s.}{janela; dispositivo de ventilação e transmissão de luz nas paredes}
  \end{phonetics}
\end{entry}

\begin{entry}{窗台}{12,5}{⽳、⼝}
  \begin{phonetics}{窗台}{chuang1 tai2}[][HSK 4]
    \definition{s.}{parapeito da janela; peitoril; parte plana de uma janela que segura a moldura}
  \end{phonetics}
\end{entry}

\begin{entry}{窗帘}{12,8}{⽳、⼱}
  \begin{phonetics}{窗帘}{chuang1lian2}[][HSK 5]
    \definition[副,幅,个,套,片,对]{s.}{cortinas para janelas}
  \end{phonetics}
\end{entry}

\begin{entry}{童年}{12,6}{⽴、⼲}
  \begin{phonetics}{童年}{tong2 nian2}[][HSK 4]
    \definition{s.}{infância; primeiros anos de vida}
  \end{phonetics}
\end{entry}

\begin{entry}{童话}{12,8}{⽴、⾔}
  \begin{phonetics}{童话}{tong2hua4}[][HSK 4]
    \definition[个,部]{s.}{conto de fadas; gênero de literatura infantil no qual as histórias adequadas para a diversão das crianças são escritas com muita imaginação, fantasia e exagero}
  \end{phonetics}
\end{entry}

\begin{entry}{等}{12}{⽵}
  \begin{phonetics}{等}{deng3}[][HSK 1,2]
    \definition{adj.}{igual}
    \definition{clas.}{para classe, grau, classificação | para tipo}
    \definition{prep.}{quando | até}
    \definition{v.}{esperar | aguardar}
  \end{phonetics}
\end{entry}

\begin{entry}{等于}{12,3}{⽵、⼆}
  \begin{phonetics}{等于}{deng3yu2}[][HSK 2]
    \definition{adv.}{igual a | equivalente a}
    \definition{v.}{equivaler a | ser equivalente a}
  \end{phonetics}
\end{entry}

\begin{entry}{等级}{12,6}{⽵、⽷}
  \begin{phonetics}{等级}{deng3ji2}[][HSK 5]
    \definition{s.}{grau; classificação; posição; distinções por qualidade, grau, status, etc. | estado social; estrato social; ordem e grau; grupos sociais desiguais em termos de status social e legal}
  \end{phonetics}
\end{entry}

\begin{entry}{等到}{12,8}{⽵、⼑}
  \begin{phonetics}{等到}{deng3 dao4}[][HSK 2]
    \definition{prep.}{pelo tempo | quando | espere até}
  \end{phonetics}
\end{entry}

\begin{entry}{等待}{12,9}{⽵、⼻}
  \begin{phonetics}{等待}{deng3dai4}[][HSK 3]
    \definition{v.}{esperar; aguardar}
  \end{phonetics}
\end{entry}

\begin{entry}{等候}{12,10}{⽵、⼈}
  \begin{phonetics}{等候}{deng3hou4}[][HSK 5]
    \definition{v.}{esperar; aguardar; expectar; usado principalmente para objetos específicos}
  \end{phonetics}
\end{entry}

\begin{entry}{筏}{12}{⽵}
  \begin{phonetics}{筏}{fa2}
    \definition{s.}{jangada (de troncos, bambus, etc.)}
  \end{phonetics}
\end{entry}

\begin{entry}{答}{12}{⽵}
  \begin{phonetics}{答}{da1}[][HSK 5]
    \definition{v.}{concordar; responder | responder; prestar atenção}
  \end{phonetics}
  \begin{phonetics}{答}{da2}[][HSK 5]
    \definition{v.}{responder; dar resposta a; responder a | retribuir; devolver (uma visita, etc.); retribuir um favor feito a alguém por outro; fazer o bem}
  \end{phonetics}
\end{entry}

\begin{entry}{答应}{12,7}{⽵、⼴}
  \begin{phonetics}{答应}{da1ying5}[][HSK 2]
    \definition{v.}{responder | concordar | prometer | cumprir com}
  \end{phonetics}
\end{entry}

\begin{entry}{答复}{12,9}{⽵、⼢}
  \begin{phonetics}{答复}{da2fu4}[][HSK 5]
    \definition[个]{s.}{resposta; respostas a perguntas ou solicitações}
    \definition{v.}{responder; dar uma resposta}
  \end{phonetics}
\end{entry}

\begin{entry}{答案}{12,10}{⽵、⽊}
  \begin{phonetics}{答案}{da2'an4}[][HSK 4]
    \definition[个]{s.}{chave; resposta; solução}
  \end{phonetics}
\end{entry}

\begin{entry}{策划}{12,6}{⽵、⼑}
  \begin{phonetics}{策划}{ce4hua4}
    \definition{s.}{planejador | produtor | plano}
    \definition{v.}{esquematizar | engenhar | planejar}
  \end{phonetics}
\end{entry}

\begin{entry}{粤语}{12,9}{⾔、⾔}
  \begin{phonetics}{粤语}{yue4yu3}
    \definition{s.}{cantonês | língua cantonesa}
  \end{phonetics}
\end{entry}

\begin{entry}{紫}{12}{⽷}
  \begin{phonetics}{紫}{zi3}
    \definition{adj.}{púrpura | violeta}
  \end{phonetics}
\end{entry}

\begin{entry}{紫色}{12,6}{⽷、⾊}
  \begin{phonetics}{紫色}{zi3 se4}
    \definition{s.}{cor púrpura | cor violeta}
  \end{phonetics}
\end{entry}

\begin{entry}{絫}{12}{⽷}
  \begin{phonetics}{絫}{lei3}
    \variantof{累}
  \end{phonetics}
\end{entry}

\begin{entry}{缓解}{12,13}{⽶、⾓}
  \begin{phonetics}{缓解}{huan3jie3}[][HSK 4]
    \definition{v.}{facilitar; aliviar; atenuar; amenizar; reduzir}
  \end{phonetics}
\end{entry}

\begin{entry}{编}{12}{⽷}
  \begin{phonetics}{编}{bian1}[][HSK 4]
    \definition*{s.}{sobrenome Bian}
    \definition{s.}{livro; volume; parte de um livro}
    \definition{v.}{tecer; trançar; entrançar | fazer uma lista; organizar em uma lista; organizar; agrupar | editar; compilar | compor; escrever | fabricar; inventar; fazer; preparar}
  \end{phonetics}
\end{entry}

\begin{entry}{编程}{12,12}{⽷、⽲}
  \begin{phonetics}{编程}{bian1cheng2}
    \definition{s.}{programa de computador}
    \definition{v.}{programar computador}
  \end{phonetics}
\end{entry}

\begin{entry}{编辑}{12,13}{⽷、⾞}
  \begin{phonetics}{编辑}{bian1ji2}[][HSK 5]
    \definition{v.}{editar; compilar; organizar e processar dados ou trabalhos existentes}
  \end{phonetics}
  \begin{phonetics}{编辑}{bian1ji5}[][HSK 5]
    \definition{s.}{editor; compilador; pessoa que organiza e processa dados ou trabalhos existentes}
  \end{phonetics}
\end{entry}

\begin{entry}{缘}{12}{⽷}
  \begin{phonetics}{缘}{yuan2}
    \definition{s.}{causa | razão | karma | destino | predestinação}
  \end{phonetics}
\end{entry}

\begin{entry}{缘分}{12,4}{⽷、⼑}
  \begin{phonetics}{缘分}{yuan2fen4}
    \definition{s.}{destino ou acaso que une as pessoas | afinidade ou relacionamento predestinado | destino (Budismo)}
  \end{phonetics}
\end{entry}

\begin{entry}{羡慕}{12,14}{⽺、⼼}
  \begin{phonetics}{羡慕}{xian4mu4}
    \definition{v.}{invejar | admirar}
  \end{phonetics}
\end{entry}

\begin{entry}{联合}{12,6}{⽿、⼝}
  \begin{phonetics}{联合}{lian2he2}[][HSK 3]
    \definition{adj.}{conjunto; unido; federal; combinado}
    \definition{s.}{aliado; união; aliança}
  \end{phonetics}
\end{entry}

\begin{entry}{联合会}{12,6,6}{⽿、⼝、⼈}
  \begin{phonetics}{联合会}{lian2he2hui4}
    \definition{s.}{federação}
  \end{phonetics}
\end{entry}

\begin{entry}{联合国}{12,6,8}{⽿、⼝、⼞}
  \begin{phonetics}{联合国}{lian2 he2 guo2}[][HSK 3]
    \definition*{s.}{Nações Unidas}
  \end{phonetics}
\end{entry}

\begin{entry}{联系}{12,7}{⽿、⽷}
  \begin{phonetics}{联系}{lian2xi4}[][HSK 3]
    \definition{s.}{relacionamento; conexão}
    \definition[个,种,层]{v.}{entrar em contato; contatar | organizar; entrar em contato com | relacionar; combinar; integrar}
  \end{phonetics}
\end{entry}

\begin{entry}{脾气}{12,4}{⾁、⽓}
  \begin{phonetics}{脾气}{pi2qi5}
    \definition{s.}{temperamento | humor | disposição | caráter}
  \end{phonetics}
\end{entry}

\begin{entry}{舒服}{12,8}{⾆、⽉}
  \begin{phonetics}{舒服}{shu1fu5}[][HSK 2]
    \definition{adj.}{estar confortável | bem disposto | sentir-se bem}
  \end{phonetics}
\end{entry}

\begin{entry}{舒适}{12,9}{⾆、⾡}
  \begin{phonetics}{舒适}{shu1shi4}[][HSK 4]
    \definition{adj.}{aconchegante; confortável; acolhedor; cômodo}
  \end{phonetics}
\end{entry}

\begin{entry}{落}{12}{⾋}
  \begin{phonetics}{落}{la4}
    \definition{v.}{deixar de fora; estar ausente | deixar para trás; esquecer de trazer | ficar para trás (ou cair)}
  \end{phonetics}
  \begin{phonetics}{落}{lao4}
    \definition{v.}{cair | descer | ficar; fazer escala; deixar para trás | obter; ter; receber}
  \end{phonetics}
  \begin{phonetics}{落}{luo4}[][HSK 4]
    \definition*{s.}{sobrenome Luo}
    \definition{s.}{paradeiro; lugar para ficar; local de descanso | assentamento; local de reunião | parte curta; área pequena; refere-se a um pequeno lugar ou área}
    \definition{v.}{cair | descer | baixar; deixar cair (ou descer) | afundar; declinar; cair; (figurativo) mudança da prosperidade para o declínio | ficar para trás; deixar para trás ou do lado de fora | ficar; parar; deixar para trás | cair em cima de; descansar com | obter; ter; receber | escrever; colocar a caneta no papel | cair em; entrar em}
  \end{phonetics}
\end{entry}

\begin{entry}{落日}{12,4}{⾋、⽇}
  \begin{phonetics}{落日}{luo4ri4}
    \definition{s.}{pôr do sol}
  \end{phonetics}
\end{entry}

\begin{entry}{落后}{12,6}{⾋、⼝}
  \begin{phonetics}{落后}{luo4hou4}[][HSK 3]
    \definition{adj.}{atrasado}
    \definition{s.}{atraso}
    \definition{v.}{ficar para trás | atrasar}
    \definition{v.}{atrasar-se; ficar para trás}
  \end{phonetics}
\end{entry}

\begin{entry}{落汤鸡}{12,6,7}{⾋、⽔、⿃}
  \begin{phonetics}{落汤鸡}{luo4tang1ji1}
    \definition{s.}{uma pessoa que parece encharcada e acamada| sofrimento profundo}
  \end{phonetics}
\end{entry}

\begin{entry}{葡}{12}{⾋}
  \begin{phonetics}{葡}{pu2}
    \definition*{s.}{Portugal, abreviação de 葡萄牙}
    \seeref{葡萄牙}{pu2tao2ya2}
  \end{phonetics}
\end{entry}

\begin{entry}{葡文}{12,4}{⾋、⽂}
  \begin{phonetics}{葡文}{pu2wen2}
    \definition{s.}{português, língua portuguesa}
    \seeref{葡萄牙文}{pu2tao2ya2wen2}
  \end{phonetics}
\end{entry}

\begin{entry}{葡汉词典}{12,5,7,8}{⾋、⽔、⾔、⼋}
  \begin{phonetics}{葡汉词典}{pu2-han4 ci2dian3}
    \definition{s.}{dicionário português-chinês}
  \seealsoref{汉葡词典}{han4-pu2 ci2dian3}
  \end{phonetics}
\end{entry}

\begin{entry}{葡语}{12,9}{⾋、⾔}
  \begin{phonetics}{葡语}{pu2yu3}
    \definition{s.}{português, língua portuguesa}
    \seeref{葡萄牙语}{pu2tao2ya2yu3}
  \end{phonetics}
\end{entry}

\begin{entry}{葡萄}{12,11}{⾋、⾋}
  \begin{phonetics}{葡萄}{pu2tao5}
    \definition{s.}{uva}
  \end{phonetics}
\end{entry}

\begin{entry}{葡萄牙}{12,11,4}{⾋、⾋、⽛}
  \begin{phonetics}{葡萄牙}{pu2tao2ya2}
    \definition{s.}{Portugal}
    \seeref{葡}{pu2}
  \end{phonetics}
\end{entry}

\begin{entry}{葡萄牙文}{12,11,4,4}{⾋、⾋、⽛、⽂}
  \begin{phonetics}{葡萄牙文}{pu2tao2ya2wen2}
    \definition{s.}{português, língua portuguesa}
    \seeref{葡文}{pu2wen2}
  \end{phonetics}
\end{entry}

\begin{entry}{葡萄牙语}{12,11,4,9}{⾋、⾋、⽛、⾔}
  \begin{phonetics}{葡萄牙语}{pu2tao2ya2yu3}
    \definition{s.}{português, língua portuguesa}
    \seeref{葡语}{pu2yu3}
  \end{phonetics}
\end{entry}

\begin{entry}{葫芦}{12,7}{⾋、⾋}
  \begin{phonetics}{葫芦}{hu2lu5}
    \definition{adj.}{confuso}
    \definition{s.}{cabaça | termo genérico para bloco e equipamento (ou partes dele)}
  \end{phonetics}
\end{entry}

\begin{entry}{葬}{12}{⾋}
  \begin{phonetics}{葬}{zang4}
    \definition{v.}{enterrar (os mortos) | sepultar}
  \end{phonetics}
\end{entry}

\begin{entry}{葱}{12}{⾋}
  \begin{phonetics}{葱}{cong1}
    \definition{s.}{cebolinha}
  \end{phonetics}
\end{entry}

\begin{entry}{葵花}{12,7}{⾋、⾋}
  \begin{phonetics}{葵花}{kui2hua1}
    \definition{s.}{girassol (flor)}
  \end{phonetics}
\end{entry}

\begin{entry}{街}{12}{⾏}
  \begin{phonetics}{街}{jie1}[][HSK 2]
    \definition[条]{s.}{rua}
  \end{phonetics}
\end{entry}

\begin{entry}{街道}{12,12}{⾏、⾡}
  \begin{phonetics}{街道}{jie1dao4}[][HSK 4]
    \definition[条]{s.}{caminho; rua; estrada; via pública com casas em ambos os lados, relativamente larga | escritório do subdistrito; tipo de organização responsável por gerenciar determinados aspectos da rua}
  \end{phonetics}
\end{entry}

\begin{entry}{街舞}{12,14}{⾏、⾇}
  \begin{phonetics}{街舞}{jie1wu3}
    \definition{s.}{dança de rua, \emph{street dance} (por exemplo, \emph{breakdance})}
  \end{phonetics}
\end{entry}

\begin{entry}{裁}{12}{⾐}
  \begin{phonetics}{裁}{cai2}
    \definition{s.}{decisão | julgamento}
    \definition{v.}{recortar (tecido de uma roupa) | cortar | aparar | reduzir | diminuir | cortar pessoal de uma equipe}
  \end{phonetics}
\end{entry}

\begin{entry}{裁判}{12,7}{⾐、⼑}
  \begin{phonetics}{裁判}{cai2pan4}[][HSK 5]
    \definition[个,位,名]{s.}{árbitro; juiz; pessoa que desempenha funções de arbitragem em esportes e outras competições}
    \definition{v.}{arbitrar; atuar como árbitro; em esportes e outras atividades competitivas, julgar o desempenho dos atletas, vitórias e derrotas, classificações e problemas que ocorrem durante a competição de acordo com as regras da competição | julgar; refere-se a um terceiro que faz um julgamento quando surge uma disputa entre duas partes}
  \end{phonetics}
\end{entry}

\begin{entry}{装}{12}{⾐}
  \begin{phonetics}{装}{zhuang1}[][HSK 2]
    \definition{s.}{adorno | roupa | traje (de um ator em uma peça)}
    \definition{v.}{adornar | vestir | desepenhar um papel | fingir | instalar | consertar | embrulhar (algo em um saco) | empacotar}
  \end{phonetics}
\end{entry}

\begin{entry}{装扮}{12,7}{⾐、⼿}
  \begin{phonetics}{装扮}{zhuang1ban4}
    \definition{v.}{enfeitar | decorar | disfarçar-me | vestir-se}
  \end{phonetics}
\end{entry}

\begin{entry}{装备}{12,8}{⾐、⼡}
  \begin{phonetics}{装备}{zhuang1bei4}
    \definition{s.}{equipamento}
    \definition{v.}{equipar}
  \end{phonetics}
\end{entry}

\begin{entry}{装修}{12,9}{⾐、⼈}
  \begin{phonetics}{装修}{zhuang1 xiu1}[][HSK 4]
    \definition{v.}{equipar; renovar; decorar (equipar uma sala ou prédio com equipamentos ou decorações)}
  \end{phonetics}
\end{entry}

\begin{entry}{装配}{12,10}{⾐、⾣}
  \begin{phonetics}{装配}{zhuang1pei4}
    \definition{v.}{montar | encaixar}
  \end{phonetics}
\end{entry}

\begin{entry}{装置}{12,13}{⾐、⽹}
  \begin{phonetics}{装置}{zhuang1 zhi4}[][HSK 4]
    \definition{s.}{dispositivo; equipamento; máquinas, instrumentos ou outros equipamentos de construção mais complexa e com alguma função independente}
    \definition{v.}{instalar; ajustar; configurar; equipar; montar}
  \end{phonetics}
\end{entry}

\begin{entry}{裙子}{12,3}{⾐、⼦}
  \begin{phonetics}{裙子}{qun2zi5}[][HSK 3]
    \definition[条,件]{s.}{saia (peça de vestuário)}
  \end{phonetics}
\end{entry}

\begin{entry}{裤子}{12,3}{⾐、⼦}
  \begin{phonetics}{裤子}{ku4zi5}[][HSK 3]
    \definition[条]{s.}{calças}
  \end{phonetics}
\end{entry}

\begin{entry}{詈骂}{12,9}{⾔、⾺}
  \begin{phonetics}{詈骂}{li4ma4}
    \definition{v.}{xingar | abusar}
  \end{phonetics}
\end{entry}

\begin{entry}{谢天谢地}{12,4,12,6}{⾔、⼤、⾔、⼟}
  \begin{phonetics}{谢天谢地}{xie4tian1xie4di4}
    \definition{expr.}{agradecer a Deus | agradecer aos céus}
  \end{phonetics}
\end{entry}

\begin{entry}{谢世}{12,5}{⾔、⼀}
  \begin{phonetics}{谢世}{xie4shi4}
    \definition{v.}{morrer | falecer}
  \end{phonetics}
\end{entry}

\begin{entry}{谢恩}{12,10}{⾔、⼼}
  \begin{phonetics}{谢恩}{xie4'en1}
    \definition{v.}{agradecer a alguém pelo favor (especialmente imperador ou oficial superior)}
  \end{phonetics}
\end{entry}

\begin{entry}{谢病}{12,10}{⾔、⽧}
  \begin{phonetics}{谢病}{xie4bing4}
    \definition{v.}{desculpar-se por causa de doença}
  \end{phonetics}
\end{entry}

\begin{entry}{谢媒}{12,12}{⾔、⼥}
  \begin{phonetics}{谢媒}{xie4mei2}
    \definition{v.}{agradecer ao casamenteiro}
  \end{phonetics}
\end{entry}

\begin{entry}{谢谢}{12,12}{⾔、⾔}
  \begin{phonetics}{谢谢}{xie4xie5}[][HSK 1]
    \definition{interj.}{Obrigado!}
    \definition{v.}{agradecer}
  \end{phonetics}
\end{entry}

\begin{entry}{谢意}{12,13}{⾔、⼼}
  \begin{phonetics}{谢意}{xie4yi4}
    \definition{s.}{gratidão}
  \end{phonetics}
\end{entry}

\begin{entry}{貂}{12}{⾘}
  \begin{phonetics}{貂}{diao1}
    \definition{s.}{marta | fuinha}
  \end{phonetics}
\end{entry}

\begin{entry}{赏}{12}{⾙}
  \begin{phonetics}{赏}{shang3}[][HSK 4]
    \definition*{s.}{sobrenome Shang}
    \definition{s.}{recompensa; prêmio}
    \definition{v.}{conceder (outorgar) uma recompensa; recompensar; premiar | admirar; desfrutar; apreciar; valorizar}
  \end{phonetics}
\end{entry}

\begin{entry}{赏心悦目}{12,4,10,5}{⾙、⼼、⼼、⽬}
  \begin{phonetics}{赏心悦目}{shang3xin1yue4mu4}
    \definition{expr.}{``Aquece o coração e encanta os olhos.''}
  \end{phonetics}
\end{entry}

\begin{entry}{赏赐}{12,12}{⾙、⾙}
  \begin{phonetics}{赏赐}{shang3ci4}
    \definition{s.}{recompensa | prêmio}
    \definition{v.}{recompensar | premiar}
  \end{phonetics}
\end{entry}

\begin{entry}{赔钱}{12,10}{⾙、⾦}
  \begin{phonetics}{赔钱}{pei2qian2}
    \definition{v.+compl.}{perder dinheiro | pagar pelos danos}
  \end{phonetics}
\end{entry}

\begin{entry}{超市}{12,5}{⾛、⼱}
  \begin{phonetics}{超市}{chao1shi4}[][HSK 2]
    \definition[家]{s.}{supermercado}
  \end{phonetics}
\end{entry}

\begin{entry}{超级}{12,6}{⾛、⽷}
  \begin{phonetics}{超级}{chao1ji2}[][HSK 3]
    \definition{adj.}{super}
    \definition{pref.}{super-; ultra-; hiper-}
  \end{phonetics}
\end{entry}

\begin{entry}{超过}{12,6}{⾛、⾡}
  \begin{phonetics}{超过}{chao1guo4}[][HSK 2]
    \definition{v.}{passar | ultrapassar (alguém ou algo) | exceder | ser mais do que | estar acima de (um padrão)}
  \end{phonetics}
\end{entry}

\begin{entry}{超声}{12,7}{⾛、⼠}
  \begin{phonetics}{超声}{chao1sheng1}
    \definition{adj.}{ultrasônico}
    \definition{s.}{ultrasom}
  \end{phonetics}
\end{entry}

\begin{entry}{超越}{12,12}{⾛、⾛}
  \begin{phonetics}{超越}{chao1yue4}[][HSK 5]
    \definition{v.}{ultrapassar; superar; passar por cima; transcender}
  \end{phonetics}
\end{entry}

\begin{entry}{越}{12}{⾛}
  \begin{phonetics}{越}{yue4}[][HSK 2]
    \definition{adv.}{quanto mais\dots mais}
    \definition{v.}{subir | exceder | superar}
  \end{phonetics}
\end{entry}

\begin{entry}{越来越……}{12,7,12}{⾛、⽊、⾛}
  \begin{phonetics}{越来越……}{yue4lai2yue4}[][HSK 2]
    \definition{adv.}{cada vez mais\dots}
  \end{phonetics}
\end{entry}

\begin{entry}{越……越……}{12,12}{⾛、⾛}
  \begin{phonetics}{越……越……}{yue4 yue4}[][HSK 2]
    \definition{expr.}{quanto mais\dots tanto mais\dots}
  \end{phonetics}
\end{entry}

\begin{entry}{越障}{12,13}{⾛、⾩}
  \begin{phonetics}{越障}{yue4zhang4}
    \definition{s.}{curso com obstáculos para treinamento de tropas}
    \definition{v.}{superar obstáculos}
  \end{phonetics}
\end{entry}

\begin{entry}{越境}{12,14}{⾛、⼟}
  \begin{phonetics}{越境}{yue4jing4}
    \definition{v.}{cruzar uma fronteira (geralmente ilegalmente) | entrar ou sair furtivamente de um país}
  \end{phonetics}
\end{entry}

\begin{entry}{趋势}{12,8}{⾛、⼒}
  \begin{phonetics}{趋势}{qu1shi4}[][HSK 4]
    \definition{s.}{tendência; tendência; direção; impulso das coisas que se movem em uma direção ou outra}
  \end{phonetics}
\end{entry}

\begin{entry}{跑}{12}{⾜}
  \begin{phonetics}{跑}{pao2}
    \definition{v.}{(de um animal) dar patadas (no chão)}
  \end{phonetics}
  \begin{phonetics}{跑}{pao3}[][HSK 1]
    \definition{v.}{vazar ou evaporar (sobre um gás ou líquido) | escapar | correr | correr (em tarefas, etc.) | fugir}
  \end{phonetics}
\end{entry}

\begin{entry}{跑马}{12,3}{⾜、⾺}
  \begin{phonetics}{跑马}{pao3ma3}
    \definition{s.}{corrida de cavalos}
    \definition{v.}{andar a cavalo em ritmo acelerado}
  \end{phonetics}
\end{entry}

\begin{entry}{跑步}{12,7}{⾜、⽌}
  \begin{phonetics}{跑步}{pao3bu4}[][HSK 3]
    \definition{s.}{corrida}
    \definition{v.+compl.}{correr; trotar}
  \end{phonetics}
\end{entry}

\begin{entry}{跑肚}{12,7}{⾜、⾁}
  \begin{phonetics}{跑肚}{pao3du4}
    \definition{v.}{(coloquial) ter diarréia}
  \end{phonetics}
\end{entry}

\begin{entry}{跑调}{12,10}{⾜、⾔}
  \begin{phonetics}{跑调}{pao3diao4}
    \definition{v.}{(coloquial) estar fora do tom ou desafinado (enquanto canta)}
  \end{phonetics}
\end{entry}

\begin{entry}{跑掉}{12,11}{⾜、⼿}
  \begin{phonetics}{跑掉}{pao3diao4}
    \definition{v.}{fugir}
  \end{phonetics}
\end{entry}

\begin{entry}{跑腿}{12,13}{⾜、⾁}
  \begin{phonetics}{跑腿}{pao3tui3}
    \definition{v.}{realizar tarefas}
  \end{phonetics}
\end{entry}

\begin{entry}{跑酷}{12,14}{⾜、⾣}
  \begin{phonetics}{跑酷}{pao3ku4}
    \definition*{s.}{(empréstimo linguístico) \emph{Parkour}}
  \end{phonetics}
\end{entry}

\begin{entry}{跑题}{12,15}{⾜、⾴}
  \begin{phonetics}{跑题}{pao3ti2}
    \definition{v.}{divagar | fugir do assunto | tergiversar}
  \end{phonetics}
\end{entry}

\begin{entry}{辈}{12}{⾞}
  \begin{phonetics}{辈}{bei4}[][HSK 5]
    \definition{s.}{geração da família | semelhante; círculo familiar; pessoas de um determinado tipo | vida útil; tempo de vida}
  \end{phonetics}
\end{entry}

\begin{entry}{遇}{12}{⾡}
  \begin{phonetics}{遇}{yu4}[][HSK 4]
    \definition*{s.}{sobrenome Yu}
    \definition{s.}{chance; oportunidade}
    \definition{v.}{encontrar; deparar-se com; encontrar-se | tratar; receber}
  \end{phonetics}
\end{entry}

\begin{entry}{遇见}{12,4}{⾡、⾒}
  \begin{phonetics}{遇见}{yu4 jian4}[][HSK 4]
    \definition{v.}{encontrar; deparar-se com}
  \end{phonetics}
\end{entry}

\begin{entry}{遇到}{12,8}{⾡、⼑}
  \begin{phonetics}{遇到}{yu4dao4}[][HSK 4]
    \definition{v.}{esbarrar em; encontrar; deparar-se com; conhecer alguém ou algo (inesperado)}
  \end{phonetics}
\end{entry}

\begin{entry}{遍}{12}{⾡}
  \begin{phonetics}{遍}{bian4}[][HSK 2]
    \definition{adv.}{em todos os lugares | por toda parte}
    \definition{clas.}{para a repetição de ações de leitura, fala ou escrita}
  \end{phonetics}
\end{entry}

\begin{entry}{道}{12}{⾡}
  \begin{phonetics}{道}{dao4}[][HSK 2]
    \definition*{s.}{Taoism | Taoist}
    \definition*{s.}{sobrenome Dao}
    \definition{s.}{estrada | caminho | rota | caminho | canal | curso | maneira | método | moral | moralidade | doutrina | corpo de ensinamentos morais | o Caminho da Natureza que não pode receber um nome | princípio | seita supersticiosa | linha | trato | habilidade}
  \end{phonetics}
\end{entry}

\begin{entry}{道理}{12,11}{⾡、⽟}
  \begin{phonetics}{道理}{dao4li5}[][HSK 2]
    \definition[个]{s.}{razão | argumento | sentido | princípio | base | justificativa}
  \end{phonetics}
\end{entry}

\begin{entry}{道路}{12,13}{⾡、⾜}
  \begin{phonetics}{道路}{dao4 lu4}[][HSK 2]
    \definition{s.}{estrada | caminho | processo}
  \end{phonetics}
\end{entry}

\begin{entry}{道歉}{12,14}{⾡、⽋}
  \begin{phonetics}{道歉}{dao4qian4}
    \definition{v.+compl.}{desculpar-se | fazer um pedido de desculpas}
  \end{phonetics}
\end{entry}

\begin{entry}{道德}{12,15}{⾡、⼻}
  \begin{phonetics}{道德}{dao4de2}[][HSK 5]
    \definition{adj.}{moral; descreve uma pessoa ou comportamento que atende aos requisitos morais; mais usado em situações negativas}
    \definition{s.}{moral; ética; moralidade; regras e normas para que as pessoas vivam juntas e se comportem em comum}
  \end{phonetics}
\end{entry}

\begin{entry}{遗产}{12,6}{⾡、⼇}
  \begin{phonetics}{遗产}{yi2chan3}[][HSK 4]
    \definition[笔,份]{s.}{legado; herança; patrimônio; propriedade deixada pelo falecido | patrimônio; riqueza cultural ou riqueza material transmitida pela história}
  \end{phonetics}
\end{entry}

\begin{entry}{遗传}{12,6}{⾡、⼈}
  \begin{phonetics}{遗传}{yi2chuan2}[][HSK 4]
    \definition{v.}{herdar, descender, transmitir, passar adiante}
  \end{phonetics}
\end{entry}

\begin{entry}{遗男}{12,7}{⾡、⽥}
  \begin{phonetics}{遗男}{yi2nan2}
    \definition{s.}{órfão | filho póstumo}
  \end{phonetics}
\end{entry}

\begin{entry}{遗迹}{12,9}{⾡、⾡}
  \begin{phonetics}{遗迹}{yi2ji4}
    \definition{s.}{vestígios históricos | remanescente | vestígio}
  \end{phonetics}
\end{entry}

\begin{entry}{遗案}{12,10}{⾡、⽊}
  \begin{phonetics}{遗案}{yi2'an4}
    \definition{s.}{(lei) caso não resolvido}
  \end{phonetics}
\end{entry}

\begin{entry}{遗落}{12,12}{⾡、⾋}
  \begin{phonetics}{遗落}{yi2luo4}
    \definition{v.}{esquecer | deixar para trás (inadvertidamente) | deixar de fora | omitir}
  \end{phonetics}
\end{entry}

\begin{entry}{遗嘱}{12,15}{⾡、⼝}
  \begin{phonetics}{遗嘱}{yi2zhu3}
    \definition{s.}{testamento}
  \end{phonetics}
\end{entry}

\begin{entry}{遗骸}{12,15}{⾡、⾻}
  \begin{phonetics}{遗骸}{yi2hai2}
    \definition{v.}{restos mortais}
  \end{phonetics}
\end{entry}

\begin{entry}{遗憾}{12,16}{⾡、⼼}
  \begin{phonetics}{遗憾}{yi2han4}
    \definition{v.}{ter pena de | lamentar}
  \end{phonetics}
\end{entry}

\begin{entry}{酢}{12}{⾣}
  \begin{phonetics}{酢}{cu4}
    \variantof{醋}
  \end{phonetics}
  \begin{phonetics}{酢}{zuo4}
    \definition{v.}{brindar o anfitrião com vinho}
  \end{phonetics}
\end{entry}

\begin{entry}{量}{12}{⾥}
  \begin{phonetics}{量}{liang2}[][HSK 4]
    \definition{v.}{medir | estimar; dimensionar}
  \end{phonetics}
  \begin{phonetics}{量}{liang4}
    \definition{s.}{instrumento de medida; antigamente, o termo se referia a objetos como baldes e litros, que medem o volume | capacidade (para tolerância ou ingestão de alimentos ou bebidas); refere-se ao limite do que pode ser acomodado | quantidade; valor; volume; número}
    \definition{v.}{estimar; medir; pesar}
  \end{phonetics}
\end{entry}

\begin{entry}{铺}{12}{⾦}
  \begin{phonetics}{铺}{pu1}
    \definition{v.}{espalhar | exibir | montar}
  \end{phonetics}
  \begin{phonetics}{铺}{pu4}
    \definition{s.}{cama de tábua | lugar para dormir | loja | depósito}
  \end{phonetics}
\end{entry}

\begin{entry}{铺垫}{12,9}{⾦、⼟}
  \begin{phonetics}{铺垫}{pu1dian4}
    \definition{s.}{cobre leito | colcha | roupa de cama}
    \definition{v.}{pavimentar}
  \end{phonetics}
\end{entry}

\begin{entry}{销售}{12,11}{⾦、⼝}
  \begin{phonetics}{销售}{xiao1shou4}[][HSK 4]
    \definition{v.}{vender; comercializar}
  \end{phonetics}
\end{entry}

\begin{entry}{锅}{12}{⾦}
  \begin{phonetics}{锅}{guo1}
    \definition[口,只]{s.}{panela | frigideira | \emph{wok} | caldeirão | coisa em forma de pote}
  \end{phonetics}
\end{entry}

\begin{entry}{隔}{12}{⾩}
  \begin{phonetics}{隔}{ge2}[][HSK 4]
    \definition{adj.}{seguinte; vizinho}
    \definition{v.}{dividir; separar; bloquear; obstruir | estar a uma distância de, após ou em um intervalo de}
  \end{phonetics}
\end{entry}

\begin{entry}{隔开}{12,4}{⾩、⼶}
  \begin{phonetics}{隔开}{ge2 kai1}[][HSK 4]
    \definition{v.}{separar; manter separado; barricar; separar completamente duas pessoas (ou coisas) ou duas partes de uma coisa que estão intimamente unidas}
  \end{phonetics}
\end{entry}

\begin{entry}{隔壁}{12,16}{⾩、⼟}
  \begin{phonetics}{隔壁}{ge2bi4}[][HSK 5]
    \definition{s.}{vizinho; casas ou pessoas vizinhas | septo; distante (socialmente distante) | anteparo; partição}
  \end{phonetics}
\end{entry}

\begin{entry}{集中}{12,4}{⾫、⼁}
  \begin{phonetics}{集中}{ji2zhong1}[][HSK 3]
    \definition{adj.}{centralizado; concentrado}
    \definition{v.}{concentrar; juntar}
  \end{phonetics}
\end{entry}

\begin{entry}{集合}{12,6}{⾫、⼝}
  \begin{phonetics}{集合}{ji2he2}[][HSK 4]
    \definition{v.}{reunir-se; juntar-se | reunir, juntar, convocar}
  \end{phonetics}
\end{entry}

\begin{entry}{集团}{12,6}{⾫、⼞}
  \begin{phonetics}{集团}{ji2tuan2}
    \definition{s.}{grupo | bloco | corporação | conglomerado}
  \end{phonetics}
\end{entry}

\begin{entry}{集体}{12,7}{⾫、⼈}
  \begin{phonetics}{集体}{ji2ti3}[][HSK 3]
    \definition{s.}{coletivo; comunidade; grupo; equipe}
  \end{phonetics}
\end{entry}

\begin{entry}{韩国}{12,8}{⾱、⼞}
  \begin{phonetics}{韩国}{han2guo2}
    \definition*{s.}{Coréia do Sul}
  \end{phonetics}
\end{entry}

\begin{entry}{韩国人}{12,8,2}{⾱、⼞、⼈}
  \begin{phonetics}{韩国人}{han2guo2ren2}
    \definition{s.}{coreano | pessoa ou povo da Coréia}
  \end{phonetics}
\end{entry}

\begin{entry}{骚乱}{12,7}{⾺、⼄}
  \begin{phonetics}{骚乱}{sao1luan4}
    \definition{s.}{rebelião | perturbação | tumulto}
    \definition{v.}{criar um distúrbio}
  \end{phonetics}
\end{entry}

\begin{entry}{黍}{12}{⿉}[Kangxi 202]
  \begin{phonetics}{黍}{shu3}
    \definition{s.}{painço}
  \end{phonetics}
\end{entry}

\begin{entry}{黑}{12}{⿊}[Kangxi 203]
  \begin{phonetics}{黑}{hei1}[][HSK 2]
    \definition{adj.}{preto | escuro | ilegal | secreto | sombrio | sinistro}
    \definition{v.}{esconder (algo) | difamar | (empréstimo linguístico) (computador) hackear}
  \end{phonetics}
\end{entry}

\begin{entry}{黑色}{12,6}{⿊、⾊}
  \begin{phonetics}{黑色}{hei1 se4}[][HSK 2]
    \definition{s.}{cor preta}
  \end{phonetics}
\end{entry}

\begin{entry}{黑板}{12,8}{⿊、⽊}
  \begin{phonetics}{黑板}{hei1ban3}[][HSK 2]
    \definition[块,个]{s.}{quadro negro}
  \end{phonetics}
\end{entry}

\begin{entry}{黑客}{12,9}{⿊、⼧}
  \begin{phonetics}{黑客}{hei1ke4}
    \definition{s.}{(empréstimo linguístico) (computação) \emph{hacker}}
  \end{phonetics}
\end{entry}

\begin{entry}{黑暗}{12,13}{⿊、⽇}
  \begin{phonetics}{黑暗}{hei1 an4}[][HSK 4]
    \definition{adj.}{escuro; sombrio; sem luz | maligno; corrupto; reacionário}
  \end{phonetics}
\end{entry}

\begin{entry}{黹}{12}{⿋}[Kangxi 204]
  \begin{phonetics}{黹}{zhi3}
    \definition{v.}{costurar; bordar}
  \end{phonetics}
\end{entry}

\begin{entry}{鼎}{12}{⿍}[Kangxi 206]
  \begin{phonetics}{鼎}{ding3}
    \definition{adj.}{importante; significativo}
    \definition{adv.}{exatamente quando; exatamente o momento para}
    \definition[尊]{s.}{um antigo recipiente de cozinha com duas alças e três ou quatro pernas | pote; caldeirão}
  \end{phonetics}
\end{entry}

%%%%% EOF %%%%%

