%%%
%%% 8画
%%%

\section*{8画}\addcontentsline{toc}{section}{8画}

\begin{entry}{丧钟}{8,9}
  \begin{phonetics}{丧钟}{sang1zhong1}
    \definition{s.}{sentença de morte}
  \end{phonetics}
\end{entry}

\begin{entry}{乖乖}{8,8}
  \begin{phonetics}{乖乖}{guai1guai1}
    \definition{adj.}{bem-comportado (criança) | obediente}
  \end{phonetics}
  \begin{phonetics}{乖乖}{guai1guai5}
    \definition{expr.}{Graças a Deus! | Oh meu Deus!}
  \end{phonetics}
\end{entry}

\begin{entry}{乳房}{8,8}
  \begin{phonetics}{乳房}{ru3fang2}
    \definition{s.}{seio | mama | úbere}
  \end{phonetics}
\end{entry}

\begin{entry}{事}{8}[Radical 亅]
  \begin{phonetics}{事}{shi4}[][HSK 1]
    \definition[件,桩,回]{s.}{coisa | assunto | item | matéria | coisa de trabalho | caso}
  \end{phonetics}
\end{entry}

\begin{entry}{事儿}{8,2}
  \begin{phonetics}{事儿}{shi4r5}
    \definition[件,桩]{s.}{o emprego | negócio | afazeres | assunto que precisa ser resolvido | matéria}
  \end{phonetics}
\end{entry}

\begin{entry}{事故}{8,9}
  \begin{phonetics}{事故}{shi4gu4}
    \definition[桩,起,次]{s.}{acidente}
  \end{phonetics}
\end{entry}

\begin{entry}{事情}{8,11}
  \begin{phonetics}{事情}{shi4qing5}[][HSK 2]
    \definition[件,桩]{s.}{assunto | coisa | erro | acidente | trabalho; emprego}
  \end{phonetics}
\end{entry}

\begin{entry}{些}{8}[Radical 二]
  \begin{phonetics}{些}{xie1}
    \definition{adv.}{uns | alguns | vários}
    \definition{clas.}{que indica uma pequena quantidade ou pequeno número maior que 1}
  \end{phonetics}
\end{entry}

\begin{entry}{些许}{8,6}
  \begin{phonetics}{些许}{xie1xu3}
    \definition{num.}{um pouco}
  \end{phonetics}
\end{entry}

\begin{entry}{享受}{8,8}
  \begin{phonetics}{享受}{xiang3shou4}
    \definition[种]{s.}{prazer}
    \definition{v.}{desfrutar | viver}
  \end{phonetics}
\end{entry}

\begin{entry}{京}{8}[Radical 亠]
  \begin{phonetics}{京}{jing1}
    \definition*{s.}{Beijing, abreviação de~北京 | sobrenome Jing}
    \definition{s.}{capital de um país}
    \seeref{北京}{bei3jing1}
  \end{phonetics}
\end{entry}

\begin{entry}{京剧}{8,10}
  \begin{phonetics}{京剧}{jing1ju4}
    \definition*{s.}{Ópera de Beijing (Pequim)}
  \end{phonetics}
\end{entry}

\begin{entry}{佩服}{8,8}
  \begin{phonetics}{佩服}{pei4fu2}
    \definition{v.}{admirar}
  \end{phonetics}
\end{entry}

\begin{entry}{使用}{8,5}
  \begin{phonetics}{使用}{shi3yong4}[][HSK 2]
    \definition{v.}{usar | empregar | aplicar}
  \end{phonetics}
\end{entry}

\begin{entry}{例子}{8,3}
  \begin{phonetics}{例子}{li4 zi5}[][HSK 2]
    \definition[个]{s.}{exemplo}
  \end{phonetics}
\end{entry}

\begin{entry}{例如}{8,6}
  \begin{phonetics}{例如}{li4ru2}[][HSK 2]
    \definition{conj.}{por exemplo | como}
  \end{phonetics}
\end{entry}

\begin{entry}{依偎}{8,11}
  \begin{phonetics}{依偎}{yi1wei1}
    \definition{v.}{aninhar-se | aconchegar-se}
  \end{phonetics}
\end{entry}

\begin{entry}{依然}{8,12}
  \begin{phonetics}{依然}{yi1ran2}
    \definition{adv.}{como era antes | ainda}
  \end{phonetics}
\end{entry}

\begin{entry}{兔子}{8,3}
  \begin{phonetics}{兔子}{tu4zi5}
    \definition[只]{s.}{coelho | lebre}
  \end{phonetics}
\end{entry}

\begin{entry}{其中}{8,4}
  \begin{phonetics}{其中}{qi2zhong1}[][HSK 2]
    \definition{pron.}{dentro | entre (o qual, eles, etc.) | em (o qual, isso, etc.)}
  \end{phonetics}
\end{entry}

\begin{entry}{其他}{8,5}
  \begin{phonetics}{其他}{qi2ta1}[][HSK 2]
    \definition{pron.}{todos os outro(s) | o resto}
  \end{phonetics}
\end{entry}

\begin{entry}{其实}{8,8}
  \begin{phonetics}{其实}{qi2shi2}
    \definition{adv.}{na verdade | de fato}
  \end{phonetics}
\end{entry}

\begin{entry}{函数}{8,13}
  \begin{phonetics}{函数}{han2shu4}
    \definition{s.}{função (matemática)}
  \end{phonetics}
\end{entry}

\begin{entry}{刮}{8}[Radical 刀]
  \begin{phonetics}{刮}{gua1}
    \definition{v.}{ventar | soprar (vento)}
  \end{phonetics}
\end{entry}

\begin{entry}{刮风}{8,4}
  \begin{phonetics}{刮风}{gua1feng1}
    \definition{v.+compl.}{ventar | fazer vento}
  \end{phonetics}
\end{entry}

\begin{entry}{到}{8}[Radical 刀]
  \begin{phonetics}{到}{dao4}[][HSK 1]
    \definition{prep.}{a | até | para}
    \definition{v.}{chegar}
  \end{phonetics}
\end{entry}

\begin{entry}{到处}{8,5}
  \begin{phonetics}{到处}{dao4chu4}[][HSK 2]
    \definition{adv.}{em todos os lugares}
  \end{phonetics}
\end{entry}

\begin{entry}{到底}{8,8}
  \begin{phonetics}{到底}{dao4di3}
    \definition{adv.}{na verdade | exatamente | são ou não são | afinal | no final | no final das contas | finalmente | quando tudo estiver dito e feito}
  \end{phonetics}
\end{entry}

\begin{entry}{制裁}{8,12}
  \begin{phonetics}{制裁}{zhi4cai2}
    \definition{s.}{punição | sanção (inclusive econômica)}
    \definition{v.}{punir}
  \end{phonetics}
\end{entry}

\begin{entry}{刷子}{8,3}
  \begin{phonetics}{刷子}{shua1zi5}
    \definition[把]{s.}{pincel | escova | escovão}
  \end{phonetics}
\end{entry}

\begin{entry}{刹}{8}[Radical 刀]
  \begin{phonetics}{刹}{cha4}
    \definition{s.}{mosteiro, templo ou santuário budista | abreviação de 刹多罗 | sânscrito "ksetra"}
    \seeref{刹多罗}{cha4duo1luo2}
  \end{phonetics}
  \begin{phonetics}{刹}{sha1}
    \definition{v.}{frear}
  \end{phonetics}
\end{entry}

\begin{entry}{刹多罗}{8,6,8}
  \begin{phonetics}{刹多罗}{cha4duo1luo2}
    \definition*{s.}{Kshatara, sânscrito ``ksetra''}
  \end{phonetics}
\end{entry}

\begin{entry}{刺}{8}[Radical 刀]
  \begin{phonetics}{刺}{ci4}
    \definition{s.}{espinho | picada}
    \definition{v.}{picar | perfurar | esfaquear | assassinar}
  \end{phonetics}
\end{entry}

\begin{entry}{刺猬}{8,12}
  \begin{phonetics}{刺猬}{ci4wei5}
    \definition{s.}{porco-espinho | ouriço}
  \end{phonetics}
\end{entry}

\begin{entry}{刻}{8}[Radical 刀]
  \begin{phonetics}{刻}{ke4}[][HSK 2]
    \definition{clas.}{para curtos intervalos de tempo}
    \definition{s.}{quarto (de hora)}
    \definition{v.}{esculpir | cortar | gravar}
  \end{phonetics}
\end{entry}

\begin{entry}{刻画}{8,8}
  \begin{phonetics}{刻画}{ke4hua4}
    \definition{v.}{retratar | tirar um retrato}
  \end{phonetics}
\end{entry}

\begin{entry}{刻钟}{8,9}
  \begin{phonetics}{刻钟}{ke4 zhong1}
    \definition{s.}{um quarto de hora}
  \end{phonetics}
\end{entry}

\begin{entry}{单}{8}[Radical 十]
  \begin{phonetics}{单}{chan2}
    \definition{s.}{usado em 单于 \dpy{chan2yu2}}
    \seeref{单于}{chan2yu2}
  \end{phonetics}
  \begin{phonetics}{单}{dan1}
    \definition{adj.}{solteiro | único}
    \definition{adv.}{apenas}
    \definition[个]{s.}{conta | lista | formulário | número ímpar}
  \end{phonetics}
  \begin{phonetics}{单}{shan4}
    \definition*{s.}{sobrenome Shan}
  \end{phonetics}
\end{entry}

\begin{entry}{单于}{8,3}
  \begin{phonetics}{单于}{chan2yu2}
    \definition{s.}{rei de Xiongnu (匈奴)}
  \seealsoref{匈奴}{xiong1nu2}
  \end{phonetics}
\end{entry}

\begin{entry}{单位}{8,7}
  \begin{phonetics}{单位}{dan1wei4}[][HSK 2]
    \definition[个]{s.}{unidade (como padrão de medida) | unidade (como uma organização, departamento, divisão, seção, etc.)}
  \end{phonetics}
\end{entry}

\begin{entry}{单质}{8,8}
  \begin{phonetics}{单质}{dan1zhi4}
    \definition{s.}{substância simples (consistindo puramente de um elemento, como diamante, grafite, etc.)}
  \end{phonetics}
\end{entry}

\begin{entry}{单调}{8,10}
  \begin{phonetics}{单调}{dan1diao4}
    \definition{adj.}{monótono}
  \end{phonetics}
\end{entry}

\begin{entry}{单脚滑行车}{8,11,12,6,4}
  \begin{phonetics}{单脚滑行车}{dan1jiao3hua2xing2che1}
    \definition{s.}{\emph{scooter}}
  \end{phonetics}
\end{entry}

\begin{entry}{卖}{8}[Radical 十]
  \begin{phonetics}{卖}{mai4}[][HSK 2]
    \definition{v.}{vender}
  \end{phonetics}
\end{entry}

\begin{entry}{卧}{8}[Radical 卜]
  \begin{phonetics}{卧}{wo4}
    \definition{v.}{agachar | deitar}
  \end{phonetics}
\end{entry}

\begin{entry}{卧车}{8,4}
  \begin{phonetics}{卧车}{wo4che1}
    \definition{s.}{um carro-leito | vagão-leito}
  \end{phonetics}
\end{entry}

\begin{entry}{卧式}{8,6}
  \begin{phonetics}{卧式}{wo4shi4}
    \definition{adj.}{horizontal}
  \end{phonetics}
\end{entry}

\begin{entry}{卧床}{8,7}
  \begin{phonetics}{卧床}{wo4chuang2}
    \definition{adj.}{acamado}
    \definition{s.}{cama}
    \definition{v.}{deitar na cama}
  \end{phonetics}
\end{entry}

\begin{entry}{卧室}{8,9}
  \begin{phonetics}{卧室}{wo4shi4}
    \definition[间]{s.}{quarto de dormir}
  \end{phonetics}
\end{entry}

\begin{entry}{卧倒}{8,10}
  \begin{phonetics}{卧倒}{wo4dao3}
    \definition{v.}{cair no chão | deitar-se}
  \end{phonetics}
\end{entry}

\begin{entry}{卧病}{8,10}
  \begin{phonetics}{卧病}{wo4bing4}
    \definition{s.}{acamado | doente na cama}
  \end{phonetics}
\end{entry}

\begin{entry}{卧舱}{8,10}
  \begin{phonetics}{卧舱}{wo4cang1}
    \definition{s.}{cabine de dormir em um barco ou trem}
  \end{phonetics}
\end{entry}

\begin{entry}{卧推}{8,11}
  \begin{phonetics}{卧推}{wo4tui1}
    \definition{s.}{supino}
  \end{phonetics}
\end{entry}

\begin{entry}{卧榻}{8,14}
  \begin{phonetics}{卧榻}{wo4ta4}
    \definition{s.}{um sofá | uma cama estreita}
  \end{phonetics}
\end{entry}

\begin{entry}{卷}{8}[Radical 卩]
  \begin{phonetics}{卷}{juan3}
    \definition{clas.}{para pequenas coisas enroladas (maço de papel dinheiro, carretel de filme, etc.) | para rolos, carretéis, bobinas, etc.}
    \definition{s.}{rolo | carretel | bobina}
    \definition{v.}{rolar | varrer | carregar}
  \end{phonetics}
  \begin{phonetics}{卷}{juan4}
    \definition{clas.}{para livros, pinturas: volumes, rolos}
    \definition{s.}{rolo com texto | livro | volume | capítulo | artigo}
  \end{phonetics}
\end{entry}

\begin{entry}{厕纸}{8,7}
  \begin{phonetics}{厕纸}{ce4zhi3}
    \definition{s.}{papel higiênico}
  \end{phonetics}
\end{entry}

\begin{entry}{厕所}{8,8}
  \begin{phonetics}{厕所}{ce4suo3}
    \definition[间,处]{s.}{lavatório | \emph{toilette}}
  \end{phonetics}
\end{entry}

\begin{entry}{参加}{8,5}
  \begin{phonetics}{参加}{can1jia1}[][HSK 2]
    \definition{v.}{participar de | tomar parte em | assistir}
  \end{phonetics}
\end{entry}

\begin{entry}{参观}{8,6}
  \begin{phonetics}{参观}{can1guan1}[][HSK 2]
    \definition{v.}{visitar}
  \end{phonetics}
\end{entry}

\begin{entry}{取}{8}[Radical 又]
  \begin{phonetics}{取}{qu3}[][HSK 2]
    \definition{v.}{buscar | obter | escolher}
  \end{phonetics}
\end{entry}

\begin{entry}{取水}{8,4}
  \begin{phonetics}{取水}{qu3shui3}
    \definition{v.}{obter água (de um poço, etc.)}
  \end{phonetics}
\end{entry}

\begin{entry}{取现}{8,8}
  \begin{phonetics}{取现}{qu3xian4}
    \definition{v.}{sacar dinheiro}
  \end{phonetics}
\end{entry}

\begin{entry}{取胜}{8,9}
  \begin{phonetics}{取胜}{qu3sheng4}
    \definition{v.}{prevalecer sobre os oponentes | marcar uma vitória}
  \end{phonetics}
\end{entry}

\begin{entry}{取悦}{8,10}
  \begin{phonetics}{取悦}{qu3yue4}
    \definition{v.}{tentar agradar}
  \end{phonetics}
\end{entry}

\begin{entry}{取得}{8,11}
  \begin{phonetics}{取得}{qu3 de2}[][HSK 2]
    \definition{v.}{ganhar | adquirir | obter}
  \end{phonetics}
\end{entry}

\begin{entry}{受到}{8,8}
  \begin{phonetics}{受到}{shou4dao4}[][HSK 2]
    \definition{v.}{receber (elogio, educação, punição, etc.) | ser elogiado, educado, punido, etc.}
  \end{phonetics}
\end{entry}

\begin{entry}{受限}{8,8}
  \begin{phonetics}{受限}{shou4xian4}
    \definition{v.}{ser limitado | ser restrito | ser constrangido}
  \end{phonetics}
\end{entry}

\begin{entry}{受得了}{8,11,2}
  \begin{phonetics}{受得了}{shou4de5liao3}
    \definition{v.}{suportar | aguentar}
  \end{phonetics}
\end{entry}

\begin{entry}{变}{8}[Radical 又]
  \begin{phonetics}{变}{bian4}[][HSK 2]
    \definition{v.}{mudar | transformar | variar}
  \end{phonetics}
\end{entry}

\begin{entry}{变化}{8,4}
  \begin{phonetics}{变化}{bian4hua4}
    \definition[个]{s.}{mudança | variação}
    \definition{v.}{(intransitivo) mudar, variar}
  \end{phonetics}
\end{entry}

\begin{entry}{变心}{8,4}
  \begin{phonetics}{变心}{bian4xin1}
    \definition{v.+compl.}{deixar de ser fiel}
  \end{phonetics}
\end{entry}

\begin{entry}{变节}{8,5}
  \begin{phonetics}{变节}{bian4jie2}
    \definition{s.}{traição | deserção | vira-casaca}
    \definition{v.}{mudar de lado politicamente}
  \end{phonetics}
\end{entry}

\begin{entry}{变异}{8,6}
  \begin{phonetics}{变异}{bian4yi4}
    \definition{s.}{variação | mutação}
  \end{phonetics}
\end{entry}

\begin{entry}{变成}{8,6}
  \begin{phonetics}{变成}{bian4 cheng2}[][HSK 2]
    \definition{v.}{mudar | transformar-se em | tornar-se}
  \end{phonetics}
\end{entry}

\begin{entry}{变迁}{8,6}
  \begin{phonetics}{变迁}{bian4qian1}
    \definition{s.}{mudanças | vicissitudes}
  \end{phonetics}
\end{entry}

\begin{entry}{变更}{8,7}
  \begin{phonetics}{变更}{bian4geng1}
    \definition{v.}{alterar | mudar | modificar}
  \end{phonetics}
\end{entry}

\begin{entry}{变性}{8,8}
  \begin{phonetics}{变性}{bian4xing4}
    \definition{s.}{desnaturação | transexual}
    \definition{v.}{desnaturar | mudar de sexo}
  \end{phonetics}
\end{entry}

\begin{entry}{变装}{8,12}
  \begin{phonetics}{变装}{bian4zhuang1}
    \definition{v.}{trocar de roupa | vestir-se | vestir uma fantasia | disfarçar-se ou fantasiar-se de personagem real ou ficcional, \emph{cosplay} | travestir-se}
  \end{phonetics}
\end{entry}

\begin{entry}{变数}{8,13}
  \begin{phonetics}{变数}{bian4shu4}
    \definition{s.}{(matemática) variável}
  \end{phonetics}
\end{entry}

\begin{entry}{呢}{8}[Radical 口]
  \begin{phonetics}{呢}{ne5}
    \definition{part.}{(no final de uma frase declarativa) partícula que indica a continuação de um estado ou ação |  partícula para perguntar sobre a localização (``Onde está\dots?'') | partícula indicando  afirmação forte | partícula indicando que uma pergunta feita anteriormente deve ser aplicada à palavra anterior (``E quanto a\dots?'', ``E\dots?'') | partícula sinalizando uma pausa, para enfatizar as palavras anteriores e permitir que o ouvinte tenha tempo para compreendê-las (``ok?'', ``você está comigo ?'')}
  \end{phonetics}
  \begin{phonetics}{呢}{ni2}
    \definition{s.}{material de lã}
  \end{phonetics}
\end{entry}

\begin{entry}{周}{8}[Radical 口]
  \begin{phonetics}{周}{zhou1}[][HSK 2]
    \definition*{s.}{sobrenome Zhou | Dinastia Zhou (1046-256 BC)}
    \definition{adv.}{semanalmente}
    \definition{s.}{círculo | circunferência | ciclo | uma volta (em um circuito) | semana}
    \definition{v.}{fazer um circuito |circular | ajudar financeiramente}
  \end{phonetics}
\end{entry}

\begin{entry}{周末}{8,5}
  \begin{phonetics}{周末}{zhou1mo4}[][HSK 2]
    \definition{s.}{final-de-semana}
  \end{phonetics}
\end{entry}

\begin{entry}{周年}{8,6}
  \begin{phonetics}{周年}{zhou1nian2}[][HSK 2]
    \definition{s.}{aniversário}
  \end{phonetics}
\end{entry}

\begin{entry}{味}{8}[Radical 口]
  \begin{phonetics}{味}{wei4}
    \definition{clas.}{para medicamentos}
    \definition{s.}{cheiro | gosto}
  \end{phonetics}
\end{entry}

\begin{entry}{味儿}{8,2}
  \begin{phonetics}{味儿}{wei4r5}
    \definition{s.}{sabor}
  \end{phonetics}
\end{entry}

\begin{entry}{味道}{8,12}
  \begin{phonetics}{味道}{wei4dao5}[][HSK 2]
    \definition{s.}{sabor | (dialeto) odor, cheiro | (figurativo) sentimento (de…), dica (de…) | (figurativo) interesse, prazer}
  \end{phonetics}
\end{entry}

\begin{entry}{呵}{8}[Radical 口]
  \begin{phonetics}{呵}{a1}
    \variantof{啊}
  \end{phonetics}
  \begin{phonetics}{呵}{he1}
    \definition{expr.}{Meu Deus! | expelir a respiração}
  \end{phonetics}
\end{entry}

\begin{entry}{呼吸}{8,6}
  \begin{phonetics}{呼吸}{hu1xi1}
    \definition{v.}{respirar}
  \end{phonetics}
\end{entry}

\begin{entry}{呼啸}{8,11}
  \begin{phonetics}{呼啸}{hu1xiao4}
    \definition{v.}{assobiar}
  \end{phonetics}
\end{entry}

\begin{entry}{命运}{8,7}
  \begin{phonetics}{命运}{ming4yun4}
    \definition[个]{s.}{destino}
  \end{phonetics}
\end{entry}

\begin{entry}{和}{8}[Radical 口]
  \begin{phonetics}{和}{he2}
    \definition*{s.}{sobrenome He}
    \definition{conj.}{e (somente para palavras) | junto com}
    \definition{s.}{união | paz | japonês (comida, roupa, etc.) | harmonia}
  \end{phonetics}
  \begin{phonetics}{和}{he4}
    \definition{v.}{compor um poema em resposta (ao poema de alguém) usando a mesma sequência de rimas | juntar-se à cantoria | cantar junto com outros}
  \end{phonetics}
  \begin{phonetics}{和}{hu2}
    \definition{v.}{completar um conjunto de Mahjong ou cartas de baralho}
  \end{phonetics}
  \begin{phonetics}{和}{huo2}
    \definition{v.}{combinar uma substância em pó (farinha, gesso, etc.) com água}
  \end{phonetics}
  \begin{phonetics}{和}{huo4}
    \definition{clas.}{para enxágues de roupas | para fervuras de ervas medicinais}
    \definition{v.}{misturar (ingredientes) | misturar}
  \end{phonetics}
\end{entry}

\begin{entry}{和平}{8,5}
  \begin{phonetics}{和平}{he2ping2}
    \definition{adj.}{pacífico}
    \definition{s.}{paz}
  \end{phonetics}
\end{entry}

\begin{entry}{和平共处}{8,5,6,5}
  \begin{phonetics}{和平共处}{he2ping2gong4chu3}
    \definition{s.}{coexistência pacífica de nações, sociedades, etc.}
  \end{phonetics}
\end{entry}

\begin{entry}{和谐}{8,11}
  \begin{phonetics}{和谐}{he2xie2}
    \definition{adj.}{harmonioso}
    \definition{s.}{harmonia}
    \definition{v.}{(eufemismo) censurar}
  \end{phonetics}
\end{entry}

\begin{entry}{咒骂}{8,9}
  \begin{phonetics}{咒骂}{zhou4ma4}
    \definition{v.}{xingar | amaldiçoar | execrar}
  \end{phonetics}
\end{entry}

\begin{entry}{咖啡}{8,11}
  \begin{phonetics}{咖啡}{ka1fei1}
    \definition[杯]{s.}{(empréstimo linguístico) café}
  \end{phonetics}
\end{entry}

\begin{entry}{咖啡色}{8,11,6}
  \begin{phonetics}{咖啡色}{ka1fei1se4}
    \definition{s.}{cor café}
  \end{phonetics}
\end{entry}

\begin{entry}{咖啡馆}{8,11,11}
  \begin{phonetics}{咖啡馆}{ka1fei1guan3}
    \definition[家]{s.}{cafeteria}
  \end{phonetics}
\end{entry}

\begin{entry}{国}{8}[Radical ⼞]
  \begin{phonetics}{国}{guo2}[][HSK 1]
    \definition*{s.}{sobrenome Guo}
    \definition[个]{s.}{país | nação}
  \end{phonetics}
\end{entry}

\begin{entry}{国人}{8,2}
  \begin{phonetics}{国人}{guo2ren2}
    \definition{s.}{compatriota}
  \end{phonetics}
\end{entry}

\begin{entry}{国内}{8,4}
  \begin{phonetics}{国内}{guo2nei4}
    \definition{adj.}{doméstico | interno (a um país) | civil}
  \end{phonetics}
\end{entry}

\begin{entry}{国外}{8,5}
  \begin{phonetics}{国外}{guo2 wai4}[][HSK 1]
    \definition{adj.}{no exterior | externo (assuntos) | estrangeiro}
  \end{phonetics}
\end{entry}

\begin{entry}{国庆节}{8,6,5}
  \begin{phonetics}{国庆节}{guo2qing4jie2}
    \definition*{s.}{Dia Nacional (1~de~outubro)}
  \end{phonetics}
\end{entry}

\begin{entry}{国际}{8,7}
  \begin{phonetics}{国际}{guo2ji4}[][HSK 2]
    \definition{adj.}{internacional}
  \end{phonetics}
\end{entry}

\begin{entry}{国际儿童节}{8,7,2,12,5}
  \begin{phonetics}{国际儿童节}{guo2ji4 er2tong2jie2}
    \definition*{s.}{Dia Internacional das Crianças (1~de~junho)}
  \end{phonetics}
\end{entry}

\begin{entry}{国际妇女节}{8,7,6,3,5}
  \begin{phonetics}{国际妇女节}{guo2ji4 fu4nv3jie2}
    \definition*{s.}{Dia Internacional das Mulheres (8~de~março)}
  \end{phonetics}
\end{entry}

\begin{entry}{国际劳动节}{8,7,7,6,5}
  \begin{phonetics}{国际劳动节}{guo2ji4 lao2dong4 jie2}
    \definition*{s.}{Dia Internacional dos Trabalhadores (1~de~maio)}
  \end{phonetics}
\end{entry}

\begin{entry}{国语}{8,9}
  \begin{phonetics}{国语}{guo2yu3}
    \definition*{s.}{Língua Chinesa (Mandarim), enfatizando sua natureza nacional}
  \end{phonetics}
\end{entry}

\begin{entry}{国家}{8,10}
  \begin{phonetics}{国家}{guo2jia1}[][HSK 1]
    \definition[个]{s.}{país | nação | estado}
  \end{phonetics}
\end{entry}

\begin{entry}{国宾馆}{8,10,11}
  \begin{phonetics}{国宾馆}{guo2bin1guan3}
    \definition{s.}{pousada estadual}
  \end{phonetics}
\end{entry}

\begin{entry}{国旗}{8,14}
  \begin{phonetics}{国旗}{guo2qi2}
    \definition[面]{s.}{bandeira (de um país)}
  \end{phonetics}
\end{entry}

\begin{entry}{国歌}{8,14}
  \begin{phonetics}{国歌}{guo2ge1}
    \definition{s.}{hino nacional}
  \end{phonetics}
\end{entry}

\begin{entry}{图}{8}[Radical 囗]
  \begin{phonetics}{图}{tu2}
    \definition[张]{s.}{diagrama | imagem | desenho | gráfico | mapa}
    \definition{v.}{planejar | esquematizar | tentar | perseguir | procurar}
  \end{phonetics}
\end{entry}

\begin{entry}{图书馆}{8,4,11}
  \begin{phonetics}{图书馆}{tu2shu1guan3}[][HSK 1]
    \definition[家,个]{s.}{biblioteca}
  \end{phonetics}
\end{entry}

\begin{entry}{图片}{8,4}
  \begin{phonetics}{图片}{tu2 pian4}[][HSK 2]
    \definition[张,幅]{s.}{imagem | fotografia}
  \end{phonetics}
\end{entry}

\begin{entry}{坦克}{8,7}
  \begin{phonetics}{坦克}{tan3ke4}
    \definition{s.}{(empréstimo linguístico) tanque (veículo militar)}
  \end{phonetics}
\end{entry}

\begin{entry}{垃圾}{8,6}
  \begin{phonetics}{垃圾}{la1ji1}
    \definition[把]{s.}{lixo}
  \end{phonetics}
\end{entry}

\begin{entry}{垃圾工}{8,6,3}
  \begin{phonetics}{垃圾工}{la1ji1gong1}
    \definition{s.}{lixeiro | gari}
  \end{phonetics}
\end{entry}

\begin{entry}{垃圾车}{8,6,4}
  \begin{phonetics}{垃圾车}{la1ji1che1}
    \definition{s.}{caminhão de lixo}
  \end{phonetics}
\end{entry}

\begin{entry}{垃圾电邮}{8,6,5,7}
  \begin{phonetics}{垃圾电邮}{la1ji1dian4you2}
    \definition{s.}{\emph{e-mail} de \emph{spam}}
  \end{phonetics}
\end{entry}

\begin{entry}{垃圾邮件}{8,6,7,6}
  \begin{phonetics}{垃圾邮件}{la1ji1you2jian4}
    \definition{s.}{\emph{spam}, \emph{e-mail} não solicitado}
  \end{phonetics}
\end{entry}

\begin{entry}{垃圾食品}{8,6,9,9}
  \begin{phonetics}{垃圾食品}{la1ji1shi2pin3}
    \definition{s.}{\emph{junk food}}
  \end{phonetics}
\end{entry}

\begin{entry}{垃圾堆}{8,6,11}
  \begin{phonetics}{垃圾堆}{la1ji1dui1}
    \definition{s.}{depósito de lixo}
  \end{phonetics}
\end{entry}

\begin{entry}{垃圾筒}{8,6,12}
  \begin{phonetics}{垃圾筒}{la1ji1tong3}
    \definition{s.}{cesto de lixo}
  \end{phonetics}
\end{entry}

\begin{entry}{垃圾箱}{8,6,15}
  \begin{phonetics}{垃圾箱}{la1ji1xiang1}
    \definition{s.}{cesto de lixo}
  \end{phonetics}
\end{entry}

\begin{entry}{备份}{8,6}
  \begin{phonetics}{备份}{bei4fen4}
    \definition{s.}{cópia de segurança | \emph{backup}}
  \end{phonetics}
\end{entry}

\begin{entry}{备胎}{8,9}
  \begin{phonetics}{备胎}{bei4tai1}
    \definition{s.}{pneu sobressalente | (gíria) substituto}
  \end{phonetics}
\end{entry}

\begin{entry}{夜}{8}[Radical 夕]
  \begin{phonetics}{夜}{ye4}[][HSK 2]
    \definition{s.}{noite}
  \end{phonetics}
\end{entry}

\begin{entry}{夜生活}{8,5,9}
  \begin{phonetics}{夜生活}{ye4sheng1huo2}
    \definition{s.}{vida noturna}
  \end{phonetics}
\end{entry}

\begin{entry}{夜鸟}{8,5}
  \begin{phonetics}{夜鸟}{ye4niao3}
    \definition{s.}{ave noturna}
  \end{phonetics}
\end{entry}

\begin{entry}{夜场}{8,6}
  \begin{phonetics}{夜场}{ye4chang3}
    \definition{s.}{show noturno (em um teatro, etc.) | local de entretenimento noturno (bar, boate, discoteca, etc.)}
  \end{phonetics}
\end{entry}

\begin{entry}{夜里}{8,7}
  \begin{phonetics}{夜里}{ye4li5}[][HSK 2]
    \definition{adv.}{à noite | durante a noite | período noturno}
  \end{phonetics}
\end{entry}

\begin{entry}{夜夜}{8,8}
  \begin{phonetics}{夜夜}{ye4ye4}
    \definition{adv.}{toda noite}
  \end{phonetics}
\end{entry}

\begin{entry}{夜店}{8,8}
  \begin{phonetics}{夜店}{ye4dian4}
    \definition{s.}{boate | \emph{nightclub}}
  \end{phonetics}
\end{entry}

\begin{entry}{夜晚}{8,11}
  \begin{phonetics}{夜晚}{ye4wan3}
    \definition[个]{s.}{noite}
  \end{phonetics}
\end{entry}

\begin{entry}{夜深人静}{8,11,2,14}
  \begin{phonetics}{夜深人静}{ye4shen1ren2jing4}
    \definition{expr.}{"Na calada da noite."}
  \end{phonetics}
\end{entry}

\begin{entry}{夜幕}{8,13}
  \begin{phonetics}{夜幕}{ye4mu4}
    \definition{s.}{cortina da noite}
  \end{phonetics}
\end{entry}

\begin{entry}{奇怪}{8,8}
  \begin{phonetics}{奇怪}{qi2guai4}
    \definition{adj.}{estranho}
    \definition{v.}{ficar perplexo | maravilhar-se}
  \end{phonetics}
\end{entry}

\begin{entry}{奇迹}{8,9}
  \begin{phonetics}{奇迹}{qi2ji4}
    \definition{adj.}{milagroso}
    \definition{s.}{milagre}
  \end{phonetics}
\end{entry}

\begin{entry}{奋战}{8,9}
  \begin{phonetics}{奋战}{fen4zhan4}
    \definition{v.}{lutar bravamente | trabalhar duro}
  \end{phonetics}
\end{entry}

\begin{entry}{妹}{8}[Radical 女]
  \begin{phonetics}{妹}{mei4}[][HSK 1]
    \definition[个]{s.}{irmã mais nova}
    \seeref{妹妹}{mei4mei5}
  \end{phonetics}
\end{entry}

\begin{entry}{妹夫}{8,4}
  \begin{phonetics}{妹夫}{mei4fu5}
    \definition{s.}{marido da irmã mais nova}
  \end{phonetics}
\end{entry}

\begin{entry}{妹妹}{8,8}
  \begin{phonetics}{妹妹}{mei4mei5}[][HSK 1]
    \definition[个]{s.}{irmã mais nova | mulher jovem}
  \end{phonetics}
\end{entry}

\begin{entry}{姐}{8}[Radical 女]
  \begin{phonetics}{姐}{jie3}[][HSK 1]
    \definition{s.}{irmã mais velha | um termo geral para mulheres jovens}
    \seeref{姐姐}{jie3jie5}
  \end{phonetics}
\end{entry}

\begin{entry}{姐夫}{8,4}
  \begin{phonetics}{姐夫}{jie3fu5}
    \definition{s.}{marido da irmã mais velha}
  \end{phonetics}
\end{entry}

\begin{entry}{姐姐}{8,8}
  \begin{phonetics}{姐姐}{jie3jie5}[][HSK 1]
    \definition[个]{s.}{irmã mais velha}
  \end{phonetics}
\end{entry}

\begin{entry}{姑且}{8,5}
  \begin{phonetics}{姑且}{gu1qie3}
    \definition{adv.}{provisoriamente | por enquanto}
  \end{phonetics}
\end{entry}

\begin{entry}{姑娘}{8,10}
  \begin{phonetics}{姑娘}{gu1niang5}
    \definition[个]{s.}{menina | mulher jovem | senhorita jovem | filha}
  \end{phonetics}
\end{entry}

\begin{entry}{姓}{8}[Radical 女]
  \begin{phonetics}{姓}{xing4}[][HSK 2]
    \definition[个]{s.}{sobrenome}
    \definition{v.}{ter o sobrenome}
  \end{phonetics}
\end{entry}

\begin{entry}{姓氏}{8,4}
  \begin{phonetics}{姓氏}{xing4shi4}
    \definition{s.}{sobrenome}
  \end{phonetics}
\end{entry}

\begin{entry}{姓名}{8,6}
  \begin{phonetics}{姓名}{xing4ming2}[][HSK 2]
    \definition{s.}{nome completo}
  \end{phonetics}
\end{entry}

\begin{entry}{委内瑞拉}{8,4,13,8}
  \begin{phonetics}{委内瑞拉}{wei3nei4rui4la1}
    \definition*{s.}{Venezuela}
  \end{phonetics}
\end{entry}

\begin{entry}{季节}{8,5}
  \begin{phonetics}{季节}{ji4jie2}
    \definition[个]{s.}{estação (clima)}
  \end{phonetics}
\end{entry}

\begin{entry}{孤独}{8,9}
  \begin{phonetics}{孤独}{gu1du2}
    \definition{adj.}{solitário}
  \end{phonetics}
\end{entry}

\begin{entry}{学}{8}[Radical 子]
  \begin{phonetics}{学}{xue2}[][HSK 1]
    \definition{v.}{aprender | estudar}
  \end{phonetics}
\end{entry}

\begin{entry}{学习}{8,3}
  \begin{phonetics}{学习}{xue2xi2}[][HSK 1]
    \definition{v.}{estudar | aprender}
  \end{phonetics}
\end{entry}

\begin{entry}{学分}{8,4}
  \begin{phonetics}{学分}{xue2fen1}
    \definition{s.}{créditos de um curso}
  \end{phonetics}
\end{entry}

\begin{entry}{学术}{8,5}
  \begin{phonetics}{学术}{xue2shu4}
    \definition[个]{s.}{aprendizagem | ciência}
  \end{phonetics}
\end{entry}

\begin{entry}{学生}{8,5}
  \begin{phonetics}{学生}{xue2sheng5}[][HSK 1]
    \definition{s.}{estudante | aluno}
  \end{phonetics}
\end{entry}

\begin{entry}{学生证}{8,5,7}
  \begin{phonetics}{学生证}{xue2sheng5zheng4}
    \definition{s.}{cartão de identidade de estudante}
  \end{phonetics}
\end{entry}

\begin{entry}{学会}{8,6}
  \begin{phonetics}{学会}{xue2hui4}
    \definition{s.}{instituto | associação (acadêmica) | sociedade científica, douta ou erudita}
    \definition{v.}{aprender | dominar (um assunto)}
  \end{phonetics}
\end{entry}

\begin{entry}{学好}{8,6}
  \begin{phonetics}{学好}{xue2hao3}
    \definition{v.}{seguir bons exemplos | aprender bem}
  \end{phonetics}
\end{entry}

\begin{entry}{学问}{8,6}
  \begin{phonetics}{学问}{xue2wen4}
    \definition[个]{s.}{conhecimento | aprendizagem}
  \end{phonetics}
\end{entry}

\begin{entry}{学费}{8,9}
  \begin{phonetics}{学费}{xue2fei4}
    \definition[个]{s.}{mensalidade}
  \end{phonetics}
\end{entry}

\begin{entry}{学院}{8,9}
  \begin{phonetics}{学院}{xue2yuan4}[][HSK 1]
    \definition[所]{s.}{instituto}
  \end{phonetics}
\end{entry}

\begin{entry}{学校}{8,10}
  \begin{phonetics}{学校}{xue2xiao4}[][HSK 1]
    \definition{s.}{escola | instituição de ensino}
  \end{phonetics}
\end{entry}

\begin{entry}{学期}{8,12}
  \begin{phonetics}{学期}{xue2qi1}[][HSK 2]
    \definition[个]{s.}{semestre}
  \end{phonetics}
\end{entry}

\begin{entry}{官桂}{8,10}
  \begin{phonetics}{官桂}{guan1gui4}
    \definition{s.}{canela}
  \seealsoref{肉桂}{rou4gui4}
  \end{phonetics}
\end{entry}

\begin{entry}{宝石}{8,5}
  \begin{phonetics}{宝石}{bao3shi2}
    \definition[枚,颗]{s.}{pedra preciosa | gema}
  \end{phonetics}
\end{entry}

\begin{entry}{宝宝}{8,8}
  \begin{phonetics}{宝宝}{bao3bao5}
    \definition{s.}{querida | \emph{darling} | \emph{baby}}
  \end{phonetics}
\end{entry}

\begin{entry}{实力}{8,2}
  \begin{phonetics}{实力}{shi2li4}
    \definition{s.}{força}
  \end{phonetics}
\end{entry}

\begin{entry}{实习}{8,3}
  \begin{phonetics}{实习}{shi2xi2}[][HSK 2]
    \definition{s.}{estagiário | estágio}
  \end{phonetics}
\end{entry}

\begin{entry}{实在}{8,6}
  \begin{phonetics}{实在}{shi2zai4}[][HSK 2]
    \definition{adv.}{realmente | verdadeiramente | de fato | na verdade}
  \end{phonetics}
\end{entry}

\begin{entry}{实际}{8,7}
  \begin{phonetics}{实际}{shi2ji4}[][HSK 2]
    \definition{adj.}{real | atual | concreto | prático | factual | realista}
    \definition{s.}{realidade | prática}
  \end{phonetics}
\end{entry}

\begin{entry}{实现}{8,8}
  \begin{phonetics}{实现}{shi2xian4}[][HSK 2]
    \definition{v.}{alcançar | implementar | constatar}
  \end{phonetics}
\end{entry}

\begin{entry}{宠物}{8,8}
  \begin{phonetics}{宠物}{chong3wu4}
    \definition{s.}{animal de estimação}
  \end{phonetics}
\end{entry}

\begin{entry}{尚且}{8,5}
  \begin{phonetics}{尚且}{shang4qie3}
    \definition{conj.}{até | ainda}
  \end{phonetics}
\end{entry}

\begin{entry}{尚且……何况……}{8,5,7,7}
  \begin{phonetics}{尚且……何况……}{shang4qie3 he2kuang4}
    \definition{conj.}{ainda que\dots, \dots}
  \end{phonetics}
\end{entry}

\begin{entry}{居然}{8,12}
  \begin{phonetics}{居然}{ju1ran2}
    \definition{adv.}{inesperadamente | na verdade | para surpresa de alguém}
  \end{phonetics}
\end{entry}

\begin{entry}{岭}{8}[Radical 山]
  \begin{phonetics}{岭}{ling3}
    \definition{s.}{cordilheira}
  \end{phonetics}
\end{entry}

\begin{entry}{帘}{8}[Radical 巾]
  \begin{phonetics}{帘}{lian2}
    \definition{s.}{cortina | tela (pendurada) | bandeira usada como placa de loja}
  \end{phonetics}
\end{entry}

\begin{entry}{幷}{8}[Radical 干]
  \begin{phonetics}{幷}{bing4}
    \variantof{并}
  \end{phonetics}
\end{entry}

\begin{entry}{幸亏}{8,3}
  \begin{phonetics}{幸亏}{xing4kui1}
    \definition{adv.}{felizmente}
  \end{phonetics}
\end{entry}

\begin{entry}{幸运}{8,7}
  \begin{phonetics}{幸运}{xing4yun4}
    \definition{adj.}{afortunado | feliz | sortudo}
  \end{phonetics}
\end{entry}

\begin{entry}{幸运儿}{8,7,2}
  \begin{phonetics}{幸运儿}{xing4yun4'er2}
    \definition{s.}{pessoa de sorte}
  \end{phonetics}
\end{entry}

\begin{entry}{幸运抽奖}{8,7,8,9}
  \begin{phonetics}{幸运抽奖}{xing4yun4chou1jiang3}
    \definition{s.}{loteria | sorteio}
  \end{phonetics}
\end{entry}

\begin{entry}{幸福}{8,13}
  \begin{phonetics}{幸福}{xing4fu2}
    \definition{adj.}{feliz | abençoado}
    \definition{s.}{felicidade}
  \end{phonetics}
\end{entry}

\begin{entry}{底气}{8,4}
  \begin{phonetics}{底气}{di3qi4}
    \definition{s.}{capacidade pulmonar | ousadia | confiança | autoconfiança | vigor}
  \end{phonetics}
\end{entry}

\begin{entry}{店}{8}[Radical 广]
  \begin{phonetics}{店}{dian4}[][HSK 2]
    \definition{s.}{loja | pousada}
  \end{phonetics}
\end{entry}

\begin{entry}{店主}{8,5}
  \begin{phonetics}{店主}{dian4zhu3}
    \definition{s.}{lojista | dono de loja}
  \end{phonetics}
\end{entry}

\begin{entry}{店员}{8,7}
  \begin{phonetics}{店员}{dian4yuan2}
    \definition{s.}{assistente de loja | balconista | vendedor}
  \end{phonetics}
\end{entry}

\begin{entry}{建立者}{8,5,8}
  \begin{phonetics}{建立者}{jian4li4zhe3}
    \definition{s.}{fundador}
  \end{phonetics}
\end{entry}

\begin{entry}{建议}{8,5}
  \begin{phonetics}{建议}{jian4yi4}
    \definition[个,点]{s.}{proposta | recomendação | sugestão}
    \definition{v.}{propor | recomendar | sugerir}
  \end{phonetics}
\end{entry}

\begin{entry}{建设}{8,6}
  \begin{phonetics}{建设}{jian4she4}
    \definition{s.}{construção}
    \definition{v.}{construir}
  \end{phonetics}
\end{entry}

\begin{entry}{建设性}{8,6,8}
  \begin{phonetics}{建设性}{jian4she4xing4}
    \definition{adj.}{construtivo}
    \definition{s.}{construtividade}
  \end{phonetics}
\end{entry}

\begin{entry}{建设者}{8,6,8}
  \begin{phonetics}{建设者}{jian4she4zhe3}
    \definition{s.}{construtor}
  \end{phonetics}
\end{entry}

\begin{entry}{建筑}{8,12}
  \begin{phonetics}{建筑}{jian4zhu4}
    \definition[个]{s.}{construção | prédio | edifício}
    \definition{v.}{construir}
  \end{phonetics}
\end{entry}

\begin{entry}{廻}{8}[Radical 廴]
  \begin{phonetics}{廻}{hui2}
    \variantof{回}
  \end{phonetics}
\end{entry}

\begin{entry}{录音}{8,9}
  \begin{phonetics}{录音}{lu4yin1}
    \definition[个]{s.}{gravação de som}
    \definition{v.+compl.}{gravar (som)}
  \end{phonetics}
\end{entry}

\begin{entry}{录音机}{8,9,6}
  \begin{phonetics}{录音机}{lu4yin1ji1}
    \definition[台]{s.}{gravador de áudio}
  \end{phonetics}
\end{entry}

\begin{entry}{录像机}{8,13,6}
  \begin{phonetics}{录像机}{lu4xiang4ji1}
    \definition[台]{s.}{gravador de vídeo | VCR}
  \end{phonetics}
\end{entry}

\begin{entry}{录像带}{8,13,9}
  \begin{phonetics}{录像带}{lu4xiang4dai4}
    \definition[盘]{s.}{video-cassete}
  \end{phonetics}
\end{entry}

\begin{entry}{往}{8}[Radical 彳]
  \begin{phonetics}{往}{wang3}[][HSK 2]
    \definition{prep.}{para | em direção a}
  \end{phonetics}
\end{entry}

\begin{entry}{往日}{8,4}
  \begin{phonetics}{往日}{wang3ri4}
    \definition{adv.}{dias passados}
    \definition{s.}{o passado}
  \end{phonetics}
\end{entry}

\begin{entry}{往生}{8,5}
  \begin{phonetics}{往生}{wang3sheng1}
    \definition{v.}{renascer | morrer | (Budismo) viver no paraíso}
  \end{phonetics}
\end{entry}

\begin{entry}{往来}{8,7}
  \begin{phonetics}{往来}{wang3lai2}
    \definition{s.}{contatos | negociações}
  \end{phonetics}
\end{entry}

\begin{entry}{往返}{8,7}
  \begin{phonetics}{往返}{wang3fan3}
    \definition{s.}{ida e volta}
    \definition{v.}{ir e voltar | ir e vir}
  \end{phonetics}
\end{entry}

\begin{entry}{往事}{8,8}
  \begin{phonetics}{往事}{wang3shi4}
    \definition{s.}{acontecimentos anteriores | eventos passados}
  \end{phonetics}
\end{entry}

\begin{entry}{往例}{8,8}
  \begin{phonetics}{往例}{wang3li4}
    \definition{s.}{prática (habitual) do passado | precedente}
  \end{phonetics}
\end{entry}

\begin{entry}{往往}{8,8}
  \begin{phonetics}{往往}{wang3wang3}
    \definition{adv.}{em muitos casos | mais frequentes do que não | geralmente | em todos os lugares (chinês clássico)}
  \end{phonetics}
\end{entry}

\begin{entry}{往昔}{8,8}
  \begin{phonetics}{往昔}{wang3xi1}
    \definition{s.}{o passado}
  \end{phonetics}
\end{entry}

\begin{entry}{往复}{8,9}
  \begin{phonetics}{往复}{wang3fu4}
    \definition{s.}{para trás e para frente (por exemplo, da ação do pistão ou da bomba)}
    \definition{v.}{ir e voltar | fazer uma viagem de volta}
  \end{phonetics}
\end{entry}

\begin{entry}{往迹}{8,9}
  \begin{phonetics}{往迹}{wang3ji4}
    \definition{s.}{eventos passados}
  \end{phonetics}
\end{entry}

\begin{entry}{往程}{8,12}
  \begin{phonetics}{往程}{wang3cheng2}
    \definition{s.}{saída (de uma viagem de ônibus ou trem, etc.)}
  \end{phonetics}
\end{entry}

\begin{entry}{忽然}{8,12}
  \begin{phonetics}{忽然}{hu1ran2}[][HSK 2]
    \definition{adv.}{de repente}
  \end{phonetics}
\end{entry}

\begin{entry}{态度}{8,9}
  \begin{phonetics}{态度}{tai4du5}[][HSK 2]
    \definition[个]{s.}{maneira | comportamento | atitude | atitude | abordagem}
  \end{phonetics}
\end{entry}

\begin{entry}{怕}{8}[Radical 心]
  \begin{phonetics}{怕}{pa4}[][HSK 2]
    \definition*{s.}{sobrenome Pa}
    \definition{v.}{ter medo | ser incapaz de suportar | temer}
  \end{phonetics}
\end{entry}

\begin{entry}{性}{8}[Radical 心]
  \begin{phonetics}{性}{xing4}
    \definition{adj.}{sufixo formando adjetivo a partir de verbo}
    \definition[个]{s.}{natureza | carácter | propriedade | qualidade | atributo | sexualidade | sexo | gênero | essência}
    \definition{s.}{sufixo formando substantivo a partir de adjetivo}
  \end{phonetics}
\end{entry}

\begin{entry}{性生活}{8,5,9}
  \begin{phonetics}{性生活}{xing4sheng1huo2}
    \definition{s.}{vida sexual}
  \end{phonetics}
\end{entry}

\begin{entry}{性侵}{8,9}
  \begin{phonetics}{性侵}{xing4qin1}
    \definition{s.}{agressão sexual}
    \definition{v.}{agredir sexualmente}
  \end{phonetics}
\end{entry}

\begin{entry}{怪}{8}[Radical 心]
  \begin{phonetics}{怪}{guai4}
    \definition{adv.}{bastante (linguagem falada)}
  \end{phonetics}
\end{entry}

\begin{entry}{怪兽}{8,11}
  \begin{phonetics}{怪兽}{guai4shou4}
    \definition{s.}{animal raro | animal mítico | monstro}
  \end{phonetics}
\end{entry}

\begin{entry}{怪癖}{8,18}
  \begin{phonetics}{怪癖}{guai4pi3}
    \definition{adj.}{peculiar}
    \definition{s.}{excentricidade | peculiaridade | hobby estranho}
  \end{phonetics}
\end{entry}

\begin{entry}{或}{8}[Radical 戈]
  \begin{phonetics}{或}{huo4}[][HSK 2]
    \definition{conj.}{ou | ou\dots ou\dots}
  \end{phonetics}
\end{entry}

\begin{entry}{或者}{8,8}
  \begin{phonetics}{或者}{huo4zhe3}[][HSK 2]
    \definition{conj.}{ou (usado em expressões afirmativas)}
  \end{phonetics}
\end{entry}

\begin{entry}{房子}{8,3}
  \begin{phonetics}{房子}{fang2zi5}[][HSK 1]
    \definition[栋,幢,座,套,间,个]{s.}{apartamento | casa | quarto}
  \end{phonetics}
\end{entry}

\begin{entry}{房东}{8,5}
  \begin{phonetics}{房东}{fang2dong1}
    \definition{s.}{dono | proprietário | senhorio}
  \end{phonetics}
\end{entry}

\begin{entry}{房主}{8,5}
  \begin{phonetics}{房主}{fang2zhu3}
    \definition{s.}{proprietário | dono de um imóvel}
  \end{phonetics}
\end{entry}

\begin{entry}{房间}{8,7}
  \begin{phonetics}{房间}{fang2jian1}[][HSK 1]
    \definition[间,个]{s.}{quarto}
  \end{phonetics}
\end{entry}

\begin{entry}{所以}{8,4}
  \begin{phonetics}{所以}{suo3 yi3}[][HSK 2]
    \definition{adv.}{portanto | então | como resultado}
    \definition{conj.}{por isso | como resultado | a razão porque}
  \end{phonetics}
\end{entry}

\begin{entry}{所有}{8,6}
  \begin{phonetics}{所有}{suo3you3}[][HSK 2]
    \definition{adj.}{todo | tudo}
    \definition{s.}{posses}
    \definition{v.}{possuir | ser dono de}
  \end{phonetics}
\end{entry}

\begin{entry}{承认}{8,4}
  \begin{phonetics}{承认}{cheng2ren4}
    \definition{s.}{reconhecimento (diplomático, artístico, etc.)}
    \definition{v.}{admitir | conceder | reconhecer}
  \end{phonetics}
\end{entry}

\begin{entry}{抬杠}{8,7}
  \begin{phonetics}{抬杠}{tai2gang4}
    \definition{v.+compl.}{discutir pelo prazer em discutir | discutir obstinadamente | brigar}
  \end{phonetics}
\end{entry}

\begin{entry}{抱怨}{8,9}
  \begin{phonetics}{抱怨}{bao4yuan4}
    \definition{v.}{reclamar | resmungar | abrir uma reclamação | sentir-se insatisfeito}
  \end{phonetics}
\end{entry}

\begin{entry}{抵抗}{8,7}
  \begin{phonetics}{抵抗}{di3kang4}
    \definition{s.}{resistência}
    \definition{v.}{resistir}
  \end{phonetics}
\end{entry}

\begin{entry}{抹泪}{8,8}
  \begin{phonetics}{抹泪}{mo3lei4}
    \definition{v.}{limpar as lágrimas | (figurativo) derramar lágrimas}
  \end{phonetics}
\end{entry}

\begin{entry}{押}{8}[Radical 手]
  \begin{phonetics}{押}{ya1}
    \definition{v.}{deter sob custódia | escoltar e proteger | hipotecar | penhorar}
  \end{phonetics}
\end{entry}

\begin{entry}{押后}{8,6}
  \begin{phonetics}{押后}{ya1hou4}
    \definition{v.}{encerrar | adiar}
  \end{phonetics}
\end{entry}

\begin{entry}{押运}{8,7}
  \begin{phonetics}{押运}{ya1yun4}
    \definition{v.}{escoltar sob guarda | escoltar (bens ou fundos)}
  \end{phonetics}
\end{entry}

\begin{entry}{押注}{8,8}
  \begin{phonetics}{押注}{ya1zhu4}
    \definition{v.}{apostar}
  \end{phonetics}
\end{entry}

\begin{entry}{押金}{8,8}
  \begin{phonetics}{押金}{ya1jin1}
    \definition{s.}{caução | sinal | depósito}
  \end{phonetics}
\end{entry}

\begin{entry}{押送}{8,9}
  \begin{phonetics}{押送}{ya1song4}
    \definition{v.}{enviar sob escolta | transportar um detido}
  \end{phonetics}
\end{entry}

\begin{entry}{押租}{8,10}
  \begin{phonetics}{押租}{ya1zu1}
    \definition{s.}{depósito de aluguel}
  \end{phonetics}
\end{entry}

\begin{entry}{押韵}{8,13}
  \begin{phonetics}{押韵}{ya1yun4}
    \definition{v.}{rimar}
  \end{phonetics}
\end{entry}

\begin{entry}{担心}{8,4}
  \begin{phonetics}{担心}{dan1xin1}
    \definition{v.}{preocupar-se | estar preocupado}
  \end{phonetics}
\end{entry}

\begin{entry}{拆}{8}[Radical 手]
  \begin{phonetics}{拆}{chai1}
    \definition{v.}{remover | tirar do seu lugar | desfazer | desmontar}
  \end{phonetics}
\end{entry}

\begin{entry}{拉}{8}[Radical 手]
  \begin{phonetics}{拉}{la1}
    \definition{v.}{puxar | arrastar | desenhar | conversar | (coloquial) esvaziar as entranhas}
  \end{phonetics}
  \begin{phonetics}{拉}{la4}
    \definition{s.}{usado em 拉拉蛄 \dpy{la4la4gu3}}
    \seeref{拉拉蛄}{la4la4gu3}
  \end{phonetics}
\end{entry}

\begin{entry}{拉拉队}{8,8,4}
  \begin{phonetics}{拉拉队}{la1la1dui4}
    \definition{s.}{claque | torcida}
  \end{phonetics}
\end{entry}

\begin{entry}{拉拉蛄}{8,8,11}
  \begin{phonetics}{拉拉蛄}{la4la4gu3}
    \variantof{蝲蝲蛄}
  \end{phonetics}
\end{entry}

\begin{entry}{拍马}{8,3}
  \begin{phonetics}{拍马}{pai1ma3}
    \definition{v.}{instigar um cavalo dando tapinhas em seu traseiro | lisonjear | bajular}
  \seealsoref{拍马屁}{pai1ma3pi4}
  \end{phonetics}
\end{entry}

\begin{entry}{拍马屁}{8,3,7}
  \begin{phonetics}{拍马屁}{pai1ma3pi4}
    \definition{s.}{puxa-saco | bajulador}
    \definition{v.}{puxar o saco | bajular}
  \seealsoref{拍马}{pai1ma3}
  \end{phonetics}
\end{entry}

\begin{entry}{拍照}{8,13}
  \begin{phonetics}{拍照}{pai1zhao4}
    \definition{v.+compl.}{tirar fotografia}
  \end{phonetics}
\end{entry}

\begin{entry}{拐}{8}[Radical 手]
  \begin{phonetics}{拐}{guai3}
    \definition{s.}{bengala | muleta}
    \definition{v.}{virar (uma esquina, etc.) | cortar | sequestrar | fraudar | apropriar-se indevidamente}
  \end{phonetics}
\end{entry}

\begin{entry}{拔尖}{8,6}
  \begin{phonetics}{拔尖}{ba2jian1}
    \definition{adj.}{topo de linha | fora do comum | o melhor}
    \definition{v.+compl.}{empurrar-se para a frente | sentir que é superior aos outros}
  \end{phonetics}
\end{entry}

\begin{entry}{拖拉机}{8,8,6}
  \begin{phonetics}{拖拉机}{tuo1la1ji1}
    \definition[台]{s.}{trator}
  \end{phonetics}
\end{entry}

\begin{entry}{拖鞋}{8,15}
  \begin{phonetics}{拖鞋}{tuo1xie2}
    \definition[双,只]{s.}{chinelos | sandálias}
  \end{phonetics}
\end{entry}

\begin{entry}{招}{8}[Radical 手]
  \begin{phonetics}{招}{zhao1}
    \definition{adj.}{contagioso}
    \definition{s.}{um movimento (xadrez) | uma manobra | dispositivo | truque}
    \definition{v.}{recrutar | provocar | acenar | incorrer | infectar | confessar}
  \end{phonetics}
\end{entry}

\begin{entry}{招手}{8,4}
  \begin{phonetics}{招手}{zhao1shou3}
    \definition{v.+compl.}{acenar}
  \end{phonetics}
\end{entry}

\begin{entry}{招数}{8,13}
  \begin{phonetics}{招数}{zhao1shu4}
    \definition{s.}{estratégia | movimento (no xadrez, no palco, nas artes marciais) | esquema | truque}
  \end{phonetics}
\end{entry}

\begin{entry}{拧开}{8,4}
  \begin{phonetics}{拧开}{ning3kai1}
    \definition{v.}{desaparafusar | desatarrachar | torcer (uma tampa) | abrir (uma torneira) | ligar (girando um botão) | girar (maçaneta da porta)}
  \end{phonetics}
\end{entry}

\begin{entry}{拨转}{8,8}
  \begin{phonetics}{拨转}{bo1zhuan3}
    \definition{v.}{transferir (fundos, etc.) | virar | dar a volta}
  \end{phonetics}
\end{entry}

\begin{entry}{放}{8}[Radical 攴]
  \begin{phonetics}{放}{fang4}[][HSK 1]
    \definition{v.}{liberar | libertar | deixar ir | colocar | por | detonar (fogos de artifício)}
  \end{phonetics}
\end{entry}

\begin{entry}{放下}{8,3}
  \begin{phonetics}{放下}{fang4 xia4}[][HSK 2]
    \definition{v.}{deitar | colocar para baixo | deixar ir | liberar | desistir | colocar em algum lugar}
  \end{phonetics}
\end{entry}

\begin{entry}{放大}{8,3}
  \begin{phonetics}{放大}{fang4da4}
    \definition{v.}{ampliar}
  \end{phonetics}
\end{entry}

\begin{entry}{放飞}{8,3}
  \begin{phonetics}{放飞}{fang4fei1}
    \definition{s.}{deixar voar}
  \end{phonetics}
\end{entry}

\begin{entry}{放心}{8,4}
  \begin{phonetics}{放心}{fang4xin1}[][HSK 2]
    \definition{adj.}{despreocupado}
    \definition{v.}{sentir-se aliviado | sentir-se tranquilo | ficar à vontade}
    \definition{v.+compl.}{confiar | ter confiança em alguém | estar à vontade | sentir-se aliviado}
  \end{phonetics}
\end{entry}

\begin{entry}{放出}{8,5}
  \begin{phonetics}{放出}{fang4chu1}
    \definition{v.}{liberar | libertar}
  \end{phonetics}
\end{entry}

\begin{entry}{放电}{8,5}
  \begin{phonetics}{放电}{fang4dian4}
    \definition{s.}{descarga elétrica}
  \end{phonetics}
\end{entry}

\begin{entry}{放任}{8,6}
  \begin{phonetics}{放任}{fang4ren4}
    \definition{v.}{ignorar | saciar-se | deixar sozinho}
  \end{phonetics}
\end{entry}

\begin{entry}{放过}{8,6}
  \begin{phonetics}{放过}{fang4guo4}
    \definition{v.}{deixar | deixar alguém escapar impune | passar despercebido}
  \end{phonetics}
\end{entry}

\begin{entry}{放弃}{8,7}
  \begin{phonetics}{放弃}{fang4qi4}
    \definition{v.}{abandonar | desistir de | renunciar}
  \end{phonetics}
\end{entry}

\begin{entry}{放弃权利}{8,7,6,7}
  \begin{phonetics}{放弃权利}{fang4qi4 quan2li4}
    \definition{s.}{renúncia}
  \end{phonetics}
\end{entry}

\begin{entry}{放弃者}{8,7,8}
  \begin{phonetics}{放弃者}{fang4qi4zhe3}
    \definition{s.}{desistente}
  \end{phonetics}
\end{entry}

\begin{entry}{放走}{8,7}
  \begin{phonetics}{放走}{fang4zou3}
    \definition{v.}{permitir (uma pessoa ou um animal) ir | liberar | libertar}
  \end{phonetics}
\end{entry}

\begin{entry}{放学}{8,8}
  \begin{phonetics}{放学}{fang4 xue2}[][HSK 1]
    \definition{v.+compl.}{sair da escola | acabar as aulas | terminar a aula (por hoje)}
  \end{phonetics}
\end{entry}

\begin{entry}{放松}{8,8}
  \begin{phonetics}{放松}{fang4song1}
    \definition{adj.}{relaxado | afrouxado}
    \definition{v.}{relaxar | afrouxar}
  \end{phonetics}
\end{entry}

\begin{entry}{放养}{8,9}
  \begin{phonetics}{放养}{fang4yang3}
    \definition{v.}{criar (gado, peixes, culturas, etc.) | crescer | criar}
  \end{phonetics}
\end{entry}

\begin{entry}{放假}{8,11}
  \begin{phonetics}{放假}{fang4 jia4}[][HSK 1]
    \definition{v.}{ter férias ou feriado}
  \end{phonetics}
\end{entry}

\begin{entry}{放肆}{8,13}
  \begin{phonetics}{放肆}{fang4si4}
    \definition{adj.}{atrevido | pesunçoso | devasso}
  \end{phonetics}
\end{entry}

\begin{entry}{放鞭炮}{8,18,9}
  \begin{phonetics}{放鞭炮}{fang4bian1pao4}
    \definition{s.}{um conjunto de bombinhas ou traques}
  \end{phonetics}
\end{entry}

\begin{entry}{斩获}{8,10}
  \begin{phonetics}{斩获}{zhan3huo4}
    \definition{v.}{matar ou capturar (em batalha) | (figurativo) (esportes) marcar (um gol), ganhar (uma medalha) | (figurativo) colher recompensas, obter ganhos}
  \end{phonetics}
\end{entry}

\begin{entry}{明天}{8,4}
  \begin{phonetics}{明天}{ming2tian1}[][HSK 1]
    \definition{adv.}{amanhã}
  \end{phonetics}
\end{entry}

\begin{entry}{明白}{8,5}
  \begin{phonetics}{明白}{ming2bai5}[][HSK 1]
    \definition{adj.}{compreendido | percebido | óbvio | inequívoco}
    \definition{v.}{compreender | perceber}
  \end{phonetics}
\end{entry}

\begin{entry}{明年}{8,6}
  \begin{phonetics}{明年}{ming2nian2}[][HSK 1]
    \definition{adv.}{próximo ano}
  \end{phonetics}
\end{entry}

\begin{entry}{明明}{8,8}
  \begin{phonetics}{明明}{ming2ming2}
    \definition{adv.}{obviamente | claramente}
  \end{phonetics}
\end{entry}

\begin{entry}{明星}{8,9}
  \begin{phonetics}{明星}{ming2xing1}[][HSK 2]
    \definition[个,位,颗]{s.}{estrela | talento de ponta | estrela (artista) | estrela brilhante | estrela brilhante}
  \end{phonetics}
\end{entry}

\begin{entry}{明显}{8,9}
  \begin{phonetics}{明显}{ming2xian3}
    \definition{adj.}{óbvio | claro | distinto}
  \end{phonetics}
\end{entry}

\begin{entry}{明珠}{8,10}
  \begin{phonetics}{明珠}{ming2zhu1}
    \definition{s.}{pérola | jóia (de grande valor)}
  \end{phonetics}
\end{entry}

\begin{entry}{昔日}{8,4}
  \begin{phonetics}{昔日}{xi1ri4}
    \definition{adj.}{passado}
  \end{phonetics}
\end{entry}

\begin{entry}{朋友}{8,4}
  \begin{phonetics}{朋友}{peng2you5}[][HSK 1]
    \definition[个,位]{s.}{amigo}
  \end{phonetics}
\end{entry}

\begin{entry}{服}{8}[Radical ⽉]
  \begin{phonetics}{服}{fu2}
    \definition{s.}{roupas | vestido | vestuário | roupa de luto}
    \definition{v.}{servir (nas forças armadas, uma sentença de prisão, etc.) | obedecer | ser convencido (por um argumento) | convencer | admirar | aclimatar | tomar (medicamento) | usar roupas de luto}
  \end{phonetics}
  \begin{phonetics}{服}{fu4}
    \definition{clas.}{(para remédio) dose}
  \end{phonetics}
\end{entry}

\begin{entry}{服务}{8,5}
  \begin{phonetics}{服务}{fu2 wu4}[][HSK 2]
    \definition{v.}{prestar serviço a | estar a serviço de | servir | trabalhar | servir}
  \end{phonetics}
\end{entry}

\begin{entry}{服务员}{8,5,7}
  \begin{phonetics}{服务员}{fu2wu4yuan2}
    \definition{s.}{atendente | garçom | garçonete | pessoal de atendimento ao cliente}
  \end{phonetics}
\end{entry}

\begin{entry}{杯}{8}[Radical 木]
  \begin{phonetics}{杯}{bei1}[][HSK 1]
    \definition{clas.}{para certos recipientes de líquidos: copo, xícara, etc.}
    \definition{s.}{copo | caneca | xícara | taça | troféu}
  \end{phonetics}
\end{entry}

\begin{entry}{杯子}{8,3}
  \begin{phonetics}{杯子}{bei1zi5}[][HSK 1]
    \definition[个,只]{s.}{copo | caneca | xícara | taça}
  \end{phonetics}
\end{entry}

\begin{entry}{杯具}{8,8}
  \begin{phonetics}{杯具}{bei1ju4}
    \definition{s.}{parachoque | fiasco | (gíria) tragédia}
  \end{phonetics}
\end{entry}

\begin{entry}{松木}{8,4}
  \begin{phonetics}{松木}{song1mu4}
    \definition{s.}{pinheiro}
  \end{phonetics}
\end{entry}

\begin{entry}{构}{8}[Radical ⽊]
  \begin{phonetics}{构}{gou4}
    \definition{s.}{composição literária}
    \definition{v.}{construir | formar | compor}
    \variantof{够}
  \end{phonetics}
\end{entry}

\begin{entry}{枕}{8}[Radical 木]
  \begin{phonetics}{枕}{zhen3}
    \definition{s.}{travesseiro | almofada}
  \end{phonetics}
\end{entry}

\begin{entry}{果子}{8,3}
  \begin{phonetics}{果子}{guo3zi5}
    \definition{s.}{fruta}
  \end{phonetics}
\end{entry}

\begin{entry}{果然}{8,12}
  \begin{phonetics}{果然}{guo3ran2}
    \definition{adv.}{como esperado}
  \end{phonetics}
\end{entry}

\begin{entry}{果酱}{8,13}
  \begin{phonetics}{果酱}{guo3jiang4}
    \definition{s.}{geléia | compota ou doce (de frutas)}
  \end{phonetics}
\end{entry}

\begin{entry}{枫叶}{8,5}
  \begin{phonetics}{枫叶}{feng1ye4}
    \definition{s.}{folha de bordo (maple, tipo de árvore)}
  \end{phonetics}
\end{entry}

\begin{entry}{欧}{8}[Radical 欠]
  \begin{phonetics}{欧}{ou1}
    \definition*{s.}{Europa, abreviação de~欧洲 | sobrenome Ou}
    \seeref{欧洲}{ou1zhou1}
  \end{phonetics}
\end{entry}

\begin{entry}{欧洲}{8,9}
  \begin{phonetics}{欧洲}{ou1zhou1}
    \definition*{s.}{Europa}
  \end{phonetics}
\end{entry}

\begin{entry}{欧洲人}{8,9,2}
  \begin{phonetics}{欧洲人}{ou1zhou1ren2}
    \definition{s.}{europeu | pessoa ou povo da Europa}
  \end{phonetics}
\end{entry}

\begin{entry}{欧洲共同体}{8,9,6,6,7}
  \begin{phonetics}{欧洲共同体}{ou1zhou1 gong4tong2ti3}
    \definition*{s.}{Comunidade Europeia}
  \end{phonetics}
\end{entry}

\begin{entry}{欧盟}{8,13}
  \begin{phonetics}{欧盟}{ou1meng2}
    \definition*{s.}{Uniáo Europeia}
  \end{phonetics}
\end{entry}

\begin{entry}{武}{8}[Radical 止]
  \begin{phonetics}{武}{wu3}
    \definition*{s.}{sobrenome Wu}
    \definition{s.}{arte marcial}
  \end{phonetics}
\end{entry}

\begin{entry}{武力}{8,2}
  \begin{phonetics}{武力}{wu3li4}
    \definition{s.}{forças armadas | militares}
  \end{phonetics}
\end{entry}

\begin{entry}{武士}{8,3}
  \begin{phonetics}{武士}{wu3shi4}
    \definition{s.}{samurai | guerreiro}
  \end{phonetics}
\end{entry}

\begin{entry}{武大戏}{8,3,6}
  \begin{phonetics}{武大戏}{wu3 da4xi4}
    \definition*{s.}{Drama de Luta Acrobática | Drama Wu}
  \end{phonetics}
\end{entry}

\begin{entry}{武艺}{8,4}
  \begin{phonetics}{武艺}{wu3yi4}
    \definition{s.}{arte marcial | habilidade militar}
  \end{phonetics}
\end{entry}

\begin{entry}{武术}{8,5}
  \begin{phonetics}{武术}{wu3shu4}
    \definition[种]{s.}{arte marcial | autodefesa}
  \end{phonetics}
\end{entry}

\begin{entry}{武官}{8,8}
  \begin{phonetics}{武官}{wu3guan1}
    \definition{s.}{oficial militar}
  \end{phonetics}
\end{entry}

\begin{entry}{武断}{8,11}
  \begin{phonetics}{武断}{wu3duan4}
    \definition{adj.}{arbitrário | dogmático | subjetivo}
  \end{phonetics}
\end{entry}

\begin{entry}{武装}{8,12}
  \begin{phonetics}{武装}{wu3zhuang1}
    \definition{s.}{forças armadas | militar | arma}
    \definition{v.}{armar}
  \end{phonetics}
\end{entry}

\begin{entry}{武器}{8,16}
  \begin{phonetics}{武器}{wu3qi4}
    \definition[种]{s.}{arma}
  \end{phonetics}
\end{entry}

\begin{entry}{河}{8}[Radical 水]
  \begin{phonetics}{河}{he2}[][HSK 2]
    \definition[条,道]{s.}{rio}
  \end{phonetics}
\end{entry}

\begin{entry}{河蚌}{8,10}
  \begin{phonetics}{河蚌}{he2bang4}
    \definition{s.}{mexilhões | bivalves cultivados em rios e lagos}
  \end{phonetics}
\end{entry}

\begin{entry}{油}{8}[Radical 水]
  \begin{phonetics}{油}{you2}[][HSK 2]
    \definition{adj.}{oleoso | gorduroso | superficial | astuto}
    \definition{s.}{óleo | gordura | graxa | petróleo}
    \definition{v.}{aplicar óleo de tungue, tinta ou verniz}
  \end{phonetics}
\end{entry}

\begin{entry}{治理}{8,11}
  \begin{phonetics}{治理}{zhi4li3}
    \definition{s.}{governança | governo}
    \definition{v.}{gerir para melhor | administrar | por em ordem}
  \end{phonetics}
\end{entry}

\begin{entry}{治愈}{8,13}
  \begin{phonetics}{治愈}{zhi4yu4}
    \definition{v.}{curar | restaurar a saúde}
  \end{phonetics}
\end{entry}

\begin{entry}{泄气}{8,4}
  \begin{phonetics}{泄气}{xie4qi4}
    \definition{adj.}{decepcionante | frustrante | patético}
    \definition{v.+compl.}{perder o coração | sentir-se desencorajado | ficar desanimado}
  \end{phonetics}
\end{entry}

\begin{entry}{法}{8}[Radical 水]
  \begin{phonetics}{法}{fa3}
    \definition*{s.}{França, abreviação de~法国}
  \seealsoref{法国}{fa3guo2}
  \end{phonetics}
\end{entry}

\begin{entry}{法文}{8,4}
  \begin{phonetics}{法文}{fa3wen2}
    \definition*{s.}{françês, língua francesa}
  \end{phonetics}
\end{entry}

\begin{entry}{法网}{8,6}
  \begin{phonetics}{法网}{fa3wang3}
    \definition*{s.}{Torneio de Roland Garros (French Open), torneio de tênis}
  \end{phonetics}
\end{entry}

\begin{entry}{法国}{8,8}
  \begin{phonetics}{法国}{fa3guo2}
    \definition*{s.}{França}
  \end{phonetics}
\end{entry}

\begin{entry}{法国人}{8,8,2}
  \begin{phonetics}{法国人}{fa3guo2ren2}
    \definition{s.}{francês | pessoa ou povo da França}
  \end{phonetics}
\end{entry}

\begin{entry}{法语}{8,9}
  \begin{phonetics}{法语}{fa3yu3}
    \definition{s.}{françês, língua francesa}
  \end{phonetics}
\end{entry}

\begin{entry}{泡}{8}[Radical 水]
  \begin{phonetics}{泡}{pao1}
    \definition{adj.}{estufado | inchado | esponjoso}
    \definition{clas.}{para urina ou fezes}
    \definition{s.}{pequeno lago (especialmente em nomes de lugares)}
  \end{phonetics}
  \begin{phonetics}{泡}{pao4}
    \definition{clas.}{para ocorrências de uma ação | para número de infusões}
    \definition{s.}{bolha | espuma}
    \definition{v.}{encharcar | infundir | pegar (uma garota) | sair com (um parceiro sexual)}
  \end{phonetics}
\end{entry}

\begin{entry}{波}{8}[Radical 水]
  \begin{phonetics}{波}{bo1}
    \definition*{s.}{Polônia, abreviação de 波兰}
    \definition{s.}{onda | ondulação | tempestade | surto}
    \seeref{波兰}{bo1lan2}
  \end{phonetics}
\end{entry}

\begin{entry}{波兰}{8,5}
  \begin{phonetics}{波兰}{bo1lan2}
    \definition*{s.}{Polônia}
  \end{phonetics}
\end{entry}

\begin{entry}{波音}{8,9}
  \begin{phonetics}{波音}{bo1yin1}
    \definition*{s.}{Boeing (empresa aeroespacial)}
    \definition{s.}{mordente (música)}
  \end{phonetics}
\end{entry}

\begin{entry}{泥}{8}[Radical 水]
  \begin{phonetics}{泥}{ni2}
    \definition{s.}{lama | argila | pasta | polpa}
  \end{phonetics}
  \begin{phonetics}{泥}{ni4}
    \definition{adj.}{contido}
  \end{phonetics}
\end{entry}

\begin{entry}{泥潭}{8,15}
  \begin{phonetics}{泥潭}{ni2tan2}
    \definition{s.}{atoleiro | lamaçal | charco | pântano}
  \end{phonetics}
\end{entry}

\begin{entry}{注册}{8,5}
  \begin{phonetics}{注册}{zhu4ce4}
    \definition{v.}{inscrever-se | matricular-se | registrar-se}
  \end{phonetics}
\end{entry}

\begin{entry}{注册人}{8,5,2}
  \begin{phonetics}{注册人}{zhu4ce4ren2}
    \definition{s.}{registrante}
  \end{phonetics}
\end{entry}

\begin{entry}{注册表}{8,5,8}
  \begin{phonetics}{注册表}{zhu4ce4biao3}
    \definition*{s.}{Registro do Windows}
  \end{phonetics}
\end{entry}

\begin{entry}{注册商标}{8,5,11,9}
  \begin{phonetics}{注册商标}{zhu4ce4shang1biao1}
    \definition{s.}{marca registrada}
  \end{phonetics}
\end{entry}

\begin{entry}{注意}{8,13}
  \begin{phonetics}{注意}{zhu4yi4}
    \definition{v.}{tomar nota de | prestar atenção em}
  \end{phonetics}
\end{entry}

\begin{entry}{注意力}{8,13,2}
  \begin{phonetics}{注意力}{zhu4yi4li4}
    \definition{s.}{atenção}
  \end{phonetics}
\end{entry}

\begin{entry}{注意力缺失症}{8,13,2,10,5,10}
  \begin{phonetics}{注意力缺失症}{zhu4yi4li4que1shi1zheng4}
    \definition{s.}{transtorno de déficit de atenção}
  \end{phonetics}
\end{entry}

\begin{entry}{注意地}{8,13,6}
  \begin{phonetics}{注意地}{zhu4yi4di4}
    \definition{s.}{área de cuidado, de observação}
  \end{phonetics}
\end{entry}

\begin{entry}{泳池}{8,6}
  \begin{phonetics}{泳池}{yong3chi2}
    \definition{s.}{piscina}
  \seealsoref{游泳池}{you2yong3chi2}
  \seealsoref{游泳馆}{you2yong3guan3}
  \end{phonetics}
\end{entry}

\begin{entry}{泳衣}{8,6}
  \begin{phonetics}{泳衣}{yong3yi1}
    \definition{s.}{roupa de banho | maiô}
  \seealsoref{游泳衣}{you2yong3yi1}
  \end{phonetics}
\end{entry}

\begin{entry}{炎热}{8,10}
  \begin{phonetics}{炎热}{yan2re4}
    \definition{adj.}{extremamente quente | escaldante (clima)}
  \end{phonetics}
\end{entry}

\begin{entry}{炒}{8}[Radical ⽕]
  \begin{phonetics}{炒}{chao3}
    \definition{v.}{saltear | demitir (alguém)}
  \end{phonetics}
\end{entry}

\begin{entry}{爬}{8}[Radical 爪]
  \begin{phonetics}{爬}{pa2}[][HSK 2]
    \definition{v.}{escalar | subir | trepar | rastejar}
  \end{phonetics}
\end{entry}

\begin{entry}{爬上}{8,3}
  \begin{phonetics}{爬上}{pa2shang4}
    \definition{v.}{escalar}
  \end{phonetics}
\end{entry}

\begin{entry}{爬山}{8,3}
  \begin{phonetics}{爬山}{pa2shan1}[][HSK 2]
    \definition{s.}{alpinista | montanhismo}
    \definition{v.}{escalar uma montanha}
  \end{phonetics}
\end{entry}

\begin{entry}{爬升}{8,4}
  \begin{phonetics}{爬升}{pa2sheng1}
    \definition{v.}{ascender | ganhar promoção | subir (números de vendas, etc.) | aumentar}
  \end{phonetics}
\end{entry}

\begin{entry}{爬行}{8,6}
  \begin{phonetics}{爬行}{pa2xing2}
    \definition{v.}{rastejar | arrastar | engatinhar}
  \end{phonetics}
\end{entry}

\begin{entry}{爬杆}{8,7}
  \begin{phonetics}{爬杆}{pa2gan1}
    \definition{s.}{escalada em poste}
    \definition{v.}{escalar um poste}
  \end{phonetics}
\end{entry}

\begin{entry}{爬竿}{8,9}
  \begin{phonetics}{爬竿}{pa2gan1}
    \definition{s.}{poste de escalada | escalada em poste (como ginástica ou ato de circo)}
  \end{phonetics}
\end{entry}

\begin{entry}{爬梳}{8,11}
  \begin{phonetics}{爬梳}{pa2shu1}
    \definition{v.}{vasculhar (documentos históricos, etc.) | desvendar}
  \end{phonetics}
\end{entry}

\begin{entry}{爬犁}{8,11}
  \begin{phonetics}{爬犁}{pa2li2}
    \definition{s.}{trenó}
    \seeref{扒犁}{pa2li2}
  \end{phonetics}
\end{entry}

\begin{entry}{爬墙}{8,14}
  \begin{phonetics}{爬墙}{pa2qiang2}
    \definition{v.}{escalar uma parede}
  \end{phonetics}
\end{entry}

\begin{entry}{爸}{8}[Radical 父]
  \begin{phonetics}{爸}{ba4}[][HSK 1]
    \definition[个,位]{s.}{pai}
  \seealsoref{爸爸}{ba4ba5}
  \end{phonetics}
\end{entry}

\begin{entry}{爸妈}{8,6}
  \begin{phonetics}{爸妈}{ba4ma1}
    \definition{s.}{pai e mãe}
  \end{phonetics}
\end{entry}

\begin{entry}{爸爸}{8,8}
  \begin{phonetics}{爸爸}{ba4ba5}[][HSK 1]
    \definition[个,位]{s.}{papai, pai (informal)}
  \end{phonetics}
\end{entry}

\begin{entry}{牦牛}{8,4}
  \begin{phonetics}{牦牛}{mao2niu2}
    \definition{s.}{iaque}
  \end{phonetics}
\end{entry}

\begin{entry}{物理}{8,11}
  \begin{phonetics}{物理}{wu4li3}
    \definition{s.}{física (disciplina)}
  \end{phonetics}
\end{entry}

\begin{entry}{狒狒}{8,8}
  \begin{phonetics}{狒狒}{fei4fei4}
    \definition{s.}{babuíno}
  \end{phonetics}
\end{entry}

\begin{entry}{狗}{8}[Radical 犬]
  \begin{phonetics}{狗}{gou3}[][HSK 2]
    \definition[条,只]{s.}{cão | cachorro}
  \end{phonetics}
\end{entry}

\begin{entry}{玩}{8}[Radical 玉]
  \begin{phonetics}{玩}{wan2}
    \definition{s.}{brinquedo | algo usado para diversão}
    \definition{v.}{divertir-se | manter algo para entretenimento | brincar com}
  \end{phonetics}
\end{entry}

\begin{entry}{玩儿}{8,2}
  \begin{phonetics}{玩儿}{wan2r5}[][HSK 1]
    \definition{v.}{divertir-se}
  \end{phonetics}
\end{entry}

\begin{entry}{玩艺}{8,4}
  \begin{phonetics}{玩艺}{wan2yi4}
    \variantof{玩意}
  \end{phonetics}
\end{entry}

\begin{entry}{玩伴}{8,7}
  \begin{phonetics}{玩伴}{wan2ban4}
    \definition{s.}{parceiro de brincadeira}
  \end{phonetics}
\end{entry}

\begin{entry}{玩具}{8,8}
  \begin{phonetics}{玩具}{wan2ju4}
    \definition{s.}{brinquedo}
  \end{phonetics}
\end{entry}

\begin{entry}{玩具厂}{8,8,2}
  \begin{phonetics}{玩具厂}{wan2ju4chang3}
    \definition{s.}{fábrica de brinquedos}
  \end{phonetics}
\end{entry}

\begin{entry}{玩具车}{8,8,4}
  \begin{phonetics}{玩具车}{wan2ju4 che1}
    \definition{s.}{carrinho de brinquedo}
  \end{phonetics}
\end{entry}

\begin{entry}{玩味}{8,8}
  \begin{phonetics}{玩味}{wan2wei4}
    \definition{v.}{ponderar sutilezas | ruminar (pensamentos)}
  \end{phonetics}
\end{entry}

\begin{entry}{玩者}{8,8}
  \begin{phonetics}{玩者}{wan2zhe3}
    \definition{s.}{jogador}
  \end{phonetics}
\end{entry}

\begin{entry}{玩耍}{8,9}
  \begin{phonetics}{玩耍}{wan2shua3}
    \definition{v.}{divertir-me | brincar (como as crianças fazem)}
  \end{phonetics}
\end{entry}

\begin{entry}{玩家}{8,10}
  \begin{phonetics}{玩家}{wan2jia1}
    \definition{s.}{entusiasta (áudio, modelos de aviões, etc.) | jogador (de um jogo)}
  \end{phonetics}
\end{entry}

\begin{entry}{玩偶}{8,11}
  \begin{phonetics}{玩偶}{wan2'ou3}
    \definition{s.}{estatueta de brinquedo | boneco de ação | bicho de pelúcia | boneca}
  \end{phonetics}
\end{entry}

\begin{entry}{玩遍}{8,12}
  \begin{phonetics}{玩遍}{wan2bian4}
    \definition{v.}{passear (todo o país, toda a cidade, etc.) | visitar (um grande número de lugares)}
  \end{phonetics}
\end{entry}

\begin{entry}{玩意}{8,13}
  \begin{phonetics}{玩意}{wan2yi4}
    \definition{s.}{ato | brinquedo | coisa | truque (em uma performance, show de palco, acrobacias, etc.)}
  \end{phonetics}
\end{entry}

\begin{entry}{环卫}{8,3}
  \begin{phonetics}{环卫}{huan2wei4}
    \definition{s.}{limpeza pública | saneamento urbano | saneamento ambiental | abreviação de 环境卫生}
    \seeref{环境卫生}{huan2jing4wei4sheng1}
  \end{phonetics}
\end{entry}

\begin{entry}{环境}{8,14}
  \begin{phonetics}{环境}{huan2jing4}
    \definition[个]{s.}{ambiente | arredores | circunstâncias}
  \end{phonetics}
\end{entry}

\begin{entry}{环境卫生}{8,14,3,5}
  \begin{phonetics}{环境卫生}{huan2jing4wei4sheng1}
    \definition{s.}{saneamento ambiental}
  \seealsoref{环卫}{huan2wei4}
  \end{phonetics}
\end{entry}

\begin{entry}{现}{8}[Radical 見]
  \begin{phonetics}{现}{xian4}
    \definition{adj.}{presente | atual}
    \definition{v.}{aparecer}
    \seeref{见}{xian4}
  \end{phonetics}
\end{entry}

\begin{entry}{现代}{8,5}
  \begin{phonetics}{现代}{xian4dai4}
    \definition*{s.}{Hyundai, empresa sul-coreana}
    \definition{s.}{tempos modernos | era moderna}
  \end{phonetics}
\end{entry}

\begin{entry}{现在}{8,6}
  \begin{phonetics}{现在}{xian4zai4}[][HSK 1]
    \definition{adv.}{agora | neste momento}
  \end{phonetics}
\end{entry}

\begin{entry}{现有}{8,6}
  \begin{phonetics}{现有}{xian4you3}
    \definition{adj.}{disponível atualmente | atualmente existente}
  \end{phonetics}
\end{entry}

\begin{entry}{现抓}{8,7}
  \begin{phonetics}{现抓}{xian4zhua1}
    \definition{v.}{improvisar}
  \end{phonetics}
\end{entry}

\begin{entry}{现实}{8,8}
  \begin{phonetics}{现实}{xian4shi2}
    \definition{adj.}{real | realístico}
    \definition{s.}{realidade}
  \end{phonetics}
\end{entry}

\begin{entry}{现货}{8,8}
  \begin{phonetics}{现货}{xian4huo4}
    \definition{s.}{produtos à vista}
  \end{phonetics}
\end{entry}

\begin{entry}{现货的}{8,8,8}
  \begin{phonetics}{现货的}{xian4huo4 de5}
    \definition{s.}{produtos em estoque}
  \end{phonetics}
\end{entry}

\begin{entry}{现做}{8,11}
  \begin{phonetics}{现做}{xian4zuo4}
    \definition{adj.}{fresco}
    \definition{v.}{fazer (comida) no local}
  \end{phonetics}
\end{entry}

\begin{entry}{现象}{8,11}
  \begin{phonetics}{现象}{xian4xiang4}
    \definition[个,种]{s.}{fenômeno}
  \end{phonetics}
\end{entry}

\begin{entry}{画}{8}[Radical 田]
  \begin{phonetics}{画}{hua4}[][HSK 2]
    \definition[幅,张]{s.}{quadro | pintura | traço de um caractere chinês (variante de 划) | (caligrafia) traço horizontal (variante de traço 划)}
    \definition{v.}{desenhar | pintar | traçar uma linha (variante de 划)}
    \seeref{划}{hua4}
  \end{phonetics}
\end{entry}

\begin{entry}{画儿}{8,2}
  \begin{phonetics}{画儿}{hua4r5}[][HSK 2]
    \definition{s.}{imagem | desenho | pintura}
  \end{phonetics}
\end{entry}

\begin{entry}{画地为牢}{8,6,4,7}
  \begin{phonetics}{画地为牢}{hua4di4wei2lao2}
    \definition{expr.}{(literalmente) ser confinado dentro de um círculo desenhado no chão | (figurativo) limitar-se a uma gama restrita de atividades}
  \end{phonetics}
\end{entry}

\begin{entry}{画家}{8,10}
  \begin{phonetics}{画家}{hua4 jia1}[][HSK 2]
    \definition[个,名,位]{s.}{pintor}
  \end{phonetics}
\end{entry}

\begin{entry}{的}{8}[Radical 白]
  \begin{phonetics}{的}{de5}
    \definition{part.}{de | partícula usada em possessivos | utilizada entre adjetivos e substantivos (opcional se o adjetivo possui apenas um caracter) | usado após um atributo | usado para formar uma expressão nominal | usado no final de uma frase declarativa para dar ênfase}
  \end{phonetics}
  \begin{phonetics}{的}{di1}
    \definition{s.}{abreviação de 的士: um táxi}
  \seealsoref{的士}{di1shi4}
  \end{phonetics}
  \begin{phonetics}{的}{di2}
    \definition{adv.}{realmente e verdadeiramente}
  \end{phonetics}
  \begin{phonetics}{的}{di4}
    \definition{adj.}{objetivo | claro}
  \end{phonetics}
\end{entry}

\begin{entry}{的士}{8,3}
  \begin{phonetics}{的士}{di1shi4}
    \definition{s.}{(empréstimo linguístico) táxi}
  \end{phonetics}
\end{entry}

\begin{entry}{的话}{8,8}
  \begin{phonetics}{的话}{de5 hua4}[][HSK 2]
    \definition{part.}{se | no caso | suponha que}
  \end{phonetics}
\end{entry}

\begin{entry}{的确}{8,12}
  \begin{phonetics}{的确}{di2que4}
    \definition{adv.}{de fato | realmente}
  \end{phonetics}
\end{entry}

\begin{entry}{盲目}{8,5}
  \begin{phonetics}{盲目}{mang2mu4}
    \definition{adj.}{ignorante | sem compreensão}
    \definition{adv.}{cegamente}
    \definition{s.}{cego}
  \end{phonetics}
\end{entry}

\begin{entry}{直译}{8,7}
  \begin{phonetics}{直译}{zhi2yi4}
    \definition{s.}{tradução literal}
  \seealsoref{意译}{yi4yi4}
  \end{phonetics}
\end{entry}

\begin{entry}{直译器}{8,7,16}
  \begin{phonetics}{直译器}{zhi2yi4qi4}
    \definition{s.}{(computação) interpretador}
  \end{phonetics}
\end{entry}

\begin{entry}{直接}{8,11}
  \begin{phonetics}{直接}{zhi2jie1}[][HSK 2]
    \definition{adj.}{direto (oposto: indireto 间接) | imediato}
    \seeref{间接}{jian4jie1}
  \end{phonetics}
\end{entry}

\begin{entry}{直播}{8,15}
  \begin{phonetics}{直播}{zhi2bo1}
    \definition{s.}{transmissão ao vivo | (agricultura) semeadura direta}
    \definition{v.}{(TV, rádio, Internet) transmitir ao vivo}
  \end{phonetics}
\end{entry}

\begin{entry}{知识}{8,7}
  \begin{phonetics}{知识}{zhi1shi5}[][HSK 1]
    \definition[门]{s.}{conhecimento}
    \definition{s.}{intelectual}
  \end{phonetics}
\end{entry}

\begin{entry}{知道}{8,12}
  \begin{phonetics}{知道}{zhi1dao4}[][HSK 1]
    \definition{v.}{conhecer | saber}
  \end{phonetics}
\end{entry}

\begin{entry}{知道了}{8,12,2}
  \begin{phonetics}{知道了}{zhi1dao4le5}
    \definition{interj.}{Entendi! | OK!}
  \end{phonetics}
\end{entry}

\begin{entry}{矿泉水}{8,9,4}
  \begin{phonetics}{矿泉水}{kuang4quan2shui3}
    \definition[瓶,杯]{s.}{água mineral}
  \end{phonetics}
\end{entry}

\begin{entry}{祅}{8}[Radical 礻]
  \begin{phonetics}{祅}{yao1}
    \definition{s.}{espírito maligno | \emph{goblin} | bruxaria}
    \variantof{妖}
  \end{phonetics}
\end{entry}

\begin{entry}{空儿}{8,2}
  \begin{phonetics}{空儿}{kong4r5}
    \definition{s.}{tempo livre}
    \definition{v.}{ter tempo livre}
  \end{phonetics}
\end{entry}

\begin{entry}{空中小姐}{8,4,3,8}
  \begin{phonetics}{空中小姐}{kong1zhong1xiao3jie3}
    \definition{s.}{aeromoça}
  \end{phonetics}
\end{entry}

\begin{entry}{空心菜}{8,4,11}
  \begin{phonetics}{空心菜}{kong1xin1cai4}
    \definition{s.}{espinafre aquático | \emph{ong choy} | repolho do pântano | convolvulus aquático | glória-da-manhã aquática}
  \seealsoref{蕹菜}{weng4cai4}
  \end{phonetics}
\end{entry}

\begin{entry}{空气}{8,4}
  \begin{phonetics}{空气}{kong1qi4}[][HSK 2]
    \definition{s.}{ar | atmosfera}
  \end{phonetics}
\end{entry}

\begin{entry}{空间}{8,7}
  \begin{phonetics}{空间}{kong1jian1}
    \definition{s.}{espaço | sala | (figurativo) escopo | (astronomia) espaço sideral | (matemática, física) espaço}
  \end{phonetics}
\end{entry}

\begin{entry}{空间站}{8,7,10}
  \begin{phonetics}{空间站}{kong1jian1zhan4}
    \definition{s.}{estação espacial}
  \end{phonetics}
\end{entry}

\begin{entry}{空姐}{8,8}
  \begin{phonetics}{空姐}{kong1jie3}
    \definition{s.}{aeromoça | comissária de bordo | abreviação de 空中小姐}
    \seeref{空中小姐}{kong1zhong1xiao3jie3}
  \end{phonetics}
\end{entry}

\begin{entry}{空调}{8,10}
  \begin{phonetics}{空调}{kong1tiao2}
    \definition[台]{s.}{ar-condicionado | condicionador de ar}
  \end{phonetics}
\end{entry}

\begin{entry}{线香}{8,9}
  \begin{phonetics}{线香}{xian4xiang1}
    \definition{s.}{bastão ou vareta de incenso}
  \end{phonetics}
\end{entry}

\begin{entry}{练}{8}[Radical 糸]
  \begin{phonetics}{练}{lian4}[][HSK 2]
    \definition{s.}{exercício | (literário) seda branca}
    \definition{v.}{praticar | treinar | aperfeiçoar (habilidade) | ferver e esfregar seda crua}
  \end{phonetics}
\end{entry}

\begin{entry}{练习}{8,3}
  \begin{phonetics}{练习}{lian4xi2}[][HSK 2]
    \definition[个]{s.}{prática | exercício}
    \definition{v.}{praticar | exercitar}
  \end{phonetics}
\end{entry}

\begin{entry}{组}{8}[Radical 糸]
  \begin{phonetics}{组}{zu3}[][HSK 2]
    \definition*{s.}{sobrenome Zu}
    \definition{clas.}{para conjuntos, séries, suítes, baterias}
    \definition{s.}{grupo}
    \definition{v.}{organizar | formar}
  \end{phonetics}
\end{entry}

\begin{entry}{组长}{8,4}
  \begin{phonetics}{组长}{zu3 zhang3}[][HSK 2]
    \definition[名,位,个]{s.}{líder de grupo}
  \end{phonetics}
\end{entry}

\begin{entry}{组成}{8,6}
  \begin{phonetics}{组成}{zu3cheng2}[][HSK 2]
    \definition{v.}{formar | compor | inventar}
  \end{phonetics}
\end{entry}

\begin{entry}{细节}{8,5}
  \begin{phonetics}{细节}{xi4jie2}
    \definition{s.}{detalhe | particularidade}
  \end{phonetics}
\end{entry}

\begin{entry}{细菌战}{8,11,9}
  \begin{phonetics}{细菌战}{xi4jun1zhan4}
    \definition{s.}{guerra biológica}
  \end{phonetics}
\end{entry}

\begin{entry}{织}{8}[Radical 糸]
  \begin{phonetics}{织}{zhi1}
    \definition{v.}{tecer | tricotar}
  \end{phonetics}
\end{entry}

\begin{entry}{经}{8}[Radical 糸]
  \begin{phonetics}{经}{jing1}
    \definition*{s.}{sobrenome Jing}
    \definition{s.}{livro sagrado | escritura | clássicos | longitude | menstruação | canal}
    \definition{v.}{passar | sofrer | suportar | deformar (têxtil)}
  \end{phonetics}
\end{entry}

\begin{entry}{经过}{8,6}
  \begin{phonetics}{经过}{jing1guo4}[][HSK 2]
    \definition[个]{s.}{processo | curso}
    \definition{v.}{passar | passar por}
  \end{phonetics}
\end{entry}

\begin{entry}{经济}{8,9}
  \begin{phonetics}{经济}{jing1ji4}
    \definition{s.}{economia}
  \end{phonetics}
\end{entry}

\begin{entry}{经常}{8,11}
  \begin{phonetics}{经常}{jing1chang2}[][HSK 2]
    \definition{adv.}{constantemente | diariamente | dia-a-dia | todo dia | frequentemente | sempre | regularmente}
  \end{phonetics}
\end{entry}

\begin{entry}{经理}{8,11}
  \begin{phonetics}{经理}{jing1li3}[][HSK 2]
    \definition[个,位,名]{s.}{diretor | gerente}
  \end{phonetics}
\end{entry}

\begin{entry}{罔}{8}[Radical 网]
  \begin{phonetics}{罔}{wang3}
    \definition{v.}{enganar}
  \end{phonetics}
\end{entry}

\begin{entry}{罗}{8}[Radical 网]
  \begin{phonetics}{罗}{luo2}
    \definition*{s.}{sobrenome Luo}
    \definition{v.}{coletar | juntar | pegar | peneirar}
  \end{phonetics}
\end{entry}

\begin{entry}{肏}{8}[Radical 肉]
  \begin{phonetics}{肏}{cao4}
    \definition{v.}{(vulgar) foder}
  \end{phonetics}
\end{entry}

\begin{entry}{肩膀}{8,14}
  \begin{phonetics}{肩膀}{jian1bang3}
    \definition{s.}{ombro}
  \end{phonetics}
\end{entry}

\begin{entry}{肯定}{8,8}
  \begin{phonetics}{肯定}{ken3ding4}
    \definition{adv.}{com certeza | certamente | definitivamente | afirmativo (resposta)}
    \definition{v.}{afirmar | ter a certeza | ser positivo | dar reconhecimento}
  \end{phonetics}
\end{entry}

\begin{entry}{苦瓜}{8,5}
  \begin{phonetics}{苦瓜}{ku3gua1}
    \definition{s.}{melão amargo (cabaça amarga, pêra bálsamo, maçã bálsamo, pepino amargo)}
  \end{phonetics}
\end{entry}

\begin{entry}{英文}{8,4}
  \begin{phonetics}{英文}{ying1 wen2}[][HSK 2]
    \definition{s.}{inglês, língua inglesa}
  \end{phonetics}
\end{entry}

\begin{entry}{英国}{8,8}
  \begin{phonetics}{英国}{ying1guo2}
    \definition*{s.}{Reino Unido}
  \end{phonetics}
\end{entry}

\begin{entry}{英国人}{8,8,2}
  \begin{phonetics}{英国人}{ying1guo2ren2}
    \definition{s.}{inglês | pessoa ou povo do Reino Unido}
  \end{phonetics}
\end{entry}

\begin{entry}{英语}{8,9}
  \begin{phonetics}{英语}{ying1 yu3}[][HSK 2]
    \definition{s.}{inglês, língua inglesa}
  \end{phonetics}
\end{entry}

\begin{entry}{英雄}{8,12}
  \begin{phonetics}{英雄}{ying1xiong2}
    \definition[个]{s.}{herói}
  \end{phonetics}
\end{entry}

\begin{entry}{苹果}{8,8}
  \begin{phonetics}{苹果}{ping2guo3}
    \definition[个,颗]{s.}{maçã}
  \end{phonetics}
\end{entry}

\begin{entry}{茄子}{8,3}
  \begin{phonetics}{茄子}{qie2zi5}
    \definition{s.}{berinjela chinesa | ``xis'' fonético (ao ser fotografado), equivale ao ``diga xis''}
  \end{phonetics}
\end{entry}

\begin{entry}{虎}{8}[Radical ⾌]
  \begin{phonetics}{虎}{hu3}
    \definition{s.}{tigre}
  \seealsoref{老虎}{lao3hu3}
  \end{phonetics}
\end{entry}

\begin{entry}{虎口}{8,3}
  \begin{phonetics}{虎口}{hu3kou3}
    \definition{s.}{lugar perigoso | cova do tigre}
  \end{phonetics}
\end{entry}

\begin{entry}{虎虎}{8,8}
  \begin{phonetics}{虎虎}{hu3hu3}
    \definition{adj.}{formidável | forte | vigoroso}
  \end{phonetics}
\end{entry}

\begin{entry}{虎鼬}{8,18}
  \begin{phonetics}{虎鼬}{hu3you4}
    \definition{s.}{doninha}
  \end{phonetics}
\end{entry}

\begin{entry}{表}{8}[Radical 衣]
  \begin{phonetics}{表}{biao3}[][HSK 2]
    \definition*{s.}{sobrenome Biao}
    \definition{s.}{superfície externa | a relação entre os filhos ou netos de um irmão e uma irmã ou de irmãs | exemplo | modelo | memorial a um imperador dos tempos antigos | gráfico | formulário | lista | tabela | medidor | relógio de pulso}
  \end{phonetics}
\end{entry}

\begin{entry}{表白}{8,5}
  \begin{phonetics}{表白}{biao3bai2}
    \definition{s.}{declaração | confissão}
    \definition{v.}{confessar a si mesmo | expressar | revelar pensamentos ou sentimentos de alguém}
  \end{phonetics}
\end{entry}

\begin{entry}{表示}{8,5}
  \begin{phonetics}{表示}{biao3shi4}[][HSK 2]
    \definition{s.}{expressão | indicação}
    \definition{v.}{expressar | mostrar | indicar | significar}
  \end{phonetics}
\end{entry}

\begin{entry}{表扬}{8,6}
  \begin{phonetics}{表扬}{biao3yang2}
    \definition{v.}{elogiar | louvar}
  \end{phonetics}
\end{entry}

\begin{entry}{表扬信}{8,6,9}
  \begin{phonetics}{表扬信}{biao3yang2 xin4}
    \definition{s.}{carta de elogio | depoimento}
  \end{phonetics}
\end{entry}

\begin{entry}{表明}{8,8}
  \begin{phonetics}{表明}{biao3ming2}
    \definition{v.}{deixar claro | tornar conhecido | declarar claramente}
  \end{phonetics}
\end{entry}

\begin{entry}{表情}{8,11}
  \begin{phonetics}{表情}{biao3qing2}
    \definition{s.}{expressão (facial)}
    \definition{v.}{expressar os sentimentos de alguém}
  \end{phonetics}
\end{entry}

\begin{entry}{表演}{8,14}
  \begin{phonetics}{表演}{biao3yan3}
    \definition[场]{s.}{representação | atuação | exposição}
    \definition{v.}{representar | atuar}
  \end{phonetics}
\end{entry}

\begin{entry}{表演艺术}{8,14,4,5}
  \begin{phonetics}{表演艺术}{biao3yan3 yi4shu4}
    \definition{s.}{arte performática}
  \end{phonetics}
\end{entry}

\begin{entry}{表演者}{8,14,8}
  \begin{phonetics}{表演者}{biao3yan3 zhe3}
    \definition{s.}{ator}
  \end{phonetics}
\end{entry}

\begin{entry}{表演特技}{8,14,10,7}
  \begin{phonetics}{表演特技}{biao3yan3 te4ji4}
    \definition{s.}{acrobacia | pirueta | façanha}
  \end{phonetics}
\end{entry}

\begin{entry}{表演游戏}{8,14,12,6}
  \begin{phonetics}{表演游戏}{biao3yan3 you2xi4}
    \definition{s.}{exibição dramática}
  \end{phonetics}
\end{entry}

\begin{entry}{表演赛}{8,14,14}
  \begin{phonetics}{表演赛}{biao3yan3sai4}
    \definition{s.}{partida ou jogo de exibição}
  \end{phonetics}
\end{entry}

\begin{entry}{衬衫}{8,8}
  \begin{phonetics}{衬衫}{chen4shan1}
    \definition[件]{s.}{camisa | blusa}
  \end{phonetics}
\end{entry}

\begin{entry}{规定}{8,8}
  \begin{phonetics}{规定}{gui1ding4}
    \definition[个]{s.}{regulamento | regra}
    \definition{v.}{estipular}
  \end{phonetics}
\end{entry}

\begin{entry}{视角}{8,7}
  \begin{phonetics}{视角}{shi4jiao3}
    \definition{s.}{ângulo do qual se observa um objeto | (figurativo) perspectiva, ponto de vista, quadro de referência | (cinematografia) ângulo da câmera | (percepção visual) ângulo visual (o ângulo que um objeto visto subtende no olho) | (fotografia) ângulo de visão}
  \end{phonetics}
\end{entry}

\begin{entry}{视频}{8,13}
  \begin{phonetics}{视频}{shi4pin2}
    \definition{s.}{vídeo}
  \end{phonetics}
\end{entry}

\begin{entry}{试}{8}[Radical 言]
  \begin{phonetics}{试}{shi4}[][HSK 1]
    \definition{s.}{exame | experimento | prova | teste}
    \definition{v.}{experimentar | provar | testar}
  \end{phonetics}
\end{entry}

\begin{entry}{诗句}{8,5}
  \begin{phonetics}{诗句}{shi1ju4}
    \definition[行]{s.}{verso | versículo}
  \end{phonetics}
\end{entry}

\begin{entry}{诗词}{8,7}
  \begin{phonetics}{诗词}{shi1ci2}
    \definition{s.}{verso}
  \end{phonetics}
\end{entry}

\begin{entry}{诗意}{8,13}
  \begin{phonetics}{诗意}{shi1yi4}
    \definition{adj.}{poético}
    \definition{s.}{poesia}
  \end{phonetics}
\end{entry}

\begin{entry}{诚实}{8,8}
  \begin{phonetics}{诚实}{cheng2shi2}
    \definition{adj.}{honesto}
  \end{phonetics}
\end{entry}

\begin{entry}{诚实地}{8,8,6}
  \begin{phonetics}{诚实地}{cheng2shi2 di4}
    \definition{adv.}{honestamente}
  \end{phonetics}
\end{entry}

\begin{entry}{话}{8}[Radical 言]
  \begin{phonetics}{话}{hua4}[][HSK 1]
    \definition[种,席,句,口,番]{s.}{fala | linguagem | dialeto}
  \end{phonetics}
\end{entry}

\begin{entry}{诟骂}{8,9}
  \begin{phonetics}{诟骂}{gou4ma4}
    \definition{v.}{abusar verbalmente | insultar | criticar}
  \end{phonetics}
\end{entry}

\begin{entry}{该}{8}[Radical 言]
  \begin{phonetics}{该}{gai1}[][HSK 2]
    \definition{v.}{deveria | é a vez de alguém fazer algo | merecer | dever}
  \end{phonetics}
\end{entry}

\begin{entry}{责怪}{8,8}
  \begin{phonetics}{责怪}{ze2guai4}
    \definition{v.}{repreender | censurar}
  \end{phonetics}
\end{entry}

\begin{entry}{货车}{8,4}
  \begin{phonetics}{货车}{huo4che1}
    \definition{s.}{caminhão | van | vagão de carga}
  \end{phonetics}
\end{entry}

\begin{entry}{贪婪}{8,11}
  \begin{phonetics}{贪婪}{tan1lan2}
    \definition{adj.}{avaro | ambicioso | voraz | insaciável}
  \end{phonetics}
\end{entry}

\begin{entry}{贫民窟}{8,5,13}
  \begin{phonetics}{贫民窟}{pin2min2ku1}
    \definition{s.}{favela}
  \end{phonetics}
\end{entry}

\begin{entry}{转}{8}[Radical 車]
  \begin{phonetics}{转}{zhuan3}
    \definition{v.}{mudar de direção | transferir | encaminhar (correio) | virar}
  \end{phonetics}
  \begin{phonetics}{转}{zhuan4}
    \definition{clas.}{para ações repetidas | para rotações (por minuto, etc.): RPM}
    \definition{v.}{circular sobre | dar voltas | andar por aí}
  \end{phonetics}
\end{entry}

\begin{entry}{转产}{8,6}
  \begin{phonetics}{转产}{zhuan3chan3}
    \definition{v.}{mudar a produção | mudar para novos produtos}
  \end{phonetics}
\end{entry}

\begin{entry}{转告}{8,7}
  \begin{phonetics}{转告}{zhuan3gao4}
    \definition{v.}{comunicar | transmitir}
  \end{phonetics}
\end{entry}

\begin{entry}{转念}{8,8}
  \begin{phonetics}{转念}{zhuan3nian4}
    \definition{v.}{ter dúvidas sobre algo | pensar melhor}
  \end{phonetics}
\end{entry}

\begin{entry}{转账}{8,8}
  \begin{phonetics}{转账}{zhuan3zhang4}
    \definition{v.+compl.}{transferir entre contas | trazer à frente | incluir uma soma de dinheiro do balanço anterior no seguinte}
  \end{phonetics}
\end{entry}

\begin{entry}{转递}{8,10}
  \begin{phonetics}{转递}{zhuan3di4}
    \definition{v.}{passar | retransmitir}
  \end{phonetics}
\end{entry}

\begin{entry}{转悠}{8,11}
  \begin{phonetics}{转悠}{zhuan4you5}
    \definition{v.}{aparecer repetidamente | rolar | passear por aí}
  \end{phonetics}
\end{entry}

\begin{entry}{转游}{8,12}
  \begin{phonetics}{转游}{zhuan4you5}
    \variantof{转悠}
  \end{phonetics}
\end{entry}

\begin{entry}{轮回}{8,6}
  \begin{phonetics}{轮回}{lun2hui2}
    \definition[个]{s.}{reencarnação (Budismo) | ciclo}
    \definition{v.}{reencarnar}
  \end{phonetics}
\end{entry}

\begin{entry}{软件}{8,6}
  \begin{phonetics}{软件}{ruan3jian4}
    \definition{v.}{\emph{software}}
  \end{phonetics}
\end{entry}

\begin{entry}{轰鸣}{8,8}
  \begin{phonetics}{轰鸣}{hong1ming2}
    \definition{s.}{bum (som de explosão) | estrondo}
  \end{phonetics}
\end{entry}

\begin{entry}{轰炸机}{8,9,6}
  \begin{phonetics}{轰炸机}{hong1zha4ji1}
    \definition{s.}{bombardeiro (aeronave)}
  \end{phonetics}
\end{entry}

\begin{entry}{郁郁葱葱}{8,8,12,12}
  \begin{phonetics}{郁郁葱葱}{yu4yu4cong1cong1}
    \definition{expr.}{verdejante e exuberante}
  \end{phonetics}
\end{entry}

\begin{entry}{郊区}{8,4}
  \begin{phonetics}{郊区}{jiao1qu1}
    \definition[个]{s.}{subúrbio | distrito suburbano | arredores}
  \end{phonetics}
\end{entry}

\begin{entry}{采访}{8,6}
  \begin{phonetics}{采访}{cai3fang3}
    \definition{s.}{entrevista}
    \definition{v.}{entrevistar | reunir notícias | cobrir (eventos)}
  \end{phonetics}
\end{entry}

\begin{entry}{金刚石}{8,6,5}
  \begin{phonetics}{金刚石}{jin1gang1shi2}
    \definition{s.}{diamante, também chamado de 钻石}
    \seeref{钻石}{zuan4shi2}
  \end{phonetics}
\end{entry}

\begin{entry}{金色}{8,6}
  \begin{phonetics}{金色}{jin1se4}
    \definition{s.}{dourado}
  \end{phonetics}
\end{entry}

\begin{entry}{金融}{8,16}
  \begin{phonetics}{金融}{jin1rong2}
    \definition{s.}{finança}
  \end{phonetics}
\end{entry}

\begin{entry}{钓鱼}{8,8}
  \begin{phonetics}{钓鱼}{diao4yu2}
    \definition{v.}{pescar (com linha e anzol) | (figurativo) aprisionar}
  \end{phonetics}
\end{entry}

\begin{entry}{闸门}{8,3}
  \begin{phonetics}{闸门}{zha2men2}
    \definition{s.}{eclusa | comporta}
  \end{phonetics}
\end{entry}

\begin{entry}{雨}{8}[Radical 雨][Kangxi 173]
  \begin{phonetics}{雨}{yu3}[][HSK 1]
    \definition[阵,场]{s.}{chuva}
  \end{phonetics}
  \begin{phonetics}{雨}{yu4}[][HSK 0]
    \definition{v.}{cair (chuva, neve, etc.) | precipitar | chover | molhar}
  \end{phonetics}
\end{entry}

\begin{entry}{雨伞}{8,6}
  \begin{phonetics}{雨伞}{yu3san3}
    \definition[把]{s.}{guarda-chuva}
  \end{phonetics}
\end{entry}

\begin{entry}{雨衣}{8,6}
  \begin{phonetics}{雨衣}{yu3yi1}
    \definition[件]{s.}{impermeável}
  \end{phonetics}
\end{entry}

\begin{entry}{雨蚀}{8,9}
  \begin{phonetics}{雨蚀}{yu3shi2}
    \definition{s.}{erosão da chuva}
  \end{phonetics}
\end{entry}

\begin{entry}{雨靴}{8,13}
  \begin{phonetics}{雨靴}{yu3xue1}
    \definition[双]{s.}{botas de chuva}
  \end{phonetics}
\end{entry}

\begin{entry}{青天}{8,4}
  \begin{phonetics}{青天}{qing1tian1}
    \definition{s.}{céu claro, limpo ou azul}
  \end{phonetics}
\end{entry}

\begin{entry}{青少年}{8,4,6}
  \begin{phonetics}{青少年}{qing1shao4nian2}[][HSK 2]
    \definition[位,个]{s.}{adolescentes}
  \end{phonetics}
\end{entry}

\begin{entry}{青玉米}{8,5,6}
  \begin{phonetics}{青玉米}{qing1yu4mi3}
    \definition{s.}{milho verde}
  \end{phonetics}
\end{entry}

\begin{entry}{青年}{8,6}
  \begin{phonetics}{青年}{qing1 nian2}[][HSK 2]
    \definition[个,名,位]{s.}{juventude | jovem}
  \end{phonetics}
\end{entry}

\begin{entry}{青年节}{8,6,5}
  \begin{phonetics}{青年节}{qing1nian2jie2}
    \definition*{s.}{Dia da Juventude (4 de maio)}
  \end{phonetics}
\end{entry}

\begin{entry}{青春}{8,9}
  \begin{phonetics}{青春}{qing1chun1}
    \definition{s.}{juventude}
  \end{phonetics}
\end{entry}

\begin{entry}{青菜}{8,11}
  \begin{phonetics}{青菜}{qing1cai4}
    \definition{s.}{verduras}
  \end{phonetics}
\end{entry}

\begin{entry}{青铜}{8,11}
  \begin{phonetics}{青铜}{qing1tong2}
    \definition{s.}{bronze (liga de cobre, 銅, e estanho, 锡)}
  \end{phonetics}
\end{entry}

\begin{entry}{青椒}{8,12}
  \begin{phonetics}{青椒}{qing1jiao1}
    \definition{s.}{pimenta verde}
  \end{phonetics}
\end{entry}

\begin{entry}{青蛙}{8,12}
  \begin{phonetics}{青蛙}{qing1wa1}
    \definition{adj.}{(gíria velha) cara feio}
    \definition[只]{s.}{sapo}
  \end{phonetics}
\end{entry}

\begin{entry}{非}{8}[Radical ⾮][Kangxi 175]
  \begin{phonetics}{非}{fei1}
    \definition*{s.}{África, abreviação de 非洲}
    \definition{adv.}{não ser | não é | não}
  \seealsoref{非洲}{fei1zhou1}
  \end{phonetics}
\end{entry}

\begin{entry}{非洲}{8,9}
  \begin{phonetics}{非洲}{fei1zhou1}
    \definition*{s.}{África}
  \end{phonetics}
\end{entry}

\begin{entry}{非洲人}{8,9,2}
  \begin{phonetics}{非洲人}{fei1zhou1ren2}
    \definition{s.}{africano | pessoa ou povo da África}
  \end{phonetics}
\end{entry}

\begin{entry}{非常}{8,11}
  \begin{phonetics}{非常}{fei1chang2}[][HSK 1]
    \definition{adv.}{extraordinário | altamente | muito}
  \end{phonetics}
\end{entry}

\begin{entry}{靣}{8}[Radical 靣]
  \begin{phonetics}{靣}{mian4}
    \variantof{面}
  \end{phonetics}
\end{entry}

\begin{entry}{顶}{8}[Radical 頁]
  \begin{phonetics}{顶}{ding3}
    \definition{adv.}{mais | extremamente | melhor | muito (linguagem falada)}
  \end{phonetics}
\end{entry}

\begin{entry}{饱}{8}[Radical 食]
  \begin{phonetics}{饱}{bao3}[][HSK 2]
    \definition{adj.}{ter comido até ficar satisfeito | estar cheio | cheio}
    \definition{adv.}{completamente | até estar cheio}
    \definition{v.}{satisfazer}
  \end{phonetics}
\end{entry}

\begin{entry}{驻军}{8,6}
  \begin{phonetics}{驻军}{zhu4jun1}
    \definition{s.}{guarnição}
    \definition{v.}{guarcener ou prover uma tropa}
  \end{phonetics}
\end{entry}

\begin{entry}{驾照}{8,13}
  \begin{phonetics}{驾照}{jia4zhao4}
    \definition{s.}{carteira de motorista}
  \end{phonetics}
\end{entry}

\begin{entry}{鱼}{8}[Radical 魚]
  \begin{phonetics}{鱼}{yu2}[][HSK 2]
    \definition*{s.}{sobrenome Yu}
    \definition[条,尾]{s.}{peixe}
  \end{phonetics}
\end{entry}

\begin{entry}{鱼片}{8,4}
  \begin{phonetics}{鱼片}{yu2pian4}
    \definition{s.}{fatia de peixe | filé de peixe}
  \end{phonetics}
\end{entry}

\begin{entry}{鱼汛}{8,6}
  \begin{phonetics}{鱼汛}{yu2xun4}
    \variantof{渔汛}
  \end{phonetics}
\end{entry}

\begin{entry}{鱼网}{8,6}
  \begin{phonetics}{鱼网}{yu2wang3}
    \variantof{渔网}
  \end{phonetics}
\end{entry}

\begin{entry}{鱼具}{8,8}
  \begin{phonetics}{鱼具}{yu2ju4}
    \variantof{渔具}
  \end{phonetics}
\end{entry}

\begin{entry}{鱼香}{8,9}
  \begin{phonetics}{鱼香}{yu2xiang1}
    \definition{s.}{um tempero da culinária chinesa que normalmente contém alho, cebolinha, gengibre, açúcar, sal, pimenta, etc. (Embora 鱼香 signifique literalmente ``fragrância de peixe'', não contém frutos do mar.)}
  \end{phonetics}
\end{entry}

\begin{entry}{鱼香肉丝}{8,9,6,5}
  \begin{phonetics}{鱼香肉丝}{yu2xiang1rou4si1}
    \definition{s.}{tiras de carne de porco salteadas com molho picante (prato)}
  \seealsoref{鱼香}{yu2xiang1}
  \end{phonetics}
\end{entry}

\begin{entry}{鱼船}{8,11}
  \begin{phonetics}{鱼船}{yu2chuan2}
    \definition{s.}{barco de pesca}
  \seealsoref{渔船}{yu2chuan2}
  \end{phonetics}
\end{entry}

\begin{entry}{鸣}{8}[Radical 鳥]
  \begin{phonetics}{鸣}{ming2}
    \definition{v.}{chorar (pássaros, animais e insetos) | fazer um som | dar voz (gratidão, queixas, etc.)}
  \end{phonetics}
\end{entry}

%%%%% EOF %%%%%

