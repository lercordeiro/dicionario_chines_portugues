%%%
%%% 8画
%%%

\section*{8画}\addcontentsline{toc}{section}{8画}

\begin{Entry}{丧}{8}{⼗}
  \begin{Phonetics}{丧}{sang1}
    \definition{adj.}{decepcionado; deprimido; desanimado}
    \definition{v.}{perder | desanimar; frustrar}
  \end{Phonetics}
  \begin{Phonetics}{丧}{sang4}
    \definition{adj.}{decepcionado | desanimado}
    \definition{v.}{estar enlutado (do cônjuge etc.) | morrer}
  \end{Phonetics}
\end{Entry}

\begin{Entry}{丧失}{8,5}{⼗、⼤}
  \begin{Phonetics}{丧失}{sang4shi1}[][HSK 6]
    \definition{v.}{perder (algo que se tem)}
  \end{Phonetics}
\end{Entry}

\begin{Entry}{丧钟}{8,9}{⼗、⾦}
  \begin{Phonetics}{丧钟}{sang1zhong1}
    \definition{s.}{sentença de morte}
  \end{Phonetics}
\end{Entry}

\begin{Entry}{乖}{8}{⼃}
  \begin{Phonetics}{乖}{guai1}
    \definition{adj.}{(de uma criança) bem comportado; bom; obediente | inteligente; astuto; esperto | (de caráter, comportamento, etc.) estranho; anormal; irracional}
    \definition{v.}{perverter; ser contrário à razão; ir contra | (de caráter, comportamento, etc.) ser anormal; ser estranho}
  \end{Phonetics}
\end{Entry}

\begin{Entry}{乖乖}{8,8}{⼃、⼃}
  \begin{Phonetics}{乖乖}{guai1guai1}
    \definition{adj.}{bem-comportado (criança) | obediente}
  \end{Phonetics}
  \begin{Phonetics}{乖乖}{guai1guai5}
    \definition{expr.}{Graças a Deus! | Oh meu Deus!}
  \end{Phonetics}
\end{Entry}

\begin{Entry}{乳}{8}{⼄}
  \begin{Phonetics}{乳}{ru3}
    \definition{adj.}{recém-nascido (animal); lactente}
    \definition{s.}{mama; peito | leite (em geral) | qualquer líquido semelhante ao leite}
    \definition{v.}{dar à luz}
  \end{Phonetics}
\end{Entry}

\begin{Entry}{乳制品}{8,8,9}{⼄、⼑、⼝}
  \begin{Phonetics}{乳制品}{ru3 zhi4 pin3}[][HSK 6]
    \definition{s.}{produtos lácteos}
  \end{Phonetics}
\end{Entry}

\begin{Entry}{乳房}{8,8}{⼄、⼾}
  \begin{Phonetics}{乳房}{ru3fang2}
    \definition{s.}{seio | mama | úbere}
  \end{Phonetics}
\end{Entry}

\begin{Entry}{事}{8}{⼅}
  \begin{Phonetics}{事}{shi4}[][HSK 1]
    \definition[件,桩,回]{s.}{assunto; questão; coisa; negócio | problema; acidente | emprego; trabalho | responsabilidade; envolvimento | caso, coisa; o que aconteceu}
    \definition{v.}{servir; atender | estar envolvido em; dedicar-se a}
  \end{Phonetics}
\end{Entry}

\begin{Entry}{事儿}{8,2}{⼅、⼉}
  \begin{Phonetics}{事儿}{shi4r5}
    \definition[件,桩]{s.}{o emprego | negócio | afazeres | assunto que precisa ser resolvido | matéria}
  \end{Phonetics}
\end{Entry}

\begin{Entry}{事业}{8,5}{⼅、⼀}
  \begin{Phonetics}{事业}{shi4ye4}[][HSK 3]
    \definition[个]{s.}{causa; carreira; empreendimento; atividades regulares realizadas por pessoas com um determinado objetivo, escala e sistema que têm impacto no desenvolvimento social | instituição; instalações; unidade de trabalho apoiada financeiramente pelo governo; refere-se especificamente a empresas que não têm rendimentos de produção, são financiadas pelo Estado e não realizam contabilidade económica}
  \end{Phonetics}
\end{Entry}

\begin{Entry}{事件}{8,6}{⼅、⼈}
  \begin{Phonetics}{事件}{shi4jian4}[][HSK 3]
    \definition[个,件,次]{s.}{evento; incidente; grandes eventos na história ou na sociedade}
  \end{Phonetics}
\end{Entry}

\begin{Entry}{事先}{8,6}{⼅、⼉}
  \begin{Phonetics}{事先}{shi4xian1}[][HSK 4]
    \definition{adv.}{antes; de antemão; com antecedência; antecipadamente}
  \end{Phonetics}
\end{Entry}

\begin{Entry}{事后}{8,6}{⼅、⼝}
  \begin{Phonetics}{事后}{shi4 hou4}[][HSK 6]
    \definition{s.}{depois; depois do evento; após o incidente ocorrer ou o problema ser resolvido}
  \end{Phonetics}
\end{Entry}

\begin{Entry}{事实}{8,8}{⼅、⼧}
  \begin{Phonetics}{事实}{shi4shi2}[][HSK 3]
    \definition[个,件]{s.}{mito; lenda; uma narrativa sobre alguém ou algo que foi transmitida oralmente}
    \definition{v.}{dizer; contar; ser dito; contar a história}
  \end{Phonetics}
\end{Entry}

\begin{Entry}{事实上}{8,8,3}{⼅、⼧、⼀}
  \begin{Phonetics}{事实上}{shi4 shi2 shang4}[][HSK 3]
    \definition{adv.}{realmente; de fato; na realidade; na verdade; de fato}
  \end{Phonetics}
\end{Entry}

\begin{Entry}{事物}{8,8}{⼅、⽜}
  \begin{Phonetics}{事物}{shi4wu4}[][HSK 4]
    \definition[种,类,个]{s.}{coisa; objeto; todos os objetos e fenômenos que existem objetivamente}
  \end{Phonetics}
\end{Entry}

\begin{Entry}{事故}{8,9}{⼅、⽁}
  \begin{Phonetics}{事故}{shi4gu4}[][HSK 3]
    \definition[起,桩,次,场]{s.}{acidente; perdas ou desastres repentinos, muitas vezes relacionados ao transporte, produção, trabalho e segurança pessoal}
  \end{Phonetics}
\end{Entry}

\begin{Entry}{事情}{8,11}{⼅、⼼}
  \begin{Phonetics}{事情}{shi4qing5}[][HSK 2]
    \definition[件,个,些,种]{s.}{assunto; questão; coisa; negócio | erro; acidente; infortúnio | (coloquial) emprego; trabalho}
  \end{Phonetics}
\end{Entry}

\begin{Entry}{些}{8}{⼆}
  \begin{Phonetics}{些}{xie1}[][HSK 4]
    \definition{adv.}{um pouco; um pouco mais; usado após um adjetivo ou parte de um verbo para indicar uma pequena quantidade, equivalente a 一点儿}
    \definition{clas.}{alguns; um pouco; denota uma quantidade indefinida}
  \seealsoref{一点儿}{yi4dian3r5}
  \end{Phonetics}
\end{Entry}

\begin{Entry}{些许}{8,6}{⼆、⾔}
  \begin{Phonetics}{些许}{xie1xu3}
    \definition{num.}{um pouco}
  \end{Phonetics}
\end{Entry}

\begin{Entry}{享}{8}{⼇}
  \begin{Phonetics}{享}{xiang3}
    \definition{v.}{aproveitar}
  \end{Phonetics}
\end{Entry}

\begin{Entry}{享受}{8,8}{⼇、⼜}
  \begin{Phonetics}{享受}{xiang3shou4}[][HSK 5]
    \definition{v.}{aproveitar; desfrutar; estar satisfeito material ou espiritualmente}
  \end{Phonetics}
\end{Entry}

\begin{Entry}{京}{8}{⼇}
  \begin{Phonetics}{京}{jing1}
    \definition*{s.}{Pequim (Beijing), abreviação de 北京 | Sobrenome Jing}
    \definition{num.}{dez milhões (um numeral antigo); 10.000.000; 1000.0000}
    \definition{s.}{capital de um país}
  \seealsoref{北京}{bei3 jing1}
  \end{Phonetics}
\end{Entry}

\begin{Entry}{京二胡}{8,2,9}{⼇、⼆、⾁}
  \begin{Phonetics}{京二胡}{jing1'er4hu2}
    \definition{s.}{um tipo de violino chinês semelhante ao 二胡 de duas cordas, usado principalmente para acompanhamento do canto da ópera de Pequim | também chamado de 京胡 | jing'erhu, um violino de duas cordas, intermediário em tamanho e tom entre o 京胡 e o 二胡, usado para acompanhar a ópera chinesa}
  \seealsoref{二胡}{er4hu2}
  \seealsoref{京胡}{jing1hu2}
  \end{Phonetics}
\end{Entry}

\begin{Entry}{京胡}{8,9}{⼇、⾁}
  \begin{Phonetics}{京胡}{jing1hu2}
    \definition{s.}{jinghu, um instrumento de arco de duas cordas com registro agudo; violino da ópera de Pequim | também chamado de 京二胡 | jinghu, um 二胡 (violino de duas cordas) menor e mais agudo, usado para acompanhar a ópera chinesa}
  \seealsoref{二胡}{er4hu2}
  \seealsoref{胡琴}{hu2qin2}
  \seealsoref{京二胡}{jing1'er4hu2}
  \end{Phonetics}
\end{Entry}

\begin{Entry}{京剧}{8,10}{⼇、⼑}
  \begin{Phonetics}{京剧}{jing1ju4}[][HSK 3]
    \definition*[场,段]{s.}{Ópera de Pequim}
  \end{Phonetics}
\end{Entry}

\begin{Entry}{佩}{8}{⼈}
  \begin{Phonetics}{佩}{pei4}
    \definition{s.}{um ornamento usado como pingente amarrados em cintos nos tempos antigos}
    \definition{v.}{vestir (na cintura, etc.) | (arcaico) admirar | (arcaico) usar, especialmente uma pistola ou espada, na cintura}
  \end{Phonetics}
\end{Entry}

\begin{Entry}{佩服}{8,8}{⼈、⽉}
  \begin{Phonetics}{佩服}{pei4fu2}
    \definition{v.}{admirar}
  \end{Phonetics}
\end{Entry}

\begin{Entry}{使}{8}{⼈}
  \begin{Phonetics}{使}{shi3}[][HSK 3]
    \definition{conj.}{se; supondo; usado como a primeira cláusula de uma frase complexa; indica uma relação hipotética; equivalente a 假如}
    \definition{s.}{enviado; mensageiro; pessoas em uma missão}
    \definition{v.}{enviar; despachar; dizer a alguém para fazer algo | usar; empregar; aplicar | deixar; chamar; habilitar}
  \seealsoref{假如}{jia3ru2}
  \end{Phonetics}
\end{Entry}

\begin{Entry}{使用}{8,5}{⼈、⽤}
  \begin{Phonetics}{使用}{shi3yong4}[][HSK 2]
    \definition{v.}{usar; empregar; aplicar; fazer com que pessoas, equipamentos, fundos, etc. sirvam a um determinado propósito}
  \end{Phonetics}
\end{Entry}

\begin{Entry}{使劲}{8,7}{⼈、⼒}
  \begin{Phonetics}{使劲}{shi3/jin4}[][HSK 4]
    \definition{v.+compl.}{colocar energia; exercer toda a sua força | esforçar-se para ajudar; colocar energia para ajudar}
  \end{Phonetics}
\end{Entry}

\begin{Entry}{使得}{8,11}{⼈、⼻}
  \begin{Phonetics}{使得}{shi3 de5}[][HSK 5]
    \definition{v.}{ser utilizável; poder ser usado | ser viável; ser exequível; ser possível;  poder fazer | fazer; tornar; causar um determinado resultado (intenção, plano, coisa)}
  \end{Phonetics}
\end{Entry}

\begin{Entry}{例}{8}{⼈}
  \begin{Phonetics}{例}{li4}
    \definition{adj.}{regular; rotineiro}
    \definition{s.}{exemplo; instância | precedente | caso; instância | regras; estatutos; regulamentos}
    \definition{v.}{analogizar}
  \end{Phonetics}
\end{Entry}

\begin{Entry}{例子}{8,3}{⼈、⼦}
  \begin{Phonetics}{例子}{li4 zi5}[][HSK 2]
    \definition[个]{s.}{exemplo; algo usado para ajudar a explicar ou provar uma determinada situação ou afirmação}
  \end{Phonetics}
\end{Entry}

\begin{Entry}{例外}{8,5}{⼈、⼣}
  \begin{Phonetics}{例外}{li4wai4}[][HSK 5]
    \definition[个,种]{s.}{exceção; situações que não se enquadram nas regras gerais ou nas leis comuns}
    \definition{v.}{ser excepcional; ser uma exceção}
  \end{Phonetics}
\end{Entry}

\begin{Entry}{例如}{8,6}{⼈、⼥}
  \begin{Phonetics}{例如}{li4ru2}[][HSK 2]
    \definition{conj.}{por exemplo; tal como; como por exemplo; colocado antes do exemplo, indica que o exemplo vem a seguir}
  \end{Phonetics}
\end{Entry}

\begin{Entry}{供}{8}{⼈}
  \begin{Phonetics}{供}{gong1}
    \definition*{s.}{Sobrenome Gong}
    \definition{v.}{fornecer; alimentar |  fornecer algo (para uso ou conveniência de); fornecer algumas condições de exploração à outra parte}
  \end{Phonetics}
  \begin{Phonetics}{供}{gong4}
    \definition{s.}{oferendas | confissão}
    \definition{v.}{depositar (oferendas) | confessar}
  \end{Phonetics}
\end{Entry}

\begin{Entry}{供应}{8,7}{⼈、⼴}
  \begin{Phonetics}{供应}{gong1 ying4}[][HSK 4]
    \definition{v.}{fornecer; prover de}
  \end{Phonetics}
\end{Entry}

\begin{Entry}{供给}{8,9}{⼈、⽷}
  \begin{Phonetics}{供给}{gong1ji3}[][HSK 6]
    \definition{s.}{fornecer; prover; fornecer produção e necessidades de vida, dinheiro, etc. para aqueles que precisam}
  \end{Phonetics}
\end{Entry}

\begin{Entry}{依}{8}{⼈}
  \begin{Phonetics}{依}{yi1}
    \definition*{s.}{Sobrenome Yi}
    \definition{prep.}{de acordo com; à luz de; julgando por}
    \definition{v.}{depender de; ser dependente de; confiar em | cumprir; ouvir; ceder a | inclinar-se; descansar sobre (ou contra)}
  \end{Phonetics}
\end{Entry}

\begin{Entry}{依旧}{8,5}{⼈、⽇}
  \begin{Phonetics}{依旧}{yi1jiu4}[][HSK 5]
    \definition{adv.}{ainda; como antes; como sempre}
  \end{Phonetics}
\end{Entry}

\begin{Entry}{依次}{8,6}{⼈、⽋}
  \begin{Phonetics}{依次}{yi1 ci4}[][HSK 6]
    \definition{adv.}{sucessivamente; na ordem correta; em ordem}
  \end{Phonetics}
\end{Entry}

\begin{Entry}{依法}{8,8}{⼈、⽔}
  \begin{Phonetics}{依法}{yi1 fa3}[][HSK 5]
    \definition{adv.}{e acordo com regras (ou métodos) fixas | de acordo com a lei; por força da lei; em conformidade com as disposições legais}
  \end{Phonetics}
\end{Entry}

\begin{Entry}{依偎}{8,11}{⼈、⼈}
  \begin{Phonetics}{依偎}{yi1wei1}
    \definition{v.}{aninhar-se | aconchegar-se}
  \end{Phonetics}
\end{Entry}

\begin{Entry}{依据}{8,11}{⼈、⼿}
  \begin{Phonetics}{依据}{yi1ju4}[][HSK 5]
    \definition{prep.}{julgando por; de acordo com; à luz de; com base em; de acordo com; introduzir algo que possa servir como premissa ou base}
    \definition[个]{s.}{base; evidência; fundamento; base para tomar uma decisão ou realizar uma ação}
    \definition{v.}{basear-se em; confiar em; depdender de; usar algo como premissa ou base}
  \end{Phonetics}
\end{Entry}

\begin{Entry}{依然}{8,12}{⼈、⽕}
  \begin{Phonetics}{依然}{yi1ran2}[][HSK 4]
    \definition{adv.}{ainda; como antes}
    \definition{v.}{estar quieto; estar como antes; estar como o original, sem alterações}
  \end{Phonetics}
\end{Entry}

\begin{Entry}{依照}{8,13}{⼈、⽕}
  \begin{Phonetics}{依照}{yi1 zhao4}[][HSK 5]
    \definition{prep.}{de acordo com; à luz de; introduzir certos padrões para os eventos, o que equivale a 按照}
    \definition{v.}{seguir (com base em algo)}
  \seealsoref{按照}{an4zhao4}
  \end{Phonetics}
\end{Entry}

\begin{Entry}{依赖}{8,13}{⼈、⾙}
  \begin{Phonetics}{依赖}{yi1lai4}[][HSK 6]
    \definition{v.}{confiar em; ser dependente de; ser completamente dependente e inseparável | depender de; ser mutuamente dependentes e inseparáveis}
  \end{Phonetics}
\end{Entry}

\begin{Entry}{依靠}{8,15}{⼈、⾮}
  \begin{Phonetics}{依靠}{yi1kao4}[][HSK 4]
    \definition{s.}{apoio; suporte; algo em que se apoiar; alguém ou algo em quem você pode confiar}
    \definition{v.}{depender de; confiar em (alguém ou alguma coisa para atingir um determinado objetivo)}
  \end{Phonetics}
\end{Entry}

\begin{Entry}{侧}{8}{⼈}
  \begin{Phonetics}{侧}{ce4}[][HSK 6]
    \definition*{s.}{Sobrenome Ce}
    \definition{s.}{lado | inclinação}
    \definition{v.}{inclinar; inclinar-se}
  \end{Phonetics}
  \begin{Phonetics}{侧}{zhai1}
    \definition{adj.}{inclinado; torto}
  \end{Phonetics}
\end{Entry}

\begin{Entry}{侧重}{8,9}{⼈、⾥}
  \begin{Phonetics}{侧重}{ce4zhong4}[][HSK 7-9]
    \definition{v.}{enfatizar; focar em (um certo aspecto)}
  \end{Phonetics}
\end{Entry}

\begin{Entry}{侧面}{8,9}{⼈、⾯}
  \begin{Phonetics}{侧面}{ce4mian4}[][HSK 7-9]
    \definition[个]{s.}{lado; flanco | vias indiretas; canais informais; algum aspecto; outro aspecto}
  \end{Phonetics}
\end{Entry}

\begin{Entry}{兔}{8}{⼉}
  \begin{Phonetics}{兔}{tu4}[][HSK 5]
    \definition[只]{s.}{lebre; coelho}
  \end{Phonetics}
\end{Entry}

\begin{Entry}{兔子}{8,3}{⼉、⼦}
  \begin{Phonetics}{兔子}{tu4zi5}
    \definition[只]{s.}{coelho | lebre}
  \end{Phonetics}
\end{Entry}

\begin{Entry}{其}{8}{⼋}
  \begin{Phonetics}{其}{qi2}[][HSK 5]
    \definition*{s.}{Sobrenome Qi}
    \definition{adv.}{fazer uma suposição ou uma réplica | expressar comando, ordem}
    \definition{pron.}{dele (dela, deles, delas) | ele, ela, isso, eles; elas | isso; tal | isso (referindo-se a nenhuma pessoa ou coisa específica)}
    \definition{suf.}{sufixo de palavra, anexado ao advérbio}
  \end{Phonetics}
\end{Entry}

\begin{Entry}{其中}{8,4}{⼋、⼁}
  \begin{Phonetics}{其中}{qi2zhong1}[][HSK 2]
    \definition{pron.}{dentro; entre (os quais, eles, etc.); em (o qual, ele, etc.); nas pessoas ou coisas mencionadas anteriormente}
  \end{Phonetics}
\end{Entry}

\begin{Entry}{其他}{8,5}{⼋、⼈}
  \begin{Phonetics}{其他}{qi2ta1}[][HSK 2]
    \definition{pron.}{outra pessoa/outra coisa | outras coisas; outras pessoas; em substituição de outras pessoas ou coisas}
  \end{Phonetics}
\end{Entry}

\begin{Entry}{其次}{8,6}{⼋、⽋}
  \begin{Phonetics}{其次}{qi2ci4}[][HSK 3]
    \definition{adj.}{secundário}
    \definition{conj.}{próximo; então; em segundo lugar; mais tarde na ordem}
  \end{Phonetics}
\end{Entry}

\begin{Entry}{其余}{8,7}{⼋、⼈}
  \begin{Phonetics}{其余}{qi2yu2}[][HSK 4]
    \definition{pron.}{o resto; os outros; o restante}
  \end{Phonetics}
\end{Entry}

\begin{Entry}{其实}{8,8}{⼋、⼧}
  \begin{Phonetics}{其实}{qi2shi2}[][HSK 3]
    \definition{adv.}{na verdade; na realidade; a primeira parte é a situação aparente, e 其实 é usado para introduzir a situação real}
  \end{Phonetics}
\end{Entry}

\begin{Entry}{具}{8}{⼋}
  \begin{Phonetics}{具}{ju4}
    \definition*{s.}{Sobrenome Ju}
    \definition{clas.}{(literário) usado para caixões, cadáveres e certos objetos}
    \definition{s.}{utensílio; ferramenta; implemento | capacidade; habilidade}
    \definition{v.}{possuir; ter | fornecer; prover | declarar; enumerar}
  \end{Phonetics}
\end{Entry}

\begin{Entry}{具有}{8,6}{⼋、⽉}
  \begin{Phonetics}{具有}{ju4 you3}[][HSK 3]
    \definition{v.}{ter; possuir; ser provido de}
  \end{Phonetics}
\end{Entry}

\begin{Entry}{具体}{8,7}{⼋、⼈}
  \begin{Phonetics}{具体}{ju4ti3}[][HSK 3]
    \definition{adj.}{específico; particular | concreto; específico; mais detalhado; muito detalhado; muito claro | concreto; real; não é abstrato, tem uma forma definida; pode ser visto ou sentido}
    \definition{v.}{incorporar; objetivar; combinar teorias, princípios, padrões, etc. com pessoas ou coisas específicas}
  \end{Phonetics}
\end{Entry}

\begin{Entry}{具备}{8,8}{⼋、⼡}
  \begin{Phonetics}{具备}{ju4bei4}[][HSK 4]
    \definition{v.}{ter; possuir; ser provido de}
  \end{Phonetics}
\end{Entry}

\begin{Entry}{典}{8}{⼋}
  \begin{Phonetics}{典}{dian3}
    \definition{s.}{lei; cânone; padrão; sistema; regulamentos | trabalho padrão de bolsa de estudos; livros que podem servir como padrões ou especificações | alusão; citação literária | cerimônia; uma grande cerimônia (nos tempos antigos, a etiqueta era um dos sistemas importantes do estado) | modelo; normas; regras}
    \definition{v.}{estar no comando de | hipotecar; usar imóveis ou casas como garantia ao pedir dinheiro emprestado}
  \end{Phonetics}
\end{Entry}

\begin{Entry}{典礼}{8,5}{⼋、⽰}
  \begin{Phonetics}{典礼}{dian3li3}[][HSK 5]
    \definition[个,次,场]{s.}{cerimônia; celebração; comemoração}
  \end{Phonetics}
\end{Entry}

\begin{Entry}{典型}{8,9}{⼋、⼟}
  \begin{Phonetics}{典型}{dian3xing2}[][HSK 4]
    \definition{adj.}{típico; representativo}
    \definition[个,种]{s.}{modelo; caso típico; indivíduo ou evento representativo | personagens típicos; personalidades modelo (em obras literárias); personagens na literatura e na arte que refletem a natureza de uma determinada sociedade e têm uma personalidade distinta}
  \end{Phonetics}
\end{Entry}

\begin{Entry}{净}{8}{⼎}
  \begin{Phonetics}{净}{jing4}[][HSK 6]
    \definition{adj.}{limpo | (depois de um verbo) terminado; sem nada sobrando | líquido | vazio; oco; nu}
    \definition{adv.}{todo; o tempo todo | somente; meramente; nada além de | inteiramente; indica puro e nada mais}
    \definition{s.}{o “rosto pintado”, comumente conhecido como Hualian, 花脸, um tipo de personagem da ópera de Pequim, etc.}
    \definition{v.}{tornar limpo | limpar; lavar; esfregar para limpar}
  \seealsoref{花脸}{hua1lian3}
  \end{Phonetics}
\end{Entry}

\begin{Entry}{凭}{8}{⼏}
  \begin{Phonetics}{凭}{ping2}[][HSK 5]
    \definition{conj.}{não importa (o que, como, etc.); conecta frases complexas condicionais para expressar incondicionalidade, equivalente a 任凭 ou 不论}
    \definition{prep.}{introduzir a ação ou o comportamento com base em algo; quando a frase nominal após 凭 é longa, pode-se adicionar 着 após 凭}
    \definition[张]{s.}{prova; evidência}
    \definition{v.}{apoiar-se; encostar-se | confiar em; depender de | basear-se em; tomar como base}
  \seealsoref{不论}{bu2 lun4}
  \seealsoref{任凭}{ren4 ping2}
  \seealsoref{着}{zhe5}
  \end{Phonetics}
\end{Entry}

\begin{Entry}{函}{8}{⼐}
  \begin{Phonetics}{函}{han2}
    \definition*{s.}{Sobrenome Han}
    \definition[封]{s.}{caixa; envelope; capa | carta}
  \end{Phonetics}
\end{Entry}

\begin{Entry}{函数}{8,13}{⼐、⽁}
  \begin{Phonetics}{函数}{han2shu4}
    \definition{s.}{função (matemática)}
  \end{Phonetics}
\end{Entry}

\begin{Entry}{刮}{8}{⼑}
  \begin{Phonetics}{刮}{gua1}[][HSK 6]
    \definition{v.}{barbear; raspar; depilar | untar com (pasta, etc.)  | extorquir; pilhar; adquirir avidamente (propriedade) por vários meios | (do vento) soprar}
  \end{Phonetics}
\end{Entry}

\begin{Entry}{刮风}{8,4}{⼑、⾵}
  \begin{Phonetics}{刮风}{gua1/feng1}
    \definition{v.+compl.}{ventar | fazer vento}
  \end{Phonetics}
\end{Entry}

\begin{Entry}{到}{8}{⼑}
  \begin{Phonetics}{到}{dao4}[][HSK 1]
    \definition*{s.}{Sobrenome Dao}
    \definition{adj.}{atencioso}
    \definition{prep.}{a; até; para; indica o tempo em que a ação ou comportamento foi alcançado}
    \definition{v.}{ir para; partir para | chegar; alcançar; chegar a | como complemento de um verbo para mostrar o resultado de uma ação}
  \end{Phonetics}
\end{Entry}

\begin{Entry}{到处}{8,5}{⼑、⼡}
  \begin{Phonetics}{到处}{dao4chu4}[][HSK 2]
    \definition{adv.}{em todos os lugares; em todos os locais; por toda parte}
  \end{Phonetics}
\end{Entry}

\begin{Entry}{到达}{8,6}{⼑、⾡}
  \begin{Phonetics}{到达}{dao4da2}[][HSK 3]
    \definition{v.}{chegar (a um determinado local, a uma determinada fase); alcançar}
  \end{Phonetics}
\end{Entry}

\begin{Entry}{到来}{8,7}{⼑、⽊}
  \begin{Phonetics}{到来}{dao4 lai2}[][HSK 5]
    \definition{v.}{chegar; chegar aqui de outro lugar}
  \end{Phonetics}
\end{Entry}

\begin{Entry}{到底}{8,8}{⼑、⼴}
  \begin{Phonetics}{到底}{dao4di3}[][HSK 3]
    \definition{adv.}{na terra (usado em frases interrogativas para expressar a determinação de alguém em encontrar uma resposta definitiva) | afinal | finalmente; por fim; no fim; indica uma situação que finalmente se concretizou após várias mudanças ou reviravoltas}
  \end{Phonetics}
\end{Entry}

\begin{Entry}{到期}{8,12}{⼑、⽉}
  \begin{Phonetics}{到期}{dao4/qi1}[][HSK 6]
    \definition{v.+compl.}{expirar; amadurecer; tornar-se devido; tornar-se devido}
  \end{Phonetics}
\end{Entry}

\begin{Entry}{制}{8}{⼑}
  \begin{Phonetics}{制}{zhi4}
    \definition[套,项]{s.}{sistema | regras; regulamentos}
    \definition{v.}{formular; elaborar | fazer; fabricar | restringir; limitar; controlar; disciplinar}
  \end{Phonetics}
\end{Entry}

\begin{Entry}{制订}{8,4}{⼑、⾔}
  \begin{Phonetics}{制订}{zhi4 ding4}[][HSK 4]
    \definition{v.}{esboçar; formular; elaborar; mapear}
  \end{Phonetics}
\end{Entry}

\begin{Entry}{制成}{8,6}{⼑、⼽}
  \begin{Phonetics}{制成}{zhi4 cheng2}[][HSK 5]
    \definition{v.}{fabricar; ser feito de; produzir}
  \end{Phonetics}
\end{Entry}

\begin{Entry}{制约}{8,6}{⼑、⽷}
  \begin{Phonetics}{制约}{zhi4yue1}[][HSK 5]
    \definition{v.}{limitar; verificar; restringir; a existência e a mudança de uma coisa determinam a existência e a mudança de outra coisa}
  \end{Phonetics}
\end{Entry}

\begin{Entry}{制作}{8,7}{⼑、⼈}
  \begin{Phonetics}{制作}{zhi4zuo4}[][HSK 3]
    \definition{v.}{fazer; produzir; itens feitos com matérias-primas, geralmente pequenos e feitos à mão | fazer; produzir; criar gráficos, anúncios, filmes, jogos, etc., utilizando texto, imagens, sons, imagens, etc.}
  \end{Phonetics}
\end{Entry}

\begin{Entry}{制定}{8,8}{⼑、⼧}
  \begin{Phonetics}{制定}{zhi4ding4}[][HSK 3]
    \definition{v.}{rascunhar; formular; elaborar; estabelecer (leis, regulamentos, planos, etc.)}
  \end{Phonetics}
\end{Entry}

\begin{Entry}{制度}{8,9}{⼑、⼴}
  \begin{Phonetics}{制度}{zhi4du4}[][HSK 3]
    \definition[项,条,套,种]{s.}{regulamentação; regulamento; procedimentos operacionais ou diretrizes de conduta que todos devem seguir | sistema; o sistema político, econômico e cultural formado sob determinadas condições históricas}
  \end{Phonetics}
\end{Entry}

\begin{Entry}{制造}{8,10}{⼑、⾡}
  \begin{Phonetics}{制造}{zhi4zao4}[][HSK 3]
    \definition{v.}{fazer; produzir; manufaturar; transformar matérias-primas em produtos acabados | criar; agitar; criar artificialmente uma situação ou atmosfera desfavorável}
  \end{Phonetics}
\end{Entry}

\begin{Entry}{制裁}{8,12}{⼑、⾐}
  \begin{Phonetics}{制裁}{zhi4cai2}
    \definition{s.}{punição | sanção (inclusive econômica)}
    \definition{v.}{punir}
  \end{Phonetics}
\end{Entry}

\begin{Entry}{刷}{8}{⼑}
  \begin{Phonetics}{刷}{shua1}[][HSK 4]
    \definition{s.}{escova; pincel | (onomatopéia) farfalhar; descreve o som de uma passagem rápida}
    \definition{v.}{escovar; esfregar; remover com uma escova | borrar; colar; aplicar com um pincel | eliminar; remover; limpar | rolar; navegar; visualizar grandes quantidades de informações muito rapidamente em um curto período de tempo online ou em dispositivos móveis | deslizar (passar o cartão magnético)}
  \end{Phonetics}
  \begin{Phonetics}{刷}{shua4}
    \definition{adj.}{pálido ou branco-azulado}
    \definition{adv.}{bastante; completamente; extremamente; descreve movimentos ágeis}
  \end{Phonetics}
\end{Entry}

\begin{Entry}{刷子}{8,3}{⼑、⼦}
  \begin{Phonetics}{刷子}{shua1zi5}[][HSK 4]
    \definition[把,个]{s.}{escova; escovão; utensílio feito de lã, fio de plástico, fio de metal, etc., para remover sujeira ou aplicar óleo de unção, etc., geralmente longo ou oval, alguns com alças}
  \end{Phonetics}
\end{Entry}

\begin{Entry}{刷牙}{8,4}{⼑、⽛}
  \begin{Phonetics}{刷牙}{shua1ya2}[][HSK 4]
    \definition{s.}{escovar os dentes}
  \end{Phonetics}
\end{Entry}

\begin{Entry}{券}{8}{⼑}
  \begin{Phonetics}{券}{quan4}[][HSK 6]
    \definition[张]{s.}{certificado; bilhete; ingresso; uma conta ou pedaço de papel que serve como recibo}
  \end{Phonetics}
\end{Entry}

\begin{Entry}{刹}{8}{⼑}
  \begin{Phonetics}{刹}{cha4}
    \definition*{s.}{abreviação de Kshatara, 刹多罗, sânscrito ``ksetra''}
    \definition{s.}{mosteiro, templo ou santuário budista}
  \seealsoref{刹多罗}{sha1duo1luo2}
  \end{Phonetics}
  \begin{Phonetics}{刹}{sha1}
    \definition{v.}{acionar o(s) freio(s); frear; brecar}
  \end{Phonetics}
\end{Entry}

\begin{Entry}{刹多罗}{8,6,8}{⼑、⼣、⽹}
  \begin{Phonetics}{刹多罗}{sha1duo1luo2}
    \definition*{s.}{Kshatara, sânscrito ``ksetra''}
  \end{Phonetics}
\end{Entry}

\begin{Entry}{刺}{8}{⼑}
  \begin{Phonetics}{刺}{ci1}
    \definition{s.}{(onomatopéia) som de rasgo, fricção, etc.}
  \end{Phonetics}
  \begin{Phonetics}{刺}{ci4}[][HSK 4]
    \definition*{s.}{Sobrenome Ci}
    \definition{s.}{espinho; farpa; algo afiado como uma agulha | cartão de visita | saliências; projeções pequenas e pontiagudas na superfície de um objeto ou na pele de uma pessoa}
    \definition{v.}{esfaquear; perfurar | irritar; estimular | assassinar | espionar; detectar | criticar}
  \end{Phonetics}
\end{Entry}

\begin{Entry}{刺耳}{8,6}{⼑、⽿}
  \begin{Phonetics}{刺耳}{ci4'er3}[][HSK 7-9]
    \definition{adj.}{irritante (desagradável) para o ouvido; estridente; penetrante; áspero}
  \end{Phonetics}
\end{Entry}

\begin{Entry}{刺骨}{8,9}{⼑、⾻}
  \begin{Phonetics}{刺骨}{ci4gu3}[][HSK 7-9]
    \definition{adj.}{perfurante (até os ossos); cortante}
  \end{Phonetics}
\end{Entry}

\begin{Entry}{刺绣}{8,10}{⼑、⽷}
  \begin{Phonetics}{刺绣}{ci4xiu4}[][HSK 7-9]
    \definition{s.}{bordado; um artesanato popular tradicional que usa fios de seda coloridos para bordar padrões ou imagens em tecidos; produtos bordados}
    \definition{v.}{bordar; bordar padrões ou imagens em tecido usando fios de seda coloridos}
  \end{Phonetics}
\end{Entry}

\begin{Entry}{刺猬}{8,12}{⼑、⽝}
  \begin{Phonetics}{刺猬}{ci4wei5}
    \definition{s.}{porco-espinho | ouriço}
  \end{Phonetics}
\end{Entry}

\begin{Entry}{刺激}{8,16}{⼑、⽔}
  \begin{Phonetics}{刺激}{ci4ji1}[][HSK 4]
    \definition{adj.}{animado; entusiasmado; sensação de empolgação e nervosismo}
    \definition[个]{s.}{estímulo; estimulação; fortes efeitos físicos ou psicológicos}
    \definition{v.}{irritar; provocar; estimular | incentivar; estimular; incitar; (por algum meio) para mudar as coisas para melhor, para fazer coisas positivas}
  \end{Phonetics}
\end{Entry}

\begin{Entry}{刻}{8}{⼑}
  \begin{Phonetics}{刻}{ke4}[][HSK 2,5]
    \definition{adj.}{cruel; severo; rude; indelicado | no mais alto grau}
    \definition{clas.}{um quarto (de uma hora, 15min)}
    \definition[件]{s.}{quarto (de hora); momento}
    \definition{v.}{esculpir; inscrever; gravar; talhar com uma faca (padrões, texto, etc.) | definir um limite de tempo | imprimir (CD)}
  \end{Phonetics}
\end{Entry}

\begin{Entry}{刻画}{8,8}{⼑、⽥}
  \begin{Phonetics}{刻画}{ke4hua4}
    \definition{v.}{retratar | tirar um retrato}
  \end{Phonetics}
\end{Entry}

\begin{Entry}{刻钟}{8,9}{⼑、⾦}
  \begin{Phonetics}{刻钟}{ke4 zhong1}
    \definition{s.}{um quarto de hora}
  \end{Phonetics}
\end{Entry}

\begin{Entry}{势}{8}{⼒}
  \begin{Phonetics}{势}{shi4}
    \definition{s.}{poder; força; influência | momentum; tendência | aparência externa de um objeto natural; fenômenos ou situações naturais | situação; estado de coisas; circunstâncias | sinal; gesto | genitais masculinos}
  \end{Phonetics}
\end{Entry}

\begin{Entry}{势力}{8,2}{⼒、⼒}
  \begin{Phonetics}{势力}{shi4li4}[][HSK 5]
    \definition[股]{s.}{força; poder; influência; forças políticas, econômicas, militares, etc.}
  \end{Phonetics}
\end{Entry}

\begin{Entry}{卑}{8}{⼗}
  \begin{Phonetics}{卑}{bei1}
    \definition{adj.}{Literário: baixo | inferior; médio | Literário: modesto; humilde}
  \end{Phonetics}
\end{Entry}

\begin{Entry}{卑鄙}{8,13}{⼗、⾢}
  \begin{Phonetics}{卑鄙}{bei1bi3}[][HSK 7-9]
    \definition{adj.}{ruim; vulgar; vil; desprezível}
  \end{Phonetics}
\end{Entry}

\begin{Entry}{单}{8}{⼗}
  \begin{Phonetics}{单}{chan2}
    \definition{s.}{usado em 单于 \dpy{chan2yu2}}
  \seealsoref{单于}{chan2yu2}
  \end{Phonetics}
  \begin{Phonetics}{单}{dan1}[][HSK 4]
    \definition*{s.}{Sobrenome Dan}
    \definition{adj.}{sozinho; único | ímpar; número ímpar (oposto a 双) | simples; poucos projetos e tipos; estrutura e ideias simples | fino; fraco; frágil}
    \definition{adv.}{isoladamente; sozinho; indica que uma ação ou coisa está dentro de um escopo limitado e não é combinada com outras; equivale a 只 ou 仅}
    \definition[个]{s.}{lençol; um único pedaço grande de pano usado para cobrir | conta; lista; pedaços de papel para anotações detalhadas (geralmente folhas soltas)}
  \seealsoref{仅}{jin3}
  \seealsoref{双}{shuang1}
  \seealsoref{只}{zhi3}
  \end{Phonetics}
  \begin{Phonetics}{单}{shan4}
    \definition*{s.}{Sobrenome Shan}
    \definition{s.}{material de tecido de largura simples (dupla) | número singular (plural)}
  \end{Phonetics}
\end{Entry}

\begin{Entry}{单一}{8,1}{⼗、⼀}
  \begin{Phonetics}{单一}{dan1 yi1}[][HSK 5]
    \definition{adj.}{único; unitário; exclusivo}
  \end{Phonetics}
\end{Entry}

\begin{Entry}{单于}{8,3}{⼗、⼆}
  \begin{Phonetics}{单于}{chan2yu2}
    \definition{s.}{rei de Xiongnu (匈奴)}
  \seealsoref{匈奴}{xiong1nu2}
  \end{Phonetics}
\end{Entry}

\begin{Entry}{单元}{8,4}{⼗、⼉}
  \begin{Phonetics}{单元}{dan1yuan2}[][HSK 3]
    \definition[个,组,套]{s.}{unidade (de algo); um conjunto completo, com parágrafos e sistemas próprios, que forma uma unidade independente}
  \end{Phonetics}
\end{Entry}

\begin{Entry}{单方面}{8,4,9}{⼗、⽅、⾯}
  \begin{Phonetics}{单方面}{dan1fang1mian4}[][HSK 7-9]
    \definition{adj.}{unilateral}
    \definition{adv.}{unilateralmente}
  \end{Phonetics}
\end{Entry}

\begin{Entry}{单打}{8,5}{⼗、⼿}
  \begin{Phonetics}{单打}{dan1 da3}[][HSK 6]
    \definition[场,局,次]{s.}{Esporte: simples; competição um contra um}
  \end{Phonetics}
\end{Entry}

\begin{Entry}{单边}{8,5}{⼗、⾡}
  \begin{Phonetics}{单边}{dan1bian1}[][HSK 7-9]
    \definition{adj.}{unilateral}
  \end{Phonetics}
\end{Entry}

\begin{Entry}{单位}{8,7}{⼗、⼈}
  \begin{Phonetics}{单位}{dan1wei4}[][HSK 2]
    \definition[个,家]{s.}{unidade (como padrão de medida) | unidade (como uma organização, departamento, divisão, seção, etc.) | unidade (grupo de pessoas como um todo) | unidade de trabalho (local de trabalho, especialmente na República Popular da China antes da reforma econômica)}
  \end{Phonetics}
\end{Entry}

\begin{Entry}{单纯}{8,7}{⼗、⽷}
  \begin{Phonetics}{单纯}{dan1chun2}[][HSK 4]
    \definition{adj.}{puro; simples; descomplicado}
    \definition{adv.}{sozinho; puramente; meramente}
  \end{Phonetics}
\end{Entry}

\begin{Entry}{单身}{8,7}{⼗、⾝}
  \begin{Phonetics}{单身}{dan1shen1}[][HSK 7-9]
    \definition{s.}{solteiro}
  \end{Phonetics}
\end{Entry}

\begin{Entry}{单质}{8,8}{⼗、⾙}
  \begin{Phonetics}{单质}{dan1zhi4}
    \definition{s.}{substância simples (consistindo puramente de um elemento, como diamante, grafite, etc.)}
  \end{Phonetics}
\end{Entry}

\begin{Entry}{单独}{8,9}{⼗、⽝}
  \begin{Phonetics}{单独}{dan1du2}[][HSK 4]
    \definition{adv.}{solo; sozinho; por si mesmo; por conta própria}
  \end{Phonetics}
\end{Entry}

\begin{Entry}{单调}{8,10}{⼗、⾔}
  \begin{Phonetics}{单调}{dan1diao4}[][HSK 4]
    \definition{adj.}{maçante; monótono}
  \end{Phonetics}
\end{Entry}

\begin{Entry}{单脚滑行车}{8,11,12,6,4}{⼗、⾁、⽔、⾏、⾞}
  \begin{Phonetics}{单脚滑行车}{dan1jiao3hua2xing2che1}
    \definition{s.}{\emph{scooter}}
  \end{Phonetics}
\end{Entry}

\begin{Entry}{单薄}{8,16}{⼗、⾋}
  \begin{Phonetics}{单薄}{dan1bo2}[][HSK 7-9]
    \definition{adj.}{fino; pouco | frágil; magro e fraco | fino; frágil; insubstancial}
  \end{Phonetics}
\end{Entry}

\begin{Entry}{卖}{8}{⼗}
  \begin{Phonetics}{卖}{mai4}[][HSK 2]
    \definition*{s.}{Sobrenome Mai}
    \definition{clas.}{um prato (nos tempos antigos); antigamente, os restaurantes chamavam cada prato vendido de 一卖 (uma porção)}
    \definition{v.}{vender (oposto de 买) | trair (o próprio país ou amigos); alcançar objetivos pessoais à custa dos interesses do país, da nação e dos outros | não poupar esforços; esforçar-se ao máximo; tentar fazer o máximo possível | mostrar-se intencionalmente; exibir-se | vender o próprio trabalho; trabalhar em troca de dinheiro}
  \seealsoref{买}{mai3}
  \end{Phonetics}
\end{Entry}

\begin{Entry}{卧}{8}{⾂}
  \begin{Phonetics}{卧}{wo4}
    \definition{adj.}{para dormir}
    \definition{s.}{vagão-leito (ou carruagem); leito | beliche | quarto | Dialeto: pochê (ovos)}
    \definition{v.}{deitar | Dialeto: deitar um bebê | (animais ou pássaros) agachar-se; sentar-se; empoleirar-se | Figurativo: viver em reclusão}
  \end{Phonetics}
\end{Entry}

\begin{Entry}{卧车}{8,4}{⾂、⾞}
  \begin{Phonetics}{卧车}{wo4che1}
    \definition{s.}{um carro-leito | vagão-leito}
  \end{Phonetics}
\end{Entry}

\begin{Entry}{卧式}{8,6}{⾂、⼷}
  \begin{Phonetics}{卧式}{wo4shi4}
    \definition{adj.}{horizontal}
  \end{Phonetics}
\end{Entry}

\begin{Entry}{卧床}{8,7}{⾂、⼴}
  \begin{Phonetics}{卧床}{wo4chuang2}
    \definition{adj.}{acamado}
    \definition{s.}{cama}
    \definition{v.}{deitar na cama}
  \end{Phonetics}
\end{Entry}

\begin{Entry}{卧室}{8,9}{⾂、⼧}
  \begin{Phonetics}{卧室}{wo4shi4}[][HSK 5]
    \definition[间,个]{s.}{quarto de dormir; quarto de uma casa usado para dormir}
  \end{Phonetics}
\end{Entry}

\begin{Entry}{卧倒}{8,10}{⾂、⼈}
  \begin{Phonetics}{卧倒}{wo4dao3}
    \definition{v.}{cair no chão | deitar-se}
  \end{Phonetics}
\end{Entry}

\begin{Entry}{卧病}{8,10}{⾂、⽧}
  \begin{Phonetics}{卧病}{wo4bing4}
    \definition{s.}{acamado | doente na cama}
  \end{Phonetics}
\end{Entry}

\begin{Entry}{卧舱}{8,10}{⾂、⾈}
  \begin{Phonetics}{卧舱}{wo4cang1}
    \definition{s.}{cabine de dormir em um barco ou trem}
  \end{Phonetics}
\end{Entry}

\begin{Entry}{卧推}{8,11}{⾂、⼿}
  \begin{Phonetics}{卧推}{wo4tui1}
    \definition{s.}{supino}
  \end{Phonetics}
\end{Entry}

\begin{Entry}{卧铺}{8,12}{⾂、⾦}
  \begin{Phonetics}{卧铺}{wo4 pu4}[][HSK 6]
    \definition[个,排]{s.}{beliche para dormir; um beliche em um trem ou ônibus de longa distância}
  \end{Phonetics}
\end{Entry}

\begin{Entry}{卧榻}{8,14}{⾂、⽊}
  \begin{Phonetics}{卧榻}{wo4ta4}
    \definition{s.}{um sofá | uma cama estreita}
  \end{Phonetics}
\end{Entry}

\begin{Entry}{卷}{8}{⼙}
  \begin{Phonetics}{卷}{juan3}[][HSK 4]
    \definition{clas.}{usado para pequenas coisas enroladas (maço de papel dinheiro, carretel de filme, etc.) | usado para rolos, carretéis, bobinas, etc.}
    \definition[张]{s.}{rolo; carretel; bobina}
    \definition{v.}{enrolar; dobrar algo em um cilindro ou semicírculo | varrer; carregar; levar junto | envolver-se; participar}
  \end{Phonetics}
  \begin{Phonetics}{卷}{juan4}[][HSK 4]
    \definition{clas.}{usado para capítulos, seções ou volumes; fascículos}
    \definition[大,小]{s.}{livro; livros e pinturas que são enrolados para coleção; geralmente se refere a pinturas e caligrafia | papel de exame | arquivo; dossiê}
  \end{Phonetics}
\end{Entry}

\begin{Entry}{厕}{8}{⼚}
  \begin{Phonetics}{厕}{ce4}
    \definition[个,间]{s.}{latrina; fossa sanitária; (componente formador de palavras)}
  \seealsoref{茅厕}{mao2ce4}
  \end{Phonetics}
  \begin{Phonetics}{厕}{si5}
    \definition{s.}{componente formador de palavras | latrina; fossa sanitária}
  \seealsoref{茅厕}{mao2ce4}
  \end{Phonetics}
\end{Entry}

\begin{Entry}{厕纸}{8,7}{⼚、⽷}
  \begin{Phonetics}{厕纸}{ce4zhi3}
    \definition{s.}{papel higiênico}
  \end{Phonetics}
\end{Entry}

\begin{Entry}{厕所}{8,8}{⼚、⼾}
  \begin{Phonetics}{厕所}{ce4suo3}[][HSK 6]
    \definition[个,间]{s.}{banheiro; lavatório; sanitário; latrina; um lugar para as pessoas urinarem e defecarem}
  \end{Phonetics}
\end{Entry}

\begin{Entry}{参}{8}{⼛}
  \begin{Phonetics}{参}{can1}
    \definition{v.}{juntar-se; entrar; tomar parte em; participar | referir; consultar; comparar com outros materiais | ligar para prestar homenagem a; fazer uma visita |  (significado antigo) acusar um funcionário perante o imperador; relatar ou expor ao imperador | explorar e compreender (verdade, significado, etc.)}
  \end{Phonetics}
\end{Entry}

\begin{Entry}{参与}{8,3}{⼛、⼀}
  \begin{Phonetics}{参与}{can1yu4}[][HSK 4]
    \definition{v.}{participar de; tomar parte em; ter uma mão em; envolver-se em; participar (no planejamento, discussão e condução dos assuntos)}
  \end{Phonetics}
\end{Entry}

\begin{Entry}{参见}{8,4}{⼛、⾒}
  \begin{Phonetics}{参见}{can1jian4}[][HSK 7-9]
    \definition{v.}{(em referências) ver; ver também | consultar; visualizar | ler algo para referência | prestar homenagem a (um superior, etc.)}
  \end{Phonetics}
\end{Entry}

\begin{Entry}{参加}{8,5}{⼛、⼒}
  \begin{Phonetics}{参加}{can1jia1}[][HSK 2]
    \definition{v.}{aderir (a organizações); participar; participar (de atividades); participar de alguma organização ou atividade | dar (conselho, sugestão, etc.)}
  \end{Phonetics}
\end{Entry}

\begin{Entry}{参军}{8,6}{⼛、⼍}
  \begin{Phonetics}{参军}{can1/jun1}[][HSK 7-9]
    \definition{s.}{oficial do estado-maior militar; título oficial antigo}
    \definition{v.+compl.}{juntar-se ao exército; alistar-se}
  \end{Phonetics}
\end{Entry}

\begin{Entry}{参考}{8,6}{⼛、⽼}
  \begin{Phonetics}{参考}{can1kao3}[][HSK 4]
    \definition{v.}{consultar; referir-se a; acessar informações relevantes para estudo ou pesquisa | consultar; referir-se a; lidar com coisas, observar, ler, aprender e usar materiais relevantes}
  \end{Phonetics}
\end{Entry}

\begin{Entry}{参观}{8,6}{⼛、⾒}
  \begin{Phonetics}{参观}{can1guan1}[][HSK 2]
    \definition{v.}{visitar; dar uma olhada; observação no local (resultados do trabalho, carreira, instalações, locais históricos e pontos turísticos, etc.)}
  \end{Phonetics}
\end{Entry}

\begin{Entry}{参展}{8,10}{⼛、⼫}
  \begin{Phonetics}{参展}{can1 zhan3}[][HSK 6]
    \definition{v.}{expor ou participar de uma feira comercial, etc.}
  \end{Phonetics}
\end{Entry}

\begin{Entry}{参谋}{8,11}{⼛、⾔}
  \begin{Phonetics}{参谋}{can1mou2}[][HSK 7-9]
    \definition{s.}{oficial de estado-maior; pessoal militar envolvido em planejamento militar e outros assuntos |conselheiro; consultor}
    \definition{v.}{aconselhar; dar conselhos}
  \end{Phonetics}
\end{Entry}

\begin{Entry}{参照}{8,13}{⼛、⽕}
  \begin{Phonetics}{参照}{can1zhao4}[][HSK 7-9]
    \definition{v.}{consultar; referir-se a; referir-se e imitar (métodos, experiências, etc.)}
  \end{Phonetics}
\end{Entry}

\begin{Entry}{参赛}{8,14}{⼛、⾙}
  \begin{Phonetics}{参赛}{can1 sai4}[][HSK 6]
    \definition{v.}{participar de uma partida (ou competição); competir}
  \end{Phonetics}
\end{Entry}

\begin{Entry}{叔}{8}{⼜}
  \begin{Phonetics}{叔}{shu1}
    \definition*{s.}{Sobrenome Shu}
    \definition{s.}{irmão mais novo do pai; tio (por parte de pai)| irmão mais novo do marido | terceiro entre irmãos | tio | uma forma de tratamento para um homem um pouco mais jovem que o pai; tio | terceiro tio (de quatro irmãos) | primo mais novo da mãe}
  \end{Phonetics}
\end{Entry}

\begin{Entry}{叔叔}{8,8}{⼜、⼜}
  \begin{Phonetics}{叔叔}{shu1shu5}
    \definition[个,位,名]{s.}{tio; irmão mais novo do pai | tio, dirigindo-se a um homem da mesma geração que o pai e mais jovem em idade}
  \end{Phonetics}
\end{Entry}

\begin{Entry}{取}{8}{⼜}
  \begin{Phonetics}{取}{qu3}[][HSK 2]
    \definition{v.}{pegar; obter; buscar; pegar de um lugar; pegar nas mãos | visar; procurar; obter; provocar | adotar; assumir; escolher; selecionar}
  \end{Phonetics}
\end{Entry}

\begin{Entry}{取水}{8,4}{⼜、⽔}
  \begin{Phonetics}{取水}{qu3shui3}
    \definition{v.}{obter água (de um poço, etc.)}
  \end{Phonetics}
\end{Entry}

\begin{Entry}{取现}{8,8}{⼜、⾒}
  \begin{Phonetics}{取现}{qu3xian4}
    \definition{v.}{sacar dinheiro}
  \end{Phonetics}
\end{Entry}

\begin{Entry}{取胜}{8,9}{⼜、⾁}
  \begin{Phonetics}{取胜}{qu3sheng4}
    \definition{v.}{prevalecer sobre os oponentes | marcar uma vitória}
  \end{Phonetics}
\end{Entry}

\begin{Entry}{取悦}{8,10}{⼜、⼼}
  \begin{Phonetics}{取悦}{qu3yue4}
    \definition{v.}{tentar agradar}
  \end{Phonetics}
\end{Entry}

\begin{Entry}{取消}{8,10}{⼜、⽔}
  \begin{Phonetics}{取消}{qu3xiao1}[][HSK 3]
    \definition{v.}{cancelar; suspender; anular; abolir; revogar; rescindir; tornar o sistema original, regulamentos, qualificações, direitos, etc. inválidos}
  \end{Phonetics}
\end{Entry}

\begin{Entry}{取得}{8,11}{⼜、⼻}
  \begin{Phonetics}{取得}{qu3 de2}[][HSK 2]
    \definition{v.}{ganhar; adquirir; obter; ser o primeiro a conseguir}
  \end{Phonetics}
\end{Entry}

\begin{Entry}{取款}{8,12}{⼜、⽋}
  \begin{Phonetics}{取款}{qu3kuan3}[][HSK 6]
    \definition{v.}{sacar dinheiro (de um banco); retirar o dinheiro que você depositou (geralmente se refere a retirar dinheiro do banco)}
  \end{Phonetics}
\end{Entry}

\begin{Entry}{取款机}{8,12,6}{⼜、⽋、⽊}
  \begin{Phonetics}{取款机}{qu3 kuan3 ji1}[][HSK 6]
    \definition{s.}{ATM; caixa eletrônico; um caixa eletrônico é uma máquina que pode concluir automaticamente operações bancárias, como saques e consultas de saldo}
  \end{Phonetics}
\end{Entry}

\begin{Entry}{受}{8}{⼜}
  \begin{Phonetics}{受}{shou4}[][HSK 3]
    \definition{v.}{receber; aceitar | sofrer; ser submetido a | aguentar; suportar; tolerar | ser agradável}
  \end{Phonetics}
\end{Entry}

\begin{Entry}{受不了}{8,4,2}{⼜、⼀、⼅}
  \begin{Phonetics}{受不了}{shou4bu5liao3}[][HSK 4]
    \definition{v.}{ser insuportável; não poder suportar algo; não suportar algo}
  \end{Phonetics}
\end{Entry}

\begin{Entry}{受伤}{8,6}{⼜、⼈}
  \begin{Phonetics}{受伤}{shou4shang1}[][HSK 3]
    \definition{v.}{ser ferido; sofrer uma lesão}
  \end{Phonetics}
\end{Entry}

\begin{Entry}{受灾}{8,7}{⼜、⽕}
  \begin{Phonetics}{受灾}{shou4 zai1}[][HSK 5]
    \definition{v.}{ser atingido por um desastre natural (ou calamidade) | ser atingido por uma adversidade natural}
  \end{Phonetics}
\end{Entry}

\begin{Entry}{受到}{8,8}{⼜、⼑}
  \begin{Phonetics}{受到}{shou4dao4}[][HSK 2]
    \definition{v.}{receber; receber itens, mensagens, instruções, etc. fornecidos por outras pessoas}
  \end{Phonetics}
\end{Entry}

\begin{Entry}{受限}{8,8}{⼜、⾩}
  \begin{Phonetics}{受限}{shou4xian4}
    \definition{v.}{ser limitado | ser restrito | ser constrangido}
  \end{Phonetics}
\end{Entry}

\begin{Entry}{受得了}{8,11,2}{⼜、⼻、⼅}
  \begin{Phonetics}{受得了}{shou4de5liao3}
    \definition{v.}{suportar | aguentar}
  \end{Phonetics}
\end{Entry}

\begin{Entry}{变}{8}{⼜}
  \begin{Phonetics}{变}{bian4}[][HSK 2]
    \definition{adj.}{alterado; mutável; que pode mudar; que está mudando ou já mudou}
    \definition{s.}{uma reviravolta inesperada nos acontecimentos; mudanças significativas repentinas}
    \definition{v.}{mudar; tornar-se diferente; fazer mudanças | tornar-se; transformar-se; natureza, estado ou situação diferentes dos originais | alterar; mudar; transformar}
  \end{Phonetics}
\end{Entry}

\begin{Entry}{变为}{8,4}{⼜、⼂}
  \begin{Phonetics}{变为}{bian4 wei2}[][HSK 3]
    \definition{v.}{transformar-se em; tornar-se | mudar para}
  \end{Phonetics}
\end{Entry}

\begin{Entry}{变化}{8,4}{⼜、⼔}
  \begin{Phonetics}{变化}{bian4hua4}[][HSK 3]
    \definition[个]{s.}{mudança; variação; a nova situação após uma mudança em pessoas ou coisas}
    \definition{v.}{mudar;  variar}
  \end{Phonetics}
\end{Entry}

\begin{Entry}{变幻莫测}{8,4,10,9}{⼜、⼳、⾋、⽔}
  \begin{Phonetics}{变幻莫测}{bian4huan4-mo4ce4}[][HSK 7-9]
    \definition{expr.}{mutável; imprevisível | errático | mudar imprevisivelmente | traiçoeiro}
  \end{Phonetics}
\end{Entry}

\begin{Entry}{变心}{8,4}{⼜、⼼}
  \begin{Phonetics}{变心}{bian4/xin1}
    \definition{v.+compl.}{deixar de ser fiel}
  \end{Phonetics}
\end{Entry}

\begin{Entry}{变节}{8,5}{⼜、⾋}
  \begin{Phonetics}{变节}{bian4jie2}
    \definition{s.}{traição | deserção | vira-casaca}
    \definition{v.}{retratar-se e submeter-se; renunciar e render-se | mudar de lado politicamente}
  \end{Phonetics}
\end{Entry}

\begin{Entry}{变动}{8,6}{⼜、⼒}
  \begin{Phonetics}{变动}{bian4 dong4}[][HSK 5]
    \definition{s.}{mudança; alteração; oscilação; modificação; variação}
    \definition{v.}{mudar; alterar; oscilar; modificar; variar}
  \end{Phonetics}
\end{Entry}

\begin{Entry}{变异}{8,6}{⼜、⼶}
  \begin{Phonetics}{变异}{bian4yi4}[][HSK 7-9]
    \definition{s.}{variação; mutação; muta; diferenças nas características morfológicas e fisiológicas entre gerações da mesma espécie ou entre indivíduos da mesma geração}
    \definition{v.}{variar; mudar}
  \end{Phonetics}
\end{Entry}

\begin{Entry}{变成}{8,6}{⼜、⼽}
  \begin{Phonetics}{变成}{bian4 cheng2}[][HSK 2]
    \definition{v.}{crescer; tornar-se; fazer; desenvolver-se; revelar-se; resultar; acontecer; passar a ser; passar para; acumular-se; converter-se; transformar-se; transformar-se em; mudar-se em; transformação da situação ou condição anterior para a situação ou condição atual}
  \end{Phonetics}
\end{Entry}

\begin{Entry}{变迁}{8,6}{⼜、⾡}
  \begin{Phonetics}{变迁}{bian4qian1}[][HSK 7-9]
    \definition{s.}{mudanças; transição; vicissitudes; mudança em tendências ou condições; mudança de situação ou estágio}
  \end{Phonetics}
\end{Entry}

\begin{Entry}{变形}{8,7}{⼜、⼺}
  \begin{Phonetics}{变形}{bian4/xing2}[][HSK 6]
    \definition{v.+compl.}{deformar; ficar fora de forma | transformar; transformar-se em outras formas}
  \end{Phonetics}
\end{Entry}

\begin{Entry}{变更}{8,7}{⼜、⽈}
  \begin{Phonetics}{变更}{bian4 geng1}[][HSK 6]
    \definition{v.}{alterar; mudar; modificar}
  \end{Phonetics}
\end{Entry}

\begin{Entry}{变性}{8,8}{⼜、⼼}
  \begin{Phonetics}{变性}{bian4xing4}
    \definition{s.}{desnaturação | transexual}
    \definition{v.}{desnaturar | mudar de sexo}
  \end{Phonetics}
\end{Entry}

\begin{Entry}{变质}{8,8}{⼜、⾙}
  \begin{Phonetics}{变质}{bian4/zhi4}[][HSK 7-9]
    \definition{v.+compl.}{deteriorar-se; estragar-se | tornar-se moralmente degenerado}
  \end{Phonetics}
\end{Entry}

\begin{Entry}{变革}{8,9}{⼜、⾰}
  \begin{Phonetics}{变革}{bian4ge2}[][HSK 7-9]
    \definition{s.}{mudança; transformação; a natureza das coisas foi reformada}
    \definition{v.}{transformar; mudar (de sistemas sociais, políticas, etc.); mudar o antigo e inovar; mudar a essência das coisas (referindo-se principalmente aos sistemas sociais)}
  \end{Phonetics}
\end{Entry}

\begin{Entry}{变换}{8,10}{⼜、⼿}
  \begin{Phonetics}{变换}{bian4 huan4}[][HSK 6]
    \definition{v.}{variar; alternar; mudar a forma ou o conteúdo de algo de uma coisa para outra}
  \end{Phonetics}
\end{Entry}

\begin{Entry}{变装}{8,12}{⼜、⾐}
  \begin{Phonetics}{变装}{bian4zhuang1}
    \definition{v.}{trocar de roupa | vestir-se | vestir uma fantasia | disfarçar-se ou fantasiar-se de personagem real ou ficcional, \emph{cosplay} | travestir-se}
  \end{Phonetics}
\end{Entry}

\begin{Entry}{变数}{8,13}{⼜、⽁}
  \begin{Phonetics}{变数}{bian4shu4}
    \definition{s.}{(matemática) variável | fatores variáveis}
  \end{Phonetics}
\end{Entry}

\begin{Entry}{呢}{8}{⼝}
  \begin{Phonetics}{呢}{ne5}[][HSK 1]
    \definition{part.}{usada no final de frases interrogativas (especificamente perguntas, perguntas de escolha e perguntas retóricas) para indicar um tom interrogativo | usada no final de uma frase declarativa, indica que uma ação ou situação está em andamento | usada em frases para indicar uma pausa (muitas vezes em pares) | usada no final de uma frase declarativa para confirmar um fato e convencer o interlocutor (com um tom de indicação e exagero)}
  \end{Phonetics}
  \begin{Phonetics}{呢}{ni2}
    \definition{s.}{(tecido feito de) lã; tecido de lã (para roupas pesadas); tecido de lã pesada; revestimento ou roupa de lã}
  \end{Phonetics}
\end{Entry}

\begin{Entry}{周}{8}{⼝}
  \begin{Phonetics}{周}{zhou1}[][HSK 2]
    \definition*{s.}{Dinastia Zhou (1046-256 BC) | Dinastia Zhou do Norte (557-581), uma das Dinastias do Norte | Dinastia Zhou Posterior (951-960), uma das Cinco Dinastias | Sobrenome Zhou}
    \definition{adj.}{universal; inteiro; por toda parte | atencioso; pensativo; completo; minucioso}
    \definition{adv.}{semanalmente}
    \definition{clas.}{usado para rodadas, voltas}
    \definition{s.}{periferia; arredores; círculo | semana | ciclo}
    \definition{v.}{fazer um circuito; mover-se em um curso circular | ajudar alguém}
  \end{Phonetics}
\end{Entry}

\begin{Entry}{周末}{8,5}{⼝、⽊}
  \begin{Phonetics}{周末}{zhou1mo4}[][HSK 2]
    \definition[个]{s.}{final-de-semana}
  \end{Phonetics}
\end{Entry}

\begin{Entry}{周年}{8,6}{⼝、⼲}
  \begin{Phonetics}{周年}{zhou1nian2}[][HSK 2]
    \definition{s.}{aniversário}
  \end{Phonetics}
\end{Entry}

\begin{Entry}{周围}{8,7}{⼝、⼞}
  \begin{Phonetics}{周围}{zhou1wei2}[][HSK 3]
    \definition{s.}{ao redor; redondeza; vizinhança; a parte ao redor do centro}
  \end{Phonetics}
\end{Entry}

\begin{Entry}{周期}{8,12}{⼝、⽉}
  \begin{Phonetics}{周期}{zhou1qi1}[][HSK 5]
    \definition[个]{s.}{período; ciclo; no processo de mudança e movimento das coisas, certas características se repetem várias vezes, com um intervalo de tempo entre cada repetição | período; ciclo; refere-se a um processo em que certas características se repetem várias vezes, e o tempo decorrido entre duas ocorrências consecutivas | classificação dos elementos na tabela periódica}
  \end{Phonetics}
\end{Entry}

\begin{Entry}{味}{8}{⼝}
  \begin{Phonetics}{味}{wei4}
    \definition{clas.}{usado para ingredientes (de uma receita de medicina chinesa)}
    \definition{s.}{gosto; sabor | cheiro; odor | interesse; deleite | acepipe; \emph{delicacy} | significância; significado}
    \definition{v.}{distinguir (provar) o sabor de; saborear}
  \end{Phonetics}
\end{Entry}

\begin{Entry}{味儿}{8,2}{⼝、⼉}
  \begin{Phonetics}{味儿}{wei4r5}[][HSK 4]
    \definition{s.}{gosto; sabor; propriedade de uma substância que dá à língua uma determinada sensação de sabor | cheiro; odor; propriedade de uma substância que dá ao nariz um determinado sentido de cheiro | interesse; significado; deleite}
  \end{Phonetics}
\end{Entry}

\begin{Entry}{味道}{8,12}{⼝、⾡}
  \begin{Phonetics}{味道}{wei4dao5}[][HSK 2]
    \definition[个,股,种]{s.}{gosto; sabor | sensação; gosto; experiência | interesse; deleite | cheiro; odor}
  \end{Phonetics}
\end{Entry}

\begin{Entry}{呵}{8}{⼝}
  \begin{Phonetics}{呵}{a1}
    \variantof{啊}
  \end{Phonetics}
  \begin{Phonetics}{呵}{he1}
    \definition{interj.}{Meu Deus!| Ah!; Oh!}
    \definition{v.}{expirar (com a boca aberta) | repreender}
  \end{Phonetics}
\end{Entry}

\begin{Entry}{呶}{8}{⼝}
  \begin{Phonetics}{呶}{nao2}
    \definition{interj.}{(onomatopéia) ruído alto e contínuo}
    \definition{v.}{(literário) gritar; clamar; falar ruidosamente}
  \seealsoref{努}{nu3}
  \end{Phonetics}
\end{Entry}

\begin{Entry}{呼}{8}{⼝}
  \begin{Phonetics}{呼}{hu1}
    \definition*{s.}{Sobrenome Hu}
    \definition{s.}{Onomatopéia: descreve o som do vento}
    \definition{v.}{expirar | gritar; clamar | chamar; ligar; ligar para alguém}
  \end{Phonetics}
\end{Entry}

\begin{Entry}{呼吸}{8,6}{⼝、⼝}
  \begin{Phonetics}{呼吸}{hu1xi1}[][HSK 4]
    \definition{s.}{um suspiro; metáfora para um período de tempo muito curto}
    \definition{v.}{respirar}
  \end{Phonetics}
\end{Entry}

\begin{Entry}{呼啦啦}{8,11,11}{⼝、⼝、⼝}
  \begin{Phonetics}{呼啦啦}{hu1 la1 la1}
    \definition{s.}{Onomatopéia: som de bater asas}
  \end{Phonetics}
\end{Entry}

\begin{Entry}{呼啸}{8,11}{⼝、⼝}
  \begin{Phonetics}{呼啸}{hu1xiao4}
    \definition{v.}{assobiar}
  \end{Phonetics}
\end{Entry}

\begin{Entry}{命}{8}{⼝}
  \begin{Phonetics}{命}{ming4}[][HSK 6]
    \definition[条]{s.}{vida | sorte; destino; fado | ordem; comando; instrução | atribuição de um nome, título etc.}
    \definition{v.}{ordenar; nomear | atribuir (um nome etc.)}
  \end{Phonetics}
\end{Entry}

\begin{Entry}{命令}{8,5}{⼝、⼈}
  \begin{Phonetics}{命令}{ming4ling4}[][HSK 5]
    \definition[条,项,道,个]{s.}{ordem; comando; instruções emitidas pelos superiores aos subordinados}
    \definition{v.}{ordenar; comandar}
  \end{Phonetics}
\end{Entry}

\begin{Entry}{命运}{8,7}{⼝、⾡}
  \begin{Phonetics}{命运}{ming4yun4}[][HSK 3]
    \definition[个]{s.}{tendência de desenvolvimento; tendência de futuro; metáfora para a direção e tendência do desenvolvimento e das mudanças | destino; sina; sorte; refere-se à vida e à morte, à riqueza e à pobreza e a todas as experiências da vida}
  \end{Phonetics}
\end{Entry}

\begin{Entry}{和}{8}{⼝}
  \begin{Phonetics}{和}{he2}[][HSK 1]
    \definition*{s.}{Sobrenome He}
    \definition{adj.}{gentil; suave; amável | harmonioso; em boas condições}
    \definition{conj.}{e (somente para palavras); unidos com}
    \definition{prep.}{relacionado com | para; com; indica correlação; comparação, etc.}
    \definition{s.}{soma; soma total | japonês; refere-se ao Japão}
    \definition{v.}{disputar; reconciliar; acabar com a guerra ou a disputa | empatar; (próxima edição ou torneio) sem vencedor}
  \end{Phonetics}
  \begin{Phonetics}{和}{he4}
    \definition{v.}{compor um poema em resposta (ao poema de alguém) usando a mesma sequência de rimas | juntar-se à cantoria; cantar junto com outros em harmonia}
  \end{Phonetics}
  \begin{Phonetics}{和}{hu2}
    \definition{v.}{completar um conjunto de Mahjong, 麻将, ou cartas de baralho}
  \seealsoref{麻将}{ma2jiang4}
  \end{Phonetics}
  \begin{Phonetics}{和}{huo2}
    \definition{v.}{combinar uma substância em pó (farinha, gesso, etc.) com água; adicionar líquido ao pó e mexer ou amassar até ficar pegajoso ou espesso}
  \end{Phonetics}
  \begin{Phonetics}{和}{huo4}
    \definition{clas.}{usado para enxágues de roupas | usado para fervuras de ervas medicinais}
    \definition{v.}{misturar (ingredientes); misturar pós ou grãos; misturar com água para obter uma consistência mais líquida}
  \end{Phonetics}
\end{Entry}

\begin{Entry}{和平}{8,5}{⼝、⼲}
  \begin{Phonetics}{和平}{he2ping2}[][HSK 3]
    \definition{adj.}{pacífico; moderado; não violento | pacífico; tranquilo; sereno}
    \definition{s.}{paz ;ausência de guerra}
  \end{Phonetics}
\end{Entry}

\begin{Entry}{和平共处}{8,5,6,5}{⼝、⼲、⼋、⼡}
  \begin{Phonetics}{和平共处}{he2ping2gong4chu3}
    \definition{s.}{coexistência pacífica de nações, sociedades, etc.}
  \end{Phonetics}
\end{Entry}

\begin{Entry}{和谐}{8,11}{⼝、⾔}
  \begin{Phonetics}{和谐}{he2xie2}[][HSK 6]
    \definition{adj.}{harmonioso; sem conflitos | em perfeita harmonia; ajuste adequado e simétrico}
    \definition{v.}{(eufemismo) censurar}
  \end{Phonetics}
\end{Entry}

\begin{Entry}{咒}{8}{⼝}
  \begin{Phonetics}{咒}{zhou4}
    \definition[个,句]{s.}{encantamento; feitiço}
    \definition{v.}{amaldiçoar; condenar | maltratar; dizer que você espera que as pessoas não tenham sucesso}
  \end{Phonetics}
\end{Entry}

\begin{Entry}{咒骂}{8,9}{⼝、⾺}
  \begin{Phonetics}{咒骂}{zhou4ma4}
    \definition{v.}{xingar | amaldiçoar | execrar}
  \end{Phonetics}
\end{Entry}

\begin{Entry}{咖}{8}{⼝}
  \begin{Phonetics}{咖}{ka1}
    \definition[杯]{s.}{classe | café | graduação}
  \end{Phonetics}
\end{Entry}

\begin{Entry}{咖啡}{8,11}{⼝、⼝}
  \begin{Phonetics}{咖啡}{ka1fei1}[][HSK 3]
    \definition[杯,瓶,罐,壶,包,袋,盒]{s.}{(empréstimo linguístico) café}
  \end{Phonetics}
\end{Entry}

\begin{Entry}{咖啡色}{8,11,6}{⼝、⼝、⾊}
  \begin{Phonetics}{咖啡色}{ka1fei1 se4}
    \definition{s.}{cor café}
  \end{Phonetics}
\end{Entry}

\begin{Entry}{咖啡馆}{8,11,11}{⼝、⼝、⾷}
  \begin{Phonetics}{咖啡馆}{ka1fei1guan3}
    \definition[家]{s.}{cafeteria}
  \end{Phonetics}
\end{Entry}

\begin{Entry}{哎}{8}{⼝}
  \begin{Phonetics}{哎}{ai1}[][HSK 7-9]
    \definition{interj.}{Por que?; Ei!; Ai!; expressar surpresa ou insatisfação | Ei!; Cuidado!}
  \end{Phonetics}
\end{Entry}

\begin{Entry}{哎呀}{8,7}{⼝、⼝}
  \begin{Phonetics}{哎呀}{ai1ya1}[][HSK 7-9]
    \definition{interj.}{expressar surpresa ou espanto | expressar reclamação, impaciência, etc.}
  \end{Phonetics}
\end{Entry}

\begin{Entry}{固}{8}{⼞}
  \begin{Phonetics}{固}{gu4}
    \definition*{s.}{Sobrenome Gu}
    \definition{adj.}{sólido; firme; forte | duro; sólido | mal informado; superficial; ignorante}
    \definition{adv.}{firmemente; resolutamente | originalmente; em primeiro lugar | certamente; reconhecidamente; seguramente}
    \definition{conj.}{usado da mesma forma que 固然}
    \definition{v.}{solidificar; consolidar; fortalecer | defender; proteger}
  \seealsoref{固然}{gu4ran2}
  \end{Phonetics}
\end{Entry}

\begin{Entry}{固定}{8,8}{⼞、⼧}
  \begin{Phonetics}{固定}{gu4ding4}[][HSK 4]
    \definition{adj.}{fixo; regular; inalterado ou imóvel}
    \definition{v.}{consertar; tornar fixo, não mover novamente; colocar as coisas em ordem, não mudá-las novamente}
  \end{Phonetics}
\end{Entry}

\begin{Entry}{固然}{8,12}{⼞、⽕}
  \begin{Phonetics}{固然}{gu4ran2}
    \definition{conj.}{usado para introduzir uma cláusula adversativa admitindo primeiro um certo fato | admitir um fato sem negar outro}
  \end{Phonetics}
\end{Entry}

\begin{Entry}{国}{8}{⼞}
  \begin{Phonetics}{国}{guo2}[][HSK 1]
    \definition*{s.}{Sobrenome Guo}
    \definition{adj.}{nacional; do estado; representante do país | o melhor de um país}
    \definition[个]{s.}{estado; nação; país}
  \end{Phonetics}
\end{Entry}

\begin{Entry}{国人}{8,2}{⼞、⼈}
  \begin{Phonetics}{国人}{guo2ren2}
    \definition{s.}{compatriota}
  \end{Phonetics}
\end{Entry}

\begin{Entry}{国内}{8,4}{⼞、⼌}
  \begin{Phonetics}{国内}{guo2 nei4}[][HSK 3]
    \definition{s.}{interno (a um país); doméstico; lar; dentro de um determinado país}
  \end{Phonetics}
\end{Entry}

\begin{Entry}{国王}{8,4}{⼞、⽟}
  \begin{Phonetics}{国王}{guo2wang2}[][HSK 6]
    \definition[位,名,个,些]{s.}{rei; soberanos; o governante supremo de algumas monarquias antigas; nos tempos modernos, refere-se ao chefe de estado de algumas monarquias}
  \end{Phonetics}
\end{Entry}

\begin{Entry}{国外}{8,5}{⼞、⼣}
  \begin{Phonetics}{国外}{guo2 wai4}[][HSK 1]
    \definition{adj.}{externo; no exterior; fora do país; outros lugares fora do país; geralmente chamados de exterior;  exterior não é o mesmo que estrangeiro}
  \end{Phonetics}
\end{Entry}

\begin{Entry}{国民}{8,5}{⼞、⽒}
  \begin{Phonetics}{国民}{guo2 min2}[][HSK 5]
    \definition{adj.}{nacional}
    \definition[个]{s.}{membro de uma nação; povo de uma nação}
  \end{Phonetics}
\end{Entry}

\begin{Entry}{国产}{8,6}{⼞、⼇}
  \begin{Phonetics}{国产}{guo2 chan3}[][HSK 6]
    \definition{adj.}{doméstico; feito na China; produzido internamente, especificamente na China}
  \end{Phonetics}
\end{Entry}

\begin{Entry}{国会}{8,6}{⼞、⼈}
  \begin{Phonetics}{国会}{guo2 hui4}[][HSK 6]
    \definition{s.}{parlamento; congresso}
  \end{Phonetics}
\end{Entry}

\begin{Entry}{国庆}{8,6}{⼞、⼴}
  \begin{Phonetics}{国庆}{guo2 qing4}[][HSK 3]
    \definition*{s.}{Dia Nacional, o dia em que um país comemora sua independência ou fundação}
  \end{Phonetics}
\end{Entry}

\begin{Entry}{国庆节}{8,6,5}{⼞、⼴、⾋}
  \begin{Phonetics}{国庆节}{guo2qing4jie2}
    \definition*{s.}{Dia Nacional (1~de~outubro)}
  \end{Phonetics}
\end{Entry}

\begin{Entry}{国际}{8,7}{⼞、⾩}
  \begin{Phonetics}{国际}{guo2ji4}[][HSK 2]
    \definition{adj.}{internacional; entre países; entre nações}
    \definition{s.}{internacional; o mundo; entre nações; entre países de todo o mundo}
  \end{Phonetics}
\end{Entry}

\begin{Entry}{国际儿童节}{8,7,2,12,5}{⼞、⾩、⼉、⽴、⾋}
  \begin{Phonetics}{国际儿童节}{guo2ji4 er2tong2jie2}
    \definition*{s.}{Dia Internacional das Crianças (1~de~junho)}
  \end{Phonetics}
\end{Entry}

\begin{Entry}{国际妇女节}{8,7,6,3,5}{⼞、⾩、⼥、⼥、⾋}
  \begin{Phonetics}{国际妇女节}{guo2ji4 fu4nv3jie2}
    \definition*{s.}{Dia Internacional das Mulheres (8~de~março)}
  \end{Phonetics}
\end{Entry}

\begin{Entry}{国际劳动节}{8,7,7,6,5}{⼞、⾩、⼒、⼒、⾋}
  \begin{Phonetics}{国际劳动节}{guo2ji4 lao2dong4 jie2}
    \definition*{s.}{Dia Internacional dos Trabalhadores (1~de~maio)}
  \end{Phonetics}
\end{Entry}

\begin{Entry}{国语}{8,9}{⼞、⾔}
  \begin{Phonetics}{国语}{guo2yu3}
    \definition*{s.}{Língua Chinesa (Mandarim), enfatizando sua natureza nacional}
  \end{Phonetics}
\end{Entry}

\begin{Entry}{国家}{8,10}{⼞、⼧}
  \begin{Phonetics}{国家}{guo2jia1}[][HSK 1]
    \definition[个]{s.}{país; estado; nação; um lugar reconhecido internacionalmente e com soberania independente, incluindo as pessoas e as instituições administrativas desse lugar}
  \end{Phonetics}
\end{Entry}

\begin{Entry}{国宾馆}{8,10,11}{⼞、⼧、⾷}
  \begin{Phonetics}{国宾馆}{guo2bin1guan3}
    \definition{s.}{pousada estadual}
  \end{Phonetics}
\end{Entry}

\begin{Entry}{国旗}{8,14}{⼞、⽅}
  \begin{Phonetics}{国旗}{guo2qi2}
    \definition[面]{s.}{bandeira (de um país)}
  \end{Phonetics}
\end{Entry}

\begin{Entry}{国歌}{8,14}{⼞、⽋}
  \begin{Phonetics}{国歌}{guo2 ge1}[][HSK 6]
    \definition[首,支]{s.}{hino nacional; o hino nacional da China, oficialmente designado pelo estado como a música que representa o país, é "Marcha dos Voluntários"}
  \end{Phonetics}
\end{Entry}

\begin{Entry}{国籍}{8,20}{⼞、⽵}
  \begin{Phonetics}{国籍}{guo2ji2}[][HSK 5]
    \definition[个]{s.}{nacionalidade; cidadania; refere-se à identidade de um indivíduo como pertencente a um Estado | identidade nacional (de um avião, navio, etc.)}
  \end{Phonetics}
\end{Entry}

\begin{Entry}{图}{8}{⼞}
  \begin{Phonetics}{图}{tu2}[][HSK 3]
    \definition*{s.}{Sobrenome Tu}
    \definition[张]{s.}{mapa; gráfico; imagem; desenho | plano; esquema; tentativa}
    \definition{v.}{procurar; perseguir; esperar obter| desenhar; retratar; pintar | imaginar; planejar; pensar; maquinar}
  \end{Phonetics}
\end{Entry}

\begin{Entry}{图书}{8,4}{⼞、⼄}
  \begin{Phonetics}{图书}{tu2 shu1}[][HSK 6]
    \definition{s.}{livros; um termo geral para publicações como livros e álbuns de imagens}[这些图书都可以借阅。===Esses livros estão disponíveis para empréstimo.]
  \end{Phonetics}
\end{Entry}

\begin{Entry}{图书馆}{8,4,11}{⼞、⼄、⾷}
  \begin{Phonetics}{图书馆}{tu2shu1guan3}[][HSK 1]
    \definition[个,家]{s.}{biblioteca; instituição que coleta, organiza e armazena livros e materiais para leitura e consulta}
  \end{Phonetics}
\end{Entry}

\begin{Entry}{图片}{8,4}{⼞、⽚}
  \begin{Phonetics}{图片}{tu2 pian4}[][HSK 2]
    \definition[张,幅]{s.}{imagem; fotografia; um termo geral para imagens, fotografias, decalques, etc. usados para ilustrar algo}
  \end{Phonetics}
\end{Entry}

\begin{Entry}{图画}{8,8}{⼞、⽥}
  \begin{Phonetics}{图画}{tu2 hua4}[][HSK 3]
    \definition[幅,张,套]{s.}{desenho; imagem; pintura}
  \end{Phonetics}
\end{Entry}

\begin{Entry}{图案}{8,10}{⼞、⽊}
  \begin{Phonetics}{图案}{tu2'an4}[][HSK 4]
    \definition[种,个]{s.}{padrão; desenho; padrões e gráficos usados para decoração de edifícios, tecidos, artes e artesanato, etc.}
  \end{Phonetics}
\end{Entry}

\begin{Entry}{坡}{8}{⼟}
  \begin{Phonetics}{坡}{po1}[][HSK 6]
    \definition{adj.}{inclinado}
    \definition{s.}{declive | encosta}
  \end{Phonetics}
\end{Entry}

\begin{Entry}{坦}{8}{⼟}
  \begin{Phonetics}{坦}{tan3}
    \definition*{s.}{Sobrenome Tan}
    \definition{adj.}{nivelado; suave; plano | calmo; composto | aberto; sincero; franco}
  \end{Phonetics}
\end{Entry}

\begin{Entry}{坦克}{8,7}{⼟、⼗}
  \begin{Phonetics}{坦克}{tan3ke4}
    \definition{s.}{(empréstimo linguístico) tanque (veículo militar)}
  \end{Phonetics}
\end{Entry}

\begin{Entry}{垂}{8}{⼠}
  \begin{Phonetics}{垂}{chui2}[][HSK 7-9]
    \definition{adv.}{perto de; uma palavra respeitosa usada para se referir às ações de outros (geralmente anciãos ou superiores) em relação a si mesmo.}
    \definition{v.}{cair; deixar cair; pendurar (objeto de cabeça para baixo)  | Literário: (mais velhos ou superiores) condescender; um termo respeitoso usado para se referir às ações de outros (geralmente anciãos ou superiores) em relação a si mesmo | transmitir; legar à posteridade; passar para as gerações posteriores}
  \end{Phonetics}
\end{Entry}

\begin{Entry}{垂头丧气}{8,5,8,4}{⼠、⼤、⼗、⽓}
  \begin{Phonetics}{垂头丧气}{chui2tou2-sang4qi4}[][HSK 7-9]
    \definition{v.}{estar desanimado; estar abatido; abaixar a cabeça com um olhar abatido}
  \end{Phonetics}
\end{Entry}

\begin{Entry}{垃}{8}{⼟}
  \begin{Phonetics}{垃}{la1}
    \definition[堆]{s.}{lixo}
  \end{Phonetics}
\end{Entry}

\begin{Entry}{垃圾}{8,6}{⼟、⼟}
  \begin{Phonetics}{垃圾}{la1 ji1}[][HSK 4]
    \definition{adj.}{lixo; inútil, ruim ou prejudicial}
    \definition[袋,桶,堆,车,片]{s.}{entulho; lixo; refugo; rejeito; resíduo; coisa inútil que é jogada fora; metáfora para alguém ou algo que perdeu seu valor ou serve a um propósito ruim}
  \end{Phonetics}
\end{Entry}

\begin{Entry}{垃圾工}{8,6,3}{⼟、⼟、⼯}
  \begin{Phonetics}{垃圾工}{la1ji1gong1}
    \definition{s.}{lixeiro | gari}
  \end{Phonetics}
\end{Entry}

\begin{Entry}{垃圾车}{8,6,4}{⼟、⼟、⾞}
  \begin{Phonetics}{垃圾车}{la1ji1che1}
    \definition{s.}{caminhão de lixo}
  \end{Phonetics}
\end{Entry}

\begin{Entry}{垃圾电邮}{8,6,5,7}{⼟、⼟、⽥、⾢}
  \begin{Phonetics}{垃圾电邮}{la1ji1 dian4you2}
    \definition{s.}{\emph{e-mail} de \emph{spam}}
  \seealsoref{垃圾邮件}{la1ji1 you2jian4}
  \end{Phonetics}
\end{Entry}

\begin{Entry}{垃圾邮件}{8,6,7,6}{⼟、⼟、⾢、⼈}
  \begin{Phonetics}{垃圾邮件}{la1ji1 you2jian4}
    \definition{s.}{\emph{spam}, \emph{e-mail} não solicitado}
  \seealsoref{垃圾电邮}{la1ji1 dian4you2}
  \end{Phonetics}
\end{Entry}

\begin{Entry}{垃圾食品}{8,6,9,9}{⼟、⼟、⾷、⼝}
  \begin{Phonetics}{垃圾食品}{la1ji1shi2pin3}
    \definition{s.}{\emph{junk food}}
  \end{Phonetics}
\end{Entry}

\begin{Entry}{垃圾堆}{8,6,11}{⼟、⼟、⼟}
  \begin{Phonetics}{垃圾堆}{la1ji1dui1}
    \definition{s.}{depósito de lixo}
  \end{Phonetics}
\end{Entry}

\begin{Entry}{垃圾筒}{8,6,12}{⼟、⼟、⽵}
  \begin{Phonetics}{垃圾筒}{la1ji1tong3}
    \definition{s.}{cesto de lixo}
  \end{Phonetics}
\end{Entry}

\begin{Entry}{垃圾箱}{8,6,15}{⼟、⼟、⾋}
  \begin{Phonetics}{垃圾箱}{la1ji1xiang1}
    \definition{s.}{cesto de lixo}
  \end{Phonetics}
\end{Entry}

\begin{Entry}{备}{8}{⼡}
  \begin{Phonetics}{备}{bei4}
    \definition*{s.}{Sobrenome Bei}
    \definition{adv.}{totalmente; de todas as maneiras possíveis | todos; tudo}
    \definition{s.}{equipamento}
    \definition{v.}{estar equipar com; ter; possuir | preparar; ficar pronto | providenciar (ou preparar) contra; tomar precauções contra}
  \end{Phonetics}
\end{Entry}

\begin{Entry}{备用}{8,5}{⼡、⽤}
  \begin{Phonetics}{备用}{bei4yong4}[][HSK 7-9]
    \definition{v.}{reservar; guardar algo para uso futuro}
  \end{Phonetics}
\end{Entry}

\begin{Entry}{备份}{8,6}{⼡、⼈}
  \begin{Phonetics}{备份}{bei4fen4}
    \definition{s.}{cópia de segurança | \emph{backup}}
  \end{Phonetics}
\end{Entry}

\begin{Entry}{备受}{8,8}{⼡、⼜}
  \begin{Phonetics}{备受}{bei4shou4}[][HSK 7-9]
    \definition{v.}{experimentar plenamente (o bem ou o mal)}
  \end{Phonetics}
\end{Entry}

\begin{Entry}{备胎}{8,9}{⼡、⾁}
  \begin{Phonetics}{备胎}{bei4tai1}
    \definition{s.}{pneu sobressalente | (gíria) substituto}
  \end{Phonetics}
\end{Entry}

\begin{Entry}{备课}{8,10}{⼡、⾔}
  \begin{Phonetics}{备课}{bei4/ke4}[][HSK 7-9]
    \definition{v.+compl.}{(professor) preparar aulas}
  \end{Phonetics}
\end{Entry}

\begin{Entry}{夜}{8}{⼣}
  \begin{Phonetics}{夜}{ye4}[][HSK 2]
    \definition{s.}{noite; tarde; noturno; o período do anoitecer ao amanhecer (em oposição a 日 ou 昼); em meteorologia, refere-se especificamente ao período das 20h do dia atual às 8h do dia seguinte}
  \seealsoref{日}{ri4}
  \seealsoref{昼}{zhou4}
  \end{Phonetics}
\end{Entry}

\begin{Entry}{夜生活}{8,5,9}{⼣、⽣、⽔}
  \begin{Phonetics}{夜生活}{ye4sheng1huo2}
    \definition{s.}{vida noturna}
  \end{Phonetics}
\end{Entry}

\begin{Entry}{夜鸟}{8,5}{⼣、⿃}
  \begin{Phonetics}{夜鸟}{ye4niao3}
    \definition{s.}{ave noturna}
  \end{Phonetics}
\end{Entry}

\begin{Entry}{夜场}{8,6}{⼣、⼟}
  \begin{Phonetics}{夜场}{ye4chang3}
    \definition{s.}{show noturno (em um teatro, etc.) | local de entretenimento noturno (bar, boate, discoteca, etc.)}
  \end{Phonetics}
\end{Entry}

\begin{Entry}{夜里}{8,7}{⼣、⾥}
  \begin{Phonetics}{夜里}{ye4li5}[][HSK 2]
    \definition{s.}{noturno; à noite; o período do anoitecer ao amanhecer}
  \end{Phonetics}
\end{Entry}

\begin{Entry}{夜间}{8,7}{⼣、⾨}
  \begin{Phonetics}{夜间}{ye4 jian1}[][HSK 5]
    \definition{s.}{noite; à noite; noturno; durante a noite}
  \end{Phonetics}
\end{Entry}

\begin{Entry}{夜夜}{8,8}{⼣、⼣}
  \begin{Phonetics}{夜夜}{ye4ye4}
    \definition{adv.}{toda noite}
  \end{Phonetics}
\end{Entry}

\begin{Entry}{夜店}{8,8}{⼣、⼴}
  \begin{Phonetics}{夜店}{ye4dian4}
    \definition{s.}{boate | \emph{nightclub}}
  \end{Phonetics}
\end{Entry}

\begin{Entry}{夜晚}{8,11}{⼣、⽇}
  \begin{Phonetics}{夜晚}{ye4wan3}
    \definition[个]{s.}{noite}
  \end{Phonetics}
\end{Entry}

\begin{Entry}{夜深人静}{8,11,2,14}{⼣、⽔、⼈、⾭}
  \begin{Phonetics}{夜深人静}{ye4shen1ren2jing4}
    \definition{expr.}{``Na calada da noite.''; ``No silêncio (ou silêncio) da noite.''}
  \end{Phonetics}
\end{Entry}

\begin{Entry}{夜幕}{8,13}{⼣、⼱}
  \begin{Phonetics}{夜幕}{ye4mu4}
    \definition{s.}{cortina da noite}
  \end{Phonetics}
\end{Entry}

\begin{Entry}{奇}{8}{⼤}
  \begin{Phonetics}{奇}{qi2}
    \definition{adj.}{ímpar (número); singular; solteiro; não em pares (ao contrário de 偶)}
    \definition{s.}{lotes ímpares; quantidade fracionária (acima daquela mencionada em um número redondo)}
  \seealsoref{偶}{ou3}
  \end{Phonetics}
\end{Entry}

\begin{Entry}{奇妙}{8,7}{⼤、⼥}
  \begin{Phonetics}{奇妙}{qi2miao4}[][HSK 6]
    \definition{adj.}{maravilhoso; milagroso; intrigante; muito inteligente e engenhoso (usado principalmente para descrever coisas interessantes e novas)}
  \end{Phonetics}
\end{Entry}

\begin{Entry}{奇怪}{8,8}{⼤、⼼}
  \begin{Phonetics}{奇怪}{qi2guai4}[][HSK 3]
    \definition{adj.}{estranho; diferente do habitual; raramente visto, até um pouco irracional | estranho; esquisito; a descrição é diferente do imaginado e é difícil de entender}
    \definition{v.}{ficar perplexo; maravilhar-se; sentir-se surpreso; sentir-se estranho; sentir-se incompreensível}
  \end{Phonetics}
\end{Entry}

\begin{Entry}{奇迹}{8,9}{⼤、⾡}
  \begin{Phonetics}{奇迹}{qi2ji4}
    \definition[个,种]{s.}{milagre; maravilha; coisas extraordinárias inimagináveis}
  \end{Phonetics}
\end{Entry}

\begin{Entry}{奉}{8}{⼤}
  \begin{Phonetics}{奉}{feng4}
    \definition*{s.}{Sobrenome Feng}
    \definition{v.}{Literário: dedicar ou presentear com respeito | receber (pedidos, instruções, etc.) | Literário: estimar; reverenciar | Litrário: acreditar em  | esperar; atender; servir}
  \end{Phonetics}
\end{Entry}

\begin{Entry}{奉献}{8,13}{⼤、⽝}
  \begin{Phonetics}{奉献}{feng4xian4}[][HSK 6]
    \definition{v.}{dedicar; oferecer como tributo; apresentar com todo respeito; entregar respeitosamente}
  \end{Phonetics}
\end{Entry}

\begin{Entry}{奋}{8}{⼤}
  \begin{Phonetics}{奋}{fen4}
    \definition{adv.}{energicamente; com força e espírito}
    \definition{v.}{esforçar-se; agir vigorosamente; preparar-se | levantar | aplicar energia; resolver; animar-se | acenar; sacudir; levantar}
  \end{Phonetics}
\end{Entry}

\begin{Entry}{奋斗}{8,4}{⼤、⽃}
  \begin{Phonetics}{奋斗}{fen4dou4}[][HSK 4]
    \definition{v.}{lutar; esforçar-se; batalhar; trabalhar duro para atingir um determinado objetivo}
  \end{Phonetics}
\end{Entry}

\begin{Entry}{奋战}{8,9}{⼤、⼽}
  \begin{Phonetics}{奋战}{fen4zhan4}
    \definition{v.}{lutar bravamente | trabalhar duro}
  \end{Phonetics}
\end{Entry}

\begin{Entry}{奔}{8}{⼤}
  \begin{Phonetics}{奔}{ben1}
    \definition{v.}{correr rápido; correr com pressa | apressar | fugir; escapar | galopar | fugir; termo antigo para uma mulher que foge com um homem}
  \end{Phonetics}
  \begin{Phonetics}{奔}{ben4}[][HSK 7-9]
    \definition{prep.}{em direção a}
    \definition{v.}{ir direto em direção a; seguir em direção a; ir direto para o seu destino | aproximar-se; estar prestes a | estar ocupado correndo por aí; correr por algo}
  \end{Phonetics}
\end{Entry}

\begin{Entry}{奔驰}{8,6}{⼤、⾺}
  \begin{Phonetics}{奔驰}{ben1chi2}
    \definition*{s.}{Benz de Mercedes-Benz}
    \definition{v.}{acelerar; galopar; (carro, cavalo, etc.) mover-se ou correr rapidamente}
  \seealsoref{梅赛德斯-奔驰}{mei2sai4de2si1-ben1chi2}
  \end{Phonetics}
\end{Entry}

\begin{Entry}{奔波}{8,8}{⼤、⽔}
  \begin{Phonetics}{奔波}{ben1bo1}[][HSK 7-9]
    \definition{v.}{correr; estar ocupado correndo; correr para frente e para trás, com dificuldade e ocupado}
  \end{Phonetics}
\end{Entry}

\begin{Entry}{奔赴}{8,9}{⼤、⾛}
  \begin{Phonetics}{奔赴}{ben1fu4}[][HSK 7-9]
    \definition{v.}{correr para; apressar-se para; correr em direção a (um certo destino)}
  \end{Phonetics}
\end{Entry}

\begin{Entry}{奔跑}{8,12}{⼤、⾜}
  \begin{Phonetics}{奔跑}{ben1 pao3}[][HSK 6]
    \definition{v.}{correr; correr muito rápido, com uma gama de aplicações mais ampla do que 奔驰, usado principalmente na linguagem falada}
  \seealsoref{奔驰}{ben1chi2}
  \end{Phonetics}
\end{Entry}

\begin{Entry}{妹}{8}{⼥}
  \begin{Phonetics}{妹}{mei4}[][HSK 1]
    \definition*{s.}{Sobrenome Mei}
    \definition[个]{s.}{irmã mais nova | parente do sexo feminino da mesma geração | jovem garota; jovem mulher ou menina}
  \seealsoref{妹妹}{mei4 mei5}
  \end{Phonetics}
\end{Entry}

\begin{Entry}{妹夫}{8,4}{⼥、⼤}
  \begin{Phonetics}{妹夫}{mei4fu5}
    \definition{s.}{marido da irmã mais nova}
  \end{Phonetics}
\end{Entry}

\begin{Entry}{妹妹}{8,8}{⼥、⼥}
  \begin{Phonetics}{妹妹}{mei4 mei5}[][HSK 1]
    \definition[个]{s.}{irmã mais nova}
  \end{Phonetics}
\end{Entry}

\begin{Entry}{妻}{8}{⼥}
  \begin{Phonetics}{妻}{qi1}
    \definition{s.}{esposa}
  \end{Phonetics}
  \begin{Phonetics}{妻}{qi4}
    \definition{v.}{casar uma mulher com (alguém)}
  \end{Phonetics}
\end{Entry}

\begin{Entry}{妻子}{8,3}{⼥、⼦}
  \begin{Phonetics}{妻子}{qi1zi3}
    \definition[个]{s.}{esposa e filhos; (chinês antigo) refere-se a esposas, filhos e filhas}
  \end{Phonetics}
  \begin{Phonetics}{妻子}{qi1zi5}[][HSK 4]
    \definition[个]{s.}{esposa (não é usado como um termo carinhoso)}
  \end{Phonetics}
\end{Entry}

\begin{Entry}{始}{8}{⼥}
  \begin{Phonetics}{始}{shi3}
    \definition*{s.}{Sobrenome Shi}
    \definition{adv.}{somente então; não\dots até}
    \definition{s.}{começo; início}
    \definition{v.}{começar; iniciar}
  \end{Phonetics}
\end{Entry}

\begin{Entry}{始终}{8,8}{⼥、⽷}
  \begin{Phonetics}{始终}{shi3zhong1}[][HSK 3]
    \definition{adv.}{sempre; o tempo todo; durante todo; do começo ao fim; indica continuidade do início ao fim}
    \definition{s.}{todo o processo do começo ao fim}
  \end{Phonetics}
\end{Entry}

\begin{Entry}{姐}{8}{⼥}
  \begin{Phonetics}{姐}{jie3}[][HSK 1]
    \definition[个,位]{s.}{irmã mais velha; irmã | termo genérico para mulheres jovens | mulheres da mesma geração que são mais velhas do que você (geralmente não inclui aquelas que podem ser chamadas de cunhadas) | um título respeitoso para mulheres jovens profissionais em determinados cargos}
  \seealsoref{姐姐}{jie3 jie5}
  \end{Phonetics}
\end{Entry}

\begin{Entry}{姐夫}{8,4}{⼥、⼤}
  \begin{Phonetics}{姐夫}{jie3fu5}
    \definition{s.}{marido da irmã mais velha}
  \end{Phonetics}
\end{Entry}

\begin{Entry}{姐妹}{8,8}{⼥、⼥}
  \begin{Phonetics}{姐妹}{jie3 mei4}[][HSK 4]
    \definition[个]{s.}{irmãs}
  \end{Phonetics}
\end{Entry}

\begin{Entry}{姐姐}{8,8}{⼥、⼥}
  \begin{Phonetics}{姐姐}{jie3 jie5}[][HSK 1]
    \definition[个]{s.}{irmã mais velha}
  \end{Phonetics}
\end{Entry}

\begin{Entry}{姑}{8}{⼥}
  \begin{Phonetics}{姑}{gu1}
    \definition{adv.}{provisoriamente; por enquanto}
    \definition[个,位,名,些]{s.}{irmã do pai; tia | irmã do marido; cunhada | mãe do marido; sogra | freira; mulher que exerce uma ocupação religiosa | a irmã do pai de alguém | mulheres jovens (no campo)}
  \end{Phonetics}
\end{Entry}

\begin{Entry}{姑且}{8,5}{⼥、⼀}
  \begin{Phonetics}{姑且}{gu1qie3}
    \definition{adv.}{provisoriamente | por enquanto}
  \end{Phonetics}
\end{Entry}

\begin{Entry}{姑姑}{8,8}{⼥、⼥}
  \begin{Phonetics}{姑姑}{gu1gu5}[][HSK 6]
    \definition[个,位,名]{s.}{tia; tia paterna}
  \end{Phonetics}
\end{Entry}

\begin{Entry}{姑娘}{8,10}{⼥、⼥}
  \begin{Phonetics}{姑娘}{gu1niang5}[][HSK 3]
    \definition[位,名,个,些]{s.}{menina; jovem senhora; mulher solteira | filha}
  \end{Phonetics}
\end{Entry}

\begin{Entry}{姓}{8}{⼥}
  \begin{Phonetics}{姓}{xing4}[][HSK 2]
    \definition[个]{s.}{sobrenome; nome de família; um caractere que representa um sistema familiar, os chineses colocam o sobrenome em primeiro lugar e o nome em segundo}
    \definition{v.}{ter como sobrenome; tratar um ou mais caracteres como sobrenome}
  \end{Phonetics}
\end{Entry}

\begin{Entry}{姓氏}{8,4}{⼥、⽒}
  \begin{Phonetics}{姓氏}{xing4shi4}
    \definition{s.}{sobrenome}
  \end{Phonetics}
\end{Entry}

\begin{Entry}{姓名}{8,6}{⼥、⼝}
  \begin{Phonetics}{姓名}{xing4ming2}[][HSK 2]
    \definition{s.}{nome; nome completo; sobrenome e nome próprio}
  \end{Phonetics}
\end{Entry}

\begin{Entry}{委}{8}{⼥}
  \begin{Phonetics}{委}{wei1}
    \definition{adj./adv.}{o mesmo que 逶 em 逶迤 sinuoso, curvo}
  \seealsoref{逶}{wei1}
  \seealsoref{逶迤}{wei1yi2}
  \end{Phonetics}
  \begin{Phonetics}{委}{wei3}
    \definition*{s.}{Sobrenome Wei}
    \definition{adj.}{indireto; desviado | apático; abatido | sinuoso; tortuoso | desanimado; apático; sem inspiração}
    \definition{adv.}{realmente; certamente; na verdade}
    \definition{s.}{membro do comitê | comitê; comissão; conselho}
    \definition{v.}{confiar; nomear |  jogar fora; deixar de lado | culpar os outros | confiar | descartar; abandonar | mudar; empurrar | acumular}
  \end{Phonetics}
\end{Entry}

\begin{Entry}{委内瑞拉}{8,4,13,8}{⼥、⼌、⽟、⼿}
  \begin{Phonetics}{委内瑞拉}{wei3nei4rui4la1}
    \definition*{s.}{Venezuela}
  \end{Phonetics}
\end{Entry}

\begin{Entry}{委托}{8,6}{⼥、⼿}
  \begin{Phonetics}{委托}{wei3tuo1}[][HSK 5]
    \definition{v.}{confiar; confiar uma tarefa a outra pessoa ou instituição (para que seja realizada)}
  \end{Phonetics}
\end{Entry}

\begin{Entry}{季}{8}{⼦}
  \begin{Phonetics}{季}{ji4}[][HSK 4]
    \definition*{s.}{Sobrenome Ji}
    \definition{s.}{estação; o ano é dividido em quatro estações, primavera, verão, outono e inverno, e uma estação dura três meses | temporada | o fim de uma era | o último mês de uma temporada | o quarto ou mais novo entre irmãos; último na ordem de precedência}
  \end{Phonetics}
\end{Entry}

\begin{Entry}{季节}{8,5}{⼦、⾋}
  \begin{Phonetics}{季节}{ji4jie2}[][HSK 4]
    \definition[个]{s.}{estação (clima); um período característico do ano}
  \end{Phonetics}
\end{Entry}

\begin{Entry}{季度}{8,9}{⼦、⼴}
  \begin{Phonetics}{季度}{ji4du4}[][HSK 4]
    \definition[个]{s.}{trimestre; período de tempo trimestral}
  \end{Phonetics}
\end{Entry}

\begin{Entry}{孤}{8}{⼦}
  \begin{Phonetics}{孤}{gu1}
    \definition*{s.}{Sobrenome Gu}
    \definition{adj.}{sozinho; solitário; isolado}
    \definition{pron.}{eu; meu humilde eu (usado por príncipes feudais); título autoproclamado dos príncipes feudais}
    \definition[个,名,位]{s.}{órfão}
  \end{Phonetics}
\end{Entry}

\begin{Entry}{孤儿}{8,2}{⼦、⼉}
  \begin{Phonetics}{孤儿}{gu1 er2}[][HSK 6]
    \definition[个,名,位]{s.}{órfão; criança sem pais; crianças que perderam os pais}
  \end{Phonetics}
\end{Entry}

\begin{Entry}{孤独}{8,9}{⼦、⽝}
  \begin{Phonetics}{孤独}{gu1du2}[][HSK 6]
    \definition{adj.}{sozinho; solitário}
  \end{Phonetics}
\end{Entry}

\begin{Entry}{学}{8}{⼦}
  \begin{Phonetics}{学}{xue2}[][HSK 1]
    \definition[所]{s.}{aprendizagem; conhecimento; sabedoria; erudição | objeto de estudo; ramo do conhecimento | escola; faculdade | teoria; doutrina}
    \definition{v.}{estudar; aprender | imitar; copiar}
  \end{Phonetics}
\end{Entry}

\begin{Entry}{学习}{8,3}{⼦、⼄}
  \begin{Phonetics}{学习}{xue2xi2}[][HSK 1]
    \definition{s.}{estudo}
    \definition{v.}{estudar; aprender; adquirir conhecimentos ou habilidades através da leitura, da audição, da pesquisa e da prática}
  \end{Phonetics}
\end{Entry}

\begin{Entry}{学分}{8,4}{⼦、⼑}
  \begin{Phonetics}{学分}{xue2fen1}[][HSK 4]
    \definition{s.}{créditos de um curso; uma unidade de medida do peso e do tempo do curso no ensino superior; cada curso vale um crédito para uma aula por semana durante um semestre; alunos devem concluir o número necessário de créditos para se formar}
  \end{Phonetics}
\end{Entry}

\begin{Entry}{学术}{8,5}{⼦、⽊}
  \begin{Phonetics}{学术}{xue2shu4}[][HSK 4]
    \definition[种]{s.}{aprendizagem; aprendizado; ciências; aprendizado sistemático e especializado}
  \end{Phonetics}
\end{Entry}

\begin{Entry}{学生}{8,5}{⼦、⽣}
  \begin{Phonetics}{学生}{xue2sheng5}[][HSK 1]
    \definition{s.}{aluno; estudante; pupilo}
  \end{Phonetics}
\end{Entry}

\begin{Entry}{学生证}{8,5,7}{⼦、⽣、⾔}
  \begin{Phonetics}{学生证}{xue2sheng5zheng4}
    \definition{s.}{cartão de identidade de estudante}
  \end{Phonetics}
\end{Entry}

\begin{Entry}{学会}{8,6}{⼦、⼈}
  \begin{Phonetics}{学会}{xue2 hui4}[][HSK 6]
    \definition[个]{s.}{sociedade; instituto; sociedade científica; um grupo acadêmico composto por pessoas que estudam um determinado assunto, como a Sociedade de Física, a Sociedade de Biologia, etc.}
    \definition{v.}{aprender; dominar; aprender e aplicar}
  \end{Phonetics}
\end{Entry}

\begin{Entry}{学好}{8,6}{⼦、⼥}
  \begin{Phonetics}{学好}{xue2hao3}
    \definition{v.}{seguir bons exemplos | aprender bem}
  \end{Phonetics}
\end{Entry}

\begin{Entry}{学年}{8,6}{⼦、⼲}
  \begin{Phonetics}{学年}{xue2 nian2}[][HSK 4]
    \definition{s.}{ano letivo; ano acadêmico}
  \end{Phonetics}
\end{Entry}

\begin{Entry}{学问}{8,6}{⼦、⾨}
  \begin{Phonetics}{学问}{xue2wen4}[][HSK 4]
    \definition[门,种,个,项]{s.}{aprendizado, conhecimento, erudição; a compreensão correta do mundo objetivo que alguém tem | conhecimento; aprendizado sistemático; conhecimento sistemático sobre algo ou uma ciência que pode ser aprendido em um livro ou em uma experiência prática}
  \end{Phonetics}
\end{Entry}

\begin{Entry}{学位}{8,7}{⼦、⼈}
  \begin{Phonetics}{学位}{xue2wei4}[][HSK 5]
    \definition[个]{s.}{grau; grau acadêmico; título concedido com base no nível acadêmico profissional, como doutorado, mestrado, etc.}
  \end{Phonetics}
\end{Entry}

\begin{Entry}{学员}{8,7}{⼦、⼝}
  \begin{Phonetics}{学员}{xue2 yuan2}[][HSK 6]
    \definition[位,名,批,个]{s.}{estudante; estagiário; geralmente se refere a pessoas que estudam em escolas ou cursos de treinamento diferentes de faculdades, escolas de ensino médio e escolas primárias}
  \end{Phonetics}
\end{Entry}

\begin{Entry}{学时}{8,7}{⼦、⽇}
  \begin{Phonetics}{学时}{xue2 shi2}[][HSK 4]
    \definition{s.}{hora-aula; hora de aula; período}
  \end{Phonetics}
\end{Entry}

\begin{Entry}{学者}{8,8}{⼦、⽼}
  \begin{Phonetics}{学者}{xue2 zhe3}[][HSK 5]
    \definition[位]{s.}{erudito; homem culto; pessoas que fazem pesquisas acadêmicas geralmente se referem àquelas que alcançaram certo sucesso acadêmico}
  \end{Phonetics}
\end{Entry}

\begin{Entry}{学科}{8,9}{⼦、⽲}
  \begin{Phonetics}{学科}{xue2 ke1}[][HSK 5]
    \definition[门,级]{s.}{ramo do aprendizado; disciplina | disciplina escolar; curso de estudo | cursos teóricos oferecidos em treinamento militar ou físico (oposto a 术科)  | disciplina acadêmica | curso | assunto; tema}
  \seealsoref{术科}{shu4ke1}
  \end{Phonetics}
\end{Entry}

\begin{Entry}{学费}{8,9}{⼦、⾙}
  \begin{Phonetics}{学费}{xue2 fei4}[][HSK 3]
    \definition[笔]{s.}{mensalidade (taxa); prêmio; taxas que os alunos devem pagar para estudar na escola, conforme estabelecido pela escola | preço pelo que se aprendeu ao custo do próprio bolso; a metáfora do preço a pagar para obter uma determinada experiência | custo; preço; todas as despesas necessárias durante o período de estudos do aluno}
  \end{Phonetics}
\end{Entry}

\begin{Entry}{学院}{8,9}{⼦、⾩}
  \begin{Phonetics}{学院}{xue2yuan4}[][HSK 1]
    \definition[个,所]{s.}{academia; instituto; um tipo de instituição de ensino superior que se concentra em uma determinada área de especialização, como faculdades de engenharia, faculdades de música, faculdades de educação, etc.}
  \end{Phonetics}
\end{Entry}

\begin{Entry}{学校}{8,10}{⼦、⽊}
  \begin{Phonetics}{学校}{xue2xiao4}[][HSK 1]
    \definition[所,个]{s.}{escola; instituição de ensino}
  \end{Phonetics}
\end{Entry}

\begin{Entry}{学期}{8,12}{⼦、⽉}
  \begin{Phonetics}{学期}{xue2qi1}[][HSK 2]
    \definition[个,段]{s.}{semestre; período escolar; um ano acadêmico é dividido em dois semestres, um semestre do início do outono até as férias de inverno e um semestre do início da primavera até as férias de verão}
  \end{Phonetics}
\end{Entry}

\begin{Entry}{宗}{8}{⼧}
  \begin{Phonetics}{宗}{zong1}
    \definition*{s.}{Sobrenome Zong}
    \definition{adj.}{do mesmo clã; da mesma família}
    \definition{clas.}{usado para matérias, cargas, etc.}
    \definition{s.}{ancestral; antepassado | clã; família | seita; facção; escola | objetivo principal; propósito | modelo; grande mestre | Datado: unidade administrativa no Tibete, equivalente a um condado | templo ancestral}
    \definition{v.}{(no trabalho acadêmico ou artístico) tomar como modelo; modelar-se em}
  \end{Phonetics}
\end{Entry}

\begin{Entry}{宗教}{8,11}{⼧、⽁}
  \begin{Phonetics}{宗教}{zong1jiao4}[][HSK 6]
    \definition[种]{s.}{religião; uma ideologia social é um reflexo ilusório do mundo objetivo, exigindo que as pessoas acreditem em Deus, no Xintoísmo, em espíritos, no carma, etc., e que depositem suas esperanças no chamado céu ou vida após a morte}
  \end{Phonetics}
\end{Entry}

\begin{Entry}{官}{8}{⼧}
  \begin{Phonetics}{官}{guan1}[][HSK 4]
    \definition*{s.}{Sobrenome Guan}
    \definition{adj.}{propriedade do governo; pertencente ao governo ou ao público | público}
    \definition[个,位,名,些]{s.}{funcionário do governo; oficial; servidor público; titular de cargo; funcionário público nomeado acima de um determinado nível | órgão (parte do tecido do corpo)}
  \end{Phonetics}
\end{Entry}

\begin{Entry}{官方}{8,4}{⼧、⽅}
  \begin{Phonetics}{官方}{guan1fang1}[][HSK 4]
    \definition{s.}{autoridade; (do ou pelo) governo | oficial (de uma organização ou instituição)}
  \end{Phonetics}
\end{Entry}

\begin{Entry}{官司}{8,5}{⼧、⼝}
  \begin{Phonetics}{官司}{guan1 si5}[][HSK 6]
    \definition[场,个]{s.}{ação judicial}
  \end{Phonetics}
\end{Entry}

\begin{Entry}{官桂}{8,10}{⼧、⽊}
  \begin{Phonetics}{官桂}{guan1gui4}
    \definition{s.}{canela}
  \seealsoref{肉桂}{rou4gui4}
  \end{Phonetics}
\end{Entry}

\begin{Entry}{定}{8}{⼧}
  \begin{Phonetics}{定}{ding4}[][HSK 4]
    \definition{adj.}{calmo; estável}
    \definition{adv.}{certamente; com certeza; definitivamente; espressa certeza ou necessidade}
    \definition{v.}{decidir; fixar; definir; determinar; ter certeza | acalmar; estabilizar; tornar estável | assinar (um jornal, etc.); reservar (assentos, ingressos, etc.); encomendar (mercadorias, etc.)}
  \end{Phonetics}
\end{Entry}

\begin{Entry}{定价}{8,6}{⼧、⼈}
  \begin{Phonetics}{定价}{ding4 jia4}[][HSK 6]
    \definition{s.}{fixação de preços; preço especificado}
    \definition{v.}{fixar um preço | fazer um preço; definir um preço}
  \end{Phonetics}
\end{Entry}

\begin{Entry}{定位}{8,7}{⼧、⼈}
  \begin{Phonetics}{定位}{ding4 wei4}[][HSK 6]
    \definition{s.}{posição; localização; posição medida ou definida}
    \definition{v.}{localizar; posicionar; orientar; avaliar algo; usar instrumentos para determinar a localização de objetos; definir o \emph{status} das coisas}
  \end{Phonetics}
\end{Entry}

\begin{Entry}{定时}{8,7}{⼧、⽇}
  \begin{Phonetics}{定时}{ding4 shi2}[][HSK 6]
    \definition{s.}{em um horário fixo; em intervalos regulares}
    \definition{v.}{cronometrar; fixar um tempo (para fazer algo)}
  \end{Phonetics}
\end{Entry}

\begin{Entry}{定期}{8,12}{⼧、⽉}
  \begin{Phonetics}{定期}{ding4qi1}[][HSK 3]
    \definition{adj.}{regular; periódico; em intervalos regulares; com prazo determinado; por tempo limitado}
    \definition{v.}{fixar (definir) uma data; determinar a data; confirmar a data}
  \end{Phonetics}
\end{Entry}

\begin{Entry}{宝}{8}{⼧}
  \begin{Phonetics}{宝}{bao3}[][HSK 4]
    \definition*{s.}{Sobrenome Bao}
    \definition{adj.}{antigo; precioso; estimado}
    \definition{pron.}{estimado; um termo educado usado para se referir à família, loja, etc. de alguém}
    \definition[个,件]{s.}{tesouro; objeto estimado; coisa preciosa | dinheiro; moeda; moeda antiga com furo quadrado no centro; moeda de prata}
  \end{Phonetics}
\end{Entry}

\begin{Entry}{宝贝}{8,4}{⼧、⾙}
  \begin{Phonetics}{宝贝}{bao3bei4}[][HSK 4]
    \definition{adj.}{excêntrico; estranho; imprestável; um termo depreciativo para uma pessoa incompetente ou ridícula}
    \definition[个,件]{s.}{tesouro; objeto estimado; coisa preciosa | querida; \emph{darling}; \emph{baby}; apelido para crianças}
  \end{Phonetics}
\end{Entry}

\begin{Entry}{宝石}{8,5}{⼧、⽯}
  \begin{Phonetics}{宝石}{bao3 shi2}[][HSK 4]
    \definition[颗,枚,块,粒]{s.}{gema; jóia; pedra preciosa; mineral precioso que tem um brilho lindo e uma dureza de mais de sete graus, não é afetado pela atmosfera ou por produtos químicos e pode ser usado como decoração, suporte de instrumentos ou abrasivos}
  \end{Phonetics}
\end{Entry}

\begin{Entry}{宝库}{8,7}{⼧、⼴}
  \begin{Phonetics}{宝库}{bao3ku4}[][HSK 7-9]
    \definition[座,个]{s.}{tesouro; casa de tesouro; um lugar onde coisas preciosas são armazenadas (frequentemente usado metaforicamente)}[图书馆是知识的宝库。===A biblioteca é um tesouro de conhecimento.]
  \end{Phonetics}
\end{Entry}

\begin{Entry}{宝宝}{8,8}{⼧、⼧}
  \begin{Phonetics}{宝宝}{bao3 bao5}[][HSK 4]
    \definition[个,位]{s.}{querida; \emph{darling}; \emph{baby}; apelido para crianças}
  \end{Phonetics}
\end{Entry}

\begin{Entry}{宝贵}{8,9}{⼧、⾙}
  \begin{Phonetics}{宝贵}{bao3gui4}[][HSK 4]
    \definition{adj.}{precioso; extremamente valioso, muito raro, pode ser usado para descrever coisas específicas, também pode ser usado para descrever coisas abstratas | valioso; como um tesouro}
  \end{Phonetics}
\end{Entry}

\begin{Entry}{宝藏}{8,17}{⼧、⾋}
  \begin{Phonetics}{宝藏}{bao3zang4}[][HSK 7-9]
    \definition[座,个]{s.}{depósitos preciosos (minerais); tesouros ou riquezas armazenadas, principalmente minerais}
  \end{Phonetics}
\end{Entry}

\begin{Entry}{实}{8}{⼧}
  \begin{Phonetics}{实}{shi2}
    \definition{adj.}{sólido; cheio por dentro; sem espaços vazios (oposto de 虚) | verdadeiro; real; atual; sincero | forte; eficaz; concreto; real}
    \definition{adv.}{verdadeiramente; realmente; de fato; originalmente}
    \definition{s.}{fato; realidade | semente; fruto}
    \definition{v.}{preencher}
  \seealsoref{虚}{xu1}
  \end{Phonetics}
\end{Entry}

\begin{Entry}{实力}{8,2}{⼧、⼒}
  \begin{Phonetics}{实力}{shi2li4}[][HSK 3]
    \definition{s.}{força real; geralmente se refere à força militar e econômica de um país, grupo ou indivíduo, e também se refere à capacidade de um indivíduo ou grupo em uma competição}
  \end{Phonetics}
\end{Entry}

\begin{Entry}{实习}{8,3}{⼧、⼄}
  \begin{Phonetics}{实习}{shi2xi2}[][HSK 2]
    \definition{s.}{estagiário; prática; estágio}
    \definition{v.}{aplicar e testar os conhecimentos teóricos aprendidos no trabalho prático, a fim de exercitar a capacidade profissional}
  \end{Phonetics}
\end{Entry}

\begin{Entry}{实用}{8,5}{⼧、⽤}
  \begin{Phonetics}{实用}{shi2yong4}[][HSK 4]
    \definition{adj.}{prático; pragmático; funcional; atende aos requisitos reais da aplicação}
    \definition{v.}{colocar em uso prático}
  \end{Phonetics}
\end{Entry}

\begin{Entry}{实在}{8,6}{⼧、⼟}
  \begin{Phonetics}{实在}{shi2zai4}[][HSK 2]
    \definition{adj.}{honesto; sincero | verdadeiro; honesto; realista; não é falso, não é enganador}
    \definition{adv.}{verdadeiramente; de fato; na verdade; usado para reforçar o tom afirmativo, enfatizando que a situação é realmente assim}
  \end{Phonetics}
\end{Entry}

\begin{Entry}{实行}{8,6}{⼧、⾏}
  \begin{Phonetics}{实行}{shi2xing2}[][HSK 3]
    \definition{v.}{praticar; implementar; executar; colocar em prática; realizar (programa, política, plano, etc.) por meio de ação}
  \end{Phonetics}
\end{Entry}

\begin{Entry}{实际}{8,7}{⼧、⾩}
  \begin{Phonetics}{实际}{shi2ji4}[][HSK 2]
    \definition{adj.}{real; efetivo; concreto; prático | factual; prático; realista; de acordo com os fatos}
    \definition{s.}{realidade; prática; coisas e situações que existem objetivamente}
  \end{Phonetics}
\end{Entry}

\begin{Entry}{实际上}{8,7,3}{⼧、⾩、⼀}
  \begin{Phonetics}{实际上}{shi2 ji4 shang4}[][HSK 3]
    \definition{adv.}{de fato; na verdade}
  \end{Phonetics}
\end{Entry}

\begin{Entry}{实现}{8,8}{⼧、⾒}
  \begin{Phonetics}{实现}{shi2xian4}[][HSK 2]
    \definition{v.}{alcançar; atingir; realizar; concretizar; tornar (ideais, planos, etc.) realidade}
  \end{Phonetics}
\end{Entry}

\begin{Entry}{实施}{8,9}{⼧、⽅}
  \begin{Phonetics}{实施}{shi2shi1}[][HSK 4]
    \definition{v.}{colocar em vigor; implementar (leis, políticas, etc.); executar; trazer (colocar) algo em vigor; fazer cumprir; colocar algo em (prática)}
  \end{Phonetics}
\end{Entry}

\begin{Entry}{实验}{8,10}{⼧、⾺}
  \begin{Phonetics}{实验}{shi2yan4}[][HSK 3]
    \definition[个,次]{s.}{teste; experimento; trabalho de laboratório}
    \definition{v.}{testar; experimentar; realizar uma operação ou se envolver em uma atividade para testar uma teoria ou hipótese científica}
  \end{Phonetics}
\end{Entry}

\begin{Entry}{实验室}{8,10,9}{⼧、⾺、⼧}
  \begin{Phonetics}{实验室}{shi2 yan4 shi4}[][HSK 3]
    \definition[个,间]{s.}{laboratório; salas especiais para experimentos científicos}
  \end{Phonetics}
\end{Entry}

\begin{Entry}{实惠}{8,12}{⼧、⼼}
  \begin{Phonetics}{实惠}{shi2hui4}[][HSK 5]
    \definition{adj.}{sólido; substancial; benefícios práticos}
    \definition{s.}{benefício material; benefícios tangíveis; benefícios reais}
  \end{Phonetics}
\end{Entry}

\begin{Entry}{实践}{8,12}{⼧、⾜}
  \begin{Phonetics}{实践}{shi2jian4}[][HSK 6]
    \definition{s.}{prática; filosoficamente, refere-se às ações conscientes das pessoas para transformar a natureza e a sociedade; as atividades de produção são as atividades práticas mais básicas e também incluem atividades políticas, experimentos científicos, educação cultural, etc.}
    \definition{v.}{praticar; realizar; implementar planos e intenções em ações específicas}
  \end{Phonetics}
\end{Entry}

\begin{Entry}{宠}{8}{⼧}
  \begin{Phonetics}{宠}{chong3}[][HSK 7-9]
    \definition*{s.}{Sobrenome Chong}
    \definition{v.}{mimar; estragar; conceder favor a | regalar; encontrar favor com alguém; estar nas boas graças de alguém}
  \end{Phonetics}
\end{Entry}

\begin{Entry}{宠物}{8,8}{⼧、⽜}
  \begin{Phonetics}{宠物}{chong3wu4}[][HSK 6]
    \definition[只]{s.}{animal de estimação; refere-se a pequenos animais criados na família}
  \end{Phonetics}
\end{Entry}

\begin{Entry}{宠爱}{8,10}{⼧、⽖}
  \begin{Phonetics}{宠爱}{chong3'ai4}[][HSK 7-9]
    \definition{v.}{mimar; fazer de alguém um animal de estimação}[爷爷总是宠爱他的孙子。===O avô sempre mima o neto.]
  \end{Phonetics}
\end{Entry}

\begin{Entry}{审}{8}{⼧}
  \begin{Phonetics}{审}{shen3}[][HSK 6]
    \definition*{s.}{Sobrenome Shen}
    \definition{adj.}{cuidadoso; detalhado; completo}
    \definition{adv.}{Literpario: realmente; de ​​fato; como esperado}
    \definition{v.}{examinar; analizar | julgar; interrogar | Literário: saber}
  \end{Phonetics}
\end{Entry}

\begin{Entry}{审查}{8,9}{⼧、⽊}
  \begin{Phonetics}{审查}{shen3cha2}[][HSK 6]
    \definition{v.}{examinar; investigar; verificar se algo está correto e apropriado (geralmente referindo-se a planos, propostas, escritos, qualificações pessoais, etc.); ler e avaliar (provas ou trabalhos de exame)}
  \end{Phonetics}
\end{Entry}

\begin{Entry}{尚}{8}{⼩}
  \begin{Phonetics}{尚}{shang4}
    \definition*{s.}{Sobrenome Shang}
    \definition{adv.}{ainda}
    \definition{s.}{costume predominante; refere-se à tendência predominante na sociedade; coisas que geralmente são admiradas pelas pessoas}
    \definition{v.}{valorizar; estimar; dar grande importância a}
  \end{Phonetics}
\end{Entry}

\begin{Entry}{尚且}{8,5}{⼩、⼀}
  \begin{Phonetics}{尚且}{shang4 qie3}
    \definition{conj.}{nem\dots; muito menos\dots; é usado antes do verbo da primeira oração de uma frase complexa para apresentar alguns exemplos óbvios para comparação, a segunda oração frequentemente usa 何况 ou 更 para ecoar e tirar conclusões inevitáveis ​​sobre exemplos semelhantes com diferentes graus de gravidade}
  \seealsoref{更}{geng4}
  \seealsoref{何况}{he2kuang4}
  \end{Phonetics}
\end{Entry}

\begin{Entry}{尚且……何况……}{8,5,7,7}{⼩、⼀、⼈、⼎}
  \begin{Phonetics}{尚且……何况……}{shang4qie3 he2kuang4}
    \definition{conj.}{ainda que\dots, \dots; além do mais\dots e muito menos\dots}
  \end{Phonetics}
\end{Entry}

\begin{Entry}{居}{8}{⼫}
  \begin{Phonetics}{居}{ju1}
    \definition*{s.}{Sobrenome Ju}
    \definition{s.}{residência; casa | restaurante (em nomes de restaurantes)}
    \definition{v.}{residir; morar; viver | ocupar uma determinada posição; ocupar (um lugar); estar (em uma determinada posição) | reivindicar; afirmar | armazenar; guardar | ficar parado; estar parado}
  \end{Phonetics}
\end{Entry}

\begin{Entry}{居民}{8,5}{⼫、⽒}
  \begin{Phonetics}{居民}{ju1min2}[][HSK 4]
    \definition[个,户,位]{s.}{residente; habitante; pessoas que estão fixas em um único lugar}
  \end{Phonetics}
\end{Entry}

\begin{Entry}{居住}{8,7}{⼫、⼈}
  \begin{Phonetics}{居住}{ju1zhu4}[][HSK 4]
    \definition{v.}{viver; residir; morar; habitar}
  \end{Phonetics}
\end{Entry}

\begin{Entry}{居然}{8,12}{⼫、⽕}
  \begin{Phonetics}{居然}{ju1ran2}[][HSK 5]
    \definition{adv.}{inesperadamente; para surpresa de alguém; além da expectativa (expressão idiomática)}
    \definition{v.}{ir tão longe a ponto de; ter a impudência de; ter o descaramento de}
  \end{Phonetics}
\end{Entry}

\begin{Entry}{屈}{8}{⼫}
  \begin{Phonetics}{屈}{qu1}
    \definition*{s.}{Sobrenome Qu}
    \definition[个]{s.}{injustiça; tratamento injusto | erro; queixa; injustiça}
    \definition{v.}{dobrar; curvar; encurvar | subjugar; submeter | tratar mal; tratar injustamente (ou deslealmente) | estar errado}
  \end{Phonetics}
\end{Entry}

\begin{Entry}{屈原}{8,10}{⼫、⼚}
  \begin{Phonetics}{屈原}{qu1yuan2}
    \definition*{s.}{Qu Yuan, poeta, é uma figura histórica famosa na cultura chinesa que viveu durante o Período dos Reinos Combatentes (340-278 a.C.).}
  \end{Phonetics}
\end{Entry}

\begin{Entry}{届}{8}{⼫}
  \begin{Phonetics}{届}{jie4}[][HSK 5]
    \definition{clas.}{sessões (de uma conferência); anos (de graduação); quantificador, ligeiramente equivalente a 次, usado para reuniões regulares ou turmas de formandos, etc.}
    \definition{v.}{vencer o prazo}
  \seealsoref{次}{ci4}
  \end{Phonetics}
\end{Entry}

\begin{Entry}{岭}{8}{⼭}
  \begin{Phonetics}{岭}{ling3}
    \definition{s.}{cordilheira}
  \end{Phonetics}
\end{Entry}

\begin{Entry}{岸}{8}{⼭}
  \begin{Phonetics}{岸}{an4}[][HSK 5]
    \definition{adj.}{arrogante; orgulhoso; grandioso (de maneira sombria ou condescendente)}
    \definition[条,道,段,面]{s.}{margem; costa; litoral; terreno à beira da água}
  \end{Phonetics}
\end{Entry}

\begin{Entry}{岸上}{8,3}{⼭、⼀}
  \begin{Phonetics}{岸上}{an4 shang4}[][HSK 5]
    \definition{s.}{em terra; costa; margem | na margem do rio; na beira do rio}
  \end{Phonetics}
\end{Entry}

\begin{Entry}{帘}{8}{⼱}
  \begin{Phonetics}{帘}{lian2}
    \definition[块,个]{s.}{bandeira em mastro sobre adega; bandeira como placa de loja | cortina; tela de bambu ou tecido; objetos para cobrir portas e janelas}
  \end{Phonetics}
\end{Entry}

\begin{Entry}{幷}{8}{⼲}
  \begin{Phonetics}{幷}{bing4}
    \variantof{并}
  \end{Phonetics}
\end{Entry}

\begin{Entry}{幸}{8}{⼲}
  \begin{Phonetics}{幸}{xing4}
    \definition*{s.}{Sobrenome Xing}
    \definition{adj.}{feliz}
    \definition{adv.}{afortunadamente; felizmente}
    \definition{s.}{felicidade}
    \definition{v.}{alegrar-se; sentir-se feliz e contente | favorecer; patrocinar | vir; chegar; antigamente, referia-se à chegada de um monarca a um determinado lugar}
  \end{Phonetics}
\end{Entry}

\begin{Entry}{幸亏}{8,3}{⼲、⼆}
  \begin{Phonetics}{幸亏}{xing4kui1}
    \definition{adv.}{felizmente}
  \end{Phonetics}
\end{Entry}

\begin{Entry}{幸运}{8,7}{⼲、⾡}
  \begin{Phonetics}{幸运}{xing4yun4}[][HSK 3]
    \definition{adj.}{sortudo; feliz; afortunado}
    \definition[个,点,丝]{s.}{boa sorte; boa fortuna}
  \end{Phonetics}
\end{Entry}

\begin{Entry}{幸运儿}{8,7,2}{⼲、⾡、⼉}
  \begin{Phonetics}{幸运儿}{xing4yun4'er2}
    \definition{s.}{pessoa de sorte}
  \end{Phonetics}
\end{Entry}

\begin{Entry}{幸运抽奖}{8,7,8,9}{⼲、⾡、⼿、⼤}
  \begin{Phonetics}{幸运抽奖}{xing4yun4chou1jiang3}
    \definition{s.}{loteria | sorteio}
  \end{Phonetics}
\end{Entry}

\begin{Entry}{幸福}{8,13}{⼲、⽰}
  \begin{Phonetics}{幸福}{xing4fu2}[][HSK 3]
    \definition{adj.}{feliz; a vida, a família e outras circunstâncias deixam as pessoas satisfeitas e felizes}
    \definition{s.}{felicidade; bem estar; sensação ou experiência satisfatória e feliz, etc.}
  \end{Phonetics}
\end{Entry}

\begin{Entry}{底}{8}{⼴}
  \begin{Phonetics}{底}{de5}
    \definition{part.}{usada após uma palavra ou frase que é usada como determinante para indicar subordinação à palavra central}
  \end{Phonetics}
  \begin{Phonetics}{底}{di3}[][HSK 4]
    \definition*{s.}{Sobrenome Di}
    \definition{pron.}{o que? |  isto; isso; aqui | assim; tal}
    \definition{s.}{base; fundo; parte inferior de um objeto | detalhes; o cerne da questão; base, fonte ou contexto de uma coisa | rascunho; cópia mantida como registro; rascunho que pode ser usado como base | final de um ano ou mês | chão; fundo; fundação | a última parte de algo}
  \end{Phonetics}
\end{Entry}

\begin{Entry}{底下}{8,3}{⼴、⼀}
  \begin{Phonetics}{底下}{di3 xia4}[][HSK 3]
    \definition{adv.}{em baixo; abaixo; sob | próximo; mais tarde; depois; daqui para a frente}
  \end{Phonetics}
\end{Entry}

\begin{Entry}{底气}{8,4}{⼴、⽓}
  \begin{Phonetics}{底气}{di3qi4}
    \definition{s.}{capacidade pulmonar | ousadia | confiança | autoconfiança | vigor}
  \end{Phonetics}
\end{Entry}

\begin{Entry}{店}{8}{⼴}
  \begin{Phonetics}{店}{dian4}[][HSK 2]
    \definition[家,间,个]{s.}{loja; armazém; loja de venda de mercadorias | pousada; pequena pousada com instalações simples | usado para nomes de lugares}
  \end{Phonetics}
\end{Entry}

\begin{Entry}{店主}{8,5}{⼴、⼂}
  \begin{Phonetics}{店主}{dian4zhu3}
    \definition{s.}{lojista | dono de loja}
  \end{Phonetics}
\end{Entry}

\begin{Entry}{店员}{8,7}{⼴、⼝}
  \begin{Phonetics}{店员}{dian4yuan2}
    \definition{s.}{assistente de loja | balconista | vendedor}
  \end{Phonetics}
\end{Entry}

\begin{Entry}{建}{8}{⼵}
  \begin{Phonetics}{建}{jian4}[][HSK 3]
    \definition*{s.}{Província de Fujian | Rio Jian Jiang (na província de Fujian) | Sobrenome Jian}
    \definition{v.}{construir; construir; erigir | estabelecer; configurar; fundar | propor; defender; apresentar (suas próprias opiniões)}
  \end{Phonetics}
\end{Entry}

\begin{Entry}{建立}{8,5}{⼵、⽴}
  \begin{Phonetics}{建立}{jian4li4}[][HSK 3]
    \definition{v.}{estabelecer; construir; começar a construir | vir a ser; começar a surgir; começar a se formar}
  \end{Phonetics}
\end{Entry}

\begin{Entry}{建立者}{8,5,8}{⼵、⽴、⽼}
  \begin{Phonetics}{建立者}{jian4li4zhe3}
    \definition{s.}{fundador; construtor}
  \end{Phonetics}
\end{Entry}

\begin{Entry}{建议}{8,5}{⼵、⾔}
  \begin{Phonetics}{建议}{jian4yi4}[][HSK 3]
    \definition[个,点,条]{s.}{proposta; sugestão; recomendação; para que alguém ou alguma coisa evolua para melhor, para o coletivo; pontos de vista e opiniões apresentados pelos líderes, etc.}
    \definition{v.}{propor; sugerir; recomendar; em relação a determinada pessoa ou situação, apresentar seus pontos de vista e opiniões ao coletivo, aos líderes ou a indivíduos, para que as coisas evoluam para melhor}
  \end{Phonetics}
\end{Entry}

\begin{Entry}{建成}{8,6}{⼵、⼽}
  \begin{Phonetics}{建成}{jian4 cheng2}[][HSK 3]
    \definition{v.}{terminar a construção}
  \end{Phonetics}
\end{Entry}

\begin{Entry}{建设}{8,6}{⼵、⾔}
  \begin{Phonetics}{建设}{jian4she4}[][HSK 3]
    \definition{s.}{reconstrução; desenvolvimento; trabalhos relacionados com a construção}
    \definition{v.}{construir; edificar; (Estado ou coletividade) criar novos empreendimentos ou aumento de novas instalações}
  \end{Phonetics}
\end{Entry}

\begin{Entry}{建设性}{8,6,8}{⼵、⾔、⼼}
  \begin{Phonetics}{建设性}{jian4she4xing4}
    \definition{adj.}{construtivo}
    \definition{s.}{construtividade}
  \end{Phonetics}
\end{Entry}

\begin{Entry}{建设者}{8,6,8}{⼵、⾔、⽼}
  \begin{Phonetics}{建设者}{jian4she4zhe3}
    \definition{s.}{construtor}
  \end{Phonetics}
\end{Entry}

\begin{Entry}{建造}{8,10}{⼵、⾡}
  \begin{Phonetics}{建造}{jian4 zao4}[][HSK 5]
    \definition{v.}{construir; edificar}
  \end{Phonetics}
\end{Entry}

\begin{Entry}{建筑}{8,12}{⼵、⽵}
  \begin{Phonetics}{建筑}{jian4zhu4}[][HSK 5]
    \definition[座,幢,排]{s.}{construção; estrutura; edifício; prédio}
    \definition{v.}{construir; erguer; edificar; construir casas, estradas, pontes, etc.}
  \end{Phonetics}
\end{Entry}

\begin{Entry}{廻}{8}{⼵}
  \begin{Phonetics}{廻}{hui2}
    \variantof{回}
  \end{Phonetics}
\end{Entry}

\begin{Entry}{录}{8}{⼹}
  \begin{Phonetics}{录}{lu4}[][HSK 3]
    \definition{s.}{registro; cadastro; coleção; seleções}
    \definition{v.}{copiar; gravar; escrever; copiar; registrar | contratar; selecionar; empregar; adotar ou nomear | gravar em fita magnética}
  \end{Phonetics}
\end{Entry}

\begin{Entry}{录取}{8,8}{⼹、⼜}
  \begin{Phonetics}{录取}{lu4qu3}[][HSK 4]
    \definition{v.}{aceitar; admitir; recrutar; entrar; matricular (os aprovados no exame)}
  \end{Phonetics}
\end{Entry}

\begin{Entry}{录音}{8,9}{⼹、⾳}
  \begin{Phonetics}{录音}{lu4/yin1}[][HSK 3]
    \definition[段,个]{s.}{gravação de som; som gravado com equipamento especializado}
    \definition{v.+compl.}{gravar; converter o som em sinal elétrico e, em seguida, gravá-lo por meios mecânicos, ópticos ou eletromagnéticos}
  \end{Phonetics}
\end{Entry}

\begin{Entry}{录音机}{8,9,6}{⼹、⾳、⽊}
  \begin{Phonetics}{录音机}{lu4 yin1 ji1}[][HSK 6]
    \definition[台]{s.}{gravador de som; máquina de gravação (de fita)}
  \end{Phonetics}
\end{Entry}

\begin{Entry}{录像}{8,13}{⼹、⼈}
  \begin{Phonetics}{录像}{lu4/xiang4}[][HSK 6]
    \definition[段,个,些,盘]{s.}{vídeo; gravação; fita de vídeo; imagens gravadas com celulares, câmeras, etc.}
    \definition{v.+compl.}{gravar bídeo; gravar em fita de vídeo | usar celulares, câmeras e outros dispositivos para salvar registros de vídeo}
  \end{Phonetics}
\end{Entry}

\begin{Entry}{录像机}{8,13,6}{⼹、⼈、⽊}
  \begin{Phonetics}{录像机}{lu4xiang4ji1}
    \definition[台]{s.}{gravador de vídeo | VCR}
  \end{Phonetics}
\end{Entry}

\begin{Entry}{录像带}{8,13,9}{⼹、⼈、⼱}
  \begin{Phonetics}{录像带}{lu4xiang4dai4}
    \definition[盘]{s.}{video-cassete}
  \end{Phonetics}
\end{Entry}

\begin{Entry}{彼}{8}{⼻}
  \begin{Phonetics}{彼}{bi3}
    \definition{s.}{aquele; aquilo (oposto a 此) ; outro | a outra parte}
  \seealsoref{此}{ci3}
  \end{Phonetics}
\end{Entry}

\begin{Entry}{彼此}{8,6}{⼻、⽌}
  \begin{Phonetics}{彼此}{bi3ci3}[][HSK 5]
    \definition{pron.}{um ao outro; uns com os outros; este e aquele têm algum tipo de relacionamento; ambas as partes}
  \end{Phonetics}
\end{Entry}

\begin{Entry}{往}{8}{⼻}
  \begin{Phonetics}{往}{wang3}[][HSK 2]
    \definition{adj.}{passado; anterior}
    \definition{prep.}{para; em direção a; na direção de}
    \definition{v.}{ir}
  \end{Phonetics}
\end{Entry}

\begin{Entry}{往日}{8,4}{⼻、⽇}
  \begin{Phonetics}{往日}{wang3ri4}
    \definition{adv.}{dias passados}
    \definition{s.}{o passado}
  \end{Phonetics}
\end{Entry}

\begin{Entry}{往生}{8,5}{⼻、⽣}
  \begin{Phonetics}{往生}{wang3sheng1}
    \definition{v.}{renascer | morrer | (Budismo) viver no paraíso}
  \end{Phonetics}
\end{Entry}

\begin{Entry}{往后}{8,6}{⼻、⼝}
  \begin{Phonetics}{往后}{wang3 hou4}[][HSK 6]
    \definition{s.}{de agora em diante; mais tarde; no futuro | na parte traseira; na parte de trás | para trás; depois; à ré}
  \end{Phonetics}
\end{Entry}

\begin{Entry}{往年}{8,6}{⼻、⼲}
  \begin{Phonetics}{往年}{wang3 nian2}[][HSK 6]
    \definition{s.}{(em) anos anteriores}
  \end{Phonetics}
\end{Entry}

\begin{Entry}{往来}{8,7}{⼻、⽊}
  \begin{Phonetics}{往来}{wang3 lai2}[][HSK 6]
    \definition{s.}{contatos comerciais; relações comerciais; relações diplomáticas | negociações; visitas mútuas; comunicação}
    \definition{v.}{ir e vir | contatar; ter relações}
  \end{Phonetics}
\end{Entry}

\begin{Entry}{往返}{8,7}{⼻、⾡}
  \begin{Phonetics}{往返}{wang3fan3}
    \definition{s.}{ida e volta}
    \definition{v.}{ir e voltar | ir e vir}
  \end{Phonetics}
\end{Entry}

\begin{Entry}{往事}{8,8}{⼻、⼅}
  \begin{Phonetics}{往事}{wang3shi4}
    \definition{s.}{acontecimentos anteriores | eventos passados}
  \end{Phonetics}
\end{Entry}

\begin{Entry}{往例}{8,8}{⼻、⼈}
  \begin{Phonetics}{往例}{wang3li4}
    \definition{s.}{prática (habitual) do passado | precedente}
  \end{Phonetics}
\end{Entry}

\begin{Entry}{往往}{8,8}{⼻、⼻}
  \begin{Phonetics}{往往}{wang3wang3}[][HSK 3]
    \definition{adv.}{frequentemente; muitas vezes; mais frequentemente do que não; indica que uma situação existe ou ocorre com frequência}
  \end{Phonetics}
\end{Entry}

\begin{Entry}{往昔}{8,8}{⼻、⽇}
  \begin{Phonetics}{往昔}{wang3xi1}
    \definition{s.}{o passado}
  \end{Phonetics}
\end{Entry}

\begin{Entry}{往复}{8,9}{⼻、⼢}
  \begin{Phonetics}{往复}{wang3fu4}
    \definition{s.}{para trás e para frente (por exemplo, da ação do pistão ou da bomba)}
    \definition{v.}{ir e voltar | fazer uma viagem de volta}
  \end{Phonetics}
\end{Entry}

\begin{Entry}{往迹}{8,9}{⼻、⾡}
  \begin{Phonetics}{往迹}{wang3ji4}
    \definition{s.}{eventos passados}
  \end{Phonetics}
\end{Entry}

\begin{Entry}{往程}{8,12}{⼻、⽲}
  \begin{Phonetics}{往程}{wang3cheng2}
    \definition{s.}{saída (de uma viagem de ônibus ou trem, etc.)}
  \end{Phonetics}
\end{Entry}

\begin{Entry}{征}{8}{⼻}
  \begin{Phonetics}{征}{zheng1}
    \definition{s.}{prova; evidência | sinal; símbolo; presságio; sinais de manifestação; fenômeno}
    \definition{v.}{viajar; fazer uma jornada; pegar o caminho mais longo | iniciar uma campanha; fazer uma expedição punitiva | convocar; selecionar; recrutar | cobrar; impor; coletar | solicitar; pedir; procurar}
  \end{Phonetics}
\end{Entry}

\begin{Entry}{征求}{8,7}{⼻、⽔}
  \begin{Phonetics}{征求}{zheng1qiu2}[][HSK 4]
    \definition{v.}{procurar; buscar; solicitar; pedir abertamente opiniões, pontos de vista, etc.}
  \end{Phonetics}
\end{Entry}

\begin{Entry}{征服}{8,8}{⼻、⽉}
  \begin{Phonetics}{征服}{zheng1fu2}[][HSK 4]
    \definition{v.}{conquistar; cativar; usar a força para fazer a outra parte se submeter | subjugar; dominar; convencer as pessoas com poder infeccioso}
  \end{Phonetics}
\end{Entry}

\begin{Entry}{忠}{8}{⼼}
  \begin{Phonetics}{忠}{zhong1}
    \definition{adj.}{leal; fiel; devotado | honesto}
  \end{Phonetics}
\end{Entry}

\begin{Entry}{忠心}{8,4}{⼼、⼼}
  \begin{Phonetics}{忠心}{zhong1 xin1}[][HSK 6]
    \definition{s.}{lealdade; devoção; fidelidade}
  \end{Phonetics}
\end{Entry}

\begin{Entry}{念}{8}{⼼}
  \begin{Phonetics}{念}{nian4}[][HSK 3]
    \definition*{s.}{Sobrenome Nian}
    \definition{num.}{vinte; 20; capitalização do número 廿}
    \definition{s.}{ideia; pensamento; pensamentos ou intenções internas}
    \definition{v.}{ler em voz alta | estudar; frequentar a escola | considerar; levar em conta | sentir falta; pensar em; pensar sobre; pensar frequentemente sobre}
  \seealsoref{廿}{nian4}
  \end{Phonetics}
\end{Entry}

\begin{Entry}{忽}{8}{⼼}
  \begin{Phonetics}{忽}{hu1}
    \definition*{s.}{Sobrenome Hu}
    \definition{adv.}{agora\dots, agora\dots | de repente; subitamente}[天气忽冷忽热。===O clima está frio em um minuto e quente no outro.]
    \definition{v.}{negligenciar; ignorar; não prestar atenção; não levar a sério}
  \end{Phonetics}
\end{Entry}

\begin{Entry}{忽视}{8,8}{⼼、⾒}
  \begin{Phonetics}{忽视}{hu1shi4}[][HSK 4]
    \definition{v.}{ignorar; negligenciar; menosprezar; desprezar; dar de ombros}
  \end{Phonetics}
\end{Entry}

\begin{Entry}{忽略}{8,11}{⼼、⽥}
  \begin{Phonetics}{忽略}{hu1lve4}[][HSK 6]
    \definition{v.}{negligenciar; ignorar; não perceber}
  \end{Phonetics}
\end{Entry}

\begin{Entry}{忽然}{8,12}{⼼、⽕}
  \begin{Phonetics}{忽然}{hu1ran2}[][HSK 2]
    \definition{adv.}{repentinamente; de repente; sem aviso prévio; significa que algo aconteceu de forma rápida e inesperada}
  \end{Phonetics}
\end{Entry}

\begin{Entry}{态}{8}{⼼}
  \begin{Phonetics}{态}{tai4}
    \definition{s.}{forma; aparência; condição | (física) estado | (linguística) voz}[气态===estado gasoso | 被动态===voz passiva]
  \end{Phonetics}
\end{Entry}

\begin{Entry}{态度}{8,9}{⼼、⼴}
  \begin{Phonetics}{态度}{tai4du5}[][HSK 2]
    \definition[种,个]{s.}{maneira; comportamento; atitude; comportamento e expressão facial das pessoas | atitude; abordagem; opinião sobre o assunto e medidas tomadas}
  \end{Phonetics}
\end{Entry}

\begin{Entry}{怕}{8}{⼼}
  \begin{Phonetics}{怕}{pa4}[][HSK 2]
    \definition{adv.}{(expressando suposição, julgamento, estimativa, etc.) talvez; suponho; receio (que)}
    \definition{adv.}{por medo; talvez; suponho}
    \definition{v.}{temer; ter medo; recear; sentir medo, ficar nervoso | estar preocupado com; estar preocupado por (ou sobre); ter medo de que algo possa acontecer | ser afetado por; não conseguir suportar; não aguentar mais}
  \end{Phonetics}
\end{Entry}

\begin{Entry}{性}{8}{⼼}
  \begin{Phonetics}{性}{xing4}[][HSK 3]
    \definition[个]{s.}{natureza; caráter; personalidade | propriedade; qualidade; natureza e características das coisas | sexo; gênero | sexualidade; relacionado com a reprodução e a sexualidade | caráter; temperamento}
    \definition{suf.}{indica uma determinada propriedade ou característica de algo; segue um substantivo, verbo ou adjetivo, formando um substantivo abstrato ou um adjetivo que expressa uma propriedade}
  \end{Phonetics}
\end{Entry}

\begin{Entry}{性生活}{8,5,9}{⼼、⽣、⽔}
  \begin{Phonetics}{性生活}{xing4sheng1huo2}
    \definition{s.}{vida sexual}
  \end{Phonetics}
\end{Entry}

\begin{Entry}{性别}{8,7}{⼼、⼑}
  \begin{Phonetics}{性别}{xing4bie2}[][HSK 3]
    \definition[种]{s.}{sexo; gênero}
  \end{Phonetics}
\end{Entry}

\begin{Entry}{性质}{8,8}{⼼、⾙}
  \begin{Phonetics}{性质}{xing4zhi4}[][HSK 4]
    \definition[个,种,类]{s.}{natureza; qualidade; caráter; propriedade; propriedade fundamental que distingue uma coisa de outra}
  \end{Phonetics}
\end{Entry}

\begin{Entry}{性侵}{8,9}{⼼、⼈}
  \begin{Phonetics}{性侵}{xing4qin1}
    \definition{s.}{agressão sexual}
    \definition{v.}{agredir sexualmente}
  \end{Phonetics}
\end{Entry}

\begin{Entry}{性格}{8,10}{⼼、⽊}
  \begin{Phonetics}{性格}{xing4ge2}[][HSK 3]
    \definition[种,个]{s.}{caráter; temperamento; as características psicológicas manifestadas na atitude e no comportamento em relação às pessoas e às coisas}
  \end{Phonetics}
\end{Entry}

\begin{Entry}{性能}{8,10}{⼼、⾁}
  \begin{Phonetics}{性能}{xing4neng2}[][HSK 5]
    \definition{s.}{natureza; propriedade; desempenho; função (de uma máquina, etc.); grau de conformidade dos produtos mecânicos ou outros produtos industriais com os requisitos de projeto}
  \end{Phonetics}
\end{Entry}

\begin{Entry}{怪}{8}{⼼}
  \begin{Phonetics}{怪}{guai4}[][HSK 4,5]
    \definition*{s.}{Sobrenome Guai}
    \definition{adj.}{estranho; esquisito; desconcertante | peculiar; excêntrico; pitoresco; monstruoso; anormal; incomum}
    \definition{adv.}{bastante; muito}
    \definition{s.}{monstro; demônio | diabo; ser maligno}
    \definition{v.}{culpar | achar algo estranho; maravilhar-se com; ficar surpreso | repreender; culpar; reclamar}
  \end{Phonetics}
\end{Entry}

\begin{Entry}{怪兽}{8,11}{⼼、⼋}
  \begin{Phonetics}{怪兽}{guai4shou4}
    \definition{s.}{animal raro | animal mítico | monstro}
  \end{Phonetics}
\end{Entry}

\begin{Entry}{怪癖}{8,18}{⼼、⽧}
  \begin{Phonetics}{怪癖}{guai4pi3}
    \definition{adj.}{peculiar}
    \definition{s.}{excentricidade | peculiaridade | hobby estranho}
  \end{Phonetics}
\end{Entry}

\begin{Entry}{或}{8}{⼽}
  \begin{Phonetics}{或}{huo4}[][HSK 2]
    \definition{adv.}{talvez; possivelmente; provavelmente | (geralmente na forma negativa) um pouco; ligeiramente}
    \definition{conj.}{ou (indicando escolha); ou\dots ou\dots}
    \definition{pron.}{alguém; algumas pessoas; refere-se a alguém ou algo, equivalente a 有人 ou 有的}
  \seealsoref{有的}{you3 de5}
  \seealsoref{有人}{you3 ren2}
  \end{Phonetics}
\end{Entry}

\begin{Entry}{或许}{8,6}{⼽、⾔}
  \begin{Phonetics}{或许}{huo4xu3}[][HSK 4]
    \definition{adv.}{talvez; possivelmente; receio; não tenho certeza}
  \end{Phonetics}
\end{Entry}

\begin{Entry}{或者}{8,8}{⼽、⽼}
  \begin{Phonetics}{或者}{huo4zhe3}[][HSK 2]
    \definition{adv.}{talvez; possivelmente}
    \definition{conj.}{ou (usado em expressões afirmativas); ou\dots ou\dots; usado em frases narrativas para indicar uma relação de escolha | ou (usado para indicar equação); indica relação de equivalência, indicando que os objetos anterior e posterior são iguais}
  \end{Phonetics}
\end{Entry}

\begin{Entry}{或是}{8,9}{⼽、⽇}
  \begin{Phonetics}{或是}{huo4 shi4}[][HSK 5]
    \definition{adv.}{um ou outro; o outro}
    \definition{conj.}{ou; às vezes, é apenas uma de duas coisas}
  \end{Phonetics}
\end{Entry}

\begin{Entry}{房}{8}{⼾}
  \begin{Phonetics}{房}{fang2}
    \definition*{s.}{Fang, a quarta das vinte e oito constelações nas quais a esfera celeste foi dividida, consistindo de quatro estrelas quase em linha reta em Escorpião | Sobrenome Fang}
    \definition[幢,个,间]{s.}{casa; edifício | sala; quarto; câmara | estrutura semelhante a uma casa | um ramo de uma família extensa | loja; estoque | local de trabalho do artesão; oficina; moinho}
  \end{Phonetics}
\end{Entry}

\begin{Entry}{房子}{8,3}{⼾、⼦}
  \begin{Phonetics}{房子}{fang2 zi5}[][HSK 1]
    \definition[栋,幢,座,套,间]{s.}{casa; edifício; prédio}
  \end{Phonetics}
\end{Entry}

\begin{Entry}{房东}{8,5}{⼾、⼀}
  \begin{Phonetics}{房东}{fang2dong1}[][HSK 3]
    \definition[个,位,名]{s.}{dono;  proprietário; senhorio; pessoas que alugam ou emprestam imóveis (para os 房客 )}
  \seealsoref{房客}{fang2ke4}
  \end{Phonetics}
\end{Entry}

\begin{Entry}{房主}{8,5}{⼾、⼂}
  \begin{Phonetics}{房主}{fang2zhu3}
    \definition{s.}{proprietário | dono de um imóvel}
  \end{Phonetics}
\end{Entry}

\begin{Entry}{房价}{8,6}{⼾、⼈}
  \begin{Phonetics}{房价}{fang2 jia4}[][HSK 6]
    \definition{s.}{custo de moradia; tarifa de quarto | preço da casa}
  \end{Phonetics}
\end{Entry}

\begin{Entry}{房间}{8,7}{⼾、⾨}
  \begin{Phonetics}{房间}{fang2jian1}[][HSK 1]
    \definition[个,间,套]{s.}{sala; câmara; escritório; apartamento; divisões internas da casa}
  \end{Phonetics}
\end{Entry}

\begin{Entry}{房客}{8,9}{⼾、⼧}
  \begin{Phonetics}{房客}{fang2ke4}[][HSK 3]
    \definition{s.}{inquilino (de um quarto ou casa); hóspede (oposto a 房东) | inquilino; hóspede; pessoas que alugam ou emprestam imóveis para moradia (para o 房东)}
  \seealsoref{房东}{fang2dong1}
  \end{Phonetics}
\end{Entry}

\begin{Entry}{房屋}{8,9}{⼾、⼫}
  \begin{Phonetics}{房屋}{fang2 wu1}[][HSK 3]
    \definition[间,所,套]{s.}{casas; habitação; edifícios}
  \end{Phonetics}
\end{Entry}

\begin{Entry}{房租}{8,10}{⼾、⽲}
  \begin{Phonetics}{房租}{fang2 zu1}[][HSK 3]
    \definition[笔]{s.}{aluguel}
  \end{Phonetics}
\end{Entry}

\begin{Entry}{所}{8}{⼾}
  \begin{Phonetics}{所}{suo3}[][HSK 3,6]
    \definition*{s.}{Sobrenome Suo}
    \definition{clas.}{usado para casas, etc.}
    \definition{part.}{usado com 为 ou 被 para indicar voz passiva | usado antes do verbo para formar um substantivo ou para qualificar um substantivo | usado antes do verbo na estrutura sujeito-predicado usada como complemento, indica que o termo central é o objeto}
    \definition{s.}{lugar | usado como nome de órgãos governamentais ou outros locais de trabalho}
  \seealsoref{被}{bei4}
  \seealsoref{为}{wei4}
  \end{Phonetics}
\end{Entry}

\begin{Entry}{所以}{8,4}{⼾、⼈}
  \begin{Phonetics}{所以}{suo3 yi3}[][HSK 2]
    \definition{conj.}{assim; portanto; como resultado; conecta frases, expressa resultados e costuma corresponder a expressões como 因为 e 由于}
    \definition[个]{s.}{motivo real; causa real; comportamento adequado}
  \seealsoref{因为}{yin1wei4}
  \seealsoref{由于}{you2yu2}
  \end{Phonetics}
\end{Entry}

\begin{Entry}{所长}{8,4}{⼾、⾧}
  \begin{Phonetics}{所长}{suo3 chang2}
    \definition{s.}{aquilo em que alguém é bom; o ponto forte de alguém; o forte de alguém}
  \end{Phonetics}
  \begin{Phonetics}{所长}{suo3 zhang3}[][HSK 3]
    \definition{s.}{chefe de um instituto, etc. | superintendente}
  \end{Phonetics}
\end{Entry}

\begin{Entry}{所在}{8,6}{⼾、⼟}
  \begin{Phonetics}{所在}{suo3 zai4}[][HSK 5]
    \definition[个]{s.}{lugar; local; localização | o lugar onde alguém ou algo está}
  \end{Phonetics}
\end{Entry}

\begin{Entry}{所有}{8,6}{⼾、⽉}
  \begin{Phonetics}{所有}{suo3you3}[][HSK 2]
    \definition{adj.}{todo | tudo}
    \definition{adj.}{tudo}
    \definition{s.}{bens; posses;}
    \definition{v.}{possuir; ter}
  \end{Phonetics}
\end{Entry}

\begin{Entry}{承}{8}{⼿}
  \begin{Phonetics}{承}{cheng2}
    \definition*{s.}{Sobrenome Cheng}
    \definition{v.}{suportar; segurar; carregar; sustentar | empreender; contratar (para fazer um trabalho) | estar em dívida (com alguém por uma gentileza); receber um favor | continuar; prosseguir | receber de cima (instruções, mandato)}
  \end{Phonetics}
\end{Entry}

\begin{Entry}{承办}{8,4}{⼿、⼒}
  \begin{Phonetics}{承办}{cheng2ban4}[][HSK 5]
    \definition{v.}{ocupar-se de; encarregar-se de; (pessoas, organizações, instituições) aceitar (atividades, reuniões, negócios, etc.)}
  \end{Phonetics}
\end{Entry}

\begin{Entry}{承认}{8,4}{⼿、⾔}
  \begin{Phonetics}{承认}{cheng2ren4}[][HSK 4]
    \definition{s.}{reconhecimento (diplomático, artístico, etc.)}
    \definition{v.}{admitir; reconhecer | dar reconhecimento diplomático; reconhecer}
  \end{Phonetics}
\end{Entry}

\begin{Entry}{承包}{8,5}{⼿、⼓}
  \begin{Phonetics}{承包}{cheng2bao1}[][HSK 7-9]
    \definition{v.}{contratar (com; para); aceitar projetos ou pedidos em massa, etc. e ser responsável por concluí-los}[他承包了这个工程。===Ele foi contratado para esse projeto.]
  \end{Phonetics}
\end{Entry}

\begin{Entry}{承受}{8,8}{⼿、⼜}
  \begin{Phonetics}{承受}{cheng2shou4}[][HSK 4]
    \definition{v.}{suportar; resistir; realizar (tarefas, dificuldades, pressões, etc.); submeter-se a (testes, etc.) | herdar}
  \end{Phonetics}
\end{Entry}

\begin{Entry}{承担}{8,8}{⼿、⼿}
  \begin{Phonetics}{承担}{cheng2dan1}[][HSK 4]
    \definition{v.}{suportar; empreender; assumir; tomar conta de algo}
  \end{Phonetics}
\end{Entry}

\begin{Entry}{承诺}{8,10}{⼿、⾔}
  \begin{Phonetics}{承诺}{cheng2nuo4}[][HSK 6]
    \definition[个,句,份]{s.}{juramento; promessa; compromisso}
    \definition{v.}{prometer fazer algo; prometer empreender; comprometer-se a fazer algo}
  \end{Phonetics}
\end{Entry}

\begin{Entry}{承载}{8,10}{⼿、⾞}
  \begin{Phonetics}{承载}{cheng2zai4}[][HSK 7-9]
    \definition{v.}{suportar o peso; segurar o objeto e suportar seu peso}[桥梁承载着巨大的重量。===A ponte suporta uma carga pesada.]
  \end{Phonetics}
\end{Entry}

\begin{Entry}{披}{8}{⼿}
  \begin{Phonetics}{披}{pi1}[][HSK 5]
    \definition{v.}{colocar sobre os ombros; enrolar em volta; cobrir ou colocar sobre os ombros | abrir; desenrolar; espalhar | abrir-se; rachar}
  \end{Phonetics}
\end{Entry}

\begin{Entry}{抬}{8}{⼿}
  \begin{Phonetics}{抬}{tai2}[][HSK 5]
    \definition{clas.}{usado para objetos que precisam ser carregados por pessoas quando transportados (por exemplo, móveis)}
    \definition{v.}{levantar; elevar; puxar para cima | (por duas ou mais pessoas) carregar; transportar; duas ou mais pessoas carregando algo com as mãos ou nos ombros | discutir, debater (geralmente sem sentido ou sem importância)}
  \end{Phonetics}
\end{Entry}

\begin{Entry}{抬头}{8,5}{⼿、⼤}
  \begin{Phonetics}{抬头}{tai2 tou2}[][HSK 5]
    \definition{s.}{(em recibos, contas, etc.) nome do comprador ou beneficiário, o local no documento onde o nome do beneficiário ou destinatário é escrito}
    \definition{v.}{levantar a cabeça}
  \end{Phonetics}
\end{Entry}

\begin{Entry}{抬杠}{8,7}{⼿、⽊}
  \begin{Phonetics}{抬杠}{tai2/gang4}
    \definition{v.+compl.}{discutir pelo prazer em discutir | discutir obstinadamente | brigar}
  \end{Phonetics}
\end{Entry}

\begin{Entry}{抱}{8}{⼿}
  \begin{Phonetics}{抱}{bao4}[][HSK 4]
    \definition*{s.}{Sobrenome Bao}
    \definition{clas.}{braçada; medida dos dois braços}
    \definition{v.}{carregar no peito; segurar com ambos os braços; abraçar | ter o primeiro filho ou neto | adotar um bebê ou criança | ficar juntos, unidos | encaixar ou servir perfeitamente (roupas e sapatos do tamanho certo) | estimar; nutrir; abrigar; ter em mente | continuar; sobrecarregar com | chocar ovos}
  \end{Phonetics}
\end{Entry}

\begin{Entry}{抱负}{8,6}{⼿、⾙}
  \begin{Phonetics}{抱负}{bao4fu4}[][HSK 7-9]
    \definition{s.}{aspiração; ambição; objetivo elevado; grandes intenções e determinação, frequentemente usados na linguagem escrita}
  \end{Phonetics}
\end{Entry}

\begin{Entry}{抱怨}{8,9}{⼿、⼼}
  \begin{Phonetics}{抱怨}{bao4yuan4}[][HSK 5]
    \definition{v.}{reclamar ou expressar descontentamento ou insatisfação; falar com os outros sobre pessoas ou coisas com as quais você não está satisfeito}
  \end{Phonetics}
\end{Entry}

\begin{Entry}{抱歉}{8,14}{⼿、⽋}
  \begin{Phonetics}{抱歉}{bao4qian4}[][HSK 6]
    \definition{adj.}{pesaroso; arrependido; sentir pena de alguém porque você causou perda, inconveniência ou não atendeu às suas necessidades}
  \end{Phonetics}
\end{Entry}

\begin{Entry}{抵}{8}{⼿}
  \begin{Phonetics}{抵}{di3}
    \definition{v.}{apoiar; sustentar | resistir; suportar | compensar; fazer o bem | hipotecar; dar como garantia; garantir | equilibrar; cancelar; compensar | ser igual a; corresponder | alcançar; chegar a | colidir; dar cabeçada (por animais com chifres)}
  \end{Phonetics}
\end{Entry}

\begin{Entry}{抵达}{8,6}{⼿、⾡}
  \begin{Phonetics}{抵达}{di3da2}[][HSK 6]
    \definition{v.}{chegar; alcançar}
  \end{Phonetics}
\end{Entry}

\begin{Entry}{抵抗}{8,7}{⼿、⼿}
  \begin{Phonetics}{抵抗}{di3kang4}[][HSK 6]
    \definition{s.}{resistência}
    \definition{v.}{resistir; usar ação para resistir ou parar o ataque da outra parte}
  \end{Phonetics}
\end{Entry}

\begin{Entry}{抹}{8}{⼿}
  \begin{Phonetics}{抹}{ma1}
    \definition{v.}{esfregar; limpar | deslizar algo para fora; tirar}
  \end{Phonetics}
  \begin{Phonetics}{抹}{mo3}
    \definition{v.}{colocar; aplicar; untar; engessar | limpar | anular; apagar | (para nuvem, etc.) irradiar; raiar; riscar; traçar | riscar; cancelar; marcar; remover; excluir}
  \end{Phonetics}
  \begin{Phonetics}{抹}{mo4}
    \definition{v.}{rebocar; engessar; alisar a massa ou o gesso com uma espátula | virar; contornar; dar uma volta de perto}
  \end{Phonetics}
\end{Entry}

\begin{Entry}{抹泪}{8,8}{⼿、⽔}
  \begin{Phonetics}{抹泪}{mo3lei4}
    \definition{v.}{limpar as lágrimas | (figurativo) derramar lágrimas}
  \end{Phonetics}
\end{Entry}

\begin{Entry}{押}{8}{⼿}
  \begin{Phonetics}{押}{ya1}
    \definition*{s.}{Sobrenome Ya}
    \definition{s.}{assinatura; marca em vez de assinatura; nome assinado ou símbolo desenhado}
    \definition{v.}{dar como garantia; hipotecar; penhorar | deter; levar sob custódia | escoltar | assinar (um documento, contrato, etc.); colocar sua assinatura (ou marcar no lugar da assinatura)}
  \end{Phonetics}
\end{Entry}

\begin{Entry}{押后}{8,6}{⼿、⼝}
  \begin{Phonetics}{押后}{ya1hou4}
    \definition{v.}{encerrar | adiar}
  \end{Phonetics}
\end{Entry}

\begin{Entry}{押运}{8,7}{⼿、⾡}
  \begin{Phonetics}{押运}{ya1yun4}
    \definition{v.}{escoltar sob guarda | escoltar (bens ou fundos)}
  \end{Phonetics}
\end{Entry}

\begin{Entry}{押注}{8,8}{⼿、⽔}
  \begin{Phonetics}{押注}{ya1zhu4}
    \definition{v.}{apostar}
  \end{Phonetics}
\end{Entry}

\begin{Entry}{押金}{8,8}{⼿、⾦}
  \begin{Phonetics}{押金}{ya1jin1}[][HSK 5]
    \definition[笔,份,些]{s.}{caução; sinal; depósito; dinheiro como garantia}
  \end{Phonetics}
\end{Entry}

\begin{Entry}{押送}{8,9}{⼿、⾡}
  \begin{Phonetics}{押送}{ya1song4}
    \definition{v.}{enviar sob escolta | transportar um detido}
  \end{Phonetics}
\end{Entry}

\begin{Entry}{押租}{8,10}{⼿、⽲}
  \begin{Phonetics}{押租}{ya1zu1}
    \definition{s.}{depósito de aluguel}
  \end{Phonetics}
\end{Entry}

\begin{Entry}{押韵}{8,13}{⼿、⾳}
  \begin{Phonetics}{押韵}{ya1yun4}
    \definition{v.}{rimar}
  \end{Phonetics}
\end{Entry}

\begin{Entry}{抽}{8}{⼿}
  \begin{Phonetics}{抽}{chou1}[][HSK 4]
    \definition{v.}{retirar; tirar (do meio); retirar, puxar ou arrancar algo que está preso ou emaranhado em outra coisa | tirar, retirar (uma parte de um todo) | (certas plantas) começar a crescer, produzir | bombear | encolher; contrair | chicotear; açoitar; surrar | dirigir; conduzir | encontrar tempo; libertar-se; sair de alguma coisa}
  \end{Phonetics}
\end{Entry}

\begin{Entry}{抽屉}{8,8}{⼿、⼫}
  \begin{Phonetics}{抽屉}{chou1ti4}[][HSK 7-9]
    \definition[个,层,组]{s.}{gaveta}
  \end{Phonetics}
\end{Entry}

\begin{Entry}{抽奖}{8,9}{⼿、⼤}
  \begin{Phonetics}{抽奖}{chou1 jiang3}[][HSK 4]
    \definition{s.}{loteria; sorteio de loteria}
  \end{Phonetics}
\end{Entry}

\begin{Entry}{抽烟}{8,10}{⼿、⽕}
  \begin{Phonetics}{抽烟}{chou1/yan1}[][HSK 4]
    \definition{v.+compl.}{fumar (um cigarro ou um cachimbo)}
  \end{Phonetics}
\end{Entry}

\begin{Entry}{抽象}{8,11}{⼿、⾗}
  \begin{Phonetics}{抽象}{chou1xiang4}[][HSK 7-9]
    \definition{adj.}{abstrato}[抽象的艺术需要想象力。===A arte abstrata requer imaginação.]
    \definition{v.}{abstrair}[这个理论很难抽象。===Essa teoria é difícil de abstrair.]
  \end{Phonetics}
\end{Entry}

\begin{Entry}{抽签}{8,13}{⼿、⽵}
  \begin{Phonetics}{抽签}{chou1/qian1}[][HSK 7-9]
    \definition{v.+compl.}{tirar/lançar sorte; realizar/fazer um sorteio}[他们抽签决定胜者。===Eles fizeram um sorteio para decidir o vencedor.]
  \end{Phonetics}
\end{Entry}

\begin{Entry}{担}{8}{⼿}
  \begin{Phonetics}{担}{dan1}[][HSK 7-9]
    \definition{v.}{carregar em uma vara de ombro e baldes; carregar nos ombros | assumir; empreender; não ter medo de correr riscos}
  \end{Phonetics}
  \begin{Phonetics}{担}{dan4}[][HSK 7-9]
    \definition{clas.}{dan, uma unidade de peso (=50 quilogramas) ; 100 jin = 1 dan | usado em coisas usadas para transportar cargas}
    \definition{s.}{carga; fardo; cargas de mercadorias transportadas em uma vara de ombro por um mascate itinerante}
  \end{Phonetics}
\end{Entry}

\begin{Entry}{担子}{8,3}{⼿、⼦}
  \begin{Phonetics}{担子}{dan4zi5}[][HSK 7-9]
    \definition[副,个]{s.}{vara de transporte (ou de ombro) e as cargas sob ela; canga; carga; fardo | tarefa}
  \end{Phonetics}
\end{Entry}

\begin{Entry}{担心}{8,4}{⼿、⼼}
  \begin{Phonetics}{担心}{dan1xin1}[][HSK 4]
    \definition{v.}{preocupar-se; ficar ansioso; sentir-se desconfortável com algo}
  \end{Phonetics}
\end{Entry}

\begin{Entry}{担任}{8,6}{⼿、⼈}
  \begin{Phonetics}{担任}{dan1ren4}[][HSK 4]
    \definition{v.}{servir como; assumir o cargo de; ocupar o posto de; ocupar um determinado cargo ou emprego}
  \end{Phonetics}
\end{Entry}

\begin{Entry}{担当}{8,6}{⼿、⼹}
  \begin{Phonetics}{担当}{dan1dang1}[][HSK 7-9]
    \definition{v.}{aceitar e assumir responsabilidade; empreender (responsabilidade, trabalho, despesas)}
  \end{Phonetics}
\end{Entry}

\begin{Entry}{担负}{8,6}{⼿、⾙}
  \begin{Phonetics}{担负}{dan1fu4}[][HSK 7-9]
    \definition{v.}{suportar; carregar; assumir; ser encarregado de}
  \end{Phonetics}
\end{Entry}

\begin{Entry}{担忧}{8,7}{⼿、⼼}
  \begin{Phonetics}{担忧}{dan1 you1}[][HSK 6]
    \definition[项,条,套,种]{v.}{preocupar-se; estar ansioso}
  \end{Phonetics}
\end{Entry}

\begin{Entry}{担保}{8,9}{⼿、⼈}
  \begin{Phonetics}{担保}{dan1bao3}[][HSK 4]
    \definition{v.}{garantir; atestar; expressar responsabilidade e garantir que não haverá problemas ou que eles serão resolvidos}
  \end{Phonetics}
\end{Entry}

\begin{Entry}{拆}{8}{⼿}
  \begin{Phonetics}{拆}{chai1}[][HSK 5]
    \definition{v.}{rasgar; desmontar; separar o que está unido | derrubar; desmantelar; demolir; refere-se especificamente à demolição de edifícios}
  \end{Phonetics}
\end{Entry}

\begin{Entry}{拆迁}{8,6}{⼿、⾡}
  \begin{Phonetics}{拆迁}{chai1 qian1}[][HSK 6]
    \definition{v.}{demolir uma casa velha e realocar seus ocupantes em outro lugar; devido às necessidades de construção, unidades ou casas residenciais são demolidas e realocadas em outros lugares}
  \end{Phonetics}
\end{Entry}

\begin{Entry}{拆除}{8,9}{⼿、⾩}
  \begin{Phonetics}{拆除}{chai1 chu2}[][HSK 5]
    \definition{v.}{desmantelar; demolir; derrubar; remover (um edifício, etc.)}
  \end{Phonetics}
\end{Entry}

\begin{Entry}{拉}{8}{⼿}
  \begin{Phonetics}{拉}{la1}[][HSK 2]
    \definition{s.}{abreviação de América Latina, 拉丁美洲}
    \definition{v.}{puxar; arrastar; rebocar | transportar por veículo; rebocar | arrastar (ou puxar) para fora | mover (tropas para um lugar) | dar uma mãozinha; ajudar | arrastar para dentro; implicar; envolver | criar (criança) | atrair; conquistar; solicitar; angariar votos | bater-papo | organizar; preparar | ter dívidas; estar endividado | pressionar; recrutar à força | (no tênis, tênis de mesa, etc.) levantar (a bola) | tocar (certos instrumentos musicais); puxar uma parte do instrumento para que ele emita som | prolongar; espaçar | envolver-se em | (coloquial) esvaziar os intestinos | levantar, uma das técnicas do tênis de mesa | destruir; esmagar; quebrar}
  \seealsoref{拉丁美洲}{la1ding1 mei3zhou1}
  \end{Phonetics}
  \begin{Phonetics}{拉}{la4}
    \definition{s.}{usado em 拉拉蛄 \dpy{la4la4gu3}}
  \seealsoref{拉拉蛄}{la4la4gu3}
  \end{Phonetics}
\end{Entry}

\begin{Entry}{拉丁美洲}{8,2,9,9}{⼿、⼀、⽺、⽔}
  \begin{Phonetics}{拉丁美洲}{la1ding1 mei3zhou1}
    \definition*{s.}{América Latina, nome coletivo dos países da América Central e do Sul, devido ao fato de a maioria de seus habitantes ser descendente de povos latinos e de a língua falada ser do grupo latino}
  \end{Phonetics}
\end{Entry}

\begin{Entry}{拉开}{8,4}{⼿、⼶}
  \begin{Phonetics}{拉开}{la1 kai1}[][HSK 4]
    \definition{v.}{puxar para abrir; recuar| ampliar; espaçar; distanciar; afastar; separar}
  \end{Phonetics}
\end{Entry}

\begin{Entry}{拉布布}{8,5,5}{⼿、⼱、⼱}
  \begin{Phonetics}{拉布布}{la1bu4bu4}
    \definition*{s.}{Labubu}
  \end{Phonetics}
\end{Entry}

\begin{Entry}{拉拉队}{8,8,4}{⼿、⼿、⾩}
  \begin{Phonetics}{拉拉队}{la1la1dui4}
    \definition{s.}{claque | torcida}
  \end{Phonetics}
\end{Entry}

\begin{Entry}{拉拉蛄}{8,8,11}{⼿、⼿、⾍}
  \begin{Phonetics}{拉拉蛄}{la4la4gu3}
    \variantof{蝲蝲蛄}
  \end{Phonetics}
\end{Entry}

\begin{Entry}{拉萨}{8,11}{⼿、⾋}
  \begin{Phonetics}{拉萨}{la1sa4}
    \definition*{s.}{Lhasa, capital da Região Autônoma do Tibete, 西藏自治区}
  \seealsoref{西藏自治区}{xi1zang4 zi4zhi4qu1}
  \end{Phonetics}
\end{Entry}

\begin{Entry}{拌}{8}{⼿}
  \begin{Phonetics}{拌}{ban4}[][HSK 7-9]
    \definition{v.}{misturar | mexer e misturar | discutir; brigar; ter uma discussão}
  \end{Phonetics}
\end{Entry}

\begin{Entry}{拍}{8}{⼿}
  \begin{Phonetics}{拍}{pai1}[][HSK 3]
    \definition[个,副,对]{s.}{bastão; raquete | batida; tempo; (música) uma unidade para medir a duração de uma nota musical}
    \definition{v.}{tirar (uma foto); usar uma câmera para capturar imagens de pessoas e objetos em filme | dar um tapinha; bater suavemente com as mãos ou ferramentas | bater asas | bater (ondas do mar) | enviar (um telegrama, etc.) | bajular}
  \end{Phonetics}
\end{Entry}

\begin{Entry}{拍马}{8,3}{⼿、⾺}
  \begin{Phonetics}{拍马}{pai1ma3}
    \definition{v.}{instigar um cavalo dando tapinhas em seu traseiro | lisonjear | bajular}
  \seealsoref{拍马屁}{pai1ma3pi4}
  \end{Phonetics}
\end{Entry}

\begin{Entry}{拍马屁}{8,3,7}{⼿、⾺、⼫}
  \begin{Phonetics}{拍马屁}{pai1ma3pi4}
    \definition{s.}{puxa-saco | bajulador}
    \definition{v.}{puxar o saco | bajular}
  \seealsoref{拍马}{pai1ma3}
  \end{Phonetics}
\end{Entry}

\begin{Entry}{拍摄}{8,13}{⼿、⼿}
  \begin{Phonetics}{拍摄}{pai1 she4}[][HSK 5]
    \definition{s.}{fotografar; tirar (uma foto); usar uma câmera fotográfica para capturar imagens de pessoas e objetos}
  \end{Phonetics}
\end{Entry}

\begin{Entry}{拍照}{8,13}{⼿、⽕}
  \begin{Phonetics}{拍照}{pai1/zhao4}[][HSK 4]
    \definition{v.+compl.}{fotografar; tirar uma foto}
  \end{Phonetics}
\end{Entry}

\begin{Entry}{拐}{8}{⼿}
  \begin{Phonetics}{拐}{guai3}[][HSK 6]
    \definition[支,根,副]{s.}{muleta; bengala; uma bengala com uma barra horizontal na parte superior, usada por pessoas com doenças ou deficiências nos membros inferiores para ajudá-las a caminhar |
sete; forma falada do numeral 七 | esquina; curva; canto}
    \definition{v.}{virar; girar; mudar de direção enquanto se move | enganar | mudar; transformar | mancar}
  \seealsoref{七}{qi1}
  \end{Phonetics}
\end{Entry}

\begin{Entry}{拔}{8}{⼿}
  \begin{Phonetics}{拔}{ba2}[][HSK 5]
    \definition{v.aux.}{puxar para cima; puxar para fora; arrastar para fora | extrair; sugar | escolher; selecionar | superar; destacar-se entre | apreender; capturar | esfriar na água; mergulhar algo em água fria para que esfrie}
  \end{Phonetics}
\end{Entry}

\begin{Entry}{拔尖}{8,6}{⼿、⼩}
  \begin{Phonetics}{拔尖}{ba2/jian1}
    \definition{adj.}{topo de linha | fora do comum | o melhor}
    \definition{v.+compl.}{empurrar-se para a frente | sentir que é superior aos outros}
  \end{Phonetics}
\end{Entry}

\begin{Entry}{拖}{8}{⼿}
  \begin{Phonetics}{拖}{tuo1}[][HSK 6]
    \definition{v.}{puxar; arrastar; transportar; puxar um objeto para movê-lo contra o solo ou outra superfície | esfregar; limpar o chão com uma ferramenta especial para esfregar | atrasar; prolongar; procrastinar; arrastar; coisas que deveriam ser feitas nunca são iniciadas ou concluídas; uma certa nota é prolongada por um longo tempo | atrasar; conter; segurar; restringir}
  \end{Phonetics}
\end{Entry}

\begin{Entry}{拖拉机}{8,8,6}{⼿、⼿、⽊}
  \begin{Phonetics}{拖拉机}{tuo1la1ji1}
    \definition[台]{s.}{trator}
  \end{Phonetics}
\end{Entry}

\begin{Entry}{拖鞋}{8,15}{⼿、⾰}
  \begin{Phonetics}{拖鞋}{tuo1 xie2}[][HSK 6]
    \definition[双,只]{s.}{chinelos; samdálias; babouche; sapatos sem cabedal geralmente são usados ​​em ambientes fechados}
  \end{Phonetics}
\end{Entry}

\begin{Entry}{招}{8}{⼿}
  \begin{Phonetics}{招}{zhao1}[][HSK 6]
    \definition*{s.}{Sobrenome Zhao}
    \definition{s.}{\emph{banner}; Faixas e outros itens costumavam ser pendurados nas entradas de hotéis, restaurantes ou lojas para atrair clientes | movimento; estratagema; artifício; meios ou táticas | movimentos de artes marciais}
    \definition{v.}{acenar; gestuar para alguém ver | alistar; inscrever; recrutar | incorrer; cortejar; atrair; provocar (um certo resultado ou reação) | provocar; tocar ou provocar a outra pessoa com palavras ou ações | confessar (culpa); assumir (culpa) | infectar; ser contagioso}
  \end{Phonetics}
\end{Entry}

\begin{Entry}{招手}{8,4}{⼿、⼿}
  \begin{Phonetics}{招手}{zhao1/shou3}[][HSK 5]
    \definition{v.+compl.}{acenar; chamar a atenção; levantar a mão e acenar com a palma, para indicar que a outra pessoa se aproxime ou para cumprimentá-la}
  \end{Phonetics}
\end{Entry}

\begin{Entry}{招生}{8,5}{⼿、⽣}
  \begin{Phonetics}{招生}{zhao1/sheng1}[][HSK 5]
    \definition{v.+compl.}{conseguir alunos; matricular novos alunos; recrutar novos alunos}
  \end{Phonetics}
\end{Entry}

\begin{Entry}{招呼}{8,8}{⼿、⼝}
  \begin{Phonetics}{招呼}{zhao1 hu5}[][HSK 4]
    \definition{v.}{chamar; chamar a atenção com palavras ou gestos | cumprimentar; saudar; cumprimentar ou despedir-se das pessoas com palavras ou gestos | pedir a alguém para fazer algo; fazer solicitações, pedir ajuda ou fazer coisas | receber e dar boas-vindas aos convidados}
  \end{Phonetics}
\end{Entry}

\begin{Entry}{招数}{8,13}{⼿、⽁}
  \begin{Phonetics}{招数}{zhao1shu4}
    \definition{s.}{estratégia | movimento (no xadrez, no palco, nas artes marciais) | esquema | truque}
  \end{Phonetics}
\end{Entry}

\begin{Entry}{招聘}{8,13}{⼿、⽿}
  \begin{Phonetics}{招聘}{zhao1pin4}[][HSK 6]
    \definition{v.}{contratar; procurar; recrutar; convidar candidatos para um emprego}
  \end{Phonetics}
\end{Entry}

\begin{Entry}{拥}{8}{⼿}
  \begin{Phonetics}{拥}{yong1}
    \definition{v.}{segurar nos braços; abraçar | reunir em volta; envolver em volta | aglomerar-se; enxamear | para apoiar | (literário) ter; possuir}
  \end{Phonetics}
\end{Entry}

\begin{Entry}{拥有}{8,6}{⼿、⽉}
  \begin{Phonetics}{拥有}{yong1you3}[][HSK 5]
    \definition{v.}{possuir; deter; ter (grande quantidade de terras, população, bens, etc.)}
  \end{Phonetics}
\end{Entry}

\begin{Entry}{拥抱}{8,8}{⼿、⼿}
  \begin{Phonetics}{拥抱}{yong1bao4}[][HSK 5]
    \definition[个,次]{s.}{abraço}
    \definition{v.}{abraçar; segurar em seus braços; abraçar para demonstrar afeto}
  \end{Phonetics}
\end{Entry}

\begin{Entry}{拧}{8}{⼿}
  \begin{Phonetics}{拧}{ning2}
    \definition{v.}{torcer | beliscar; torcer a pele com os dedos e virá-la com força}
  \end{Phonetics}
  \begin{Phonetics}{拧}{ning3}
    \definition{adj.}{errado; equivocado; de cabeça para baixo; oposto}
    \definition{v.}{torcer; parafusar | divergir; discordar; estar em desacordo}
  \end{Phonetics}
  \begin{Phonetics}{拧}{ning4}
    \definition{adj.}{teimoso}
  \end{Phonetics}
\end{Entry}

\begin{Entry}{拧开}{8,4}{⼿、⼶}
  \begin{Phonetics}{拧开}{ning3kai1}
    \definition{v.}{desaparafusar | desatarrachar | torcer (uma tampa) | abrir (uma torneira) | ligar (girando um botão) | girar (maçaneta da porta)}
  \end{Phonetics}
\end{Entry}

\begin{Entry}{拨}{8}{⼿}
  \begin{Phonetics}{拨}{bo1}[][HSK 7-9]
    \definition{clas.}{usado para agrupar pessoas; grupo; lote}
    \definition{v.}{mover (mexer) com a mão, o pé, o bastão, etc.; usar as mãos, os pés ou os bastões para mover objetos | atribuir; alocar; reservar | virar-se; inverter a marcha | dedilhar (uma corda de violão) com os dedos ou com um instrumento | chamar (alguém)}
  \end{Phonetics}
\end{Entry}

\begin{Entry}{拨及}{8,3}{⼿、⼃}
  \begin{Phonetics}{拨及}{bo1ji2}[][HSK 7-9]
    \definition{v.}{espalhar para; envolver; afetar}
  \end{Phonetics}
\end{Entry}

\begin{Entry}{拨打}{8,5}{⼿、⼿}
  \begin{Phonetics}{拨打}{bo1 da3}[][HSK 6]
    \definition{v.}{ligar; discar; de acordo com o número da chamada, discar o número no telefone ou pressionar as teclas numéricas para fazer uma chamada}
  \end{Phonetics}
\end{Entry}

\begin{Entry}{拨转}{8,8}{⼿、⾞}
  \begin{Phonetics}{拨转}{bo1zhuan3}
    \definition{v.}{transferir (fundos, etc.) | virar | dar a volta}
  \end{Phonetics}
\end{Entry}

\begin{Entry}{拨通}{8,10}{⼿、⾡}
  \begin{Phonetics}{拨通}{bo1/tong1}[][HSK 7-9]
    \definition{v.+compl.}{discar (os números de um telefone, etc.)}
  \end{Phonetics}
\end{Entry}

\begin{Entry}{拨款}{8,12}{⼿、⽋}
  \begin{Phonetics}{拨款}{bo1kuan3}[][HSK 7-9]
    \definition[项,笔]{s.}{dinheiro apropriado; apropriação; subsídio financeiro do estado; alocação de fundos; financiamento alocado}
    \definition{v.}{apropriar-se de dinheiro; alocar fundos}
  \end{Phonetics}
\end{Entry}

\begin{Entry}{放}{8}{⽅}
  \begin{Phonetics}{放}{fang4}[][HSK 1]
    \definition{v.}{deixar ir; libertar; soltar | ceder; deixar-se levar | levar para se alimentar; pastar | soltar; liberar (ou expelir) | exibir (um filme, etc.); reproduzir (um disco, etc.) | acender; inflamar | emprestar (dinheiro) com juros | tornar maior ou mais longo; soltar; abaixar | moderar (a atitude ou o comportamento de alguém) | (de flores) florescer; abrir | colocar; posicionar; deitar | fazer com que algo (ou alguém) caia no chão | deixar de lado; guardar (para uso futuro); conservar | (seguido por 着\dots 不\dots) permitir que algo permaneça (por fazer, por pegar, por usar, etc.) | adicionar; colocar | colocar em pastagem; soltar para caçar | deixar de lado; suspender; interromper | remover; aliviar; livrar-se; proteger; libertar | deixar-se levar; sem restrições; libertino | mandar embora; tirar o prisioneiro da prisão e deportá-lo para uma região remota | distribuir; emitir; lançar | atear fogo | expandir; ampliar; prolongar | reajustar-se até certo ponto; controlar suas ações, adotar uma determinada atitude, atingir um certo equilíbrio | derrubar}
  \end{Phonetics}
\end{Entry}

\begin{Entry}{放下}{8,3}{⽅、⼀}
  \begin{Phonetics}{放下}{fang4 xia4}[][HSK 2]
    \definition{v.}{deitar-se; colocar no chão| deixar ir; soltar; desistir; largar | colocar; acomodar; depositar}
  \end{Phonetics}
\end{Entry}

\begin{Entry}{放大}{8,3}{⽅、⼤}
  \begin{Phonetics}{放大}{fang4da4}[][HSK 5]
    \definition{v.}{amplificar; magnificar; aumentar; ampliar; aumentar o tamanho de imagens, textos, sons, etc.}
  \end{Phonetics}
\end{Entry}

\begin{Entry}{放飞}{8,3}{⽅、⾶}
  \begin{Phonetics}{放飞}{fang4fei1}
    \definition{s.}{deixar voar}
  \end{Phonetics}
\end{Entry}

\begin{Entry}{放心}{8,4}{⽅、⼼}
  \begin{Phonetics}{放心}{fang4xin1}[][HSK 2]
    \definition{adj.}{despreocupado}
    \definition{v.}{confiar; ter confiança em alguém; sentir-se aliviado; ficar tranquilo; ficar com a consciência tranquila}
  \end{Phonetics}
\end{Entry}

\begin{Entry}{放出}{8,5}{⽅、⼐}
  \begin{Phonetics}{放出}{fang4chu1}
    \definition{v.}{liberar | libertar}
  \end{Phonetics}
\end{Entry}

\begin{Entry}{放电}{8,5}{⽅、⽥}
  \begin{Phonetics}{放电}{fang4dian4}
    \definition{s.}{descarga elétrica}
  \end{Phonetics}
\end{Entry}

\begin{Entry}{放任}{8,6}{⽅、⼈}
  \begin{Phonetics}{放任}{fang4ren4}
    \definition{v.}{ignorar | saciar-se | deixar sozinho}
  \end{Phonetics}
\end{Entry}

\begin{Entry}{放过}{8,6}{⽅、⾡}
  \begin{Phonetics}{放过}{fang4guo4}
    \definition{v.}{deixar | deixar alguém escapar impune | passar despercebido}
  \end{Phonetics}
\end{Entry}

\begin{Entry}{放弃}{8,7}{⽅、⼶}
  \begin{Phonetics}{放弃}{fang4qi4}[][HSK 5]
    \definition{v.}{desistir, abandonar; descartar (direitos originais, reivindicações, opiniões, etc.)}
  \end{Phonetics}
\end{Entry}

\begin{Entry}{放弃权利}{8,7,6,7}{⽅、⼶、⽊、⼑}
  \begin{Phonetics}{放弃权利}{fang4qi4 quan2li4}
    \definition{s.}{renúncia}
  \end{Phonetics}
\end{Entry}

\begin{Entry}{放弃者}{8,7,8}{⽅、⼶、⽼}
  \begin{Phonetics}{放弃者}{fang4qi4zhe3}
    \definition{s.}{desistente}
  \end{Phonetics}
\end{Entry}

\begin{Entry}{放走}{8,7}{⽅、⾛}
  \begin{Phonetics}{放走}{fang4zou3}
    \definition{v.}{permitir (uma pessoa ou um animal) ir | liberar | libertar}
  \end{Phonetics}
\end{Entry}

\begin{Entry}{放到}{8,8}{⽅、⼑}
  \begin{Phonetics}{放到}{fang4 dao4}[][HSK 3]
    \definition{v.}{colocar em; meter}
  \end{Phonetics}
\end{Entry}

\begin{Entry}{放学}{8,8}{⽅、⼦}
  \begin{Phonetics}{放学}{fang4/xue2}[][HSK 1]
    \definition{v.+compl.}{encerrar; sair da escola; as aulas terminaram; a escola acabou (por hoje); voltar para casa depois de um dia ou meio dia de aula}
  \end{Phonetics}
\end{Entry}

\begin{Entry}{放松}{8,8}{⽅、⽊}
  \begin{Phonetics}{放松}{fang4song1}[][HSK 4]
    \definition{v.}{relaxar; afrouxar; soltar; desprender}
  \end{Phonetics}
\end{Entry}

\begin{Entry}{放养}{8,9}{⽅、⼋}
  \begin{Phonetics}{放养}{fang4yang3}
    \definition{v.}{criar (gado, peixes, culturas, etc.) | crescer | criar}
  \end{Phonetics}
\end{Entry}

\begin{Entry}{放假}{8,11}{⽅、⼈}
  \begin{Phonetics}{放假}{fang4/jia4}[][HSK 1]
    \definition{v.}{tirar férias (ou feriado); ter um dia de folga}
    \definition{v.+compl.}{tirar férias (ou feriado); começar as férias; ter um dia de folga; estar de férias (feriado)}
  \end{Phonetics}
\end{Entry}

\begin{Entry}{放肆}{8,13}{⽅、⾀}
  \begin{Phonetics}{放肆}{fang4si4}
    \definition{adj.}{atrevido | pesunçoso | devasso}
  \end{Phonetics}
\end{Entry}

\begin{Entry}{放鞭炮}{8,18,9}{⽅、⾰、⽕}
  \begin{Phonetics}{放鞭炮}{fang4bian1pao4}
    \definition{s.}{um conjunto de bombinhas ou traques}
  \end{Phonetics}
\end{Entry}

\begin{Entry}{斩}{8}{⽄}
  \begin{Phonetics}{斩}{zhan3}
    \definition*{s.}{Sobrenome Zhan}
    \definition{v.}{matar; cortar; picar | (dialeto) tosquiar; chantagear | decapitar}
  \end{Phonetics}
\end{Entry}

\begin{Entry}{斩获}{8,10}{⽄、⾋}
  \begin{Phonetics}{斩获}{zhan3huo4}
    \definition{v.}{matar ou capturar (em batalha) | (figurativo) (esportes) marcar (um gol), ganhar (uma medalha) | (figurativo) colher recompensas, obter ganhos}
  \end{Phonetics}
\end{Entry}

\begin{Entry}{旺}{8}{⽇}
  \begin{Phonetics}{旺}{wang4}
    \definition{adj.}{próspero; florescente; vigoroso | abundante; numeroso}
  \end{Phonetics}
\end{Entry}

\begin{Entry}{旺季}{8,8}{⽇、⼦}
  \begin{Phonetics}{旺季}{wang4ji4}
    \definition{s.}{alta temporada; período de pico; temporada movimentada; a estação em que um determinado produto é produzido em grandes quantidades ou quando os negócios estão crescendo (diferente de 淡季)}
  \seealsoref{淡季}{dan4ji4}
  \end{Phonetics}
\end{Entry}

\begin{Entry}{昂}{8}{⽇}
  \begin{Phonetics}{昂}{ang2}
    \definition{adj.}{alto; subindo}
    \definition{v.}{manter (a cabeça) erguida | elevar; levantar; olhar para cima}
  \end{Phonetics}
\end{Entry}

\begin{Entry}{昂贵}{8,9}{⽇、⾙}
  \begin{Phonetics}{昂贵}{ang2gui4}[][HSK 7-9]
    \definition{adj.}{caro; dispendioso; algo é muito caro, o preço é particularmente alto; metaforicamente, o custo de fazer algo é particularmente alto}
  \end{Phonetics}
\end{Entry}

\begin{Entry}{昌}{8}{⽇}
  \begin{Phonetics}{昌}{chang1}
    \definition*{s.}{Sobrenome Chang}
    \definition{adj.}{próspero; florescente | adequado; bom}
  \end{Phonetics}
\end{Entry}

\begin{Entry}{昌盛}{8,11}{⽇、⽫}
  \begin{Phonetics}{昌盛}{chang1 sheng4}[][HSK 6]
    \definition{adj.}{(país, nação, etc.) próspero; florescente}
  \end{Phonetics}
\end{Entry}

\begin{Entry}{明}{8}{⽇}
  \begin{Phonetics}{明}{ming2}
    \definition*{s.}{Dinastia Ming (1368-1644) | Sobrenome Ming}
    \definition{adj.}{claro; brilhante; brilhante | claro; distinto; de fácil entendimento | aberto; evidente; explícito; exposto | de ​​olhos aguçados; boa visão; visão nítida | honesto}
    \definition{adv.}{claramente; definitivamente; aparentemente; de fato}
    \definition{s.}{imediatamente a seguir no tempo; ao lado deste ano e hoje; visão}
    \definition{v.}{mostrar; revelar; tornar conhecido; deixar claro | entender; compreender}
  \end{Phonetics}
\end{Entry}

\begin{Entry}{明天}{8,4}{⽇、⼤}
  \begin{Phonetics}{明天}{ming2tian1}[][HSK 1]
    \definition{s.}{amanhã | futuro próximo}
  \end{Phonetics}
\end{Entry}

\begin{Entry}{明日}{8,4}{⽇、⽇}
  \begin{Phonetics}{明日}{ming2 ri4}[][HSK 6]
    \definition{s.}{amanhã}
  \seealsoref{明天}{ming2tian1}
  \end{Phonetics}
\end{Entry}

\begin{Entry}{明白}{8,5}{⽇、⽩}
  \begin{Phonetics}{明白}{ming2bai5}[][HSK 1]
    \definition{adj.}{claro; óbvio; evidente; inequívoco | sensato; razoável | aberto; franco; inequívoco; explícito}
    \definition{v.}{entender; compreender; saber}
  \end{Phonetics}
\end{Entry}

\begin{Entry}{明年}{8,6}{⽇、⼲}
  \begin{Phonetics}{明年}{ming2 nian2}[][HSK 1]
    \definition{s.}{próximo ano}
  \end{Phonetics}
\end{Entry}

\begin{Entry}{明明}{8,8}{⽇、⽇}
  \begin{Phonetics}{明明}{ming2ming2}[][HSK 5]
    \definition{adv.}{obviamente; claramente; sem dúvida; indica que o fenômeno ou princípio é evidente}
  \end{Phonetics}
\end{Entry}

\begin{Entry}{明码}{8,8}{⽇、⽯}
  \begin{Phonetics}{明码}{ming2ma3}
    \definition{s.}{código simples, em claro (oposto a 密码) | preço claramente marcado}
  \seealsoref{密码}{mi4ma3}
  \end{Phonetics}
\end{Entry}

\begin{Entry}{明亮}{8,9}{⽇、⼇}
  \begin{Phonetics}{明亮}{ming2 liang4}[][HSK 5]
    \definition{adj.}{claro; bem iluminado | brilhante; resplandecente | claro; simples; compreensível}
  \end{Phonetics}
\end{Entry}

\begin{Entry}{明星}{8,9}{⽇、⽇}
  \begin{Phonetics}{明星}{ming2xing1}[][HSK 2]
    \definition[个,位,颗,名]{s.}{estrela; ator, atleta, cantor famosos, etc. | talento de ponta; profissional de destaque; também é usado como metáfora para pessoas ou grupos que se destacam pelo seu bom desempenho ou excelência | estrela brilhante; estrela resplandecente; referindo-se a estrelas muito brilhantes}
  \end{Phonetics}
\end{Entry}

\begin{Entry}{明显}{8,9}{⽇、⽇}
  \begin{Phonetics}{明显}{ming2xian3}[][HSK 3]
    \definition{adj.}{claro; óbvio; distinto; claramente visível}
  \end{Phonetics}
\end{Entry}

\begin{Entry}{明珠}{8,10}{⽇、⽟}
  \begin{Phonetics}{明珠}{ming2zhu1}
    \definition{s.}{pérola | jóia (de grande valor)}
  \end{Phonetics}
\end{Entry}

\begin{Entry}{明确}{8,12}{⽇、⽯}
  \begin{Phonetics}{明确}{ming2que4}[][HSK 3]
    \definition{adj.}{claro; definido; específico}
    \definition{v.}{deixar claro; tornar definitivo; tornar um ponto de vista, uma tarefa, etc. claro, compreensível e definitivo}
  \end{Phonetics}
\end{Entry}

\begin{Entry}{昏}{8}{⽇}
  \begin{Phonetics}{昏}{hun1}
    \definition*{s.}{Sobrenome Hun}
    \definition{adj.}{escuro; fraco; embaçado | confuso; embaraçado; inconsciente}
    \definition{s.}{crepúsculo; tarde}
    \definition{v.}{perder a consciência; desmaiar}
  \end{Phonetics}
\end{Entry}

\begin{Entry}{易}{8}{⽇}
  \begin{Phonetics}{易}{yi4}
    \definition*{s.}{Sobrenome Yi}
    \definition{adj.}{fácil | amigável; pacífico}
    \definition{v.}{modificar; transformar | trocar | subestimar; desprezar}
  \end{Phonetics}
\end{Entry}

\begin{Entry}{昔}{8}{⽇}
  \begin{Phonetics}{昔}{xi1}
    \definition{s.}{tempos antigos; o passado; era uma vez}
  \end{Phonetics}
\end{Entry}

\begin{Entry}{昔日}{8,4}{⽇、⽇}
  \begin{Phonetics}{昔日}{xi1ri4}
    \definition{adj.}{passado}
  \end{Phonetics}
\end{Entry}

\begin{Entry}{朋}{8}{⽉}
  \begin{Phonetics}{朋}{peng2}
    \definition*{s.}{Sobrenome Peng}
    \definition{s.}{amigo}
    \definition{v.}{(literário) rivalizar; igualar; comparar | (literário) reunir-se em grupo; juntar-se em grupo}
  \end{Phonetics}
\end{Entry}

\begin{Entry}{朋友}{8,4}{⽉、⼜}
  \begin{Phonetics}{朋友}{peng2you5}[][HSK 1]
    \definition[个,位,帮,群]{s.}{amigo; pessoas que têm um bom relacionamento, uma boa relação, se entendem e se ajudam mutuamente | namorado; namorada}
  \end{Phonetics}
\end{Entry}

\begin{Entry}{服}{8}{⽉}
  \begin{Phonetics}{服}{fu2}[][HSK 6]
    \definition*{s.}{Sobrenome Fu}
    \definition{s.}{roupas | vestuário de luto; refere-se a roupas de luto}
    \definition{v.}{vestir (roupas) | tomar (remédio) | envolver-se em; servir | obedecer; ser convencido | convencer; persuadir | adaptar-se; acostumar-se a}
  \end{Phonetics}
  \begin{Phonetics}{服}{fu4}
    \definition{clas.}{usado para remédio: dose; usado na medicina tradicional chinesa}
  \end{Phonetics}
\end{Entry}

\begin{Entry}{服从}{8,4}{⽉、⼈}
  \begin{Phonetics}{服从}{fu2cong2}[][HSK 5]
    \definition{v.}{obedecer; submeter-se a; estar subordinado a}
  \end{Phonetics}
\end{Entry}

\begin{Entry}{服务}{8,5}{⽉、⼒}
  \begin{Phonetics}{服务}{fu2 wu4}[][HSK 2]
    \definition{v.}{prestar serviço a; estar a serviço de; servir; trabalhar para o benefício coletivo (ou de outras pessoas) ou para uma causa específica | trabalhar; servir}
  \end{Phonetics}
\end{Entry}

\begin{Entry}{服务员}{8,5,7}{⽉、⼒、⼝}
  \begin{Phonetics}{服务员}{fu2wu4yuan2}
    \definition{s.}{atendente | garçom | garçonete | pessoal de atendimento ao cliente}
  \end{Phonetics}
\end{Entry}

\begin{Entry}{服装}{8,12}{⽉、⾐}
  \begin{Phonetics}{服装}{fu2zhuang1}[][HSK 3]
    \definition[套,件,身]{s.}{roupas; vestuário; trajes; termo genérico para roupas, sapatos e chapéus, geralmente referido especificamente a roupas}
  \end{Phonetics}
\end{Entry}

\begin{Entry}{杯}{8}{⽊}
  \begin{Phonetics}{杯}{bei1}[][HSK 1]
    \definition{clas.}{para certos recipientes de líquidos: copo, xícara, etc.}
    \definition[只,个]{s.}{copo; caneca; xícara | taça; troféu; prêmio em forma de taça}
  \end{Phonetics}
\end{Entry}

\begin{Entry}{杯子}{8,3}{⽊、⼦}
  \begin{Phonetics}{杯子}{bei1 zi5}[][HSK 1]
    \definition[个,只,种]{s.}{xícara; copo; recipiente para bebidas ou outros líquidos, geralmente cilíndrico ou com a parte inferior ligeiramente mais estreita, com capacidade geralmente pequena}
  \end{Phonetics}
\end{Entry}

\begin{Entry}{杯具}{8,8}{⽊、⼋}
  \begin{Phonetics}{杯具}{bei1ju4}
    \definition{s.}{parachoque | fiasco | (gíria) tragédia}
  \end{Phonetics}
\end{Entry}

\begin{Entry}{杰}{8}{⽊}
  \begin{Phonetics}{杰}{jie2}
    \definition{adj.}{notável; proeminente; fora do comum}
    \definition[位,名,个,些]{s.}{pessoa excepcional; herói; uma pessoa com talentos excepcionais}
  \end{Phonetics}
\end{Entry}

\begin{Entry}{杰出}{8,5}{⽊、⼐}
  \begin{Phonetics}{杰出}{jie2chu1}[][HSK 6]
    \definition{adj.}{notável; proeminente; (talento, realização) excepcional}
  \end{Phonetics}
\end{Entry}

\begin{Entry}{松}{8}{⽊}
  \begin{Phonetics}{松}{song1}[][HSK 4]
    \definition*{s.}{Sobrenome Song}
    \definition{adj.}{solto; frouxo; folgado | leve e crocante; macio | relaxado; confortável}
    \definition[棵]{s.}{pinheiro | fio de carne seca; carne moída seca; alimentos macios ou quebradiços}
    \definition{v.}{afrouxar; relaxar; abrandar | desamarrar; desatar; liberar}
  \end{Phonetics}
\end{Entry}

\begin{Entry}{松木}{8,4}{⽊、⽊}
  \begin{Phonetics}{松木}{song1mu4}
    \definition{s.}{pinheiro}
  \end{Phonetics}
\end{Entry}

\begin{Entry}{松树}{8,9}{⽊、⽊}
  \begin{Phonetics}{松树}{song1 shu4}[][HSK 4]
    \definition[棵]{s.}{pinheiro; conífera comum, geralmente com folhas longas e pontiagudas e cones lenhosos}
  \end{Phonetics}
\end{Entry}

\begin{Entry}{板}{8}{⽊}
  \begin{Phonetics}{板}{ban3}[][HSK 3]
    \definition{adj.}{rígido; não natural; inflexível}
    \definition[块,个]{s.}{tábua; placa; prato; objeto rígido em forma de placa | veneziana; persiana; refere-se especificamente aos painéis de portas de lojas | badalos (instrumento musical que marca o ritmo) | uma batida acentuada (ritmo) na música e na ópera tradicional | chefe}
    \definition{v.}{parecer sério | corrigir maus hábitos ou defeitos | ser rígido como uma tábua}
  \end{Phonetics}
\end{Entry}

\begin{Entry}{板块}{8,7}{⽊、⼟}
  \begin{Phonetics}{板块}{ban3kuai4}[][HSK 7-9]
    \definition[个]{s.}{placa tectônica; segmentos móveis da crosta terrestre | seção; uma metáfora para uma combinação de partes que têm algo em comum ou conectado}
  \end{Phonetics}
\end{Entry}

\begin{Entry}{构}{8}{⽊}
  \begin{Phonetics}{构}{gou4}
    \definition{s.}{composição literária}
    \definition{v.}{construir | formar | compor}
    \variantof{够}
  \end{Phonetics}
\end{Entry}

\begin{Entry}{构成}{8,6}{⽊、⼽}
  \begin{Phonetics}{构成}{gou4cheng2}[][HSK 4]
    \definition{s.}{parte; componente; composição; estrutura}
    \definition{v.}{formar; compor; constituir; compor; encaixar muitas partes para formar um todo | consistir; causar; formar (principalmente em termos jurídicos)}
  \end{Phonetics}
\end{Entry}

\begin{Entry}{构建}{8,8}{⽊、⼵}
  \begin{Phonetics}{构建}{gou4 jian4}[][HSK 6]
    \definition{v.}{estabelecer (usado principalmente para coisas abstratas); montar; instalar}
  \end{Phonetics}
\end{Entry}

\begin{Entry}{构造}{8,10}{⽊、⾡}
  \begin{Phonetics}{构造}{gou4 zao4}[][HSK 4]
    \definition[种]{s.}{estrutura; construção; disposição, organização e inter-relação dos componentes}
    \definition{v.}{formar; construir}
  \end{Phonetics}
\end{Entry}

\begin{Entry}{枕}{8}{⽊}
  \begin{Phonetics}{枕}{zhen3}
    \definition*{s.}{Sobrenome Zhen}
    \definition[个]{s.}{travesseiro; almofada | Mecânica: bloco}
    \definition{v.}{descansar a cabeça no travesseiro, almofada}
  \end{Phonetics}
\end{Entry}

\begin{Entry}{果}{8}{⽊}
  \begin{Phonetics}{果}{guo3}
    \definition*{s.}{Sobrenome Guo}
    \definition{adj.}{resoluto; determinado; sem exitação}
    \definition{adv.}{realmente; como esperado; com certeza; isso significa que as coisas são consistentes com as expectativas, equivalente a 果然}
    \definition{conj.}{se realmente; se de fato}
    \definition[个,些,种]{s.}{fruta; fruto da planta | resultado; consequência; o resultado final de um assunto (em oposição à 因)}
  \seealsoref{果然}{guo3ran2}
  \seealsoref{因}{yin1}
  \end{Phonetics}
\end{Entry}

\begin{Entry}{果子}{8,3}{⽊、⼦}
  \begin{Phonetics}{果子}{guo3zi5}
    \definition{s.}{fruta}
  \end{Phonetics}
\end{Entry}

\begin{Entry}{果汁}{8,5}{⽊、⽔}
  \begin{Phonetics}{果汁}{guo3zhi1}[][HSK 3]
    \definition[杯,瓶,种]{s.}{suco; suco de frutas frescas; também se refere a bebidas feitas com suco de frutas frescas}
  \end{Phonetics}
\end{Entry}

\begin{Entry}{果实}{8,8}{⽊、⼧}
  \begin{Phonetics}{果实}{guo3shi2}[][HSK 4]
    \definition[种]{s.}{fruta; o órgão que se desenvolve a partir do ovário ou com outras partes da flor após a fertilização da flor | ganhos; frutos;  uma metáfora para conquista ou recompensa por trabalho árduo}
  \end{Phonetics}
\end{Entry}

\begin{Entry}{果树}{8,9}{⽊、⽊}
  \begin{Phonetics}{果树}{guo3 shu4}[][HSK 6]
    \definition[棵,个,片]{s.}{árvore frutífera; árvores cujos frutos são principalmente comestíveis, como pessegueiros e macieiras}
  \end{Phonetics}
\end{Entry}

\begin{Entry}{果然}{8,12}{⽊、⽕}
  \begin{Phonetics}{果然}{guo3ran2}[][HSK 3]
    \definition{adv.}{realmente; como esperado; com certeza; indica que os fatos correspondem ao que foi dito ou esperado}
    \definition{conj.}{se realmente; se de fato; suponha que os fatos correspondam ao que foi dito ou esperado}
  \end{Phonetics}
\end{Entry}

\begin{Entry}{果酱}{8,13}{⽊、⾣}
  \begin{Phonetics}{果酱}{guo3 jiang4}[][HSK 6]
    \definition{s.}{geléia | compota ou doce (de frutas); fruta em conserva}
  \end{Phonetics}
\end{Entry}

\begin{Entry}{枝}{8}{⽊}
  \begin{Phonetics}{枝}{zhi1}[][HSK 6]
    \definition*{s.}{Sobrenome Zhi}
    \definition{clas.}{usado para flores com galhos, ramos | usado para objetos em forma de haste}
    \definition{s.}{ramo; galho}
  \end{Phonetics}
\end{Entry}

\begin{Entry}{枪}{8}{⽊}
  \begin{Phonetics}{枪}{qiang1}[][HSK 5]
    \definition*{s.}{Sobrenome Qiang}
    \definition[把,杆,支,挺]{s.}{lança | arma; rifle; arma de fogo | uma coisa em forma de arma | enxada; ferramenta para cavar a terra}
    \definition{v.}{escrever artigos ou responder perguntas para outras pessoas}
  \end{Phonetics}
\end{Entry}

\begin{Entry}{枫}{8}{⽊}
  \begin{Phonetics}{枫}{feng1}
    \definition[棵]{s.}{goma doce chinesa | bordo; \emph{maple}}
  \end{Phonetics}
\end{Entry}

\begin{Entry}{枫叶}{8,5}{⽊、⼝}
  \begin{Phonetics}{枫叶}{feng1ye4}
    \definition{s.}{folha de bordo (maple, tipo de árvore)}
  \end{Phonetics}
\end{Entry}

\begin{Entry}{柜}{8}{⽊}
  \begin{Phonetics}{柜}{gui4}
    \definition{s.}{baú; armário; gabinete | loja; balcão}
  \end{Phonetics}
  \begin{Phonetics}{柜}{ju3}
    \definition{s.}{faia; salgueiro}
  \end{Phonetics}
\end{Entry}

\begin{Entry}{柜子}{8,3}{⽊、⼦}
  \begin{Phonetics}{柜子}{gui4 zi5}[][HSK 5]
    \definition[个]{s.}{gabinete; armário; dispositivo para guardar roupas, documentos, livros, etc.}
  \end{Phonetics}
\end{Entry}

\begin{Entry}{欣}{8}{⽋}
  \begin{Phonetics}{欣}{xin1}
    \definition*{s.}{Sobrenome Xin}
    \definition{adj.}{alegre; feliz; contente}
  \end{Phonetics}
\end{Entry}

\begin{Entry}{欣赏}{8,12}{⽋、⾙}
  \begin{Phonetics}{欣赏}{xin1shang3}[][HSK 5]
    \definition{v.}{apreciar; admirar; valorizar; apreciar as coisas boas e descubrir o prazer que elas proporcionam | apreciar; gostar; considerar bom}
  \end{Phonetics}
\end{Entry}

\begin{Entry}{欧}{8}{⽋}
  \begin{Phonetics}{欧}{ou1}
    \definition*{s.}{Europa, abreviação de 欧洲 | Sobrenome Ou}
  \seealsoref{欧洲}{ou1zhou1}
  \end{Phonetics}
\end{Entry}

\begin{Entry}{欧洲}{8,9}{⽋、⽔}
  \begin{Phonetics}{欧洲}{ou1zhou1}
    \definition*{s.}{Europa}
  \end{Phonetics}
\end{Entry}

\begin{Entry}{欧洲人}{8,9,2}{⽋、⽔、⼈}
  \begin{Phonetics}{欧洲人}{ou1zhou1ren2}
    \definition{s.}{europeu | pessoa ou povo da Europa}
  \end{Phonetics}
\end{Entry}

\begin{Entry}{欧洲共同体}{8,9,6,6,7}{⽋、⽔、⼋、⼝、⼈}
  \begin{Phonetics}{欧洲共同体}{ou1zhou1 gong4tong2ti3}
    \definition*{s.}{Comunidade Europeia}
  \end{Phonetics}
\end{Entry}

\begin{Entry}{欧盟}{8,13}{⽋、⽫}
  \begin{Phonetics}{欧盟}{ou1meng2}
    \definition*{s.}{União Europeia (EU)}
  \end{Phonetics}
\end{Entry}

\begin{Entry}{武}{8}{⽌}
  \begin{Phonetics}{武}{wu3}
    \definition*{s.}{Sobrenome Wu}
    \definition{adj.}{valente; corajoso}
    \definition{s.}{militar; atividades e comportamentos relacionados a habilidades militares e de combate (em oposição a 文) | arte marcial | passo; meio passo; pegadas}
  \seealsoref{文}{wen2}
  \end{Phonetics}
\end{Entry}

\begin{Entry}{武力}{8,2}{⽌、⼒}
  \begin{Phonetics}{武力}{wu3li4}
    \definition{s.}{forças armadas | militares}
  \end{Phonetics}
\end{Entry}

\begin{Entry}{武士}{8,3}{⽌、⼠}
  \begin{Phonetics}{武士}{wu3shi4}
    \definition{s.}{samurai | guerreiro}
  \end{Phonetics}
\end{Entry}

\begin{Entry}{武大戏}{8,3,6}{⽌、⼤、⼽}
  \begin{Phonetics}{武大戏}{wu3 da4xi4}
    \definition*{s.}{Drama de Luta Acrobática | Drama Wu}
  \end{Phonetics}
\end{Entry}

\begin{Entry}{武艺}{8,4}{⽌、⾋}
  \begin{Phonetics}{武艺}{wu3yi4}
    \definition{s.}{arte marcial | habilidade militar}
  \end{Phonetics}
\end{Entry}

\begin{Entry}{武术}{8,5}{⽌、⽊}
  \begin{Phonetics}{武术}{wu3shu4}[][HSK 3]
    \definition[种,套,门]{s.}{arte marcial; autodefesa; \emph{wushu}; um esporte tradicional chinês que utiliza técnicas com os punhos, pernas, pés ou armas como facas e espadas}
  \end{Phonetics}
\end{Entry}

\begin{Entry}{武官}{8,8}{⽌、⼧}
  \begin{Phonetics}{武官}{wu3guan1}
    \definition{s.}{oficial militar}
  \end{Phonetics}
\end{Entry}

\begin{Entry}{武断}{8,11}{⽌、⽄}
  \begin{Phonetics}{武断}{wu3duan4}
    \definition{adj.}{arbitrário | dogmático | subjetivo}
  \end{Phonetics}
\end{Entry}

\begin{Entry}{武装}{8,12}{⽌、⾐}
  \begin{Phonetics}{武装}{wu3zhuang1}
    \definition{s.}{forças armadas | militar | arma}
    \definition{v.}{armar}
  \end{Phonetics}
\end{Entry}

\begin{Entry}{武器}{8,16}{⽌、⼝}
  \begin{Phonetics}{武器}{wu3qi4}[][HSK 3]
    \definition[批,个,件,种]{s.}{arma; equipamentos e dispositivos utilizados diretamente para matar inimigos ou destruir suas instalações defensivas e ofensivas | armas; armamento; metáfora usada como ferramenta de luta}
  \end{Phonetics}
\end{Entry}

\begin{Entry}{河}{8}{⽔}
  \begin{Phonetics}{河}{he2}[][HSK 2]
    \definition*{s.}{Astronomia: o sistema da Via Láctea | O Rio Amarelo; O Rio Huanghe | Sobrenome He}
    \definition[条,道]{s.}{rio; refere-se a grandes cursos de água}
  \end{Phonetics}
\end{Entry}

\begin{Entry}{河南}{8,9}{⽔、⼗}
  \begin{Phonetics}{河南}{he2nan2}
    \definition*{s.}{Província de Henan}
  \end{Phonetics}
\end{Entry}

\begin{Entry}{河蚌}{8,10}{⽔、⾍}
  \begin{Phonetics}{河蚌}{he2bang4}
    \definition{s.}{mexilhões | bivalves cultivados em rios e lagos}
  \end{Phonetics}
\end{Entry}

\begin{Entry}{油}{8}{⽔}
  \begin{Phonetics}{油}{you2}[][HSK 2]
    \definition*{s.}{Sobrenome You}
    \definition{adj.}{oleoso; gorduroso}
    \definition[瓶,滴,层]{s.}{óleo; gordura; graxa; petróleo}
    \definition{v.}{aplicar óleo de tungue, verniz ou tinta | estar manchado ou sujo com óleo ou graxa | aplicar óleo de tungue ou tinta}
  \end{Phonetics}
\end{Entry}

\begin{Entry}{治}{8}{⽔}
  \begin{Phonetics}{治}{zhi4}[][HSK 4]
    \definition*{s.}{Sobrenome Zhi}
    \definition{adj.}{calmo e pacífico}
    \definition{s.}{sede de um antigo governo local}
    \definition{v.}{reger; administrar; governar; gerenciar; gerir | tratar (uma doença); curar; sarar | eliminar; controlar pragas | controlar (um rio); restaurar um curso d'água por meio de dragagem | punir; castigar | estudar; pesquisar; explorar}
  \end{Phonetics}
\end{Entry}

\begin{Entry}{治安}{8,6}{⽔、⼧}
  \begin{Phonetics}{治安}{zhi4'an1}[][HSK 5]
    \definition{s.}{ordem pública; segurança pública; ordem social estável}
  \end{Phonetics}
\end{Entry}

\begin{Entry}{治疗}{8,7}{⽔、⽧}
  \begin{Phonetics}{治疗}{zhi4liao2}[][HSK 4]
    \definition{s.}{diagnóstico; tratamento}
    \definition{v.}{tratar; curar; remediar; eliminar doenças por meio de medicamentos, cirurgia, etc.}
  \end{Phonetics}
\end{Entry}

\begin{Entry}{治病}{8,10}{⽔、⽧}
  \begin{Phonetics}{治病}{zhi4 bing4}[][HSK 6]
    \definition{v.}{tratar uma doença; eliminar doenças por meio de medicamentos, cirurgias, etc.}
  \end{Phonetics}
\end{Entry}

\begin{Entry}{治理}{8,11}{⽔、⽟}
  \begin{Phonetics}{治理}{zhi4li3}[][HSK 5]
    \definition{s.}{governo | governança}
    \definition{v.}{dirigir; gerenciar; governar; administrar | tratar; aproveitar; colocar sob controle; colocar em ordem}
  \end{Phonetics}
\end{Entry}

\begin{Entry}{治愈}{8,13}{⽔、⼼}
  \begin{Phonetics}{治愈}{zhi4yu4}
    \definition{v.}{curar | restaurar a saúde}
  \end{Phonetics}
\end{Entry}

\begin{Entry}{沿}{8}{⽔}
  \begin{Phonetics}{沿}{yan2}[][HSK 6]
    \definition{prep.}{ao longo}
    \definition{s.}{beira; borda; acabamento}
    \definition{v.}{seguir (uma tradição, padrão, etc.) | enfeitar (com fita, faixa, etc.)}
  \end{Phonetics}
\end{Entry}

\begin{Entry}{沿海}{8,10}{⽔、⽔}
  \begin{Phonetics}{沿海}{yan2hai3}[][HSK 6]
    \definition{s.}{costa; litoral; área ou região ao longo da costa}
  \end{Phonetics}
\end{Entry}

\begin{Entry}{沿着}{8,11}{⽔、⽬}
  \begin{Phonetics}{沿着}{yan2 zhe5}[][HSK 6]
    \definition{prep.}{ao longo (de uma determinada rota)}
  \end{Phonetics}
\end{Entry}

\begin{Entry}{泄}{8}{⽔}
  \begin{Phonetics}{泄}{xie4}
    \definition*{s.}{Sobrenome Xie}
    \definition{v.}{deixar sair (um fluido ou gás); descarregar; liberar | revelar (um segredo); vazar (notícias, segredos, etc.) | dar vazão a; desabafar}
  \end{Phonetics}
\end{Entry}

\begin{Entry}{泄气}{8,4}{⽔、⽓}
  \begin{Phonetics}{泄气}{xie4/qi4}
    \definition{adj.}{decepcionante | frustrante | patético}
    \definition{v.+compl.}{perder o coração | sentir-se desencorajado | ficar desanimado}
  \end{Phonetics}
\end{Entry}

\begin{Entry}{泄底}{8,8}{⽔、⼴}
  \begin{Phonetics}{泄底}{xie4di3}
    \definition{v.}{revelar ou expor o que está no fundo de algo | divulgar a história interna; vazar segredos}
  \end{Phonetics}
\end{Entry}

\begin{Entry}{泄洪}{8,9}{⽔、⽔}
  \begin{Phonetics}{泄洪}{xie4hong2}
    \definition{v.}{liberar água da enchente (descarga de inundação)}
  \end{Phonetics}
\end{Entry}

\begin{Entry}{泄愤}{8,12}{⽔、⼼}
  \begin{Phonetics}{泄愤}{xie4fen4}
    \definition{v.}{dar vazão à raiva}
  \end{Phonetics}
\end{Entry}

\begin{Entry}{泄露}{8,21}{⽔、⾬}
  \begin{Phonetics}{泄露}{xie4lou4}
    \definition{v.}{vazar; deixar escapar; divulgar; revelar (um segredo, etc.) | vazar; escapar; descarregar (um fluido ou gás)}
  \end{Phonetics}
\end{Entry}

\begin{Entry}{法}{8}{⽔}
  \begin{Phonetics}{法}{fa3}[][HSK 4]
    \definition*{s.}{Doutrina budista; o dharma | França, abreviação de 法国 | Sobrenome Fa}
    \definition{adj.}{(usado após advérbios negativos) legal; cumpridor da lei}
    \definition{clas.}{F; Farad, medida de capacitância}
    \definition{s.}{lei; termo geral para regras de comportamento estabelecidas ou endossadas pelo Estado | maneira; método; modo; meios | padrão; modelo | artes mágicas; feitiço}
    \definition{v.}{seguir; imitar; aprender (os pontos fortes dos outros) |}
  \seealsoref{法国}{fa3guo2}
  \end{Phonetics}
\end{Entry}

\begin{Entry}{法文}{8,4}{⽔、⽂}
  \begin{Phonetics}{法文}{fa3wen2}
    \definition[份]{s.}{françês, língua francesa}
  \end{Phonetics}
\end{Entry}

\begin{Entry}{法网}{8,6}{⽔、⽹}
  \begin{Phonetics}{法网}{fa3wang3}
    \definition*{s.}{Torneio de Roland Garros (French Open), torneio de tênis}
  \end{Phonetics}
\end{Entry}

\begin{Entry}{法制}{8,8}{⽔、⼑}
  \begin{Phonetics}{法制}{fa3 zhi4}[][HSK 5]
    \definition{s.}{legalidade; instituições jurídicas; sistema jurídico}
  \end{Phonetics}
\end{Entry}

\begin{Entry}{法国}{8,8}{⽔、⼞}
  \begin{Phonetics}{法国}{fa3guo2}
    \definition*{s.}{França}
  \end{Phonetics}
\end{Entry}

\begin{Entry}{法国人}{8,8,2}{⽔、⼞、⼈}
  \begin{Phonetics}{法国人}{fa3guo2ren2}
    \definition{s.}{francês | pessoa ou povo da França}
  \end{Phonetics}
\end{Entry}

\begin{Entry}{法官}{8,8}{⽔、⼧}
  \begin{Phonetics}{法官}{fa3 guan1}[][HSK 4]
    \definition[位,名,个,些]{s.}{juiz; justiça; termo genérico para um membro do judiciário em um tribunal de justiça}
  \end{Phonetics}
\end{Entry}

\begin{Entry}{法规}{8,8}{⽔、⾒}
  \begin{Phonetics}{法规}{fa3 gui1}[][HSK 5]
    \definition[部,项,条,套,个]{s.}{lei e regulamento; estatuto; termo geral para leis, decretos, regulamentos, regras, estatutos, etc.}
  \end{Phonetics}
\end{Entry}

\begin{Entry}{法庭}{8,9}{⽔、⼴}
  \begin{Phonetics}{法庭}{fa3 ting2}[][HSK 6]
    \definition{s.}{corte; tribunal | tribunal; um órgão estatal que exerce o poder judicial de forma independente}
  \end{Phonetics}
\end{Entry}

\begin{Entry}{法律}{8,9}{⽔、⼻}
  \begin{Phonetics}{法律}{fa3lv4}[][HSK 4]
    \definition[项,条,套,个]{s.}{lei; estatuto; regras de conduta formuladas pelo legislativo e cuja aplicação é garantida pelo poder estatal}
  \end{Phonetics}
\end{Entry}

\begin{Entry}{法语}{8,9}{⽔、⾔}
  \begin{Phonetics}{法语}{fa3 yu3}[][HSK 6]
    \definition[种,门,句,段]{s.}{françês, língua francesa}
  \end{Phonetics}
\end{Entry}

\begin{Entry}{法院}{8,9}{⽔、⾩}
  \begin{Phonetics}{法院}{fa3yuan4}[][HSK 4]
    \definition[所,座]{s.}{tribunal; corte; órgãos estatais que exercem poder judicial independente}
  \end{Phonetics}
\end{Entry}

\begin{Entry}{泡}{8}{⽔}
  \begin{Phonetics}{泡}{pao1}
    \definition{adj.}{esponjoso; oco e macio; não duro}
    \definition{clas.}{usado para fezes e urina}
    \definition[串,个]{s.}{algo fofo e macio | pequeno lago}
  \end{Phonetics}
  \begin{Phonetics}{泡}{pao4}[][HSK 6]
    \definition[串,个]{s.}{bolha | algo em forma de bolha}
    \definition{v.}{mergulhar; encharcar | despejar água fervente em (chá, sopa instantânea, etc.) | enrolar; demorar-se; ficar por aí | (coloquial) (de um homem) brincar no campo; brincar com uma mulher | perder tempo; matar o tempo deliberadamente}
  \end{Phonetics}
\end{Entry}

\begin{Entry}{波}{8}{⽔}
  \begin{Phonetics}{波}{bo1}
    \definition*{s.}{Polônia, abreviação de 波兰 | Sobrenome Bo}
    \definition{s.}{ondas, a superfície irregular da água em rios, lagos e oceanos | onda, o processo de propagação da vibração | mudanças inesperadas; uma reviravolta inesperada nos acontecimentos; metáfora para mudanças inesperadas nas coisas | olhos; metáfora do olhar errante}
  \seealsoref{波兰}{bo1lan2}
  \end{Phonetics}
\end{Entry}

\begin{Entry}{波兰}{8,5}{⽔、⼋}
  \begin{Phonetics}{波兰}{bo1lan2}
    \definition*{s.}{Polônia}
  \end{Phonetics}
\end{Entry}

\begin{Entry}{波动}{8,6}{⽔、⼒}
  \begin{Phonetics}{波动}{bo1 dong4}[][HSK 6]
    \definition{s.}{ondulação; flutuação; movimento de onda}
    \definition{v.}{ondular; flutuar}
  \end{Phonetics}
\end{Entry}

\begin{Entry}{波折}{8,7}{⽔、⼿}
  \begin{Phonetics}{波折}{bo1zhe2}[][HSK 7-9]
    \definition{s.}{reviravoltas; contratempo; as reviravoltas que ocorrem durante o curso das coisas, o que significa que você sofre dificuldades ou contratempos}
  \end{Phonetics}
\end{Entry}

\begin{Entry}{波音}{8,9}{⽔、⾳}
  \begin{Phonetics}{波音}{bo1yin1}
    \definition*{s.}{Boeing (empresa aeroespacial)}
    \definition{s.}{mordente (música)}
  \end{Phonetics}
\end{Entry}

\begin{Entry}{波浪}{8,10}{⽔、⽔}
  \begin{Phonetics}{波浪}{bo1lang4}[][HSK 6]
    \definition{s.}{onda; a superfície irregular da água nos rios, lagos e oceanos, geralmente se refere a águas menores e mais bonitas, frequentemente usada na linguagem falada}
  \end{Phonetics}
\end{Entry}

\begin{Entry}{波涛}{8,10}{⽔、⽔}
  \begin{Phonetics}{波涛}{bo1tao1}[][HSK 7-9]
    \definition{s.}{grandes (enormes) ondas; ondas de maré; ondas grandes costumam se referir a paisagens espetaculares ou emocionantes; são usadas tanto na linguagem falada quanto na escrita.}
  \end{Phonetics}
\end{Entry}

\begin{Entry}{波澜}{8,15}{⽔、⽔}
  \begin{Phonetics}{波澜}{bo1lan2}[][HSK 7-9]
    \definition[个,场,阵]{s.}{ondas grandes}
  \end{Phonetics}
\end{Entry}

\begin{Entry}{泥}{8}{⽔}
  \begin{Phonetics}{泥}{ni2}[][HSK 6]
    \definition*{s.}{Sobrenome Ni}
    \definition{s.}{lama; atoleiro | pasta ou polpa; amassado | qualquer matéria pastosa; purê de vegetais ou frutas}
  \end{Phonetics}
  \begin{Phonetics}{泥}{ni4}
    \definition{adj.}{fanático; teimoso; obstinado; cabeçudo}
    \definition{v.}{cobrir ou rebocar com gesso, massa de vidraceiro, etc.}
  \end{Phonetics}
\end{Entry}

\begin{Entry}{泥潭}{8,15}{⽔、⽔}
  \begin{Phonetics}{泥潭}{ni2tan2}
    \definition{s.}{atoleiro | lamaçal | charco | pântano}
  \end{Phonetics}
\end{Entry}

\begin{Entry}{注}{8}{⽔}
  \begin{Phonetics}{注}{zhu4}
    \definition{s.}{apostas (em jogos de azar) | notas (em um texto)}
    \definition{v.}{derramar; encher | concentrar-se em; fixar-se em; focar em  | anotar; explicar com notas | registrar; gravar | irrigar | dar exegese ou explicação}
  \end{Phonetics}
\end{Entry}

\begin{Entry}{注册}{8,5}{⽔、⼌}
  \begin{Phonetics}{注册}{zhu4ce4}[][HSK 5]
    \definition{v.}{inscrever-se; matricular-se; registrar-se; registrar-se junto à autoridade ou escola competente para obter status legal; refere-se especificamente ao usuário de uma determinada rede de computadores que insere o nome de usuário, senha, etc. na rede para obter permissão para usar a rede}
  \end{Phonetics}
\end{Entry}

\begin{Entry}{注册人}{8,5,2}{⽔、⼌、⼈}
  \begin{Phonetics}{注册人}{zhu4ce4ren2}
    \definition{s.}{registrante}
  \end{Phonetics}
\end{Entry}

\begin{Entry}{注册表}{8,5,8}{⽔、⼌、⾐}
  \begin{Phonetics}{注册表}{zhu4ce4biao3}
    \definition[份,个,张]{s.}{registro do Windows}
  \end{Phonetics}
\end{Entry}

\begin{Entry}{注册商标}{8,5,11,9}{⽔、⼌、⼝、⽊}
  \begin{Phonetics}{注册商标}{zhu4ce4 shang1biao1}
    \definition{s.}{marca registrada}
  \end{Phonetics}
\end{Entry}

\begin{Entry}{注视}{8,8}{⽔、⾒}
  \begin{Phonetics}{注视}{zhu4shi4}[][HSK 5]
    \definition{v.}{olhar atentamente para; observar atentamente}
  \end{Phonetics}
\end{Entry}

\begin{Entry}{注重}{8,9}{⽔、⾥}
  \begin{Phonetics}{注重}{zhu4zhong4}[][HSK 5]
    \definition{v.}{enfatizar; dar ênfase a; dar ênfase a; prestar atenção a; dar importância a}
  \end{Phonetics}
\end{Entry}

\begin{Entry}{注射}{8,10}{⽔、⼨}
  \begin{Phonetics}{注射}{zhu4she4}[][HSK 5]
    \definition{v.}{injetar; usar uma seringa para administrar medicamento líquido em um organismo}
  \end{Phonetics}
\end{Entry}

\begin{Entry}{注意}{8,13}{⽔、⼼}
  \begin{Phonetics}{注意}{zhu4yi4}[][HSK 3]
    \definition{v.}{prestar atenção; notar; ficar de olho; concentrar os pensamentos em um aspecto específico}
  \end{Phonetics}
\end{Entry}

\begin{Entry}{注意力}{8,13,2}{⽔、⼼、⼒}
  \begin{Phonetics}{注意力}{zhu4yi4li4}
    \definition{s.}{atenção}
  \end{Phonetics}
\end{Entry}

\begin{Entry}{注意力缺失症}{8,13,2,10,5,10}{⽔、⼼、⼒、⽸、⼤、⽧}
  \begin{Phonetics}{注意力缺失症}{zhu4yi4li4que1shi1zheng4}
    \definition{s.}{transtorno de déficit de atenção}
  \end{Phonetics}
\end{Entry}

\begin{Entry}{注意地}{8,13,6}{⽔、⼼、⼟}
  \begin{Phonetics}{注意地}{zhu4yi4di4}
    \definition{s.}{área de cuidado, de observação}
  \end{Phonetics}
\end{Entry}

\begin{Entry}{泪}{8}{⽔}
  \begin{Phonetics}{泪}{lei4}[][HSK 4]
    \definition[滴,行]{s.}{lágrima | algo semelhante a uma lágrima}
  \end{Phonetics}
\end{Entry}

\begin{Entry}{泪水}{8,4}{⽔、⽔}
  \begin{Phonetics}{泪水}{lei4 shui3}[][HSK 4]
    \definition[滴,行]{s.}{lágrima}
  \end{Phonetics}
\end{Entry}

\begin{Entry}{泳}{8}{⽔}
  \begin{Phonetics}{泳}{yong3}
    \definition{v.}{nadar}
  \end{Phonetics}
\end{Entry}

\begin{Entry}{泳池}{8,6}{⽔、⽔}
  \begin{Phonetics}{泳池}{yong3chi2}
    \definition{s.}{piscina}
  \seealsoref{游泳池}{you2 yong3 chi2}
  \seealsoref{游泳馆}{you2yong3guan3}
  \end{Phonetics}
\end{Entry}

\begin{Entry}{泳衣}{8,6}{⽔、⾐}
  \begin{Phonetics}{泳衣}{yong3yi1}
    \definition{s.}{roupa de banho | maiô}
  \seealsoref{游泳衣}{you2yong3yi1}
  \end{Phonetics}
\end{Entry}

\begin{Entry}{泼}{8}{⽔}
  \begin{Phonetics}{泼}{po1}[][HSK 5]
    \definition{adj.}{rude e irracional; mal-humorado | Dialeto: ousado e vigoroso; ousado e resoluto}
    \definition{v.}{espalhar; salpicar; derramar; derramar ou espalhar o líquido com força para fora}
  \end{Phonetics}
\end{Entry}

\begin{Entry}{浅}{8}{⽔}
  \begin{Phonetics}{浅}{jian1}
    \definition{adj.}{murmurando, fluindo suavemente, gorgolejando suavemente}
    \definition{s.}{Onomatopéia: som de água em movimento}
  \end{Phonetics}
  \begin{Phonetics}{浅}{qian3}[][HSK 4]
    \definition{adj.}{raso; superficial;  (em oposição a 深) | fácil; simples; redação, conteúdo, etc. simples e fáceis de entender | superficial; não é profundo em aprendizado, percepção e sabedoria | não próximo; não íntimo; sentimentos não profundos | (cor) claro; pálido;  cor pouco intensa; leve |experiência breve; duração de tempo breve | baixo grau; peso leve; nível baixo}
  \seealsoref{深}{shen1}
  \end{Phonetics}
\end{Entry}

\begin{Entry}{炎}{8}{⽕}
  \begin{Phonetics}{炎}{yan2}
    \definition{adj.}{escaldante; ardente}
    \definition{s.}{inflamação | poder; influência}
  \end{Phonetics}
\end{Entry}

\begin{Entry}{炎热}{8,10}{⽕、⽕}
  \begin{Phonetics}{炎热}{yan2re4}
    \definition{adj.}{extremamente quente | escaldante (clima)}
  \end{Phonetics}
\end{Entry}

\begin{Entry}{炒}{8}{⽕}
  \begin{Phonetics}{炒}{chao3}[][HSK 6]
    \definition{v.}{saltear; refogar; aquecer os alimentos em uma panela e mexer repetidamente para cozinhá-los ou secá-los | especular (na bolsa de valores, etc.) | exagerar; dar publicidade exagerada; a fim de ampliar a influência, por meio de publicidade repetida e exagerada na mídia | demitir; despedir}
  \end{Phonetics}
\end{Entry}

\begin{Entry}{炒作}{8,7}{⽕、⼈}
  \begin{Phonetics}{炒作}{chao3 zuo4}[][HSK 6]
    \definition{v.}{promover (na mídia); exagerar artificialmente e promover ou desvalorizar de forma inadequada | especular; comprar e vender frequentemente no mercado de negociação para obter lucros}
  \end{Phonetics}
\end{Entry}

\begin{Entry}{炒股}{8,8}{⽕、⾁}
  \begin{Phonetics}{炒股}{chao3/gu3}[][HSK 6]
    \definition{v.+compl.}{especular em ações; comprar e vender ações; jogar no mercado}
  \end{Phonetics}
\end{Entry}

\begin{Entry}{爬}{8}{⽖}
  \begin{Phonetics}{爬}{pa2}[][HSK 2]
    \definition{v.}{rastejar; arrastar-se; engatinhar | escalar; trepar; subir com dificuldade | sentar-se; levantar-se; levantar-se da posição deitada ou sentada}
  \end{Phonetics}
\end{Entry}

\begin{Entry}{爬上}{8,3}{⽖、⼀}
  \begin{Phonetics}{爬上}{pa2shang4}
    \definition{v.}{escalar}
  \end{Phonetics}
\end{Entry}

\begin{Entry}{爬山}{8,3}{⽖、⼭}
  \begin{Phonetics}{爬山}{pa2/shan1}[][HSK 2]
    \definition{v.+compl.}{escalar uma montanha;}
  \end{Phonetics}
\end{Entry}

\begin{Entry}{爬升}{8,4}{⽖、⼗}
  \begin{Phonetics}{爬升}{pa2sheng1}
    \definition{v.}{ascender | ganhar promoção | subir (números de vendas, etc.) | aumentar}
  \end{Phonetics}
\end{Entry}

\begin{Entry}{爬行}{8,6}{⽖、⾏}
  \begin{Phonetics}{爬行}{pa2xing2}
    \definition{v.}{rastejar | arrastar | engatinhar}
  \end{Phonetics}
\end{Entry}

\begin{Entry}{爬杆}{8,7}{⽖、⽊}
  \begin{Phonetics}{爬杆}{pa2gan1}
    \definition{s.}{escalada em poste}
    \definition{v.}{escalar um poste}
  \end{Phonetics}
\end{Entry}

\begin{Entry}{爬竿}{8,9}{⽖、⽵}
  \begin{Phonetics}{爬竿}{pa2gan1}
    \definition{s.}{poste de escalada | escalada em poste (como ginástica ou ato de circo)}
  \end{Phonetics}
\end{Entry}

\begin{Entry}{爬梳}{8,11}{⽖、⽊}
  \begin{Phonetics}{爬梳}{pa2shu1}
    \definition{v.}{vasculhar (documentos históricos, etc.) | desvendar}
  \end{Phonetics}
\end{Entry}

\begin{Entry}{爬犁}{8,11}{⽖、⽜}
  \begin{Phonetics}{爬犁}{pa2li2}
    \definition{s.}{trenó}
  \seealsoref{扒犁}{pa2li2}
  \end{Phonetics}
\end{Entry}

\begin{Entry}{爬墙}{8,14}{⽖、⼟}
  \begin{Phonetics}{爬墙}{pa2qiang2}
    \definition{v.}{escalar uma parede}
  \end{Phonetics}
\end{Entry}

\begin{Entry}{爸}{8}{⽗}
  \begin{Phonetics}{爸}{ba4}[][HSK 1]
    \definition[个,位]{s.}{(informal) pai}
  \seealsoref{爸爸}{ba4ba5}
  \end{Phonetics}
\end{Entry}

\begin{Entry}{爸妈}{8,6}{⽗、⼥}
  \begin{Phonetics}{爸妈}{ba4ma1}
    \definition{s.}{pai e mãe}
  \end{Phonetics}
\end{Entry}

\begin{Entry}{爸爸}{8,8}{⽗、⽗}
  \begin{Phonetics}{爸爸}{ba4ba5}[][HSK 1]
    \definition[个,位,名,群]{s.}{(informal) pai; papai; papa}
  \seealsoref{爸}{ba4}
  \end{Phonetics}
\end{Entry}

\begin{Entry}{版}{8}{⽚}
  \begin{Phonetics}{版}{ban3}[][HSK 5]
    \definition{clas.}{usado como uma palavra de medida para materiais impressos (por exemplo, livros, jornais, edições)}
    \definition{s.}{chapa, placa ou bloco de impressão | edição (livros impressos) | página (de um jornal) | moldes ou fromas de construção}
  \end{Phonetics}
\end{Entry}

\begin{Entry}{牦}{8}{⽜}
  \begin{Phonetics}{牦}{mao2}
    \definition[头]{s.}{iaque; boi da Tartária}
  \end{Phonetics}
\end{Entry}

\begin{Entry}{牦牛}{8,4}{⽜、⽜}
  \begin{Phonetics}{牦牛}{mao2niu2}
    \definition{s.}{iaque}
  \end{Phonetics}
\end{Entry}

\begin{Entry}{物}{8}{⽜}
  \begin{Phonetics}{物}{wu4}
    \definition{s.}{coisa; matéria; objeto | mundo exterior distinto de si mesmo; outras pessoas; refere-se a outras pessoas além de si mesmo ou ao ambiente em relação a si mesmo | essência; conteúdo; substância | criatura; criação}
  \end{Phonetics}
\end{Entry}

\begin{Entry}{物业}{8,5}{⽜、⼀}
  \begin{Phonetics}{物业}{wu4ye4}[][HSK 5]
    \definition[处]{s.}{propriedade; gestão de propriedades; gestão patrimonial; administração de imóveis | empresa de administração de imóveis; empresa de gestão imobiliária; empresa de administração de bens imóveis}
  \end{Phonetics}
\end{Entry}

\begin{Entry}{物价}{8,6}{⽜、⼈}
  \begin{Phonetics}{物价}{wu4 jia4}[][HSK 5]
    \definition[个]{s.}{preços das commodities; preços das matérias-primas; preço das mercadorias}
  \end{Phonetics}
\end{Entry}

\begin{Entry}{物质}{8,8}{⽜、⾙}
  \begin{Phonetics}{物质}{wu4zhi4}[][HSK 5]
    \definition[种,类,个]{s.}{matéria; substância; algo que existe além do espírito, que pode ser visto, tocado, cheirado ou detectado por instrumentos científicos | material; meios de subsistência; coisas que permitem às pessoas viver ou viver melhor, como comida, roupas, casas, dinheiro, etc.}
  \end{Phonetics}
\end{Entry}

\begin{Entry}{物品}{8,9}{⽜、⼝}
  \begin{Phonetics}{物品}{wu4 pin3}[][HSK 6]
    \definition[件,个]{s.}{artigos; itens; bens}
  \end{Phonetics}
\end{Entry}

\begin{Entry}{物理}{8,11}{⽜、⽟}
  \begin{Phonetics}{物理}{wu4li3}
    \definition{s.}{física (disciplina)}
  \end{Phonetics}
\end{Entry}

\begin{Entry}{狒}{8}{⽝}
  \begin{Phonetics}{狒}{fei4}
    \definition{s.}{babuíno (uma espécie de macaco)}
  \end{Phonetics}
\end{Entry}

\begin{Entry}{狒狒}{8,8}{⽝、⽝}
  \begin{Phonetics}{狒狒}{fei4fei4}
    \definition{s.}{babuíno}
  \end{Phonetics}
\end{Entry}

\begin{Entry}{狗}{8}{⽝}
  \begin{Phonetics}{狗}{gou3}[][HSK 2]
    \definition[条,只,群]{s.}{cão; cachorro | palavrão usado para se referir a pessoas más ou seus capangas}
  \end{Phonetics}
\end{Entry}

\begin{Entry}{玩}{8}{⽟}
  \begin{Phonetics}{玩}{wan2}
    \definition*{s.}{Sobrenome Wan}
    \definition{s.}{objeto de apreciação; coisas para assistir}
    \definition{v.}{(~儿) divertir-se; entreter-se; fazer atividades que te deixem feliz | jogar; praticar algum tipo de atividade cultural, de entretenimento ou esportiva | recorrer a; usar métodos e meios impróprios para atingir o objetivo | provocar; subestimar; tratar com uma atitude frívola; desprezar | desfrutar; apreciar; observar | (~儿) envolver-se em; tomar parte em; perseguir ou expressar deliberadamente um certo sentimento | ponderar; pensar cuidadosamente; apreciar}
  \end{Phonetics}
\end{Entry}

\begin{Entry}{玩儿}{8,2}{⽟、⼉}
  \begin{Phonetics}{玩儿}{wan2r5}[][HSK 1]
    \definition{v.}{divertir-se; (entretenimento) relaxar ou experimentar alguma atividade}
  \end{Phonetics}
\end{Entry}

\begin{Entry}{玩艺}{8,4}{⽟、⾋}
  \begin{Phonetics}{玩艺}{wan2yi4}
    \variantof{玩意}
  \end{Phonetics}
\end{Entry}

\begin{Entry}{玩伴}{8,7}{⽟、⼈}
  \begin{Phonetics}{玩伴}{wan2ban4}
    \definition{s.}{parceiro de brincadeira}
  \end{Phonetics}
\end{Entry}

\begin{Entry}{玩具}{8,8}{⽟、⼋}
  \begin{Phonetics}{玩具}{wan2ju4}[][HSK 3]
    \definition[个,件,套]{s.}{brinquedo; coisas para brincar}
  \end{Phonetics}
\end{Entry}

\begin{Entry}{玩具厂}{8,8,2}{⽟、⼋、⼚}
  \begin{Phonetics}{玩具厂}{wan2ju4chang3}
    \definition{s.}{fábrica de brinquedos}
  \end{Phonetics}
\end{Entry}

\begin{Entry}{玩具车}{8,8,4}{⽟、⼋、⾞}
  \begin{Phonetics}{玩具车}{wan2ju4 che1}
    \definition{s.}{carrinho de brinquedo}
  \end{Phonetics}
\end{Entry}

\begin{Entry}{玩味}{8,8}{⽟、⼝}
  \begin{Phonetics}{玩味}{wan2wei4}
    \definition{v.}{ponderar sutilezas | ruminar (pensamentos)}
  \end{Phonetics}
\end{Entry}

\begin{Entry}{玩者}{8,8}{⽟、⽼}
  \begin{Phonetics}{玩者}{wan2zhe3}
    \definition{s.}{jogador}
  \end{Phonetics}
\end{Entry}

\begin{Entry}{玩耍}{8,9}{⽟、⽽}
  \begin{Phonetics}{玩耍}{wan2shua3}
    \definition{v.}{divertir-me | brincar (como as crianças fazem)}
  \end{Phonetics}
\end{Entry}

\begin{Entry}{玩家}{8,10}{⽟、⼧}
  \begin{Phonetics}{玩家}{wan2jia1}
    \definition{s.}{entusiasta (áudio, modelos de aviões, etc.) | jogador (de um jogo)}
  \end{Phonetics}
\end{Entry}

\begin{Entry}{玩偶}{8,11}{⽟、⼈}
  \begin{Phonetics}{玩偶}{wan2'ou3}
    \definition{s.}{estatueta de brinquedo | boneco de ação | bicho de pelúcia | boneca}
  \end{Phonetics}
\end{Entry}

\begin{Entry}{玩遍}{8,12}{⽟、⾡}
  \begin{Phonetics}{玩遍}{wan2bian4}
    \definition{v.}{passear (todo o país, toda a cidade, etc.) | visitar (um grande número de lugares)}
  \end{Phonetics}
\end{Entry}

\begin{Entry}{玩意}{8,13}{⽟、⼼}
  \begin{Phonetics}{玩意}{wan2yi4}
    \definition{s.}{ato | brinquedo | coisa | truque (em uma performance, show de palco, acrobacias, etc.)}
  \end{Phonetics}
\end{Entry}

\begin{Entry}{环}{8}{⽟}
  \begin{Phonetics}{环}{huan2}[][HSK 3]
    \definition*{s.}{Sobrenome Huan}
    \definition{clas.}{usado para anéis}
    \definition[个,串]{s.}{anel; arco | elo; \emph{link}; passo; etapa | anel; objeto em forma de círculo | arredores}
    \definition{v.}{cercar; rodear; circular; circundar}
  \end{Phonetics}
\end{Entry}

\begin{Entry}{环卫}{8,3}{⽟、⼙}
  \begin{Phonetics}{环卫}{huan2wei4}
    \definition{s.}{limpeza pública; saneamento ambiental; saneamento geral; abreviação de 环境卫生 | Arcaico: guardas imperiais; guardas}
  \seealsoref{环境卫生}{huan2jing4wei4sheng1}
  \end{Phonetics}
\end{Entry}

\begin{Entry}{环节}{8,5}{⽟、⾋}
  \begin{Phonetics}{环节}{huan2jie2}[][HSK 5]
    \definition[个]{s.}{\emph{link}; ligação; vínculo; uma das muitas coisas que estão inter-relacionadas | segmento; estrutura anelar de alguns animais inferiores}
  \end{Phonetics}
\end{Entry}

\begin{Entry}{环保}{8,9}{⽟、⼈}
  \begin{Phonetics}{环保}{huan2 bao3}[][HSK 3]
    \definition{adj.}{ecológico; benefício para o meio ambiente; não prejudica o meio ambiente}
    \definition{s.}{proteção ambiental}
  \end{Phonetics}
\end{Entry}

\begin{Entry}{环境}{8,14}{⽟、⼟}
  \begin{Phonetics}{环境}{huan2jing4}[][HSK 3]
    \definition[个]{s.}{ambiente; os arredores | arredores; circunstâncias; condições políticas, econômicas, culturais, etc., dentro de um determinado âmbito}
  \end{Phonetics}
\end{Entry}

\begin{Entry}{环境卫生}{8,14,3,5}{⽟、⼟、⼙、⽣}
  \begin{Phonetics}{环境卫生}{huan2jing4wei4sheng1}
    \definition{s.}{saneamento ambiental}
  \seealsoref{环卫}{huan2wei4}
  \end{Phonetics}
\end{Entry}

\begin{Entry}{现}{8}{⾒}
  \begin{Phonetics}{现}{xian4}
    \definition{adj.}{(dinheiro, etc.) em mãos}
    \definition{adv.}{recente; de improviso; naquela época; temporariamente}
    \definition{s.}{presente; atual; existente | dinheiro; dinheiro de pronto}
    \definition{v.}{mostrar; revelar; aparecer; tornar-se visível}
  \seealsoref{见}{xian4}
  \end{Phonetics}
\end{Entry}

\begin{Entry}{现代}{8,5}{⾒、⼈}
  \begin{Phonetics}{现代}{xian4dai4}[][HSK 3]
    \definition*{s.}{Hyundai, empresa sul-coreana}
    \definition{adj.}{moderno; contemporâneo; com características, estilo e conceitos modernos, refletindo a vanguarda, a moda e a inovação da atualidade}
    \definition{s.}{tempos modernos; era contemporânea; atualmente, na divisão cronológica da história da China, refere-se principalmente ao período desde o Movimento 4 de Maio até os dias atuais}
  \end{Phonetics}
\end{Entry}

\begin{Entry}{现在}{8,6}{⾒、⼟}
  \begin{Phonetics}{现在}{xian4zai4}[][HSK 1]
    \definition{adv.}{agora; no momento; atualmente; neste momento, quando se fala, às vezes inclui um período de tempo mais ou menos longo antes ou depois da fala (diferente de 过去 ou 将来)}
  \seealsoref{过去}{guo4 qu4}
  \seealsoref{将来}{jiang1lai2}
  \end{Phonetics}
\end{Entry}

\begin{Entry}{现场}{8,6}{⾒、⼟}
  \begin{Phonetics}{现场}{xian4chang3}[][HSK 3]
    \definition[个,处]{s.}{local onde ocorreu o acidente, incidente ou desastre| local; ponto; local onde se realizam diretamente atividades como produção, apresentações e competições}
  \end{Phonetics}
\end{Entry}

\begin{Entry}{现有}{8,6}{⾒、⽉}
  \begin{Phonetics}{现有}{xian4 you3}[][HSK 5]
    \definition{adj.}{agora disponível; existente}
    \definition{v.}{estar disponível agora; existir | Literário: ter em mãos; ter em posse}
  \end{Phonetics}
\end{Entry}

\begin{Entry}{现抓}{8,7}{⾒、⼿}
  \begin{Phonetics}{现抓}{xian4zhua1}
    \definition{v.}{improvisar}
  \end{Phonetics}
\end{Entry}

\begin{Entry}{现状}{8,7}{⾒、⽝}
  \begin{Phonetics}{现状}{xian4zhuang4}[][HSK 5]
    \definition{s.}{situação atual}
  \end{Phonetics}
\end{Entry}

\begin{Entry}{现实}{8,8}{⾒、⼧}
  \begin{Phonetics}{现实}{xian4shi2}[][HSK 3]
    \definition{adj.}{real; efetivo; verdadeiro; de acordo com circunstâncias objetivas}
    \definition[个]{s.}{realidade; factualidade; coisas que existem objetivamente}
  \end{Phonetics}
\end{Entry}

\begin{Entry}{现货}{8,8}{⾒、⾙}
  \begin{Phonetics}{现货}{xian4huo4}
    \definition{s.}{produtos à vista}
  \end{Phonetics}
\end{Entry}

\begin{Entry}{现货的}{8,8,8}{⾒、⾙、⽩}
  \begin{Phonetics}{现货的}{xian4huo4 de5}
    \definition{s.}{produtos em estoque}
  \end{Phonetics}
\end{Entry}

\begin{Entry}{现金}{8,8}{⾒、⾦}
  \begin{Phonetics}{现金}{xian4jin1}[][HSK 3]
    \definition[笔]{s.}{dinheiro; dinheiro vivo; moeda que pode ser usada diretamente | reserva de dinheiro em um banco; o dinheiro guardado no cofre do banco}
  \end{Phonetics}
\end{Entry}

\begin{Entry}{现做}{8,11}{⾒、⼈}
  \begin{Phonetics}{现做}{xian4zuo4}
    \definition{adj.}{fresco}
    \definition{v.}{fazer (comida) no local}
  \end{Phonetics}
\end{Entry}

\begin{Entry}{现象}{8,11}{⾒、⾗}
  \begin{Phonetics}{现象}{xian4xiang4}[][HSK 3]
    \definition[个,种]{s.}{aparência (das coisas); fenômeno; a forma externa e as relações manifestadas pelas coisas em seu desenvolvimento e mudança}
  \end{Phonetics}
\end{Entry}

\begin{Entry}{画}{8}{⽥}
  \begin{Phonetics}{画}{hua4}[][HSK 2]
    \definition*{s.}{Sobrenome Hua}
    \definition{clas.}{traços (de um caractere chinês)}
    \definition[张,幅]{s.}{desenho; pintura; imagem; figura desenhada | traço horizontal (em caracteres chineses)}
    \definition{v.}{desenhar; pintar | desenhar; marcar; assinar}
  \seealsoref{划}{hua4}
  \end{Phonetics}
\end{Entry}

\begin{Entry}{画儿}{8,2}{⽥、⼉}
  \begin{Phonetics}{画儿}{hua4r5}[][HSK 2]
    \definition[幅,张]{s.}{imagem; desenho; pintura; obra de arte pintada}
  \end{Phonetics}
\end{Entry}

\begin{Entry}{画地为牢}{8,6,4,7}{⽥、⼟、⼂、⼧}
  \begin{Phonetics}{画地为牢}{hua4di4wei2lao2}
    \definition{expr.}{desenhar um círculo no chão para servir como uma prisão; restringir as atividades de alguém a uma área ou esfera designada; limitar; restringir | (literário) ser confinado dentro de um círculo desenhado no chão | (figurativo) limitar-se a uma gama restrita de atividades}
  \end{Phonetics}
\end{Entry}

\begin{Entry}{画面}{8,9}{⽥、⾯}
  \begin{Phonetics}{画面}{hua4 mian4}[][HSK 5]
    \definition[个,幅,帧]{s.}{quadro; aparência geral de uma imagem; imagem apresentada no quadro, na tela, etc.}
  \end{Phonetics}
\end{Entry}

\begin{Entry}{画家}{8,10}{⽥、⼧}
  \begin{Phonetics}{画家}{hua4 jia1}[][HSK 2]
    \definition[个,位,名,些]{s.}{pintor; pessoa com talento para pintura}
  \end{Phonetics}
\end{Entry}

\begin{Entry}{畅}{8}{⽥}
  \begin{Phonetics}{畅}{chang4}
    \definition*{s.}{Sobrenome Chang}
    \definition{adj.}{suave; desimpedido; sem obstáculos; desobstruído | livre; desinibido}
  \end{Phonetics}
\end{Entry}

\begin{Entry}{畅谈}{8,10}{⽥、⾔}
  \begin{Phonetics}{畅谈}{chang4tan2}[][HSK 7-9]
    \definition{v.}{falar livremente e com entusiasmo; falar livremente e com satisfação; falar com entusiasmo sobre}
  \end{Phonetics}
\end{Entry}

\begin{Entry}{畅销}{8,12}{⽥、⾦}
  \begin{Phonetics}{畅销}{chang4xiao1}[][HSK 7-9]
    \definition{adj.}{mais vendido; \emph{best-seller}}
    \definition{v.}{vender bem; ter grande procura; ter um mercado pronto; ser um \emph{best-seller}}
  \end{Phonetics}
\end{Entry}

\begin{Entry}{的}{8}{⽩}
  \begin{Phonetics}{的}{de5}
    \definition{part.}{usado para indicar posse | formar uma frase nominal ou expressão nominal | substituir a pessoa ou coisa mencionada anteriormente | no final de uma frase declarativa, para dar ênfase; usado após o verbo predicativo, enfatiza o agente da ação, o tempo, o local, etc. | usado no final de uma frase declarativa, expressa afirmação, ênfase, certeza, etc. | indica que alguém obteve uma determinada posição ou status | usado com 是 para indicar predicado ou ênfase; indica que alguém é o objeto da ação | e assim por diante; e assim por diante; e similares; usado após palavras paralelas, significa 等等, 之类 | indica uma ação (o pronome é o objeto da ação); combinado com o verbo anterior, expressa uma ação, e o pronome é o objeto dessa ação}
  \seealsoref{等等}{deng3 deng3}
  \seealsoref{是}{shi4}
  \seealsoref{之类}{zhi1 lei4}
  \end{Phonetics}
  \begin{Phonetics}{的}{di1}
    \definition{s.}{abreviação de 的士: um táxi}
  \seealsoref{的士}{di1shi4}
  \end{Phonetics}
  \begin{Phonetics}{的}{di2}
    \definition{adv.}{verdadeiramente; exatamente; realmente}
  \end{Phonetics}
  \begin{Phonetics}{的}{di4}
    \definition{adj.}{alvo; centro do alvo}
  \end{Phonetics}
\end{Entry}

\begin{Entry}{的士}{8,3}{⽩、⼠}
  \begin{Phonetics}{的士}{di1shi4}
    \definition{s.}{(empréstimo linguístico) táxi}
  \end{Phonetics}
\end{Entry}

\begin{Entry}{的时候}{8,7,10}{⽩、⽇、⼈}
  \begin{Phonetics}{的时候}{de5 shi2hou4}
    \definition{part.}{naquele momento; quando; em; descreve o momento específico em que um evento ocorreu}
  \end{Phonetics}
\end{Entry}

\begin{Entry}{的话}{8,8}{⽩、⾔}
  \begin{Phonetics}{的话}{de5 hua4}[][HSK 2]
    \definition{part.}{se; caso; suponha que; partícula usada após uma frase hipotética para introduzir o texto seguinte}
  \end{Phonetics}
\end{Entry}

\begin{Entry}{的确}{8,12}{⽩、⽯}
  \begin{Phonetics}{的确}{di2que4}[][HSK 4]
    \definition{adv.}{realmente; de fato, ao expressar certeza sobre a situação}
  \end{Phonetics}
\end{Entry}

\begin{Entry}{盲}{8}{⽬}
  \begin{Phonetics}{盲}{mang2}
    \definition{adj.}{cego | incapaz de distinguir coisas | totalmente incompetente}
    \definition{adv.}{cegamente}
  \end{Phonetics}
\end{Entry}

\begin{Entry}{盲人}{8,2}{⽬、⼈}
  \begin{Phonetics}{盲人}{mang2 ren2}[][HSK 6]
    \definition[个,位,名]{s.}{cego; pessoa cega; pessoas com deficiência visual}
  \end{Phonetics}
\end{Entry}

\begin{Entry}{盲目}{8,5}{⽬、⽬}
  \begin{Phonetics}{盲目}{mang2mu4}
    \definition{adj.}{ignorante | sem compreensão}
    \definition{adv.}{cegamente}
    \definition{s.}{cego}
  \end{Phonetics}
\end{Entry}

\begin{Entry}{直}{8}{⽬}
  \begin{Phonetics}{直}{zhi2}[][HSK 3]
    \definition*{s.}{Sobrenome Zhi}
    \definition{adj.}{reto (oposto a 弯,曲) | vertical; perpendicular (oposto a 横) | justo; íntegro; imparcial | franco; sincero; direto ao ponto | rígido; entorpecido | direto; em linha reta; rígido | ereto; perpendicular ao solo}
    \definition{adv.}{diretamente; sempre; reto | continuamente; constantemente | apenas; simplesmente | de ​​fato}
    \definition[条]{s.}{traço vertical (em caracteres chineses, 竖)}
    \definition{v.}{endireitar; tornar reto | alongar}
  \seealsoref{横}{heng2}
  \seealsoref{曲}{qu1}
  \seealsoref{竖}{shu4}
  \seealsoref{弯}{wan1}
  \end{Phonetics}
\end{Entry}

\begin{Entry}{直升机}{8,4,6}{⽬、⼗、⽊}
  \begin{Phonetics}{直升机}{zhi2 sheng1 ji1}[][HSK 6]
    \definition[架,台,个]{s.}{helicóptero; uma aeronave que pode decolar e pousar verticalmente, com uma hélice montada na parte superior da fuselagem que gira horizontalmente, permitindo que ela permaneça no ar e decole e pouse em uma pequena área}
  \end{Phonetics}
\end{Entry}

\begin{Entry}{直译}{8,7}{⽬、⾔}
  \begin{Phonetics}{直译}{zhi2yi4}
    \definition{s.}{tradução literal}
  \seealsoref{意译}{yi4yi4}
  \end{Phonetics}
\end{Entry}

\begin{Entry}{直译器}{8,7,16}{⽬、⾔、⼝}
  \begin{Phonetics}{直译器}{zhi2yi4qi4}
    \definition{s.}{(computação) interpretador}
  \end{Phonetics}
\end{Entry}

\begin{Entry}{直到}{8,8}{⽬、⼑}
  \begin{Phonetics}{直到}{zhi2 dao4}[][HSK 3]
    \definition{adv.}{até (geralmente se refere ao tempo); até que}
  \end{Phonetics}
\end{Entry}

\begin{Entry}{直线}{8,8}{⽬、⽷}
  \begin{Phonetics}{直线}{zhi2 xian4}[][HSK 5]
    \definition{adj.}{direto; retilíneo | íngreme; acentuada (subida ou descida)}
    \definition[条]{s.}{linha reta}
  \end{Phonetics}
\end{Entry}

\begin{Entry}{直接}{8,11}{⽬、⼿}
  \begin{Phonetics}{直接}{zhi2jie1}[][HSK 2]
    \definition{adj.}{direto (oposto: indireto 间接) | imediato}
  \seealsoref{间接}{jian4jie1}
  \end{Phonetics}
\end{Entry}

\begin{Entry}{直播}{8,15}{⽬、⼿}
  \begin{Phonetics}{直播}{zhi2bo1}[][HSK 3]
    \definition{v.}{transmissão ao vivo; transmitir diretamente, sem gravar}
  \end{Phonetics}
\end{Entry}

\begin{Entry}{知}{8}{⽮}
  \begin{Phonetics}{知}{zhi1}
    \definition{s.}{conhecimento | amigo íntimo; refere-se a um confidente}
    \definition{v.}{saber; entender; perceber; estar ciente de | contar; informar; notificar; tornar conhecido | administrar; estar encarregado de; supervisionar}
  \end{Phonetics}
\end{Entry}

\begin{Entry}{知了}{8,2}{⽮、⼅}
  \begin{Phonetics}{知了}{zhi1liao3}
    \definition[通]{s.}{cigarra}
  \end{Phonetics}
\end{Entry}

\begin{Entry}{知名}{8,6}{⽮、⼝}
  \begin{Phonetics}{知名}{zhi1 ming2}[][HSK 6]
    \definition{adj.}{notável; famoso; celebrado; bem conhecido}
  \end{Phonetics}
\end{Entry}

\begin{Entry}{知识}{8,7}{⽮、⾔}
  \begin{Phonetics}{知识}{zhi1shi5}[][HSK 1]
    \definition[个,门,种]{s.}{conhecimento; conjunto de conhecimentos e experiências adquiridos pelas pessoas na prática de transformar o mundo | intelectual; refere-se à cultura acadêmica}
  \end{Phonetics}
\end{Entry}

\begin{Entry}{知道}{8,12}{⽮、⾡}
  \begin{Phonetics}{知道}{zhi1dao4}[][HSK 1]
    \definition{v.}{saber; perceber; estar ciente de; ter conhecimento dos fatos ou da razão; ser sensato}
  \end{Phonetics}
\end{Entry}

\begin{Entry}{知道了}{8,12,2}{⽮、⾡、⼅}
  \begin{Phonetics}{知道了}{zhi1dao4le5}
    \definition{interj.}{Entendi! | OK!}
  \end{Phonetics}
\end{Entry}

\begin{Entry}{矿}{8}{⽯}
  \begin{Phonetics}{矿}{kuang4}[][HSK 6]
    \definition[个,座]{s.}{depósito de minério | minério | mina}
  \end{Phonetics}
\end{Entry}

\begin{Entry}{矿泉水}{8,9,4}{⽯、⽔、⽔}
  \begin{Phonetics}{矿泉水}{kuang4quan2shui3}[][HSK 4]
    \definition[瓶,杯,口]{s.}{água mineral de nascente}
  \end{Phonetics}
\end{Entry}

\begin{Entry}{码}{8}{⽯}
  \begin{Phonetics}{码}{ma3}
    \definition{clas.}{refere-se a um assunto específico ou a uma categoria de assuntos; refere-se a uma coisa ou a uma classe de coisas | jarda; unidade de comprimento britânica e americana}
    \definition{s.}{um sinal ou objeto que indica número; código; símbolos ou ferramentas que indicam números}
    \definition{v.}{empilhar; acumular}
  \end{Phonetics}
\end{Entry}

\begin{Entry}{码头}{8,5}{⽯、⼤}
  \begin{Phonetics}{码头}{ma3tou2}[][HSK 5]
    \definition[个,座]{s.}{doca; cais; píer; molhe; edifícios à beira-mar ou à beira do rio destinados exclusivamente à atracação de embarcações, embarque e desembarque de passageiros e carga e descarga de mercadorias | cidade portuária; centro comercial e de transportes; refere-se a uma cidade comercial com transporte terrestre e marítimo bem desenvolvido.}
  \end{Phonetics}
\end{Entry}

\begin{Entry}{祅}{8}{⽰}
  \begin{Phonetics}{祅}{yao1}
    \definition{s.}{espírito maligno | \emph{goblin} | bruxaria}
    \variantof{妖}
  \end{Phonetics}
\end{Entry}

\begin{Entry}{秉}{8}{⽲}
  \begin{Phonetics}{秉}{bing3}
    \definition{clas.}{unidade antiga de volume; 16 hu}
    \definition{s.}{Sobrenome Bing}
    \definition{v.}{Literário: segurar; agarrar | Literário: controlar; presidir; assumir o comando de}
  \seealsoref{斛}{hu2}
  \end{Phonetics}
\end{Entry}

\begin{Entry}{秉承}{8,8}{⽲、⼿}
  \begin{Phonetics}{秉承}{bing3cheng2}[][HSK 7-9]
    \definition{v.}{receber (ordens); receber (comandos); aceitar e seguir (uma ordem ou instrução)}
  \end{Phonetics}
\end{Entry}

\begin{Entry}{空}{8}{⽳}
  \begin{Phonetics}{空}{kong1}[][HSK 3]
    \definition*{s.}{Sobrenome Kong}
    \definition{adj.}{vazio; oco; nulo; não inclui nada; não contém nada ou não tem conteúdo; irrealista}
    \definition{adv.}{por nada; em vão; sem efeito}
    \definition{s.}{céu; ar | vazio; vazio do mundo dos sentidos}
  \end{Phonetics}
  \begin{Phonetics}{空}{kong4}[][HSK 4]
    \definition*{s.}{Sobrenome Kong}
    \definition{adj.}{vazio; oco; nulo; que não contém nada; que não tem nada ou nenhum conteúdo; impraticável}
    \definition{adv.}{para nada; em vão; sem efeito}
    \definition{s.}{céu; ar | vazio; ausência do mundo dos sentidos}
  \end{Phonetics}
\end{Entry}

\begin{Entry}{空儿}{8,2}{⽳、⼉}
  \begin{Phonetics}{空儿}{kong4r5}[][HSK 3]
    \definition[个]{s.}{tempo livre; sem horário específico | sala; espaço (não utilizado); área ainda não utilizada}
    \definition{v.}{ter tempo livre}
  \end{Phonetics}
\end{Entry}

\begin{Entry}{空中}{8,4}{⽳、⼁}
  \begin{Phonetics}{空中}{kong1 zhong1}[][HSK 5]
    \definition{adj.}{aéreo; aerotransportado; refere-se à transmissão de sinais de rádio}
    \definition{s.}{no céu; no ar}
  \end{Phonetics}
\end{Entry}

\begin{Entry}{空中小姐}{8,4,3,8}{⽳、⼁、⼩、⼥}
  \begin{Phonetics}{空中小姐}{kong1zhong1xiao3jie3}
    \definition{s.}{aeromoça}
  \end{Phonetics}
\end{Entry}

\begin{Entry}{空心菜}{8,4,11}{⽳、⼼、⾋}
  \begin{Phonetics}{空心菜}{kong1xin1cai4}
    \definition{s.}{espinafre aquático | \emph{ong choy} | repolho do pântano | convolvulus aquático | glória-da-manhã aquática}
  \seealsoref{蕹菜}{weng4cai4}
  \end{Phonetics}
\end{Entry}

\begin{Entry}{空气}{8,4}{⽳、⽓}
  \begin{Phonetics}{空气}{kong1qi4}[][HSK 2]
    \definition[缕,股,份,阵]{s.}{ar; gases que compõe a atmosfera terrestre | atmosfera}
  \end{Phonetics}
\end{Entry}

\begin{Entry}{空军}{8,6}{⽳、⼍}
  \begin{Phonetics}{空军}{kong1 jun1}[][HSK 6]
    \definition[名,位,个,支]{s.}{força aérea; um exército que luta no ar, geralmente composto por várias unidades de aviação e unidades terrestres da força aérea}
  \end{Phonetics}
\end{Entry}

\begin{Entry}{空间}{8,7}{⽳、⾨}
  \begin{Phonetics}{空间}{kong1jian1}[][HSK 4]
    \definition[个]{s.}{espaço; recinto; cômodo; espaço em branco; interespaço}
  \end{Phonetics}
\end{Entry}

\begin{Entry}{空间站}{8,7,10}{⽳、⾨、⽴}
  \begin{Phonetics}{空间站}{kong1jian1zhan4}
    \definition{s.}{estação espacial}
  \end{Phonetics}
\end{Entry}

\begin{Entry}{空姐}{8,8}{⽳、⼥}
  \begin{Phonetics}{空姐}{kong1jie3}
    \definition[名,位,个]{s.}{aeromoça; comissária de bordo; abreviação de 空中小姐}
  \seealsoref{空中小姐}{kong1zhong1xiao3jie3}
  \end{Phonetics}
\end{Entry}

\begin{Entry}{空调}{8,10}{⽳、⾔}
  \begin{Phonetics}{空调}{kong1tiao2}[][HSK 3]
    \definition[台,个]{s.}{ar-condicionado;  condicionador de ar}
  \end{Phonetics}
\end{Entry}

\begin{Entry}{线}{8}{⽷}
  \begin{Phonetics}{线}{xian4}[][HSK 3]
    \definition{clas.}{usado para coisas abstratas, o número é limitado a ``一''}
    \definition[根,个]{s.}{fio; corda; arame; objetos finos e longos feitos de seda, algodão, metal, etc. | linha; figura formada pelo movimento arbitrário de um ponto| feito de fio de algodão | algo em forma de linha, fio, etc. | rota de transporte; linha | linha de demarcação; limite; zona de fronteira; zona de transição | beira; borda | linha ideológica e política | pista; fio}
  \end{Phonetics}
\end{Entry}

\begin{Entry}{线香}{8,9}{⽷、⾹}
  \begin{Phonetics}{线香}{xian4xiang1}
    \definition{s.}{bastão ou vareta de incenso}
  \end{Phonetics}
\end{Entry}

\begin{Entry}{线索}{8,10}{⽷、⽷}
  \begin{Phonetics}{线索}{xian4suo3}[][HSK 5]
    \definition[条,个]{s.}{pista; fio; metáfora para o desenvolvimento das coisas ou a maneira de explorar um problema | fio; linha; refere-se ao contexto de desenvolvimento do enredo em obras literárias}
  \end{Phonetics}
\end{Entry}

\begin{Entry}{线路}{8,13}{⽷、⾜}
  \begin{Phonetics}{线路}{xian4 lu4}[][HSK 6]
    \definition[条]{s.}{linha; rota; as rotas que os veículos de transporte percorrem, etc., que as pessoas podem usar para chegar aos seus destinos | Eletricidade: linha; circuito; a rota da corrente elétrica}
  \end{Phonetics}
\end{Entry}

\begin{Entry}{练}{8}{⽷}
  \begin{Phonetics}{练}{lian4}[][HSK 2]
    \definition*{s.}{Sobrenome Lian}
    \definition{adj.}{habilidoso; experiente; bem treinado}
    \definition{s.}{seda branca}
    \definition{v.}{tratar, amaciar e branquear a seda por meio de fervura; cozinhar seda crua ou tecidos de seda crua | treinar; praticar; exercitar}
  \end{Phonetics}
\end{Entry}

\begin{Entry}{练习}{8,3}{⽷、⼄}
  \begin{Phonetics}{练习}{lian4xi2}[][HSK 2]
    \definition[项,次]{s.}{exercício (em livros); tarefas ou exercícios organizados para consolidar os resultados da aprendizagem}
    \definition{v.}{praticar; exercitar; repitir várias vezes até ficar bem treinado}
  \end{Phonetics}
\end{Entry}

\begin{Entry}{组}{8}{⽷}
  \begin{Phonetics}{组}{zu3}[][HSK 2]
    \definition{clas.}{usado para conjuntos, séries, suítes, baterias}
    \definition[个]{s.}{grupo; uma unidade composta por um pequeno número de pessoas}
    \definition{v.}{formar; organizar; combinar pessoas ou coisas dispersas em um todo ou sistema}
  \end{Phonetics}
\end{Entry}

\begin{Entry}{组长}{8,4}{⽷、⾧}
  \begin{Phonetics}{组长}{zu3 zhang3}[][HSK 2]
    \definition[名,位,个]{s.}{líder de grupo; um supervisor de grupo}
  \end{Phonetics}
\end{Entry}

\begin{Entry}{组合}{8,6}{⽷、⼝}
  \begin{Phonetics}{组合}{zu3he2}[][HSK 3]
    \definition{s.}{associação; combinação; o todo organizado | combinação; retirar n elementos diferentes de m elementos e agrupá-los, independentemente da ordem, em que cada grupo contenha pelo menos um elemento diferente, o resultado obtido é chamado de combinação de n elementos de m}
    \definition{v.}{compor; constituir; formar}
  \end{Phonetics}
\end{Entry}

\begin{Entry}{组成}{8,6}{⽷、⼽}
  \begin{Phonetics}{组成}{zu3cheng2}[][HSK 2]
    \definition{v.}{formar; compor; inventar}
  \end{Phonetics}
\end{Entry}

\begin{Entry}{组织}{8,8}{⽷、⽷}
  \begin{Phonetics}{组织}{zu3zhi1}[][HSK 5]
    \definition{s.}{organização; um coletivo ou grupo estabelecido de acordo com determinados objetivos e princípios | sistema organizado; vários fatores interligados de determinada maneira, formando um sistema | tecer; a combinação de linhas horizontais e verticais nos têxteis | tecido; os seres humanos, os animais, as plantas e outros seres vivos são compostos por uma combinação de células com formas e funções semelhantes, que formam os tecidos; os tecidos são as unidades que compõem os diversos órgãos}
  \end{Phonetics}
\end{Entry}

\begin{Entry}{细}{8}{⽷}
  \begin{Phonetics}{细}{xi4}[][HSK 4]
    \definition{adj.}{fino; delgado; esguio; esbelto; em oposição a 粗 | fino; em partículas pequenas; grãos pequenos | fino e macio;  um sussuro | fino; requintado; delicado | cuidadoso; detalhado; meticuloso | ínfimo; minúsculo; insignificante; diminuto | jovem; pequeno}
  \seealsoref{粗}{cu1}
  \end{Phonetics}
\end{Entry}

\begin{Entry}{细节}{8,5}{⽷、⾋}
  \begin{Phonetics}{细节}{xi4jie2}[][HSK 4]
    \definition[处]{s.}{detalhe; particularidade; aspectos secundários ou partes sutis de um enredo ou episódios secundários usados em uma obra literária para expressar o caráter de uma pessoa ou as características essenciais de uma coisa}
  \end{Phonetics}
\end{Entry}

\begin{Entry}{细胞}{8,9}{⽷、⾁}
  \begin{Phonetics}{细胞}{xi4bao1}[][HSK 6]
    \definition[个]{s.}{célula; unidade estrutural e funcional básica de um organismo, com uma variedade de formas, composta principalmente pelo núcleo, citoplasma e membrana celular; as plantas também possuem paredes celulares fora da membrana celular}
  \end{Phonetics}
\end{Entry}

\begin{Entry}{细致}{8,10}{⽷、⾄}
  \begin{Phonetics}{细致}{xi4zhi4}[][HSK 4]
    \definition{adj.}{meticuloso; cuidadoso; minucioso | intrincado; delicado}
  \end{Phonetics}
\end{Entry}

\begin{Entry}{细菌}{8,11}{⽷、⾋}
  \begin{Phonetics}{细菌}{xi4jun1}[][HSK 6]
    \definition[个]{s.}{germe; bactéria; um organismo muito pequeno, invisível aos olhos humanos}
  \end{Phonetics}
\end{Entry}

\begin{Entry}{细菌战}{8,11,9}{⽷、⾋、⼽}
  \begin{Phonetics}{细菌战}{xi4jun1zhan4}
    \definition{s.}{guerra biológica}
  \end{Phonetics}
\end{Entry}

\begin{Entry}{织}{8}{⽷}
  \begin{Phonetics}{织}{zhi1}[][HSK 6]
    \definition{v.}{tecer; fazer fios ou linhas cruzarem para fazer seda, tecido, lã, etc. | tricotar; usar agulhas para fazer fios ou linhas entrelaçados para confeccionar suéteres, meias, rendas, redes, etc. | sobrepor-se; entrelaçar-se; cruzar; entrelaçar}
  \end{Phonetics}
\end{Entry}

\begin{Entry}{终}{8}{⽷}
  \begin{Phonetics}{终}{zhong1}
    \definition*{s.}{Sobrenome Zhong}
    \definition{adj.}{tudo; todo; inteiro; o tempo todo do começo ao fim}
    \definition{adv.}{afinal; no final; eventualmente; finalmente}
    \definition{s.}{fim; término | tempo todo; período inteiro; o tempo todo | final | morte; refere-se à morte}
  \end{Phonetics}
\end{Entry}

\begin{Entry}{终于}{8,3}{⽷、⼆}
  \begin{Phonetics}{终于}{zhong1yu2}[][HSK 3]
    \definition{adv.}{finalmente; por fim; eventualmente; no final; indica uma situação que surge após várias mudanças ou espera}
  \end{Phonetics}
\end{Entry}

\begin{Entry}{终止}{8,4}{⽷、⽌}
  \begin{Phonetics}{终止}{zhong1zhi3}[][HSK 5]
    \definition{v.}{parar; terminar | anular; encerrar; expirar; revogar}
  \end{Phonetics}
\end{Entry}

\begin{Entry}{终究}{8,7}{⽷、⽳}
  \begin{Phonetics}{终究}{zhong1jiu1}
    \definition{adv.}{afinal de contas; enfatiza que, não importa o que aconteça, a natureza das pessoas e das coisas não mudará e que as características básicas devem ser reconhecidas (tem o efeito de fortalecer o tom) |  no final; indica que um determinado resultado ocorrerá ou não, frequentemente usado em especulações, julgamentos etc. | afinal de contas; indica que, apesar do grande esforço ou da grande esperança, o resultado objetivo ainda é insatisfatório, geralmente com o significado de pesar ou pena | afinal de contas; indica que um resultado desejado finalmente aparece}
  \end{Phonetics}
\end{Entry}

\begin{Entry}{终身}{8,7}{⽷、⾝}
  \begin{Phonetics}{终身}{zhong1shen1}[][HSK 5]
    \definition{s.}{vida inteira; por toda a vida; por toda a vida}
  \end{Phonetics}
\end{Entry}

\begin{Entry}{终点}{8,9}{⽷、⽕}
  \begin{Phonetics}{终点}{zhong1dian3}[][HSK 5]
    \definition[个]{s.}{destino; ponto terminal; ponto de chegada; lugar onde uma jornada termina | final; refere-se especificamente ao local onde a corrida é interrompida}
  \end{Phonetics}
\end{Entry}

\begin{Entry}{经}{8}{⽷}
  \begin{Phonetics}{经}{jing1}
    \definition*{s.}{Sobrenome Jing}
    \definition{adj.}{constante; regular}
    \definition{prep.}{como resultado de; depois; através de}
    \definition{s.}{urdidura, os fios longitudinais de um tecido (oposto a 纬) | Medicina chinesa: canais principais e colaterais | Geografia: longitude (oposto a 纬) | escritura; sutra; cânone; clássico | menstruação}
    \definition{v.}{Literário: gerenciar; lidar com; envolver-se em | enforcar-se | suportar; ficar de pé; aguentar; resistir | passar por; sofrer; experimentar}
  \seealsoref{纬}{wei3}
  \end{Phonetics}
\end{Entry}

\begin{Entry}{经历}{8,4}{⽷、⼚}
  \begin{Phonetics}{经历}{jing1li4}[][HSK 3]
    \definition[个,次,段,种]{s.}{experiência; coisas que você viu, fez ou sofreu pessoalmente}
    \definition{v.}{passar por; atravessar; ter visto, feito ou sofrido pessoalmente}
  \end{Phonetics}
\end{Entry}

\begin{Entry}{经过}{8,6}{⽷、⾡}
  \begin{Phonetics}{经过}{jing1guo4}[][HSK 2]
    \definition{prep.}{depois; através; como resultado de; passar por uma atividade ou evento que traz novas mudanças para pessoas ou coisas}
    \definition[个,段,番]{s.}{processo; curso; experiência}
    \definition{v.}{passar; atravessar; passar por; através de (local, tempo, ação, etc.)}
  \end{Phonetics}
\end{Entry}

\begin{Entry}{经典}{8,8}{⽷、⼋}
  \begin{Phonetics}{经典}{jing1dian3}[][HSK 4]
    \definition{adj.}{clássico; (escritos ou obras, etc.) que são típicos, autorizados}
    \definition{s.}{clássicos; escritos tradicionais e valiosos; os livros mais importantes e fundamentais da religião | escrituras; escritos de doutrinas religiosas}
  \end{Phonetics}
\end{Entry}

\begin{Entry}{经济}{8,9}{⽷、⽔}
  \begin{Phonetics}{经济}{jing1ji4}[][HSK 3]
    \definition{adj.}{econômico;  parcimonioso; descreve algo que custa pouco e rende muito; preço acessível}
    \definition{s.}{economia; a soma das relações de produção social em um determinado período histórico|econômico; de valor industrial ou econômico; refere-se à economia nacional; também se refere a um determinado setor da economia nacional | economia; refere-se às atividades econômicas, incluindo produção, circulação, distribuição e consumo, bem como atividades ou processos financeiros, de seguros, etc. | renda; situação financeira; refere-se à situação financeira de uma pessoa}
    \definition{v.}{governar o país e beneficiar o povo}
  \end{Phonetics}
\end{Entry}

\begin{Entry}{经费}{8,9}{⽷、⾙}
  \begin{Phonetics}{经费}{jing1fei4}[][HSK 5]
    \definition[笔]{s.}{fundos; desembolso; gastos | despesas; gastos}
  \end{Phonetics}
\end{Entry}

\begin{Entry}{经验}{8,10}{⽷、⾺}
  \begin{Phonetics}{经验}{jing1yan4}[][HSK 3]
    \definition[个,次,种]{s.}{experiência; conhecimento ou habilidades adquiridos através da prática}
    \definition{v.}{experimentar; passar por; ter visto, feito ou sofrido pessoalmente}
  \end{Phonetics}
\end{Entry}

\begin{Entry}{经常}{8,11}{⽷、⼱}
  \begin{Phonetics}{经常}{jing1chang2}[][HSK 2]
    \definition{adj.}{habitual; cotidiano; diário; do dia a dia}
    \definition{adv.}{frequentemente; regularmente; constantemente; com frequência; indica que a ação ocorre repetidamente}
  \end{Phonetics}
\end{Entry}

\begin{Entry}{经理}{8,11}{⽷、⽟}
  \begin{Phonetics}{经理}{jing1li3}[][HSK 2]
    \definition[个,位,名]{s.}{gerente; diretor; pessoas responsáveis pela gestão e administração de empresas ou corporações}
  \end{Phonetics}
\end{Entry}

\begin{Entry}{经营}{8,11}{⽷、⾋}
  \begin{Phonetics}{经营}{jing1ying2}[][HSK 3]
    \definition{v.}{executar; gerenciar; operar; envolver-se em; planejar e gerenciar (empresas, etc.) | gerenciar; refere-se a planos e organizações em geral}
  \end{Phonetics}
\end{Entry}

\begin{Entry}{罔}{8}{⼌}
  \begin{Phonetics}{罔}{wang3}
    \definition{v.}{enganar}
  \end{Phonetics}
\end{Entry}

\begin{Entry}{罗}{8}{⽹}
  \begin{Phonetics}{罗}{luo2}
    \definition*{s.}{Sobrenome Luo}
    \definition{clas.}{uma grosa; uma bruta; doze dúzias; 144 unidades}
    \definition{s.}{uma rede para capturar pássaros | peneira; tela | uma espécie de gaze de seda}
    \definition{v.}{pegar pássaros com uma rede | espalhar; exibir; mostrar | coletar; reunir; recrutar | peneirar}
  \end{Phonetics}
\end{Entry}

\begin{Entry}{者}{8}{⽼}
  \begin{Phonetics}{者}{zhe3}[][HSK 3]
    \definition*{s.}{Sobrenome Zhe}
    \definition{part.}{significa 是 e é usado após palavras, frases e orações para indicar uma pausa}
    \definition{pron.}{usado para se referir à pessoa, coisa ou assunto que realiza uma ação ou possui um determinado atributo | pessoas; caras (Usado para se referir a alguém envolvido em uma determinada profissão, que acredita em uma determinada ideologia ou que tem uma forte tendência para algo) | usado após certos números ou palavras direcionais para se referir a coisas mencionadas anteriormente | significado semelhante a 这 (mais comum na linguagem coloquial antiga)}
  \seealsoref{是}{shi4}
  \seealsoref{这}{zhe4}
  \end{Phonetics}
\end{Entry}

\begin{Entry}{肏}{8}{⼊}
  \begin{Phonetics}{肏}{cao4}
    \definition{v.}{(vulgar) foder; palavras sujas usadas para insultar pessoas; refere-se à relação sexual masculina}
  \end{Phonetics}
\end{Entry}

\begin{Entry}{股}{8}{⾁}
  \begin{Phonetics}{股}{gu3}[][HSK 6]
    \definition*{s.}{Sobrenome Gu}
    \definition{clas.}{usado para coisas em tiras, longas e estreitas | usado para gás, odor, força, etc. | Pejorativo: usado para um grupo de pessoas}
    \definition{s.}{coxa; ancas | seção (de um escritório, empresa, etc.); unidades organizacionais em agências governamentais, empresas e grupos | fio; camada | uma das várias partes iguais de propriedade | ação; \emph{stock}; ação do capital social; uma parte igual de fundos ou propriedade | a perna mais longa de um triângulo retângulo}
  \end{Phonetics}
\end{Entry}

\begin{Entry}{股东}{8,5}{⾁、⼀}
  \begin{Phonetics}{股东}{gu3dong1}[][HSK 6]
    \definition[个,位,名,家]{s.}{acionista de uma sociedade anônima com direito a participar e votar nas assembleias gerais; refere-se também a investidores em outras empresas industriais e comerciais administradas por sociedades}
  \end{Phonetics}
\end{Entry}

\begin{Entry}{股票}{8,11}{⾁、⽰}
  \begin{Phonetics}{股票}{gu3piao4}[][HSK 6]
    \definition[只,股]{s.}{ação; quotas; certificado de ações; título de capital; capital social; títulos utilizados para representar ações}
  \end{Phonetics}
\end{Entry}

\begin{Entry}{肥}{8}{⾁}
  \begin{Phonetics}{肥}{fei2}[][HSK 4]
    \definition{adj.}{gordo; gorduroso; contém muita gordura (o oposto de 瘦, geralmente não usado para descrever pessoas) | fértil; rico | solto; largo; folgado; (roupas, etc.) largas (em oposição a 瘦) | lucrativo; rendendo bons lucros}
    \definition{s.}{fertilizante; esterco}
    \definition{v.}{fertilizar; tornar fértil ou obeso | enriquecer com renda ilegal, ilícita}
  \seealsoref{瘦}{shou4}
  \end{Phonetics}
\end{Entry}

\begin{Entry}{肩}{8}{⾁}
  \begin{Phonetics}{肩}{jian1}[][HSK 5]
    \definition*{s.}{Sobrenome Jian}
    \definition{s.}{ombro; torso}
    \definition{v.}{assumir; empreender; carregar; suportar; suportar um fardo}
  \end{Phonetics}
\end{Entry}

\begin{Entry}{肩膀}{8,14}{⾁、⾁}
  \begin{Phonetics}{肩膀}{jian1bang3}
    \definition{s.}{ombro}
  \end{Phonetics}
\end{Entry}

\begin{Entry}{肯}{8}{⾁}
  \begin{Phonetics}{肯}{ken3}[][HSK 6]
    \definition{s.}{carne presa ao osso}
    \definition{v.}{concordar; consentir}
    \definition{v.aux.}{estar disposto a; estar pronto para; para expressar vontade subjetiva; vontade de aceitar}
  \end{Phonetics}
\end{Entry}

\begin{Entry}{肯定}{8,8}{⾁、⼧}
  \begin{Phonetics}{肯定}{ken3ding4}[][HSK 5]
    \definition{adj.}{certo; definitivo; positivo; afirmativo | positivo; afirmativo; aceitável}
    \definition{adv.}{certamente; definitivamente; sem dúvida; sem dúvida alguma}
    \definition{v.}{afirmar; aprovar; confirmar; considerar positivo; reconhecer a existência de algo ou sua autenticidade ou racionalidade (em oposição à 否定)}
  \seealsoref{否定}{fou3ding4}
  \end{Phonetics}
\end{Entry}

\begin{Entry}{肺}{8}{⾁}
  \begin{Phonetics}{肺}{fei4}[][HSK 6]
    \definition[叶]{s.}{pulmão | pulmões; órgãos respiratórios de humanos e animais superiores}
  \end{Phonetics}
\end{Entry}

\begin{Entry}{肿}{8}{⾁}
  \begin{Phonetics}{肿}{zhong3}[][HSK 6]
    \definition{s.}{inchaço; protuberância}
    \definition{v.}{inchar; estar inchado}
  \end{Phonetics}
\end{Entry}

\begin{Entry}{舍}{8}{⾆}
  \begin{Phonetics}{舍}{she3}
    \definition{v.}{abandonar; desistir; descartar; jogar fora | dar esmola; dispensar caridade}
  \end{Phonetics}
  \begin{Phonetics}{舍}{she4}
    \definition*{s.}{Sobrenome She}
    \definition{clas.}{uma unidade antiga de distância igual a 30 li, 里}
    \definition{pron.}{meu, uma palavra humilde usada para se referir aos parentes mais jovens ou de geração inferior}
    \definition{s.}{cabana; casa | minha casa; minha humilde morada | chiqueiro; galpão; curral de gado}
  \seealsoref{里}{li3}
  \end{Phonetics}
\end{Entry}

\begin{Entry}{舍不得}{8,4,11}{⾆、⼀、⼻}
  \begin{Phonetics}{舍不得}{she3bu5de5}[][HSK 5]
    \definition{v.}{não se pode abandonar ou deixar, não se quer usar ou descartar; detestar separar-me ou usar}
  \end{Phonetics}
\end{Entry}

\begin{Entry}{舍得}{8,11}{⾆、⼻}
  \begin{Phonetics}{舍得}{she3 de5}[][HSK 5]
    \definition{v.}{não guardar rancor; estar disposto a abrir mão de algo; estar disposto a gastar dinheiro, tempo, etc.; estar disposto a abrir mão de pessoas, oportunidades, coisas, etc. que são importantes para você}
  \end{Phonetics}
\end{Entry}

\begin{Entry}{艰}{8}{⾉}
  \begin{Phonetics}{艰}{jian1}
    \definition{adj.}{difícil; duro}
  \end{Phonetics}
\end{Entry}

\begin{Entry}{艰苦}{8,8}{⾉、⾋}
  \begin{Phonetics}{艰苦}{jian1ku3}[][HSK 5]
    \definition{adj.}{duro; resistente; árduo; difícil; condições de trabalho ou de vida ruins que tornam as pessoas miseráveis}
  \end{Phonetics}
\end{Entry}

\begin{Entry}{艰难}{8,10}{⾉、⾫}
  \begin{Phonetics}{艰难}{jian1nan2}[][HSK 5]
    \definition{adj.}{duro; árduo; difícil}
  \end{Phonetics}
\end{Entry}

\begin{Entry}{若}{8}{⾋}
  \begin{Phonetics}{若}{ruo4}[][HSK 6]
    \definition*{s.}{Sobrenome Ruo}
    \definition{adv.}{como se; como se fosse; usado antes do verbo para indicar que o que foi dito é mais ou menos assim, equivalente a 好像}
    \definition{conj.}{se; usado na primeira parte de uma frase composta, expressa uma relação hipotética, equivalente a 如果}
    \definition{pron.}{você; referir-se ao interlocutor como 你 ou 你的}
    \definition{v.}{parecer}
  \seealsoref{好像}{hao3xiang4}
  \seealsoref{你}{ni3}
  \seealsoref{你的}{ni3 de5}
  \seealsoref{如果}{ru2guo3}
  \end{Phonetics}
\end{Entry}

\begin{Entry}{苦}{8}{⾋}
  \begin{Phonetics}{苦}{ku3}[][HSK 4]
    \definition{adj.}{amargo; descreve um sabor parecido com o de melão amargo ou raiz de coptis (em oposição a 甘 ou 甜) | difícil; doloroso; sofrido}
    \definition{adv.}{meticulosamente; diligentemente; pacientemente}
    \definition{v.}{causar sofrimento a alguém; dificultar a vida de alguém; causar dor; tornar desconfortável | sofrer de; ser incomodado por; sentir-se angustiado com uma situação | estar desgastado; cortar demais; descrever a superação de um certo nível em algum aspecto}
  \seealsoref{甘}{gan1}
  \seealsoref{甜}{tian2}
  \end{Phonetics}
\end{Entry}

\begin{Entry}{苦瓜}{8,5}{⾋、⽠}
  \begin{Phonetics}{苦瓜}{ku3gua1}
    \definition{s.}{melão amargo (cabaça amarga, pêra bálsamo, maçã bálsamo, pepino amargo)}
  \end{Phonetics}
\end{Entry}

\begin{Entry}{英}{8}{⾋}
  \begin{Phonetics}{英}{ying1}
    \definition*{s.}{Reino Unido, abreviação de 英国 | Sobrenome Ying}
    \definition{s.}{flor | herói; pessoa excepcional | uma pessoa de talento ou sabedoria extraordinários}
  \seealsoref{英国}{ying1guo2}
  \end{Phonetics}
\end{Entry}

\begin{Entry}{英文}{8,4}{⾋、⽂}
  \begin{Phonetics}{英文}{ying1 wen2}[][HSK 2]
    \definition{s.}{inglês, língua inglesa; a forma escrita do inglês}
  \end{Phonetics}
\end{Entry}

\begin{Entry}{英国}{8,8}{⾋、⼞}
  \begin{Phonetics}{英国}{ying1guo2}
    \definition*{s.}{Reino Unido; Grã-Bretanha; Inglaterra}
  \end{Phonetics}
\end{Entry}

\begin{Entry}{英国人}{8,8,2}{⾋、⼞、⼈}
  \begin{Phonetics}{英国人}{ying1guo2ren2}
    \definition{s.}{inglês | pessoa ou povo do Reino Unido}
  \end{Phonetics}
\end{Entry}

\begin{Entry}{英勇}{8,9}{⾋、⼒}
  \begin{Phonetics}{英勇}{ying1yong3}[][HSK 4]
    \definition{adj.}{heroico; valente; bravo; corajoso; extraordinariamente corajoso}
  \end{Phonetics}
\end{Entry}

\begin{Entry}{英语}{8,9}{⾋、⾔}
  \begin{Phonetics}{英语}{ying1 yu3}[][HSK 2]
    \definition{s.}{inglês, língua inglesa}
  \end{Phonetics}
\end{Entry}

\begin{Entry}{英雄}{8,12}{⾋、⾫}
  \begin{Phonetics}{英雄}{ying1xiong2}[][HSK 6]
    \definition{adj.}{heróico}
    \definition[名,个,位]{s.}{herói; uma pessoa cujas habilidades e coragem superam as das pessoas comuns | herói; aqueles que não têm medo das dificuldades, dos perigos ou da morte e que lutam bravamente pelos interesses do povo, mesmo ao custo das suas próprias vidas}
  \end{Phonetics}
\end{Entry}

\begin{Entry}{苹}{8}{⾋}
  \begin{Phonetics}{苹}{ping2}
    \definition[个]{s.}{uma espécie de artemísia | maçã | lentilha-d'água}
  \end{Phonetics}
\end{Entry}

\begin{Entry}{苹果}{8,8}{⾋、⽊}
  \begin{Phonetics}{苹果}{ping2guo3}[][HSK 3]
    \definition[个,斤,筐,箱,棵,种]{s.}{maçã}
  \end{Phonetics}
\end{Entry}

\begin{Entry}{茄}{8}{⾋}
  \begin{Phonetics}{茄}{jia1}
    \definition{s.}{caracter fonético usado em empréstimos linguísticos para o som "jia", embora 夹 seja mais comum}
  \seealsoref{夹}{jia1}
  \end{Phonetics}
  \begin{Phonetics}{茄}{qie2}
    \definition[只]{s.}{berinjela}
  \end{Phonetics}
\end{Entry}

\begin{Entry}{茄子}{8,3}{⾋、⼦}
  \begin{Phonetics}{茄子}{qie2 zi5}[][HSK 6]
    \definition{interj.}{Onomatopéia: ``xis'' fonético (ao ser fotografado), equivale ao ``diga xis''}
    \definition[个,根]{s.}{berinjela (fruto e planta)}
  \end{Phonetics}
\end{Entry}

\begin{Entry}{茅}{8}{⾋}
  \begin{Phonetics}{茅}{mao2}
    \definition*{s.}{Sobrenome Mao}
    \definition[座]{s.}{capim-cogon | planta semelhante ao capim-cogon (como palha)}
  \end{Phonetics}
\end{Entry}

\begin{Entry}{茅厕}{8,8}{⾋、⼚}
  \begin{Phonetics}{茅厕}{mao2ce4}
    \definition{s.}{(dialeto) latrina}
  \end{Phonetics}
\end{Entry}

\begin{Entry}{虎}{8}{⾌}
  \begin{Phonetics}{虎}{hu3}[][HSK 5]
    \definition*{s.}{Sobrenome Hu}
    \definition{adj.}{corajoso; bravo; valente; vigoroso}
    \definition[只]{s.}{tigre}
    \definition{v.}{blefar; o mesmo que 唬 | parecer feroz; mostrar a aparência feroz de alguém}
  \seealsoref{唬}{hu3}
  \seealsoref{老虎}{lao3hu3}
  \end{Phonetics}
\end{Entry}

\begin{Entry}{虎口}{8,3}{⾌、⼝}
  \begin{Phonetics}{虎口}{hu3kou3}
    \definition{s.}{lugar perigoso | cova do tigre}
  \end{Phonetics}
\end{Entry}

\begin{Entry}{虎虎}{8,8}{⾌、⾌}
  \begin{Phonetics}{虎虎}{hu3hu3}
    \definition{adj.}{formidável | forte | vigoroso}
  \end{Phonetics}
\end{Entry}

\begin{Entry}{虎鼬}{8,18}{⾌、⿏}
  \begin{Phonetics}{虎鼬}{hu3you4}
    \definition{s.}{doninha}
  \end{Phonetics}
\end{Entry}

\begin{Entry}{表}{8}{⾐}
  \begin{Phonetics}{表}{biao3}[][HSK 2]
    \definition*{s.}{Sobrenome Biao}
    \definition{s.}{exterior; superfície; externo | a relação entre os filhos ou netos de um irmão e uma irmã ou de irmãs | modelo; exemplo; padrão | memorial a um imperador; um tipo de petição da antiguidade, frequentemente usado para expressar intenções; mais tarde, também usado para expressar opiniões sobre eventos importantes | formulário; lista; gráfico; tabela | medidor; instrumento para medir uma determinada quantidade | relógio; um dispositivo para medir o tempo, menor que um relógio, que geralmente pode ser carregado no bolso | medidor de luz solar; antiga vara de madeira para medir o tempo através da sombra do sol | coluna usada antigamente para marcação}
    \definition{v.}{mostrar; expressar; expressar ideias, pensamentos, sentimentos, etc. | administrar medicamentos para aliviar o resfriado; na medicina tradicional chinesa refere-se ao uso de medicamentos para dissipar o frio e o vento que afetam o corpo}
  \end{Phonetics}
\end{Entry}

\begin{Entry}{表白}{8,5}{⾐、⽩}
  \begin{Phonetics}{表白}{biao3bai2}[][HSK 7-9]
    \definition{v.}{justificar; explicar-se; expressar ou declarar claramente; explicar (as próprias intenções) aos outros}
  \end{Phonetics}
\end{Entry}

\begin{Entry}{表示}{8,5}{⾐、⽰}
  \begin{Phonetics}{表示}{biao3shi4}[][HSK 2]
    \definition{s.}{expressão; indicação}
    \definition{v.}{mostrar; expressar; indicar | significar | expressar pensamentos e sentimentos através de palavras, ações ou expressões faciais}
  \end{Phonetics}
\end{Entry}

\begin{Entry}{表决}{8,6}{⾐、⼎}
  \begin{Phonetics}{表决}{biao3jue2}[][HSK 7-9]
    \definition{v.}{votar; colocar em votação; decidir por votação}
  \end{Phonetics}
\end{Entry}

\begin{Entry}{表扬}{8,6}{⾐、⼿}
  \begin{Phonetics}{表扬}{biao3yang2}[][HSK 4]
    \definition{v.}{elogiar; louvar; elogiar publicamente as pessoas boas e as boas ações}
  \end{Phonetics}
\end{Entry}

\begin{Entry}{表扬信}{8,6,9}{⾐、⼿、⼈}
  \begin{Phonetics}{表扬信}{biao3yang2 xin4}
    \definition{s.}{carta de elogio; depoimento}
  \end{Phonetics}
\end{Entry}

\begin{Entry}{表达}{8,6}{⾐、⾡}
  \begin{Phonetics}{表达}{biao3da2}[][HSK 3]
    \definition{v.}{entregar; expressar; mostrar; manifestar; transmitir; comunicar; refere-se ao processo de transmitir pensamentos, sentimentos ou opiniões pessoais a outras pessoas por meio de linguagem, texto, ações, etc.}
  \end{Phonetics}
\end{Entry}

\begin{Entry}{表态}{8,8}{⾐、⼼}
  \begin{Phonetics}{表态}{biao3/tai4}[][HSK 7-9]
    \definition{v.+compl.}{tornar conhecida a sua posição; declarar onde se posiciona; comprometer-se; expressar claramente a atitude de alguém em relação a algo}
  \end{Phonetics}
\end{Entry}

\begin{Entry}{表明}{8,8}{⾐、⽇}
  \begin{Phonetics}{表明}{biao3ming2}[][HSK 3]
    \definition{v.}{indicar; demonstrar; expressar; marcar; expressar claramente; expressar de forma clara}
  \end{Phonetics}
\end{Entry}

\begin{Entry}{表现}{8,8}{⾐、⾒}
  \begin{Phonetics}{表现}{biao3xian4}[][HSK 3]
    \definition[个,种,份]{s.}{desempenho; expressão; manifestação; comportamento; as ideias, o estilo, as qualidades, o nível ou as capacidades demonstrados em ação.}
    \definition{v.}{mostrar; expressar; exibir; manifestar; descrever; demonstrar algum tipo de pensamento, espírito, qualidade, sentimento ou habilidade, etc. | exibir-se; demonstrar de forma inadequada e intencional alguma habilidade, ponto forte ou vantagem.}
  \end{Phonetics}
\end{Entry}

\begin{Entry}{表述}{8,8}{⾐、⾡}
  \begin{Phonetics}{表述}{biao3shu4}[][HSK 7-9]
    \definition{s.}{formulação; expressão}
    \definition{v.}{declarar; explicar; descrever em palavras ou texto}
  \end{Phonetics}
\end{Entry}

\begin{Entry}{表面}{8,9}{⾐、⾯}
  \begin{Phonetics}{表面}{biao3mian4}[][HSK 3]
    \definition{s.}{superfície; face; exterior; aparência | aparência; superficialidade | mostrador (placa); mostrador do relógio | aparência; a aparência externa das coisas ou a parte não essencial delas}
  \end{Phonetics}
\end{Entry}

\begin{Entry}{表面上}{8,9,3}{⾐、⾯、⼀}
  \begin{Phonetics}{表面上}{biao3 mian4 shang4}[][HSK 6]
    \definition{adj.}{superficial; ostensivo; aparente}
  \end{Phonetics}
\end{Entry}

\begin{Entry}{表格}{8,10}{⾐、⽊}
  \begin{Phonetics}{表格}{biao3ge2}[][HSK 3]
    \definition[张,份,个]{s.}{tabela; formulário}
  \end{Phonetics}
\end{Entry}

\begin{Entry}{表情}{8,11}{⾐、⼼}
  \begin{Phonetics}{表情}{biao3qing2}[][HSK 4]
    \definition[个,种,幅]{s.}{expressão; expressão facial; expressão de pensamentos e sentimentos internos por meio de mudanças faciais ou de gestos}
    \definition{v.}{expressar pensamentos e sentimentos internos por meio de mudanças faciais ou de gestos}
  \end{Phonetics}
\end{Entry}

\begin{Entry}{表率}{8,11}{⾐、⽞}
  \begin{Phonetics}{表率}{biao3shuai4}[][HSK 7-9]
    \definition{s.}{modelo; exemplo}
  \end{Phonetics}
\end{Entry}

\begin{Entry}{表彰}{8,14}{⾐、⼺}
  \begin{Phonetics}{表彰}{biao3zhang1}[][HSK 7-9]
    \definition{v.}{citar; honrar; elogiar}
  \end{Phonetics}
\end{Entry}

\begin{Entry}{表演}{8,14}{⾐、⽔}
  \begin{Phonetics}{表演}{biao3yan3}[][HSK 3]
    \definition[场]{s.}{performance; exposição; refere-se às atividades expressas pelos atores por meio da linguagem, voz, expressões faciais, instrumentos musicais ou movimentos}
    \definition{v.}{atuar; representar; interpretar | demonstrar; fazer demonstrações | fingir; agir de forma afetada; metáfora para fingir deliberadamente uma determinada atitude para enganar alguém}
  \end{Phonetics}
\end{Entry}

\begin{Entry}{表演艺术}{8,14,4,5}{⾐、⽔、⾋、⽊}
  \begin{Phonetics}{表演艺术}{biao3yan3 yi4shu4}
    \definition{s.}{arte performática}
  \end{Phonetics}
\end{Entry}

\begin{Entry}{表演者}{8,14,8}{⾐、⽔、⽼}
  \begin{Phonetics}{表演者}{biao3yan3 zhe3}
    \definition{s.}{artista; intérprete}
  \end{Phonetics}
\end{Entry}

\begin{Entry}{表演特技}{8,14,10,7}{⾐、⽔、⽜、⼿}
  \begin{Phonetics}{表演特技}{biao3yan3 te4ji4}
    \definition{s.}{acrobacia | pirueta | façanha}
  \end{Phonetics}
\end{Entry}

\begin{Entry}{表演游戏}{8,14,12,6}{⾐、⽔、⽔、⼽}
  \begin{Phonetics}{表演游戏}{biao3yan3 you2xi4}
    \definition{s.}{exibição dramática}
  \end{Phonetics}
\end{Entry}

\begin{Entry}{表演赛}{8,14,14}{⾐、⽔、⾙}
  \begin{Phonetics}{表演赛}{biao3yan3sai4}
    \definition{s.}{partida de exibição; jogo de exibição; uma competição realizada para celebração, comemoração, demonstração, publicidade, etc.}
  \end{Phonetics}
\end{Entry}

\begin{Entry}{衬}{8}{⾐}
  \begin{Phonetics}{衬}{chen4}
    \definition[件,个]{s.}{forro}
    \definition{v.}{forrar; colocar algo embaixo | fornecer um pano de fundo para; destacar; servir como contraste para}
  \end{Phonetics}
\end{Entry}

\begin{Entry}{衬托}{8,6}{⾐、⼿}
  \begin{Phonetics}{衬托}{chen4tuo1}[][HSK 7-9]
    \definition{v.}{destacar; acentuar}[红色衬托了她的笑容。===O vermelho acentuava seu sorriso.]
  \end{Phonetics}
\end{Entry}

\begin{Entry}{衬衣}{8,6}{⾐、⾐}
  \begin{Phonetics}{衬衣}{chen4 yi1}[][HSK 3]
    \definition[件,个]{s.}{camisa; também se refere a uma peça de roupa usada por baixo do casaco}
  \end{Phonetics}
\end{Entry}

\begin{Entry}{衬衫}{8,8}{⾐、⾐}
  \begin{Phonetics}{衬衫}{chen4shan1}[][HSK 3]
    \definition[件,个]{s.}{camisa; blusa; camisa ocidental usada por baixo}
  \end{Phonetics}
\end{Entry}

\begin{Entry}{规}{8}{⾒}
  \begin{Phonetics}{规}{gui1}
    \definition*{s.}{Sobrenome Gui}
    \definition[个,种]{s.}{bússola | regulamentação; regra | (mecânica) medidor | compasso; ferramenta para desenhar círculos}
    \definition{v.}{admoestar; aconselhar; advertir | planejar; fazer planos}
  \end{Phonetics}
\end{Entry}

\begin{Entry}{规划}{8,6}{⾒、⼑}
  \begin{Phonetics}{规划}{gui1hua4}[][HSK 5]
    \definition[个,项]{s.}{plano; projeto; planejamento; programa; programação; esquematização; plano de desenvolvimento de longo prazo mais abrangente}
    \definition{v.}{planejar; programar}
  \end{Phonetics}
\end{Entry}

\begin{Entry}{规则}{8,6}{⾒、⼑}
  \begin{Phonetics}{规则}{gui1ze2}[][HSK 4]
    \definition{adj.}{ordenado; regular; descreve a forma, estrutura, arranjo, etc., que se conformam a uma determinada maneira organizada}
    \definition{s.}{regra; regulamento; sistema ou código de conduta prescrito para observância comum | lei; norma}
  \end{Phonetics}
\end{Entry}

\begin{Entry}{规定}{8,8}{⾒、⼧}
  \begin{Phonetics}{规定}{gui1ding4}[][HSK 3]
    \definition[个,条,项,款]{s.}{regra; regulamento; estipulação; tomar decisões sobre a forma, o método, a quantidade ou a qualidade de algo}
    \definition{v.}{estipular; prover; prescrever; estabelecer requisitos ou restrições em termos de métodos, qualidade, quantidade, tempo, etc.}
  \end{Phonetics}
\end{Entry}

\begin{Entry}{规律}{8,9}{⾒、⼻}
  \begin{Phonetics}{规律}{gui1lv4}[][HSK 4]
    \definition{adj.}{estável; regular; coisas, comportamentos, fenômenos, etc. que ocorrem em um determinado momento}
    \definition{s.}{lei; padrão regular; conexão essencial e recorrente entre as coisas}
  \end{Phonetics}
\end{Entry}

\begin{Entry}{规范}{8,9}{⾒、⾋}
  \begin{Phonetics}{规范}{gui1fan4}[][HSK 3]
    \definition{adj.}{regular; normal; padrão; que atende às especificações; em conformidade com as normas}
    \definition{s.}{norma; padrão; diretriz}
    \definition{v.}{regular; padronizar; tornar conforme as normas}
  \end{Phonetics}
\end{Entry}

\begin{Entry}{规模}{8,14}{⾒、⽊}
  \begin{Phonetics}{规模}{gui1mo2}[][HSK 4]
    \definition[个,种]{s.}{escala; escopo; dimensões; padrão, forma ou escopo (de um empreendimento, instituição, projeto, movimento, etc.)}
  \end{Phonetics}
\end{Entry}

\begin{Entry}{视}{8}{⾒}
  \begin{Phonetics}{视}{shi4}
    \definition{v.}{olhar para | considerar; olhar para | inspecionar; observar}
  \end{Phonetics}
\end{Entry}

\begin{Entry}{视为}{8,4}{⾒、⼂}
  \begin{Phonetics}{视为}{shi4 wei2}[][HSK 5]
    \definition{v.}{considerar; ver como; considerar como; considerar ser; achar que é}
  \end{Phonetics}
\end{Entry}

\begin{Entry}{视角}{8,7}{⾒、⾓}
  \begin{Phonetics}{视角}{shi4jiao3}
    \definition{s.}{ângulo do qual se observa um objeto | (figurativo) perspectiva, ponto de vista, quadro de referência | (cinematografia) ângulo da câmera | (percepção visual) ângulo visual (o ângulo que um objeto visto subtende no olho) | (fotografia) ângulo de visão}
  \end{Phonetics}
\end{Entry}

\begin{Entry}{视频}{8,13}{⾒、⾴}
  \begin{Phonetics}{视频}{shi4pin2}[][HSK 5]
    \definition[个,段,条]{s.}{vídeo; videoclipe}
  \end{Phonetics}
\end{Entry}

\begin{Entry}{试}{8}{⾔}
  \begin{Phonetics}{试}{shi4}[][HSK 1]
    \definition{s.}{teste; exame; avaliação de conhecimentos ou habilidades através de métodos específicos}
    \definition{v.}{tentar; investigar resultados ou verificar a natureza, não se envolver formalmente (em determinada atividade)}
  \end{Phonetics}
\end{Entry}

\begin{Entry}{试卷}{8,8}{⾔、⼙}
  \begin{Phonetics}{试卷}{shi4juan4}[][HSK 4]
    \definition[分,张]{s.}{folha de teste; folha de exame; papel usado para escrever as respostas nos exames}
  \end{Phonetics}
\end{Entry}

\begin{Entry}{试图}{8,8}{⾔、⼞}
  \begin{Phonetics}{试图}{shi4tu2}[][HSK 5]
    \definition{v.}{tentar; pretender, fazer o possível para realizar algo}
  \end{Phonetics}
\end{Entry}

\begin{Entry}{试点}{8,9}{⾔、⽕}
  \begin{Phonetics}{试点}{shi4 dian3}[][HSK 6]
    \definition[个]{s.}{local onde um experimento é conduzido; unidade experimental; local de teste; um lugar para pequenos experimentos}
    \definition{v.}{experimentar; fazer experimentos; realizar testes em pontos selecionados; lançar um projeto piloto}
  \end{Phonetics}
\end{Entry}

\begin{Entry}{试验}{8,10}{⾔、⾺}
  \begin{Phonetics}{试验}{shi4yan4}[][HSK 3]
    \definition{v.}{testar; fazer um teste; fazer um experimento; para examinar o efeito ou desempenho de algo, primeiro experimente em um laboratório ou em uma escala menor}
  \end{Phonetics}
\end{Entry}

\begin{Entry}{试题}{8,15}{⾔、⾴}
  \begin{Phonetics}{试题}{shi4 ti2}[][HSK 3]
    \definition[道]{s.}{questões de um exame}
  \end{Phonetics}
\end{Entry}

\begin{Entry}{诗}{8}{⾔}
  \begin{Phonetics}{诗}{shi1}[][HSK 4]
    \definition[首,句,行]{s.}{poesia; verso; poema; um gênero literário que reflete a vida e expressa emoções por meio de uma linguagem rítmica e rimada}
  \seealsoref{诗经}{shi1jing1}
  \end{Phonetics}
\end{Entry}

\begin{Entry}{诗人}{8,2}{⾔、⼈}
  \begin{Phonetics}{诗人}{shi1 ren2}[][HSK 4]
    \definition[个,位,名,些]{s.}{poeta; escritor de poesia}
  \end{Phonetics}
\end{Entry}

\begin{Entry}{诗句}{8,5}{⾔、⼝}
  \begin{Phonetics}{诗句}{shi1ju4}
    \definition[行]{s.}{verso | versículo}
  \end{Phonetics}
\end{Entry}

\begin{Entry}{诗词}{8,7}{⾔、⾔}
  \begin{Phonetics}{诗词}{shi1ci2}
    \definition{s.}{verso}
  \end{Phonetics}
\end{Entry}

\begin{Entry}{诗经}{8,8}{⾔、⽷}
  \begin{Phonetics}{诗经}{shi1jing1}
    \definition*{s.}{Shijing, o Livro das Canções, antiga coleção de poemas chineses e um dos Cinco Clássicos do Confucionismo}
  \end{Phonetics}
\end{Entry}

\begin{Entry}{诗意}{8,13}{⾔、⼼}
  \begin{Phonetics}{诗意}{shi1yi4}
    \definition{adj.}{poético}
    \definition{s.}{poesia}
  \end{Phonetics}
\end{Entry}

\begin{Entry}{诗歌}{8,14}{⾔、⽋}
  \begin{Phonetics}{诗歌}{shi1 ge1}[][HSK 5]
    \definition[本,首,段]{s.}{poesia; poemas e canções; refere-se a todos os tipos de poesia}
  \end{Phonetics}
\end{Entry}

\begin{Entry}{诚}{8}{⾔}
  \begin{Phonetics}{诚}{cheng2}
    \definition{adj.}{sincero; honesto; verdadeiro}
    \definition{adv.}{na verdade; realmente; de fato}
    \definition{s.}{sinceridade; genuinidade; seriedade}
  \end{Phonetics}
\end{Entry}

\begin{Entry}{诚心诚意}{8,4,8,13}{⾔、⼼、⾔、⼼}
  \begin{Phonetics}{诚心诚意}{cheng2xin1-cheng2yi4}[][HSK 7-9]
    \definition{expr.}{sincero e sério; com toda a sinceridade; é uma expressão idiomática chinesa que vem da Biografia de Ma Yuan no Livro da Dinastia Han Posterior | genuíno; sincero}
  \end{Phonetics}
\end{Entry}

\begin{Entry}{诚实}{8,8}{⾔、⼧}
  \begin{Phonetics}{诚实}{cheng2shi2}[][HSK 4]
    \definition{adj.}{honesto; sincero e honesto, não hipócrita}
  \end{Phonetics}
\end{Entry}

\begin{Entry}{诚实地}{8,8,6}{⾔、⼧、⼟}
  \begin{Phonetics}{诚实地}{cheng2shi2 di4}
    \definition{adv.}{honestamente}
  \end{Phonetics}
\end{Entry}

\begin{Entry}{诚信}{8,9}{⾔、⼈}
  \begin{Phonetics}{诚信}{cheng2 xin4}[][HSK 4]
    \definition{adj.}{honesto e confiável}
    \definition[种]{s.}{fé; honestidade; padrão e princípio de comportamento: não contar mentiras, prometer aos outros o que eles podem fazer e ter a confiança dos outros}
  \end{Phonetics}
\end{Entry}

\begin{Entry}{诚恳}{8,10}{⾔、⼼}
  \begin{Phonetics}{诚恳}{cheng2ken3}[][HSK 7-9]
    \definition{adj.}{sincero; sério; a atitude é muito real e pé no chão}
  \end{Phonetics}
\end{Entry}

\begin{Entry}{诚挚}{8,10}{⾔、⼿}
  \begin{Phonetics}{诚挚}{cheng2zhi4}[][HSK 7-9]
    \definition{adj.}{sincero; cordial; honesto}
  \end{Phonetics}
\end{Entry}

\begin{Entry}{诚意}{8,13}{⾔、⼼}
  \begin{Phonetics}{诚意}{cheng2yi4}[][HSK 7-9]
    \definition{s.}{boa fé; sinceridade; intenções sinceras}
  \end{Phonetics}
\end{Entry}

\begin{Entry}{话}{8}{⾔}
  \begin{Phonetics}{话}{hua4}[][HSK 1]
    \definition[句,段,番,种]{s.}{palavra; conversa; a voz que expressa os pensamentos quando falada, ou os caracteres que registram essa voz}
    \definition{v.}{falar sobre; falar a respeito}
  \end{Phonetics}
\end{Entry}

\begin{Entry}{话剧}{8,10}{⾔、⼑}
  \begin{Phonetics}{话剧}{hua4 ju4}[][HSK 3]
    \definition[场,幕,部,出,台]{s.}{drama moderno; peça de teatro; peça teatral representada através de diálogos e ações}
  \end{Phonetics}
\end{Entry}

\begin{Entry}{话题}{8,15}{⾔、⾴}
  \begin{Phonetics}{话题}{hua4ti2}[][HSK 3]
    \definition[个,种,项]{s.}{assunto de uma palestra; tópico de uma conversa; o foco da conversa}
  \end{Phonetics}
\end{Entry}

\begin{Entry}{诞}{8}{⾔}
  \begin{Phonetics}{诞}{dan4}
    \definition{adj.}{absurdo; fantástico; irreal; irracional}
    \definition{adv.}{absurdamente; fantasticamente}
    \definition{s.}{aniversário de nascimento | nascimento}
    \definition{v.}{nascer | dar à luz}
  \end{Phonetics}
\end{Entry}

\begin{Entry}{诞生}{8,5}{⾔、⽣}
  \begin{Phonetics}{诞生}{dan4sheng1}[][HSK 6]
    \definition{v.}{nascer; vir a existir; uma pessoa nasce; também significa que algo novo surgiu e tem um impacto positivo na sociedade}
  \end{Phonetics}
\end{Entry}

\begin{Entry}{诞辰}{8,7}{⾔、⾠}
  \begin{Phonetics}{诞辰}{dan4chen2}[][HSK 7-9]
    \definition[周年]{s.}{aniversário (usado principalmente para pessoas respeitadas)}[9月28日是孔子诞辰日。===28 de setembro é o aniversário de Confúcio.]
  \end{Phonetics}
\end{Entry}

\begin{Entry}{诟}{8}{⾔}
  \begin{Phonetics}{诟}{gou4}
    \definition*{s.}{Sobrenome Gou}
    \definition{s.}{vergonha; humilhação}
    \definition{v.}{insultar; xingar; falar de forma abusiva}
  \end{Phonetics}
\end{Entry}

\begin{Entry}{诟骂}{8,9}{⾔、⾺}
  \begin{Phonetics}{诟骂}{gou4ma4}
    \definition{v.}{abusar verbalmente | insultar | criticar}
  \end{Phonetics}
\end{Entry}

\begin{Entry}{询}{8}{⾔}
  \begin{Phonetics}{询}{xun2}
    \definition{v.}{perguntar; indagar; reunir informações | consultar; buscar conselho}
  \end{Phonetics}
\end{Entry}

\begin{Entry}{询问}{8,6}{⾔、⾨}
  \begin{Phonetics}{询问}{xun2wen4}[][HSK 5]
    \definition{v.}{indagar; perguntar sobre; pedir conselho}
  \end{Phonetics}
\end{Entry}

\begin{Entry}{该}{8}{⾔}
  \begin{Phonetics}{该}{gai1}[][HSK 2]
    \definition{adj.}{completo; integral; abrangente; inclusivo; o mesmo que 赅}
    \definition{pron.}{isto; aquilo; o referido; o acima mencionado; indica a pessoa ou coisa mencionada acima, equivalente a 此, 这个, etc.}
    \definition{v.}{deveria ser; deveria ser assim | caber a alguém; ser a vez (ou dever) de alguém fazer algo | merecer; servir a alguém de direito; indica que algo deve ser feito | dever | deve; provavelmente irá; muito provavelmente; pode ser razoavelmente ou naturalmente esperado que; expressa uma conclusão lógica ou provável com base na razão ou na experiência}
    \definition{v.aux.}{usado em frases exclamativas, tem a função de reforçar o tom}
  \seealsoref{此}{ci3}
  \seealsoref{赅}{gai1}
  \seealsoref{这个}{zhe4ge5}
  \end{Phonetics}
\end{Entry}

\begin{Entry}{详}{8}{⾔}
  \begin{Phonetics}{详}{xiang2}
    \definition{adj.}{conhecido; reconhecido; saber claramente | detalhado; minucioso; pormenorizado (oposto a 略)}
    \definition{s.}{detalhes; particularidades}
    \definition{v.}{contar; explicar; elaborar | saber claramente}
  \seealsoref{略}{lve4}
  \end{Phonetics}
\end{Entry}

\begin{Entry}{详细}{8,8}{⾔、⽷}
  \begin{Phonetics}{详细}{xiang2xi4}[][HSK 5]
    \definition{adj.}{explícito; detalhado; minucioso; circunstancial; meticuloso}
  \end{Phonetics}
\end{Entry}

\begin{Entry}{诧}{8}{⾔}
  \begin{Phonetics}{诧}{cha4}
    \definition{v.}{ficar surpreso}
  \end{Phonetics}
\end{Entry}

\begin{Entry}{诧异}{8,6}{⾔、⼶}
  \begin{Phonetics}{诧异}{cha4yi4}[][HSK 7-9]
    \definition{v.}{ficar surpreso; ficar espantado}
  \end{Phonetics}
\end{Entry}

\begin{Entry}{责}{8}{⾙}
  \begin{Phonetics}{责}{ze2}
    \definition{s.}{dever; responsabilidade}
    \definition{v.}{exigir; requerer; exigir que algo seja feito ou que atenda a certos padrões | questionar atentamente; chamar alguém para prestar contas; interrogar| reprovar; culpar | punir}
  \end{Phonetics}
\end{Entry}

\begin{Entry}{责任}{8,6}{⾙、⼈}
  \begin{Phonetics}{责任}{ze2ren4}[][HSK 3]
    \definition[个,种,份]{s.}{dever; responsabilidade; de acordo com a profissão, cargo, identidade, etc., as coisas que você deve fazer ou as tarefas que deve assumir | culpa; responsabilidade por uma falha ou erro; não ter feito o que era sua obrigação e, portanto, ser responsável pela falha}
  \end{Phonetics}
\end{Entry}

\begin{Entry}{责怪}{8,8}{⾙、⼼}
  \begin{Phonetics}{责怪}{ze2guai4}
    \definition{v.}{repreender | censurar}
  \end{Phonetics}
\end{Entry}

\begin{Entry}{败}{8}{⾒}
  \begin{Phonetics}{败}{bai4}[][HSK 4]
    \definition{adj.}{ruim; deteriorado; murcho; dilapidado; decadente}
    \definition{v.}{ser derrotado; perder (oposto a 胜) | derrotar; bater | falha (oposto a 成) | estragar; arruinar | decair; murchar | quebrar; neutralizar; dissipar}
  \seealsoref{成}{cheng2}
  \seealsoref{胜}{sheng4}
  \end{Phonetics}
\end{Entry}

\begin{Entry}{账}{8}{⾙}
  \begin{Phonetics}{账}{zhang4}[][HSK 6]
    \definition[笔,本]{s.}{conta | livro de contas | dívida; conta | crédito (de dívidas)}
  \end{Phonetics}
\end{Entry}

\begin{Entry}{账户}{8,4}{⾙、⼾}
  \begin{Phonetics}{账户}{zhang4hu4}[][HSK 6]
    \definition[个]{s.}{conta; refere-se à classificação de vários usos de fundos, fontes e processos de rotatividade no livro contábil}
  \end{Phonetics}
\end{Entry}

\begin{Entry}{货}{8}{⾙}
  \begin{Phonetics}{货}{huo4}[][HSK 4]
    \definition{s.}{dinheiro; moeda | bens; mercadorias; \emph{commodity} | refere-se a uma pessoa com um certo mau caráter (usado como um insulto) | riqueza; fortuna; um termo geral para dinheiro, joias, tecidos, etc.}
    \definition{v.}{vender}
  \end{Phonetics}
\end{Entry}

\begin{Entry}{货车}{8,4}{⾙、⾞}
  \begin{Phonetics}{货车}{huo4che1}
    \definition{s.}{caminhão | van | vagão de carga}
  \end{Phonetics}
\end{Entry}

\begin{Entry}{质}{8}{⾙}
  \begin{Phonetics}{质}{zhi4}
    \definition*{s.}{Sobrenome Zhi}
    \definition{adj.}{simples; claro; sem adornos}
    \definition{s.}{natureza; caráter; essência; substância | qualidade | matéria; substância | segurança; penhor; garantia}
    \definition{v.}{penhorar | hipotecar | questionar; chamar à responsabilidade; acusar}
  \end{Phonetics}
\end{Entry}

\begin{Entry}{质量}{8,12}{⾙、⾥}
  \begin{Phonetics}{质量}{zhi4liang4}[][HSK 4]
    \definition{s.}{qualidade; o quão bom ou ruim é o produto ou o trabalho | Física: massa}
  \end{Phonetics}
\end{Entry}

\begin{Entry}{贪}{8}{⾙}
  \begin{Phonetics}{贪}{tan1}
    \definition{adj.}{corrupto; venal | ganancioso; avarento; ambicioso}
    \definition{v.}{apropriar-se indevidamente; praticar corrupção; ser corrupto | ter um desejo insaciável por; ter um desejo voraz por | cobiçar; ansiar por; ser ganancioso por}
  \end{Phonetics}
\end{Entry}

\begin{Entry}{贪婪}{8,11}{⾙、⼥}
  \begin{Phonetics}{贪婪}{tan1lan2}
    \definition{adj.}{avaro | ambicioso | voraz | insaciável}
  \end{Phonetics}
\end{Entry}

\begin{Entry}{贫}{8}{⾙}
  \begin{Phonetics}{贫}{pin2}
    \definition{adj.}{pobre; empobrecido | inadequado; deficiente; insuficiente | tagarela; loquaz; falante; chato e irritante}
  \end{Phonetics}
\end{Entry}

\begin{Entry}{贫民窟}{8,5,13}{⾙、⽒、⽳}
  \begin{Phonetics}{贫民窟}{pin2min2ku1}
    \definition{s.}{favela}
  \end{Phonetics}
\end{Entry}

\begin{Entry}{贫困}{8,7}{⾙、⼞}
  \begin{Phonetics}{贫困}{pin2kun4}[][HSK 6]
    \definition{adj.}{pobre; indigente; necessitado; empobrecido; assolado pela pobreza; em circunstâncias difíceis}
  \end{Phonetics}
\end{Entry}

\begin{Entry}{贬}{8}{⾙}
  \begin{Phonetics}{贬}{bian3}
    \definition{adj.}{depreciativo; derrogativo; rebaixante}
    \definition{v.}{(nos tempos antigos) rebaixar de posição; (nos tempos modernos) diminuir de valor | reduzir valor; desvalorizar | censurar; menosprezar; depreciar; dar uma avaliação ruim | degradar; rebaixar; relegar}
  \end{Phonetics}
\end{Entry}

\begin{Entry}{贬值}{8,10}{⾙、⼈}
  \begin{Phonetics}{贬值}{bian3zhi2}[][HSK 7-9]
    \definition{v.}{depreciar; tornar-se desvalorizado; refere-se à diminuição do poder de compra do dinheiro | depreciar; geralmente se refere à diminuição do valor de algo | desvalorizar; reduzir o teor de ouro da moeda de um país ou reduzir a taxa de câmbio da moeda de um país em relação às moedas estrangeiras}
  \end{Phonetics}
\end{Entry}

\begin{Entry}{购}{8}{⾙}
  \begin{Phonetics}{购}{gou4}
    \definition{v.}{comprar}
  \end{Phonetics}
\end{Entry}

\begin{Entry}{购买}{8,6}{⾙、⼄}
  \begin{Phonetics}{购买}{gou4 mai3}[][HSK 4]
    \definition{v.}{comprar; adquirir; usar dinheiro para obter itens}
  \end{Phonetics}
\end{Entry}

\begin{Entry}{购物}{8,8}{⾙、⽜}
  \begin{Phonetics}{购物}{gou4wu4}[][HSK 4]
    \definition{s.}{compras; itens comprados; \emph{shopping}}
    \definition{v.}{ir às compras; fazer compras}
  \end{Phonetics}
\end{Entry}

\begin{Entry}{转}{8}{⾞}
  \begin{Phonetics}{转}{zhuai3}
  \end{Phonetics}
  \begin{Phonetics}{转}{zhuan3}
    \definition{v.}{mudar; deslocar; transferir; virar; mudar de direção, posição, situação, circunstâncias, etc. | transmitir; transferir; passar adiante}
  \end{Phonetics}
  \begin{Phonetics}{转}{zhuan4}[][HSK 3,6]
    \definition{clas.}{usado para rotações (por minuto, por segundo, etc.): RPM}
    \definition{v.}{girar; rodar; revolver; movimento em torno de um centro | passear; dar uma volta}
  \end{Phonetics}
\end{Entry}

\begin{Entry}{转化}{8,4}{⾞、⼔}
  \begin{Phonetics}{转化}{zhuan3 hua4}[][HSK 5]
    \definition{v.}{mudar; transformar | inverter; converter}
  \end{Phonetics}
\end{Entry}

\begin{Entry}{转让}{8,5}{⾞、⾔}
  \begin{Phonetics}{转让}{zhuan3rang4}[][HSK 5]
    \definition{v.}{ceder; fazer a entrega; transferir a posse de; ceder seus bens ou direitos a outra pessoa}
  \end{Phonetics}
\end{Entry}

\begin{Entry}{转产}{8,6}{⾞、⼇}
  \begin{Phonetics}{转产}{zhuan3chan3}
    \definition{v.}{mudar a produção | mudar para novos produtos}
  \end{Phonetics}
\end{Entry}

\begin{Entry}{转动}{8,6}{⾞、⼒}
  \begin{Phonetics}{转动}{zhuan3 dong4}[][HSK 4]
    \definition{v.}{girar; rodar; dar voltas; torcer | dar a volta em algo}
  \end{Phonetics}
  \begin{Phonetics}{转动}{zhuan4 dong4}[][HSK 6]
    \definition{s.}{tambor; roda}
    \definition{v.}{girar; correr; rolar; revolver; rotacionar; torcer}
  \end{Phonetics}
\end{Entry}

\begin{Entry}{转向}{8,6}{⾞、⼝}
  \begin{Phonetics}{转向}{zhuan3 xiang4}[][HSK 5]
    \definition{v.}{desviar; desviar-se; mudar a direção do avanço | mudar a posição política de alguém | mudar de direção; virar-se para (a outra parte)}
  \end{Phonetics}
  \begin{Phonetics}{转向}{zhuan4/xiang4}
    \definition{v.+compl.}{perder-se; perder o rumo; não consiguir distinguir a direção; estar perdido}
  \end{Phonetics}
\end{Entry}

\begin{Entry}{转告}{8,7}{⾞、⼝}
  \begin{Phonetics}{转告}{zhuan3gao4}[][HSK 4]
    \definition{v.}{passar adiante; comunicar; transmitir; ser instruído a dizer a outra parte o que uma pessoa diz, o que está acontecendo, etc.}
  \end{Phonetics}
\end{Entry}

\begin{Entry}{转身}{8,7}{⾞、⾝}
  \begin{Phonetics}{转身}{zhuan3 shen1}[][HSK 4]
    \definition{adv.}{em um instante; em um piscar de olhos}
    \definition{v.}{dar a volta; dar meia-volta; dar a volta por cima | virar; girar; refere-se a uma mudança de direção, localização, natureza, etc.}
  \end{Phonetics}
\end{Entry}

\begin{Entry}{转变}{8,8}{⾞、⼜}
  \begin{Phonetics}{转变}{zhuan3bian4}[][HSK 3]
    \definition{v.}{mudar; converter; transformar}
  \end{Phonetics}
\end{Entry}

\begin{Entry}{转念}{8,8}{⾞、⼼}
  \begin{Phonetics}{转念}{zhuan3nian4}
    \definition{v.}{ter dúvidas sobre algo | pensar melhor}
  \end{Phonetics}
\end{Entry}

\begin{Entry}{转账}{8,8}{⾞、⾙}
  \begin{Phonetics}{转账}{zhuan3/zhang4}
    \definition{v.+compl.}{transferir entre contas | trazer à frente | incluir uma soma de dinheiro do balanço anterior no seguinte}
  \end{Phonetics}
\end{Entry}

\begin{Entry}{转弯}{8,9}{⾞、⼸}
  \begin{Phonetics}{转弯}{zhuan4 wan1}[][HSK 4]
    \definition{s.}{esquina; curva}[小心急转弯。===Tenha cuidado em curvas fechadas.]
    \definition{v.}{rodar; desviar; virar uma esquina; fazer uma curva; fazer uma curva de 180º}
  \end{Phonetics}
\end{Entry}

\begin{Entry}{转换}{8,10}{⾞、⼿}
  \begin{Phonetics}{转换}{zhuan3 huan4}[][HSK 5]
    \definition{v.}{mudar; trocar; converter; transformar; alterar}
  \end{Phonetics}
\end{Entry}

\begin{Entry}{转递}{8,10}{⾞、⾡}
  \begin{Phonetics}{转递}{zhuan3di4}
    \definition{v.}{passar | retransmitir}
  \end{Phonetics}
\end{Entry}

\begin{Entry}{转悠}{8,11}{⾞、⼼}
  \begin{Phonetics}{转悠}{zhuan4you5}
    \definition{v.}{aparecer repetidamente | rolar | passear por aí}
  \end{Phonetics}
\end{Entry}

\begin{Entry}{转移}{8,11}{⾞、⽲}
  \begin{Phonetics}{转移}{zhuan3yi2}[][HSK 4]
    \definition{v.}{deslocar; desviar; transferir; redirecionar; reposicionar; reorientar | mudar; transformar}
  \end{Phonetics}
\end{Entry}

\begin{Entry}{转游}{8,12}{⾞、⽔}
  \begin{Phonetics}{转游}{zhuan4you5}
  \seealsoref{转悠}{zhuan4you5}
  \end{Phonetics}
\end{Entry}

\begin{Entry}{轮}{8}{⾞}
  \begin{Phonetics}{轮}{lun2}[][HSK 4]
    \definition{clas.}{usado para sol vermelho, lua brilhante, etc. | usado para rodadas | doze anos de idade (os doze ramos terrestres são usados para lembrar o gênero humano e cada doze anos de idade é um ciclo)}
    \definition{s.}{roda | anel; disco; objeto semelhante a uma roda | navio a vapor; barco a vapor}
    \definition{v.}{revezar; substituir um ao outro em sequência (para fazer algo)}
  \end{Phonetics}
\end{Entry}

\begin{Entry}{轮子}{8,3}{⾞、⼦}
  \begin{Phonetics}{轮子}{lun2 zi5}[][HSK 4]
    \definition[个,只]{s.}{roda; peças circulares de veículos ou máquinas com capacidade de rotação}
  \end{Phonetics}
\end{Entry}

\begin{Entry}{轮回}{8,6}{⾞、⼞}
  \begin{Phonetics}{轮回}{lun2hui2}
    \definition[个]{s.}{reencarnação (Budismo) | ciclo}
    \definition{v.}{reencarnar}
  \end{Phonetics}
\end{Entry}

\begin{Entry}{轮船}{8,11}{⾞、⾈}
  \begin{Phonetics}{轮船}{lun2chuan2}[][HSK 4]
    \definition[艘,班]{s.}{vapor; navio a vapor; barco a vapor}
  \end{Phonetics}
\end{Entry}

\begin{Entry}{轮椅}{8,12}{⾞、⽊}
  \begin{Phonetics}{轮椅}{lun2 yi3}[][HSK 4]
    \definition{s.}{cadeira de rodas; dispositivo de assento especialmente projetado com rodas para pessoas com dificuldade de locomoção, que pode ser acionado por um disco de roda ou manivela operados manualmente}
  \end{Phonetics}
\end{Entry}

\begin{Entry}{软}{8}{⾞}
  \begin{Phonetics}{软}{ruan3}[][HSK 5]
    \definition*{s.}{Sobrenome Ruan}
    \definition{adj.}{macio; flexível; maleável; maleável (oposto de 硬) | suave; brando; delicado | fraco; débil | de baixa qualidade, capacidade, etc. | facilmente movido (ou influenciado) | de maneira suave (ou gentil) | indulgente; tolerante | maleável; flexível | fácil de se emocionar ou abalar}
  \seealsoref{硬}{ying4}
  \end{Phonetics}
\end{Entry}

\begin{Entry}{软件}{8,6}{⾞、⼈}
  \begin{Phonetics}{软件}{ruan3jian4}[][HSK 5]
    \definition[款,个]{s.}{\emph{software}; programas de computador, procedimentos, regras e quaisquer arquivos, documentos e dados relacionados à operação do sistema de computador}
  \end{Phonetics}
\end{Entry}

\begin{Entry}{轰}{8}{⾞}
  \begin{Phonetics}{轰}{hong1}
    \definition{interj.}{(onomatopéia) Bum!; estrondo; refere-se aos ruídos altos feitos por trovões, fogo de artilharia, etc.}
    \definition{v.}{retumbar; bombardear; explodir | espantar; expulsar}
  \end{Phonetics}
\end{Entry}

\begin{Entry}{轰鸣}{8,8}{⾞、⿃}
  \begin{Phonetics}{轰鸣}{hong1ming2}
    \definition{s.}{bum (som de explosão) | estrondo}
  \end{Phonetics}
\end{Entry}

\begin{Entry}{轰炸机}{8,9,6}{⾞、⽕、⽊}
  \begin{Phonetics}{轰炸机}{hong1zha4ji1}
    \definition{s.}{bombardeiro (aeronave)}
  \end{Phonetics}
\end{Entry}

\begin{Entry}{迫}{8}{⾡}
  \begin{Phonetics}{迫}{po4}
    \definition{adj.}{urgente; premente}
    \definition{s.}{morteiro; artilharia}
    \definition{v.}{compelir; forçar; pressionar | aproximar-se; ir em direção a (ou perto de)}
  \end{Phonetics}
\end{Entry}

\begin{Entry}{迫切}{8,4}{⾡、⼑}
  \begin{Phonetics}{迫切}{po4qie4}[][HSK 4]
    \definition{adj.}{urgente; premente; muito ansiosamente, a ponto de ser difícil esperar}
  \end{Phonetics}
\end{Entry}

\begin{Entry}{郁}{8}{⾢}
  \begin{Phonetics}{郁}{yu4}
    \definition*{s.}{Sobrenome Yu}
    \definition{adj.}{fortemente perfumado | luxuriante; exuberante | sombrio; deprimido}
  \end{Phonetics}
\end{Entry}

\begin{Entry}{郁郁葱葱}{8,8,12,12}{⾢、⾢、⾋、⾋}
  \begin{Phonetics}{郁郁葱葱}{yu4yu4cong1cong1}
    \definition{adj.}{exuberante e verde}
    \definition{expr.}{verdejante e exuberante; uma profusão selvagem de vegetação; luxuriantemente verde; ela cresce mais verde e mais fresca}
  \end{Phonetics}
\end{Entry}

\begin{Entry}{郊}{8}{⾢}
  \begin{Phonetics}{郊}{jiao1}
    \definition*{s.}{Sobrenome Jiao}
    \definition{s.}{subúrbios; periferias; áreas ao redor da cidade}
  \end{Phonetics}
\end{Entry}

\begin{Entry}{郊区}{8,4}{⾢、⼖}
  \begin{Phonetics}{郊区}{jiao1 qu1}[][HSK 5]
    \definition[个,片,块]{s.}{subúrbios; arredores; periferia; área ao redor da cidade que está administrativamente sob a jurisdição da cidade}
  \end{Phonetics}
\end{Entry}

\begin{Entry}{采}{8}{⾤}
  \begin{Phonetics}{采}{cai3}[][HSK 7-9]
    \definition*{s.}{Sobrenome Cai}
    \definition{s.}{espírito; tez; cor e expressão facial | cores}
    \definition{v.}{escolher; arrancar; reunir; colher (flores, folhas, frutas) | minerar; extrair | reunir; coletar | adotar; pegar; selecionar}
  \end{Phonetics}
  \begin{Phonetics}{采}{cai4}
    \definition{s.}{atribuição a um nobre feudal; a terra (incluindo os escravos que cultivavam a terra) concedida pelos antigos príncipes aos nobres; também chamada de feudo}
  \end{Phonetics}
\end{Entry}

\begin{Entry}{采用}{8,5}{⾤、⽤}
  \begin{Phonetics}{采用}{cai3 yong4}[][HSK 3]
    \definition{v.}{selecionar e usar; adotar; considerar adequado e utilizar}
  \end{Phonetics}
\end{Entry}

\begin{Entry}{采访}{8,6}{⾤、⾔}
  \begin{Phonetics}{采访}{cai3fang3}[][HSK 4]
    \definition{s.}{cobertura; entrevista; coleta de notícias; entrevistas, pesquisas, gravações de áudio e vídeo, etc., com o objetivo de coletar os materiais necessários}
    \definition{v.}{cobrir; entrevistar; reunir novas informações}
  \end{Phonetics}
\end{Entry}

\begin{Entry}{采纳}{8,7}{⾤、⽷}
  \begin{Phonetics}{采纳}{cai3na4}[][HSK 6]
    \definition{v.}{aceitar; adotar; tomar (opiniões, sugestões, solicitações, etc.)}
  \end{Phonetics}
\end{Entry}

\begin{Entry}{采取}{8,8}{⾤、⼜}
  \begin{Phonetics}{采取}{cai3qu3}[][HSK 3]
    \definition{v.}{adotar; escolha da implementação (diretrizes, políticas, métodos, ações, etc.) | reunir; coletar; tomar; assumir}
  \end{Phonetics}
\end{Entry}

\begin{Entry}{采矿}{8,8}{⾤、⽯}
  \begin{Phonetics}{采矿}{cai3/kuang4}[][HSK 7-9]
    \definition{s.}{mina}
    \definition{v.+compl.}{minerar; extrair minerais}
  \end{Phonetics}
\end{Entry}

\begin{Entry}{采购}{8,8}{⾤、⾙}
  \begin{Phonetics}{采购}{cai3gou4}[][HSK 5]
    \definition[名]{s.}{comprador; responsável pelas compras}
    \definition{v.}{adquirir; comprar; fazer compras para uma organização; fazer compras para uma empresa}
  \end{Phonetics}
\end{Entry}

\begin{Entry}{采集}{8,12}{⾤、⾫}
  \begin{Phonetics}{采集}{cai3ji2}[][HSK 7-9]
    \definition{v.}{reunir; coletar}
  \end{Phonetics}
\end{Entry}

\begin{Entry}{金}{8}{⾦}[Kangxi 167]
  \begin{Phonetics}{金}{jin1}[][HSK 3]
    \definition*{s.}{Dinastia Jin (1115-1234) | Sobrenome Jin}
    \definition{adj.}{dourado | altamente respeitado; precioso. metáfora de nobreza}
    \definition[锭,块]{s.}{ouro | metal | dinheiro | instrumento antigo de percussão de metal}
  \end{Phonetics}
\end{Entry}

\begin{Entry}{金子}{8,3}{⾦、⼦}
  \begin{Phonetics}{金子}{jin1zi5}
    \definition{s.}{ouro; elemento metálico, símbolo Au (aurum) amarelo-avermelhado, macio, dúctil, quimicamente estável é um metal precioso, usado para fabricar dinheiro, ornamentos etc.}
  \end{Phonetics}
\end{Entry}

\begin{Entry}{金刚石}{8,6,5}{⾦、⼑、⽯}
  \begin{Phonetics}{金刚石}{jin1gang1shi2}
    \definition{s.}{diamante, também chamado de 钻石}[金刚石比什么金属都硬。===O diamante é mais duro que qualquer metal.]
  \seealsoref{钻石}{zuan4shi2}
  \end{Phonetics}
\end{Entry}

\begin{Entry}{金色}{8,6}{⾦、⾊}
  \begin{Phonetics}{金色}{jin1 se4}
    \definition{s.}{cor ouro; dourado}
  \end{Phonetics}
\end{Entry}

\begin{Entry}{金钱}{8,10}{⾦、⾦}
  \begin{Phonetics}{金钱}{jin1 qian2}[][HSK 6]
    \definition[沓,笔,堆]{s.}{dinheiro; moeda}
  \end{Phonetics}
\end{Entry}

\begin{Entry}{金牌}{8,12}{⾦、⽚}
  \begin{Phonetics}{金牌}{jin1 pai2}[][HSK 3]
    \definition[枚]{s.}{medalha de ouro; refere-se à medalha conquistada pelo campeão em uma competição esportiva | ficha de ouro; placa de ouro usada como símbolo}
  \end{Phonetics}
\end{Entry}

\begin{Entry}{金额}{8,15}{⾦、⾴}
  \begin{Phonetics}{金额}{jin1 e2}[][HSK 6]
    \definition[份,笔]{s.}{quantidade de dinheiro; soma de dinheiro}
  \end{Phonetics}
\end{Entry}

\begin{Entry}{金融}{8,16}{⾦、⿀}
  \begin{Phonetics}{金融}{jin1rong2}[][HSK 6]
    \definition{s.}{finanças; serviços bancários; refere-se a atividades econômicas como a emissão, circulação e retirada de moeda, a concessão e retirada de empréstimos, o depósito e retirada de depósitos e transações de câmbio}
  \end{Phonetics}
\end{Entry}

\begin{Entry}{钓}{8}{⾦}
  \begin{Phonetics}{钓}{diao4}
    \definition{v.}{pescar com anzol e linha | buscar (fama e ganho pessoal) | fisgar; defraudar por meios desleais}
  \end{Phonetics}
\end{Entry}

\begin{Entry}{钓鱼}{8,8}{⾦、⿂}
  \begin{Phonetics}{钓鱼}{diao4yu2}
    \definition{v.}{pescar (com linha e anzol) | (figurativo) aprisionar}
  \end{Phonetics}
\end{Entry}

\begin{Entry}{闸}{8}{⾨}
  \begin{Phonetics}{闸}{zha2}
    \definition[个,道]{s.}{comporta; comporta | freio | (coloquial) interruptor}
    \definition{v.}{represar um córrego, rio, etc. | represar a água; parar a água}
  \end{Phonetics}
\end{Entry}

\begin{Entry}{闸门}{8,3}{⾨、⾨}
  \begin{Phonetics}{闸门}{zha2men2}
    \definition{s.}{eclusa | comporta}
  \end{Phonetics}
\end{Entry}

\begin{Entry}{闹}{8}{⾾}
  \begin{Phonetics}{闹}{nao4}[][HSK 4]
    \definition{adj.}{barulhento}
    \definition{v.}{fazer barulho; provocar problemas | dar vazão (à sua raiva, ressentimento, etc.) | sofrer de; ser incomodado por; ocorrer (um desastre ou coisa ruim) | fazer;  entrar em ação | agitar; perturbar | brincar; fazer bagunça}
  \end{Phonetics}
\end{Entry}

\begin{Entry}{闹钟}{8,9}{⾾、⾦}
  \begin{Phonetics}{闹钟}{nao4 zhong1}[][HSK 4]
    \definition[个,台,只,款]{s.}{despertador; relógios capazes de tocar alarmes em horários predeterminados}
  \end{Phonetics}
\end{Entry}

\begin{Entry}{降}{8}{⾩}
  \begin{Phonetics}{降}{jiang4}[][HSK 4]
    \definition*{s.}{Sobrenome Jiang}
    \definition{v.}{cair; descer; quedar-se (oposto de 升 ) | diminuir; reduzir; cair; abaixar | nascer}
  \seealsoref{升}{sheng1}
  \end{Phonetics}
\end{Entry}

\begin{Entry}{降价}{8,6}{⾩、⼈}
  \begin{Phonetics}{降价}{jiang4 jia4}[][HSK 4]
    \definition{v.}{ficar mais barato; cortar o preço; reduzir o preço}
  \end{Phonetics}
\end{Entry}

\begin{Entry}{降低}{8,7}{⾩、⼈}
  \begin{Phonetics}{降低}{jiang4di1}[][HSK 4]
    \definition{v.}{reduzir; cortar; diminuir; rebaixar; cair; abaixar}
  \end{Phonetics}
\end{Entry}

\begin{Entry}{降温}{8,12}{⾩、⽔}
  \begin{Phonetics}{降温}{jiang4 wen1}[][HSK 4]
    \definition{v.}{baixar a temperatura (como em uma oficina);  recusar | cair a temperatura | esfriar; resfriar; metáfora para um declínio no entusiasmo ou uma diminuição no ímpeto de algo}
  \end{Phonetics}
\end{Entry}

\begin{Entry}{降落}{8,12}{⾩、⾋}
  \begin{Phonetics}{降落}{jiang4luo4}[][HSK 4]
    \definition{v.}{aterrissar; descer; descer do céu}
  \end{Phonetics}
\end{Entry}

\begin{Entry}{限}{8}{⾩}
  \begin{Phonetics}{限}{xian4}
    \definition{s.}{limite | limiar}
    \definition{v.}{definir um limite; limitar; restringir}
  \end{Phonetics}
\end{Entry}

\begin{Entry}{限制}{8,8}{⾩、⼑}
  \begin{Phonetics}{限制}{xian4zhi4}[][HSK 4]
    \definition[些]{s.}{limite; restrição; confinamento}
    \definition{v.}{limitar; adstringir; restringir; confinar; fechar em (sobre)}
  \end{Phonetics}
\end{Entry}

\begin{Entry}{隶}{8}{⾪}[Kangxi 171]
  \begin{Phonetics}{隶}{li4}
    \definition*{s.}{Sobrenome Li}
    \definition{s.}{escravo; pessoa em servidão; pessoas escravizadas | Arcaico: corredor de cargo governamental na China feudal | um dos estilos antigos da caligrafia chinesa}
    \definition{v.}{estar subordinado a; estar afiliado a (ou com)}
  \end{Phonetics}
\end{Entry}

\begin{Entry}{雨}{8}{⾬}[Kangxi 173]
  \begin{Phonetics}{雨}{yu3}[][HSK 1]
    \definition*{s.}{Sobrenome Yu}
    \definition[场,阵,滴]{s.}{chuva; água que cai das nuvens para o solo}
  \end{Phonetics}
  \begin{Phonetics}{雨}{yu4}
    \definition{v.}{cair (chuva, neve, etc.) | precipitar | chover | molhar}
  \end{Phonetics}
\end{Entry}

\begin{Entry}{雨水}{8,4}{⾬、⽔}
  \begin{Phonetics}{雨水}{yu3 shui3}[][HSK 5]
    \definition{s.}{água da chuva; precipitação; chuva; água proveniente da chuva}
  \end{Phonetics}
\end{Entry}

\begin{Entry}{雨伞}{8,6}{⾬、⼈}
  \begin{Phonetics}{雨伞}{yu3san3}
    \definition[把]{s.}{guarda-chuva}
  \end{Phonetics}
\end{Entry}

\begin{Entry}{雨衣}{8,6}{⾬、⾐}
  \begin{Phonetics}{雨衣}{yu3 yi1}[][HSK 6]
    \definition[件,个]{s.}{capa de chuva; jaqueta impermeável; roupas impermeáveis}
  \end{Phonetics}
\end{Entry}

\begin{Entry}{雨蚀}{8,9}{⾬、⾷}
  \begin{Phonetics}{雨蚀}{yu3shi2}
    \definition{s.}{erosão da chuva}
  \end{Phonetics}
\end{Entry}

\begin{Entry}{雨靴}{8,13}{⾬、⾰}
  \begin{Phonetics}{雨靴}{yu3xue1}
    \definition[双]{s.}{botas de chuva}
  \end{Phonetics}
\end{Entry}

\begin{Entry}{青}{8}{⾭}[Kangxi 174]
  \begin{Phonetics}{青}{qing1}[][HSK 5]
    \definition*{s.}{Província de Qinghai, abreviação de 青海 | Sobrenome Qing}
    \definition{adj.}{azul ou verde | preto | jovens (pessoas)}
    \definition{s.}{grama verde | colheitas jovens (não maduras) | tiras de bambu verde}
  \seealsoref{青海}{qing1hai3}
  \end{Phonetics}
\end{Entry}

\begin{Entry}{青天}{8,4}{⾭、⼤}
  \begin{Phonetics}{青天}{qing1tian1}
    \definition{s.}{céu claro, limpo ou azul}
  \end{Phonetics}
\end{Entry}

\begin{Entry}{青少年}{8,4,6}{⾭、⼩、⼲}
  \begin{Phonetics}{青少年}{qing1shao4nian2}[][HSK 2]
    \definition[位,名,个,些]{s.}{adolescentes}
  \end{Phonetics}
\end{Entry}

\begin{Entry}{青玉米}{8,5,6}{⾭、⽟、⽶}
  \begin{Phonetics}{青玉米}{qing1yu4mi3}
    \definition{s.}{milho verde}
  \end{Phonetics}
\end{Entry}

\begin{Entry}{青年}{8,6}{⾭、⼲}
  \begin{Phonetics}{青年}{qing1 nian2}[][HSK 2]
    \definition[个,位,名,些]{s.}{juventude; jovem; refere-se ao período entre os 15 e os 30 anos de idade.}
  \end{Phonetics}
\end{Entry}

\begin{Entry}{青年节}{8,6,5}{⾭、⼲、⾋}
  \begin{Phonetics}{青年节}{qing1nian2jie2}
    \definition*{s.}{Dia da Juventude (4 de maio)}
  \end{Phonetics}
\end{Entry}

\begin{Entry}{青春}{8,9}{⾭、⽇}
  \begin{Phonetics}{青春}{qing1chun1}[][HSK 4]
    \definition[个]{s.}{juventude; jovialidade}
  \end{Phonetics}
\end{Entry}

\begin{Entry}{青海}{8,10}{⾭、⽔}
  \begin{Phonetics}{青海}{qing1hai3}
    \definition*{s.}{Província de Qinghai}
  \end{Phonetics}
\end{Entry}

\begin{Entry}{青菜}{8,11}{⾭、⾋}
  \begin{Phonetics}{青菜}{qing1cai4}
    \definition{s.}{verduras}
  \end{Phonetics}
\end{Entry}

\begin{Entry}{青铜}{8,11}{⾭、⾦}
  \begin{Phonetics}{青铜}{qing1tong2}
    \definition{s.}{bronze (liga de cobre, 銅, e estanho, 锡)}
  \end{Phonetics}
\end{Entry}

\begin{Entry}{青椒}{8,12}{⾭、⽊}
  \begin{Phonetics}{青椒}{qing1jiao1}
    \definition{s.}{pimenta verde}
  \end{Phonetics}
\end{Entry}

\begin{Entry}{青蛙}{8,12}{⾭、⾍}
  \begin{Phonetics}{青蛙}{qing1wa1}
    \definition{adj.}{(gíria velha) cara feio}
    \definition[只]{s.}{sapo}
  \end{Phonetics}
\end{Entry}

\begin{Entry}{非}{8}{⾮}[Kangxi 175]
  \begin{Phonetics}{非}{fei1}[][HSK 4]
    \definition*{s.}{África, abreviação de 非洲 | Sobrenome Fei}
    \definition{adv.}{Em resposta a 不, indica necessidade (deve)}
    \definition{pref.}{indicando negatividade ou exclusão}
    \definition{s.}{engano; erro}
    \definition{v.}{opor-se a; culpar; censurar | não estar em conformidade com; ser contrário a | não ser | ter que; simplesmente precisar (fazer algo)}
  \seealsoref{不}{bu4}
  \seealsoref{非洲}{fei1zhou1}
  \end{Phonetics}
\end{Entry}

\begin{Entry}{非洲}{8,9}{⾮、⽔}
  \begin{Phonetics}{非洲}{fei1zhou1}
    \definition*{s.}{África}
  \end{Phonetics}
\end{Entry}

\begin{Entry}{非洲人}{8,9,2}{⾮、⽔、⼈}
  \begin{Phonetics}{非洲人}{fei1zhou1ren2}
    \definition{s.}{africano | pessoa ou povo da África}
  \end{Phonetics}
\end{Entry}

\begin{Entry}{非常}{8,11}{⾮、⼱}
  \begin{Phonetics}{非常}{fei1chang2}[][HSK 1]
    \definition{adj.}{extraordinário; incomum; especial}
    \definition{adv.}{muito; extremamente; altamente}
  \end{Phonetics}
\end{Entry}

\begin{Entry}{靣}{8}{⼀}[Kangxi 176]
  \begin{Phonetics}{靣}{mian4}
    \variantof{面}
  \end{Phonetics}
\end{Entry}

\begin{Entry}{顶}{8}{⾴}
  \begin{Phonetics}{顶}{ding3}[][HSK 4]
    \definition{adv.}{muito (linguagem falada); a maioria; extremamente; expressa o grau mais alto, equivalente a 最 e 极}
    \definition{clas.}{usado para coisas que têm um topo}
    \definition{prep.}{até}
    \definition{s.}{coroa da cabeça; parte mais alta do corpo ou objeto | topo; limite superior; ponto mais alto}
    \definition{v.}{carregar na cabeça; carregar em sua cabeça | empurrar (ou apoiar) para cima; empurrar por baixo (ou por trás) | dar cabeçadas; dar uma coronhada | sustentar; apoiar; suportar | resistir; ir contra; enfrentar | rebater; retorquir; responder de volta | cooperar; enfrentar; apoiar; dar suporte | igualar; ser equivalente a | substituir; tomar o lugar de | assumir o controle; transferir ou adquirir o direito de administrar um negócio ou alugar uma casa ou terreno}
  \seealsoref{极}{ji2}
  \seealsoref{最}{zui4}
  \end{Phonetics}
\end{Entry}

\begin{Entry}{饱}{8}{⾷}
  \begin{Phonetics}{饱}{bao3}[][HSK 2]
    \definition{adj.}{cheio; comer até ficar satisfeito | cheio; rechonchudo}
    \definition{adv.}{totalmente; completamente; plenamente}
    \definition{v.}{satisfazer}
  \end{Phonetics}
\end{Entry}

\begin{Entry}{饱和}{8,8}{⾷、⼝}
  \begin{Phonetics}{饱和}{bao3he2}[][HSK 7-9]
    \definition{v.}{estar saturado; a uma certa temperatura ou pressão, a quantidade de soluto contida na solução atinge seu limite máximo e não consegue mais se dissolver | estar saturado; metaforicamente, a quantidade de algo atinge um máximo dentro de um certo intervalo}
  \end{Phonetics}
\end{Entry}

\begin{Entry}{饱满}{8,13}{⾷、⽔}
  \begin{Phonetics}{饱满}{bao3man3}[][HSK 7-9]
    \definition{adj.}{cheio; rechonchudo; bem empilhado; preenchido | robusto; abundante; pleno; vigoroso}
  \end{Phonetics}
\end{Entry}

\begin{Entry}{驻}{8}{⾺}
  \begin{Phonetics}{驻}{zhu4}[][HSK 6]
    \definition{v.}{parar; ficar | estar estacionado; acampar; (tropas ou pessoal) viver no local onde desempenham suas funções; (organização) estar localizada em um determinado lugar}
  \end{Phonetics}
\end{Entry}

\begin{Entry}{驻军}{8,6}{⾺、⼍}
  \begin{Phonetics}{驻军}{zhu4jun1}
    \definition{s.}{guarnição}
    \definition{v.}{guarcener ou prover uma tropa}
  \end{Phonetics}
\end{Entry}

\begin{Entry}{驾}{8}{⾺}
  \begin{Phonetics}{驾}{jia4}
    \definition*{s.}{Sobrenome Jia}
    \definition{s.}{carruagem do imperador; refere-se especificamente ao carro do imperador, referindo-se ao imperador | referindo-se a um veículo, usado como um termo respeitoso para uma pessoa}
    \definition{v.}{atrelar; puxar (uma carroça, etc.) | dirigir (um veículo); pilotar (um avião); velejar (um barco) | montar; cavalgar}
  \end{Phonetics}
\end{Entry}

\begin{Entry}{驾驶}{8,8}{⾺、⾺}
  \begin{Phonetics}{驾驶}{jia4shi3}[][HSK 5]
    \definition{v.}{dirigir; pilotar; conduzir; guiar; operar (um carro, navio, avião, trator, etc.) para fazê-lo mover}
  \end{Phonetics}
\end{Entry}

\begin{Entry}{驾照}{8,13}{⾺、⽕}
  \begin{Phonetics}{驾照}{jia4 zhao4}[][HSK 5]
    \definition[本,张]{s.}{carteira de motorista}
  \end{Phonetics}
\end{Entry}

\begin{Entry}{鱼}{8}{⿂}[Kangxi 195]
  \begin{Phonetics}{鱼}{yu2}[][HSK 2]
    \definition*{s.}{Sobrenome Yu}
    \definition[条,种,尾]{s.}{peixe; um vertebrado que vive na água; geralmente possui um corpo achatado lateralmente, fusiforme e com muitas escamas; nada com as nadadeiras e respira com as brânquias; sua temperatura corporal varia de acordo com a temperatura externa; existem muitas espécies, a maioria das quais comestíveis | carne de peixe; peixe (como alimento)}
  \end{Phonetics}
\end{Entry}

\begin{Entry}{鱼片}{8,4}{⿂、⽚}
  \begin{Phonetics}{鱼片}{yu2pian4}
    \definition{s.}{fatia de peixe | filé de peixe}
  \end{Phonetics}
\end{Entry}

\begin{Entry}{鱼汛}{8,6}{⿂、⽔}
  \begin{Phonetics}{鱼汛}{yu2xun4}
    \variantof{渔汛}
  \end{Phonetics}
\end{Entry}

\begin{Entry}{鱼网}{8,6}{⿂、⽹}
  \begin{Phonetics}{鱼网}{yu2wang3}
    \variantof{渔网}
  \end{Phonetics}
\end{Entry}

\begin{Entry}{鱼具}{8,8}{⿂、⼋}
  \begin{Phonetics}{鱼具}{yu2ju4}
    \variantof{渔具}
  \end{Phonetics}
\end{Entry}

\begin{Entry}{鱼香}{8,9}{⿂、⾹}
  \begin{Phonetics}{鱼香}{yu2xiang1}
    \definition{s.}{um tempero da culinária chinesa que normalmente contém alho, cebolinha, gengibre, açúcar, sal, pimenta, etc. (Embora 鱼香 signifique literalmente ``fragrância de peixe'', não contém frutos do mar.)}
  \end{Phonetics}
\end{Entry}

\begin{Entry}{鱼香肉丝}{8,9,6,5}{⿂、⾹、⾁、⼀}
  \begin{Phonetics}{鱼香肉丝}{yu2xiang1rou4si1}
    \definition{s.}{tiras de carne de porco salteadas com molho picante (prato)}
  \seealsoref{鱼香}{yu2xiang1}
  \end{Phonetics}
\end{Entry}

\begin{Entry}{鱼船}{8,11}{⿂、⾈}
  \begin{Phonetics}{鱼船}{yu2chuan2}
    \definition{s.}{barco de pesca}
  \seealsoref{渔船}{yu2chuan2}
  \end{Phonetics}
\end{Entry}

\begin{Entry}{鸣}{8}{⿃}
  \begin{Phonetics}{鸣}{ming2}
    \definition{v.}{chorar (pássaros, animais e insetos) | fazer um som | dar voz (gratidão, queixas, etc.)}
  \end{Phonetics}
\end{Entry}

\begin{Entry}{齿}{8}{⿒}[Kangxi 211]
  \begin{Phonetics}{齿}{chi3}
    \definition[颗]{s.}{dente | uma parte de qualquer coisa semelhante a um dente; parte dentada de um objeto | idade (de uma pessoa); faixa etária}
    \definition{v.}{mencionar; falar de}
  \end{Phonetics}
\end{Entry}

\begin{Entry}{齿儿}{8,2}{⿒、⼉}
  \begin{Phonetics}{齿儿}{chi3r5}
    \definition{s.}{dentes}
  \end{Phonetics}
\end{Entry}

%%%%% EOF %%%%%

