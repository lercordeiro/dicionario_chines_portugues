%%%
%%% 8画
%%%

\section*{8画}\addcontentsline{toc}{section}{8画}

\begin{entry}{丧钟}{8,9}[Radicais ⼗、⾦]
  \begin{phonetics}{丧钟}{sang1zhong1}
    \definition{s.}{sentença de morte}
  \end{phonetics}
\end{entry}

\begin{entry}{乖乖}{8,8}[Radicais ⼃、⼃]
  \begin{phonetics}{乖乖}{guai1guai1}
    \definition{adj.}{bem-comportado (criança) | obediente}
  \end{phonetics}
  \begin{phonetics}{乖乖}{guai1guai5}
    \definition{expr.}{Graças a Deus! | Oh meu Deus!}
  \end{phonetics}
\end{entry}

\begin{entry}{乳房}{8,8}[Radicais ⼄、⼾]
  \begin{phonetics}{乳房}{ru3fang2}
    \definition{s.}{seio | mama | úbere}
  \end{phonetics}
\end{entry}

\begin{entry}{事}{8}[Radical ⼅]
  \begin{phonetics}{事}{shi4}[][HSK 1]
    \definition[件,桩,回]{s.}{coisa | assunto | item | matéria | coisa de trabalho | caso}
  \end{phonetics}
\end{entry}

\begin{entry}{事儿}{8,2}[Radicais ⼅、⼉]
  \begin{phonetics}{事儿}{shi4r5}
    \definition[件,桩]{s.}{o emprego | negócio | afazeres | assunto que precisa ser resolvido | matéria}
  \end{phonetics}
\end{entry}

\begin{entry}{事业}{8,5}[Radicais ⼅、⼀]
  \begin{phonetics}{事业}{shi4ye4}[][HSK 3]
    \definition[个]{s.}{causa; carreira; empreendimento | instituição; instalações; unidade de trabalho apoiada financeiramente pelo governo}
  \end{phonetics}
\end{entry}

\begin{entry}{事件}{8,6}[Radicais ⼅、⼈]
  \begin{phonetics}{事件}{shi4jian4}[][HSK 3]
    \definition[个,件,次]{s.}{evento; incidente}
  \end{phonetics}
\end{entry}

\begin{entry}{事实}{8,8}[Radicais ⼅、⼧]
  \begin{phonetics}{事实}{shi4shi2}[][HSK 3]
    \definition{s.}{mito; lenda}
    \definition{v.}{dizer; contar; ser dito}
  \end{phonetics}
\end{entry}

\begin{entry}{事实上}{8,8,3}[Radicais ⼅、⼧、⼀]
  \begin{phonetics}{事实上}{shi4 shi2 shang4}[][HSK 3]
    \definition{adv.}{realmente; de ​​fato; na verdade}
  \end{phonetics}
\end{entry}

\begin{entry}{事故}{8,9}[Radicais ⼅、⽁]
  \begin{phonetics}{事故}{shi4gu4}[][HSK 3]
    \definition[桩,起,次]{s.}{acidente}
  \end{phonetics}
\end{entry}

\begin{entry}{事情}{8,11}[Radicais ⼅、⼼]
  \begin{phonetics}{事情}{shi4qing5}[][HSK 2]
    \definition[件,桩]{s.}{assunto | coisa | erro | acidente | trabalho; emprego}
  \end{phonetics}
\end{entry}

\begin{entry}{些}{8}[Radical ⼆]
  \begin{phonetics}{些}{xie1}
    \definition{adv.}{uns | alguns | vários}
    \definition{clas.}{que indica uma pequena quantidade ou pequeno número maior que 1}
  \end{phonetics}
\end{entry}

\begin{entry}{些许}{8,6}[Radicais ⼆、⾔]
  \begin{phonetics}{些许}{xie1xu3}
    \definition{num.}{um pouco}
  \end{phonetics}
\end{entry}

\begin{entry}{享受}{8,8}[Radicais ⼇、⼜]
  \begin{phonetics}{享受}{xiang3shou4}
    \definition[种]{s.}{prazer}
    \definition{v.}{desfrutar | viver}
  \end{phonetics}
\end{entry}

\begin{entry}{京}{8}[Radical ⼇]
  \begin{phonetics}{京}{jing1}
    \definition*{s.}{Beijing, abreviação de~北京 | sobrenome Jing}
    \definition{s.}{capital de um país}
    \seeref{北京}{bei3 jing1}
  \end{phonetics}
\end{entry}

\begin{entry}{京剧}{8,10}[Radicais ⼇、⼑]
  \begin{phonetics}{京剧}{jing1ju4}[][HSK 3]
    \definition*[场,段]{s.}{Ópera de Beijing (Pequim)}
  \end{phonetics}
\end{entry}

\begin{entry}{佩服}{8,8}[Radicais ⼈、⽉]
  \begin{phonetics}{佩服}{pei4fu2}
    \definition{v.}{admirar}
  \end{phonetics}
\end{entry}

\begin{entry}{使}{8}[Radical ⼈]
  \begin{phonetics}{使}{shi3}[][HSK 3]
    \definition*{s.}{sobrenome Shi}
    \definition{conj.}{se; supondo}
    \definition{s.}{enviado; mensageiro}
    \definition{v.}{enviar; dizer a alguém para fazer algo | usar; empregar; aplicar | fazer; causar; habilitar}
  \end{phonetics}
\end{entry}

\begin{entry}{使用}{8,5}[Radicais ⼈、⽤]
  \begin{phonetics}{使用}{shi3yong4}[][HSK 2]
    \definition{v.}{usar | empregar | aplicar}
  \end{phonetics}
\end{entry}

\begin{entry}{例子}{8,3}[Radicais ⼈、⼦]
  \begin{phonetics}{例子}{li4 zi5}[][HSK 2]
    \definition[个]{s.}{exemplo}
  \end{phonetics}
\end{entry}

\begin{entry}{例如}{8,6}[Radicais ⼈、⼥]
  \begin{phonetics}{例如}{li4ru2}[][HSK 2]
    \definition{conj.}{por exemplo | como}
  \end{phonetics}
\end{entry}

\begin{entry}{依偎}{8,11}[Radicais ⼈、⼈]
  \begin{phonetics}{依偎}{yi1wei1}
    \definition{v.}{aninhar-se | aconchegar-se}
  \end{phonetics}
\end{entry}

\begin{entry}{依然}{8,12}[Radicais ⼈、⽕]
  \begin{phonetics}{依然}{yi1ran2}
    \definition{adv.}{como era antes | ainda}
  \end{phonetics}
\end{entry}

\begin{entry}{兔子}{8,3}[Radicais ⼉、⼦]
  \begin{phonetics}{兔子}{tu4zi5}
    \definition[只]{s.}{coelho | lebre}
  \end{phonetics}
\end{entry}

\begin{entry}{其中}{8,4}[Radicais ⼋、⼁]
  \begin{phonetics}{其中}{qi2zhong1}[][HSK 2]
    \definition{pron.}{dentro | entre (o qual, eles, etc.) | em (o qual, isso, etc.)}
  \end{phonetics}
\end{entry}

\begin{entry}{其他}{8,5}[Radicais ⼋、⼈]
  \begin{phonetics}{其他}{qi2ta1}[][HSK 2]
    \definition{pron.}{todos os outro(s) | o resto}
  \end{phonetics}
\end{entry}

\begin{entry}{其次}{8,6}[Radicais ⼋、⽋]
  \begin{phonetics}{其次}{qi2ci4}[][HSK 3]
    \definition{adj.}{secundário}
    \definition{conj.}{próximo; então; em segundo lugar}
  \end{phonetics}
\end{entry}

\begin{entry}{其实}{8,8}[Radicais ⼋、⼧]
  \begin{phonetics}{其实}{qi2shi2}[][HSK 3]
    \definition{adv.}{na verdade; na realidade; de fato}
  \end{phonetics}
\end{entry}

\begin{entry}{具有}{8,6}[Radicais ⼋、⽉]
  \begin{phonetics}{具有}{ju4 you3}[][HSK 3]
    \definition{v.}{ter; possuir; ser provido de}
  \end{phonetics}
\end{entry}

\begin{entry}{具体}{8,7}[Radicais ⼋、⼈]
  \begin{phonetics}{具体}{ju4ti3}[][HSK 3]
    \definition{adj.}{específico; particular | concreto; específico | concreto; real}
    \definition{v.}{incorporar; objetivar}
  \end{phonetics}
\end{entry}

\begin{entry}{典型}{8,9}[Radicais ⼋、⼟]
  \begin{phonetics}{典型}{dian3xing2}[][HSK 4]
    \definition{adj.}{típico; representativo}
    \definition[个]{s.}{modelo; caso típico; indivíduo ou evento representativo | personagens típicos; personalidades modelo (em obras literárias); personagens na literatura e na arte que refletem a natureza de uma determinada sociedade e têm uma personalidade distinta}
  \end{phonetics}
\end{entry}

\begin{entry}{函数}{8,13}[Radicais ⼐、⽁]
  \begin{phonetics}{函数}{han2shu4}
    \definition{s.}{função (matemática)}
  \end{phonetics}
\end{entry}

\begin{entry}{刮}{8}[Radical ⼑]
  \begin{phonetics}{刮}{gua1}
    \definition{v.}{ventar | soprar (vento)}
  \end{phonetics}
\end{entry}

\begin{entry}{刮风}{8,4}[Radicais ⼑、⾵]
  \begin{phonetics}{刮风}{gua1feng1}
    \definition{v.+compl.}{ventar | fazer vento}
  \end{phonetics}
\end{entry}

\begin{entry}{到}{8}[Radical ⼑]
  \begin{phonetics}{到}{dao4}[][HSK 1]
    \definition{prep.}{a | até | para}
    \definition{v.}{chegar}
  \end{phonetics}
\end{entry}

\begin{entry}{到处}{8,5}[Radicais ⼑、⼡]
  \begin{phonetics}{到处}{dao4chu4}[][HSK 2]
    \definition{adv.}{em todos os lugares}
  \end{phonetics}
\end{entry}

\begin{entry}{到达}{8,6}[Radicais ⼑、⾡]
  \begin{phonetics}{到达}{dao4da2}[][HSK 3]
    \definition{v.}{chegar; alcançar}
  \end{phonetics}
\end{entry}

\begin{entry}{到底}{8,8}[Radicais ⼑、⼴]
  \begin{phonetics}{到底}{dao4di3}[][HSK 3]
    \definition{adv.}{na terra (usado em frases interrogativas para expressar a determinação de alguém em encontrar uma resposta definitiva) | afinal | finalmente; por fim; no fim}
  \end{phonetics}
\end{entry}

\begin{entry}{制作}{8,7}[Radicais ⼑、⼈]
  \begin{phonetics}{制作}{zhi4zuo4}[][HSK 3]
    \definition{v.}{fazer; fabricar; fazer a partir de matérias-primas, usado principalmente para itens menores e artesanais; criar designs com palavras, imagens, sons, etc. para criar gráficos, anúncios, filmes, jogos, etc.}
  \end{phonetics}
\end{entry}

\begin{entry}{制定}{8,8}[Radicais ⼑、⼧]
  \begin{phonetics}{制定}{zhi4ding4}[][HSK 3]
    \definition{v.}{rascunhar; formular; elaborar; estabelecer}
  \end{phonetics}
\end{entry}

\begin{entry}{制度}{8,9}[Radicais ⼑、⼴]
  \begin{phonetics}{制度}{zhi4du4}[][HSK 3]
    \definition[个]{s.}{regulamento; regulação; regulamentação | sistema; um sistema político, econômico, cultural e outro formado sob certas condições históricas}
  \end{phonetics}
\end{entry}

\begin{entry}{制造}{8,10}[Radicais ⼑、⾡]
  \begin{phonetics}{制造}{zhi4zao4}[][HSK 3]
    \definition{v.}{fazer; produzir; manufaturar; transformar matérias-primas em produtos acabados | criar; agitar; criar artificialmente uma situação ou atmosfera ruim}
  \end{phonetics}
\end{entry}

\begin{entry}{制裁}{8,12}[Radicais ⼑、⾐]
  \begin{phonetics}{制裁}{zhi4cai2}
    \definition{s.}{punição | sanção (inclusive econômica)}
    \definition{v.}{punir}
  \end{phonetics}
\end{entry}

\begin{entry}{刷子}{8,3}[Radicais ⼑、⼦]
  \begin{phonetics}{刷子}{shua1zi5}
    \definition[把]{s.}{pincel | escova | escovão}
  \end{phonetics}
\end{entry}

\begin{entry}{刹}{8}[Radical ⼑]
  \begin{phonetics}{刹}{cha4}
    \definition{s.}{mosteiro, templo ou santuário budista | abreviação de 刹多罗 | sânscrito "ksetra"}
    \seeref{刹多罗}{cha4duo1luo2}
  \end{phonetics}
  \begin{phonetics}{刹}{sha1}
    \definition{v.}{frear}
  \end{phonetics}
\end{entry}

\begin{entry}{刹多罗}{8,6,8}[Radicais ⼑、⼣、⽹]
  \begin{phonetics}{刹多罗}{cha4duo1luo2}
    \definition*{s.}{Kshatara, sânscrito ``ksetra''}
  \end{phonetics}
\end{entry}

\begin{entry}{刺}{8}[Radical ⼑]
  \begin{phonetics}{刺}{ci1}
    \definition{s.}{(onomatopéia) som de rasgo, fricção, etc.}
  \end{phonetics}
  \begin{phonetics}{刺}{ci4}[][HSK 4]
    \definition*{s.}{sobrenome Ci}
    \definition{s.}{espinho; farpa; algo afiado como uma agulha | cartão de visita | saliências; projeções pequenas e pontiagudas na superfície de um objeto ou na pele de uma pessoa}
    \definition{v.}{esfaquear; perfurar | irritar; estimular | assassinar | espionar; detectar | criticar}
  \end{phonetics}
\end{entry}

\begin{entry}{刺猬}{8,12}[Radicais ⼑、⽝]
  \begin{phonetics}{刺猬}{ci4wei5}
    \definition{s.}{porco-espinho | ouriço}
  \end{phonetics}
\end{entry}

\begin{entry}{刺激}{8,16}[Radicais ⼑、⽔]
  \begin{phonetics}{刺激}{ci4ji1}[][HSK 4]
    \definition{adj.}{animado; entusiasmado; sensação de empolgação e nervosismo}
    \definition[个]{s.}{estímulo; estimulação; fortes efeitos físicos ou psicológicos}
    \definition{v.}{irritar; provocar; estimular | incentivar; estimular; incitar; (por algum meio) para mudar as coisas para melhor, para fazer coisas positivas}
  \end{phonetics}
\end{entry}

\begin{entry}{刻}{8}[Radical ⼑]
  \begin{phonetics}{刻}{ke4}[][HSK 2]
    \definition{clas.}{para curtos intervalos de tempo}
    \definition{s.}{quarto (de hora)}
    \definition{v.}{esculpir | cortar | gravar}
  \end{phonetics}
\end{entry}

\begin{entry}{刻画}{8,8}[Radicais ⼑、⽥]
  \begin{phonetics}{刻画}{ke4hua4}
    \definition{v.}{retratar | tirar um retrato}
  \end{phonetics}
\end{entry}

\begin{entry}{刻钟}{8,9}[Radicais ⼑、⾦]
  \begin{phonetics}{刻钟}{ke4 zhong1}
    \definition{s.}{um quarto de hora}
  \end{phonetics}
\end{entry}

\begin{entry}{单}{8}[Radical ⼗]
  \begin{phonetics}{单}{chan2}
    \definition{s.}{usado em 单于 \dpy{chan2yu2}}
    \seeref{单于}{chan2yu2}
  \end{phonetics}
  \begin{phonetics}{单}{dan1}[][HSK 4]
    \definition{adj.}{sozinho; único | ímpar | sem forro (vestuário) | simples; poucos itens ou categorias; não é complexo | fino; fraco; frágil}
    \definition{adv.}{isoladamente; sozinho | somente; sozinho; unicamente | somente; apenas}
    \definition[个]{s.}{lençol; um pano grande para cobrir a cama | conta; lista; pedaços de papel que detalham coisas}
  \end{phonetics}
  \begin{phonetics}{单}{shan4}
    \definition*{s.}{sobrenome Shan}
  \end{phonetics}
\end{entry}

\begin{entry}{单于}{8,3}[Radicais ⼗、⼆]
  \begin{phonetics}{单于}{chan2yu2}
    \definition{s.}{rei de Xiongnu (匈奴)}
  \seealsoref{匈奴}{xiong1nu2}
  \end{phonetics}
\end{entry}

\begin{entry}{单元}{8,4}[Radicais ⼗、⼉]
  \begin{phonetics}{单元}{dan1yuan2}[][HSK 3]
    \definition[个,组,套]{s.}{unidade (de algo)}
  \end{phonetics}
\end{entry}

\begin{entry}{单位}{8,7}[Radicais ⼗、⼈]
  \begin{phonetics}{单位}{dan1wei4}[][HSK 2]
    \definition[个]{s.}{unidade (como padrão de medida) | unidade (como uma organização, departamento, divisão, seção, etc.)}
  \end{phonetics}
\end{entry}

\begin{entry}{单纯}{8,7}[Radicais ⼗、⽷]
  \begin{phonetics}{单纯}{dan1chun2}[][HSK 4]
    \definition{adj.}{puro; simples; descomplicado}
    \definition{adv.}{sozinho; puramente; meramente}
  \end{phonetics}
\end{entry}

\begin{entry}{单质}{8,8}[Radicais ⼗、⾙]
  \begin{phonetics}{单质}{dan1zhi4}
    \definition{s.}{substância simples (consistindo puramente de um elemento, como diamante, grafite, etc.)}
  \end{phonetics}
\end{entry}

\begin{entry}{单独}{8,9}[Radicais ⼗、⽝]
  \begin{phonetics}{单独}{dan1du2}[][HSK 4]
    \definition{adv.}{solo; sozinho; por si mesmo; por conta própria}
  \end{phonetics}
\end{entry}

\begin{entry}{单调}{8,10}[Radicais ⼗、⾔]
  \begin{phonetics}{单调}{dan1diao4}[][HSK 4]
    \definition{adj.}{maçante; monótono}
  \end{phonetics}
\end{entry}

\begin{entry}{单脚滑行车}{8,11,12,6,4}[Radicais ⼗、⾁、⽔、⾏、⾞]
  \begin{phonetics}{单脚滑行车}{dan1jiao3hua2xing2che1}
    \definition{s.}{\emph{scooter}}
  \end{phonetics}
\end{entry}

\begin{entry}{卖}{8}[Radical ⼗]
  \begin{phonetics}{卖}{mai4}[][HSK 2]
    \definition{v.}{vender}
  \end{phonetics}
\end{entry}

\begin{entry}{卧}{8}[Radical ⾂]
  \begin{phonetics}{卧}{wo4}
    \definition{v.}{agachar | deitar}
  \end{phonetics}
\end{entry}

\begin{entry}{卧车}{8,4}[Radicais ⾂、⾞]
  \begin{phonetics}{卧车}{wo4che1}
    \definition{s.}{um carro-leito | vagão-leito}
  \end{phonetics}
\end{entry}

\begin{entry}{卧式}{8,6}[Radicais ⾂、⼷]
  \begin{phonetics}{卧式}{wo4shi4}
    \definition{adj.}{horizontal}
  \end{phonetics}
\end{entry}

\begin{entry}{卧床}{8,7}[Radicais ⾂、⼴]
  \begin{phonetics}{卧床}{wo4chuang2}
    \definition{adj.}{acamado}
    \definition{s.}{cama}
    \definition{v.}{deitar na cama}
  \end{phonetics}
\end{entry}

\begin{entry}{卧室}{8,9}[Radicais ⾂、⼧]
  \begin{phonetics}{卧室}{wo4shi4}
    \definition[间]{s.}{quarto de dormir}
  \end{phonetics}
\end{entry}

\begin{entry}{卧倒}{8,10}[Radicais ⾂、⼈]
  \begin{phonetics}{卧倒}{wo4dao3}
    \definition{v.}{cair no chão | deitar-se}
  \end{phonetics}
\end{entry}

\begin{entry}{卧病}{8,10}[Radicais ⾂、⽧]
  \begin{phonetics}{卧病}{wo4bing4}
    \definition{s.}{acamado | doente na cama}
  \end{phonetics}
\end{entry}

\begin{entry}{卧舱}{8,10}[Radicais ⾂、⾈]
  \begin{phonetics}{卧舱}{wo4cang1}
    \definition{s.}{cabine de dormir em um barco ou trem}
  \end{phonetics}
\end{entry}

\begin{entry}{卧推}{8,11}[Radicais ⾂、⼿]
  \begin{phonetics}{卧推}{wo4tui1}
    \definition{s.}{supino}
  \end{phonetics}
\end{entry}

\begin{entry}{卧榻}{8,14}[Radicais ⾂、⽊]
  \begin{phonetics}{卧榻}{wo4ta4}
    \definition{s.}{um sofá | uma cama estreita}
  \end{phonetics}
\end{entry}

\begin{entry}{卷}{8}[Radical ⼙]
  \begin{phonetics}{卷}{juan3}
    \definition{clas.}{para pequenas coisas enroladas (maço de papel dinheiro, carretel de filme, etc.) | para rolos, carretéis, bobinas, etc.}
    \definition{s.}{rolo | carretel | bobina}
    \definition{v.}{rolar | varrer | carregar}
  \end{phonetics}
  \begin{phonetics}{卷}{juan4}
    \definition{clas.}{para livros, pinturas: volumes, rolos}
    \definition{s.}{rolo com texto | livro | volume | capítulo | artigo}
  \end{phonetics}
\end{entry}

\begin{entry}{厕纸}{8,7}[Radicais ⼚、⽷]
  \begin{phonetics}{厕纸}{ce4zhi3}
    \definition{s.}{papel higiênico}
  \end{phonetics}
\end{entry}

\begin{entry}{厕所}{8,8}[Radicais ⼚、⼾]
  \begin{phonetics}{厕所}{ce4suo3}
    \definition[间,处]{s.}{lavatório | \emph{toilette}}
  \end{phonetics}
\end{entry}

\begin{entry}{参与}{8,3}[Radicais ⼛、⼀]
  \begin{phonetics}{参与}{can1yu4}[][HSK 4]
    \definition{v.}{participar de; tomar parte em; ter uma mão em; envolver-se em; participar (no planejamento, discussão e condução dos assuntos)}
  \end{phonetics}
\end{entry}

\begin{entry}{参加}{8,5}[Radicais ⼛、⼒]
  \begin{phonetics}{参加}{can1jia1}[][HSK 2]
    \definition{v.}{participar de | tomar parte em | assistir}
  \end{phonetics}
\end{entry}

\begin{entry}{参考}{8,6}[Radicais ⼛、⽼]
  \begin{phonetics}{参考}{can1kao3}[][HSK 4]
    \definition{v.}{consultar; referir-se a; acessar informações relevantes para estudo ou pesquisa | consultar; referir-se a; lidar com coisas, observar, ler, aprender e usar materiais relevantes}
  \end{phonetics}
\end{entry}

\begin{entry}{参观}{8,6}[Radicais ⼛、⾒]
  \begin{phonetics}{参观}{can1guan1}[][HSK 2]
    \definition{v.}{visitar}
  \end{phonetics}
\end{entry}

\begin{entry}{取}{8}[Radical ⼜]
  \begin{phonetics}{取}{qu3}[][HSK 2]
    \definition{v.}{buscar | obter | escolher}
  \end{phonetics}
\end{entry}

\begin{entry}{取水}{8,4}[Radicais ⼜、⽔]
  \begin{phonetics}{取水}{qu3shui3}
    \definition{v.}{obter água (de um poço, etc.)}
  \end{phonetics}
\end{entry}

\begin{entry}{取现}{8,8}[Radicais ⼜、⾒]
  \begin{phonetics}{取现}{qu3xian4}
    \definition{v.}{sacar dinheiro}
  \end{phonetics}
\end{entry}

\begin{entry}{取胜}{8,9}[Radicais ⼜、⾁]
  \begin{phonetics}{取胜}{qu3sheng4}
    \definition{v.}{prevalecer sobre os oponentes | marcar uma vitória}
  \end{phonetics}
\end{entry}

\begin{entry}{取悦}{8,10}[Radicais ⼜、⼼]
  \begin{phonetics}{取悦}{qu3yue4}
    \definition{v.}{tentar agradar}
  \end{phonetics}
\end{entry}

\begin{entry}{取消}{8,10}[Radicais ⼜、⽔]
  \begin{phonetics}{取消}{qu3xiao1}[][HSK 3]
    \definition{v.}{cancelar; suspender; anular; abolir; revogar; rescindir}
  \end{phonetics}
\end{entry}

\begin{entry}{取得}{8,11}[Radicais ⼜、⼻]
  \begin{phonetics}{取得}{qu3 de2}[][HSK 2]
    \definition{v.}{ganhar | adquirir | obter}
  \end{phonetics}
\end{entry}

\begin{entry}{受}{8}[Radical ⼜]
  \begin{phonetics}{受}{shou4}[][HSK 3]
    \definition{v.}{receber; aceitar | sofrer; ser submetido a | aguentar; suportar; tolerar | ser agradável}
  \end{phonetics}
\end{entry}

\begin{entry}{受伤}{8,6}[Radicais ⼜、⼈]
  \begin{phonetics}{受伤}{shou4shang1}[][HSK 3]
    \definition{v.}{ser ferido; sofrer uma lesão}
  \end{phonetics}
\end{entry}

\begin{entry}{受到}{8,8}[Radicais ⼜、⼑]
  \begin{phonetics}{受到}{shou4dao4}[][HSK 2]
    \definition{v.}{receber (elogio, educação, punição, etc.) | ser elogiado, educado, punido, etc.}
  \end{phonetics}
\end{entry}

\begin{entry}{受限}{8,8}[Radicais ⼜、⾩]
  \begin{phonetics}{受限}{shou4xian4}
    \definition{v.}{ser limitado | ser restrito | ser constrangido}
  \end{phonetics}
\end{entry}

\begin{entry}{受得了}{8,11,2}[Radicais ⼜、⼻、⼅]
  \begin{phonetics}{受得了}{shou4de5liao3}
    \definition{v.}{suportar | aguentar}
  \end{phonetics}
\end{entry}

\begin{entry}{变}{8}[Radical ⼜]
  \begin{phonetics}{变}{bian4}[][HSK 2]
    \definition{v.}{mudar | transformar | variar}
  \end{phonetics}
\end{entry}

\begin{entry}{变为}{8,4}[Radicais ⼜、⼂]
  \begin{phonetics}{变为}{bian4 wei2}[][HSK 3]
    \definition{v.}{transformar-se em | tornar-se | mudar para}
  \end{phonetics}
\end{entry}

\begin{entry}{变化}{8,4}[Radicais ⼜、⼔]
  \begin{phonetics}{变化}{bian4hua4}[][HSK 3]
    \definition[个]{s.}{mudança | variação}
    \definition{v.}{(intransitivo) mudar, variar}
  \end{phonetics}
\end{entry}

\begin{entry}{变心}{8,4}[Radicais ⼜、⼼]
  \begin{phonetics}{变心}{bian4xin1}
    \definition{v.+compl.}{deixar de ser fiel}
  \end{phonetics}
\end{entry}

\begin{entry}{变节}{8,5}[Radicais ⼜、⾋]
  \begin{phonetics}{变节}{bian4jie2}
    \definition{s.}{traição | deserção | vira-casaca}
    \definition{v.}{mudar de lado politicamente}
  \end{phonetics}
\end{entry}

\begin{entry}{变异}{8,6}[Radicais ⼜、⼶]
  \begin{phonetics}{变异}{bian4yi4}
    \definition{s.}{variação | mutação}
  \end{phonetics}
\end{entry}

\begin{entry}{变成}{8,6}[Radicais ⼜、⼽]
  \begin{phonetics}{变成}{bian4 cheng2}[][HSK 2]
    \definition{v.}{mudar | transformar-se em | tornar-se}
  \end{phonetics}
\end{entry}

\begin{entry}{变迁}{8,6}[Radicais ⼜、⾡]
  \begin{phonetics}{变迁}{bian4qian1}
    \definition{s.}{mudanças | vicissitudes}
  \end{phonetics}
\end{entry}

\begin{entry}{变更}{8,7}[Radicais ⼜、⽈]
  \begin{phonetics}{变更}{bian4geng1}
    \definition{v.}{alterar | mudar | modificar}
  \end{phonetics}
\end{entry}

\begin{entry}{变性}{8,8}[Radicais ⼜、⼼]
  \begin{phonetics}{变性}{bian4xing4}
    \definition{s.}{desnaturação | transexual}
    \definition{v.}{desnaturar | mudar de sexo}
  \end{phonetics}
\end{entry}

\begin{entry}{变装}{8,12}[Radicais ⼜、⾐]
  \begin{phonetics}{变装}{bian4zhuang1}
    \definition{v.}{trocar de roupa | vestir-se | vestir uma fantasia | disfarçar-se ou fantasiar-se de personagem real ou ficcional, \emph{cosplay} | travestir-se}
  \end{phonetics}
\end{entry}

\begin{entry}{变数}{8,13}[Radicais ⼜、⽁]
  \begin{phonetics}{变数}{bian4shu4}
    \definition{s.}{(matemática) variável}
  \end{phonetics}
\end{entry}

\begin{entry}{呢}{8}[Radical ⼝]
  \begin{phonetics}{呢}{ne5}[][HSK 1]
    \definition{part.}{(no final de uma frase declarativa) partícula que indica a continuação de um estado ou ação |  partícula para perguntar sobre a localização (``Onde está\dots?'') | partícula indicando  afirmação forte | partícula indicando que uma pergunta feita anteriormente deve ser aplicada à palavra anterior (``E quanto a\dots?'', ``E\dots?'') | partícula sinalizando uma pausa, para enfatizar as palavras anteriores e permitir que o ouvinte tenha tempo para compreendê-las (``ok?'', ``você está comigo ?'')}
  \end{phonetics}
  \begin{phonetics}{呢}{ni2}
    \definition{s.}{material de lã}
  \end{phonetics}
\end{entry}

\begin{entry}{周}{8}[Radical ⼝]
  \begin{phonetics}{周}{zhou1}[][HSK 2]
    \definition*{s.}{sobrenome Zhou | Dinastia Zhou (1046-256 BC)}
    \definition{adv.}{semanalmente}
    \definition{s.}{círculo | circunferência | ciclo | uma volta (em um circuito) | semana}
    \definition{v.}{fazer um circuito |circular | ajudar financeiramente}
  \end{phonetics}
\end{entry}

\begin{entry}{周末}{8,5}[Radicais ⼝、⽊]
  \begin{phonetics}{周末}{zhou1mo4}[][HSK 2]
    \definition{s.}{final-de-semana}
  \end{phonetics}
\end{entry}

\begin{entry}{周年}{8,6}[Radicais ⼝、⼲]
  \begin{phonetics}{周年}{zhou1nian2}[][HSK 2]
    \definition{s.}{aniversário}
  \end{phonetics}
\end{entry}

\begin{entry}{周围}{8,7}[Radicais ⼝、⼞]
  \begin{phonetics}{周围}{zhou1wei2}[][HSK 3]
    \definition{s.}{ao redor; em torno; vizinhança; a parte que circunda o centro}
  \end{phonetics}
\end{entry}

\begin{entry}{味}{8}[Radical ⼝]
  \begin{phonetics}{味}{wei4}
    \definition{clas.}{para medicamentos}
    \definition{s.}{cheiro | gosto}
  \end{phonetics}
\end{entry}

\begin{entry}{味儿}{8,2}[Radicais ⼝、⼉]
  \begin{phonetics}{味儿}{wei4r5}
    \definition{s.}{sabor}
  \end{phonetics}
\end{entry}

\begin{entry}{味道}{8,12}[Radicais ⼝、⾡]
  \begin{phonetics}{味道}{wei4dao5}[][HSK 2]
    \definition{s.}{sabor | (dialeto) odor, cheiro | (figurativo) sentimento (de…), dica (de…) | (figurativo) interesse, prazer}
  \end{phonetics}
\end{entry}

\begin{entry}{呵}{8}[Radical ⼝]
  \begin{phonetics}{呵}{a1}
    \variantof{啊}
  \end{phonetics}
  \begin{phonetics}{呵}{he1}
    \definition{expr.}{Meu Deus! | expelir a respiração}
  \end{phonetics}
\end{entry}

\begin{entry}{呼吸}{8,6}[Radicais ⼝、⼝]
  \begin{phonetics}{呼吸}{hu1xi1}
    \definition{v.}{respirar}
  \end{phonetics}
\end{entry}

\begin{entry}{呼啸}{8,11}[Radicais ⼝、⼝]
  \begin{phonetics}{呼啸}{hu1xiao4}
    \definition{v.}{assobiar}
  \end{phonetics}
\end{entry}

\begin{entry}{命运}{8,7}[Radicais ⼝、⾡]
  \begin{phonetics}{命运}{ming4yun4}[][HSK 3]
    \definition[个]{s.}{tendência de desenvolvimento; tendência de futuro | destino; sina; sorte}
  \end{phonetics}
\end{entry}

\begin{entry}{和}{8}[Radical ⼝]
  \begin{phonetics}{和}{he2}[][HSK 1]
    \definition*{s.}{sobrenome He}
    \definition{conj.}{e (somente para palavras) | junto com}
    \definition{s.}{união | paz | japonês (comida, roupa, etc.) | harmonia}
  \end{phonetics}
  \begin{phonetics}{和}{he4}
    \definition{v.}{compor um poema em resposta (ao poema de alguém) usando a mesma sequência de rimas | juntar-se à cantoria | cantar junto com outros}
  \end{phonetics}
  \begin{phonetics}{和}{hu2}
    \definition{v.}{completar um conjunto de Mahjong ou cartas de baralho}
  \end{phonetics}
  \begin{phonetics}{和}{huo2}
    \definition{v.}{combinar uma substância em pó (farinha, gesso, etc.) com água}
  \end{phonetics}
  \begin{phonetics}{和}{huo4}
    \definition{clas.}{para enxágues de roupas | para fervuras de ervas medicinais}
    \definition{v.}{misturar (ingredientes) | misturar}
  \end{phonetics}
\end{entry}

\begin{entry}{和平}{8,5}[Radicais ⼝、⼲]
  \begin{phonetics}{和平}{he2ping2}[][HSK 3]
    \definition{adj.}{pacífico; não violento}
    \definition{s.}{paz}
  \end{phonetics}
\end{entry}

\begin{entry}{和平共处}{8,5,6,5}[Radicais ⼝、⼲、⼋、⼡]
  \begin{phonetics}{和平共处}{he2ping2gong4chu3}
    \definition{s.}{coexistência pacífica de nações, sociedades, etc.}
  \end{phonetics}
\end{entry}

\begin{entry}{和谐}{8,11}[Radicais ⼝、⾔]
  \begin{phonetics}{和谐}{he2xie2}
    \definition{adj.}{harmonioso}
    \definition{s.}{harmonia}
    \definition{v.}{(eufemismo) censurar}
  \end{phonetics}
\end{entry}

\begin{entry}{咒骂}{8,9}[Radicais ⼝、⾺]
  \begin{phonetics}{咒骂}{zhou4ma4}
    \definition{v.}{xingar | amaldiçoar | execrar}
  \end{phonetics}
\end{entry}

\begin{entry}{咖啡}{8,11}[Radicais ⼝、⼝]
  \begin{phonetics}{咖啡}{ka1fei1}[][HSK 3]
    \definition[杯]{s.}{(empréstimo linguístico) café}
  \end{phonetics}
\end{entry}

\begin{entry}{咖啡色}{8,11,6}[Radicais ⼝、⼝、⾊]
  \begin{phonetics}{咖啡色}{ka1fei1 se4}
    \definition{s.}{cor café}
  \end{phonetics}
\end{entry}

\begin{entry}{咖啡馆}{8,11,11}[Radicais ⼝、⼝、⾷]
  \begin{phonetics}{咖啡馆}{ka1fei1guan3}
    \definition[家]{s.}{cafeteria}
  \end{phonetics}
\end{entry}

\begin{entry}{国}{8}[Radical ⼞]
  \begin{phonetics}{国}{guo2}[][HSK 1]
    \definition*{s.}{sobrenome Guo}
    \definition[个]{s.}{país | nação}
  \end{phonetics}
\end{entry}

\begin{entry}{国人}{8,2}[Radicais ⼞、⼈]
  \begin{phonetics}{国人}{guo2ren2}
    \definition{s.}{compatriota}
  \end{phonetics}
\end{entry}

\begin{entry}{国内}{8,4}[Radicais ⼞、⼌]
  \begin{phonetics}{国内}{guo2 nei4}[][HSK 3]
    \definition{s.}{interno (a um país); doméstico; lar}
  \end{phonetics}
\end{entry}

\begin{entry}{国外}{8,5}[Radicais ⼞、⼣]
  \begin{phonetics}{国外}{guo2 wai4}[][HSK 1]
    \definition{adj.}{no exterior | externo (assuntos) | estrangeiro}
  \end{phonetics}
\end{entry}

\begin{entry}{国庆}{8,6}[Radicais ⼞、⼴]
  \begin{phonetics}{国庆}{guo2 qing4}[][HSK 3]
    \definition*{s.}{Dia Nacional}
  \end{phonetics}
\end{entry}

\begin{entry}{国庆节}{8,6,5}[Radicais ⼞、⼴、⾋]
  \begin{phonetics}{国庆节}{guo2qing4jie2}
    \definition*{s.}{Dia Nacional (1~de~outubro)}
  \end{phonetics}
\end{entry}

\begin{entry}{国际}{8,7}[Radicais ⼞、⾩]
  \begin{phonetics}{国际}{guo2ji4}[][HSK 2]
    \definition{adj.}{internacional}
  \end{phonetics}
\end{entry}

\begin{entry}{国际儿童节}{8,7,2,12,5}[Radicais ⼞、⾩、⼉、⽴、⾋]
  \begin{phonetics}{国际儿童节}{guo2ji4 er2tong2jie2}
    \definition*{s.}{Dia Internacional das Crianças (1~de~junho)}
  \end{phonetics}
\end{entry}

\begin{entry}{国际妇女节}{8,7,6,3,5}[Radicais ⼞、⾩、⼥、⼥、⾋]
  \begin{phonetics}{国际妇女节}{guo2ji4 fu4nv3jie2}
    \definition*{s.}{Dia Internacional das Mulheres (8~de~março)}
  \end{phonetics}
\end{entry}

\begin{entry}{国际劳动节}{8,7,7,6,5}[Radicais ⼞、⾩、⼒、⼒、⾋]
  \begin{phonetics}{国际劳动节}{guo2ji4 lao2dong4 jie2}
    \definition*{s.}{Dia Internacional dos Trabalhadores (1~de~maio)}
  \end{phonetics}
\end{entry}

\begin{entry}{国语}{8,9}[Radicais ⼞、⾔]
  \begin{phonetics}{国语}{guo2yu3}
    \definition*{s.}{Língua Chinesa (Mandarim), enfatizando sua natureza nacional}
  \end{phonetics}
\end{entry}

\begin{entry}{国家}{8,10}[Radicais ⼞、⼧]
  \begin{phonetics}{国家}{guo2jia1}[][HSK 1]
    \definition[个]{s.}{país | nação | estado}
  \end{phonetics}
\end{entry}

\begin{entry}{国宾馆}{8,10,11}[Radicais ⼞、⼧、⾷]
  \begin{phonetics}{国宾馆}{guo2bin1guan3}
    \definition{s.}{pousada estadual}
  \end{phonetics}
\end{entry}

\begin{entry}{国旗}{8,14}[Radicais ⼞、⽅]
  \begin{phonetics}{国旗}{guo2qi2}
    \definition[面]{s.}{bandeira (de um país)}
  \end{phonetics}
\end{entry}

\begin{entry}{国歌}{8,14}[Radicais ⼞、⽋]
  \begin{phonetics}{国歌}{guo2ge1}
    \definition{s.}{hino nacional}
  \end{phonetics}
\end{entry}

\begin{entry}{图}{8}[Radical ⼞]
  \begin{phonetics}{图}{tu2}[][HSK 3]
    \definition*{s.}{sobrenome Tu}
    \definition[张]{s.}{mapa; gráfico; imagem; desenho | plano; esquema; tentativa}
    \definition{v.}{procurar; perseguir | desenhar; pintar | inventar; planejar; tentar}
  \end{phonetics}
\end{entry}

\begin{entry}{图书馆}{8,4,11}[Radicais ⼞、⼄、⾷]
  \begin{phonetics}{图书馆}{tu2shu1guan3}[][HSK 1]
    \definition[家,个]{s.}{biblioteca}
  \end{phonetics}
\end{entry}

\begin{entry}{图片}{8,4}[Radicais ⼞、⽚]
  \begin{phonetics}{图片}{tu2 pian4}[][HSK 2]
    \definition[张,幅]{s.}{imagem | fotografia}
  \end{phonetics}
\end{entry}

\begin{entry}{图画}{8,8}[Radicais ⼞、⽥]
  \begin{phonetics}{图画}{tu2 hua4}[][HSK 3]
    \definition{s.}{desenho; imagem; pintura}
  \end{phonetics}
\end{entry}

\begin{entry}{坦克}{8,7}[Radicais ⼟、⼗]
  \begin{phonetics}{坦克}{tan3ke4}
    \definition{s.}{(empréstimo linguístico) tanque (veículo militar)}
  \end{phonetics}
\end{entry}

\begin{entry}{垃圾}{8,6}[Radicais ⼟、⼟]
  \begin{phonetics}{垃圾}{la1ji1}
    \definition[把]{s.}{lixo}
  \end{phonetics}
\end{entry}

\begin{entry}{垃圾工}{8,6,3}[Radicais ⼟、⼟、⼯]
  \begin{phonetics}{垃圾工}{la1ji1gong1}
    \definition{s.}{lixeiro | gari}
  \end{phonetics}
\end{entry}

\begin{entry}{垃圾车}{8,6,4}[Radicais ⼟、⼟、⾞]
  \begin{phonetics}{垃圾车}{la1ji1che1}
    \definition{s.}{caminhão de lixo}
  \end{phonetics}
\end{entry}

\begin{entry}{垃圾电邮}{8,6,5,7}[Radicais ⼟、⼟、⽥、⾢]
  \begin{phonetics}{垃圾电邮}{la1ji1dian4you2}
    \definition{s.}{\emph{e-mail} de \emph{spam}}
  \end{phonetics}
\end{entry}

\begin{entry}{垃圾邮件}{8,6,7,6}[Radicais ⼟、⼟、⾢、⼈]
  \begin{phonetics}{垃圾邮件}{la1ji1you2jian4}
    \definition{s.}{\emph{spam}, \emph{e-mail} não solicitado}
  \end{phonetics}
\end{entry}

\begin{entry}{垃圾食品}{8,6,9,9}[Radicais ⼟、⼟、⾷、⼝]
  \begin{phonetics}{垃圾食品}{la1ji1shi2pin3}
    \definition{s.}{\emph{junk food}}
  \end{phonetics}
\end{entry}

\begin{entry}{垃圾堆}{8,6,11}[Radicais ⼟、⼟、⼟]
  \begin{phonetics}{垃圾堆}{la1ji1dui1}
    \definition{s.}{depósito de lixo}
  \end{phonetics}
\end{entry}

\begin{entry}{垃圾筒}{8,6,12}[Radicais ⼟、⼟、⽵]
  \begin{phonetics}{垃圾筒}{la1ji1tong3}
    \definition{s.}{cesto de lixo}
  \end{phonetics}
\end{entry}

\begin{entry}{垃圾箱}{8,6,15}[Radicais ⼟、⼟、⾋]
  \begin{phonetics}{垃圾箱}{la1ji1xiang1}
    \definition{s.}{cesto de lixo}
  \end{phonetics}
\end{entry}

\begin{entry}{备份}{8,6}[Radicais ⼡、⼈]
  \begin{phonetics}{备份}{bei4fen4}
    \definition{s.}{cópia de segurança | \emph{backup}}
  \end{phonetics}
\end{entry}

\begin{entry}{备胎}{8,9}[Radicais ⼡、⾁]
  \begin{phonetics}{备胎}{bei4tai1}
    \definition{s.}{pneu sobressalente | (gíria) substituto}
  \end{phonetics}
\end{entry}

\begin{entry}{夜}{8}[Radical ⼣]
  \begin{phonetics}{夜}{ye4}[][HSK 2]
    \definition{s.}{noite}
  \end{phonetics}
\end{entry}

\begin{entry}{夜生活}{8,5,9}[Radicais ⼣、⽣、⽔]
  \begin{phonetics}{夜生活}{ye4sheng1huo2}
    \definition{s.}{vida noturna}
  \end{phonetics}
\end{entry}

\begin{entry}{夜鸟}{8,5}[Radicais ⼣、⿃]
  \begin{phonetics}{夜鸟}{ye4niao3}
    \definition{s.}{ave noturna}
  \end{phonetics}
\end{entry}

\begin{entry}{夜场}{8,6}[Radicais ⼣、⼟]
  \begin{phonetics}{夜场}{ye4chang3}
    \definition{s.}{show noturno (em um teatro, etc.) | local de entretenimento noturno (bar, boate, discoteca, etc.)}
  \end{phonetics}
\end{entry}

\begin{entry}{夜里}{8,7}[Radicais ⼣、⾥]
  \begin{phonetics}{夜里}{ye4li5}[][HSK 2]
    \definition{adv.}{à noite | durante a noite | período noturno}
  \end{phonetics}
\end{entry}

\begin{entry}{夜夜}{8,8}[Radicais ⼣、⼣]
  \begin{phonetics}{夜夜}{ye4ye4}
    \definition{adv.}{toda noite}
  \end{phonetics}
\end{entry}

\begin{entry}{夜店}{8,8}[Radicais ⼣、⼴]
  \begin{phonetics}{夜店}{ye4dian4}
    \definition{s.}{boate | \emph{nightclub}}
  \end{phonetics}
\end{entry}

\begin{entry}{夜晚}{8,11}[Radicais ⼣、⽇]
  \begin{phonetics}{夜晚}{ye4wan3}
    \definition[个]{s.}{noite}
  \end{phonetics}
\end{entry}

\begin{entry}{夜深人静}{8,11,2,14}[Radicais ⼣、⽔、⼈、⾭]
  \begin{phonetics}{夜深人静}{ye4shen1ren2jing4}
    \definition{expr.}{"Na calada da noite."}
  \end{phonetics}
\end{entry}

\begin{entry}{夜幕}{8,13}[Radicais ⼣、⼱]
  \begin{phonetics}{夜幕}{ye4mu4}
    \definition{s.}{cortina da noite}
  \end{phonetics}
\end{entry}

\begin{entry}{奇怪}{8,8}[Radicais ⼤、⼼]
  \begin{phonetics}{奇怪}{qi2guai4}[][HSK 3]
    \definition{adj.}{estranho; esquisito}
    \definition{v.}{ficar perplexo; maravilhar-se; sentir-se surpreso}
  \end{phonetics}
\end{entry}

\begin{entry}{奇迹}{8,9}[Radicais ⼤、⾡]
  \begin{phonetics}{奇迹}{qi2ji4}
    \definition{adj.}{milagroso}
    \definition{s.}{milagre}
  \end{phonetics}
\end{entry}

\begin{entry}{奋战}{8,9}[Radicais ⼤、⼽]
  \begin{phonetics}{奋战}{fen4zhan4}
    \definition{v.}{lutar bravamente | trabalhar duro}
  \end{phonetics}
\end{entry}

\begin{entry}{妹}{8}[Radical ⼥]
  \begin{phonetics}{妹}{mei4}[][HSK 1]
    \definition[个]{s.}{irmã mais nova}
    \seeref{妹妹}{mei4mei5}
  \end{phonetics}
\end{entry}

\begin{entry}{妹夫}{8,4}[Radicais ⼥、⼤]
  \begin{phonetics}{妹夫}{mei4fu5}
    \definition{s.}{marido da irmã mais nova}
  \end{phonetics}
\end{entry}

\begin{entry}{妹妹}{8,8}[Radicais ⼥、⼥]
  \begin{phonetics}{妹妹}{mei4mei5}[][HSK 1]
    \definition[个]{s.}{irmã mais nova | mulher jovem}
  \end{phonetics}
\end{entry}

\begin{entry}{始终}{8,8}[Radicais ⼥、⽷]
  \begin{phonetics}{始终}{shi3zhong1}[][HSK 3]
    \definition{adv.}{sempre; o tempo todo; durante todo; do começo ao fim}
    \definition{s.}{todo o processo do começo ao fim}
  \end{phonetics}
\end{entry}

\begin{entry}{姐}{8}[Radical ⼥]
  \begin{phonetics}{姐}{jie3}[][HSK 1]
    \definition{s.}{irmã mais velha | um termo geral para mulheres jovens}
    \seeref{姐姐}{jie3 jie5}
  \end{phonetics}
\end{entry}

\begin{entry}{姐夫}{8,4}[Radicais ⼥、⼤]
  \begin{phonetics}{姐夫}{jie3fu5}
    \definition{s.}{marido da irmã mais velha}
  \end{phonetics}
\end{entry}

\begin{entry}{姐姐}{8,8}[Radicais ⼥、⼥]
  \begin{phonetics}{姐姐}{jie3 jie5}[][HSK 1]
    \definition[个]{s.}{irmã mais velha}
  \end{phonetics}
\end{entry}

\begin{entry}{姑且}{8,5}[Radicais ⼥、⼀]
  \begin{phonetics}{姑且}{gu1qie3}
    \definition{adv.}{provisoriamente | por enquanto}
  \end{phonetics}
\end{entry}

\begin{entry}{姑娘}{8,10}[Radicais ⼥、⼥]
  \begin{phonetics}{姑娘}{gu1niang5}[][HSK 3]
    \definition[位,个]{s.}{menina; jovem senhora; mulher solteira | filha}
  \end{phonetics}
\end{entry}

\begin{entry}{姓}{8}[Radical ⼥]
  \begin{phonetics}{姓}{xing4}[][HSK 2]
    \definition[个]{s.}{sobrenome}
    \definition{v.}{ter o sobrenome}
  \end{phonetics}
\end{entry}

\begin{entry}{姓氏}{8,4}[Radicais ⼥、⽒]
  \begin{phonetics}{姓氏}{xing4shi4}
    \definition{s.}{sobrenome}
  \end{phonetics}
\end{entry}

\begin{entry}{姓名}{8,6}[Radicais ⼥、⼝]
  \begin{phonetics}{姓名}{xing4ming2}[][HSK 2]
    \definition{s.}{nome completo}
  \end{phonetics}
\end{entry}

\begin{entry}{委内瑞拉}{8,4,13,8}[Radicais ⼥、⼌、⽟、⼿]
  \begin{phonetics}{委内瑞拉}{wei3nei4rui4la1}
    \definition*{s.}{Venezuela}
  \end{phonetics}
\end{entry}

\begin{entry}{季节}{8,5}[Radicais ⼦、⾋]
  \begin{phonetics}{季节}{ji4jie2}
    \definition[个]{s.}{estação (clima)}
  \end{phonetics}
\end{entry}

\begin{entry}{孤独}{8,9}[Radicais ⼦、⽝]
  \begin{phonetics}{孤独}{gu1du2}
    \definition{adj.}{solitário}
  \end{phonetics}
\end{entry}

\begin{entry}{学}{8}[Radical ⼦]
  \begin{phonetics}{学}{xue2}[][HSK 1]
    \definition{v.}{aprender | estudar}
  \end{phonetics}
\end{entry}

\begin{entry}{学习}{8,3}[Radicais ⼦、⼄]
  \begin{phonetics}{学习}{xue2xi2}[][HSK 1]
    \definition{v.}{estudar | aprender}
  \end{phonetics}
\end{entry}

\begin{entry}{学分}{8,4}[Radicais ⼦、⼑]
  \begin{phonetics}{学分}{xue2fen1}
    \definition{s.}{créditos de um curso}
  \end{phonetics}
\end{entry}

\begin{entry}{学术}{8,5}[Radicais ⼦、⽊]
  \begin{phonetics}{学术}{xue2shu4}
    \definition[个]{s.}{aprendizagem | ciência}
  \end{phonetics}
\end{entry}

\begin{entry}{学生}{8,5}[Radicais ⼦、⽣]
  \begin{phonetics}{学生}{xue2sheng5}[][HSK 1]
    \definition{s.}{estudante | aluno}
  \end{phonetics}
\end{entry}

\begin{entry}{学生证}{8,5,7}[Radicais ⼦、⽣、⾔]
  \begin{phonetics}{学生证}{xue2sheng5zheng4}
    \definition{s.}{cartão de identidade de estudante}
  \end{phonetics}
\end{entry}

\begin{entry}{学会}{8,6}[Radicais ⼦、⼈]
  \begin{phonetics}{学会}{xue2hui4}
    \definition{s.}{instituto | associação (acadêmica) | sociedade científica, douta ou erudita}
    \definition{v.}{aprender | dominar (um assunto)}
  \end{phonetics}
\end{entry}

\begin{entry}{学好}{8,6}[Radicais ⼦、⼥]
  \begin{phonetics}{学好}{xue2hao3}
    \definition{v.}{seguir bons exemplos | aprender bem}
  \end{phonetics}
\end{entry}

\begin{entry}{学问}{8,6}[Radicais ⼦、⾨]
  \begin{phonetics}{学问}{xue2wen4}
    \definition[个]{s.}{conhecimento | aprendizagem}
  \end{phonetics}
\end{entry}

\begin{entry}{学费}{8,9}[Radicais ⼦、⾙]
  \begin{phonetics}{学费}{xue2 fei4}[][HSK 3]
    \definition[个]{s.}{mensalidade (taxa); prêmio; taxas que os alunos devem pagar para estudar na escola, conforme estipulado pela escola | preço pelo que se aprendeu ao custo de cada um; uma metáfora para o preço pago para ganhar uma certa experiência | custo; preço; todas as despesas necessárias para os alunos estudarem}
  \end{phonetics}
\end{entry}

\begin{entry}{学院}{8,9}[Radicais ⼦、⾩]
  \begin{phonetics}{学院}{xue2yuan4}[][HSK 1]
    \definition[所]{s.}{instituto}
  \end{phonetics}
\end{entry}

\begin{entry}{学校}{8,10}[Radicais ⼦、⽊]
  \begin{phonetics}{学校}{xue2xiao4}[][HSK 1]
    \definition{s.}{escola | instituição de ensino}
  \end{phonetics}
\end{entry}

\begin{entry}{学期}{8,12}[Radicais ⼦、⽉]
  \begin{phonetics}{学期}{xue2qi1}[][HSK 2]
    \definition[个]{s.}{semestre}
  \end{phonetics}
\end{entry}

\begin{entry}{官桂}{8,10}[Radicais ⼧、⽊]
  \begin{phonetics}{官桂}{guan1gui4}
    \definition{s.}{canela}
  \seealsoref{肉桂}{rou4gui4}
  \end{phonetics}
\end{entry}

\begin{entry}{定}{8}[Radical ⼧]
  \begin{phonetics}{定}{ding4}[][HSK 4]
    \definition{adj.}{calmo; estável | fixo; estabelecido; fixado; inalterado}
    \definition{adv.}{certamente; com certeza; definitivamente}
    \definition{s.}{sobrenome Ding}
    \definition{v.}{decidir; fixar; definir; determinar; ter certeza | consertar; fazer com que seja consertado | acalmar; estabilizar; tornar estável | assinar (um jornal, etc.); reservar (assentos, ingressos, etc.); encomendar (mercadorias, etc.)}
  \end{phonetics}
\end{entry}

\begin{entry}{定期}{8,12}[Radicais ⼧、⽉]
  \begin{phonetics}{定期}{ding4qi1}[][HSK 3]
    \definition{adj.}{regular; periódico; em intervalos regulares}
    \definition{v.}{fixar (definir) uma data}
  \end{phonetics}
\end{entry}

\begin{entry}{宝}{8}[Radical ⼧]
  \begin{phonetics}{宝}{bao3}[][HSK 4]
    \definition{adj.}{antigo; precioso; estimado}
    \definition[个,件]{s.}{tesouro; objeto estimado; coisa preciosa | dispositivo de jogo; ferramenta de jogo | dinheiro; moeda; moeda antiga com furo quadrado no centro; moeda de prata}
    \definition{s.}{sobrenome Bao}
  \end{phonetics}
\end{entry}

\begin{entry}{宝贝}{8,4}[Radicais ⼧、⾙]
  \begin{phonetics}{宝贝}{bao3bei4}[][HSK 4]
    \definition{adj.}{excêntrico; estranho; imprestável; um termo depreciativo para uma pessoa incompetente ou ridícula}
    \definition[个,件]{s.}{tesouro; objeto estimado; coisa preciosa | querida; \emph{darling}; \emph{baby}; apelido para crianças}
  \end{phonetics}
\end{entry}

\begin{entry}{宝石}{8,5}[Radicais ⼧、⽯]
  \begin{phonetics}{宝石}{bao3 shi2}[][HSK 4]
    \definition[颗,枚,块]{s.}{gema; jóia; pedra preciosa; mineral precioso que tem um brilho lindo e uma dureza de mais de sete graus, não é afetado pela atmosfera ou por produtos químicos e pode ser usado como decoração, suporte de instrumentos ou abrasivos}
  \end{phonetics}
\end{entry}

\begin{entry}{宝宝}{8,8}[Radicais ⼧、⼧]
  \begin{phonetics}{宝宝}{bao3 bao5}[][HSK 4]
    \definition[个]{s.}{querida; \emph{darling}; \emph{baby}; apelido para crianças}
  \end{phonetics}
\end{entry}

\begin{entry}{宝贵}{8,9}[Radicais ⼧、⾙]
  \begin{phonetics}{宝贵}{bao3gui4}[][HSK 4]
    \definition{adj.}{precioso; extremamente valioso, muito raro, pode ser usado para descrever coisas específicas, também pode ser usado para descrever coisas abstratas | valioso; como um tesouro}
  \end{phonetics}
\end{entry}

\begin{entry}{实力}{8,2}[Radicais ⼧、⼒]
  \begin{phonetics}{实力}{shi2li4}[][HSK 3]
    \definition{s.}{força | geralmente se refere à força militar e econômica de um país, grupo ou indivíduo, e também se refere à habilidade de um indivíduo ou grupo em um jogo.}
  \end{phonetics}
\end{entry}

\begin{entry}{实习}{8,3}[Radicais ⼧、⼄]
  \begin{phonetics}{实习}{shi2xi2}[][HSK 2]
    \definition{s.}{estagiário | estágio}
  \end{phonetics}
\end{entry}

\begin{entry}{实在}{8,6}[Radicais ⼧、⼟]
  \begin{phonetics}{实在}{shi2zai4}[][HSK 2]
    \definition{adv.}{realmente | verdadeiramente | de fato | na verdade}
  \end{phonetics}
\end{entry}

\begin{entry}{实行}{8,6}[Radicais ⼧、⾏]
  \begin{phonetics}{实行}{shi2xing2}[][HSK 3]
    \definition{v.}{praticar; implementar; executar; pôr em prática}
  \end{phonetics}
\end{entry}

\begin{entry}{实际}{8,7}[Radicais ⼧、⾩]
  \begin{phonetics}{实际}{shi2ji4}[][HSK 2]
    \definition{adj.}{real | atual | concreto | prático | factual | realista}
    \definition{s.}{realidade | prática}
  \end{phonetics}
\end{entry}

\begin{entry}{实际上}{8,7,3}[Radicais ⼧、⾩、⼀]
  \begin{phonetics}{实际上}{shi2 ji4 shang4}[][HSK 3]
    \definition{adv.}{de fato; na verdade; como aliás}
  \end{phonetics}
\end{entry}

\begin{entry}{实现}{8,8}[Radicais ⼧、⾒]
  \begin{phonetics}{实现}{shi2xian4}[][HSK 2]
    \definition{v.}{alcançar | implementar | constatar}
  \end{phonetics}
\end{entry}

\begin{entry}{实验}{8,10}[Radicais ⼧、⾺]
  \begin{phonetics}{实验}{shi2yan4}[][HSK 3]
    \definition[个,次]{s.}{teste; experimento; trabalho de laboratório}
    \definition{v.}{testar; experimentar}
  \end{phonetics}
\end{entry}

\begin{entry}{实验室}{8,10,9}[Radicais ⼧、⾺、⼧]
  \begin{phonetics}{实验室}{shi2 yan4 shi4}[][HSK 3]
    \definition[个,间]{s.}{laboratório}
  \end{phonetics}
\end{entry}

\begin{entry}{宠物}{8,8}[Radicais ⼧、⽜]
  \begin{phonetics}{宠物}{chong3wu4}
    \definition{s.}{animal de estimação}
  \end{phonetics}
\end{entry}

\begin{entry}{尚且}{8,5}[Radicais ⼩、⼀]
  \begin{phonetics}{尚且}{shang4qie3}
    \definition{conj.}{até | ainda}
  \end{phonetics}
\end{entry}

\begin{entry}{尚且……何况……}{8,5,7,7}[Radicais ⼩、⼀、⼈、⼎]
  \begin{phonetics}{尚且……何况……}{shang4qie3 he2kuang4}
    \definition{conj.}{ainda que\dots, \dots}
  \end{phonetics}
\end{entry}

\begin{entry}{居然}{8,12}[Radicais ⼫、⽕]
  \begin{phonetics}{居然}{ju1ran2}
    \definition{adv.}{inesperadamente | na verdade | para surpresa de alguém}
  \end{phonetics}
\end{entry}

\begin{entry}{岭}{8}[Radical ⼭]
  \begin{phonetics}{岭}{ling3}
    \definition{s.}{cordilheira}
  \end{phonetics}
\end{entry}

\begin{entry}{帘}{8}[Radical ⼱]
  \begin{phonetics}{帘}{lian2}
    \definition{s.}{cortina | tela (pendurada) | bandeira usada como placa de loja}
  \end{phonetics}
\end{entry}

\begin{entry}{幷}{8}[Radical ⼲]
  \begin{phonetics}{幷}{bing4}
    \variantof{并}
  \end{phonetics}
\end{entry}

\begin{entry}{幸亏}{8,3}[Radicais ⼲、⼆]
  \begin{phonetics}{幸亏}{xing4kui1}
    \definition{adv.}{felizmente}
  \end{phonetics}
\end{entry}

\begin{entry}{幸运}{8,7}[Radicais ⼲、⾡]
  \begin{phonetics}{幸运}{xing4yun4}[][HSK 3]
    \definition{adj.}{sortudo; feliz; afortunado}
    \definition[个]{s.}{boa sorte; boa fortuna}
  \end{phonetics}
\end{entry}

\begin{entry}{幸运儿}{8,7,2}[Radicais ⼲、⾡、⼉]
  \begin{phonetics}{幸运儿}{xing4yun4'er2}
    \definition{s.}{pessoa de sorte}
  \end{phonetics}
\end{entry}

\begin{entry}{幸运抽奖}{8,7,8,9}[Radicais ⼲、⾡、⼿、⼤]
  \begin{phonetics}{幸运抽奖}{xing4yun4chou1jiang3}
    \definition{s.}{loteria | sorteio}
  \end{phonetics}
\end{entry}

\begin{entry}{幸福}{8,13}[Radicais ⼲、⽰]
  \begin{phonetics}{幸福}{xing4fu2}[][HSK 3]
    \definition{adj.}{feliz | a vida, a família, etc. deixam as pessoas satisfeitas e felizes}
    \definition{s.}{felicidade; bem estar | uma sensação ou experiência satisfatória e feliz}
  \end{phonetics}
\end{entry}

\begin{entry}{底}{8}[Radical ⼴]
  \begin{phonetics}{底}{de5}
    \definition{part.}{usada após uma palavra ou frase que é usada como determinante para indicar subordinação à palavra central}
  \end{phonetics}
  \begin{phonetics}{底}{di3}[][HSK 4]
    \definition*{s.}{sobrenome Di}
    \definition{pron.}{o que |  isto; isso; aqui | tal | dessa forma}
    \definition{s.}{base; fundo; parte inferior de um objeto | detalhes; o cerne da questão; base, fonte ou contexto de uma coisa | rascunho; cópia mantida como registro; rascunho que pode ser usado como base | final de um ano ou mês | chão; fundo; fundação | a última parte de algo}
  \end{phonetics}
\end{entry}

\begin{entry}{底下}{8,3}[Radicais ⼴、⼀]
  \begin{phonetics}{底下}{di3 xia4}[][HSK 3]
    \definition{adv.}{em baixo; abaixo; sob | próximo; mais tarde; depois}
  \end{phonetics}
\end{entry}

\begin{entry}{底气}{8,4}[Radicais ⼴、⽓]
  \begin{phonetics}{底气}{di3qi4}
    \definition{s.}{capacidade pulmonar | ousadia | confiança | autoconfiança | vigor}
  \end{phonetics}
\end{entry}

\begin{entry}{店}{8}[Radical ⼴]
  \begin{phonetics}{店}{dian4}[][HSK 2]
    \definition{s.}{loja | pousada}
  \end{phonetics}
\end{entry}

\begin{entry}{店主}{8,5}[Radicais ⼴、⼂]
  \begin{phonetics}{店主}{dian4zhu3}
    \definition{s.}{lojista | dono de loja}
  \end{phonetics}
\end{entry}

\begin{entry}{店员}{8,7}[Radicais ⼴、⼝]
  \begin{phonetics}{店员}{dian4yuan2}
    \definition{s.}{assistente de loja | balconista | vendedor}
  \end{phonetics}
\end{entry}

\begin{entry}{建}{8}[Radical ⼵]
  \begin{phonetics}{建}{jian4}[][HSK 3]
    \definition*{s.}{sobrenome Jian}
    \definition{v.}{construir; construir; erigir | estabelecer; configurar; fundar}
  \end{phonetics}
\end{entry}

\begin{entry}{建立}{8,5}[Radicais ⼵、⽴]
  \begin{phonetics}{建立}{jian4li4}[][HSK 3]
    \definition{v.}{estabelecer; construir | vir a ser}
  \end{phonetics}
\end{entry}

\begin{entry}{建立者}{8,5,8}[Radicais ⼵、⽴、⽼]
  \begin{phonetics}{建立者}{jian4li4zhe3}
    \definition{s.}{fundador}
  \end{phonetics}
\end{entry}

\begin{entry}{建议}{8,5}[Radicais ⼵、⾔]
  \begin{phonetics}{建议}{jian4yi4}[][HSK 3]
    \definition[个,点,条]{s.}{proposta; sugestão; recomendação}
    \definition{v.}{propor; sugerir; recomendar}
  \end{phonetics}
\end{entry}

\begin{entry}{建成}{8,6}[Radicais ⼵、⼽]
  \begin{phonetics}{建成}{jian4 cheng2}[][HSK 3]
    \definition{v.}{terminar a construção}
  \end{phonetics}
\end{entry}

\begin{entry}{建设}{8,6}[Radicais ⼵、⾔]
  \begin{phonetics}{建设}{jian4she4}[][HSK 3]
    \definition{s.}{reconstrução; desenvolvimento}
    \definition{v.}{construir}
  \end{phonetics}
\end{entry}

\begin{entry}{建设性}{8,6,8}[Radicais ⼵、⾔、⼼]
  \begin{phonetics}{建设性}{jian4she4xing4}
    \definition{adj.}{construtivo}
    \definition{s.}{construtividade}
  \end{phonetics}
\end{entry}

\begin{entry}{建设者}{8,6,8}[Radicais ⼵、⾔、⽼]
  \begin{phonetics}{建设者}{jian4she4zhe3}
    \definition{s.}{construtor}
  \end{phonetics}
\end{entry}

\begin{entry}{建筑}{8,12}[Radicais ⼵、⽵]
  \begin{phonetics}{建筑}{jian4zhu4}
    \definition[个]{s.}{construção | prédio | edifício}
    \definition{v.}{construir}
  \end{phonetics}
\end{entry}

\begin{entry}{廻}{8}[Radical ⼵]
  \begin{phonetics}{廻}{hui2}
    \variantof{回}
  \end{phonetics}
\end{entry}

\begin{entry}{录}{8}[Radical ⼹]
  \begin{phonetics}{录}{lu4}[][HSK 3]
    \definition*{s.}{sobrenome Lu}
    \definition{s.}{nota; ata; registro; coleção; seleções}
    \definition{v.}{gravar; escrever; copiar | selecionar; empregar; contratar | gravar em fita}
  \end{phonetics}
\end{entry}

\begin{entry}{录音}{8,9}[Radicais ⼹、⾳]
  \begin{phonetics}{录音}{lu4yin1}[][HSK 3]
    \definition[段,个]{s.}{gravação de som; gravação de filme}
    \definition{v.+compl.}{gravar (som, filme)}
  \end{phonetics}
\end{entry}

\begin{entry}{录音机}{8,9,6}[Radicais ⼹、⾳、⽊]
  \begin{phonetics}{录音机}{lu4yin1ji1}
    \definition[台]{s.}{gravador de áudio}
  \end{phonetics}
\end{entry}

\begin{entry}{录像机}{8,13,6}[Radicais ⼹、⼈、⽊]
  \begin{phonetics}{录像机}{lu4xiang4ji1}
    \definition[台]{s.}{gravador de vídeo | VCR}
  \end{phonetics}
\end{entry}

\begin{entry}{录像带}{8,13,9}[Radicais ⼹、⼈、⼱]
  \begin{phonetics}{录像带}{lu4xiang4dai4}
    \definition[盘]{s.}{video-cassete}
  \end{phonetics}
\end{entry}

\begin{entry}{往}{8}[Radical ⼻]
  \begin{phonetics}{往}{wang3}[][HSK 2]
    \definition{prep.}{para | em direção a}
  \end{phonetics}
\end{entry}

\begin{entry}{往日}{8,4}[Radicais ⼻、⽇]
  \begin{phonetics}{往日}{wang3ri4}
    \definition{adv.}{dias passados}
    \definition{s.}{o passado}
  \end{phonetics}
\end{entry}

\begin{entry}{往生}{8,5}[Radicais ⼻、⽣]
  \begin{phonetics}{往生}{wang3sheng1}
    \definition{v.}{renascer | morrer | (Budismo) viver no paraíso}
  \end{phonetics}
\end{entry}

\begin{entry}{往来}{8,7}[Radicais ⼻、⽊]
  \begin{phonetics}{往来}{wang3lai2}
    \definition{s.}{contatos | negociações}
  \end{phonetics}
\end{entry}

\begin{entry}{往返}{8,7}[Radicais ⼻、⾡]
  \begin{phonetics}{往返}{wang3fan3}
    \definition{s.}{ida e volta}
    \definition{v.}{ir e voltar | ir e vir}
  \end{phonetics}
\end{entry}

\begin{entry}{往事}{8,8}[Radicais ⼻、⼅]
  \begin{phonetics}{往事}{wang3shi4}
    \definition{s.}{acontecimentos anteriores | eventos passados}
  \end{phonetics}
\end{entry}

\begin{entry}{往例}{8,8}[Radicais ⼻、⼈]
  \begin{phonetics}{往例}{wang3li4}
    \definition{s.}{prática (habitual) do passado | precedente}
  \end{phonetics}
\end{entry}

\begin{entry}{往往}{8,8}[Radicais ⼻、⼻]
  \begin{phonetics}{往往}{wang3wang3}[][HSK 3]
    \definition{adv.}{frequentemente; muitas vezes; mais frequentemente do que não}
  \end{phonetics}
\end{entry}

\begin{entry}{往昔}{8,8}[Radicais ⼻、⽇]
  \begin{phonetics}{往昔}{wang3xi1}
    \definition{s.}{o passado}
  \end{phonetics}
\end{entry}

\begin{entry}{往复}{8,9}[Radicais ⼻、⼢]
  \begin{phonetics}{往复}{wang3fu4}
    \definition{s.}{para trás e para frente (por exemplo, da ação do pistão ou da bomba)}
    \definition{v.}{ir e voltar | fazer uma viagem de volta}
  \end{phonetics}
\end{entry}

\begin{entry}{往迹}{8,9}[Radicais ⼻、⾡]
  \begin{phonetics}{往迹}{wang3ji4}
    \definition{s.}{eventos passados}
  \end{phonetics}
\end{entry}

\begin{entry}{往程}{8,12}[Radicais ⼻、⽲]
  \begin{phonetics}{往程}{wang3cheng2}
    \definition{s.}{saída (de uma viagem de ônibus ou trem, etc.)}
  \end{phonetics}
\end{entry}

\begin{entry}{念}{8}[Radical ⼼]
  \begin{phonetics}{念}{nian4}[][HSK 3]
    \definition*{s.}{sobrenome Nian}
    \definition{num.}{vinte; 20}
    \definition{s.}{ideia; pensamento}
    \definition{v.}{ler em voz alta | estudar; frequentar a escola | considerar; levar em conta | sentir falta; pensar em}
  \end{phonetics}
\end{entry}

\begin{entry}{忽然}{8,12}[Radicais ⼼、⽕]
  \begin{phonetics}{忽然}{hu1ran2}[][HSK 2]
    \definition{adv.}{de repente}
  \end{phonetics}
\end{entry}

\begin{entry}{态度}{8,9}[Radicais ⼼、⼴]
  \begin{phonetics}{态度}{tai4du5}[][HSK 2]
    \definition[个]{s.}{maneira | comportamento | atitude | atitude | abordagem}
  \end{phonetics}
\end{entry}

\begin{entry}{怕}{8}[Radical ⼼]
  \begin{phonetics}{怕}{pa4}[][HSK 2]
    \definition*{s.}{sobrenome Pa}
    \definition{adv.}{por medo; talvez; suponho}
    \definition{v.}{recear; temer; ter medo de | ser incapaz de suportar (ficar de pé, suportar) | ter medo de; ter medo de}
  \end{phonetics}
\end{entry}

\begin{entry}{性}{8}[Radical ⼼]
  \begin{phonetics}{性}{xing4}[][HSK 3]
    \definition*{s.}{sobrenome Xing}
    \definition[个]{s.}{natureza; caráter; disposição | propriedade; qualidade | sexo; gênero}
    \definition{suf.}{forma substantivo a partir de adjetivo | indica natureza, escopo ou maneira}
  \end{phonetics}
\end{entry}

\begin{entry}{性生活}{8,5,9}[Radicais ⼼、⽣、⽔]
  \begin{phonetics}{性生活}{xing4sheng1huo2}
    \definition{s.}{vida sexual}
  \end{phonetics}
\end{entry}

\begin{entry}{性别}{8,7}[Radicais ⼼、⼑]
  \begin{phonetics}{性别}{xing4bie2}[][HSK 3]
    \definition[种]{s.}{sexo; gênero}
  \end{phonetics}
\end{entry}

\begin{entry}{性侵}{8,9}[Radicais ⼼、⼈]
  \begin{phonetics}{性侵}{xing4qin1}
    \definition{s.}{agressão sexual}
    \definition{v.}{agredir sexualmente}
  \end{phonetics}
\end{entry}

\begin{entry}{性格}{8,10}[Radicais ⼼、⽊]
  \begin{phonetics}{性格}{xing4ge2}[][HSK 3]
    \definition[种,个]{s.}{caráter; temperamento}
  \end{phonetics}
\end{entry}

\begin{entry}{怪}{8}[Radical ⼼]
  \begin{phonetics}{怪}{guai4}
    \definition{adv.}{bastante (linguagem falada)}
  \end{phonetics}
\end{entry}

\begin{entry}{怪兽}{8,11}[Radicais ⼼、⼋]
  \begin{phonetics}{怪兽}{guai4shou4}
    \definition{s.}{animal raro | animal mítico | monstro}
  \end{phonetics}
\end{entry}

\begin{entry}{怪癖}{8,18}[Radicais ⼼、⽧]
  \begin{phonetics}{怪癖}{guai4pi3}
    \definition{adj.}{peculiar}
    \definition{s.}{excentricidade | peculiaridade | hobby estranho}
  \end{phonetics}
\end{entry}

\begin{entry}{或}{8}[Radical ⼽]
  \begin{phonetics}{或}{huo4}[][HSK 2]
    \definition{conj.}{ou | ou\dots ou\dots}
  \end{phonetics}
\end{entry}

\begin{entry}{或者}{8,8}[Radicais ⼽、⽼]
  \begin{phonetics}{或者}{huo4zhe3}[][HSK 2]
    \definition{conj.}{ou (usado em expressões afirmativas)}
  \end{phonetics}
\end{entry}

\begin{entry}{房子}{8,3}[Radicais ⼾、⼦]
  \begin{phonetics}{房子}{fang2 zi5}[][HSK 1]
    \definition[栋,幢,座,套,间,个]{s.}{apartamento | casa | quarto}
  \end{phonetics}
\end{entry}

\begin{entry}{房东}{8,5}[Radicais ⼾、⼀]
  \begin{phonetics}{房东}{fang2dong1}[][HSK 3]
    \definition[个,位]{s.}{dono;  proprietário; senhorio}
  \end{phonetics}
\end{entry}

\begin{entry}{房主}{8,5}[Radicais ⼾、⼂]
  \begin{phonetics}{房主}{fang2zhu3}
    \definition{s.}{proprietário | dono de um imóvel}
  \end{phonetics}
\end{entry}

\begin{entry}{房间}{8,7}[Radicais ⼾、⾨]
  \begin{phonetics}{房间}{fang2jian1}[][HSK 1]
    \definition[间,个]{s.}{quarto}
  \end{phonetics}
\end{entry}

\begin{entry}{房屋}{8,9}[Radicais ⼾、⼫]
  \begin{phonetics}{房屋}{fang2 wu1}[][HSK 3]
    \definition[间,所,套]{s.}{casas; habitação; edifícios}
  \end{phonetics}
\end{entry}

\begin{entry}{房租}{8,10}[Radicais ⼾、⽲]
  \begin{phonetics}{房租}{fang2 zu1}[][HSK 3]
    \definition[笔]{s.}{aluguel}
  \end{phonetics}
\end{entry}

\begin{entry}{所}{8}[Radical ⼾]
  \begin{phonetics}{所}{suo3}[][HSK 3]
    \definition*{s.}{sobrenome Suo | usado como nome de uma agência ou outro escritório}
    \definition{clas.}{para casas, etc.}
    \definition{part.}{usado com "为" ou "被" para indicar voz passiva | usado com verbos, representa a entidade que recebe a ação | usado em conjunto com verbos, seguido de um substantivo que recebe a ação | usado com verbos, "者" ou "的" depois do verbo representa a entidade que recebe a ação}
    \definition{s.}{lugar}
  \end{phonetics}
\end{entry}

\begin{entry}{所以}{8,4}[Radicais ⼾、⼈]
  \begin{phonetics}{所以}{suo3 yi3}[][HSK 2]
    \definition{adv.}{portanto | então | como resultado}
    \definition{conj.}{por isso | como resultado | a razão porque}
  \end{phonetics}
\end{entry}

\begin{entry}{所长}{8,4}[Radicais ⼾、⾧]
  \begin{phonetics}{所长}{suo3 chang2}
    \definition{s.}{aquilo em que alguém é bom; o ponto forte de alguém; o forte de alguém}
  \end{phonetics}
  \begin{phonetics}{所长}{suo3 zhang3}[][HSK 3]
    \definition{s.}{chefe de um instituto, etc. | superintendente}
  \end{phonetics}
\end{entry}

\begin{entry}{所有}{8,6}[Radicais ⼾、⽉]
  \begin{phonetics}{所有}{suo3you3}[][HSK 2]
    \definition{adj.}{todo | tudo}
    \definition{s.}{posses}
    \definition{v.}{possuir | ser dono de}
  \end{phonetics}
\end{entry}

\begin{entry}{承认}{8,4}[Radicais ⼿、⾔]
  \begin{phonetics}{承认}{cheng2ren4}[][HSK 4]
    \definition{s.}{reconhecimento (diplomático, artístico, etc.)}
    \definition{v.}{admitir; reconhecer | dar reconhecimento diplomático; reconhecer}
  \end{phonetics}
\end{entry}

\begin{entry}{承受}{8,8}[Radicais ⼿、⼜]
  \begin{phonetics}{承受}{cheng2shou4}[][HSK 4]
    \definition{v.}{suportar; resistir; realizar (tarefas, dificuldades, pressões, etc.); submeter-se a (testes, etc.) | herdar}
  \end{phonetics}
\end{entry}

\begin{entry}{承担}{8,8}[Radicais ⼿、⼿]
  \begin{phonetics}{承担}{cheng2dan1}[][HSK 4]
    \definition{v.}{suportar; empreender; assumir; tomar conta de algo}
  \end{phonetics}
\end{entry}

\begin{entry}{抬杠}{8,7}[Radicais ⼿、⽊]
  \begin{phonetics}{抬杠}{tai2gang4}
    \definition{v.+compl.}{discutir pelo prazer em discutir | discutir obstinadamente | brigar}
  \end{phonetics}
\end{entry}

\begin{entry}{抱}{8}[Radical ⼿]
  \begin{phonetics}{抱}{bao4}[][HSK 4]
    \definition*{s.}{sobrenome Bao}
    \definition{clas.}{braçada; medida dos dois braços}
    \definition{v.}{carregar no peito; segurar com ambos os braços; abraçar | ter o primeiro filho ou neto | adotar um bebê ou criança | ficar juntos, unidos | encaixar ou servir perfeitamente (roupas e sapatos do tamanho certo) | estimar; nutrir; abrigar; ter em mente | continuar; sobrecarregar com | chocar ovos}
  \end{phonetics}
\end{entry}

\begin{entry}{抱怨}{8,9}[Radicais ⼿、⼼]
  \begin{phonetics}{抱怨}{bao4yuan4}
    \definition{v.}{reclamar | resmungar | abrir uma reclamação | sentir-se insatisfeito}
  \end{phonetics}
\end{entry}

\begin{entry}{抵抗}{8,7}[Radicais ⼿、⼿]
  \begin{phonetics}{抵抗}{di3kang4}
    \definition{s.}{resistência}
    \definition{v.}{resistir}
  \end{phonetics}
\end{entry}

\begin{entry}{抹泪}{8,8}[Radicais ⼿、⽔]
  \begin{phonetics}{抹泪}{mo3lei4}
    \definition{v.}{limpar as lágrimas | (figurativo) derramar lágrimas}
  \end{phonetics}
\end{entry}

\begin{entry}{押}{8}[Radical ⼿]
  \begin{phonetics}{押}{ya1}
    \definition{v.}{deter sob custódia | escoltar e proteger | hipotecar | penhorar}
  \end{phonetics}
\end{entry}

\begin{entry}{押后}{8,6}[Radicais ⼿、⼝]
  \begin{phonetics}{押后}{ya1hou4}
    \definition{v.}{encerrar | adiar}
  \end{phonetics}
\end{entry}

\begin{entry}{押运}{8,7}[Radicais ⼿、⾡]
  \begin{phonetics}{押运}{ya1yun4}
    \definition{v.}{escoltar sob guarda | escoltar (bens ou fundos)}
  \end{phonetics}
\end{entry}

\begin{entry}{押注}{8,8}[Radicais ⼿、⽔]
  \begin{phonetics}{押注}{ya1zhu4}
    \definition{v.}{apostar}
  \end{phonetics}
\end{entry}

\begin{entry}{押金}{8,8}[Radicais ⼿、⾦]
  \begin{phonetics}{押金}{ya1jin1}
    \definition{s.}{caução | sinal | depósito}
  \end{phonetics}
\end{entry}

\begin{entry}{押送}{8,9}[Radicais ⼿、⾡]
  \begin{phonetics}{押送}{ya1song4}
    \definition{v.}{enviar sob escolta | transportar um detido}
  \end{phonetics}
\end{entry}

\begin{entry}{押租}{8,10}[Radicais ⼿、⽲]
  \begin{phonetics}{押租}{ya1zu1}
    \definition{s.}{depósito de aluguel}
  \end{phonetics}
\end{entry}

\begin{entry}{押韵}{8,13}[Radicais ⼿、⾳]
  \begin{phonetics}{押韵}{ya1yun4}
    \definition{v.}{rimar}
  \end{phonetics}
\end{entry}

\begin{entry}{抽}{8}[Radical ⼿]
  \begin{phonetics}{抽}{chou1}[][HSK 4]
    \definition{v.}{retirar; tirar (do meio); retirar, puxar ou arrancar algo que está preso ou emaranhado em outra coisa | tirar, retirar (uma parte de um todo) | (certas plantas) começar a crescer, produzir | bombear | encolher; contrair |
chicotear; açoitar; surrar | dirigir; conduzir | encontrar tempo; libertar-se; sair de alguma coisa}
  \end{phonetics}
\end{entry}

\begin{entry}{抽奖}{8,9}[Radicais ⼿、⼤]
  \begin{phonetics}{抽奖}{chou1 jiang3}[][HSK 4]
    \definition{s.}{loteria; sorteio de loteria}
  \end{phonetics}
\end{entry}

\begin{entry}{抽烟}{8,10}[Radicais ⼿、⽕]
  \begin{phonetics}{抽烟}{chou1yan1}[][HSK 4]
    \definition{v.+compl.}{fumar (um cigarro ou um cachimbo)}
  \end{phonetics}
\end{entry}

\begin{entry}{担心}{8,4}[Radicais ⼿、⼼]
  \begin{phonetics}{担心}{dan1xin1}[][HSK 4]
    \definition{v.}{preocupar-se; ficar ansioso; sentir-se desconfortável com algo}
  \end{phonetics}
\end{entry}

\begin{entry}{担任}{8,6}[Radicais ⼿、⼈]
  \begin{phonetics}{担任}{dan1ren4}[][HSK 4]
    \definition{v.}{servir como; assumir o cargo de; ocupar o posto de; ocupar um determinado cargo ou emprego}
  \end{phonetics}
\end{entry}

\begin{entry}{担保}{8,9}[Radicais ⼿、⼈]
  \begin{phonetics}{担保}{dan1bao3}[][HSK 4]
    \definition{v.}{garantir; atestar; expressar responsabilidade e garantir que não haverá problemas ou que eles serão resolvidos}
  \end{phonetics}
\end{entry}

\begin{entry}{拆}{8}[Radical ⼿]
  \begin{phonetics}{拆}{chai1}
    \definition{v.}{remover | tirar do seu lugar | desfazer | desmontar}
  \end{phonetics}
\end{entry}

\begin{entry}{拉}{8}[Radical ⼿]
  \begin{phonetics}{拉}{la1}[][HSK 2]
    \definition{v.}{puxar | arrastar | desenhar | conversar | (coloquial) esvaziar as entranhas}
  \end{phonetics}
  \begin{phonetics}{拉}{la4}
    \definition{s.}{usado em 拉拉蛄 \dpy{la4la4gu3}}
    \seeref{拉拉蛄}{la4la4gu3}
  \end{phonetics}
\end{entry}

\begin{entry}{拉拉队}{8,8,4}[Radicais ⼿、⼿、⾩]
  \begin{phonetics}{拉拉队}{la1la1dui4}
    \definition{s.}{claque | torcida}
  \end{phonetics}
\end{entry}

\begin{entry}{拉拉蛄}{8,8,11}[Radicais ⼿、⼿、⾍]
  \begin{phonetics}{拉拉蛄}{la4la4gu3}
    \variantof{蝲蝲蛄}
  \end{phonetics}
\end{entry}

\begin{entry}{拍}{8}[Radical ⼿]
  \begin{phonetics}{拍}{pai1}[][HSK 3]
    \definition[只,把]{s.}{bastão; raquete | batida; tempo}
    \definition{v.}{bater palmas; bater; dar um tapa | chicotear; açoitar; bater | enviar (um telegrama, etc.) | tirar (uma foto); fotografar | bajular; lisonjear; adular}
  \end{phonetics}
\end{entry}

\begin{entry}{拍马}{8,3}[Radicais ⼿、⾺]
  \begin{phonetics}{拍马}{pai1ma3}
    \definition{v.}{instigar um cavalo dando tapinhas em seu traseiro | lisonjear | bajular}
  \seealsoref{拍马屁}{pai1ma3pi4}
  \end{phonetics}
\end{entry}

\begin{entry}{拍马屁}{8,3,7}[Radicais ⼿、⾺、⼫]
  \begin{phonetics}{拍马屁}{pai1ma3pi4}
    \definition{s.}{puxa-saco | bajulador}
    \definition{v.}{puxar o saco | bajular}
  \seealsoref{拍马}{pai1ma3}
  \end{phonetics}
\end{entry}

\begin{entry}{拍照}{8,13}[Radicais ⼿、⽕]
  \begin{phonetics}{拍照}{pai1zhao4}
    \definition{v.+compl.}{tirar fotografia}
  \end{phonetics}
\end{entry}

\begin{entry}{拐}{8}[Radical ⼿]
  \begin{phonetics}{拐}{guai3}
    \definition{s.}{bengala | muleta}
    \definition{v.}{virar (uma esquina, etc.) | cortar | sequestrar | fraudar | apropriar-se indevidamente}
  \end{phonetics}
\end{entry}

\begin{entry}{拔尖}{8,6}[Radicais ⼿、⼩]
  \begin{phonetics}{拔尖}{ba2jian1}
    \definition{adj.}{topo de linha | fora do comum | o melhor}
    \definition{v.+compl.}{empurrar-se para a frente | sentir que é superior aos outros}
  \end{phonetics}
\end{entry}

\begin{entry}{拖拉机}{8,8,6}[Radicais ⼿、⼿、⽊]
  \begin{phonetics}{拖拉机}{tuo1la1ji1}
    \definition[台]{s.}{trator}
  \end{phonetics}
\end{entry}

\begin{entry}{拖鞋}{8,15}[Radicais ⼿、⾰]
  \begin{phonetics}{拖鞋}{tuo1xie2}
    \definition[双,只]{s.}{chinelos | sandálias}
  \end{phonetics}
\end{entry}

\begin{entry}{招}{8}[Radical ⼿]
  \begin{phonetics}{招}{zhao1}
    \definition{adj.}{contagioso}
    \definition{s.}{um movimento (xadrez) | uma manobra | dispositivo | truque}
    \definition{v.}{recrutar | provocar | acenar | incorrer | infectar | confessar}
  \end{phonetics}
\end{entry}

\begin{entry}{招手}{8,4}[Radicais ⼿、⼿]
  \begin{phonetics}{招手}{zhao1shou3}
    \definition{v.+compl.}{acenar}
  \end{phonetics}
\end{entry}

\begin{entry}{招数}{8,13}[Radicais ⼿、⽁]
  \begin{phonetics}{招数}{zhao1shu4}
    \definition{s.}{estratégia | movimento (no xadrez, no palco, nas artes marciais) | esquema | truque}
  \end{phonetics}
\end{entry}

\begin{entry}{拧开}{8,4}[Radicais ⼿、⼶]
  \begin{phonetics}{拧开}{ning3kai1}
    \definition{v.}{desaparafusar | desatarrachar | torcer (uma tampa) | abrir (uma torneira) | ligar (girando um botão) | girar (maçaneta da porta)}
  \end{phonetics}
\end{entry}

\begin{entry}{拨转}{8,8}[Radicais ⼿、⾞]
  \begin{phonetics}{拨转}{bo1zhuan3}
    \definition{v.}{transferir (fundos, etc.) | virar | dar a volta}
  \end{phonetics}
\end{entry}

\begin{entry}{放}{8}[Radical ⽅]
  \begin{phonetics}{放}{fang4}[][HSK 1]
    \definition{v.}{liberar | libertar | deixar ir | colocar | por | detonar (fogos de artifício)}
  \end{phonetics}
\end{entry}

\begin{entry}{放下}{8,3}[Radicais ⽅、⼀]
  \begin{phonetics}{放下}{fang4 xia4}[][HSK 2]
    \definition{v.}{deitar | colocar para baixo | deixar ir | liberar | desistir | colocar em algum lugar}
  \end{phonetics}
\end{entry}

\begin{entry}{放大}{8,3}[Radicais ⽅、⼤]
  \begin{phonetics}{放大}{fang4da4}
    \definition{v.}{ampliar}
  \end{phonetics}
\end{entry}

\begin{entry}{放飞}{8,3}[Radicais ⽅、⾶]
  \begin{phonetics}{放飞}{fang4fei1}
    \definition{s.}{deixar voar}
  \end{phonetics}
\end{entry}

\begin{entry}{放心}{8,4}[Radicais ⽅、⼼]
  \begin{phonetics}{放心}{fang4xin1}[][HSK 2]
    \definition{adj.}{despreocupado}
    \definition{v.}{sentir-se aliviado | sentir-se tranquilo | ficar à vontade}
    \definition{v.+compl.}{confiar | ter confiança em alguém | estar à vontade | sentir-se aliviado}
  \end{phonetics}
\end{entry}

\begin{entry}{放出}{8,5}[Radicais ⽅、⼐]
  \begin{phonetics}{放出}{fang4chu1}
    \definition{v.}{liberar | libertar}
  \end{phonetics}
\end{entry}

\begin{entry}{放电}{8,5}[Radicais ⽅、⽥]
  \begin{phonetics}{放电}{fang4dian4}
    \definition{s.}{descarga elétrica}
  \end{phonetics}
\end{entry}

\begin{entry}{放任}{8,6}[Radicais ⽅、⼈]
  \begin{phonetics}{放任}{fang4ren4}
    \definition{v.}{ignorar | saciar-se | deixar sozinho}
  \end{phonetics}
\end{entry}

\begin{entry}{放过}{8,6}[Radicais ⽅、⾡]
  \begin{phonetics}{放过}{fang4guo4}
    \definition{v.}{deixar | deixar alguém escapar impune | passar despercebido}
  \end{phonetics}
\end{entry}

\begin{entry}{放弃}{8,7}[Radicais ⽅、⼶]
  \begin{phonetics}{放弃}{fang4qi4}
    \definition{v.}{abandonar | desistir de | renunciar}
  \end{phonetics}
\end{entry}

\begin{entry}{放弃权利}{8,7,6,7}[Radicais ⽅、⼶、⽊、⼑]
  \begin{phonetics}{放弃权利}{fang4qi4 quan2li4}
    \definition{s.}{renúncia}
  \end{phonetics}
\end{entry}

\begin{entry}{放弃者}{8,7,8}[Radicais ⽅、⼶、⽼]
  \begin{phonetics}{放弃者}{fang4qi4zhe3}
    \definition{s.}{desistente}
  \end{phonetics}
\end{entry}

\begin{entry}{放走}{8,7}[Radicais ⽅、⾛]
  \begin{phonetics}{放走}{fang4zou3}
    \definition{v.}{permitir (uma pessoa ou um animal) ir | liberar | libertar}
  \end{phonetics}
\end{entry}

\begin{entry}{放到}{8,8}[Radicais ⽅、⼑]
  \begin{phonetics}{放到}{fang4 dao4}[][HSK 3]
    \definition{v.}{colocar em; meter}
  \end{phonetics}
\end{entry}

\begin{entry}{放学}{8,8}[Radicais ⽅、⼦]
  \begin{phonetics}{放学}{fang4 xue2}[][HSK 1]
    \definition{v.+compl.}{sair da escola | acabar as aulas | terminar a aula (por hoje)}
  \end{phonetics}
\end{entry}

\begin{entry}{放松}{8,8}[Radicais ⽅、⽊]
  \begin{phonetics}{放松}{fang4song1}
    \definition{adj.}{relaxado | afrouxado}
    \definition{v.}{relaxar | afrouxar}
  \end{phonetics}
\end{entry}

\begin{entry}{放养}{8,9}[Radicais ⽅、⼋]
  \begin{phonetics}{放养}{fang4yang3}
    \definition{v.}{criar (gado, peixes, culturas, etc.) | crescer | criar}
  \end{phonetics}
\end{entry}

\begin{entry}{放假}{8,11}[Radicais ⽅、⼈]
  \begin{phonetics}{放假}{fang4 jia4}[][HSK 1]
    \definition{v.}{ter férias ou feriado}
  \end{phonetics}
\end{entry}

\begin{entry}{放肆}{8,13}[Radicais ⽅、⾀]
  \begin{phonetics}{放肆}{fang4si4}
    \definition{adj.}{atrevido | pesunçoso | devasso}
  \end{phonetics}
\end{entry}

\begin{entry}{放鞭炮}{8,18,9}[Radicais ⽅、⾰、⽕]
  \begin{phonetics}{放鞭炮}{fang4bian1pao4}
    \definition{s.}{um conjunto de bombinhas ou traques}
  \end{phonetics}
\end{entry}

\begin{entry}{斩获}{8,10}[Radicais ⽄、⾋]
  \begin{phonetics}{斩获}{zhan3huo4}
    \definition{v.}{matar ou capturar (em batalha) | (figurativo) (esportes) marcar (um gol), ganhar (uma medalha) | (figurativo) colher recompensas, obter ganhos}
  \end{phonetics}
\end{entry}

\begin{entry}{明天}{8,4}[Radicais ⽇、⼤]
  \begin{phonetics}{明天}{ming2tian1}[][HSK 1]
    \definition{adv.}{amanhã}
  \end{phonetics}
\end{entry}

\begin{entry}{明白}{8,5}[Radicais ⽇、⽩]
  \begin{phonetics}{明白}{ming2bai5}[][HSK 1]
    \definition{adj.}{compreendido | percebido | óbvio | inequívoco}
    \definition{v.}{compreender | perceber}
  \end{phonetics}
\end{entry}

\begin{entry}{明年}{8,6}[Radicais ⽇、⼲]
  \begin{phonetics}{明年}{ming2nian2}[][HSK 1]
    \definition{adv.}{próximo ano}
  \end{phonetics}
\end{entry}

\begin{entry}{明明}{8,8}[Radicais ⽇、⽇]
  \begin{phonetics}{明明}{ming2ming2}
    \definition{adv.}{obviamente | claramente}
  \end{phonetics}
\end{entry}

\begin{entry}{明星}{8,9}[Radicais ⽇、⽇]
  \begin{phonetics}{明星}{ming2xing1}[][HSK 2]
    \definition[个,位,颗]{s.}{estrela | talento de ponta | estrela (artista) | estrela brilhante | estrela brilhante}
  \end{phonetics}
\end{entry}

\begin{entry}{明显}{8,9}[Radicais ⽇、⽇]
  \begin{phonetics}{明显}{ming2xian3}[][HSK 3]
    \definition{adj.}{claro; óbvio; distinto}
  \end{phonetics}
\end{entry}

\begin{entry}{明珠}{8,10}[Radicais ⽇、⽟]
  \begin{phonetics}{明珠}{ming2zhu1}
    \definition{s.}{pérola | jóia (de grande valor)}
  \end{phonetics}
\end{entry}

\begin{entry}{明确}{8,12}[Radicais ⽇、⽯]
  \begin{phonetics}{明确}{ming2que4}[][HSK 3]
    \definition{adj.}{claro; definido; específico}
    \definition{v.}{deixar claro; tornar definitivo}
  \end{phonetics}
\end{entry}

\begin{entry}{昔日}{8,4}[Radicais ⽇、⽇]
  \begin{phonetics}{昔日}{xi1ri4}
    \definition{adj.}{passado}
  \end{phonetics}
\end{entry}

\begin{entry}{朋友}{8,4}[Radicais ⽉、⼜]
  \begin{phonetics}{朋友}{peng2you5}[][HSK 1]
    \definition[个,位]{s.}{amigo}
  \end{phonetics}
\end{entry}

\begin{entry}{服}{8}[Radical ⽉]
  \begin{phonetics}{服}{fu2}
    \definition{s.}{roupas | vestido | vestuário | roupa de luto}
    \definition{v.}{servir (nas forças armadas, uma sentença de prisão, etc.) | obedecer | ser convencido (por um argumento) | convencer | admirar | aclimatar | tomar (medicamento) | usar roupas de luto}
  \end{phonetics}
  \begin{phonetics}{服}{fu4}
    \definition{clas.}{(para remédio) dose}
  \end{phonetics}
\end{entry}

\begin{entry}{服务}{8,5}[Radicais ⽉、⼒]
  \begin{phonetics}{服务}{fu2 wu4}[][HSK 2]
    \definition{v.}{prestar serviço a | estar a serviço de | servir | trabalhar | servir}
  \end{phonetics}
\end{entry}

\begin{entry}{服务员}{8,5,7}[Radicais ⽉、⼒、⼝]
  \begin{phonetics}{服务员}{fu2wu4yuan2}
    \definition{s.}{atendente | garçom | garçonete | pessoal de atendimento ao cliente}
  \end{phonetics}
\end{entry}

\begin{entry}{服装}{8,12}[Radicais ⽉、⾐]
  \begin{phonetics}{服装}{fu2zhuang1}[][HSK 3]
    \definition[套,件,身]{s.}{roupas; trajes; fantasias}
  \end{phonetics}
\end{entry}

\begin{entry}{杯}{8}[Radical ⽊]
  \begin{phonetics}{杯}{bei1}[][HSK 1]
    \definition{clas.}{para certos recipientes de líquidos: copo, xícara, etc.}
    \definition{s.}{copo | caneca | xícara | taça | troféu}
  \end{phonetics}
\end{entry}

\begin{entry}{杯子}{8,3}[Radicais ⽊、⼦]
  \begin{phonetics}{杯子}{bei1 zi5}[][HSK 1]
    \definition[个,只]{s.}{copo | caneca | xícara | taça}
  \end{phonetics}
\end{entry}

\begin{entry}{杯具}{8,8}[Radicais ⽊、⼋]
  \begin{phonetics}{杯具}{bei1ju4}
    \definition{s.}{parachoque | fiasco | (gíria) tragédia}
  \end{phonetics}
\end{entry}

\begin{entry}{松木}{8,4}[Radicais ⽊、⽊]
  \begin{phonetics}{松木}{song1mu4}
    \definition{s.}{pinheiro}
  \end{phonetics}
\end{entry}

\begin{entry}{板}{8}[Radical ⽊]
  \begin{phonetics}{板}{ban3}[][HSK 3]
    \definition{adj.}{rígido; não natural | duro}
    \definition{clas.}{para cartões, papéis}
    \definition{s.}{tábua; placa; prato | veneziana; persiana; refere-se especificamente aos painéis de portas de lojas | badalos (instrumento musical que marca o ritmo) | uma batida acentuada (ritmo) na música e na ópera tradicional | chefe}
    \definition{v.}{parecer sério | livrar-se de maus hábitos ou falhas}
  \end{phonetics}
\end{entry}

\begin{entry}{构}{8}[Radical ⽊]
  \begin{phonetics}{构}{gou4}
    \definition{s.}{composição literária}
    \definition{v.}{construir | formar | compor}
    \variantof{够}
  \end{phonetics}
\end{entry}

\begin{entry}{枕}{8}[Radical ⽊]
  \begin{phonetics}{枕}{zhen3}
    \definition{s.}{travesseiro | almofada}
  \end{phonetics}
\end{entry}

\begin{entry}{果子}{8,3}[Radicais ⽊、⼦]
  \begin{phonetics}{果子}{guo3zi5}
    \definition{s.}{fruta}
  \end{phonetics}
\end{entry}

\begin{entry}{果汁}{8,5}[Radicais ⽊、⽔]
  \begin{phonetics}{果汁}{guo3zhi1}[][HSK 3]
    \definition[杯,瓶,种]{s.}{suco; suco de fruta}
  \end{phonetics}
\end{entry}

\begin{entry}{果然}{8,12}[Radicais ⽊、⽕]
  \begin{phonetics}{果然}{guo3ran2}[][HSK 3]
    \definition{adv.}{realmente; como esperado; com certeza}
    \definition{conj.}{se realmente; se de fato}
  \end{phonetics}
\end{entry}

\begin{entry}{果酱}{8,13}[Radicais ⽊、⾣]
  \begin{phonetics}{果酱}{guo3jiang4}
    \definition{s.}{geléia | compota ou doce (de frutas)}
  \end{phonetics}
\end{entry}

\begin{entry}{枫叶}{8,5}[Radicais ⽊、⼝]
  \begin{phonetics}{枫叶}{feng1ye4}
    \definition{s.}{folha de bordo (maple, tipo de árvore)}
  \end{phonetics}
\end{entry}

\begin{entry}{欧}{8}[Radical ⽋]
  \begin{phonetics}{欧}{ou1}
    \definition*{s.}{Europa, abreviação de~欧洲 | sobrenome Ou}
    \seeref{欧洲}{ou1zhou1}
  \end{phonetics}
\end{entry}

\begin{entry}{欧洲}{8,9}[Radicais ⽋、⽔]
  \begin{phonetics}{欧洲}{ou1zhou1}
    \definition*{s.}{Europa}
  \end{phonetics}
\end{entry}

\begin{entry}{欧洲人}{8,9,2}[Radicais ⽋、⽔、⼈]
  \begin{phonetics}{欧洲人}{ou1zhou1ren2}
    \definition{s.}{europeu | pessoa ou povo da Europa}
  \end{phonetics}
\end{entry}

\begin{entry}{欧洲共同体}{8,9,6,6,7}[Radicais ⽋、⽔、⼋、⼝、⼈]
  \begin{phonetics}{欧洲共同体}{ou1zhou1 gong4tong2ti3}
    \definition*{s.}{Comunidade Europeia}
  \end{phonetics}
\end{entry}

\begin{entry}{欧盟}{8,13}[Radicais ⽋、⽫]
  \begin{phonetics}{欧盟}{ou1meng2}
    \definition*{s.}{Uniáo Europeia}
  \end{phonetics}
\end{entry}

\begin{entry}{武}{8}[Radical ⽌]
  \begin{phonetics}{武}{wu3}
    \definition*{s.}{sobrenome Wu}
    \definition{s.}{arte marcial}
  \end{phonetics}
\end{entry}

\begin{entry}{武力}{8,2}[Radicais ⽌、⼒]
  \begin{phonetics}{武力}{wu3li4}
    \definition{s.}{forças armadas | militares}
  \end{phonetics}
\end{entry}

\begin{entry}{武士}{8,3}[Radicais ⽌、⼠]
  \begin{phonetics}{武士}{wu3shi4}
    \definition{s.}{samurai | guerreiro}
  \end{phonetics}
\end{entry}

\begin{entry}{武大戏}{8,3,6}[Radicais ⽌、⼤、⼽]
  \begin{phonetics}{武大戏}{wu3 da4xi4}
    \definition*{s.}{Drama de Luta Acrobática | Drama Wu}
  \end{phonetics}
\end{entry}

\begin{entry}{武艺}{8,4}[Radicais ⽌、⾋]
  \begin{phonetics}{武艺}{wu3yi4}
    \definition{s.}{arte marcial | habilidade militar}
  \end{phonetics}
\end{entry}

\begin{entry}{武术}{8,5}[Radicais ⽌、⽊]
  \begin{phonetics}{武术}{wu3shu4}[][HSK 3]
    \definition[种,套,门]{s.}{arte marcial; autodefesa; \emph{wushu}}
  \end{phonetics}
\end{entry}

\begin{entry}{武官}{8,8}[Radicais ⽌、⼧]
  \begin{phonetics}{武官}{wu3guan1}
    \definition{s.}{oficial militar}
  \end{phonetics}
\end{entry}

\begin{entry}{武断}{8,11}[Radicais ⽌、⽄]
  \begin{phonetics}{武断}{wu3duan4}
    \definition{adj.}{arbitrário | dogmático | subjetivo}
  \end{phonetics}
\end{entry}

\begin{entry}{武装}{8,12}[Radicais ⽌、⾐]
  \begin{phonetics}{武装}{wu3zhuang1}
    \definition{s.}{forças armadas | militar | arma}
    \definition{v.}{armar}
  \end{phonetics}
\end{entry}

\begin{entry}{武器}{8,16}[Radicais ⽌、⼝]
  \begin{phonetics}{武器}{wu3qi4}[][HSK 3]
    \definition[批,种]{s.}{arma; armamento}
  \end{phonetics}
\end{entry}

\begin{entry}{河}{8}[Radical ⽔]
  \begin{phonetics}{河}{he2}[][HSK 2]
    \definition[条,道]{s.}{rio}
  \end{phonetics}
\end{entry}

\begin{entry}{河蚌}{8,10}[Radicais ⽔、⾍]
  \begin{phonetics}{河蚌}{he2bang4}
    \definition{s.}{mexilhões | bivalves cultivados em rios e lagos}
  \end{phonetics}
\end{entry}

\begin{entry}{油}{8}[Radical ⽔]
  \begin{phonetics}{油}{you2}[][HSK 2]
    \definition{adj.}{oleoso | gorduroso | superficial | astuto}
    \definition{s.}{óleo | gordura | graxa | petróleo}
    \definition{v.}{aplicar óleo de tungue, tinta ou verniz}
  \end{phonetics}
\end{entry}

\begin{entry}{治理}{8,11}[Radicais ⽔、⽟]
  \begin{phonetics}{治理}{zhi4li3}
    \definition{s.}{governança | governo}
    \definition{v.}{gerir para melhor | administrar | por em ordem}
  \end{phonetics}
\end{entry}

\begin{entry}{治愈}{8,13}[Radicais ⽔、⼼]
  \begin{phonetics}{治愈}{zhi4yu4}
    \definition{v.}{curar | restaurar a saúde}
  \end{phonetics}
\end{entry}

\begin{entry}{泄气}{8,4}[Radicais ⽔、⽓]
  \begin{phonetics}{泄气}{xie4qi4}
    \definition{adj.}{decepcionante | frustrante | patético}
    \definition{v.+compl.}{perder o coração | sentir-se desencorajado | ficar desanimado}
  \end{phonetics}
\end{entry}

\begin{entry}{法}{8}[Radical ⽔]
  \begin{phonetics}{法}{fa3}
    \definition*{s.}{França, abreviação de~法国}
  \seealsoref{法国}{fa3guo2}
  \end{phonetics}
\end{entry}

\begin{entry}{法文}{8,4}[Radicais ⽔、⽂]
  \begin{phonetics}{法文}{fa3wen2}
    \definition*{s.}{françês, língua francesa}
  \end{phonetics}
\end{entry}

\begin{entry}{法网}{8,6}[Radicais ⽔、⽹]
  \begin{phonetics}{法网}{fa3wang3}
    \definition*{s.}{Torneio de Roland Garros (French Open), torneio de tênis}
  \end{phonetics}
\end{entry}

\begin{entry}{法国}{8,8}[Radicais ⽔、⼞]
  \begin{phonetics}{法国}{fa3guo2}
    \definition*{s.}{França}
  \end{phonetics}
\end{entry}

\begin{entry}{法国人}{8,8,2}[Radicais ⽔、⼞、⼈]
  \begin{phonetics}{法国人}{fa3guo2ren2}
    \definition{s.}{francês | pessoa ou povo da França}
  \end{phonetics}
\end{entry}

\begin{entry}{法语}{8,9}[Radicais ⽔、⾔]
  \begin{phonetics}{法语}{fa3yu3}
    \definition{s.}{françês, língua francesa}
  \end{phonetics}
\end{entry}

\begin{entry}{泡}{8}[Radical ⽔]
  \begin{phonetics}{泡}{pao1}
    \definition{adj.}{estufado | inchado | esponjoso}
    \definition{clas.}{para urina ou fezes}
    \definition{s.}{pequeno lago (especialmente em nomes de lugares)}
  \end{phonetics}
  \begin{phonetics}{泡}{pao4}
    \definition{clas.}{para ocorrências de uma ação | para número de infusões}
    \definition{s.}{bolha | espuma}
    \definition{v.}{encharcar | infundir | pegar (uma garota) | sair com (um parceiro sexual)}
  \end{phonetics}
\end{entry}

\begin{entry}{波}{8}[Radical ⽔]
  \begin{phonetics}{波}{bo1}
    \definition*{s.}{Polônia, abreviação de 波兰}
    \definition{s.}{onda | ondulação | tempestade | surto}
    \seeref{波兰}{bo1lan2}
  \end{phonetics}
\end{entry}

\begin{entry}{波兰}{8,5}[Radicais ⽔、⼋]
  \begin{phonetics}{波兰}{bo1lan2}
    \definition*{s.}{Polônia}
  \end{phonetics}
\end{entry}

\begin{entry}{波音}{8,9}[Radicais ⽔、⾳]
  \begin{phonetics}{波音}{bo1yin1}
    \definition*{s.}{Boeing (empresa aeroespacial)}
    \definition{s.}{mordente (música)}
  \end{phonetics}
\end{entry}

\begin{entry}{泥}{8}[Radical ⽔]
  \begin{phonetics}{泥}{ni2}
    \definition{s.}{lama | argila | pasta | polpa}
  \end{phonetics}
  \begin{phonetics}{泥}{ni4}
    \definition{adj.}{contido}
  \end{phonetics}
\end{entry}

\begin{entry}{泥潭}{8,15}[Radicais ⽔、⽔]
  \begin{phonetics}{泥潭}{ni2tan2}
    \definition{s.}{atoleiro | lamaçal | charco | pântano}
  \end{phonetics}
\end{entry}

\begin{entry}{注册}{8,5}[Radicais ⽔、⼌]
  \begin{phonetics}{注册}{zhu4ce4}
    \definition{v.}{inscrever-se | matricular-se | registrar-se}
  \end{phonetics}
\end{entry}

\begin{entry}{注册人}{8,5,2}[Radicais ⽔、⼌、⼈]
  \begin{phonetics}{注册人}{zhu4ce4ren2}
    \definition{s.}{registrante}
  \end{phonetics}
\end{entry}

\begin{entry}{注册表}{8,5,8}[Radicais ⽔、⼌、⾐]
  \begin{phonetics}{注册表}{zhu4ce4biao3}
    \definition*{s.}{Registro do Windows}
  \end{phonetics}
\end{entry}

\begin{entry}{注册商标}{8,5,11,9}[Radicais ⽔、⼌、⼝、⽊]
  \begin{phonetics}{注册商标}{zhu4ce4shang1biao1}
    \definition{s.}{marca registrada}
  \end{phonetics}
\end{entry}

\begin{entry}{注意}{8,13}[Radicais ⽔、⼼]
  \begin{phonetics}{注意}{zhu4yi4}[][HSK 3]
    \definition{v.}{prestar atenção em; tomar nota de; ficar de olho em; concentrar seus pensamentos em uma coisa}
  \end{phonetics}
\end{entry}

\begin{entry}{注意力}{8,13,2}[Radicais ⽔、⼼、⼒]
  \begin{phonetics}{注意力}{zhu4yi4li4}
    \definition{s.}{atenção}
  \end{phonetics}
\end{entry}

\begin{entry}{注意力缺失症}{8,13,2,10,5,10}[Radicais ⽔、⼼、⼒、⽸、⼤、⽧]
  \begin{phonetics}{注意力缺失症}{zhu4yi4li4que1shi1zheng4}
    \definition{s.}{transtorno de déficit de atenção}
  \end{phonetics}
\end{entry}

\begin{entry}{注意地}{8,13,6}[Radicais ⽔、⼼、⼟]
  \begin{phonetics}{注意地}{zhu4yi4di4}
    \definition{s.}{área de cuidado, de observação}
  \end{phonetics}
\end{entry}

\begin{entry}{泳池}{8,6}[Radicais ⽔、⽔]
  \begin{phonetics}{泳池}{yong3chi2}
    \definition{s.}{piscina}
  \seealsoref{游泳池}{you2yong3chi2}
  \seealsoref{游泳馆}{you2yong3guan3}
  \end{phonetics}
\end{entry}

\begin{entry}{泳衣}{8,6}[Radicais ⽔、⾐]
  \begin{phonetics}{泳衣}{yong3yi1}
    \definition{s.}{roupa de banho | maiô}
  \seealsoref{游泳衣}{you2yong3yi1}
  \end{phonetics}
\end{entry}

\begin{entry}{炎热}{8,10}[Radicais ⽕、⽕]
  \begin{phonetics}{炎热}{yan2re4}
    \definition{adj.}{extremamente quente | escaldante (clima)}
  \end{phonetics}
\end{entry}

\begin{entry}{炒}{8}[Radical ⽕]
  \begin{phonetics}{炒}{chao3}
    \definition{v.}{saltear | demitir (alguém)}
  \end{phonetics}
\end{entry}

\begin{entry}{爬}{8}[Radical ⽖]
  \begin{phonetics}{爬}{pa2}[][HSK 2]
    \definition{v.}{escalar | subir | trepar | rastejar}
  \end{phonetics}
\end{entry}

\begin{entry}{爬上}{8,3}[Radicais ⽖、⼀]
  \begin{phonetics}{爬上}{pa2shang4}
    \definition{v.}{escalar}
  \end{phonetics}
\end{entry}

\begin{entry}{爬山}{8,3}[Radicais ⽖、⼭]
  \begin{phonetics}{爬山}{pa2shan1}[][HSK 2]
    \definition{s.}{alpinista | montanhismo}
    \definition{v.}{escalar uma montanha}
  \end{phonetics}
\end{entry}

\begin{entry}{爬升}{8,4}[Radicais ⽖、⼗]
  \begin{phonetics}{爬升}{pa2sheng1}
    \definition{v.}{ascender | ganhar promoção | subir (números de vendas, etc.) | aumentar}
  \end{phonetics}
\end{entry}

\begin{entry}{爬行}{8,6}[Radicais ⽖、⾏]
  \begin{phonetics}{爬行}{pa2xing2}
    \definition{v.}{rastejar | arrastar | engatinhar}
  \end{phonetics}
\end{entry}

\begin{entry}{爬杆}{8,7}[Radicais ⽖、⽊]
  \begin{phonetics}{爬杆}{pa2gan1}
    \definition{s.}{escalada em poste}
    \definition{v.}{escalar um poste}
  \end{phonetics}
\end{entry}

\begin{entry}{爬竿}{8,9}[Radicais ⽖、⽵]
  \begin{phonetics}{爬竿}{pa2gan1}
    \definition{s.}{poste de escalada | escalada em poste (como ginástica ou ato de circo)}
  \end{phonetics}
\end{entry}

\begin{entry}{爬梳}{8,11}[Radicais ⽖、⽊]
  \begin{phonetics}{爬梳}{pa2shu1}
    \definition{v.}{vasculhar (documentos históricos, etc.) | desvendar}
  \end{phonetics}
\end{entry}

\begin{entry}{爬犁}{8,11}[Radicais ⽖、⽜]
  \begin{phonetics}{爬犁}{pa2li2}
    \definition{s.}{trenó}
    \seeref{扒犁}{pa2li2}
  \end{phonetics}
\end{entry}

\begin{entry}{爬墙}{8,14}[Radicais ⽖、⼟]
  \begin{phonetics}{爬墙}{pa2qiang2}
    \definition{v.}{escalar uma parede}
  \end{phonetics}
\end{entry}

\begin{entry}{爸}{8}[Radical ⽗]
  \begin{phonetics}{爸}{ba4}[][HSK 1]
    \definition[个,位]{s.}{(informal) pai}
    \seeref{爸爸}{ba4ba5}
  \seealsoref{爸爸}{ba4ba5}
  \end{phonetics}
\end{entry}

\begin{entry}{爸妈}{8,6}[Radicais ⽗、⼥]
  \begin{phonetics}{爸妈}{ba4ma1}
    \definition{s.}{pai e mãe}
  \end{phonetics}
\end{entry}

\begin{entry}{爸爸}{8,8}[Radicais ⽗、⽗]
  \begin{phonetics}{爸爸}{ba4ba5}[][HSK 1]
    \definition[个,位,名,群]{s.}{(informal) pai; papai; papa}
    \seeref{爸}{ba4}
  \end{phonetics}
\end{entry}

\begin{entry}{牦牛}{8,4}[Radicais ⽜、⽜]
  \begin{phonetics}{牦牛}{mao2niu2}
    \definition{s.}{iaque}
  \end{phonetics}
\end{entry}

\begin{entry}{物理}{8,11}[Radicais ⽜、⽟]
  \begin{phonetics}{物理}{wu4li3}
    \definition{s.}{física (disciplina)}
  \end{phonetics}
\end{entry}

\begin{entry}{狒狒}{8,8}[Radicais ⽝、⽝]
  \begin{phonetics}{狒狒}{fei4fei4}
    \definition{s.}{babuíno}
  \end{phonetics}
\end{entry}

\begin{entry}{狗}{8}[Radical ⽝]
  \begin{phonetics}{狗}{gou3}[][HSK 2]
    \definition[条,只]{s.}{cão | cachorro}
  \end{phonetics}
\end{entry}

\begin{entry}{玩}{8}[Radical ⽟]
  \begin{phonetics}{玩}{wan2}
    \definition{s.}{brinquedo | algo usado para diversão}
    \definition{v.}{divertir-se | manter algo para entretenimento | brincar com}
  \end{phonetics}
\end{entry}

\begin{entry}{玩儿}{8,2}[Radicais ⽟、⼉]
  \begin{phonetics}{玩儿}{wan2r5}[][HSK 1]
    \definition{v.}{divertir-se}
  \end{phonetics}
\end{entry}

\begin{entry}{玩艺}{8,4}[Radicais ⽟、⾋]
  \begin{phonetics}{玩艺}{wan2yi4}
    \variantof{玩意}
  \end{phonetics}
\end{entry}

\begin{entry}{玩伴}{8,7}[Radicais ⽟、⼈]
  \begin{phonetics}{玩伴}{wan2ban4}
    \definition{s.}{parceiro de brincadeira}
  \end{phonetics}
\end{entry}

\begin{entry}{玩具}{8,8}[Radicais ⽟、⼋]
  \begin{phonetics}{玩具}{wan2ju4}[][HSK 3]
    \definition[个,件,套,只,辆]{s.}{brinquedo; brincadeira}
  \end{phonetics}
\end{entry}

\begin{entry}{玩具厂}{8,8,2}[Radicais ⽟、⼋、⼚]
  \begin{phonetics}{玩具厂}{wan2ju4chang3}
    \definition{s.}{fábrica de brinquedos}
  \end{phonetics}
\end{entry}

\begin{entry}{玩具车}{8,8,4}[Radicais ⽟、⼋、⾞]
  \begin{phonetics}{玩具车}{wan2ju4 che1}
    \definition{s.}{carrinho de brinquedo}
  \end{phonetics}
\end{entry}

\begin{entry}{玩味}{8,8}[Radicais ⽟、⼝]
  \begin{phonetics}{玩味}{wan2wei4}
    \definition{v.}{ponderar sutilezas | ruminar (pensamentos)}
  \end{phonetics}
\end{entry}

\begin{entry}{玩者}{8,8}[Radicais ⽟、⽼]
  \begin{phonetics}{玩者}{wan2zhe3}
    \definition{s.}{jogador}
  \end{phonetics}
\end{entry}

\begin{entry}{玩耍}{8,9}[Radicais ⽟、⽽]
  \begin{phonetics}{玩耍}{wan2shua3}
    \definition{v.}{divertir-me | brincar (como as crianças fazem)}
  \end{phonetics}
\end{entry}

\begin{entry}{玩家}{8,10}[Radicais ⽟、⼧]
  \begin{phonetics}{玩家}{wan2jia1}
    \definition{s.}{entusiasta (áudio, modelos de aviões, etc.) | jogador (de um jogo)}
  \end{phonetics}
\end{entry}

\begin{entry}{玩偶}{8,11}[Radicais ⽟、⼈]
  \begin{phonetics}{玩偶}{wan2'ou3}
    \definition{s.}{estatueta de brinquedo | boneco de ação | bicho de pelúcia | boneca}
  \end{phonetics}
\end{entry}

\begin{entry}{玩遍}{8,12}[Radicais ⽟、⾡]
  \begin{phonetics}{玩遍}{wan2bian4}
    \definition{v.}{passear (todo o país, toda a cidade, etc.) | visitar (um grande número de lugares)}
  \end{phonetics}
\end{entry}

\begin{entry}{玩意}{8,13}[Radicais ⽟、⼼]
  \begin{phonetics}{玩意}{wan2yi4}
    \definition{s.}{ato | brinquedo | coisa | truque (em uma performance, show de palco, acrobacias, etc.)}
  \end{phonetics}
\end{entry}

\begin{entry}{环}{8}[Radical ⽟]
  \begin{phonetics}{环}{huan2}[][HSK 3]
    \definition*{s.}{sobrenome Huan}
    \definition{clas.}{para anéis}
    \definition{s.}{anel; arco | \emph{link}; ligação}
    \definition{v.}{cercar; rodear; circular; circundar}
  \end{phonetics}
\end{entry}

\begin{entry}{环卫}{8,3}[Radicais ⽟、⼙]
  \begin{phonetics}{环卫}{huan2wei4}
    \definition{s.}{limpeza pública | saneamento urbano | saneamento ambiental | abreviação de 环境卫生}
    \seeref{环境卫生}{huan2jing4wei4sheng1}
  \end{phonetics}
\end{entry}

\begin{entry}{环保}{8,9}[Radicais ⽟、⼈]
  \begin{phonetics}{环保}{huan2 bao3}[][HSK 3]
    \definition{adj.}{bom para o meio ambiente; não danifica o meio ambiente}
    \definition{s.}{proteção ambiental}
  \end{phonetics}
\end{entry}

\begin{entry}{环境}{8,14}[Radicais ⽟、⼟]
  \begin{phonetics}{环境}{huan2jing4}[][HSK 3]
    \definition[个]{s.}{ambiente | arredores; circunstâncias}
  \end{phonetics}
\end{entry}

\begin{entry}{环境卫生}{8,14,3,5}[Radicais ⽟、⼟、⼙、⽣]
  \begin{phonetics}{环境卫生}{huan2jing4wei4sheng1}
    \definition{s.}{saneamento ambiental}
  \seealsoref{环卫}{huan2wei4}
  \end{phonetics}
\end{entry}

\begin{entry}{现}{8}[Radical ⾒]
  \begin{phonetics}{现}{xian4}
    \definition{adj.}{presente | atual}
    \definition{v.}{aparecer}
    \seeref{见}{xian4}
  \end{phonetics}
\end{entry}

\begin{entry}{现代}{8,5}[Radicais ⾒、⼈]
  \begin{phonetics}{现代}{xian4dai4}[][HSK 3]
    \definition*{s.}{Hyundai, empresa sul-coreana}
    \definition{adj.}{moderno; contemporâneo}
    \definition{s.}{tempos modernos; era contemporânea}
  \end{phonetics}
\end{entry}

\begin{entry}{现在}{8,6}[Radicais ⾒、⼟]
  \begin{phonetics}{现在}{xian4zai4}[][HSK 1]
    \definition{adv.}{agora | neste momento}
  \end{phonetics}
\end{entry}

\begin{entry}{现场}{8,6}[Radicais ⾒、⼟]
  \begin{phonetics}{现场}{xian4chang3}[][HSK 3]
    \definition[个,处]{s.}{cena (de um incidente) | local; lugar; sítio}
  \end{phonetics}
\end{entry}

\begin{entry}{现有}{8,6}[Radicais ⾒、⽉]
  \begin{phonetics}{现有}{xian4you3}
    \definition{adj.}{disponível atualmente | atualmente existente}
  \end{phonetics}
\end{entry}

\begin{entry}{现抓}{8,7}[Radicais ⾒、⼿]
  \begin{phonetics}{现抓}{xian4zhua1}
    \definition{v.}{improvisar}
  \end{phonetics}
\end{entry}

\begin{entry}{现实}{8,8}[Radicais ⾒、⼧]
  \begin{phonetics}{现实}{xian4shi2}[][HSK 3]
    \definition{adj.}{real; atual}
    \definition[个]{s.}{realidade; atualidade}
  \end{phonetics}
\end{entry}

\begin{entry}{现货}{8,8}[Radicais ⾒、⾙]
  \begin{phonetics}{现货}{xian4huo4}
    \definition{s.}{produtos à vista}
  \end{phonetics}
\end{entry}

\begin{entry}{现货的}{8,8,8}[Radicais ⾒、⾙、⽩]
  \begin{phonetics}{现货的}{xian4huo4 de5}
    \definition{s.}{produtos em estoque}
  \end{phonetics}
\end{entry}

\begin{entry}{现金}{8,8}[Radicais ⾒、⾦]
  \begin{phonetics}{现金}{xian4jin1}[][HSK 3]
    \definition[笔]{s.}{dinheiro; dinheiro vivo | reserva de dinheiro em um banco}
  \end{phonetics}
\end{entry}

\begin{entry}{现做}{8,11}[Radicais ⾒、⼈]
  \begin{phonetics}{现做}{xian4zuo4}
    \definition{adj.}{fresco}
    \definition{v.}{fazer (comida) no local}
  \end{phonetics}
\end{entry}

\begin{entry}{现象}{8,11}[Radicais ⾒、⾗]
  \begin{phonetics}{现象}{xian4xiang4}[][HSK 3]
    \definition[个,种]{s.}{aparência (das coisas); fenômeno}
  \end{phonetics}
\end{entry}

\begin{entry}{画}{8}[Radical ⽥]
  \begin{phonetics}{画}{hua4}[][HSK 2]
    \definition[幅,张]{s.}{quadro | pintura | traço de um caractere chinês (variante de 划) | (caligrafia) traço horizontal (variante de traço 划)}
    \definition{v.}{desenhar | pintar | traçar uma linha (variante de 划)}
    \seeref{划}{hua4}
  \end{phonetics}
\end{entry}

\begin{entry}{画儿}{8,2}[Radicais ⽥、⼉]
  \begin{phonetics}{画儿}{hua4r5}[][HSK 2]
    \definition{s.}{imagem | desenho | pintura}
  \end{phonetics}
\end{entry}

\begin{entry}{画地为牢}{8,6,4,7}[Radicais ⽥、⼟、⼂、⼧]
  \begin{phonetics}{画地为牢}{hua4di4wei2lao2}
    \definition{expr.}{(literalmente) ser confinado dentro de um círculo desenhado no chão | (figurativo) limitar-se a uma gama restrita de atividades}
  \end{phonetics}
\end{entry}

\begin{entry}{画家}{8,10}[Radicais ⽥、⼧]
  \begin{phonetics}{画家}{hua4 jia1}[][HSK 2]
    \definition[个,名,位]{s.}{pintor}
  \end{phonetics}
\end{entry}

\begin{entry}{的}{8}[Radical ⽩]
  \begin{phonetics}{的}{de5}
    \definition{part.}{de | partícula usada em possessivos | utilizada entre adjetivos e substantivos (opcional se o adjetivo possui apenas um caracter) | usado após um atributo | usado para formar uma expressão nominal | usado no final de uma frase declarativa para dar ênfase}
  \end{phonetics}
  \begin{phonetics}{的}{di1}
    \definition{s.}{abreviação de 的士: um táxi}
  \seealsoref{的士}{di1shi4}
  \end{phonetics}
  \begin{phonetics}{的}{di2}
    \definition{adv.}{realmente e verdadeiramente}
  \end{phonetics}
  \begin{phonetics}{的}{di4}
    \definition{adj.}{objetivo | claro}
  \end{phonetics}
\end{entry}

\begin{entry}{的士}{8,3}[Radicais ⽩、⼠]
  \begin{phonetics}{的士}{di1shi4}
    \definition{s.}{(empréstimo linguístico) táxi}
  \end{phonetics}
\end{entry}

\begin{entry}{的话}{8,8}[Radicais ⽩、⾔]
  \begin{phonetics}{的话}{de5 hua4}[][HSK 2]
    \definition{part.}{se | no caso | suponha que}
  \end{phonetics}
\end{entry}

\begin{entry}{的确}{8,12}[Radicais ⽩、⽯]
  \begin{phonetics}{的确}{di2que4}[][HSK 4]
    \definition{adv.}{realmente; de fato, ao expressar certeza sobre a situação}
  \end{phonetics}
\end{entry}

\begin{entry}{盲目}{8,5}[Radicais ⽬、⽬]
  \begin{phonetics}{盲目}{mang2mu4}
    \definition{adj.}{ignorante | sem compreensão}
    \definition{adv.}{cegamente}
    \definition{s.}{cego}
  \end{phonetics}
\end{entry}

\begin{entry}{直}{8}[Radical ⽬]
  \begin{phonetics}{直}{zhi2}[][HSK 3]
    \definition*{s.}{sobrenome Zhi}
    \definition{adj.}{reto; rígido | ereto; vertical; perpendicular; vertical ao solo; de cima para baixo; da frente para trás | justo; honesto; correto | franco; direto}
    \definition{adv.}{diretamente; sempre; reto | continuamente; constantemente | apenas; simplesmente; de ​​fato}
    \definition[条]{s.}{traço vertical (em caracteres chineses, ``竖'')}
    \definition{v.}{endireitar; esticar}
  \end{phonetics}
\end{entry}

\begin{entry}{直译}{8,7}[Radicais ⽬、⾔]
  \begin{phonetics}{直译}{zhi2yi4}
    \definition{s.}{tradução literal}
  \seealsoref{意译}{yi4yi4}
  \end{phonetics}
\end{entry}

\begin{entry}{直译器}{8,7,16}[Radicais ⽬、⾔、⼝]
  \begin{phonetics}{直译器}{zhi2yi4qi4}
    \definition{s.}{(computação) interpretador}
  \end{phonetics}
\end{entry}

\begin{entry}{直到}{8,8}[Radicais ⽬、⼑]
  \begin{phonetics}{直到}{zhi2 dao4}[][HSK 3]
    \definition{adv.}{até (na maior parte do tempo)}
  \end{phonetics}
\end{entry}

\begin{entry}{直接}{8,11}[Radicais ⽬、⼿]
  \begin{phonetics}{直接}{zhi2jie1}[][HSK 2]
    \definition{adj.}{direto (oposto: indireto 间接) | imediato}
    \seeref{间接}{jian4jie1}
  \end{phonetics}
\end{entry}

\begin{entry}{直播}{8,15}[Radicais ⽬、⼿]
  \begin{phonetics}{直播}{zhi2bo1}[][HSK 3]
    \definition{s.}{transmissão ao vivo; transmissão sem gravação por estações de rádio ou sem gravação por canais de televisão | (agricultura) semeadura direta}
    \definition[次]{v.}{(TV, rádio, Internet) transmitir ao vivo}
  \end{phonetics}
\end{entry}

\begin{entry}{知识}{8,7}[Radicais ⽮、⾔]
  \begin{phonetics}{知识}{zhi1shi5}[][HSK 1]
    \definition[门]{s.}{conhecimento}
    \definition{s.}{intelectual}
  \end{phonetics}
\end{entry}

\begin{entry}{知道}{8,12}[Radicais ⽮、⾡]
  \begin{phonetics}{知道}{zhi1dao4}[][HSK 1]
    \definition{v.}{conhecer | saber}
  \end{phonetics}
\end{entry}

\begin{entry}{知道了}{8,12,2}[Radicais ⽮、⾡、⼅]
  \begin{phonetics}{知道了}{zhi1dao4le5}
    \definition{interj.}{Entendi! | OK!}
  \end{phonetics}
\end{entry}

\begin{entry}{矿泉水}{8,9,4}[Radicais ⽯、⽔、⽔]
  \begin{phonetics}{矿泉水}{kuang4quan2shui3}
    \definition[瓶,杯]{s.}{água mineral}
  \end{phonetics}
\end{entry}

\begin{entry}{祅}{8}[Radical ⽰]
  \begin{phonetics}{祅}{yao1}
    \definition{s.}{espírito maligno | \emph{goblin} | bruxaria}
    \variantof{妖}
  \end{phonetics}
\end{entry}

\begin{entry}{空}{8}[Radical ⽳]
  \begin{phonetics}{空}{kong1}[][HSK 3]
    \definition*{s.}{sobrenome Kong}
    \definition{adj.}{vazio; oco; nulo}
    \definition{adv.}{por nada; em vão}
    \definition{s.}{céu; ar | vazio; vazio do mundo dos sentidos}
  \end{phonetics}
  \begin{phonetics}{空}{kong4}
    \definition{adj.}{desocupado; vago; em branco}
    \definition{s.}{espaço vazio | tempo livre}
    \definition{v.}{deixar em branco ou vazio; esvaziar; desocupar}
  \end{phonetics}
\end{entry}

\begin{entry}{空儿}{8,2}[Radicais ⽳、⼉]
  \begin{phonetics}{空儿}{kong4r5}[][HSK 3]
    \definition{s.}{tempo livre | espaço (não utilizado)}
    \definition{v.}{ter tempo livre}
  \end{phonetics}
\end{entry}

\begin{entry}{空中小姐}{8,4,3,8}[Radicais ⽳、⼁、⼩、⼥]
  \begin{phonetics}{空中小姐}{kong1zhong1xiao3jie3}
    \definition{s.}{aeromoça}
  \end{phonetics}
\end{entry}

\begin{entry}{空心菜}{8,4,11}[Radicais ⽳、⼼、⾋]
  \begin{phonetics}{空心菜}{kong1xin1cai4}
    \definition{s.}{espinafre aquático | \emph{ong choy} | repolho do pântano | convolvulus aquático | glória-da-manhã aquática}
  \seealsoref{蕹菜}{weng4cai4}
  \end{phonetics}
\end{entry}

\begin{entry}{空气}{8,4}[Radicais ⽳、⽓]
  \begin{phonetics}{空气}{kong1qi4}[][HSK 2]
    \definition{s.}{ar | atmosfera}
  \end{phonetics}
\end{entry}

\begin{entry}{空间}{8,7}[Radicais ⽳、⾨]
  \begin{phonetics}{空间}{kong1jian1}
    \definition{s.}{espaço | sala | (figurativo) escopo | (astronomia) espaço sideral | (matemática, física) espaço}
  \end{phonetics}
\end{entry}

\begin{entry}{空间站}{8,7,10}[Radicais ⽳、⾨、⽴]
  \begin{phonetics}{空间站}{kong1jian1zhan4}
    \definition{s.}{estação espacial}
  \end{phonetics}
\end{entry}

\begin{entry}{空姐}{8,8}[Radicais ⽳、⼥]
  \begin{phonetics}{空姐}{kong1jie3}
    \definition{s.}{aeromoça | comissária de bordo | abreviação de 空中小姐}
    \seeref{空中小姐}{kong1zhong1xiao3jie3}
  \end{phonetics}
\end{entry}

\begin{entry}{空调}{8,10}[Radicais ⽳、⾔]
  \begin{phonetics}{空调}{kong1tiao2}[][HSK 3]
    \definition[台]{s.}{ar-condicionado;  condicionador de ar}
  \end{phonetics}
\end{entry}

\begin{entry}{线}{8}[Radical ⽷]
  \begin{phonetics}{线}{xian4}[][HSK 3]
    \definition{clas.}{para coisas abstratas, o número é limitado a "一"}
    \definition{s.}{fio; corda; arame | linha | feito de fio de algodão | algo em forma de linha, fio, etc. | rota; linha | linha de demarcação; limite | beira; borda | linha ideológica e política | pista; fio}
  \end{phonetics}
\end{entry}

\begin{entry}{线香}{8,9}[Radicais ⽷、⾹]
  \begin{phonetics}{线香}{xian4xiang1}
    \definition{s.}{bastão ou vareta de incenso}
  \end{phonetics}
\end{entry}

\begin{entry}{练}{8}[Radical ⽷]
  \begin{phonetics}{练}{lian4}[][HSK 2]
    \definition{s.}{exercício | (literário) seda branca}
    \definition{v.}{praticar | treinar | aperfeiçoar (habilidade) | ferver e esfregar seda crua}
  \end{phonetics}
\end{entry}

\begin{entry}{练习}{8,3}[Radicais ⽷、⼄]
  \begin{phonetics}{练习}{lian4xi2}[][HSK 2]
    \definition[个]{s.}{prática | exercício}
    \definition{v.}{praticar | exercitar}
  \end{phonetics}
\end{entry}

\begin{entry}{组}{8}[Radical ⽷]
  \begin{phonetics}{组}{zu3}[][HSK 2]
    \definition*{s.}{sobrenome Zu}
    \definition{clas.}{para conjuntos, séries, suítes, baterias}
    \definition{s.}{grupo}
    \definition{v.}{organizar | formar}
  \end{phonetics}
\end{entry}

\begin{entry}{组长}{8,4}[Radicais ⽷、⾧]
  \begin{phonetics}{组长}{zu3 zhang3}[][HSK 2]
    \definition[名,位,个]{s.}{líder de grupo}
  \end{phonetics}
\end{entry}

\begin{entry}{组合}{8,6}[Radicais ⽷、⼝]
  \begin{phonetics}{组合}{zu3he2}[][HSK 3]
    \definition{s.}{associação; combinação
combinação; pegue n elementos de m elementos diferentes e agrupe-os em grupos, independentemente da ordem, onde cada grupo contém pelo menos um componente diferente, o resultado é chamado de combinação de n de m.}
    \definition{v.}{criar; compor; constituir}
  \end{phonetics}
\end{entry}

\begin{entry}{组成}{8,6}[Radicais ⽷、⼽]
  \begin{phonetics}{组成}{zu3cheng2}[][HSK 2]
    \definition{v.}{formar | compor | inventar}
  \end{phonetics}
\end{entry}

\begin{entry}{细节}{8,5}[Radicais ⽷、⾋]
  \begin{phonetics}{细节}{xi4jie2}
    \definition{s.}{detalhe | particularidade}
  \end{phonetics}
\end{entry}

\begin{entry}{细菌战}{8,11,9}[Radicais ⽷、⾋、⼽]
  \begin{phonetics}{细菌战}{xi4jun1zhan4}
    \definition{s.}{guerra biológica}
  \end{phonetics}
\end{entry}

\begin{entry}{织}{8}[Radical ⽷]
  \begin{phonetics}{织}{zhi1}
    \definition{v.}{tecer | tricotar}
  \end{phonetics}
\end{entry}

\begin{entry}{终于}{8,3}[Radicais ⽷、⼆]
  \begin{phonetics}{终于}{zhong1yu2}[][HSK 3]
    \definition{adv.}{finalmente; eventualmente; no final; indica uma situação que ocorre após várias mudanças ou esperas}
  \end{phonetics}
\end{entry}

\begin{entry}{经}{8}[Radical ⽷]
  \begin{phonetics}{经}{jing1}
    \definition*{s.}{sobrenome Jing}
    \definition{s.}{livro sagrado | escritura | clássicos | longitude | menstruação | canal}
    \definition{v.}{passar | sofrer | suportar | deformar (têxtil)}
  \end{phonetics}
\end{entry}

\begin{entry}{经历}{8,4}[Radicais ⽷、⼚]
  \begin{phonetics}{经历}{jing1li4}[][HSK 3]
    \definition[个,次,段,种]{s.}{experiência}
    \definition{v.}{passar por}
  \end{phonetics}
\end{entry}

\begin{entry}{经过}{8,6}[Radicais ⽷、⾡]
  \begin{phonetics}{经过}{jing1guo4}[][HSK 2]
    \definition[个]{s.}{processo | curso}
    \definition{v.}{passar | passar por}
  \end{phonetics}
\end{entry}

\begin{entry}{经济}{8,9}[Radicais ⽷、⽔]
  \begin{phonetics}{经济}{jing1ji4}[][HSK 3]
    \definition{adj.}{econômico;  parcimonioso}
    \definition{s.}{economia |economia nacional; setor da economia nacional | renda; condição financeira}
    \definition{v.}{governar o país}
  \end{phonetics}
\end{entry}

\begin{entry}{经验}{8,10}[Radicais ⽷、⾺]
  \begin{phonetics}{经验}{jing1yan4}[][HSK 3]
    \definition[个,次,种]{s.}{experiência}
    \definition{v.}{experimentar; passar por}
  \end{phonetics}
\end{entry}

\begin{entry}{经常}{8,11}[Radicais ⽷、⼱]
  \begin{phonetics}{经常}{jing1chang2}[][HSK 2]
    \definition{adv.}{constantemente | diariamente | dia-a-dia | todo dia | frequentemente | sempre | regularmente}
  \end{phonetics}
\end{entry}

\begin{entry}{经理}{8,11}[Radicais ⽷、⽟]
  \begin{phonetics}{经理}{jing1li3}[][HSK 2]
    \definition[个,位,名]{s.}{diretor | gerente}
  \end{phonetics}
\end{entry}

\begin{entry}{经营}{8,11}[Radicais ⽷、⾋]
  \begin{phonetics}{经营}{jing1ying2}[][HSK 3]
    \definition{v.}{executar; gerenciar; operar; envolver-se em | gerenciar}
  \end{phonetics}
\end{entry}

\begin{entry}{罔}{8}[Radical ⼌]
  \begin{phonetics}{罔}{wang3}
    \definition{v.}{enganar}
  \end{phonetics}
\end{entry}

\begin{entry}{罗}{8}[Radical ⽹]
  \begin{phonetics}{罗}{luo2}
    \definition*{s.}{sobrenome Luo}
    \definition{v.}{coletar | juntar | pegar | peneirar}
  \end{phonetics}
\end{entry}

\begin{entry}{者}{8}[Radical ⽼]
  \begin{phonetics}{者}{zhe3}[][HSK 3]
    \definition{part.}{usado depois de um adjetivo ou verbo, ou depois de uma frase com um adjetivo ou verbo, para indicar uma pessoa ou coisa que tem esse atributo ou realiza essa ação | usado depois de ´´trabalho'' e ´´ismo'' para se referir a pessoas que fazem um determinado trabalho ou acreditam em uma determinada ideologia | usado depois de numerais como "dois", "três" etc., referindo-se a vários itens mencionados no contexto | usado depois de palavras, frases ou cláusulas para indicar uma pausa}
    \definition{pron.}{usado principalmente no vernáculo antigo, significando o mesmo que "这"}
    \definition{suf.}{voluntário}
  \seealsoref{这}{zhe4}
  \end{phonetics}
\end{entry}

\begin{entry}{肏}{8}[Radical ⼊]
  \begin{phonetics}{肏}{cao4}
    \definition{v.}{(vulgar) foder}
  \end{phonetics}
\end{entry}

\begin{entry}{肩膀}{8,14}[Radicais ⾁、⾁]
  \begin{phonetics}{肩膀}{jian1bang3}
    \definition{s.}{ombro}
  \end{phonetics}
\end{entry}

\begin{entry}{肯定}{8,8}[Radicais ⾁、⼧]
  \begin{phonetics}{肯定}{ken3ding4}
    \definition{adv.}{com certeza | certamente | definitivamente | afirmativo (resposta)}
    \definition{v.}{afirmar | ter a certeza | ser positivo | dar reconhecimento}
  \end{phonetics}
\end{entry}

\begin{entry}{苦瓜}{8,5}[Radicais ⾋、⽠]
  \begin{phonetics}{苦瓜}{ku3gua1}
    \definition{s.}{melão amargo (cabaça amarga, pêra bálsamo, maçã bálsamo, pepino amargo)}
  \end{phonetics}
\end{entry}

\begin{entry}{英文}{8,4}[Radicais ⾋、⽂]
  \begin{phonetics}{英文}{ying1 wen2}[][HSK 2]
    \definition{s.}{inglês, língua inglesa}
  \end{phonetics}
\end{entry}

\begin{entry}{英国}{8,8}[Radicais ⾋、⼞]
  \begin{phonetics}{英国}{ying1guo2}
    \definition*{s.}{Reino Unido}
  \end{phonetics}
\end{entry}

\begin{entry}{英国人}{8,8,2}[Radicais ⾋、⼞、⼈]
  \begin{phonetics}{英国人}{ying1guo2ren2}
    \definition{s.}{inglês | pessoa ou povo do Reino Unido}
  \end{phonetics}
\end{entry}

\begin{entry}{英语}{8,9}[Radicais ⾋、⾔]
  \begin{phonetics}{英语}{ying1 yu3}[][HSK 2]
    \definition{s.}{inglês, língua inglesa}
  \end{phonetics}
\end{entry}

\begin{entry}{英雄}{8,12}[Radicais ⾋、⾫]
  \begin{phonetics}{英雄}{ying1xiong2}
    \definition[个]{s.}{herói}
  \end{phonetics}
\end{entry}

\begin{entry}{苹果}{8,8}[Radicais ⾋、⽊]
  \begin{phonetics}{苹果}{ping2guo3}[][HSK 3]
    \definition[个,颗]{s.}{maçã}
  \end{phonetics}
\end{entry}

\begin{entry}{茄子}{8,3}[Radicais ⾋、⼦]
  \begin{phonetics}{茄子}{qie2zi5}
    \definition{s.}{berinjela chinesa | ``xis'' fonético (ao ser fotografado), equivale ao ``diga xis''}
  \end{phonetics}
\end{entry}

\begin{entry}{虎}{8}[Radical ⾌]
  \begin{phonetics}{虎}{hu3}
    \definition{s.}{tigre}
  \seealsoref{老虎}{lao3hu3}
  \end{phonetics}
\end{entry}

\begin{entry}{虎口}{8,3}[Radicais ⾌、⼝]
  \begin{phonetics}{虎口}{hu3kou3}
    \definition{s.}{lugar perigoso | cova do tigre}
  \end{phonetics}
\end{entry}

\begin{entry}{虎虎}{8,8}[Radicais ⾌、⾌]
  \begin{phonetics}{虎虎}{hu3hu3}
    \definition{adj.}{formidável | forte | vigoroso}
  \end{phonetics}
\end{entry}

\begin{entry}{虎鼬}{8,18}[Radicais ⾌、⿏]
  \begin{phonetics}{虎鼬}{hu3you4}
    \definition{s.}{doninha}
  \end{phonetics}
\end{entry}

\begin{entry}{表}{8}[Radical ⾐]
  \begin{phonetics}{表}{biao3}[][HSK 2]
    \definition*{s.}{sobrenome Biao}
    \definition{s.}{superfície externa | a relação entre os filhos ou netos de um irmão e uma irmã ou de irmãs | exemplo | modelo | memorial a um imperador dos tempos antigos | gráfico | formulário | lista | tabela | medidor | relógio de pulso}
  \end{phonetics}
\end{entry}

\begin{entry}{表白}{8,5}[Radicais ⾐、⽩]
  \begin{phonetics}{表白}{biao3bai2}
    \definition{s.}{declaração | confissão}
    \definition{v.}{confessar a si mesmo | expressar | revelar pensamentos ou sentimentos de alguém}
  \end{phonetics}
\end{entry}

\begin{entry}{表示}{8,5}[Radicais ⾐、⽰]
  \begin{phonetics}{表示}{biao3shi4}[][HSK 2]
    \definition{s.}{expressão | indicação}
    \definition{v.}{expressar | mostrar | indicar | significar}
  \end{phonetics}
\end{entry}

\begin{entry}{表扬}{8,6}[Radicais ⾐、⼿]
  \begin{phonetics}{表扬}{biao3yang2}[][HSK 4]
    \definition[次,种,份]{s.}{elogios públicos por boas ações}
    \definition{v.}{elogiar; louvar}
  \end{phonetics}
\end{entry}

\begin{entry}{表扬信}{8,6,9}[Radicais ⾐、⼿、⼈]
  \begin{phonetics}{表扬信}{biao3yang2 xin4}
    \definition{s.}{carta de elogio | depoimento}
  \end{phonetics}
\end{entry}

\begin{entry}{表达}{8,6}[Radicais ⾐、⾡]
  \begin{phonetics}{表达}{biao3da2}[][HSK 3]
    \definition{v.}{entregar | expressar | mostrar | transmitir | comunicar}
  \end{phonetics}
\end{entry}

\begin{entry}{表明}{8,8}[Radicais ⾐、⽇]
  \begin{phonetics}{表明}{biao3ming2}[][HSK 3]
    \definition{v.}{deixar claro | tornar conhecido | declarar claramente}
  \end{phonetics}
\end{entry}

\begin{entry}{表现}{8,8}[Radicais ⾐、⾒]
  \begin{phonetics}{表现}{biao3xian4}[][HSK 3]
    \definition[个,种,份]{s.}{desempenho | expressão  manifestação | comportamento}
    \definition{v.}{mostrar | expressar | exibir | manifestar | descrever}
  \end{phonetics}
\end{entry}

\begin{entry}{表面}{8,9}[Radicais ⾐、⾯]
  \begin{phonetics}{表面}{biao3mian4}[][HSK 3]
    \definition{s.}{superfície | lado de fora | aparência | superficialidade}
  \end{phonetics}
\end{entry}

\begin{entry}{表格}{8,10}[Radicais ⾐、⽊]
  \begin{phonetics}{表格}{biao3ge2}[][HSK 3]
    \definition[份,张]{s.}{tabela | formulário}
  \end{phonetics}
\end{entry}

\begin{entry}{表情}{8,11}[Radicais ⾐、⼼]
  \begin{phonetics}{表情}{biao3qing2}[][HSK 4]
    \definition[个,种,幅]{s.}{expressão; expressão facial; expressão de pensamentos e sentimentos internos por meio de mudanças faciais ou de gestos}
    \definition{v.}{expressar pensamentos e sentimentos internos por meio de mudanças faciais ou de gestos}
  \end{phonetics}
\end{entry}

\begin{entry}{表演}{8,14}[Radicais ⾐、⽔]
  \begin{phonetics}{表演}{biao3yan3}[][HSK 3]
    \definition[场]{s.}{representação | atuação | exposição}
    \definition{v.}{executar | atuar | jogar | demonstrar | agir | fingir}
  \end{phonetics}
\end{entry}

\begin{entry}{表演艺术}{8,14,4,5}[Radicais ⾐、⽔、⾋、⽊]
  \begin{phonetics}{表演艺术}{biao3yan3 yi4shu4}
    \definition{s.}{arte performática}
  \end{phonetics}
\end{entry}

\begin{entry}{表演者}{8,14,8}[Radicais ⾐、⽔、⽼]
  \begin{phonetics}{表演者}{biao3yan3 zhe3}
    \definition{s.}{ator}
  \end{phonetics}
\end{entry}

\begin{entry}{表演特技}{8,14,10,7}[Radicais ⾐、⽔、⽜、⼿]
  \begin{phonetics}{表演特技}{biao3yan3 te4ji4}
    \definition{s.}{acrobacia | pirueta | façanha}
  \end{phonetics}
\end{entry}

\begin{entry}{表演游戏}{8,14,12,6}[Radicais ⾐、⽔、⽔、⼽]
  \begin{phonetics}{表演游戏}{biao3yan3 you2xi4}
    \definition{s.}{exibição dramática}
  \end{phonetics}
\end{entry}

\begin{entry}{表演赛}{8,14,14}[Radicais ⾐、⽔、⾙]
  \begin{phonetics}{表演赛}{biao3yan3sai4}
    \definition{s.}{partida ou jogo de exibição}
  \end{phonetics}
\end{entry}

\begin{entry}{衬衣}{8,6}[Radicais ⾐、⾐]
  \begin{phonetics}{衬衣}{chen4 yi1}[][HSK 3]
    \definition[件]{s.}{camisa}
  \end{phonetics}
\end{entry}

\begin{entry}{衬衫}{8,8}[Radicais ⾐、⾐]
  \begin{phonetics}{衬衫}{chen4shan1}[][HSK 3]
    \definition[件]{s.}{camisa; blusa}
  \end{phonetics}
\end{entry}

\begin{entry}{规定}{8,8}[Radicais ⾒、⼧]
  \begin{phonetics}{规定}{gui1ding4}[][HSK 3]
    \definition[个,条,项]{s.}{regra; regulamento; estipulação}
    \definition{v.}{estipular; prover; prescrever}
  \end{phonetics}
\end{entry}

\begin{entry}{规范}{8,9}[Radicais ⾒、⾋]
  \begin{phonetics}{规范}{gui1fan4}[][HSK 3]
    \definition{adj.}{regular; normal; padrão; atendendo às especificações}
    \definition{s.}{norma; padrão}
    \definition{v.}{regular; padronizar}
  \end{phonetics}
\end{entry}

\begin{entry}{视角}{8,7}[Radicais ⾒、⾓]
  \begin{phonetics}{视角}{shi4jiao3}
    \definition{s.}{ângulo do qual se observa um objeto | (figurativo) perspectiva, ponto de vista, quadro de referência | (cinematografia) ângulo da câmera | (percepção visual) ângulo visual (o ângulo que um objeto visto subtende no olho) | (fotografia) ângulo de visão}
  \end{phonetics}
\end{entry}

\begin{entry}{视频}{8,13}[Radicais ⾒、⾴]
  \begin{phonetics}{视频}{shi4pin2}
    \definition{s.}{vídeo}
  \end{phonetics}
\end{entry}

\begin{entry}{试}{8}[Radical ⾔]
  \begin{phonetics}{试}{shi4}[][HSK 1]
    \definition{s.}{exame | experimento | prova | teste}
    \definition{v.}{experimentar | provar | testar}
  \end{phonetics}
\end{entry}

\begin{entry}{试验}{8,10}[Radicais ⾔、⾺]
  \begin{phonetics}{试验}{shi4yan4}[][HSK 3]
    \definition{v.}{testar; fazer um teste; fazer um experimento}
  \end{phonetics}
\end{entry}

\begin{entry}{试题}{8,15}[Radicais ⾔、⾴]
  \begin{phonetics}{试题}{shi4 ti2}[][HSK 3]
    \definition[道]{s.}{perguntas de teste; perguntas de exame}
  \end{phonetics}
\end{entry}

\begin{entry}{诗句}{8,5}[Radicais ⾔、⼝]
  \begin{phonetics}{诗句}{shi1ju4}
    \definition[行]{s.}{verso | versículo}
  \end{phonetics}
\end{entry}

\begin{entry}{诗词}{8,7}[Radicais ⾔、⾔]
  \begin{phonetics}{诗词}{shi1ci2}
    \definition{s.}{verso}
  \end{phonetics}
\end{entry}

\begin{entry}{诗意}{8,13}[Radicais ⾔、⼼]
  \begin{phonetics}{诗意}{shi1yi4}
    \definition{adj.}{poético}
    \definition{s.}{poesia}
  \end{phonetics}
\end{entry}

\begin{entry}{诚实}{8,8}[Radicais ⾔、⼧]
  \begin{phonetics}{诚实}{cheng2shi2}[][HSK 4]
    \definition{adj.}{honesto; sincero e honesto, não hipócrita}
  \end{phonetics}
\end{entry}

\begin{entry}{诚实地}{8,8,6}[Radicais ⾔、⼧、⼟]
  \begin{phonetics}{诚实地}{cheng2shi2 di4}
    \definition{adv.}{honestamente}
  \end{phonetics}
\end{entry}

\begin{entry}{诚信}{8,9}[Radicais ⾔、⼈]
  \begin{phonetics}{诚信}{cheng2 xin4}[][HSK 4]
    \definition{adj.}{honesto e confiável}
    \definition[种]{s.}{fé; honestidade; padrão e princípio de comportamento: não contar mentiras, prometer aos outros o que eles podem fazer e ter a confiança dos outros}
  \end{phonetics}
\end{entry}

\begin{entry}{话}{8}[Radical ⾔]
  \begin{phonetics}{话}{hua4}[][HSK 1]
    \definition[种,席,句,口,番]{s.}{fala | linguagem | dialeto}
  \end{phonetics}
\end{entry}

\begin{entry}{话剧}{8,10}[Radicais ⾔、⼑]
  \begin{phonetics}{话剧}{hua4 ju4}[][HSK 3]
    \definition[台,部]{s.}{drama moderno; peça de teatro}
  \end{phonetics}
\end{entry}

\begin{entry}{话题}{8,15}[Radicais ⾔、⾴]
  \begin{phonetics}{话题}{hua4ti2}[][HSK 3]
    \definition[个,种,项]{s.}{assunto de uma palestra; tópico de uma conversa}
  \end{phonetics}
\end{entry}

\begin{entry}{诟骂}{8,9}[Radicais ⾔、⾺]
  \begin{phonetics}{诟骂}{gou4ma4}
    \definition{v.}{abusar verbalmente | insultar | criticar}
  \end{phonetics}
\end{entry}

\begin{entry}{该}{8}[Radical ⾔]
  \begin{phonetics}{该}{gai1}[][HSK 2]
    \definition{v.}{deveria | é a vez de alguém fazer algo | merecer | dever}
  \end{phonetics}
\end{entry}

\begin{entry}{责任}{8,6}[Radicais ⾙、⼈]
  \begin{phonetics}{责任}{ze2ren4}[][HSK 3]
    \definition{s.}{dever; responsabilidade; de acordo com sua ocupação, posição, identidade, etc., as coisas que você deve fazer ou as tarefas que você deve realizar | culpa; responsabilidade por uma falha ou erro; não fazer o que se deve fazer e assumir a culpa}
  \end{phonetics}
\end{entry}

\begin{entry}{责怪}{8,8}[Radicais ⾙、⼼]
  \begin{phonetics}{责怪}{ze2guai4}
    \definition{v.}{repreender | censurar}
  \end{phonetics}
\end{entry}

\begin{entry}{败}{8}[Radical ⾒]
  \begin{phonetics}{败}{bai4}[][HSK 4]
    \definition{adj.}{dilapidado; decadente; murcho; em declínio}
    \definition{v.}{derrota; bater | falhar | quebrar; neutralizar; dissipar | arruinar; estragar; corromper | ser derrotado; perder}
  \end{phonetics}
\end{entry}

\begin{entry}{货车}{8,4}[Radicais ⾙、⾞]
  \begin{phonetics}{货车}{huo4che1}
    \definition{s.}{caminhão | van | vagão de carga}
  \end{phonetics}
\end{entry}

\begin{entry}{贪婪}{8,11}[Radicais ⾙、⼥]
  \begin{phonetics}{贪婪}{tan1lan2}
    \definition{adj.}{avaro | ambicioso | voraz | insaciável}
  \end{phonetics}
\end{entry}

\begin{entry}{贫民窟}{8,5,13}[Radicais ⾙、⽒、⽳]
  \begin{phonetics}{贫民窟}{pin2min2ku1}
    \definition{s.}{favela}
  \end{phonetics}
\end{entry}

\begin{entry}{转}{8}[Radical ⾞]
  \begin{phonetics}{转}{zhuai3}
  \end{phonetics}
  \begin{phonetics}{转}{zhuan3}
    \definition{v.}{mudar; deslocar; transferir; virar; mudar de direção, posição, situação, circunstâncias, etc. | transmitir; transferir; passar adiante}
  \end{phonetics}
  \begin{phonetics}{转}{zhuan4}[][HSK 3]
    \definition{clas.}{para ações repetidas | para rotações (por minuto, etc.): RPM}
    \definition{v.}{rodar; girar; virar; dar voltas | passear; andar por aí}
  \end{phonetics}
\end{entry}

\begin{entry}{转产}{8,6}[Radicais ⾞、⼇]
  \begin{phonetics}{转产}{zhuan3chan3}
    \definition{v.}{mudar a produção | mudar para novos produtos}
  \end{phonetics}
\end{entry}

\begin{entry}{转告}{8,7}[Radicais ⾞、⼝]
  \begin{phonetics}{转告}{zhuan3gao4}
    \definition{v.}{comunicar | transmitir}
  \end{phonetics}
\end{entry}

\begin{entry}{转变}{8,8}[Radicais ⾞、⼜]
  \begin{phonetics}{转变}{zhuan3bian4}[][HSK 3]
    \definition{v.}{mudar; converter; transformar}
  \end{phonetics}
\end{entry}

\begin{entry}{转念}{8,8}[Radicais ⾞、⼼]
  \begin{phonetics}{转念}{zhuan3nian4}
    \definition{v.}{ter dúvidas sobre algo | pensar melhor}
  \end{phonetics}
\end{entry}

\begin{entry}{转账}{8,8}[Radicais ⾞、⾙]
  \begin{phonetics}{转账}{zhuan3zhang4}
    \definition{v.+compl.}{transferir entre contas | trazer à frente | incluir uma soma de dinheiro do balanço anterior no seguinte}
  \end{phonetics}
\end{entry}

\begin{entry}{转递}{8,10}[Radicais ⾞、⾡]
  \begin{phonetics}{转递}{zhuan3di4}
    \definition{v.}{passar | retransmitir}
  \end{phonetics}
\end{entry}

\begin{entry}{转悠}{8,11}[Radicais ⾞、⼼]
  \begin{phonetics}{转悠}{zhuan4you5}
    \definition{v.}{aparecer repetidamente | rolar | passear por aí}
  \end{phonetics}
\end{entry}

\begin{entry}{转游}{8,12}[Radicais ⾞、⽔]
  \begin{phonetics}{转游}{zhuan4you5}
    \variantof{转悠}
  \end{phonetics}
\end{entry}

\begin{entry}{轮回}{8,6}[Radicais ⾞、⼞]
  \begin{phonetics}{轮回}{lun2hui2}
    \definition[个]{s.}{reencarnação (Budismo) | ciclo}
    \definition{v.}{reencarnar}
  \end{phonetics}
\end{entry}

\begin{entry}{软件}{8,6}[Radicais ⾞、⼈]
  \begin{phonetics}{软件}{ruan3jian4}
    \definition{v.}{\emph{software}}
  \end{phonetics}
\end{entry}

\begin{entry}{轰鸣}{8,8}[Radicais ⾞、⿃]
  \begin{phonetics}{轰鸣}{hong1ming2}
    \definition{s.}{bum (som de explosão) | estrondo}
  \end{phonetics}
\end{entry}

\begin{entry}{轰炸机}{8,9,6}[Radicais ⾞、⽕、⽊]
  \begin{phonetics}{轰炸机}{hong1zha4ji1}
    \definition{s.}{bombardeiro (aeronave)}
  \end{phonetics}
\end{entry}

\begin{entry}{郁郁葱葱}{8,8,12,12}[Radicais ⾢、⾢、⾋、⾋]
  \begin{phonetics}{郁郁葱葱}{yu4yu4cong1cong1}
    \definition{expr.}{verdejante e exuberante}
  \end{phonetics}
\end{entry}

\begin{entry}{郊区}{8,4}[Radicais ⾢、⼖]
  \begin{phonetics}{郊区}{jiao1qu1}
    \definition[个]{s.}{subúrbio | distrito suburbano | arredores}
  \end{phonetics}
\end{entry}

\begin{entry}{采用}{8,5}[Radicais ⾤、⽤]
  \begin{phonetics}{采用}{cai3 yong4}[][HSK 3]
    \definition{v.}{selecionar e usar | adotar}
  \end{phonetics}
\end{entry}

\begin{entry}{采访}{8,6}[Radicais ⾤、⾔]
  \begin{phonetics}{采访}{cai3fang3}[][HSK 4]
    \definition{s.}{cobertura; entrevista; coleta de notícias; entrevistas, pesquisas, gravações de áudio e vídeo, etc., com o objetivo de coletar os materiais necessários}
    \definition{v.}{cobrir; entrevistar; reunir novas informações}
  \end{phonetics}
\end{entry}

\begin{entry}{采取}{8,8}[Radicais ⾤、⼜]
  \begin{phonetics}{采取}{cai3qu3}[][HSK 3]
    \definition{v.}{adotar | reunir | coletar | tomar | assumir}
  \end{phonetics}
\end{entry}

\begin{entry}{金}{8}[Kangxi 167][Radical ⾦]
  \begin{phonetics}{金}{jin1}[][HSK 3]
    \definition*{s.}{sobrenome Jin}
    \definition{adj.}{dourado | altamente respeitado; precioso}
    \definition[锭,块]{s.}{ouro | metal | dinheiro | instrumento antigo de percussão de metal | a Dinastia Jin (1115-1234)}
  \end{phonetics}
\end{entry}

\begin{entry}{金刚石}{8,6,5}[Radicais ⾦、⼑、⽯]
  \begin{phonetics}{金刚石}{jin1gang1shi2}
    \definition{s.}{diamante, também chamado de 钻石}
    \seeref{钻石}{zuan4shi2}
  \end{phonetics}
\end{entry}

\begin{entry}{金色}{8,6}[Radicais ⾦、⾊]
  \begin{phonetics}{金色}{jin1 se4}
    \definition{s.}{cor ouro; dourado}
  \end{phonetics}
\end{entry}

\begin{entry}{金牌}{8,12}[Radicais ⾦、⽚]
  \begin{phonetics}{金牌}{jin1 pai2}[][HSK 3]
    \definition[枚]{s.}{medalha de ouro | ficha de ouro}
  \end{phonetics}
\end{entry}

\begin{entry}{金融}{8,16}[Radicais ⾦、⿀]
  \begin{phonetics}{金融}{jin1rong2}
    \definition{s.}{finança}
  \end{phonetics}
\end{entry}

\begin{entry}{钓鱼}{8,8}[Radicais ⾦、⿂]
  \begin{phonetics}{钓鱼}{diao4yu2}
    \definition{v.}{pescar (com linha e anzol) | (figurativo) aprisionar}
  \end{phonetics}
\end{entry}

\begin{entry}{闸门}{8,3}[Radicais ⾨、⾨]
  \begin{phonetics}{闸门}{zha2men2}
    \definition{s.}{eclusa | comporta}
  \end{phonetics}
\end{entry}

\begin{entry}{雨}{8}[Radical ⾬]
  \begin{phonetics}{雨}{yu3}[][HSK 1]
    \definition[阵,场]{s.}{chuva}
  \end{phonetics}
  \begin{phonetics}{雨}{yu4}
    \definition{v.}{cair (chuva, neve, etc.) | precipitar | chover | molhar}
  \end{phonetics}
\end{entry}

\begin{entry}{雨伞}{8,6}[Radicais ⾬、⼈]
  \begin{phonetics}{雨伞}{yu3san3}
    \definition[把]{s.}{guarda-chuva}
  \end{phonetics}
\end{entry}

\begin{entry}{雨衣}{8,6}[Radicais ⾬、⾐]
  \begin{phonetics}{雨衣}{yu3yi1}
    \definition[件]{s.}{impermeável}
  \end{phonetics}
\end{entry}

\begin{entry}{雨蚀}{8,9}[Radicais ⾬、⾷]
  \begin{phonetics}{雨蚀}{yu3shi2}
    \definition{s.}{erosão da chuva}
  \end{phonetics}
\end{entry}

\begin{entry}{雨靴}{8,13}[Radicais ⾬、⾰]
  \begin{phonetics}{雨靴}{yu3xue1}
    \definition[双]{s.}{botas de chuva}
  \end{phonetics}
\end{entry}

\begin{entry}{青天}{8,4}[Radicais ⾭、⼤]
  \begin{phonetics}{青天}{qing1tian1}
    \definition{s.}{céu claro, limpo ou azul}
  \end{phonetics}
\end{entry}

\begin{entry}{青少年}{8,4,6}[Radicais ⾭、⼩、⼲]
  \begin{phonetics}{青少年}{qing1shao4nian2}[][HSK 2]
    \definition[位,个]{s.}{adolescentes}
  \end{phonetics}
\end{entry}

\begin{entry}{青玉米}{8,5,6}[Radicais ⾭、⽟、⽶]
  \begin{phonetics}{青玉米}{qing1yu4mi3}
    \definition{s.}{milho verde}
  \end{phonetics}
\end{entry}

\begin{entry}{青年}{8,6}[Radicais ⾭、⼲]
  \begin{phonetics}{青年}{qing1 nian2}[][HSK 2]
    \definition[个,名,位]{s.}{juventude | jovem}
  \end{phonetics}
\end{entry}

\begin{entry}{青年节}{8,6,5}[Radicais ⾭、⼲、⾋]
  \begin{phonetics}{青年节}{qing1nian2jie2}
    \definition*{s.}{Dia da Juventude (4 de maio)}
  \end{phonetics}
\end{entry}

\begin{entry}{青春}{8,9}[Radicais ⾭、⽇]
  \begin{phonetics}{青春}{qing1chun1}
    \definition{s.}{juventude}
  \end{phonetics}
\end{entry}

\begin{entry}{青菜}{8,11}[Radicais ⾭、⾋]
  \begin{phonetics}{青菜}{qing1cai4}
    \definition{s.}{verduras}
  \end{phonetics}
\end{entry}

\begin{entry}{青铜}{8,11}[Radicais ⾭、⾦]
  \begin{phonetics}{青铜}{qing1tong2}
    \definition{s.}{bronze (liga de cobre, 銅, e estanho, 锡)}
  \end{phonetics}
\end{entry}

\begin{entry}{青椒}{8,12}[Radicais ⾭、⽊]
  \begin{phonetics}{青椒}{qing1jiao1}
    \definition{s.}{pimenta verde}
  \end{phonetics}
\end{entry}

\begin{entry}{青蛙}{8,12}[Radicais ⾭、⾍]
  \begin{phonetics}{青蛙}{qing1wa1}
    \definition{adj.}{(gíria velha) cara feio}
    \definition[只]{s.}{sapo}
  \end{phonetics}
\end{entry}

\begin{entry}{非}{8}[Kangxi 175][Radical ⾮]
  \begin{phonetics}{非}{fei1}
    \definition*{s.}{África, abreviação de 非洲}
    \definition{adv.}{não ser | não é | não}
  \seealsoref{非洲}{fei1zhou1}
  \end{phonetics}
\end{entry}

\begin{entry}{非洲}{8,9}[Radicais ⾮、⽔]
  \begin{phonetics}{非洲}{fei1zhou1}
    \definition*{s.}{África}
  \end{phonetics}
\end{entry}

\begin{entry}{非洲人}{8,9,2}[Radicais ⾮、⽔、⼈]
  \begin{phonetics}{非洲人}{fei1zhou1ren2}
    \definition{s.}{africano | pessoa ou povo da África}
  \end{phonetics}
\end{entry}

\begin{entry}{非常}{8,11}[Radicais ⾮、⼱]
  \begin{phonetics}{非常}{fei1chang2}[][HSK 1]
    \definition{adv.}{extraordinário | altamente | muito}
  \end{phonetics}
\end{entry}

\begin{entry}{靣}{8}[Radical ⼀]
  \begin{phonetics}{靣}{mian4}
    \variantof{面}
  \end{phonetics}
\end{entry}

\begin{entry}{顶}{8}[Radical ⾴]
  \begin{phonetics}{顶}{ding3}[][HSK 4]
    \definition{adv.}{muito (linguagem falada); a maioria; extremamente; expressa o grau mais alto, equivalente a ``最'' e ``极''}
    \definition{clas.}{para coisas que têm um topo}
    \definition{prep.}{até}
    \definition{s.}{coroa da cabeça; parte mais alta do corpo ou objeto | topo; limite superior; ponto mais alto}
    \definition{v.}{carregar na cabeça; carregar em sua cabeça | empurrar (ou apoiar) para cima; empurrar por baixo (ou por trás) |
dar cabeçadas; dar uma coronhada | sustentar; apoiar; suportar | resistir; ir contra; enfrentar | rebater; retorquir; responder de volta | cooperar; enfrentar; apoiar; dar suporte | igualar; ser equivalente a | substituir; tomar o lugar de | assumir o controle; transferir ou adquirir o direito de administrar um negócio ou alugar uma casa ou terreno}
  \seealsoref{极}{ji2}
  \seealsoref{最}{zui4}
  \end{phonetics}
\end{entry}

\begin{entry}{饱}{8}[Radical ⾷]
  \begin{phonetics}{饱}{bao3}[][HSK 2]
    \definition{adj.}{ter comido até ficar satisfeito | estar cheio | cheio}
    \definition{adv.}{completamente | até estar cheio}
    \definition{v.}{satisfazer}
  \end{phonetics}
\end{entry}

\begin{entry}{驻军}{8,6}[Radicais ⾺、⼍]
  \begin{phonetics}{驻军}{zhu4jun1}
    \definition{s.}{guarnição}
    \definition{v.}{guarcener ou prover uma tropa}
  \end{phonetics}
\end{entry}

\begin{entry}{驾照}{8,13}[Radicais ⾺、⽕]
  \begin{phonetics}{驾照}{jia4zhao4}
    \definition{s.}{carteira de motorista}
  \end{phonetics}
\end{entry}

\begin{entry}{鱼}{8}[Radical ⿂]
  \begin{phonetics}{鱼}{yu2}[][HSK 2]
    \definition*{s.}{sobrenome Yu}
    \definition[条,尾]{s.}{peixe}
  \end{phonetics}
\end{entry}

\begin{entry}{鱼片}{8,4}[Radicais ⿂、⽚]
  \begin{phonetics}{鱼片}{yu2pian4}
    \definition{s.}{fatia de peixe | filé de peixe}
  \end{phonetics}
\end{entry}

\begin{entry}{鱼汛}{8,6}[Radicais ⿂、⽔]
  \begin{phonetics}{鱼汛}{yu2xun4}
    \variantof{渔汛}
  \end{phonetics}
\end{entry}

\begin{entry}{鱼网}{8,6}[Radicais ⿂、⽹]
  \begin{phonetics}{鱼网}{yu2wang3}
    \variantof{渔网}
  \end{phonetics}
\end{entry}

\begin{entry}{鱼具}{8,8}[Radicais ⿂、⼋]
  \begin{phonetics}{鱼具}{yu2ju4}
    \variantof{渔具}
  \end{phonetics}
\end{entry}

\begin{entry}{鱼香}{8,9}[Radicais ⿂、⾹]
  \begin{phonetics}{鱼香}{yu2xiang1}
    \definition{s.}{um tempero da culinária chinesa que normalmente contém alho, cebolinha, gengibre, açúcar, sal, pimenta, etc. (Embora 鱼香 signifique literalmente ``fragrância de peixe'', não contém frutos do mar.)}
  \end{phonetics}
\end{entry}

\begin{entry}{鱼香肉丝}{8,9,6,5}[Radicais ⿂、⾹、⾁、⼀]
  \begin{phonetics}{鱼香肉丝}{yu2xiang1rou4si1}
    \definition{s.}{tiras de carne de porco salteadas com molho picante (prato)}
  \seealsoref{鱼香}{yu2xiang1}
  \end{phonetics}
\end{entry}

\begin{entry}{鱼船}{8,11}[Radicais ⿂、⾈]
  \begin{phonetics}{鱼船}{yu2chuan2}
    \definition{s.}{barco de pesca}
  \seealsoref{渔船}{yu2chuan2}
  \end{phonetics}
\end{entry}

\begin{entry}{鸣}{8}[Radical ⿃]
  \begin{phonetics}{鸣}{ming2}
    \definition{v.}{chorar (pássaros, animais e insetos) | fazer um som | dar voz (gratidão, queixas, etc.)}
  \end{phonetics}
\end{entry}

%%%%% EOF %%%%%

