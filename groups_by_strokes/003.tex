%%%
%%% 3画
%%%

\section*{3画}\addcontentsline{toc}{section}{3画}

\begin{entry}{万}{3}[Radical 一]
  \begin{phonetics}{万}{wan4}[][HSK 2]
    \definition*{s.}{sobrenome Wan}
    \definition{adj.}{um grande número}
    \definition{num.}{dez mil; 10.000; 1.0000}
  \end{phonetics}
\end{entry}

\begin{entry}{万万}{3,3}
  \begin{phonetics}{万万}{wan4wan4}
    \definition{adv.}{absolutamente | totalmente}
  \end{phonetics}
\end{entry}

\begin{entry}{万圣节}{3,5,5}
  \begin{phonetics}{万圣节}{wan4sheng4jie2}
    \definition*{s.}{Dia de Todos os Santos}
  \seealsoref{万圣节前夕}{wan4sheng4jie2qian2xi1}
  \end{phonetics}
\end{entry}

\begin{entry}{万圣节前夕}{3,5,5,9,3}
  \begin{phonetics}{万圣节前夕}{wan4sheng4jie2qian2xi1}
    \definition*{s.}{Véspera do Dia de Todos os Santos | \emph{Halloween}}
  \seealsoref{万圣节}{wan4sheng4jie2}
  \end{phonetics}
\end{entry}

\begin{entry}{三}{3}[Radical 一]
  \begin{phonetics}{三}{san1}[][HSK 1]
    \definition*{s.}{sobrenome San}
    \definition{num.}{três; 3}
  \end{phonetics}
\end{entry}

\begin{entry}{三角}{3,7}
  \begin{phonetics}{三角}{san1jiao3}
    \definition{s.}{triângulo}
  \end{phonetics}
\end{entry}

\begin{entry}{三角恋爱}{3,7,10,10}
  \begin{phonetics}{三角恋爱}{san1jiao3lian4'ai4}
    \definition{s.}{triângulo amoroso}
  \end{phonetics}
\end{entry}

\begin{entry}{三明治}{3,8,8}
  \begin{phonetics}{三明治}{san1ming2zhi4}
    \definition{s.}{(empréstimo linguístico) sanduíche}
  \end{phonetics}
\end{entry}

\begin{entry}{三轮车}{3,8,4}
  \begin{phonetics}{三轮车}{san1lun2che1}
    \definition{s.}{triciclo}
  \end{phonetics}
\end{entry}

\begin{entry}{上}{3}[Radical 一]
  \begin{phonetics}{上}{shang4}[][HSK 1]
    \definition{adv.}{acima | em cima | sobre}
    \definition{v.}{subir | entrar em | frequentar (aula ou universidade)}
  \end{phonetics}
\end{entry}

\begin{entry}{上午}{3,4}
  \begin{phonetics}{上午}{shang4wu3}[][HSK 1]
    \definition{adv.}{manhã | de manhã | período antes do meio-dia}
  \end{phonetics}
\end{entry}

\begin{entry}{上车}{3,4}
  \begin{phonetics}{上车}{shang4 che1}[][HSK 1]
    \definition{v.}{entrar (em ônibus, trem, carro, etc.)}
  \end{phonetics}
\end{entry}

\begin{entry}{上去}{3,5}
  \begin{phonetics}{上去}{shang4qu4}
    \definition{v.}{subir (a partir da minha localização)}
  \end{phonetics}
\end{entry}

\begin{entry}{上古}{3,5}
  \begin{phonetics}{上古}{shang4gu3}
    \definition{s.}{o passado distante | tempos antigos | antiguidade}
  \end{phonetics}
\end{entry}

\begin{entry}{上边}{3,5}
  \begin{phonetics}{上边}{shang4bian5}[][HSK 1]
    \definition{adv.}{acima de | parte de cima | por cima}
  \end{phonetics}
\end{entry}

\begin{entry}{上当}{3,6}
  \begin{phonetics}{上当}{shang4dang4}
    \definition{v.+compl.}{ser enganado | morder uma isca | ser manipulado | ser joguete nas mãos de alguém}
  \end{phonetics}
\end{entry}

\begin{entry}{上次}{3,6}
  \begin{phonetics}{上次}{shang4 ci4}[][HSK 1]
    \definition{adv.}{última vez}
  \end{phonetics}
\end{entry}

\begin{entry}{上网}{3,6}
  \begin{phonetics}{上网}{shang4 wang3}[][HSK 1]
    \definition{v.}{conectar à \emph{Internet} | fazer \emph{upload} | ficar \emph{online}}
  \end{phonetics}
\end{entry}

\begin{entry}{上访}{3,6}
  \begin{phonetics}{上访}{shang4fang3}
    \definition{v.}{buscar uma audiência com superiores (especialmente funcionários do governo) para fazer uma petição por algo}
  \end{phonetics}
\end{entry}

\begin{entry}{上声}{3,7}
  \begin{phonetics}{上声}{shang3sheng1}
    \definition{s.}{tom descendente e ascendente | terceiro tom no mandarim moderno}
  \end{phonetics}
\end{entry}

\begin{entry}{上来}{3,7}
  \begin{phonetics}{上来}{shang4lai2}
    \definition{v.}{subir (para a minha localização)}
  \end{phonetics}
\end{entry}

\begin{entry}{上周}{3,8}
  \begin{phonetics}{上周}{shang4 zhou1}[][HSK 2]
    \definition{s.}{semana passada}
  \end{phonetics}
\end{entry}

\begin{entry}{上坡路}{3,8,13}
  \begin{phonetics}{上坡路}{shang4po1lu4}
    \definition{s.}{aclive | progresso | (fig.) tendência ascendente}
  \end{phonetics}
\end{entry}

\begin{entry}{上学}{3,8}
  \begin{phonetics}{上学}{shang4 xue2}[][HSK 1]
    \definition{v.}{ir à escola | frequentar a escola | estar na escola | iniciar as aulas}
  \end{phonetics}
\end{entry}

\begin{entry}{上询}{3,8}
  \begin{phonetics}{上询}{shang4 xun2}
    \definition{adv.}{primeira dezena do mês}
  \end{phonetics}
\end{entry}

\begin{entry}{上面}{3,9}
  \begin{phonetics}{上面}{shang4mian4}
    \definition{adv.}{acima de | parte de cima | por cima}
  \end{phonetics}
\end{entry}

\begin{entry}{上海}{3,10}
  \begin{phonetics}{上海}{shang4hai3}
    \definition*{s.}{Shangai (Xangai)}
  \end{phonetics}
\end{entry}

\begin{entry}{上班}{3,10}
  \begin{phonetics}{上班}{shang4 ban1}[][HSK 1]
    \definition{v.+compl.}{ir para o trabalho | ir para o emprego | estar de plantão}
  \end{phonetics}
\end{entry}

\begin{entry}{上课}{3,10}
  \begin{phonetics}{上课}{shang4 ke4}[][HSK 1]
    \definition{v.}{assistir à aula | ir para a aula | ir dar uma aula}
  \end{phonetics}
\end{entry}

\begin{entry}{上演}{3,14}
  \begin{phonetics}{上演}{shang4yan3}
    \definition{s.}{exibição | encenação}
    \definition{v.}{exibir (um filme) | encenar (uma peça)}
  \end{phonetics}
\end{entry}

\begin{entry}{下}{3}[Radical 一]
  \begin{phonetics}{下}{xia4}[][HSK 1]
    \definition{adv.}{abaixo | em baixo de}
    \definition{clas.}{para número de vezes para ações}
    \definition{v.}{descer | chegar a (uma decisão, conclusão, etc.) | recusar}
  \end{phonetics}
\end{entry}

\begin{entry}{下午}{3,4}
  \begin{phonetics}{下午}{xia4wu3}[][HSK 1]
    \definition{adv.}{tarde | à tarde | período logo após o meio-dia}
  \end{phonetics}
\end{entry}

\begin{entry}{下午茶}{3,4,9}
  \begin{phonetics}{下午茶}{xia4wu3cha2}
    \definition{s.}{chá da tarde (normalmente chás com doces)}
  \end{phonetics}
\end{entry}

\begin{entry}{下巴}{3,4}
  \begin{phonetics}{下巴}{xia4ba5}
    \definition[个]{s.}{queixo}
  \end{phonetics}
\end{entry}

\begin{entry}{下水道}{3,4,12}
  \begin{phonetics}{下水道}{xia4shui3dao4}
    \definition{s.}{esgoto}
  \end{phonetics}
\end{entry}

\begin{entry}{下车}{3,4}
  \begin{phonetics}{下车}{xia4 che1}[][HSK 1]
    \definition{v.}{descer ou sair (de ônibus, carro, etc.)}
  \end{phonetics}
\end{entry}

\begin{entry}{下去}{3,5}
  \begin{phonetics}{下去}{xia4qu4}
    \definition{v.}{descer (a partir da minha localização)}
  \end{phonetics}
\end{entry}

\begin{entry}{下边}{3,5}
  \begin{phonetics}{下边}{xia4bian5}[][HSK 1]
    \definition{adv.}{em baixo | abaixo | parte de baixo}
  \end{phonetics}
\end{entry}

\begin{entry}{下旬}{3,6}
  \begin{phonetics}{下旬}{xia4xun2}
    \definition{adv.}{última dezena do mês}
  \end{phonetics}
\end{entry}

\begin{entry}{下次}{3,6}
  \begin{phonetics}{下次}{xia4 ci4}[][HSK 1]
    \definition{s.}{próxima vez}
  \end{phonetics}
\end{entry}

\begin{entry}{下来}{3,7}
  \begin{phonetics}{下来}{xia4lai5}
    \definition{v.}{descer (para a minha localização)}
  \end{phonetics}
\end{entry}

\begin{entry}{下周}{3,8}
  \begin{phonetics}{下周}{xia4 zhou1}[][HSK 2]
    \definition{s.}{próxima semana}
  \end{phonetics}
\end{entry}

\begin{entry}{下线}{3,8}
  \begin{phonetics}{下线}{xia4xian4}
    \definition{v.}{ficar \emph{offline} | (um produto) sair da linha de montagem | pessoa abaixo de si em um esquema de pirâmide}
  \end{phonetics}
\end{entry}

\begin{entry}{下雨}{3,8}
  \begin{phonetics}{下雨}{xia4 yu3}[][HSK 1]
    \definition{v.+compl.}{chover}
  \end{phonetics}
\end{entry}

\begin{entry}{下面}{3,9}
  \begin{phonetics}{下面}{xia4mian4}
    \definition{adv.}{em baixo | abaixo | parte de baixo}
    \definition{v.}{cozinhar macarrão}
  \end{phonetics}
\end{entry}

\begin{entry}{下海}{3,10}
  \begin{phonetics}{下海}{xia4hai3}
    \definition{v.+compl.}{ir para o mar; (barco) deixar o porto e iniciar uma jornada | ir pescar no mar | tornar-se ator profissional}
  \end{phonetics}
\end{entry}

\begin{entry}{下班}{3,10}
  \begin{phonetics}{下班}{xia4 ban1}[][HSK 1]
    \definition{v.+compl.}{sair do trabalho}
  \end{phonetics}
\end{entry}

\begin{entry}{下课}{3,10}
  \begin{phonetics}{下课}{xia4 ke4}[][HSK 1]
    \definition{v.+compl.}{acabar a aula | terminar a aula}
  \end{phonetics}
\end{entry}

\begin{entry}{下载}{3,10}
  \begin{phonetics}{下载}{xia4zai3}
    \definition{v.}{baixar | \emph{download}}
  \end{phonetics}
\end{entry}

\begin{entry}{下蛋}{3,11}
  \begin{phonetics}{下蛋}{xia4dan4}
    \definition{v.}{botar ovos}
  \end{phonetics}
\end{entry}

\begin{entry}{下雪}{3,11}
  \begin{phonetics}{下雪}{xia4 xue3}[][HSK 2]
    \definition[场,次]{s.}{neve}
    \definition{v.+compl.}{nevar}
  \end{phonetics}
\end{entry}

\begin{entry}{下崽}{3,12}
  \begin{phonetics}{下崽}{xia4zai3}
    \definition{v.}{dar à luz (animais) | parir}
  \end{phonetics}
\end{entry}

\begin{entry}{与}{3}[Radical 一]
  \begin{phonetics}{与}{yu3}
    \definition{conj.}{e, com}
  \end{phonetics}
  \begin{phonetics}{与}{yu4}
    \definition{v.}{fazer parte de}
  \end{phonetics}
\end{entry}

\begin{entry}{与其}{3,8}
  \begin{phonetics}{与其}{yu3qi2}
    \definition{conj.}{mais do que}
  \end{phonetics}
\end{entry}

\begin{entry}{与其……不如……}{3,8,4,6}
  \begin{phonetics}{与其……不如……}{yu3qi2 bu4ru2}
    \definition{conj.}{ao invés de\dots melhor que\dots}
  \end{phonetics}
\end{entry}

\begin{entry}{与其……宁可……}{3,8,5,5}
  \begin{phonetics}{与其……宁可……}{yu3qi2 ning4ke3}
    \definition{conj.}{ao invés de\dots melhor que\dots}
  \end{phonetics}
\end{entry}

\begin{entry}{个}{3}[Radical 人]
  \begin{phonetics}{个}{ge3}
    \definition{pron.}{usado em 自个儿}
    \seeref{自个儿}{zi4ge3r5}
  \end{phonetics}
  \begin{phonetics}{个}{ge4}
    \definition{clas.}{para objetos e pessoas em geral}
    \definition{pron.}{isto | aquilo}
    \definition{s.}{indivíduo | tamanho}
  \end{phonetics}
\end{entry}

\begin{entry}{个子}{3,3}
  \begin{phonetics}{个子}{ge4zi5}[][HSK 2]
    \definition{s.}{altura | estatura}
  \end{phonetics}
\end{entry}

\begin{entry}{个性}{3,8}
  \begin{phonetics}{个性}{ge4xing4}
    \definition{s.}{individualidade | personalidade}
  \end{phonetics}
\end{entry}

\begin{entry}{久}{3}[Radical 丿]
  \begin{phonetics}{久}{jiu3}
    \definition{adj.}{por muito tempo}
    \definition{s.}{de uma duração específica}
  \end{phonetics}
\end{entry}

\begin{entry}{之外}{3,5}
  \begin{phonetics}{之外}{zhi1wai4}
    \definition{adv.}{lado de fora}
  \end{phonetics}
\end{entry}

\begin{entry}{也}{3}[Radical 乙]
  \begin{phonetics}{也}{ye3}[][HSK 1]
    \definition*{s.}{sobrenome Ye}
    \definition{adv.}{também | (em frases negativas) nem, tampouco}
  \end{phonetics}
\end{entry}

\begin{entry}{也有今天}{3,6,4,4}
  \begin{phonetics}{也有今天}{ye3you3jin1tian1}
    \definition{expr.}{obter apenas o que merece | todo cachorro tem seu dia | obter a sua parte (coisas boas ou ruins) | servir alguém bem}
  \end{phonetics}
\end{entry}

\begin{entry}{也许}{3,6}
  \begin{phonetics}{也许}{ye3xu3}[][HSK 2]
    \definition{adv.}{possivelmente | talvez}
  \end{phonetics}
\end{entry}

\begin{entry}{也就是}{3,12,9}
  \begin{phonetics}{也就是}{ye3jiu4shi4}
    \definition{adv.}{i.e., isso é | ou seja}
  \end{phonetics}
\end{entry}

\begin{entry}{也就是说}{3,12,9,9}
  \begin{phonetics}{也就是说}{ye3jiu4shi4shuo1}
    \definition{adv.}{em outras palavras | então | isto é | por isso}
  \end{phonetics}
\end{entry}

\begin{entry}{习惯}{3,11}
  \begin{phonetics}{习惯}{xi2guan4}[][HSK 2]
    \definition[个]{s.}{hábito | costume | prática usual}
    \definition{v.}{ser acostumado a | ter o hábito de}
  \end{phonetics}
\end{entry}

\begin{entry}{乡巴佬}{3,4,8}
  \begin{phonetics}{乡巴佬}{xiang1ba1lao3}
    \definition{s.}{aldeão | caipira}
  \end{phonetics}
\end{entry}

\begin{entry}{乡村}{3,7}
  \begin{phonetics}{乡村}{xiang1cun1}
    \definition{adj.}{rural | rústico}
    \definition{s.}{vila | campo}
  \end{phonetics}
\end{entry}

\begin{entry}{于是}{3,9}
  \begin{phonetics}{于是}{yu2shi4}
    \definition{conj.}{então | portanto | é por isso}
  \end{phonetics}
\end{entry}

\begin{entry}{亿}{3}[Radical 人]
  \begin{phonetics}{亿}{yi4}[][HSK 2]
    \definition{num.}{cem milhões; 100.000.000; 1.0000.0000}
  \end{phonetics}
\end{entry}

\begin{entry}{千}{3}[Radical 十]
  \begin{phonetics}{千}{qian1}[][HSK 2]
    \definition{num.}{mil; 1.000; 1000}
  \end{phonetics}
\end{entry}

\begin{entry}{千万}{3,3}
  \begin{phonetics}{千万}{qian1wan4}
    \definition{adv.}{absolutamente | por todos os meios | (quando usado negativamente) sob nehuma circunstância | nunca | pelo amor de Deus | por favor | não}
  \end{phonetics}
\end{entry}

\begin{entry}{千千万万}{3,3,3,3}
  \begin{phonetics}{千千万万}{qian1qian1wan4wan4}
    \definition{num.}{inumerável | números incontáveis | milhares e milhares}
  \end{phonetics}
\end{entry}

\begin{entry}{千古}{3,5}
  \begin{phonetics}{千古}{qian1gu3}
    \definition{adv.}{por toda a eternidade | em todas as idades}
    \definition{s.}{eternidade (usada em um dístico elegíaco, coroa de flores, etc., dedicada aos mortos)}
  \end{phonetics}
\end{entry}

\begin{entry}{千年}{3,6}
  \begin{phonetics}{千年}{qian1nian2}
    \definition{s.}{milênio}
  \end{phonetics}
\end{entry}

\begin{entry}{千克}{3,7}
  \begin{phonetics}{千克}{qian1 ke4}[][HSK 2]
    \definition{clas.}{kg | quilo | quilograma}
  \end{phonetics}
\end{entry}

\begin{entry}{卫生}{3,5}
  \begin{phonetics}{卫生}{wei4sheng1}
    \definition{s.}{saúde | higiene | saneamento}
  \end{phonetics}
\end{entry}

\begin{entry}{卫生巾}{3,5,3}
  \begin{phonetics}{卫生巾}{wei4sheng1jin1}
    \definition{s.}{absorvente higiênico}
  \end{phonetics}
\end{entry}

\begin{entry}{卫生厅}{3,5,4}
  \begin{phonetics}{卫生厅}{wei4sheng1ting1}
    \definition*{s.}{Departamento de Saúde (da província)}
  \end{phonetics}
\end{entry}

\begin{entry}{卫生防疫}{3,5,6,9}
  \begin{phonetics}{卫生防疫}{wei4sheng1 fang2yi4}
    \definition{s.}{prevenção contra a epidemia}
  \end{phonetics}
\end{entry}

\begin{entry}{卫生局}{3,5,7}
  \begin{phonetics}{卫生局}{wei4sheng1ju2}
    \definition*{s.}{Departamento de Saúde | Escritório de Saúde}
  \end{phonetics}
\end{entry}

\begin{entry}{卫生纸}{3,5,7}
  \begin{phonetics}{卫生纸}{wei4sheng1zhi3}
    \definition{s.}{papel higiênico}
  \end{phonetics}
\end{entry}

\begin{entry}{卫生间}{3,5,7}
  \begin{phonetics}{卫生间}{wei4sheng1jian1}
    \definition[间]{s.}{banheiro | \emph{toilette}}
  \end{phonetics}
\end{entry}

\begin{entry}{卫生套}{3,5,10}
  \begin{phonetics}{卫生套}{wei4sheng1tao4}
    \definition[只]{s.}{preservativo | camisinha}
  \end{phonetics}
\end{entry}

\begin{entry}{卫生部}{3,5,10}
  \begin{phonetics}{卫生部}{wei4sheng1bu4}
    \definition*{s.}{Ministério da Saúde}
  \end{phonetics}
\end{entry}

\begin{entry}{卫生球}{3,5,11}
  \begin{phonetics}{卫生球}{wei4sheng1qiu2}
    \definition{s.}{naftalina}
  \end{phonetics}
\end{entry}

\begin{entry}{卫生棉}{3,5,12}
  \begin{phonetics}{卫生棉}{wei4sheng1mian2}
    \definition{s.}{absorvente | algodão absorvente esterilizado (usado para curativos ou limpeza de feridas) | absorvente tampão}
  \end{phonetics}
\end{entry}

\begin{entry}{卫生署}{3,5,13}
  \begin{phonetics}{卫生署}{wei4sheng1shu3}
    \definition*{s.}{Agência de Saúde (ou Escritório, ou Departamento)}
  \end{phonetics}
\end{entry}

\begin{entry}{及}{3}[Radical 又]
  \begin{phonetics}{及}{ji2}
    \definition{conj.}{e | bem como}
  \end{phonetics}
\end{entry}

\begin{entry}{及格}{3,10}
  \begin{phonetics}{及格}{ji2ge2}
    \definition{v.}{atender a um padrão mínimo | passar em um exame ou teste}
  \end{phonetics}
\end{entry}

\begin{entry}{口}{3}[Radical 口][Kangxi 30]
  \begin{phonetics}{口}{kou3}[][HSK 1]
    \definition{clas.}{para coisas com bocas (pessoas, animais domésticos, canhões, etc.) | para mordidas ou bocados}
    \definition{s.}{boca}
  \end{phonetics}
\end{entry}

\begin{entry}{口语}{3,9}
  \begin{phonetics}{口语}{kou3yu3}
    \definition[门]{s.}{linguagem oral | linguagem falada | fofoca | calúnia}
  \end{phonetics}
\end{entry}

\begin{entry}{口音}{3,9}
  \begin{phonetics}{口音}{kou3yin1}
    \definition{s.}{sons da fala oral (linguística)}
  \end{phonetics}
  \begin{phonetics}{口音}{kou3yin5}
    \definition{s.}{sotaque | voz}
  \end{phonetics}
\end{entry}

\begin{entry}{口香糖}{3,9,16}
  \begin{phonetics}{口香糖}{kou3xiang1tang2}
    \definition{s.}{goma de mascar | chiclete}
  \end{phonetics}
\end{entry}

\begin{entry}{口袋}{3,11}
  \begin{phonetics}{口袋}{kou3dai4}
    \definition{s.}{bolso | saco}
  \end{phonetics}
\end{entry}

\begin{entry}{口袋妖怪}{3,11,7,8}
  \begin{phonetics}{口袋妖怪}{kou3dai4 yao1guai4}
    \definition*{s.}{\emph{Pokémon}}
  \end{phonetics}
\end{entry}

\begin{entry}{土地}{3,6}
  \begin{phonetics}{土地}{tu3di4}
    \definition[片,块]{s.}{terra | solo | território}
  \end{phonetics}
  \begin{phonetics}{土地}{tu3di5}
    \definition{s.}{deus local | \emph{genius loci} deidade protetora de um local}
  \end{phonetics}
\end{entry}

\begin{entry}{土豆}{3,7}
  \begin{phonetics}{土豆}{tu3dou4}
    \definition[个,颗]{s.}{batata}
  \end{phonetics}
\end{entry}

\begin{entry}{土豆泥}{3,7,8}
  \begin{phonetics}{土豆泥}{tu3dou4ni2}
    \definition{s.}{purê de batatas}
  \end{phonetics}
\end{entry}

\begin{entry}{土鸡}{3,7}
  \begin{phonetics}{土鸡}{tu3ji1}
    \definition{s.}{galinha caipira}
  \end{phonetics}
\end{entry}

\begin{entry}{夕阳}{3,6}
  \begin{phonetics}{夕阳}{xi1yang2}
    \definition{s.}{pôr do sol}
  \seealsoref{日出}{ri4chu1}
  \end{phonetics}
\end{entry}

\begin{entry}{大}{3}[Radical ⼈]
  \begin{phonetics}{大}{da4}
    \definition{adj.}{grande | enorme | maior | largo | profundo | mais velho (que) | mais antigo | mais velho | muito}
    \definition{s.}{(dialeto) pai | irmão mais velho ou mais novo do pai}
  \end{phonetics}
  \begin{phonetics}{大}{dai4}
    \definition{s.}{usado em 大夫 \dpy{dai4fu5}: médico, doutor}
    \seeref{大夫}{dai4fu5}
  \end{phonetics}
\end{entry}

\begin{entry}{大人}{3,2}
  \begin{phonetics}{大人}{da4 ren2}[][HSK 2]
    \definition{s.}{adulto}
  \end{phonetics}
\end{entry}

\begin{entry}{大口}{3,3}
  \begin{phonetics}{大口}{da4kou3}
    \definition{s.}{grande bocado (de comida, bebida, fumo, etc.)}
  \end{phonetics}
\end{entry}

\begin{entry}{大大}{3,3}
  \begin{phonetics}{大大}{da4 da4}[][HSK 2]
    \definition{adv.}{muito; enormemente}
  \end{phonetics}
\end{entry}

\begin{entry}{大小}{3,3}
  \begin{phonetics}{大小}{da4 xiao3}[][HSK 2]
    \definition{adv.}{no mínimo}
    \definition[家]{s.}{tamanho | grau de antiguidade | adultos e crianças | grande ou pequeno}
  \end{phonetics}
\end{entry}

\begin{entry}{大门}{3,3}
  \begin{phonetics}{大门}{da4 men2}[][HSK 2]
    \definition{s.}{portão | entrada}
  \end{phonetics}
\end{entry}

\begin{entry}{大马}{3,3}
  \begin{phonetics}{大马}{da4ma3}
    \definition*{s.}{Malásia}
  \end{phonetics}
\end{entry}

\begin{entry}{大夫}{3,4}
  \begin{phonetics}{大夫}{da4fu1}
    \definition{s.}{oficial sênior (na China Imperial)}
  \end{phonetics}
  \begin{phonetics}{大夫}{dai4fu5}
    \definition{s.}{médico, doutor}
  \end{phonetics}
\end{entry}

\begin{entry}{大全}{3,6}
  \begin{phonetics}{大全}{da4quan2}
    \definition{s.}{coleção abrangente}
  \end{phonetics}
\end{entry}

\begin{entry}{大后天}{3,6,4}
  \begin{phonetics}{大后天}{da4hou4tian1}
    \definition{adv.}{daqui a três dias}
  \end{phonetics}
\end{entry}

\begin{entry}{大多数}{3,6,13}
  \begin{phonetics}{大多数}{da4 duo1 shu4}[][HSK 2]
    \definition{s.}{a grande maioria | a vasta maioria | a maior parte}
  \end{phonetics}
\end{entry}

\begin{entry}{大戏}{3,6}
  \begin{phonetics}{大戏}{da4xi4}
    \definition*{s.}{Drama, Ópera Chinesa}
  \end{phonetics}
\end{entry}

\begin{entry}{大约}{3,6}
  \begin{phonetics}{大约}{da4yue1}
    \definition{adv.}{aproximadamente | provavelmente}
  \end{phonetics}
\end{entry}

\begin{entry}{大自然}{3,6,12}
  \begin{phonetics}{大自然}{da4 zi4 ran2}[][HSK 2]
    \definition{s.}{natureza}
  \end{phonetics}
\end{entry}

\begin{entry}{大衣}{3,6}
  \begin{phonetics}{大衣}{da4 yi1}[][HSK 2]
    \definition{s.}{sobretudo}
  \end{phonetics}
\end{entry}

\begin{entry}{大声}{3,7}
  \begin{phonetics}{大声}{da4 sheng1}[][HSK 2]
    \definition{adj.}{alto volume | em voz alta}
  \end{phonetics}
\end{entry}

\begin{entry}{大豆}{3,7}
  \begin{phonetics}{大豆}{da4dou4}
    \definition{s.}{soja}
  \end{phonetics}
\end{entry}

\begin{entry}{大学}{3,8}
  \begin{phonetics}{大学}{da4xue2}[][HSK 1]
    \definition[所]{s.}{faculdade | universidade}
  \end{phonetics}
\end{entry}

\begin{entry}{大学生}{3,8,5}
  \begin{phonetics}{大学生}{da4xue2sheng1}[][HSK 1]
    \definition{s.}{estudante universitário}
  \end{phonetics}
\end{entry}

\begin{entry}{大规模}{3,8,14}
  \begin{phonetics}{大规模}{da4gui1mo2}
    \definition{adj.}{em grande escala | em larga escala | extenso}
  \end{phonetics}
\end{entry}

\begin{entry}{大雨}{3,8}
  \begin{phonetics}{大雨}{da4yu3}
    \definition[场]{s.}{chuva pesada, forte}
  \end{phonetics}
\end{entry}

\begin{entry}{大前天}{3,9,4}
  \begin{phonetics}{大前天}{da4qian2tian1}
    \definition{adv.}{três dias atrás}
  \end{phonetics}
\end{entry}

\begin{entry}{大战}{3,9}
  \begin{phonetics}{大战}{da4zhan4}
    \definition{s.}{guerra}
    \definition{v.}{guerrear | lutar em uma guerra}
  \end{phonetics}
\end{entry}

\begin{entry}{大洋洲}{3,9,9}
  \begin{phonetics}{大洋洲}{da4yang2zhou1}
    \definition*{s.}{Oceania}
  \end{phonetics}
\end{entry}

\begin{entry}{大神}{3,9}
  \begin{phonetics}{大神}{da4shen2}
    \definition{s.}{deidade | (gíria da Internet) guru | \emph{expert} | gênio}
  \end{phonetics}
\end{entry}

\begin{entry}{大胆}{3,9}
  \begin{phonetics}{大胆}{da4dan3}
    \definition{adj.}{audacioso | ousado | destemido}
  \end{phonetics}
\end{entry}

\begin{entry}{大家}{3,10}
  \begin{phonetics}{大家}{da4jia1}[][HSK 2]
    \definition{pron.}{todos}
  \end{phonetics}
\end{entry}

\begin{entry}{大海}{3,10}
  \begin{phonetics}{大海}{da4 hai3}[][HSK 2]
    \definition{s.}{mar | oceano}
  \end{phonetics}
\end{entry}

\begin{entry}{大部分}{3,10,4}
  \begin{phonetics}{大部分}{da4 bu2 fen4}[][HSK 2]
    \definition{s.}{a maioria | a maior parte}
  \end{phonetics}
\end{entry}

\begin{entry}{大猩猩}{3,12,12}
  \begin{phonetics}{大猩猩}{da4xing1xing5}
    \definition{s.}{gorila}
  \end{phonetics}
\end{entry}

\begin{entry}{大量}{3,12}
  \begin{phonetics}{大量}{da4 liang4}[][HSK 2]
    \definition{adj.}{numeroso | em massa | grande em número ou quantidade | generoso | magnânimo}
  \end{phonetics}
\end{entry}

\begin{entry}{大概}{3,13}
  \begin{phonetics}{大概}{da4gai4}
    \definition{adv.}{aproximadamente | por volta de}
  \end{phonetics}
\end{entry}

\begin{entry}{大腿}{3,13}
  \begin{phonetics}{大腿}{da4tui3}
    \definition{s.}{coxa}
  \end{phonetics}
\end{entry}

\begin{entry}{大蒜}{3,13}
  \begin{phonetics}{大蒜}{da4suan4}
    \definition[瓣,头]{s.}{alho}
  \end{phonetics}
\end{entry}

\begin{entry}{大赛}{3,14}
  \begin{phonetics}{大赛}{da4sai4}
    \definition{s.}{grande concurso, competição}
  \end{phonetics}
\end{entry}

\begin{entry}{女}{3}[Radical 女][Kangxi 38]
  \begin{phonetics}{女}{nv3}[][HSK 1]
    \definition{adj.}{feminino}
  \end{phonetics}
\end{entry}

\begin{entry}{女人}{3,2}
  \begin{phonetics}{女人}{nv3ren2}[][HSK 1]
    \definition[个,位]{s.}{mulher}
  \end{phonetics}
\end{entry}

\begin{entry}{女儿}{3,2}
  \begin{phonetics}{女儿}{nv3'er2}[][HSK 1]
    \definition{s.}{filha}
  \seealsoref{儿子}{er2zi5}
  \end{phonetics}
\end{entry}

\begin{entry}{女王}{3,4}
  \begin{phonetics}{女王}{nv3wang2}
    \definition{s.}{rainha}
  \end{phonetics}
\end{entry}

\begin{entry}{女生}{3,5}
  \begin{phonetics}{女生}{nv3sheng1}[][HSK 1]
    \definition[个]{s.}{aluna | estudante so sexo feminino}
  \end{phonetics}
\end{entry}

\begin{entry}{女朋友}{3,8,4}
  \begin{phonetics}{女朋友}{nv3peng2you5}[][HSK 1]
    \definition{s.}{namorada}
  \end{phonetics}
\end{entry}

\begin{entry}{女孩}{3,9}
  \begin{phonetics}{女孩}{nv3hai2}
    \definition{s.}{menina | garota}
  \end{phonetics}
\end{entry}

\begin{entry}{女孩儿}{3,9,2}
  \begin{phonetics}{女孩儿}{nv3hai2r5}[][HSK 1]
  \end{phonetics}
\end{entry}

\begin{entry}{女婿}{3,12}
  \begin{phonetics}{女婿}{nv3xu5}
    \definition{s.}{marido da filha}
  \end{phonetics}
\end{entry}

\begin{entry}{子}{3}[Radical 子]
  \begin{phonetics}{子}{zi3}
    \definition{adj.}{jovem | pequeno | tenro}
    \definition{clas.}{para objetos finos que podem ser pinçados com os dedos}
    \definition{pron.}{você}
    \definition{s.}{filho | pessoa | antigo título de respeito para um homem culto ou virtuoso | semente | ovo; ova | coisas pequenas e duras | moeda de cobre; cobre | o quarto título da classificação dos cinco títulos feudais de nobreza; visconde}
  \end{phonetics}
  \begin{phonetics}{子}{zi5}
    \definition{clas.}{sufixos de palavras de medida individuais}
    \definition{suf.}{sufixo para substantivos}
  \end{phonetics}
\end{entry}

\begin{entry}{子弹}{3,11}
  \begin{phonetics}{子弹}{zi3dan4}
    \definition[粒,颗,发]{s.}{bala (de revólver)}
  \end{phonetics}
\end{entry}

\begin{entry}{小}{3}[Radical 小][Kangxi 42]
  \begin{phonetics}{小}{xiao3}[][HSK 1]
    \definition{adj.}{pequeno | jovem}
  \end{phonetics}
\end{entry}

\begin{entry}{小小}{3,3}
  \begin{phonetics}{小小}{xiao3xiao3}
    \definition{adj.}{muito pequeno}
  \end{phonetics}
\end{entry}

\begin{entry}{小区}{3,4}
  \begin{phonetics}{小区}{xiao3qu1}
    \definition{s.}{conjunto habitacional, comunidade, bairro | célula (telecomunicações)}
  \end{phonetics}
\end{entry}

\begin{entry}{小心}{3,4}
  \begin{phonetics}{小心}{xiao3xin1}[][HSK 2]
    \definition{adj.}{cuidado}
  \end{phonetics}
\end{entry}

\begin{entry}{小气鬼}{3,4,9}
  \begin{phonetics}{小气鬼}{xiao3qi4gui3}
    \definition{adj.}{avarento | mão-de-vaca | miserável | pão-duro}
  \end{phonetics}
\end{entry}

\begin{entry}{小白菜}{3,5,11}
  \begin{phonetics}{小白菜}{xiao3bai2cai4}
    \definition[棵]{s.}{\emph{bok choy} | couve chinesa}
  \end{phonetics}
\end{entry}

\begin{entry}{小众}{3,6}
  \begin{phonetics}{小众}{xiao3zhong4}
    \definition{s.}{minoria da população | nicho (mercado, etc.)}
  \end{phonetics}
\end{entry}

\begin{entry}{小吃}{3,6}
  \begin{phonetics}{小吃}{xiao3chi1}
    \definition{s.}{refeição leve | petisco}
  \end{phonetics}
\end{entry}

\begin{entry}{小声}{3,7}
  \begin{phonetics}{小声}{xiao3 sheng1}[][HSK 2]
    \definition{v.}{falar em voz baixa | sussurar}
  \end{phonetics}
\end{entry}

\begin{entry}{小时}{3,7}
  \begin{phonetics}{小时}{xiao3shi2}[][HSK 1]
    \definition{adv.}{hora | para horas}
    \definition[个]{s.}{hora}
  \end{phonetics}
\end{entry}

\begin{entry}{小时候}{3,7,10}
  \begin{phonetics}{小时候}{xiao3 shi2 hou5}[][HSK 2]
    \definition{s.}{na infância | quando alguém era jovem}
  \end{phonetics}
\end{entry}

\begin{entry}{小姐}{3,8}
  \begin{phonetics}{小姐}{xiao3jie5}[][HSK 1]
    \definition[个,位]{s.}{senhorita | jovem senhora | (gíria) prostituta}
  \end{phonetics}
\end{entry}

\begin{entry}{小学}{3,8}
  \begin{phonetics}{小学}{xiao3xue2}[][HSK 1]
    \definition{s.}{escola ensino fundamental}
  \end{phonetics}
\end{entry}

\begin{entry}{小学生}{3,8,5}
  \begin{phonetics}{小学生}{xiao3xue2sheng1}[][HSK 1]
    \definition{s.}{aluno, estudante de escola primária}
  \end{phonetics}
\end{entry}

\begin{entry}{小朋友}{3,8,4}
  \begin{phonetics}{小朋友}{xiao3peng2you3}[][HSK 1]
    \definition{s.}{criança | [termo de tratamento usado por um adulto para uma criança] amiguinho}
  \end{phonetics}
\end{entry}

\begin{entry}{小狗}{3,8}
  \begin{phonetics}{小狗}{xiao3 gou3}
    \definition{s.}{filhote de cachorro}
  \end{phonetics}
\end{entry}

\begin{entry}{小组}{3,8}
  \begin{phonetics}{小组}{xiao3 zu3}[][HSK 2]
    \definition[个]{s.}{grupo}
  \end{phonetics}
\end{entry}

\begin{entry}{小孩儿}{3,9,2}
  \begin{phonetics}{小孩儿}{xiao3hai2r5}[][HSK 1]
    \definition[个]{s.}{criança | bebê}
  \end{phonetics}
\end{entry}

\begin{entry}{小屋}{3,9}
  \begin{phonetics}{小屋}{xiao3wu1}
    \definition{s.}{cabana | chalé | cabine}
  \end{phonetics}
\end{entry}

\begin{entry}{小树}{3,9}
  \begin{phonetics}{小树}{xiao3shu4}
    \definition[棵]{s.}{muda | arbusto | árvore pequena}
  \end{phonetics}
\end{entry}

\begin{entry}{小洋白菜}{3,9,5,11}
  \begin{phonetics}{小洋白菜}{xiao3 yang2bai2cai4}
    \definition{s.}{couve de bruxelas}
  \end{phonetics}
\end{entry}

\begin{entry}{小说}{3,9}
  \begin{phonetics}{小说}{xiao3shuo1}[][HSK 2]
    \definition[本,部]{s.}{romance | ficção}
  \end{phonetics}
\end{entry}

\begin{entry}{小腿}{3,13}
  \begin{phonetics}{小腿}{xiao3tui3}
    \definition{s.}{perna (do joelho ao calcanhar) | haste}
  \end{phonetics}
\end{entry}

\begin{entry}{山}{3}[Radical 山][Kangxi 46]
  \begin{phonetics}{山}{shan1}[][HSK 1]
    \definition*{s.}{sobrenome Shan}
    \definition[座]{s.}{montanha | monte | qualquer coisa que se assemelhe a uma montanha}
  \end{phonetics}
\end{entry}

\begin{entry}{山区}{3,4}
  \begin{phonetics}{山区}{shan1qu1}
    \definition[个]{s.}{área montanhosa | montanhas}
  \end{phonetics}
\end{entry}

\begin{entry}{山东}{3,5}
  \begin{phonetics}{山东}{shan1dong1}
    \definition*{s.}{Shandong}
  \end{phonetics}
\end{entry}

\begin{entry}{山羊}{3,6}
  \begin{phonetics}{山羊}{shan1yang2}
    \definition{s.}{cabra | (ginástica) cavalo de salto de pequeno porte}
  \end{phonetics}
\end{entry}

\begin{entry}{山体}{3,7}
  \begin{phonetics}{山体}{shan1ti3}
    \definition{s.}{forma de uma montanha}
  \end{phonetics}
\end{entry}

\begin{entry}{山谷}{3,7}
  \begin{phonetics}{山谷}{shan1gu3}
    \definition{s.}{vale | ravina}
  \end{phonetics}
\end{entry}

\begin{entry}{山顶}{3,8}
  \begin{phonetics}{山顶}{shan1ding3}
    \definition{s.}{cume da montanha}
  \end{phonetics}
\end{entry}

\begin{entry}{山寨}{3,14}
  \begin{phonetics}{山寨}{shan1zhai4}
    \definition{s.}{fortaleza fortificada da vila | fortaleza da montanha (especialmente de bandidos) | falsificação | imitação | (fig.) pechincha}
  \end{phonetics}
\end{entry}

\begin{entry}{工}{3}[Radical 工][Kangxi 48]
  \begin{phonetics}{工}{gong1}
    \definition{s.}{trabalho | trabalhador | habilidade | profissão | comércio | ofício}
  \end{phonetics}
\end{entry}

\begin{entry}{工人}{3,2}
  \begin{phonetics}{工人}{gong1ren2}[][HSK 1]
    \definition{s.}{trabalhador | operário | mão de obra de fábrica}
  \end{phonetics}
\end{entry}

\begin{entry}{工艺}{3,4}
  \begin{phonetics}{工艺}{gong1yi4}
    \definition{s.}{artesanato}
  \end{phonetics}
\end{entry}

\begin{entry}{工艺品}{3,4,9}
  \begin{phonetics}{工艺品}{gong1yi4pin3}
    \definition[个]{s.}{artigo de artesanato | trabalho manual}
  \end{phonetics}
\end{entry}

\begin{entry}{工作}{3,7}
  \begin{phonetics}{工作}{gong1zuo4}[][HSK 1]
    \definition[个,份,项]{s.}{trabalho | tarefa}
    \definition{v.}{trabalhar | operar (uma máquina)}
  \end{phonetics}
\end{entry}

\begin{entry}{工资}{3,10}
  \begin{phonetics}{工资}{gong1zi1}
    \definition[份]{s.}{salário | remuneração}
  \end{phonetics}
\end{entry}

\begin{entry}{工龄}{3,13}
  \begin{phonetics}{工龄}{gong1ling2}
    \definition{s.}{tempo de serviço | senioridade}
  \end{phonetics}
\end{entry}

\begin{entry}{已}{3}[Radical 己]
  \begin{phonetics}{已}{yi3}
    \definition{adv.}{já | após | então}
  \end{phonetics}
\end{entry}

\begin{entry}{已久}{3,3}
  \begin{phonetics}{已久}{yi3jiu3}
    \definition{adv.}{já faz muito tempo}
  \end{phonetics}
\end{entry}

\begin{entry}{已灭}{3,5}
  \begin{phonetics}{已灭}{yi3mie4}
    \definition{adj.}{extinto}
  \end{phonetics}
\end{entry}

\begin{entry}{已知}{3,8}
  \begin{phonetics}{已知}{yi3zhi1}
    \definition{v.}{conhecido (ter ciência)}
  \end{phonetics}
\end{entry}

\begin{entry}{已经}{3,8}
  \begin{phonetics}{已经}{yi3jing1}[][HSK 2]
    \definition{adv.}{já}
  \end{phonetics}
\end{entry}

\begin{entry}{已故}{3,9}
  \begin{phonetics}{已故}{yi3gu4}
    \definition{adj.}{morto | atrasado}
  \end{phonetics}
\end{entry}

\begin{entry}{已婚}{3,11}
  \begin{phonetics}{已婚}{yi3hun1}
    \definition{adj.}{casado}
  \end{phonetics}
\end{entry}

\begin{entry}{已然}{3,12}
  \begin{phonetics}{已然}{yi3ran2}
    \definition{adv.}{já | já ser assim}
  \end{phonetics}
\end{entry}

\begin{entry}{干}{3}[Radical 干]
  \begin{phonetics}{干}{gan1}[][HSK 1]
    \definition*{s.}{sobrenome Gan}
    \definition{v.}{preocupar | ignorar | interferir}
  \end{phonetics}
  \begin{phonetics}{干}{gan4}[][HSK 1]
    \definition{v.}{fazer | gerenciar | trabalhar | (gíria) matar | (vulgar) foder}
  \end{phonetics}
\end{entry}

\begin{entry}{干与}{3,3}
  \begin{phonetics}{干与}{gan1yu4}
    \variantof{干预}
  \end{phonetics}
\end{entry}

\begin{entry}{干什么}{3,4,3}
  \begin{phonetics}{干什么}{gan4 shen2me5}[][HSK 1]
    \definition{v.}{o que fazer? | o que está fazendo?}
  \end{phonetics}
\end{entry}

\begin{entry}{干你屁事}{3,7,7,8}
  \begin{phonetics}{干你屁事}{gan1 ni3 pi4shi4}
    \definition{interj.}{Foda-se!}
  \end{phonetics}
\end{entry}

\begin{entry}{干净}{3,8}
  \begin{phonetics}{干净}{gan1jing4}[][HSK 1]
    \definition{adj.}{limpo | arrumado}
  \end{phonetics}
\end{entry}

\begin{entry}{干杯}{3,8}
  \begin{phonetics}{干杯}{gan1bei1}[][HSK 2]
    \definition{interj.}{Saúde!}
    \definition{v.+compl.}{fazer um brinde | brindar até a última gota}
  \end{phonetics}
\end{entry}

\begin{entry}{干活}{3,9}
  \begin{phonetics}{干活}{gan4huo2}
    \definition{v.+compl.}{trabalhar | trabalhar em um emprego}
  \end{phonetics}
\end{entry}

\begin{entry}{干活儿}{3,9,2}
  \begin{phonetics}{干活儿}{gan4huo2r5}[][HSK 2]
    \definition{v.}{trabalhar em um emprego}
  \end{phonetics}
\end{entry}

\begin{entry}{干预}{3,10}
  \begin{phonetics}{干预}{gan1yu4}
    \definition{s.}{intervenção}
    \definition{v.}{intervir | intrometer-se}
  \end{phonetics}
\end{entry}

\begin{entry}{广东}{3,5}
  \begin{phonetics}{广东}{guang3dong1}
    \definition*{s.}{Guangdong}
  \end{phonetics}
\end{entry}

\begin{entry}{广场}{3,6}
  \begin{phonetics}{广场}{guang3chang3}[][HSK 2]
    \definition{s.}{praça | praça pública | esplanada}
  \end{phonetics}
\end{entry}

\begin{entry}{广场舞}{3,6,14}
  \begin{phonetics}{广场舞}{guang3chang3wu3}
    \definition{s.}{quadrilha, uma rotina de exercícios tocada com música em quadrados públicos, parques e praças, popular especialmente entre mulheres de meia-idade e aposentados na China}
  \end{phonetics}
\end{entry}

\begin{entry}{广告}{3,7}
  \begin{phonetics}{广告}{guang3gao4}[][HSK 2]
    \definition[项]{s.}{publicidade | anúncio publicitário}
  \end{phonetics}
\end{entry}

\begin{entry}{才}{3}[Radical 手]
  \begin{phonetics}{才}{cai2}[][HSK 2]
    \definition{adv.}{apenas (seguido por uma cláusula numérica) | meramente | só (indicando que algo está acontecendo mais tarde do que o esperado) | só depois | só então | não\dots até (precedido por uma cláusula de condição ou razão) | há um momento atrás | há pouco tempo}
    \definition{conj.}{apenas quando}
    \definition{s.}{um indivíduo capaz | habilidade | talento}
  \end{phonetics}
\end{entry}

\begin{entry}{才华}{3,6}
  \begin{phonetics}{才华}{cai2hua2}
    \definition[份]{s.}{talento}
  \end{phonetics}
\end{entry}

\begin{entry}{才能}{3,10}
  \begin{phonetics}{才能}{cai2neng2}
    \definition{s.}{talento | abilidade | capacidade}
  \end{phonetics}
\end{entry}

\begin{entry}{才略}{3,11}
  \begin{phonetics}{才略}{cai2lve4}
    \definition{s.}{habilidade e sagacidade}
  \end{phonetics}
\end{entry}

\begin{entry}{门}{3}[Radical 門]
  \begin{phonetics}{门}{men2}[][HSK 1]
    \definition*{s.}{sobrenome Men}
    \definition{clas.}{para canhão | para lição de casa, tecnologia, etc.}
    \definition{s.}{porta | portão | entrada; saída | interruptor | válvula |maneira | método | acesso | família | casa | escola (de pensamento) | seita (religiosa) | ramo de estudo | categoria; classe | filo}
  \end{phonetics}
\end{entry}

\begin{entry}{门口}{3,3}
  \begin{phonetics}{门口}{men2kou3}[][HSK 1]
    \definition[个]{s.}{porta | portão}
  \end{phonetics}
\end{entry}

\begin{entry}{门票}{3,11}
  \begin{phonetics}{门票}{men2piao4}[][HSK 1]
    \definition{s.}{bilhete de entrada | bilhete de admissão}
  \end{phonetics}
\end{entry}

\begin{entry}{飞}{3}[Radical 飛]
  \begin{phonetics}{飞}{fei1}[][HSK 1]
    \definition*{s.}{sobrenome Fei}
    \definition{adj.}{inesperado | acidental | infundado | sem fundamento}
    \definition{adv.}{rapidamente}
    \definition{s.}{roda livre de uma bicicleta}
    \definition{v.}{voar | esvoaçar | flutuar no ar | volatilizar}
  \end{phonetics}
\end{entry}

\begin{entry}{飞机}{3,6}
  \begin{phonetics}{飞机}{fei1ji1}[][HSK 1]
    \definition[架]{s.}{avião}
  \end{phonetics}
\end{entry}

\begin{entry}{飞机票}{3,6,11}
  \begin{phonetics}{飞机票}{fei1ji1piao4}
    \definition[张]{s.}{bilhete de avião}
  \seealsoref{机票}{ji1piao4}
  \end{phonetics}
\end{entry}

\begin{entry}{飞行}{3,6}
  \begin{phonetics}{飞行}{fei1xing2}
    \definition{s.}{voo | aviação}
    \definition{v.}{(aviões, etc.) voar}
  \end{phonetics}
\end{entry}

\begin{entry}{飞船}{3,11}
  \begin{phonetics}{飞船}{fei1chuan2}
    \definition{s.}{espaçonave | dirigível | aeronave}
  \end{phonetics}
\end{entry}

\begin{entry}{飞碟}{3,14}
  \begin{phonetics}{飞碟}{fei1die2}
    \definition{s.}{disco-voador, OVNI, \emph{UFO} | \emph{frisbee}}
  \end{phonetics}
\end{entry}

\begin{entry}{马}{3}[Radical 馬]
  \begin{phonetics}{马}{ma3}
    \definition*{s.}{sobrenome Ma}
    \definition[匹]{s.}{cavalo}
  \end{phonetics}
\end{entry}

\begin{entry}{马上}{3,3}
  \begin{phonetics}{马上}{ma3shang4}[][HSK 1]
    \definition{adv.}{já | imediatamente | de imediato | sem demora}
  \end{phonetics}
\end{entry}

\begin{entry}{马马虎虎}{3,3,8,8}
  \begin{phonetics}{马马虎虎}{ma3ma3hu3hu3}
    \definition{adj.}{descuidado | casual | tolerável | vago | mais ou menos}
  \end{phonetics}
\end{entry}

\begin{entry}{马耳他}{3,6,5}
  \begin{phonetics}{马耳他}{ma3'er3ta1}
    \definition*{s.}{Malta}
  \end{phonetics}
\end{entry}

\begin{entry}{马克思列宁主义}{3,7,9,6,5,5,3}
  \begin{phonetics}{马克思列宁主义}{ma3ke4si1 lie4ning2zhu3yi4}
    \definition*{s.}{Marxismo-Leninismo}
  \end{phonetics}
\end{entry}

\begin{entry}{马尾}{3,7}
  \begin{phonetics}{马尾}{ma3wei3}
    \definition{s.}{(penteado) rabo de cavalo | cauda de cavalo}
  \end{phonetics}
\end{entry}

\begin{entry}{马路}{3,13}
  \begin{phonetics}{马路}{ma3lu4}[][HSK 1]
    \definition[条]{s.}{rua | estrada}
  \end{phonetics}
\end{entry}

%%%%% EOF %%%%%

