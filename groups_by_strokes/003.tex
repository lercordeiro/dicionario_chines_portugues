%%%
%%% 3画
%%%

\section*{3画}\addcontentsline{toc}{section}{3画}

\begin{entry}{万}{3}{⼀}
  \begin{phonetics}{万}{wan4}[][HSK 2]
    \definition*{s.}{Sobrenome Wan}
    \definition{adv.}{absolutamente; indica um grau extremamente alto, equivalente a 完全, 绝对 e 极}
    \definition{num.}{dez mil; 10.000; 1.0000 | miríade; um número muito grande}
  \seealsoref{极}{ji2}
  \seealsoref{绝对}{jue2dui4}
  \seealsoref{完全}{wan2quan2}
  \end{phonetics}
\end{entry}

\begin{entry}{万一}{3,1}{⼀、⼀}
  \begin{phonetics}{万一}{wan4yi1}[][HSK 4]
    \definition{conj.}{por via das dúvidas; se por acaso; só por precaução; expressa uma suposição muito improvável (usado para coisas desagradáveis)}
    \definition{num.}{um décimo milionésimo; uma porcentagem muito pequena}
    \definition{s.}{contingência; eventualidade; contingências muito improváveis}
  \end{phonetics}
\end{entry}

\begin{entry}{万万}{3,3}{⼀、⼀}
  \begin{phonetics}{万万}{wan4wan4}
    \definition{adv.}{absolutamente | totalmente}
  \end{phonetics}
\end{entry}

\begin{entry}{万圣节}{3,5,5}{⼀、⼟、⾋}
  \begin{phonetics}{万圣节}{wan4sheng4jie2}
    \definition*{s.}{Dia de Todos os Santos}
  \seealsoref{万圣节前夕}{wan4sheng4jie2qian2xi1}
  \end{phonetics}
\end{entry}

\begin{entry}{万圣节前夕}{3,5,5,9,3}{⼀、⼟、⾋、⼑、⼣}
  \begin{phonetics}{万圣节前夕}{wan4sheng4jie2qian2xi1}
    \definition*{s.}{Véspera do Dia de Todos os Santos | Halloween}
  \seealsoref{万圣节}{wan4sheng4jie2}
  \end{phonetics}
\end{entry}

\begin{entry}{万福}{3,13}{⼀、⽰}
  \begin{phonetics}{万福}{wan4fu2}
    \definition{s.}{(antigo) reverência feminina; reverência}
  \end{phonetics}
\end{entry}

\begin{entry}{丈}{3}{⼀}
  \begin{phonetics}{丈}{zhang4}
    \definition{clas.}{zhang, uma unidade tradicional de comprimento, igual a 10 市尺 e equivalente a 3,333 metros ou 3,65 jardas}
    \definition{s.}{zhang, uma unidade de comprimento (= 3,333 metros)}
    \definition{s.}{sênior; ancião | marido (em certos termos de parentesco) | tratamento respeitoso ao idoso na China antiga; um título respeitoso para homens idosos nos tempos antigos | uma forma de tratamento para certos parentes do sexo masculino por casamento}
  \seealsoref{市尺}{shi4 chi3}
  \end{phonetics}
\end{entry}

\begin{entry}{丈夫}{3,4}{⼀、⼤}
  \begin{phonetics}{丈夫}{zhang4fu5}[][HSK 4]
    \definition[个,位,名]{s.}{marido; esposo}
  \end{phonetics}
\end{entry}

\begin{entry}{三}{3}{⼀}
  \begin{phonetics}{三}{san1}[][HSK 1]
    \definition*{s.}{Sobrenome San}
    \definition{num.}{três; 3 | muitos; vários; mais de dois; referindo-se a muitos ou à maioria | alguns; poucos; menos; não muitos}
  \end{phonetics}
\end{entry}

\begin{entry}{三角}{3,7}{⼀、⾓}
  \begin{phonetics}{三角}{san1jiao3}
    \definition{s.}{triângulo}
  \end{phonetics}
\end{entry}

\begin{entry}{三角恋爱}{3,7,10,10}{⼀、⾓、⼼、⽖}
  \begin{phonetics}{三角恋爱}{san1jiao3lian4'ai4}
    \definition{s.}{triângulo amoroso}
  \end{phonetics}
\end{entry}

\begin{entry}{三明治}{3,8,8}{⼀、⽇、⽔}
  \begin{phonetics}{三明治}{san1 ming2 zhi4}[][HSK 6]
    \definition[个,些,块]{s.}{Empréstimo linguístico: sanduíche, \emph{sandwich}}
  \end{phonetics}
\end{entry}

\begin{entry}{三轮车}{3,8,4}{⼀、⾞、⾞}
  \begin{phonetics}{三轮车}{san1lun2che1}
    \definition{s.}{triciclo}
  \end{phonetics}
\end{entry}

\begin{entry}{上}{3}{⼀}
  \begin{phonetics}{上}{shang3}
    \definition{s.}{tom descendente-ascendente; significa o segundo tom dos quatro tons do mandarim, e também se refere ao terceiro tom do mandarim padrão}
  \end{phonetics}
  \begin{phonetics}{上}{shang4}[][HSK 1]
    \definition{adj.}{mais recente; último; anterior; tempo ou a sequência anterior | superior; mais alto; melhor; indica uma posição elevada em termos de qualidade, nível, etc. | lugar elevado; posição superior (em oposição a 下)}
    \definition{s.}{superior; acima; para cima; um lugar alto ou mais alto do que um determinado local | na superfície de um objeto; usado após um substantivo, indica a superfície de um objeto | indica estar dentro do escopo de algo; usado após um substantivo, indica que algo está dentro do âmbito de determinada coisa | indica um aspecto específico | antigamente, referia-se ao imperador | usado após palavras que indicam idade, equivale a ``\dots 的时候'' | o primeiro nível da escala da música folclórica chinesa, usado como um símbolo de nota na notação musical, equivalente ao '1' na notação simplificada.}
    \definition{v.}{subir; montar; embarcar; entrar | ir para; partir para | estar ocupado (com trabalho, estudos, etc.) em um horário fixo; começar a trabalhar ou estudar na hora marcada, etc. | seguir em frente; prosseguir | encher; abastecer; servir; melhorar; aumentar | aparecer no palco; entrar | colocar algo em posição; ajustar; fixar; montar as duas partes de algo | aplicar; pintar; espalhar | ser registrado; ser publicado (em uma publicação) | atingir; ser suficiente (uma determinada quantidade ou grau) | submeter; enviar; apresentar; submeter à aprovação superior | ventilar; apertar; torcer | trazer; servir; colocar comida, pratos, chá e outras coisas na mesa para os convidados | indicar que uma ação tem um resultado | pesquisar na \emph{Internet} | emaranhar-se; ficar emaranhado; enredar-se}
    \definition{v.aux.}{usado após um verbo para indicar início e continuidade}
  \seealsoref{的时候}{de5 shi2hou4}
  \seealsoref{下}{xia4}
  \end{phonetics}
\end{entry}

\begin{entry}{上下}{3,3}{⼀、⼀}
  \begin{phonetics}{上下}{shang4 xia4}[][HSK 5]
    \definition{adv.}{para cima e para baixo}
    \definition[顶]{s.}{alto e baixo | de cima para baixo; para cima e para baixo | superioridade ou inferioridade relativa | (após números redondos) aproximadamente; mais ou menos; por aí | velhos e jovens; hierarquia em termos de cargo e posição social}
    \definition{v.}{subir ou descer | subir e descer; da alta para a baixa ou da baixa para a alta}
  \end{phonetics}
\end{entry}

\begin{entry}{上个月}{3,3,4}{⼀、⼈、⽉}
  \begin{phonetics}{上个月}{shang4 ge4 yue4}[][HSK 4]
    \definition{s.}{mês passado; refere-se à hora de um mês atrás, ou seja, o mês passado mais próximo da hora atual}
  \end{phonetics}
\end{entry}

\begin{entry}{上门}{3,3}{⼀、⾨}
  \begin{phonetics}{上门}{shang4 men2}[][HSK 4]
    \definition{v.}{chamar; visitar; aparecer; ir ou vir para ver alguém; ir até a porta; ir até a casa de alguém | trancar a porta; fechar a porta durante a noite | casar-se e morar com a família da noiva}
  \end{phonetics}
\end{entry}

\begin{entry}{上升}{3,4}{⼀、⼗}
  \begin{phonetics}{上升}{shang4 sheng1}[][HSK 3]
    \definition{v.}{elevar; subir; mover-se para cima; mover de baixo para cima; aumentar em nível, grau, quantidade, etc.}
  \end{phonetics}
\end{entry}

\begin{entry}{上午}{3,4}{⼀、⼗}
  \begin{phonetics}{上午}{shang4wu3}[][HSK 1]
    \definition[个]{s.}{manhã; \emph{ante meridiem} (a.m.); geralmente refere-se ao período entre a manhã e o meio-dia}
  \end{phonetics}
\end{entry}

\begin{entry}{上车}{3,4}{⼀、⾞}
  \begin{phonetics}{上车}{shang4 che1}[][HSK 1]
    \definition{v.}{entrar; subir (em um ônibus, trem, carro etc.)}
  \end{phonetics}
\end{entry}

\begin{entry}{上去}{3,5}{⼀、⼛}
  \begin{phonetics}{上去}{shang4 qu4}[][HSK 3]
    \definition{v.}{subir (a partir da minha localização) | ascender a um lugar (ou estado) considerado mais elevado (ou acima); usado depois de um verbo para indicar movimento, de baixo para cima ou de perto para longe}
  \end{phonetics}
\end{entry}

\begin{entry}{上古}{3,5}{⼀、⼝}
  \begin{phonetics}{上古}{shang4gu3}
    \definition{s.}{o passado distante | tempos antigos | antiguidade}
  \end{phonetics}
\end{entry}

\begin{entry}{上台}{3,5}{⼀、⼝}
  \begin{phonetics}{上台}{shang4 tai2}[][HSK 6]
    \definition{v.}{aparecer no palco; subir na plataforma; ir para o palco ou pódio | assumir o poder; chegar (subir) ao poder; começar a assumir papéis de liderança ou a ganhar algum tipo de poder}
  \end{phonetics}
\end{entry}

\begin{entry}{上市}{3,5}{⼀、⼱}
  \begin{phonetics}{上市}{shang4 shi4}[][HSK 6]
    \definition{v.}{listar; abrir o capital; ser listado (na bolsa de valores) | estar na estação; estar (aparecer) no mercado | ir ao mercado (para fazer compras)}
  \end{phonetics}
\end{entry}

\begin{entry}{上边}{3,5}{⼀、⾡}
  \begin{phonetics}{上边}{shang4 bian5}[][HSK 1]
    \definition{s.}{topo; acima; sobre; superior}
  \end{phonetics}
\end{entry}

\begin{entry}{上当}{3,6}{⼀、⼹}
  \begin{phonetics}{上当}{shang4dang4}[][HSK 6]
    \definition{v.+compl.}{ser enganado; ser ludibriado; morder a isca; cair nas mãos de alguém}
  \end{phonetics}
\end{entry}

\begin{entry}{上次}{3,6}{⼀、⽋}
  \begin{phonetics}{上次}{shang4 ci4}[][HSK 1]
    \definition{adv.}{última vez}
  \end{phonetics}
\end{entry}

\begin{entry}{上级}{3,6}{⼀、⽷}
  \begin{phonetics}{上级}{shang4ji2}[][HSK 5]
    \definition[个]{s.}{nível superior; organização ou pessoa em nível superior; organizações ou pessoas de nível superior dentro do mesmo sistema organizacional}
  \end{phonetics}
\end{entry}

\begin{entry}{上网}{3,6}{⼀、⽹}
  \begin{phonetics}{上网}{shang4wang3}[][HSK 1]
    \definition{v.}{conectar-se à \emph{Internet}; acessar a \emph{Internet}; entrar na \emph{Internet}; acessar a rede; refere-se especificamente ao computador do usuário conectado à Internet para pesquisar e consultar informações, etc.}
  \end{phonetics}
\end{entry}

\begin{entry}{上衣}{3,6}{⼀、⾐}
  \begin{phonetics}{上衣}{shang4 yi1}[][HSK 3]
    \definition[件]{s.}{jaqueta; roupas para a parte superior do corpo}
  \end{phonetics}
\end{entry}

\begin{entry}{上访}{3,6}{⼀、⾔}
  \begin{phonetics}{上访}{shang4fang3}
    \definition{v.}{buscar uma audiência com superiores (especialmente funcionários do governo) para fazer uma petição por algo}
  \end{phonetics}
\end{entry}

\begin{entry}{上声}{3,7}{⼀、⼠}
  \begin{phonetics}{上声}{shang3sheng1}
    \definition{s.}{tom descendente e ascendente | terceiro tom no mandarim moderno}
  \end{phonetics}
\end{entry}

\begin{entry}{上来}{3,7}{⼀、⽊}
  \begin{phonetics}{上来}{shang4 lai2}[][HSK 3]
    \definition{v.}{subir (para a minha localização) | estar no começo; começar; iniciar | surgir; de um lugar baixo para um lugar alto (o interlocutor está em um lugar alto) | usado após o verbo, indica que algo foi concluído com sucesso}
  \end{phonetics}
\end{entry}

\begin{entry}{上周}{3,8}{⼀、⼝}
  \begin{phonetics}{上周}{shang4 zhou1}[][HSK 2]
    \definition{s.}{semana passada}
  \end{phonetics}
\end{entry}

\begin{entry}{上坡路}{3,8,13}{⼀、⼟、⾜}
  \begin{phonetics}{上坡路}{shang4po1lu4}
    \definition{s.}{aclive | progresso | (fig.) tendência ascendente}
  \end{phonetics}
\end{entry}

\begin{entry}{上学}{3,8}{⼀、⼦}
  \begin{phonetics}{上学}{shang4 xue2}[][HSK 1]
    \definition{v.}{ir à escola; frequentar a escola; estar na escola; ir à escola para estudar | começar a escola; começar a estudar no ensino fundamental}
  \end{phonetics}
\end{entry}

\begin{entry}{上询}{3,8}{⼀、⾔}
  \begin{phonetics}{上询}{shang4 xun2}
    \definition{adv.}{primeira dezena do mês}
  \end{phonetics}
\end{entry}

\begin{entry}{上帝}{3,9}{⼀、⼱}
  \begin{phonetics}{上帝}{shang4 di4}[][HSK 6]
    \definition*{s.}{Deus; O Deus Supremo no Cristianismo | O Imperador do Céu; um deus na antiga crença chinesa que pode controlar tudo no mundo}
    \definition[个]{s.}{(figurado) cliente; metáfora para consumidores}
  \end{phonetics}
\end{entry}

\begin{entry}{上面}{3,9}{⼀、⾯}
  \begin{phonetics}{上面}{shang4 mian4}[][HSK 3]
    \definition{s.}{uma posição mais alta que algo; uma posição acima/acima de algo | superfície do objeto | aspecto | a parte acima mencionada; a parte que vem primeiro na ordem; a parte de um artigo ou discurso que vem antes da presente | autoridades superiores | os mais velhos; a geração mais velha da família}
  \end{phonetics}
\end{entry}

\begin{entry}{上海}{3,10}{⼀、⽔}
  \begin{phonetics}{上海}{shang4hai3}
    \definition*{s.}{Município de Xangai (Shanghai), centro-leste da China}
  \end{phonetics}
\end{entry}

\begin{entry}{上涨}{3,10}{⼀、⽔}
  \begin{phonetics}{上涨}{shang4 zhang3}[][HSK 5]
    \definition{v.}{subir; ir para cima; ascender}
  \end{phonetics}
\end{entry}

\begin{entry}{上班}{3,10}{⼀、⽟}
  \begin{phonetics}{上班}{shang4ban1}[][HSK 1]
    \definition{v.+compl.}{ir trabalhar; começar a trabalhar; estar de plantão; ir trabalhar no local de trabalho regular no horário especificado}
  \end{phonetics}
\end{entry}

\begin{entry}{上课}{3,10}{⼀、⾔}
  \begin{phonetics}{上课}{shang4 ke4}[][HSK 1]
    \definition{v.+compl.}{frequentar aulas; ir às aulas; dar uma aula}
  \end{phonetics}
\end{entry}

\begin{entry}{上楼}{3,13}{⼀、⽊}
  \begin{phonetics}{上楼}{shang4 lou2}[][HSK 4]
    \definition{v.}{subir as escadas; ir para o andar de cima}
  \end{phonetics}
\end{entry}

\begin{entry}{上演}{3,14}{⼀、⽔}
  \begin{phonetics}{上演}{shang4 yan3}[][HSK 6]
    \definition{s.}{exibição | encenação}
    \definition{v.}{exibir (um filme); encenar (uma peça); atuar; colocar no palco}
  \end{phonetics}
\end{entry}

\begin{entry}{下}{3}{⼀}
  \begin{phonetics}{下}{xia4}[][HSK 1,2]
    \definition{clas.}{número de vezes usado para a ação | volume de um contêiner; quantidade de objetos que cabem em um utensílio | usado depois de 两 e 几 para expressar habilidade, capacidade, destreza}
    \definition{s.}{abaixo | próximo; último; segundo; referindo-se ao que está por vir ou ao que vem em seguida | mais baixo; inferior; de baixo nível ou grau | próximo; último; segundo; em ordem ou em ordem cronológica | indica pertencer a uma determinada faixa, situação, condição, etc. | indica uma determinada época ou estação | usado após um número para indicar posição ou direção | para baixo (após uma preposição) | sob (depois de um substantivo) | para baixo (antes de um verbo)}
    \definition{v.}{desembarcar; descer; sair | cair (chuva, neve, etc.) | enviar; emitir; entregar | ir para | sair; partir; retirar-se | lançar; colocar | descarregar; desmontar; tirar (fora) | formar (uma opinião, ideia, etc.); tomar decisões, fazer julgamentos, etc. | usar; aplicar | dar à luz (animais) | tomar; capturar; conquistar | ceder | terminar; deixar de lado; terminar o trabalho ou os estudos diários na hora prevista | para negação; ser inferior a; ser menor que}
  \seealsoref{几}{ji3}
  \seealsoref{两}{liang3}
  \end{phonetics}
\end{entry}

\begin{entry}{下个月}{3,3,4}{⼀、⼈、⽉}
  \begin{phonetics}{下个月}{xia4 ge4 yue4}[][HSK 4]
    \definition{s.}{próximo mês; mês que vem; refere-se ao próximo mês do mês atual}
  \end{phonetics}
\end{entry}

\begin{entry}{下午}{3,4}{⼀、⼗}
  \begin{phonetics}{下午}{xia4wu3}[][HSK 1]
    \definition[个]{s.}{tarde; \emph{post meridiem} (p.m.); refere-se ao período entre o meio-dia e o pôr do sol}
  \end{phonetics}
\end{entry}

\begin{entry}{下午茶}{3,4,9}{⼀、⼗、⾋}
  \begin{phonetics}{下午茶}{xia4wu3cha2}
    \definition{s.}{chá da tarde (normalmente chás com doces)}
  \end{phonetics}
\end{entry}

\begin{entry}{下巴}{3,4}{⼀、⼰}
  \begin{phonetics}{下巴}{xia4ba5}
    \definition[个]{s.}{queixo}
  \end{phonetics}
\end{entry}

\begin{entry}{下水道}{3,4,12}{⼀、⽔、⾡}
  \begin{phonetics}{下水道}{xia4shui3dao4}
    \definition{s.}{esgoto}
  \end{phonetics}
\end{entry}

\begin{entry}{下车}{3,4}{⼀、⾞}
  \begin{phonetics}{下车}{xia4 che1}[][HSK 1]
    \definition{v.}{descer ou sair de (um ônibus, trem, carro etc.)}
  \end{phonetics}
\end{entry}

\begin{entry}{下去}{3,5}{⼀、⼛}
  \begin{phonetics}{下去}{xia4 qu4}[][HSK 3]
    \definition{part.}{usado depois de verbos para indicar de alto a baixo | usado depois de um verbo para indicar continuação}
    \definition{v.}{descer; baixar (a partir da minha localização) | (após um verbo) continuar (fazendo algo); prosseguir | usado após o verbo, indica uma descida de um ponto alto para um ponto baixo | usado após o verbo, indica continuidade | usado após um adjetivo, indica que o grau continua aumentando}
  \end{phonetics}
\end{entry}

\begin{entry}{下边}{3,5}{⼀、⾡}
  \begin{phonetics}{下边}{xia4 bian5}[][HSK 1]
    \definition{s.}{abaixo; sob; por baixo | próximo em ordem; seguinte | nível inferior; subordinado | a parte inferior}
  \end{phonetics}
\end{entry}

\begin{entry}{下旬}{3,6}{⼀、⽇}
  \begin{phonetics}{下旬}{xia4xun2}
    \definition{adv.}{última dezena do mês}
  \end{phonetics}
\end{entry}

\begin{entry}{下次}{3,6}{⼀、⽋}
  \begin{phonetics}{下次}{xia4 ci4}[][HSK 1]
    \definition{s.}{na próxima vez; na próxima oportunidade ou no próximo evento}
  \end{phonetics}
\end{entry}

\begin{entry}{下来}{3,7}{⼀、⽊}
  \begin{phonetics}{下来}{xia4 lai5}[][HSK 3]
    \definition{part.}{usado após o verbo, indica que a ação ou o comportamento se dirige para a posição do falante ou que a ação é contínua ou concluída | usado após um adjetivo, indica que uma determinada situação começou a ocorrer e continuará a se desenvolver}
    \definition{v.}{descer (para a minha localização) | (colheitas/frutas/vegetais, etc.) ser colhido; estar maduro o suficiente para ser colhido | (período de tempo) acabar; passar; chegar ao fim; indicar o fim de um período de tempo}
  \end{phonetics}
\end{entry}

\begin{entry}{下周}{3,8}{⼀、⼝}
  \begin{phonetics}{下周}{xia4 zhou1}[][HSK 2]
    \definition{s.}{próxima semana}
  \end{phonetics}
\end{entry}

\begin{entry}{下线}{3,8}{⼀、⽷}
  \begin{phonetics}{下线}{xia4xian4}
    \definition{v.}{ficar \emph{offline} | (um produto) sair da linha de montagem | pessoa abaixo de si em um esquema de pirâmide}
  \end{phonetics}
\end{entry}

\begin{entry}{下降}{3,8}{⼀、⾩}
  \begin{phonetics}{下降}{xia4 jiang4}[][HSK 4]
    \definition{v.}{cair; despencar; declinar; descer; diminuir; ir para baixo}
  \end{phonetics}
\end{entry}

\begin{entry}{下雨}{3,8}{⼀、⾬}
  \begin{phonetics}{下雨}{xia4 yu3}[][HSK 1]
    \definition{v.+compl.}{chover}
  \end{phonetics}
\end{entry}

\begin{entry}{下面}{3,9}{⼀、⾯}
  \begin{phonetics}{下面}{xia4 mian4}[][HSK 3]
    \definition{s.}{em baixo; abaixo; parte de baixo | próximo; seguinte; a parte posterior; a parte posterior de um artigo ou discurso em relação ao que está sendo narrado no momento | subordinado; o nível inferior; homens nos níveis inferiores | por baixo}
  \end{phonetics}
\end{entry}

\begin{entry}{下海}{3,10}{⼀、⽔}
  \begin{phonetics}{下海}{xia4hai3}
    \definition{v.+compl.}{ir para o mar; (barco) deixar o porto e iniciar uma jornada | ir pescar no mar | tornar-se ator profissional}
  \end{phonetics}
\end{entry}

\begin{entry}{下班}{3,10}{⼀、⽟}
  \begin{phonetics}{下班}{xia4 ban1}[][HSK 1]
    \definition{v.+compl.}{sair do trabalho; bater ponto; terminar o trabalho na hora prevista e sair do local de trabalho}
  \end{phonetics}
\end{entry}

\begin{entry}{下课}{3,10}{⼀、⾔}
  \begin{phonetics}{下课}{xia4 ke4}[][HSK 1]
    \definition{v.+compl.}{terminar a aula; sair da aula}
  \end{phonetics}
\end{entry}

\begin{entry}{下载}{3,10}{⼀、⾞}
  \begin{phonetics}{下载}{xia4zai3}[][HSK 4]
    \definition{v.}{\emph{download}; baixar; salvar informações da \emph{Web} em um dispositivo, como um computador}
  \end{phonetics}
\end{entry}

\begin{entry}{下蛋}{3,11}{⼀、⾍}
  \begin{phonetics}{下蛋}{xia4dan4}
    \definition{v.}{botar ovos}
  \end{phonetics}
\end{entry}

\begin{entry}{下雪}{3,11}{⼀、⾬}
  \begin{phonetics}{下雪}{xia4 xue3}[][HSK 2]
    \definition{v.+compl.}{nevar}
  \end{phonetics}
\end{entry}

\begin{entry}{下崽}{3,12}{⼀、⼭}
  \begin{phonetics}{下崽}{xia4zai3}
    \definition{v.}{dar à luz (animais) | parir}
  \end{phonetics}
\end{entry}

\begin{entry}{下楼}{3,13}{⼀、⽊}
  \begin{phonetics}{下楼}{xia4 lou2}[][HSK 4]
    \definition{v.}{descer as escadas}
  \end{phonetics}
\end{entry}

\begin{entry}{与}{3}{⼀}
  \begin{phonetics}{与}{yu3}
    \definition{conj.}{e, com}
  \end{phonetics}
  \begin{phonetics}{与}{yu4}
    \definition{v.}{fazer parte de}
  \end{phonetics}
\end{entry}

\begin{entry}{与其}{3,8}{⼀、⼋}
  \begin{phonetics}{与其}{yu3qi2}
    \definition{conj.}{mais do que}
  \end{phonetics}
\end{entry}

\begin{entry}{与其……不如……}{3,8,4,6}{⼀、⼋、⼀、⼥}
  \begin{phonetics}{与其……不如……}{yu3qi2 bu4ru2}
    \definition{conj.}{ao invés de\dots melhor que\dots}
  \end{phonetics}
\end{entry}

\begin{entry}{与其……宁可……}{3,8,5,5}{⼀、⼋、⼧、⼝}
  \begin{phonetics}{与其……宁可……}{yu3qi2 ning4ke3}
    \definition{conj.}{ao invés de\dots melhor que\dots}
  \end{phonetics}
\end{entry}

\begin{entry}{个}{3}{⼈}
  \begin{phonetics}{个}{ge3}
    \definition{pron.}{usado em 自个儿}
  \seealsoref{自个儿}{zi4ge3r5}
  \end{phonetics}
  \begin{phonetics}{个}{ge4}[][HSK 1]
    \definition{adj.}{individual}
    \definition{clas.}{usado antes de substantivos que não têm palavras de medida específicas | usado na frente do divisor; usado na frente do número aproximado | usado após verbos com objeto direto |  usado entre verbos e complementos}
    \definition{part.}{usado após pronomes demonstrativos | adicionado após certas palavras de tempo}
  \end{phonetics}
\end{entry}

\begin{entry}{个人}{3,2}{⼈、⼈}
  \begin{phonetics}{个人}{ge4ren2}[][HSK 3]
    \definition{pron.}{pessoal; si mesmo}
    \definition[个]{s.}{indivíduo; pessoa}
  \end{phonetics}
\end{entry}

\begin{entry}{个儿}{3,2}{⼈、⼉}
  \begin{phonetics}{个儿}{ge4r5}[][HSK 5]
    \definition{s.}{tamanho; altura; estatura; tamanho do corpo ou do objeto | pessoas ou coisas consideradas isoladamente; referir-se a uma pessoa ou coisa individualmente}
  \end{phonetics}
\end{entry}

\begin{entry}{个子}{3,3}{⼈、⼦}
  \begin{phonetics}{个子}{ge4zi5}[][HSK 2]
    \definition[个,种,些]{s.}{altura; estatura; refere-se ao tamanho do corpo humano e também ao tamanho do corpo dos animais}
  \end{phonetics}
\end{entry}

\begin{entry}{个体}{3,7}{⼈、⼈}
  \begin{phonetics}{个体}{ge4ti3}[][HSK 4]
    \definition{s.}{pessoa ou organismo individual}
  \end{phonetics}
\end{entry}

\begin{entry}{个别}{3,7}{⼈、⼑}
  \begin{phonetics}{个别}{ge4bie2}[][HSK 4]
    \definition{adj.}{muito poucos; excepcionais}
    \definition{adv.}{separadamente; individualmente; isoladamente}
  \end{phonetics}
\end{entry}

\begin{entry}{个性}{3,8}{⼈、⼼}
  \begin{phonetics}{个性}{ge4xing4}[][HSK 3]
    \definition[种,点儿]{s.}{individualidade; personalidade; caráter individual; as características relativamente fixas de uma pessoa, formadas sob determinadas condições sociais e influências educacionais | propriedade específica; caráter específico; a propriedade ou característica especial que distingue uma coisa de outras coisas}
  \end{phonetics}
\end{entry}

\begin{entry}{久}{3}{⼃}
  \begin{phonetics}{久}{jiu3}[][HSK 3]
    \definition{adj.}{por muito tempo; longo período de tempo | duração de tempo especificada}
  \end{phonetics}
\end{entry}

\begin{entry}{义}{3}{⼂}
  \begin{phonetics}{义}{yi4}
    \definition*{s.}{Sobrenome Yi}
    \definition{adj.}{justo; equitativo | adotado; adotivo | juramentado | artificial; falso}
    \definition[个]{s.}{justiça; retidão | laços humanos; relacionamento | significado; importância}
  \end{phonetics}
\end{entry}

\begin{entry}{义务}{3,5}{⼂、⼒}
  \begin{phonetics}{义务}{yi4wu4}[][HSK 4]
    \definition{s.}{dever; obrigação; responsabilidades perante a lei, em oposição a 权利}
  \seealsoref{权利}{quan2li4}
  \end{phonetics}
\end{entry}

\begin{entry}{之}{3}{⼂}
  \begin{phonetics}{之}{zhi1}
    \definition*{s.}{Sobrenome Zhi}
    \definition{part.}{entre um atributivo e a palavra que ele modifica; equivalente a "的" | usado entre o sujeito e o predicado, em estruturas sujeito-predicado, de modo a torná-lo nominalizado}
    \definition{pron.}{substituto de uma pessoa ou coisa, limitado a ser usado como um objeto; substituir a pessoa ou coisa mencionada anteriormente | isto; isso; não substitui uma pessoa ou coisa específica, mas serve apenas para complementar sílabas}
    \definition{v.}{ir; deixar}
  \seealsoref{的}{de5}
  \end{phonetics}
\end{entry}

\begin{entry}{之一}{3,1}{⼂、⼀}
  \begin{phonetics}{之一}{zhi1 yi1}[][HSK 4]
    \definition{s.}{um de (algo); pertence a um ou a todo um grupo de coisas com as mesmas características}
  \end{phonetics}
\end{entry}

\begin{entry}{之下}{3,3}{⼂、⼀}
  \begin{phonetics}{之下}{zhi1 xia4}[][HSK 5]
    \definition{s.}{usado para indicar algo abaixo de um determinado intervalo, posição, grau, etc.; indica um aspecto inferior em termos de alcance, posição, status, nível, Chengdu, etc. | usado para indicar as condições sob as quais algo acontece | usado para indicar o humor, estado em que alguém faz algo; expressa um determinado comportamento em um determinado estado de espírito ou situação}
  \end{phonetics}
\end{entry}

\begin{entry}{之中}{3,4}{⼂、⼁}
  \begin{phonetics}{之中}{zhi1 zhong1}[][HSK 5]
    \definition{prep.}{em; no meio de; entre}
  \end{phonetics}
\end{entry}

\begin{entry}{之内}{3,4}{⼂、⼌}
  \begin{phonetics}{之内}{zhi1 nei4}[][HSK 5]
    \definition{adv.}{em; dentro de; indica dentro de um determinado intervalo, limite ou período de tempo, etc.}
  \end{phonetics}
\end{entry}

\begin{entry}{之外}{3,5}{⼂、⼣}
  \begin{phonetics}{之外}{zhi1 wai4}[][HSK 5]
    \definition{adv.}{lado de fora; exceto; além de; além disso; refere-se a algo que excede um determinado limite}
  \end{phonetics}
\end{entry}

\begin{entry}{之后}{3,6}{⼂、⼝}
  \begin{phonetics}{之后}{zhi1 hou4}[][HSK 4]
    \definition{adv.}{mais tarde; posteriormente; depois; desde então; para indicar que é depois de um determinado tempo ou de uma determinada coisa, 以后 é usado com frequência na linguagem falada; às vezes, também pode indicar que é depois de um determinado lugar ou local,  后面 é usado com frequência na linguagem falada}
  \seealsoref{后面}{hou4mian4}
  \seealsoref{以后}{yi3 hou4}
  \end{phonetics}
\end{entry}

\begin{entry}{之间}{3,7}{⼂、⾨}
  \begin{phonetics}{之间}{zhi1 jian1}[][HSK 4]
    \definition{prep.}{(depois de um substantivo) entre; dentro de duas delimitações de tempo, local ou quantitativas | colocado após certos verbos ou advérbios de duas sílabas para indicar um curto período de tempo}
  \end{phonetics}
\end{entry}

\begin{entry}{之前}{3,9}{⼂、⼑}
  \begin{phonetics}{之前}{zhi1 qian2}[][HSK 4]
    \definition{adv.}{(referindo-se ao tempo) antes, antes de, atrás | (referindo-se ao local físico) na frente de | (usado independentemente) no passado, antes disso}
  \end{phonetics}
\end{entry}

\begin{entry}{之类}{3,9}{⼂、⽶}
  \begin{phonetics}{之类}{zhi1 lei4}
    \definition{s.}{usado para dar exemplos (coisas do tipo, desse tipo, assim); uma categoria de pessoas ou coisas que compartilham as mesmas características das pessoas ou coisas mencionadas anteriormente}[我喜欢香蕉、苹果之类的水果。___Eu gosto de frutas como bananas e maçãs.]
  \end{phonetics}
\end{entry}

\begin{entry}{也}{3}{⼄}
  \begin{phonetics}{也}{ye3}[][HSK 1]
    \definition*{s.}{Sobrenome Ye}
    \definition{adv.}{também; igualmente; assim como; da mesma forma; usado em frases simples, implica que é igual a outra coisa | assim como (expressar ênfase) | (expressar que as consequências são as mesmas) | também (expressar ufemismo; expressar um tom diplomático) | usado em frases compostas paralelas, indica que duas ou mais coisas têm algo em comum (pode ser usado em todas as frases ou apenas na última frase)}
    \definition{part.}{usado no meio de uma frase, destacando um elemento da frase sobre o qual deve ser feita uma afirmação | usado no final de uma frase, indicando uma explicação ou um julgamento; usado no final da frase, indica tom afirmativo e também pode reforçar o tom interrogativo, exclamativo ou imperativo}
  \end{phonetics}
\end{entry}

\begin{entry}{也好}{3,6}{⼄、⼥}
  \begin{phonetics}{也好}{ye3 hao3}[][HSK 5]
    \definition{part.}{pode não ser uma má ideia; também pode ser | (reduplicado) se\dots ou\dots; não importa se | pode não ser uma má ideia | se\dots ou\dots; usado em conjunto, significa que não está condicionado a uma determinada situação}
  \end{phonetics}
\end{entry}

\begin{entry}{也有今天}{3,6,4,4}{⼄、⽉、⼈、⼤}
  \begin{phonetics}{也有今天}{ye3you3jin1tian1}
    \definition{expr.}{obter apenas o que merece | todo cachorro tem seu dia | obter a sua parte (coisas boas ou ruins) | servir alguém bem}
  \end{phonetics}
\end{entry}

\begin{entry}{也许}{3,6}{⼄、⾔}
  \begin{phonetics}{也许}{ye3xu3}[][HSK 2]
    \definition{adv.}{talvez; provavelmente; estou com medo; para expressar incerteza; para expressar uma alta probabilidade}
  \end{phonetics}
\end{entry}

\begin{entry}{也就是}{3,12,9}{⼄、⼪、⽇}
  \begin{phonetics}{也就是}{ye3jiu4shi4}
    \definition{adv.}{i.e., isso é | ou seja}
  \end{phonetics}
\end{entry}

\begin{entry}{也就是说}{3,12,9,9}{⼄、⼪、⽇、⾔}
  \begin{phonetics}{也就是说}{ye3jiu4shi4shuo1}
    \definition{adv.}{em outras palavras | então | isto é | por isso}
  \end{phonetics}
\end{entry}

\begin{entry}{习}{3}{⼄}
  \begin{phonetics}{习}{xi2}
    \definition*{s.}{Sobrenome Xi}
    \definition{s.}{hábito; costume; prática usual; um comportamento que se desenvolve inconscientemente por meio de ações repetidas ao longo de um longo período de tempo}
    \definition{v.}{revisar; praticar; exercitar | acostumado a; familiarizado com; familiarizado com algo por meio de contato frequente | estudar; aprender (pássaro)}
  \end{phonetics}
\end{entry}

\begin{entry}{习惯}{3,11}{⼄、⼼}
  \begin{phonetics}{习惯}{xi2guan4}[][HSK 2]
    \definition[个,种]{s.}{hábito; costume; prática usual; comportamentos, tendências ou tendências sociais que se desenvolvem gradualmente ao longo de um longo período de tempo e são difíceis de mudar}
    \definition{v.}{estar acostumado a; ter o hábito de}
  \end{phonetics}
\end{entry}

\begin{entry}{乡}{3}{⼄}
  \begin{phonetics}{乡}{xiang1}[][HSK 5]
    \definition[个]{s.}{país; campo; vilarejo; área rural | local de origem; vila ou cidade natal | município (uma unidade administrativa rural subordinada ao condado) | vila natal; cidade natal | terra ou local famoso por produzir algo}
  \end{phonetics}
\end{entry}

\begin{entry}{乡巴佬}{3,4,8}{⼄、⼰、⼈}
  \begin{phonetics}{乡巴佬}{xiang1ba1lao3}
    \definition{s.}{aldeão | caipira}
  \end{phonetics}
\end{entry}

\begin{entry}{乡村}{3,7}{⼄、⽊}
  \begin{phonetics}{乡村}{xiang1 cun1}[][HSK 5]
    \definition{adj.}{rural | rústico}
    \definition{s.}{vila; campo; área rural; principalmente envolvido na agricultura; áreas com distribuição populacional mais dispersa em relação às cidades}
  \end{phonetics}
\end{entry}

\begin{entry}{于}{3}{⼆}
  \begin{phonetics}{于}{yu2}[][HSK 6]
    \definition*{s.}{Sobrenome Yu}
    \definition{prep.}{indica hora, lugar, alcance, etc. | indica a direção da ação | usado depois de um verbo para indicar dar, entregar, etc. | apresentar a relação do objeto ou entidade introduzida | indica o ponto de início ou de partida | indica comparação | indica passividade}
  \end{phonetics}
\end{entry}

\begin{entry}{于是}{3,9}{⼆、⽇}
  \begin{phonetics}{于是}{yu2shi4}[][HSK 4]
    \definition{conj.}{então; portanto; consequentemente; como resultado; indica que o último segue o primeiro e que o último é frequentemente causado pelo primeiro}
  \end{phonetics}
\end{entry}

\begin{entry}{亏}{3}{⼆}
  \begin{phonetics}{亏}{kui1}[][HSK 5]
    \definition{adv.}{felizmente; por sorte; graças a | contrariamente, expressando sarcasmo}
    \definition{s.}{prejuízo}
    \definition{v.}{perder dinheiro, etc.; ter um déficit; ter prejuízo | ter falta de; ser deficiente; carecer de | tratar injustamente; causar prejuízo; trair a confiança}
  \end{phonetics}
\end{entry}

\begin{entry}{亿}{3}{⼈}
  \begin{phonetics}{亿}{yi4}[][HSK 2]
    \definition*{s.}{Sobrenome Yi}
    \definition{num.}{cem milhões; 100.000.000; 1.0000.0000}
  \end{phonetics}
\end{entry}

\begin{entry}{凡}{3}{⼏}
  \begin{phonetics}{凡}{fan2}
    \definition*{s.}{Sobrenome Fan}
    \definition{adj.}{comum; ordinário}
    \definition{adv.}{qualquer; todos; todo | em tudo; completamente}
    \definition{s.}{este mundo mortal; a terra | o mundo secular; refere-se ao mundo humano | uma nota da escala em Gongchepu (工尺谱), correspondente a 4 na notação musical numerada | ideia geral; esboço}
  \seealsoref{工尺谱}{gong1 che3 pu3}
  \end{phonetics}
\end{entry}

\begin{entry}{凡是}{3,9}{⼏、⽇}
  \begin{phonetics}{凡是}{fan2shi4}[][HSK 6]
    \definition{adv.}{todos; qualquer; cada; resumir tudo dentro de um determinado âmbito}
  \end{phonetics}
\end{entry}

\begin{entry}{勺}{3}{⼓}
  \begin{phonetics}{勺}{shao2}[][HSK 6]
    \definition{clas.}{shao; uma unidade tradicional de volume, igual a 0,01 市升, e equivalente a 1 centilitro ou 0,018 \emph{pint}}
    \definition{s.}{colher; concha}
  \seealsoref{市升}{shi4sheng1}
  \end{phonetics}
\end{entry}

\begin{entry}{千}{3}{⼗}
  \begin{phonetics}{千}{qian1}[][HSK 2]
    \definition*{s.}{Sobrenome Qian}
    \definition{num.}{mil; 1.000; 1000 | a grande quantidade de; um grande número de}
  \end{phonetics}
\end{entry}

\begin{entry}{千万}{3,3}{⼗、⼀}
  \begin{phonetics}{千万}{qian1wan4}[][HSK 3]
    \definition{adv.}{(usado para indicar desejos fortes) por todos os meios; sob quaisquer circunstâncias; expressa uma exortação sincera, equivalente a 务必}
    \definition{num.}{dez milhões; 10.000.000; 1000.0000; milhões e milhões; um número aproximado, indicando um grande número}
  \seealsoref{务必}{wu4bi4}
  \end{phonetics}
\end{entry}

\begin{entry}{千千万万}{3,3,3,3}{⼗、⼗、⼀、⼀}
  \begin{phonetics}{千千万万}{qian1qian1wan4wan4}
    \definition{num.}{inumerável | números incontáveis | milhares e milhares}
  \end{phonetics}
\end{entry}

\begin{entry}{千古}{3,5}{⼗、⼝}
  \begin{phonetics}{千古}{qian1gu3}
    \definition{adv.}{por toda a eternidade | em todas as idades}
    \definition{s.}{eternidade (usada em um dístico elegíaco, coroa de flores, etc., dedicada aos mortos)}
  \end{phonetics}
\end{entry}

\begin{entry}{千年}{3,6}{⼗、⼲}
  \begin{phonetics}{千年}{qian1nian2}
    \definition{s.}{milênio}
  \end{phonetics}
\end{entry}

\begin{entry}{千克}{3,7}{⼗、⼗}
  \begin{phonetics}{千克}{qian1 ke4}[][HSK 2]
    \definition{clas.}{kg; quilo; quilograma; 1 quilograma equivale a 1.000 gramas, ou 2 jin (斤)}
  \seealsoref{斤}{jin1}
  \end{phonetics}
\end{entry}

\begin{entry}{卫}{3}{⼙}
  \begin{phonetics}{卫}{wei4}
    \definition*{s.}{Wei, um estado da Dinastia Zhou | Sobrenome Wei}
    \definition{s.}{uma palavra usada no nome do lugar | outro nome para um burro}
    \definition{v.}{defender; guardar; proteger}
  \end{phonetics}
\end{entry}

\begin{entry}{卫生}{3,5}{⼙、⽣}
  \begin{phonetics}{卫生}{wei4 sheng1}[][HSK 3]
    \definition{adj.}{bom para a saúde; higiênico; limpo; capaz de prevenir doenças e benéfico para a saúde}
    \definition{s.}{higiene; saneamento; situação limpa}
  \end{phonetics}
\end{entry}

\begin{entry}{卫生巾}{3,5,3}{⼙、⽣、⼱}
  \begin{phonetics}{卫生巾}{wei4sheng1jin1}
    \definition{s.}{absorvente higiênico}
  \end{phonetics}
\end{entry}

\begin{entry}{卫生厅}{3,5,4}{⼙、⽣、⼚}
  \begin{phonetics}{卫生厅}{wei4sheng1ting1}
    \definition*{s.}{Departamento de Saúde (da Província)}
  \end{phonetics}
\end{entry}

\begin{entry}{卫生防疫}{3,5,6,9}{⼙、⽣、⾩、⽧}
  \begin{phonetics}{卫生防疫}{wei4sheng1 fang2yi4}
    \definition{s.}{prevenção contra a epidemia}
  \end{phonetics}
\end{entry}

\begin{entry}{卫生局}{3,5,7}{⼙、⽣、⼫}
  \begin{phonetics}{卫生局}{wei4sheng1ju2}
    \definition*{s.}{Departamento de Saúde | Escritório de Saúde}
  \end{phonetics}
\end{entry}

\begin{entry}{卫生纸}{3,5,7}{⼙、⽣、⽷}
  \begin{phonetics}{卫生纸}{wei4sheng1zhi3}
    \definition{s.}{papel higiênico}
  \end{phonetics}
\end{entry}

\begin{entry}{卫生间}{3,5,7}{⼙、⽣、⾨}
  \begin{phonetics}{卫生间}{wei4sheng1jian1}[][HSK 3]
    \definition[间,个]{s.}{banheiro; sanitário; \emph{toilette}; quartos com instalações sanitárias em hotéis ou residências}
  \end{phonetics}
\end{entry}

\begin{entry}{卫生套}{3,5,10}{⼙、⽣、⼤}
  \begin{phonetics}{卫生套}{wei4sheng1tao4}
    \definition[只]{s.}{preservativo | camisinha}
  \end{phonetics}
\end{entry}

\begin{entry}{卫生部}{3,5,10}{⼙、⽣、⾢}
  \begin{phonetics}{卫生部}{wei4sheng1bu4}
    \definition*{s.}{Ministério da Saúde}
  \end{phonetics}
\end{entry}

\begin{entry}{卫生球}{3,5,11}{⼙、⽣、⽟}
  \begin{phonetics}{卫生球}{wei4sheng1qiu2}
    \definition{s.}{naftalina}
  \end{phonetics}
\end{entry}

\begin{entry}{卫生棉}{3,5,12}{⼙、⽣、⽊}
  \begin{phonetics}{卫生棉}{wei4sheng1mian2}
    \definition{s.}{absorvente | algodão absorvente esterilizado (usado para curativos ou limpeza de feridas) | absorvente tampão}
  \end{phonetics}
\end{entry}

\begin{entry}{卫生署}{3,5,13}{⼙、⽣、⽹}
  \begin{phonetics}{卫生署}{wei4sheng1shu3}
    \definition*{s.}{Agência de Saúde (ou Escritório, ou Departamento)}
  \end{phonetics}
\end{entry}

\begin{entry}{卫星}{3,9}{⼙、⽇}
  \begin{phonetics}{卫星}{wei4xing1}[][HSK 5]
    \definition[个,颗]{s.}{satélite; lua; corpos celestes orbitando planetas | satélite artificial | algo que gira em torno de um centro}
  \end{phonetics}
\end{entry}

\begin{entry}{叉}{3}{⼜}
  \begin{phonetics}{叉}{cha1}[][HSK 5]
    \definition{s.}{garfo; forquilha | símbolo de cruz, ``×''}
    \definition{v.}{trabalhar com um garfo; garfar; pegar coisas com um garfo}
  \end{phonetics}
  \begin{phonetics}{叉}{cha2}
    \definition{v.}{bloquear; emperrar; congestionar}
  \end{phonetics}
  \begin{phonetics}{叉}{cha3}
    \definition{v.}{separar de modo a formar uma bifurcação; bifurcar}
  \end{phonetics}
\end{entry}

\begin{entry}{叉子}{3,3}{⼜、⼦}
  \begin{phonetics}{叉子}{cha1zi5}[][HSK 5]
    \definition[把]{s.}{garfo; ferramenta com mais de duas pontas em uma extremidade | tridente; forquilha; ferramentas de agricultura antigas}
  \end{phonetics}
\end{entry}

\begin{entry}{及}{3}{⼃}
  \begin{phonetics}{及}{ji2}
    \definition{conj.}{e | bem como}
  \end{phonetics}
\end{entry}

\begin{entry}{及时}{3,7}{⼃、⽇}
  \begin{phonetics}{及时}{ji2shi2}[][HSK 3]
    \definition{adj.}{oportuno; na hora certa; adequado; na ocasião certa}
    \definition{adv.}{prontamente; sem demora; imediatamente}
  \end{phonetics}
\end{entry}

\begin{entry}{及格}{3,10}{⼃、⽊}
  \begin{phonetics}{及格}{ji2ge2}[][HSK 4]
    \definition{v.+compl.}{passar; passar em um teste, exame, etc.}
  \end{phonetics}
\end{entry}

\begin{entry}{口}{3}{⼝}[Kangxi 30]
  \begin{phonetics}{口}{kou3}[][HSK 1]
    \definition*{s.}{Sobrenome Kou}
    \definition{clas.}{usado para coisas com bocas (pessoas, animais domésticos, canhões, etc.) | usado para mordidas ou bocados | usado para idiomas}
    \definition{s.}{boca | borda; boca; o espaço externo ao recipiente | saída; entrada; local de entrada e saída | o gosto de alguém | corte; buraco; ferida |  a borda de uma faca; lâminas de facas, espadas, tesouras, etc. | a idade de um animal de tração | seção; departamento; sistema integrado de departamentos relacionados | conversa, discurso; pronunciamento; referência à fala | um portão da Grande Muralha (frequentemente usado em nomes de lugares)}
  \end{phonetics}
\end{entry}

\begin{entry}{口号}{3,5}{⼝、⼝}
  \begin{phonetics}{口号}{kou3 hao4}[][HSK 5]
    \definition[个]{s.}{\emph{slogan}; palavra de ordem; lema}
  \end{phonetics}
\end{entry}

\begin{entry}{口吃}{3,6}{⼝、⼝}
  \begin{phonetics}{口吃}{kou3chi1}
    \definition{s.}{gagueira; espasmofemia; balbucinato; mogilalia; battarismo; battarismo; iscnofonia; pselismo; o fenômeno de repetir palavras ou interromper frases ao falar é um defeito habitual de linguagem comumente conhecido como gagueira}
  \end{phonetics}
\end{entry}

\begin{entry}{口吃病}{3,6,10}{⼝、⼝、⽧}
  \begin{phonetics}{口吃病}{kou3chi1 bing4}
    \definition{s.}{doença da gagueira}
  \end{phonetics}
\end{entry}

\begin{entry}{口试}{3,8}{⼝、⾔}
  \begin{phonetics}{口试}{kou3 shi4}[][HSK 6]
    \definition{s.}{exame oral (ou teste); um tipo de exame que exige que os candidatos respondam a perguntas oralmente (em oposição a 笔试)}
    \definition{v.}{examinar oralmente}
  \seealsoref{笔试}{bi3 shi4}
  \end{phonetics}
\end{entry}

\begin{entry}{口语}{3,9}{⼝、⾔}
  \begin{phonetics}{口语}{kou3 yu3}[][HSK 4]
    \definition[门]{s.}{linguagem oral; linguagem falada; linguagem coloquial; linguagem usada em conversas}
  \end{phonetics}
\end{entry}

\begin{entry}{口音}{3,9}{⼝、⾳}
  \begin{phonetics}{口音}{kou3yin1}
    \definition{s.}{sons da fala oral (linguística)}
  \end{phonetics}
  \begin{phonetics}{口音}{kou3yin5}
    \definition{s.}{sotaque | voz}
  \end{phonetics}
\end{entry}

\begin{entry}{口香糖}{3,9,16}{⼝、⾹、⽶}
  \begin{phonetics}{口香糖}{kou3xiang1tang2}
    \definition{s.}{goma de mascar | chiclete}
  \end{phonetics}
\end{entry}

\begin{entry}{口袋}{3,11}{⼝、⾐}
  \begin{phonetics}{口袋}{kou3dai4}[][HSK 4]
    \definition[个]{s.}{bolso | saco; sacola; artigos de tecido ou couro}
  \end{phonetics}
\end{entry}

\begin{entry}{口袋妖怪}{3,11,7,8}{⼝、⾐、⼥、⼼}
  \begin{phonetics}{口袋妖怪}{kou3dai4 yao1guai4}
    \definition*{s.}{Pokémon}
  \end{phonetics}
\end{entry}

\begin{entry}{土}{3}{⼟}[Kangxi 32]
  \begin{phonetics}{土}{tu3}[][HSK 3,6]
    \definition*{s.}{Sobrenome Tu}
    \definition{adj.}{local; nativo; local com características regionais| caseiro; indígena; o que é tradicional no país; popular | não refinado; não esclarecido; não está na moda; não é popular}
    \definition[堆,捧,层]{s.}{solo; terra | terra; território | ópio bruto | cidade natal; terra natal; pátria}
  \end{phonetics}
\end{entry}

\begin{entry}{土地}{3,6}{⼟、⼟}
  \begin{phonetics}{土地}{tu3di4}[][HSK 4]
    \definition[片,块]{s.}{terra; solo; chão; superfície terrestre da Terra usada para cultivar, construir edifícios e viver | território}
  \end{phonetics}
  \begin{phonetics}{土地}{tu3di5}
    \definition{s.}{deus da audeia; deus local; \emph{genius loci} deidade protetora de um local; (superstição) refere-se ao deus da terra que governa uma pequena área}
  \end{phonetics}
\end{entry}

\begin{entry}{土豆}{3,7}{⼟、⾖}
  \begin{phonetics}{土豆}{tu3dou4}[][HSK 5]
    \definition[个,片,块,斤]{s.}{batata; denominação comum da batata}
  \end{phonetics}
\end{entry}

\begin{entry}{土豆泥}{3,7,8}{⼟、⾖、⽔}
  \begin{phonetics}{土豆泥}{tu3dou4ni2}
    \definition{s.}{purê de batatas}
  \end{phonetics}
\end{entry}

\begin{entry}{土鸡}{3,7}{⼟、⿃}
  \begin{phonetics}{土鸡}{tu3ji1}
    \definition{s.}{galinha caipira}
  \end{phonetics}
\end{entry}

\begin{entry}{士}{3}{⼠}[Kangxi 33]
  \begin{phonetics}{士}{shi4}
    \definition*{s.}{Sobrenome Shi}
    \definition[位,名,个]{s.}{soldado; militar | oficial não comissionado; primeira classe de soldados | pessoa treinada em uma determinada área; algum tipo de técnico | pessoa (louvável) | bacharel (na China antiga) | classe social, entre os oficiais, 大夫, e o povo comum, 庶民 | estudioso | guarda-costas, uma das peças do xadrez chinês}
  \seealsoref{大夫}{da4fu1}
  \seealsoref{庶民}{shu4min2}
  \end{phonetics}
\end{entry}

\begin{entry}{士兵}{3,7}{⼠、⼋}
  \begin{phonetics}{士兵}{shi4bing1}[][HSK 4]
    \definition[名,个]{s.}{soldado; militar; termo coletivo para oficiais não comissionados e soldados; os membros mais jovens do exército}
  \end{phonetics}
\end{entry}

\begin{entry}{夕}{3}{⼣}[Kangxi 36]
  \begin{phonetics}{夕}{xi1}
    \definition*{s.}{Sobrenome Xi}
    \definition{s.}{pôr do sol; crepúsculo | tarde; noite}
  \end{phonetics}
\end{entry}

\begin{entry}{夕阳}{3,6}{⼣、⾩}
  \begin{phonetics}{夕阳}{xi1yang2}
    \definition{s.}{pôr do sol}
  \seealsoref{日出}{ri4chu1}
  \end{phonetics}
\end{entry}

\begin{entry}{大}{3}{⼤}[Kangxi 37]
  \begin{phonetics}{大}{da4}[][HSK 1]
    \definition*{s.}{Sobrenome Da}
    \definition{adj.}{grande; amplo; grande em volume, área, etc. | mais velho; em primeiro lugar no ranking | tamanho; descreve o grau de grandeza | usado em certas épocas do ano, condições climáticas, feriados ou antes de um determinado momento, para enfatizar | o tempo mais distante; há muito tempo}
    \definition{adv.}{grandemente; totalmente; expressa um grau muito profundo | não muito; não frequentemente; usado após 不, indica um grau baixo ou poucas vezes}
    \definition{s.}{adulto; crescido; pessoas idosas | pai | irmão do pai de alguém; tio}
  \seealsoref{不}{bu4}
  \end{phonetics}
  \begin{phonetics}{大}{dai4}
    \definition{s.}{usado em 大夫: médico, doutor | usado em 大王: grande rei}
  \seealsoref{大夫}{dai4fu5}
  \seealsoref{大王}{dai4wang5}
  \end{phonetics}
\end{entry}

\begin{entry}{大人}{3,2}{⼤、⼈}
  \begin{phonetics}{大人}{da4 ren2}[][HSK 2]
    \definition[个,位]{s.}{senhor; ilustre; sua excelência; antigo título honorífico para funcionários públicos | adulto; crescido; maduro;}
  \end{phonetics}
\end{entry}

\begin{entry}{大力}{3,2}{⼤、⼒}
  \begin{phonetics}{大力}{da4 li4}[][HSK 6]
    \definition{adv.}{energicamente; vigorosamente; indica uso de grande força}
    \definition{s.}{grande força, poder}
  \end{phonetics}
\end{entry}

\begin{entry}{大于}{3,3}{⼤、⼆}
  \begin{phonetics}{大于}{da4 yu2}[][HSK 5]
    \definition{v.}{ser maior, mais numeroso, mais importante, etc. do que}
  \end{phonetics}
\end{entry}

\begin{entry}{大口}{3,3}{⼤、⼝}
  \begin{phonetics}{大口}{da4kou3}
    \definition{s.}{grande bocado (de comida, bebida, fumo, etc.)}
  \end{phonetics}
\end{entry}

\begin{entry}{大大}{3,3}{⼤、⼤}
  \begin{phonetics}{大大}{da4 da4}[][HSK 2]
    \definition{adv.}{grandemente; enormemente; enfatizar grande quantidade ou grau profundo}
  \end{phonetics}
\end{entry}

\begin{entry}{大小}{3,3}{⼤、⼩}
  \begin{phonetics}{大小}{da4 xiao3}[][HSK 2]
    \definition{adv.}{no mínimo; grande ou pequeno (geralmente pequeno), significa que ainda pode ser considerado}
    \definition[家]{s.}{tamanho; o grau de tamanho | ordem de senioridade; hierarquia | adultos e crianças | grande ou pequeno}
  \end{phonetics}
\end{entry}

\begin{entry}{大门}{3,3}{⼤、⾨}
  \begin{phonetics}{大门}{da4 men2}[][HSK 2]
    \definition{s.}{portão; entrada; portão grande, referindo-se especificamente ao portão principal de um edifício (como uma casa, pátio ou parque) que dá para a rua (em contraste com o segundo portão e as portas das várias divisões)}
  \end{phonetics}
\end{entry}

\begin{entry}{大马}{3,3}{⼤、⾺}
  \begin{phonetics}{大马}{da4ma3}
    \definition*{s.}{Malásia}
  \end{phonetics}
\end{entry}

\begin{entry}{大厅}{3,4}{⼤、⼚}
  \begin{phonetics}{大厅}{da4 ting1}[][HSK 5]
    \definition{s.}{\emph{hall}; saguão, uma sala grande para reuniões ou atividades em um edifício de grande porte}
  \end{phonetics}
\end{entry}

\begin{entry}{大夫}{3,4}{⼤、⼤}
  \begin{phonetics}{大夫}{da4fu1}
    \definition[个,位,名]{s.}{oficial sênior (na China Imperial)}
  \end{phonetics}
  \begin{phonetics}{大夫}{dai4fu5}[][HSK 3]
    \definition[个,位,名]{s.}{médico, doutor}
  \end{phonetics}
\end{entry}

\begin{entry}{大巴}{3,4}{⼤、⼰}
  \begin{phonetics}{大巴}{da4 ba1}[][HSK 4]
    \definition{s.}{ônibus}
  \end{phonetics}
\end{entry}

\begin{entry}{大方}{3,4}{⼤、⽅}
  \begin{phonetics}{大方}{da4fang5}[][HSK 4]
    \definition{adj.}{generoso | não afetado; natural e equilibrado |  de bom gosto}
  \end{phonetics}
\end{entry}

\begin{entry}{大王}{3,4}{⼤、⽟}
  \begin{phonetics}{大王}{da4wang2}
    \definition{s.}{rei; magnata | pessoa da mais alta classe ou habilidade em algo; ás | barões | pessoa com habilidade especializada em algo}
  \end{phonetics}
  \begin{phonetics}{大王}{dai4wang5}
    \definition{s.}{magnata; barões | barão ladrão (em ópera, histórias antigas)}
  \end{phonetics}
\end{entry}

\begin{entry}{大众}{3,6}{⼤、⼈}
  \begin{phonetics}{大众}{da4 zhong4}[][HSK 4]
    \definition{s.}{massas; população; pessoas comuns; público em geral}
  \end{phonetics}
\end{entry}

\begin{entry}{大伙儿}{3,6,2}{⼤、⼈、⼉}
  \begin{phonetics}{大伙儿}{da4huo3r5}[][HSK 5]
    \definition{pron.}{todos nós; todos vocês; todo mundo; todos | equivalente a 大家}
  \seealsoref{大家}{da4jia1}
  \end{phonetics}
\end{entry}

\begin{entry}{大会}{3,6}{⼤、⼈}
  \begin{phonetics}{大会}{da4 hui4}[][HSK 4]
    \definition{s.}{sessão plenária; reunião geral de membros; reuniões convocadas por partidos políticos socialistas | reunião de massa; comício de massa}
  \end{phonetics}
\end{entry}

\begin{entry}{大全}{3,6}{⼤、⼊}
  \begin{phonetics}{大全}{da4quan2}
    \definition{s.}{coleção abrangente}
  \end{phonetics}
\end{entry}

\begin{entry}{大后天}{3,6,4}{⼤、⼝、⼤}
  \begin{phonetics}{大后天}{da4 hou4 tian1}
    \definition{adv.}{daqui a três dias}
  \end{phonetics}
\end{entry}

\begin{entry}{大多}{3,6}{⼤、⼣}
  \begin{phonetics}{大多}{da4 duo1}[][HSK 4]
    \definition{adv.}{majoritariamente; em sua maior parte; em sua maioria; em grande parte}
  \end{phonetics}
\end{entry}

\begin{entry}{大多数}{3,6,13}{⼤、⼣、⽁}
  \begin{phonetics}{大多数}{da4 duo1 shu4}[][HSK 2]
    \definition{s.}{grande maioria; vasta maioria; a maior parte; mais da metade, um número significativo}
  \end{phonetics}
\end{entry}

\begin{entry}{大妈}{3,6}{⼤、⼥}
  \begin{phonetics}{大妈}{da4 ma1}[][HSK 4]
    \definition{s.}{tia; esposa do irmão mais velho do pai | tia (homenagem às mulheres idosas)}
  \end{phonetics}
\end{entry}

\begin{entry}{大师}{3,6}{⼤、⼱}
  \begin{phonetics}{大师}{da4 shi1}[][HSK 6]
    \definition*{s.}{Grande Mestre, título de cortesia usado para se dirigir a um monge budista}
    \definition{s.}{grande mestre; mestre; maestro; uma pessoa com realizações profundas}
  \end{phonetics}
\end{entry}

\begin{entry}{大戏}{3,6}{⼤、⼽}
  \begin{phonetics}{大戏}{da4xi4}
    \definition*{s.}{Drama, Ópera Chinesa}
  \end{phonetics}
\end{entry}

\begin{entry}{大爷}{3,6}{⼤、⽗}
  \begin{phonetics}{大爷}{da4 ye5}[][HSK 4]
    \definition{s.}{irmão mais velho do pai; tio | tio (homenagem aos homens mais velhos)}
  \end{phonetics}
\end{entry}

\begin{entry}{大米}{3,6}{⼤、⽶}
  \begin{phonetics}{大米}{da4 mi3}[][HSK 6]
    \definition[颗,粒,斤,包,袋]{s.}{arroz; arroz descascado; arroz bom}
  \end{phonetics}
\end{entry}

\begin{entry}{大约}{3,6}{⼤、⽷}
  \begin{phonetics}{大约}{da4yue1}[][HSK 3]
    \definition{adv.}{aproximadamente; sobre; estimativa não muito precisa| provavelmente; expressar suposições sobre a situação}
  \end{phonetics}
\end{entry}

\begin{entry}{大自然}{3,6,12}{⼤、⾃、⽕}
  \begin{phonetics}{大自然}{da4 zi4 ran2}[][HSK 2]
    \definition{s.}{natureza}
  \end{phonetics}
\end{entry}

\begin{entry}{大衣}{3,6}{⼤、⾐}
  \begin{phonetics}{大衣}{da4 yi1}[][HSK 2]
    \definition[件,个]{s.}{sobretudo; casaco; casaco ocidental mais comprido}
  \end{phonetics}
\end{entry}

\begin{entry}{大声}{3,7}{⼤、⼠}
  \begin{phonetics}{大声}{da4 sheng1}[][HSK 2]
    \definition{adj.}{alto; volume alto; em voz alta}
  \end{phonetics}
\end{entry}

\begin{entry}{大批}{3,7}{⼤、⼿}
  \begin{phonetics}{大批}{da4 pi1}[][HSK 6]
    \definition{num.}{grandes quantidades de; exércitos; inundações}[大批书籍被印刷出来。___Grandes quantidades de livros foram impressas.]
  \end{phonetics}
\end{entry}

\begin{entry}{大纲}{3,7}{⼤、⽷}
  \begin{phonetics}{大纲}{da4 gang1}[][HSK 5]
    \definition{s.}{esboço; compêndio; programa de estudos; resumo; fundamentos da organização sistemática de conteúdos (livros, discursos, programas, etc.)}
  \end{phonetics}
\end{entry}

\begin{entry}{大豆}{3,7}{⼤、⾖}
  \begin{phonetics}{大豆}{da4dou4}
    \definition{s.}{soja}
  \end{phonetics}
\end{entry}

\begin{entry}{大陆}{3,7}{⼤、⾩}
  \begin{phonetics}{大陆}{da4 lu4}[][HSK 4]
    \definition*{s.}{China continental; refere-se especificamente à vasta porção terrestre do território da China}
    \definition[个,块]{s.}{terra firme; continente; vasta extensão de terra}
  \end{phonetics}
\end{entry}

\begin{entry}{大事}{3,8}{⼤、⼅}
  \begin{phonetics}{大事}{da4 shi4}[][HSK 5]
    \definition[件,桩]{s.}{grande evento; grande acontecimento; assunto importante; grande questão; algo importante | situação geral | em grande escala; em grande estilo; em grande parte}
  \end{phonetics}
\end{entry}

\begin{entry}{大使}{3,8}{⼤、⼈}
  \begin{phonetics}{大使}{da4 shi3}[][HSK 6]
    \definition[位,任]{s.}{embaixador; o representante diplomático de mais alto nível enviado por um país a outro país}
  \seealsoref{全称特命全权大使}{quan2cheng1 te4ming4 quan2quan2 da4shi3}
  \end{phonetics}
\end{entry}

\begin{entry}{大使馆}{3,8,11}{⼤、⼈、⾷}
  \begin{phonetics}{大使馆}{da4shi3guan3}[][HSK 3]
    \definition[座,个]{s.}{embaixada; uma representação diplomática de um país em outro país, chefiada por um embaixador}
  \end{phonetics}
\end{entry}

\begin{entry}{大姐}{3,8}{⼤、⼥}
  \begin{phonetics}{大姐}{da4 jie3}[][HSK 4]
    \definition[个]{s.}{irmã mais velha (também um termo educado para se dirigir a uma garota ou mulher um pouco mais velha do que a pessoa que fala)}
  \end{phonetics}
\end{entry}

\begin{entry}{大学}{3,8}{⼤、⼦}
  \begin{phonetics}{大学}{da4 xue2}[][HSK 1]
    \definition[所,座]{s.}{universidade; faculdade; tipo de instituição de ensino superior que, na China, geralmente se refere a uma universidade abrangente}
  \end{phonetics}
\end{entry}

\begin{entry}{大学生}{3,8,5}{⼤、⼦、⽣}
  \begin{phonetics}{大学生}{da4 xue2 sheng1}[][HSK 1]
    \definition[名,个]{s.}{estudante universitário; estudante de faculdade; estudantes de graduação ou cursos técnicos em instituições de ensino superior}
  \end{phonetics}
\end{entry}

\begin{entry}{大抵}{3,8}{⼤、⼿}
  \begin{phonetics}{大抵}{da4di3}
    \definition{adv.}{no geral; de um modo geral; provavelmente; principalmente}
  \end{phonetics}
\end{entry}

\begin{entry}{大规模}{3,8,14}{⼤、⾒、⽊}
  \begin{phonetics}{大规模}{da4 gui1 mo2}[][HSK 4]
    \definition{adj.}{em larga escala; extensivo; maciço; massa}
    \definition{adj.}{em larga escala; extensivo; maciço; massivo}
  \end{phonetics}
\end{entry}

\begin{entry}{大雨}{3,8}{⼤、⾬}
  \begin{phonetics}{大雨}{da4yu3}
    \definition[场]{s.}{chuva pesada, forte}
  \end{phonetics}
\end{entry}

\begin{entry}{大前天}{3,9,4}{⼤、⼑、⼤}
  \begin{phonetics}{大前天}{da4qian2tian1}
    \definition{adv.}{três dias atrás}
  \end{phonetics}
\end{entry}

\begin{entry}{大型}{3,9}{⼤、⼟}
  \begin{phonetics}{大型}{da4xing2}[][HSK 4]
    \definition{adj.}{grande; em larga escala; tamanho e volume grandes | larga escala (importante e influente)}
  \end{phonetics}
\end{entry}

\begin{entry}{大城}{3,9}{⼤、⼟}
  \begin{phonetics}{大城}{da4cheng2}
    \definition*[个,座]{s.}{Condado de Dacheng em Langfang 廊坊, Hebei | Município de Tacheng no condado de Changhua | Condado de Changhua, Taiwan}
  \seealsoref{廊坊}{lang2fang2}
  \end{phonetics}
\end{entry}

\begin{entry}{大奖赛}{3,9,14}{⼤、⼤、⾙}
  \begin{phonetics}{大奖赛}{da4 jiang3 sai4}[][HSK 5]
    \definition{s.}{grande competição; grande prêmio; \emph{grand prix}}
  \end{phonetics}
\end{entry}

\begin{entry}{大战}{3,9}{⼤、⼽}
  \begin{phonetics}{大战}{da4zhan4}
    \definition{s.}{guerra}
    \definition{v.}{guerrear | lutar em uma guerra}
  \end{phonetics}
\end{entry}

\begin{entry}{大洋洲}{3,9,9}{⼤、⽔、⽔}
  \begin{phonetics}{大洋洲}{da4yang2zhou1}
    \definition*{s.}{Oceania}
  \end{phonetics}
\end{entry}

\begin{entry}{大神}{3,9}{⼤、⽰}
  \begin{phonetics}{大神}{da4shen2}
    \definition{s.}{deidade | (gíria da Internet) guru | \emph{expert} | gênio}
  \end{phonetics}
\end{entry}

\begin{entry}{大胆}{3,9}{⼤、⾁}
  \begin{phonetics}{大胆}{da4 dan3}[][HSK 5]
    \definition{adj.}{ousado; atrevido; audacioso; corajoso; destemido}
  \end{phonetics}
\end{entry}

\begin{entry}{大哥}{3,10}{⼤、⼝}
  \begin{phonetics}{大哥}{da4 ge1}[][HSK 4]
    \definition{s.}{irmão mais velho | \emph{big brother} (endereço educado para um homem da mesma idade que você) | líder de gangue; pessoa mais poderosa em uma organização que realiza atividades ilegais na sociedade}
  \end{phonetics}
\end{entry}

\begin{entry}{大家}{3,10}{⼤、⼧}
  \begin{phonetics}{大家}{da4jia1}[][HSK 2]
    \definition{pron.}{todos; toda a gente; refere-se a todas as pessoas dentro de um determinado âmbito}
    \definition{s.}{grande mestre; autoridade; especialista renomado | família nobre; família rica e influente; família tradicional}
  \end{phonetics}
\end{entry}

\begin{entry}{大海}{3,10}{⼤、⽔}
  \begin{phonetics}{大海}{da4 hai3}[][HSK 2]
    \definition{s.}{o mar; o oceano; o mar aberto, ou seja, a parte do oceano que não está fechada entre cabos nem incluída em estreitos}
  \end{phonetics}
\end{entry}

\begin{entry}{大脑}{3,10}{⼤、⾁}
  \begin{phonetics}{大脑}{da4 nao3}[][HSK 5]
    \definition{s.}{cérebro; encéfalo}
  \end{phonetics}
\end{entry}

\begin{entry}{大致}{3,10}{⼤、⾄}
  \begin{phonetics}{大致}{da4zhi4}[][HSK 5]
    \definition{adj.}{geral; no todo}
    \definition{adv.}{grosso modo; aproximadamente; mais ou menos; indica uma estimativa aproximada da situação}
  \end{phonetics}
\end{entry}

\begin{entry}{大部分}{3,10,4}{⼤、⾢、⼑}
  \begin{phonetics}{大部分}{da4 bu4 fen4}[][HSK 2]
    \definition[把]{s.}{a maioria; a maior parte; em grande parte; refere-se a uma quantidade superior a metade do total}
  \end{phonetics}
\end{entry}

\begin{entry}{大都}{3,10}{⼤、⾢}
  \begin{phonetics}{大都}{da4 dou1}[][HSK 5]
  \end{phonetics}
  \begin{phonetics}{大都}{da4 du1}
    \definition{adv.}{em sua maior parte; na maior parte; indica que a maioria das pessoas ou coisas em um determinado intervalo tem a mesma natureza e características | também pronunciado como \dpy{da4dou1} na língua falada}
  \end{phonetics}
\end{entry}

\begin{entry}{大象}{3,11}{⼤、⾗}
  \begin{phonetics}{大象}{da4xiang4}[][HSK 5]
    \definition[只,头,群,个]{s.}{elefante}
  \end{phonetics}
\end{entry}

\begin{entry}{大黄}{3,11}{⼤、⿈}
  \begin{phonetics}{大黄}{da4huang2}
    \definition{s.}{ruibarbo chinês}
  \end{phonetics}
\end{entry}

\begin{entry}{大猩猩}{3,12,12}{⼤、⽝、⽝}
  \begin{phonetics}{大猩猩}{da4xing1xing5}
    \definition{s.}{gorila}
  \end{phonetics}
\end{entry}

\begin{entry}{大街}{3,12}{⼤、⾏}
  \begin{phonetics}{大街}{da4 jie1}[][HSK 6]
    \definition[条,个]{s.}{avenida; rua; rua principal}
  \end{phonetics}
\end{entry}

\begin{entry}{大道}{3,12}{⼤、⾡}
  \begin{phonetics}{大道}{da4 dao4}[][HSK 6]
    \definition*{s.}{O Grande Tao; O Grande Caminho}
    \definition[条]{s.}{estrada principal | o caminho da justiça | avenida | rua principal}
  \end{phonetics}
\end{entry}

\begin{entry}{大量}{3,12}{⼤、⾥}
  \begin{phonetics}{大量}{da4 liang4}[][HSK 2]
    \definition{adj.}{numeroso; em grande quantidade; grande em número ou quantidade | generoso; magnânimo; descreve uma pessoa que não fica zangada quando os outros cometem erros e que costuma perdoar os outros}
  \end{phonetics}
\end{entry}

\begin{entry}{大楼}{3,13}{⼤、⽊}
  \begin{phonetics}{大楼}{da4 lou2}[][HSK 4]
    \definition[座,幢]{s.}{edifício; mansão; edifício de vários andares disponível para uso residencial e comercial}
  \end{phonetics}
\end{entry}

\begin{entry}{大概}{3,13}{⼤、⽊}
  \begin{phonetics}{大概}{da4gai4}[][HSK 3]
    \definition{adj.}{geral; grosseiro; aproximado; não é muito preciso ou muito detalhado}
    \definition{adv.}{sobre; provavelmente; estimativas ou suposições imprecisas sobre eventos, quantidades, tempo, localização, etc.| geralmente; brevemente; não muito seriamente, casualmente; não muito cuidadosamente}
    \definition{s.}{ideia geral; esboço geral; conteúdo geral ou situação}
  \end{phonetics}
\end{entry}

\begin{entry}{大腿}{3,13}{⼤、⾁}
  \begin{phonetics}{大腿}{da4tui3}
    \definition{s.}{coxa}
  \end{phonetics}
\end{entry}

\begin{entry}{大蒜}{3,13}{⼤、⾋}
  \begin{phonetics}{大蒜}{da4suan4}
    \definition[瓣,头]{s.}{alho}
  \end{phonetics}
\end{entry}

\begin{entry}{大熊猫}{3,14,11}{⼤、⽕、⽝}
  \begin{phonetics}{大熊猫}{da4 xiong2 mao1}[][HSK 5]
    \definition{s.}{panda gigante}
  \end{phonetics}
\end{entry}

\begin{entry}{大赛}{3,14}{⼤、⾙}
  \begin{phonetics}{大赛}{da4 sai4}[][HSK 6]
    \definition{s.}{grande torneio; competição importante; um evento de grande porte e alto nível; um grande evento}
  \end{phonetics}
\end{entry}

\begin{entry}{女}{3}{⼥}[Kangxi 38]
  \begin{phonetics}{女}{nv3}[][HSK 1]
    \definition{adj.}{mulher; feminino (em oposição a 男) | fêmea (de certos animais)}
    \definition{s.}{menina; filha | nü, uma das mansões lunares | mulher}
  \seealsoref{男}{nan2}
  \end{phonetics}
\end{entry}

\begin{entry}{女人}{3,2}{⼥、⼈}
  \begin{phonetics}{女人}{nv3 ren2}[][HSK 1]
    \definition[个,位]{s.}{mulher adulta}
  \end{phonetics}
\end{entry}

\begin{entry}{女儿}{3,2}{⼥、⼉}
  \begin{phonetics}{女儿}{nv3'er2}[][HSK 1]
    \definition[个]{s.}{menina; filha}
  \seealsoref{儿子}{er2zi5}
  \end{phonetics}
\end{entry}

\begin{entry}{女士}{3,3}{⼥、⼠}
  \begin{phonetics}{女士}{nv3shi4}[][HSK 4]
    \definition{pron.}{Sra.; Senhorita; Senhora; título honorífico para mulheres (agora usado em contextos diplomáticos)}
    \definition[位,个]{s.}{senhora; madame}
  \end{phonetics}
\end{entry}

\begin{entry}{女子}{3,3}{⼥、⼦}
  \begin{phonetics}{女子}{nv3 zi3}[][HSK 3]
    \definition[位,名,个]{s.}{mulher; feminino; pessoa do sexo feminino}
  \end{phonetics}
\end{entry}

\begin{entry}{女王}{3,4}{⼥、⽟}
  \begin{phonetics}{女王}{nv3wang2}
    \definition{s.}{rainha}
  \end{phonetics}
\end{entry}

\begin{entry}{女生}{3,5}{⼥、⽣}
  \begin{phonetics}{女生}{nv3 sheng1}[][HSK 1]
    \definition[个]{s.}{estudante; aluna; estudante do sexo feminino | menina; jovem mulher}
  \end{phonetics}
\end{entry}

\begin{entry}{女性}{3,8}{⼥、⼼}
  \begin{phonetics}{女性}{nv3 xing4}[][HSK 5]
    \definition[位,名]{s.}{mulher; feminino; feminilidade}
  \end{phonetics}
\end{entry}

\begin{entry}{女朋友}{3,8,4}{⼥、⽉、⼜}
  \begin{phonetics}{女朋友}{nv3 peng2 you5}[][HSK 1]
    \definition{s.}{namorada}
  \end{phonetics}
\end{entry}

\begin{entry}{女孩}{3,9}{⼥、⼦}
  \begin{phonetics}{女孩}{nv3hai2}
    \definition{s.}{menina | garota}
  \end{phonetics}
\end{entry}

\begin{entry}{女孩儿}{3,9,2}{⼥、⼦、⼉}
  \begin{phonetics}{女孩儿}{nv3 hai2r5}[][HSK 1]
    \definition{s.}{garota; menina; atualmente também se refere a mulher adolescente | filha}
  \end{phonetics}
\end{entry}

\begin{entry}{女婿}{3,12}{⼥、⼥}
  \begin{phonetics}{女婿}{nv3xu5}
    \definition{s.}{marido da filha}
  \end{phonetics}
\end{entry}

\begin{entry}{子}{3}{⼦}[Kangxi 39]
  \begin{phonetics}{子}{zi3}
    \definition*{s.}{Sobrenome Zi}
    \definition{adj.}{pequeno; jovem; tenro | subsidiário; subordinado; derivado}
    \definition{clas.}{usado para objetos finos que podem ser pinçados com os dedos}
    \definition{pron.}{você;  antigamente, era uma forma de tratamento respeitosa para se referir a outras pessoas, equivalente a 您}
    \definition[个,位,名]{s.}{filho, criança; antigamente, referia-se aos filhos, mas atualmente refere-se especificamente aos filhos homens | pessoa | antigo título de respeito para um homem culto ou virtuoso; na antiguidade, referia-se especificamente a homens eruditos | visconde; o quarto posto na hierarquia dos cinco títulos feudais da nobreza | ovo | semente | coisas pequenas e duras; pequenos fragmentos ou grãos duros e sólidos | cobre; moeda de cobre | o primeiro dos doze ramos terrestres}
  \seealsoref{您}{nin2}
  \end{phonetics}
  \begin{phonetics}{子}{zi5}[][HSK 1]
    \definition{suf.}{sufixo para substantivos | sufixos de palavras de medida individuais; anexado a certas palavras classificadoras}
  \end{phonetics}
\end{entry}

\begin{entry}{子女}{3,3}{⼦、⼥}
  \begin{phonetics}{子女}{zi3 nv3}[][HSK 3]
    \definition[个]{s.}{crianças; descendentes; filhos e filhas}
  \end{phonetics}
\end{entry}

\begin{entry}{子弹}{3,11}{⼦、⼸}
  \begin{phonetics}{子弹}{zi3dan4}[][HSK 5]
    \definition[粒,颗,发]{s.}{bala; cartucho; munição;}
  \end{phonetics}
\end{entry}

\begin{entry}{寸}{3}{⼨}[Kangxi 41]
  \begin{phonetics}{寸}{cun4}[][HSK 5]
    \definition*{s.}{Sobrenome Cun}
    \definition{adj.}{muito pouco; muito curto; pequeno | Ddialeto: coincidência}
    \definition{clas.}{cun, uma unidade tradicional de comprimento, igual a 0,1 市尺 e equivalente a 3,333 centímetros ou 1,312 polegadas | cun, uma unidade de comprimento (=13 decímetros)}
  \seealsoref{市尺}{shi4 chi3}
  \end{phonetics}
\end{entry}

\begin{entry}{小}{3}{⼩}[Kangxi 42]
  \begin{phonetics}{小}{xiao3}[][HSK 1,2]
    \definition*{s.}{Sobrenome Xiao}
    \definition{adj.}{menor; pequeno; insignificante; pouco; volume, área, quantidade, intensidade, etc. não são grandes | jovem | expressões humildes, referindo-se a si mesmo ou a pessoas ou coisas relacionadas a si mesmo | por um tempo; por um curto período; por um curto período de tempo | o mais novo; o último na ordem de antiguidade; em último lugar na classificação}
    \definition{pref.}{usado antes do sobrenome, nome, posição na família, etc.}
    \definition{s.}{os jovens; pessoas mais jovens | concubina}
  \end{phonetics}
\end{entry}

\begin{entry}{小小}{3,3}{⼩、⼩}
  \begin{phonetics}{小小}{xiao3xiao3}
    \definition{adj.}{muito pequeno}
  \end{phonetics}
\end{entry}

\begin{entry}{小区}{3,4}{⼩、⼖}
  \begin{phonetics}{小区}{xiao3qu1}
    \definition{s.}{conjunto habitacional, comunidade, bairro | célula (telecomunicações)}
  \end{phonetics}
\end{entry}

\begin{entry}{小心}{3,4}{⼩、⼼}
  \begin{phonetics}{小心}{xiao3xin1}[][HSK 2]
    \definition{adj.}{cuidadoso; atento; com cautela}
    \definition{v.}{ter cuidado; ser cauteloso; estar atento; tomar cuidado; prestar atenção}
  \end{phonetics}
\end{entry}

\begin{entry}{小气鬼}{3,4,9}{⼩、⽓、⿁}
  \begin{phonetics}{小气鬼}{xiao3qi4gui3}
    \definition{adj.}{avarento | mão-de-vaca | miserável | pão-duro}
  \end{phonetics}
\end{entry}

\begin{entry}{小白菜}{3,5,11}{⼩、⽩、⾋}
  \begin{phonetics}{小白菜}{xiao3bai2cai4}
    \definition[棵]{s.}{\emph{bok choy} | couve chinesa}
  \end{phonetics}
\end{entry}

\begin{entry}{小众}{3,6}{⼩、⼈}
  \begin{phonetics}{小众}{xiao3zhong4}
    \definition{s.}{minoria da população | nicho (mercado, etc.)}
  \end{phonetics}
\end{entry}

\begin{entry}{小伙子}{3,6,3}{⼩、⼈、⼦}
  \begin{phonetics}{小伙子}{xiao3huo3zi5}[][HSK 4]
    \definition[个]{s.}{rapaz jovem; jovem colega}
  \end{phonetics}
\end{entry}

\begin{entry}{小吃}{3,6}{⼩、⼝}
  \begin{phonetics}{小吃}{xiao3chi1}[][HSK 4]
    \definition{s.}{lanche; petiscos; comida com especialidades locais, não muito para uma porção | prato frio; prato feito; cortes de frios na culinária ocidental | pratos pequenos e baratos; pratos simples em restaurantes com porções pequenas e preços baixos}
  \end{phonetics}
\end{entry}

\begin{entry}{小声}{3,7}{⼩、⼠}
  \begin{phonetics}{小声}{xiao3 sheng1}[][HSK 2]
    \definition{v.}{falar em voz baixa; falar baixinho; sussurar}
  \end{phonetics}
\end{entry}

\begin{entry}{小时}{3,7}{⼩、⽇}
  \begin{phonetics}{小时}{xiao3shi2}[][HSK 1]
    \definition{clas.}{hora; unidade de medida legal do tempo, 1 hora equivale a 60 minutos, é 1/24 de um dia}
    \definition[个]{s.}{hora; refere-se a um período de uma hora}
  \end{phonetics}
\end{entry}

\begin{entry}{小时候}{3,7,10}{⼩、⽇、⼈}
  \begin{phonetics}{小时候}{xiao3 shi2 hou5}[][HSK 2]
    \definition{s.}{na infância; quando alguém era jovem; refere-se à infância}
  \end{phonetics}
\end{entry}

\begin{entry}{小姐}{3,8}{⼩、⼥}
  \begin{phonetics}{小姐}{xiao3jie5}[][HSK 1]
    \definition[个,位]{s.}{jovem senhora; anteriormente, era assim que se referiam às filhas de famílias ricas. | senhorita; título honorífico para mulheres jovens | (gíria) prostituta}
  \end{phonetics}
\end{entry}

\begin{entry}{小学}{3,8}{⼩、⼦}
  \begin{phonetics}{小学}{xiao3 xue2}[][HSK 1]
    \definition[个]{s.}{escola primária (ou fundamental); escolas que oferecem ensino fundamental básico | estudos filológicos; antigamente, referia-se ao estudo da escrita, da fonética e da exegese}
  \end{phonetics}
\end{entry}

\begin{entry}{小学生}{3,8,5}{⼩、⼦、⽣}
  \begin{phonetics}{小学生}{xiao3 xue2 sheng1}[][HSK 1]
    \definition{s.}{aluno; estudante; estudante do sexo masculino (男); estudante do sexo feminino (女) | um aluno mais novo (do que os outros da sua turma) | (dialeto) um menino pequeno}
  \seealsoref{男}{nan2}
  \seealsoref{女}{nv3}
  \end{phonetics}
\end{entry}

\begin{entry}{小朋友}{3,8,4}{⼩、⽉、⼜}
  \begin{phonetics}{小朋友}{xiao3 peng2 you3}[][HSK 1]
    \definition[个]{s.}{criança; crianças; refere-se a crianças e adolescentes | (termo usado por um adulto para se dirigir a uma criança) amiguinho; menino (ou menina); termo carinhoso para se referir a crianças e adolescentes}
  \end{phonetics}
\end{entry}

\begin{entry}{小狗}{3,8}{⼩、⽝}
  \begin{phonetics}{小狗}{xiao3 gou3}
    \definition{s.}{filhote de cachorro}
  \end{phonetics}
\end{entry}

\begin{entry}{小组}{3,8}{⼩、⽷}
  \begin{phonetics}{小组}{xiao3 zu3}[][HSK 2]
    \definition[个,名,位]{s.}{grupo; um pequeno grupo de pessoas}
  \end{phonetics}
\end{entry}

\begin{entry}{小型}{3,9}{⼩、⼟}
  \begin{phonetics}{小型}{xiao3 xing2}[][HSK 4]
    \definition{adj.}{de tamanho pequeno; em pequena escala; miniatura; tipo pequeno; tamanho de bolso; tipo compacto}
    \definition{s.}{(Mediterrâneo) escunas, pequenos veleiros de pesca ou turismo | pequeno \emph{rover} lunar (duas pessoas)}
  \end{phonetics}
\end{entry}

\begin{entry}{小孩儿}{3,9,2}{⼩、⼦、⼉}
  \begin{phonetics}{小孩儿}{xiao3hai2r5}[][HSK 1]
    \definition[个]{s.}{criança; bebê}
  \end{phonetics}
\end{entry}

\begin{entry}{小屋}{3,9}{⼩、⼫}
  \begin{phonetics}{小屋}{xiao3wu1}
    \definition{s.}{cabana | chalé | cabine}
  \end{phonetics}
\end{entry}

\begin{entry}{小树}{3,9}{⼩、⽊}
  \begin{phonetics}{小树}{xiao3shu4}
    \definition[棵]{s.}{muda | arbusto | árvore pequena}
  \end{phonetics}
\end{entry}

\begin{entry}{小洋白菜}{3,9,5,11}{⼩、⽔、⽩、⾋}
  \begin{phonetics}{小洋白菜}{xiao3 yang2bai2cai4}
    \definition{s.}{couve de bruxelas}
  \end{phonetics}
\end{entry}

\begin{entry}{小说}{3,9}{⼩、⾔}
  \begin{phonetics}{小说}{xiao3shuo1}[][HSK 2]
    \definition[本,部,篇,章]{s.}{história; romance; ficção; uma forma literária que reflete a vida social por meio da descrição de personagens, ambiente e enredo}
  \end{phonetics}
\end{entry}

\begin{entry}{小偷儿}{3,11,2}{⼩、⼈、⼉}
  \begin{phonetics}{小偷儿}{xiao3 tou1er5}[][HSK 5]
    \definition{s.}{ladrão insignificante (ou furtivo); ladrãozinho | ladrão}
  \end{phonetics}
\end{entry}

\begin{entry}{小腿}{3,13}{⼩、⾁}
  \begin{phonetics}{小腿}{xiao3tui3}
    \definition{s.}{perna (do joelho ao calcanhar) | haste}
  \end{phonetics}
\end{entry}

\begin{entry}{山}{3}{⼭}[Kangxi 46]
  \begin{phonetics}{山}{shan1}[][HSK 1]
    \definition*{s.}{Sobrenome Shan}
    \definition[座]{s.}{colina; maciço; montanha | qualquer coisa que se assemelhe a uma montanha | arbustos nos quais os bichos-da-seda tecem seus casulos; referindo-se a casulos de bicho-da-seda | eco; metáfora para um som muito alto}
  \end{phonetics}
\end{entry}

\begin{entry}{山区}{3,4}{⼭、⼖}
  \begin{phonetics}{山区}{shan1 qu1}[][HSK 5]
    \definition[个]{s.}{área montanhosa; região montanhosa | colina; serra; montanha | distrito montanhoso}
  \end{phonetics}
\end{entry}

\begin{entry}{山东}{3,5}{⼭、⼀}
  \begin{phonetics}{山东}{shan1dong1}
    \definition*{s.}{Província de Shandong (Shantung) no nordeste da China}
  \end{phonetics}
\end{entry}

\begin{entry}{山羊}{3,6}{⼭、⽺}
  \begin{phonetics}{山羊}{shan1yang2}
    \definition{s.}{cabra | (ginástica) cavalo de salto de pequeno porte}
  \end{phonetics}
\end{entry}

\begin{entry}{山体}{3,7}{⼭、⼈}
  \begin{phonetics}{山体}{shan1ti3}
    \definition{s.}{forma de uma montanha}
  \end{phonetics}
\end{entry}

\begin{entry}{山谷}{3,7}{⼭、⾕}
  \begin{phonetics}{山谷}{shan1 gu3}[][HSK 6]
    \definition[条,个]{s.}{vale; desfiladeiro; ravina; a área baixa e estreita entre duas montanhas geralmente tem riachos no meio}
  \end{phonetics}
\end{entry}

\begin{entry}{山坡}{3,8}{⼭、⼟}
  \begin{phonetics}{山坡}{shan1 po1}[][HSK 6]
    \definition[个,座,片]{s.}{encosta; encosta da montanha; a inclinação entre o topo da montanha e o terreno plano}
  \end{phonetics}
\end{entry}

\begin{entry}{山顶}{3,8}{⼭、⾴}
  \begin{phonetics}{山顶}{shan1ding3}
    \definition{s.}{cume da montanha}
  \end{phonetics}
\end{entry}

\begin{entry}{山峰}{3,10}{⼭、⼭}
  \begin{phonetics}{山峰}{shan1 feng1}[][HSK 6]
    \definition[座,个]{s.}{pico (montanha); topo alto e pontudo da montanha}
  \end{phonetics}
\end{entry}

\begin{entry}{山寨}{3,14}{⼭、⼧}
  \begin{phonetics}{山寨}{shan1zhai4}
    \definition{s.}{fortaleza fortificada da vila | fortaleza da montanha (especialmente de bandidos) | falsificação | imitação | (fig.) pechincha}
  \end{phonetics}
\end{entry}

\begin{entry}{工}{3}{⼯}[Kangxi 48]
  \begin{phonetics}{工}{gong1}
    \definition*{s.}{Sobrenome Gong}
    \definition{adj.}{fino; requintado; delicado}
    \definition{s.}{trabalhador; operário; artesão | trabalho; labor; trabalho produtivo | projeto; construção; refere-se à engenharia | indústria; refere-se à indústria | homem-dia; a quantidade de trabalho que um trabalhador faz em um dia | uma nota da escala em Gongchepu (工尺谱), correspondente a 3 na notação musical numerada | engenheiro; refere-se a engenheiros}
    \definition{v.}{ser versado em; ser bom em | trabalhar em; agora geralmente escrito como 功}
  \seealsoref{功}{gong1}
  \seealsoref{工尺谱}{gong1 che3 pu3}
  \end{phonetics}
\end{entry}

\begin{entry}{工人}{3,2}{⼯、⼈}
  \begin{phonetics}{工人}{gong1ren2}[][HSK 1]
    \definition[个,名]{s.}{trabalhador; operário; mão de obra; trabalhadores braçais que vivem do salário}
  \end{phonetics}
\end{entry}

\begin{entry}{工厂}{3,2}{⼯、⼚}
  \begin{phonetics}{工厂}{gong1chang3}[][HSK 3]
    \definition[个,家,座,间]{s.}{fábrica; moinho; planta; unidades que realizam atividades de produção industrial diretamente, geralmente incluindo diferentes oficinas}
  \end{phonetics}
\end{entry}

\begin{entry}{工夫}{3,4}{⼯、⼤}
  \begin{phonetics}{工夫}{gong1 fu1}
    \definition[个]{s.}{tempo | tempo livre; lazer}
  \end{phonetics}
  \begin{phonetics}{工夫}{gong1 fu5}[][HSK 3]
    \definition[个]{s.}{(um período de) tempo; o tempo ou energia gastos para realizar uma tarefa | tempo livre}
  \end{phonetics}
\end{entry}

\begin{entry}{工尺谱}{3,4,14}{⼯、⼫、⾔}
  \begin{phonetics}{工尺谱}{gong1 che3 pu3}
    \definition*{s.}{Gongchepu, notação musical tradicional chinesa}
    \definition{s.}{notação musical tradicional chinesa que usa caracteres chineses para representar notas musicais}
  \end{phonetics}
\end{entry}

\begin{entry}{工艺}{3,4}{⼯、⾋}
  \begin{phonetics}{工艺}{gong1 yi4}[][HSK 5]
    \definition{s.}{técnica; tecnologia; arte industrial; técnicas ou métodos de fabricação e processamento de produtos | artesanato; arte artesanal}
  \end{phonetics}
\end{entry}

\begin{entry}{工艺品}{3,4,9}{⼯、⾋、⼝}
  \begin{phonetics}{工艺品}{gong1 yi4 pin3}[][HSK 5]
    \definition[个,件]{s.}{trabalho manual; artesanato; habilidade manual; artigo artesanal; itens delicados produzidos com técnicas artesanais. Por exemplo, esculturas em jade, esmaltes Jingtailan, bordados, etc.}
  \end{phonetics}
\end{entry}

\begin{entry}{工业}{3,5}{⼯、⼀}
  \begin{phonetics}{工业}{gong1ye4}[][HSK 3]
    \definition{s.}{indústria; utilização de recursos naturais; fabricação de meios de produção; meios de subsistência; ou processamento de produtos agrícolas, produtos semiacabados, etc.}
  \end{phonetics}
\end{entry}

\begin{entry}{工作}{3,7}{⼯、⼈}
  \begin{phonetics}{工作}{gong1zuo4}[][HSK 1]
    \definition[份,个,分,项]{s.}{trabalho; emprego | dever; tarefa; negócio}
    \definition{v.}{trabalhar; operar (uma máquina); envolver-se em trabalho físico ou intelectual, também se refere de maneira geral a máquinas e ferramentas operadas por pessoas para realizar funções produtivas}
  \end{phonetics}
\end{entry}

\begin{entry}{工作日}{3,7,4}{⼯、⼈、⽇}
  \begin{phonetics}{工作日}{gong1 zuo4 ri4}[][HSK 5]
    \definition{s.}{dia de trabalho; dia útil; dias em que você deveria estar trabalhando de acordo com as regras | horas de trabalho por dia; horas do dia para fazer o trabalho necessário}
  \end{phonetics}
\end{entry}

\begin{entry}{工具}{3,8}{⼯、⼋}
  \begin{phonetics}{工具}{gong1ju4}[][HSK 3]
    \definition[个,件,套]{s.}{ferramenta; ferramentas utilizadas na produção| ferramenta; meio; instrumento; (metáfora) algo ou meio utilizado para atingir um determinado objetivo}
  \end{phonetics}
\end{entry}

\begin{entry}{工资}{3,10}{⼯、⾙}
  \begin{phonetics}{工资}{gong1zi1}[][HSK 3]
    \definition[份,笔,月,天]{s.}{pagamento; salário; remuneração; vencimentos; o pagamento em dinheiro ou em espécie feito ao trabalhador como remuneração pelo trabalho realizado}
  \end{phonetics}
\end{entry}

\begin{entry}{工商}{3,11}{⼯、⼝}
  \begin{phonetics}{工商}{gong1 shang1}[][HSK 6]
    \definition{s.}{indústria e comércio; um termo combinado para indústria e comércio}
  \end{phonetics}
\end{entry}

\begin{entry}{工程}{3,12}{⼯、⽲}
  \begin{phonetics}{工程}{gong1 cheng2}[][HSK 4]
    \definition[个,项]{s.}{projeto; programa; trabalhos que utilizam equipamentos grandes e complexos, como projetos de reconstrução urbana e projetos de cestas de alimentos, etc. | engenharia; departamentos de produção e manufatura usam equipamentos grandes e complexos para realizar seu trabalho}
  \end{phonetics}
\end{entry}

\begin{entry}{工程师}{3,12,6}{⼯、⽲、⼱}
  \begin{phonetics}{工程师}{gong1cheng2shi1}[][HSK 3]
    \definition[个,位,名,些]{s.}{engenheiro; um dos cargos técnicos é o de especialista capaz de realizar de forma independente o projeto e a execução de uma tarefa técnica específica}
  \end{phonetics}
\end{entry}

\begin{entry}{工龄}{3,13}{⼯、⿒}
  \begin{phonetics}{工龄}{gong1ling2}
    \definition{s.}{tempo de serviço | senioridade}
  \end{phonetics}
\end{entry}

\begin{entry}{已}{3}{⼰}
  \begin{phonetics}{已}{yi3}[][HSK 3]
    \definition{adv.}{já | posteriormente; mais tarde; depois de algum tempo | demasiadamente; excessivamente}
    \definition{v.}{terminar; parar; cessar}
  \end{phonetics}
\end{entry}

\begin{entry}{已久}{3,3}{⼰、⼃}
  \begin{phonetics}{已久}{yi3jiu3}
    \definition{adv.}{já faz muito tempo}
  \end{phonetics}
\end{entry}

\begin{entry}{已灭}{3,5}{⼰、⽕}
  \begin{phonetics}{已灭}{yi3mie4}
    \definition{adj.}{extinto}
  \end{phonetics}
\end{entry}

\begin{entry}{已知}{3,8}{⼰、⽮}
  \begin{phonetics}{已知}{yi3zhi1}
    \definition{v.}{conhecido (ter ciência)}
  \end{phonetics}
\end{entry}

\begin{entry}{已经}{3,8}{⼰、⽷}
  \begin{phonetics}{已经}{yi3jing1}[][HSK 2]
    \definition{adv.}{já; indica que uma ação ou mudança foi concluída ou atingiu um determinado nível}
  \end{phonetics}
\end{entry}

\begin{entry}{已故}{3,9}{⼰、⽁}
  \begin{phonetics}{已故}{yi3gu4}
    \definition{adj.}{morto | atrasado}
  \end{phonetics}
\end{entry}

\begin{entry}{已婚}{3,11}{⼰、⼥}
  \begin{phonetics}{已婚}{yi3hun1}
    \definition{adj.}{casado}
  \end{phonetics}
\end{entry}

\begin{entry}{已然}{3,12}{⼰、⽕}
  \begin{phonetics}{已然}{yi3ran2}
    \definition{adv.}{já | já ser assim}
  \end{phonetics}
\end{entry}

\begin{entry}{干}{3}{⼲}[Kangxi 51]
  \begin{phonetics}{干}{gan1}[][HSK 1]
    \definition*{s.}{Sobrenome Gan}
    \definition{adj.}{seco (oposto a 湿) | vazio; oco; seco | sem substância; vazio | de parentesco nominal; (parentes) não ligados por laços sanguíneos | sem água; (água) esgotada; completamente vazia | assumido como parente nominal; relação familiar reconhecida por adoção | rude; grosseiro; mal-educado; descreve alguém que fala de forma muito direta e rude (sem delicadeza).}
    \definition{adv.}{em vão; fútil; sem propósito; para nada; sem resultado | apenas; sem nada mais | inutilmente; sem uso, sem aproveitamento | superficialmente; significa que não há conteúdo, apenas forma}
    \definition{s.}{(arcaico) escudo | margem; ribeira; margem das águas | alimentos desidratados | abreviação para os dez troncos celestiais}
    \definition{v.}{ofender; afrontar | ter a ver com; estar relacionado com; estar implicado em; interferir com | (antiquado) buscar (cargo público, remuneração, etc.) | (dialeto) deixar alguém de fora; tratar alguém com indiferença; desprezar | assediar; perturbar; criar confusão; causar estragos; bagunçar | solicitar; procurar; buscar (cargo, salário, etc.) | beber até o fim | tratar com indiferença; ignorar}
  \seealsoref{干儿}{gan1 er2}
  \seealsoref{干儿}{gan1r5}
  \seealsoref{湿}{shi1}
  \end{phonetics}
  \begin{phonetics}{干}{gan4}[][HSK 1]
    \definition{adj.}{capaz; competente; habilidoso}
    \definition{s.}{tronco; parte principal; corpo principal ou parte importante de algo | habilidade; capacidade; competência}
    \definition{v.}{fazer; trabalhar; cuidar; fazer coisas | ocupar o cargo de; estar envolvido em; assumir, exercer | lutar; golpear; esforçar-se}
  \end{phonetics}
\end{entry}

\begin{entry}{干儿}{3,2}{⼲、⼉}
  \begin{phonetics}{干儿}{gan1 er2}
    \definition{s.}{filho adotivo (adoção tradicional, ou seja, sem implicações legais)}
  \end{phonetics}
  \begin{phonetics}{干儿}{gan1r5}
    \definition{s.}{alimentos secos, desidratados}
  \end{phonetics}
\end{entry}

\begin{entry}{干与}{3,3}{⼲、⼀}
  \begin{phonetics}{干与}{gan1yu4}
    \variantof{干预}
  \end{phonetics}
\end{entry}

\begin{entry}{干什么}{3,4,3}{⼲、⼈、⼃}
  \begin{phonetics}{干什么}{gan4 shen2 me5}[][HSK 1]
    \definition{adv.}{o que fazer; o que ele está fazendo?; o que você está fazendo?; perguntar a razão ou o objetivo}
  \end{phonetics}
\end{entry}

\begin{entry}{干吗}{3,6}{⼲、⼝}
  \begin{phonetics}{干吗}{gan4 ma2}[][HSK 3]
    \definition{pron.}{por que?}
    \definition{v.}{o que fazer?}
  \end{phonetics}
\end{entry}

\begin{entry}{干你屁事}{3,7,7,8}{⼲、⼈、⼫、⼅}
  \begin{phonetics}{干你屁事}{gan1 ni3 pi4shi4}
    \definition{interj.}{Foda-se!}
  \end{phonetics}
\end{entry}

\begin{entry}{干扰}{3,7}{⼲、⼿}
  \begin{phonetics}{干扰}{gan1rao3}[][HSK 5]
    \definition{v.}{perturbar; incomodar | interferir; interromper o funcionamento adequado de equipamentos eletrônicos com sinais eletrônicos dispersos}
  \end{phonetics}
\end{entry}

\begin{entry}{干净}{3,8}{⼲、⼎}
  \begin{phonetics}{干净}{gan1jing4}[][HSK 1]
    \definition{adj.}{limpo; limpo e arrumado; sem poeira, impurezas, etc. |}
    \definition{adv.}{completely; totally; sem deixar nada para trás}
  \end{phonetics}
\end{entry}

\begin{entry}{干杯}{3,8}{⼲、⽊}
  \begin{phonetics}{干杯}{gan1bei1}[][HSK 2]
    \definition{interj.}{Saúde!}
    \definition{v.+compl.}{fazer um brinde;  brindar até a última gota}
  \end{phonetics}
\end{entry}

\begin{entry}{干活}{3,9}{⼲、⽔}
  \begin{phonetics}{干活}{gan4huo2}
    \definition{v.+compl.}{trabalhar | trabalhar em um emprego}
  \end{phonetics}
\end{entry}

\begin{entry}{干活儿}{3,9,2}{⼲、⽔、⼉}
  \begin{phonetics}{干活儿}{gan4huo2r5}[][HSK 2]
    \definition{v.}{trabalhar; gastar energia física ou mental para fazer algo, especialmente trabalho árduo ou esforçado.}
  \end{phonetics}
\end{entry}

\begin{entry}{干涉}{3,10}{⼲、⽔}
  \begin{phonetics}{干涉}{gan1she4}[][HSK 6]
    \definition{s.}{interferência; refere-se ao ato ou comportamento de interferir nos assuntos dos outros}
    \definition{v.}{interferir; intervir; intrometer-se; pedir ou impedir algo geralmente significa interferir quando não se deve}
  \end{phonetics}
\end{entry}

\begin{entry}{干脆}{3,10}{⼲、⾁}
  \begin{phonetics}{干脆}{gan1cui4}[][HSK 5]
    \definition{adj.}{claro; direto; (falar, fazer coisas) sem hesitação; atitude clara}
    \definition{adv.}{justamente; diretamente; sem maiores considerações}
  \end{phonetics}
\end{entry}

\begin{entry}{干预}{3,10}{⼲、⾴}
  \begin{phonetics}{干预}{gan1yu4}[][HSK 5]
    \definition{s.}{intromissão; intervenção}
    \definition{v.}{intrometer-se; intervir; interpor-se;}
  \end{phonetics}
\end{entry}

\begin{entry}{广}{3}{⼴}[Kangxi 53]
  \begin{phonetics}{广}{an1}
    \definition{s.}{mais comum em nomes de pessoas; o mesmo que 庵}[广安是我的朋友。___An'an é meu amigo.]
  \seealsoref{庵}{an1}
  \end{phonetics}
  \begin{phonetics}{广}{guang3}[][HSK 5]
    \definition*{s.}{Sobrenome Guang}
    \definition{adj.}{largo; vasto; amplo; extenso (oposto a 狭) | numeroso | comum; universal}
    \definition{s.}{Guangdong, 广东, e Guangxi, 广州}
    \definition{v.}{expandir; espalhar; ampliar}
  \seealsoref{广东}{guang3dong1}
  \seealsoref{广州}{guang3zhou1}
  \seealsoref{狭}{xia2}
  \end{phonetics}
  \begin{phonetics}{广}{yan3}
    \definition[家]{s.}{casa ou edifício construído contra ou ao longo da encosta de uma montanha ou penhasco}
  \end{phonetics}
\end{entry}

\begin{entry}{广大}{3,3}{⼴、⼤}
  \begin{phonetics}{广大}{guang3da4}[][HSK 3]
    \definition{adj.}{muito difundido; enorme (alcance, escala) | (uma área ou espaço) vasto; extenso; em grande escala; amplo (área, espaço) | numeroso; muitos (número de pessoas)}
  \end{phonetics}
\end{entry}

\begin{entry}{广东}{3,5}{⼴、⼀}
  \begin{phonetics}{广东}{guang3dong1}
    \definition*{s.}{Província de Guangdong}
  \seealsoref{粤}{yue4}
  \end{phonetics}
\end{entry}

\begin{entry}{广场}{3,6}{⼴、⼟}
  \begin{phonetics}{广场}{guang3chang3}[][HSK 2]
    \definition{s.}{praça; praça pública; esplanada; área ampla, especificamente uma área ampla na cidade}
  \end{phonetics}
\end{entry}

\begin{entry}{广场舞}{3,6,14}{⼴、⼟、⾇}
  \begin{phonetics}{广场舞}{guang3chang3wu3}
    \definition{s.}{quadrilha, uma rotina de exercícios tocada com música em quadrados públicos, parques e praças, popular especialmente entre mulheres de meia-idade e aposentados na China}
  \end{phonetics}
\end{entry}

\begin{entry}{广州}{3,6}{⼴、⼮}
  \begin{phonetics}{广州}{guang3zhou1}
    \definition*{s.}{Guangzhou, antigamente Cantão; Capital da Província de Guangdong}
  \end{phonetics}
\end{entry}

\begin{entry}{广告}{3,7}{⼴、⼝}
  \begin{phonetics}{广告}{guang3gao4}[][HSK 2]
    \definition[则,条,段,项,个]{s.}{anúncio; propaganda; uma forma de divulgação ao público de produtos, serviços ou programas culturais e esportivos, geralmente realizada por meio de jornais, televisão, rádio, cartazes, etc.}
    \definition{v.}{anunciar; a ação ou ato de promover ou divulgar algo}
  \end{phonetics}
\end{entry}

\begin{entry}{广泛}{3,7}{⼴、⽔}
  \begin{phonetics}{广泛}{guang3fan4}[][HSK 5]
    \definition{adj.}{amplo; extenso; de grande alcance; disseminado; escopo e cobertura amplos}
  \end{phonetics}
\end{entry}

\begin{entry}{广阔}{3,12}{⼴、⾨}
  \begin{phonetics}{广阔}{guang3kuo4}[][HSK 6]
    \definition{adj.}{vasto; largo; amplo}
  \end{phonetics}
\end{entry}

\begin{entry}{广播}{3,15}{⼴、⼿}
  \begin{phonetics}{广播}{guang3bo1}[][HSK 3]
    \definition[个,次,段,则,条]{s.}{programa de rádio; transmissão (de rádio); refere-se a programas transmitidos por estações de rádio ou televisão a cabo}
    \definition{v.}{transmitir; estar no ar | espalhar-se amplamente; ser conhecido em toda parte; divulgar amplamente}
  \end{phonetics}
\end{entry}

\begin{entry}{才}{3}{⼿}
  \begin{phonetics}{才}{cai2}[][HSK 2,4]
    \definition*{s.}{Sobrenome Cai}
    \definition{adv.}{há pouco; agora mesmo | (precedido por uma expressão de tempo) não até | (precedido por uma expressão de razão ou condição) não a menos que; não até que; então e somente então; por nenhuma outra razão | (seguido por uma expressão numérica) apenas; indica um intervalo pequeno ou uma quantidade reduzida, equivalente a 仅仅 ou 只 | (em uma afirmação ou negação, enfatizando o que vem antes de 才, geralmente com 呢 no final da frase) na verdade; realmente | dica que algo acontece tarde ou termina tarde | (precedido por uma expressão de tempo) não até; indicando que não era assim, mas agora surgiu uma nova situação | (precedido por uma expressão de razão ou condição) a menos que; indica que só em determinadas condições e, em seguida, como | (expressa ênfase )}
    \definition{s.}{habilidade; talento; dom | pessoa competente | pessoas de um determinado tipo (frequentemente usado como sufixo) | dotação; talento; habilidade}
  \seealsoref{呢}{ne5}
  \end{phonetics}
\end{entry}

\begin{entry}{才华}{3,6}{⼿、⼗}
  \begin{phonetics}{才华}{cai2hua2}
    \definition[份]{s.}{talento}
  \end{phonetics}
\end{entry}

\begin{entry}{才能}{3,10}{⼿、⾁}
  \begin{phonetics}{才能}{cai2 neng2}[][HSK 3]
    \definition[间]{s.}{talento; habilidade; dom; capacidade; inteligência e habilidade}
  \end{phonetics}
\end{entry}

\begin{entry}{才略}{3,11}{⼿、⽥}
  \begin{phonetics}{才略}{cai2lve4}
    \definition{s.}{habilidade e sagacidade}
  \end{phonetics}
\end{entry}

\begin{entry}{门}{3}{⾨}[Kangxi 169]
  \begin{phonetics}{门}{men2}[][HSK 1]
    \definition*{s.}{Sobrenome Men}
    \definition{clas.}{para equipamentos de artilharia (por exemplo: canhões) | para trabalhos escolares, ciência e tecnologia, etc. | para idiomas | para casamentos | para parentes}
    \definition[个,把,道,扇]{s.}{entradas e saídas de edifícios, veículos, navios, aviões, etc. | válvula; interruptor; algo que funciona como um interruptor ou como uma porta | habilidade; método; acesso; maneira de fazer algo | família; ramo de uma família ou clã | seita (religiosa); escola (de pensamento); faculdades acadêmicas, ideológicas ou religiosas | classe; categoria; ramo de estudo; refere-se à categoria geral de coisas | filo; segundo nível da classificação biológica | (computador) \emph{gate}; porta (lógica) | porta; portão; entrada; refere-se a uma porta que pode ser aberta e fechada, instalada na entrada e saída | qualquer abertura; partes de objetos que podem ser abertas e fechadas | orifício no corpo humano; refere-se especificamente aos orifícios do corpo humano | estudar com o mesmo professor; refere-se especificamente ao professor ou mestre | posição em um jogo de apostas (em relação ao local onde se senta ou onde se faz uma aposta)}
  \end{phonetics}
\end{entry}

\begin{entry}{门口}{3,3}{⾨、⼝}
  \begin{phonetics}{门口}{men2 kou3}[][HSK 1]
    \definition[个]{s.}{porta; portão; entrada; porta de entrada}
  \end{phonetics}
\end{entry}

\begin{entry}{门诊}{3,7}{⾨、⾔}
  \begin{phonetics}{门诊}{men2 zhen3}[][HSK 5]
    \definition{s.}{(no hospital) clínica ambulatorial; seção para pacientes ambulatoriais; local onde os médicos atendem pacientes que não estão internados no hospital}
  \end{phonetics}
\end{entry}

\begin{entry}{门票}{3,11}{⾨、⽰}
  \begin{phonetics}{门票}{men2 piao4}[][HSK 1]
    \definition{s.}{bilhete de entrada; bilhete de admissão; ingressos para locais de turismo, entretenimento, etc.}
  \end{phonetics}
\end{entry}

\begin{entry}{飞}{3}{⾶}[Kangxi 183]
  \begin{phonetics}{飞}{fei1}[][HSK 1]
    \definition{adj.}{inesperado; acidental; surgido do nada}
    \definition{adv.}{rapidamente; velozmente}
    \definition{s.}{roda livre de uma bicicleta}
    \definition{v.}{voar; esvoaçar; (pássaros, insetos, etc.) voar pelo ar batendo as asas | voar; utilizar máquinas motorizadas para se deslocar no ar | voar; (objetos naturais) flutuar ou esvoaçar no ar | volatilizar; evaporar; um gás se dissipar no ar | ir muito rapidamente; movimentar-se rapidamente, como se estivesse voando}
  \end{phonetics}
\end{entry}

\begin{entry}{飞机}{3,6}{⾶、⽊}
  \begin{phonetics}{飞机}{fei1ji1}[][HSK 1]
    \definition[架,个]{s.}{avião; aeronave; aroplano}
  \end{phonetics}
\end{entry}

\begin{entry}{飞机票}{3,6,11}{⾶、⽊、⽰}
  \begin{phonetics}{飞机票}{fei1ji1 piao4}
    \definition[张]{s.}{bilhete de avião}
  \seealsoref{机票}{ji1 piao4}
  \end{phonetics}
\end{entry}

\begin{entry}{飞行}{3,6}{⾶、⾏}
  \begin{phonetics}{飞行}{fei1 xing2}[][HSK 3]
    \definition{s.}{voo; aviação}
    \definition{v.}{voar; fazer um voo; (aviões, foguetes, etc.) voar no ar}
  \end{phonetics}
\end{entry}

\begin{entry}{飞行员}{3,6,7}{⾶、⾏、⼝}
  \begin{phonetics}{飞行员}{fei1 xing2 yuan2}[][HSK 6]
    \definition[名,班]{s.}{piloto; aviador; pilotos de aeronaves}
  \end{phonetics}
\end{entry}

\begin{entry}{飞船}{3,11}{⾶、⾈}
  \begin{phonetics}{飞船}{fei1 chuan2}[][HSK 6]
    \definition{s.}{nave espacial; espaçonave | dirigível; aerobarco}
  \end{phonetics}
\end{entry}

\begin{entry}{飞碟}{3,14}{⾶、⽯}
  \begin{phonetics}{飞碟}{fei1die2}
    \definition{s.}{disco-voador, OVNI, \emph{UFO} | \emph{frisbee}}
  \end{phonetics}
\end{entry}

\begin{entry}{马}{3}{⾺}[Kangxi 187]
  \begin{phonetics}{马}{ma3}[][HSK 3]
    \definition*{s.}{Sobrenome Ma}
    \definition{adj.}{grande; extenso; amplo}
    \definition[匹,头,只,群]{s.}{cavalo | a peça do cavalo no xadrez chinês}
  \end{phonetics}
\end{entry}

\begin{entry}{马上}{3,3}{⾺、⼀}
  \begin{phonetics}{马上}{ma3shang4}[][HSK 1]
    \definition{adv.}{imediatamente; de uma só vez; em um piscar de olhos | em breve; em um futuro próximo; em um curto espaço de tempo}
  \end{phonetics}
\end{entry}

\begin{entry}{马马虎虎}{3,3,8,8}{⾺、⾺、⾌、⾌}
  \begin{phonetics}{马马虎虎}{ma3ma3hu3hu3}
    \definition{adj.}{descuidado | casual | tolerável | vago | mais ou menos}
  \end{phonetics}
\end{entry}

\begin{entry}{马车}{3,4}{⾺、⾞}
  \begin{phonetics}{马车}{ma3 che1}[][HSK 6]
    \definition[辆]{s.}{carruagem (puxada por cavalo); carroça; charrete}
  \end{phonetics}
\end{entry}

\begin{entry}{马耳他}{3,6,5}{⾺、⽿、⼈}
  \begin{phonetics}{马耳他}{ma3'er3ta1}
    \definition*{s.}{Malta}
  \end{phonetics}
\end{entry}

\begin{entry}{马克思列宁主义}{3,7,9,6,5,5,3}{⾺、⼗、⼼、⼑、⼧、⼂、⼂}
  \begin{phonetics}{马克思列宁主义}{ma3ke4si1 lie4ning2 zhu3yi4}
    \definition*{s.}{Marxismo-Leninismo}
  \end{phonetics}
\end{entry}

\begin{entry}{马尾}{3,7}{⾺、⼫}
  \begin{phonetics}{马尾}{ma3wei3}
    \definition{s.}{(penteado) rabo de cavalo | cauda de cavalo}
  \end{phonetics}
\end{entry}

\begin{entry}{马路}{3,13}{⾺、⾜}
  \begin{phonetics}{马路}{ma3lu4}[][HSK 1]
    \definition[条]{s.}{estrada; rua; avenida; estradas largas e planas para o tráfego de carros e cavalos nas cidades ou nos subúrbios}
  \end{phonetics}
\end{entry}

%%%%% EOF %%%%%

