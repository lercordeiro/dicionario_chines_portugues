%%%
%%% 9画
%%%

\section*{9画}\addcontentsline{toc}{section}{9画}

\begin{entry}{临}{9}{⼁}
  \begin{phonetics}{临}{lin2}
    \definition*{s.}{sobrenome Lin}
    \definition{adv.}{pouco antes; prestes a; no ponto de}
    \definition{v.}{encarar; enfrentar; aproximar-se | chegar; estar presente | copiar (um modelo de caligrafia ou pintura); traçar sobre as palavras ou figuras | olhar de cima para baixo | ir de cima para baixo}
  \end{phonetics}
\end{entry}

\begin{entry}{临时}{9,7}{⼁、⽇}
  \begin{phonetics}{临时}{lin2shi2}[][HSK 4]
    \definition{adj.}{temporário; provisório; por um breve período}
    \definition{adv.}{no momento em que algo acontece (quando as coisas dão errado)}
  \end{phonetics}
\end{entry}

\begin{entry}{举}{9}{⼂}
  \begin{phonetics}{举}{ju3}[][HSK 2]
    \definition{v.}{levantar | segurar | iniciar | começar | dar à luz a | eleger | escolher | citar| enumerar}
  \end{phonetics}
\end{entry}

\begin{entry}{举办}{9,4}{⼂、⼒}
  \begin{phonetics}{举办}{ju3ban4}[][HSK 3]
    \definition{v.}{segurar; conduzir}
  \end{phonetics}
\end{entry}

\begin{entry}{举手}{9,4}{⼂、⼿}
  \begin{phonetics}{举手}{ju3 shou3}[][HSK 2]
    \definition{v.}{levantar (colocar) a mão ou mãos}
  \end{phonetics}
\end{entry}

\begin{entry}{举动}{9,6}{⼂、⼒}
  \begin{phonetics}{举动}{ju3dong4}[][HSK 5]
    \definition{s.}{ato; atividade; movimento; ação}
  \end{phonetics}
\end{entry}

\begin{entry}{举行}{9,6}{⼂、⾏}
  \begin{phonetics}{举行}{ju3xing2}[][HSK 2]
    \definition{v.}{realizar (uma reunião, cerimônia, etc.) | ter lugar}
  \end{phonetics}
\end{entry}

\begin{entry}{亭}{9}{⼇}
  \begin{phonetics}{亭}{ting2}
    \definition{s.}{pavilhão | cabine | quiosque}
  \end{phonetics}
\end{entry}

\begin{entry}{亮}{9}{⼇}
  \begin{phonetics}{亮}{liang4}[][HSK 2]
    \definition*{s.}{sobrenome Lian}
    \definition{adj.}{brilhante | alto e claro | retumbante | iluminado | aberto e claro}
    \definition{s.}{luz}
    \definition{v.}{iluminar | brilhar | elevar a voz | ressoar | revelar | mostrar | aparecer | exibir}
  \end{phonetics}
\end{entry}

\begin{entry}{亲}{9}{⼇}
  \begin{phonetics}{亲}{qin1}[][HSK 3]
    \definition{adj.}{parente próximo; relacionado por sangue; de ​​relação de sangue | querido; próximo; íntimo | em si mesmo; pessoalmente}
    \definition[位]{s.}{pais | parente | casal; casamento | noiva}
    \definition{v.}{beijar | (países, partidos, etc.) a favor de; apoiar; estar perto de}
  \end{phonetics}
  \begin{phonetics}{亲}{qing4}
    \definition{s.}{parentes por afinidade; parentes por casamento}
  \end{phonetics}
\end{entry}

\begin{entry}{亲人}{9,2}{⼇、⼈}
  \begin{phonetics}{亲人}{qin1 ren2}[][HSK 3]
    \definition{s.}{um membro da família; os pais, o cônjuge, os filhos, etc. | queridos; entes queridos; aqueles queridos para alguém}
  \end{phonetics}
\end{entry}

\begin{entry}{亲切}{9,4}{⼇、⼑}
  \begin{phonetics}{亲切}{qin1qie4}[][HSK 3]
    \definition{adj.}{gentil; cordial | próximo; íntimo}
  \end{phonetics}
\end{entry}

\begin{entry}{亲自}{9,6}{⼇、⾃}
  \begin{phonetics}{亲自}{qin1zi4}[][HSK 3]
    \definition{adv.}{pessoalmente; em pessoa; si mesmo}
  \end{phonetics}
\end{entry}

\begin{entry}{亲爱}{9,10}{⼇、⽖}
  \begin{phonetics}{亲爱}{qin1'ai4}[][HSK 4]
    \definition{adj.}{querido; amado; termo carinhoso que expressa intimidade e afeto}
  \end{phonetics}
\end{entry}

\begin{entry}{亲密}{9,11}{⼇、⼧}
  \begin{phonetics}{亲密}{qin1mi4}[][HSK 4]
    \definition{adj.}{próximo; íntimo; relacionamento afetuoso e próximo}
  \end{phonetics}
\end{entry}

\begin{entry}{侵略}{9,11}{⼈、⽥}
  \begin{phonetics}{侵略}{qin1lve4}
    \definition{s.}{invasão}
    \definition{v.}{invadir}
  \end{phonetics}
\end{entry}

\begin{entry}{便于}{9,3}{⼈、⼆}
  \begin{phonetics}{便于}{bian4yu2}[][HSK 5]
    \definition{v.}{ser fácil para; ser conveniente para (algo ou fazer algo)}
  \end{phonetics}
\end{entry}

\begin{entry}{便利}{9,7}{⼈、⼑}
  \begin{phonetics}{便利}{bian4li4}[][HSK 5]
    \definition{adj.}{fácil; conveniente;}
    \definition{s.}{facilidade; conveniência; coisas ou condições convenientes}
    \definition{v.}{facilitar; fornecer ajuda para que os outros se sintam confortáveis}
  \end{phonetics}
\end{entry}

\begin{entry}{便条}{9,7}{⼈、⽊}
  \begin{phonetics}{便条}{bian4tiao2}[][HSK 5]
    \definition[张,个]{s.}{nota ou mensagem informal; geralmente uma mensagem ou notificação}
  \end{phonetics}
\end{entry}

\begin{entry}{便宜}{9,8}{⼈、⼧}
  \begin{phonetics}{便宜}{pian2yi5}[][HSK 2]
    \definition{adj.}{barato}
    \definition{v.}{deixar alguém levemente de lado}
  \end{phonetics}
\end{entry}

\begin{entry}{促进}{9,7}{⼈、⾡}
  \begin{phonetics}{促进}{cu4jin4}[][HSK 4]
    \definition{v.}{impulsionar; promover; avançar; incentivar o desenvolvimento}
  \end{phonetics}
\end{entry}

\begin{entry}{促使}{9,8}{⼈、⼈}
  \begin{phonetics}{促使}{cu4shi3}[][HSK 4]
    \definition{v.}{incitar; estimular; impelir; causar; provocar uma mudança em alguém ou em algo}
  \end{phonetics}
\end{entry}

\begin{entry}{促销}{9,12}{⼈、⾦}
  \begin{phonetics}{促销}{cu4 xiao1}[][HSK 4]
    \definition{v.}{promover vendas}
  \end{phonetics}
\end{entry}

\begin{entry}{俄}{9}{⼈}
  \begin{phonetics}{俄}{e2}
    \definition*{s.}{Rússia, abreviação de 俄罗斯}
  \seealsoref{俄罗斯}{e2luo2si1}
  \end{phonetics}
\end{entry}

\begin{entry}{俄罗斯}{9,8,12}{⼈、⽹、⽄}
  \begin{phonetics}{俄罗斯}{e2luo2si1}
    \definition*{s.}{Rússia}
  \end{phonetics}
\end{entry}

\begin{entry}{俄罗斯人}{9,8,12,2}{⼈、⽹、⽄、⼈}
  \begin{phonetics}{俄罗斯人}{e2luo2si1ren2}
    \definition{s.}{russo | pessoa ou povo da Rússia}
  \end{phonetics}
\end{entry}

\begin{entry}{保}{9}{⼈}
  \begin{phonetics}{保}{bao3}[][HSK 3]
    \definition*{s.}{sobrenome Bao}
    \definition{s.}{fiador
oficial responsável
sistema administrativo}
    \definition{v.}{defender | proteger |manter | preservar | conservar em boas condições | garantir | assegurar | ficar como fiador de alguém.}
  \end{phonetics}
\end{entry}

\begin{entry}{保卫}{9,3}{⼈、⼙}
  \begin{phonetics}{保卫}{bao3wei4}[][HSK 5]
    \definition{v.}{defender; proteger; salvaguardar}
  \end{phonetics}
\end{entry}

\begin{entry}{保存}{9,6}{⼈、⼦}
  \begin{phonetics}{保存}{bao3cun2}[][HSK 3]
    \definition{v.}{conservar | preservar | (computação) salvar (um arquivo, etc.)}
  \end{phonetics}
\end{entry}

\begin{entry}{保守}{9,6}{⼈、⼧}
  \begin{phonetics}{保守}{bao3shou3}[][HSK 4]
    \definition{adj.}{retrógrado; conservador; pensamentos e conceitos que são retrógrados e não conseguem acompanhar o desenvolvimento da situação}
    \definition{v.}{manter; guardar; evitar perder}
  \end{phonetics}
\end{entry}

\begin{entry}{保安}{9,6}{⼈、⼧}
  \begin{phonetics}{保安}{bao3 an1}[][HSK 3]
    \definition{s.}{guarda de segurança}
    \definition{v.}{manter seguro | garantir a segurança}
  \end{phonetics}
\end{entry}

\begin{entry}{保护}{9,7}{⼈、⼿}
  \begin{phonetics}{保护}{bao3hu4}[][HSK 3]
    \definition{s.}{proteção | salvaguarda}
    \definition{v.}{proteger | defender | salvaguardar}
  \end{phonetics}
\end{entry}

\begin{entry}{保护区}{9,7,4}{⼈、⼿、⼖}
  \begin{phonetics}{保护区}{bao3hu4qu1}
    \definition{s.}{área protegida | área de conservação}
  \end{phonetics}
\end{entry}

\begin{entry}{保护主义}{9,7,5,3}{⼈、⼿、⼂、⼂}
  \begin{phonetics}{保护主义}{bao3hu4zhu3yi4}
    \definition{s.}{protecionismo}
  \end{phonetics}
\end{entry}

\begin{entry}{保护色}{9,7,6}{⼈、⼿、⾊}
  \begin{phonetics}{保护色}{bao3hu4se4}
    \definition{s.}{camuflagem}
  \end{phonetics}
\end{entry}

\begin{entry}{保护剂}{9,7,8}{⼈、⼿、⼑}
  \begin{phonetics}{保护剂}{bao3hu4ji4}
    \definition{s.}{agente protetor}
  \end{phonetics}
\end{entry}

\begin{entry}{保护国}{9,7,8}{⼈、⼿、⼞}
  \begin{phonetics}{保护国}{bao3hu4guo2}
    \definition{s.}{protetorado}
  \end{phonetics}
\end{entry}

\begin{entry}{保护性}{9,7,8}{⼈、⼿、⼼}
  \begin{phonetics}{保护性}{bao3hu4xing4}
    \definition{s.}{proteção}
  \end{phonetics}
\end{entry}

\begin{entry}{保护物}{9,7,8}{⼈、⼿、⽜}
  \begin{phonetics}{保护物}{bao3hu4 wu4}
    \definition{s.}{protetor}
  \end{phonetics}
\end{entry}

\begin{entry}{保护者}{9,7,8}{⼈、⼿、⽼}
  \begin{phonetics}{保护者}{bao3hu4zhe3}
    \definition{s.}{protetor | segurador}
  \end{phonetics}
\end{entry}

\begin{entry}{保护神}{9,7,9}{⼈、⼿、⽰}
  \begin{phonetics}{保护神}{bao3hu4shen2}
    \definition{s.}{anjo da guarda | santo patrono}
  \end{phonetics}
\end{entry}

\begin{entry}{保证}{9,7}{⼈、⾔}
  \begin{phonetics}{保证}{bao3zheng4}[][HSK 3]
    \definition[个]{s.}{garantia}
    \definition{v.}{garantir}
  \end{phonetics}
\end{entry}

\begin{entry}{保养}{9,9}{⼈、⼋}
  \begin{phonetics}{保养}{bao3yang3}[][HSK 5]
    \definition{v.}{preservar; cuidar bem (ou conservar) da saúde |  fazer manutenção; conservar; manter; manter em bom estado de conservação}
  \end{phonetics}
\end{entry}

\begin{entry}{保持}{9,9}{⼈、⼿}
  \begin{phonetics}{保持}{bao3chi2}[][HSK 3]
    \definition{v.}{manter | segurar | reter | preservar}
  \end{phonetics}
\end{entry}

\begin{entry}{保险}{9,9}{⼈、⾩}
  \begin{phonetics}{保险}{bao3xian3}[][HSK 3]
    \definition[个]{adj./s.}{seguro}
    \definition{v.}{ter certeza | estar vinculado a}
  \end{phonetics}
\end{entry}

\begin{entry}{保留}{9,10}{⼈、⽥}
  \begin{phonetics}{保留}{bao3liu2}[][HSK 3]
    \definition{v.}{reter | continuar a ter | segurar | reservar}
  \end{phonetics}
\end{entry}

\begin{entry}{保密}{9,11}{⼈、⼧}
  \begin{phonetics}{保密}{bao3mi4}[][HSK 4]
    \definition{v.}{manter segredo; manter algo em segredo; manter a confidencialidade}
  \end{phonetics}
\end{entry}

\begin{entry}{信}{9}{⼈}
  \begin{phonetics}{信}{xin4}[][HSK 2,3]
    \definition*{s.}{sobrenome Xin}
    \definition{adj.}{verdade}
    \definition{adv.}{à vontade; ao acaso; sem plano}
    \definition[封,个,张]{s.}{carta; correio
mensagem; palavra; informação
sinal; evidência
confiança; fé
fusível
arsênico}
    \definition{v.}{acreditar; fazer um balanço; dar crédito | professar fé em; acreditar em}
  \end{phonetics}
\end{entry}

\begin{entry}{信心}{9,4}{⼈、⼼}
  \begin{phonetics}{信心}{xin4xin1}[][HSK 2]
    \definition[个]{s.}{confiança | fé (em alguém ou algo)}
  \end{phonetics}
\end{entry}

\begin{entry}{信号}{9,5}{⼈、⼝}
  \begin{phonetics}{信号}{xin4hao4}[][HSK 2]
    \definition[个]{s.}{sinal | ponte de sinalização}
  \end{phonetics}
\end{entry}

\begin{entry}{信用}{9,5}{⼈、⽤}
  \begin{phonetics}{信用}{xin4yong4}
    \definition{s.}{crédito (comércio)}
  \end{phonetics}
\end{entry}

\begin{entry}{信用卡}{9,5,5}{⼈、⽤、⼘}
  \begin{phonetics}{信用卡}{xin4yong4ka3}[][HSK 2]
    \definition[些]{s.}{cartão de crédito}
  \end{phonetics}
\end{entry}

\begin{entry}{信任}{9,6}{⼈、⼈}
  \begin{phonetics}{信任}{xin4ren4}[][HSK 3]
    \definition[个]{s.}{confiança; certeza; convicção}
    \definition{v.}{confiar; ter confiança em}
  \end{phonetics}
\end{entry}

\begin{entry}{信访}{9,6}{⼈、⾔}
  \begin{phonetics}{信访}{xin4fang3}
    \definition{s.}{carta de reclamação | carta de petição}
  \seealsoref{上访}{shang4fang3}
  \end{phonetics}
\end{entry}

\begin{entry}{信念}{9,8}{⼈、⼼}
  \begin{phonetics}{信念}{xin4nian4}[][HSK 5]
    \definition[个]{s.}{fé; crença; convicção; concepções consideradas corretas e acreditadas com convicção}
  \end{phonetics}
\end{entry}

\begin{entry}{信经}{9,8}{⼈、⽷}
  \begin{phonetics}{信经}{xin4jing1}
    \definition[个]{s.}{crença | credo (seção da missa católica)}
  \end{phonetics}
\end{entry}

\begin{entry}{信封}{9,9}{⼈、⼨}
  \begin{phonetics}{信封}{xin4feng1}[][HSK 3]
    \definition[个]{s.}{envelope de carta}
  \end{phonetics}
\end{entry}

\begin{entry}{信息}{9,10}{⼈、⼼}
  \begin{phonetics}{信息}{xin4xi1}[][HSK 2]
    \definition[个,条]{s.}{notícias | informação | mensagem}
  \end{phonetics}
\end{entry}

\begin{entry}{信箱}{9,15}{⼈、⾋}
  \begin{phonetics}{信箱}{xin4 xiang1}[][HSK 5]
    \definition{s.}{caixa de correio; caixa postal instalada pelos correios para que as pessoas possam depositar cartas | caixa postal; caixas com números, localizadas nos correios, que podem ser alugadas para receber correspondência; chamadas de caixas postais exclusivas}
  \end{phonetics}
\end{entry}

\begin{entry}{俩}{9}{⼈}
  \begin{phonetics}{俩}{lia3}[][HSK 4]
    \definition{num.}{ambos; dois; contração de ``两个'' | alguns; vários; refere-se a um pequeno número}
  \end{phonetics}
\end{entry}

\begin{entry}{俩钱}{9,10}{⼈、⾦}
  \begin{phonetics}{俩钱}{lia3qian2}
    \definition{s.}{uma pequena quantia de dinheiro}
  \end{phonetics}
\end{entry}

\begin{entry}{俭省}{9,9}{⼈、⽬}
  \begin{phonetics}{俭省}{jian3sheng3}
    \definition{adj.}{econômico}
  \end{phonetics}
\end{entry}

\begin{entry}{修}{9}{⼈}
  \begin{phonetics}{修}{xiu1}[][HSK 3]
    \definition*{s.}{sobrenome Xiu}
    \definition{adj.}{comprido; alto e esbelto}
    \definition{s.}{revisionismo}
    \definition{v.}{embelezar; decorar
consertar; reparar; reformar
escrever; compilar
estudar; cultivar
construir; edificar
aparar; podar}
  \end{phonetics}
\end{entry}

\begin{entry}{修改}{9,7}{⼈、⽁}
  \begin{phonetics}{修改}{xiu1gai3}[][HSK 3]
    \definition{v.}{revisar; alterar}
  \end{phonetics}
\end{entry}

\begin{entry}{修建}{9,8}{⼈、⼵}
  \begin{phonetics}{修建}{xiu1jian4}[][HSK 5]
    \definition{v.}{construir; erguer; animar; edificar; construir com tijolos, telhas, madeira, cimento, areia, etc.}
  \end{phonetics}
\end{entry}

\begin{entry}{修规}{9,8}{⼈、⾒}
  \begin{phonetics}{修规}{xiu1gui1}
    \definition{s.}{plano de construção}
  \end{phonetics}
\end{entry}

\begin{entry}{修养}{9,9}{⼈、⼋}
  \begin{phonetics}{修养}{xiu1yang3}[][HSK 5]
    \definition[种]{s.}{treinamento; domínio; realização; refere-se a um determinado nível em termos de teoria, conhecimento, arte, pensamento, etc. | auto-cultivo; refere-se à atitude e ao comportamento cultivados ao longo do tempo, em conformidade com as exigências sociais}
  \end{phonetics}
\end{entry}

\begin{entry}{修复}{9,9}{⼈、⼢}
  \begin{phonetics}{修复}{xiu1fu4}[][HSK 5]
    \definition{v.}{reparar; restaurar; renovar | reparar; melhorar e restaurar (o relacionamento)}
  \end{phonetics}
\end{entry}

\begin{entry}{修理}{9,11}{⼈、⽟}
  \begin{phonetics}{修理}{xiu1li3}[][HSK 4]
    \definition{v.}{consertar; reparar; restaurar algo danificado à sua forma ou função original | aparar; podar; cortar com tesouras e outras ferramentas para deixar árvores, flores, cabelos, etc. arrumados | culpar; punir; criticar ou punir uma pessoa para mostrar que ela está errada}
  \end{phonetics}
\end{entry}

\begin{entry}{养}{9}{⼋}
  \begin{phonetics}{养}{yang3}[][HSK 2]
    \definition{v.}{criar (animais ou filhos), plantar (flores), etc. | dar a luz}
  \end{phonetics}
\end{entry}

\begin{entry}{养分}{9,4}{⼋、⼑}
  \begin{phonetics}{养分}{yang3fen4}
    \definition{s.}{nutriente}
  \end{phonetics}
\end{entry}

\begin{entry}{养成}{9,6}{⼋、⼽}
  \begin{phonetics}{养成}{yang3cheng2}[][HSK 4]
    \definition{v.}{cultivar; desenvolver; cultivar para formar; nutrir para crescer}
  \end{phonetics}
\end{entry}

\begin{entry}{养料}{9,10}{⼋、⽃}
  \begin{phonetics}{养料}{yang3liao4}
    \definition{s.}{nutrição}
  \end{phonetics}
\end{entry}

\begin{entry}{冒}{9}{⽇}
  \begin{phonetics}{冒}{mao4}[][HSK 5]
    \definition*{s.}{sobrenome Mao}
    \definition{adv.}{com ousadia; precipitadamente | fingidamente; falsamente; fraudulentamente}
    \definition{v.}{emitir; liberar; enviar (para cima) | arriscar; ser corajoso}
  \end{phonetics}
\end{entry}

\begin{entry}{冒险}{9,9}{⽇、⾩}
  \begin{phonetics}{冒险}{mao4xian3}
    \definition{adj.}{corajoso}
    \definition{s.}{risco | aventura}
    \definition{v.+compl.}{correr risco | arriscar-se | aventurar-se em}
  \end{phonetics}
\end{entry}

\begin{entry}{冠}{9}{⼍}
  \begin{phonetics}{冠}{guan1}
    \definition{s.}{chapéu | coroa | brasão | boné}
  \end{phonetics}
  \begin{phonetics}{冠}{guan4}
    \definition*{s.}{sobrenome Guan}
    \definition{v.}{colocar um chapéu | ser o primeiro | dublar}
  \end{phonetics}
\end{entry}

\begin{entry}{冠军}{9,6}{⼍、⼍}
  \begin{phonetics}{冠军}{guan4jun1}[][HSK 5]
    \definition[个]{s.}{campeão; medalhista de ouro; primeiro lugar em esportes e outras competições}
  \end{phonetics}
\end{entry}

\begin{entry}{前}{9}{⼑}
  \begin{phonetics}{前}{qian2}[][HSK 1]
    \definition{adv.}{frente; em frente de | A.C. (Antes de~Cristo)}[前293年  (293 a.C.)]
  \seealsoref{公元}{gong1yuan2}
  \end{phonetics}
\end{entry}

\begin{entry}{前天}{9,4}{⼑、⼤}
  \begin{phonetics}{前天}{qian2tian1}[][HSK 1]
    \definition{adv.}{anteontem}
  \end{phonetics}
\end{entry}

\begin{entry}{前头}{9,5}{⼑、⼤}
  \begin{phonetics}{前头}{qian2 tou5}[][HSK 4]
    \definition{s.}{à frente; na frente; adiante}
  \end{phonetics}
\end{entry}

\begin{entry}{前边}{9,5}{⼑、⾡}
  \begin{phonetics}{前边}{qian2bian5}[][HSK 1]
    \definition{adv.}{à frente | da frente}
  \end{phonetics}
\end{entry}

\begin{entry}{前后}{9,6}{⼑、⼝}
  \begin{phonetics}{前后}{qian2 hou4}[][HSK 3]
    \definition{s.}{em volta; sobre | do início ao fim | frente e verso}
  \end{phonetics}
\end{entry}

\begin{entry}{前年}{9,6}{⼑、⼲}
  \begin{phonetics}{前年}{qian2 nian2}[][HSK 2]
    \definition{adv.}{há dois anos}
  \end{phonetics}
\end{entry}

\begin{entry}{前进}{9,7}{⼑、⾡}
  \begin{phonetics}{前进}{qian2 jin4}[][HSK 3]
    \definition{v.}{marchar; avançar; para ir em frente; seguir em frente}
  \end{phonetics}
\end{entry}

\begin{entry}{前往}{9,8}{⼑、⼻}
  \begin{phonetics}{前往}{qian2 wang3}[][HSK 3]
    \definition{v.}{ir para; prosseguir para; partir para}
  \end{phonetics}
\end{entry}

\begin{entry}{前面}{9,9}{⼑、⾯}
  \begin{phonetics}{前面}{qian2mian4}[][HSK 3]
    \definition{s.}{frente | parte anterior; acima}
  \end{phonetics}
\end{entry}

\begin{entry}{前途}{9,10}{⼑、⾡}
  \begin{phonetics}{前途}{qian2tu2}[][HSK 4]
    \definition[片,段,种]{s.}{futuro; perspectiva; prospecto; originalmente, refere-se à jornada à frente, mas, metaforicamente, refere-se ao futuro.}
  \end{phonetics}
\end{entry}

\begin{entry}{前提}{9,12}{⼑、⼿}
  \begin{phonetics}{前提}{qian2ti2}[][HSK 5]
    \definition[个,项]{s.}{premissa; pressuposto | pré-requisito; pressuposição; condições prévias para que algo aconteça ou se desenvolva}
  \end{phonetics}
\end{entry}

\begin{entry}{前景}{9,12}{⼑、⽇}
  \begin{phonetics}{前景}{qian2jing3}[][HSK 5]
    \definition{s.}{primeiro plano (de uma vista, imagem, foto, etc.); as imagens que parecem mais próximas do espectador em pinturas, palcos e telas | vista; perspectiva; prospecto; ponto de vista; situações que podem ocorrer no trabalho, na carreira, etc.}
  \end{phonetics}
\end{entry}

\begin{entry}{剑}{9}{⼑}
  \begin{phonetics}{剑}{jian4}
    \definition{clas.}{para golpes de uma espada}
    \definition[口,把]{s.}{espada de dois gumes}
  \end{phonetics}
\end{entry}

\begin{entry}{剑客}{9,9}{⼑、⼧}
  \begin{phonetics}{剑客}{jian4ke4}
    \definition{s.}{espada | esgrimista, espadachim}
  \end{phonetics}
\end{entry}

\begin{entry}{勇士}{9,3}{⼒、⼠}
  \begin{phonetics}{勇士}{yong3shi4}
    \definition{s.}{um guerreiro | uma pessoa corajosa}
  \end{phonetics}
\end{entry}

\begin{entry}{勇气}{9,4}{⼒、⽓}
  \begin{phonetics}{勇气}{yong3qi4}[][HSK 4]
    \definition[种,股]{s.}{coragem; arrojo; nervos; coragem para agir sem medo}
  \end{phonetics}
\end{entry}

\begin{entry}{勇敢}{9,11}{⼒、⽁}
  \begin{phonetics}{勇敢}{yong3gan3}[][HSK 4]
    \definition{adj.}{bravo; valente; galante; corajoso}
  \end{phonetics}
\end{entry}

\begin{entry}{南}{9}{⼗}
  \begin{phonetics}{南}{nan2}[][HSK 1]
    \definition*{s.}{sobrenome Nan}
    \definition{s.}{sul}
  \end{phonetics}
\end{entry}

\begin{entry}{南方}{9,4}{⼗、⽅}
  \begin{phonetics}{南方}{nan2 fang1}[][HSK 2]
    \definition{s.}{sul | o Sul | a parte sul do país}
  \end{phonetics}
\end{entry}

\begin{entry}{南北}{9,5}{⼗、⼔}
  \begin{phonetics}{南北}{nan2 bei3}[][HSK 5]
    \definition{s.}{norte e sul | de norte a sul}
  \end{phonetics}
\end{entry}

\begin{entry}{南边}{9,5}{⼗、⾡}
  \begin{phonetics}{南边}{nan2bian5}[][HSK 1]
    \definition{adv.}{sul | lado sul | parte sul | ao sul de}
  \end{phonetics}
\end{entry}

\begin{entry}{南极}{9,7}{⼗、⽊}
  \begin{phonetics}{南极}{nan2ji2}[][HSK 5]
    \definition*{s.}{Polo Sul; Polo Antártico | Polo sul magnético}
    \definition{s.}{pólo sul magnético}
  \end{phonetics}
\end{entry}

\begin{entry}{南面}{9,9}{⼗、⾯}
  \begin{phonetics}{南面}{nan2mian4}
    \definition{s.}{sul | lado sul}
  \end{phonetics}
\end{entry}

\begin{entry}{南部}{9,10}{⼗、⾢}
  \begin{phonetics}{南部}{nan2 bu4}[][HSK 3]
    \definition{s.}{parte sul; sul | a parte sul}
  \end{phonetics}
\end{entry}

\begin{entry}{厘米}{9,6}{⼚、⽶}
  \begin{phonetics}{厘米}{li2mi3}[][HSK 4]
    \definition{clas.}{centímetro; unidade de comprimento, símbolo cm, 1 metro é igual a 100 centímetros}
  \end{phonetics}
\end{entry}

\begin{entry}{厚}{9}{⼚}
  \begin{phonetics}{厚}{hou4}[][HSK 4]
    \definition*{s.}{sobrenome Hou}
    \definition{adj.}{grosso; espesso | profundo | bondoso; gentil; magnânimo | grande; generoso | rico ou forte em sabor}
    \definition{s.}{espessura; profundidade}
    \definition{v.}{favorecer; enfatizar}
  \end{phonetics}
\end{entry}

\begin{entry}{咬}{9}{⼝}
  \begin{phonetics}{咬}{yao3}[][HSK 5]
    \definition{v.}{morder; estalar; pressionar os dentes superiores e inferiores com força | latir | agarrar; morder | incriminar outra pessoa (geralmente inocente) quando culpada ou interrogada | pronunciar; articular; pronunciar corretamente | corroer (metais); irritar (a pele) | ser minucioso (com relação ao uso de palavras) | aproximar-se de; pressionar em direção a; avançar sobre}
  \end{phonetics}
\end{entry}

\begin{entry}{咱}{9}{⼝}
  \begin{phonetics}{咱}{zan2}[][HSK 2]
    \definition{pron.}{eu}
  \end{phonetics}
\end{entry}

\begin{entry}{咱们}{9,5}{⼝、⼈}
  \begin{phonetics}{咱们}{zan2men5}[][HSK 2]
    \definition{pron.}{nós (incluindo o orador e a(s) pessoa(s) com quem se fala)}
  \end{phonetics}
\end{entry}

\begin{entry}{咱俩}{9,9}{⼝、⼈}
  \begin{phonetics}{咱俩}{zan2lia3}
    \definition{pron.}{nós dois}
  \end{phonetics}
\end{entry}

\begin{entry}{咱家}{9,10}{⼝、⼧}
  \begin{phonetics}{咱家}{za2jia1}
    \definition{pron.}{eu (frequentemente usado na literatura vernácula antiga) | me | mim | comigo}
  \end{phonetics}
\end{entry}

\begin{entry}{咳}{9}{⼝}
  \begin{phonetics}{咳}{hai1}
    \definition{interj.}{expressa tristeza, arrependimento ou espanto}
  \end{phonetics}
  \begin{phonetics}{咳}{ke2}[][HSK 5]
    \definition{v.}{tossir}
  \end{phonetics}
\end{entry}

\begin{entry}{咳嗽}{9,14}{⼝、⼝}
  \begin{phonetics}{咳嗽}{ke2sou5}
    \definition{v.}{ter tosse | tossir}
  \end{phonetics}
\end{entry}

\begin{entry}{咸}{9}{⼝}
  \begin{phonetics}{咸}{xian2}[][HSK 4]
    \definition*{s.}{sobrenome Xian}
    \definition{adj.}{salgado; em conserva; sabor salgado}
    \definition{adv.}{todos; indica a totalidade de um intervalo, equivalente a ``全'' e ``都''}
  \seealsoref{都}{dou1}
  \seealsoref{全}{quan2}
  \end{phonetics}
\end{entry}

\begin{entry}{咸水}{9,4}{⼝、⽔}
  \begin{phonetics}{咸水}{xian2shui3}
    \definition{s.}{salmora | água salgada}
  \end{phonetics}
\end{entry}

\begin{entry}{咸肉}{9,6}{⼝、⾁}
  \begin{phonetics}{咸肉}{xian2rou4}
    \definition{s.}{\emph{bacon} | carne curada com sal}
  \end{phonetics}
\end{entry}

\begin{entry}{咸鱼}{9,8}{⼝、⿂}
  \begin{phonetics}{咸鱼}{xian2yu2}
    \definition{s.}{peixe salgado}
  \end{phonetics}
\end{entry}

\begin{entry}{咸涩}{9,10}{⼝、⽔}
  \begin{phonetics}{咸涩}{xian2se4}
    \definition{s.}{ácido | salgado e amargo}
  \end{phonetics}
\end{entry}

\begin{entry}{咸盐}{9,10}{⼝、⽫}
  \begin{phonetics}{咸盐}{xian2yan2}
    \definition{s.}{(coloquial) sal | sal de mesa}
  \end{phonetics}
\end{entry}

\begin{entry}{咸淡}{9,11}{⼝、⽔}
  \begin{phonetics}{咸淡}{xian2dan4}
    \definition{s.}{água salobra | grau de salinidade | salgado e sem sal (sabores)}
  \end{phonetics}
\end{entry}

\begin{entry}{咸菜}{9,11}{⼝、⾋}
  \begin{phonetics}{咸菜}{xian2cai4}
    \definition{s.}{legumes salgados | \emph{pickles}}
  \end{phonetics}
\end{entry}

\begin{entry}{品}{9}{⼝}
  \begin{phonetics}{品}{pin3}[][HSK 5]
    \definition*{s.}{sobrenome Pin}
    \definition{s.}{artigo; produto | grau; classe; classificação; nível | caráter; qualidade | classificação; os graus dos funcionários públicos antigos, num total de nove graus}
    \definition{v.}{provar; saborear; degustar algo com discernimento | soprar; tocar (instrumentos de sopro) | avaliar; distinguir o bom do ruim}
  \end{phonetics}
\end{entry}

\begin{entry}{品质}{9,8}{⼝、⾙}
  \begin{phonetics}{品质}{pin3zhi4}[][HSK 4]
    \definition[个,种]{s.}{qualidade; caráter; natureza do pensamento, da compreensão, do caráter, etc., conforme expresso no comportamento, no estilo, etc. | qualidade (de produtos, mercadorias, etc.)}
  \end{phonetics}
\end{entry}

\begin{entry}{品种}{9,9}{⼝、⽲}
  \begin{phonetics}{品种}{pin3zhong3}[][HSK 5]
    \definition[个]{s.}{raça; linhagem; variedade; refere-se a um grupo de organismos com características genéticas comuns, formados por meio da seleção e cultivo artificiais de culturas, gado, aves, etc. | variedade; sortimento; referência geral ao tipo de item}
  \end{phonetics}
\end{entry}

\begin{entry}{品德}{9,15}{⼝、⼻}
  \begin{phonetics}{品德}{pin3de2}
    \definition{s.}{caráter moral | moralidade}
  \end{phonetics}
\end{entry}

\begin{entry}{哄}{9}{⼝}
  \begin{phonetics}{哄}{hong1}
    \definition{s.}{gargalhadas | risadas ruidosas | algazarra | rugido | clamor}
  \end{phonetics}
  \begin{phonetics}{哄}{hong3}
    \definition{v.}{enganar | persuadir | divertir (uma criança)}
  \end{phonetics}
  \begin{phonetics}{哄}{hong4}
    \definition{s.}{tumulto | agitação | perturbação}
  \end{phonetics}
\end{entry}

\begin{entry}{哇塞}{9,13}{⼝、⼟}
  \begin{phonetics}{哇塞}{wa1sai1}
    \definition{interj.}{(gíria) Uau!}
  \end{phonetics}
\end{entry}

\begin{entry}{哇噻}{9,16}{⼝、⼝}
  \begin{phonetics}{哇噻}{wa1sai1}
    \variantof{哇塞}
  \end{phonetics}
\end{entry}

\begin{entry}{哈马斯}{9,3,12}{⼝、⾺、⽄}
  \begin{phonetics}{哈马斯}{ha1ma3si1}
    \definition*{s.}{Hamas (Grupo Palestino)}
  \end{phonetics}
\end{entry}

\begin{entry}{哈哈}{9,9}{⼝、⼝}
  \begin{phonetics}{哈哈}{ha1 ha1}[][HSK 3]
    \definition{expr.}{(onomatopéia)  ha ha; o som de uma risada alta}
  \end{phonetics}
\end{entry}

\begin{entry}{响}{9}{⼝}
  \begin{phonetics}{响}{xiang3}[][HSK 2]
    \definition{adj.}{barulhento}
    \definition[声,阵]{s.}{som | barulho | eco}
    \definition{v.}{fazer um som | soar | tocar}
  \end{phonetics}
\end{entry}

\begin{entry}{哪}{9}{⼝}
  \begin{phonetics}{哪}{na3}[][HSK 1,4]
    \definition{adv.}{para expressar uma pergunta retórica}
    \definition{pron.}{qual?; o que? | qualquer; ser usado em um sentido geral}
  \end{phonetics}
  \begin{phonetics}{哪}{na5}
    \definition{part.}{usado depois de uma palavra com a terminação -n, é equivalente a ``啊''}
  \seealsoref{啊}{a5}
  \end{phonetics}
  \begin{phonetics}{哪}{nei3}
    \definition{part.}{qual? (interrogativo, seguido de classificador ou numeral-classificador)}
  \end{phonetics}
\end{entry}

\begin{entry}{哪儿}{9,2}{⼝、⼉}
  \begin{phonetics}{哪儿}{na3r5}[][HSK 1]
    \definition{adv.}{onde?}
  \end{phonetics}
\end{entry}

\begin{entry}{哪里}{9,7}{⼝、⾥}
  \begin{phonetics}{哪里}{na3 li3}[][HSK 1]
    \definition{adv.}{onde?}
  \end{phonetics}
\end{entry}

\begin{entry}{哪些}{9,8}{⼝、⼆}
  \begin{phonetics}{哪些}{na3xie1}[][HSK 1]
    \definition{pron.}{quais?}
  \end{phonetics}
\end{entry}

\begin{entry}{哪国人}{9,8,2}{⼝、⼞、⼈}
  \begin{phonetics}{哪国人}{na3 guo2ren2}
    \definition{expr.}{de qual país?}
  \end{phonetics}
\end{entry}

\begin{entry}{哪怕}{9,8}{⼝、⼼}
  \begin{phonetics}{哪怕}{na3pa4}[][HSK 4]
    \definition{conj.}{mesmo; mesmo se; mesmo que; não importa o quão}
  \end{phonetics}
\end{entry}

\begin{entry}{型}{9}{⼟}
  \begin{phonetics}{型}{xing2}[][HSK 4]
    \definition{s.}{molde; modelo | modelo; tipo; padrão}
  \end{phonetics}
\end{entry}

\begin{entry}{型号}{9,5}{⼟、⼝}
  \begin{phonetics}{型号}{xing2 hao4}[][HSK 4]
    \definition[个,种]{s.}{modelo; tipo; refere-se ao desempenho, às especificações e ao tamanho de aeronaves, máquinas, implementos agrícolas, etc.}
  \end{phonetics}
\end{entry}

\begin{entry}{垫子}{9,3}{⼟、⼦}
  \begin{phonetics}{垫子}{dian4zi5}
    \definition{s.}{colchão | esteira | almofada}
  \end{phonetics}
\end{entry}

\begin{entry}{城}{9}{⼟}
  \begin{phonetics}{城}{cheng2}[][HSK 3]
    \definition*{s.}{sobrenome Cheng}
    \definition[座,道,个]{s.}{muralha da cidade; muro | cidade}
  \end{phonetics}
\end{entry}

\begin{entry}{城市}{9,5}{⼟、⼱}
  \begin{phonetics}{城市}{cheng2shi4}[][HSK 3]
    \definition[个,座]{s.}{cidade}
  \end{phonetics}
\end{entry}

\begin{entry}{城里}{9,7}{⼟、⾥}
  \begin{phonetics}{城里}{cheng2 li3}[][HSK 5]
    \definition{s.}{na cidade; dentro da cidade; originalmente referia-se à área dentro das muralhas da cidade, agora refere-se principalmente à área urbana}
  \end{phonetics}
\end{entry}

\begin{entry}{城度}{9,9}{⼟、⼴}
  \begin{phonetics}{城度}{cheng2du4}[][HSK 3]
    \definition*{s.}{Cidade}
  \end{phonetics}
\end{entry}

\begin{entry}{城堡}{9,12}{⼟、⼟}
  \begin{phonetics}{城堡}{cheng2bao3}
    \definition*{s.}{castelo | torre (peça de xadrez)}
  \end{phonetics}
\end{entry}

\begin{entry}{复习}{9,3}{⼢、⼄}
  \begin{phonetics}{复习}{fu4xi2}[][HSK 2]
    \definition{s.}{revisão}
    \definition{v.}{rever | revisar}
  \end{phonetics}
\end{entry}

\begin{entry}{复印}{9,5}{⼢、⼙}
  \begin{phonetics}{复印}{fu4yin4}[][HSK 3]
    \definition{v.}{fotografar; fotocopiar; duplicar}
  \end{phonetics}
\end{entry}

\begin{entry}{复杂}{9,6}{⼢、⽊}
  \begin{phonetics}{复杂}{fu4za2}[][HSK 3]
    \definition{adj.}{complexo; complicado}
  \end{phonetics}
\end{entry}

\begin{entry}{复制}{9,8}{⼢、⼑}
  \begin{phonetics}{复制}{fu4zhi4}[][HSK 4]
    \definition{v.}{copiar; duplicar; reproduzir; fazer uma cópia de; fazer uma cópia do original ou reproduzi-lo, reimprimi-lo ou copiá-lo em sua forma original (geralmente referindo-se a relíquias culturais ou obras de arte)}
  \end{phonetics}
\end{entry}

\begin{entry}{复刻}{9,8}{⼢、⼑}
  \begin{phonetics}{复刻}{fu4ke4}
    \definition{v.}{reimprimir (um trabalho que esteve fora do catálogo) | reeditar (um disco de vinil, um CD, etc.) | replicar | recriar | (empréstimo linguístico) (computação) \emph{fork}}
  \end{phonetics}
\end{entry}

\begin{entry}{复活节}{9,9,5}{⼢、⽔、⾋}
  \begin{phonetics}{复活节}{fu4huo2jie2}
    \definition*{s.}{Páscoa}
  \end{phonetics}
\end{entry}

\begin{entry}{奏效}{9,10}{⼤、⽁}
  \begin{phonetics}{奏效}{zou4xiao4}
    \definition{v.}{mostrar resultados | ser eficaz}
  \end{phonetics}
\end{entry}

\begin{entry}{奖}{9}{⼤}
  \begin{phonetics}{奖}{jiang3}[][HSK 4]
    \definition[个,次]{s.}{prêmio; recompensa | elogio; loa}
    \definition{v.}{elogiar; recompensar; recomendar; incentivar}
  \end{phonetics}
\end{entry}

\begin{entry}{奖励}{9,7}{⼤、⼒}
  \begin{phonetics}{奖励}{jiang3li4}[][HSK 5]
    \definition{s.}{prêmio; recompensa; dinheiro ou honras dadas em troca de elogios ou incentivos}
    \definition{v.}{recompensar; incentivar; encorajar}
  \end{phonetics}
\end{entry}

\begin{entry}{奖学金}{9,8,8}{⼤、⼦、⾦}
  \begin{phonetics}{奖学金}{jiang3 xue2 jin1}[][HSK 4]
    \definition[笔]{s.}{bolsa de estudos; exposição; prêmios concedidos por escolas, organizações ou indivíduos a alunos com bom desempenho acadêmico}
  \end{phonetics}
\end{entry}

\begin{entry}{奖金}{9,8}{⼤、⾦}
  \begin{phonetics}{奖金}{jiang3jin1}[][HSK 4]
    \definition[个,笔]{s.}{bônus; recompensa; prêmio; prêmio em dinheiro; dinheiro de recompensa, dinheiro dado às pessoas para incentivá-las ou elogiá-las por terem se saído bem em alguma coisa}
  \end{phonetics}
\end{entry}

\begin{entry}{姜}{9}{⼥}
  \begin{phonetics}{姜}{jiang1}
    \definition*{s.}{sobrenome Jiang}
    \definition{s.}{gengibre}
  \end{phonetics}
\end{entry}

\begin{entry}{孩子}{9,3}{⼦、⼦}
  \begin{phonetics}{孩子}{hai2zi5}[][HSK 1]
    \definition{s.}{criança | filho}
  \end{phonetics}
\end{entry}

\begin{entry}{客人}{9,2}{⼧、⼈}
  \begin{phonetics}{客人}{ke4ren2}[][HSK 2]
    \definition{s.}{visitante | convidado | cliente | passageiro | viajante}
  \end{phonetics}
\end{entry}

\begin{entry}{客厅}{9,4}{⼧、⼚}
  \begin{phonetics}{客厅}{ke4ting1}[][HSK 5]
    \definition[间,个]{s.}{sala de estar; sala de visitas; sala para receber convidados}
  \end{phonetics}
\end{entry}

\begin{entry}{客户}{9,4}{⼧、⼾}
  \begin{phonetics}{客户}{ke4hu4}[][HSK 5]
    \definition{s.}{cliente; consumidor}
  \end{phonetics}
\end{entry}

\begin{entry}{客气}{9,4}{⼧、⽓}
  \begin{phonetics}{客气}{ke4qi5}[][HSK 5]
    \definition{adj.}{educado; modesto; cortês}
    \definition{v.}{ser educado; ser cortês; fazer comentários educados ou agir educadamente}
  \end{phonetics}
\end{entry}

\begin{entry}{客观}{9,6}{⼧、⾒}
  \begin{phonetics}{客观}{ke4guan1}[][HSK 3]
    \definition{adj.}{objetivo; justo e razoável; imparcial}
    \definition{s.}{objetivo}
  \end{phonetics}
\end{entry}

\begin{entry}{宣布}{9,5}{⼧、⼱}
  \begin{phonetics}{宣布}{xuan1bu4}[][HSK 3]
    \definition{v.}{declarar; proclamar; pronunciar; anunciar | anunciar oficialmente e publicamente as últimas decisões e situações a todos}
  \end{phonetics}
\end{entry}

\begin{entry}{宣传}{9,6}{⼧、⼈}
  \begin{phonetics}{宣传}{xuan1chuan2}[][HSK 3]
    \definition{v.}{propagar; disseminar; conduzir propaganda | explicar às massas para que elas possam acreditar e agir de acordo}
  \end{phonetics}
\end{entry}

\begin{entry}{宣扬}{9,6}{⼧、⼿}
  \begin{phonetics}{宣扬}{xuan1yang2}
    \definition{v.}{divulgar | anunciar | espalhar por toda parte}
  \end{phonetics}
\end{entry}

\begin{entry}{室}{9}{⼧}
  \begin{phonetics}{室}{shi4}[][HSK 3]
    \definition*{s.}{sobrenome Shi | Shi, uma das mansões lunares}
    \definition{s.}{sala; aposento; cômodo |  seção; escritório | esposa}
  \end{phonetics}
\end{entry}

\begin{entry}{宪制}{9,8}{⼧、⼑}
  \begin{phonetics}{宪制}{xian4zhi4}
    \definition{adj.}{constitucional}
    \definition{s.}{sistema de governo constitucional}
  \end{phonetics}
\end{entry}

\begin{entry}{宪法法院}{9,8,8,9}{⼧、⽔、⽔、⾩}
  \begin{phonetics}{宪法法院}{xian4fa3fa3yuan4}
    \definition{s.}{tribunal constitucional}
  \end{phonetics}
\end{entry}

\begin{entry}{宪政}{9,9}{⼧、⽁}
  \begin{phonetics}{宪政}{xian4zheng4}
    \definition{s.}{governo constitucional}
  \end{phonetics}
\end{entry}

\begin{entry}{封}{9}{⼨}
  \begin{phonetics}{封}{feng1}[][HSK 2,5]
    \definition*{s.}{sobrenome Feng}
    \definition{clas.}{para objetos selados, especialmente cartas}
    \definition{s.}{envelope; pacote; embalagens ou sacos de papel que são lacrados ou usados para lacrar coisas |
feudalismo}
    \definition{v.}{fechar; selar; vedar; lacrar | acender uma fogueira | conceder (um título, território, etc.) a; nos tempos antigos, o imperador concedia títulos ou titulações aos seus súditos}
  \end{phonetics}
\end{entry}

\begin{entry}{封口}{9,3}{⼨、⼝}
  \begin{phonetics}{封口}{feng1kou3}
    \definition{v.}{selar | fechar | curar (uma ferida) | manter os lábios selados}
  \end{phonetics}
\end{entry}

\begin{entry}{封印}{9,5}{⼨、⼙}
  \begin{phonetics}{封印}{feng1yin4}
    \definition{s.}{selo (em envelopes)}
  \end{phonetics}
\end{entry}

\begin{entry}{封闭}{9,6}{⼨、⾨}
  \begin{phonetics}{封闭}{feng1bi4}[][HSK 4]
    \definition{adj.}{fechado; aqueles que não têm contato com o mundo exterior; aqueles que são muito conservadores (em seu pensamento) e não se comunicam com os outros}
    \definition{v.}{selar; fechar; lacrar; vedar; de modo a impedir a passagem, o uso ou a abertura}
  \end{phonetics}
\end{entry}

\begin{entry}{封冻}{9,7}{⼨、⼎}
  \begin{phonetics}{封冻}{feng1dong4}
    \definition{v.}{congelar (água ou terra)}
  \end{phonetics}
\end{entry}

\begin{entry}{封底}{9,8}{⼨、⼴}
  \begin{phonetics}{封底}{feng1di3}
    \definition{s.}{contracapa de um livro}
  \end{phonetics}
\end{entry}

\begin{entry}{封建}{9,8}{⼨、⼵}
  \begin{phonetics}{封建}{feng1jian4}
    \definition{adj.}{feudal}
    \definition{s.}{feudalismo}
  \end{phonetics}
\end{entry}

\begin{entry}{封面}{9,9}{⼨、⾯}
  \begin{phonetics}{封面}{feng1mian4}
    \definition{s.}{capa (de uma publicação) | sobrecapa}
  \end{phonetics}
\end{entry}

\begin{entry}{封斋}{9,10}{⼨、⽂}
  \begin{phonetics}{封斋}{feng1zhai1}
    \definition*{s.}{Ramadã (Islã)}
  \end{phonetics}
\end{entry}

\begin{entry}{封盖}{9,11}{⼨、⽫}
  \begin{phonetics}{封盖}{feng1gai4}
    \definition{s.}{boné | capa | selo}
    \definition{v.}{cobrir}
  \end{phonetics}
\end{entry}

\begin{entry}{将}{9}{⼨}
  \begin{phonetics}{将}{jiang1}[][HSK 5]
    \definition*{s.}{sobrenome Jiang}
    \definition{adv.}{estar indo para; parcialmente\dots parcialmente\dots}
    \definition{part.}{expressar uma direção, como ``进来'', ``出去''; usado no meio de verbos e complementos que indicam tendência, como ``进来'', ``出去'' etc.}
    \definition{prep.}{com; por meio de; por | usado da mesma forma que ``把''}
    \definition{v.}{fazer algo; lidar com (um assunto) | dar um cheque-mate | cuidar (da saúde) | incitar alguém a agir; desafiar; estimular | segurar; pegar | colocar; tirar | levar; trazer | dar suporte; dar apoio}
  \seealsoref{把}{ba3}
  \seealsoref{出去}{chu1 qu4}
  \seealsoref{进来}{jin4 lai2}
  \end{phonetics}
  \begin{phonetics}{将}{jiang4}
    \definition{s.}{general; nome do posto; abaixo de marechal de campo; acima de coronel}
    \definition{v.}{comandar; liderar}
  \end{phonetics}
  \begin{phonetics}{将}{qiang1}
    \definition{v.}{pedir; apelar para}
  \end{phonetics}
\end{entry}

\begin{entry}{将来}{9,7}{⼨、⽊}
  \begin{phonetics}{将来}{jiang1lai2}[][HSK 3]
    \definition[个]{s.}{futuro}
  \end{phonetics}
\end{entry}

\begin{entry}{将近}{9,7}{⼨、⾡}
  \begin{phonetics}{将近}{jiang1jin4}[][HSK 3]
    \definition{adv.}{quase}
  \end{phonetics}
\end{entry}

\begin{entry}{将要}{9,9}{⼨、⾑}
  \begin{phonetics}{将要}{jiang1 yao4}[][HSK 5]
    \definition{adv.}{irá; deverá; estará prestes a; irá a; indica que um ato ou situação ocorre logo em seguida}
  \end{phonetics}
\end{entry}

\begin{entry}{尝}{9}{⼩}
  \begin{phonetics}{尝}{chang2}[][HSK 5]
    \definition{adv.}{alguma vez; uma vez}
    \definition{v.}{provar; experimentar o sabor de | provar; experimentar; conhecer | tentar}
  \end{phonetics}
\end{entry}

\begin{entry}{尝试}{9,8}{⼩、⾔}
  \begin{phonetics}{尝试}{chang2shi4}[][HSK 5]
    \definition{v.}{tentar; provar; experimentar}
  \end{phonetics}
\end{entry}

\begin{entry}{屋}{9}{⼫}
  \begin{phonetics}{屋}{wu1}[][HSK 5]
    \definition[间,座]{s.}{casa | quarto}
  \end{phonetics}
\end{entry}

\begin{entry}{屋子}{9,3}{⼫、⼦}
  \begin{phonetics}{屋子}{wu1zi5}[][HSK 3]
    \definition[间,座,栋]{s.}{casa}
  \end{phonetics}
\end{entry}

\begin{entry}{屌丝}{9,5}{⼫、⼀}
  \begin{phonetics}{屌丝}{diao3si1}
    \definition{adj.}{panaca | zé-ninguém | (gíria de \emph{Internet}) \emph{looser}}
  \end{phonetics}
\end{entry}

\begin{entry}{屎}{9}{⼫}
  \begin{phonetics}{屎}{shi3}
    \definition{s.}{fezes | excrementos | (forma ligada) secreção (do ouvido, olho, etc.)}
  \end{phonetics}
\end{entry}

\begin{entry}{差}{9}{⼯}
  \begin{phonetics}{差}{cha4}[][HSK 1]
    \definition{adv.}{ligeiramente | comparativamente | um pouco}
    \definition{s.}{differença | dissimilaridade | engano | equívoco}
  \end{phonetics}
\end{entry}

\begin{entry}{差(一)点儿}{9,1,9,2}{⼯、⼀、⽕、⼉}
  \begin{phonetics}{差(一)点儿}{cha1yi4dian3r5}[][HSK 5]
    \definition{adv.}{quase; à beira de; praticamente; aproximadamente; significa que algo está perto de ser alcançado, mas não foi alcançado, ou algo foi alcançado, mas mal foi alcançado}
  \end{phonetics}
\end{entry}

\begin{entry}{差不多}{9,4,6}{⼯、⼀、⼣}
  \begin{phonetics}{差不多}{cha4bu5duo1}[][HSK 2]
    \definition{adj.}{mais ou menos}
    \definition{adv.}{quase perto}
  \end{phonetics}
\end{entry}

\begin{entry}{差别}{9,7}{⼯、⼑}
  \begin{phonetics}{差别}{cha1bie2}[][HSK 5]
    \definition{s.}{diferença; disparidade; dissimilaridade; distinção; não semelhança; diferenças na forma ou no conteúdo}
  \end{phonetics}
\end{entry}

\begin{entry}{差点儿}{9,9,2}{⼯、⽕、⼉}
  \begin{phonetics}{差点儿}{cha4dian3r5}
    \definition{adv.}{por pouco | por um triz | quase}
  \end{phonetics}
\end{entry}

\begin{entry}{差距}{9,11}{⼯、⾜}
  \begin{phonetics}{差距}{cha1ju4}[][HSK 5]
    \definition[个,些,段]{s.}{lacuna; disparidade; discrepância; diferença; grau de diferença entre as coisas, especialmente em termos de distância de algum padrão.}
  \end{phonetics}
\end{entry}

\begin{entry}{帝国}{9,8}{⼱、⼞}
  \begin{phonetics}{帝国}{di4guo2}
    \definition{adj.}{imperial}
    \definition{s.}{império}
  \end{phonetics}
\end{entry}

\begin{entry}{带}{9}{⼱}
  \begin{phonetics}{带}{dai4}[][HSK 2]
    \definition{v.}{levar | trazer}
  \end{phonetics}
\end{entry}

\begin{entry}{带动}{9,6}{⼱、⼒}
  \begin{phonetics}{带动}{dai4 dong4}[][HSK 3]
    \definition{v.}{dirigir; ativar; fazer algo funcionar; acionar | liderar; trazer; estimular; motivar; atrair}
  \end{phonetics}
\end{entry}

\begin{entry}{带有}{9,6}{⼱、⽉}
  \begin{phonetics}{带有}{dai4 you3}[][HSK 5]
    \definition{v.}{ter; envolver; carregar}
  \end{phonetics}
\end{entry}

\begin{entry}{带来}{9,7}{⼱、⽊}
  \begin{phonetics}{带来}{dai4 lai2}[][HSK 2]
    \definition{v.}{trazer | (figurativo) provocar, produzir}
  \end{phonetics}
\end{entry}

\begin{entry}{带领}{9,11}{⼱、⾴}
  \begin{phonetics}{带领}{dai4ling3}[][HSK 3]
    \definition{v.}{guiar | liderar}
  \end{phonetics}
\end{entry}

\begin{entry}{帮}{9}{⼱}
  \begin{phonetics}{帮}{bang1}[][HSK 1]
    \definition{clas.}{para alguém (como uma ajuda)}
    \definition{s.}{gangue | grupo | contratado (como trabalhador) | camada externa | festa | sociedade secreta}
    \definition{v.}{ajudar | apoiar}
  \end{phonetics}
\end{entry}

\begin{entry}{帮忙}{9,6}{⼱、⼼}
  \begin{phonetics}{帮忙}{bang1 mang2}[][HSK 1]
    \definition{v.+compl.}{ajudar | dar uma mão | estender a mão | fazer um favor}
  \end{phonetics}
\end{entry}

\begin{entry}{帮佣}{9,7}{⼱、⼈}
  \begin{phonetics}{帮佣}{bang1yong1}
    \definition{s.}{ajudante doméstico | servo}
  \end{phonetics}
\end{entry}

\begin{entry}{帮助}{9,7}{⼱、⼒}
  \begin{phonetics}{帮助}{bang1zhu4}[][HSK 2]
    \definition[种]{s.}{ajuda | assistência}
    \definition{v.}{ajudar | dar assistência}
  \end{phonetics}
\end{entry}

\begin{entry}{帮教}{9,11}{⼱、⽁}
  \begin{phonetics}{帮教}{bang1jiao4}
    \definition{v.}{orientar}
  \end{phonetics}
\end{entry}

\begin{entry}{幽默}{9,16}{⼳、⿊}
  \begin{phonetics}{幽默}{you1mo4}[][HSK 5]
    \definition{adj.}{humorístico; interessante ou engraçado, mas com um significado profundo}
    \definition{s.}{humor; lado engraçado; graça; características, temperamento, palavras ou comportamentos interessantes, engraçados ou significativos}
  \end{phonetics}
\end{entry}

\begin{entry}{度}{9}{⼴}
  \begin{phonetics}{度}{du4}[][HSK 2]
    \definition{clas.}{para temperatura, etc. | para eventos e ocorrências}
    \definition{s.}{grau (ângulo, temperatura, etc.) | kilowatt-hora}
  \end{phonetics}
  \begin{phonetics}{度}{duo2}
    \definition{v.}{estimar}
  \end{phonetics}
\end{entry}

\begin{entry}{度过}{9,6}{⼴、⾡}
  \begin{phonetics}{度过}{du4guo4}[][HSK 4]
    \definition{s.}{passar o tempo; fazer o tempo desaparecer no trabalho, na vida, no lazer e no descanso}
  \end{phonetics}
\end{entry}

\begin{entry}{弯}{9}{⼸}
  \begin{phonetics}{弯}{wan1}[][HSK 4]
    \definition{adj.}{curvo; dobrado; torto; flexível; tortuoso}
    \definition{s.}{curva; dobra}
    \definition{v.}{curvar; dobrar; flexionar}
  \end{phonetics}
\end{entry}

\begin{entry}{待}{9}{⼻}
  \begin{phonetics}{待}{dai1}[][HSK 5]
    \definition{v.}{ficar; permanecer | ir além (de um período de tempo)}
  \end{phonetics}
  \begin{phonetics}{待}{dai4}
    \definition{s.}{sobrenome Dai}
    \definition{v.}{tratar; lidar com | entreter; receber (convidados) | aguardar; esperar por | precisar; necessitar | desejar; pretender; querer}
  \end{phonetics}
\end{entry}

\begin{entry}{待遇}{9,12}{⼻、⾡}
  \begin{phonetics}{待遇}{dai4yu4}[][HSK 4]
    \definition[种,项,份]{s.}{tratamento; refere-se a direitos, status social, etc. | salário; ordenado; remuneração}
  \end{phonetics}
\end{entry}

\begin{entry}{很}{9}{⼻}
  \begin{phonetics}{很}{hen3}[][HSK 1]
    \definition{adv.}{bastante | muito | terrivelmente | advérbio de grau}
  \end{phonetics}
\end{entry}

\begin{entry}{律师}{9,6}{⼻、⼱}
  \begin{phonetics}{律师}{lv4shi1}[][HSK 4]
    \definition[名,个,位]{s.}{advogado; procurador; profissionais encarregados pelas partes ou nomeados pelo tribunal para auxiliar as partes no litígio, para comparecer ao tribunal para defesa e para tratar de assuntos jurídicos relacionados, de acordo com a lei}
  \end{phonetics}
\end{entry}

\begin{entry}{怎}{9}{⼼}
  \begin{phonetics}{怎}{zen3}
    \definition{adv.}{como}
  \end{phonetics}
\end{entry}

\begin{entry}{怎么}{9,3}{⼼、⼃}
  \begin{phonetics}{怎么}{zen3me5}[][HSK 1]
    \definition{pron.}{como? | o que?}
  \end{phonetics}
\end{entry}

\begin{entry}{怎么了}{9,3,2}{⼼、⼃、⼅}
  \begin{phonetics}{怎么了}{zen3me5le5}
    \definition{expr.}{O que aconteceu? | O que está acontecendo? | E aí?}
  \end{phonetics}
\end{entry}

\begin{entry}{怎么办}{9,3,4}{⼼、⼃、⼒}
  \begin{phonetics}{怎么办}{zen3 me5 ban4}[][HSK 2]
    \definition{adv.}{o que fazer?}
  \end{phonetics}
\end{entry}

\begin{entry}{怎么回事}{9,3,6,8}{⼼、⼃、⼞、⼅}
  \begin{phonetics}{怎么回事}{zen3me5hui2shi4}
    \definition{expr.}{O que aconteceu? | O que se passou?}
  \end{phonetics}
\end{entry}

\begin{entry}{怎么样}{9,3,10}{⼼、⼃、⽊}
  \begin{phonetics}{怎么样}{zen3me5yang4}[][HSK 2]
    \definition{adv.}{como? | que tal?}
  \end{phonetics}
\end{entry}

\begin{entry}{怎么得了}{9,3,11,2}{⼼、⼃、⼻、⼅}
  \begin{phonetics}{怎么得了}{zen3me5de2liao3}
    \definition{expr.}{Como isso pode ser? | Que bagunça horrível! | O que deve ser feito?}
  \end{phonetics}
\end{entry}

\begin{entry}{怎么搞的}{9,3,13,8}{⼼、⼃、⼿、⽩}
  \begin{phonetics}{怎么搞的}{zen3me5gao3de5}
    \definition{expr.}{Como isso aconteceu? | O que deu errado? | E aí? | O que está errado?}
  \end{phonetics}
\end{entry}

\begin{entry}{怎样}{9,10}{⼼、⽊}
  \begin{phonetics}{怎样}{zen3 yang4}[][HSK 2]
    \definition{pron.}{como | o que | de uma certa maneira | de qualquer maneira | não importa o quão}
  \end{phonetics}
\end{entry}

\begin{entry}{怒骂}{9,9}{⼼、⾺}
  \begin{phonetics}{怒骂}{nu4ma4}
    \definition{v.}{praguejar de raiva}
  \end{phonetics}
\end{entry}

\begin{entry}{思考}{9,6}{⼼、⽼}
  \begin{phonetics}{思考}{si1kao3}[][HSK 4]
    \definition{v.}{pensar; ponderar; considerar; deliberar; envolver-se em atividades de pensamento, como análise, síntese, julgamento, raciocínio e generalização}
  \end{phonetics}
\end{entry}

\begin{entry}{思维}{9,11}{⼼、⽷}
  \begin{phonetics}{思维}{si1wei2}[][HSK 5]
    \definition[种]{s.}{pensamento; reflexão; organizar e transformar os materiais obtidos através do conhecimento sensorial para formar conceitos, julgamentos e raciocínios}
    \definition{v.}{pensar;}
  \end{phonetics}
\end{entry}

\begin{entry}{思想}{9,13}{⼼、⼼}
  \begin{phonetics}{思想}{si1xiang3}[][HSK 3]
    \definition[个]{s.}{reflexão; pensamento; ideologia | ideia}
  \end{phonetics}
\end{entry}

\begin{entry}{急}{9}{⼼}
  \begin{phonetics}{急}{ji2}[][HSK 2]
    \definition{adj.}{impaciente |ansioso | irritado | aborrecido |violento | urgente | premente}
    \definition{s.}{urgência | emergência}
    \definition{v.}{preocupar | estar ansioso para ajudar}
  \end{phonetics}
\end{entry}

\begin{entry}{急忙}{9,6}{⼼、⼼}
  \begin{phonetics}{急忙}{ji2mang2}[][HSK 4]
    \definition{adv.}{apressadamente; com pressa}
  \end{phonetics}
\end{entry}

\begin{entry}{急救}{9,11}{⼼、⽁}
  \begin{phonetics}{急救}{ji2jiu4}
    \definition{s.}{primeiros socorros}
    \definition{v.}{dar tratamento de emergência}
  \end{phonetics}
\end{entry}

\begin{entry}{怨}{9}{⼼}
  \begin{phonetics}{怨}{yuan4}[][HSK 5]
    \definition{s.}{ressentimento; inimizade; rancor}
    \definition{v.}{culpar; reclamar}
  \end{phonetics}
\end{entry}

\begin{entry}{怹}{9}{⼼}
  \begin{phonetics}{怹}{tan1}
    \definition{pron.}{ele, ela (cortês, em oposição a 他)}
    \seeref{他}{ta1}
  \end{phonetics}
\end{entry}

\begin{entry}{总}{9}{⼼}
  \begin{phonetics}{总}{zong3}[][HSK 3]
    \definition{adj.}{total; geral; global | responsável (liderança)}
    \definition{adv.}{sempre; invariavelmente | de qualquer forma; afinal; eventualmente; mais cedo ou mais tarde | seguramente; provavelmente; certamente}
    \definition{v.}{resumir; juntar; reunir}
  \end{phonetics}
\end{entry}

\begin{entry}{总之}{9,3}{⼼、⼂}
  \begin{phonetics}{总之}{zong3zhi1}[][HSK 4]
    \definition{conj.}{em uma palavra; em suma; em resumo; indica que a declaração seguinte é uma declaração geral}
  \end{phonetics}
\end{entry}

\begin{entry}{总长}{9,4}{⼼、⾧}
  \begin{phonetics}{总长}{zong3chang2}
    \definition{s.}{comprimento total}
  \end{phonetics}
\end{entry}

\begin{entry}{总务}{9,5}{⼼、⼒}
  \begin{phonetics}{总务}{zong3wu4}
    \definition{s.}{divisão de assuntos gerais | assuntos gerais | pessoa responsável geral}
  \end{phonetics}
\end{entry}

\begin{entry}{总台}{9,5}{⼼、⼝}
  \begin{phonetics}{总台}{zong3tai2}
    \definition{s.}{recepção | balcão de recepção}
  \end{phonetics}
\end{entry}

\begin{entry}{总价}{9,6}{⼼、⼈}
  \begin{phonetics}{总价}{zong3jia4}
    \definition{s.}{preço total}
  \end{phonetics}
\end{entry}

\begin{entry}{总共}{9,6}{⼼、⼋}
  \begin{phonetics}{总共}{zong3gong4}[][HSK 4]
    \definition{adv.}{em tudo; em todos; no total; completamente; totalmente; em conjunto}
  \end{phonetics}
\end{entry}

\begin{entry}{总体}{9,7}{⼼、⼈}
  \begin{phonetics}{总体}{zong3 ti3}[][HSK 5]
    \definition{s.}{total; geral; conjunto; totalidade; massa; população; o todo formado pela união de vários indivíduos; a totalidade das coisas}
  \end{phonetics}
\end{entry}

\begin{entry}{总线}{9,8}{⼼、⽷}
  \begin{phonetics}{总线}{zong3xian4}
    \definition{s.}{barramento (computador) | \emph{computer bus}}
  \end{phonetics}
\end{entry}

\begin{entry}{总是}{9,9}{⼼、⽇}
  \begin{phonetics}{总是}{zong3shi4}[][HSK 3]
    \definition{adv.}{sempre; indica que algo está acontecendo por um período de tempo; um certo estado permanece inalterado
 | afinal; significa que não importa o que aconteça, haverá um resultado.}
  \end{phonetics}
\end{entry}

\begin{entry}{总结}{9,9}{⼼、⽷}
  \begin{phonetics}{总结}{zong3jie2}[][HSK 3]
    \definition[个,份]{s.}{resumo; conclusão obtida}
    \definition{v.}{resumir; sumariar; analisar a experiência da pesquisa e tirar conclusões}
  \end{phonetics}
\end{entry}

\begin{entry}{总统}{9,9}{⼼、⽷}
  \begin{phonetics}{总统}{zong3tong3}[][HSK 4]
    \definition*[个,位,名]{s.}{Presidente (de um país); Título dos líderes de determinadas repúblicas}
  \end{phonetics}
\end{entry}

\begin{entry}{总值}{9,10}{⼼、⼈}
  \begin{phonetics}{总值}{zong3zhi2}
    \definition{s.}{valor total}
  \end{phonetics}
\end{entry}

\begin{entry}{总站}{9,10}{⼼、⽴}
  \begin{phonetics}{总站}{zong3zhan4}
    \definition{s.}{terminal}
  \end{phonetics}
\end{entry}

\begin{entry}{总得}{9,11}{⼼、⼻}
  \begin{phonetics}{总得}{zong3dei3}
    \definition{adv.}{prestes a}
    \definition{v.}{dever | precisar}
  \end{phonetics}
\end{entry}

\begin{entry}{总理}{9,11}{⼼、⽟}
  \begin{phonetics}{总理}{zong3li3}[][HSK 4]
    \definition*[个,位,名]{s.}{Primeiro-Ministro do Conselho de Estado; Título do líder do Conselho de Estado da China | Título do chefe de governo em determinados países | Primeiro-Ministro; Título de líderes de determinados partidos políticos | Título dos chefes de determinadas instituições e empresas nos velhos tempos}
    \definition{v.}{assumir a responsabilidade total;}
  \end{phonetics}
\end{entry}

\begin{entry}{总裁}{9,12}{⼼、⾐}
  \begin{phonetics}{总裁}{zong3cai2}[][HSK 5]
    \definition[位,名]{s.}{presidente (de uma empresa); nomes de certos líderes de partidos políticos ou grandes empresas}
  \end{phonetics}
\end{entry}

\begin{entry}{总数}{9,13}{⼼、⽁}
  \begin{phonetics}{总数}{zong3 shu4}[][HSK 5]
    \definition{s.}{soma; total; totalidade; inventário; número total; soma total}
  \end{phonetics}
\end{entry}

\begin{entry}{总督}{9,13}{⼼、⽬}
  \begin{phonetics}{总督}{zong3du1}
    \definition*{s.}{Governador-Geral | Governador | Vice-Rei}
  \end{phonetics}
\end{entry}

\begin{entry}{总算}{9,14}{⼼、⽵}
  \begin{phonetics}{总算}{zong3suan4}[][HSK 5]
    \definition{adv.}{finalmente; por fim; indica que, após um longo período de tempo, um desejo finalmente se tornou realidade | suficiente; considerando tudo; no geral; considerando todos os aspectos; significa que, em geral, está tudo bem}
  \end{phonetics}
\end{entry}

\begin{entry}{恒星系}{9,9,7}{⼼、⽇、⽷}
  \begin{phonetics}{恒星系}{heng2xing1xi4}
    \definition{s.}{sistema estelar | galáxia}
  \end{phonetics}
\end{entry}

\begin{entry}{恢复}{9,9}{⼼、⼢}
  \begin{phonetics}{恢复}{hui1fu4}[][HSK 5]
    \definition{v.}{retomar; renovar; restaurar; voltar a | reviver; recuperar; reencontrar | restaurar; restabelecer; reabilitar; regenerar; ressurgir; restabelecer alguém em; recuperar o que foi perdido}
  \end{phonetics}
\end{entry}

\begin{entry}{T-恤}{9}{⼼}
  \begin{phonetics}{T-恤}{xu4}
    \definition{s.}{camiseta | pulôver | suéter}
  \end{phonetics}
\end{entry}

\begin{entry}{恨}{9}{⼼}
  \begin{phonetics}{恨}{hen4}[][HSK 5]
    \definition{s.}{ódio; resentimento}
    \definition{v.}{odiar}
  \end{phonetics}
\end{entry}

\begin{entry}{恰}{9}{⼼}
  \begin{phonetics}{恰}{qia4}
    \definition{adv.}{exatamente | apenas}
  \end{phonetics}
\end{entry}

\begin{entry}{恰好}{9,6}{⼼、⼥}
  \begin{phonetics}{恰好}{qia4hao3}
    \definition{adv.}{certo | por sorte | ao que parece | por sorte coincidência}
  \end{phonetics}
\end{entry}

\begin{entry}{恰到好处}{9,8,6,5}{⼼、⼑、⼥、⼡}
  \begin{phonetics}{恰到好处}{qia4dao4hao3chu4}
    \definition{expr.}{é simplesmente perfeito | é simplesmente correto}
  \end{phonetics}
\end{entry}

\begin{entry}{战}{9}{⼽}
  \begin{phonetics}{战}{zhan4}
    \definition{s.}{luta | guerra | batalha}
    \definition{v.}{lutar}
  \end{phonetics}
\end{entry}

\begin{entry}{战士}{9,3}{⼽、⼠}
  \begin{phonetics}{战士}{zhan4shi4}[][HSK 4]
    \definition[个]{s.}{soldado; membros mais jovens do exército | campeão; guerreiro; lutador; geralmente, uma pessoa que se engaja em alguma causa justa ou participa de alguma luta justa}
  \end{phonetics}
\end{entry}

\begin{entry}{战斗}{9,4}{⼽、⽃}
  \begin{phonetics}{战斗}{zhan4dou4}[][HSK 4]
    \definition[场,次]{s.}{luta; batalha; combate; ação; conflito armado entre as partes oponentes}
    \definition{v.}{lutar | trabalhar sob pressão}
  \end{phonetics}
\end{entry}

\begin{entry}{战争}{9,6}{⼽、⼑}
  \begin{phonetics}{战争}{zhan4zheng1}[][HSK 4]
    \definition[场,次]{s.}{guerra; conflito; luta armada entre povos, entre nações, entre classes ou entre grupos políticos}
  \end{phonetics}
\end{entry}

\begin{entry}{战胜}{9,9}{⼽、⾁}
  \begin{phonetics}{战胜}{zhan4 sheng4}[][HSK 4]
    \definition{v.}{derrotar; vencer; superar; triunfar sobre; metáfora para superar dificuldades e alcançar o sucesso}
  \end{phonetics}
\end{entry}

\begin{entry}{拜访}{9,6}{⼿、⾔}
  \begin{phonetics}{拜访}{bai4fang3}[][HSK 5]
    \definition{v.}{visitar; fazer uma visita (respeitosamente)}
  \end{phonetics}
\end{entry}

\begin{entry}{括号}{9,5}{⼿、⼝}
  \begin{phonetics}{括号}{kuo4 hao4}[][HSK 4]
    \definition{s.}{chaves, colchetes e parênteses (em fórmulas aritméticas ou algébricas, os símbolos que indicam a combinação e a ordem de vários números ou termos) | colchetes e parênteses usados como um tipo de sinal de pontuação para mostrar a parte explicativa de uma passagem em um texto}
  \end{phonetics}
\end{entry}

\begin{entry}{拼}{9}{⼿}
  \begin{phonetics}{拼}{pin1}[][HSK 5]
    \definition{v.}{montar; juntar as peças | dar tudo de si no trabalho; estar disposto a arriscar a vida (em lutas, no trabalho, etc.); fazer tudo o que for preciso; arriscar tudo}
  \end{phonetics}
\end{entry}

\begin{entry}{拼命}{9,8}{⼿、⼝}
  \begin{phonetics}{拼命}{pin1ming4}
    \definition{adv.}{com toda a força | desesperadamente}
    \definition{v.+compl.}{arriscar a vida de alguém | desafiar a morte | colocar-se em uma luta desesperada | fazer algo desesperadamente | exercer a maior força}
  \end{phonetics}
\end{entry}

\begin{entry}{拼音}{9,9}{⼿、⾳}
  \begin{phonetics}{拼音}{pin1yin1}
    \definition{s.}{escrita fonética | pinyin (romanização chinesa)}
  \end{phonetics}
\end{entry}

\begin{entry}{拾}{9}{⼿}
  \begin{phonetics}{拾}{shi2}[][HSK 5]
    \definition{num.}{dez (usado no lugar do numeral 十 em cheques, notas bancárias, etc., para evitar erros ou alterações)}
    \definition{v.}{pegar (do chão); recolher}
  \end{phonetics}
\end{entry}

\begin{entry}{持续}{9,11}{⼿、⽷}
  \begin{phonetics}{持续}{chi2xu4}[][HSK 3]
    \definition{v.}{durar; continuar; sustentar}
  \end{phonetics}
\end{entry}

\begin{entry}{挂}{9}{⼿}
  \begin{phonetics}{挂}{gua4}[][HSK 3]
    \definition{clas.}{para conjuntos ou sequência de itens}
    \definition{v.}{pendurar; colocar; suspender | interromper chamada (telefônica) | colocar alguém em contato com; ligar; telefonar
pegar carona; ser pego | ter em mente; estar preocupado com | ser revestido com; ser coberto com | colocar em registro; registrar}
  \end{phonetics}
\end{entry}

\begin{entry}{挂号}{9,5}{⼿、⼝}
  \begin{phonetics}{挂号}{gua4hao4}
    \definition{v.+compl.}{registrar-se (em um hospital, etc.) | enviar através de carta registrada}
  \end{phonetics}
\end{entry}

\begin{entry}{挂号信}{9,5,9}{⼿、⼝、⼈}
  \begin{phonetics}{挂号信}{gua4hao4xin4}
    \definition{s.}{carta registrada}
  \end{phonetics}
\end{entry}

\begin{entry}{指}{9}{⼿}
  \begin{phonetics}{指}{zhi3}[][HSK 3]
    \definition*{s.}{sobrenome Zhi}
    \definition{clas.}{dígito; largura do dedo; a largura de um dedo é chamada de ``一指'', que é usado para medir profundidade, largura, etc.}
    \definition{s.}{dedo}
    \definition{v.}{apontar para | (pelo) eriçar | indicar; mostrar-se; apontar; demonstrar | referir-se a; dirigir-se a | confiar em; contar com; depender de | criticar; repreender}
  \end{phonetics}
\end{entry}

\begin{entry}{指出}{9,5}{⼿、⼐}
  \begin{phonetics}{指出}{zhi3 chu1}[][HSK 3]
    \definition{v.}{apontar; indicar}
  \end{phonetics}
\end{entry}

\begin{entry}{指甲}{9,5}{⼿、⽥}
  \begin{phonetics}{指甲}{zhi3jia5}[][HSK 5]
    \definition{s.}{unha; unha de agulha; unha de dedo; camada córnea na ponta dos dedos}
  \end{phonetics}
\end{entry}

\begin{entry}{指示}{9,5}{⼿、⽰}
  \begin{phonetics}{指示}{zhi3shi4}[][HSK 5]
    \definition{s.}{diretriz; instruções; para orientar o trabalho, os superiores emitem opiniões verbais ou escritas aos subordinados}
    \definition{v.}{indicar; apontar; apontar para alguém | instruir; superiores emitem opiniões verbais ou escritas para orientar o trabalho dos subordinados}
  \end{phonetics}
\end{entry}

\begin{entry}{指导}{9,6}{⼿、⼨}
  \begin{phonetics}{指导}{zhi3dao3}[][HSK 3]
    \definition{s.}{guia; pessoa que faz trabalho de orientação}
    \definition{v.}{guiar; dirigir; instruir}
  \end{phonetics}
\end{entry}

\begin{entry}{指责}{9,8}{⼿、⾙}
  \begin{phonetics}{指责}{zhi3ze2}[][HSK 5]
    \definition{v.}{censurar; criticar; encontrar falhas; repreender}
  \end{phonetics}
\end{entry}

\begin{entry}{指南针}{9,9,7}{⼿、⼗、⾦}
  \begin{phonetics}{指南针}{zhi3nan2zhen1}
    \definition{s.}{bússola}
  \end{phonetics}
\end{entry}

\begin{entry}{指挥}{9,9}{⼿、⼿}
  \begin{phonetics}{指挥}{zhi3hui1}[][HSK 4]
    \definition[个]{s.}{diretor; comandante; despachante; operador | maestro; condutor; pessoa na frente de uma orquestra ou coro que dá instruções sobre como tocar ou cantar}
    \definition{v.}{dirigir; conduzir; comandar; direcionar; emitir ordens de agendamento}
  \end{phonetics}
\end{entry}

\begin{entry}{指标}{9,9}{⼿、⽊}
  \begin{phonetics}{指标}{zhi3biao1}[][HSK 5]
    \definition{s.}{meta; cota; norma; índice; objetivos a serem alcançados | alvo; índice; refletir os requisitos de desenvolvimento em determinados aspectos através de números absolutos ou percentagens de aumento ou diminuição, inclui indicadores quantitativos e qualitativos}
  \end{phonetics}
\end{entry}

\begin{entry}{按}{9}{⼿}
  \begin{phonetics}{按}{an4}[][HSK 3]
    \definition{v.}{pressionar | empurrar para baixo | deixar de lado | arquivar | restringir | controlar}
  \end{phonetics}
\end{entry}

\begin{entry}{按时}{9,7}{⼿、⽇}
  \begin{phonetics}{按时}{an4shi2}[][HSK 4]
    \definition{adv.}{na hora; no horário; pontualmente; de acordo com o tempo estipulado}
  \end{phonetics}
\end{entry}

\begin{entry}{按照}{9,13}{⼿、⽕}
  \begin{phonetics}{按照}{an4zhao4}[][HSK 3]
    \definition{prep.}{de acordo com; em conformidade com; à luz de; com base em}
  \end{phonetics}
\end{entry}

\begin{entry}{按摩}{9,15}{⼿、⼿}
  \begin{phonetics}{按摩}{an4mo2}[][HSK 5]
    \definition{s.}{massagem; empurrar, pressionar, beliscar e amassar o corpo de uma pessoa com as mãos para promover a circulação sanguínea, aumentar a resistência da pele e regular a função dos nervos}
  \end{phonetics}
\end{entry}

\begin{entry}{挑}{9}{⼿}
  \begin{phonetics}{挑}{tiao1}[][HSK 4]
    \definition{clas.}{para coisas que são escolhidas ou selecionadas | para coisas que podem ser usadas como palhetas}
    \definition{s.}{vara comprida com algo pendurado nas pontas; haste de transporte}
    \definition{v.}{escolher; selecionar | fazer picuinhas; ser hipercrítico; ser meticuloso; ser excessivamente rigoroso nos detalhes | carregar com uma haste de transporte; carregar no ombro; pendurar coisas nas pontas de varas longas e carregá-las em seus ombros}
  \end{phonetics}
  \begin{phonetics}{挑}{tiao3}[][HSK 4]
    \definition{s.}{um dos traços dos caracteres chineses; inclinado para cima da esquerda para a direita}
    \definition{v.}{levantar; elevar; erguer | levantar ou apoiar com uma extremidade de uma vara ou objeto semelhante; segurar ou apoiar com a ponta de uma vara etc. | causar conflitos deliberadamente; provocar deliberadamente um conflito | (método de bordado) usar uma agulha para levantar os fios de urdidura ou trama, com a agulha e a linha passando por baixo para formar padrões e desenhos}
  \end{phonetics}
\end{entry}

\begin{entry}{挑战}{9,9}{⼿、⼽}
  \begin{phonetics}{挑战}{tiao3zhan4}[][HSK 4]
    \definition{v.}{desafiar; deixar um oponente deliberadamente irritado e sair para lutar ou lutar consigo mesmo; estimular um oponente a lutar consigo mesmo}
  \end{phonetics}
\end{entry}

\begin{entry}{挑选}{9,9}{⼿、⾡}
  \begin{phonetics}{挑选}{tiao1 xuan3}[][HSK 4]
    \definition{v.}{escolher; optar; selecionar; escolher a pessoa ou coisa certa para o trabalho}
  \end{phonetics}
\end{entry}

\begin{entry}{挑衅}{9,11}{⼿、⾎}
  \begin{phonetics}{挑衅}{tiao3xin4}
    \definition{s.}{provocação}
    \definition{v.}{provocar}
  \end{phonetics}
\end{entry}

\begin{entry}{挖}{9}{⼿}
  \begin{phonetics}{挖}{wa1}
    \definition{v.}{cavar | escavar}
  \end{phonetics}
\end{entry}

\begin{entry}{挖掘机}{9,11,6}{⼿、⼿、⽊}
  \begin{phonetics}{挖掘机}{wa1jue2ji1}
    \definition{s.}{escavadeira | escavador | escavadora | pá mecânica}
  \end{phonetics}
\end{entry}

\begin{entry}{挡}{9}{⼿}
  \begin{phonetics}{挡}{dang3}[][HSK 5]
    \definition{s.}{persiana; veneziana; paralama; coisas para cobrir ou bloquear | caixa de câmbio (automóvel)}
    \definition{v.}{bloquear; resistir; manter afastado; afastar | cobrir; bloquear; atrapalhar}
  \end{phonetics}
  \begin{phonetics}{挡}{dang4}
    \definition{v.}{organizar}
  \end{phonetics}
\end{entry}

\begin{entry}{挡风玻璃}{9,4,9,14}{⼿、⾵、⽟、⽟}
  \begin{phonetics}{挡风玻璃}{dang3feng1bo1li5}
    \definition{s.}{parabrisa}
  \end{phonetics}
\end{entry}

\begin{entry}{挣}{9}{⼿}
  \begin{phonetics}{挣}{zheng4}[][HSK 5]
    \definition{v.}{empurrar e puxar; tentar se livrar; lutar para se libertar; esforçar-se para se libertar das amarras | ganhar; fazer; trabalhar para; trocar trabalho por}
  \end{phonetics}
\end{entry}

\begin{entry}{挣扎}{9,4}{⼿、⼿}
  \begin{phonetics}{挣扎}{zheng1zha2}
    \definition{v.}{lutar}
  \end{phonetics}
\end{entry}

\begin{entry}{挣钱}{9,10}{⼿、⾦}
  \begin{phonetics}{挣钱}{zheng4qian2}[][HSK 5]
    \definition{v.+compl.}{ganhar dinheiro; fazer dinheiro; lucrar; trabalhar para ganhar dinheiro}
  \end{phonetics}
\end{entry}

\begin{entry}{挣得}{9,11}{⼿、⼻}
  \begin{phonetics}{挣得}{zheng4de2}
    \definition{v.}{ganhar renda ou dinheiro}
  \end{phonetics}
\end{entry}

\begin{entry}{挤}{9}{⼿}
  \begin{phonetics}{挤}{ji3}[][HSK 5]
    \definition{adj.}{lotado; congestionado; descreve um grande número de pessoas ou coisas e muito pouco espaço}
    \definition{v.}{empacotar; amontoar; aglomerar | sacudir; empurrar contra; empurrar alguém ou algo para longe com seu corpo com toda a força que puder| pressionar; apertar; expulsar por pressão}
  \end{phonetics}
\end{entry}

\begin{entry}{挥汗如雨}{9,6,6,8}{⼿、⽔、⼥、⾬}
  \begin{phonetics}{挥汗如雨}{hui1han4ru2yu3}
    \definition{s.}{suor derramado}
    \definition{v.}{pingar com suor}
  \end{phonetics}
\end{entry}

\begin{entry}{挺}{9}{⼿}
  \begin{phonetics}{挺}{ting3}[][HSK 2,4]
    \definition{adj.}{rígido; ereto; vertical; reto | distinto. que se destaca; que se sobressai; excepcional}
    \definition{adv.}{muito; bastante}
    \definition{clas.}{para metralhadoras}
    \definition{v.}{sobressair; endireitar-se; saliente ou protuberante | suportar; aguentar; ficar de pé; resistir}
  \end{phonetics}
\end{entry}

\begin{entry}{挺尸}{9,3}{⼿、⼫}
  \begin{phonetics}{挺尸}{ting3shi1}
    \definition{v.}{(coloquial) dormir | (literalmente) ficar deitado duro como um cadáver}
  \end{phonetics}
\end{entry}

\begin{entry}{挺立}{9,5}{⼿、⽴}
  \begin{phonetics}{挺立}{ting3li4}
    \definition{v.}{ficar ereto | ficar de pé}
  \end{phonetics}
\end{entry}

\begin{entry}{挺好}{9,6}{⼿、⼥}
  \begin{phonetics}{挺好}{ting3 hao3}[][HSK 2]
    \definition{adj.}{muito bom}
  \end{phonetics}
\end{entry}

\begin{entry}{挺过}{9,6}{⼿、⾡}
  \begin{phonetics}{挺过}{ting3guo4}
    \definition{s.}{sobreviver}
  \end{phonetics}
\end{entry}

\begin{entry}{挺住}{9,7}{⼿、⼈}
  \begin{phonetics}{挺住}{ting3zhu4}
    \definition{v.}{permanecer firme | manter-se firme (diante da adversidade ou da dor)}
  \end{phonetics}
\end{entry}

\begin{entry}{挺杆}{9,7}{⼿、⽊}
  \begin{phonetics}{挺杆}{ting3gan3}
    \definition{s.}{tucho (peça de máquina)}
  \end{phonetics}
\end{entry}

\begin{entry}{挺身}{9,7}{⼿、⾝}
  \begin{phonetics}{挺身}{ting3shen1}
    \definition{v.}{endireitar as costas}
  \end{phonetics}
\end{entry}

\begin{entry}{挺进}{9,7}{⼿、⾡}
  \begin{phonetics}{挺进}{ting3jin4}
    \definition{s.}{progresso | avanço}
    \definition{v.}{progredir | avançar}
  \end{phonetics}
\end{entry}

\begin{entry}{挺拔}{9,8}{⼿、⼿}
  \begin{phonetics}{挺拔}{ting3ba2}
    \definition{adj.}{alto e reto}
  \end{phonetics}
\end{entry}

\begin{entry}{挺腰}{9,13}{⼿、⾁}
  \begin{phonetics}{挺腰}{ting3yao1}
    \definition{v.}{arquear as costas | endireitar as costas}
  \end{phonetics}
\end{entry}

\begin{entry}{政纲}{9,7}{⽁、⽷}
  \begin{phonetics}{政纲}{zheng4gang1}
    \definition{s.}{programa ou plataforma política}
  \end{phonetics}
\end{entry}

\begin{entry}{政府}{9,8}{⽁、⼴}
  \begin{phonetics}{政府}{zheng4fu3}[][HSK 4]
    \definition[个]{s.}{governo;  órgãos executivos do poder do Estado, ou seja, órgãos administrativos do Estado, como o Conselho de Estado (Governo Popular Central) e os governos populares locais em todos os níveis na China}
  \end{phonetics}
\end{entry}

\begin{entry}{政治}{9,8}{⽁、⽔}
  \begin{phonetics}{政治}{zheng4zhi4}[][HSK 4]
    \definition{s.}{política; assuntos políticos; questões políticas}
  \end{phonetics}
\end{entry}

\begin{entry}{政治局}{9,8,7}{⽁、⽔、⼫}
  \begin{phonetics}{政治局}{zheng4zhi4ju2}
    \definition{s.}{o principal comitê de políticas de um partido comunista}
  \end{phonetics}
\end{entry}

\begin{entry}{故}{9}{⽁}
  \begin{phonetics}{故}{gu4}
    \definition{conj.}{por isso | portanto | então}
  \end{phonetics}
\end{entry}

\begin{entry}{故乡}{9,3}{⽁、⼄}
  \begin{phonetics}{故乡}{gu4xiang1}[][HSK 3]
    \definition[个]{s.}{cidade natal; terra natal}
  \end{phonetics}
\end{entry}

\begin{entry}{故事}{9,8}{⽁、⼅}
  \begin{phonetics}{故事}{gu4shi4}
    \definition{s.}{prática antiga}
  \end{phonetics}
  \begin{phonetics}{故事}{gu4shi5}[][HSK 2]
    \definition{s.}{narrativa | história | conto}
  \end{phonetics}
\end{entry}

\begin{entry}{故宫}{9,9}{⽁、⼧}
  \begin{phonetics}{故宫}{gu4gong1}
    \definition*{s.}{Palácio Imperial | Cidade Proibida}
  \end{phonetics}
\end{entry}

\begin{entry}{故意}{9,13}{⽁、⼼}
  \begin{phonetics}{故意}{gu4yi4}[][HSK 2]
    \definition{adv.}{intencionalmente | deliberadamente | propositalmente}
  \end{phonetics}
\end{entry}

\begin{entry}{既}{9}{⽆}
  \begin{phonetics}{既}{ji4}[][HSK 4]
    \definition*{s.}{sobrenome Ji}
    \definition{adv.}{já}
    \definition{conj.}{desde; como; agora que | assim como; e também; ambos\dots e\dots; usado em conjunto com advérbios como ``且、又、也'' para indicar uma combinação de ambas as situações}
  \seealsoref{且}{qie3}
  \seealsoref{也}{ye3}
  \seealsoref{又}{you4}
  \end{phonetics}
\end{entry}

\begin{entry}{既又}{9,2}{⽆、⼜}
  \begin{phonetics}{既又}{ji4you4}
    \definition{conj.}{desde | como | agora isso | os dois e | assim como}
  \end{phonetics}
\end{entry}

\begin{entry}{既不……又不……}{9,4,2,4}{⽆、⼀、⼜、⼀}
  \begin{phonetics}{既不……又不……}{ji4bu4 you4bu4}
    \definition{conj.}{nem mesmo\dots}
  \end{phonetics}
\end{entry}

\begin{entry}{既然}{9,12}{⽆、⽕}
  \begin{phonetics}{既然}{ji4ran2}[][HSK 4]
    \definition{conj.}{como; desde; agora que; usado na primeira metade de uma frase, muitas vezes repetido na segunda metade pelos advérbios ``就、也、还'' para indicar que a premissa é primeiro declarada e depois inferida}
  \seealsoref{还}{hai2}
  \seealsoref{就}{jiu4}
  \seealsoref{也}{ye3}
  \end{phonetics}
\end{entry}

\begin{entry}{星火}{9,4}{⽇、⽕}
  \begin{phonetics}{星火}{xing1huo3}
    \definition{s.}{trilha de meteoro (usada principalmente em expressões como 急如星火) | faísca}
  \end{phonetics}
\end{entry}

\begin{entry}{星辰}{9,7}{⽇、⾠}
  \begin{phonetics}{星辰}{xing1chen2}
    \definition{s.}{estrelas}
  \end{phonetics}
\end{entry}

\begin{entry}{星表}{9,8}{⽇、⾐}
  \begin{phonetics}{星表}{xing1biao3}
    \definition{s.}{catálogo de estrelas}
  \end{phonetics}
\end{entry}

\begin{entry}{星星}{9,9}{⽇、⽇}
  \begin{phonetics}{星星}{xing1 xing5}[][HSK 2]
    \definition{s.}{estrela}
  \end{phonetics}
\end{entry}

\begin{entry}{星座}{9,10}{⽇、⼴}
  \begin{phonetics}{星座}{xing1zuo4}
    \definition[张]{s.}{signo astrológico | constelação}
  \end{phonetics}
\end{entry}

\begin{entry}{星期}{9,12}{⽇、⽉}
  \begin{phonetics}{星期}{xing1qi1}[][HSK 1]
    \definition[个]{s.}{semana}
  \end{phonetics}
\end{entry}

\begin{entry}{星期一}{9,12,1}{⽇、⽉、⼀}
  \begin{phonetics}{星期一}{xing1qi1yi1}[][HSK 1]
    \definition{s.}{segunda-feira}
  \end{phonetics}
\end{entry}

\begin{entry}{星期二}{9,12,2}{⽇、⽉、⼆}
  \begin{phonetics}{星期二}{xing1qi1'er4}[][HSK 1]
    \definition{s.}{terça-feira}
  \end{phonetics}
\end{entry}

\begin{entry}{星期三}{9,12,3}{⽇、⽉、⼀}
  \begin{phonetics}{星期三}{xing1qi1san1}[][HSK 1]
    \definition{s.}{quarta-feira}
  \end{phonetics}
\end{entry}

\begin{entry}{星期五}{9,12,4}{⽇、⽉、⼆}
  \begin{phonetics}{星期五}{xing1qi1wu3}[][HSK 1]
    \definition{s.}{sexta-feira}
  \end{phonetics}
\end{entry}

\begin{entry}{星期六}{9,12,4}{⽇、⽉、⼋}
  \begin{phonetics}{星期六}{xing1qi1liu4}[][HSK 1]
    \definition{s.}{sábado}
  \end{phonetics}
\end{entry}

\begin{entry}{星期天}{9,12,4}{⽇、⽉、⼤}
  \begin{phonetics}{星期天}{xing1qi1tian1}[][HSK 1]
    \definition{s.}{domingo}
  \seealsoref{星期日}{xing1qi1ri4}
  \end{phonetics}
\end{entry}

\begin{entry}{星期日}{9,12,4}{⽇、⽉、⽇}
  \begin{phonetics}{星期日}{xing1qi1ri4}[][HSK 1]
    \definition{s.}{domingo}
  \seealsoref{星期天}{xing1qi1tian1}
  \end{phonetics}
\end{entry}

\begin{entry}{星期四}{9,12,5}{⽇、⽉、⼞}
  \begin{phonetics}{星期四}{xing1qi1si4}[][HSK 1]
    \definition{s.}{quinta-feira}
  \end{phonetics}
\end{entry}

\begin{entry}{春}{9}{⽇}
  \begin{phonetics}{春}{chun1}
    \definition*{s.}{sobrenome Chun}
    \definition{s.}{primavera | amor | luxúria | vida | vitalidade}
  \end{phonetics}
\end{entry}

\begin{entry}{春天}{9,4}{⽇、⼤}
  \begin{phonetics}{春天}{chun1 tian1}
    \definition[个]{s.}{primavera}
  \end{phonetics}
\end{entry}

\begin{entry}{春节}{9,5}{⽇、⾋}
  \begin{phonetics}{春节}{chun1 jie2}[][HSK 2]
    \definition*{s.}{Festival da Primavera (Ano Novo Chinês)}
  \end{phonetics}
\end{entry}

\begin{entry}{春季}{9,8}{⽇、⼦}
  \begin{phonetics}{春季}{chun1 ji4}[][HSK 4]
    \definition{s.}{primavera; primeiro trimestre do ano, que na China se refere ao período de três meses entre o início da primavera e o início do verão, e também se refere aos três meses do calendário lunar, a saber, o primeiro, o segundo e o terceiro meses}
  \end{phonetics}
\end{entry}

\begin{entry}{昨}{9}{⽇}
  \begin{phonetics}{昨}{zuo2}
    \definition{s.}{ontem}
  \end{phonetics}
\end{entry}

\begin{entry}{昨天}{9,4}{⽇、⼤}
  \begin{phonetics}{昨天}{zuo2tian1}[][HSK 1]
    \definition{adv.}{ontem}
  \end{phonetics}
\end{entry}

\begin{entry}{昨日}{9,4}{⽇、⽇}
  \begin{phonetics}{昨日}{zuo2ri4}
    \definition{adv.}{ontem}
  \end{phonetics}
\end{entry}

\begin{entry}{昨夜}{9,8}{⽇、⼣}
  \begin{phonetics}{昨夜}{zuo2ye4}
    \definition{adv.}{noite passada}
  \end{phonetics}
\end{entry}

\begin{entry}{昨晚}{9,11}{⽇、⽇}
  \begin{phonetics}{昨晚}{zuo2wan3}
    \definition{adv.}{noite passada | ontem à noite}
  \end{phonetics}
\end{entry}

\begin{entry}{是}{9}{⽇}
  \begin{phonetics}{是}{shi4}[][HSK 1]
    \definition{adj.}{correto | certo | verdadeiro | (reconhecimento respeitoso de um comando) muito bem}
    \definition{adv.}{(advérbio para afirmação enfática)}
    \definition{v.}{ser (somente seguido por substantivos)}
  \end{phonetics}
\end{entry}

\begin{entry}{是否}{9,7}{⽇、⼝}
  \begin{phonetics}{是否}{shi4fou3}[][HSK 4]
    \definition{adv.}{se; se ou não}
  \end{phonetics}
\end{entry}

\begin{entry}{是的}{9,8}{⽇、⽩}
  \begin{phonetics}{是的}{shi4de5}
    \definition{adv.}{sim | está certo}
  \end{phonetics}
\end{entry}

\begin{entry}{显}{9}{⽇}
  \begin{phonetics}{显}{xian3}[][HSK 5]
    \definition*{s.}{sobrenome Xian}
    \definition{adj.}{aparente; óbvio; perceptível | ilustre e influente | evidente; óbvio}
    \definition{v.}{mostrar; exibir; manifestar | aparecer; mostrar; revelar}
  \end{phonetics}
\end{entry}

\begin{entry}{显示}{9,5}{⽇、⽰}
  \begin{phonetics}{显示}{xian3shi4}[][HSK 3]
    \definition{v.}{mostrar | exibir}
  \end{phonetics}
\end{entry}

\begin{entry}{显得}{9,11}{⽇、⼻}
  \begin{phonetics}{显得}{xian3de5}[][HSK 3]
    \definition{v.}{parecer; aparecer}
  \end{phonetics}
\end{entry}

\begin{entry}{显著}{9,11}{⽇、⽬}
  \begin{phonetics}{显著}{xian3zhu4}[][HSK 4]
    \definition{adj.}{notável; significativo; notável; extraordinário; muito óbvio; muito claramente demonstrado; muito facilmente visto ou sentido}
  \end{phonetics}
\end{entry}

\begin{entry}{显然}{9,12}{⽇、⽕}
  \begin{phonetics}{显然}{xian3ran2}[][HSK 3]
    \definition{adj.}{claro; evidente; óbvio}
    \definition{adv.}{claramente; evidentemente; obviamente}
  \end{phonetics}
\end{entry}

\begin{entry}{枯木}{9,4}{⽊、⽊}
  \begin{phonetics}{枯木}{ku1mu4}
    \definition{s.}{árvore morta | madeira morta}
  \end{phonetics}
\end{entry}

\begin{entry}{架}{9}{⽊}
  \begin{phonetics}{架}{jia4}[][HSK 3]
    \definition{clas.}{para coisas com pilares ou componentes mecânicos | quadrado (usado para montanhas)}
    \definition{s.}{quadro; prateleira; suporte | briga; discussão}
    \definition{v.}{colocar para cima; erigir | afastar; resistir | suportar; ajudar | sequestrar; levar alguém embora à força}
  \end{phonetics}
\end{entry}

\begin{entry}{架式}{9,6}{⽊、⼷}
  \begin{phonetics}{架式}{jia4shi5}
    \variantof{架势}
  \end{phonetics}
\end{entry}

\begin{entry}{架势}{9,8}{⽊、⼒}
  \begin{phonetics}{架势}{jia4shi5}
    \definition{s.}{postura | atitude | posição (sobre um assunto, etc.)}
  \end{phonetics}
\end{entry}

\begin{entry}{柏树}{9,9}{⽊、⽊}
  \begin{phonetics}{柏树}{bai3shu4}
    \definition{s.}{cipreste}
  \end{phonetics}
\end{entry}

\begin{entry}{某}{9}{⽊}
  \begin{phonetics}{某}{mou3}[][HSK 3]
    \definition{pron.}{um certo alguém ou coisa; algum | usado para substituir seu próprio nome}
  \end{phonetics}
\end{entry}

\begin{entry}{染}{9}{⽊}
  \begin{phonetics}{染}{ran3}[][HSK 5]
    \definition*{s.}{sobrenome Ran}
    \definition{s.}{soja fermentada e temperada em forma de pasta}
    \definition{v.}{tingir; pintar | pegar (uma doença); cair em (um mau hábito, etc.) | sujar; contaminar | pegar (contrair) (uma doença) | adquirir (um mau hábito, etc.); contaminar}
  \end{phonetics}
\end{entry}

\begin{entry}{柔软}{9,8}{⽊、⾞}
  \begin{phonetics}{柔软}{rou2ruan3}
    \definition{adj.}{macio | suave}
  \end{phonetics}
\end{entry}

\begin{entry}{柠檬}{9,17}{⽊、⽊}
  \begin{phonetics}{柠檬}{ning2meng2}
    \definition{s.}{limão}
  \end{phonetics}
\end{entry}

\begin{entry}{查}{9}{⽊}
  \begin{phonetics}{查}{cha2}[][HSK 2]
    \definition{v.}{verificar | examinar | investigar |consultar}
  \end{phonetics}
  \begin{phonetics}{查}{zha1}
    \definition*{s.}{sobrenome Zha}
    \definition{s.}{espinheiro}
  \end{phonetics}
\end{entry}

\begin{entry}{查询}{9,8}{⽊、⾔}
  \begin{phonetics}{查询}{cha2 xun2}[][HSK 5]
    \definition{v.}{indagar; inquirir; perguntar sobre}
  \end{phonetics}
\end{entry}

\begin{entry}{柬埔寨}{9,10,14}{⽊、⼟、⼧}
  \begin{phonetics}{柬埔寨}{jian3pu3zhai4}
    \definition*{s.}{Camboja}
  \end{phonetics}
\end{entry}

\begin{entry}{柳}{9}{⽊}
  \begin{phonetics}{柳}{liu3}
    \definition*{s.}{sobrenome Liu}
    \definition{s.}{salgueiro}
  \end{phonetics}
\end{entry}

\begin{entry}{柳橙汁}{9,16,5}{⽊、⽊、⽔}
  \begin{phonetics}{柳橙汁}{liu3cheng2zhi1}
    \definition[瓶,杯,罐,盒]{s.}{suco de laranja}
  \seealsoref{橙汁}{cheng2zhi1}
  \seealsoref{橘子汁}{ju2zi5zhi1}
  \end{phonetics}
\end{entry}

\begin{entry}{标志}{9,7}{⽊、⼼}
  \begin{phonetics}{标志}{biao1zhi4}[][HSK 4]
    \definition[个,种]{s.}{sinal; marca; logotipo; símbolo; emblema; marcações que caracterizam um objeto para facilitar a identificação}
    \definition{v.}{marcar; indicar; simbolizar; identificar}
  \end{phonetics}
\end{entry}

\begin{entry}{标准}{9,10}{⽊、⼎}
  \begin{phonetics}{标准}{biao1zhun3}[][HSK 3]
    \definition{adj.}{criterioso | padronizado | normatizado}
    \definition[个]{s.}{critério | padrão (oficial) | norma}
  \end{phonetics}
\end{entry}

\begin{entry}{标题}{9,15}{⽊、⾴}
  \begin{phonetics}{标题}{biao1ti2}[][HSK 3]
    \definition[个,条,篇]{s.}{título | manchete | cabeçalho}
  \end{phonetics}
\end{entry}

\begin{entry}{树}{9}{⽊}
  \begin{phonetics}{树}{shu4}[][HSK 1]
    \definition[棵]{s.}{árvore}
    \definition{v.}{cultivar}
  \end{phonetics}
\end{entry}

\begin{entry}{树木}{9,4}{⽊、⽊}
  \begin{phonetics}{树木}{shu4mu4}
    \definition{s.}{árvore}
  \end{phonetics}
\end{entry}

\begin{entry}{树叶}{9,5}{⽊、⼝}
  \begin{phonetics}{树叶}{shu4ye4}[][HSK 4]
    \definition[片,枚,堆]{s.}{folha; folhagem;}
  \end{phonetics}
\end{entry}

\begin{entry}{树林}{9,8}{⽊、⽊}
  \begin{phonetics}{树林}{shu4 lin2}[][HSK 4]
    \definition{s.}{bosque; muitas árvores que crescem em fragmentos, menores que as florestas}
  \end{phonetics}
\end{entry}

\begin{entry}{树莓}{9,10}{⽊、⾋}
  \begin{phonetics}{树莓}{shu4mei2}
    \definition{s.}{framboesa}
  \end{phonetics}
\end{entry}

\begin{entry}{歪}{9}{⽌}
  \begin{phonetics}{歪}{wai1}
    \definition{adj.}{torto | tortuoso | nocivo}
  \end{phonetics}
\end{entry}

\begin{entry}{歪果仁}{9,8,4}{⽌、⽊、⼈}
  \begin{phonetics}{歪果仁}{wai1guo3ren2}
    \definition{s.}{gíria na \emph{Internet} para 外国人}
    \seeref{外国人}{wai4guo2ren2}
  \end{phonetics}
\end{entry}

\begin{entry}{残疾人}{9,10,2}{⽍、⽧、⼈}
  \begin{phonetics}{残疾人}{can2ji2ren2}
    \definition{s.}{pessoa com deficiência}
  \end{phonetics}
\end{entry}

\begin{entry}{残酷}{9,14}{⽍、⾣}
  \begin{phonetics}{残酷}{can2ku4}
    \definition{adj.}{cruel}
    \definition{s.}{crueldade}
  \end{phonetics}
\end{entry}

\begin{entry}{段}{9}{⽎}
  \begin{phonetics}{段}{duan4}[][HSK 2]
    \definition*{s.}{sobrenome Duan}
    \definition{clas.}{para histórias, períodos de tempo, desenvolvimento de um tópico, etc.}
    \definition{s.}{parágrafo | seção | segmento | estágio (de um processo)}
  \end{phonetics}
\end{entry}

\begin{entry}{毒}{9}{⽏}
  \begin{phonetics}{毒}{du2}[][HSK 5]
    \definition*{s.}{sobrenome Du}
    \definition{adj.}{veneno; toxina; propriedade ou substância prejudicial aos organismos vivos | droga; narcóticos}
    \definition{adj.}{venenoso; tóxico; envenenado}
    \definition{v.}{matar com veneno; envenenar}
  \end{phonetics}
\end{entry}

\begin{entry}{毒杀}{9,6}{⽏、⽊}
  \begin{phonetics}{毒杀}{du2sha1}
    \definition{v.}{matar por envenenamento}
  \end{phonetics}
\end{entry}

\begin{entry}{毒物}{9,8}{⽏、⽜}
  \begin{phonetics}{毒物}{du2wu4}
    \definition{s.}{substância venenosa | toxina}
  \end{phonetics}
\end{entry}

\begin{entry}{毒害}{9,10}{⽏、⼧}
  \begin{phonetics}{毒害}{du2hai4}
    \definition{s.}{envenenamento}
    \definition{v.}{envenenar (prejudicar com uma substância tóxica) | envenenar (as mentes das pessoas)}
  \end{phonetics}
\end{entry}

\begin{entry}{毒蛇}{9,11}{⽏、⾍}
  \begin{phonetics}{毒蛇}{du2she2}
    \definition{s.}{víbora | cobra venenosa}
  \end{phonetics}
\end{entry}

\begin{entry}{泉}{9}{⽔}
  \begin{phonetics}{泉}{quan2}[][HSK 5]
    \definition*{s.}{sobrenome Quan}
    \definition[股,眼,汪]{s.}{fonte (de água mineral) | a nascente de um rio | termo antigo para moeda}
  \end{phonetics}
\end{entry}

\begin{entry}{洋葱}{9,12}{⽔、⾋}
  \begin{phonetics}{洋葱}{yang2cong1}
    \definition{s.}{cebola}
  \end{phonetics}
\end{entry}

\begin{entry}{洒}{9}{⽔}
  \begin{phonetics}{洒}{sa3}[][HSK 5]
    \definition{adj.}{natural e sem restrições; confortável (sem restrições)}
    \definition{v.}{derramar; espalhar; borrifar; salpicar; fazer com que (água ou outra coisa) caia de forma dispersa | derramar; cair de forma dispersa}
  \end{phonetics}
\end{entry}

\begin{entry}{洒水}{9,4}{⽔、⽔}
  \begin{phonetics}{洒水}{sa3shui3}
    \definition{v.}{borrifar}
  \end{phonetics}
\end{entry}

\begin{entry}{洗}{9}{⽔}
  \begin{phonetics}{洗}{xi3}[][HSK 1]
    \definition{v.}{lavar | revelar (fotos) | tomar banho}
  \end{phonetics}
\end{entry}

\begin{entry}{洗手}{9,4}{⽔、⼿}
  \begin{phonetics}{洗手}{xi3shou3}
    \definition{v.}{ir ao banheiro | lavar as mãos}
  \end{phonetics}
\end{entry}

\begin{entry}{洗手不干}{9,4,4,3}{⽔、⼿、⼀、⼲}
  \begin{phonetics}{洗手不干}{xi3shou3bu2gan4}
    \definition{v.}{parar totalmente de fazer algo}
  \end{phonetics}
\end{entry}

\begin{entry}{洗手池}{9,4,6}{⽔、⼿、⽔}
  \begin{phonetics}{洗手池}{xi3shou3chi2}
    \definition{s.}{pia de banheiro | lavatório}
  \seealsoref{洗手盆}{xi3shou3pen2}
  \end{phonetics}
\end{entry}

\begin{entry}{洗手间}{9,4,7}{⽔、⼿、⾨}
  \begin{phonetics}{洗手间}{xi3shou3jian1}[][HSK 1]
    \definition{s.}{sanitário | toilette | banheiro}
  \end{phonetics}
\end{entry}

\begin{entry}{洗手乳}{9,4,8}{⽔、⼿、⼄}
  \begin{phonetics}{洗手乳}{xi3shou3ru3}
    \definition{s.}{sabonete líquido para lavar as mãos}
  \seealsoref{洗手液}{xi3shou3ye4}
  \end{phonetics}
\end{entry}

\begin{entry}{洗手盆}{9,4,9}{⽔、⼿、⽫}
  \begin{phonetics}{洗手盆}{xi3shou3pen2}
    \definition{s.}{pia de banheiro | lavatório}
  \seealsoref{洗手池}{xi3shou3chi2}
  \end{phonetics}
\end{entry}

\begin{entry}{洗手液}{9,4,11}{⽔、⼿、⽔}
  \begin{phonetics}{洗手液}{xi3shou3ye4}
    \definition{s.}{sabonete líquido para lavar as mãos}
  \seealsoref{洗手乳}{xi3shou3ru3}
  \end{phonetics}
\end{entry}

\begin{entry}{洗礼}{9,5}{⽔、⽰}
  \begin{phonetics}{洗礼}{xi3li3}
    \definition{s.}{batismo}
    \definition{v.}{batizar}
  \end{phonetics}
\end{entry}

\begin{entry}{洗衣机}{9,6,6}{⽔、⾐、⽊}
  \begin{phonetics}{洗衣机}{xi3 yi1 ji1}[][HSK 2]
    \definition[台]{s.}{máquina de lavar roupa}
  \end{phonetics}
\end{entry}

\begin{entry}{洗劫}{9,7}{⽔、⼒}
  \begin{phonetics}{洗劫}{xi3jie2}
    \definition{v.}{saquear | pilhar | roubar}
  \end{phonetics}
\end{entry}

\begin{entry}{洗净}{9,8}{⽔、⼎}
  \begin{phonetics}{洗净}{xi3jing4}
    \definition{v.}{lavar (limpeza)}
  \end{phonetics}
\end{entry}

\begin{entry}{洗胃}{9,9}{⽔、⾁}
  \begin{phonetics}{洗胃}{xi3wei4}
    \definition{s.}{(medicina) lavagem gástrica}
    \definition{v.}{ter o estômago lavado}
  \end{phonetics}
\end{entry}

\begin{entry}{洗涤}{9,10}{⽔、⽔}
  \begin{phonetics}{洗涤}{xi3di2}
    \definition{s.}{enxágue | lava}
    \definition{v.}{enxaguar | lavar}
  \end{phonetics}
\end{entry}

\begin{entry}{洗涤间}{9,10,7}{⽔、⽔、⾨}
  \begin{phonetics}{洗涤间}{xi3di2jian1}
    \definition{s.}{lavanderia}
  \end{phonetics}
\end{entry}

\begin{entry}{洗脱}{9,11}{⽔、⾁}
  \begin{phonetics}{洗脱}{xi3tuo1}
    \definition{v.}{limpar | purgar | lavar}
  \end{phonetics}
\end{entry}

\begin{entry}{洗碗}{9,13}{⽔、⽯}
  \begin{phonetics}{洗碗}{xi3wan3}
    \definition{v.}{lavar pratos}
  \end{phonetics}
\end{entry}

\begin{entry}{洗澡}{9,16}{⽔、⽔}
  \begin{phonetics}{洗澡}{xi3zao3}[][HSK 2]
    \definition{v.+compl.}{tomar banho | duchar-se | lavar-se}
  \end{phonetics}
\end{entry}

\begin{entry}{洗澡间}{9,16,7}{⽔、⽔、⾨}
  \begin{phonetics}{洗澡间}{xi3zao3jian1}
    \definition[间]{s.}{banheiro}
  \end{phonetics}
\end{entry}

\begin{entry}{洞}{9}{⽔}
  \begin{phonetics}{洞}{dong4}[][HSK 5]
    \definition{adj.}{profundo; minucioso; claro; completo; abrangente}
    \definition{s.}{buraco; cavidade; orifício; furo; parte penetrante ou profundamente recuada de um objeto; uma caverna}
  \end{phonetics}
\end{entry}

\begin{entry}{洞穴}{9,5}{⽔、⽳}
  \begin{phonetics}{洞穴}{dong4xue2}
    \definition{s.}{caverna}
  \end{phonetics}
\end{entry}

\begin{entry}{洪水}{9,4}{⽔、⽔}
  \begin{phonetics}{洪水}{hong2shui3}
    \definition{s.}{enchente | inundação | dilúvio}
  \end{phonetics}
\end{entry}

\begin{entry}{洲}{9}{⽔}
  \begin{phonetics}{洲}{zhou1}
    \definition{s.}{continente | ilha em um rio}
  \end{phonetics}
\end{entry}

\begin{entry}{活}{9}{⽔}
  \begin{phonetics}{活}{huo2}[][HSK 3]
    \definition{adj.}{vivo; vivendo | vívido; animado; ativo | móvel; em movimento}
    \definition{adv.}{exatamente; simplesmente}
    \definition{s.}{trabalho | produto}
    \definition{v.}{viver | salvar (a vida de uma pessoa)}
  \end{phonetics}
\end{entry}

\begin{entry}{活力}{9,2}{⽔、⼒}
  \begin{phonetics}{活力}{huo2li4}[][HSK 5]
    \definition{s.}{vigor; vitalidade; energia; muito forte, geralmente usado para descrever pessoas, cidades, empresas, economias, etc.}
  \end{phonetics}
\end{entry}

\begin{entry}{活动}{9,6}{⽔、⼒}
  \begin{phonetics}{活动}{huo2dong4}[][HSK 2]
    \definition[项,个]{s.}{atividade | evento | campanha}
    \definition{v.}{exercer | operar}
  \end{phonetics}
\end{entry}

\begin{entry}{活泼}{9,8}{⽔、⽔}
  \begin{phonetics}{活泼}{huo2po1}[][HSK 5]
    \definition{adj.}{vívido; ativo; animado; brilhante; vivaz; cheio de vida | reativo; (química) significa que a substância é ativa e reage facilmente com outras substâncias}
  \end{phonetics}
\end{entry}

\begin{entry}{活着}{9,11}{⽔、⽬}
  \begin{phonetics}{活着}{huo2zhe5}
    \definition{adj.}{vivo}
  \end{phonetics}
\end{entry}

\begin{entry}{活路}{9,13}{⽔、⾜}
  \begin{phonetics}{活路}{huo2lu4}
    \definition{s.}{maneira de sobreviver | meio de subsistência}
  \end{phonetics}
  \begin{phonetics}{活路}{huo2lu5}
    \definition{s.}{labor | trabalho físico}
  \end{phonetics}
\end{entry}

\begin{entry}{派}{9}{⽔}
  \begin{phonetics}{派}{pai4}[][HSK 3]
    \definition{adj.}{elegante; bonito}
    \definition{clas.}{para grupos, escolas de pensamento ou arte, etc. | para um discursos, atmosferas, cenas, etc.}
    \definition{s.}{panelinha; grupo exclusivo; facção | torta | estilo | afluente; braço de rio}
    \definition{v.}{enviar; despachar | alocar; repartir; distribuir}
  \end{phonetics}
\end{entry}

\begin{entry}{测}{9}{⽔}
  \begin{phonetics}{测}{ce4}[][HSK 4]
    \definition{v.}{pesquisar; sondar; medir | conjecturar; inferir}
  \end{phonetics}
\end{entry}

\begin{entry}{测试}{9,8}{⽔、⾔}
  \begin{phonetics}{测试}{ce4 shi4}[][HSK 4]
    \definition[个]{s.}{exame; teste; medição do conhecimento humano, das habilidades ou do funcionamento de máquinas, ferramentas ou instrumentos}
    \definition{v.}{examinar | testar, medição do desempenho e da precisão de máquinas, instrumentos, aparelhos, etc.}
  \end{phonetics}
\end{entry}

\begin{entry}{测量}{9,12}{⽔、⾥}
  \begin{phonetics}{测量}{ce4liang2}[][HSK 4]
    \definition{v.}{aferir; pesquisar; medir; determinar valores relevantes para espaço, tempo, temperatura, velocidade, função, etc.}
  \end{phonetics}
\end{entry}

\begin{entry}{浓}{9}{⽔}
  \begin{phonetics}{浓}{nong2}[][HSK 4]
    \definition{adj.}{denso; espesso; concentrado; um líquido ou gás que contém mais de um determinado ingrediente | grande; forte; profundo (de grau ou extensão) | profundo; (algumas cores) escuro}
  \end{phonetics}
\end{entry}

\begin{entry}{点}{9}{⽕}
  \begin{phonetics}{点}{dian3}[][HSK 1]
    \definition{clas.}{para itens | hora cheia}
    \definition{s.}{ponto | gota | mancha | horas | ponto (no espaço ou no tempo) | traço de ponto em caracteres chineses}
    \definition{v.}{desenhar um ponto | verificar uma lista | escolher | pedir (comida em um restaurante) | tocar brevemente | sugerir | acender | derramar um líquido gota a gota}
  \end{phonetics}
\end{entry}

\begin{entry}{点火}{9,4}{⽕、⽕}
  \begin{phonetics}{点火}{dian3huo3}
    \definition{s.}{ignição}
    \definition{v.}{inflamar | acender um fogo | agitar | dar partida em um motor | (figurativo) provocar problemas}
  \end{phonetics}
\end{entry}

\begin{entry}{点头}{9,5}{⽕、⼤}
  \begin{phonetics}{点头}{dian3 tou2}[][HSK 2]
    \definition{v.}{acenar com a cabeça}
  \end{phonetics}
\end{entry}

\begin{entry}{点名}{9,6}{⽕、⼝}
  \begin{phonetics}{点名}{dian3 ming2}[][HSK 4]
    \definition{v.}{fazer a lista de chamada; manter o controle da presença de alguém; chamar nomes para controle de presença | mencionar alguém pelo nome}
  \end{phonetics}
\end{entry}

\begin{entry}{点燃}{9,16}{⽕、⽕}
  \begin{phonetics}{点燃}{dian3 ran2}[][HSK 5]
    \definition{v.}{acender; inflamar; acender uma fogueira, para iluminar}
  \end{phonetics}
\end{entry}

\begin{entry}{烂}{9}{⽕}
  \begin{phonetics}{烂}{lan4}[][HSK 5]
    \definition{adj.}{macio; pastoso; amassado | podre; deteriorado | quebrado; esfarrapado; gasto | desorganizado; indigno}
    \definition{adv.}{totalmente; extremamente; completamente; expressa um grau muito profundo}
    \definition{v.}{apodrecer; infeccionar; decompor-se}
  \end{phonetics}
\end{entry}

\begin{entry}{独}{9}{⽝}
  \begin{phonetics}{独}{du2}
    \definition{adj.}{sozinho | solitário | solteiro}
    \definition{adv.}{apenas}
  \end{phonetics}
\end{entry}

\begin{entry}{独立}{9,5}{⽝、⽴}
  \begin{phonetics}{独立}{du2li4}[][HSK 4]
    \definition{adj.}{independente; por conta própria | separado; respectivo; descreve algo que é separado e não está em contato com outra coisa}
    \definition{prep.}{independente de; separado de; não mais anexado à unidade original, mas uma unidade separada}
    \definition{v.}{ficar sozinho | alcançar a independência; tornar-se um país independente; liberdade de um Estado, regime ou organização contra interferência, controle e dominação por forças externas}
  \end{phonetics}
\end{entry}

\begin{entry}{独自}{9,6}{⽝、⾃}
  \begin{phonetics}{独自}{du2 zi4}[][HSK 4]
    \definition{adj.}{sozinho; por si mesmo; por conta própria}
  \end{phonetics}
\end{entry}

\begin{entry}{独特}{9,10}{⽝、⽜}
  \begin{phonetics}{独特}{du2te4}[][HSK 4]
    \definition{adj.}{único; distinto; original; especial}
  \end{phonetics}
\end{entry}

\begin{entry}{狭}{9}{⽝}
  \begin{phonetics}{狭}{xia2}
    \definition{adj.}{estreito}
  \end{phonetics}
\end{entry}

\begin{entry}{玻璃}{9,14}{⽟、⽟}
  \begin{phonetics}{玻璃}{bo1li5}[][HSK 5]
    \definition[张,块]{s.}{vidro; corpo duro, quebradiço e transparente, geralmente feito de areia, calcário, carbonato de sódio, etc. | \emph{nylon}; plástico; refere-se a determinados plásticos que se assemelham ao vidro.}
  \end{phonetics}
\end{entry}

\begin{entry}{珍贵}{9,9}{⽟、⾙}
  \begin{phonetics}{珍贵}{zhen1gui4}[][HSK 5]
    \definition{adj.}{raro; valioso; precioso; de grande valor; profundo significado}
  \end{phonetics}
\end{entry}

\begin{entry}{珍珠}{9,10}{⽟、⽟}
  \begin{phonetics}{珍珠}{zhen1zhu1}[][HSK 5]
    \definition[颗,串]{s.}{pérola; grânulos redondos produzidos nas conchas de certos animais aquáticos, de cor branca, rosa, etc., bonitos e brilhantes, frequentemente usados como adornos}
  \end{phonetics}
\end{entry}

\begin{entry}{珍惜}{9,11}{⽟、⼼}
  \begin{phonetics}{珍惜}{zhen1xi1}[][HSK 5]
    \definition{v.}{valorizar; estimar; valorizar e evitar o desperdício}
  \end{phonetics}
\end{entry}

\begin{entry}{甚而}{9,6}{⽢、⽽}
  \begin{phonetics}{甚而}{shen4'er2}
    \definition{conj.}{(ir) tão longe quanto | tanto que}
  \end{phonetics}
\end{entry}

\begin{entry}{甚至}{9,6}{⽢、⾄}
  \begin{phonetics}{甚至}{shen4zhi4}[][HSK 4]
    \definition{conj.}{e até mesmo; nem mesmo; para apresentar uma situação típica e especial, para enfatizar a profundidade e a seriedade de uma situação}
  \end{phonetics}
\end{entry}

\begin{entry}{甚或}{9,8}{⽢、⼽}
  \begin{phonetics}{甚或}{shen4huo4}
    \definition{conj.}{(ir) tão longe quanto | tanto que}
  \end{phonetics}
\end{entry}

\begin{entry}{甭}{9}{⽤}
  \begin{phonetics}{甭}{beng2}
    \definition{v.}{contração de 不用 | não precisar}
    \seeref{不用}{bu2 yong4}
  \end{phonetics}
\end{entry}

\begin{entry}{界碑}{9,13}{⽥、⽯}
  \begin{phonetics}{界碑}{jie4bei1}
    \definition{s.}{marco de fronteira}
  \end{phonetics}
\end{entry}

\begin{entry}{疯}{9}{⽧}
  \begin{phonetics}{疯}{feng1}[][HSK 5]
    \definition{adj.}{louco; doido; maluco; insano | termo usado para se referir ao crescimento vigoroso de plantas e culturas que não dão frutos}
  \end{phonetics}
\end{entry}

\begin{entry}{疯狂}{9,7}{⽧、⽝}
  \begin{phonetics}{疯狂}{feng1kuang2}[][HSK 5]
    \definition{adj.}{louco; insano; frenético; desenfreado}
  \end{phonetics}
\end{entry}

\begin{entry}{皆}{9}{⽩}
  \begin{phonetics}{皆}{jie1}
    \definition{adv.}{todos | em todos os casos}
  \end{phonetics}
\end{entry}

\begin{entry}{皇帝}{9,9}{⽩、⼱}
  \begin{phonetics}{皇帝}{huang2di4}
    \definition[个]{s.}{imperador}
  \end{phonetics}
\end{entry}

\begin{entry}{盆}{9}{⽫}
  \begin{phonetics}{盆}{pen2}[][HSK 5]
    \definition*{s.}{sobrenome Pen}
    \definition[个]{s.}{bacia; banheira; panela; utensílios para guardar ou lavar coisas}
  \end{phonetics}
\end{entry}

\begin{entry}{盆友}{9,4}{⽫、⼜}
  \begin{phonetics}{盆友}{pen2you3}
    \definition{s.}{(gíria na \emph{Internet}) amigo (trocadilho com 朋友)}
    \seeref{朋友}{peng2you5}
  \end{phonetics}
\end{entry}

\begin{entry}{相互}{9,4}{⽬、⼆}
  \begin{phonetics}{相互}{xiang1 hu4}[][HSK 3]
    \definition{adj.}{mútuo; recíproco}
    \definition{adv.}{mutuamente; um ao outro}
  \end{phonetics}
\end{entry}

\begin{entry}{相反}{9,4}{⽬、⼜}
  \begin{phonetics}{相反}{xiang1fan3}[][HSK 4]
    \definition{adj.}{oposto; contrário; dois aspectos das coisas são contraditórios e mutuamente exclusivos}
    \definition{conj.}{pelo contrário; usado no início ou no meio de uma frase para indicar uma contradição de significado com o que foi dito anteriormente.}
  \end{phonetics}
\end{entry}

\begin{entry}{相比}{9,4}{⽬、⽐}
  \begin{phonetics}{相比}{xiang1 bi3}[][HSK 3]
    \definition{v.}{combinar; comparar com |comparar uma coisa com outra, usar uma coisa como padrão para ver as características de outra coisa ou para obter um ponto de vista}
  \end{phonetics}
\end{entry}

\begin{entry}{相片}{9,4}{⽬、⽚}
  \begin{phonetics}{相片}{xiang4 pian4}[][HSK 4]
    \definition[张]{s.}{foto; fotografia; uma imagem de uma pessoa ou objeto feita pela exposição de papel fotográfico a um negativo fotográfico e, em seguida, revelando e fixando a imagem.}
  \end{phonetics}
\end{entry}

\begin{entry}{相处}{9,5}{⽬、⼡}
  \begin{phonetics}{相处}{xiang1chu3}[][HSK 4]
    \definition{v.}{dar-se bem; viver juntos; dar-se bem (uns com os outros); viver uns com os outros; entrar em contato uns com os outros, tratar uns aos outros}
  \end{phonetics}
\end{entry}

\begin{entry}{相似}{9,6}{⽬、⼈}
  \begin{phonetics}{相似}{xiang1si4}[][HSK 3]
    \definition{v.}{assemelhar-se; ser semelhante; ser igual}
  \end{phonetics}
\end{entry}

\begin{entry}{相关}{9,6}{⽬、⼋}
  \begin{phonetics}{相关}{xiang1guan1}[][HSK 3]
    \definition{v.}{mutuamente relacionados; inter-relacionados}
  \end{phonetics}
\end{entry}

\begin{entry}{相同}{9,6}{⽬、⼝}
  \begin{phonetics}{相同}{xiang1tong2}[][HSK 2]
    \definition{adj.}{igual | idêntico | o mesmo}
  \end{phonetics}
\end{entry}

\begin{entry}{相当}{9,6}{⽬、⼹}
  \begin{phonetics}{相当}{xiang1dang1}[][HSK 3]
    \definition{adj.}{adequado; ajustado; apropriado}
    \definition{adv.}{bastante; razoavelmente; consideravelmente}
    \definition{v.}{combinar; equilibrar; corresponder a; ser aproximadamente igual a; ser compatível com}
  \end{phonetics}
\end{entry}

\begin{entry}{相机}{9,6}{⽬、⽊}
  \begin{phonetics}{相机}{xiang4 ji1}[][HSK 2]
    \definition[台,个]{s.}{câmera | máquina fotográfica}
    \definition{v.}{ficar atento a uma oportunidade}
  \end{phonetics}
\end{entry}

\begin{entry}{相声}{9,7}{⽬、⼠}
  \begin{phonetics}{相声}{xiang4sheng5}[][HSK 5]
    \definition[个,段]{s.}{conversa cruzada; diálogo cômico; forma de performance humorística, em que os atores usam piadas, canções e imitações para satirizar e elogiar}
  \end{phonetics}
\end{entry}

\begin{entry}{相应}{9,7}{⽬、⼴}
  \begin{phonetics}{相应}{xiang1ying4}[][HSK 5]
    \definition{v.}{corresponder}
  \end{phonetics}
\end{entry}

\begin{entry}{相宜}{9,8}{⽬、⼧}
  \begin{phonetics}{相宜}{xiang1yi2}
    \definition{adj.}{adequado | apropriado}
    \definition{v.}{ser adequado ou apropriado}
  \end{phonetics}
\end{entry}

\begin{entry}{相亲}{9,9}{⽬、⼇}
  \begin{phonetics}{相亲}{xiang1qin1}
    \definition{s.}{encontro às cegas | entrevista arranjada para avaliar a proposta de um parceiro de casamento | apegar-se profundamente um ao outro}
  \end{phonetics}
\end{entry}

\begin{entry}{相信}{9,9}{⽬、⼈}
  \begin{phonetics}{相信}{xiang1xin4}[][HSK 2]
    \definition{v.}{acreditar | estar convencido | aceitar como verdadeiro}
  \end{phonetics}
\end{entry}

\begin{entry}{相思病}{9,9,10}{⽬、⼼、⽧}
  \begin{phonetics}{相思病}{xiang1si1bing4}
    \definition{s.}{saudade de amor}
  \end{phonetics}
\end{entry}

\begin{entry}{相等}{9,12}{⽬、⽵}
  \begin{phonetics}{相等}{xiang1deng3}[][HSK 5]
    \definition{v.}{ser igual a; possuir a mesma quantidade, peso, tamanho e grau}
  \end{phonetics}
\end{entry}

\begin{entry}{相遇}{9,12}{⽬、⾡}
  \begin{phonetics}{相遇}{xiang1yu4}
    \definition{v.}{encontrar (reunião, encontro, etc.)}
  \end{phonetics}
\end{entry}

\begin{entry}{相聚}{9,14}{⽬、⽿}
  \begin{phonetics}{相聚}{xiang1ju4}
    \definition{v.}{reunir-se | montar}
  \end{phonetics}
\end{entry}

\begin{entry}{省}{9}{⽬}
  \begin{phonetics}{省}{sheng3}[][HSK 2]
    \definition{s.}{província | capital provincial}
    \definition{v.}{economizar | guardar | ser frugal | omitir | excluir | deixar de fora}
  \end{phonetics}
  \begin{phonetics}{省}{xing3}
    \definition[个]{s.}{governadoria}
    \definition{v.}{examinar minuciosamente | refletir (sobre a conduta de alguém) | realizar | fazer uma visita (aos pais ou idosos)}
  \end{phonetics}
\end{entry}

\begin{entry}{省力}{9,2}{⽬、⼒}
  \begin{phonetics}{省力}{sheng3li4}
    \definition{v.}{economizar esforço ou trabalho}
  \end{phonetics}
\end{entry}

\begin{entry}{省心}{9,4}{⽬、⼼}
  \begin{phonetics}{省心}{sheng3xin1}
    \definition{adj.}{despreocupado}
    \definition{v.}{ser poupado de preocupações | despreocupar-se}
  \end{phonetics}
\end{entry}

\begin{entry}{省长}{9,4}{⽬、⾧}
  \begin{phonetics}{省长}{sheng3zhang3}
    \definition*{s.}{Governador | governador de uma província}
  \end{phonetics}
\end{entry}

\begin{entry}{省会}{9,6}{⽬、⼈}
  \begin{phonetics}{省会}{sheng3hui4}
    \definition{s.}{capital da província}
  \end{phonetics}
\end{entry}

\begin{entry}{省却}{9,7}{⽬、⼙}
  \begin{phonetics}{省却}{sheng3que4}
    \definition{v.}{livrar-se (para economizar espaço) | salvar}
  \end{phonetics}
\end{entry}

\begin{entry}{省俭}{9,9}{⽬、⼈}
  \begin{phonetics}{省俭}{sheng3jian3}
    \definition{s.}{econômico | frugal}
    \definition{v.}{economizar}
  \end{phonetics}
\end{entry}

\begin{entry}{省城}{9,9}{⽬、⼟}
  \begin{phonetics}{省城}{sheng3cheng2}
    \definition{s.}{capital da província}
  \end{phonetics}
\end{entry}

\begin{entry}{省悟}{9,10}{⽬、⼼}
  \begin{phonetics}{省悟}{xing3wu4}
    \definition{v.}{voltar a si | constatar | ver a verdade | acordar para a realidade}
  \end{phonetics}
\end{entry}

\begin{entry}{省钱}{9,10}{⽬、⾦}
  \begin{phonetics}{省钱}{sheng3qian2}
    \definition{v.}{economizar dinheiro}
  \end{phonetics}
\end{entry}

\begin{entry}{眉}{9}{⽬}
  \begin{phonetics}{眉}{mei2}
    \definition{s.}{sobrancelha | margem superior}
  \end{phonetics}
\end{entry}

\begin{entry}{眉毛}{9,4}{⽬、⽑}
  \begin{phonetics}{眉毛}{mei2mao5}
    \definition[根]{s.}{sobrancelha}
  \end{phonetics}
\end{entry}

\begin{entry}{眉头}{9,5}{⽬、⼤}
  \begin{phonetics}{眉头}{mei2tou2}
    \definition{s.}{testa}
  \end{phonetics}
\end{entry}

\begin{entry}{看}{9}{⽬}
  \begin{phonetics}{看}{kan1}
    \definition{v.}{cuidar | vigiar}
  \end{phonetics}
  \begin{phonetics}{看}{kan4}[][HSK 1]
    \definition{interj.}{Cuidado! (para um perigo)}
    \definition{part.}{(depois de um verbo) tentar}
    \definition{v.}{olhar | ver | assistir | ler | visitar (pessoas)}
  \end{phonetics}
\end{entry}

\begin{entry}{看上去}{9,3,5}{⽬、⼀、⼛}
  \begin{phonetics}{看上去}{kan4 shang4 qu4}[][HSK 3]
    \definition{adv.}{parece que}
  \end{phonetics}
\end{entry}

\begin{entry}{看不起}{9,4,10}{⽬、⼀、⾛}
  \begin{phonetics}{看不起}{kan4bu5qi3}[][HSK 4]
    \definition{v.}{desprezar; desdenhar; menosprezar; ter desprezo; olhar de cima para baixo}
  \end{phonetics}
\end{entry}

\begin{entry}{看见}{9,4}{⽬、⾒}
  \begin{phonetics}{看见}{kan4 jian4}[][HSK 1]
    \definition{v.}{encontrar | enxergar | ver | avistar}
  \end{phonetics}
\end{entry}

\begin{entry}{看出}{9,5}{⽬、⼐}
  \begin{phonetics}{看出}{kan4 chu1}[][HSK 5]
    \definition{v.}{perceber; descobrir; estar ciente de; ver}
  \end{phonetics}
\end{entry}

\begin{entry}{看成}{9,6}{⽬、⼽}
  \begin{phonetics}{看成}{kan4 cheng2}[][HSK 5]
    \definition{v.}{olhar como; considerar como; tratar como; pensar como; ter como}
  \end{phonetics}
\end{entry}

\begin{entry}{看来}{9,7}{⽬、⽊}
  \begin{phonetics}{看来}{kan4 lai2}[][HSK 4]
    \definition{adv.}{parecer; parecer como se (ou embora); refere-se a um julgamento aproximado; expressa um julgamento por observação}
    \definition{v.}{ser considerado; na visão de alguém; na opinião de alguém; expressar a ideia aproximada que o locutor tem da situação}
  \end{phonetics}
\end{entry}

\begin{entry}{看到}{9,8}{⽬、⼑}
  \begin{phonetics}{看到}{kan4 dao4}[][HSK 1]
    \definition{v.}{ver}
  \end{phonetics}
\end{entry}

\begin{entry}{看法}{9,8}{⽬、⽔}
  \begin{phonetics}{看法}{kan4fa3}[][HSK 2]
    \definition[个]{s.}{modo de olhar alguma coisa | ponto de vista | opinião}
  \end{phonetics}
\end{entry}

\begin{entry}{看待}{9,9}{⽬、⼻}
  \begin{phonetics}{看待}{kan4dai4}[][HSK 5]
    \definition{v.}{tratar; considerar; olhar com atenção; ter uma certa atitude ou visão em relação a alguém ou alguma coisa}
  \end{phonetics}
\end{entry}

\begin{entry}{看病}{9,10}{⽬、⽧}
  \begin{phonetics}{看病}{kan4 bing4}[][HSK 1]
    \definition{v.+compl.}{(médico) ver um paciente | (paciente) consultar (ver) um médico}
  \end{phonetics}
\end{entry}

\begin{entry}{看起来}{9,10,7}{⽬、⾛、⽊}
  \begin{phonetics}{看起来}{kan4 qi3 lai5}[][HSK 3]
    \definition{v.}{parecer; parecer com}
  \end{phonetics}
\end{entry}

\begin{entry}{看望}{9,11}{⽬、⽉}
  \begin{phonetics}{看望}{kan4wang4}[][HSK 4]
    \definition{v.}{ver; visitar; ligar; dar uma olhada; ir até os pais, idosos, professores ou amigos para cumprimentá-los}
  \end{phonetics}
\end{entry}

\begin{entry}{看淡}{9,11}{⽬、⽔}
  \begin{phonetics}{看淡}{kan4dan4}
    \definition{v.}{considerar sem importância | ser indiferente a (fama, riqueza, etc.) | (de uma economia ou mercado) enfraquecer, ficar mais lento, diminuir a velocidade}
  \end{phonetics}
\end{entry}

\begin{entry}{矜}{9}{⽭}
  \begin{phonetics}{矜}{jin1}
    \definition{adj.}{presunçoso; vaidoso | contido; reservado; determinado}
    \definition{v.}{ter pena; simpatizar com; compadecer-se}
  \end{phonetics}
\end{entry}

\begin{entry}{砂}{9}{⽯}
  \begin{phonetics}{砂}{sha1}
    \variantof{沙}
  \end{phonetics}
\end{entry}

\begin{entry}{砍}{9}{⽯}
  \begin{phonetics}{砍}{kan3}
    \definition{v.}{cortar}
  \end{phonetics}
\end{entry}

\begin{entry}{砍刀}{9,2}{⽯、⼑}
  \begin{phonetics}{砍刀}{kan3dao1}
    \definition{s.}{facão | machete}
  \end{phonetics}
\end{entry}

\begin{entry}{砍头}{9,5}{⽯、⼤}
  \begin{phonetics}{砍头}{kan3tou2}
    \definition{v.}{decapitar}
  \end{phonetics}
\end{entry}

\begin{entry}{砍价}{9,6}{⽯、⼈}
  \begin{phonetics}{砍价}{kan3jia4}
    \definition{v.}{barganhar | cortar ou derrubar um preço}
  \end{phonetics}
\end{entry}

\begin{entry}{砍伤}{9,6}{⽯、⼈}
  \begin{phonetics}{砍伤}{kan3shang1}
    \definition{v.}{ferir com lâmina ou machado}
  \end{phonetics}
\end{entry}

\begin{entry}{砍杀}{9,6}{⽯、⽊}
  \begin{phonetics}{砍杀}{kan3sha1}
    \definition{v.}{atacar com arma branca}
  \end{phonetics}
\end{entry}

\begin{entry}{砍死}{9,6}{⽯、⽍}
  \begin{phonetics}{砍死}{kan3si3}
    \definition{v.}{matar com um machado}
  \end{phonetics}
\end{entry}

\begin{entry}{砍树}{9,9}{⽯、⽊}
  \begin{phonetics}{砍树}{kan3shu4}
    \definition{v.}{derrubar árvores}
  \end{phonetics}
\end{entry}

\begin{entry}{砍掉}{9,11}{⽯、⼿}
  \begin{phonetics}{砍掉}{kan3diao4}
    \definition{v.}{amputar}
  \end{phonetics}
\end{entry}

\begin{entry}{砍断}{9,11}{⽯、⽄}
  \begin{phonetics}{砍断}{kan3duan4}
    \definition{v.}{cortar}
  \end{phonetics}
\end{entry}

\begin{entry}{研究}{9,7}{⽯、⽳}
  \begin{phonetics}{研究}{yan2jiu1}[][HSK 4]
    \definition{v.}{estudar; pesquisar | discutir; considerar}
  \end{phonetics}
\end{entry}

\begin{entry}{研究生}{9,7,5}{⽯、⽳、⽣}
  \begin{phonetics}{研究生}{yan2 jiu1 sheng1}[][HSK 4]
    \definition[位,名]{s.}{pós-graduado; estudante de pós-graduação}
  \end{phonetics}
\end{entry}

\begin{entry}{研究所}{9,7,8}{⽯、⽳、⼾}
  \begin{phonetics}{研究所}{yan2 jiu1 suo3}[][HSK 5]
    \definition[个]{s.}{instituto de pesquisa; instituição de pesquisa científica envolvida em pesquisas em um determinado campo}
  \end{phonetics}
\end{entry}

\begin{entry}{研制}{9,8}{⽯、⼑}
  \begin{phonetics}{研制}{yan2 zhi4}[][HSK 4]
    \definition{v.}{desenvolver; fabricar; produzir | triturar; (medicina chinesa) moer}
  \end{phonetics}
\end{entry}

\begin{entry}{砖}{9}{⽯}
  \begin{phonetics}{砖}{zhuan1}
    \definition[块]{s.}{tijolo}
  \end{phonetics}
\end{entry}

\begin{entry}{祖国}{9,8}{⽰、⼞}
  \begin{phonetics}{祖国}{zu3guo2}
    \definition{s.}{pátria | terra natal}
  \end{phonetics}
\end{entry}

\begin{entry}{祝}{9}{⽰}
  \begin{phonetics}{祝}{zhu4}[][HSK 3]
    \definition*{s.}{sobrenome Zhu}
    \definition{v.}{expressar bons desejos; desejar; abençoar | orar aos deuses ou espíritos por bênçãos}
  \end{phonetics}
\end{entry}

\begin{entry}{祝好}{9,6}{⽰、⼥}
  \begin{phonetics}{祝好}{zhu4hao3}
    \definition{expr.}{desejo-lhe tudo de melhor! (ao encerrar uma correspondência)}
  \end{phonetics}
\end{entry}

\begin{entry}{祝寿}{9,7}{⽰、⼨}
  \begin{phonetics}{祝寿}{zhu4shou4}
    \definition{v.}{dar parabéns pelo aniversário (a uma pessoa idosa)}
  \end{phonetics}
\end{entry}

\begin{entry}{祝贺}{9,9}{⽰、⾙}
  \begin{phonetics}{祝贺}{zhu4he4}[][HSK 5]
    \definition[个]{s.}{congratulações; felicitações}
    \definition{v.}{congratular; felicitar; parabenizar}
  \end{phonetics}
\end{entry}

\begin{entry}{祝酒}{9,10}{⽰、⾣}
  \begin{phonetics}{祝酒}{zhu4jiu3}
    \definition{v.}{parabenizar e fazer um brinde | brindar}
  \end{phonetics}
\end{entry}

\begin{entry}{祝颂}{9,10}{⽰、⾴}
  \begin{phonetics}{祝颂}{zhu4song4}
    \definition{v.}{expressar bons desejos}
  \end{phonetics}
\end{entry}

\begin{entry}{祝祷}{9,11}{⽰、⽰}
  \begin{phonetics}{祝祷}{zhu4dao3}
    \definition{v.}{rezar | orar}
  \end{phonetics}
\end{entry}

\begin{entry}{祝谢}{9,12}{⽰、⾔}
  \begin{phonetics}{祝谢}{zhu4xie4}
    \definition{v.}{agradecer | dar parabéns}
  \end{phonetics}
\end{entry}

\begin{entry}{祝福}{9,13}{⽰、⽰}
  \begin{phonetics}{祝福}{zhu4fu2}[][HSK 4]
    \definition[个]{s.}{bênção; benzedura; benzimento; originalmente, referia-se à oração para obter as bênçãos de Deus, mas, mais tarde, refere-se a desejar paz e felicidade às pessoas}
    \definition{v.}{desejar boa sorte a alguém}
  \end{phonetics}
\end{entry}

\begin{entry}{祝愿}{9,14}{⽰、⽕}
  \begin{phonetics}{祝愿}{zhu4yuan4}
    \definition{v.}{desejar}
  \end{phonetics}
\end{entry}

\begin{entry}{神}{9}{⽰}
  \begin{phonetics}{神}{shen2}[][HSK 5]
    \definition*{s.}{Deus | sobrenome Shen}
    \definition{adj.}{inteligente; esperto |
mágico; sobrenatural;}
    \definition[个,位,尊]{s.}{divindade; deidade | espírito; mente; refere-se ao espírito, energia ou atenção de uma pessoa | olhar; expressão; expressões que refletem o estado interior}
  \end{phonetics}
\end{entry}

\begin{entry}{神奇}{9,8}{⽰、⼤}
  \begin{phonetics}{神奇}{shen2qi2}[][HSK 5]
    \definition{adj.}{mágico; peculiar; místico; milagroso; algo que parece muito novo; algo que ninguém imaginaria, mas que geralmente traz bons resultados}
    \definition{s.}{mágica; milagre}
  \end{phonetics}
\end{entry}

\begin{entry}{神明}{9,8}{⽰、⽇}
  \begin{phonetics}{神明}{shen2ming2}
    \definition{s.}{divindades | deuses}
  \end{phonetics}
\end{entry}

\begin{entry}{神经}{9,8}{⽰、⽷}
  \begin{phonetics}{神经}{shen2jing1}[][HSK 5]
    \definition{adj.}{excêntrico; estranho; peculiar; descreve anormalidade neurológica}
    \definition{s.}{nervo; um tipo de tecido presente no corpo humano ou animal que conecta o cérebro aos órgãos, transmitindo as sensações ao cérebro e as informações do cérebro aos órgãos}
  \end{phonetics}
\end{entry}

\begin{entry}{神经病学}{9,8,10,8}{⽰、⽷、⽧、⼦}
  \begin{phonetics}{神经病学}{shen2jing1bing4xue2}
    \definition{s.}{neurologia}
  \end{phonetics}
\end{entry}

\begin{entry}{神经病的}{9,8,10,8}{⽰、⽷、⽧、⽩}
  \begin{phonetics}{神经病的}{shen2jing1bing4de5}
    \definition{adj.}{neurótico}
  \end{phonetics}
\end{entry}

\begin{entry}{神话}{9,8}{⽰、⾔}
  \begin{phonetics}{神话}{shen2hua4}[][HSK 4]
    \definition[个]{s.}{mito; mitologia; conto de fadas; refere-se a deuses e deusas lendários e histórias de heróis antigos deificados | lorota; refere-se a alegações ridículas e infundadas}
  \end{phonetics}
\end{entry}

\begin{entry}{神秘}{9,10}{⽰、⽲}
  \begin{phonetics}{神秘}{shen2mi4}[][HSK 4]
    \definition{adj.}{místico; misterioso}
  \end{phonetics}
\end{entry}

\begin{entry}{神兽}{9,11}{⽰、⼋}
  \begin{phonetics}{神兽}{shen2shou4}
    \definition{s.}{animal mitológico | fera}
  \end{phonetics}
\end{entry}

\begin{entry}{神情}{9,11}{⽰、⼼}
  \begin{phonetics}{神情}{shen2 qing2}[][HSK 5]
    \definition{s.}{aparência; expressão; atividades internas reveladas no rosto das pessoas}
  \end{phonetics}
\end{entry}

\begin{entry}{神器}{9,16}{⽰、⼝}
  \begin{phonetics}{神器}{shen2qi4}
    \definition{s.}{objeto mágico | objeto simbólico do poder imperial | arma fina | ferramenta muito útil}
  \end{phonetics}
\end{entry}

\begin{entry}{秋}{9}{⽲}
  \begin{phonetics}{秋}{qiu1}
    \definition*{s.}{sobrenome Qiu}
    \definition{s.}{outono | colheita}
  \end{phonetics}
\end{entry}

\begin{entry}{秋天}{9,4}{⽲、⼤}
  \begin{phonetics}{秋天}{qiu1 tian1}[][HSK 2]
    \definition[个]{s.}{outono}
  \end{phonetics}
\end{entry}

\begin{entry}{秋季}{9,8}{⽲、⼦}
  \begin{phonetics}{秋季}{qiu1 ji4}[][HSK 4]
    \definition[个]{s.}{outono; terceiro trimestre do ano, segundo o costume chinês, refere-se ao período de três meses entre o outono e o inverno, também se refere aos sétimo, oitavo e nono meses do calendário lunar}
  \end{phonetics}
\end{entry}

\begin{entry}{种}{9}{⽲}
  \begin{phonetics}{种}{zhong3}[][HSK 3,4]
    \definition{clas.}{para tipos, espécies e gêneros de pessoas ou qualquer coisa}
    \definition{s.}{espécie | semente; estirpe; linhagem | entranhas; intestino; espinha dorsal | tipo; variedade; indica tipo, usado para pessoas e qualquer coisa}
  \end{phonetics}
  \begin{phonetics}{种}{zhong4}
    \definition{v.}{plantar; semear; crescer; cultivar}
  \end{phonetics}
\end{entry}

\begin{entry}{种子}{9,3}{⽲、⼦}
  \begin{phonetics}{种子}{zhong3zi5}[][HSK 3]
    \definition[颗,粒]{s.}{semente; um órgão exclusivo de certas plantas, geralmente composto de três partes: tegumento, embrião e endosperma, as sementes podem germinar e se tornar novas plantas sob certas condições | jogador cabeça de chave; na competição, quando é realizada a fase eliminatória, são escolhidos os jogadores mais fortes de cada equipe}
  \end{phonetics}
\end{entry}

\begin{entry}{种地}{9,6}{⽲、⼟}
  \begin{phonetics}{种地}{zhong4di4}
    \definition{v.}{cultivar | trabalhar a terra}
  \end{phonetics}
\end{entry}

\begin{entry}{种种}{9,9}{⽲、⽲}
  \begin{phonetics}{种种}{zhong3zhong3}
    \definition{adj.}{todos os tipos de}
  \end{phonetics}
\end{entry}

\begin{entry}{种类}{9,9}{⽲、⽶}
  \begin{phonetics}{种类}{zhong3lei4}[][HSK 4]
    \definition{s.}{espécie; classe; tipo; variedade; categoria; classificação de alguma coisa de acordo com sua natureza e características}
  \end{phonetics}
\end{entry}

\begin{entry}{种族灭绝}{9,11,5,9}{⽲、⽅、⽕、⽷}
  \begin{phonetics}{种族灭绝}{zhong3zu2mie4jue2}
    \definition{s.}{genocídio | extinção étnica}
  \end{phonetics}
\end{entry}

\begin{entry}{种麻}{9,11}{⽲、⿇}
  \begin{phonetics}{种麻}{zhong3ma2}
    \definition{s.}{planta de cânhamo (feminina)}
  \end{phonetics}
\end{entry}

\begin{entry}{种植}{9,12}{⽲、⽊}
  \begin{phonetics}{种植}{zhong4zhi2}[][HSK 4]
    \definition{v.}{plantar; crescer; cultivar; enterrar as sementes de uma planta no solo; plantar as mudas de uma planta no solo}
  \end{phonetics}
\end{entry}

\begin{entry}{种薯}{9,16}{⽲、⾋}
  \begin{phonetics}{种薯}{zhong3shu3}
    \definition{s.}{tubérculo semente}
  \end{phonetics}
\end{entry}

\begin{entry}{科}{9}{⽲}
  \begin{phonetics}{科}{ke1}[][HSK 2]
    \definition*{s.}{sobrenome Ke}
    \definition{s.}{um ramo de estudo acadêmico ou profissional |uma divisão ou subdivisão de uma unidade administrativa | família | instruções de palco no drama chinês clássico}
  \end{phonetics}
\end{entry}

\begin{entry}{科技}{9,7}{⽲、⼿}
  \begin{phonetics}{科技}{ke1 ji4}[][HSK 3]
    \definition{s.}{ciência e tecnologia}
  \end{phonetics}
\end{entry}

\begin{entry}{科学}{9,8}{⽲、⼦}
  \begin{phonetics}{科学}{ke1xue2}[][HSK 2]
    \definition{adj.}{científico}
    \definition[门]{s.}{ciência}
  \end{phonetics}
\end{entry}

\begin{entry}{科学家}{9,8,10}{⽲、⼦、⼧}
  \begin{phonetics}{科学家}{ke1xue2jia1}
    \definition[个]{s.}{cientista}
  \end{phonetics}
\end{entry}

\begin{entry}{秒}{9}{⽲}
  \begin{phonetics}{秒}{miao3}[][HSK 5]
    \definition{adv.}{(coloquial) instantaneamente}
    \definition{s.}{segundo (unidade de tempo) | segundo (unidade de medida angular)}
  \end{phonetics}
\end{entry}

\begin{entry}{穿}{9}{⽳}
  \begin{phonetics}{穿}{chuan1}[][HSK 1]
    \definition{v.}{vestir}
  \end{phonetics}
\end{entry}

\begin{entry}{穿上}{9,3}{⽳、⼀}
  \begin{phonetics}{穿上}{chuan1 shang4}[][HSK 4]
    \definition{v.}{vestir (roupas, etc.); colocar roupas}
  \end{phonetics}
\end{entry}

\begin{entry}{突出}{9,5}{⽳、⼐}
  \begin{phonetics}{突出}{tu1chu1}[][HSK 3]
    \definition{adj.}{proeminente; excelente}
    \definition{v.}{romper | enfatizar; destacar; dar destaque a | sobressair; projetar-se; destacar-se}
  \end{phonetics}
\end{entry}

\begin{entry}{突破}{9,10}{⽳、⽯}
  \begin{phonetics}{突破}{tu1po4}[][HSK 5]
    \definition{v.}{romper; fazer uma descoberta revolucionária; concentrar esforços em um único ponto de ataque, reunir o sucesso | quebrar (limite); superar (dificuldade); superar dificuldades; ultrapassar os números ou limites anteriores, superar recordes anteriores, etc.; romper com as limitações e restrições anteriores}
  \end{phonetics}
\end{entry}

\begin{entry}{突然}{9,12}{⽳、⽕}
  \begin{phonetics}{突然}{tu1ran2}[][HSK 3]
    \definition{adj.}{repentino; abrupto; inesperado}
    \definition{adv.}{de repente; abruptamente; inesperadamente}
  \end{phonetics}
\end{entry}

\begin{entry}{类}{9}{⽶}
  \begin{phonetics}{类}{lei4}[][HSK 3]
    \definition*{s.}{sobrenome Lei}
    \definition{s.}{classe; categoria; tipo; espécie}
    \definition{v.}{assemelhar-se a; ser semelhante a}
  \end{phonetics}
\end{entry}

\begin{entry}{类似}{9,6}{⽶、⼈}
  \begin{phonetics}{类似}{lei4si4}[][HSK 3]
    \definition{adj.}{semelhante; análogo}
  \end{phonetics}
\end{entry}

\begin{entry}{类型}{9,9}{⽶、⼟}
  \begin{phonetics}{类型}{lei4xing2}[][HSK 4]
    \definition[种,个]{s.}{tipo; espécie; categoria; tipos formados por coisas com características comuns}
  \end{phonetics}
\end{entry}

\begin{entry}{结}{9}{⽷}
  \begin{phonetics}{结}{jie1}
    \definition{v.}{dar (frutos); formar (sementes); produzir frutos ou sementes (uma planta)}
  \end{phonetics}
  \begin{phonetics}{结}{jie2}[][HSK 4]
    \definition*{s.}{sobrenome Jie}
    \definition{s.}{nó | declaração juramentada; garantia por escrito; documento que, antigamente, significava um reconhecimento de encerramento ou uma garantia de responsabilidade}
    \definition{v.}{amarrar; tricotar; dar nó; tecer | formar; forjar; cimentar; solidificar | resolver; concluir | combinar; formar um relacionamento}
  \end{phonetics}
\end{entry}

\begin{entry}{结合}{9,6}{⽷、⼝}
  \begin{phonetics}{结合}{jie2he2}[][HSK 3]
    \definition{v.}{ligar; unir; combinar; integrar | casar-se; unir-se em matrimônio}
  \end{phonetics}
\end{entry}

\begin{entry}{结论}{9,6}{⽷、⾔}
  \begin{phonetics}{结论}{jie2lun4}[][HSK 4]
    \definition[个]{s.}{conclusão; palavra final sobre uma pessoa ou coisa após investigação e pesquisa | veredito; julgamento deduzido de premissas também é chamado de conclusão}
  \end{phonetics}
\end{entry}

\begin{entry}{结局}{9,7}{⽷、⼫}
  \begin{phonetics}{结局}{jie2ju2}
    \definition{s.}{conclusão | fim | final}
  \end{phonetics}
\end{entry}

\begin{entry}{结束}{9,7}{⽷、⽊}
  \begin{phonetics}{结束}{jie2shu4}[][HSK 3]
    \definition{v.}{finalizar; fechar; terminar; concluir; encerrar}
  \end{phonetics}
\end{entry}

\begin{entry}{结束工作}{9,7,3,7}{⽷、⽊、⼯、⼈}
  \begin{phonetics}{结束工作}{jie2shu4gong1zuo4}
    \definition{s.}{trabalho final}
    \definition{v.}{terminar o trabalho}
  \end{phonetics}
\end{entry}

\begin{entry}{结束区}{9,7,4}{⽷、⽊、⼖}
  \begin{phonetics}{结束区}{jie2shu4 qu1}
    \definition{s.}{zona final}
  \end{phonetics}
\end{entry}

\begin{entry}{结束文本}{9,7,4,5}{⽷、⽊、⽂、⽊}
  \begin{phonetics}{结束文本}{jie2shu4 wen2ben3}
    \definition{s.}{texto final}
  \end{phonetics}
\end{entry}

\begin{entry}{结束剂}{9,7,8}{⽷、⽊、⼑}
  \begin{phonetics}{结束剂}{jie2shu4 ji4}
    \definition{s.}{finalizador}
  \end{phonetics}
\end{entry}

\begin{entry}{结束语}{9,7,9}{⽷、⽊、⾔}
  \begin{phonetics}{结束语}{jie2shu4yu3}
    \definition{s.}{conclusões finais | considerações finais}
  \end{phonetics}
\end{entry}

\begin{entry}{结束辩论}{9,7,16,6}{⽷、⽊、⾟、⾔}
  \begin{phonetics}{结束辩论}{jie2shu4 bian4 lun4}
    \definition{s.}{debate de encerramento}
  \end{phonetics}
\end{entry}

\begin{entry}{结社自由}{9,7,6,5}{⽷、⽰、⾃、⽥}
  \begin{phonetics}{结社自由}{jie2she4zi4you2}
    \definition{s.}{(constitucional) liberdade de associação}
  \end{phonetics}
\end{entry}

\begin{entry}{结实}{9,8}{⽷、⼧}
  \begin{phonetics}{结实}{jie1shi5}[][HSK 3]
    \definition{adj.}{sólido; resistente; durável | forte; resistente; robusto}
  \end{phonetics}
\end{entry}

\begin{entry}{结构}{9,8}{⽷、⽊}
  \begin{phonetics}{结构}{jie2gou4}[][HSK 4]
    \definition[个,座]{s.}{estrutura; composição; construção; formação; constituição; tecido; forma; sistematização; mecânica; organização | arquitetura; estrutura; construção; construção de partes de edifícios com suporte de carga ou com carga externa | textura (geológico)}
  \end{phonetics}
\end{entry}

\begin{entry}{结果}{9,8}{⽷、⽊}
  \begin{phonetics}{结果}{jie1guo3}
    \definition{v.}{dar frutos}
  \end{phonetics}
  \begin{phonetics}{结果}{jie2guo3}[][HSK 2]
    \definition{s.}{resultado | conclusão}
    \definition{v.}{despachar | matar}
  \end{phonetics}
\end{entry}

\begin{entry}{结婚}{9,11}{⽷、⼥}
  \begin{phonetics}{结婚}{jie2hun1}[][HSK 3]
    \definition{v.+compl.}{casar; casar-se}
  \end{phonetics}
\end{entry}

\begin{entry}{结婚礼服}{9,11,5,8}{⽷、⼥、⽰、⽉}
  \begin{phonetics}{结婚礼服}{jie2hun1 li3 fu2}
    \definition{s.}{vestido de casamento}
  \end{phonetics}
\end{entry}

\begin{entry}{绕}{9}{⽷}
  \begin{phonetics}{绕}{rao4}[][HSK 5]
    \definition*{s.}{sobrenome Rao}
    \definition{v.}{enrolar; bobinar; rebobinar | mover-se em círculo; girar; revolver | fazer um desvio; contornar; dar a volta | confundir; desorientar}
  \end{phonetics}
\end{entry}

\begin{entry}{给}{9}{⽷}
  \begin{phonetics}{给}{gei3}[][HSK 1]
    \definition{prep.}{a | para}
    \definition{v.}{dar | permitir | fazer alguma coisa (para alguém)}
  \end{phonetics}
  \begin{phonetics}{给}{ji3}
    \definition{v.}{fornecer | prover}
  \end{phonetics}
\end{entry}

\begin{entry}{给……打电话}{9,5,5,8}{⽷、⼿、⽥、⾔}
  \begin{phonetics}{给……打电话}{gei3 da3 dian4 hua4}
    \definition{expr.}{telefonar para alguém}
    \seeref{打电话}{da3 dian4 hua4}
  \end{phonetics}
\end{entry}

\begin{entry}{绝不}{9,4}{⽷、⼀}
  \begin{phonetics}{绝不}{jue2bu4}
    \definition{adv.}{definitivamente não | de forma alguma | sob nenhuma circunstância}
  \end{phonetics}
\end{entry}

\begin{entry}{绝对}{9,5}{⽷、⼨}
  \begin{phonetics}{绝对}{jue2dui4}[][HSK 3]
    \definition{adj.}{absoluto; extremo}
    \definition{adv.}{absolutamente}
  \end{phonetics}
\end{entry}

\begin{entry}{绝招}{9,8}{⽷、⼿}
  \begin{phonetics}{绝招}{jue2zhao1}
    \definition{s.}{habilidade única | movimento delicado inesperado (como último recurso) | golpe de mestre | golpe final}
  \end{phonetics}
\end{entry}

\begin{entry}{绝版}{9,8}{⽷、⽚}
  \begin{phonetics}{绝版}{jue2ban3}
    \definition{adj.}{esgotado | fora de catálogo}
  \end{phonetics}
\end{entry}

\begin{entry}{绝望}{9,11}{⽷、⽉}
  \begin{phonetics}{绝望}{jue2 wang4}[][HSK 5]
    \definition{v.+compl.}{desesperar; desistir de toda esperança; perder toda esperança de}
  \end{phonetics}
\end{entry}

\begin{entry}{统一}{9,1}{⽷、⼀}
  \begin{phonetics}{统一}{tong3yi1}[][HSK 4]
    \definition{adj.}{unificado; unitário; centralizado; consistente}
    \definition{v.}{unificar; unir; integrar; padronizar}
  \end{phonetics}
\end{entry}

\begin{entry}{统计}{9,4}{⽷、⾔}
  \begin{phonetics}{统计}{tong3ji4}[][HSK 4]
    \definition{v.}{compilar estatísticas; refere-se à realização de trabalho estatístico, ou seja, coletar, reunir, analisar e extrapolar dados sobre um fenômeno | somar; adicionar; contar}
  \end{phonetics}
\end{entry}

\begin{entry}{罚}{9}{⽹}
  \begin{phonetics}{罚}{fa2}[][HSK 5]
    \definition{s.}{punição; penalidade}
    \definition{v.}{punir; penalizar; multar; confiscar}
  \end{phonetics}
\end{entry}

\begin{entry}{罚款}{9,12}{⽹、⽋}
  \begin{phonetics}{罚款}{fa2kuan3}[][HSK 5]
    \definition[笔,次,宗]{s.}{multa; penalidade; refere-se ao dinheiro pago por uma pessoa ou entidade de acordo com as disposições de um delito ou violação de contrato ou contrato}
    \definition{v.+compl.}{multar; penalizar; exigir, de acordo com os regulamentos, uma determinada quantia de dinheiro de uma pessoa ou entidade que tenha violado a lei ou descumprido um regulamento ou contrato}
  \end{phonetics}
\end{entry}

\begin{entry}{美}{9}{⽺}
  \begin{phonetics}{美}{mei3}[][HSK 3]
    \definition*{s.}{Abreviatura de América (美洲) | Abreviatura de Estados Unidos da América (美国)}
    \definition{adj.}{lindo; bonito; belo; atraente | satisfatório; bom; agradável}
    \definition{v.}{embelezar; enfeitar | orgulhar-se de; estar satisfeito consigo mesmo}
  \seealsoref{美国}{mei3guo2}
  \seealsoref{美洲}{mei3zhou1}
  \end{phonetics}
\end{entry}

\begin{entry}{美女}{9,3}{⽺、⼥}
  \begin{phonetics}{美女}{mei3 nv3}[][HSK 4]
    \definition[个,位]{s.}{beldade; mulher bonita; uma jovem linda}
  \end{phonetics}
\end{entry}

\begin{entry}{美元}{9,4}{⽺、⼉}
  \begin{phonetics}{美元}{mei3yuan2}[][HSK 3]
    \definition*[元,笔,沓]{s.}{Dólar Americano}
  \end{phonetics}
\end{entry}

\begin{entry}{美术}{9,5}{⽺、⽊}
  \begin{phonetics}{美术}{mei3shu4}[][HSK 3]
    \definition[种]{s.}{arte; belas artes | pintura}
  \end{phonetics}
\end{entry}

\begin{entry}{美甲}{9,5}{⽺、⽥}
  \begin{phonetics}{美甲}{mei3jia3}
    \definition{s.}{manicure e/ou pedicure}
  \end{phonetics}
\end{entry}

\begin{entry}{美好}{9,6}{⽺、⼥}
  \begin{phonetics}{美好}{mei3 hao3}[][HSK 3]
    \definition{adj.}{bem; feliz; glorioso}
  \end{phonetics}
\end{entry}

\begin{entry}{美丽}{9,7}{⽺、⼀}
  \begin{phonetics}{美丽}{mei3li4}[][HSK 3]
    \definition{adj.}{bonito; lindo}
  \end{phonetics}
\end{entry}

\begin{entry}{美味}{9,8}{⽺、⼝}
  \begin{phonetics}{美味}{mei3wei4}
    \definition{adj.}{delicioso}
    \definition{s.}{comida deliciosa | delicadeza (\emph{delicacy})}
  \end{phonetics}
\end{entry}

\begin{entry}{美国}{9,8}{⽺、⼞}
  \begin{phonetics}{美国}{mei3guo2}
    \definition*{s.}{Estados Unidos da América}
  \end{phonetics}
\end{entry}

\begin{entry}{美国人}{9,8,2}{⽺、⼞、⼈}
  \begin{phonetics}{美国人}{mei3guo2ren2}
    \definition{s.}{americano | pessoa ou povo dos Estados Unidos da América}
  \end{phonetics}
\end{entry}

\begin{entry}{美学}{9,8}{⽺、⼦}
  \begin{phonetics}{美学}{mei3xue2}
    \definition{s.}{estética}
  \end{phonetics}
\end{entry}

\begin{entry}{美金}{9,8}{⽺、⾦}
  \begin{phonetics}{美金}{mei3 jin1}[][HSK 4]
    \definition{s.}{USD; dólar americano: a moeda local dos Estados Unidos}
  \end{phonetics}
\end{entry}

\begin{entry}{美洲}{9,9}{⽺、⽔}
  \begin{phonetics}{美洲}{mei3zhou1}
    \definition*{s.}{América (incluindo Norte, Central e Sul)}
  \end{phonetics}
\end{entry}

\begin{entry}{美洲人}{9,9,2}{⽺、⽔、⼈}
  \begin{phonetics}{美洲人}{mei3zhou1ren2}
    \definition{s.}{americano | pessoa ou povo do continente Americano}
  \end{phonetics}
\end{entry}

\begin{entry}{美食}{9,9}{⽺、⾷}
  \begin{phonetics}{美食}{mei3 shi2}[][HSK 3]
    \definition[种,道,桌]{s.}{iguaria; comida deliciosa}
  \end{phonetics}
\end{entry}

\begin{entry}{耍}{9}{⽽}
  \begin{phonetics}{耍}{shua3}
    \definition{v.}{brincar com | empunhar | agir (legal, calmo, tranquilo, descolado, etc.) | exibir (uma habilidade, o temperamento de alguém, etc.)}
  \end{phonetics}
\end{entry}

\begin{entry}{耍赖}{9,13}{⽽、⾙}
  \begin{phonetics}{耍赖}{shua3lai4}
    \definition{v.}{agir descaradamente | recusar -se a reconhecer que alguém perdeu o jogo ou fez uma promessa, etc. | agir como um idiota | agir como se algo nunca tivesse acontecido}
  \end{phonetics}
\end{entry}

\begin{entry}{耐心}{9,4}{⽽、⼼}
  \begin{phonetics}{耐心}{nai4xin1}[][HSK 5]
    \definition{adj.}{paciente}
    \definition{s.}{paciência; não se incomoda com as dificuldades e tem um caráter tolerante}
    \definition{v.}{ser paciente}
  \end{phonetics}
\end{entry}

\begin{entry}{胃}{9}{⾁}
  \begin{phonetics}{胃}{wei4}[][HSK 5]
    \definition*{s.}{Wei, uma das mansões lunares; uma das vinte e oito constelações}
    \definition{s.}{estômago; parte do aparelho digestivo}
  \end{phonetics}
\end{entry}

\begin{entry}{胃口}{9,3}{⾁、⼝}
  \begin{phonetics}{胃口}{wei4kou3}
    \definition{s.}{apetite}
  \end{phonetics}
\end{entry}

\begin{entry}{胆}{9}{⾁}
  \begin{phonetics}{胆}{dan3}[][HSK 5]
    \definition[个,颗]{s.}{vesícula biliar | coragem; bravura | um recipiente interno semelhante a uma bexiga; algo que se encaixa dentro de um objeto e pode conter água, ar, etc.}
  \end{phonetics}
\end{entry}

\begin{entry}{胆小}{9,3}{⾁、⼩}
  \begin{phonetics}{胆小}{dan3 xiao3}[][HSK 5]
    \definition{adj.}{tímido; covarde}
  \end{phonetics}
\end{entry}

\begin{entry}{胆小鬼}{9,3,9}{⾁、⼩、⿁}
  \begin{phonetics}{胆小鬼}{dan3xiao3gui3}
    \definition{adj.}{covarde | medroso}
  \end{phonetics}
\end{entry}

\begin{entry}{背}{9}{⾁}
  \begin{phonetics}{背}{bei1}[][HSK 2]
    \definition{v.}{estar sobrecarregado | carregar nas costas ou no ombro}
  \end{phonetics}
  \begin{phonetics}{背}{bei4}[][HSK 3]
    \definition{adv.}{a parte de trás de um corpo ou objeto}
    \definition{s.}{costas | (gíria) azarado}
    \definition{v.}{esconder algo de | decorar | recitar de memória | virar as costas}
  \end{phonetics}
\end{entry}

\begin{entry}{背包}{9,5}{⾁、⼓}
  \begin{phonetics}{背包}{bei1 bao1}[][HSK 5]
    \definition[个]{s.}{mochila; mochila de ataque; mochila de infantaria; pacotes de roupas carregados nas costas quando marcham}
  \end{phonetics}
\end{entry}

\begin{entry}{背后}{9,6}{⾁、⼝}
  \begin{phonetics}{背后}{bei4 hou4}[][HSK 3]
    \definition{s.}{parte de trás | traseira | nas costas de alguém}
  \end{phonetics}
\end{entry}

\begin{entry}{背景}{9,12}{⾁、⽇}
  \begin{phonetics}{背景}{bei4jing3}[][HSK 4]
    \definition[种]{s.}{pano de fundo; fundo; cenário de teatro, filme ou drama de TV | fundo; cenário que permeia a imagem principal na tela | condições sociais; ambientes históricos (significativamente influentes para algo ou alguém) | poder que dá suporte a alguém}
  \end{phonetics}
\end{entry}

\begin{entry}{胖}{9}{⾁}
  \begin{phonetics}{胖}{pan2}
    \definition{adj.}{saudável}
  \end{phonetics}
  \begin{phonetics}{胖}{pang4}[][HSK 3]
    \definition{adj.}{gordo; robusto; rechonchudo}
  \end{phonetics}
\end{entry}

\begin{entry}{胖子}{9,3}{⾁、⼦}
  \begin{phonetics}{胖子}{pang4 zi5}[][HSK 4]
    \definition{s.}{obeso; gordo; pessoa gorda}
  \end{phonetics}
\end{entry}

\begin{entry}{胚}{9}{⾁}
  \begin{phonetics}{胚}{pei1}
    \definition{s.}{embrião}
  \end{phonetics}
\end{entry}

\begin{entry}{胜}{9}{⾁}
  \begin{phonetics}{胜}{sheng4}[][HSK 3]
    \definition{adj.}{soberbo; maravilhoso; adorável}
    \definition[场]{s.}{vitória; sucesso | penteado de mulher}
    \definition{v.}{ganhar; derrotar; vencer; ter sucesso | superar; ser superior a; levar a melhor sobre | ser igual a; poder suportar}
  \end{phonetics}
\end{entry}

\begin{entry}{胜负}{9,6}{⾁、⾙}
  \begin{phonetics}{胜负}{sheng4fu4}[][HSK 5]
    \definition{s.}{vitória ou derrota; sucesso ou fracasso}
  \end{phonetics}
\end{entry}

\begin{entry}{胜利}{9,7}{⾁、⼑}
  \begin{phonetics}{胜利}{sheng4li4}[][HSK 3]
    \definition{adv.}{com sucesso; triunfantemente}
    \definition[场,个]{s.}{vitória; triunfo; sucesso}
    \definition{v.}{ganhar; vencer; triunfar; ter sucesso}
  \end{phonetics}
\end{entry}

\begin{entry}{胜算}{9,14}{⾁、⽵}
  \begin{phonetics}{胜算}{sheng4suan4}
    \definition{s.}{probabilidade de sucesso | estratégia que garante o sucesso}
    \definition{v.}{ter certeza do sucesso}
  \end{phonetics}
\end{entry}

\begin{entry}{胡子}{9,3}{⾁、⼦}
  \begin{phonetics}{胡子}{hu2 zi5}[][HSK 5]
    \definition[团,根,个]{s.}{barba; bigode | bandido; salteador}
  \end{phonetics}
\end{entry}

\begin{entry}{胡同儿}{9,6,2}{⾁、⼝、⼉}
  \begin{phonetics}{胡同儿}{hu2 tong4r5}[][HSK 5]
    \definition{s.}{beco; via; rua}
  \end{phonetics}
\end{entry}

\begin{entry}{胡萝卜}{9,11,2}{⾁、⾋、⼘}
  \begin{phonetics}{胡萝卜}{hu2luo2bo5}
    \definition{s.}{cenoura}
  \end{phonetics}
\end{entry}

\begin{entry}{舁}{9}{⾅}
  \begin{phonetics}{舁}{yu2}
    \definition{v.}{levantar; elevar | aumentar}
  \end{phonetics}
\end{entry}

\begin{entry}{范围}{9,7}{⾋、⼞}
  \begin{phonetics}{范围}{fan4wei2}[][HSK 3]
    \definition[个]{s.}{escopo; limite; alcance}
    \definition{v.}{estabelecer limites para; limitar o escopo de}
  \end{phonetics}
\end{entry}

\begin{entry}{茶}{9}{⾋}
  \begin{phonetics}{茶}{cha2}[][HSK 1]
    \definition[杯,壶]{s.}{chá | pé (planta) de chá}
  \end{phonetics}
\end{entry}

\begin{entry}{茶叶}{9,5}{⾋、⼝}
  \begin{phonetics}{茶叶}{cha2 ye4}[][HSK 4]
    \definition[盒,罐,包,片]{s.}{chá; folhas de chá; as folhas jovens da planta do chá que são processadas para produzir bebidas}
  \end{phonetics}
\end{entry}

\begin{entry}{草}{9}{⾋}
  \begin{phonetics}{草}{cao3}[][HSK 2]
    \definition[棵,撮,株,根]{s.}{erva | grama}
  \end{phonetics}
\end{entry}

\begin{entry}{草地}{9,6}{⾋、⼟}
  \begin{phonetics}{草地}{cao3 di4}[][HSK 2]
    \definition[片]{s.}{relva | pastagem}
  \end{phonetics}
\end{entry}

\begin{entry}{草纸}{9,7}{⾋、⽷}
  \begin{phonetics}{草纸}{cao3zhi3}
    \definition{s.}{papel pardo | pergaminho | papel de palha áspero | papel higiênico}
  \end{phonetics}
\end{entry}

\begin{entry}{草原}{9,10}{⾋、⼚}
  \begin{phonetics}{草原}{cao3 yuan2}[][HSK 5]
    \definition[片,个]{s.}{estepe; pradaria; grandes áreas de terra coberta de vegetação em áreas semiáridas, intercaladas com árvores tolerantes à seca}
  \end{phonetics}
\end{entry}

\begin{entry}{草莓}{9,10}{⾋、⾋}
  \begin{phonetics}{草莓}{cao3mei2}
    \definition[颗]{s.}{morango}
  \end{phonetics}
\end{entry}

\begin{entry}{荒芜}{9,7}{⾋、⾋}
  \begin{phonetics}{荒芜}{huang1wu2}
    \definition{adj.}{estéril}
  \end{phonetics}
\end{entry}

\begin{entry}{荔枝}{9,8}{⾋、⽊}
  \begin{phonetics}{荔枝}{li4zhi1}
    \definition{s.}{lichia}
  \end{phonetics}
\end{entry}

\begin{entry}{药}{9}{⾋}
  \begin{phonetics}{药}{yao4}[][HSK 2]
    \definition[种,服,味]{s.}{medicamento | remédio | droga}
  \end{phonetics}
\end{entry}

\begin{entry}{药丸}{9,3}{⾋、⼂}
  \begin{phonetics}{药丸}{yao4wan2}
    \definition[粒]{s.}{pílula}
  \end{phonetics}
\end{entry}

\begin{entry}{药水}{9,4}{⾋、⽔}
  \begin{phonetics}{药水}{yao4 shui3}[][HSK 2]
    \definition{s.}{remédio engarrafado | loção | medicamento em forma líquida}
  \end{phonetics}
\end{entry}

\begin{entry}{药片}{9,4}{⾋、⽚}
  \begin{phonetics}{药片}{yao4 pian4}[][HSK 2]
    \definition[片]{s.}{uma pílula ou comprimido (remédio)}
  \end{phonetics}
\end{entry}

\begin{entry}{药补}{9,7}{⾋、⾐}
  \begin{phonetics}{药补}{yao4bu3}
    \definition{s.}{suplemento dietético medicinal que ajuda a melhorar a saúde}
  \end{phonetics}
\end{entry}

\begin{entry}{药典}{9,8}{⾋、⼋}
  \begin{phonetics}{药典}{yao4dian3}
    \definition{s.}{farmacopéia}
  \end{phonetics}
\end{entry}

\begin{entry}{药店}{9,8}{⾋、⼴}
  \begin{phonetics}{药店}{yao4 dian4}[][HSK 2]
    \definition{s.}{farmácia | drogaria | loja de produtos químicos}
  \end{phonetics}
\end{entry}

\begin{entry}{药房}{9,8}{⾋、⼾}
  \begin{phonetics}{药房}{yao4fang2}
    \definition{s.}{farmácia | drogaria}
  \end{phonetics}
\end{entry}

\begin{entry}{药物}{9,8}{⾋、⽜}
  \begin{phonetics}{药物}{yao4 wu4}[][HSK 4]
    \definition{s.}{droga; medicamento; remédio; substâncias que controlam doenças, pragas, etc.}
  \end{phonetics}
\end{entry}

\begin{entry}{药品}{9,9}{⾋、⼝}
  \begin{phonetics}{药品}{yao4pin3}
    \definition{s.}{medicamento | remédio | droga}
  \end{phonetics}
\end{entry}

\begin{entry}{药签}{9,13}{⾋、⽵}
  \begin{phonetics}{药签}{yao4qian1}
    \definition{s.}{cotonete médico}
  \end{phonetics}
\end{entry}

\begin{entry}{药膳}{9,16}{⾋、⾁}
  \begin{phonetics}{药膳}{yao4shan4}
    \definition{s.}{dieta medicinal}
  \end{phonetics}
\end{entry}

\begin{entry}{药罐}{9,23}{⾋、⽸}
  \begin{phonetics}{药罐}{yao4guan4}
    \definition{s.}{frasco de remédio}
  \end{phonetics}
\end{entry}

\begin{entry}{虽}{9}{⾍}
  \begin{phonetics}{虽}{sui1}
    \definition{conj.}{no entanto | embora | mesmo se/embora}
  \end{phonetics}
\end{entry}

\begin{entry}{虽然}{9,12}{⾍、⽕}
  \begin{phonetics}{虽然}{sui1 ran2}[][HSK 2]
    \definition{conj.}{embora (frequentemente usado correlativamente com 可是, 但是, etc); geralmente é usado no início de uma frase para indicar que o fato anterior foi reconhecido, mas não mudará o que acontecerá em seguida}
  \seealsoref{但是}{dan4 shi4}
  \seealsoref{可是}{ke3shi4}
  \end{phonetics}
\end{entry}

\begin{entry}{虾}{9}{⾍}
  \begin{phonetics}{虾}{xia1}
    \definition{s.}{camarão}
  \end{phonetics}
\end{entry}

\begin{entry}{蚂蚁}{9,9}{⾍、⾍}
  \begin{phonetics}{蚂蚁}{ma3yi3}
    \definition{s.}{formiga}
  \end{phonetics}
\end{entry}

\begin{entry}{要}{9}{⾑}
  \begin{phonetics}{要}{yao1}[][HSK 1]
    \definition*{s.}{sobrenome Yao}
    \definition{v.}{exigir; pedir; requerer; solicitar; ter algo de que depender e forçar | forçar; coagir; ameaçar}
  \end{phonetics}
  \begin{phonetics}{要}{yao4}[][HSK 4]
    \definition{conj.}{suponha; no caso; se, indicando um relacionamento hipotético | ou; ou\dots ou\dots}
    \definition{s.}{ponto principal; essencial; manchete}
    \definition{v.}{querer; desejar | querer; pedir; deseja; querer obter; querer manter | recuperar algo; dizer a alguém que espere por algo para lhe dar ou devolver | pedir (ou querer) que alguém faça algo; pedir a alguém para fazer algo, quando usado para conseguir que alguém faça algo, tem um tom de comando e pode ser indelicado | precisar; tomar; pegar | deve; deveria; é necessário (imperativo, essencial) que\dots | estar indo para | querer; ter um desejo por; expressar determinação ou desejo de fazer algo | pode; deve;  indica uma estimativa, usada para comparação}
  \seealsoref{要是}{yao4shi5}
  \end{phonetics}
\end{entry}

\begin{entry}{要么……要么……}{9,3,9,3}{⾑、⼃、⾑、⼃}
  \begin{phonetics}{要么……要么……}{yao4me5 yao4me5}
    \definition{conj.}{ou\dots ou\dots}
  \end{phonetics}
\end{entry}

\begin{entry}{要义}{9,3}{⾑、⼂}
  \begin{phonetics}{要义}{yao4yi4}
    \definition{s.}{resumo | o essencial}
  \end{phonetics}
\end{entry}

\begin{entry}{要不}{9,4}{⾑、⼀}
  \begin{phonetics}{要不}{yao4bu4}
    \definition{conj.}{de outra forma | se não | outro | ou}
  \end{phonetics}
\end{entry}

\begin{entry}{要不然}{9,4,12}{⾑、⼀、⽕}
  \begin{phonetics}{要不然}{yao4bu4ran2}
    \definition{conj.}{de outra forma | se não | outro | ou}
  \end{phonetics}
\end{entry}

\begin{entry}{要好}{9,6}{⾑、⼥}
  \begin{phonetics}{要好}{yao4hao3}
    \definition{v.}{ser amigos íntimos | estar em boas condições}
  \end{phonetics}
\end{entry}

\begin{entry}{要死}{9,6}{⾑、⽍}
  \begin{phonetics}{要死}{yao4si3}
    \definition{adv.}{extremamente | muito}
  \end{phonetics}
\end{entry}

\begin{entry}{要求}{9,7}{⾑、⽔}
  \begin{phonetics}{要求}{yao1qiu2}[][HSK 2]
    \definition[点]{s.}{requerimento}
    \definition{v.}{pedir | exigir | solicitar | fazer uma reivindicação}
  \end{phonetics}
\end{entry}

\begin{entry}{要挟}{9,9}{⾑、⼿}
  \begin{phonetics}{要挟}{yao1xie2}
    \definition{v.}{chantagear | ameaçar}
  \end{phonetics}
\end{entry}

\begin{entry}{要是}{9,9}{⾑、⽇}
  \begin{phonetics}{要是}{yao4shi5}[][HSK 3]
    \definition{conj.}{se; no caso; orações de conexão, expressando relações hipotéticas, equivalentes a ``se'', podem ser usadas com ``então''}
  \end{phonetics}
\end{entry}

\begin{entry}{要是……的话}{9,9,8,8}{⾑、⽇、⽩、⾔}
  \begin{phonetics}{要是……的话}{yao4shi5 de5hua4}[][HSK 2,3]
    \definition{conj.}{se\dots no caso de}
  \end{phonetics}
\end{entry}

\begin{entry}{要点}{9,9}{⾑、⽕}
  \begin{phonetics}{要点}{yao4dian3}
    \definition{s.}{pontos principais | essencial}
  \end{phonetics}
\end{entry}

\begin{entry}{要谎}{9,11}{⾑、⾔}
  \begin{phonetics}{要谎}{yao4huang3}
    \definition{v.}{pedir um preço enorme (como primeiro passo de negociação)}
  \end{phonetics}
\end{entry}

\begin{entry}{要强}{9,12}{⾑、⼸}
  \begin{phonetics}{要强}{yao4qiang2}
    \definition{adj.}{ansioso para se destacar | ansioso para progredir na vida | obstinado}
  \end{phonetics}
\end{entry}

\begin{entry}{觉得}{9,11}{⾒、⼻}
  \begin{phonetics}{觉得}{jue2de5}[][HSK 1]
    \definition{v.}{pensar que\dots | sentir que\dots | sentir (desconfortável, etc.)}
  \end{phonetics}
\end{entry}

\begin{entry}{语}{9}{⾔}
  \begin{phonetics}{语}{yu3}
    \definition{s.}{língua; linguagem | dito; provérbio; refere-se especialmente a coloquialismos, provérbios, expressões idiomáticas ou palavras de livros antigos | sinal; meio não linguístico de comunicar ideias ; ações ou sinais que substituem palavras para expressar significado | palavras; expressão; refere-se a uma palavra, frase ou sentença}
    \definition{v.}{dizer; falar | (de pássaros, insetos, etc.) gorjear; pipilar}
  \end{phonetics}
  \begin{phonetics}{语}{yu4}
    \definition{v.}{contar; informar}
  \end{phonetics}
\end{entry}

\begin{entry}{语气}{9,4}{⾔、⽓}
  \begin{phonetics}{语气}{yu3qi4}
    \definition[个]{s.}{maneira de falar | tom}
  \end{phonetics}
\end{entry}

\begin{entry}{语言}{9,7}{⾔、⾔}
  \begin{phonetics}{语言}{yu3yan2}[][HSK 2]
    \definition[门,种]{s.}{linguagem | língua}
  \end{phonetics}
\end{entry}

\begin{entry}{语言实验室}{9,7,8,10,9}{⾔、⾔、⼧、⾺、⼧}
  \begin{phonetics}{语言实验室}{yu3yan2shi2yan4shi4}
    \definition{s.}{laboratório de línguas}
  \end{phonetics}
\end{entry}

\begin{entry}{语法}{9,8}{⾔、⽔}
  \begin{phonetics}{语法}{yu3fa3}[][HSK 4]
    \definition[个]{s.}{gramática; maneira como o idioma é estruturado, incluindo a formação e as variações de palavras, a organização de frases e sentenças | estudo da gramática; estudo das regras de estrutura linguística}
  \end{phonetics}
\end{entry}

\begin{entry}{语法术语}{9,8,5,9}{⾔、⽔、⽊、⾔}
  \begin{phonetics}{语法术语}{yu3fa3shu4yu3}
    \definition{s.}{termo gramatical}
  \end{phonetics}
\end{entry}

\begin{entry}{语音}{9,9}{⾔、⾳}
  \begin{phonetics}{语音}{yu3 yin1}[][HSK 4]
    \definition{s.}{voz; pronúncia; sons da fala; som de alguém falando | pronúncia; som do idioma}
  \end{phonetics}
\end{entry}

\begin{entry}{语调}{9,10}{⾔、⾔}
  \begin{phonetics}{语调}{yu3diao4}
    \definition[个]{s.}{entonação}
  \end{phonetics}
\end{entry}

\begin{entry}{误会}{9,6}{⾔、⼈}
  \begin{phonetics}{误会}{wu4hui4}
    \definition[场]{s.}{mal-entendido; desentendimentos ou conflitos decorrentes de mal-entendidos}
    \definition{v.}{entender mal; entender errado; interpretar mal; não entender; não entender corretamente o significado}
  \end{phonetics}
\end{entry}

\begin{entry}{误点}{9,9}{⾔、⽕}
  \begin{phonetics}{误点}{wu4dian3}
    \definition{v.+compl.}{atrasar | chegar tarde}
  \end{phonetics}
\end{entry}

\begin{entry}{误解}{9,13}{⾔、⾓}
  \begin{phonetics}{误解}{wu4jie3}[][HSK 5]
    \definition[种]{s.}{equívoco; mal-entendido; desentendimento}
    \definition{v.}{interpretar mal; interpretar erroneamente; não compreender corretamente}
  \end{phonetics}
\end{entry}

\begin{entry}{诱人}{9,2}{⾔、⼈}
  \begin{phonetics}{诱人}{you4ren2}
    \definition{adj.}{atraente | cativante}
  \end{phonetics}
\end{entry}

\begin{entry}{说}{9}{⾔}
  \begin{phonetics}{说}{shui4}
    \definition{v.}{persuadir}
  \end{phonetics}
  \begin{phonetics}{说}{shuo1}[][HSK 1]
    \definition{s.}{uma teoria (normalmente o último caractere, como em 日心说, teoria heliocêntrica)}
    \definition{v.}{falar | dizer | explicar | contar}
  \end{phonetics}
\end{entry}

\begin{entry}{说不定}{9,4,8}{⾔、⼀、⼧}
  \begin{phonetics}{说不定}{shuo1bu5ding4}[][HSK 4]
    \definition{adv.}{talvez; indica uma estimativa, possivelmente, provavelmente}
    \definition{v.}{não ter certeza; não estar certo; ser impreciso}
  \end{phonetics}
\end{entry}

\begin{entry}{说好}{9,6}{⾔、⼥}
  \begin{phonetics}{说好}{shuo1hao3}
    \definition{v.}{chegar a um acordo | concluir negociações}
  \end{phonetics}
\end{entry}

\begin{entry}{说完}{9,7}{⾔、⼧}
  \begin{phonetics}{说完}{shuo1-wan2}
    \definition{expr.}{acabar/terminar palavras}
  \end{phonetics}
\end{entry}

\begin{entry}{说明}{9,8}{⾔、⽇}
  \begin{phonetics}{说明}{shuo1ming2}[][HSK 2]
    \definition[本,个]{s.}{legenda | instrução | explicação}
    \definition{v.}{mostrar | explicar | ilustrar | indicar | provar | demonstrar}
  \end{phonetics}
\end{entry}

\begin{entry}{说服}{9,8}{⾔、⽉}
  \begin{phonetics}{说服}{shuo1fu2}[][HSK 4]
    \definition{v.}{persuadir; convencer; convencer a outra parte com palavras bem fundamentadas}
  \end{phonetics}
\end{entry}

\begin{entry}{说法}{9,8}{⾔、⽔}
  \begin{phonetics}{说法}{shuo1 fa3}[][HSK 5]
    \definition[种]{s.}{maneira de dizer uma coisa; palavras ou frases usadas para expressar significado | declaração; versão; argumento; opinião; ponto de vista | motivo; razão; motivos ou bases para a resolução do problema}
  \end{phonetics}
\end{entry}

\begin{entry}{说话}{9,8}{⾔、⾔}
  \begin{phonetics}{说话}{shuo1 hua4}[][HSK 1]
    \definition{adv.}{imediatamente | em um minuto}
    \definition{v.}{falar | dizer | bater-papo | conversar | fofocar}
  \end{phonetics}
\end{entry}

\begin{entry}{说理}{9,11}{⾔、⽟}
  \begin{phonetics}{说理}{shuo1li3}
    \definition{v.}{racionalizar | discutir logicamente}
  \end{phonetics}
\end{entry}

\begin{entry}{说谎}{9,11}{⾔、⾔}
  \begin{phonetics}{说谎}{shuo1huang3}
    \definition{v.+compl.}{mentir | contar uma mentira}
  \end{phonetics}
\end{entry}

\begin{entry}{贴}{9}{⾙}
  \begin{phonetics}{贴}{tie1}[][HSK 4]
    \definition{adj.}{submisso; obediente}
    \definition{clas.}{para uso em gessos, emplastros}
    \definition{s.}{subsídio; subvenção}
    \definition{v.}{grudar; colar | aninhar-se a; aconchegar-se a | subsidiar; ajudar financeiramente}
  \end{phonetics}
\end{entry}

\begin{entry}{贵}{9}{⾙}
  \begin{phonetics}{贵}[⻉]{gui4}[中一⻉][HSK 1]
    \definition{adj.}{caro | nobre | precioso}
  \end{phonetics}
\end{entry}

\begin{entry}{贵姓}{9,8}{⾙、⼥}
  \begin{phonetics}{贵姓}{gui4xing4}
    \definition{expr.}{qual seu sobrenome?}
  \end{phonetics}
\end{entry}

\begin{entry}{贷款}{9,12}{⾙、⽋}
  \begin{phonetics}{贷款}{dai4kuan3}[][HSK 5]
    \definition[个,笔]{s.}{empréstimo; crédito;}
    \definition{v.}{fornecer um empréstimo; conceder um empréstimo; conceder crédito a; emprestar dinheiro para quem precisa}
  \end{phonetics}
\end{entry}

\begin{entry}{贸易}{9,8}{⾙、⽇}
  \begin{phonetics}{贸易}{mao4yi4}[][HSK 5]
    \definition[笔,宗,项]{s.}{comércio; troca; negócios; refere-se a atividades comerciais, como a troca de mercadorias}
    \definition{v.}{fazer uma transação comercial}
  \end{phonetics}
\end{entry}

\begin{entry}{费}{9}{⾙}
  \begin{phonetics}{费}{fei4}[][HSK 3]
    \definition*{s.}{Fei}
    \definition{s.}{taxa; despesa; encargo}
    \definition{v.}{custar; gastar; desperdiçar}
  \end{phonetics}
\end{entry}

\begin{entry}{费用}{9,5}{⾙、⽤}
  \begin{phonetics}{费用}{fei4 yong4}[][HSK 3]
    \definition[笔,个]{s.}{custo; despesa; desembolso}
  \end{phonetics}
\end{entry}

\begin{entry}{贺}{9}{⾙}
  \begin{phonetics}{贺}{he4}
    \definition*{s.}{sobrenome He}
    \definition{v.}{parabenizar | congratular}
  \end{phonetics}
\end{entry}

\begin{entry}{贺卡}{9,5}{⾙、⼘}
  \begin{phonetics}{贺卡}{he4 ka3}[][HSK 5]
    \definition[张]{s.}{cartão de felicitações; pedaço de papel para parabenizar amigos e parentes em seu casamento, aniversário ou festivais, geralmente impresso com palavras e desenhos de felicitações}
  \end{phonetics}
\end{entry}

\begin{entry}{轴承}{9,8}{⾞、⼿}
  \begin{phonetics}{轴承}{zhou2cheng2}
    \definition{s.}{(mecânico) rolamento}
  \end{phonetics}
\end{entry}

\begin{entry}{轻}{9}{⾞}
  \begin{phonetics}{轻}{qing1}[][HSK 2]
    \definition{adj.}{leve | pequeno em número, grau, etc. | não importante | relaxado}
    \definition{adv.}{suavemente | levemente | precipitadamente}
    \definition{v.}{menosprezar}
  \end{phonetics}
\end{entry}

\begin{entry}{轻易}{9,8}{⾞、⽇}
  \begin{phonetics}{轻易}{qing1yi4}[][HSK 4]
    \definition{adj.}{fácil; simples}
    \definition{adv.}{facilmente; prontamente | facilmente; precipitadamente; indica que uma ação é realizada casualmente, geralmente usado em frases negativas}
  \end{phonetics}
\end{entry}

\begin{entry}{轻松}{9,8}{⾞、⽊}
  \begin{phonetics}{轻松}{qing1song1}[][HSK 4]
    \definition{adj.}{leve; relaxado; livre de fardos; não se sentir nervoso ou cansado}
    \definition{v.}{relaxar; levar as coisas menos a sério}
  \end{phonetics}
\end{entry}

\begin{entry}{迷}{9}{⾡}
  \begin{phonetics}{迷}{mi2}[][HSK 3]
    \definition*{s.}{sobrenome Mi}
    \definition{adj.}{perdido; confuso}
    \definition{s.}{fã; entusiasta; fanático}
    \definition{v.}{estar confuso; perder o rumo; se perder-se | ficar fascinado por; entregar-se a; ficar encantado com (por); ser louco por | confundir; desorientar; fascinar; encantar}
  \end{phonetics}
\end{entry}

\begin{entry}{迷人}{9,2}{⾡、⼈}
  \begin{phonetics}{迷人}{mi2ren2}[][HSK 5]
    \definition{adj.}{encantador; fascinante; sedutor; hipnotizante}
    \definition{v.}{confundir; intrigar; enganar}
  \end{phonetics}
\end{entry}

\begin{entry}{迷你}{9,7}{⾡、⼈}
  \begin{phonetics}{迷你}{mi2ni3}
    \definition{adj.}{(empréstimo linguístico) mini, como em minissaia ou \emph{Mini Cooper}}
  \end{phonetics}
\end{entry}

\begin{entry}{迷信}{9,9}{⾡、⼈}
  \begin{phonetics}{迷信}{mi2xin4}[][HSK 5]
    \definition{s.}{superstição; crença supersticiosa; fé cega; adoração cega; crença em deuses, espíritos e fantasmas}
    \definition{v.}{ter fé cega em; fazer um fetiche de}
  \end{phonetics}
\end{entry}

\begin{entry}{迷宫}{9,9}{⾡、⼧}
  \begin{phonetics}{迷宫}{mi2gong1}
    \definition{s.}{labirinto}
  \end{phonetics}
\end{entry}

\begin{entry}{迷恋}{9,10}{⾡、⼼}
  \begin{phonetics}{迷恋}{mi2lian4}
    \definition{adj.}{obcecado}
    \definition{v.}{estar/ser apaixonado por | ficar encantado por | estar/ser obcecado por}
  \end{phonetics}
\end{entry}

\begin{entry}{迷路}{9,13}{⾡、⾜}
  \begin{phonetics}{迷路}{mi2lu4}
    \definition{s.}{labirinto | ouvido interno}
    \definition{v.+compl.}{perder o caminho | perder-se | seguir pelo caminho errado | não conseguir encontrar o caminho}
  \end{phonetics}
\end{entry}

\begin{entry}{追}{9}{⾡}
  \begin{phonetics}{追}{zhui1}[][HSK 3]
    \definition*{s.}{sobrenome Zhui}
    \definition{v.}{perseguir; correr atrás; ir atrás de; alcançar | rastrear; investigar; chegar ao fundo de | ansiar por (depois); ir atrás; procurar | recordar; relembrar; lembrar | agir retroativamente; fazer postumamente}
  \end{phonetics}
\end{entry}

\begin{entry}{追求}{9,7}{⾡、⽔}
  \begin{phonetics}{追求}{zhui1qiu2}[][HSK 4]
    \definition{s.}{perseguição (ações e metas positivas)}[她的追求是获得成功。(Sua meta é alcançar o sucesso.)]
    \definition{v.}{buscar; aspirar; perseguir | cortejar, uma referência especial ao namoro}
  \end{phonetics}
\end{entry}

\begin{entry}{追赶}{9,10}{⾡、⾛}
  \begin{phonetics}{追赶}{zhui1gan3}
    \definition{v.}{perseguir | acelerar | alcançar | ultrapassar}
  \end{phonetics}
\end{entry}

\begin{entry}{退}{9}{⾡}
  \begin{phonetics}{退}{tui4}[][HSK 3]
    \definition{v.}{recuar; mover-se para trás | fazer recuar; remover; retirar | desistir; retirar-se de | retroceder; refluir; declinar | desaparecer; desvanecer | devolver; retornar | cancelar; rescindir; romper}
  \end{phonetics}
\end{entry}

\begin{entry}{退出}{9,5}{⾡、⼐}
  \begin{phonetics}{退出}{tui4 chu1}[][HSK 3]
    \definition{v.}{desistir; retirar-se; separar-se; retirar-se de}
  \end{phonetics}
\end{entry}

\begin{entry}{退休}{9,6}{⾡、⼈}
  \begin{phonetics}{退休}{tui4xiu1}[][HSK 3]
    \definition{v.+compl.}{aposentar-se}
  \end{phonetics}
\end{entry}

\begin{entry}{送}{9}{⾡}
  \begin{phonetics}{送}{song4}[][HSK 1]
    \definition{v.}{distribuir | entregar | dar | oferecer (alguma coisa como presente) | enviar | remeter}
  \end{phonetics}
\end{entry}

\begin{entry}{送到}{9,8}{⾡、⼑}
  \begin{phonetics}{送到}{song4 dao4}[][HSK 2]
    \definition{v.}{enviar para (lugar)}
  \end{phonetics}
\end{entry}

\begin{entry}{送给}{9,9}{⾡、⽷}
  \begin{phonetics}{送给}{song4 gei3}[][HSK 2]
    \definition{v.}{dar a (alguém ou organização)}
  \end{phonetics}
\end{entry}

\begin{entry}{适用}{9,5}{⾡、⽤}
  \begin{phonetics}{适用}{shi4 yong4}[][HSK 3]
    \definition{adj.}{adequado; aplicável}
    \definition{v.}{ser aplicável; ser adequado}
  \end{phonetics}
\end{entry}

\begin{entry}{适合}{9,6}{⾡、⼝}
  \begin{phonetics}{适合}{shi4he2}[][HSK 3]
    \definition{v.}{servir (uma roupa); caber; se adequar}
  \end{phonetics}
\end{entry}

\begin{entry}{适应}{9,7}{⾡、⼴}
  \begin{phonetics}{适应}{shi4ying4}[][HSK 3]
    \definition{v.}{ajustar-se; adequar-se; adaptar-se}
  \end{phonetics}
\end{entry}

\begin{entry}{逃}{9}{⾡}
  \begin{phonetics}{逃}{tao2}[][HSK 5]
    \definition{v.}{fugir; escapar; correr; dar no pé | evadir; esquivar-se; escapar}
  \end{phonetics}
\end{entry}

\begin{entry}{逃走}{9,7}{⾡、⾛}
  \begin{phonetics}{逃走}{tao2 zou3}[][HSK 5]
    \definition{v.}{escapar; afastar-se de pessoas, coisas ou lugares que não são bons para você ou que você não gosta}
  \end{phonetics}
\end{entry}

\begin{entry}{逃跑}{9,12}{⾡、⾜}
  \begin{phonetics}{逃跑}{tao2 pao3}[][HSK 5]
    \definition{v.}{fugir; escapar; correr; partir para fugir de um ambiente ou de coisas que não lhe são favoráveis}
  \end{phonetics}
\end{entry}

\begin{entry}{逆境}{9,14}{⾡、⼟}
  \begin{phonetics}{逆境}{ni4jing4}
    \definition{s.}{adversidade | tribulação}
  \end{phonetics}
\end{entry}

\begin{entry}{选}{9}{⾡}
  \begin{phonetics}{选}{xuan3}[][HSK 2]
    \definition{s.}{seleções | antologia}
    \definition{v.}{selecionar | escolher | eleger}
  \end{phonetics}
\end{entry}

\begin{entry}{选手}{9,4}{⾡、⼿}
  \begin{phonetics}{选手}{xuan3shou3}[][HSK 3]
    \definition[位]{s.}{jogador; competidor (selecionado); atleta selecionado para uma competição esportiva}
  \end{phonetics}
\end{entry}

\begin{entry}{选择}{9,8}{⾡、⼿}
  \begin{phonetics}{选择}{xuan3ze2}[][HSK 4]
    \definition[个,种,次]{s.}{escolha; opção; resultado da escolha; possibilidade de escolha}
    \definition{v.}{selecionar; escolher}
  \end{phonetics}
\end{entry}

\begin{entry}{选修}{9,9}{⾡、⼈}
  \begin{phonetics}{选修}{xuan3 xiu1}[][HSK 5]
    \definition{v.}{fazer um curso eletivo; selecionar os cursos a serem estudados entre os cursos disponíveis}
  \end{phonetics}
\end{entry}

\begin{entry}{重}{9}{⾥}
  \begin{phonetics}{重}{chong2}
    \definition*{s.}{sobrenome Chong}
    \definition{adv.}{novamente; mais uma vez}
    \definition{clas.}{para camadas}
    \definition{v.}{repetir; duplicar}
  \end{phonetics}
  \begin{phonetics}{重}{zhong4}[][HSK 1,3]
    \definition{adj.}{pesado | profundo; sério | importante; momentoso | discreto; prudente | considerável em quantidade ou valor}
    \definition{adv.}{pesadamente; severamente}
    \definition{v.}{colocar (pôr) ênfase em; dar valor a; atribuir importância a}
  \end{phonetics}
\end{entry}

\begin{entry}{重大}{9,3}{⾥、⼤}
  \begin{phonetics}{重大}{zhong4da4}[][HSK 3]
    \definition{adj.}{grande; importante; significativo; de grande importância}
  \end{phonetics}
\end{entry}

\begin{entry}{重阳节}{9,6,5}{⾥、⾩、⾋}
  \begin{phonetics}{重阳节}{chong2yang2jie2}
    \definition*{s.}{Festa do Duplo Nove, Festival Yang, dia de subir aos lugares mais altos para evitar calamidades e dia do culto aos antepassados (9º dia do nono mês lunar)}
  \end{phonetics}
\end{entry}

\begin{entry}{重视}{9,8}{⾥、⾒}
  \begin{phonetics}{重视}{zhong4shi4}[][HSK 2]
    \definition{v.}{atribuir valor a | dar peso a | atribuir importância a | prestar atenção a}
  \end{phonetics}
\end{entry}

\begin{entry}{重复}{9,9}{⾥、⼢}
  \begin{phonetics}{重复}{chong2fu4}[][HSK 2]
    \definition{v.}{repetir | iterar | duplicar | reduplicar | fazer algo de novo}
  \end{phonetics}
\end{entry}

\begin{entry}{重点}{9,9}{⾥、⽕}
  \begin{phonetics}{重点}{chong2dian3}
    \definition{adj./adv./s.}{nota-chave | ponto-chave | ponto focal | ênfase}
  \end{phonetics}
  \begin{phonetics}{重点}{zhong4dian3}[][HSK 2]
    \definition{s.}{nota-chave | ponto-chave | ponto focal | ênfase}
  \end{phonetics}
\end{entry}

\begin{entry}{重要}{9,9}{⾥、⾑}
  \begin{phonetics}{重要}{zhong4yao4}[][HSK 1]
    \definition{adj.}{importante | significativo | principal}
  \end{phonetics}
\end{entry}

\begin{entry}{重重}{9,9}{⾥、⾥}
  \begin{phonetics}{重重}{chong2chong2}
    \definition{adv.}{camada após camada | um após o outro}
  \end{phonetics}
  \begin{phonetics}{重重}{zhong4zhong4}
    \definition{adv.}{fortemente | severamente}
  \end{phonetics}
\end{entry}

\begin{entry}{重逢}{9,10}{⾥、⾡}
  \begin{phonetics}{重逢}{chong2feng2}
    \definition{s.}{reunião}
    \definition{v.}{encontrar-se novamente | reunir-se}
  \end{phonetics}
\end{entry}

\begin{entry}{重量}{9,12}{⾥、⾥}
  \begin{phonetics}{重量}{zhong4liang4}[][HSK 4]
    \definition[个]{s.}{peso; a magnitude da força da gravidade em um objeto}
  \end{phonetics}
\end{entry}

\begin{entry}{重新}{9,13}{⾥、⽄}
  \begin{phonetics}{重新}{chong2xin1}[][HSK 2]
    \definition{adv.}{de novo | novamente}
  \end{phonetics}
\end{entry}

\begin{entry}{钟}{9}{⾦}
  \begin{phonetics}{钟}{zhong1}[][HSK 3]
    \definition*{s.}{sobrenome Zhong}
    \definition[顶,个,口]{s.}{sino; campainha; um chocalho feito de cobre ou ferro | relógio; temporizador | tempo; refere-se à hora, ao tempo | copo sem alça; xícara sem alça}
    \definition{v.}{concentrar (as afeições de alguém, etc.)}
  \end{phonetics}
\end{entry}

\begin{entry}{钟室}{9,9}{⾦、⼧}
  \begin{phonetics}{钟室}{zhong1shi4}
    \definition{s.}{campanário | sala do relógio}
  \end{phonetics}
\end{entry}

\begin{entry}{钟罩}{9,13}{⾦、⽹}
  \begin{phonetics}{钟罩}{zhong1zhao4}
    \definition{s.}{redoma | dossel de sino}
  \end{phonetics}
\end{entry}

\begin{entry}{钢}{9}{⾦}
  \begin{phonetics}{钢}{gang1}
    \definition{s.}{aço}
  \end{phonetics}
\end{entry}

\begin{entry}{钢丝}{9,5}{⾦、⼀}
  \begin{phonetics}{钢丝}{gang1si1}
    \definition{s.}{cabo de aço | corda bamba}
  \end{phonetics}
\end{entry}

\begin{entry}{钢笔}{9,10}{⾦、⽵}
  \begin{phonetics}{钢笔}{gang1 bi3}[][HSK 5]
    \definition[支]{s.}{caneta-tinteiro; canetas com ponta metálica}
  \end{phonetics}
\end{entry}

\begin{entry}{钢琴}{9,12}{⾦、⽟}
  \begin{phonetics}{钢琴}{gang1qin2}[][HSK 5]
    \definition[架]{s.}{piano}
  \end{phonetics}
\end{entry}

\begin{entry}{钥匙}{9,11}{⾦、⼔}
  \begin{phonetics}{钥匙}{yao4shi5}
    \definition[把]{s.}{chave}
  \end{phonetics}
\end{entry}

\begin{entry}{钥匙孔}{9,11,4}{⾦、⼔、⼦}
  \begin{phonetics}{钥匙孔}{yao4shi5kong3}
    \definition{s.}{buraco da fechadura}
  \end{phonetics}
\end{entry}

\begin{entry}{钥匙卡}{9,11,5}{⾦、⼔、⼘}
  \begin{phonetics}{钥匙卡}{yao4shi5ka3}
    \definition{s.}{cartão de acesso}
  \end{phonetics}
\end{entry}

\begin{entry}{钥匙洞孔}{9,11,9,4}{⾦、⼔、⽔、⼦}
  \begin{phonetics}{钥匙洞孔}{yao4shi5dong4kong3}
    \definition{s.}{buraco da fechadura}
  \end{phonetics}
\end{entry}

\begin{entry}{钥匙圈}{9,11,11}{⾦、⼔、⼞}
  \begin{phonetics}{钥匙圈}{yao4shi5quan1}
    \definition{s.}{chaveiro}
  \end{phonetics}
\end{entry}

\begin{entry}{钩}{9}{⾦}
  \begin{phonetics}{钩}{gou1}
    \definition{s.}{gancho | \emph{check mark} | \emph{tick}}
    \definition{v.}{enganchar | costurar}
  \end{phonetics}
\end{entry}

\begin{entry}{闻}{9}{⾨}
  \begin{phonetics}{闻}{wen2}[][HSK 2]
    \definition*{s.}{sobrenome Wen}
    \definition{s.}{notícias | reputação | fama}
    \definition{v.}{ouvir | cheirar | farejar}
  \end{phonetics}
\end{entry}

\begin{entry}{阁下}{9,3}{⾨、⼀}
  \begin{phonetics}{阁下}{ge2xia4}
    \definition{pron.}{Sua Excelência | Sua Majestade | \emph{Sire}}
  \end{phonetics}
\end{entry}

\begin{entry}{院}{9}{⾩}
  \begin{phonetics}{院}{yuan4}[][HSK 2]
    \definition[个]{s.}{pátio | instituição}
  \end{phonetics}
\end{entry}

\begin{entry}{院子}{9,3}{⾩、⼦}
  \begin{phonetics}{院子}{yuan4zi5}[][HSK 2]
    \definition[个]{s.}{pátio | jardim | quintal}
  \end{phonetics}
\end{entry}

\begin{entry}{院长}{9,4}{⾩、⾧}
  \begin{phonetics}{院长}{yuan4zhang3}[][HSK 2]
    \definition[个]{s.}{presidente de um conselho | reitor | chefe de departamento | primeiro-ministro da República da China | presidente de uma universidade}
  \end{phonetics}
\end{entry}

\begin{entry}{除了}{9,2}{⾩、⼅}
  \begin{phonetics}{除了}{chu2le5}[][HSK 3]
    \definition{prep.}{exceto; à parte | além disso; além de | ou \dots ou \dots}
  \end{phonetics}
\end{entry}

\begin{entry}{除夕}{9,3}{⾩、⼣}
  \begin{phonetics}{除夕}{chu2xi1}[][HSK 5]
    \definition*{s.}{Véspera de Ano Novo Lunar; a noite do último dia do ano, também se refere ao último dia do ano}
  \end{phonetics}
\end{entry}

\begin{entry}{除非}{9,8}{⾩、⾮}
  \begin{phonetics}{除非}{chu2fei1}[][HSK 5]
    \definition{conj.}{a menos que; somente se; indica a única condição, equivalente a ``只有'', frequentemente combinada com ``才、否则、不然'', etc.}
  \seealsoref{不然}{bu4ran2}
  \seealsoref{才}{cai2}
  \seealsoref{否则}{fou3ze2}
  \seealsoref{只有}{zhi3 you3}
  \end{phonetics}
\end{entry}

\begin{entry}{面}{9}{⾯}[Kangxi 176]
  \begin{phonetics}{面}{mian4}[][HSK 2]
    \definition{clas.}{para objetos com superfície plana como tambores, espelhos, bandeiras, etc.}
    \definition{s.}{farinha | massa | (gíria) (uma pessoa) ineficaz | face | superfície | lado | lado de fora}
  \end{phonetics}
\end{entry}

\begin{entry}{面子}{9,3}{⾯、⼦}
  \begin{phonetics}{面子}{mian4zi5}[][HSK 5]
    \definition{s.}{face; exterior; parte externa; superfície do objeto | imagem; reputação; prestígio; decência; vaidade superficial | sentimentos; sensibilidades | pó}
  \end{phonetics}
\end{entry}

\begin{entry}{面包}{9,5}{⾯、⼓}
  \begin{phonetics}{面包}{mian4bao1}[][HSK 1]
    \definition[个,片,袋,块]{s.}{pão}[我买八个面包了。(Comprei oito pães.) | 他在吃两片面包。(Ele está comendo duas fatias de pão.) | 我在家里带了一袋面包。(Trouxe um saco de pão para casa.) | 我拿了一块面包。(Peguei um pedaço de pão.)]
  \end{phonetics}
\end{entry}

\begin{entry}{面对}{9,5}{⾯、⼨}
  \begin{phonetics}{面对}{mian4dui4}[][HSK 3]
    \definition{v.}{enfrentar; defrontar | confrontar (problema)}
  \end{phonetics}
\end{entry}

\begin{entry}{面对面}{9,5,9}{⾯、⼨、⾯}
  \begin{phonetics}{面对面}{mian4dui4mian4}
    \definition{expr.}{cara a cara}
  \end{phonetics}
\end{entry}

\begin{entry}{面对面吃面}{9,5,9,6,9}{⾯、⼨、⾯、⼝、⾯}
  \begin{phonetics}{面对面吃面}{mian4dui4mian4 chi1 mian4}
    \definition{expr.}{Comer macarrão cara a cara; indica que o seu estado atual, ou algumas das posições em que você está, ou algumas das coisas que você fez são muito claras}
  \end{phonetics}
\end{entry}

\begin{entry}{面团}{9,6}{⾯、⼞}
  \begin{phonetics}{面团}{mian4tuan2}
    \definition{s.}{massa | pasta}
  \end{phonetics}
\end{entry}

\begin{entry}{面条}{9,7}{⾯、⽊}
  \begin{phonetics}{面条}{mian4tiao2}
    \definition{s.}{macarrão | espaguete}
  \end{phonetics}
\end{entry}

\begin{entry}{面条儿}{9,7,2}{⾯、⽊、⼉}
  \begin{phonetics}{面条儿}{mian4tiao2r5}[][HSK 1]
    \definition{s.}{macarrão | \emph{noodles}}
  \end{phonetics}
\end{entry}

\begin{entry}{面试}{9,8}{⾯、⾔}
  \begin{phonetics}{面试}{mian4 shi4}[][HSK 4]
    \definition[次]{s.}{entrevista; audição}
  \end{phonetics}
\end{entry}

\begin{entry}{面临}{9,9}{⾯、⼁}
  \begin{phonetics}{面临}{mian4lin2}[][HSK 4]
    \definition{v.}{ser confrontado com; encontrar (uma situação) na frente de}
  \end{phonetics}
\end{entry}

\begin{entry}{面前}{9,9}{⾯、⼑}
  \begin{phonetics}{面前}{mian4 qian2}[][HSK 2]
    \definition{adv.}{antes | na frente de | na (frente) de}
  \end{phonetics}
\end{entry}

\begin{entry}{面积}{9,10}{⾯、⽲}
  \begin{phonetics}{面积}{mian4ji1}[][HSK 3]
    \definition{s.}{área (de um andar, pedaço de terreno, etc.); área de uma superfície}
  \end{phonetics}
\end{entry}

\begin{entry}{面貌}{9,14}{⾯、⾘}
  \begin{phonetics}{面貌}{mian4mao4}[][HSK 5]
    \definition[种,个]{s.}{rosto; traços faciais; formato do rosto; aparência |
aparência; aspecto; aparência (das coisas) |}
  \end{phonetics}
\end{entry}

\begin{entry}{韭菜}{9,11}{⾲、⾋}
  \begin{phonetics}{韭菜}{jiu3cai4}
    \definition{s.}{cebolinha chinesa | (figurativo) investidores de varejo que perdem seu dinheiro para operadores mais experientes (ou seja, são ``colhidos'' como cebolinhas)}
  \end{phonetics}
\end{entry}

\begin{entry}{音乐}{9,5}{⾳、⼃}
  \begin{phonetics}{音乐}{yin1yue4}[][HSK 2]
    \definition[张,曲,段]{s.}{música}
  \end{phonetics}
\end{entry}

\begin{entry}{音乐厅}{9,5,4}{⾳、⼃、⼚}
  \begin{phonetics}{音乐厅}{yin1yue4ting1}
    \definition{s.}{auditório | teatro | \emph{concert hall}}
  \end{phonetics}
\end{entry}

\begin{entry}{音乐节}{9,5,5}{⾳、⼃、⾋}
  \begin{phonetics}{音乐节}{yin1yue4jie2}
    \definition{s.}{festival de música}
  \end{phonetics}
\end{entry}

\begin{entry}{音乐会}{9,5,6}{⾳、⼃、⼈}
  \begin{phonetics}{音乐会}{yin1 yue4 hui4}[][HSK 2]
    \definition[场]{s.}{concerto}
  \end{phonetics}
\end{entry}

\begin{entry}{音乐光碟}{9,5,6,14}{⾳、⼃、⼉、⽯}
  \begin{phonetics}{音乐光碟}{yin1yue4guang1die2}
    \definition{s.}{CD de música}
  \end{phonetics}
\end{entry}

\begin{entry}{音乐学}{9,5,8}{⾳、⼃、⼦}
  \begin{phonetics}{音乐学}{yin1yue4xue2}
    \definition{s.}{musicologia}
  \end{phonetics}
\end{entry}

\begin{entry}{音乐学院}{9,5,8,9}{⾳、⼃、⼦、⾩}
  \begin{phonetics}{音乐学院}{yin1yue4xue2yuan4}
    \definition{s.}{conservatório | academia de música}
  \end{phonetics}
\end{entry}

\begin{entry}{音乐院}{9,5,9}{⾳、⼃、⾩}
  \begin{phonetics}{音乐院}{yin1yue4yuan4}
    \definition{s.}{conservatório | instituto de música}
  \end{phonetics}
\end{entry}

\begin{entry}{音乐家}{9,5,10}{⾳、⼃、⼧}
  \begin{phonetics}{音乐家}{yin1yue4jia1}
    \definition{s.}{músico}
  \end{phonetics}
\end{entry}

\begin{entry}{音节}{9,5}{⾳、⾋}
  \begin{phonetics}{音节}{yin1 jie2}[][HSK 2]
    \definition{s.}{sílaba}
  \end{phonetics}
\end{entry}

\begin{entry}{项}{9}{⾴}
  \begin{phonetics}{项}{xiang4}[][HSK 4]
    \definition*{s.}{sobrenome Xiang}
    \definition{clas.}{para itens discriminados; taxonomia}
    \definition{s.}{nuca (do pescoço); a parte de trás do pescoço |
soma (de dinheiro); fundos para fins especiais |
termo; em álgebra, significa uma única fórmula que não é unida por um sinal de mais ou de menos | item}
  \end{phonetics}
\end{entry}

\begin{entry}{项目}{9,5}{⾴、⽬}
  \begin{phonetics}{项目}{xiang4mu4}[][HSK 4]
    \definition{s.}{evento | item; projeto; trabalhos de engenharia, acadêmicos, etc., de conteúdo específico}
  \end{phonetics}
\end{entry}

\begin{entry}{顺}{9}{⾴}
  \begin{phonetics}{顺}{shun4}
    \definition{adj.}{correr bem | favorável}
  \end{phonetics}
\end{entry}

\begin{entry}{顺从}{9,4}{⾴、⼈}
  \begin{phonetics}{顺从}{shun4cong2}
    \definition{v.}{obedecer | submeter-se}
  \end{phonetics}
\end{entry}

\begin{entry}{顺心}{9,4}{⾴、⼼}
  \begin{phonetics}{顺心}{shun4xin1}
    \definition{adj.}{satisfatório | satisfeito}
  \end{phonetics}
\end{entry}

\begin{entry}{顺水}{9,4}{⾴、⽔}
  \begin{phonetics}{顺水}{shun4shui3}
    \definition{v.}{ir com o fluxo}
  \end{phonetics}
\end{entry}

\begin{entry}{顺延}{9,6}{⾴、⼵}
  \begin{phonetics}{顺延}{shun4yan2}
    \definition{v.}{adiar | procrastinar}
  \end{phonetics}
\end{entry}

\begin{entry}{顺当}{9,6}{⾴、⼹}
  \begin{phonetics}{顺当}{shun4dang5}
    \definition{adv.}{suavemente}
  \end{phonetics}
\end{entry}

\begin{entry}{顺耳}{9,6}{⾴、⽿}
  \begin{phonetics}{顺耳}{shun4'er3}
    \definition{adj.}{agradável ao ouvido}
  \end{phonetics}
\end{entry}

\begin{entry}{顺利}{9,7}{⾴、⼑}
  \begin{phonetics}{顺利}{shun4li4}[][HSK 2]
    \definition{adv.}{suavemente | sem problemas}
  \end{phonetics}
\end{entry}

\begin{entry}{顺序}{9,7}{⾴、⼴}
  \begin{phonetics}{顺序}{shun4xu4}[][HSK 4]
    \definition{adv.}{por sua vez; na ordem correta; na devida ordem; na ordem adequada; na ordem apropriada}
    \definition[个]{s.}{ordem; sequência; sucessão; subsequência; sequência simples; ordem de prioridade}
  \end{phonetics}
\end{entry}

\begin{entry}{顺畅}{9,8}{⾴、⽥}
  \begin{phonetics}{顺畅}{shun4chang4}
    \definition{adj.}{liso e sem obstáculos | fluente}
  \end{phonetics}
\end{entry}

\begin{entry}{顺便}{9,9}{⾴、⼈}
  \begin{phonetics}{顺便}{shun4bian4}
    \definition{adv.}{convenientemente | de passagem | sem muito esforço extra}
  \end{phonetics}
\end{entry}

\begin{entry}{顺叙}{9,9}{⾴、⼜}
  \begin{phonetics}{顺叙}{shun4xu4}
    \definition{s.}{narrativa cronológica}
  \end{phonetics}
\end{entry}

\begin{entry}{顺眼}{9,11}{⾴、⽬}
  \begin{phonetics}{顺眼}{shun4yan3}
    \definition{adj.}{agradável aos olhos}
  \end{phonetics}
\end{entry}

\begin{entry}{顺境}{9,14}{⾴、⼟}
  \begin{phonetics}{顺境}{shun4jing4}
    \definition{s.}{circunstâncias favoráveis}
  \end{phonetics}
\end{entry}

\begin{entry}{顺嘴}{9,16}{⾴、⼝}
  \begin{phonetics}{顺嘴}{shun4zui3}
    \definition{v.}{deixar escapar (sem pensar) | ler suavemente (texto) | adequar-se  ao gosto (comida)}
  \end{phonetics}
\end{entry}

\begin{entry}{飒飒}{9,9}{⾵、⾵}
  \begin{phonetics}{飒飒}{sa4sa4}
    \definition{s.}{o farfalhar | sussurro | murmúrio (do vento nas árvores, o mar, etc.)}
  \end{phonetics}
\end{entry}

\begin{entry}{食物}{9,8}{⾷、⽜}
  \begin{phonetics}{食物}{shi2wu4}[][HSK 2]
    \definition[种]{s.}{comida}
  \end{phonetics}
\end{entry}

\begin{entry}{食品}{9,9}{⾷、⼝}
  \begin{phonetics}{食品}{shi2 pin3}[][HSK 3]
    \definition[种]{s.}{comida; gêneros alimentícios; provisões}
  \end{phonetics}
\end{entry}

\begin{entry}{食堂}{9,11}{⾷、⼟}
  \begin{phonetics}{食堂}{shi2 tang2}[][HSK 4]
    \definition[个,间]{s.}{cantina; refeitório}
  \end{phonetics}
\end{entry}

\begin{entry}{饺子}{9,3}{⾷、⼦}
  \begin{phonetics}{饺子}{jiao3zi5}[][HSK 2]
    \definition[个,只]{s.}{jiaozi | bolinhos chineses | bolinho de massa}
  \end{phonetics}
\end{entry}

\begin{entry}{饼}{9}{⾷}
  \begin{phonetics}{饼}{bing3}[][HSK 5]
    \definition[张]{s.}{um bolo redondo e plano; massa assada ou cozida no vapor | algo que tem o formato de um bolo; semelhante a uma torta}
  \end{phonetics}
\end{entry}

\begin{entry}{饼干}{9,3}{⾷、⼲}
  \begin{phonetics}{饼干}{bing3gan1}[][HSK 5]
    \definition[块,片,包,盒,袋]{s.}{biscoito; bolacha; \emph{cookie}; alimentos, pedaços pequenos e finos cozidos em farinha com açúcar, ovos, leite, etc.}
  \end{phonetics}
\end{entry}

\begin{entry}{首}{9}{⾸}[Kangxi 185]
  \begin{phonetics}{首}{shou3}[][HSK 4]
    \definition*{s.}{sobrenome Shou}
    \definition{adj.}{primeiro}
    \definition{adv.}{inicialmente; como o primeiro; em primeiro lugar}
    \definition{clas.}{para canções e poemas}
    \definition{s.}{cabeça | cabeça; chefe; líder | capital (cidade)}
    \definition{v.}{apresentar acusações contra alguém}
  \end{phonetics}
\end{entry}

\begin{entry}{首先}{9,6}{⾸、⼉}
  \begin{phonetics}{首先}{shou3xian1}[][HSK 3]
    \definition{adv.}{primeiramente; antes de todos os outros}
    \definition{conj.}{acima de tudo; primeiramente; em primeiro lugar}
  \end{phonetics}
\end{entry}

\begin{entry}{首相}{9,9}{⾸、⽬}
  \begin{phonetics}{首相}{shou3xiang4}
    \definition*{s.}{Primeiro-Ministro (Japão, UK, etc.)}
  \end{phonetics}
\end{entry}

\begin{entry}{首席执行官}{9,10,6,6,8}{⾸、⼱、⼿、⾏、⼧}
  \begin{phonetics}{首席执行官}{shou3xi2 zhi2xing2 guan1}
    \definition{s.}{\emph{chief executive officer}, CEO}
  \end{phonetics}
\end{entry}

\begin{entry}{首都}{9,10}{⾸、⾢}
  \begin{phonetics}{首都}{shou3du1}[][HSK 3]
    \definition[个]{s.}{capital (cidade)}
  \end{phonetics}
\end{entry}

\begin{entry}{香}{9}{⾹}[Kangxi 186]
  \begin{phonetics}{香}{xiang1}[][HSK 3]
    \definition*{s.}{sobrenome Xiang}
    \definition{adj.}{aromático; perfumado; fragrante; cheiroso | saboroso; saboroso; delicioso; apetitoso | com gosto; com bom apetite | (sono) profundo | popular; bem-vindo}
    \definition[束,根,炷]{s.}{especiaria; perfume; fragrância; aromatizante | incenso | relacionado a mulheres ou às próprias mulheres}
  \end{phonetics}
\end{entry}

\begin{entry}{香气}{9,4}{⾹、⽓}
  \begin{phonetics}{香气}{xiang1qi4}
    \definition{s.}{fragrância | aroma | incenso}
  \end{phonetics}
\end{entry}

\begin{entry}{香皂}{9,7}{⾹、⽩}
  \begin{phonetics}{香皂}{xiang1zao4}
    \definition{s.}{sabonete | sabonete perfumado}
  \end{phonetics}
\end{entry}

\begin{entry}{香肠}{9,7}{⾹、⾁}
  \begin{phonetics}{香肠}{xiang1chang2}[][HSK 5]
    \definition[根]{s.}{salsicha; linguiça; alimento feito com intestino de porco, recheado com carne picada e temperos}
  \end{phonetics}
\end{entry}

\begin{entry}{香味}{9,8}{⾹、⼝}
  \begin{phonetics}{香味}{xiang1wei4}
    \definition[股]{s.}{fragrância | cheiro doce}
  \end{phonetics}
\end{entry}

\begin{entry}{香波}{9,8}{⾹、⽔}
  \begin{phonetics}{香波}{xiang1bo1}
    \definition{s.}{xampu}
  \end{phonetics}
\end{entry}

\begin{entry}{香炉}{9,8}{⾹、⽕}
  \begin{phonetics}{香炉}{xiang1lu2}
    \definition{s.}{incensário (para queimar incenso) | queimador de incenso | insensório, turíbulo}
  \end{phonetics}
\end{entry}

\begin{entry}{香烟}{9,10}{⾹、⽕}
  \begin{phonetics}{香烟}{xiang1yan1}
    \definition[支,条]{s.}{cigarro | fumaça de incenso queimado}
  \end{phonetics}
\end{entry}

\begin{entry}{香艳}{9,10}{⾹、⾊}
  \begin{phonetics}{香艳}{xiang1yan4}
    \definition{adj.}{atraente | erótico | romântico}
  \end{phonetics}
\end{entry}

\begin{entry}{香港}{9,12}{⾹、⽔}
  \begin{phonetics}{香港}{xiang1gang3}
    \definition*{s.}{Hong Kong}
  \seealsoref{香港岛}{xiang1gang3 dao3}
  \end{phonetics}
\end{entry}

\begin{entry}{香港岛}{9,12,7}{⾹、⽔、⼭}
  \begin{phonetics}{香港岛}{xiang1gang3 dao3}
    \definition*{s.}{Ilha de Hong Kong}
  \seealsoref{香港}{xiang1gang3}
  \end{phonetics}
\end{entry}

\begin{entry}{香槟酒}{9,14,10}{⾹、⽊、⾣}
  \begin{phonetics}{香槟酒}{xiang1bin1jiu3}
    \definition[杯]{s.}{(empréstimo linguístico) \emph{champagne}}
  \end{phonetics}
\end{entry}

\begin{entry}{香蕈}{9,15}{⾹、⾋}
  \begin{phonetics}{香蕈}{xiang1xun4}
    \definition{s.}{\emph{shiitake}, cogumelo comestível}
  \end{phonetics}
\end{entry}

\begin{entry}{香蕉}{9,15}{⾹、⾋}
  \begin{phonetics}{香蕉}{xiang1jiao1}[][HSK 3]
    \definition[枝,根,个,把,串,束,弓]{s.}{banana}
  \end{phonetics}
\end{entry}

\begin{entry}{骂}{9}{⾺}
  \begin{phonetics}{骂}{ma4}[][HSK 5]
    \definition{v.}{abusar; xingar; insultar; insultar alguém com palavras grosseiras ou maliciosas | repreender; censurar; condenar}
  \end{phonetics}
\end{entry}

\begin{entry}{骂名}{9,6}{⾺、⼝}
  \begin{phonetics}{骂名}{ma4ming2}
    \definition{s.}{infâmia}
  \end{phonetics}
\end{entry}

\begin{entry}{骂街}{9,12}{⾺、⾏}
  \begin{phonetics}{骂街}{ma4jie1}
    \definition{v.}{gritar abusos na rua}
  \end{phonetics}
\end{entry}

\begin{entry}{骆驼}{9,8}{⾺、⾺}
  \begin{phonetics}{骆驼}{luo4tuo5}
    \definition[峰,匹,头]{s.}{camelo | (coloquial) cabeça-dura, idiota}
  \end{phonetics}
\end{entry}

\begin{entry}{骨}{9}{⾻}[Kangxi 188]
  \begin{phonetics}{骨}{gu3}
    \definition{s.}{osso}
  \end{phonetics}
\end{entry}

\begin{entry}{骨头}{9,5}{⾻、⼤}
  \begin{phonetics}{骨头}{gu3tou5}[][HSK 4]
    \definition[根,块]{s.}{osso; tecidos mais duros no corpo de uma pessoa ou de alguns animais que sustentam o corpo ou protegem os órgãos do corpo | caráter de uma pessoa; refere-se à qualidade do caráter de uma pessoa}
  \end{phonetics}
\end{entry}

\begin{entry}{鬼}{9}{⿁}
  \begin{phonetics}{鬼}{gui3}[][HSK 5]
    \definition*{s.}{Gui, uma das mansões lunares | sobrenome Gui}
    \definition{adj.}{evasivo; furtivo; sub-reptício; ardiloso; enganoso, malicioso; obscuro | terrível; ruim; severo; vil | esperto; astuto; inteligente}
    \definition{s.}{espírito; fantasma; aparição; refere-se à alma de uma pessoa após a morte | usado para formar um termo de abuso para caráter ignóbil; refere-se a pessoas que têm maus hábitos ou cujo comportamento é repugnante | companheiro; pessoa que é considerada divertida}
  \end{phonetics}
\end{entry}

\begin{entry}{鬼火}{9,4}{⿁、⽕}
  \begin{phonetics}{鬼火}{gui3huo3}
    \definition{s.}{fogo-fátuo | boitatá | fogo corredor | fogo de santelmo}
  \end{phonetics}
\end{entry}

\begin{entry}{鬼怪}{9,8}{⿁、⼼}
  \begin{phonetics}{鬼怪}{gui3guai4}
    \definition{s.}{\emph{hobgoblin} | bicho-papão | fantasma}
  \end{phonetics}
\end{entry}

%%%%% EOF %%%%%

