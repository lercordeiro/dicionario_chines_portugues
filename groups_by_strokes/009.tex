%%%
%%% 9画
%%%

\section*{9画}\addcontentsline{toc}{section}{9画}

\begin{entry}{临}{9}[Radical ⼁]
  \begin{phonetics}{临}{lin2}
    \definition*{s.}{sobrenome Lin}
    \definition{adv.}{pouco antes; prestes a; no ponto de}
    \definition{v.}{encarar; enfrentar; aproximar-se | chegar; estar presente | copiar (um modelo de caligrafia ou pintura); traçar sobre as palavras ou figuras | olhar de cima para baixo | ir de cima para baixo}
  \end{phonetics}
\end{entry}

\begin{entry}{临时}{9,7}[Radicais ⼁、⽇]
  \begin{phonetics}{临时}{lin2shi2}[][HSK 4]
    \definition{adj.}{temporário; provisório; por um breve período}
    \definition{adv.}{no momento em que algo acontece (quando as coisas dão errado)}
  \end{phonetics}
\end{entry}

\begin{entry}{举}{9}[Radical ⼂]
  \begin{phonetics}{举}{ju3}[][HSK 2]
    \definition{v.}{levantar | segurar | iniciar | começar | dar à luz a | eleger | escolher | citar| enumerar}
  \end{phonetics}
\end{entry}

\begin{entry}{举办}{9,4}[Radicais ⼂、⼒]
  \begin{phonetics}{举办}{ju3ban4}[][HSK 3]
    \definition{v.}{segurar; conduzir}
  \end{phonetics}
\end{entry}

\begin{entry}{举手}{9,4}[Radicais ⼂、⼿]
  \begin{phonetics}{举手}{ju3 shou3}[][HSK 2]
    \definition{v.}{levantar (colocar) a mão ou mãos}
  \end{phonetics}
\end{entry}

\begin{entry}{举行}{9,6}[Radicais ⼂、⾏]
  \begin{phonetics}{举行}{ju3xing2}[][HSK 2]
    \definition{v.}{realizar (uma reunião, cerimônia, etc.) | ter lugar}
  \end{phonetics}
\end{entry}

\begin{entry}{亭}{9}[Radical ⼇]
  \begin{phonetics}{亭}{ting2}
    \definition{s.}{pavilhão | cabine | quiosque}
  \end{phonetics}
\end{entry}

\begin{entry}{亮}{9}[Radical ⼇]
  \begin{phonetics}{亮}{liang4}[][HSK 2]
    \definition*{s.}{sobrenome Lian}
    \definition{adj.}{brilhante | alto e claro | retumbante | iluminado | aberto e claro}
    \definition{s.}{luz}
    \definition{v.}{iluminar | brilhar | elevar a voz | ressoar | revelar | mostrar | aparecer | exibir}
  \end{phonetics}
\end{entry}

\begin{entry}{亲}{9}[Radical ⼇]
  \begin{phonetics}{亲}{qin1}[][HSK 3]
    \definition{adj.}{parente próximo; relacionado por sangue; de ​​relação de sangue | querido; próximo; íntimo | em si mesmo; pessoalmente}
    \definition[位]{s.}{pais | parente | casal; casamento | noiva}
    \definition{v.}{beijar | (países, partidos, etc.) a favor de; apoiar; estar perto de}
  \end{phonetics}
  \begin{phonetics}{亲}{qing4}
    \definition{s.}{parentes por afinidade; parentes por casamento}
  \end{phonetics}
\end{entry}

\begin{entry}{亲人}{9,2}[Radicais ⼇、⼈]
  \begin{phonetics}{亲人}{qin1 ren2}[][HSK 3]
    \definition{s.}{um membro da família; os pais, o cônjuge, os filhos, etc. | queridos; entes queridos; aqueles queridos para alguém}
  \end{phonetics}
\end{entry}

\begin{entry}{亲切}{9,4}[Radicais ⼇、⼑]
  \begin{phonetics}{亲切}{qin1qie4}[][HSK 3]
    \definition{adj.}{gentil; cordial | próximo; íntimo}
  \end{phonetics}
\end{entry}

\begin{entry}{亲自}{9,6}[Radicais ⼇、⾃]
  \begin{phonetics}{亲自}{qin1zi4}[][HSK 3]
    \definition{adv.}{pessoalmente; em pessoa; si mesmo}
  \end{phonetics}
\end{entry}

\begin{entry}{亲爱}{9,10}[Radicais ⼇、⽖]
  \begin{phonetics}{亲爱}{qin1'ai4}[][HSK 4]
    \definition{adj.}{querido; amado; termo carinhoso que expressa intimidade e afeto}
  \end{phonetics}
\end{entry}

\begin{entry}{亲密}{9,11}[Radicais ⼇、⼧]
  \begin{phonetics}{亲密}{qin1mi4}[][HSK 4]
    \definition{adj.}{próximo; íntimo; relacionamento afetuoso e próximo}
  \end{phonetics}
\end{entry}

\begin{entry}{侵略}{9,11}[Radicais ⼈、⽥]
  \begin{phonetics}{侵略}{qin1lve4}
    \definition{s.}{invasão}
    \definition{v.}{invadir}
  \end{phonetics}
\end{entry}

\begin{entry}{便宜}{9,8}[Radicais ⼈、⼧]
  \begin{phonetics}{便宜}{pian2yi5}[][HSK 2]
    \definition{adj.}{barato}
    \definition{v.}{deixar alguém levemente de lado}
  \end{phonetics}
\end{entry}

\begin{entry}{促进}{9,7}[Radicais ⼈、⾡]
  \begin{phonetics}{促进}{cu4jin4}[][HSK 4]
    \definition{v.}{impulsionar; promover; avançar; incentivar o desenvolvimento}
  \end{phonetics}
\end{entry}

\begin{entry}{促使}{9,8}[Radicais ⼈、⼈]
  \begin{phonetics}{促使}{cu4shi3}[][HSK 4]
    \definition{v.}{incitar; estimular; impelir; causar; provocar uma mudança em alguém ou em algo}
  \end{phonetics}
\end{entry}

\begin{entry}{促销}{9,12}[Radicais ⼈、⾦]
  \begin{phonetics}{促销}{cu4 xiao1}[][HSK 4]
    \definition{v.}{promover vendas}
  \end{phonetics}
\end{entry}

\begin{entry}{俄}{9}[Radical ⼈]
  \begin{phonetics}{俄}{e2}
    \definition*{s.}{Rússia, abreviação de 俄罗斯}
  \seealsoref{俄罗斯}{e2luo2si1}
  \end{phonetics}
\end{entry}

\begin{entry}{俄罗斯}{9,8,12}[Radicais ⼈、⽹、⽄]
  \begin{phonetics}{俄罗斯}{e2luo2si1}
    \definition*{s.}{Rússia}
  \end{phonetics}
\end{entry}

\begin{entry}{俄罗斯人}{9,8,12,2}[Radicais ⼈、⽹、⽄、⼈]
  \begin{phonetics}{俄罗斯人}{e2luo2si1ren2}
    \definition{s.}{russo | pessoa ou povo da Rússia}
  \end{phonetics}
\end{entry}

\begin{entry}{保}{9}[Radical ⼈]
  \begin{phonetics}{保}{bao3}[][HSK 3]
    \definition*{s.}{sobrenome Bao}
    \definition{s.}{fiador
oficial responsável
sistema administrativo}
    \definition{v.}{defender | proteger |manter | preservar | conservar em boas condições | garantir | assegurar | ficar como fiador de alguém.}
  \end{phonetics}
\end{entry}

\begin{entry}{保存}{9,6}[Radicais ⼈、⼦]
  \begin{phonetics}{保存}{bao3cun2}[][HSK 3]
    \definition{v.}{conservar | preservar | (computação) salvar (um arquivo, etc.)}
  \end{phonetics}
\end{entry}

\begin{entry}{保守}{9,6}[Radicais ⼈、⼧]
  \begin{phonetics}{保守}{bao3shou3}[][HSK 4]
    \definition{adj.}{retrógrado; conservador; pensamentos e conceitos que são retrógrados e não conseguem acompanhar o desenvolvimento da situação}
    \definition{v.}{manter; guardar; evitar perder}
  \end{phonetics}
\end{entry}

\begin{entry}{保安}{9,6}[Radicais ⼈、⼧]
  \begin{phonetics}{保安}{bao3 an1}[][HSK 3]
    \definition{s.}{guarda de segurança}
    \definition{v.}{manter seguro | garantir a segurança}
  \end{phonetics}
\end{entry}

\begin{entry}{保护}{9,7}[Radicais ⼈、⼿]
  \begin{phonetics}{保护}{bao3hu4}[][HSK 3]
    \definition{s.}{proteção | salvaguarda}
    \definition{v.}{proteger | defender | salvaguardar}
  \end{phonetics}
\end{entry}

\begin{entry}{保护区}{9,7,4}[Radicais ⼈、⼿、⼖]
  \begin{phonetics}{保护区}{bao3hu4qu1}
    \definition{s.}{área protegida | área de conservação}
  \end{phonetics}
\end{entry}

\begin{entry}{保护主义}{9,7,5,3}[Radicais ⼈、⼿、⼂、⼂]
  \begin{phonetics}{保护主义}{bao3hu4zhu3yi4}
    \definition{s.}{protecionismo}
  \end{phonetics}
\end{entry}

\begin{entry}{保护色}{9,7,6}[Radicais ⼈、⼿、⾊]
  \begin{phonetics}{保护色}{bao3hu4se4}
    \definition{s.}{camuflagem}
  \end{phonetics}
\end{entry}

\begin{entry}{保护剂}{9,7,8}[Radicais ⼈、⼿、⼑]
  \begin{phonetics}{保护剂}{bao3hu4ji4}
    \definition{s.}{agente protetor}
  \end{phonetics}
\end{entry}

\begin{entry}{保护国}{9,7,8}[Radicais ⼈、⼿、⼞]
  \begin{phonetics}{保护国}{bao3hu4guo2}
    \definition{s.}{protetorado}
  \end{phonetics}
\end{entry}

\begin{entry}{保护性}{9,7,8}[Radicais ⼈、⼿、⼼]
  \begin{phonetics}{保护性}{bao3hu4xing4}
    \definition{s.}{proteção}
  \end{phonetics}
\end{entry}

\begin{entry}{保护物}{9,7,8}[Radicais ⼈、⼿、⽜]
  \begin{phonetics}{保护物}{bao3hu4 wu4}
    \definition{s.}{protetor}
  \end{phonetics}
\end{entry}

\begin{entry}{保护者}{9,7,8}[Radicais ⼈、⼿、⽼]
  \begin{phonetics}{保护者}{bao3hu4zhe3}
    \definition{s.}{protetor | segurador}
  \end{phonetics}
\end{entry}

\begin{entry}{保护神}{9,7,9}[Radicais ⼈、⼿、⽰]
  \begin{phonetics}{保护神}{bao3hu4shen2}
    \definition{s.}{anjo da guarda | santo patrono}
  \end{phonetics}
\end{entry}

\begin{entry}{保证}{9,7}[Radicais ⼈、⾔]
  \begin{phonetics}{保证}{bao3zheng4}[][HSK 3]
    \definition[个]{s.}{garantia}
    \definition{v.}{garantir}
  \end{phonetics}
\end{entry}

\begin{entry}{保持}{9,9}[Radicais ⼈、⼿]
  \begin{phonetics}{保持}{bao3chi2}[][HSK 3]
    \definition{v.}{manter | segurar | reter | preservar}
  \end{phonetics}
\end{entry}

\begin{entry}{保险}{9,9}[Radicais ⼈、⾩]
  \begin{phonetics}{保险}{bao3xian3}[][HSK 3]
    \definition[个]{adj./s.}{seguro}
    \definition{v.}{ter certeza | estar vinculado a}
  \end{phonetics}
\end{entry}

\begin{entry}{保留}{9,10}[Radicais ⼈、⽥]
  \begin{phonetics}{保留}{bao3liu2}[][HSK 3]
    \definition{v.}{reter | continuar a ter | segurar | reservar}
  \end{phonetics}
\end{entry}

\begin{entry}{保密}{9,11}[Radicais ⼈、⼧]
  \begin{phonetics}{保密}{bao3mi4}[][HSK 4]
    \definition{v.}{manter segredo; manter algo em segredo; manter a confidencialidade}
  \end{phonetics}
\end{entry}

\begin{entry}{信}{9}[Radical ⼈]
  \begin{phonetics}{信}{xin4}[][HSK 2,3]
    \definition*{s.}{sobrenome Xin}
    \definition{adj.}{verdade}
    \definition{adv.}{à vontade; ao acaso; sem plano}
    \definition[封,个,张]{s.}{carta; correio
mensagem; palavra; informação
sinal; evidência
confiança; fé
fusível
arsênico}
    \definition{v.}{acreditar; fazer um balanço; dar crédito | professar fé em; acreditar em}
  \end{phonetics}
\end{entry}

\begin{entry}{信心}{9,4}[Radicais ⼈、⼼]
  \begin{phonetics}{信心}{xin4xin1}[][HSK 2]
    \definition[个]{s.}{confiança | fé (em alguém ou algo)}
  \end{phonetics}
\end{entry}

\begin{entry}{信号}{9,5}[Radicais ⼈、⼝]
  \begin{phonetics}{信号}{xin4hao4}[][HSK 2]
    \definition[个]{s.}{sinal | ponte de sinalização}
  \end{phonetics}
\end{entry}

\begin{entry}{信用}{9,5}[Radicais ⼈、⽤]
  \begin{phonetics}{信用}{xin4yong4}
    \definition{s.}{crédito (comércio)}
  \end{phonetics}
\end{entry}

\begin{entry}{信用卡}{9,5,5}[Radicais ⼈、⽤、⼘]
  \begin{phonetics}{信用卡}{xin4yong4ka3}[][HSK 2]
    \definition[些]{s.}{cartão de crédito}
  \end{phonetics}
\end{entry}

\begin{entry}{信任}{9,6}[Radicais ⼈、⼈]
  \begin{phonetics}{信任}{xin4ren4}[][HSK 3]
    \definition[个]{s.}{confiança; certeza; convicção}
    \definition{v.}{confiar; ter confiança em}
  \end{phonetics}
\end{entry}

\begin{entry}{信访}{9,6}[Radicais ⼈、⾔]
  \begin{phonetics}{信访}{xin4fang3}
    \definition{s.}{carta de reclamação | carta de petição}
  \seealsoref{上访}{shang4fang3}
  \end{phonetics}
\end{entry}

\begin{entry}{信经}{9,8}[Radicais ⼈、⽷]
  \begin{phonetics}{信经}{xin4jing1}
    \definition[个]{s.}{crença | credo (seção da missa católica)}
  \end{phonetics}
\end{entry}

\begin{entry}{信封}{9,9}[Radicais ⼈、⼨]
  \begin{phonetics}{信封}{xin4feng1}[][HSK 3]
    \definition[个]{s.}{envelope de carta}
  \end{phonetics}
\end{entry}

\begin{entry}{信息}{9,10}[Radicais ⼈、⼼]
  \begin{phonetics}{信息}{xin4xi1}[][HSK 2]
    \definition[个,条]{s.}{notícias | informação | mensagem}
  \end{phonetics}
\end{entry}

\begin{entry}{俩}{9}[Radical ⼈]
  \begin{phonetics}{俩}{lia3}[][HSK 4]
    \definition{num.}{ambos; dois; contração de ``两个'' | alguns; vários; refere-se a um pequeno número}
  \end{phonetics}
\end{entry}

\begin{entry}{俩钱}{9,10}[Radicais ⼈、⾦]
  \begin{phonetics}{俩钱}{lia3qian2}
    \definition{s.}{uma pequena quantia de dinheiro}
  \end{phonetics}
\end{entry}

\begin{entry}{俭省}{9,9}[Radicais ⼈、⽬]
  \begin{phonetics}{俭省}{jian3sheng3}
    \definition{adj.}{econômico}
  \end{phonetics}
\end{entry}

\begin{entry}{修}{9}[Radical ⼈]
  \begin{phonetics}{修}{xiu1}[][HSK 3]
    \definition*{s.}{sobrenome Xiu}
    \definition{adj.}{comprido; alto e esbelto}
    \definition{s.}{revisionismo}
    \definition{v.}{embelezar; decorar
consertar; reparar; reformar
escrever; compilar
estudar; cultivar
construir; edificar
aparar; podar}
  \end{phonetics}
\end{entry}

\begin{entry}{修改}{9,7}[Radicais ⼈、⽁]
  \begin{phonetics}{修改}{xiu1gai3}[][HSK 3]
    \definition{v.}{revisar; alterar}
  \end{phonetics}
\end{entry}

\begin{entry}{修规}{9,8}[Radicais ⼈、⾒]
  \begin{phonetics}{修规}{xiu1gui1}
    \definition{s.}{plano de construção}
  \end{phonetics}
\end{entry}

\begin{entry}{养}{9}[Radical ⼋]
  \begin{phonetics}{养}{yang3}[][HSK 2]
    \definition{v.}{criar (animais ou filhos), plantar (flores), etc. | dar a luz}
  \end{phonetics}
\end{entry}

\begin{entry}{养分}{9,4}[Radicais ⼋、⼑]
  \begin{phonetics}{养分}{yang3fen4}
    \definition{s.}{nutriente}
  \end{phonetics}
\end{entry}

\begin{entry}{养料}{9,10}[Radicais ⼋、⽃]
  \begin{phonetics}{养料}{yang3liao4}
    \definition{s.}{nutrição}
  \end{phonetics}
\end{entry}

\begin{entry}{冒险}{9,9}[Radicais ⽇、⾩]
  \begin{phonetics}{冒险}{mao4xian3}
    \definition{adj.}{corajoso}
    \definition{s.}{risco | aventura}
    \definition{v.+compl.}{correr risco | arriscar-se | aventurar-se em}
  \end{phonetics}
\end{entry}

\begin{entry}{冠}{9}[Radical ⼍]
  \begin{phonetics}{冠}{guan1}
    \definition{s.}{chapéu | coroa | brasão | boné}
  \end{phonetics}
  \begin{phonetics}{冠}{guan4}
    \definition*{s.}{sobrenome Guan}
    \definition{v.}{colocar um chapéu | ser o primeiro | dublar}
  \end{phonetics}
\end{entry}

\begin{entry}{前}{9}[Radical ⼑]
  \begin{phonetics}{前}{qian2}[][HSK 1]
    \definition{adv.}{frente; em frente de | A.C. (Antes de~Cristo)}[前293年  (293 a.C.)]
  \seealsoref{公元}{gong1yuan2}
  \end{phonetics}
\end{entry}

\begin{entry}{前天}{9,4}[Radicais ⼑、⼤]
  \begin{phonetics}{前天}{qian2tian1}[][HSK 1]
    \definition{adv.}{anteontem}
  \end{phonetics}
\end{entry}

\begin{entry}{前头}{9,5}[Radicais ⼑、⼤]
  \begin{phonetics}{前头}{qian2 tou5}[][HSK 4]
    \definition{s.}{à frente; na frente; adiante}
  \end{phonetics}
\end{entry}

\begin{entry}{前边}{9,5}[Radicais ⼑、⾡]
  \begin{phonetics}{前边}{qian2bian5}[][HSK 1]
    \definition{adv.}{à frente | da frente}
  \end{phonetics}
\end{entry}

\begin{entry}{前后}{9,6}[Radicais ⼑、⼝]
  \begin{phonetics}{前后}{qian2 hou4}[][HSK 3]
    \definition{s.}{em volta; sobre | do início ao fim | frente e verso}
  \end{phonetics}
\end{entry}

\begin{entry}{前年}{9,6}[Radicais ⼑、⼲]
  \begin{phonetics}{前年}{qian2 nian2}[][HSK 2]
    \definition{adv.}{há dois anos}
  \end{phonetics}
\end{entry}

\begin{entry}{前进}{9,7}[Radicais ⼑、⾡]
  \begin{phonetics}{前进}{qian2 jin4}[][HSK 3]
    \definition{v.}{marchar; avançar; para ir em frente; seguir em frente}
  \end{phonetics}
\end{entry}

\begin{entry}{前往}{9,8}[Radicais ⼑、⼻]
  \begin{phonetics}{前往}{qian2 wang3}[][HSK 3]
    \definition{v.}{ir para; prosseguir para; partir para}
  \end{phonetics}
\end{entry}

\begin{entry}{前面}{9,9}[Radicais ⼑、⾯]
  \begin{phonetics}{前面}{qian2mian4}[][HSK 3]
    \definition{s.}{frente | parte anterior; acima}
  \end{phonetics}
\end{entry}

\begin{entry}{前途}{9,10}[Radicais ⼑、⾡]
  \begin{phonetics}{前途}{qian2tu2}[][HSK 4]
    \definition[片,段,种]{s.}{futuro; perspectiva; prospecto; originalmente, refere-se à jornada à frente, mas, metaforicamente, refere-se ao futuro.}
  \end{phonetics}
\end{entry}

\begin{entry}{剑}{9}[Radical ⼑]
  \begin{phonetics}{剑}{jian4}
    \definition{clas.}{para golpes de uma espada}
    \definition[口,把]{s.}{espada de dois gumes}
  \end{phonetics}
\end{entry}

\begin{entry}{剑客}{9,9}[Radicais ⼑、⼧]
  \begin{phonetics}{剑客}{jian4ke4}
    \definition{s.}{espada | esgrimista, espadachim}
  \end{phonetics}
\end{entry}

\begin{entry}{勇士}{9,3}[Radicais ⼒、⼠]
  \begin{phonetics}{勇士}{yong3shi4}
    \definition{s.}{um guerreiro | uma pessoa corajosa}
  \end{phonetics}
\end{entry}

\begin{entry}{勇气}{9,4}[Radicais ⼒、⽓]
  \begin{phonetics}{勇气}{yong3qi4}
    \definition{adj.}{coragem | valor}
  \end{phonetics}
\end{entry}

\begin{entry}{勇敢}{9,11}[Radicais ⼒、⽁]
  \begin{phonetics}{勇敢}{yong3gan3}
    \definition{adj.}{bravo | corajoso}
  \end{phonetics}
\end{entry}

\begin{entry}{南}{9}[Radical ⼗]
  \begin{phonetics}{南}{nan2}[][HSK 1]
    \definition*{s.}{sobrenome Nan}
    \definition{s.}{sul}
  \end{phonetics}
\end{entry}

\begin{entry}{南方}{9,4}[Radicais ⼗、⽅]
  \begin{phonetics}{南方}{nan2 fang1}[][HSK 2]
    \definition{s.}{sul | o Sul | a parte sul do país}
  \end{phonetics}
\end{entry}

\begin{entry}{南边}{9,5}[Radicais ⼗、⾡]
  \begin{phonetics}{南边}{nan2bian5}[][HSK 1]
    \definition{adv.}{sul | lado sul | parte sul | ao sul de}
  \end{phonetics}
\end{entry}

\begin{entry}{南极}{9,7}[Radicais ⼗、⽊]
  \begin{phonetics}{南极}{nan2ji2}
    \definition*{s.}{Antártico | Pólo Sul}
    \definition{s.}{pólo sul magnético}
  \end{phonetics}
\end{entry}

\begin{entry}{南面}{9,9}[Radicais ⼗、⾯]
  \begin{phonetics}{南面}{nan2mian4}
    \definition{s.}{sul | lado sul}
  \end{phonetics}
\end{entry}

\begin{entry}{南部}{9,10}[Radicais ⼗、⾢]
  \begin{phonetics}{南部}{nan2 bu4}[][HSK 3]
    \definition{s.}{parte sul; sul | a parte sul}
  \end{phonetics}
\end{entry}

\begin{entry}{厘米}{9,6}[Radicais ⼚、⽶]
  \begin{phonetics}{厘米}{li2mi3}[][HSK 4]
    \definition{clas.}{centímetro; unidade de comprimento, símbolo cm, 1 metro é igual a 100 centímetros}
  \end{phonetics}
\end{entry}

\begin{entry}{厚}{9}[Radical ⼚]
  \begin{phonetics}{厚}{hou4}[][HSK 4]
    \definition*{s.}{sobrenome Hou}
    \definition{adj.}{grosso; espesso | profundo | bondoso; gentil; magnânimo | grande; generoso | rico ou forte em sabor}
    \definition{s.}{espessura; profundidade}
    \definition{v.}{favorecer; enfatizar}
  \end{phonetics}
\end{entry}

\begin{entry}{咱}{9}[Radical ⼝]
  \begin{phonetics}{咱}{zan2}[][HSK 2]
    \definition{pron.}{eu}
  \end{phonetics}
\end{entry}

\begin{entry}{咱们}{9,5}[Radicais ⼝、⼈]
  \begin{phonetics}{咱们}{zan2men5}[][HSK 2]
    \definition{pron.}{nós (incluindo o orador e a(s) pessoa(s) com quem se fala)}
  \end{phonetics}
\end{entry}

\begin{entry}{咱俩}{9,9}[Radicais ⼝、⼈]
  \begin{phonetics}{咱俩}{zan2lia3}
    \definition{pron.}{nós dois}
  \end{phonetics}
\end{entry}

\begin{entry}{咱家}{9,10}[Radicais ⼝、⼧]
  \begin{phonetics}{咱家}{za2jia1}
    \definition{pron.}{eu (frequentemente usado na literatura vernácula antiga) | me | mim | comigo}
  \end{phonetics}
\end{entry}

\begin{entry}{咳嗽}{9,14}[Radicais ⼝、⼝]
  \begin{phonetics}{咳嗽}{ke2sou5}
    \definition{v.}{ter tosse | tossir}
  \end{phonetics}
\end{entry}

\begin{entry}{咸}{9}[Radical ⼝]
  \begin{phonetics}{咸}{xian2}[][HSK 4]
    \definition*{s.}{sobrenome Xian}
    \definition{adj.}{salgado; em conserva; sabor salgado}
    \definition{adv.}{todos; indica a totalidade de um intervalo, equivalente a ``全'' e ``都''}
  \seealsoref{都}{dou1}
  \seealsoref{全}{quan2}
  \end{phonetics}
\end{entry}

\begin{entry}{咸水}{9,4}[Radicais ⼝、⽔]
  \begin{phonetics}{咸水}{xian2shui3}
    \definition{s.}{salmora | água salgada}
  \end{phonetics}
\end{entry}

\begin{entry}{咸肉}{9,6}[Radicais ⼝、⾁]
  \begin{phonetics}{咸肉}{xian2rou4}
    \definition{s.}{\emph{bacon} | carne curada com sal}
  \end{phonetics}
\end{entry}

\begin{entry}{咸鱼}{9,8}[Radicais ⼝、⿂]
  \begin{phonetics}{咸鱼}{xian2yu2}
    \definition{s.}{peixe salgado}
  \end{phonetics}
\end{entry}

\begin{entry}{咸涩}{9,10}[Radicais ⼝、⽔]
  \begin{phonetics}{咸涩}{xian2se4}
    \definition{s.}{ácido | salgado e amargo}
  \end{phonetics}
\end{entry}

\begin{entry}{咸盐}{9,10}[Radicais ⼝、⽫]
  \begin{phonetics}{咸盐}{xian2yan2}
    \definition{s.}{(coloquial) sal | sal de mesa}
  \end{phonetics}
\end{entry}

\begin{entry}{咸淡}{9,11}[Radicais ⼝、⽔]
  \begin{phonetics}{咸淡}{xian2dan4}
    \definition{s.}{água salobra | grau de salinidade | salgado e sem sal (sabores)}
  \end{phonetics}
\end{entry}

\begin{entry}{咸菜}{9,11}[Radicais ⼝、⾋]
  \begin{phonetics}{咸菜}{xian2cai4}
    \definition{s.}{legumes salgados | \emph{pickles}}
  \end{phonetics}
\end{entry}

\begin{entry}{品质}{9,8}[Radicais ⼝、⾙]
  \begin{phonetics}{品质}{pin3zhi4}[][HSK 4]
    \definition[个,种]{s.}{qualidade; caráter; natureza do pensamento, da compreensão, do caráter, etc., conforme expresso no comportamento, no estilo, etc. | qualidade (de produtos, mercadorias, etc.)}
  \end{phonetics}
\end{entry}

\begin{entry}{品德}{9,15}[Radicais ⼝、⼻]
  \begin{phonetics}{品德}{pin3de2}
    \definition{s.}{caráter moral | moralidade}
  \end{phonetics}
\end{entry}

\begin{entry}{哄}{9}[Radical ⼝]
  \begin{phonetics}{哄}{hong1}
    \definition{s.}{gargalhadas | risadas ruidosas | algazarra | rugido | clamor}
  \end{phonetics}
  \begin{phonetics}{哄}{hong3}
    \definition{v.}{enganar | persuadir | divertir (uma criança)}
  \end{phonetics}
  \begin{phonetics}{哄}{hong4}
    \definition{s.}{tumulto | agitação | perturbação}
  \end{phonetics}
\end{entry}

\begin{entry}{哇塞}{9,13}[Radicais ⼝、⼟]
  \begin{phonetics}{哇塞}{wa1sai1}
    \definition{interj.}{(gíria) Uau!}
  \end{phonetics}
\end{entry}

\begin{entry}{哇噻}{9,16}[Radicais ⼝、⼝]
  \begin{phonetics}{哇噻}{wa1sai1}
    \variantof{哇塞}
  \end{phonetics}
\end{entry}

\begin{entry}{哈马斯}{9,3,12}[Radicais ⼝、⾺、⽄]
  \begin{phonetics}{哈马斯}{ha1ma3si1}
    \definition*{s.}{Hamas (Grupo Palestino)}
  \end{phonetics}
\end{entry}

\begin{entry}{哈哈}{9,9}[Radicais ⼝、⼝]
  \begin{phonetics}{哈哈}{ha1 ha1}[][HSK 3]
    \definition{expr.}{(onomatopéia)  ha ha; o som de uma risada alta}
  \end{phonetics}
\end{entry}

\begin{entry}{响}{9}[Radical ⼝]
  \begin{phonetics}{响}{xiang3}[][HSK 2]
    \definition{adj.}{barulhento}
    \definition[声,阵]{s.}{som | barulho | eco}
    \definition{v.}{fazer um som | soar | tocar}
  \end{phonetics}
\end{entry}

\begin{entry}{哪}{9}[Radical ⼝]
  \begin{phonetics}{哪}{na3}[][HSK 1,4]
    \definition{adv.}{para expressar uma pergunta retórica}
    \definition{pron.}{qual?; o que? | qualquer; ser usado em um sentido geral}
  \end{phonetics}
  \begin{phonetics}{哪}{na5}
    \definition{part.}{usado depois de uma palavra com a terminação -n, é equivalente a ``啊''}
  \seealsoref{啊}{a5}
  \end{phonetics}
  \begin{phonetics}{哪}{nei3}
    \definition{part.}{qual? (interrogativo, seguido de classificador ou numeral-classificador)}
  \end{phonetics}
\end{entry}

\begin{entry}{哪儿}{9,2}[Radicais ⼝、⼉]
  \begin{phonetics}{哪儿}{na3r5}[][HSK 1]
    \definition{adv.}{onde?}
  \end{phonetics}
\end{entry}

\begin{entry}{哪里}{9,7}[Radicais ⼝、⾥]
  \begin{phonetics}{哪里}{na3 li3}[][HSK 1]
    \definition{adv.}{onde?}
  \end{phonetics}
\end{entry}

\begin{entry}{哪些}{9,8}[Radicais ⼝、⼆]
  \begin{phonetics}{哪些}{na3xie1}[][HSK 1]
    \definition{pron.}{quais?}
  \end{phonetics}
\end{entry}

\begin{entry}{哪国人}{9,8,2}[Radicais ⼝、⼞、⼈]
  \begin{phonetics}{哪国人}{na3 guo2ren2}
    \definition{expr.}{de qual país?}
  \end{phonetics}
\end{entry}

\begin{entry}{哪怕}{9,8}[Radicais ⼝、⼼]
  \begin{phonetics}{哪怕}{na3pa4}[][HSK 4]
    \definition{conj.}{mesmo; mesmo se; mesmo que; não importa o quão}
  \end{phonetics}
\end{entry}

\begin{entry}{垫子}{9,3}[Radicais ⼟、⼦]
  \begin{phonetics}{垫子}{dian4zi5}
    \definition{s.}{colchão | esteira | almofada}
  \end{phonetics}
\end{entry}

\begin{entry}{城}{9}[Radical ⼟]
  \begin{phonetics}{城}{cheng2}[][HSK 3]
    \definition*{s.}{sobrenome Cheng}
    \definition[座,道,个]{s.}{muralha da cidade; muro | cidade}
  \end{phonetics}
\end{entry}

\begin{entry}{城市}{9,5}[Radicais ⼟、⼱]
  \begin{phonetics}{城市}{cheng2shi4}[][HSK 3]
    \definition[个,座]{s.}{cidade}
  \end{phonetics}
\end{entry}

\begin{entry}{城度}{9,9}[Radicais ⼟、⼴]
  \begin{phonetics}{城度}{cheng2du4}[][HSK 3]
    \definition*{s.}{Cidade}
  \end{phonetics}
\end{entry}

\begin{entry}{城堡}{9,12}[Radicais ⼟、⼟]
  \begin{phonetics}{城堡}{cheng2bao3}
    \definition*{s.}{castelo | torre (peça de xadrez)}
  \end{phonetics}
\end{entry}

\begin{entry}{复习}{9,3}[Radicais ⼢、⼄]
  \begin{phonetics}{复习}{fu4xi2}[][HSK 2]
    \definition{s.}{revisão}
    \definition{v.}{rever | revisar}
  \end{phonetics}
\end{entry}

\begin{entry}{复印}{9,5}[Radicais ⼢、⼙]
  \begin{phonetics}{复印}{fu4yin4}[][HSK 3]
    \definition{v.}{fotografar; fotocopiar; duplicar}
  \end{phonetics}
\end{entry}

\begin{entry}{复杂}{9,6}[Radicais ⼢、⽊]
  \begin{phonetics}{复杂}{fu4za2}[][HSK 3]
    \definition{adj.}{complexo; complicado}
  \end{phonetics}
\end{entry}

\begin{entry}{复制}{9,8}[Radicais ⼢、⼑]
  \begin{phonetics}{复制}{fu4zhi4}[][HSK 4]
    \definition{v.}{copiar; duplicar; reproduzir; fazer uma cópia de; fazer uma cópia do original ou reproduzi-lo, reimprimi-lo ou copiá-lo em sua forma original (geralmente referindo-se a relíquias culturais ou obras de arte)}
  \end{phonetics}
\end{entry}

\begin{entry}{复刻}{9,8}[Radicais ⼢、⼑]
  \begin{phonetics}{复刻}{fu4ke4}
    \definition{v.}{reimprimir (um trabalho que esteve fora do catálogo) | reeditar (um disco de vinil, um CD, etc.) | replicar | recriar | (empréstimo linguístico) (computação) \emph{fork}}
  \end{phonetics}
\end{entry}

\begin{entry}{复活节}{9,9,5}[Radicais ⼢、⽔、⾋]
  \begin{phonetics}{复活节}{fu4huo2jie2}
    \definition*{s.}{Páscoa}
  \end{phonetics}
\end{entry}

\begin{entry}{奏效}{9,10}[Radicais ⼤、⽁]
  \begin{phonetics}{奏效}{zou4xiao4}
    \definition{v.}{mostrar resultados | ser eficaz}
  \end{phonetics}
\end{entry}

\begin{entry}{奖}{9}[Radical ⼤]
  \begin{phonetics}{奖}{jiang3}[][HSK 4]
    \definition[个,次]{s.}{prêmio; recompensa | elogio; loa}
    \definition{v.}{elogiar; recompensar; recomendar; incentivar}
  \end{phonetics}
\end{entry}

\begin{entry}{奖学金}{9,8,8}[Radicais ⼤、⼦、⾦]
  \begin{phonetics}{奖学金}{jiang3 xue2 jin1}[][HSK 4]
    \definition[笔]{s.}{bolsa de estudos; exposição; prêmios concedidos por escolas, organizações ou indivíduos a alunos com bom desempenho acadêmico}
  \end{phonetics}
\end{entry}

\begin{entry}{奖金}{9,8}[Radicais ⼤、⾦]
  \begin{phonetics}{奖金}{jiang3jin1}[][HSK 4]
    \definition[个,笔]{s.}{bônus; recompensa; prêmio; prêmio em dinheiro; dinheiro de recompensa, dinheiro dado às pessoas para incentivá-las ou elogiá-las por terem se saído bem em alguma coisa}
  \end{phonetics}
\end{entry}

\begin{entry}{姜}{9}[Radical ⼥]
  \begin{phonetics}{姜}{jiang1}
    \definition*{s.}{sobrenome Jiang}
    \definition{s.}{gengibre}
  \end{phonetics}
\end{entry}

\begin{entry}{孩子}{9,3}[Radicais ⼦、⼦]
  \begin{phonetics}{孩子}{hai2zi5}[][HSK 1]
    \definition{s.}{criança | filho}
  \end{phonetics}
\end{entry}

\begin{entry}{客人}{9,2}[Radicais ⼧、⼈]
  \begin{phonetics}{客人}{ke4ren2}[][HSK 2]
    \definition{s.}{visitante | convidado | cliente | passageiro | viajante}
  \end{phonetics}
\end{entry}

\begin{entry}{客厅}{9,4}[Radicais ⼧、⼚]
  \begin{phonetics}{客厅}{ke4ting1}
    \definition[间]{s.}{sala de estar | sala de visitas}
  \end{phonetics}
\end{entry}

\begin{entry}{客气}{9,4}[Radicais ⼧、⽓]
  \begin{phonetics}{客气}{ke4qi5}
    \definition{adj.}{cortês | delicado | modesto | educado}
    \definition{v.}{fazer cerimônia}
  \end{phonetics}
\end{entry}

\begin{entry}{客观}{9,6}[Radicais ⼧、⾒]
  \begin{phonetics}{客观}{ke4guan1}[][HSK 3]
    \definition{adj.}{objetivo; justo e razoável; imparcial}
    \definition{s.}{objetivo}
  \end{phonetics}
\end{entry}

\begin{entry}{宣布}{9,5}[Radicais ⼧、⼱]
  \begin{phonetics}{宣布}{xuan1bu4}[][HSK 3]
    \definition{v.}{declarar; proclamar; pronunciar; anunciar | anunciar oficialmente e publicamente as últimas decisões e situações a todos}
  \end{phonetics}
\end{entry}

\begin{entry}{宣传}{9,6}[Radicais ⼧、⼈]
  \begin{phonetics}{宣传}{xuan1chuan2}[][HSK 3]
    \definition{v.}{propagar; disseminar; conduzir propaganda | explicar às massas para que elas possam acreditar e agir de acordo}
  \end{phonetics}
\end{entry}

\begin{entry}{宣扬}{9,6}[Radicais ⼧、⼿]
  \begin{phonetics}{宣扬}{xuan1yang2}
    \definition{v.}{divulgar | anunciar | espalhar por toda parte}
  \end{phonetics}
\end{entry}

\begin{entry}{室}{9}[Radical ⼧]
  \begin{phonetics}{室}{shi4}[][HSK 3]
    \definition*{s.}{sobrenome Shi | Shi, uma das mansões lunares}
    \definition{s.}{sala; aposento; cômodo |  seção; escritório | esposa}
  \end{phonetics}
\end{entry}

\begin{entry}{宪制}{9,8}[Radicais ⼧、⼑]
  \begin{phonetics}{宪制}{xian4zhi4}
    \definition{adj.}{constitucional}
    \definition{s.}{sistema de governo constitucional}
  \end{phonetics}
\end{entry}

\begin{entry}{宪法法院}{9,8,8,9}[Radicais ⼧、⽔、⽔、⾩]
  \begin{phonetics}{宪法法院}{xian4fa3fa3yuan4}
    \definition{s.}{tribunal constitucional}
  \end{phonetics}
\end{entry}

\begin{entry}{宪政}{9,9}[Radicais ⼧、⽁]
  \begin{phonetics}{宪政}{xian4zheng4}
    \definition{s.}{governo constitucional}
  \end{phonetics}
\end{entry}

\begin{entry}{封}{9}[Radical ⼨]
  \begin{phonetics}{封}{feng1}[][HSK 2]
    \definition*{s.}{sobrenome Feng}
    \definition{clas.}{para objetos selados, especialmente cartas}
    \definition{v.}{conceder um título | conferir | conceder | selar}
  \end{phonetics}
\end{entry}

\begin{entry}{封口}{9,3}[Radicais ⼨、⼝]
  \begin{phonetics}{封口}{feng1kou3}
    \definition{v.}{selar | fechar | curar (uma ferida) | manter os lábios selados}
  \end{phonetics}
\end{entry}

\begin{entry}{封印}{9,5}[Radicais ⼨、⼙]
  \begin{phonetics}{封印}{feng1yin4}
    \definition{s.}{selo (em envelopes)}
  \end{phonetics}
\end{entry}

\begin{entry}{封闭}{9,6}[Radicais ⼨、⾨]
  \begin{phonetics}{封闭}{feng1bi4}[][HSK 4]
    \definition{adj.}{fechado; aqueles que não têm contato com o mundo exterior; aqueles que são muito conservadores (em seu pensamento) e não se comunicam com os outros}
    \definition{v.}{selar; fechar; lacrar; vedar; de modo a impedir a passagem, o uso ou a abertura}
  \end{phonetics}
\end{entry}

\begin{entry}{封冻}{9,7}[Radicais ⼨、⼎]
  \begin{phonetics}{封冻}{feng1dong4}
    \definition{v.}{congelar (água ou terra)}
  \end{phonetics}
\end{entry}

\begin{entry}{封底}{9,8}[Radicais ⼨、⼴]
  \begin{phonetics}{封底}{feng1di3}
    \definition{s.}{contracapa de um livro}
  \end{phonetics}
\end{entry}

\begin{entry}{封建}{9,8}[Radicais ⼨、⼵]
  \begin{phonetics}{封建}{feng1jian4}
    \definition{adj.}{feudal}
    \definition{s.}{feudalismo}
  \end{phonetics}
\end{entry}

\begin{entry}{封面}{9,9}[Radicais ⼨、⾯]
  \begin{phonetics}{封面}{feng1mian4}
    \definition{s.}{capa (de uma publicação) | sobrecapa}
  \end{phonetics}
\end{entry}

\begin{entry}{封斋}{9,10}[Radicais ⼨、⽂]
  \begin{phonetics}{封斋}{feng1zhai1}
    \definition*{s.}{Ramadã (Islã)}
  \end{phonetics}
\end{entry}

\begin{entry}{封盖}{9,11}[Radicais ⼨、⽫]
  \begin{phonetics}{封盖}{feng1gai4}
    \definition{s.}{boné | capa | selo}
    \definition{v.}{cobrir}
  \end{phonetics}
\end{entry}

\begin{entry}{将来}{9,7}[Radicais ⼨、⽊]
  \begin{phonetics}{将来}{jiang1lai2}[][HSK 3]
    \definition[个]{s.}{futuro}
  \end{phonetics}
\end{entry}

\begin{entry}{将近}{9,7}[Radicais ⼨、⾡]
  \begin{phonetics}{将近}{jiang1jin4}[][HSK 3]
    \definition{adv.}{quase}
  \end{phonetics}
\end{entry}

\begin{entry}{将要}{9,9}[Radicais ⼨、⾑]
  \begin{phonetics}{将要}{jiang1yao4}
    \definition{adv.}{vai | deve}
  \end{phonetics}
\end{entry}

\begin{entry}{屋子}{9,3}[Radicais ⼫、⼦]
  \begin{phonetics}{屋子}{wu1zi5}[][HSK 3]
    \definition[间,座,栋]{s.}{casa}
  \end{phonetics}
\end{entry}

\begin{entry}{屌丝}{9,5}[Radicais ⼫、⼀]
  \begin{phonetics}{屌丝}{diao3si1}
    \definition{adj.}{panaca | zé-ninguém | (gíria de \emph{Internet}) \emph{looser}}
  \end{phonetics}
\end{entry}

\begin{entry}{屎}{9}[Radical ⼫]
  \begin{phonetics}{屎}{shi3}
    \definition{s.}{fezes | excrementos | (forma ligada) secreção (do ouvido, olho, etc.)}
  \end{phonetics}
\end{entry}

\begin{entry}{差}{9}[Radical ⼯]
  \begin{phonetics}{差}{cha4}[][HSK 1]
    \definition{adv.}{ligeiramente | comparativamente | um pouco}
    \definition{s.}{differença | dissimilaridade | engano | equívoco}
  \end{phonetics}
\end{entry}

\begin{entry}{差不多}{9,4,6}[Radicais ⼯、⼀、⼣]
  \begin{phonetics}{差不多}{cha4bu5duo1}[][HSK 2]
    \definition{adj.}{mais ou menos}
    \definition{adv.}{quase perto}
  \end{phonetics}
\end{entry}

\begin{entry}{差点儿}{9,9,2}[Radicais ⼯、⽕、⼉]
  \begin{phonetics}{差点儿}{cha4dian3r5}
    \definition{adv.}{por pouco | por um triz | quase}
  \end{phonetics}
\end{entry}

\begin{entry}{帝国}{9,8}[Radicais ⼱、⼞]
  \begin{phonetics}{帝国}{di4guo2}
    \definition{adj.}{imperial}
    \definition{s.}{império}
  \end{phonetics}
\end{entry}

\begin{entry}{带}{9}[Radical ⼱]
  \begin{phonetics}{带}{dai4}[][HSK 2]
    \definition{v.}{levar | trazer}
  \end{phonetics}
\end{entry}

\begin{entry}{带动}{9,6}[Radicais ⼱、⼒]
  \begin{phonetics}{带动}{dai4 dong4}[][HSK 3]
    \definition{v.}{dirigir; ativar; fazer algo funcionar; acionar | liderar; trazer; estimular; motivar; atrair}
  \end{phonetics}
\end{entry}

\begin{entry}{带来}{9,7}[Radicais ⼱、⽊]
  \begin{phonetics}{带来}{dai4 lai2}[][HSK 2]
    \definition{v.}{trazer | (figurativo) provocar, produzir}
  \end{phonetics}
\end{entry}

\begin{entry}{带领}{9,11}[Radicais ⼱、⾴]
  \begin{phonetics}{带领}{dai4ling3}[][HSK 3]
    \definition{v.}{guiar | liderar}
  \end{phonetics}
\end{entry}

\begin{entry}{帮}{9}[Radical ⼱]
  \begin{phonetics}{帮}{bang1}[][HSK 1]
    \definition{clas.}{para alguém (como uma ajuda)}
    \definition{s.}{gangue | grupo | contratado (como trabalhador) | camada externa | festa | sociedade secreta}
    \definition{v.}{ajudar | apoiar}
  \end{phonetics}
\end{entry}

\begin{entry}{帮忙}{9,6}[Radicais ⼱、⼼]
  \begin{phonetics}{帮忙}{bang1 mang2}[][HSK 1]
    \definition{v.+compl.}{ajudar | dar uma mão | estender a mão | fazer um favor}
  \end{phonetics}
\end{entry}

\begin{entry}{帮佣}{9,7}[Radicais ⼱、⼈]
  \begin{phonetics}{帮佣}{bang1yong1}
    \definition{s.}{ajudante doméstico | servo}
  \end{phonetics}
\end{entry}

\begin{entry}{帮助}{9,7}[Radicais ⼱、⼒]
  \begin{phonetics}{帮助}{bang1zhu4}[][HSK 2]
    \definition[种]{s.}{ajuda | assistência}
    \definition{v.}{ajudar | dar assistência}
  \end{phonetics}
\end{entry}

\begin{entry}{帮教}{9,11}[Radicais ⼱、⽁]
  \begin{phonetics}{帮教}{bang1jiao4}
    \definition{v.}{orientar}
  \end{phonetics}
\end{entry}

\begin{entry}{度}{9}[Radical ⼴]
  \begin{phonetics}{度}{du4}[][HSK 2]
    \definition{clas.}{para temperatura, etc. | para eventos e ocorrências}
    \definition{s.}{grau (ângulo, temperatura, etc.) | kilowatt-hora}
  \end{phonetics}
  \begin{phonetics}{度}{duo2}
    \definition{v.}{estimar}
  \end{phonetics}
\end{entry}

\begin{entry}{度过}{9,6}[Radicais ⼴、⾡]
  \begin{phonetics}{度过}{du4guo4}[][HSK 4]
    \definition{s.}{passar o tempo; fazer o tempo desaparecer no trabalho, na vida, no lazer e no descanso}
  \end{phonetics}
\end{entry}

\begin{entry}{弯}{9}[Radical ⼸]
  \begin{phonetics}{弯}{wan1}[][HSK 4]
    \definition{adj.}{curvo; dobrado; torto; flexível; tortuoso}
    \definition{s.}{curva; dobra}
    \definition{v.}{curvar; dobrar; flexionar}
  \end{phonetics}
\end{entry}

\begin{entry}{待遇}{9,12}[Radicais ⼻、⾡]
  \begin{phonetics}{待遇}{dai4yu4}[][HSK 4]
    \definition[种,项,份]{s.}{tratamento; refere-se a direitos, status social, etc. | salário; ordenado; remuneração}
  \end{phonetics}
\end{entry}

\begin{entry}{很}{9}[Radical ⼻]
  \begin{phonetics}{很}{hen3}[][HSK 1]
    \definition{adv.}{bastante | muito | terrivelmente | advérbio de grau}
  \end{phonetics}
\end{entry}

\begin{entry}{律师}{9,6}[Radicais ⼻、⼱]
  \begin{phonetics}{律师}{lv4shi1}[][HSK 4]
    \definition[名,个,位]{s.}{advogado; procurador; profissionais encarregados pelas partes ou nomeados pelo tribunal para auxiliar as partes no litígio, para comparecer ao tribunal para defesa e para tratar de assuntos jurídicos relacionados, de acordo com a lei}
  \end{phonetics}
\end{entry}

\begin{entry}{怎}{9}[Radical ⼼]
  \begin{phonetics}{怎}{zen3}
    \definition{adv.}{como}
  \end{phonetics}
\end{entry}

\begin{entry}{怎么}{9,3}[Radicais ⼼、⼃]
  \begin{phonetics}{怎么}{zen3me5}[][HSK 1]
    \definition{pron.}{como? | o que?}
  \end{phonetics}
\end{entry}

\begin{entry}{怎么了}{9,3,2}[Radicais ⼼、⼃、⼅]
  \begin{phonetics}{怎么了}{zen3me5le5}
    \definition{expr.}{O que aconteceu? | O que está acontecendo? | E aí?}
  \end{phonetics}
\end{entry}

\begin{entry}{怎么办}{9,3,4}[Radicais ⼼、⼃、⼒]
  \begin{phonetics}{怎么办}{zen3 me5 ban4}[][HSK 2]
    \definition{adv.}{o que fazer?}
  \end{phonetics}
\end{entry}

\begin{entry}{怎么回事}{9,3,6,8}[Radicais ⼼、⼃、⼞、⼅]
  \begin{phonetics}{怎么回事}{zen3me5hui2shi4}
    \definition{expr.}{O que aconteceu? | O que se passou?}
  \end{phonetics}
\end{entry}

\begin{entry}{怎么样}{9,3,10}[Radicais ⼼、⼃、⽊]
  \begin{phonetics}{怎么样}{zen3me5yang4}[][HSK 2]
    \definition{adv.}{como? | que tal?}
  \end{phonetics}
\end{entry}

\begin{entry}{怎么得了}{9,3,11,2}[Radicais ⼼、⼃、⼻、⼅]
  \begin{phonetics}{怎么得了}{zen3me5de2liao3}
    \definition{expr.}{Como isso pode ser? | Que bagunça horrível! | O que deve ser feito?}
  \end{phonetics}
\end{entry}

\begin{entry}{怎么搞的}{9,3,13,8}[Radicais ⼼、⼃、⼿、⽩]
  \begin{phonetics}{怎么搞的}{zen3me5gao3de5}
    \definition{expr.}{Como isso aconteceu? | O que deu errado? | E aí? | O que está errado?}
  \end{phonetics}
\end{entry}

\begin{entry}{怎样}{9,10}[Radicais ⼼、⽊]
  \begin{phonetics}{怎样}{zen3 yang4}[][HSK 2]
    \definition{pron.}{como | o que | de uma certa maneira | de qualquer maneira | não importa o quão}
  \end{phonetics}
\end{entry}

\begin{entry}{怒骂}{9,9}[Radicais ⼼、⾺]
  \begin{phonetics}{怒骂}{nu4ma4}
    \definition{v.}{praguejar de raiva}
  \end{phonetics}
\end{entry}

\begin{entry}{思考}{9,6}[Radicais ⼼、⽼]
  \begin{phonetics}{思考}{si1kao3}[][HSK 4]
    \definition{v.}{pensar; ponderar; considerar; deliberar; envolver-se em atividades de pensamento, como análise, síntese, julgamento, raciocínio e generalização}
  \end{phonetics}
\end{entry}

\begin{entry}{思想}{9,13}[Radicais ⼼、⼼]
  \begin{phonetics}{思想}{si1xiang3}[][HSK 3]
    \definition[个]{s.}{reflexão; pensamento; ideologia | ideia}
  \end{phonetics}
\end{entry}

\begin{entry}{急}{9}[Radical ⼼]
  \begin{phonetics}{急}{ji2}[][HSK 2]
    \definition{adj.}{impaciente |ansioso | irritado | aborrecido |violento | urgente | premente}
    \definition{s.}{urgência | emergência}
    \definition{v.}{preocupar | estar ansioso para ajudar}
  \end{phonetics}
\end{entry}

\begin{entry}{急忙}{9,6}[Radicais ⼼、⼼]
  \begin{phonetics}{急忙}{ji2mang2}[][HSK 4]
    \definition{adv.}{apressadamente; com pressa}
  \end{phonetics}
\end{entry}

\begin{entry}{急救}{9,11}[Radicais ⼼、⽁]
  \begin{phonetics}{急救}{ji2jiu4}
    \definition{s.}{primeiros socorros}
    \definition{v.}{dar tratamento de emergência}
  \end{phonetics}
\end{entry}

\begin{entry}{怹}{9}[Radical ⼼]
  \begin{phonetics}{怹}{tan1}
    \definition{pron.}{ele, ela (cortês, em oposição a 他)}
    \seeref{他}{ta1}
  \end{phonetics}
\end{entry}

\begin{entry}{总}{9}[Radical ⼼]
  \begin{phonetics}{总}{zong3}[][HSK 3]
    \definition{adj.}{total; geral; global | responsável (liderança)}
    \definition{adv.}{sempre; invariavelmente | de qualquer forma; afinal; eventualmente; mais cedo ou mais tarde | seguramente; provavelmente; certamente}
    \definition{v.}{resumir; juntar; reunir}
  \end{phonetics}
\end{entry}

\begin{entry}{总长}{9,4}[Radicais ⼼、⾧]
  \begin{phonetics}{总长}{zong3chang2}
    \definition{s.}{comprimento total}
  \end{phonetics}
\end{entry}

\begin{entry}{总务}{9,5}[Radicais ⼼、⼒]
  \begin{phonetics}{总务}{zong3wu4}
    \definition{s.}{divisão de assuntos gerais | assuntos gerais | pessoa responsável geral}
  \end{phonetics}
\end{entry}

\begin{entry}{总台}{9,5}[Radicais ⼼、⼝]
  \begin{phonetics}{总台}{zong3tai2}
    \definition{s.}{recepção | balcão de recepção}
  \end{phonetics}
\end{entry}

\begin{entry}{总价}{9,6}[Radicais ⼼、⼈]
  \begin{phonetics}{总价}{zong3jia4}
    \definition{s.}{preço total}
  \end{phonetics}
\end{entry}

\begin{entry}{总线}{9,8}[Radicais ⼼、⽷]
  \begin{phonetics}{总线}{zong3xian4}
    \definition{s.}{barramento (computador) | \emph{computer bus}}
  \end{phonetics}
\end{entry}

\begin{entry}{总是}{9,9}[Radicais ⼼、⽇]
  \begin{phonetics}{总是}{zong3shi4}[][HSK 3]
    \definition{adv.}{sempre; indica que algo está acontecendo por um período de tempo; um certo estado permanece inalterado
 | afinal; significa que não importa o que aconteça, haverá um resultado.}
  \end{phonetics}
\end{entry}

\begin{entry}{总结}{9,9}[Radicais ⼼、⽷]
  \begin{phonetics}{总结}{zong3jie2}[][HSK 3]
    \definition[个,份]{s.}{resumo; conclusão obtida}
    \definition{v.}{resumir; sumariar; analisar a experiência da pesquisa e tirar conclusões}
  \end{phonetics}
\end{entry}

\begin{entry}{总统}{9,9}[Radicais ⼼、⽷]
  \begin{phonetics}{总统}{zong3tong3}
    \definition*[个,位,名,届]{s.}{Presidente (de um país)}
  \end{phonetics}
\end{entry}

\begin{entry}{总值}{9,10}[Radicais ⼼、⼈]
  \begin{phonetics}{总值}{zong3zhi2}
    \definition{s.}{valor total}
  \end{phonetics}
\end{entry}

\begin{entry}{总站}{9,10}[Radicais ⼼、⽴]
  \begin{phonetics}{总站}{zong3zhan4}
    \definition{s.}{terminal}
  \end{phonetics}
\end{entry}

\begin{entry}{总得}{9,11}[Radicais ⼼、⼻]
  \begin{phonetics}{总得}{zong3dei3}
    \definition{adv.}{prestes a}
    \definition{v.}{dever | precisar}
  \end{phonetics}
\end{entry}

\begin{entry}{总理}{9,11}[Radicais ⼼、⽟]
  \begin{phonetics}{总理}{zong3li3}
    \definition*{s.}{Primeiro-Ministro}
  \end{phonetics}
\end{entry}

\begin{entry}{总督}{9,13}[Radicais ⼼、⽬]
  \begin{phonetics}{总督}{zong3du1}
    \definition*{s.}{Governador-Geral | Governador | Vice-Rei}
  \end{phonetics}
\end{entry}

\begin{entry}{恒星系}{9,9,7}[Radicais ⼼、⽇、⽷]
  \begin{phonetics}{恒星系}{heng2xing1xi4}
    \definition{s.}{sistema estelar | galáxia}
  \end{phonetics}
\end{entry}

\begin{entry}{T-恤}{9}[Radical ⼼]
  \begin{phonetics}{T-恤}{xu4}
    \definition{s.}{camiseta | pulôver | suéter}
  \end{phonetics}
\end{entry}

\begin{entry}{恨}{9}[Radical ⼼]
  \begin{phonetics}{恨}{hen4}
    \definition{s.}{ódio}
    \definition{v.}{odiar}
  \end{phonetics}
\end{entry}

\begin{entry}{恰}{9}[Radical ⼼]
  \begin{phonetics}{恰}{qia4}
    \definition{adv.}{exatamente | apenas}
  \end{phonetics}
\end{entry}

\begin{entry}{恰好}{9,6}[Radicais ⼼、⼥]
  \begin{phonetics}{恰好}{qia4hao3}
    \definition{adv.}{certo | por sorte | ao que parece | por sorte coincidência}
  \end{phonetics}
\end{entry}

\begin{entry}{恰到好处}{9,8,6,5}[Radicais ⼼、⼑、⼥、⼡]
  \begin{phonetics}{恰到好处}{qia4dao4hao3chu4}
    \definition{expr.}{é simplesmente perfeito | é simplesmente correto}
  \end{phonetics}
\end{entry}

\begin{entry}{战}{9}[Radical ⼽]
  \begin{phonetics}{战}{zhan4}
    \definition{s.}{luta | guerra | batalha}
    \definition{v.}{lutar}
  \end{phonetics}
\end{entry}

\begin{entry}{战士}{9,3}[Radicais ⼽、⼠]
  \begin{phonetics}{战士}{zhan4shi4}
    \definition[个]{s.}{lutador | soldado | guerreiro}
  \end{phonetics}
\end{entry}

\begin{entry}{战争}{9,6}[Radicais ⼽、⼑]
  \begin{phonetics}{战争}{zhan4zheng1}
    \definition[場,次]{s.}{guerra | conflito}
  \end{phonetics}
\end{entry}

\begin{entry}{括号}{9,5}[Radicais ⼿、⼝]
  \begin{phonetics}{括号}{kuo4 hao4}[][HSK 4]
    \definition{s.}{chaves, colchetes e parênteses (em fórmulas aritméticas ou algébricas, os símbolos que indicam a combinação e a ordem de vários números ou termos) | colchetes e parênteses usados como um tipo de sinal de pontuação para mostrar a parte explicativa de uma passagem em um texto}
  \end{phonetics}
\end{entry}

\begin{entry}{拼}{9}[Radical ⼿]
  \begin{phonetics}{拼}{pin1}
    \definition{v.}{soletrar | juntar | unir}
  \end{phonetics}
\end{entry}

\begin{entry}{拼命}{9,8}[Radicais ⼿、⼝]
  \begin{phonetics}{拼命}{pin1ming4}
    \definition{adv.}{com toda a força | desesperadamente}
    \definition{v.+compl.}{arriscar a vida de alguém | desafiar a morte | colocar-se em uma luta desesperada | fazer algo desesperadamente | exercer a maior força}
  \end{phonetics}
\end{entry}

\begin{entry}{拼音}{9,9}[Radicais ⼿、⾳]
  \begin{phonetics}{拼音}{pin1yin1}
    \definition{s.}{escrita fonética | pinyin (romanização chinesa)}
  \end{phonetics}
\end{entry}

\begin{entry}{持续}{9,11}[Radicais ⼿、⽷]
  \begin{phonetics}{持续}{chi2xu4}[][HSK 3]
    \definition{v.}{durar; continuar; sustentar}
  \end{phonetics}
\end{entry}

\begin{entry}{挂}{9}[Radical ⼿]
  \begin{phonetics}{挂}{gua4}[][HSK 3]
    \definition{clas.}{para conjuntos ou sequência de itens}
    \definition{v.}{pendurar; colocar; suspender | interromper chamada (telefônica) | colocar alguém em contato com; ligar; telefonar
pegar carona; ser pego | ter em mente; estar preocupado com | ser revestido com; ser coberto com | colocar em registro; registrar}
  \end{phonetics}
\end{entry}

\begin{entry}{挂号}{9,5}[Radicais ⼿、⼝]
  \begin{phonetics}{挂号}{gua4hao4}
    \definition{v.+compl.}{registrar-se (em um hospital, etc.) | enviar através de carta registrada}
  \end{phonetics}
\end{entry}

\begin{entry}{挂号信}{9,5,9}[Radicais ⼿、⼝、⼈]
  \begin{phonetics}{挂号信}{gua4hao4xin4}
    \definition{s.}{carta registrada}
  \end{phonetics}
\end{entry}

\begin{entry}{指}{9}[Radical ⼿]
  \begin{phonetics}{指}{zhi3}[][HSK 3]
    \definition*{s.}{sobrenome Zhi}
    \definition{clas.}{dígito; largura do dedo; a largura de um dedo é chamada de ``一指'', que é usado para medir profundidade, largura, etc.}
    \definition{s.}{dedo}
    \definition{v.}{apontar para | (pelo) eriçar | indicar; mostrar-se; apontar; demonstrar | referir-se a; dirigir-se a | confiar em; contar com; depender de | criticar; repreender}
  \end{phonetics}
\end{entry}

\begin{entry}{指出}{9,5}[Radicais ⼿、⼐]
  \begin{phonetics}{指出}{zhi3 chu1}[][HSK 3]
    \definition{v.}{apontar; indicar}
  \end{phonetics}
\end{entry}

\begin{entry}{指甲}{9,5}[Radicais ⼿、⽥]
  \begin{phonetics}{指甲}{zhi3jia5}
    \definition{s.}{unha da mão}
  \end{phonetics}
\end{entry}

\begin{entry}{指导}{9,6}[Radicais ⼿、⼨]
  \begin{phonetics}{指导}{zhi3dao3}[][HSK 3]
    \definition{s.}{guia; pessoa que faz trabalho de orientação}
    \definition{v.}{guiar; dirigir; instruir}
  \end{phonetics}
\end{entry}

\begin{entry}{指南针}{9,9,7}[Radicais ⼿、⼗、⾦]
  \begin{phonetics}{指南针}{zhi3nan2zhen1}
    \definition{s.}{bússola}
  \end{phonetics}
\end{entry}

\begin{entry}{指挥}{9,9}[Radicais ⼿、⼿]
  \begin{phonetics}{指挥}{zhi3hui1}
    \definition[个]{s.}{condutor (de uma orquestra)}
    \definition{v.}{conduzir | comandar | direcionar}
  \end{phonetics}
\end{entry}

\begin{entry}{按}{9}[Radical ⼿]
  \begin{phonetics}{按}{an4}[][HSK 3]
    \definition{v.}{pressionar | empurrar para baixo | deixar de lado | arquivar | restringir | controlar}
  \end{phonetics}
\end{entry}

\begin{entry}{按时}{9,7}[Radicais ⼿、⽇]
  \begin{phonetics}{按时}{an4shi2}[][HSK 4]
    \definition{adv.}{na hora; no horário; pontualmente; de acordo com o tempo estipulado}
  \end{phonetics}
\end{entry}

\begin{entry}{按照}{9,13}[Radicais ⼿、⽕]
  \begin{phonetics}{按照}{an4zhao4}[][HSK 3]
    \definition{prep.}{de acordo com | em conformidade com | à luz de | com base em}
  \end{phonetics}
\end{entry}

\begin{entry}{挑}{9}[Radical ⼿]
  \begin{phonetics}{挑}{tiao1}[][HSK 4]
    \definition{clas.}{para coisas que são escolhidas ou selecionadas | para coisas que podem ser usadas como palhetas}
    \definition{s.}{vara comprida com algo pendurado nas pontas; haste de transporte}
    \definition{v.}{escolher; selecionar | fazer picuinhas; ser hipercrítico; ser meticuloso; ser excessivamente rigoroso nos detalhes | carregar com uma haste de transporte; carregar no ombro; pendurar coisas nas pontas de varas longas e carregá-las em seus ombros}
  \end{phonetics}
  \begin{phonetics}{挑}{tiao3}[][HSK 4]
    \definition{s.}{um dos traços dos caracteres chineses; inclinado para cima da esquerda para a direita}
    \definition{v.}{levantar; elevar; erguer | levantar ou apoiar com uma extremidade de uma vara ou objeto semelhante; segurar ou apoiar com a ponta de uma vara etc. | causar conflitos deliberadamente; provocar deliberadamente um conflito | (método de bordado) usar uma agulha para levantar os fios de urdidura ou trama, com a agulha e a linha passando por baixo para formar padrões e desenhos}
  \end{phonetics}
\end{entry}

\begin{entry}{挑战}{9,9}[Radicais ⼿、⼽]
  \begin{phonetics}{挑战}{tiao3zhan4}[][HSK 4]
    \definition{v.}{desafiar; deixar um oponente deliberadamente irritado e sair para lutar ou lutar consigo mesmo; estimular um oponente a lutar consigo mesmo}
  \end{phonetics}
\end{entry}

\begin{entry}{挑选}{9,9}[Radicais ⼿、⾡]
  \begin{phonetics}{挑选}{tiao1 xuan3}[][HSK 4]
    \definition{v.}{escolher; optar; selecionar; escolher a pessoa ou coisa certa para o trabalho}
  \end{phonetics}
\end{entry}

\begin{entry}{挑衅}{9,11}[Radicais ⼿、⾎]
  \begin{phonetics}{挑衅}{tiao3xin4}
    \definition{s.}{provocação}
    \definition{v.}{provocar}
  \end{phonetics}
\end{entry}

\begin{entry}{挖}{9}[Radical ⼿]
  \begin{phonetics}{挖}{wa1}
    \definition{v.}{cavar | escavar}
  \end{phonetics}
\end{entry}

\begin{entry}{挖掘机}{9,11,6}[Radicais ⼿、⼿、⽊]
  \begin{phonetics}{挖掘机}{wa1jue2ji1}
    \definition{s.}{escavadeira | escavador | escavadora | pá mecânica}
  \end{phonetics}
\end{entry}

\begin{entry}{挡风玻璃}{9,4,9,14}[Radicais ⼿、⾵、⽟、⽟]
  \begin{phonetics}{挡风玻璃}{dang3feng1bo1li5}
    \definition{s.}{parabrisa}
  \end{phonetics}
\end{entry}

\begin{entry}{挣}{9}[Radical ⼿]
  \begin{phonetics}{挣}{zheng4}
    \definition{v.}{ganhar dinheiro | esforçar-se para adquirir | lutar para se libertar}
  \end{phonetics}
\end{entry}

\begin{entry}{挣扎}{9,4}[Radicais ⼿、⼿]
  \begin{phonetics}{挣扎}{zheng1zha2}
    \definition{v.}{lutar}
  \end{phonetics}
\end{entry}

\begin{entry}{挣钱}{9,10}[Radicais ⼿、⾦]
  \begin{phonetics}{挣钱}{zheng4qian2}
    \definition{v.+compl.}{ganhar dinheiro}
  \end{phonetics}
\end{entry}

\begin{entry}{挣得}{9,11}[Radicais ⼿、⼻]
  \begin{phonetics}{挣得}{zheng4de2}
    \definition{v.}{ganhar renda ou dinheiro}
  \end{phonetics}
\end{entry}

\begin{entry}{挥汗如雨}{9,6,6,8}[Radicais ⼿、⽔、⼥、⾬]
  \begin{phonetics}{挥汗如雨}{hui1han4ru2yu3}
    \definition{s.}{suor derramado}
    \definition{v.}{pingar com suor}
  \end{phonetics}
\end{entry}

\begin{entry}{挺}{9}[Radical ⼿]
  \begin{phonetics}{挺}{ting3}[][HSK 2,4]
    \definition{adj.}{rígido; ereto; vertical; reto | distinto. que se destaca; que se sobressai; excepcional}
    \definition{adv.}{muito; bastante}
    \definition{clas.}{para metralhadoras}
    \definition{v.}{sobressair; endireitar-se; saliente ou protuberante | suportar; aguentar; ficar de pé; resistir}
  \end{phonetics}
\end{entry}

\begin{entry}{挺尸}{9,3}[Radicais ⼿、⼫]
  \begin{phonetics}{挺尸}{ting3shi1}
    \definition{v.}{(coloquial) dormir | (literalmente) ficar deitado duro como um cadáver}
  \end{phonetics}
\end{entry}

\begin{entry}{挺立}{9,5}[Radicais ⼿、⽴]
  \begin{phonetics}{挺立}{ting3li4}
    \definition{v.}{ficar ereto | ficar de pé}
  \end{phonetics}
\end{entry}

\begin{entry}{挺好}{9,6}[Radicais ⼿、⼥]
  \begin{phonetics}{挺好}{ting3 hao3}[][HSK 2]
    \definition{adj.}{muito bom}
  \end{phonetics}
\end{entry}

\begin{entry}{挺过}{9,6}[Radicais ⼿、⾡]
  \begin{phonetics}{挺过}{ting3guo4}
    \definition{s.}{sobreviver}
  \end{phonetics}
\end{entry}

\begin{entry}{挺住}{9,7}[Radicais ⼿、⼈]
  \begin{phonetics}{挺住}{ting3zhu4}
    \definition{v.}{permanecer firme | manter-se firme (diante da adversidade ou da dor)}
  \end{phonetics}
\end{entry}

\begin{entry}{挺杆}{9,7}[Radicais ⼿、⽊]
  \begin{phonetics}{挺杆}{ting3gan3}
    \definition{s.}{tucho (peça de máquina)}
  \end{phonetics}
\end{entry}

\begin{entry}{挺身}{9,7}[Radicais ⼿、⾝]
  \begin{phonetics}{挺身}{ting3shen1}
    \definition{v.}{endireitar as costas}
  \end{phonetics}
\end{entry}

\begin{entry}{挺进}{9,7}[Radicais ⼿、⾡]
  \begin{phonetics}{挺进}{ting3jin4}
    \definition{s.}{progresso | avanço}
    \definition{v.}{progredir | avançar}
  \end{phonetics}
\end{entry}

\begin{entry}{挺拔}{9,8}[Radicais ⼿、⼿]
  \begin{phonetics}{挺拔}{ting3ba2}
    \definition{adj.}{alto e reto}
  \end{phonetics}
\end{entry}

\begin{entry}{挺腰}{9,13}[Radicais ⼿、⾁]
  \begin{phonetics}{挺腰}{ting3yao1}
    \definition{v.}{arquear as costas | endireitar as costas}
  \end{phonetics}
\end{entry}

\begin{entry}{政纲}{9,7}[Radicais ⽁、⽷]
  \begin{phonetics}{政纲}{zheng4gang1}
    \definition{s.}{programa ou plataforma política}
  \end{phonetics}
\end{entry}

\begin{entry}{政府}{9,8}[Radicais ⽁、⼴]
  \begin{phonetics}{政府}{zheng4fu3}
    \definition[个]{s.}{governo}
  \end{phonetics}
\end{entry}

\begin{entry}{政治局}{9,8,7}[Radicais ⽁、⽔、⼫]
  \begin{phonetics}{政治局}{zheng4zhi4ju2}
    \definition{s.}{o principal comitê de políticas de um partido comunista}
  \end{phonetics}
\end{entry}

\begin{entry}{故}{9}[Radical ⽁]
  \begin{phonetics}{故}{gu4}
    \definition{conj.}{por isso | portanto | então}
  \end{phonetics}
\end{entry}

\begin{entry}{故乡}{9,3}[Radicais ⽁、⼄]
  \begin{phonetics}{故乡}{gu4xiang1}[][HSK 3]
    \definition[个]{s.}{cidade natal; terra natal}
  \end{phonetics}
\end{entry}

\begin{entry}{故事}{9,8}[Radicais ⽁、⼅]
  \begin{phonetics}{故事}{gu4shi4}
    \definition{s.}{prática antiga}
  \end{phonetics}
  \begin{phonetics}{故事}{gu4shi5}[][HSK 2]
    \definition{s.}{narrativa | história | conto}
  \end{phonetics}
\end{entry}

\begin{entry}{故宫}{9,9}[Radicais ⽁、⼧]
  \begin{phonetics}{故宫}{gu4gong1}
    \definition*{s.}{Palácio Imperial | Cidade Proibida}
  \end{phonetics}
\end{entry}

\begin{entry}{故意}{9,13}[Radicais ⽁、⼼]
  \begin{phonetics}{故意}{gu4yi4}[][HSK 2]
    \definition{adv.}{intencionalmente | deliberadamente | propositalmente}
  \end{phonetics}
\end{entry}

\begin{entry}{既}{9}[Radical ⽆]
  \begin{phonetics}{既}{ji4}[][HSK 4]
    \definition*{s.}{sobrenome Ji}
    \definition{adv.}{já}
    \definition{conj.}{desde; como; agora que | assim como; e também; ambos\dots e\dots; usado em conjunto com advérbios como ``且、又、也'' para indicar uma combinação de ambas as situações}
  \seealsoref{且}{qie3}
  \seealsoref{也}{ye3}
  \seealsoref{又}{you4}
  \end{phonetics}
\end{entry}

\begin{entry}{既又}{9,2}[Radicais ⽆、⼜]
  \begin{phonetics}{既又}{ji4you4}
    \definition{conj.}{desde | como | agora isso | os dois e | assim como}
  \end{phonetics}
\end{entry}

\begin{entry}{既不……又不……}{9,4,2,4}[Radicais ⽆、⼀、⼜、⼀]
  \begin{phonetics}{既不……又不……}{ji4bu4 you4bu4}
    \definition{conj.}{nem mesmo\dots}
  \end{phonetics}
\end{entry}

\begin{entry}{既然}{9,12}[Radicais ⽆、⽕]
  \begin{phonetics}{既然}{ji4ran2}[][HSK 4]
    \definition{conj.}{como; desde; agora que; usado na primeira metade de uma frase, muitas vezes repetido na segunda metade pelos advérbios ``就、也、还'' para indicar que a premissa é primeiro declarada e depois inferida}
  \seealsoref{还}{hai2}
  \seealsoref{就}{jiu4}
  \seealsoref{也}{ye3}
  \end{phonetics}
\end{entry}

\begin{entry}{星火}{9,4}[Radicais ⽇、⽕]
  \begin{phonetics}{星火}{xing1huo3}
    \definition{s.}{trilha de meteoro (usada principalmente em expressões como 急如星火) | faísca}
  \end{phonetics}
\end{entry}

\begin{entry}{星辰}{9,7}[Radicais ⽇、⾠]
  \begin{phonetics}{星辰}{xing1chen2}
    \definition{s.}{estrelas}
  \end{phonetics}
\end{entry}

\begin{entry}{星表}{9,8}[Radicais ⽇、⾐]
  \begin{phonetics}{星表}{xing1biao3}
    \definition{s.}{catálogo de estrelas}
  \end{phonetics}
\end{entry}

\begin{entry}{星星}{9,9}[Radicais ⽇、⽇]
  \begin{phonetics}{星星}{xing1 xing5}[][HSK 2]
    \definition{s.}{estrela}
  \end{phonetics}
\end{entry}

\begin{entry}{星座}{9,10}[Radicais ⽇、⼴]
  \begin{phonetics}{星座}{xing1zuo4}
    \definition[张]{s.}{signo astrológico | constelação}
  \end{phonetics}
\end{entry}

\begin{entry}{星期}{9,12}[Radicais ⽇、⽉]
  \begin{phonetics}{星期}{xing1qi1}[][HSK 1]
    \definition[个]{s.}{semana}
  \end{phonetics}
\end{entry}

\begin{entry}{星期一}{9,12,1}[Radicais ⽇、⽉、⼀]
  \begin{phonetics}{星期一}{xing1qi1yi1}[][HSK 1]
    \definition{s.}{segunda-feira}
  \end{phonetics}
\end{entry}

\begin{entry}{星期二}{9,12,2}[Radicais ⽇、⽉、⼆]
  \begin{phonetics}{星期二}{xing1qi1'er4}[][HSK 1]
    \definition{s.}{terça-feira}
  \end{phonetics}
\end{entry}

\begin{entry}{星期三}{9,12,3}[Radicais ⽇、⽉、⼀]
  \begin{phonetics}{星期三}{xing1qi1san1}[][HSK 1]
    \definition{s.}{quarta-feira}
  \end{phonetics}
\end{entry}

\begin{entry}{星期五}{9,12,4}[Radicais ⽇、⽉、⼆]
  \begin{phonetics}{星期五}{xing1qi1wu3}[][HSK 1]
    \definition{s.}{sexta-feira}
  \end{phonetics}
\end{entry}

\begin{entry}{星期六}{9,12,4}[Radicais ⽇、⽉、⼋]
  \begin{phonetics}{星期六}{xing1qi1liu4}[][HSK 1]
    \definition{s.}{sábado}
  \end{phonetics}
\end{entry}

\begin{entry}{星期天}{9,12,4}[Radicais ⽇、⽉、⼤]
  \begin{phonetics}{星期天}{xing1qi1tian1}[][HSK 1]
    \definition{s.}{domingo}
  \seealsoref{星期日}{xing1qi1ri4}
  \end{phonetics}
\end{entry}

\begin{entry}{星期日}{9,12,4}[Radicais ⽇、⽉、⽇]
  \begin{phonetics}{星期日}{xing1qi1ri4}[][HSK 1]
    \definition{s.}{domingo}
  \seealsoref{星期天}{xing1qi1tian1}
  \end{phonetics}
\end{entry}

\begin{entry}{星期四}{9,12,5}[Radicais ⽇、⽉、⼞]
  \begin{phonetics}{星期四}{xing1qi1si4}[][HSK 1]
    \definition{s.}{quinta-feira}
  \end{phonetics}
\end{entry}

\begin{entry}{春}{9}[Radical ⽇]
  \begin{phonetics}{春}{chun1}
    \definition*{s.}{sobrenome Chun}
    \definition{s.}{primavera | amor | luxúria | vida | vitalidade}
  \end{phonetics}
\end{entry}

\begin{entry}{春天}{9,4}[Radicais ⽇、⼤]
  \begin{phonetics}{春天}{chun1 tian1}
    \definition[个]{s.}{primavera}
  \end{phonetics}
\end{entry}

\begin{entry}{春节}{9,5}[Radicais ⽇、⾋]
  \begin{phonetics}{春节}{chun1 jie2}[][HSK 2]
    \definition*{s.}{Festival da Primavera (Ano Novo Chinês)}
  \end{phonetics}
\end{entry}

\begin{entry}{春季}{9,8}[Radicais ⽇、⼦]
  \begin{phonetics}{春季}{chun1 ji4}[][HSK 4]
    \definition{s.}{primavera; primeiro trimestre do ano, que na China se refere ao período de três meses entre o início da primavera e o início do verão, e também se refere aos três meses do calendário lunar, a saber, o primeiro, o segundo e o terceiro meses}
  \end{phonetics}
\end{entry}

\begin{entry}{昨}{9}[Radical ⽇]
  \begin{phonetics}{昨}{zuo2}
    \definition{s.}{ontem}
  \end{phonetics}
\end{entry}

\begin{entry}{昨天}{9,4}[Radicais ⽇、⼤]
  \begin{phonetics}{昨天}{zuo2tian1}[][HSK 1]
    \definition{adv.}{ontem}
  \end{phonetics}
\end{entry}

\begin{entry}{昨日}{9,4}[Radicais ⽇、⽇]
  \begin{phonetics}{昨日}{zuo2ri4}
    \definition{adv.}{ontem}
  \end{phonetics}
\end{entry}

\begin{entry}{昨夜}{9,8}[Radicais ⽇、⼣]
  \begin{phonetics}{昨夜}{zuo2ye4}
    \definition{adv.}{noite passada}
  \end{phonetics}
\end{entry}

\begin{entry}{昨晚}{9,11}[Radicais ⽇、⽇]
  \begin{phonetics}{昨晚}{zuo2wan3}
    \definition{adv.}{noite passada | ontem à noite}
  \end{phonetics}
\end{entry}

\begin{entry}{是}{9}[Radical ⽇]
  \begin{phonetics}{是}{shi4}[][HSK 1]
    \definition{adj.}{correto | certo | verdadeiro | (reconhecimento respeitoso de um comando) muito bem}
    \definition{adv.}{(advérbio para afirmação enfática)}
    \definition{v.}{ser (somente seguido por substantivos)}
  \end{phonetics}
\end{entry}

\begin{entry}{是否}{9,7}[Radicais ⽇、⼝]
  \begin{phonetics}{是否}{shi4fou3}[][HSK 4]
    \definition{adv.}{se; se ou não}
  \end{phonetics}
\end{entry}

\begin{entry}{是的}{9,8}[Radicais ⽇、⽩]
  \begin{phonetics}{是的}{shi4de5}
    \definition{adv.}{sim | está certo}
  \end{phonetics}
\end{entry}

\begin{entry}{显示}{9,5}[Radicais ⽇、⽰]
  \begin{phonetics}{显示}{xian3shi4}[][HSK 3]
    \definition{v.}{mostrar | exibir}
  \end{phonetics}
\end{entry}

\begin{entry}{显得}{9,11}[Radicais ⽇、⼻]
  \begin{phonetics}{显得}{xian3de5}[][HSK 3]
    \definition{v.}{parecer; aparecer}
  \end{phonetics}
\end{entry}

\begin{entry}{显著}{9,11}[Radicais ⽇、⽬]
  \begin{phonetics}{显著}{xian3zhu4}[][HSK 4]
    \definition{adj.}{notável; significativo; notável; extraordinário; muito óbvio; muito claramente demonstrado; muito facilmente visto ou sentido}
  \end{phonetics}
\end{entry}

\begin{entry}{显然}{9,12}[Radicais ⽇、⽕]
  \begin{phonetics}{显然}{xian3ran2}[][HSK 3]
    \definition{adj.}{claro; evidente; óbvio}
    \definition{adv.}{claramente; evidentemente; obviamente}
  \end{phonetics}
\end{entry}

\begin{entry}{枯木}{9,4}[Radicais ⽊、⽊]
  \begin{phonetics}{枯木}{ku1mu4}
    \definition{s.}{árvore morta | madeira morta}
  \end{phonetics}
\end{entry}

\begin{entry}{架}{9}[Radical ⽊]
  \begin{phonetics}{架}{jia4}[][HSK 3]
    \definition{clas.}{para coisas com pilares ou componentes mecânicos | quadrado (usado para montanhas)}
    \definition{s.}{quadro; prateleira; suporte | briga; discussão}
    \definition{v.}{colocar para cima; erigir | afastar; resistir | suportar; ajudar | sequestrar; levar alguém embora à força}
  \end{phonetics}
\end{entry}

\begin{entry}{架式}{9,6}[Radicais ⽊、⼷]
  \begin{phonetics}{架式}{jia4shi5}
    \variantof{架势}
  \end{phonetics}
\end{entry}

\begin{entry}{架势}{9,8}[Radicais ⽊、⼒]
  \begin{phonetics}{架势}{jia4shi5}
    \definition{s.}{postura | atitude | posição (sobre um assunto, etc.)}
  \end{phonetics}
\end{entry}

\begin{entry}{柏树}{9,9}[Radicais ⽊、⽊]
  \begin{phonetics}{柏树}{bai3shu4}
    \definition{s.}{cipreste}
  \end{phonetics}
\end{entry}

\begin{entry}{某}{9}[Radical ⽊]
  \begin{phonetics}{某}{mou3}[][HSK 3]
    \definition{pron.}{um certo alguém ou coisa; algum | usado para substituir seu próprio nome}
  \end{phonetics}
\end{entry}

\begin{entry}{柔软}{9,8}[Radicais ⽊、⾞]
  \begin{phonetics}{柔软}{rou2ruan3}
    \definition{adj.}{macio | suave}
  \end{phonetics}
\end{entry}

\begin{entry}{柠檬}{9,17}[Radicais ⽊、⽊]
  \begin{phonetics}{柠檬}{ning2meng2}
    \definition{s.}{limão}
  \end{phonetics}
\end{entry}

\begin{entry}{查}{9}[Radical ⽊]
  \begin{phonetics}{查}{cha2}[][HSK 2]
    \definition{v.}{verificar | examinar | investigar |consultar}
  \end{phonetics}
  \begin{phonetics}{查}{zha1}
    \definition*{s.}{sobrenome Zha}
    \definition{s.}{espinheiro}
  \end{phonetics}
\end{entry}

\begin{entry}{柬埔寨}{9,10,14}[Radicais ⽊、⼟、⼧]
  \begin{phonetics}{柬埔寨}{jian3pu3zhai4}
    \definition*{s.}{Camboja}
  \end{phonetics}
\end{entry}

\begin{entry}{柳}{9}[Radical ⽊]
  \begin{phonetics}{柳}{liu3}
    \definition*{s.}{sobrenome Liu}
    \definition{s.}{salgueiro}
  \end{phonetics}
\end{entry}

\begin{entry}{柳橙汁}{9,16,5}[Radicais ⽊、⽊、⽔]
  \begin{phonetics}{柳橙汁}{liu3cheng2zhi1}
    \definition[瓶,杯,罐,盒]{s.}{suco de laranja}
  \seealsoref{橙汁}{cheng2zhi1}
  \seealsoref{橘子汁}{ju2zi5zhi1}
  \end{phonetics}
\end{entry}

\begin{entry}{标志}{9,7}[Radicais ⽊、⼼]
  \begin{phonetics}{标志}{biao1zhi4}[][HSK 4]
    \definition[个,种]{s.}{sinal; marca; logotipo; símbolo; emblema; marcações que caracterizam um objeto para facilitar a identificação}
    \definition{v.}{marcar; indicar; simbolizar; identificar}
  \end{phonetics}
\end{entry}

\begin{entry}{标准}{9,10}[Radicais ⽊、⼎]
  \begin{phonetics}{标准}{biao1zhun3}[][HSK 3]
    \definition{adj.}{criterioso | padronizado | normatizado}
    \definition[个]{s.}{critério | padrão (oficial) | norma}
  \end{phonetics}
\end{entry}

\begin{entry}{标题}{9,15}[Radicais ⽊、⾴]
  \begin{phonetics}{标题}{biao1ti2}[][HSK 3]
    \definition[个,条,篇]{s.}{título | manchete | cabeçalho}
  \end{phonetics}
\end{entry}

\begin{entry}{树}{9}[Radical ⽊]
  \begin{phonetics}{树}{shu4}[][HSK 1]
    \definition[棵]{s.}{árvore}
    \definition{v.}{cultivar}
  \end{phonetics}
\end{entry}

\begin{entry}{树木}{9,4}[Radicais ⽊、⽊]
  \begin{phonetics}{树木}{shu4mu4}
    \definition{s.}{árvore}
  \end{phonetics}
\end{entry}

\begin{entry}{树叶}{9,5}[Radicais ⽊、⼝]
  \begin{phonetics}{树叶}{shu4ye4}[][HSK 4]
    \definition[片,枚,堆]{s.}{folha; folhagem;}
  \end{phonetics}
\end{entry}

\begin{entry}{树林}{9,8}[Radicais ⽊、⽊]
  \begin{phonetics}{树林}{shu4 lin2}[][HSK 4]
    \definition{s.}{bosque; muitas árvores que crescem em fragmentos, menores que as florestas}
  \end{phonetics}
\end{entry}

\begin{entry}{树莓}{9,10}[Radicais ⽊、⾋]
  \begin{phonetics}{树莓}{shu4mei2}
    \definition{s.}{framboesa}
  \end{phonetics}
\end{entry}

\begin{entry}{歪}{9}[Radical ⽌]
  \begin{phonetics}{歪}{wai1}
    \definition{adj.}{torto | tortuoso | nocivo}
  \end{phonetics}
\end{entry}

\begin{entry}{歪果仁}{9,8,4}[Radicais ⽌、⽊、⼈]
  \begin{phonetics}{歪果仁}{wai1guo3ren2}
    \definition{s.}{gíria na \emph{Internet} para 外国人}
    \seeref{外国人}{wai4guo2ren2}
  \end{phonetics}
\end{entry}

\begin{entry}{残疾人}{9,10,2}[Radicais ⽍、⽧、⼈]
  \begin{phonetics}{残疾人}{can2ji2ren2}
    \definition{s.}{pessoa com deficiência}
  \end{phonetics}
\end{entry}

\begin{entry}{残酷}{9,14}[Radicais ⽍、⾣]
  \begin{phonetics}{残酷}{can2ku4}
    \definition{adj.}{cruel}
    \definition{s.}{crueldade}
  \end{phonetics}
\end{entry}

\begin{entry}{段}{9}[Radical ⽎]
  \begin{phonetics}{段}{duan4}[][HSK 2]
    \definition*{s.}{sobrenome Duan}
    \definition{clas.}{para histórias, períodos de tempo, desenvolvimento de um tópico, etc.}
    \definition{s.}{parágrafo | seção | segmento | estágio (de um processo)}
  \end{phonetics}
\end{entry}

\begin{entry}{毒}{9}[Radical ⽏]
  \begin{phonetics}{毒}{du2}
    \definition{adj.}{venenoso | tóxico}
    \definition{s.}{veneno | tóxico}
    \definition{v.}{intoxicar}
  \end{phonetics}
\end{entry}

\begin{entry}{毒杀}{9,6}[Radicais ⽏、⽊]
  \begin{phonetics}{毒杀}{du2sha1}
    \definition{v.}{matar por envenenamento}
  \end{phonetics}
\end{entry}

\begin{entry}{毒物}{9,8}[Radicais ⽏、⽜]
  \begin{phonetics}{毒物}{du2wu4}
    \definition{s.}{substância venenosa | toxina}
  \end{phonetics}
\end{entry}

\begin{entry}{毒害}{9,10}[Radicais ⽏、⼧]
  \begin{phonetics}{毒害}{du2hai4}
    \definition{s.}{envenenamento}
    \definition{v.}{envenenar (prejudicar com uma substância tóxica) | envenenar (as mentes das pessoas)}
  \end{phonetics}
\end{entry}

\begin{entry}{毒蛇}{9,11}[Radicais ⽏、⾍]
  \begin{phonetics}{毒蛇}{du2she2}
    \definition{s.}{víbora | cobra venenosa}
  \end{phonetics}
\end{entry}

\begin{entry}{洋葱}{9,12}[Radicais ⽔、⾋]
  \begin{phonetics}{洋葱}{yang2cong1}
    \definition{s.}{cebola}
  \end{phonetics}
\end{entry}

\begin{entry}{洒水}{9,4}[Radicais ⽔、⽔]
  \begin{phonetics}{洒水}{sa3shui3}
    \definition{v.}{borrifar}
  \end{phonetics}
\end{entry}

\begin{entry}{洗}{9}[Radical ⽔]
  \begin{phonetics}{洗}{xi3}[][HSK 1]
    \definition{v.}{lavar | revelar (fotos) | tomar banho}
  \end{phonetics}
\end{entry}

\begin{entry}{洗手}{9,4}[Radicais ⽔、⼿]
  \begin{phonetics}{洗手}{xi3shou3}
    \definition{v.}{ir ao banheiro | lavar as mãos}
  \end{phonetics}
\end{entry}

\begin{entry}{洗手不干}{9,4,4,3}[Radicais ⽔、⼿、⼀、⼲]
  \begin{phonetics}{洗手不干}{xi3shou3bu2gan4}
    \definition{v.}{parar totalmente de fazer algo}
  \end{phonetics}
\end{entry}

\begin{entry}{洗手池}{9,4,6}[Radicais ⽔、⼿、⽔]
  \begin{phonetics}{洗手池}{xi3shou3chi2}
    \definition{s.}{pia de banheiro | lavatório}
  \seealsoref{洗手盆}{xi3shou3pen2}
  \end{phonetics}
\end{entry}

\begin{entry}{洗手间}{9,4,7}[Radicais ⽔、⼿、⾨]
  \begin{phonetics}{洗手间}{xi3shou3jian1}[][HSK 1]
    \definition{s.}{sanitário | toilette | banheiro}
  \end{phonetics}
\end{entry}

\begin{entry}{洗手乳}{9,4,8}[Radicais ⽔、⼿、⼄]
  \begin{phonetics}{洗手乳}{xi3shou3ru3}
    \definition{s.}{sabonete líquido para lavar as mãos}
  \seealsoref{洗手液}{xi3shou3ye4}
  \end{phonetics}
\end{entry}

\begin{entry}{洗手盆}{9,4,9}[Radicais ⽔、⼿、⽫]
  \begin{phonetics}{洗手盆}{xi3shou3pen2}
    \definition{s.}{pia de banheiro | lavatório}
  \seealsoref{洗手池}{xi3shou3chi2}
  \end{phonetics}
\end{entry}

\begin{entry}{洗手液}{9,4,11}[Radicais ⽔、⼿、⽔]
  \begin{phonetics}{洗手液}{xi3shou3ye4}
    \definition{s.}{sabonete líquido para lavar as mãos}
  \seealsoref{洗手乳}{xi3shou3ru3}
  \end{phonetics}
\end{entry}

\begin{entry}{洗礼}{9,5}[Radicais ⽔、⽰]
  \begin{phonetics}{洗礼}{xi3li3}
    \definition{s.}{batismo}
    \definition{v.}{batizar}
  \end{phonetics}
\end{entry}

\begin{entry}{洗衣机}{9,6,6}[Radicais ⽔、⾐、⽊]
  \begin{phonetics}{洗衣机}{xi3 yi1 ji1}[][HSK 2]
    \definition[台]{s.}{máquina de lavar roupa}
  \end{phonetics}
\end{entry}

\begin{entry}{洗劫}{9,7}[Radicais ⽔、⼒]
  \begin{phonetics}{洗劫}{xi3jie2}
    \definition{v.}{saquear | pilhar | roubar}
  \end{phonetics}
\end{entry}

\begin{entry}{洗净}{9,8}[Radicais ⽔、⼎]
  \begin{phonetics}{洗净}{xi3jing4}
    \definition{v.}{lavar (limpeza)}
  \end{phonetics}
\end{entry}

\begin{entry}{洗胃}{9,9}[Radicais ⽔、⾁]
  \begin{phonetics}{洗胃}{xi3wei4}
    \definition{s.}{(medicina) lavagem gástrica}
    \definition{v.}{ter o estômago lavado}
  \end{phonetics}
\end{entry}

\begin{entry}{洗涤}{9,10}[Radicais ⽔、⽔]
  \begin{phonetics}{洗涤}{xi3di2}
    \definition{s.}{enxágue | lava}
    \definition{v.}{enxaguar | lavar}
  \end{phonetics}
\end{entry}

\begin{entry}{洗涤间}{9,10,7}[Radicais ⽔、⽔、⾨]
  \begin{phonetics}{洗涤间}{xi3di2jian1}
    \definition{s.}{lavanderia}
  \end{phonetics}
\end{entry}

\begin{entry}{洗脱}{9,11}[Radicais ⽔、⾁]
  \begin{phonetics}{洗脱}{xi3tuo1}
    \definition{v.}{limpar | purgar | lavar}
  \end{phonetics}
\end{entry}

\begin{entry}{洗碗}{9,13}[Radicais ⽔、⽯]
  \begin{phonetics}{洗碗}{xi3wan3}
    \definition{v.}{lavar pratos}
  \end{phonetics}
\end{entry}

\begin{entry}{洗澡}{9,16}[Radicais ⽔、⽔]
  \begin{phonetics}{洗澡}{xi3zao3}[][HSK 2]
    \definition{v.+compl.}{tomar banho | duchar-se | lavar-se}
  \end{phonetics}
\end{entry}

\begin{entry}{洗澡间}{9,16,7}[Radicais ⽔、⽔、⾨]
  \begin{phonetics}{洗澡间}{xi3zao3jian1}
    \definition[间]{s.}{banheiro}
  \end{phonetics}
\end{entry}

\begin{entry}{洞穴}{9,5}[Radicais ⽔、⽳]
  \begin{phonetics}{洞穴}{dong4xue2}
    \definition{s.}{caverna}
  \end{phonetics}
\end{entry}

\begin{entry}{洪水}{9,4}[Radicais ⽔、⽔]
  \begin{phonetics}{洪水}{hong2shui3}
    \definition{s.}{enchente | inundação | dilúvio}
  \end{phonetics}
\end{entry}

\begin{entry}{洲}{9}[Radical ⽔]
  \begin{phonetics}{洲}{zhou1}
    \definition{s.}{continente | ilha em um rio}
  \end{phonetics}
\end{entry}

\begin{entry}{活}{9}[Radical ⽔]
  \begin{phonetics}{活}{huo2}[][HSK 3]
    \definition{adj.}{vivo; vivendo | vívido; animado; ativo | móvel; em movimento}
    \definition{adv.}{exatamente; simplesmente}
    \definition{s.}{trabalho | produto}
    \definition{v.}{viver | salvar (a vida de uma pessoa)}
  \end{phonetics}
\end{entry}

\begin{entry}{活力}{9,2}[Radicais ⽔、⼒]
  \begin{phonetics}{活力}{huo2li4}
    \definition{s.}{energia | vitalidade | vigor | força vital}
  \end{phonetics}
\end{entry}

\begin{entry}{活动}{9,6}[Radicais ⽔、⼒]
  \begin{phonetics}{活动}{huo2dong4}[][HSK 2]
    \definition[项,个]{s.}{atividade | evento | campanha}
    \definition{v.}{exercer | operar}
  \end{phonetics}
\end{entry}

\begin{entry}{活着}{9,11}[Radicais ⽔、⽬]
  \begin{phonetics}{活着}{huo2zhe5}
    \definition{adj.}{vivo}
  \end{phonetics}
\end{entry}

\begin{entry}{活路}{9,13}[Radicais ⽔、⾜]
  \begin{phonetics}{活路}{huo2lu4}
    \definition{s.}{maneira de sobreviver | meio de subsistência}
  \end{phonetics}
  \begin{phonetics}{活路}{huo2lu5}
    \definition{s.}{labor | trabalho físico}
  \end{phonetics}
\end{entry}

\begin{entry}{派}{9}[Radical ⽔]
  \begin{phonetics}{派}{pai4}[][HSK 3]
    \definition{adj.}{elegante; bonito}
    \definition{clas.}{para grupos, escolas de pensamento ou arte, etc. | para um discursos, atmosferas, cenas, etc.}
    \definition{s.}{panelinha; grupo exclusivo; facção | torta | estilo | afluente; braço de rio}
    \definition{v.}{enviar; despachar | alocar; repartir; distribuir}
  \end{phonetics}
\end{entry}

\begin{entry}{测}{9}[Radical ⽔]
  \begin{phonetics}{测}{ce4}[][HSK 4]
    \definition{v.}{pesquisar; sondar; medir | conjecturar; inferir}
  \end{phonetics}
\end{entry}

\begin{entry}{测试}{9,8}[Radicais ⽔、⾔]
  \begin{phonetics}{测试}{ce4 shi4}[][HSK 4]
    \definition[个]{s.}{exame; teste; medição do conhecimento humano, das habilidades ou do funcionamento de máquinas, ferramentas ou instrumentos}
    \definition{v.}{examinar | testar, medição do desempenho e da precisão de máquinas, instrumentos, aparelhos, etc.}
  \end{phonetics}
\end{entry}

\begin{entry}{测量}{9,12}[Radicais ⽔、⾥]
  \begin{phonetics}{测量}{ce4liang2}[][HSK 4]
    \definition{v.}{aferir; pesquisar; medir; determinar valores relevantes para espaço, tempo, temperatura, velocidade, função, etc.}
  \end{phonetics}
\end{entry}

\begin{entry}{浓}{9}[Radical ⽔]
  \begin{phonetics}{浓}{nong2}[][HSK 4]
    \definition{adj.}{denso; espesso; concentrado; um líquido ou gás que contém mais de um determinado ingrediente | grande; forte; profundo (de grau ou extensão) | profundo; (algumas cores) escuro}
  \end{phonetics}
\end{entry}

\begin{entry}{点}{9}[Radical ⽕]
  \begin{phonetics}{点}{dian3}[][HSK 1]
    \definition{clas.}{para itens | hora cheia}
    \definition{s.}{ponto | gota | mancha | horas | ponto (no espaço ou no tempo) | traço de ponto em caracteres chineses}
    \definition{v.}{desenhar um ponto | verificar uma lista | escolher | pedir (comida em um restaurante) | tocar brevemente | sugerir | acender | derramar um líquido gota a gota}
  \end{phonetics}
\end{entry}

\begin{entry}{点火}{9,4}[Radicais ⽕、⽕]
  \begin{phonetics}{点火}{dian3huo3}
    \definition{s.}{ignição}
    \definition{v.}{inflamar | acender um fogo | agitar | dar partida em um motor | (figurativo) provocar problemas}
  \end{phonetics}
\end{entry}

\begin{entry}{点头}{9,5}[Radicais ⽕、⼤]
  \begin{phonetics}{点头}{dian3 tou2}[][HSK 2]
    \definition{v.}{acenar com a cabeça}
  \end{phonetics}
\end{entry}

\begin{entry}{点名}{9,6}[Radicais ⽕、⼝]
  \begin{phonetics}{点名}{dian3 ming2}[][HSK 4]
    \definition{v.}{fazer a lista de chamada; manter o controle da presença de alguém; chamar nomes para controle de presença | mencionar alguém pelo nome}
  \end{phonetics}
\end{entry}

\begin{entry}{点燃}{9,16}[Radicais ⽕、⽕]
  \begin{phonetics}{点燃}{dian3ran2}
    \definition{v.}{inflamar | incendiar}
  \end{phonetics}
\end{entry}

\begin{entry}{独}{9}[Radical ⽝]
  \begin{phonetics}{独}{du2}
    \definition{adj.}{sozinho | solitário | solteiro}
    \definition{adv.}{apenas}
  \end{phonetics}
\end{entry}

\begin{entry}{独立}{9,5}[Radicais ⽝、⽴]
  \begin{phonetics}{独立}{du2li4}[][HSK 4]
    \definition{adj.}{independente; por conta própria | separado; respectivo; descreve algo que é separado e não está em contato com outra coisa}
    \definition{prep.}{independente de; separado de; não mais anexado à unidade original, mas uma unidade separada}
    \definition{v.}{ficar sozinho | alcançar a independência; tornar-se um país independente; liberdade de um Estado, regime ou organização contra interferência, controle e dominação por forças externas}
  \end{phonetics}
\end{entry}

\begin{entry}{独自}{9,6}[Radicais ⽝、⾃]
  \begin{phonetics}{独自}{du2 zi4}[][HSK 4]
    \definition{adj.}{sozinho; por si mesmo; por conta própria}
  \end{phonetics}
\end{entry}

\begin{entry}{独特}{9,10}[Radicais ⽝、⽜]
  \begin{phonetics}{独特}{du2te4}[][HSK 4]
    \definition{adj.}{único; distinto; original; especial}
  \end{phonetics}
\end{entry}

\begin{entry}{玻璃}{9,14}[Radicais ⽟、⽟]
  \begin{phonetics}{玻璃}{bo1li5}
    \definition[张,塊]{s.}{vidro | (gíria) homossexual masculino}
  \end{phonetics}
\end{entry}

\begin{entry}{珍贵}{9,9}[Radicais ⽟、⾙]
  \begin{phonetics}{珍贵}{zhen1gui4}
    \definition{adj.}{precioso}
  \end{phonetics}
\end{entry}

\begin{entry}{珍珠}{9,10}[Radicais ⽟、⽟]
  \begin{phonetics}{珍珠}{zhen1zhu1}
    \definition[颗]{s.}{pérola}
  \end{phonetics}
\end{entry}

\begin{entry}{甚而}{9,6}[Radicais ⽢、⽽]
  \begin{phonetics}{甚而}{shen4'er2}
    \definition{conj.}{(ir) tão longe quanto | tanto que}
  \end{phonetics}
\end{entry}

\begin{entry}{甚至}{9,6}[Radicais ⽢、⾄]
  \begin{phonetics}{甚至}{shen4zhi4}[][HSK 4]
    \definition{conj.}{e até mesmo; nem mesmo; para apresentar uma situação típica e especial, para enfatizar a profundidade e a seriedade de uma situação}
  \end{phonetics}
\end{entry}

\begin{entry}{甚或}{9,8}[Radicais ⽢、⼽]
  \begin{phonetics}{甚或}{shen4huo4}
    \definition{conj.}{(ir) tão longe quanto | tanto que}
  \end{phonetics}
\end{entry}

\begin{entry}{甭}{9}[Radical ⽤]
  \begin{phonetics}{甭}{beng2}
    \definition{v.}{contração de 不用 | não precisar}
    \seeref{不用}{bu2 yong4}
  \end{phonetics}
\end{entry}

\begin{entry}{界碑}{9,13}[Radicais ⽥、⽯]
  \begin{phonetics}{界碑}{jie4bei1}
    \definition{s.}{marco de fronteira}
  \end{phonetics}
\end{entry}

\begin{entry}{疯狂}{9,7}[Radicais ⽧、⽝]
  \begin{phonetics}{疯狂}{feng1kuang2}
    \definition{adj.}{louco | frenético | selvagem}
  \end{phonetics}
\end{entry}

\begin{entry}{皆}{9}[Radical ⽩]
  \begin{phonetics}{皆}{jie1}
    \definition{adv.}{todos | em todos os casos}
  \end{phonetics}
\end{entry}

\begin{entry}{皇帝}{9,9}[Radicais ⽩、⼱]
  \begin{phonetics}{皇帝}{huang2di4}
    \definition[个]{s.}{imperador}
  \end{phonetics}
\end{entry}

\begin{entry}{盆}{9}[Radical ⽫]
  \begin{phonetics}{盆}{pen2}
    \definition[个]{s.}{panela | bacia | vaso de flores}
  \end{phonetics}
\end{entry}

\begin{entry}{盆友}{9,4}[Radicais ⽫、⼜]
  \begin{phonetics}{盆友}{pen2you3}
    \definition{s.}{(gíria na \emph{Internet}) amigo (trocadilho com 朋友)}
    \seeref{朋友}{peng2you5}
  \end{phonetics}
\end{entry}

\begin{entry}{相互}{9,4}[Radicais ⽬、⼆]
  \begin{phonetics}{相互}{xiang1 hu4}[][HSK 3]
    \definition{adj.}{mútuo; recíproco}
    \definition{adv.}{mutuamente; um ao outro}
  \end{phonetics}
\end{entry}

\begin{entry}{相比}{9,4}[Radicais ⽬、⽐]
  \begin{phonetics}{相比}{xiang1 bi3}[][HSK 3]
    \definition{v.}{combinar; comparar com |comparar uma coisa com outra, usar uma coisa como padrão para ver as características de outra coisa ou para obter um ponto de vista}
  \end{phonetics}
\end{entry}

\begin{entry}{相处}{9,5}[Radicais ⽬、⼡]
  \begin{phonetics}{相处}{xiang1chu3}
    \definition{v.}{entrar em contato (com alguém) | associar | interagir | se dar bem (bem, mal)}
  \end{phonetics}
\end{entry}

\begin{entry}{相似}{9,6}[Radicais ⽬、⼈]
  \begin{phonetics}{相似}{xiang1si4}[][HSK 3]
    \definition{v.}{assemelhar-se; ser semelhante; ser igual}
  \end{phonetics}
\end{entry}

\begin{entry}{相关}{9,6}[Radicais ⽬、⼋]
  \begin{phonetics}{相关}{xiang1guan1}[][HSK 3]
    \definition{v.}{mutuamente relacionados; inter-relacionados}
  \end{phonetics}
\end{entry}

\begin{entry}{相同}{9,6}[Radicais ⽬、⼝]
  \begin{phonetics}{相同}{xiang1tong2}[][HSK 2]
    \definition{adj.}{igual | idêntico | o mesmo}
  \end{phonetics}
\end{entry}

\begin{entry}{相当}{9,6}[Radicais ⽬、⼹]
  \begin{phonetics}{相当}{xiang1dang1}[][HSK 3]
    \definition{adj.}{adequado; ajustado; apropriado}
    \definition{adv.}{bastante; razoavelmente; consideravelmente}
    \definition{v.}{combinar; equilibrar; corresponder a; ser aproximadamente igual a; ser compatível com}
  \end{phonetics}
\end{entry}

\begin{entry}{相机}{9,6}[Radicais ⽬、⽊]
  \begin{phonetics}{相机}{xiang4 ji1}[][HSK 2]
    \definition[台,个]{s.}{câmera | máquina fotográfica}
    \definition{v.}{ficar atento a uma oportunidade}
  \end{phonetics}
\end{entry}

\begin{entry}{相宜}{9,8}[Radicais ⽬、⼧]
  \begin{phonetics}{相宜}{xiang1yi2}
    \definition{adj.}{adequado | apropriado}
    \definition{v.}{ser adequado ou apropriado}
  \end{phonetics}
\end{entry}

\begin{entry}{相亲}{9,9}[Radicais ⽬、⼇]
  \begin{phonetics}{相亲}{xiang1qin1}
    \definition{s.}{encontro às cegas | entrevista arranjada para avaliar a proposta de um parceiro de casamento | apegar-se profundamente um ao outro}
  \end{phonetics}
\end{entry}

\begin{entry}{相信}{9,9}[Radicais ⽬、⼈]
  \begin{phonetics}{相信}{xiang1xin4}[][HSK 2]
    \definition{v.}{acreditar | estar convencido | aceitar como verdadeiro}
  \end{phonetics}
\end{entry}

\begin{entry}{相思病}{9,9,10}[Radicais ⽬、⼼、⽧]
  \begin{phonetics}{相思病}{xiang1si1bing4}
    \definition{s.}{saudade de amor}
  \end{phonetics}
\end{entry}

\begin{entry}{相遇}{9,12}[Radicais ⽬、⾡]
  \begin{phonetics}{相遇}{xiang1yu4}
    \definition{v.}{encontrar (reunião, encontro, etc.)}
  \end{phonetics}
\end{entry}

\begin{entry}{相聚}{9,14}[Radicais ⽬、⽿]
  \begin{phonetics}{相聚}{xiang1ju4}
    \definition{v.}{reunir-se | montar}
  \end{phonetics}
\end{entry}

\begin{entry}{省}{9}[Radical ⽬]
  \begin{phonetics}{省}{sheng3}[][HSK 2]
    \definition{s.}{província | capital provincial}
    \definition{v.}{economizar | guardar | ser frugal | omitir | excluir | deixar de fora}
  \end{phonetics}
  \begin{phonetics}{省}{xing3}
    \definition[个]{s.}{governadoria}
    \definition{v.}{examinar minuciosamente | refletir (sobre a conduta de alguém) | realizar | fazer uma visita (aos pais ou idosos)}
  \end{phonetics}
\end{entry}

\begin{entry}{省力}{9,2}[Radicais ⽬、⼒]
  \begin{phonetics}{省力}{sheng3li4}
    \definition{v.}{economizar esforço ou trabalho}
  \end{phonetics}
\end{entry}

\begin{entry}{省心}{9,4}[Radicais ⽬、⼼]
  \begin{phonetics}{省心}{sheng3xin1}
    \definition{adj.}{despreocupado}
    \definition{v.}{ser poupado de preocupações | despreocupar-se}
  \end{phonetics}
\end{entry}

\begin{entry}{省长}{9,4}[Radicais ⽬、⾧]
  \begin{phonetics}{省长}{sheng3zhang3}
    \definition*{s.}{Governador | governador de uma província}
  \end{phonetics}
\end{entry}

\begin{entry}{省会}{9,6}[Radicais ⽬、⼈]
  \begin{phonetics}{省会}{sheng3hui4}
    \definition{s.}{capital da província}
  \end{phonetics}
\end{entry}

\begin{entry}{省却}{9,7}[Radicais ⽬、⼙]
  \begin{phonetics}{省却}{sheng3que4}
    \definition{v.}{livrar-se (para economizar espaço) | salvar}
  \end{phonetics}
\end{entry}

\begin{entry}{省俭}{9,9}[Radicais ⽬、⼈]
  \begin{phonetics}{省俭}{sheng3jian3}
    \definition{s.}{econômico | frugal}
    \definition{v.}{economizar}
  \end{phonetics}
\end{entry}

\begin{entry}{省城}{9,9}[Radicais ⽬、⼟]
  \begin{phonetics}{省城}{sheng3cheng2}
    \definition{s.}{capital da província}
  \end{phonetics}
\end{entry}

\begin{entry}{省悟}{9,10}[Radicais ⽬、⼼]
  \begin{phonetics}{省悟}{xing3wu4}
    \definition{v.}{voltar a si | constatar | ver a verdade | acordar para a realidade}
  \end{phonetics}
\end{entry}

\begin{entry}{省钱}{9,10}[Radicais ⽬、⾦]
  \begin{phonetics}{省钱}{sheng3qian2}
    \definition{v.}{economizar dinheiro}
  \end{phonetics}
\end{entry}

\begin{entry}{眉}{9}[Radical ⽬]
  \begin{phonetics}{眉}{mei2}
    \definition{s.}{sobrancelha | margem superior}
  \end{phonetics}
\end{entry}

\begin{entry}{眉毛}{9,4}[Radicais ⽬、⽑]
  \begin{phonetics}{眉毛}{mei2mao5}
    \definition[根]{s.}{sobrancelha}
  \end{phonetics}
\end{entry}

\begin{entry}{眉头}{9,5}[Radicais ⽬、⼤]
  \begin{phonetics}{眉头}{mei2tou2}
    \definition{s.}{testa}
  \end{phonetics}
\end{entry}

\begin{entry}{看}{9}[Radical ⽬]
  \begin{phonetics}{看}{kan1}
    \definition{v.}{cuidar | vigiar}
  \end{phonetics}
  \begin{phonetics}{看}{kan4}[][HSK 1]
    \definition{interj.}{Cuidado! (para um perigo)}
    \definition{part.}{(depois de um verbo) tentar}
    \definition{v.}{olhar | ver | assistir | ler | visitar (pessoas)}
  \end{phonetics}
\end{entry}

\begin{entry}{看上去}{9,3,5}[Radicais ⽬、⼀、⼛]
  \begin{phonetics}{看上去}{kan4 shang4 qu4}[][HSK 3]
    \definition{adv.}{parece que}
  \end{phonetics}
\end{entry}

\begin{entry}{看不起}{9,4,10}[Radicais ⽬、⼀、⾛]
  \begin{phonetics}{看不起}{kan4bu5qi3}[][HSK 4]
    \definition{v.}{desprezar; desdenhar; menosprezar; ter desprezo; olhar de cima para baixo}
  \end{phonetics}
\end{entry}

\begin{entry}{看见}{9,4}[Radicais ⽬、⾒]
  \begin{phonetics}{看见}{kan4 jian4}[][HSK 1]
    \definition{v.}{encontrar | enxergar | ver | avistar}
  \end{phonetics}
\end{entry}

\begin{entry}{看来}{9,7}[Radicais ⽬、⽊]
  \begin{phonetics}{看来}{kan4 lai2}[][HSK 4]
    \definition{adv.}{parecer; parecer como se (ou embora); refere-se a um julgamento aproximado; expressa um julgamento por observação}
    \definition{v.}{ser considerado; na visão de alguém; na opinião de alguém; expressar a ideia aproximada que o locutor tem da situação}
  \end{phonetics}
\end{entry}

\begin{entry}{看到}{9,8}[Radicais ⽬、⼑]
  \begin{phonetics}{看到}{kan4 dao4}[][HSK 1]
    \definition{v.}{ver}
  \end{phonetics}
\end{entry}

\begin{entry}{看法}{9,8}[Radicais ⽬、⽔]
  \begin{phonetics}{看法}{kan4fa3}[][HSK 2]
    \definition[个]{s.}{modo de olhar alguma coisa | ponto de vista | opinião}
  \end{phonetics}
\end{entry}

\begin{entry}{看病}{9,10}[Radicais ⽬、⽧]
  \begin{phonetics}{看病}{kan4 bing4}[][HSK 1]
    \definition{v.+compl.}{(médico) ver um paciente | (paciente) consultar (ver) um médico}
  \end{phonetics}
\end{entry}

\begin{entry}{看起来}{9,10,7}[Radicais ⽬、⾛、⽊]
  \begin{phonetics}{看起来}{kan4 qi3 lai5}[][HSK 3]
    \definition{v.}{parecer; parecer com}
  \end{phonetics}
\end{entry}

\begin{entry}{看望}{9,11}[Radicais ⽬、⽉]
  \begin{phonetics}{看望}{kan4wang4}[][HSK 4]
    \definition{v.}{ver; visitar; ligar; dar uma olhada; ir até os pais, idosos, professores ou amigos para cumprimentá-los}
  \end{phonetics}
\end{entry}

\begin{entry}{看淡}{9,11}[Radicais ⽬、⽔]
  \begin{phonetics}{看淡}{kan4dan4}
    \definition{v.}{considerar sem importância | ser indiferente a (fama, riqueza, etc.) | (de uma economia ou mercado) enfraquecer, ficar mais lento, diminuir a velocidade}
  \end{phonetics}
\end{entry}

\begin{entry}{砂}{9}[Radical ⽯]
  \begin{phonetics}{砂}{sha1}
    \variantof{沙}
  \end{phonetics}
\end{entry}

\begin{entry}{砍}{9}[Radical ⽯]
  \begin{phonetics}{砍}{kan3}
    \definition{v.}{cortar}
  \end{phonetics}
\end{entry}

\begin{entry}{砍刀}{9,2}[Radicais ⽯、⼑]
  \begin{phonetics}{砍刀}{kan3dao1}
    \definition{s.}{facão | machete}
  \end{phonetics}
\end{entry}

\begin{entry}{砍头}{9,5}[Radicais ⽯、⼤]
  \begin{phonetics}{砍头}{kan3tou2}
    \definition{v.}{decapitar}
  \end{phonetics}
\end{entry}

\begin{entry}{砍价}{9,6}[Radicais ⽯、⼈]
  \begin{phonetics}{砍价}{kan3jia4}
    \definition{v.}{barganhar | cortar ou derrubar um preço}
  \end{phonetics}
\end{entry}

\begin{entry}{砍伤}{9,6}[Radicais ⽯、⼈]
  \begin{phonetics}{砍伤}{kan3shang1}
    \definition{v.}{ferir com lâmina ou machado}
  \end{phonetics}
\end{entry}

\begin{entry}{砍杀}{9,6}[Radicais ⽯、⽊]
  \begin{phonetics}{砍杀}{kan3sha1}
    \definition{v.}{atacar com arma branca}
  \end{phonetics}
\end{entry}

\begin{entry}{砍死}{9,6}[Radicais ⽯、⽍]
  \begin{phonetics}{砍死}{kan3si3}
    \definition{v.}{matar com um machado}
  \end{phonetics}
\end{entry}

\begin{entry}{砍树}{9,9}[Radicais ⽯、⽊]
  \begin{phonetics}{砍树}{kan3shu4}
    \definition{v.}{derrubar árvores}
  \end{phonetics}
\end{entry}

\begin{entry}{砍掉}{9,11}[Radicais ⽯、⼿]
  \begin{phonetics}{砍掉}{kan3diao4}
    \definition{v.}{amputar}
  \end{phonetics}
\end{entry}

\begin{entry}{砍断}{9,11}[Radicais ⽯、⽄]
  \begin{phonetics}{砍断}{kan3duan4}
    \definition{v.}{cortar}
  \end{phonetics}
\end{entry}

\begin{entry}{砖}{9}[Radical ⽯]
  \begin{phonetics}{砖}{zhuan1}
    \definition[块]{s.}{tijolo}
  \end{phonetics}
\end{entry}

\begin{entry}{祖国}{9,8}[Radicais ⽰、⼞]
  \begin{phonetics}{祖国}{zu3guo2}
    \definition{s.}{pátria | terra natal}
  \end{phonetics}
\end{entry}

\begin{entry}{祝}{9}[Radical ⽰]
  \begin{phonetics}{祝}{zhu4}[][HSK 3]
    \definition*{s.}{sobrenome Zhu}
    \definition{v.}{expressar bons desejos; desejar; abençoar | orar aos deuses ou espíritos por bênçãos}
  \end{phonetics}
\end{entry}

\begin{entry}{祝好}{9,6}[Radicais ⽰、⼥]
  \begin{phonetics}{祝好}{zhu4hao3}
    \definition{expr.}{desejo-lhe tudo de melhor! (ao encerrar uma correspondência)}
  \end{phonetics}
\end{entry}

\begin{entry}{祝寿}{9,7}[Radicais ⽰、⼨]
  \begin{phonetics}{祝寿}{zhu4shou4}
    \definition{v.}{dar parabéns pelo aniversário (a uma pessoa idosa)}
  \end{phonetics}
\end{entry}

\begin{entry}{祝贺}{9,9}[Radicais ⽰、⾙]
  \begin{phonetics}{祝贺}{zhu4he4}
    \definition[个]{s.}{congratulações}
    \definition{v.}{congratular}
  \end{phonetics}
\end{entry}

\begin{entry}{祝酒}{9,10}[Radicais ⽰、⾣]
  \begin{phonetics}{祝酒}{zhu4jiu3}
    \definition{v.}{parabenizar e fazer um brinde | brindar}
  \end{phonetics}
\end{entry}

\begin{entry}{祝颂}{9,10}[Radicais ⽰、⾴]
  \begin{phonetics}{祝颂}{zhu4song4}
    \definition{v.}{expressar bons desejos}
  \end{phonetics}
\end{entry}

\begin{entry}{祝祷}{9,11}[Radicais ⽰、⽰]
  \begin{phonetics}{祝祷}{zhu4dao3}
    \definition{v.}{rezar | orar}
  \end{phonetics}
\end{entry}

\begin{entry}{祝谢}{9,12}[Radicais ⽰、⾔]
  \begin{phonetics}{祝谢}{zhu4xie4}
    \definition{v.}{agradecer | dar parabéns}
  \end{phonetics}
\end{entry}

\begin{entry}{祝福}{9,13}[Radicais ⽰、⽰]
  \begin{phonetics}{祝福}{zhu4fu2}
    \definition{s.}{bênçãos}
    \definition{v.}{desejar boa sorte a alguém}
  \end{phonetics}
\end{entry}

\begin{entry}{祝愿}{9,14}[Radicais ⽰、⽕]
  \begin{phonetics}{祝愿}{zhu4yuan4}
    \definition{v.}{desejar}
  \end{phonetics}
\end{entry}

\begin{entry}{神}{9}[Radical ⽰]
  \begin{phonetics}{神}{shen2}
    \definition*{s.}{Deus}
    \definition{s.}{deus | divindade}
  \end{phonetics}
\end{entry}

\begin{entry}{神奇}{9,8}[Radicais ⽰、⼤]
  \begin{phonetics}{神奇}{shen2qi2}
    \definition{adj.}{mágico | místico | milagroso}
    \definition{s.}{mágica | milagre}
  \end{phonetics}
\end{entry}

\begin{entry}{神明}{9,8}[Radicais ⽰、⽇]
  \begin{phonetics}{神明}{shen2ming2}
    \definition{s.}{divindades | deuses}
  \end{phonetics}
\end{entry}

\begin{entry}{神经}{9,8}[Radicais ⽰、⽷]
  \begin{phonetics}{神经}{shen2jing1}
    \definition{adj.}{desequilibrado | louco | insano}
    \definition{s.}{nervo}
  \end{phonetics}
\end{entry}

\begin{entry}{神经病学}{9,8,10,8}[Radicais ⽰、⽷、⽧、⼦]
  \begin{phonetics}{神经病学}{shen2jing1bing4xue2}
    \definition{s.}{neurologia}
  \end{phonetics}
\end{entry}

\begin{entry}{神经病的}{9,8,10,8}[Radicais ⽰、⽷、⽧、⽩]
  \begin{phonetics}{神经病的}{shen2jing1bing4de5}
    \definition{adj.}{neurótico}
  \end{phonetics}
\end{entry}

\begin{entry}{神话}{9,8}[Radicais ⽰、⾔]
  \begin{phonetics}{神话}{shen2hua4}[][HSK 4]
    \definition[个]{s.}{mito; mitologia; conto de fadas; refere-se a deuses e deusas lendários e histórias de heróis antigos deificados | lorota; refere-se a alegações ridículas e infundadas}
  \end{phonetics}
\end{entry}

\begin{entry}{神秘}{9,10}[Radicais ⽰、⽲]
  \begin{phonetics}{神秘}{shen2mi4}[][HSK 4]
    \definition{adj.}{místico; misterioso}
  \end{phonetics}
\end{entry}

\begin{entry}{神兽}{9,11}[Radicais ⽰、⼋]
  \begin{phonetics}{神兽}{shen2shou4}
    \definition{s.}{animal mitológico | fera}
  \end{phonetics}
\end{entry}

\begin{entry}{神器}{9,16}[Radicais ⽰、⼝]
  \begin{phonetics}{神器}{shen2qi4}
    \definition{s.}{objeto mágico | objeto simbólico do poder imperial | arma fina | ferramenta muito útil}
  \end{phonetics}
\end{entry}

\begin{entry}{秋}{9}[Radical ⽲]
  \begin{phonetics}{秋}{qiu1}
    \definition*{s.}{sobrenome Qiu}
    \definition{s.}{outono | colheita}
  \end{phonetics}
\end{entry}

\begin{entry}{秋天}{9,4}[Radicais ⽲、⼤]
  \begin{phonetics}{秋天}{qiu1 tian1}[][HSK 2]
    \definition[个]{s.}{outono}
  \end{phonetics}
\end{entry}

\begin{entry}{秋季}{9,8}[Radicais ⽲、⼦]
  \begin{phonetics}{秋季}{qiu1 ji4}[][HSK 4]
    \definition[个]{s.}{outono; terceiro trimestre do ano, segundo o costume chinês, refere-se ao período de três meses entre o outono e o inverno, também se refere aos sétimo, oitavo e nono meses do calendário lunar}
  \end{phonetics}
\end{entry}

\begin{entry}{种}{9}[Radical ⽲]
  \begin{phonetics}{种}{zhong3}[][HSK 3]
    \definition*{s.}{sobrenome Zhong}
    \definition{clas.}{para tipos, espécies e gêneros}
    \definition{s.}{espécie | semente; estirpe; raça | entranhas; brio; coragem; espinha dorsal | tipo; variedade; indica tipo, usado para pessoas e qualquer coisa}
  \end{phonetics}
  \begin{phonetics}{种}{zhong4}
    \definition{v.}{plantar; semear; crescer; cultivar}
  \end{phonetics}
\end{entry}

\begin{entry}{种子}{9,3}[Radicais ⽲、⼦]
  \begin{phonetics}{种子}{zhong3zi5}[][HSK 3]
    \definition[颗,粒]{s.}{semente; um órgão exclusivo de certas plantas, geralmente composto de três partes: tegumento, embrião e endosperma, as sementes podem germinar e se tornar novas plantas sob certas condições | jogador cabeça de chave; na competição, quando é realizada a fase eliminatória, são escolhidos os jogadores mais fortes de cada equipe}
  \end{phonetics}
\end{entry}

\begin{entry}{种地}{9,6}[Radicais ⽲、⼟]
  \begin{phonetics}{种地}{zhong4di4}
    \definition{v.}{cultivar | trabalhar a terra}
  \end{phonetics}
\end{entry}

\begin{entry}{种种}{9,9}[Radicais ⽲、⽲]
  \begin{phonetics}{种种}{zhong3zhong3}
    \definition{adj.}{todos os tipos de}
  \end{phonetics}
\end{entry}

\begin{entry}{种族灭绝}{9,11,5,9}[Radicais ⽲、⽅、⽕、⽷]
  \begin{phonetics}{种族灭绝}{zhong3zu2mie4jue2}
    \definition{s.}{genocídio | extinção étnica}
  \end{phonetics}
\end{entry}

\begin{entry}{种麻}{9,11}[Radicais ⽲、⿇]
  \begin{phonetics}{种麻}{zhong3ma2}
    \definition{s.}{planta de cânhamo (feminina)}
  \end{phonetics}
\end{entry}

\begin{entry}{种薯}{9,16}[Radicais ⽲、⾋]
  \begin{phonetics}{种薯}{zhong3shu3}
    \definition{s.}{tubérculo semente}
  \end{phonetics}
\end{entry}

\begin{entry}{科}{9}[Radical ⽲]
  \begin{phonetics}{科}{ke1}[][HSK 2]
    \definition*{s.}{sobrenome Ke}
    \definition{s.}{um ramo de estudo acadêmico ou profissional |uma divisão ou subdivisão de uma unidade administrativa | família | instruções de palco no drama chinês clássico}
  \end{phonetics}
\end{entry}

\begin{entry}{科技}{9,7}[Radicais ⽲、⼿]
  \begin{phonetics}{科技}{ke1 ji4}[][HSK 3]
    \definition{s.}{ciência e tecnologia}
  \end{phonetics}
\end{entry}

\begin{entry}{科学}{9,8}[Radicais ⽲、⼦]
  \begin{phonetics}{科学}{ke1xue2}[][HSK 2]
    \definition{adj.}{científico}
    \definition[门]{s.}{ciência}
  \end{phonetics}
\end{entry}

\begin{entry}{科学家}{9,8,10}[Radicais ⽲、⼦、⼧]
  \begin{phonetics}{科学家}{ke1xue2jia1}
    \definition[个]{s.}{cientista}
  \end{phonetics}
\end{entry}

\begin{entry}{秒}{9}[Radical ⽲]
  \begin{phonetics}{秒}{miao3}
    \definition{adv.}{(coloquial) instantaneamente}
    \definition{s.}{segundo (unidade de tempo) | segundo (unidade de medida angular)}
  \end{phonetics}
\end{entry}

\begin{entry}{穿}{9}[Radical ⽳]
  \begin{phonetics}{穿}{chuan1}[][HSK 1]
    \definition{v.}{vestir}
  \end{phonetics}
\end{entry}

\begin{entry}{穿上}{9,3}[Radicais ⽳、⼀]
  \begin{phonetics}{穿上}{chuan1 shang4}[][HSK 4]
    \definition{v.}{vestir (roupas, etc.); colocar roupas}
  \end{phonetics}
\end{entry}

\begin{entry}{突出}{9,5}[Radicais ⽳、⼐]
  \begin{phonetics}{突出}{tu1chu1}[][HSK 3]
    \definition{adj.}{proeminente; excelente}
    \definition{v.}{romper | enfatizar; destacar; dar destaque a | sobressair; projetar-se; destacar-se}
  \end{phonetics}
\end{entry}

\begin{entry}{突然}{9,12}[Radicais ⽳、⽕]
  \begin{phonetics}{突然}{tu1ran2}[][HSK 3]
    \definition{adj.}{repentino; abrupto; inesperado}
    \definition{adv.}{de repente; abruptamente; inesperadamente}
  \end{phonetics}
\end{entry}

\begin{entry}{类}{9}[Radical ⽶]
  \begin{phonetics}{类}{lei4}[][HSK 3]
    \definition*{s.}{sobrenome Lei}
    \definition{s.}{classe; categoria; tipo; espécie}
    \definition{v.}{assemelhar-se a; ser semelhante a}
  \end{phonetics}
\end{entry}

\begin{entry}{类似}{9,6}[Radicais ⽶、⼈]
  \begin{phonetics}{类似}{lei4si4}[][HSK 3]
    \definition{adj.}{semelhante; análogo}
  \end{phonetics}
\end{entry}

\begin{entry}{类型}{9,9}[Radicais ⽶、⼟]
  \begin{phonetics}{类型}{lei4xing2}[][HSK 4]
    \definition[种,个]{s.}{tipo; espécie; categoria; tipos formados por coisas com características comuns}
  \end{phonetics}
\end{entry}

\begin{entry}{结}{9}[Radical ⽷]
  \begin{phonetics}{结}{jie1}
    \definition{v.}{dar (frutos); formar (sementes); produzir frutos ou sementes (uma planta)}
  \end{phonetics}
  \begin{phonetics}{结}{jie2}[][HSK 4]
    \definition*{s.}{sobrenome Jie}
    \definition{s.}{nó | declaração juramentada; garantia por escrito; documento que, antigamente, significava um reconhecimento de encerramento ou uma garantia de responsabilidade}
    \definition{v.}{amarrar; tricotar; dar nó; tecer | formar; forjar; cimentar; solidificar | resolver; concluir | combinar; formar um relacionamento}
  \end{phonetics}
\end{entry}

\begin{entry}{结合}{9,6}[Radicais ⽷、⼝]
  \begin{phonetics}{结合}{jie2he2}[][HSK 3]
    \definition{v.}{ligar; unir; combinar; integrar | casar-se; unir-se em matrimônio}
  \end{phonetics}
\end{entry}

\begin{entry}{结论}{9,6}[Radicais ⽷、⾔]
  \begin{phonetics}{结论}{jie2lun4}[][HSK 4]
    \definition[个]{s.}{conclusão; palavra final sobre uma pessoa ou coisa após investigação e pesquisa | veredito; julgamento deduzido de premissas também é chamado de conclusão}
  \end{phonetics}
\end{entry}

\begin{entry}{结局}{9,7}[Radicais ⽷、⼫]
  \begin{phonetics}{结局}{jie2ju2}
    \definition{s.}{conclusão | fim | final}
  \end{phonetics}
\end{entry}

\begin{entry}{结束}{9,7}[Radicais ⽷、⽊]
  \begin{phonetics}{结束}{jie2shu4}[][HSK 3]
    \definition{v.}{finalizar; fechar; terminar; concluir; encerrar}
  \end{phonetics}
\end{entry}

\begin{entry}{结束工作}{9,7,3,7}[Radicais ⽷、⽊、⼯、⼈]
  \begin{phonetics}{结束工作}{jie2shu4gong1zuo4}
    \definition{s.}{trabalho final}
    \definition{v.}{terminar o trabalho}
  \end{phonetics}
\end{entry}

\begin{entry}{结束区}{9,7,4}[Radicais ⽷、⽊、⼖]
  \begin{phonetics}{结束区}{jie2shu4 qu1}
    \definition{s.}{zona final}
  \end{phonetics}
\end{entry}

\begin{entry}{结束文本}{9,7,4,5}[Radicais ⽷、⽊、⽂、⽊]
  \begin{phonetics}{结束文本}{jie2shu4 wen2ben3}
    \definition{s.}{texto final}
  \end{phonetics}
\end{entry}

\begin{entry}{结束剂}{9,7,8}[Radicais ⽷、⽊、⼑]
  \begin{phonetics}{结束剂}{jie2shu4 ji4}
    \definition{s.}{finalizador}
  \end{phonetics}
\end{entry}

\begin{entry}{结束语}{9,7,9}[Radicais ⽷、⽊、⾔]
  \begin{phonetics}{结束语}{jie2shu4yu3}
    \definition{s.}{conclusões finais | considerações finais}
  \end{phonetics}
\end{entry}

\begin{entry}{结束辩论}{9,7,16,6}[Radicais ⽷、⽊、⾟、⾔]
  \begin{phonetics}{结束辩论}{jie2shu4 bian4 lun4}
    \definition{s.}{debate de encerramento}
  \end{phonetics}
\end{entry}

\begin{entry}{结社自由}{9,7,6,5}[Radicais ⽷、⽰、⾃、⽥]
  \begin{phonetics}{结社自由}{jie2she4zi4you2}
    \definition{s.}{(constitucional) liberdade de associação}
  \end{phonetics}
\end{entry}

\begin{entry}{结实}{9,8}[Radicais ⽷、⼧]
  \begin{phonetics}{结实}{jie1shi5}[][HSK 3]
    \definition{adj.}{sólido; resistente; durável | forte; resistente; robusto}
  \end{phonetics}
\end{entry}

\begin{entry}{结构}{9,8}[Radicais ⽷、⽊]
  \begin{phonetics}{结构}{jie2gou4}[][HSK 4]
    \definition[个,座]{s.}{estrutura; composição; construção; formação; constituição; tecido; forma; sistematização; mecânica; organização | arquitetura; estrutura; construção; construção de partes de edifícios com suporte de carga ou com carga externa | textura (geológico)}
  \end{phonetics}
\end{entry}

\begin{entry}{结果}{9,8}[Radicais ⽷、⽊]
  \begin{phonetics}{结果}{jie1guo3}
    \definition{v.}{dar frutos}
  \end{phonetics}
  \begin{phonetics}{结果}{jie2guo3}[][HSK 2]
    \definition{s.}{resultado | conclusão}
    \definition{v.}{despachar | matar}
  \end{phonetics}
\end{entry}

\begin{entry}{结婚}{9,11}[Radicais ⽷、⼥]
  \begin{phonetics}{结婚}{jie2hun1}[][HSK 3]
    \definition{v.+compl.}{casar; casar-se}
  \end{phonetics}
\end{entry}

\begin{entry}{结婚礼服}{9,11,5,8}[Radicais ⽷、⼥、⽰、⽉]
  \begin{phonetics}{结婚礼服}{jie2hun1 li3 fu2}
    \definition{s.}{vestido de casamento}
  \end{phonetics}
\end{entry}

\begin{entry}{给}{9}[Radical ⽷]
  \begin{phonetics}{给}{gei3}[][HSK 1]
    \definition{prep.}{a | para}
    \definition{v.}{dar | permitir | fazer alguma coisa (para alguém)}
  \end{phonetics}
  \begin{phonetics}{给}{ji3}
    \definition{v.}{fornecer | prover}
  \end{phonetics}
\end{entry}

\begin{entry}{给……打电话}{9,5,5,8}[Radicais ⽷、⼿、⽥、⾔]
  \begin{phonetics}{给……打电话}{gei3 da3 dian4 hua4}
    \definition{expr.}{telefonar para alguém}
    \seeref{打电话}{da3 dian4 hua4}
  \end{phonetics}
\end{entry}

\begin{entry}{绝不}{9,4}[Radicais ⽷、⼀]
  \begin{phonetics}{绝不}{jue2bu4}
    \definition{adv.}{definitivamente não | de forma alguma | sob nenhuma circunstância}
  \end{phonetics}
\end{entry}

\begin{entry}{绝对}{9,5}[Radicais ⽷、⼨]
  \begin{phonetics}{绝对}{jue2dui4}[][HSK 3]
    \definition{adj.}{absoluto; extremo}
    \definition{adv.}{absolutamente}
  \end{phonetics}
\end{entry}

\begin{entry}{绝招}{9,8}[Radicais ⽷、⼿]
  \begin{phonetics}{绝招}{jue2zhao1}
    \definition{s.}{habilidade única | movimento delicado inesperado (como último recurso) | golpe de mestre | golpe final}
  \end{phonetics}
\end{entry}

\begin{entry}{绝版}{9,8}[Radicais ⽷、⽚]
  \begin{phonetics}{绝版}{jue2ban3}
    \definition{adj.}{esgotado | fora de catálogo}
  \end{phonetics}
\end{entry}

\begin{entry}{统一}{9,1}[Radicais ⽷、⼀]
  \begin{phonetics}{统一}{tong3yi1}[][HSK 4]
    \definition{adj.}{unificado; unitário; centralizado; consistente}
    \definition{v.}{unificar; unir; integrar; padronizar}
  \end{phonetics}
\end{entry}

\begin{entry}{统计}{9,4}[Radicais ⽷、⾔]
  \begin{phonetics}{统计}{tong3ji4}[][HSK 4]
    \definition{v.}{compilar estatísticas; refere-se à realização de trabalho estatístico, ou seja, coletar, reunir, analisar e extrapolar dados sobre um fenômeno | somar; adicionar; contar}
  \end{phonetics}
\end{entry}

\begin{entry}{罚}{9}[Radical ⽹]
  \begin{phonetics}{罚}{fa2}
    \definition{v.}{castigar | punir}
  \end{phonetics}
\end{entry}

\begin{entry}{罚款}{9,12}[Radicais ⽹、⽋]
  \begin{phonetics}{罚款}{fa2kuan3}
    \definition{s.}{multa (monetária) | pena}
    \definition{v.+compl.}{aplicar uma multa | multar}
  \end{phonetics}
\end{entry}

\begin{entry}{美}{9}[Radical ⽺]
  \begin{phonetics}{美}{mei3}[][HSK 3]
    \definition*{s.}{Abreviatura de América (美洲) | Abreviatura de Estados Unidos da América (美国)}
    \definition{adj.}{lindo; bonito; belo; atraente | satisfatório; bom; agradável}
    \definition{v.}{embelezar; enfeitar | orgulhar-se de; estar satisfeito consigo mesmo}
  \seealsoref{美国}{mei3guo2}
  \seealsoref{美洲}{mei3zhou1}
  \end{phonetics}
\end{entry}

\begin{entry}{美女}{9,3}[Radicais ⽺、⼥]
  \begin{phonetics}{美女}{mei3 nv3}[][HSK 4]
    \definition[个,位]{s.}{beldade; mulher bonita; uma jovem linda}
  \end{phonetics}
\end{entry}

\begin{entry}{美元}{9,4}[Radicais ⽺、⼉]
  \begin{phonetics}{美元}{mei3yuan2}[][HSK 3]
    \definition*[元,笔,沓]{s.}{Dólar Americano}
  \end{phonetics}
\end{entry}

\begin{entry}{美术}{9,5}[Radicais ⽺、⽊]
  \begin{phonetics}{美术}{mei3shu4}[][HSK 3]
    \definition[种]{s.}{arte; belas artes | pintura}
  \end{phonetics}
\end{entry}

\begin{entry}{美甲}{9,5}[Radicais ⽺、⽥]
  \begin{phonetics}{美甲}{mei3jia3}
    \definition{s.}{manicure e/ou pedicure}
  \end{phonetics}
\end{entry}

\begin{entry}{美好}{9,6}[Radicais ⽺、⼥]
  \begin{phonetics}{美好}{mei3 hao3}[][HSK 3]
    \definition{adj.}{bem; feliz; glorioso}
  \end{phonetics}
\end{entry}

\begin{entry}{美丽}{9,7}[Radicais ⽺、⼀]
  \begin{phonetics}{美丽}{mei3li4}[][HSK 3]
    \definition{adj.}{bonito; lindo}
  \end{phonetics}
\end{entry}

\begin{entry}{美味}{9,8}[Radicais ⽺、⼝]
  \begin{phonetics}{美味}{mei3wei4}
    \definition{adj.}{delicioso}
    \definition{s.}{comida deliciosa | delicadeza (\emph{delicacy})}
  \end{phonetics}
\end{entry}

\begin{entry}{美国}{9,8}[Radicais ⽺、⼞]
  \begin{phonetics}{美国}{mei3guo2}
    \definition*{s.}{Estados Unidos da América}
  \end{phonetics}
\end{entry}

\begin{entry}{美国人}{9,8,2}[Radicais ⽺、⼞、⼈]
  \begin{phonetics}{美国人}{mei3guo2ren2}
    \definition{s.}{americano | pessoa ou povo dos Estados Unidos da América}
  \end{phonetics}
\end{entry}

\begin{entry}{美学}{9,8}[Radicais ⽺、⼦]
  \begin{phonetics}{美学}{mei3xue2}
    \definition{s.}{estética}
  \end{phonetics}
\end{entry}

\begin{entry}{美金}{9,8}[Radicais ⽺、⾦]
  \begin{phonetics}{美金}{mei3 jin1}[][HSK 4]
    \definition{s.}{USD; dólar americano: a moeda local dos Estados Unidos}
  \end{phonetics}
\end{entry}

\begin{entry}{美洲}{9,9}[Radicais ⽺、⽔]
  \begin{phonetics}{美洲}{mei3zhou1}
    \definition*{s.}{América (incluindo Norte, Central e Sul)}
  \end{phonetics}
\end{entry}

\begin{entry}{美洲人}{9,9,2}[Radicais ⽺、⽔、⼈]
  \begin{phonetics}{美洲人}{mei3zhou1ren2}
    \definition{s.}{americano | pessoa ou povo do continente Americano}
  \end{phonetics}
\end{entry}

\begin{entry}{美食}{9,9}[Radicais ⽺、⾷]
  \begin{phonetics}{美食}{mei3 shi2}[][HSK 3]
    \definition[种,道,桌]{s.}{iguaria; comida deliciosa}
  \end{phonetics}
\end{entry}

\begin{entry}{耍}{9}[Radical ⽽]
  \begin{phonetics}{耍}{shua3}
    \definition{v.}{brincar com | empunhar | agir (legal, calmo, tranquilo, descolado, etc.) | exibir (uma habilidade, o temperamento de alguém, etc.)}
  \end{phonetics}
\end{entry}

\begin{entry}{耍赖}{9,13}[Radicais ⽽、⾙]
  \begin{phonetics}{耍赖}{shua3lai4}
    \definition{v.}{agir descaradamente | recusar -se a reconhecer que alguém perdeu o jogo ou fez uma promessa, etc. | agir como um idiota | agir como se algo nunca tivesse acontecido}
  \end{phonetics}
\end{entry}

\begin{entry}{耐心}{9,4}[Radicais ⽽、⼼]
  \begin{phonetics}{耐心}{nai4xin1}
    \definition{s.}{paciência}
    \definition{v.}{ser paciente}
  \end{phonetics}
\end{entry}

\begin{entry}{胃口}{9,3}[Radicais ⾁、⼝]
  \begin{phonetics}{胃口}{wei4kou3}
    \definition{s.}{apetite}
  \end{phonetics}
\end{entry}

\begin{entry}{胆小鬼}{9,3,9}[Radicais ⾁、⼩、⿁]
  \begin{phonetics}{胆小鬼}{dan3xiao3gui3}
    \definition{adj.}{covarde | medroso}
  \end{phonetics}
\end{entry}

\begin{entry}{背}{9}[Radical ⾁]
  \begin{phonetics}{背}{bei1}[][HSK 2]
    \definition{v.}{estar sobrecarregado | carregar nas costas ou no ombro}
  \end{phonetics}
  \begin{phonetics}{背}{bei4}[][HSK 3]
    \definition{adv.}{a parte de trás de um corpo ou objeto}
    \definition{s.}{costas | (gíria) azarado}
    \definition{v.}{esconder algo de | decorar | recitar de memória | virar as costas}
  \end{phonetics}
\end{entry}

\begin{entry}{背后}{9,6}[Radicais ⾁、⼝]
  \begin{phonetics}{背后}{bei4 hou4}[][HSK 3]
    \definition{s.}{parte de trás | traseira | nas costas de alguém}
  \end{phonetics}
\end{entry}

\begin{entry}{背景}{9,12}[Radicais ⾁、⽇]
  \begin{phonetics}{背景}{bei4jing3}[][HSK 4]
    \definition[种]{s.}{pano de fundo; fundo; cenário de teatro, filme ou drama de TV | fundo; cenário que permeia a imagem principal na tela | condições sociais; ambientes históricos (significativamente influentes para algo ou alguém) | poder que dá suporte a alguém}
  \end{phonetics}
\end{entry}

\begin{entry}{胖}{9}[Radical ⾁]
  \begin{phonetics}{胖}{pan2}
    \definition{adj.}{saudável}
  \end{phonetics}
  \begin{phonetics}{胖}{pang4}[][HSK 3]
    \definition{adj.}{gordo; robusto; rechonchudo}
  \end{phonetics}
\end{entry}

\begin{entry}{胖子}{9,3}[Radicais ⾁、⼦]
  \begin{phonetics}{胖子}{pang4 zi5}[][HSK 4]
    \definition{s.}{obeso; gordo; pessoa gorda}
  \end{phonetics}
\end{entry}

\begin{entry}{胚}{9}[Radical ⾁]
  \begin{phonetics}{胚}{pei1}
    \definition{s.}{embrião}
  \end{phonetics}
\end{entry}

\begin{entry}{胜}{9}[Radical ⾁]
  \begin{phonetics}{胜}{sheng4}[][HSK 3]
    \definition{adj.}{soberbo; maravilhoso; adorável}
    \definition[场]{s.}{vitória; sucesso | penteado de mulher}
    \definition{v.}{ganhar; derrotar; vencer; ter sucesso | superar; ser superior a; levar a melhor sobre | ser igual a; poder suportar}
  \end{phonetics}
\end{entry}

\begin{entry}{胜利}{9,7}[Radicais ⾁、⼑]
  \begin{phonetics}{胜利}{sheng4li4}[][HSK 3]
    \definition{adv.}{com sucesso; triunfantemente}
    \definition[场,个]{s.}{vitória; triunfo; sucesso}
    \definition{v.}{ganhar; vencer; triunfar; ter sucesso}
  \end{phonetics}
\end{entry}

\begin{entry}{胜算}{9,14}[Radicais ⾁、⽵]
  \begin{phonetics}{胜算}{sheng4suan4}
    \definition{s.}{probabilidade de sucesso | estratégia que garante o sucesso}
    \definition{v.}{ter certeza do sucesso}
  \end{phonetics}
\end{entry}

\begin{entry}{胡萝卜}{9,11,2}[Radicais ⾁、⾋、⼘]
  \begin{phonetics}{胡萝卜}{hu2luo2bo5}
    \definition{s.}{cenoura}
  \end{phonetics}
\end{entry}

\begin{entry}{范围}{9,7}[Radicais ⾋、⼞]
  \begin{phonetics}{范围}{fan4wei2}[][HSK 3]
    \definition[个]{s.}{escopo; limite; alcance}
    \definition{v.}{estabelecer limites para; limitar o escopo de}
  \end{phonetics}
\end{entry}

\begin{entry}{茶}{9}[Radical ⾋]
  \begin{phonetics}{茶}{cha2}[][HSK 1]
    \definition[杯,壶]{s.}{chá | pé (planta) de chá}
  \end{phonetics}
\end{entry}

\begin{entry}{茶叶}{9,5}[Radicais ⾋、⼝]
  \begin{phonetics}{茶叶}{cha2 ye4}[][HSK 4]
    \definition[盒,罐,包,片]{s.}{chá; folhas de chá; as folhas jovens da planta do chá que são processadas para produzir bebidas}
  \end{phonetics}
\end{entry}

\begin{entry}{草}{9}[Radical ⾋]
  \begin{phonetics}{草}{cao3}[][HSK 2]
    \definition[棵,撮,株,根]{s.}{erva | grama}
  \end{phonetics}
\end{entry}

\begin{entry}{草地}{9,6}[Radicais ⾋、⼟]
  \begin{phonetics}{草地}{cao3 di4}[][HSK 2]
    \definition[片]{s.}{relva | pastagem}
  \end{phonetics}
\end{entry}

\begin{entry}{草纸}{9,7}[Radicais ⾋、⽷]
  \begin{phonetics}{草纸}{cao3zhi3}
    \definition{s.}{papel pardo | pergaminho | papel de palha áspero | papel higiênico}
  \end{phonetics}
\end{entry}

\begin{entry}{草莓}{9,10}[Radicais ⾋、⾋]
  \begin{phonetics}{草莓}{cao3mei2}
    \definition[颗]{s.}{morango}
  \end{phonetics}
\end{entry}

\begin{entry}{荒芜}{9,7}[Radicais ⾋、⾋]
  \begin{phonetics}{荒芜}{huang1wu2}
    \definition{adj.}{estéril}
  \end{phonetics}
\end{entry}

\begin{entry}{荔枝}{9,8}[Radicais ⾋、⽊]
  \begin{phonetics}{荔枝}{li4zhi1}
    \definition{s.}{lichia}
  \end{phonetics}
\end{entry}

\begin{entry}{药}{9}[Radical ⾋]
  \begin{phonetics}{药}{yao4}[][HSK 2]
    \definition[种,服,味]{s.}{medicamento | remédio | droga}
  \end{phonetics}
\end{entry}

\begin{entry}{药丸}{9,3}[Radicais ⾋、⼂]
  \begin{phonetics}{药丸}{yao4wan2}
    \definition[粒]{s.}{pílula}
  \end{phonetics}
\end{entry}

\begin{entry}{药水}{9,4}[Radicais ⾋、⽔]
  \begin{phonetics}{药水}{yao4 shui3}[][HSK 2]
    \definition{s.}{remédio engarrafado | loção | medicamento em forma líquida}
  \end{phonetics}
\end{entry}

\begin{entry}{药片}{9,4}[Radicais ⾋、⽚]
  \begin{phonetics}{药片}{yao4 pian4}[][HSK 2]
    \definition[片]{s.}{uma pílula ou comprimido (remédio)}
  \end{phonetics}
\end{entry}

\begin{entry}{药补}{9,7}[Radicais ⾋、⾐]
  \begin{phonetics}{药补}{yao4bu3}
    \definition{s.}{suplemento dietético medicinal que ajuda a melhorar a saúde}
  \end{phonetics}
\end{entry}

\begin{entry}{药典}{9,8}[Radicais ⾋、⼋]
  \begin{phonetics}{药典}{yao4dian3}
    \definition{s.}{farmacopéia}
  \end{phonetics}
\end{entry}

\begin{entry}{药店}{9,8}[Radicais ⾋、⼴]
  \begin{phonetics}{药店}{yao4 dian4}[][HSK 2]
    \definition{s.}{farmácia | drogaria | loja de produtos químicos}
  \end{phonetics}
\end{entry}

\begin{entry}{药房}{9,8}[Radicais ⾋、⼾]
  \begin{phonetics}{药房}{yao4fang2}
    \definition{s.}{farmácia | drogaria}
  \end{phonetics}
\end{entry}

\begin{entry}{药品}{9,9}[Radicais ⾋、⼝]
  \begin{phonetics}{药品}{yao4pin3}
    \definition{s.}{medicamento | remédio | droga}
  \end{phonetics}
\end{entry}

\begin{entry}{药签}{9,13}[Radicais ⾋、⽵]
  \begin{phonetics}{药签}{yao4qian1}
    \definition{s.}{cotonete médico}
  \end{phonetics}
\end{entry}

\begin{entry}{药膳}{9,16}[Radicais ⾋、⾁]
  \begin{phonetics}{药膳}{yao4shan4}
    \definition{s.}{dieta medicinal}
  \end{phonetics}
\end{entry}

\begin{entry}{药罐}{9,23}[Radicais ⾋、⽸]
  \begin{phonetics}{药罐}{yao4guan4}
    \definition{s.}{frasco de remédio}
  \end{phonetics}
\end{entry}

\begin{entry}{虽}{9}[Radical ⾍]
  \begin{phonetics}{虽}{sui1}
    \definition{conj.}{no entanto | embora | mesmo se/embora}
  \end{phonetics}
\end{entry}

\begin{entry}{虽然}{9,12}[Radicais ⾍、⽕]
  \begin{phonetics}{虽然}{sui1 ran2}[][HSK 2]
    \definition{conj.}{embora (frequentemente usado correlativamente com 可是, 但是, etc); geralmente é usado no início de uma frase para indicar que o fato anterior foi reconhecido, mas não mudará o que acontecerá em seguida}
  \seealsoref{但是}{dan4 shi4}
  \seealsoref{可是}{ke3shi4}
  \end{phonetics}
\end{entry}

\begin{entry}{虾}{9}[Radical ⾍]
  \begin{phonetics}{虾}{xia1}
    \definition{s.}{camarão}
  \end{phonetics}
\end{entry}

\begin{entry}{蚂蚁}{9,9}[Radicais ⾍、⾍]
  \begin{phonetics}{蚂蚁}{ma3yi3}
    \definition{s.}{formiga}
  \end{phonetics}
\end{entry}

\begin{entry}{要}{9}[Radical ⾑]
  \begin{phonetics}{要}{yao1}[][HSK 1]
    \definition{v.}{(forma ligada) demandar, coagir}
  \end{phonetics}
  \begin{phonetics}{要}{yao4}
    \definition{adj.}{(forma ligada) importante}
    \definition{conj.}{se (o mesmo que  要是)}
    \definition{v.}{querer | precisar | pedir por | precisar de}
  \seealsoref{要是}{yao4shi5}
  \end{phonetics}
\end{entry}

\begin{entry}{要么……要么……}{9,3,9,3}[Radicais ⾑、⼃、⾑、⼃]
  \begin{phonetics}{要么……要么……}{yao4me5 yao4me5}
    \definition{conj.}{ou\dots ou\dots}
  \end{phonetics}
\end{entry}

\begin{entry}{要义}{9,3}[Radicais ⾑、⼂]
  \begin{phonetics}{要义}{yao4yi4}
    \definition{s.}{resumo | o essencial}
  \end{phonetics}
\end{entry}

\begin{entry}{要不}{9,4}[Radicais ⾑、⼀]
  \begin{phonetics}{要不}{yao4bu4}
    \definition{conj.}{de outra forma | se não | outro | ou}
  \end{phonetics}
\end{entry}

\begin{entry}{要不然}{9,4,12}[Radicais ⾑、⼀、⽕]
  \begin{phonetics}{要不然}{yao4bu4ran2}
    \definition{conj.}{de outra forma | se não | outro | ou}
  \end{phonetics}
\end{entry}

\begin{entry}{要好}{9,6}[Radicais ⾑、⼥]
  \begin{phonetics}{要好}{yao4hao3}
    \definition{v.}{ser amigos íntimos | estar em boas condições}
  \end{phonetics}
\end{entry}

\begin{entry}{要死}{9,6}[Radicais ⾑、⽍]
  \begin{phonetics}{要死}{yao4si3}
    \definition{adv.}{extremamente | muito}
  \end{phonetics}
\end{entry}

\begin{entry}{要求}{9,7}[Radicais ⾑、⽔]
  \begin{phonetics}{要求}{yao1qiu2}[][HSK 2]
    \definition[点]{s.}{requerimento}
    \definition{v.}{pedir | exigir | solicitar | fazer uma reivindicação}
  \end{phonetics}
\end{entry}

\begin{entry}{要挟}{9,9}[Radicais ⾑、⼿]
  \begin{phonetics}{要挟}{yao1xie2}
    \definition{v.}{chantagear | ameaçar}
  \end{phonetics}
\end{entry}

\begin{entry}{要是}{9,9}[Radicais ⾑、⽇]
  \begin{phonetics}{要是}{yao4shi5}[][HSK 3]
    \definition{conj.}{se; no caso; orações de conexão, expressando relações hipotéticas, equivalentes a ``se'', podem ser usadas com ``então''}
  \end{phonetics}
\end{entry}

\begin{entry}{要是……的话}{9,9,8,8}[Radicais ⾑、⽇、⽩、⾔]
  \begin{phonetics}{要是……的话}{yao4shi5 de5hua4}[][HSK 2,3]
    \definition{conj.}{se\dots no caso de}
  \end{phonetics}
\end{entry}

\begin{entry}{要点}{9,9}[Radicais ⾑、⽕]
  \begin{phonetics}{要点}{yao4dian3}
    \definition{s.}{pontos principais | essencial}
  \end{phonetics}
\end{entry}

\begin{entry}{要谎}{9,11}[Radicais ⾑、⾔]
  \begin{phonetics}{要谎}{yao4huang3}
    \definition{v.}{pedir um preço enorme (como primeiro passo de negociação)}
  \end{phonetics}
\end{entry}

\begin{entry}{要强}{9,12}[Radicais ⾑、⼸]
  \begin{phonetics}{要强}{yao4qiang2}
    \definition{adj.}{ansioso para se destacar | ansioso para progredir na vida | obstinado}
  \end{phonetics}
\end{entry}

\begin{entry}{觉得}{9,11}[Radicais ⾒、⼻]
  \begin{phonetics}{觉得}{jue2de5}[][HSK 1]
    \definition{v.}{pensar que\dots | sentir que\dots | sentir (desconfortável, etc.)}
  \end{phonetics}
\end{entry}

\begin{entry}{语}{9}[Radical ⾔]
  \begin{phonetics}{语}{yu3}
    \definition{s.}{língua; linguagem | dito; provérbio; refere-se especialmente a coloquialismos, provérbios, expressões idiomáticas ou palavras de livros antigos | sinal; meio não linguístico de comunicar ideias ; ações ou sinais que substituem palavras para expressar significado | palavras; expressão; refere-se a uma palavra, frase ou sentença}
    \definition{v.}{dizer; falar | (de pássaros, insetos, etc.) gorjear; pipilar}
  \end{phonetics}
  \begin{phonetics}{语}{yu4}
    \definition{v.}{contar; informar}
  \end{phonetics}
\end{entry}

\begin{entry}{语气}{9,4}[Radicais ⾔、⽓]
  \begin{phonetics}{语气}{yu3qi4}
    \definition[个]{s.}{maneira de falar | tom}
  \end{phonetics}
\end{entry}

\begin{entry}{语言}{9,7}[Radicais ⾔、⾔]
  \begin{phonetics}{语言}{yu3yan2}[][HSK 2]
    \definition[门,种]{s.}{linguagem | língua}
  \end{phonetics}
\end{entry}

\begin{entry}{语言实验室}{9,7,8,10,9}[Radicais ⾔、⾔、⼧、⾺、⼧]
  \begin{phonetics}{语言实验室}{yu3yan2shi2yan4shi4}
    \definition{s.}{laboratório de línguas}
  \end{phonetics}
\end{entry}

\begin{entry}{语法}{9,8}[Radicais ⾔、⽔]
  \begin{phonetics}{语法}{yu3fa3}
    \definition{s.}{gramática}
  \end{phonetics}
\end{entry}

\begin{entry}{语法术语}{9,8,5,9}[Radicais ⾔、⽔、⽊、⾔]
  \begin{phonetics}{语法术语}{yu3fa3shu4yu3}
    \definition{s.}{termo gramatical}
  \end{phonetics}
\end{entry}

\begin{entry}{语调}{9,10}[Radicais ⾔、⾔]
  \begin{phonetics}{语调}{yu3diao4}
    \definition[个]{s.}{entonação}
  \end{phonetics}
\end{entry}

\begin{entry}{误会}{9,6}[Radicais ⾔、⼈]
  \begin{phonetics}{误会}{wu4hui4}
    \definition[场]{s.}{mal-entendido; desentendimentos ou conflitos decorrentes de mal-entendidos}
    \definition{v.}{entender mal; entender errado; interpretar mal; não entender; não entender corretamente o significado}
  \end{phonetics}
\end{entry}

\begin{entry}{误点}{9,9}[Radicais ⾔、⽕]
  \begin{phonetics}{误点}{wu4dian3}
    \definition{v.+compl.}{atrasar | chegar tarde}
  \end{phonetics}
\end{entry}

\begin{entry}{诱人}{9,2}[Radicais ⾔、⼈]
  \begin{phonetics}{诱人}{you4ren2}
    \definition{adj.}{atraente | cativante}
  \end{phonetics}
\end{entry}

\begin{entry}{说}{9}[Radical ⾔]
  \begin{phonetics}{说}{shui4}
    \definition{v.}{persuadir}
  \end{phonetics}
  \begin{phonetics}{说}{shuo1}[][HSK 1]
    \definition{s.}{uma teoria (normalmente o último caractere, como em 日心说, teoria heliocêntrica)}
    \definition{v.}{falar | dizer | explicar | contar}
  \end{phonetics}
\end{entry}

\begin{entry}{说不定}{9,4,8}[Radicais ⾔、⼀、⼧]
  \begin{phonetics}{说不定}{shuo1bu5ding4}[][HSK 4]
    \definition{adv.}{talvez; indica uma estimativa, possivelmente, provavelmente}
    \definition{v.}{não ter certeza; não estar certo; ser impreciso}
  \end{phonetics}
\end{entry}

\begin{entry}{说好}{9,6}[Radicais ⾔、⼥]
  \begin{phonetics}{说好}{shuo1hao3}
    \definition{v.}{chegar a um acordo | concluir negociações}
  \end{phonetics}
\end{entry}

\begin{entry}{说完}{9,7}[Radicais ⾔、⼧]
  \begin{phonetics}{说完}{shuo1-wan2}
    \definition{expr.}{acabar/terminar palavras}
  \end{phonetics}
\end{entry}

\begin{entry}{说明}{9,8}[Radicais ⾔、⽇]
  \begin{phonetics}{说明}{shuo1ming2}[][HSK 2]
    \definition[本,个]{s.}{legenda | instrução | explicação}
    \definition{v.}{mostrar | explicar | ilustrar | indicar | provar | demonstrar}
  \end{phonetics}
\end{entry}

\begin{entry}{说服}{9,8}[Radicais ⾔、⽉]
  \begin{phonetics}{说服}{shuo1fu2}[][HSK 4]
    \definition{v.}{persuadir; convencer; convencer a outra parte com palavras bem fundamentadas}
  \end{phonetics}
\end{entry}

\begin{entry}{说话}{9,8}[Radicais ⾔、⾔]
  \begin{phonetics}{说话}{shuo1 hua4}[][HSK 1]
    \definition{adv.}{imediatamente | em um minuto}
    \definition{v.}{falar | dizer | bater-papo | conversar | fofocar}
  \end{phonetics}
\end{entry}

\begin{entry}{说理}{9,11}[Radicais ⾔、⽟]
  \begin{phonetics}{说理}{shuo1li3}
    \definition{v.}{racionalizar | discutir logicamente}
  \end{phonetics}
\end{entry}

\begin{entry}{说谎}{9,11}[Radicais ⾔、⾔]
  \begin{phonetics}{说谎}{shuo1huang3}
    \definition{v.+compl.}{mentir | contar uma mentira}
  \end{phonetics}
\end{entry}

\begin{entry}{贴}{9}[Radical ⾙]
  \begin{phonetics}{贴}{tie1}[][HSK 4]
    \definition{adj.}{submisso; obediente}
    \definition{clas.}{para uso em gessos, emplastros}
    \definition{s.}{subsídio; subvenção}
    \definition{v.}{grudar; colar | aninhar-se a; aconchegar-se a | subsidiar; ajudar financeiramente}
  \end{phonetics}
\end{entry}

\begin{entry}{贵}{9}[Radical ⾙]
  \begin{phonetics}{贵}[⻉]{gui4}[中一⻉][HSK 1]
    \definition{adj.}{caro | nobre | precioso}
  \end{phonetics}
\end{entry}

\begin{entry}{贵姓}{9,8}[Radicais ⾙、⼥]
  \begin{phonetics}{贵姓}{gui4xing4}
    \definition{expr.}{qual seu sobrenome?}
  \end{phonetics}
\end{entry}

\begin{entry}{贸易}{9,8}[Radicais ⾙、⽇]
  \begin{phonetics}{贸易}{mao4yi4}
    \definition[个]{s.}{transação comercial}
    \definition{v.}{fazer uma transação comercial}
  \end{phonetics}
\end{entry}

\begin{entry}{费}{9}[Radical ⾙]
  \begin{phonetics}{费}{fei4}[][HSK 3]
    \definition*{s.}{Fei}
    \definition{s.}{taxa; despesa; encargo}
    \definition{v.}{custar; gastar; desperdiçar}
  \end{phonetics}
\end{entry}

\begin{entry}{费用}{9,5}[Radicais ⾙、⽤]
  \begin{phonetics}{费用}{fei4 yong4}[][HSK 3]
    \definition[笔,个]{s.}{custo; despesa; desembolso}
  \end{phonetics}
\end{entry}

\begin{entry}{贺}{9}[Radical ⾙]
  \begin{phonetics}{贺}{he4}
    \definition*{s.}{sobrenome He}
    \definition{v.}{parabenizar | congratular}
  \end{phonetics}
\end{entry}

\begin{entry}{轴承}{9,8}[Radicais ⾞、⼿]
  \begin{phonetics}{轴承}{zhou2cheng2}
    \definition{s.}{(mecânico) rolamento}
  \end{phonetics}
\end{entry}

\begin{entry}{轻}{9}[Radical ⾞]
  \begin{phonetics}{轻}{qing1}[][HSK 2]
    \definition{adj.}{leve | pequeno em número, grau, etc. | não importante | relaxado}
    \definition{adv.}{suavemente | levemente | precipitadamente}
    \definition{v.}{menosprezar}
  \end{phonetics}
\end{entry}

\begin{entry}{轻易}{9,8}[Radicais ⾞、⽇]
  \begin{phonetics}{轻易}{qing1yi4}[][HSK 4]
    \definition{adj.}{fácil; simples}
    \definition{adv.}{facilmente; prontamente | facilmente; precipitadamente; indica que uma ação é realizada casualmente, geralmente usado em frases negativas}
  \end{phonetics}
\end{entry}

\begin{entry}{轻松}{9,8}[Radicais ⾞、⽊]
  \begin{phonetics}{轻松}{qing1song1}[][HSK 4]
    \definition{adj.}{leve; relaxado; livre de fardos; não se sentir nervoso ou cansado}
    \definition{v.}{relaxar; levar as coisas menos a sério}
  \end{phonetics}
\end{entry}

\begin{entry}{迷}{9}[Radical ⾡]
  \begin{phonetics}{迷}{mi2}[][HSK 3]
    \definition*{s.}{sobrenome Mi}
    \definition{adj.}{perdido; confuso}
    \definition{s.}{fã; entusiasta; fanático}
    \definition{v.}{estar confuso; perder o rumo; se perder-se | ficar fascinado por; entregar-se a; ficar encantado com (por); ser louco por | confundir; desorientar; fascinar; encantar}
  \end{phonetics}
\end{entry}

\begin{entry}{迷人}{9,2}[Radicais ⾡、⼈]
  \begin{phonetics}{迷人}{mi2ren2}
    \definition{adj.}{fascinante | encantador | tentador}
  \end{phonetics}
\end{entry}

\begin{entry}{迷你}{9,7}[Radicais ⾡、⼈]
  \begin{phonetics}{迷你}{mi2ni3}
    \definition{adj.}{(empréstimo linguístico) mini, como em minissaia ou \emph{Mini Cooper}}
  \end{phonetics}
\end{entry}

\begin{entry}{迷宫}{9,9}[Radicais ⾡、⼧]
  \begin{phonetics}{迷宫}{mi2gong1}
    \definition{s.}{labirinto}
  \end{phonetics}
\end{entry}

\begin{entry}{迷恋}{9,10}[Radicais ⾡、⼼]
  \begin{phonetics}{迷恋}{mi2lian4}
    \definition{adj.}{obcecado}
    \definition{v.}{estar/ser apaixonado por | ficar encantado por | estar/ser obcecado por}
  \end{phonetics}
\end{entry}

\begin{entry}{迷路}{9,13}[Radicais ⾡、⾜]
  \begin{phonetics}{迷路}{mi2lu4}
    \definition{s.}{labirinto | ouvido interno}
    \definition{v.+compl.}{perder o caminho | perder-se | seguir pelo caminho errado | não conseguir encontrar o caminho}
  \end{phonetics}
\end{entry}

\begin{entry}{追}{9}[Radical ⾡]
  \begin{phonetics}{追}{zhui1}[][HSK 3]
    \definition*{s.}{sobrenome Zhui}
    \definition{v.}{perseguir; correr atrás; ir atrás de; alcançar | rastrear; investigar; chegar ao fundo de | ansiar por (depois); ir atrás; procurar | recordar; relembrar; lembrar | agir retroativamente; fazer postumamente}
  \end{phonetics}
\end{entry}

\begin{entry}{追赶}{9,10}[Radicais ⾡、⾛]
  \begin{phonetics}{追赶}{zhui1gan3}
    \definition{v.}{perseguir | acelerar | alcançar | ultrapassar}
  \end{phonetics}
\end{entry}

\begin{entry}{退}{9}[Radical ⾡]
  \begin{phonetics}{退}{tui4}[][HSK 3]
    \definition{v.}{recuar; mover-se para trás | fazer recuar; remover; retirar | desistir; retirar-se de | retroceder; refluir; declinar | desaparecer; desvanecer | devolver; retornar | cancelar; rescindir; romper}
  \end{phonetics}
\end{entry}

\begin{entry}{退出}{9,5}[Radicais ⾡、⼐]
  \begin{phonetics}{退出}{tui4 chu1}[][HSK 3]
    \definition{v.}{desistir; retirar-se; separar-se; retirar-se de}
  \end{phonetics}
\end{entry}

\begin{entry}{退休}{9,6}[Radicais ⾡、⼈]
  \begin{phonetics}{退休}{tui4xiu1}[][HSK 3]
    \definition{v.+compl.}{aposentar-se}
  \end{phonetics}
\end{entry}

\begin{entry}{送}{9}[Radical ⾡]
  \begin{phonetics}{送}{song4}[][HSK 1]
    \definition{v.}{distribuir | entregar | dar | oferecer (alguma coisa como presente) | enviar | remeter}
  \end{phonetics}
\end{entry}

\begin{entry}{送到}{9,8}[Radicais ⾡、⼑]
  \begin{phonetics}{送到}{song4 dao4}[][HSK 2]
    \definition{v.}{enviar para (lugar)}
  \end{phonetics}
\end{entry}

\begin{entry}{送给}{9,9}[Radicais ⾡、⽷]
  \begin{phonetics}{送给}{song4 gei3}[][HSK 2]
    \definition{v.}{dar a (alguém ou organização)}
  \end{phonetics}
\end{entry}

\begin{entry}{适用}{9,5}[Radicais ⾡、⽤]
  \begin{phonetics}{适用}{shi4 yong4}[][HSK 3]
    \definition{adj.}{adequado; aplicável}
    \definition{v.}{ser aplicável; ser adequado}
  \end{phonetics}
\end{entry}

\begin{entry}{适合}{9,6}[Radicais ⾡、⼝]
  \begin{phonetics}{适合}{shi4he2}[][HSK 3]
    \definition{v.}{servir (uma roupa); caber; se adequar}
  \end{phonetics}
\end{entry}

\begin{entry}{适应}{9,7}[Radicais ⾡、⼴]
  \begin{phonetics}{适应}{shi4ying4}[][HSK 3]
    \definition{v.}{ajustar-se; adequar-se; adaptar-se}
  \end{phonetics}
\end{entry}

\begin{entry}{逃}{9}[Radical ⾡]
  \begin{phonetics}{逃}{tao2}
    \definition{v.}{escapar | fugir}
  \end{phonetics}
\end{entry}

\begin{entry}{逆境}{9,14}[Radicais ⾡、⼟]
  \begin{phonetics}{逆境}{ni4jing4}
    \definition{s.}{adversidade | tribulação}
  \end{phonetics}
\end{entry}

\begin{entry}{选}{9}[Radical ⾡]
  \begin{phonetics}{选}{xuan3}[][HSK 2]
    \definition{s.}{seleções | antologia}
    \definition{v.}{selecionar | escolher | eleger}
  \end{phonetics}
\end{entry}

\begin{entry}{选手}{9,4}[Radicais ⾡、⼿]
  \begin{phonetics}{选手}{xuan3shou3}[][HSK 3]
    \definition[位]{s.}{jogador; competidor (selecionado); atleta selecionado para uma competição esportiva}
  \end{phonetics}
\end{entry}

\begin{entry}{选择}{9,8}[Radicais ⾡、⼿]
  \begin{phonetics}{选择}{xuan3ze2}
    \definition{s.}{escolha | opção | alternativa}
    \definition{v.}{selecionar | escolher}
  \end{phonetics}
\end{entry}

\begin{entry}{重}{9}[Radical ⾥]
  \begin{phonetics}{重}{chong2}
    \definition*{s.}{sobrenome Chong}
    \definition{adv.}{novamente; mais uma vez}
    \definition{clas.}{para camadas}
    \definition{v.}{repetir; duplicar}
  \end{phonetics}
  \begin{phonetics}{重}{zhong4}[][HSK 1,3]
    \definition{adj.}{pesado | profundo; sério | importante; momentoso | discreto; prudente | considerável em quantidade ou valor}
    \definition{adv.}{pesadamente; severamente}
    \definition{v.}{colocar (pôr) ênfase em; dar valor a; atribuir importância a}
  \end{phonetics}
\end{entry}

\begin{entry}{重大}{9,3}[Radicais ⾥、⼤]
  \begin{phonetics}{重大}{zhong4da4}[][HSK 3]
    \definition{adj.}{grande; importante; significativo; de grande importância}
  \end{phonetics}
\end{entry}

\begin{entry}{重阳节}{9,6,5}[Radicais ⾥、⾩、⾋]
  \begin{phonetics}{重阳节}{chong2yang2jie2}
    \definition*{s.}{Festa do Duplo Nove, Festival Yang, dia de subir aos lugares mais altos para evitar calamidades e dia do culto aos antepassados (9º dia do nono mês lunar)}
  \end{phonetics}
\end{entry}

\begin{entry}{重视}{9,8}[Radicais ⾥、⾒]
  \begin{phonetics}{重视}{zhong4shi4}[][HSK 2]
    \definition{v.}{atribuir valor a | dar peso a | atribuir importância a | prestar atenção a}
  \end{phonetics}
\end{entry}

\begin{entry}{重复}{9,9}[Radicais ⾥、⼢]
  \begin{phonetics}{重复}{chong2fu4}[][HSK 2]
    \definition{v.}{repetir | iterar | duplicar | reduplicar | fazer algo de novo}
  \end{phonetics}
\end{entry}

\begin{entry}{重点}{9,9}[Radicais ⾥、⽕]
  \begin{phonetics}{重点}{chong2dian3}
    \definition{adj./adv./s.}{nota-chave | ponto-chave | ponto focal | ênfase}
  \end{phonetics}
  \begin{phonetics}{重点}{zhong4dian3}[][HSK 2]
    \definition{s.}{nota-chave | ponto-chave | ponto focal | ênfase}
  \end{phonetics}
\end{entry}

\begin{entry}{重要}{9,9}[Radicais ⾥、⾑]
  \begin{phonetics}{重要}{zhong4yao4}[][HSK 1]
    \definition{adj.}{importante | significativo | principal}
  \end{phonetics}
\end{entry}

\begin{entry}{重重}{9,9}[Radicais ⾥、⾥]
  \begin{phonetics}{重重}{chong2chong2}
    \definition{adv.}{camada após camada | um após o outro}
  \end{phonetics}
  \begin{phonetics}{重重}{zhong4zhong4}
    \definition{adv.}{fortemente | severamente}
  \end{phonetics}
\end{entry}

\begin{entry}{重逢}{9,10}[Radicais ⾥、⾡]
  \begin{phonetics}{重逢}{chong2feng2}
    \definition{s.}{reunião}
    \definition{v.}{encontrar-se novamente | reunir-se}
  \end{phonetics}
\end{entry}

\begin{entry}{重量}{9,12}[Radicais ⾥、⾥]
  \begin{phonetics}{重量}{zhong4liang4}
    \definition[个]{s.}{peso}
  \end{phonetics}
\end{entry}

\begin{entry}{重新}{9,13}[Radicais ⾥、⽄]
  \begin{phonetics}{重新}{chong2xin1}[][HSK 2]
    \definition{adv.}{de novo | novamente}
  \end{phonetics}
\end{entry}

\begin{entry}{钟}{9}[Radical ⾦]
  \begin{phonetics}{钟}{zhong1}[][HSK 3]
    \definition*{s.}{sobrenome Zhong}
    \definition[顶,个,口]{s.}{sino; campainha; um chocalho feito de cobre ou ferro | relógio; temporizador | tempo; refere-se à hora, ao tempo | copo sem alça; xícara sem alça}
    \definition{v.}{concentrar (as afeições de alguém, etc.)}
  \end{phonetics}
\end{entry}

\begin{entry}{钟室}{9,9}[Radicais ⾦、⼧]
  \begin{phonetics}{钟室}{zhong1shi4}
    \definition{s.}{campanário | sala do relógio}
  \end{phonetics}
\end{entry}

\begin{entry}{钟罩}{9,13}[Radicais ⾦、⽹]
  \begin{phonetics}{钟罩}{zhong1zhao4}
    \definition{s.}{redoma | dossel de sino}
  \end{phonetics}
\end{entry}

\begin{entry}{钢}{9}[Radical ⾦]
  \begin{phonetics}{钢}{gang1}
    \definition{s.}{aço}
  \end{phonetics}
\end{entry}

\begin{entry}{钢丝}{9,5}[Radicais ⾦、⼀]
  \begin{phonetics}{钢丝}{gang1si1}
    \definition{s.}{cabo de aço | corda bamba}
  \end{phonetics}
\end{entry}

\begin{entry}{钢琴}{9,12}[Radicais ⾦、⽟]
  \begin{phonetics}{钢琴}{gang1qin2}
    \definition[架,台]{s.}{piano}
  \end{phonetics}
\end{entry}

\begin{entry}{钥匙}{9,11}[Radicais ⾦、⼔]
  \begin{phonetics}{钥匙}{yao4shi5}
    \definition[把]{s.}{chave}
  \end{phonetics}
\end{entry}

\begin{entry}{钥匙孔}{9,11,4}[Radicais ⾦、⼔、⼦]
  \begin{phonetics}{钥匙孔}{yao4shi5kong3}
    \definition{s.}{buraco da fechadura}
  \end{phonetics}
\end{entry}

\begin{entry}{钥匙卡}{9,11,5}[Radicais ⾦、⼔、⼘]
  \begin{phonetics}{钥匙卡}{yao4shi5ka3}
    \definition{s.}{cartão de acesso}
  \end{phonetics}
\end{entry}

\begin{entry}{钥匙洞孔}{9,11,9,4}[Radicais ⾦、⼔、⽔、⼦]
  \begin{phonetics}{钥匙洞孔}{yao4shi5dong4kong3}
    \definition{s.}{buraco da fechadura}
  \end{phonetics}
\end{entry}

\begin{entry}{钥匙圈}{9,11,11}[Radicais ⾦、⼔、⼞]
  \begin{phonetics}{钥匙圈}{yao4shi5quan1}
    \definition{s.}{chaveiro}
  \end{phonetics}
\end{entry}

\begin{entry}{钩}{9}[Radical ⾦]
  \begin{phonetics}{钩}{gou1}
    \definition{s.}{gancho | \emph{check mark} | \emph{tick}}
    \definition{v.}{enganchar | costurar}
  \end{phonetics}
\end{entry}

\begin{entry}{闻}{9}[Radical ⾨]
  \begin{phonetics}{闻}{wen2}[][HSK 2]
    \definition*{s.}{sobrenome Wen}
    \definition{s.}{notícias | reputação | fama}
    \definition{v.}{ouvir | cheirar | farejar}
  \end{phonetics}
\end{entry}

\begin{entry}{阁下}{9,3}[Radicais ⾨、⼀]
  \begin{phonetics}{阁下}{ge2xia4}
    \definition{pron.}{Sua Excelência | Sua Majestade | \emph{Sire}}
  \end{phonetics}
\end{entry}

\begin{entry}{院}{9}[Radical ⾩]
  \begin{phonetics}{院}{yuan4}[][HSK 2]
    \definition[个]{s.}{pátio | instituição}
  \end{phonetics}
\end{entry}

\begin{entry}{院子}{9,3}[Radicais ⾩、⼦]
  \begin{phonetics}{院子}{yuan4zi5}[][HSK 2]
    \definition[个]{s.}{pátio | jardim | quintal}
  \end{phonetics}
\end{entry}

\begin{entry}{院长}{9,4}[Radicais ⾩、⾧]
  \begin{phonetics}{院长}{yuan4zhang3}[][HSK 2]
    \definition[个]{s.}{presidente de um conselho | reitor | chefe de departamento | primeiro-ministro da República da China | presidente de uma universidade}
  \end{phonetics}
\end{entry}

\begin{entry}{除了}{9,2}[Radicais ⾩、⼅]
  \begin{phonetics}{除了}{chu2le5}[][HSK 3]
    \definition{prep.}{exceto; à parte | além disso; além de | ou \dots ou \dots}
  \end{phonetics}
\end{entry}

\begin{entry}{除非}{9,8}[Radicais ⾩、⾮]
  \begin{phonetics}{除非}{chu2fei1}
    \definition{conj.}{a menos que | somente se}
  \end{phonetics}
\end{entry}

\begin{entry}{面}{9}[Kangxi 176][Radical ⾯]
  \begin{phonetics}{面}{mian4}[][HSK 2]
    \definition{clas.}{para objetos com superfície plana como tambores, espelhos, bandeiras, etc.}
    \definition{s.}{farinha | massa | (gíria) (uma pessoa) ineficaz | face | superfície | lado | lado de fora}
  \end{phonetics}
\end{entry}

\begin{entry}{面包}{9,5}[Radicais ⾯、⼓]
  \begin{phonetics}{面包}{mian4bao1}[][HSK 1]
    \definition[个,片,袋,块]{s.}{pão}[我买八个面包了。(Comprei oito pães.) | 他在吃两片面包。(Ele está comendo duas fatias de pão.) | 我在家里带了一袋面包。(Trouxe um saco de pão para casa.) | 我拿了一块面包。(Peguei um pedaço de pão.)]
  \end{phonetics}
\end{entry}

\begin{entry}{面对}{9,5}[Radicais ⾯、⼨]
  \begin{phonetics}{面对}{mian4dui4}[][HSK 3]
    \definition{v.}{enfrentar; defrontar | confrontar (problema)}
  \end{phonetics}
\end{entry}

\begin{entry}{面对面}{9,5,9}[Radicais ⾯、⼨、⾯]
  \begin{phonetics}{面对面}{mian4dui4mian4}
    \definition{expr.}{cara a cara}
  \end{phonetics}
\end{entry}

\begin{entry}{面对面吃面}{9,5,9,6,9}[Radicais ⾯、⼨、⾯、⼝、⾯]
  \begin{phonetics}{面对面吃面}{mian4dui4mian4 chi1 mian4}
    \definition{expr.}{Comer macarrão cara a cara; indica que o seu estado atual, ou algumas das posições em que você está, ou algumas das coisas que você fez são muito claras}
  \end{phonetics}
\end{entry}

\begin{entry}{面团}{9,6}[Radicais ⾯、⼞]
  \begin{phonetics}{面团}{mian4tuan2}
    \definition{s.}{massa | pasta}
  \end{phonetics}
\end{entry}

\begin{entry}{面条}{9,7}[Radicais ⾯、⽊]
  \begin{phonetics}{面条}{mian4tiao2}
    \definition{s.}{macarrão | espaguete}
  \end{phonetics}
\end{entry}

\begin{entry}{面条儿}{9,7,2}[Radicais ⾯、⽊、⼉]
  \begin{phonetics}{面条儿}{mian4tiao2r5}[][HSK 1]
    \definition{s.}{macarrão | \emph{noodles}}
  \end{phonetics}
\end{entry}

\begin{entry}{面试}{9,8}[Radicais ⾯、⾔]
  \begin{phonetics}{面试}{mian4 shi4}[][HSK 4]
    \definition[次]{s.}{entrevista; audição}
  \end{phonetics}
\end{entry}

\begin{entry}{面临}{9,9}[Radicais ⾯、⼁]
  \begin{phonetics}{面临}{mian4lin2}[][HSK 4]
    \definition{v.}{ser confrontado com; encontrar (uma situação) na frente de}
  \end{phonetics}
\end{entry}

\begin{entry}{面前}{9,9}[Radicais ⾯、⼑]
  \begin{phonetics}{面前}{mian4 qian2}[][HSK 2]
    \definition{adv.}{antes | na frente de | na (frente) de}
  \end{phonetics}
\end{entry}

\begin{entry}{面积}{9,10}[Radicais ⾯、⽲]
  \begin{phonetics}{面积}{mian4ji1}[][HSK 3]
    \definition{s.}{área (de um andar, pedaço de terreno, etc.); área de uma superfície}
  \end{phonetics}
\end{entry}

\begin{entry}{韭菜}{9,11}[Radicais ⾲、⾋]
  \begin{phonetics}{韭菜}{jiu3cai4}
    \definition{s.}{cebolinha chinesa | (figurativo) investidores de varejo que perdem seu dinheiro para operadores mais experientes (ou seja, são ``colhidos'' como cebolinhas)}
  \end{phonetics}
\end{entry}

\begin{entry}{音乐}{9,5}[Radicais ⾳、⼃]
  \begin{phonetics}{音乐}{yin1yue4}[][HSK 2]
    \definition[张,曲,段]{s.}{música}
  \end{phonetics}
\end{entry}

\begin{entry}{音乐厅}{9,5,4}[Radicais ⾳、⼃、⼚]
  \begin{phonetics}{音乐厅}{yin1yue4ting1}
    \definition{s.}{auditório | teatro | \emph{concert hall}}
  \end{phonetics}
\end{entry}

\begin{entry}{音乐节}{9,5,5}[Radicais ⾳、⼃、⾋]
  \begin{phonetics}{音乐节}{yin1yue4jie2}
    \definition{s.}{festival de música}
  \end{phonetics}
\end{entry}

\begin{entry}{音乐会}{9,5,6}[Radicais ⾳、⼃、⼈]
  \begin{phonetics}{音乐会}{yin1 yue4 hui4}[][HSK 2]
    \definition[场]{s.}{concerto}
  \end{phonetics}
\end{entry}

\begin{entry}{音乐光碟}{9,5,6,14}[Radicais ⾳、⼃、⼉、⽯]
  \begin{phonetics}{音乐光碟}{yin1yue4guang1die2}
    \definition{s.}{CD de música}
  \end{phonetics}
\end{entry}

\begin{entry}{音乐学}{9,5,8}[Radicais ⾳、⼃、⼦]
  \begin{phonetics}{音乐学}{yin1yue4xue2}
    \definition{s.}{musicologia}
  \end{phonetics}
\end{entry}

\begin{entry}{音乐学院}{9,5,8,9}[Radicais ⾳、⼃、⼦、⾩]
  \begin{phonetics}{音乐学院}{yin1yue4xue2yuan4}
    \definition{s.}{conservatório | academia de música}
  \end{phonetics}
\end{entry}

\begin{entry}{音乐院}{9,5,9}[Radicais ⾳、⼃、⾩]
  \begin{phonetics}{音乐院}{yin1yue4yuan4}
    \definition{s.}{conservatório | instituto de música}
  \end{phonetics}
\end{entry}

\begin{entry}{音乐家}{9,5,10}[Radicais ⾳、⼃、⼧]
  \begin{phonetics}{音乐家}{yin1yue4jia1}
    \definition{s.}{músico}
  \end{phonetics}
\end{entry}

\begin{entry}{音节}{9,5}[Radicais ⾳、⾋]
  \begin{phonetics}{音节}{yin1 jie2}[][HSK 2]
    \definition{s.}{sílaba}
  \end{phonetics}
\end{entry}

\begin{entry}{顺}{9}[Radical ⾴]
  \begin{phonetics}{顺}{shun4}
    \definition{adj.}{correr bem | favorável}
  \end{phonetics}
\end{entry}

\begin{entry}{顺从}{9,4}[Radicais ⾴、⼈]
  \begin{phonetics}{顺从}{shun4cong2}
    \definition{v.}{obedecer | submeter-se}
  \end{phonetics}
\end{entry}

\begin{entry}{顺心}{9,4}[Radicais ⾴、⼼]
  \begin{phonetics}{顺心}{shun4xin1}
    \definition{adj.}{satisfatório | satisfeito}
  \end{phonetics}
\end{entry}

\begin{entry}{顺水}{9,4}[Radicais ⾴、⽔]
  \begin{phonetics}{顺水}{shun4shui3}
    \definition{v.}{ir com o fluxo}
  \end{phonetics}
\end{entry}

\begin{entry}{顺延}{9,6}[Radicais ⾴、⼵]
  \begin{phonetics}{顺延}{shun4yan2}
    \definition{v.}{adiar | procrastinar}
  \end{phonetics}
\end{entry}

\begin{entry}{顺当}{9,6}[Radicais ⾴、⼹]
  \begin{phonetics}{顺当}{shun4dang5}
    \definition{adv.}{suavemente}
  \end{phonetics}
\end{entry}

\begin{entry}{顺耳}{9,6}[Radicais ⾴、⽿]
  \begin{phonetics}{顺耳}{shun4'er3}
    \definition{adj.}{agradável ao ouvido}
  \end{phonetics}
\end{entry}

\begin{entry}{顺利}{9,7}[Radicais ⾴、⼑]
  \begin{phonetics}{顺利}{shun4li4}[][HSK 2]
    \definition{adv.}{suavemente | sem problemas}
  \end{phonetics}
\end{entry}

\begin{entry}{顺序}{9,7}[Radicais ⾴、⼴]
  \begin{phonetics}{顺序}{shun4xu4}[][HSK 4]
    \definition{adv.}{por sua vez; na ordem correta; na devida ordem; na ordem adequada; na ordem apropriada}
    \definition[个]{s.}{ordem; sequência; sucessão; subsequência; sequência simples; ordem de prioridade}
  \end{phonetics}
\end{entry}

\begin{entry}{顺畅}{9,8}[Radicais ⾴、⽥]
  \begin{phonetics}{顺畅}{shun4chang4}
    \definition{adj.}{liso e sem obstáculos | fluente}
  \end{phonetics}
\end{entry}

\begin{entry}{顺便}{9,9}[Radicais ⾴、⼈]
  \begin{phonetics}{顺便}{shun4bian4}
    \definition{adv.}{convenientemente | de passagem | sem muito esforço extra}
  \end{phonetics}
\end{entry}

\begin{entry}{顺叙}{9,9}[Radicais ⾴、⼜]
  \begin{phonetics}{顺叙}{shun4xu4}
    \definition{s.}{narrativa cronológica}
  \end{phonetics}
\end{entry}

\begin{entry}{顺眼}{9,11}[Radicais ⾴、⽬]
  \begin{phonetics}{顺眼}{shun4yan3}
    \definition{adj.}{agradável aos olhos}
  \end{phonetics}
\end{entry}

\begin{entry}{顺境}{9,14}[Radicais ⾴、⼟]
  \begin{phonetics}{顺境}{shun4jing4}
    \definition{s.}{circunstâncias favoráveis}
  \end{phonetics}
\end{entry}

\begin{entry}{顺嘴}{9,16}[Radicais ⾴、⼝]
  \begin{phonetics}{顺嘴}{shun4zui3}
    \definition{v.}{deixar escapar (sem pensar) | ler suavemente (texto) | adequar-se  ao gosto (comida)}
  \end{phonetics}
\end{entry}

\begin{entry}{飒飒}{9,9}[Radicais ⾵、⾵]
  \begin{phonetics}{飒飒}{sa4sa4}
    \definition{s.}{o farfalhar | sussurro | murmúrio (do vento nas árvores, o mar, etc.)}
  \end{phonetics}
\end{entry}

\begin{entry}{食物}{9,8}[Radicais ⾷、⽜]
  \begin{phonetics}{食物}{shi2wu4}[][HSK 2]
    \definition[种]{s.}{comida}
  \end{phonetics}
\end{entry}

\begin{entry}{食品}{9,9}[Radicais ⾷、⼝]
  \begin{phonetics}{食品}{shi2 pin3}[][HSK 3]
    \definition[种]{s.}{comida; gêneros alimentícios; provisões}
  \end{phonetics}
\end{entry}

\begin{entry}{食堂}{9,11}[Radicais ⾷、⼟]
  \begin{phonetics}{食堂}{shi2 tang2}[][HSK 4]
    \definition[个,间]{s.}{cantina; refeitório}
  \end{phonetics}
\end{entry}

\begin{entry}{饺子}{9,3}[Radicais ⾷、⼦]
  \begin{phonetics}{饺子}{jiao3zi5}[][HSK 2]
    \definition[个,只]{s.}{jiaozi | bolinhos chineses | bolinho de massa}
  \end{phonetics}
\end{entry}

\begin{entry}{饼}{9}[Radical ⾷]
  \begin{phonetics}{饼}{bing3}
    \definition[张]{s.}{panqueca | biscoito | torta}
  \end{phonetics}
\end{entry}

\begin{entry}{饼干}{9,3}[Radicais ⾷、⼲]
  \begin{phonetics}{饼干}{bing3gan1}
    \definition[片,块]{s.}{bolacha | biscoito}
  \end{phonetics}
\end{entry}

\begin{entry}{首}{9}[Radical ⾸]
  \begin{phonetics}{首}{shou3}[][HSK 4]
    \definition*{s.}{sobrenome Shou}
    \definition{adj.}{primeiro}
    \definition{adv.}{inicialmente; como o primeiro; em primeiro lugar}
    \definition{clas.}{para canções e poemas}
    \definition{s.}{cabeça | cabeça; chefe; líder | capital (cidade)}
    \definition{v.}{apresentar acusações contra alguém}
  \end{phonetics}
\end{entry}

\begin{entry}{首先}{9,6}[Radicais ⾸、⼉]
  \begin{phonetics}{首先}{shou3xian1}[][HSK 3]
    \definition{adv.}{primeiramente; antes de todos os outros}
    \definition{conj.}{acima de tudo; primeiramente; em primeiro lugar}
  \end{phonetics}
\end{entry}

\begin{entry}{首相}{9,9}[Radicais ⾸、⽬]
  \begin{phonetics}{首相}{shou3xiang4}
    \definition*{s.}{Primeiro-Ministro (Japão, UK, etc.)}
  \end{phonetics}
\end{entry}

\begin{entry}{首席执行官}{9,10,6,6,8}[Radicais ⾸、⼱、⼿、⾏、⼧]
  \begin{phonetics}{首席执行官}{shou3xi2 zhi2xing2 guan1}
    \definition{s.}{\emph{chief executive officer}, CEO}
  \end{phonetics}
\end{entry}

\begin{entry}{首都}{9,10}[Radicais ⾸、⾢]
  \begin{phonetics}{首都}{shou3du1}[][HSK 3]
    \definition[个]{s.}{capital (cidade)}
  \end{phonetics}
\end{entry}

\begin{entry}{香}{9}[Kangxi 186][Radical ⾹]
  \begin{phonetics}{香}{xiang1}[][HSK 3]
    \definition*{s.}{sobrenome Xiang}
    \definition{adj.}{aromático; perfumado; fragrante; cheiroso | saboroso; saboroso; delicioso; apetitoso | com gosto; com bom apetite | (sono) profundo | popular; bem-vindo}
    \definition[束,根,炷]{s.}{especiaria; perfume; fragrância; aromatizante | incenso | relacionado a mulheres ou às próprias mulheres}
  \end{phonetics}
\end{entry}

\begin{entry}{香气}{9,4}[Radicais ⾹、⽓]
  \begin{phonetics}{香气}{xiang1qi4}
    \definition{s.}{fragrância | aroma | incenso}
  \end{phonetics}
\end{entry}

\begin{entry}{香皂}{9,7}[Radicais ⾹、⽩]
  \begin{phonetics}{香皂}{xiang1zao4}
    \definition{s.}{sabonete | sabonete perfumado}
  \end{phonetics}
\end{entry}

\begin{entry}{香肠}{9,7}[Radicais ⾹、⾁]
  \begin{phonetics}{香肠}{xiang1chang2}
    \definition[根]{s.}{salsicha}
  \end{phonetics}
\end{entry}

\begin{entry}{香味}{9,8}[Radicais ⾹、⼝]
  \begin{phonetics}{香味}{xiang1wei4}
    \definition[股]{s.}{fragrância | cheiro doce}
  \end{phonetics}
\end{entry}

\begin{entry}{香波}{9,8}[Radicais ⾹、⽔]
  \begin{phonetics}{香波}{xiang1bo1}
    \definition{s.}{xampu}
  \end{phonetics}
\end{entry}

\begin{entry}{香炉}{9,8}[Radicais ⾹、⽕]
  \begin{phonetics}{香炉}{xiang1lu2}
    \definition{s.}{incensário (para queimar incenso) | queimador de incenso | insensório, turíbulo}
  \end{phonetics}
\end{entry}

\begin{entry}{香烟}{9,10}[Radicais ⾹、⽕]
  \begin{phonetics}{香烟}{xiang1yan1}
    \definition[支,条]{s.}{cigarro | fumaça de incenso queimado}
  \end{phonetics}
\end{entry}

\begin{entry}{香艳}{9,10}[Radicais ⾹、⾊]
  \begin{phonetics}{香艳}{xiang1yan4}
    \definition{adj.}{atraente | erótico | romântico}
  \end{phonetics}
\end{entry}

\begin{entry}{香港}{9,12}[Radicais ⾹、⽔]
  \begin{phonetics}{香港}{xiang1gang3}
    \definition*{s.}{Hong Kong}
  \seealsoref{香港岛}{xiang1gang3 dao3}
  \end{phonetics}
\end{entry}

\begin{entry}{香港岛}{9,12,7}[Radicais ⾹、⽔、⼭]
  \begin{phonetics}{香港岛}{xiang1gang3 dao3}
    \definition*{s.}{Ilha de Hong Kong}
  \seealsoref{香港}{xiang1gang3}
  \end{phonetics}
\end{entry}

\begin{entry}{香槟酒}{9,14,10}[Radicais ⾹、⽊、⾣]
  \begin{phonetics}{香槟酒}{xiang1bin1jiu3}
    \definition[杯]{s.}{(empréstimo linguístico) \emph{champagne}}
  \end{phonetics}
\end{entry}

\begin{entry}{香蕈}{9,15}[Radicais ⾹、⾋]
  \begin{phonetics}{香蕈}{xiang1xun4}
    \definition{s.}{\emph{shiitake}, cogumelo comestível}
  \end{phonetics}
\end{entry}

\begin{entry}{香蕉}{9,15}[Radicais ⾹、⾋]
  \begin{phonetics}{香蕉}{xiang1jiao1}[][HSK 3]
    \definition[枝,根,个,把,串,束,弓]{s.}{banana}
  \end{phonetics}
\end{entry}

\begin{entry}{骂}{9}[Radical ⾺]
  \begin{phonetics}{骂}{ma4}
    \definition{v.}{insultar | maldizer | ralhar}
  \end{phonetics}
\end{entry}

\begin{entry}{骂名}{9,6}[Radicais ⾺、⼝]
  \begin{phonetics}{骂名}{ma4ming2}
    \definition{s.}{infâmia}
  \end{phonetics}
\end{entry}

\begin{entry}{骂街}{9,12}[Radicais ⾺、⾏]
  \begin{phonetics}{骂街}{ma4jie1}
    \definition{v.}{gritar abusos na rua}
  \end{phonetics}
\end{entry}

\begin{entry}{骆驼}{9,8}[Radicais ⾺、⾺]
  \begin{phonetics}{骆驼}{luo4tuo5}
    \definition[峰,匹,头]{s.}{camelo | (coloquial) cabeça-dura, idiota}
  \end{phonetics}
\end{entry}

\begin{entry}{骨}{9}[Kangxi 188][Radical ⾻]
  \begin{phonetics}{骨}{gu3}
    \definition{s.}{osso}
  \end{phonetics}
\end{entry}

\begin{entry}{骨头}{9,5}[Radicais ⾻、⼤]
  \begin{phonetics}{骨头}{gu3tou5}[][HSK 4]
    \definition[根,块]{s.}{osso; tecidos mais duros no corpo de uma pessoa ou de alguns animais que sustentam o corpo ou protegem os órgãos do corpo | caráter de uma pessoa; refere-se à qualidade do caráter de uma pessoa}
  \end{phonetics}
\end{entry}

\begin{entry}{鬼火}{9,4}[Radicais ⿁、⽕]
  \begin{phonetics}{鬼火}{gui3huo3}
    \definition{s.}{fogo-fátuo | boitatá | fogo corredor | fogo de santelmo}
  \end{phonetics}
\end{entry}

\begin{entry}{鬼怪}{9,8}[Radicais ⿁、⼼]
  \begin{phonetics}{鬼怪}{gui3guai4}
    \definition{s.}{\emph{hobgoblin} | bicho-papão | fantasma}
  \end{phonetics}
\end{entry}

%%%%% EOF %%%%%

