%%%
%%% 9画
%%%

\section*{9画}\addcontentsline{toc}{section}{9画}

\begin{Entry}{临}{9}{⼁}
  \begin{Phonetics}{临}{lin2}
    \definition*{s.}{Sobrenome Lin}
    \definition{adv.}{pouco antes; prestes a; no ponto de; indica que uma ação está prestes a ocorrer}
    \definition{v.}{encarar; enfrentar; aproximar-se | chegar; estar presente | copiar (um modelo de caligrafia ou pintura); traçar sobre as palavras ou figuras | olhar de cima para baixo | ir de cima para baixo}
  \end{Phonetics}
\end{Entry}

\begin{Entry}{临时}{9,7}{⼁、⽇}
  \begin{Phonetics}{临时}{lin2shi2}[][HSK 4]
    \definition{adj.}{temporário; provisório; por um breve período}
    \definition{adv.}{no momento em que algo acontece (quando as coisas dão errado)}
  \end{Phonetics}
\end{Entry}

\begin{Entry}{临近}{9,7}{⼁、⾡}
  \begin{Phonetics}{临近}{lin2jin4}
    \definition{v.}{aproximar-se; estar perto de}
  \end{Phonetics}
\end{Entry}

\begin{Entry}{举}{9}{⼂}
  \begin{Phonetics}{举}{ju3}[][HSK 2]
    \definition*{s.}{Sobrenome Ju}
    \definition{adj.}{inteiro; completo}
    \definition{s.}{ato; ação; movimento; comportamento | (nas dinastias Ming e Qing) candidato aprovado nos exames imperiais a nível provincial}
    \definition{v.}{levantar; erguer; sustentar | começar; iniciar; surgir | eleger; escolher; recomendar; selecionar | citar; enumerar; propor; revelar}
  \end{Phonetics}
\end{Entry}

\begin{Entry}{举办}{9,4}{⼂、⼒}
  \begin{Phonetics}{举办}{ju3ban4}[][HSK 3]
    \definition{v.}{conduzir; organizar; realizar}
  \end{Phonetics}
\end{Entry}

\begin{Entry}{举手}{9,4}{⼂、⼿}
  \begin{Phonetics}{举手}{ju3 shou3}[][HSK 2]
    \definition{v.}{levantar a mão ou as mãos; levantar a mão para sinalizar ou responder a uma pergunta}
  \end{Phonetics}
\end{Entry}

\begin{Entry}{举动}{9,6}{⼂、⼒}
  \begin{Phonetics}{举动}{ju3dong4}[][HSK 5]
    \definition{s.}{ato; atividade; movimento; ação}
  \end{Phonetics}
\end{Entry}

\begin{Entry}{举行}{9,6}{⼂、⾏}
  \begin{Phonetics}{举行}{ju3xing2}[][HSK 2]
    \definition{v.}{realizar (uma reunião, cerimônia, etc.); realizar (atividades formais ou solenes)}
  \end{Phonetics}
\end{Entry}

\begin{Entry}{亭}{9}{⼇}
  \begin{Phonetics}{亭}{ting2}
    \definition{s.}{pavilhão | cabine | quiosque}
  \end{Phonetics}
\end{Entry}

\begin{Entry}{亮}{9}{⼇}
  \begin{Phonetics}{亮}{liang4}[][HSK 2]
    \definition*{s.}{Sobrenome Lian}
    \definition{adj.}{brilhante; claro | alto e claro; retumbante | esclarecido; aberto e claro}
    \definition{s.}{luz}
    \definition{v.}{iluminar; clarear; brilhar | elevar a voz; ressoar; tornar o som mais alto | revelar; mostrar; aparecer; exibir}
  \end{Phonetics}
\end{Entry}

\begin{Entry}{亲}{9}{⼇}
  \begin{Phonetics}{亲}{qin1}[][HSK 3]
    \definition{adj.}{parente próximo; relacionado por sangue; de ​​parentesco consanguíneo; parente consanguíneo mais próximo | querido; próximo; íntimo; relações próximas entre pessoas; sentimentos profundos (em oposição a 疏) | em si mesmo; pessoalmente}
    \definition[位]{s.}{pais; refere-se aos pais; também se refere apenas ao pai ou à mãe | parente; refere-se a pessoas que são relacionadas por sangue ou casamento| casal; casamento; refere-se ao casamento ou relacionamento conjugal | noiva; refere-se especificamente à noiva}
    \definition{v.}{beijar | (de países, partidos, etc.) a favor de; apoiar; estar perto de}
  \seealsoref{疏}{shu1}
  \end{Phonetics}
  \begin{Phonetics}{亲}{qing4}
    \definition{s.}{parentes por afinidade; parentes por casamento}
  \end{Phonetics}
\end{Entry}

\begin{Entry}{亲人}{9,2}{⼇、⼈}
  \begin{Phonetics}{亲人}{qin1 ren2}[][HSK 3]
    \definition[个,位]{s.}{um membro da família; os pais, o cônjuge, os filhos, etc.; refere-se a parentes ou cônjuges | queridos; entes queridos; aqueles queridos para alguém; uma metáfora para pessoas que têm um relacionamento próximo e sentimentos profundos}
  \end{Phonetics}
\end{Entry}

\begin{Entry}{亲切}{9,4}{⼇、⼑}
  \begin{Phonetics}{亲切}{qin1qie4}[][HSK 3]
    \definition{adj.}{gentil; cordial; cheio de sinceridade e cuidado, fazendo com que as pessoas se sintam acolhidas e acessíveis | próximo; íntimo; por familiaridade e afeição}
  \end{Phonetics}
\end{Entry}

\begin{Entry}{亲自}{9,6}{⼇、⾃}
  \begin{Phonetics}{亲自}{qin1zi4}[][HSK 3]
    \definition{adv.}{pessoalmente; em pessoa; si mesmo; fazer algo diretamente por si mesmo}
  \end{Phonetics}
\end{Entry}

\begin{Entry}{亲爱}{9,10}{⼇、⽖}
  \begin{Phonetics}{亲爱}{qin1'ai4}[][HSK 4]
    \definition{adj.}{querido; amado; termo carinhoso que expressa intimidade e afeto}
  \end{Phonetics}
\end{Entry}

\begin{Entry}{亲密}{9,11}{⼇、⼧}
  \begin{Phonetics}{亲密}{qin1mi4}[][HSK 4]
    \definition{adj.}{próximo; íntimo; relacionamento afetuoso e próximo}
  \end{Phonetics}
\end{Entry}

\begin{Entry}{亲眼}{9,11}{⼇、⽬}
  \begin{Phonetics}{亲眼}{qin1 yan3}[][HSK 6]
    \definition{adv.}{pessoalmente; com os próprios olhos}
  \end{Phonetics}
\end{Entry}

\begin{Entry}{亲属}{9,12}{⼇、⼫}
  \begin{Phonetics}{亲属}{qin1 shu3}[][HSK 6]
    \definition{s.}{parentes; cognatos}
  \end{Phonetics}
\end{Entry}

\begin{Entry}{侵}{9}{⼈}
  \begin{Phonetics}{侵}{qin1}
    \definition*{s.}{Sobrenome Qin}
    \definition{prep.}{aproximando-se; aproximar}
    \definition{v.}{invadir; intrometer-se em; infringir | aproximar-se (amanhecer)}
  \end{Phonetics}
\end{Entry}

\begin{Entry}{侵犯}{9,5}{⼈、⽝}
  \begin{Phonetics}{侵犯}{qin1fan4}[][HSK 6]
    \definition{v.}{violar; invadir; infringir; interferência ilegal com terceiros e violação de seus direitos | violar; fazer incursões; invadir o território de outro país}
  \end{Phonetics}
\end{Entry}

\begin{Entry}{侵略}{9,11}{⼈、⽥}
  \begin{Phonetics}{侵略}{qin1lve4}
    \definition{s.}{invasão}
    \definition{v.}{invadir}
  \end{Phonetics}
\end{Entry}

\begin{Entry}{便}{9}{⼈}
  \begin{Phonetics}{便}{bian4}[][HSK 6]
    \definition{adj.}{prático; conveniente | simples; comum; informal}
    \definition{adv.}{então; apenas no caso de; mesmo significado e uso de 就}
    \definition{conj.}{mesmo que; expressa uma concessão hipotética}
    \definition{s.}{facilidade; conveniência; o momento certo; a oportunidade | fezes ou urina}
    \definition{v.}{aliviar-se; excretar fezes e urina}
  \seealsoref{就}{jiu4}
  \end{Phonetics}
  \begin{Phonetics}{便}{pian2}
    \definition*{s.}{Sobrenome Pian}
    \definition{adj.}{silencioso e confortável}
  \end{Phonetics}
\end{Entry}

\begin{Entry}{便于}{9,3}{⼈、⼆}
  \begin{Phonetics}{便于}{bian4yu2}[][HSK 5]
    \definition{v.}{ser fácil para; ser conveniente para (algo ou fazer algo)}
  \end{Phonetics}
\end{Entry}

\begin{Entry}{便利}{9,7}{⼈、⼑}
  \begin{Phonetics}{便利}{bian4li4}[][HSK 5]
    \definition{adj.}{fácil; conveniente}
    \definition{s.}{facilidade; conveniência; coisas ou condições convenientes}
    \definition{v.}{facilitar; fornecer ajuda para que os outros se sintam confortáveis}
  \end{Phonetics}
\end{Entry}

\begin{Entry}{便条}{9,7}{⼈、⽊}
  \begin{Phonetics}{便条}{bian4tiao2}[][HSK 5]
    \definition[张,个]{s.}{nota ou mensagem informal; geralmente uma mensagem ou notificação}
  \end{Phonetics}
\end{Entry}

\begin{Entry}{便宜}{9,8}{⼈、⼧}
  \begin{Phonetics}{便宜}{bian4yi2}
    \definition{adj.}{prático; conveniente; adequado}
  \end{Phonetics}
  \begin{Phonetics}{便宜}{pian2yi5}[][HSK 2]
    \definition{adj.}{barato; acessível}
    \definition[个,份,件]{s.}{vantagem em algum aspecto | ganho; lucro; vantagem; benefício indevido}
    \definition{v.}{deixar alguém escapar impune; obter algum benefício}
  \end{Phonetics}
\end{Entry}

\begin{Entry}{便是}{9,9}{⼈、⽇}
  \begin{Phonetics}{便是}{bian4 shi4}[][HSK 6]
    \definition{adv.}{exatamente; precisamente; para expressar afirmação ou ênfase}
    \definition{conj.}{mesmo; mesmo que; usado para introduzir um caso extremo hipotético, enfatizando que o mesmo resultado ocorreria em circunstâncias tão extremas, sem mencionar circunstâncias normais; você também pode usar 即便是}
    \definition{part.}{usada no final de uma frase para expressar afirmação}
  \seealsoref{即便是}{ji2bian4 shi4}
  \end{Phonetics}
\end{Entry}

\begin{Entry}{促}{9}{⼈}
  \begin{Phonetics}{促}{cu4}
    \definition{adj.}{curto; apressado; urgente}
    \definition{v.}{urgir; promover | estar perto de; estar perto}
  \end{Phonetics}
\end{Entry}

\begin{Entry}{促进}{9,7}{⼈、⾡}
  \begin{Phonetics}{促进}{cu4jin4}[][HSK 4]
    \definition{v.}{impulsionar; promover; avançar; incentivar o desenvolvimento}
  \end{Phonetics}
\end{Entry}

\begin{Entry}{促使}{9,8}{⼈、⼈}
  \begin{Phonetics}{促使}{cu4shi3}[][HSK 4]
    \definition{v.}{incitar; estimular; impelir; causar; provocar uma mudança em alguém ou em algo}
  \end{Phonetics}
\end{Entry}

\begin{Entry}{促销}{9,12}{⼈、⾦}
  \begin{Phonetics}{促销}{cu4 xiao1}[][HSK 4]
    \definition{v.}{promover vendas}
  \end{Phonetics}
\end{Entry}

\begin{Entry}{俄}{9}{⼈}
  \begin{Phonetics}{俄}{e2}
    \definition*{s.}{Rússia, abreviação de 俄罗斯}
    \definition{adv.}{muito em breve; em breve; de repente}
  \seealsoref{俄罗斯}{e2luo2si1}
  \end{Phonetics}
\end{Entry}

\begin{Entry}{俄罗斯}{9,8,12}{⼈、⽹、⽄}
  \begin{Phonetics}{俄罗斯}{e2luo2si1}
    \definition*{s.}{Rússia}
  \end{Phonetics}
\end{Entry}

\begin{Entry}{俄罗斯人}{9,8,12,2}{⼈、⽹、⽄、⼈}
  \begin{Phonetics}{俄罗斯人}{e2luo2si1ren2}
    \definition{s.}{russo | pessoa ou povo da Rússia}
  \end{Phonetics}
\end{Entry}

\begin{Entry}{保}{9}{⼈}
  \begin{Phonetics}{保}{bao3}[][HSK 3]
    \definition*{s.}{Sobrenome Bao}
    \definition{s.}{fiador; babá ou responsável pela guarda de crianças | oficial responsável; sistema administrativo; unidade administrativa do antigo registro civil}
    \definition{v.}{defender; proteger | manter; preservar; conservar em boas condições | assegurar; garantir | ser fiador de alguém}
  \end{Phonetics}
\end{Entry}

\begin{Entry}{保卫}{9,3}{⼈、⼙}
  \begin{Phonetics}{保卫}{bao3wei4}[][HSK 5]
    \definition{v.}{defender; proteger; salvaguardar; proteger-se de ser violado}
  \end{Phonetics}
\end{Entry}

\begin{Entry}{保存}{9,6}{⼈、⼦}
  \begin{Phonetics}{保存}{bao3cun2}[][HSK 3]
    \definition{v.}{salvar; preservar; conservar; manter a existência com ênfase em que as coisas, as propriedades, os significados, os estilos, etc. não sofram perdas ou mudanças | (computação) salvar (um arquivo, etc.)}
  \end{Phonetics}
\end{Entry}

\begin{Entry}{保守}{9,6}{⼈、⼧}
  \begin{Phonetics}{保守}{bao3shou3}[][HSK 4]
    \definition{adj.}{retrógrado; conservador; pensamentos e conceitos que são retrógrados e não conseguem acompanhar o desenvolvimento da situação}
    \definition{v.}{manter; guardar; evitar perder}
  \end{Phonetics}
\end{Entry}

\begin{Entry}{保安}{9,6}{⼈、⼧}
  \begin{Phonetics}{保安}{bao3 an1}[][HSK 3]
    \definition[个,位,名]{s.}{guarda de segurança; segurança}
    \definition{v.}{proteger; manter em segurança; defender a segurança social | garantir a segurança; proteger a segurança dos trabalhadores e prevenir acidentes durante o processo de produção}
  \end{Phonetics}
\end{Entry}

\begin{Entry}{保护}{9,7}{⼈、⼿}
  \begin{Phonetics}{保护}{bao3hu4}[][HSK 3]
    \definition{v.}{proteger, guardar, cuidar; salvaguardar; cuidar ao máximo, para que não seja danificado, referindo-se principalmente a coisas concretas}
  \end{Phonetics}
\end{Entry}

\begin{Entry}{保护区}{9,7,4}{⼈、⼿、⼖}
  \begin{Phonetics}{保护区}{bao3hu4qu1}
    \definition[个,片]{s.}{zona de proteção | área de preservação; reserva natural}
  \end{Phonetics}
\end{Entry}

\begin{Entry}{保护主义}{9,7,5,3}{⼈、⼿、⼂、⼂}
  \begin{Phonetics}{保护主义}{bao3hu4zhu3yi4}
    \definition{s.}{protecionismo}
  \end{Phonetics}
\end{Entry}

\begin{Entry}{保护色}{9,7,6}{⼈、⼿、⾊}
  \begin{Phonetics}{保护色}{bao3hu4se4}
    \definition{s.}{camuflagem | coloração protetora}
  \end{Phonetics}
\end{Entry}

\begin{Entry}{保护剂}{9,7,8}{⼈、⼿、⼑}
  \begin{Phonetics}{保护剂}{bao3hu4ji4}
    \definition{s.}{agente protetor; protetor}
  \end{Phonetics}
\end{Entry}

\begin{Entry}{保护国}{9,7,8}{⼈、⼿、⼞}
  \begin{Phonetics}{保护国}{bao3hu4guo2}
    \definition{s.}{protetorado}
  \end{Phonetics}
\end{Entry}

\begin{Entry}{保护性}{9,7,8}{⼈、⼿、⼼}
  \begin{Phonetics}{保护性}{bao3hu4xing4}
    \definition{s.}{proteção; protetor}
  \end{Phonetics}
\end{Entry}

\begin{Entry}{保护物}{9,7,8}{⼈、⼿、⽜}
  \begin{Phonetics}{保护物}{bao3hu4 wu4}
    \definition{s.}{protetor}
  \end{Phonetics}
\end{Entry}

\begin{Entry}{保护者}{9,7,8}{⼈、⼿、⽼}
  \begin{Phonetics}{保护者}{bao3hu4zhe3}
    \definition{s.}{protetor | segurador}
  \end{Phonetics}
\end{Entry}

\begin{Entry}{保护神}{9,7,9}{⼈、⼿、⽰}
  \begin{Phonetics}{保护神}{bao3hu4shen2}
    \definition{s.}{anjo da guarda | santo patrono}
  \end{Phonetics}
\end{Entry}

\begin{Entry}{保证}{9,7}{⼈、⾔}
  \begin{Phonetics}{保证}{bao3zheng4}[][HSK 3]
    \definition[种,份]{s.}{compromisso; garantia; caução; aval; condições ou coisas que garantem a realização de algo}
    \definition{v.}{prometer; garantir; assegurar; certamente concluir algo; garantir que determinados padrões e requisitos sejam alcançados}
  \end{Phonetics}
\end{Entry}

\begin{Entry}{保养}{9,9}{⼈、⼋}
  \begin{Phonetics}{保养}{bao3yang3}[][HSK 5]
    \definition{v.}{preservar; cuidar bem (ou conservar) da saúde |  fazer manutenção; conservar; manter; manter em bom estado de conservação}
  \end{Phonetics}
\end{Entry}

\begin{Entry}{保持}{9,9}{⼈、⼿}
  \begin{Phonetics}{保持}{bao3chi2}[][HSK 3]
    \definition{v.}{manter; conservar; reter; preservar; manter um determinado estado, para que não desapareça ou não se altere}
  \end{Phonetics}
\end{Entry}

\begin{Entry}{保险}{9,9}{⼈、⾩}
  \begin{Phonetics}{保险}{bao3xian3}[][HSK 3]
    \definition{adj.}{seguro; pode ficar tranquilo}
    \definition[个,份,种]{s.}{seguro; um tipo de seguro comercial que garante que o segurado receba uma indenização em caso de prejuízo}
    \definition{v.}{ter certeza; estar obrigado a; garantir que algo aconteça (o que as pessoas desejam)}
  \end{Phonetics}
\end{Entry}

\begin{Entry}{保健}{9,10}{⼈、⼈}
  \begin{Phonetics}{保健}{bao3 jian4}[][HSK 6]
    \definition{s.}{cuidados de saúde; proteção da saúde}
    \definition{v.}{cuidar da sua saúde; proteger sua saúde}
  \end{Phonetics}
\end{Entry}

\begin{Entry}{保留}{9,10}{⼈、⽥}
  \begin{Phonetics}{保留}{bao3liu2}[][HSK 3]
    \definition{v.}{manter; continuar a ter; manter o estado original inalterado | conter; reter; deixar ficar; não tirar | reservar; colocar os direitos, opiniões, etc. de lado, não exercê-los ou expressá-los por enquanto}
  \end{Phonetics}
\end{Entry}

\begin{Entry}{保密}{9,11}{⼈、⼧}
  \begin{Phonetics}{保密}{bao3mi4}[][HSK 4]
    \definition{v.}{manter segredo; manter algo em segredo; manter a confidencialidade}
  \end{Phonetics}
\end{Entry}

\begin{Entry}{信}{9}{⼈}
  \begin{Phonetics}{信}{xin4}[][HSK 2,3]
    \definition*{s.}{Sobrenome Xin}
    \definition{adj.}{verdade}
    \definition[封,个,张]{s.}{carta; correio | mensagem; notícia; informação | sinal; evidência | confiança; fé; crédito | detonador (de bombas, etc.) | arsênico}
    \definition{v.}{acreditar; fazer um balanço; dar crédito | deixar à vontade; deixar à mercê; deixar ao acaso | professar fé em; acreditar em}
  \end{Phonetics}
\end{Entry}

\begin{Entry}{信心}{9,4}{⼈、⼼}
  \begin{Phonetics}{信心}{xin4xin1}[][HSK 2]
    \definition[个]{s.}{confiança; fé (em alguém ou algo) ; a crença de que os desejos se tornarão realidade}
  \end{Phonetics}
\end{Entry}

\begin{Entry}{信号}{9,5}{⼈、⼝}
  \begin{Phonetics}{信号}{xin4hao4}[][HSK 2]
    \definition[个,道]{s.}{sinal; luz, ondas de rádio, som, movimento, etc. usados para transmitir mensagens ou comandos | ponte de sinalização; marcação para chamar a atenção, ajudar na identificação e na memória}
  \end{Phonetics}
\end{Entry}

\begin{Entry}{信用}{9,5}{⼈、⽤}
  \begin{Phonetics}{信用}{xin4 yong4}[][HSK 6]
    \definition{adj.}{crédito; não é necessária nenhuma garantia material e o dinheiro pode ser reembolsado no prazo}
    \definition[些]{s.}{crédito; confiabilidade; a confiança que você ganha ao fazer o que prometeu | crédito; uma relação de empréstimo-pagamento ou situação em que o empréstimo é condicionado ao pagamento; uma situação em que um banco empresta dinheiro temporariamente a um cliente e este posteriormente devolve o dinheiro ao banco}
  \end{Phonetics}
\end{Entry}

\begin{Entry}{信用卡}{9,5,5}{⼈、⽤、⼘}
  \begin{Phonetics}{信用卡}{xin4yong4ka3}[][HSK 2]
    \definition[张]{s.}{cartão de crédito; moeda eletrônica emitida por um banco ou outra instituição especializada para consumidores; os titulares do cartão podem usá-lo para sacar dinheiro ou fazer compras de acordo com os regulamentos}
  \end{Phonetics}
\end{Entry}

\begin{Entry}{信仰}{9,6}{⼈、⼈}
  \begin{Phonetics}{信仰}{xin4yang3}[][HSK 6]
    \definition[种]{s.}{crença; religião; refere-se à ideia de acreditar, adorar e tomar algo como padrão e guia para palavras e ações}
    \definition{v.}{acreditar; crer em; acreditar e adorar uma determinada religião ou doutrina e tomá-la como guia para palavras e ações}
  \end{Phonetics}
\end{Entry}

\begin{Entry}{信任}{9,6}{⼈、⼈}
  \begin{Phonetics}{信任}{xin4ren4}[][HSK 3]
    \definition{s.}{confiança; um estado mental positivo e conexão emocional}
    \definition{v.}{confiar; ter confiança em; acreditar e ousar confiar}
  \end{Phonetics}
\end{Entry}

\begin{Entry}{信访}{9,6}{⼈、⾔}
  \begin{Phonetics}{信访}{xin4fang3}
    \definition{s.}{carta de reclamação | carta de petição}
  \seealsoref{上访}{shang4fang3}
  \end{Phonetics}
\end{Entry}

\begin{Entry}{信念}{9,8}{⼈、⼼}
  \begin{Phonetics}{信念}{xin4nian4}[][HSK 5]
    \definition[个,种]{s.}{fé; crença; convicção; concepções consideradas corretas e acreditadas com convicção}
  \end{Phonetics}
\end{Entry}

\begin{Entry}{信经}{9,8}{⼈、⽷}
  \begin{Phonetics}{信经}{xin4jing1}
    \definition[个]{s.}{crença | credo (seção da missa católica)}
  \end{Phonetics}
\end{Entry}

\begin{Entry}{信封}{9,9}{⼈、⼨}
  \begin{Phonetics}{信封}{xin4feng1}[][HSK 3]
    \definition[个,封]{s.}{envelope para cartas}
  \end{Phonetics}
\end{Entry}

\begin{Entry}{信息}{9,10}{⼈、⼼}
  \begin{Phonetics}{信息}{xin4xi1}[][HSK 2]
    \definition[个,条,段,些]{s.}{notícias; informações; as últimas notícias sobre alguém ou alguma coisa | mensagem; informação; na teoria da informação, uma mensagem transmitida usando símbolos, cujo conteúdo é desconhecido pelo receptor}
  \end{Phonetics}
\end{Entry}

\begin{Entry}{信箱}{9,15}{⼈、⾋}
  \begin{Phonetics}{信箱}{xin4 xiang1}[][HSK 5]
    \definition{s.}{caixa de correio; caixa postal instalada pelos correios para que as pessoas possam depositar cartas | caixa postal; caixas com números, localizadas nos correios, que podem ser alugadas para receber correspondência; chamadas de caixas postais exclusivas}
  \end{Phonetics}
\end{Entry}

\begin{Entry}{俩}{9}{⼈}
  \begin{Phonetics}{俩}{lia3}[][HSK 4]
    \definition{num.}{ambos; dois; contração de 两个 | alguns; vários; refere-se a um pequeno número}
  \end{Phonetics}
\end{Entry}

\begin{Entry}{俩钱}{9,10}{⼈、⾦}
  \begin{Phonetics}{俩钱}{lia3qian2}
    \definition{s.}{uma pequena quantia de dinheiro}
  \end{Phonetics}
\end{Entry}

\begin{Entry}{俭}{9}{⼈}
  \begin{Phonetics}{俭}{jian3}
    \definition*{s.}{Sobrenome Jian}
    \definition{adj.}{econômico; frugal | querendo; faltando; curto}
  \end{Phonetics}
\end{Entry}

\begin{Entry}{俭省}{9,9}{⼈、⽬}
  \begin{Phonetics}{俭省}{jian3sheng3}
    \definition{adj.}{econômico}
  \end{Phonetics}
\end{Entry}

\begin{Entry}{修}{9}{⼈}
  \begin{Phonetics}{修}{xiu1}[][HSK 3]
    \definition*{s.}{Sobrenome Xiu}
    \definition{adj.}{comprido; alto e esbelto}
    \definition{s.}{revisionismo}
    \definition{v.}{embelezar; decorar | consertar; reparar; reformar | escrever; redigir; compilar | estudar; cultivar; aprender e praticar para aperfeiçoar ou melhorar (o caráter e o conhecimento) | construir; edificar | cortar ou aparar, para deixar bonito e arrumado | dedicar-se à prática da religião}
  \end{Phonetics}
\end{Entry}

\begin{Entry}{修车}{9,4}{⼈、⾞}
  \begin{Phonetics}{修车}{xiu1 che1}[][HSK 6]
    \definition{v.}{consertar uma bicicleta (carro etc.)}[我打算明天去修车。===Pretendo consertar meu carro amanhã.]
  \end{Phonetics}
\end{Entry}

\begin{Entry}{修改}{9,7}{⼈、⽁}
  \begin{Phonetics}{修改}{xiu1gai3}[][HSK 3]
    \definition{v.}{revisar; retocar; corrigir erros e falhas em artigos, planos, etc.}
  \end{Phonetics}
\end{Entry}

\begin{Entry}{修建}{9,8}{⼈、⼵}
  \begin{Phonetics}{修建}{xiu1jian4}[][HSK 5]
    \definition{v.}{construir; erguer; animar; edificar; construir com tijolos, telhas, madeira, cimento, areia, etc.}
  \end{Phonetics}
\end{Entry}

\begin{Entry}{修规}{9,8}{⼈、⾒}
  \begin{Phonetics}{修规}{xiu1gui1}
    \definition{s.}{plano de construção}
  \end{Phonetics}
\end{Entry}

\begin{Entry}{修养}{9,9}{⼈、⼋}
  \begin{Phonetics}{修养}{xiu1yang3}[][HSK 5]
    \definition[种]{s.}{treinamento; domínio; realização; refere-se a um determinado nível em termos de teoria, conhecimento, arte, pensamento, etc. | auto-cultivo; refere-se à atitude e ao comportamento cultivados ao longo do tempo, em conformidade com as exigências sociais}
  \end{Phonetics}
\end{Entry}

\begin{Entry}{修复}{9,9}{⼈、⼢}
  \begin{Phonetics}{修复}{xiu1fu4}[][HSK 5]
    \definition{v.}{reparar; restaurar; renovar | reparar; melhorar e restaurar (o relacionamento)}
  \end{Phonetics}
\end{Entry}

\begin{Entry}{修理}{9,11}{⼈、⽟}
  \begin{Phonetics}{修理}{xiu1li3}[][HSK 4]
    \definition{v.}{consertar; reparar; restaurar algo danificado à sua forma ou função original | aparar; podar; cortar com tesouras e outras ferramentas para deixar árvores, flores, cabelos, etc. arrumados | culpar; punir; criticar ou punir uma pessoa para mostrar que ela está errada}
  \end{Phonetics}
\end{Entry}

\begin{Entry}{养}{9}{⼋}
  \begin{Phonetics}{养}{yang3}[][HSK 2]
    \definition*{s.}{Sobrenome Yang}
    \definition{adj.}{adotivo; órfão; adotado; não biológico}
    \definition{s.}{qualidade; (caráter moral, desempenho acadêmico, etc.) boas qualidades}
    \definition{v.}{apoiar; prover; fornecer dinheiro e materiais necessários para viver | aumentar; manter; crescer; alimentar os animais e cuidar de suas vidas para que possam crescer | dar à luz | formar; adquirir; cultivar | descansar; curar; convalescer; recuperar a saúde | manter; manter em bom estado | deixar (o cabelo) crescer | ajudar; apoiar | cultivar (plantações ou flores)}
  \end{Phonetics}
\end{Entry}

\begin{Entry}{养分}{9,4}{⼋、⼑}
  \begin{Phonetics}{养分}{yang3fen4}
    \definition{s.}{nutriente}
  \end{Phonetics}
\end{Entry}

\begin{Entry}{养成}{9,6}{⼋、⼽}
  \begin{Phonetics}{养成}{yang3cheng2}[][HSK 4]
    \definition{v.}{cultivar; desenvolver; cultivar para formar; nutrir para crescer}
  \end{Phonetics}
\end{Entry}

\begin{Entry}{养老}{9,6}{⼋、⽼}
  \begin{Phonetics}{养老}{yang3 lao3}[][HSK 6]
    \definition{v.}{prover assistência aos idosos (geralmente os pais) | viver a vida na aposentadoria; refere-se ao idoso que descansa em casa}
  \end{Phonetics}
\end{Entry}

\begin{Entry}{养料}{9,10}{⼋、⽃}
  \begin{Phonetics}{养料}{yang3liao4}
    \definition{s.}{nutrição}
  \end{Phonetics}
\end{Entry}

\begin{Entry}{冒}{9}{⽇}
  \begin{Phonetics}{冒}{mao4}[][HSK 5]
    \definition*{s.}{Sobrenome Mao}
    \definition{adv.}{com ousadia; precipitadamente | fingidamente; falsamente; fraudulentamente}
    \definition{v.}{emitir; liberar; enviar (para cima) | arriscar; ser corajoso}
  \end{Phonetics}
\end{Entry}

\begin{Entry}{冒险}{9,9}{⽇、⾩}
  \begin{Phonetics}{冒险}{mao4xian3}
    \definition{adj.}{corajoso}
    \definition{s.}{risco | aventura}
    \definition{v.+compl.}{correr risco | arriscar-se | aventurar-se em}
  \end{Phonetics}
\end{Entry}

\begin{Entry}{冠}{9}{⼍}
  \begin{Phonetics}{冠}{guan1}
    \definition{s.}{chapéu | corona; coroa; copa | crista}
  \end{Phonetics}
  \begin{Phonetics}{冠}{guan4}
    \definition*{s.}{Sobrenome Guan}
    \definition{s.}{primeiro lugar; o melhor; classificado em primeiro lugar}
    \definition{v.}{colocar um chapéu (boné) | preceder com (por); coroar com; adicionar um nome ou texto na frente}
  \end{Phonetics}
\end{Entry}

\begin{Entry}{冠军}{9,6}{⼍、⼍}
  \begin{Phonetics}{冠军}{guan4jun1}[][HSK 5]
    \definition[位,名,项,个]{s.}{campeão; medalhista de ouro; primeiro lugar em esportes e outras competições}
  \end{Phonetics}
\end{Entry}

\begin{Entry}{前}{9}{⼑}
  \begin{Phonetics}{前}{qian2}[][HSK 1]
    \definition*{s.}{Sobrenome Qian}
    \definition{s.}{frente | futuro; perspectiva | atrás; antes; mais cedo do que uma coisa ou um momento | à frente; para a frente; na parte frontal (referindo-se ao espaço, em oposição a 后) | precedente; antes que algo aconteça | antigo; antigamente | topo; primeiro; primeiro na ordem | frente; campo de batalha | A.C. (Antes de~Cristo)}[前293年===293 a.C.]
    \definition{v.}{seguir em frente; ir em frente}
  \seealsoref{公元}{gong1yuan2}
  \seealsoref{后}{hou4}
  \end{Phonetics}
\end{Entry}

\begin{Entry}{前天}{9,4}{⼑、⼤}
  \begin{Phonetics}{前天}{qian2 tian1}[][HSK 1]
    \definition{adv.}{anteontem; dia anterior a ontem}
  \end{Phonetics}
\end{Entry}

\begin{Entry}{前方}{9,4}{⼑、⽅}
  \begin{Phonetics}{前方}{qian2 fang1}[][HSK 6]
    \definition{s.}{frente; o espaço à frente; a direção voltada para a frente; a frente (em oposição à 后方) | linha de frente; frente de batalha; áreas onde os exércitos de ambos os lados estão se aproximando ou lutando}
  \seealsoref{后方}{hou4 fang1}
  \end{Phonetics}
\end{Entry}

\begin{Entry}{前头}{9,5}{⼑、⼤}
  \begin{Phonetics}{前头}{qian2 tou5}[][HSK 4]
    \definition{s.}{à frente; na frente; adiante}
  \end{Phonetics}
\end{Entry}

\begin{Entry}{前边}{9,5}{⼑、⾡}
  \begin{Phonetics}{前边}{qian2 bian5}[][HSK 1]
    \definition{adv.}{à frente; na frente}
  \end{Phonetics}
\end{Entry}

\begin{Entry}{前后}{9,6}{⼑、⼝}
  \begin{Phonetics}{前后}{qian2 hou4}[][HSK 3]
    \definition{s.}{em volta; sobre; um período de tempo ligeiramente anterior ou posterior a um horário específico| do início ao fim; refere-se ao período de tempo do início ao fim de algo | frente e verso; na frente e atrás de algo}
  \end{Phonetics}
\end{Entry}

\begin{Entry}{前年}{9,6}{⼑、⼲}
  \begin{Phonetics}{前年}{qian2 nian2}[][HSK 2]
    \definition{adv.}{há dois anos; dois anos atrás}
  \end{Phonetics}
\end{Entry}

\begin{Entry}{前来}{9,7}{⼑、⽊}
  \begin{Phonetics}{前来}{qian2 lai2}[][HSK 6]
    \definition{v.}{vir; em direção à localização e direção do falante}
  \end{Phonetics}
\end{Entry}

\begin{Entry}{前进}{9,7}{⼑、⾡}
  \begin{Phonetics}{前进}{qian2 jin4}[][HSK 3]
    \definition{v.}{marchar; avançar; para ir em frente; seguir em frente; geralmente se refere ao desenvolvimento futuro}
  \end{Phonetics}
\end{Entry}

\begin{Entry}{前往}{9,8}{⼑、⼻}
  \begin{Phonetics}{前往}{qian2 wang3}[][HSK 3]
    \definition{v.}{ir para; prosseguir para; partir para; ir em frente}
  \end{Phonetics}
\end{Entry}

\begin{Entry}{前线}{9,8}{⼑、⽷}
  \begin{Phonetics}{前线}{qian2 xian4}
    \definition{s.}{linha de frente; frente (oposto à 后方) | frente de batalha; a área onde os dois exércitos se aproximam durante uma batalha (em oposição à 后方)}
  \seealsoref{后方}{hou4 fang1}
  \end{Phonetics}
\end{Entry}

\begin{Entry}{前面}{9,9}{⼑、⾯}
  \begin{Phonetics}{前面}{qian2mian4}[][HSK 3]
    \definition{s.}{frente; a parte frontal do espaço ou posição | parte anterior; acima; a parte que vem primeiro na ordem; a parte de um artigo ou discurso que precede a narração atual}
  \end{Phonetics}
\end{Entry}

\begin{Entry}{前途}{9,10}{⼑、⾡}
  \begin{Phonetics}{前途}{qian2tu2}[][HSK 4]
    \definition[片,段,种]{s.}{futuro; perspectiva; prospecto; originalmente, refere-se à jornada à frente, mas, metaforicamente, refere-se ao futuro.}
  \end{Phonetics}
\end{Entry}

\begin{Entry}{前提}{9,12}{⼑、⼿}
  \begin{Phonetics}{前提}{qian2ti2}[][HSK 5]
    \definition[个,项]{s.}{premissa; pressuposto | pré-requisito; pressuposição; condições prévias para que algo aconteça ou se desenvolva}
  \end{Phonetics}
\end{Entry}

\begin{Entry}{前景}{9,12}{⼑、⽇}
  \begin{Phonetics}{前景}{qian2jing3}[][HSK 5]
    \definition{s.}{primeiro plano (de uma vista, imagem, foto, etc.); as imagens que parecem mais próximas do espectador em pinturas, palcos e telas | vista; perspectiva; prospecto; ponto de vista; situações que podem ocorrer no trabalho, na carreira, etc.}
  \end{Phonetics}
\end{Entry}

\begin{Entry}{剑}{9}{⼑}
  \begin{Phonetics}{剑}{jian4}[][HSK 6]
    \definition[把,口]{s.}{espada; sabre; florete}
  \end{Phonetics}
\end{Entry}

\begin{Entry}{剑客}{9,9}{⼑、⼧}
  \begin{Phonetics}{剑客}{jian4ke4}
    \definition{s.}{espada | esgrimista, espadachim}
  \end{Phonetics}
\end{Entry}

\begin{Entry}{勇}{9}{⼒}
  \begin{Phonetics}{勇}{yong3}
    \definition*{s.}{Sobrenome Yong}
    \definition{adj.}{bravo; valente; corajoso}
    \definition{s.}{recrutas temporários em tempos de guerra na Dinastia Qing}
  \end{Phonetics}
\end{Entry}

\begin{Entry}{勇士}{9,3}{⼒、⼠}
  \begin{Phonetics}{勇士}{yong3shi4}
    \definition{s.}{um guerreiro | uma pessoa corajosa}
  \end{Phonetics}
\end{Entry}

\begin{Entry}{勇气}{9,4}{⼒、⽓}
  \begin{Phonetics}{勇气}{yong3qi4}[][HSK 4]
    \definition[种,股]{s.}{coragem; arrojo; nervos; coragem para agir sem medo}
  \end{Phonetics}
\end{Entry}

\begin{Entry}{勇敢}{9,11}{⼒、⽁}
  \begin{Phonetics}{勇敢}{yong3gan3}[][HSK 4]
    \definition{adj.}{bravo; valente; galante; corajoso}
  \end{Phonetics}
\end{Entry}

\begin{Entry}{南}{9}{⼗}
  \begin{Phonetics}{南}{nan2}[][HSK 1]
    \definition*{s.}{Sobrenome Nan}
    \definition{s.}{sul; uma das quatro direções básicas, o lado direito quando se está de frente para o sol pela manhã (oposto ao 北) | especificamente no sul da China}
  \seealsoref{北}{bei3}
  \end{Phonetics}
\end{Entry}

\begin{Entry}{南方}{9,4}{⼗、⽅}
  \begin{Phonetics}{南方}{nan2 fang1}[][HSK 2]
    \definition{s.}{sul; indica a direção sul | o sul; a região sul}
  \end{Phonetics}
\end{Entry}

\begin{Entry}{南北}{9,5}{⼗、⼔}
  \begin{Phonetics}{南北}{nan2 bei3}[][HSK 5]
    \definition{s.}{(território) norte e sul | (distância) de norte a sul}
  \end{Phonetics}
\end{Entry}

\begin{Entry}{南边}{9,5}{⼗、⾡}
  \begin{Phonetics}{南边}{nan2 bian5}[][HSK 1]
    \definition{s.}{sul; lado sul}
  \end{Phonetics}
\end{Entry}

\begin{Entry}{南极}{9,7}{⼗、⽊}
  \begin{Phonetics}{南极}{nan2ji2}[][HSK 5]
    \definition*{s.}{Polo Sul; Polo Antártico | Polo sul magnético}
    \definition{s.}{polo sul magnético}
  \end{Phonetics}
\end{Entry}

\begin{Entry}{南京}{9,8}{⼗、⼇}
  \begin{Phonetics}{南京}{nan2jing1}
    \definition*{s.}{Nanquim, capital da província de Jiangsu, 江苏}
  \seealsoref{江苏}{jiang1su1}
  \end{Phonetics}
\end{Entry}

\begin{Entry}{南面}{9,9}{⼗、⾯}
  \begin{Phonetics}{南面}{nan2mian4}
    \definition{s.}{sul | lado sul}
  \end{Phonetics}
\end{Entry}

\begin{Entry}{南部}{9,10}{⼗、⾢}
  \begin{Phonetics}{南部}{nan2 bu4}[][HSK 3]
    \definition{s.}{parte sul; sul | a parte sul}
  \end{Phonetics}
\end{Entry}

\begin{Entry}{厘}{9}{⼚}
  \begin{Phonetics}{厘}{li2}
    \definition*{s.}{Sobrenome Li}
    \definition{clas.}{li, uma unidade tradicional de comprimento, igual a 0,001 chi (市尺), e equivalente a 0,333 milímetro ou 0,013 polegada | li, uma unidade tradicional de peso, igual a 0,0001 jin (市斤), e equivalente a 5 centigramas ou 0,771 grãos | li, uma unidade tradicional de área, igual a 0,01 mu (市亩), e equivalente a 0,667 metro quadrado ou 0,797 jarda quadrada | li, unidade monetária chinesa, igual a 0,1 fen ou 0,001 yuan | li, unidade de taxa de juros, igual a 0,1\% de juros mensais ou 1\% de juros anuais | quantidade muito pequena; fração; o mínimo}
    \definition{v.}{regular; retificar | administrar}
  \seealsoref{市尺}{shi4 chi3}
  \seealsoref{市斤}{shi4jin1}
  \seealsoref{市亩}{shi4mu3}
  \end{Phonetics}
\end{Entry}

\begin{Entry}{厘米}{9,6}{⼚、⽶}
  \begin{Phonetics}{厘米}{li2mi3}[][HSK 4]
    \definition{clas.}{centímetro; unidade de comprimento, símbolo cm, 1 metro é igual a 100 centímetros}
  \end{Phonetics}
\end{Entry}

\begin{Entry}{厚}{9}{⼚}
  \begin{Phonetics}{厚}{hou4}[][HSK 4]
    \definition*{s.}{Sobrenome Hou}
    \definition{adj.}{espesso; grosso (oposto a 薄) | profundo | gentil; magnânimo | grande; generoso | rico ou forte em sabor}
    \definition[米,厘米]{s.}{espessura | profundidade}
    \definition{v.}{favorecer; enfatizar}
  \seealsoref{薄}{bao2}
  \end{Phonetics}
\end{Entry}

\begin{Entry}{咨}{9}{⼝}
  \begin{Phonetics}{咨}{zi1}
    \definition[行]{s.}{comunicação oficial; relatório entregue pelo chefe de um governo sobre assuntos de Estado}
    \definition{v.}{consultar; discutir com}
  \end{Phonetics}
\end{Entry}

\begin{Entry}{咨询}{9,8}{⼝、⾔}
  \begin{Phonetics}{咨询}{zi1xun2}[][HSK 6]
    \definition{v.}{consultar; aconselhar-se com; buscar conselho de; pedir conselhos}
  \end{Phonetics}
\end{Entry}

\begin{Entry}{咬}{9}{⼝}
  \begin{Phonetics}{咬}{yao3}[][HSK 5]
    \definition{v.}{morder; estalar; pressionar os dentes superiores e inferiores com força | latir | agarrar; morder | incriminar outra pessoa (geralmente inocente) quando culpada ou interrogada | pronunciar; articular; pronunciar corretamente | corroer (metais); irritar (a pele) | ser minucioso (com relação ao uso de palavras) | aproximar-se de; pressionar em direção a; avançar sobre}
  \end{Phonetics}
\end{Entry}

\begin{Entry}{咱}{9}{⼝}
  \begin{Phonetics}{咱}{za2}
  \end{Phonetics}
  \begin{Phonetics}{咱}{zan2}[][HSK 2]
    \definition{pron.}{nós; nos (incluindo tanto o falante quanto a pessoa ou pessoas às quais se dirige) | eu; mim |}
  \end{Phonetics}
  \begin{Phonetics}{咱}{zan5}
    \definition{adv.}{quando; agora; então; naquele momento; usado em 这咱, 那咱, 多咱, uma combinação das duas palavras 早晚}
  \seealsoref{多咱}{duo1 zan5}
  \seealsoref{那咱}{na4 zan5}
  \seealsoref{早晚}{zao3 wan3}
  \seealsoref{这咱}{zhe4 zan5}
  \end{Phonetics}
\end{Entry}

\begin{Entry}{咱们}{9,5}{⼝、⼈}
  \begin{Phonetics}{咱们}{zan2men5}[][HSK 2]
    \definition{pron.}{dirige-se tanto ao falante (eu, nós) quanto ao ouvinte (você, vocês) | eu; mim; refere-se ao próprio orador, eu}
  \end{Phonetics}
\end{Entry}

\begin{Entry}{咱俩}{9,9}{⼝、⼈}
  \begin{Phonetics}{咱俩}{zan2lia3}
    \definition{pron.}{nós dois}
  \end{Phonetics}
\end{Entry}

\begin{Entry}{咱家}{9,10}{⼝、⼧}
  \begin{Phonetics}{咱家}{za2jia1}
    \definition{pron.}{eu (frequentemente usado na literatura vernácula antiga) | me | mim | comigo}
  \end{Phonetics}
\end{Entry}

\begin{Entry}{咳}{9}{⼝}
  \begin{Phonetics}{咳}{hai1}
    \definition{interj.}{expressa tristeza, arrependimento ou espanto}
  \end{Phonetics}
  \begin{Phonetics}{咳}{ke2}[][HSK 5]
    \definition{v.}{tossir}
  \end{Phonetics}
\end{Entry}

\begin{Entry}{咳嗽}{9,14}{⼝、⼝}
  \begin{Phonetics}{咳嗽}{ke2sou5}
    \definition{v.}{ter tosse | tossir}
  \end{Phonetics}
\end{Entry}

\begin{Entry}{咸}{9}{⼝}
  \begin{Phonetics}{咸}{xian2}[][HSK 4]
    \definition*{s.}{Sobrenome Xian}
    \definition{adj.}{salgado; em conserva; sabor salgado}
    \definition{adv.}{todos; indica a totalidade de um intervalo, equivalente a 全 e 都}
  \seealsoref{都}{dou1}
  \seealsoref{全}{quan2}
  \end{Phonetics}
\end{Entry}

\begin{Entry}{咸水}{9,4}{⼝、⽔}
  \begin{Phonetics}{咸水}{xian2shui3}
    \definition{s.}{salmora | água salgada}
  \end{Phonetics}
\end{Entry}

\begin{Entry}{咸肉}{9,6}{⼝、⾁}
  \begin{Phonetics}{咸肉}{xian2rou4}
    \definition{s.}{\emph{bacon} | carne curada com sal}
  \end{Phonetics}
\end{Entry}

\begin{Entry}{咸鱼}{9,8}{⼝、⿂}
  \begin{Phonetics}{咸鱼}{xian2yu2}
    \definition{s.}{peixe salgado}
  \end{Phonetics}
\end{Entry}

\begin{Entry}{咸涩}{9,10}{⼝、⽔}
  \begin{Phonetics}{咸涩}{xian2se4}
    \definition{s.}{ácido | salgado e amargo}
  \end{Phonetics}
\end{Entry}

\begin{Entry}{咸盐}{9,10}{⼝、⽫}
  \begin{Phonetics}{咸盐}{xian2yan2}
    \definition{s.}{(coloquial) sal | sal de mesa}
  \end{Phonetics}
\end{Entry}

\begin{Entry}{咸淡}{9,11}{⼝、⽔}
  \begin{Phonetics}{咸淡}{xian2dan4}
    \definition{s.}{água salobra | grau de salinidade | salgado e sem sal (sabores)}
  \end{Phonetics}
\end{Entry}

\begin{Entry}{咸菜}{9,11}{⼝、⾋}
  \begin{Phonetics}{咸菜}{xian2cai4}
    \definition{s.}{legumes salgados | \emph{pickles}}
  \end{Phonetics}
\end{Entry}

\begin{Entry}{哀}{9}{⼝}
  \begin{Phonetics}{哀}{ai1}
    \definition*{s.}{Sobrenome Ai}
    \definition{adj.}{triste; pesaroso}
    \definition{adv.}{tristemente; lamentavelmente}
    \definition{s.}{luto | tristeza; pesar | pena; misericórdia}
    \definition{v.}{lamentar; lamentar-se por | Literário: estar triste}
  \end{Phonetics}
\end{Entry}

\begin{Entry}{哀求}{9,7}{⼝、⽔}
  \begin{Phonetics}{哀求}{ai1qiu2}[][HSK 7,8,9]
    \definition{v.}{suplicar; implorar | suplicar; implorar piedosamente}
  \end{Phonetics}
\end{Entry}

\begin{Entry}{品}{9}{⼝}
  \begin{Phonetics}{品}{pin3}[][HSK 5]
    \definition*{s.}{Sobrenome Pin}
    \definition{s.}{artigo; produto | grau; classe; classificação; nível | caráter; qualidade | classificação; os graus dos funcionários públicos antigos, num total de nove graus}
    \definition{v.}{provar; saborear; degustar algo com discernimento | soprar; tocar (instrumentos de sopro) | avaliar; distinguir o bom do ruim}
  \end{Phonetics}
\end{Entry}

\begin{Entry}{品质}{9,8}{⼝、⾙}
  \begin{Phonetics}{品质}{pin3zhi4}[][HSK 4]
    \definition[个,种]{s.}{qualidade; caráter; natureza do pensamento, da compreensão, do caráter, etc., conforme expresso no comportamento, no estilo, etc. | qualidade (de produtos, mercadorias, etc.)}
  \end{Phonetics}
\end{Entry}

\begin{Entry}{品种}{9,9}{⼝、⽲}
  \begin{Phonetics}{品种}{pin3zhong3}[][HSK 5]
    \definition[个,些]{s.}{raça; linhagem; variedade; refere-se a um grupo de organismos com características genéticas comuns, formados por meio da seleção e cultivo artificiais de culturas, gado, aves, etc. | variedade; sortimento; referência geral ao tipo de item}
  \end{Phonetics}
\end{Entry}

\begin{Entry}{品牌}{9,12}{⼝、⽚}
  \begin{Phonetics}{品牌}{pin3 pai2}[][HSK 6]
    \definition[个,种]{s.}{marca registrada; nome de marca}
  \end{Phonetics}
\end{Entry}

\begin{Entry}{品德}{9,15}{⼝、⼻}
  \begin{Phonetics}{品德}{pin3de2}
    \definition{s.}{caráter moral | moralidade}
  \end{Phonetics}
\end{Entry}

\begin{Entry}{哄}{9}{⼝}
  \begin{Phonetics}{哄}{hong1}
    \definition{interj.}{(onomatopéia) gargalhadas ou alvoroço}
    \definition{s.}{rugido; clamor; comoção}
  \end{Phonetics}
  \begin{Phonetics}{哄}{hong3}
    \definition{v.}{brincar; enganar; tapear | persuadir; agradar os outros com palavras ou ações, especialmente observando ou cuidando de crianças}
  \end{Phonetics}
  \begin{Phonetics}{哄}{hong4}
    \definition{s.}{comoção; tumulto}
  \end{Phonetics}
\end{Entry}

\begin{Entry}{哇}{9}{⼝}
  \begin{Phonetics}{哇}{wa1}
    \definition{interj.}{(onomatopéia) som de choro ou vômito | Uau!; expressa surpresa}
  \end{Phonetics}
  \begin{Phonetics}{哇}{wa5}[][HSK 6]
    \definition{part.}{a mudança do som de 啊 devido à influência do som final da palavra anterior, ``u'' ou ``ao''}
  \seealsoref{啊}{a5}
  \end{Phonetics}
\end{Entry}

\begin{Entry}{哇塞}{9,13}{⼝、⼟}
  \begin{Phonetics}{哇塞}{wa1sai1}
    \definition{interj.}{Uau!; uma exclamação de espanto, admiração, etc.}
  \end{Phonetics}
\end{Entry}

\begin{Entry}{哇噻}{9,16}{⼝、⼝}
  \begin{Phonetics}{哇噻}{wa1sai1}
    \variantof{哇塞}
  \end{Phonetics}
\end{Entry}

\begin{Entry}{哈}{9}{⼝}
  \begin{Phonetics}{哈}{ha1}
    \definition{interj.}{Onomatopéia: ha; descreve o riso, usado principalmente em duplicata | indica orgulho ou satisfação, frequentemente usado de forma duplicada}
    \definition{v.}{soprar; expirar (com a boca aberta) | dobrar}
  \seealsoref{哈哈}{ha1 ha1}
  \end{Phonetics}
  \begin{Phonetics}{哈}{ha3}
    \definition*{s.}{Sobrenome Ha}
    \definition{v.}{repreender}
  \end{Phonetics}
\end{Entry}

\begin{Entry}{哈马斯}{9,3,12}{⼝、⾺、⽄}
  \begin{Phonetics}{哈马斯}{ha1ma3si1}
    \definition*{s.}{Hamas (Grupo Palestino)}
  \end{Phonetics}
\end{Entry}

\begin{Entry}{哈哈}{9,9}{⼝、⼝}
  \begin{Phonetics}{哈哈}{ha1 ha1}[][HSK 3]
    \definition{expr.}{(onomatopéia)  ha ha; o som de uma gargalhada}
  \end{Phonetics}
\end{Entry}

\begin{Entry}{响}{9}{⼝}
  \begin{Phonetics}{响}{xiang3}[][HSK 2]
    \definition{adj.}{barulhento; ressonante}
    \definition[声,阵]{s.}{som; ruído; barulho | eco}
    \definition{v.}{tocar; soar; ressoar; fazer um som | soar; fazer algo emitir um som}
  \end{Phonetics}
\end{Entry}

\begin{Entry}{响声}{9,7}{⼝、⼠}
  \begin{Phonetics}{响声}{xiang3 sheng1}[][HSK 6]
    \definition{s.}{som; ruído}
  \end{Phonetics}
\end{Entry}

\begin{Entry}{哗}{9}{⼝}
  \begin{Phonetics}{哗}{hua1}
    \definition{s.}{(onomatopéia) sons de impacto, batida, fluxo de água, etc.}
  \end{Phonetics}
  \begin{Phonetics}{哗}{hua2}
    \definition{v.}{ser barulhento; fazer alvoroço}
  \end{Phonetics}
\end{Entry}

\begin{Entry}{哗啦啦}{9,11,11}{⼝、⼝、⼝}
  \begin{Phonetics}{哗啦啦}{hua1la1 la5}
    \definition{s.}{(onomatopéia) som de colisão, batida}
  \end{Phonetics}
\end{Entry}

\begin{Entry}{哪}{9}{⼝}
  \begin{Phonetics}{哪}{na3}[][HSK 1,4]
    \definition{adv.}{para expressar uma pergunta retórica, indicando que é impossível}
    \definition{pron.}{qual?; o que?; expressa a necessidade de determinar um entre várias pessoas ou coisas | qualquer; ser usado em um sentido geral | qual?; o que?; (usado sozinho, o mesmo que 什么, frequentemente usado de forma intercambiável com 什么) | qualquer; qualquer que seja; refere-se a qualquer um, geralmente seguido por 都 ou 也, ou usando dois 哪 antes e depois | qual (indica algo incerto)}
  \seealsoref{都}{dou1}
  \seealsoref{什么}{shen2me5}
  \seealsoref{也}{ye3}
  \end{Phonetics}
  \begin{Phonetics}{哪}{na5}
    \definition{part.}{usado depois de uma palavra com a terminação -n, é equivalente a 啊}
  \seealsoref{啊}{a5}
  \end{Phonetics}
  \begin{Phonetics}{哪}{nei3}
    \definition{part.}{qual? (interrogativo, seguido de classificador ou numeral-classificador)}
  \end{Phonetics}
\end{Entry}

\begin{Entry}{哪儿}{9,2}{⼝、⼉}
  \begin{Phonetics}{哪儿}{na3r5}[][HSK 1]
    \definition{adv.}{usado para perguntas retóricas, indicando negação}
    \definition{pron.}{onde? | onde quer que seja; em qualquer lugar | usado como uma resposta educada a um elogio}
  \end{Phonetics}
\end{Entry}

\begin{Entry}{哪个}{9,3}{⼝、⼈}
  \begin{Phonetics}{哪个}{na3ge5}
    \definition{pron.}{qual deles (pergunta sobre o objeto) | quem (perguntar a alguém ou indicar qualquer pessoa)}
  \end{Phonetics}
\end{Entry}

\begin{Entry}{哪里}{9,7}{⼝、⾥}
  \begin{Phonetics}{哪里}{na3 li3}[][HSK 1]
    \definition{adv.}{usado em perguntas retóricas para expressar um significado negativo}
    \definition{pron.}{onde?; em que lugar? | onde quer que seja; em qualquer lugar | usado como uma resposta educada a um elogio}
  \end{Phonetics}
\end{Entry}

\begin{Entry}{哪些}{9,8}{⼝、⼆}
  \begin{Phonetics}{哪些}{na3xie1}[][HSK 1]
    \definition{pron.}{quais?}
  \end{Phonetics}
\end{Entry}

\begin{Entry}{哪国人}{9,8,2}{⼝、⼞、⼈}
  \begin{Phonetics}{哪国人}{na3 guo2ren2}
    \definition{expr.}{de qual país?}
  \end{Phonetics}
\end{Entry}

\begin{Entry}{哪怕}{9,8}{⼝、⼼}
  \begin{Phonetics}{哪怕}{na3pa4}[][HSK 4]
    \definition{conj.}{mesmo; mesmo se; mesmo que; não importa o quão}
  \end{Phonetics}
\end{Entry}

\begin{Entry}{型}{9}{⼟}
  \begin{Phonetics}{型}{xing2}[][HSK 4]
    \definition{s.}{molde; modelo | modelo; tipo; padrão}
  \end{Phonetics}
\end{Entry}

\begin{Entry}{型号}{9,5}{⼟、⼝}
  \begin{Phonetics}{型号}{xing2 hao4}[][HSK 4]
    \definition[个,种]{s.}{modelo; tipo; refere-se ao desempenho, às especificações e ao tamanho de aeronaves, máquinas, implementos agrícolas, etc.}
  \end{Phonetics}
\end{Entry}

\begin{Entry}{垫}{9}{⼟}
  \begin{Phonetics}{垫}{dian4}
    \definition[个]{s.}{almofada}
    \definition{v.}{colocar algo sob; elevar ou nivelar; encher; preencher | pagar por alguém e esperar ser reembolsado mais tarde | colocar algo sob algo para elevá-lo ou nivelá-lo; usar algo para apoiar, espalhar ou forrar algo para torná-lo mais alto, mais grosso ou mais plano | preencher uma vaga; preencher uma lacuna}
  \end{Phonetics}
\end{Entry}

\begin{Entry}{垫子}{9,3}{⼟、⼦}
  \begin{Phonetics}{垫子}{dian4zi5}
    \definition{s.}{colchão | esteira | almofada}
  \end{Phonetics}
\end{Entry}

\begin{Entry}{城}{9}{⼟}
  \begin{Phonetics}{城}{cheng2}[][HSK 3]
    \definition*{s.}{Sobrenome Cheng}
    \definition[座,道,个]{s.}{muralha da cidade; muralha | cidade | centro de um determinado tipo (por exemplo, negócios, entretenimento, etc.)}
  \end{Phonetics}
\end{Entry}

\begin{Entry}{城乡}{9,3}{⼟、⼄}
  \begin{Phonetics}{城乡}{cheng2 xiang1}[][HSK 6]
    \definition{s.}{cidade e campo; áreas urbanas e rurais; a cidade e o campo | cidade e campo; urbano e rural}
  \end{Phonetics}
\end{Entry}

\begin{Entry}{城区}{9,4}{⼟、⼖}
  \begin{Phonetics}{城区}{cheng2 qu1}[][HSK 6]
    \definition{s.}{cidade propriamente dita (oposto a 郊区) | área metropolitana; área urbana; áreas urbanas e suburbanas (diferentes de 郊区)}
  \seealsoref{郊区}{jiao1 qu1}
  \end{Phonetics}
\end{Entry}

\begin{Entry}{城市}{9,5}{⼟、⼱}
  \begin{Phonetics}{城市}{cheng2shi4}[][HSK 3]
    \definition[个,座]{s.}{cidade; regiões com alta densidade populacional, comércio e indústria desenvolvidos e cuja população é predominantemente não agrícola são geralmente centros políticos, econômicos e culturais das regiões vizinhas}
  \end{Phonetics}
\end{Entry}

\begin{Entry}{城里}{9,7}{⼟、⾥}
  \begin{Phonetics}{城里}{cheng2 li3}[][HSK 5]
    \definition{s.}{na cidade; dentro da cidade; originalmente referia-se à área dentro das muralhas da cidade, agora refere-se principalmente à área urbana}
  \end{Phonetics}
\end{Entry}

\begin{Entry}{城度}{9,9}{⼟、⼴}
  \begin{Phonetics}{城度}{cheng2du4}[][HSK 3]
    \definition{s.}{na cidade; dentro da cidade; originalmente se referia à área dentro das muralhas da cidade, agora se refere principalmente à área urbana}
  \end{Phonetics}
\end{Entry}

\begin{Entry}{城堡}{9,12}{⼟、⼟}
  \begin{Phonetics}{城堡}{cheng2bao3}
    \definition[座,个]{s.}{forte; castelo; cidadela; uma pequena cidade com muralhas que facilitam a defesa}
  \end{Phonetics}
\end{Entry}

\begin{Entry}{城镇}{9,15}{⼟、⾦}
  \begin{Phonetics}{城镇}{cheng2 zhen4}[][HSK 6]
    \definition[个]{s.}{cidade; cidades e vilas}
  \end{Phonetics}
\end{Entry}

\begin{Entry}{复}{9}{⼢}
  \begin{Phonetics}{复}{fu4}
    \definition*{s.}{Sobrenome Fu}
    \definition{adj.}{composto; complexo; nem um único; dois ou mais}
    \definition{adv.}{de novo; novamente; indica o reaparecimento de uma situação, equivalente a 再}
    \definition{s.}{jaqueta; roupas forradas}
    \definition{v.}{virar; virar-se | responder; retornar | recuperar; retornar a; restaurar | vingar | duplicar; repetir}
  \seealsoref{再}{zai4}
  \end{Phonetics}
\end{Entry}

\begin{Entry}{复习}{9,3}{⼢、⼄}
  \begin{Phonetics}{复习}{fu4xi2}[][HSK 2]
    \definition{s.}{revisão}
    \definition{v.}{revisar; corrigir (lições, etc.); repetir o que já aprendeu para consolidar o conhecimento}
  \end{Phonetics}
\end{Entry}

\begin{Entry}{复印}{9,5}{⼢、⼙}
  \begin{Phonetics}{复印}{fu4yin4}[][HSK 3]
    \definition{v.}{fotografar; fotocopiar; duplicar; sem passar pelo processo de impressão, obter uma cópia diretamente do original (geralmente referindo-se à cópia feita com uma copiadora)}
  \end{Phonetics}
\end{Entry}

\begin{Entry}{复杂}{9,6}{⼢、⽊}
  \begin{Phonetics}{复杂}{fu4za2}[][HSK 3]
    \definition{adj.}{complexo; complicado; em oposição a 单纯 e 简单}
  \seealsoref{单纯}{dan1chun2}
  \seealsoref{简单}{jian3dan1}
  \end{Phonetics}
\end{Entry}

\begin{Entry}{复苏}{9,7}{⼢、⾋}
  \begin{Phonetics}{复苏}{fu4 su1}[][HSK 6]
    \definition{s.}{recuperação}
    \definition{v.}{reviver; recuperar; ressuscitar; voltar à vida}
  \end{Phonetics}
\end{Entry}

\begin{Entry}{复制}{9,8}{⼢、⼑}
  \begin{Phonetics}{复制}{fu4zhi4}[][HSK 4]
    \definition{v.}{copiar; duplicar; reproduzir; fazer uma cópia de; fazer uma cópia do original ou reproduzi-lo, reimprimi-lo ou copiá-lo em sua forma original (geralmente referindo-se a relíquias culturais ou obras de arte)}
  \end{Phonetics}
\end{Entry}

\begin{Entry}{复刻}{9,8}{⼢、⼑}
  \begin{Phonetics}{复刻}{fu4ke4}
    \definition{v.}{reimprimir (um trabalho que esteve fora do catálogo) | reeditar (um disco de vinil, um CD, etc.) | replicar | recriar | (empréstimo linguístico) (computação) \emph{fork}}
  \end{Phonetics}
\end{Entry}

\begin{Entry}{复活节}{9,9,5}{⼢、⽔、⾋}
  \begin{Phonetics}{复活节}{fu4huo2jie2}
    \definition*{s.}{Páscoa}
  \end{Phonetics}
\end{Entry}

\begin{Entry}{奏}{9}{⼤}
  \begin{Phonetics}{奏}{zou4}[][HSK 6]
    \definition{v.}{tocar (música); executar (em um instrumento musical)  | alcançar; produzir; alcançar ou estabelecer (desempenho ou realização) | (antigo) apresentar um memorial a um imperador; fazer uma petição}
  \end{Phonetics}
\end{Entry}

\begin{Entry}{奏效}{9,10}{⼤、⽁}
  \begin{Phonetics}{奏效}{zou4xiao4}
    \definition{v.}{mostrar resultados | ser eficaz}
  \end{Phonetics}
\end{Entry}

\begin{Entry}{奖}{9}{⼤}
  \begin{Phonetics}{奖}{jiang3}[][HSK 4]
    \definition[个,次]{s.}{prêmio; recompensa | elogio; loa}
    \definition{v.}{elogiar; recompensar; recomendar; incentivar}
  \end{Phonetics}
\end{Entry}

\begin{Entry}{奖励}{9,7}{⼤、⼒}
  \begin{Phonetics}{奖励}{jiang3li4}[][HSK 5]
    \definition{s.}{prêmio; recompensa; dinheiro ou honras dadas em troca de elogios ou incentivos}
    \definition{v.}{recompensar; incentivar; encorajar}
  \end{Phonetics}
\end{Entry}

\begin{Entry}{奖学金}{9,8,8}{⼤、⼦、⾦}
  \begin{Phonetics}{奖学金}{jiang3 xue2 jin1}[][HSK 4]
    \definition[笔]{s.}{bolsa de estudos; exposição; prêmios concedidos por escolas, organizações ou indivíduos a alunos com bom desempenho acadêmico}
  \end{Phonetics}
\end{Entry}

\begin{Entry}{奖金}{9,8}{⼤、⾦}
  \begin{Phonetics}{奖金}{jiang3jin1}[][HSK 4]
    \definition[个,笔]{s.}{bônus; recompensa; prêmio; prêmio em dinheiro; dinheiro de recompensa, dinheiro dado às pessoas para incentivá-las ou elogiá-las por terem se saído bem em alguma coisa}
  \end{Phonetics}
\end{Entry}

\begin{Entry}{姜}{9}{⼥}
  \begin{Phonetics}{姜}{jiang1}
    \definition*{s.}{Sobrenome Jiang}
    \definition[磅,斤,两]{s.}{gengibre; rizoma de gengibre}
  \end{Phonetics}
\end{Entry}

\begin{Entry}{威}{9}{⼥}
  \begin{Phonetics}{威}{wei1}
    \definition*{s.}{Sobrenome Wei}
    \definition{adj.}{forte; poderoso}
    \definition{s.}{força impressionante; poder; força}
    \definition{v.}{ameaçar pela força; intimidar com força}
  \end{Phonetics}
\end{Entry}

\begin{Entry}{威胁}{9,8}{⼥、⾁}
  \begin{Phonetics}{威胁}{wei1xie2}[][HSK 6]
    \definition{v.}{pôr em perigo; ameaçar; intimidar}
  \end{Phonetics}
\end{Entry}

\begin{Entry}{娃}{9}{⼥}
  \begin{Phonetics}{娃}{wa2}
    \definition[个,名,位,只]{s.}{bebê; criança | filho ou filha; criança | Dialeto: animal recém-nascido | Literário: menina; jovem mulher | Literário: menina bonita}
  \end{Phonetics}
\end{Entry}

\begin{Entry}{娃娃}{9,9}{⼥、⼥}
  \begin{Phonetics}{娃娃}{wa2wa5}[][HSK 6]
    \definition[个,名,位]{s.}{bebê; criança; criança pequena | boneca; brinquedos em forma de crianças}
  \end{Phonetics}
\end{Entry}

\begin{Entry}{孩}{9}{⼦}
  \begin{Phonetics}{孩}{hai2}
    \definition[个]{s.}{criança}
  \end{Phonetics}
\end{Entry}

\begin{Entry}{孩子}{9,3}{⼦、⼦}
  \begin{Phonetics}{孩子}{hai2 zi5}[][HSK 1]
    \definition[个]{s.}{criança; crianças; pessoas com idade entre alguns anos ou na adolescência, geralmente com menos de 14 anos | crianças; filho ou filha}
  \end{Phonetics}
\end{Entry}

\begin{Entry}{客}{9}{⼧}
  \begin{Phonetics}{客}{ke4}
    \definition*{s.}{Sobrenome Ke}
    \definition{adj.}{objetivo; independente da consciência humana | estrangeiro; não desta região, unidade ou indústria}
    \definition{clas.}{porção (de comida, bebida, etc.); em algumas áreas, é usado para vender alimentos e bebidas em porções}
    \definition[个,位,名,些]{s.}{convidado; visitante; aquele que é convidado; aquele que vem visitar (em oposição a 主) | viajante; passageiro | comerciante viajante; refere-se especificamente a comerciantes que transportam mercadorias de um lugar para o outro | cliente; patrono; consumidor | uma pessoa envolvida em alguma atividade específica; pessoas que viajam fazendo algum tipo de atividade}
    \definition{v.}{ser um estranho; estabelecer-se (ou viver) em um lugar estranho; estar longe de casa ou morar no exterior}
  \seealsoref{主}{zhu3}
  \end{Phonetics}
\end{Entry}

\begin{Entry}{客人}{9,2}{⼧、⼈}
  \begin{Phonetics}{客人}{ke4ren2}[][HSK 2]
    \definition[位,个,桌,拨,批]{s.}{visitante; convidado | cliente; passageiro; hóspede; viajante}
  \end{Phonetics}
\end{Entry}

\begin{Entry}{客厅}{9,4}{⼧、⼚}
  \begin{Phonetics}{客厅}{ke4ting1}[][HSK 5]
    \definition[间,个]{s.}{sala de estar; sala de visitas; sala para receber convidados}
  \end{Phonetics}
\end{Entry}

\begin{Entry}{客户}{9,4}{⼧、⼾}
  \begin{Phonetics}{客户}{ke4hu4}[][HSK 5]
    \definition[位,个,家,批]{s.}{cliente; consumidor}
  \end{Phonetics}
\end{Entry}

\begin{Entry}{客气}{9,4}{⼧、⽓}
  \begin{Phonetics}{客气}{ke4qi5}[][HSK 5]
    \definition{adj.}{educado; modesto; cortês}
    \definition{v.}{ser educado; ser cortês; fazer comentários educados ou agir educadamente}
  \end{Phonetics}
\end{Entry}

\begin{Entry}{客车}{9,4}{⼧、⾞}
  \begin{Phonetics}{客车}{ke4 che1}[][HSK 6]
    \definition[辆,列,次,趟]{s.}{ônibus; veículo de passageiros; veículos que transportam passageiros em ferrovias e estradas}
  \end{Phonetics}
\end{Entry}

\begin{Entry}{客观}{9,6}{⼧、⾒}
  \begin{Phonetics}{客观}{ke4guan1}[][HSK 3]
    \definition{adj.}{objetivo; justo e razoável; imparcial; com base na situação real, sem preconceitos pessoais}
    \definition{s.}{objetivo; existe fora da consciência, sem depender da consciência subjetiva}
  \end{Phonetics}
\end{Entry}

\begin{Entry}{宣}{9}{⼧}
  \begin{Phonetics}{宣}{xuan1}
    \definition*{s.}{Sobrenome Xuan}
    \definition{v.}{declarar; proclamar; anunciar; falar publicamente | drenar (líquidos)}
  \end{Phonetics}
\end{Entry}

\begin{Entry}{宣布}{9,5}{⼧、⼱}
  \begin{Phonetics}{宣布}{xuan1bu4}[][HSK 3]
    \definition{v.}{declarar; proclamar; pronunciar; anunciar; informar oficialmente a todos sobre as últimas decisões e situações}
  \end{Phonetics}
\end{Entry}

\begin{Entry}{宣传}{9,6}{⼧、⼈}
  \begin{Phonetics}{宣传}{xuan1chuan2}[][HSK 3]
    \definition[个]{v.}{propagar; divulgar; fazer propaganda; explicar e esclarecer às pessoas, para que elas acreditem e sigam as ações}
  \end{Phonetics}
\end{Entry}

\begin{Entry}{宣扬}{9,6}{⼧、⼿}
  \begin{Phonetics}{宣扬}{xuan1yang2}
    \definition{v.}{divulgar | anunciar | espalhar por toda parte}
  \end{Phonetics}
\end{Entry}

\begin{Entry}{室}{9}{⼧}
  \begin{Phonetics}{室}{shi4}[][HSK 3]
    \definition*{s.}{Shi, a décima terceira das vinte e oito constelações da esfera celeste, composta por duas estrelas em linha reta na constelação de Pégaso | Sobrenome Shi}
    \definition{s.}{sala; quarto; casa | departamento; sala como unidade administrativa ou de trabalho; órgãos públicos, fábricas, escolas e outras unidades de trabalho internas | esposa; familiares ou esposa | família; clã | cavidade; órgão com forma semelhante a uma câmara}
  \end{Phonetics}
\end{Entry}

\begin{Entry}{宪}{9}{⼧}
  \begin{Phonetics}{宪}{xian4}
    \definition*{s.}{Sobrenome Xian}
    \definition{s.}{estatuto; decreto | constituição}
  \end{Phonetics}
\end{Entry}

\begin{Entry}{宪制}{9,8}{⼧、⼑}
  \begin{Phonetics}{宪制}{xian4zhi4}
    \definition{adj.}{constitucional}
    \definition{s.}{sistema de governo constitucional}
  \end{Phonetics}
\end{Entry}

\begin{Entry}{宪法法院}{9,8,8,9}{⼧、⽔、⽔、⾩}
  \begin{Phonetics}{宪法法院}{xian4fa3fa3yuan4}
    \definition{s.}{tribunal constitucional}
  \end{Phonetics}
\end{Entry}

\begin{Entry}{宪政}{9,9}{⼧、⽁}
  \begin{Phonetics}{宪政}{xian4zheng4}
    \definition{s.}{governo constitucional}
  \end{Phonetics}
\end{Entry}

\begin{Entry}{宫}{9}{⼧}
  \begin{Phonetics}{宫}{gong1}[][HSK 6]
    \definition*{s.}{Sobrenome Gong}
    \definition[座]{s.}{palácio imperial; palácio; casas onde o imperador, a imperatriz, o príncipe, etc. vivem | morada de seres sobrenaturais; palácio; paraíso; casas onde vivem os deuses na mitologia | templo (usado em um nome de templo) | local para atividades culturais e recreativas; um edifício para atividades culturais e recreativas; casas para fins culturais e de entretenimento | útero | uma nota da antiga escala chinesa de cinco tons, correspondente a 1 na notação musical numerada}
  \end{Phonetics}
\end{Entry}

\begin{Entry}{封}{9}{⼨}
  \begin{Phonetics}{封}{feng1}[][HSK 2,5]
    \definition*{s.}{Sobrenome Feng}
    \definition{clas.}{usado para objetos selados, especialmente cartas}
    \definition{s.}{feudalismo | embalagem; envelope | pacote}
    \definition{v.}{conferir (um título, território, etc.) a | selar | acender uma fogueira | fechar}
  \end{Phonetics}
\end{Entry}

\begin{Entry}{封口}{9,3}{⼨、⼝}
  \begin{Phonetics}{封口}{feng1kou3}
    \definition{v.}{selar | fechar | curar (uma ferida) | manter os lábios selados}
  \end{Phonetics}
\end{Entry}

\begin{Entry}{封印}{9,5}{⼨、⼙}
  \begin{Phonetics}{封印}{feng1yin4}
    \definition{s.}{selo (em envelopes)}
  \end{Phonetics}
\end{Entry}

\begin{Entry}{封闭}{9,6}{⼨、⾨}
  \begin{Phonetics}{封闭}{feng1bi4}[][HSK 4]
    \definition{adj.}{fechado; aqueles que não têm contato com o mundo exterior; aqueles que são muito conservadores (em seu pensamento) e não se comunicam com os outros}
    \definition{v.}{selar; fechar; lacrar; vedar; de modo a impedir a passagem, o uso ou a abertura}
  \end{Phonetics}
\end{Entry}

\begin{Entry}{封冻}{9,7}{⼨、⼎}
  \begin{Phonetics}{封冻}{feng1dong4}
    \definition{v.}{congelar (água ou terra)}
  \end{Phonetics}
\end{Entry}

\begin{Entry}{封底}{9,8}{⼨、⼴}
  \begin{Phonetics}{封底}{feng1di3}
    \definition{s.}{contracapa de um livro}
  \end{Phonetics}
\end{Entry}

\begin{Entry}{封建}{9,8}{⼨、⼵}
  \begin{Phonetics}{封建}{feng1jian4}
    \definition{adj.}{feudal}
    \definition{s.}{feudalismo}
  \end{Phonetics}
\end{Entry}

\begin{Entry}{封面}{9,9}{⼨、⾯}
  \begin{Phonetics}{封面}{feng1mian4}
    \definition{s.}{capa (de uma publicação) | sobrecapa}
  \end{Phonetics}
\end{Entry}

\begin{Entry}{封斋}{9,10}{⼨、⽂}
  \begin{Phonetics}{封斋}{feng1zhai1}
    \definition*{s.}{Ramadã (Islã)}
  \end{Phonetics}
\end{Entry}

\begin{Entry}{封盖}{9,11}{⼨、⽫}
  \begin{Phonetics}{封盖}{feng1gai4}
    \definition{s.}{boné | capa | selo}
    \definition{v.}{cobrir}
  \end{Phonetics}
\end{Entry}

\begin{Entry}{将}{9}{⼨}
  \begin{Phonetics}{将}{jiang1}[][HSK 5]
    \definition*{s.}{Sobrenome Jiang}
    \definition{adv.}{estar indo para; parcialmente\dots parcialmente\dots}
    \definition{part.}{expressar uma direção, como 进来, 出去; usado no meio de verbos e complementos que indicam tendência, como 进来, 出去, etc.}
    \definition{prep.}{com; por meio de; por | usado da mesma forma que 把}
    \definition{v.}{fazer algo; lidar com (um assunto) | dar um cheque-mate | cuidar (da saúde) | incitar alguém a agir; desafiar; estimular | segurar; pegar | colocar; tirar | levar; trazer | dar suporte; dar apoio}
  \seealsoref{把}{ba3}
  \seealsoref{出去}{chu1 qu4}
  \seealsoref{进来}{jin4 lai2}
  \end{Phonetics}
  \begin{Phonetics}{将}{jiang4}
    \definition{s.}{general; nome do posto; abaixo de marechal de campo; acima de coronel}
    \definition{v.}{comandar; liderar}
  \end{Phonetics}
  \begin{Phonetics}{将}{qiang1}
    \definition{v.}{pedir; apelar para}
  \end{Phonetics}
\end{Entry}

\begin{Entry}{将军}{9,6}{⼨、⼍}
  \begin{Phonetics}{将军}{jiang1jun1}[][HSK 6]
    \definition[位,名]{s.}{general; geralmente se refere a generais seniores}
    \definition{v.+compl.}{dar xeque-mate; atacar o general ou rei do oponente no xadrez; colocar alguém em grandes apuros; metáfora para dar a alguém um problema difícil ou dificultar a tarefa para essa pessoa}
  \end{Phonetics}
\end{Entry}

\begin{Entry}{将来}{9,7}{⼨、⽊}
  \begin{Phonetics}{将来}{jiang1lai2}[][HSK 3]
    \definition[个]{s.}{no futuro (geralmente se refere a um período mais longo)}
  \end{Phonetics}
\end{Entry}

\begin{Entry}{将近}{9,7}{⼨、⾡}
  \begin{Phonetics}{将近}{jiang1jin4}[][HSK 3]
    \definition{adv.}{quase}
  \end{Phonetics}
\end{Entry}

\begin{Entry}{将要}{9,9}{⼨、⾑}
  \begin{Phonetics}{将要}{jiang1 yao4}[][HSK 5]
    \definition{adv.}{irá; deverá; estará prestes a; irá a; indica que um ato ou situação ocorre logo em seguida}
  \end{Phonetics}
\end{Entry}

\begin{Entry}{尝}{9}{⼩}
  \begin{Phonetics}{尝}{chang2}[][HSK 5]
    \definition{adv.}{alguma vez; uma vez}
    \definition{v.}{provar; experimentar o sabor de | provar; experimentar; conhecer | tentar; testar}
  \end{Phonetics}
\end{Entry}

\begin{Entry}{尝试}{9,8}{⼩、⾔}
  \begin{Phonetics}{尝试}{chang2shi4}[][HSK 5]
    \definition{v.}{tentar; provar; experimentar}
  \end{Phonetics}
\end{Entry}

\begin{Entry}{屋}{9}{⼫}
  \begin{Phonetics}{屋}{wu1}[][HSK 5]
    \definition[间,座]{s.}{casa | quarto}
  \end{Phonetics}
\end{Entry}

\begin{Entry}{屋子}{9,3}{⼫、⼦}
  \begin{Phonetics}{屋子}{wu1zi5}[][HSK 3]
    \definition[间,座,栋]{s.}{quarto; sala}
  \end{Phonetics}
\end{Entry}

\begin{Entry}{屌}{9}{⼫}
  \begin{Phonetics}{屌}{diao3}
    \definition{adj.}{(gíria) legal ou extraordinário}
    \definition{s.}{órgão genital masculino; pênis}
    \definition{v.}{(cantonês) foder}
  \end{Phonetics}
\end{Entry}

\begin{Entry}{屌丝}{9,5}{⼫、⼀}
  \begin{Phonetics}{屌丝}{diao3si1}
    \definition{adj.}{panaca | zé-ninguém | (gíria de \emph{Internet}) \emph{looser}}
  \end{Phonetics}
\end{Entry}

\begin{Entry}{屎}{9}{⼫}
  \begin{Phonetics}{屎}{shi3}
    \definition{s.}{fezes | excrementos | (forma ligada) secreção (do ouvido, olho, etc.)}
  \end{Phonetics}
\end{Entry}

\begin{Entry}{屏}{9}{⼫}
  \begin{Phonetics}{屏}{bing1}
    \definition{s.}{antigamente, referia-se à pequena parede de tela em frente ao portão de um antigo palácio; no chinês moderno, também é usado como uma palavra humilde para expressar o significado de 惶恐}
  \seealsoref{惶恐}{huang2kong3}
  \end{Phonetics}
  \begin{Phonetics}{屏}{bing3}
    \definition*{s.}{Sobrenome Bing}
    \definition{v.}{prender (a respiração); conter a respiração | rejeitar; livrar-se de; remover; pôr (colocar) de lado; abandonar; descartar}
  \end{Phonetics}
  \begin{Phonetics}{屏}{ping2}
    \definition{s.}{tela | um conjunto de pergaminhos; tiras de tela}
    \definition{v.}{proteger alguém ou algo; resguardar}
  \end{Phonetics}
\end{Entry}

\begin{Entry}{屏幕}{9,13}{⼫、⼱}
  \begin{Phonetics}{屏幕}{ping2 mu4}[][HSK 6]
    \definition[个,块]{s.}{tela; a parte dos computadores, televisores, celulares, etc. que exibe texto, imagens, etc.}
  \end{Phonetics}
\end{Entry}

\begin{Entry}{差}{9}{⼯}
  \begin{Phonetics}{差}{cha1}
    \definition{adj.}{diferente; diferente ou inconsistente com um determinado padrão}
    \definition{adv.}{ligeiramente; comparativamente; um pouco}
    \definition{s.}{diferença; resto após a subtração de dois números | erro; engano}
  \end{Phonetics}
  \begin{Phonetics}{差}{cha4}[][HSK 1]
    \definition{adj.}{não está de acordo com o padrão; pobre; ruim; inferior | errado; incorreto | mesmo significado de 差 \dpy{cha1}}
    \definition{v.}{faltar}
  \end{Phonetics}
  \begin{Phonetics}{差}{chai1}
    \definition{s.}{tarefa; trabalho; ser enviado para fazer algo; deveres oficiais; posição | corvéia; mensageiro ou oficial de justiça em um yamen feudal; (velho) refere-se a pessoas que são enviadas para fazer coisas}
    \definition{v.}{enviar uma mensagem; despachar; fnviar (para fazer algo)}
  \end{Phonetics}
\end{Entry}

\begin{Entry}{差(一)点儿}{9,1,9,2}{⼯、⼀、⽕、⼉}
  \begin{Phonetics}{差(一)点儿}{cha1yi4dian3r5}[][HSK 5]
    \definition{adj.}{não é bom o suficiente; ligeiramente inferior a; não está à altura da marca;  (qualidade, tecnologia, desempenho, etc.) ligeiramente inferior}
    \definition{adv.}{quase; à beira de; praticamente; aproximadamente; significa que algo está perto de ser alcançado, mas não foi alcançado, ou algo foi alcançado, mas mal foi alcançado}
  \end{Phonetics}
\end{Entry}

\begin{Entry}{差不多}{9,4,6}{⼯、⼀、⼣}
  \begin{Phonetics}{差不多}{cha4bu5duo1}[][HSK 2]
    \definition{adj.}{semelhante; aproximadamente igual | não muito longe; quase certo (suficiente); basicamente, próximo dos padrões e requisitos; normal | prestes a (terminar; acabar); descreve que (algo) está quase acabando; (uma tarefa) está quase concluída}
    \definition{adv.}{quase; perto; indica proximidade}
  \end{Phonetics}
\end{Entry}

\begin{Entry}{差异}{9,6}{⼯、⼶}
  \begin{Phonetics}{差异}{cha1 yi4}[][HSK 6]
    \definition{s.}{diferença; divergência; discrepância}
  \end{Phonetics}
\end{Entry}

\begin{Entry}{差别}{9,7}{⼯、⼑}
  \begin{Phonetics}{差别}{cha1bie2}[][HSK 5]
    \definition{s.}{diferença; disparidade; dissimilaridade; distinção; não semelhança; diferenças na forma ou no conteúdo}
  \end{Phonetics}
\end{Entry}

\begin{Entry}{差点儿}{9,9,2}{⼯、⽕、⼉}
  \begin{Phonetics}{差点儿}{cha4dian3r5}
    \definition{adv.}{por pouco | por um triz | quase}
  \end{Phonetics}
\end{Entry}

\begin{Entry}{差距}{9,11}{⼯、⾜}
  \begin{Phonetics}{差距}{cha1ju4}[][HSK 5]
    \definition[个,些,段]{s.}{lacuna; disparidade; discrepância; diferença; grau de diferença entre as coisas, especialmente em termos de distância de algum padrão.}
  \end{Phonetics}
\end{Entry}

\begin{Entry}{帝}{9}{⼱}
  \begin{Phonetics}{帝}{di4}
    \definition*{s.}{Ser Supremo; Deus}
    \definition[位,名,个]{s.}{imperador | (abreviação) imperialismo}
  \end{Phonetics}
\end{Entry}

\begin{Entry}{帝国}{9,8}{⼱、⼞}
  \begin{Phonetics}{帝国}{di4guo2}
    \definition{adj.}{imperial}
    \definition{s.}{império}
  \end{Phonetics}
\end{Entry}

\begin{Entry}{带}{9}{⼱}
  \begin{Phonetics}{带}{dai4}[][HSK 2]
    \definition*{s.}{Sobrenome Dai}
    \definition[根]{s.}{cinto; faixa; banda; fita; fita adesiva; algo parecido com uma fita | pneu | zona; área; faixa; cinturão; região; uma determinada área geográfica com determinadas características | leucorreia; corrimento branco; corrimento vaginal}
    \definition{v.}{levar; trazer; transportar | liderar; dirigir; conduzir; assumir | cuidar de crianças; criar filhos; educar | fazer uma coisa e, ao mesmo tempo, fazer outra coisa |suportar; conter | ter algo anexado, simultâneo | trazer consigo | carregar consigo | demonstrar; parecer | incluir; acrescentar}
  \end{Phonetics}
\end{Entry}

\begin{Entry}{带动}{9,6}{⼱、⼒}
  \begin{Phonetics}{带动}{dai4 dong4}[][HSK 3]
    \definition{v.}{dirigir; ativar; fazer algo funcionar; acionar | liderar; trazer; estimular; motivar; atrair; liderar o avanço; dar o exemplo e fazer com que os outros sigam o exemplo}
  \end{Phonetics}
\end{Entry}

\begin{Entry}{带有}{9,6}{⼱、⽉}
  \begin{Phonetics}{带有}{dai4 you3}[][HSK 5]
    \definition{v.}{ter; envolver; carregar; implicar}
  \end{Phonetics}
\end{Entry}

\begin{Entry}{带来}{9,7}{⼱、⽊}
  \begin{Phonetics}{带来}{dai4 lai2}[][HSK 2]
    \definition{v.}{provocar; produzir; causar}
  \end{Phonetics}
\end{Entry}

\begin{Entry}{带领}{9,11}{⼱、⾴}
  \begin{Phonetics}{带领}{dai4ling3}[][HSK 3]
    \definition{v.}{guiar, na frente, liderando | liderar e comandar}
  \end{Phonetics}
\end{Entry}

\begin{Entry}{帮}{9}{⼱}
  \begin{Phonetics}{帮}{bang1}[][HSK 1]
    \definition*{s.}{Sobrenome Bang}
    \definition{clas.}{um grupo de; um bando de; uma gangue de; um grupo de pessoas}
    \definition{s.}{lateral; superior; partes ao lado ou ao redor do objeto | folha externa; parte mais grossa das folhas externas dos vegetais | gangue; banda; grupo; conglomerado}
    \definition{v.}{ajudar; assistir; auxiliar | trabalho; refere-se ao envolvimento em trabalho assalariado}
  \end{Phonetics}
\end{Entry}

\begin{Entry}{帮忙}{9,6}{⼱、⼼}
  \begin{Phonetics}{帮忙}{bang1 mang2}[][HSK 1]
    \definition{v.+compl.}{ajudar; dar uma mão; dar uma mãozinha; fazer um favor; fazer uma boa ação; ajudar os outros a fazer algo, referindo-se, de maneira geral, a oferecer ajuda quando alguém está com dificuldades}
  \end{Phonetics}
\end{Entry}

\begin{Entry}{帮佣}{9,7}{⼱、⼈}
  \begin{Phonetics}{帮佣}{bang1yong1}
    \definition{s.}{trabalhador doméstico; empregada doméstica; servo; servente}
    \definition{v.}{trabalhar ou ser contratado como trabalhador doméstico, servo, etc.}
  \end{Phonetics}
\end{Entry}

\begin{Entry}{帮助}{9,7}{⼱、⼒}
  \begin{Phonetics}{帮助}{bang1zhu4}[][HSK 2]
    \definition[个,次,回,份,种]{s.}{ajuda; auxílio; socorro; função de promoção ou auxílio}
    \definition{v.}{ajudar; assistir; apoiar; quando alguém está passando por dificuldades, oferecer apoio financeiro ou material, ou ainda apoio moral, dar conselhos, pensar em soluções, fazer coisas por essa pessoa, etc.}
  \end{Phonetics}
\end{Entry}

\begin{Entry}{帮教}{9,11}{⼱、⽁}
  \begin{Phonetics}{帮教}{bang1jiao4}
    \definition{v.}{orientar}
  \end{Phonetics}
\end{Entry}

\begin{Entry}{幽}{9}{⼳}
  \begin{Phonetics}{幽}{you1}
    \definition*{s.}{Sobrenome You}
    \definition{adj.}{profundo e remoto; isolado; escuro | secreto; escondido; oculto; não público | quieto; tranquilo; sereno | do mundo inferior}
    \definition{s.}{mundo inferior}
  \end{Phonetics}
\end{Entry}

\begin{Entry}{幽默}{9,16}{⼳、⿊}
  \begin{Phonetics}{幽默}{you1mo4}[][HSK 5]
    \definition{adj.}{humorístico; interessante ou engraçado, mas com um significado profundo}
    \definition{s.}{humor; lado engraçado; graça; características, temperamento, palavras ou comportamentos interessantes, engraçados ou significativos}
  \end{Phonetics}
\end{Entry}

\begin{Entry}{度}{9}{⼴}
  \begin{Phonetics}{度}{du4}[][HSK 2]
    \definition*{s.}{Sobrenome Du}
    \definition{clas.}{grau; unidade de medida para ângulos, temperatura, etc. | quilowatt-hora (kWh) | usado para indicar a quantidade de álcool presente no vinho | usado para arcos e ângulos | usado para indicar o grau de curvatura da lente dos óculos ou o grau de miopia | tempo; número de vezes | usado para longitude e latitude, localização geográfica}
    \definition{s.}{medida linear; padrões e instrumentos para medir comprimentos | grau de intensidade; refere-se especificamente ao grau alcançado por uma determinada propriedade de uma coisa | limite; extensão; grau; quota | regras; código de conduta; diretrizes | tolerância; magnanimidade; refere-se especificamente ao grau de tolerância | maneira; temperamento; disposição; a personalidade ou aparência de uma pessoa | indicador de grau, nível alcançado por algo | tempo ou espaço limitado; um determinado período de tempo ou espaço}
    \definition{v.}{passar; atravessar; passar por cima | (em termos de tempo) passar; passar por | (de monges ou monjas budistas, ou sacerdotes taoístas) pregar; converter; proselitar}
  \end{Phonetics}
  \begin{Phonetics}{度}{duo2}
    \definition{v.}{supor; estimar; especular}
  \end{Phonetics}
\end{Entry}

\begin{Entry}{度过}{9,6}{⼴、⾡}
  \begin{Phonetics}{度过}{du4guo4}[][HSK 4]
    \definition{s.}{passar o tempo; fazer o tempo desaparecer no trabalho, na vida, no lazer e no descanso}
  \end{Phonetics}
\end{Entry}

\begin{Entry}{弯}{9}{⼸}
  \begin{Phonetics}{弯}{wan1}[][HSK 4]
    \definition{adj.}{curvo; tortuoso; torto | para algo curvo, como a lua, etc. | dobrado; flexível}
    \definition[个,道]{s.}{curva; dobra; volta}
    \definition{v.}{dobrar; flexionar; curvar | Literário: desenhar}
  \end{Phonetics}
\end{Entry}

\begin{Entry}{弯曲}{9,6}{⼸、⽈}
  \begin{Phonetics}{弯曲}{wan1 qu1}[][HSK 6]
    \definition{s.}{torto; curvo; sinuoso; tortuoso; não reto}
    \definition{v.}{dobrar; curvar; flexionar}
  \end{Phonetics}
\end{Entry}

\begin{Entry}{待}{9}{⼻}
  \begin{Phonetics}{待}{dai1}[][HSK 5]
    \definition{v.}{ficar; permanecer | ir além (de um período de tempo)}
  \end{Phonetics}
  \begin{Phonetics}{待}{dai4}
    \definition*{s.}{Sobrenome Dai}
    \definition{v.}{tratar; lidar com | entreter; receber (convidados) | aguardar; esperar por | precisar; necessitar | desejar; pretender; querer}
  \end{Phonetics}
\end{Entry}

\begin{Entry}{待会儿}{9,6,2}{⼻、⼈、⼉}
  \begin{Phonetics}{待会儿}{dai1 hui4r5}[][HSK 6]
    \definition{adv.}{em um momento; depois de um tempo | mais tarde; depois}
  \end{Phonetics}
\end{Entry}

\begin{Entry}{待遇}{9,12}{⼻、⾡}
  \begin{Phonetics}{待遇}{dai4yu4}[][HSK 4]
    \definition[种,项,份]{s.}{tratamento; refere-se a direitos, status social, etc. | salário; ordenado; remuneração}
  \end{Phonetics}
\end{Entry}

\begin{Entry}{很}{9}{⼻}
  \begin{Phonetics}{很}{hen3}[][HSK 1]
    \definition{adv.}{muito; bastante; terrivelmente; indica um grau bastante elevado; definitivo; o mais alto}
  \end{Phonetics}
\end{Entry}

\begin{Entry}{很难说}{9,10,9}{⼻、⾫、⾔}
  \begin{Phonetics}{很难说}{hen3 nan2 shuo1}[][HSK 6]
    \definition{adj.}{difícil dizer}
  \end{Phonetics}
\end{Entry}

\begin{Entry}{律}{9}{⼻}
  \begin{Phonetics}{律}{lv4}
    \definition*{s.}{Sobrenome Lü}
    \definition{s.}{lei; regra; estatuto; regulamento}
    \definition{v.}{restringir; disciplinar; manter sob controle}
  \end{Phonetics}
\end{Entry}

\begin{Entry}{律师}{9,6}{⼻、⼱}
  \begin{Phonetics}{律师}{lv4shi1}[][HSK 4]
    \definition[名,个,位]{s.}{advogado; procurador; profissionais encarregados pelas partes ou nomeados pelo tribunal para auxiliar as partes no litígio, para comparecer ao tribunal para defesa e para tratar de assuntos jurídicos relacionados, de acordo com a lei}
  \end{Phonetics}
\end{Entry}

\begin{Entry}{怎}{9}{⼼}
  \begin{Phonetics}{怎}{zen3}
    \definition{adv.}{como}
  \end{Phonetics}
\end{Entry}

\begin{Entry}{怎么}{9,3}{⼼、⼃}
  \begin{Phonetics}{怎么}{zen3me5}[][HSK 1]
    \definition{pron.}{como?; o quê?; perguntas sobre natureza, situação, método, motivo, etc. | de qualquer maneira; não importa como; de uma certa maneira; referência geral à natureza, condição ou modo | que? (usado sozinho no início de uma frase para expressar surpresa) | usado após 不 e 没, indica um grau baixo e é uma forma mais educada de se expressar | usado em perguntas retóricas}
  \seealsoref{不}{bu4}
  \seealsoref{没}{mei2}
  \end{Phonetics}
\end{Entry}

\begin{Entry}{怎么了}{9,3,2}{⼼、⼃、⼅}
  \begin{Phonetics}{怎么了}{zen3me5le5}
    \definition{expr.}{O que aconteceu? | O que está acontecendo? | E aí?}
  \end{Phonetics}
\end{Entry}

\begin{Entry}{怎么办}{9,3,4}{⼼、⼃、⼒}
  \begin{Phonetics}{怎么办}{zen3 me5 ban4}[][HSK 2]
    \definition{adv.}{o que fazer?; o que deve ser feito?}
  \end{Phonetics}
\end{Entry}

\begin{Entry}{怎么回事}{9,3,6,8}{⼼、⼃、⼞、⼅}
  \begin{Phonetics}{怎么回事}{zen3me5hui2shi4}
    \definition{expr.}{O que aconteceu? | O que se passou?}
  \end{Phonetics}
\end{Entry}

\begin{Entry}{怎么样}{9,3,10}{⼼、⼃、⽊}
  \begin{Phonetics}{怎么样}{zen3me5yang4}[][HSK 2]
    \definition{adv.}{como?; o que?; como é?; como estão as coisas?; o que você acha?; pergunte sobre o método, natureza, situação, opinião, etc. | substitui uma ação ou situação não dita (usado apenas na forma negativa, mais eufemístico do que uma declaração direta); indaga sobre a natureza, condição, método, razão, etc.}
  \end{Phonetics}
\end{Entry}

\begin{Entry}{怎么得了}{9,3,11,2}{⼼、⼃、⼻、⼅}
  \begin{Phonetics}{怎么得了}{zen3me5de2liao3}
    \definition{expr.}{Como isso pode ser? | Que bagunça horrível! | O que deve ser feito?}
  \end{Phonetics}
\end{Entry}

\begin{Entry}{怎么搞的}{9,3,13,8}{⼼、⼃、⼿、⽩}
  \begin{Phonetics}{怎么搞的}{zen3me5gao3de5}
    \definition{expr.}{Como isso aconteceu? | O que deu errado? | E aí? | O que está errado?}
  \end{Phonetics}
\end{Entry}

\begin{Entry}{怎样}{9,10}{⼼、⽊}
  \begin{Phonetics}{怎样}{zen3 yang4}[][HSK 2]
    \definition{pron.}{como?; o que?; indagar sobre a natureza, condição ou método, etc. | como?; indica uma referência virtual | de uma certa maneira; de qualquer maneira; não importa como; indica qualquer | como?; usado como predicado, objeto ou complemento para indagar sobre uma situação}
  \end{Phonetics}
\end{Entry}

\begin{Entry}{怒}{9}{⼼}
  \begin{Phonetics}{怒}{nu4}
    \definition{adj.}{zangado; furioso | feroz; forte; descreve um forte impulso}
    \definition{adv.}{com força; vigorosamente; dinamicamente | com raiva}
    \definition{s.}{raiva; fúria}
    \definition{v.}{enfurecer-se; ficar com raiva}
  \end{Phonetics}
\end{Entry}

\begin{Entry}{怒放}{9,8}{⼼、⽅}
  \begin{Phonetics}{怒放}{nu4fang4}
    \definition{v.}{florescer em plena floração}
  \end{Phonetics}
\end{Entry}

\begin{Entry}{怒骂}{9,9}{⼼、⾺}
  \begin{Phonetics}{怒骂}{nu4ma4}
    \definition{v.}{praguejar de raiva}
  \end{Phonetics}
\end{Entry}

\begin{Entry}{思}{9}{⼼}
  \begin{Phonetics}{思}{si1}
    \definition*{s.}{Sobrenome Si}
    \definition{s.}{pensamento; ideias | pensamentos; emoções; humor}
    \definition{v.}{pensar; considerar; deliberar | pensar em; ansiar por}
  \end{Phonetics}
\end{Entry}

\begin{Entry}{思考}{9,6}{⼼、⽼}
  \begin{Phonetics}{思考}{si1kao3}[][HSK 4]
    \definition{v.}{pensar; ponderar; considerar; deliberar; envolver-se em atividades de pensamento, como análise, síntese, julgamento, raciocínio e generalização}
  \end{Phonetics}
\end{Entry}

\begin{Entry}{思维}{9,11}{⼼、⽷}
  \begin{Phonetics}{思维}{si1wei2}[][HSK 5]
    \definition[种]{s.}{pensamento; reflexão; organizar e transformar os materiais obtidos através do conhecimento sensorial para formar conceitos, julgamentos e raciocínios}
    \definition{v.}{pensar}
  \end{Phonetics}
\end{Entry}

\begin{Entry}{思想}{9,13}{⼼、⼼}
  \begin{Phonetics}{思想}{si1xiang3}[][HSK 3]
    \definition[个,种]{s.}{reflexão; pensamento; ideologia; a existência objetiva é refletida na consciência das pessoas por meio de atividades de pensamento, que pertencem à cognição racional | ideia; pensamento}
  \end{Phonetics}
\end{Entry}

\begin{Entry}{急}{9}{⼼}
  \begin{Phonetics}{急}{ji2}[][HSK 2]
    \definition{adj.}{impaciente; ansioso | irritado; aborrecido; incomodado | rápido e intenso (em oposição a 缓); veloz | urgente; premente}
    \definition{s.}{urgência; emergência; assunto urgente e grave}
    \definition{v.}{preocupar; deixar ansioso | estar ansioso para ajudar; tratar os problemas dos outros como se fossem urgentes e ajudar a resolvê-los imediatamente}
  \seealsoref{缓}{huan3}
  \end{Phonetics}
\end{Entry}

\begin{Entry}{急忙}{9,6}{⼼、⼼}
  \begin{Phonetics}{急忙}{ji2mang2}[][HSK 4]
    \definition{adv.}{apressadamente; com pressa}
  \end{Phonetics}
\end{Entry}

\begin{Entry}{急救}{9,11}{⼼、⽁}
  \begin{Phonetics}{急救}{ji2 jiu4}[][HSK 6]
    \definition{s.}{primeiros socorros; tratamento médico de emergência (para pessoas gravemente doentes ou gravemente feridas)}
    \definition{v.}{prestar primeiros socorros; dar tratamento de emergência}
  \end{Phonetics}
\end{Entry}

\begin{Entry}{怨}{9}{⼼}
  \begin{Phonetics}{怨}{yuan4}[][HSK 5]
    \definition{s.}{ressentimento; inimizade; rancor}
    \definition{v.}{culpar; reclamar}
  \end{Phonetics}
\end{Entry}

\begin{Entry}{怹}{9}{⼼}
  \begin{Phonetics}{怹}{tan1}
    \definition{pron.}{ele, ela (cortês, em oposição a 他)}
  \seealsoref{他}{ta1}
  \end{Phonetics}
\end{Entry}

\begin{Entry}{总}{9}{⼼}
  \begin{Phonetics}{总}{zong3}[][HSK 3]
    \definition{adj.}{total; geral; global | responsável (liderança)}
    \definition{adv.}{sempre; invariavelmente | de qualquer forma; afinal; eventualmente; mais cedo ou mais tarde; no fim das contas | certamente; provavelmente; com certeza; expressa estimativa; suposição; equivalente a 大概}
    \definition{v.}{reunir; resumir; juntar; compilar}
  \seealsoref{大概}{da4gai4}
  \end{Phonetics}
\end{Entry}

\begin{Entry}{总之}{9,3}{⼼、⼂}
  \begin{Phonetics}{总之}{zong3zhi1}[][HSK 4]
    \definition{conj.}{em uma palavra; em suma; em resumo; indica que a declaração seguinte é uma declaração geral}
  \end{Phonetics}
\end{Entry}

\begin{Entry}{总长}{9,4}{⼼、⾧}
  \begin{Phonetics}{总长}{zong3chang2}
    \definition{s.}{comprimento total}
  \end{Phonetics}
\end{Entry}

\begin{Entry}{总务}{9,5}{⼼、⼒}
  \begin{Phonetics}{总务}{zong3wu4}
    \definition{s.}{divisão de assuntos gerais | assuntos gerais | pessoa responsável geral}
  \end{Phonetics}
\end{Entry}

\begin{Entry}{总台}{9,5}{⼼、⼝}
  \begin{Phonetics}{总台}{zong3tai2}
    \definition{s.}{recepção | balcão de recepção}
  \end{Phonetics}
\end{Entry}

\begin{Entry}{总价}{9,6}{⼼、⼈}
  \begin{Phonetics}{总价}{zong3jia4}
    \definition{s.}{preço total}
  \end{Phonetics}
\end{Entry}

\begin{Entry}{总共}{9,6}{⼼、⼋}
  \begin{Phonetics}{总共}{zong3gong4}[][HSK 4]
    \definition{adv.}{em tudo; em todos; no total; completamente; totalmente; em conjunto}
  \end{Phonetics}
\end{Entry}

\begin{Entry}{总体}{9,7}{⼼、⼈}
  \begin{Phonetics}{总体}{zong3 ti3}[][HSK 5]
    \definition{s.}{total; geral; conjunto; totalidade; massa; população; o todo formado pela união de vários indivíduos; a totalidade das coisas}
  \end{Phonetics}
\end{Entry}

\begin{Entry}{总线}{9,8}{⼼、⽷}
  \begin{Phonetics}{总线}{zong3xian4}
    \definition{s.}{barramento (computador) | \emph{computer bus}}
  \end{Phonetics}
\end{Entry}

\begin{Entry}{总经理}{9,8,11}{⼼、⽷、⽟}
  \begin{Phonetics}{总经理}{zong3 jing1 li3}[][HSK 6]
    \definition[位,名,个,些]{s.}{CEO; gerente geral; o mais alto executivo de uma empresa ou organização similar, que geralmente tem o poder de decidir políticas administrativas e de gestão}
  \end{Phonetics}
\end{Entry}

\begin{Entry}{总是}{9,9}{⼼、⽇}
  \begin{Phonetics}{总是}{zong3shi4}[][HSK 3]
    \definition{adv.}{sempre; indica como tem sido durante um determinado período de tempo; um determinado estado permanece inalterado | afinal; significa que, independentemente do que acontecer, haverá ou será um resultado}
  \end{Phonetics}
\end{Entry}

\begin{Entry}{总结}{9,9}{⼼、⽷}
  \begin{Phonetics}{总结}{zong3jie2}[][HSK 3]
    \definition[个,篇]{s.}{resumo; síntese; conclusão resumida}
    \definition{v.}{resumir; sumariar; sintetizar; analisar e estudar as experiências para chegar a conclusões}
  \end{Phonetics}
\end{Entry}

\begin{Entry}{总统}{9,9}{⼼、⽷}
  \begin{Phonetics}{总统}{zong3tong3}[][HSK 4]
    \definition*[个,位,名,家]{s.}{Presidente (de um país); Título dos líderes de determinadas repúblicas}
  \end{Phonetics}
\end{Entry}

\begin{Entry}{总值}{9,10}{⼼、⼈}
  \begin{Phonetics}{总值}{zong3zhi2}
    \definition{s.}{valor total}
  \end{Phonetics}
\end{Entry}

\begin{Entry}{总监}{9,10}{⼼、⽫}
  \begin{Phonetics}{总监}{zong3 jian1}[][HSK 6]
    \definition[名,位]{s.}{inspetor geral; inspetor-chefe}
  \end{Phonetics}
\end{Entry}

\begin{Entry}{总站}{9,10}{⼼、⽴}
  \begin{Phonetics}{总站}{zong3zhan4}
    \definition{s.}{terminal}
  \end{Phonetics}
\end{Entry}

\begin{Entry}{总部}{9,10}{⼼、⾢}
  \begin{Phonetics}{总部}{zong3 bu4}[][HSK 6]
    \definition{s.}{sede geral; escritório central}
  \end{Phonetics}
\end{Entry}

\begin{Entry}{总得}{9,11}{⼼、⼻}
  \begin{Phonetics}{总得}{zong3dei3}
    \definition{adv.}{prestes a}
    \definition{v.}{dever | precisar}
  \end{Phonetics}
\end{Entry}

\begin{Entry}{总理}{9,11}{⼼、⽟}
  \begin{Phonetics}{总理}{zong3li3}[][HSK 4]
    \definition*[个,位,名]{s.}{Primeiro-Ministro do Conselho de Estado; Título do líder do Conselho de Estado da China | Título do chefe de governo em determinados países | Primeiro-Ministro; Título de líderes de determinados partidos políticos | Título dos chefes de determinadas instituições e empresas nos velhos tempos}
    \definition{v.}{assumir a responsabilidade total}
  \end{Phonetics}
\end{Entry}

\begin{Entry}{总裁}{9,12}{⼼、⾐}
  \begin{Phonetics}{总裁}{zong3cai2}[][HSK 5]
    \definition[位,名,个]{s.}{presidente (de uma empresa); nomes de certos líderes de partidos políticos ou grandes empresas}
  \end{Phonetics}
\end{Entry}

\begin{Entry}{总量}{9,12}{⼼、⾥}
  \begin{Phonetics}{总量}{zong3 liang4}[][HSK 6]
    \definition{s.}{capacidade total; quantidade bruta | valor total | total}
  \end{Phonetics}
\end{Entry}

\begin{Entry}{总数}{9,13}{⼼、⽁}
  \begin{Phonetics}{总数}{zong3 shu4}[][HSK 5]
    \definition{s.}{soma; total; totalidade; inventário; número total; soma total}
  \end{Phonetics}
\end{Entry}

\begin{Entry}{总督}{9,13}{⼼、⽬}
  \begin{Phonetics}{总督}{zong3du1}
    \definition*{s.}{Governador-Geral | Governador | Vice-Rei}
  \end{Phonetics}
\end{Entry}

\begin{Entry}{总算}{9,14}{⼼、⽵}
  \begin{Phonetics}{总算}{zong3suan4}[][HSK 5]
    \definition{adv.}{finalmente; por fim; indica que, após um longo período de tempo, um desejo finalmente se tornou realidade | suficiente; considerando tudo; no geral; considerando todos os aspectos; significa que, em geral, está tudo bem}
  \end{Phonetics}
\end{Entry}

\begin{Entry}{恒}{9}{⼼}
  \begin{Phonetics}{恒}{heng2}
    \definition*{s.}{Sobrenome Heng}
    \definition{adj.}{permanente; duradouro | usual; comum; constante | usual; frequente; constante}
    \definition{s.}{perseverança; constância}
  \end{Phonetics}
\end{Entry}

\begin{Entry}{恒星系}{9,9,7}{⼼、⽇、⽷}
  \begin{Phonetics}{恒星系}{heng2xing1xi4}
    \definition{s.}{sistema estelar | galáxia}
  \end{Phonetics}
\end{Entry}

\begin{Entry}{恢}{9}{⼼}
  \begin{Phonetics}{恢}{hui1}
    \definition{adj.}{extenso; vasto | grande; ótimo}
    \definition{v.}{recuperar; restaurar; restabelecer}
  \end{Phonetics}
\end{Entry}

\begin{Entry}{恢复}{9,9}{⼼、⼢}
  \begin{Phonetics}{恢复}{hui1fu4}[][HSK 5]
    \definition{v.}{retomar; renovar; restaurar; voltar a | reviver; recuperar; reencontrar | restaurar; restabelecer; reabilitar; regenerar; ressurgir; restabelecer alguém em; recuperar o que foi perdido}
  \end{Phonetics}
\end{Entry}

\begin{Entry}{恤}{9}{⼼}
  \begin{Phonetics}{恤}{xu4}
    \definition{v.}{ter pena; simpatizar | dar alívio; compensar}
  \end{Phonetics}
\end{Entry}

\begin{Entry}{恨}{9}{⼼}
  \begin{Phonetics}{恨}{hen4}[][HSK 5]
    \definition{s.}{ódio; resentimento}
    \definition{v.}{odiar; ressentir-se}
  \end{Phonetics}
\end{Entry}

\begin{Entry}{恰}{9}{⼼}
  \begin{Phonetics}{恰}{qia4}
    \definition{adv.}{exatamente | apenas}
  \end{Phonetics}
\end{Entry}

\begin{Entry}{恰好}{9,6}{⼼、⼥}
  \begin{Phonetics}{恰好}{qia4 hao3}[][HSK 6]
    \definition{adv.}{na medida certa; como a sorte quis}
  \end{Phonetics}
\end{Entry}

\begin{Entry}{恰当}{9,6}{⼼、⼹}
  \begin{Phonetics}{恰当}{qia4dang4}[][HSK 6]
    \definition{adj.}{adequado; apropriado; conveniente; apropriado; a linguagem ou abordagem é muito apropriada}
  \end{Phonetics}
\end{Entry}

\begin{Entry}{恰到好处}{9,8,6,5}{⼼、⼑、⼥、⼡}
  \begin{Phonetics}{恰到好处}{qia4dao4hao3chu4}
    \definition{expr.}{é simplesmente perfeito | é simplesmente correto}
  \end{Phonetics}
\end{Entry}

\begin{Entry}{恰恰}{9,9}{⼼、⼼}
  \begin{Phonetics}{恰恰}{qia4 qia4}[][HSK 6]
    \definition{adv.}{justamente; exatamente; precisamente; bem na hora}
  \end{Phonetics}
\end{Entry}

\begin{Entry}{战}{9}{⼽}
  \begin{Phonetics}{战}{zhan4}
    \definition{s.}{luta | guerra | batalha}
    \definition{v.}{lutar}
  \end{Phonetics}
\end{Entry}

\begin{Entry}{战士}{9,3}{⼽、⼠}
  \begin{Phonetics}{战士}{zhan4shi4}[][HSK 4]
    \definition[个,些,名,位]{s.}{soldado; membros mais jovens do exército | campeão; guerreiro; lutador; geralmente, uma pessoa que se engaja em alguma causa justa ou participa de alguma luta justa}
  \end{Phonetics}
\end{Entry}

\begin{Entry}{战友}{9,4}{⼽、⼜}
  \begin{Phonetics}{战友}{zhan4 you3}[][HSK 6]
    \definition{s.}{camarada de armas; companheiro de guerra | companheiro de batalha; pessoas lutando juntas}
  \end{Phonetics}
\end{Entry}

\begin{Entry}{战斗}{9,4}{⼽、⽃}
  \begin{Phonetics}{战斗}{zhan4dou4}[][HSK 4]
    \definition[场,次]{s.}{luta; batalha; combate; ação; conflito armado entre as partes oponentes}
    \definition{v.}{lutar; combater | lutar; metáfora para trabalho duro ou labor}
  \end{Phonetics}
\end{Entry}

\begin{Entry}{战术}{9,5}{⼽、⽊}
  \begin{Phonetics}{战术}{zhan4shu4}[][HSK 6]
    \definition[种,套]{s.}{táticas para resolver problemas; geralmente se refere ao método de resolução de problemas locais | táticas (militares); princípios e métodos de condução de combate}
  \end{Phonetics}
\end{Entry}

\begin{Entry}{战争}{9,6}{⼽、⼑}
  \begin{Phonetics}{战争}{zhan4zheng1}[][HSK 4]
    \definition[场,次]{s.}{guerra; conflito; luta armada entre povos, entre nações, entre classes ou entre grupos políticos}
  \end{Phonetics}
\end{Entry}

\begin{Entry}{战场}{9,6}{⼽、⼟}
  \begin{Phonetics}{战场}{zhan4 chang3}[][HSK 6]
    \definition[个,片,处]{s.}{campo de batalha; frente de batalha}
  \end{Phonetics}
\end{Entry}

\begin{Entry}{战胜}{9,9}{⼽、⾁}
  \begin{Phonetics}{战胜}{zhan4 sheng4}[][HSK 4]
    \definition{v.}{derrotar; vencer; superar; triunfar sobre; metáfora para superar dificuldades e alcançar o sucesso}
  \end{Phonetics}
\end{Entry}

\begin{Entry}{战略}{9,11}{⼽、⽥}
  \begin{Phonetics}{战略}{zhan4lve4}[][HSK 6]
    \definition[个,条]{s.}{estratégia; uma estratégia que orienta todo o processo de guerra (diferente de 战术) | estratégia; refere-se a uma diretriz geral}
  \seealsoref{战术}{zhan4shu4}
  \end{Phonetics}
\end{Entry}

\begin{Entry}{扁}{9}{⼾}
  \begin{Phonetics}{扁}{bian3}[][HSK 6]
    \definition{adj.}{plano}
    \definition{v.}{(coloquial)  bater em alguém}
  \end{Phonetics}
  \begin{Phonetics}{扁}{pian1}
    \definition{adj.}{pequeno | fora do caminho; remoto}
  \end{Phonetics}
\end{Entry}

\begin{Entry}{扁舟}{9,6}{⼾、⾈}
  \begin{Phonetics}{扁舟}{pian1 zhou1}
    \definition[叶,艘]{s.}{pequeno barco; esquife}
  \end{Phonetics}
\end{Entry}

\begin{Entry}{拜}{9}{⼿}
  \begin{Phonetics}{拜}{bai4}
    \definition*{s.}{Sobrenome Bai}
    \definition{adv.}{respeitosamente (usado na comunicação interpessoal);}
    \definition{v.}{fazer uma visita de cortesia | adorar; prestar homenagem | fazer uma chamada cerimonial | ligar; fazer uma visita | intitular alguém com cerimônia; conceder uma posição oficial ou um determinado título com certa etiqueta | estabelecer ou jurar formalmente relacionamentos}
  \end{Phonetics}
\end{Entry}

\begin{Entry}{拜访}{9,6}{⼿、⾔}
  \begin{Phonetics}{拜访}{bai4fang3}[][HSK 5]
    \definition{v.}{visitar; fazer uma visita (respeitosamente)}
  \end{Phonetics}
\end{Entry}

\begin{Entry}{括}{9}{⼿}
  \begin{Phonetics}{括}{kuo4}
    \definition{v.}{unir (músculos, etc.); contrair | incluir | adicionar colchetes a | amarrar; empacotar}
  \end{Phonetics}
\end{Entry}

\begin{Entry}{括号}{9,5}{⼿、⼝}
  \begin{Phonetics}{括号}{kuo4 hao4}[][HSK 4]
    \definition{s.}{chaves, colchetes e parênteses (em fórmulas aritméticas ou algébricas, os símbolos que indicam a combinação e a ordem de vários números ou termos) | colchetes e parênteses usados como um tipo de sinal de pontuação para mostrar a parte explicativa de uma passagem em um texto}
  \end{Phonetics}
\end{Entry}

\begin{Entry}{拮}{9}{⼿}
  \begin{Phonetics}{拮}{jie2}
    \definition{adj.}{trabalhoso | sem dinheiro | antagônico | trabalhando duro | pressionado}
  \end{Phonetics}
\end{Entry}

\begin{Entry}{拮据}{9,11}{⼿、⼿}
  \begin{Phonetics}{拮据}{jie2ju1}
    \definition{adj.}{em circunstâncias difíceis; sem dinheiro; em dificuldades}
  \end{Phonetics}
\end{Entry}

\begin{Entry}{拼}{9}{⼿}
  \begin{Phonetics}{拼}{pin1}[][HSK 5]
    \definition{v.}{montar; juntar as peças | dar tudo de si no trabalho; estar disposto a arriscar a vida (em lutas, no trabalho, etc.); fazer tudo o que for preciso; arriscar tudo}
  \end{Phonetics}
\end{Entry}

\begin{Entry}{拼命}{9,8}{⼿、⼝}
  \begin{Phonetics}{拼命}{pin1ming4}
    \definition{adv.}{com toda a força | desesperadamente}
    \definition{v.+compl.}{arriscar a vida de alguém | desafiar a morte | colocar-se em uma luta desesperada | fazer algo desesperadamente | exercer a maior força}
  \end{Phonetics}
\end{Entry}

\begin{Entry}{拼音}{9,9}{⼿、⾳}
  \begin{Phonetics}{拼音}{pin1yin1}
    \definition{s.}{escrita fonética | pinyin (romanização chinesa)}
  \end{Phonetics}
\end{Entry}

\begin{Entry}{拾}{9}{⼿}
  \begin{Phonetics}{拾}{shi2}[][HSK 5]
    \definition{num.}{dez (usado no lugar do numeral 十 em cheques, notas bancárias, etc., para evitar erros ou alterações)}
    \definition{v.}{pegar (do chão); recolher}
  \end{Phonetics}
\end{Entry}

\begin{Entry}{持}{9}{⼿}
  \begin{Phonetics}{持}{chi2}
    \definition{v.}{segurar; agarrar | opor; confrontar | apoiar; manter | gerenciar; supervisionar | sequestrar; agarrar (controlar; forçar)}
  \end{Phonetics}
\end{Entry}

\begin{Entry}{持有}{9,6}{⼿、⽉}
  \begin{Phonetics}{持有}{chi2 you3}[][HSK 6]
    \definition{v.}{segurar; possuir | segurar; ter; abrigar; ter em mente (ideias, opiniões, etc.)}
  \end{Phonetics}
\end{Entry}

\begin{Entry}{持续}{9,11}{⼿、⽷}
  \begin{Phonetics}{持续}{chi2xu4}[][HSK 3]
    \definition{v.}{durar; continuar; sustentar; manter a situação ou as condições como estão, sem alterações}
  \end{Phonetics}
\end{Entry}

\begin{Entry}{挂}{9}{⼿}
  \begin{Phonetics}{挂}{gua4}[][HSK 3]
    \definition{clas.}{usado principalmente para coisas que vêm em conjuntos ou séries}
    \definition{v.}{pendurar; colocar; suspender; usando cordas, ganchos, pregos e outros itens para prender objetos em um ou mais pontos específicos | interromper chamada (telefônica) | colocar alguém em contato com; ligar; telefonar; refere-se a ligar o telefone, bem como a fazer uma chamada | falhar; fracassar | colocar em registro; registrarpegar carona; ser pego | preocupar-se com | ser revestido com; ser coberto com | estar pendente; deixar algo sem solução}
  \end{Phonetics}
\end{Entry}

\begin{Entry}{挂号}{9,5}{⼿、⼝}
  \begin{Phonetics}{挂号}{gua4hao4}
    \definition{v.+compl.}{registrar-se (em um hospital, etc.) | enviar através de carta registrada}
  \end{Phonetics}
\end{Entry}

\begin{Entry}{挂号信}{9,5,9}{⼿、⼝、⼈}
  \begin{Phonetics}{挂号信}{gua4hao4xin4}
    \definition{s.}{carta registrada}
  \end{Phonetics}
\end{Entry}

\begin{Entry}{指}{9}{⼿}
  \begin{Phonetics}{指}{zhi3}[][HSK 3]
    \definition*{s.}{Sobrenome Zhi}
    \definition{clas.}{dígito; largura do dedo; a largura de um dedo é chamada de 一指, que é usado para medir profundidade, largura, etc.}
    \definition{s.}{dedo}
    \definition{v.}{apontar para; indicar; usar o dedo ou a ponta de um objeto para apontar | (pelo) eriçar;  (cabelo) ficar em pé | indicar; mostrar; apontar; demonstrar | referir-se a; dirigir-se a | confiar em; contar com; depender de | criticar; repreender}
  \end{Phonetics}
\end{Entry}

\begin{Entry}{指出}{9,5}{⼿、⼐}
  \begin{Phonetics}{指出}{zhi3 chu1}[][HSK 3]
    \definition{v.}{apontar; indicar}
  \end{Phonetics}
\end{Entry}

\begin{Entry}{指头}{9,5}{⼿、⼤}
  \begin{Phonetics}{指头}{zhi3 tou5}[][HSK 6]
    \definition[个,根,只]{s.}{dedo da mão ou do pé}
  \end{Phonetics}
\end{Entry}

\begin{Entry}{指甲}{9,5}{⼿、⽥}
  \begin{Phonetics}{指甲}{zhi3jia5}[][HSK 5]
    \definition[个,种]{s.}{unha; unha de agulha; unha de dedo; camada córnea na ponta dos dedos}
  \end{Phonetics}
\end{Entry}

\begin{Entry}{指示}{9,5}{⼿、⽰}
  \begin{Phonetics}{指示}{zhi3shi4}[][HSK 5]
    \definition[点,条,项,个]{s.}{diretriz; instruções; para orientar o trabalho, os superiores emitem opiniões verbais ou escritas aos subordinados}
    \definition{v.}{indicar; apontar; apontar para alguém | instruir; superiores emitem opiniões verbais ou escritas para orientar o trabalho dos subordinados}
  \end{Phonetics}
\end{Entry}

\begin{Entry}{指导}{9,6}{⼿、⼨}
  \begin{Phonetics}{指导}{zhi3dao3}[][HSK 3]
    \definition[位]{s.}{guia; diretor; pessoa que dá orientações}
    \definition{v.}{orientar; dirigir; instruir}
  \end{Phonetics}
\end{Entry}

\begin{Entry}{指定}{9,8}{⼿、⼧}
  \begin{Phonetics}{指定}{zhi3ding4}[][HSK 6]
    \definition{adv.}{certamente; com certeza; reforça o tom de palpite e estimativa}
    \definition{v.}{nomear; atribuir; determinar a pessoa, evento, lugar, conteúdo, etc. que faz algo}
  \end{Phonetics}
\end{Entry}

\begin{Entry}{指责}{9,8}{⼿、⾙}
  \begin{Phonetics}{指责}{zhi3ze2}[][HSK 5]
    \definition{v.}{censurar; criticar; encontrar falhas; repreender}
  \end{Phonetics}
\end{Entry}

\begin{Entry}{指南针}{9,9,7}{⼿、⼗、⾦}
  \begin{Phonetics}{指南针}{zhi3nan2zhen1}
    \definition{s.}{bússola}
  \end{Phonetics}
\end{Entry}

\begin{Entry}{指挥}{9,9}{⼿、⼿}
  \begin{Phonetics}{指挥}{zhi3hui1}[][HSK 4]
    \definition[个,位,名]{s.}{diretor; comandante; despachante; operador | maestro; condutor; pessoa na frente de uma orquestra ou coro que dá instruções sobre como tocar ou cantar}
    \definition{v.}{dirigir; conduzir; comandar; direcionar; emitir ordens de agendamento}
  \end{Phonetics}
\end{Entry}

\begin{Entry}{指标}{9,9}{⼿、⽊}
  \begin{Phonetics}{指标}{zhi3biao1}[][HSK 5]
    \definition[个,种]{s.}{meta; cota; norma; índice; objetivos a serem alcançados | alvo; índice; refletir os requisitos de desenvolvimento em determinados aspectos através de números absolutos ou percentagens de aumento ou diminuição, inclui indicadores quantitativos e qualitativos}
  \end{Phonetics}
\end{Entry}

\begin{Entry}{指着}{9,11}{⼿、⽬}
  \begin{Phonetics}{指着}{zhi3 zhe5}[][HSK 6]
    \definition{v.}{apontar}
  \end{Phonetics}
\end{Entry}

\begin{Entry}{指数}{9,13}{⼿、⽁}
  \begin{Phonetics}{指数}{zhi3 shu4}[][HSK 6]
    \definition{s.}{Matemática: expoente; refere-se ao número de vezes que um número é multiplicado por si mesmo, o que é registrado no canto superior direito do número | Estatística: índice, refere-se à razão entre o valor de um fenômeno econômico em um determinado período e o valor de outro período usado como padrão de comparação, geralmente expresso como uma porcentagem, como o índice de preços ao consumidor}
  \end{Phonetics}
\end{Entry}

\begin{Entry}{按}{9}{⼿}
  \begin{Phonetics}{按}{an4}[][HSK 3]
    \definition{prep.}{de acordo com; à luz de; com base em; em conformidade com}
    \definition{v.}{pressionar; empurrar para baixo; pressionar ou apertar com a mão ou os dedos | pôr de parte; deixar de lado; deixar para mais tarde | restringir; controlar; inibir; parar | verificar; consultar | comentar ou anotar (por um editor ou autor)}
  \end{Phonetics}
\end{Entry}

\begin{Entry}{按时}{9,7}{⼿、⽇}
  \begin{Phonetics}{按时}{an4shi2}[][HSK 4]
    \definition{adv.}{na hora; no horário; pontualmente; de acordo com o tempo estipulado}
  \end{Phonetics}
\end{Entry}

\begin{Entry}{按照}{9,13}{⼿、⽕}
  \begin{Phonetics}{按照}{an4zhao4}[][HSK 3]
    \definition{prep.}{de acordo com; em conformidade com; à luz de; com base em; apresentar os fundamentos ou critérios de julgamento para fazer algo}
  \end{Phonetics}
\end{Entry}

\begin{Entry}{按摩}{9,15}{⼿、⼿}
  \begin{Phonetics}{按摩}{an4mo2}[][HSK 5]
    \definition{s.}{massagem; empurrar, pressionar, beliscar e amassar o corpo de uma pessoa com as mãos para promover a circulação sanguínea, aumentar a resistência da pele e regular a função dos nervos}
  \end{Phonetics}
\end{Entry}

\begin{Entry}{挑}{9}{⼿}
  \begin{Phonetics}{挑}{tiao1}[][HSK 4]
    \definition{clas.}{usado para coisas que são escolhidas ou selecionadas | usado para coisas que podem ser usadas como palhetas}
    \definition{s.}{vara comprida com algo pendurado nas pontas; haste de transporte}
    \definition{v.}{escolher; selecionar | fazer picuinhas; ser hipercrítico; ser meticuloso; ser excessivamente rigoroso nos detalhes | carregar com uma haste de transporte; carregar no ombro; pendurar coisas nas pontas de varas longas e carregá-las em seus ombros}
  \end{Phonetics}
  \begin{Phonetics}{挑}{tiao3}[][HSK 4]
    \definition{s.}{um dos traços dos caracteres chineses; inclinado para cima da esquerda para a direita}
    \definition{v.}{levantar; elevar; erguer | levantar ou apoiar com uma extremidade de uma vara ou objeto semelhante; segurar ou apoiar com a ponta de uma vara etc. | causar conflitos deliberadamente; provocar deliberadamente um conflito | (método de bordado) usar uma agulha para levantar os fios de urdidura ou trama, com a agulha e a linha passando por baixo para formar padrões e desenhos}
  \end{Phonetics}
\end{Entry}

\begin{Entry}{挑战}{9,9}{⼿、⼽}
  \begin{Phonetics}{挑战}{tiao3zhan4}[][HSK 4]
    \definition{v.}{desafiar; deixar um oponente deliberadamente irritado e sair para lutar ou lutar consigo mesmo; estimular um oponente a lutar consigo mesmo}
  \end{Phonetics}
\end{Entry}

\begin{Entry}{挑选}{9,9}{⼿、⾡}
  \begin{Phonetics}{挑选}{tiao1 xuan3}[][HSK 4]
    \definition{v.}{escolher; optar; selecionar; escolher a pessoa ou coisa certa para o trabalho}
  \end{Phonetics}
\end{Entry}

\begin{Entry}{挑衅}{9,11}{⼿、⾎}
  \begin{Phonetics}{挑衅}{tiao3xin4}
    \definition{s.}{provocação}
    \definition{v.}{provocar; causar problemas; tentar causar conflito ou guerra}
  \end{Phonetics}
\end{Entry}

\begin{Entry}{挖}{9}{⼿}
  \begin{Phonetics}{挖}{wa1}[][HSK 6]
    \definition{v.}{cavar; escavar; arrancar | explorar; sondar | (dialeto) arranhar | escavar a superfície de um objeto com ferramentas ou mãos}
  \end{Phonetics}
\end{Entry}

\begin{Entry}{挖掘机}{9,11,6}{⼿、⼿、⽊}
  \begin{Phonetics}{挖掘机}{wa1jue2ji1}
    \definition{s.}{escavadeira | escavador | escavadora | pá mecânica}
  \end{Phonetics}
\end{Entry}

\begin{Entry}{挡}{9}{⼿}
  \begin{Phonetics}{挡}{dang3}[][HSK 5]
    \definition{s.}{persiana; veneziana; paralama; coisas para cobrir ou bloquear | caixa de câmbio (automóvel)}
    \definition{v.}{bloquear; resistir; manter afastado; afastar | cobrir; bloquear; atrapalhar}
  \end{Phonetics}
  \begin{Phonetics}{挡}{dang4}
    \definition{v.}{organizar}
  \end{Phonetics}
\end{Entry}

\begin{Entry}{挡风玻璃}{9,4,9,14}{⼿、⾵、⽟、⽟}
  \begin{Phonetics}{挡风玻璃}{dang3feng1bo1li5}
    \definition{s.}{parabrisa}
  \end{Phonetics}
\end{Entry}

\begin{Entry}{挣}{9}{⼿}
  \begin{Phonetics}{挣}{zheng4}[][HSK 5]
    \definition{v.}{empurrar e puxar; tentar se livrar; lutar para se libertar; esforçar-se para se libertar das amarras | ganhar; fazer; trabalhar para; trocar trabalho por}
  \end{Phonetics}
\end{Entry}

\begin{Entry}{挣扎}{9,4}{⼿、⼿}
  \begin{Phonetics}{挣扎}{zheng1zha2}
    \definition{v.}{lutar}
  \end{Phonetics}
\end{Entry}

\begin{Entry}{挣钱}{9,10}{⼿、⾦}
  \begin{Phonetics}{挣钱}{zheng4qian2}[][HSK 5]
    \definition{v.+compl.}{ganhar dinheiro; fazer dinheiro; lucrar; trabalhar para ganhar dinheiro}
  \end{Phonetics}
\end{Entry}

\begin{Entry}{挣得}{9,11}{⼿、⼻}
  \begin{Phonetics}{挣得}{zheng4de2}
    \definition{v.}{ganhar renda ou dinheiro}
  \end{Phonetics}
\end{Entry}

\begin{Entry}{挤}{9}{⼿}
  \begin{Phonetics}{挤}{ji3}[][HSK 5]
    \definition{adj.}{lotado; congestionado; descreve um grande número de pessoas ou coisas e muito pouco espaço}
    \definition{v.}{empacotar; amontoar; aglomerar | sacudir; empurrar contra; empurrar alguém ou algo para longe com seu corpo com toda a força que puder| pressionar; apertar; expulsar por pressão}
  \end{Phonetics}
\end{Entry}

\begin{Entry}{挥}{9}{⼿}
  \begin{Phonetics}{挥}{hui1}
    \definition{v.}{acenar; empunhar; socar | limpar lágrimas, suor, etc. com as mãos | comandar (um exército) | espalhar; dispersar | afastar-se; livrar-se de}
  \end{Phonetics}
\end{Entry}

\begin{Entry}{挥汗如雨}{9,6,6,8}{⼿、⽔、⼥、⾬}
  \begin{Phonetics}{挥汗如雨}{hui1han4ru2yu3}
    \definition{s.}{suor derramado}
    \definition{v.}{pingar com suor}
  \end{Phonetics}
\end{Entry}

\begin{Entry}{挺}{9}{⼿}
  \begin{Phonetics}{挺}{ting3}[][HSK 2,4]
    \definition{adj.}{rígido; ereto; vertical; reto | notável; destacado; distinto}
    \definition{adv.}{muito; bastante}
    \definition{clas.}{usado para metralhadoras}
    \definition{v.}{sobressair; endireitar-se; protrudir (protuberância ou saliência) | suportar; aguentar; resistir; perseverar}
  \end{Phonetics}
\end{Entry}

\begin{Entry}{挺尸}{9,3}{⼿、⼫}
  \begin{Phonetics}{挺尸}{ting3shi1}
    \definition{v.}{(coloquial) dormir | (literalmente) ficar deitado duro como um cadáver}
  \end{Phonetics}
\end{Entry}

\begin{Entry}{挺立}{9,5}{⼿、⽴}
  \begin{Phonetics}{挺立}{ting3li4}
    \definition{v.}{ficar ereto | ficar de pé}
  \end{Phonetics}
\end{Entry}

\begin{Entry}{挺好}{9,6}{⼿、⼥}
  \begin{Phonetics}{挺好}{ting3 hao3}[][HSK 2]
    \definition{adj.}{nada mal; surpreendentemente bom}
  \end{Phonetics}
\end{Entry}

\begin{Entry}{挺过}{9,6}{⼿、⾡}
  \begin{Phonetics}{挺过}{ting3guo4}
    \definition{s.}{sobreviver}
  \end{Phonetics}
\end{Entry}

\begin{Entry}{挺住}{9,7}{⼿、⼈}
  \begin{Phonetics}{挺住}{ting3zhu4}
    \definition{v.}{permanecer firme | manter-se firme (diante da adversidade ou da dor)}
  \end{Phonetics}
\end{Entry}

\begin{Entry}{挺杆}{9,7}{⼿、⽊}
  \begin{Phonetics}{挺杆}{ting3gan3}
    \definition{s.}{tucho (peça de máquina)}
  \end{Phonetics}
\end{Entry}

\begin{Entry}{挺身}{9,7}{⼿、⾝}
  \begin{Phonetics}{挺身}{ting3shen1}
    \definition{v.}{endireitar as costas}
  \end{Phonetics}
\end{Entry}

\begin{Entry}{挺进}{9,7}{⼿、⾡}
  \begin{Phonetics}{挺进}{ting3jin4}
    \definition{s.}{progresso | avanço}
    \definition{v.}{progredir | avançar}
  \end{Phonetics}
\end{Entry}

\begin{Entry}{挺拔}{9,8}{⼿、⼿}
  \begin{Phonetics}{挺拔}{ting3ba2}
    \definition{adj.}{alto e reto}
  \end{Phonetics}
\end{Entry}

\begin{Entry}{挺腰}{9,13}{⼿、⾁}
  \begin{Phonetics}{挺腰}{ting3yao1}
    \definition{v.}{arquear as costas | endireitar as costas}
  \end{Phonetics}
\end{Entry}

\begin{Entry}{政}{9}{⽁}
  \begin{Phonetics}{政}{zheng4}
    \definition*{s.}{Sobrenome Zheng}
    \definition{s.}{política; assuntos políticos | certos aspectos administrativos do governo | assuntos de uma família ou de uma organização; refere-se a assuntos familiares ou de grupo}
  \end{Phonetics}
\end{Entry}

\begin{Entry}{政权}{9,6}{⽁、⽊}
  \begin{Phonetics}{政权}{zheng4quan2}[][HSK 6]
    \definition{s.}{poder político ou estatal; regime}
  \end{Phonetics}
\end{Entry}

\begin{Entry}{政纲}{9,7}{⽁、⽷}
  \begin{Phonetics}{政纲}{zheng4gang1}
    \definition{s.}{programa ou plataforma política}
  \end{Phonetics}
\end{Entry}

\begin{Entry}{政府}{9,8}{⽁、⼴}
  \begin{Phonetics}{政府}{zheng4fu3}[][HSK 4]
    \definition{s.}{governo;  órgãos executivos do poder do Estado, ou seja, órgãos administrativos do Estado, como o Conselho de Estado (Governo Popular Central) e os governos populares locais em todos os níveis na China}
  \end{Phonetics}
\end{Entry}

\begin{Entry}{政治}{9,8}{⽁、⽔}
  \begin{Phonetics}{政治}{zheng4zhi4}[][HSK 4]
    \definition{s.}{política; assuntos políticos; questões políticas; as atividades de governos, partidos políticos, grupos sociais e indivíduos em assuntos internos e relações internacionais}
  \end{Phonetics}
\end{Entry}

\begin{Entry}{政治局}{9,8,7}{⽁、⽔、⼫}
  \begin{Phonetics}{政治局}{zheng4zhi4ju2}
    \definition{s.}{o principal comitê de políticas de um partido comunista}
  \end{Phonetics}
\end{Entry}

\begin{Entry}{政党}{9,10}{⽁、⼉}
  \begin{Phonetics}{政党}{zheng4 dang3}[][HSK 6]
    \definition[个,些]{s.}{partido político; uma organização política que representa um determinado estágio, classe ou grupo e luta para concretizar seus interesses}
  \end{Phonetics}
\end{Entry}

\begin{Entry}{政策}{9,12}{⽁、⽵}
  \begin{Phonetics}{政策}{zheng4ce4}[][HSK 6]
    \definition[项,条,个]{s.}{política; um código de conduta formulado por um país ou partido político para alcançar sua política em um determinado período histórico}
  \end{Phonetics}
\end{Entry}

\begin{Entry}{故}{9}{⽁}
  \begin{Phonetics}{故}{gu4}
    \definition*{s.}{Sobrenome Gu}
    \definition{adj.}{velho; antigo}
    \definition{adv.}{propositalmente; intencionalmente; deliberadamente}
    \definition{conj.}{assim; portanto; consequentemente; pelo contrário}
    \definition{s.}{evento; incidente; acontecimento; acidente | causa; razão | amigo; conhecido | o velho; refere-se a coisas antigas e passadas}
    \definition{v.}{morrer}
  \end{Phonetics}
\end{Entry}

\begin{Entry}{故乡}{9,3}{⽁、⼄}
  \begin{Phonetics}{故乡}{gu4xiang1}[][HSK 3]
    \definition[个]{s.}{cidade natal; terra natal; local de nascimento ou onde viveu por muito tempo}
  \end{Phonetics}
\end{Entry}

\begin{Entry}{故事}{9,8}{⽁、⼅}
  \begin{Phonetics}{故事}{gu4shi5}[][HSK 2]
    \definition[个,段,篇,则]{s.}{história; conto; coisas reais ou fictícias usadas como objeto de narrativa, com coerência, atraentes e capazes de emocionar as pessoas | enredo; trama; enredo que consegue mostrar a personalidade dos personagens e refletir a ideia central da obra literária}
  \end{Phonetics}
\end{Entry}

\begin{Entry}{故宫}{9,9}{⽁、⼧}
  \begin{Phonetics}{故宫}{gu4gong1}
    \definition*{s.}{Palácio Imperial | Cidade Proibida}
  \end{Phonetics}
\end{Entry}

\begin{Entry}{故意}{9,13}{⽁、⼼}
  \begin{Phonetics}{故意}{gu4yi4}[][HSK 2]
    \definition{adv.}{deliberadamente; intencionalmente; não é por descuido, mas sim conscientemente (geralmente coisas que não se devem fazer ou que não são necessárias)}
    \definition{s.}{intenção; um tipo de mentalidade, uma pessoa sabe claramente que seus atos podem causar danos a outras pessoas ou trazer consequências negativas para a sociedade, mas mesmo assim não faz nada para impedir isso}
  \end{Phonetics}
\end{Entry}

\begin{Entry}{故障}{9,13}{⽁、⾩}
  \begin{Phonetics}{故障}{gu4zhang4}[][HSK 6]
    \definition[出]{s.}{problema; falha; parada; mau funcionamento; avaria; situações em que máquinas, instrumentos, etc. não podem funcionar normalmente devido a problemas}
  \end{Phonetics}
\end{Entry}

\begin{Entry}{既}{9}{⽆}
  \begin{Phonetics}{既}{ji4}[][HSK 4]
    \definition*{s.}{Sobrenome Ji}
    \definition{adv.}{já}
    \definition{conj.}{desde; como; agora que | assim como; e também; ambos\dots e\dots; usado em conjunto com advérbios como 且, 又, 也 para indicar uma combinação de ambas as situações}
  \seealsoref{且}{qie3}
  \seealsoref{也}{ye3}
  \seealsoref{又}{you4}
  \end{Phonetics}
\end{Entry}

\begin{Entry}{既又}{9,2}{⽆、⼜}
  \begin{Phonetics}{既又}{ji4you4}
    \definition{conj.}{desde | como | agora isso | os dois e | assim como}
  \end{Phonetics}
\end{Entry}

\begin{Entry}{既不……又不……}{9,4,2,4}{⽆、⼀、⼜、⼀}
  \begin{Phonetics}{既不……又不……}{ji4bu4 you4bu4}
    \definition{conj.}{nem\dots nem\dots}
  \end{Phonetics}
\end{Entry}

\begin{Entry}{既然}{9,12}{⽆、⽕}
  \begin{Phonetics}{既然}{ji4ran2}[][HSK 4]
    \definition{conj.}{como; desde; agora que; usado na primeira metade de uma frase, muitas vezes repetido na segunda metade pelos advérbios 就, 也, 还 para indicar que a premissa é primeiro declarada e depois inferida}
  \seealsoref{还}{hai2}
  \seealsoref{就}{jiu4}
  \seealsoref{也}{ye3}
  \end{Phonetics}
\end{Entry}

\begin{Entry}{星}{9}{⽇}
  \begin{Phonetics}{星}{xing1}
    \definition*{s.}{Xing, a vigésima quinta das vinte e oito constelações em que a esfera celeste era dividida na antiga astronomia chinesa, consistindo em sete estrelas em Hydra}
    \definition[颗]{s.}{estrela | (astronomia) corpo celeste | partícula | pequenas marcas no braço de uma balança romana indicando jin e suas frações | artista famoso (estrela de cinema, estrela de jogos de bola, etc.) | satélite (artificial) | pequena quantidade}
  \end{Phonetics}
\end{Entry}

\begin{Entry}{星火}{9,4}{⽇、⽕}
  \begin{Phonetics}{星火}{xing1huo3}
    \definition{s.}{trilha de meteoro (usada principalmente em expressões como 急如星火) | faísca}
  \end{Phonetics}
\end{Entry}

\begin{Entry}{星辰}{9,7}{⽇、⾠}
  \begin{Phonetics}{星辰}{xing1chen2}
    \definition{s.}{estrelas}
  \end{Phonetics}
\end{Entry}

\begin{Entry}{星表}{9,8}{⽇、⾐}
  \begin{Phonetics}{星表}{xing1biao3}
    \definition{s.}{catálogo de estrelas}
  \end{Phonetics}
\end{Entry}

\begin{Entry}{星星}{9,9}{⽇、⽇}
  \begin{Phonetics}{星星}{xing1 xing5}[][HSK 2]
    \definition[颗,群,片]{s.}{estrela; em astronomia, refere-se aos corpos celestes luminosos no universo, como as estrelas que brilham no céu noturno | estrela; uma metáfora para alguém ou algo que se destaca em um determinado campo e atrai atenção | objetos em forma de estrela}
  \end{Phonetics}
\end{Entry}

\begin{Entry}{星座}{9,10}{⽇、⼴}
  \begin{Phonetics}{星座}{xing1zuo4}
    \definition[张]{s.}{signo astrológico | constelação}
  \end{Phonetics}
\end{Entry}

\begin{Entry}{星期}{9,12}{⽇、⽉}
  \begin{Phonetics}{星期}{xing1qi1}[][HSK 1]
    \definition[个]{s.}{semana | dias da semana; usado em conjunto com 日, 一, 二, 三, 四, 五, 六, 天, indica um determinado dia da semana | abreviação de domingo}
  \seealsoref{星期二}{xing1 qi1 er4}
  \seealsoref{星期六}{xing1 qi1 liu4}
  \seealsoref{星期日}{xing1 qi1 ri4}
  \seealsoref{星期三}{xing1 qi1 san1}
  \seealsoref{星期四}{xing1 qi1 si4}
  \seealsoref{星期天}{xing1 qi1 tian1}
  \seealsoref{星期五}{xing1 qi1 wu3}
  \seealsoref{星期一}{xing1 qi1 yi1}
  \end{Phonetics}
\end{Entry}

\begin{Entry}{星期一}{9,12,1}{⽇、⽉、⼀}
  \begin{Phonetics}{星期一}{xing1 qi1 yi1}[][HSK 1]
    \definition{s.}{segunda-feira}
  \end{Phonetics}
\end{Entry}

\begin{Entry}{星期二}{9,12,2}{⽇、⽉、⼆}
  \begin{Phonetics}{星期二}{xing1 qi1 er4}[][HSK 1]
    \definition{s.}{terça-feira}
  \end{Phonetics}
\end{Entry}

\begin{Entry}{星期三}{9,12,3}{⽇、⽉、⼀}
  \begin{Phonetics}{星期三}{xing1 qi1 san1}[][HSK 1]
    \definition{s.}{quarta-feira}
  \end{Phonetics}
\end{Entry}

\begin{Entry}{星期五}{9,12,4}{⽇、⽉、⼆}
  \begin{Phonetics}{星期五}{xing1 qi1 wu3}[][HSK 1]
    \definition{s.}{sexta-feira}
  \end{Phonetics}
\end{Entry}

\begin{Entry}{星期六}{9,12,4}{⽇、⽉、⼋}
  \begin{Phonetics}{星期六}{xing1 qi1 liu4}[][HSK 1]
    \definition{s.}{sábado}
  \end{Phonetics}
\end{Entry}

\begin{Entry}{星期天}{9,12,4}{⽇、⽉、⼤}
  \begin{Phonetics}{星期天}{xing1 qi1 tian1}[][HSK 1]
    \definition{s.}{domingo}
  \seealsoref{星期日}{xing1 qi1 ri4}
  \end{Phonetics}
\end{Entry}

\begin{Entry}{星期日}{9,12,4}{⽇、⽉、⽇}
  \begin{Phonetics}{星期日}{xing1 qi1 ri4}[][HSK 1]
    \definition{s.}{domingo}
  \seealsoref{星期天}{xing1 qi1 tian1}
  \end{Phonetics}
\end{Entry}

\begin{Entry}{星期四}{9,12,5}{⽇、⽉、⼞}
  \begin{Phonetics}{星期四}{xing1 qi1 si4}[][HSK 1]
    \definition{s.}{quinta-feira}
  \end{Phonetics}
\end{Entry}

\begin{Entry}{春}{9}{⽇}
  \begin{Phonetics}{春}{chun1}
    \definition*{s.}{Sobrenome Chun}
    \definition{s.}{primavera | amor; luxúria | vida; vitalidade}
  \end{Phonetics}
\end{Entry}

\begin{Entry}{春天}{9,4}{⽇、⼤}
  \begin{Phonetics}{春天}{chun1 tian1}
    \definition[个,段,季,番]{s.}{primavera; época da primavera | primavera; renascimento; uma atmosfera cheia de energia e esperança}
  \end{Phonetics}
\end{Entry}

\begin{Entry}{春节}{9,5}{⽇、⾋}
  \begin{Phonetics}{春节}{chun1 jie2}[][HSK 2]
    \definition*[个]{s.}{Festival da Primavera (Ano Novo Chinês); o primeiro dia do primeiro mês do calendário lunar, também se refere aos dias seguintes ao primeiro dia do primeiro mês}
  \end{Phonetics}
\end{Entry}

\begin{Entry}{春季}{9,8}{⽇、⼦}
  \begin{Phonetics}{春季}{chun1 ji4}[][HSK 4]
    \definition{s.}{primavera; primeiro trimestre do ano, que na China se refere ao período de três meses entre o início da primavera e o início do verão, e também se refere aos três meses do calendário lunar, a saber, o primeiro, o segundo e o terceiro meses}
  \end{Phonetics}
\end{Entry}

\begin{Entry}{昨}{9}{⽇}
  \begin{Phonetics}{昨}{zuo2}
    \definition{s.}{ontem | o passado}
  \end{Phonetics}
\end{Entry}

\begin{Entry}{昨天}{9,4}{⽇、⼤}
  \begin{Phonetics}{昨天}{zuo2tian1}[][HSK 1]
    \definition{s.}{ontem}
  \end{Phonetics}
\end{Entry}

\begin{Entry}{昨日}{9,4}{⽇、⽇}
  \begin{Phonetics}{昨日}{zuo2ri4}
    \definition{adv.}{ontem}
  \end{Phonetics}
\end{Entry}

\begin{Entry}{昨夜}{9,8}{⽇、⼣}
  \begin{Phonetics}{昨夜}{zuo2ye4}
    \definition{adv.}{noite passada}
  \end{Phonetics}
\end{Entry}

\begin{Entry}{昨晚}{9,11}{⽇、⽇}
  \begin{Phonetics}{昨晚}{zuo2wan3}
    \definition{adv.}{noite passada | ontem à noite}
  \end{Phonetics}
\end{Entry}

\begin{Entry}{是}{9}{⽇}
  \begin{Phonetics}{是}{shi4}[][HSK 1]
    \definition*{s.}{Sobrenome Shi}
    \definition{adj.}{correto; certo | verdadeiro}
    \definition{adv.}{(expressar afirmação firme) de fato; realmente}
    \definition{pron.}{isso; isto |  todos; qualquer um; usado antes de substantivos, tem o significado de 凡是}
    \definition{s.}{assuntos (importantes); grandes planos}
    \definition{v.}{usado como “ser” antes de substantivos ou pronomes para identificar, descrever ou ampliar o sujeito; indica que duas coisas são iguais, ou que a segunda explica a primeira | usado entre duas palavras idênticas; relacionar duas palavras semelhantes |  (usado antes de substantivos) ser exatamente; ser corretamente; usado antes de substantivos, tem o significado de 适合 | elogiar; justificar | expressar afirmação ou concordância (frequentemente usado sozinho) | usado para escolher perguntas, perguntas sim/não ou perguntas retóricas | (usado no início de uma frase) enfatizar uma determinada parte de uma frase | usado em perguntas sim-não}
  \seealsoref{凡是}{fan2shi4}
  \seealsoref{适合}{shi4he2}
  \end{Phonetics}
\end{Entry}

\begin{Entry}{是不是}{9,4,9}{⽇、⼀、⽇}
  \begin{Phonetics}{是不是}{shi4 bu2 shi4}[][HSK 1]
    \definition{expr.}{sim ou não; é ou não é; se ou não; questões levantadas sobre a confirmação e a negação dos fatos}
  \end{Phonetics}
\end{Entry}

\begin{Entry}{是否}{9,7}{⽇、⼝}
  \begin{Phonetics}{是否}{shi4fou3}[][HSK 4]
    \definition{adv.}{se; se ou não; sim ou não}
  \end{Phonetics}
\end{Entry}

\begin{Entry}{是的}{9,8}{⽇、⽩}
  \begin{Phonetics}{是的}{shi4de5}
    \definition{adv.}{sim | está certo}
  \end{Phonetics}
\end{Entry}

\begin{Entry}{昼}{9}{⽇}
  \begin{Phonetics}{昼}{zhou4}
    \definition*{s.}{Sobrenome Zhou}
    \definition{s.}{diurno; luz do dia; dia (oposição à 夜) | dia; o período do amanhecer ao anoitecer; diurno}
  \seealsoref{夜}{ye4}
  \end{Phonetics}
\end{Entry}

\begin{Entry}{显}{9}{⽇}
  \begin{Phonetics}{显}{xian3}[][HSK 5]
    \definition*{s.}{Sobrenome Xian}
    \definition{adj.}{aparente; óbvio; perceptível | ilustre e influente | evidente; óbvio}
    \definition{v.}{mostrar; exibir; manifestar | aparecer; mostrar; revelar}
  \end{Phonetics}
\end{Entry}

\begin{Entry}{显出}{9,5}{⽇、⼐}
  \begin{Phonetics}{显出}{xian3 chu1}[][HSK 6]
    \definition{v.}{mostrar; revelar | dar provas; expressar; exibir}
  \end{Phonetics}
\end{Entry}

\begin{Entry}{显示}{9,5}{⽇、⽰}
  \begin{Phonetics}{显示}{xian3shi4}[][HSK 3]
    \definition{v.}{mostrar; manifestar-se claramente| exibir; ostentar}
  \end{Phonetics}
\end{Entry}

\begin{Entry}{显得}{9,11}{⽇、⼻}
  \begin{Phonetics}{显得}{xian3de5}[][HSK 3]
    \definition{v.}{parecer; aparecer; manifestar (alguma situação)}
  \end{Phonetics}
\end{Entry}

\begin{Entry}{显著}{9,11}{⽇、⽬}
  \begin{Phonetics}{显著}{xian3zhu4}[][HSK 4]
    \definition{adj.}{notável; significativo; notável; extraordinário; muito óbvio; muito claramente demonstrado; muito facilmente visto ou sentido}
  \end{Phonetics}
\end{Entry}

\begin{Entry}{显然}{9,12}{⽇、⽕}
  \begin{Phonetics}{显然}{xian3ran2}[][HSK 3]
    \definition{adj.}{claro; evidente; óbvio; fatos, verdades e outras coisas que são fáceis de descobrir, perceber ou sentir claramente}
  \end{Phonetics}
\end{Entry}

\begin{Entry}{枯}{9}{⽊}
  \begin{Phonetics}{枯}{ku1}
    \definition{adj.}{murcho | (de um poço, rio, etc.) seco | chato; desinteressante | magro e abatido; emaciado}
    \definition[片]{s.}{borra; resíduo}
  \end{Phonetics}
\end{Entry}

\begin{Entry}{枯木}{9,4}{⽊、⽊}
  \begin{Phonetics}{枯木}{ku1mu4}
    \definition{s.}{árvore morta | madeira morta}
  \end{Phonetics}
\end{Entry}

\begin{Entry}{架}{9}{⽊}
  \begin{Phonetics}{架}{jia4}[][HSK 3]
    \definition{clas.}{usado para coisas com pilares ou componentes mecânicos | quadrado (usado para montanhas)}
    \definition{s.}{estrutura; organização do corpo humano ou das coisas | prateleira; estante; suporte; componentes que sustentam objetos ou utensílios para colocar objetos, etc.}
    \definition{v.}{colocar para cima; erigir | brigar; discutir | resistir; repelir; afastar | sequestrar; levar alguém à força}
  \end{Phonetics}
\end{Entry}

\begin{Entry}{架式}{9,6}{⽊、⼷}
  \begin{Phonetics}{架式}{jia4shi5}
    \variantof{架势}
  \end{Phonetics}
\end{Entry}

\begin{Entry}{架势}{9,8}{⽊、⼒}
  \begin{Phonetics}{架势}{jia4shi5}
    \definition{s.}{postura | atitude | posição (sobre um assunto, etc.)}
  \end{Phonetics}
\end{Entry}

\begin{Entry}{柏}{9}{⽊}
  \begin{Phonetics}{柏}{bai3}
  \seealsoref{柏树}{bai3shu4}
  \end{Phonetics}
  \begin{Phonetics}{柏}{bo2}
    \definition{s.}{cipreste | usado para transcrever nomes}[柏林,德国城市名。===Berlim, uma cidade alemã.]
  \end{Phonetics}
  \begin{Phonetics}{柏}{bo4}
    \definition{s.}{cedro; cipreste amarelo}
  \end{Phonetics}
\end{Entry}

\begin{Entry}{柏林}{9,8}{⽊、⽊}
  \begin{Phonetics}{柏林}{bo2lin2}
    \definition*{s.}{Berlim, capital da Alemanha}
  \end{Phonetics}
\end{Entry}

\begin{Entry}{柏树}{9,9}{⽊、⽊}
  \begin{Phonetics}{柏树}{bai3shu4}
    \definition{s.}{cipreste}
  \end{Phonetics}
\end{Entry}

\begin{Entry}{某}{9}{⽊}
  \begin{Phonetics}{某}{mou3}[][HSK 3]
    \definition{pron.}{alguém ou algo indefinido; refere-se a pessoas ou coisas incertas | referindo-se a si mesmo; em vez do seu próprio nome | alguns; certos; refere-se a uma pessoa ou coisa específica cujo nome não se sabe ou não se pode revelar | tal e tal; substituir o nome de outra pessoa (geralmente com um tom rude)}
  \end{Phonetics}
\end{Entry}

\begin{Entry}{染}{9}{⽊}
  \begin{Phonetics}{染}{ran3}[][HSK 5]
    \definition*{s.}{Sobrenome Ran}
    \definition{s.}{soja fermentada e temperada em forma de pasta}
    \definition{v.}{tingir; pintar | pegar (uma doença); cair em (um mau hábito, etc.) | sujar; contaminar | pegar (contrair) (uma doença) | adquirir (um mau hábito, etc.); contaminar}
  \end{Phonetics}
\end{Entry}

\begin{Entry}{柔}{9}{⽊}
  \begin{Phonetics}{柔}{rou2}
    \definition*{s.}{Sobrenome Rou}
    \definition{adj.}{macio; flexível; maleável | gentil; flexível; brando}
    \definition{v.}{tornar macio; amolecer | apaziguar}
  \end{Phonetics}
\end{Entry}

\begin{Entry}{柔软}{9,8}{⽊、⾞}
  \begin{Phonetics}{柔软}{rou2ruan3}
    \definition{adj.}{macio | suave}
  \end{Phonetics}
\end{Entry}

\begin{Entry}{柠}{9}{⽊}
  \begin{Phonetics}{柠}{ning2}
    \definition{s.}{limão}
  \end{Phonetics}
\end{Entry}

\begin{Entry}{柠檬}{9,17}{⽊、⽊}
  \begin{Phonetics}{柠檬}{ning2meng2}
    \definition[个,片,只]{s.}{limão}
  \end{Phonetics}
\end{Entry}

\begin{Entry}{查}{9}{⽊}
  \begin{Phonetics}{查}{cha2}[][HSK 2]
    \definition{v.}{examinar; verificar cuidadosamente | examinar; investigar; entender bem a situação | procurar; consultar; revisar (documentos bibliográficos)}
  \end{Phonetics}
  \begin{Phonetics}{查}{zha1}
    \definition*{s.}{Sobrenome Zha}
    \definition{s.}{espinheiro-chinês}
  \end{Phonetics}
\end{Entry}

\begin{Entry}{查出}{9,5}{⽊、⼐}
  \begin{Phonetics}{查出}{cha2 chu1}[][HSK 6]
    \definition{v.}{rastrear; desentocar}
  \end{Phonetics}
\end{Entry}

\begin{Entry}{查询}{9,8}{⽊、⾔}
  \begin{Phonetics}{查询}{cha2 xun2}[][HSK 5]
    \definition{v.}{indagar; inquirir; perguntar sobre}
  \end{Phonetics}
\end{Entry}

\begin{Entry}{查看}{9,9}{⽊、⽬}
  \begin{Phonetics}{查看}{cha2 kan4}[][HSK 6]
    \definition{v.}{verificar; examinar; checar; investigar; verificar e observar a existência das coisas}
  \end{Phonetics}
\end{Entry}

\begin{Entry}{柬}{9}{⽊}
  \begin{Phonetics}{柬}{jian3}
    \definition*{s.}{Sobrenome Jian}
    \definition[张,封]{s.}{cartão; nota; carta; um termo geral para cartas, cartões de visita, postagens, etc.}
  \end{Phonetics}
\end{Entry}

\begin{Entry}{柬埔寨}{9,10,14}{⽊、⼟、⼧}
  \begin{Phonetics}{柬埔寨}{jian3pu3zhai4}
    \definition*{s.}{Camboja}
  \end{Phonetics}
\end{Entry}

\begin{Entry}{柱}{9}{⽊}
  \begin{Phonetics}{柱}{zhu4}
    \definition*{s.}{Sobrenome Zhu}
    \definition[根]{s.}{poste; pilar; coluna | algo em forma de coluna | Matemática: cilindro}
  \end{Phonetics}
\end{Entry}

\begin{Entry}{柱子}{9,3}{⽊、⼦}
  \begin{Phonetics}{柱子}{zhu4 zi5}[][HSK 6]
    \definition{s.}{poste; pilar; coluna; estrutura de suporte vertical de um edifício, feita de madeira, pedra, aço, concreto armado, etc.}
  \end{Phonetics}
\end{Entry}

\begin{Entry}{柳}{9}{⽊}
  \begin{Phonetics}{柳}{liu3}
    \definition*{s.}{Liu, a vigésima quarta das vinte e oito constelações, consistindo de oito estrelas em Hydra | Liu, uma das mansões lunares | Sobrenome Liu}
    \definition[棵]{s.}{salgueiro}
  \end{Phonetics}
\end{Entry}

\begin{Entry}{柳橙汁}{9,16,5}{⽊、⽊、⽔}
  \begin{Phonetics}{柳橙汁}{liu3cheng2zhi1}
    \definition[瓶,杯,罐,盒]{s.}{suco de laranja}
  \seealsoref{橙汁}{cheng2zhi1}
  \seealsoref{橘子汁}{ju2zi5zhi1}
  \end{Phonetics}
\end{Entry}

\begin{Entry}{标}{9}{⽊}
  \begin{Phonetics}{标}{biao1}
    \definition{clas.}{usado para equipes (o numeral é limitado a um, 一, o que é comum no chinês moderno)}
    \definition[个]{s.}{copa da árvore (significado original) | marca; sinal | padrão; cota | sinal externo; sintoma | prêmio; troféu | oferta; licitação comercial pública | a ponta de uma árvore | aparência externa; ramos ou superfícies | partes aéreas das plantas | rótulo; etiqueta; identificação; sinal | regimento na Dinastia Qing; uma das organizações militares no final da Dinastia Qing}
    \definition{v.}{colocar uma marca, etiqueta ou rótulo em; rotular | agrupar; formar equipe | marcar; expressar com palavras ou outras coisas |}
  \end{Phonetics}
\end{Entry}

\begin{Entry}{标志}{9,7}{⽊、⼼}
  \begin{Phonetics}{标志}{biao1zhi4}[][HSK 4]
    \definition[个,种]{s.}{sinal; marca; logotipo; símbolo; emblema; marcações que caracterizam um objeto para facilitar a identificação}
    \definition{v.}{marcar; indicar; simbolizar; identificar}
  \end{Phonetics}
\end{Entry}

\begin{Entry}{标准}{9,10}{⽊、⼎}
  \begin{Phonetics}{标准}{biao1zhun3}[][HSK 3]
    \definition{adj.}{padrão (que serve como ou está em conformidade com um padrão); em conformidade com os documentos e princípios regulamentares}
    \definition[个,条,项,种]{s.}{padrão; critério; critérios de avaliação das coisas}
  \end{Phonetics}
\end{Entry}

\begin{Entry}{标题}{9,15}{⽊、⾴}
  \begin{Phonetics}{标题}{biao1ti2}[][HSK 3]
    \definition[个,条,篇]{s.}{título; manchete; cabeçalho; resumo conciso do conteúdo da obra}
  \end{Phonetics}
\end{Entry}

\begin{Entry}{栏}{9}{⽊}
  \begin{Phonetics}{栏}{lan2}
    \definition{s.}{cerca; corrimão; balaustrada | curral; galpão; celeiro; chiqueiro | coluna (de uma página ou tabela, ou de um jornal) | quadro (de avisos); prancha; tabuleiro | Esporte: obstáculo}
  \end{Phonetics}
\end{Entry}

\begin{Entry}{栏目}{9,5}{⽊、⽬}
  \begin{Phonetics}{栏目}{lan2mu4}[][HSK 6]
    \definition[个,档]{s.}{coluna; programa; seções nomeadas de jornais, revistas, etc. divididas de acordo com a natureza de seu conteúdo}
  \end{Phonetics}
\end{Entry}

\begin{Entry}{树}{9}{⽊}
  \begin{Phonetics}{树}{shu4}[][HSK 1]
    \definition*{s.}{Sobrenome Shu}
    \definition[棵,株]{s.}{árvore; nome comum das plantas lenhosas}
    \definition{v.}{plantar; cultivar | configurar; manter; estabelecer}
  \end{Phonetics}
\end{Entry}

\begin{Entry}{树木}{9,4}{⽊、⽊}
  \begin{Phonetics}{树木}{shu4mu4}
    \definition{s.}{árvore}
  \end{Phonetics}
\end{Entry}

\begin{Entry}{树叶}{9,5}{⽊、⼝}
  \begin{Phonetics}{树叶}{shu4ye4}[][HSK 4]
    \definition[片,枚,堆]{s.}{folha; folhagem}
  \end{Phonetics}
\end{Entry}

\begin{Entry}{树林}{9,8}{⽊、⽊}
  \begin{Phonetics}{树林}{shu4 lin2}[][HSK 4]
    \definition[片,座]{s.}{bosque; muitas árvores que crescem em fragmentos, menores que as florestas}
  \end{Phonetics}
\end{Entry}

\begin{Entry}{树莓}{9,10}{⽊、⾋}
  \begin{Phonetics}{树莓}{shu4mei2}
    \definition{s.}{framboesa}
  \end{Phonetics}
\end{Entry}

\begin{Entry}{歪}{9}{⽌}
  \begin{Phonetics}{歪}{wai1}
    \definition{adj.}{torto | tortuoso | nocivo}
  \end{Phonetics}
\end{Entry}

\begin{Entry}{歪果仁}{9,8,4}{⽌、⽊、⼈}
  \begin{Phonetics}{歪果仁}{wai1 guo3 ren2}
    \definition{s.}{gíria na \emph{Internet} para estrangeiro (外国人)}
  \seealsoref{外国人}{wai4 guo2 ren2}
  \end{Phonetics}
\end{Entry}

\begin{Entry}{残}{9}{⽍}
  \begin{Phonetics}{残}{can2}
    \definition{adj.}{incompleto; fragmentário; deficiente | remanescente; restante | cruel; feroz | opressivo; selvagem; bárbaro}
    \definition{v.}{ferir; danificar | estragar; prejudicar; destruir}
  \end{Phonetics}
\end{Entry}

\begin{Entry}{残疾}{9,10}{⽍、⽧}
  \begin{Phonetics}{残疾}{can2ji2}[][HSK 6]
    \definition{s.}{deformidade; deficiência; deficiência física; defeitos de membros, órgãos ou funções fisiológicas}
  \end{Phonetics}
\end{Entry}

\begin{Entry}{残疾人}{9,10,2}{⽍、⽧、⼈}
  \begin{Phonetics}{残疾人}{can2 ji2 ren2}[][HSK 6]
    \definition[位,名]{s.}{pessoa com deficiência (ou incapacitada); o incapacitado (ou deficiente); pessoas com deficiência visual, auditiva, de linguagem, intelectual, física ou mental são os principais alvos da medicina de reabilitação}
  \end{Phonetics}
\end{Entry}

\begin{Entry}{残酷}{9,14}{⽍、⾣}
  \begin{Phonetics}{残酷}{can2ku4}[][HSK 6]
    \definition{adj.}{cruel; brutal; implacável}
  \end{Phonetics}
\end{Entry}

\begin{Entry}{段}{9}{⽎}
  \begin{Phonetics}{段}{duan4}[][HSK 2]
    \definition*{s.}{Sobrenome Duan}
    \definition{clas.}{parte; seção; segmento; usado para dividir objetos em várias partes | passagem; parágrafo; parte de algo que tem características de continuidade | seção; período; usado para uma certa distância no tempo ou no espaço}
    \definition{s.}{nível; dan (no judô, weiqi, etc.) | seção (como nível administrativo em uma mina ou fábrica) | parte; etapa; estágio}
    \definition{v.}{cortar; separar}
  \end{Phonetics}
\end{Entry}

\begin{Entry}{毒}{9}{⽏}
  \begin{Phonetics}{毒}{du2}[][HSK 5]
    \definition*{s.}{Sobrenome Du}
    \definition{adj.}{veneno; toxina; propriedade ou substância prejudicial aos organismos vivos | droga; narcóticos | vírus; vírus de computador | influência venenosa}
    \definition{adj.}{venenoso; tóxico; envenenado | malicioso; cruel; feroz}
    \definition{v.}{matar com veneno; envenenar | envenenar (a mente de alguém)}
  \end{Phonetics}
\end{Entry}

\begin{Entry}{毒杀}{9,6}{⽏、⽊}
  \begin{Phonetics}{毒杀}{du2sha1}
    \definition{v.}{matar por envenenamento}
  \end{Phonetics}
\end{Entry}

\begin{Entry}{毒物}{9,8}{⽏、⽜}
  \begin{Phonetics}{毒物}{du2wu4}
    \definition{s.}{substância venenosa | toxina}
  \end{Phonetics}
\end{Entry}

\begin{Entry}{毒品}{9,9}{⽏、⼝}
  \begin{Phonetics}{毒品}{du2pin3}[][HSK 6]
    \definition[种,点]{s.}{drogas; veneno; narcóticos; refere-se ao ópio, morfina, heroína, etc. usados ​​como vício}
  \end{Phonetics}
\end{Entry}

\begin{Entry}{毒害}{9,10}{⽏、⼧}
  \begin{Phonetics}{毒害}{du2hai4}
    \definition{s.}{envenenamento}
    \definition{v.}{envenenar (prejudicar com uma substância tóxica) | envenenar (as mentes das pessoas)}
  \end{Phonetics}
\end{Entry}

\begin{Entry}{毒蛇}{9,11}{⽏、⾍}
  \begin{Phonetics}{毒蛇}{du2she2}
    \definition{s.}{víbora | cobra venenosa}
  \end{Phonetics}
\end{Entry}

\begin{Entry}{泉}{9}{⽔}
  \begin{Phonetics}{泉}{quan2}[][HSK 5]
    \definition*{s.}{Sobrenome Quan}
    \definition[股,眼,汪]{s.}{fonte (de água mineral) | a nascente de um rio | termo antigo para moeda}
  \end{Phonetics}
\end{Entry}

\begin{Entry}{洋}{9}{⽔}
  \begin{Phonetics}{洋}{yang2}[][HSK 6]
    \definition*{s.}{Sobrenome Yang}
    \definition{adj.}{vasto; rico; transbordante | estrangeiro (especialmente ocidental) | moderno (oposto a 土)}
    \definition[个,片]{s.}{oceano | moeda de prata}
  \seealsoref{土}{tu3}
  \end{Phonetics}
\end{Entry}

\begin{Entry}{洋葱}{9,12}{⽔、⾋}
  \begin{Phonetics}{洋葱}{yang2cong1}
    \definition{s.}{cebola}
  \end{Phonetics}
\end{Entry}

\begin{Entry}{洒}{9}{⽔}
  \begin{Phonetics}{洒}{sa3}[][HSK 5]
    \definition{adj.}{natural e sem restrições; confortável (sem restrições)}
    \definition{v.}{derramar; espalhar; borrifar; salpicar; fazer com que (água ou outra coisa) caia de forma dispersa | derramar; cair de forma dispersa}
  \end{Phonetics}
\end{Entry}

\begin{Entry}{洒水}{9,4}{⽔、⽔}
  \begin{Phonetics}{洒水}{sa3shui3}
    \definition{v.}{borrifar}
  \end{Phonetics}
\end{Entry}

\begin{Entry}{洗}{9}{⽔}
  \begin{Phonetics}{洗}{xi3}[][HSK 1]
    \definition[个]{s.}{pequeno recipiente contendo água para enxaguar os pincéis de escrever | batismo}
    \definition{v.}{lavar; tomar banho; remover a sujeira do objeto com água ou outro solvente | batizar | eliminar; corrigir; reparar | saquear; matar e pilhar; matar ou roubar tudo, como se tivesse sido lavado | revelar filmes; imprimir fotos | apagar; limpar (uma gravação, etc.) | embaralhar (cartas, etc.)}
  \end{Phonetics}
\end{Entry}

\begin{Entry}{洗手}{9,4}{⽔、⼿}
  \begin{Phonetics}{洗手}{xi3shou3}
    \definition{v.}{ir ao banheiro | lavar as mãos}
  \end{Phonetics}
\end{Entry}

\begin{Entry}{洗手不干}{9,4,4,3}{⽔、⼿、⼀、⼲}
  \begin{Phonetics}{洗手不干}{xi3shou3bu2gan4}
    \definition{v.}{parar totalmente de fazer algo}
  \end{Phonetics}
\end{Entry}

\begin{Entry}{洗手池}{9,4,6}{⽔、⼿、⽔}
  \begin{Phonetics}{洗手池}{xi3shou3chi2}
    \definition{s.}{pia de banheiro | lavatório}
  \seealsoref{洗手盆}{xi3shou3pen2}
  \end{Phonetics}
\end{Entry}

\begin{Entry}{洗手间}{9,4,7}{⽔、⼿、⾨}
  \begin{Phonetics}{洗手间}{xi3shou3jian1}[][HSK 1]
    \definition[个]{s.}{banheiro; lavatório; lavabo}
  \end{Phonetics}
\end{Entry}

\begin{Entry}{洗手乳}{9,4,8}{⽔、⼿、⼄}
  \begin{Phonetics}{洗手乳}{xi3shou3ru3}
    \definition{s.}{sabonete líquido para lavar as mãos}
  \seealsoref{洗手液}{xi3shou3ye4}
  \end{Phonetics}
\end{Entry}

\begin{Entry}{洗手盆}{9,4,9}{⽔、⼿、⽫}
  \begin{Phonetics}{洗手盆}{xi3shou3pen2}
    \definition{s.}{pia de banheiro | lavatório}
  \seealsoref{洗手池}{xi3shou3chi2}
  \end{Phonetics}
\end{Entry}

\begin{Entry}{洗手液}{9,4,11}{⽔、⼿、⽔}
  \begin{Phonetics}{洗手液}{xi3shou3ye4}
    \definition{s.}{sabonete líquido para lavar as mãos}
  \seealsoref{洗手乳}{xi3shou3ru3}
  \end{Phonetics}
\end{Entry}

\begin{Entry}{洗礼}{9,5}{⽔、⽰}
  \begin{Phonetics}{洗礼}{xi3li3}
    \definition{s.}{batismo}
    \definition{v.}{batizar}
  \end{Phonetics}
\end{Entry}

\begin{Entry}{洗衣机}{9,6,6}{⽔、⾐、⽊}
  \begin{Phonetics}{洗衣机}{xi3 yi1 ji1}[][HSK 2]
    \definition[台]{s.}{máquina de lavar roupa; eletrodomésticos para lavagem automática ou semiautomática de roupas}
  \end{Phonetics}
\end{Entry}

\begin{Entry}{洗衣粉}{9,6,10}{⽔、⾐、⽶}
  \begin{Phonetics}{洗衣粉}{xi3 yi1 fen3}[][HSK 6]
    \definition[袋,包,勺]{s.}{sabão em pó; detergente para roupa (em pó); detergente em pó sintetizado quimicamente, específico para uso em lavanderia}
  \end{Phonetics}
\end{Entry}

\begin{Entry}{洗劫}{9,7}{⽔、⼒}
  \begin{Phonetics}{洗劫}{xi3jie2}
    \definition{v.}{saquear | pilhar | roubar}
  \end{Phonetics}
\end{Entry}

\begin{Entry}{洗净}{9,8}{⽔、⼎}
  \begin{Phonetics}{洗净}{xi3jing4}
    \definition{v.}{lavar (limpeza)}
  \end{Phonetics}
\end{Entry}

\begin{Entry}{洗胃}{9,9}{⽔、⾁}
  \begin{Phonetics}{洗胃}{xi3wei4}
    \definition{s.}{(medicina) lavagem gástrica}
    \definition{v.}{ter o estômago lavado}
  \end{Phonetics}
\end{Entry}

\begin{Entry}{洗涤}{9,10}{⽔、⽔}
  \begin{Phonetics}{洗涤}{xi3di2}
    \definition{s.}{enxágue | lava}
    \definition{v.}{enxaguar | lavar}
  \end{Phonetics}
\end{Entry}

\begin{Entry}{洗涤间}{9,10,7}{⽔、⽔、⾨}
  \begin{Phonetics}{洗涤间}{xi3di2jian1}
    \definition{s.}{lavanderia}
  \end{Phonetics}
\end{Entry}

\begin{Entry}{洗脱}{9,11}{⽔、⾁}
  \begin{Phonetics}{洗脱}{xi3tuo1}
    \definition{v.}{limpar | purgar | lavar}
  \end{Phonetics}
\end{Entry}

\begin{Entry}{洗碗}{9,13}{⽔、⽯}
  \begin{Phonetics}{洗碗}{xi3wan3}
    \definition{v.}{lavar pratos}
  \end{Phonetics}
\end{Entry}

\begin{Entry}{洗澡}{9,16}{⽔、⽔}
  \begin{Phonetics}{洗澡}{xi3zao3}[][HSK 2]
    \definition{v.+compl.}{tomar banho; tomar banho de chuveiro; lavar-se}
  \end{Phonetics}
\end{Entry}

\begin{Entry}{洗澡间}{9,16,7}{⽔、⽔、⾨}
  \begin{Phonetics}{洗澡间}{xi3zao3jian1}
    \definition[间]{s.}{banheiro}
  \end{Phonetics}
\end{Entry}

\begin{Entry}{洞}{9}{⽔}
  \begin{Phonetics}{洞}{dong4}[][HSK 5]
    \definition{adj.}{profundo; minucioso; claro; completo; abrangente}
    \definition{s.}{buraco; cavidade; orifício; furo; parte penetrante ou profundamente recuada de um objeto; uma caverna}
  \end{Phonetics}
\end{Entry}

\begin{Entry}{洞穴}{9,5}{⽔、⽳}
  \begin{Phonetics}{洞穴}{dong4xue2}
    \definition{s.}{caverna}
  \end{Phonetics}
\end{Entry}

\begin{Entry}{洪}{9}{⽔}
  \begin{Phonetics}{洪}{hong2}
    \definition*{s.}{Sobrenome Hong}
    \definition{adj.}{alto; vasto | grande; grandioso}
    \definition[场]{s.}{enchente; inundação}
  \end{Phonetics}
\end{Entry}

\begin{Entry}{洪水}{9,4}{⽔、⽔}
  \begin{Phonetics}{洪水}{hong2shui3}[][HSK 6]
    \definition[场]{s.}{dilúvio; inundação; enchente; um aumento repentino em um rio causado por chuva forte ou derretimento de neve}
  \end{Phonetics}
\end{Entry}

\begin{Entry}{洲}{9}{⽔}
  \begin{Phonetics}{洲}{zhou1}
    \definition{s.}{continente | ilha em um rio}
  \end{Phonetics}
\end{Entry}

\begin{Entry}{活}{9}{⽔}
  \begin{Phonetics}{活}{huo2}[][HSK 3]
    \definition{adj.}{vivo; vivendo; indica que (alguma ação) foi realizada enquanto a pessoa ainda estava viva | vívido; animado; ativo | móvel; em movimento; ativo}
    \definition{adv.}{exatamente; simplesmente; expressa um grau elevado, equivalente a 真正 ou 简直}
    \definition{s.}{emprego; meios de subsistência; trabalho (geralmente refere-se a trabalho físico) | produto; algo fabricado}
    \definition{v.}{viver; ter vida; sobreviver (em oposição a 死) | salvar (a vida de uma pessoa); fazer sobreviver; manter a vida}
  \seealsoref{简直}{jian3zhi2}
  \seealsoref{死}{si3}
  \seealsoref{真正}{zhen1zheng4}
  \end{Phonetics}
\end{Entry}

\begin{Entry}{活力}{9,2}{⽔、⼒}
  \begin{Phonetics}{活力}{huo2li4}[][HSK 5]
    \definition{s.}{vigor; vitalidade; energia; muito forte, geralmente usado para descrever pessoas, cidades, empresas, economias, etc.}
  \end{Phonetics}
\end{Entry}

\begin{Entry}{活动}{9,6}{⽔、⼒}
  \begin{Phonetics}{活动}{huo2dong4}[][HSK 2]
    \definition{adj.}{móvel; flexível para alterações ou mudanças}
    \definition[些,个,种,类,次]{s.}{atividade; ação tomada com o objetivo de alcançar um determinado objetivo}
    \definition{v.}{fazer exercício; movimentar-se | usar influência pessoal; usar meios irregulares | mover-se}
  \end{Phonetics}
\end{Entry}

\begin{Entry}{活泼}{9,8}{⽔、⽔}
  \begin{Phonetics}{活泼}{huo2po1}[][HSK 5]
    \definition{adj.}{vívido; ativo; animado; brilhante; vivaz; cheio de vida | Química: reativo; significa que a substância é ativa e reage facilmente com outras substâncias}
  \end{Phonetics}
\end{Entry}

\begin{Entry}{活着}{9,11}{⽔、⽬}
  \begin{Phonetics}{活着}{huo2zhe5}
    \definition{adj.}{vivo}
  \end{Phonetics}
\end{Entry}

\begin{Entry}{活跃}{9,11}{⽔、⾜}
  \begin{Phonetics}{活跃}{huo2yue4}[][HSK 6]
    \definition{adj.}{ativo; dinâmico; pensamentos, ações ou atividades positivas; ocorrências frequentes | rápido; ativo; dinâmico}
    \definition{v.}{animar; tornar ativo | ser ativo}
  \end{Phonetics}
\end{Entry}

\begin{Entry}{活路}{9,13}{⽔、⾜}
  \begin{Phonetics}{活路}{huo2lu4}
    \definition{s.}{maneira de sobreviver | meio de subsistência}
  \end{Phonetics}
  \begin{Phonetics}{活路}{huo2lu5}
    \definition{s.}{labor | trabalho físico}
  \end{Phonetics}
\end{Entry}

\begin{Entry}{派}{9}{⽔}
  \begin{Phonetics}{派}{pai4}[][HSK 3]
    \definition{adj.}{elegante; bonito; imponente}
    \definition{clas.}{usado para grupos, escolas de pensamento ou arte, etc. | usado para um discursos, situações, cenas, etc.}
    \definition[个,块,种]{s.}{panelinha; facção; pessoas com ideias, visões e estilos semelhantes | torta; um alimento recheado comumente consumido pelos ocidentais, geralmente doce | maneira e ar; estilo ou comportamento | afluente; braço de rio}
    \definition{v.}{enviar; despachar; arranjar ou ordenar que uma pessoa faça algo; providenciar transporte | alocar; repartir; distribuir}
  \end{Phonetics}
\end{Entry}

\begin{Entry}{派出}{9,5}{⽔、⼐}
  \begin{Phonetics}{派出}{pai4 chu1}[][HSK 6]
    \definition{v.}{despachar; expedi | enviar}
  \end{Phonetics}
\end{Entry}

\begin{Entry}{浊}{9}{⽔}
  \begin{Phonetics}{浊}{zhuo2}
    \definition*{s.}{Sobrenome Zhuo}
    \definition{adj.}{turvo; lamacento; imundo (oposto a 清) | profundo e espesso | caótico; confuso; corrompido}
  \seealsoref{清}{qing1}
  \end{Phonetics}
\end{Entry}

\begin{Entry}{测}{9}{⽔}
  \begin{Phonetics}{测}{ce4}[][HSK 4]
    \definition{v.}{pesquisar; sondar; medir | conjecturar; advinhar}
  \end{Phonetics}
\end{Entry}

\begin{Entry}{测定}{9,8}{⽔、⼧}
  \begin{Phonetics}{测定}{ce4 ding4}[][HSK 6]
    \definition{v.}{verificar por medição (ou levantamento); determinar; medir; avaliar}
  \end{Phonetics}
\end{Entry}

\begin{Entry}{测试}{9,8}{⽔、⾔}
  \begin{Phonetics}{测试}{ce4 shi4}[][HSK 4]
    \definition[次,场]{s.}{exame; teste; medição do conhecimento humano, das habilidades ou do funcionamento de máquinas, ferramentas ou instrumentos}
    \definition{v.}{examinar | testar, medição do desempenho e da precisão de máquinas, instrumentos, aparelhos, etc.}
  \end{Phonetics}
\end{Entry}

\begin{Entry}{测量}{9,12}{⽔、⾥}
  \begin{Phonetics}{测量}{ce4liang2}[][HSK 4]
    \definition{v.}{aferir; pesquisar; medir; determinar valores relevantes para espaço, tempo, temperatura, velocidade, função, etc.}
  \end{Phonetics}
\end{Entry}

\begin{Entry}{浓}{9}{⽔}
  \begin{Phonetics}{浓}{nong2}[][HSK 4]
    \definition{adj.}{denso; espesso; concentrado; um líquido ou gás que contém mais de um determinado ingrediente | grande; forte; profundo (de grau ou extensão) | profundo; (algumas cores) escuro}
  \end{Phonetics}
\end{Entry}

\begin{Entry}{炮}{9}{⽕}
  \begin{Phonetics}{炮}{bao1}
    \definition{v.}{processar; o método de preparação da medicina chinesa é colocar as ervas cruas em uma panela de ferro em alta temperatura e fritá-las até que fiquem marrons e estourem | secar alimentos pelo calor; refogar}
  \end{Phonetics}
  \begin{Phonetics}{炮}{pao2}
    \definition{v.}{(medicina tradicional chinesa) preparar a medicina chinesa assando-a em uma panela de ferro quente até dourar e estalar}
  \end{Phonetics}
  \begin{Phonetics}{炮}{pao4}[][HSK 6]
    \definition{s.}{arma grande; canhão; peça de artilharia | fogo de artifício | buraco de explosão cheio de dinamite | canhão, uma das peças do xadrez chinês}
  \end{Phonetics}
\end{Entry}

\begin{Entry}{炸}{9}{⽕}
  \begin{Phonetics}{炸}{zha2}
    \definition{v.}{explodir; estourar; romper | dinamitar; bombardear; explodir; detonar com explosivos | encolerizar-se; explodir em fúria | correr; fugir em pânico}
  \end{Phonetics}
  \begin{Phonetics}{炸}{zha4}[][HSK 6]
    \definition{v.}{fritar em gordura ou óleo | escaldar (como forma de cozinhar)}
  \end{Phonetics}
\end{Entry}

\begin{Entry}{炸药}{9,9}{⽕、⾋}
  \begin{Phonetics}{炸药}{zha4 yao4}[][HSK 6]
    \definition[包,种]{s.}{explosivo; cargas explosivas; dinamite; substâncias que explodem quando aquecidas ou impactadas, produzindo grandes quantidades de energia e gases de alta temperatura, como dinamite e pólvora negra}
  \end{Phonetics}
\end{Entry}

\begin{Entry}{炸弹}{9,11}{⽕、⼸}
  \begin{Phonetics}{炸弹}{zha4 dan4}[][HSK 6]
    \definition{s.}{bomba; uma arma com invólucro de ferro e explosivos dentro que explodem quando um fusível é acionado, geralmente lançada de um avião}
  \end{Phonetics}
\end{Entry}

\begin{Entry}{点}{9}{⽕}
  \begin{Phonetics}{点}{dian3}[][HSK 1]
    \definition{clas.}{hora cheia | ponto, uma unidade de medida para tipos; antigamente, a contagem do tempo durante a noite era feita por turnos, sendo cada turno dividido em cinco pontos | quantidade ínfima; um pouco; um pouquinho; alguma coisa; indica uma pequena quantidade | usado para itens}
    \definition{s.}{gota (de líquido); (ponto) pequena gota de líquido | mancha; ponto; salpico; (um pouco) Um pequeno vestígio | (ponto) Traço de um caractere chinês, cuja forma é ``、''  | ponto; (matemática) refere-se a uma figura geométrica que não tem comprimento, largura ou altura, mas apenas uma posição | gongo, instrumento musical de metal | ponto decimal; refere-se ao ponto decimal, símbolo matemático que representa os números decimais | lugar específico | lanche leve; petisco | lugar; grau; sinalização de um determinado local ou grau | hora marcada; hora regulamentar | aspecto; característica; partes ou aspectos específicos de algo | ritmo; batida}
    \definition{v.}{andar na ponta dos pés | dar uma dica, sugestão | tocar levemente com o dedo, pincel ou vara; tocar muito brevemente; passar rapidamente | acenar; baixar ligeiramente a cabeça e levantar rapidamente | gotejar; fazer cair líquido | semear em buracos; plantar com um plantador | verificar um por um | colocar um ponto; usar caneta e outras ferramentas para adicionar ideias | sugerir; indicar; dar uma dica | decorar; realçar | selecionar; escolher; especificar o que é exigido | acender; queimar; inflamar | (pedido) comer uma pequena quantidade de comida para saciar a fome}
  \end{Phonetics}
\end{Entry}

\begin{Entry}{点火}{9,4}{⽕、⽕}
  \begin{Phonetics}{点火}{dian3huo3}
    \definition{s.}{ignição}
    \definition{v.}{inflamar | acender um fogo | agitar | dar partida em um motor | (figurativo) provocar problemas}
  \end{Phonetics}
\end{Entry}

\begin{Entry}{点头}{9,5}{⽕、⼤}
  \begin{Phonetics}{点头}{dian3 tou2}[][HSK 2]
    \definition{v.}{acenar com a cabeça; balançar a cabeça; mover ligeiramente a cabeça para baixo; indicar permissão, aprovação, compreensão ou saudação}
  \end{Phonetics}
\end{Entry}

\begin{Entry}{点名}{9,6}{⽕、⼝}
  \begin{Phonetics}{点名}{dian3 ming2}[][HSK 4]
    \definition{v.}{fazer a lista de chamada; manter o controle da presença de alguém; chamar nomes para controle de presença | mencionar alguém pelo nome}
  \end{Phonetics}
\end{Entry}

\begin{Entry}{点燃}{9,16}{⽕、⽕}
  \begin{Phonetics}{点燃}{dian3 ran2}[][HSK 5]
    \definition{v.}{acender; inflamar; acender uma fogueira, para iluminar}
  \end{Phonetics}
\end{Entry}

\begin{Entry}{炼}{9}{⽕}
  \begin{Phonetics}{炼}{lian4}
    \definition{v.}{fundir; refinar | temperar (um metal) com fogo | pesar a palavra; procurar a frase certa; polir | trabalhar; tornar uma substância pura ou resistente por aquecimento, etc. | polir; fazer as palavras bonitas e concisas}
  \end{Phonetics}
\end{Entry}

\begin{Entry}{烂}{9}{⽕}
  \begin{Phonetics}{烂}{lan4}[][HSK 5]
    \definition{adj.}{macio; pastoso; amassado | podre; deteriorado | quebrado; esfarrapado; gasto | desorganizado; indigno}
    \definition{adv.}{totalmente; extremamente; completamente; expressa um grau muito profundo}
    \definition{v.}{apodrecer; infeccionar; decompor-se}
  \end{Phonetics}
\end{Entry}

\begin{Entry}{牵}{9}{⽜}
  \begin{Phonetics}{牵}{qian1}[][HSK 6]
    \definition{v.}{conduzir (segurando a mão, o cabresto, etc.); puxar | envolver-se | sentir falta; preocupar-se com | controlar; restringir; ser retido; ser constrangido}
  \end{Phonetics}
\end{Entry}

\begin{Entry}{狠}{9}{⽝}
  \begin{Phonetics}{狠}{hen3}[][HSK 6]
    \definition{adj.}{impiedoso; implacável; feroz | firme; resoluto; severo; determinado}
    \definition{adv.}{muito; bastante; bastante | também, frequentemente usado antes de um adjetivo sem intensificar seu significado, ou seja, como um elemento sintático sem sentido}
    \definition{v.}{endurecer (o coração); suprimir (os próprios sentimentos)}
    \variantof{很}
  \seealsoref{很}{hen3}
  \end{Phonetics}
\end{Entry}

\begin{Entry}{独}{9}{⽝}
  \begin{Phonetics}{独}{du2}
    \definition*{s.}{Sobrenome Du}
    \definition{adj.}{só; solteiro | (coloquial) distante | único; só}
    \definition{adv.}{unicamente; somente | sozinho; por si mesmo; em solidão}
    \definition{s.}{idosos sem descendência; os sem filhos}
  \end{Phonetics}
\end{Entry}

\begin{Entry}{独立}{9,5}{⽝、⽴}
  \begin{Phonetics}{独立}{du2li4}[][HSK 4]
    \definition{adj.}{independente; por conta própria | separado; respectivo; descreve algo que é separado e não está em contato com outra coisa}
    \definition{v.}{ficar sozinho | alcançar a independência; tornar-se um país independente; liberdade de um Estado, regime ou organização contra interferência, controle e dominação por forças externas}
  \end{Phonetics}
\end{Entry}

\begin{Entry}{独自}{9,6}{⽝、⾃}
  \begin{Phonetics}{独自}{du2 zi4}[][HSK 4]
    \definition{adv.}{sozinho; por si mesmo; por conta própria}
  \end{Phonetics}
\end{Entry}

\begin{Entry}{独特}{9,10}{⽝、⽜}
  \begin{Phonetics}{独特}{du2te4}[][HSK 4]
    \definition{adj.}{único; distinto; original; especial}
  \end{Phonetics}
\end{Entry}

\begin{Entry}{狭}{9}{⽝}
  \begin{Phonetics}{狭}{xia2}
    \definition{adj.}{estreito (oposto a 广)}
  \seealsoref{广}{guang3}
  \end{Phonetics}
\end{Entry}

\begin{Entry}{玻}{9}{⽟}
  \begin{Phonetics}{玻}{bo1}
    \definition{s.}{vidro}
  \end{Phonetics}
\end{Entry}

\begin{Entry}{玻璃}{9,14}{⽟、⽟}
  \begin{Phonetics}{玻璃}{bo1li5}[][HSK 5]
    \definition[张,块]{s.}{vidro; corpo duro, quebradiço e transparente, geralmente feito de areia, calcário, carbonato de sódio, etc. | \emph{nylon}; plástico; refere-se a determinados plásticos que se assemelham ao vidro.}
  \end{Phonetics}
\end{Entry}

\begin{Entry}{珍}{9}{⽟}
  \begin{Phonetics}{珍}{zhen1}
    \definition{adj.}{precioso; valioso; raro | inestimável}
    \definition{s.}{tesouro | objetos de valor}
    \definition{v.}{valorizar muito; estimar}
  \end{Phonetics}
\end{Entry}

\begin{Entry}{珍贵}{9,9}{⽟、⾙}
  \begin{Phonetics}{珍贵}{zhen1gui4}[][HSK 5]
    \definition{adj.}{raro; valioso; precioso; de grande valor; profundo significado}
  \end{Phonetics}
\end{Entry}

\begin{Entry}{珍珠}{9,10}{⽟、⽟}
  \begin{Phonetics}{珍珠}{zhen1zhu1}[][HSK 5]
    \definition[颗,串]{s.}{pérola; grânulos redondos produzidos nas conchas de certos animais aquáticos, de cor branca, rosa, etc., bonitos e brilhantes, frequentemente usados como adornos}
  \end{Phonetics}
\end{Entry}

\begin{Entry}{珍惜}{9,11}{⽟、⼼}
  \begin{Phonetics}{珍惜}{zhen1xi1}[][HSK 5]
    \definition{v.}{valorizar; estimar; valorizar e evitar o desperdício}
  \end{Phonetics}
\end{Entry}

\begin{Entry}{甚}{9}{⽢}
  \begin{Phonetics}{甚}{shen4}
    \definition{adv.}{muito; extremamente}
    \definition{pron.}{o que}
    \definition{v.}{exceder; superar}
  \seealsoref{什么}{shen2me5}
  \end{Phonetics}
\end{Entry}

\begin{Entry}{甚而}{9,6}{⽢、⽽}
  \begin{Phonetics}{甚而}{shen4'er2}
    \definition{conj.}{(ir) tão longe quanto | tanto que}
  \end{Phonetics}
\end{Entry}

\begin{Entry}{甚至}{9,6}{⽢、⾄}
  \begin{Phonetics}{甚至}{shen4zhi4}[][HSK 4]
    \definition{conj.}{e até mesmo; nem mesmo; para apresentar uma situação típica e especial, para enfatizar a profundidade e a seriedade de uma situação}
  \end{Phonetics}
\end{Entry}

\begin{Entry}{甚或}{9,8}{⽢、⼽}
  \begin{Phonetics}{甚或}{shen4huo4}
    \definition{conj.}{(ir) tão longe quanto | tanto que}
  \end{Phonetics}
\end{Entry}

\begin{Entry}{甭}{9}{⽤}
  \begin{Phonetics}{甭}{beng2}
    \definition{adv.}{não; não precisa; não tem que; contração de 不用}
  \seealsoref{不用}{bu2 yong4}
  \end{Phonetics}
\end{Entry}

\begin{Entry}{界}{9}{⽥}
  \begin{Phonetics}{界}{jie4}[][HSK 6]
    \definition{s.}{fronteira; limite | escopo; extensão | círculos | divisão primária; reino | era geológica | (matemática) limite | mundo; faixa dividida por ocupação, emprego ou gênero, etc. | grupo}
  \end{Phonetics}
\end{Entry}

\begin{Entry}{界碑}{9,13}{⽥、⽯}
  \begin{Phonetics}{界碑}{jie4bei1}
    \definition{s.}{marco de fronteira}
  \end{Phonetics}
\end{Entry}

\begin{Entry}{疯}{9}{⽧}
  \begin{Phonetics}{疯}{feng1}[][HSK 5]
    \definition{adj.}{louco; insano | tolo; leviano | (de uma planta, safra de grãos, etc.) esguia; refere-se ao crescimento vigoroso das plantações, mas sem frutos | com todas as forças; fazer o máximo possível}
    \definition{v.}{jogar sem restrições}
  \end{Phonetics}
\end{Entry}

\begin{Entry}{疯狂}{9,7}{⽧、⽝}
  \begin{Phonetics}{疯狂}{feng1kuang2}[][HSK 5]
    \definition{adj.}{louco; insano; frenético; desenfreado}
  \end{Phonetics}
\end{Entry}

\begin{Entry}{皆}{9}{⽩}
  \begin{Phonetics}{皆}{jie1}
    \definition{adv.}{todos | em todos os casos}
  \end{Phonetics}
\end{Entry}

\begin{Entry}{皇}{9}{⽩}
  \begin{Phonetics}{皇}{huang2}
    \definition*{s.}{Sobrenome Huang}
    \definition{adj.}{grandioso; magnífico}
    \definition{s.}{imperador, o governante supremo de uma dinastia feudal após a Dinastia Qin; soberano}
  \end{Phonetics}
\end{Entry}

\begin{Entry}{皇帝}{9,9}{⽩、⼱}
  \begin{Phonetics}{皇帝}{huang2di4}[][HSK 6]
    \definition[个,位,任]{s.}{imperador; o título do mais alto governante feudal na China começou com o título de Imperador Qin Shi Huang}
  \end{Phonetics}
\end{Entry}

\begin{Entry}{盆}{9}{⽫}
  \begin{Phonetics}{盆}{pen2}[][HSK 5]
    \definition*{s.}{Sobrenome Pen}
    \definition{s.}{bacia; banheira; panela; utensílios para guardar ou lavar coisas}
  \end{Phonetics}
\end{Entry}

\begin{Entry}{盆友}{9,4}{⽫、⼜}
  \begin{Phonetics}{盆友}{pen2you3}
    \definition{s.}{Gíria da \emph{Internet}: amigo (trocadilho com 朋友)}
  \seealsoref{朋友}{peng2you5}
  \end{Phonetics}
\end{Entry}

\begin{Entry}{相}{9}{⽬}
  \begin{Phonetics}{相}{xiang1}
    \definition*{s.}{Sobrenome Xiang}
    \definition{adv.}{uns aos outros; mutuamente | (para uma ação realizada por uma pessoa em relação a outra) | indica a ação de uma parte em relação à outra parte}
    \definition{s.}{qualidade; substância}
    \definition{v.}{ver por si mesmo (se algo ou algo é do seu agrado)}
  \end{Phonetics}
  \begin{Phonetics}{相}{xiang4}
    \definition*{s.}{Sobrenome Xiang}
    \definition{s.}{aparência | postura; porte; postura sentada, em pé, etc. | (física) fase; refere-se a uma parte homogênea de uma substância com a mesma composição e as mesmas propriedades físicas e químicas | fotografia | primeiro-ministro (na China antiga) | ministro; títulos oficiais de certos países | fácies marinha (carvão) | elefante, uma das peças do xadrez chinês | recepcionista (pessoa que ajuda o anfitrião a receber o hóspede); antigamente, referia-se a alguém que ajudava o anfitrião a receber convidados}
    \definition{v.}{olhar e avaliar; observe a aparência das coisas; julgar sua qualidade | assistir; ajudar; auxiliar}
  \end{Phonetics}
\end{Entry}

\begin{Entry}{相互}{9,4}{⽬、⼆}
  \begin{Phonetics}{相互}{xiang1 hu4}[][HSK 3]
    \definition{adj.}{mútuo; recíproco; entre duas pessoas ou coisas}
    \definition{adv.}{mutuamente; um ao outro; tratamento recíproco}
  \end{Phonetics}
\end{Entry}

\begin{Entry}{相反}{9,4}{⽬、⼜}
  \begin{Phonetics}{相反}{xiang1fan3}[][HSK 4]
    \definition{adj.}{oposto; contrário; dois aspectos das coisas são contraditórios e mutuamente exclusivos}
    \definition{conj.}{pelo contrário; usado no início ou no meio de uma frase para indicar uma contradição de significado com o que foi dito anteriormente.}
  \end{Phonetics}
\end{Entry}

\begin{Entry}{相比}{9,4}{⽬、⽐}
  \begin{Phonetics}{相比}{xiang1 bi3}[][HSK 3]
    \definition{v.}{combinar; comparar com | comparar mutuamente, usar uma coisa como padrão, perceber as características de outra coisa ou obter uma opinião}
  \end{Phonetics}
\end{Entry}

\begin{Entry}{相片}{9,4}{⽬、⽚}
  \begin{Phonetics}{相片}{xiang4 pian4}[][HSK 4]
    \definition[张]{s.}{foto; fotografia; uma imagem de uma pessoa ou objeto feita pela exposição de papel fotográfico a um negativo fotográfico e, em seguida, revelando e fixando a imagem.}
  \end{Phonetics}
\end{Entry}

\begin{Entry}{相处}{9,5}{⽬、⼡}
  \begin{Phonetics}{相处}{xiang1chu3}[][HSK 4]
    \definition{v.}{dar-se bem; viver juntos; dar-se bem (uns com os outros); viver uns com os outros; entrar em contato uns com os outros, tratar uns aos outros}
  \end{Phonetics}
\end{Entry}

\begin{Entry}{相似}{9,6}{⽬、⼈}
  \begin{Phonetics}{相似}{xiang1si4}[][HSK 3]
    \definition{v.}{assemelhar-se; ser semelhante; ser parecido}
  \end{Phonetics}
\end{Entry}

\begin{Entry}{相关}{9,6}{⽬、⼋}
  \begin{Phonetics}{相关}{xiang1guan1}[][HSK 3]
    \definition{v.}{estar mutuamente relacionado; estar intimamente relacionado; estar inter-relacionado}
  \end{Phonetics}
\end{Entry}

\begin{Entry}{相同}{9,6}{⽬、⼝}
  \begin{Phonetics}{相同}{xiang1tong2}[][HSK 2]
    \definition{adj.}{semelhante; similar; igual; idêntico; o mesmo; consistentes entre si, sem diferença}
  \end{Phonetics}
\end{Entry}

\begin{Entry}{相当}{9,6}{⽬、⼹}
  \begin{Phonetics}{相当}{xiang1dang1}[][HSK 3]
    \definition{adj.}{adequado; apropriado}
    \definition{adv.}{bastante; razoavelmente; consideravelmente; indica um grau relativamente alto e profundo}
    \definition{v.}{combinar; equilibrar; corresponder a; ser aproximadamente igual a; ser proporcional a}
  \end{Phonetics}
\end{Entry}

\begin{Entry}{相机}{9,6}{⽬、⽊}
  \begin{Phonetics}{相机}{xiang4 ji1}[][HSK 2]
    \definition[台,部,架,个]{s.}{câmera; máquina fotográfica}
    \definition{v.}{ficar atento a uma oportunidade; procurar oportunidades}
  \end{Phonetics}
\end{Entry}

\begin{Entry}{相声}{9,7}{⽬、⼠}
  \begin{Phonetics}{相声}{xiang4sheng5}[][HSK 5]
    \definition[个,段]{s.}{conversa cruzada; diálogo cômico; forma de performance humorística, em que os atores usam piadas, canções e imitações para satirizar e elogiar}
  \end{Phonetics}
\end{Entry}

\begin{Entry}{相应}{9,7}{⽬、⼴}
  \begin{Phonetics}{相应}{xiang1ying4}[][HSK 5]
    \definition{adj.}{Dialeto: barato}
    \definition{v.}{corresponder}
  \end{Phonetics}
\end{Entry}

\begin{Entry}{相宜}{9,8}{⽬、⼧}
  \begin{Phonetics}{相宜}{xiang1yi2}
    \definition{adj.}{adequado | apropriado}
    \definition{v.}{ser adequado ou apropriado}
  \end{Phonetics}
\end{Entry}

\begin{Entry}{相亲}{9,9}{⽬、⼇}
  \begin{Phonetics}{相亲}{xiang1qin1}
    \definition{s.}{encontro às cegas | entrevista arranjada para avaliar a proposta de um parceiro de casamento | apegar-se profundamente um ao outro}
  \end{Phonetics}
\end{Entry}

\begin{Entry}{相信}{9,9}{⽬、⼈}
  \begin{Phonetics}{相信}{xiang1xin4}[][HSK 2]
    \definition{v.}{acreditar em; estar convencido de; ter fé em; acreditar que algo é certo ou verdadeiro sem dúvida}
  \end{Phonetics}
\end{Entry}

\begin{Entry}{相思病}{9,9,10}{⽬、⼼、⽧}
  \begin{Phonetics}{相思病}{xiang1si1bing4}
    \definition{s.}{saudade de amor}
  \end{Phonetics}
\end{Entry}

\begin{Entry}{相等}{9,12}{⽬、⽵}
  \begin{Phonetics}{相等}{xiang1deng3}[][HSK 5]
    \definition{v.}{ser igual a; possuir a mesma quantidade, peso, tamanho e grau}
  \end{Phonetics}
\end{Entry}

\begin{Entry}{相遇}{9,12}{⽬、⾡}
  \begin{Phonetics}{相遇}{xiang1yu4}
    \definition{v.}{encontrar (reunião, encontro, etc.)}
  \end{Phonetics}
\end{Entry}

\begin{Entry}{相聚}{9,14}{⽬、⽿}
  \begin{Phonetics}{相聚}{xiang1ju4}
    \definition{v.}{reunir-se | montar}
  \end{Phonetics}
\end{Entry}

\begin{Entry}{盼}{9}{⽬}
  \begin{Phonetics}{盼}{pan4}
    \definition*{s.}{Sobrenome Pan}
    \definition{adj.}{(de olhos) com preto e branco fortemente contrastados; olhos claros}
    \definition{v.}{olhar | esperar por; ansiar por | sentir falta de ; continuar pensando sobre}
  \end{Phonetics}
\end{Entry}

\begin{Entry}{盼望}{9,11}{⽬、⽉}
  \begin{Phonetics}{盼望}{pan4wang4}[][HSK 6]
    \definition{v.}{esperar por; ansiar por; esperar que algo aconteça em breve}
  \end{Phonetics}
\end{Entry}

\begin{Entry}{省}{9}{⽬}
  \begin{Phonetics}{省}{sheng3}[][HSK 2]
    \definition*{s.}{Sobrenome Sheng}
    \definition{s.}{província; unidade administrativa, subordinada diretamente ao governo central | capital provincial; refere-se à capital da província, localização da administração provincial | abreviação (de palavras)}
    \definition{v.}{economizar; poupar; reduzir o consumo (em oposição a 费) | omitir; deixar de fora}
  \seealsoref{费}{fei4}
  \end{Phonetics}
  \begin{Phonetics}{省}{xing3}
    \definition{v.}{examinar-se criticamente; verificar (os próprios pensamentos, palavras e ações) | visitar (especialmente os pais ou pessoas mais velhas) | estar ciente; tornar-se consciente; compreender; tomar consciência | examinar minuciosamente; inspecionar; escrutinar}
  \end{Phonetics}
\end{Entry}

\begin{Entry}{省力}{9,2}{⽬、⼒}
  \begin{Phonetics}{省力}{sheng3li4}
    \definition{v.}{economizar esforço ou trabalho}
  \end{Phonetics}
\end{Entry}

\begin{Entry}{省心}{9,4}{⽬、⼼}
  \begin{Phonetics}{省心}{sheng3xin1}
    \definition{adj.}{despreocupado}
    \definition{v.}{ser poupado de preocupações | despreocupar-se}
  \end{Phonetics}
\end{Entry}

\begin{Entry}{省长}{9,4}{⽬、⾧}
  \begin{Phonetics}{省长}{sheng3zhang3}
    \definition[位,任]{s.}{governador; governador de uma província}
  \end{Phonetics}
\end{Entry}

\begin{Entry}{省会}{9,6}{⽬、⼈}
  \begin{Phonetics}{省会}{sheng3hui4}
    \definition{s.}{capital da província}
  \end{Phonetics}
\end{Entry}

\begin{Entry}{省却}{9,7}{⽬、⼙}
  \begin{Phonetics}{省却}{sheng3que4}
    \definition{v.}{livrar-se (para economizar espaço) | salvar}
  \end{Phonetics}
\end{Entry}

\begin{Entry}{省俭}{9,9}{⽬、⼈}
  \begin{Phonetics}{省俭}{sheng3jian3}
    \definition{s.}{econômico | frugal}
    \definition{v.}{economizar}
  \end{Phonetics}
\end{Entry}

\begin{Entry}{省城}{9,9}{⽬、⼟}
  \begin{Phonetics}{省城}{sheng3cheng2}
    \definition{s.}{capital da província}
  \end{Phonetics}
\end{Entry}

\begin{Entry}{省悟}{9,10}{⽬、⼼}
  \begin{Phonetics}{省悟}{xing3wu4}
    \definition{v.}{voltar a si | constatar | ver a verdade | acordar para a realidade}
  \end{Phonetics}
\end{Entry}

\begin{Entry}{省钱}{9,10}{⽬、⾦}
  \begin{Phonetics}{省钱}{sheng3 qian2}[][HSK 6]
    \definition{adj.}{barato; não caro}
    \definition{v.}{economizar dinheiro}
  \end{Phonetics}
\end{Entry}

\begin{Entry}{眉}{9}{⽬}
  \begin{Phonetics}{眉}{mei2}
    \definition{s.}{sobrancelha | margem superior}
  \end{Phonetics}
\end{Entry}

\begin{Entry}{眉毛}{9,4}{⽬、⽑}
  \begin{Phonetics}{眉毛}{mei2mao5}
    \definition[根]{s.}{sobrancelha}
  \end{Phonetics}
\end{Entry}

\begin{Entry}{眉头}{9,5}{⽬、⼤}
  \begin{Phonetics}{眉头}{mei2tou2}
    \definition{s.}{testa}
  \end{Phonetics}
\end{Entry}

\begin{Entry}{看}{9}{⽬}
  \begin{Phonetics}{看}{kan1}
    \definition{v.}{cuidar de; tomar conta de; cuidar de; proteger | manter sob vigilância}
  \end{Phonetics}
  \begin{Phonetics}{看}{kan4}[][HSK 1,6]
    \definition{interj.}{Cuidado! (para um perigo)}
    \definition{part.}{tentar, usado depois de outros verbos}
    \definition{v.}{ver; olhar para; observar; fazer contato visual com pessoas ou objetos | pensar; considerar; observar; julgar; observar e analisar | visitar; ver; fazer uma visita | olhar para; considerar; tratar | tratar (um paciente ou uma doença) | cuidar | ficar atento; ficar de olho | depender de; ser dependente de | ler}
  \end{Phonetics}
\end{Entry}

\begin{Entry}{看上去}{9,3,5}{⽬、⼀、⼛}
  \begin{Phonetics}{看上去}{kan4 shang4 qu4}[][HSK 3]
    \definition{adv.}{parece que}
  \end{Phonetics}
\end{Entry}

\begin{Entry}{看不起}{9,4,10}{⽬、⼀、⾛}
  \begin{Phonetics}{看不起}{kan4bu5qi3}[][HSK 4]
    \definition{v.}{desprezar; desdenhar; menosprezar; ter desprezo; olhar de cima para baixo}
  \end{Phonetics}
\end{Entry}

\begin{Entry}{看见}{9,4}{⽬、⾒}
  \begin{Phonetics}{看见}{kan4jian4}[][HSK 1]
    \definition{v.}{ver; avistar; ao olhar, descobrir alguém ou algo}
  \end{Phonetics}
\end{Entry}

\begin{Entry}{看出}{9,5}{⽬、⼐}
  \begin{Phonetics}{看出}{kan4 chu1}[][HSK 5]
    \definition{v.}{decifrar; ver; sondar; encontrar; discernir; perceber | descobrir; estar ciente de}
  \end{Phonetics}
\end{Entry}

\begin{Entry}{看好}{9,6}{⽬、⼥}
  \begin{Phonetics}{看好}{kan4 hao3}[][HSK 6]
    \definition{v.}{elogiar; apreciar; encorajar; acreditar que pessoas ou coisas terão uma boa tendência | estar prestes a surgir uma boa tendência}
  \end{Phonetics}
\end{Entry}

\begin{Entry}{看成}{9,6}{⽬、⼽}
  \begin{Phonetics}{看成}{kan4 cheng2}[][HSK 5]
    \definition{v.}{ser capaz de ver ou assistir | tomar como; olhar como; considerar como | tratar como; considerar como; pensar como; ter como}
  \end{Phonetics}
\end{Entry}

\begin{Entry}{看作}{9,7}{⽬、⼈}
  \begin{Phonetics}{看作}{kan4 zuo4}[][HSK 6]
    \definition{v.}{considerar como; olhar como}
  \end{Phonetics}
\end{Entry}

\begin{Entry}{看来}{9,7}{⽬、⽊}
  \begin{Phonetics}{看来}{kan4 lai2}[][HSK 4]
    \definition{adv.}{parecer; parecer como se (ou embora); refere-se a um julgamento aproximado; expressa um julgamento por observação}
    \definition{v.}{ser considerado; na visão de alguém; na opinião de alguém; expressar a ideia aproximada que o locutor tem da situação}
  \end{Phonetics}
\end{Entry}

\begin{Entry}{看到}{9,8}{⽬、⼑}
  \begin{Phonetics}{看到}{kan4 dao4}[][HSK 1]
    \definition{v.}{ver; avistar}
  \end{Phonetics}
\end{Entry}

\begin{Entry}{看法}{9,8}{⽬、⽔}
  \begin{Phonetics}{看法}{kan4fa3}[][HSK 2]
    \definition[个,种,点]{s.}{opinião; perspectiva; (ponto de) vista; uma maneira de ver uma coisa | opinião desfavorável (ou crítica) sobre alguém}
  \end{Phonetics}
\end{Entry}

\begin{Entry}{看待}{9,9}{⽬、⼻}
  \begin{Phonetics}{看待}{kan4dai4}[][HSK 5]
    \definition{v.}{tratar; considerar; olhar com atenção; ter uma certa atitude ou visão em relação a alguém ou alguma coisa}
  \end{Phonetics}
\end{Entry}

\begin{Entry}{看病}{9,10}{⽬、⽧}
  \begin{Phonetics}{看病}{kan4 bing4}[][HSK 1]
    \definition{v.+compl.}{(de um médico) ver um paciente | (de um paciente) ver (consultar) um médico}
  \end{Phonetics}
\end{Entry}

\begin{Entry}{看起来}{9,10,7}{⽬、⾛、⽊}
  \begin{Phonetics}{看起来}{kan4 qi3 lai5}[][HSK 3]
    \definition{v.}{parecer; aparentar; dar a impressão de (ou como se)}
  \end{Phonetics}
\end{Entry}

\begin{Entry}{看得见}{9,11,4}{⽬、⼻、⾒}
  \begin{Phonetics}{看得见}{kan4 de5 jian4}[][HSK 6]
    \definition{adj.}{perceptível; visível; tangível}
  \end{Phonetics}
\end{Entry}

\begin{Entry}{看得起}{9,11,10}{⽬、⼻、⾛}
  \begin{Phonetics}{看得起}{kan4 de5 qi3}[][HSK 6]
    \definition{v.}{ter uma boa opinião sobre; pensar muito (ou muito) sobre}
  \end{Phonetics}
\end{Entry}

\begin{Entry}{看望}{9,11}{⽬、⽉}
  \begin{Phonetics}{看望}{kan4wang4}[][HSK 4]
    \definition{v.}{ver; visitar; ligar; dar uma olhada; ir até os pais, idosos, professores ou amigos para cumprimentá-los}
  \end{Phonetics}
\end{Entry}

\begin{Entry}{看淡}{9,11}{⽬、⽔}
  \begin{Phonetics}{看淡}{kan4dan4}
    \definition{v.}{considerar sem importância | ser indiferente a (fama, riqueza, etc.) | (de uma economia ou mercado) enfraquecer, ficar mais lento, diminuir a velocidade}
  \end{Phonetics}
\end{Entry}

\begin{Entry}{看管}{9,14}{⽬、⽵}
  \begin{Phonetics}{看管}{kan1 guan3}[][HSK 6]
    \definition{v.}{cuidar; atender | guardar; vigiar; ficar de olho em | assumir o comando; estar no comando}
  \end{Phonetics}
\end{Entry}

\begin{Entry}{矜}{9}{⽭}
  \begin{Phonetics}{矜}{jin1}
    \definition{adj.}{presunçoso; vaidoso | contido; reservado; determinado}
    \definition{v.}{ter pena; simpatizar com; compadecer-se}
  \end{Phonetics}
\end{Entry}

\begin{Entry}{砂}{9}{⽯}
  \begin{Phonetics}{砂}{sha1}
    \variantof{沙}
  \end{Phonetics}
\end{Entry}

\begin{Entry}{砍}{9}{⽯}
  \begin{Phonetics}{砍}{kan3}
    \definition{v.}{cortar}
  \end{Phonetics}
\end{Entry}

\begin{Entry}{砍刀}{9,2}{⽯、⼑}
  \begin{Phonetics}{砍刀}{kan3dao1}
    \definition{s.}{facão | machete}
  \end{Phonetics}
\end{Entry}

\begin{Entry}{砍头}{9,5}{⽯、⼤}
  \begin{Phonetics}{砍头}{kan3tou2}
    \definition{v.}{decapitar}
  \end{Phonetics}
\end{Entry}

\begin{Entry}{砍价}{9,6}{⽯、⼈}
  \begin{Phonetics}{砍价}{kan3jia4}
    \definition{v.}{barganhar | cortar ou derrubar um preço}
  \end{Phonetics}
\end{Entry}

\begin{Entry}{砍伤}{9,6}{⽯、⼈}
  \begin{Phonetics}{砍伤}{kan3shang1}
    \definition{v.}{ferir com lâmina ou machado}
  \end{Phonetics}
\end{Entry}

\begin{Entry}{砍杀}{9,6}{⽯、⽊}
  \begin{Phonetics}{砍杀}{kan3sha1}
    \definition{v.}{atacar com arma branca}
  \end{Phonetics}
\end{Entry}

\begin{Entry}{砍死}{9,6}{⽯、⽍}
  \begin{Phonetics}{砍死}{kan3si3}
    \definition{v.}{matar com um machado}
  \end{Phonetics}
\end{Entry}

\begin{Entry}{砍树}{9,9}{⽯、⽊}
  \begin{Phonetics}{砍树}{kan3shu4}
    \definition{v.}{derrubar árvores}
  \end{Phonetics}
\end{Entry}

\begin{Entry}{砍掉}{9,11}{⽯、⼿}
  \begin{Phonetics}{砍掉}{kan3diao4}
    \definition{v.}{amputar}
  \end{Phonetics}
\end{Entry}

\begin{Entry}{砍断}{9,11}{⽯、⽄}
  \begin{Phonetics}{砍断}{kan3duan4}
    \definition{v.}{cortar}
  \end{Phonetics}
\end{Entry}

\begin{Entry}{研}{9}{⽯}
  \begin{Phonetics}{研}{yan2}
    \definition{s.}{(abreviação)  pesquisador adjunto, 副研}
    \definition{v.}{moer; esmerilhar; triturar; pulverizar | estudar; pesquisar}
  \seealsoref{副研}{fu4yan2}
  \end{Phonetics}
\end{Entry}

\begin{Entry}{研发}{9,5}{⽯、⼜}
  \begin{Phonetics}{研发}{yan2 fa1}[][HSK 6]
    \definition{s.}{pesquisa e desenvolvimento; P\&D}
    \definition{v.}{pesquisar e/ou desenvolver}
  \end{Phonetics}
\end{Entry}

\begin{Entry}{研究}{9,7}{⽯、⽳}
  \begin{Phonetics}{研究}{yan2jiu1}[][HSK 4]
    \definition{v.}{estudar; pesquisar | discutir; considerar}
  \end{Phonetics}
\end{Entry}

\begin{Entry}{研究生}{9,7,5}{⽯、⽳、⽣}
  \begin{Phonetics}{研究生}{yan2 jiu1 sheng1}[][HSK 4]
    \definition[位,名,个,些]{s.}{pós-graduado; estudante de pós-graduação}
  \end{Phonetics}
\end{Entry}

\begin{Entry}{研究所}{9,7,8}{⽯、⽳、⼾}
  \begin{Phonetics}{研究所}{yan2 jiu1 suo3}[][HSK 5]
    \definition[家,个]{s.}{instituto de pesquisa; instituição de pesquisa científica envolvida em pesquisas em um determinado campo}
  \end{Phonetics}
\end{Entry}

\begin{Entry}{研制}{9,8}{⽯、⼑}
  \begin{Phonetics}{研制}{yan2 zhi4}[][HSK 4]
    \definition{v.}{desenvolver; fabricar; produzir | triturar; (medicina chinesa) moer}
  \end{Phonetics}
\end{Entry}

\begin{Entry}{砖}{9}{⽯}
  \begin{Phonetics}{砖}{zhuan1}
    \definition[块]{s.}{tijolo}
  \end{Phonetics}
\end{Entry}

\begin{Entry}{祖}{9}{⽰}
  \begin{Phonetics}{祖}{zu3}
    \definition*{s.}{Sobrenome Zu}
    \definition{s.}{avô; geração anterior dos pais | ancestral; antepassado | fundador (de um negócio, facção, seita religiosa, etc.); originador; fundador; mestre fundador}
  \end{Phonetics}
\end{Entry}

\begin{Entry}{祖父}{9,4}{⽰、⽗}
  \begin{Phonetics}{祖父}{zu3fu4}[][HSK 6]
    \definition{s.}{avô (paterno)}
  \end{Phonetics}
\end{Entry}

\begin{Entry}{祖母}{9,5}{⽰、⽏}
  \begin{Phonetics}{祖母}{zu3 mu3}[][HSK 6]
    \definition{s.}{avó (paterna)}
  \end{Phonetics}
\end{Entry}

\begin{Entry}{祖国}{9,8}{⽰、⼞}
  \begin{Phonetics}{祖国}{zu3guo2}[][HSK 6]
    \definition{s.}{país; pátria; próprio país}
  \end{Phonetics}
\end{Entry}

\begin{Entry}{祝}{9}{⽰}
  \begin{Phonetics}{祝}{zhu4}[][HSK 3]
    \definition*{s.}{Sobrenome Zhu}
    \definition{v.}{expressar bons votos; desejar; abençoar | rezar aos deuses ou espíritos para obter bênçãos}
  \end{Phonetics}
\end{Entry}

\begin{Entry}{祝好}{9,6}{⽰、⼥}
  \begin{Phonetics}{祝好}{zhu4hao3}
    \definition{expr.}{desejo-lhe tudo de melhor! (ao encerrar uma correspondência)}
  \end{Phonetics}
\end{Entry}

\begin{Entry}{祝寿}{9,7}{⽰、⼨}
  \begin{Phonetics}{祝寿}{zhu4shou4}
    \definition{v.}{dar parabéns pelo aniversário (a uma pessoa idosa)}
  \end{Phonetics}
\end{Entry}

\begin{Entry}{祝贺}{9,9}{⽰、⾙}
  \begin{Phonetics}{祝贺}{zhu4he4}[][HSK 5]
    \definition[个]{s.}{congratulações; felicitações}
    \definition{v.}{congratular; felicitar; parabenizar}
  \end{Phonetics}
\end{Entry}

\begin{Entry}{祝酒}{9,10}{⽰、⾣}
  \begin{Phonetics}{祝酒}{zhu4jiu3}
    \definition{v.}{parabenizar e fazer um brinde | brindar}
  \end{Phonetics}
\end{Entry}

\begin{Entry}{祝颂}{9,10}{⽰、⾴}
  \begin{Phonetics}{祝颂}{zhu4song4}
    \definition{v.}{expressar bons desejos}
  \end{Phonetics}
\end{Entry}

\begin{Entry}{祝祷}{9,11}{⽰、⽰}
  \begin{Phonetics}{祝祷}{zhu4dao3}
    \definition{v.}{rezar | orar}
  \end{Phonetics}
\end{Entry}

\begin{Entry}{祝谢}{9,12}{⽰、⾔}
  \begin{Phonetics}{祝谢}{zhu4xie4}
    \definition{v.}{agradecer | dar parabéns}
  \end{Phonetics}
\end{Entry}

\begin{Entry}{祝福}{9,13}{⽰、⽰}
  \begin{Phonetics}{祝福}{zhu4fu2}[][HSK 4]
    \definition[个]{s.}{bênção; benzedura; benzimento; originalmente, referia-se à oração para obter as bênçãos de Deus, mas, mais tarde, refere-se ao desejo de paz e felicidade às pessoas}
    \definition{v.}{desejar boa sorte a alguém}
  \end{Phonetics}
\end{Entry}

\begin{Entry}{祝愿}{9,14}{⽰、⽕}
  \begin{Phonetics}{祝愿}{zhu4 yuan4}[][HSK 6]
    \definition{v.}{desejar; expressar bons desejos}
  \end{Phonetics}
\end{Entry}

\begin{Entry}{神}{9}{⽰}
  \begin{Phonetics}{神}{shen2}[][HSK 5]
    \definition*{s.}{Deus | Sobrenome Shen}
    \definition{adj.}{inteligente; esperto | mágico; sobrenatural}
    \definition[个,位,尊,名]{s.}{divindade; deidade | espírito; mente; refere-se ao espírito, energia ou atenção de uma pessoa | olhar; expressão; expressões que refletem o estado interior}
  \end{Phonetics}
\end{Entry}

\begin{Entry}{神奇}{9,8}{⽰、⼤}
  \begin{Phonetics}{神奇}{shen2qi2}[][HSK 5]
    \definition{adj.}{mágico; peculiar; místico; milagroso; faz as pessoas se sentirem muito revigoradas; é completamente inesperado e geralmente traz boas influências}
    \definition{adj.}{mágico; peculiar; místico; milagroso; algo que parece muito novo; algo que ninguém imaginaria, mas que geralmente traz bons resultados}
  \end{Phonetics}
\end{Entry}

\begin{Entry}{神明}{9,8}{⽰、⽇}
  \begin{Phonetics}{神明}{shen2ming2}
    \definition{s.}{divindades | deuses}
  \end{Phonetics}
\end{Entry}

\begin{Entry}{神经}{9,8}{⽰、⽷}
  \begin{Phonetics}{神经}{shen2jing1}[][HSK 5]
    \definition{adj.}{excêntrico; estranho; peculiar; descreve anormalidade neurológica}
    \definition[根,条]{s.}{nervo; um tipo de tecido presente no corpo humano ou animal que conecta o cérebro aos órgãos, transmitindo as sensações ao cérebro e as informações do cérebro aos órgãos}
  \end{Phonetics}
\end{Entry}

\begin{Entry}{神经病学}{9,8,10,8}{⽰、⽷、⽧、⼦}
  \begin{Phonetics}{神经病学}{shen2jing1bing4 xue2}
    \definition{s.}{neurologia}
  \end{Phonetics}
\end{Entry}

\begin{Entry}{神经病的}{9,8,10,8}{⽰、⽷、⽧、⽩}
  \begin{Phonetics}{神经病的}{shen2jing1bing4 de5}
    \definition{adj.}{neuropático; neurótico}
  \end{Phonetics}
\end{Entry}

\begin{Entry}{神话}{9,8}{⽰、⾔}
  \begin{Phonetics}{神话}{shen2hua4}[][HSK 4]
    \definition[段,篇]{s.}{mito; mitologia; conto de fadas; refere-se a deuses e deusas lendários e histórias de heróis antigos deificados | lorota; refere-se a alegações ridículas e infundadas}
  \end{Phonetics}
\end{Entry}

\begin{Entry}{神秘}{9,10}{⽰、⽲}
  \begin{Phonetics}{神秘}{shen2mi4}[][HSK 4]
    \definition{adj.}{místico; misterioso}
  \end{Phonetics}
\end{Entry}

\begin{Entry}{神兽}{9,11}{⽰、⼋}
  \begin{Phonetics}{神兽}{shen2shou4}
    \definition{s.}{animal mitológico | fera}
  \end{Phonetics}
\end{Entry}

\begin{Entry}{神情}{9,11}{⽰、⼼}
  \begin{Phonetics}{神情}{shen2 qing2}[][HSK 5]
    \definition{s.}{aparência; expressão; atividades internas reveladas no rosto das pessoas}
  \end{Phonetics}
\end{Entry}

\begin{Entry}{神器}{9,16}{⽰、⼝}
  \begin{Phonetics}{神器}{shen2qi4}
    \definition{s.}{objeto mágico | objeto simbólico do poder imperial | arma fina | ferramenta muito útil}
  \end{Phonetics}
\end{Entry}

\begin{Entry}{秋}{9}{⽲}
  \begin{Phonetics}{秋}{qiu1}
    \definition*{s.}{Sobrenome Qiu}
    \definition{s.}{outono | época da colheita; a estação em que as colheitas amadurecem; colheitas maduras no outono | ano; refere-se a um ano | um período de tempo (geralmente conturbado)}
  \end{Phonetics}
\end{Entry}

\begin{Entry}{秋天}{9,4}{⽲、⼤}
  \begin{Phonetics}{秋天}{qiu1 tian1}[][HSK 2]
    \definition[个,段,季,番]{s.}{outono}
  \end{Phonetics}
\end{Entry}

\begin{Entry}{秋季}{9,8}{⽲、⼦}
  \begin{Phonetics}{秋季}{qiu1 ji4}[][HSK 4]
    \definition[个]{s.}{outono; terceiro trimestre do ano, segundo o costume chinês, refere-se ao período de três meses entre o outono e o inverno, também se refere aos sétimo, oitavo e nono meses do calendário lunar}
  \end{Phonetics}
\end{Entry}

\begin{Entry}{种}{9}{⽲}
  \begin{Phonetics}{种}{zhong3}[][HSK 3,4]
    \definition{clas.}{indica tipo, usado para pessoas e qualquer coisa}
    \definition{s.}{espécie | etnia | semente; estirpe; linhagem; material para reprodução biológica em cadeia | coragem; determinação; garra; força de caráter; refere-se à coragem ou determinação}
  \end{Phonetics}
  \begin{Phonetics}{种}{zhong4}
    \definition{v.}{semear; cultivar; plantar}
  \end{Phonetics}
\end{Entry}

\begin{Entry}{种子}{9,3}{⽲、⼦}
  \begin{Phonetics}{种子}{zhong3zi5}[][HSK 3]
    \definition[颗,粒,个,号]{s.}{semente; um órgão exclusivo de certas plantas, geralmente composto de três partes: tegumento, embrião e endosperma, as sementes podem germinar e se tornar novas plantas sob certas condições | jogador classificado; durante a competição, nas eliminatórias, os jogadores mais fortes de cada equipe são escalados}
  \end{Phonetics}
\end{Entry}

\begin{Entry}{种地}{9,6}{⽲、⼟}
  \begin{Phonetics}{种地}{zhong4di4}
    \definition{v.}{cultivar | trabalhar a terra}
  \end{Phonetics}
\end{Entry}

\begin{Entry}{种种}{9,9}{⽲、⽲}
  \begin{Phonetics}{种种}{zhong3 zhong3}[][HSK 6]
    \definition{adj.}{todos os tipos de; uma variedade de}
  \end{Phonetics}
\end{Entry}

\begin{Entry}{种类}{9,9}{⽲、⽶}
  \begin{Phonetics}{种类}{zhong3lei4}[][HSK 4]
    \definition[个]{s.}{espécie; classe; tipo; variedade; categoria; classificação de alguma coisa de acordo com sua natureza e características}
  \end{Phonetics}
\end{Entry}

\begin{Entry}{种族灭绝}{9,11,5,9}{⽲、⽅、⽕、⽷}
  \begin{Phonetics}{种族灭绝}{zhong3zu2mie4jue2}
    \definition{s.}{genocídio | extinção étnica}
  \end{Phonetics}
\end{Entry}

\begin{Entry}{种麻}{9,11}{⽲、⿇}
  \begin{Phonetics}{种麻}{zhong3ma2}
    \definition{s.}{planta de cânhamo (feminina)}
  \end{Phonetics}
\end{Entry}

\begin{Entry}{种植}{9,12}{⽲、⽊}
  \begin{Phonetics}{种植}{zhong4zhi2}[][HSK 4]
    \definition{v.}{plantar; crescer; cultivar; enterrar as sementes de uma planta no solo; plantar as mudas de uma planta no solo}
  \end{Phonetics}
\end{Entry}

\begin{Entry}{种薯}{9,16}{⽲、⾋}
  \begin{Phonetics}{种薯}{zhong3shu3}
    \definition{s.}{tubérculo semente}
  \end{Phonetics}
\end{Entry}

\begin{Entry}{科}{9}{⽲}
  \begin{Phonetics}{科}{ke1}[][HSK 2]
    \definition*{s.}{Sobrenome Ke}
    \definition{s.}{um ramo de estudo acadêmico ou profissional |uma divisão ou subdivisão de uma unidade administrativa | família | instruções de palco no drama chinês clássico; nos roteiros de peças clássicas, termos usados para indicar as ações dos personagens | nível; classificação; categoria | sessão de exames; refere-se às disciplinas, notas e anos das provas para a seleção de candidatos a cargos públicos militares e civis na antiguidade | tecnológico | assunto | lei; regulamento; decreto | penalidade; pena; punição | treinamento profissional ou formal; curso profissionalizante}
    \definition{v.}{proferir uma sentença (penal)}
  \end{Phonetics}
\end{Entry}

\begin{Entry}{科技}{9,7}{⽲、⼿}
  \begin{Phonetics}{科技}{ke1 ji4}[][HSK 3]
    \definition{s.}{ciência e tecnologia}
  \end{Phonetics}
\end{Entry}

\begin{Entry}{科学}{9,8}{⽲、⼦}
  \begin{Phonetics}{科学}{ke1xue2}[][HSK 2]
    \definition{adj.}{científico; em conformidade com as leis da ciência}
    \definition[门,个,种]{s.}{ciência; um conjunto de conhecimentos que reflete as leis objetivas da natureza, da sociedade, do pensamento, etc.}
  \end{Phonetics}
\end{Entry}

\begin{Entry}{科学家}{9,8,10}{⽲、⼦、⼧}
  \begin{Phonetics}{科学家}{ke1xue2jia1}
    \definition[位,名,个]{s.}{cientista; pessoas com realizações significativas no campo da pesquisa científica}
  \end{Phonetics}
\end{Entry}

\begin{Entry}{科研}{9,9}{⽲、⽯}
  \begin{Phonetics}{科研}{ke1 yan2}[][HSK 6]
    \definition{s.}{pesquisa científica}
    \definition{v.}{envolver-se em pesquisa científica}
  \end{Phonetics}
\end{Entry}

\begin{Entry}{秒}{9}{⽲}
  \begin{Phonetics}{秒}{miao3}[][HSK 5]
    \definition{adv.}{instantaneamente}
    \definition{s.}{segundo (unidade de tempo) | segundo (unidade de medida angular)}
  \end{Phonetics}
\end{Entry}

\begin{Entry}{穿}{9}{⽳}
  \begin{Phonetics}{穿}{chuan1}[][HSK 1]
    \definition{adj.}{direto; através; usado após certos verbos, indica um estado de revelação completa}
    \definition{s.}{vestuário; roupas; refere-se a roupas, sapatos, meias, etc.}
    \definition{v.}{usar; vestir; estar vestido; ter \dots vestido;  vestir roupas, sapatos, meias, etc. | perfurar através de; penetrar; formar orifícios por meio de cinzéis, brocas ou pontas afiadas | enfiar; amarrar; usar cordas e fios para ligar coisas | passar por; atravessar; passar por; através de (buracos, fendas, espaços vazios, etc.)}
  \end{Phonetics}
\end{Entry}

\begin{Entry}{穿上}{9,3}{⽳、⼀}
  \begin{Phonetics}{穿上}{chuan1 shang4}[][HSK 4]
    \definition{v.}{vestir (roupas, etc.); colocar roupas}
  \end{Phonetics}
\end{Entry}

\begin{Entry}{突}{9}{⽳}
  \begin{Phonetics}{突}{tu1}
    \definition{adv.}{de repente; abruptamente; inesperadamente}
    \definition{s.}{chaminé}
    \definition{v.}{avançar rapidamente; atacar | projetar; destacar-se | romper | projetar-se; inchar; fazer bojo}
  \end{Phonetics}
\end{Entry}

\begin{Entry}{突出}{9,5}{⽳、⼐}
  \begin{Phonetics}{突出}{tu1chu1}[][HSK 3]
    \definition{adj.}{proeminente; excelente; mais que a média}
    \definition{v.}{romper | enfatizar; destacar; dar destaque a | sobressair; projetar-se; destacar-se}
  \end{Phonetics}
\end{Entry}

\begin{Entry}{突破}{9,10}{⽳、⽯}
  \begin{Phonetics}{突破}{tu1po4}[][HSK 5]
    \definition{v.}{romper; fazer uma descoberta revolucionária; concentrar esforços em um único ponto de ataque, reunir o sucesso | quebrar (limite); superar (dificuldade); superar dificuldades; ultrapassar os números ou limites anteriores, superar recordes anteriores, etc.; romper com as limitações e restrições anteriores}
  \end{Phonetics}
\end{Entry}

\begin{Entry}{突然}{9,12}{⽳、⽕}
  \begin{Phonetics}{突然}{tu1ran2}[][HSK 3]
    \definition{adj.}{repentino; abrupto; inesperado}
    \definition{adv.}{de repente; abruptamente; inesperadamente; subitamente}
  \end{Phonetics}
\end{Entry}

\begin{Entry}{竖}{9}{⽴}
  \begin{Phonetics}{竖}{shu4}
    \definition*{s.}{Sobrenome Shu}
    \definition{adj.}{vertical; ereto; perpendicular ao solo}
    \definition{s.}{traço vertical (em caracteres chineses) | empregados domésticos; jovens criados}
    \definition{v.}{colocar em pé; erguer; ficar de pé; colocar o objeto perpendicular ao solo}
  \end{Phonetics}
\end{Entry}

\begin{Entry}{类}{9}{⽶}
  \begin{Phonetics}{类}{lei4}[][HSK 3]
    \definition*{s.}{Sobrenome Lei}
    \definition{clas.}{tipo; espécie; categoria usada para pessoas ou coisas}
    \definition{s.}{classe; categoria; tipo; variedade; a combinação de muitas coisas semelhantes ou iguais}
    \definition{v.}{assemelhar-se a; ser semelhante a}
  \end{Phonetics}
\end{Entry}

\begin{Entry}{类似}{9,6}{⽶、⼈}
  \begin{Phonetics}{类似}{lei4si4}[][HSK 3]
    \definition{adj.}{semelhante; análogo}
  \end{Phonetics}
\end{Entry}

\begin{Entry}{类型}{9,9}{⽶、⼟}
  \begin{Phonetics}{类型}{lei4xing2}[][HSK 4]
    \definition[种,个]{s.}{tipo; espécie; categoria; tipos formados por coisas com características comuns}
  \end{Phonetics}
\end{Entry}

\begin{Entry}{结}{9}{⽷}
  \begin{Phonetics}{结}{jie1}
    \definition{v.}{dar (frutos); formar (sementes); produzir frutos ou sementes (uma planta)}
  \end{Phonetics}
  \begin{Phonetics}{结}{jie2}[][HSK 4]
    \definition*{s.}{Sobrenome Jie}
    \definition{s.}{nó | declaração juramentada; garantia por escrito; documento que, antigamente, significava um reconhecimento de encerramento ou uma garantia de responsabilidade}
    \definition{v.}{amarrar; tricotar; dar nó; tecer | formar; forjar; cimentar; solidificar | resolver; concluir | combinar; formar um relacionamento}
  \end{Phonetics}
\end{Entry}

\begin{Entry}{结合}{9,6}{⽷、⼝}
  \begin{Phonetics}{结合}{jie2he2}[][HSK 3]
    \definition{v.}{ligar; unir; combinar; integrar; formar uma relação estreita entre pessoas ou coisas | casar-se; unir-se em matrimônio; referir-se especificamente a casais}
  \end{Phonetics}
\end{Entry}

\begin{Entry}{结论}{9,6}{⽷、⾔}
  \begin{Phonetics}{结论}{jie2lun4}[][HSK 4]
    \definition[个]{s.}{conclusão; palavra final sobre uma pessoa ou coisa após investigação e pesquisa | veredito; julgamento deduzido de premissas também é chamado de conclusão}
  \end{Phonetics}
\end{Entry}

\begin{Entry}{结局}{9,7}{⽷、⼫}
  \begin{Phonetics}{结局}{jie2ju2}
    \definition{s.}{conclusão | fim | final}
  \end{Phonetics}
\end{Entry}

\begin{Entry}{结束}{9,7}{⽷、⽊}
  \begin{Phonetics}{结束}{jie2shu4}[][HSK 3]
    \definition{v.}{finalizar; fechar; terminar; concluir; encerrar; desenvolver ou avançar até a fase final, sem continuidade}
  \end{Phonetics}
\end{Entry}

\begin{Entry}{结束工作}{9,7,3,7}{⽷、⽊、⼯、⼈}
  \begin{Phonetics}{结束工作}{jie2shu4gong1zuo4}
    \definition{s.}{trabalho final}
    \definition{v.}{terminar o trabalho}
  \end{Phonetics}
\end{Entry}

\begin{Entry}{结束区}{9,7,4}{⽷、⽊、⼖}
  \begin{Phonetics}{结束区}{jie2shu4 qu1}
    \definition{s.}{zona final}
  \end{Phonetics}
\end{Entry}

\begin{Entry}{结束文本}{9,7,4,5}{⽷、⽊、⽂、⽊}
  \begin{Phonetics}{结束文本}{jie2shu4 wen2ben3}
    \definition{s.}{texto final}
  \end{Phonetics}
\end{Entry}

\begin{Entry}{结束剂}{9,7,8}{⽷、⽊、⼑}
  \begin{Phonetics}{结束剂}{jie2shu4 ji4}
    \definition{s.}{finalizador}
  \end{Phonetics}
\end{Entry}

\begin{Entry}{结束语}{9,7,9}{⽷、⽊、⾔}
  \begin{Phonetics}{结束语}{jie2shu4yu3}
    \definition{s.}{conclusões finais | considerações finais}
  \end{Phonetics}
\end{Entry}

\begin{Entry}{结束辩论}{9,7,16,6}{⽷、⽊、⾟、⾔}
  \begin{Phonetics}{结束辩论}{jie2shu4 bian4 lun4}
    \definition{s.}{debate de encerramento}
  \end{Phonetics}
\end{Entry}

\begin{Entry}{结社自由}{9,7,6,5}{⽷、⽰、⾃、⽥}
  \begin{Phonetics}{结社自由}{jie2she4zi4you2}
    \definition{s.}{(constitucional) liberdade de associação}
  \end{Phonetics}
\end{Entry}

\begin{Entry}{结实}{9,8}{⽷、⼧}
  \begin{Phonetics}{结实}{jie1shi5}[][HSK 3]
    \definition{adj.}{sólido; resistente; durável | forte; resistente; robusto}
  \end{Phonetics}
\end{Entry}

\begin{Entry}{结构}{9,8}{⽷、⽊}
  \begin{Phonetics}{结构}{jie2gou4}[][HSK 4]
    \definition[个]{s.}{estrutura; composição; construção; formação; constituição; tecido; forma; sistematização; mecânica; organização | arquitetura; estrutura; construção; construção de partes de edifícios com suporte de carga ou com carga externa | Geologia: textura}[这些矿物质具有致密结构。===Esses minerais têm uma estrutura densa.]
  \end{Phonetics}
\end{Entry}

\begin{Entry}{结果}{9,8}{⽷、⽊}
  \begin{Phonetics}{结果}{jie1guo3}
    \definition{v.}{dar frutos}
  \end{Phonetics}
  \begin{Phonetics}{结果}{jie2guo3}[][HSK 2]
    \definition{conj.}{como resultado | no final}
    \definition{s.}{resultado | conclusão | consequência}
    \definition{v.}{despachar | matar}
  \end{Phonetics}
\end{Entry}

\begin{Entry}{结婚}{9,11}{⽷、⼥}
  \begin{Phonetics}{结婚}{jie2hun1}[][HSK 3]
    \definition{v.+compl.}{casar; casar-se; casar-se bem;}
  \end{Phonetics}
\end{Entry}

\begin{Entry}{结婚礼服}{9,11,5,8}{⽷、⼥、⽰、⽉}
  \begin{Phonetics}{结婚礼服}{jie2hun1 li3 fu2}
    \definition{s.}{vestido de casamento}
  \end{Phonetics}
\end{Entry}

\begin{Entry}{绕}{9}{⽷}
  \begin{Phonetics}{绕}{rao4}[][HSK 5]
    \definition*{s.}{Sobrenome Rao}
    \definition{v.}{enrolar; bobinar; rebobinar | mover-se em círculo; girar; revolver | fazer um desvio; contornar; dar a volta | confundir; desorientar}
  \end{Phonetics}
\end{Entry}

\begin{Entry}{绘}{9}{⽷}
  \begin{Phonetics}{绘}{hui4}
    \definition{v.}{pintar; desenhar}
  \end{Phonetics}
\end{Entry}

\begin{Entry}{绘画}{9,8}{⽷、⽥}
  \begin{Phonetics}{绘画}{hui4 hua4}[][HSK 6]
    \definition{s.}{desenho; pintura}
    \definition{v.}{desenhar; pintar}
  \end{Phonetics}
\end{Entry}

\begin{Entry}{给}{9}{⽷}
  \begin{Phonetics}{给}{gei3}[][HSK 1]
    \definition{prep.}{por; expressa significado passivo; tem o mesmo significado que 被, 叫; pode ser seguido pelo agente da ação; o agente da ação também pode não aparecer na frase | para; a; seguido por quem se beneficia da ação; igual a 为 | em direção a; seguido pelo destinatário da ação; o mesmo que 向 | indica transmissão}
    \definition{v.}{dar; conceder; fazer com que a outra parte obtenha algo | passar; pagar; indicar que a outra pessoa faça algo | deixar; permitir que alguém faça algo; autorizar alguém a fazer algo}
    \definition{v.aux.}{usado antes de verbos predicativos que expressam passividade, disposição, etc., para reforçar o tom}
  \seealsoref{被}{bei4}
  \seealsoref{叫}{jiao4}
  \seealsoref{为}{wei4}
  \seealsoref{向}{xiang4}
  \end{Phonetics}
  \begin{Phonetics}{给}{ji3}
    \definition{adj.}{abundante; próspero; bem provido para}
    \definition{v.}{fornecer; prover}
  \end{Phonetics}
\end{Entry}

\begin{Entry}{给予}{9,4}{⽷、⼅}
  \begin{Phonetics}{给予}{ji3yu3}[][HSK 6]
    \definition{v.}{dar; conceder; dar em troca}
  \end{Phonetics}
\end{Entry}

\begin{Entry}{给……打电话}{9,5,5,8}{⽷、⼿、⽥、⾔}
  \begin{Phonetics}{给……打电话}{gei3 da3 dian4 hua4}
    \definition{expr.}{dar um telefonema para alguém}
  \seealsoref{打电话}{da3 dian4 hua4}
  \end{Phonetics}
\end{Entry}

\begin{Entry}{绝}{9}{⽷}
  \begin{Phonetics}{绝}{jue2}[][HSK 6]
    \definition{adj.}{exausto; esgotado; acabado | desesperado; sem esperança | único; soberbo; incomparável | não deixar margem de manobra; não fazer concessões; intransigente}
    \definition{adv.}{extremamente; mais | (antes de uma negativa) absolutamente; no mínimo; por qualquer meio; em qualquer conta}
    \definition{s.}{(literário) jueju, um poema de quatro linhas}
    \definition{v.}{cortar; romper | parar de respirar; morrer}
  \end{Phonetics}
\end{Entry}

\begin{Entry}{绝大多数}{9,3,6,13}{⽷、⼤、⼣、⽁}
  \begin{Phonetics}{绝大多数}{jue2 da4 duo1 shu4}[][HSK 6]
    \definition{expr.}{maioria absoluta | uma maioria esmagadora}
  \end{Phonetics}
\end{Entry}

\begin{Entry}{绝不}{9,4}{⽷、⼀}
  \begin{Phonetics}{绝不}{jue2bu4}
    \definition{adv.}{definitivamente não | de forma alguma | sob nenhuma circunstância}
  \end{Phonetics}
\end{Entry}

\begin{Entry}{绝对}{9,5}{⽷、⼨}
  \begin{Phonetics}{绝对}{jue2dui4}[][HSK 3]
    \definition{adj.}{absoluto; sem condições; sem restrições | absoluto; extremo; incompleto; sem margem para negociação ou alteração}
    \definition{adv.}{absolutamente; completamente; com certeza}
  \end{Phonetics}
\end{Entry}

\begin{Entry}{绝招}{9,8}{⽷、⼿}
  \begin{Phonetics}{绝招}{jue2zhao1}
    \definition{s.}{habilidade única | movimento delicado inesperado (como último recurso) | golpe de mestre | golpe final}
  \end{Phonetics}
\end{Entry}

\begin{Entry}{绝版}{9,8}{⽷、⽚}
  \begin{Phonetics}{绝版}{jue2ban3}
    \definition{adj.}{esgotado | fora de catálogo}
  \end{Phonetics}
\end{Entry}

\begin{Entry}{绝望}{9,11}{⽷、⽉}
  \begin{Phonetics}{绝望}{jue2 wang4}[][HSK 5]
    \definition{v.+compl.}{desesperar; desistir de toda esperança; perder toda esperança de}
  \end{Phonetics}
\end{Entry}

\begin{Entry}{统}{9}{⽷}
  \begin{Phonetics}{统}{tong3}
    \definition{adv.}{todos; juntos; de forma unificada | inteiramente; totalmente}
    \definition{s.}{interligado; inter-relacionado | sistema interconectado | qualquer parte em forma de tubo de uma peça de roupa, etc.; o mesmo que 筒}
    \definition{v.}{reunir em um; unir | unir; liderar; comandar}
  \seealsoref{筒}{tong3}
  \end{Phonetics}
\end{Entry}

\begin{Entry}{统一}{9,1}{⽷、⼀}
  \begin{Phonetics}{统一}{tong3yi1}[][HSK 4]
    \definition{adj.}{unificado; unitário; centralizado; consistente}
    \definition{v.}{unificar; unir; integrar; padronizar}
  \end{Phonetics}
\end{Entry}

\begin{Entry}{统计}{9,4}{⽷、⾔}
  \begin{Phonetics}{统计}{tong3ji4}[][HSK 4]
    \definition{v.}{compilar estatísticas; refere-se à realização de trabalho estatístico, ou seja, coletar, reunir, analisar e extrapolar dados sobre um fenômeno | somar; adicionar; contar}
  \end{Phonetics}
\end{Entry}

\begin{Entry}{罚}{9}{⽹}
  \begin{Phonetics}{罚}{fa2}[][HSK 5]
    \definition{s.}{punição; penalidade}
    \definition{v.}{punir; penalizar; multar; confiscar}
  \end{Phonetics}
\end{Entry}

\begin{Entry}{罚款}{9,12}{⽹、⽋}
  \begin{Phonetics}{罚款}{fa2kuan3}[][HSK 5]
    \definition[笔,次,宗]{s.}{multa; penalidade; refere-se ao dinheiro pago por uma pessoa ou entidade de acordo com as disposições de um delito ou violação de contrato ou contrato}
    \definition{v.+compl.}{multar; penalizar; exigir, de acordo com os regulamentos, uma determinada quantia de dinheiro de uma pessoa ou entidade que tenha violado a lei ou descumprido um regulamento ou contrato}
  \end{Phonetics}
\end{Entry}

\begin{Entry}{美}{9}{⽺}
  \begin{Phonetics}{美}{mei3}[][HSK 3]
    \definition*{s.}{Abreviatura de América, 美洲 | Abreviatura de Estados Unidos da América, 美国 | As Américas, 美洲}
    \definition{adj.}{belo; bonito (oposto de 丑) | muito satisfatório; bom; agradável}
    \definition{s.}{beleza (oposto de 丑)}
    \definition{v.}{embelezar; tornar mais bonito | estar satisfeito consigo mesmo; orgulhar-se; sentir-se presunçoso}
  \seealsoref{丑}{chou3}
  \seealsoref{美国}{mei3guo2}
  \seealsoref{美洲}{mei3zhou1}
  \end{Phonetics}
\end{Entry}

\begin{Entry}{美女}{9,3}{⽺、⼥}
  \begin{Phonetics}{美女}{mei3 nv3}[][HSK 4]
    \definition[个,位,名,些]{s.}{beldade; mulher bonita; uma jovem linda}
  \end{Phonetics}
\end{Entry}

\begin{Entry}{美元}{9,4}{⽺、⼉}
  \begin{Phonetics}{美元}{mei3yuan2}[][HSK 3]
    \definition*[元,笔,沓]{s.}{Dólar Americano; a moeda dos Estados Unidos}
  \end{Phonetics}
\end{Entry}

\begin{Entry}{美术}{9,5}{⽺、⽊}
  \begin{Phonetics}{美术}{mei3shu4}[][HSK 3]
    \definition[种]{s.}{arte; artes plásticas: arte que ocupa um determinado espaço, compõe imagens estéticas e permite que as pessoas apreciem visualmente, incluindo pintura, escultura, arquitetura, etc. | pintura; pintura tradicional chinesa}
  \end{Phonetics}
\end{Entry}

\begin{Entry}{美甲}{9,5}{⽺、⽥}
  \begin{Phonetics}{美甲}{mei3jia3}
    \definition{s.}{manicure e/ou pedicure}
  \end{Phonetics}
\end{Entry}

\begin{Entry}{美好}{9,6}{⽺、⼥}
  \begin{Phonetics}{美好}{mei3 hao3}[][HSK 3]
    \definition{adj.}{bem; feliz; glorioso; descreve a vida, os desejos, etc. como sendo muito bons e satisfatórios}
  \end{Phonetics}
\end{Entry}

\begin{Entry}{美丽}{9,7}{⽺、⼀}
  \begin{Phonetics}{美丽}{mei3li4}[][HSK 3]
    \definition{adj.}{bonito; lindo; capaz de proporcionar uma sensação de beleza}
  \end{Phonetics}
\end{Entry}

\begin{Entry}{美味}{9,8}{⽺、⼝}
  \begin{Phonetics}{美味}{mei3wei4}
    \definition{adj.}{delicioso}
    \definition{s.}{comida deliciosa | delicadeza (\emph{delicacy})}
  \end{Phonetics}
\end{Entry}

\begin{Entry}{美国}{9,8}{⽺、⼞}
  \begin{Phonetics}{美国}{mei3guo2}
    \definition*{s.}{Estados Unidos da América}
  \end{Phonetics}
\end{Entry}

\begin{Entry}{美国人}{9,8,2}{⽺、⼞、⼈}
  \begin{Phonetics}{美国人}{mei3guo2ren2}
    \definition{s.}{americano | pessoa ou povo dos Estados Unidos da América}
  \end{Phonetics}
\end{Entry}

\begin{Entry}{美学}{9,8}{⽺、⼦}
  \begin{Phonetics}{美学}{mei3xue2}
    \definition{s.}{estética; a ciência que estuda as leis e os princípios gerais da beleza na natureza, na sociedade e na arte explora principalmente a natureza da beleza, a relação entre arte e realidade e as leis gerais da criação artística}
  \end{Phonetics}
\end{Entry}

\begin{Entry}{美金}{9,8}{⽺、⾦}
  \begin{Phonetics}{美金}{mei3 jin1}[][HSK 4]
    \definition{s.}{USD; dólar americano: a moeda local dos Estados Unidos}
  \end{Phonetics}
\end{Entry}

\begin{Entry}{美洲}{9,9}{⽺、⽔}
  \begin{Phonetics}{美洲}{mei3zhou1}
    \definition*{s.}{América (incluindo Norte, Central e Sul)}
  \end{Phonetics}
\end{Entry}

\begin{Entry}{美洲人}{9,9,2}{⽺、⽔、⼈}
  \begin{Phonetics}{美洲人}{mei3zhou1ren2}
    \definition{s.}{americano | pessoa ou povo do continente Americano}
  \end{Phonetics}
\end{Entry}

\begin{Entry}{美食}{9,9}{⽺、⾷}
  \begin{Phonetics}{美食}{mei3 shi2}[][HSK 3]
    \definition[种,道,桌]{s.}{iguaria; (gastronomia) comida saborosa}
  \end{Phonetics}
\end{Entry}

\begin{Entry}{美容}{9,10}{⽺、⼧}
  \begin{Phonetics}{美容}{mei3 rong2}[][HSK 6]
    \definition{v.}{embelezar; melhorar a aparência de alguém; deixar seu rosto bonito retocando, cuidando, etc.}
  \end{Phonetics}
\end{Entry}

\begin{Entry}{耍}{9}{⽽}
  \begin{Phonetics}{耍}{shua3}
    \definition{v.}{brincar com | empunhar | agir (legal, calmo, tranquilo, descolado, etc.) | exibir (uma habilidade, o temperamento de alguém, etc.)}
  \end{Phonetics}
\end{Entry}

\begin{Entry}{耍赖}{9,13}{⽽、⾙}
  \begin{Phonetics}{耍赖}{shua3lai4}
    \definition{v.}{agir descaradamente | recusar -se a reconhecer que alguém perdeu o jogo ou fez uma promessa, etc. | agir como um idiota | agir como se algo nunca tivesse acontecido}
  \end{Phonetics}
\end{Entry}

\begin{Entry}{耐}{9}{⽽}
  \begin{Phonetics}{耐}{nai4}
    \definition{v.}{ser capaz de suportar; aguentar}
  \end{Phonetics}
\end{Entry}

\begin{Entry}{耐心}{9,4}{⽽、⼼}
  \begin{Phonetics}{耐心}{nai4xin1}[][HSK 5]
    \definition{adj.}{paciente}
    \definition[些]{s.}{paciência; uma pessoa que não se importa com problemas e é paciente}
  \end{Phonetics}
\end{Entry}

\begin{Entry}{胃}{9}{⾁}
  \begin{Phonetics}{胃}{wei4}[][HSK 5]
    \definition*{s.}{Wei, uma das mansões lunares | Wei, uma das vinte e oito constelações}
    \definition{s.}{estômago; parte do aparelho digestivo}
  \end{Phonetics}
\end{Entry}

\begin{Entry}{胃口}{9,3}{⾁、⼝}
  \begin{Phonetics}{胃口}{wei4kou3}
    \definition{s.}{apetite}
  \end{Phonetics}
\end{Entry}

\begin{Entry}{胆}{9}{⾁}
  \begin{Phonetics}{胆}{dan3}[][HSK 5]
    \definition[个,颗]{s.}{vesícula biliar | coragem; bravura | um recipiente interno semelhante a uma bexiga; algo que se encaixa dentro de um objeto e pode conter água, ar, etc.}
  \end{Phonetics}
\end{Entry}

\begin{Entry}{胆小}{9,3}{⾁、⼩}
  \begin{Phonetics}{胆小}{dan3 xiao3}[][HSK 5]
    \definition{adj.}{tímido; covarde}
  \end{Phonetics}
\end{Entry}

\begin{Entry}{胆小鬼}{9,3,9}{⾁、⼩、⿁}
  \begin{Phonetics}{胆小鬼}{dan3xiao3gui3}
    \definition{adj.}{covarde | medroso}
  \end{Phonetics}
\end{Entry}

\begin{Entry}{背}{9}{⾁}
  \begin{Phonetics}{背}{bei1}[][HSK 2]
    \definition{clas.}{carga; pacote; para transportar coisas nas costas}
    \definition{v.}{carregar nas costas | suportar; carregar}
  \end{Phonetics}
  \begin{Phonetics}{背}{bei4}[][HSK 3]
    \definition{adj.}{azarado | fora do caminho; um lugar muito distante do centro movimentado, onde poucas pessoas aparecem | deficiente auditivo}
    \definition{s.}{parte posterior do corpo; costas; coluna vertebral; parte do tronco entre os ombros e a região lombar | parte de trás de um objeto}
    \definition{v.}{afastar-se; virar as costas | decorar; memorizar; recitar de memória | esconder algo de; fazer algo em segredo | sair, ir embora; partir; abandonar | quebrar; violar; agir de forma contrária a}
  \end{Phonetics}
\end{Entry}

\begin{Entry}{背心}{9,4}{⾁、⼼}
  \begin{Phonetics}{背心}{bei4 xin1}[][HSK 6]
    \definition[件]{s.}{colete; vestimenta sem mangas; \emph{tops} sem gola e sem mangas}
  \end{Phonetics}
\end{Entry}

\begin{Entry}{背包}{9,5}{⾁、⼓}
  \begin{Phonetics}{背包}{bei1 bao1}[][HSK 5]
    \definition[个,只,款]{s.}{mochila; mochila de ataque; mochila de infantaria; pacotes de roupas carregados nas costas quando marcham}
  \end{Phonetics}
\end{Entry}

\begin{Entry}{背后}{9,6}{⾁、⼝}
  \begin{Phonetics}{背后}{bei4 hou4}[][HSK 3]
    \definition{s.}{parte posterior; parte de trás; traseira | pelas costas de alguém}
  \end{Phonetics}
\end{Entry}

\begin{Entry}{背着}{9,11}{⾁、⽬}
  \begin{Phonetics}{背着}{bei4 zhe5}[][HSK 6]
    \definition{adv.}{pelas costas; atrás de alguém}
    \definition{v.}{carregar nas costas}
  \end{Phonetics}
\end{Entry}

\begin{Entry}{背景}{9,12}{⾁、⽇}
  \begin{Phonetics}{背景}{bei4jing3}[][HSK 4]
    \definition[种]{s.}{pano de fundo; fundo; cenário de teatro, filme ou drama de TV | fundo; cenário que permeia a imagem principal na tela | condições sociais; ambientes históricos (significativamente influentes para algo ou alguém) | poder que dá suporte a alguém}
  \end{Phonetics}
\end{Entry}

\begin{Entry}{胖}{9}{⾁}
  \begin{Phonetics}{胖}{pan2}
    \definition{adj.}{saudável}
  \end{Phonetics}
  \begin{Phonetics}{胖}{pang4}[][HSK 3]
    \definition{adj.}{gordo; robusto; rechonchudo; (corpo humano) com muita gordura ou carne (em oposição a 瘦)}
  \seealsoref{瘦}{shou4}
  \end{Phonetics}
\end{Entry}

\begin{Entry}{胖子}{9,3}{⾁、⼦}
  \begin{Phonetics}{胖子}{pang4 zi5}[][HSK 4]
    \definition[个]{s.}{obeso; gordo; pessoa gorda}
  \end{Phonetics}
\end{Entry}

\begin{Entry}{胚}{9}{⾁}
  \begin{Phonetics}{胚}{pei1}
    \definition{s.}{embrião}
  \end{Phonetics}
\end{Entry}

\begin{Entry}{胜}{9}{⾁}
  \begin{Phonetics}{胜}{sheng4}[][HSK 3]
    \definition{adj.}{soberbo; maravilhoso; adorável}
    \definition[场]{s.}{vitória; sucesso | penteado de mulher; joias usadas pelas mulheres na antiguidade}
    \definition{v.}{vencer (oposto de 负, 败) | derrotar | (frequentemente seguido por 于, etc.) superar; ser superior a; levar a melhor sobre | vencer; ter sucesso; derrotar o adversário | ultrapassar; ser superior ao outro | suportar; ser capaz de suportar ou aguentar}
  \seealsoref{败}{bai4}
  \seealsoref{负}{fu4}
  \seealsoref{于}{yu2}
  \end{Phonetics}
\end{Entry}

\begin{Entry}{胜负}{9,6}{⾁、⾙}
  \begin{Phonetics}{胜负}{sheng4fu4}[][HSK 5]
    \definition{s.}{vitória ou derrota; sucesso ou fracasso}
  \end{Phonetics}
\end{Entry}

\begin{Entry}{胜利}{9,7}{⾁、⼑}
  \begin{Phonetics}{胜利}{sheng4li4}[][HSK 3]
    \definition{adv.}{com sucesso; triunfantemente; atingir o objetivo previsto}
    \definition{v.}{ganhar; vencer; triunfar; ter sucesso}
  \end{Phonetics}
\end{Entry}

\begin{Entry}{胜算}{9,14}{⾁、⽵}
  \begin{Phonetics}{胜算}{sheng4suan4}
    \definition{s.}{probabilidade de sucesso | estratégia que garante o sucesso}
    \definition{v.}{ter certeza do sucesso}
  \end{Phonetics}
\end{Entry}

\begin{Entry}{胡}{9}{⾁}
  \begin{Phonetics}{胡}{hu2}
    \definition*{s.}{Sobrenome Hu}
    \definition{adj.}{introduzidos de nacionalidades do norte e do oeste ou do exterior | nos tempos antigos, o termo "Oriente e Ocidente" se referia às minorias étnicas do norte e do oeste, e também, de modo geral, às pessoas do exterior}
    \definition{adv.}{imprudentemente; desenfreadamente; escandalosamente; sem lei, ordem ou razão}
    \definition{pron.}{Por que?; palavras interrogativas: 为什么, 何故}
    \definition{s.}{nos tempos antigos, geralmente se referia às minorias étnicas do norte e do oeste | violino chinês | barba; bigode}
  \seealsoref{何故}{he2gu4}
  \seealsoref{为什么}{wei4shen2me5}
  \end{Phonetics}
\end{Entry}

\begin{Entry}{胡子}{9,3}{⾁、⼦}
  \begin{Phonetics}{胡子}{hu2 zi5}[][HSK 5]
    \definition[团,根,个,撮]{s.}{barba; bigode | bandido; salteador}
  \end{Phonetics}
\end{Entry}

\begin{Entry}{胡同}{9,6}{⾁、⼝}
  \begin{Phonetics}{胡同}{hu2tong5}
    \definition[条,个]{s.}{beco; rua pequena}
  \end{Phonetics}
\end{Entry}

\begin{Entry}{胡同儿}{9,6,2}{⾁、⼝、⼉}
  \begin{Phonetics}{胡同儿}{hu2 tong4r5}[][HSK 5]
    \definition{s.}{beco}
  \end{Phonetics}
\end{Entry}

\begin{Entry}{胡萝卜}{9,11,2}{⾁、⾋、⼘}
  \begin{Phonetics}{胡萝卜}{hu2luo2bo5}
    \definition{s.}{cenoura}
  \end{Phonetics}
\end{Entry}

\begin{Entry}{胡琴}{9,12}{⾁、⽟}
  \begin{Phonetics}{胡琴}{hu2qin2}
    \definition{s.}{huqin, um termo geral para certos instrumentos de arco de duas cordas, como 二胡, 京胡, etc. | família de violinos chineses de duas cordas, com caixa de ressonância de madeira revestida de pele de cobra e arco de bambu com corda de crina de cavalo}
  \seealsoref{二胡}{er4hu2}
  \seealsoref{京胡}{jing1hu2}
  \end{Phonetics}
\end{Entry}

\begin{Entry}{舁}{9}{⾅}
  \begin{Phonetics}{舁}{yu2}
    \definition{v.}{levantar; elevar | aumentar}
  \end{Phonetics}
\end{Entry}

\begin{Entry}{范}{9}{⾋}
  \begin{Phonetics}{范}{fan4}
    \definition*{s.}{Sobrenome Fan}
    \definition{s.}{padrão; molde; matriz | modelo; exemplo; modelo a seguir | limites; escopo | restrição; limite}
  \end{Phonetics}
\end{Entry}

\begin{Entry}{范围}{9,7}{⾋、⼞}
  \begin{Phonetics}{范围}{fan4wei2}[][HSK 3]
    \definition[个]{s.}{escopo; limite; alcance}
    \definition{v.}{estabelecer limites para; limitar o escopo de}
  \end{Phonetics}
\end{Entry}

\begin{Entry}{茶}{9}{⾋}
  \begin{Phonetics}{茶}{cha2}[][HSK 1]
    \definition{adj.}{moreno; fulvo; amarelo-acastanhado}
    \definition[杯,壶]{s.}{chá (a bebida); bebida feita com folhas de chá | chá (a planta) | certos tipos de bebidas ou alimentos líquidos | árvore de chá-de-óleo | camélia}
  \end{Phonetics}
\end{Entry}

\begin{Entry}{茶叶}{9,5}{⾋、⼝}
  \begin{Phonetics}{茶叶}{cha2 ye4}[][HSK 4]
    \definition[包,袋,盒,斤,把,种]{s.}{chá; folhas de chá; as folhas jovens da planta do chá que são processadas para produzir bebidas}
  \end{Phonetics}
\end{Entry}

\begin{Entry}{草}{9}{⾋}
  \begin{Phonetics}{草}{cao3}[][HSK 2]
    \definition*{s.}{Sobrenome Cao}
    \definition{adj.}{descuidado; rude | rascunho; inicial | femea; na linguagem coloquial, refere-se a animais domésticos e aves fêmeas | precipitado; pouco cuidadoso | rascunho; não definitivo; preliminar; informal}
    \definition[种,棵,撮,株,根]{s.}{grama; gramado | palha | campo; zona rural; área selvagem | letra cursiva | letra cursiva (ou caligráfica) de um alfabeto fonético | rascunho | caligrafia cursiva; um tipo de escrita chinesa}
    \definition{v.}{esboçar; redigir}
  \end{Phonetics}
\end{Entry}

\begin{Entry}{草地}{9,6}{⾋、⼟}
  \begin{Phonetics}{草地}{cao3 di4}[][HSK 2]
    \definition[片,块]{s.}{prado; gramado; campo; pastagem ou grande área de terra plantada com pastagem | gramado; relvado; local com grama alta ou gramado}
  \end{Phonetics}
\end{Entry}

\begin{Entry}{草纸}{9,7}{⾋、⽷}
  \begin{Phonetics}{草纸}{cao3zhi3}
    \definition{s.}{papel pardo | pergaminho | papel de palha áspero | papel higiênico}
  \end{Phonetics}
\end{Entry}

\begin{Entry}{草原}{9,10}{⾋、⼚}
  \begin{Phonetics}{草原}{cao3 yuan2}[][HSK 5]
    \definition[片,个]{s.}{estepe; pradaria; grandes áreas de terra coberta de vegetação em áreas semiáridas, intercaladas com árvores tolerantes à seca}
  \end{Phonetics}
\end{Entry}

\begin{Entry}{草莓}{9,10}{⾋、⾋}
  \begin{Phonetics}{草莓}{cao3mei2}
    \definition[颗]{s.}{morango}
  \end{Phonetics}
\end{Entry}

\begin{Entry}{荒}{9}{⾋}
  \begin{Phonetics}{荒}{huang1}
    \definition*{s.}{Sobrenome Huang}
    \definition{adj.}{(terra) não utilizada; não cultivada | desolado; estéril | irracional; delirante; fantástico; absurdo | incerto; duvidoso | dissoluto; autoindulgente | grosseiramente processado; bruto}
    \definition[片,块]{s.}{terra devastada; terra inculta; deserto | fome; quebra de safra | escassez | lixo; restos | terra selvagem (floresta)}
    \definition{v.}{(coloquial) negligenciar; estar fora de prática}
  \end{Phonetics}
\end{Entry}

\begin{Entry}{荒芜}{9,7}{⾋、⾋}
  \begin{Phonetics}{荒芜}{huang1wu2}
    \definition{adj.}{estéril}
  \end{Phonetics}
\end{Entry}

\begin{Entry}{荔}{9}{⾋}
  \begin{Phonetics}{荔}{li4}
    \definition[颗]{s.}{lichia | (arcaico) uma espécie de grama semelhante à taboa}
  \end{Phonetics}
\end{Entry}

\begin{Entry}{荔枝}{9,8}{⾋、⽊}
  \begin{Phonetics}{荔枝}{li4zhi1}
    \definition{s.}{lichia}
  \end{Phonetics}
\end{Entry}

\begin{Entry}{荣}{9}{⾋}
  \begin{Phonetics}{荣}{rong2}
    \definition*{s.}{Sobrenome Rong}
    \definition{adj.}{próspero; florescente | exuberante | glorioso}
    \definition{s.}{honra; glória (oposto a 辱) | guarda-sol chinês | flor; flor de planta herbácea | beirais virados para cima}
    \definition{v.}{glorificar; luxuriar; crescer abundantemente; florescer | florescer | lançar}
  \seealsoref{辱}{ru3}
  \end{Phonetics}
\end{Entry}

\begin{Entry}{荤}{9}{⾋}[Kangxi 9]
  \begin{Phonetics}{荤}{hun1}
    \definition{adj.}{obsceno; lascivo; vulgar}
    \definition{s.}{carne ou peixe (oposto a 素) | Budismo: vegetais picantes proibidos aos vegetarianos budistas, como cebola, alho-poró, alho, etc. | alimentos não vegetarianos (carne, peixe etc.) | vegetais com cheiro forte (alho etc.)}
  \seealsoref{素}{su4}
  \end{Phonetics}
\end{Entry}

\begin{Entry}{药}{9}{⾋}
  \begin{Phonetics}{药}{yao4}[][HSK 2]
    \definition*{s.}{Sobrenome Yao}
    \definition[片,粒,颗,瓶,服]{s.}{droga; loção; remédio; medicamento; substâncias que podem prevenir e tratar doenças, pragas ou melhorar funções corporais | certos produtos químicos com efeitos específicos}
    \definition{v.}{curar com remédios; tomar remédios para tratar doenças | matar com veneno; envenenar}
  \end{Phonetics}
\end{Entry}

\begin{Entry}{药丸}{9,3}{⾋、⼂}
  \begin{Phonetics}{药丸}{yao4wan2}
    \definition[粒]{s.}{pílula}
  \end{Phonetics}
\end{Entry}

\begin{Entry}{药水}{9,4}{⾋、⽔}
  \begin{Phonetics}{药水}{yao4 shui3}[][HSK 2]
    \definition*{s.}{Yaksu na Coreia do Norte, perto da fronteira com Liaoning e a província de Jilin}
    \definition{s.}{medicamento líquido; líquido medicinal | loção | remédio engarrafado | medicamento em forma líquida}
  \end{Phonetics}
\end{Entry}

\begin{Entry}{药片}{9,4}{⾋、⽚}
  \begin{Phonetics}{药片}{yao4 pian4}[][HSK 2]
    \definition[颗,片]{s.}{pílula; comprimido; preparações em comprimidos}
  \end{Phonetics}
\end{Entry}

\begin{Entry}{药补}{9,7}{⾋、⾐}
  \begin{Phonetics}{药补}{yao4bu3}
    \definition{s.}{suplemento dietético medicinal que ajuda a melhorar a saúde}
  \end{Phonetics}
\end{Entry}

\begin{Entry}{药典}{9,8}{⾋、⼋}
  \begin{Phonetics}{药典}{yao4dian3}
    \definition{s.}{farmacopéia}
  \end{Phonetics}
\end{Entry}

\begin{Entry}{药店}{9,8}{⾋、⼴}
  \begin{Phonetics}{药店}{yao4 dian4}[][HSK 2]
    \definition[家]{s.}{farmácia; drogaria; lojas que vendem medicamentos}
  \end{Phonetics}
\end{Entry}

\begin{Entry}{药房}{9,8}{⾋、⼾}
  \begin{Phonetics}{药房}{yao4fang2}
    \definition{s.}{farmácia | drogaria}
  \end{Phonetics}
\end{Entry}

\begin{Entry}{药物}{9,8}{⾋、⽜}
  \begin{Phonetics}{药物}{yao4 wu4}[][HSK 4]
    \definition[种]{s.}{droga; medicamento; remédio; substâncias que controlam doenças, pragas, etc.}
  \end{Phonetics}
\end{Entry}

\begin{Entry}{药品}{9,9}{⾋、⼝}
  \begin{Phonetics}{药品}{yao4pin3}[][HSK 6]
    \definition[个,些,种,类,批]{s.}{medicamentos e reagentes químicos; um termo geral para vários medicamentos e reagentes químicos}
  \end{Phonetics}
\end{Entry}

\begin{Entry}{药签}{9,13}{⾋、⽵}
  \begin{Phonetics}{药签}{yao4qian1}
    \definition{s.}{cotonete médico}
  \end{Phonetics}
\end{Entry}

\begin{Entry}{药膳}{9,16}{⾋、⾁}
  \begin{Phonetics}{药膳}{yao4shan4}
    \definition{s.}{alimentos medicamentosos; alimentos cozidos com ervas medicinais | cozinha medicinal}
  \end{Phonetics}
\end{Entry}

\begin{Entry}{药罐}{9,23}{⾋、⽸}
  \begin{Phonetics}{药罐}{yao4guan4}
    \definition{s.}{frasco de remédio; pote de remédio}
  \end{Phonetics}
\end{Entry}

\begin{Entry}{虽}{9}{⾍}
  \begin{Phonetics}{虽}{sui1}[][HSK 6]
    \definition{conj.}{no entanto; embora | mesmo se}
  \end{Phonetics}
\end{Entry}

\begin{Entry}{虽然}{9,12}{⾍、⽕}
  \begin{Phonetics}{虽然}{sui1 ran2}[][HSK 2]
    \definition{conj.}{apesar de; embora (frequentemente usado correlativamente com 可是, 但是, etc); geralmente é usado no início de uma frase para indicar que o fato anterior foi reconhecido, mas não mudará o que acontecerá em seguida}
  \seealsoref{但是}{dan4 shi4}
  \seealsoref{可是}{ke3shi4}
  \end{Phonetics}
\end{Entry}

\begin{Entry}{虾}{9}{⾍}
  \begin{Phonetics}{虾}{xia1}
    \definition{s.}{camarão}
  \end{Phonetics}
\end{Entry}

\begin{Entry}{蚂}{9}{⾍}
  \begin{Phonetics}{蚂}{ma1}
    \definition{part.}{caracter formador de palavras}
    \definition[只]{s.}{libélula}
  \end{Phonetics}
  \begin{Phonetics}{蚂}{ma3}
    \definition{part.}{caracter formador de palavras}
  \end{Phonetics}
  \begin{Phonetics}{蚂}{ma4}
    \definition{part.}{caracter formador de palavras}
  \end{Phonetics}
\end{Entry}

\begin{Entry}{蚂蚁}{9,9}{⾍、⾍}
  \begin{Phonetics}{蚂蚁}{ma3yi3}
    \definition{s.}{formiga}
  \end{Phonetics}
\end{Entry}

\begin{Entry}{要}{9}{⾑}
  \begin{Phonetics}{要}{yao1}[][HSK 1]
    \definition*{s.}{Sobrenome Yao}
    \definition{v.}{exigir; pedir; requerer; solicitar; buscar; insistir com base em algo em que se apoia | forçar; coagir; ameaçar}
  \end{Phonetics}
  \begin{Phonetics}{要}{yao4}[][HSK 4]
    \definition{adj.}{importante; essencial}
    \definition{conj.}{suponha; no caso; se, indicando um relacionamento hipotético | ou; ou\dots ou\dots}
    \definition{s.}{ponto principal; manchete; conteúdo importante}
    \definition{v.}{querer; desejar; pensar | querer; pedir; deseja; querer obter; querer manter | recuperar algo; dizer a alguém para guardar algo para você ou devolver | pedir (ou querer) que alguém faça algo; pedir a alguém para fazer algo, quando usado para conseguir que alguém faça algo, tem um tom de comando e pode ser indelicado | precisar; tomar; pegar | deve; deveria; é necessário (imperativo, essencial) que\dots | estar indo para | querer; ter um desejo por; expressar determinação ou desejo de fazer algo | poder; dever;  indica uma estimativa, usada para comparação}
  \seealsoref{要是}{yao4shi5}
  \end{Phonetics}
\end{Entry}

\begin{Entry}{要么}{9,3}{⾑、⼃}
  \begin{Phonetics}{要么}{yao4 me5}[][HSK 6]
    \definition{conj.}{ou; ou\dots ou\dots; indica uma escolha entre duas situações ou dois desejos}
  \seealsoref{要么……要么……}{yao4 me5 yao4 me5}
  \end{Phonetics}
\end{Entry}

\begin{Entry}{要么……要么……}{9,3,9,3}{⾑、⼃、⾑、⼃}
  \begin{Phonetics}{要么……要么……}{yao4 me5 yao4 me5}[][HSK 6]
    \definition{conj.}{ou\dots ou\dots}
  \seealsoref{要么}{yao4 me5}
  \end{Phonetics}
\end{Entry}

\begin{Entry}{要义}{9,3}{⾑、⼂}
  \begin{Phonetics}{要义}{yao4yi4}
    \definition{s.}{resumo | o essencial}
  \end{Phonetics}
\end{Entry}

\begin{Entry}{要不}{9,4}{⾑、⼀}
  \begin{Phonetics}{要不}{yao4 bu4}
    \definition{conj.}{ou então; caso contrário; se você não fizer isso (haverá um resultado ruim) | usado para propor educadamente; usado para fazer uma sugestão educadamente | ou; se você não fizer isso, faça aquilo}
  \end{Phonetics}
\end{Entry}

\begin{Entry}{要不然}{9,4,12}{⾑、⼀、⽕}
  \begin{Phonetics}{要不然}{yao4 bu4 ran2}[][HSK 6]
    \definition{conj.}{caso contrário; ou então; se você não fizer isso (haverá um resultado ruim) | ou então; usado entre duas frases em um relacionamento de escolha; significa escolher uma entre as duas; equivalente a 要不}
  \seealsoref{要不}{yao4 bu4}
  \end{Phonetics}
\end{Entry}

\begin{Entry}{要好}{9,6}{⾑、⼥}
  \begin{Phonetics}{要好}{yao4 hao3}[][HSK 6]
    \definition{adj.}{estar em bons termos; ser amigos próximos; relacionamento harmonioso | estar ansioso para melhorar a si mesmo; esforçar-se para progredir | ansioso para melhorar a si mesmo; esforçar-se para progredir}
  \end{Phonetics}
\end{Entry}

\begin{Entry}{要死}{9,6}{⾑、⽍}
  \begin{Phonetics}{要死}{yao4si3}
    \definition{adv.}{extremamente | muito}
  \end{Phonetics}
\end{Entry}

\begin{Entry}{要求}{9,7}{⾑、⽔}
  \begin{Phonetics}{要求}{yao1qiu2}[][HSK 2]
    \definition[个,点]{s.}{exigência; demanda; reivindicação; desejos ou condições específicas propostas}
    \definition{v.}{pedir; exigir; exigir; reivindicar; apresentar desejos ou condições específicas, esperando que sejam satisfeitos ou realizados}
  \end{Phonetics}
\end{Entry}

\begin{Entry}{要挟}{9,9}{⾑、⼿}
  \begin{Phonetics}{要挟}{yao1xie2}
    \definition{v.}{chantagear | ameaçar}
  \end{Phonetics}
\end{Entry}

\begin{Entry}{要是}{9,9}{⾑、⽇}
  \begin{Phonetics}{要是}{yao4shi5}[][HSK 3]
    \definition{conj.}{se; suponha; no caso de; conecta frases, expressa uma relação hipotética, equivalente a 如果, e pode ser usado em conjunto com 的话}
  \seealsoref{的话}{de5 hua4}
  \seealsoref{如果}{ru2guo3}
  \end{Phonetics}
\end{Entry}

\begin{Entry}{要是……的话}{9,9,8,8}{⾑、⽇、⽩、⾔}
  \begin{Phonetics}{要是……的话}{yao4shi5 de5hua4}
    \definition{conj.}{se for assim\dots}
  \end{Phonetics}
\end{Entry}

\begin{Entry}{要点}{9,9}{⾑、⽕}
  \begin{Phonetics}{要点}{yao4dian3}
    \definition{s.}{pontos principais | essencial}
  \end{Phonetics}
\end{Entry}

\begin{Entry}{要素}{9,10}{⾑、⽷}
  \begin{Phonetics}{要素}{yao4su4}[][HSK 6]
    \definition[个]{s.}{fator essencial; elemento-chave; os elementos essenciais que compõem as coisas}
  \end{Phonetics}
\end{Entry}

\begin{Entry}{要谎}{9,11}{⾑、⾔}
  \begin{Phonetics}{要谎}{yao4huang3}
    \definition{v.}{pedir um preço enorme (como primeiro passo de negociação)}
  \end{Phonetics}
\end{Entry}

\begin{Entry}{要强}{9,12}{⾑、⼸}
  \begin{Phonetics}{要强}{yao4qiang2}
    \definition{adj.}{ansioso para se destacar | ansioso para progredir na vida | obstinado}
  \end{Phonetics}
\end{Entry}

\begin{Entry}{觉}{9}{⾒}
  \begin{Phonetics}{觉}{jiao4}[][HSK 6]
    \definition[个]{s.}{sono; o processo desde adormecer até acordar}
  \end{Phonetics}
  \begin{Phonetics}{觉}{jue2}
    \definition{s.}{sentimento; senso; percepção e discriminação de estímulos externos}
    \definition{v.}{sentir; perceber | acordar | tornar-se consciente; tornar-se desperto; despertar; entender}
  \end{Phonetics}
\end{Entry}

\begin{Entry}{觉悟}{9,10}{⾒、⼼}
  \begin{Phonetics}{觉悟}{jue2wu4}[][HSK 6]
    \definition{s.}{consciência; percepção; compreensão; nível de consciência}
    \definition{v.}{vir a compreender; tornar-se consciente de; tornar-se politicamente desperto; despertar}
  \end{Phonetics}
\end{Entry}

\begin{Entry}{觉得}{9,11}{⾒、⼻}
  \begin{Phonetics}{觉得}{jue2de5}[][HSK 1]
    \definition{v.}{sentir; estar ciente; pressentir; causar uma sensação | pensar; sentir; encontrar; considerar (tom menos assertivo)}
  \end{Phonetics}
\end{Entry}

\begin{Entry}{语}{9}{⾔}
  \begin{Phonetics}{语}{yu3}
    \definition{s.}{língua; linguagem | dito; provérbio; refere-se especialmente a coloquialismos, provérbios, expressões idiomáticas ou palavras de livros antigos | sinal; meio não linguístico de comunicar ideias ; ações ou sinais que substituem palavras para expressar significado | palavras; expressão; refere-se a uma palavra, frase ou sentença}
    \definition{v.}{dizer; falar | (pássaros, insetos, etc.) gorjear; pipilar}
  \end{Phonetics}
  \begin{Phonetics}{语}{yu4}
    \definition{v.}{contar; informar}
  \end{Phonetics}
\end{Entry}

\begin{Entry}{语气}{9,4}{⾔、⽓}
  \begin{Phonetics}{语气}{yu3qi4}
    \definition[个]{s.}{maneira de falar | tom}
  \end{Phonetics}
\end{Entry}

\begin{Entry}{语言}{9,7}{⾔、⾔}
  \begin{Phonetics}{语言}{yu3yan2}[][HSK 2]
    \definition[种,门]{s.}{linguagem; é uma ferramenta exclusiva dos humanos para expressar ideias e comunicar pensamentos; é um fenômeno social especial e consiste em um sistema específico de pronúncia, vocabulário e gramática | linguagem falada}
  \end{Phonetics}
\end{Entry}

\begin{Entry}{语言实验室}{9,7,8,10,9}{⾔、⾔、⼧、⾺、⼧}
  \begin{Phonetics}{语言实验室}{yu3yan2shi2yan4shi4}
    \definition{s.}{laboratório de línguas}
  \end{Phonetics}
\end{Entry}

\begin{Entry}{语法}{9,8}{⾔、⽔}
  \begin{Phonetics}{语法}{yu3fa3}[][HSK 4]
    \definition[个]{s.}{gramática; maneira como o idioma é estruturado, incluindo a formação e as variações de palavras, a organização de frases e sentenças | estudo da gramática; estudo das regras de estrutura linguística}
  \end{Phonetics}
\end{Entry}

\begin{Entry}{语法术语}{9,8,5,9}{⾔、⽔、⽊、⾔}
  \begin{Phonetics}{语法术语}{yu3fa3 shu4yu3}
    \definition{s.}{termo gramatical}
  \end{Phonetics}
\end{Entry}

\begin{Entry}{语音}{9,9}{⾔、⾳}
  \begin{Phonetics}{语音}{yu3 yin1}[][HSK 4]
    \definition{s.}{voz; pronúncia; sons da fala; som de alguém falando | pronúncia; som do idioma}
  \end{Phonetics}
\end{Entry}

\begin{Entry}{语调}{9,10}{⾔、⾔}
  \begin{Phonetics}{语调}{yu3diao4}
    \definition[个]{s.}{entonação}
  \end{Phonetics}
\end{Entry}

\begin{Entry}{误}{9}{⾔}
  \begin{Phonetics}{误}{wu4}[][HSK 6]
    \definition{adj.}{errado; falso; impreciso | acidental}
    \definition{adv.}{por engano; por acidente; não intencional}
    \definition{s.}{engano; erro}
    \definition{v.}{perder | dificultar; impedir; prejudicar | confundir; entender mal; cometer um erro | causar desvantagem a. causar dano}
  \end{Phonetics}
\end{Entry}

\begin{Entry}{误会}{9,6}{⾔、⼈}
  \begin{Phonetics}{误会}{wu4hui4}
    \definition[场]{s.}{mal-entendido; desentendimentos ou conflitos decorrentes de mal-entendidos}
    \definition{v.}{entender mal; entender errado; interpretar mal; não entender; não entender corretamente o significado}
  \end{Phonetics}
\end{Entry}

\begin{Entry}{误点}{9,9}{⾔、⽕}
  \begin{Phonetics}{误点}{wu4dian3}
    \definition{v.+compl.}{atrasar | chegar tarde}
  \end{Phonetics}
\end{Entry}

\begin{Entry}{误解}{9,13}{⾔、⾓}
  \begin{Phonetics}{误解}{wu4jie3}[][HSK 5]
    \definition[个,种]{s.}{equívoco; mal-entendido; desentendimento}
    \definition{v.}{interpretar mal; interpretar erroneamente; não compreender corretamente}
  \end{Phonetics}
\end{Entry}

\begin{Entry}{诱}{9}{⾔}
  \begin{Phonetics}{诱}{you4}
    \definition{v.}{guiar; liderar; dirigir | atrair; seduzir; aliciar | induzir; causar; resultar de; levar a}
  \end{Phonetics}
\end{Entry}

\begin{Entry}{诱人}{9,2}{⾔、⼈}
  \begin{Phonetics}{诱人}{you4ren2}
    \definition{adj.}{atraente | cativante}
  \end{Phonetics}
\end{Entry}

\begin{Entry}{说}{9}{⾔}
  \begin{Phonetics}{说}{shui4}
    \definition{v.}{persuadir}
  \end{Phonetics}
  \begin{Phonetics}{说}{shuo1}[][HSK 1]
    \definition{s.}{uma teoria (normalmente o último caractere, como em 日心说, teoria heliocêntrica); ensinamentos; doutrina}
    \definition{v.}{falar; conversar; dizer | explicar | repreender | atuar como casamenteiro | referir-se a; indicar | criticar; aconselhar | fazer uma combinação; conciliar; mediar | discutir; falar sobre; conversar sobre | uma forma de expressão linguística da arte cênica}
  \seealsoref{日心说}{ri4 xin1 shuo1}
  \end{Phonetics}
\end{Entry}

\begin{Entry}{说不定}{9,4,8}{⾔、⼀、⼧}
  \begin{Phonetics}{说不定}{shuo1bu5ding4}[][HSK 4]
    \definition{adv.}{talvez; indica uma estimativa, possivelmente, provavelmente}
    \definition{v.}{não ter certeza; não estar certo; ser impreciso}
  \end{Phonetics}
\end{Entry}

\begin{Entry}{说好}{9,6}{⾔、⼥}
  \begin{Phonetics}{说好}{shuo1hao3}
    \definition{v.}{chegar a um acordo | concluir negociações}
  \end{Phonetics}
\end{Entry}

\begin{Entry}{说完}{9,7}{⾔、⼧}
  \begin{Phonetics}{说完}{shuo1-wan2}
    \definition{expr.}{acabar/terminar palavras}
  \end{Phonetics}
\end{Entry}

\begin{Entry}{说实话}{9,8,8}{⾔、⼧、⾔}
  \begin{Phonetics}{说实话}{shuo1 shi2 hua4}[][HSK 6]
    \definition{v.}{falar a verdade; dizer a verdade sobre (os próprios erros ou crimes)}
  \end{Phonetics}
\end{Entry}

\begin{Entry}{说明}{9,8}{⾔、⽇}
  \begin{Phonetics}{说明}{shuo1ming2}[][HSK 2]
    \definition[本,个]{s.}{legenda; instrução; explicação}
    \definition{v.}{mostrar; explicar; ilustrar | indicar; mostrar; provar; demonstrar; usar materiais confiáveis para demonstrar ou determinar a autenticidade de pessoas ou coisas}
  \end{Phonetics}
\end{Entry}

\begin{Entry}{说明书}{9,8,4}{⾔、⽇、⼄}
  \begin{Phonetics}{说明书}{shuo1 ming2 shu1}[][HSK 6]
    \definition[本]{s.}{manual; livro de instruções; descrições textuais da finalidade, especificações, desempenho e uso de itens, bem como enredos de peças e filmes, etc.}
  \end{Phonetics}
\end{Entry}

\begin{Entry}{说服}{9,8}{⾔、⽉}
  \begin{Phonetics}{说服}{shuo1fu2}[][HSK 4]
    \definition{v.}{persuadir; convencer; convencer a outra parte com palavras bem fundamentadas}
  \end{Phonetics}
\end{Entry}

\begin{Entry}{说法}{9,8}{⾔、⽔}
  \begin{Phonetics}{说法}{shuo1 fa3}[][HSK 5]
    \definition[种,个]{s.}{formulação; maneira de dizer uma coisa; formas de expressar opiniões | versão; argumento; declaração; opinião | explicação; acordo; palavras justas; razões ou fundamentos legítimos}
  \end{Phonetics}
\end{Entry}

\begin{Entry}{说话}{9,8}{⾔、⾔}
  \begin{Phonetics}{说话}{shuo1hua4}[][HSK 1]
    \definition{adv.}{imediatamente; em um minuto; refere-se ao tempo que leva para falar, indicando um período muito curto}
    \definition{v.}{falar; conversar; dizer; expressar o significado através da linguagem | conversar (conversa fiada); bater papo | fofocar; conversar; criticar; censurar}
  \end{Phonetics}
\end{Entry}

\begin{Entry}{说理}{9,11}{⾔、⽟}
  \begin{Phonetics}{说理}{shuo1li3}
    \definition{v.}{racionalizar | discutir logicamente}
  \end{Phonetics}
\end{Entry}

\begin{Entry}{说谎}{9,11}{⾔、⾔}
  \begin{Phonetics}{说谎}{shuo1huang3}
    \definition{v.+compl.}{mentir | contar uma mentira}
  \end{Phonetics}
\end{Entry}

\begin{Entry}{贱}{9}{⾙}
  \begin{Phonetics}{贱}{jian4}
    \definition*{s.}{Sobrenome Jian}
    \definition{adj.}{baixo preço; barato (oposto a 贵) | humilde (oposto a 贵) | baixo; básico; desprezível | humilde; baixa posição social}
    \definition{pron.}{meu (autodepreciativo)}
  \seealsoref{贵}{gui4}
  \end{Phonetics}
\end{Entry}

\begin{Entry}{贴}{9}{⾙}
  \begin{Phonetics}{贴}{tie1}[][HSK 4]
    \definition{adj.}{submisso; obediente | apropriado}
    \definition{clas.}{usado em gessos, emplastros}
    \definition{s.}{subsídio; subvenção}
    \definition{v.}{grudar; colar | aninhar-se a; aconchegar-se a; aconchegar-se em | subsidiar; ajudar financeiramente}
  \end{Phonetics}
\end{Entry}

\begin{Entry}{贵}{9}{⾙}
  \begin{Phonetics}{贵}{gui4}[][HSK 1]
    \definition*{s.}{Província de Guizhou, abreviação de 贵州 | Sobrenome Gui}
    \definition{adj.}{caro; dispendioso (oposto de 贱) | altamente valorizado; valioso | de alta patente; nobre (oposto de 贱) | caro; preço ou valor elevado (em oposição a 贱) | digno de ser valorizado ou apreciado | nobre; honrado; posição social elevada}
    \definition{pron.}{honrado: Seu}
  \seealsoref{贵州}{gui4zhou1}
  \seealsoref{贱}{jian4}
  \end{Phonetics}
\end{Entry}

\begin{Entry}{贵州}{9,6}{⾙、⼮}
  \begin{Phonetics}{贵州}{gui4zhou1}
    \definition*{s.}{Província de Guizhou}
  \end{Phonetics}
\end{Entry}

\begin{Entry}{贵姓}{9,8}{⾙、⼥}
  \begin{Phonetics}{贵姓}{gui4xing4}
    \definition{expr.}{qual seu sobrenome?}
  \end{Phonetics}
\end{Entry}

\begin{Entry}{贷}{9}{⾙}
  \begin{Phonetics}{贷}{dai4}
    \definition[笔]{s.}{empréstimo; valor do empréstimo}
    \definition{v.}{pedir dinheiro emprestado ou emprestar dinheiro | fugir da responsabilidade | perdoar}
  \end{Phonetics}
\end{Entry}

\begin{Entry}{贷款}{9,12}{⾙、⽋}
  \begin{Phonetics}{贷款}{dai4kuan3}[][HSK 5]
    \definition[个,笔]{s.}{empréstimo; crédito}
    \definition{v.}{fornecer um empréstimo; conceder um empréstimo; conceder crédito a; emprestar dinheiro para quem precisa}
  \end{Phonetics}
\end{Entry}

\begin{Entry}{贸}{9}{⾙}
  \begin{Phonetics}{贸}{mao4}
    \definition*{s.}{Sobrenome Mao}
    \definition{s.}{comércio; negociação}
  \end{Phonetics}
\end{Entry}

\begin{Entry}{贸易}{9,8}{⾙、⽇}
  \begin{Phonetics}{贸易}{mao4yi4}[][HSK 5]
    \definition[笔,宗,项,个]{s.}{comércio; troca; negócios; refere-se a atividades comerciais, como a troca de mercadorias}
    \definition{v.}{fazer uma transação comercial}
  \end{Phonetics}
\end{Entry}

\begin{Entry}{费}{9}{⾙}
  \begin{Phonetics}{费}{fei4}[][HSK 3]
    \definition*{s.}{Sobrenome Fei}
    \definition{s.}{taxa; despesa; encargo}
    \definition{v.}{custar; gastar; despender | ser desperdiçador; consumir em excesso; gastar algo muito rapidamente; consumo excessivo (oposto a 省)}
  \seealsoref{省}{sheng3}
  \end{Phonetics}
\end{Entry}

\begin{Entry}{费用}{9,5}{⾙、⽤}
  \begin{Phonetics}{费用}{fei4 yong4}[][HSK 3]
    \definition[笔,个]{s.}{custo; despesa; desembolso}
  \end{Phonetics}
\end{Entry}

\begin{Entry}{贺}{9}{⾙}
  \begin{Phonetics}{贺}{he4}
    \definition*{s.}{Sobrenome He}
    \definition{v.}{parabenizar; congratular | celebrar; comemorar}
  \end{Phonetics}
\end{Entry}

\begin{Entry}{贺卡}{9,5}{⾙、⼘}
  \begin{Phonetics}{贺卡}{he4 ka3}[][HSK 5]
    \definition[张]{s.}{cartão de felicitações; pedaço de papel para parabenizar amigos e parentes em seu casamento, aniversário ou festivais, geralmente impresso com palavras e desenhos de felicitações}
  \end{Phonetics}
\end{Entry}

\begin{Entry}{轴}{9}{⾞}
  \begin{Phonetics}{轴}{zhou2}
    \definition{adj.}{(movimento) inflexível; rígido; desajeitado | direto; franco; decidido | em pergaminho}
    \definition{clas.}{usado para as linhas enroladas ao redor do eixo e as pinturas montadas no eixo}
    \definition{s.}{eixo | carretel; haste | rolo; pergaminho; objeto de enrolamento cilíndrico}
  \end{Phonetics}
  \begin{Phonetics}{轴}{zhou4}
    \definition{s.}{a parte final da performance; a última e central peça de uma peça dramática}
  \end{Phonetics}
\end{Entry}

\begin{Entry}{轴承}{9,8}{⾞、⼿}
  \begin{Phonetics}{轴承}{zhou2cheng2}
    \definition{s.}{(mecânico) rolamento}
  \end{Phonetics}
\end{Entry}

\begin{Entry}{轻}{9}{⾞}
  \begin{Phonetics}{轻}{qing1}[][HSK 2]
    \definition{adj.}{de pouco peso; leve (oposto de 重) | (de carga, equipamento, etc.) pequeno; simples | pequeno em número, grau, etc. | não sério; relaxante; leve | sem importância | suave; delicado | levianos, crédulos | leve; peso leve; densidade baixa | leve; descontraído; fácil | imprudente; descuidado | inconstante; frívolo}
    \definition{v.}{menosprezar; subestimar}
  \seealsoref{重}{zhong4}
  \end{Phonetics}
\end{Entry}

\begin{Entry}{轻易}{9,8}{⾞、⽇}
  \begin{Phonetics}{轻易}{qing1yi4}[][HSK 4]
    \definition{adv.}{facilmente; prontamente | facilmente; precipitadamente; indica que uma ação é realizada casualmente, geralmente usado em frases negativas}
  \end{Phonetics}
\end{Entry}

\begin{Entry}{轻松}{9,8}{⾞、⽊}
  \begin{Phonetics}{轻松}{qing1song1}[][HSK 4]
    \definition{adj.}{leve; relaxado; livre de fardos; não nervoso; não cansado}
    \definition{v.}{sentir-se livre de fardos; não se sentir nervoso ou cansado}
  \end{Phonetics}
\end{Entry}

\begin{Entry}{迷}{9}{⾡}
  \begin{Phonetics}{迷}{mi2}[][HSK 3]
    \definition[个]{s.}{fã; entusiasta; aficionado; pessoa que gosta excessivamente de algo}
    \definition{v.}{estar confuso; perder o rumo; se perder-se; perda da capacidade de discernimento e julgamento | ficar fascinado por; entregar-se a; ficar encantado com (por); ser louco por | confundir; desorientar; fascinar; encantar; tornar indistinto; deixar encantado e fascinado}
  \end{Phonetics}
\end{Entry}

\begin{Entry}{迷人}{9,2}{⾡、⼈}
  \begin{Phonetics}{迷人}{mi2ren2}[][HSK 5]
    \definition{adj.}{encantador; fascinante; sedutor; hipnotizante}
    \definition{v.}{confundir; intrigar; enganar}
  \end{Phonetics}
\end{Entry}

\begin{Entry}{迷你}{9,7}{⾡、⼈}
  \begin{Phonetics}{迷你}{mi2ni3}
    \definition{adj.}{(empréstimo linguístico) mini, como em minissaia ou \emph{Mini Cooper}}
  \end{Phonetics}
\end{Entry}

\begin{Entry}{迷信}{9,9}{⾡、⼈}
  \begin{Phonetics}{迷信}{mi2xin4}[][HSK 5]
    \definition{s.}{superstição; crença supersticiosa | fé cega; adoração cega}
    \definition{v.}{ter fé cega em; ter um fetiche de}
  \end{Phonetics}
\end{Entry}

\begin{Entry}{迷宫}{9,9}{⾡、⼧}
  \begin{Phonetics}{迷宫}{mi2gong1}
    \definition{s.}{labirinto}
  \end{Phonetics}
\end{Entry}

\begin{Entry}{迷恋}{9,10}{⾡、⼼}
  \begin{Phonetics}{迷恋}{mi2lian4}
    \definition{adj.}{obcecado}
    \definition{v.}{estar/ser apaixonado por | ficar encantado por | estar/ser obcecado por}
  \end{Phonetics}
\end{Entry}

\begin{Entry}{迷路}{9,13}{⾡、⾜}
  \begin{Phonetics}{迷路}{mi2lu4}
    \definition{s.}{labirinto | ouvido interno}
    \definition{v.+compl.}{perder o caminho | perder-se | seguir pelo caminho errado | não conseguir encontrar o caminho}
  \end{Phonetics}
\end{Entry}

\begin{Entry}{追}{9}{⾡}
  \begin{Phonetics}{追}{zhui1}[][HSK 3]
    \definition{v.}{perseguir; correr atrás; seguir de perto | rastrear; investigar; chegar ao fundo de | procurar; ir atrás; esforçar-se para alcançar um determinado objetivo | recordar; relembrar | fazer depois do ocorrido; retrabalhar | cortejar (uma mulher)}
  \end{Phonetics}
\end{Entry}

\begin{Entry}{追求}{9,7}{⾡、⽔}
  \begin{Phonetics}{追求}{zhui1qiu2}[][HSK 4]
    \definition{s.}{perseguição (ações e metas positivas)}[她的追求是获得成功。===Sua meta é alcançar o sucesso.]
    \definition{v.}{procurar; aspirar; perseguir | cortejar; refere-se especificamente ao namoro}
  \end{Phonetics}
\end{Entry}

\begin{Entry}{追究}{9,7}{⾡、⽳}
  \begin{Phonetics}{追究}{zhui1jiu1}[][HSK 6]
    \definition{v.}{descobrir; investigar}
  \end{Phonetics}
\end{Entry}

\begin{Entry}{追赶}{9,10}{⾡、⾛}
  \begin{Phonetics}{追赶}{zhui1gan3}
    \definition{v.}{perseguir | acelerar | alcançar | ultrapassar}
  \end{Phonetics}
\end{Entry}

\begin{Entry}{退}{9}{⾡}
  \begin{Phonetics}{退}{tui4}[][HSK 3]
    \definition{v.}{recuar; mover-se para trás  (oposto de 進) | remover; retirar; fazer recuar; mover para trás | desistir; retirar-se de | refluir; declinar; retroceder | aposentar-se; deixar o emprego por atingir a idade estipulada ou por problemas de saúde | retornar; reembolsar; devolver | romper; cancelar o que foi decidido}
  \seealsoref{进}{jin4}
  \end{Phonetics}
\end{Entry}

\begin{Entry}{退出}{9,5}{⾡、⼐}
  \begin{Phonetics}{退出}{tui4 chu1}[][HSK 3]
    \definition{v.}{desistir; retirar-se; separar-se; retirar-se de; abandonar o local ou outro lugar e parar de participar; abandonaar o grupo ou organização}
  \end{Phonetics}
\end{Entry}

\begin{Entry}{退休}{9,6}{⾡、⼈}
  \begin{Phonetics}{退休}{tui4xiu1}[][HSK 3]
    \definition{v.+compl.}{aposentar-se; os trabalhadores que deixarem o emprego por velhice ou invalidez causada pelo trabalho receberão as despesas de subsistência conforme o cronograma}
  \end{Phonetics}
\end{Entry}

\begin{Entry}{退票}{9,11}{⾡、⽰}
  \begin{Phonetics}{退票}{tui4 piao4}[][HSK 6]
    \definition{s.}{bilhete devolvido (ou não utilizado) | reembolso do bilhete}
    \definition{v.}{devolver um bilhete; obter um reembolso por um bilhete | devolver (um cheque)}
  \end{Phonetics}
\end{Entry}

\begin{Entry}{送}{9}{⾡}
  \begin{Phonetics}{送}{song4}[][HSK 1]
    \definition*{s.}{Sobrenome Song}
    \definition{v.}{transportar; entregar | dar; dar como presente; presentear | acompanhar; despedir-se de alguém (ao sair); acompanhar a pessoa que está partindo até o destino ou caminhar um trecho com ela | escoltar}
  \end{Phonetics}
\end{Entry}

\begin{Entry}{送礼}{9,5}{⾡、⽰}
  \begin{Phonetics}{送礼}{song4 li3}[][HSK 6]
    \definition{v.}{dar um presente a alguém; presentear alguém com um presente | enviar presentes (para obter favores) | dar um presente; enviar um presente}
  \end{Phonetics}
\end{Entry}

\begin{Entry}{送行}{9,6}{⾡、⾏}
  \begin{Phonetics}{送行}{song4 xing2}[][HSK 6]
    \definition{v.}{ver alguém partir; ir até o local onde o viajante iniciou sua jornada, despedir-se dele e observar ele partir | dar uma festa de despedida; realizar uma festa de despedida | despedir-se do falecido}
  \end{Phonetics}
\end{Entry}

\begin{Entry}{送到}{9,8}{⾡、⼑}
  \begin{Phonetics}{送到}{song4 dao4}[][HSK 2]
    \definition{v.}{enviar para (lugar)}
  \end{Phonetics}
\end{Entry}

\begin{Entry}{送给}{9,9}{⾡、⽷}
  \begin{Phonetics}{送给}{song4 gei3}[][HSK 2]
    \definition{v.}{dar a (alguém ou organização); dar como algo gratuito; dar como presente}
  \end{Phonetics}
\end{Entry}

\begin{Entry}{适}{9}{⾡}
  \begin{Phonetics}{适}{shi4}
    \definition*{s.}{Sobrenome Shi}
    \definition{adj.}{confortável; bem | adequado; apropriado | certo; oportuno}
    \definition{v.}{ser apto; ser adequado; ser apropriado | ir; seguir; perseguir | (de uma mulher) casar}
  \end{Phonetics}
\end{Entry}

\begin{Entry}{适用}{9,5}{⾡、⽤}
  \begin{Phonetics}{适用}{shi4 yong4}[][HSK 3]
    \definition{adj.}{adequado; aplicável}
  \end{Phonetics}
\end{Entry}

\begin{Entry}{适合}{9,6}{⾡、⼝}
  \begin{Phonetics}{适合}{shi4he2}[][HSK 3]
    \definition{v.}{servir; caber; se adequar; atender às necessidades de uma determinada situação ou pessoa}
  \end{Phonetics}
\end{Entry}

\begin{Entry}{适当}{9,6}{⾡、⼹}
  \begin{Phonetics}{适当}{shi4 dang4}[][HSK 6]
    \definition{s.}{adequado; apropriado}
  \end{Phonetics}
\end{Entry}

\begin{Entry}{适应}{9,7}{⾡、⼴}
  \begin{Phonetics}{适应}{shi4ying4}[][HSK 3]
    \definition{v.}{ajustar-se; adequar-se; adaptar-se; fazer as alterações correspondentes para se adequar à medida que as condições mudam}
  \end{Phonetics}
\end{Entry}

\begin{Entry}{逃}{9}{⾡}
  \begin{Phonetics}{逃}{tao2}[][HSK 5]
    \definition{v.}{fugir; escapar; correr; dar no pé | evadir; esquivar-se; escapar}
  \end{Phonetics}
\end{Entry}

\begin{Entry}{逃走}{9,7}{⾡、⾛}
  \begin{Phonetics}{逃走}{tao2 zou3}[][HSK 5]
    \definition{v.}{escapar; afastar-se de pessoas, coisas ou lugares que não são bons para você ou que você não gosta}
  \end{Phonetics}
\end{Entry}

\begin{Entry}{逃跑}{9,12}{⾡、⾜}
  \begin{Phonetics}{逃跑}{tao2 pao3}[][HSK 5]
    \definition{v.}{fugir; escapar; correr; partir para fugir de um ambiente ou de coisas que não lhe são favoráveis}
  \end{Phonetics}
\end{Entry}

\begin{Entry}{逆}{9}{⾡}
  \begin{Phonetics}{逆}{ni4}
    \definition{adj.}{contrário (oposto a 顺) ; contra; oposto; inverso | traidor; rebelde}
    \definition{adv.}{antecipadamente; com antecedência}
    \definition{s.}{traidor; rebelde}
    \definition{v.}{ir contra; opor-se; desobedecer; resistir; desafiar (oposto a 顺) | (literário) saudar; cumprimentar}
  \seealsoref{顺}{shun4}
  \end{Phonetics}
\end{Entry}

\begin{Entry}{逆境}{9,14}{⾡、⼟}
  \begin{Phonetics}{逆境}{ni4jing4}
    \definition{s.}{adversidade | tribulação}
  \end{Phonetics}
\end{Entry}

\begin{Entry}{选}{9}{⾡}
  \begin{Phonetics}{选}{xuan3}[][HSK 2]
    \definition{s.}{pessoa ou coisa selecionada | seleções; antologia; trabalhos selecionados e compilados}
    \definition{v.}{selecionar; escolher | eleger}
  \end{Phonetics}
\end{Entry}

\begin{Entry}{选手}{9,4}{⾡、⼿}
  \begin{Phonetics}{选手}{xuan3shou3}[][HSK 3]
    \definition[位,名,个,些]{s.}{jogador; (selecionado) competidor; atleta selecionado para uma competição esportiva; participantes selecionados entre um grande número de candidatos}
  \end{Phonetics}
\end{Entry}

\begin{Entry}{选拔}{9,8}{⾡、⼿}
  \begin{Phonetics}{选拔}{xuan3ba2}[][HSK 6]
    \definition{v.}{selecionar; escolher}
  \end{Phonetics}
\end{Entry}

\begin{Entry}{选择}{9,8}{⾡、⼿}
  \begin{Phonetics}{选择}{xuan3ze2}[][HSK 4]
    \definition[个,种,次]{s.}{escolha; opção; resultado da escolha; possibilidade de escolha}
    \definition{v.}{selecionar; escolher}
  \end{Phonetics}
\end{Entry}

\begin{Entry}{选举}{9,9}{⾡、⼂}
  \begin{Phonetics}{选举}{xuan3ju3}[][HSK 6]
    \definition[次,个]{s.}{eleição; as eleições são o processo pelo qual os cidadãos escolhem os seus representantes ou líderes através do voto}
    \definition{v.}{votar; eleger; eleger representantes ou responsáveis ​​votando ou levantando as mãos}
  \end{Phonetics}
\end{Entry}

\begin{Entry}{选修}{9,9}{⾡、⼈}
  \begin{Phonetics}{选修}{xuan3 xiu1}[][HSK 5]
    \definition{v.}{fazer como disciplina eletiva; escolher entre uma seleção de cursos disponíveis}
  \end{Phonetics}
\end{Entry}

\begin{Entry}{重}{9}{⾥}
  \begin{Phonetics}{重}{chong2}
    \definition*{s.}{Sobrenome Chong}
    \definition{adv.}{novamente; mais uma vez}
    \definition{clas.}{usado para camadas}
    \definition{v.}{repetir; duplicar}
  \end{Phonetics}
  \begin{Phonetics}{重}{zhong4}[][HSK 1,3]
    \definition{adj.}{pesado; densidade elevada | profundo; sério; grau profundo | importante; significativo | discreto; prudente | considerável em quantidade ou valor}
    \definition[斤,公,斤,吨]{s.}{peso}
    \definition{v.}{colocar (colocar, pôr) ênfase em; dar valor a; atribuir importância a}
  \end{Phonetics}
\end{Entry}

\begin{Entry}{重大}{9,3}{⾥、⼤}
  \begin{Phonetics}{重大}{zhong4da4}[][HSK 3]
    \definition{adj.}{excelente; importante; significativo; de grande importância}
  \end{Phonetics}
\end{Entry}

\begin{Entry}{重阳节}{9,6,5}{⾥、⾩、⾋}
  \begin{Phonetics}{重阳节}{chong2yang2jie2}
    \definition*{s.}{Festa do Duplo Nove, Festival Yang, dia de subir aos lugares mais altos para evitar calamidades e dia do culto aos antepassados (9º dia do nono mês lunar)}
  \end{Phonetics}
\end{Entry}

\begin{Entry}{重建}{9,8}{⾥、⼵}
  \begin{Phonetics}{重建}{chong2 jian4}[][HSK 6]
    \definition{s.}{restabelecimento; reconstrução}
    \definition{v.}{reconstruir; reconstruir; restabelecer; reabilitar}
  \end{Phonetics}
\end{Entry}

\begin{Entry}{重组}{9,8}{⾥、⽷}
  \begin{Phonetics}{重组}{chong2 zu3}[][HSK 6]
    \definition{v.}{reestruturar; reorganizar; remanejar}
  \end{Phonetics}
\end{Entry}

\begin{Entry}{重视}{9,8}{⾥、⾒}
  \begin{Phonetics}{重视}{zhong4shi4}[][HSK 2]
    \definition{v.}{valorizar; dar peso a; atribuir importância a; prestar atenção a; considerar a virtude ou o talento de uma pessoa ou o papel de algo como importante e levá-lo a sério}
  \end{Phonetics}
\end{Entry}

\begin{Entry}{重复}{9,9}{⾥、⼢}
  \begin{Phonetics}{重复}{chong2fu4}[][HSK 2]
    \definition{v.}{repetir; iterar; duplicar; reduplicar | fazer algo novamente; repetir as mesmas palavras, fazer as mesmas coisas}
  \end{Phonetics}
\end{Entry}

\begin{Entry}{重点}{9,9}{⾥、⽕}
  \begin{Phonetics}{重点}{chong2dian3}
    \definition[个]{adj./adv./s.}{nota principal; ponto-chave; ponto focal; ênfase}
  \end{Phonetics}
  \begin{Phonetics}{重点}{zhong4dian3}[][HSK 2]
    \definition[个]{s.}{nota principal; ponto-chave; ponto}
  \end{Phonetics}
\end{Entry}

\begin{Entry}{重要}{9,9}{⾥、⾑}
  \begin{Phonetics}{重要}{zhong4yao4}[][HSK 1]
    \definition{adj.}{importante; significativo; relevante; de grande importância, função e impacto}
  \end{Phonetics}
\end{Entry}

\begin{Entry}{重重}{9,9}{⾥、⾥}
  \begin{Phonetics}{重重}{chong2chong2}
    \definition{adv.}{camada após camada | um após o outro}
  \end{Phonetics}
  \begin{Phonetics}{重重}{zhong4zhong4}
    \definition{adv.}{fortemente | severamente}
  \end{Phonetics}
\end{Entry}

\begin{Entry}{重逢}{9,10}{⾥、⾡}
  \begin{Phonetics}{重逢}{chong2feng2}
    \definition{s.}{reunião}
    \definition{v.}{encontrar-se novamente | reunir-se}
  \end{Phonetics}
\end{Entry}

\begin{Entry}{重量}{9,12}{⾥、⾥}
  \begin{Phonetics}{重量}{zhong4liang4}[][HSK 4]
    \definition[个]{s.}{peso; a magnitude da força da gravidade em um objeto}
  \end{Phonetics}
\end{Entry}

\begin{Entry}{重新}{9,13}{⾥、⽄}
  \begin{Phonetics}{重新}{chong2xin1}[][HSK 2]
    \definition{adv.}{novamente; de novo; significa repetir uma ação ou comportamento já realizado | indica que se deve começar do início (mudança de método ou conteúdo)}
  \end{Phonetics}
\end{Entry}

\begin{Entry}{钝}{9}{⾦}
  \begin{Phonetics}{钝}{dun4}
    \definition{adj.}{sem corte; opaco (oposto a 快, 利, 锐) | estúpido; sem noção | maçante}
  \seealsoref{快}{kuai4}
  \seealsoref{利}{li4}
  \seealsoref{锐}{rui4}
  \end{Phonetics}
\end{Entry}

\begin{Entry}{钟}{9}{⾦}
  \begin{Phonetics}{钟}{zhong1}[][HSK 3]
    \definition*{s.}{Sobrenome Zhong}
    \definition[顶,个,口]{s.}{sino; campainha; um instrumento de percussão antigo, oco, feito de cobre ou ferro | relógio; um aparelho para medir o tempo que não se leva consigo | tempo medido em horas e minutos; referindo-se ao tempo ou momento| um recipiente antigo para guardar vinho, com barriga grande e gargalo pequeno | sino; refere-se especificamente aos sinos pendurados em templos ou outros locais, cujo som é usado para marcar as horas, alertar ou convocar pessoas}
    \definition{v.}{focar; concentrar (as afeições de alguém, etc.)}
  \end{Phonetics}
\end{Entry}

\begin{Entry}{钟头}{9,5}{⾦、⼤}
  \begin{Phonetics}{钟头}{zhong1 tou2}[][HSK 6]
    \definition[个]{s.}{hora; 60 minutos}[三四个钟头过去了。===Três ou quatro horas se passaram.]
  \end{Phonetics}
\end{Entry}

\begin{Entry}{钟室}{9,9}{⾦、⼧}
  \begin{Phonetics}{钟室}{zhong1shi4}
    \definition{s.}{campanário | sala do relógio}
  \end{Phonetics}
\end{Entry}

\begin{Entry}{钟罩}{9,13}{⾦、⽹}
  \begin{Phonetics}{钟罩}{zhong1zhao4}
    \definition{s.}{redoma | dossel de sino}
  \end{Phonetics}
\end{Entry}

\begin{Entry}{钢}{9}{⾦}
  \begin{Phonetics}{钢}{gang1}
    \definition[吨,块,根]{s.}{aço; liga de ferro e carbono}
  \end{Phonetics}
\end{Entry}

\begin{Entry}{钢丝}{9,5}{⾦、⼀}
  \begin{Phonetics}{钢丝}{gang1si1}
    \definition{s.}{cabo de aço | corda bamba}
  \end{Phonetics}
\end{Entry}

\begin{Entry}{钢笔}{9,10}{⾦、⽵}
  \begin{Phonetics}{钢笔}{gang1 bi3}[][HSK 5]
    \definition[支,杆]{s.}{caneta-tinteiro; canetas com ponta metálica}
  \end{Phonetics}
\end{Entry}

\begin{Entry}{钢琴}{9,12}{⾦、⽟}
  \begin{Phonetics}{钢琴}{gang1qin2}[][HSK 5]
    \definition[架,台]{s.}{piano}
  \end{Phonetics}
\end{Entry}

\begin{Entry}{钥}{9}{⾦}
  \begin{Phonetics}{钥}{yao4}
    \definition{s.}{chave}
  \end{Phonetics}
\end{Entry}

\begin{Entry}{钥匙}{9,11}{⾦、⼔}
  \begin{Phonetics}{钥匙}{yao4shi5}
    \definition[把]{s.}{chave}
  \end{Phonetics}
\end{Entry}

\begin{Entry}{钥匙孔}{9,11,4}{⾦、⼔、⼦}
  \begin{Phonetics}{钥匙孔}{yao4shi5kong3}
    \definition{s.}{buraco da fechadura}
  \end{Phonetics}
\end{Entry}

\begin{Entry}{钥匙卡}{9,11,5}{⾦、⼔、⼘}
  \begin{Phonetics}{钥匙卡}{yao4shi5ka3}
    \definition{s.}{cartão de acesso}
  \end{Phonetics}
\end{Entry}

\begin{Entry}{钥匙洞孔}{9,11,9,4}{⾦、⼔、⽔、⼦}
  \begin{Phonetics}{钥匙洞孔}{yao4shi5dong4kong3}
    \definition{s.}{buraco da fechadura}
  \end{Phonetics}
\end{Entry}

\begin{Entry}{钥匙圈}{9,11,11}{⾦、⼔、⼞}
  \begin{Phonetics}{钥匙圈}{yao4shi5quan1}
    \definition{s.}{chaveiro}
  \end{Phonetics}
\end{Entry}

\begin{Entry}{钩}{9}{⾦}
  \begin{Phonetics}{钩}{gou1}
    \definition*{s.}{Sobrenome Gou}
    \definition[只,个]{s.}{gancho | traço de gancho em caracteres chineses | marca de verificação; visto; \emph{tick}; \emph{check mark} | marca em forma de gancho | uma espada em forma de gancho | forma falada do numeral 9 em certas ocasiões}
    \definition{v.}{prender com um gancho; enganchar | fazer crochê | costurar com pontos grandes | costurar com pontos longos}
  \end{Phonetics}
\end{Entry}

\begin{Entry}{闻}{9}{⾨}
  \begin{Phonetics}{闻}{wen2}[][HSK 2]
    \definition*{s.}{Sobrenome Wen}
    \definition{adj.}{bem conhecido; famoso}
    \definition{s.}{notícia; história | reputação | boato; rumor}
    \definition{v.}{cheirar | ouvir}
  \end{Phonetics}
\end{Entry}

\begin{Entry}{阁}{9}{⾨}
  \begin{Phonetics}{阁}{ge2}
    \definition{s.}{pavilhão (geralmente de dois andares) | gabinete (de um governo) | (datado) quarto da mulher; \emph{boudoir} | prateleira}
  \end{Phonetics}
\end{Entry}

\begin{Entry}{阁下}{9,3}{⾨、⼀}
  \begin{Phonetics}{阁下}{ge2xia4}
    \definition{pron.}{Sua Excelência | Sua Majestade | \emph{Sire}}
  \end{Phonetics}
\end{Entry}

\begin{Entry}{院}{9}{⾩}
  \begin{Phonetics}{院}{yuan4}[][HSK 2]
    \definition*{s.}{Sobrenome Yuan}
    \definition[个]{s.}{pátio; quintal; complexo | designação para certos escritórios governamentais e locais públicos | faculdade; academia; instituto de ensino superior | hospital}
  \end{Phonetics}
\end{Entry}

\begin{Entry}{院子}{9,3}{⾩、⼦}
  \begin{Phonetics}{院子}{yuan4zi5}[][HSK 2]
    \definition[个,座,处]{s.}{quintal; pátio; o espaço aberto na frente ou atrás de uma casa cercado por muros ou cercas}
  \end{Phonetics}
\end{Entry}

\begin{Entry}{院长}{9,4}{⾩、⾧}
  \begin{Phonetics}{院长}{yuan4zhang3}[][HSK 2]
    \definition[个,位,名]{s.}{reitor; diretor; o mais alto funcionário de qualquer instituição ou escola pública ou privada}
  \end{Phonetics}
\end{Entry}

\begin{Entry}{除}{9}{⾩}
  \begin{Phonetics}{除}{chu2}[][HSK 6]
    \definition*{s.}{Sobrenome Chu}
    \definition{prep.}{exceto; não incluído | além do mais}
    \definition{s.}{degraus de uma casa; degraus de uma porta; escadaria}
    \definition{v.}{remover; livrar-se de; eliminar; limpar | dividir; executar operação de divisão | nomear para o cargo}
  \end{Phonetics}
\end{Entry}

\begin{Entry}{除了}{9,2}{⾩、⼅}
  \begin{Phonetics}{除了}{chu2le5}[][HSK 3]
    \definition{prep.}{exceto; à parte; indica que o que foi dito não é levado em consideração | além disso; além de; usado em conjunto com 还, 也 e 只, indica que, além de algo, há ainda outra coisa | ou \dots ou \dots; usado em conjunto com 就是, significa "ou assim ou assado"}
  \seealsoref{还}{hai2}
  \seealsoref{就是}{jiu4 shi4}
  \seealsoref{也}{ye3}
  \seealsoref{只}{zhi3}
  \end{Phonetics}
\end{Entry}

\begin{Entry}{除夕}{9,3}{⾩、⼣}
  \begin{Phonetics}{除夕}{chu2xi1}[][HSK 5]
    \definition*{s.}{Véspera de Ano Novo Lunar; a noite do último dia do ano, também se refere ao último dia do ano}
  \end{Phonetics}
\end{Entry}

\begin{Entry}{除非}{9,8}{⾩、⾮}
  \begin{Phonetics}{除非}{chu2fei1}[][HSK 5]
    \definition{conj.}{a menos que; somente se; indica a única condição, equivalente a 只有, frequentemente combinada com 才, 否则, 不然, etc.}
  \seealsoref{不然}{bu4ran2}
  \seealsoref{才}{cai2}
  \seealsoref{否则}{fou3ze2}
  \seealsoref{只有}{zhi3 you3}
  \end{Phonetics}
\end{Entry}

\begin{Entry}{险}{9}{⾩}
  \begin{Phonetics}{险}{xian3}[][HSK 6]
    \definition{adj.}{perigoso; arriscado | sinistro; cruel; venenoso}
    \definition{adv.}{por um fio de cabelo; por centímetros; quase}
    \definition{s.}{lugar de difícil acesso; lugar perigoso e difícil de atravessar; passagem estreita; desfiladeiro | abreviação de seguro, 保险 | perigo; risco}
  \seealsoref{保险}{bao3xian3}
  \end{Phonetics}
\end{Entry}

\begin{Entry}{面}{9}{⾯}[Kangxi 176]
  \begin{Phonetics}{面}{mian4}[][HSK 2]
    \definition*{s.}{Sobrenome Mian}
    \definition{adj.}{macio e farinhento; descreve algo que é muito macio ao comer | superficial}
    \definition{adv.}{diretamente; pessoalmente; na frente de alguém; cara a cara}
    \definition{clas.}{usado para objetos planos | usado para indicar o número de vezes que as pessoas se encontram}
    \definition[斤,两,碗]{s.}{face; parte frontal da cabeça; rosto | topo; superfície | capa; exterior; a parte externa de um objeto ou a face frontal de um tecido (em oposição à 里) | (matemática) superfície | cara; sentimento; emoção | geral; área total; abrangente; toda a região | lado; aspecto | escopo; escala; extensão; alcance; âmbito | farinha; farinha de trigo | pó; algo em pó | macarrão; \emph{noodle}}
    \definition{suf.}{sufixo para localização ou direção; anexado ao final de palavras que indicam localização, equivalente a 边}
    \definition{v.}{encarar algo | encontrar; revelar-se}
  \seealsoref{边}{bian1}
  \seealsoref{里}{li3}
  \end{Phonetics}
\end{Entry}

\begin{Entry}{面子}{9,3}{⾯、⼦}
  \begin{Phonetics}{面子}{mian4zi5}[][HSK 5]
    \definition{s.}{face; exterior; parte externa; superfície do objeto | imagem; reputação; prestígio; decência; vaidade superficial | sentimentos; sensibilidades | pó}
  \end{Phonetics}
\end{Entry}

\begin{Entry}{面包}{9,5}{⾯、⼓}
  \begin{Phonetics}{面包}{mian4bao1}[][HSK 1]
    \definition[个,片,袋,块]{s.}{pão}[我买八个面包了。===Comprei oito pães. | 他在吃两片面包。===Ele está comendo duas fatias de pão. | 我在家里带了一袋面包。===Trouxe um saco de pão para casa. | 我拿了一块面包。===Peguei um pedaço de pão.]
  \end{Phonetics}
\end{Entry}

\begin{Entry}{面对}{9,5}{⾯、⼨}
  \begin{Phonetics}{面对}{mian4dui4}[][HSK 3]
    \definition{v.}{enfrentar; defrontar; olhar para (uma pessoa ou um objeto específico) | confrontar (problema); problemas, dificuldades e outras questões que precisam ser resolvidas e que merecem atenção}
  \end{Phonetics}
\end{Entry}

\begin{Entry}{面对面}{9,5,9}{⾯、⼨、⾯}
  \begin{Phonetics}{面对面}{mian4 dui4 mian4}[][HSK 6]
    \definition*{expr.}{frente a frente; cara a cara; vis-à-vis}
  \end{Phonetics}
\end{Entry}

\begin{Entry}{面对面吃面}{9,5,9,6,9}{⾯、⼨、⾯、⼝、⾯}
  \begin{Phonetics}{面对面吃面}{mian4dui4mian4 chi1 mian4}
    \definition{expr.}{Comer macarrão cara a cara; indica que o seu estado atual, ou algumas das posições em que você está, ou algumas das coisas que você fez são muito claras}
  \end{Phonetics}
\end{Entry}

\begin{Entry}{面向}{9,6}{⾯、⼝}
  \begin{Phonetics}{面向}{mian4 xiang4}[][HSK 6]
    \definition{v.}{virar o rosto para; virar na direção de; defrontar; voltado para algum lugar | estar orientado para as necessidades de; atender a; principalmente para um certo tipo de pessoas}
  \end{Phonetics}
\end{Entry}

\begin{Entry}{面团}{9,6}{⾯、⼞}
  \begin{Phonetics}{面团}{mian4tuan2}
    \definition{s.}{massa | pasta}
  \end{Phonetics}
\end{Entry}

\begin{Entry}{面条}{9,7}{⾯、⽊}
  \begin{Phonetics}{面条}{mian4tiao2}
    \definition{s.}{macarrão | espaguete}
  \end{Phonetics}
\end{Entry}

\begin{Entry}{面条儿}{9,7,2}{⾯、⽊、⼉}
  \begin{Phonetics}{面条儿}{mian4 tiao2r5}[][HSK 1]
    \definition{s.}{macarrão; \emph{noodles}}
  \end{Phonetics}
\end{Entry}

\begin{Entry}{面试}{9,8}{⾯、⾔}
  \begin{Phonetics}{面试}{mian4 shi4}[][HSK 4]
    \definition{v.}{entrevistar (é realizado na forma de perguntas e respostas orais presenciais)}
  \end{Phonetics}
\end{Entry}

\begin{Entry}{面临}{9,9}{⾯、⼁}
  \begin{Phonetics}{面临}{mian4lin2}[][HSK 4]
    \definition{v.}{ser confrontado com; encontrar (uma situação) na frente de}
  \end{Phonetics}
\end{Entry}

\begin{Entry}{面前}{9,9}{⾯、⼑}
  \begin{Phonetics}{面前}{mian4 qian2}[][HSK 2]
    \definition{s.}{antes; na frente de; diante de}
  \end{Phonetics}
\end{Entry}

\begin{Entry}{面积}{9,10}{⾯、⽲}
  \begin{Phonetics}{面积}{mian4ji1}[][HSK 3]
    \definition{s.}{área (de um andar, pedaço de terreno, etc.); área de uma superfície; o tamanho de uma superfície plana ou da superfície de um objeto}
  \end{Phonetics}
\end{Entry}

\begin{Entry}{面貌}{9,14}{⾯、⾘}
  \begin{Phonetics}{面貌}{mian4mao4}[][HSK 5]
    \definition[种,个]{s.}{rosto; traços faciais; formato do rosto; aparência | aparência; aspecto; aparência (das coisas)}
  \end{Phonetics}
\end{Entry}

\begin{Entry}{革}{9}{⾰}[Kangxi 177]
  \begin{Phonetics}{革}{ge2}
    \definition*{s.}{Sobrenome Ge}
    \definition{s.}{couro; pele; peles de animais depiladas e processadas}
    \definition{v.}{mudar; transformar; reformar | demitir; remover do cargo; expulsar}
  \end{Phonetics}
\end{Entry}

\begin{Entry}{革新}{9,13}{⾰、⽄}
  \begin{Phonetics}{革新}{ge2 xin1}[][HSK 6]
    \definition{v.}{inovar; renovar; livrar-se do velho e criar o novo}
  \end{Phonetics}
\end{Entry}

\begin{Entry}{韭}{9}{⾲}[Kangxi 179]
  \begin{Phonetics}{韭}{jiu3}
    \definition{s.}{alho de flor perfumada; cebolinha chinesa}
  \end{Phonetics}
\end{Entry}

\begin{Entry}{韭菜}{9,11}{⾲、⾋}
  \begin{Phonetics}{韭菜}{jiu3cai4}
    \definition{s.}{cebolinha chinesa | (figurativo) investidores de varejo que perdem seu dinheiro para operadores mais experientes (ou seja, são ``colhidos'' como cebolinhas)}
  \end{Phonetics}
\end{Entry}

\begin{Entry}{音}{9}{⾳}[Kangxi 180]
  \begin{Phonetics}{音}{yin1}
    \definition[个,种]{s.}{som; som musical | notícias; novidades; informação | tom; refere-se especificamente a uma sílaba ou fonética | sílaba; refere-se a sílabas (um caractere chinês é uma sílaba)}
    \definition{v.}{vocalizar}
  \end{Phonetics}
\end{Entry}

\begin{Entry}{音乐}{9,5}{⾳、⼃}
  \begin{Phonetics}{音乐}{yin1yue4}[][HSK 2]
    \definition[种,段,张,曲]{s.}{música; ramo da arte que cria imagens artísticas, expressa pensamentos e sentimentos e reflete a vida real por meio da melodia e do ritmo da música; geralmente é dividido em duas categorias: música vocal e música instrumental}
  \end{Phonetics}
\end{Entry}

\begin{Entry}{音乐厅}{9,5,4}{⾳、⼃、⼚}
  \begin{Phonetics}{音乐厅}{yin1yue4ting1}
    \definition{s.}{auditório | teatro | \emph{concert hall}}
  \end{Phonetics}
\end{Entry}

\begin{Entry}{音乐节}{9,5,5}{⾳、⼃、⾋}
  \begin{Phonetics}{音乐节}{yin1yue4jie2}
    \definition{s.}{festival de música}
  \end{Phonetics}
\end{Entry}

\begin{Entry}{音乐会}{9,5,6}{⾳、⼃、⼈}
  \begin{Phonetics}{音乐会}{yin1 yue4 hui4}[][HSK 2]
    \definition[场]{s.}{concerto; atividades de execução de obras musicais}
  \end{Phonetics}
\end{Entry}

\begin{Entry}{音乐光碟}{9,5,6,14}{⾳、⼃、⼉、⽯}
  \begin{Phonetics}{音乐光碟}{yin1yue4guang1die2}
    \definition{s.}{CD de música}
  \end{Phonetics}
\end{Entry}

\begin{Entry}{音乐学}{9,5,8}{⾳、⼃、⼦}
  \begin{Phonetics}{音乐学}{yin1yue4xue2}
    \definition{s.}{musicologia}
  \end{Phonetics}
\end{Entry}

\begin{Entry}{音乐学院}{9,5,8,9}{⾳、⼃、⼦、⾩}
  \begin{Phonetics}{音乐学院}{yin1yue4xue2yuan4}
    \definition{s.}{conservatório | academia de música}
  \end{Phonetics}
\end{Entry}

\begin{Entry}{音乐院}{9,5,9}{⾳、⼃、⾩}
  \begin{Phonetics}{音乐院}{yin1yue4yuan4}
    \definition{s.}{conservatório | instituto de música}
  \end{Phonetics}
\end{Entry}

\begin{Entry}{音乐家}{9,5,10}{⾳、⼃、⼧}
  \begin{Phonetics}{音乐家}{yin1yue4jia1}
    \definition{s.}{músico}
  \end{Phonetics}
\end{Entry}

\begin{Entry}{音节}{9,5}{⾳、⾋}
  \begin{Phonetics}{音节}{yin1 jie2}[][HSK 2]
    \definition{s.}{sílaba}
  \end{Phonetics}
\end{Entry}

\begin{Entry}{音量}{9,12}{⾳、⾥}
  \begin{Phonetics}{音量}{yin1 liang4}[][HSK 6]
    \definition[把]{s.}{volume; volume do som; a força de um som}
  \end{Phonetics}
\end{Entry}

\begin{Entry}{音像}{9,13}{⾳、⼈}
  \begin{Phonetics}{音像}{yin1 xiang4}[][HSK 6]
    \definition{s.}{audiovisual; produtos audiovisuais; o nome coletivo para gravações de áudio e vídeo}
  \end{Phonetics}
\end{Entry}

\begin{Entry}{项}{9}{⾴}
  \begin{Phonetics}{项}{xiang4}[][HSK 4]
    \definition*{s.}{Sobrenome Xiang}
    \definition{clas.}{usado para itens discriminados; taxonomia}
    \definition{s.}{nuca (do pescoço); a parte de trás do pescoço | soma (de dinheiro); fundos para fins especiais | termo; em álgebra, significa uma única fórmula que não é unida por um sinal de mais ou de menos | item}
  \end{Phonetics}
\end{Entry}

\begin{Entry}{项目}{9,5}{⾴、⽬}
  \begin{Phonetics}{项目}{xiang4mu4}[][HSK 4]
    \definition{s.}{evento; categorias em que as coisas são divididas | item; projeto; trabalhos de engenharia, acadêmicos, etc., de conteúdo específico}
  \end{Phonetics}
\end{Entry}

\begin{Entry}{顺}{9}{⾴}
  \begin{Phonetics}{顺}{shun4}[][HSK 6]
    \definition{adj.}{(de escritos) legível; claro e bem escrito; organizado | favorável; harmonioso | favorável; bem-sucedido}
    \definition{prep.}{conforme a conveniência de alguém | ao longo; a introdução da rota, situação ou oportunidade que a ação segue pode ser seguida por 着 | com a corrente; na mesma direção |  com; na mesma direção que}
    \definition{v.}{organizar; colocar em ordem; tornar as coisas organizadas ou ordenadas | obedecer; ceder a; agir em submissão a | ser adequado; ser agradável}
  \seealsoref{着}{zhe5}
  \end{Phonetics}
\end{Entry}

\begin{Entry}{顺从}{9,4}{⾴、⼈}
  \begin{Phonetics}{顺从}{shun4cong2}
    \definition{v.}{obedecer | submeter-se}
  \end{Phonetics}
\end{Entry}

\begin{Entry}{顺心}{9,4}{⾴、⼼}
  \begin{Phonetics}{顺心}{shun4xin1}
    \definition{adj.}{satisfatório | satisfeito}
  \end{Phonetics}
\end{Entry}

\begin{Entry}{顺水}{9,4}{⾴、⽔}
  \begin{Phonetics}{顺水}{shun4shui3}
    \definition{v.}{ir com o fluxo}
  \end{Phonetics}
\end{Entry}

\begin{Entry}{顺延}{9,6}{⾴、⼵}
  \begin{Phonetics}{顺延}{shun4yan2}
    \definition{v.}{adiar | procrastinar}
  \end{Phonetics}
\end{Entry}

\begin{Entry}{顺当}{9,6}{⾴、⼹}
  \begin{Phonetics}{顺当}{shun4dang5}
    \definition{adv.}{suavemente}
  \end{Phonetics}
\end{Entry}

\begin{Entry}{顺耳}{9,6}{⾴、⽿}
  \begin{Phonetics}{顺耳}{shun4'er3}
    \definition{adj.}{agradável ao ouvido}
  \end{Phonetics}
\end{Entry}

\begin{Entry}{顺利}{9,7}{⾴、⼑}
  \begin{Phonetics}{顺利}{shun4li4}[][HSK 2]
    \definition{adj.}{sem problemas; com sucesso; sem dificuldades; sem contratempos; sem obstáculos; sem obstáculos ou dificuldades significativas no desempenho das tarefas}
  \end{Phonetics}
\end{Entry}

\begin{Entry}{顺序}{9,7}{⾴、⼴}
  \begin{Phonetics}{顺序}{shun4xu4}[][HSK 4]
    \definition{adv.}{por sua vez; na ordem correta; na devida ordem; na ordem adequada; na ordem apropriada}
    \definition[个]{s.}{ordem; sequência; sucessão; subsequência; sequência simples; ordem de prioridade}
  \end{Phonetics}
\end{Entry}

\begin{Entry}{顺畅}{9,8}{⾴、⽥}
  \begin{Phonetics}{顺畅}{shun4chang4}
    \definition{adj.}{liso e sem obstáculos | fluente}
  \end{Phonetics}
\end{Entry}

\begin{Entry}{顺便}{9,9}{⾴、⼈}
  \begin{Phonetics}{顺便}{shun4bian4}
    \definition{adv.}{convenientemente | de passagem | sem muito esforço extra}
  \end{Phonetics}
\end{Entry}

\begin{Entry}{顺叙}{9,9}{⾴、⼜}
  \begin{Phonetics}{顺叙}{shun4xu4}
    \definition{s.}{narrativa cronológica}
  \end{Phonetics}
\end{Entry}

\begin{Entry}{顺眼}{9,11}{⾴、⽬}
  \begin{Phonetics}{顺眼}{shun4yan3}
    \definition{adj.}{agradável aos olhos}
  \end{Phonetics}
\end{Entry}

\begin{Entry}{顺境}{9,14}{⾴、⼟}
  \begin{Phonetics}{顺境}{shun4jing4}
    \definition{s.}{circunstâncias favoráveis}
  \end{Phonetics}
\end{Entry}

\begin{Entry}{顺嘴}{9,16}{⾴、⼝}
  \begin{Phonetics}{顺嘴}{shun4zui3}
    \definition{v.}{deixar escapar (sem pensar) | ler suavemente (texto) | adequar-se  ao gosto (comida)}
  \end{Phonetics}
\end{Entry}

\begin{Entry}{飒}{9}{⾵}
  \begin{Phonetics}{飒}{sa4}
    \definition{adj.}{(das mulheres) natural e desenfreada; elegante; valente}
    \definition{interj.}{(onomatopéia) farfalhar; sussurrar | (onomatopéia) som do vento}
    \definition{v.}{murchar}
  \end{Phonetics}
\end{Entry}

\begin{Entry}{飒飒}{9,9}{⾵、⾵}
  \begin{Phonetics}{飒飒}{sa4sa4}
    \definition{s.}{o farfalhar | sussurro | murmúrio (do vento nas árvores, o mar, etc.)}
  \end{Phonetics}
\end{Entry}

\begin{Entry}{食}{9}{⾷}[Kangxi 184]
  \begin{Phonetics}{食}{shi2}
    \definition{adj.}{para cozinhar; comestível}
    \definition{s.}{refeição; comida; o que as pessoas e os animais comem | alimentação; alimento para animais; ração | eclipse solar; eclipse lunar}
    \definition{v.}{comer}
  \end{Phonetics}
  \begin{Phonetics}{食}{si4}
    \definition{v.}{alimentar; dar comida a}
  \end{Phonetics}
\end{Entry}

\begin{Entry}{食物}{9,8}{⾷、⽜}
  \begin{Phonetics}{食物}{shi2wu4}[][HSK 2]
    \definition[种]{s.}{comida; alimentos; comestíveis}
  \end{Phonetics}
\end{Entry}

\begin{Entry}{食品}{9,9}{⾷、⼝}
  \begin{Phonetics}{食品}{shi2 pin3}[][HSK 3]
    \definition[种]{s.}{comida; gêneros alimentícios; provisões; alimentos vendidos em lojas que passaram por algum processamento}
  \end{Phonetics}
\end{Entry}

\begin{Entry}{食堂}{9,11}{⾷、⼟}
  \begin{Phonetics}{食堂}{shi2 tang2}[][HSK 4]
    \definition[个,间]{s.}{cantina; refeitório}
  \end{Phonetics}
\end{Entry}

\begin{Entry}{食欲}{9,11}{⾷、⽋}
  \begin{Phonetics}{食欲}{shi2 yu4}[][HSK 6]
    \definition{adj.}{apetitoso}
    \definition{s.}{apetite; desejo humano de comer}
  \end{Phonetics}
\end{Entry}

\begin{Entry}{饺}{9}{⾷}
  \begin{Phonetics}{饺}{jiao3}
    \definition[盘,碗,顿,个]{s.}{bolinho de massa; \emph{dumpling}}
  \end{Phonetics}
\end{Entry}

\begin{Entry}{饺子}{9,3}{⾷、⼦}
  \begin{Phonetics}{饺子}{jiao3zi5}[][HSK 2]
    \definition[个,盘,碗,锅]{s.}{jiaozi; bolinho chinês; bolinho de massa}
  \end{Phonetics}
\end{Entry}

\begin{Entry}{饼}{9}{⾷}
  \begin{Phonetics}{饼}{bing3}[][HSK 5]
    \definition[张]{s.}{um bolo redondo e plano; massa assada ou cozida no vapor | algo que tem o formato de um bolo; semelhante a uma torta}
  \end{Phonetics}
\end{Entry}

\begin{Entry}{饼干}{9,3}{⾷、⼲}
  \begin{Phonetics}{饼干}{bing3gan1}[][HSK 5]
    \definition[块,片,包,盒,袋]{s.}{biscoito; bolacha; \emph{cookie}; alimentos, pedaços pequenos e finos cozidos em farinha com açúcar, ovos, leite, etc.}
  \end{Phonetics}
\end{Entry}

\begin{Entry}{首}{9}{⾸}[Kangxi 185]
  \begin{Phonetics}{首}{shou3}[][HSK 4,6]
    \definition*{s.}{Sobrenome Shou}
    \definition{adj.}{primeiro}
    \definition{adv.}{inicialmente; como o primeiro; em primeiro lugar}
    \definition{clas.}{usado para canções e poemas}
    \definition{s.}{cabeça | cabeça; chefe; líder | capital (cidade)}
    \definition{v.}{apresentar acusações contra alguém}
  \end{Phonetics}
\end{Entry}

\begin{Entry}{首先}{9,6}{⾸、⼉}
  \begin{Phonetics}{首先}{shou3xian1}[][HSK 3]
    \definition{adv.}{primeiramente; antes de todos os outros}
    \definition{conj.}{acima de tudo; primeiramente; em primeiro lugar}
  \end{Phonetics}
\end{Entry}

\begin{Entry}{首次}{9,6}{⾸、⽋}
  \begin{Phonetics}{首次}{shou3 ci4}[][HSK 6]
    \definition{s.}{o primeiro; pela primeira vez}
  \end{Phonetics}
\end{Entry}

\begin{Entry}{首相}{9,9}{⾸、⽬}
  \begin{Phonetics}{首相}{shou3 xiang4}[][HSK 6]
    \definition*[个,名,位]{s.}{Primeiro-Ministro (Japão, UK, etc.); o mais alto cargo oficial no gabinete de uma monarquia; o chefe do governo central de alguns países não monárquicos às vezes usa esse nome}
  \end{Phonetics}
\end{Entry}

\begin{Entry}{首席}{9,10}{⾸、⼱}
  \begin{Phonetics}{首席}{shou3 xi2}[][HSK 6]
    \definition{adj.}{chefe; a primeira; a posição mais alta}
    \definition{s.}{assento de honra; o assento mais honroso}
  \end{Phonetics}
\end{Entry}

\begin{Entry}{首席执行官}{9,10,6,6,8}{⾸、⼱、⼿、⾏、⼧}
  \begin{Phonetics}{首席执行官}{shou3xi2 zhi2xing2 guan1}
    \definition{s.}{\emph{chief executive officer}, CEO}
  \end{Phonetics}
\end{Entry}

\begin{Entry}{首脑}{9,10}{⾸、⾁}
  \begin{Phonetics}{首脑}{shou3 nao3}[][HSK 6]
    \definition[位]{s.}{cabeça; líder; chefe}
  \end{Phonetics}
\end{Entry}

\begin{Entry}{首都}{9,10}{⾸、⾢}
  \begin{Phonetics}{首都}{shou3du1}[][HSK 3]
    \definition[个,座]{s.}{capital (cidade); a sede do mais alto poder político do país e o centro político do país}
  \end{Phonetics}
\end{Entry}

\begin{Entry}{香}{9}{⾹}[Kangxi 186]
  \begin{Phonetics}{香}{xiang1}[][HSK 3]
    \definition*{s.}{Sobrenome Xiang}
    \definition{adj.}{aromático; perfumado; fragrante; cheiroso; oposto a 臭 | saboroso; saboroso; delicioso; apetitoso | com gosto; com bom apetite | (sono) profundo; dormir confortavelmente e tranquilamente | popular; valorizado; apreciado}
    \definition[根,炷]{s.}{especiaria; perfume; fragrância; aromatizante; substância com aroma intenso | incenso; bastão de incenso; tiras finas feitas de serragem e especiarias, queimadas em rituais para honrar os antepassados ou deuses e budas, e também usadas para afastar odores desagradáveis ou mosquitos| antigamente, referia-se a coisas relacionadas com mulheres ou mulheres}
  \seealsoref{臭}{chou4}
  \end{Phonetics}
\end{Entry}

\begin{Entry}{香气}{9,4}{⾹、⽓}
  \begin{Phonetics}{香气}{xiang1qi4}
    \definition{s.}{fragrância | aroma | incenso}
  \end{Phonetics}
\end{Entry}

\begin{Entry}{香皂}{9,7}{⾹、⽩}
  \begin{Phonetics}{香皂}{xiang1zao4}
    \definition{s.}{sabonete | sabonete perfumado}
  \end{Phonetics}
\end{Entry}

\begin{Entry}{香肠}{9,7}{⾹、⾁}
  \begin{Phonetics}{香肠}{xiang1chang2}[][HSK 5]
    \definition[根]{s.}{salsicha; linguiça; alimento feito com intestino de porco, recheado com carne picada e temperos}
  \end{Phonetics}
\end{Entry}

\begin{Entry}{香味}{9,8}{⾹、⼝}
  \begin{Phonetics}{香味}{xiang1wei4}
    \definition[股]{s.}{fragrância | cheiro doce}
  \end{Phonetics}
\end{Entry}

\begin{Entry}{香波}{9,8}{⾹、⽔}
  \begin{Phonetics}{香波}{xiang1bo1}
    \definition{s.}{xampu}
  \end{Phonetics}
\end{Entry}

\begin{Entry}{香炉}{9,8}{⾹、⽕}
  \begin{Phonetics}{香炉}{xiang1lu2}
    \definition{s.}{incensário (para queimar incenso) | queimador de incenso | insensório, turíbulo}
  \end{Phonetics}
\end{Entry}

\begin{Entry}{香烟}{9,10}{⾹、⽕}
  \begin{Phonetics}{香烟}{xiang1yan1}
    \definition[支,条]{s.}{cigarro | fumaça de incenso queimado}
  \end{Phonetics}
\end{Entry}

\begin{Entry}{香艳}{9,10}{⾹、⾊}
  \begin{Phonetics}{香艳}{xiang1yan4}
    \definition{adj.}{atraente | erótico | romântico}
  \end{Phonetics}
\end{Entry}

\begin{Entry}{香港}{9,12}{⾹、⽔}
  \begin{Phonetics}{香港}{xiang1gang3}
    \definition*{s.}{Hong Kong}
  \seealsoref{香港岛}{xiang1gang3 dao3}
  \end{Phonetics}
\end{Entry}

\begin{Entry}{香港岛}{9,12,7}{⾹、⽔、⼭}
  \begin{Phonetics}{香港岛}{xiang1gang3 dao3}
    \definition*{s.}{Ilha de Hong Kong}
  \seealsoref{香港}{xiang1gang3}
  \end{Phonetics}
\end{Entry}

\begin{Entry}{香槟酒}{9,14,10}{⾹、⽊、⾣}
  \begin{Phonetics}{香槟酒}{xiang1bin1jiu3}
    \definition[杯]{s.}{(empréstimo linguístico) \emph{champagne}}
  \end{Phonetics}
\end{Entry}

\begin{Entry}{香蕈}{9,15}{⾹、⾋}
  \begin{Phonetics}{香蕈}{xiang1xun4}
    \definition{s.}{\emph{shiitake}, cogumelo comestível}
  \end{Phonetics}
\end{Entry}

\begin{Entry}{香蕉}{9,15}{⾹、⾋}
  \begin{Phonetics}{香蕉}{xiang1jiao1}[][HSK 3]
    \definition[枝,根,个,把,串,束,弓]{s.}{banana}
  \end{Phonetics}
\end{Entry}

\begin{Entry}{骂}{9}{⾺}
  \begin{Phonetics}{骂}{ma4}[][HSK 5]
    \definition{v.}{abusar; xingar; insultar; insultar alguém com palavras grosseiras ou maliciosas | repreender; censurar; condenar}
  \end{Phonetics}
\end{Entry}

\begin{Entry}{骂名}{9,6}{⾺、⼝}
  \begin{Phonetics}{骂名}{ma4ming2}
    \definition{s.}{infâmia}
  \end{Phonetics}
\end{Entry}

\begin{Entry}{骂街}{9,12}{⾺、⾏}
  \begin{Phonetics}{骂街}{ma4jie1}
    \definition{v.}{gritar abusos na rua}
  \end{Phonetics}
\end{Entry}

\begin{Entry}{骄}{9}{⾺}
  \begin{Phonetics}{骄}{jiao1}
    \definition{adj.}{orgulhoso; arrogante; vaidoso | Literário: feroz; intenso; forte; violento}
  \end{Phonetics}
\end{Entry}

\begin{Entry}{骄傲}{9,12}{⾺、⼈}
  \begin{Phonetics}{骄傲}{jiao1'ao4}[][HSK 6]
    \definition{adj.}{arrogante; vaidoso; orgulhoso}
    \definition{s.}{orgulho; pessoas ou coisas das quais se orgulhar}
  \end{Phonetics}
\end{Entry}

\begin{Entry}{骆}{9}{⾺}
  \begin{Phonetics}{骆}{luo4}
    \definition*{s.}{Sobrenome Luo}
    \definition[只]{s.}{Arcaico: um cavalo branco com crina preta, mencionado em antigos livros chineses}
  \end{Phonetics}
\end{Entry}

\begin{Entry}{骆驼}{9,8}{⾺、⾺}
  \begin{Phonetics}{骆驼}{luo4tuo5}
    \definition[头,只,匹]{s.}{camelo | coloquial: cabeça-dura, idiota}
  \end{Phonetics}
\end{Entry}

\begin{Entry}{骨}{9}{⾻}[Kangxi 188]
  \begin{Phonetics}{骨}{gu3}
    \definition*{s.}{Sobrenome Gu}
    \definition[根,块]{s.}{osso | esqueleto; estrutura | caráter; espírito | cadáver; corpo}
  \end{Phonetics}
\end{Entry}

\begin{Entry}{骨头}{9,5}{⾻、⼤}
  \begin{Phonetics}{骨头}{gu3tou5}[][HSK 4]
    \definition[根,块]{s.}{osso; tecidos mais duros no corpo de uma pessoa ou de alguns animais que sustentam o corpo ou protegem os órgãos do corpo | caráter de uma pessoa; refere-se à qualidade do caráter de uma pessoa}
  \end{Phonetics}
\end{Entry}

\begin{Entry}{鬼}{9}{⿁}[Kangxi 194]
  \begin{Phonetics}{鬼}{gui3}[][HSK 5]
    \definition*{s.}{Gui, uma das mansões lunares | Gui, a vigésima terceira das vinte e oito constelações em que a esfera celeste foi dividida, consistindo de quatro estrelas em Câncer | Sobrenome Gui}
    \definition{adj.}{evasivo; furtivo; sub-reptício; ardiloso; enganoso, malicioso; obscuro | terrível; ruim; severo; vil | esperto; astuto; inteligente}
    \definition{s.}{espírito; fantasma; aparição; refere-se à alma de uma pessoa após a morte | usado para formar um termo de abuso para caráter ignóbil; refere-se a pessoas que têm maus hábitos ou cujo comportamento é repugnante | companheiro; pessoa que é considerada divertida}
  \end{Phonetics}
\end{Entry}

\begin{Entry}{鬼火}{9,4}{⿁、⽕}
  \begin{Phonetics}{鬼火}{gui3huo3}
    \definition{s.}{fogo-fátuo | boitatá | fogo corredor | fogo de santelmo}
  \end{Phonetics}
\end{Entry}

\begin{Entry}{鬼怪}{9,8}{⿁、⼼}
  \begin{Phonetics}{鬼怪}{gui3guai4}
    \definition{s.}{\emph{hobgoblin} | bicho-papão | fantasma}
  \end{Phonetics}
\end{Entry}

%%%%% EOF %%%%%

