%%%
%%% 11画
%%%

\section*{11画}\addcontentsline{toc}{section}{11画}

\begin{entry}{假}{11}[Radical ⼈]
  \begin{phonetics}{假}{jia3}[][HSK 2]
    \definition{adj.}{falso | artificial}
    \definition{v.}{emprestar}
  \end{phonetics}
  \begin{phonetics}{假}{jia4}
    \definition{s.}{férias}
  \end{phonetics}
\end{entry}

\begin{entry}{假如}{11,6}[Radicais ⼈、⼥]
  \begin{phonetics}{假如}{jia3ru2}
    \definition{conj.}{se | supondo | em caso}
  \end{phonetics}
\end{entry}

\begin{entry}{假声}{11,7}[Radicais ⼈、⼠]
  \begin{phonetics}{假声}{jia3sheng1}
    \definition{s.}{falsete}
  \seealsoref{真声}{zhen1sheng1}
  \end{phonetics}
\end{entry}

\begin{entry}{假证件}{11,7,6}[Radicais ⼈、⾔、⼈]
  \begin{phonetics}{假证件}{jia3zheng4jian4}
    \definition{s.}{documentos falsos}
  \end{phonetics}
\end{entry}

\begin{entry}{假使}{11,8}[Radicais ⼈、⼈]
  \begin{phonetics}{假使}{jia3shi3}
    \definition{conj.}{se | supondo | em caso}
  \end{phonetics}
\end{entry}

\begin{entry}{假的}{11,8}[Radicais ⼈、⽩]
  \begin{phonetics}{假的}{jia3de5}
    \definition{adj.}{falso | substituto | simulado}
  \end{phonetics}
\end{entry}

\begin{entry}{假期}{11,12}[Radicais ⼈、⽉]
  \begin{phonetics}{假期}{jia4 qi1}[][HSK 2]
    \definition[个]{s.}{férias | feriados | período de licença}
  \end{phonetics}
\end{entry}

\begin{entry}{偏偏}{11,11}[Radicais ⼈、⼈]
  \begin{phonetics}{偏偏}{pian1pian1}
    \definition{adv.}{voluntariamente | insistentemente | persistentemente | ao contrário da expectativa | infelizmente (indicando que alguma coisa aconteceu ao contrário do que se esperava) | teimosamente (indicando que algo é o oposto ao que seria normal ou razoável) | precisamente (indicando que alguém ou um grupo é escolhido)}
  \end{phonetics}
\end{entry}

\begin{entry}{做}{11}[Radical ⼈]
  \begin{phonetics}{做}{zuo4}[][HSK 1]
    \definition{v.}{fazer}
  \end{phonetics}
\end{entry}

\begin{entry}{做生活}{11,5,9}[Radicais ⼈、⽣、⽔]
  \begin{phonetics}{做生活}{zuo4sheng1huo2}
    \definition{v.}{fazer tabalhos manuais}
  \end{phonetics}
\end{entry}

\begin{entry}{做戏}{11,6}[Radicais ⼈、⼽]
  \begin{phonetics}{做戏}{zuo4xi4}
    \definition{v.}{atuar em uma peça | fazer uma peça}
  \end{phonetics}
\end{entry}

\begin{entry}{做作}{11,7}[Radicais ⼈、⼈]
  \begin{phonetics}{做作}{zuo4zuo5}
    \definition{adj.}{afetado | artificial}
  \end{phonetics}
\end{entry}

\begin{entry}{做饭}{11,7}[Radicais ⼈、⾷]
  \begin{phonetics}{做饭}{zuo4 fan4}[][HSK 2]
    \definition{v.}{preparar uma refeição | cozinhar}
  \end{phonetics}
\end{entry}

\begin{entry}{做到}{11,8}[Radicais ⼈、⼑]
  \begin{phonetics}{做到}{zuo4 dao4}[][HSK 2]
    \definition{v.}{realizar | alcançar}
  \end{phonetics}
\end{entry}

\begin{entry}{做法}{11,8}[Radicais ⼈、⽔]
  \begin{phonetics}{做法}{zuo4fa3}[][HSK 2]
    \definition[个]{s.}{método para fazer | prática | receita | maneira de lidar com algo | método de trabalho}
  \end{phonetics}
\end{entry}

\begin{entry}{做客}{11,9}[Radicais ⼈、⼧]
  \begin{phonetics}{做客}{zuo4 ke4}[][HSK 3]
    \definition{v.}{visitar; ser um convidado; visitar outras pessoas e ser você mesmo o convidado}
  \end{phonetics}
\end{entry}

\begin{entry}{做活}{11,9}[Radicais ⼈、⽔]
  \begin{phonetics}{做活}{zuo4huo2}
    \definition{v.}{trabalhar para ganhar a vida (especialmente de mulher costureira)}
  \end{phonetics}
\end{entry}

\begin{entry}{做眼}{11,11}[Radicais ⼈、⽬]
  \begin{phonetics}{做眼}{zuo4yan3}
    \definition{v.}{agir como um guia | trabalhar como espião}
  \end{phonetics}
\end{entry}

\begin{entry}{停}{11}[Radical ⼈]
  \begin{phonetics}{停}{ting2}[][HSK 2]
    \definition{v.}{parar | estacionar (um carro)}
  \end{phonetics}
\end{entry}

\begin{entry}{停工}{11,3}[Radicais ⼈、⼯]
  \begin{phonetics}{停工}{ting2gong1}
    \definition{v.}{parar de trabalhar | parar a produção}
  \end{phonetics}
\end{entry}

\begin{entry}{停办}{11,4}[Radicais ⼈、⼒]
  \begin{phonetics}{停办}{ting2ban4}
    \definition{v.}{cancelar | sair do negócio | desligar | terminar}
  \end{phonetics}
\end{entry}

\begin{entry}{停止}{11,4}[Radicais ⼈、⽌]
  \begin{phonetics}{停止}{ting2 zhi3}[][HSK 3]
    \definition{v.}{parar; suspender; cessar; cancelar}
  \end{phonetics}
\end{entry}

\begin{entry}{停火}{11,4}[Radicais ⼈、⽕]
  \begin{phonetics}{停火}{ting2huo3}
    \definition{s.}{cessar-fogo}
    \definition{v.+compl.}{cessar fogo}
  \end{phonetics}
\end{entry}

\begin{entry}{停车}{11,4}[Radicais ⼈、⾞]
  \begin{phonetics}{停车}{ting2 che1}[][HSK 2]
    \definition{v.}{parar de trabalhar (uma máquina) | estacionar | parar (um veículo) | paralisar}
  \end{phonetics}
\end{entry}

\begin{entry}{停车场}{11,4,6}[Radicais ⼈、⾞、⼟]
  \begin{phonetics}{停车场}{ting2 che1 chang3}[][HSK 2]
    \definition{s.}{parque de estacionamento}
  \end{phonetics}
\end{entry}

\begin{entry}{停业}{11,5}[Radicais ⼈、⼀]
  \begin{phonetics}{停业}{ting2ye4}
    \definition{v.}{cessar a negociação (temporária ou permanentemente) | fechar}
  \end{phonetics}
\end{entry}

\begin{entry}{停用}{11,5}[Radicais ⼈、⽤]
  \begin{phonetics}{停用}{ting2yong4}
    \definition{v.}{desabilitar | descontinuar | parar de usar | suspender}
  \end{phonetics}
\end{entry}

\begin{entry}{停电}{11,5}[Radicais ⼈、⽥]
  \begin{phonetics}{停电}{ting2dian4}
    \definition{s.}{corte de energia}
    \definition{v.}{ter uma falha de energia}
  \end{phonetics}
\end{entry}

\begin{entry}{停当}{11,6}[Radicais ⼈、⼹]
  \begin{phonetics}{停当}{ting2dang5}
    \definition{adj.}{realizado | preparado | assentado}
  \end{phonetics}
\end{entry}

\begin{entry}{停息}{11,10}[Radicais ⼈、⼼]
  \begin{phonetics}{停息}{ting2xi1}
    \definition{v.}{cessar | parar}
  \end{phonetics}
\end{entry}

\begin{entry}{停留}{11,10}[Radicais ⼈、⽥]
  \begin{phonetics}{停留}{ting2liu2}
    \definition{v.}{ficar em algum lugar temporariamente | demorar | permanecer}
  \end{phonetics}
\end{entry}

\begin{entry}{停课}{11,10}[Radicais ⼈、⾔]
  \begin{phonetics}{停课}{ting2ke4}
    \definition{v.}{fechar (escola) | parar as aulas}
  \end{phonetics}
\end{entry}

\begin{entry}{停歇}{11,13}[Radicais ⼈、⽋]
  \begin{phonetics}{停歇}{ting2xie1}
    \definition{v.}{parar para descansar}
  \end{phonetics}
\end{entry}

\begin{entry}{偶然}{11,12}[Radicais ⼈、⽕]
  \begin{phonetics}{偶然}{ou3ran2}
    \definition{adv.}{por acaso | fortuitamente}
  \end{phonetics}
\end{entry}

\begin{entry}{偷}{11}[Radical ⼈]
  \begin{phonetics}{偷}{tou1}
    \definition{adv.}{furtivamente}
    \definition{v.}{furtar | roubar}
  \end{phonetics}
\end{entry}

\begin{entry}{偷安}{11,6}[Radicais ⼈、⼧]
  \begin{phonetics}{偷安}{tou1'an1}
    \definition{v.}{buscar facilidade temporária}
  \end{phonetics}
\end{entry}

\begin{entry}{偷听}{11,7}[Radicais ⼈、⼝]
  \begin{phonetics}{偷听}{tou1ting1}
    \definition{v.}{bisbilhotar; monitorar (secretamente)}
  \end{phonetics}
\end{entry}

\begin{entry}{偷窃}{11,9}[Radicais ⼈、⽳]
  \begin{phonetics}{偷窃}{tou1qie4}
    \definition{v.}{furtar | roubar}
  \end{phonetics}
\end{entry}

\begin{entry}{偷情}{11,11}[Radicais ⼈、⼼]
  \begin{phonetics}{偷情}{tou1qing2}
    \definition{v.}{manter um caso de amor clandestino}
  \end{phonetics}
\end{entry}

\begin{entry}{偷袭}{11,11}[Radicais ⼈、⾐]
  \begin{phonetics}{偷袭}{tou1xi2}
    \definition{s.}{ataque surpresa}
    \definition{v.}{montar um ataque furtivo | invadir}
  \end{phonetics}
\end{entry}

\begin{entry}{偷渡}{11,12}[Radicais ⼈、⽔]
  \begin{phonetics}{偷渡}{tou1du4}
    \definition{s.}{contrabando | imigração ilegal | clandestino (em um navio)}
    \definition{v.}{executar um bloqueio | roubar através da fronteira internacional}
  \end{phonetics}
\end{entry}

\begin{entry}{偷税}{11,12}[Radicais ⼈、⽲]
  \begin{phonetics}{偷税}{tou1shui4}
    \definition{s.}{evasão fiscal}
  \end{phonetics}
\end{entry}

\begin{entry}{偸}{11}[Radical ⼈]
  \begin{phonetics}{偸}{tou1}
    \variantof{偷}
  \end{phonetics}
\end{entry}

\begin{entry}{副}{11}[Radical ⼑]
  \begin{phonetics}{副}{fu4}
    \definition{clas.}{para pares, conjuntos de coisas e expressões faciais | para óculos, luvas, etc.}
  \end{phonetics}
\end{entry}

\begin{entry}{唱}{11}[Radical ⼝]
  \begin{phonetics}{唱}{chang4}[][HSK 1]
    \definition{v.}{cantar}
  \end{phonetics}
\end{entry}

\begin{entry}{唱片}{11,4}[Radicais ⼝、⽚]
  \begin{phonetics}{唱片}{chang4 pian4}[][HSK 4]
    \definition[枚,张]{s.}{disco; disco feito de goma-laca, plástico, etc. com ranhuras em espiral na superfície para registrar alterações no som que podem reproduzir o som gravado em um fonógrafo}
  \end{phonetics}
\end{entry}

\begin{entry}{唱歌}{11,14}[Radicais ⼝、⽋]
  \begin{phonetics}{唱歌}{chang4 ge1}[][HSK 1]
    \definition{v.+compl.}{cantar}
  \end{phonetics}
\end{entry}

\begin{entry}{唾骂}{11,9}[Radicais ⼝、⾺]
  \begin{phonetics}{唾骂}{tuo4ma4}
    \definition{v.}{insultar | amaldiçoar}
  \end{phonetics}
\end{entry}

\begin{entry}{商人}{11,2}[Radicais ⼝、⼈]
  \begin{phonetics}{商人}{shang1 ren2}[][HSK 2]
    \definition[位,名]{s.}{comerciante | mercador | homem de negócios}
  \end{phonetics}
\end{entry}

\begin{entry}{商业}{11,5}[Radicais ⼝、⼀]
  \begin{phonetics}{商业}{shang1ye4}[][HSK 3]
    \definition[个]{s.}{barganha; negócio; comércio}
  \end{phonetics}
\end{entry}

\begin{entry}{商场}{11,6}[Radicais ⼝、⼟]
  \begin{phonetics}{商场}{shang1chang3}[][HSK 1]
    \definition[家]{s.}{mercado | shopping | loja de departamentos | o mundo dos negócios}
  \end{phonetics}
\end{entry}

\begin{entry}{商店}{11,8}[Radicais ⼝、⼴]
  \begin{phonetics}{商店}{shang1dian4}[][HSK 1]
    \definition[家,个]{s.}{loja}
  \end{phonetics}
\end{entry}

\begin{entry}{商品}{11,9}[Radicais ⼝、⼝]
  \begin{phonetics}{商品}{shang1pin3}[][HSK 3]
    \definition[种,个,件,批]{s.}{bens; mercadoria; \emph{merchandising}}
  \end{phonetics}
\end{entry}

\begin{entry}{商贸}{11,9}[Radicais ⼝、⾙]
  \begin{phonetics}{商贸}{shang1mao4}
    \definition{s.}{comércio}
  \end{phonetics}
\end{entry}

\begin{entry}{商量}{11,12}[Radicais ⼝、⾥]
  \begin{phonetics}{商量}{shang1liang5}[][HSK 2]
    \definition{v.}{consultar | discutir | falar sobre}
  \end{phonetics}
\end{entry}

\begin{entry}{啤酒}{11,10}[Radicais ⼝、⾣]
  \begin{phonetics}{啤酒}{pi2jiu3}[][HSK 3]
    \definition[杯,瓶,罐,桶,缸]{s.}{(empréstimo linguístico) cerveja}
  \end{phonetics}
\end{entry}

\begin{entry}{啤酒馆}{11,10,11}[Radicais ⼝、⾣、⾷]
  \begin{phonetics}{啤酒馆}{pi2jiu3guan3}
    \definition{s.}{cervejaria}
  \end{phonetics}
\end{entry}

\begin{entry}{啥}{11}[Radical ⼝]
  \begin{phonetics}{啥}{sha2}
    \definition{adv.}{Equivalente a 什么 (dialeto)}
  \end{phonetics}
\end{entry}

\begin{entry}{啵}{11}[Radical ⼝]
  \begin{phonetics}{啵}{bo1}
    \definition{s.}{(onomatopéia) borbulhar}
  \end{phonetics}
  \begin{phonetics}{啵}{bo5}
    \definition{part.}{partícula gramaticalmente equivalente a 吧}
  \seealsoref{吧}{ba5}
  \end{phonetics}
\end{entry}

\begin{entry}{圈粉}{11,10}[Radicais ⼞、⽶]
  \begin{phonetics}{圈粉}{quan1fen3}
    \definition{s.}{(neologismo, coloquial) ganhar alguém como fã, obter novos fãs}
  \end{phonetics}
\end{entry}

\begin{entry}{埦}{11}[Radical ⼟]
  \begin{phonetics}{埦}{wan3}
    \variantof{碗}
  \end{phonetics}
\end{entry}

\begin{entry}{基本}{11,5}[Radicais ⼟、⽊]
  \begin{phonetics}{基本}{ji1ben3}[][HSK 3]
    \definition{adj.}{básico; fundamental; elementar | principal}
    \definition{adv.}{basicamente; em geral}
    \definition{s.}{fundação}
  \end{phonetics}
\end{entry}

\begin{entry}{基本上}{11,5,3}[Radicais ⼟、⽊、⼀]
  \begin{phonetics}{基本上}{ji1 ben3 shang4}[][HSK 3]
    \definition{adv.}{basicamente; no principal | em geral}
  \end{phonetics}
\end{entry}

\begin{entry}{基本功}{11,5,5}[Radicais ⼟、⽊、⼒]
  \begin{phonetics}{基本功}{ji1ben3gong1}
    \definition{s.}{habilidades | fundamentos básicos}
  \end{phonetics}
\end{entry}

\begin{entry}{基本法}{11,5,8}[Radicais ⼟、⽊、⽔]
  \begin{phonetics}{基本法}{ji1ben3fa3}
    \definition{s.}{lei básica (constituição)}
  \end{phonetics}
\end{entry}

\begin{entry}{基因}{11,6}[Radicais ⼟、⼞]
  \begin{phonetics}{基因}{ji1yin1}
    \definition{s.}{gene}
  \end{phonetics}
\end{entry}

\begin{entry}{基础}{11,10}[Radicais ⼟、⽯]
  \begin{phonetics}{基础}{ji1chu3}[][HSK 3]
    \definition{adj.}{básico; fundamental}
    \definition[个]{s.}{base; fundamento; fundação}
  \end{phonetics}
\end{entry}

\begin{entry}{基督教}{11,13,11}[Radicais ⼟、⽬、⽁]
  \begin{phonetics}{基督教}{ji1du1jiao4}
    \definition*{s.}{Cristianismo | Cristão}
  \end{phonetics}
\end{entry}

\begin{entry}{堵}{11}[Radical ⼟]
  \begin{phonetics}{堵}{du3}[][HSK 4]
    \definition{adj.}{asfixiado; abafado; sufocado; oprimido}
    \definition{clas.}{para paredes}
    \definition{s.}{parede}
    \definition{s.}{sobrenome Du}
    \definition{v.}{impedir; bloquear}
  \end{phonetics}
\end{entry}

\begin{entry}{堵车}{11,4}[Radicais ⼟、⾞]
  \begin{phonetics}{堵车}{du3che1}[][HSK 4]
    \definition{v.}{congestionar (trânsito)}
    \definition{v.+compl.}{congestionamento; tráfego intenso; ficar congestionado (no tráfego); bloqueio de vias devido ao excesso de tráfego, etc.}
  \end{phonetics}
\end{entry}

\begin{entry}{够}{11}[Radical ⼣]
  \begin{phonetics}{够}{gou4}[][HSK 2]
    \definition{adj.}{suficiente}
    \definition{adv.}{(antes do adj.) realmente}
    \definition{v.}{bastar | chegar}
  \end{phonetics}
\end{entry}

\begin{entry}{够不着}{11,4,11}[Radicais ⼣、⼀、⽬]
  \begin{phonetics}{够不着}{gou4bu5zhao2}
    \definition{v.}{ser incapaz de alcançar}
  \end{phonetics}
\end{entry}

\begin{entry}{够本}{11,5}[Radicais ⼣、⽊]
  \begin{phonetics}{够本}{gou4ben3}
    \definition{v.}{empatar | fazer valer o dinheiro}
  \end{phonetics}
\end{entry}

\begin{entry}{够呛}{11,7}[Radicais ⼣、⼝]
  \begin{phonetics}{够呛}{gou4qiang4}
    \definition{adj.}{suficiente | terrível | insuportável | improvável}
  \end{phonetics}
\end{entry}

\begin{entry}{够味}{11,8}[Radicais ⼣、⼝]
  \begin{phonetics}{够味}{gou4wei4}
    \definition{adj.}{excelente | na medida}
  \end{phonetics}
\end{entry}

\begin{entry}{够戗}{11,8}[Radicais ⼣、⼽]
  \begin{phonetics}{够戗}{gou4qiang4}
    \variantof{够呛}
  \end{phonetics}
\end{entry}

\begin{entry}{够朋友}{11,8,4}[Radicais ⼣、⽉、⼜]
  \begin{phonetics}{够朋友}{gou4peng2you5}
    \definition{v.}{ser um amigo verdadeiro}
  \end{phonetics}
\end{entry}

\begin{entry}{够格}{11,10}[Radicais ⼣、⽊]
  \begin{phonetics}{够格}{gou4ge2}
    \definition{adj.}{apto | qualificado | apresentável}
  \end{phonetics}
\end{entry}

\begin{entry}{够得着}{11,11,11}[Radicais ⼣、⼻、⽬]
  \begin{phonetics}{够得着}{gou4de5zhao2}
    \definition{v.}{estar à altura | alcançar}
  \end{phonetics}
\end{entry}

\begin{entry}{婚礼}{11,5}[Radicais ⼥、⽰]
  \begin{phonetics}{婚礼}{hun1li3}
    \definition[场]{s.}{casamento | núpcias | cerimônia de casamento}
  \end{phonetics}
\end{entry}

\begin{entry}{宿舍}{11,8}[Radicais ⼧、⾆]
  \begin{phonetics}{宿舍}{su4she4}
    \definition[间]{s.}{dormitório | quarto de dormir | hostel}
  \end{phonetics}
\end{entry}

\begin{entry}{寂寞}{11,13}[Radicais ⼧、⼧]
  \begin{phonetics}{寂寞}{ji4mo4}
    \definition{adj.}{sozinho | solitário | (de um lugar) silencioso}
  \end{phonetics}
\end{entry}

\begin{entry}{寂寥}{11,14}[Radicais ⼧、⼧]
  \begin{phonetics}{寂寥}{ji4liao2}
    \definition{s.}{solidão | vasto e vazio | quieto e desolado (literário)}
  \end{phonetics}
\end{entry}

\begin{entry}{寄}{11}[Radical ⼧]
  \begin{phonetics}{寄}{ji4}
    \definition{v.}{enviar | mandar}
  \end{phonetics}
\end{entry}

\begin{entry}{寄予}{11,4}[Radicais ⼧、⼅]
  \begin{phonetics}{寄予}{ji4yu3}
    \definition{v.}{expressar | colocar (esperança, importância, etc.) em | mostrar}
  \end{phonetics}
\end{entry}

\begin{entry}{寄生}{11,5}[Radicais ⼧、⽣]
  \begin{phonetics}{寄生}{ji4sheng1}
    \definition{s.}{parasita | parasitismo}
    \definition{v.}{viver tirando vantagem dos outros | viver dentro ou sobre outro organismo como um parasita}
  \end{phonetics}
\end{entry}

\begin{entry}{寄生生活}{11,5,5,9}[Radicais ⼧、⽣、⽣、⽔]
  \begin{phonetics}{寄生生活}{ji4sheng1sheng1huo2}
    \definition{s.}{parasitismo | vida parasitária}
  \end{phonetics}
\end{entry}

\begin{entry}{寄存}{11,6}[Radicais ⼧、⼦]
  \begin{phonetics}{寄存}{ji4cun2}
    \definition{v.}{depositar | deixar algo com alguém | armazenar}
  \end{phonetics}
\end{entry}

\begin{entry}{寄托}{11,6}[Radicais ⼧、⼿]
  \begin{phonetics}{寄托}{ji4tuo1}
    \definition{v.}{investir (sua esperança, energia, etc.) em algo | confiar (a alguém) | colocar (a esperança, a energia, etc.) em}
  \end{phonetics}
\end{entry}

\begin{entry}{寄卖}{11,8}[Radicais ⼧、⼗]
  \begin{phonetics}{寄卖}{ji4mai4}
    \definition{v.}{consignar para venda}
  \end{phonetics}
\end{entry}

\begin{entry}{寄居}{11,8}[Radicais ⼧、⼫]
  \begin{phonetics}{寄居}{ji4ju1}
    \definition{s.}{morar longe de casa}
  \end{phonetics}
\end{entry}

\begin{entry}{寄放}{11,8}[Radicais ⼧、⽅]
  \begin{phonetics}{寄放}{ji4fang4}
    \definition{v.}{deixar algo com alguém}
  \end{phonetics}
\end{entry}

\begin{entry}{寄养}{11,9}[Radicais ⼧、⼋]
  \begin{phonetics}{寄养}{ji4yang3}
    \definition{v.}{embarcar | promover | colocar sob os cuidados de alguém (uma criança, animal de estimação, etc.)}
  \end{phonetics}
\end{entry}

\begin{entry}{寄送}{11,9}[Radicais ⼧、⾡]
  \begin{phonetics}{寄送}{ji4song4}
    \definition{v.}{enviar | transmitir}
  \end{phonetics}
\end{entry}

\begin{entry}{寄递}{11,10}[Radicais ⼧、⾡]
  \begin{phonetics}{寄递}{ji4di4}
    \definition{s.}{entrega de correspondência}
  \end{phonetics}
\end{entry}

\begin{entry}{寄售}{11,11}[Radicais ⼧、⼝]
  \begin{phonetics}{寄售}{ji4shou4}
    \definition{v.}{venda em consignação}
  \end{phonetics}
\end{entry}

\begin{entry}{寄宿}{11,11}[Radicais ⼧、⼧]
  \begin{phonetics}{寄宿}{ji4su4}
    \definition{s.}{embarque}
    \definition{v.}{embarcar}
  \end{phonetics}
\end{entry}

\begin{entry}{寄望}{11,11}[Radicais ⼧、⽉]
  \begin{phonetics}{寄望}{ji4wang4}
    \definition{v.}{depositar esperanças em}
  \end{phonetics}
\end{entry}

\begin{entry}{密切}{11,4}[Radicais ⼧、⼑]
  \begin{phonetics}{密切}{mi4qie4}
    \definition{adj.}{perto | familiar | íntimo}
    \definition{v.}{promover laços estreitos (relacionamento) | prestar muita atenção}
  \end{phonetics}
\end{entry}

\begin{entry}{崇}{11}[Radical ⼭]
  \begin{phonetics}{崇}{chong2}
    \definition*{s.}{sobrenome Chong}
    \definition{adj.}{alto | sublime | elevado}
    \definition{v.}{estimar | adorar}
  \end{phonetics}
\end{entry}

\begin{entry}{崖}{11}[Radical ⼭]
  \begin{phonetics}{崖}{ya2}
    \definition{s.}{precipício | penhasco}
  \end{phonetics}
\end{entry}

\begin{entry}{崩}{11}[Radical ⼭]
  \begin{phonetics}{崩}{beng1}
    \definition{s.}{morte de rei ou imperador | desaparecimento}
    \definition{v.}{entrar em colapso | cair em ruínas}
  \end{phonetics}
\end{entry}

\begin{entry}{巢}{11}[Radical ⼮]
  \begin{phonetics}{巢}{chao2}
    \definition*{s.}{sobrenome Chao}
    \definition{s.}{ninho (de aves, etc.)}
  \end{phonetics}
\end{entry}

\begin{entry}{常}{11}[Radical ⼱]
  \begin{phonetics}{常}{chang2}[][HSK 1]
    \definition*{s.}{sobrenome Chang}
    \definition{adv.}{muitas vezes | frequentemente}
  \end{phonetics}
\end{entry}

\begin{entry}{常见}{11,4}[Radicais ⼱、⾒]
  \begin{phonetics}{常见}{chang2 jian4}[][HSK 2]
    \definition{adj.}{comum}
  \end{phonetics}
\end{entry}

\begin{entry}{常用}{11,5}[Radicais ⼱、⽤]
  \begin{phonetics}{常用}{chang2 yong4}[][HSK 2]
    \definition{adj.}{em uso comum}
  \end{phonetics}
\end{entry}

\begin{entry}{常问问题}{11,6,6,15}[Radicais ⼱、⾨、⾨、⾴]
  \begin{phonetics}{常问问题}{chang2wen4wen4ti2}
    \definition{s.}{FAQ; perguntas frequentes}
  \end{phonetics}
\end{entry}

\begin{entry}{常识}{11,7}[Radicais ⼱、⾔]
  \begin{phonetics}{常识}{chang2shi2}[][HSK 4]
    \definition{s.}{senso comum; conhecimento geral; conhecimento que uma pessoa comum deve ter}
  \end{phonetics}
\end{entry}

\begin{entry}{常常}{11,11}[Radicais ⼱、⼱]
  \begin{phonetics}{常常}{chang2 chang2}[][HSK 1]
    \definition{adv.}{frequentemente | com frequência}
  \end{phonetics}
\end{entry}

\begin{entry}{彩色}{11,6}[Radicais ⼺、⾊]
  \begin{phonetics}{彩色}{cai3 se4}[][HSK 3]
    \definition{s.}{multicolorido; cor}
  \end{phonetics}
\end{entry}

\begin{entry}{彩虹}{11,9}[Radicais ⼺、⾍]
  \begin{phonetics}{彩虹}{cai3hong2}
    \definition[道]{s.}{arco-íris}
  \end{phonetics}
\end{entry}

\begin{entry}{得}{11}[Radical ⼻]
  \begin{phonetics}{得}{de2}[][HSK 2]
    \definition{adj.}{adequado; apropriado | satisfeito; complacente; orgulhoso de si mesmo}
    \definition{interj.}{usado para encerrar uma conversa para indicar concordância ou proibição | usado quando a situação não é a esperada, para expressar impotência}
    \definition{v.}{obter (em vez de ``perder''); conseguir; ganhar | (de um cálculo) igual; resultar em; efetuar cálculos para produzir resultados | estar terminado; estar pronto; cumprir | contrair uma doença}
    \definition{v.aux.}{usado antes de outros verbos para expressar permissão | usado antes de outros verbos para indicar que é possível (usado principalmente na forma negativa) | usado em conversas para indicar que não há necessidade de dizer mais nada}
  \end{phonetics}
  \begin{phonetics}{得}{de5}[][HSK 2]
    \definition{part.}{depois de um verbo ou adjetivo para expressar possibilidade ou capacidade | entre um verbo e seu complemento para expressar possibilidade | ligando um verbo ou um adjetivo a um complemento que descreve a maneira ou o grau}
  \end{phonetics}
  \begin{phonetics}{得}{dei3}[][HSK 4]
    \definition{v.}{precisar; dever; expressa uma necessidade racional, factual ou subjetiva | dever; ter que; necessidade expressa, volitiva ou factual}
  \end{phonetics}
\end{entry}

\begin{entry}{得了}{11,2}[Radicais ⼻、⼅]
  \begin{phonetics}{得了}{de2le5}
    \definition{expr.}{Tudo bem!; É o bastante!}
  \end{phonetics}
  \begin{phonetics}{得了}{de2liao3}
    \definition{adj.}{(enfaticamente, em perguntas retóricas) possível}
  \end{phonetics}
\end{entry}

\begin{entry}{得分}{11,4}[Radicais ⼻、⼑]
  \begin{phonetics}{得分}{de2 fen1}[][HSK 3]
    \definition{v.}{fazer pontos; pontuar}
  \end{phonetics}
\end{entry}

\begin{entry}{得出}{11,5}[Radicais ⼻、⼐]
  \begin{phonetics}{得出}{de2 chu1}[][HSK 2]
    \definition{v.}{chegar (a uma conclusão) | obter (a um resultado)}
  \end{phonetics}
\end{entry}

\begin{entry}{得到}{11,8}[Radicais ⼻、⼑]
  \begin{phonetics}{得到}{de2 dao4}[][HSK 1]
    \definition{v.}{obter | receber}
  \end{phonetics}
\end{entry}

\begin{entry}{得意}{11,13}[Radicais ⼻、⼼]
  \begin{phonetics}{得意}{de2yi4}[][HSK 4]
    \definition{adj.}{complacente; orgulhoso de si mesmo; satisfeito consigo mesmo}
    \definition{v.+compl.}{orgulhar-se de si mesmo; ter satisfação consigo mesmo; ser complacente}
  \end{phonetics}
\end{entry}

\begin{entry}{悉心}{11,4}[Radicais ⼼、⼼]
  \begin{phonetics}{悉心}{xi1xin1}
    \definition{adv.}{colocar o coração (e a alma) em algo | com muito cuidado}
  \end{phonetics}
\end{entry}

\begin{entry}{悉尼}{11,5}[Radicais ⼼、⼫]
  \begin{phonetics}{悉尼}{xi1ni2}
    \definition*{s.}{Sidney}
  \end{phonetics}
\end{entry}

\begin{entry}{悉数}{11,13}[Radicais ⼼、⽁]
  \begin{phonetics}{悉数}{xi1shu3}
    \definition{adv.}{enumerar em detalhes | explicar claramente}
  \end{phonetics}
  \begin{phonetics}{悉数}{xi1shu4}
    \definition{adv.}{todos | cada um | toda a soma}
  \end{phonetics}
\end{entry}

\begin{entry}{您}{11}[Radical ⼼]
  \begin{phonetics}{您}{nin2}[][HSK 1]
    \definition{pron.}{você (formal) | tu | te | ti | contigo}
    \seeref{你}{ni3}
  \end{phonetics}
\end{entry}

\begin{entry}{悬挂}{11,9}[Radicais ⼼、⼿]
  \begin{phonetics}{悬挂}{xuan2gua4}
    \definition{s.}{(veículo) suspensão}
    \definition{v.}{suspender}
  \end{phonetics}
\end{entry}

\begin{entry}{悬崖}{11,11}[Radicais ⼼、⼭]
  \begin{phonetics}{悬崖}{xuan2ya2}
    \definition{s.}{precipício | penhasco}
  \end{phonetics}
\end{entry}

\begin{entry}{情况}{11,7}[Radicais ⼼、⼎]
  \begin{phonetics}{情况}{qing2kuang4}[][HSK 3]
    \definition[种,个,些]{s.}{condição; situação; circunstâncias; estado das coisas | mudança notável}
  \end{phonetics}
\end{entry}

\begin{entry}{情绪}{11,11}[Radicais ⼼、⽷]
  \begin{phonetics}{情绪}{qing2xu4}
    \definition[种]{s.}{humor | estado da mente | mau humor}
  \end{phonetics}
\end{entry}

\begin{entry}{情感}{11,13}[Radicais ⼼、⼼]
  \begin{phonetics}{情感}{qing2 gan3}[][HSK 3]
    \definition[份]{s.}{emoção; sentimento | afeição; apego}
    \definition{v.}{mover-se (emocionalmente)}
  \end{phonetics}
\end{entry}

\begin{entry}{惊呆}{11,7}[Radicais ⼼、⼝]
  \begin{phonetics}{惊呆}{jing1dai1}
    \definition{adj.}{estupefato | chocado}
  \end{phonetics}
\end{entry}

\begin{entry}{惊喜}{11,12}[Radicais ⼼、⼝]
  \begin{phonetics}{惊喜}{jing1xi3}
    \definition{s.}{boa surpresa}
    \definition{v.}{ser agradavelmente surpreendido}
  \end{phonetics}
\end{entry}

\begin{entry}{惨}{11}[Radical ⽕]
  \begin{phonetics}{惨}{can3}
    \definition{adj.}{miserável | cruel | desumano | desastroso | trágico | sombrio}
  \end{phonetics}
\end{entry}

\begin{entry}{据说}{11,9}[Radicais ⼿、⾔]
  \begin{phonetics}{据说}{ju4shuo1}[][HSK 3]
    \definition{v.}{ser dito; ser relatado}
  \end{phonetics}
\end{entry}

\begin{entry}{捷径}{11,8}[Radicais ⼿、⼻]
  \begin{phonetics}{捷径}{jie2jing4}
    \definition{s.}{atalho}
  \end{phonetics}
\end{entry}

\begin{entry}{掉}{11}[Radical ⼿]
  \begin{phonetics}{掉}{diao4}[][HSK 2]
    \definition{v.}{cair | deixar cair}
  \end{phonetics}
\end{entry}

\begin{entry}{掉队}{11,4}[Radicais ⼿、⾩]
  \begin{phonetics}{掉队}{diao4dui4}
    \definition{v.}{abandonar | ficar para trás}
  \end{phonetics}
\end{entry}

\begin{entry}{掉包}{11,5}[Radicais ⼿、⼓]
  \begin{phonetics}{掉包}{diao4bao1}
    \definition{v.}{vender uma falsificação pelo artigo genuíno | roubar o item valioso de alguém e substituí-lo por um item de aparência semelhante, mas sem valor}
  \end{phonetics}
\end{entry}

\begin{entry}{掉线}{11,8}[Radicais ⼿、⽷]
  \begin{phonetics}{掉线}{diao4xian4}
    \definition{v.}{desconectar-se (da \emph{Internet})}
  \end{phonetics}
\end{entry}

\begin{entry}{掉转}{11,8}[Radicais ⼿、⾞]
  \begin{phonetics}{掉转}{diao4zhuan3}
    \definition{v.}{dar a volta}
  \end{phonetics}
\end{entry}

\begin{entry}{掉落}{11,12}[Radicais ⼿、⾋]
  \begin{phonetics}{掉落}{diao4luo4}
    \definition{v.}{derrubar}
  \end{phonetics}
\end{entry}

\begin{entry}{掉膘}{11,15}[Radicais ⼿、⾁]
  \begin{phonetics}{掉膘}{diao4biao1}
    \definition{v.}{perder peso (gado)}
  \end{phonetics}
\end{entry}

\begin{entry}{排}{11}[Radical ⼿]
  \begin{phonetics}{排}{pai2}[][HSK 2,3]
    \definition{clas.}{para linhas}
    \definition{s.}{linha | pelotão | jangada; balsa | torta}
    \definition{v.}{organizar; colocar em ordem | ensaiar | excluir; ejetar; descarregar | empurrar}
  \end{phonetics}
\end{entry}

\begin{entry}{排水}{11,4}[Radicais ⼿、⽔]
  \begin{phonetics}{排水}{pai2shui3}
    \definition{v.}{drenar}
  \end{phonetics}
\end{entry}

\begin{entry}{排队}{11,4}[Radicais ⼿、⾩]
  \begin{phonetics}{排队}{pai2dui4}[][HSK 2]
    \definition{v.+compl.}{formar uma fila | alinhar | listar | classificar}
  \end{phonetics}
\end{entry}

\begin{entry}{排名}{11,6}[Radicais ⼿、⼝]
  \begin{phonetics}{排名}{pai2 ming2}[][HSK 3]
    \definition{s.}{classificação; resultado}
  \end{phonetics}
\end{entry}

\begin{entry}{排球}{11,11}[Radicais ⼿、⽟]
  \begin{phonetics}{排球}{pai2 qiu2}[][HSK 2]
    \definition[个]{s.}{voleibol}
  \end{phonetics}
\end{entry}

\begin{entry}{探亲}{11,9}[Radicais ⼿、⼇]
  \begin{phonetics}{探亲}{tan4qin1}
    \definition{v.+compl.}{ir para casa para visitar a família}
  \end{phonetics}
\end{entry}

\begin{entry}{接}{11}[Radical ⼿]
  \begin{phonetics}{接}{jie1}[][HSK 2]
    \definition{v.}{ir buscar (alguém) |  ir ao encontro de (alguém) | receber}
  \end{phonetics}
\end{entry}

\begin{entry}{接下来}{11,3,7}[Radicais ⼿、⼀、⽊]
  \begin{phonetics}{接下来}{jie1 xia4 lai2}[][HSK 2]
    \definition{expr.}{próximo | seguinte | aceitar}
  \end{phonetics}
\end{entry}

\begin{entry}{接(电话)}{11,5,8}[Radicais ⼿、⽥、⾔]
  \begin{phonetics}{接(电话)}{jie1(dian4hua4)}
    \definition{v.}{atender (o telefone) | receber (uma ligação telefônica)}
  \end{phonetics}
\end{entry}

\begin{entry}{接近}{11,7}[Radicais ⼿、⾡]
  \begin{phonetics}{接近}{jie1jin4}[][HSK 3]
    \definition{adj.}{perto; próximo}
    \definition{v.}{estar perto de; aproximar; aproximar-se}
  \end{phonetics}
\end{entry}

\begin{entry}{接到}{11,8}[Radicais ⼿、⼑]
  \begin{phonetics}{接到}{jie1 dao4}[][HSK 2]
    \definition{v.}{receber (carta, etc.)}
  \end{phonetics}
\end{entry}

\begin{entry}{接受}{11,8}[Radicais ⼿、⼜]
  \begin{phonetics}{接受}{jie1shou4}[][HSK 2]
    \definition{v.}{aceitar | concordar}
  \end{phonetics}
\end{entry}

\begin{entry}{接待}{11,9}[Radicais ⼿、⼻]
  \begin{phonetics}{接待}{jie1dai4}[][HSK 3]
    \definition{v.}{receber (alguém); acolher; recepcionar}
  \end{phonetics}
\end{entry}

\begin{entry}{接班人}{11,10,2}[Radicais ⼿、⽟、⼈]
  \begin{phonetics}{接班人}{jie1ban1ren2}
    \definition{s.}{sucessor}
  \end{phonetics}
\end{entry}

\begin{entry}{接着}{11,11}[Radicais ⼿、⽬]
  \begin{phonetics}{接着}{jie1zhe5}[][HSK 2]
    \definition{adv.}{por sua vez | com um seguindo o outro}
    \definition{v.}{seguir | prosseguir | continuar | prosseguir | pegar}
  \end{phonetics}
\end{entry}

\begin{entry}{控制}{11,8}[Radicais ⼿、⼑]
  \begin{phonetics}{控制}{kong4zhi4}
    \definition{v.}{controlar}
  \end{phonetics}
\end{entry}

\begin{entry}{推}{11}[Radical ⼿]
  \begin{phonetics}{推}{tui1}[][HSK 2]
    \definition{v.}{empurrar | girar um moinho ou uma pedra de amolar | moer | impulsionar | promover | avançar | estender | deduzir | inferir | declinar | empurrar para longe | deslocar | adiar | diferir | eleger | selecionar | escolher | ter em alta estima | elogiar muito}
  \end{phonetics}
\end{entry}

\begin{entry}{推广}{11,3}[Radicais ⼿、⼴]
  \begin{phonetics}{推广}{tui1guang3}[][HSK 3]
    \definition{s.}{extensão}
    \definition{v.}{espalhar; estender; promover; popularizar}
  \end{phonetics}
\end{entry}

\begin{entry}{推介}{11,4}[Radicais ⼿、⼈]
  \begin{phonetics}{推介}{tui1jie4}
    \definition{s.}{promoção}
    \definition{v.}{promover | introduzir e recomendar}
  \end{phonetics}
\end{entry}

\begin{entry}{推开}{11,4}[Radicais ⼿、⼶]
  \begin{phonetics}{推开}{tui1 kai1}[][HSK 3]
    \definition{v.}{declinar; rejeitar | empurrar para longe | empurrar para abrir (um portão, etc.) | estender; popularizar}
  \end{phonetics}
\end{entry}

\begin{entry}{推动}{11,6}[Radicais ⼿、⼒]
  \begin{phonetics}{推动}{tui1 dong4}[][HSK 3]
    \definition{v.}{promover; atuar; impulsionar; empurrar para a frente; dar ímpeto a}
  \end{phonetics}
\end{entry}

\begin{entry}{推进}{11,7}[Radicais ⼿、⾡]
  \begin{phonetics}{推进}{tui1 jin4}[][HSK 3]
    \definition{v.}{avançar; empurrar; levar adiante; dar ímpeto a | empurrar; dirigir; avançar; seguir em frente; seguir em frente; pressionar para frente}
  \end{phonetics}
\end{entry}

\begin{entry}{推迟}{11,7}[Radicais ⼿、⾡]
  \begin{phonetics}{推迟}{tui1chi2}
    \definition{v.}{adiar | deixar para mais tarde | tardar}
  \end{phonetics}
\end{entry}

\begin{entry}{措施}{11,9}[Radicais ⼿、⽅]
  \begin{phonetics}{措施}{cuo4shi1}[][HSK 4]
    \definition{s.}{medida; etapa; passo; abordagem adotada para lidar com as coisas}
  \end{phonetics}
\end{entry}

\begin{entry}{敎}{11}[Radical ⽁]
  \begin{phonetics}{敎}{jiao4}
    \variantof{教}
  \end{phonetics}
\end{entry}

\begin{entry}{救}{11}[Radical ⽁]
  \begin{phonetics}{救}{jiu4}[][HSK 3]
    \definition*{s.}{sobrenome Jiu}
    \definition{v.}{resgatar; salvar | ajudar; aliviar; socorrer}
  \end{phonetics}
\end{entry}

\begin{entry}{救出}{11,5}[Radicais ⽁、⼐]
  \begin{phonetics}{救出}{jiu4chu1}
    \definition{v.}{resgatar | tirar do perigo}
  \end{phonetics}
\end{entry}

\begin{entry}{救护车}{11,7,4}[Radicais ⽁、⼿、⾞]
  \begin{phonetics}{救护车}{jiu4hu4che1}
    \definition[辆]{s.}{ambulância}
  \end{phonetics}
\end{entry}

\begin{entry}{救命}{11,8}[Radicais ⽁、⼝]
  \begin{phonetics}{救命}{jiu4ming4}
    \definition{interj.}{Socorro! | Salve-me!}
    \definition{v.+compl.}{salvar a vida de alguém}
  \end{phonetics}
\end{entry}

\begin{entry}{教}{11}[Radical ⽁]
  \begin{phonetics}{教}{jiao1}[][HSK 1]
    \definition{v.}{ensinar | lecionar}
  \end{phonetics}
  \begin{phonetics}{教}{jiao4}
    \definition*{s.}{sobrenome Jiao}
    \definition{s.}{religião | ensinamento}
    \definition{v.}{causar | como fazer algo | contar (explicar como fazer algo)}
  \end{phonetics}
\end{entry}

\begin{entry}{教长}{11,4}[Radicais ⽁、⾧]
  \begin{phonetics}{教长}{jiao4zhang3}
    \definition{s.}{imã (Islã) | mulá}
  \end{phonetics}
\end{entry}

\begin{entry}{教会}{11,6}[Radicais ⽁、⼈]
  \begin{phonetics}{教会}{jiao1hui4}
    \definition{v.}{mostrar | ensinar}
  \end{phonetics}
  \begin{phonetics}{教会}{jiao4hui4}
    \definition{s.}{igreja cristã}
  \end{phonetics}
\end{entry}

\begin{entry}{教导}{11,6}[Radicais ⽁、⼨]
  \begin{phonetics}{教导}{jiao4dao3}
    \definition{s.}{instrução | orientação | ensino}
    \definition{v.}{instruir | orientar | ensinar}
  \end{phonetics}
\end{entry}

\begin{entry}{教师}{11,6}[Radicais ⽁、⼱]
  \begin{phonetics}{教师}{jiao4 shi1}[][HSK 2]
    \definition[个]{s.}{professor | mestre}
  \end{phonetics}
\end{entry}

\begin{entry}{教材}{11,7}[Radicais ⽁、⽊]
  \begin{phonetics}{教材}{jiao4cai2}[][HSK 3]
    \definition[本,套]{s.}{livro didático; material didático}
  \end{phonetics}
\end{entry}

\begin{entry}{教学}{11,8}[Radicais ⽁、⼦]
  \begin{phonetics}{教学}{jiao1xue2}
    \definition{v.}{ensinar (como um professor)}
  \end{phonetics}
  \begin{phonetics}{教学}{jiao4 xue2}[][HSK 2]
    \definition[次]{s.}{ensino | instrução}
  \end{phonetics}
\end{entry}

\begin{entry}{教学楼}{11,8,13}[Radicais ⽁、⼦、⽊]
  \begin{phonetics}{教学楼}{jiao4 xue2 lou2}[][HSK 1]
    \definition{s.}{edifício de salas de aula}
  \end{phonetics}
\end{entry}

\begin{entry}{教官}{11,8}[Radicais ⽁、⼧]
  \begin{phonetics}{教官}{jiao4guan1}
    \definition{s.}{instrutor militar}
  \end{phonetics}
\end{entry}

\begin{entry}{教练}{11,8}[Radicais ⽁、⽷]
  \begin{phonetics}{教练}{jiao4lian4}[][HSK 3]
    \definition[个,位,名]{s.}{instrutor; treinador (esportes)}
    \definition{v.}{treinar}
  \end{phonetics}
\end{entry}

\begin{entry}{教育}{11,8}[Radicais ⽁、⾁]
  \begin{phonetics}{教育}{jiao4yu4}[][HSK 2]
    \definition{s.}{educação}
    \definition{v.}{ensinar | educar}
  \end{phonetics}
\end{entry}

\begin{entry}{教室}{11,9}[Radicais ⽁、⼧]
  \begin{phonetics}{教室}{jiao4shi4}[][HSK 2]
    \definition[间]{s.}{sala de aula}
  \end{phonetics}
\end{entry}

\begin{entry}{教堂}{11,11}[Radicais ⽁、⼟]
  \begin{phonetics}{教堂}{jiao4tang2}
    \definition[间]{s.}{igreja | capela}
  \end{phonetics}
\end{entry}

\begin{entry}{教授}{11,11}[Radicais ⽁、⼿]
  \begin{phonetics}{教授}{jiao4shou4}
    \definition[个,位]{s.}{professor (universitário)}
    \definition{v.}{instruir | palestrar sobre}
  \end{phonetics}
\end{entry}

\begin{entry}{敢}{11}[Radical ⽁]
  \begin{phonetics}{敢}{gan3}[][HSK 3]
    \definition{adj.}{ousado; corajoso; bravo}
    \definition{adv.}{talvez; provavelmente}
    \definition{v.}{ousar; aventurar-se | ter a confiança de; ter certeza | tornar ousado; aventurar-se}
  \end{phonetics}
\end{entry}

\begin{entry}{敢情}{11,11}[Radicais ⽁、⼼]
  \begin{phonetics}{敢情}{gan3qing5}
    \definition{adv.}{claro | acontece que\dots}
  \end{phonetics}
\end{entry}

\begin{entry}{斜阳}{11,6}[Radicais ⽃、⾩]
  \begin{phonetics}{斜阳}{xie2yang2}
    \definition{s.}{sol poente}
  \end{phonetics}
\end{entry}

\begin{entry}{断}{11}[Radical ⽄]
  \begin{phonetics}{断}{duan4}[][HSK 3]
    \definition*{s.}{sobrenome Duan}
    \definition{adv.}{absolutamente; decididamente}
    \definition{v.}{quebrar; estalar | interromper; cortar; parar | desistir; abster-se de | julgar; decidir}
  \end{phonetics}
\end{entry}

\begin{entry}{断交}{11,6}[Radicais ⽄、⼇]
  \begin{phonetics}{断交}{duan4jiao1}
    \definition{v.+compl.}{terminar uma amizade | romper relações diplomáticas}
  \end{phonetics}
\end{entry}

\begin{entry}{旋转}{11,8}[Radicais ⽅、⾞]
  \begin{phonetics}{旋转}{xuan2zhuan3}
    \definition{v.}{girar}
  \end{phonetics}
\end{entry}

\begin{entry}{族}{11}[Radical ⽅]
  \begin{phonetics}{族}{zu2}
    \definition{s.}{raça | nacionalidade | etnia | clã | por extensão, grupo social}
  \end{phonetics}
\end{entry}

\begin{entry}{旣}{11}[Radical ⽆]
  \begin{phonetics}{旣}{ji4}
    \variantof{既}
  \end{phonetics}
\end{entry}

\begin{entry}{晚}{11}[Radical ⽇]
  \begin{phonetics}{晚}{wan3}[][HSK 1]
    \definition{adj.}{tarde | noite}
  \end{phonetics}
\end{entry}

\begin{entry}{晚上}{11,3}[Radicais ⽇、⼀]
  \begin{phonetics}{晚上}{wan3shang5}[][HSK 1]
    \definition{adv.}{noite | à noite}
  \end{phonetics}
\end{entry}

\begin{entry}{晚会}{11,6}[Radicais ⽇、⼈]
  \begin{phonetics}{晚会}{wan3hui4}[][HSK 2]
    \definition[个]{s.}{festa noturna}
  \end{phonetics}
\end{entry}

\begin{entry}{晚安}{11,6}[Radicais ⽇、⼧]
  \begin{phonetics}{晚安}{wan3'an1}[][HSK 2]
    \definition{v.}{boa noite}
  \end{phonetics}
\end{entry}

\begin{entry}{晚报}{11,7}[Radicais ⽇、⼿]
  \begin{phonetics}{晚报}{wan3 bao4}[][HSK 2]
    \definition{s.}{jornal da noite}
  \end{phonetics}
\end{entry}

\begin{entry}{晚近}{11,7}[Radicais ⽇、⾡]
  \begin{phonetics}{晚近}{wan3jin4}
    \definition{adj.}{recente | mais recente no passado}
    \definition{adv.}{ultimamente | recentemente}
  \end{phonetics}
\end{entry}

\begin{entry}{晚饭}{11,7}[Radicais ⽇、⾷]
  \begin{phonetics}{晚饭}{wan3fan4}[][HSK 1]
    \definition[份,顿,次,餐]{s.}{jantar}
  \end{phonetics}
\end{entry}

\begin{entry}{晚育}{11,8}[Radicais ⽇、⾁]
  \begin{phonetics}{晚育}{wan3yu4}
    \definition{s.}{parto tardio}
    \definition{v.}{ter um filho mais tarde}
  \end{phonetics}
\end{entry}

\begin{entry}{晚点}{11,9}[Radicais ⽇、⽕]
  \begin{phonetics}{晚点}{wan3dian3}
    \definition{adj.}{atrasado}
    \definition{s.}{jantar leve}
  \end{phonetics}
\end{entry}

\begin{entry}{晚景}{11,12}[Radicais ⽇、⽇]
  \begin{phonetics}{晚景}{wan3jing3}
    \definition{s.}{circunstâncias dos anos de declínio de alguém | cena noturna}
  \end{phonetics}
\end{entry}

\begin{entry}{晚餐}{11,16}[Radicais ⽇、⾷]
  \begin{phonetics}{晚餐}{wan3can1}[][HSK 2]
    \definition[份,顿,次]{s.}{jantar | refeição noturna}
  \end{phonetics}
\end{entry}

\begin{entry}{梦}{11}[Radical ⼣]
  \begin{phonetics}{梦}{meng4}
    \definition[场,个]{s.}{sonho}
    \definition{v.}{sonhar}
  \end{phonetics}
\end{entry}

\begin{entry}{梯恩梯}{11,10,11}[Radicais ⽊、⼼、⽊]
  \begin{phonetics}{梯恩梯}{ti1'en1ti1}
    \definition{s.}{(empréstimo linguístico) TNT, trinitrotolueno}
  \end{phonetics}
\end{entry}

\begin{entry}{检查}{11,9}[Radicais ⽊、⽊]
  \begin{phonetics}{检查}{jian3cha2}[][HSK 2]
    \definition[次]{s.}{inspeção}
    \definition{v.}{examinar | inspecionar}
  \end{phonetics}
\end{entry}

\begin{entry}{欲}{11}[Radical ⽋]
  \begin{phonetics}{欲}{yu4}
    \definition{adj.}{desejo | apetite | paixão | luxúria | ganância}
    \definition{v.}{desejar}
  \end{phonetics}
\end{entry}

\begin{entry}{毫不费力}{11,4,9,2}[Radicais ⽊、⼀、⾙、⼒]
  \begin{phonetics}{毫不费力}{hao2bu2fei4li4}
    \definition{adj.}{sem esforço | não gastando o menor esforço}
  \end{phonetics}
\end{entry}

\begin{entry}{毫米}{11,6}[Radicais ⽊、⽶]
  \begin{phonetics}{毫米}{hao2mi3}
    \definition{s.}{milímetro}
  \end{phonetics}
\end{entry}

\begin{entry}{液体}{11,7}[Radicais ⽔、⼈]
  \begin{phonetics}{液体}{ye4ti3}
    \definition{adj./s.}{líquido}
  \end{phonetics}
\end{entry}

\begin{entry}{涵}{11}[Radical ⽔]
  \begin{phonetics}{涵}{han2}
    \definition{s.}{bueiro | galeria}
    \definition{v.}{conter | incluir | entupir}
  \end{phonetics}
\end{entry}

\begin{entry}{淀}{11}[Radical ⽔]
  \begin{phonetics}{淀}{dian4}
    \definition{adj.}{pantanoso}
    \definition{s.}{lago raso | pântano}
    \definition{v.}{formar sedimentos | precipitar}
  \end{phonetics}
\end{entry}

\begin{entry}{淋}{11}[Radical ⽔]
  \begin{phonetics}{淋}{lin2}
    \definition{v.}{borrifar | pingar | derramar | encharcar}
  \end{phonetics}
  \begin{phonetics}{淋}{lin4}
    \definition{s.}{gonorréia}
    \definition{v.}{filtrar | coar | drenar}
  \end{phonetics}
\end{entry}

\begin{entry}{淡}{11}[Radical ⽔]
  \begin{phonetics}{淡}{dan4}[][HSK 4]
    \definition*{s.}{sobrenome Dan}
    \definition{adj.}{fino; leve | sem gosto; fraco; não tem sabor forte; não é salgado | leve; fraco; pálido | indiferente; frio; sem entusiasmo | frouxo; sem brilho | sem sentido; trivial}
  \end{phonetics}
\end{entry}

\begin{entry}{淤泥}{11,8}[Radicais ⽔、⽔]
  \begin{phonetics}{淤泥}{yu1ni2}
    \definition{s.}{lodo}
  \end{phonetics}
\end{entry}

\begin{entry}{深}{11}[Radical ⽔]
  \begin{phonetics}{深}{shen1}[][HSK 3]
    \definition*{s.}{sobrenome Shen}
    \definition{adj.}{profundo
difícil; intenso; profundo
completo; penetrante; intenso; profundo
próximo; íntimo
escuro; profundo
tardio}
    \definition{adv.}{muito; grandemente; profundamente}
    \definition{s.}{profundidade}
  \end{phonetics}
\end{entry}

\begin{entry}{深入}{11,2}[Radicais ⽔、⼊]
  \begin{phonetics}{深入}{shen1 ru4}[][HSK 3]
    \definition{adj.}{minucioso; meticuloso; profundo}
    \definition{v.}{ir fundo em; penetrar em}
  \end{phonetics}
\end{entry}

\begin{entry}{深刻}{11,8}[Radicais ⽔、⼑]
  \begin{phonetics}{深刻}{shen1ke4}[][HSK 3]
    \definition{adj.}{profundo; instenso}
  \end{phonetics}
\end{entry}

\begin{entry}{深夜}{11,8}[Radicais ⽔、⼣]
  \begin{phonetics}{深夜}{shen1ye4}
    \definition{adv.}{tarde da noite}
  \end{phonetics}
\end{entry}

\begin{entry}{深厚}{11,9}[Radicais ⽔、⼚]
  \begin{phonetics}{深厚}{shen1hou4}
    \definition{adj.}{profundo}
  \end{phonetics}
\end{entry}

\begin{entry}{深深}{11,11}[Radicais ⽔、⽔]
  \begin{phonetics}{深深}{shen1shen1}
    \definition{adj.}{profundo}
    \definition{adv.}{profundamente}
  \end{phonetics}
\end{entry}

\begin{entry}{混乱}{11,7}[Radicais ⽔、⼄]
  \begin{phonetics}{混乱}{hun4luan4}
    \definition{adj.}{confuso | caótico | desordenado}
    \definition{s.}{caos}
  \end{phonetics}
\end{entry}

\begin{entry}{混饭}{11,7}[Radicais ⽔、⾷]
  \begin{phonetics}{混饭}{hun4fan4}
    \definition{v.+compl.}{trabalhar para viver}
  \end{phonetics}
\end{entry}

\begin{entry}{清}{11}[Radical ⽔]
  \begin{phonetics}{清}{qing1}
    \definition*{s.}{sobrenome Qing}
    \definition{adj.}{claro | limpo (água, etc.) | tranquilo | quieto | puro | não corrompido | distinto}
    \definition{v.}{limpar | resolver (contas)}
  \end{phonetics}
\end{entry}

\begin{entry}{清彻}{11,7}[Radicais ⽔、⼻]
  \begin{phonetics}{清彻}{qing1che4}
    \variantof{清澈}
  \end{phonetics}
\end{entry}

\begin{entry}{清明节}{11,8,5}[Radicais ⽔、⽇、⾋]
  \begin{phonetics}{清明节}{qing1ming2jie2}
    \definition*{s.}{Dia Qingming, Dia dos Finados (uma das 24~divisões do ano solar no calendário lunar chinês:~dia~4 ou 5~de~abril solar)}
  \end{phonetics}
\end{entry}

\begin{entry}{清凉}{11,10}[Radicais ⽔、⼎]
  \begin{phonetics}{清凉}{qing1liang2}
    \definition{adj.}{fresco | refrescante | (roupa) ousada, reveladora}
  \end{phonetics}
\end{entry}

\begin{entry}{清唱}{11,11}[Radicais ⽔、⼝]
  \begin{phonetics}{清唱}{qing1chang4}
    \definition{v.}{cantar à capela}
  \end{phonetics}
\end{entry}

\begin{entry}{清爽}{11,11}[Radicais ⽔、⽘]
  \begin{phonetics}{清爽}{qing1shuang3}
    \definition{adj.}{refrescante | relaxado}
  \end{phonetics}
\end{entry}

\begin{entry}{清理}{11,11}[Radicais ⽔、⽟]
  \begin{phonetics}{清理}{qing1li3}
    \definition{v.}{limpar | arrumar | descartar}
  \end{phonetics}
\end{entry}

\begin{entry}{清晰}{11,12}[Radicais ⽔、⽇]
  \begin{phonetics}{清晰}{qing1xi1}
    \definition{adj.}{claro | distinto}
  \end{phonetics}
\end{entry}

\begin{entry}{清楚}{11,13}[Radicais ⽔、⽊]
  \begin{phonetics}{清楚}{qing1chu5}[][HSK 2]
    \definition{adj.}{claro | límpido}
    \definition{v.}{ser claro sobre | entender completamente}
  \end{phonetics}
\end{entry}

\begin{entry}{清澈}{11,15}[Radicais ⽔、⽔]
  \begin{phonetics}{清澈}{qing1che4}
    \definition{adj.}{claro | límpido}
  \end{phonetics}
\end{entry}

\begin{entry}{渐渐}{11,11}[Radicais ⽔、⽔]
  \begin{phonetics}{渐渐}{jian4jian4}
    \definition{adv.}{pouco a pouco | passo a passo | progressivamente}
  \end{phonetics}
\end{entry}

\begin{entry}{渔}{11}[Radical ⽔]
  \begin{phonetics}{渔}{yu2}
    \definition[条]{s.}{pescador}
    \definition{v.}{pescar}
  \end{phonetics}
\end{entry}

\begin{entry}{渔夫}{11,4}[Radicais ⽔、⼤]
  \begin{phonetics}{渔夫}{yu2fu1}
    \definition{s.}{pescador}
  \end{phonetics}
\end{entry}

\begin{entry}{渔民}{11,5}[Radicais ⽔、⽒]
  \begin{phonetics}{渔民}{yu2min2}
    \definition{s.}{pescadores | povo pescador}
  \end{phonetics}
\end{entry}

\begin{entry}{渔场}{11,6}[Radicais ⽔、⼟]
  \begin{phonetics}{渔场}{yu2chang3}
    \definition{s.}{área de pesca}
  \end{phonetics}
\end{entry}

\begin{entry}{渔汛}{11,6}[Radicais ⽔、⽔]
  \begin{phonetics}{渔汛}{yu2xun4}
    \definition{s.}{temporada de pesca}
  \end{phonetics}
\end{entry}

\begin{entry}{渔网}{11,6}[Radicais ⽔、⽹]
  \begin{phonetics}{渔网}{yu2wang3}
    \definition{s.}{rede de pesca}
  \end{phonetics}
\end{entry}

\begin{entry}{渔具}{11,8}[Radicais ⽔、⼋]
  \begin{phonetics}{渔具}{yu2ju4}
    \definition{s.}{equipamento de pesca}
  \end{phonetics}
\end{entry}

\begin{entry}{渔轮}{11,8}[Radicais ⽔、⾞]
  \begin{phonetics}{渔轮}{yu2lun2}
    \definition{s.}{navio de pesca}
  \end{phonetics}
\end{entry}

\begin{entry}{渔捞}{11,10}[Radicais ⽔、⼿]
  \begin{phonetics}{渔捞}{yu2lao1}
    \definition{s.}{pesca (como atividade comercial)}
  \end{phonetics}
\end{entry}

\begin{entry}{渔猎}{11,11}[Radicais ⽔、⽝]
  \begin{phonetics}{渔猎}{yu2lie4}
    \definition{s.}{pesca e caça}
    \definition{v.}{saquear | pilhar}
  \end{phonetics}
\end{entry}

\begin{entry}{渔笼}{11,11}[Radicais ⽔、⽵]
  \begin{phonetics}{渔笼}{yu2long2}
    \definition{s.}{gaiola de pesca | armadilha de pesca}
  \end{phonetics}
\end{entry}

\begin{entry}{渔船}{11,11}[Radicais ⽔、⾈]
  \begin{phonetics}{渔船}{yu2chuan2}
    \definition[条]{s.}{barco de pesca}
  \seealsoref{鱼船}{yu2chuan2}
  \end{phonetics}
\end{entry}

\begin{entry}{渔船队}{11,11,4}[Radicais ⽔、⾈、⾩]
  \begin{phonetics}{渔船队}{yu2chuan2dui4}
    \definition{s.}{frota pesqueira}
  \end{phonetics}
\end{entry}

\begin{entry}{焊}{11}[Radical ⽕]
  \begin{phonetics}{焊}{han4}
    \definition{v.}{soldar}
  \end{phonetics}
\end{entry}

\begin{entry}{猎物}{11,8}[Radicais ⽝、⽜]
  \begin{phonetics}{猎物}{lie4wu4}
    \definition{s.}{presa (vítima de um predador)}
  \end{phonetics}
\end{entry}

\begin{entry}{猛}{11}[Radical ⽝]
  \begin{phonetics}{猛}{meng3}
    \definition{adj.}{feroz | violento | corajoso | abrupto | (gíria) incrível}
    \definition{adv.}{de repente}
  \end{phonetics}
\end{entry}

\begin{entry}{猛然}{11,12}[Radicais ⽝、⽕]
  \begin{phonetics}{猛然}{meng3ran2}
    \definition{adv.}{de repente | abruptamente}
  \end{phonetics}
\end{entry}

\begin{entry}{猜}{11}[Radical ⽝]
  \begin{phonetics}{猜}{cai1}
    \definition{v.}{advinhar}
  \end{phonetics}
\end{entry}

\begin{entry}{猪}{11}[Radical ⽝]
  \begin{phonetics}{猪}{zhu1}[][HSK 3]
    \definition[口,头]{s.}{porco; suíno}
  \end{phonetics}
\end{entry}

\begin{entry}{猪头}{11,5}[Radicais ⽝、⼤]
  \begin{phonetics}{猪头}{zhu1tou2}
    \definition{s.}{tolo | idiota}
  \end{phonetics}
\end{entry}

\begin{entry}{猪柳}{11,9}[Radicais ⽝、⽊]
  \begin{phonetics}{猪柳}{zhu1liu3}
    \definition{s.}{filé de porco}
  \end{phonetics}
\end{entry}

\begin{entry}{猪笼}{11,11}[Radicais ⽝、⽵]
  \begin{phonetics}{猪笼}{zhu1long2}
    \definition{s.}{estrutura cilíndrica de bambu ou arame usada para restringir um porco durante o transporte}
  \end{phonetics}
\end{entry}

\begin{entry}{猪窠}{11,13}[Radicais ⽝、⽳]
  \begin{phonetics}{猪窠}{zhu1ke1}
    \definition{s.}{chiqueiro}
  \end{phonetics}
\end{entry}

\begin{entry}{猫}{11}[Radical ⽝]
  \begin{phonetics}{猫}{mao1}[][HSK 2]
    \definition[只]{s.}{gato |  (empréstimo linguístico) (coloquial) MODEM}
    \definition{v.}{(dialeto) esconder-se}
  \end{phonetics}
  \begin{phonetics}{猫}{mao2}
    \definition{v.}{utilizado em 猫腰 \dpy{mao2yao1}}
    \seeref{猫腰}{mao2yao1}
  \end{phonetics}
\end{entry}

\begin{entry}{猫腰}{11,13}[Radicais ⽝、⾁]
  \begin{phonetics}{猫腰}{mao2yao1}
    \definition{v.}{curvar-se}
  \end{phonetics}
\end{entry}

\begin{entry}{猫熊}{11,14}[Radicais ⽝、⽕]
  \begin{phonetics}{猫熊}{mao1xiong2}
    \definition[把,只]{s.}{panda gigante}
  \seealsoref{熊猫}{xiong2mao1}
  \end{phonetics}
\end{entry}

\begin{entry}{球}{11}[Radical ⽟]
  \begin{phonetics}{球}{qiu2}[][HSK 1]
    \definition[个]{s.}{bola | esfera | globo}
    \definition[场]{s.}{jogo | partida de bola}
  \end{phonetics}
\end{entry}

\begin{entry}{球队}{11,4}[Radicais ⽟、⾩]
  \begin{phonetics}{球队}{qiu2 dui4}[][HSK 2]
    \definition{s.}{time (basquete, futebol, etc.)}
  \end{phonetics}
\end{entry}

\begin{entry}{球场}{11,6}[Radicais ⽟、⼟]
  \begin{phonetics}{球场}{qiu2 chang3}[][HSK 2]
    \definition{s.}{um campo onde são jogados jogos de bola | tribunal | campo | curso | \emph{links}}
  \end{phonetics}
\end{entry}

\begin{entry}{球拍}{11,8}[Radicais ⽟、⼿]
  \begin{phonetics}{球拍}{qiu2pai1}
    \definition{s.}{raquete}
  \end{phonetics}
\end{entry}

\begin{entry}{球迷}{11,9}[Radicais ⽟、⾡]
  \begin{phonetics}{球迷}{qiu2mi2}[][HSK 3]
    \definition[个]{s.}{fã (de esportes de bola)}
  \end{phonetics}
\end{entry}

\begin{entry}{球鞋}{11,15}[Radicais ⽟、⾰]
  \begin{phonetics}{球鞋}{qiu2 xie2}[][HSK 2]
    \definition{s.}{calçados esportivos | tênis | tênis de ginástica}
  \end{phonetics}
\end{entry}

\begin{entry}{理发}{11,5}[Radicais ⽟、⼜]
  \begin{phonetics}{理发}{li3fa4}[][HSK 3]
    \definition{v.+compl.}{cortar o cabelo; ter (dar) um corte de cabelo}
  \end{phonetics}
\end{entry}

\begin{entry}{理由}{11,5}[Radicais ⽟、⽥]
  \begin{phonetics}{理由}{li3you2}[][HSK 3]
    \definition[个]{s.}{razão; justificativa; fundamento}
  \end{phonetics}
\end{entry}

\begin{entry}{理论}{11,6}[Radicais ⽟、⾔]
  \begin{phonetics}{理论}{li3lun4}[][HSK 3]
    \definition[套,个]{s.}{teoria}
    \definition{v.}{argumentar; raciocinar com alguém}
  \end{phonetics}
\end{entry}

\begin{entry}{理想}{11,13}[Radicais ⽟、⼼]
  \begin{phonetics}{理想}{li3xiang3}[][HSK 2]
    \definition[个]{adj./s.}{ideal}
    \definition{adv.}{idealmente}
  \end{phonetics}
\end{entry}

\begin{entry}{理解}{11,13}[Radicais ⽟、⾓]
  \begin{phonetics}{理解}{li3jie3}[][HSK 3]
    \definition{v.}{entender; compreender | entender com empatia}
  \end{phonetics}
\end{entry}

\begin{entry}{甜}{11}[Radical ⽢]
  \begin{phonetics}{甜}{tian2}[][HSK 3]
    \definition{adj.}{doce; melado | agradável; confortável | (sono) profundo | feliz}
  \end{phonetics}
\end{entry}

\begin{entry}{甜心}{11,4}[Radicais ⽢、⼼]
  \begin{phonetics}{甜心}{tian2xin1}
    \definition{s.}{querido}
  \end{phonetics}
\end{entry}

\begin{entry}{甜头}{11,5}[Radicais ⽢、⼤]
  \begin{phonetics}{甜头}{tian2tou5}
    \definition{s.}{benefício | sabor doce (de poder, sucesso, etc.)}
  \end{phonetics}
\end{entry}

\begin{entry}{甜玉米}{11,5,6}[Radicais ⽢、⽟、⽶]
  \begin{phonetics}{甜玉米}{tian2 yu4mi3}
    \definition{s.}{milho doce}
  \end{phonetics}
\end{entry}

\begin{entry}{甜言}{11,7}[Radicais ⽢、⾔]
  \begin{phonetics}{甜言}{tian2yan2}
    \definition{s.}{boa conversa | palavras amáveis}
  \end{phonetics}
\end{entry}

\begin{entry}{甜品}{11,9}[Radicais ⽢、⼝]
  \begin{phonetics}{甜品}{tian2pin3}
    \definition{s.}{sobremesa}
  \end{phonetics}
\end{entry}

\begin{entry}{甜食}{11,9}[Radicais ⽢、⾷]
  \begin{phonetics}{甜食}{tian2shi2}
    \definition{s.}{doces | sobremesa}
  \end{phonetics}
\end{entry}

\begin{entry}{甜酒}{11,10}[Radicais ⽢、⾣]
  \begin{phonetics}{甜酒}{tian2jiu3}
    \definition{s.}{licor doce}
  \end{phonetics}
\end{entry}

\begin{entry}{甜甜圈}{11,11,11}[Radicais ⽢、⽢、⼞]
  \begin{phonetics}{甜甜圈}{tian2tian2quan1}
    \definition{s.}{rosquinha | \emph{doughnut}}
  \end{phonetics}
\end{entry}

\begin{entry}{甜菊}{11,11}[Radicais ⽢、⾋]
  \begin{phonetics}{甜菊}{tian2ju2}
    \definition{s.}{estévia, arbusto cujas folhas produzem um substituto para o açúcar}
  \end{phonetics}
\end{entry}

\begin{entry}{甜筒}{11,12}[Radicais ⽢、⽵]
  \begin{phonetics}{甜筒}{tian2tong3}
    \definition{s.}{sorvete de casquinha}
  \end{phonetics}
\end{entry}

\begin{entry}{甜稚}{11,13}[Radicais ⽢、⽲]
  \begin{phonetics}{甜稚}{tian2zhi4}
    \definition{s.}{doce e inocente}
  \end{phonetics}
\end{entry}

\begin{entry}{甜酸}{11,14}[Radicais ⽢、⾣]
  \begin{phonetics}{甜酸}{tian2suan1}
    \definition{adj.}{agridoce}
  \end{phonetics}
\end{entry}

\begin{entry}{略}{11}[Radical ⽥]
  \begin{phonetics}{略}{lve4}
    \definition{adv.}{ligeiramente | marginalmente | aproximadamente}
  \end{phonetics}
\end{entry}

\begin{entry}{略微}{11,13}[Radicais ⽥、⼻]
  \begin{phonetics}{略微}{lve4wei1}
    \definition{adv.}{ligeiramente | marginalmente | aproximadamente}
  \end{phonetics}
\end{entry}

\begin{entry}{盒}{11}[Radical ⽫]
  \begin{phonetics}{盒}{he2}
    \definition{clas.}{caixa pequena}
    \definition{s.}{caixa pequena | estojo}
  \end{phonetics}
\end{entry}

\begin{entry}{盘}{11}[Radical ⽫]
  \begin{phonetics}{盘}{pan2}
    \definition{clas.}{para bobinas de fio | (de comida) pratos, serviços | para jogos de xadrez}
    \definition{s.}{tabuleiro | prato | bandeja | (computação) disco rígido}
    \definition{v.}{construir | checar | enrolar | examinar | transferir (propriedade)}
  \end{phonetics}
\end{entry}

\begin{entry}{盘子}{11,3}[Radicais ⽫、⼦]
  \begin{phonetics}{盘子}{pan2zi5}
    \definition{s.}{prato| travessa | situação de mercado | taxa de mercado | transação de negócio}
  \end{phonetics}
\end{entry}

\begin{entry}{盛宴}{11,10}[Radicais ⽫、⼧]
  \begin{phonetics}{盛宴}{sheng4yan4}
    \definition{s.}{celebração}
  \end{phonetics}
\end{entry}

\begin{entry}{眯}{11}[Radical ⽬]
  \begin{phonetics}{眯}{mi1}
    \definition{v.}{estreitar os olhos | esmagar | (dialeto) tirar uma soneca}
  \end{phonetics}
  \begin{phonetics}{眯}{mi2}
    \definition{v.}{cegar (como com poeira)}
  \end{phonetics}
\end{entry}

\begin{entry}{眼}{11}[Radical ⽬]
  \begin{phonetics}{眼}{yan3}[][HSK 2]
    \definition{clas.}{para grandes coisas ocas: poços, fogões, panelas, etc.}
    \definition[只,双]{s.}{ponto crucial (de um assunto) | olho | pequeno buraco}
  \end{phonetics}
\end{entry}

\begin{entry}{眼花缭乱}{11,7,15,7}[Radicais ⽬、⾋、⽷、⼄]
  \begin{phonetics}{眼花缭乱}{yan3hua1liao2luan4}
    \definition{v.}{ficar deslumbrado | deslumbrar}
  \end{phonetics}
\end{entry}

\begin{entry}{眼证}{11,7}[Radicais ⽬、⾔]
  \begin{phonetics}{眼证}{yan3zheng4}
    \definition{s.}{testemunha ocular}
  \end{phonetics}
\end{entry}

\begin{entry}{眼泪}{11,8}[Radicais ⽬、⽔]
  \begin{phonetics}{眼泪}{yan3lei4}
    \definition[滴]{s.}{choro | lágrimas}
  \end{phonetics}
\end{entry}

\begin{entry}{眼前}{11,9}[Radicais ⽬、⼑]
  \begin{phonetics}{眼前}{yan3 qian2}[][HSK 3]
    \definition{adv.}{agora; (no) momento}
    \definition{s.}{diante dos olhos; diante de; momento}
  \end{phonetics}
\end{entry}

\begin{entry}{眼柄}{11,9}[Radicais ⽬、⽊]
  \begin{phonetics}{眼柄}{yan3bing3}
    \definition{s.}{pedúnculo ocular (de crustáceo, etc.)}
  \end{phonetics}
\end{entry}

\begin{entry}{眼袋}{11,11}[Radicais ⽬、⾐]
  \begin{phonetics}{眼袋}{yan3dai4}
    \definition{s.}{inchaço sob os olhos}
  \end{phonetics}
\end{entry}

\begin{entry}{眼睛}{11,13}[Radicais ⽬、⽬]
  \begin{phonetics}{眼睛}{yan3jing5}[][HSK 2]
    \definition[只,双]{s.}{olho(s)}
  \end{phonetics}
\end{entry}

\begin{entry}{眼镜}{11,16}[Radicais ⽬、⾦]
  \begin{phonetics}{眼镜}{yan3jing4}
    \definition[副]{s.}{óculos}
  \end{phonetics}
\end{entry}

\begin{entry}{着}{11}[Radical ⽬]
  \begin{phonetics}{着}{zhao1}
    \definition{interj.}{Tudo bem!}
    \definition{s.}{movimento (xadrez) | truque}
  \end{phonetics}
  \begin{phonetics}{着}{zhao2}
    \definition{v.}{ser afetado por | queimar | pegar fogo | entrar em contato com | sentir | tocar}
  \end{phonetics}
  \begin{phonetics}{着}{zhe5}[][HSK 1]
    \definition{part.}{indicando ação em andamento ou estado em andamento}
  \end{phonetics}
  \begin{phonetics}{着}{zhuo2}
    \definition{v.}{aplicar | contactar | usar | vestir (roupas)}
  \end{phonetics}
\end{entry}

\begin{entry}{着手}{11,4}[Radicais ⽬、⼿]
  \begin{phonetics}{着手}{zhuo2shou3}
    \definition{v.}{colocar a mão nisso | estabelecer | começar uma tarefa}
  \end{phonetics}
\end{entry}

\begin{entry}{着地}{11,6}[Radicais ⽬、⼟]
  \begin{phonetics}{着地}{zhao2di4}
    \definition{v.}{pousar | tocar o chão}
  \end{phonetics}
\end{entry}

\begin{entry}{着花}{11,7}[Radicais ⽬、⾋]
  \begin{phonetics}{着花}{zhao2hua1}
    \definition{v.}{florescer}
  \end{phonetics}
  \begin{phonetics}{着花}{zhuo2hua1}
    \definition{s.}{floração}
    \definition{v.}{florescer}
  \end{phonetics}
\end{entry}

\begin{entry}{着急}{11,9}[Radicais ⽬、⼼]
  \begin{phonetics}{着急}{zhao2ji2}
    \definition{adj.}{inquieto | ansioso}
    \definition{s.}{preocupação | ansiedade}
    \definition{v.+compl.}{preocupar-se | sentir-se ansioso | sentir uma sensação de urgência}
  \end{phonetics}
\end{entry}

\begin{entry}{着凉}{11,10}[Radicais ⽬、⼎]
  \begin{phonetics}{着凉}{zhao2liang2}
    \definition{v.}{pegar um resfriado}
  \end{phonetics}
\end{entry}

\begin{entry}{着眼}{11,11}[Radicais ⽬、⽬]
  \begin{phonetics}{着眼}{zhuo2yan3}
    \definition{v.}{ter seus olhos em (um objetivo) | ter algo em mente | concentrar-se}
  \end{phonetics}
\end{entry}

\begin{entry}{着装}{11,12}[Radicais ⽬、⾐]
  \begin{phonetics}{着装}{zhuo2zhuang1}
    \definition{s.}{roupa | vestimenta}
    \definition{v.}{vestir}
  \end{phonetics}
\end{entry}

\begin{entry}{着想}{11,13}[Radicais ⽬、⼼]
  \begin{phonetics}{着想}{zhuo2xiang3}
    \definition{v.}{considerar (as necessidades de outras pessoas) | pensar (para os outros)}
  \end{phonetics}
\end{entry}

\begin{entry}{着数}{11,13}[Radicais ⽬、⽁]
  \begin{phonetics}{着数}{zhao1shu4}
    \definition{s.}{estratégia | movimento (no xadrez, no palco, nas artes marciais) | esquema | truque}
  \end{phonetics}
\end{entry}

\begin{entry}{票}{11}[Radical ⽰]
  \begin{phonetics}{票}{piao4}[][HSK 1]
    \definition{clas.}{para grupos, lotes, transações comerciais}
    \definition[张]{s.}{performance amadora de ópera chinesa | cédula eleitoral | nota | bilhete | pessoa detida por resgate | refém}
  \end{phonetics}
\end{entry}

\begin{entry}{票价}{11,6}[Radicais ⽰、⼈]
  \begin{phonetics}{票价}{piao4 jia4}[][HSK 3]
    \definition[个]{s.}{o preço de um bilhete; taxa de admissão; taxa de entrada}
  \end{phonetics}
\end{entry}

\begin{entry}{章}{11}[Radical ⾳]
  \begin{phonetics}{章}{zhang1}
    \definition*{s.}{sobrenome Zhang}
    \definition{s.}{capítulo | seção | cláusula |  movimento (de sinfonia) | selo | crachá | regulamento}
  \end{phonetics}
\end{entry}

\begin{entry}{章鱼}{11,8}[Radicais ⾳、⿂]
  \begin{phonetics}{章鱼}{zhang1yu2}
    \definition{s.}{polvo | octópode}
  \end{phonetics}
\end{entry}

\begin{entry}{笛}{11}[Radical ⽵]
  \begin{phonetics}{笛}{di2}
    \definition{s.}{flauta}
  \end{phonetics}
\end{entry}

\begin{entry}{符合}{11,6}[Radicais ⽵、⼝]
  \begin{phonetics}{符合}{fu2he2}
    \definition{conj.}{de acordo com | concordando com | contando com | alinhado com}
    \definition{v.}{concordar com | estar em conformidade com | corresponder com | gerenciar | lidar}
  \end{phonetics}
\end{entry}

\begin{entry}{笨}{11}[Radical ⽵]
  \begin{phonetics}{笨}{ben4}[][HSK 4]
    \definition{adj.}{estúpido; sem graça; tolo; de pouca habilidade; sem inteligência | desajeitado; tosco; inflexível | incômodo; pesado; desajeitado; difícil de manejar; trabalhoso}
  \end{phonetics}
\end{entry}

\begin{entry}{笨蛋}{11,11}[Radicais ⽵、⾍]
  \begin{phonetics}{笨蛋}{ben4dan4}
    \definition{s.}{bobalhão | cabeça-oca | cabeça-dura}
    \definition{v.}{iludir | enganar}
  \end{phonetics}
\end{entry}

\begin{entry}{第}{11}[Radical ⽵]
  \begin{phonetics}{第}{di4}[][HSK 1]
    \definition{num.}{prefixo para expressar números ordinais}
  \end{phonetics}
\end{entry}

\begin{entry}{笼}{11}[Radical ⽵]
  \begin{phonetics}{笼}{long2}
    \definition{s.}{armação fechada de bambu, arame, etc. | jaula | gaiola}
  \end{phonetics}
  \begin{phonetics}{笼}{long3}
    \definition{v.}{envolver | cobrir}
  \end{phonetics}
\end{entry}

\begin{entry}{笼子}{11,3}[Radicais ⽵、⼦]
  \begin{phonetics}{笼子}{long2zi5}
    \definition{s.}{jaula | cesta | gaiola | recipiente}
  \end{phonetics}
  \begin{phonetics}{笼子}{long3zi5}
    \definition{s.}{caixa grande | porta-malas}
  \end{phonetics}
\end{entry}

\begin{entry}{粗}{11}[Radical ⽶]
  \begin{phonetics}{粗}{cu1}[][HSK 4]
    \definition{adj.}{largo (em diâmetro); grosso | grosseiro; rude; áspero | áspero; rouco | descuidado; negligente | rude; sem refinamento; vulgar}
    \definition{adv.}{grosseiramente; vagamente}
  \end{phonetics}
\end{entry}

\begin{entry}{粗心}{11,4}[Radicais ⽶、⼼]
  \begin{phonetics}{粗心}{cu1xin1}[][HSK 4]
    \definition{adj.}{descuidado; irrefletido; (fazer as coisas) de forma desleixada, sem cuidado}
  \end{phonetics}
\end{entry}

\begin{entry}{粗心地做}{11,4,6,11}[Radicais ⽶、⼼、⼟、⼈]
  \begin{phonetics}{粗心地做}{cu1xin1 di4 zuo4}
    \definition{adj.}{feito descuidadamente}
  \end{phonetics}
\end{entry}

\begin{entry}{粗糙}{11,16}[Radicais ⽶、⽶]
  \begin{phonetics}{粗糙}{cu1cao1}
    \definition{adj.}{áspero | grosseiro}
  \end{phonetics}
\end{entry}

\begin{entry}{累}{11}[Radical ⽷]
  \begin{phonetics}{累}{lei2}
    \definition*{s.}{sobrenome Lei}
    \definition{s.}{corda}
    \definition{v.}{amarrar | torcer}
  \end{phonetics}
  \begin{phonetics}{累}{lei3}
    \definition{adj.}{contínuo | repetido}
    \definition{v.}{acumular | envolver ou implicar}
  \end{phonetics}
  \begin{phonetics}{累}{lei4}[][HSK 1]
    \definition{adj.}{cansado | fatigado}
    \definition{v.}{forçar | desgastar | trabalhar duro}
  \end{phonetics}
\end{entry}

\begin{entry}{绰号}{11,5}[Radicais ⽷、⼝]
  \begin{phonetics}{绰号}{chuo4hao4}
    \definition{s.}{apelido}
  \end{phonetics}
\end{entry}

\begin{entry}{绳子}{11,3}[Radicais ⽷、⼦]
  \begin{phonetics}{绳子}{sheng2zi5}
    \definition[条]{s.}{corda | cordão}
  \end{phonetics}
\end{entry}

\begin{entry}{维吾尔}{11,7,5}[Radicais ⽷、⼝、⼩]
  \begin{phonetics}{维吾尔}{wei2wu2'er3}
    \definition*{s.}{Grupo étnico Uigur de Xinjiang}
  \end{phonetics}
\end{entry}

\begin{entry}{绷带}{11,9}[Radicais ⽷、⼱]
  \begin{phonetics}{绷带}{beng1dai4}
    \definition{s.}{curativo | (empréstimo linguístico) \emph{bandage}}
  \end{phonetics}
\end{entry}

\begin{entry}{绿}{11}[Radical ⽷]
  \begin{phonetics}{绿}{lv4}[][HSK 2]
    \definition{adj.}{verde}
  \end{phonetics}
\end{entry}

\begin{entry}{绿色}{11,6}[Radicais ⽷、⾊]
  \begin{phonetics}{绿色}{lv4 se4}[][HSK 2]
    \definition{s.}{cor verde}
  \end{phonetics}
\end{entry}

\begin{entry}{绿豆}{11,7}[Radicais ⽷、⾖]
  \begin{phonetics}{绿豆}{lv4dou4}
    \definition{s.}{vagens}
  \end{phonetics}
\end{entry}

\begin{entry}{绿豆芽}{11,7,7}[Radicais ⽷、⾖、⾋]
  \begin{phonetics}{绿豆芽}{lv4dou4 ya2}
    \definition{s.}{broto de feijão verde}
  \end{phonetics}
\end{entry}

\begin{entry}{绿茶}{11,9}[Radicais ⽷、⾋]
  \begin{phonetics}{绿茶}{lv4 cha2}[][HSK 3]
    \definition{s.}{chá verde}
  \end{phonetics}
\end{entry}

\begin{entry}{聊天}{11,4}[Radicais ⽿、⼤]
  \begin{phonetics}{聊天}{liao2tian1}
    \definition{v.+compl.}{papear | bater papo}
  \end{phonetics}
\end{entry}

\begin{entry}{职工}{11,3}[Radicais ⽿、⼯]
  \begin{phonetics}{职工}{zhi2 gong1}[][HSK 3]
    \definition[个]{s.}{pessoal; trabalhadores e pessoal de escritório}
  \end{phonetics}
\end{entry}

\begin{entry}{职业}{11,5}[Radicais ⽿、⼀]
  \begin{phonetics}{职业}{zhi2ye4}[][HSK 3]
    \definition{adj.}{profissional; não amador}
    \definition[种,份,个]{s.}{ocupação; profissão; vocação; o trabalho que um indivíduo realiza na sociedade como sua principal fonte de subsistência}
  \end{phonetics}
\end{entry}

\begin{entry}{职员}{11,7}[Radicais ⽿、⼝]
  \begin{phonetics}{职员}{zhi2yuan2}
    \definition[个,位]{s.}{empregado | trabalhador de escritório | membro da equipe}
  \end{phonetics}
\end{entry}

\begin{entry}{脖子}{11,3}[Radicais ⾁、⼦]
  \begin{phonetics}{脖子}{bo2zi5}
    \definition[个]{s.}{pescoço}
  \end{phonetics}
\end{entry}

\begin{entry}{脚}{11}[Radical ⾁]
  \begin{phonetics}{脚}{jiao3}[][HSK 2]
    \definition{clas.}{para chutes}
    \definition[双,只]{s.}{pé | base (de um objeto) | perna (de um animal ou objeto)}
  \end{phonetics}
  \begin{phonetics}{脚}{jue2}
    \variantof{角}
  \end{phonetics}
\end{entry}

\begin{entry}{脱毛}{11,4}[Radicais ⾁、⽑]
  \begin{phonetics}{脱毛}{tuo1mao2}
    \definition{s.}{depilação}
    \definition{v.}{perder cabelo ou penas | depilar | fazer a barba}
  \end{phonetics}
\end{entry}

\begin{entry}{脱险}{11,9}[Radicais ⾁、⾩]
  \begin{phonetics}{脱险}{tuo1xian3}
    \definition{v.}{sair do perigo}
  \end{phonetics}
\end{entry}

\begin{entry}{脸}{11}[Radical ⾁]
  \begin{phonetics}{脸}{lian3}[][HSK 2]
    \definition[张,个]{s.}{cara | rosto | face}
  \end{phonetics}
\end{entry}

\begin{entry}{脸色}{11,6}[Radicais ⾁、⾊]
  \begin{phonetics}{脸色}{lian3se4}
    \definition{s.}{compleição; tez; face}
  \end{phonetics}
\end{entry}

\begin{entry}{船}{11}[Radical ⾈]
  \begin{phonetics}{船}{chuan2}[][HSK 2]
    \definition[条,艘,只]{s.}{barco | navio}
  \end{phonetics}
\end{entry}

\begin{entry}{菜}{11}[Radical ⾋]
  \begin{phonetics}{菜}{cai4}[][HSK 1]
    \definition[棵]{s.}{hortaliça | verdura | legume}
    \definition[样,道,盘]{s.}{prato (tipo de alimento) | o tipo (de alguém) | (características de alguém, etc.) fraco, pobre}
  \end{phonetics}
\end{entry}

\begin{entry}{菜刀}{11,2}[Radicais ⾋、⼑]
  \begin{phonetics}{菜刀}{cai4dao1}
    \definition[把]{s.}{faca de vegetais | faca de cozinha | cutelo}
  \end{phonetics}
\end{entry}

\begin{entry}{菜单}{11,8}[Radicais ⾋、⼗]
  \begin{phonetics}{菜单}{cai4dan1}[][HSK 2]
    \definition[份,张]{s.}{menu | cardápio}
  \end{phonetics}
\end{entry}

\begin{entry}{菠菜}{11,11}[Radicais ⾋、⾋]
  \begin{phonetics}{菠菜}{bo1cai4}
    \definition[棵]{s.}{espinafre}
  \end{phonetics}
\end{entry}

\begin{entry}{菱角}{11,7}[Radicais ⾋、⾓]
  \begin{phonetics}{菱角}{ling2jiao5}
    \definition{s.}{castanha d'água}
  \end{phonetics}
\end{entry}

\begin{entry}{营养}{11,9}[Radicais ⾋、⼋]
  \begin{phonetics}{营养}{ying2yang3}[][HSK 3]
    \definition[种]{s.}{nutrição; a função de um organismo de absorver as substâncias necessárias do mundo exterior para manter o crescimento, o desenvolvimento e outras atividades vitais; o ato ou processo de fornecer nutrição}
  \end{phonetics}
\end{entry}

\begin{entry}{虚伪}{11,6}[Radicais ⾌、⼈]
  \begin{phonetics}{虚伪}{xu1wei3}
    \definition{adj.}{falso | hipócrita | artificial}
  \end{phonetics}
\end{entry}

\begin{entry}{蛇}{11}[Radical ⾍]
  \begin{phonetics}{蛇}{she2}
    \definition[条]{s.}{cobra | serpente}
  \end{phonetics}
\end{entry}

\begin{entry}{蛋}{11}[Radical ⾍]
  \begin{phonetics}{蛋}{dan4}[][HSK 2]
    \definition[个,打]{s.}{ovo | objeto de formato oval}
  \end{phonetics}
\end{entry}

\begin{entry}{蛋糕}{11,16}[Radicais ⾍、⽶]
  \begin{phonetics}{蛋糕}{dan4gao1}
    \definition[块,个]{s.}{bolo}
  \end{phonetics}
\end{entry}

\begin{entry}{袋}{11}[Radical ⾐]
  \begin{phonetics}{袋}{dai4}[][HSK 4]
    \definition{clas.}{para armazenamento em sacolas | para cachimbos, cigarros ou tabaco seco}
    \definition[口]{s.}{saco; sacola; bolso; bolsa}
  \end{phonetics}
\end{entry}

\begin{entry}{袭击}{11,5}[Radicais ⾐、⼐]
  \begin{phonetics}{袭击}{xi2ji1}
    \definition{s.}{ataque (especialmente um ataque surpresa) | invasão}
    \definition{v.}{atacar}
  \end{phonetics}
\end{entry}

\begin{entry}{谎话}{11,8}[Radicais ⾔、⾔]
  \begin{phonetics}{谎话}{huang3hua4}
    \definition{s.}{mentira}
  \end{phonetics}
\end{entry}

\begin{entry}{谐}{11}[Radical ⾔]
  \begin{phonetics}{谐}{xie2}
    \definition{adj.}{harmonioso | humorístico}
  \end{phonetics}
\end{entry}

\begin{entry}{距离}{11,10}[Radicais ⾜、⼇]
  \begin{phonetics}{距离}{ju4li2}
    \definition[个]{s.}{distância}
    \definition{v.}{estar distante de}
  \end{phonetics}
\end{entry}

\begin{entry}{辆}{11}[Radical ⾞]
  \begin{phonetics}{辆}{liang4}[][HSK 2]
    \definition{clas.}{para automóveis, veículos, etc.}
  \end{phonetics}
\end{entry}

\begin{entry}{逮}{11}[Radical ⾡]
  \begin{phonetics}{逮}{dai3}
    \definition{v.}{(coloquial) pegar, aproveitar, capturar}
  \end{phonetics}
  \begin{phonetics}{逮}{dai4}
    \definition{v.}{(literário) alcançar, usado em 逮捕}
  \seealsoref{逮捕}{dai4bu3}
  \end{phonetics}
\end{entry}

\begin{entry}{逮捕}{11,10}[Radicais ⾡、⼿]
  \begin{phonetics}{逮捕}{dai4bu3}
    \definition{v.}{prender | apreender | levar sob custódia}
  \end{phonetics}
\end{entry}

\begin{entry}{野}{11}[Radical ⾥]
  \begin{phonetics}{野}{ye3}
    \definition{adj.}{selvagem | rude}
    \definition{s.}{campo | espaço aberto | limite}
  \end{phonetics}
\end{entry}

\begin{entry}{野生}{11,5}[Radicais ⾥、⽣]
  \begin{phonetics}{野生}{ye3sheng1}
    \definition{adj.}{selvagem | não domesticado}
  \end{phonetics}
\end{entry}

\begin{entry}{铲车}{11,4}[Radicais ⾦、⾞]
  \begin{phonetics}{铲车}{chan3che1}
    \definition[台]{s.}{empilhadeira}
  \end{phonetics}
\end{entry}

\begin{entry}{银}{11}[Radical ⾦]
  \begin{phonetics}{银}{yin2}[][HSK 3]
    \definition*{s.}{sobrenome Yin}
    \definition{adj.}{relativo a moeda ou dinheiro
prateado; como a cor da prata}
    \definition[锭]{s.}{prata (Ag)}
  \end{phonetics}
\end{entry}

\begin{entry}{银色}{11,6}[Radicais ⾦、⾊]
  \begin{phonetics}{银色}{yin2 se4}
    \definition{s.}{cor prata; prateado}
  \end{phonetics}
\end{entry}

\begin{entry}{银行}{11,6}[Radicais ⾦、⾏]
  \begin{phonetics}{银行}{yin2hang2}[][HSK 2]
    \definition[家,个]{s.}{banco | agência bancária}
  \end{phonetics}
\end{entry}

\begin{entry}{银行卡}{11,6,5}[Radicais ⾦、⾏、⼘]
  \begin{phonetics}{银行卡}{yin2 hang2 ka3}[][HSK 2]
    \definition{s.}{cartão bancário}
  \end{phonetics}
\end{entry}

\begin{entry}{银河}{11,8}[Radicais ⾦、⽔]
  \begin{phonetics}{银河}{yin2he2}
    \definition*{s.}{Via Láctea}
  \seealsoref{银河系}{yin2he2xi4}
  \end{phonetics}
\end{entry}

\begin{entry}{银河系}{11,8,7}[Radicais ⾦、⽔、⽷]
  \begin{phonetics}{银河系}{yin2he2xi4}
    \definition*{s.}{Galáxia Via Láctea}
  \seealsoref{银河}{yin2he2}
  \end{phonetics}
\end{entry}

\begin{entry}{银牌}{11,12}[Radicais ⾦、⽚]
  \begin{phonetics}{银牌}{yin2 pai2}[][HSK 3]
    \definition[枚]{s.}{medalha de prata; um tipo de medalha concedida ao segundo colocado}
  \end{phonetics}
\end{entry}

\begin{entry}{随}{11}[Radical ⾩]
  \begin{phonetics}{随}{sui2}[][HSK 3]
    \definition*{s.}{sobrenome Sui}
    \definition{v.}{seguir | cumprir com; adaptar-se a | deixar (alguém fazer o que ele gosta) | parecer-se com; assemelhar-se a}
  \end{phonetics}
\end{entry}

\begin{entry}{随处}{11,5}[Radicais ⾩、⼡]
  \begin{phonetics}{随处}{sui2chu4}
    \definition{adv.}{em qualquer lugar}
  \end{phonetics}
\end{entry}

\begin{entry}{随地}{11,6}[Radicais ⾩、⼟]
  \begin{phonetics}{随地}{sui2di4}
    \definition{adv.}{qualquer lugar | todo lugar}
  \end{phonetics}
\end{entry}

\begin{entry}{随机存取记忆体}{11,6,6,8,5,4,7}[Radicais ⾩、⽊、⼦、⼜、⾔、⼼、⼈]
  \begin{phonetics}{随机存取记忆体}{sui2ji1cun2qu3ji4yi4ti3}
    \definition{s.}{RAM (\emph{random access memory})}
  \seealsoref{内存}{nei4cun2}
  \seealsoref{随机存取存储器}{sui2ji1cun2qu3cun2chu3qi4}
  \end{phonetics}
\end{entry}

\begin{entry}{随机存取存储器}{11,6,6,8,6,12,16}[Radicais ⾩、⽊、⼦、⼜、⼦、⼈、⼝]
  \begin{phonetics}{随机存取存储器}{sui2ji1cun2qu3cun2chu3qi4}
    \definition{s.}{RAM (\emph{random access memory})}
  \seealsoref{内存}{nei4cun2}
  \seealsoref{随机存取记忆体}{sui2ji1cun2qu3ji4yi4ti3}
  \end{phonetics}
\end{entry}

\begin{entry}{随时}{11,7}[Radicais ⾩、⽇]
  \begin{phonetics}{随时}{sui2shi2}[][HSK 2]
    \definition{adv.}{a qualquer momento | sempre que necessário}
  \end{phonetics}
\end{entry}

\begin{entry}{随便}{11,9}[Radicais ⾩、⼈]
  \begin{phonetics}{随便}{sui2bian4}[][HSK 2]
    \definition{adj.}{à vontade | como queira | como desejar | casual | negligente | devasso}
    \definition{adv.}{aleatoriamente}
  \end{phonetics}
\end{entry}

\begin{entry}{雪}{11}[Radical ⾬]
  \begin{phonetics}{雪}{xue3}[][HSK 2]
    \definition*{s.}{sobrenome Xue}
    \definition[场]{s.}{neve}
  \end{phonetics}
\end{entry}

\begin{entry}{雪人}{11,2}[Radicais ⾬、⼈]
  \begin{phonetics}{雪人}{xue3ren2}
    \definition{s.}{boneco de neve | \emph{Yeti}}
  \end{phonetics}
\end{entry}

\begin{entry}{雪山}{11,3}[Radicais ⾬、⼭]
  \begin{phonetics}{雪山}{xue3shan1}
    \definition{s.}{montanha coberta de neve}
  \end{phonetics}
\end{entry}

\begin{entry}{雪花}{11,7}[Radicais ⾬、⾋]
  \begin{phonetics}{雪花}{xue3hua1}
    \definition{s.}{floco de neve}
  \end{phonetics}
\end{entry}

\begin{entry}{雪板}{11,8}[Radicais ⾬、⽊]
  \begin{phonetics}{雪板}{xue3ban3}
    \definition{s.}{prancha de \emph{snowboard}}
    \definition{v.}{praticar \textit{snowboard}}
  \end{phonetics}
\end{entry}

\begin{entry}{雪葩}{11,12}[Radicais ⾬、⾋]
  \begin{phonetics}{雪葩}{xue3pa1}
    \definition{s.}{sorvete}
  \end{phonetics}
\end{entry}

\begin{entry}{雪鞋}{11,15}[Radicais ⾬、⾰]
  \begin{phonetics}{雪鞋}{xue3xie2}
    \definition[双]{s.}{sapatos de neve}
  \end{phonetics}
\end{entry}

\begin{entry}{雪糕}{11,16}[Radicais ⾬、⽶]
  \begin{phonetics}{雪糕}{xue3gao1}
    \definition{s.}{picolé}
  \end{phonetics}
\end{entry}

\begin{entry}{领}{11}[Radical ⾴]
  \begin{phonetics}{领}{ling3}[][HSK 3]
    \definition{adj.}{territorial (sob jurisdição de; em posse de)}
    \definition{clas.}{para roupas, tapetes, telas, etc.}
    \definition{s.}{pescoço; gargalo | colarinho; faixa de pescoço | esboço; ponto principal; essência}
    \definition{v.}{encabeçar; liderar; conduzir | possuir; ser o possuidor de | receber; obter; conseguir | aceitar; tomar |entender; compreender | adotar}
  \end{phonetics}
\end{entry}

\begin{entry}{领先}{11,6}[Radicais ⾴、⼉]
  \begin{phonetics}{领先}{ling3xian1}[][HSK 3]
    \definition{v.}{liderar; assumir a liderança; estar na liderança}
  \end{phonetics}
\end{entry}

\begin{entry}{领导}{11,6}[Radicais ⾴、⼨]
  \begin{phonetics}{领导}{ling3dao3}[][HSK 3]
    \definition[个,位]{s.}{líder; liderança}
    \definition{v.}{liderar; exercer liderança}
  \end{phonetics}
\end{entry}

\begin{entry}{领情}{11,11}[Radicais ⾴、⼼]
  \begin{phonetics}{领情}{ling3qing2}
    \definition{v.+compl.}{sentir-se grato a alguém}
  \end{phonetics}
\end{entry}

\begin{entry}{颇}{11}[Radical ⽪]
  \begin{phonetics}{颇}{po1}
    \definition*{s.}{sobrenome Po}
    \definition{adv.}{muito, bastante (linguagem escrita)}
  \end{phonetics}
\end{entry}

\begin{entry}{骑}{11}[Radical ⾺]
  \begin{phonetics}{骑}{qi2}[][HSK 2]
    \definition{clas.}{para cavalos de sela}
    \definition{v.}{andar (cavalo, bicicleta, etc.) | sentar-se montado | montar}
  \end{phonetics}
\end{entry}

\begin{entry}{骑车}{11,4}[Radicais ⾺、⾞]
  \begin{phonetics}{骑车}{qi2 che1}[][HSK 2]
    \definition{v.}{andar de bicicleta | pedalar}
  \end{phonetics}
\end{entry}

\begin{entry}{鸽子}{11,3}[Radicais ⿃、⼦]
  \begin{phonetics}{鸽子}{ge1zi5}
    \definition{s.}{pombo}
  \end{phonetics}
\end{entry}

\begin{entry}{鹿}{11}[Kangxi 198][Radical ⿅]
  \begin{phonetics}{鹿}{lu4}
    \definition{s.}{cervo | veado}
  \end{phonetics}
\end{entry}

\begin{entry}{麻将}{11,9}[Radicais ⿇、⼨]
  \begin{phonetics}{麻将}{ma2jiang4}
    \definition[副]{s.}{\emph{mahjong}}
  \end{phonetics}
\end{entry}

\begin{entry}{麻烦}{11,10}[Radicais ⿇、⽕]
  \begin{phonetics}{麻烦}{ma2fan5}[][HSK 3]
    \definition{adj.}{problemático; inconveniente}
    \definition[个]{s.}{problema; inconveniência}
    \definition{v.}{preocupar; incomodar; amolar; azucrinar; incomodar alguém; enfadar; aborrecer}
  \end{phonetics}
\end{entry}

\begin{entry}{麻辣豆腐}{11,14,7,14}[Radicais ⿇、⾟、⾖、⾁]
  \begin{phonetics}{麻辣豆腐}{ma2la4 dou4fu5}
    \definition{s.}{tofú guisado em molho picante (prato)}
  \end{phonetics}
\end{entry}

\begin{entry}{黄}{11}[Kangxi 201][Radical ⿈]
  \begin{phonetics}{黄}{huang2}[][HSK 2]
    \definition*{s.}{sobrenome Huang ou Hwang}
    \definition{adj.}{amarelo | pornográfico}
  \end{phonetics}
\end{entry}

\begin{entry}{黄瓜}{11,5}[Radicais ⿈、⽠]
  \begin{phonetics}{黄瓜}{huang2gua1}
    \definition[条]{s.}{pepino}
  \end{phonetics}
\end{entry}

\begin{entry}{黄色}{11,6}[Radicais ⿈、⾊]
  \begin{phonetics}{黄色}{huang2 se4}[][HSK 2]
    \definition{s.}{cor amarela}
  \end{phonetics}
\end{entry}

\begin{entry}{黄昏}{11,8}[Radicais ⿈、⽇]
  \begin{phonetics}{黄昏}{huang2hun1}
    \definition{s.}{anoitecer}
  \end{phonetics}
\end{entry}

\begin{entry}{黄河}{11,8}[Radicais ⿈、⽔]
  \begin{phonetics}{黄河}{huang2he2}
    \definition*{s.}{Rio Amarelo | Rio Huang He}
  \end{phonetics}
\end{entry}

\begin{entry}{黄油}{11,8}[Radicais ⿈、⽔]
  \begin{phonetics}{黄油}{huang2you2}
    \definition[盒]{s.}{manteiga}
  \end{phonetics}
\end{entry}

%%%%% EOF %%%%%

