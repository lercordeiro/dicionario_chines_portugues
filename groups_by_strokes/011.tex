%%%
%%% 11画
%%%

\section*{11画}\addcontentsline{toc}{section}{11画}

\begin{Entry}{假}{11}{⼈}
  \begin{Phonetics}{假}{jia3}[][HSK 2]
    \definition{adj.}{falso; artificial}
    \definition{conj.}{se; caso; no caso de; conecta frases, expressa relação hipotética, geralmente usada com 如, 若 e 使, equivalente a 如果}
    \definition[个,天]{s.}{falsificação; coisas falsas, irreais ou forjadas}
    \definition{v.}{emprestar | valer-se de; aproveitar; utilizar | supor; presumir; pressupor}
  \seealsoref{如}{ru2}
  \seealsoref{如果}{ru2guo3}
  \seealsoref{若}{ruo4}
  \seealsoref{使}{shi3}
  \end{Phonetics}
  \begin{Phonetics}{假}{jia4}
    \definition[个,天]{s.}{feriado; férias; período de suspensão temporária do trabalho ou dos estudos, legal ou aprovado | licença; afastamento temporário; período de licença temporária do trabalho ou dos estudos, após aprovação}
  \end{Phonetics}
\end{Entry}

\begin{Entry}{假日}{11,4}{⼈、⽇}
  \begin{Phonetics}{假日}{jia4 ri4}[][HSK 6]
    \definition[节]{s.}{feriado; dia de folga}
  \end{Phonetics}
\end{Entry}

\begin{Entry}{假如}{11,6}{⼈、⼥}
  \begin{Phonetics}{假如}{jia3ru2}[][HSK 4]
    \definition{conj.}{se; supondo; no caso}
  \end{Phonetics}
\end{Entry}

\begin{Entry}{假声}{11,7}{⼈、⼠}
  \begin{Phonetics}{假声}{jia3sheng1}
    \definition{s.}{falsete}
  \seealsoref{真声}{zhen1sheng1}
  \end{Phonetics}
\end{Entry}

\begin{Entry}{假证件}{11,7,6}{⼈、⾔、⼈}
  \begin{Phonetics}{假证件}{jia3zheng4jian4}
    \definition{s.}{documentos falsos}
  \end{Phonetics}
\end{Entry}

\begin{Entry}{假使}{11,8}{⼈、⼈}
  \begin{Phonetics}{假使}{jia3shi3}
    \definition{conj.}{se | supondo | em caso}
  \end{Phonetics}
\end{Entry}

\begin{Entry}{假的}{11,8}{⼈、⽩}
  \begin{Phonetics}{假的}{jia3de5}
    \definition{adj.}{falso | substituto | simulado}
  \end{Phonetics}
\end{Entry}

\begin{Entry}{假期}{11,12}{⼈、⽉}
  \begin{Phonetics}{假期}{jia4 qi1}[][HSK 2]
    \definition[个,段,次,种]{s.}{férias; feriados; período de licença}
  \end{Phonetics}
\end{Entry}

\begin{Entry}{偏}{11}{⼈}
  \begin{Phonetics}{偏}{pian1}[][HSK 6]
    \definition{adj.}{parcial; preconceituoso; injusto; focando apenas em um lado | torto; inclinado (oposto de 正) | não dominante; auxiliar | remoto; periférico; longe do centro; incomum}
    \definition{adv.}{intencionalmente; insistentemente; persistentemente; indica ir intencionalmente contra o senso comum ou a solicitação de outra pessoa}
    \definition{expr.}{uma expressão educada para indicar que alguém já tomou chá ou comeu}
    \definition{v.}{divergir; não ser igual a; ser diferente de; exceder ou ficar aquém dos padrões normais | desviar-se; afastar-se; sair na direção certa}
  \seealsoref{正}{zheng4}
  \end{Phonetics}
\end{Entry}

\begin{Entry}{偏偏}{11,11}{⼈、⼈}
  \begin{Phonetics}{偏偏}{pian1pian1}
    \definition{adv.}{voluntariamente | insistentemente | persistentemente | ao contrário da expectativa | infelizmente (indicando que alguma coisa aconteceu ao contrário do que se esperava) | teimosamente (indicando que algo é o oposto ao que seria normal ou razoável) | precisamente (indicando que alguém ou um grupo é escolhido)}
  \end{Phonetics}
\end{Entry}

\begin{Entry}{做}{11}{⼈}
  \begin{Phonetics}{做}{zuo4}[][HSK 1]
    \definition{v.}{fabricar; produzir; criar | escrever; compor | fazer; trabalhar em; dedicar-se a; exercer uma determinada profissão ou atividade | realizar uma festa em família; comemorar | ser; tornar-se; agir como; atuar como | ser usado como | formar ou estabelecer um relacionamento; conectar-se (em algum tipo de relação) | fingir (alguma coisa) | cozinhar; preparar}
  \end{Phonetics}
\end{Entry}

\begin{Entry}{做生活}{11,5,9}{⼈、⽣、⽔}
  \begin{Phonetics}{做生活}{zuo4sheng1huo2}
    \definition{v.}{fazer tabalhos manuais}
  \end{Phonetics}
\end{Entry}

\begin{Entry}{做戏}{11,6}{⼈、⼽}
  \begin{Phonetics}{做戏}{zuo4xi4}
    \definition{v.}{atuar em uma peça | fazer uma peça}
  \end{Phonetics}
\end{Entry}

\begin{Entry}{做作}{11,7}{⼈、⼈}
  \begin{Phonetics}{做作}{zuo4zuo5}
    \definition{adj.}{afetado | artificial}
  \end{Phonetics}
\end{Entry}

\begin{Entry}{做饭}{11,7}{⼈、⾷}
  \begin{Phonetics}{做饭}{zuo4 fan4}[][HSK 2]
    \definition{v.}{cozinhar; preparar uma refeição; cozinhar refeições e transformar alimentos crus em alimentos cozidos}
  \end{Phonetics}
\end{Entry}

\begin{Entry}{做到}{11,8}{⼈、⼑}
  \begin{Phonetics}{做到}{zuo4 dao4}[][HSK 2]
    \definition{v.}{alcançar; realizar; atingir um determinado objetivo; atingir um determinado padrão}
  \end{Phonetics}
\end{Entry}

\begin{Entry}{做法}{11,8}{⼈、⽔}
  \begin{Phonetics}{做法}{zuo4fa3}[][HSK 2]
    \definition[种,个]{s.}{método; maneira de fazer algo; métodos de lidar com coisas ou fazer coisas}
  \end{Phonetics}
\end{Entry}

\begin{Entry}{做客}{11,9}{⼈、⼧}
  \begin{Phonetics}{做客}{zuo4 ke4}[][HSK 3]
    \definition{v.}{visitar; ser um convidado; ser hóspede}
  \end{Phonetics}
\end{Entry}

\begin{Entry}{做活}{11,9}{⼈、⽔}
  \begin{Phonetics}{做活}{zuo4huo2}
    \definition{v.}{trabalhar para ganhar a vida (especialmente de mulher costureira)}
  \end{Phonetics}
\end{Entry}

\begin{Entry}{做梦}{11,11}{⼈、⼣}
  \begin{Phonetics}{做梦}{zuo4 meng4}[][HSK 4]
    \definition{s.}{sonho; ilusões e visões na consciência durante o sono}
    \definition{v.}{sonhar; ter um sonho | sonhar acordado, ter um sonho impossível; metáfora para fantasia irrealista}[别​做​梦​了​,她​不​会​嫁​给​你​的​。===Pare de sonhar, ela não se casará com você.]
  \end{Phonetics}
\end{Entry}

\begin{Entry}{做眼}{11,11}{⼈、⽬}
  \begin{Phonetics}{做眼}{zuo4yan3}
    \definition{v.}{agir como um guia | trabalhar como espião}
  \end{Phonetics}
\end{Entry}

\begin{Entry}{停}{11}{⼈}
  \begin{Phonetics}{停}{ting2}[][HSK 2]
    \definition{adj.}{pronto; resolvido; bem organizado}
    \definition{clas.}{usado para partes (de um total); porções}
    \definition{v.}{parar; interromper; cessar; fazer uma pausa | permanecer; ficar; fazer uma parada (para descansar) | estacionar; ancorar; atracar}
  \end{Phonetics}
\end{Entry}

\begin{Entry}{停下}{11,3}{⼈、⼀}
  \begin{Phonetics}{停下}{ting2 xia4}[][HSK 4]
    \definition{v.}{encerrar; desligar; parar}
  \end{Phonetics}
\end{Entry}

\begin{Entry}{停工}{11,3}{⼈、⼯}
  \begin{Phonetics}{停工}{ting2gong1}
    \definition{v.}{parar de trabalhar | parar a produção}
  \end{Phonetics}
\end{Entry}

\begin{Entry}{停办}{11,4}{⼈、⼒}
  \begin{Phonetics}{停办}{ting2ban4}
    \definition{v.}{cancelar | sair do negócio | desligar | terminar}
  \end{Phonetics}
\end{Entry}

\begin{Entry}{停止}{11,4}{⼈、⽌}
  \begin{Phonetics}{停止}{ting2 zhi3}[][HSK 3]
    \definition{v.}{parar; suspender; cessar; cancelar}
  \end{Phonetics}
\end{Entry}

\begin{Entry}{停火}{11,4}{⼈、⽕}
  \begin{Phonetics}{停火}{ting2/huo3}
    \definition{s.}{cessar-fogo}
    \definition{v.+compl.}{cessar fogo}
  \end{Phonetics}
\end{Entry}

\begin{Entry}{停车}{11,4}{⼈、⾞}
  \begin{Phonetics}{停车}{ting2 che1}[][HSK 2]
    \definition{v.}{(veículo) parar; frear | estacionar o veículo | parar; deixar de funcionar}
  \end{Phonetics}
\end{Entry}

\begin{Entry}{停车场}{11,4,6}{⼈、⾞、⼟}
  \begin{Phonetics}{停车场}{ting2 che1 chang3}[][HSK 2]
    \definition[个]{s.}{estacionamento; área de estacionamento; local para estacionamento de veículos}
  \end{Phonetics}
\end{Entry}

\begin{Entry}{停业}{11,5}{⼈、⼀}
  \begin{Phonetics}{停业}{ting2ye4}
    \definition{v.}{cessar a negociação (temporária ou permanentemente) | fechar}
  \end{Phonetics}
\end{Entry}

\begin{Entry}{停用}{11,5}{⼈、⽤}
  \begin{Phonetics}{停用}{ting2yong4}
    \definition{v.}{desabilitar | descontinuar | parar de usar | suspender}
  \end{Phonetics}
\end{Entry}

\begin{Entry}{停电}{11,5}{⼈、⽥}
  \begin{Phonetics}{停电}{ting2dian4}
    \definition{s.}{corte de energia}
    \definition{v.}{ter uma falha de energia}
  \end{Phonetics}
\end{Entry}

\begin{Entry}{停当}{11,6}{⼈、⼹}
  \begin{Phonetics}{停当}{ting2dang5}
    \definition{adj.}{realizado | preparado | assentado}
  \end{Phonetics}
\end{Entry}

\begin{Entry}{停息}{11,10}{⼈、⼼}
  \begin{Phonetics}{停息}{ting2xi1}
    \definition{v.}{cessar | parar}
  \end{Phonetics}
\end{Entry}

\begin{Entry}{停留}{11,10}{⼈、⽥}
  \begin{Phonetics}{停留}{ting2 liu2}[][HSK 5]
    \definition{v.}{permanecer; ficar por muito tempo; parar temporariamente em algum lugar, sem continuar avançando | permanecer; parar por um longo tempo; parar em um determinado estágio ou nível, sem evoluir}
  \end{Phonetics}
\end{Entry}

\begin{Entry}{停课}{11,10}{⼈、⾔}
  \begin{Phonetics}{停课}{ting2ke4}
    \definition{v.}{fechar (escola) | parar as aulas}
  \end{Phonetics}
\end{Entry}

\begin{Entry}{停歇}{11,13}{⼈、⽋}
  \begin{Phonetics}{停歇}{ting2xie1}
    \definition{v.}{parar para descansar}
  \end{Phonetics}
\end{Entry}

\begin{Entry}{偶}{11}{⼈}
  \begin{Phonetics}{偶}{ou3}
    \definition{adv.}{por acaso; por acidente; de vez em quando; ocasionalmente | par; número par; pareado (em oposição a 奇)}
    \definition{s.}{imagem; ídolo; figuras feitas de madeira, barro, etc. | companheiro; cônjuge; parceiro; refere-se a um casal ou a um dos casais}
  \seealsoref{奇}{qi2}
  \end{Phonetics}
\end{Entry}

\begin{Entry}{偶尔}{11,5}{⼈、⼩}
  \begin{Phonetics}{偶尔}{ou3'er3}[][HSK 5]
    \definition{adj.}{ocasional}
    \definition{adv.}{ocasionalmente; de vez em quando; às vezes}
  \end{Phonetics}
\end{Entry}

\begin{Entry}{偶然}{11,12}{⼈、⽕}
  \begin{Phonetics}{偶然}{ou3ran2}[][HSK 5]
    \definition{adj.}{acidental; ocasional}
    \definition{adv.}{por acaso; acidentalmente; sem querer; inesperadamente | ocasionalmente; de vez em quando; às vezes}
  \end{Phonetics}
\end{Entry}

\begin{Entry}{偶像}{11,13}{⼈、⼈}
  \begin{Phonetics}{偶像}{ou3xiang4}[][HSK 5]
    \definition[位,个,名]{s.}{ídolo; pessoa amada pelas pessoas; refere-se a uma pessoa que é apreciada por todos e que, em certos aspectos, é digna de admiração e respeito}
  \end{Phonetics}
\end{Entry}

\begin{Entry}{偷}{11}{⼈}
  \begin{Phonetics}{偷}{tou1}[][HSK 5]
    \definition{adv.}{furtivamente; secretamente; às escondidas}
    \definition{s.}{ladrão; furtador}
    \definition{v.}{roubar; furtar; levar sem pagar; roubar os bens alheios às escondidas | encontrar (tempo) | deixar-se levar; viver apenas para o presente, sem se preocupar com o futuro}
  \end{Phonetics}
\end{Entry}

\begin{Entry}{偷安}{11,6}{⼈、⼧}
  \begin{Phonetics}{偷安}{tou1'an1}
    \definition{v.}{buscar facilidade temporária}
  \end{Phonetics}
\end{Entry}

\begin{Entry}{偷听}{11,7}{⼈、⼝}
  \begin{Phonetics}{偷听}{tou1ting1}
    \definition{v.}{bisbilhotar; monitorar (secretamente)}
  \end{Phonetics}
\end{Entry}

\begin{Entry}{偷窃}{11,9}{⼈、⽳}
  \begin{Phonetics}{偷窃}{tou1qie4}
    \definition{v.}{furtar | roubar}
  \end{Phonetics}
\end{Entry}

\begin{Entry}{偷偷}{11,11}{⼈、⼈}
  \begin{Phonetics}{偷偷}{tou1 tou1}[][HSK 5]
    \definition{adv.}{secretamente; dissimuladamente; furtivamente; às escondidas; descreve uma ação que não é notada pelos outros; em segredo ou em privado, não revelada}
  \end{Phonetics}
\end{Entry}

\begin{Entry}{偷情}{11,11}{⼈、⼼}
  \begin{Phonetics}{偷情}{tou1qing2}
    \definition{v.}{manter um caso de amor clandestino}
  \end{Phonetics}
\end{Entry}

\begin{Entry}{偷袭}{11,11}{⼈、⾐}
  \begin{Phonetics}{偷袭}{tou1xi2}
    \definition{s.}{ataque surpresa}
    \definition{v.}{montar um ataque furtivo | invadir}
  \end{Phonetics}
\end{Entry}

\begin{Entry}{偷渡}{11,12}{⼈、⽔}
  \begin{Phonetics}{偷渡}{tou1du4}
    \definition{s.}{contrabando | imigração ilegal | clandestino (em um navio)}
    \definition{v.}{executar um bloqueio | roubar através da fronteira internacional}
  \end{Phonetics}
\end{Entry}

\begin{Entry}{偷税}{11,12}{⼈、⽲}
  \begin{Phonetics}{偷税}{tou1shui4}
    \definition{s.}{evasão fiscal}
  \end{Phonetics}
\end{Entry}

\begin{Entry}{偸}{11}{⼈}
  \begin{Phonetics}{偸}{tou1}
    \variantof{偷}
  \end{Phonetics}
\end{Entry}

\begin{Entry}{偿}{11}{⼈}
  \begin{Phonetics}{偿}{chang2}
    \definition{v.}{reembolsar; compensar | realizar; cumprir | cumprir; satisfazer}
  \end{Phonetics}
\end{Entry}

\begin{Entry}{偿还}{11,7}{⼈、⾡}
  \begin{Phonetics}{偿还}{chang2huan2}[][HSK 7-9]
    \definition{v.}{reembolsar; pagar de volta; pagar (uma dívida)}
  \end{Phonetics}
\end{Entry}

\begin{Entry}{兜}{11}{⼉}
  \begin{Phonetics}{兜}{dou1}[][HSK 7-9]
    \definition{s.}{bolsa; bolso; coisas tipo bolso}
    \definition{v.}{embrulhar em um pedaço de pano, etc.; fazer um formato de bolso para guardar coisas | mover-se; dar uma volta; fazer um desvio | solicitar; sondar; recrutamentar | assumir a responsabilidade por algo; assumir o controle de}
  \end{Phonetics}
\end{Entry}

\begin{Entry}{兜儿}{11,2}{⼉、⼉}
  \begin{Phonetics}{兜儿}{dou1r5}[][HSK 7-9]
    \definition{s.}{bolso}
  \end{Phonetics}
\end{Entry}

\begin{Entry}{兜售}{11,11}{⼉、⼝}
  \begin{Phonetics}{兜售}{dou1shou4}[][HSK 7-9]
    \definition{v.}{vender; apregoar | vender; fazer uma venda de}
  \end{Phonetics}
\end{Entry}

\begin{Entry}{兽}{11}{⼋}
  \begin{Phonetics}{兽}{shou4}
    \definition{adj.}{bestial; brutal}
    \definition{s.}{besta; animal}
  \end{Phonetics}
\end{Entry}

\begin{Entry}{兽力车}{11,2,4}{⼋、⼒、⾞}
  \begin{Phonetics}{兽力车}{shou4 li4 che1}
    \definition{s.}{veículo puxado por animais  (oposto a 人力车) | carruagem; carroça}
  \seealsoref{人力车}{ren2 li4 che1}
  \end{Phonetics}
\end{Entry}

\begin{Entry}{兽行}{11,6}{⼋、⾏}
  \begin{Phonetics}{兽行}{shou4xing2}
    \definition{s.}{ato brutal; brutalidade | bestialidade}
  \end{Phonetics}
\end{Entry}

\begin{Entry}{减}{11}{⼎}
  \begin{Phonetics}{减}{jian3}[][HSK 4]
    \definition*{s.}{Sobrenome Jian}
    \definition{v.}{subtrair; remover uma parte da quantidade original | reduzir; diminuir; cortar}
  \end{Phonetics}
\end{Entry}

\begin{Entry}{减少}{11,4}{⼎、⼩}
  \begin{Phonetics}{减少}{jian3shao3}[][HSK 4]
    \definition{v.}{cair; reduzir; diminuir; subtrair uma parte}
  \end{Phonetics}
\end{Entry}

\begin{Entry}{减肥}{11,8}{⼎、⾁}
  \begin{Phonetics}{减肥}{jian3/fei2}[][HSK 4]
    \definition{v.+compl.}{perder peso; dieta, exercícios, medicamentos, massagem, cirurgia, etc., para reduzir o excesso de gordura corporal, de modo que o grau de obesidade seja reduzido}
  \end{Phonetics}
\end{Entry}

\begin{Entry}{减轻}{11,9}{⼎、⾞}
  \begin{Phonetics}{减轻}{jian3 qing1}[][HSK 5]
    \definition{v.}{aliviar; remeter; clarear; facilitar; mitigar}
  \end{Phonetics}
\end{Entry}

\begin{Entry}{凑}{11}{⼎}
  \begin{Phonetics}{凑}{cou4}[][HSK 7-9]
    \definition{v.}{reunir; coletar; ajuntar | acontecer por acaso; aproveitar; esbarrar em; alcançar; tirar vantagem de | aproximar; mover-se para perto de}
  \end{Phonetics}
\end{Entry}

\begin{Entry}{凑巧}{11,5}{⼎、⼯}
  \begin{Phonetics}{凑巧}{cou4qiao3}[][HSK 7-9]
    \definition{adj.}{afortunado; sortudo; coincidente; significa que é o momento certo ou que algo que você quer ou não quer está acontecendo}
  \end{Phonetics}
\end{Entry}

\begin{Entry}{凑合}{11,6}{⼎、⼝}
  \begin{Phonetics}{凑合}{cou4he5}[][HSK 7-9]
    \definition{v.}{contentar-se com algo; ser razoável; ser razoavelmente bom, mas não excelente; aceitar relutantemente coisas ou condições de um nível ou grau inferior | improvisar | reunir}
  \end{Phonetics}
\end{Entry}

\begin{Entry}{凰}{11}{⼏}
  \begin{Phonetics}{凰}{huang2}
    \definition[只]{s.}{Mitologia: fênix fêmea}
  \end{Phonetics}
\end{Entry}

\begin{Entry}{剪}{11}{⼑}
  \begin{Phonetics}{剪}{jian3}[][HSK 5]
    \definition[把]{s.}{tesouras; tesouras de poda; cortadores | pinças; tenazes}
    \definition{v.}{cortar; aparar; tosquiar; cortar (com uma tesoura) | exterminar; eliminar; acabar com}
  \end{Phonetics}
\end{Entry}

\begin{Entry}{剪刀}{11,2}{⼑、⼑}
  \begin{Phonetics}{剪刀}{jian3dao1}[][HSK 5]
    \definition[把,个]{s.}{tesoura; tesoura de jardim; instrumento de ferro para cortar tecido, papel, barbante, etc., com duas lâminas interligadas que podem ser abertas e fechadas}
  \end{Phonetics}
\end{Entry}

\begin{Entry}{剪子}{11,3}{⼑、⼦}
  \begin{Phonetics}{剪子}{jian3 zi5}[][HSK 5]
    \definition[把]{s.}{tesouras; tesouras de podar; tosquiadeiras}
  \end{Phonetics}
\end{Entry}

\begin{Entry}{副}{11}{⼑}
  \begin{Phonetics}{副}{fu4}[][HSK 6]
    \definition{adj.}{segundo em exercício; deputado; auxiliar | subsidiário; incidental; secundário}
    \definition{clas.}{usado para conjuntos completos de itens; usado para \emph{kits} | usado para expressões faciais | usado para som ou voz}
    \definition{pref.}{vice-}
    \definition{s.}{assistente; ajudante; auxiliar; posição auxiliar; pessoa que ocupa uma posição auxiliar}
    \definition{v.}{ajustar; corresponder a; conformar-se a}
  \end{Phonetics}
\end{Entry}

\begin{Entry}{副作用}{11,7,5}{⼑、⼈、⽤}
  \begin{Phonetics}{副作用}{fu4zuo4yong4}[][HSK 7-9]
    \definition{s.}{efeito colateral; efeitos adversos além dos efeitos principais}
  \end{Phonetics}
\end{Entry}

\begin{Entry}{副研}{11,9}{⼑、⽯}
  \begin{Phonetics}{副研}{fu4yan2}
    \definition{s.}{pesquisador adjunto}
  \end{Phonetics}
\end{Entry}

\begin{Entry}{唬}{11}{⼝}
  \begin{Phonetics}{唬}{hu3}
    \definition{v.}{blefar, exagerar para assustar ou confundir}
  \end{Phonetics}
\end{Entry}

\begin{Entry}{售}{11}{⼝}
  \begin{Phonetics}{售}{shou4}
    \definition{v.}{vender | fazer (o plano, truque, etc.) funcionar; continuar (as intrigas) | realizar (intrigas)}
  \end{Phonetics}
\end{Entry}

\begin{Entry}{售货员}{11,8,7}{⼝、⾙、⼝}
  \begin{Phonetics}{售货员}{shou4huo4yuan2}[][HSK 4]
    \definition[名,位]{s.}{vendedor; balconista; assistente de loja; equipe que vende produtos em lojas}
  \end{Phonetics}
\end{Entry}

\begin{Entry}{唯}{11}{⼝}
  \begin{Phonetics}{唯}{wei2}
    \definition{adv.}{somente; sozinho | ainda; somente; exceto que}
  \end{Phonetics}
  \begin{Phonetics}{唯}{wei3}
    \definition{interj.}{Sim!; Yea!; significa uma palavra que indica acordo}
  \end{Phonetics}
\end{Entry}

\begin{Entry}{唯一}{11,1}{⼝、⼀}
  \begin{Phonetics}{唯一}{wei2yi1}[][HSK 5]
    \definition{adj.}{único; exclusivo; singular; apenas um; nenhum outro}
  \end{Phonetics}
\end{Entry}

\begin{Entry}{唱}{11}{⼝}
  \begin{Phonetics}{唱}{chang4}[][HSK 1]
    \definition*{s.}{Sobrenome Chang}
    \definition{s.}{uma música ou uma parte cantada de uma ópera chinesa; canções; letras de óperas tradicionais}
    \definition{v.}{cantar; seguir o ritmo da música | chorar; chamar; gritar, falar ou recitar em voz alta}
  \end{Phonetics}
\end{Entry}

\begin{Entry}{唱片}{11,4}{⼝、⽚}
  \begin{Phonetics}{唱片}{chang4 pian4}[][HSK 4]
    \definition[枚,张]{s.}{disco; disco feito de goma-laca, plástico, etc. com ranhuras em espiral na superfície para registrar alterações no som que podem reproduzir o som gravado em um fonógrafo}
  \end{Phonetics}
\end{Entry}

\begin{Entry}{唱歌}{11,14}{⼝、⽋}
  \begin{Phonetics}{唱歌}{chang4/ge1}[][HSK 1]
    \definition{v.+compl.}{cantar (uma música); emitir sons com entonação ritmada e melodiosa; emitir sons (musicais) com a boca; emitir sons de acordo com a melodia}
  \end{Phonetics}
\end{Entry}

\begin{Entry}{唾}{11}{⼝}
  \begin{Phonetics}{唾}{tuo4}
    \definition[口]{s.}{saliva; cuspe}
    \definition{v.}{cuspir (mostrar desprezo) | rejeitar}
  \end{Phonetics}
\end{Entry}

\begin{Entry}{唾骂}{11,9}{⼝、⾺}
  \begin{Phonetics}{唾骂}{tuo4ma4}
    \definition{v.}{insultar | amaldiçoar}
  \end{Phonetics}
\end{Entry}

\begin{Entry}{商}{11}{⼝}
  \begin{Phonetics}{商}{shang1}
    \definition*{s.}{Dinastia Shang (1600-1046 a.C.) | Shang, nome da estrela da constelação do coração entre as vinte e oito constelações | Sobrenome Shang}
    \definition{s.}{comércio; negócio; a atividade econômica de compra e venda de mercadorias | comerciante; negociante; comerciante; empresário; pessoas que compram e vendem mercadorias | (matemática) quociente;  o resultado de uma operação de divisão em aritmética | uma nota da antiga escala chinesa de cinco tons, correspondente a 2 na notação musical numerada}
    \definition{v.}{discutir; consultar; trocar ideias}
  \end{Phonetics}
\end{Entry}

\begin{Entry}{商人}{11,2}{⼝、⼈}
  \begin{Phonetics}{商人}{shang1 ren2}[][HSK 2]
    \definition[位,名]{s.}{comerciante; mercador; empresário; homem de negócios; pessoas que trabalham com a distribuição de mercadorias}
  \end{Phonetics}
\end{Entry}

\begin{Entry}{商业}{11,5}{⼝、⼀}
  \begin{Phonetics}{商业}{shang1ye4}[][HSK 3]
    \definition[个,种]{s.}{barganha; negócio; comércio; atividade econômica que circula mercadorias por meio de compra e venda}
  \end{Phonetics}
\end{Entry}

\begin{Entry}{商务}{11,5}{⼝、⼒}
  \begin{Phonetics}{商务}{shang1wu4}[][HSK 4]
    \definition[种,类,项]{s.}{negócios; assuntos de negócios; assuntos comerciais}
  \end{Phonetics}
\end{Entry}

\begin{Entry}{商场}{11,6}{⼝、⼟}
  \begin{Phonetics}{商场}{shang1 chang3}[][HSK 1]
    \definition[家]{s.}{mercado; shopping center; loja de departamentos; loja de grande área com uma variedade completa de produtos | o mundo dos negócios; referindo-se ao mundo dos negócios | mercado; mercado composto por várias lojas reunidas em um ou vários edifícios interligados}
  \end{Phonetics}
\end{Entry}

\begin{Entry}{商店}{11,8}{⼝、⼴}
  \begin{Phonetics}{商店}{shang1dian4}[][HSK 1]
    \definition[间,家,个]{s.}{loja; armazém; local de venda de mercadorias em recinto fechado}
  \end{Phonetics}
\end{Entry}

\begin{Entry}{商品}{11,9}{⼝、⼝}
  \begin{Phonetics}{商品}{shang1pin3}[][HSK 3]
    \definition[种,个,件,批]{s.}{bens; mercadoria; \emph{merchande}; os produtos do trabalho produzidos para troca têm a dupla natureza de valor de uso e valor; as mercadorias incorporam diferentes relações de produção em diferentes sistemas sociais}
  \end{Phonetics}
\end{Entry}

\begin{Entry}{商城}{11,9}{⼝、⼟}
  \begin{Phonetics}{商城}{shang1 cheng2}[][HSK 6]
    \definition{s.}{um mercado; um centro comercial; um \emph{shopping center}; refere-se a um complexo comercial contíguo com um grande espaço de construção}
  \end{Phonetics}
\end{Entry}

\begin{Entry}{商标}{11,9}{⼝、⽊}
  \begin{Phonetics}{商标}{shang1biao1}[][HSK 5]
    \definition[个]{s.}{marca; marca registrada; \emph{trademark}; marca ou símbolo (desenho, padrão, texto, etc.) gravado ou impresso na superfície ou embalagem de um produto, para diferenciá-lo de outros produtos semelhantes}
  \end{Phonetics}
\end{Entry}

\begin{Entry}{商贸}{11,9}{⼝、⾙}
  \begin{Phonetics}{商贸}{shang1mao4}
    \definition{s.}{comércio}
  \end{Phonetics}
\end{Entry}

\begin{Entry}{商量}{11,12}{⼝、⾥}
  \begin{Phonetics}{商量}{shang1liang5}[][HSK 2]
    \definition{v.}{consultar; discutir; conversar sobre; discutir e trocar opiniões}
  \end{Phonetics}
\end{Entry}

\begin{Entry}{啤}{11}{⼝}
  \begin{Phonetics}{啤}{pi2}
    \definition{s.}{cerveja}
  \end{Phonetics}
\end{Entry}

\begin{Entry}{啤酒}{11,10}{⼝、⾣}
  \begin{Phonetics}{啤酒}{pi2jiu3}[][HSK 3]
    \definition[杯,瓶,罐,桶,缸]{s.}{(empréstimo linguístico) cerveja; uma bebida de baixo teor alcoólico feita de malte de cevada e lúpulo, com espuma e aroma especial}
  \end{Phonetics}
\end{Entry}

\begin{Entry}{啤酒馆}{11,10,11}{⼝、⾣、⾷}
  \begin{Phonetics}{啤酒馆}{pi2jiu3guan3}
    \definition{s.}{cervejaria}
  \end{Phonetics}
\end{Entry}

\begin{Entry}{啥}{11}{⼝}
  \begin{Phonetics}{啥}{sha2}
    \definition{pron.}{Dialeto: O que?; equivalente a 什么}
  \end{Phonetics}
\end{Entry}

\begin{Entry}{啦}{11}{⼝}
  \begin{Phonetics}{啦}{la1}
    \definition{s.}{(onomatoméia) som de canto, aplausos etc.; usado para palavras como 呼啦啦, 哗啦啦, 哩哩啦啦, etc.}
  \seealsoref{呼啦啦}{hu1 la1 la1}
  \seealsoref{哗啦啦}{hua1la1 la5}
  \seealsoref{哩哩啦啦}{li1 li1 la1 la1}
  \end{Phonetics}
  \begin{Phonetics}{啦}{la5}[][HSK 6]
    \definition{part.}{uma palavra composta de 了 e 啊, que tem o significado de ambos}
  \seealsoref{啊}{a5}
  \seealsoref{了}{le5}
  \end{Phonetics}
\end{Entry}

\begin{Entry}{啵}{11}{⼝}
  \begin{Phonetics}{啵}{bo1}
    \definition{part.}{denotando pedido, comando, etc.; o uso é semelhante ao de 吧, que é mais comum no vernáculo antigo}
    \definition{v.aux.}{indicando uma sugestão, pedido ou comando suave | indicando consentimento ou aprovação | em uma pergunta tendenciosa que pede a confirmação de uma suposição | indicando alguma dúvida na mente do falante | marcando uma pausa após suposições como alternativas}
  \seealsoref{吧}{ba5}
  \end{Phonetics}
  \begin{Phonetics}{啵}{bo5}
    \definition{part.}{partícula gramaticalmente equivalente a 吧}
  \seealsoref{吧}{ba5}
  \end{Phonetics}
\end{Entry}

\begin{Entry}{圈}{11}{⼞}
  \begin{Phonetics}{圈}{juan1}
    \definition{v.}{prender aves e animais de criação | prender; colocar na cadeia, prisão | confinar}
  \end{Phonetics}
  \begin{Phonetics}{圈}{juan4}
    \definition*{s.}{Sobrenome Juan}
    \definition{s.}{curral; local onde o gado ou as aves são mantidos, geralmente cercado ou murado, alguns com galpões}
  \end{Phonetics}
  \begin{Phonetics}{圈}{quan1}[][HSK 4]
    \definition[个]{s.}{anel; círculo; refere-se a algo em forma de anel | domínio; grupo; escopo; círculo(s)}
    \definition{v.}{cercar; rodear; circundar | marcar com um círculo}
  \end{Phonetics}
\end{Entry}

\begin{Entry}{圈粉}{11,10}{⼞、⽶}
  \begin{Phonetics}{圈粉}{quan1fen3}
    \definition{s.}{(neologismo, coloquial) ganhar alguém como fã, obter novos fãs}
  \end{Phonetics}
\end{Entry}

\begin{Entry}{埦}{11}{⼟}
  \begin{Phonetics}{埦}{wan3}
    \variantof{碗}
  \end{Phonetics}
\end{Entry}

\begin{Entry}{培}{11}{⼟}
  \begin{Phonetics}{培}{pei2}
    \definition{v.}{aterrar com terra; aterrar | fomentar; treinar | cultivar; crescer e desenvolver-se propositalmente}
  \end{Phonetics}
\end{Entry}

\begin{Entry}{培训}{11,5}{⼟、⾔}
  \begin{Phonetics}{培训}{pei2xun4}[][HSK 4]
    \definition{v.}{treinar (trabalhadores técnicos, quadros profissionais, etc.)}
  \end{Phonetics}
\end{Entry}

\begin{Entry}{培训班}{11,5,10}{⼟、⾔、⽟}
  \begin{Phonetics}{培训班}{pei2 xun4 ban1}[][HSK 4]
    \definition{s.}{aula de treinamento; curso de treinamento}
  \end{Phonetics}
\end{Entry}

\begin{Entry}{培育}{11,8}{⼟、⾁}
  \begin{Phonetics}{培育}{pei2yu4}[][HSK 4]
    \definition{v.}{criar; fomentar; educar; procriar; nutrir; cultivar}
  \end{Phonetics}
\end{Entry}

\begin{Entry}{培养}{11,9}{⼟、⼋}
  \begin{Phonetics}{培养}{pei2yang3}[][HSK 4]
    \definition{v.}{cultivar (plantas, microorganismos) | promover; treinar ou desenvolver; educar e treinar para um determinado propósito durante um longo período de tempo; fazer crescer | progredir gradualmente; desenvolver ou cultivar gradualmente (hábito, qualidade, caráter, emoção, estilo, interesse, habilidade, etc.)}
  \end{Phonetics}
\end{Entry}

\begin{Entry}{基}{11}{⼟}
  \begin{Phonetics}{基}{ji1}
    \definition{adj.}{chave; básico; primário; cardinal; fundamental}
    \definition{s.}{base; fundação | base; grupo; radical; (química) uma parte dos átomos contidos na molécula de um composto, quando considerada como uma unidade, é chamada de base}
  \end{Phonetics}
\end{Entry}

\begin{Entry}{基本}{11,5}{⼟、⽊}
  \begin{Phonetics}{基本}{ji1ben3}[][HSK 3]
    \definition{adj.}{básico; fundamental; elementar | principal}
    \definition{adv.}{basicamente; em geral; no geral; em termos gerais}
    \definition{s.}{fundação}
  \end{Phonetics}
\end{Entry}

\begin{Entry}{基本上}{11,5,3}{⼟、⽊、⼀}
  \begin{Phonetics}{基本上}{ji1 ben3 shang4}[][HSK 3]
    \definition{adv.}{basicamente; principalmente | em geral; de modo geral}
  \end{Phonetics}
\end{Entry}

\begin{Entry}{基本功}{11,5,5}{⼟、⽊、⼒}
  \begin{Phonetics}{基本功}{ji1ben3gong1}
    \definition{s.}{treinamento básico; habilidade básica; técnica essencial}
  \end{Phonetics}
\end{Entry}

\begin{Entry}{基本法}{11,5,8}{⼟、⽊、⽔}
  \begin{Phonetics}{基本法}{ji1ben3fa3}
    \definition{s.}{lei básica (constituição)}
  \end{Phonetics}
\end{Entry}

\begin{Entry}{基因}{11,6}{⼟、⼞}
  \begin{Phonetics}{基因}{ji1yin1}
    \definition{s.}{gene}
  \end{Phonetics}
\end{Entry}

\begin{Entry}{基地}{11,6}{⼟、⼟}
  \begin{Phonetics}{基地}{ji1di4}[][HSK 5]
    \definition{s.}{base; como base para alguns negócios | base; um local dedicado à realização de um negócio}
  \end{Phonetics}
\end{Entry}

\begin{Entry}{基金}{11,8}{⼟、⾦}
  \begin{Phonetics}{基金}{ji1jin1}[][HSK 5]
    \definition[只,笔]{s.}{fundo; fundos reservados ou destinados ao estabelecimento ou desenvolvimento de uma empresa}
  \end{Phonetics}
\end{Entry}

\begin{Entry}{基础}{11,10}{⼟、⽯}
  \begin{Phonetics}{基础}{ji1chu3}[][HSK 3]
    \definition[个,种,点,层]{s.}{base; fundamento; fundação; a essência ou o ponto de partida do desenvolvimento das coisas | básico; fundamental; refere-se às condições mínimas | fundação do edifício; base do edifício}
  \end{Phonetics}
\end{Entry}

\begin{Entry}{基督教}{11,13,11}{⼟、⽬、⽁}
  \begin{Phonetics}{基督教}{ji1 du1 jiao4}[][HSK 6]
    \definition*{s.}{Cristianismo; A Religião Cristã | Cristão}
  \end{Phonetics}
\end{Entry}

\begin{Entry}{堆}{11}{⼟}
  \begin{Phonetics}{堆}{dui1}[][HSK 5]
    \definition{clas.}{amontoado; pilha; multidão; usado para pilhas de coisas}
    \definition{s.}{amontoado; pilha; empilhamento | (em nomes de lugares)  colina; monte| multidão de pessoas ou coisas}
    \definition{v.}{empilhar; amontoar; acumular; juntar; reunir}
  \end{Phonetics}
\end{Entry}

\begin{Entry}{堆砌}{11,9}{⼟、⽯}
  \begin{Phonetics}{堆砌}{dui1qi4}[][HSK 7-9]
    \definition{v.}{Figurativo: preencher (escrever com frases elaboradas) | Literário: empilhar (tijolos) | embalar}
  \end{Phonetics}
\end{Entry}

\begin{Entry}{堕}{11}{⼟}
  \begin{Phonetics}{堕}{duo4}
    \definition{v.}{cair; afundar}
  \end{Phonetics}
\end{Entry}

\begin{Entry}{堕落}{11,12}{⼟、⾋}
  \begin{Phonetics}{堕落}{duo4luo4}[][HSK 7-9]
    \definition{adj.}{corrupto; decadente}
    \definition{v.}{cair; afundar; degenerar; deteriorar}
  \end{Phonetics}
\end{Entry}

\begin{Entry}{堵}{11}{⼟}
  \begin{Phonetics}{堵}{du3}[][HSK 4]
    \definition*{s.}{Sobrenome Du}
    \definition{adj.}{asfixiado; abafado; sufocado; oprimido}
    \definition{clas.}{usado para paredes}
    \definition{s.}{parede}
    \definition{v.}{impedir; bloquear}
  \end{Phonetics}
\end{Entry}

\begin{Entry}{堵车}{11,4}{⼟、⾞}
  \begin{Phonetics}{堵车}{du3/che1}[][HSK 4]
    \definition{s.}{congestionamento; tráfego intenso; ficar congestionado (no tráfego); bloqueio de vias devido ao excesso de tráfego, etc.}
    \definition{v.+compl.}{congestionar (trânsito)}
  \end{Phonetics}
\end{Entry}

\begin{Entry}{堵塞}{11,13}{⼟、⼟}
  \begin{Phonetics}{堵塞}{du3se4}[][HSK 7-9]
    \definition{v.}{parar; bloquear; tornar obstruído}
  \end{Phonetics}
\end{Entry}

\begin{Entry}{够}{11}{⼣}
  \begin{Phonetics}{够}{gou4}[][HSK 2]
    \definition{adj.}{suficiente; adequado; apropriado; atingir e ultrapassar um determinado limite, difícil de suportar}
    \definition{adv.}{suficientemente; o suficiente (para atingir um determinado nível); indica que atingiu um determinado padrão ou nível elevado}
    \definition{v.}{alcançar (algo, esticando-se); (usando membros, etc.) esticar-se para alcançar ou tocar em locais de difícil acesso | atingir (um padrão ou nível); satisfazer ou atingir a quantidade, os padrões, etc. necessários}
  \end{Phonetics}
\end{Entry}

\begin{Entry}{够不着}{11,4,11}{⼣、⼀、⽬}
  \begin{Phonetics}{够不着}{gou4bu5zhao2}
    \definition{v.}{ser incapaz de alcançar}
  \end{Phonetics}
\end{Entry}

\begin{Entry}{够本}{11,5}{⼣、⽊}
  \begin{Phonetics}{够本}{gou4ben3}
    \definition{v.}{empatar | fazer valer o dinheiro}
  \end{Phonetics}
\end{Entry}

\begin{Entry}{够呛}{11,7}{⼣、⼝}
  \begin{Phonetics}{够呛}{gou4qiang4}[][HSK 7-9]
    \definition{adj.}{terrível; insuportável; descreve uma situação extremamente grave e insuportável | improvável; bastante improvável; quase impossível; descreve como difícil de alcançar}
  \end{Phonetics}
\end{Entry}

\begin{Entry}{够味}{11,8}{⼣、⼝}
  \begin{Phonetics}{够味}{gou4wei4}
    \definition{adj.}{excelente | na medida}
  \end{Phonetics}
\end{Entry}

\begin{Entry}{够戗}{11,8}{⼣、⼽}
  \begin{Phonetics}{够戗}{gou4qiang4}
    \variantof{够呛}
  \end{Phonetics}
\end{Entry}

\begin{Entry}{够朋友}{11,8,4}{⼣、⽉、⼜}
  \begin{Phonetics}{够朋友}{gou4peng2you5}
    \definition{v.}{ser um amigo verdadeiro}
  \end{Phonetics}
\end{Entry}

\begin{Entry}{够格}{11,10}{⼣、⽊}
  \begin{Phonetics}{够格}{gou4ge2}
    \definition{adj.}{apto | qualificado | apresentável}
  \end{Phonetics}
\end{Entry}

\begin{Entry}{够得着}{11,11,11}{⼣、⼻、⽬}
  \begin{Phonetics}{够得着}{gou4de5zhao2}
    \definition{v.}{estar à altura | alcançar}
  \end{Phonetics}
\end{Entry}

\begin{Entry}{婚}{11}{⼥}
  \begin{Phonetics}{婚}{hun1}
    \definition{s.}{casamento}
    \definition{v.}{casar}
  \end{Phonetics}
\end{Entry}

\begin{Entry}{婚礼}{11,5}{⼥、⽰}
  \begin{Phonetics}{婚礼}{hun1li3}[][HSK 4]
    \definition[场]{s.}{casamento; núpcias; cerimônia de casamento}
  \end{Phonetics}
\end{Entry}

\begin{Entry}{宿}{11}{⼧}
  \begin{Phonetics}{宿}{su4}
    \definition*{s.}{Sobrenome Su}
    \definition{adj.}{de longa data; antigo; velho | veterano; velho; experiente}
    \definition{v.}{hospedar-se para passar a noite; passar a noite}
  \end{Phonetics}
  \begin{Phonetics}{宿}{xiu3}
    \definition{s.}{usado para calcular a noite}[谈了半宿。===Conversamos por metade da noite.]
  \end{Phonetics}
  \begin{Phonetics}{宿}{xiu4}
    \definition{s.}{(astronomia) um termo antigo para constelação}
  \end{Phonetics}
\end{Entry}

\begin{Entry}{宿舍}{11,8}{⼧、⾆}
  \begin{Phonetics}{宿舍}{su4she4}[][HSK 5]
    \definition[间,幢]{s.}{alojamento; dormitório; república; albergue; casas onde escolas, empresas, etc. acomodam seus alunos ou funcionários}
  \end{Phonetics}
\end{Entry}

\begin{Entry}{寂}{11}{⼧}
  \begin{Phonetics}{寂}{ji4}
    \definition{adj.}{quieto; parado; silencioso | solitário}
  \end{Phonetics}
\end{Entry}

\begin{Entry}{寂寞}{11,13}{⼧、⼧}
  \begin{Phonetics}{寂寞}{ji4mo4}
    \definition{adj.}{sozinho | solitário | (de um lugar) silencioso}
  \end{Phonetics}
\end{Entry}

\begin{Entry}{寂寥}{11,14}{⼧、⼧}
  \begin{Phonetics}{寂寥}{ji4liao2}
    \definition{s.}{solidão | vasto e vazio | quieto e desolado (literário)}
  \end{Phonetics}
\end{Entry}

\begin{Entry}{寄}{11}{⼧}
  \begin{Phonetics}{寄}{ji4}[][HSK 4]
    \definition*{s.}{Sobrenome Ji}
    \definition{adj.}{adotado; fomentado; promovido}
    \definition{v.}{enviar; postar; remeter | confiar; depositar; colocar | depender de; apegar-se a}
  \end{Phonetics}
\end{Entry}

\begin{Entry}{寄予}{11,4}{⼧、⼅}
  \begin{Phonetics}{寄予}{ji4yu3}
    \definition{v.}{expressar | colocar (esperança, importância, etc.) em | mostrar}
  \end{Phonetics}
\end{Entry}

\begin{Entry}{寄生}{11,5}{⼧、⽣}
  \begin{Phonetics}{寄生}{ji4sheng1}
    \definition{s.}{parasita | parasitismo}
    \definition{v.}{viver tirando vantagem dos outros | viver dentro ou sobre outro organismo como um parasita}
  \end{Phonetics}
\end{Entry}

\begin{Entry}{寄生生活}{11,5,5,9}{⼧、⽣、⽣、⽔}
  \begin{Phonetics}{寄生生活}{ji4sheng1sheng1huo2}
    \definition{s.}{parasitismo | vida parasitária}
  \end{Phonetics}
\end{Entry}

\begin{Entry}{寄存}{11,6}{⼧、⼦}
  \begin{Phonetics}{寄存}{ji4cun2}
    \definition{v.}{depositar | deixar algo com alguém | armazenar}
  \end{Phonetics}
\end{Entry}

\begin{Entry}{寄托}{11,6}{⼧、⼿}
  \begin{Phonetics}{寄托}{ji4tuo1}
    \definition{v.}{investir (sua esperança, energia, etc.) em algo | confiar (a alguém) | colocar (a esperança, a energia, etc.) em}
  \end{Phonetics}
\end{Entry}

\begin{Entry}{寄卖}{11,8}{⼧、⼗}
  \begin{Phonetics}{寄卖}{ji4mai4}
    \definition{v.}{consignar para venda}
  \end{Phonetics}
\end{Entry}

\begin{Entry}{寄居}{11,8}{⼧、⼫}
  \begin{Phonetics}{寄居}{ji4ju1}
    \definition{s.}{morar longe de casa}
  \end{Phonetics}
\end{Entry}

\begin{Entry}{寄放}{11,8}{⼧、⽅}
  \begin{Phonetics}{寄放}{ji4fang4}
    \definition{v.}{deixar algo com alguém}
  \end{Phonetics}
\end{Entry}

\begin{Entry}{寄养}{11,9}{⼧、⼋}
  \begin{Phonetics}{寄养}{ji4yang3}
    \definition{v.}{embarcar | promover | colocar sob os cuidados de alguém (uma criança, animal de estimação, etc.)}
  \end{Phonetics}
\end{Entry}

\begin{Entry}{寄送}{11,9}{⼧、⾡}
  \begin{Phonetics}{寄送}{ji4song4}
    \definition{v.}{enviar | transmitir}
  \end{Phonetics}
\end{Entry}

\begin{Entry}{寄递}{11,10}{⼧、⾡}
  \begin{Phonetics}{寄递}{ji4di4}
    \definition{s.}{entrega de correspondência}
  \end{Phonetics}
\end{Entry}

\begin{Entry}{寄售}{11,11}{⼧、⼝}
  \begin{Phonetics}{寄售}{ji4shou4}
    \definition{v.}{venda em consignação}
  \end{Phonetics}
\end{Entry}

\begin{Entry}{寄宿}{11,11}{⼧、⼧}
  \begin{Phonetics}{寄宿}{ji4su4}
    \definition{s.}{embarque}
    \definition{v.}{embarcar}
  \end{Phonetics}
\end{Entry}

\begin{Entry}{寄望}{11,11}{⼧、⽉}
  \begin{Phonetics}{寄望}{ji4wang4}
    \definition{v.}{depositar esperanças em}
  \end{Phonetics}
\end{Entry}

\begin{Entry}{密}{11}{⼧}
  \begin{Phonetics}{密}{mi4}[][HSK 4]
    \definition*{s.}{Sobrenome Mi}
    \definition{adj.}{fechado; denso; espesso | íntimo; próximo; afetuoso | delicado; fino; cuidadoso; meticuloso}
    \definition{adv.}{secretamente}
    \definition{s.}{segredo | densidade | senha; \emph{password}}
  \end{Phonetics}
\end{Entry}

\begin{Entry}{密切}{11,4}{⼧、⼑}
  \begin{Phonetics}{密切}{mi4qie4}[][HSK 4]
    \definition{adj.}{próximo; íntimo; relacionamento próximo}
    \definition{adv.}{cuidadosamente; atentamente; intimamente}
    \definition{v.}{tornar-se próximo; tornar-se íntimo; conectar-se}
  \end{Phonetics}
\end{Entry}

\begin{Entry}{密码}{11,8}{⼧、⽯}
  \begin{Phonetics}{密码}{mi4ma3}[][HSK 4]
    \definition[个,种]{s.}{código; senha; um código secreto especialmente formulado usado entre as partes acordadas (diferente do 明码)}
  \seealsoref{明码}{ming2ma3}
  \end{Phonetics}
\end{Entry}

\begin{Entry}{崇}{11}{⼭}
  \begin{Phonetics}{崇}{chong2}
    \definition*{s.}{Sobrenome Chong}
    \definition{adj.}{alto; elevado; sublime}
    \definition{v.}{adorar; reverenciar; venerar; estimar | respeitar}
  \end{Phonetics}
\end{Entry}

\begin{Entry}{崇尚}{11,8}{⼭、⼩}
  \begin{Phonetics}{崇尚}{chong2shang4}[][HSK 7-9]
    \definition{v.}{sustentar; defender; valorizar}[我们崇尚公平与正义。===Nós defendemos a justiça e a equidade.]
  \end{Phonetics}
\end{Entry}

\begin{Entry}{崇拜}{11,9}{⼭、⼿}
  \begin{Phonetics}{崇拜}{chong2bai4}[][HSK 6]
    \definition{v.}{adorar; idolatrar; venerar}
  \end{Phonetics}
\end{Entry}

\begin{Entry}{崇高}{11,10}{⼭、⾼}
  \begin{Phonetics}{崇高}{chong2gao1}[][HSK 7-9]
    \definition{adj.}{alto; elevado; sublime; nobre}
  \end{Phonetics}
\end{Entry}

\begin{Entry}{崖}{11}{⼭}
  \begin{Phonetics}{崖}{ya2}
    \definition{s.}{precipício | penhasco}
  \end{Phonetics}
\end{Entry}

\begin{Entry}{崩}{11}{⼭}
  \begin{Phonetics}{崩}{beng1}
    \definition{v.}{colapsar |  estourar; quebrar | atingir por explosão | matar atirando; atirar; executar | (de um imperador) morrer | rachar; romper | atingir | executar atirando}
  \end{Phonetics}
\end{Entry}

\begin{Entry}{崩溃}{11,12}{⼭、⽔}
  \begin{Phonetics}{崩溃}{beng1kui4}[][HSK 7-9]
    \definition{v.}{colapsar; desmoronar; cair aos pedaços; as coisas estão destruídas; as emoções das pessoas estão fora de controle}
  \end{Phonetics}
\end{Entry}

\begin{Entry}{巢}{11}{⼮}
  \begin{Phonetics}{巢}{chao2}
    \definition*{s.}{Sobrenome Chao}
    \definition[个]{s.}{ninho (de aves, insetos, etc.)}
  \end{Phonetics}
\end{Entry}

\begin{Entry}{常}{11}{⼱}
  \begin{Phonetics}{常}{chang2}[][HSK 1]
    \definition*{s.}{Sobrenome Chang}
    \definition{adj.}{normal; comum; ordinário; indica frequência, normalidade, universalidade | constante; invariável; imutável; permanente}
    \definition{adv.}{frequentemente; geralmente; com frequência;}
    \definition{s.}{normas; disciplina, ordem social e lei e ordem do Estado}
  \end{Phonetics}
\end{Entry}

\begin{Entry}{常人}{11,2}{⼱、⼈}
  \begin{Phonetics}{常人}{chang2ren2}[][HSK 7-9]
    \definition{s.}{pessoa comum; homem da rua}
  \end{Phonetics}
\end{Entry}

\begin{Entry}{常见}{11,4}{⼱、⾒}
  \begin{Phonetics}{常见}{chang2 jian4}[][HSK 2]
    \definition{adj.}{comum; frequentemente visto}
  \end{Phonetics}
\end{Entry}

\begin{Entry}{常用}{11,5}{⼱、⽤}
  \begin{Phonetics}{常用}{chang2 yong4}[][HSK 2]
    \definition{adj.}{em uso comum; frequentemente utilizado}
  \end{Phonetics}
\end{Entry}

\begin{Entry}{常年}{11,6}{⼱、⼲}
  \begin{Phonetics}{常年}{chang2 nian2}[][HSK 6]
    \definition{adj.}{perene; anual}
    \definition{adv.}{ano após ano; ao longo do ano; durante todo o ano; longo prazo}
  \end{Phonetics}
\end{Entry}

\begin{Entry}{常问问题}{11,6,6,15}{⼱、⾨、⾨、⾴}
  \begin{Phonetics}{常问问题}{chang2wen4wen4ti2}
    \definition{s.}{FAQ; perguntas frequentes}
  \end{Phonetics}
\end{Entry}

\begin{Entry}{常识}{11,7}{⼱、⾔}
  \begin{Phonetics}{常识}{chang2shi2}[][HSK 4]
    \definition[门]{s.}{senso comum; conhecimento geral; conhecimento que uma pessoa comum deve ter}
  \end{Phonetics}
\end{Entry}

\begin{Entry}{常态}{11,8}{⼱、⼼}
  \begin{Phonetics}{常态}{chang2tai4}[][HSK 7-9]
    \definition{s.}{normalidade; \emph{habitus}; comportamento normal; condições normais; estado normal ou usual}
  \end{Phonetics}
\end{Entry}

\begin{Entry}{常规}{11,8}{⼱、⾒}
  \begin{Phonetics}{常规}{chang2 gui1}[][HSK 6]
    \definition[个,种]{s.}{convenção; prática comum; rotina | (medicina) rotina | regra; sulco}
  \end{Phonetics}
\end{Entry}

\begin{Entry}{常常}{11,11}{⼱、⼱}
  \begin{Phonetics}{常常}{chang2 chang2}[][HSK 1]
    \definition{adv.}{frequentemente; muitas vezes; geralmente; indica que a ação ocorreu várias vezes}
  \end{Phonetics}
\end{Entry}

\begin{Entry}{常理}{11,11}{⼱、⽟}
  \begin{Phonetics}{常理}{chang2li3}[][HSK 7-9]
    \definition{s.}{regra geral; o que é normal | senso comum; pensamento lógico | raciocínio convencional e moral}
  \end{Phonetics}
\end{Entry}

\begin{Entry}{常温}{11,12}{⼱、⽔}
  \begin{Phonetics}{常温}{chang2wen1}[][HSK 7-9]
    \definition{s.}{temperatura atmosférica normal; temperatura ordinária | homeotermia}
  \end{Phonetics}
\end{Entry}

\begin{Entry}{庵}{11}{⼴}
  \begin{Phonetics}{庵}{an1}
    \definition*{s.}{Sobrenome An}
    \definition[个,座]{s.}{cabana | convento de freiras; templos budistas, principalmente onde vivem as freiras}
  \end{Phonetics}
\end{Entry}

\begin{Entry}{庶}{11}{⼴}
  \begin{Phonetics}{庶}{shu4}
    \definition*{s.}{Sobrenome Shu}
    \definition{adj.}{multitudinário; numeroso}
    \definition{conj.}{para que; de ​​modo a}
    \definition{s.}{da ou pela concubina (diferentemente da esposa legal); no sistema patriarcal, refere-se ao ramo lateral da família}
  \end{Phonetics}
\end{Entry}

\begin{Entry}{庶民}{11,5}{⼴、⽒}
  \begin{Phonetics}{庶民}{shu4min2}
    \definition{s.}{(antigo) pessoas comuns | (antigo) plebeu; plebeus | (antigo) a multidão de pessoas comuns (na literatura erudita)}
  \end{Phonetics}
\end{Entry}

\begin{Entry}{康}{11}{⼴}
  \begin{Phonetics}{康}{kang1}
    \definition*{s.}{Sobrenome Kang}
    \definition{adj.}{saudável |  fácil; pacífico; abundante | amplo; largo | Dialeto: de baixa qualidade; inferior}
    \definition{s.}{bem-estar; saúde | palha; farelo; casca}
    \definition{v.}{(normalmente de um rabanete) tornar-se esponjoso}
  \end{Phonetics}
\end{Entry}

\begin{Entry}{康复}{11,9}{⼴、⼢}
  \begin{Phonetics}{康复}{kang1 fu4}[][HSK 6]
    \definition{v.}{Saúde: estaurar; recuperar; reabilitar}
  \end{Phonetics}
\end{Entry}

\begin{Entry}{廊}{11}{⼴}
  \begin{Phonetics}{廊}{lang2}
    \definition[个]{s.}{varanda; corredor}
  \end{Phonetics}
\end{Entry}

\begin{Entry}{廊坊}{11,7}{⼴、⼟}
  \begin{Phonetics}{廊坊}{lang2fang2}
    \definition*{s.}{Cidade de Langfang em Hebei}
  \end{Phonetics}
\end{Entry}

\begin{Entry}{弹}{11}{⼸}
  \begin{Phonetics}{弹}{dan4}
    \definition{s.}{bola; pelota; pequenas bolas disparadas com um estilingue | bomba; bala; explosivos que podem ser lançados ou arremessados, com poder destrutivo e letal}
  \end{Phonetics}
  \begin{Phonetics}{弹}{tan2}[][HSK 5]
    \definition{v.}{enviar; atirar (como com uma catapulta, etc.); usar a elasticidade de um objeto para lançar outro objeto | afofar; preparar fibras; usar um dispositivo elástico para amolecer as fibras | virar; sacudir | dedilhar; tocar (um instrumento musical de cordas) | acusar; atacar; criticar; relatar | saltar; quicar}
  \end{Phonetics}
\end{Entry}

\begin{Entry}{彩}{11}{⼺}
  \begin{Phonetics}{彩}{cai3}
    \definition{s.}{cor | aplausos; vivas | variedade; brilho; esplendor | prêmio; loteria | sangue de uma ferida | habilidades especiais empregadas em mágica ou ópera para alcançar um efeito desejado | seda colorida | cores variadas | graça na arte; graciosidade | prêmio de loteria; ganhos | efeitos especiais no teatro chinês (simbolizando sangue, fogo, etc.)}
  \end{Phonetics}
\end{Entry}

\begin{Entry}{彩电}{11,5}{⼺、⽥}
  \begin{Phonetics}{彩电}{cai3dian4}[][HSK 7-9]
    \definition[台,个]{s.}{TV à cores}
  \end{Phonetics}
\end{Entry}

\begin{Entry}{彩色}{11,6}{⼺、⾊}
  \begin{Phonetics}{彩色}{cai3 se4}[][HSK 3]
    \definition[个,种]{s.}{multicolorido; cor; várias cores}
  \end{Phonetics}
\end{Entry}

\begin{Entry}{彩虹}{11,9}{⼺、⾍}
  \begin{Phonetics}{彩虹}{cai3hong2}[][HSK 7-9]
    \definition[道,条]{s.}{arco-íris}
  \end{Phonetics}
\end{Entry}

\begin{Entry}{彩票}{11,11}{⼺、⽰}
  \begin{Phonetics}{彩票}{cai3piao4}[][HSK 5]
    \definition[张,注]{s.}{bilhete de loteria; um título com números, vendido pelo valor de face; após o sorteio, o portador do bilhete premiado pode reivindicar o prêmio de acordo com o regulamento}
  \end{Phonetics}
\end{Entry}

\begin{Entry}{彩霞}{11,17}{⼺、⾬}
  \begin{Phonetics}{彩霞}{cai3xia2}[][HSK 7-9]
    \definition[片]{s.}{nuvens rosadas (ou cor-de-rosa) | nuvens tingidas com tons de pôr do sol; nuvens coloridas}
  \end{Phonetics}
\end{Entry}

\begin{Entry}{彪}{11}{⾌}
  \begin{Phonetics}{彪}{biao1}
    \definition*{s.}{Sobrenome Biao}
    \definition{adj.}{semelhante a um tigre (metáfora para estatura alta)}
    \definition{s.}{tigre jovem}
  \end{Phonetics}
\end{Entry}

\begin{Entry}{彬}{11}{⼺}
  \begin{Phonetics}{彬}{bin1}
    \definition*{s.}{Sobrenome Bin}
    \definition{adj.}{Literário: fino; elegante}
  \end{Phonetics}
\end{Entry}

\begin{Entry}{彬彬有礼}{11,11,6,5}{⼺、⼺、⽉、⽰}
  \begin{Phonetics}{彬彬有礼}{bin1bin1-you3li3}[][HSK 7-9]
    \definition{expr.}{refinado e cortês; urbano}
  \end{Phonetics}
\end{Entry}

\begin{Entry}{得}{11}{⼻}
  \begin{Phonetics}{得}{de2}[][HSK 2]
    \definition{adj.}{adequado; apropriado | satisfeito; complacente; orgulhoso de si mesmo}
    \definition{interj.}{usado para encerrar uma conversa para indicar concordância ou proibição | usado quando a situação não é a esperada, para expressar impotência}
    \definition{v.}{obter (em oposição a 失); conseguir; ganhar |  (de um cálculo) igual; resultar em | estar pronto; estar acabado | pegar; apanhar; contrair uma doença}
    \definition{v.aux.}{usado antes de outros verbos para expressar permissão | usado antes de outros verbos para indicar que é possível (usado principalmente na forma negativa) | usado em conversas para indicar que não há necessidade de dizer mais nada}
  \seealsoref{失}{shi1}
  \end{Phonetics}
  \begin{Phonetics}{得}{de5}[][HSK 2]
    \definition{part.}{depois de um verbo ou adjetivo para expressar possibilidade ou capacidade | entre um verbo e seu complemento para expressar possibilidade | ligando um verbo ou um adjetivo a um complemento que descreve a maneira ou o grau}
  \end{Phonetics}
  \begin{Phonetics}{得}{dei3}[][HSK 4]
    \definition{v.}{precisar; expressa uma necessidade lógica, factual ou subjetiva; deve; é necessário | ter de; ser obrigado a; indica uma necessidade de vontade ou de fato | certamente irá; expressa a inevitabilidade da especulação}
  \end{Phonetics}
\end{Entry}

\begin{Entry}{得了}{11,2}{⼻、⼅}
  \begin{Phonetics}{得了}{de2le5}[][HSK 5]
    \definition{expr.}{Tudo bem!; É o bastante!}
  \end{Phonetics}
  \begin{Phonetics}{得了}{de2liao3}
    \definition{adj.}{(enfaticamente, em perguntas retóricas) possível; indica que a situação é séria (usado principalmente em perguntas retóricas ou formas negativas)}
  \end{Phonetics}
\end{Entry}

\begin{Entry}{得力}{11,2}{⼻、⼒}
  \begin{Phonetics}{得力}{de2li4}[][HSK 7-9]
    \definition{adj.}{capaz; competente; capaz de fazer coisas | eficiente; poderoso}
    \definition{v.}{beneficiar-se de; obter ajuda de; beneficiar}
  \end{Phonetics}
\end{Entry}

\begin{Entry}{得不偿失}{11,4,11,5}{⼻、⼀、⼈、⼤}
  \begin{Phonetics}{得不偿失}{de2bu4chang2shi1}[][HSK 7-9]
    \definition{expr.}{``A perda supera o ganho.''; ``Os ganhos não compensam as perdas.''; perder mais do que ganhar; ``O jogo não vale a pena.''; ``O que é ganho não compensa o que é perdido.''}
  \end{Phonetics}
\end{Entry}

\begin{Entry}{得以}{11,4}{⼻、⼈}
  \begin{Phonetics}{得以}{de2 yi3}[][HSK 5]
    \definition{v.}{ser capaz de; para que\dots possa (ou possa)\dots}
  \end{Phonetics}
\end{Entry}

\begin{Entry}{得分}{11,4}{⼻、⼑}
  \begin{Phonetics}{得分}{de2 fen1}[][HSK 3]
    \definition{s.}{pontuação; classificação; nota; pontuação obtida em jogos ou competições}
    \definition{v.}{fazer pontos; pontuar}
  \end{Phonetics}
\end{Entry}

\begin{Entry}{得天独厚}{11,4,9,9}{⼻、⼤、⽝、⼚}
  \begin{Phonetics}{得天独厚}{de2tian1du2hou4}[][HSK 7-9]
    \definition{expr.}{ser ricamente dotado pela natureza; abundar em dádivas da natureza; desfrutar de vantagens excepcionais | abençoado pelo céu | desfrutar de vantagens excepcionais | favorecido pela natureza}
  \end{Phonetics}
\end{Entry}

\begin{Entry}{得手}{11,4}{⼻、⼿}
  \begin{Phonetics}{得手}{de2shou3}[][HSK 7-9]
    \definition{adj.}{Coloquial: prático; conveniente e fácil de usar}
    \definition{v.}{fazer algo suavemente; ter sucesso; atingir seu objetivo | ir suavemente; sair; fazer bem; fazer as coisas sem problemas}
  \end{Phonetics}
\end{Entry}

\begin{Entry}{得出}{11,5}{⼻、⼐}
  \begin{Phonetics}{得出}{de2 chu1}[][HSK 2]
    \definition{v.}{chegar (a uma conclusão); obter (a um resultado); deduzir ou calcular (conclusão ou resultado)}
  \end{Phonetics}
\end{Entry}

\begin{Entry}{得失}{11,5}{⼻、⼤}
  \begin{Phonetics}{得失}{de2shi1}[][HSK 7-9]
    \definition{s.}{ganho e perda; sucesso e fracasso | méritos e deméritos; vantagens e desvantagens; prós e contras}
  \end{Phonetics}
\end{Entry}

\begin{Entry}{得当}{11,6}{⼻、⼹}
  \begin{Phonetics}{得当}{de2dang4}[][HSK 7-9]
    \definition{adj.}{apropriado; próprio; adequado | apto}
  \end{Phonetics}
\end{Entry}

\begin{Entry}{得体}{11,7}{⼻、⼈}
  \begin{Phonetics}{得体}{de2ti3}[][HSK 7-9]
    \definition{adj.}{(fala, comportamento, etc.) apropriado; moderado}
  \end{Phonetics}
\end{Entry}

\begin{Entry}{得到}{11,8}{⼻、⼑}
  \begin{Phonetics}{得到}{de2 dao4}[][HSK 1]
    \definition{v.}{obter; conseguir; ganhar; receber; possuir algo; adquirir}
  \end{Phonetics}
\end{Entry}

\begin{Entry}{得知}{11,8}{⼻、⽮}
  \begin{Phonetics}{得知}{de2zhi1}[][HSK 7-9]
    \definition{v.}{saber; ser informado de; aprender}
  \end{Phonetics}
\end{Entry}

\begin{Entry}{得益于}{11,10,3}{⼻、⽫、⼆}
  \begin{Phonetics}{得益于}{de2yi4 yu2}[][HSK 7-9]
    \definition{s.}{correlação positiva; benefício}
  \end{Phonetics}
\end{Entry}

\begin{Entry}{得意}{11,13}{⼻、⼼}
  \begin{Phonetics}{得意}{de2yi4}[][HSK 4]
    \definition{adj.}{complacente; orgulhoso de si mesmo; satisfeito consigo mesmo}
  \end{Phonetics}
\end{Entry}

\begin{Entry}{得意扬扬}{11,13,6,6}{⼻、⼼、⼿、⼿}
  \begin{Phonetics}{得意扬扬}{de2yi4-yang2yang2}[][HSK 7-9]
    \definition{expr.}{orgulhoso e complacente | estar imensamente orgulhoso; parecer triunfante}
  \end{Phonetics}
\end{Entry}

\begin{Entry}{得罪}{11,13}{⼻、⽹}
  \begin{Phonetics}{得罪}{de2zui4}[][HSK 7-9]
    \definition{v.}{ofender; desagradar; causar desprazer ou ressentimento}
  \end{Phonetics}
\end{Entry}

\begin{Entry}{悉}{11}{⼼}
  \begin{Phonetics}{悉}{xi1}
    \definition*{s.}{Sobrenome Xi}
    \definition{adj.}{tudo; inteiro; total | detalhado}
    \definition{v.}{saber; aprender; ser informado de}
  \end{Phonetics}
\end{Entry}

\begin{Entry}{悉心}{11,4}{⼼、⼼}
  \begin{Phonetics}{悉心}{xi1xin1}
    \definition{adv.}{colocar o coração (e a alma) em algo | com muito cuidado}
  \end{Phonetics}
\end{Entry}

\begin{Entry}{悉尼}{11,5}{⼼、⼫}
  \begin{Phonetics}{悉尼}{xi1ni2}
    \definition*{s.}{Sidney}
  \end{Phonetics}
\end{Entry}

\begin{Entry}{悉数}{11,13}{⼼、⽁}
  \begin{Phonetics}{悉数}{xi1shu3}
    \definition{adv.}{enumerar em detalhes | explicar claramente}
  \end{Phonetics}
  \begin{Phonetics}{悉数}{xi1shu4}
    \definition{adv.}{todos | cada um | toda a soma}
  \end{Phonetics}
\end{Entry}

\begin{Entry}{患}{11}{⼼}
  \begin{Phonetics}{患}{huan4}
    \definition*{s.}{Sobrenome Huan}
    \definition{s.}{perigo; problema; desastre; flagelo | preocupação; ansiedade}
    \definition{v.}{contrair (doença); sofrer de}
  \end{Phonetics}
\end{Entry}

\begin{Entry}{患者}{11,8}{⼼、⽼}
  \begin{Phonetics}{患者}{huan4zhe3}[][HSK 6]
    \definition[个,位,名]{s.}{paciente; sofredor; pessoas com certas doenças}
  \end{Phonetics}
\end{Entry}

\begin{Entry}{您}{11}{⼼}
  \begin{Phonetics}{您}{nin2}[][HSK 1]
    \definition{pron.}{você; a forma de tratamento respeitosa da segunda pessoa do singular 你}
  \seealsoref{你}{ni3}
  \end{Phonetics}
\end{Entry}

\begin{Entry}{悬}{11}{⼼}
  \begin{Phonetics}{悬}{xuan2}[][HSK 6]
    \definition{adj.}{pendente; não resolvido; sem nenhum resultado | distante; a distância é grande; a diferença é grande | (dialeto) perigoso}
    \definition{v.}{pendurar; suspender | levantar; elevar | sentir-se ansioso; ser solícito | imaginar}
  \end{Phonetics}
\end{Entry}

\begin{Entry}{悬挂}{11,9}{⼼、⼿}
  \begin{Phonetics}{悬挂}{xuan2gua4}
    \definition{v.}{pendurar; pender; suspender; prender um objeto em um ou mais pontos em algum lugar com a ajuda de uma corda, gancho, prego, etc.}
  \end{Phonetics}
\end{Entry}

\begin{Entry}{悬崖}{11,11}{⼼、⼭}
  \begin{Phonetics}{悬崖}{xuan2ya2}
    \definition{s.}{precipício | penhasco}
  \end{Phonetics}
\end{Entry}

\begin{Entry}{悼}{11}{⼼}
  \begin{Phonetics}{悼}{dao4}
    \definition*{s.}{Sobrenome Dao}
    \definition{v.}{lamentar; expressar pesar}
  \end{Phonetics}
\end{Entry}

\begin{Entry}{悼念}{11,8}{⼼、⼼}
  \begin{Phonetics}{悼念}{dao4nian4}[][HSK 7-9]
    \definition{v.}{lamentar; lamentar-se por | lamentar por; expressar pesar}
  \end{Phonetics}
\end{Entry}

\begin{Entry}{情}{11}{⼼}
  \begin{Phonetics}{情}{qing2}
    \definition{s.}{sentimento; afeição | amor; paixão | paixão sexual; luxúria | favor; gentileza | situação; circunstâncias; condição | razão; sentido | sensibilidades; sentimentos}
  \end{Phonetics}
\end{Entry}

\begin{Entry}{情节}{11,5}{⼼、⾋}
  \begin{Phonetics}{情节}{qing2jie2}[][HSK 5]
    \definition[个,段]{s.}{enredo; trama; desenrolar específico dos acontecimentos | circunstância; detalhes do crime ou erro | enredo; roteiro; refere-se especificamente ao processo de desenvolvimento e evolução dos conflitos e contradições em obras literárias narrativas}
  \end{Phonetics}
\end{Entry}

\begin{Entry}{情况}{11,7}{⼼、⼎}
  \begin{Phonetics}{情况}{qing2kuang4}[][HSK 3]
    \definition[种,个,些]{s.}{condição; situação; circunstâncias; estado das coisas | mudanças notáveis e impactantes}
  \end{Phonetics}
\end{Entry}

\begin{Entry}{情形}{11,7}{⼼、⼺}
  \begin{Phonetics}{情形}{qing2xing2}[][HSK 5]
    \definition[个,种]{s.}{situação; condição; circunstâncias; estado de coisas; a situação específica das coisas}
  \end{Phonetics}
\end{Entry}

\begin{Entry}{情绪}{11,11}{⼼、⽷}
  \begin{Phonetics}{情绪}{qing2xu4}[][HSK 6]
    \definition[种,片,股,丝]{s.}{mau humor; depressão; um sentimento ruim no coração, especialmente um estado mental desagradável quando se sente injusto | emoção; humor; moral; sentimento; o estado mental de uma pessoa ao longo de um período de tempo}
  \end{Phonetics}
\end{Entry}

\begin{Entry}{情景}{11,12}{⼼、⽇}
  \begin{Phonetics}{情景}{qing2jing3}[][HSK 4]
    \definition[个,幕,种]{s.}{cena; vista; circunstâncias}
  \end{Phonetics}
\end{Entry}

\begin{Entry}{情感}{11,13}{⼼、⼼}
  \begin{Phonetics}{情感}{qing2 gan3}[][HSK 3]
    \definition[份]{s.}{emoção; sentimento | afeição; apego; reações psicológicas positivas ou negativas a estímulos externos, como gosto, raiva, tristeza, medo, amor, nojo, etc.}
  \end{Phonetics}
\end{Entry}

\begin{Entry}{惊}{11}{⼼}
  \begin{Phonetics}{惊}{jing1}
    \definition{v.}{assustar; ficar assustado; ficar nervoso devido a estímulo repentino; ficar com medo | surpreender; chocar; alarmar}
  \end{Phonetics}
\end{Entry}

\begin{Entry}{惊人}{11,2}{⼼、⼈}
  \begin{Phonetics}{惊人}{jing1 ren2}[][HSK 6]
    \definition{adj.}{surpreso; espantado; atônito; surpreendente}
  \end{Phonetics}
\end{Entry}

\begin{Entry}{惊呆}{11,7}{⼼、⼝}
  \begin{Phonetics}{惊呆}{jing1dai1}
    \definition{adj.}{estupefato | chocado}
  \end{Phonetics}
\end{Entry}

\begin{Entry}{惊喜}{11,12}{⼼、⼝}
  \begin{Phonetics}{惊喜}{jing1 xi3}[][HSK 6]
    \definition{s.}{boa surpresa; agradavelmente surpreso}
  \end{Phonetics}
\end{Entry}

\begin{Entry}{惦}{11}{⼼}
  \begin{Phonetics}{惦}{dian4}
    \definition{v.}{lembrar com preocupação; estar preocupado com; continuar pensando sobre; ficar pensando em alguém ou em alguma coisa e se preocupar com eles; sentir falta deles}
  \end{Phonetics}
\end{Entry}

\begin{Entry}{惦记}{11,5}{⼼、⾔}
  \begin{Phonetics}{惦记}{dian4ji4}[][HSK 7-9]
    \definition{s.}{lembrar com preocupação; estar preocupado com; continuar pensando sobre; (sobre uma pessoa ou coisa) continuar pensando nisso e não deixar passar}
  \end{Phonetics}
\end{Entry}

\begin{Entry}{惨}{11}{⽕}
  \begin{Phonetics}{惨}{can3}[][HSK 6]
    \definition{adj.}{miserável; trágico | cruel; brutal; implacável | desastroso; terrível; esmagador | lamentável; desaventurado | em um grau sério; grau grave; dano grave | selvagem; desumano; vicioso; cruel}
  \end{Phonetics}
\end{Entry}

\begin{Entry}{惨白}{11,5}{⽕、⽩}
  \begin{Phonetics}{惨白}{can3bai2}[][HSK 7-9]
    \definition{adj.}{(cenário) escuro; fraco; sombrio; sombrio | (rosto) mortalmente pálido; medonho | terrivelmente (fantasmagórico) pálido; pálido; escuro}
  \end{Phonetics}
\end{Entry}

\begin{Entry}{惨重}{11,9}{⽕、⾥}
  \begin{Phonetics}{惨重}{can3zhong4}[][HSK 7-9]
    \definition{adj.}{pesado; doloroso; desastroso | calamitoso (perdas extremamente graves)}
  \end{Phonetics}
\end{Entry}

\begin{Entry}{惨痛}{11,12}{⽕、⽧}
  \begin{Phonetics}{惨痛}{can3tong4}[][HSK 7-9]
    \definition{adj.}{grave; terrivelmente doloroso; amargo; agonizante | profundamente triste; doloroso}
  \end{Phonetics}
\end{Entry}

\begin{Entry}{惭}{11}{⼼}
  \begin{Phonetics}{惭}{can2}
    \definition{adj.}{envergonhado}
    \definition{s.}{vergonha}
    \definition{v.}{sentir vergonha}
  \end{Phonetics}
\end{Entry}

\begin{Entry}{惭愧}{11,12}{⼼、⼼}
  \begin{Phonetics}{惭愧}{can2kui4}[][HSK 7-9]
    \definition{adj.}{envergonhado; sentir-se inseguro por ter deficiências, fazer algo errado ou não cumprir responsabilidades}
  \end{Phonetics}
\end{Entry}

\begin{Entry}{惯}{11}{⼼}
  \begin{Phonetics}{惯}{guan4}[][HSK 7-9]
    \definition{adj.}{habitual; costumeiro; usual | incorrigível; endurecido}
    \definition{v.}{estar acostumado a; ter o hábito de | mimar; estragar}
  \end{Phonetics}
\end{Entry}

\begin{Entry}{惯例}{11,8}{⼼、⼈}
  \begin{Phonetics}{惯例}{guan4li4}[][HSK 7-9]
    \definition[个]{s.}{rotina; convenção; prática usual; prática habitual | precedente; embora não haja nenhuma disposição explícita na lei, há práticas que foram implementadas no passado e podem ser imitadas}
  \end{Phonetics}
\end{Entry}

\begin{Entry}{惯性}{11,8}{⼼、⼼}
  \begin{Phonetics}{惯性}{guan4xing4}[][HSK 7-9]
    \definition{s.}{Física: inércia; a força da inércia}
  \end{Phonetics}
\end{Entry}

\begin{Entry}{据}{11}{⼿}
  \begin{Phonetics}{据}{ju1}
    \definition{part.}{elemento formador de palavras}
  \seealsoref{拮据}{jie2ju1}
  \end{Phonetics}
  \begin{Phonetics}{据}{ju4}[][HSK 6]
    \definition*{s.}{Sobrenome Ju}
    \definition{prep.}{de acordo com; com base em}
    \definition{s.}{evidência; certificado; prova}
    \definition{v.}{ocupar; apreender | confiar em; depender de}
  \end{Phonetics}
\end{Entry}

\begin{Entry}{据说}{11,9}{⼿、⾔}
  \begin{Phonetics}{据说}{ju4shuo1}[][HSK 3]
    \definition{v.}{é o que dizem; é o que se diz}
  \end{Phonetics}
\end{Entry}

\begin{Entry}{捶}{11}{⼿}
  \begin{Phonetics}{捶}{chui2}[][HSK 7-9]
    \definition{v.}{bater (com um pedaço de pau, martelo ou punho)}
  \end{Phonetics}
\end{Entry}

\begin{Entry}{捶子}{11,3}{⼿、⼦}
  \begin{Phonetics}{捶子}{chui2zi5}[][HSK 7-9]
    \definition[把]{s.}{martelo}
  \end{Phonetics}
\end{Entry}

\begin{Entry}{捷}{11}{⼿}
  \begin{Phonetics}{捷}{jie2}
    \definition*{s.}{Sobrenome Jie}
    \definition{adj.}{rápido; ágil}
    \definition{s.}{vitória; triunfo; sucesso}
  \end{Phonetics}
\end{Entry}

\begin{Entry}{捷径}{11,8}{⼿、⼻}
  \begin{Phonetics}{捷径}{jie2jing4}
    \definition{s.}{atalho}
  \end{Phonetics}
\end{Entry}

\begin{Entry}{掉}{11}{⼿}
  \begin{Phonetics}{掉}{diao4}[][HSK 2]
    \definition{v.}{cair; soltar-se; desprender-se | ficar para trás | perder; desaparecer; omitir | diminuir; reduzir | balançar; abanar; oscilar | virar; voltar; retornar | alterar; trocar; intercambiar}
    \definition{v.aux.}{usado após certos verbos para indicar a conclusão de uma ação}
  \end{Phonetics}
\end{Entry}

\begin{Entry}{掉队}{11,4}{⼿、⾩}
  \begin{Phonetics}{掉队}{diao4/dui4}[][HSK 7-9]
    \definition{v.+compl.}{abandonar (ou sair); ficar para trás; cair fora}
  \end{Phonetics}
\end{Entry}

\begin{Entry}{掉包}{11,5}{⼿、⼓}
  \begin{Phonetics}{掉包}{diao4bao1}
    \definition{v.}{vender uma falsificação pelo artigo genuíno | roubar o item valioso de alguém e substituí-lo por um item de aparência semelhante, mas sem valor}
  \end{Phonetics}
\end{Entry}

\begin{Entry}{掉头}{11,5}{⼿、⼤}
  \begin{Phonetics}{掉头}{diao4/tou2}[][HSK 7-9]
    \definition{v.+compl.}{virar-se; afastar-se (das pessoas) | dar meia-volta (carro, barco, etc.); (carro, barco, etc.) virar na direção oposta}
  \end{Phonetics}
\end{Entry}

\begin{Entry}{掉线}{11,8}{⼿、⽷}
  \begin{Phonetics}{掉线}{diao4xian4}
    \definition{v.}{desconectar-se (da \emph{Internet})}
  \end{Phonetics}
\end{Entry}

\begin{Entry}{掉转}{11,8}{⼿、⾞}
  \begin{Phonetics}{掉转}{diao4zhuan3}
    \definition{v.}{dar a volta}
  \end{Phonetics}
\end{Entry}

\begin{Entry}{掉落}{11,12}{⼿、⾋}
  \begin{Phonetics}{掉落}{diao4luo4}
    \definition{v.}{derrubar}
  \end{Phonetics}
\end{Entry}

\begin{Entry}{掉膘}{11,15}{⼿、⾁}
  \begin{Phonetics}{掉膘}{diao4biao1}
    \definition{v.}{perder peso (gado)}
  \end{Phonetics}
\end{Entry}

\begin{Entry}{掏}{11}{⼿}
  \begin{Phonetics}{掏}{tao1}[][HSK 6]
    \definition{v.}{extrair; retirar; pescar | cavar (um buraco, etc.); escavar; retirar | (coloquial) roubar do bolso de alguém | tirar}
  \end{Phonetics}
\end{Entry}

\begin{Entry}{排}{11}{⼿}
  \begin{Phonetics}{排}{pai2}[][HSK 2,3]
    \definition{clas.}{usado para linhas, filas; coisas usadas para formar filas}
    \definition{s.}{linha; fileira; fileiras horizontais | pelotão; unidade militar, abaixo do nível de companhia, acima do nível de pelotão | jangada; balsa; um meio de transporte aquático feito de bambu e madeira unidos lado a lado; também se refere a bambu e madeira amarrados em fileiras para facilitar o transporte aquático | torta; bolo de carne; bolinho assado; comida cozida no vapor}
    \definition{v.}{organizar; alinhar; colocar em ordem; posicionar ou organizar em uma determinada ordem; ordenar | ensaiar | ejetar; excluir; dispensar; remover; eliminar | empurrar o obstáculo para fora do caminho}
  \end{Phonetics}
\end{Entry}

\begin{Entry}{排水}{11,4}{⼿、⽔}
  \begin{Phonetics}{排水}{pai2shui3}
    \definition{v.}{drenar}
  \end{Phonetics}
\end{Entry}

\begin{Entry}{排队}{11,4}{⼿、⾩}
  \begin{Phonetics}{排队}{pai2/dui4}[][HSK 2]
    \definition{v.+compl.}{formar uma fila; alinhar-se; enfileirar-se; organizar em sequência | listar; classificar}
  \end{Phonetics}
\end{Entry}

\begin{Entry}{排列}{11,6}{⼿、⼑}
  \begin{Phonetics}{排列}{pai2lie4}[][HSK 4]
    \definition{v.}{classificar; colocar; variar; organizar; pôr em ordem}
  \end{Phonetics}
\end{Entry}

\begin{Entry}{排名}{11,6}{⼿、⼝}
  \begin{Phonetics}{排名}{pai2 ming2}[][HSK 3]
    \definition{s.}{classificação; resultado; organizado de acordo com determinados critérios}
  \end{Phonetics}
\end{Entry}

\begin{Entry}{排行榜}{11,6,14}{⼿、⾏、⽊}
  \begin{Phonetics}{排行榜}{pai2 hang2 bang3}[][HSK 6]
    \definition{s.}{lista; classificação; lista de classificação; (de registros) os gráficos; uma lista em uma determinada ordem publicada com base em certos resultados estatísticos}
  \end{Phonetics}
\end{Entry}

\begin{Entry}{排除}{11,9}{⼿、⾩}
  \begin{Phonetics}{排除}{pai2chu2}[][HSK 5]
    \definition{v.}{remover; superar; excluir; eliminar; livrar-se de}
  \end{Phonetics}
\end{Entry}

\begin{Entry}{排球}{11,11}{⼿、⽟}
  \begin{Phonetics}{排球}{pai2 qiu2}[][HSK 2]
    \definition[场,只,个]{s.}{voleibol; bola de voleibol}
  \end{Phonetics}
\end{Entry}

\begin{Entry}{探}{11}{⼿}
  \begin{Phonetics}{探}{tan4}
    \definition[个,位,名]{s.}{batedor; espião; detetive}
    \definition{v.}{tentar descobrir; explorar; soar | explorar; espionar | visitar; fazer uma visita em | se destacar | preocupar-se com; envolver-se em | ver; invocar}
  \end{Phonetics}
\end{Entry}

\begin{Entry}{探讨}{11,5}{⼿、⾔}
  \begin{Phonetics}{探讨}{tan4tao3}[][HSK 6]
    \definition{v.}{examinar; indagar; investigar; discutir}
  \end{Phonetics}
\end{Entry}

\begin{Entry}{探亲}{11,9}{⼿、⼇}
  \begin{Phonetics}{探亲}{tan4/qin1}
    \definition{v.+compl.}{ir para casa para visitar a família}
  \end{Phonetics}
\end{Entry}

\begin{Entry}{探索}{11,10}{⼿、⽷}
  \begin{Phonetics}{探索}{tan4suo3}[][HSK 6]
    \definition{v.}{sondar; explorar; procurar respostas de várias fontes para resolver dúvidas}
  \end{Phonetics}
\end{Entry}

\begin{Entry}{接}{11}{⼿}
  \begin{Phonetics}{接}{jie1}[][HSK 2]
    \definition*{s.}{Sobrenome Jie}
    \definition{v.}{entrar em contato com; aproximar-se de | conectar; unir; juntar | continuar; prosseguir | assumir o controle; assumir o trabalho de outra pessoa e continuar a fazê-lo | pegar; agarrar; segurar ou sustentar com as mãos | receber; aceitar | encontrar; dar as boas-vindas}
  \end{Phonetics}
\end{Entry}

\begin{Entry}{接下来}{11,3,7}{⼿、⼀、⽊}
  \begin{Phonetics}{接下来}{jie1 xia4 lai2}[][HSK 2]
    \definition{expr.}{próximo; seguinte; indica uma sequência temporal subsequente}
  \end{Phonetics}
\end{Entry}

\begin{Entry}{接(电话)}{11,5,8}{⼿、⽥、⾔}
  \begin{Phonetics}{接(电话)}{jie1(dian4hua4)}
    \definition{v.}{atender (o telefone) | receber (uma ligação telefônica)}
  \end{Phonetics}
\end{Entry}

\begin{Entry}{接收}{11,6}{⼿、⽁}
  \begin{Phonetics}{接收}{jie1 shou1}[][HSK 6]
    \definition{v.}{aceitar; receber | assumir; expropriar; tomar posse (de uma instituição, propriedade, etc.) de acordo com a lei | admitir; aceitar; absorver}
  \end{Phonetics}
\end{Entry}

\begin{Entry}{接近}{11,7}{⼿、⾡}
  \begin{Phonetics}{接近}{jie1jin4}[][HSK 3]
    \definition{adj.}{perto; próximo; a diferença entre os dois é mínima}
    \definition{v.}{estar perto de; aproximar; aproximar-se}
  \end{Phonetics}
\end{Entry}

\begin{Entry}{接连}{11,7}{⼿、⾡}
  \begin{Phonetics}{接连}{jie1lian2}[][HSK 5]
    \definition{adv.}{no final; em sucessão; em uma fileira; um após o outro; seguindo o anterior; continuando}
  \end{Phonetics}
\end{Entry}

\begin{Entry}{接到}{11,8}{⼿、⼑}
  \begin{Phonetics}{接到}{jie1 dao4}[][HSK 2]
    \definition{v.}{receber (carta, etc.)}
  \end{Phonetics}
\end{Entry}

\begin{Entry}{接受}{11,8}{⼿、⼜}
  \begin{Phonetics}{接受}{jie1shou4}[][HSK 2]
    \definition{v.}{aceitar; não recusar (o que os outros oferecem) | concordar; não recusar (opiniões/sugestões/críticas/convites de outras pessoas, etc.)}
  \end{Phonetics}
\end{Entry}

\begin{Entry}{接待}{11,9}{⼿、⼻}
  \begin{Phonetics}{接待}{jie1dai4}[][HSK 3]
    \definition{v.}{receber (alguém); acolher; recepcionar; receber com cordialidade e generosidade}
  \end{Phonetics}
\end{Entry}

\begin{Entry}{接班人}{11,10,2}{⼿、⽟、⼈}
  \begin{Phonetics}{接班人}{jie1ban1ren2}
    \definition{s.}{sucessor}
  \end{Phonetics}
\end{Entry}

\begin{Entry}{接着}{11,11}{⼿、⽬}
  \begin{Phonetics}{接着}{jie1zhe5}[][HSK 2]
    \definition{adv.}{por sua vez; um após o outro; sucessivamente; conectado (à frase anterior); imediatamente após (a ação anterior)}
    \definition{v.}{seguir; prosseguir; continuar; seguir em frente; ficar ao lado | pegar com as mãos; apanhar}
  \end{Phonetics}
\end{Entry}

\begin{Entry}{接触}{11,13}{⼿、⾓}
  \begin{Phonetics}{接触}{jie1chu4}[][HSK 5]
    \definition{v.}{entrar em contato com | entrar em contato; tocar; interagir | engajar; o termo militar refere-se a fogo cruzado}
  \end{Phonetics}
\end{Entry}

\begin{Entry}{控}{11}{⼿}
  \begin{Phonetics}{控}{kong4}
    \definition{v.}{acusar; cobrar | controlar; dominar | manter (parte do corpo em uma determinada posição) sem apoio | virar (um recipiente) de cabeça para baixo para deixar o líquido escorrer}
  \end{Phonetics}
\end{Entry}

\begin{Entry}{控制}{11,8}{⼿、⼑}
  \begin{Phonetics}{控制}{kong4zhi4}[][HSK 5]
    \definition{v.}{controlar; restringir; dominar; fazer com que não ultrapasse um determinado limite | controlar; dominar; comandar; ocupar, fazer com que não se perca}
  \end{Phonetics}
\end{Entry}

\begin{Entry}{推}{11}{⼿}
  \begin{Phonetics}{推}{tui1}[][HSK 2]
    \definition{v.}{empurrar; dar um encontrão | girar um moinho ou uma pedra de amolar; moer | cortar; aparar | impulsionar; promover; avançar | inferir; deduzir | afastar; fugir; deslocar | adiar | eleger; escolher | ter em alta estima; elogiar muito | declinar | selecionar | elogiar muito}
  \end{Phonetics}
\end{Entry}

\begin{Entry}{推广}{11,3}{⼿、⼴}
  \begin{Phonetics}{推广}{tui1guang3}[][HSK 3]
    \definition{v.}{espalhar; estender; promover; popularizar; expandir o escopo de uso ou função de algo}
  \end{Phonetics}
\end{Entry}

\begin{Entry}{推介}{11,4}{⼿、⼈}
  \begin{Phonetics}{推介}{tui1jie4}
    \definition{s.}{promoção}
    \definition{v.}{promover | introduzir e recomendar}
  \end{Phonetics}
\end{Entry}

\begin{Entry}{推开}{11,4}{⼿、⼶}
  \begin{Phonetics}{推开}{tui1 kai1}[][HSK 3]
    \definition{v.}{declinar; rejeitar | empurrar para longe; aplicar força em uma determinada direção para mover uma pessoa ou objeto para longe de seu lugar original | empurrar para abrir (um portão, etc.); empurrar para fora para abrir algo que está fechado | estender; popularizar; promover para um alcance mais amplo e realizar em uma escala mais ampla}
  \end{Phonetics}
\end{Entry}

\begin{Entry}{推出}{11,5}{⼿、⼐}
  \begin{Phonetics}{推出}{tui1 chu1}[][HSK 6]
    \definition{v.}{lançar; apresentar; fazer com que apareça diante do público | deduzir; tirar conclusões da análise}
  \end{Phonetics}
\end{Entry}

\begin{Entry}{推动}{11,6}{⼿、⼒}
  \begin{Phonetics}{推动}{tui1 dong4}[][HSK 3]
    \definition{v.}{promover; atuar; impulsionar; empurrar para a frente; dar ímpeto a; começar ou avançar algo (com alguma força); começar a trabalhar}
  \end{Phonetics}
\end{Entry}

\begin{Entry}{推行}{11,6}{⼿、⾏}
  \begin{Phonetics}{推行}{tui1 xing2}[][HSK 5]
    \definition{v.}{realizar; prosseguir; praticar | implementar; praticar; implementação generalizada; divulgar (experiências, métodos, etc.)}
  \end{Phonetics}
\end{Entry}

\begin{Entry}{推进}{11,7}{⼿、⾡}
  \begin{Phonetics}{推进}{tui1 jin4}[][HSK 3]
    \definition{v.}{avançar; empurrar; levar adiante; dar ímpeto a; promover o trabalho e fazê-lo avançar | empurrar; dirigir; avançar; seguir em frente; seguir em frente}
  \end{Phonetics}
\end{Entry}

\begin{Entry}{推迟}{11,7}{⼿、⾡}
  \begin{Phonetics}{推迟}{tui1chi2}[][HSK 4]
    \definition{v.}{adiar; postergar; tardar; deixar para mais tarde}
  \end{Phonetics}
\end{Entry}

\begin{Entry}{推销}{11,12}{⼿、⾦}
  \begin{Phonetics}{推销}{tui1xiao1}[][HSK 4]
    \definition{v.}{vender; comercializar; promover vendas; promover a comercialização de mercadorias}
  \end{Phonetics}
\end{Entry}

\begin{Entry}{措}{11}{⼿}
  \begin{Phonetics}{措}{cuo4}
    \definition{s.}{iniciativa; solução; medida}
    \definition{v.}{organizar; gerenciar; lidar | fazer planos; administrar; organizar}
  \end{Phonetics}
\end{Entry}

\begin{Entry}{措手不及}{11,4,4,3}{⼿、⼿、⼀、⼃}
  \begin{Phonetics}{措手不及}{cuo4shou3-bu4ji2}[][HSK 7-9]
    \definition{expr.}{ser pego de surpresa; ser pego de surpresa (despreparado); ser tarde demais para fazer algo a respeito; ficar surpreso demais para se defender; não conseguir fazer uma defesa adequada; não conseguir pensar a tempo em uma maneira de se defender; não ter tempo para colocar em prática; surpreender alguém; pegar alguém desprevenido | ser pego desprevenido; ser pego de surpresa}
  \end{Phonetics}
\end{Entry}

\begin{Entry}{措施}{11,9}{⼿、⽅}
  \begin{Phonetics}{措施}{cuo4shi1}[][HSK 4]
    \definition[项,个]{s.}{medida; etapa; passo; abordagem adotada para lidar com as coisas}
  \end{Phonetics}
\end{Entry}

\begin{Entry}{掺}{11}{⼿}
  \begin{Phonetics}{掺}{can4}
    \definition{s.}{um estilo antigo de tocar bateria; uma antiga canção de tambor}
  \end{Phonetics}
  \begin{Phonetics}{掺}{chan1}[][HSK 7-9]
    \definition{v.}{misturar; mesclar; adicionar}
  \end{Phonetics}
  \begin{Phonetics}{掺}{shan3}
    \definition{v.}{misturar; mesclar | conter; reter}
  \end{Phonetics}
\end{Entry}

\begin{Entry}{描}{11}{⼿}
  \begin{Phonetics}{描}{miao2}
    \definition{v.}{traçar; copiar | retocar; retocar | traçar um desenho | retratar | esboçar}
  \end{Phonetics}
\end{Entry}

\begin{Entry}{描写}{11,5}{⼿、⼍}
  \begin{Phonetics}{描写}{miao2xie3}[][HSK 4]
    \definition{v.}{representar; retratar; descrever; usar a linguagem e as palavras para transmitir uma imagem concreta de uma pessoa, evento ou situação}
  \end{Phonetics}
\end{Entry}

\begin{Entry}{描述}{11,8}{⼿、⾡}
  \begin{Phonetics}{描述}{miao2 shu4}[][HSK 4]
    \definition[段,种]{s.}{descrição; trecho que descreve um evento ou uma cena}
    \definition{v.}{descrever; representar}
  \end{Phonetics}
\end{Entry}

\begin{Entry}{敎}{11}{⽁}
  \begin{Phonetics}{敎}{jiao4}
    \variantof{教}
  \end{Phonetics}
\end{Entry}

\begin{Entry}{敏}{11}{⽁}
  \begin{Phonetics}{敏}{min3}
    \definition*{s.}{Sobrenome Min}
    \definition{adj.}{rápido; ágil | perspicaz; inteligente; rápido | inteligente; esperto}
  \end{Phonetics}
\end{Entry}

\begin{Entry}{敏感}{11,13}{⽁、⼼}
  \begin{Phonetics}{敏感}{min3gan3}[][HSK 5]
    \definition{adj.}{sensível; descreve pessoas ou animais que rapidamente percebem mudanças ou estímulos externos | reativo; sensível; fácil de causar reações intensas}
  \end{Phonetics}
\end{Entry}

\begin{Entry}{救}{11}{⽁}
  \begin{Phonetics}{救}{jiu4}[][HSK 3]
    \definition*{s.}{Sobrenome Jiu}
    \definition{v.}{resgatar; salvar | salvar de; aliviar (angústia, etc.) | resgatar; livrar alguém de um desastre ou perigo | ajudar; aliviar; socorrer; livrar pessoas e coisas de desastres e perigos}
  \end{Phonetics}
\end{Entry}

\begin{Entry}{救出}{11,5}{⽁、⼐}
  \begin{Phonetics}{救出}{jiu4chu1}
    \definition{v.}{resgatar | tirar do perigo}
  \end{Phonetics}
\end{Entry}

\begin{Entry}{救护车}{11,7,4}{⽁、⼿、⾞}
  \begin{Phonetics}{救护车}{jiu4hu4che1}
    \definition[辆]{s.}{ambulância}
  \end{Phonetics}
\end{Entry}

\begin{Entry}{救灾}{11,7}{⽁、⽕}
  \begin{Phonetics}{救灾}{jiu4 zai1}[][HSK 5]
    \definition{v.}{ajudar as vítimas de desastres, aliviar o desastre; resgatar pessoas afetadas por desastres; recuperar danos causados por desastres}
  \end{Phonetics}
\end{Entry}

\begin{Entry}{救命}{11,8}{⽁、⼝}
  \begin{Phonetics}{救命}{jiu4/ming4}[][HSK 6]
    \definition{interj.}{Socorro!; Salve-me!}
    \definition{v.+compl.}{ajudar; salvar a vida de alguém}
  \end{Phonetics}
\end{Entry}

\begin{Entry}{救援}{11,12}{⽁、⼿}
  \begin{Phonetics}{救援}{jiu4 yuan2}[][HSK 6]
    \definition{v.}{resgatar; socorrer; vir em auxílio de alguém (resgate)}
  \end{Phonetics}
\end{Entry}

\begin{Entry}{教}{11}{⽁}
  \begin{Phonetics}{教}{jiao1}[][HSK 1]
    \definition*{s.}{Sobrenome Jiao}
    \definition{prep.}{em uma frase passiva para introduzir o executor da ação}
    \definition{s.}{religião | professor; referência à educação ou aos professores}
    \definition{v.}{ensinar; instruir |  pedir; ordenar; dizer | permitir; possibilitar}
  \end{Phonetics}
  \begin{Phonetics}{教}{jiao4}
    \definition*{s.}{Sobrenome Jiao}
    \definition{prep.}{em uma frase passiva para apresentar o autor da ação}
    \definition{s.}{religião | educação; professor}
    \definition{v.}{ensinar; instruir | perguntar; ordenar; contar | permitir; permitir}
  \end{Phonetics}
\end{Entry}

\begin{Entry}{教长}{11,4}{⽁、⾧}
  \begin{Phonetics}{教长}{jiao4zhang3}
    \definition{s.}{imã (Islã) | mulá}
  \end{Phonetics}
\end{Entry}

\begin{Entry}{教训}{11,5}{⽁、⾔}
  \begin{Phonetics}{教训}{jiao4xun4}[][HSK 4]
    \definition[个,次,番,顿]{s.}{moral; lição}
    \definition{v.}{repreender; educar; ensinar uma lição a alguém; dar uma bronca em alguém; dar um sermão em alguém (por ter cometido um erro, etc.)}
  \end{Phonetics}
\end{Entry}

\begin{Entry}{教会}{11,6}{⽁、⼈}
  \begin{Phonetics}{教会}{jiao1hui4}
    \definition{v.}{mostrar | ensinar}
  \end{Phonetics}
  \begin{Phonetics}{教会}{jiao4hui4}
    \definition{s.}{igreja cristã}
  \end{Phonetics}
\end{Entry}

\begin{Entry}{教导}{11,6}{⽁、⼨}
  \begin{Phonetics}{教导}{jiao4dao3}
    \definition{s.}{instrução | orientação | ensino}
    \definition{v.}{instruir | orientar | ensinar}
  \end{Phonetics}
\end{Entry}

\begin{Entry}{教师}{11,6}{⽁、⼱}
  \begin{Phonetics}{教师}{jiao4 shi1}[][HSK 2]
    \definition[个,位,名]{s.}{professor; professor de escola}
  \end{Phonetics}
\end{Entry}

\begin{Entry}{教材}{11,7}{⽁、⽊}
  \begin{Phonetics}{教材}{jiao4cai2}[][HSK 3]
    \definition[本,套]{s.}{livro didático; materiais didáticos, incluindo livros didáticos, apostilas, materiais de referência, vídeos, imagens, etc.}
  \end{Phonetics}
\end{Entry}

\begin{Entry}{教学}{11,8}{⽁、⼦}
  \begin{Phonetics}{教学}{jiao4 xue2}[][HSK 2]
    \definition[个,门]{s.}{ensino; educação; o processo de transmissão de conhecimentos e habilidades}
  \end{Phonetics}
\end{Entry}

\begin{Entry}{教学楼}{11,8,13}{⽁、⼦、⽊}
  \begin{Phonetics}{教学楼}{jiao4 xue2 lou2}[][HSK 1]
    \definition{s.}{prédio da escola; bloco de ensino; edifícios utilizados para atividades educacionais, geralmente incluindo salas de aula, laboratórios, auditórios, etc.}
  \end{Phonetics}
\end{Entry}

\begin{Entry}{教官}{11,8}{⽁、⼧}
  \begin{Phonetics}{教官}{jiao4guan1}
    \definition{s.}{instrutor militar; um oficial que serviu como treinador no antigo exército ou escola}
  \end{Phonetics}
\end{Entry}

\begin{Entry}{教练}{11,8}{⽁、⽷}
  \begin{Phonetics}{教练}{jiao4lian4}[][HSK 3]
    \definition[个,位,名]{s.}{instrutor; treinador (esportes); pessoas que trabalham como treinadores}
    \definition{v.}{treinar; treinar outras pessoas para dominarem uma determinada técnica (como esportes, dirigir carros, pilotar aviões, etc.)}
  \end{Phonetics}
\end{Entry}

\begin{Entry}{教育}{11,8}{⽁、⾁}
  \begin{Phonetics}{教育}{jiao4yu4}[][HSK 2]
    \definition{s.}{educação; refere-se a atividades sociais cujo objetivo direto é influenciar o desenvolvimento físico e mental das pessoas; refere-se principalmente ao processo de formação dos alunos nas escolas}
    \definition{v.}{ensinar; educar; inspirar, fazer compreender a razão}
  \end{Phonetics}
\end{Entry}

\begin{Entry}{教育部}{11,8,10}{⽁、⾁、⾢}
  \begin{Phonetics}{教育部}{jiao4 yu4 bu4}[][HSK 6]
    \definition*{s.}{Ministério da Educação}
  \end{Phonetics}
\end{Entry}

\begin{Entry}{教室}{11,9}{⽁、⼧}
  \begin{Phonetics}{教室}{jiao4shi4}[][HSK 2]
    \definition[间]{s.}{sala de aula}
  \end{Phonetics}
\end{Entry}

\begin{Entry}{教堂}{11,11}{⽁、⼟}
  \begin{Phonetics}{教堂}{jiao4tang2}[][HSK 6]
    \definition[座,所,间]{s.}{igreja; capela; catedral; casa de deus; um lugar onde os cristãos realizam cerimônias religiosas}
  \end{Phonetics}
\end{Entry}

\begin{Entry}{教授}{11,11}{⽁、⼿}
  \begin{Phonetics}{教授}{jiao4shou4}[][HSK 4]
    \definition[个,位,名]{s.}{professor (universitário); o professor com a classificação mais alta em uma universidade}
    \definition{v.}{ensinar; instruir; dar aulas; dar palestras}
  \end{Phonetics}
\end{Entry}

\begin{Entry}{敢}{11}{⽁}
  \begin{Phonetics}{敢}{gan3}[][HSK 3]
    \definition{adj.}{ousado; corajoso; audacioso; valente}
    \definition{adv.}{talvez; provavelmente}
    \definition{v.}{ser ousado o suficiente; atrever-se | ter confiança em; ter certeza; estar certo | aventurar-se; ter coragem de fazer algo | ser ousado; arriscar-se}
  \end{Phonetics}
\end{Entry}

\begin{Entry}{敢于}{11,3}{⽁、⼆}
  \begin{Phonetics}{敢于}{gan3 yu2}[][HSK 6]
    \definition{v.}{ousar; ser ousado em; ter determinação; ter coragem (para fazer ou se esforçar para fazer)}
  \end{Phonetics}
\end{Entry}

\begin{Entry}{敢情}{11,11}{⽁、⼼}
  \begin{Phonetics}{敢情}{gan3qing5}[][HSK 7-9]
    \definition{adv.}{por que; então; eu digo; indica a descoberta de algo que não foi descoberto anteriormente | claro; de fato; realmente; isso significa que a razão é óbvia e não há necessidade de duvidar dela}
  \end{Phonetics}
\end{Entry}

\begin{Entry}{斛}{11}{⽃}
  \begin{Phonetics}{斛}{hu2}
    \definition*{s.}{Sobrenome Hu}
    \definition{s.}{Arcaico: uma medida seca usada antigamente, originalmente igual a 10 dou (斗), mais tarde 5 dou}
  \seealsoref{斗}{dou4}
  \end{Phonetics}
\end{Entry}

\begin{Entry}{斜}{11}{⽃}
  \begin{Phonetics}{斜}{xie2}[][HSK 5]
    \definition{adj.}{oblíquo; inclinado | enviesado; chanfrado; diagonal; torto; nem paralelo nem perpendicular a um plano ou linha}
    \definition{v.}{virar de lado; inclinar}
  \end{Phonetics}
\end{Entry}

\begin{Entry}{斜阳}{11,6}{⽃、⾩}
  \begin{Phonetics}{斜阳}{xie2yang2}
    \definition{s.}{sol poente}
  \end{Phonetics}
\end{Entry}

\begin{Entry}{断}{11}{⽄}
  \begin{Phonetics}{断}{duan4}[][HSK 3]
    \definition*{s.}{Sobrenome Duan}
    \definition{adv.}{(geralmente na forma negativa) absolutamente; decididamente}
    \definition{v.}{quebrar; partir; (objetos longos) dividir em segmentos não conectados | parar; interromper; romper; isolar; fazer com que não se sucedam mais | desistir; abster-se de; parar de fumar, beber, etc. | julgar; decidir | interceptar}
  \end{Phonetics}
\end{Entry}

\begin{Entry}{断交}{11,6}{⽄、⼇}
  \begin{Phonetics}{断交}{duan4/jiao1}
    \definition{v.+compl.}{terminar uma amizade | romper relações diplomáticas}
  \end{Phonetics}
\end{Entry}

\begin{Entry}{断定}{11,8}{⽄、⼧}
  \begin{Phonetics}{断定}{duan4ding4}[][HSK 7-9]
    \definition{v.}{decidir; determinar; concluir; formar um julgamento; fazer um julgamento definitivo}
  \end{Phonetics}
\end{Entry}

\begin{Entry}{断断续续}{11,11,11,11}{⽄、⽄、⽷、⽷}
  \begin{Phonetics}{断断续续}{duan4duan4xu4xu4}[][HSK 7-9]
    \definition{adj.}{intermitente; esporádico; ocasional; aos trancos e barrancos}
  \end{Phonetics}
\end{Entry}

\begin{Entry}{断裂}{11,12}{⽄、⾐}
  \begin{Phonetics}{断裂}{duan4lie4}[][HSK 7-9]
    \definition{s.}{fraturar; quebrar; surtar | fender; rachar; romper}
  \end{Phonetics}
\end{Entry}

\begin{Entry}{旋}{11}{⽅}
  \begin{Phonetics}{旋}{xuan2}
    \definition*{s.}{Sobrenome Xuan}
    \definition{adv.}{em breve; rapidamente}
    \definition{s.}{redemoinho; turbilhão; vórtice}
    \definition{v.}{girar; circular; rodar | retornar; voltar}
  \end{Phonetics}
\end{Entry}

\begin{Entry}{旋转}{11,8}{⽅、⾞}
  \begin{Phonetics}{旋转}{xuan2zhuan3}[][HSK 6]
    \definition{v.}{girar; rodar; revolver; rodopiar; o movimento circular de um objeto em torno de um ponto ou eixo}
  \end{Phonetics}
\end{Entry}

\begin{Entry}{族}{11}{⽅}
  \begin{Phonetics}{族}{zu2}[][HSK 6]
    \definition{s.}{clã; família | uma pena de morte na China antiga, imposta ao infrator e a toda a sua família, ou mesmo às famílias de sua mãe e esposa; uma antiga forma de tortura |
etnia; nacionalidade | uma grande categoria de coisas que compartilham algum atributo comum}
  \end{Phonetics}
\end{Entry}

\begin{Entry}{旣}{11}{⽆}
  \begin{Phonetics}{旣}{ji4}
    \variantof{既}
  \end{Phonetics}
\end{Entry}

\begin{Entry}{晚}{11}{⽇}
  \begin{Phonetics}{晚}{wan3}[][HSK 1]
    \definition*{s.}{Sobrenome Wan}
    \definition{adj.}{tarde; tardio; passado o prazo acordado | júnior; mais jovem | mais tarde no tempo}
    \definition{s.}{noite; à noite; após o pôr do sol | últimos anos; última vida; um período posterior; refere-se especificamente à velhice de uma pessoa | pôr do sol; ao pôr do sol}
  \end{Phonetics}
\end{Entry}

\begin{Entry}{晚上}{11,3}{⽇、⼀}
  \begin{Phonetics}{晚上}{wan3shang5}[][HSK 1]
    \definition[个]{s.}{noite; o período entre o pôr do sol e a madrugada}
  \end{Phonetics}
\end{Entry}

\begin{Entry}{晚会}{11,6}{⽇、⼈}
  \begin{Phonetics}{晚会}{wan3hui4}[][HSK 2]
    \definition[场,个,次]{s.}{festa noturna; entretenimento noturno}
  \end{Phonetics}
\end{Entry}

\begin{Entry}{晚安}{11,6}{⽇、⼧}
  \begin{Phonetics}{晚安}{wan3'an1}[][HSK 2]
    \definition{expr.}{Tenha uma boa noite; uma frase educada usada para se despedir ou cumprimentar as pessoas à noite}
  \end{Phonetics}
\end{Entry}

\begin{Entry}{晚报}{11,7}{⽇、⼿}
  \begin{Phonetics}{晚报}{wan3 bao4}[][HSK 2]
    \definition[份,张]{s.}{jornal vespertino; um jornal publicado todas as tardes}
  \end{Phonetics}
\end{Entry}

\begin{Entry}{晚近}{11,7}{⽇、⾡}
  \begin{Phonetics}{晚近}{wan3jin4}
    \definition{adj.}{recente | mais recente no passado}
    \definition{adv.}{ultimamente | recentemente}
  \end{Phonetics}
\end{Entry}

\begin{Entry}{晚饭}{11,7}{⽇、⾷}
  \begin{Phonetics}{晚饭}{wan3 fan4}[][HSK 1]
    \definition[顿]{s.}{jantar}
  \end{Phonetics}
\end{Entry}

\begin{Entry}{晚育}{11,8}{⽇、⾁}
  \begin{Phonetics}{晚育}{wan3yu4}
    \definition{s.}{parto tardio}
    \definition{v.}{ter um filho mais tarde}
  \end{Phonetics}
\end{Entry}

\begin{Entry}{晚点}{11,9}{⽇、⽕}
  \begin{Phonetics}{晚点}{wan3 dian3}[][HSK 4]
    \definition{v.}{atrasar; adiar; (veículo, navio ou avião) partir, operar ou chegar mais tarde do que o horário especificado}
  \end{Phonetics}
\end{Entry}

\begin{Entry}{晚景}{11,12}{⽇、⽇}
  \begin{Phonetics}{晚景}{wan3jing3}
    \definition{s.}{circunstâncias dos anos de declínio de alguém | cena noturna}
  \end{Phonetics}
\end{Entry}

\begin{Entry}{晚餐}{11,16}{⽇、⾷}
  \begin{Phonetics}{晚餐}{wan3 can1}[][HSK 2]
    \definition[份,顿,次]{s.}{ceia; jantar}
  \end{Phonetics}
\end{Entry}

\begin{Entry}{望}{11}{⽉}
  \begin{Phonetics}{望}{wang4}
    \definition*{s.}{Sobrenome Wang}
    \definition{prep.}{para; em direção a; em ``olhando para frente, 望前看'', ``olhando para o leste, 望东走'', etc.; 望 é frequentemente escrito como 往}
    \definition{s.}{prestígio; reputação; fama | lua cheia | o 15º dia de um mês lunar}
    \definition{v.}{olhar por cima; olhar para a distância; olhar para longe na distância | visitar; ligar para | ter esperança; esperar | odiar; ressentir-se | pensar em atingir um determinado objetivo ou uma determinada situação em mente}
  \seealsoref{往}{wang3}
  \end{Phonetics}
\end{Entry}

\begin{Entry}{望见}{11,4}{⽉、⾒}
  \begin{Phonetics}{望见}{wang4 jian4}[][HSK 6]
    \definition{v.}{espiar; ver; pôr os olhos em | detectar}
  \end{Phonetics}
\end{Entry}

\begin{Entry}{梅}{11}{⽊}
  \begin{Phonetics}{梅}{mei2}
    \definition*{s.}{Sobrenome Mei}
    \definition{s.}{ameixa | flor de ameixa | ameixeira | estação chuvosa}
  \end{Phonetics}
\end{Entry}

\begin{Entry}{梅花}{11,7}{⽊、⾋}
  \begin{Phonetics}{梅花}{mei2 hua1}[][HSK 6]
    \definition[朵,枝,片,瓣,束,株]{s.}{paus ♣ (um naipe em jogos de cartas) | flor de ameixa | doçura-de-inverno; refere-se especificamente à flor-de-inverno ; também se refere a algo que se parece com esta flor}
  \seealsoref{方片}{fang1 pian4}
  \seealsoref{黑桃}{hei1 tao2}
  \seealsoref{红心}{hong2 xin1}
  \end{Phonetics}
\end{Entry}

\begin{Entry}{梅赛德斯-奔驰}{11,14,15,12,8,6}{⽊、⾙、⼻、⽄、⼤、⾺}
  \begin{Phonetics}{梅赛德斯-奔驰}{mei2sai4de2si1-ben1chi2}
    \definition*{s.}{Mercedes-Benz}
  \end{Phonetics}
\end{Entry}

\begin{Entry}{梦}{11}{⼣}
  \begin{Phonetics}{梦}{meng4}[][HSK 4]
    \definition*{s.}{Sobrenome Meng}
    \definition[个,场]{s.}{sonho; atividade de representação no cérebro durante o sono}
    \definition{v.}{sonhar; ter um sonho}
  \end{Phonetics}
\end{Entry}

\begin{Entry}{梦见}{11,4}{⼣、⾒}
  \begin{Phonetics}{梦见}{meng4 jian4}[][HSK 4]
    \definition{v.}{sonhar; sonhar com; ver em um sonho}
  \end{Phonetics}
\end{Entry}

\begin{Entry}{梦想}{11,13}{⼣、⼼}
  \begin{Phonetics}{梦想}{meng4xiang3}[][HSK 4]
    \definition[个,种,些,番]{s.}{sonho; esperança vã; sonho irreal; divagação; um desejo ou ideia que você espera particularmente realizar}
    \definition{v.}{sonhar; desejar sinceramente; ansiar}
  \end{Phonetics}
\end{Entry}

\begin{Entry}{梨}{11}{⽊}
  \begin{Phonetics}{梨}{li2}[][HSK 5]
    \definition*{s.}{Sobrenome Li}
    \definition[个,只,斤,棵,种]{s.}{perira; árvore de pera | pera}
  \end{Phonetics}
\end{Entry}

\begin{Entry}{梯}{11}{⽊}
  \begin{Phonetics}{梯}{ti1}
    \definition*{s.}{Sobrenome Ti}
    \definition{adj.}{em forma de escada; em socalcos}
    \definition[个]{s.}{escada; degrau; socalco (são plataformas niveladas, semelhantes a degraus, cortadas em encostas de morros para permitir o cultivo agrícola e evitar a erosão do solo)}
  \end{Phonetics}
\end{Entry}

\begin{Entry}{梯恩梯}{11,10,11}{⽊、⼼、⽊}
  \begin{Phonetics}{梯恩梯}{ti1'en1ti1}
    \definition{s.}{(empréstimo linguístico) TNT, trinitrotolueno}
  \end{Phonetics}
\end{Entry}

\begin{Entry}{检}{11}{⽊}
  \begin{Phonetics}{检}{jian3}
    \definition*{s.}{Sobrenome Jian}
    \definition{v.}{verificar; inspecionar; examinar | conter-se; ter cuidado na conduta}
  \end{Phonetics}
\end{Entry}

\begin{Entry}{检查}{11,9}{⽊、⽊}
  \begin{Phonetics}{检查}{jian3cha2}[][HSK 2]
    \definition[份,个,次]{s.}{autocrítica; reconhecer e criticar os próprios erros verbais ou escritos}
    \definition{v.}{verificar; inspecionar; examinar; verificar cuidadosamente para descobrir o problema | criticar a si mesmo; identificar seus pontos fracos e erros, e criticar seu próprio comportamento}
  \end{Phonetics}
\end{Entry}

\begin{Entry}{检测}{11,9}{⽊、⽔}
  \begin{Phonetics}{检测}{jian3 ce4}[][HSK 4]
    \definition{v.}{testar; detectar; verificar}
  \end{Phonetics}
\end{Entry}

\begin{Entry}{检验}{11,10}{⽊、⾺}
  \begin{Phonetics}{检验}{jian3yan4}[][HSK 5]
    \definition{v.}{testar; examinar; inspecionar}
  \end{Phonetics}
\end{Entry}

\begin{Entry}{欲}{11}{⽋}
  \begin{Phonetics}{欲}{yu4}
    \definition{adj.}{desejo | apetite | paixão | luxúria | ganância}
    \definition{v.}{desejar}
  \end{Phonetics}
\end{Entry}

\begin{Entry}{毫}{11}{⽊}
  \begin{Phonetics}{毫}{hao2}
    \definition{adv.}{nem um pouco; absolutamente nenhum; completamente sem}
    \definition{clas.}{hao, uma unidade de comprimento igual a um milésimo de polegada ou 1/30 de milímetro | hao, uma unidade de peso igual a um milésimo de um centavo ou 0,005 grama |
uma fração minúscula; uma parte muito pequena}
    \definition{pref.}{mili-, usado com a unidade de uma quantidade física para representar um milésimo dessa quantidade}
    \definition{s.}{cabelo longo e fino | pincel de escrita | uma das duas ou três alças de uma balança para pendurar na mão do usuário | cerda; uma corda de mão em uma balança ou equilíbrio | fio de cabelo}
  \end{Phonetics}
\end{Entry}

\begin{Entry}{毫不}{11,4}{⽊、⼀}
  \begin{Phonetics}{毫不}{hao2 bu4}[][HSK 7-9]
    \definition{adv.}{dificilmente; de ​​jeito nenhum; nem um pouco}
  \end{Phonetics}
\end{Entry}

\begin{Entry}{毫不犹豫}{11,4,7,15}{⽊、⼀、⽝、⾗}
  \begin{Phonetics}{毫不犹豫}{hao2 bu4 you2yu4}[][HSK 7-9]
    \definition{expr.}{sem hesitação; sem a menor hesitação}
  \end{Phonetics}
\end{Entry}

\begin{Entry}{毫不费力}{11,4,9,2}{⽊、⼀、⾙、⼒}
  \begin{Phonetics}{毫不费力}{hao2bu2fei4li4}
    \definition{expr.}{sem esforço; não despender o menor esforço}
  \end{Phonetics}
\end{Entry}

\begin{Entry}{毫升}{11,4}{⽊、⼗}
  \begin{Phonetics}{毫升}{hao2 sheng1}[][HSK 4]
    \definition{clas.}{mililitro; unidade de volume, milésimo de um litro (ml)}
  \end{Phonetics}
\end{Entry}

\begin{Entry}{毫无}{11,4}{⽊、⽆}
  \begin{Phonetics}{毫无}{hao2wu2}[][HSK 7-9]
    \definition{adv.}{não; nada; de jeito nenhum}
  \end{Phonetics}
\end{Entry}

\begin{Entry}{毫米}{11,6}{⽊、⽶}
  \begin{Phonetics}{毫米}{hao2mi3}[][HSK 4]
    \definition{clas.}{milímetro; unidade legal de medida de comprimento, 1 mm equivale a 0,1 cm}
  \end{Phonetics}
\end{Entry}

\begin{Entry}{液}{11}{⽔}
  \begin{Phonetics}{液}{ye4}
    \definition{s.}{líquido; fluido; suco}
  \end{Phonetics}
\end{Entry}

\begin{Entry}{液体}{11,7}{⽔、⼈}
  \begin{Phonetics}{液体}{ye4ti3}
    \definition{adj./s.}{líquido}
  \end{Phonetics}
\end{Entry}

\begin{Entry}{涵}{11}{⽔}
  \begin{Phonetics}{涵}{han2}
    \definition{s.}{bueiro; galeria}
    \definition{v.}{conter; incorporar}
  \end{Phonetics}
\end{Entry}

\begin{Entry}{涵义}{11,3}{⽔、⼂}
  \begin{Phonetics}{涵义}{han2yi4}[][HSK 7-9]
    \definition[层,种]{s.}{significado; implicação | conotação | conteúdo}
  \end{Phonetics}
\end{Entry}

\begin{Entry}{涵盖}{11,11}{⽔、⽫}
  \begin{Phonetics}{涵盖}{han2gai4}[][HSK 7-9]
    \definition{v.}{cobrir; incluir; conter; conter completamente}
  \end{Phonetics}
\end{Entry}

\begin{Entry}{淀}{11}{⽔}
  \begin{Phonetics}{淀}{dian4}
    \definition{s.}{lago raso, frequentemente usado em nomes de lugares}
    \definition{v.}{formar sedimentos | sedimentar; precipitar}
  \end{Phonetics}
\end{Entry}

\begin{Entry}{淀粉}{11,10}{⽔、⽶}
  \begin{Phonetics}{淀粉}{dian4fen3}[][HSK 7-9]
    \definition[包,勺,克,的]{s.}{amido; amilo; os carboidratos são os principais componentes das sementes, raízes e tubérculos}
  \end{Phonetics}
\end{Entry}

\begin{Entry}{淋}{11}{⽔}
  \begin{Phonetics}{淋}{lin2}
    \definition{v.}{borrifar | pingar | derramar | encharcar}
  \end{Phonetics}
  \begin{Phonetics}{淋}{lin4}
    \definition{s.}{gonorréia}
    \definition{v.}{filtrar | coar | drenar}
  \end{Phonetics}
\end{Entry}

\begin{Entry}{淡}{11}{⽔}
  \begin{Phonetics}{淡}{dan4}[][HSK 4]
    \definition*{s.}{Sobrenome Dan}
    \definition{adj.}{sem gosto; fraco; não tem sabor forte; não é salgado | leve; fraco; pálido | indiferente; frio; sem entusiasmo | frouxo; sem brilho | sem sentido; trivial | fino; leve}
  \end{Phonetics}
\end{Entry}

\begin{Entry}{淡化}{11,4}{⽔、⼔}
  \begin{Phonetics}{淡化}{dan4hua4}[][HSK 7-9]
    \definition{v.}{dessalinizar; transformar água com alto teor de sal em água com baixo teor de sal | desaparecer; enfraquecer; tornar ou tornar-se menos importante}
  \end{Phonetics}
\end{Entry}

\begin{Entry}{淡季}{11,8}{⽔、⼦}
  \begin{Phonetics}{淡季}{dan4ji4}[][HSK 7-9]
    \definition{s.}{baixa temporada; temporada fraca (ou monótona, fora de temporada) | uma estação em que a produção de um determinado produto é baixa; uma estação em que os negócios estão lentos (diferente da 旺季)}
  \seealsoref{旺季}{wang4ji4}
  \end{Phonetics}
\end{Entry}

\begin{Entry}{淤}{11}{⽔}
  \begin{Phonetics}{淤}{yu1}
    \definition{adj.}{assoreado}
    \definition{s.}{lodo}
    \definition[出]{s.}{(medicina chinesa) estase de sangue}
    \definition{v.}{ficar assoreado; ficar sufocado com lodo | derramar; transbordar}
  \end{Phonetics}
\end{Entry}

\begin{Entry}{淤泥}{11,8}{⽔、⽔}
  \begin{Phonetics}{淤泥}{yu1ni2}
    \definition{s.}{lodo}
  \end{Phonetics}
\end{Entry}

\begin{Entry}{深}{11}{⽔}
  \begin{Phonetics}{深}{shen1}[][HSK 3]
    \definition*{s.}{Sobrenome Shen}
    \definition{adj.}{profundo | difícil; intenso; profundo | completo; penetrante; intenso; profundo | próximo; íntimo; afeição profunda; relacionamento próximo | escuro; profundo | tardio}
    \definition{adv.}{muito; grandemente; profundamente}
    \definition{s.}{profundidade}
  \seealsoref{浅}{qian3}
  \end{Phonetics}
\end{Entry}

\begin{Entry}{深入}{11,2}{⽔、⼊}
  \begin{Phonetics}{深入}{shen1 ru4}[][HSK 3]
    \definition{adj.}{profundo; completo}
    \definition{v.}{ir fundo em; penetrar em; penetrar o exterior; alcançar o interior ou o centro de algo}
  \end{Phonetics}
\end{Entry}

\begin{Entry}{深化}{11,4}{⽔、⼔}
  \begin{Phonetics}{深化}{shen1 hua4}[][HSK 6]
    \definition{v.}{aprofundar; avançar; intensificar; tornar-se mais profundo; tornar mais profundo}
  \end{Phonetics}
\end{Entry}

\begin{Entry}{深处}{11,5}{⽔、⼡}
  \begin{Phonetics}{深处}{shen1 chu4}[][HSK 5]
    \definition{s.}{profundidades; recantos; recessos | profundezas}
  \end{Phonetics}
\end{Entry}

\begin{Entry}{深刻}{11,8}{⽔、⼑}
  \begin{Phonetics}{深刻}{shen1ke4}[][HSK 3]
    \definition{adj.}{profundo; instenso; chegar à essência de um assunto ou problema}
  \end{Phonetics}
\end{Entry}

\begin{Entry}{深夜}{11,8}{⽔、⼣}
  \begin{Phonetics}{深夜}{shen1ye4}
    \definition{adv.}{tarde da noite}
  \end{Phonetics}
\end{Entry}

\begin{Entry}{深厚}{11,9}{⽔、⼚}
  \begin{Phonetics}{深厚}{shen1hou4}[][HSK 4]
    \definition{adj.}{profundo; sentimentos fortes | sólido; profundamente enraizado; fundação sólida}
  \end{Phonetics}
\end{Entry}

\begin{Entry}{深度}{11,9}{⽔、⼴}
  \begin{Phonetics}{深度}{shen1 du4}[][HSK 5]
    \definition{adj.}{(em grau ou extensão) profundo; sério; grave}
    \definition{s.}{profundidade; grau de profundidade; | profundidade; rigor; meticulosidade; grau de contato com a essência das coisas | estágio avançado (ou em deterioração) de desenvolvimento; grau de crescimento e desenvolvimento das coisas}
  \end{Phonetics}
\end{Entry}

\begin{Entry}{深深}{11,11}{⽔、⽔}
  \begin{Phonetics}{深深}{shen1 shen1}[][HSK 6]
    \definition{adj.}{profundo; intenso}
    \definition{adv.}{profundamente; intensamente; descreve um grau profundo ou forte}
  \end{Phonetics}
\end{Entry}

\begin{Entry}{混}{11}{⽔}
  \begin{Phonetics}{混}{hun4}[][HSK 6]
    \definition{adj.}{confuso; imundo; turvo; lamacento; impuro}
    \definition{adv.}{de forma imprudente; irresponsável; irrefletidamente}
    \definition{v.}{misturar; confundir; misturar verdadeiro e falso | passar por; esgueirar-se | vagar à deriva; arrastar-se; sobreviver de maneira superficial; contentar-se com | se dar bem com alguém}
  \end{Phonetics}
\end{Entry}

\begin{Entry}{混合}{11,6}{⽔、⼝}
  \begin{Phonetics}{混合}{hun4he2}[][HSK 6]
    \definition{s.}{híbrido; composto; refere-se a duas ou mais substâncias misturadas sem reação química, mas ainda mantendo suas respectivas propriedades (diferente de 化合)}
    \definition{v.}{misturar; mixar; misturar-se}
  \seealsoref{化合}{hua4he2}
  \end{Phonetics}
\end{Entry}

\begin{Entry}{混乱}{11,7}{⽔、⼄}
  \begin{Phonetics}{混乱}{hun4luan4}[][HSK 6]
    \definition{adj.}{caótico; confuso; desordenado; desorganizado; fora de ordem}
    \definition[片]{s.}{caos; confusão}
  \end{Phonetics}
\end{Entry}

\begin{Entry}{混饭}{11,7}{⽔、⾷}
  \begin{Phonetics}{混饭}{hun4/fan4}
    \definition{v.+compl.}{trabalhar para viver}
  \end{Phonetics}
\end{Entry}

\begin{Entry}{添}{11}{⽔}
  \begin{Phonetics}{添}{tian1}[][HSK 6]
    \definition{v.}{adicionar; aumentar | dar à luz}
  \end{Phonetics}
\end{Entry}

\begin{Entry}{清}{11}{⽔}
  \begin{Phonetics}{清}{qing1}[][HSK 6]
    \definition*{s.}{Dinastia Qing (1644-1911) | Sobrenome Qing}
    \definition{adj.}{claro; não misturado; (líquido ou gasoso) puro e sem mistura (em oposição a 浊) | silencioso; quieto | justo e honesto | distinto; claro; esclarecido | simples; puro, sem qualquer adulteração ou combinação | limpo; puro}
    \definition{v.}{limpar; tornar limpo | resolver; esclarecer; pagar; liquidar | contar; inspecionar}
  \seealsoref{浊}{zhuo2}
  \end{Phonetics}
\end{Entry}

\begin{Entry}{清彻}{11,7}{⽔、⼻}
  \begin{Phonetics}{清彻}{qing1che4}
    \variantof{清澈}
  \end{Phonetics}
\end{Entry}

\begin{Entry}{清明节}{11,8,5}{⽔、⽇、⾋}
  \begin{Phonetics}{清明节}{qing1 ming2 jie2}[][HSK 6]
    \definition*{s.}{Qingming ou Festival do Brilho Puro ou Dia da Varredura de Túmulos, Dia dos Finados (uma das 24~divisões do ano solar no calendário lunar chinês:~dia~4 ou 5~de~abril solar)}
  \end{Phonetics}
\end{Entry}

\begin{Entry}{清洁}{11,9}{⽔、⽔}
  \begin{Phonetics}{清洁}{qing1jie2}[][HSK 6]
    \definition{adj.}{limpo; sem poeira, gordura, etc.}
    \definition{v.}{limpar}
  \end{Phonetics}
\end{Entry}

\begin{Entry}{清洁工}{11,9,3}{⽔、⽔、⼯}
  \begin{Phonetics}{清洁工}{qing1 jie2 gong1}[][HSK 6]
    \definition{s.}{coletor de lixo; trabalhador de saneamento; limpador de rua; trabalhadores envolvidos na limpeza do ambiente, remoção de lixo e fezes, etc.}
  \end{Phonetics}
\end{Entry}

\begin{Entry}{清洗}{11,9}{⽔、⽔}
  \begin{Phonetics}{清洗}{qing1 xi3}[][HSK 6]
    \definition{v.}{enxaguar; lavar; limpar | purgar; limpar | eliminar}
  \end{Phonetics}
\end{Entry}

\begin{Entry}{清凉}{11,10}{⽔、⼎}
  \begin{Phonetics}{清凉}{qing1liang2}
    \definition{adj.}{fresco | refrescante | (roupa) ousada, reveladora}
  \end{Phonetics}
\end{Entry}

\begin{Entry}{清唱}{11,11}{⽔、⼝}
  \begin{Phonetics}{清唱}{qing1chang4}
    \definition{v.}{cantar à capela}
  \end{Phonetics}
\end{Entry}

\begin{Entry}{清晨}{11,11}{⽔、⽇}
  \begin{Phonetics}{清晨}{qing1chen2}[][HSK 5]
    \definition{s.}{matinal; manhã cedo; geralmente se refere ao período do amanhecer até logo após o nascer do sol}
  \end{Phonetics}
\end{Entry}

\begin{Entry}{清爽}{11,11}{⽔、⽘}
  \begin{Phonetics}{清爽}{qing1shuang3}
    \definition{adj.}{refrescante | relaxado}
  \end{Phonetics}
\end{Entry}

\begin{Entry}{清理}{11,11}{⽔、⽟}
  \begin{Phonetics}{清理}{qing1li3}[][HSK 5]
    \definition{v.}{esclarecer; resolver; verificar; colocar em ordem; organizar tudo e jogar fora o que não for útil}
  \end{Phonetics}
\end{Entry}

\begin{Entry}{清晰}{11,12}{⽔、⽇}
  \begin{Phonetics}{清晰}{qing1xi1}
    \definition{adj.}{claro | distinto}
  \end{Phonetics}
\end{Entry}

\begin{Entry}{清楚}{11,13}{⽔、⽊}
  \begin{Phonetics}{清楚}{qing1chu5}[][HSK 2]
    \definition{adj.}{claro; distinto; compreensível; organizado; fácil de identificar e entender | plenamente consciente de; claro sobre}
    \definition{v.}{ter clareza sobre; compreender; ação que expressa compreensão e conhecimento}
  \end{Phonetics}
\end{Entry}

\begin{Entry}{清澈}{11,15}{⽔、⽔}
  \begin{Phonetics}{清澈}{qing1che4}
    \definition{adj.}{claro | límpido}
  \end{Phonetics}
\end{Entry}

\begin{Entry}{清醒}{11,16}{⽔、⾣}
  \begin{Phonetics}{清醒}{qing1xing3}[][HSK 4]
    \definition{adj.}{sóbrio; lúcido}
    \definition{v.}{recuperar a consciência; recuperar-se de um coma}
  \end{Phonetics}
\end{Entry}

\begin{Entry}{渐}{11}{⽔}
  \begin{Phonetics}{渐}{jian1}
    \definition{v.}{encharcar; ficar saturado com | fluir para}
  \end{Phonetics}
  \begin{Phonetics}{渐}{jian4}
    \definition{adv.}{gradualmente; por graus}
  \end{Phonetics}
\end{Entry}

\begin{Entry}{渐渐}{11,11}{⽔、⽔}
  \begin{Phonetics}{渐渐}{jian4 jian4}[][HSK 4]
    \definition{adv.}{gradualmente; pouco a pouco; passo a passo; indica um aumento ou diminuição gradual em grau ou quantidade}
  \end{Phonetics}
\end{Entry}

\begin{Entry}{渔}{11}{⽔}
  \begin{Phonetics}{渔}{yu2}
    \definition[条]{s.}{pescador}
    \definition{v.}{pescar}
  \end{Phonetics}
\end{Entry}

\begin{Entry}{渔夫}{11,4}{⽔、⼤}
  \begin{Phonetics}{渔夫}{yu2fu1}
    \definition{s.}{pescador}
  \end{Phonetics}
\end{Entry}

\begin{Entry}{渔民}{11,5}{⽔、⽒}
  \begin{Phonetics}{渔民}{yu2min2}
    \definition{s.}{pescadores | povo pescador}
  \end{Phonetics}
\end{Entry}

\begin{Entry}{渔场}{11,6}{⽔、⼟}
  \begin{Phonetics}{渔场}{yu2chang3}
    \definition{s.}{área de pesca}
  \end{Phonetics}
\end{Entry}

\begin{Entry}{渔汛}{11,6}{⽔、⽔}
  \begin{Phonetics}{渔汛}{yu2xun4}
    \definition{s.}{temporada de pesca}
  \end{Phonetics}
\end{Entry}

\begin{Entry}{渔网}{11,6}{⽔、⽹}
  \begin{Phonetics}{渔网}{yu2wang3}
    \definition{s.}{rede de pesca | tresmalho}
  \end{Phonetics}
\end{Entry}

\begin{Entry}{渔具}{11,8}{⽔、⼋}
  \begin{Phonetics}{渔具}{yu2ju4}
    \definition{s.}{equipamento de pesca}
  \end{Phonetics}
\end{Entry}

\begin{Entry}{渔轮}{11,8}{⽔、⾞}
  \begin{Phonetics}{渔轮}{yu2lun2}
    \definition{s.}{navio de pesca}
  \end{Phonetics}
\end{Entry}

\begin{Entry}{渔捞}{11,10}{⽔、⼿}
  \begin{Phonetics}{渔捞}{yu2lao1}
    \definition{s.}{pesca (como atividade comercial)}
  \end{Phonetics}
\end{Entry}

\begin{Entry}{渔猎}{11,11}{⽔、⽝}
  \begin{Phonetics}{渔猎}{yu2lie4}
    \definition{s.}{pesca e caça}
    \definition{v.}{saquear | pilhar}
  \end{Phonetics}
\end{Entry}

\begin{Entry}{渔笼}{11,11}{⽔、⽵}
  \begin{Phonetics}{渔笼}{yu2long2}
    \definition{s.}{gaiola de pesca | armadilha de pesca}
  \end{Phonetics}
\end{Entry}

\begin{Entry}{渔船}{11,11}{⽔、⾈}
  \begin{Phonetics}{渔船}{yu2chuan2}
    \definition[条]{s.}{barco de pesca}
  \seealsoref{鱼船}{yu2chuan2}
  \end{Phonetics}
\end{Entry}

\begin{Entry}{渔船队}{11,11,4}{⽔、⾈、⾩}
  \begin{Phonetics}{渔船队}{yu2chuan2 dui4}
    \definition{s.}{frota pesqueira}
  \end{Phonetics}
\end{Entry}

\begin{Entry}{渠}{11}{⽊}
  \begin{Phonetics}{渠}{qu2}
    \definition*{s.}{Sobrenome Qu}
    \definition{adj.}{Literário: grande}
    \definition{pron.}{Dialeto: ele; ela}
    \definition[条]{s.}{canal; vala; fosso; trincheira | borda externa da roda | escudo}
  \end{Phonetics}
\end{Entry}

\begin{Entry}{渠道}{11,12}{⽊、⾡}
  \begin{Phonetics}{渠道}{qu2dao4}[][HSK 6]
    \definition[条,个,种]{s.}{vala de irrigação; os cursos de água escavados pelos trabalhadores para drenagem e irrigação | maneira; meio; caminho}
  \end{Phonetics}
\end{Entry}

\begin{Entry}{焊}{11}{⽕}
  \begin{Phonetics}{焊}{han4}[][HSK 7-9]
    \definition{v.}{soldar; usar metal fundido para reparar objetos de metal ou conectar peças de metal}
  \end{Phonetics}
\end{Entry}

\begin{Entry}{爽}{11}{⽘}
  \begin{Phonetics}{爽}{shuang3}[][HSK 6]
    \definition{adj.}{claro; nítido; brilhante | franco; de coração aberto; direto | relaxado; confortável}
    \definition{v.}{desviar; afastar | tornar confortável; ficar confortável}
  \end{Phonetics}
\end{Entry}

\begin{Entry}{猎}{11}{⽝}
  \begin{Phonetics}{猎}{lie4}
    \definition[个]{s.}{traje de caça}
    \definition{v.}{caçar | procurar; perseguir}
  \end{Phonetics}
\end{Entry}

\begin{Entry}{猎物}{11,8}{⽝、⽜}
  \begin{Phonetics}{猎物}{lie4wu4}
    \definition{s.}{presa (vítima de um predador)}
  \end{Phonetics}
\end{Entry}

\begin{Entry}{猖}{11}{⽝}
  \begin{Phonetics}{猖}{chang1}
    \definition{adj.}{louco; indisciplinado; dissoluto; licencioso; precipitado; imprudente | Literário: feroz}
  \end{Phonetics}
\end{Entry}

\begin{Entry}{猖狂}{11,7}{⽝、⽝}
  \begin{Phonetics}{猖狂}{chang1kuang2}[][HSK 7-9]
    \definition{adj.}{selvagem; desenfreado; furioso; imprudente; arrogante e presunçoso}
  \end{Phonetics}
\end{Entry}

\begin{Entry}{猛}{11}{⽝}
  \begin{Phonetics}{猛}{meng3}[][HSK 6]
    \definition*{s.}{Sobrenome Meng}
    \definition{adj.}{feroz; violento | enérgico; vigoroso | valente}
    \definition{adv.}{de repente; abruptamente | vigorosamente; com força repentina | (coloquial) ao contentamento do coração; de todo o coração | ferozmente; violentamente}
  \end{Phonetics}
\end{Entry}

\begin{Entry}{猛然}{11,12}{⽝、⽕}
  \begin{Phonetics}{猛然}{meng3ran2}
    \definition{adv.}{de repente; abruptamente; indica ação repentina e rápida}
  \end{Phonetics}
\end{Entry}

\begin{Entry}{猜}{11}{⽝}
  \begin{Phonetics}{猜}{cai1}[][HSK 5]
    \definition{v.}{adivinhar; conjecturar; especular | suspeitar; ser cauteloso com os outros; desconfiar dos outros}
  \end{Phonetics}
\end{Entry}

\begin{Entry}{猜忌}{11,7}{⽝、⼼}
  \begin{Phonetics}{猜忌}{cai1 ji4}
    \definition{v.}{ser desconfiado e invejoso | ser desconfiado e ciumento de}
  \end{Phonetics}
\end{Entry}

\begin{Entry}{猜测}{11,9}{⽝、⽔}
  \begin{Phonetics}{猜测}{cai1 ce4}[][HSK 5]
    \definition[个,种]{s.}{advinhação; conjectura; suposição; especulação}
    \definition{v.}{adivinhar; conjecturar; especular; estimar a partir da imaginação}
  \end{Phonetics}
\end{Entry}

\begin{Entry}{猜谜}{11,11}{⽝、⾔}
  \begin{Phonetics}{猜谜}{cai1 mi2}[][HSK 7-9]
    \definition{v.}{adivinhar um enigma}
  \end{Phonetics}
\end{Entry}

\begin{Entry}{猜想}{11,13}{⽝、⼼}
  \begin{Phonetics}{猜想}{cai1xiang3}[][HSK 7-9]
    \definition{s.}{suposição; conjectura; palpite; especulação}
    \definition{v.}{supor; adivinhar; suspeitar}
  \end{Phonetics}
\end{Entry}

\begin{Entry}{猜疑}{11,14}{⽝、⽦}
  \begin{Phonetics}{猜疑}{cai1 yi2}
    \definition{v.}{abrigar suspeitas; ser desconfiado; ter receios; levantar suspeitas do nada}
  \end{Phonetics}
\end{Entry}

\begin{Entry}{猪}{11}{⽝}
  \begin{Phonetics}{猪}{zhu1}[][HSK 3]
    \definition[头,只,口]{s.}{porco; suíno}
  \end{Phonetics}
\end{Entry}

\begin{Entry}{猪头}{11,5}{⽝、⼤}
  \begin{Phonetics}{猪头}{zhu1tou2}
    \definition{s.}{tolo | idiota}
  \end{Phonetics}
\end{Entry}

\begin{Entry}{猪柳}{11,9}{⽝、⽊}
  \begin{Phonetics}{猪柳}{zhu1liu3}
    \definition{s.}{filé de porco}
  \end{Phonetics}
\end{Entry}

\begin{Entry}{猪笼}{11,11}{⽝、⽵}
  \begin{Phonetics}{猪笼}{zhu1long2}
    \definition{s.}{estrutura cilíndrica de bambu ou arame usada para restringir um porco durante o transporte}
  \end{Phonetics}
\end{Entry}

\begin{Entry}{猪窠}{11,13}{⽝、⽳}
  \begin{Phonetics}{猪窠}{zhu1ke1}
    \definition{s.}{chiqueiro}
  \end{Phonetics}
\end{Entry}

\begin{Entry}{猫}{11}{⽝}
  \begin{Phonetics}{猫}{mao1}[][HSK 2]
    \definition*[只,种,群,窝,个]{s.}{gato |  Empréstimo linguístico: MODEM}
    \definition{v.}{esconder-se; entrar em esconderijo | inclinar-se para a frente; curvar-se}
  \end{Phonetics}
  \begin{Phonetics}{猫}{mao2}
    \definition{v.}{utilizado em 猫腰 \dpy{mao2yao1}}
  \seealsoref{猫腰}{mao2yao1}
  \end{Phonetics}
\end{Entry}

\begin{Entry}{猫腰}{11,13}{⽝、⾁}
  \begin{Phonetics}{猫腰}{mao2yao1}
    \definition{v.}{curvar-se}
  \end{Phonetics}
\end{Entry}

\begin{Entry}{猫熊}{11,14}{⽝、⽕}
  \begin{Phonetics}{猫熊}{mao1xiong2}
    \definition[把,只]{s.}{panda gigante}
  \seealsoref{熊猫}{xiong2mao1}
  \end{Phonetics}
\end{Entry}

\begin{Entry}{率}{11}{⽞}
  \begin{Phonetics}{率}{lv4}
    \definition{s.}{taxa; razão; proporção; a relação proporcional entre duas grandezas relacionadas}
  \end{Phonetics}
  \begin{Phonetics}{率}{shuai4}
    \definition*{s.}{Sobrenome Shuai}
    \definition{adj.}{precipitado; não cuidadoso; não cauteloso | franco; direto | elegante; bonito; o mesmo que 帅}
    \definition{adv.}{geralmente; expressa uma estimativa incerta, equivalente a 大约 e 大抵}
    \definition{s.}{modelo; exemplo}
    \definition{v.}{liderar; comandar | obedecer; seguir}
  \seealsoref{大抵}{da4di3}
  \seealsoref{大约}{da4yue1}
  \seealsoref{帅}{shuai4}
  \end{Phonetics}
\end{Entry}

\begin{Entry}{率先}{11,6}{⽞、⼉}
  \begin{Phonetics}{率先}{shuai4 xian1}[][HSK 4]
    \definition{v.}{tomar a iniciativa de fazer algo; ser o primeiro a fazer algo; assumir a liderança}
  \end{Phonetics}
\end{Entry}

\begin{Entry}{率领}{11,11}{⽞、⾴}
  \begin{Phonetics}{率领}{shuai4ling3}[][HSK 5]
    \definition{v.}{liderar (equipe ou grupo); chefiar; comandar}
  \end{Phonetics}
\end{Entry}

\begin{Entry}{球}{11}{⽟}
  \begin{Phonetics}{球}{qiu2}[][HSK 1]
    \definition[个,颗,筐]{s.}{esfera; globo; equipamento de jogo antigo, objeto tridimensional circular, feito de couro, recheado com penas, para ser chutado com os pés ou batido com um bastão | qualquer coisa com formato de bola; algo esférico ou quase esférico | bola; refere-se a certos artigos esportivos (geralmente redondos e tridimensionais) | jogo; partida; referência a esportes com bola | o Globo; a Terra; referindo-se especificamente à Terra}
  \end{Phonetics}
\end{Entry}

\begin{Entry}{球队}{11,4}{⽟、⾩}
  \begin{Phonetics}{球队}{qiu2 dui4}[][HSK 2]
    \definition[个,支]{s.}{equipe (basquete, futebol, etc.); equipe de atletas formada para competições esportivas com bola, como times de basquete, futebol, etc.}
  \end{Phonetics}
\end{Entry}

\begin{Entry}{球场}{11,6}{⽟、⼟}
  \begin{Phonetics}{球场}{qiu2 chang3}[][HSK 2]
    \definition[个,座]{s.}{quadra; campo; terreno para jogos com bola; campos para a prática de esportes com bola, como basquete, futebol, tênis e vôlei, cuja forma, tamanho e equipamentos variam de acordo com as exigências de cada esporte}
  \end{Phonetics}
\end{Entry}

\begin{Entry}{球员}{11,7}{⽟、⼝}
  \begin{Phonetics}{球员}{qiu2 yuan2}[][HSK 6]
    \definition[名,位,个]{s.}{Esporte: jogador | membro do clube esportivo}
  \end{Phonetics}
\end{Entry}

\begin{Entry}{球拍}{11,8}{⽟、⼿}
  \begin{Phonetics}{球拍}{qiu2 pai1}[][HSK 6]
    \definition[支]{s.}{(tênis, badminton, etc.) raquete}
  \end{Phonetics}
\end{Entry}

\begin{Entry}{球星}{11,9}{⽟、⽇}
  \begin{Phonetics}{球星}{qiu2 xing1}[][HSK 6]
    \definition[位,名]{s.}{estrela do esporte (esporte com bola)}
  \end{Phonetics}
\end{Entry}

\begin{Entry}{球迷}{11,9}{⽟、⾡}
  \begin{Phonetics}{球迷}{qiu2mi2}[][HSK 3]
    \definition[个,位,名,些]{s.}{fã (de esportes de bola); pessoas obcecadas por jogar ou assistir jogos de bola}
  \end{Phonetics}
\end{Entry}

\begin{Entry}{球鞋}{11,15}{⽟、⾰}
  \begin{Phonetics}{球鞋}{qiu2 xie2}[][HSK 2]
    \definition[双,只,款]{s.}{tênis de ginástica; tênis de tênis; tênis esportivos}
  \end{Phonetics}
\end{Entry}

\begin{Entry}{理}{11}{⽟}
  \begin{Phonetics}{理}{li3}[][HSK 6]
    \definition*{s.}{Sobrenome Li}
    \definition{s.}{textura; grão (em madeira, pele, etc.) | ordem; sequência | razão; lógica; verdade | ciências naturais (especialmente física)}
    \definition{v.}{gerenciar; executar | colocar em ordem; arrumar | (geralmente no negativo) prestar atenção a; fazer um gesto ou falar com | tratar | colocar em ordem; limpar | tomar conhecimento de; prestar atenção a; expressar uma atitude; expressar uma opinião}
  \end{Phonetics}
\end{Entry}

\begin{Entry}{理发}{11,5}{⽟、⼜}
  \begin{Phonetics}{理发}{li3/fa4}[][HSK 3]
    \definition{v.+compl.}{cortar e aparar o cabelo; ter (dar) um corte de cabelo}
  \end{Phonetics}
\end{Entry}

\begin{Entry}{理由}{11,5}{⽟、⽥}
  \begin{Phonetics}{理由}{li3you2}[][HSK 3]
    \definition[个,条,种,堆]{s.}{razão; justificativa; fundamento; a razão pela qual as coisas são feitas desta ou daquela maneira}
  \end{Phonetics}
\end{Entry}

\begin{Entry}{理论}{11,6}{⽟、⾔}
  \begin{Phonetics}{理论}{li3lun4}[][HSK 3]
    \definition[套,个]{s.}{teoria; uma série de conclusões tiradas pelas pessoas sobre atividades naturais ou sociais}
    \definition{v.}{argumentar; raciocinar com alguém; discutir com outras pessoas sobre quem está certo ou errado}
  \end{Phonetics}
\end{Entry}

\begin{Entry}{理财}{11,7}{⽟、⾙}
  \begin{Phonetics}{理财}{li3 cai2}[][HSK 6]
    \definition{v.}{administrar questões financeiras; conduzir transações financeiras; administrar propriedade; ser responsável pelo trabalho financeiro}
  \end{Phonetics}
\end{Entry}

\begin{Entry}{理智}{11,12}{⽟、⽇}
  \begin{Phonetics}{理智}{li3zhi4}[][HSK 6]
    \definition{adj.}{racional; sensato; cabeça fria; sóbrio; calmo}
    \definition{s.}{sentido; razão; intelecto; a capacidade de distinguir o certo do errado, analisar e julgar e controlar as emoções e o comportamento de acordo}
  \end{Phonetics}
\end{Entry}

\begin{Entry}{理想}{11,13}{⽟、⼼}
  \begin{Phonetics}{理想}{li3xiang3}[][HSK 2]
    \definition{adj.}{ideal; perfeito | conforme o desejado; satisfatório}
    \definition{adv.}{idealmente}
    \definition[个,种]{s.}{ideal; sonho; aspiração}
  \end{Phonetics}
\end{Entry}

\begin{Entry}{理解}{11,13}{⽟、⾓}
  \begin{Phonetics}{理解}{li3jie3}[][HSK 3]
    \definition{v.}{entender; compreender; compreender o significado por trás de algo através da reflexão e do aprendizado | entender com empatia; achar que os outros não conseguem fazer determinada coisa e demonstrar compaixão, perdão e não crítica}
  \end{Phonetics}
\end{Entry}

\begin{Entry}{甜}{11}{⽢}
  \begin{Phonetics}{甜}{tian2}[][HSK 3]
    \definition{adj.}{doce; melado | agradável; confortável; fazer as pessoas se sentirem confortáveis e felizes | (sono) profundo | feliz; descreve o sentimento de felicidade}
  \end{Phonetics}
\end{Entry}

\begin{Entry}{甜心}{11,4}{⽢、⼼}
  \begin{Phonetics}{甜心}{tian2xin1}
    \definition{s.}{querido}
  \end{Phonetics}
\end{Entry}

\begin{Entry}{甜头}{11,5}{⽢、⼤}
  \begin{Phonetics}{甜头}{tian2tou5}
    \definition{s.}{benefício | sabor doce (de poder, sucesso, etc.)}
  \end{Phonetics}
\end{Entry}

\begin{Entry}{甜玉米}{11,5,6}{⽢、⽟、⽶}
  \begin{Phonetics}{甜玉米}{tian2 yu4mi3}
    \definition{s.}{milho doce}
  \end{Phonetics}
\end{Entry}

\begin{Entry}{甜言}{11,7}{⽢、⾔}
  \begin{Phonetics}{甜言}{tian2yan2}
    \definition{s.}{boa conversa | palavras amáveis}
  \end{Phonetics}
\end{Entry}

\begin{Entry}{甜品}{11,9}{⽢、⼝}
  \begin{Phonetics}{甜品}{tian2pin3}
    \definition{s.}{sobremesa}
  \end{Phonetics}
\end{Entry}

\begin{Entry}{甜食}{11,9}{⽢、⾷}
  \begin{Phonetics}{甜食}{tian2shi2}
    \definition{s.}{doces | sobremesa}
  \end{Phonetics}
\end{Entry}

\begin{Entry}{甜酒}{11,10}{⽢、⾣}
  \begin{Phonetics}{甜酒}{tian2jiu3}
    \definition{s.}{licor doce}
  \end{Phonetics}
\end{Entry}

\begin{Entry}{甜甜圈}{11,11,11}{⽢、⽢、⼞}
  \begin{Phonetics}{甜甜圈}{tian2tian2quan1}
    \definition{s.}{rosquinha | \emph{doughnut}}
  \end{Phonetics}
\end{Entry}

\begin{Entry}{甜菊}{11,11}{⽢、⾋}
  \begin{Phonetics}{甜菊}{tian2ju2}
    \definition{s.}{estévia, arbusto cujas folhas produzem um substituto para o açúcar}
  \end{Phonetics}
\end{Entry}

\begin{Entry}{甜筒}{11,12}{⽢、⽵}
  \begin{Phonetics}{甜筒}{tian2tong3}
    \definition{s.}{sorvete de casquinha}
  \end{Phonetics}
\end{Entry}

\begin{Entry}{甜稚}{11,13}{⽢、⽲}
  \begin{Phonetics}{甜稚}{tian2zhi4}
    \definition{s.}{doce e inocente}
  \end{Phonetics}
\end{Entry}

\begin{Entry}{甜酸}{11,14}{⽢、⾣}
  \begin{Phonetics}{甜酸}{tian2suan1}
    \definition{adj.}{agridoce}
  \end{Phonetics}
\end{Entry}

\begin{Entry}{略}{11}{⽥}
  \begin{Phonetics}{略}{lve4}
    \definition{adv.}{ligeiramente | marginalmente | aproximadamente}
  \end{Phonetics}
\end{Entry}

\begin{Entry}{略微}{11,13}{⽥、⼻}
  \begin{Phonetics}{略微}{lve4wei1}
    \definition{adv.}{ligeiramente | marginalmente | aproximadamente}
  \end{Phonetics}
\end{Entry}

\begin{Entry}{盒}{11}{⽫}
  \begin{Phonetics}{盒}{he2}[][HSK 5]
    \definition{clas.}{caixa (de pequena dimensão)}
    \definition[个]{s.}{caixa; estojo; recipiente; receptáculo}
  \end{Phonetics}
\end{Entry}

\begin{Entry}{盒子}{11,3}{⽫、⼦}
  \begin{Phonetics}{盒子}{he2zi5}[][HSK 5]
    \definition[个,只,堆]{s.}{caixa; recipiente que têm tampas na parte superior e podem conter coisas dentro, geralmente é pequeno e plano}
  \end{Phonetics}
\end{Entry}

\begin{Entry}{盒饭}{11,7}{⽫、⾷}
  \begin{Phonetics}{盒饭}{he2 fan4}[][HSK 5]
    \definition[份]{s.}{refeição embalada; marmita; \emph{fast-food} vendida em caixas}
  \end{Phonetics}
\end{Entry}

\begin{Entry}{盖}{11}{⽫}
  \begin{Phonetics}{盖}{gai4}[][HSK 4]
    \definition*{s.}{Sobrenome Gai}
    \definition{adj.}{excelente; soberbo; fantástico}
    \definition{adv.}{cerca de; ao redor; aproximadamente; expressa um julgamento especulativo sobre algo, ou uma explicação da causa, o que é equivalente a 大概 ou 原来}
    \definition{conj.}{para; porque; dando continuidade à frase anterior, afirmando a razão ou causa, com tom incerto}
    \definition{s.}{tampa; capa; cobertura; algo que cobre ou sela a parte superior de um objeto | carapaça; concha (de tartaruga, caranguejo, etc.); ossos em formato de crânio em certas partes do corpo humano; as conchas nas costas de certos animais | dossel; capota; toldo | nivelador (uma ferramenta agrícola usada para nivelar terras)}
    \definition{v.}{cobrir; proteger; colocar uma capa em; colocar uma tampa em um objeto | selar; afixar um selo em | superar; sobressair; sobrepujar; ultrapassar | construir; colocar para cima | esconder; ocultar; encobrir | nivelar o terreno com um nivelador (ferramenta agrícola)}
  \seealsoref{大概}{da4gai4}
  \seealsoref{原来}{yuan2lai2}
  \end{Phonetics}
  \begin{Phonetics}{盖}{ge3}
    \definition*{s.}{Sobrenome Ge}
  \end{Phonetics}
\end{Entry}

\begin{Entry}{盖子}{11,3}{⽫、⼦}
  \begin{Phonetics}{盖子}{gai4zi5}[][HSK 7-9]
    \definition[个]{s.}{tampa; cobertura; capa; topo; algo que tem um efeito de proteção na parte superior de um objeto | casco (de tartaruga, etc.); conchas nas costas dos animais}
  \end{Phonetics}
\end{Entry}

\begin{Entry}{盗}{11}{⽫}
  \begin{Phonetics}{盗}{dao4}[][HSK 7-9]
    \definition[个,伙,帮,窝]{s.}{ladrão; assaltante}
    \definition{v.}{roubar; saquear | usurpar; buscar ganho pessoal ou ganho por meios impróprios}
  \end{Phonetics}
\end{Entry}

\begin{Entry}{盗版}{11,8}{⽫、⽚}
  \begin{Phonetics}{盗版}{dao4 ban3}[][HSK 6]
    \definition{s.}{cópia ilegal; cópia pirata; refere-se a livros, periódicos e produtos audiovisuais pirateados (diferentes dos 正版)}
    \definition{v.}{piratear; copiar ou vender ilegalmente; para obter lucros enormes, reimprimir ou copiar livros, periódicos ou produtos audiovisuais em grandes quantidades sem o consentimento do detentor dos direitos autorais}
  \seealsoref{正版}{zheng4 ban3}
  \end{Phonetics}
\end{Entry}

\begin{Entry}{盗窃}{11,9}{⽫、⽳}
  \begin{Phonetics}{盗窃}{dao4qie4}[][HSK 7-9]
    \definition{v.}{roubar; furtar; obter ilegalmente por meios secretos}
  \end{Phonetics}
\end{Entry}

\begin{Entry}{盘}{11}{⽫}
  \begin{Phonetics}{盘}{pan2}[][HSK 4]
    \definition*{s.}{Sobrenome Pan}
    \definition{clas.}{usado para pratos, pedras de moer, etc. | usado para jogos de xadrez e de bola | usado para as coisas que estão entrelaçadas, emaranhadas}
    \definition{s.}{bandeja; tabuleiro | recipiente plano e raso, como uma bandeja, prato, travessa etc.  | preço atual; cotação de mercado; refere-se ao preço básico pelo qual as commodities são negociadas}
    \definition{v.}{enrolar; torcer; enrolar (para cima); entrelaçar; cercar | construir (assentando tijolos, pedras, etc.) | checar; examinar; interrogar; verificar um por um ou repetidamente (quantidade, situação, etc.) | transferir a propriedade de; passar para outra pessoa | carregar; transportar}
  \end{Phonetics}
\end{Entry}

\begin{Entry}{盘子}{11,3}{⽫、⼦}
  \begin{Phonetics}{盘子}{pan2zi5}[][HSK 4]
    \definition[个,叠,套,只]{s.}{prato; utensílio de fundo raso para guardar objetos, maior do que um pires, geralmente de formato redondo | situação de mercado; taxa de mercado; transação comercial}
  \end{Phonetics}
\end{Entry}

\begin{Entry}{盛}{11}{⽫}
  \begin{Phonetics}{盛}{cheng2}[][HSK 7-9]
    \definition{v.}{encher; encher com uma concha; colocar as coisas em recipientes; especialmente colocar alimentos em tigelas, pratos e outros recipientes | segurar; conter; acomodar}
  \end{Phonetics}
  \begin{Phonetics}{盛}{sheng4}
    \definition*{s.}{Sobrenome Sheng}
    \definition{adj.}{florescente; próspero | vigoroso; enérgico | grandioso; magnífico | abundante; profundo | popular; comum; difundido; universal | amplo; generoso; abundante; suficiente | ótimo}
    \definition{adv.}{muito; profundamente}
  \end{Phonetics}
\end{Entry}

\begin{Entry}{盛行}{11,6}{⽫、⾏}
  \begin{Phonetics}{盛行}{sheng4xing2}[][HSK 6]
    \definition{v.}{predominar; estar atual; estar na moda; ser amplamente popular}
  \end{Phonetics}
\end{Entry}

\begin{Entry}{盛宴}{11,10}{⽫、⼧}
  \begin{Phonetics}{盛宴}{sheng4yan4}
    \definition{s.}{celebração}
  \end{Phonetics}
\end{Entry}

\begin{Entry}{眯}{11}{⽬}
  \begin{Phonetics}{眯}{mi1}
    \definition{v.}{estreitar os olhos | esmagar | (dialeto) tirar uma soneca}
  \end{Phonetics}
  \begin{Phonetics}{眯}{mi2}
    \definition{v.}{cegar (como com poeira)}
  \end{Phonetics}
\end{Entry}

\begin{Entry}{眼}{11}{⽬}
  \begin{Phonetics}{眼}{yan3}[][HSK 2]
    \definition{clas.}{usado para grandes coisas ocas: poços, fogões, panelas, etc.}
    \definition[双,只]{s.}{olho; o órgão visual dos humanos ou animais | abertura; pequeno furo; pequeno buraco | ponto-chave; refere-se ao ponto-chave das coisas | armadilha; um termo do jogo Go que se refere a um espaço vazio cercado pelas peças de um jogador, onde o outro jogador não pode colocar uma peça, a menos que haja circunstâncias especiais | uma batida sem acento na música tradicional chinesa}
  \end{Phonetics}
\end{Entry}

\begin{Entry}{眼光}{11,6}{⽬、⼉}
  \begin{Phonetics}{眼光}{yan3guang1}[][HSK 5]
    \definition{s.}{olho; visão | visão; percepção; previsão; capacidade de observar e identificar coisas | vista; ponto de vista}
  \end{Phonetics}
\end{Entry}

\begin{Entry}{眼花缭乱}{11,7,15,7}{⽬、⾋、⽷、⼄}
  \begin{Phonetics}{眼花缭乱}{yan3hua1liao2luan4}
    \definition{v.}{ficar deslumbrado | deslumbrar}
  \end{Phonetics}
\end{Entry}

\begin{Entry}{眼证}{11,7}{⽬、⾔}
  \begin{Phonetics}{眼证}{yan3zheng4}
    \definition{s.}{testemunha ocular}
  \end{Phonetics}
\end{Entry}

\begin{Entry}{眼里}{11,7}{⽬、⾥}
  \begin{Phonetics}{眼里}{yan3 li3}[][HSK 4]
    \definition{s.}{aos olhos de alguém; na opinião (ou visão) de alguém}
  \end{Phonetics}
\end{Entry}

\begin{Entry}{眼泪}{11,8}{⽬、⽔}
  \begin{Phonetics}{眼泪}{yan3 lei4}[][HSK 4]
    \definition[滴,行]{s.}{lágrimas; termo genérico para lágrimas; fluido incolor e transparente secretado pelas glândulas lacrimais no olho, que serve para proteger o olho}
  \end{Phonetics}
\end{Entry}

\begin{Entry}{眼前}{11,9}{⽬、⼑}
  \begin{Phonetics}{眼前}{yan3 qian2}[][HSK 3]
    \definition{adv.}{agora; (no) momento}
    \definition{s.}{diante dos olhos; diante de | agora; (no) momento}
  \end{Phonetics}
\end{Entry}

\begin{Entry}{眼柄}{11,9}{⽬、⽊}
  \begin{Phonetics}{眼柄}{yan3bing3}
    \definition{s.}{pedúnculo ocular (de crustáceo, etc.)}
  \end{Phonetics}
\end{Entry}

\begin{Entry}{眼看}{11,9}{⽬、⽬}
  \begin{Phonetics}{眼看}{yan3 kan4}[][HSK 6]
    \definition{adv.}{em breve; em um momento; imediatamente}
    \definition{v.}{observar impotentemente; olhar passivamente; observar (o que está acontecendo)}
  \end{Phonetics}
\end{Entry}

\begin{Entry}{眼袋}{11,11}{⽬、⾐}
  \begin{Phonetics}{眼袋}{yan3dai4}
    \definition{s.}{inchaço sob os olhos}
  \end{Phonetics}
\end{Entry}

\begin{Entry}{眼睛}{11,13}{⽬、⽬}
  \begin{Phonetics}{眼睛}{yan3jing5}[][HSK 2]
    \definition[双,只]{s.}{olho(s)}
  \end{Phonetics}
\end{Entry}

\begin{Entry}{眼镜}{11,16}{⽬、⾦}
  \begin{Phonetics}{眼镜}{yan3jing4}[][HSK 4]
    \definition[副]{s.}{óculos; óculos de grau; lentes usadas nos olhos para melhorar a visão ou proteger os olhos, feitas de vidro ou cristal incolor ou colorido}
  \end{Phonetics}
\end{Entry}

\begin{Entry}{着}{11}{⽬}
  \begin{Phonetics}{着}{zhao1}
    \definition{interj.}{tudo bem; tudo certo; \emph{O.K.}}
    \definition{s.}{uma jogada no xadrez | truque; meio; artifício; manobra; estratégia}
    \definition{v.}{colocar dentro; guardar}
  \end{Phonetics}
  \begin{Phonetics}{着}{zhao2}
    \definition{v.}{tocar (contato físico) | sentir; ser afetado por | queimar; acender | adormecer; cair no sono | acertar em cheio; ter sucesso em; usado após o verbo, indica que o objetivo foi alcançado ou que houve um resultado}
  \end{Phonetics}
  \begin{Phonetics}{着}{zhe5}[][HSK 1,4]
    \definition{part.}{adicionar a um verbo ou adjetivo para indicar uma ação ou estado contínuo | em frases que começam com uma palavra que indica um lugar, acrescente ao verbo para indicar um estado resultante | em frases imperativas, usado após verbos ou adjetivos para dar ênfase | adicionado após certos verbos, transforma-se em preposição}
    \definition{s.}{um movimento no xadrez |movimento; estratégia; estratagema}
  \end{Phonetics}
  \begin{Phonetics}{着}{zhuo2}
    \definition{v.}{vestir (roupas); vestir-se | tocar; entrar em contato com; aproximar-se de; (contato físico) | enviar; despachar | expressão usada em documentos oficiais antigos, indicando um tom de ordem | aplicar; usar; adicionar; anexar}
  \end{Phonetics}
\end{Entry}

\begin{Entry}{着手}{11,4}{⽬、⼿}
  \begin{Phonetics}{着手}{zhuo2shou3}
    \definition{v.}{colocar a mão nisso | estabelecer | começar uma tarefa}
  \end{Phonetics}
\end{Entry}

\begin{Entry}{着火}{11,4}{⽬、⽕}
  \begin{Phonetics}{着火}{zhao2huo3}[][HSK 4]
    \definition{v.}{pegar fogo; estar em chamas}
  \end{Phonetics}
\end{Entry}

\begin{Entry}{着地}{11,6}{⽬、⼟}
  \begin{Phonetics}{着地}{zhao2di4}
    \definition{v.}{pousar | tocar o chão}
  \end{Phonetics}
\end{Entry}

\begin{Entry}{着花}{11,7}{⽬、⾋}
  \begin{Phonetics}{着花}{zhao2hua1}
    \definition{v.}{florescer}
  \end{Phonetics}
  \begin{Phonetics}{着花}{zhuo2hua1}
    \definition{s.}{floração}
    \definition{v.}{florescer}
  \end{Phonetics}
\end{Entry}

\begin{Entry}{着急}{11,9}{⽬、⼼}
  \begin{Phonetics}{着急}{zhao2ji2}[][HSK 4]
    \definition{adj.}{ansioso; preocupado}
    \definition{s.}{preocupação; ansiedade}
  \end{Phonetics}
\end{Entry}

\begin{Entry}{着凉}{11,10}{⽬、⼎}
  \begin{Phonetics}{着凉}{zhao2liang2}
    \definition{v.}{pegar um resfriado}
  \end{Phonetics}
\end{Entry}

\begin{Entry}{着眼}{11,11}{⽬、⽬}
  \begin{Phonetics}{着眼}{zhuo2yan3}
    \definition{v.}{ter seus olhos em (um objetivo) | ter algo em mente | concentrar-se}
  \end{Phonetics}
\end{Entry}

\begin{Entry}{着装}{11,12}{⽬、⾐}
  \begin{Phonetics}{着装}{zhuo2zhuang1}
    \definition{s.}{roupa | vestimenta}
    \definition{v.}{vestir}
  \end{Phonetics}
\end{Entry}

\begin{Entry}{着想}{11,13}{⽬、⼼}
  \begin{Phonetics}{着想}{zhuo2xiang3}
    \definition{v.}{considerar (as necessidades de outras pessoas) | pensar (para os outros)}
  \end{Phonetics}
\end{Entry}

\begin{Entry}{着数}{11,13}{⽬、⽁}
  \begin{Phonetics}{着数}{zhao1shu4}
    \definition{s.}{estratégia | movimento (no xadrez, no palco, nas artes marciais) | esquema | truque}
  \end{Phonetics}
\end{Entry}

\begin{Entry}{硕}{11}{⽯}
  \begin{Phonetics}{硕}{shuo4}
    \definition*{s.}{Sobrenome Shuo}
    \definition{adj.}{grande; enorme}
    \definition{s.}{mestrado (MBA)}
  \end{Phonetics}
\end{Entry}

\begin{Entry}{硕士}{11,3}{⽯、⼠}
  \begin{Phonetics}{硕士}{shuo4shi4}[][HSK 5]
    \definition[个,位,名]{s.}{mestrado; um diploma concedido por uma universidade ou faculdade a um aluno após um ou dois anos de estudo adicional após o bacharelado}
  \end{Phonetics}
\end{Entry}

\begin{Entry}{票}{11}{⽰}
  \begin{Phonetics}{票}{piao4}[][HSK 1]
    \definition{clas.}{para grupos, lotes, transações comerciais}
    \definition[张]{s.}{bilhete; passagem; ingresso | cédula | nota bancária; conta | pessoa mantida em cativeiro por sequestradores para obter resgate; refém | apresentação amadora (de ópera de Pequim, etc.); peças teatrais amadoras}
    \definition{v.}{atuar como amador (na ópera de Pequim)}
  \end{Phonetics}
\end{Entry}

\begin{Entry}{票价}{11,6}{⽰、⼈}
  \begin{Phonetics}{票价}{piao4 jia4}[][HSK 3]
    \definition[个]{s.}{o preço de um ingresso; taxa de admissão; taxa de entrada}
  \end{Phonetics}
\end{Entry}

\begin{Entry}{祸}{11}{⽰}
  \begin{Phonetics}{祸}{huo4}
    \definition[场]{s.}{infortúnio; desastre; calamidade (oposto de 福) | desgraça; catástrofe}
    \definition{v.}{trazer desastre; arruinar | causar problemas}
  \seealsoref{福}{fu2}
  \end{Phonetics}
\end{Entry}

\begin{Entry}{移}{11}{⽲}
  \begin{Phonetics}{移}{yi2}[][HSK 4]
    \definition*{s.}{Sobrenome Yi}
    \definition{v.}{mover; remover; deslocar; mudar | mudar; alterar}
  \end{Phonetics}
\end{Entry}

\begin{Entry}{移民}{11,5}{⽲、⽒}
  \begin{Phonetics}{移民}{yi2min2}[][HSK 4]
    \definition[个,批]{s.}{emigrante; migrantes; aqueles que se mudam para um país ou estado estrangeiro para se estabelecer}
    \definition{v.}{migrar; imigrar}
  \end{Phonetics}
\end{Entry}

\begin{Entry}{移动}{11,6}{⽲、⼒}
  \begin{Phonetics}{移动}{yi2dong4}[][HSK 4]
    \definition{v.}{deslocar; mover; mudar}
  \end{Phonetics}
\end{Entry}

\begin{Entry}{竟}{11}{⾳}
  \begin{Phonetics}{竟}{jing4}
    \definition{adj.}{todo; por toda parte; do começo ao fim}
    \definition{adv.}{no final; eventualmente | na verdade; inesperadamente; significa algo inesperado, equivalente a 居然}
    \definition{v.}{terminar; completar | investigar}
  \seealsoref{居然}{ju1ran2}
  \end{Phonetics}
\end{Entry}

\begin{Entry}{竟然}{11,12}{⾳、⽕}
  \begin{Phonetics}{竟然}{jing4ran2}[][HSK 4]
    \definition{adv.}{de fato; inesperadamente; para surpresa de alguém; chegar ao ponto de; indica que algo é um pouco inesperado}
  \end{Phonetics}
\end{Entry}

\begin{Entry}{章}{11}{⾳}
  \begin{Phonetics}{章}{zhang1}[][HSK 6]
    \definition*{s.}{Sobrenome Zhang}
    \definition[枚,个,方]{s.}{(para um livro, carta, etc.) capítulo; seção | ordem | regras; regulamentos; constituição | item; cláusula | (arcaico) memorial ao imperador; memorial ao trono | (arcaico) figura; padrão decorativo | selo; carimbo | distintivo; insígnia; medalha | verso; trecho do poema | escrita literária}
  \end{Phonetics}
\end{Entry}

\begin{Entry}{章鱼}{11,8}{⾳、⿂}
  \begin{Phonetics}{章鱼}{zhang1yu2}
    \definition{s.}{polvo | octópode}
  \end{Phonetics}
\end{Entry}

\begin{Entry}{笛}{11}{⽵}
  \begin{Phonetics}{笛}{di2}
    \definition[只]{s.}{flauta de bambu | sirene; apito; buzina}
  \end{Phonetics}
\end{Entry}

\begin{Entry}{笛子}{11,3}{⽵、⼦}
  \begin{Phonetics}{笛子}{di2zi5}[][HSK 7-9]
    \definition[管]{s.}{flauta; flauta de bambu; um instrumento de sopro transversal feito de bambu ou metal com seis furos de tom dispostos em uma fileira de acordo com o tom}
  \end{Phonetics}
\end{Entry}

\begin{Entry}{符}{11}{⽵}
  \begin{Phonetics}{符}{fu2}
    \definition*{s.}{Sobrenome Fu}
    \definition[个]{s.}{registro emitido por um governante para generais, enviados, etc., como credenciais na China antiga | símbolo; emblema | figuras mágicas desenhadas por sacerdotes taoístas para invocar ou expulsar espíritos e trazer boa ou má sorte | marca; sinal}
    \definition{v.}{(usado com 相 xiāng ou 不) coincidir com; concordar com | encaixar bem; combinar com; em conformidade com}
  \seealsoref{不}{bu4}
  \seealsoref{相}{xiang1}
  \end{Phonetics}
\end{Entry}

\begin{Entry}{符号}{11,5}{⽵、⼝}
  \begin{Phonetics}{符号}{fu2hao4}[][HSK 4]
    \definition[个]{s.}{marca; símbolo; sinais que marcam as coisas | insígnia; emblema; um símbolo usado no corpo para indicar posição, \emph{status}, etc.}
  \end{Phonetics}
\end{Entry}

\begin{Entry}{符合}{11,6}{⽵、⼝}
  \begin{Phonetics}{符合}{fu2he2}[][HSK 4]
    \definition{v.}{conformar-se com, estar de acordo com, estar em conformidade com}
  \end{Phonetics}
\end{Entry}

\begin{Entry}{笨}{11}{⽵}
  \begin{Phonetics}{笨}{ben4}[][HSK 4]
    \definition{adj.}{estúpido; sem graça; tolo; de pouca habilidade; sem inteligência | desajeitado; tosco; inflexível | incômodo; pesado; desajeitado; difícil de manejar; trabalhoso}
  \end{Phonetics}
\end{Entry}

\begin{Entry}{笨重}{11,9}{⽵、⾥}
  \begin{Phonetics}{笨重}{ben4zhong4}[][HSK 7-9]
    \definition{adj.}{pesado; desajeitado; incômodo; grande e pesado, inconveniente de usar | pesado; difícil de manejar; pesado e trabalhoso}
  \end{Phonetics}
\end{Entry}

\begin{Entry}{笨蛋}{11,11}{⽵、⾍}
  \begin{Phonetics}{笨蛋}{ben4dan4}[][HSK 7-9]
    \definition[个]{s.}{tolo; idiota; (depreciativo) refere-se a uma pessoa muito estúpida ou sem cérebro; geralmente usado para insultar pessoas}
  \end{Phonetics}
\end{Entry}

\begin{Entry}{第}{11}{⽵}
  \begin{Phonetics}{第}{di4}[][HSK 1]
    \definition*{s.}{Sobrenome Di}
    \definition{adv.}{mas, apenas, somente; Indica que a ação não está sujeita a restrições ou condições; equivalente a 只管}
    \definition{conj.}{mas; contudo; orações de conexão; indicando uma relação de transição; equivalente a 但是}
    \definition{pref.}{palavra auxiliar para números ordinais; usado antes de números inteiros, indica ordem}
    \definition{s.}{diferentes notas dos candidatos aprovados nos exames imperiais | a residência de um alto funcionário; grandes residências dos burocratas da era feudal}
  \seealsoref{但是}{dan4 shi4}
  \seealsoref{只管}{zhi3 guan3}
  \end{Phonetics}
\end{Entry}

\begin{Entry}{第一手}{11,1,4}{⽵、⼀、⼿}
  \begin{Phonetics}{第一手}{di4yi1shou3}[][HSK 7-9]
    \definition{s.}{em primeira mão; obtido por meio de prática e investigação pessoal; obtido diretamente}
  \end{Phonetics}
\end{Entry}

\begin{Entry}{第一线}{11,1,8}{⽵、⼀、⽷}
  \begin{Phonetics}{第一线}{di4yi1xian4}[][HSK 7-9]
    \definition{s.}{vanguarda; linha de frente; primeira linha | frente (linha), primeira linha; a linha de frente do campo de batalha também se refere ao local onde um determinado trabalho é realizado diretamente}
  \end{Phonetics}
\end{Entry}

\begin{Entry}{笼}{11}{⽵}
  \begin{Phonetics}{笼}{long2}
    \definition{s.}{armação fechada de bambu, arame, etc. | jaula | gaiola}
  \end{Phonetics}
  \begin{Phonetics}{笼}{long3}
    \definition{v.}{envolver | cobrir}
  \end{Phonetics}
\end{Entry}

\begin{Entry}{笼子}{11,3}{⽵、⼦}
  \begin{Phonetics}{笼子}{long2zi5}
    \definition{s.}{jaula | cesta | gaiola | recipiente}
  \end{Phonetics}
  \begin{Phonetics}{笼子}{long3zi5}
    \definition{s.}{caixa grande | porta-malas}
  \end{Phonetics}
\end{Entry}

\begin{Entry}{粗}{11}{⽶}
  \begin{Phonetics}{粗}{cu1}[][HSK 4]
    \definition{adj.}{largo (em diâmetro); grosso | grosseiro; rude; áspero | áspero; rouco | descuidado; negligente | rude; sem refinamento; vulgar}
    \definition{adv.}{grosseiramente; vagamente}
  \end{Phonetics}
\end{Entry}

\begin{Entry}{粗心}{11,4}{⽶、⼼}
  \begin{Phonetics}{粗心}{cu1xin1}[][HSK 4]
    \definition{adj.}{descuidado; irrefletido; (fazer as coisas) de forma desleixada, sem cuidado}
  \end{Phonetics}
\end{Entry}

\begin{Entry}{粗心大意}{11,4,3,13}{⽶、⼼、⼤、⼼}
  \begin{Phonetics}{粗心大意}{cu1xin1-da4yi4}[][HSK 7-9]
    \definition{expr.}{ser negligente; descuidado; inadvertido; desmiolado; descuidado e negligente; negligente; remisso; refere-se a fazer as coisas de forma descuidada}
  \end{Phonetics}
\end{Entry}

\begin{Entry}{粗心地做}{11,4,6,11}{⽶、⼼、⼟、⼈}
  \begin{Phonetics}{粗心地做}{cu1xin1 di4 zuo4}
    \definition{adj.}{feito descuidadamente}
  \end{Phonetics}
\end{Entry}

\begin{Entry}{粗略}{11,11}{⽶、⽥}
  \begin{Phonetics}{粗略}{cu1lve4}[][HSK 7-9]
    \definition{adj.}{grosseiro; rudimentar; superficial}
  \end{Phonetics}
\end{Entry}

\begin{Entry}{粗鲁}{11,12}{⽶、⿂}
  \begin{Phonetics}{粗鲁}{cu1lu3}[][HSK 7-9]
    \definition{adj.}{rude; grosseiro; incivilizado}
  \end{Phonetics}
\end{Entry}

\begin{Entry}{粗暴}{11,15}{⽶、⽇}
  \begin{Phonetics}{粗暴}{cu1bao4}[][HSK 7-9]
    \definition{adj.}{rude; áspero; bruto; brutal; violento}
  \end{Phonetics}
\end{Entry}

\begin{Entry}{粗糙}{11,16}{⽶、⽶}
  \begin{Phonetics}{粗糙}{cu1cao1}[][HSK 7-9]
    \definition{adj.}{áspero; grosseiro; não é liso; não é redondo; não é fino | desleixado; descuidado; não meticuloso}
  \end{Phonetics}
\end{Entry}

\begin{Entry}{累}{11}{⽷}
  \begin{Phonetics}{累}{lei2}
    \definition*{s.}{Sobrenome Lei}
    \definition{adj.}{incômodo; complicado}
    \definition{s.}{corda; cordão | touro na época de acasalamento}
    \definition{v.}{amarrar; prender; atar | copular}
  \end{Phonetics}
  \begin{Phonetics}{累}{lei3}
    \definition*{s.}{Sobrenome Lei}
    \definition{adj.}{em andamento; repetido; contínuo}
    \definition{v.}{acumular; empilhar; colocar em cima de outro | envolver; implicar | construir empilhando tijolos, pedras, terra, etc.}
  \end{Phonetics}
  \begin{Phonetics}{累}{lei4}[][HSK 1]
    \definition{adj.}{cansado; exausto; fatigado}
    \definition{v.}{cansar; desgastar; fatigar; esgotar | labutar; trabalhar duro}
  \end{Phonetics}
\end{Entry}

\begin{Entry}{绯}{11}{⽷}
  \begin{Phonetics}{绯}{fei1}
    \definition{adj.}{escarlate; vermelho; vermelho escuro; vermelho profundo}
  \end{Phonetics}
\end{Entry}

\begin{Entry}{绯闻}{11,9}{⽷、⾨}
  \begin{Phonetics}{绯闻}{fei1wen2}[][HSK 7-9]
    \definition{s.}{boato/fofoca sobre escândalos sexuais | escândalo sexual; rumores sobre relacionamentos entre homens e mulheres}
  \end{Phonetics}
\end{Entry}

\begin{Entry}{绰}{11}{⽷}
  \begin{Phonetics}{绰}{chuo4}
    \definition{adj.}{amplo; espaçoso | (do porte de uma menina) graciosa; flexível}
  \end{Phonetics}
\end{Entry}

\begin{Entry}{绰号}{11,5}{⽷、⼝}
  \begin{Phonetics}{绰号}{chuo4hao4}[][HSK 7-9]
    \definition[个]{s.}{apelido; um nome informal dado a alguém com base em suas características, muitas vezes expressando afeição, antipatia ou brincadeira; também chamado de 外号}
  \seealsoref{外号}{wai4hao4}
  \end{Phonetics}
\end{Entry}

\begin{Entry}{绳}{11}{⽷}
  \begin{Phonetics}{绳}{sheng2}
    \definition*{s.}{Sobrenome Sheng}
    \definition[根]{s.}{corda; cordão; barbante | a linha no marcador de tinta de carpinteiro}
    \definition{v.}{restringir; corrigir; sancionar | medir | continuar}
  \end{Phonetics}
\end{Entry}

\begin{Entry}{绳子}{11,3}{⽷、⼦}
  \begin{Phonetics}{绳子}{sheng2zi5}
    \definition[条]{s.}{corda | cordão}
  \end{Phonetics}
\end{Entry}

\begin{Entry}{维}{11}{⽷}
  \begin{Phonetics}{维}{wei2}
    \definition*{s.}{Sobrenome Wei}
    \definition{s.}{pensamento | dimensão; conceitos básicos de geometria e teoria do espaço}
    \definition{v.}{ligar; amarrar; manter unido; conectar | manter; manter; salvaguardar; preservar}
  \end{Phonetics}
\end{Entry}

\begin{Entry}{维生素}{11,5,10}{⽷、⽣、⽷}
  \begin{Phonetics}{维生素}{wei2sheng1su4}[][HSK 6]
    \definition[点]{s.}{vitamina}[西瓜中含丰富的维生素。===A melancia é rica em vitaminas.]
  \end{Phonetics}
\end{Entry}

\begin{Entry}{维吾尔}{11,7,5}{⽷、⼝、⼩}
  \begin{Phonetics}{维吾尔}{wei2wu2'er3}
    \definition*{s.}{Etnia Uigur de Xinjiang}
  \end{Phonetics}
\end{Entry}

\begin{Entry}{维护}{11,7}{⽷、⼿}
  \begin{Phonetics}{维护}{wei2hu4}[][HSK 4]
    \definition{v.}{defender; proteger; manter; preservar}
  \end{Phonetics}
\end{Entry}

\begin{Entry}{维修}{11,9}{⽷、⼈}
  \begin{Phonetics}{维修}{wei2xiu1}[][HSK 4]
    \definition{v.}{prestar serviços; manter; reparar; manter em (bom) estado de conservação}
  \end{Phonetics}
\end{Entry}

\begin{Entry}{维持}{11,9}{⽷、⼿}
  \begin{Phonetics}{维持}{wei2chi2}[][HSK 4]
    \definition{v.}{manter; conservar; guardar; manter vivo}
  \end{Phonetics}
\end{Entry}

\begin{Entry}{绷}{11}{⽷}
  \begin{Phonetics}{绷}{beng1}[][HSK 7-9]
    \definition{s.}{estrutura de cama amarrada com cordas, tiras de vime, etc.}
    \definition{v.}{esticar (ou puxar) com força | saltar; quicar | alinhavar; fixar | Dialeto: conseguir fazer algo com dificuldade | (roupas) apertar | costurar ou alfinetar com parcimônia |Dialeto: fraudar; roubar dinheiro}
  \end{Phonetics}
  \begin{Phonetics}{绷}{beng3}
    \definition{v.}{mostrar uma cara sombria, tensa; parecer descontente | conter o próprio temperamento}
  \end{Phonetics}
  \begin{Phonetics}{绷}{beng4}
    \definition{adv.}{muito; extremamente; altamente; usado antes de certos adjetivos para indicar um alto grau de severidade}
    \definition{v.}{rachar; dividir; rasgar; fissurar}
  \end{Phonetics}
\end{Entry}

\begin{Entry}{绷带}{11,9}{⽷、⼱}
  \begin{Phonetics}{绷带}{beng1dai4}[][HSK 7-9]
    \definition[条,卷]{s.}{curativo | Empréstimo linguístico: \emph{bandage}; a atadura de gaze usada para enfaixar feridas ou áreas afetadas}
  \end{Phonetics}
\end{Entry}

\begin{Entry}{综}{11}{⽷}
  \begin{Phonetics}{综}{zeng4}
    \definition{s.}{liço; fuso; um dispositivo em um tear que separa os fios da urdidura em um padrão alternado para permitir a passagem da lançadeira}
  \end{Phonetics}
  \begin{Phonetics}{综}{zong1}
    \definition*{s.}{Sobrenome Zong}
    \definition{v.}{reunir; resumir | combinar; reunir}
  \end{Phonetics}
\end{Entry}

\begin{Entry}{综合}{11,6}{⽷、⼝}
  \begin{Phonetics}{综合}{zong1he2}[][HSK 4]
    \definition{s.}{síntese}
    \definition{v.}{sintetizar; resumir as partes de uma coisa em um todo unificado após análise (em oposição a 分析); reunir coisas de um tipo ou natureza diferente}
  \seealsoref{分析}{fen1xi1}
  \end{Phonetics}
\end{Entry}

\begin{Entry}{绿}{11}{⽷}
  \begin{Phonetics}{绿}{lv4}[][HSK 2]
    \definition*{s.}{Sobrenome Lü}
    \definition{adj.}{verde}
    \definition{v.}{tornar-se verde; ficar verde}
  \end{Phonetics}
\end{Entry}

\begin{Entry}{绿化}{11,4}{⽷、⼔}
  \begin{Phonetics}{绿化}{lv4 hua4}[][HSK 6]
    \definition{v.}{tornar verde plantando árvores, flores, etc.; arborizar; reflorestar; plantar árvores, flores e plantas para embelezar o ambiente ou prevenir a erosão do solo}
  \end{Phonetics}
\end{Entry}

\begin{Entry}{绿色}{11,6}{⽷、⾊}
  \begin{Phonetics}{绿色}{lv4 se4}[][HSK 2]
    \definition{adj.}{verde; ecológico; sem poluição; em conformidade com os requisitos ambientais}
    \definition{s.}{cor verde}
  \end{Phonetics}
\end{Entry}

\begin{Entry}{绿豆}{11,7}{⽷、⾖}
  \begin{Phonetics}{绿豆}{lv4dou4}
    \definition{s.}{vagens}
  \end{Phonetics}
\end{Entry}

\begin{Entry}{绿豆芽}{11,7,7}{⽷、⾖、⾋}
  \begin{Phonetics}{绿豆芽}{lv4dou4 ya2}
    \definition{s.}{broto de feijão verde}
  \end{Phonetics}
\end{Entry}

\begin{Entry}{绿茶}{11,9}{⽷、⾋}
  \begin{Phonetics}{绿茶}{lv4 cha2}[][HSK 3]
    \definition{s.}{chá verde; chá produzido apenas através dos processos de maturação, enrolamento (ou sem enrolamento) e secagem, sem passar por fermentação, com cor verde-claro}
  \end{Phonetics}
\end{Entry}

\begin{Entry}{聊}{11}{⽿}
  \begin{Phonetics}{聊}{liao2}[][HSK 6]
    \definition*{s.}{Sobrenome Liao}
    \definition{adv.}{apenas; meramente; provisoriamente; por enquanto | um pouco; ligeiramente}
    \definition{v.}{tagarelar; conversar; bater papo | confiar (ou depender, recorrer) a}
  \end{Phonetics}
\end{Entry}

\begin{Entry}{聊天}{11,4}{⽿、⼤}
  \begin{Phonetics}{聊天}{liao2/tian1}
    \definition{v.+compl.}{papear | bater papo}
  \end{Phonetics}
\end{Entry}

\begin{Entry}{聊天儿}{11,4,2}{⽿、⼤、⼉}
  \begin{Phonetics}{聊天儿}{liao2/tian1r5}[][HSK 6]
    \definition{v.+compl.}{conversar; fofocar; bater papo; duas ou mais pessoas conversando sem um tópico ou propósito específico}
  \end{Phonetics}
\end{Entry}

\begin{Entry}{职}{11}{⽿}
  \begin{Phonetics}{职}{zhi2}
    \definition*{s.}{Sobrenome Zhi}
    \definition{prep.}{para; devido a; por causa de}
    \definition{prep.}{(datado) Eu (em relatórios oficiais aos superiores)}
    \definition{s.}{dever; trabalho | cargo; posto; função; responsabilidades; posição}
    \definition{v.}{gerenciar; dirigir | administrar}
  \end{Phonetics}
\end{Entry}

\begin{Entry}{职工}{11,3}{⽿、⼯}
  \begin{Phonetics}{职工}{zhi2 gong1}[][HSK 3]
    \definition[个,位,名,些]{s.}{pessoal; trabalhadores e funcionários administrativos}
  \end{Phonetics}
\end{Entry}

\begin{Entry}{职业}{11,5}{⽿、⼀}
  \begin{Phonetics}{职业}{zhi2ye4}[][HSK 3]
    \definition{adj.}{profissional; não amador}
    \definition[种,份,个]{s.}{ocupação; profissão; vocação; o trabalho que um indivíduo realiza na sociedade como sua principal fonte de subsistência}
  \end{Phonetics}
\end{Entry}

\begin{Entry}{职务}{11,5}{⽿、⼒}
  \begin{Phonetics}{职务}{zhi2wu4}[][HSK 5]
    \definition{s.}{cargo; posto; deveres; função; funções que devem ser desempenhadas de acordo com as especificações do cargo}
  \end{Phonetics}
\end{Entry}

\begin{Entry}{职位}{11,7}{⽿、⼈}
  \begin{Phonetics}{职位}{zhi2wei4}[][HSK 5]
    \definition[个]{s.}{posto; posição; cargo que exerce determinadas funções em órgãos ou entidades}
  \end{Phonetics}
\end{Entry}

\begin{Entry}{职员}{11,7}{⽿、⼝}
  \begin{Phonetics}{职员}{zhi2yuan2}
    \definition[个,位]{s.}{empregado | trabalhador de escritório | membro da equipe}
  \end{Phonetics}
\end{Entry}

\begin{Entry}{职责}{11,8}{⽿、⾙}
  \begin{Phonetics}{职责}{zhi2 ze2}[][HSK 6]
    \definition[种]{s.}{dever; obrigação; responsabilidade; coisas que você deve fazer por causa de sua profissão ou identidade}
  \end{Phonetics}
\end{Entry}

\begin{Entry}{职能}{11,10}{⽿、⾁}
  \begin{Phonetics}{职能}{zhi2neng2}[][HSK 5]
    \definition[种,项]{s.}{função; funções ou papéis que as organizações, instituições, etc. devem desempenhar}
  \end{Phonetics}
\end{Entry}

\begin{Entry}{脖}{11}{⾁}
  \begin{Phonetics}{脖}{bo2}
    \definition[个]{s.}{pescoço | em forma de pescoço | parte semelhante ao pescoço}
  \end{Phonetics}
\end{Entry}

\begin{Entry}{脖子}{11,3}{⾁、⼦}
  \begin{Phonetics}{脖子}{bo2zi5}[][HSK 7-9]
    \definition[条,个]{s.}{pescoço; a parte onde a cabeça e o tronco se conectam}
  \end{Phonetics}
\end{Entry}

\begin{Entry}{脚}{11}{⾁}
  \begin{Phonetics}{脚}{jiao3}[][HSK 2]
    \definition{clas.}{usado para chutes}
    \definition[只,双]{s.}{pé; a parte inferior das pernas de pessoas ou animais, que entra em contato com o solo | base; pé; a parte inferior do objeto | antigamente, referia-se ao trabalho físico de transporte de cargas | resíduos; sobras}
  \end{Phonetics}
  \begin{Phonetics}{脚}{jue2}
    \variantof{角}
  \end{Phonetics}
\end{Entry}

\begin{Entry}{脚印}{11,5}{⾁、⼙}
  \begin{Phonetics}{脚印}{jiao3 yin4}[][HSK 6]
    \definition{s.}{trilha; pegada; marca de pé; os rastros deixados pelos passos}
  \end{Phonetics}
\end{Entry}

\begin{Entry}{脚步}{11,7}{⾁、⽌}
  \begin{Phonetics}{脚步}{jiao3 bu4}[][HSK 5]
    \definition{s.}{pé; passo; pisada; refere-se ao movimento das pernas ao caminhar | ritmo; passo; distância entre os pés dianteiros e traseiros ao caminhar}
  \end{Phonetics}
\end{Entry}

\begin{Entry}{脱}{11}{⾁}
  \begin{Phonetics}{脱}{tuo1}[][HSK 4]
    \definition{conj.}{se; no caso}
    \definition{v.}{(cabelo, pele) soltar-se; desprender-se; cair | retirar peça de roupa do corpo | sair de; escapar de | perder (palavras) | livrar-se de algo}
  \end{Phonetics}
\end{Entry}

\begin{Entry}{脱毛}{11,4}{⾁、⽑}
  \begin{Phonetics}{脱毛}{tuo1mao2}
    \definition{s.}{depilação}
    \definition{v.}{perder cabelo ou penas | depilar | fazer a barba}
  \end{Phonetics}
\end{Entry}

\begin{Entry}{脱险}{11,9}{⾁、⾩}
  \begin{Phonetics}{脱险}{tuo1xian3}
    \definition{v.}{sair do perigo}
  \end{Phonetics}
\end{Entry}

\begin{Entry}{脱离}{11,10}{⾁、⼇}
  \begin{Phonetics}{脱离}{tuo1li2}[][HSK 5]
    \definition{v.}{separar-se; divorciar-se; afastar-se; sair (de um determinado ambiente ou situação); romper (uma determinada relação)}
  \end{Phonetics}
\end{Entry}

\begin{Entry}{脸}{11}{⾁}
  \begin{Phonetics}{脸}{lian3}[][HSK 2]
    \definition[张,个]{s.}{rosto (de pessoas ou animais); a parte frontal da cabeça, da testa ao queixo | parte frontal de algo | cara; autoestima; aparência | rosto; expressões faciais}
  \end{Phonetics}
\end{Entry}

\begin{Entry}{脸色}{11,6}{⾁、⾊}
  \begin{Phonetics}{脸色}{lian3 se4}[][HSK 5]
    \definition{s.}{aparência; tez; cor da pele | aparência; expressão facial | (indicando a condição física de alguém) aparência; cor}
  \end{Phonetics}
\end{Entry}

\begin{Entry}{脸盆}{11,9}{⾁、⽫}
  \begin{Phonetics}{脸盆}{lian3 pen2}[][HSK 5]
    \definition[个]{s.}{lavatório; bacia para lavar as mãos e o rosto}
  \end{Phonetics}
\end{Entry}

\begin{Entry}{舵}{11}{⾈}
  \begin{Phonetics}{舵}{duo4}
    \definition{s.}{leme; dispositivos para controlar a direção de navios, aeronaves, etc.}
  \seealsoref{柁}{tuo2}
  \end{Phonetics}
\end{Entry}

\begin{Entry}{舵手}{11,4}{⾈、⼿}
  \begin{Phonetics}{舵手}{duo4shou3}[][HSK 7-9]
    \definition{s.}{timoneiro}
  \end{Phonetics}
\end{Entry}

\begin{Entry}{船}{11}{⾈}
  \begin{Phonetics}{船}{chuan2}[][HSK 2]
    \definition*{s.}{Sobrenome Chuan}
    \definition[条,艘,叶,只]{s.}{barco; navio | embarcação; meio de transporte aquático, nome genérico para embarcações}
  \end{Phonetics}
\end{Entry}

\begin{Entry}{船长}{11,4}{⾈、⾧}
  \begin{Phonetics}{船长}{chuan2 zhang3}[][HSK 6]
    \definition{s.}{capitão do navio; mestre; marinheiro; comandante; o oficial chefe a bordo}
  \end{Phonetics}
\end{Entry}

\begin{Entry}{船只}{11,5}{⾈、⼝}
  \begin{Phonetics}{船只}{chuan2 zhi1}[][HSK 6]
    \definition[艘,条]{s.}{transporte marítimo; embarcação | navio; veleiro}
  \end{Phonetics}
\end{Entry}

\begin{Entry}{船员}{11,7}{⾈、⼝}
  \begin{Phonetics}{船员}{chuan2 yuan2}[][HSK 6]
    \definition[名,位,个]{s.}{tripulação (do navio) | membro da tripulação (do navio); marinheiro; marujo; barqueiro; velejador}
  \end{Phonetics}
\end{Entry}

\begin{Entry}{船桨}{11,10}{⾈、⽊}
  \begin{Phonetics}{船桨}{chuan2jiang3}[][HSK 7-9]
    \definition{s.}{remo}
  \end{Phonetics}
\end{Entry}

\begin{Entry}{船舶}{11,11}{⾈、⾈}
  \begin{Phonetics}{船舶}{chuan2bo2}[][HSK 7-9]
    \definition[艘,条,只]{s.}{transporte marítimo; barcos e navios; refere-se a vários navios}
  \end{Phonetics}
\end{Entry}

\begin{Entry}{菜}{11}{⾋}
  \begin{Phonetics}{菜}{cai4}[][HSK 1]
    \definition*{s.}{Sobrenome Cai}
    \definition{adj.}{pouca habilidade; baixo nível; baixa capacidade}
    \definition[棵,个,道]{s.}{legumes; verduras; plantas que podem ser usadas como alimentos complementares | óleo de canola | prato; item ou prato do cardápio (seja de carne ou de vegetais)}
  \end{Phonetics}
\end{Entry}

\begin{Entry}{菜刀}{11,2}{⾋、⼑}
  \begin{Phonetics}{菜刀}{cai4dao1}
    \definition[把]{s.}{faca de vegetais | faca de cozinha | cutelo}
  \end{Phonetics}
\end{Entry}

\begin{Entry}{菜市场}{11,5,6}{⾋、⼱、⼟}
  \begin{Phonetics}{菜市场}{cai4shi4chang3}[][HSK 7-9]
    \definition[个,家]{s.}{mercado de alimentos; mercearia verde; mercado de produtos agrícolas; mercado de vegetais; um mercado em uma cidade ou município que vende vegetais, carne, ovos e outros alimentos não básicos}
  \end{Phonetics}
\end{Entry}

\begin{Entry}{菜单}{11,8}{⾋、⼗}
  \begin{Phonetics}{菜单}{cai4dan1}[][HSK 2]
    \definition[个,分,张]{s.}{menu; lista de pratos | menu (para computadores); lista utilizada para selecionar várias operações diferentes}
  \end{Phonetics}
\end{Entry}

\begin{Entry}{菠}{11}{⾋}
  \begin{Phonetics}{菠}{bo1}
    \definition{s.}{espinafre}
  \end{Phonetics}
\end{Entry}

\begin{Entry}{菠菜}{11,11}{⾋、⾋}
  \begin{Phonetics}{菠菜}{bo1cai4}
    \definition[棵]{s.}{espinafre}
  \end{Phonetics}
\end{Entry}

\begin{Entry}{菱}{11}{⾋}
  \begin{Phonetics}{菱}{ling2}
    \definition{s.}{maruca; caltrop aquático; castanha d'água}
  \end{Phonetics}
\end{Entry}

\begin{Entry}{菱角}{11,7}{⾋、⾓}
  \begin{Phonetics}{菱角}{ling2jiao5}
    \definition{s.}{castanha d'água}
  \end{Phonetics}
\end{Entry}

\begin{Entry}{营}{11}{⾋}
  \begin{Phonetics}{营}{ying2}
    \definition*{s.}{Sobrenome Ying}
    \definition{s.}{acampamento; quartel; onde o exército está estacionado | batalhão; unidades militares}
    \definition{v.}{procurar | operar; executar; gerenciar}
  \end{Phonetics}
\end{Entry}

\begin{Entry}{营业}{11,5}{⾋、⼀}
  \begin{Phonetics}{营业}{ying2ye4}[][HSK 4]
    \definition{v.}{fazer negócios; estar aberto para negócios}
  \end{Phonetics}
\end{Entry}

\begin{Entry}{营养}{11,9}{⾋、⼋}
  \begin{Phonetics}{营养}{ying2yang3}[][HSK 3]
    \definition[种]{s.}{nutrição; alimentação; a função do organismo de absorver as substâncias necessárias do meio externo para manter atividades vitais, como crescimento e desenvolvimento | nutrição; alimentação; ato ou processo de fornecer nutrição}
  \end{Phonetics}
\end{Entry}

\begin{Entry}{著}{11}{⽬}
  \begin{Phonetics}{著}{zhu4}
    \definition{adj.}{marcado; excelente; óbvio}
    \definition{s.}{livro; trabalho | nativo; pessoa/povo indígena; refere-se a pessoas que se estabeleceram em um lugar por gerações}
    \definition{v.}{mostrar; provar; revelar | escrever}
  \end{Phonetics}
\end{Entry}

\begin{Entry}{著名}{11,6}{⽬、⼝}
  \begin{Phonetics}{著名}{zhu4ming2}[][HSK 4]
    \definition[位]{adj.}{famoso; bem conhecido; célebre}
  \end{Phonetics}
\end{Entry}

\begin{Entry}{著作}{11,7}{⽬、⼈}
  \begin{Phonetics}{著作}{zhu4zuo4}[][HSK 4]
    \definition[部,本,类]{s.}{obra; livro; escritos}
    \definition{v.}{escrever; usar palavras para expressar opiniões, conhecimentos, ideias, sentimentos, etc.}
  \end{Phonetics}
\end{Entry}

\begin{Entry}{虚}{11}{⾌}
  \begin{Phonetics}{虚}{xu1}
    \definition*{s.}{Xu, a décima primeira das vinte e oito constelações em que a esfera celeste foi dividida, consistindo de duas estrelas em linha reta, uma em Aquário e a outra em Equuleus | Xu, uma das mansões lunares | Sobrenome Xu}
    \definition{adj.}{vazio; oco; desocupado | desconfiado; tímido | falso; nominal (oposto a 实) | humilde; modesto | fraco; com saúde debilitada | (física) virtual}
    \definition{adv.}{em vão}
    \definition{s.}{vazio; nulidade; anulação | resumo; teoria; princípios orientadores; ideologia política e outros aspectos}
    \definition{v.}{reservar espaço}
  \seealsoref{实}{shi2}
  \end{Phonetics}
\end{Entry}

\begin{Entry}{虚心}{11,4}{⾌、⼼}
  \begin{Phonetics}{虚心}{xu1xin1}[][HSK 5]
    \definition{adj.}{modesto; humilde; de mente aberta; não ser presunçoso, ser capaz de aceitar as opiniões dos outros}
  \end{Phonetics}
\end{Entry}

\begin{Entry}{虚伪}{11,6}{⾌、⼈}
  \begin{Phonetics}{虚伪}{xu1wei3}
    \definition{adj.}{falso | hipócrita | artificial}
  \end{Phonetics}
\end{Entry}

\begin{Entry}{蛇}{11}{⾍}
  \begin{Phonetics}{蛇}{she2}[][HSK 5]
    \definition[条]{s.}{cobra; serpente; répteis}
  \end{Phonetics}
\end{Entry}

\begin{Entry}{蛋}{11}{⾍}
  \begin{Phonetics}{蛋}{dan4}[][HSK 2]
    \definition[个,只]{s.}{ovo; ovos produzidos por aves, tartarugas, cobras, etc. | algo em forma de ovo | tolo; idiota; metáfora para pessoas com determinadas características (com conotação pejorativa) | se perder; colocado após certos verbos, forma um verbo com conotação pejorativa | testículos; em algumas regiões, refere-se aos testículos de certos animais ou pessoas}
  \end{Phonetics}
\end{Entry}

\begin{Entry}{蛋白质}{11,5,8}{⾍、⽩、⾙}
  \begin{Phonetics}{蛋白质}{dan4bai2zhi4}[][HSK 7-9]
    \definition{s.}{proteína}
  \end{Phonetics}
\end{Entry}

\begin{Entry}{蛋糕}{11,16}{⾍、⽶}
  \begin{Phonetics}{蛋糕}{dan4gao1}[][HSK 5]
    \definition[个,块,盒]{s.}{bolo; bolo fofo feito de ovos e farinha com açúcar e óleo}
  \end{Phonetics}
\end{Entry}

\begin{Entry}{袋}{11}{⾐}
  \begin{Phonetics}{袋}{dai4}[][HSK 4]
    \definition{clas.}{usado para coisas que podem ser colocadas nos bolsos | usado para cigarros, narguilé ou tabaco seco}
    \definition[口]{s.}{saco; sacola; bolso; bolsa}
  \end{Phonetics}
\end{Entry}

\begin{Entry}{袭}{11}{⾐}
  \begin{Phonetics}{袭}{xi2}
    \definition*{s.}{Sobrenome Xi}
    \definition{clas.}{usado para conjuntos completos de roupas}
    \definition{v.}{fazer um ataque surpresa a; invadir | seguir o padrão de; continuar como antes; fazer o mesmo}
  \end{Phonetics}
\end{Entry}

\begin{Entry}{袭击}{11,5}{⾐、⼐}
  \begin{Phonetics}{袭击}{xi2ji1}
    \definition{s.}{ataque (especialmente um ataque surpresa) | invasão}
    \definition{v.}{atacar}
  \end{Phonetics}
\end{Entry}

\begin{Entry}{谎}{11}{⾔}
  \begin{Phonetics}{谎}{huang3}
    \definition[句]{s.}{mentira; falsidade}
    \definition{v.}{contar uma mentira; mentir}
  \end{Phonetics}
\end{Entry}

\begin{Entry}{谎话}{11,8}{⾔、⾔}
  \begin{Phonetics}{谎话}{huang3hua4}
    \definition{s.}{mentira}
  \end{Phonetics}
\end{Entry}

\begin{Entry}{谐}{11}{⾔}
  \begin{Phonetics}{谐}{xie2}
    \definition{adj.}{harmonioso | humorístico}
  \end{Phonetics}
\end{Entry}

\begin{Entry}{象}{11}{⾗}
  \begin{Phonetics}{象}{xiang4}
    \definition*{s.}{Sobrenome Xiang}
    \definition[头,群,个]{s.}{elefante | elefante, uma das peças do xadrez chinês | aparência; forma; imagem}
    \definition{v.}{imitar | latir}
  \end{Phonetics}
\end{Entry}

\begin{Entry}{象征}{11,8}{⾗、⼻}
  \begin{Phonetics}{象征}{xiang4zheng1}[][HSK 5]
    \definition[种]{s.}{símbolo; emblema; insígnia; \emph{token}; objeto concreto que simboliza um significado especial}
    \definition{v.}{simbolizar; significar; representar; expressar um significado especial através de algo concreto}
  \end{Phonetics}
\end{Entry}

\begin{Entry}{象棋}{11,12}{⾗、⽊}
  \begin{Phonetics}{象棋}{xiang4qi2}
    \definition[副]{s.}{xadrez chinês; um tipo de jogo de xadrez em que dois jogadores têm dezesseis peças cada: um general, dois soldados, dois elefantes, duas carruagens, dois cavalos, dois canhões e cinco soldados ; cada jogador joga de acordo com as regras e o vencedor é aquele que der o xeque no general do adversário}
  \end{Phonetics}
\end{Entry}

\begin{Entry}{距}{11}{⾜}
  \begin{Phonetics}{距}{ju4}
    \definition{s.}{distância | espora (de um galo, etc.)}
    \definition{v.}{estar separado (longe) de; estar distante de}
  \end{Phonetics}
\end{Entry}

\begin{Entry}{距离}{11,10}{⾜、⼇}
  \begin{Phonetics}{距离}{ju4li2}[][HSK 4]
    \definition[个,段]{s.}{distância}
    \definition{v.}{estar distante de}
  \end{Phonetics}
\end{Entry}

\begin{Entry}{辅}{11}{⾞}
  \begin{Phonetics}{辅}{fu3}
    \definition*{s.}{Sobrenome Fu}
    \definition{adj.}{subsidiário}
    \definition{s.}{barras laterais do carrinho atuando como proteção da roda; duas barras retas de madeira são adicionadas na parte externa da roda para prender o cubo | maçã do rosto | assistente oficial; títulos oficiais antigos | (literário) território que circunda a capital}
    \definition{v.}{auxiliar; complementar; suplementar | ajudar}
  \end{Phonetics}
\end{Entry}

\begin{Entry}{辅导}{11,6}{⾞、⼨}
  \begin{Phonetics}{辅导}{fu3dao3}[][HSK 7-9]
    \definition{v.}{orientar no estudo ou treinamento; treinar; guiar; dar aulas particulares}
  \end{Phonetics}
\end{Entry}

\begin{Entry}{辅助}{11,7}{⾞、⼒}
  \begin{Phonetics}{辅助}{fu3zhu4}[][HSK 5]
    \definition{adj.}{auxiliar; suplementar; complementar}
    \definition{v.}{auxiliar; ajudar; colocar os outros em primeiro lugar e dar-lhes alguma ajuda externa}
  \end{Phonetics}
\end{Entry}

\begin{Entry}{辆}{11}{⾞}
  \begin{Phonetics}{辆}{liang4}[][HSK 2]
    \definition{clas.}{usado para automóveis, veículos, etc.}
  \end{Phonetics}
\end{Entry}

\begin{Entry}{逮}{11}{⾡}
  \begin{Phonetics}{逮}{dai3}[][HSK 7-9]
    \definition{v.}{capturar; pegar}[猫逮老鼠。===Gatos pegam ratos.]
  \end{Phonetics}
  \begin{Phonetics}{逮}{dai4}
    \definition*{s.}{Sobrenome Dai}
    \definition{v.}{alcançar | prender, usado em 逮捕}
  \seealsoref{逮捕}{dai4bu3}
  \end{Phonetics}
\end{Entry}

\begin{Entry}{逮捕}{11,10}{⾡、⼿}
  \begin{Phonetics}{逮捕}{dai4bu3}[][HSK 7-9]
    \definition{v.}{prender; apreender; levar sob custódia}
  \end{Phonetics}
\end{Entry}

\begin{Entry}{逶}{11}{⾡}
  \begin{Phonetics}{逶}{wei1}
    \definition{adj.}{sinuoso; tortuoso}
  \end{Phonetics}
\end{Entry}

\begin{Entry}{逶迤}{11,8}{⾡、⾡}
  \begin{Phonetics}{逶迤}{wei1yi2}
    \definition{adj.}{sinuoso; tortuoso; descreve a aparência sinuosa e contínua de estradas, montanhas, rios, etc.}
  \end{Phonetics}
\end{Entry}

\begin{Entry}{逻}{11}{⾡}
  \begin{Phonetics}{逻}{luo2}
    \definition{s.}{patrulha | (literário) a beira de um riacho de montanha}
    \definition{v.}{patrulhar; fazer rondas}
  \end{Phonetics}
\end{Entry}

\begin{Entry}{逻辑}{11,13}{⾡、⾞}
  \begin{Phonetics}{逻辑}{luo2ji5}[][HSK 5]
    \definition[套,条,种]{s.}{lógica; lei objetiva; a objetividade das leis que regem o desenvolvimento das coisas | lógica; razão; regras para o pensamento | lógica como ciência do raciocínio, do pensamento; disciplina que estuda a lógica}
  \end{Phonetics}
\end{Entry}

\begin{Entry}{野}{11}{⾥}
  \begin{Phonetics}{野}{ye3}[][HSK 6]
    \definition*{s.}{Sobrenome Ye}
    \definition{adj.}{(de plantas ou animais) selvagem; incultivado; não domesticado; indomável (opp. 家) | rude; áspero | desenfreado; abandonado; indisciplinado | ilícito; sem licença}
    \definition{s.}{espaço aberto; o aberto | limite; fronteira | não está no poder; fora do cargo}
  \seealsoref{家}{jia1}
  \end{Phonetics}
\end{Entry}

\begin{Entry}{野生}{11,5}{⾥、⽣}
  \begin{Phonetics}{野生}{ye3 sheng1}[][HSK 6]
    \definition{adj.}{selvagem; não cultivado; não domesticado}
  \end{Phonetics}
\end{Entry}

\begin{Entry}{铜}{11}{⾦}
  \begin{Phonetics}{铜}{tong2}
    \definition[块]{s.}{cobre (Cu)}
  \end{Phonetics}
\end{Entry}

\begin{Entry}{铜牌}{11,12}{⾦、⽚}
  \begin{Phonetics}{铜牌}{tong2 pai2}[][HSK 6]
    \definition[枚]{s.}{medalha de bronze; o bronze | placa de bronze com nome ou logotipo comercial, etc.}
  \end{Phonetics}
\end{Entry}

\begin{Entry}{铲}{11}{⾦}
  \begin{Phonetics}{铲}{chan3}[][HSK 7-9]
    \definition[个,把]{s.}{pá}
    \definition{v.}{trabalhar com uma pá (ou enxada) | levantar (mover) com uma pá}
  \end{Phonetics}
\end{Entry}

\begin{Entry}{铲子}{11,3}{⾦、⼦}
  \begin{Phonetics}{铲子}{chan3zi5}[][HSK 7-9]
    \definition[把]{s.}{pá; uma ferramenta com uma chapa de ferro grossa, quase quadrada, em uma extremidade e um cabo longo na outra}
  \end{Phonetics}
\end{Entry}

\begin{Entry}{铲车}{11,4}{⾦、⾞}
  \begin{Phonetics}{铲车}{chan3che1}
    \definition[台]{s.}{empilhadeira}
  \end{Phonetics}
\end{Entry}

\begin{Entry}{银}{11}{⾦}
  \begin{Phonetics}{银}{yin2}[][HSK 3]
    \definition*{s.}{Sobrenome Yin}
    \definition{adj.}{prateado; como a cor da prata}
    \definition[锭]{s.}{Ag, prata | refere-se a moeda ou a coisas relacionadas com moeda}
  \end{Phonetics}
\end{Entry}

\begin{Entry}{银色}{11,6}{⾦、⾊}
  \begin{Phonetics}{银色}{yin2 se4}
    \definition{s.}{cor prata; prateado}
  \end{Phonetics}
\end{Entry}

\begin{Entry}{银行}{11,6}{⾦、⾏}
  \begin{Phonetics}{银行}{yin2hang2}[][HSK 2]
    \definition[个,家,所]{s.}{banco; instituições financeiras que operam depósitos, empréstimos, câmbio, poupança e outros negócios}
  \end{Phonetics}
\end{Entry}

\begin{Entry}{银行卡}{11,6,5}{⾦、⾏、⼘}
  \begin{Phonetics}{银行卡}{yin2 hang2 ka3}[][HSK 2]
    \definition{s.}{cartão bancário; cartão ATM}
  \end{Phonetics}
\end{Entry}

\begin{Entry}{银河}{11,8}{⾦、⽔}
  \begin{Phonetics}{银河}{yin2he2}
    \definition*{s.}{Via Láctea}
  \seealsoref{银河系}{yin2he2xi4}
  \end{Phonetics}
\end{Entry}

\begin{Entry}{银河系}{11,8,7}{⾦、⽔、⽷}
  \begin{Phonetics}{银河系}{yin2he2xi4}
    \definition*{s.}{Galáxia Via Láctea}
  \seealsoref{银河}{yin2he2}
  \end{Phonetics}
\end{Entry}

\begin{Entry}{银牌}{11,12}{⾦、⽚}
  \begin{Phonetics}{银牌}{yin2 pai2}[][HSK 3]
    \definition[枚]{s.}{medalha de prata; um tipo de medalha, concedida ao segundo colocado}
  \end{Phonetics}
\end{Entry}

\begin{Entry}{阐}{11}{⾨}
  \begin{Phonetics}{阐}{chan3}
    \definition{v.}{explicar; expor; expressar; divulgar; esclarecer; elucidar}
  \end{Phonetics}
\end{Entry}

\begin{Entry}{阐述}{11,8}{⾨、⾡}
  \begin{Phonetics}{阐述}{chan3shu4}[][HSK 7-9]
    \definition{v.}{explicar; expor; elaborar; discutir}
  \end{Phonetics}
\end{Entry}

\begin{Entry}{随}{11}{⾩}
  \begin{Phonetics}{随}{sui2}[][HSK 3]
    \definition*{s.}{Sobrenome Sui}
    \definition{adv.}{fazer algo imediatamente assim que ocorre, sem demora ou hesitação; usado antes de dois verbos ou frases verbais para indicar que a última ação segue a anterior}
    \definition{prep.}{junto com (alguma outra ação) | apresentando as condições das quais a ação depende}
    \definition{v.}{seguir; vir (ou ir) junto com | concordar com; adaptar-se a | deixar (alguém fazer o que quiser) | (dialeto) parecer-se com; assemelhar-se a | seguir ou agir de acordo com a condição ou circunstância da qual a ação depende}
  \end{Phonetics}
\end{Entry}

\begin{Entry}{随手}{11,4}{⾩、⼿}
  \begin{Phonetics}{随手}{sui2shou3}[][HSK 4]
    \definition{adv.}{convenientemente; sem problemas adicionais; casualmente}
  \end{Phonetics}
\end{Entry}

\begin{Entry}{随处}{11,5}{⾩、⼡}
  \begin{Phonetics}{随处}{sui2chu4}
    \definition{adv.}{em qualquer lugar}
  \end{Phonetics}
\end{Entry}

\begin{Entry}{随后}{11,6}{⾩、⼝}
  \begin{Phonetics}{随后}{sui2 hou4}[][HSK 5]
    \definition{adv.}{logo em seguida; logo depois; indica que segue imediatamente após a ação ou situação anterior (geralmente usado em conjunto com 就)}
  \seealsoref{就}{jiu4}
  \end{Phonetics}
\end{Entry}

\begin{Entry}{随地}{11,6}{⾩、⼟}
  \begin{Phonetics}{随地}{sui2di4}
    \definition{adv.}{qualquer lugar | todo lugar}
  \end{Phonetics}
\end{Entry}

\begin{Entry}{随机存取记忆体}{11,6,6,8,5,4,7}{⾩、⽊、⼦、⼜、⾔、⼼、⼈}
  \begin{Phonetics}{随机存取记忆体}{sui2ji1cun2qu3ji4yi4ti3}
    \definition{s.}{RAM (\emph{random access memory})}
  \seealsoref{内存}{nei4cun2}
  \seealsoref{随机存取存储器}{sui2ji1cun2qu3cun2chu3qi4}
  \end{Phonetics}
\end{Entry}

\begin{Entry}{随机存取存储器}{11,6,6,8,6,12,16}{⾩、⽊、⼦、⼜、⼦、⼈、⼝}
  \begin{Phonetics}{随机存取存储器}{sui2ji1cun2qu3cun2chu3qi4}
    \definition{s.}{RAM (\emph{random access memory})}
  \seealsoref{内存}{nei4cun2}
  \seealsoref{随机存取记忆体}{sui2ji1cun2qu3ji4yi4ti3}
  \end{Phonetics}
\end{Entry}

\begin{Entry}{随时}{11,7}{⾩、⽇}
  \begin{Phonetics}{随时}{sui2shi2}[][HSK 2]
    \definition{adv.}{a qualquer momento; em todos os momentos}
  \end{Phonetics}
\end{Entry}

\begin{Entry}{随便}{11,9}{⾩、⼈}
  \begin{Phonetics}{随便}{sui2bian4}[][HSK 2]
    \definition{adj.}{relaxado; descontraído; sem restrições; sem limitações | aleatório; casual; descuidado; indiferente; distraído, não pensa bem antes de falar ou agir | casual; informal; não dá importância aos detalhes}
    \definition{conj.}{qualquer; qualquer que seja; não importa}
    \definition{v.}{deixar alguém à vontade}
  \end{Phonetics}
\end{Entry}

\begin{Entry}{随着}{11,11}{⾩、⽬}
  \begin{Phonetics}{随着}{sui2zhe5}[][HSK 5]
    \definition{prep.}{junto com; na esteira de; em sintonia com; usado no início da frase ou antes do verbo, indica as condições necessárias para que uma ação, comportamento ou evento ocorra}
  \end{Phonetics}
\end{Entry}

\begin{Entry}{随意}{11,13}{⾩、⼼}
  \begin{Phonetics}{随意}{sui2yi4}[][HSK 5]
    \definition{adj.}{aleatório; casual; à vontade; como se deseja}
  \end{Phonetics}
\end{Entry}

\begin{Entry}{隐}{11}{⾩}
  \begin{Phonetics}{隐}{yin3}[][HSK 6]
    \definition*{s.}{Sobrenome Yin}
    \definition{adj.}{escondido; escondido profundamente | latente; adormecido; à espreita}
    \definition{pref.}{cripto-}
    \definition{s.}{segredo; assuntos ocultos}
    \definition{v.}{esconder; esconder da vista; ocultar}
  \end{Phonetics}
\end{Entry}

\begin{Entry}{隐私}{11,7}{⾩、⽲}
  \begin{Phonetics}{隐私}{yin3si1}[][HSK 6]
    \definition[点,些]{s.}{privacidade; segredos de alguém; assuntos pessoais que você não quer contar ou tornar públicos}
  \end{Phonetics}
\end{Entry}

\begin{Entry}{隐藏}{11,17}{⾩、⾋}
  \begin{Phonetics}{隐藏}{yin3 cang2}[][HSK 6]
    \definition{v.}{esconder; ocultar}
  \end{Phonetics}
\end{Entry}

\begin{Entry}{雪}{11}{⾬}
  \begin{Phonetics}{雪}{xue3}[][HSK 2]
    \definition*{s.}{Sobrenome Xue}
    \definition[场,层]{s.}{neve | algo parecido com neve}
    \definition{v.}{limpar; enxugar; remover}
  \end{Phonetics}
\end{Entry}

\begin{Entry}{雪人}{11,2}{⾬、⼈}
  \begin{Phonetics}{雪人}{xue3ren2}
    \definition{s.}{boneco de neve | \emph{Yeti}}
  \end{Phonetics}
\end{Entry}

\begin{Entry}{雪山}{11,3}{⾬、⼭}
  \begin{Phonetics}{雪山}{xue3shan1}
    \definition{s.}{montanha coberta de neve}
  \end{Phonetics}
\end{Entry}

\begin{Entry}{雪花}{11,7}{⾬、⾋}
  \begin{Phonetics}{雪花}{xue3hua1}
    \definition{s.}{floco de neve}
  \end{Phonetics}
\end{Entry}

\begin{Entry}{雪板}{11,8}{⾬、⽊}
  \begin{Phonetics}{雪板}{xue3ban3}
    \definition{s.}{prancha de \emph{snowboard}}
    \definition{v.}{praticar \textit{snowboard}}
  \end{Phonetics}
\end{Entry}

\begin{Entry}{雪葩}{11,12}{⾬、⾋}
  \begin{Phonetics}{雪葩}{xue3pa1}
    \definition{s.}{sorvete}
  \end{Phonetics}
\end{Entry}

\begin{Entry}{雪鞋}{11,15}{⾬、⾰}
  \begin{Phonetics}{雪鞋}{xue3xie2}
    \definition[双]{s.}{sapatos de neve}
  \end{Phonetics}
\end{Entry}

\begin{Entry}{雪糕}{11,16}{⾬、⽶}
  \begin{Phonetics}{雪糕}{xue3gao1}
    \definition{s.}{picolé}
  \end{Phonetics}
\end{Entry}

\begin{Entry}{领}{11}{⾴}
  \begin{Phonetics}{领}{ling3}[][HSK 3]
    \definition{clas.}{usado para roupas, mantos, esteiras, tapetes, telas, etc.}
    \definition{s.}{pescoço; gargalo | gola; colarinho; faixa de pescoço | esboço; ponto principal; essência}
    \definition{v.}{conduzir; guiar; orientar | possuir; ser o possuidor de; ter jurisdição sobre | obter; conseguir; receber (o que foi distribuído) | aceitar; tomar |entender; compreender (o significado)}
  \end{Phonetics}
\end{Entry}

\begin{Entry}{领先}{11,6}{⾴、⼉}
  \begin{Phonetics}{领先}{ling3xian1}[][HSK 3]
    \definition{v.}{liderar; assumir a liderança; estar na liderança; (velocidade, desempenho, etc.) superar pessoas ou coisas semelhantes, estar na vanguarda}
  \end{Phonetics}
\end{Entry}

\begin{Entry}{领导}{11,6}{⾴、⼨}
  \begin{Phonetics}{领导}{ling3dao3}[][HSK 3]
    \definition[个,位,名,些]{s.}{líder; liderança; pessoa que ocupa uma posição de liderança}
    \definition{v.}{liderar; exercer liderança; (elogio) liderar, gerenciar outras pessoas;  trabalhar com outras pessoas ou avançar em direção a um objetivo}
  \end{Phonetics}
\end{Entry}

\begin{Entry}{领取}{11,8}{⾴、⼜}
  \begin{Phonetics}{领取}{ling3 qu3}[][HSK 6]
    \definition{v.}{sacar; receber; obter; receber o que lhe é enviado}
  \end{Phonetics}
\end{Entry}

\begin{Entry}{领带}{11,9}{⾴、⼱}
  \begin{Phonetics}{领带}{ling3 dai4}[][HSK 5]
    \definition[条]{s.}{colar; gargantilha; gravata}
  \end{Phonetics}
\end{Entry}

\begin{Entry}{领袖}{11,10}{⾴、⾐}
  \begin{Phonetics}{领袖}{ling3xiu4}[][HSK 6]
    \definition[个,位,名]{s.}{líder de estados, grupos políticos, organizações de massa, etc.}
  \end{Phonetics}
\end{Entry}

\begin{Entry}{领情}{11,11}{⾴、⼼}
  \begin{Phonetics}{领情}{ling3/qing2}
    \definition{v.+compl.}{sentir-se grato a alguém}
  \end{Phonetics}
\end{Entry}

\begin{Entry}{颇}{11}{⽪}
  \begin{Phonetics}{颇}{po1}
    \definition*{s.}{Sobrenome Po}
    \definition{adj.}{oblíquo; inclinado para um lado | Literário: tendencioso; incorreto}
    \definition{adv.}{muito; bastante; consideravelmente}
  \end{Phonetics}
\end{Entry}

\begin{Entry}{骑}{11}{⾺}
  \begin{Phonetics}{骑}{qi2}[][HSK 2]
    \definition{s.}{cavalos ou outros animais para montaria | cavalaria; cavaleiro, também se refere genericamente a qualquer pessoa que monta a cavalo}
    \definition{v.}{montar (um animal ou bicicleta); sentar-se na parte de trás de | montar; abranger ambos os lados}
  \end{Phonetics}
\end{Entry}

\begin{Entry}{骑车}{11,4}{⾺、⾞}
  \begin{Phonetics}{骑车}{qi2 che1}[][HSK 2]
    \definition{v.}{andar de bicicleta; pedalar}
  \end{Phonetics}
\end{Entry}

\begin{Entry}{鸽}{11}{⿃}
  \begin{Phonetics}{鸽}{ge1}
    \definition[只]{s.}{pombo}[和平鸽。===Pomba da Paz.]
  \end{Phonetics}
\end{Entry}

\begin{Entry}{鸽子}{11,3}{⿃、⼦}
  \begin{Phonetics}{鸽子}{ge1zi5}[][HSK 7-9]
    \definition[只,对,群]{s.}{pombo}
  \end{Phonetics}
\end{Entry}

\begin{Entry}{鹿}{11}{⿅}[Kangxi 198]
  \begin{Phonetics}{鹿}{lu4}
    \definition*{s.}{Sobrenome Lu}
    \definition[只,头,群]{s.}{cervo | veado}
  \end{Phonetics}
\end{Entry}

\begin{Entry}{麻}{11}{⿇}[Kangxi 200]
  \begin{Phonetics}{麻}{ma2}
    \definition*{s.}{Sobrenome Ma}
    \definition{adj.}{áspero; grosseiro | marcado; manchado | espinhas; manchas ásperas; cicatrizes deixadas após a varíola}
    \definition[棵,株]{s.}{nome geral para cânhamo, linho, etc. | fibra de cânhamo, linho, etc. para têxteis | sésamo; gergelim | marcas de varíola; um rosto com marcas de varíola}
    \definition{v.}{anestesiar | corromper (a mente de alguém); envenenar}
  \end{Phonetics}
\end{Entry}

\begin{Entry}{麻将}{11,9}{⿇、⼨}
  \begin{Phonetics}{麻将}{ma2jiang4}
    \definition*[副]{s.}{Mahjong}
  \end{Phonetics}
\end{Entry}

\begin{Entry}{麻烦}{11,10}{⿇、⽕}
  \begin{Phonetics}{麻烦}{ma2fan5}[][HSK 3]
    \definition{adj.}{incômodo; inconveniente; complicado; trabalhoso; burocrático | incômodo; inconveniente; (a situação) é confusa e complicada}
    \definition[个,些,点,堆]{s.}{problema; inconveniência; assuntos complicados e difíceis de resolver}
    \definition{v.}{incomodar; perturbar; incomodar alguém; irritar; aborrecer; causar incômodo ou sobrecarregar outras pessoas}
  \end{Phonetics}
\end{Entry}

\begin{Entry}{麻辣豆腐}{11,14,7,14}{⿇、⾟、⾖、⾁}
  \begin{Phonetics}{麻辣豆腐}{ma2la4 dou4fu5}
    \definition{s.}{tofú guisado em molho picante (prato)}
  \end{Phonetics}
\end{Entry}

\begin{Entry}{黄}{11}{⿈}[Kangxi 201]
  \begin{Phonetics}{黄}{huang2}[][HSK 2]
    \definition*{s.}{Rio Huanghe, abreviação de 黄河 | Refere-se ao Imperador Amarelo, um imperador da mitologia chinesa antiga | Sobrenome Huang ou Hwang}
    \definition{adj.}{amarelo | obsceno; indecente; pornográfico; símbolo de corrupção e decadência, referindo-se especificamente à pornografia}
    \definition{s.}{gema; ovas de caranguejo; refere-se a certas coisas de cor amarela}
    \definition{v.}{fracassar; dar errado}
  \seealsoref{黄河}{huang2he2}
  \end{Phonetics}
\end{Entry}

\begin{Entry}{黄瓜}{11,5}{⿈、⽠}
  \begin{Phonetics}{黄瓜}{huang2 gua1}[][HSK 4]
    \definition[根,棵,株,条]{s.}{pepino}
  \end{Phonetics}
\end{Entry}

\begin{Entry}{黄色}{11,6}{⿈、⾊}
  \begin{Phonetics}{黄色}{huang2 se4}[][HSK 2]
    \definition{adj.}{decadente; obsceno; erótico; pornográfico; símbolo de corrupção e decadência, referindo-se especificamente à pornografia}
    \definition[种]{s.}{cor amarela}
  \end{Phonetics}
\end{Entry}

\begin{Entry}{黄昏}{11,8}{⿈、⽇}
  \begin{Phonetics}{黄昏}{huang2hun1}
    \definition{s.}{anoitecer}
  \end{Phonetics}
\end{Entry}

\begin{Entry}{黄河}{11,8}{⿈、⽔}
  \begin{Phonetics}{黄河}{huang2he2}
    \definition*{s.}{Rio Amarelo | Rio Huang He}
  \end{Phonetics}
\end{Entry}

\begin{Entry}{黄油}{11,8}{⿈、⽔}
  \begin{Phonetics}{黄油}{huang2you2}
    \definition[盒]{s.}{manteiga}
  \end{Phonetics}
\end{Entry}

\begin{Entry}{黄金}{11,8}{⿈、⾦}
  \begin{Phonetics}{黄金}{huang2jin1}[][HSK 4]
    \definition{adj.}{de primeira qualidade; dourado;}
    \definition[块,克,两]{s.}{ouro; \emph{aurum}; um tipo de metal, de cor amarela, mais precioso, abreviação de 金, frequentemente falado como 金子}
  \seealsoref{金}{jin1}
  \seealsoref{金子}{jin1zi5}
  \end{Phonetics}
\end{Entry}

%%%%% EOF %%%%%

