%%%
%%% 11画
%%%

\section*{11画}\addcontentsline{toc}{section}{11画}

\begin{entry}{假}{11}[Radical 人]
  \begin{phonetics}{假}{jia3}
    \definition{adj.}{falso | artificial}
    \definition{v.}{emprestar}
  \end{phonetics}
  \begin{phonetics}{假}{jia4}
    \definition{s.}{férias}
  \end{phonetics}
\end{entry}

\begin{entry}{假如}{11,6}
  \begin{phonetics}{假如}{jia3ru2}
    \definition{conj.}{se | supondo | em caso}
  \end{phonetics}
\end{entry}

\begin{entry}{假声}{11,7}
  \begin{phonetics}{假声}{jia3sheng1}
    \definition{s.}{falsete}
  \seealsoref{真声}{zhen1sheng1}
  \end{phonetics}
\end{entry}

\begin{entry}{假证件}{11,7,6}
  \begin{phonetics}{假证件}{jia3zheng4jian4}
    \definition{s.}{documentos falsos}
  \end{phonetics}
\end{entry}

\begin{entry}{假使}{11,8}
  \begin{phonetics}{假使}{jia3shi3}
    \definition{conj.}{se | supondo | em caso}
  \end{phonetics}
\end{entry}

\begin{entry}{假的}{11,8}
  \begin{phonetics}{假的}{jia3de5}
    \definition{adj.}{falso | substituto | simulado}
  \end{phonetics}
\end{entry}

\begin{entry}{偏偏}{11,11}
  \begin{phonetics}{偏偏}{pian1pian1}
    \definition{adv.}{voluntariamente | insistentemente | persistentemente | ao contrário da expectativa | infelizmente (indicando que alguma coisa aconteceu ao contrário do que se esperava) | teimosamente (indicando que algo é o oposto ao que seria normal ou razoável) | precisamente (indicando que alguém ou um grupo é escolhido)}
  \end{phonetics}
\end{entry}

\begin{entry}{做}{11}[Radical 人]
  \begin{phonetics}{做}{zuo4}
    \definition{v.}{fazer}
  \end{phonetics}
\end{entry}

\begin{entry}{做生活}{11,5,9}
  \begin{phonetics}{做生活}{zuo4sheng1huo2}
    \definition{v.}{fazer tabalhos manuais}
  \end{phonetics}
\end{entry}

\begin{entry}{做戏}{11,6}
  \begin{phonetics}{做戏}{zuo4xi4}
    \definition{v.}{atuar em uma peça | fazer uma peça}
  \end{phonetics}
\end{entry}

\begin{entry}{做作}{11,7}
  \begin{phonetics}{做作}{zuo4zuo5}
    \definition{adj.}{afetado | artificial}
  \end{phonetics}
\end{entry}

\begin{entry}{做饭}{11,7}
  \begin{phonetics}{做饭}{zuo4fan4}
    \definition{v.}{preparar uma refeição | cozinhar}
  \end{phonetics}
\end{entry}

\begin{entry}{做法}{11,8}
  \begin{phonetics}{做法}{zuo4fa3}
    \definition[个]{s.}{método para fazer | prática | receita | maneira de lidar com algo | método de trabalho}
  \end{phonetics}
\end{entry}

\begin{entry}{做活}{11,9}
  \begin{phonetics}{做活}{zuo4huo2}
    \definition{v.}{trabalhar para ganhar a vida (especialmente de mulher costureira)}
  \end{phonetics}
\end{entry}

\begin{entry}{做眼}{11,11}
  \begin{phonetics}{做眼}{zuo4yan3}
    \definition{v.}{agir como um guia | trabalhar como espião}
  \end{phonetics}
\end{entry}

\begin{entry}{停}{11}[Radical 人]
  \begin{phonetics}{停}{ting2}
    \definition{v.}{parar | estacionar (um carro)}
  \end{phonetics}
\end{entry}

\begin{entry}{停工}{11,3}
  \begin{phonetics}{停工}{ting2gong1}
    \definition{v.}{parar de trabalhar | parar a produção}
  \end{phonetics}
\end{entry}

\begin{entry}{停办}{11,4}
  \begin{phonetics}{停办}{ting2ban4}
    \definition{v.}{cancelar | sair do negócio | desligar | terminar}
  \end{phonetics}
\end{entry}

\begin{entry}{停止}{11,4}
  \begin{phonetics}{停止}{ting2zhi3}
    \definition{v.}{cessar | encerrar | parar}
  \end{phonetics}
\end{entry}

\begin{entry}{停火}{11,4}
  \begin{phonetics}{停火}{ting2huo3}
    \definition{s.}{cessar-fogo}
    \definition{v.+compl.}{cessar fogo}
  \end{phonetics}
\end{entry}

\begin{entry}{停车}{11,4}
  \begin{phonetics}{停车}{ting2che1}
    \definition{v.}{parar de trabalhar (uma máquina) | estacionar | parar (um veículo) | paralisar}
  \end{phonetics}
\end{entry}

\begin{entry}{停车场}{11,4,6}
  \begin{phonetics}{停车场}{ting2che1chang3}
    \definition{s.}{parque de estacionamento}
  \end{phonetics}
\end{entry}

\begin{entry}{停业}{11,5}
  \begin{phonetics}{停业}{ting2ye4}
    \definition{v.}{cessar a negociação (temporária ou permanentemente) | fechar}
  \end{phonetics}
\end{entry}

\begin{entry}{停用}{11,5}
  \begin{phonetics}{停用}{ting2yong4}
    \definition{v.}{desabilitar | descontinuar | parar de usar | suspender}
  \end{phonetics}
\end{entry}

\begin{entry}{停电}{11,5}
  \begin{phonetics}{停电}{ting2dian4}
    \definition{s.}{corte de energia}
    \definition{v.}{ter uma falha de energia}
  \end{phonetics}
\end{entry}

\begin{entry}{停当}{11,6}
  \begin{phonetics}{停当}{ting2dang5}
    \definition{adj.}{realizado | preparado | assentado}
  \end{phonetics}
\end{entry}

\begin{entry}{停息}{11,10}
  \begin{phonetics}{停息}{ting2xi1}
    \definition{v.}{cessar | parar}
  \end{phonetics}
\end{entry}

\begin{entry}{停留}{11,10}
  \begin{phonetics}{停留}{ting2liu2}
    \definition{v.}{ficar em algum lugar temporariamente | demorar | permanecer}
  \end{phonetics}
\end{entry}

\begin{entry}{停课}{11,10}
  \begin{phonetics}{停课}{ting2ke4}
    \definition{v.}{fechar (escola) | parar as aulas}
  \end{phonetics}
\end{entry}

\begin{entry}{停歇}{11,13}
  \begin{phonetics}{停歇}{ting2xie1}
    \definition{v.}{parar para descansar}
  \end{phonetics}
\end{entry}

\begin{entry}{偶然}{11,12}
  \begin{phonetics}{偶然}{ou3ran2}
    \definition{adv.}{por acaso | fortuitamente}
  \end{phonetics}
\end{entry}

\begin{entry}{偷}{11}[Radical 人]
  \begin{phonetics}{偷}{tou1}
    \definition{adv.}{furtivamente}
    \definition{v.}{furtar | roubar}
  \end{phonetics}
\end{entry}

\begin{entry}{偷安}{11,6}
  \begin{phonetics}{偷安}{tou1'an1}
    \definition{v.}{buscar facilidade temporária}
  \end{phonetics}
\end{entry}

\begin{entry}{偷听}{11,7}
  \begin{phonetics}{偷听}{tou1ting1}
    \definition{v.}{bisbilhotar; monitorar (secretamente)}
  \end{phonetics}
\end{entry}

\begin{entry}{偷窃}{11,9}
  \begin{phonetics}{偷窃}{tou1qie4}
    \definition{v.}{furtar | roubar}
  \end{phonetics}
\end{entry}

\begin{entry}{偷情}{11,11}
  \begin{phonetics}{偷情}{tou1qing2}
    \definition{v.}{manter um caso de amor clandestino}
  \end{phonetics}
\end{entry}

\begin{entry}{偷袭}{11,11}
  \begin{phonetics}{偷袭}{tou1xi2}
    \definition{s.}{ataque surpresa}
    \definition{v.}{montar um ataque furtivo | invadir}
  \end{phonetics}
\end{entry}

\begin{entry}{偷渡}{11,12}
  \begin{phonetics}{偷渡}{tou1du4}
    \definition{s.}{contrabando | imigração ilegal | clandestino (em um navio)}
    \definition{v.}{executar um bloqueio | roubar através da fronteira internacional}
  \end{phonetics}
\end{entry}

\begin{entry}{偷税}{11,12}
  \begin{phonetics}{偷税}{tou1shui4}
    \definition{s.}{evasão fiscal}
  \end{phonetics}
\end{entry}

\begin{entry}{偸}{11}
  \begin{phonetics}{偸}{tou1}
    \variantof{偷}
  \end{phonetics}
\end{entry}

\begin{entry}{副}{11}[Radical 刀]
  \begin{phonetics}{副}{fu4}
    \definition{clas.}{para pares, conjuntos de coisas e expressões faciais | para óculos, luvas, etc.}
  \end{phonetics}
\end{entry}

\begin{entry}{唱}{11}[Radical ⼝]
  \begin{phonetics}{唱}{chang4}
    \definition{v.}{cantar}
  \end{phonetics}
\end{entry}

\begin{entry}{唱歌}{11,14}
  \begin{phonetics}{唱歌}{chang4ge1}
    \definition{v.+compl.}{cantar}
  \end{phonetics}
\end{entry}

\begin{entry}{唾骂}{11,9}
  \begin{phonetics}{唾骂}{tuo4ma4}
    \definition{v.}{insultar | amaldiçoar}
  \end{phonetics}
\end{entry}

\begin{entry}{商店}{11,8}
  \begin{phonetics}{商店}{shang1dian4}
    \definition[家,个]{s.}{loja}
  \end{phonetics}
\end{entry}

\begin{entry}{商贸}{11,9}
  \begin{phonetics}{商贸}{shang1mao4}
    \definition{s.}{comércio}
  \end{phonetics}
\end{entry}

\begin{entry}{啤酒}{11,10}
  \begin{phonetics}{啤酒}{pi2jiu3}
    \definition[杯,瓶,罐,桶,缸]{s.}{(empréstimo linguístico) cerveja}
  \end{phonetics}
\end{entry}

\begin{entry}{啤酒馆}{11,10,11}
  \begin{phonetics}{啤酒馆}{pi2jiu3guan3}
    \definition{s.}{cervejaria}
  \end{phonetics}
\end{entry}

\begin{entry}{啥}{11}[Radical 口]
  \begin{phonetics}{啥}{sha2}
    \definition{adv.}{Equivalente a 什么 (dialeto), também pronunciado como \dpy{sha4}}
    \seeref{什么}{shen2me5}
  \end{phonetics}
  \begin{phonetics}{啥}{sha4}
    \definition{adv.}{Equivalente a 什么 (dialeto), também pronunciado como \dpy{sha2}}
  \end{phonetics}
\end{entry}

\begin{entry}{啵}{11}[Radical 口]
  \begin{phonetics}{啵}{bo1}
    \definition{s.}{(onomatopéia) borbulhar}
  \end{phonetics}
  \begin{phonetics}{啵}{bo5}
    \definition{part.}{partícula gramaticalmente equivalente a 吧}
  \seealsoref{吧}{ba5}
  \end{phonetics}
\end{entry}

\begin{entry}{圈粉}{11,10}
  \begin{phonetics}{圈粉}{quan1fen3}
    \definition{s.}{(neologismo, coloquial) ganhar alguém como fã, obter novos fãs}
  \end{phonetics}
\end{entry}

\begin{entry}{埦}{11}
  \begin{phonetics}{埦}{wan3}
    \variantof{碗}
  \end{phonetics}
\end{entry}

\begin{entry}{基本功}{11,5,5}
  \begin{phonetics}{基本功}{ji1ben3gong1}
    \definition{s.}{habilidades | fundamentos básicos}
  \end{phonetics}
\end{entry}

\begin{entry}{基本法}{11,5,8}
  \begin{phonetics}{基本法}{ji1ben3fa3}
    \definition{s.}{lei básica (constituição)}
  \end{phonetics}
\end{entry}

\begin{entry}{基因}{11,6}
  \begin{phonetics}{基因}{ji1yin1}
    \definition{s.}{gene}
  \end{phonetics}
\end{entry}

\begin{entry}{基督教}{11,13,11}
  \begin{phonetics}{基督教}{ji1du1jiao4}
    \definition*{s.}{Cristianismo | Cristão}
  \end{phonetics}
\end{entry}

\begin{entry}{堵车}{11,4}
  \begin{phonetics}{堵车}{du3che1}
    \definition{v.}{congestionar (trânsito)}
    \definition{v.+compl.}{congestionamento | engarrafamento (de trânsito)}
  \end{phonetics}
\end{entry}

\begin{entry}{够}{11}[Radical ⼣]
  \begin{phonetics}{够}{gou4}
    \definition{adj.}{suficiente}
    \definition{adv.}{(antes do adj.) realmente}
    \definition{v.}{bastar | chegar}
  \end{phonetics}
\end{entry}

\begin{entry}{够不着}{11,4,11}
  \begin{phonetics}{够不着}{gou4bu5zhao2}
    \definition{v.}{ser incapaz de alcançar}
  \end{phonetics}
\end{entry}

\begin{entry}{够本}{11,5}
  \begin{phonetics}{够本}{gou4ben3}
    \definition{v.}{empatar | fazer valer o dinheiro}
  \end{phonetics}
\end{entry}

\begin{entry}{够呛}{11,7}
  \begin{phonetics}{够呛}{gou4qiang4}
    \definition{adj.}{suficiente | terrível | insuportável | improvável}
  \end{phonetics}
\end{entry}

\begin{entry}{够味}{11,8}
  \begin{phonetics}{够味}{gou4wei4}
    \definition{adj.}{excelente | na medida}
  \end{phonetics}
\end{entry}

\begin{entry}{够戗}{11,8}
  \begin{phonetics}{够戗}{gou4qiang4}
    \variantof{够呛}
  \end{phonetics}
\end{entry}

\begin{entry}{够朋友}{11,8,4}
  \begin{phonetics}{够朋友}{gou4peng2you5}
    \definition{v.}{ser um amigo verdadeiro}
  \end{phonetics}
\end{entry}

\begin{entry}{够格}{11,10}
  \begin{phonetics}{够格}{gou4ge2}
    \definition{adj.}{apto | qualificado | apresentável}
  \end{phonetics}
\end{entry}

\begin{entry}{够得着}{11,11,11}
  \begin{phonetics}{够得着}{gou4de5zhao2}
    \definition{v.}{estar à altura | alcançar}
  \end{phonetics}
\end{entry}

\begin{entry}{婚礼}{11,5}
  \begin{phonetics}{婚礼}{hun1li3}
    \definition[场]{s.}{casamento | núpcias | cerimônia de casamento}
  \end{phonetics}
\end{entry}

\begin{entry}{宿舍}{11,8}
  \begin{phonetics}{宿舍}{su4she4}
    \definition[间]{s.}{dormitório | quarto de dormir | hostel}
  \end{phonetics}
\end{entry}

\begin{entry}{寂寞}{11,13}
  \begin{phonetics}{寂寞}{ji4mo4}
    \definition{adj.}{sozinho | solitário | (de um lugar) silencioso}
  \end{phonetics}
\end{entry}

\begin{entry}{寂寥}{11,14}
  \begin{phonetics}{寂寥}{ji4liao2}
    \definition{s.}{solidão | vasto e vazio | quieto e desolado (literário)}
  \end{phonetics}
\end{entry}

\begin{entry}{寄}{11}[Radical 宀]
  \begin{phonetics}{寄}{ji4}
    \definition{v.}{enviar | mandar}
  \end{phonetics}
\end{entry}

\begin{entry}{寄予}{11,4}
  \begin{phonetics}{寄予}{ji4yu3}
    \definition{v.}{expressar | colocar (esperança, importância, etc.) em | mostrar}
  \end{phonetics}
\end{entry}

\begin{entry}{寄生}{11,5}
  \begin{phonetics}{寄生}{ji4sheng1}
    \definition{s.}{parasita | parasitismo}
    \definition{v.}{viver tirando vantagem dos outros | viver dentro ou sobre outro organismo como um parasita}
  \end{phonetics}
\end{entry}

\begin{entry}{寄生生活}{11,5,5,9}
  \begin{phonetics}{寄生生活}{ji4sheng1sheng1huo2}
    \definition{s.}{parasitismo | vida parasitária}
  \end{phonetics}
\end{entry}

\begin{entry}{寄存}{11,6}
  \begin{phonetics}{寄存}{ji4cun2}
    \definition{v.}{depositar | deixar algo com alguém | armazenar}
  \end{phonetics}
\end{entry}

\begin{entry}{寄托}{11,6}
  \begin{phonetics}{寄托}{ji4tuo1}
    \definition{v.}{investir (sua esperança, energia, etc.) em algo | confiar (a alguém) | colocar (a esperança, a energia, etc.) em}
  \end{phonetics}
\end{entry}

\begin{entry}{寄卖}{11,8}
  \begin{phonetics}{寄卖}{ji4mai4}
    \definition{v.}{consignar para venda}
  \end{phonetics}
\end{entry}

\begin{entry}{寄居}{11,8}
  \begin{phonetics}{寄居}{ji4ju1}
    \definition{s.}{morar longe de casa}
  \end{phonetics}
\end{entry}

\begin{entry}{寄放}{11,8}
  \begin{phonetics}{寄放}{ji4fang4}
    \definition{v.}{deixar algo com alguém}
  \end{phonetics}
\end{entry}

\begin{entry}{寄养}{11,9}
  \begin{phonetics}{寄养}{ji4yang3}
    \definition{v.}{embarcar | promover | colocar sob os cuidados de alguém (uma criança, animal de estimação, etc.)}
  \end{phonetics}
\end{entry}

\begin{entry}{寄送}{11,9}
  \begin{phonetics}{寄送}{ji4song4}
    \definition{v.}{enviar | transmitir}
  \end{phonetics}
\end{entry}

\begin{entry}{寄递}{11,10}
  \begin{phonetics}{寄递}{ji4di4}
    \definition{s.}{entrega de correspondência}
  \end{phonetics}
\end{entry}

\begin{entry}{寄售}{11,11}
  \begin{phonetics}{寄售}{ji4shou4}
    \definition{v.}{venda em consignação}
  \end{phonetics}
\end{entry}

\begin{entry}{寄宿}{11,11}
  \begin{phonetics}{寄宿}{ji4su4}
    \definition{s.}{embarque}
    \definition{v.}{embarcar}
  \end{phonetics}
\end{entry}

\begin{entry}{寄望}{11,11}
  \begin{phonetics}{寄望}{ji4wang4}
    \definition{v.}{depositar esperanças em}
  \end{phonetics}
\end{entry}

\begin{entry}{密切}{11,4}
  \begin{phonetics}{密切}{mi4qie4}
    \definition{adj.}{perto | familiar | íntimo}
    \definition{v.}{promover laços estreitos (relacionamento) | prestar muita atenção}
  \end{phonetics}
\end{entry}

\begin{entry}{崇}{11}[Radical ⼭]
  \begin{phonetics}{崇}{chong2}
    \definition*{s.}{sobrenome Chong}
    \definition{adj.}{alto | sublime | elevado}
    \definition{v.}{estimar | adorar}
  \end{phonetics}
\end{entry}

\begin{entry}{崖}{11}[Radical 山]
  \begin{phonetics}{崖}{ya2}
    \definition{s.}{precipício | penhasco}
  \end{phonetics}
\end{entry}

\begin{entry}{崩}{11}[Radical 山]
  \begin{phonetics}{崩}{beng1}
    \definition{s.}{morte de rei ou imperador | desaparecimento}
    \definition{v.}{entrar em colapso | cair em ruínas}
  \end{phonetics}
\end{entry}

\begin{entry}{巢}{11}[Radical ⼮]
  \begin{phonetics}{巢}{chao2}
    \definition*{s.}{sobrenome Chao}
    \definition{s.}{ninho (de aves, etc.)}
  \end{phonetics}
\end{entry}

\begin{entry}{常}{11}[Radical 巾]
  \begin{phonetics}{常}{chang2}
    \definition*{s.}{sobrenome Chang}
    \definition{adv.}{muitas vezes | frequentemente}
  \end{phonetics}
\end{entry}

\begin{entry}{常问问题}{11,6,6,15}
  \begin{phonetics}{常问问题}{chang2wen4wen4ti2}
    \definition{s.}{FAQ; perguntas frequentes}
  \end{phonetics}
\end{entry}

\begin{entry}{常常}{11,11}
  \begin{phonetics}{常常}{chang2chang2}
    \definition{adv.}{frequentemente | com frequência}
  \end{phonetics}
\end{entry}

\begin{entry}{彩虹}{11,9}
  \begin{phonetics}{彩虹}{cai3hong2}
    \definition[道]{s.}{arco-íris}
  \end{phonetics}
\end{entry}

\begin{entry}{得}{11}[Radical 彳]
  \begin{phonetics}{得}{de2}
    \definition{v.}{obter | ganhar | pegar (uma doença)}
  \end{phonetics}
  \begin{phonetics}{得}{de5}
    \definition{part.}{(estrutural) ligando um verbo à frase seguinte indicando efeito, grau, possibilidade, etc.}
  \end{phonetics}
  \begin{phonetics}{得}{dei3}
    \definition{v.}{haver de | ter de}
  \end{phonetics}
\end{entry}

\begin{entry}{得了}{11,2}
  \begin{phonetics}{得了}{de2le5}
    \definition{expr.}{Tudo bem!; É o bastante!}
  \end{phonetics}
  \begin{phonetics}{得了}{de2liao3}
    \definition{adj.}{(enfaticamente, em perguntas retóricas) possível}
  \end{phonetics}
\end{entry}

\begin{entry}{得到}{11,8}
  \begin{phonetics}{得到}{de2dao4}
    \definition{v.}{obter | receber}
  \end{phonetics}
\end{entry}

\begin{entry}{得意}{11,13}
  \begin{phonetics}{得意}{de2yi4}
    \definition{adj.}{orgulhoso de si mesmo | satisfeito consigo mesmo | complacente}
    \definition{v.+compl.}{orgulhar-se de si mesmo; ter satisfação consigo mesmo; ser complacente}
  \end{phonetics}
\end{entry}

\begin{entry}{悉心}{11,4}
  \begin{phonetics}{悉心}{xi1xin1}
    \definition{adv.}{colocar o coração (e a alma) em algo | com muito cuidado}
  \end{phonetics}
\end{entry}

\begin{entry}{悉尼}{11,5}
  \begin{phonetics}{悉尼}{xi1ni2}
    \definition*{s.}{Sidney}
  \end{phonetics}
\end{entry}

\begin{entry}{悉数}{11,13}
  \begin{phonetics}{悉数}{xi1shu3}
    \definition{adv.}{enumerar em detalhes | explicar claramente}
  \end{phonetics}
  \begin{phonetics}{悉数}{xi1shu4}
    \definition{adv.}{todos | cada um | toda a soma}
  \end{phonetics}
\end{entry}

\begin{entry}{您}{11}[Radical 心]
  \begin{phonetics}{您}{nin2}
    \definition{pron.}{você (formal) | tu | te | ti | contigo}
    \seeref{你}{ni3}
  \end{phonetics}
\end{entry}

\begin{entry}{悬挂}{11,9}
  \begin{phonetics}{悬挂}{xuan2gua4}
    \definition{s.}{(veículo) suspensão}
    \definition{v.}{suspender}
  \end{phonetics}
\end{entry}

\begin{entry}{悬崖}{11,11}
  \begin{phonetics}{悬崖}{xuan2ya2}
    \definition{s.}{precipício | penhasco}
  \end{phonetics}
\end{entry}

\begin{entry}{情况}{11,7}
  \begin{phonetics}{情况}{qing2kuang4}
    \definition[个,种]{s.}{circunstância | situação | estado das coisas}
  \end{phonetics}
\end{entry}

\begin{entry}{情绪}{11,11}
  \begin{phonetics}{情绪}{qing2xu4}
    \definition[种]{s.}{humor | estado da mente | mau humor}
  \end{phonetics}
\end{entry}

\begin{entry}{情感}{11,13}
  \begin{phonetics}{情感}{qing2gan3}
    \definition{s.}{sentimento | emoção}
    \definition{v.}{mover-se (emocionalmente)}
  \end{phonetics}
\end{entry}

\begin{entry}{惊呆}{11,7}
  \begin{phonetics}{惊呆}{jing1dai1}
    \definition{adj.}{estupefato | chocado}
  \end{phonetics}
\end{entry}

\begin{entry}{惊喜}{11,12}
  \begin{phonetics}{惊喜}{jing1xi3}
    \definition{s.}{boa surpresa}
    \definition{v.}{ser agradavelmente surpreendido}
  \end{phonetics}
\end{entry}

\begin{entry}{惨}{11}[Radical 心]
  \begin{phonetics}{惨}{can3}
    \definition{adj.}{miserável | cruel | desumano | desastroso | trágico | sombrio}
  \end{phonetics}
\end{entry}

\begin{entry}{捷径}{11,8}
  \begin{phonetics}{捷径}{jie2jing4}
    \definition{s.}{atalho}
  \end{phonetics}
\end{entry}

\begin{entry}{掉}{11}[Radical 手]
  \begin{phonetics}{掉}{diao4}
    \definition{v.}{cair | deixar cair}
  \end{phonetics}
\end{entry}

\begin{entry}{掉队}{11,4}
  \begin{phonetics}{掉队}{diao4dui4}
    \definition{v.}{abandonar | ficar para trás}
  \end{phonetics}
\end{entry}

\begin{entry}{掉包}{11,5}
  \begin{phonetics}{掉包}{diao4bao1}
    \definition{v.}{vender uma falsificação pelo artigo genuíno | roubar o item valioso de alguém e substituí-lo por um item de aparência semelhante, mas sem valor}
  \end{phonetics}
\end{entry}

\begin{entry}{掉线}{11,8}
  \begin{phonetics}{掉线}{diao4xian4}
    \definition{v.}{desconectar-se (da \emph{Internet})}
  \end{phonetics}
\end{entry}

\begin{entry}{掉转}{11,8}
  \begin{phonetics}{掉转}{diao4zhuan3}
    \definition{v.}{dar a volta}
  \end{phonetics}
\end{entry}

\begin{entry}{掉落}{11,12}
  \begin{phonetics}{掉落}{diao4luo4}
    \definition{v.}{derrubar}
  \end{phonetics}
\end{entry}

\begin{entry}{掉膘}{11,15}
  \begin{phonetics}{掉膘}{diao4biao1}
    \definition{v.}{perder peso (gado)}
  \end{phonetics}
\end{entry}

\begin{entry}{排水}{11,4}
  \begin{phonetics}{排水}{pai2shui3}
    \definition{v.}{drenar}
  \end{phonetics}
\end{entry}

\begin{entry}{排队}{11,4}
  \begin{phonetics}{排队}{pai2dui4}
    \definition{v.+compl.}{formar uma fila | alinhar | listar | classificar}
  \end{phonetics}
\end{entry}

\begin{entry}{排球}{11,11}
  \begin{phonetics}{排球}{pai2qiu2}
    \definition[个]{s.}{voleibol}
  \end{phonetics}
\end{entry}

\begin{entry}{探亲}{11,9}
  \begin{phonetics}{探亲}{tan4qin1}
    \definition{v.+compl.}{ir para casa para visitar a família}
  \end{phonetics}
\end{entry}

\begin{entry}{接}{11}[Radical 手]
  \begin{phonetics}{接}{jie1}
    \definition{v.}{ir buscar (alguém) |  ir ao encontro de (alguém) | receber}
  \end{phonetics}
\end{entry}

\begin{entry}{接(电话)}{11,5,8}
  \begin{phonetics}{接(电话)}{jie1(dian4hua4)}
    \definition{v.}{atender (o telefone) | receber (uma ligação telefônica)}
  \end{phonetics}
\end{entry}

\begin{entry}{接待}{11,9}
  \begin{phonetics}{接待}{jie1dai4}
    \definition{v.}{receber (alguém) | acolher | recepcionar}
  \end{phonetics}
\end{entry}

\begin{entry}{接班人}{11,10,2}
  \begin{phonetics}{接班人}{jie1ban1ren2}
    \definition{s.}{sucessor}
  \end{phonetics}
\end{entry}

\begin{entry}{控制}{11,8}
  \begin{phonetics}{控制}{kong4zhi4}
    \definition{v.}{controlar}
  \end{phonetics}
\end{entry}

\begin{entry}{推介}{11,4}
  \begin{phonetics}{推介}{tui1jie4}
    \definition{s.}{promoção}
    \definition{v.}{promover | introduzir e recomendar}
  \end{phonetics}
\end{entry}

\begin{entry}{推迟}{11,7}
  \begin{phonetics}{推迟}{tui1chi2}
    \definition{v.}{adiar | deixar para mais tarde | tardar}
  \end{phonetics}
\end{entry}

\begin{entry}{敎}{11}
  \begin{phonetics}{敎}{jiao4}
    \variantof{教}
  \end{phonetics}
\end{entry}

\begin{entry}{救出}{11,5}
  \begin{phonetics}{救出}{jiu4chu1}
    \definition{v.}{resgatar | tirar do perigo}
  \end{phonetics}
\end{entry}

\begin{entry}{救护车}{11,7,4}
  \begin{phonetics}{救护车}{jiu4hu4che1}
    \definition[辆]{s.}{ambulância}
  \end{phonetics}
\end{entry}

\begin{entry}{救命}{11,8}
  \begin{phonetics}{救命}{jiu4ming4}
    \definition{interj.}{Socorro! | Salve-me!}
    \definition{v.+compl.}{salvar a vida de alguém}
  \end{phonetics}
\end{entry}

\begin{entry}{教}{11}[Radical 攴]
  \begin{phonetics}{教}{jiao1}
    \definition{v.}{ensinar | lecionar}
  \end{phonetics}
  \begin{phonetics}{教}{jiao4}
    \definition*{s.}{sobrenome Jiao}
    \definition{s.}{religião | ensinamento}
    \definition{v.}{causar | como fazer algo | contar (explicar como fazer algo)}
  \end{phonetics}
\end{entry}

\begin{entry}{教长}{11,4}
  \begin{phonetics}{教长}{jiao4zhang3}
    \definition{s.}{imã (Islã) | mulá}
  \end{phonetics}
\end{entry}

\begin{entry}{教会}{11,6}
  \begin{phonetics}{教会}{jiao1hui4}
    \definition{v.}{mostrar | ensinar}
  \end{phonetics}
  \begin{phonetics}{教会}{jiao4hui4}
    \definition{s.}{igreja cristã}
  \end{phonetics}
\end{entry}

\begin{entry}{教导}{11,6}
  \begin{phonetics}{教导}{jiao4dao3}
    \definition{s.}{instrução | orientação | ensino}
    \definition{v.}{instruir | orientar | ensinar}
  \end{phonetics}
\end{entry}

\begin{entry}{教师}{11,6}
  \begin{phonetics}{教师}{jiao4shi1}
    \definition[个]{s.}{professor | mestre}
  \end{phonetics}
\end{entry}

\begin{entry}{教学}{11,8}
  \begin{phonetics}{教学}{jiao1xue2}
    \definition{v.}{ensinar (como um professor)}
  \end{phonetics}
  \begin{phonetics}{教学}{jiao4xue2}
    \definition[次]{s.}{ensino | instrução}
  \end{phonetics}
\end{entry}

\begin{entry}{教学楼}{11,8,13}
  \begin{phonetics}{教学楼}{jiao4xue2lou2}
    \definition{s.}{edifício de salas de aula}
  \end{phonetics}
\end{entry}

\begin{entry}{教官}{11,8}
  \begin{phonetics}{教官}{jiao4guan1}
    \definition{s.}{instrutor militar}
  \end{phonetics}
\end{entry}

\begin{entry}{教练}{11,8}
  \begin{phonetics}{教练}{jiao4lian4}
    \definition[个,位,名]{s.}{instrutor | treinador (esportes)}
  \end{phonetics}
\end{entry}

\begin{entry}{教室}{11,9}
  \begin{phonetics}{教室}{jiao4shi4}
    \definition[间]{s.}{sala de aula}
  \end{phonetics}
\end{entry}

\begin{entry}{教堂}{11,11}
  \begin{phonetics}{教堂}{jiao4tang2}
    \definition[间]{s.}{igreja | capela}
  \end{phonetics}
\end{entry}

\begin{entry}{教授}{11,11}
  \begin{phonetics}{教授}{jiao4shou4}
    \definition[个,位]{s.}{professor (universitário)}
    \definition{v.}{instruir | palestrar sobre}
  \end{phonetics}
\end{entry}

\begin{entry}{敢情}{11,11}
  \begin{phonetics}{敢情}{gan3qing5}
    \definition{adv.}{claro | acontece que\dots}
  \end{phonetics}
\end{entry}

\begin{entry}{斜阳}{11,6}
  \begin{phonetics}{斜阳}{xie2yang2}
    \definition{s.}{sol poente}
  \end{phonetics}
\end{entry}

\begin{entry}{断交}{11,6}
  \begin{phonetics}{断交}{duan4jiao1}
    \definition{v.+compl.}{terminar uma amizade | romper relações diplomáticas}
  \end{phonetics}
\end{entry}

\begin{entry}{旋转}{11,8}
  \begin{phonetics}{旋转}{xuan2zhuan3}
    \definition{v.}{girar}
  \end{phonetics}
\end{entry}

\begin{entry}{族}{11}[Radical 方]
  \begin{phonetics}{族}{zu2}
    \definition{s.}{raça | nacionalidade | etnia | clã | por extensão, grupo social}
  \end{phonetics}
\end{entry}

\begin{entry}{旣}{11}
  \begin{phonetics}{旣}{ji4}
    \variantof{既}
  \end{phonetics}
\end{entry}

\begin{entry}{晚}{11}[Radical 日]
  \begin{phonetics}{晚}{wan3}
    \definition{adj.}{tarde | noite}
  \end{phonetics}
\end{entry}

\begin{entry}{晚上}{11,3}
  \begin{phonetics}{晚上}{wan3shang5}
    \definition{adv.}{noite | à noite}
  \end{phonetics}
\end{entry}

\begin{entry}{晚会}{11,6}
  \begin{phonetics}{晚会}{wan3hui4}
    \definition[个]{s.}{festa noturna}
  \end{phonetics}
\end{entry}

\begin{entry}{晚报}{11,7}
  \begin{phonetics}{晚报}{wan3bao4}
    \definition{s.}{jornal da noite}
  \end{phonetics}
\end{entry}

\begin{entry}{晚近}{11,7}
  \begin{phonetics}{晚近}{wan3jin4}
    \definition{adj.}{recente | mais recente no passado}
    \definition{adv.}{ultimamente | recentemente}
  \end{phonetics}
\end{entry}

\begin{entry}{晚饭}{11,7}
  \begin{phonetics}{晚饭}{wan3fan4}
    \definition[份,顿,次,餐]{s.}{jantar}
  \end{phonetics}
\end{entry}

\begin{entry}{晚育}{11,8}
  \begin{phonetics}{晚育}{wan3yu4}
    \definition{s.}{parto tardio}
    \definition{v.}{ter um filho mais tarde}
  \end{phonetics}
\end{entry}

\begin{entry}{晚点}{11,9}
  \begin{phonetics}{晚点}{wan3dian3}
    \definition{adj.}{atrasado}
    \definition{s.}{jantar leve}
  \end{phonetics}
\end{entry}

\begin{entry}{晚景}{11,12}
  \begin{phonetics}{晚景}{wan3jing3}
    \definition{s.}{circunstâncias dos anos de declínio de alguém | cena noturna}
  \end{phonetics}
\end{entry}

\begin{entry}{晚餐}{11,16}
  \begin{phonetics}{晚餐}{wan3can1}
    \definition[份,顿,次]{s.}{jantar | refeição noturna}
  \end{phonetics}
\end{entry}

\begin{entry}{梦}{11}[Radical 木]
  \begin{phonetics}{梦}{meng4}
    \definition[场,个]{s.}{sonho}
    \definition{v.}{sonhar}
  \end{phonetics}
\end{entry}

\begin{entry}{梯恩梯}{11,10,11}
  \begin{phonetics}{梯恩梯}{ti1'en1ti1}
    \definition{s.}{(empréstimo linguístico) TNT, trinitrotolueno}
  \end{phonetics}
\end{entry}

\begin{entry}{检查}{11,9}
  \begin{phonetics}{检查}{jian3cha2}
    \definition[次]{s.}{inspeção}
    \definition{v.}{examinar | inspecionar}
  \end{phonetics}
\end{entry}

\begin{entry}{欲}{11}[Radical 欠]
  \begin{phonetics}{欲}{yu4}
    \definition{adj.}{desejo | apetite | paixão | luxúria | ganância}
    \definition{v.}{desejar}
  \end{phonetics}
\end{entry}

\begin{entry}{毫不费力}{11,4,9,2}
  \begin{phonetics}{毫不费力}{hao2bu2fei4li4}
    \definition{adj.}{sem esforço | não gastando o menor esforço}
  \end{phonetics}
\end{entry}

\begin{entry}{毫米}{11,6}
  \begin{phonetics}{毫米}{hao2mi3}
    \definition{s.}{milímetro}
  \end{phonetics}
\end{entry}

\begin{entry}{液体}{11,7}
  \begin{phonetics}{液体}{ye4ti3}
    \definition{adj./s.}{líquido}
  \end{phonetics}
\end{entry}

\begin{entry}{涵}{11}[Radical 水]
  \begin{phonetics}{涵}{han2}
    \definition{s.}{bueiro | galeria}
    \definition{v.}{conter | incluir | entupir}
  \end{phonetics}
\end{entry}

\begin{entry}{淀}{11}[Radical 水]
  \begin{phonetics}{淀}{dian4}
    \definition{adj.}{pantanoso}
    \definition{s.}{lago raso | pântano}
    \definition{v.}{formar sedimentos | precipitar}
  \end{phonetics}
\end{entry}

\begin{entry}{淋}{11}[Radical 水]
  \begin{phonetics}{淋}{lin2}
    \definition{v.}{borrifar | pingar | derramar | encharcar}
  \end{phonetics}
  \begin{phonetics}{淋}{lin4}
    \definition{s.}{gonorréia}
    \definition{v.}{filtrar | coar | drenar}
  \end{phonetics}
\end{entry}

\begin{entry}{淤泥}{11,8}
  \begin{phonetics}{淤泥}{yu1ni2}
    \definition{s.}{lodo}
  \end{phonetics}
\end{entry}

\begin{entry}{深}{11}[Radical 水]
  \begin{phonetics}{深}{shen1}
    \definition{adj.}{profundo}
  \end{phonetics}
\end{entry}

\begin{entry}{深夜}{11,8}
  \begin{phonetics}{深夜}{shen1ye4}
    \definition{adv.}{tarde da noite}
  \end{phonetics}
\end{entry}

\begin{entry}{深厚}{11,9}
  \begin{phonetics}{深厚}{shen1hou4}
    \definition{adj.}{profundo}
  \end{phonetics}
\end{entry}

\begin{entry}{深深}{11,11}
  \begin{phonetics}{深深}{shen1shen1}
    \definition{adj.}{profundo}
    \definition{adv.}{profundamente}
  \end{phonetics}
\end{entry}

\begin{entry}{混乱}{11,7}
  \begin{phonetics}{混乱}{hun4luan4}
    \definition{adj.}{confuso | caótico | desordenado}
    \definition{s.}{caos}
  \end{phonetics}
\end{entry}

\begin{entry}{混饭}{11,7}
  \begin{phonetics}{混饭}{hun4fan4}
    \definition{v.+compl.}{trabalhar para viver}
  \end{phonetics}
\end{entry}

\begin{entry}{清}{11}[Radical 水]
  \begin{phonetics}{清}{qing1}
    \definition*{s.}{sobrenome Qing}
    \definition{adj.}{claro | limpo (água, etc.) | tranquilo | quieto | puro | não corrompido | distinto}
    \definition{v.}{limpar | resolver (contas)}
  \end{phonetics}
\end{entry}

\begin{entry}{清彻}{11,7}
  \begin{phonetics}{清彻}{qing1che4}
    \variantof{清澈}
  \end{phonetics}
\end{entry}

\begin{entry}{清明节}{11,8,5}
  \begin{phonetics}{清明节}{qing1ming2jie2}
    \definition*{s.}{Dia Qingming, Dia dos Finados (uma das 24~divisões do ano solar no calendário lunar chinês:~dia~4 ou 5~de~abril solar)}
  \end{phonetics}
\end{entry}

\begin{entry}{清凉}{11,10}
  \begin{phonetics}{清凉}{qing1liang2}
    \definition{adj.}{fresco | refrescante | (roupa) ousada, reveladora}
  \end{phonetics}
\end{entry}

\begin{entry}{清唱}{11,11}
  \begin{phonetics}{清唱}{qing1chang4}
    \definition{v.}{cantar à capela}
  \end{phonetics}
\end{entry}

\begin{entry}{清爽}{11,11}
  \begin{phonetics}{清爽}{qing1shuang3}
    \definition{adj.}{refrescante | relaxado}
  \end{phonetics}
\end{entry}

\begin{entry}{清理}{11,11}
  \begin{phonetics}{清理}{qing1li3}
    \definition{v.}{limpar | arrumar | descartar}
  \end{phonetics}
\end{entry}

\begin{entry}{清晰}{11,12}
  \begin{phonetics}{清晰}{qing1xi1}
    \definition{adj.}{claro | distinto}
  \end{phonetics}
\end{entry}

\begin{entry}{清楚}{11,13}
  \begin{phonetics}{清楚}{qing1chu5}
    \definition{adj.}{claro | límpido}
    \definition{v.}{ser claro sobre | entender completamente}
  \end{phonetics}
\end{entry}

\begin{entry}{清澈}{11,15}
  \begin{phonetics}{清澈}{qing1che4}
    \definition{adj.}{claro | límpido}
  \end{phonetics}
\end{entry}

\begin{entry}{渐渐}{11,11}
  \begin{phonetics}{渐渐}{jian4jian4}
    \definition{adv.}{pouco a pouco | passo a passo | progressivamente}
  \end{phonetics}
\end{entry}

\begin{entry}{渔}{11}[Radical 水]
  \begin{phonetics}{渔}{yu2}
    \definition[条]{s.}{pescador}
    \definition{v.}{pescar}
  \end{phonetics}
\end{entry}

\begin{entry}{渔夫}{11,4}
  \begin{phonetics}{渔夫}{yu2fu1}
    \definition{s.}{pescador}
  \end{phonetics}
\end{entry}

\begin{entry}{渔民}{11,5}
  \begin{phonetics}{渔民}{yu2min2}
    \definition{s.}{pescadores | povo pescador}
  \end{phonetics}
\end{entry}

\begin{entry}{渔场}{11,6}
  \begin{phonetics}{渔场}{yu2chang3}
    \definition{s.}{área de pesca}
  \end{phonetics}
\end{entry}

\begin{entry}{渔汛}{11,6}
  \begin{phonetics}{渔汛}{yu2xun4}
    \definition{s.}{temporada de pesca}
  \end{phonetics}
\end{entry}

\begin{entry}{渔网}{11,6}
  \begin{phonetics}{渔网}{yu2wang3}
    \definition{s.}{rede de pesca}
  \end{phonetics}
\end{entry}

\begin{entry}{渔具}{11,8}
  \begin{phonetics}{渔具}{yu2ju4}
    \definition{s.}{equipamento de pesca}
  \end{phonetics}
\end{entry}

\begin{entry}{渔轮}{11,8}
  \begin{phonetics}{渔轮}{yu2lun2}
    \definition{s.}{navio de pesca}
  \end{phonetics}
\end{entry}

\begin{entry}{渔捞}{11,10}
  \begin{phonetics}{渔捞}{yu2lao1}
    \definition{s.}{pesca (como atividade comercial)}
  \end{phonetics}
\end{entry}

\begin{entry}{渔猎}{11,11}
  \begin{phonetics}{渔猎}{yu2lie4}
    \definition{s.}{pesca e caça}
    \definition{v.}{saquear | pilhar}
  \end{phonetics}
\end{entry}

\begin{entry}{渔笼}{11,11}
  \begin{phonetics}{渔笼}{yu2long2}
    \definition{s.}{gaiola de pesca | armadilha de pesca}
  \end{phonetics}
\end{entry}

\begin{entry}{渔船}{11,11}
  \begin{phonetics}{渔船}{yu2chuan2}
    \definition[条]{s.}{barco de pesca}
  \seealsoref{鱼船}{yu2chuan2}
  \end{phonetics}
\end{entry}

\begin{entry}{渔船队}{11,11,4}
  \begin{phonetics}{渔船队}{yu2chuan2dui4}
    \definition{s.}{frota pesqueira}
  \end{phonetics}
\end{entry}

\begin{entry}{焊}{11}[Radical 火]
  \begin{phonetics}{焊}{han4}
    \definition{v.}{soldar}
  \end{phonetics}
\end{entry}

\begin{entry}{猎物}{11,8}
  \begin{phonetics}{猎物}{lie4wu4}
    \definition{s.}{presa (vítima de um predador)}
  \end{phonetics}
\end{entry}

\begin{entry}{猛}{11}[Radical 犬]
  \begin{phonetics}{猛}{meng3}
    \definition{adj.}{feroz | violento | corajoso | abrupto | (gíria) incrível}
    \definition{adv.}{de repente}
  \end{phonetics}
\end{entry}

\begin{entry}{猛然}{11,12}
  \begin{phonetics}{猛然}{meng3ran2}
    \definition{adv.}{de repente | abruptamente}
  \end{phonetics}
\end{entry}

\begin{entry}{猜}{11}[Radical 犬]
  \begin{phonetics}{猜}{cai1}
    \definition{v.}{advinhar}
  \end{phonetics}
\end{entry}

\begin{entry}{猪}{11}[Radical 犬]
  \begin{phonetics}{猪}{zhu1}
    \definition[口,头]{s.}{porco | suíno}
  \end{phonetics}
\end{entry}

\begin{entry}{猪头}{11,5}
  \begin{phonetics}{猪头}{zhu1tou2}
    \definition{s.}{tolo | idiota}
  \end{phonetics}
\end{entry}

\begin{entry}{猪柳}{11,9}
  \begin{phonetics}{猪柳}{zhu1liu3}
    \definition{s.}{filé de porco}
  \end{phonetics}
\end{entry}

\begin{entry}{猪笼}{11,11}
  \begin{phonetics}{猪笼}{zhu1long2}
    \definition{s.}{estrutura cilíndrica de bambu ou arame usada para restringir um porco durante o transporte}
  \end{phonetics}
\end{entry}

\begin{entry}{猪窠}{11,13}
  \begin{phonetics}{猪窠}{zhu1ke1}
    \definition{s.}{chiqueiro}
  \end{phonetics}
\end{entry}

\begin{entry}{猫}{11}[Radical 犬]
  \begin{phonetics}{猫}{mao1}
    \definition[只]{s.}{gato |  (empréstimo linguístico) (coloquial) MODEM}
    \definition{v.}{(dialeto) esconder-se}
  \end{phonetics}
  \begin{phonetics}{猫}{mao2}
    \definition{v.}{utilizado em 猫腰 \dpy{mao2yao1}}
    \seeref{猫腰}{mao2yao1}
  \end{phonetics}
\end{entry}

\begin{entry}{猫腰}{11,13}
  \begin{phonetics}{猫腰}{mao2yao1}
    \definition{v.}{curvar-se}
  \end{phonetics}
\end{entry}

\begin{entry}{猫熊}{11,14}
  \begin{phonetics}{猫熊}{mao1xiong2}
    \definition[把,只]{s.}{panda gigante}
  \seealsoref{熊猫}{xiong2mao1}
  \end{phonetics}
\end{entry}

\begin{entry}{球}{11}[Radical 玉]
  \begin{phonetics}{球}{qiu2}
    \definition[个]{s.}{bola | esfera | globo}
    \definition[场]{s.}{jogo | partida de bola}
  \end{phonetics}
\end{entry}

\begin{entry}{球拍}{11,8}
  \begin{phonetics}{球拍}{qiu2pai1}
    \definition{s.}{raquete}
  \end{phonetics}
\end{entry}

\begin{entry}{球迷}{11,9}
  \begin{phonetics}{球迷}{qiu2mi2}
    \definition[个]{s.}{fã (esportes de bola)}
  \end{phonetics}
\end{entry}

\begin{entry}{理发}{11,5}
  \begin{phonetics}{理发}{li3fa4}
    \definition{v.+compl.}{fazer um corte de cabelo | cortar o cabelo de alguém}
  \end{phonetics}
\end{entry}

\begin{entry}{理由}{11,5}
  \begin{phonetics}{理由}{li3you2}
    \definition[个]{s.}{razão | justificativa}
  \end{phonetics}
\end{entry}

\begin{entry}{甜}{11}[Radical 甘]
  \begin{phonetics}{甜}{tian2}
    \definition{adj.}{doce}
  \end{phonetics}
\end{entry}

\begin{entry}{甜心}{11,4}
  \begin{phonetics}{甜心}{tian2xin1}
    \definition{s.}{querido}
  \end{phonetics}
\end{entry}

\begin{entry}{甜头}{11,5}
  \begin{phonetics}{甜头}{tian2tou5}
    \definition{s.}{benefício | sabor doce (de poder, sucesso, etc.)}
  \end{phonetics}
\end{entry}

\begin{entry}{甜玉米}{11,5,6}
  \begin{phonetics}{甜玉米}{tian2 yu4mi3}
    \definition{s.}{milho doce}
  \end{phonetics}
\end{entry}

\begin{entry}{甜言}{11,7}
  \begin{phonetics}{甜言}{tian2yan2}
    \definition{s.}{boa conversa | palavras amáveis}
  \end{phonetics}
\end{entry}

\begin{entry}{甜品}{11,9}
  \begin{phonetics}{甜品}{tian2pin3}
    \definition{s.}{sobremesa}
  \end{phonetics}
\end{entry}

\begin{entry}{甜食}{11,9}
  \begin{phonetics}{甜食}{tian2shi2}
    \definition{s.}{doces | sobremesa}
  \end{phonetics}
\end{entry}

\begin{entry}{甜酒}{11,10}
  \begin{phonetics}{甜酒}{tian2jiu3}
    \definition{s.}{licor doce}
  \end{phonetics}
\end{entry}

\begin{entry}{甜甜圈}{11,11,11}
  \begin{phonetics}{甜甜圈}{tian2tian2quan1}
    \definition{s.}{rosquinha | \emph{doughnut}}
  \end{phonetics}
\end{entry}

\begin{entry}{甜菊}{11,11}
  \begin{phonetics}{甜菊}{tian2ju2}
    \definition{s.}{estévia, arbusto cujas folhas produzem um substituto para o açúcar}
  \end{phonetics}
\end{entry}

\begin{entry}{甜筒}{11,12}
  \begin{phonetics}{甜筒}{tian2tong3}
    \definition{s.}{sorvete de casquinha}
  \end{phonetics}
\end{entry}

\begin{entry}{甜稚}{11,13}
  \begin{phonetics}{甜稚}{tian2zhi4}
    \definition{s.}{doce e inocente}
  \end{phonetics}
\end{entry}

\begin{entry}{甜酸}{11,14}
  \begin{phonetics}{甜酸}{tian2suan1}
    \definition{adj.}{agridoce}
  \end{phonetics}
\end{entry}

\begin{entry}{略}{11}[Radical 田]
  \begin{phonetics}{略}{lve4}
    \definition{adv.}{ligeiramente | marginalmente | aproximadamente}
  \end{phonetics}
\end{entry}

\begin{entry}{略微}{11,13}
  \begin{phonetics}{略微}{lve4wei1}
    \definition{adv.}{ligeiramente | marginalmente | aproximadamente}
  \end{phonetics}
\end{entry}

\begin{entry}{盒}{11}[Radical ⽫]
  \begin{phonetics}{盒}{he2}
    \definition{clas.}{caixa pequena}
    \definition{s.}{caixa pequena | estojo}
  \end{phonetics}
\end{entry}

\begin{entry}{盘}{11}[Radical 皿]
  \begin{phonetics}{盘}{pan2}
    \definition{clas.}{para bobinas de fio | (de comida) pratos, serviços | para jogos de xadrez}
    \definition{s.}{tabuleiro | prato | bandeja | (computação) disco rígido}
    \definition{v.}{construir | checar | enrolar | examinar | transferir (propriedade)}
  \end{phonetics}
\end{entry}

\begin{entry}{盛宴}{11,10}
  \begin{phonetics}{盛宴}{sheng4yan4}
    \definition{s.}{celebração}
  \end{phonetics}
\end{entry}

\begin{entry}{眯}{11}[Radical 目]
  \begin{phonetics}{眯}{mi1}
    \definition{v.}{estreitar os olhos | esmagar | (dialeto) tirar uma soneca}
  \end{phonetics}
  \begin{phonetics}{眯}{mi2}
    \definition{v.}{cegar (como com poeira)}
  \end{phonetics}
\end{entry}

\begin{entry}{眼}{11}[Radical 目]
  \begin{phonetics}{眼}{yan3}
    \definition{clas.}{para grandes coisas ocas: poços, fogões, panelas, etc.}
    \definition[只,双]{s.}{ponto crucial (de um assunto) | olho | pequeno buraco}
  \end{phonetics}
\end{entry}

\begin{entry}{眼花缭乱}{11,7,15,7}
  \begin{phonetics}{眼花缭乱}{yan3hua1liao2luan4}
    \definition{v.}{ficar deslumbrado | deslumbrar}
  \end{phonetics}
\end{entry}

\begin{entry}{眼证}{11,7}
  \begin{phonetics}{眼证}{yan3zheng4}
    \definition{s.}{testemunha ocular}
  \end{phonetics}
\end{entry}

\begin{entry}{眼泪}{11,8}
  \begin{phonetics}{眼泪}{yan3lei4}
    \definition[滴]{s.}{choro | lágrimas}
  \end{phonetics}
\end{entry}

\begin{entry}{眼柄}{11,9}
  \begin{phonetics}{眼柄}{yan3bing3}
    \definition{s.}{pedúnculo ocular (de crustáceo, etc.)}
  \end{phonetics}
\end{entry}

\begin{entry}{眼袋}{11,11}
  \begin{phonetics}{眼袋}{yan3dai4}
    \definition{s.}{inchaço sob os olhos}
  \end{phonetics}
\end{entry}

\begin{entry}{眼睛}{11,13}
  \begin{phonetics}{眼睛}{yan3jing5}
    \definition[只,双]{s.}{olho(s)}
  \end{phonetics}
\end{entry}

\begin{entry}{眼镜}{11,16}
  \begin{phonetics}{眼镜}{yan3jing4}
    \definition[副]{s.}{óculos}
  \end{phonetics}
\end{entry}

\begin{entry}{着}{11}[Radical 目]
  \begin{phonetics}{着}{zhao1}
    \definition{interj.}{Tudo bem!}
    \definition{s.}{movimento (xadrez) | truque}
  \end{phonetics}
  \begin{phonetics}{着}{zhao2}
    \definition{v.}{ser afetado por | queimar | pegar fogo | entrar em contato com | sentir | tocar}
  \end{phonetics}
  \begin{phonetics}{着}{zhe5}
    \definition{part.}{indicando ação em andamento ou estado em andamento}
  \end{phonetics}
  \begin{phonetics}{着}{zhuo2}
    \definition{v.}{aplicar | contactar | usar | vestir (roupas)}
  \end{phonetics}
\end{entry}

\begin{entry}{着手}{11,4}
  \begin{phonetics}{着手}{zhuo2shou3}
    \definition{v.}{colocar a mão nisso | estabelecer | começar uma tarefa}
  \end{phonetics}
\end{entry}

\begin{entry}{着地}{11,6}
  \begin{phonetics}{着地}{zhao2di4}
    \definition{v.}{pousar | tocar o chão}
  \end{phonetics}
\end{entry}

\begin{entry}{着花}{11,7}
  \begin{phonetics}{着花}{zhao2hua1}
    \definition{v.}{florescer}
  \end{phonetics}
  \begin{phonetics}{着花}{zhuo2hua1}
    \definition{s.}{floração}
    \definition{v.}{florescer}
  \end{phonetics}
\end{entry}

\begin{entry}{着急}{11,9}
  \begin{phonetics}{着急}{zhao2ji2}
    \definition{adj.}{inquieto | ansioso}
    \definition{s.}{preocupação | ansiedade}
    \definition{v.+compl.}{preocupar-se | sentir-se ansioso | sentir uma sensação de urgência}
  \end{phonetics}
\end{entry}

\begin{entry}{着凉}{11,10}
  \begin{phonetics}{着凉}{zhao2liang2}
    \definition{v.}{pegar um resfriado}
  \end{phonetics}
\end{entry}

\begin{entry}{着眼}{11,11}
  \begin{phonetics}{着眼}{zhuo2yan3}
    \definition{v.}{ter seus olhos em (um objetivo) | ter algo em mente | concentrar-se}
  \end{phonetics}
\end{entry}

\begin{entry}{着装}{11,12}
  \begin{phonetics}{着装}{zhuo2zhuang1}
    \definition{s.}{roupa | vestimenta}
    \definition{v.}{vestir}
  \end{phonetics}
\end{entry}

\begin{entry}{着想}{11,13}
  \begin{phonetics}{着想}{zhuo2xiang3}
    \definition{v.}{considerar (as necessidades de outras pessoas) | pensar (para os outros)}
  \end{phonetics}
\end{entry}

\begin{entry}{着数}{11,13}
  \begin{phonetics}{着数}{zhao1shu4}
    \definition{s.}{estratégia | movimento (no xadrez, no palco, nas artes marciais) | esquema | truque}
  \end{phonetics}
\end{entry}

\begin{entry}{票}{11}[Radical 示]
  \begin{phonetics}{票}{piao4}
    \definition{clas.}{para grupos, lotes, transações comerciais}
    \definition[张]{s.}{performance amadora de ópera chinesa | cédula eleitoral | nota | bilhete | pessoa detida por resgate}
  \end{phonetics}
\end{entry}

\begin{entry}{章}{11}[Radical 音]
  \begin{phonetics}{章}{zhang1}
    \definition*{s.}{sobrenome Zhang}
    \definition{s.}{capítulo | seção | cláusula |  movimento (de sinfonia) | selo | crachá | regulamento}
  \end{phonetics}
\end{entry}

\begin{entry}{章鱼}{11,8}
  \begin{phonetics}{章鱼}{zhang1yu2}
    \definition{s.}{polvo | octópode}
  \end{phonetics}
\end{entry}

\begin{entry}{笛}{11}[Radical 竹]
  \begin{phonetics}{笛}{di2}
    \definition{s.}{flauta}
  \end{phonetics}
\end{entry}

\begin{entry}{符合}{11,6}
  \begin{phonetics}{符合}{fu2he2}
    \definition{conj.}{de acordo com | concordando com | contando com | alinhado com}
    \definition{v.}{concordar com | estar em conformidade com | corresponder com | gerenciar | lidar}
  \end{phonetics}
\end{entry}

\begin{entry}{笨蛋}{11,11}
  \begin{phonetics}{笨蛋}{ben4dan4}
    \definition{s.}{bobalhão | cabeça-oca | cabeça-dura}
    \definition{v.}{iludir | enganar}
  \end{phonetics}
\end{entry}

\begin{entry}{第}{11}[Radical 竹]
  \begin{phonetics}{第}{di4}
    \definition{num.}{prefixo para expressar números ordinais}
  \end{phonetics}
\end{entry}

\begin{entry}{笼}{11}[Radical 竹]
  \begin{phonetics}{笼}{long2}
    \definition{s.}{armação fechada de bambu, arame, etc. | jaula | gaiola}
  \end{phonetics}
  \begin{phonetics}{笼}{long3}
    \definition{v.}{envolver | cobrir}
  \end{phonetics}
\end{entry}

\begin{entry}{笼子}{11,3}
  \begin{phonetics}{笼子}{long2zi5}
    \definition{s.}{jaula | cesta | gaiola | recipiente}
  \end{phonetics}
  \begin{phonetics}{笼子}{long3zi5}
    \definition{s.}{caixa grande | porta-malas}
  \end{phonetics}
\end{entry}

\begin{entry}{粗心}{11,4}
  \begin{phonetics}{粗心}{cu1xin1}
    \definition{adj.}{descuido}
  \end{phonetics}
\end{entry}

\begin{entry}{粗心地做}{11,4,6,11}
  \begin{phonetics}{粗心地做}{cu1xin1 di4 zuo4}
    \definition{adj.}{feito descuidadamente}
  \end{phonetics}
\end{entry}

\begin{entry}{粗糙}{11,16}
  \begin{phonetics}{粗糙}{cu1cao1}
    \definition{adj.}{áspero | grosseiro}
  \end{phonetics}
\end{entry}

\begin{entry}{累}{11}[Radical 糸]
  \begin{phonetics}{累}{lei2}
    \definition*{s.}{sobrenome Lei}
    \definition{s.}{corda}
    \definition{v.}{amarrar | torcer}
  \end{phonetics}
  \begin{phonetics}{累}{lei3}
    \definition{adj.}{contínuo | repetido}
    \definition{v.}{acumular | envolver ou implicar}
  \end{phonetics}
  \begin{phonetics}{累}{lei4}
    \definition{adj.}{cansado | fatigado}
    \definition{v.}{forçar | desgastar | trabalhar duro}
  \end{phonetics}
\end{entry}

\begin{entry}{绰号}{11,5}
  \begin{phonetics}{绰号}{chuo4hao4}
    \definition{s.}{apelido}
  \end{phonetics}
\end{entry}

\begin{entry}{绳子}{11,3}
  \begin{phonetics}{绳子}{sheng2zi5}
    \definition[条]{s.}{corda | cordão}
  \end{phonetics}
\end{entry}

\begin{entry}{维吾尔}{11,7,5}
  \begin{phonetics}{维吾尔}{wei2wu2'er3}
    \definition*{s.}{Grupo étnico Uigur de Xinjiang}
  \end{phonetics}
\end{entry}

\begin{entry}{绷带}{11,9}
  \begin{phonetics}{绷带}{beng1dai4}
    \definition{s.}{curativo | (empréstimo linguístico) \emph{bandage}}
  \end{phonetics}
\end{entry}

\begin{entry}{绿}{11}[Radical 糸]
  \begin{phonetics}{绿}{lv4}
    \definition{adj.}{verde}
  \end{phonetics}
\end{entry}

\begin{entry}{绿色}{11,6}
  \begin{phonetics}{绿色}{lv4se4}
    \definition{s.}{cor verde}
  \end{phonetics}
\end{entry}

\begin{entry}{绿豆}{11,7}
  \begin{phonetics}{绿豆}{lv4dou4}
    \definition{s.}{vagens}
  \end{phonetics}
\end{entry}

\begin{entry}{绿豆芽}{11,7,7}
  \begin{phonetics}{绿豆芽}{lv4dou4 ya2}
    \definition{s.}{broto de feijão verde}
  \end{phonetics}
\end{entry}

\begin{entry}{聊天}{11,4}
  \begin{phonetics}{聊天}{liao2tian1}
    \definition{v.+compl.}{papear | bater papo}
  \end{phonetics}
\end{entry}

\begin{entry}{职业}{11,5}
  \begin{phonetics}{职业}{zhi2ye4}
    \definition{adj.}{profissional}
    \definition{s.}{ocupação | profissão | vocação}
  \end{phonetics}
\end{entry}

\begin{entry}{职员}{11,7}
  \begin{phonetics}{职员}{zhi2yuan2}
    \definition[个,位]{s.}{empregado | trabalhador de escritório | membro da equipe}
  \end{phonetics}
\end{entry}

\begin{entry}{脖子}{11,3}
  \begin{phonetics}{脖子}{bo2zi5}
    \definition[个]{s.}{pescoço}
  \end{phonetics}
\end{entry}

\begin{entry}{脚}{11}[Radical 肉]
  \begin{phonetics}{脚}{jiao3}
    \definition{clas.}{para chutes}
    \definition[双,只]{s.}{pé | base (de um objeto) | perna (de um animal ou objeto)}
  \end{phonetics}
  \begin{phonetics}{脚}{jue2}
    \variantof{角}
  \end{phonetics}
\end{entry}

\begin{entry}{脱毛}{11,4}
  \begin{phonetics}{脱毛}{tuo1mao2}
    \definition{s.}{depilação}
    \definition{v.}{perder cabelo ou penas | depilar | fazer a barba}
  \end{phonetics}
\end{entry}

\begin{entry}{脱险}{11,9}
  \begin{phonetics}{脱险}{tuo1xian3}
    \definition{v.}{sair do perigo}
  \end{phonetics}
\end{entry}

\begin{entry}{脸}{11}[Radical 肉]
  \begin{phonetics}{脸}{lian3}
    \definition[张,个]{s.}{cara | rosto | face}
  \end{phonetics}
\end{entry}

\begin{entry}{脸色}{11,6}
  \begin{phonetics}{脸色}{lian3se4}
    \definition{s.}{compleição; tez; face}
  \end{phonetics}
\end{entry}

\begin{entry}{船}{11}[Radical ⾈]
  \begin{phonetics}{船}{chuan2}
    \definition[条,艘,只]{s.}{barco | navio}
  \end{phonetics}
\end{entry}

\begin{entry}{菜}{11}[Radical 艸]
  \begin{phonetics}{菜}{cai4}
    \definition[棵]{s.}{hortaliça | verdura | legume}
    \definition[样,道,盘]{s.}{prato (tipo de alimento) | o tipo (de alguém) | (características de alguém, etc.) fraco, pobre}
  \end{phonetics}
\end{entry}

\begin{entry}{菜刀}{11,2}
  \begin{phonetics}{菜刀}{cai4dao1}
    \definition[把]{s.}{faca de vegetais | faca de cozinha | cutelo}
  \end{phonetics}
\end{entry}

\begin{entry}{菜单}{11,8}
  \begin{phonetics}{菜单}{cai4dan1}
    \definition[份,张]{s.}{menu | cardápio}
  \end{phonetics}
\end{entry}

\begin{entry}{菠菜}{11,11}
  \begin{phonetics}{菠菜}{bo1cai4}
    \definition[棵]{s.}{espinafre}
  \end{phonetics}
\end{entry}

\begin{entry}{菱角}{11,7}
  \begin{phonetics}{菱角}{ling2jiao5}
    \definition{s.}{castanha d'água}
  \end{phonetics}
\end{entry}

\begin{entry}{虚伪}{11,6}
  \begin{phonetics}{虚伪}{xu1wei3}
    \definition{adj.}{falso | hipócrita | artificial}
  \end{phonetics}
\end{entry}

\begin{entry}{蛇}{11}[Radical 虫]
  \begin{phonetics}{蛇}{she2}
    \definition[条]{s.}{cobra | serpente}
  \end{phonetics}
\end{entry}

\begin{entry}{蛋}{11}[Radical 足]
  \begin{phonetics}{蛋}{dan4}
    \definition[个,打]{s.}{ovo | objeto de formato oval}
  \end{phonetics}
\end{entry}

\begin{entry}{蛋糕}{11,16}
  \begin{phonetics}{蛋糕}{dan4gao1}
    \definition[块,个]{s.}{bolo}
  \end{phonetics}
\end{entry}

\begin{entry}{袭击}{11,5}
  \begin{phonetics}{袭击}{xi2ji1}
    \definition{s.}{ataque (especialmente um ataque surpresa) | invasão}
    \definition{v.}{atacar}
  \end{phonetics}
\end{entry}

\begin{entry}{谎话}{11,8}
  \begin{phonetics}{谎话}{huang3hua4}
    \definition{s.}{mentira}
  \end{phonetics}
\end{entry}

\begin{entry}{谐}{11}[Radical 言]
  \begin{phonetics}{谐}{xie2}
    \definition{adj.}{harmonioso | humorístico}
  \end{phonetics}
\end{entry}

\begin{entry}{距离}{11,10}
  \begin{phonetics}{距离}{ju4li2}
    \definition[个]{s.}{distância}
    \definition{v.}{estar distante de}
  \end{phonetics}
\end{entry}

\begin{entry}{辆}{11}[Radical 車]
  \begin{phonetics}{辆}{liang4}
    \definition{clas.}{para automóveis, veículos, etc.}
  \end{phonetics}
\end{entry}

\begin{entry}{逮}{11}[Radical 辵]
  \begin{phonetics}{逮}{dai3}
    \definition{v.}{(coloquial) pegar, aproveitar, capturar}
  \end{phonetics}
  \begin{phonetics}{逮}{dai4}
    \definition{v.}{(literário) alcançar, usado em 逮捕}
  \seealsoref{逮捕}{dai4bu3}
  \end{phonetics}
\end{entry}

\begin{entry}{逮捕}{11,10}
  \begin{phonetics}{逮捕}{dai4bu3}
    \definition{v.}{prender | apreender | levar sob custódia}
  \end{phonetics}
\end{entry}

\begin{entry}{野}{11}[Radical 里]
  \begin{phonetics}{野}{ye3}
    \definition{adj.}{selvagem | rude}
    \definition{s.}{campo | espaço aberto | limite}
  \end{phonetics}
\end{entry}

\begin{entry}{野生}{11,5}
  \begin{phonetics}{野生}{ye3sheng1}
    \definition{adj.}{selvagem | não domesticado}
  \end{phonetics}
\end{entry}

\begin{entry}{铲车}{11,4}
  \begin{phonetics}{铲车}{chan3che1}
    \definition[台]{s.}{empilhadeira}
  \end{phonetics}
\end{entry}

\begin{entry}{银色}{11,6}
  \begin{phonetics}{银色}{yin2se4}
    \definition{s.}{prateado}
  \end{phonetics}
\end{entry}

\begin{entry}{银行}{11,6}
  \begin{phonetics}{银行}{yin2hang2}
    \definition[家,个]{s.}{banco | agência bancária}
  \end{phonetics}
\end{entry}

\begin{entry}{银河}{11,8}
  \begin{phonetics}{银河}{yin2he2}
    \definition*{s.}{Via Láctea}
  \seealsoref{银河系}{yin2he2xi4}
  \end{phonetics}
\end{entry}

\begin{entry}{银河系}{11,8,7}
  \begin{phonetics}{银河系}{yin2he2xi4}
    \definition*{s.}{Galáxia Via Láctea}
  \seealsoref{银河}{yin2he2}
  \end{phonetics}
\end{entry}

\begin{entry}{随处}{11,5}
  \begin{phonetics}{随处}{sui2chu4}
    \definition{adv.}{em qualquer lugar}
  \end{phonetics}
\end{entry}

\begin{entry}{随地}{11,6}
  \begin{phonetics}{随地}{sui2di4}
    \definition{adv.}{qualquer lugar | todo lugar}
  \end{phonetics}
\end{entry}

\begin{entry}{随机存取记忆体}{11,6,6,8,5,4,7}
  \begin{phonetics}{随机存取记忆体}{sui2ji1cun2qu3ji4yi4ti3}
    \definition{s.}{RAM (\emph{random access memory})}
  \seealsoref{内存}{nei4cun2}
  \seealsoref{随机存取存储器}{sui2ji1cun2qu3cun2chu3qi4}
  \end{phonetics}
\end{entry}

\begin{entry}{随机存取存储器}{11,6,6,8,6,12,16}
  \begin{phonetics}{随机存取存储器}{sui2ji1cun2qu3cun2chu3qi4}
    \definition{s.}{RAM (\emph{random access memory})}
  \seealsoref{内存}{nei4cun2}
  \seealsoref{随机存取记忆体}{sui2ji1cun2qu3ji4yi4ti3}
  \end{phonetics}
\end{entry}

\begin{entry}{随时}{11,7}
  \begin{phonetics}{随时}{sui2shi2}
    \definition{adv.}{a qualquer momento | sempre que necessário}
  \end{phonetics}
\end{entry}

\begin{entry}{随便}{11,9}
  \begin{phonetics}{随便}{sui2bian4}
    \definition{adj.}{à vontade | como queira | como desejar | casual | negligente | devasso}
    \definition{adv.}{aleatoriamente}
  \end{phonetics}
\end{entry}

\begin{entry}{雪}{11}[Radical 雨]
  \begin{phonetics}{雪}{xue3}
    \definition*{s.}{sobrenome Xue}
    \definition[场]{s.}{neve}
  \end{phonetics}
\end{entry}

\begin{entry}{雪人}{11,2}
  \begin{phonetics}{雪人}{xue3ren2}
    \definition{s.}{boneco de neve | \emph{Yeti}}
  \end{phonetics}
\end{entry}

\begin{entry}{雪山}{11,3}
  \begin{phonetics}{雪山}{xue3shan1}
    \definition{s.}{montanha coberta de neve}
  \end{phonetics}
\end{entry}

\begin{entry}{雪花}{11,7}
  \begin{phonetics}{雪花}{xue3hua1}
    \definition{s.}{floco de neve}
  \end{phonetics}
\end{entry}

\begin{entry}{雪板}{11,8}
  \begin{phonetics}{雪板}{xue3ban3}
    \definition{s.}{prancha de \emph{snowboard}}
    \definition{v.}{praticar \textit{snowboard}}
  \end{phonetics}
\end{entry}

\begin{entry}{雪葩}{11,12}
  \begin{phonetics}{雪葩}{xue3pa1}
    \definition{s.}{sorvete}
  \end{phonetics}
\end{entry}

\begin{entry}{雪鞋}{11,15}
  \begin{phonetics}{雪鞋}{xue3xie2}
    \definition[双]{s.}{sapatos de neve}
  \end{phonetics}
\end{entry}

\begin{entry}{雪糕}{11,16}
  \begin{phonetics}{雪糕}{xue3gao1}
    \definition{s.}{picolé}
  \end{phonetics}
\end{entry}

\begin{entry}{领}{11}[Radical 頁]
  \begin{phonetics}{领}{ling3}
    \definition{clas.}{para roupas, tapetes, telas, etc.}
    \definition{s.}{pescoço | colarinho}
    \definition{v.}{liderar | receber}
  \end{phonetics}
\end{entry}

\begin{entry}{领导}{11,6}
  \begin{phonetics}{领导}{ling3dao3}
    \definition[位,个]{s.}{líder | liderança}
    \definition{v.}{liderar}
  \end{phonetics}
\end{entry}

\begin{entry}{领情}{11,11}
  \begin{phonetics}{领情}{ling3qing2}
    \definition{v.+compl.}{sentir-se grato a alguém}
  \end{phonetics}
\end{entry}

\begin{entry}{颇}{11}[Radical 頁]
  \begin{phonetics}{颇}{po1}
    \definition*{s.}{sobrenome Po}
    \definition{adv.}{muito, bastante (linguagem escrita)}
  \end{phonetics}
\end{entry}

\begin{entry}{骑}{11}[Radical 馬]
  \begin{phonetics}{骑}{qi2}
    \definition{clas.}{para cavalos de sela}
    \definition{v.}{andar (cavalo, bicicleta, etc.) | sentar-se montado | montar}
  \end{phonetics}
\end{entry}

\begin{entry}{骑车}{11,4}
  \begin{phonetics}{骑车}{qi2che1}
    \definition{v.}{andar de bicicleta | pedalar}
  \end{phonetics}
\end{entry}

\begin{entry}{鸽子}{11,3}
  \begin{phonetics}{鸽子}{ge1zi5}
    \definition{s.}{pombo}
  \end{phonetics}
\end{entry}

\begin{entry}{鹿}{11}[Radical 鹿]
  \begin{phonetics}{鹿}{lu4}
    \definition{s.}{cervo | veado}
  \end{phonetics}
\end{entry}

\begin{entry}{麻将}{11,9}
  \begin{phonetics}{麻将}{ma2jiang4}
    \definition[副]{s.}{\emph{mahjong}}
  \end{phonetics}
\end{entry}

\begin{entry}{麻烦}{11,10}
  \begin{phonetics}{麻烦}{ma2fan5}
    \definition{adj.}{fastidioso | maçante | inconveniente | problemático}
    \definition{s.}{incômodo}
    \definition{v.}{incomodar alguém | colocar alguém em apuros}
  \end{phonetics}
\end{entry}

\begin{entry}{麻辣豆腐}{11,14,7,14}
  \begin{phonetics}{麻辣豆腐}{ma2la4 dou4fu5}
    \definition{s.}{tofú guisado em molho picante (prato)}
  \end{phonetics}
\end{entry}

\begin{entry}{黄}{11}[Radical ⻩][Kangxi 201]
  \begin{phonetics}{黄}{huang2}
    \definition*{s.}{sobrenome Huang ou Hwang}
    \definition{adj.}{amarelo | pornográfico}
  \end{phonetics}
\end{entry}

\begin{entry}{黄瓜}{11,5}
  \begin{phonetics}{黄瓜}{huang2gua1}
    \definition[条]{s.}{pepino}
  \end{phonetics}
\end{entry}

\begin{entry}{黄色}{11,6}
  \begin{phonetics}{黄色}{huang2se4}
    \definition{s.}{cor amarela}
  \end{phonetics}
\end{entry}

\begin{entry}{黄昏}{11,8}
  \begin{phonetics}{黄昏}{huang2hun1}
    \definition{s.}{anoitecer}
  \end{phonetics}
\end{entry}

\begin{entry}{黄油}{11,8}
  \begin{phonetics}{黄油}{huang2you2}
    \definition[盒]{s.}{manteiga}
  \end{phonetics}
\end{entry}

%%%%% EOF %%%%%

