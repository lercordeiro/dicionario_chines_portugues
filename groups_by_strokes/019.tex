%%%
%%% 19画
%%%

\section*{19画}\addcontentsline{toc}{section}{19画}

\begin{entry}{攀}{19}{⼿}
  \begin{phonetics}{攀}{pan1}
    \definition{v.}{escalar; escalar | buscar conexões em altos cargos | envolver; implicar | agarrar; agarrar-se; segurar-se a}
  \end{phonetics}
\end{entry}

\begin{entry}{攀岩}{19,8}{⼿、⼭}
  \begin{phonetics}{攀岩}{pan1yan2}
    \definition{s.}{alpinista}
    \definition{v.}{escalar uma montanha}
  \end{phonetics}
\end{entry}

\begin{entry}{攀爬}{19,8}{⼿、⽖}
  \begin{phonetics}{攀爬}{pan1pa2}
    \definition{v.}{escalar}
  \end{phonetics}
\end{entry}

\begin{entry}{爆}{19}{⽕}
  \begin{phonetics}{爆}{bao4}[][HSK 6]
    \definition{v.}{explodir; estourar | fritar rapidamente; ferver rapidamente | aparecer (ou ocorrer) inesperadamente}
  \end{phonetics}
\end{entry}

\begin{entry}{爆发}{19,5}{⽕、⼜}
  \begin{phonetics}{爆发}{bao4fa1}[][HSK 6]
    \definition{v.}{entrar em erupção; explodir | estourar; irromper; ocorrer de forma repentina e violenta}
  \end{phonetics}
\end{entry}

\begin{entry}{爆米花}{19,6,7}{⽕、⽶、⾋}
  \begin{phonetics}{爆米花}{bao4mi3hua1}
    \definition{s.}{pipoca (de milho) | pipoca de arroz}
  \end{phonetics}
\end{entry}

\begin{entry}{爆炸}{19,9}{⽕、⽕}
  \begin{phonetics}{爆炸}{bao4zha4}[][HSK 6]
    \definition{s.}{explosão}
    \definition{v.}{explodir; explodir; detonar | aumentar bruscamente em um curto espaço de tempo (de quantidade)}
  \end{phonetics}
\end{entry}

\begin{entry}{聼}{19}{⼼}
  \begin{phonetics}{聼}{ting1}
    \variantof{听}
  \end{phonetics}
\end{entry}

\begin{entry}{蘑}{19}{⾋}
  \begin{phonetics}{蘑}{mo2}
    \definition{s.}{cogumelo}
  \end{phonetics}
\end{entry}

\begin{entry}{蘑菇}{19,11}{⾋、⾋}
  \begin{phonetics}{蘑菇}{mo2gu5}
    \definition{s.}{cogumelo}
    \definition{v.}{mandriar | embromar | amofinar | incomodar alguém com solicitações ou interrupções frequentes ou persistentes}
  \end{phonetics}
\end{entry}

\begin{entry}{警}{19}{⾔}
  \begin{phonetics}{警}{jing3}
    \definition{s.}{policial}
    \definition{v.}{alertar | avisar}
  \end{phonetics}
\end{entry}

\begin{entry}{警告}{19,7}{⾔、⼝}
  \begin{phonetics}{警告}{jing3gao4}[][HSK 5]
    \definition[个]{s.}{advertência (como medida disciplinar); uma forma de punição}
    \definition{v.}{avisar; advertir; admoestar}
  \end{phonetics}
\end{entry}

\begin{entry}{警官}{19,8}{⾔、⼧}
  \begin{phonetics}{警官}{jing3guan1}
    \definition{s.}{polícia | policial}
  \end{phonetics}
\end{entry}

\begin{entry}{警察}{19,14}{⾔、⼧}
  \begin{phonetics}{警察}{jing3cha2}[][HSK 3]
    \definition[个,位,名,群,队]{s.}{polícia; policial; oficial de polícia; as forças armadas que mantêm a segurança social do país são uma parte importante do aparato estatal; também se refere aos membros dessas forças armadas}
  \end{phonetics}
\end{entry}

\begin{entry}{蹲}{19}{⾜}
  \begin{phonetics}{蹲}{dun1}[][HSK 6]
    \definition{v.}{agachamento sobre os calcanhares; dobrar as pernas o máximo possível, como se estivesse sentado, mas não deixar as nádegas tocarem o chão | ficar; metáfora para ficar ocioso em casa}
  \end{phonetics}
\end{entry}

\begin{entry}{蹲下}{19,3}{⾜、⼀}
  \begin{phonetics}{蹲下}{dun1xia4}
    \definition{v.}{agachar | agachar-se}
  \end{phonetics}
\end{entry}

%%%%% EOF %%%%%

