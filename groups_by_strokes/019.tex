%%%
%%% 19画
%%%

\section*{19画}\addcontentsline{toc}{section}{19画}

\begin{Entry}{巅}{19}{⼭}
  \begin{Phonetics}{巅}{dian1}
    \definition[个]{s.}{pico da montanha; cume; topo da montanha}
  \end{Phonetics}
\end{Entry}

\begin{Entry}{巅峰}{19,10}{⼭、⼭}
  \begin{Phonetics}{巅峰}{dian1feng1}[][HSK 7-9]
    \definition{s.}{um cume; um pico de montanha}
  \end{Phonetics}
\end{Entry}

\begin{Entry}{攀}{19}{⼿}
  \begin{Phonetics}{攀}{pan1}
    \definition{v.}{escalar; escalar | buscar conexões em altos cargos | envolver; implicar | agarrar; agarrar-se; segurar-se a}
  \end{Phonetics}
\end{Entry}

\begin{Entry}{攀岩}{19,8}{⼿、⼭}
  \begin{Phonetics}{攀岩}{pan1yan2}
    \definition{s.}{alpinista}
    \definition{v.}{escalar uma montanha}
  \end{Phonetics}
\end{Entry}

\begin{Entry}{攀爬}{19,8}{⼿、⽖}
  \begin{Phonetics}{攀爬}{pan1pa2}
    \definition{v.}{escalar}
  \end{Phonetics}
\end{Entry}

\begin{Entry}{曝}{19}{⽇}
  \begin{Phonetics}{曝}{bao4}
    \definition{v.}{usado em  曝光}
  \seealsoref{曝光}{bao4/guang1}
  \end{Phonetics}
  \begin{Phonetics}{曝}{pu4}
    \definition{v.}{expor ao sol}
  \end{Phonetics}
\end{Entry}

\begin{Entry}{曝光}{19,6}{⽇、⼉}
  \begin{Phonetics}{曝光}{bao4/guang1}[][HSK 7-9]
    \definition{v.+compl.}{expor; sensibilizar filme fotográfico ou papel fotossensível para formar uma imagem latente | expor; tornar (algo ruim) público; metáfora para revelar coisas secretas (geralmente vergonhosas) ao mundo}
  \end{Phonetics}
\end{Entry}

\begin{Entry}{爆}{19}{⽕}
  \begin{Phonetics}{爆}{bao4}[][HSK 6]
    \definition{v.}{explodir; estourar | fritar rapidamente; ferver rapidamente | aparecer (ou ocorrer) inesperadamente}
  \end{Phonetics}
\end{Entry}

\begin{Entry}{爆发}{19,5}{⽕、⼜}
  \begin{Phonetics}{爆发}{bao4fa1}[][HSK 6]
    \definition{v.}{entrar em erupção; explodir | estourar; irromper; ocorrer de forma repentina e violenta}
  \end{Phonetics}
\end{Entry}

\begin{Entry}{爆竹}{19,6}{⽕、⽵}
  \begin{Phonetics}{爆竹}{bao4 zhu2}[][HSK 7-9]
    \definition[串,个]{s.}{fogo de artifício}
  \end{Phonetics}
\end{Entry}

\begin{Entry}{爆米花}{19,6,7}{⽕、⽶、⾋}
  \begin{Phonetics}{爆米花}{bao4mi3hua1}
    \definition{s.}{pipoca (de milho) | pipoca de arroz}
  \end{Phonetics}
\end{Entry}

\begin{Entry}{爆冷门}{19,7,3}{⽕、⼎、⾨}
  \begin{Phonetics}{爆冷门}{bao4 leng3men2}[][HSK 7-9]
    \definition{s.}{um avanço | uma reviravolta (especialmente nos esportes) | reviravolta inesperada dos acontecimentos}
    \definition{v.}{dar um golpe}
  \end{Phonetics}
\end{Entry}

\begin{Entry}{爆炸}{19,9}{⽕、⽕}
  \begin{Phonetics}{爆炸}{bao4zha4}[][HSK 6]
    \definition{s.}{explosão}
    \definition{v.}{explodir; explodir; detonar | aumentar bruscamente em um curto espaço de tempo (de quantidade)}
  \end{Phonetics}
\end{Entry}

\begin{Entry}{爆满}{19,13}{⽕、⽔}
  \begin{Phonetics}{爆满}{bao4man3}[][HSK 7-9]
    \definition{v.}{(teatro, cinema, estádio, etc.) lotar; ter casa cheia | estar lotado}
  \end{Phonetics}
\end{Entry}

\begin{Entry}{聼}{19}{⼼}
  \begin{Phonetics}{聼}{ting1}
    \variantof{听}
  \end{Phonetics}
\end{Entry}

\begin{Entry}{蘑}{19}{⾋}
  \begin{Phonetics}{蘑}{mo2}
    \definition{s.}{cogumelo}
  \end{Phonetics}
\end{Entry}

\begin{Entry}{蘑菇}{19,11}{⾋、⾋}
  \begin{Phonetics}{蘑菇}{mo2gu5}
    \definition{s.}{cogumelo}
    \definition{v.}{mandriar | embromar | amofinar | incomodar alguém com solicitações ou interrupções frequentes ou persistentes}
  \end{Phonetics}
\end{Entry}

\begin{Entry}{警}{19}{⾔}
  \begin{Phonetics}{警}{jing3}
    \definition{s.}{policial}
    \definition{v.}{alertar | avisar}
  \end{Phonetics}
\end{Entry}

\begin{Entry}{警告}{19,7}{⾔、⼝}
  \begin{Phonetics}{警告}{jing3gao4}[][HSK 5]
    \definition[个]{s.}{advertência (como medida disciplinar); uma forma de punição}
    \definition{v.}{avisar; advertir; admoestar}
  \end{Phonetics}
\end{Entry}

\begin{Entry}{警官}{19,8}{⾔、⼧}
  \begin{Phonetics}{警官}{jing3guan1}
    \definition[名]{s.}{polícia | policial}
  \end{Phonetics}
\end{Entry}

\begin{Entry}{警察}{19,14}{⾔、⼧}
  \begin{Phonetics}{警察}{jing3cha2}[][HSK 3]
    \definition[个,位,名,群,队]{s.}{polícia; policial; oficial de polícia; as forças armadas que mantêm a segurança social do país são uma parte importante do aparato estatal; também se refere aos membros dessas forças armadas}
  \end{Phonetics}
\end{Entry}

\begin{Entry}{蹬}{19}{⾜}
  \begin{Phonetics}{蹬}{deng1}[][HSK 7-9]
    \definition{v.}{pressionar com o pé; pisar; pisar em | Dialeto: calçar (sapatos ou calças); usar (sapatos) | Gíria: despejar (algo)}
  \end{Phonetics}
  \begin{Phonetics}{蹬}{deng4}
    \definition{s.}{lutar; ter dificuldade}
  \seealsoref{蹭蹬}{ceng4deng4}
  \end{Phonetics}
\end{Entry}

\begin{Entry}{蹭}{19}{⾜}
  \begin{Phonetics}{蹭}{ceng4}[][HSK 7-9]
    \definition{v.}{esfregar; raspar; arranhar | esfregar em algo e ficar manchado; ser manchado com; manchar por fricção | mover-se lentamente; demorar-se; arrastar-se | Dialeto: roubar}
  \end{Phonetics}
\end{Entry}

\begin{Entry}{蹭蹬}{19,19}{⾜、⾜}
  \begin{Phonetics}{蹭蹬}{ceng4deng4}
    \definition{interj.}{Droga!}
    \definition{v.}{enfrentar contratempos; estar sem sorte; ter má sorte}
  \end{Phonetics}
\end{Entry}

\begin{Entry}{蹲}{19}{⾜}
  \begin{Phonetics}{蹲}{dun1}[][HSK 6]
    \definition{v.}{agachamento sobre os calcanhares; dobrar as pernas o máximo possível, como se estivesse sentado, mas não deixar as nádegas tocarem o chão | ficar; metáfora para ficar ocioso em casa}
  \end{Phonetics}
\end{Entry}

\begin{Entry}{蹲下}{19,3}{⾜、⼀}
  \begin{Phonetics}{蹲下}{dun1xia4}
    \definition{v.}{agachar | agachar-se}
  \end{Phonetics}
\end{Entry}

\begin{Entry}{颤}{19}{⾴}
  \begin{Phonetics}{颤}{chan4}
    \definition{v.}{tremer; estremecer | vibrar; tremer; sacudir}
  \end{Phonetics}
\end{Entry}

\begin{Entry}{颤抖}{19,7}{⾴、⼿}
  \begin{Phonetics}{颤抖}{chan4dou3}[][HSK 7-9]
    \definition{v.}{tremer; estremecer; tremular; tiritar}
  \end{Phonetics}
\end{Entry}

%%%%% EOF %%%%%

