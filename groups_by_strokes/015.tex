%%%
%%% 15画
%%%

\section*{15画}\addcontentsline{toc}{section}{15画}

\begin{entry}{嘱}{15}{⼝}
  \begin{phonetics}{嘱}{zhu3}
    \definition{v.}{juntar-se | implorar | incitar}
  \end{phonetics}
\end{entry}

\begin{entry}{嘱托}{15,6}{⼝、⼿}
  \begin{phonetics}{嘱托}{zhu3tuo1}
    \definition{v.}{confiar uma tarefa a alguém}
  \end{phonetics}
\end{entry}

\begin{entry}{嘱咐}{15,8}{⼝、⼝}
  \begin{phonetics}{嘱咐}{zhu3fu5}
    \definition{v.}{ordenar | dizer | exortar}
  \end{phonetics}
\end{entry}

\begin{entry}{噎}{15}{⼝}
  \begin{phonetics}{噎}{ye1}
    \definition{v.}{engasgar | sufocar}
  \end{phonetics}
\end{entry}

\begin{entry}{增}{15}{⼟}
  \begin{phonetics}{增}{zeng1}[][HSK 5]
    \definition*{s.}{sobrenome Zeng}
    \definition{v.}{aumentar; ganhar; adicionar}
  \end{phonetics}
\end{entry}

\begin{entry}{增大}{15,3}{⼟、⼤}
  \begin{phonetics}{增大}{zeng1 da4}[][HSK 5]
    \definition{v.}{ampliar; expandir; estender | amplificar}
  \end{phonetics}
\end{entry}

\begin{entry}{增长}{15,4}{⼟、⾧}
  \begin{phonetics}{增长}{zeng1 zhang3}[][HSK 3]
    \definition{v.}{subir; crescer; aumentar; melhorar a partir da base existente}
  \end{phonetics}
\end{entry}

\begin{entry}{增加}{15,5}{⼟、⼒}
  \begin{phonetics}{增加}{zeng1jia1}[][HSK 3]
    \definition{v.}{adicionar; aumentar; incrementar; adicionar mais ao que já existe}
  \end{phonetics}
\end{entry}

\begin{entry}{增产}{15,6}{⼟、⼇}
  \begin{phonetics}{增产}{zeng1 chan3}[][HSK 5]
    \definition{v.+compl.}{aumentar a produção}
  \end{phonetics}
\end{entry}

\begin{entry}{增多}{15,6}{⼟、⼣}
  \begin{phonetics}{增多}{zeng1 duo1}[][HSK 5]
    \definition{v.}{aumentar; crescer em número ou quantidade}
  \end{phonetics}
\end{entry}

\begin{entry}{增速}{15,10}{⼟、⾡}
  \begin{phonetics}{增速}{zeng1su4}
    \definition{s.}{(economia) taxa de crescimento}
    \definition{v.}{acelerar;}
  \end{phonetics}
\end{entry}

\begin{entry}{增强}{15,12}{⼟、⼸}
  \begin{phonetics}{增强}{zeng1 qiang2}[][HSK 5]
    \definition{v.}{impulsionar; aprimorar; aumentar; fortalecer; tornar mais forte ou mais poderoso}
  \end{phonetics}
\end{entry}

\begin{entry}{墨镜}{15,16}{⿊、⾦}
  \begin{phonetics}{墨镜}{mo4jing4}
    \definition[只,双,副]{s.}{óculos escuros}
  \end{phonetics}
\end{entry}

\begin{entry}{影子}{15,3}{⼺、⼦}
  \begin{phonetics}{影子}{ying3zi5}[][HSK 4]
    \definition[个,片]{s.}{sombra; imagem projetada por um objeto, etc., que bloqueia a luz | reflexão; reflexo; imagem de um objeto, etc., conforme aparece em um refletor, como um espelho, uma superfície de água, etc. | sinal; vestígio; vaga impressão}
  \end{phonetics}
\end{entry}

\begin{entry}{影片}{15,4}{⼺、⽚}
  \begin{phonetics}{影片}{ying3 pian4}[][HSK 2]
    \definition[部,盘,盒,卷]{s.}{filme; imagem | filme; película usada para reproduzir filmes}
  \end{phonetics}
\end{entry}

\begin{entry}{影视}{15,8}{⼺、⾒}
  \begin{phonetics}{影视}{ying3 shi4}[][HSK 3]
    \definition{s.}{cinema e televisão combinados; denominação conjunta para cinema e TV}
  \end{phonetics}
\end{entry}

\begin{entry}{影响}{15,9}{⼺、⼝}
  \begin{phonetics}{影响}{ying3xiang3}[][HSK 2]
    \definition{s.}{efeito; influência; efeitos sobre pessoas ou coisas}
    \definition{v.}{afetar; influenciar; influência sobre os pensamentos ou ações dos outros}
  \end{phonetics}
\end{entry}

\begin{entry}{影像}{15,13}{⼺、⼈}
  \begin{phonetics}{影像}{ying3xiang4}
    \definition{s.}{imagem}
  \end{phonetics}
\end{entry}

\begin{entry}{德}{15}{⼻}
  \begin{phonetics}{德}{de2}
    \definition*{s.}{Alemanha, abreviação de 德国}
    \definition*{s.}{sobrenome De}
    \definition{s.}{virtude; moral; caráter moral; moralidade; conduta; qualidades políticas | coração; mente; pensamentos | bondade; favor; graça}
  \seealsoref{德国}{de2guo2}
  \end{phonetics}
\end{entry}

\begin{entry}{德国}{15,8}{⼻、⼞}
  \begin{phonetics}{德国}{de2guo2}
    \definition*{s.}{Alemanha}
  \end{phonetics}
\end{entry}

\begin{entry}{德国人}{15,8,2}{⼻、⼞、⼈}
  \begin{phonetics}{德国人}{de2guo2ren2}
    \definition{s.}{alemão | pessoa ou povo da Alemanha}
  \end{phonetics}
\end{entry}

\begin{entry}{慰问}{15,6}{⼼、⾨}
  \begin{phonetics}{慰问}{wei4wen4}[][HSK 5]
    \definition{v.}{visitar; consolar; expressar simpatia por; confortar e cumprimentar com palavras e presentes;  enfatizar o conforto e o cumprimento, frequentemente usado por superiores para subordinados}
  \end{phonetics}
\end{entry}

\begin{entry}{憧}{15}{⼼}
  \begin{phonetics}{憧}{chong1}
    \definition{adj.}{irresoluto; indeciso | estúpido; imbecil; confuso}
  \end{phonetics}
\end{entry}

\begin{entry}{憧憬}{15,15}{⼼、⼼}
  \begin{phonetics}{憧憬}{chong1jing3}
    \definition{v.}{ansiar por | esperar por}
  \end{phonetics}
\end{entry}

\begin{entry}{懂}{15}{⼼}
  \begin{phonetics}{懂}{dong3}[][HSK 2]
    \definition*{s.}{sobrenome Dong}
    \definition{v.}{compreender; entender}
  \end{phonetics}
\end{entry}

\begin{entry}{懂得}{15,11}{⼼、⼻}
  \begin{phonetics}{懂得}{dong3 de5}[][HSK 2]
    \definition{v.}{saber (significado, prática, etc.); compreender; entender}
  \end{phonetics}
\end{entry}

\begin{entry}{摩托}{15,6}{⼿、⼿}
  \begin{phonetics}{摩托}{mo2 tuo1}[][HSK 5]
    \definition[辆]{s.}{(empréstimo linguístico) motor; motor de combustão interna | (empréstimo linguístico) motocicleta, abreviação de 摩托车}
  \seealsoref{摩托车}{mo2tuo1che1}
  \end{phonetics}
\end{entry}

\begin{entry}{摩托车}{15,6,4}{⼿、⼿、⾞}
  \begin{phonetics}{摩托车}{mo2tuo1che1}
    \definition[辆,部]{s.}{(empréstimo linguístico) motocicleta}
  \end{phonetics}
\end{entry}

\begin{entry}{摩擦}{15,17}{⼿、⼿}
  \begin{phonetics}{摩擦}{mo2ca1}[][HSK 5]
    \definition{s.}{atrito; desacordo; conflito (entre duas partes); a ação de impedir o movimento relativo entre dois objetos em contato, produzida na superfície de contato | atrito; metáfora para o conflito entre as duas partes}
    \definition{v.}{esfregar}
  \end{phonetics}
\end{entry}

\begin{entry}{撒旦}{15,5}{⼿、⽇}
  \begin{phonetics}{撒旦}{sa1dan4}
    \definition*{s.}{Satã}
  \end{phonetics}
\end{entry}

\begin{entry}{撒旦主义}{15,5,5,3}{⼿、⽇、⼂、⼂}
  \begin{phonetics}{撒旦主义}{sa1dan4 zhu3yi4}
    \definition*{s.}{Satanismo}
  \end{phonetics}
\end{entry}

\begin{entry}{撒但}{15,7}{⼿、⼈}
  \begin{phonetics}{撒但}{sa1dan4}
    \variantof{撒旦}
  \end{phonetics}
\end{entry}

\begin{entry}{撞}{15}{⼿}
  \begin{phonetics}{撞}{zhuang4}[][HSK 5]
    \definition{v.}{chocar-se contra; chocar-se com; bater; colidir | encontrar-se por acaso; esbarrar em; deparar-se com | apressar; correr; empurrar | aproveitar a chance | esbarrar de repente em |  encontrar | confiar em; tentar | agir precipitadamente; invadir}
  \end{phonetics}
\end{entry}

\begin{entry}{撞车}{15,4}{⼿、⾞}
  \begin{phonetics}{撞车}{zhuang4che1}
    \definition{v.+compl.}{(figurativo) colidir (opiniões, cronogramas, etc.) | ser o mesmo (assunto) | colidir (com outro veículo)}
  \end{phonetics}
\end{entry}

\begin{entry}{撞运气}{15,7,4}{⼿、⾡、⽓}
  \begin{phonetics}{撞运气}{zhuang4yun4qi5}
    \definition{v.}{confiar no destino | tentar a sorte}
  \end{phonetics}
\end{entry}

\begin{entry}{撤}{15}{⼿}
  \begin{phonetics}{撤}{che4}
    \definition{v.}{remover, tirar | demitir; liberar | retirar-se; evacuar}
  \end{phonetics}
\end{entry}

\begin{entry}{播}{15}{⼿}
  \begin{phonetics}{播}{bo1}
    \definition{v.}{espalhar; transmitir | semear | mover-se; migrar; ir para o exílio}
  \end{phonetics}
\end{entry}

\begin{entry}{播出}{15,5}{⼿、⼐}
  \begin{phonetics}{播出}{bo1 chu1}[][HSK 3]
    \definition{v.}{radiodifundir; transmitir; estar no ar; transmitir via rádio e televisão}
  \end{phonetics}
\end{entry}

\begin{entry}{播放}{15,8}{⼿、⽅}
  \begin{phonetics}{播放}{bo1fang4}[][HSK 3]
    \definition{v.}{ir ao ar; transmitir por rádio | mostrar; exibir; transmitir (um programa de TV)}
  \end{phonetics}
\end{entry}

\begin{entry}{播音}{15,9}{⼿、⾳}
  \begin{phonetics}{播音}{bo1yin1}
    \definition{s.}{transmissão}
    \definition{v.+compl.}{transmitir}
  \end{phonetics}
\end{entry}

\begin{entry}{擒获}{15,10}{⼿、⾋}
  \begin{phonetics}{擒获}{qin2huo4}
    \definition{v.}{apreender | capturar}
  \end{phonetics}
\end{entry}

\begin{entry}{暴}{15}{⽇}
  \begin{phonetics}{暴}{bao4}
    \definition*{s.}{sobrenome Bao}
    \definition{adj.}{repentino e violento | cruel; selvagem; feroz | temperamental | severo e tirânico; brutal | irritável; irascível; impaciente}
    \definition{adv.}{de repente e ferozmente}
    \definition{s.}{violência; ferocidade}
    \definition{v.}{sobressair; destacar-se; inchar | expor; transmitir | desperdiçar; arruinar; estragar}
  \end{phonetics}
\end{entry}

\begin{entry}{暴力}{15,2}{⽇、⼒}
  \begin{phonetics}{暴力}{bao4li4}
    \definition{adj.}{violento}
    \definition{s.}{violência}
  \end{phonetics}
\end{entry}

\begin{entry}{暴行}{15,6}{⽇、⾏}
  \begin{phonetics}{暴行}{bao4xing2}
    \definition{s.}{ato selvagem | atrocidade | indignação}
  \end{phonetics}
\end{entry}

\begin{entry}{暴乱}{15,7}{⽇、⼄}
  \begin{phonetics}{暴乱}{bao4luan4}
    \definition{s.}{rebelião | revolta | tumulto}
  \end{phonetics}
\end{entry}

\begin{entry}{暴雨}{15,8}{⽇、⾬}
  \begin{phonetics}{暴雨}{bao4yu3}
    \definition[场,阵]{s.}{tempestade | chuva torrencial}
  \end{phonetics}
\end{entry}

\begin{entry}{暴躁}{15,20}{⽇、⾜}
  \begin{phonetics}{暴躁}{bao4zao4}
    \definition{adj.}{irascível | irritável}
  \end{phonetics}
\end{entry}

\begin{entry}{槽}{15}{⽊}
  \begin{phonetics}{槽}{cao2}
    \definition{clas.}{usado para portas | usado para porcos}
    \definition[个,道]{s.}{cocho | sulco; entalhe | canal | manjedoura (para água, ração animal, vinho, cuba); um recipiente para alimentar o gado, geralmente é retangular, alto em todos os lados e côncavo no meio, como uma caixa sem tampa | tanque de fermentação; cuba de vinho; geralmente se refere a certos utensílios com lados altos e côncavos no meio | leito do rio; fossa; refere-se a certos cursos d'água ou valas com lados altos e um meio côncavo | ranhura; fenda; uma depressão semelhante a um sulco em um objeto}
  \end{phonetics}
\end{entry}

\begin{entry}{横}{15}{⽊}
  \begin{phonetics}{横}{heng2}
    \definition{adj.}{horizontal; transversal; paralelo ao plano horizontal (oposto de 竖 e 直) | em ângulo reto com; direção esquerda-direita (em oposição à 竖, 直 ou 纵) | e leste a oeste ou de oeste a leste; direção leste-oeste (oposta a 纵) | desenfreado; turbulento | violento; feroz; irracional}
    \definition{adv.}{de qualquer forma; em qualquer caso | provavelmente; muito provavelmente}
    \definition{s.}{traço horizontal (em caracteres chineses)}
    \definition{v.}{deitar-se transversalmente; estar de lado | colocar algo transversalmente (ou horizontalmente)}
  \seealsoref{竖}{shu4}
  \seealsoref{直}{zhi2}
  \seealsoref{纵}{zong4}
  \end{phonetics}
\end{entry}

\begin{entry}{横竖}{15,9}{⽊、⽴}
  \begin{phonetics}{横竖}{heng2shu5}
    \definition{adv.}{de qualquer maneira | independentemente (linguagem falada)}
  \end{phonetics}
\end{entry}

\begin{entry}{樱桃}{15,10}{⽊、⽊}
  \begin{phonetics}{樱桃}{ying1tao2}
    \definition{s.}{cereja}
  \end{phonetics}
\end{entry}

\begin{entry}{橄}{15}{⽊}
  \begin{phonetics}{橄}{gan3}
    \definition*{s.}{sobrenome Gan}
  \end{phonetics}
\end{entry}

\begin{entry}{橄榄球}{15,13,11}{⽊、⽊、⽟}
  \begin{phonetics}{橄榄球}{gan3lan3qiu2}
    \definition{s.}{futebol jogado com bola oval (rúgbi, futebol americano, regras australianas, etc.)}
  \end{phonetics}
\end{entry}

\begin{entry}{潜在}{15,6}{⽔、⼟}
  \begin{phonetics}{潜在}{qian2zai4}
    \definition{adj.}{oculto | latente}
    \definition{s.}{potencial}
  \end{phonetics}
\end{entry}

\begin{entry}{潮}{15}{⽔}
  \begin{phonetics}{潮}{chao2}[][HSK 4]
    \definition{adj.}{úmido; molhado | inferior; de qualidade ruim | inferior; não muito habilidoso}
    \definition{s.}{maré; água da maré | surto; corrente; maré; uma metáfora para mudanças sociais em grande escala ou para os altos e baixos de um movimento (social)}
    \definition{s.}{Chaozhou, uma cidade na província de Guangdong}
  \end{phonetics}
\end{entry}

\begin{entry}{潮流}{15,10}{⽔、⽔}
  \begin{phonetics}{潮流}{chao2liu2}[][HSK 4]
    \definition{s.}{maré; corrente de maré; movimento da água devido às marés | tendência; analogia com mudanças sociais ou tendências de desenvolvimento}
  \end{phonetics}
\end{entry}

\begin{entry}{潮绣}{15,10}{⽔、⽷}
  \begin{phonetics}{潮绣}{chao2xiu4}
    \definition*{s.}{Bordado Chaozhou}
  \end{phonetics}
\end{entry}

\begin{entry}{潮湿}{15,12}{⽔、⽔}
  \begin{phonetics}{潮湿}{chao2shi1}[][HSK 4]
    \definition{adj.}{molhado; úmido; umedecido; que contém mais água do que o normal}
  \end{phonetics}
\end{entry}

\begin{entry}{澳}{15}{⽔}
  \begin{phonetics}{澳}{ao4}
    \definition*{s.}{Austrália, abreviação de 澳大利亚}
    \definition{s.}{sobrenome Ao}
    \definition{s.}{baía; uma entrada do mar; um lugar curvo na costa onde os barcos podem ser atracados, frequentemente usado em nomes de lugares}
  \seealsoref{澳大利亚}{ao4da4li4ya4}
  \end{phonetics}
\end{entry}

\begin{entry}{澳大利亚}{15,3,7,6}{⽔、⼤、⼑、⼆}
  \begin{phonetics}{澳大利亚}{ao4da4li4ya4}
    \definition*{s.}{Austrália}
  \end{phonetics}
\end{entry}

\begin{entry}{熟}{15}{⽕}
  \begin{phonetics}{熟}{shu2}[][HSK 2]
    \definition{adj.}{maduro (frutos) | pronto; cozido | processado, fabricado ou exercitado | familiar, bem conhecido; conhecido por ser comum ou frequentemente utilizado | habilidoso;  (trabalho, tecnologia) experiente; não é novato | profundo; sólido}
  \end{phonetics}
\end{entry}

\begin{entry}{熟人}{15,2}{⽕、⼈}
  \begin{phonetics}{熟人}{shu2 ren2}[][HSK 3]
    \definition[位,名,个,些]{s.}{amigo; conhecido; pessoas que se conhecem há muito tempo; pessoas que são muito familiares}
  \end{phonetics}
\end{entry}

\begin{entry}{熟练}{15,8}{⽕、⽷}
  \begin{phonetics}{熟练}{shu2lian4}[][HSK 4]
    \definition{adj.}{especializado; proficiente; qualificado; habilidoso}
  \end{phonetics}
\end{entry}

\begin{entry}{熟悉}{15,11}{⽕、⼼}
  \begin{phonetics}{熟悉}{shu2xi1}[][HSK 5]
    \definition{adj.}{familiarizado com; não ser estranho}
    \definition{v.}{estar familiarizado com; saber claramente que | conhecer bem algo ou alguém; compreender e dominar (a situação) através da observação ou da experiência}
  \end{phonetics}
\end{entry}

\begin{entry}{碾碎}{15,13}{⽯、⽯}
  \begin{phonetics}{碾碎}{nian3sui4}
    \definition{v.}{pulverizar | esmagar}
  \end{phonetics}
\end{entry}

\begin{entry}{稿}{15}{⽲}
  \begin{phonetics}{稿}{gao3}
    \definition[篇]{s.}{(significado original) talo de grão; palha | rascunho; esboço; manuscrito | texto original}
  \end{phonetics}
\end{entry}

\begin{entry}{稿纸}{15,7}{⽲、⽷}
  \begin{phonetics}{稿纸}{gao3zhi3}
    \definition{s.}{rascunho | manuscrito}
  \end{phonetics}
\end{entry}

\begin{entry}{箱}{15}{⾋}
  \begin{phonetics}{箱}{xiang1}[][HSK 4]
    \definition{s.}{caixa; estojo; baú | qualquer coisa no formato de caixa}
  \end{phonetics}
\end{entry}

\begin{entry}{箱子}{15,3}{⾋、⼦}
  \begin{phonetics}{箱子}{xiang1 zi5}[][HSK 4]
    \definition[个,只]{s.}{baú; caixa; estojo; maleta; pasta executiva}
  \end{phonetics}
\end{entry}

\begin{entry}{篇}{15}{⽵}
  \begin{phonetics}{篇}{pian1}[][HSK 2]
    \definition*{s.}{sobrenome Pian}
    \definition{clas.}{usado para folhas de papel, páginas de livros, artigos, etc.}
    \definition{s.}{um pedaço de escrita | folha (de papel, etc.) | (para papel, folhas de livros, artigos, etc.) folha; página; pedaço}
  \end{phonetics}
\end{entry}

\begin{entry}{糆}{15}{⽶}
  \begin{phonetics}{糆}{mian4}
    \variantof{面}
  \end{phonetics}
\end{entry}

\begin{entry}{糊里糊涂}{15,7,15,10}{⽶、⾥、⽶、⽔}
  \begin{phonetics}{糊里糊涂}{hu2li5hu2tu5}
    \definition{adj.}{desnorteado | perturbado}
  \end{phonetics}
\end{entry}

\begin{entry}{聪}{15}{⽿}
  \begin{phonetics}{聪}{cong1}
    \definition{adj.}{audição aguçada | brilhante; inteligente; esperto | perspicaz}
    \definition{s.}{(literário) faculdades auditivas}
  \end{phonetics}
\end{entry}

\begin{entry}{聪明}{15,8}{⽿、⽇}
  \begin{phonetics}{聪明}{cong1ming5}[][HSK 5]
    \definition{adj.}{brilhante; esperto; inteligente; intelecto bem desenvolvido com boa memória e capacidade de compreensão}
  \end{phonetics}
\end{entry}

\begin{entry}{聪慧}{15,15}{⽿、⼼}
  \begin{phonetics}{聪慧}{cong1hui4}
    \definition{adj.}{inteligente | brilhante}
  \end{phonetics}
\end{entry}

\begin{entry}{蔬菜}{15,11}{⾋、⾋}
  \begin{phonetics}{蔬菜}{shu1cai4}[][HSK 5]
    \definition[样,种]{s.}{verduras; legumes; vegetais; ervas que podem ser usadas na culinária}
  \end{phonetics}
\end{entry}

\begin{entry}{蕃茄}{15,8}{⾋、⾋}
  \begin{phonetics}{蕃茄}{fan1qie2}
    \variantof{番茄}
  \end{phonetics}
\end{entry}

\begin{entry}{蝌蚪}{15,10}{⾍、⾍}
  \begin{phonetics}{蝌蚪}{ke1dou3}
    \definition{s.}{girino}
  \end{phonetics}
\end{entry}

\begin{entry}{蝲蝲蛄}{15,15,11}{⾍、⾍、⾍}
  \begin{phonetics}{蝲蝲蛄}{la4la4gu3}
    \definition{s.}{grilo toupeira}
  \end{phonetics}
\end{entry}

\begin{entry}{蝴蝶}{15,15}{⾍、⾍}
  \begin{phonetics}{蝴蝶}{hu2die2}
    \definition[只]{s.}{borboleta}
  \end{phonetics}
\end{entry}

\begin{entry}{豌豆}{15,7}{⾖、⾖}
  \begin{phonetics}{豌豆}{wan1dou4}
    \definition{s.}{ervilha}
  \end{phonetics}
\end{entry}

\begin{entry}{豫}{15}{⾗}
  \begin{phonetics}{豫}{yu4}
    \definition{adj.}{feliz despreocupado | à vontade}
  \seealsoref{预}{yu4}
  \end{phonetics}
\end{entry}

\begin{entry}{踏板}{15,8}{⾜、⽊}
  \begin{phonetics}{踏板}{ta4ban3}
    \definition{s.}{pedal (em um carro, em um piano, etc.) |  apoio para os pés | estribo}
  \end{phonetics}
\end{entry}

\begin{entry}{踢}{15}{⾜}
  \begin{phonetics}{踢}{ti1}
    \definition{v.}{chutar | jogar (por exemplo, futebol) | dar pontapés em}
  \end{phonetics}
\end{entry}

\begin{entry}{踢蹋舞}{15,17,14}{⾜、⾜、⾇}
  \begin{phonetics}{踢蹋舞}{ti1ta4wu3}
    \definition{s.}{sapateado | passo de dança}
  \end{phonetics}
\end{entry}

\begin{entry}{踢爆}{15,19}{⾜、⽕}
  \begin{phonetics}{踢爆}{ti1bao4}
    \definition{v.}{expor | revelar}
  \end{phonetics}
\end{entry}

\begin{entry}{躺}{15}{⾝}
  \begin{phonetics}{躺}{tang3}[][HSK 4]
    \definition{v.}{deitar; reclinar}
  \end{phonetics}
\end{entry}

\begin{entry}{遵守}{15,6}{⾡、⼧}
  \begin{phonetics}{遵守}{zun1shou3}[][HSK 5]
    \definition{v.}{obedecer; observar; cumprir; respeitar; atuar de acordo com as regras; não infringir}
  \end{phonetics}
\end{entry}

\begin{entry}{醉}{15}{⾣}
  \begin{phonetics}{醉}{zui4}[][HSK 5]
    \definition{v.}{embriagar-se; ficar bêbado; intoxicar-se; beber em excesso e perder o controle | (de certos alimentos) ser embebido em licor; ser mergulhado em vinho; marinar (alimentos) em vinho | entregar-se a; ser viciado em; gostar demais, a ponto de chegar à obsessão}
  \end{phonetics}
\end{entry}

\begin{entry}{醋}{15}{⾣}
  \begin{phonetics}{醋}{cu4}
    \definition[瓶,坛,碟,碗]{s.}{(condimento) vinagre | ciúme (como em caso de amor); uma metáfora para o ciúme, referindo-se principalmente aos relacionamentos entre pessoas}
  \end{phonetics}
\end{entry}

\begin{entry}{震惊}{15,11}{⾬、⼼}
  \begin{phonetics}{震惊}{zhen4jing1}[][HSK 5]
    \definition{adj.}{cado; atordoado; espantado; atônito}
    \definition{v.}{chocar; surpreender; espantar}
  \end{phonetics}
\end{entry}

\begin{entry}{震撼}{15,16}{⾬、⼿}
  \begin{phonetics}{震撼}{zhen4han4}
    \definition{v.}{sacudir | chocar | atordoar}
  \end{phonetics}
\end{entry}

\begin{entry}{靠}{15}{⾮}
  \begin{phonetics}{靠}{kao4}[][HSK 2]
    \definition{prep.}{manter (em); aproximar-se (de); ao longo de | por; graças a; com base em; de acordo com}
    \definition{s.}{armadura de palco (feita de seda bordada); armadura usada pelos generais militares antigos nas peças teatrais}
    \definition{v.}{inclinar-se; sentado ou em pé, deixar parte do peso do corpo ser suportado por outra pessoa ou objeto (pessoa) | encostar-se (em); apoiar-se ou levantar-se com a ajuda de alguma coisa | aproximar-se; estar perto de | confiar em; depender de | confiar}
  \end{phonetics}
\end{entry}

\begin{entry}{靠近}{15,7}{⾮、⾡}
  \begin{phonetics}{靠近}{kao4 jin4}[][HSK 5]
    \definition{adv.}{próximo; perto de; ao lado de}
    \definition{v.}{aproximar-se; chegar perto; avançar em direção a um determinado objetivo de modo que a distância fique cada vez menor}
  \end{phonetics}
\end{entry}

\begin{entry}{鞋}{15}{⾰}
  \begin{phonetics}{鞋}{xie2}[][HSK 2]
    \definition[双,只]{s.}{sapatos; usado nos pés; algo que toca o chão ao caminhar; sem cano alto}
  \end{phonetics}
\end{entry}

\begin{entry}{题}{15}{⾴}
  \begin{phonetics}{题}{ti2}[][HSK 2]
    \definition*{s.}{sobrenome Ti}
    \definition[个,道]{s.}{tópico; título; assunto; problema; frases que indicam o conteúdo de poemas ou discursos | questão; questões que devem ser respondidas durante os exercícios ou exames | antigamente, referia-se à testa}
    \definition{v.}{inscrever; escrever; assinar}
  \end{phonetics}
\end{entry}

\begin{entry}{题目}{15,5}{⾴、⽬}
  \begin{phonetics}{题目}{ti2mu4}[][HSK 3]
    \definition[个,道]{s.}{título; assunto; tópico; o título de um poema ou discurso | quebra-cabeça; problema de exercício; questões a serem respondidas em exercícios ou provas}
  \end{phonetics}
\end{entry}

\begin{entry}{题材}{15,7}{⾴、⽊}
  \begin{phonetics}{题材}{ti2cai2}[][HSK 5]
    \definition{s.}{tema; assunto; material que compõe as obras literárias e artísticas, ou seja, os eventos ou fenômenos da vida descritos concretamente nas obras}
  \end{phonetics}
\end{entry}

\begin{entry}{颜}{15}{⾴}
  \begin{phonetics}{颜}{yan2}
    \definition*{s.}{sobrenome Yan}
    \definition{s.}{cor | face | semblante}
  \end{phonetics}
\end{entry}

\begin{entry}{颜色}{15,6}{⾴、⾊}
  \begin{phonetics}{颜色}{yan2 se4}[][HSK 2]
    \definition[个,种]{s.}{cor; a sensação visual de um objeto é uma impressão diferente produzida pelas diferentes quantidades de luz absorvidas e refletidas pelo objeto | tez; semblante; aparência; geralmente se refere à aparência de uma garota | olhar severo no rosto como um aviso; um olhar ou ação que faz os outros parecerem particularmente ferozes | a expressão mostrada no rosto}
  \end{phonetics}
\end{entry}

\begin{entry}{飘}{15}{⾵}
  \begin{phonetics}{飘}{piao1}
    \definition{adj.}{complacente | frívolo | fraco | instável | bambo | cambaleante}
    \definition{v.}{flutuar (no ar) | esvoaçar | tremular}
  \end{phonetics}
\end{entry}

\begin{entry}{鲨鱼}{15,8}{⿂、⿂}
  \begin{phonetics}{鲨鱼}{sha1yu2}
    \definition{s.}{tubarão}
  \end{phonetics}
\end{entry}

\begin{entry}{鹤}{15}{⿃}
  \begin{phonetics}{鹤}{he4}
    \definition{s.}{grou (ave)}
  \end{phonetics}
\end{entry}

\begin{entry}{麫}{15}{⿆}
  \begin{phonetics}{麫}{mian4}
    \variantof{面}
  \end{phonetics}
\end{entry}

\begin{entry}{黎}{15}{⿉}
  \begin{phonetics}{黎}{li2}
    \definition*{s.}{sobrenome Li}
    \definition*{s.}{a nacionalidade Li, uma das minorias nacionais da província de Hainan}
    \definition{adj.}{numeroso}
    \definition{s.}{multidão}
  \end{phonetics}
\end{entry}

%%%%% EOF %%%%%

