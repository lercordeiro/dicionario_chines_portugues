%%%
%%% 15画
%%%

\section*{15画}\addcontentsline{toc}{section}{15画}

\begin{Entry}{嘱}{15}{⼝}
  \begin{Phonetics}{嘱}{zhu3}
    \definition{v.}{juntar-se | implorar | incitar}
  \end{Phonetics}
\end{Entry}

\begin{Entry}{嘱托}{15,6}{⼝、⼿}
  \begin{Phonetics}{嘱托}{zhu3tuo1}
    \definition{v.}{confiar uma tarefa a alguém}
  \end{Phonetics}
\end{Entry}

\begin{Entry}{嘱咐}{15,8}{⼝、⼝}
  \begin{Phonetics}{嘱咐}{zhu3fu5}
    \definition{v.}{ordenar | dizer | exortar}
  \end{Phonetics}
\end{Entry}

\begin{Entry}{嘲}{15}{⼝}
  \begin{Phonetics}{嘲}{chao2}
    \definition{v.}{ridicularizar; zombar; fazer piada de}
  \end{Phonetics}
  \begin{Phonetics}{嘲}{zhao1}
    \definition{s.}{Onomatopéia: barulho clamoroso feito por várias pessoas falando ou cantando, ou por instrumentos musicais, ou pássaros cantando; descreve um som caótico e fragmentado}
  \end{Phonetics}
\end{Entry}

\begin{Entry}{嘲弄}{15,7}{⼝、⼶}
  \begin{Phonetics}{嘲弄}{chao2nong4}[][HSK 7-9]
    \definition{v.}{zombar; zombar de}
  \end{Phonetics}
\end{Entry}

\begin{Entry}{嘲笑}{15,10}{⼝、⽵}
  \begin{Phonetics}{嘲笑}{chao2xiao4}[][HSK 7-9]
    \definition{v.}{ridicularizar; zombar; rir de; zombar de; fazer graça de; usar palavras para zombar de alguém}
  \end{Phonetics}
\end{Entry}

\begin{Entry}{嘹}{15}{⼝}
  \begin{Phonetics}{嘹}{liao2}
    \definition{adj.}{(som) alto e claro | som claro | grito (de guindastes, etc.)}
  \end{Phonetics}
\end{Entry}

\begin{Entry}{嘹亮}{15,9}{⼝、⼇}
  \begin{Phonetics}{嘹亮}{liao2liang4}
    \definition{adj.}{ressonante; alto e claro}
  \end{Phonetics}
\end{Entry}

\begin{Entry}{嘿}{15}{⼝}
  \begin{Phonetics}{嘿}{hei1}[][HSK 7-9]
    \definition{interj.}{Ei!; indicando uma saudação ou chamar a atenção | expressando orgulho ou satisfação | expressando espanto, surpresa}
  \end{Phonetics}
  \begin{Phonetics}{嘿}{mo4}
    \definition{adj.}{quieto; silencioso; tácito}
  \end{Phonetics}
\end{Entry}

\begin{Entry}{噎}{15}{⼝}
  \begin{Phonetics}{噎}{ye1}
    \definition{v.}{engasgar | sufocar}
  \end{Phonetics}
\end{Entry}

\begin{Entry}{增}{15}{⼟}
  \begin{Phonetics}{增}{zeng1}[][HSK 5]
    \definition*{s.}{Sobrenome Zeng}
    \definition{v.}{aumentar; ganhar; adicionar}
  \end{Phonetics}
\end{Entry}

\begin{Entry}{增大}{15,3}{⼟、⼤}
  \begin{Phonetics}{增大}{zeng1 da4}[][HSK 5]
    \definition{v.}{ampliar; expandir; estender | amplificar}
  \end{Phonetics}
\end{Entry}

\begin{Entry}{增长}{15,4}{⼟、⾧}
  \begin{Phonetics}{增长}{zeng1 zhang3}[][HSK 3]
    \definition{v.}{subir; crescer; aumentar; melhorar a partir da base existente}
  \end{Phonetics}
\end{Entry}

\begin{Entry}{增加}{15,5}{⼟、⼒}
  \begin{Phonetics}{增加}{zeng1jia1}[][HSK 3]
    \definition{v.}{adicionar; aumentar; incrementar; adicionar mais ao que já existe}
  \end{Phonetics}
\end{Entry}

\begin{Entry}{增产}{15,6}{⼟、⼇}
  \begin{Phonetics}{增产}{zeng1/chan3}[][HSK 5]
    \definition{v.+compl.}{aumentar a produção}
  \end{Phonetics}
\end{Entry}

\begin{Entry}{增多}{15,6}{⼟、⼣}
  \begin{Phonetics}{增多}{zeng1 duo1}[][HSK 5]
    \definition{v.}{aumentar; crescer em número ou quantidade}
  \end{Phonetics}
\end{Entry}

\begin{Entry}{增进}{15,7}{⼟、⾡}
  \begin{Phonetics}{增进}{zeng1 jin4}[][HSK 6]
    \definition{v.}{melhorar; promover; aprofundar}
  \end{Phonetics}
\end{Entry}

\begin{Entry}{增值}{15,10}{⼟、⼈}
  \begin{Phonetics}{增值}{zeng1 zhi2}[][HSK 6]
    \definition{s.}{aumento de valor; apreciação; incremento | valor agregado}
  \end{Phonetics}
\end{Entry}

\begin{Entry}{增速}{15,10}{⼟、⾡}
  \begin{Phonetics}{增速}{zeng1su4}
    \definition{s.}{(economia) taxa de crescimento}
    \definition{v.}{acelerar;}
  \end{Phonetics}
\end{Entry}

\begin{Entry}{增强}{15,12}{⼟、⼸}
  \begin{Phonetics}{增强}{zeng1 qiang2}[][HSK 5]
    \definition{v.}{impulsionar; aprimorar; aumentar; fortalecer; tornar mais forte ou mais poderoso}
  \end{Phonetics}
\end{Entry}

\begin{Entry}{墨}{15}{⿊}
  \begin{Phonetics}{墨}{mo4}
    \definition*{s.}{Escola Moísta; Moísmo | México, abreviação de 墨西哥}
    \definition{adj.}{preto; escuro como breu | corrupto | escuro}
    \definition{s.}{tinta chinesa; bastão de tinta | pigmento; tinta | caligrafia ou pintura | aprendizagem; alfabetização | marcador de linha de carpinteiro; marcador de tinta | tatuar o rosto (um castigo); uma punição na China antiga | corrupção; peculato; fraude}
  \seealsoref{墨西哥}{mo4xi1ge1}
  \end{Phonetics}
\end{Entry}

\begin{Entry}{墨水}{15,4}{⿊、⽔}
  \begin{Phonetics}{墨水}{mo4 shui3}[][HSK 6]
    \definition[瓶]{s.}{tinta chinesa preparada; tinta (para caneta-tinteiro) | aprendizagem; alfabetização; uma metáfora para o conhecimento ou a capacidade de ler e escrever}
  \end{Phonetics}
\end{Entry}

\begin{Entry}{墨西哥}{15,6,10}{⿊、⾑、⼝}
  \begin{Phonetics}{墨西哥}{mo4xi1ge1}
    \definition*{s.}{México; Planalto no México}
  \end{Phonetics}
\end{Entry}

\begin{Entry}{墨镜}{15,16}{⿊、⾦}
  \begin{Phonetics}{墨镜}{mo4jing4}
    \definition[只,双,副]{s.}{óculos escuros}
  \end{Phonetics}
\end{Entry}

\begin{Entry}{影}{15}{⼺}
  \begin{Phonetics}{影}{ying3}
    \definition*{s.}{Sobrenome Ying}
    \definition{s.}{sombra | reflexão; imagem | traço; sinal; impressão vaga | fotografia; imagem | filme | jogo de sombras; pantomima de sombra}
    \definition{v.}{(dialeto) esconder; ocultar | copiar; rastrear | fotocopiar}
  \end{Phonetics}
\end{Entry}

\begin{Entry}{影子}{15,3}{⼺、⼦}
  \begin{Phonetics}{影子}{ying3zi5}[][HSK 4]
    \definition[个,片]{s.}{sombra; imagem projetada por um objeto, etc., que bloqueia a luz | reflexão; reflexo; imagem de um objeto, etc., conforme aparece em um refletor, como um espelho, uma superfície de água, etc. | sinal; vestígio; vaga impressão}
  \end{Phonetics}
\end{Entry}

\begin{Entry}{影片}{15,4}{⼺、⽚}
  \begin{Phonetics}{影片}{ying3 pian4}[][HSK 2]
    \definition[部,盘,盒,卷]{s.}{filme; imagem | filme; película usada para reproduzir filmes}
  \end{Phonetics}
\end{Entry}

\begin{Entry}{影视}{15,8}{⼺、⾒}
  \begin{Phonetics}{影视}{ying3 shi4}[][HSK 3]
    \definition{s.}{cinema e televisão combinados; denominação conjunta para cinema e TV}
  \end{Phonetics}
\end{Entry}

\begin{Entry}{影响}{15,9}{⼺、⼝}
  \begin{Phonetics}{影响}{ying3xiang3}[][HSK 2]
    \definition{s.}{efeito; influência; efeitos sobre pessoas ou coisas}
    \definition{v.}{afetar; influenciar; influência sobre os pensamentos ou ações dos outros}
  \end{Phonetics}
\end{Entry}

\begin{Entry}{影响力}{15,9,2}{⼺、⼝、⼒}
  \begin{Phonetics}{影响力}{ying3 xiang3 li4}[][HSK 6]
    \definition{s.}{impacto | influência}
  \end{Phonetics}
\end{Entry}

\begin{Entry}{影星}{15,9}{⼺、⽇}
  \begin{Phonetics}{影星}{ying3 xing1}[][HSK 6]
    \definition{s.}{estrela de cinema}
  \end{Phonetics}
\end{Entry}

\begin{Entry}{影迷}{15,9}{⼺、⾡}
  \begin{Phonetics}{影迷}{ying3 mi2}[][HSK 6]
    \definition[个,名,位]{s.}{fã de cinema; entusiasta de cinema; pessoas viciadas em assistir filmes}
  \end{Phonetics}
\end{Entry}

\begin{Entry}{影像}{15,13}{⼺、⼈}
  \begin{Phonetics}{影像}{ying3xiang4}
    \definition{s.}{imagem}
  \end{Phonetics}
\end{Entry}

\begin{Entry}{德}{15}{⼻}
  \begin{Phonetics}{德}{de2}[][HSK 7-9]
    \definition*{s.}{Alemanha, abreviação de 德国 | Sobrenome De}
    \definition{s.}{virtude; moral; caráter moral; moralidade; conduta; qualidades políticas | coração; mente; pensamentos | bondade; favor; graça}
  \seealsoref{德国}{de2guo2}
  \end{Phonetics}
\end{Entry}

\begin{Entry}{德国}{15,8}{⼻、⼞}
  \begin{Phonetics}{德国}{de2guo2}
    \definition*{s.}{Alemanha}
  \end{Phonetics}
\end{Entry}

\begin{Entry}{德国人}{15,8,2}{⼻、⼞、⼈}
  \begin{Phonetics}{德国人}{de2guo2ren2}
    \definition{s.}{alemão | pessoa ou povo da Alemanha}
  \end{Phonetics}
\end{Entry}

\begin{Entry}{慰}{15}{⼼}
  \begin{Phonetics}{慰}{wei4}
    \definition{adj.}{aliviado; em paz; confortável}
    \definition{v.}{consolar; confortar | ser (ficar) aliviado}
  \end{Phonetics}
\end{Entry}

\begin{Entry}{慰问}{15,6}{⼼、⾨}
  \begin{Phonetics}{慰问}{wei4wen4}[][HSK 5]
    \definition{v.}{visitar; consolar; expressar simpatia por; confortar e cumprimentar com palavras e presentes;  enfatizar o conforto e o cumprimento, frequentemente usado por superiores para subordinados}
  \end{Phonetics}
\end{Entry}

\begin{Entry}{憋}{15}{⼼}
  \begin{Phonetics}{憋}{bie1}[][HSK 7-9]
    \definition{adj.}{sufocado; oprimido}
    \definition{v.}{suprimir; conter | Dialeto: obrigar | Dialeto: ponderar; contemplar | Dialeto: ficar de olho em | Dialeto: destruir (por pressão interna) | calar a boca; inibir; bloquear | sufocar; abafar}
  \end{Phonetics}
\end{Entry}

\begin{Entry}{憧}{15}{⼼}
  \begin{Phonetics}{憧}{chong1}
    \definition{adj.}{irresoluto; indeciso | estúpido; imbecil; confuso}
  \end{Phonetics}
\end{Entry}

\begin{Entry}{憧憬}{15,15}{⼼、⼼}
  \begin{Phonetics}{憧憬}{chong1jing3}
    \definition{v.}{ansiar por | esperar por}
  \end{Phonetics}
\end{Entry}

\begin{Entry}{懂}{15}{⼼}
  \begin{Phonetics}{懂}{dong3}[][HSK 2]
    \definition*{s.}{Sobrenome Dong}
    \definition{v.}{compreender; entender}
  \end{Phonetics}
\end{Entry}

\begin{Entry}{懂事}{15,8}{⼼、⼅}
  \begin{Phonetics}{懂事}{dong3shi4}[][HSK 7-9]
    \definition{adj.}{sensato; inteligente; muito compreensivo da natureza e da razão humana}
  \end{Phonetics}
\end{Entry}

\begin{Entry}{懂得}{15,11}{⼼、⼻}
  \begin{Phonetics}{懂得}{dong3 de5}[][HSK 2]
    \definition{v.}{saber (significado, prática, etc.); compreender; entender}
  \end{Phonetics}
\end{Entry}

\begin{Entry}{摩}{15}{⼿}
  \begin{Phonetics}{摩}{mo2}
    \definition{v.}{esfregar; raspar; tocar | refletir; estudar | afagar}
  \end{Phonetics}
\end{Entry}

\begin{Entry}{摩托}{15,6}{⼿、⼿}
  \begin{Phonetics}{摩托}{mo2 tuo1}[][HSK 5]
    \definition[辆]{s.}{Empréstimo linguístico: motor; motor de combustão interna | Empréstimo linguístico: motocicleta, abreviação de 摩托车}
  \seealsoref{摩托车}{mo2tuo1che1}
  \end{Phonetics}
\end{Entry}

\begin{Entry}{摩托车}{15,6,4}{⼿、⼿、⾞}
  \begin{Phonetics}{摩托车}{mo2tuo1che1}
    \definition[辆,部]{s.}{(empréstimo linguístico) motocicleta}
  \end{Phonetics}
\end{Entry}

\begin{Entry}{摩擦}{15,17}{⼿、⼿}
  \begin{Phonetics}{摩擦}{mo2ca1}[][HSK 5]
    \definition{s.}{atrito; desacordo; conflito (entre duas partes); a ação de impedir o movimento relativo entre dois objetos em contato, produzida na superfície de contato | atrito; metáfora para o conflito entre as duas partes}
    \definition{v.}{esfregar}
  \end{Phonetics}
\end{Entry}

\begin{Entry}{撑}{15}{⼿}
  \begin{Phonetics}{撑}{cheng1}[][HSK 6]
    \definition{s.}{suporte; escora;  apoio; esteio}
    \definition{v.}{sustentar; apoiar; resistir a | empurrar (ou mover) com uma vara; usar um mastro para empurrar a margem ou o leito do rio para fazer o barco avançar | manter; manter-se atualizado | abrir; desdobrar; expandir (um objeto contraído) | encher até estourar (inchaço devido a excesso de comida ou alimentação excessiva)}
  \end{Phonetics}
\end{Entry}

\begin{Entry}{撒}{15}{⼿}
  \begin{Phonetics}{撒}{sa1}
    \definition{v.}{lançar; deixar ir; deixar sair; liberar | livrar-se de todas as restrições; deixar-se levar; tentar usá-lo ou exibi-lo o máximo possível}
  \end{Phonetics}
\end{Entry}

\begin{Entry}{撒旦}{15,5}{⼿、⽇}
  \begin{Phonetics}{撒旦}{sa1dan4}
    \definition*{s.}{Satã}
  \end{Phonetics}
\end{Entry}

\begin{Entry}{撒旦主义}{15,5,5,3}{⼿、⽇、⼂、⼂}
  \begin{Phonetics}{撒旦主义}{sa1dan4 zhu3yi4}
    \definition*{s.}{Satanismo}
  \end{Phonetics}
\end{Entry}

\begin{Entry}{撒但}{15,7}{⼿、⼈}
  \begin{Phonetics}{撒但}{sa1dan4}
    \variantof{撒旦}
  \end{Phonetics}
\end{Entry}

\begin{Entry}{撞}{15}{⼿}
  \begin{Phonetics}{撞}{zhuang4}[][HSK 5]
    \definition{v.}{chocar-se contra; chocar-se com; bater; colidir | encontrar-se por acaso; esbarrar em; deparar-se com | apressar; correr; empurrar | aproveitar a chance | esbarrar de repente em |  encontrar | confiar em; tentar | agir precipitadamente; invadir}
  \end{Phonetics}
\end{Entry}

\begin{Entry}{撞车}{15,4}{⼿、⾞}
  \begin{Phonetics}{撞车}{zhuang4/che1}
    \definition{v.+compl.}{(figurativo) colidir (opiniões, cronogramas, etc.) | ser o mesmo (assunto) | colidir (com outro veículo)}
  \end{Phonetics}
\end{Entry}

\begin{Entry}{撞运气}{15,7,4}{⼿、⾡、⽓}
  \begin{Phonetics}{撞运气}{zhuang4yun4qi5}
    \definition{v.}{confiar no destino | tentar a sorte}
  \end{Phonetics}
\end{Entry}

\begin{Entry}{撤}{15}{⼿}
  \begin{Phonetics}{撤}{che4}[][HSK 7-9]
    \definition{v.}{remover, tirar | demitir; liberar | retirar-se; evacuar}
  \end{Phonetics}
\end{Entry}

\begin{Entry}{撤换}{15,10}{⼿、⼿}
  \begin{Phonetics}{撤换}{che4huan4}[][HSK 7-9]
    \definition{v.}{demitir e substituir (alguém); revogar; substituir (alguém ou alguma coisa)}
  \end{Phonetics}
\end{Entry}

\begin{Entry}{撤离}{15,10}{⼿、⼇}
  \begin{Phonetics}{撤离}{che4 li2}[][HSK 6]
    \definition{v.}{retirar-se de; deixar; evacuar}
  \end{Phonetics}
\end{Entry}

\begin{Entry}{撤销}{15,12}{⼿、⾦}
  \begin{Phonetics}{撤销}{che4xiao1}[][HSK 6]
    \definition{v.}{cancelar; rescindir; revogar; remover}
  \end{Phonetics}
\end{Entry}

\begin{Entry}{播}{15}{⼿}
  \begin{Phonetics}{播}{bo1}[][HSK 6]
    \definition{v.}{espalhar; transmitir | semear | mover-se; migrar; ir para o exílio}
  \end{Phonetics}
\end{Entry}

\begin{Entry}{播出}{15,5}{⼿、⼐}
  \begin{Phonetics}{播出}{bo1 chu1}[][HSK 3]
    \definition{v.}{radiodifundir; transmitir; estar no ar; transmitir via rádio e televisão}
  \end{Phonetics}
\end{Entry}

\begin{Entry}{播放}{15,8}{⼿、⽅}
  \begin{Phonetics}{播放}{bo1fang4}[][HSK 3]
    \definition{v.}{ir ao ar; transmitir por rádio | mostrar; exibir; transmitir (um programa de TV)}
  \end{Phonetics}
\end{Entry}

\begin{Entry}{播音}{15,9}{⼿、⾳}
  \begin{Phonetics}{播音}{bo1/yin1}
    \definition{s.}{transmissão}
    \definition{v.+compl.}{transmitir}
  \end{Phonetics}
\end{Entry}

\begin{Entry}{擒}{15}{⼿}
  \begin{Phonetics}{擒}{qin2}
    \definition{v.}{capturar; pegar; apreender}
  \end{Phonetics}
\end{Entry}

\begin{Entry}{擒获}{15,10}{⼿、⾋}
  \begin{Phonetics}{擒获}{qin2huo4}
    \definition{v.}{apreender | capturar}
  \end{Phonetics}
\end{Entry}

\begin{Entry}{敷}{15}{⽁}
  \begin{Phonetics}{敷}{fu1}[][HSK 7-9]
    \definition*{s.}{Sobrenome Fu}
    \definition{v.}{aplicar (pó, pomada, etc.) | espalhar; dispor | ser suficiente para | espalhar-se}
  \end{Phonetics}
\end{Entry}

\begin{Entry}{暴}{15}{⽇}
  \begin{Phonetics}{暴}{bao4}
    \definition*{s.}{Sobrenome Bao}
    \definition{adj.}{repentino e violento | cruel; selvagem; feroz | temperamental | severo e tirânico; brutal | irritável; irascível; impaciente}
    \definition{adv.}{de repente e ferozmente}
    \definition{s.}{violência; ferocidade}
    \definition{v.}{sobressair; destacar-se; inchar | expor; transmitir | desperdiçar; arruinar; estragar}
  \end{Phonetics}
\end{Entry}

\begin{Entry}{暴力}{15,2}{⽇、⼒}
  \begin{Phonetics}{暴力}{bao4li4}[][HSK 6]
    \definition{s.}{violência; força (usada em tempos de conflito); poder de coerção}
  \end{Phonetics}
\end{Entry}

\begin{Entry}{暴风雨}{15,4,8}{⽇、⾵、⾬}
  \begin{Phonetics}{暴风雨}{bao4 feng1 yu3}[][HSK 6]
    \definition{s.}{tempestade; tormenta; temporal; borrasca; vento e chuva fortes e violentos}
  \end{Phonetics}
\end{Entry}

\begin{Entry}{暴风骤雨}{15,4,17,8}{⽇、⾵、⾺、⾬}
  \begin{Phonetics}{暴风骤雨}{bao4feng1-zhou4yu3}[][HSK 7-9]
    \definition{expr.}{tempestade violenta; furacão; tempestade | vento violento e tempestade de chuva}
  \end{Phonetics}
\end{Entry}

\begin{Entry}{暴行}{15,6}{⽇、⾏}
  \begin{Phonetics}{暴行}{bao4xing2}
    \definition{s.}{ato selvagem | atrocidade | indignação}
  \end{Phonetics}
\end{Entry}

\begin{Entry}{暴乱}{15,7}{⽇、⼄}
  \begin{Phonetics}{暴乱}{bao4luan4}
    \definition{s.}{rebelião | revolta | tumulto}
  \end{Phonetics}
\end{Entry}

\begin{Entry}{暴利}{15,7}{⽇、⼑}
  \begin{Phonetics}{暴利}{bao4li4}[][HSK 7-9]
    \definition{s.}{lucros enormes repentinos | lucros exorbitantes; lucros extravagantes; lucros excessivos}
  \end{Phonetics}
\end{Entry}

\begin{Entry}{暴雨}{15,8}{⽇、⾬}
  \begin{Phonetics}{暴雨}{bao4yu3}[][HSK 6]
    \definition[场,次,阵]{s.}{tempestade; chuva torrencial; chuva forte com precipitação intensa; em meteorologia, refere-se a chuvas de 16 mm ou mais em uma hora ou 50 mm ou mais em 24 horas}
  \end{Phonetics}
\end{Entry}

\begin{Entry}{暴躁}{15,20}{⽇、⾜}
  \begin{Phonetics}{暴躁}{bao4zao4}[][HSK 7-9]
    \definition{adj.}{irascível; febril; irritável; temperamental; descreve uma pessoa que é impaciente, não consegue controlar suas emoções e fica com raiva facilmente}
  \end{Phonetics}
\end{Entry}

\begin{Entry}{暴露}{15,21}{⽇、⾬}
  \begin{Phonetics}{暴露}{bao4lu4}[][HSK 6]
    \definition{adj.}{reveladoras (roupas inadequadas que expõem muito o corpo)}
    \definition{v.}{expor; desnudar; revelar; tornar público algo oculto}
  \end{Phonetics}
\end{Entry}

\begin{Entry}{槽}{15}{⽊}
  \begin{Phonetics}{槽}{cao2}[][HSK 7-9]
    \definition{clas.}{usado para portas | usado para porcos}
    \definition[个,道]{s.}{cocho | sulco; entalhe | canal | manjedoura (para água, ração animal, vinho, cuba); um recipiente para alimentar o gado, geralmente é retangular, alto em todos os lados e côncavo no meio, como uma caixa sem tampa | tanque de fermentação; cuba de vinho; geralmente se refere a certos utensílios com lados altos e côncavos no meio | leito do rio; fossa; refere-se a certos cursos d'água ou valas com lados altos e um meio côncavo | ranhura; fenda; uma depressão semelhante a um sulco em um objeto}
  \end{Phonetics}
\end{Entry}

\begin{Entry}{横}{15}{⽊}
  \begin{Phonetics}{横}{heng2}[][HSK 6]
    \definition{adj.}{horizontal; transversal; paralelo ao plano horizontal (oposto de 竖 e 直) | em ângulo reto com; direção esquerda-direita (em oposição à 竖, 直 ou 纵) | e leste a oeste ou de oeste a leste; direção leste-oeste (oposta a 纵) | desenfreado; turbulento | violento; feroz; irracional}
    \definition{adv.}{de qualquer forma; em qualquer caso | provavelmente; muito provavelmente}
    \definition{s.}{traço horizontal (em caracteres chineses)}
    \definition{v.}{deitar-se transversalmente; estar de lado | colocar algo transversalmente (ou horizontalmente)}
  \seealsoref{竖}{shu4}
  \seealsoref{直}{zhi2}
  \seealsoref{纵}{zong4}
  \end{Phonetics}
  \begin{Phonetics}{横}{heng4}[][HSK 7-9]
    \definition{adj.}{chocante e irracional; inesperado}
  \end{Phonetics}
\end{Entry}

\begin{Entry}{横七竖八}{15,2,9,2}{⽊、⼀、⽴、⼋}
  \begin{Phonetics}{横七竖八}{heng2qi1-shu4ba1}[][HSK 7-9]
    \definition{expr.}{em desordem; em seis e sete; desorganizado}
  \end{Phonetics}
\end{Entry}

\begin{Entry}{横向}{15,6}{⽊、⼝}
  \begin{Phonetics}{横向}{heng2xiang4}[][HSK 7-9]
    \definition{adj.}{horizontal; transversal (oposto a 竖向,纵向) | lateral | ortogonal | perpendicular}
  \seealsoref{竖向}{shu4xiang4}
  \seealsoref{纵向}{zong4xiang4}
  \end{Phonetics}
\end{Entry}

\begin{Entry}{横竖}{15,9}{⽊、⽴}
  \begin{Phonetics}{横竖}{heng2shu5}
    \definition{adv.}{de qualquer forma; em qualquer maneira; isso significa que não importa o que aconteça, o resultado ou a conclusão não mudará; equivale a 反正}
  \seealsoref{反正}{fan3zheng4}
  \end{Phonetics}
\end{Entry}

\begin{Entry}{樱}{15}{⽊}
  \begin{Phonetics}{樱}{ying1}
    \definition[个,棵,朵]{s.}{cereja | cerejeira oriental; flores de cerejeira}
  \end{Phonetics}
\end{Entry}

\begin{Entry}{樱桃}{15,10}{⽊、⽊}
  \begin{Phonetics}{樱桃}{ying1tao2}
    \definition{s.}{cereja}
  \end{Phonetics}
\end{Entry}

\begin{Entry}{橄}{15}{⽊}
  \begin{Phonetics}{橄}{gan3}
    \definition*{s.}{Sobrenome Gan}
  \end{Phonetics}
\end{Entry}

\begin{Entry}{橄榄球}{15,13,11}{⽊、⽊、⽟}
  \begin{Phonetics}{橄榄球}{gan3lan3qiu2}
    \definition{s.}{futebol jogado com bola oval (rúgbi, futebol americano, regras australianas, etc.)}
  \end{Phonetics}
\end{Entry}

\begin{Entry}{潜}{15}{⽔}
  \begin{Phonetics}{潜}{qian2}
    \definition*{s.}{Sobrenome Qian}
    \definition{adj.}{latente; oculto}
    \definition{adv.}{furtivamente; secretamente; às escondidas}
    \definition{v.}{ir para debaixo d'água; esconder-se debaixo d'água; mergulhar | esconder | vadear (atravessar) na água | enterrar | fugir de casa}
  \end{Phonetics}
\end{Entry}

\begin{Entry}{潜力}{15,2}{⽔、⼒}
  \begin{Phonetics}{潜力}{qian2li4}[][HSK 6]
    \definition{s.}{potencial; potencialidade; capacidade latente; as habilidades e possibilidades de desenvolvimento que as pessoas e as coisas ainda não demonstraram}
  \end{Phonetics}
\end{Entry}

\begin{Entry}{潜在}{15,6}{⽔、⼟}
  \begin{Phonetics}{潜在}{qian2zai4}
    \definition{adj.}{oculto | latente}
    \definition{s.}{potencial}
  \end{Phonetics}
\end{Entry}

\begin{Entry}{潮}{15}{⽔}
  \begin{Phonetics}{潮}{chao2}[][HSK 4]
    \definition{adj.}{úmido; molhado | inferior; de qualidade ruim | inferior; não muito habilidoso}
    \definition{s.}{maré; água da maré | surto; corrente; maré; uma metáfora para mudanças sociais em grande escala ou para os altos e baixos de um movimento (social)}
    \definition{s.}{Chaozhou, uma cidade na província de Guangdong}
  \end{Phonetics}
\end{Entry}

\begin{Entry}{潮流}{15,10}{⽔、⽔}
  \begin{Phonetics}{潮流}{chao2liu2}[][HSK 4]
    \definition[种,股,个]{s.}{maré; corrente de maré; movimento da água devido às marés | tendência; analogia com mudanças sociais ou tendências de desenvolvimento}
  \end{Phonetics}
\end{Entry}

\begin{Entry}{潮绣}{15,10}{⽔、⽷}
  \begin{Phonetics}{潮绣}{chao2xiu4}
    \definition*{s.}{Bordado Chaozhou}
  \end{Phonetics}
\end{Entry}

\begin{Entry}{潮湿}{15,12}{⽔、⽔}
  \begin{Phonetics}{潮湿}{chao2shi1}[][HSK 4]
    \definition{adj.}{molhado; úmido; umedecido; que contém mais água do que o normal}
  \end{Phonetics}
\end{Entry}

\begin{Entry}{澄}{15}{⽔}
  \begin{Phonetics}{澄}{cheng2}
    \definition*{s.}{Sobrenome Cheng}
    \definition{adj.}{claro; transparente}
    \definition{v.}{esclarecer; purificar}
  \end{Phonetics}
  \begin{Phonetics}{澄}{deng4}
    \definition{adj.}{(água, ar, etc.) claro; transparente; límpido}
    \definition{v.}{esclarecer; aclarar | sedimentar; fazer com que impurezas em um líquido afundem}
  \end{Phonetics}
\end{Entry}

\begin{Entry}{澄清}{15,11}{⽔、⽔}
  \begin{Phonetics}{澄清}{cheng2qing1}[][HSK 7-9]
    \definition{adj.}{claro; transparente}
    \definition{v.}{esclarecer; deixar claro; entender | purificar; limpar; esclarecer a turbidez, uma metáfora para esclarecer uma situação caótica}
  \end{Phonetics}
\end{Entry}

\begin{Entry}{澳}{15}{⽔}
  \begin{Phonetics}{澳}{ao4}
    \definition*{s.}{Abreviação de Austrália, 澳大利亚 | Sobrenome Ao}
    \definition{s.}{baía; uma entrada do mar; um lugar curvo na costa onde os barcos podem ser atracados, frequentemente usado em nomes de lugares}
  \seealsoref{澳大利亚}{ao4da4li4ya4}
  \end{Phonetics}
\end{Entry}

\begin{Entry}{澳大利亚}{15,3,7,6}{⽔、⼤、⼑、⼆}
  \begin{Phonetics}{澳大利亚}{ao4da4li4ya4}
    \definition*{s.}{Austrália}
  \end{Phonetics}
\end{Entry}

\begin{Entry}{熟}{15}{⽕}
  \begin{Phonetics}{熟}{shu2}[][HSK 2]
    \definition{adj.}{maduro (frutos) | pronto; cozido | processado, fabricado ou exercitado | familiar, bem conhecido; conhecido por ser comum ou frequentemente utilizado | habilidoso;  (trabalho, tecnologia) experiente; não é novato | profundo; sólido}
  \end{Phonetics}
\end{Entry}

\begin{Entry}{熟人}{15,2}{⽕、⼈}
  \begin{Phonetics}{熟人}{shu2 ren2}[][HSK 3]
    \definition[位,名,个,些]{s.}{amigo; conhecido; pessoas que se conhecem há muito tempo; pessoas que são muito familiares}
  \end{Phonetics}
\end{Entry}

\begin{Entry}{熟练}{15,8}{⽕、⽷}
  \begin{Phonetics}{熟练}{shu2lian4}[][HSK 4]
    \definition{adj.}{especializado; proficiente; qualificado; habilidoso}
  \end{Phonetics}
\end{Entry}

\begin{Entry}{熟悉}{15,11}{⽕、⼼}
  \begin{Phonetics}{熟悉}{shu2xi1}[][HSK 5]
    \definition{adj.}{familiarizado com; não ser estranho}
    \definition{v.}{estar familiarizado com; saber claramente que | conhecer bem algo ou alguém; compreender e dominar (a situação) através da observação ou da experiência}
  \end{Phonetics}
\end{Entry}

\begin{Entry}{獞}{15}{⽝}
  \begin{Phonetics}{獞}{tong2}
    \definition{s.}{nome de uma variedade de cão | tribos selvagens no sul da China}
  \end{Phonetics}
  \begin{Phonetics}{獞}{zhuang4}
    \variantof{壮}
  \end{Phonetics}
\end{Entry}

\begin{Entry}{碾}{15}{⽯}
  \begin{Phonetics}{碾}{nian3}
    \definition[台,个]{s.}{rolo e mó; rolo de pedra | rolo compressor}
    \definition{v.}{moer ou descascar com um rolo; esmagar | (literário) cortar e polir (jade, vidro, etc.) | achatar | pisar; pisotear, 轧}
  \seealsoref{辗}{zhan3}
  \end{Phonetics}
\end{Entry}

\begin{Entry}{碾碎}{15,13}{⽯、⽯}
  \begin{Phonetics}{碾碎}{nian3sui4}
    \definition{v.}{pulverizar | esmagar}
  \end{Phonetics}
\end{Entry}

\begin{Entry}{磅}{15}{⽯}
  \begin{Phonetics}{磅}{bang4}[][HSK 7-9]
    \definition{clas.}{libra | Tipografia: pt, ponto (tamanho de letra, por exemplo: 10pt)}
    \definition{s.}{escalas}
    \definition{v.}{pesar com uma balança}
  \end{Phonetics}
  \begin{Phonetics}{磅}{pang2}
    \definition{adj.}{majestoso; abundante; cheio de energia; magnífico}
  \end{Phonetics}
\end{Entry}

\begin{Entry}{稻}{15}{⽲}
  \begin{Phonetics}{稻}{dao4}
    \definition{s.}{arroz; arroz com casca}
  \end{Phonetics}
\end{Entry}

\begin{Entry}{稻草}{15,9}{⽲、⾋}
  \begin{Phonetics}{稻草}{dao4cao3}[][HSK 7-9]
    \definition[捆,根,抱,束]{s.}{palha de arroz (pode ser usada para fazer cordas ou esteiras de palha, para fazer papel, ou para ser usada como ração, combustível, etc.)}
  \end{Phonetics}
\end{Entry}

\begin{Entry}{稿}{15}{⽲}
  \begin{Phonetics}{稿}{gao3}
    \definition[篇]{s.}{(significado original) talo de grão; palha | rascunho; esboço; manuscrito | texto original}
  \end{Phonetics}
\end{Entry}

\begin{Entry}{稿子}{15,3}{⽲、⼦}
  \begin{Phonetics}{稿子}{gao3 zi5}[][HSK 6]
    \definition[篇,份,堆,叠]{s.}{rascunho; esboço; rascunhos de poemas, ensaios, desenhos, etc. | rascunho; manuscrito; poemas escritos | ideia; plano; plano preliminar ou conceito de trabalho}
  \end{Phonetics}
\end{Entry}

\begin{Entry}{稿纸}{15,7}{⽲、⽷}
  \begin{Phonetics}{稿纸}{gao3zhi3}
    \definition{s.}{rascunho | manuscrito}
  \end{Phonetics}
\end{Entry}

\begin{Entry}{箭}{15}{⽵}
  \begin{Phonetics}{箭}{jian4}[][HSK 6]
    \definition[支]{s.}{seta | distância percorrida por uma flecha}
  \end{Phonetics}
\end{Entry}

\begin{Entry}{箱}{15}{⾋}
  \begin{Phonetics}{箱}{xiang1}[][HSK 4]
    \definition{s.}{caixa; estojo; baú | qualquer coisa no formato de caixa}
  \end{Phonetics}
\end{Entry}

\begin{Entry}{箱子}{15,3}{⾋、⼦}
  \begin{Phonetics}{箱子}{xiang1 zi5}[][HSK 4]
    \definition[个,只]{s.}{baú; caixa; estojo; maleta; pasta executiva}
  \end{Phonetics}
\end{Entry}

\begin{Entry}{篇}{15}{⽵}
  \begin{Phonetics}{篇}{pian1}[][HSK 2]
    \definition*{s.}{Sobrenome Pian}
    \definition{clas.}{usado para folhas de papel, páginas de livros, artigos, etc.}
    \definition{s.}{um pedaço de escrita | folha (de papel, etc.) | (para papel, folhas de livros, artigos, etc.) folha; página; pedaço}
  \end{Phonetics}
\end{Entry}

\begin{Entry}{糆}{15}{⽶}
  \begin{Phonetics}{糆}{mian4}
    \variantof{面}
  \end{Phonetics}
\end{Entry}

\begin{Entry}{糊}{15}{⽶}
  \begin{Phonetics}{糊}{hu1}
    \definition{v.}{colar; untar; usar uma pasta mais espessa para revestir costuras, furos ou superfícies planas}
  \end{Phonetics}
  \begin{Phonetics}{糊}{hu2}[][HSK 7-9]
    \definition{adj.}{queimado}
    \definition{s.}{mingau; pasta; papa}
    \definition{v.}{colar com pasta; colar | (comida) ser queimado}
  \end{Phonetics}
  \begin{Phonetics}{糊}{hu4}
    \definition{s.}{pasta; comida que parece mingau}
  \end{Phonetics}
\end{Entry}

\begin{Entry}{糊里糊涂}{15,7,15,10}{⽶、⾥、⽶、⽔}
  \begin{Phonetics}{糊里糊涂}{hu2 li5 hu2tu5}
    \definition{adj.}{desnorteado | perturbado}
  \end{Phonetics}
\end{Entry}

\begin{Entry}{糊涂}{15,10}{⽶、⽔}
  \begin{Phonetics}{糊涂}{hu2tu5}[][HSK 7-9]
    \definition{adj.}{confuso; perplexo; desnorteado; com compreensão pouco clara ou confusa das coisas | confuso; com conteúdo confuso}
  \end{Phonetics}
\end{Entry}

\begin{Entry}{聪}{15}{⽿}
  \begin{Phonetics}{聪}{cong1}
    \definition{adj.}{audição aguçada | brilhante; inteligente; esperto | perspicaz}
    \definition{s.}{(literário) faculdades auditivas}
  \end{Phonetics}
\end{Entry}

\begin{Entry}{聪明}{15,8}{⽿、⽇}
  \begin{Phonetics}{聪明}{cong1ming5}[][HSK 5]
    \definition{adj.}{brilhante; esperto; inteligente; intelecto bem desenvolvido com boa memória e capacidade de compreensão}
  \end{Phonetics}
\end{Entry}

\begin{Entry}{聪慧}{15,15}{⽿、⼼}
  \begin{Phonetics}{聪慧}{cong1hui4}
    \definition{adj.}{inteligente | brilhante}
  \end{Phonetics}
\end{Entry}

\begin{Entry}{蔬}{15}{⾋}
  \begin{Phonetics}{蔬}{shu1}
    \definition{s.}{vegetais}
  \end{Phonetics}
\end{Entry}

\begin{Entry}{蔬菜}{15,11}{⾋、⾋}
  \begin{Phonetics}{蔬菜}{shu1cai4}[][HSK 5]
    \definition[样,种]{s.}{verduras; legumes; vegetais; ervas que podem ser usadas na culinária}
  \end{Phonetics}
\end{Entry}

\begin{Entry}{蕃}{15}{⾋}
  \begin{Phonetics}{蕃}{bo1}
    \definition[种]{s.}{estrangeiros}
  \end{Phonetics}
  \begin{Phonetics}{蕃}{fan1}
    \definition[种]{s.}{estrangeiros; aborígenes}
  \end{Phonetics}
  \begin{Phonetics}{蕃}{fan2}
    \definition{adj.}{exuberante; próspero}
    \definition{v.}{multiplicar; proliferar}
  \end{Phonetics}
\end{Entry}

\begin{Entry}{蕃茄}{15,8}{⾋、⾋}
  \begin{Phonetics}{蕃茄}{fan1 qie2}
    \variantof{番茄}
  \end{Phonetics}
\end{Entry}

\begin{Entry}{蝌}{15}{⾍}
  \begin{Phonetics}{蝌}{ke1}
    \definition[只]{s.}{girino}
  \end{Phonetics}
\end{Entry}

\begin{Entry}{蝌蚪}{15,10}{⾍、⾍}
  \begin{Phonetics}{蝌蚪}{ke1dou3}
    \definition{s.}{girino}
  \end{Phonetics}
\end{Entry}

\begin{Entry}{蝲}{15}{⾍}
  \begin{Phonetics}{蝲}{la4}
    \definition{s.}{lagostim de água doce}
  \seealsoref{蝲蛄}{la4gu3}
  \end{Phonetics}
\end{Entry}

\begin{Entry}{蝲蛄}{15,11}{⾍、⾍}
  \begin{Phonetics}{蝲蛄}{la4gu3}
    \definition{s.}{lagostim; lagostim de água doce}
  \end{Phonetics}
\end{Entry}

\begin{Entry}{蝲蝲蛄}{15,15,11}{⾍、⾍、⾍}
  \begin{Phonetics}{蝲蝲蛄}{la4la4gu3}
    \definition{s.}{grilo toupeira}
  \end{Phonetics}
\end{Entry}

\begin{Entry}{蝴}{15}{⾍}
  \begin{Phonetics}{蝴}{hu2}
    \definition[对]{s.}{borboleta}
  \end{Phonetics}
\end{Entry}

\begin{Entry}{蝴蝶}{15,15}{⾍、⾍}
  \begin{Phonetics}{蝴蝶}{hu2die2}
    \definition[只]{s.}{borboleta}
  \end{Phonetics}
\end{Entry}

\begin{Entry}{豌}{15}{⾖}
  \begin{Phonetics}{豌}{wan1}
    \definition[粒]{s.}{ervilhas}
  \end{Phonetics}
\end{Entry}

\begin{Entry}{豌豆}{15,7}{⾖、⾖}
  \begin{Phonetics}{豌豆}{wan1dou4}
    \definition{s.}{ervilha}
  \end{Phonetics}
\end{Entry}

\begin{Entry}{豫}{15}{⾗}
  \begin{Phonetics}{豫}{yu4}
    \definition*{s.}{Província de Henan, abreviatura de 河南}
    \definition{adj.}{satisfeito; encantado | anterior; preliminar; preparatório}
    \definition{adv.}{com antecedência; antecipadamente}
    \definition{v.}{viver com facilidade e conforto | participar de}
  \seealsoref{河南}{he2nan2}
  \seealsoref{预}{yu4}
  \end{Phonetics}
\end{Entry}

\begin{Entry}{趟}{15}{⾛}
  \begin{Phonetics}{趟}{tang1}
    \definition{v.}{atravessar; andar na grama ou onde não haja caminho | usar arados, capinadores, etc. para virar o solo e remover ervas daninhas | vadear; atravessar a vau; caminhar por águas rasas}[我们趟水去那小岛。===Nós vadeamos até a ilha.]
  \end{Phonetics}
  \begin{Phonetics}{趟}{tang4}[][HSK 6]
    \definition{clas.}{usado para o número de vezes de viagens de ida e volta |  usado para coisas dispostas em fileiras ou tiras | usado para a programação de veículos, navios, etc. que circulam em uma determinada ordem | usado em conjuntos de movimentos de artes marciais}
    \definition{s.}{marcha; procissão; jornada; viagem}
  \end{Phonetics}
\end{Entry}

\begin{Entry}{踏}{15}{⾜}
  \begin{Phonetics}{踏}{ta1}
    \definition{part.}{Caracter formador de palavras}
  \end{Phonetics}
  \begin{Phonetics}{踏}{ta4}[][HSK 6]
    \definition{v.}{por os pés em; pisar em; esmagar com o pé | fazer uma investigação ou levantamento no local}
  \end{Phonetics}
\end{Entry}

\begin{Entry}{踏实}{15,8}{⾜、⼧}
  \begin{Phonetics}{踏实}{ta1shi5}[][HSK 6]
    \definition{adj.}{confiável; sério; estável e seguro; descreve uma atitude séria em relação ao trabalho ou estudo | à vontade; livre de ansiedade; descreve uma mente ou sentimento estável, sem qualquer preocupação ou ansiedade}
  \end{Phonetics}
\end{Entry}

\begin{Entry}{踏板}{15,8}{⾜、⽊}
  \begin{Phonetics}{踏板}{ta4ban3}
    \definition{s.}{pedal (em um carro, em um piano, etc.) |  apoio para os pés | estribo}
  \end{Phonetics}
\end{Entry}

\begin{Entry}{踢}{15}{⾜}
  \begin{Phonetics}{踢}{ti1}[][HSK 6]
    \definition{v.}{chutar | jogar (por exemplo, futebol)}
  \end{Phonetics}
\end{Entry}

\begin{Entry}{踢蹋舞}{15,17,14}{⾜、⾜、⾇}
  \begin{Phonetics}{踢蹋舞}{ti1ta4wu3}
    \definition{s.}{sapateado | passo de dança}
  \end{Phonetics}
\end{Entry}

\begin{Entry}{踢爆}{15,19}{⾜、⽕}
  \begin{Phonetics}{踢爆}{ti1bao4}
    \definition{v.}{expor | revelar}
  \end{Phonetics}
\end{Entry}

\begin{Entry}{踩}{15}{⾜}
  \begin{Phonetics}{踩}{cai3}[][HSK 6]
    \definition{v.}{pisar; pisotear | pisar; metáfora: depreciar ou estragar | rastrear; antigamente significava rastrear (bandidos) ou investigar (casos)}
  \end{Phonetics}
\end{Entry}

\begin{Entry}{躺}{15}{⾝}
  \begin{Phonetics}{躺}{tang3}[][HSK 4]
    \definition{v.}{deitar; reclinar; cair no chão ou sobre um objeto}
  \end{Phonetics}
\end{Entry}

\begin{Entry}{遵}{15}{⾡}
  \begin{Phonetics}{遵}{zun1}
    \definition{v.}{cumprir; obedecer; observar; seguir}
  \end{Phonetics}
\end{Entry}

\begin{Entry}{遵守}{15,6}{⾡、⼧}
  \begin{Phonetics}{遵守}{zun1shou3}[][HSK 5]
    \definition{v.}{obedecer; observar; cumprir; respeitar; atuar de acordo com as regras; não infringir}
  \end{Phonetics}
\end{Entry}

\begin{Entry}{醇}{15}{⾣}
  \begin{Phonetics}{醇}{chun2}
    \definition{adj.}{Literário: puro; puro e suave; não misturado}
    \definition{s.}{Literário: vinho suave; bom vinho ; Química: álcool}
  \end{Phonetics}
\end{Entry}

\begin{Entry}{醇厚}{15,9}{⾣、⼚}
  \begin{Phonetics}{醇厚}{chun2hou4}[][HSK 7-9]
    \definition{adj.}{suave; rico; cheiro e sabor puros e ricos | puro e honesto; simples e gentil}
  \end{Phonetics}
\end{Entry}

\begin{Entry}{醉}{15}{⾣}
  \begin{Phonetics}{醉}{zui4}[][HSK 5]
    \definition{v.}{embriagar-se; ficar bêbado; intoxicar-se; beber em excesso e perder o controle | (de certos alimentos) ser embebido em licor; ser mergulhado em vinho; marinar (alimentos) em vinho | entregar-se a; ser viciado em; gostar demais, a ponto de chegar à obsessão}
  \end{Phonetics}
\end{Entry}

\begin{Entry}{醋}{15}{⾣}
  \begin{Phonetics}{醋}{cu4}[][HSK 6]
    \definition[瓶,坛,碟,碗]{s.}{(condimento) vinagre | ciúme (como em caso de amor); uma metáfora para o ciúme, referindo-se principalmente aos relacionamentos entre pessoas}
  \end{Phonetics}
\end{Entry}

\begin{Entry}{镇}{15}{⾦}
  \begin{Phonetics}{镇}{zhen4}[][HSK 6]
    \definition{adj.}{inteiro; indica um período inteiro de tempo}
    \definition{adv.}{frequentemente; muitas vezes}
    \definition{s.}{posto de guarnição | cidade; divisão administrativa | centro comercial}
    \definition{v.}{suprimir; segurar; manter pressionado |  acalmar-se; recompor-se; estabilizar | guardar; guarnecer; fortalecer; usar a força para manter a estabilidade | resfriar com gelo; esfriar em água fria | acalmar; suprimir; dissuadir | suprimir pela força; sancionar}
  \end{Phonetics}
\end{Entry}

\begin{Entry}{震}{15}{⾬}
  \begin{Phonetics}{震}{zhen4}
    \definition*{s.}{Zhen, um dos Oito Trigramas que representa o trovão | Sobrenome Zhen}
    \definition{adj.}{(coloquial) muito animado; profundamente surpreso; chocado}
    \definition{s.}{vibração; trepidação; tremor; abalo | terremoto; refere-se especificamente a terremotos | trovão; relâmpago}
    \definition{v.}{sacudir; chocar; vibrar; estremecer | ficar muito animado; ficar profundamente surpreso; ficar chocado | superar; vencer}
  \end{Phonetics}
\end{Entry}

\begin{Entry}{震惊}{15,11}{⾬、⼼}
  \begin{Phonetics}{震惊}{zhen4jing1}[][HSK 5]
    \definition{adj.}{chocado; atordoado; espantado; atônito}
    \definition{v.}{chocar; surpreender; espantar}
  \end{Phonetics}
\end{Entry}

\begin{Entry}{震撼}{15,16}{⾬、⼿}
  \begin{Phonetics}{震撼}{zhen4han4}
    \definition{v.}{sacudir | chocar | atordoar}
  \end{Phonetics}
\end{Entry}

\begin{Entry}{靠}{15}{⾮}
  \begin{Phonetics}{靠}{kao4}[][HSK 2]
    \definition{prep.}{manter (em); aproximar-se (de); ao longo de | por; graças a; com base em; de acordo com}
    \definition{s.}{armadura de palco (feita de seda bordada); armadura usada pelos generais militares antigos nas peças teatrais}
    \definition{v.}{inclinar-se; sentado ou em pé, deixar parte do peso do corpo ser suportado por outra pessoa ou objeto (pessoa) | encostar-se (em); apoiar-se ou levantar-se com a ajuda de alguma coisa | aproximar-se; estar perto de | confiar em; depender de | confiar}
  \end{Phonetics}
\end{Entry}

\begin{Entry}{靠近}{15,7}{⾮、⾡}
  \begin{Phonetics}{靠近}{kao4 jin4}[][HSK 5]
    \definition{adv.}{próximo; perto de; ao lado de}
    \definition{v.}{aproximar-se; chegar perto; avançar em direção a um determinado objetivo de modo que a distância fique cada vez menor}
  \end{Phonetics}
\end{Entry}

\begin{Entry}{鞋}{15}{⾰}
  \begin{Phonetics}{鞋}{xie2}[][HSK 2]
    \definition[双,只]{s.}{sapatos; usado nos pés; algo que toca o chão ao caminhar; sem cano alto}
  \end{Phonetics}
\end{Entry}

\begin{Entry}{题}{15}{⾴}
  \begin{Phonetics}{题}{ti2}[][HSK 2]
    \definition*{s.}{Sobrenome Ti}
    \definition[个,道]{s.}{tópico; título; assunto; problema; frases que indicam o conteúdo de poemas ou discursos | questão; questões que devem ser respondidas durante os exercícios ou exames | antigamente, referia-se à testa}
    \definition{v.}{inscrever; escrever; assinar}
  \end{Phonetics}
\end{Entry}

\begin{Entry}{题目}{15,5}{⾴、⽬}
  \begin{Phonetics}{题目}{ti2mu4}[][HSK 3]
    \definition[个,道]{s.}{título; assunto; tópico; o título de um poema ou discurso | quebra-cabeça; problema de exercício; questões a serem respondidas em exercícios ou provas}
  \end{Phonetics}
\end{Entry}

\begin{Entry}{题材}{15,7}{⾴、⽊}
  \begin{Phonetics}{题材}{ti2cai2}[][HSK 5]
    \definition{s.}{tema; assunto; material que compõe as obras literárias e artísticas, ou seja, os eventos ou fenômenos da vida descritos concretamente nas obras}
  \end{Phonetics}
\end{Entry}

\begin{Entry}{颜}{15}{⾴}
  \begin{Phonetics}{颜}{yan2}
    \definition*{s.}{Sobrenome Yan}
    \definition{s.}{rosto; semblante; expressão facial | rosto; prestígio; dignidade | cor}
  \end{Phonetics}
\end{Entry}

\begin{Entry}{颜色}{15,6}{⾴、⾊}
  \begin{Phonetics}{颜色}{yan2 se4}[][HSK 2]
    \definition[个,种]{s.}{cor; a sensação visual de um objeto é uma impressão diferente produzida pelas diferentes quantidades de luz absorvidas e refletidas pelo objeto | tez; semblante; aparência; geralmente se refere à aparência de uma garota | olhar severo no rosto como um aviso; um olhar ou ação que faz os outros parecerem particularmente ferozes | a expressão mostrada no rosto}
  \end{Phonetics}
\end{Entry}

\begin{Entry}{额}{15}{⾴}
  \begin{Phonetics}{额}{e2}
    \definition*{s.}{Sobrenome E}
    \definition[块]{s.}{testa; a área abaixo do cabelo e acima das sobrancelhas em humanos; a área aproximadamente equivalente na cabeça de alguns animais | uma tábua horizontal; placa horizontal inscrita; uma placa pendurada no lintel de uma porta ou na parede | um número específico (ou quantidade); limite superior de número; número limitado | a parte superior de algo}
  \end{Phonetics}
\end{Entry}

\begin{Entry}{额外}{15,5}{⾴、⼣}
  \begin{Phonetics}{额外}{e2wai4}[][HSK 7-9]
    \definition{adj.}{extra; adicional; excede a quantidade ou intervalo prescrito}
  \end{Phonetics}
\end{Entry}

\begin{Entry}{飘}{15}{⾵}
  \begin{Phonetics}{飘}{piao1}
    \definition{adj.}{complacente | frívolo | fraco | instável | bambo | cambaleante}
    \definition{v.}{flutuar (no ar) | esvoaçar | tremular}
  \end{Phonetics}
\end{Entry}

\begin{Entry}{鲨}{15}{⿂}
  \begin{Phonetics}{鲨}{sha1}
    \definition[只,条]{s.}{tubarão}
  \end{Phonetics}
\end{Entry}

\begin{Entry}{鲨鱼}{15,8}{⿂、⿂}
  \begin{Phonetics}{鲨鱼}{sha1yu2}
    \definition{s.}{tubarão}
  \end{Phonetics}
\end{Entry}

\begin{Entry}{鹤}{15}{⿃}
  \begin{Phonetics}{鹤}{he4}
    \definition*{s.}{Sobrenome He}
    \definition[只]{s.}{grou (ave)}
  \end{Phonetics}
\end{Entry}

\begin{Entry}{鹤立鸡群}{15,5,7,13}{⿃、⽴、⿃、⽺}
  \begin{Phonetics}{鹤立鸡群}{he4li4ji1qun2}[][HSK 7-9]
    \definition{expr.}{destaque-se da multidão; manifestamente superior; muito acima do comum; como um guindaste em pé entre galinhas --- fique de pé acima dos outros}
  \end{Phonetics}
\end{Entry}

\begin{Entry}{麫}{15}{⿆}
  \begin{Phonetics}{麫}{mian4}
    \variantof{面}
  \end{Phonetics}
\end{Entry}

\begin{Entry}{黎}{15}{⿉}
  \begin{Phonetics}{黎}{li2}
    \definition*{s.}{Etnia Li, uma das minorias nacionais da província de Hainan | Sobrenome Li}
    \definition{adj.}{Literário: preto; escuro | Literário: numeroso}
    \definition{s.}{multidão; as massas; a população}
  \end{Phonetics}
\end{Entry}

%%%%% EOF %%%%%

