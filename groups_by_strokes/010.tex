%%%
%%% 10画
%%%

\section*{10画}\addcontentsline{toc}{section}{10画}

\begin{entry}{乘}{10}{⽲}
  \begin{phonetics}{乘}{cheng2}[][HSK 5]
    \definition*{s.}{sobrenome Cheng}
    \definition{s.}{uma divisão principal das escolas budistas; uma seita ou doutrina do budismo}
    \definition{v.}{cavalgar; utilizar um veículo ou animal em vez de caminhar | aproveitar-se de; valer-se de; tirar vantagem de | multiplicar; realizar multiplicação | perseguir; caçar}
  \end{phonetics}
  \begin{phonetics}{乘}{sheng4}
    \definition{clas.}{para carruagens de guerra puxada por quatro cavalos}
    \definition{s.}{obras históricas; livros de história geral}
  \end{phonetics}
\end{entry}

\begin{entry}{乘车}{10,4}{⽲、⾞}
  \begin{phonetics}{乘车}{cheng2 che1}[][HSK 5]
    \definition{v.}{montar; dirigir; conduzir; andar a cavalo, de moto, de bicicleta, etc.}
  \end{phonetics}
\end{entry}

\begin{entry}{乘坐}{10,7}{⽲、⼟}
  \begin{phonetics}{乘坐}{cheng2zuo4}[][HSK 5]
    \definition{v.}{pegar (um trem, ônibus, etc.); andar de (bicicleta, moto, etc.)}
  \end{phonetics}
\end{entry}

\begin{entry}{乘客}{10,9}{⽲、⼧}
  \begin{phonetics}{乘客}{cheng2 ke4}[][HSK 5]
    \definition[个,位,名]{s.}{passageiro}
  \end{phonetics}
\end{entry}

\begin{entry}{乘客数}{10,9,13}{⽲、⼧、⽁}
  \begin{phonetics}{乘客数}{cheng2ke4 shu4}
    \definition{s.}{número de passageiros}
  \end{phonetics}
\end{entry}

\begin{entry}{俱乐部}{10,5,10}{⼈、⼃、⾢}
  \begin{phonetics}{俱乐部}{ju4le4bu4}[][HSK 5]
    \definition[个]{s.}{clube; grupos e locais para atividades sociais, políticas, literárias, recreativas e outras}
  \end{phonetics}
\end{entry}

\begin{entry}{倂}{10}{⼈}
  \begin{phonetics}{倂}{bing4}
    \variantof{并}
  \end{phonetics}
\end{entry}

\begin{entry}{倍}{10}{⼈}
  \begin{phonetics}{倍}{bei4}[][HSK 4]
    \definition{adv.}{mais; especialmente}
    \definition{clas.}{vezes; para obter um número igual ao número original, você pode multiplicar o número por esse múltiplo}
    \definition{s.}{dobro; duas vezes mais}
  \end{phonetics}
\end{entry}

\begin{entry}{倒}{10}{⼈}
  \begin{phonetics}{倒}{dao3}[][HSK 2]
    \definition{v.}{cair no chão | deitar-se no chão | colapsar | ir à falência}
  \end{phonetics}
  \begin{phonetics}{倒}{dao4}[][HSK 2]
    \definition{adv.}{ao contrário da expectativa | ao contrário}
    \definition{v.}{inverter | colocar de cabeça para baixo ou de frente para trás | derramar | tombar}
  \end{phonetics}
\end{entry}

\begin{entry}{倒车}{10,4}{⼈、⾞}
  \begin{phonetics}{倒车}{dao3che1}[][HSK 4]
    \definition{v.}{mudar de trem ou ônibus; trocar de trem ou ônibus no meio do caminho}
  \end{phonetics}
  \begin{phonetics}{倒车}{dao4che1}[][HSK 4]
    \definition{v.}{dar marcha à ré (em um veículo)}
  \end{phonetics}
\end{entry}

\begin{entry}{倒地}{10,6}{⼈、⼟}
  \begin{phonetics}{倒地}{dao3di4}
    \definition{v.}{cair no chão}
  \end{phonetics}
\end{entry}

\begin{entry}{倒血霉}{10,6,15}{⼈、⾎、⾬}
  \begin{phonetics}{倒血霉}{dao3xue4mei2}
    \definition{v.}{ter muito azar (versão mais forte de 倒霉)}
  \seealsoref{倒霉}{dao3mei2}
  \end{phonetics}
\end{entry}

\begin{entry}{倒闭}{10,6}{⼈、⾨}
  \begin{phonetics}{倒闭}{dao3bi4}[][HSK 4]
    \definition{v.}{fechar; ir à falência; entrar em liquidação; sair do negócio}
  \end{phonetics}
\end{entry}

\begin{entry}{倒是}{10,9}{⼈、⽇}
  \begin{phonetics}{倒是}{dao4 shi4}[][HSK 5]
    \definition{adv.}{usado para indicar o oposto do que geralmente é verdade; ao contrário do senso comum; pelo contrário | usado para indicar o que é contrário aos fatos, com um toque de crítica; indica que as coisas não são assim (com um sentimento de culpa) | usado de algo inesperado; expressando surpresa | usado para indicar concessão | usado para indicar uma mudança de significado; indica um ponto de virada | usado para modificar ou suavizar uma declaração anterior; para suavizar o tom | usado para pressionar ou questionar alguém; para instar ou perguntar}
  \end{phonetics}
\end{entry}

\begin{entry}{倒楣}{10,13}{⼈、⽊}
  \begin{phonetics}{倒楣}{dao3mei2}
    \variantof{倒霉}
  \end{phonetics}
\end{entry}

\begin{entry}{倒霉}{10,15}{⼈、⾬}
  \begin{phonetics}{倒霉}{dao3mei2}
    \definition{adj.}{azarado}
    \definition{s.}{azar | má sorte}
    \definition{v.}{estar sem sorte | ter azar}
  \seealsoref{倒血霉}{dao3xue4mei2}
  \end{phonetics}
\end{entry}

\begin{entry}{倘使}{10,8}{⼈、⼈}
  \begin{phonetics}{倘使}{tang3shi3}
    \definition{conj.}{se | supondo que | no caso}
  \end{phonetics}
\end{entry}

\begin{entry}{倘或}{10,8}{⼈、⼽}
  \begin{phonetics}{倘或}{tang3huo4}
    \definition{conj.}{se | supondo que | no caso}
  \end{phonetics}
\end{entry}

\begin{entry}{倘若}{10,8}{⼈、⾋}
  \begin{phonetics}{倘若}{tang3ruo4}
    \definition{conj.}{se | supondo que | no caso}
  \end{phonetics}
\end{entry}

\begin{entry}{借}{10}{⼈}
  \begin{phonetics}{借}{jie4}[][HSK 2]
    \definition{adv.}{por meio de}
    \definition{v.}{pedir emprestado | emprestar | aproveitar (uma oportunidade)}
  \end{phonetics}
\end{entry}

\begin{entry}{借书证}{10,4,7}{⼈、⼄、⾔}
  \begin{phonetics}{借书证}{jie4shu1zheng4}
    \definition{s.}{cartão de biblioteca | (literalmente) cartão para pedir emprestado livros}
  \end{phonetics}
\end{entry}

\begin{entry}{倡导}{10,6}{⼈、⼨}
  \begin{phonetics}{倡导}{chang4dao3}[][HSK 5]
    \definition{v.}{iniciar; propor; promover; defender; advogar}
  \end{phonetics}
\end{entry}

\begin{entry}{值}{10}{⼈}
  \begin{phonetics}{值}{zhi2}[][HSK 3]
    \definition{s.}{preço; valor; valor numérico}
    \definition{v.}{valer a pena | acontecer com; ir de encontro | estar de serviço; ter sua vez em algo; assumir a posição de turno}
  \end{phonetics}
\end{entry}

\begin{entry}{值得}{10,11}{⼈、⼻}
  \begin{phonetics}{值得}{zhi2de5}[][HSK 3]
    \definition{adj.}{que vale a pena}
    \definition{v.}{merecer; valer a pena; ser digno; significa que fazer isso terá bons resultados; é valioso e significativo}
  \end{phonetics}
\end{entry}

\begin{entry}{倾城}{10,9}{⼈、⼟}
  \begin{phonetics}{倾城}{qing1cheng2}
    \definition{adj.}{sedutora (mulher)}
    \definition{adv.}{de todo o lugar | vindo de todos os lugares}
    \definition{v.}{arruinar e derrubar o estado}
  \end{phonetics}
\end{entry}

\begin{entry}{健全}{10,6}{⼈、⼊}
  \begin{phonetics}{健全}{jian4quan2}[][HSK 5]
    \definition{adj.}{saudável; íntegro; capaz; apto; robusto e sem mácula | sólido; completo; perfeito}
    \definition{v.}{aperfeiçoar; melhorar; fortalecer; reforçar}
  \end{phonetics}
\end{entry}

\begin{entry}{健身}{10,7}{⼈、⾝}
  \begin{phonetics}{健身}{jian4shen1}[][HSK 4]
    \definition{s.}{exercício físico | \emph{fitness}}
    \definition{v.+compl.}{exercitar-se; manter a forma; praticar um esporte, especialmente a ginástica, inclusive em aparelhos, para desenvolver força, flexibilidade, aumentar a resistência, melhorar a coordenação e o controle de todas as partes do corpo}
  \end{phonetics}
\end{entry}

\begin{entry}{健康}{10,11}{⼈、⼴}
  \begin{phonetics}{健康}{jian4kang1}[][HSK 2]
    \definition{adj.}{em forma | saudável | curado}
    \definition{s.}{saúde | físico}
  \end{phonetics}
\end{entry}

\begin{entry}{兼}{10}{⼋}
  \begin{phonetics}{兼}{jian1}
    \definition{conj.}{e (ocupando dois ou mais cargos (oficiais) ao mesmo tempo)}
  \end{phonetics}
\end{entry}

\begin{entry}{准}{10}{⼎}
  \begin{phonetics}{准}{zhun3}[][HSK 3]
    \definition{adj.}{exato; preciso; algo determinado a ser imutável | preciso; exato; correto | perto; parcialmente; quase; próximo de algo em termos de padrão}
    \definition{adv.}{definitivamente; certamente}
    \definition{pref.}{quasi-; para-}
    \definition{prep.}{de acordo com; baseado em}
    \definition{s.}{norma; padrão; critério | confiança certa; uma ideia definida, certeza, etc. (geralmente usada depois de ``有'' ou ``没有'')}
    \definition{v.}{autorizar; conceder; consentir; permitir}
  \seealsoref{没有}{mei2you3}
  \seealsoref{有}{you3}
  \end{phonetics}
\end{entry}

\begin{entry}{准时}{10,7}{⼎、⽇}
  \begin{phonetics}{准时}{zhun3shi2}[][HSK 4]
    \definition{adj.}{pontual}
    \definition{adv.}{na hora certa; dentro do prazo; no horário especificado}
  \end{phonetics}
\end{entry}

\begin{entry}{准备}{10,8}{⼎、⼡}
  \begin{phonetics}{准备}{zhun3bei4}[][HSK 1]
    \definition{v.}{preparar | ficar pronto | pretender | planejar}
  \end{phonetics}
\end{entry}

\begin{entry}{准确}{10,12}{⼎、⽯}
  \begin{phonetics}{准确}{zhun3que4}[][HSK 2]
    \definition{adj.}{exato | preciso | acurado}
  \end{phonetics}
\end{entry}

\begin{entry}{凉}{10}{⼎}
  \begin{phonetics}{凉}{liang2}[][HSK 2]
    \definition{adj.}{frio | legal}
  \end{phonetics}
  \begin{phonetics}{凉}{liang4}
    \definition{v.}{esfriar | tornar ou tornar-se frio | deixar esfriar pelo ar}
  \end{phonetics}
\end{entry}

\begin{entry}{凉水}{10,4}{⼎、⽔}
  \begin{phonetics}{凉水}{liang2 shui3}[][HSK 3]
    \definition{s.}{água fria | água gelada | água não fervida}
  \end{phonetics}
\end{entry}

\begin{entry}{凉快}{10,7}{⼎、⼼}
  \begin{phonetics}{凉快}{liang2kuai5}[][HSK 2]
    \definition{adj.}{agradável e frio | agradavelmente fresco}
  \end{phonetics}
\end{entry}

\begin{entry}{凉鞋}{10,15}{⼎、⾰}
  \begin{phonetics}{凉鞋}{liang2xie2}
    \definition{s.}{sandália | alpargata | alpercata | alparca}
  \end{phonetics}
\end{entry}

\begin{entry}{剧本}{10,5}{⼑、⽊}
  \begin{phonetics}{剧本}{ju4ben3}[][HSK 5]
    \definition{s.}{cenário; roteiro (para drama, filme, etc.); gênero de obra literária que consiste em diálogos entre personagens (às vezes cantados) e indicações de palco}
  \end{phonetics}
\end{entry}

\begin{entry}{剧场}{10,6}{⼑、⼟}
  \begin{phonetics}{剧场}{ju4 chang3}[][HSK 3]
    \definition[个,坐]{s.}{teatro; um lugar para apresentações teatrais, canto e dança, etc.}
  \end{phonetics}
\end{entry}

\begin{entry}{原木}{10,4}{⼚、⽊}
  \begin{phonetics}{原木}{yuan2mu4}
    \definition{s.}{registro | \emph{logs}}
  \end{phonetics}
\end{entry}

\begin{entry}{原则}{10,6}{⼚、⼑}
  \begin{phonetics}{原则}{yuan2ze2}[][HSK 4]
    \definition{adv.}{em geral; em princípio; refere-se a um aspecto geral; geralmente}
    \definition[个]{s.}{princípios; leis ou padrões pelos quais alguém fala ou age}
  \end{phonetics}
\end{entry}

\begin{entry}{原因}{10,6}{⼚、⼞}
  \begin{phonetics}{原因}{yuan2yin1}[][HSK 2]
    \definition[个]{s.}{causa | razão | motivo}
  \end{phonetics}
\end{entry}

\begin{entry}{原色}{10,6}{⼚、⾊}
  \begin{phonetics}{原色}{yuan2 se4}
    \definition{s.}{cor primária}
  \end{phonetics}
\end{entry}

\begin{entry}{原来}{10,7}{⼚、⽊}
  \begin{phonetics}{原来}{yuan2lai2}[][HSK 2]
    \definition{adv.}{originalmente | como se vê | na verdade}
    \definition{v.}{vir a ser}
  \end{phonetics}
\end{entry}

\begin{entry}{原料}{10,10}{⼚、⽃}
  \begin{phonetics}{原料}{yuan2liao4}[][HSK 4]
    \definition[种,个]{s.}{matéria-prima; refere-se a materiais que não foram processados e fabricados, como minérios para metalurgia e algodão para têxteis}
  \end{phonetics}
\end{entry}

\begin{entry}{原理}{10,11}{⼚、⽟}
  \begin{phonetics}{原理}{yuan2li3}
    \definition{s.}{princípio | teoria}
  \end{phonetics}
\end{entry}

\begin{entry}{哥}{10}{⼝}
  \begin{phonetics}{哥}{ge1}[][HSK 1]
    \definition{s.}{irmão mais velho}
    \seeref{哥哥}{ge1 ge5}
  \end{phonetics}
\end{entry}

\begin{entry}{哥们}{10,5}{⼝、⼈}
  \begin{phonetics}{哥们}{ge1men5}
    \definition{expr.}{\emph{Brothers!}}
    \definition{s.}{(coloquial) cara | irmão (forma diminuta de tratamento entre homens)}
  \end{phonetics}
\end{entry}

\begin{entry}{哥哥}{10,10}{⼝、⼝}
  \begin{phonetics}{哥哥}{ge1 ge5}[][HSK 1]
    \definition[个,位]{s.}{irmão mais velho}
  \end{phonetics}
\end{entry}

\begin{entry}{哥斯拉}{10,12,8}{⼝、⽄、⼿}
  \begin{phonetics}{哥斯拉}{ge1si1la1}
    \definition*{s.}{Godzilla}
  \seealsoref{酷斯拉}{ku4si1la1}
  \end{phonetics}
\end{entry}

\begin{entry}{哦}{10}{⼝}
  \begin{phonetics}{哦}{e2}
    \definition{v.}{entoar cântico}
  \end{phonetics}
  \begin{phonetics}{哦}{o2}
    \definition{interj.}{Oh! (indicando dúvida ou surpresa)}
  \end{phonetics}
  \begin{phonetics}{哦}{o4}
    \definition{interj.}{Oh! (indicando que acabou de aprender algo)}
  \end{phonetics}
  \begin{phonetics}{哦}{o5}
    \definition{part.}{final da frase que transmite informalidade, calor, simpatia ou intimidade; também pode indicar que alguém está declarando um fato de que a outra pessoa não está ciente}
  \end{phonetics}
\end{entry}

\begin{entry}{哭}{10}{⼝}
  \begin{phonetics}{哭}{ku1}[][HSK 2]
    \definition{v.}{chorar}
  \end{phonetics}
\end{entry}

\begin{entry}{哭墙}{10,14}{⼝、⼟}
  \begin{phonetics}{哭墙}{ku1qiang2}
    \definition*{s.}{Muro das Lamentações (Jerusalém)}
  \end{phonetics}
\end{entry}

\begin{entry}{哮喘}{10,12}{⼝、⼝}
  \begin{phonetics}{哮喘}{xiao4chuan3}
    \definition{s.}{asma}
  \end{phonetics}
\end{entry}

\begin{entry}{哲理}{10,11}{⼝、⽟}
  \begin{phonetics}{哲理}{zhe2li3}
    \definition{s.}{filosofia | teoria filosófica}
  \end{phonetics}
\end{entry}

\begin{entry}{唇}{10}{⼝}
  \begin{phonetics}{唇}{chun2}
    \definition{s.}{lábios}
  \end{phonetics}
\end{entry}

\begin{entry}{唐人街}{10,2,12}{⼝、⼈、⾏}
  \begin{phonetics}{唐人街}{tang2ren2 jie1}
    \definition*{s.}{Bairro Chinês | \emph{Chinatown}}
  \seealsoref{中国城}{zhong1guo2cheng2}
  \end{phonetics}
\end{entry}

\begin{entry}{啊}{10}{⼝}
  \begin{phonetics}{啊}{a1}[][HSK 2]
    \definition{interj.}{Ah! | Oh! | interjeição de surpresa}
  \end{phonetics}
  \begin{phonetics}{啊}{a2}[][HSK 2]
    \definition{interj.}{Eh? | Que? | interjeição expressando dúvida ou exigindo resposta}
  \end{phonetics}
  \begin{phonetics}{啊}{a3}[][HSK 2]
    \definition{interj.}{Eh? | Meu! | E aí? | Que? | interjeição de surpresa ou dúvida}
  \end{phonetics}
  \begin{phonetics}{啊}{a4}[][HSK 2]
    \definition{interj.}{Ah! | OK! | Oh, é você! | Hum! | expressão de reconhecimento | interjeição de acordo}
  \end{phonetics}
  \begin{phonetics}{啊}{a5}[][HSK 2,4]
    \definition{adv.}{assim por diante}
    \definition{part.}{no final de sentença para expressar admiração | no final de sentença mostrando afirmação, aprovação, urgência, aconselhamento, etc. | no final de sentença para indicar uma pergunta | para pausar ligeiramente uma frase, chamando a atenção para as palavras seguintes | após cada um dos itens listados}
  \end{phonetics}
\end{entry}

\begin{entry}{啊呀}{10,7}{⼝、⼝}
  \begin{phonetics}{啊呀}{a1ya1}
    \definition{interj.}{Oh meu Deus! | interjeição de surpresa}
  \end{phonetics}
\end{entry}

\begin{entry}{啊哟}{10,9}{⼝、⼝}
  \begin{phonetics}{啊哟}{a1yo5}
    \definition{interj.}{Meu Deus! | Oh! | Ai! | interjeição de surpresa ou dor}
  \end{phonetics}
\end{entry}

\begin{entry}{圆}{10}{⼞}
  \begin{phonetics}{圆}{yuan2}[][HSK 4]
    \definition*{s.}{sobrenome Yuan}
    \definition{adj.}{redondo; circular; esférico; arredondado | diplomático; satisfatório}
    \definition[个]{s.}{círculo; circunferência | uma moeda de valor e peso fixos}
    \definition{v.}{tornar plausível; justificar; tornar completo; completar}
  \end{phonetics}
\end{entry}

\begin{entry}{圆满}{10,13}{⼞、⽔}
  \begin{phonetics}{圆满}{yuan2man3}[][HSK 4]
    \definition{adj.}{perfeito; satisfatório; sem defeitos}
  \end{phonetics}
\end{entry}

\begin{entry}{埋伏}{10,6}{⼟、⼈}
  \begin{phonetics}{埋伏}{mai2fu2}
    \definition{s.}{emboscada}
    \definition{v.}{emboscar}
  \end{phonetics}
\end{entry}

\begin{entry}{夏天}{10,4}{⼢、⼤}
  \begin{phonetics}{夏天}{xia4 tian1}[][HSK 2]
    \definition[个]{s.}{verão}
  \end{phonetics}
\end{entry}

\begin{entry}{夏日}{10,4}{⼢、⽇}
  \begin{phonetics}{夏日}{xia4ri4}
    \definition{s.}{horário de verão}
  \end{phonetics}
\end{entry}

\begin{entry}{夏季}{10,8}{⼢、⼦}
  \begin{phonetics}{夏季}{xia4 ji4}[][HSK 4]
    \definition{s.}{verão; segundo trimestre do ano, habitualmente chamado na China de período de três meses, do início do verão ao início do outono, também chamado de ``quarto, quinto e sexto'' meses do calendário lunar}
  \end{phonetics}
\end{entry}

\begin{entry}{套}{10}{⼤}
  \begin{phonetics}{套}{tao4}[][HSK 2]
    \definition{clas.}{para conjuntos, coleções}
    \definition{s.}{cobertura | fórmula | laço de corda}
    \definition{v.}{cobrir | envolver | intercalar | sobrepor}
  \end{phonetics}
\end{entry}

\begin{entry}{套问}{10,6}{⼤、⾨}
  \begin{phonetics}{套问}{tao4wen4}
    \definition{s.}{retórica}
    \definition{v.}{descobrir por meio de questionamento indireto diplomático}
  \end{phonetics}
\end{entry}

\begin{entry}{套餐}{10,16}{⼤、⾷}
  \begin{phonetics}{套餐}{tao4 can1}[][HSK 4]
    \definition{s.}{combo; pacote de produtos; pacote de serviços; metaforicamente, bens ou projetos que são combinados e levados ao mercado | refeição preparada; pacotes de refeições completos}
  \end{phonetics}
\end{entry}

\begin{entry}{孬}{10}{⼥}
  \begin{phonetics}{孬}{nao1}
    \definition{adj.}{(dialeto) não (é) bom (contração de 不+好)}
  \end{phonetics}
\end{entry}

\begin{entry}{害}{10}{⼧}
  \begin{phonetics}{害}{hai4}[][HSK 5]
    \definition{adj.}{prejudicial; destrutivo; injurioso; nocivo}
    \definition{s.}{mal; maldade; dano; calamidade}
    \definition{v.}{prejudicar; fazer mal a; causar problemas a | matar; assassinar | sofrer de; contrair (uma doença) | sentir-se (envergonhado, com medo, etc.); despertar (um sentimento ou uma emoção)}
  \end{phonetics}
\end{entry}

\begin{entry}{害怕}{10,8}{⼧、⼼}
  \begin{phonetics}{害怕}{hai4pa4}[][HSK 3]
    \definition{v.}{estar assustado; ter medo}
  \end{phonetics}
\end{entry}

\begin{entry}{害羞}{10,10}{⼧、⽺}
  \begin{phonetics}{害羞}{hai4xiu1}
    \definition{adj.}{tímido | envergonhado}
  \end{phonetics}
\end{entry}

\begin{entry}{家}{10}{⼧}
  \begin{phonetics}{家}{jia1}[][HSK 1,2]
    \definition{clas.}{para famílias ou empresas}
    \definition{pron.}{(educado) meu (irmã, tio, etc.)}
    \definition[个]{s.}{casa | família}
    \definition{suf.}{sufixo substantivo para designar um especialista em alguma atividade, como um músico ou revolucionário, para designar uma profissão como em -eiro, -ista}
  \end{phonetics}
\end{entry}

\begin{entry}{家人}{10,2}{⼧、⼈}
  \begin{phonetics}{家人}{jia1ren2}[][HSK 1]
    \definition{s.}{(a) família | membro da família}
  \end{phonetics}
\end{entry}

\begin{entry}{家乡}{10,3}{⼧、⼄}
  \begin{phonetics}{家乡}{jia1xiang1}[][HSK 3]
    \definition[个]{s.}{cidade natal}
  \end{phonetics}
\end{entry}

\begin{entry}{家长}{10,4}{⼧、⾧}
  \begin{phonetics}{家长}{jia1 zhang3}[][HSK 2]
    \definition[位,名,个]{s.}{pais | patriarca | guardião}
  \end{phonetics}
\end{entry}

\begin{entry}{家务}{10,5}{⼧、⼒}
  \begin{phonetics}{家务}{jia1wu4}[][HSK 4]
    \definition[堆,次,件]{s.}{trabalho doméstico; tarefas domésticas}
  \end{phonetics}
\end{entry}

\begin{entry}{家伙}{10,6}{⼧、⼈}
  \begin{phonetics}{家伙}{jia1huo5}
    \definition{s.}{prato, implemento ou móvel doméstico | animal doméstico | (coloquial) o cara | indivíduo | arma}
  \end{phonetics}
\end{entry}

\begin{entry}{家里}{10,7}{⼧、⾥}
  \begin{phonetics}{家里}{jia1 li3}[][HSK 1]
    \definition{adv.}{em casa}
  \end{phonetics}
\end{entry}

\begin{entry}{家具}{10,8}{⼧、⼋}
  \begin{phonetics}{家具}{jia1ju4}[][HSK 3]
    \definition[件,套]{s.}{móveis; mobiliário de casa}
  \end{phonetics}
\end{entry}

\begin{entry}{家庭}{10,9}{⼧、⼴}
  \begin{phonetics}{家庭}{jia1ting2}[][HSK 2]
    \definition[个,户]{s.}{família}
  \end{phonetics}
\end{entry}

\begin{entry}{家俱}{10,10}{⼧、⼈}
  \begin{phonetics}{家俱}{jia1ju4}
    \variantof{家具}
  \end{phonetics}
\end{entry}

\begin{entry}{家属}{10,12}{⼧、⼫}
  \begin{phonetics}{家属}{jia1shu3}[][HSK 3]
    \definition{s.}{membros da família; dependentes (familiares)}
  \end{phonetics}
\end{entry}

\begin{entry}{容易}{10,8}{⼧、⽇}
  \begin{phonetics}{容易}{rong2yi4}[][HSK 3]
    \definition{adj.}{fácil; simples | provável; responsável; apto}
  \end{phonetics}
\end{entry}

\begin{entry}{容貌}{10,14}{⼧、⾘}
  \begin{phonetics}{容貌}{rong2mao4}
    \definition{s.}{aparência | aspecto | características}
  \end{phonetics}
\end{entry}

\begin{entry}{宽}{10}{⼧}
  \begin{phonetics}{宽}{kuan1}[][HSK 4]
    \definition*{s.}{sobrenome Kuan}
    \definition{adj.}{largo; amplo; grandes distâncias horizontais | leniente; generoso; indulgente | bem de vida; confortável | espaçoso}
    \definition{s.}{largura; amplitude}
    \definition{v.}{relaxar; aliviar}
  \end{phonetics}
\end{entry}

\begin{entry}{宽广}{10,3}{⼧、⼴}
  \begin{phonetics}{宽广}{kuan1 guang3}[][HSK 4]
    \definition{adj.}{vasto; amplo; espaçoso; extenso}
  \end{phonetics}
\end{entry}

\begin{entry}{宽度}{10,9}{⼧、⼴}
  \begin{phonetics}{宽度}{kuan1 du4}[][HSK 5]
    \definition{s.}{largura; amplitude; duração; o grau de largura e estreiteza; a distância horizontal (no caso de um retângulo, a distância entre os dois lados mais longos)}
  \end{phonetics}
\end{entry}

\begin{entry}{宽影片}{10,15,4}{⼧、⼺、⽚}
  \begin{phonetics}{宽影片}{kuan1ying3pian4}
    \definition{s.}{filme \emph{widescreen}}
  \end{phonetics}
\end{entry}

\begin{entry}{宾馆}{10,11}{⼧、⾷}
  \begin{phonetics}{宾馆}{bin1guan3}[][HSK 5]
    \definition[家,个,座]{s.}{hotel; acomodações públicas para hóspedes}
  \end{phonetics}
\end{entry}

\begin{entry}{射}{10}{⼨}
  \begin{phonetics}{射}{she4}[][HSK 5]
    \definition*{s.}{sobrenome She}
    \definition{v.}{atirar; disparar | descarregar em jato; jorrar | emitir (luz, calor, etc.) | irradiar | aludir a algo ou alguém; insinuar}
  \end{phonetics}
\end{entry}

\begin{entry}{射击}{10,5}{⼨、⼐}
  \begin{phonetics}{射击}{she4ji1}[][HSK 5]
    \definition{s.}{tiro; tiro ao alvo}
    \definition{v.}{disparar; atirar}
  \end{phonetics}
\end{entry}

\begin{entry}{展开}{10,4}{⼫、⼶}
  \begin{phonetics}{展开}{zhan3kai1}[][HSK 3]
    \definition{s.}{desenvolvimento; expansão; explosão; evolução}
    \definition{v.}{desenvolver; espalhar; desdobrar; abrir; desenrolar; amplificar; esticar; ventilar | lançar; desdobrar; desenvolver; executar}
  \end{phonetics}
\end{entry}

\begin{entry}{展示}{10,5}{⼫、⽰}
  \begin{phonetics}{展示}{zhan3shi4}
    \definition{v.}{revelar | mostrar | exibir}
  \end{phonetics}
\end{entry}

\begin{entry}{席卷}{10,8}{⼱、⼙}
  \begin{phonetics}{席卷}{xi2juan3}
    \definition{v.}{engolfar | varrer | levar tudo para fora}
  \end{phonetics}
\end{entry}

\begin{entry}{座}{10}{⼴}
  \begin{phonetics}{座}{zuo4}[][HSK 2]
    \definition{clas.}{frequentemente usado para objetos maiores ou fixos}
    \definition{s.}{assento | lugar | base | suporte | pedestal | constelação}
  \end{phonetics}
\end{entry}

\begin{entry}{座子}{10,3}{⼴、⼦}
  \begin{phonetics}{座子}{zuo4zi5}
    \definition{s.}{soquete | pedestal | sela}
  \end{phonetics}
\end{entry}

\begin{entry}{座位}{10,7}{⼴、⼈}
  \begin{phonetics}{座位}{zuo4wei4}[][HSK 2]
    \definition[个]{s.}{assento | lugar}
  \end{phonetics}
\end{entry}

\begin{entry}{座标}{10,9}{⼴、⽊}
  \begin{phonetics}{座标}{zuo4biao1}
    \variantof{坐标}
  \end{phonetics}
\end{entry}

\begin{entry}{弱}{10}{⼸}
  \begin{phonetics}{弱}{ruo4}[][HSK 4]
    \definition{adj.}{fraco; debilitado | jovem | inferior; pior | colocado depois de uma fração ou decimal para indicar que é um pouco menor que esse número}
    \definition{v.}{perder (através da morte)}
  \end{phonetics}
\end{entry}

\begin{entry}{徒手}{10,4}{⼻、⼿}
  \begin{phonetics}{徒手}{tu2shou3}
    \definition{adj.}{com as mãos vazias | desarmado | mão livre (desenho) | lutando mão-a-mão}
  \end{phonetics}
\end{entry}

\begin{entry}{恋爱}{10,10}{⼼、⽖}
  \begin{phonetics}{恋爱}{lian4'ai4}[][HSK 5]
    \definition[个,场]{s.}{amor (romântico)}
    \definition[个,场]{s.}{namoro; afeto; amor romântico}
    \definition{v.}{amar; estar apaixonado}
  \end{phonetics}
\end{entry}

\begin{entry}{恐龙}{10,5}{⼼、⿓}
  \begin{phonetics}{恐龙}{kong3long2}
    \definition[头,只]{s.}{dinossauro}
  \end{phonetics}
\end{entry}

\begin{entry}{恐怕}{10,8}{⼼、⼼}
  \begin{phonetics}{恐怕}{kong3pa4}[][HSK 3]
    \definition{adv.}{talvez; provavelmente; pode ser | por medo de}
    \definition{v.}{ter medo de; temer; recear}
  \end{phonetics}
\end{entry}

\begin{entry}{恐怖主义}{10,8,5,3}{⼼、⼼、⼂、⼂}
  \begin{phonetics}{恐怖主义}{kong3bu4zhu3yi4}
    \definition{adj.}{terrorista}
    \definition{s.}{terrorismo}
  \end{phonetics}
\end{entry}

\begin{entry}{恩赐}{10,12}{⼼、⾙}
  \begin{phonetics}{恩赐}{en1ci4}
    \definition{s.}{favor | caridade}
    \definition{v.}{conceder (favor, caridade)}
  \end{phonetics}
\end{entry}

\begin{entry}{恶心}{10,4}{⼼、⼼}
  \begin{phonetics}{恶心}{e3xin1}[][HSK 4]
    \definition{adj.}{nauseante; repugnante}
    \definition{s.}{enjoo; náusea; repugnância; sensação de enjoo; vontade de vomitar}
    \definition{v.}{repugnar; ser nauseante; vomitar}
  \end{phonetics}
  \begin{phonetics}{恶心}{e4xin1}
    \definition{s.}{mau hábito | hábito vicioso | vício}
  \end{phonetics}
\end{entry}

\begin{entry}{悄悄}{10,10}{⼼、⼼}
  \begin{phonetics}{悄悄}{qiao1qiao1}[][HSK 5]
    \definition{adv.}{silenciosamente; em silêncio; aos sussuros; sem som ou em voz baixa; com o mínimo de ruído possível}
  \end{phonetics}
\end{entry}

\begin{entry}{扇}{10}{⼾}
  \begin{phonetics}{扇}{shan1}[][HSK 5]
    \definition{s.}{ventilar; agitar um leque para fazer o ar circular | dar um tapa; bater com a palma da mão | bater asas; esvoaçar | incitar; instigar; estimular; agitar}
  \end{phonetics}
  \begin{phonetics}{扇}{shan4}[][HSK 5]
    \definition{clas.}{para portas, janelas, etc.}
    \definition[把]{s.}{leque | folha; algo em forma de placa ou folha}
  \end{phonetics}
\end{entry}

\begin{entry}{扇子}{10,3}{⼾、⼦}
  \begin{phonetics}{扇子}{shan4zi5}[][HSK 5]
    \definition[把]{s.}{leque; abano; abanador; utensílios que produzem vento ao serem agitados}
  \end{phonetics}
\end{entry}

\begin{entry}{拳王}{10,4}{⼿、⽟}
  \begin{phonetics}{拳王}{quan2wang2}
    \definition{s.}{pugilista | boxeador}
  \end{phonetics}
\end{entry}

\begin{entry}{拳法}{10,8}{⼿、⽔}
  \begin{phonetics}{拳法}{quan2fa3}
    \definition{s.}{boxe | luta}
  \end{phonetics}
\end{entry}

\begin{entry}{拿}{10}{⼿}
  \begin{phonetics}{拿}{na2}[][HSK 1]
    \definition{part.}{usado da mesma forma que 把: para marcar o seguinte substantivo seguinte como objeto direto}
    \definition{v.}{segurar | tomar | pegar em}
  \end{phonetics}
\end{entry}

\begin{entry}{拿出}{10,5}{⼿、⼐}
  \begin{phonetics}{拿出}{na2 chu1}[][HSK 2]
    \definition{v.}{apresentar (evidências) | prover | apresentar (uma proposta) | colocar para fora | retirar}
  \end{phonetics}
\end{entry}

\begin{entry}{拿到}{10,8}{⼿、⼑}
  \begin{phonetics}{拿到}{na2 dao4}[][HSK 2]
    \definition{v.}{pegar | obter}
  \end{phonetics}
\end{entry}

\begin{entry}{挫折}{10,7}{⼿、⼿}
  \begin{phonetics}{挫折}{cuo4zhe2}
    \definition{s.}{revés | reverso | derrota | frustração | decepção}
    \definition{v.}{frustrar | desencorajar | subjugar}
  \end{phonetics}
\end{entry}

\begin{entry}{捞}{10}{⼿}
  \begin{phonetics}{捞}{lao1}
    \definition{v.}{pescar | dragar}
  \end{phonetics}
\end{entry}

\begin{entry}{损失}{10,5}{⼿、⼤}
  \begin{phonetics}{损失}{sun3shi1}[][HSK 5]
    \definition{s.}{perda; desperdício; algo que se consome ou se perde sem custo algum}
    \definition{v.}{perder; consumir ou perder}
  \end{phonetics}
\end{entry}

\begin{entry}{损害}{10,10}{⼿、⼧}
  \begin{phonetics}{损害}{sun3 hai4}[][HSK 5]
    \definition{v.}{prejudicar; danificar; causar danos}
  \end{phonetics}
\end{entry}

\begin{entry}{捡}{10}{⼿}
  \begin{phonetics}{捡}{jian3}
    \definition{v.}{apanhar | recolher | coletar}
  \end{phonetics}
\end{entry}

\begin{entry}{换}{10}{⼿}
  \begin{phonetics}{换}{huan4}[][HSK 2]
    \definition{v.}{mudar | trocar | substituir | converter (moedas)}
  \end{phonetics}
\end{entry}

\begin{entry}{换钱}{10,10}{⼿、⾦}
  \begin{phonetics}{换钱}{huan4qian2}
    \definition{v.+compl.}{trocar dinheiro (em pequenas valores ou em outra moeda) | trocar (mercadorias) por dinheiro | vender}
  \end{phonetics}
\end{entry}

\begin{entry}{效果}{10,8}{⽁、⽊}
  \begin{phonetics}{效果}{xiao4guo3}[][HSK 3]
    \definition[种,个]{s.}{efeito; resultado | efeitos sonoros; vários sons ou fenômenos naturais criados para combinar com o enredo em dramas e filmes, como vento e chuva, tiros, fogo, neve, etc.}
  \end{phonetics}
\end{entry}

\begin{entry}{效率}{10,11}{⽁、⽞}
  \begin{phonetics}{效率}{xiao4lv4}[][HSK 4]
    \definition[种]{s.}{eficiência; produtividade}
  \end{phonetics}
\end{entry}

\begin{entry}{敌人}{10,2}{⾆、⼈}
  \begin{phonetics}{敌人}{di2ren2}[][HSK 4]
    \definition[群,伙,股,批,帮,个]{s.}{inimigo; pessoa hostil; parte hostil}
  \end{phonetics}
\end{entry}

\begin{entry}{旁}{10}{⽅}
  \begin{phonetics}{旁}{pang2}[][HSK 5]
    \definition{adj.}{outro | abundante; abrangente}
    \definition{s.}{lado | radical lateral de um caractere chinês}
  \end{phonetics}
\end{entry}

\begin{entry}{旁边}{10,5}{⽅、⾡}
  \begin{phonetics}{旁边}{pang2bian1}[][HSK 1]
    \definition{adv.}{junto a | próximo de | ao lado}
  \end{phonetics}
\end{entry}

\begin{entry}{旅行}{10,6}{⽅、⾏}
  \begin{phonetics}{旅行}{lv3xing2}[][HSK 2]
    \definition{v.}{viajar}
  \end{phonetics}
\end{entry}

\begin{entry}{旅行社}{10,6,7}{⽅、⾏、⽰}
  \begin{phonetics}{旅行社}{lv3 xing2 she4}[][HSK 3]
    \definition[家]{s.}{agência de viagens}
  \end{phonetics}
\end{entry}

\begin{entry}{旅客}{10,9}{⽅、⼧}
  \begin{phonetics}{旅客}{lv3 ke4}[][HSK 2]
    \definition{s.}{viajante | turista}
  \end{phonetics}
\end{entry}

\begin{entry}{旅馆}{10,11}{⽅、⾷}
  \begin{phonetics}{旅馆}{lv3 guan3}[][HSK 3]
    \definition[家,个,所]{s.}{pousada; hotel}
  \end{phonetics}
\end{entry}

\begin{entry}{旅游}{10,12}{⽅、⽔}
  \begin{phonetics}{旅游}{lv3you2}[][HSK 2]
    \definition[趟,次,个]{s.}{jornada | viagem}
    \definition{v.}{viajar}
  \end{phonetics}
\end{entry}

\begin{entry}{旅程}{10,12}{⽅、⽲}
  \begin{phonetics}{旅程}{lv3cheng2}
    \definition{s.}{jornada | viagem}
  \end{phonetics}
\end{entry}

\begin{entry}{晒}{10}{⽇}
  \begin{phonetics}{晒}{shai4}[][HSK 4]
    \definition{v.}{(sol) brilhar sobre | aquecer-se; secar ao sol; colocar algo sob a luz do sol para secar | ignorar (alguém) | mostrar; divulgar o conteúdo de sua vida privada na Internet}
  \end{phonetics}
\end{entry}

\begin{entry}{晒干}{10,3}{⽇、⼲}
  \begin{phonetics}{晒干}{shai4gan1}
    \definition{v.}{secar ao sol}
  \end{phonetics}
\end{entry}

\begin{entry}{朗读}{10,10}{⽉、⾔}
  \begin{phonetics}{朗读}{lang3du2}[][HSK 5]
    \definition{v.}{ler em voz alta; recitar com voz clara e alta}
  \end{phonetics}
\end{entry}

\begin{entry}{校}{10}{⽊}
  \begin{phonetics}{校}{jiao4}
    \definition{v.}{verificar | comparar | revisar}
  \end{phonetics}
  \begin{phonetics}{校}{xiao4}
    \definition[所]{s.}{oficial militar | escola}
  \end{phonetics}
\end{entry}

\begin{entry}{校长}{10,4}{⽊、⾧}
  \begin{phonetics}{校长}{xiao4zhang3}[][HSK 2]
    \definition[个,位,名]{s.}{diretor de escola | reitor (universidade)}
  \end{phonetics}
\end{entry}

\begin{entry}{校园}{10,7}{⽊、⼞}
  \begin{phonetics}{校园}{xiao4 yuan2}[][HSK 2]
    \definition{s.}{campus}
  \end{phonetics}
\end{entry}

\begin{entry}{校服}{10,8}{⽊、⽉}
  \begin{phonetics}{校服}{xiao4fu2}
    \definition{s.}{uniforme escolar}
  \end{phonetics}
\end{entry}

\begin{entry}{校规}{10,8}{⽊、⾒}
  \begin{phonetics}{校规}{xiao4gui1}
    \definition{s.}{regras e regulamentos escolares}
  \end{phonetics}
\end{entry}

\begin{entry}{校监}{10,10}{⽊、⽫}
  \begin{phonetics}{校监}{xiao4jian1}
    \definition{s.}{diretor | supervisor (de escola)}
  \end{phonetics}
\end{entry}

\begin{entry}{样}{10}{⽊}
  \begin{phonetics}{样}{yang4}
    \definition{s.}{aparência | forma | modelo}
  \end{phonetics}
\end{entry}

\begin{entry}{样儿}{10,2}{⽊、⼉}
  \begin{phonetics}{样儿}{yang4r5}
    \definition{s.}{aparência | forma | modelo}
  \seealsoref{样子}{yang4zi5}
  \end{phonetics}
\end{entry}

\begin{entry}{样子}{10,3}{⽊、⼦}
  \begin{phonetics}{样子}{yang4zi5}[][HSK 2]
    \definition{s.}{aparência | forma | modelo}
  \seealsoref{样儿}{yang4r5}
  \end{phonetics}
\end{entry}

\begin{entry}{样品}{10,9}{⽊、⼝}
  \begin{phonetics}{样品}{yang4pin3}
    \definition{s.}{amostra | espécime}
  \end{phonetics}
\end{entry}

\begin{entry}{样样}{10,10}{⽊、⽊}
  \begin{phonetics}{样样}{yang4yang4}
    \definition{adv.}{todos os tipos}
  \end{phonetics}
\end{entry}

\begin{entry}{样章}{10,11}{⽊、⾳}
  \begin{phonetics}{样章}{yang4zhang1}
    \definition{s.}{capítulo de amostra}
  \end{phonetics}
\end{entry}

\begin{entry}{核}{10}{⽊}
  \begin{phonetics}{核}{he2}
    \definition{adj.}{nuclear}
    \definition{s.}{poço | pedra | núcleo}
    \definition{v.}{examinar | checar | verificar}
  \end{phonetics}
\end{entry}

\begin{entry}{根}{10}{⽊}
  \begin{phonetics}{根}{gen1}[][HSK 4]
    \definition*{s.}{sobrenome Gen}
    \definition{adv.}{completamente; minuciosamente; radicalmente}
    \definition{clas.}{para objetos finos, alongados}
    \definition{s.}{raiz (de uma planta) | descendentes; posteridade; analogia com as gerações futuras | raiz (abreviação de raiz quadrada) | radical (química, refere-se a radicais carregados) | base; pé; raiz; parte inferior, base ou parte de um objeto que está presa a outra coisa | a parte de baixo das coisas; fonte; a origem  das coisas | base; fundamento}
  \end{phonetics}
\end{entry}

\begin{entry}{根本}{10,5}{⽊、⽊}
  \begin{phonetics}{根本}{gen1ben3}[][HSK 3]
    \definition{adj.}{básico; essencial; fundamental}
    \definition{adv.}{sempre; simplesmente; absolutamente; de qualquer modo | radically; thoroughly}
    \definition[个]{s.}{base; raiz; fundação}
  \end{phonetics}
\end{entry}

\begin{entry}{根据}{10,11}{⽊、⼿}
  \begin{phonetics}{根据}{gen1ju4}[][HSK 4]
    \definition[个]{prep.}{com base em; de acordo com; à luz de}
    \definition{s.}{base; fundamentos; razão; fundo; alicerce}
    \definition{v.}{basear; usar algo como premissa para uma conclusão ou como base para uma ação verbal}
  \end{phonetics}
\end{entry}

\begin{entry}{格兰菜}{10,5,11}{⽊、⼋、⾋}
  \begin{phonetics}{格兰菜}{ge2lan2cai4}
    \definition{s.}{brócolis chinês | couve chinesa | mostarda}
    \seeref{芥蓝}{gai4lan2}
  \end{phonetics}
\end{entry}

\begin{entry}{格外}{10,5}{⽊、⼣}
  \begin{phonetics}{格外}{ge2wai4}[][HSK 4]
    \definition{adv.}{especialmente; particularmente; ainda mais; indica mais do que a média | adicionalmente; indica adicional ou extra}
  \end{phonetics}
\end{entry}

\begin{entry}{栽}{10}{⽊}
  \begin{phonetics}{栽}{zai1}
    \definition{v.}{cultivar | plantar}
  \end{phonetics}
\end{entry}

\begin{entry}{栽种}{10,9}{⽊、⽲}
  \begin{phonetics}{栽种}{zai1zhong4}
    \definition{v.}{plantar}
  \end{phonetics}
\end{entry}

\begin{entry}{栽倒}{10,10}{⽊、⼈}
  \begin{phonetics}{栽倒}{zai1dao3}
    \definition{v.}{cair | sofrer uma queda}
  \end{phonetics}
\end{entry}

\begin{entry}{栽赃}{10,10}{⽊、⾙}
  \begin{phonetics}{栽赃}{zai1zang1}
    \definition{v.}{enquadrar alguém (plantar provas nele)}
  \end{phonetics}
\end{entry}

\begin{entry}{栽培}{10,11}{⽊、⼟}
  \begin{phonetics}{栽培}{zai1pei2}
    \definition{v.}{cultivar | educar | patrocinar | treinar}
  \end{phonetics}
\end{entry}

\begin{entry}{栽培种}{10,11,9}{⽊、⼟、⽲}
  \begin{phonetics}{栽培种}{zai1pei2 zhong3}
    \definition{s.}{espécies cultivadas}
  \end{phonetics}
\end{entry}

\begin{entry}{栽植}{10,12}{⽊、⽊}
  \begin{phonetics}{栽植}{zai1zhi2}
    \definition{v.}{plantar | transplantar}
  \end{phonetics}
\end{entry}

\begin{entry}{桃}{10}{⽊}
  \begin{phonetics}{桃}{tao2}[][HSK 5]
    \definition*{s.}{sobrenome Tao}
    \definition{s.}{pêssego | em forma de pêssego | pessegueiro}
  \end{phonetics}
\end{entry}

\begin{entry}{桃花}{10,7}{⽊、⾋}
  \begin{phonetics}{桃花}{tao2 hua1}[][HSK 5]
    \definition{s.}{(figurativo) caso amoroso | flor de pessegueiro}
  \end{phonetics}
\end{entry}

\begin{entry}{桃树}{10,9}{⽊、⽊}
  \begin{phonetics}{桃树}{tao2 shu4}[][HSK 5]
    \definition[株]{s.}{pêssego (árvore) | pessegueiro; pêssegos}
  \end{phonetics}
\end{entry}

\begin{entry}{桌}{10}{⽊}
  \begin{phonetics}{桌}{zhuo1}
    \definition{clas.}{para mesas de convidados em um banquete etc.}
    \definition{s.}{mesa}
  \end{phonetics}
\end{entry}

\begin{entry}{桌子}{10,3}{⽊、⼦}
  \begin{phonetics}{桌子}{zhuo1zi5}[][HSK 1]
    \definition[张,套]{s.}{mesa}
  \end{phonetics}
\end{entry}

\begin{entry}{桌布}{10,5}{⽊、⼱}
  \begin{phonetics}{桌布}{zhuo1bu4}
    \definition[条,块,张]{s.}{(computação) plano de fundo da área de trabalho | toalha de mesa | papel de parede}
  \end{phonetics}
\end{entry}

\begin{entry}{桌机}{10,6}{⽊、⽊}
  \begin{phonetics}{桌机}{zhuo1ji1}
    \definition{s.}{computador \emph{desktop}}
  \end{phonetics}
\end{entry}

\begin{entry}{桌灯}{10,6}{⽊、⽕}
  \begin{phonetics}{桌灯}{zhuo1deng1}
    \definition{s.}{luminária | lâmpada de mesa}
  \end{phonetics}
\end{entry}

\begin{entry}{桌面}{10,9}{⽊、⾯}
  \begin{phonetics}{桌面}{zhuo1mian4}
    \definition{s.}{área de trabalho | mesa}
  \end{phonetics}
\end{entry}

\begin{entry}{桌球}{10,11}{⽊、⽟}
  \begin{phonetics}{桌球}{zhuo1qiu2}
    \definition{s.}{bilhar | sinuca | mesa de ping-pong}
  \end{phonetics}
\end{entry}

\begin{entry}{桌游}{10,12}{⽊、⽔}
  \begin{phonetics}{桌游}{zhuo1you2}
    \definition{s.}{jogo de tabuleiro}
  \end{phonetics}
\end{entry}

\begin{entry}{桑}{10}{⽊}
  \begin{phonetics}{桑}{sang1}
    \definition*{s.}{sobrenome Sang}
    \definition{s.}{amoreira}
  \end{phonetics}
\end{entry}

\begin{entry}{桑巴舞}{10,4,14}{⽊、⼰、⾇}
  \begin{phonetics}{桑巴舞}{sang1ba1wu3}
    \definition{s.}{samba}
  \end{phonetics}
\end{entry}

\begin{entry}{桑树}{10,9}{⽊、⽊}
  \begin{phonetics}{桑树}{sang1shu4}
    \definition{s.}{amoreira, suas folhas são utilizadas para alimentar bichos-da-seda}
  \end{phonetics}
\end{entry}

\begin{entry}{桥}{10}{⽊}
  \begin{phonetics}{桥}{qiao2}[][HSK 3]
    \definition*{s.}{sobrenome Qiao}
    \definition[座]{s.}{ponte}
  \end{phonetics}
\end{entry}

\begin{entry}{桩}{10}{⽊}
  \begin{phonetics}{桩}{zhuang1}
    \definition{clas.}{para eventos, casos, transações, assuntos, etc.}
    \definition{s.}{toco | estaca | pilha}
  \end{phonetics}
\end{entry}

\begin{entry}{欱}{10}{⽋}
  \begin{phonetics}{欱}{he1}
    \variantof{喝}
  \end{phonetics}
\end{entry}

\begin{entry}{氧}{10}{⽓}
  \begin{phonetics}{氧}{yang3}
    \definition{s.}{oxigênio}
  \end{phonetics}
\end{entry}

\begin{entry}{流}{10}{⽔}
  \begin{phonetics}{流}{liu2}[][HSK 2]
    \definition[名,个]{s.}{fluxo de água | correnteza | córrego | algo que se assemelha a um fluxo de água | corrente | fluxo | classe | grau | taxa (de variação)}
    \definition{v.}{fluir
deriva; mover; vagar
espalhar
degenerar; mudar para pior
enviar para o exílio; banir}
  \end{phonetics}
\end{entry}

\begin{entry}{流水}{10,4}{⽔、⽔}
  \begin{phonetics}{流水}{liu2shui3}
    \definition{s.}{água corrente | (negócio) rotatividade}
  \end{phonetics}
\end{entry}

\begin{entry}{流传}{10,6}{⽔、⼈}
  \begin{phonetics}{流传}{liu2chuan2}[][HSK 4]
    \definition{v.}{espalhar; circular; passar adiante}
  \end{phonetics}
\end{entry}

\begin{entry}{流动}{10,6}{⽔、⼒}
  \begin{phonetics}{流动}{liu2 dong4}[][HSK 5]
    \definition{v.}{fluir; correr; circular | ir de um lugar para outro; mover-se; mudar frequentemente de posição}
  \end{phonetics}
\end{entry}

\begin{entry}{流行}{10,6}{⽔、⾏}
  \begin{phonetics}{流行}{liu2xing2}[][HSK 2]
    \definition{adj.}{(estilo de roupa, música, etc.) popular, na moda}
    \definition{v.}{(doença contagiosa, etc.) espalhar | propagar}
  \end{phonetics}
\end{entry}

\begin{entry}{流利}{10,7}{⽔、⼑}
  \begin{phonetics}{流利}{liu2li4}[][HSK 2]
    \definition{adj.}{fluente (em uma língua)}
  \end{phonetics}
\end{entry}

\begin{entry}{流星}{10,9}{⽔、⽇}
  \begin{phonetics}{流星}{liu2xing1}
    \definition{s.}{meteoro | estrela cadente}
  \end{phonetics}
\end{entry}

\begin{entry}{流通}{10,10}{⽔、⾡}
  \begin{phonetics}{流通}{liu2tong1}[][HSK 5]
    \definition{v.}{(de ar, dinheiro, mercadorias, etc.) fluir; circular}
  \end{phonetics}
\end{entry}

\begin{entry}{浙江}{10,6}{⽔、⽔}
  \begin{phonetics}{浙江}{zhe4jiang1}
    \definition*{s.}{Zhejiang}
  \end{phonetics}
\end{entry}

\begin{entry}{浪花}{10,7}{⽔、⾋}
  \begin{phonetics}{浪花}{lang4hua1}
    \definition[朵]{s.}{\emph{spray} | \emph{spray} do oceano | (figurativo) acontecimentos de sua vida}
  \end{phonetics}
\end{entry}

\begin{entry}{浪费}{10,9}{⽔、⾙}
  \begin{phonetics}{浪费}{lang4fei4}[][HSK 3]
    \definition{adj.}{desperdiçado}
    \definition{adv.}{extravagantemente}
    \definition{v.}{desperdiçar; dissipar; esbanjar; ser extravagante}
  \end{phonetics}
\end{entry}

\begin{entry}{浪漫}{10,14}{⽔、⽔}
  \begin{phonetics}{浪漫}{lang4man4}[][HSK 5]
    \definition{adj.}{romântico; poético | não convencional; boêmio; abandonado; libertino; devasso; comportar-se de maneira descuidada e descuidada (geralmente se referindo a relacionamentos entre homens e mulheres) | irrealista; impraticável}
  \end{phonetics}
\end{entry}

\begin{entry}{浮力}{10,2}{⽔、⼒}
  \begin{phonetics}{浮力}{fu2li4}
    \definition{s.}{flutuabilidade}
  \end{phonetics}
\end{entry}

\begin{entry}{浮图}{10,8}{⽔、⼞}
  \begin{phonetics}{浮图}{fu2tu2}
    \definition*{s.}{Termo alternativo para 佛陀}
    \variantof{浮屠}
  \seealsoref{佛陀}{fo2tuo2}
  \end{phonetics}
\end{entry}

\begin{entry}{浮屠}{10,11}{⽔、⼫}
  \begin{phonetics}{浮屠}{fu2tu2}
    \definition*{s.}{Buda | Templo (Stupa) Budista (transliteração de Pali Thuo)}
  \end{phonetics}
\end{entry}

\begin{entry}{海}{10}{⽔}
  \begin{phonetics}{海}{hai3}[][HSK 2]
    \definition*{s.}{sobrenome Hai}
    \definition[个,片]{s.}{mar | oceano}
  \end{phonetics}
\end{entry}

\begin{entry}{海水}{10,4}{⽔、⽔}
  \begin{phonetics}{海水}{hai3 shui3}[][HSK 4]
    \definition[把]{s.}{água do mar; salmoura}
  \end{phonetics}
\end{entry}

\begin{entry}{海风}{10,4}{⽔、⾵}
  \begin{phonetics}{海风}{hai3feng1}
    \definition{s.}{brisa do mar | vento que vem do mar}
  \end{phonetics}
\end{entry}

\begin{entry}{海边}{10,5}{⽔、⾡}
  \begin{phonetics}{海边}{hai3 bian1}[][HSK 2]
    \definition{s.}{costa marítima | litoral | beira-mar | praia}
  \end{phonetics}
\end{entry}

\begin{entry}{海关}{10,6}{⽔、⼋}
  \begin{phonetics}{海关}{hai3guan1}[][HSK 3]
    \definition{s.}{alfândega}
  \end{phonetics}
\end{entry}

\begin{entry}{海里}{10,7}{⽔、⾥}
  \begin{phonetics}{海里}{hai3li3}
    \definition{s.}{milha náutica}
  \end{phonetics}
\end{entry}

\begin{entry}{海底}{10,8}{⽔、⼴}
  \begin{phonetics}{海底}{hai3di3}
    \definition{adj.}{submarino}
    \definition{s.}{fundo do mar | solo oceânico | fundo do oceano}
  \end{phonetics}
\end{entry}

\begin{entry}{海鸥}{10,9}{⽔、⿃}
  \begin{phonetics}{海鸥}{hai3'ou1}
    \definition{s.}{gaivota}
  \end{phonetics}
\end{entry}

\begin{entry}{海浪}{10,10}{⽔、⽔}
  \begin{phonetics}{海浪}{hai3lang4}
    \definition{s.}{ondas do mar}
  \end{phonetics}
\end{entry}

\begin{entry}{海绵}{10,11}{⽔、⽷}
  \begin{phonetics}{海绵}{hai3mian2}
    \definition{s.}{(zoologia) esponja do mar | esponja (feita de poliéster ou celulose, etc.) | espuma de borracha}
  \end{phonetics}
\end{entry}

\begin{entry}{海棠}{10,12}{⽔、⽊}
  \begin{phonetics}{海棠}{hai3tang2}
    \definition{s.}{begônia}
  \end{phonetics}
\end{entry}

\begin{entry}{海鲜}{10,14}{⽔、⿂}
  \begin{phonetics}{海鲜}{hai3xian1}[][HSK 4]
    \definition[种,份,桌,批,些]{s.}{frutos do mar; mariscos; peixes marinhos frescos, camarões, etc., para consumo |}
  \end{phonetics}
\end{entry}

\begin{entry}{消化}{10,4}{⽔、⼔}
  \begin{phonetics}{消化}{xiao1hua4}[][HSK 4]
    \definition{v.}{digerir (alimentos) | digerir (conhecimento); pensar e absorver; uma metáfora para a compreensão total de novos conhecimentos ou informações e a capacidade de transformá-los em algo que possa ser usado}
  \end{phonetics}
\end{entry}

\begin{entry}{消失}{10,5}{⽔、⼤}
  \begin{phonetics}{消失}{xiao1shi1}[][HSK 3]
    \definition{v.}{desaparecer; desvanecer; dissolver; dissipar; evaporar; sumir}
  \end{phonetics}
\end{entry}

\begin{entry}{消防}{10,6}{⽔、⾩}
  \begin{phonetics}{消防}{xiao1fang2}
    \definition{s.}{combate a incêncios | controle de incêndios}
  \end{phonetics}
\end{entry}

\begin{entry}{消防员}{10,6,7}{⽔、⾩、⼝}
  \begin{phonetics}{消防员}{xiao1fang2yuan2}
    \definition{s.}{bombeiro}
  \end{phonetics}
\end{entry}

\begin{entry}{消费}{10,9}{⽔、⾙}
  \begin{phonetics}{消费}{xiao1fei4}[][HSK 3]
    \definition{v.}{gastar; consumir | consumir (recursos naturais)}
  \end{phonetics}
\end{entry}

\begin{entry}{消息}{10,10}{⽔、⼼}
  \begin{phonetics}{消息}{xiao1xi5}[][HSK 3]
    \definition[个,条,篇]{s.}{notícias; informação}
  \end{phonetics}
\end{entry}

\begin{entry}{涨价}{10,6}{⽔、⼈}
  \begin{phonetics}{涨价}{zhang3jia4}
    \definition{s.}{aumento de preços}
    \definition{v.+compl.}{avaliar (em valor) | dar preço | aumentar o preço}
  \end{phonetics}
\end{entry}

\begin{entry}{烈士}{10,3}{⽕、⼠}
  \begin{phonetics}{烈士}{lie4shi4}
    \definition{s.}{mártir}
  \end{phonetics}
\end{entry}

\begin{entry}{烟}{10}{⽕}
  \begin{phonetics}{烟}{yan1}[][HSK 3]
    \definition*[竜]{s.}{sobrenome Yan}
    \definition[根]{s.}{fumaça; gases produzidos quando substâncias queimam | névoa; vapor; algo como fumaça | tabaco; planta de tabaco | fumo; cigarro; termo geral para cigarros, charutos, etc. | ópio}
    \definition{v.}{ficar irritado com a fumaça (olhos)}
  \end{phonetics}
\end{entry}

\begin{entry}{烟火}{10,4}{⽕、⽕}
  \begin{phonetics}{烟火}{yan1huo3}
    \definition{s.}{fogo de artifício}
  \end{phonetics}
\end{entry}

\begin{entry}{烟叶}{10,5}{⽕、⼝}
  \begin{phonetics}{烟叶}{yan1ye4}
    \definition{s.}{folha de tabaco}
  \end{phonetics}
\end{entry}

\begin{entry}{烟头}{10,5}{⽕、⼤}
  \begin{phonetics}{烟头}{yan1tou2}
    \definition[根]{s.}{bituca de cigarro}
  \end{phonetics}
\end{entry}

\begin{entry}{烟囱}{10,7}{⽕、⼞}
  \begin{phonetics}{烟囱}{yan1cong1}
    \definition{s.}{chaminé}
  \end{phonetics}
\end{entry}

\begin{entry}{烟花}{10,7}{⽕、⾋}
  \begin{phonetics}{烟花}{yan1hua1}
    \definition{s.}{fogos de artifício}
  \end{phonetics}
\end{entry}

\begin{entry}{烟雨}{10,8}{⽕、⾬}
  \begin{phonetics}{烟雨}{yan1yu3}
    \definition{s.}{chuvisco | garoa}
  \end{phonetics}
\end{entry}

\begin{entry}{烟草}{10,9}{⽕、⾋}
  \begin{phonetics}{烟草}{yan1cao3}
    \definition{s.}{tabaco}
  \end{phonetics}
\end{entry}

\begin{entry}{烤}{10}{⽕}
  \begin{phonetics}{烤}{kao3}
    \definition{v.}{assar | grelhar}
  \end{phonetics}
\end{entry}

\begin{entry}{烤肉}{10,6}{⽕、⾁}
  \begin{phonetics}{烤肉}{kao3 rou4}[][HSK 5]
    \definition[块,串,片,盘]{s.}{churrasco (literalmente carne assada)}
  \end{phonetics}
\end{entry}

\begin{entry}{烤鸭}{10,10}{⽕、⿃}
  \begin{phonetics}{烤鸭}{kao3ya1}[][HSK 5]
    \definition{s.}{pato assado; pato recheado e assado em um forno especial após ser abatido}
  \end{phonetics}
\end{entry}

\begin{entry}{烦}{10}{⽕}
  \begin{phonetics}{烦}{fan2}[][HSK 4]
    \definition{adj.}{(estar) cansado; (estar) aborrecido; irritado; incomodado | supérfluo e confuso, bagunçado}
    \definition{s.}{problema; incômodo}
    \definition{v.}{incomodar; solicitar; colocar alguém em dificuldade (para fazer algo)}
  \end{phonetics}
\end{entry}

\begin{entry}{烧}{10}{⽕}
  \begin{phonetics}{烧}{shao1}[][HSK 4]
    \definition[次]{s.}{febre; temperatura corporal mais alta do que o normal}
    \definition{v.}{queimar; pegar fogo | cozinhar; aquecer; assar | guisar depois de fritar ou fritar depois de guisar | assar; grelhar os ingredientes dos alimentos diretamente sobre o fogo | ter febre; estar com febre | danificar (matar ou murchar) as plantas pelo uso excessivo (ou inadequado) de fertilizantes | tornar-se arrogante ou presunçoso; metáfora de estar em uma boa posição e se deixar levar}
  \end{phonetics}
\end{entry}

\begin{entry}{烧烤}{10,10}{⽕、⽕}
  \begin{phonetics}{烧烤}{shao1kao3}
    \definition{s.}{churrasco}
    \definition{v.}{assar}
  \end{phonetics}
\end{entry}

\begin{entry}{热}{10}{⽕}
  \begin{phonetics}{热}{re4}[][HSK 1]
    \definition{adj.}{quente (clima) | fervente | ardente | fervoroso}
    \definition{v.}{aquecer | ferver}
  \end{phonetics}
\end{entry}

\begin{entry}{热门}{10,3}{⽕、⾨}
  \begin{phonetics}{热门}{re4men2}[][HSK 5]
    \definition{adj.}{popular}
    \definition{s.}{algo que desperta o interesse popular; metáfora para algo que está na moda e recebe a atenção de todos (em contraste com ``冷门'').}
  \seealsoref{冷门}{leng3men2}
  \end{phonetics}
\end{entry}

\begin{entry}{热心}{10,4}{⽕、⼼}
  \begin{phonetics}{热心}{re4xin1}[][HSK 4]
    \definition{adj.}{ardente; sincero; entusiasmado; afetuoso; apaixonado; interessado}
    \definition{v.}{ser entusiasmado com alguma coisa}
  \end{phonetics}
\end{entry}

\begin{entry}{热血沸腾}{10,6,8,13}{⽕、⾎、⽔、⾁}
  \begin{phonetics}{热血沸腾}{re4xue4fei4teng2}
    \definition{expr.}{ferver o sangue | apaixonar-se}
  \end{phonetics}
\end{entry}

\begin{entry}{热泪盈眶}{10,8,9,11}{⽕、⽔、⽫、⽬}
  \begin{phonetics}{热泪盈眶}{re4lei4ying2kuang4}
    \definition{expr.}{olhos cheios de lágrimas de emoção | extremamente emocionado}
  \end{phonetics}
\end{entry}

\begin{entry}{热闹}{10,8}{⽕、⾾}
  \begin{phonetics}{热闹}{re4nao5}[][HSK 4]
    \definition{adj.}{animado; agitado; movimentado com barulho e excitação; descreve uma cena animada com uma atmosfera calorosa}
    \definition{s.}{uma vista emocionante; uma cena de agitação e excitação; atmosfera acolhedora}
    \definition{v.}{animar; divertir-se}
  \end{phonetics}
\end{entry}

\begin{entry}{热烈}{10,10}{⽕、⽕}
  \begin{phonetics}{热烈}{re4lie4}[][HSK 3]
    \definition{adj.}{caloroso; calorosamente; fervoroso; ardente; entusiasmado}
  \end{phonetics}
\end{entry}

\begin{entry}{热爱}{10,10}{⽕、⽖}
  \begin{phonetics}{热爱}{re4'ai4}[][HSK 3]
    \definition{v.}{amar ardentemente; amar de coração; ter amor profundo por}
  \end{phonetics}
\end{entry}

\begin{entry}{热情}{10,11}{⽕、⼼}
  \begin{phonetics}{热情}{re4qing2}[][HSK 2]
    \definition{adj.}{caloroso | fervoroso | entusiasmado}
    \definition{s.}{entusiasmo | ardor | devoção | calor | zelo}
  \end{phonetics}
\end{entry}

\begin{entry}{热量}{10,12}{⽕、⾥}
  \begin{phonetics}{热量}{re4 liang4}[][HSK 5]
    \definition{s.}{calor; quantidade de calor; calorias; em física, refere-se à energia transferida entre objetos com temperaturas diferentes, do objeto com temperatura mais alta para o objeto com temperatura mais baixa}
  \end{phonetics}
\end{entry}

\begin{entry}{爱}{10}{⽖}
  \begin{phonetics}{爱}{ai4}[][HSK 1]
    \definition*{s.}{sobrenome Ai}
    \definition[个]{s.}{amor; afeição; afeição profunda; preocupação profunda; especialmente amor entre pessoas}[爱是理解和包容。(O amor é compreensão e tolerância.)]
    \definition{v.}{amar; ter sentimentos profundos por pessoas ou coisas | gostar; gostar de; estar interessado em |  cuidar; valorizar; ter em alta estima; cuidar bem de | estar apto a; ter o hábito de}[他们深深爱着对方。(Eles se amam profundamente.) | 我爱我的家人。(Eu amo minha família.) | 我爱旅行。(Eu adoro viajar.)]
  \end{phonetics}
\end{entry}

\begin{entry}{爱人}{10,2}{⽖、⼈}
  \begin{phonetics}{爱人}{ai4ren5}[][HSK 2]
    \definition[个]{s.}{amante; amado | marido ou esposa; usado principalmente em situações formais}[这是我的爱人。(Este é o meu/minha esposo/companheiro.) | 她是我一生的爱人。(Ela é o amor da minha vida.) | 请携带爱人出席晚宴。(Por favor, traga seu cônjuge para o jantar.)]
  \end{phonetics}
\end{entry}

\begin{entry}{爱上}{10,3}{⽖、⼀}
  \begin{phonetics}{爱上}{ai4shang4}
    \definition{v.}{perder o coração por; apaixonar-se por}[他在旅行时爱上了一位法国女孩。(Ele se apaixonou por uma garota francesa durante a viagem.)  | 来到杭州后,我爱上了龙井茶。(Depois de chegar em Hangzhou, me apaixonei pelo chá Longjing.) | 我从来没想过自己会爱上健身。(Eu nunca imaginei que iria me apaixonar por academia.)]
  \end{phonetics}
\end{entry}

\begin{entry}{爱心}{10,4}{⽖、⼼}
  \begin{phonetics}{爱心}{ai4xin1}[][HSK 3]
    \definition[片]{s.}{amor | cuidado | compaixão}
  \end{phonetics}
\end{entry}

\begin{entry}{爱好}{10,6}{⽖、⼥}
  \begin{phonetics}{爱好}{ai4 hao4}[][HSK 1]
    \definition[个,种]{s.}{passatempo; interesse; \emph{hobby}; sentimentos de interesse especial ou afeição por algo | ``爱好'' é mais usado para atividades regulares (esportes, música), enquanto ``喜欢'' é para preferências gerais.}[他的爱好是收集邮票。(Seu hobby era colecionar selos.)  | 我的爱好是读书和旅行。(Meus hobbies são ler e viajar.)]
    \definition{v.}{estar interessado em; ter prazer em; ter um forte interesse em algo; ter sentimentos profundos por alguém ou algo}
  \seealsoref{喜欢}{xi3huan5}
  \end{phonetics}
\end{entry}

\begin{entry}{爱好者}{10,6,8}{⽖、⼥、⽼}
  \begin{phonetics}{爱好者}{ai4 hao4 zhe3}
    \definition{s.}{hobbista; amador; entusiasta; fã; amante (de arte, esportes, etc.)}[他是一位摄影爱好者。(Ele é um entusiasta de fotografia.) | 她是位潜水爱好者,经常去东南亚潜水。(Ela é uma mergulhadora amadora e frequentemente mergulha no Sudeste Asiático.)  | 我们为书法爱好者创建了一个微信群。(Criamos um grupo no WeChat para amantes de caligrafia.)]
  \end{phonetics}
\end{entry}

\begin{entry}{爱抚}{10,7}{⽖、⼿}
  \begin{phonetics}{爱抚}{ai4fu3}
    \definition{v.}{acariciar; afagar; cuidar (com ternura)}[他轻轻爱抚她的头发。(Ele afagou suavemente o cabelo dela.) | 母亲爱抚婴儿的脸颊。(A mãe acaricia a bochecha do bebê.) | 她爱抚着小猫的耳朵。(Ela acariciou as orelhas do gatinho.)]
  \end{phonetics}
\end{entry}

\begin{entry}{爱护}{10,7}{⽖、⼿}
  \begin{phonetics}{爱护}{ai4hu4}[][HSK 4]
    \definition{v.}{acalentar; valorizar; salvaguardar; cuidar bem de}[全社会都应爱护老年人。(Toda a sociedade deve tratar os idosos com cuidado e respeito.) | 请爱护公园里的小动物。(Por favor, tratem os animais do parque com cuidado.)]
  \end{phonetics}
\end{entry}

\begin{entry}{爱国}{10,8}{⽖、⼞}
  \begin{phonetics}{爱国}{ai4 guo2}[][HSK 4]
    \definition{adj.}{patriótico; patriotismo}[爱国是每个公民的责任。(O patriotismo é o dever de todo cidadão.) | 这部电影讲述了英雄的爱国故事。(Este filme conta a história patriótica de um herói.)]
    \definition{v.}{ser patriota; amar o seu país}
  \end{phonetics}
\end{entry}

\begin{entry}{爱爱}{10,10}{⽖、⽖}
  \begin{phonetics}{爱爱}{ai4'ai5}
    \definition{v.}{(coloquial) fazer amor ou relações íntimas | pode ser usado como um apelido entre casais, transmitindo ternura | pode soar vulgar se usado em contextos inadequados}[他们俩刚结婚,天天都想爱爱。(Eles acabaram de se casar e querem fazer amor todo dia.) | 爱爱,你今天好漂亮!(Amor, você está linda hoje!)]
  \end{phonetics}
\end{entry}

\begin{entry}{爱情}{10,11}{⽖、⼼}
  \begin{phonetics}{爱情}{ai4qing2}[][HSK 2]
    \definition{s.}{amor (entre pessoas); afeição}[爱情是盲目的。(O amor é cego.) | 爱情如同玫瑰,美丽却带刺。(O amor é como uma rosa, bela mas com espinhos.)  | 这首歌讲述了破碎的爱情故事。(Esta música conta uma história de amor fracassado.)]
  \end{phonetics}
\end{entry}

\begin{entry}{特价}{10,6}{⽜、⼈}
  \begin{phonetics}{特价}{te4 jia4}[][HSK 4]
    \definition{s.}{oferta especial; preço de barganha; preço especial reduzido}
  \end{phonetics}
\end{entry}

\begin{entry}{特地}{10,6}{⽜、⼟}
  \begin{phonetics}{特地}{te4di4}
    \definition{adv.}{especialmente | propositalmente}
  \end{phonetics}
\end{entry}

\begin{entry}{特有}{10,6}{⽜、⽉}
  \begin{phonetics}{特有}{te4 you3}[][HSK 5]
    \definition{adj.}{específico; peculiar; característico; único; exclusivo; especial}
  \end{phonetics}
\end{entry}

\begin{entry}{特色}{10,6}{⽜、⾊}
  \begin{phonetics}{特色}{te4se4}[][HSK 3]
    \definition{s.}{característica; característica distintiva | a cor única, estilo, etc. de um objeto}
  \end{phonetics}
\end{entry}

\begin{entry}{特别}{10,7}{⽜、⼑}
  \begin{phonetics}{特别}{te4bie2}[][HSK 2]
    \definition{adj.}{especial | paricular | incomum}
    \definition{adv.}{especialmente | particularmente | propositalmente}
  \end{phonetics}
\end{entry}

\begin{entry}{特技}{10,7}{⽜、⼿}
  \begin{phonetics}{特技}{te4ji4}
    \definition{s.}{efeito especial | dublê}
  \end{phonetics}
\end{entry}

\begin{entry}{特定}{10,8}{⽜、⼧}
  \begin{phonetics}{特定}{te4ding4}[][HSK 5]
    \definition{adj.}{específico; especialmente designado | dado; especificado; específico (pessoa, hora, lugar, local, ambiente, etc.)}
  \end{phonetics}
\end{entry}

\begin{entry}{特征}{10,8}{⽜、⼻}
  \begin{phonetics}{特征}{te4zheng1}[][HSK 4]
    \definition[个,种]{s.}{característica; aparência ou o fenômeno característico de uma pessoa ou coisa que pode ser visto de fora}
  \end{phonetics}
\end{entry}

\begin{entry}{特性}{10,8}{⽜、⼼}
  \begin{phonetics}{特性}{te4 xing4}[][HSK 5]
    \definition[个]{s.}{propriedade específica (ou característica) | característica; sabores | propriedade}
  \end{phonetics}
\end{entry}

\begin{entry}{特点}{10,9}{⽜、⽕}
  \begin{phonetics}{特点}{te4dian3}[][HSK 2]
    \definition[个]{s.}{característica | peculiaridade | característica distintiva}
  \end{phonetics}
\end{entry}

\begin{entry}{特殊}{10,10}{⽜、⽍}
  \begin{phonetics}{特殊}{te4shu1}[][HSK 4]
    \definition{adj.}{especial; particular; peculiar; excepcional; incomum}
  \end{phonetics}
\end{entry}

\begin{entry}{牺牲}{10,9}{⽜、⽜}
  \begin{phonetics}{牺牲}{xi1sheng1}
    \definition{s.}{abate de um animal como sacrifício}
    \definition{v.}{sacrificar a vida de alguém | sacrificar (algo de valor)}
  \end{phonetics}
\end{entry}

\begin{entry}{猃狁}{10,7}{⽝、⽝}
  \begin{phonetics}{猃狁}{xian3yun3}
    \definition*{s.}{Termo da dinastia Zhou para uma tribo nômade do norte mais tarde chamou o Xiongnu (匈奴) nas dinastias Qin e Han}
  \seealsoref{匈奴}{xiong1nu2}
  \end{phonetics}
\end{entry}

\begin{entry}{珠子}{10,3}{⽟、⼦}
  \begin{phonetics}{珠子}{zhu1zi5}
    \definition[粒,颗]{s.}{pérola | contas}
  \end{phonetics}
\end{entry}

\begin{entry}{班}{10}{⽟}
  \begin{phonetics}{班}{ban1}[][HSK 1]
    \definition*{s.}{sobrenome Ban}
    \definition{clas.}{para grupos}
    \definition[个]{s.}{equipe| time | esquadrão | turno de trabalho | classificação}
  \end{phonetics}
\end{entry}

\begin{entry}{班长}{10,4}{⽟、⾧}
  \begin{phonetics}{班长}{ban1 zhang3}[][HSK 2]
    \definition[个]{s.}{monitor de classe | líder de equipe | líder de esquadrão}
  \end{phonetics}
\end{entry}

\begin{entry}{班级}{10,6}{⽟、⽷}
  \begin{phonetics}{班级}{ban1 ji2}[][HSK 3]
    \definition[个]{s.}{classe | série (na escola)}
  \end{phonetics}
\end{entry}

\begin{entry}{瓶}{10}{⽡}
  \begin{phonetics}{瓶}{ping2}[][HSK 2]
    \definition{clas.}{para vinho ou líquidos}
    \definition[个]{s.}{garrafa | jarro| vaso}
  \end{phonetics}
\end{entry}

\begin{entry}{瓶子}{10,3}{⽡、⼦}
  \begin{phonetics}{瓶子}{ping2zi5}[][HSK 2]
    \definition[个]{s.}{garrafa}
  \end{phonetics}
\end{entry}

\begin{entry}{瓶盖}{10,11}{⽡、⽫}
  \begin{phonetics}{瓶盖}{ping2gai4}
    \definition{s.}{tampa de garrafa}
  \end{phonetics}
\end{entry}

\begin{entry}{瓶装}{10,12}{⽡、⾐}
  \begin{phonetics}{瓶装}{ping2zhuang1}
    \definition{adj.}{engarrafado}
  \end{phonetics}
\end{entry}

\begin{entry}{瓷}{10}{⽡}
  \begin{phonetics}{瓷}{ci2}
    \definition{s.}{artigos de porcelana}
  \end{phonetics}
\end{entry}

\begin{entry}{留}{10}{⽥}
  \begin{phonetics}{留}{liu2}[][HSK 2]
    \definition{v.}{permanecer | ficar | pedir para alguém ficar | manter alguém onde ele está | concentrar-se em | reservar | manter | salvar | deixar crescer | crescer | vestir | aceitar | tomar | deixar para trás | estudar no exterior}
  \end{phonetics}
\end{entry}

\begin{entry}{留下}{10,3}{⽥、⼀}
  \begin{phonetics}{留下}{liu2 xia4}[][HSK 2]
    \definition{v.}{deixar}
  \end{phonetics}
\end{entry}

\begin{entry}{留学}{10,8}{⽥、⼦}
  \begin{phonetics}{留学}{liu2xue2}[][HSK 3]
    \definition{v.}{estudar no exterior}
  \end{phonetics}
\end{entry}

\begin{entry}{留学生}{10,8,5}{⽥、⼦、⽣}
  \begin{phonetics}{留学生}{liu2 xue2 sheng1}[][HSK 2]
    \definition[个,位,名,批]{s.}{estudante estrangeiro | estudante estudando no exterior}
  \end{phonetics}
\end{entry}

\begin{entry}{留神}{10,9}{⽥、⽰}
  \begin{phonetics}{留神}{liu2shen2}
    \definition{v.+compl.}{tomar cuidado | prestar atenção | manter os olhos abertos}
  \end{phonetics}
\end{entry}

\begin{entry}{畜}{10}{⽥}
  \begin{phonetics}{畜}{chu4}
    \definition{s.}{gado | animal domesticado | animal doméstico}
  \end{phonetics}
  \begin{phonetics}{畜}{xu4}
    \definition{v.}{criar (animais)}
  \end{phonetics}
\end{entry}

\begin{entry}{疼}{10}{⽧}
  \begin{phonetics}{疼}{teng2}[][HSK 2]
    \definition{adj.}{dolorido | doído}
    \definition{v.}{doer | amar ternamente}
  \end{phonetics}
\end{entry}

\begin{entry}{病}{10}{⽧}
  \begin{phonetics}{病}{bing4}[][HSK 1]
    \definition[场]{s.}{doença}
    \definition{v.}{adoecer | estar doente}
  \end{phonetics}
\end{entry}

\begin{entry}{病人}{10,2}{⽧、⼈}
  \begin{phonetics}{病人}{bing4 ren2}[][HSK 1]
    \definition{s.}{doente | paciente}
  \end{phonetics}
\end{entry}

\begin{entry}{病毒}{10,9}{⽧、⽏}
  \begin{phonetics}{病毒}{bing4du2}[][HSK 5]
    \definition[种,株,类]{s.}{vírus; patógenos que são menores que os germes e visíveis somente com um microscópio eletrônico | vírus de computador}
  \end{phonetics}
\end{entry}

\begin{entry}{盏}{10}{⽫}
  \begin{phonetics}{盏}{zhan3}
    \definition{clas.}{para lâmpadas}
    \definition{s.}{copo pequeno}
  \end{phonetics}
\end{entry}

\begin{entry}{盐}{10}{⽫}
  \begin{phonetics}{盐}{yan2}[][HSK 4]
    \definition{s.}{sal; sais}
    \definition{s.}{sobrenome Yan}
  \end{phonetics}
\end{entry}

\begin{entry}{监狱}{10,9}{⽫、⽝}
  \begin{phonetics}{监狱}{jian1yu4}
    \definition{s.}{prisão}
  \end{phonetics}
\end{entry}

\begin{entry}{眞}{10}{⽬}
  \begin{phonetics}{眞}{zhen1}
    \variantof{真}
  \end{phonetics}
\end{entry}

\begin{entry}{真}{10}{⼗}
  \begin{phonetics}{真}{zhen1}[][HSK 1]
    \definition{adj.}{genuíno}
    \definition{adv.}{que\dots tão\dots! | realmente}
  \end{phonetics}
\end{entry}

\begin{entry}{真切}{10,4}{⼗、⼑}
  \begin{phonetics}{真切}{zhen1qie4}
    \definition{adj.}{claro | distinto | honesto | sincero | vívido}
  \end{phonetics}
\end{entry}

\begin{entry}{真心}{10,4}{⼗、⼼}
  \begin{phonetics}{真心}{zhen1xin1}
    \definition{adj.}{sincero}
    \definition[片]{s.}{sinceridade}
  \end{phonetics}
\end{entry}

\begin{entry}{真牛}{10,4}{⼗、⽜}
  \begin{phonetics}{真牛}{zhen1niu2}
    \definition{adj.}{(gíria) muito legal, incrível}
  \end{phonetics}
\end{entry}

\begin{entry}{真正}{10,5}{⼗、⽌}
  \begin{phonetics}{真正}{zhen1zheng4}[][HSK 2]
    \definition{adj.}{verdadeiro | real | genuíno}
    \definition{adv.}{realmente | de ​​fato}
  \end{phonetics}
\end{entry}

\begin{entry}{真声}{10,7}{⼗、⼠}
  \begin{phonetics}{真声}{zhen1sheng1}
    \definition{s.}{voz natural | voz verdadeira}
    \seeref{假声}{jia3sheng1}
  \end{phonetics}
\end{entry}

\begin{entry}{真实}{10,8}{⼗、⼧}
  \begin{phonetics}{真实}{zhen1shi2}[][HSK 3]
    \definition{adj.}{verdadeiro; real; autêntico}
  \end{phonetics}
\end{entry}

\begin{entry}{真的}{10,8}{⼗、⽩}
  \begin{phonetics}{真的}{zhen1 de5}[][HSK 1]
    \definition{adv.}{realmente | verdadeiramente}
  \end{phonetics}
\end{entry}

\begin{entry}{真珠}{10,10}{⼗、⽟}
  \begin{phonetics}{真珠}{zhen1zhu1}
    \variantof{珍珠}
  \end{phonetics}
\end{entry}

\begin{entry}{真真}{10,10}{⼗、⼗}
  \begin{phonetics}{真真}{zhen1zhen1}
    \definition{adv.}{genuinamente | realmente | escrupulosamente}
  \end{phonetics}
\end{entry}

\begin{entry}{真理}{10,11}{⼗、⽟}
  \begin{phonetics}{真理}{zhen1li3}
    \definition[个]{s.}{verdade}
  \end{phonetics}
\end{entry}

\begin{entry}{真释}{10,12}{⼗、⾤}
  \begin{phonetics}{真释}{zhen1shi4}
    \definition{s.}{razão genuína | explicação verdadeira}
  \end{phonetics}
\end{entry}

\begin{entry}{破}{10}{⽯}
  \begin{phonetics}{破}{po4}[][HSK 3]
    \definition{adj.}{quebrado; danificado; rasgado; desgastado | pobre; ruim; insignificante; péssimo; miserável}
    \definition{v.}{estar quebrado; estar danificado | quebrar; avariar; danificar | quebrar; dividir; cortar; cinzelar | trocar (dinheiro) | romper; quebrar (avanço) | livrar-se de; destruir; romper com
derrotar; capturar (uma cidade, etc.) | despender; gastar (dinheiro) | expor a verdade de; desnudar}
  \end{phonetics}
\end{entry}

\begin{entry}{破产}{10,6}{⽯、⼇}
  \begin{phonetics}{破产}{po4chan3}[][HSK 4]
    \definition{v.+compl.}{falir; ir à falência; tornar-se insolvente; entrar em liquidação; perder todo o patrimônio | falhar; fracassar; não dar em nada; figura de linguagem (geralmente com uma conotação depreciativa)}
  \end{phonetics}
\end{entry}

\begin{entry}{破坏}{10,7}{⽯、⼟}
  \begin{phonetics}{破坏}{po4huai4}[][HSK 3]
    \definition{s.}{destruição | dano}
    \definition{v.}{demolir; naufragar; soçobrar; destruir; obliterar | quebrar; violar (um acordo, regulamento, etc.) | prejudicar; perturbar; sabotar; causar grande dano | reverter; mudar (um sistema social, costume, etc.) completamente ou violentamente | destruir; decompor}
  \end{phonetics}
\end{entry}

\begin{entry}{破坏性}{10,7,8}{⽯、⼟、⼼}
  \begin{phonetics}{破坏性}{po4huai4xing4}
    \definition{adj.}{destrutivo}
    \definition{s.}{poder destrutivo}
  \end{phonetics}
\end{entry}

\begin{entry}{砸}{10}{⽯}
  \begin{phonetics}{砸}{za2}
    \definition{v.}{esmagar | bater | falhar | estragar}
  \end{phonetics}
\end{entry}

\begin{entry}{离}{10}{⼇}
  \begin{phonetics}{离}{li2}[][HSK 2]
    \definition*{s.}{sobrenome Li}
    \definition{prep.}{(ser longe) de\dots até\dots}
    \definition{v.}{ficar longe de | deixar | separar-se de}
  \end{phonetics}
\end{entry}

\begin{entry}{离不开}{10,4,4}{⼇、⼀、⼶}
  \begin{phonetics}{离不开}{li2 bu4 kai1}[][HSK 4]
    \definition{v.}{não pode prescindir; ser inseparável de; não ser capaz de se separar ou deixar uma pessoa, coisa ou circunstância}
  \end{phonetics}
\end{entry}

\begin{entry}{离开}{10,4}{⼇、⼶}
  \begin{phonetics}{离开}{li2kai1}[][HSK 2]
    \definition{v.}{partir| deixar}
  \end{phonetics}
\end{entry}

\begin{entry}{离婚}{10,11}{⼇、⼥}
  \begin{phonetics}{离婚}{li2hun1}[][HSK 3]
    \definition{v.+compl.}{divórciar; romper um casamento; obter o divórcio}
  \end{phonetics}
\end{entry}

\begin{entry}{秘书}{10,4}{⽲、⼄}
  \begin{phonetics}{秘书}{mi4shu1}[][HSK 4]
    \definition[个,位,名]{s.}{o cargo de secretário; funções de secretariado | secretário; pessoas encarregadas da correspondência e que auxiliam o chefe do órgão ou departamento na condução diária de seu trabalho}
  \end{phonetics}
\end{entry}

\begin{entry}{秘密}{10,11}{⽲、⼧}
  \begin{phonetics}{秘密}{mi4mi4}[][HSK 4]
    \definition{adj.}{secreto}
    \definition[个]{s.}{segredo; algo secreto; coisas que você não quer que as pessoas saibam}
  \end{phonetics}
\end{entry}

\begin{entry}{租}{10}{⽲}
  \begin{phonetics}{租}{zu1}[][HSK 2]
    \definition{s.}{imposto sobre propriedade urbana ou rural}
    \definition{v.}{alugar | tomar de aluguel}
  \end{phonetics}
\end{entry}

\begin{entry}{租用}{10,5}{⽲、⽤}
  \begin{phonetics}{租用}{zu1yong4}
    \definition{v.}{contratar | alugar | alugar (algo de alguém)}
  \end{phonetics}
\end{entry}

\begin{entry}{租让}{10,5}{⽲、⾔}
  \begin{phonetics}{租让}{zu1rang4}
    \definition{v.}{alugar | alugar (a propriedade de alguém para outra pessoa)}
  \end{phonetics}
\end{entry}

\begin{entry}{租约}{10,6}{⽲、⽷}
  \begin{phonetics}{租约}{zu1yue1}
    \definition{s.}{aluguel}
  \end{phonetics}
\end{entry}

\begin{entry}{租房}{10,8}{⽲、⼾}
  \begin{phonetics}{租房}{zu1fang2}
    \definition{v.}{alugar um apartamento}
  \end{phonetics}
\end{entry}

\begin{entry}{租金}{10,8}{⽲、⾦}
  \begin{phonetics}{租金}{zu1jin1}
    \definition{s.}{aluguel}
    \seeref{租钱}{zu1qian5}
  \end{phonetics}
\end{entry}

\begin{entry}{租赁}{10,10}{⽲、⾙}
  \begin{phonetics}{租赁}{zu1lin4}
    \definition{v.}{contratar | alugar}
  \end{phonetics}
\end{entry}

\begin{entry}{租钱}{10,10}{⽲、⾦}
  \begin{phonetics}{租钱}{zu1qian5}
    \definition{s.}{aluguel}
    \seeref{租金}{zu1jin1}
  \end{phonetics}
\end{entry}

\begin{entry}{租船}{10,11}{⽲、⾈}
  \begin{phonetics}{租船}{zu1chuan2}
    \definition{v.}{fretar um navio | alugar um navio}
  \end{phonetics}
\end{entry}

\begin{entry}{积木}{10,4}{⽲、⽊}
  \begin{phonetics}{积木}{ji1mu4}
    \definition{s.}{blocos de montar (brinquedo)}
  \end{phonetics}
\end{entry}

\begin{entry}{积极}{10,7}{⽲、⽊}
  \begin{phonetics}{积极}{ji1ji2}[][HSK 3]
    \definition{adj.}{ativo | positivo}
  \end{phonetics}
\end{entry}

\begin{entry}{积累}{10,11}{⽲、⽷}
  \begin{phonetics}{积累}{ji1lei3}[][HSK 4]
    \definition{s.}{acúmulo; acumulação}
    \definition{v.}{acumular}
  \end{phonetics}
\end{entry}

\begin{entry}{称}{10}{⽲}
  \begin{phonetics}{称}{chen4}
    \definition{adj.}{ajustado; encaixado}
    \definition{v.}{ajustar; adequar; combinar | ter; possuir}
  \end{phonetics}
  \begin{phonetics}{称}{cheng1}[][HSK 2,5]
    \definition*{s.}{sobrenome Cheng}
    \definition{s.}{nome}
    \definition{v.}{chamar |
dizer; declarar |
elogiar; louvar; usar palavras para expressar afirmação ou elogio a pessoas ou coisas |
pesar; medir o peso |
elevar; levantar; erguer |
aplaudir; concordar; expressar a opinião ou os sentimentos em palavras ou gestos |
declarar-se como; declarar que é; se apresentar como poderoso ou se fazer passar por tal |}
  \end{phonetics}
\end{entry}

\begin{entry}{称为}{10,4}{⽲、⼂}
  \begin{phonetics}{称为}{cheng1 wei2}[][HSK 3]
    \definition{v.}{chamar; ser chamado; ser conhecido como}
  \end{phonetics}
\end{entry}

\begin{entry}{称号}{10,5}{⽲、⼝}
  \begin{phonetics}{称号}{cheng1hao4}[][HSK 5]
    \definition{s.}{título; nome; designação; nome dado a alguém, a uma organização ou a alguma coisa (geralmente usado de forma honrosa)}
  \end{phonetics}
\end{entry}

\begin{entry}{称赞}{10,16}{⽲、⾙}
  \begin{phonetics}{称赞}{cheng1zan4}[][HSK 4]
    \definition[句,声,番,次]{s.}{elogio; aclamação; louvor; avaliação positiva de um desempenho ou conquista}
    \definition{v.}{elogiar; aclamar; louvar; usar palavras para expressar um carinho pelas virtudes de uma pessoa ou coisa}
  \end{phonetics}
\end{entry}

\begin{entry}{站}{10}{⽴}
  \begin{phonetics}{站}{zhan4}[][HSK 1]
    \definition{s.}{estação | ponto | parada}
  \end{phonetics}
\end{entry}

\begin{entry}{站长}{10,4}{⽴、⾧}
  \begin{phonetics}{站长}{zhan4zhang3}
    \definition{s.}{pessoa responsável pela estação de trem | chefe da estação | \emph{webmaster} | gerente de centro de voluntariado}
  \end{phonetics}
\end{entry}

\begin{entry}{站台}{10,5}{⽴、⼝}
  \begin{phonetics}{站台}{zhan4tai2}
    \definition{s.}{plataforma (em uma estação ferroviária)}
  \end{phonetics}
\end{entry}

\begin{entry}{站住}{10,7}{⽴、⼈}
  \begin{phonetics}{站住}{zhan4 zhu4}[][HSK 2]
    \definition{v.}{parar | deter | ficar firme em pé | manter os pés firmes | manter a própria posição | consolidar a própria posição | reter água | ser sustentável}
  \end{phonetics}
\end{entry}

\begin{entry}{站姿}{10,9}{⽴、⼥}
  \begin{phonetics}{站姿}{zhan4zi1}
    \definition{s.}{postura}
  \end{phonetics}
\end{entry}

\begin{entry}{站点}{10,9}{⽴、⽕}
  \begin{phonetics}{站点}{zhan4dian3}
    \definition{s.}{\emph{website}}
  \end{phonetics}
\end{entry}

\begin{entry}{竞争}{10,6}{⽴、⼑}
  \begin{phonetics}{竞争}{jing4zheng1}[][HSK 5]
    \definition{v.}{competir; disputar; lutar; entre duas ou mais partes; em prol de seus próprios interesses; lutar pela vitória por meio de uma disputa de sua própria força contra outra}
  \end{phonetics}
\end{entry}

\begin{entry}{竞赛}{10,14}{⽴、⾙}
  \begin{phonetics}{竞赛}{jing4sai4}[][HSK 5]
    \definition[个]{s.}{concurso; competição; partida; corrida}
    \definition{v.}{correr; competir; competir uns com os outros por superioridade; em esportes, produção e outras atividades, para comparar competência, habilidade etc., usado principalmente na linguagem falada}
  \end{phonetics}
\end{entry}

\begin{entry}{笋}{10}{⽵}
  \begin{phonetics}{笋}{sun3}
    \definition{s.}{broto de bambu}
  \end{phonetics}
\end{entry}

\begin{entry}{笑}{10}{⽵}
  \begin{phonetics}{笑}{xiao4}[][HSK 1]
    \definition{v.}{sorrir | rir | rir de}
  \end{phonetics}
\end{entry}

\begin{entry}{笑话}{10,8}{⽵、⾔}
  \begin{phonetics}{笑话}{xiao4hua5}[][HSK 2]
    \definition{adj.}{absurdo | ridículo}
    \definition[个]{s.}{piada | brincadeira}
    \definition{v.}{rir de algo | zombar | ridicularizar}
  \end{phonetics}
\end{entry}

\begin{entry}{笑话儿}{10,8,2}{⽵、⾔、⼉}
  \begin{phonetics}{笑话儿}{xiao4 hua4r5}[][HSK 2]
    \definition{s.}{piada | gracejo}
  \end{phonetics}
\end{entry}

\begin{entry}{笑容}{10,10}{⽵、⼧}
  \begin{phonetics}{笑容}{xiao4rong2}
    \definition[副]{s.}{sorriso | expressão sorridente}
  \end{phonetics}
\end{entry}

\begin{entry}{笔}{10}{⽵}
  \begin{phonetics}{笔}{bi3}[][HSK 2]
    \definition{clas.}{para somas de dinheiro, negócios}
    \definition[支,枝]{s.}{caneta | lápis}
  \end{phonetics}
\end{entry}

\begin{entry}{笔记}{10,5}{⽵、⾔}
  \begin{phonetics}{笔记}{bi3 ji4}[][HSK 2]
    \definition[篇,本,个]{s.}{notas | ensaios | esboços}
    \definition{v.}{tomar nota (por escrito)}
  \end{phonetics}
\end{entry}

\begin{entry}{笔记本}{10,5,5}{⽵、⾔、⽊}
  \begin{phonetics}{笔记本}{bi3ji4ben3}[][HSK 2]
    \definition[本]{s.}{caderno}
    \definition{s.}{\emph{laptop}}
  \end{phonetics}
\end{entry}

\begin{entry}{粉}{10}{⽶}
  \begin{phonetics}{粉}{fen3}
    \definition{s.}{pó | pó cosmético facial | alimento preparado a partir de amido | macarrão feito de qualquer tipo de farinha}
    \definition{v.}{tornar algo em pó | ser um fã de}
  \end{phonetics}
\end{entry}

\begin{entry}{粉丝}{10,5}{⽶、⼀}
  \begin{phonetics}{粉丝}{fen3si1}
    \definition{s.}{(empréstimo linguístico) fã | entusiasta de alguém ou alguma coisa}
    \definition[把]{s.}{aletria de amido de feijão | aletria chinesa | macarrão de celofane ou macarrão de vidro (transparente)}
  \end{phonetics}
\end{entry}

\begin{entry}{粉色}{10,6}{⽶、⾊}
  \begin{phonetics}{粉色}{fen3 se4}
    \definition{s.}{cor-de-rosa}
  \end{phonetics}
\end{entry}

\begin{entry}{索性}{10,8}{⽷、⼼}
  \begin{phonetics}{索性}{suo3xing4}
    \definition{adv.}{poderia muito bem | simplesmente | apenas}
  \end{phonetics}
\end{entry}

\begin{entry}{紧}{10}{⽷}
  \begin{phonetics}{紧}{jin3}[][HSK 3]
    \definition{adj.}{tenso; apertado | seguro; firme | cerrado; apertado | urgente; premente; tenso | rigoroso; rígido; severo | difícil; sem dinheiro}
    \definition{v.}{apertar}
  \end{phonetics}
\end{entry}

\begin{entry}{紧张}{10,7}{⽷、⼸}
  \begin{phonetics}{紧张}{jin3zhang1}[][HSK 3]
    \definition{adj.}{nervoso; tenso | apertado; em falta | tenso; intenso; coado}
  \end{phonetics}
\end{entry}

\begin{entry}{紧急}{10,9}{⽷、⼼}
  \begin{phonetics}{紧急}{jin3ji2}[][HSK 3]
    \definition{adj.}{urgente}
    \definition{adj.}{urgente; premente; crítico}
    \definition{s.}{emergência}
  \end{phonetics}
\end{entry}

\begin{entry}{紧紧}{10,10}{⽷、⽷}
  \begin{phonetics}{紧紧}{jin3 jin3}[][HSK 5]
    \definition{adv.}{firmemente; estreitamente; apertadamente; prestar muita atenção (em algo)}
  \end{phonetics}
\end{entry}

\begin{entry}{紧密}{10,11}{⽷、⼧}
  \begin{phonetics}{紧密}{jin3 mi4}[][HSK 4]
    \definition{adj.}{próximos; inseparáveis | incessante; rápido e intenso}
  \end{phonetics}
\end{entry}

\begin{entry}{绣}{10}{⽷}
  \begin{phonetics}{绣}{xiu4}
    \definition{s.}{bordado}
    \definition{v.}{bordar}
  \end{phonetics}
\end{entry}

\begin{entry}{继承}{10,8}{⽷、⼿}
  \begin{phonetics}{继承}{ji4cheng2}[][HSK 5]
    \definition{v.}{herdar (o patrimônio de uma pessoa falecida, etc.) de acordo com a lei | continuar; geralmente se refere à aceitação do estilo, da cultura, do conhecimento, etc., daqueles que nos precederam | continuar; os descendentes continuam o trabalho deixado por seus antecessores.}
  \end{phonetics}
\end{entry}

\begin{entry}{继续}{10,11}{⽷、⽷}
  \begin{phonetics}{继续}{ji4xu4}[][HSK 3]
    \definition{v.}{continuar; prosseguir}
  \end{phonetics}
\end{entry}

\begin{entry}{缺}{10}{⽸}
  \begin{phonetics}{缺}{que1}[][HSK 3]
    \definition{adj.}{incompleto; imperfeito}
    \definition[种]{s.}{vaga; abertura; falta}
    \definition{v.}{estar com falta de; faltar | estar faltando; estar incompleto | estar ausente}
  \end{phonetics}
\end{entry}

\begin{entry}{缺乏}{10,4}{⽸、⼃}
  \begin{phonetics}{缺乏}{que1fa2}[][HSK 5]
    \definition{v.}{faltar; estar em falta de; não ter ou não ter totalmente (algo que deveria possuir ou é desejaria possuir)}
  \end{phonetics}
\end{entry}

\begin{entry}{缺少}{10,4}{⽸、⼩}
  \begin{phonetics}{缺少}{que1shao3}[][HSK 3]
    \definition{v.}{falta; estar com falta de; estar em falta de}
  \end{phonetics}
\end{entry}

\begin{entry}{缺点}{10,9}{⽸、⽕}
  \begin{phonetics}{缺点}{que1dian3}[][HSK 3]
    \definition[个]{s.}{desvantagem; deficiência; inconveniência; ponto fraco}
  \end{phonetics}
\end{entry}

\begin{entry}{缺勤}{10,13}{⽸、⼒}
  \begin{phonetics}{缺勤}{que1qin2}
    \definition{v.+compl.}{ausentar-se do dever (trabalho)}
  \end{phonetics}
\end{entry}

\begin{entry}{罢}{10}{⽹}
  \begin{phonetics}{罢}{ba4}
    \definition{v.}{parar | cessar | demitir | suspender | desistir | terminar}
  \end{phonetics}
  \begin{phonetics}{罢}{ba5}
    \definition{part.}{partícula final, a mesma que 吧}
  \seealsoref{吧}{ba5}
  \end{phonetics}
\end{entry}

\begin{entry}{耕}{10}{⽾}
  \begin{phonetics}{耕}{geng1}
    \definition{v.}{lavrar; arar; cultivar | ganhar a vida; buscar o próprio sustento}
  \end{phonetics}
\end{entry}

\begin{entry}{耽心}{10,4}{⽿、⼼}
  \begin{phonetics}{耽心}{dan1xin1}
    \variantof{担心}
  \end{phonetics}
\end{entry}

\begin{entry}{胳肢窝}{10,8,12}{⾁、⾁、⽳}
  \begin{phonetics}{胳肢窝}{ga1 zhi1 wo1}
    \definition{s.}{axila; sovaco; também escrito ``夹肢窝''}
  \seealsoref{夹肢窝}{jia1 zhi1 wo1}
  \end{phonetics}
\end{entry}

\begin{entry}{胶水}{10,4}{⾁、⽔}
  \begin{phonetics}{胶水}{jiao1shui3}[][HSK 5]
    \definition[瓶]{s.}{cola; mucilagem; cola líquida}
  \end{phonetics}
\end{entry}

\begin{entry}{胶卷}{10,8}{⾁、⼙}
  \begin{phonetics}{胶卷}{jiao1juan3}
    \definition{s.}{filme | rolo de filme}
  \end{phonetics}
\end{entry}

\begin{entry}{胶带}{10,9}{⾁、⼱}
  \begin{phonetics}{胶带}{jiao1 dai4}[][HSK 5]
    \definition[卷]{s.}{fita adesiva | fita de gravação | correia de borracha}
  \end{phonetics}
\end{entry}

\begin{entry}{胸}{10}{⾁}
  \begin{phonetics}{胸}{xiong1}
    \definition{s.}{peito | tórax}
  \end{phonetics}
\end{entry}

\begin{entry}{胸部}{10,10}{⾁、⾢}
  \begin{phonetics}{胸部}{xiong1 bu4}[][HSK 4]
    \definition{s.}{peito; tórax; seios}
  \end{phonetics}
\end{entry}

\begin{entry}{能}{10}{⾁}
  \begin{phonetics}{能}{neng2}[][HSK 1]
    \definition*{s.}{sobrenome Neng}
    \definition{adv.}{talvez}
    \definition{s.}{(física)nenergia | habilidade}
    \definition{v.}{poder | ser capaz de}
  \end{phonetics}
\end{entry}

\begin{entry}{能力}{10,2}{⾁、⼒}
  \begin{phonetics}{能力}{neng2li4}[][HSK 3]
    \definition{s.}{habilidade; capacidade; aptidão}
  \end{phonetics}
\end{entry}

\begin{entry}{能上能下}{10,3,10,3}{⾁、⼀、⾁、⼀}
  \begin{phonetics}{能上能下}{neng2shang4neng2xia4}
    \definition{s.}{pronto para aceitar qualquer trabalho, alto ou baixo}
  \end{phonetics}
\end{entry}

\begin{entry}{能干}{10,3}{⾁、⼲}
  \begin{phonetics}{能干}{neng2gan4}[][HSK 4]
    \definition{adj.}{apto; capaz; competente}
  \end{phonetics}
\end{entry}

\begin{entry}{能不能}{10,4,10}{⾁、⼀、⾁}
  \begin{phonetics}{能不能}{neng2 bu4 neng2}[][HSK 3]
    \definition{adv.}{pode ou não pode\dots?}
  \end{phonetics}
\end{entry}

\begin{entry}{能够}{10,11}{⾁、⼣}
  \begin{phonetics}{能够}{neng2 gou4}[][HSK 2]
    \definition{v.}{ser capaz de}
  \end{phonetics}
\end{entry}

\begin{entry}{能量}{10,12}{⾁、⾥}
  \begin{phonetics}{能量}{neng2liang4}[][HSK 5]
    \definition[种]{s.}{energia; quantidade de energia; Uma grandeza física que mede a capacidade da matéria de realizar trabalho | capacidade; competências; capacidade e papel que uma pessoa pode desempenhar}
  \end{phonetics}
\end{entry}

\begin{entry}{脂麻}{10,11}{⾁、⿇}
  \begin{phonetics}{脂麻}{zhi1ma5}
    \variantof{芝麻}
  \end{phonetics}
\end{entry}

\begin{entry}{脆}{10}{⾁}
  \begin{phonetics}{脆}{cui4}[][HSK 5]
    \definition{adj.}{frágil; quebradiço | crocante | clara; nítida; (voz)| puro}
  \end{phonetics}
\end{entry}

\begin{entry}{脏}{10}{⾁}
  \begin{phonetics}{脏}{zang1}[][HSK 2]
    \definition{adj.}{sujo | imundo}
  \end{phonetics}
  \begin{phonetics}{脏}{zang4}
    \definition{s.}{órgão (anatomia) | víscera}
  \end{phonetics}
\end{entry}

\begin{entry}{脏土}{10,3}{⾁、⼟}
  \begin{phonetics}{脏土}{zang1tu3}
    \definition{s.}{solo sujo | lama | lixo}
  \end{phonetics}
\end{entry}

\begin{entry}{脏字}{10,6}{⾁、⼦}
  \begin{phonetics}{脏字}{zang1zi4}
    \definition{s.}{obscenidade}
  \end{phonetics}
\end{entry}

\begin{entry}{脏病}{10,10}{⾁、⽧}
  \begin{phonetics}{脏病}{zang1bing4}
    \definition{s.}{doença venérea}
  \end{phonetics}
\end{entry}

\begin{entry}{脏脏}{10,10}{⾁、⾁}
  \begin{phonetics}{脏脏}{zang1zang1}
    \definition{adj.}{sujo}
  \end{phonetics}
\end{entry}

\begin{entry}{脏煤}{10,13}{⾁、⽕}
  \begin{phonetics}{脏煤}{zang1mei2}
    \definition{s.}{carvão sujo | sujeira (de uma mina de carvão)}
  \end{phonetics}
\end{entry}

\begin{entry}{脏器}{10,16}{⾁、⼝}
  \begin{phonetics}{脏器}{zang4qi4}
    \definition{s.}{órgãos internos}
  \end{phonetics}
\end{entry}

\begin{entry}{脏辫}{10,17}{⾁、⾟}
  \begin{phonetics}{脏辫}{zang1bian4}
    \definition{s.}{\emph{dreadlocks}}
  \end{phonetics}
\end{entry}

\begin{entry}{脑子}{10,3}{⾁、⼦}
  \begin{phonetics}{脑子}{nao3 zi5}[][HSK 5]
    \definition[个]{s.}{cérebro | mente; cabeça; cérebro; inteligência; poder mental; refere-se à capacidade de pensar, memorizar, raciocinar, etc.; inteligência}
  \end{phonetics}
\end{entry}

\begin{entry}{脑瓜}{10,5}{⾁、⽠}
  \begin{phonetics}{脑瓜}{nao3gua1}
    \definition{s.}{crânio | cérebro | cabeça | mente | mentalidade | ideia}
  \seealsoref{脑瓜子}{nao3gua1zi5}
  \end{phonetics}
\end{entry}

\begin{entry}{脑瓜子}{10,5,3}{⾁、⽠、⼦}
  \begin{phonetics}{脑瓜子}{nao3gua1zi5}
    \definition{s.}{crânio | cérebro | cabeça | mente | mentalidade | ideia}
  \seealsoref{脑瓜}{nao3gua1}
  \end{phonetics}
\end{entry}

\begin{entry}{脑袋}{10,11}{⾁、⾐}
  \begin{phonetics}{脑袋}{nao3dai5}[][HSK 4]
    \definition[颗,个]{s.}{cabeça; a parte mais alta do corpo humano ou a parte mais alta de um animal que contém órgãos como a boca, o nariz, os olhos etc. | mente; cérebro; capacidade de pensar, lembrar, etc.}
  \end{phonetics}
\end{entry}

\begin{entry}{臭}{10}{⾃}
  \begin{phonetics}{臭}{chou4}[][HSK 5]
    \definition{adj.}{sujo; malcheiroso; fedorento; contrário de ``香'' | repugnante; nojento; repulsivo | ruim; pobre; péssimo}
    \definition{adv.}{severamente; firmemente}
    \definition{v.}{falhar em detonar (bala)}
  \seealsoref{香}{xiang1}
  \end{phonetics}
  \begin{phonetics}{臭}{xiu4}
    \definition{s.}{odor; cheiro;}
    \definition{v.}{cheirar; farejar; o mesmo que "嗅"}
  \seealsoref{嗅}{xiu4}
  \end{phonetics}
\end{entry}

\begin{entry}{臭气}{10,4}{⾃、⽓}
  \begin{phonetics}{臭气}{chou4qi4}
    \definition{s.}{fedor}
  \end{phonetics}
\end{entry}

\begin{entry}{致敬}{10,12}{⾄、⽁}
  \begin{phonetics}{致敬}{zhi4jing4}
    \definition{v.}{saudar | prestar respeitos a | prestar homenagem a}
  \end{phonetics}
\end{entry}

\begin{entry}{航天员}{10,4,7}{⾈、⼤、⼝}
  \begin{phonetics}{航天员}{hang2tian1yuan2}
    \definition{s.}{astronauta}
  \end{phonetics}
\end{entry}

\begin{entry}{航空}{10,8}{⾈、⽳}
  \begin{phonetics}{航空}{hang2kong1}[][HSK 4]
    \definition{s.}{viagem; aviação; refere-se ao voo de uma aeronave no ar}
  \end{phonetics}
\end{entry}

\begin{entry}{航班}{10,10}{⾈、⽟}
  \begin{phonetics}{航班}{hang2ban1}[][HSK 4]
    \definition[个,次]{s.}{número do voo; voo programado}
  \end{phonetics}
\end{entry}

\begin{entry}{般}{10}{⾈}
  \begin{phonetics}{般}{ban1}
    \definition{s.}{espécie | tipo | classe | caminho | maneira}
  \end{phonetics}
  \begin{phonetics}{般}{bo1}
    \definition{s.}{utilizado em 般若 \dpy{bo1re3}}
    \seeref{般若}{bo1re3}
  \end{phonetics}
  \begin{phonetics}{般}{pan2}
    \definition{s.}{utilizado em 般乐 \dpy{pan2le4}}
    \seeref{般乐}{pan2le4}
  \end{phonetics}
\end{entry}

\begin{entry}{般乐}{10,5}{⾈、⼃}
  \begin{phonetics}{般乐}{pan2le4}
    \definition{v.}{jogar | divertir-se}
  \end{phonetics}
\end{entry}

\begin{entry}{般若}{10,8}{⾈、⾋}
  \begin{phonetics}{般若}{bo1re3}
    \definition*{s.}{Prajna (sânscrito), \emph{insight} sobre a verdadeira natureza da realidade | (Budismo) sabedoria}
  \end{phonetics}
\end{entry}

\begin{entry}{舱}{10}{⾈}
  \begin{phonetics}{舱}{cang1}
    \definition{s.}{cabine | porão (de carga) de um navio ou avião}
  \end{phonetics}
\end{entry}

\begin{entry}{荷}{10}{⾋}
  \begin{phonetics}{荷}{he2}
    \definition{s.}{lótus}
  \end{phonetics}
  \begin{phonetics}{荷}{he4}
    \definition{s.}{carga | responsabilidade}
    \definition{v.}{carregar no ombro ou nas costas}
  \end{phonetics}
\end{entry}

\begin{entry}{荷花}{10,7}{⾋、⾋}
  \begin{phonetics}{荷花}{he2hua1}
    \definition{s.}{lótus}
  \end{phonetics}
\end{entry}

\begin{entry}{莎莎舞}{10,10,14}{⾋、⾋、⾇}
  \begin{phonetics}{莎莎舞}{sha1sha1wu3}
    \definition{s.}{salsa (dança)}
  \end{phonetics}
\end{entry}

\begin{entry}{莫名其妙}{10,6,8,7}{⾋、⼝、⼋、⼥}
  \begin{phonetics}{莫名其妙}{mo4ming2qi2miao4}
    \definition{adj.}{desconcertante | bizzaro | inexplicável | perplexo}
  \end{phonetics}
\end{entry}

\begin{entry}{莲花}{10,7}{⾋、⾋}
  \begin{phonetics}{莲花}{lian2hua1}
    \definition{s.}{flor de lótus | lírio aquático}
  \end{phonetics}
\end{entry}

\begin{entry}{莲藕}{10,18}{⾋、⾋}
  \begin{phonetics}{莲藕}{lian2'ou3}
    \definition{s.}{raiz de Lotus}
  \end{phonetics}
\end{entry}

\begin{entry}{获}{10}{⾋}
  \begin{phonetics}{获}{huo4}[][HSK 4]
    \definition*{s.}{sobrenome Huo}
    \definition{v.}{capturar; pegar | obter; ganhar; colher | colher; ceifar}
  \end{phonetics}
\end{entry}

\begin{entry}{获取}{10,8}{⾋、⼜}
  \begin{phonetics}{获取}{huo4 qu3}[][HSK 4]
    \definition{v.}{adquirir; obter; ganhar; colher}
  \end{phonetics}
\end{entry}

\begin{entry}{获奖}{10,9}{⾋、⼤}
  \begin{phonetics}{获奖}{huo4 jiang3}[][HSK 4]
    \definition{v.}{ganhar prêmio; ser recompensado; ganhar um prêmio; receber um prêmio}
  \end{phonetics}
\end{entry}

\begin{entry}{获得}{10,11}{⾋、⼻}
  \begin{phonetics}{获得}{huo4de2}[][HSK 4]
    \definition{v.}{adquirir; ganhar; obter; alcançar}
  \end{phonetics}
\end{entry}

\begin{entry}{蚊子}{10,3}{⾍、⼦}
  \begin{phonetics}{蚊子}{wen2zi5}
    \definition{s.}{pernilongo}
  \end{phonetics}
\end{entry}

\begin{entry}{蚊香}{10,9}{⾍、⾹}
  \begin{phonetics}{蚊香}{wen2xiang1}
    \definition{s.}{incenso ou espiral repelente de mosquitos}
  \end{phonetics}
\end{entry}

\begin{entry}{蚕纸}{10,7}{⾍、⽷}
  \begin{phonetics}{蚕纸}{can2zhi3}
    \definition{s.}{papel onde o bicho-da-seda põe seus ovos}
  \end{phonetics}
\end{entry}

\begin{entry}{蚝}{10}{⾍}
  \begin{phonetics}{蚝}{hao2}
    \definition{s.}{ostra}
  \end{phonetics}
\end{entry}

\begin{entry}{袖}{10}{⾐}
  \begin{phonetics}{袖}{xiu4}
    \definition{s.}{manga (de camisa, de camiseta, etc.)}
  \end{phonetics}
\end{entry}

\begin{entry}{袜子}{10,3}{⾐、⼦}
  \begin{phonetics}{袜子}{wa4zi5}[][HSK 4]
    \definition[双,只,对]{s.}{meias; peúgas; meias-calças}
  \end{phonetics}
\end{entry}

\begin{entry}{被}{10}{⾐}
  \begin{phonetics}{被}{bei4}[][HSK 3]
    \definition*{s.}{sobrenome Bei}
    \definition{part.}{usada antes de verbos para formar frases verbais passivas}
    \definition{prep.}{usado em uma frase para indicar que o sujeito é o receptor da ação}
    \definition{s.}{colcha}
    \definition{v.}{cobrir; espalhar
sofrer}
  \end{phonetics}
\end{entry}

\begin{entry}{被子}{10,3}{⾐、⼦}
  \begin{phonetics}{被子}{bei4zi5}[][HSK 3]
    \definition[床]{s.}{colcha}
  \end{phonetics}
\end{entry}

\begin{entry}{被动}{10,6}{⾐、⼒}
  \begin{phonetics}{被动}{bei4dong4}[][HSK 5]
    \definition{adj.}{passivo;  agir com base em um impulso externo (o oposto de ``主动'') | passivo; impossibilidade de prosseguir como pretendido devido a resistência ou interferência}
  \seealsoref{主动}{zhu3dong4}
  \end{phonetics}
\end{entry}

\begin{entry}{被告}{10,7}{⾐、⼝}
  \begin{phonetics}{被告}{bei4gao4}
    \definition{s.}{réu}
  \end{phonetics}
\end{entry}

\begin{entry}{被单}{10,8}{⾐、⼗}
  \begin{phonetics}{被单}{bei4dan1}
    \definition[床]{s.}{lençol}
  \end{phonetics}
\end{entry}

\begin{entry}{被迫}{10,8}{⾐、⾡}
  \begin{phonetics}{被迫}{bei4 po4}[][HSK 4]
    \definition{v.}{ser forçado; ser coagido; ser compelido; ser constrangido; ser forçado a fazer algo por força externa}
  \end{phonetics}
\end{entry}

\begin{entry}{被套}{10,10}{⾐、⼤}
  \begin{phonetics}{被套}{bei4tao4}
    \definition{s.}{capa de \emph{edredon}}
    \definition{v.}{ter dinheiro preso (em ações, imóveis, etc.)}
  \end{phonetics}
\end{entry}

\begin{entry}{被窝}{10,12}{⾐、⽳}
  \begin{phonetics}{被窝}{bei4wo1}
    \definition{s.}{colcha}
  \end{phonetics}
\end{entry}

\begin{entry}{请}{10}{⾔}
  \begin{phonetics}{请}{qing3}[][HSK 1]
    \definition{v.}{por favor (fazer alguma coisa) | perguntar | convidar | solicitar}
  \end{phonetics}
\end{entry}

\begin{entry}{请问}{10,6}{⾔、⾨}
  \begin{phonetics}{请问}{qing3wen4}[][HSK 1]
    \definition{expr.}{Com licença, posso perguntar\dots? (para perguntar por qualquer coisa)}
  \end{phonetics}
\end{entry}

\begin{entry}{请坐}{10,7}{⾔、⼟}
  \begin{phonetics}{请坐}{qing3 zuo4}[][HSK 1]
    \definition{v.}{por favor, sente-se}
  \end{phonetics}
\end{entry}

\begin{entry}{请求}{10,7}{⾔、⽔}
  \begin{phonetics}{请求}{qing3qiu2}[][HSK 2]
    \definition[个]{s.}{solicitação}
    \definition{v.}{solicitar | perguntar}
  \end{phonetics}
\end{entry}

\begin{entry}{请进}{10,7}{⾔、⾡}
  \begin{phonetics}{请进}{qing3 jin4}[][HSK 1]
    \definition{v.}{por favor entre}
  \end{phonetics}
\end{entry}

\begin{entry}{请客}{10,9}{⾔、⼧}
  \begin{phonetics}{请客}{qing3ke4}[][HSK 2]
    \definition{v.+compl.}{entreter os convidados | dar um jantar | convidar para jantar}
  \end{phonetics}
\end{entry}

\begin{entry}{请假}{10,11}{⾔、⼈}
  \begin{phonetics}{请假}{qing3 jia4}[][HSK 1]
    \definition{v.+compl.}{pedir licença para sair}
  \end{phonetics}
\end{entry}

\begin{entry}{请假条}{10,11,7}{⾔、⼈、⽊}
  \begin{phonetics}{请假条}{qing3jia4tiao2}
    \definition{s.}{pedido de licença de ausência (do trabalho ou da escola)}
  \end{phonetics}
\end{entry}

\begin{entry}{请教}{10,11}{⾔、⽁}
  \begin{phonetics}{请教}{qing3jiao4}[][HSK 3]
    \definition{v.}{consultar; pedir conselho}
  \end{phonetics}
\end{entry}

\begin{entry}{诺贝尔奖}{10,4,5,9}{⾔、⾙、⼩、⼤}
  \begin{phonetics}{诺贝尔奖}{nuo4bei4'er3 jiang3}
    \definition*{s.}{Prêmio Nobel}
  \end{phonetics}
\end{entry}

\begin{entry}{诺奖}{10,9}{⾔、⼤}
  \begin{phonetics}{诺奖}{nuo4jiang3}
    \definition*{s.}{Prêmio Nobel, abreviação de 诺贝尔奖}
    \seeref{诺贝尔奖}{nuo4bei4'er3 jiang3}
  \end{phonetics}
\end{entry}

\begin{entry}{读}{10}{⾔}
  \begin{phonetics}{读}{dou4}
    \definition{s.}{vírgula | frase marcada por pausa}
  \end{phonetics}
  \begin{phonetics}{读}{du2}[][HSK 1]
    \definition{v.}{ler em voz alta | ler | frequentar (escola) | estudar (uma matéria na escola) | pronunciar}
  \end{phonetics}
\end{entry}

\begin{entry}{读书}{10,4}{⾔、⼄}
  \begin{phonetics}{读书}{du2 shu1}[][HSK 1]
    \definition{v.+compl.}{ler | estudar | frequentar a escola}
  \end{phonetics}
\end{entry}

\begin{entry}{读者}{10,8}{⾔、⽼}
  \begin{phonetics}{读者}{du2 zhe3}[][HSK 3]
    \definition[个,位]{s.}{leitor}
  \end{phonetics}
\end{entry}

\begin{entry}{读音}{10,9}{⾔、⾳}
  \begin{phonetics}{读音}{du2 yin1}[][HSK 2]
    \definition{s.}{pronúncia}
  \end{phonetics}
\end{entry}

\begin{entry}{课}{10}{⾔}
  \begin{phonetics}{课}{ke4}[][HSK 1]
    \definition{s.}{aula | curso | lição | imposto | taxa |seção}
  \end{phonetics}
\end{entry}

\begin{entry}{课文}{10,4}{⾔、⽂}
  \begin{phonetics}{课文}{ke4 wen2}[][HSK 1]
    \definition{s.}{texto (de uma lição)}
  \end{phonetics}
\end{entry}

\begin{entry}{课本}{10,5}{⾔、⽊}
  \begin{phonetics}{课本}{ke4 ben3}[][HSK 1]
    \definition[本]{s.}{livro do aluno | manual}
  \end{phonetics}
\end{entry}

\begin{entry}{课堂}{10,11}{⾔、⼟}
  \begin{phonetics}{课堂}{ke4 tang2}[][HSK 2]
    \definition[间]{s.}{sala de aula}
  \end{phonetics}
\end{entry}

\begin{entry}{课程}{10,12}{⾔、⽲}
  \begin{phonetics}{课程}{ke4cheng2}[][HSK 3]
    \definition[个,堂,节,门]{s.}{curso; currículo}
  \end{phonetics}
\end{entry}

\begin{entry}{课题}{10,15}{⾔、⾴}
  \begin{phonetics}{课题}{ke4ti2}[][HSK 5]
    \definition{s.}{uma questão para estudo ou discussão; principais questões a serem pesquisadas ou discutidas, ou assuntos importantes que precisam ser resolvidos com urgência | tarefa; problema; questões a serem resolvidas}
  \end{phonetics}
\end{entry}

\begin{entry}{谁}{10}{⾔}
  \begin{phonetics}{谁}{shei2}[][HSK 1]
    \definition{pron.}{quem?}
  \end{phonetics}
  \begin{phonetics}{谁}{shui2}[][HSK 1]
    \definition{pron.}{quem?}
  \end{phonetics}
\end{entry}

\begin{entry}{调}{10}{⾔}
  \begin{phonetics}{调}{diao4}[][HSK 3]
    \definition{s.}{sotaque | nota (musical) | melodia; tom}
    \definition{v.}{transferir; deslocar; mover | distribuir; alocar | trocar; permutar; comutar}
  \end{phonetics}
  \begin{phonetics}{调}{tiao2}[][HSK 3]
    \definition*{s.}{sobrenome Tiao}
    \definition{v.}{harmonizar; adequar-se bem; encaixar-se perfeitamente | mesclar; ajustar; ajustar; misturar | mediar |gracejar; tirar sarro de | provocar; alienar}
  \end{phonetics}
\end{entry}

\begin{entry}{调皮}{10,5}{⾔、⽪}
  \begin{phonetics}{调皮}{tiao2pi2}[][HSK 4]
    \definition{adj.}{travesso; malicioso; malandro | indisciplinado; desordeiro; indomável; astuto | inteligente e desonesto}
  \end{phonetics}
\end{entry}

\begin{entry}{调节}{10,5}{⾔、⾋}
  \begin{phonetics}{调节}{tiao2jie2}[][HSK 5]
    \definition{v.}{regular; ajustar; ajustar e controlar de várias maneiras para atender aos requisitos}
  \end{phonetics}
\end{entry}

\begin{entry}{调动}{10,6}{⾔、⼒}
  \begin{phonetics}{调动}{diao4dong4}[][HSK 5]
    \definition{v.}{mudar; transferir; pessoal, trabalho | mobilizar; despertar; pôr em jogo; melhorar (motivação, entusiasmo, etc.) por meio de alguns meios | reunir; manobrar; mover (tropas); mobilizar forças militares}
  \end{phonetics}
\end{entry}

\begin{entry}{调律}{10,9}{⾔、⼻}
  \begin{phonetics}{调律}{tiao2lv4}
    \definition{v.}{afinar (por exemplo, um piano)}
  \end{phonetics}
\end{entry}

\begin{entry}{调查}{10,9}{⾔、⽊}
  \begin{phonetics}{调查}{diao4cha2}[][HSK 3]
    \definition[项,个]{s.}{pesquisa; investigação}
    \definition{v.}{investigar; indagar; inquerir}
  \end{phonetics}
\end{entry}

\begin{entry}{调解}{10,13}{⾔、⾓}
  \begin{phonetics}{调解}{tiao2jie3}[][HSK 5]
    \definition{v.}{mediar; fazer as pazes; resolver conflitos através da persuasão}
  \end{phonetics}
\end{entry}

\begin{entry}{调整}{10,16}{⾔、⽁}
  \begin{phonetics}{调整}{tiao2zheng3}[][HSK 3]
    \definition{v.}{ajustar; revisar; regular}
  \end{phonetics}
\end{entry}

\begin{entry}{谈}{10}{⾔}
  \begin{phonetics}{谈}{tan2}[][HSK 3]
    \definition*{s.}{sobrenome Tan}
    \definition{s.}{o que é dito ou falado}
    \definition{v.}{falar; bater papo; discutir}
  \end{phonetics}
\end{entry}

\begin{entry}{谈判}{10,7}{⾔、⼑}
  \begin{phonetics}{谈判}{tan2pan4}[][HSK 3]
    \definition{v.}{negociar; manter conversações}
  \end{phonetics}
\end{entry}

\begin{entry}{谈话}{10,8}{⾔、⾔}
  \begin{phonetics}{谈话}{tan2 hua4}[][HSK 3]
    \definition[次]{s.}{declaração}
    \definition{v.+compl.}{conversar; discutir | falar}
  \end{phonetics}
\end{entry}

\begin{entry}{谈恋爱}{10,10,10}{⾔、⼼、⽖}
  \begin{phonetics}{谈恋爱}{tan2lian4'ai4}
    \definition{v.}{namorar | apaixonar-se}
  \end{phonetics}
\end{entry}

\begin{entry}{豹子}{10,3}{⾘、⼦}
  \begin{phonetics}{豹子}{bao4zi5}
    \definition[头]{s.}{leopardo}
  \end{phonetics}
\end{entry}

\begin{entry}{资}{10}{⾙}
  \begin{phonetics}{资}{zi1}
    \definition{s.}{recursos | capital | dinheiro | despesa}
    \definition{v.}{fornecer | suprir}
  \end{phonetics}
\end{entry}

\begin{entry}{资助}{10,7}{⾙、⼒}
  \begin{phonetics}{资助}{zi1zhu4}
    \definition{s.}{subsídio}
    \definition{v.}{subsidiar | fornecer ajuda financeira}
  \end{phonetics}
\end{entry}

\begin{entry}{资金}{10,8}{⾙、⾦}
  \begin{phonetics}{资金}{zi1jin1}[][HSK 3]
    \definition[笔]{s.}{fundo; capital; capital para atividades empresariais}
  \end{phonetics}
\end{entry}

\begin{entry}{资料}{10,10}{⾙、⽃}
  \begin{phonetics}{资料}{zi1liao4}[][HSK 4]
    \definition[份,个]{s.}{dados; material; material informativo para referência ou para ser considerado confiável | material de produção; meios de subsistência; requisitos de produção ou subsistência}
  \end{phonetics}
\end{entry}

\begin{entry}{资格}{10,10}{⾙、⽊}
  \begin{phonetics}{资格}{zi1ge2}[][HSK 3]
    \definition{s.}{qualificação; as condições e identidades necessárias para exercer uma determinada atividade | senioridade; uma identidade formada pelo tempo gasto realizando um determinado trabalho ou atividade}
  \end{phonetics}
\end{entry}

\begin{entry}{资源}{10,13}{⾙、⽔}
  \begin{phonetics}{资源}{zi1yuan2}[][HSK 4]
    \definition{s.}{recurso; fontes naturais de meios de produção ou subsistência}
  \end{phonetics}
\end{entry}

\begin{entry}{赶}{10}{⾛}
  \begin{phonetics}{赶}{gan3}[][HSK 3]
    \definition*{s.}{sobrenome Gan}
    \definition{prep.}{por; até}
    \definition{v.}{ultrapassar; alcançar | perseguir; correr para; correr atrás; tentar pegar | dirigir | expulsar; afastar | encontrar; deparar-se com; esbarrar em; acontecer com; encontrar-se em (uma situação); aproveitar-se de (uma oportunidade) | ir para}
  \end{phonetics}
\end{entry}

\begin{entry}{赶上}{10,3}{⾛、⼀}
  \begin{phonetics}{赶上}{gan3shang4}
    \definition{adv.}{a tempo para}
    \definition{v.}{alcançar | ultrapassar}
  \end{phonetics}
\end{entry}

\begin{entry}{赶忙}{10,6}{⾛、⼼}
  \begin{phonetics}{赶忙}{gan3mang2}
    \definition{v.}{acelerar | apressar | se apressar}
  \end{phonetics}
\end{entry}

\begin{entry}{赶早}{10,6}{⾛、⽇}
  \begin{phonetics}{赶早}{gan3zao3}
    \definition{adv.}{o mais breve possível | na primeira oportunidade | antes que seja tarde | quanto antes melhor}
  \end{phonetics}
\end{entry}

\begin{entry}{赶快}{10,7}{⾛、⼼}
  \begin{phonetics}{赶快}{gan3kuai4}[][HSK 3]
    \definition{adv.}{rapidamente; imediatamente}
  \end{phonetics}
\end{entry}

\begin{entry}{赶走}{10,7}{⾛、⾛}
  \begin{phonetics}{赶走}{gan3zou3}
    \definition{v.}{expulsar | voltar atrás}
  \end{phonetics}
\end{entry}

\begin{entry}{赶到}{10,8}{⾛、⼑}
  \begin{phonetics}{赶到}{gan3 dao4}[][HSK 3]
    \definition{v.}{apressar (para algum lugar); avançar de súbito}
  \end{phonetics}
\end{entry}

\begin{entry}{赶赴}{10,9}{⾛、⾛}
  \begin{phonetics}{赶赴}{gan3fu4}
    \definition{v.}{apressar}
  \end{phonetics}
\end{entry}

\begin{entry}{赶紧}{10,10}{⾛、⽷}
  \begin{phonetics}{赶紧}{gan3jin3}[][HSK 3]
    \definition{adv.}{apressadamente; sem demora}
  \end{phonetics}
\end{entry}

\begin{entry}{赶脚}{10,11}{⾛、⾁}
  \begin{phonetics}{赶脚}{gan3jiao3}
    \definition{v.}{transportar mercadorias para ganhar a vida (especialmente de burro) | trabalhar como carroceiro ou porteiro}
  \end{phonetics}
\end{entry}

\begin{entry}{赶跑}{10,12}{⾛、⾜}
  \begin{phonetics}{赶跑}{gan3pao3}
    \definition{v.}{afastar | forçar a saída | repelir}
  \end{phonetics}
\end{entry}

\begin{entry}{赶集}{10,12}{⾛、⾫}
  \begin{phonetics}{赶集}{gan3ji2}
    \definition{v.}{ir a uma feira | ir ao mercado}
  \end{phonetics}
\end{entry}

\begin{entry}{赶路}{10,13}{⾛、⾜}
  \begin{phonetics}{赶路}{gan3lu4}
    \definition{v.}{apressar a jornada | apressar-se}
  \end{phonetics}
\end{entry}

\begin{entry}{起}{10}{⾛}
  \begin{phonetics}{起}{qi3}[][HSK 1]
    \definition*{s.}{sobrenome Qi}
    \definition{clas.}{caso; instância | lote; grupo}
    \definition{v.}{levantar | levantar-se | extrair| remover | puxar | aparecer | crescer | construir | configurar | começar | iniciar}
  \end{phonetics}
\end{entry}

\begin{entry}{起飞}{10,3}{⾛、⾶}
  \begin{phonetics}{起飞}{qi3fei1}[][HSK 2]
    \definition{v.}{decolar}
  \end{phonetics}
\end{entry}

\begin{entry}{起床}{10,7}{⾛、⼴}
  \begin{phonetics}{起床}{qi3 chuang2}[][HSK 1]
    \definition{v.+compl.}{sair da cama | levantar-se}
  \end{phonetics}
\end{entry}

\begin{entry}{起来}{10,7}{⾛、⽊}
  \begin{phonetics}{起来}{qi3 lai2}[][HSK 1]
    \definition{v.+compl.}{levantar-se}
  \end{phonetics}
\end{entry}

\begin{entry}{起到}{10,8}{⾛、⼑}
  \begin{phonetics}{起到}{qi3 dao4}[][HSK 5]
    \definition{v.}{ter (um efeito motivador, etc.); desempenhar (um papel estabilizador, etc.)}
  \end{phonetics}
\end{entry}

\begin{entry}{起码}{10,8}{⾛、⽯}
  \begin{phonetics}{起码}{qi3ma3}[][HSK 5]
    \definition{adj.}{mínimo; elementar; rudimentar}
    \definition{adv.}{mínimamente; pelo menos;}
  \end{phonetics}
\end{entry}

\begin{entry}{起跳}{10,13}{⾛、⾜}
  \begin{phonetics}{起跳}{qi3tiao4}
    \definition{v.}{(atletismo) decolar (no início de um salto) | (de preço, salário, etc.) começar (de um determinado nível)}
  \end{phonetics}
\end{entry}

\begin{entry}{较}{10}{⾞}
  \begin{phonetics}{较}{jiao4}[][HSK 3]
    \definition{adj.}{claro; óbvio; marcado}
    \definition{adv.}{comparativamente; relativamente; razoavelmente; bastante; bastante}
    \definition{prep.}{usado para comparar características e graus}
    \definition{v.}{comparar | disputar}
  \end{phonetics}
\end{entry}

\begin{entry}{辱骂}{10,9}{⾠、⾺}
  \begin{phonetics}{辱骂}{ru3ma4}
    \definition{v.}{insultar | abusar}
  \end{phonetics}
\end{entry}

\begin{entry}{透}{10}{⾡}
  \begin{phonetics}{透}{tou4}[][HSK 4]
    \definition{adv.}{totalmente; completamente; minuciosamente | profundamente; extremamente}
    \definition{v.}{penetrar; passar através de; infiltrar-se através de | revelar; deixar transparecer; contar secretamente |mostrar; aparecer}
  \end{phonetics}
\end{entry}

\begin{entry}{透支}{10,4}{⾡、⽀}
  \begin{phonetics}{透支}{tou4zhi1}
    \definition{v.}{cheque especial (bancário) | saque a descoberto}
  \end{phonetics}
\end{entry}

\begin{entry}{透气}{10,4}{⾡、⽓}
  \begin{phonetics}{透气}{tou4qi4}
    \definition{v.}{respirar (sobre tecido, etc.) | fluir livremente (sobre ar) | respirar ar fresco | ventilar}
  \end{phonetics}
\end{entry}

\begin{entry}{透水}{10,4}{⾡、⽔}
  \begin{phonetics}{透水}{tou4shui3}
    \definition{adj.}{permeável}
    \definition{s.}{vazamento de água}
  \end{phonetics}
\end{entry}

\begin{entry}{透过}{10,6}{⾡、⾡}
  \begin{phonetics}{透过}{tou4guo4}
    \definition{v.}{passar através | penetrar}
  \end{phonetics}
\end{entry}

\begin{entry}{透彻}{10,7}{⾡、⼻}
  \begin{phonetics}{透彻}{tou4che4}
    \definition{adj.}{minucioso | incisivo | penetrante}
  \end{phonetics}
\end{entry}

\begin{entry}{透明}{10,8}{⾡、⽇}
  \begin{phonetics}{透明}{tou4ming2}[][HSK 4]
    \definition{adj.}{transparente; diáfano; capaz de transmitir luz | evidente; transparente; situação ou assunto que seja aberto e não oculto | transparente; diáfano; indica pureza, ausência de impurezas}
  \end{phonetics}
\end{entry}

\begin{entry}{透顶}{10,8}{⾡、⾴}
  \begin{phonetics}{透顶}{tou4ding3}
    \definition{adv.}{completamente}
  \end{phonetics}
\end{entry}

\begin{entry}{透亮}{10,9}{⾡、⼇}
  \begin{phonetics}{透亮}{tou4liang4}
    \definition{adj.}{brilhante | claro como cristal}
  \end{phonetics}
\end{entry}

\begin{entry}{透辟}{10,13}{⾡、⾟}
  \begin{phonetics}{透辟}{tou4pi4}
    \definition{adj.}{incisivo | penetrante}
  \end{phonetics}
\end{entry}

\begin{entry}{透澈}{10,15}{⾡、⽔}
  \begin{phonetics}{透澈}{tou4che4}
    \variantof{透彻}
  \end{phonetics}
\end{entry}

\begin{entry}{透露}{10,21}{⾡、⾬}
  \begin{phonetics}{透露}{tou4lu4}
    \definition{v.}{divulgar | vazar | revelar}
  \end{phonetics}
\end{entry}

\begin{entry}{逐步}{10,7}{⾡、⽌}
  \begin{phonetics}{逐步}{zhu2bu4}[][HSK 4]
    \definition{adv.}{gradualmente; passo a passo; progressivamente}
  \end{phonetics}
\end{entry}

\begin{entry}{逐渐}{10,11}{⾡、⽔}
  \begin{phonetics}{逐渐}{zhu2jian4}[][HSK 4]
    \definition{adv.}{gradualmente; aos poucos; por etapas; indica mudanças lentas e ordenadas no grau, na quantidade, etc.}
  \end{phonetics}
\end{entry}

\begin{entry}{递}{10}{⾡}
  \begin{phonetics}{递}{di4}[][HSK 5]
    \definition{adv.}{na ordem correta; sucessivamente}
    \definition{v.}{entregar; passar; dar; transmitir}
  \end{phonetics}
\end{entry}

\begin{entry}{递给}{10,9}{⾡、⽷}
  \begin{phonetics}{递给}{di4 gei3}[][HSK 5]
    \definition{v.}{entregar algo a alguém; passar itens ou coisas para outras pessoas}
  \end{phonetics}
\end{entry}

\begin{entry}{途中}{10,4}{⾡、⼁}
  \begin{phonetics}{途中}{tu2 zhong1}[][HSK 4]
    \definition{adv.}{no caminho; ao longo do caminho}
  \end{phonetics}
\end{entry}

\begin{entry}{通}{10}{⾡}
  \begin{phonetics}{通}{tong1}[][HSK 2]
    \definition{clas.}{para cartas, telegramas, telefonemas, etc.}
    \definition{suf.}{especialista}
    \definition{v.}{ligar para | conseguir a ligação}
  \end{phonetics}
  \begin{phonetics}{通}{tong4}
    \definition{clas.}{para uma atividade, tomada em sua totalidade (discurso de abuso, período de reprodução de música, bebedeira, etc.)}
  \end{phonetics}
\end{entry}

\begin{entry}{通用}{10,5}{⾡、⽤}
  \begin{phonetics}{通用}{tong1yong4}[][HSK 5]
    \definition{adj.}{de uso comum; universal; (em um determinado âmbito) de uso generalizado | intercambiável; alguns caracteres chineses com grafia diferente, mas pronúncia igual, podem ser usados indistintamente (alguns limitados a um determinado significado)}
  \end{phonetics}
\end{entry}

\begin{entry}{通观}{10,6}{⾡、⾒}
  \begin{phonetics}{通观}{tong1guan1}
    \definition{v.}{ter uma visão geral de algo}
  \end{phonetics}
\end{entry}

\begin{entry}{通过}{10,6}{⾡、⾡}
  \begin{phonetics}{通过}{tong1guo4}[][HSK 2]
    \definition{adv.}{por meio de | através de | via}
    \definition{v.}{passar por | adotar (uma resolução), aprovar (legislação) | passar (em um teste)}
  \end{phonetics}
\end{entry}

\begin{entry}{通识}{10,7}{⾡、⾔}
  \begin{phonetics}{通识}{tong1shi2}
    \definition{s.}{conhecimento comum | erudição | conhecimento geral | amplamente conhecido}
  \end{phonetics}
\end{entry}

\begin{entry}{通知}{10,8}{⾡、⽮}
  \begin{phonetics}{通知}{tong1zhi1}[][HSK 2]
    \definition[份,个,张]{s.}{aviso | circular}
    \definition{v.}{aconselhar | notificar | informar | dar aviso}
  \end{phonetics}
\end{entry}

\begin{entry}{通知书}{10,8,4}{⾡、⽮、⼄}
  \begin{phonetics}{通知书}{tong1 zhi1 shu1}[][HSK 4]
    \definition{s.}{aviso; observação; notificação}
  \end{phonetics}
\end{entry}

\begin{entry}{通信}{10,9}{⾡、⼈}
  \begin{phonetics}{通信}{tong1 xin4}[][HSK 3]
    \definition{v.+compl.}{corresponder; comunicar por carta | transmitir (ou transportar) mensagem; passar (ou transmitir) informação}
  \end{phonetics}
\end{entry}

\begin{entry}{通常}{10,11}{⾡、⼱}
  \begin{phonetics}{通常}{tong1chang2}[][HSK 3]
    \definition{adj.}{usual; normal; geral}
    \definition{adv.}{habitualmente; usualmente; geralmente; ordinariamente}
  \end{phonetics}
\end{entry}

\begin{entry}{通牒}{10,13}{⾡、⽚}
  \begin{phonetics}{通牒}{tong1die2}
    \definition{s.}{nota diplomática}
  \end{phonetics}
\end{entry}

\begin{entry}{逛}{10}{⾡}
  \begin{phonetics}{逛}{guang4}[][HSK 4]
    \definition{v.}{perambular; passear; vaguear}
  \end{phonetics}
\end{entry}

\begin{entry}{速度}{10,9}{⾡、⼴}
  \begin{phonetics}{速度}{su4du4}[][HSK 3]
    \definition[个,种]{s.}{velocidade; taxa; ritmo; andamento | velocidade; rapidez}
  \end{phonetics}
\end{entry}

\begin{entry}{造}{10}{⾡}
  \begin{phonetics}{造}{zao4}[][HSK 3]
    \definition*{s.}{sobrenome Zao}
    \definition{clas.}{para colheitas ou número de colheitas de safras}
    \definition{s.}{uma das duas partes em um acordo legal ou uma ação judicial | colheita; safra}
    \definition{v.}{fazer; construir; criar; produzir | cozinhar; fabricar; inventar | ir para; chegar a | alcançar; atingir | treinar; educar}
  \end{phonetics}
\end{entry}

\begin{entry}{造成}{10,6}{⾡、⼽}
  \begin{phonetics}{造成}{zao4cheng2}[][HSK 3]
    \definition{v.}{criar; causar; acarretar; dar origem a; formar; levar a (principalmente resultados ruins)}
  \end{phonetics}
\end{entry}

\begin{entry}{造型}{10,9}{⾡、⼟}
  \begin{phonetics}{造型}{zao4xing2}[][HSK 4]
    \definition{s.}{modelo; formato; forma; moldagem}
    \definition{v.}{modelar; moldar}
  \end{phonetics}
\end{entry}

\begin{entry}{部}{10}{⾢}
  \begin{phonetics}{部}{bu4}[][HSK 3]
    \definition{clas.}{para obras de literatura, filmes, máquinas etc.}
    \definition[根]{s.}{departamento | divisão | ministério | seção | parte | tropas}
  \end{phonetics}
\end{entry}

\begin{entry}{部下}{10,3}{⾢、⼀}
  \begin{phonetics}{部下}{bu4xia4}
    \definition{s.}{subordinado | tropas sob comando de alguém}
  \end{phonetics}
\end{entry}

\begin{entry}{部门}{10,3}{⾢、⾨}
  \begin{phonetics}{部门}{bu4men2}[][HSK 3]
    \definition[个]{s.}{filial | departamento | divisão | seção}
  \end{phonetics}
\end{entry}

\begin{entry}{部分}{10,4}{⾢、⼑}
  \begin{phonetics}{部分}{bu4fen5}[][HSK 2]
    \definition[个]{s.}{parte | parte de | uma parte de | pedaço | secção}
  \end{phonetics}
\end{entry}

\begin{entry}{部长}{10,4}{⾢、⾧}
  \begin{phonetics}{部长}{bu4 zhang3}[][HSK 3]
    \definition[个,位,名]{s.}{ministro | chefe de departamento | chefe de seção}
  \end{phonetics}
\end{entry}

\begin{entry}{部队}{10,4}{⾢、⾩}
  \begin{phonetics}{部队}{bu4dui4}
    \definition[个]{s.}{exército | forças armadas | tropas | unidades}
  \end{phonetics}
\end{entry}

\begin{entry}{部位}{10,7}{⾢、⼈}
  \begin{phonetics}{部位}{bu4wei4}[][HSK 5]
    \definition{s.}{lugar; posição (usado principalmente para o corpo humano)}
  \end{phonetics}
\end{entry}

\begin{entry}{部族}{10,11}{⾢、⽅}
  \begin{phonetics}{部族}{bu4zu2}
    \definition{adj.}{tribal}
    \definition{s.}{tribo}
  \end{phonetics}
\end{entry}

\begin{entry}{部属}{10,12}{⾢、⼫}
  \begin{phonetics}{部属}{bu4shu3}
    \definition{s.}{afiliado a um ministério | subordinado | tropas sob comando de alguém}
  \end{phonetics}
\end{entry}

\begin{entry}{部署}{10,13}{⾢、⽹}
  \begin{phonetics}{部署}{bu4shu3}
    \definition{s.}{implantação}
    \definition{v.}{implantar}
  \end{phonetics}
\end{entry}

\begin{entry}{都}{10}{⾢}
  \begin{phonetics}{都}{dou1}[][HSK 1]
    \definition{adv.}{todos | ambos | inteiramente | até | já (usado para dar ênfase) | (não) em tudo}
  \end{phonetics}
  \begin{phonetics}{都}{du1}
    \definition*{s.}{sobrenome Du}
    \definition{s.}{capital | metrópole}
  \end{phonetics}
\end{entry}

\begin{entry}{配}{10}{⾣}
  \begin{phonetics}{配}{pei4}[][HSK 3]
    \definition{s.}{esposa}
    \definition{v.}{unir-se em matrimônio | acasalar (animais) | compor; combinar; mesclar; amalgamar; misturar |distribuir de acordo com o plano; repartir | encontrar algo para encaixar ou substituir outra coisa | corresponder; combinar; equiparar | merecer; ser digno de; ser qualificado}
  \end{phonetics}
\end{entry}

\begin{entry}{配合}{10,6}{⾣、⼝}
  \begin{phonetics}{配合}{pei4he2}[][HSK 3]
    \definition{s.}{coordenação}
    \definition{v.}{cooperar; coordenar}
  \end{phonetics}
\end{entry}

\begin{entry}{配备}{10,8}{⾣、⼡}
  \begin{phonetics}{配备}{pei4bei4}[][HSK 5]
    \definition{s.}{equipamento; material; conjunto completo de utensílios, etc.}
    \definition{v.}{fornecer; alocar; equipar; distribuir conforme necessário | posicionar; dispor (tropas, etc.)}
  \end{phonetics}
\end{entry}

\begin{entry}{配套}{10,10}{⾣、⼤}
  \begin{phonetics}{配套}{pei4tao4}[][HSK 5]
    \definition{v.+compl.}{formar um conjunto ou sistema completo; combinar vários elementos relacionados em um conjunto completo}
  \end{phonetics}
\end{entry}

\begin{entry}{酒}{10}{⾣}
  \begin{phonetics}{酒}{jiu3}[][HSK 2]
    \definition[杯,瓶,罐,桶,缸]{s.}{bebida alcoólica | vinho (especialmente vinho de arroz) | aguardente | licor | espíritos}
  \end{phonetics}
\end{entry}

\begin{entry}{酒吧}{10,7}{⾣、⼝}
  \begin{phonetics}{酒吧}{jiu3ba1}[][HSK 4]
    \definition[家,个]{s.}{bar; \emph{pub}; um local onde são vendidas bebidas alcoólicas e onde as pessoas podem beber e conversar, referindo-se principalmente a um restaurante ou hotel de estilo ocidental especializado na venda de bebidas alcoólicas.}
  \end{phonetics}
\end{entry}

\begin{entry}{酒店}{10,8}{⾣、⼴}
  \begin{phonetics}{酒店}{jiu3 dian4}[][HSK 2]
    \definition[家]{s.}{hotel | restaurante}
  \end{phonetics}
\end{entry}

\begin{entry}{酒鬼}{10,9}{⾣、⿁}
  \begin{phonetics}{酒鬼}{jiu3gui3}[][HSK 5]
    \definition{s.}{bebedor de vinho; beberrão; ébrio | alcoólatra}
  \end{phonetics}
\end{entry}

\begin{entry}{酒馆}{10,11}{⾣、⾷}
  \begin{phonetics}{酒馆}{jiu3guan3}
    \definition{s.}{bar | taverna | adega}
  \end{phonetics}
\end{entry}

\begin{entry}{钱}{10}{⾦}
  \begin{phonetics}{钱}{qian2}[][HSK 1]
    \definition*{s.}{sobrenome Qian}
    \definition[笔]{s.}{moeda | dinheiro}
  \end{phonetics}
\end{entry}

\begin{entry}{钱包}{10,5}{⾦、⼓}
  \begin{phonetics}{钱包}{qian2bao1}[][HSK 1]
    \definition{s.}{carteira | bolsa}
  \end{phonetics}
\end{entry}

\begin{entry}{钻石}{10,5}{⾦、⽯}
  \begin{phonetics}{钻石}{zuan4shi2}
    \definition[颗]{s.}{diamante}
  \end{phonetics}
\end{entry}

\begin{entry}{钻戒}{10,7}{⾦、⼽}
  \begin{phonetics}{钻戒}{zuan4jie4}
    \definition[只]{s.}{anel de diamante}
  \end{phonetics}
\end{entry}

\begin{entry}{钿}{10}{⾦}
  \begin{phonetics}{钿}{dian4}
    \definition{s.}{ornamento incrustado antigo em forma de flor}
    \definition{v.}{incrustar com ouro, prata, etc.}
  \end{phonetics}
  \begin{phonetics}{钿}{tian2}
    \definition{s.}{(dialeto) moeda, dinheiro}
  \end{phonetics}
\end{entry}

\begin{entry}{铁}{10}{⾦}
  \begin{phonetics}{铁}{tie3}[][HSK 3]
    \definition*{s.}{sobrenome Tie}
    \definition{adj.}{duro; forte; sólido como ferro | violento | inabalável; inalterável; determinado | (gíria) apertado}
    \definition{s.}{ferro (Fe) | arma; armamento}
    \definition{v.}{resolver; determinar}
  \end{phonetics}
\end{entry}

\begin{entry}{铁轨}{10,6}{⾦、⾞}
  \begin{phonetics}{铁轨}{tie3gui3}
    \definition[根]{s.}{trilho | trilho ferroviário}
  \end{phonetics}
\end{entry}

\begin{entry}{铁路}{10,13}{⾦、⾜}
  \begin{phonetics}{铁路}{tie3 lu4}[][HSK 3]
    \definition[条]{s.}{ferrovia; estrada de ferro}
  \end{phonetics}
\end{entry}

\begin{entry}{铃}{10}{⾦}
  \begin{phonetics}{铃}{ling2}[][HSK 5]
    \definition{s.}{sino; instrumento musical feito de metal | objetos em forma de sino | cápsula; botão; broto}
  \end{phonetics}
\end{entry}

\begin{entry}{铃声}{10,7}{⾦、⼠}
  \begin{phonetics}{铃声}{ling2 sheng1}[][HSK 5]
    \definition{s.}{o tilintar de sinos; o som de um sino tocando}
  \end{phonetics}
\end{entry}

\begin{entry}{阅兵式}{10,7,6}{⾨、⼋、⼷}
  \begin{phonetics}{阅兵式}{yue4bing1shi4}
    \definition{s.}{parada militar}
  \end{phonetics}
\end{entry}

\begin{entry}{阅览室}{10,9,9}{⾨、⾒、⼧}
  \begin{phonetics}{阅览室}{yue4lan3shi4}
    \definition[间]{s.}{sala de leitura}
  \end{phonetics}
\end{entry}

\begin{entry}{阅读}{10,10}{⾨、⾔}
  \begin{phonetics}{阅读}{yue4du2}[][HSK 4]
    \definition{s.}{leitura}
    \definition{v.}{ler; examinar; olhar (livros, jornais, etc.) e entender seu conteúdo}
  \end{phonetics}
\end{entry}

\begin{entry}{阅读广度}{10,10,3,9}{⾨、⾔、⼴、⼴}
  \begin{phonetics}{阅读广度}{yue4du2guang3du4}
    \definition{s.}{intervalo de leitura}
  \end{phonetics}
\end{entry}

\begin{entry}{阅读时间}{10,10,7,7}{⾨、⾔、⽇、⾨}
  \begin{phonetics}{阅读时间}{yue4du2shi2jian1}
    \definition{s.}{tempo de leitura}
  \end{phonetics}
\end{entry}

\begin{entry}{阅读理解}{10,10,11,13}{⾨、⾔、⽟、⾓}
  \begin{phonetics}{阅读理解}{yue4du2li3jie3}
    \definition{s.}{compreensão de leitura}
  \end{phonetics}
\end{entry}

\begin{entry}{阅读装置}{10,10,12,13}{⾨、⾔、⾐、⽹}
  \begin{phonetics}{阅读装置}{yue4du2zhuang1zhi4}
    \definition{s.}{dispositivo de leitura (por exemplo, para códigos de barras, etiquetas RFID, etc.)}
  \end{phonetics}
\end{entry}

\begin{entry}{阅读障碍}{10,10,13,13}{⾨、⾔、⾩、⽯}
  \begin{phonetics}{阅读障碍}{yue4du2zhang4ai4}
    \definition{s.}{dislexia}
  \end{phonetics}
\end{entry}

\begin{entry}{阅读器}{10,10,16}{⾨、⾔、⼝}
  \begin{phonetics}{阅读器}{yue4du2qi4}
    \definition{s.}{leitor (\emph{software})}
  \end{phonetics}
\end{entry}

\begin{entry}{陪}{10}{⾩}
  \begin{phonetics}{陪}{pei2}[][HSK 5]
    \definition{v.}{servir; acompanhar; cuidar; fazer companhia a alguém | auxiliar; ajudar}
  \end{phonetics}
\end{entry}

\begin{entry}{陵园}{10,7}{⾩、⼞}
  \begin{phonetics}{陵园}{ling2yuan2}
    \definition{s.}{cemitério}
  \end{phonetics}
\end{entry}

\begin{entry}{陷入}{10,2}{⾩、⼊}
  \begin{phonetics}{陷入}{xian4ru4}
    \definition{v.}{afundar | ser pego em | pousar (em uma situação)}
  \end{phonetics}
\end{entry}

\begin{entry}{难}{10}{⾫}
  \begin{phonetics}{难}{nan2}[][HSK 1]
    \definition{adj.}{difícil}
    \definition{s.}{dificuldade}
  \end{phonetics}
  \begin{phonetics}{难}{nan4}
    \definition{s.}{desastre}
    \definition{v.}{repreender}
  \end{phonetics}
\end{entry}

\begin{entry}{难以}{10,4}{⾫、⼈}
  \begin{phonetics}{难以}{nan2 yi3}[][HSK 5]
    \definition{adj.}{difícil; complicado}
  \end{phonetics}
\end{entry}

\begin{entry}{难过}{10,6}{⾫、⾡}
  \begin{phonetics}{难过}{nan2guo4}[][HSK 2]
    \definition{adj.}{triste | ruim | pesaroso | arrependido | difícil}
  \end{phonetics}
\end{entry}

\begin{entry}{难免}{10,7}{⾫、⼉}
  \begin{phonetics}{难免}{nan4mian3}[][HSK 4]
    \definition{adj.}{inevitável; difícil de evitar}
  \end{phonetics}
\end{entry}

\begin{entry}{难听}{10,7}{⾫、⼝}
  \begin{phonetics}{难听}{nan2 ting1}[][HSK 2]
    \definition{adj.}{desagradável de ouvir | ofensivo | grosseiro | escandaloso}
  \end{phonetics}
\end{entry}

\begin{entry}{难受}{10,8}{⾫、⼜}
  \begin{phonetics}{难受}{nan2shou4}[][HSK 2]
    \definition{adj.}{sofrer dor | sentir-se mal | desconfortável | sentir-se infeliz}
  \end{phonetics}
\end{entry}

\begin{entry}{难度}{10,9}{⾫、⼴}
  \begin{phonetics}{难度}{nan2 du4}[][HSK 3]
    \definition{s.}{dificuldade; grau de dificuldade}
  \end{phonetics}
\end{entry}

\begin{entry}{难看}{10,9}{⾫、⽬}
  \begin{phonetics}{难看}{nan2 kan4}[][HSK 2]
    \definition{adj.}{feio | antiestético | vergonhoso | embaraçoso | vergonhoso}
  \end{phonetics}
\end{entry}

\begin{entry}{难得}{10,11}{⾫、⼻}
  \begin{phonetics}{难得}{nan2de2}[][HSK 5]
    \definition{adj.}{raro; difícil de encontrar; difícil de obter ou realizar, indicando que é valioso}
    \definition{adv.}{raramente; com pouca frequência}
  \end{phonetics}
\end{entry}

\begin{entry}{难道}{10,12}{⾫、⾡}
  \begin{phonetics}{难道}{nan2dao4}[][HSK 3]
    \definition{adv.}{indica uma pergunta retórica | certamente não significa que\dots?; é possível que\dots?; não me diga\dots; poderia ser que\dots?}
  \end{phonetics}
\end{entry}

\begin{entry}{难题}{10,15}{⾫、⾴}
  \begin{phonetics}{难题}{nan2 ti2}[][HSK 2]
    \definition[出]{s.}{desafio | problema difícil | pergunta difícil}
  \end{phonetics}
\end{entry}

\begin{entry}{顽强}{10,12}{⾴、⼸}
  \begin{phonetics}{顽强}{wan2qiang2}
    \definition{adj.}{persistente | tenaz | difícil de derrotar}
  \end{phonetics}
\end{entry}

\begin{entry}{顾问}{10,6}{⾴、⾨}
  \begin{phonetics}{顾问}{gu4wen4}[][HSK 5]
    \definition{s.}{conselheiro; consultor; assessor; pessoas com conhecimento especializado ou experiência contratadas para prestar consultoria a organizações ou indivíduos}
  \end{phonetics}
\end{entry}

\begin{entry}{顾客}{10,9}{⾴、⼧}
  \begin{phonetics}{顾客}{gu4ke4}[][HSK 2]
    \definition[位]{s.}{cliente}
  \end{phonetics}
\end{entry}

\begin{entry}{顿}{10}{⾴}
  \begin{phonetics}{顿}{dun4}[][HSK 3]
    \definition*{s.}{sobrenome Dun}
    \definition{adj.}{cansado; fatigado}
    \definition{adv.}{de repente; imediatamente}
    \definition{clas.}{para refeições | para surras, repreensões, etc.}
    \definition{s.}{um lugar para ficar}
    \definition{v.}{pausar
pausar na escrita para reforçar o início ou o fim de um traço
tocar o chão (com a cabeça)
pisar (o pé)
resolver; arranjar
montar acampamento; ficar temporariamente}
  \end{phonetics}
\end{entry}

\begin{entry}{预}{10}{⾴}
  \begin{phonetics}{预}{yu4}
    \definition{adv.}{antecipadamente}
    \definition{v.}{avançar | preparar}
  \end{phonetics}
\end{entry}

\begin{entry}{预习}{10,3}{⾴、⼄}
  \begin{phonetics}{预习}{yu4xi2}[][HSK 3]
    \definition{v.}{pré-visualizar; preparar uma lição; estudar com antecedência as aulas que irá assistir}
  \end{phonetics}
\end{entry}

\begin{entry}{预见}{10,4}{⾴、⾒}
  \begin{phonetics}{预见}{yu4jian4}
    \definition{s.}{previsão; intuição; vislumbre}
    \definition{v.}{prever}
  \end{phonetics}
\end{entry}

\begin{entry}{预计}{10,4}{⾴、⾔}
  \begin{phonetics}{预计}{yu4 ji4}[][HSK 3]
    \definition{v.}{estimar; esperar; calcular com antecedência}
  \end{phonetics}
\end{entry}

\begin{entry}{预订}{10,4}{⾴、⾔}
  \begin{phonetics}{预订}{yu4ding4}[][HSK 4]
    \definition{v.}{reservar; fazer uma reserva}
  \end{phonetics}
\end{entry}

\begin{entry}{预付}{10,5}{⾴、⼈}
  \begin{phonetics}{预付}{yu4fu4}
    \definition{s.}{pré-pago}
    \definition{v.}{pagar antecipadamente}
  \end{phonetics}
\end{entry}

\begin{entry}{预约}{10,6}{⾴、⽷}
  \begin{phonetics}{预约}{yu4yue1}
    \definition{s.}{reserva}
    \definition{v.}{agendar | marcar compromisso}
  \end{phonetics}
\end{entry}

\begin{entry}{预防}{10,6}{⾴、⾩}
  \begin{phonetics}{预防}{yu4fang2}[][HSK 3]
    \definition{v.}{prevenir; proteger-se contra; tomar precauções contra}
  \end{phonetics}
\end{entry}

\begin{entry}{预判}{10,7}{⾴、⼑}
  \begin{phonetics}{预判}{yu4pan4}
    \definition{v.}{prever | antecipar}
  \end{phonetics}
\end{entry}

\begin{entry}{预报}{10,7}{⾴、⼿}
  \begin{phonetics}{预报}{yu4bao4}[][HSK 3]
    \definition[个]{s.}{boletim meteorológico; previsões meteorológicas antecipadas}
    \definition{v.}{prever (o tempo); relato de coisas antes que elas aconteçam, usado principalmente em clima, astronomia, desastres naturais, etc.}
  \end{phonetics}
\end{entry}

\begin{entry}{预定}{10,8}{⾴、⼧}
  \begin{phonetics}{预定}{yu4ding4}
    \definition{v.}{agendar com antecedência}
  \end{phonetics}
\end{entry}

\begin{entry}{预购}{10,8}{⾴、⾙}
  \begin{phonetics}{预购}{yu4gou4}
    \definition{s.}{compra antecipada}
    \definition{v.}{comprar antecipadamente}
  \end{phonetics}
\end{entry}

\begin{entry}{预测}{10,9}{⾴、⽔}
  \begin{phonetics}{预测}{yu4 ce4}[][HSK 4]
    \definition{v.}{prever; prognosticar; predizer}
  \end{phonetics}
\end{entry}

\begin{entry}{预祝}{10,9}{⾴、⽰}
  \begin{phonetics}{预祝}{yu4zhu4}
    \definition{v.}{parabenizar de antemão | oferecer os melhores votos para}
  \end{phonetics}
\end{entry}

\begin{entry}{预览}{10,9}{⾴、⾒}
  \begin{phonetics}{预览}{yu4lan3}
    \definition{s.}{visualização}
    \definition{v.}{visualizar}
  \end{phonetics}
\end{entry}

\begin{entry}{预留}{10,10}{⾴、⽥}
  \begin{phonetics}{预留}{yu4liu2}
    \definition{v.}{separar | reservar}
  \end{phonetics}
\end{entry}

\begin{entry}{预配}{10,10}{⾴、⾣}
  \begin{phonetics}{预配}{yu4pei4}
    \definition{s.}{pré-alocado | pré-cabeado}
    \definition{v.}{pré-alocar | pré-cabear}
  \end{phonetics}
\end{entry}

\begin{entry}{预谋}{10,11}{⾴、⾔}
  \begin{phonetics}{预谋}{yu4mou2}
    \definition{adj.}{premeditado}
    \definition{v.}{planejar algo com antecedência (especialmente um crime)}
  \end{phonetics}
\end{entry}

\begin{entry}{预提}{10,12}{⾴、⼿}
  \begin{phonetics}{预提}{yu4ti2}
    \definition{s.}{retenção}
    \definition{v.}{reter (imposto)}
  \end{phonetics}
\end{entry}

\begin{entry}{预感}{10,13}{⾴、⼼}
  \begin{phonetics}{预感}{yu4gan3}
    \definition{s.}{premonição}
    \definition{v.}{ter uma premonição}
  \end{phonetics}
\end{entry}

\begin{entry}{预警}{10,19}{⾴、⾔}
  \begin{phonetics}{预警}{yu4jing3}
    \definition{s.}{aviso | aviso antecipado}
  \end{phonetics}
\end{entry}

\begin{entry}{饿}{10}{⾷}
  \begin{phonetics}{饿}{e4}[][HSK 1]
    \definition{adj.}{faminto}
    \definition{s.}{fome}
    \definition{v.}{morrer de fome}
  \end{phonetics}
\end{entry}

\begin{entry}{高}{10}{⾼}[Kangxi 189]
  \begin{phonetics}{高}{gao1}[][HSK 1]
    \definition*{s.}{sobrenome Gao}
    \definition{adj.}{alto | acima da média}
    \definition{pron.}{Seu (honorífico)}
  \end{phonetics}
\end{entry}

\begin{entry}{高于}{10,3}{⾼、⼆}
  \begin{phonetics}{高于}{gao1 yu2}[][HSK 5]
    \definition{v.}{ser mais alto do que; sobrepujar}
  \end{phonetics}
\end{entry}

\begin{entry}{高大}{10,3}{⾼、⼤}
  \begin{phonetics}{高大}{gao1 da4}[][HSK 5]
    \definition{adj.}{alto e grande; alto | elevado; sublime; nobre}
  \end{phonetics}
\end{entry}

\begin{entry}{高中}{10,4}{⾼、⼁}
  \begin{phonetics}{高中}{gao1 zhong1}[][HSK 2]
    \definition{s.}{escola secundária | escola de segundo grau}
  \end{phonetics}
\end{entry}

\begin{entry}{高手}{10,4}{⾼、⼿}
  \begin{phonetics}{高手}{gao1shou3}
    \definition{s.}{\emph{expert} | mestre}
  \end{phonetics}
\end{entry}

\begin{entry}{高尔夫}{10,5,4}{⾼、⼩、⼤}
  \begin{phonetics}{高尔夫}{gao1'er3fu1}
    \definition{s.}{(empréstimo linguístico) \emph{golf}}
  \end{phonetics}
\end{entry}

\begin{entry}{高价}{10,6}{⾼、⼈}
  \begin{phonetics}{高价}{gao1 jia4}[][HSK 4]
    \definition{s.}{preço alto; bilhete caro; custo elevado; dispendioso}
  \end{phonetics}
\end{entry}

\begin{entry}{高兴}{10,6}{⾼、⼋}
  \begin{phonetics}{高兴}{gao1xing4}[][HSK 1]
    \definition{adj.}{feliz | contente | disposto (a fazer alguma coisa) | de bom humor}
  \end{phonetics}
\end{entry}

\begin{entry}{高级}{10,6}{⾼、⽷}
  \begin{phonetics}{高级}{gao1ji2}[][HSK 2]
    \definition{adj.}{sênior | alto escalão | alto nível | alto grau | grau superior | alta qualidade | avançado}
  \end{phonetics}
\end{entry}

\begin{entry}{高尚}{10,8}{⾼、⼩}
  \begin{phonetics}{高尚}{gao1shang4}[][HSK 4]
    \definition{adj.}{nobre; elevado; descreve um alto padrão moral e uma boa qualidade de pensamento | significativo e não de mau gosto}
  \end{phonetics}
\end{entry}

\begin{entry}{高度}{10,9}{⾼、⼴}
  \begin{phonetics}{高度}{gao1 du4}[][HSK 5]
    \definition{adj.}{alto; elevado; avançado; alto grau | alta concentração; intenso}
    \definition[个]{s.}{altura; altitude; elevação; distância de baixo para cima; o grau e o nível em que as coisas se desenvolveram}
  \end{phonetics}
\end{entry}

\begin{entry}{高原}{10,10}{⾼、⼚}
  \begin{phonetics}{高原}{gao1 yuan2}[][HSK 5]
    \definition[片]{s.}{planalto continental; planalto | platô}
  \end{phonetics}
\end{entry}

\begin{entry}{高效}{10,10}{⾼、⽁}
  \begin{phonetics}{高效}{gao1xiao4}
    \definition{adj.}{eficiente | altamente eficaz}
  \end{phonetics}
\end{entry}

\begin{entry}{高速}{10,10}{⾼、⾡}
  \begin{phonetics}{高速}{gao1 su4}[][HSK 3]
    \definition{adj.}{alta velocidade}
    \definition{s.}{auto-estrada; via expressa}
  \end{phonetics}
\end{entry}

\begin{entry}{高速公路}{10,10,4,13}{⾼、⾡、⼋、⾜}
  \begin{phonetics}{高速公路}{gao1su4gong1lu4}[][HSK 3]
    \definition[条]{s.}{via expressa; rodovia; auto-estrada}
  \end{phonetics}
\end{entry}

\begin{entry}{高铁}{10,10}{⾼、⾦}
  \begin{phonetics}{高铁}{gao1 tie3}[][HSK 4]
    \definition{s.}{trem de alta velocidade; trem bala}
  \end{phonetics}
\end{entry}

\begin{entry}{高温}{10,12}{⾼、⽔}
  \begin{phonetics}{高温}{gao1 wen1}[][HSK 5]
    \definition{s.}{alta temperatura; temperatura elevada; hipertermia; megatemperatura; inferno;}
  \end{phonetics}
\end{entry}

\begin{entry}{高楼}{10,13}{⾼、⽊}
  \begin{phonetics}{高楼}{gao1lou2}
    \definition[座]{s.}{edifício alto | edifício de muitos andares | arranha-céu}
  \end{phonetics}
\end{entry}

\begin{entry}{高跟鞋}{10,13,15}{⾼、⾜、⾰}
  \begin{phonetics}{高跟鞋}{gao1 gen1 xie2}[][HSK 5]
    \definition{s.}{salto alto; sapatos de salto alto; sapato feminino com salto mais alto e mais distante do chão}
  \end{phonetics}
\end{entry}

\begin{entry}{高潮}{10,15}{⾼、⽔}
  \begin{phonetics}{高潮}{gao1chao2}[][HSK 4]
    \definition[个,场]{s.}{maré alta; o nível mais alto da maré em um ciclo de maré | pico; aumento; maré alta; uma metáfora para o estágio mais próspero de desenvolvimento das coisas (diferente de ``低潮'') | (ficção, drama e filmes) clímax}
  \seealsoref{低潮}{di1chao2}
  \end{phonetics}
\end{entry}

\begin{entry}{髟}{10}{⾽}[Kangxi 190]
  \begin{phonetics}{髟}{biao1}
    \definition{adj.}{cabelo solto, caído}
  \end{phonetics}
\end{entry}

\begin{entry}{鬯}{10}{⾿}[Kangxi 192]
  \begin{phonetics}{鬯}{chang4}
    \definition{s.}{um antigo vinho para sacrifícios | estojo para arco | o mesmo que ``畅''}
  \seealsoref{畅}{chang4}
  \end{phonetics}
\end{entry}

\begin{entry}{鬲}{10}{⿀}[Kangxi 193]
  \begin{phonetics}{鬲}{ge2}
    \definition{s.}{um antigo tripé de cozinha com pernas ocas; uma grande panela de barro}
  \end{phonetics}
  \begin{phonetics}{鬲}{li4}
    \definition{s.}{um antigo tripé de cozinha com pernas ocas; uma grande panela de barro}
  \end{phonetics}
\end{entry}

\begin{entry}{鸭}{10}{⿃}
  \begin{phonetics}{鸭}{ya1}
    \definition[只]{s.}{pato | (gíria) prostituto}
  \end{phonetics}
\end{entry}

\begin{entry}{鸭子}{10,3}{⿃、⼦}
  \begin{phonetics}{鸭子}{ya1zi5}
    \definition[只]{s.}{pato | (gíria) prostituto}
  \end{phonetics}
\end{entry}

\begin{entry}{鸵鸟}{10,5}{⿃、⿃}
  \begin{phonetics}{鸵鸟}{tuo2niao3}
    \definition{s.}{avestruz}
  \end{phonetics}
\end{entry}

%%%%% EOF %%%%%

