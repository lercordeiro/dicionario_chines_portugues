%%%
%%% 10画
%%%

\section*{10画}\addcontentsline{toc}{section}{10画}

\begin{Entry}{乘}{10}{⽲}
  \begin{Phonetics}{乘}{cheng2}[][HSK 5]
    \definition*{s.}{Sobrenome Cheng}
    \definition{s.}{uma divisão principal das escolas budistas; uma seita ou doutrina do budismo}
    \definition{v.}{cavalgar; andar a cavalo; utilizar um veículo ou animal em vez de caminhar | aproveitar-se de; valer-se de; tirar vantagem de; tirar proveito de | multiplicar; realizar multiplicação | perseguir; caçar}
  \end{Phonetics}
  \begin{Phonetics}{乘}{sheng4}
    \definition{clas.}{usado para carruagens de guerra puxada por quatro cavalos}
    \definition{s.}{obras históricas; livros de história geral | antigamente, uma carruagem puxada por quatro cavalos}
  \end{Phonetics}
\end{Entry}

\begin{Entry}{乘人之危}{10,2,3,6}{⽲、⼈、⼂、⼙}
  \begin{Phonetics}{乘人之危}{cheng2ren2zhi1wei1}[][HSK 7-9]
    \definition{expr.}{tirar vantagem das dificuldades dos outros (posição precária; problema); capitalizar as dificuldades de alguém (desastres); fazer uso (utilizar) da situação precária em que alguém se encontra; tentar usar o dilema de alguém para\dots; aproveitar-se da angústia dos outros para prejudicá-los; atacar em um momento de crise}
  \end{Phonetics}
\end{Entry}

\begin{Entry}{乘车}{10,4}{⽲、⾞}
  \begin{Phonetics}{乘车}{cheng2 che1}[][HSK 5]
    \definition{v.}{montar; dirigir; conduzir; andar a cavalo, de moto, de bicicleta, etc.}
  \end{Phonetics}
\end{Entry}

\begin{Entry}{乘坐}{10,7}{⽲、⼟}
  \begin{Phonetics}{乘坐}{cheng2zuo4}[][HSK 5]
    \definition{v.}{pegar (um trem, ônibus, etc.); andar de (bicicleta, moto, etc.)}
  \end{Phonetics}
\end{Entry}

\begin{Entry}{乘客}{10,9}{⽲、⼧}
  \begin{Phonetics}{乘客}{cheng2 ke4}[][HSK 5]
    \definition[个,位,名]{s.}{passageiro; pessoas viajando de carro, navio ou avião}
  \end{Phonetics}
\end{Entry}

\begin{Entry}{乘客数}{10,9,13}{⽲、⼧、⽁}
  \begin{Phonetics}{乘客数}{cheng2ke4 shu4}
    \definition{s.}{número de passageiros}
  \end{Phonetics}
\end{Entry}

\begin{Entry}{乘积}{10,10}{⽲、⽲}
  \begin{Phonetics}{乘积}{cheng2ji1}
    \definition{s.}{(matemática) produto (resultado da multiplicação)}
  \end{Phonetics}
\end{Entry}

\begin{Entry}{俯}{10}{⼈}
  \begin{Phonetics}{俯}{fu3}
    \definition{v.}{curvar (a cabeça) (oposto a 仰) | inclinar-se | (datado, em documentos ou cartas oficiais) condescender com | curvar-se; fazer uma reverência}
  \seealsoref{仰}{yang3}
  \end{Phonetics}
\end{Entry}

\begin{Entry}{俯首}{10,9}{⼈、⾸}
  \begin{Phonetics}{俯首}{fu3shou3}[][HSK 7-9]
    \definition{v.}{abaixar a cabeça; curvar-se; inclinar-se}
  \end{Phonetics}
\end{Entry}

\begin{Entry}{俱}{10}{⼈}
  \begin{Phonetics}{俱}{ju4}
    \definition{adv.}{(literário) tudo; completamente; inteiramente}
  \end{Phonetics}
\end{Entry}

\begin{Entry}{俱乐部}{10,5,10}{⼈、⼃、⾢}
  \begin{Phonetics}{俱乐部}{ju4le4bu4}[][HSK 5]
    \definition[个,家,间]{s.}{clube; grupos e locais para atividades sociais, políticas, literárias, recreativas e outras}
  \end{Phonetics}
\end{Entry}

\begin{Entry}{倂}{10}{⼈}
  \begin{Phonetics}{倂}{bing4}
    \variantof{并}
  \end{Phonetics}
\end{Entry}

\begin{Entry}{倍}{10}{⼈}
  \begin{Phonetics}{倍}{bei4}[][HSK 4]
    \definition{adv.}{ainda mais; especialmente | (antes de certos adjetivos) muito; particularmente; é pronunciado como um som erhua e é usado antes de certos adjetivos para expressar um alto grau de profundidade, equivalente a 非常 ou 特别}
    \definition{clas.}{vezes; usado após um numeral, significa que o valor anterior é multiplicado por este número}[增长了五倍。===Aumentou cinco vezes. | 二的三倍是六。===Três vezes dois é seis.]
  \seealsoref{非常}{fei1chang2}
  \seealsoref{特别}{te4bie2}
  \end{Phonetics}
\end{Entry}

\begin{Entry}{倒}{10}{⼈}
  \begin{Phonetics}{倒}{dao3}[][HSK 2]
    \definition{v.}{cair; tombar | falhar; entrar em colapso | ficar rouco | mudar; trocar; transferir; converter | movimentar-se; manobrar | oferecer (casa, loja) para venda; vender mercadorias ou lojas a terceiros a um preço fixo | derrubar; derrubar com}
  \end{Phonetics}
  \begin{Phonetics}{倒}{dao4}[][HSK 2]
    \definition{adj.}{inverso; invertido; de cabeça para baixo}
    \definition{adv.}{mas; pelo contrário; expressa o contrário do esperado, equivalente a 反倒 | indicando que algo não é o que se pensa; indica que as coisas não são assim | usado para indicar uma transição ou concessão | transmitindo a sensação de ``urgência''; expressa pressa ou insistência, com um tom impaciente}
    \definition{v.}{ser inverso; estar invertido; estar de cabeça para baixo; inverter a posição original para cima e para baixo ou para a frente e para trás | recuar; virar de cabeça para baixo; fazer mover na direção oposta ou inverter | inclinar ou virar o recipiente para retirar o conteúdo; inclinar; derramar}
  \seealsoref{反倒}{fan3dao4}
  \end{Phonetics}
\end{Entry}

\begin{Entry}{倒下}{10,3}{⼈、⼀}
  \begin{Phonetics}{倒下}{dao3xia4}[][HSK 7-9]
    \definition{v.}{entrar em colapso | tombar}
  \end{Phonetics}
\end{Entry}

\begin{Entry}{倒计时}{10,4,7}{⼈、⾔、⽇}
  \begin{Phonetics}{倒计时}{dao4ji4shi2}[][HSK 7-9]
    \definition{s.}{contagem regressiva; contagem de tempo a partir de um determinado ponto no futuro até o presente, de mais para menos, até que o tempo chegue a zero; é frequentemente usado para expressar que um certo momento está se aproximando}
  \end{Phonetics}
\end{Entry}

\begin{Entry}{倒车}{10,4}{⼈、⾞}
  \begin{Phonetics}{倒车}{dao3che1}[][HSK 4]
    \definition{v.}{trocar de trem ou ônibus (no meio do caminho)}
  \end{Phonetics}
  \begin{Phonetics}{倒车}{dao4che1}[][HSK 4]
    \definition{v.}{dar marcha à ré (em um veículo)}
  \end{Phonetics}
\end{Entry}

\begin{Entry}{倒地}{10,6}{⼈、⼟}
  \begin{Phonetics}{倒地}{dao3di4}
    \definition{v.}{cair no chão}
  \end{Phonetics}
\end{Entry}

\begin{Entry}{倒血霉}{10,6,15}{⼈、⾎、⾬}
  \begin{Phonetics}{倒血霉}{dao3xue4mei2}
    \definition{v.}{ter muito azar (versão mais forte de 倒霉)}
  \seealsoref{倒霉}{dao3/mei2}
  \end{Phonetics}
\end{Entry}

\begin{Entry}{倒闭}{10,6}{⼈、⾨}
  \begin{Phonetics}{倒闭}{dao3bi4}[][HSK 4]
    \definition{v.}{fechar; ir à falência; entrar em liquidação; sair do negócio; (empresa, loja ou banco) deixar de operar devido ao baixo desempenho}
  \end{Phonetics}
\end{Entry}

\begin{Entry}{倒卖}{10,8}{⼈、⼗}
  \begin{Phonetics}{倒卖}{dao3mai4}[][HSK 7-9]
    \definition{v.}{revender com lucro}
  \end{Phonetics}
\end{Entry}

\begin{Entry}{倒是}{10,9}{⼈、⽇}
  \begin{Phonetics}{倒是}{dao4 shi4}[][HSK 5]
    \definition{adv.}{usado para indicar o oposto do que geralmente é verdade; ao contrário do senso comum; pelo contrário | usado para indicar o que é contrário aos fatos, com um toque de crítica; indica que as coisas não são assim (com um sentimento de culpa) | usado de algo inesperado; expressando surpresa | usado para indicar concessão | usado para indicar uma mudança de significado; indica um ponto de virada | usado para modificar ou suavizar uma declaração anterior; para suavizar o tom | usado para pressionar ou questionar alguém; para instar ou perguntar}
  \end{Phonetics}
\end{Entry}

\begin{Entry}{倒塌}{10,13}{⼈、⼟}
  \begin{Phonetics}{倒塌}{dao3ta1}[][HSK 7-9]
    \definition{v.}{colapsar; desabar}
  \end{Phonetics}
\end{Entry}

\begin{Entry}{倒数}{10,13}{⼈、⽁}
  \begin{Phonetics}{倒数}{dao4shu3}[][HSK 7-9]
    \definition{v.}{contar de trás para frente (contagem regressiva)}
  \end{Phonetics}
  \begin{Phonetics}{倒数}{dao4shu4}
    \definition{s.}{número inverso; Matemática: recíproco}
  \end{Phonetics}
\end{Entry}

\begin{Entry}{倒楣}{10,13}{⼈、⽊}
  \begin{Phonetics}{倒楣}{dao3mei2}
    \definition{adj.}{azarado; infeliz; tendo má sorte}
  \end{Phonetics}
\end{Entry}

\begin{Entry}{倒霉}{10,15}{⼈、⾬}
  \begin{Phonetics}{倒霉}{dao3/mei2}[][HSK 7-9]
    \definition{adj.}{azarado}
    \definition{s.}{azar; má sorte}
    \definition{v.+compl.}{cair em dias maus; cair em tempos difíceis; encontrar coisas desfavoráveis; ter má sorte}
  \seealsoref{倒血霉}{dao3xue4mei2}
  \end{Phonetics}
\end{Entry}

\begin{Entry}{倘}{10}{⼈}
  \begin{Phonetics}{倘}{chang2}
  \end{Phonetics}
  \begin{Phonetics}{倘}{tang3}
    \definition{conj.}{se; supondo; no caso}
  \end{Phonetics}
\end{Entry}

\begin{Entry}{倘使}{10,8}{⼈、⼈}
  \begin{Phonetics}{倘使}{tang3shi3}
    \definition{conj.}{se | supondo que | no caso}
  \end{Phonetics}
\end{Entry}

\begin{Entry}{倘或}{10,8}{⼈、⼽}
  \begin{Phonetics}{倘或}{tang3huo4}
    \definition{conj.}{se | supondo que | no caso}
  \end{Phonetics}
\end{Entry}

\begin{Entry}{倘若}{10,8}{⼈、⾋}
  \begin{Phonetics}{倘若}{tang3ruo4}
    \definition{conj.}{se | supondo que | no caso}
  \end{Phonetics}
\end{Entry}

\begin{Entry}{候}{10}{⼈}
  \begin{Phonetics}{候}{hou4}
    \definition{s.}{tempo; estação | condição; estado | situação meteorológica | uma unidade tradicional de tempo no antigo calendário chinês; antigamente, cinco dias constituíam uma estação, o que ainda é usado na meteorologia hoje em dia}
    \definition{s.}{Sobrenome Hou}
    \definition{v.}{esperar; aguardar | perguntar depois | assistir; observar}
  \end{Phonetics}
\end{Entry}

\begin{Entry}{候选人}{10,9,2}{⼈、⾡、⼈}
  \begin{Phonetics}{候选人}{hou4xuan3ren2}[][HSK 7-9]
    \definition[个,名,位]{s.}{candidato}
  \end{Phonetics}
\end{Entry}

\begin{Entry}{借}{10}{⼈}
  \begin{Phonetics}{借}{jie4}[][HSK 2]
    \definition{adv.}{por meio de}
    \definition{v.}{emprestar | pedir emprestado | usar como pretexto | aproveitar; tirar proveito (de uma oportunidade, etc.)}
  \end{Phonetics}
\end{Entry}

\begin{Entry}{借书证}{10,4,7}{⼈、⼄、⾔}
  \begin{Phonetics}{借书证}{jie4shu1zheng4}
    \definition{s.}{cartão de biblioteca | (literalmente) cartão para pedir emprestado livros}
  \end{Phonetics}
\end{Entry}

\begin{Entry}{借用}{10,5}{⼈、⽤}
  \begin{Phonetics}{借用}{jie4yong4}
    \definition{v.}{tomar emprestado; ter o empréstimo de | usar algo para outro propósito}
  \end{Phonetics}
\end{Entry}

\begin{Entry}{借鉴}{10,13}{⼈、⾦}
  \begin{Phonetics}{借鉴}{jie4jian4}[][HSK 6]
    \definition{s.}{tirar lições de; aproveitar a experiência de; ganhar experiência e lições com o passado ou com as experiências de outras pessoas}
  \end{Phonetics}
\end{Entry}

\begin{Entry}{倡}{10}{⼈}
  \begin{Phonetics}{倡}{chang4}
    \definition{v.}{iniciar; propor; defender | promover; assumir a liderança}
  \end{Phonetics}
\end{Entry}

\begin{Entry}{倡议}{10,5}{⼈、⾔}
  \begin{Phonetics}{倡议}{chang4yi4}[][HSK 7-9]
    \definition[项,条,次]{s.}{proposta; iniciativa; primeiras sugestões}
    \definition{v.}{propor; iniciar; defender}
  \end{Phonetics}
\end{Entry}

\begin{Entry}{倡导}{10,6}{⼈、⼨}
  \begin{Phonetics}{倡导}{chang4dao3}[][HSK 5]
    \definition{v.}{iniciar; propor; promover; defender; advogar}
  \end{Phonetics}
\end{Entry}

\begin{Entry}{债}{10}{⼈}
  \begin{Phonetics}{债}{zhai4}[][HSK 6]
    \definition[笔]{s.}{dívida | empréstimo}
  \end{Phonetics}
\end{Entry}

\begin{Entry}{值}{10}{⼈}
  \begin{Phonetics}{值}{zhi2}[][HSK 3]
    \definition{adj.}{significativo; valioso; digno de nota}
    \definition{prep.}{quando; introduz o momento em que algo acontece ou existe, equivalente a 当 ou 在}
    \definition{s.}{preço; valor | valor de um número, de uma variável}
    \definition{v.}{valer; custar; a mercadoria é adequada ao preço | ir de encontro; encontrar; cruzar | estar de serviço; ter sua vez em algo; assumir o cargo que lhe cabe | é a vez de (executar uma determinada função pública)}
  \seealsoref{当}{dang1}
  \seealsoref{在}{zai4}
  \end{Phonetics}
\end{Entry}

\begin{Entry}{值班}{10,10}{⼈、⽟}
  \begin{Phonetics}{值班}{zhi2ban1}[][HSK 5]
    \definition{v.}{estar em serviço ou plantão; trabalhar em um turno; (em rodízio) desempenhar funções durante um período de tempo determinado}
  \end{Phonetics}
\end{Entry}

\begin{Entry}{值得}{10,11}{⼈、⼻}
  \begin{Phonetics}{值得}{zhi2de5}[][HSK 3]
    \definition{adj.}{que tem valor; (fazer algo) é vantajoso, sem prejuízos}
    \definition{v.}{merecer; ter valor; significa que fazer isso terá bons resultados; que é valioso e significativo}
  \end{Phonetics}
\end{Entry}

\begin{Entry}{倾}{10}{⼈}
  \begin{Phonetics}{倾}{qing1}
    \definition{s.}{desvio; tendência}
    \definition{v.}{inclinar; inclinar-se; dobrar-se | colapsar | virar e despejar; esvaziar | fazer tudo o que puder; usar todos os recursos | sobrecarregar; dominar; dominar | admirar | superar}
  \end{Phonetics}
\end{Entry}

\begin{Entry}{倾向}{10,6}{⼈、⼝}
  \begin{Phonetics}{倾向}{qing1xiang4}[][HSK 6]
    \definition{s.}{tendência; desvio; inclinação; direção do desenvolvimento}
    \definition{v.}{preferir; estar inclinado a; concordar com uma determinada opinião}
  \end{Phonetics}
\end{Entry}

\begin{Entry}{倾城}{10,9}{⼈、⼟}
  \begin{Phonetics}{倾城}{qing1cheng2}
    \definition{adj.}{sedutora (mulher)}
    \definition{adv.}{de todo o lugar | vindo de todos os lugares}
    \definition{v.}{arruinar e derrubar o estado}
  \end{Phonetics}
\end{Entry}

\begin{Entry}{健}{10}{⼈}
  \begin{Phonetics}{健}{jian4}
    \definition{adj.}{forte; saudável; bem definido | ser forte em; ser bom em; apresentar um grau superior à média em determinado aspecto}
    \definition{v.}{fortalecer; endurecer; revigorar}
  \end{Phonetics}
\end{Entry}

\begin{Entry}{健全}{10,6}{⼈、⼊}
  \begin{Phonetics}{健全}{jian4quan2}[][HSK 5]
    \definition{adj.}{saudável; íntegro; capaz; apto; robusto e sem mácula | sólido; completo; perfeito}
    \definition{v.}{aperfeiçoar; melhorar; fortalecer; reforçar}
  \end{Phonetics}
\end{Entry}

\begin{Entry}{健身}{10,7}{⼈、⾝}
  \begin{Phonetics}{健身}{jian4/shen1}[][HSK 4]
    \definition{s.}{exercício físico | \emph{fitness}}
    \definition{v.+compl.}{exercitar-se; manter a forma; praticar um esporte, especialmente a ginástica, inclusive em aparelhos, para desenvolver força, flexibilidade, aumentar a resistência, melhorar a coordenação e o controle de todas as partes do corpo}
  \end{Phonetics}
\end{Entry}

\begin{Entry}{健康}{10,11}{⼈、⼴}
  \begin{Phonetics}{健康}{jian4kang1}[][HSK 2]
    \definition{adj.}{em forma; saudável; descreve que a pessoa está em ótimo estado físico ou mental, sem nenhum problema | sudável; tudo está normal, sem problemas | saudável; livre de doenças; bom para a saúde}
    \definition{s.}{saúde; físico; estado de saúde}
  \end{Phonetics}
\end{Entry}

\begin{Entry}{党}{10}{⼉}
  \begin{Phonetics}{党}{dang3}[][HSK 6]
    \definition*{s.}{O Partido (Partido Comunista da China) | Sobrenome Dang}
    \definition{s.}{partido político; partido | camarilha; facção; gangue | Datado: parentes}
    \definition{v.}{ser parcial; tomar partido de}
  \end{Phonetics}
\end{Entry}

\begin{Entry}{兼}{10}{⼋}
  \begin{Phonetics}{兼}{jian1}
    \definition{conj.}{e (ocupando dois ou mais cargos (oficiais) ao mesmo tempo)}
  \end{Phonetics}
\end{Entry}

\begin{Entry}{准}{10}{⼎}
  \begin{Phonetics}{准}{zhun3}[][HSK 3]
    \definition{adj.}{exato; preciso; algo determinado a ser imutável | preciso; exato; correto | perto; parcialmente; quase; próximo de algo em termos de padrão}
    \definition{adv.}{definitivamente; certamente}
    \definition{pref.}{quasi-; para-}
    \definition{prep.}{de acordo com; baseado em}
    \definition{s.}{norma; padrão; critério | confiança certa; uma ideia definida, certeza, etc. (geralmente usada depois de 有 ou 没有)}
    \definition{v.}{autorizar; conceder; consentir; permitir}
  \seealsoref{没有}{mei2 you3}
  \seealsoref{有}{you3}
  \end{Phonetics}
\end{Entry}

\begin{Entry}{准时}{10,7}{⼎、⽇}
  \begin{Phonetics}{准时}{zhun3shi2}[][HSK 4]
    \definition{adj.}{pontual}
    \definition{adv.}{na hora certa; dentro do prazo; no horário especificado}
  \end{Phonetics}
\end{Entry}

\begin{Entry}{准备}{10,8}{⼎、⼡}
  \begin{Phonetics}{准备}{zhun3bei4}[][HSK 1]
    \definition{v.}{preparar; ficar pronto; planejar ou organizar com antecedência | pretender; planejar}
  \end{Phonetics}
\end{Entry}

\begin{Entry}{准确}{10,12}{⼎、⽯}
  \begin{Phonetics}{准确}{zhun3que4}[][HSK 2]
    \definition{adj.}{exato; preciso; acurado; os resultados da ação são completamente consistentes com os resultados reais ou esperados}
  \end{Phonetics}
\end{Entry}

\begin{Entry}{凉}{10}{⼎}
  \begin{Phonetics}{凉}{liang2}[][HSK 2]
    \definition{adj.}{frio; gelado; ligeiramente fria (menos do que 冷) | sombrio; desolado; sem animação | desanimado; desapontado | usado para prevenir o calor e manter a temperatura amena; para proteção contra o calor}
    \definition{s.}{frio; refere-se a um ambiente fresco ou a uma brisa fresca}
  \seealsoref{冷}{leng3}
  \end{Phonetics}
  \begin{Phonetics}{凉}{liang4}
    \definition{v.}{deixar algo esfriar; deixar um objeto quente descansar por um tempo para que a temperatura diminua}
  \end{Phonetics}
\end{Entry}

\begin{Entry}{凉水}{10,4}{⼎、⽔}
  \begin{Phonetics}{凉水}{liang2 shui3}[][HSK 3]
    \definition{s.}{água fria; água não aquecida | água não fervida}
  \end{Phonetics}
\end{Entry}

\begin{Entry}{凉快}{10,7}{⼎、⼼}
  \begin{Phonetics}{凉快}{liang2kuai5}[][HSK 2]
    \definition{adj.}{fresco; refrescante}
    \definition{v.}{refrescar; refrescar-se; deixar o corpo fresco e revigorado}
  \end{Phonetics}
\end{Entry}

\begin{Entry}{凉鞋}{10,15}{⼎、⾰}
  \begin{Phonetics}{凉鞋}{liang2 xie2}[][HSK 6]
    \definition[双,只]{s.}{sandália; alpargata; alpercata; alparca ; sapatos de verão com cabedal ventilado}
  \end{Phonetics}
\end{Entry}

\begin{Entry}{剥}{10}{⼑}
  \begin{Phonetics}{剥}{bao1}[][HSK 7-9]
    \definition{v.}{descascar; despelar; remover a casca ou pele externa}
  \end{Phonetics}
  \begin{Phonetics}{剥}{bo1}
    \definition{v.}{Dialeto: descascar; despelar; remover a casca ou pele externa | (pele, tinta, etc.) sair; descascar | privar; explorar}
  \end{Phonetics}
\end{Entry}

\begin{Entry}{剥夺}{10,6}{⼑、⼤}
  \begin{Phonetics}{剥夺}{bo1duo2}[][HSK 7-9]
    \definition{v.}{roubar algo de alguém; tirar algo de alguém à força | privar; privar por lei; cancelar de acordo com a lei}
  \end{Phonetics}
\end{Entry}

\begin{Entry}{剥削}{10,9}{⼑、⼑}
  \begin{Phonetics}{剥削}{bo1xue1}[][HSK 7-9]
    \definition{v.}{explorar; apropriar-se do trabalho ou dos frutos do trabalho de outrem sem remuneração}
  \end{Phonetics}
\end{Entry}

\begin{Entry}{剧}{10}{⼑}
  \begin{Phonetics}{剧}{ju4}[][HSK 6]
    \definition*{s.}{Sobrenome Ju}
    \definition{adj.}{agudo; grave; intenso; violento}
    \definition[部,个,种]{s.}{obra teatral; drama; peça; ópera}
  \end{Phonetics}
\end{Entry}

\begin{Entry}{剧本}{10,5}{⼑、⽊}
  \begin{Phonetics}{剧本}{ju4ben3}[][HSK 5]
    \definition[部,个]{s.}{cenário; roteiro (para drama, filme, etc.); gênero de obra literária que consiste em diálogos entre personagens (às vezes cantados) e indicações de palco}
  \end{Phonetics}
\end{Entry}

\begin{Entry}{剧场}{10,6}{⼑、⼟}
  \begin{Phonetics}{剧场}{ju4 chang3}[][HSK 3]
    \definition[个,坐]{s.}{teatro; local para apresentações teatrais, musicais, etc.}
  \end{Phonetics}
\end{Entry}

\begin{Entry}{原}{10}{⼚}
  \begin{Phonetics}{原}{yuan2}[][HSK 6]
    \definition*{s.}{Sobrenome Yuan}
    \definition{adj.}{inicial; básico; primitivo | cru; bruto; não processado | virgem; primário; original; antigo; inalterado}
    \definition{adv.}{originalmente}
    \definition[项,条,片]{s.}{planície; país aberto; terreno plano e amplo | início; fonte; origem; aparência original | origem; a raiz ou o começo das coisas}
    \definition{v.}{desculpar; perdoar; tolerar; compreender | rastrear; sondar; investigar (a origem das coisas)}
  \end{Phonetics}
\end{Entry}

\begin{Entry}{原木}{10,4}{⼚、⽊}
  \begin{Phonetics}{原木}{yuan2mu4}
    \definition{s.}{registro | \emph{logs}}
  \end{Phonetics}
\end{Entry}

\begin{Entry}{原先}{10,6}{⼚、⼉}
  \begin{Phonetics}{原先}{yuan2xian1}[][HSK 5]
    \definition{adj.}{antigo; original}
    \definition{s.}{antigamente; no início; no passado; no começo}
  \end{Phonetics}
\end{Entry}

\begin{Entry}{原则}{10,6}{⼚、⼑}
  \begin{Phonetics}{原则}{yuan2ze2}[][HSK 4]
    \definition{adv.}{em geral; em princípio; refere-se a um aspecto geral; geralmente}
    \definition[个,条,项,点]{s.}{princípios; leis ou padrões pelos quais alguém fala ou age}
  \end{Phonetics}
\end{Entry}

\begin{Entry}{原因}{10,6}{⼚、⼞}
  \begin{Phonetics}{原因}{yuan2yin1}[][HSK 2]
    \definition[个,条,种,些]{s.}{causa; razão; motivo; as condições que fazem com que algo aconteça ou produzam um certo resultado}
  \end{Phonetics}
\end{Entry}

\begin{Entry}{原有}{10,6}{⼚、⽉}
  \begin{Phonetics}{原有}{yuan2 you3}[][HSK 5]
    \definition{v.}{já estar pronto, não é necessário fazer ou procurar nada; ser o original}
  \end{Phonetics}
\end{Entry}

\begin{Entry}{原色}{10,6}{⼚、⾊}
  \begin{Phonetics}{原色}{yuan2 se4}
    \definition{s.}{cor primária}
  \end{Phonetics}
\end{Entry}

\begin{Entry}{原告}{10,7}{⼚、⼝}
  \begin{Phonetics}{原告}{yuan2gao4}[][HSK 6]
    \definition{s.}{(em casos civis) autor; solicitante | (em casos criminais) promotor; acusador; reclamante (oposto a 被告)}
  \seealsoref{被告}{bei4gao4}
  \end{Phonetics}
\end{Entry}

\begin{Entry}{原来}{10,7}{⼚、⽊}
  \begin{Phonetics}{原来}{yuan2lai2}[][HSK 2]
    \definition{adj.}{original; anterior; em primeiro lugar; inicialmente; inalterado}
    \definition{adv.}{na verdade; de fato; como se vê; expressar compreensão repentina}
    \definition{s.}{a princípio; no passado; antigamente}
  \end{Phonetics}
\end{Entry}

\begin{Entry}{原始}{10,8}{⼚、⼥}
  \begin{Phonetics}{原始}{yuan2shi3}[][HSK 5]
    \definition{s.}{original; de primeira mão | primitivo; mais antigo; não desenvolvido; não civilizado}
  \end{Phonetics}
\end{Entry}

\begin{Entry}{原料}{10,10}{⼚、⽃}
  \begin{Phonetics}{原料}{yuan2liao4}[][HSK 4]
    \definition[种,个]{s.}{matéria-prima; refere-se a materiais que não foram processados e fabricados, como minérios para metalurgia e algodão para têxteis}
  \end{Phonetics}
\end{Entry}

\begin{Entry}{原谅}{10,10}{⼚、⾔}
  \begin{Phonetics}{原谅}{yuan2liang4}[][HSK 6]
    \definition{v.}{perdoar; perdoar a negligência, os erros ou as falhas das pessoas sem culpá-las ou puni-las}
  \end{Phonetics}
\end{Entry}

\begin{Entry}{原理}{10,11}{⼚、⽟}
  \begin{Phonetics}{原理}{yuan2li3}[][HSK 5]
    \definition[个,条]{s.}{princípio; axioma; teoria; teoria básica ou princípio científico de significado universal}
  \end{Phonetics}
\end{Entry}

\begin{Entry}{哥}{10}{⼝}
  \begin{Phonetics}{哥}{ge1}[][HSK 1]
    \definition[个,位,名,些]{s.}{irmão mais velho | forma de se dirigir a um parente masculino mais velho de sua geração | irmão; termo amigável para se dirigir a conhecidos mais velhos do sexo masculino}
  \seealsoref{哥哥}{ge1 ge5}
  \end{Phonetics}
\end{Entry}

\begin{Entry}{哥们}{10,5}{⼝、⼈}
  \begin{Phonetics}{哥们}{ge1men5}
    \definition{expr.}{\emph{Brothers!}}
    \definition{s.}{(coloquial) cara | irmão (forma diminuta de tratamento entre homens)}
  \end{Phonetics}
\end{Entry}

\begin{Entry}{哥哥}{10,10}{⼝、⼝}
  \begin{Phonetics}{哥哥}{ge1 ge5}[][HSK 1]
    \definition[个,位]{s.}{irmão mais velho | primo}
  \end{Phonetics}
\end{Entry}

\begin{Entry}{哥斯拉}{10,12,8}{⼝、⽄、⼿}
  \begin{Phonetics}{哥斯拉}{ge1si1la1}
    \definition*{s.}{Godzilla}
  \seealsoref{酷斯拉}{ku4si1la1}
  \end{Phonetics}
\end{Entry}

\begin{Entry}{哦}{10}{⼝}
  \begin{Phonetics}{哦}{e2}
    \definition{v.}{cantar suavemente (um poema)}
  \end{Phonetics}
  \begin{Phonetics}{哦}{o2}
    \definition{interj.}{Oh! (indicando dúvida ou surpresa)}
  \end{Phonetics}
  \begin{Phonetics}{哦}{o4}
    \definition{interj.}{Oh! (indicando que acabou de aprender algo)}
  \end{Phonetics}
  \begin{Phonetics}{哦}{o5}
    \definition{part.}{usada no final da frase para indicar que uma pessoa está afirmando um fato que a outra pessoa não sabe | usada no final da frase para transmitir informalidade, calor, simpatia ou intimidade}
  \end{Phonetics}
\end{Entry}

\begin{Entry}{哩}{10}{⼝}
  \begin{Phonetics}{哩}{li3}
    \definition{clas.}{milha (unidade de comprimento igual a 1.609,344 m)}
  \end{Phonetics}
  \begin{Phonetics}{哩}{li5}
    \definition{part.}{(dialeto) final modal semelhante a 呢 ou 啦, usado em um tom definido, mas um tanto exagerado}
  \seealsoref{啦}{la5}
  \seealsoref{呢}{ne5}
  \end{Phonetics}
\end{Entry}

\begin{Entry}{哩哩啦啦}{10,10,11,11}{⼝、⼝、⼝、⼝}
  \begin{Phonetics}{哩哩啦啦}{li1 li1 la1 la1}
    \definition{adj.}{espalhado; disperso; disseminado; difuso; esporádico; aqui e ali}
  \end{Phonetics}
\end{Entry}

\begin{Entry}{哭}{10}{⼝}
  \begin{Phonetics}{哭}{ku1}[][HSK 2]
    \definition{v.}{chorar; soluçar; lamentar-se; chorar de dor ou emoção}
  \end{Phonetics}
\end{Entry}

\begin{Entry}{哭墙}{10,14}{⼝、⼟}
  \begin{Phonetics}{哭墙}{ku1qiang2}
    \definition*{s.}{Muro das Lamentações (Jerusalém)}
  \end{Phonetics}
\end{Entry}

\begin{Entry}{哮}{10}{⼝}
  \begin{Phonetics}{哮}{xiao4}
    \definition{s.}{respiração pesada; chiado}
    \definition{v.}{rugir; uivar}
  \end{Phonetics}
\end{Entry}

\begin{Entry}{哮喘}{10,12}{⼝、⼝}
  \begin{Phonetics}{哮喘}{xiao4chuan3}
    \definition{s.}{asma; sintomas de dispneia: os pacientes sentem que a respiração está muito difícil; pneumonia, insuficiência cardíaca, bronquite crônica e outras doenças causadas por espasmo da musculatura lisa respiratória frequentemente apresentam esse sintoma}
    \definition{v.}{sofrer de asma}
  \end{Phonetics}
\end{Entry}

\begin{Entry}{哲}{10}{⼝}
  \begin{Phonetics}{哲}{zhe2}
    \definition{adj.}{sábio; sagaz}
    \definition[位,名,个]{s.}{pessoas sábias; sábio |sabedoria}
  \end{Phonetics}
\end{Entry}

\begin{Entry}{哲学}{10,8}{⼝、⼦}
  \begin{Phonetics}{哲学}{zhe2xue2}[][HSK 6]
    \definition{s.}{filosofia; é uma disciplina que explora questões fundamentais e conceitos básicos}
  \end{Phonetics}
\end{Entry}

\begin{Entry}{哲理}{10,11}{⼝、⽟}
  \begin{Phonetics}{哲理}{zhe2li3}
    \definition{s.}{filosofia | teoria filosófica}
  \end{Phonetics}
\end{Entry}

\begin{Entry}{哺}{10}{⼝}
  \begin{Phonetics}{哺}{bu3}
    \definition{s.}{comida na boca; chorando por comida | alimentos de mastigação; mastigando comida}
    \definition{v.}{alimentar (um bebê); amamentar}
  \end{Phonetics}
\end{Entry}

\begin{Entry}{哺育}{10,8}{⼝、⾁}
  \begin{Phonetics}{哺育}{bu3yu4}[][HSK 7-9]
    \definition{v.}{alimentar | Figurativo: nutrir; fomentar | desenvolver}
  \end{Phonetics}
\end{Entry}

\begin{Entry}{哼}{10}{⼝}
  \begin{Phonetics}{哼}{heng1}[][HSK 7-9]
    \definition{v.}{gemer; bufar | cantarolar}
  \end{Phonetics}
  \begin{Phonetics}{哼}{hng5}
    \definition{interj.}{Hmm; Humph; expressa insatisfação, desprezo, desdém ou indignação}
  \end{Phonetics}
\end{Entry}

\begin{Entry}{唇}{10}{⼝}
  \begin{Phonetics}{唇}{chun2}
    \definition[片]{s.}{lábios}
  \end{Phonetics}
\end{Entry}

\begin{Entry}{唉}{10}{⼝}
  \begin{Phonetics}{唉}{ai1}
    \definition{interj.}{Sim!; Certo!; Bem! | Ai de mim!; o som dos suspiros}
  \end{Phonetics}
  \begin{Phonetics}{唉}{ai4}[][HSK 7-9]
    \definition{interj.}{Oh!; Ah!; Bem!; interjeição que expressa tristeza ou arrependimento | Bem!; Argh!; usado para responder ou reconhecer}
  \end{Phonetics}
\end{Entry}

\begin{Entry}{唐}{10}{⼝}
  \begin{Phonetics}{唐}{tang2}
    \definition*{s.}{Dinastia estabelecida pelo Imperador Yao, 尧, no período lendário da história chinesa | Dinastia Tang (618-907) | Dinastia Tang posterior (923-936), uma das cinco dinastias | Sobrenome Tang}
    \definition{adj.}{exagerado; bombástico; orgulhoso | em vão; por nada}
  \seealsoref{尧}{yao2}
  \end{Phonetics}
\end{Entry}

\begin{Entry}{唐人街}{10,2,12}{⼝、⼈、⾏}
  \begin{Phonetics}{唐人街}{tang2ren2 jie1}
    \definition*[条,座]{s.}{Bairro Chinês; Chinatown; refere-se ao mercado de rua onde os chineses do exterior vivem e abrem muitas lojas com características chinesas}
  \seealsoref{中国城}{zhong1guo2cheng2}
  \end{Phonetics}
\end{Entry}

\begin{Entry}{唐初四大家}{10,7,5,3,10}{⼝、⾐、⼞、⼤、⼧}
  \begin{Phonetics}{唐初四大家}{tang2 chu1 si4 da4jia1}
    \definition*{s.}{Quatro grandes calígrafos do início da dinastia Tang; refere-se a Yu Shi'nan 虞世南, Ouyang Xun 欧阳询, Chu Suiliang 褚遂良 e Xue Ji 薛稷}
  \seealsoref{褚遂良}{chu3 sui4liang2}
  \seealsoref{欧阳询}{ou1yang2 xun2}
  \seealsoref{薛稷}{xue1 ji4}
  \seealsoref{虞世南}{yu2 shi4'nan2}
  \end{Phonetics}
\end{Entry}

\begin{Entry}{唤}{10}{⼝}
  \begin{Phonetics}{唤}{huan4}
    \definition{v.}{chamar; fazer um barulho alto para fazer a outra parte acordar, prestar atenção ou vir até você}
  \end{Phonetics}
\end{Entry}

\begin{Entry}{唤起}{10,10}{⼝、⾛}
  \begin{Phonetics}{唤起}{huan4qi3}[][HSK 7-9]
    \definition{v.}{despertar | chamar; evocar}
  \end{Phonetics}
\end{Entry}

\begin{Entry}{啊}{10}{⼝}
  \begin{Phonetics}{啊}{a1}[][HSK 2]
    \definition{interj.}{Ah!; Oh!; expressar surpresa ou admiração}
  \end{Phonetics}
  \begin{Phonetics}{啊}{a2}[][HSK 2]
    \definition{interj.}{Eh?; Ei?; Que?; Por que?; expressar questionamento, dúvida ou solicitar opinião}
  \end{Phonetics}
  \begin{Phonetics}{啊}{a3}[][HSK 2]
    \definition{interj.}{Eh?; Meu!; E aí?; Que?; expressar surpresa e dúvida}
  \end{Phonetics}
  \begin{Phonetics}{啊}{a4}[][HSK 2]
    \definition{interj.}{Bem!; Sim!; expressa concordância, pronúncia mais curta | Oh!; Ah!; indica que compreendeu, com pronúncia mais longa | Oh!; expressa surpresa ou admiração, com pronúncia mais longa | Desgraça!; expressa tristeza ou pesar}
  \end{Phonetics}
  \begin{Phonetics}{啊}{a5}[][HSK 2,4]
    \definition{part.}{usado no final da frase para expressar admiração | usado no final da frase para expressar afirmação, justificativa, insistência, recomendação, etc. | usado no final da frase para indicar dúvida | usado para fazer uma pequena pausa na frase, chamando a atenção para o que vem a seguir | usado após os itens enumerados | usado após verbos repetitivos, indica um processo longo}
  \end{Phonetics}
\end{Entry}

\begin{Entry}{啊呀}{10,7}{⼝、⼝}
  \begin{Phonetics}{啊呀}{a1ya1}
    \definition{interj.}{Oh meu Deus! | interjeição de surpresa}
  \end{Phonetics}
\end{Entry}

\begin{Entry}{啊哟}{10,9}{⼝、⼝}
  \begin{Phonetics}{啊哟}{a1yo5}
    \definition{interj.}{Meu Deus! | Oh! | Ai! | interjeição de surpresa ou dor}
  \end{Phonetics}
\end{Entry}

\begin{Entry}{圆}{10}{⼞}
  \begin{Phonetics}{圆}{yuan2}[][HSK 4]
    \definition*{s.}{Sobrenome Yuan}
    \definition{adj.}{redondo; circular; esférico; arredondado | diplomático; satisfatório}
    \definition[个,轮]{s.}{círculo; circunferência | uma moeda de valor e peso fixos}
    \definition{v.}{tornar plausível; justificar; tornar completo; completar}
  \end{Phonetics}
\end{Entry}

\begin{Entry}{圆珠笔}{10,10,10}{⼞、⽟、⽵}
  \begin{Phonetics}{圆珠笔}{yuan2 zhu1 bi3}[][HSK 6]
    \definition[支,枝]{s.}{caneta esferográfica}
  \end{Phonetics}
\end{Entry}

\begin{Entry}{圆满}{10,13}{⼞、⽔}
  \begin{Phonetics}{圆满}{yuan2man3}[][HSK 4]
    \definition{adj.}{perfeito; satisfatório; sem defeitos}
  \end{Phonetics}
\end{Entry}

\begin{Entry}{埋}{10}{⼟}
  \begin{Phonetics}{埋}{mai2}[][HSK 6]
    \definition{v.}{cobrir (com terra, neve, etc.); enterrar | esconder | enterrar (uma pessoa morta)}
  \end{Phonetics}
  \begin{Phonetics}{埋}{man2}
    \definition{part.}{caracter formador de palavras}
  \end{Phonetics}
\end{Entry}

\begin{Entry}{埋伏}{10,6}{⼟、⼈}
  \begin{Phonetics}{埋伏}{mai2fu2}
    \definition{s.}{emboscada}
    \definition{v.}{emboscar}
  \end{Phonetics}
\end{Entry}

\begin{Entry}{壶}{10}{⼠}
  \begin{Phonetics}{壶}{hu2}[][HSK 6]
    \definition*{s.}{Sobrenome Hu}
    \definition[个,把]{s.}{chaleira; panela | garrafa; frasco; recipiente para líquidos}
  \end{Phonetics}
\end{Entry}

\begin{Entry}{夏}{10}{⼢}
  \begin{Phonetics}{夏}{xia4}
    \definition*{s.}{Dinastia Xia (2070-1600 a.C.) | China; refere-se à China | Sobrenome Xia}
    \definition{s.}{verão}
  \end{Phonetics}
\end{Entry}

\begin{Entry}{夏天}{10,4}{⼢、⼤}
  \begin{Phonetics}{夏天}{xia4 tian1}[][HSK 2]
    \definition[个]{s.}{verão}
  \end{Phonetics}
\end{Entry}

\begin{Entry}{夏日}{10,4}{⼢、⽇}
  \begin{Phonetics}{夏日}{xia4ri4}
    \definition{s.}{horário de verão}
  \end{Phonetics}
\end{Entry}

\begin{Entry}{夏季}{10,8}{⼢、⼦}
  \begin{Phonetics}{夏季}{xia4 ji4}[][HSK 4]
    \definition[个]{s.}{verão; segundo trimestre do ano, habitualmente chamado na China de período de três meses, do início do verão ao início do outono, também chamado de ``quarto, quinto e sexto'' meses do calendário lunar}
  \end{Phonetics}
\end{Entry}

\begin{Entry}{套}{10}{⼤}
  \begin{Phonetics}{套}{tao4}[][HSK 2]
    \definition{clas.}{usado para coisas agrupadas: conjuntos, coleções, séries, etc.}
    \definition{s.}{estojo; capa; bainha | local onde o rio ou a cordilheira faz uma curva (usado principalmente em nomes de lugares) | enchimento de algodão em roupas e edredons | arreios; corda para amarrar animais | nó; laço; um objeto circular feito com corda ou algo semelhante | cortersia; convenção; conversa fiada; métodos repetitivos | armadilha; truque; conspiração}
    \definition{v.}{sobrepor; interligar | deslizar sobre; cobrir por fora | atrelar; engatar; usar um cinto de segurança | copiar; imitar; seguir o modelo de | extrair; induzir a falar; persuadir alguém a revelar um segredo; induzir; provocar | tentar vencer; aproximar-se de; aproximar-se intencionalmente de outras pessoas para algum propósito | fazer a rosca de um parafuso; usar um macho de rosca ou uma chave de rosca para fazer roscas}
  \end{Phonetics}
\end{Entry}

\begin{Entry}{套问}{10,6}{⼤、⾨}
  \begin{Phonetics}{套问}{tao4wen4}
    \definition{s.}{retórica}
    \definition{v.}{descobrir por meio de questionamento indireto diplomático}
  \end{Phonetics}
\end{Entry}

\begin{Entry}{套餐}{10,16}{⼤、⾷}
  \begin{Phonetics}{套餐}{tao4 can1}[][HSK 4]
    \definition{s.}{combo; pacote de produtos; pacote de serviços; metaforicamente, bens ou projetos que são combinados e levados ao mercado | refeição preparada; pacotes de refeições completos}
  \end{Phonetics}
\end{Entry}

\begin{Entry}{娱}{10}{⼥}
  \begin{Phonetics}{娱}{yu2}
    \definition{s.}{alegria; prazer; diversão; felicidade}
    \definition{v.}{dar prazer a; divertir; fazer feliz}
  \end{Phonetics}
\end{Entry}

\begin{Entry}{娱乐}{10,5}{⼥、⼃}
  \begin{Phonetics}{娱乐}{yu2le4}[][HSK 6]
    \definition[项]{s.}{entretenimento; diversão; recreação; passa-tempo; atividades recreativas, prazerosas e divertidas}
    \definition{v.}{recrear; divertir; distrair; entreter; passar o tempo}
  \end{Phonetics}
\end{Entry}

\begin{Entry}{孬}{10}{⼥}
  \begin{Phonetics}{孬}{nao1}
    \definition{adj.}{ruim | covarde | Dialeto: não (é) bom (contração de 不 + 好)}
  \seealsoref{不}{bu4}
  \seealsoref{好}{hao3}
  \end{Phonetics}
\end{Entry}

\begin{Entry}{害}{10}{⼧}
  \begin{Phonetics}{害}{hai4}[][HSK 5]
    \definition{adj.}{prejudicial; destrutivo; injurioso; nocivo}
    \definition{s.}{mal; maldade; dano; calamidade}
    \definition{v.}{prejudicar; fazer mal a; causar problemas a | matar; assassinar | sofrer de; contrair (uma doença) | sentir-se (envergonhado, com medo, etc.); despertar (um sentimento ou uma emoção)}
  \end{Phonetics}
\end{Entry}

\begin{Entry}{害虫}{10,6}{⼧、⾍}
  \begin{Phonetics}{害虫}{hai4chong2}[][HSK 7-9]
    \definition[种,只,个]{s.}{verme; bicho; inseto nocivo (ou destrutivo); praga (oposto a 益虫)}
  \seealsoref{益虫}{yi4chong2}
  \end{Phonetics}
\end{Entry}

\begin{Entry}{害怕}{10,8}{⼧、⼼}
  \begin{Phonetics}{害怕}{hai4pa4}[][HSK 3]
    \definition{v.}{estar assustado; ter medo; encontrar dificuldades, perigos, etc., e sentir-se inquieto ou nervoso}
  \end{Phonetics}
\end{Entry}

\begin{Entry}{害羞}{10,10}{⼧、⽺}
  \begin{Phonetics}{害羞}{hai4/xiu1}[][HSK 7-9]
    \definition{v.+compl.}{ser tímido; parecer tímido; tornar-se tímido}
  \end{Phonetics}
\end{Entry}

\begin{Entry}{害臊}{10,17}{⼧、⾁}
  \begin{Phonetics}{害臊}{hai4/sao4}[][HSK 7-9]
    \definition{v.+compl.}{sentir vergonha; ser tímido}
  \end{Phonetics}
\end{Entry}

\begin{Entry}{宴}{10}{⼧}
  \begin{Phonetics}{宴}{yan4}
    \definition{adj.}{tranquilo e confortável}
    \definition[个,场]{s.}{festa; banquete | facilidade e conforto; felicidade; lazer}
    \definition{v.}{entreter em um banquete; festejar}
  \end{Phonetics}
\end{Entry}

\begin{Entry}{宴会}{10,6}{⼧、⼈}
  \begin{Phonetics}{宴会}{yan4hui4}[][HSK 6]
    \definition[个,场,次]{s.}{festa; banquete; jantar; uma reunião onde convidados e anfitriões bebem e comem juntos (referindo-se a uma ocasião mais solene)}
  \end{Phonetics}
\end{Entry}

\begin{Entry}{家}{10}{⼧}
  \begin{Phonetics}{家}{jia1}[][HSK 1,2]
    \definition*{s.}{Sobrenome Jia}
    \definition{adj.}{domado; domesticado; criado; alimentado | interno}
    \definition{clas.}{usado para famílias ou estabelecimentos comerciais; para uso doméstico; lojas; fábricas, etc.}
    \definition{pron.}{Educado: meu (irmã, tio, etc.)}
    \definition[个]{s.}{família; domicílio; clã | lar; casa; residência da família | pessoa ou família envolvida em um determinado comércio; pessoas que trabalham em determinada profissão ou que possuem determinada identidade | especialista em um determinado campo; pessoa que possui conhecimentos especializados ou se dedica a atividades específicas | escola de pensamento; rscola acadêmica | (em cartas de baralho, mah-jong etc.) festa; lado; refere-se a jogar xadrez ou cartas, em que uma das partes joga contra a outra | nacionalidade; referindo-se à etnia | membros da família; parentes; pessoas ou famílias com quem você tem algum tipo de relação | membro do mesmo clã; pessoas com o mesmo sobrenome}
    \definition{suf.}{sufixo substantivo para designar um especialista em alguma atividade, como um músico ou revolucionário, para designar uma profissão como em -eiro, -ista, por exemplo 科学家}
  \seealsoref{科学家}{ke1xue2jia1}
  \end{Phonetics}
\end{Entry}

\begin{Entry}{家人}{10,2}{⼧、⼈}
  \begin{Phonetics}{家人}{jia1 ren2}[][HSK 1]
    \definition{s.}{família (de alguém); membro da família; os membros de uma família}
  \end{Phonetics}
\end{Entry}

\begin{Entry}{家乡}{10,3}{⼧、⼄}
  \begin{Phonetics}{家乡}{jia1xiang1}[][HSK 3]
    \definition[片,座]{s.}{cidade natal; o lugar onde sua família vive há gerações}
  \end{Phonetics}
\end{Entry}

\begin{Entry}{家长}{10,4}{⼧、⾧}
  \begin{Phonetics}{家长}{jia1 zhang3}[][HSK 2]
    \definition[位,名,个]{s.}{pais; patriarca; tutor; guardião; refere-se aos pais ou outros responsáveis legais}
  \end{Phonetics}
\end{Entry}

\begin{Entry}{家务}{10,5}{⼧、⼒}
  \begin{Phonetics}{家务}{jia1wu4}[][HSK 4]
    \definition[堆,次,件]{s.}{trabalho doméstico; tarefas domésticas}
  \end{Phonetics}
\end{Entry}

\begin{Entry}{家用}{10,5}{⼧、⽤}
  \begin{Phonetics}{家用}{jia1yong4}[][HSK 7-9]
    \definition{adj.}{doméstico; para uso doméstico}
    \definition[本]{s.}{despesas familiares; dinheiro para manutenção da casa}
    \definition{v.}{ser usado em casa; para uso doméstico}
  \end{Phonetics}
\end{Entry}

\begin{Entry}{家用电器}{10,5,5,16}{⼧、⽤、⽥、⼝}
  \begin{Phonetics}{家用电器}{jia1yong4 dian4qi4}
    \definition{s.}{eletrodoméstico; refere-se a diversos aparelhos elétricos utilizados na vida doméstica e coletiva}
  \end{Phonetics}
\end{Entry}

\begin{Entry}{家电}{10,5}{⼧、⽥}
  \begin{Phonetics}{家电}{jia1 dian4}[][HSK 6]
    \definition[件,台]{s.}{eletrodomésticos, abreviação de 家用电器}
  \seealsoref{家用电器}{jia1yong4 dian4qi4}
  \end{Phonetics}
\end{Entry}

\begin{Entry}{家伙}{10,6}{⼧、⼈}
  \begin{Phonetics}{家伙}{jia1huo5}[][HSK 7-9]
    \definition[些,个,群,帮]{s.}{ferramenta; utensílio; arma; refere-se a ferramentas ou armas | cara; companheiro; refere-se a pessoas (com desprezo ou humor)  | gado; animal doméstico}
  \end{Phonetics}
\end{Entry}

\begin{Entry}{家园}{10,7}{⼧、⼞}
  \begin{Phonetics}{家园}{jia1 yuan2}[][HSK 6]
    \definition{s.}{casa; terra natal; um jardim em casa, geralmente referindo-se à cidade natal ou à família}
  \end{Phonetics}
\end{Entry}

\begin{Entry}{家里}{10,7}{⼧、⾥}
  \begin{Phonetics}{家里}{jia1 li3}[][HSK 1]
    \definition{s.}{(em) casa; (em sua) família | esposa}
  \end{Phonetics}
\end{Entry}

\begin{Entry}{家具}{10,8}{⼧、⼋}
  \begin{Phonetics}{家具}{jia1ju4}[][HSK 3]
    \definition[件,套,些,个]{s.}{móveis; mobiliário de casa; utensílios domésticos, incluem principalmente camas, mesas, cadeiras, armários, etc.}
  \end{Phonetics}
\end{Entry}

\begin{Entry}{家庭}{10,9}{⼧、⼴}
  \begin{Phonetics}{家庭}{jia1ting2}[][HSK 2]
    \definition[个,户]{s.}{família}
  \end{Phonetics}
\end{Entry}

\begin{Entry}{家政}{10,9}{⼧、⽁}
  \begin{Phonetics}{家政}{jia1zheng4}[][HSK 7-9]
    \definition{s.}{tarefas domésticas; trabalho de gestão doméstica}
  \end{Phonetics}
\end{Entry}

\begin{Entry}{家俱}{10,10}{⼧、⼈}
  \begin{Phonetics}{家俱}{jia1ju4}
    \definition{s.}{mobília}
  \end{Phonetics}
\end{Entry}

\begin{Entry}{家家户户}{10,10,4,4}{⼧、⼧、⼾、⼾}
  \begin{Phonetics}{家家户户}{jia1jia1hu4hu4}[][HSK 7-9]
    \definition{expr.}{cada família; cada lar}
  \end{Phonetics}
\end{Entry}

\begin{Entry}{家教}{10,11}{⼧、⽁}
  \begin{Phonetics}{家教}{jia1jiao4}[][HSK 7-9]
    \definition[个,名,位]{s.}{educação; educação familiar; disciplina doméstica; a educação de moral e etiqueta que os pais dão aos seus filhos |  tutor}
  \end{Phonetics}
\end{Entry}

\begin{Entry}{家族}{10,11}{⼧、⽅}
  \begin{Phonetics}{家族}{jia1zu2}[][HSK 7-9]
    \definition[个]{s.}{clã; família; uma organização social baseada em relações de sangue, incluindo várias gerações do mesmo sangue}
  \end{Phonetics}
\end{Entry}

\begin{Entry}{家喻户晓}{10,12,4,10}{⼧、⼝、⼾、⽇}
  \begin{Phonetics}{家喻户晓}{jia1yu4-hu4xiao3}[][HSK 7-9]
    \definition{expr.}{nome familiar | conhecido por todas as famílias; amplamente conhecido; conhecido por todos}[他是家喻户晓的演员。===Ele é um ator famoso.]
  \end{Phonetics}
\end{Entry}

\begin{Entry}{家属}{10,12}{⼧、⼫}
  \begin{Phonetics}{家属}{jia1shu3}[][HSK 3]
    \definition{s.}{membros da família; dependentes (familiares); os membros da família que não sejam o próprio chefe da família, ou seja, os membros da família que não sejam o próprio trabalhador}
  \end{Phonetics}
\end{Entry}

\begin{Entry}{家禽}{10,12}{⼧、⽱}
  \begin{Phonetics}{家禽}{jia1qin2}[][HSK 7-9]
    \definition{s.}{aves domésticas | ave; pássaro doméstico}
  \end{Phonetics}
\end{Entry}

\begin{Entry}{家境}{10,14}{⼧、⼟}
  \begin{Phonetics}{家境}{jia1jing4}[][HSK 7-9]
    \definition{s.}{situação financeira familiar; circunstâncias familiares}
  \end{Phonetics}
\end{Entry}

\begin{Entry}{容}{10}{⼧}
  \begin{Phonetics}{容}{rong2}
    \definition*{s.}{Sobrenome Rong}
    \definition{adv.}{talvez; provavelmente; possivelmente}
    \definition{s.}{expressão facial e tez | aparência; o estado ou condição das coisas}
    \definition{v.}{permitir; quando os outros querem fazer algo, deixe-os fazer | tolerar; ser capaz de aceitar pessoas ou coisas com as quais você não está satisfeito | conter (número de pessoas ou coisas que podem ser colocadas em um determinado espaço)}
  \end{Phonetics}
\end{Entry}

\begin{Entry}{容易}{10,8}{⼧、⽇}
  \begin{Phonetics}{容易}{rong2yi4}[][HSK 3]
    \definition{adj.}{fácil; simples; sem complicações | provável; passível; inclinado; indica uma alta probabilidade de algo acontecer}
  \end{Phonetics}
\end{Entry}

\begin{Entry}{容貌}{10,14}{⼧、⾘}
  \begin{Phonetics}{容貌}{rong2mao4}
    \definition{s.}{aparência | aspecto | características}
  \end{Phonetics}
\end{Entry}

\begin{Entry}{宽}{10}{⼧}
  \begin{Phonetics}{宽}{kuan1}[][HSK 4]
    \definition*{s.}{Sobrenome Kuan}
    \definition{adj.}{largo; amplo; espaçoso; grandes distâncias horizontais (em oposição a 窄) | leniente; generoso; indulgente | bem de vida; rico; confortável}
    \definition[米]{s.}{largura; amplitude}[桌子有一米宽。===A mesa tem um metro de largura.]
    \definition{v.}{relaxar; aliviar}
  \seealsoref{窄}{zhai3}
  \end{Phonetics}
\end{Entry}

\begin{Entry}{宽广}{10,3}{⼧、⼴}
  \begin{Phonetics}{宽广}{kuan1 guang3}[][HSK 4]
    \definition{adj.}{vasto; amplo; espaçoso; extenso}
  \end{Phonetics}
\end{Entry}

\begin{Entry}{宽度}{10,9}{⼧、⼴}
  \begin{Phonetics}{宽度}{kuan1 du4}[][HSK 5]
    \definition{s.}{largura; amplitude; duração; o grau de largura e estreiteza; a distância horizontal (no caso de um retângulo, a distância entre os dois lados mais longos)}
  \end{Phonetics}
\end{Entry}

\begin{Entry}{宽阔}{10,12}{⼧、⾨}
  \begin{Phonetics}{宽阔}{kuan1 kuo4}[][HSK 6]
    \definition{adj.}{amplo; largo; espaçoso | tolerante; mente aberta; descreve uma mente alegre e ampla}
  \end{Phonetics}
\end{Entry}

\begin{Entry}{宽影片}{10,15,4}{⼧、⼺、⽚}
  \begin{Phonetics}{宽影片}{kuan1ying3pian4}
    \definition{s.}{filme \emph{widescreen}}
  \end{Phonetics}
\end{Entry}

\begin{Entry}{宾}{10}{⼧}
  \begin{Phonetics}{宾}{bin1}
    \definition*{s.}{Sobrenome Bin}
    \definition[个,位,名,些]{s.}{convidado}
  \end{Phonetics}
\end{Entry}

\begin{Entry}{宾馆}{10,11}{⼧、⾷}
  \begin{Phonetics}{宾馆}{bin1guan3}[][HSK 5]
    \definition[家,个,座]{s.}{hotel; acomodações públicas para hóspedes}
  \end{Phonetics}
\end{Entry}

\begin{Entry}{射}{10}{⼨}
  \begin{Phonetics}{射}{she4}[][HSK 5]
    \definition*{s.}{Sobrenome She}
    \definition{v.}{atirar; disparar | descarregar em jato; jorrar | emitir (luz, calor, etc.) | irradiar | aludir a algo ou alguém; insinuar}
  \end{Phonetics}
\end{Entry}

\begin{Entry}{射击}{10,5}{⼨、⼐}
  \begin{Phonetics}{射击}{she4ji1}[][HSK 5]
    \definition{s.}{tiro; tiro ao alvo}
    \definition{v.}{disparar; atirar}
  \end{Phonetics}
\end{Entry}

\begin{Entry}{展}{10}{⼫}
  \begin{Phonetics}{展}{zhan3}
    \definition*{s.}{Sobrenome Zhan}
    \definition{s.}{exposição}
    \definition{v.}{abrir; espalhar; desdobrar | fazer bom uso; dar liberdade para | adiar; estender; prolongar | expandir | abrir; deixar ir | exibir; mostrar}
  \end{Phonetics}
\end{Entry}

\begin{Entry}{展开}{10,4}{⼫、⼶}
  \begin{Phonetics}{展开}{zhan3kai1}[][HSK 3]
    \definition{s.}{desenvolvimento; expansão; explosão; evolução}
    \definition{v.}{espalhar; desdobrar; abrir | lançar; desdobrar; desenvolver; realizar em grande escala | espalhar; desenrolar; amplificar; desenvolver; expandir; explodir; evoluir; alongar}
  \end{Phonetics}
\end{Entry}

\begin{Entry}{展示}{10,5}{⼫、⽰}
  \begin{Phonetics}{展示}{zhan3shi4}[][HSK 5]
    \definition{v.}{mostrar; revelar; pôr a nu; abrir diante de alguém; expor claramente; manifestar de forma evidente}
  \end{Phonetics}
\end{Entry}

\begin{Entry}{展现}{10,8}{⼫、⾒}
  \begin{Phonetics}{展现}{zhan3xian4}[][HSK 5]
    \definition{v.}{mostrar; surgir; manifestar}
  \end{Phonetics}
\end{Entry}

\begin{Entry}{展览}{10,9}{⼫、⾒}
  \begin{Phonetics}{展览}{zhan3lan3}[][HSK 5]
    \definition[个,次,场]{s.}{exposição; exibição; atividades expostas; itens expostos}
    \definition{v.}{mostrar; exibir; expor; expor algo para que as pessoas vejam}
  \end{Phonetics}
\end{Entry}

\begin{Entry}{峰}{10}{⼭}
  \begin{Phonetics}{峰}{feng1}
    \definition{clas.}{usado para camelos}
    \definition{s.}{pico; cume; o pico proeminente de uma montanha | coisa parecida com um pico; coisas em forma de montanhas}
  \end{Phonetics}
\end{Entry}

\begin{Entry}{峰会}{10,6}{⼭、⼈}
  \begin{Phonetics}{峰会}{feng1 hui4}[][HSK 6]
    \definition{s.}{cúpula; reunião de cúpula}
  \end{Phonetics}
\end{Entry}

\begin{Entry}{峰回路转}{10,6,13,8}{⼭、⼞、⾜、⾞}
  \begin{Phonetics}{峰回路转}{feng1hui2-lu4zhuan3}[][HSK 7-9]
    \definition{expr.}{cume em meio a elevações circundantes e estradas sinuosas;  (estrada de montanha) torcendo e virando; a estrada da montanha serpenteia em torno de cada novo pico | boa (ou nova) reviravolta nos acontecimentos; uma oportunidade surgiu inesperadamente; as coisas tomaram um novo rumo}
  \end{Phonetics}
\end{Entry}

\begin{Entry}{席}{10}{⼱}
  \begin{Phonetics}{席}{xi2}
    \definition*{s.}{Sobrenome Xi}
    \definition[卷,张]{s.}{esteira | assento; lugar; caixa | assento (em uma assembleia legislativa) | festim; banquete; jantar}
  \end{Phonetics}
\end{Entry}

\begin{Entry}{席卷}{10,8}{⼱、⼙}
  \begin{Phonetics}{席卷}{xi2juan3}
    \definition{v.}{engolfar | varrer | levar tudo para fora}
  \end{Phonetics}
\end{Entry}

\begin{Entry}{座}{10}{⼴}
  \begin{Phonetics}{座}{zuo4}[][HSK 2]
    \definition{clas.}{usado para montanhas, edifícios e objetos imóveis semelhantes}
    \definition{s.}{assento; lugar | suporte; pedestal; base | (astronomia) constalação | (antigo) forma de tratamento a altos funcionários |}
  \end{Phonetics}
\end{Entry}

\begin{Entry}{座子}{10,3}{⼴、⼦}
  \begin{Phonetics}{座子}{zuo4zi5}
    \definition{s.}{soquete | pedestal | sela}
  \end{Phonetics}
\end{Entry}

\begin{Entry}{座位}{10,7}{⼴、⼈}
  \begin{Phonetics}{座位}{zuo4wei4}[][HSK 2]
    \definition[个,排]{s.}{assento; lugar}
  \end{Phonetics}
\end{Entry}

\begin{Entry}{座标}{10,9}{⼴、⽊}
  \begin{Phonetics}{座标}{zuo4biao1}
    \variantof{坐标}
  \end{Phonetics}
\end{Entry}

\begin{Entry}{座谈会}{10,10,6}{⼴、⾔、⼈}
  \begin{Phonetics}{座谈会}{zuo4 tan2 hui4}[][HSK 6]
    \definition{s.}{fórum; simpósio; discussão informal | conferência | sessão de rap}
  \end{Phonetics}
\end{Entry}

\begin{Entry}{弱}{10}{⼸}
  \begin{Phonetics}{弱}{ruo4}[][HSK 4]
    \definition{adj.}{fraco; debilitado | jovem | inferior; pior | colocado depois de uma fração ou decimal para indicar que é um pouco menor que esse número (oposto a 强)}
    \definition{v.}{perder (através da morte)}
  \seealsoref{强}{qiang2}
  \end{Phonetics}
\end{Entry}

\begin{Entry}{徐}{10}{⼻}
  \begin{Phonetics}{徐}{xu2}
    \definition*{s.}{Sobrenome Xu}
    \definition{adv.}{lentamente; suavemente}
  \end{Phonetics}
\end{Entry}

\begin{Entry}{徐徐}{10,10}{⼻、⼻}
  \begin{Phonetics}{徐徐}{xu2xu2}
    \definition{adv.}{lentamente; suavemente}
  \end{Phonetics}
\end{Entry}

\begin{Entry}{徒}{10}{⼻}
  \begin{Phonetics}{徒}{tu2}
    \definition{adj.}{vazio; nu}
    \definition{adv.}{somente; meramente; apenas | a pé | em vão; sem sucesso; sem sucesso}
    \definition{s.}{aprendiz; aluno | seguidor; crente | (pejorativo) pessoas da mesma facção | (pejorativo) pessoa; companheiro | (prisão) pena; prisão; sentença | estudante}
    \definition{v.}{estar a pé | andar}
  \end{Phonetics}
\end{Entry}

\begin{Entry}{徒手}{10,4}{⼻、⼿}
  \begin{Phonetics}{徒手}{tu2shou3}
    \definition{adj.}{com as mãos vazias | desarmado | mão livre (desenho) | lutando mão-a-mão}
  \end{Phonetics}
\end{Entry}

\begin{Entry}{徒弟}{10,7}{⼻、⼸}
  \begin{Phonetics}{徒弟}{tu2di4}[][HSK 6]
    \definition[位,名,个]{s.}{discípulo; aprendiz; uma pessoa que aprende com um mestre; geralmente se refere a uma pessoa que aprende com um especialista}[他是我的徒弟。===Ele é meu aprendiz.]
  \end{Phonetics}
\end{Entry}

\begin{Entry}{恋}{10}{⼼}
  \begin{Phonetics}{恋}{lian4}
    \definition*{s.}{Sobrenome Lian}
    \definition{v.}{amor (romântico) | ansiar por; sentir-se apegado a | amar; apaixonar-se por | não querendo se separar de; sentir sua falta para sempre; não suportar ficar separado}
  \end{Phonetics}
\end{Entry}

\begin{Entry}{恋爱}{10,10}{⼼、⽖}
  \begin{Phonetics}{恋爱}{lian4'ai4}[][HSK 5]
    \definition[个,场,段]{s.}{namoro; afeto; amor romântico; ações que demonstram o amor mútuo}
    \definition{v.}{amar; estar apaixonado}
  \end{Phonetics}
\end{Entry}

\begin{Entry}{恐}{10}{⼼}
  \begin{Phonetics}{恐}{kong3}
    \definition{adv.}{talvez; provavelmente}
    \definition{v.}{temer; recear; ter medo de | ameaçar; aterrorizar; intimidar}
  \end{Phonetics}
\end{Entry}

\begin{Entry}{恐龙}{10,5}{⼼、⿓}
  \begin{Phonetics}{恐龙}{kong3long2}
    \definition[头,只]{s.}{dinossauro}
  \end{Phonetics}
\end{Entry}

\begin{Entry}{恐怕}{10,8}{⼼、⼼}
  \begin{Phonetics}{恐怕}{kong3pa4}[][HSK 3]
    \definition{adv.}{talvez; provavelmente; pode ser; expressa suposição; estimativa. | por medo de; expressar estimativa e preocupação}
    \definition{v.}{ter medo de; temer; recear}
  \end{Phonetics}
\end{Entry}

\begin{Entry}{恐怖主义}{10,8,5,3}{⼼、⼼、⼂、⼂}
  \begin{Phonetics}{恐怖主义}{kong3bu4zhu3yi4}
    \definition{adj.}{terrorista}
    \definition{s.}{terrorismo}
  \end{Phonetics}
\end{Entry}

\begin{Entry}{恩}{10}{⼼}
  \begin{Phonetics}{恩}{en1}
    \definition*{s.}{Sobrenome En}
    \definition{s.}{bondade; favor; graça; gentileza}
  \end{Phonetics}
\end{Entry}

\begin{Entry}{恩人}{10,2}{⼼、⼈}
  \begin{Phonetics}{恩人}{en1 ren2}[][HSK 6]
    \definition{s.}{benfeitor; uma pessoa que ajudou significativamente alguém}
  \end{Phonetics}
\end{Entry}

\begin{Entry}{恩怨}{10,9}{⼼、⼼}
  \begin{Phonetics}{恩怨}{en1yuan4}[][HSK 7-9]
    \definition{s.}{sentimento de gratidão ou ressentimento (inimizade) | ressentimento; queixa; velhas contas}
  \end{Phonetics}
\end{Entry}

\begin{Entry}{恩情}{10,11}{⼼、⼼}
  \begin{Phonetics}{恩情}{en1qing2}[][HSK 7-9]
    \definition{s.}{amor; bondade; afeição profunda}
  \end{Phonetics}
\end{Entry}

\begin{Entry}{恩惠}{10,12}{⼼、⼼}
  \begin{Phonetics}{恩惠}{en1hui4}[][HSK 7-9]
    \definition[份]{s.}{favor; generosidade | bondade; graça; benefícios concedidos ou recebidos}
  \end{Phonetics}
\end{Entry}

\begin{Entry}{恩赐}{10,12}{⼼、⾙}
  \begin{Phonetics}{恩赐}{en1ci4}[][HSK 7-9]
    \definition{s.}{favor; caridade; esmola}
    \definition{v.}{conceder (favores, caridade, etc.); recompensar}
  \end{Phonetics}
\end{Entry}

\begin{Entry}{恭}{10}{⼼}
  \begin{Phonetics}{恭}{gong1}
    \definition{adj.}{respeitoso; reverente | educado}
  \end{Phonetics}
\end{Entry}

\begin{Entry}{恭维}{10,11}{⼼、⽷}
  \begin{Phonetics}{恭维}{gong1wei2}[][HSK 7-9]
    \definition{v.}{bajular; elogiar}
  \end{Phonetics}
\end{Entry}

\begin{Entry}{恭喜}{10,12}{⼼、⼝}
  \begin{Phonetics}{恭喜}{gong1xi3}[][HSK 7-9]
    \definition{v.}{parabenizar; uma maneira educada de parabenizar alguém por seu feliz evento}
  \end{Phonetics}
\end{Entry}

\begin{Entry}{恶}{10}{⼼}
  \begin{Phonetics}{恶}{e3}
    \definition{part.}{elementos formadores de palavras}
  \end{Phonetics}
  \begin{Phonetics}{恶}{e4}[][HSK 7-9]
    \definition{adj.}{feroz | ruim; maligno; perverso | vicioso | feio | grosseiro}
    \definition{s.}{mal; vício; crime (oposto a 善) | maldade; comportamento muito ruim; coisas criminosas}
  \seealsoref{善}{shan4}
  \end{Phonetics}
  \begin{Phonetics}{恶}{wu1}
    \definition{interj.}{Droga!; Ah não!; expressa surpresa}
    \definition{pron.}{como?; por que?; refere-se a um lugar ou coisa; expressa uma pergunta retórica; equivalente a 何 ou 怎么}
  \seealsoref{何}{he2}
  \seealsoref{怎么}{zen3me5}
  \end{Phonetics}
  \begin{Phonetics}{恶}{wu4}
    \definition{v.}{não gostar; odiar; detestar; repugnar}
  \end{Phonetics}
\end{Entry}

\begin{Entry}{恶化}{10,4}{⼼、⼔}
  \begin{Phonetics}{恶化}{e4hua4}[][HSK 7-9]
    \definition{v.}{piorar; deteriorar; exacerbar | piorar a situação}
  \end{Phonetics}
\end{Entry}

\begin{Entry}{恶心}{10,4}{⼼、⼼}
  \begin{Phonetics}{恶心}{e3xin5}[][HSK 4]
    \definition{adj.}{nauseante; repugnante}
    \definition{s.}{náusea; repugnância}
    \definition{v.}{repugnar; ser nauseante; sentir-se mal | envergonhar (deliberadamente)}
  \end{Phonetics}
  \begin{Phonetics}{恶心}{e4xin1}
    \definition{s.}{mau hábito; hábito vicioso; vício}
  \end{Phonetics}
\end{Entry}

\begin{Entry}{恶劣}{10,6}{⼼、⼒}
  \begin{Phonetics}{恶劣}{e4lie4}[][HSK 7-9]
    \definition{adj.}{mau; odioso; abominável; repugnante; desprezível; muito mau; muito ruim}
  \end{Phonetics}
\end{Entry}

\begin{Entry}{恶性}{10,8}{⼼、⼼}
  \begin{Phonetics}{恶性}{e4xing4}[][HSK 7-9]
    \definition{adj.}{maligno; pernicioso; vicioso (oposto a 良性) | produzindo o mal | rápido (declínio) | descontrolada (inflação) | vicioso (círculo) | perverso}
  \seealsoref{良性}{liang2xing4}
  \end{Phonetics}
\end{Entry}

\begin{Entry}{恶意}{10,13}{⼼、⼼}
  \begin{Phonetics}{恶意}{e4yi4}[][HSK 7-9]
    \definition[丝]{s.}{malícia; má vontade; más intenções}
  \end{Phonetics}
\end{Entry}

\begin{Entry}{悄}{10}{⼼}
  \begin{Phonetics}{悄}{qiao1}
    \definition{adj.}{quieto; silencioso}
  \end{Phonetics}
  \begin{Phonetics}{悄}{qiao3}
    \definition{adj.}{quieto; silencioso | triste; preocupado; aflito}
  \end{Phonetics}
\end{Entry}

\begin{Entry}{悄悄}{10,10}{⼼、⼼}
  \begin{Phonetics}{悄悄}{qiao1qiao1}[][HSK 5]
    \definition{adv.}{silenciosamente; em silêncio; aos sussuros; sem som ou em voz baixa; com o mínimo de ruído possível}
  \end{Phonetics}
\end{Entry}

\begin{Entry}{悔}{10}{⼼}
  \begin{Phonetics}{悔}{hui3}
    \definition{v.}{lamentar; arrepender-se}
  \end{Phonetics}
\end{Entry}

\begin{Entry}{悔恨}{10,9}{⼼、⼼}
  \begin{Phonetics}{悔恨}{hui3hen4}[][HSK 7-9]
    \definition{v.}{arrepender-se profundamente; estar amargamente arrependido}
  \end{Phonetics}
\end{Entry}

\begin{Entry}{扇}{10}{⼾}
  \begin{Phonetics}{扇}{shan1}[][HSK 5]
    \definition{s.}{ventilar; agitar um leque para fazer o ar circular | dar um tapa; bater com a palma da mão | bater asas; esvoaçar | incitar; instigar; estimular; agitar}
  \end{Phonetics}
  \begin{Phonetics}{扇}{shan4}[][HSK 5]
    \definition{clas.}{usado para portas, janelas, etc.}
    \definition[把]{s.}{leque | folha; algo em forma de placa ou folha}
  \end{Phonetics}
\end{Entry}

\begin{Entry}{扇子}{10,3}{⼾、⼦}
  \begin{Phonetics}{扇子}{shan4zi5}[][HSK 5]
    \definition[把,个]{s.}{leque; abano; abanador; utensílios que produzem vento ao serem agitados}
  \end{Phonetics}
\end{Entry}

\begin{Entry}{拳}{10}{⼿}
  \begin{Phonetics}{拳}{quan2}
    \definition*{s.}{Sobrenome Quan}
    \definition[个,记,套]{s.}{punho | boxe; pugilismo}
    \definition{v.}{enrolar}
  \end{Phonetics}
\end{Entry}

\begin{Entry}{拳王}{10,4}{⼿、⽟}
  \begin{Phonetics}{拳王}{quan2wang2}
    \definition{s.}{pugilista | boxeador}
  \end{Phonetics}
\end{Entry}

\begin{Entry}{拳法}{10,8}{⼿、⽔}
  \begin{Phonetics}{拳法}{quan2fa3}
    \definition{s.}{boxe | luta}
  \end{Phonetics}
\end{Entry}

\begin{Entry}{拿}{10}{⼿}
  \begin{Phonetics}{拿}{na2}[][HSK 1]
    \definition{part.}{usado da mesma forma que 把: para marcar o seguinte substantivo seguinte como objeto direto}
    \definition{prep.}{ferramentas, materiais, métodos, etc. utilizados para a introdução | os objetos que estão sendo manipulados para introdução}
    \definition{v.}{segurar; pegar; pegar ou mover objetos com as mãos ou de outra forma | apreender; capturar; prender; usar força bruta para capturar | ter certeza de; ser capaz de fazer; ter uma compreensão firme de | tornar as coisas difíceis para alguém; colocar alguém em uma situação difícil; obstruir; chantagear; coagir; causar dificuldades intencionalmente | fingir ou fazer (algum tipo de postura ou aparência) | ter certeza de; tomar uma decisão | obter; ganhar; receber}
  \end{Phonetics}
\end{Entry}

\begin{Entry}{拿出}{10,5}{⼿、⼐}
  \begin{Phonetics}{拿出}{na2 chu1}[][HSK 2]
    \definition{v.}{apresentar (evidências) | fornecer | apresentar (uma proposta) | oferecer; servir | retirar; tirar}
  \end{Phonetics}
\end{Entry}

\begin{Entry}{拿走}{10,7}{⼿、⾛}
  \begin{Phonetics}{拿走}{na2 zou3}[][HSK 6]
    \definition{v.}{tirar; remover}
  \end{Phonetics}
\end{Entry}

\begin{Entry}{拿到}{10,8}{⼿、⼑}
  \begin{Phonetics}{拿到}{na2 dao4}[][HSK 2]
    \definition{v.}{pegar; obter, conseguir}
  \end{Phonetics}
\end{Entry}

\begin{Entry}{挨}{10}{⼿}
  \begin{Phonetics}{挨}{ai1}
    \definition{prep.}{por turnos; em sequência; indica sequencialmente}
    \definition{v.}{estar próximo de; estar (ou chegar) perto de; abordar}
  \end{Phonetics}
  \begin{Phonetics}{挨}{ai2}[][HSK 6]
    \definition{v.}{sofrer; suportar | arrastar-se; lutar para sobreviver (tempos difíceis); passar (tempo) com dificuldade | parar; atrasar; adiar; procrastinar}
  \end{Phonetics}
\end{Entry}

\begin{Entry}{挨打}{10,5}{⼿、⼿}
  \begin{Phonetics}{挨打}{ai2/da3}[][HSK 6]
    \definition{v.+compl.}{levar uma surra; ser atacado; ser espancado}
  \end{Phonetics}
\end{Entry}

\begin{Entry}{挨家挨户}{10,10,10,4}{⼿、⼧、⼿、⼾}
  \begin{Phonetics}{挨家挨户}{ai1jia1-ai1hu4}[][HSK 7-9]
    \definition{expr.}{ir de casa em casa, de porta em porta ; um após o outro}
  \end{Phonetics}
\end{Entry}

\begin{Entry}{挨着}{10,11}{⼿、⽬}
  \begin{Phonetics}{挨着}{ai1 zhe5}[][HSK 6]
    \definition{adv.}{ao lado de; perto de; imediatamente depois}
  \end{Phonetics}
\end{Entry}

\begin{Entry}{挫}{10}{⼿}
  \begin{Phonetics}{挫}{cuo4}
    \definition{v.}{frustrar | diminuir; embotar; desinflar | pressionar para baixo; abaixar}
  \end{Phonetics}
\end{Entry}

\begin{Entry}{挫折}{10,7}{⼿、⼿}
  \begin{Phonetics}{挫折}{cuo4zhe2}[][HSK 7-9]
    \definition[个,次]{s.}{retrocesso; reversão; frustração | derrota, fracasso, insucesso}
    \definition{v.}{falhar; derrotar; fracassar}
  \end{Phonetics}
\end{Entry}

\begin{Entry}{振}{10}{⼿}
  \begin{Phonetics}{振}{zhen4}
    \definition{v.}{sacudir; acenar; bater as asas; empunhar | vibrar | recompor-se; levantar-se; animar}
  \end{Phonetics}
\end{Entry}

\begin{Entry}{振动}{10,6}{⼿、⼒}
  \begin{Phonetics}{振动}{zhen4dong4}[][HSK 5]
    \definition{s.}{vibração}
    \definition{v.}{sacudir; balançar; tremer; roncar; tagarelar; vibrar; oscilar; a física se refere ao movimento contínuo de um objeto em torno de um determinado ponto no espaço, como o movimento de um pêndulo, um diapasão ou uma corda de violão}
  \end{Phonetics}
\end{Entry}

\begin{Entry}{捉}{10}{⼿}
  \begin{Phonetics}{捉}{zhuo1}[][HSK 6]
    \definition{v.}{agarrar; segurar; apreender | pegar; capturar; aprisionar}
  \end{Phonetics}
\end{Entry}

\begin{Entry}{捍}{10}{⼿}
  \begin{Phonetics}{捍}{han4}
    \definition{v.}{defender; guardar | defender-se | afastar (um golpe) | resistir}
  \end{Phonetics}
\end{Entry}

\begin{Entry}{捍卫}{10,3}{⼿、⼙}
  \begin{Phonetics}{捍卫}{han4wei4}[][HSK 7-9]
    \definition{v.}{defender; guardar; proteger; defender-se pela força ou outros meios de ser violado ou prejudicado}
  \end{Phonetics}
\end{Entry}

\begin{Entry}{捐}{10}{⼿}
  \begin{Phonetics}{捐}{juan1}[][HSK 6]
    \definition{s.}{imposto}
    \definition{v.}{renunciar; abandonar | contribuir; doar; assinar}
  \end{Phonetics}
\end{Entry}

\begin{Entry}{捐助}{10,7}{⼿、⼒}
  \begin{Phonetics}{捐助}{juan1 zhu4}[][HSK 6]
    \definition{v.}{oferecer (assistência financeira ou material); contribuir; doar}
  \end{Phonetics}
\end{Entry}

\begin{Entry}{捐款}{10,12}{⼿、⽋}
  \begin{Phonetics}{捐款}{juan1/kuan3}[][HSK 6]
    \definition[笔]{s.}{doação; contribuição (de dinheiro); valor doado}
    \definition{v.+compl.}{doar; contribuir com dinheiro}
  \end{Phonetics}
\end{Entry}

\begin{Entry}{捐赠}{10,16}{⼿、⾙}
  \begin{Phonetics}{捐赠}{juan1 zeng4}[][HSK 6]
    \definition{v.}{apresentar; contribuir (como um presente); doar (itens para um país ou grupo)}
  \end{Phonetics}
\end{Entry}

\begin{Entry}{捕}{10}{⼿}
  \begin{Phonetics}{捕}{bu3}[][HSK 6]
    \definition{v.}{pegar; apreender; prender}
  \end{Phonetics}
\end{Entry}

\begin{Entry}{捕捉}{10,10}{⼿、⼿}
  \begin{Phonetics}{捕捉}{bu3zhuo1}[][HSK 7-9]
    \definition{v.}{caçar; perseguir; pegar; capturar; apreender; pegar; fazer uma pessoa ou animal cair nas mãos; pode ser usado tanto para pessoas quanto para coisas; tem uma ampla gama de aplicações; usado tanto na linguagem falada quanto na escrita}
  \end{Phonetics}
\end{Entry}

\begin{Entry}{捞}{10}{⼿}
  \begin{Phonetics}{捞}{lao1}
    \definition{v.}{pescar | dragar}
  \end{Phonetics}
\end{Entry}

\begin{Entry}{损}{10}{⼿}
  \begin{Phonetics}{损}{sun3}
    \definition{adj.}{sarcástico; cortante; de ​​língua afiada; maldoso; mau; cruel}
    \definition{v.}{diminuir; perder; reduzir | prejudicar; danificar; degradar; destruir; arruinar; destruir o estado original ou fazê-lo perder sua eficácia original | ser sarcástico; ser cáustico; ser cortante; ferir; insultar; usar palavras duras para zombar de alguém}
  \end{Phonetics}
\end{Entry}

\begin{Entry}{损失}{10,5}{⼿、⼤}
  \begin{Phonetics}{损失}{sun3shi1}[][HSK 5]
    \definition{s.}{perda; desperdício; algo que se consome ou se perde sem custo algum}
    \definition{v.}{perder; consumir ou perder}
  \end{Phonetics}
\end{Entry}

\begin{Entry}{损害}{10,10}{⼿、⼧}
  \begin{Phonetics}{损害}{sun3 hai4}[][HSK 5]
    \definition{v.}{prejudicar; danificar; ferir; causar danos; causar perdas}
  \end{Phonetics}
\end{Entry}

\begin{Entry}{捡}{10}{⼿}
  \begin{Phonetics}{捡}{jian3}[][HSK 6]
    \definition{v.}{coletar; reunir; apanhar; pegar coisas do chão}
  \end{Phonetics}
\end{Entry}

\begin{Entry}{换}{10}{⼿}
  \begin{Phonetics}{换}{huan4}[][HSK 2]
    \definition{v.}{negociar; trocar; permutar; dar algo a alguém e, ao mesmo tempo, obter algo dele em troca | mudar; transformar; substituir | trocar dinheiro (câmbio) | transfundir (sangue) | transplantar (um órgão)}
  \end{Phonetics}
\end{Entry}

\begin{Entry}{换成}{10,6}{⼿、⼽}
  \begin{Phonetics}{换成}{huan4cheng2}[][HSK 7-9]
    \definition{v.}{trocar (algo) por (outro); indica a substituição de um objeto, estado ou situação por outro}
  \end{Phonetics}
\end{Entry}

\begin{Entry}{换位}{10,7}{⼿、⼈}
  \begin{Phonetics}{换位}{huan4wei4}[][HSK 7-9]
    \definition{v.}{trocar posições; transpor | mudar de posição}
  \end{Phonetics}
\end{Entry}

\begin{Entry}{换言之}{10,7,3}{⼿、⾔、⼂}
  \begin{Phonetics}{换言之}{huan4yan2zhi1}[][HSK 7-9]
    \definition{adv.}{em outras palavras}
  \end{Phonetics}
\end{Entry}

\begin{Entry}{换取}{10,8}{⼿、⼜}
  \begin{Phonetics}{换取}{huan4qu3}[][HSK 7-9]
    \definition{v.}{trocar (ou escambo) algo por; obter em troca | trocar algo por; obter por troca}
  \end{Phonetics}
\end{Entry}

\begin{Entry}{换钱}{10,10}{⼿、⾦}
  \begin{Phonetics}{换钱}{huan4/qian2}
    \definition{v.+compl.}{trocar dinheiro (em pequenas valores ou em outra moeda) | trocar (mercadorias) por dinheiro | vender}
  \end{Phonetics}
\end{Entry}

\begin{Entry}{捣}{10}{⼿}
  \begin{Phonetics}{捣}{dao3}
    \definition{v.}{bater com um pilão, etc.; bater; esmagar | assediar; perturbar | bater com um pedaço de pau}
  \end{Phonetics}
\end{Entry}

\begin{Entry}{捣乱}{10,7}{⼿、⼄}
  \begin{Phonetics}{捣乱}{dao3/luan4}[][HSK 7-9]
    \definition{v.+compl.}{causar problemas; criar uma perturbação; causar intencionalmente problemas para os outros; interromper | perturbar; interferir com; causar problemas intencionalmente}
  \end{Phonetics}
\end{Entry}

\begin{Entry}{效}{10}{⽁}
  \begin{Phonetics}{效}{xiao4}
    \definition{s.}{efeito; função | eficiência; resultado}
    \definition{v.}{imitar; seguir o exemplo de | dedicar (a energia ou a vida de alguém) a; prestar (um serviço)}
  \end{Phonetics}
\end{Entry}

\begin{Entry}{效果}{10,8}{⽁、⽊}
  \begin{Phonetics}{效果}{xiao4guo3}[][HSK 3]
    \definition[种,个]{s.}{efeito; resultado | efeitos sonoros; vários sons ou fenômenos naturais criados para combinar com o enredo em dramas e filmes, como vento e chuva, tiros, fogo, neve, etc.}
  \end{Phonetics}
\end{Entry}

\begin{Entry}{效率}{10,11}{⽁、⽞}
  \begin{Phonetics}{效率}{xiao4lv4}[][HSK 4]
    \definition[种]{s.}{eficiência; produtividade; a quantidade de trabalho concluído por unidade de tempo}
  \end{Phonetics}
\end{Entry}

\begin{Entry}{敌}{10}{⾆}
  \begin{Phonetics}{敌}{di2}
    \definition[个,名,位,种]{s.}{inimigo; adversário}
    \definition{v.}{opor-se; lutar; resistir; suportar | combinar; igualar}
  \end{Phonetics}
\end{Entry}

\begin{Entry}{敌人}{10,2}{⾆、⼈}
  \begin{Phonetics}{敌人}{di2ren2}[][HSK 4]
    \definition[群,伙,帮,个,队]{s.}{inimigo; pessoa hostil; parte hostil}
  \end{Phonetics}
\end{Entry}

\begin{Entry}{料}{10}{⽃}
  \begin{Phonetics}{料}{liao4}[][HSK 6]
    \definition{clas.}{usado na medicina tradicional chinesa para preparar pílulas | unidade usada para calcular um pedaço de madeira, é a seção transversal em ambas as extremidades, que é de 1 pé (quadrado) com 7 pés de comprimento}
    \definition{s.}{material; coisa | (grão) alimento; forragem | artigos de vidro; vidros coloridos opacos | (para pílulas de medicina chinesa) prescrição}
    \definition{v.}{supor; esperar; antecipar | gerenciar; cuidar de | prever}
  \end{Phonetics}
\end{Entry}

\begin{Entry}{旁}{10}{⽅}
  \begin{Phonetics}{旁}{pang2}[][HSK 5]
    \definition{adj.}{outro | abundante; abrangente}
    \definition{s.}{lado | radical lateral de um caractere chinês}
  \end{Phonetics}
\end{Entry}

\begin{Entry}{旁边}{10,5}{⽅、⾡}
  \begin{Phonetics}{旁边}{pang2bian1}[][HSK 1]
    \definition{s.}{junto a; próximo de; ao lado}
  \end{Phonetics}
\end{Entry}

\begin{Entry}{旅}{10}{⽅}
  \begin{Phonetics}{旅}{lv3}
    \definition{adv.}{juntos; conjuntamente}
    \definition[个]{s.}{brigada; unidade organizacional militar, abaixo do nível de divisão e acima do nível de regimento ou batalhão | força; tropas; geralmente se refere aos militares | viajante; passageiro; turista | viagem; jornada | pessoas}
    \definition{v.}{viajar; ficar longe de casa; ir para longe; morar longe de casa}
  \end{Phonetics}
\end{Entry}

\begin{Entry}{旅行}{10,6}{⽅、⾏}
  \begin{Phonetics}{旅行}{lv3xing2}[][HSK 2]
    \definition{v.}{viajar; passear; para tratar de assuntos ou passear, ir de um lugar para outro (geralmente se refere a distâncias longas)}
  \end{Phonetics}
\end{Entry}

\begin{Entry}{旅行社}{10,6,7}{⽅、⾏、⽰}
  \begin{Phonetics}{旅行社}{lv3 xing2 she4}[][HSK 3]
    \definition[家]{s.}{agência de viagens; agência especializada em serviços relacionados a viagens, que providencia hospedagem, transporte e outros serviços para viajantes}
  \end{Phonetics}
\end{Entry}

\begin{Entry}{旅店}{10,8}{⽅、⼴}
  \begin{Phonetics}{旅店}{lv3 dian5}[][HSK 6]
    \definition[家,个]{s.}{pousada; albergue; hotel}
  \end{Phonetics}
\end{Entry}

\begin{Entry}{旅客}{10,9}{⽅、⼧}
  \begin{Phonetics}{旅客}{lv3 ke4}[][HSK 2]
    \definition[名,位,个,些]{s.}{viajante; passageiro; as agências de transporte e turismo referem-se às pessoas que viajam}
  \end{Phonetics}
\end{Entry}

\begin{Entry}{旅馆}{10,11}{⽅、⾷}
  \begin{Phonetics}{旅馆}{lv3 guan3}[][HSK 3]
    \definition[家,个,所]{s.}{pousada; hotel; local comercial destinado ao alojamento de viajantes}
  \end{Phonetics}
\end{Entry}

\begin{Entry}{旅游}{10,12}{⽅、⽔}
  \begin{Phonetics}{旅游}{lv3you2}[][HSK 2]
    \definition{v.}{viajar para outros lugares para passear e fazer turismo}
  \end{Phonetics}
\end{Entry}

\begin{Entry}{旅程}{10,12}{⽅、⽲}
  \begin{Phonetics}{旅程}{lv3cheng2}
    \definition{s.}{jornada | viagem}
  \end{Phonetics}
\end{Entry}

\begin{Entry}{晃}{10}{⽇}
  \begin{Phonetics}{晃}{huang3}[][HSK 7-9]
    \definition*{s.}{Sobrenome Huang}
    \definition{adj.}{deslumbrante}
    \definition{v.}{passar rapidamente | deslumbrar; cegar}
  \end{Phonetics}
  \begin{Phonetics}{晃}{huang4}[][HSK 7-9]
    \definition{v.}{sacudir; balançar}
  \end{Phonetics}
\end{Entry}

\begin{Entry}{晃荡}{10,9}{⽇、⾋}
  \begin{Phonetics}{晃荡}{huang4dang5}[][HSK 7-9]
    \definition{v.}{balançar; sacudir | Coloquial: vagar; ficar ocioso; divagar | oscilar}
  \end{Phonetics}
\end{Entry}

\begin{Entry}{晒}{10}{⽇}
  \begin{Phonetics}{晒}{shai4}[][HSK 4]
    \definition{v.}{(sol) brilhar sobre | aquecer-se; secar ao sol; colocar algo sob a luz do sol para secar | ignorar (alguém) | mostrar; divulgar o conteúdo de sua vida privada na Internet}
  \end{Phonetics}
\end{Entry}

\begin{Entry}{晒干}{10,3}{⽇、⼲}
  \begin{Phonetics}{晒干}{shai4gan1}
    \definition{v.}{secar ao sol}
  \end{Phonetics}
\end{Entry}

\begin{Entry}{晓}{10}{⽇}
  \begin{Phonetics}{晓}{xiao3}
    \definition{s.}{amanhecer; alvorada}
    \definition{v.}{(um dia) amanhecer; romper | saber; deixar alguém saber; dizer}
  \end{Phonetics}
\end{Entry}

\begin{Entry}{晓得}{10,11}{⽇、⼻}
  \begin{Phonetics}{晓得}{xiao3 de2}[][HSK 6]
    \definition{v.}{saber; entender}[我不晓得他在哪里。===Não sei onde ele está.]
  \end{Phonetics}
\end{Entry}

\begin{Entry}{晕}{10}{⽇}
  \begin{Phonetics}{晕}{yun1}[][HSK 6]
    \definition{adj.}{tonto; vertiginoso; confuso; sensação de que as coisas estão girando ao seu redor e, às vezes, sensação de que você vai cair}
    \definition{v.}{desmaiar; desfalecer}
  \end{Phonetics}
  \begin{Phonetics}{晕}{yun4}
    \definition{s.}{auréola; o círculo de luz formado pela refração da luz solar ou do luar através dos cristais de gelo nas nuvens | halo em torno de alguma cor ou luz; áreas desfocadas em torno de luz, sombra e cor}
    \definition{v.}{ficar tonto; desmaiar; desfalecer; sensação de tontura, como se os objetos ao seu redor estivessem girando e como se você estivesse prestes a cair}
  \end{Phonetics}
\end{Entry}

\begin{Entry}{晕车}{10,4}{⽇、⾞}
  \begin{Phonetics}{晕车}{yun4 che1}[][HSK 6]
    \definition{v.}{ter enjoo no carro; ter tontura e vômito ao andar de carro}
  \end{Phonetics}
\end{Entry}

\begin{Entry}{朗}{10}{⽉}
  \begin{Phonetics}{朗}{lang3}
    \definition*{s.}{Sobrenome Lang}
    \definition{adj.}{claro; brilhante | alto e claro (som)}
  \end{Phonetics}
\end{Entry}

\begin{Entry}{朗读}{10,10}{⽉、⾔}
  \begin{Phonetics}{朗读}{lang3du2}[][HSK 5]
    \definition{v.}{ler em voz alta; recitar com voz clara e alta}
  \end{Phonetics}
\end{Entry}

\begin{Entry}{校}{10}{⽊}
  \begin{Phonetics}{校}{jiao4}
    \definition{v.}{verificar | comparar | revisar}
  \end{Phonetics}
  \begin{Phonetics}{校}{xiao4}
    \definition[所]{s.}{oficial militar | escola}
  \end{Phonetics}
\end{Entry}

\begin{Entry}{校长}{10,4}{⽊、⾧}
  \begin{Phonetics}{校长}{xiao4zhang3}[][HSK 2]
    \definition[个,位,名]{s.}{diretor; presidente; reitor; o mais alto líder administrativo e empresarial de uma escola}
  \end{Phonetics}
\end{Entry}

\begin{Entry}{校园}{10,7}{⽊、⼞}
  \begin{Phonetics}{校园}{xiao4 yuan2}[][HSK 2]
    \definition[个]{s.}{campus; pátio da escola; refere-se a todos os terrenos e edifícios dentro da área escolar}
  \end{Phonetics}
\end{Entry}

\begin{Entry}{校服}{10,8}{⽊、⽉}
  \begin{Phonetics}{校服}{xiao4fu2}
    \definition{s.}{uniforme escolar}
  \end{Phonetics}
\end{Entry}

\begin{Entry}{校规}{10,8}{⽊、⾒}
  \begin{Phonetics}{校规}{xiao4gui1}
    \definition{s.}{regras e regulamentos escolares}
  \end{Phonetics}
\end{Entry}

\begin{Entry}{校监}{10,10}{⽊、⽫}
  \begin{Phonetics}{校监}{xiao4jian1}
    \definition{s.}{diretor | supervisor (de escola)}
  \end{Phonetics}
\end{Entry}

\begin{Entry}{样}{10}{⽊}
  \begin{Phonetics}{样}{yang4}[][HSK 6]
    \definition{clas.}{usado para tipos de coisas}[这里有四样东西。===Há quatro coisas aqui.]
    \definition[个]{s.}{aparência; aspecto;  forma; aparência; a forma do objeto | modelo; amostra; padrão; coisas usadas como padrões | ar; maneira; aparência; a aparência ou expressão de uma pessoa | tendência; probabilidade; a situação ou tendência das coisas}
  \end{Phonetics}
\end{Entry}

\begin{Entry}{样儿}{10,2}{⽊、⼉}
  \begin{Phonetics}{样儿}{yang4r5}
    \definition{s.}{aparência | forma | modelo}
  \seealsoref{样子}{yang4zi5}
  \end{Phonetics}
\end{Entry}

\begin{Entry}{样子}{10,3}{⽊、⼦}
  \begin{Phonetics}{样子}{yang4zi5}[][HSK 2]
    \definition[个,种,副]{s.}{forma; aparência; estilo | ar; maneira; modalidade; estado | tendência; probabilidade; usado com 看 e 照 para expressar uma estimativa de uma tendência | modelo; amostra; padrão; uma pessoa ou coisa que pode ser usada como um padrão para as pessoas verificarem, seguirem ou aprenderem com ela}
  \seealsoref{看}{kan4}
  \seealsoref{样儿}{yang4r5}
  \seealsoref{照}{zhao4}
  \end{Phonetics}
\end{Entry}

\begin{Entry}{样品}{10,9}{⽊、⼝}
  \begin{Phonetics}{样品}{yang4pin3}
    \definition{s.}{amostra | espécime}
  \end{Phonetics}
\end{Entry}

\begin{Entry}{样样}{10,10}{⽊、⽊}
  \begin{Phonetics}{样样}{yang4yang4}
    \definition{adv.}{todos os tipos}
  \end{Phonetics}
\end{Entry}

\begin{Entry}{样章}{10,11}{⽊、⾳}
  \begin{Phonetics}{样章}{yang4zhang1}
    \definition{s.}{capítulo de amostra}
  \end{Phonetics}
\end{Entry}

\begin{Entry}{核}{10}{⽊}
  \begin{Phonetics}{核}{he2}[][HSK 7-9]
    \definition{adj.}{Literário: verdadeiro; fiel}
    \definition{s.}{poço; pedra; caroço | núcleo | núcleo atômico}
    \definition{v.}{examinar; verificar}
  \end{Phonetics}
  \begin{Phonetics}{核}{hu2}
    \definition{s.}{semente; o mesmo que 核}
  \end{Phonetics}
\end{Entry}

\begin{Entry}{核心}{10,4}{⽊、⼼}
  \begin{Phonetics}{核心}{he2xin1}[][HSK 6]
    \definition[个]{s.}{núcleo; elite; coração; centro; parte principal (em termos de relacionamento entre as coisas)}
  \end{Phonetics}
\end{Entry}

\begin{Entry}{核对}{10,5}{⽊、⼨}
  \begin{Phonetics}{核对}{he2dui4}[][HSK 7-9]
    \definition{v.}{verificar; checar; verificar cuidadosamente (para ver se corresponde)}
  \end{Phonetics}
\end{Entry}

\begin{Entry}{核电站}{10,5,10}{⽊、⽥、⽴}
  \begin{Phonetics}{核电站}{he2dian4zhan4}[][HSK 7-9]
    \definition{s.}{usina nuclear; usina que utiliza energia nuclear para gerar eletricidade}
  \end{Phonetics}
\end{Entry}

\begin{Entry}{核实}{10,8}{⽊、⼧}
  \begin{Phonetics}{核实}{he2shi2}[][HSK 7-9]
    \definition{v.}{verificar; checar; verificar se é verdade}
  \end{Phonetics}
\end{Entry}

\begin{Entry}{核武器}{10,8,16}{⽊、⽌、⼝}
  \begin{Phonetics}{核武器}{he2wu3qi4}[][HSK 7-9]
    \definition[个]{s.}{arma nuclear}
  \end{Phonetics}
\end{Entry}

\begin{Entry}{核桃}{10,10}{⽊、⽊}
  \begin{Phonetics}{核桃}{he2tao5}[][HSK 7-9]
    \definition[颗,个,棵,顆]{s.}{noz | nogueira}
  \end{Phonetics}
\end{Entry}

\begin{Entry}{核能}{10,10}{⽊、⾁}
  \begin{Phonetics}{核能}{he2neng2}[][HSK 7-9]
    \definition{s.}{energia nuclear}
  \end{Phonetics}
\end{Entry}

\begin{Entry}{根}{10}{⽊}
  \begin{Phonetics}{根}{gen1}[][HSK 4]
    \definition*{s.}{Sobrenome Gen}
    \definition{adv.}{completamente; minuciosamente; radicalmente}
    \definition{clas.}{usado para objetos finos, alongados}
    \definition{s.}{raiz (de uma planta) | descendentes; posteridade; analogia com as gerações futuras | raiz (abreviação de raiz quadrada) | radical (química, refere-se a radicais carregados) | base; pé; raiz; parte inferior, base ou parte de um objeto que está presa a outra coisa | a parte de baixo das coisas; fonte; a origem  das coisas | base; fundamento}
  \end{Phonetics}
\end{Entry}

\begin{Entry}{根本}{10,5}{⽊、⽊}
  \begin{Phonetics}{根本}{gen1ben3}[][HSK 3]
    \definition{adj.}{básico; essencial; fundamental; importante; decisivo}
    \definition{adv.}{nunca; simplesmente; de forma alguma | radicalmente; completamente; nunca (mais usado em negativas)}
    \definition[个]{s.}{base; raiz; fundação; a origem, a base ou a parte mais importante das coisas}
  \end{Phonetics}
\end{Entry}

\begin{Entry}{根治}{10,8}{⽊、⽔}
  \begin{Phonetics}{根治}{gen1zhi4}[][HSK 7-9]
    \definition{v.}{efetuar uma cura radical; curar de uma vez por todas; colocar sob controle permanente; curar completamente (referindo-se à erradicação de pragas ou doenças)}
  \end{Phonetics}
\end{Entry}

\begin{Entry}{根基}{10,11}{⽊、⼟}
  \begin{Phonetics}{根基}{gen1ji1}[][HSK 7-9]
    \definition{s.}{base; fundação; alicerce; parte subterrânea de um edifício | recursos; propriedade acumulada ao longo de um longo período}
  \end{Phonetics}
\end{Entry}

\begin{Entry}{根据}{10,11}{⽊、⼿}
  \begin{Phonetics}{根据}{gen1ju4}[][HSK 4]
    \definition{prep.}{com base em; de acordo com; à luz de}
    \definition[个]{s.}{base; fundamentos; razão; fundo; alicerce}
    \definition{v.}{basear; usar algo como premissa para uma conclusão ou como base para uma ação verbal}
  \end{Phonetics}
\end{Entry}

\begin{Entry}{根深蒂固}{10,11,12,8}{⽊、⽔、⾋、⼞}
  \begin{Phonetics}{根深蒂固}{gen1shen1-di4gu4}[][HSK 7-9]
    \definition{expr.}{arraigado; inveterado; tornar-se profundamente enraizado em; profundamente enraizado; profundamente enraizado; profundamente enraizado e firmemente plantado -- bem fundado; ter uma base firme; ter raízes profundas e uma base firme; bem estabelecido; significa que a fundação é sólida e não se abala facilmente}
  \end{Phonetics}
\end{Entry}

\begin{Entry}{根源}{10,13}{⽊、⽔}
  \begin{Phonetics}{根源}{gen1yuan2}[][HSK 7-9]
    \definition{s.}{fonte; origem; raiz | raízes da grama; fonte; nascente; raiz; fundo}
    \definition{v.}{originar-se; provir de}
  \end{Phonetics}
\end{Entry}

\begin{Entry}{格}{10}{⽊}
  \begin{Phonetics}{格}{ge1}
    \definition{s.}{Onomatopéia: estalo (som); riso zombeteiro}
  \end{Phonetics}
  \begin{Phonetics}{格}{ge2}[][HSK 7-9]
    \definition*{s.}{Sobrenome Ge}
    \definition{s.}{quadrados formados por linhas cruzadas; quadriculado; grade | divisão (horizontal ou não); treliça | padrão; forma; formato; estilo | caso; as categorias morfológicas de substantivos, pronomes e adjetivos em algumas línguas}
    \definition{v.}{resistir; dificultar; obstruir; impedir | estudar cuidadosamente; investigar | lutar; bater}
  \end{Phonetics}
\end{Entry}

\begin{Entry}{格兰菜}{10,5,11}{⽊、⼋、⾋}
  \begin{Phonetics}{格兰菜}{ge2lan2cai4}
    \definition{s.}{brócolis chinês | couve chinesa | mostarda}
  \seealsoref{芥蓝}{gai4lan2}
  \end{Phonetics}
\end{Entry}

\begin{Entry}{格外}{10,5}{⽊、⼣}
  \begin{Phonetics}{格外}{ge2wai4}[][HSK 4]
    \definition{adv.}{especialmente; particularmente; ainda mais; indica mais do que a média | adicionalmente; indica adicional ou extra}
  \end{Phonetics}
\end{Entry}

\begin{Entry}{格式}{10,6}{⽊、⼷}
  \begin{Phonetics}{格式}{ge2shi5}[][HSK 7-9]
    \definition[种]{s.}{forma; estilo; \emph{layout}; padrão; formato; modo}
  \end{Phonetics}
\end{Entry}

\begin{Entry}{格局}{10,7}{⽊、⼫}
  \begin{Phonetics}{格局}{ge2ju2}[][HSK 7-9]
    \definition{s.}{padrão; configuração; estrutura; estilo; maneira; arranjo | a visão ou percepção de uma situação geral; a visão de uma pessoa, a altura e a profundidade da consideração do problema}
  \end{Phonetics}
\end{Entry}

\begin{Entry}{格格不入}{10,10,4,2}{⽊、⽊、⼀、⼊}
  \begin{Phonetics}{格格不入}{ge2ge2-bu2ru4}[][HSK 7-9]
    \definition{expr.}{incompatível com; fora de sintonia com; estranho; fora do seu elemento; como uma estaca quadrada em um buraco redondo; desarmônico}
  \end{Phonetics}
\end{Entry}

\begin{Entry}{栽}{10}{⽊}
  \begin{Phonetics}{栽}{zai1}
    \definition{v.}{cultivar | plantar}
  \end{Phonetics}
\end{Entry}

\begin{Entry}{栽种}{10,9}{⽊、⽲}
  \begin{Phonetics}{栽种}{zai1zhong4}
    \definition{v.}{plantar}
  \end{Phonetics}
\end{Entry}

\begin{Entry}{栽倒}{10,10}{⽊、⼈}
  \begin{Phonetics}{栽倒}{zai1dao3}
    \definition{v.}{cair | sofrer uma queda}
  \end{Phonetics}
\end{Entry}

\begin{Entry}{栽赃}{10,10}{⽊、⾙}
  \begin{Phonetics}{栽赃}{zai1zang1}
    \definition{v.}{enquadrar alguém (plantar provas nele)}
  \end{Phonetics}
\end{Entry}

\begin{Entry}{栽培}{10,11}{⽊、⼟}
  \begin{Phonetics}{栽培}{zai1pei2}
    \definition{v.}{cultivar | educar | patrocinar | treinar}
  \end{Phonetics}
\end{Entry}

\begin{Entry}{栽培种}{10,11,9}{⽊、⼟、⽲}
  \begin{Phonetics}{栽培种}{zai1pei2 zhong3}
    \definition{s.}{espécies cultivadas}
  \end{Phonetics}
\end{Entry}

\begin{Entry}{栽植}{10,12}{⽊、⽊}
  \begin{Phonetics}{栽植}{zai1zhi2}
    \definition{v.}{plantar | transplantar}
  \end{Phonetics}
\end{Entry}

\begin{Entry}{桂}{10}{⽊}
  \begin{Phonetics}{桂}{gui4}
    \definition*{s.}{outro nome para o rio Guijiang 桂江 (em Guangxi 广西) | outro nome para Guangxi 广西 (Região Autônoma de Zhuang) | Sobrenome Gui}
    \definition[棵]{s.}{louro; loureiro | osmanthus de aroma doce | árvore de casca de cássia | canela; osmanthus}
  \seealsoref{广西}{guang3xi1}
  \seealsoref{桂江}{gui4jiang1}
  \end{Phonetics}
\end{Entry}

\begin{Entry}{桂江}{10,6}{⽊、⽔}
  \begin{Phonetics}{桂江}{gui4jiang1}
    \definition*{s.}{Rio Guijiang}
  \end{Phonetics}
\end{Entry}

\begin{Entry}{桂花}{10,7}{⽊、⾋}
  \begin{Phonetics}{桂花}{gui4hua1}[][HSK 7-9]
    \definition{s.}{jasmim do imperador; um arbusto perene ou pequena árvore, cujas flores também são chamadas de osmanthus, são muito perfumadas e podem ser usadas para extrair óleos aromáticos ou fazer especiarias. Variedades comuns incluem Jingui 金桂 (flores amarelo-alaranjadas), Dangui 丹桂 (flores vermelho-alaranjadas), Yingui 银桂 (flores branco-amareladas) e Sijigui 四季桂 (flores branco-amareladas).}
  \end{Phonetics}
\end{Entry}

\begin{Entry}{桃}{10}{⽊}
  \begin{Phonetics}{桃}{tao2}[][HSK 5]
    \definition*{s.}{Sobrenome Tao}
    \definition[个,箱,袋,斤,棵,种]{s.}{pêssego | em forma de pêssego | pessegueiro}
  \end{Phonetics}
\end{Entry}

\begin{Entry}{桃花}{10,7}{⽊、⾋}
  \begin{Phonetics}{桃花}{tao2 hua1}[][HSK 5]
    \definition[朵,枝,株]{s.}{Figurativo: caso amoroso | flor de pessegueiro}
  \end{Phonetics}
\end{Entry}

\begin{Entry}{桃树}{10,9}{⽊、⽊}
  \begin{Phonetics}{桃树}{tao2 shu4}[][HSK 5]
    \definition[棵,株]{s.}{pêssego (árvore) | pessegueiro; pêssegos}
  \end{Phonetics}
\end{Entry}

\begin{Entry}{案}{10}{⽊}
  \begin{Phonetics}{案}{an4}
    \definition{s.}{mesa; escrivaninha; mesa longa | caso; caso de direito (legal) | registro; arquivo; arquivo de caso | um plano submetido para consideração; proposta; um documento que propõe planos, sugestões, métodos, etc.}
  \end{Phonetics}
\end{Entry}

\begin{Entry}{案件}{10,6}{⽊、⼈}
  \begin{Phonetics}{案件}{an4jian4}[][HSK 7-9]
    \definition[个,起,件,类]{s.}{caso; caso de direito; caso legal; contencioso e eventos ilegais}
  \end{Phonetics}
\end{Entry}

\begin{Entry}{桌}{10}{⽊}
  \begin{Phonetics}{桌}{zhuo1}
    \definition{clas.}{usado para mesas de convidados em um banquete etc.}
    \definition{s.}{mesa}
  \end{Phonetics}
\end{Entry}

\begin{Entry}{桌子}{10,3}{⽊、⼦}
  \begin{Phonetics}{桌子}{zhuo1zi5}[][HSK 1]
    \definition[张,套]{s.}{mesa; escrivaninha; móveis, com uma superfície plana na parte superior e uma estrutura de suporte na parte inferior, para colocar objetos ou realizar atividades}
  \end{Phonetics}
\end{Entry}

\begin{Entry}{桌布}{10,5}{⽊、⼱}
  \begin{Phonetics}{桌布}{zhuo1bu4}
    \definition[条,块,张]{s.}{(computação) plano de fundo da área de trabalho | toalha de mesa | papel de parede}
  \end{Phonetics}
\end{Entry}

\begin{Entry}{桌机}{10,6}{⽊、⽊}
  \begin{Phonetics}{桌机}{zhuo1ji1}
    \definition{s.}{computador \emph{desktop}}
  \end{Phonetics}
\end{Entry}

\begin{Entry}{桌灯}{10,6}{⽊、⽕}
  \begin{Phonetics}{桌灯}{zhuo1deng1}
    \definition{s.}{luminária | lâmpada de mesa}
  \end{Phonetics}
\end{Entry}

\begin{Entry}{桌面}{10,9}{⽊、⾯}
  \begin{Phonetics}{桌面}{zhuo1mian4}
    \definition{s.}{área de trabalho | mesa}
  \end{Phonetics}
\end{Entry}

\begin{Entry}{桌球}{10,11}{⽊、⽟}
  \begin{Phonetics}{桌球}{zhuo1qiu2}
    \definition{s.}{bilhar | sinuca | mesa de ping-pong}
  \end{Phonetics}
\end{Entry}

\begin{Entry}{桌游}{10,12}{⽊、⽔}
  \begin{Phonetics}{桌游}{zhuo1you2}
    \definition{s.}{jogo de tabuleiro}
  \end{Phonetics}
\end{Entry}

\begin{Entry}{桑}{10}{⽊}
  \begin{Phonetics}{桑}{sang1}
    \definition*{s.}{Sobrenome Sang}
    \definition[棵]{s.}{amoreira}
  \end{Phonetics}
\end{Entry}

\begin{Entry}{桑巴舞}{10,4,14}{⽊、⼰、⾇}
  \begin{Phonetics}{桑巴舞}{sang1ba1wu3}
    \definition{s.}{samba}
  \end{Phonetics}
\end{Entry}

\begin{Entry}{桑树}{10,9}{⽊、⽊}
  \begin{Phonetics}{桑树}{sang1shu4}
    \definition{s.}{amoreira, suas folhas são utilizadas para alimentar bichos-da-seda}
  \end{Phonetics}
\end{Entry}

\begin{Entry}{档}{10}{⽊}
  \begin{Phonetics}{档}{dang4}[][HSK 6]
    \definition{clas.}{festa; usado para eventos, shows}
    \definition{s.}{prateleiras (para arquivos); compartimentos para documentos | arquivos; arquivos | travessa (de uma mesa, etc.) | qualidade; nota}
  \end{Phonetics}
\end{Entry}

\begin{Entry}{档次}{10,6}{⽊、⽋}
  \begin{Phonetics}{档次}{dang4ci4}[][HSK 7-9]
    \definition{s.}{classe; grau; qualidade; nível; diferentes níveis divididos de acordo com certos padrões}
  \end{Phonetics}
\end{Entry}

\begin{Entry}{档案}{10,10}{⽊、⽊}
  \begin{Phonetics}{档案}{dang4'an4}[][HSK 6]
    \definition[份,个]{s.}{arquivos; registro; dossiê; arquivos e materiais armazenados de forma classificada para referência futura}
  \end{Phonetics}
\end{Entry}

\begin{Entry}{桥}{10}{⽊}
  \begin{Phonetics}{桥}{qiao2}[][HSK 3]
    \definition*{s.}{Sobrenome Qiao}
    \definition[座]{s.}{ponte; construção que atravessa a água conectando as duas margens}
  \end{Phonetics}
\end{Entry}

\begin{Entry}{桥梁}{10,11}{⽊、⽊}
  \begin{Phonetics}{桥梁}{qiao2liang2}[][HSK 6]
    \definition[座]{s.}{ponte; acesso; uma obra construída na superfície do rio, conectando as duas margens | ponte; metáfora para pessoas ou coisas que podem se comunicar}
  \end{Phonetics}
\end{Entry}

\begin{Entry}{桩}{10}{⽊}
  \begin{Phonetics}{桩}{zhuang1}
    \definition{clas.}{para eventos, casos, transações, assuntos, etc.}
    \definition{s.}{toco | estaca | pilha}
  \end{Phonetics}
\end{Entry}

\begin{Entry}{欱}{10}{⽋}
  \begin{Phonetics}{欱}{he1}
    \definition{v.}{beber | beber bebida alcoólica}
    \variantof{喝}
  \end{Phonetics}
\end{Entry}

\begin{Entry}{氧}{10}{⽓}
  \begin{Phonetics}{氧}{yang3}
    \definition{s.}{oxigênio}
  \end{Phonetics}
\end{Entry}

\begin{Entry}{氧气}{10,4}{⽓、⽓}
  \begin{Phonetics}{氧气}{yang3qi4}[][HSK 6]
    \definition{s.}{oxigênio (O); gás oxigênio}
  \end{Phonetics}
\end{Entry}

\begin{Entry}{流}{10}{⽔}
  \begin{Phonetics}{流}{liu2}[][HSK 2]
    \definition*{s.}{Sobrenome Liu}
    \definition{adj.}{fluente; tão suave quanto a água corrente}
    \definition{clas.}{lúmen; abreviação de lumens, 流明}
    \definition[名,个]{s.}{corrente de água | corrente; algo que se assemelha a um fluxo de água | razão; taxa; classe; grau; ramificação; facção; hierarquia}
    \definition{v.}{(de líquido) fluir | vaguear; vagar; mover-se de um lugar para outro; movimentar-se sem direção fixa | espalhar; circular; transmitir; divulgar | degenerar; mudar para pior; tender (aspecto negativo) | banir; enviar para o exílio | correr (ou fluir) como líquido; refere-se à parte do rio após deixar sua nascente (em contraste com a 源)}
  \seealsoref{流明}{liu2ming2}
  \seealsoref{源}{yuan2}
  \end{Phonetics}
\end{Entry}

\begin{Entry}{流水}{10,4}{⽔、⽔}
  \begin{Phonetics}{流水}{liu2shui3}
    \definition{s.}{água corrente | (negócio) rotatividade}
  \end{Phonetics}
\end{Entry}

\begin{Entry}{流传}{10,6}{⽔、⼈}
  \begin{Phonetics}{流传}{liu2chuan2}[][HSK 4]
    \definition[间]{v.}{espalhar; circular; passar adiante}
  \end{Phonetics}
\end{Entry}

\begin{Entry}{流动}{10,6}{⽔、⼒}
  \begin{Phonetics}{流动}{liu2 dong4}[][HSK 5]
    \definition{v.}{(água, ar, etc.) fluir; correr; circular | ir de um lugar para outro; estar em movimento; ser móvel (oposto a 固定)}
  \seealsoref{固定}{gu4ding4}
  \end{Phonetics}
\end{Entry}

\begin{Entry}{流行}{10,6}{⽔、⾏}
  \begin{Phonetics}{流行}{liu2xing2}[][HSK 2]
    \definition{adj.}{popular; na moda; muito popular}
    \definition{v.}{ser popular; prevalecer; espalhar-se amplamente; divulgar amplamente}
  \end{Phonetics}
\end{Entry}

\begin{Entry}{流行性感冒}{10,6,8,13,9}{⽔、⾏、⼼、⼼、⽇}
  \begin{Phonetics}{流行性感冒}{liu2xing2 xing4 gan3mao4}
    \definition{s.}{gripe muito forte; influenza}
  \end{Phonetics}
\end{Entry}

\begin{Entry}{流利}{10,7}{⽔、⼑}
  \begin{Phonetics}{流利}{liu2li4}[][HSK 2]
    \definition{adj.}{fluente; suave; lúcido; falar e escrever com fluência e clareza | com fluência; sem dificuldades}
  \end{Phonetics}
\end{Entry}

\begin{Entry}{流明}{10,8}{⽔、⽇}
  \begin{Phonetics}{流明}{liu2ming2}
    \definition{s.}{(empréstimo linguístico) lúmen (unidade de fluxo luminoso)}
  \end{Phonetics}
\end{Entry}

\begin{Entry}{流星}{10,9}{⽔、⽇}
  \begin{Phonetics}{流星}{liu2xing1}
    \definition{s.}{meteoro | estrela cadente}
  \end{Phonetics}
\end{Entry}

\begin{Entry}{流通}{10,10}{⽔、⾡}
  \begin{Phonetics}{流通}{liu2tong1}[][HSK 5]
    \definition{v.}{(ar, dinheiro, mercadorias, etc.) fluir; circular}
  \end{Phonetics}
\end{Entry}

\begin{Entry}{流感}{10,13}{⽔、⼼}
  \begin{Phonetics}{流感}{liu2 gan3}[][HSK 6]
    \definition{s.}{gripe; influenza; abreviação de 流行性感冒}
  \seealsoref{流行性感冒}{liu2xing2 xing4 gan3mao4}
  \end{Phonetics}
\end{Entry}

\begin{Entry}{浙}{10}{⽔}
  \begin{Phonetics}{浙}{zhe4}
    \definition{s.}{abreviação de província de Zhejiang,  浙江, no leste da China}
  \seealsoref{浙江}{zhe4jiang1}
  \end{Phonetics}
\end{Entry}

\begin{Entry}{浙江}{10,6}{⽔、⽔}
  \begin{Phonetics}{浙江}{zhe4jiang1}
    \definition*{s.}{Província de Zhejiang}
  \end{Phonetics}
\end{Entry}

\begin{Entry}{浩}{10}{⽔}
  \begin{Phonetics}{浩}{hao4}
    \definition*{s.}{Sobrenome Hao}
    \definition{adj.}{grande; vasto; grandioso; sem limites | um grande número; infinito}
  \end{Phonetics}
\end{Entry}

\begin{Entry}{浩劫}{10,7}{⽔、⼒}
  \begin{Phonetics}{浩劫}{hao4jie2}[][HSK 7-9]
    \definition[场,次]{s.}{grande calamidade; catástrofe | devastação; holocausto; flagelo}
  \end{Phonetics}
\end{Entry}

\begin{Entry}{浪}{10}{⽔}
  \begin{Phonetics}{浪}{lang4}
    \definition*{s.}{Sobrenome Lang}
    \definition{adj.}{desenfreado; perdulário}
    \definition{adv.}{livremente}
    \definition[朵,阵,波]{s.}{onda; vagalhão; rebentação | algo ondulatório | coisas ondulando como ondas}
    \definition{v.}{passear; divagar}
  \end{Phonetics}
\end{Entry}

\begin{Entry}{浪花}{10,7}{⽔、⾋}
  \begin{Phonetics}{浪花}{lang4hua1}
    \definition[朵]{s.}{\emph{spray} | \emph{spray} do oceano | (figurativo) acontecimentos de sua vida}
  \end{Phonetics}
\end{Entry}

\begin{Entry}{浪费}{10,9}{⽔、⾙}
  \begin{Phonetics}{浪费}{lang4fei4}[][HSK 3]
    \definition{adj.}{desperdiçado; extravagante; não econômico}
    \definition{v.}{desperdiçar; dissipar; esbanjar; ser extravagante; uso excessivo ou inadequado de bens, recursos humanos, tempo, etc.}
  \end{Phonetics}
\end{Entry}

\begin{Entry}{浪漫}{10,14}{⽔、⽔}
  \begin{Phonetics}{浪漫}{lang4man4}[][HSK 5]
    \definition{adj.}{romântico; poético | não convencional; boêmio; abandonado; libertino; devasso; comportar-se de maneira descuidada e descuidada (geralmente se referindo a relacionamentos entre pessoas) | irrealista; impraticável}
  \end{Phonetics}
\end{Entry}

\begin{Entry}{浮}{10}{⽔}
  \begin{Phonetics}{浮}{fu2}[][HSK 6]
    \definition*{s.}{Sobrenome Fu}
    \definition{adj.}{superficial; na superfície | móvel; removível | temporário; provisório | superficial e frívolo; volátil; impetuoso | oco; vazio; inflado | excessivo; excedente}
    \definition{v.}{flutuar (oposto a 沉) | (dialeto) nadar | flutuar; derivar; flutuar na superfície do líquido}
  \seealsoref{沉}{chen2}
  \end{Phonetics}
\end{Entry}

\begin{Entry}{浮力}{10,2}{⽔、⼒}
  \begin{Phonetics}{浮力}{fu2li4}[][HSK 7-9]
    \definition{s.}{flutuabilidade; a força de empuxo, a força ascendente exercida sobre um objeto em um fluido, é igual ao peso do fluido deslocado pelo objeto}
  \end{Phonetics}
\end{Entry}

\begin{Entry}{浮图}{10,8}{⽔、⼞}
  \begin{Phonetics}{浮图}{fu2tu2}
    \definition*{s.}{Termo alternativo para 佛陀}
    \variantof{浮屠}
  \seealsoref{佛陀}{fo2tuo2}
  \end{Phonetics}
\end{Entry}

\begin{Entry}{浮现}{10,8}{⽔、⾒}
  \begin{Phonetics}{浮现}{fu2xian4}[][HSK 7-9]
    \definition{v.}{(experiência passada) ressurgir; vir à mente | aparecer; apresentar; revelar}
  \end{Phonetics}
\end{Entry}

\begin{Entry}{浮屠}{10,11}{⽔、⼫}
  \begin{Phonetics}{浮屠}{fu2tu2}
    \definition*{s.}{Buda | Templo (Stupa) Budista (transliteração de Pali Thuo)}
  \end{Phonetics}
\end{Entry}

\begin{Entry}{浮躁}{10,20}{⽔、⾜}
  \begin{Phonetics}{浮躁}{fu2zao4}[][HSK 7-9]
    \definition{adj.}{inquieto; impetuoso; impaciente; frívolo e impaciente}
  \end{Phonetics}
\end{Entry}

\begin{Entry}{海}{10}{⽔}
  \begin{Phonetics}{海}{hai3}[][HSK 2]
    \definition*{s.}{Sobrenome Hai}
    \definition{adj.}{extragrande; de grande capacidade; descreve capacidade, tom de voz, etc.}
    \definition{adv.}{aleatoriamente; sem rumo; sem limites; sem restrições}
    \definition[片]{s.}{mar; grande lago; a parte do oceano próxima à costa, alguns grandes lagos também são chamados de mar | grande número de pessoas ou coisas reunidas; metáfora para muitas coisas semelhantes que formam um grande conjunto}
  \end{Phonetics}
\end{Entry}

\begin{Entry}{海内外}{10,4,5}{⽔、⼌、⼣}
  \begin{Phonetics}{海内外}{hai3 nei4wai4}[][HSK 7-9]
    \definition{s.}{em casa e no exterior | nacional e internacional}
  \end{Phonetics}
\end{Entry}

\begin{Entry}{海水}{10,4}{⽔、⽔}
  \begin{Phonetics}{海水}{hai3 shui3}[][HSK 4]
    \definition[把]{s.}{água do mar; salmoura}
  \end{Phonetics}
\end{Entry}

\begin{Entry}{海风}{10,4}{⽔、⾵}
  \begin{Phonetics}{海风}{hai3feng1}
    \definition{s.}{brisa do mar | vento que vem do mar}
  \end{Phonetics}
\end{Entry}

\begin{Entry}{海外}{10,5}{⽔、⼣}
  \begin{Phonetics}{海外}{hai3 wai4}[][HSK 6]
    \definition[次]{s.}{fora das fronteiras nacionais; no exterior}
  \end{Phonetics}
\end{Entry}

\begin{Entry}{海边}{10,5}{⽔、⾡}
  \begin{Phonetics}{海边}{hai3 bian1}[][HSK 2]
    \definition{s.}{praia; costa; litoral; orla marítima; a parte marginal do oceano e as grandes áreas de água salgada cercadas por terra firme, onde a terra e a água se encontram, formam a costa}
  \end{Phonetics}
\end{Entry}

\begin{Entry}{海关}{10,6}{⽔、⼋}
  \begin{Phonetics}{海关}{hai3guan1}[][HSK 3]
    \definition[个]{s.}{alfândega; órgão administrativo nacional, sua principal função é supervisionar e inspecionar os bens e meios de transporte que entram e saem do país, cobrar impostos alfandegários e reprimir o contrabando}
  \end{Phonetics}
\end{Entry}

\begin{Entry}{海军}{10,6}{⽔、⼍}
  \begin{Phonetics}{海军}{hai3 jun1}[][HSK 6]
    \definition[支,名,位,个]{s.}{marinha; o exército que luta no mar geralmente é composto por navios de superfície, submarinos, aviação naval, fuzileiros navais e outros ramos, além de diversas forças profissionais}
  \end{Phonetics}
\end{Entry}

\begin{Entry}{海报}{10,7}{⽔、⼿}
  \begin{Phonetics}{海报}{hai3 bao4}[][HSK 6]
    \definition[张,份,幅]{s.}{pôster; cartaz; cartazes anunciando apresentações culturais, exibições de filmes ou competições esportivas, etc.}
  \end{Phonetics}
\end{Entry}

\begin{Entry}{海运}{10,7}{⽔、⾡}
  \begin{Phonetics}{海运}{hai3yun4}[][HSK 7-9]
    \definition{s.}{transporte marítimo; transporte oceânico}
    \definition{v.}{transportar pelo mar}
  \end{Phonetics}
\end{Entry}

\begin{Entry}{海里}{10,7}{⽔、⾥}
  \begin{Phonetics}{海里}{hai3li3}
    \definition{s.}{milha náutica}
  \end{Phonetics}
\end{Entry}

\begin{Entry}{海岸}{10,8}{⽔、⼭}
  \begin{Phonetics}{海岸}{hai3'an4}[][HSK 7-9]
    \definition[段]{s.}{litoral; costa; praia}
  \end{Phonetics}
\end{Entry}

\begin{Entry}{海底}{10,8}{⽔、⼴}
  \begin{Phonetics}{海底}{hai3 di3}[][HSK 6]
    \definition{s.}{fundo do mar; fundo do oceano; solo oceânico}
  \end{Phonetics}
\end{Entry}

\begin{Entry}{海拔}{10,8}{⽔、⼿}
  \begin{Phonetics}{海拔}{hai3ba2}[][HSK 7-9]
    \definition{s.}{altitude; altura em relação ao nível médio do mar}
  \end{Phonetics}
\end{Entry}

\begin{Entry}{海峡}{10,9}{⽔、⼭}
  \begin{Phonetics}{海峡}{hai3xia2}[][HSK 7-9]
    \definition*{s.}{Estreito de Taiwan}
    \definition[个]{s.}{estreito; canal; um canal estreito que conecta dois oceanos entre duas massas de terra}
  \end{Phonetics}
\end{Entry}

\begin{Entry}{海洋}{10,9}{⽔、⽔}
  \begin{Phonetics}{海洋}{hai3yang2}[][HSK 6]
    \definition[片,个]{s.}{mar; oceano; um termo geral para os mares e oceanos que formam uma entidade contínua na superfície da Terra; também pode ser usado para descrever um grande número de coisas semelhantes}
  \end{Phonetics}
\end{Entry}

\begin{Entry}{海面}{10,9}{⽔、⾯}
  \begin{Phonetics}{海面}{hai3mian4}[][HSK 7-9]
    \definition{s.}{nível do mar | superfície do mar}
  \end{Phonetics}
\end{Entry}

\begin{Entry}{海鸥}{10,9}{⽔、⿃}
  \begin{Phonetics}{海鸥}{hai3'ou1}
    \definition{s.}{gaivota}
  \end{Phonetics}
\end{Entry}

\begin{Entry}{海浪}{10,10}{⽔、⽔}
  \begin{Phonetics}{海浪}{hai3 lang4}[][HSK 6]
    \definition{s.}{ondas do mar}
  \end{Phonetics}
\end{Entry}

\begin{Entry}{海啸}{10,11}{⽔、⼝}
  \begin{Phonetics}{海啸}{hai3xiao4}[][HSK 7-9]
    \definition{s.}{\emph{tsunami}; maremoto}
  \end{Phonetics}
\end{Entry}

\begin{Entry}{海域}{10,11}{⽔、⼟}
  \begin{Phonetics}{海域}{hai3yu4}[][HSK 7-9]
    \definition{s.}{área marítima; espaço marítimo; refere-se a uma determinada área do oceano (tanto acima quanto abaixo da água)}
  \end{Phonetics}
\end{Entry}

\begin{Entry}{海盗}{10,11}{⽔、⽫}
  \begin{Phonetics}{海盗}{hai3dao4}[][HSK 7-9]
    \definition{s.}{pirata; viajante do mar | bucaneiro; viking; pirata}
  \end{Phonetics}
\end{Entry}

\begin{Entry}{海绵}{10,11}{⽔、⽷}
  \begin{Phonetics}{海绵}{hai3mian2}[][HSK 7-9]
    \definition{s.}{espuma de borracha; espuma de plástico; esponja; um material poroso feito de borracha ou plástico que é elástico, como uma esponja | esponja marinha seca; poríferos; osso esponjoso; refere-se especificamente ao esqueleto queratinoso das esponjas}
  \end{Phonetics}
\end{Entry}

\begin{Entry}{海棠}{10,12}{⽔、⽊}
  \begin{Phonetics}{海棠}{hai3tang2}
    \definition{s.}{begônia}
  \end{Phonetics}
\end{Entry}

\begin{Entry}{海湾}{10,12}{⽔、⽔}
  \begin{Phonetics}{海湾}{hai3 wan1}[][HSK 6]
    \definition{s.}{baía; golfo | lago}
  \end{Phonetics}
\end{Entry}

\begin{Entry}{海量}{10,12}{⽔、⾥}
  \begin{Phonetics}{海量}{hai3liang4}[][HSK 7-9]
    \definition{s.}{magnanimidade | alta tolerância ao álcool | cargas de; uma grande quantidade de}
  \end{Phonetics}
\end{Entry}

\begin{Entry}{海滨}{10,13}{⽔、⽔}
  \begin{Phonetics}{海滨}{hai3bin1}[][HSK 7-9]
    \definition{s.}{praia; beira-mar; litoral; um lugar perto do mar}
  \end{Phonetics}
\end{Entry}

\begin{Entry}{海滩}{10,13}{⽔、⽔}
  \begin{Phonetics}{海滩}{hai3tan1}[][HSK 7-9]
    \definition[个,片]{s.}{praia; praia com declive suave em direção ao mar}
  \end{Phonetics}
\end{Entry}

\begin{Entry}{海鲜}{10,14}{⽔、⿂}
  \begin{Phonetics}{海鲜}{hai3xian1}[][HSK 4]
    \definition[种,份,桌,批,些]{s.}{frutos do mar; mariscos; peixes marinhos frescos, camarões, etc., para consumo |}
  \end{Phonetics}
\end{Entry}

\begin{Entry}{海藻}{10,19}{⽔、⾋}
  \begin{Phonetics}{海藻}{hai3zao3}[][HSK 7-9]
    \definition{s.}{alga marinha; planta marítima}
  \end{Phonetics}
\end{Entry}

\begin{Entry}{消}{10}{⽔}
  \begin{Phonetics}{消}{xiao1}
    \definition{v.}{desaparecer | dissipar; remover; eliminar; fazer desaparecer | passar o tempo de forma descontraída (recreação) | precisar; tomar (necessidade, geralmente precedido por 不, 几, 何)}
  \seealsoref{不}{bu4}
  \seealsoref{何}{he2}
  \seealsoref{几}{ji3}
  \end{Phonetics}
\end{Entry}

\begin{Entry}{消化}{10,4}{⽔、⼔}
  \begin{Phonetics}{消化}{xiao1hua4}[][HSK 4]
    \definition{v.}{digerir (alimentos) | digerir (conhecimento); pensar e absorver; uma metáfora para a compreensão total de novos conhecimentos ou informações e a capacidade de transformá-los em algo que possa ser usado}
  \end{Phonetics}
\end{Entry}

\begin{Entry}{消失}{10,5}{⽔、⼤}
  \begin{Phonetics}{消失}{xiao1shi1}[][HSK 3]
    \definition{v.}{desaparecer; desvanecer; dissolver; dissipar; evaporar; sumir}
  \end{Phonetics}
\end{Entry}

\begin{Entry}{消灭}{10,5}{⽔、⽕}
  \begin{Phonetics}{消灭}{xiao1mie4}[][HSK 6]
    \definition{v.}{perecer; morrer; falecer; desaparecer | abolir; erradicar; eliminar; aniquilar; exterminar; acabar com; fazer com que não exista}
  \end{Phonetics}
\end{Entry}

\begin{Entry}{消防}{10,6}{⽔、⾩}
  \begin{Phonetics}{消防}{xiao1fang2}[][HSK 5]
    \definition{s.}{combate a incêncios; controle de incêndios}
  \end{Phonetics}
\end{Entry}

\begin{Entry}{消防员}{10,6,7}{⽔、⾩、⼝}
  \begin{Phonetics}{消防员}{xiao1fang2yuan2}
    \definition{s.}{bombeiro}
  \end{Phonetics}
\end{Entry}

\begin{Entry}{消极}{10,7}{⽔、⽊}
  \begin{Phonetics}{消极}{xiao1ji2}[][HSK 5]
    \definition{adj.}{negativo; oposto; adverso | passivo; inativo; sem ambição; sem iniciativa; desanimado; apático}
  \end{Phonetics}
\end{Entry}

\begin{Entry}{消毒}{10,9}{⽔、⽏}
  \begin{Phonetics}{消毒}{xiao1du2}[][HSK 5]
    \definition{v.}{desinfetar; esterilizar; matar os microrganismos causadores de doenças por meios físicos ou químicos}
  \end{Phonetics}
\end{Entry}

\begin{Entry}{消费}{10,9}{⽔、⾙}
  \begin{Phonetics}{消费}{xiao1fei4}[][HSK 3]
    \definition{v.}{gastar; consumir; consumir materiais para satisfazer as necessidades de produção ou de vida (geralmente refere-se ao consumo doméstico) | consumir (recursos naturais)}
  \end{Phonetics}
\end{Entry}

\begin{Entry}{消费者}{10,9,8}{⽔、⾙、⽼}
  \begin{Phonetics}{消费者}{xiao1 fei4 zhe3}[][HSK 5]
    \definition{s.}{consumidor; cliente; consumo; indivíduos membros da sociedade que compram e utilizam bens e serviços para consumo pessoal}
  \end{Phonetics}
\end{Entry}

\begin{Entry}{消除}{10,9}{⽔、⾩}
  \begin{Phonetics}{消除}{xiao1chu2}[][HSK 5]
    \definition{v.}{dissipar; eliminar; limpar; tornar algo inexistente; remover (algo desfavorável)}
  \end{Phonetics}
\end{Entry}

\begin{Entry}{消息}{10,10}{⽔、⼼}
  \begin{Phonetics}{消息}{xiao1xi5}[][HSK 3]
    \definition[个,条,篇,些]{s.}{notícias; informação; reportagem sobre pessoas ou situações | notícias; novidades;}
  \end{Phonetics}
\end{Entry}

\begin{Entry}{消耗}{10,10}{⽔、⽾}
  \begin{Phonetics}{消耗}{xiao1hao4}[][HSK 6]
    \definition{v.}{gastar; esgotar; consumir; usar; (espírito, força, coisas, etc.) diminuir gradualmente devido ao uso ou perda}
  \end{Phonetics}
\end{Entry}

\begin{Entry}{涉}{10}{⽔}
  \begin{Phonetics}{涉}{she4}[][HSK 6]
    \definition*{s.}{Sobrenome She}
    \definition{v.}{vadear; atravessar ou passar um rio ou um obstáculo | passar por; experimentar | envolver; implicar}
  \end{Phonetics}
\end{Entry}

\begin{Entry}{涉及}{10,3}{⽔、⼃}
  \begin{Phonetics}{涉及}{she4ji2}[][HSK 6]
    \definition{v.}{envolver; relacionar-se com; referir-se a; tocar em}
  \end{Phonetics}
\end{Entry}

\begin{Entry}{涝}{10}{⽔}
  \begin{Phonetics}{涝}{lao4}
    \definition{adj.}{inundado; encharcado}
    \definition{s.}{alagamento (de terras ou plantações); água acumulada nos campos devido às fortes chuvas}
    \definition{v.}{inundar; encharcar}
  \end{Phonetics}
\end{Entry}

\begin{Entry}{涨}{10}{⽔}
  \begin{Phonetics}{涨}{zhang3}[][HSK 5,6]
    \definition{v.}{subir; inchar; aumentar; elevar; melhorar}
  \end{Phonetics}
  \begin{Phonetics}{涨}{zhang4}
    \definition{v.}{inchar; ter o volume aumentado | ser inundado por uma torrente de sangue; ter uma dor de cabeça; ficar com o rosto vermelho de raiva | ser mais, maior, etc. do que o esperado}
  \end{Phonetics}
\end{Entry}

\begin{Entry}{涨价}{10,6}{⽔、⼈}
  \begin{Phonetics}{涨价}{zhang3/jia4}[][HSK 5]
    \definition{s.}{aumento de preços}
    \definition{v.+compl.}{(preços) subir; aumentar o preço}
  \end{Phonetics}
\end{Entry}

\begin{Entry}{烈}{10}{⽕}
  \begin{Phonetics}{烈}{lie4}
    \definition*{s.}{Sobrenome Lie}
    \definition{adj.}{forte; violento; intenso; feroz | justo; severo | firme; convicto; rigoroso}
    \definition{s.}{pessoa que morreu por uma causa justa | conquistas; façanhas | mártir sacrificando-se por uma causa justa}
  \end{Phonetics}
\end{Entry}

\begin{Entry}{烈士}{10,3}{⽕、⼠}
  \begin{Phonetics}{烈士}{lie4shi4}
    \definition{s.}{mártir}
  \end{Phonetics}
\end{Entry}

\begin{Entry}{烘}{10}{⽕}
  \begin{Phonetics}{烘}{hong1}
    \definition{v.}{secar; assar; aquecer; usar fogo ou vapor para aquecer o corpo ou para cozinhar, aquecer ou secar algo | destacar}
  \end{Phonetics}
\end{Entry}

\begin{Entry}{烘干}{10,3}{⽕、⼲}
  \begin{Phonetics}{烘干}{hong1gan1}[][HSK 7-9]
    \definition{v.}{secar em fogo alto | secar ao lado ou sobre o fogo | assar; secar no forno}
  \end{Phonetics}
\end{Entry}

\begin{Entry}{烘托}{10,6}{⽕、⼿}
  \begin{Phonetics}{烘托}{hong1tuo1}[][HSK 7-9]
    \definition{v.}{adicionar sombreamento ao redor de um objeto para destacá-lo; um dos métodos de pintura chinesa, que utiliza tinta ou cores claras para pontilhar o contorno do objeto e torná-lo mais claro | destacar por contraste; colocar em nítido relevo; fazer com que se destaque}
  \end{Phonetics}
\end{Entry}

\begin{Entry}{烟}{10}{⽕}
  \begin{Phonetics}{烟}{yan1}[][HSK 3]
    \definition[股,支,根,盒,包]{s.}{fumaça; gás produzido pela combustão de materiais, misturado com pequenas partículas não completamente queimadas | névoa; neblina | tabaco; planta de tabaco | fumo; cigarro; termo geral para cigarros, charutos, etc. | ópio | fuligem; fumaça de carvão}
    \definition{v.}{ficar irritado com a fumaça (os olhos lacrimejam ou não conseguem abrir)}
  \end{Phonetics}
\end{Entry}

\begin{Entry}{烟火}{10,4}{⽕、⽕}
  \begin{Phonetics}{烟火}{yan1huo3}
    \definition{s.}{fogo de artifício}
  \end{Phonetics}
\end{Entry}

\begin{Entry}{烟叶}{10,5}{⽕、⼝}
  \begin{Phonetics}{烟叶}{yan1ye4}
    \definition{s.}{folha de tabaco}
  \end{Phonetics}
\end{Entry}

\begin{Entry}{烟头}{10,5}{⽕、⼤}
  \begin{Phonetics}{烟头}{yan1tou2}
    \definition[根]{s.}{bituca de cigarro}
  \end{Phonetics}
\end{Entry}

\begin{Entry}{烟囱}{10,7}{⽕、⼞}
  \begin{Phonetics}{烟囱}{yan1cong1}
    \definition{s.}{chaminé}
  \end{Phonetics}
\end{Entry}

\begin{Entry}{烟花}{10,7}{⽕、⾋}
  \begin{Phonetics}{烟花}{yan1 hua1}[][HSK 6]
    \definition[场,朵]{s.}{fogos de artifício; uma coisa que emite faíscas de várias cores quando exposta à observação | prostituta; antigamente, referia-se a algo relacionado à prostituição}
  \end{Phonetics}
\end{Entry}

\begin{Entry}{烟雨}{10,8}{⽕、⾬}
  \begin{Phonetics}{烟雨}{yan1yu3}
    \definition{s.}{chuvisco | garoa}
  \end{Phonetics}
\end{Entry}

\begin{Entry}{烟草}{10,9}{⽕、⾋}
  \begin{Phonetics}{烟草}{yan1cao3}
    \definition{s.}{tabaco}
  \end{Phonetics}
\end{Entry}

\begin{Entry}{烤}{10}{⽕}
  \begin{Phonetics}{烤}{kao3}
    \definition{v.}{assar | grelhar}
  \end{Phonetics}
\end{Entry}

\begin{Entry}{烤肉}{10,6}{⽕、⾁}
  \begin{Phonetics}{烤肉}{kao3 rou4}[][HSK 5]
    \definition[块,串,片,盘]{s.}{churrasco (literalmente carne assada)}
  \end{Phonetics}
\end{Entry}

\begin{Entry}{烤鸭}{10,10}{⽕、⿃}
  \begin{Phonetics}{烤鸭}{kao3ya1}[][HSK 5]
    \definition[只,盘]{s.}{pato assado; pato recheado e assado em um forno especial após ser abatido}
  \end{Phonetics}
\end{Entry}

\begin{Entry}{烦}{10}{⽕}
  \begin{Phonetics}{烦}{fan2}[][HSK 4]
    \definition{adj.}{redundante e confuso | supérfluo e confuso; muito bagunçado}
    \definition{v.}{aborrecer | irritar; incomodar; estar cansado de; ficar irritado | incomodar; solicitar}
  \end{Phonetics}
\end{Entry}

\begin{Entry}{烦闷}{10,7}{⽕、⾨}
  \begin{Phonetics}{烦闷}{fan2men4}[][HSK 7-9]
    \definition{adj.}{infeliz; deprimido; mal-humorado | desconfortável}
  \end{Phonetics}
\end{Entry}

\begin{Entry}{烦恼}{10,9}{⽕、⼼}
  \begin{Phonetics}{烦恼}{fan2nao3}[][HSK 7-9]
    \definition{adj.}{irritado; preocupado; incomodado}
    \definition[个,种,些]{s.}{aborrecimento; coisas que te incomodam}
  \end{Phonetics}
\end{Entry}

\begin{Entry}{烦躁}{10,20}{⽕、⾜}
  \begin{Phonetics}{烦躁}{fan2zao4}[][HSK 7-9]
    \definition{adj.}{inquieto; agitado; irritável}
  \end{Phonetics}
\end{Entry}

\begin{Entry}{烧}{10}{⽕}
  \begin{Phonetics}{烧}{shao1}[][HSK 4]
    \definition[次]{s.}{febre; temperatura corporal mais alta do que o normal}
    \definition{v.}{queimar; pegar fogo | cozinhar; aquecer; assar | guisar depois de fritar ou fritar depois de guisar | assar; grelhar os ingredientes dos alimentos diretamente sobre o fogo | ter febre; estar com febre | danificar (matar ou murchar) as plantas pelo uso excessivo (ou inadequado) de fertilizantes | tornar-se arrogante ou presunçoso; metáfora de estar em uma boa posição e se deixar levar}
  \end{Phonetics}
\end{Entry}

\begin{Entry}{烧烤}{10,10}{⽕、⽕}
  \begin{Phonetics}{烧烤}{shao1kao3}
    \definition{s.}{churrasco}
    \definition{v.}{assar}
  \end{Phonetics}
\end{Entry}

\begin{Entry}{热}{10}{⽕}
  \begin{Phonetics}{热}{re4}[][HSK 1]
    \definition{adj.}{quente; temperatura elevada | ardente; caloroso; profundamente afetuoso | ansioso; invejoso; descreve inveja e desejo de possuir algo | térmico; altamente radioativo | popular; muito procurado; muito apreciado por muitas pessoas}
    \definition{s.}{calor; energia liberada pelo movimento irregular das moléculas dentro de um objeto | febre; febre alta causada por doença | moda passageira; mania; febre}
    \definition{v.}{aquecer (geralmente se refere a alimentos)}
  \end{Phonetics}
\end{Entry}

\begin{Entry}{热门}{10,3}{⽕、⾨}
  \begin{Phonetics}{热门}{re4men2}[][HSK 5]
    \definition{adj.}{popular}
    \definition{s.}{algo que desperta o interesse popular; metáfora para algo que está na moda e recebe a atenção de todos (em contraste com 冷门)}
  \seealsoref{冷门}{leng3men2}
  \end{Phonetics}
\end{Entry}

\begin{Entry}{热心}{10,4}{⽕、⼼}
  \begin{Phonetics}{热心}{re4xin1}[][HSK 4]
    \definition{adj.}{ardente; sincero; entusiasmado; afetuoso; apaixonado; interessado}
    \definition{v.}{ser entusiasmado com alguma coisa}
  \end{Phonetics}
\end{Entry}

\begin{Entry}{热水}{10,4}{⽕、⽔}
  \begin{Phonetics}{热水}{re4 shui3}[][HSK 6]
    \definition{s.}{água quente; água em temperatura mais alta}
  \end{Phonetics}
\end{Entry}

\begin{Entry}{热水器}{10,4,16}{⽕、⽔、⼝}
  \begin{Phonetics}{热水器}{re4 shui3 qi4}[][HSK 6]
    \definition[台]{s.}{aquecedor de água; aparelhos que aquecem água usando eletricidade, gás natural, gás liquefeito de petróleo ou energia solar}
  \end{Phonetics}
\end{Entry}

\begin{Entry}{热血沸腾}{10,6,8,13}{⽕、⾎、⽔、⾁}
  \begin{Phonetics}{热血沸腾}{re4xue4-fei4teng2}
    \definition{expr.}{estar animado; ter o sangue correndo}
  \end{Phonetics}
\end{Entry}

\begin{Entry}{热泪盈眶}{10,8,9,11}{⽕、⽔、⽫、⽬}
  \begin{Phonetics}{热泪盈眶}{re4lei4ying2kuang4}
    \definition{expr.}{olhos cheios de lágrimas de emoção | extremamente emocionado}
  \end{Phonetics}
\end{Entry}

\begin{Entry}{热线}{10,8}{⽕、⽷}
  \begin{Phonetics}{热线}{re4 xian4}[][HSK 6]
    \definition[条]{s.}{raio infravermelho | linha direta; \emph{hot line}; uma linha telefônica ou telegráfica direta; uma linha para um ponto de acesso | rota quente (ou movimentada, popular) | raio de calor}
  \end{Phonetics}
\end{Entry}

\begin{Entry}{热闹}{10,8}{⽕、⾾}
  \begin{Phonetics}{热闹}{re4nao5}[][HSK 4]
    \definition{adj.}{animado; agitado; movimentado com barulho e excitação; descreve uma cena animada com uma atmosfera calorosa}
    \definition{s.}{uma vista emocionante; uma cena de agitação e excitação; atmosfera acolhedora}
    \definition{v.}{animar; divertir-se}
  \end{Phonetics}
\end{Entry}

\begin{Entry}{热点}{10,9}{⽕、⽕}
  \begin{Phonetics}{热点}{re4 dian3}[][HSK 6]
    \definition{s.}{ponto de acesso; \emph{hotspot}}
  \end{Phonetics}
\end{Entry}

\begin{Entry}{热烈}{10,10}{⽕、⽕}
  \begin{Phonetics}{热烈}{re4lie4}[][HSK 3]
    \definition{adj.}{caloroso; fervoroso; ardente; entusiasmado; excitado}
  \end{Phonetics}
\end{Entry}

\begin{Entry}{热爱}{10,10}{⽕、⽖}
  \begin{Phonetics}{热爱}{re4'ai4}[][HSK 3]
    \definition{v.}{amar ardentemente; amar de coração; ter amor profundo por; amar apaixonadamente}
  \end{Phonetics}
\end{Entry}

\begin{Entry}{热情}{10,11}{⽕、⼼}
  \begin{Phonetics}{热情}{re4qing2}[][HSK 2]
    \definition{adj.}{caloroso; fervoroso; entusiasmado; cordial; descreve sentimentos calorosos por alguém}
    \definition{s.}{entusiasmo; ardor; devoção; calor humano; zelo; sentimentos calorosos}
  \end{Phonetics}
\end{Entry}

\begin{Entry}{热量}{10,12}{⽕、⾥}
  \begin{Phonetics}{热量}{re4 liang4}[][HSK 5]
    \definition{s.}{calor; quantidade de calor; calorias; em física, refere-se à energia transferida entre objetos com temperaturas diferentes, do objeto com temperatura mais alta para o objeto com temperatura mais baixa}
  \end{Phonetics}
\end{Entry}

\begin{Entry}{爱}{10}{⽖}
  \begin{Phonetics}{爱}{ai4}[][HSK 1]
    \definition*{s.}{Sobrenome Ai}
    \definition[个]{s.}{amor; afeição profunda; preocupação profunda; especialmente amor entre pessoas}[爱是理解和包容。===O amor é compreensão e tolerância.]
    \definition{v.}{amar; ter sentimentos profundos por pessoas ou coisas | gostar; gostar de; estar interessado em |  cuidar; valorizar; ter em alta estima; cuidar bem de | estar apto a; ter o hábito de}[他们深深爱着对方。===Eles se amam profundamente. | 我爱我的家人。===Eu amo minha família. | 我爱旅行。===Eu adoro viajar.]
  \end{Phonetics}
\end{Entry}

\begin{Entry}{爱人}{10,2}{⽖、⼈}
  \begin{Phonetics}{爱人}{ai4 ren5}[][HSK 2]
    \definition[个]{s.}{amante; \emph{dollbaby}; namorado(a) | marido ou esposa; mais usado em ocasiões formais}[这是我的爱人。===Este é o meu/minha esposo/companheiro. | 她是我一生的爱人。===Ela é o amor da minha vida. | 请携带爱人出席晚宴。===Por favor, traga seu cônjuge para o jantar.]
  \end{Phonetics}
\end{Entry}

\begin{Entry}{爱上}{10,3}{⽖、⼀}
  \begin{Phonetics}{爱上}{ai4shang4}
    \definition{v.}{perder o coração por; apaixonar-se por}[他在旅行时爱上了一位法国女孩。===Ele se apaixonou por uma garota francesa durante a viagem.  | 来到杭州后,我爱上了龙井茶。===Depois de chegar em Hangzhou, me apaixonei pelo chá Longjing. | 我从来没想过自己会爱上健身。===Eu nunca imaginei que iria me apaixonar por academia.]
  \end{Phonetics}
\end{Entry}

\begin{Entry}{爱不释手}{10,4,12,4}{⽖、⼀、⾤、⼿}
  \begin{Phonetics}{爱不释手}{ai4bu2shi4shou3}[][HSK 7-9]
    \definition{expr.}{``Não consigo parar de ler.''; ``Amo tanto que não consigo deixar passar.''; gostar (amar) algo tanto que não se pode suportar separar-se dele}
  \end{Phonetics}
\end{Entry}

\begin{Entry}{爱心}{10,4}{⽖、⼼}
  \begin{Phonetics}{爱心}{ai4xin1}[][HSK 3]
    \definition[片]{s.}{amor; carinho; compaixão; um sentimento de preocupação e carinho por outras pessoas ou animais}
  \end{Phonetics}
\end{Entry}

\begin{Entry}{爱好}{10,6}{⽖、⼥}
  \begin{Phonetics}{爱好}{ai4 hao4}[][HSK 1]
    \definition[个,种]{s.}{passatempo; interesse; \emph{hobby}; sentimentos de interesse especial ou afeição por algo | 爱好 é mais usado para atividades regulares (esportes, música), enquanto 喜欢 é para preferências gerais}[他的爱好是收集邮票。===Seu hobby era colecionar selos.  | 我的爱好是读书和旅行。===Meus hobbies são ler e viajar.]
    \definition{v.}{estar interessado em; ter prazer em; ter um forte interesse em algo; ter sentimentos profundos por alguém ou algo}
  \seealsoref{喜欢}{xi3huan5}
  \end{Phonetics}
\end{Entry}

\begin{Entry}{爱好者}{10,6,8}{⽖、⼥、⽼}
  \begin{Phonetics}{爱好者}{ai4 hao4 zhe3}
    \definition{s.}{hobbista; amador; entusiasta; fã; amante (de arte, esportes, etc.)}[他是一位摄影爱好者。===Ele é um entusiasta de fotografia. | 她是位潜水爱好者,经常去东南亚潜水。===Ela é uma mergulhadora amadora e frequentemente mergulha no Sudeste Asiático.  | 我们为书法爱好者创建了一个微信群。===Criamos um grupo no WeChat para amantes de caligrafia.]
  \end{Phonetics}
\end{Entry}

\begin{Entry}{爱抚}{10,7}{⽖、⼿}
  \begin{Phonetics}{爱抚}{ai4fu3}
    \definition{s.}{carinho; carícia}
    \definition{v.}{acariciar; afagar; cuidar (com ternura)}[他轻轻爱抚她的头发。===Ele afagou suavemente o cabelo dela. | 母亲爱抚婴儿的脸颊。===A mãe acaricia a bochecha do bebê. | 她爱抚着小猫的耳朵。===Ela acariciou as orelhas do gatinho.]
  \end{Phonetics}
\end{Entry}

\begin{Entry}{爱护}{10,7}{⽖、⼿}
  \begin{Phonetics}{爱护}{ai4hu4}[][HSK 4]
    \definition{v.}{acalentar; valorizar; salvaguardar; cuidar bem de}[全社会都应爱护老年人。===Toda a sociedade deve tratar os idosos com cuidado e respeito. | 请爱护公园里的小动物。===Por favor, tratem os animais do parque com cuidado.]
  \end{Phonetics}
\end{Entry}

\begin{Entry}{爱国}{10,8}{⽖、⼞}
  \begin{Phonetics}{爱国}{ai4 guo2}[][HSK 4]
    \definition{adj.}{patriótico; patriotismo}[爱国是每个公民的责任。===O patriotismo é o dever de todo cidadão. | 这部电影讲述了英雄的爱国故事。===Este filme conta a história patriótica de um herói.]
    \definition{v.}{ser patriota; amar o seu país}
  \end{Phonetics}
\end{Entry}

\begin{Entry}{爱面子}{10,9,3}{⽖、⾯、⼦}
  \begin{Phonetics}{爱面子}{ai4/mian4zi5}[][HSK 7-9]
    \definition{v.+compl.}{estar preocupado em salvar a face; ser vigilante em relação à reputação; ser sensível ao próprio orgulho; valorizar minha própria dignidade e ter medo que os outros me desprezem}[他爱面子,怕别人笑话他。===Ele se importa com sua reputação e tem medo que os outros riam dele.]
  \end{Phonetics}
\end{Entry}

\begin{Entry}{爱爱}{10,10}{⽖、⽖}
  \begin{Phonetics}{爱爱}{ai4'ai5}
    \definition{v.}{coloquial: fazer amor ou relações íntimas | pode ser usado como um apelido entre casais, transmitindo ternura | pode soar vulgar se usado em contextos inadequados}[他们俩刚结婚,天天都想爱爱。===Eles acabaram de se casar e querem fazer amor todo dia. | 爱爱,你今天好漂亮!===Amor, você está linda hoje!]
  \end{Phonetics}
\end{Entry}

\begin{Entry}{爱情}{10,11}{⽖、⼼}
  \begin{Phonetics}{爱情}{ai4qing2}[][HSK 2]
    \definition{s.}{amor (entre pessoas); afeição}[爱情是盲目的。===O amor é cego. | 爱情如同玫瑰,美丽却带刺。===O amor é como uma rosa, bela mas com espinhos.  | 这首歌讲述了破碎的爱情故事。===Esta música conta uma história de amor fracassado.]
  \end{Phonetics}
\end{Entry}

\begin{Entry}{爱惜}{10,11}{⽖、⼼}
  \begin{Phonetics}{爱惜}{ai4xi1}[][HSK 7-9]
    \definition{v.}{valorizar; prezar; estimar; usar com moderação; não desperdiçar}
  \end{Phonetics}
\end{Entry}

\begin{Entry}{爱理不理}{10,11,4,11}{⽖、⽟、⼀、⽟}
  \begin{Phonetics}{爱理不理}{ai4li3-bu4li3}[][HSK 7-9]
    \definition{expr.}{frio e indiferente; distante}
  \end{Phonetics}
\end{Entry}

\begin{Entry}{爹}{10}{⽗}
  \begin{Phonetics}{爹}{die1}[][HSK 7-9]
    \definition[个]{s.}{Coloquial: pai; papai | velho pai; um título respeitoso para homens idosos em algumas áreas}
  \end{Phonetics}
\end{Entry}

\begin{Entry}{特}{10}{⽜}
  \begin{Phonetics}{特}{te4}[][HSK 6]
    \definition{adj.}{especial; incomum; particular; excepcional; diferente do geral | especial; solteiro; solitário}
    \definition{adv.}{muito; extremamente | especialmente; para um propósito especial |mas; somente}
    \definition{clas.}{TEX; abreviação para unidades de medida como TEX; a unidade de medida TEX indica a espessura de um fio têxtil através do seu peso}
    \definition{s.}{espião; agente secreto}
  \end{Phonetics}
\end{Entry}

\begin{Entry}{特大}{10,3}{⽜、⼤}
  \begin{Phonetics}{特大}{te4 da4}[][HSK 6]
    \definition{adj.}{especialmente (excepcionalmente) grande; o mais}
  \end{Phonetics}
\end{Entry}

\begin{Entry}{特价}{10,6}{⽜、⼈}
  \begin{Phonetics}{特价}{te4 jia4}[][HSK 4]
    \definition{s.}{oferta especial; preço de barganha; preço especial reduzido}
  \end{Phonetics}
\end{Entry}

\begin{Entry}{特地}{10,6}{⽜、⼟}
  \begin{Phonetics}{特地}{te4 di4}[][HSK 6]
    \definition{adv.}{especialmente; propositalmente; para um propósito especial}
  \end{Phonetics}
\end{Entry}

\begin{Entry}{特有}{10,6}{⽜、⽉}
  \begin{Phonetics}{特有}{te4 you3}[][HSK 5]
    \definition{adj.}{específico; peculiar; característico; único; exclusivo; especial}
  \end{Phonetics}
\end{Entry}

\begin{Entry}{特色}{10,6}{⽜、⾊}
  \begin{Phonetics}{特色}{te4se4}[][HSK 3]
    \definition{s.}{característica; característica distintiva; a cor única, estilo, etc. de um objeto}
  \end{Phonetics}
\end{Entry}

\begin{Entry}{特别}{10,7}{⽜、⼑}
  \begin{Phonetics}{特别}{te4bie2}[][HSK 2]
    \definition{adj.}{especial; particular; fora do comum; diferente dos outros, com características próprias}
    \definition{adv.}{especialmente; particularmente | ainda mais; em particular; frequentemente usado com 是 | especialmente; deliberadamente; para um propósito específico}
  \seealsoref{是}{shi4}
  \end{Phonetics}
\end{Entry}

\begin{Entry}{特别快车}{10,7,7,4}{⽜、⼑、⼼、⾞}
  \begin{Phonetics}{特别快车}{te4bie2 kuai4che1}
    \definition{s.}{trem expresso; expresso; expresso especial; refere-se a trens de passageiros que param em menos estações e têm menor tempo de viagem do que trens expressos diretos}
  \end{Phonetics}
\end{Entry}

\begin{Entry}{特快}{10,7}{⽜、⼼}
  \begin{Phonetics}{特快}{te4 kuai4}[][HSK 6]
    \definition{adj.}{expresso (trem, entrega etc.)}
    \definition{s.}{trem expresso (opp. 普快); abreviação de 特别快车}
  \seealsoref{特别快车}{te4bie2 kuai4che1}
  \end{Phonetics}
\end{Entry}

\begin{Entry}{特技}{10,7}{⽜、⼿}
  \begin{Phonetics}{特技}{te4ji4}
    \definition{s.}{efeito especial | dublê}
  \end{Phonetics}
\end{Entry}

\begin{Entry}{特定}{10,8}{⽜、⼧}
  \begin{Phonetics}{特定}{te4ding4}[][HSK 5]
    \definition{adj.}{específico; especialmente designado | dado; especificado; específico (pessoa, hora, lugar, local, ambiente, etc.)}
  \end{Phonetics}
\end{Entry}

\begin{Entry}{特征}{10,8}{⽜、⼻}
  \begin{Phonetics}{特征}{te4zheng1}[][HSK 4]
    \definition[个,种]{s.}{característica; aparência ou o fenômeno característico de uma pessoa ou coisa que pode ser visto de fora}
  \end{Phonetics}
\end{Entry}

\begin{Entry}{特性}{10,8}{⽜、⼼}
  \begin{Phonetics}{特性}{te4 xing4}[][HSK 5]
    \definition[种,个]{s.}{propriedade específica (ou característica) | característica; sabores | propriedade}
  \end{Phonetics}
\end{Entry}

\begin{Entry}{特点}{10,9}{⽜、⽕}
  \begin{Phonetics}{特点}{te4dian3}[][HSK 2]
    \definition[个,大]{s.}{característica; peculiaridade; traço distintivo; a singularidade de uma pessoa ou coisa}
  \end{Phonetics}
\end{Entry}

\begin{Entry}{特殊}{10,10}{⽜、⽍}
  \begin{Phonetics}{特殊}{te4shu1}[][HSK 4]
    \definition{adj.}{especial; particular; peculiar; excepcional; incomum}
  \end{Phonetics}
\end{Entry}

\begin{Entry}{特意}{10,13}{⽜、⼼}
  \begin{Phonetics}{特意}{te4yi4}[][HSK 6]
    \definition{adv.}{especialmente; para um propósito especial}
  \end{Phonetics}
\end{Entry}

\begin{Entry}{牺}{10}{⽜}
  \begin{Phonetics}{牺}{xi1}
    \definition{s.}{um animal de cor uniforme para sacrifício; sacrifício; gado com pelagem pura usado para sacrifício}
  \end{Phonetics}
\end{Entry}

\begin{Entry}{牺牲}{10,9}{⽜、⽜}
  \begin{Phonetics}{牺牲}{xi1sheng1}[][HSK 6]
    \definition[份]{s.}{sacrifício; um animal abatido para sacrifício; refere-se ao sacrifício da própria vida ou dos próprios interesses por um propósito justo, ou refere-se ao preço pago por um determinado propósito}
    \definition{v.}{sacrificar-se; morrer como mártir; dar a própria vida; sacrificar sua vida pela justiça | sacrificar; desistir; fazer algo às custas de; geralmente se refere a pagar um preço ou sofrer danos por alguém ou algo}
  \end{Phonetics}
\end{Entry}

\begin{Entry}{猃}{10}{⽝}
  \begin{Phonetics}{猃}{xian3}
    \definition{s.}{(arcaico) um tipo de cão com focinho longo}
  \end{Phonetics}
\end{Entry}

\begin{Entry}{猃狁}{10,7}{⽝、⽝}
  \begin{Phonetics}{猃狁}{xian3yun3}
    \definition*{s.}{Termo da dinastia Zhou para uma tribo nômade do norte mais tarde chamou o Xiongnu (匈奴) nas dinastias Qin e Han}
  \seealsoref{匈奴}{xiong1nu2}
  \end{Phonetics}
\end{Entry}

\begin{Entry}{珠}{10}{⽟}
  \begin{Phonetics}{珠}{zhu1}
    \definition[粒,颗]{s.}{pérola | conta (de colar, ábaco, etc.) | coisa parecida com uma bola (como um globo ocular)}
  \end{Phonetics}
\end{Entry}

\begin{Entry}{珠子}{10,3}{⽟、⼦}
  \begin{Phonetics}{珠子}{zhu1zi5}
    \definition[粒,颗]{s.}{pérola | contas}
  \end{Phonetics}
\end{Entry}

\begin{Entry}{珠宝}{10,8}{⽟、⼧}
  \begin{Phonetics}{珠宝}{zhu1 bao3}[][HSK 6]
    \definition[串]{s.}{joias; pérolas; um termo geral para pérolas, pedras preciosas e outros ornamentos}
  \end{Phonetics}
\end{Entry}

\begin{Entry}{班}{10}{⽟}
  \begin{Phonetics}{班}{ban1}[][HSK 1]
    \definition*{s.}{Sobrenome Ban}
    \definition{adj.}{regular; programado; executado regularmente; com horários fixos (meios de transporte)}
    \definition{clas.}{um grupo de; uma classe de; usado para pessoas | meios de transporte com horários fixos}
    \definition[个]{s.}{equipe; turma; organização estruturada | dever; turno; período de trabalho dentro de um dia | equipe; esquadrão; unidade básica das forças armadas | nome usado antigamente para designar uma companhia teatral}
    \definition{v.}{mover; implantar; implementar}
  \end{Phonetics}
\end{Entry}

\begin{Entry}{班长}{10,4}{⽟、⾧}
  \begin{Phonetics}{班长}{ban1 zhang3}[][HSK 2]
    \definition[个,位,名]{s.}{monitor de turma; líder de equipe; alunos responsáveis nas turmas da escola | líder de esquadrão; responsável por uma turma de soldados, geralmente com patente de sargento}
  \end{Phonetics}
\end{Entry}

\begin{Entry}{班级}{10,6}{⽟、⽷}
  \begin{Phonetics}{班级}{ban1 ji2}[][HSK 3]
    \definition[个]{s.}{classe; série (na escola); o nome geral para as séries e turmas da escola}
  \end{Phonetics}
\end{Entry}

\begin{Entry}{瓶}{10}{⽡}
  \begin{Phonetics}{瓶}{ping2}[][HSK 2]
    \definition*{s.}{Sobrenome Ping}
    \definition{clas.}{usado para coisas que são engarrafadas; quantidade contida em um frasco, vaso, garrafa}
    \definition[个]{s.}{jarra; vaso; frasco; garrafa;}
  \end{Phonetics}
\end{Entry}

\begin{Entry}{瓶子}{10,3}{⽡、⼦}
  \begin{Phonetics}{瓶子}{ping2zi5}[][HSK 2]
    \definition[个,只,种]{s.}{garrafa; recipientes com gargalo feitos de cerâmica, vidro, plástico, etc., geralmente em forma cilíndrica}
  \end{Phonetics}
\end{Entry}

\begin{Entry}{瓶盖}{10,11}{⽡、⽫}
  \begin{Phonetics}{瓶盖}{ping2gai4}
    \definition{s.}{tampa de garrafa}
  \end{Phonetics}
\end{Entry}

\begin{Entry}{瓶装}{10,12}{⽡、⾐}
  \begin{Phonetics}{瓶装}{ping2zhuang1}
    \definition{adj.}{engarrafado}
  \end{Phonetics}
\end{Entry}

\begin{Entry}{瓷}{10}{⽡}
  \begin{Phonetics}{瓷}{ci2}[][HSK 7-9]
    \definition{adj.}{Dialeto: (relação) próxima; íntima}
    \definition{s.}{artigos de porcelana}
  \end{Phonetics}
\end{Entry}

\begin{Entry}{瓷器}{10,16}{⽡、⼝}
  \begin{Phonetics}{瓷器}{ci2qi4}[][HSK 7-9]
    \definition[件,种]{s.}{porcelana; louça; utensílios feitos de argila de porcelana, feldspato, quartzo, etc.}
  \end{Phonetics}
\end{Entry}

\begin{Entry}{留}{10}{⽥}
  \begin{Phonetics}{留}{liu2}[][HSK 2]
    \definition*{s.}{Sobrenome Liu}
    \definition{v.}{ficar; permanecer; parar em um determinado local ou posição; não se afastar | estudar no exterior (geralmente seguido pelo nome de um país com uma sílaba) | pedir a alguém para ficar; manter alguém onde está | concentrar-se em; concentrar a atenção em algo | manter; guardar; reservar; não joger fora | acumular; deixar crescer | aceitar; receber | transmitir (legado); deixar para trás}
  \end{Phonetics}
\end{Entry}

\begin{Entry}{留下}{10,3}{⽥、⼀}
  \begin{Phonetics}{留下}{liu2 xia4}[][HSK 2]
    \definition{v.}{deixar; parar em algum lugar}
  \end{Phonetics}
\end{Entry}

\begin{Entry}{留言}{10,7}{⽥、⾔}
  \begin{Phonetics}{留言}{liu2 yan2}[][HSK 6]
    \definition[条]{s.}{mensagem; recado}
    \definition{v.}{deixar uma mensagem; deixar seus comentários}
  \end{Phonetics}
\end{Entry}

\begin{Entry}{留学}{10,8}{⽥、⼦}
  \begin{Phonetics}{留学}{liu2xue2}[][HSK 3]
    \definition{v.}{estudar no exterior; permanecer no estrangeiro para estudar ou pesquisar}
  \end{Phonetics}
\end{Entry}

\begin{Entry}{留学生}{10,8,5}{⽥、⼦、⽣}
  \begin{Phonetics}{留学生}{liu2 xue2 sheng1}[][HSK 2]
    \definition[个,位,名,批,帮]{s.}{estudante estrangeiro; estudante que retornou; estudante que estuda no exterior}
  \end{Phonetics}
\end{Entry}

\begin{Entry}{留神}{10,9}{⽥、⽰}
  \begin{Phonetics}{留神}{liu2/shen2}
    \definition{v.+compl.}{tomar cuidado | prestar atenção | manter os olhos abertos}
  \end{Phonetics}
\end{Entry}

\begin{Entry}{畜}{10}{⽥}
  \begin{Phonetics}{畜}{chu4}
    \definition*{s.}{Sobrenome Chu}
    \definition{s.}{animal doméstico; gado; bestas, principalmente referindo-se ao gado}
  \end{Phonetics}
  \begin{Phonetics}{畜}{xu4}
    \definition{v.}{criar (animais domésticos)}
  \end{Phonetics}
\end{Entry}

\begin{Entry}{疼}{10}{⽧}
  \begin{Phonetics}{疼}{teng2}[][HSK 2]
    \definition{adj.}{dolorido; doído; sensação de extremo desconforto causada por ferimentos, doenças, etc.}
    \definition{v.}{ferir; machucar | adorar; amar profundamente; gostar muito; cuidar}
  \end{Phonetics}
\end{Entry}

\begin{Entry}{疼痛}{10,12}{⽧、⽧}
  \begin{Phonetics}{疼痛}{teng2 tong4}[][HSK 6]
    \definition[阵,种]{s.}{dor; sofrimento; ferimento; descreve a sensação de dor causada por lesão ou doença}
  \end{Phonetics}
\end{Entry}

\begin{Entry}{疾}{10}{⽧}
  \begin{Phonetics}{疾}{ji2}
    \definition*{s.}{Sobrenome Ji}
    \definition{s.}{doença; enfermidade; moléstia; padecimento | sofrimento; dor; dificuldade; mazela}
  \end{Phonetics}
\end{Entry}

\begin{Entry}{疾病}{10,10}{⽧、⽧}
  \begin{Phonetics}{疾病}{ji2bing4}[][HSK 6]
    \definition[种]{s.}{doença; enfermidade; termo geral para doença}
  \end{Phonetics}
\end{Entry}

\begin{Entry}{病}{10}{⽧}
  \begin{Phonetics}{病}{bing4}[][HSK 1]
    \definition[种]{s.}{doença; enfermidade | doença; males | falha; defeito; desvantagem; erro}
    \definition{v.}{adoecer; ficar doente | ferir; causar danos a | angustiar; desaprovar}
  \end{Phonetics}
\end{Entry}

\begin{Entry}{病人}{10,2}{⽧、⼈}
  \begin{Phonetics}{病人}{bing4 ren2}[][HSK 1]
    \definition[个,位]{s.}{doente; paciente; pessoas doentes; pessoas em tratamento}
  \end{Phonetics}
\end{Entry}

\begin{Entry}{病床}{10,7}{⽧、⼴}
  \begin{Phonetics}{病床}{bing4chuang2}[][HSK 7-9]
    \definition[号,张]{s.}{cama de hospital | leito de doente}
  \end{Phonetics}
\end{Entry}

\begin{Entry}{病房}{10,8}{⽧、⼾}
  \begin{Phonetics}{病房}{bing4 fang2}[][HSK 6]
    \definition[个,间]{s.}{enfermaria de um hospital; quartos onde ficam os pacientes em hospitais e onde vivem em casas de repouso}
  \end{Phonetics}
\end{Entry}

\begin{Entry}{病毒}{10,9}{⽧、⽏}
  \begin{Phonetics}{病毒}{bing4du2}[][HSK 5]
    \definition[种,株,类]{s.}{vírus; patógenos que são menores que os germes e visíveis somente com um microscópio eletrônico | Computação: vírus de computador}
  \end{Phonetics}
\end{Entry}

\begin{Entry}{病症}{10,10}{⽧、⽧}
  \begin{Phonetics}{病症}{bing4zheng4}[][HSK 7-9]
    \definition[种]{s.}{doença; enfermidade}
  \end{Phonetics}
\end{Entry}

\begin{Entry}{病情}{10,11}{⽧、⼼}
  \begin{Phonetics}{病情}{bing4 qing2}[][HSK 6]
    \definition{s.}{estado de uma doença; condição do paciente; mudanças na doença}
  \end{Phonetics}
\end{Entry}

\begin{Entry}{症}{10}{⽧}
  \begin{Phonetics}{症}{zheng1}
    \definition{s.}{doença; enfermidade | (figurativo) ponto de atrito | tumor abdominal | obstrução intestinal}
  \end{Phonetics}
  \begin{Phonetics}{症}{zheng4}
    \definition{s.}{doença; enfermidade}
  \end{Phonetics}
\end{Entry}

\begin{Entry}{症状}{10,7}{⽧、⽝}
  \begin{Phonetics}{症状}{zheng4zhuang4}[][HSK 6]
    \definition[种,些]{s.}{sintoma; estado anormal de um organismo devido a uma doença, como tosse, febre, etc.}
  \end{Phonetics}
\end{Entry}

\begin{Entry}{益}{10}{⽫}
  \begin{Phonetics}{益}{yi4}
    \definition*{s.}{Sobrenome Yi}
    \definition{adj.}{benéfico}
    \definition{adv.}{Literário: mais; cada vez mais}[空气污染问题日益严重。===O problema da poluição do ar está se tornando cada vez mais sério.]
    \definition{s.}{benefício; lucro; vantagem}
    \definition{v.}{aumentar}
  \end{Phonetics}
\end{Entry}

\begin{Entry}{益虫}{10,6}{⽫、⾍}
  \begin{Phonetics}{益虫}{yi4chong2}
    \definition{s.}{inseto benéfico (oposto a 害虫)}
  \seealsoref{害虫}{hai4chong2}
  \end{Phonetics}
\end{Entry}

\begin{Entry}{盏}{10}{⽫}
  \begin{Phonetics}{盏}{zhan3}
    \definition{clas.}{usado para lâmpadas, iluminação}[一盏煤油灯。===Uma lamparina de querosene.]
    \definition{s.}{copo pequeno}
  \end{Phonetics}
\end{Entry}

\begin{Entry}{盐}{10}{⽫}
  \begin{Phonetics}{盐}{yan2}[][HSK 4]
    \definition[袋,勺,把,包,粒]{s.}{sal (de cozinha) | Química: sal (produto formado pela neutralização de um ácido por uma base)}
  \end{Phonetics}
\end{Entry}

\begin{Entry}{监}{10}{⽫}
  \begin{Phonetics}{监}{jian1}
    \definition{s.}{prisão; cadeia}
    \definition{v.}{supervisionar; inspecionar; observar}
  \end{Phonetics}
\end{Entry}

\begin{Entry}{监测}{10,9}{⽫、⽔}
  \begin{Phonetics}{监测}{jian1 ce4}[][HSK 6]
    \definition{v.}{monitorar; supervisionar e testar}
  \end{Phonetics}
\end{Entry}

\begin{Entry}{监狱}{10,9}{⽫、⽝}
  \begin{Phonetics}{监狱}{jian1yu4}
    \definition{s.}{prisão}
  \end{Phonetics}
\end{Entry}

\begin{Entry}{监督}{10,13}{⽫、⽬}
  \begin{Phonetics}{监督}{jian1du1}[][HSK 6]
    \definition[个,位,名]{s.}{monitoramento; supervisão; pessoas que supervisionam}
    \definition{v.}{controlar; supervisionar; superintender; monitorar e supervisionar de perto}
  \end{Phonetics}
\end{Entry}

\begin{Entry}{监察}{10,14}{⽫、⼧}
  \begin{Phonetics}{监察}{jian1cha2}[][HSK 7-9]
    \definition{v.}{supervisionar; controlar}
  \end{Phonetics}
\end{Entry}

\begin{Entry}{眞}{10}{⽬}
  \begin{Phonetics}{眞}{zhen1}
    \variantof{真}
  \end{Phonetics}
\end{Entry}

\begin{Entry}{真}{10}{⼗}
  \begin{Phonetics}{真}{zhen1}[][HSK 1]
    \definition*{s.}{Sobrenome Zhen}
    \definition{adj.}{verdadeiro; real; genuíno (oposto de 假, 伪) | claro; inequívoco | genuíno; conforme os fatos objetivos (em oposição a 假 e 伪) | sincero}
    \definition{adv.}{realmente; verdadeiramente; de fato}
    \definition{s.}{escrita regular | retrato; imagem; cópia exata de algo | instintos naturais (ou caráter, disposição); natureza; qualidade inerente; origem | estado original; refere-se à forma original das coisas}
  \seealsoref{假}{jia4}
  \seealsoref{伪}{wei3}
  \end{Phonetics}
\end{Entry}

\begin{Entry}{真切}{10,4}{⼗、⼑}
  \begin{Phonetics}{真切}{zhen1qie4}
    \definition{adj.}{claro | distinto | honesto | sincero | vívido}
  \end{Phonetics}
\end{Entry}

\begin{Entry}{真心}{10,4}{⼗、⼼}
  \begin{Phonetics}{真心}{zhen1xin1}
    \definition{adj.}{sincero}
    \definition[片]{s.}{sinceridade}
  \end{Phonetics}
\end{Entry}

\begin{Entry}{真牛}{10,4}{⼗、⽜}
  \begin{Phonetics}{真牛}{zhen1niu2}
    \definition{adj.}{(gíria) muito legal, incrível}
  \end{Phonetics}
\end{Entry}

\begin{Entry}{真正}{10,5}{⼗、⽌}
  \begin{Phonetics}{真正}{zhen1zheng4}[][HSK 2]
    \definition{adj.}{verdadeiro; real; genuíno}
    \definition{adv.}{realmente; de fato; expressa afirmação de uma ação ou situação, equivalente a 确实}
  \seealsoref{确实}{que4shi2}
  \end{Phonetics}
\end{Entry}

\begin{Entry}{真声}{10,7}{⼗、⼠}
  \begin{Phonetics}{真声}{zhen1sheng1}
    \definition{s.}{voz modal; voz natural; voz verdadeira (oposto a 假声)}
  \seealsoref{假声}{jia3sheng1}
  \end{Phonetics}
\end{Entry}

\begin{Entry}{真实}{10,8}{⼗、⼧}
  \begin{Phonetics}{真实}{zhen1shi2}[][HSK 3]
    \definition{adj.}{verdadeiro; real; autêntico; de acordo com fatos objetivos}
  \end{Phonetics}
\end{Entry}

\begin{Entry}{真的}{10,8}{⼗、⽩}
  \begin{Phonetics}{真的}{zhen1 de5}[][HSK 1]
    \definition{adv.}{realmente; salientar que a situação existe realmente | verdadeiramente; realmente; existente na realidade; consistente com os fatos objetivos}
  \end{Phonetics}
\end{Entry}

\begin{Entry}{真诚}{10,8}{⼗、⾔}
  \begin{Phonetics}{真诚}{zhen1 cheng2}[][HSK 5]
    \definition{adj.}{verdadeiro; honesto; sério; sincero; genuíno; descreve uma pessoa que fala e age com sinceridade, de coração, fazendo com que os outros acreditem nela}
  \end{Phonetics}
\end{Entry}

\begin{Entry}{真相}{10,9}{⼗、⽬}
  \begin{Phonetics}{真相}{zhen1xiang4}[][HSK 5]
    \definition[个]{s.}{face; verdade; verdade nua e crua; a situação real; o estado real das coisas; a verdadeira situação}
  \end{Phonetics}
\end{Entry}

\begin{Entry}{真珠}{10,10}{⼗、⽟}
  \begin{Phonetics}{真珠}{zhen1zhu1}
    \variantof{珍珠}
  \end{Phonetics}
\end{Entry}

\begin{Entry}{真真}{10,10}{⼗、⼗}
  \begin{Phonetics}{真真}{zhen1zhen1}
    \definition{adv.}{genuinamente | realmente | escrupulosamente}
  \end{Phonetics}
\end{Entry}

\begin{Entry}{真理}{10,11}{⼗、⽟}
  \begin{Phonetics}{真理}{zhen1li3}[][HSK 5]
    \definition[条,个]{s.}{verdade; o reflexo correto das coisas objetivas e suas leis no cérebro humano}
  \end{Phonetics}
\end{Entry}

\begin{Entry}{真释}{10,12}{⼗、⾤}
  \begin{Phonetics}{真释}{zhen1shi4}
    \definition{s.}{razão genuína | explicação verdadeira}
  \end{Phonetics}
\end{Entry}

\begin{Entry}{破}{10}{⽯}
  \begin{Phonetics}{破}{po4}[][HSK 3]
    \definition{adj.}{quebrado; danificado; rasgado; desgastado | insignificante; péssimo; medíocre}
    \definition{v.}{quebrar; danificar | dividir; cortar; separar | trocar (dinheiro) | livrar-se de; destruir; romper com | derrotar; capturar (uma cidade, etc.) | gastar dinheiro | revelar a verdade sobre; expor | mudar; romper; quebrar (regras, hábitos, ideias, etc.)}
  \end{Phonetics}
\end{Entry}

\begin{Entry}{破产}{10,6}{⽯、⼇}
  \begin{Phonetics}{破产}{po4/chan3}[][HSK 4]
    \definition{v.+compl.}{falir; ir à falência; tornar-se insolvente; entrar em liquidação; perder todo o patrimônio | falhar; fracassar; não dar em nada; figura de linguagem (geralmente com uma conotação depreciativa)}
  \end{Phonetics}
\end{Entry}

\begin{Entry}{破坏}{10,7}{⽯、⼟}
  \begin{Phonetics}{破坏}{po4huai4}[][HSK 3]
    \definition{v.}{demolir; naufragar; soçobrar; destruir; obliterar | quebrar; violar (um acordo, regulamento, etc.); não cumprir (disposições legais, regras, acordos, princípios, etc.) | prejudicar; perturbar; sabotar; causar grande dano; causar danos às coisas | reverter; mudar (um sistema social, costume, etc.) completamente ou violentamente | destruir; decompor; danificar o tecido ou a estrutura de um objeto}
  \end{Phonetics}
\end{Entry}

\begin{Entry}{破坏性}{10,7,8}{⽯、⼟、⼼}
  \begin{Phonetics}{破坏性}{po4huai4xing4}
    \definition{adj.}{destrutivo}
    \definition{s.}{poder destrutivo}
  \end{Phonetics}
\end{Entry}

\begin{Entry}{砸}{10}{⽯}
  \begin{Phonetics}{砸}{za2}
    \definition{v.}{esmagar | bater | falhar | estragar}
  \end{Phonetics}
\end{Entry}

\begin{Entry}{离}{10}{⼇}
  \begin{Phonetics}{离}{li2}[][HSK 2]
    \definition*{s.}{Um dos Oito Diagramas | Sobrenome Li}
    \definition{prep.}{(ser longe) de\dots até\dots}
    \definition{v.}{partir; separar-se; afastar-se; estar longe de | prescindir; dispensar; ser independente de | mudar de; desviar-se de | mudar de; desviar-se de; trair; ser incompatível}
  \end{Phonetics}
\end{Entry}

\begin{Entry}{离不开}{10,4,4}{⼇、⼀、⼶}
  \begin{Phonetics}{离不开}{li2 bu4 kai1}[][HSK 4]
    \definition{v.}{não pode prescindir; ser inseparável de; não ser capaz de se separar ou deixar uma pessoa, coisa ou circunstância}
  \end{Phonetics}
\end{Entry}

\begin{Entry}{离开}{10,4}{⼇、⼶}
  \begin{Phonetics}{离开}{li2kai1}[][HSK 2]
    \definition{v.}{deixar; partir; desviar-se; separar-se das pessoas, dos lugares e das coisas}
  \end{Phonetics}
\end{Entry}

\begin{Entry}{离婚}{10,11}{⼇、⼥}
  \begin{Phonetics}{离婚}{li2/hun1}[][HSK 3]
    \definition{v.+compl.}{divórciar; romper um casamento; obter o divórcio}
  \end{Phonetics}
\end{Entry}

\begin{Entry}{秘}{10}{⽲}
  \begin{Phonetics}{秘}{bi4}
    \definition*{s.}{Abreviação de Peru, 秘鲁 | Sobrenome Bi}
  \seealsoref{秘鲁}{bi4lu3}
  \end{Phonetics}
  \begin{Phonetics}{秘}{mi4}
    \definition{adj.}{secreto; misterioso | raro; raramente visto; estranho}
    \definition{adv.}{secretamente; privadamente}
    \definition{s.}{secretário}
    \definition{v.}{manter algo em segredo; esconder algo; guardar segredos | bloquear; obstruir; ter dificuldade para defecar}
  \end{Phonetics}
\end{Entry}

\begin{Entry}{秘书}{10,4}{⽲、⼄}
  \begin{Phonetics}{秘书}{mi4shu1}[][HSK 4]
    \definition[个,位,名]{s.}{o cargo de secretário; funções de secretariado | secretário; pessoas encarregadas da correspondência e que auxiliam o chefe do órgão ou departamento na condução diária de seu trabalho}
  \end{Phonetics}
\end{Entry}

\begin{Entry}{秘书长}{10,4,4}{⽲、⼄、⾧}
  \begin{Phonetics}{秘书长}{mi4 shu1 zhang3}[][HSK 6]
    \definition{s.}{secretário-geral}
  \end{Phonetics}
\end{Entry}

\begin{Entry}{秘密}{10,11}{⽲、⼧}
  \begin{Phonetics}{秘密}{mi4mi4}[][HSK 4]
    \definition{adj.}{secreto}
    \definition[个,条,些]{s.}{segredo; algo secreto; coisas que você não quer que as pessoas saibam}
  \end{Phonetics}
\end{Entry}

\begin{Entry}{秘鲁}{10,12}{⽲、⿂}
  \begin{Phonetics}{秘鲁}{bi4lu3}
    \definition*{s.}{Peru}
  \end{Phonetics}
\end{Entry}

\begin{Entry}{租}{10}{⽲}
  \begin{Phonetics}{租}{zu1}[][HSK 2]
    \definition{s.}{aluguel | imposto sobre a terra; tributação; (antigo) refere-se ao imposto predial}
    \definition{v.}{contratar; alugar; fretar | alugar; arrendar}
  \end{Phonetics}
\end{Entry}

\begin{Entry}{租用}{10,5}{⽲、⽤}
  \begin{Phonetics}{租用}{zu1yong4}
    \definition{v.}{contratar | alugar | alugar (algo de alguém)}
  \end{Phonetics}
\end{Entry}

\begin{Entry}{租让}{10,5}{⽲、⾔}
  \begin{Phonetics}{租让}{zu1rang4}
    \definition{v.}{alugar | alugar (a propriedade de alguém para outra pessoa)}
  \end{Phonetics}
\end{Entry}

\begin{Entry}{租约}{10,6}{⽲、⽷}
  \begin{Phonetics}{租约}{zu1yue1}
    \definition{s.}{aluguel}
  \end{Phonetics}
\end{Entry}

\begin{Entry}{租房}{10,8}{⽲、⼾}
  \begin{Phonetics}{租房}{zu1fang2}
    \definition{v.}{alugar um apartamento}
  \end{Phonetics}
\end{Entry}

\begin{Entry}{租金}{10,8}{⽲、⾦}
  \begin{Phonetics}{租金}{zu1 jin1}[][HSK 6]
    \definition[笔]{s.}{aluguel; aluguer; o custo do aluguel de terras, casas ou itens; a renda do aluguel de terras, casas ou itens}
  \seealsoref{租钱}{zu1qian5}
  \end{Phonetics}
\end{Entry}

\begin{Entry}{租赁}{10,10}{⽲、⾙}
  \begin{Phonetics}{租赁}{zu1lin4}
    \definition{v.}{contratar | alugar}
  \end{Phonetics}
\end{Entry}

\begin{Entry}{租钱}{10,10}{⽲、⾦}
  \begin{Phonetics}{租钱}{zu1qian5}
    \definition{s.}{aluguel}
  \seealsoref{租金}{zu1 jin1}
  \end{Phonetics}
\end{Entry}

\begin{Entry}{租船}{10,11}{⽲、⾈}
  \begin{Phonetics}{租船}{zu1chuan2}
    \definition{v.}{fretar um navio | alugar um navio}
  \end{Phonetics}
\end{Entry}

\begin{Entry}{秤}{10}{⽲}
  \begin{Phonetics}{秤}{cheng4}[][HSK 7-9]
    \definition[把,杆,台]{s.}{balança; balança romana; um instrumento para medir o peso de um objeto}
  \end{Phonetics}
\end{Entry}

\begin{Entry}{积}{10}{⽲}
  \begin{Phonetics}{积}{ji1}[][HSK 7-9]
    \definition{adj.}{de longa data; pendente há muito tempo | antiquíssimo; acumulado ao longo de um longo período de tempo}
    \definition{s.}{(medicina chinesa) indigestão (em bebês e crianças) | (matemática)  abreviação de produto, 乘积}
    \definition{v.}{acumular; juntar; amontoar; reunir; coletar}
  \seealsoref{乘积}{cheng2ji1}
  \end{Phonetics}
\end{Entry}

\begin{Entry}{积木}{10,4}{⽲、⽊}
  \begin{Phonetics}{积木}{ji1mu4}
    \definition{s.}{blocos de montar (brinquedo)}
  \end{Phonetics}
\end{Entry}

\begin{Entry}{积极}{10,7}{⽲、⽊}
  \begin{Phonetics}{积极}{ji1ji2}[][HSK 3]
    \definition{adj.}{ativo; descreve uma atitude proativa e esforçada | positivo; que tem um efeito positivo e ajuda no desenvolvimento das coisas}
  \end{Phonetics}
\end{Entry}

\begin{Entry}{积淀}{10,11}{⽲、⽔}
  \begin{Phonetics}{积淀}{ji1dian4}[][HSK 7-9]
    \definition{s.}{acumulação | depósitos acumulados ao longo de longos períodos | Figurativo: experiência valiosa, sabedoria acumulada}
    \definition{v.}{acumular}
  \end{Phonetics}
\end{Entry}

\begin{Entry}{积累}{10,11}{⽲、⽷}
  \begin{Phonetics}{积累}{ji1lei3}[][HSK 4]
    \definition{s.}{acúmulo; acumulação}
    \definition{v.}{acumular}
  \end{Phonetics}
\end{Entry}

\begin{Entry}{积蓄}{10,13}{⽲、⾋}
  \begin{Phonetics}{积蓄}{ji1xu4}[][HSK 7-9]
    \definition[笔]{s.}{poupança; economias}
    \definition{v.}{acumular; poupar; economizar}
  \end{Phonetics}
\end{Entry}

\begin{Entry}{称}{10}{⽲}
  \begin{Phonetics}{称}{chen4}
    \definition{adj.}{ajustado; encaixado; adequado}
    \definition{v.}{ajustar; adequar; combinar; estar em conformidade com; ser adequado para | ter; possuir}
  \end{Phonetics}
  \begin{Phonetics}{称}{cheng1}[][HSK 2,5]
    \definition*{s.}{Sobrenome Cheng}
    \definition{s.}{nome}
    \definition{v.}{chamar; ser chamado | dizer; declarar | elogiar; louvar; expressar afirmação ou elogio a pessoas ou coisas por meio de palavras | pesar; medir o peso | elevar; levantar; erguer | aplaudir; concordar; expressar suas opiniões ou sentimentos por meio de palavras ou ações | declarar-se como; declarar que é; reivindicar ser alguém em virtude do próprio poder}
  \end{Phonetics}
\end{Entry}

\begin{Entry}{称为}{10,4}{⽲、⼂}
  \begin{Phonetics}{称为}{cheng1 wei2}[][HSK 3]
    \definition{v.}{ser chamado de; ser conhecido como; denominar}
  \end{Phonetics}
\end{Entry}

\begin{Entry}{称号}{10,5}{⽲、⼝}
  \begin{Phonetics}{称号}{cheng1hao4}[][HSK 5]
    \definition{s.}{título; nome; designação; nome dado a alguém, a uma organização ou a alguma coisa (geralmente usado de forma honrosa)}
  \end{Phonetics}
\end{Entry}

\begin{Entry}{称作}{10,7}{⽲、⼈}
  \begin{Phonetics}{称作}{cheng1zuo4}[][HSK 7-9]
    \definition{v.}{ser chamado | ser conhecido como}
  \end{Phonetics}
\end{Entry}

\begin{Entry}{称呼}{10,8}{⽲、⼝}
  \begin{Phonetics}{称呼}{cheng1hu5}[][HSK 7-9]
    \definition{v.}{chamar; dirigir-se (a alguém)}[我称呼他为老师。===Eu o chamo de professor.]
  \end{Phonetics}
\end{Entry}

\begin{Entry}{称赞}{10,16}{⽲、⾙}
  \begin{Phonetics}{称赞}{cheng1zan4}[][HSK 4]
    \definition[句,声,番,次]{s.}{elogio; aclamação; louvor; avaliação positiva de um desempenho ou conquista}
    \definition{v.}{elogiar; aclamar; louvar; usar palavras para expressar um carinho pelas virtudes de uma pessoa ou coisa}
  \end{Phonetics}
\end{Entry}

\begin{Entry}{窄}{10}{⽳}
  \begin{Phonetics}{窄}{zhai3}
    \definition{adj.}{estreito; pequena distância horizontal | mesquinho; estreito; (mente) não alegre; (capacidade) pequena | difícil; mal; falta de; (vida) não bem de vida}
  \end{Phonetics}
\end{Entry}

\begin{Entry}{站}{10}{⽴}
  \begin{Phonetics}{站}{zhan4}[][HSK 1,2]
    \definition*{s.}{Sobrenome Zhan}
    \definition{s.}{parada; estação; ponto de parada | central; estação; instituição criada para um determinado tipo de atividade | filial de uma empresa ou organização; local de trabalho criado para realizar uma determinada tarefa | \emph{website}; na rede de computadores, refere-se a um \emph{site}}
    \definition{v.}{ficar em pé; estar em pé | parar; interromper; fazer uma pausa}
  \end{Phonetics}
\end{Entry}

\begin{Entry}{站长}{10,4}{⽴、⾧}
  \begin{Phonetics}{站长}{zhan4zhang3}
    \definition{s.}{pessoa responsável pela estação de trem | chefe da estação | \emph{webmaster} | gerente de centro de voluntariado}
  \end{Phonetics}
\end{Entry}

\begin{Entry}{站台}{10,5}{⽴、⼝}
  \begin{Phonetics}{站台}{zhan4 tai2}[][HSK 6]
    \definition{s.}{plataforma (em uma estação ferroviária)}
  \end{Phonetics}
\end{Entry}

\begin{Entry}{站住}{10,7}{⽴、⼈}
  \begin{Phonetics}{站住}{zhan4 zhu4}[][HSK 2]
    \definition{v.}{parar; deter; parar enquanto se move | ficar firme nos pés; manter os pés; permanecer firme | manter-se firme; consolidar a posição de alguém; estabelecer-se em uma determinada unidade ou lugar | sustentar a opinião}
  \end{Phonetics}
\end{Entry}

\begin{Entry}{站姿}{10,9}{⽴、⼥}
  \begin{Phonetics}{站姿}{zhan4zi1}
    \definition{s.}{postura}
  \end{Phonetics}
\end{Entry}

\begin{Entry}{站点}{10,9}{⽴、⽕}
  \begin{Phonetics}{站点}{zhan4dian3}
    \definition{s.}{\emph{website}}
  \end{Phonetics}
\end{Entry}

\begin{Entry}{竞}{10}{⽴}
  \begin{Phonetics}{竞}{jing4}
    \definition{adj.}{forte; poderoso}
    \definition{v.}{competir; contender; disputar | contestar}
  \end{Phonetics}
\end{Entry}

\begin{Entry}{竞争}{10,6}{⽴、⼑}
  \begin{Phonetics}{竞争}{jing4zheng1}[][HSK 5]
    \definition{v.}{competir; disputar; lutar; entre duas ou mais partes; em prol de seus próprios interesses; lutar pela vitória por meio de uma disputa de sua própria força contra outra}
  \end{Phonetics}
\end{Entry}

\begin{Entry}{竞赛}{10,14}{⽴、⾙}
  \begin{Phonetics}{竞赛}{jing4sai4}[][HSK 5]
    \definition[个]{s.}{concurso; competição; partida; corrida}
    \definition{v.}{correr; competir; competir uns com os outros por superioridade; em esportes, produção e outras atividades, para comparar competência, habilidade etc., usado principalmente na linguagem falada}
  \end{Phonetics}
\end{Entry}

\begin{Entry}{笋}{10}{⽵}
  \begin{Phonetics}{笋}{sun3}
    \definition{s.}{broto de bambu}
  \end{Phonetics}
\end{Entry}

\begin{Entry}{笑}{10}{⽵}
  \begin{Phonetics}{笑}{xiao4}[][HSK 1]
    \definition{adj.}{ridículo; engraçado; risível; hilário}
    \definition{v.}{sorrir; rir; mostrar expressão de alegria; emitir sons de alegria | ridicularizar; rir de; zombar}
  \end{Phonetics}
\end{Entry}

\begin{Entry}{笑声}{10,7}{⽵、⼠}
  \begin{Phonetics}{笑声}{xiao4 sheng1}[][HSK 6]
    \definition{s.}{riso; risada}
  \end{Phonetics}
\end{Entry}

\begin{Entry}{笑话}{10,8}{⽵、⾔}
  \begin{Phonetics}{笑话}{xiao4hua5}[][HSK 2]
    \definition[个]{s.}{piada; brincadeira; uma conversa ou história que faz as pessoas rirem; algo que as pessoas usam como piada}
    \definition{v.}{ridicularizar; zombar; rir de;}
  \end{Phonetics}
\end{Entry}

\begin{Entry}{笑话儿}{10,8,2}{⽵、⾔、⼉}
  \begin{Phonetics}{笑话儿}{xiao4 hua4r5}[][HSK 2]
    \definition{s.}{piada; brincadeira; gracejo}
  \end{Phonetics}
\end{Entry}

\begin{Entry}{笑容}{10,10}{⽵、⼧}
  \begin{Phonetics}{笑容}{xiao4 rong2}[][HSK 6]
    \definition[丝,抹,个]{s.}{sorriso; expressão sorridente; o olhar no rosto de alguém ao sorrir}
  \end{Phonetics}
\end{Entry}

\begin{Entry}{笑脸}{10,11}{⽵、⾁}
  \begin{Phonetics}{笑脸}{xiao4 lian3}[][HSK 6]
    \definition{s.}{\emph{smiley}; rosto sorridente (emoji)}
  \end{Phonetics}
\end{Entry}

\begin{Entry}{笔}{10}{⽵}
  \begin{Phonetics}{笔}{bi3}[][HSK 2]
    \definition{clas.}{usado para grandes quantias de dinheiro, compras, negócios, propriedades, etc. | usado em caligrafia e pintura, etc.}
    \definition[支,枝]{s.}{caneta; lápis; pincel para escrever; ferramentas para escrever ou desenhar | técnica de escrita; caligrafia ou desenho | traço}
    \definition{v.}{escrever à mão}
  \end{Phonetics}
\end{Entry}

\begin{Entry}{笔记}{10,5}{⽵、⾔}
  \begin{Phonetics}{笔记}{bi3 ji4}[][HSK 2]
    \definition[篇,本,个]{s.}{notas; anotações feitas durante aulas, palestras e leituras | ensaios; esboços}
    \definition{v.}{tomar nota (por escrito)}
  \end{Phonetics}
\end{Entry}

\begin{Entry}{笔记本}{10,5,5}{⽵、⾔、⽊}
  \begin{Phonetics}{笔记本}{bi3ji4ben3}[][HSK 2]
    \definition[个,本]{s.}{caderno para anotações | \emph{laptop}; refere-se a um computador portátil}
    \definition{s.}{\emph{laptop}}
  \end{Phonetics}
\end{Entry}

\begin{Entry}{笔试}{10,8}{⽵、⾔}
  \begin{Phonetics}{笔试}{bi3 shi4}[][HSK 6]
    \definition{s.}{exame escrito; um tipo de exame que exige respostas escritas; diferente de 口试}
  \seealsoref{口试}{kou3 shi4}
  \end{Phonetics}
\end{Entry}

\begin{Entry}{粉}{10}{⽶}
  \begin{Phonetics}{粉}{fen3}[][HSK 7-9]
    \definition{adj.}{branco | rosa}
    \definition{s.}{pó | cosméticos em pó | farinha de trigo | macarrão ou outro alimento feito de feijão, arroz, batata, amido de batata-doce, etc. | macarrão de arroz}
    \definition{v.}{virar pó | Dialeto: caiar}
  \end{Phonetics}
\end{Entry}

\begin{Entry}{粉丝}{10,5}{⽶、⼀}
  \begin{Phonetics}{粉丝}{fen3si1}[][HSK 7-9]
    \definition{s.}{(empréstimo linguístico) fã | entusiasta de alguém ou alguma coisa}
    \definition[个,群,位,名,些,批]{s.}{aletria de amido de feijão ou batata; aletria chinesa; macarrão de celofane ou macarrão de vidro (transparente) | Empréstimo linguístico: fã; refere-se a uma pessoa que é obcecada ou adora uma celebridade}
  \end{Phonetics}
\end{Entry}

\begin{Entry}{粉色}{10,6}{⽶、⾊}
  \begin{Phonetics}{粉色}{fen3 se4}
    \definition{s.}{cor-de-rosa}
  \end{Phonetics}
\end{Entry}

\begin{Entry}{粉碎}{10,13}{⽶、⽯}
  \begin{Phonetics}{粉碎}{fen3sui4}[][HSK 7-9]
    \definition{adj.}{pulverizado; quebrado em pedaços; descreve algo que está muito quebrado, quebrado em partículas muito pequenas}
    \definition{v.}{esmagar; transformar as coisas em partículas muito pequenas | esmagar; quebrar; estilhaçar; fazer com que a outra parte falhe ou seja completamente destruída}
  \end{Phonetics}
\end{Entry}

\begin{Entry}{素}{10}{⽷}
  \begin{Phonetics}{素}{su4}
    \definition{adj.}{branco; de cor natural | simples; natural; singelo; de cor simples | nativo; original | normal; usual; geral}
    \definition{adv.}{geralmente; sempre; habitualmente}
    \definition{s.}{vegetais, frutas e outros alimentos (em oposição à 荤) | matéria-prima; matéria-prima básico; tecidos de seda naturais e não processados | elemento; os componentes básicos de algo}
  \seealsoref{荤}{hun1}
  \end{Phonetics}
\end{Entry}

\begin{Entry}{素质}{10,8}{⽷、⾙}
  \begin{Phonetics}{素质}{su4zhi4}[][HSK 6]
    \definition[个,种]{s.}{qualidade; características; caráter; o nível físico, moral, mental, intelectual e cultural de uma pessoa}
  \end{Phonetics}
\end{Entry}

\begin{Entry}{索}{10}{⽷}
  \begin{Phonetics}{索}{suo3}
    \definition*{s.}{Sobrenome Suo}
    \definition{adj.}{completamente sozinho; sozinho | maçante; insípido; sem significado}
    \definition[根]{s.}{corda; cabo; cordão; corrente | uma corda grande}
    \definition{v.}{(literário) pesquisar | exigir; pedir}
  \end{Phonetics}
\end{Entry}

\begin{Entry}{索性}{10,8}{⽷、⼼}
  \begin{Phonetics}{索性}{suo3xing4}
    \definition{adv.}{poderia muito bem | simplesmente | apenas}
  \end{Phonetics}
\end{Entry}

\begin{Entry}{紧}{10}{⽷}
  \begin{Phonetics}{紧}{jin3}[][HSK 3]
    \definition{adj.}{tenso; apertado; o estado em que um objeto se encontra após ser submetido a uma grande força de tração ou pressão.| seguro; firme | cerrado; apertado | urgente; premente; tenso | rigoroso; rígido; severo | difícil; sem dinheiro}
    \definition{v.}{apertar; tornar mais apertado}
  \end{Phonetics}
\end{Entry}

\begin{Entry}{紧张}{10,7}{⽷、⼸}
  \begin{Phonetics}{紧张}{jin3zhang1}[][HSK 3]
    \definition{adj.}{nervoso; tenso; mentalmente em estado de alerta, excitado e inquieto | apertado; em falta; o que está disponível não satisfaz os requisitos| tenso; intenso; intenso ou urgente, causando tensão mental}
  \end{Phonetics}
\end{Entry}

\begin{Entry}{紧急}{10,9}{⽷、⼼}
  \begin{Phonetics}{紧急}{jin3ji2}[][HSK 3]
    \definition{adj./adj.}{urgente; premente; crítico}
  \end{Phonetics}
\end{Entry}

\begin{Entry}{紧紧}{10,10}{⽷、⽷}
  \begin{Phonetics}{紧紧}{jin3 jin3}[][HSK 5]
    \definition{adv.}{firmemente; estreitamente; apertadamente; prestar muita atenção (em algo)}
  \end{Phonetics}
\end{Entry}

\begin{Entry}{紧密}{10,11}{⽷、⼧}
  \begin{Phonetics}{紧密}{jin3 mi4}[][HSK 4]
    \definition{adj.}{próximos; inseparáveis | incessante; rápido e intenso}
  \end{Phonetics}
\end{Entry}

\begin{Entry}{绣}{10}{⽷}
  \begin{Phonetics}{绣}{xiu4}
    \definition{s.}{bordado}
    \definition{v.}{bordar}
  \end{Phonetics}
\end{Entry}

\begin{Entry}{继}{10}{⽷}
  \begin{Phonetics}{继}{ji4}[][HSK 7-9]
    \definition{adv.}{então; depois}
    \definition{s.}{filhos; prole}
    \definition{v.}{continuar; ter sucesso; seguir}
  \end{Phonetics}
\end{Entry}

\begin{Entry}{继父}{10,4}{⽷、⽗}
  \begin{Phonetics}{继父}{ji4fu4}[][HSK 7-9]
    \definition{s.}{padrasto}
  \end{Phonetics}
\end{Entry}

\begin{Entry}{继母}{10,5}{⽷、⽏}
  \begin{Phonetics}{继母}{ji4mu3}[][HSK 7-9]
    \definition{s.}{madrasta}
  \end{Phonetics}
\end{Entry}

\begin{Entry}{继而}{10,6}{⽷、⽽}
  \begin{Phonetics}{继而}{ji4'er2}[][HSK 7-9]
    \definition{adv.}{então; depois; mais tarde}
  \end{Phonetics}
\end{Entry}

\begin{Entry}{继承}{10,8}{⽷、⼿}
  \begin{Phonetics}{继承}{ji4cheng2}[][HSK 5]
    \definition{v.}{herdar (o patrimônio de uma pessoa falecida, etc.) de acordo com a lei | continuar; geralmente se refere à aceitação do estilo, da cultura, do conhecimento, etc., daqueles que nos precederam | continuar; os descendentes continuam o trabalho deixado por seus antecessores.}
  \end{Phonetics}
\end{Entry}

\begin{Entry}{继续}{10,11}{⽷、⽷}
  \begin{Phonetics}{继续}{ji4xu4}[][HSK 3]
    \definition{s.}{continuação}
    \definition{v.}{continuar; prosseguir | prosseguir; continuar; seguir em frente (com); (atividades, eventos, etc.) continuar após uma pausa ou um determinado período de tempo}
  \end{Phonetics}
\end{Entry}

\begin{Entry}{缺}{10}{⽸}
  \begin{Phonetics}{缺}{que1}[][HSK 3]
    \definition{adj.}{incompleto; imperfeito}
    \definition[种]{s.}{vaga; abertura; falta}
    \definition{v.}{estar com falta de; faltar | estar ausente}
  \end{Phonetics}
\end{Entry}

\begin{Entry}{缺乏}{10,4}{⽸、⼃}
  \begin{Phonetics}{缺乏}{que1fa2}[][HSK 5]
    \definition{v.}{faltar; estar em falta de; não ter ou não ter totalmente (algo que deveria possuir ou é desejaria possuir)}
  \end{Phonetics}
\end{Entry}

\begin{Entry}{缺少}{10,4}{⽸、⼩}
  \begin{Phonetics}{缺少}{que1shao3}[][HSK 3]
    \definition{v.}{falta; estar com falta de; estar em falta de; geralmente se refere à falta de pessoas ou coisas}
  \end{Phonetics}
\end{Entry}

\begin{Entry}{缺点}{10,9}{⽸、⽕}
  \begin{Phonetics}{缺点}{que1dian3}[][HSK 3]
    \definition[个,些]{s.}{desvantagem; deficiência; inconveniência; ponto fraco; uma deficiência ou imperfeição (em oposição a 优点)}
  \seealsoref{优点}{you1dian3}
  \end{Phonetics}
\end{Entry}

\begin{Entry}{缺陷}{10,10}{⽸、⾩}
  \begin{Phonetics}{缺陷}{que1xian4}[][HSK 6]
    \definition[个,处,项]{pron.}{defeito; falha; inconveniência; mancha; um lugar onde uma pessoa ou coisa está incompleta ou tem falhas porque algo está faltando}
  \end{Phonetics}
\end{Entry}

\begin{Entry}{缺勤}{10,13}{⽸、⼒}
  \begin{Phonetics}{缺勤}{que1/qin2}
    \definition{v.+compl.}{ausentar-se do dever (trabalho)}
  \end{Phonetics}
\end{Entry}

\begin{Entry}{罢}{10}{⽹}
  \begin{Phonetics}{罢}{ba4}
    \definition{v.}{parar; cessar | revogar; destituir; encerrar | terminar | abandonar uma ideia; esqueçer sobre algo; deixar estar (passar)}
  \end{Phonetics}
  \begin{Phonetics}{罢}{ba5}
    \definition{part.}{partícula final, a mesma que 吧}
  \seealsoref{吧}{ba5}
  \end{Phonetics}
\end{Entry}

\begin{Entry}{罢了}{10,2}{⽹、⼅}
  \begin{Phonetics}{罢了}{ba4 le5}[][HSK 6]
    \definition{part.}{usado no final de uma frase, significa 仅此而已, geralmente seguido de 无非, 不过, 只是}
  \seealsoref{不过}{bu2guo4}
  \seealsoref{仅此而已}{jin3ci3'er2yi3}
  \seealsoref{无非}{wu2fei1}
  \seealsoref{只是}{zhi3 shi4}
  \end{Phonetics}
  \begin{Phonetics}{罢了}{ba4 liao3}
    \definition{part.}{uma partícula modal indicando (não se preocupe, ok)}
  \end{Phonetics}
\end{Entry}

\begin{Entry}{罢工}{10,3}{⽹、⼯}
  \begin{Phonetics}{罢工}{ba4gong1}[][HSK 6]
    \definition{v.}{parar de trabalhar; entrar em greve; abandonar o emprego}
  \end{Phonetics}
\end{Entry}

\begin{Entry}{罢休}{10,6}{⽹、⼈}
  \begin{Phonetics}{罢休}{ba4xiu1}[][HSK 7-9]
    \definition{v.}{parar; desistir; deixar o assunto de lado; parar de fazer algo, enfatizando a determinação de parar de fazê-lo}
  \end{Phonetics}
\end{Entry}

\begin{Entry}{罢免}{10,7}{⽹、⼉}
  \begin{Phonetics}{罢免}{ba4mian3}[][HSK 7-9]
    \definition{v.}{destituir; remover do cargo; demitir alguém do seu posto | destituir do cargo uma pessoa eleita pelo eleitorado ou por um órgão representativo}
  \end{Phonetics}
\end{Entry}

\begin{Entry}{翅}{10}{⽻}
  \begin{Phonetics}{翅}{chi4}
    \definition[只]{s.}{asa | barbatana de tubarão | coisa parecida com uma asa}
  \end{Phonetics}
\end{Entry}

\begin{Entry}{翅膀}{10,14}{⽻、⾁}
  \begin{Phonetics}{翅膀}{chi4bang3}[][HSK 7-9]
    \definition[只,个,对]{s.}{asa; os órgãos de voo de animais como pássaros e insetos geralmente aparecem em pares | barbatana; aba; lâmina; a parte de algo que tem o formato ou age como uma asa}
  \end{Phonetics}
\end{Entry}

\begin{Entry}{耕}{10}{⽾}
  \begin{Phonetics}{耕}{geng1}
    \definition{v.}{arar; cultivar | trabalhar; fazer | ganhar a vida}
  \end{Phonetics}
\end{Entry}

\begin{Entry}{耕地}{10,6}{⽾、⼟}
  \begin{Phonetics}{耕地}{geng1/di4}[][HSK 7-9]
    \definition[块,公顷]{s.}{terra cultivada; terra para cultivo}
    \definition{v.+compl.}{lavrar; arar}
  \end{Phonetics}
\end{Entry}

\begin{Entry}{耗}{10}{⽾}
  \begin{Phonetics}{耗}{hao4}[][HSK 7-9]
    \definition{s.}{más notícias}[听到噩耗,他碎心裂胆。===Ele ficou arrasado ao ouvir as más notícias.]
    \definition{v.}{consumir; custar | perder tempo; procrastinar}
  \end{Phonetics}
\end{Entry}

\begin{Entry}{耗时}{10,7}{⽾、⽇}
  \begin{Phonetics}{耗时}{hao4shi2}[][HSK 7-9]
    \definition{adj.}{demorado; levar um período de ($x$ quantidade de tempo)}
  \end{Phonetics}
\end{Entry}

\begin{Entry}{耗费}{10,9}{⽾、⾙}
  \begin{Phonetics}{耗费}{hao4fei4}[][HSK 7-9]
    \definition{v.}{gastar; consumir; esgotar}
  \end{Phonetics}
\end{Entry}

\begin{Entry}{耻}{10}{⽿}
  \begin{Phonetics}{耻}{chi3}
    \definition{s.}{vergonha; desgraça; humilhação}
    \definition{v.}{estar envergonhado de; considerar vergonhoso}
  \end{Phonetics}
\end{Entry}

\begin{Entry}{耻笑}{10,10}{⽿、⽵}
  \begin{Phonetics}{耻笑}{chi3xiao4}[][HSK 7-9]
    \definition{v.}{ridicularizar alguém; zombar; zombar de; rir de}
  \end{Phonetics}
\end{Entry}

\begin{Entry}{耻辱}{10,10}{⽿、⾠}
  \begin{Phonetics}{耻辱}{chi3ru3}[][HSK 7-9]
    \definition{s.}{vergonha; desgraça; humilhação; danos à reputação; incidente vergonhoso}
  \end{Phonetics}
\end{Entry}

\begin{Entry}{耽}{10}{⽿}
  \begin{Phonetics}{耽}{dan1}
    \definition*{s.}{Sobrenome Dan}
    \definition{v.}{atrasar | (literário) abandonar-se a; entregar-se a}
  \end{Phonetics}
\end{Entry}

\begin{Entry}{耽心}{10,4}{⽿、⼼}
  \begin{Phonetics}{耽心}{dan1xin1}
    \variantof{担心}
  \end{Phonetics}
\end{Entry}

\begin{Entry}{耽误}{10,9}{⽿、⾔}
  \begin{Phonetics}{耽误}{dan1wu5}[][HSK 7-9]
    \definition{v.}{atrasar; segurar; perder algo devido a atraso ou oportunidade perdida; perder (oportunidade)}
  \end{Phonetics}
\end{Entry}

\begin{Entry}{耽搁}{10,12}{⽿、⼿}
  \begin{Phonetics}{耽搁}{dan1ge5}[][HSK 7-9]
    \definition{v.}{ficar; fazer uma parada | atrasar | perder (uma oportunidade, um prazo)}
  \end{Phonetics}
\end{Entry}

\begin{Entry}{耿}{10}{⽿}
  \begin{Phonetics}{耿}{geng3}
    \definition{adj.}{Literário: brilhante | honesto e justo; correto; íntegro | dedicado; leal}
    \definition{s.}{Sobrenome Geng}
  \end{Phonetics}
\end{Entry}

\begin{Entry}{耿直}{10,8}{⽿、⽬}
  \begin{Phonetics}{耿直}{geng3zhi2}[][HSK 7-9]
    \definition{adj.}{íntregro; franco; correto; honesto e franco}
  \end{Phonetics}
\end{Entry}

\begin{Entry}{胳}{10}{⾁}
  \begin{Phonetics}{胳}{ga1}
    \definition{s.}{usado em 胳肢窝}
  \seealsoref{胳肢窝}{ga1 zhi1 wo1}
  \end{Phonetics}
  \begin{Phonetics}{胳}{ge1}
    \definition{s.}{axila; sovaco}
  \end{Phonetics}
  \begin{Phonetics}{胳}{ge2}
    \definition{v.}{usado em 胳肢}
  \seealsoref{胳肢}{ge2zhi5}
  \end{Phonetics}
\end{Entry}

\begin{Entry}{胳肢}{10,8}{⾁、⾁}
  \begin{Phonetics}{胳肢}{ge2zhi5}
    \definition{v.}{(dialeto) fazer cócegas}
  \end{Phonetics}
\end{Entry}

\begin{Entry}{胳肢窝}{10,8,12}{⾁、⾁、⽳}
  \begin{Phonetics}{胳肢窝}{ga1 zhi1 wo1}
    \definition{s.}{axila; sovaco; também escrito 夹肢窝}
  \seealsoref{夹肢窝}{jia1 zhi1 wo1}
  \end{Phonetics}
\end{Entry}

\begin{Entry}{胳膊}{10,14}{⾁、⾁}
  \begin{Phonetics}{胳膊}{ge1bo5}[][HSK 7-9]
    \definition[条,双,只]{s.}{braço; a área abaixo do ombro e acima do pulso}
  \end{Phonetics}
\end{Entry}

\begin{Entry}{胶}{10}{⾁}
  \begin{Phonetics}{胶}{jiao1}
    \definition*{s.}{Sobrenome Jiao}
    \definition{adj.}{pegajoso; viscoso; grudento}
    \definition{s.}{cola; goma; adesivo | borracha | gel; colóide}
    \definition{v.}{colar com cola | colar; grudar}
  \end{Phonetics}
\end{Entry}

\begin{Entry}{胶水}{10,4}{⾁、⽔}
  \begin{Phonetics}{胶水}{jiao1shui3}[][HSK 5]
    \definition[瓶]{s.}{cola; mucilagem; cola líquida}
  \end{Phonetics}
\end{Entry}

\begin{Entry}{胶卷}{10,8}{⾁、⼙}
  \begin{Phonetics}{胶卷}{jiao1juan3}
    \definition{s.}{filme | rolo de filme}
  \end{Phonetics}
\end{Entry}

\begin{Entry}{胶带}{10,9}{⾁、⼱}
  \begin{Phonetics}{胶带}{jiao1 dai4}[][HSK 5]
    \definition[卷,条,段]{s.}{fita de embalagem transparente; fita adesiva | fita magnética de plástico; fita de gravação | fita emborrachada; cinta de borracha}
  \end{Phonetics}
\end{Entry}

\begin{Entry}{胸}{10}{⾁}
  \begin{Phonetics}{胸}{xiong1}
    \definition{s.}{peito | tórax}
  \end{Phonetics}
\end{Entry}

\begin{Entry}{胸部}{10,10}{⾁、⾢}
  \begin{Phonetics}{胸部}{xiong1 bu4}[][HSK 4]
    \definition{s.}{peito; tórax; seios}
  \end{Phonetics}
\end{Entry}

\begin{Entry}{能}{10}{⾁}
  \begin{Phonetics}{能}{neng2}[][HSK 1]
    \definition*{s.}{Sobrenome Neng}
    \definition{adv.}{talvez}
    \definition{s.}{habilidade; capacidade; competência | potência; energia; em física, refere-se à energia}
    \definition{v.}{poder fazer; ser capaz de | ser possível | entre 不 \dots 不 para expressar obrigação, certeza ou grande probabilidade | poder; ter permissão para | ser bom em fazer algo | permitir}
  \end{Phonetics}
\end{Entry}

\begin{Entry}{能力}{10,2}{⾁、⼒}
  \begin{Phonetics}{能力}{neng2li4}[][HSK 3]
    \definition[个,种]{s.}{habilidade; capacidade; aptidão; as condições subjetivas para ser competente para uma tarefa}
  \end{Phonetics}
\end{Entry}

\begin{Entry}{能上能下}{10,3,10,3}{⾁、⼀、⾁、⼀}
  \begin{Phonetics}{能上能下}{neng2shang4neng2xia4}
    \definition{s.}{pronto para aceitar qualquer trabalho, alto ou baixo}
  \end{Phonetics}
\end{Entry}

\begin{Entry}{能干}{10,3}{⾁、⼲}
  \begin{Phonetics}{能干}{neng2gan4}[][HSK 4]
    \definition{adj.}{apto; capaz; competente}
  \end{Phonetics}
\end{Entry}

\begin{Entry}{能不能}{10,4,10}{⾁、⼀、⾁}
  \begin{Phonetics}{能不能}{neng2 bu4 neng2}[][HSK 3]
    \definition{adv.}{pode ou não pode\dots?}
  \end{Phonetics}
\end{Entry}

\begin{Entry}{能否}{10,7}{⾁、⼝}
  \begin{Phonetics}{能否}{neng2 fou3}[][HSK 6]
    \definition{adv.}{é possível; se ou não; pode ou não pode; Você consegue?; expressa dúvida, frequentemente usado em perguntas de sim ou não}
  \end{Phonetics}
\end{Entry}

\begin{Entry}{能够}{10,11}{⾁、⼣}
  \begin{Phonetics}{能够}{neng2 gou4}[][HSK 2]
    \definition{v.}{poder; ser capaz de; indica que possui uma determinada capacidade ou que atingiu um determinado nível de eficiência | poder; ser capaz de; indica que algo é permitido sob certas condições ou por motivos razoáveis}
  \end{Phonetics}
\end{Entry}

\begin{Entry}{能量}{10,12}{⾁、⾥}
  \begin{Phonetics}{能量}{neng2liang4}[][HSK 5]
    \definition[种]{s.}{energia; quantidade de energia; Uma grandeza física que mede a capacidade da matéria de realizar trabalho | capacidade; competências; capacidade e papel que uma pessoa pode desempenhar}
  \end{Phonetics}
\end{Entry}

\begin{Entry}{脂}{10}{⾁}
  \begin{Phonetics}{脂}{zhi1}
    \definition*{s.}{Sobrenome Zhi}
    \definition{s.}{gordura; graxa; sebo | (cosméticos) rouge | (cosméticos) baton; protetor labial}
  \end{Phonetics}
\end{Entry}

\begin{Entry}{脂麻}{10,11}{⾁、⿇}
  \begin{Phonetics}{脂麻}{zhi1ma5}
    \variantof{芝麻}
  \end{Phonetics}
\end{Entry}

\begin{Entry}{脆}{10}{⾁}
  \begin{Phonetics}{脆}{cui4}[][HSK 5]
    \definition{adj.}{frágil; quebradiço (oposto a 韧) | crocante | (voz) clara; nítida | puro}
  \seealsoref{韧}{ren4}
  \end{Phonetics}
\end{Entry}

\begin{Entry}{脆弱}{10,10}{⾁、⼸}
  \begin{Phonetics}{脆弱}{cui4ruo4}[][HSK 7-9]
    \definition{adj.}{frágil; débil; fraco; incapaz de suportar contratempos}
  \end{Phonetics}
\end{Entry}

\begin{Entry}{脊}{10}{⾁}
  \begin{Phonetics}{脊}{ji2}
    \definition{s.}{coluna vertebral (de humanos e animais) | espinha; costas; cumeeira; a parte de um objeto em forma de espinha}
  \end{Phonetics}
  \begin{Phonetics}{脊}{ji3}
    \definition{s.}{espinha dorsal; coluna vertebral | crista; cumeeira; espinhaço | vértebra}
  \end{Phonetics}
\end{Entry}

\begin{Entry}{脊梁}{10,11}{⾁、⽊}
  \begin{Phonetics}{脊梁}{ji3liang2}[][HSK 7-9]
    \definition{s.}{espinha dorsal | coluna vertebral}
  \end{Phonetics}
\end{Entry}

\begin{Entry}{脏}{10}{⾁}
  \begin{Phonetics}{脏}{zang1}[][HSK 2]
    \definition{adj.}{sujo; imundo | imundo; metáfora para vulgaridade e obscenidade}
    \definition{v.}{tornar algo sujo ou impuro}
  \end{Phonetics}
  \begin{Phonetics}{脏}{zang4}
    \definition[处]{s.}{vísceras; órgãos internos do corpo, geralmente o coração, o fígado, o baço, os pulmões e os rins; um termo geral para órgãos nas cavidades torácica e abdominal de humanos ou animais | (anatomia) órgão; a medicina tradicional chinesa chama o coração, o fígado, o baço, os pulmões e os rins de órgãos internos}
  \end{Phonetics}
\end{Entry}

\begin{Entry}{脏土}{10,3}{⾁、⼟}
  \begin{Phonetics}{脏土}{zang1tu3}
    \definition{s.}{solo sujo | lama | lixo}
  \end{Phonetics}
\end{Entry}

\begin{Entry}{脏字}{10,6}{⾁、⼦}
  \begin{Phonetics}{脏字}{zang1zi4}
    \definition{s.}{obscenidade}
  \end{Phonetics}
\end{Entry}

\begin{Entry}{脏病}{10,10}{⾁、⽧}
  \begin{Phonetics}{脏病}{zang1bing4}
    \definition{s.}{doença venérea}
  \end{Phonetics}
\end{Entry}

\begin{Entry}{脏脏}{10,10}{⾁、⾁}
  \begin{Phonetics}{脏脏}{zang1zang1}
    \definition{adj.}{sujo}
  \end{Phonetics}
\end{Entry}

\begin{Entry}{脏煤}{10,13}{⾁、⽕}
  \begin{Phonetics}{脏煤}{zang1mei2}
    \definition{s.}{carvão sujo | sujeira (de uma mina de carvão)}
  \end{Phonetics}
\end{Entry}

\begin{Entry}{脏器}{10,16}{⾁、⼝}
  \begin{Phonetics}{脏器}{zang4qi4}
    \definition{s.}{órgãos internos}
  \end{Phonetics}
\end{Entry}

\begin{Entry}{脏辫}{10,17}{⾁、⾟}
  \begin{Phonetics}{脏辫}{zang1bian4}
    \definition{s.}{\emph{dreadlocks}}
  \end{Phonetics}
\end{Entry}

\begin{Entry}{脑}{10}{⾁}
  \begin{Phonetics}{脑}{nao3}
    \definition{s.}{(fisiologia) cérebro | tofu;  substância branca semelhante ao cérebro ou à medula espinhal cerebral | cabeça | a essência de um objeto}
  \end{Phonetics}
\end{Entry}

\begin{Entry}{脑子}{10,3}{⾁、⼦}
  \begin{Phonetics}{脑子}{nao3 zi5}[][HSK 5]
    \definition[个]{s.}{cérebro | mente; cabeça; cérebro; inteligência; poder mental; refere-se à capacidade de pensar, memorizar, raciocinar, etc.; inteligência}
  \end{Phonetics}
\end{Entry}

\begin{Entry}{脑瓜}{10,5}{⾁、⽠}
  \begin{Phonetics}{脑瓜}{nao3gua1}
    \definition{s.}{crânio | cérebro | cabeça | mente | mentalidade | ideia}
  \seealsoref{脑瓜子}{nao3gua1zi5}
  \end{Phonetics}
\end{Entry}

\begin{Entry}{脑瓜子}{10,5,3}{⾁、⽠、⼦}
  \begin{Phonetics}{脑瓜子}{nao3gua1zi5}
    \definition{s.}{Coloquial: crânio; cérebro; cabeça; mente; mentalidade; ideia}
  \seealsoref{脑瓜}{nao3gua1}
  \end{Phonetics}
\end{Entry}

\begin{Entry}{脑袋}{10,11}{⾁、⾐}
  \begin{Phonetics}{脑袋}{nao3dai5}[][HSK 4]
    \definition[颗,个]{s.}{cabeça; a parte mais alta do corpo humano ou a parte mais alta de um animal que contém órgãos como a boca, o nariz, os olhos etc. | mente; cérebro; capacidade de pensar, lembrar, etc.}
  \end{Phonetics}
\end{Entry}

\begin{Entry}{臭}{10}{⾃}
  \begin{Phonetics}{臭}{chou4}[][HSK 5]
    \definition{adj.}{sujo; malcheiroso; fedorento; contrário de 香 | repugnante; nojento; repulsivo | ruim; pobre; péssimo}
    \definition{adv.}{severamente; firmemente}
    \definition{v.}{falhar em detonar (bala)}
  \seealsoref{香}{xiang1}
  \end{Phonetics}
  \begin{Phonetics}{臭}{xiu4}
    \definition{s.}{odor; cheiro}
    \definition{v.}{cheirar; farejar; o mesmo que 嗅}
  \seealsoref{嗅}{xiu4}
  \end{Phonetics}
\end{Entry}

\begin{Entry}{臭气}{10,4}{⾃、⽓}
  \begin{Phonetics}{臭气}{chou4qi4}
    \definition{s.}{fedor}
  \end{Phonetics}
\end{Entry}

\begin{Entry}{致}{10}{⾄}
  \begin{Phonetics}{致}{zhi4}
    \definition{adj.}{fino; delicado; meticuloso; preciso}
    \definition{s.}{interesse}
    \definition{v.}{enviar; estender; entregar; dar; mostrar (cortesia, afeto, etc.) à outra parte | concentrar-se; trabalhar para; dedicar (os próprios esforços, etc.); focar em um aspecto | causar; incorrer; convidar; levar a | alcançar}
  \end{Phonetics}
\end{Entry}

\begin{Entry}{致敬}{10,12}{⾄、⽁}
  \begin{Phonetics}{致敬}{zhi4jing4}
    \definition{v.}{saudar | prestar respeitos a | prestar homenagem a}
  \end{Phonetics}
\end{Entry}

\begin{Entry}{航}{10}{⾈}
  \begin{Phonetics}{航}{hang2}
    \definition*{s.}{Sobrenome Hang}
    \definition[趟]{s.}{barco; navio}
    \definition{v.}{navegar (por água ou ar) | velejar}
  \end{Phonetics}
\end{Entry}

\begin{Entry}{航天}{10,4}{⾈、⼤}
  \begin{Phonetics}{航天}{hang2tian1}[][HSK 7-9]
    \definition{s.}{voo espacial; astronáutica}
    \definition{v.}{voar ou viajar no espaço}
  \end{Phonetics}
\end{Entry}

\begin{Entry}{航天员}{10,4,7}{⾈、⼤、⼝}
  \begin{Phonetics}{航天员}{hang2tian1yuan2}[][HSK 7-9]
    \definition[名,位,个]{s.}{astronauta}
  \end{Phonetics}
\end{Entry}

\begin{Entry}{航行}{10,6}{⾈、⾏}
  \begin{Phonetics}{航行}{hang2xing2}[][HSK 7-9]
    \definition{v.}{velejar; voar; navegar pela água, pelo ar}
  \end{Phonetics}
\end{Entry}

\begin{Entry}{航运}{10,7}{⾈、⾡}
  \begin{Phonetics}{航运}{hang2yun4}[][HSK 7-9]
    \definition{s.}{transporte hidroviário; transporte marítimo}
  \end{Phonetics}
\end{Entry}

\begin{Entry}{航空}{10,8}{⾈、⽳}
  \begin{Phonetics}{航空}{hang2kong1}[][HSK 4]
    \definition{s.}{viagem; aviação; refere-se ao voo de uma aeronave no ar}
  \end{Phonetics}
\end{Entry}

\begin{Entry}{航海}{10,10}{⾈、⽔}
  \begin{Phonetics}{航海}{hang2hai3}[][HSK 7-9]
    \definition{v.}{velejar; navegar}
  \end{Phonetics}
\end{Entry}

\begin{Entry}{航班}{10,10}{⾈、⽟}
  \begin{Phonetics}{航班}{hang2ban1}[][HSK 4]
    \definition[个,次]{s.}{número do voo; voo programado; o horário de um navio ou avião de passageiros}
  \end{Phonetics}
\end{Entry}

\begin{Entry}{般}{10}{⾈}
  \begin{Phonetics}{般}{ban1}
    \definition{clas.}{tipo; classe; gênero; amostra}
    \definition{part.}{(o mesmo) que; como; semelhante}
  \end{Phonetics}
  \begin{Phonetics}{般}{bo1}
    \definition{s.}{utilizado em 般若}
  \seealsoref{般若}{bo1re3}
  \end{Phonetics}
  \begin{Phonetics}{般}{pan2}
    \definition{adj.}{feliz; bem-aventurado}
  \end{Phonetics}
\end{Entry}

\begin{Entry}{般乐}{10,5}{⾈、⼃}
  \begin{Phonetics}{般乐}{pan2le4}
    \definition{v.}{jogar | divertir-se}
  \end{Phonetics}
\end{Entry}

\begin{Entry}{般若}{10,8}{⾈、⾋}
  \begin{Phonetics}{般若}{bo1re3}
    \definition*{s.}{Prajña (sânscrito), \emph{insight} sobre a verdadeira natureza da realidade}
    \definition{s.}{budismo: sabedoria}
  \end{Phonetics}
\end{Entry}

\begin{Entry}{舱}{10}{⾈}
  \begin{Phonetics}{舱}{cang1}[][HSK 7-9]
    \definition{s.}{cabine (de um avião ou navio) | módulo (de uma nave espacial) | espaço em um navio ou aeronave para transportar pessoas, carga ou máquinas}
  \end{Phonetics}
\end{Entry}

\begin{Entry}{荷}{10}{⾋}
  \begin{Phonetics}{荷}{he2}
    \definition*{s.}{Países Baixos; Holanda, abreviação de 荷兰 | Sobrenome He}
    \definition{s.}{lótus}
  \seealsoref{荷兰}{he2lan2}
  \end{Phonetics}
  \begin{Phonetics}{荷}{he4}
    \definition{s.}{fardo; responsabilidade}
    \definition{v.}{carregar no ombro ou nas costas | aceitar um favor, frequentemente usado em cartas para expressar cortesia}
  \end{Phonetics}
\end{Entry}

\begin{Entry}{荷兰}{10,5}{⾋、⼋}
  \begin{Phonetics}{荷兰}{he2lan2}
    \definition*{s.}{Países Baixos; Holanda}
  \end{Phonetics}
\end{Entry}

\begin{Entry}{荷花}{10,7}{⾋、⾋}
  \begin{Phonetics}{荷花}{he2hua1}[][HSK 7-9]
    \definition[朵,枝,片]{s.}{lótus; flor de lótus}
  \end{Phonetics}
\end{Entry}

\begin{Entry}{莎}{10}{⾋}
  \begin{Phonetics}{莎}{sha1}
    \definition{s.}{em nomes pessoais e de lugares | cigarra | fonético "sha" usado na transliteração}
  \end{Phonetics}
  \begin{Phonetics}{莎}{suo1}
  \end{Phonetics}
\end{Entry}

\begin{Entry}{莎莎舞}{10,10,14}{⾋、⾋、⾇}
  \begin{Phonetics}{莎莎舞}{sha1sha1wu3}
    \definition{s.}{salsa (dança)}
  \end{Phonetics}
\end{Entry}

\begin{Entry}{莫}{10}{⾋}
  \begin{Phonetics}{莫}{mo4}
    \definition*{s.}{Sobrenome Mo}
    \definition{adv.}{não, frequentemente usado em frases imperativas | não; não pode | pode ser que; não pode ser que; é possível que}
    \definition{pron.}{nenhum; nada; ninguém; significa 没有谁 ou 没有哪一种东西}
  \seealsoref{没有哪一种东西}{mei2you3 na3 yi4 zhong3 dong1xi1}
  \seealsoref{没有谁}{mei2you3 shei2}
  \end{Phonetics}
\end{Entry}

\begin{Entry}{莫名其妙}{10,6,8,7}{⾋、⼝、⼋、⼥}
  \begin{Phonetics}{莫名其妙}{mo4ming2qi2miao4}
    \definition{adj.}{desconcertante | bizzaro | inexplicável | perplexo}
  \end{Phonetics}
\end{Entry}

\begin{Entry}{莫非}{10,8}{⾋、⾮}
  \begin{Phonetics}{莫非}{mo4fei1}
    \definition{expr.}{Não é mesmo?; é frequentemente usado com 不成}
    \definition{v.}{pode ser que; é possível que}
  \seealsoref{不成}{bu4 cheng2}
  \end{Phonetics}
\end{Entry}

\begin{Entry}{莲}{10}{⾋}
  \begin{Phonetics}{莲}{lian2}
    \definition*{s.}{Sobrenome Lian}
    \definition[粒]{s.}{lótus}
  \end{Phonetics}
\end{Entry}

\begin{Entry}{莲花}{10,7}{⾋、⾋}
  \begin{Phonetics}{莲花}{lian2hua1}
    \definition{s.}{flor de lótus | lírio aquático}
  \end{Phonetics}
\end{Entry}

\begin{Entry}{莲藕}{10,18}{⾋、⾋}
  \begin{Phonetics}{莲藕}{lian2'ou3}
    \definition{s.}{raiz de Lotus}
  \end{Phonetics}
\end{Entry}

\begin{Entry}{获}{10}{⾋}
  \begin{Phonetics}{获}{huo4}[][HSK 4]
    \definition*{s.}{Sobrenome Huo}
    \definition{v.}{capturar; pegar | obter; ganhar; colher | colher; ceifar}
  \end{Phonetics}
\end{Entry}

\begin{Entry}{获取}{10,8}{⾋、⼜}
  \begin{Phonetics}{获取}{huo4 qu3}[][HSK 4]
    \definition{v.}{adquirir; obter; ganhar; colher}
  \end{Phonetics}
\end{Entry}

\begin{Entry}{获奖}{10,9}{⾋、⼤}
  \begin{Phonetics}{获奖}{huo4 jiang3}[][HSK 4]
    \definition{v.}{ganhar prêmio; ser recompensado; ganhar um prêmio; receber um prêmio}
  \end{Phonetics}
\end{Entry}

\begin{Entry}{获胜}{10,9}{⾋、⾁}
  \begin{Phonetics}{获胜}{huo4sheng4}[][HSK 7-9]
    \definition{v.}{vencer; ser vitorioso; triunfar; alcançar a vitória}
  \end{Phonetics}
\end{Entry}

\begin{Entry}{获得}{10,11}{⾋、⼻}
  \begin{Phonetics}{获得}{huo4de2}[][HSK 4]
    \definition{v.}{adquirir; ganhar; obter; alcançar}
  \end{Phonetics}
\end{Entry}

\begin{Entry}{获悉}{10,11}{⾋、⼼}
  \begin{Phonetics}{获悉}{huo4xi1}[][HSK 7-9]
    \definition{v.}{saber (de um evento); receber notícias; ser informado}
  \end{Phonetics}
\end{Entry}

\begin{Entry}{蚊}{10}{⾍}
  \begin{Phonetics}{蚊}{wen2}
    \definition{s.}{mosquito; pernilongo}
  \end{Phonetics}
\end{Entry}

\begin{Entry}{蚊子}{10,3}{⾍、⼦}
  \begin{Phonetics}{蚊子}{wen2zi5}
    \definition{s.}{pernilongo}
  \end{Phonetics}
\end{Entry}

\begin{Entry}{蚊香}{10,9}{⾍、⾹}
  \begin{Phonetics}{蚊香}{wen2xiang1}
    \definition{s.}{incenso ou espiral repelente de mosquitos}
  \end{Phonetics}
\end{Entry}

\begin{Entry}{蚕}{10}{⾍}
  \begin{Phonetics}{蚕}{can2}
    \definition[只,条]{s.}{bicho-da-seda; um inseto que pode fiar seda e fazer casulos}
  \end{Phonetics}
\end{Entry}

\begin{Entry}{蚕纸}{10,7}{⾍、⽷}
  \begin{Phonetics}{蚕纸}{can2zhi3}
    \definition{s.}{papel onde o bicho-da-seda põe seus ovos}
  \end{Phonetics}
\end{Entry}

\begin{Entry}{蚝}{10}{⾍}
  \begin{Phonetics}{蚝}{hao2}
    \definition[只]{s.}{ostra}
  \end{Phonetics}
\end{Entry}

\begin{Entry}{袖}{10}{⾐}
  \begin{Phonetics}{袖}{xiu4}
    \definition{s.}{manga (de camisa, de camiseta, etc.)}
  \end{Phonetics}
\end{Entry}

\begin{Entry}{袖珍}{10,9}{⾐、⽟}
  \begin{Phonetics}{袖珍}{xiu4 zhen1}[][HSK 6]
    \definition{adj.}{do tamanho do bolso; de bolso (livro, agenda, etc.)}
  \end{Phonetics}
\end{Entry}

\begin{Entry}{袜}{10}{⾐}
  \begin{Phonetics}{袜}{wa4}
    \definition[只,双,打]{s.}{meias; meias-calças}
  \end{Phonetics}
\end{Entry}

\begin{Entry}{袜子}{10,3}{⾐、⼦}
  \begin{Phonetics}{袜子}{wa4zi5}[][HSK 4]
    \definition[双,只,对]{s.}{meias; peúgas; meias-calças}
  \end{Phonetics}
\end{Entry}

\begin{Entry}{被}{10}{⾐}
  \begin{Phonetics}{被}{bei4}[][HSK 3]
    \definition{part.}{usada antes de verbos para formar frases verbais passivas}
    \definition{prep.}{usado em uma estrutura passiva para introduzir o executor da ação ou apenas a ação | usado em frases para expressar passividade, com o sujeito sendo o objeto}
    \definition{s.}{colcha}
    \definition{v.}{cobrir; espalhar | sofrer}
  \end{Phonetics}
\end{Entry}

\begin{Entry}{被子}{10,3}{⾐、⼦}
  \begin{Phonetics}{被子}{bei4zi5}[][HSK 3]
    \definition[条,床]{s.}{colcha; cobertor; algo com que você se cobre quando dorme, geralmente feito de pano ou seda, com forro de pano e preenchido com algodão ou fio de seda}
  \end{Phonetics}
\end{Entry}

\begin{Entry}{被动}{10,6}{⾐、⼒}
  \begin{Phonetics}{被动}{bei4dong4}[][HSK 5]
    \definition{adj.}{passivo;  agir com base em um impulso externo (oposto de 主动) | passivo; impossibilidade de prosseguir como pretendido devido a resistência ou interferência}
  \seealsoref{主动}{zhu3dong4}
  \end{Phonetics}
\end{Entry}

\begin{Entry}{被告}{10,7}{⾐、⼝}
  \begin{Phonetics}{被告}{bei4gao4}[][HSK 6]
    \definition{s.}{réu; indiciado; acusado (oposto a 原告)}
  \seealsoref{原告}{yuan2gao4}
  \end{Phonetics}
\end{Entry}

\begin{Entry}{被单}{10,8}{⾐、⼗}
  \begin{Phonetics}{被单}{bei4dan1}
    \definition[床]{s.}{lençol (de cama) | envelope para uma colcha acolchoada}
  \end{Phonetics}
\end{Entry}

\begin{Entry}{被迫}{10,8}{⾐、⾡}
  \begin{Phonetics}{被迫}{bei4 po4}[][HSK 4]
    \definition{v.}{ser forçado; ser coagido; ser compelido; ser constrangido; ser forçado a fazer algo por força externa}
  \end{Phonetics}
\end{Entry}

\begin{Entry}{被套}{10,10}{⾐、⼤}
  \begin{Phonetics}{被套}{bei4tao4}
    \definition{s.}{capa de \emph{edredon}}
    \definition{v.}{ter dinheiro preso (em ações, imóveis, etc.)}
  \end{Phonetics}
\end{Entry}

\begin{Entry}{被捕}{10,10}{⾐、⼿}
  \begin{Phonetics}{被捕}{bei4bu3}[][HSK 7-9]
    \definition{v.}{ser preso; estar sob prisão}
  \end{Phonetics}
\end{Entry}

\begin{Entry}{被窝}{10,12}{⾐、⽳}
  \begin{Phonetics}{被窝}{bei4wo1}
    \definition{s.}{colcha}
  \end{Phonetics}
\end{Entry}

\begin{Entry}{请}{10}{⾔}
  \begin{Phonetics}{请}{qing3}[][HSK 1]
    \definition*{s.}{Sobrenome Qing}
    \definition{v.}{solicitar; perguntar | convidar; envolver | por favor; uma expressão educada usada quando você quer que alguém faça algo | comprar coisas sagradas para sacrifício, como incenso, velas, cavalos de papel e santuários de Buda; superstição se refere à compra de estátuas de Buda, santuários, etc. | entreter}
  \end{Phonetics}
\end{Entry}

\begin{Entry}{请问}{10,6}{⾔、⾨}
  \begin{Phonetics}{请问}{qing3 wen4}[][HSK 1]
    \definition{expr.}{Com licença, posso perguntar\dots? (para perguntar por qualquer coisa); uma maneira educada de pedir para alguém responder a uma pergunta}
  \end{Phonetics}
\end{Entry}

\begin{Entry}{请坐}{10,7}{⾔、⼟}
  \begin{Phonetics}{请坐}{qing3 zuo4}[][HSK 1]
    \definition{v.}{por favor, sente-se; convidar outras pessoas para sentar ou descansar}
  \end{Phonetics}
\end{Entry}

\begin{Entry}{请求}{10,7}{⾔、⽔}
  \begin{Phonetics}{请求}{qing3qiu2}[][HSK 2]
    \definition[个,次]{s.}{pedido; petição; solicitação; refere-se à exigência apresentada}
    \definition{v.}{pedir; solicitar; requerer; peticionar; fazer uma solicitação e pedir que a outra parte concorde com ela}
  \end{Phonetics}
\end{Entry}

\begin{Entry}{请进}{10,7}{⾔、⾡}
  \begin{Phonetics}{请进}{qing3 jin4}[][HSK 1]
    \definition{v.}{por favor entre; convidar alguém para um espaço ou lugar}
  \end{Phonetics}
\end{Entry}

\begin{Entry}{请客}{10,9}{⾔、⼧}
  \begin{Phonetics}{请客}{qing3/ke4}[][HSK 2]
    \definition{v.+compl.}{receber convidados; hospedar convidados | oferecer; convidar; pagar a conta; arcar com os custos; convidar alguém para comer, tomar chá, etc.}
  \end{Phonetics}
\end{Entry}

\begin{Entry}{请假}{10,11}{⾔、⼈}
  \begin{Phonetics}{请假}{qing3/jia4}[][HSK 1]
    \definition{v.+compl.}{pedir licença para sair; solicitar permissão para não trabalhar ou estudar por um determinado período de tempo devido a doença ou outros motivos}
  \end{Phonetics}
\end{Entry}

\begin{Entry}{请假条}{10,11,7}{⾔、⼈、⽊}
  \begin{Phonetics}{请假条}{qing3jia4tiao2}
    \definition{s.}{pedido de licença de ausência (do trabalho ou da escola)}
  \end{Phonetics}
\end{Entry}

\begin{Entry}{请教}{10,11}{⾔、⽁}
  \begin{Phonetics}{请教}{qing3jiao4}[][HSK 3]
    \definition{v.}{consultar; pedir conselho}
  \end{Phonetics}
\end{Entry}

\begin{Entry}{诸}{10}{⾔}
  \begin{Phonetics}{诸}{zhu1}
    \definition*{s.}{Sobrenome Zhu}
    \definition{adj.}{todos; cada; vários}
    \definition{prep.}{em; para; de}
  \end{Phonetics}
\end{Entry}

\begin{Entry}{诸位}{10,7}{⾔、⼈}
  \begin{Phonetics}{诸位}{zhu1wei4}[][HSK 6]
    \definition{pron.}{senhores; todos; todos vocês; senhoras e senhores; um termo educado que se refere a várias pessoas}
  \end{Phonetics}
\end{Entry}

\begin{Entry}{诺}{10}{⾔}
  \begin{Phonetics}{诺}{nuo4}
    \definition*{s.}{Sobrenome Nuo}
    \definition{interj.}{Sim!}
    \definition{v.}{prometer}
  \end{Phonetics}
\end{Entry}

\begin{Entry}{诺贝尔奖}{10,4,5,9}{⾔、⾙、⼩、⼤}
  \begin{Phonetics}{诺贝尔奖}{nuo4bei4'er3 jiang3}
    \definition*{s.}{Prêmio Nobel}
  \end{Phonetics}
\end{Entry}

\begin{Entry}{诺奖}{10,9}{⾔、⼤}
  \begin{Phonetics}{诺奖}{nuo4jiang3}
    \definition*{s.}{Prêmio Nobel, abreviação de 诺贝尔奖}
  \seealsoref{诺贝尔奖}{nuo4bei4'er3 jiang3}
  \end{Phonetics}
\end{Entry}

\begin{Entry}{读}{10}{⾔}
  \begin{Phonetics}{读}{dou4}
    \definition{s.}{vírgula; uma breve pausa na leitura}
  \end{Phonetics}
  \begin{Phonetics}{读}{du2}[][HSK 1]
    \definition*{s.}{Sobrenome Du}
    \definition{v.}{ler em voz alta | ler; ler o texto e compreendera seu significado | frequentar a escola; refere-se a ir à escola ou estudar | (computação) ler dados}
  \end{Phonetics}
\end{Entry}

\begin{Entry}{读书}{10,4}{⾔、⼄}
  \begin{Phonetics}{读书}{du2/shu1}[][HSK 1]
    \definition{v.+compl.}{ler; estudar | frequentar a escola}
  \end{Phonetics}
\end{Entry}

\begin{Entry}{读者}{10,8}{⾔、⽼}
  \begin{Phonetics}{读者}{du2 zhe3}[][HSK 3]
    \definition[个,位,名,些,群]{s.}{leitor; (para obras, autores, revistas, etc.) Pessoas que compram ou leem livros, revistas, artigos, jornais, etc.}
  \end{Phonetics}
\end{Entry}

\begin{Entry}{读音}{10,9}{⾔、⾳}
  \begin{Phonetics}{读音}{du2 yin1}[][HSK 2]
    \definition[种]{s.}{pronúncia}
  \end{Phonetics}
\end{Entry}

\begin{Entry}{诽}{10}{⾔}
  \begin{Phonetics}{诽}{fei3}
    \definition{v.}{calúnia}
  \end{Phonetics}
\end{Entry}

\begin{Entry}{诽谤}{10,12}{⾔、⾔}
  \begin{Phonetics}{诽谤}{fei3bang4}[][HSK 7-9]
    \definition{v.}{difamar; caluniar; falar mal; espalhar boatos e caluniar os outros; falar mal dos outros e prejudicar sua reputação}
  \end{Phonetics}
\end{Entry}

\begin{Entry}{课}{10}{⾔}
  \begin{Phonetics}{课}{ke4}[][HSK 1]
    \definition{clas.}{aula; lição; unidade de tempo de ensino; parágrafo do material didático}
    \definition[门,节]{s.}{classe; aula; ensino por etapas planejado | disciplina; curso | imposto; antiga referência a impostos | seção; departamentos de escritório criados no antigo governo}
    \definition{v.}{cobrar; impor; taxar}
  \end{Phonetics}
\end{Entry}

\begin{Entry}{课文}{10,4}{⾔、⽂}
  \begin{Phonetics}{课文}{ke4 wen2}[][HSK 1]
    \definition[篇,段]{s.}{texto (de uma lição); texto principal do livro didático (diferente das notas de rodapé, exercícios, etc.)}
  \end{Phonetics}
\end{Entry}

\begin{Entry}{课本}{10,5}{⾔、⽊}
  \begin{Phonetics}{课本}{ke4 ben3}[][HSK 1]
    \definition[本]{s.}{livro didático; livro-texto}
  \end{Phonetics}
\end{Entry}

\begin{Entry}{课堂}{10,11}{⾔、⼟}
  \begin{Phonetics}{课堂}{ke4 tang2}[][HSK 2]
    \definition[间,节,个]{s.}{sala de aula; local onde se realizam as aulas; local onde se realizam as atividades de ensino}
  \end{Phonetics}
\end{Entry}

\begin{Entry}{课程}{10,12}{⾔、⽲}
  \begin{Phonetics}{课程}{ke4cheng2}[][HSK 3]
    \definition[个,堂,节,门]{s.}{curso; currículo; as disciplinas e o programa letivo da escola}
  \end{Phonetics}
\end{Entry}

\begin{Entry}{课题}{10,15}{⾔、⾴}
  \begin{Phonetics}{课题}{ke4ti2}[][HSK 5]
    \definition[组]{s.}{uma questão para estudo ou discussão; principais questões a serem pesquisadas ou discutidas, ou assuntos importantes que precisam ser resolvidos com urgência | tarefa; problema; questões a serem resolvidas}
  \end{Phonetics}
\end{Entry}

\begin{Entry}{谁}{10}{⾔}
  \begin{Phonetics}{谁}{shei2}[][HSK 1]
    \definition{pron.}{quem? | (em pergunta retórica) quem?; usado em perguntas retóricas, para indicar que não há ninguém | refere-se a pessoas que não têm certeza, incluindo aquelas que não sabem | alguém; qualquer pessoa; indica qualquer pessoa ou qualquer um | repetido em uma frase para se referir a uma pessoa | (repetido em duas frases) quem quer que seja; fazer com que o sujeito e o objeto se refiram a duas pessoas diferentes}
  \end{Phonetics}
  \begin{Phonetics}{谁}{shui2}[][HSK 1]
  \end{Phonetics}
\end{Entry}

\begin{Entry}{调}{10}{⾔}
  \begin{Phonetics}{调}{diao4}[][HSK 3]
    \definition{s.}{sotaque; pronúncia | nota (musical) | melodia; música | tom; refere-se ao tom da fala, ou seja, a elevação e descida do tom das palavras | estilo; ambiente; estilo metafórico, talento, etc. | argumento; discurso}
    \definition{v.}{deslocar; mover; transferir; mover (pessoas, objetos, etc.) de um lugar para outro | examinar; investigar}
  \end{Phonetics}
  \begin{Phonetics}{调}{tiao2}[][HSK 3]
    \definition{adj.}{harmonioso; boa coordenação}
    \definition{v.}{misturar; ajustar; fazer o ajuste uniforme e apropriado | provocar; importunar; fazer pouco de | incitar; instigar; provocar; semear discórdia | mediar; trazer harmonia}
  \end{Phonetics}
\end{Entry}

\begin{Entry}{调皮}{10,5}{⾔、⽪}
  \begin{Phonetics}{调皮}{tiao2pi2}[][HSK 4]
    \definition{adj.}{travesso; malicioso; malandro | indisciplinado; desordeiro; indomável; astuto | inteligente e desonesto}
  \end{Phonetics}
\end{Entry}

\begin{Entry}{调节}{10,5}{⾔、⾋}
  \begin{Phonetics}{调节}{tiao2jie2}[][HSK 5]
    \definition{v.}{regular; ajustar; ajustar e controlar de várias maneiras para atender aos requisitos}
  \end{Phonetics}
\end{Entry}

\begin{Entry}{调动}{10,6}{⾔、⼒}
  \begin{Phonetics}{调动}{diao4dong4}[][HSK 5]
    \definition{v.}{mudar; transferir; pessoal, trabalho | mobilizar; despertar; pôr em jogo; melhorar (motivação, entusiasmo, etc.) por meio de alguns meios | reunir; manobrar; mover (tropas); mobilizar forças militares}
  \end{Phonetics}
\end{Entry}

\begin{Entry}{调度}{10,9}{⾔、⼴}
  \begin{Phonetics}{调度}{diao4du4}[][HSK 7-9]
    \definition[位,个]{s.}{despachante; pessoal responsável pelo despacho do trabalho}
    \definition{v.}{organizar e despachar; arranjar e despachar}
  \end{Phonetics}
\end{Entry}

\begin{Entry}{调律}{10,9}{⾔、⼻}
  \begin{Phonetics}{调律}{tiao2lv4}
    \definition{v.}{afinar (por exemplo, um piano)}
  \end{Phonetics}
\end{Entry}

\begin{Entry}{调查}{10,9}{⾔、⽊}
  \begin{Phonetics}{调查}{diao4cha2}[][HSK 3]
    \definition[项,个,份]{s.}{pesquisa; investigação; informações obtidas após perguntar a outras pessoas ou investigar}
    \definition{v.}{investigar; indagar; inquerir; examinar; realizar uma investigação (geralmente no local) para entender a situação}
  \end{Phonetics}
\end{Entry}

\begin{Entry}{调研}{10,9}{⾔、⽯}
  \begin{Phonetics}{调研}{diao4 yan2}[][HSK 6]
    \definition{v.}{pesquisar e estudar; investigar e pesquisar; pesquisar}
  \end{Phonetics}
\end{Entry}

\begin{Entry}{调解}{10,13}{⾔、⾓}
  \begin{Phonetics}{调解}{tiao2jie3}[][HSK 5]
    \definition{v.}{mediar; fazer as pazes; resolver conflitos através da persuasão}
  \end{Phonetics}
\end{Entry}

\begin{Entry}{调整}{10,16}{⾔、⽁}
  \begin{Phonetics}{调整}{tiao2zheng3}[][HSK 3]
    \definition{v.}{ajustar; revisar; regularizar; fazer as alterações apropriadas no estado original para se adaptar à nova situação}
  \end{Phonetics}
\end{Entry}

\begin{Entry}{谈}{10}{⾔}
  \begin{Phonetics}{谈}{tan2}[][HSK 3]
    \definition*{s.}{Sobrenome Tan}
    \definition{s.}{o que é dito ou falado; discurso}
    \definition{v.}{falar; bater papo; discutir}
  \end{Phonetics}
\end{Entry}

\begin{Entry}{谈判}{10,7}{⾔、⼑}
  \begin{Phonetics}{谈判}{tan2pan4}[][HSK 3]
    \definition{v.}{negociar; manter conversações; para resolver um grande problema, as partes relevantes trocaram opiniões entre si, na esperança de encontrar uma solução com a qual todos pudessem concordar}
  \end{Phonetics}
\end{Entry}

\begin{Entry}{谈话}{10,8}{⾔、⾔}
  \begin{Phonetics}{谈话}{tan2/hua4}[][HSK 3]
    \definition[次]{s.}{declaração; opiniões (principalmente políticas) expressas na forma de conversas}
    \definition{v.+compl.}{conversar; discutir | falar; refere-se especificamente ao uso da conversa para entender a situação, fazer trabalho ideológico, etc. (usado principalmente por superiores para subordinados)}
  \end{Phonetics}
\end{Entry}

\begin{Entry}{谈恋爱}{10,10,10}{⾔、⼼、⽖}
  \begin{Phonetics}{谈恋爱}{tan2lian4'ai4}
    \definition{v.}{namorar | apaixonar-se}
  \end{Phonetics}
\end{Entry}

\begin{Entry}{豹}{10}{⾘}
  \begin{Phonetics}{豹}{bao4}[][HSK 7-9]
    \definition*{s.}{Sobrenome Bao}
    \definition[只]{s.}{leopardo; pantera | espécies de gato da montanha}
  \end{Phonetics}
\end{Entry}

\begin{Entry}{豹子}{10,3}{⾘、⼦}
  \begin{Phonetics}{豹子}{bao4zi5}
    \definition[头]{s.}{leopardo}
  \end{Phonetics}
\end{Entry}

\begin{Entry}{贿}{10}{⾙}
  \begin{Phonetics}{贿}{hui4}
    \definition[行]{s.}{bens; riqueza; objetos de valor; propriedade | suborno | Literário: wealth}
    \definition{v.}{subornar}
  \end{Phonetics}
\end{Entry}

\begin{Entry}{贿赂}{10,10}{⾙、⾙}
  \begin{Phonetics}{贿赂}{hui4lu4}[][HSK 7-9]
    \definition[笔]{s.}{suborno}
    \definition{v.}{subornar; subornar outros com dinheiro}
  \end{Phonetics}
\end{Entry}

\begin{Entry}{资}{10}{⾙}
  \begin{Phonetics}{资}{zi1}
    \definition{s.}{recursos | capital | dinheiro | despesa}
    \definition{v.}{fornecer | suprir}
  \end{Phonetics}
\end{Entry}

\begin{Entry}{资本}{10,5}{⾙、⽊}
  \begin{Phonetics}{资本}{zi1ben3}[][HSK 5]
    \definition{s.}{capital; meios de produção ou moeda utilizados para fins lucrativos | o que é capitalizado; algo usado em benefício próprio; metáfora para obter benefícios}
  \end{Phonetics}
\end{Entry}

\begin{Entry}{资产}{10,6}{⾙、⼇}
  \begin{Phonetics}{资产}{zi1chan3}[][HSK 5]
    \definition{s.}{propriedade; bens; patrimônio | capital; fundo de capital; recursos financeiros da empresa | ativos; na contabilidade, refere-se à utilização de fundos}
  \end{Phonetics}
\end{Entry}

\begin{Entry}{资助}{10,7}{⾙、⼒}
  \begin{Phonetics}{资助}{zi1zhu4}[][HSK 5]
    \definition{s.}{subsídio}
    \definition{v.}{subsidiar; patrocinar; ajudar financeiramente; ajudar com recursos financeiros}
  \end{Phonetics}
\end{Entry}

\begin{Entry}{资金}{10,8}{⾙、⾦}
  \begin{Phonetics}{资金}{zi1jin1}[][HSK 3]
    \definition[笔]{s.}{fundo; capital; capital necessário para atividades comerciais, etc.}
  \end{Phonetics}
\end{Entry}

\begin{Entry}{资料}{10,10}{⾙、⽃}
  \begin{Phonetics}{资料}{zi1liao4}[][HSK 4]
    \definition[份,堆,本,个]{s.}{dados; material; material informativo para referência ou para ser considerado confiável | material de produção; meios de subsistência; requisitos de produção ou subsistência}
  \end{Phonetics}
\end{Entry}

\begin{Entry}{资格}{10,10}{⾙、⽊}
  \begin{Phonetics}{资格}{zi1ge2}[][HSK 3]
    \definition{s.}{qualificação; condições e identidades necessárias para exercer uma determinada atividade | senioridade; identidade formada pelo tempo dedicado a um determinado trabalho ou atividade}
  \end{Phonetics}
\end{Entry}

\begin{Entry}{资源}{10,13}{⾙、⽔}
  \begin{Phonetics}{资源}{zi1yuan2}[][HSK 4]
    \definition{s.}{recurso; fontes naturais de meios de produção ou subsistência}
  \end{Phonetics}
\end{Entry}

\begin{Entry}{赅}{10}{⾙}
  \begin{Phonetics}{赅}{gai1}
    \definition*{s.}{Sobrenome Gai}
    \definition{adj.}{completo; integral; abrangente; inclusivo}
  \end{Phonetics}
\end{Entry}

\begin{Entry}{赶}{10}{⾛}
  \begin{Phonetics}{赶}{gan3}[][HSK 3]
    \definition*{s.}{Sobrenome Gan}
    \definition{prep.}{por; até; até que; até quando; introduzir o momento em que algo aconteceu, indicando que se espera até um determinado momento}
    \definition{v.}{ultrapassar; alcançar | perseguir; correr atrás; tentar alcançar; dar uma corrida; acelerar ou intensificar  | dirigir; conduzir | expulsar; afugentar; afastar | encontrar; deparar-se com; esbarrar em; acontecer; encontrar-se em (uma situação); aproveitar-se de (uma oportunidade) | ir para; participar (atividades com horário marcado)}
  \end{Phonetics}
\end{Entry}

\begin{Entry}{赶上}{10,3}{⾛、⼀}
  \begin{Phonetics}{赶上}{gan3 shang4}[][HSK 6]
    \definition{v.}{alcançar; manter o ritmo com; acompanhar alguém ou o padrão do planejador | chegar a tempo para; ter tempo suficiente; não ser tarde demais | encontrar; topar com; cruzar com; encontrar-se com; acontecer de encontrar; encontrar algo, em um determinado momento ou oportunidade}
  \end{Phonetics}
\end{Entry}

\begin{Entry}{赶不上}{10,4,3}{⾛、⼀、⼀}
  \begin{Phonetics}{赶不上}{gan3 bu5 shang4}[][HSK 6]
    \definition{v.}{ficar para trás; ser incapaz de alcançar; não conseguir alcançar; não conseguir acompanhar | ser tarde demais (para fazer algo); (não) existir tempo suficiente (para fazer algo) |  deixar de ter; ser incapaz de encontrar ou ter a chance de encontrar; não encontrar; não encontrar (boa oportunidade) | não poder ser comparado a}
  \end{Phonetics}
\end{Entry}

\begin{Entry}{赶忙}{10,6}{⾛、⼼}
  \begin{Phonetics}{赶忙}{gan3 mang2}[][HSK 6]
    \definition{adv.}{imediatamente; com pressa; às pressas; rapidamente}
  \end{Phonetics}
\end{Entry}

\begin{Entry}{赶早}{10,6}{⾛、⽇}
  \begin{Phonetics}{赶早}{gan3zao3}
    \definition{adv.}{o mais breve possível | na primeira oportunidade | antes que seja tarde | quanto antes melhor}
  \end{Phonetics}
\end{Entry}

\begin{Entry}{赶快}{10,7}{⾛、⼼}
  \begin{Phonetics}{赶快}{gan3kuai4}[][HSK 3]
    \definition{adv.}{rapidamente; imediatamente; aproveite o momento e acelere o ritmo}
  \end{Phonetics}
\end{Entry}

\begin{Entry}{赶走}{10,7}{⾛、⾛}
  \begin{Phonetics}{赶走}{gan3zou3}
    \definition{v.}{expulsar | voltar atrás}
  \end{Phonetics}
\end{Entry}

\begin{Entry}{赶到}{10,8}{⾛、⼑}
  \begin{Phonetics}{赶到}{gan3 dao4}[][HSK 3]
    \definition{v.}{correr (para algum lugar); apressar-se}
  \end{Phonetics}
\end{Entry}

\begin{Entry}{赶往}{10,8}{⾛、⼻}
  \begin{Phonetics}{赶往}{gan3wang3}[][HSK 7-9]
    \definition{v.}{apressar-se para (algum lugar)}
  \end{Phonetics}
\end{Entry}

\begin{Entry}{赶赴}{10,9}{⾛、⾛}
  \begin{Phonetics}{赶赴}{gan3fu4}[][HSK 7-9]
    \definition{v.}{apressar-se para; correr para | apressar-se}
  \end{Phonetics}
\end{Entry}

\begin{Entry}{赶紧}{10,10}{⾛、⽷}
  \begin{Phonetics}{赶紧}{gan3jin3}[][HSK 3]
    \definition{adv.}{apressadamente; precipitadamente; às pressas; significa agir imediatamente, sem demora}
  \end{Phonetics}
\end{Entry}

\begin{Entry}{赶脚}{10,11}{⾛、⾁}
  \begin{Phonetics}{赶脚}{gan3jiao3}
    \definition{v.}{transportar mercadorias para ganhar a vida (especialmente de burro) | trabalhar como carroceiro ou porteiro}
  \end{Phonetics}
\end{Entry}

\begin{Entry}{赶跑}{10,12}{⾛、⾜}
  \begin{Phonetics}{赶跑}{gan3pao3}
    \definition{v.}{afastar | forçar a saída | repelir}
  \end{Phonetics}
\end{Entry}

\begin{Entry}{赶集}{10,12}{⾛、⾫}
  \begin{Phonetics}{赶集}{gan3ji2}
    \definition{v.}{ir a uma feira | ir ao mercado}
  \end{Phonetics}
\end{Entry}

\begin{Entry}{赶路}{10,13}{⾛、⾜}
  \begin{Phonetics}{赶路}{gan3lu4}
    \definition{v.}{apressar a jornada | apressar-se}
  \end{Phonetics}
\end{Entry}

\begin{Entry}{起}{10}{⾛}
  \begin{Phonetics}{起}{qi3}[][HSK 1]
    \definition{clas.}{caso; instância | lote; grupo}
    \definition{prep.}{de; colocado antes de uma palavra de tempo ou lugar, indica um ponto de partida | por; colocado antes de uma palavra de lugar, indica um lugar por onde passou}
    \definition{v.}{levantar-se; ficar de pé| iniciar; lançar; deicar a posição original | subir; ascender | aparecer; levantar; crescer (bolhas, protuberâncias, brotoeja) | puxar para cima; puxar para fora; tirar o que está guardado ou incorporado | crescer; aumentar | esboçar; elaborar | construir; montar; estabelecer | receber (comprovante) | começar; iniciar; combina com 从 e 由; indica quando, onde e quem começou | buscar; pegar; usado após um verbo, indica movimento para cima | indicar se alguém tem força suficiente ou não; usado após um verbo, indica que a força é suficiente ou insuficiente | indicar que a ação envolve alguém ou algo; equivalente a 及 ou 到 | começar; iniciar; usado depois de um verbo, indica o início de uma ação | juntar; implodir; (informal) usado depois de um verbo, para unir coisas ou fechá-las}
  \seealsoref{从}{cong2}
  \seealsoref{到}{dao4}
  \seealsoref{及}{ji2}
  \seealsoref{由}{you2}
  \end{Phonetics}
\end{Entry}

\begin{Entry}{起飞}{10,3}{⾛、⾶}
  \begin{Phonetics}{起飞}{qi3fei1}[][HSK 2]
    \definition{v.}{decolar; levantar voo | crescer rapidamente; decolar; disparar; metáfora para o rápido desenvolvimento de negócios, economia, etc.}
  \end{Phonetics}
\end{Entry}

\begin{Entry}{起床}{10,7}{⾛、⼴}
  \begin{Phonetics}{起床}{qi3/chuang2}[][HSK 1]
    \definition{v.+compl.}{levantar-se; sair da cama; acordar e sair da cama (geralmente pela manhã); levantar-se da posição sentada, deitada ou deitada de bruços, ou sentar-se a partir da posição deitada}
  \end{Phonetics}
\end{Entry}

\begin{Entry}{起来}{10,7}{⾛、⽊}
  \begin{Phonetics}{起来}{qi3/lai2}[][HSK 1]
    \definition{v.+compl.}{levantar-se; passar de posições como deitado, sentado ou ajoelhado para ficar em pé | levantar-se; sair da cama | levantar-se; revoltar-se; rebelar-se; refere-se a ascensão, surgimento, levantamento, etc.}
  \end{Phonetics}
  \begin{Phonetics}{起来}{qi3lai5}
    \definition{v.aux.}{usado depois de um verbo para indicar movimento ascendente}
  \end{Phonetics}
  \begin{Phonetics}{起来}{qi5lai2}
    \definition{v.}{descrever resultados, retratar comportamentos, transmitir movimento}
  \end{Phonetics}
\end{Entry}

\begin{Entry}{起诉}{10,7}{⾛、⾔}
  \begin{Phonetics}{起诉}{qi3 su4}[][HSK 6]
    \definition{v.}{processar; entrar com uma ação judicial}
  \end{Phonetics}
\end{Entry}

\begin{Entry}{起到}{10,8}{⾛、⼑}
  \begin{Phonetics}{起到}{qi3 dao4}[][HSK 5]
    \definition{v.}{ter (um efeito motivador, etc.); desempenhar (um papel estabilizador, etc.)}
  \end{Phonetics}
\end{Entry}

\begin{Entry}{起码}{10,8}{⾛、⽯}
  \begin{Phonetics}{起码}{qi3ma3}[][HSK 5]
    \definition{adj.}{mínimo; elementar; rudimentar}
    \definition{adv.}{mínimamente; pelo menos;}
  \end{Phonetics}
\end{Entry}

\begin{Entry}{起点}{10,9}{⾛、⽕}
  \begin{Phonetics}{起点}{qi3 dian3}[][HSK 6]
    \definition[个]{s.}{ponto de partida (para o tempo ou local do início de algo); o lugar ou hora de início | ponto de partida (para o nível ou base de algo feito inicialmente); refere-se especificamente ao ponto de partida designado em um evento de pista}
  \end{Phonetics}
\end{Entry}

\begin{Entry}{起跳}{10,13}{⾛、⾜}
  \begin{Phonetics}{起跳}{qi3tiao4}
    \definition{v.}{(atletismo) decolar (no início de um salto) | (de preço, salário, etc.) começar (de um determinado nível)}
  \end{Phonetics}
\end{Entry}

\begin{Entry}{较}{10}{⾞}
  \begin{Phonetics}{较}{jiao4}[][HSK 3]
    \definition{adj.}{claro; óbvio; evidente}
    \definition{adv.}{comparativamente; relativamente; razoavelmente; bastante; bastante}
    \definition{prep.}{usado para comparar características e graus; introduzir o objeto de comparação; equivalente a 比}
    \definition{v.}{comparar | disputar}
  \seealsoref{比}{bi3}
  \end{Phonetics}
\end{Entry}

\begin{Entry}{辱}{10}{⾠}
  \begin{Phonetics}{辱}{ru3}
    \definition*{s.}{Sobrenome Ru}
    \definition{s.}{desgraça; desonra (oposto a 荣)}
    \definition{v.}{trazer desgraça (ou humilhação) para | trazer desgraça; ser uma desgraça para | estar em dívida (com alguém por uma gentileza) | humilhar; insultar}
  \seealsoref{荣}{rong2}
  \end{Phonetics}
\end{Entry}

\begin{Entry}{辱骂}{10,9}{⾠、⾺}
  \begin{Phonetics}{辱骂}{ru3ma4}
    \definition{v.}{insultar | abusar}
  \end{Phonetics}
\end{Entry}

\begin{Entry}{透}{10}{⾡}
  \begin{Phonetics}{透}{tou4}[][HSK 4]
    \definition{adv.}{totalmente; completamente; minuciosamente | profundamente; extremamente}
    \definition{v.}{penetrar; passar através de; infiltrar-se através de | revelar; deixar transparecer; contar secretamente |mostrar; aparecer}
  \end{Phonetics}
\end{Entry}

\begin{Entry}{透支}{10,4}{⾡、⽀}
  \begin{Phonetics}{透支}{tou4zhi1}
    \definition{v.}{cheque especial (bancário) | saque a descoberto}
  \end{Phonetics}
\end{Entry}

\begin{Entry}{透气}{10,4}{⾡、⽓}
  \begin{Phonetics}{透气}{tou4qi4}
    \definition{v.}{respirar (sobre tecido, etc.) | fluir livremente (sobre ar) | respirar ar fresco | ventilar}
  \end{Phonetics}
\end{Entry}

\begin{Entry}{透水}{10,4}{⾡、⽔}
  \begin{Phonetics}{透水}{tou4shui3}
    \definition{adj.}{permeável}
    \definition{s.}{vazamento de água}
  \end{Phonetics}
\end{Entry}

\begin{Entry}{透过}{10,6}{⾡、⾡}
  \begin{Phonetics}{透过}{tou4guo4}
    \definition{v.}{passar através | penetrar}
  \end{Phonetics}
\end{Entry}

\begin{Entry}{透彻}{10,7}{⾡、⼻}
  \begin{Phonetics}{透彻}{tou4che4}
    \definition{adj.}{minucioso | incisivo | penetrante}
  \end{Phonetics}
\end{Entry}

\begin{Entry}{透明}{10,8}{⾡、⽇}
  \begin{Phonetics}{透明}{tou4ming2}[][HSK 4]
    \definition{adj.}{transparente; diáfano; capaz de transmitir luz | evidente; transparente; situação ou assunto que seja aberto e não oculto | transparente; diáfano; indica pureza, ausência de impurezas}
  \end{Phonetics}
\end{Entry}

\begin{Entry}{透顶}{10,8}{⾡、⾴}
  \begin{Phonetics}{透顶}{tou4ding3}
    \definition{adv.}{completamente}
  \end{Phonetics}
\end{Entry}

\begin{Entry}{透亮}{10,9}{⾡、⼇}
  \begin{Phonetics}{透亮}{tou4liang4}
    \definition{adj.}{brilhante | claro como cristal}
  \end{Phonetics}
\end{Entry}

\begin{Entry}{透辟}{10,13}{⾡、⾟}
  \begin{Phonetics}{透辟}{tou4pi4}
    \definition{adj.}{incisivo | penetrante}
  \end{Phonetics}
\end{Entry}

\begin{Entry}{透澈}{10,15}{⾡、⽔}
  \begin{Phonetics}{透澈}{tou4che4}
    \variantof{透彻}
  \end{Phonetics}
\end{Entry}

\begin{Entry}{透露}{10,21}{⾡、⾬}
  \begin{Phonetics}{透露}{tou4lu4}[][HSK 6]
    \definition{v.}{vazar; revelar; expor; divulgar; contar deliberadamente um segredo a alguém; revelar um certo significado}
  \end{Phonetics}
\end{Entry}

\begin{Entry}{逐}{10}{⾡}
  \begin{Phonetics}{逐}{zhu2}
    \definition{prep.}{um por um; um a um}[逐月===mês a mês]
    \definition{v.}{ir atrás de; perseguir | expulsar; banir | correr atrás; alcançar}
  \end{Phonetics}
\end{Entry}

\begin{Entry}{逐步}{10,7}{⾡、⽌}
  \begin{Phonetics}{逐步}{zhu2bu4}[][HSK 4]
    \definition{adv.}{gradualmente; passo a passo; progressivamente}
  \end{Phonetics}
\end{Entry}

\begin{Entry}{逐渐}{10,11}{⾡、⽔}
  \begin{Phonetics}{逐渐}{zhu2jian4}[][HSK 4]
    \definition{adv.}{gradualmente; aos poucos; por etapas; indica mudanças lentas e ordenadas no grau, na quantidade, etc.}
  \end{Phonetics}
\end{Entry}

\begin{Entry}{递}{10}{⾡}
  \begin{Phonetics}{递}{di4}[][HSK 5]
    \definition{adv.}{na ordem correta; sucessivamente}
    \definition{v.}{entregar; passar; dar; transmitir}
  \end{Phonetics}
\end{Entry}

\begin{Entry}{递交}{10,6}{⾡、⼇}
  \begin{Phonetics}{递交}{di4jiao1}[][HSK 7-9]
    \definition{v.}{apresentar; submeter; entregar; entregar pessoalmente}
  \end{Phonetics}
\end{Entry}

\begin{Entry}{递给}{10,9}{⾡、⽷}
  \begin{Phonetics}{递给}{di4 gei3}[][HSK 5]
    \definition{v.}{entregar algo a alguém; passar itens ou coisas para outras pessoas}
  \end{Phonetics}
\end{Entry}

\begin{Entry}{途}{10}{⾡}
  \begin{Phonetics}{途}{tu2}
    \definition[条]{s.}{caminho; estrada; rota | jornada; caminho}
  \end{Phonetics}
\end{Entry}

\begin{Entry}{途中}{10,4}{⾡、⼁}
  \begin{Phonetics}{途中}{tu2 zhong1}[][HSK 4]
    \definition[家]{adv.}{no caminho; ao longo do caminho}
  \end{Phonetics}
\end{Entry}

\begin{Entry}{途径}{10,8}{⾡、⼻}
  \begin{Phonetics}{途径}{tu2jing4}[][HSK 6]
    \definition[种,条,个]{s.}{caminho; canal; metaforicamente falando, uma maneira ou método de resolver um problema ou fazer algo}
  \end{Phonetics}
\end{Entry}

\begin{Entry}{逗}{10}{⾡}
  \begin{Phonetics}{逗}{dou4}[][HSK 7-9]
    \definition{adj.}{engraçado; divertido}
    \definition{s.}{ligeira pausa na leitura; antigamente, referia-se ao lugar em um artigo onde o significado de uma frase não era completado e uma pausa era necessária durante a leitura}
    \definition{v.}{provocar; brincar com | divertir; provocar (risos, etc.) | ficar; parar}
  \end{Phonetics}
\end{Entry}

\begin{Entry}{通}{10}{⾡}
  \begin{Phonetics}{通}{tong1}[][HSK 2]
    \definition*{s.}{Sobrenome Tong}
    \definition{adj.}{lógico; coerente | geral; comum | tudo; inteiro | aberto; através de | total}
    \definition{clas.}{(antigo) usado para cartas, telegramas, documentos oficiais, etc.}
    \definition{s.}{autoridade; especialista}
    \definition{suf.}{especialista}
    \definition{v.}{abrir; atravessar | abrir ou limpar cutucando ou espetando | levar a; ir a | conectar; comunicar | notificar; informar | compreender; saber | cutucar; dar uma pancada | transmitir; conectar; interagir | dominar; compreender; entender}
  \end{Phonetics}
  \begin{Phonetics}{通}{tong4}
    \definition{clas.}{usado para uma atividade, tomada em sua totalidade (discurso de abuso, período de reprodução de música, bebedeira, etc.)}
  \end{Phonetics}
\end{Entry}

\begin{Entry}{通用}{10,5}{⾡、⽤}
  \begin{Phonetics}{通用}{tong1yong4}[][HSK 5]
    \definition[家]{adj.}{de uso comum; universal; (em um determinado âmbito) de uso generalizado | intercambiável; alguns caracteres chineses com grafia diferente, mas pronúncia igual, podem ser usados indistintamente (alguns limitados a um determinado significado)}
  \end{Phonetics}
\end{Entry}

\begin{Entry}{通讯}{10,5}{⾡、⾔}
  \begin{Phonetics}{通讯}{tong1xun4}[][HSK 6]
    \definition[个,种]{s.}{relatório; comunicação; boletim informativo; correspondência; reportagem; despacho de notícias; artigos que relatam fatos objetivos ou números típicos de forma detalhada e vívida}
    \definition{v.}{usar equipamentos de telecomunicações para transmitir mensagens}
  \end{Phonetics}
\end{Entry}

\begin{Entry}{通红}{10,6}{⾡、⽷}
  \begin{Phonetics}{通红}{tong1 hong2}[][HSK 6]
    \definition{adj.}{muito vermelho; vermelho por completo}
  \end{Phonetics}
\end{Entry}

\begin{Entry}{通行}{10,6}{⾡、⾏}
  \begin{Phonetics}{通行}{tong1 xing2}[][HSK 6]
    \definition{adj.}{atual; geral}
    \definition{v.}{passar (ou ir) através; passar por; atravessar | prevalecer; predominar; ser corrente | (pedestres, veículos, etc.) passar na linha de trânsito}
  \end{Phonetics}
\end{Entry}

\begin{Entry}{通观}{10,6}{⾡、⾒}
  \begin{Phonetics}{通观}{tong1guan1}
    \definition{v.}{ter uma visão geral de algo}
  \end{Phonetics}
\end{Entry}

\begin{Entry}{通过}{10,6}{⾡、⾡}
  \begin{Phonetics}{通过}{tong1guo4}[][HSK 2]
    \definition{prep.}{por; através de; por meio de; por meio de; meios, métodos, etc. para introduzir ações}
    \definition{v.}{atravessar; passar por; transitar | aprovar; adotar | solicitar o consentimento ou aprovação de}
  \end{Phonetics}
\end{Entry}

\begin{Entry}{通报}{10,7}{⾡、⼿}
  \begin{Phonetics}{通报}{tong1 bao4}[][HSK 6]
    \definition[份]{s.}{circular | boletim; jornal; publicação | sumário; notificação para informações gerais}
    \definition{v.}{circular um aviso (aviso por escrito) | notificar; dar informações com; compartilhar informações com}
  \end{Phonetics}
\end{Entry}

\begin{Entry}{通识}{10,7}{⾡、⾔}
  \begin{Phonetics}{通识}{tong1shi2}
    \definition{s.}{conhecimento comum | erudição | conhecimento geral | amplamente conhecido}
  \end{Phonetics}
\end{Entry}

\begin{Entry}{通知}{10,8}{⾡、⽮}
  \begin{Phonetics}{通知}{tong1zhi1}[][HSK 2]
    \definition[份,个,张]{s.}{aviso; circular; notificação por escrito ou verbal}
    \definition{v.}{aconselhar; notificar; informar; dar aviso prévio}
  \end{Phonetics}
\end{Entry}

\begin{Entry}{通知书}{10,8,4}{⾡、⽮、⼄}
  \begin{Phonetics}{通知书}{tong1 zhi1 shu1}[][HSK 4]
    \definition[份]{s.}{aviso; observação; notificação}
  \end{Phonetics}
\end{Entry}

\begin{Entry}{通话}{10,8}{⾡、⾔}
  \begin{Phonetics}{通话}{tong1 hua4}[][HSK 6]
    \definition{v.}{comunicar por telefone | conversar; comunicar; falar em uma língua que ambos possam entender}
  \end{Phonetics}
\end{Entry}

\begin{Entry}{通信}{10,9}{⾡、⼈}
  \begin{Phonetics}{通信}{tong1/xin4}[][HSK 3]
    \definition{v.+compl.}{corresponder; comunicar por carta; comunicar situações e informações escrevendo cartas | transmitir (ou transportar) mensagem; passar (ou transmitir) informação; usar ondas de rádio e outros sinais para transmitir texto, imagens, etc.}
  \end{Phonetics}
\end{Entry}

\begin{Entry}{通常}{10,11}{⾡、⼱}
  \begin{Phonetics}{通常}{tong1chang2}[][HSK 3]
    \definition{adj.}{usual; normal; geral}
    \definition{adv.}{habitualmente; usualmente; geralmente; ordinariamente}
  \end{Phonetics}
\end{Entry}

\begin{Entry}{通道}{10,12}{⾡、⾡}
  \begin{Phonetics}{通道}{tong1 dao4}[][HSK 6]
    \definition[条,个]{s.}{acesso; corredor; passagem; caminhos que levam ao exterior de teatros, minas, etc. | passagem; via pública}
  \end{Phonetics}
\end{Entry}

\begin{Entry}{通牒}{10,13}{⾡、⽚}
  \begin{Phonetics}{通牒}{tong1die2}
    \definition{s.}{nota diplomática}
  \end{Phonetics}
\end{Entry}

\begin{Entry}{逛}{10}{⾡}
  \begin{Phonetics}{逛}{guang4}[][HSK 4]
    \definition{v.}{perambular; passear; vaguear}
  \end{Phonetics}
\end{Entry}

\begin{Entry}{逞}{10}{⾡}
  \begin{Phonetics}{逞}{cheng3}
    \definition*{s.}{Sobrenome Cheng}
    \definition{v.}{exibir-se; ostentar; gabar-se | executar (um plano maligno); ter sucesso (em um esquema) | saciar; satisfazer; dar rédea solta a; deliciar-se}
  \end{Phonetics}
\end{Entry}

\begin{Entry}{逞能}{10,10}{⾡、⾁}
  \begin{Phonetics}{逞能}{cheng3/neng2}[][HSK 7-9]
    \definition{v.+compl.}{exibir a própria habilidade (ou capacidade); exibir a própria capacidade | mostrar sua habilidade ou capacidade}
  \end{Phonetics}
\end{Entry}

\begin{Entry}{逞强}{10,12}{⾡、⼸}
  \begin{Phonetics}{逞强}{cheng3/qiang2}[][HSK 7-9]
    \definition{v.+compl.}{exibir-se; ser orgulhoso; ser teimoso; ostentar a própria superioridade}
  \end{Phonetics}
\end{Entry}

\begin{Entry}{速}{10}{⾡}
  \begin{Phonetics}{速}{su4}
    \definition{adj.}{rápido; veloz}
    \definition{s.}{velocidade}
    \definition{v.aux.}{convidar}
  \end{Phonetics}
\end{Entry}

\begin{Entry}{速度}{10,9}{⾡、⼴}
  \begin{Phonetics}{速度}{su4du4}[][HSK 3]
    \definition[个,种]{s.}{velocidade; taxa; ritmo; andamento; uma quantidade física que indica a velocidade e a direção do movimento de um objeto, ou seja, a distância que um objeto percorre em uma direção por unidade de tempo | velocidade; rapidez; geralmente se refere ao grau de velocidade}
  \end{Phonetics}
\end{Entry}

\begin{Entry}{造}{10}{⾡}
  \begin{Phonetics}{造}{zao4}[][HSK 3]
    \definition*{s.}{Sobrenome Zao}
    \definition{clas.}{para colheitas ou número de colheitas de safras}
    \definition{s.}{uma das duas partes em um acordo legal ou um processo judicial | (dialeto) colheita; safra | realizações; conquistas |}
    \definition{v.}{fazer; construir; criar; produzir | forjar; inventar | correr solto; bagunçar as coisas | expor sem restrições |  treinar; educar | fabricar | alcançar; atingir}
  \end{Phonetics}
\end{Entry}

\begin{Entry}{造成}{10,6}{⾡、⼽}
  \begin{Phonetics}{造成}{zao4cheng2}[][HSK 3]
    \definition{v.}{criar; dar origem a; provocar; causar (geralmente se refere a resultados negativos)}
  \end{Phonetics}
\end{Entry}

\begin{Entry}{造型}{10,9}{⾡、⼟}
  \begin{Phonetics}{造型}{zao4xing2}[][HSK 4]
    \definition[个,种]{s.}{molde; modelo; formato; forma; moldagem}
    \definition{v.}{modelar; moldar}
  \end{Phonetics}
\end{Entry}

\begin{Entry}{逢}{10}{⾡}
  \begin{Phonetics}{逢}{feng2}[][HSK 7-9]
    \definition*{s.}{Sobrenome Feng}
    \definition{v.}{encontrar; vir até; encontrar-se por acaso}
  \end{Phonetics}
\end{Entry}

\begin{Entry}{部}{10}{⾢}
  \begin{Phonetics}{部}{bu4}[][HSK 3]
    \definition*{s.}{Sobrenome Bu}
    \definition{clas.}{usado para obras de literatura, livros, filmes, etc.}
    \definition[根]{s.}{parte; seção | unidade; ministério; departamento; conselho | sede; matriz; quartel general | tropas; forças | divisão; região}
    \definition{v.}{comandar; liderar}
  \end{Phonetics}
\end{Entry}

\begin{Entry}{部下}{10,3}{⾢、⼀}
  \begin{Phonetics}{部下}{bu4xia4}
    \definition{s.}{subordinado | tropas sob comando de alguém}
  \end{Phonetics}
\end{Entry}

\begin{Entry}{部门}{10,3}{⾢、⾨}
  \begin{Phonetics}{部门}{bu4men2}[][HSK 3]
    \definition[个]{s.}{departamento; ramo; classe; seção; partes ou unidades que compõem um todo}
  \end{Phonetics}
\end{Entry}

\begin{Entry}{部分}{10,4}{⾢、⼑}
  \begin{Phonetics}{部分}{bu4fen5}[][HSK 2]
    \definition[个,些,快,份]{s.}{parte; seção; porção; parte do todo; alguns indivíduos dentro do todo | ramo; parte separada de um sistema ou entidade}
  \end{Phonetics}
\end{Entry}

\begin{Entry}{部长}{10,4}{⾢、⾧}
  \begin{Phonetics}{部长}{bu4 zhang3}[][HSK 3]
    \definition[个,位,名]{s.}{ministro; chefe de departamento; um alto funcionário do estado encarregado pelo chefe de estado ou chefe executivo do governo da gestão das atividades governamentais de um departamento | chefe de seção; líder tribal}
  \end{Phonetics}
\end{Entry}

\begin{Entry}{部队}{10,4}{⾢、⾩}
  \begin{Phonetics}{部队}{bu4 dui4}[][HSK 6]
    \definition[支,个]{s.}{militar; exército; forças armadas | tropas; refere-se a uma parte do exército}
  \end{Phonetics}
\end{Entry}

\begin{Entry}{部件}{10,6}{⾢、⼈}
  \begin{Phonetics}{部件}{bu4jian4}[][HSK 7-9]
    \definition[个]{s.}{peças; partes; componentes; um componente de uma máquina, montado a partir de várias partes | partes; componentes (para caracteres chineses); uma unidade de caracteres chineses composta por traços, por exemplo, 氵, 礻, 口 são todos componentes de caracteres chineses}
  \end{Phonetics}
\end{Entry}

\begin{Entry}{部位}{10,7}{⾢、⼈}
  \begin{Phonetics}{部位}{bu4wei4}[][HSK 5]
    \definition{s.}{lugar; posição (usado principalmente para o corpo humano)}
  \end{Phonetics}
\end{Entry}

\begin{Entry}{部族}{10,11}{⾢、⽅}
  \begin{Phonetics}{部族}{bu4zu2}
    \definition{adj.}{tribal}
    \definition{s.}{tribo}
  \end{Phonetics}
\end{Entry}

\begin{Entry}{部属}{10,12}{⾢、⼫}
  \begin{Phonetics}{部属}{bu4shu3}
    \definition{s.}{afiliado a um ministério | subordinado | tropas sob comando de alguém}
  \end{Phonetics}
\end{Entry}

\begin{Entry}{部署}{10,13}{⾢、⽹}
  \begin{Phonetics}{部署}{bu4shu3}[][HSK 7-9]
    \definition{v.}{organizar; implantar; dispor; organizar ou dispor de maneira planejada (usado principalmente em grandes aspectos)}
  \end{Phonetics}
\end{Entry}

\begin{Entry}{都}{10}{⾢}
  \begin{Phonetics}{都}{dou1}[][HSK 1]
    \definition{adv.}{todos; representa a soma total | apenas por causa de; usado em conjunto com a palavra 是, explica o motivo | mesmo; até; indicativo de ênfase | já; significa 已经}
  \seealsoref{是}{shi4}
  \seealsoref{已经}{yi3jing1}
  \end{Phonetics}
  \begin{Phonetics}{都}{du1}
    \definition*{s.}{Sobrenome Du}
    \definition[座]{s.}{capital | cidade grande; metrópole}
  \end{Phonetics}
\end{Entry}

\begin{Entry}{都市}{10,5}{⾢、⼱}
  \begin{Phonetics}{都市}{du1 shi4}[][HSK 6]
    \definition[个]{s.}{cidade grande; grandes cidades}
  \end{Phonetics}
\end{Entry}

\begin{Entry}{都会}{10,6}{⾢、⼈}
  \begin{Phonetics}{都会}{du1hui4}[][HSK 7-9]
    \definition{s.}{cidade; metrópole}
  \end{Phonetics}
\end{Entry}

\begin{Entry}{配}{10}{⾣}
  \begin{Phonetics}{配}{pei4}[][HSK 3]
    \definition{adj.}{adequado; bem combinado}
    \definition{s.}{cônjuge (geralmente referindo-se a uma esposa)}
    \definition{v.}{unir-se em matrimônio | (animais) acasalar; copular | compor; combinar; mesclar; amalgamar; misturar | distribuir de forma planejada; repartir | encontrar algo para encaixar ou substituir outra coisa; compensar as partes faltantes de acordo com certos padrões | combinar; harmonizar com; estar em harmonia com | exilar; banir; nos tempos antigos, referia-se ao exílio de criminosos}
    \definition{v.aux.}{adequar-se a; merecer; ser qualificado; ser digno de}
  \end{Phonetics}
\end{Entry}

\begin{Entry}{配合}{10,6}{⾣、⼝}
  \begin{Phonetics}{配合}{pei4he2}[][HSK 3]
    \definition{v.}{cooperar; coordenar; todas as partes trabalham juntas para concluir tarefas comuns}
  \end{Phonetics}
\end{Entry}

\begin{Entry}{配备}{10,8}{⾣、⼡}
  \begin{Phonetics}{配备}{pei4bei4}[][HSK 5]
    \definition{s.}{equipamento; material; conjunto completo de utensílios, etc.}
    \definition{v.}{fornecer; alocar; equipar; distribuir conforme necessário | posicionar; dispor (tropas, etc.)}
  \end{Phonetics}
\end{Entry}

\begin{Entry}{配套}{10,10}{⾣、⼤}
  \begin{Phonetics}{配套}{pei4/tao4}[][HSK 5]
    \definition{v.+compl.}{formar um conjunto ou sistema completo; combinar vários elementos relacionados em um conjunto completo}
  \end{Phonetics}
\end{Entry}

\begin{Entry}{配置}{10,13}{⾣、⽹}
  \begin{Phonetics}{配置}{pei4 zhi4}[][HSK 6]
    \definition{s.}{configuração; refere-se especificamente à seleção e combinação de software e hardware em várias partes de computadores, carros, etc.}
    \definition{v.}{implantar; alocar; dispor (tropas, etc.); equipar e configurar}
  \end{Phonetics}
\end{Entry}

\begin{Entry}{酒}{10}{⾣}
  \begin{Phonetics}{酒}{jiu3}[][HSK 2]
    \definition*{s.}{Sobrenome Jiu}
    \definition[口,杯,瓶,罐,桶,缸]{s.}{bebida alcoólica; vinho; licor; bebidas destiladas}
  \end{Phonetics}
\end{Entry}

\begin{Entry}{酒水}{10,4}{⾣、⽔}
  \begin{Phonetics}{酒水}{jiu3 shui3}[][HSK 6]
    \definition{s.}{bebidas; bebidas e álcool | Dialeto: festa; banquete}
  \end{Phonetics}
\end{Entry}

\begin{Entry}{酒吧}{10,7}{⾣、⼝}
  \begin{Phonetics}{酒吧}{jiu3ba1}[][HSK 4]
    \definition[家,间]{s.}{bar; \emph{pub}; um local onde são vendidas bebidas alcoólicas e onde as pessoas podem beber e conversar, referindo-se principalmente a um restaurante ou hotel de estilo ocidental especializado na venda de bebidas alcoólicas.}
  \end{Phonetics}
\end{Entry}

\begin{Entry}{酒店}{10,8}{⾣、⼴}
  \begin{Phonetics}{酒店}{jiu3 dian4}[][HSK 2]
    \definition[家,个]{s.}{hotel; Estabelecimento comercial que oferece hospedagem e alimentação aos hóspedes | restaurante}
  \end{Phonetics}
\end{Entry}

\begin{Entry}{酒鬼}{10,9}{⾣、⿁}
  \begin{Phonetics}{酒鬼}{jiu3gui3}[][HSK 5]
    \definition[个]{s.}{bebedor de vinho; beberrão; ébrio | alcoólatra}
  \end{Phonetics}
\end{Entry}

\begin{Entry}{酒馆}{10,11}{⾣、⾷}
  \begin{Phonetics}{酒馆}{jiu3guan3}
    \definition{s.}{bar | taverna | adega}
  \end{Phonetics}
\end{Entry}

\begin{Entry}{钱}{10}{⾦}
  \begin{Phonetics}{钱}{qian2}[][HSK 1]
    \definition*{s.}{Sobrenome Qian}
    \definition{clas.}{qian, uma unidade de peso (=5 gramas) | qian, uma unidade de peso (um décimo de um tael 两)}
    \definition[笔]{s.}{dinheiro; riqueza; bens | moeda de cobre; dinheiro | objeto em forma de moeda de cobre | fundo; montante | dinheiro guardado ou gasto para algum fim específico (geralmente se refere a quantias significativas de dinheiro que entram e saem de órgãos públicos, organizações, etc.)}
  \seealsoref{两}{liang3}
  \end{Phonetics}
\end{Entry}

\begin{Entry}{钱包}{10,5}{⾦、⼓}
  \begin{Phonetics}{钱包}{qian2 bao1}[][HSK 1]
    \definition[个]{s.}{carteira; bolsa; bolsa de dinheiro}
  \end{Phonetics}
\end{Entry}

\begin{Entry}{钻}{10}{⾦}
  \begin{Phonetics}{钻}{zuan1}
    \definition{v.}{furar; perfurar; girar um objeto pontiagudo para perfurar outro objeto | perfurar; entrar; penetrar; passar por | aprofundar-se; estudar intensivamente; fazer um estudo penetrante de | buscar ganho pessoal; tramar; refere-se a esquemas}
  \end{Phonetics}
  \begin{Phonetics}{钻}{zuan4}[][HSK 6]
    \definition[把]{s.}{broca; pua; sonda; existem muitos tipos de ferramentas para perfuração, incluindo manivela, elétrica e pneumática | joia; diamante}
    \definition{v.}{furar; perfurar;  girar um objeto pontiagudo para perfurar outro objeto}
  \end{Phonetics}
\end{Entry}

\begin{Entry}{钻石}{10,5}{⾦、⽯}
  \begin{Phonetics}{钻石}{zuan4shi2}
    \definition[颗]{s.}{diamante}
  \end{Phonetics}
\end{Entry}

\begin{Entry}{钻戒}{10,7}{⾦、⼽}
  \begin{Phonetics}{钻戒}{zuan4jie4}
    \definition[只]{s.}{anel de diamante}
  \end{Phonetics}
\end{Entry}

\begin{Entry}{钿}{10}{⾦}
  \begin{Phonetics}{钿}{dian4}
    \definition{s.}{ornamento incrustado antigo em forma de flor | enfeite de cabelo feminino com flores douradas | incrustação de madrepérola; um padrão incrustado com conchas de caracóis em madeira e laca}
    \definition{v.}{incrustar com ouro, prata, etc.}
  \end{Phonetics}
  \begin{Phonetics}{钿}{tian2}
    \definition{s.}{(dialeto) moeda | dinheiro; moeda | uma quantia de dinheiro}
  \end{Phonetics}
\end{Entry}

\begin{Entry}{铁}{10}{⾦}
  \begin{Phonetics}{铁}{tie3}[][HSK 3]
    \definition*{s.}{Sobrenome Tie}
    \definition{adj.}{duro; forte; sólido como ferro; metáfora para natureza dura; vontade forte | violento | inabalável; inalterável; determinado; metáfora para violência ou crueldade}
    \definition{s.}{ferro (Fe) | arma; armamento; refere-se a facas, armas de fogo, etc.}
    \definition{v.}{resolver; determinar}
  \end{Phonetics}
\end{Entry}

\begin{Entry}{铁轨}{10,6}{⾦、⾞}
  \begin{Phonetics}{铁轨}{tie3gui3}
    \definition[根]{s.}{trilho | trilho ferroviário}
  \end{Phonetics}
\end{Entry}

\begin{Entry}{铁路}{10,13}{⾦、⾜}
  \begin{Phonetics}{铁路}{tie3 lu4}[][HSK 3]
    \definition[条,公里]{s.}{ferrovia; estrada de ferro; uma estrada com trilhos de aço dispostos no leito da estrada para a circulação de trens}
  \end{Phonetics}
\end{Entry}

\begin{Entry}{铃}{10}{⾦}
  \begin{Phonetics}{铃}{ling2}[][HSK 5]
    \definition[串,个]{s.}{sino; instrumento musical feito de metal | objetos em forma de sino | cápsula; botão; broto}
  \end{Phonetics}
\end{Entry}

\begin{Entry}{铃声}{10,7}{⾦、⼠}
  \begin{Phonetics}{铃声}{ling2 sheng1}[][HSK 5]
    \definition{s.}{o tilintar de sinos; o som de um sino tocando}
  \end{Phonetics}
\end{Entry}

\begin{Entry}{铅}{10}{⾦}
  \begin{Phonetics}{铅}{qian1}
    \definition[根,盒]{s.}{chumbo (Pb) | grafite (em um lápis); grafite preta |}
  \end{Phonetics}
\end{Entry}

\begin{Entry}{铅笔}{10,10}{⾦、⽵}
  \begin{Phonetics}{铅笔}{qian1bi3}[][HSK 6]
    \definition[支,盒,种,枝,杆]{s.}{lápis; canetas com pontas de grafite ou argila pigmentada}
  \end{Phonetics}
\end{Entry}

\begin{Entry}{阅}{10}{⾨}
  \begin{Phonetics}{阅}{yue4}
    \definition{v.}{ler; repassar; examinar | revisar; inspecionar | experimentar; passar por}
  \end{Phonetics}
\end{Entry}

\begin{Entry}{阅兵式}{10,7,6}{⾨、⼋、⼷}
  \begin{Phonetics}{阅兵式}{yue4bing1shi4}
    \definition{s.}{parada militar; desfile militar}
  \end{Phonetics}
\end{Entry}

\begin{Entry}{阅览室}{10,9,9}{⾨、⾒、⼧}
  \begin{Phonetics}{阅览室}{yue4 lan3 shi4}[][HSK 5]
    \definition[个,间]{s.}{sala de leitura; a biblioteca dispõe de salas para leitura e pesquisa, equipadas com mesas e cadeiras adequadas, livros, jornais, revistas, etc.}
  \end{Phonetics}
\end{Entry}

\begin{Entry}{阅读}{10,10}{⾨、⾔}
  \begin{Phonetics}{阅读}{yue4du2}[][HSK 4]
    \definition{v.}{ler; examinar; olhar (livros, jornais, etc.) e entender seu conteúdo}
  \end{Phonetics}
\end{Entry}

\begin{Entry}{阅读广度}{10,10,3,9}{⾨、⾔、⼴、⼴}
  \begin{Phonetics}{阅读广度}{yue4du2guang3du4}
    \definition{s.}{intervalo de leitura}
  \end{Phonetics}
\end{Entry}

\begin{Entry}{阅读时间}{10,10,7,7}{⾨、⾔、⽇、⾨}
  \begin{Phonetics}{阅读时间}{yue4 du2 shi2 jian1}
    \definition{s.}{tempo de leitura}
  \end{Phonetics}
\end{Entry}

\begin{Entry}{阅读理解}{10,10,11,13}{⾨、⾔、⽟、⾓}
  \begin{Phonetics}{阅读理解}{yue4du2li3jie3}
    \definition{s.}{compreensão de leitura}
  \end{Phonetics}
\end{Entry}

\begin{Entry}{阅读装置}{10,10,12,13}{⾨、⾔、⾐、⽹}
  \begin{Phonetics}{阅读装置}{yue4du2zhuang1zhi4}
    \definition{s.}{dispositivo de leitura (por exemplo, para códigos de barras, etiquetas RFID, etc.)}
  \end{Phonetics}
\end{Entry}

\begin{Entry}{阅读障碍}{10,10,13,13}{⾨、⾔、⾩、⽯}
  \begin{Phonetics}{阅读障碍}{yue4du2zhang4ai4}
    \definition{s.}{dislexia}
  \end{Phonetics}
\end{Entry}

\begin{Entry}{阅读器}{10,10,16}{⾨、⾔、⼝}
  \begin{Phonetics}{阅读器}{yue4du2qi4}
    \definition{s.}{leitor (\emph{software})}
  \end{Phonetics}
\end{Entry}

\begin{Entry}{陪}{10}{⾩}
  \begin{Phonetics}{陪}{pei2}[][HSK 5]
    \definition{v.}{servir; acompanhar; cuidar; fazer companhia a alguém | auxiliar; ajudar}
  \end{Phonetics}
\end{Entry}

\begin{Entry}{陪同}{10,6}{⾩、⼝}
  \begin{Phonetics}{陪同}{pei2 tong2}[][HSK 6]
    \definition{v.}{acompanhar; acompanhar alguém para fazer uma atividade ou trabalhar junto}
  \end{Phonetics}
\end{Entry}

\begin{Entry}{陵}{10}{⾩}
  \begin{Phonetics}{陵}{ling2}
    \definition*{s.}{Sobrenome Ling}
    \definition{s.}{colina; monte | túmulo imperial; mausoléu}
    \definition{v.}{(literário) intimidar; violar}
  \end{Phonetics}
\end{Entry}

\begin{Entry}{陵园}{10,7}{⾩、⼞}
  \begin{Phonetics}{陵园}{ling2yuan2}
    \definition{s.}{cemitério}
  \end{Phonetics}
\end{Entry}

\begin{Entry}{陷}{10}{⾩}
  \begin{Phonetics}{陷}{xian4}
    \definition[个]{s.}{armadilha; cilada | defeito | deficiência; desvantagem}
    \definition{v.}{ficar preso (ou atolado); enredar | afundar; desabar | acusar falsamente; incriminar; armar | (de uma cidade, etc.) ser capturado; cair | ser enquadrado; ser capturado}
  \end{Phonetics}
\end{Entry}

\begin{Entry}{陷入}{10,2}{⾩、⼊}
  \begin{Phonetics}{陷入}{xian4ru4}[][HSK 6]
    \definition{v.}{afundar em; cair em; cair em uma situação desfavorável | estar perdido em; estar profundamente em; estar imerso em; metaforicamente, estar profundamente imerso em (uma situação ou pensamento) | estar atolado (lama fofa, areia, etc.)}
  \end{Phonetics}
\end{Entry}

\begin{Entry}{难}{10}{⾫}
  \begin{Phonetics}{难}{nan2}[][HSK 1]
    \definition{adj.}{difícil; duro; problemático (oposto a 易) | dificilmente possível; inevitável | ruim; desagradável | problemático; improvável}
    \definition{s.}{dificuldade}
    \definition{v.}{colocar alguém em uma situação difícil}
  \seealsoref{易}{yi4}
  \end{Phonetics}
  \begin{Phonetics}{难}{nan4}
    \definition{s.}{catástrofe; calamidade; desastre; adversidade; grande infortúnio}
    \definition{v.}{acusar; culpar}
  \end{Phonetics}
\end{Entry}

\begin{Entry}{难以}{10,4}{⾫、⼈}
  \begin{Phonetics}{难以}{nan2 yi3}[][HSK 5]
    \definition{adj.}{difícil; complicado}
  \end{Phonetics}
\end{Entry}

\begin{Entry}{难过}{10,6}{⾫、⾡}
  \begin{Phonetics}{难过}{nan2guo4}[][HSK 2]
    \definition{adj.}{triste; ruim; psicologicamente desconfortável | difícil; árduo}
  \end{Phonetics}
\end{Entry}

\begin{Entry}{难免}{10,7}{⾫、⼉}
  \begin{Phonetics}{难免}{nan4mian3}[][HSK 4]
    \definition{adj.}{inevitável; difícil de evitar}
  \end{Phonetics}
\end{Entry}

\begin{Entry}{难听}{10,7}{⾫、⼝}
  \begin{Phonetics}{难听}{nan2 ting1}[][HSK 2]
    \definition{adj.}{desagradável de ouvir | ofensivo; grosseiro; vulgar e desagradável | escandaloso; indigno}
  \end{Phonetics}
\end{Entry}

\begin{Entry}{难忘}{10,7}{⾫、⼼}
  \begin{Phonetics}{难忘}{nan2 wang4}[][HSK 6]
    \definition{adj.}{memorável; inesquecível}
  \end{Phonetics}
\end{Entry}

\begin{Entry}{难受}{10,8}{⾫、⼜}
  \begin{Phonetics}{难受}{nan2shou4}[][HSK 2]
    \definition{adj.}{sentir dor; sentir-se mal; sentir-se desconfortável | sentir-se mal; sentir-se infeliz; de mau humor; triste}
  \end{Phonetics}
\end{Entry}

\begin{Entry}{难度}{10,9}{⾫、⼴}
  \begin{Phonetics}{难度}{nan2 du4}[][HSK 3]
    \definition{s.}{dificuldade; grau de dificuldade}
  \end{Phonetics}
\end{Entry}

\begin{Entry}{难看}{10,9}{⾫、⽬}
  \begin{Phonetics}{难看}{nan2 kan4}[][HSK 2]
    \definition{adj.}{feio; desagradável à vista | vergonhoso; embaraçoso; desonroso; sem glória; sem dignidade}
  \end{Phonetics}
\end{Entry}

\begin{Entry}{难得}{10,11}{⾫、⼻}
  \begin{Phonetics}{难得}{nan2de2}[][HSK 5]
    \definition{adj.}{raro; difícil de encontrar; difícil de obter ou realizar, indicando que é valioso}
    \definition{adv.}{raramente; com pouca frequência}
  \end{Phonetics}
\end{Entry}

\begin{Entry}{难道}{10,12}{⾫、⾡}
  \begin{Phonetics}{难道}{nan2dao4}[][HSK 3]
    \definition{adv.}{certamente não significa que\dots?; é possível que\dots?; não me diga\dots; poderia ser que\dots?; usado em frases interrogativas para reforçar o tom interrogativo; frequentemente usado com palavras como "吗" e "不成".}
  \seealsoref{不成}{bu4 cheng2}
  \seealsoref{吗}{ma5}
  \end{Phonetics}
\end{Entry}

\begin{Entry}{难题}{10,15}{⾫、⾴}
  \begin{Phonetics}{难题}{nan2 ti2}[][HSK 2]
    \definition[个,道]{s.}{desafio; problema difícil; questão difícil; questões difíceis de responder ou resolver}
  \end{Phonetics}
\end{Entry}

\begin{Entry}{顽}{10}{⾴}
  \begin{Phonetics}{顽}{wan2}
    \definition*{s.}{Sobrenome Wan}
    \definition{adj.}{estúpido; denso; insensível | teimoso; obstinado; não é facilmente persuadido ou subjugado | travesso; pernicioso | cabeça dura; estúpido e ignorante}
    \definition{v.}{brincar; divertir-se; divertir-se | empregar; recorrer a | envolver-se em; tomar parte em}
  \end{Phonetics}
\end{Entry}

\begin{Entry}{顽皮}{10,5}{⾴、⽪}
  \begin{Phonetics}{顽皮}{wan2 pi2}[][HSK 6]
    \definition{adj.}{atrevido; travesso; arteiro; levado; (crianças, adolescentes, etc.) adoram brincar e causar problemas e não dão ouvidos a conselhos}
  \end{Phonetics}
\end{Entry}

\begin{Entry}{顽强}{10,12}{⾴、⼸}
  \begin{Phonetics}{顽强}{wan2qiang2}[][HSK 6]
    \definition{adj.}{firme; tenaz; indomável; forte; resistente}
  \end{Phonetics}
\end{Entry}

\begin{Entry}{顾}{10}{⾴}
  \begin{Phonetics}{顾}{gu4}[][HSK 6]
    \definition*{s.}{Sobrenome Gu}
    \definition{adv.}{em vez disso; pelo contrário; indica o oposto, equivalente a 却 ou 反而}
    \definition{conj.}{mas; no entanto}
    \definition{v.}{olhar para trás; olhar para; virar-se e olhar para | cuidar de; atender a; levar em conta ou consideração | visitar; chamar | sentir pena de}
  \seealsoref{反而}{fan3'er2}
  \seealsoref{却}{que4}
  \end{Phonetics}
\end{Entry}

\begin{Entry}{顾及}{10,3}{⾴、⼃}
  \begin{Phonetics}{顾及}{gu4ji2}[][HSK 7-9]
    \definition{v.}{atender a; levar em conta; dar consideração a; cuidar de; notar}
  \end{Phonetics}
\end{Entry}

\begin{Entry}{顾不上}{10,4,3}{⾴、⼀、⼀}
  \begin{Phonetics}{顾不上}{gu4bu5shang4}[][HSK 7-9]
    \definition{v.}{não conseguir; não conseguir atender; incapaz de cuidar de (fazer algo)}
  \end{Phonetics}
\end{Entry}

\begin{Entry}{顾不得}{10,4,11}{⾴、⼀、⼻}
  \begin{Phonetics}{顾不得}{gu4bu5de5}[][HSK 7-9]
    \definition{v.}{incapaz de mudar algo | incapaz de lidar com}
  \end{Phonetics}
\end{Entry}

\begin{Entry}{顾全大局}{10,6,3,7}{⾴、⼊、⼤、⼫}
  \begin{Phonetics}{顾全大局}{gu4quan2-da4ju2}[][HSK 7-9]
    \definition{expr.}{``Considere a situação geral.''; levar em conta os interesses do todo; considerar a situação como um todo; levar em consideração o panorama geral; trabalhar para o benefício de todos}
  \end{Phonetics}
\end{Entry}

\begin{Entry}{顾问}{10,6}{⾴、⾨}
  \begin{Phonetics}{顾问}{gu4wen4}[][HSK 5]
    \definition[个,位,名]{s.}{conselheiro; consultor; assessor; pessoas com conhecimento especializado ou experiência contratadas para prestar consultoria a organizações ou indivíduos}
  \end{Phonetics}
\end{Entry}

\begin{Entry}{顾客}{10,9}{⾴、⼧}
  \begin{Phonetics}{顾客}{gu4ke4}[][HSK 2]
    \definition[个,位,名,些]{s.}{cliente; comprador; consumidor; paciente}
  \end{Phonetics}
\end{Entry}

\begin{Entry}{顾虑}{10,10}{⾴、⾌}
  \begin{Phonetics}{顾虑}{gu4lv4}[][HSK 7-9]
    \definition[丝,点]{s.}{preocupação; escrúpulo; receio; apreensão}
    \definition{v.}{estar apreensivo (sobre as consequências da própria ação)}
  \end{Phonetics}
\end{Entry}

\begin{Entry}{顿}{10}{⾴}
  \begin{Phonetics}{顿}{dun4}[][HSK 3]
    \definition*{s.}{Sobrenome Dun}
    \definition{adj.}{cansado; fatigado}
    \definition{adv.}{de repente; imediatamente; indica que o tempo é curto, equivalente a 立刻}
    \definition{clas.}{usado para refeições | usado para surras, repreensões, castigos físicos, etc.}
    \definition{s.}{um lugar para ficar; acomodação e alimentação}
    \definition{v.}{pausar; parar; fazer uma pausa | pausar na escrita para reforçar o início ou o fim de um traço; ao escrever com pincel, pressione o pincel com força e pare um pouco sobre o papel | tocar o chão (com a cabeça) | bater o pé); chutar o chão ou bater no chão com um objeto | resolver; arranjar | montar acampamento; ficar temporariamente; parar para se hospedar; acampar}
  \seealsoref{立刻}{li4ke4}
  \end{Phonetics}
\end{Entry}

\begin{Entry}{顿时}{10,7}{⾴、⽇}
  \begin{Phonetics}{顿时}{dun4shi2}[][HSK 7-9]
    \definition{adv.}{de repente; imediatamente; repentinamente; indica que uma ação ou comportamento ocorre sob certas circunstâncias ou imediatamente após algo; usado principalmente na escrita; usado apenas para descrever eventos passados}
  \end{Phonetics}
\end{Entry}

\begin{Entry}{颁}{10}{⾴}
  \begin{Phonetics}{颁}{ban1}
    \definition{v.}{promulgar; emitir; enviar | conceder ou conferir}
  \end{Phonetics}
\end{Entry}

\begin{Entry}{颁发}{10,5}{⾴、⼜}
  \begin{Phonetics}{颁发}{ban1fa1}[][HSK 7-9]
    \definition{v.}{promulgar; emitir (comandos, instruções, regulamentos, etc.) a um superior | premiar (prêmio, medalha, certificado, etc.)}
  \end{Phonetics}
\end{Entry}

\begin{Entry}{颁布}{10,5}{⾴、⼱}
  \begin{Phonetics}{颁布}{ban1bu4}[][HSK 7-9]
    \definition{v.}{promulgar; emitir; publicar; anunciar (leis, regulamentos, etc.), com um escopo de uso mais restrito do que 公布}
  \seealsoref{公布}{gong1bu4}
  \end{Phonetics}
\end{Entry}

\begin{Entry}{颁奖}{10,9}{⾴、⼤}
  \begin{Phonetics}{颁奖}{ban1/jiang3}[][HSK 7-9]
    \definition{v.+compl.}{conceder prêmios, bônus, certificados, etc.; distribuir prêmios, bônus, certificados, etc.}
  \end{Phonetics}
\end{Entry}

\begin{Entry}{预}{10}{⾴}
  \begin{Phonetics}{预}{yu4}
    \definition{adv.}{antecipadamente}
    \definition{v.}{avançar | preparar}
  \end{Phonetics}
\end{Entry}

\begin{Entry}{预习}{10,3}{⾴、⼄}
  \begin{Phonetics}{预习}{yu4xi2}[][HSK 3]
    \definition{v.}{pré-visualizar; preparar uma lição; estudar antecipadamente as matérias que serão abordadas nas aulas}
  \end{Phonetics}
\end{Entry}

\begin{Entry}{预见}{10,4}{⾴、⾒}
  \begin{Phonetics}{预见}{yu4jian4}
    \definition{s.}{previsão; intuição; vislumbre}
    \definition{v.}{prever}
  \end{Phonetics}
\end{Entry}

\begin{Entry}{预计}{10,4}{⾴、⾔}
  \begin{Phonetics}{预计}{yu4 ji4}[][HSK 3]
    \definition{v.}{estimar; calcular com antecedência}
  \end{Phonetics}
\end{Entry}

\begin{Entry}{预订}{10,4}{⾴、⾔}
  \begin{Phonetics}{预订}{yu4ding4}[][HSK 4]
    \definition{v.}{reservar; fazer uma reserva}
  \end{Phonetics}
\end{Entry}

\begin{Entry}{预付}{10,5}{⾴、⼈}
  \begin{Phonetics}{预付}{yu4fu4}
    \definition{s.}{pré-pago}
    \definition{v.}{pagar antecipadamente}
  \end{Phonetics}
\end{Entry}

\begin{Entry}{预约}{10,6}{⾴、⽷}
  \begin{Phonetics}{预约}{yu4 yue1}[][HSK 6]
    \definition[个]{s.}{reserva}
    \definition{v.}{reservar; agendar; marcar compromisso; marcar uma consulta}
  \end{Phonetics}
\end{Entry}

\begin{Entry}{预防}{10,6}{⾴、⾩}
  \begin{Phonetics}{预防}{yu4fang2}[][HSK 3]
    \definition{v.}{prevenir; proteger-se contra; tomar precauções contra; preparar-se com antecedência para evitar que algo ruim aconteça}
  \end{Phonetics}
\end{Entry}

\begin{Entry}{预判}{10,7}{⾴、⼑}
  \begin{Phonetics}{预判}{yu4pan4}
    \definition{v.}{prever | antecipar}
  \end{Phonetics}
\end{Entry}

\begin{Entry}{预报}{10,7}{⾴、⼿}
  \begin{Phonetics}{预报}{yu4bao4}[][HSK 3]
    \definition[个,项]{s.}{boletim meteorológico; previsões meteorológicas antecipadas}
    \definition{v.}{prever (o tempo); relatar antes que algo aconteça, usado principalmente em relação ao clima, astronomia, desastres naturais, etc.}
  \end{Phonetics}
\end{Entry}

\begin{Entry}{预备}{10,8}{⾴、⼡}
  \begin{Phonetics}{预备}{yu4 bei4}[][HSK 5]
    \definition{v.}{preparar-se; ficar pronto}
  \end{Phonetics}
\end{Entry}

\begin{Entry}{预定}{10,8}{⾴、⼧}
  \begin{Phonetics}{预定}{yu4ding4}
    \definition{v.}{agendar com antecedência}
  \end{Phonetics}
\end{Entry}

\begin{Entry}{预购}{10,8}{⾴、⾙}
  \begin{Phonetics}{预购}{yu4gou4}
    \definition{s.}{compra antecipada}
    \definition{v.}{comprar antecipadamente}
  \end{Phonetics}
\end{Entry}

\begin{Entry}{预测}{10,9}{⾴、⽔}
  \begin{Phonetics}{预测}{yu4 ce4}[][HSK 4]
    \definition{v.}{prever; prognosticar; predizer}
  \end{Phonetics}
\end{Entry}

\begin{Entry}{预祝}{10,9}{⾴、⽰}
  \begin{Phonetics}{预祝}{yu4zhu4}
    \definition{v.}{parabenizar de antemão | oferecer os melhores votos para}
  \end{Phonetics}
\end{Entry}

\begin{Entry}{预览}{10,9}{⾴、⾒}
  \begin{Phonetics}{预览}{yu4lan3}
    \definition{s.}{visualização}
    \definition{v.}{visualizar}
  \end{Phonetics}
\end{Entry}

\begin{Entry}{预留}{10,10}{⾴、⽥}
  \begin{Phonetics}{预留}{yu4liu2}
    \definition{v.}{separar | reservar}
  \end{Phonetics}
\end{Entry}

\begin{Entry}{预配}{10,10}{⾴、⾣}
  \begin{Phonetics}{预配}{yu4pei4}
    \definition{s.}{pré-alocado | pré-cabeado}
    \definition{v.}{pré-alocar | pré-cabear}
  \end{Phonetics}
\end{Entry}

\begin{Entry}{预谋}{10,11}{⾴、⾔}
  \begin{Phonetics}{预谋}{yu4mou2}
    \definition{adj.}{premeditado}
    \definition{v.}{planejar algo com antecedência (especialmente um crime)}
  \end{Phonetics}
\end{Entry}

\begin{Entry}{预提}{10,12}{⾴、⼿}
  \begin{Phonetics}{预提}{yu4ti2}
    \definition{s.}{retenção}
    \definition{v.}{reter (imposto)}
  \end{Phonetics}
\end{Entry}

\begin{Entry}{预期}{10,12}{⾴、⽉}
  \begin{Phonetics}{预期}{yu4qi1}[][HSK 5]
    \definition{v.}{esperar; antecipar; imaginar; antecipar com expectativa}
  \end{Phonetics}
\end{Entry}

\begin{Entry}{预感}{10,13}{⾴、⼼}
  \begin{Phonetics}{预感}{yu4gan3}
    \definition{s.}{premonição}
    \definition{v.}{ter uma premonição}
  \end{Phonetics}
\end{Entry}

\begin{Entry}{预警}{10,19}{⾴、⾔}
  \begin{Phonetics}{预警}{yu4jing3}
    \definition{s.}{aviso | aviso antecipado}
  \end{Phonetics}
\end{Entry}

\begin{Entry}{饿}{10}{⾷}
  \begin{Phonetics}{饿}{e4}[][HSK 1]
    \definition{adj.}{faminto}
    \definition{v.}{passar fome; causar fome}
  \end{Phonetics}
\end{Entry}

\begin{Entry}{高}{10}{⾼}[Kangxi 189]
  \begin{Phonetics}{高}{gao1}[][HSK 1]
    \definition*{s.}{Sobrenome Gao}
    \definition{adj.}{alto; elevado; grande distância de baixo para cima; longe do chão | barulhento | sofisticado; caro; de preço elevado; acima do valor real ou do preço de mercado | acima da média; de alto nível ou grau; acima do padrão geral ou da média; de nível superior}
    \definition{s.}{altura; altitude}
  \end{Phonetics}
\end{Entry}

\begin{Entry}{高于}{10,3}{⾼、⼆}
  \begin{Phonetics}{高于}{gao1 yu2}[][HSK 5]
    \definition{v.}{ser mais alto do que; sobrepujar}
  \end{Phonetics}
\end{Entry}

\begin{Entry}{高大}{10,3}{⾼、⼤}
  \begin{Phonetics}{高大}{gao1 da4}[][HSK 5]
    \definition{adj.}{alto e grande; alto | elevado; sublime; nobre}
  \end{Phonetics}
\end{Entry}

\begin{Entry}{高山}{10,3}{⾼、⼭}
  \begin{Phonetics}{高山}{gao1shan1}[][HSK 7-9]
    \definition[座]{s.}{alta montanha; alpes}
  \end{Phonetics}
\end{Entry}

\begin{Entry}{高中}{10,4}{⾼、⼁}
  \begin{Phonetics}{高中}{gao1 zhong1}[][HSK 2]
    \definition[所,个]{s.}{ensino médio; escola secundária de ensino médio}
  \end{Phonetics}
\end{Entry}

\begin{Entry}{高手}{10,4}{⾼、⼿}
  \begin{Phonetics}{高手}{gao1 shou3}[][HSK 6]
    \definition[位,个,名,些,群]{s.}{ás; mestre; especialista; \emph{expert}; uma pessoa com habilidades excepcionais}
  \end{Phonetics}
\end{Entry}

\begin{Entry}{高尔夫}{10,5,4}{⾼、⼩、⼤}
  \begin{Phonetics}{高尔夫}{gao1'er3fu1}
    \definition{s.}{(empréstimo linguístico) \emph{golf}}
  \end{Phonetics}
\end{Entry}

\begin{Entry}{高尔夫球}{10,5,4,11}{⾼、⼩、⼤、⽟}
  \begin{Phonetics}{高尔夫球}{gao1'er3fu1qiu2}[][HSK 7-9]
    \definition[个,只,场,些]{s.}{golfe | bola de golfe}
  \end{Phonetics}
\end{Entry}

\begin{Entry}{高价}{10,6}{⾼、⼈}
  \begin{Phonetics}{高价}{gao1 jia4}[][HSK 4]
    \definition{s.}{preço alto; bilhete caro; custo elevado; dispendioso}
  \end{Phonetics}
\end{Entry}

\begin{Entry}{高兴}{10,6}{⾼、⼋}
  \begin{Phonetics}{高兴}{gao1xing4}[][HSK 1]
    \definition{adj.}{contente; feliz; exultante; alegre; satisfeito; animado}
    \definition{v.}{estar contente; estar feliz; estar animado; estar de bom humor; fazer algo com alegria; gostar}
  \end{Phonetics}
\end{Entry}

\begin{Entry}{高压}{10,6}{⾼、⼚}
  \begin{Phonetics}{高压}{gao1ya1}[][HSK 7-9]
    \definition{s.}{Física, Meteorologia: alta pressão (oposto a 低压) | Eletricidade: alta tensão; alta voltagem (oposto a 低压) | Política: mão de ferro; arrogância | Medicina: pressão sistólica; pressão máxima | perseguição cruel; opressão extrema}
  \seealsoref{低压}{di1ya1}
  \end{Phonetics}
\end{Entry}

\begin{Entry}{高级}{10,6}{⾼、⽷}
  \begin{Phonetics}{高级}{gao1ji2}[][HSK 2]
    \definition{adj.}{sênior; de alto escalão; de alto nível; elevado; excelente; superior; estágio avançado | e alta qualidade; de primeira qualidade; avançado}
  \end{Phonetics}
\end{Entry}

\begin{Entry}{高考}{10,6}{⾼、⽼}
  \begin{Phonetics}{高考}{gao1 kao3}[][HSK 6]
    \definition[次,回,场]{s.}{vestibular; exame de admissão em instituições de ensino superior}
  \end{Phonetics}
\end{Entry}

\begin{Entry}{高血压}{10,6,6}{⾼、⾎、⼚}
  \begin{Phonetics}{高血压}{gao1xue4ya1}[][HSK 7-9]
    \definition{adj.}{hipertenso}
    \definition[点儿]{s.}{hipertenção; pressão alta}
  \end{Phonetics}
\end{Entry}

\begin{Entry}{高低}{10,7}{⾼、⼈}
  \begin{Phonetics}{高低}{gao1di1}[][HSK 7-9]
    \definition{adv.}{apenas; simplesmente; em qualquer caso; de qualquer forma; em qualquer conta; indica que não importa o que | finalmente; no final, depois de tudo}
    \definition{s.}{inclinação; nível; altura | diferença de grau; superioridade ou inferioridade relativa | discrição; senso de propriedade | (falar ou fazer coisas) medida; profundidade, leveza e peso}
  \end{Phonetics}
\end{Entry}

\begin{Entry}{高层}{10,7}{⾼、⼫}
  \begin{Phonetics}{高层}{gao1 ceng2}[][HSK 6]
    \definition{adj.}{(de um edifício) arranha-céu | (de posição oficial) alto nível}
    \definition{s.}{nível superior; piso, camada, etc. | arranha-céus; um prédio de apartamentos alto}
  \end{Phonetics}
\end{Entry}

\begin{Entry}{高技术}{10,7,5}{⾼、⼿、⽊}
  \begin{Phonetics}{高技术}{gao1 ji4 shu4}
    \definition{s.}{alta tecnologia; \emph{hight tech}}
  \seealsoref{高科技}{gao1 ke1 ji4}
  \end{Phonetics}
\end{Entry}

\begin{Entry}{高尚}{10,8}{⾼、⼩}
  \begin{Phonetics}{高尚}{gao1shang4}[][HSK 4]
    \definition{adj.}{nobre; elevado; descreve um alto padrão moral e uma boa qualidade de pensamento | significativo e não de mau gosto}
  \end{Phonetics}
\end{Entry}

\begin{Entry}{高昂}{10,8}{⾼、⽇}
  \begin{Phonetics}{高昂}{gao1'ang2}[][HSK 7-9]
    \definition{adj.}{alto; eufórico; exaltado | caro; exorbitante}
    \definition{v.}{manter erguida (a cabeça, etc.)}
  \end{Phonetics}
\end{Entry}

\begin{Entry}{高明}{10,8}{⾼、⽇}
  \begin{Phonetics}{高明}{gao1ming2}[][HSK 7-9]
    \definition{adj.}{sábio; brilhante; (percepção, habilidades) excelente}
    \definition{s.}{pessoa sábia; pessoa habilidosa}
  \end{Phonetics}
\end{Entry}

\begin{Entry}{高空}{10,8}{⾼、⽳}
  \begin{Phonetics}{高空}{gao1kong1}[][HSK 7-9]
    \definition{s.}{alta altitude; ar superior (oposto a 低空)}
  \seealsoref{低空}{di1kong1}
  \end{Phonetics}
\end{Entry}

\begin{Entry}{高度}{10,9}{⾼、⼴}
  \begin{Phonetics}{高度}{gao1 du4}[][HSK 5]
    \definition{adj.}{alto; elevado; avançado; alto grau | alta concentração; intenso}
    \definition[个]{s.}{altura; altitude; elevação; distância de baixo para cima; o grau e o nível em que as coisas se desenvolveram}
  \end{Phonetics}
\end{Entry}

\begin{Entry}{高科技}{10,9,7}{⾼、⽲、⼿}
  \begin{Phonetics}{高科技}{gao1 ke1 ji4}[][HSK 6]
    \definition[种,类]{s.}{alta tecnologia; \emph{high tech}}
  \seealsoref{高技术}{gao1 ji4 shu4}
  \end{Phonetics}
\end{Entry}

\begin{Entry}{高贵}{10,9}{⾼、⾙}
  \begin{Phonetics}{高贵}{gao1gui4}[][HSK 7-9]
    \definition{adj.}{(caráter pessoal) nobre; honrado; moralmente elevado; magnânimo | grandeza; extremamente valioso | elitista; altamente privilegiado; refere-se àqueles com status elevado e vida superior}
  \end{Phonetics}
\end{Entry}

\begin{Entry}{高原}{10,10}{⾼、⼚}
  \begin{Phonetics}{高原}{gao1 yuan2}[][HSK 5]
    \definition[片]{s.}{platô; terras altas; planalto | planalto continental}
  \end{Phonetics}
\end{Entry}

\begin{Entry}{高峰}{10,10}{⾼、⼭}
  \begin{Phonetics}{高峰}{gao1feng1}[][HSK 6]
    \definition[个,座]{s.}{cume; pináculo; pico da montanha | pico (de atividade, qualidade ou realização); uma metáfora para o ponto mais alto no desenvolvimento das coisas | cúpula; principais líderes; uma metáfora para o mais alto nível de liderança}
  \end{Phonetics}
\end{Entry}

\begin{Entry}{高峰期}{10,10,12}{⾼、⼭、⽉}
  \begin{Phonetics}{高峰期}{gao1feng1qi1}[][HSK 7-9]
    \definition[个]{s.}{período de pico; (de tráfego) horas de pico; o período em que ocorre com mais frequência ou se desenvolve mais próspero}
  \end{Phonetics}
\end{Entry}

\begin{Entry}{高效}{10,10}{⾼、⽁}
  \begin{Phonetics}{高效}{gao1xiao4}[][HSK 7-9]
    \definition{adj.}{altamente eficiente | eficiente | altamente eficaz}
  \end{Phonetics}
\end{Entry}

\begin{Entry}{高档}{10,10}{⾼、⽊}
  \begin{Phonetics}{高档}{gao1dang4}[][HSK 6]
    \definition{adj.}{grau superior; alta qualidade; alo grau; qualidade superior; boa qualidade, preço alto (produto)}
  \end{Phonetics}
\end{Entry}

\begin{Entry}{高涨}{10,10}{⾼、⽔}
  \begin{Phonetics}{高涨}{gao1zhang3}[][HSK 7-9]
    \definition{v.}{ascender; subir alto; avançar; (preços, sentimento, etc.) subir rapidamente (em oposição a 低落)}
  \seealsoref{低落}{di1luo4}
  \end{Phonetics}
\end{Entry}

\begin{Entry}{高调}{10,10}{⾼、⾔}
  \begin{Phonetics}{高调}{gao1diao4}[][HSK 7-9]
    \definition{adj.}{alto perfil; retaliação; significa agir de forma muito ostensiva e chamativa, deixando claro para todos; também pode significar opor-se deliberadamente aos outros, chegando até a provocar uma briga}
    \definition{s.}{tom elevado; palavras de alto som; sons mais agudos do que o normal ao cantar ou falar}
  \end{Phonetics}
\end{Entry}

\begin{Entry}{高速}{10,10}{⾼、⾡}
  \begin{Phonetics}{高速}{gao1 su4}[][HSK 3]
    \definition{adj.}{alta velocidade; veloz; rápido}
    \definition[条]{s.}{autoestrada; via expressa; rodovia}
  \end{Phonetics}
\end{Entry}

\begin{Entry}{高速公路}{10,10,4,13}{⾼、⾡、⼋、⾜}
  \begin{Phonetics}{高速公路}{gao1su4gong1lu4}[][HSK 3]
    \definition[条]{s.}{via expressa; rodovia; autoestrada; as rodovias destinadas exclusivamente ao tráfego de veículos em alta velocidade são retas e, quando cruzam outras vias, utilizam cruzamentos em nível}
  \end{Phonetics}
\end{Entry}

\begin{Entry}{高铁}{10,10}{⾼、⾦}
  \begin{Phonetics}{高铁}{gao1 tie3}[][HSK 4]
    \definition{s.}{trem de alta velocidade; trem bala}
  \end{Phonetics}
\end{Entry}

\begin{Entry}{高傲}{10,12}{⾼、⼈}
  \begin{Phonetics}{高傲}{gao1'ao4}[][HSK 7-9]
    \definition{adj.}{arrogante; altivo | orgulhoso; respeitoso; nobre}
  \end{Phonetics}
\end{Entry}

\begin{Entry}{高温}{10,12}{⾼、⽔}
  \begin{Phonetics}{高温}{gao1 wen1}[][HSK 5]
    \definition{s.}{alta temperatura (oposto a 低温); temperatura elevada; hipertermia; megatemperatura; inferno}
  \seealsoref{低温}{di1 wen1}
  \end{Phonetics}
\end{Entry}

\begin{Entry}{高等}{10,12}{⾼、⽵}
  \begin{Phonetics}{高等}{gao1 deng3}[][HSK 6]
    \definition{adj.}{superior; avançado (oposto a 低等) | alto nível}
  \seealsoref{低等}{di1 deng3}
  \end{Phonetics}
\end{Entry}

\begin{Entry}{高超}{10,12}{⾼、⾛}
  \begin{Phonetics}{高超}{gao1chao1}[][HSK 7-9]
    \definition{adj.}{soberbo; excelente; descreve um nível muito alto, excedendo a maioria dos níveis}
  \end{Phonetics}
\end{Entry}

\begin{Entry}{高雅}{10,12}{⾼、⾫}
  \begin{Phonetics}{高雅}{gao1ya3}[][HSK 7-9]
    \definition{adj.}{delicado | elegante}
    \definition{s.}{delicadeza; decoro | elegância}
  \end{Phonetics}
\end{Entry}

\begin{Entry}{高新技术}{10,13,7,5}{⾼、⽄、⼿、⽊}
  \begin{Phonetics}{高新技术}{gao1xin1-ji4shu4}[][HSK 7-9]
    \definition[项,门,套]{s.}{nova e alta tecnologia; \emph{high‐tech}}
  \end{Phonetics}
\end{Entry}

\begin{Entry}{高楼}{10,13}{⾼、⽊}
  \begin{Phonetics}{高楼}{gao1lou2}
    \definition[座]{s.}{edifício alto | edifício de muitos andares | arranha-céu}
  \end{Phonetics}
\end{Entry}

\begin{Entry}{高跟鞋}{10,13,15}{⾼、⾜、⾰}
  \begin{Phonetics}{高跟鞋}{gao1 gen1 xie2}[][HSK 5]
    \definition[双]{s.}{salto alto; sapatos de salto alto; sapato feminino com salto mais alto e mais distante do chão}
  \end{Phonetics}
\end{Entry}

\begin{Entry}{高龄}{10,13}{⾼、⿒}
  \begin{Phonetics}{高龄}{gao1ling2}[][HSK 7-9]
    \definition{adj.}{mais velho que o normal | avançado em anos}
    \definition{s.}{idade avançada; idade venerável}
  \end{Phonetics}
\end{Entry}

\begin{Entry}{高潮}{10,15}{⾼、⽔}
  \begin{Phonetics}{高潮}{gao1chao2}[][HSK 4]
    \definition[个,场]{s.}{maré alta; o nível mais alto da maré em um ciclo de maré | pico; aumento; maré alta; uma metáfora para o estágio mais próspero de desenvolvimento das coisas (diferente de 低潮) | (ficção, drama e filmes) clímax}
  \seealsoref{低潮}{di1chao2}
  \end{Phonetics}
\end{Entry}

\begin{Entry}{高额}{10,15}{⾼、⾴}
  \begin{Phonetics}{高额}{gao1'e2}[][HSK 7-9]
    \definition{s.}{quantidade enorme; cota grande | grande quantidade}
  \end{Phonetics}
\end{Entry}

\begin{Entry}{髟}{10}{⾽}[Kangxi 190]
  \begin{Phonetics}{髟}{biao1}
    \definition{adj.}{(de cabelo) solto, caído}
  \end{Phonetics}
\end{Entry}

\begin{Entry}{鬯}{10}{⾿}[Kangxi 192]
  \begin{Phonetics}{鬯}{chang4}
    \definition{adj.}{suave; desimpedido | livre; desinibido}
    \definition{s.}{um tipo de vinho usado em sacrifícios antigos | (antigo) estojo ou bolsa para arco | o mesmo que 畅}
  \seealsoref{畅}{chang4}
  \end{Phonetics}
\end{Entry}

\begin{Entry}{鬲}{10}{⿀}[Kangxi 193]
  \begin{Phonetics}{鬲}{ge2}
    \definition{s.}{um antigo utensílio de cozinha semelhante a um caldeirão; uma grande panela de barro | utilizado em nomes geográficos ou pessoais}
  \end{Phonetics}
  \begin{Phonetics}{鬲}{li4}
    \definition{s.}{recipiente de cerâmica antigo com três pernas usado para cozinhar, com marcas de cordão na parte externa e pernas ocas}
  \end{Phonetics}
\end{Entry}

\begin{Entry}{鸭}{10}{⿃}
  \begin{Phonetics}{鸭}{ya1}
    \definition[只]{s.}{pato | (gíria) prostituto}
  \end{Phonetics}
\end{Entry}

\begin{Entry}{鸭子}{10,3}{⿃、⼦}
  \begin{Phonetics}{鸭子}{ya1 zi5}[][HSK 5]
    \definition[只,群]{s.}{pato | Gíria: prostituto}
  \end{Phonetics}
\end{Entry}

\begin{Entry}{鸵}{10}{⿃}
  \begin{Phonetics}{鸵}{tuo2}
    \definition[只]{s.}{avestruz}
  \end{Phonetics}
\end{Entry}

\begin{Entry}{鸵鸟}{10,5}{⿃、⿃}
  \begin{Phonetics}{鸵鸟}{tuo2niao3}
    \definition{s.}{avestruz}
  \end{Phonetics}
\end{Entry}

%%%%% EOF %%%%%

