%%%
%%% 10画
%%%

\section*{10画}\addcontentsline{toc}{section}{10画}

\begin{entry}{乘客}{10,9}[Radicais ⽲、⼧]
  \begin{phonetics}{乘客}{cheng2ke4}
    \definition{s.}{passageiro}
  \end{phonetics}
\end{entry}

\begin{entry}{乘客数}{10,9,13}[Radicais ⽲、⼧、⽁]
  \begin{phonetics}{乘客数}{cheng2ke4 shu4}
    \definition{s.}{número de passageiros}
  \end{phonetics}
\end{entry}

\begin{entry}{倂}{10}[Radical ⼈]
  \begin{phonetics}{倂}{bing4}
    \variantof{并}
  \end{phonetics}
\end{entry}

\begin{entry}{倒}{10}[Radical ⼈]
  \begin{phonetics}{倒}{dao3}[][HSK 2]
    \definition{v.}{cair no chão | deitar-se no chão | colapsar | ir à falência}
  \end{phonetics}
  \begin{phonetics}{倒}{dao4}[][HSK 2]
    \definition{adv.}{ao contrário da expectativa | ao contrário}
    \definition{v.}{inverter | colocar de cabeça para baixo ou de frente para trás | derramar | tombar}
  \end{phonetics}
\end{entry}

\begin{entry}{倒地}{10,6}[Radicais ⼈、⼟]
  \begin{phonetics}{倒地}{dao3di4}
    \definition{v.}{cair no chão}
  \end{phonetics}
\end{entry}

\begin{entry}{倒血霉}{10,6,15}[Radicais ⼈、⾎、⾬]
  \begin{phonetics}{倒血霉}{dao3xue4mei2}
    \definition{v.}{ter muito azar (versão mais forte de 倒霉)}
  \seealsoref{倒霉}{dao3mei2}
  \end{phonetics}
\end{entry}

\begin{entry}{倒楣}{10,13}[Radicais ⼈、⽊]
  \begin{phonetics}{倒楣}{dao3mei2}
    \variantof{倒霉}
  \end{phonetics}
\end{entry}

\begin{entry}{倒霉}{10,15}[Radicais ⼈、⾬]
  \begin{phonetics}{倒霉}{dao3mei2}
    \definition{adj.}{azarado}
    \definition{s.}{azar | má sorte}
    \definition{v.}{estar sem sorte | ter azar}
  \seealsoref{倒血霉}{dao3xue4mei2}
  \end{phonetics}
\end{entry}

\begin{entry}{倘使}{10,8}[Radicais ⼈、⼈]
  \begin{phonetics}{倘使}{tang3shi3}
    \definition{conj.}{se | supondo que | no caso}
  \end{phonetics}
\end{entry}

\begin{entry}{倘或}{10,8}[Radicais ⼈、⼽]
  \begin{phonetics}{倘或}{tang3huo4}
    \definition{conj.}{se | supondo que | no caso}
  \end{phonetics}
\end{entry}

\begin{entry}{倘若}{10,8}[Radicais ⼈、⾋]
  \begin{phonetics}{倘若}{tang3ruo4}
    \definition{conj.}{se | supondo que | no caso}
  \end{phonetics}
\end{entry}

\begin{entry}{借}{10}[Radical ⼈]
  \begin{phonetics}{借}{jie4}[][HSK 2]
    \definition{adv.}{por meio de}
    \definition{v.}{pedir emprestado | emprestar | aproveitar (uma oportunidade)}
  \end{phonetics}
\end{entry}

\begin{entry}{借书证}{10,4,7}[Radicais ⼈、⼄、⾔]
  \begin{phonetics}{借书证}{jie4shu1zheng4}
    \definition{s.}{cartão de biblioteca | (literalmente) cartão para pedir emprestado livros}
  \end{phonetics}
\end{entry}

\begin{entry}{值}{10}[Radical ⼈]
  \begin{phonetics}{值}{zhi2}[][HSK 3]
    \definition{s.}{preço; valor; valor numérico}
    \definition{v.}{valer a pena | acontecer com; ir de encontro | estar de serviço; ter sua vez em algo; assumir a posição de turno}
  \end{phonetics}
\end{entry}

\begin{entry}{值得}{10,11}[Radicais ⼈、⼻]
  \begin{phonetics}{值得}{zhi2de5}[][HSK 3]
    \definition{adj.}{que vale a pena}
    \definition{v.}{merecer; valer a pena; ser digno; significa que fazer isso terá bons resultados; é valioso e significativo}
  \end{phonetics}
\end{entry}

\begin{entry}{倾城}{10,9}[Radicais ⼈、⼟]
  \begin{phonetics}{倾城}{qing1cheng2}
    \definition{adj.}{sedutora (mulher)}
    \definition{adv.}{de todo o lugar | vindo de todos os lugares}
    \definition{v.}{arruinar e derrubar o estado}
  \end{phonetics}
\end{entry}

\begin{entry}{健身}{10,7}[Radicais ⼈、⾝]
  \begin{phonetics}{健身}{jian4shen1}
    \definition{s.}{exercício físico | \emph{fitness}}
    \definition{v.}{exercitar-se | manter a forma}
  \end{phonetics}
\end{entry}

\begin{entry}{健康}{10,11}[Radicais ⼈、⼴]
  \begin{phonetics}{健康}{jian4kang1}[][HSK 2]
    \definition{adj.}{em forma | saudável | curado}
    \definition{s.}{saúde | físico}
  \end{phonetics}
\end{entry}

\begin{entry}{兼}{10}[Radical ⼋]
  \begin{phonetics}{兼}{jian1}
    \definition{conj.}{e (ocupando dois ou mais cargos (oficiais) ao mesmo tempo)}
  \end{phonetics}
\end{entry}

\begin{entry}{准}{10}[Radical ⼎]
  \begin{phonetics}{准}{zhun3}[][HSK 3]
    \definition{adj.}{exato; preciso; algo determinado a ser imutável | preciso; exato; correto | perto; parcialmente; quase; próximo de algo em termos de padrão}
    \definition{adv.}{definitivamente; certamente}
    \definition{pref.}{``quasi''; ``para''}
    \definition{prep.}{de acordo com; baseado em}
    \definition{s.}{norma; padrão; critério | confiança certa; uma ideia definida, certeza, etc. (geralmente usada depois de "有" ou "没有o")}
    \definition{v.}{autorizar; conceder; consentir; permitir}
  \seealsoref{没有}{mei2you3}
  \seealsoref{有}{you3}
  \end{phonetics}
\end{entry}

\begin{entry}{准备}{10,8}[Radicais ⼎、⼡]
  \begin{phonetics}{准备}{zhun3bei4}[][HSK 1]
    \definition{v.}{preparar | ficar pronto | pretender | planejar}
  \end{phonetics}
\end{entry}

\begin{entry}{准确}{10,12}[Radicais ⼎、⽯]
  \begin{phonetics}{准确}{zhun3que4}[][HSK 2]
    \definition{adj.}{exato | preciso | acurado}
  \end{phonetics}
\end{entry}

\begin{entry}{凉}{10}[Radical ⼎]
  \begin{phonetics}{凉}{liang2}[][HSK 2]
    \definition{adj.}{frio | legal}
  \end{phonetics}
  \begin{phonetics}{凉}{liang4}
    \definition{v.}{esfriar | tornar ou tornar-se frio | deixar esfriar pelo ar}
  \end{phonetics}
\end{entry}

\begin{entry}{凉水}{10,4}[Radicais ⼎、⽔]
  \begin{phonetics}{凉水}{liang2 shui3}[][HSK 3]
    \definition{s.}{água fria | água gelada | água não fervida}
  \end{phonetics}
\end{entry}

\begin{entry}{凉快}{10,7}[Radicais ⼎、⼼]
  \begin{phonetics}{凉快}{liang2kuai5}[][HSK 2]
    \definition{adj.}{agradável e frio | agradavelmente fresco}
  \end{phonetics}
\end{entry}

\begin{entry}{凉鞋}{10,15}[Radicais ⼎、⾰]
  \begin{phonetics}{凉鞋}{liang2xie2}
    \definition{s.}{sandália | alpargata | alpercata | alparca}
  \end{phonetics}
\end{entry}

\begin{entry}{剧场}{10,6}[Radicais ⼑、⼟]
  \begin{phonetics}{剧场}{ju4 chang3}[][HSK 3]
    \definition[个,坐]{s.}{teatro; um lugar para apresentações teatrais, canto e dança, etc.}
  \end{phonetics}
\end{entry}

\begin{entry}{原木}{10,4}[Radicais ⼚、⽊]
  \begin{phonetics}{原木}{yuan2mu4}
    \definition{s.}{registro | \emph{logs}}
  \end{phonetics}
\end{entry}

\begin{entry}{原因}{10,6}[Radicais ⼚、⼞]
  \begin{phonetics}{原因}{yuan2yin1}[][HSK 2]
    \definition[个]{s.}{causa | razão | motivo}
  \end{phonetics}
\end{entry}

\begin{entry}{原色}{10,6}[Radicais ⼚、⾊]
  \begin{phonetics}{原色}{yuan2 se4}
    \definition{s.}{cor primária}
  \end{phonetics}
\end{entry}

\begin{entry}{原来}{10,7}[Radicais ⼚、⽊]
  \begin{phonetics}{原来}{yuan2lai2}[][HSK 2]
    \definition{adv.}{originalmente | como se vê | na verdade}
    \definition{v.}{vir a ser}
  \end{phonetics}
\end{entry}

\begin{entry}{原理}{10,11}[Radicais ⼚、⽟]
  \begin{phonetics}{原理}{yuan2li3}
    \definition{s.}{princípio | teoria}
  \end{phonetics}
\end{entry}

\begin{entry}{哥}{10}[Radical ⼝]
  \begin{phonetics}{哥}{ge1}[][HSK 1]
    \definition{s.}{irmão mais velho}
    \seeref{哥哥}{ge1 ge5}
  \end{phonetics}
\end{entry}

\begin{entry}{哥们}{10,5}[Radicais ⼝、⼈]
  \begin{phonetics}{哥们}{ge1men5}
    \definition{expr.}{\emph{Brothers!}}
    \definition{s.}{(coloquial) cara | irmão (forma diminuta de tratamento entre homens)}
  \end{phonetics}
\end{entry}

\begin{entry}{哥哥}{10,10}[Radicais ⼝、⼝]
  \begin{phonetics}{哥哥}{ge1 ge5}[][HSK 1]
    \definition[个,位]{s.}{irmão mais velho}
  \end{phonetics}
\end{entry}

\begin{entry}{哥斯拉}{10,12,8}[Radicais ⼝、⽄、⼿]
  \begin{phonetics}{哥斯拉}{ge1si1la1}
    \definition*{s.}{Godzilla}
  \seealsoref{酷斯拉}{ku4si1la1}
  \end{phonetics}
\end{entry}

\begin{entry}{哦}{10}[Radical ⼝]
  \begin{phonetics}{哦}{e2}
    \definition{v.}{entoar cântico}
  \end{phonetics}
  \begin{phonetics}{哦}{o2}
    \definition{interj.}{Oh! (indicando dúvida ou surpresa)}
  \end{phonetics}
  \begin{phonetics}{哦}{o4}
    \definition{interj.}{Oh! (indicando que acabou de aprender algo)}
  \end{phonetics}
  \begin{phonetics}{哦}{o5}
    \definition{part.}{final da frase que transmite informalidade, calor, simpatia ou intimidade; também pode indicar que alguém está declarando um fato de que a outra pessoa não está ciente}
  \end{phonetics}
\end{entry}

\begin{entry}{哭}{10}[Radical ⼝]
  \begin{phonetics}{哭}{ku1}[][HSK 2]
    \definition{v.}{chorar}
  \end{phonetics}
\end{entry}

\begin{entry}{哭墙}{10,14}[Radicais ⼝、⼟]
  \begin{phonetics}{哭墙}{ku1qiang2}
    \definition*{s.}{Muro das Lamentações (Jerusalém)}
  \end{phonetics}
\end{entry}

\begin{entry}{哮喘}{10,12}[Radicais ⼝、⼝]
  \begin{phonetics}{哮喘}{xiao4chuan3}
    \definition{s.}{asma}
  \end{phonetics}
\end{entry}

\begin{entry}{哲理}{10,11}[Radicais ⼝、⽟]
  \begin{phonetics}{哲理}{zhe2li3}
    \definition{s.}{filosofia | teoria filosófica}
  \end{phonetics}
\end{entry}

\begin{entry}{唇}{10}[Radical ⼝]
  \begin{phonetics}{唇}{chun2}
    \definition{s.}{lábios}
  \end{phonetics}
\end{entry}

\begin{entry}{唐人街}{10,2,12}[Radicais ⼝、⼈、⾏]
  \begin{phonetics}{唐人街}{tang2ren2 jie1}
    \definition*{s.}{Bairro Chinês | \emph{Chinatown}}
  \seealsoref{中国城}{zhong1guo2cheng2}
  \end{phonetics}
\end{entry}

\begin{entry}{啊}{10}[Radical ⼝]
  \begin{phonetics}{啊}{a1}[][HSK 2]
    \definition{interj.}{Ah! | Oh! | interjeição de surpresa}
  \end{phonetics}
  \begin{phonetics}{啊}{a2}[][HSK 2]
    \definition{interj.}{Eh? | Que? | interjeição expressando dúvida ou exigindo resposta}
  \end{phonetics}
  \begin{phonetics}{啊}{a3}[][HSK 2]
    \definition{interj.}{Eh? | Meu! | E aí? | Que? | interjeição de surpresa ou dúvida}
  \end{phonetics}
  \begin{phonetics}{啊}{a4}[][HSK 2]
    \definition{interj.}{Ah! | OK! | Oh, é você! | Hum! | expressão de reconhecimento | interjeição de acordo}
  \end{phonetics}
  \begin{phonetics}{啊}{a5}[][HSK 2]
    \definition{adv.}{assim por diante}
    \definition{part.}{no final de sentença para expressar admiração | no final de sentença mostrando afirmação, aprovação, urgência, aconselhamento, etc. | no final de sentença para indicar uma pergunta | para pausar ligeiramente uma frase, chamando a atenção para as palavras seguintes | após cada um dos itens listados}
  \end{phonetics}
\end{entry}

\begin{entry}{啊呀}{10,7}[Radicais ⼝、⼝]
  \begin{phonetics}{啊呀}{a1ya1}
    \definition{interj.}{Oh meu Deus! | interjeição de surpresa}
  \end{phonetics}
\end{entry}

\begin{entry}{啊哟}{10,9}[Radicais ⼝、⼝]
  \begin{phonetics}{啊哟}{a1yo5}
    \definition{interj.}{Meu Deus! | Oh! | Ai! | interjeição de surpresa ou dor}
  \end{phonetics}
\end{entry}

\begin{entry}{埋伏}{10,6}[Radicais ⼟、⼈]
  \begin{phonetics}{埋伏}{mai2fu2}
    \definition{s.}{emboscada}
    \definition{v.}{emboscar}
  \end{phonetics}
\end{entry}

\begin{entry}{夏天}{10,4}[Radicais ⼢、⼤]
  \begin{phonetics}{夏天}{xia4 tian1}[][HSK 2]
    \definition[个]{s.}{verão}
  \end{phonetics}
\end{entry}

\begin{entry}{夏日}{10,4}[Radicais ⼢、⽇]
  \begin{phonetics}{夏日}{xia4ri4}
    \definition{s.}{horário de verão}
  \end{phonetics}
\end{entry}

\begin{entry}{套}{10}[Radical ⼤]
  \begin{phonetics}{套}{tao4}[][HSK 2]
    \definition{clas.}{para conjuntos, coleções}
    \definition{s.}{cobertura | fórmula | laço de corda}
    \definition{v.}{cobrir | envolver | intercalar | sobrepor}
  \end{phonetics}
\end{entry}

\begin{entry}{套问}{10,6}[Radicais ⼤、⾨]
  \begin{phonetics}{套问}{tao4wen4}
    \definition{s.}{retórica}
    \definition{v.}{descobrir por meio de questionamento indireto diplomático}
  \end{phonetics}
\end{entry}

\begin{entry}{孬}{10}[Radical ⼥]
  \begin{phonetics}{孬}{nao1}
    \definition{adj.}{(dialeto) não (é) bom (contração de 不+好)}
  \end{phonetics}
\end{entry}

\begin{entry}{害}{10}[Radical ⼧]
  \begin{phonetics}{害}{hai4}
    \definition{s.}{dano | mal | calamidade}
    \definition{v.}{causar danos a | causar problemas para}
  \end{phonetics}
\end{entry}

\begin{entry}{害怕}{10,8}[Radicais ⼧、⼼]
  \begin{phonetics}{害怕}{hai4pa4}[][HSK 3]
    \definition{v.}{estar assustado; ter medo}
  \end{phonetics}
\end{entry}

\begin{entry}{害羞}{10,10}[Radicais ⼧、⽺]
  \begin{phonetics}{害羞}{hai4xiu1}
    \definition{adj.}{tímido | envergonhado}
  \end{phonetics}
\end{entry}

\begin{entry}{家}{10}[Radical ⼧]
  \begin{phonetics}{家}{jia1}[][HSK 1,2]
    \definition{clas.}{para famílias ou empresas}
    \definition{pron.}{(educado) meu (irmã, tio, etc.)}
    \definition[个]{s.}{casa | família}
    \definition{suf.}{sufixo substantivo para designar um especialista em alguma atividade, como um músico ou revolucionário, para designar uma profissão como em -eiro, -ista}
  \end{phonetics}
\end{entry}

\begin{entry}{家人}{10,2}[Radicais ⼧、⼈]
  \begin{phonetics}{家人}{jia1ren2}[][HSK 1]
    \definition{s.}{(a) família | membro da família}
  \end{phonetics}
\end{entry}

\begin{entry}{家乡}{10,3}[Radicais ⼧、⼄]
  \begin{phonetics}{家乡}{jia1xiang1}[][HSK 3]
    \definition[个]{s.}{cidade natal}
  \end{phonetics}
\end{entry}

\begin{entry}{家长}{10,4}[Radicais ⼧、⾧]
  \begin{phonetics}{家长}{jia1 zhang3}[][HSK 2]
    \definition[位,名,个]{s.}{pais | patriarca | guardião}
  \end{phonetics}
\end{entry}

\begin{entry}{家伙}{10,6}[Radicais ⼧、⼈]
  \begin{phonetics}{家伙}{jia1huo5}
    \definition{s.}{prato, implemento ou móvel doméstico | animal doméstico | (coloquial) o cara | indivíduo | arma}
  \end{phonetics}
\end{entry}

\begin{entry}{家里}{10,7}[Radicais ⼧、⾥]
  \begin{phonetics}{家里}{jia1 li3}[][HSK 1]
    \definition{adv.}{em casa}
  \end{phonetics}
\end{entry}

\begin{entry}{家具}{10,8}[Radicais ⼧、⼋]
  \begin{phonetics}{家具}{jia1ju4}[][HSK 3]
    \definition[件,套]{s.}{móveis; mobiliário de casa}
  \end{phonetics}
\end{entry}

\begin{entry}{家庭}{10,9}[Radicais ⼧、⼴]
  \begin{phonetics}{家庭}{jia1ting2}[][HSK 2]
    \definition[个,户]{s.}{família}
  \end{phonetics}
\end{entry}

\begin{entry}{家俱}{10,10}[Radicais ⼧、⼈]
  \begin{phonetics}{家俱}{jia1ju4}
    \variantof{家具}
  \end{phonetics}
\end{entry}

\begin{entry}{家属}{10,12}[Radicais ⼧、⼫]
  \begin{phonetics}{家属}{jia1shu3}[][HSK 3]
    \definition{s.}{membros da família; dependentes (familiares)}
  \end{phonetics}
\end{entry}

\begin{entry}{容易}{10,8}[Radicais ⼧、⽇]
  \begin{phonetics}{容易}{rong2yi4}[][HSK 3]
    \definition{adj.}{fácil; simples | provável; responsável; apto}
  \end{phonetics}
\end{entry}

\begin{entry}{容貌}{10,14}[Radicais ⼧、⾘]
  \begin{phonetics}{容貌}{rong2mao4}
    \definition{s.}{aparência | aspecto | características}
  \end{phonetics}
\end{entry}

\begin{entry}{宽影片}{10,15,4}[Radicais ⼧、⼺、⽚]
  \begin{phonetics}{宽影片}{kuan1ying3pian4}
    \definition{s.}{filme \emph{widescreen}}
  \end{phonetics}
\end{entry}

\begin{entry}{宾馆}{10,11}[Radicais ⼧、⾷]
  \begin{phonetics}{宾馆}{bin1guan3}
    \definition[个,家]{s.}{casa de hóspedes | hotel}
  \end{phonetics}
\end{entry}

\begin{entry}{射}{10}[Radical ⼨]
  \begin{phonetics}{射}{she4}
    \definition{v.}{atirar | lançar}
  \end{phonetics}
\end{entry}

\begin{entry}{展开}{10,4}[Radicais ⼫、⼶]
  \begin{phonetics}{展开}{zhan3kai1}[][HSK 3]
    \definition{s.}{desenvolvimento; expansão; explosão; evolução}
    \definition{v.}{desenvolver; espalhar; desdobrar; abrir; desenrolar; amplificar; esticar; ventilar | lançar; desdobrar; desenvolver; executar}
  \end{phonetics}
\end{entry}

\begin{entry}{展示}{10,5}[Radicais ⼫、⽰]
  \begin{phonetics}{展示}{zhan3shi4}
    \definition{v.}{revelar | mostrar | exibir}
  \end{phonetics}
\end{entry}

\begin{entry}{席卷}{10,8}[Radicais ⼱、⼙]
  \begin{phonetics}{席卷}{xi2juan3}
    \definition{v.}{engolfar | varrer | levar tudo para fora}
  \end{phonetics}
\end{entry}

\begin{entry}{座}{10}[Radical ⼴]
  \begin{phonetics}{座}{zuo4}[][HSK 2]
    \definition{clas.}{frequentemente usado para objetos maiores ou fixos}
    \definition{s.}{assento | lugar | base | suporte | pedestal | constelação}
  \end{phonetics}
\end{entry}

\begin{entry}{座子}{10,3}[Radicais ⼴、⼦]
  \begin{phonetics}{座子}{zuo4zi5}
    \definition{s.}{soquete | pedestal | sela}
  \end{phonetics}
\end{entry}

\begin{entry}{座位}{10,7}[Radicais ⼴、⼈]
  \begin{phonetics}{座位}{zuo4wei4}[][HSK 2]
    \definition[个]{s.}{assento | lugar}
  \end{phonetics}
\end{entry}

\begin{entry}{座标}{10,9}[Radicais ⼴、⽊]
  \begin{phonetics}{座标}{zuo4biao1}
    \variantof{坐标}
  \end{phonetics}
\end{entry}

\begin{entry}{徒手}{10,4}[Radicais ⼻、⼿]
  \begin{phonetics}{徒手}{tu2shou3}
    \definition{adj.}{com as mãos vazias | desarmado | mão livre (desenho) | lutando mão-a-mão}
  \end{phonetics}
\end{entry}

\begin{entry}{恋爱}{10,10}[Radicais ⼼、⽖]
  \begin{phonetics}{恋爱}{lian4'ai4}
    \definition[个,场]{s.}{amor (romântico)}
    \definition{v.}{sentir-se profundamente apegado a}
  \end{phonetics}
\end{entry}

\begin{entry}{恐龙}{10,5}[Radicais ⼼、⿓]
  \begin{phonetics}{恐龙}{kong3long2}
    \definition[头,只]{s.}{dinossauro}
  \end{phonetics}
\end{entry}

\begin{entry}{恐怕}{10,8}[Radicais ⼼、⼼]
  \begin{phonetics}{恐怕}{kong3pa4}[][HSK 3]
    \definition{adv.}{talvez; provavelmente; pode ser | por medo de}
    \definition{v.}{ter medo de; temer; recear}
  \end{phonetics}
\end{entry}

\begin{entry}{恐怖主义}{10,8,5,3}[Radicais ⼼、⼼、⼂、⼂]
  \begin{phonetics}{恐怖主义}{kong3bu4zhu3yi4}
    \definition{adj.}{terrorista}
    \definition{s.}{terrorismo}
  \end{phonetics}
\end{entry}

\begin{entry}{恩赐}{10,12}[Radicais ⼼、⾙]
  \begin{phonetics}{恩赐}{en1ci4}
    \definition{s.}{favor | caridade}
    \definition{v.}{conceder (favor, caridade)}
  \end{phonetics}
\end{entry}

\begin{entry}{恶心}{10,4}[Radicais ⼼、⼼]
  \begin{phonetics}{恶心}{e3xin1}
    \definition{adj.}{nauseante | repugnante}
    \definition{s.}{enjôo | náusea | repugnância}
    \definition{v.}{envergonhar (deliberadamente) | sentir-se doente}
  \end{phonetics}
  \begin{phonetics}{恶心}{e4xin1}
    \definition{s.}{mau hábito | hábito vicioso | vício}
  \end{phonetics}
\end{entry}

\begin{entry}{扇子}{10,3}[Radicais ⼾、⼦]
  \begin{phonetics}{扇子}{shan4zi5}
    \definition[把]{s.}{leque | abano | abanador}
  \end{phonetics}
\end{entry}

\begin{entry}{拳王}{10,4}[Radicais ⼿、⽟]
  \begin{phonetics}{拳王}{quan2wang2}
    \definition{s.}{pugilista | boxeador}
  \end{phonetics}
\end{entry}

\begin{entry}{拳法}{10,8}[Radicais ⼿、⽔]
  \begin{phonetics}{拳法}{quan2fa3}
    \definition{s.}{boxe | luta}
  \end{phonetics}
\end{entry}

\begin{entry}{拿}{10}[Radical ⼿]
  \begin{phonetics}{拿}{na2}[][HSK 1]
    \definition{part.}{usado da mesma forma que 把: para marcar o seguinte substantivo seguinte como objeto direto}
    \definition{v.}{segurar | tomar | pegar em}
  \end{phonetics}
\end{entry}

\begin{entry}{拿出}{10,5}[Radicais ⼿、⼐]
  \begin{phonetics}{拿出}{na2 chu1}[][HSK 2]
    \definition{v.}{apresentar (evidências) | prover | apresentar (uma proposta) | colocar para fora | retirar}
  \end{phonetics}
\end{entry}

\begin{entry}{拿到}{10,8}[Radicais ⼿、⼑]
  \begin{phonetics}{拿到}{na2 dao4}[][HSK 2]
    \definition{v.}{pegar | obter}
  \end{phonetics}
\end{entry}

\begin{entry}{挫折}{10,7}[Radicais ⼿、⼿]
  \begin{phonetics}{挫折}{cuo4zhe2}
    \definition{s.}{revés | reverso | derrota | frustração | decepção}
    \definition{v.}{frustrar | desencorajar | subjugar}
  \end{phonetics}
\end{entry}

\begin{entry}{捞}{10}[Radical ⼿]
  \begin{phonetics}{捞}{lao1}
    \definition{v.}{pescar | dragar}
  \end{phonetics}
\end{entry}

\begin{entry}{捡}{10}[Radical ⼿]
  \begin{phonetics}{捡}{jian3}
    \definition{v.}{apanhar | recolher | coletar}
  \end{phonetics}
\end{entry}

\begin{entry}{换}{10}[Radical ⼿]
  \begin{phonetics}{换}{huan4}[][HSK 2]
    \definition{v.}{mudar | trocar | substituir | converter (moedas)}
  \end{phonetics}
\end{entry}

\begin{entry}{换钱}{10,10}[Radicais ⼿、⾦]
  \begin{phonetics}{换钱}{huan4qian2}
    \definition{v.+compl.}{trocar dinheiro (em pequenas valores ou em outra moeda) | trocar (mercadorias) por dinheiro | vender}
  \end{phonetics}
\end{entry}

\begin{entry}{效果}{10,8}[Radicais ⽁、⽊]
  \begin{phonetics}{效果}{xiao4guo3}[][HSK 3]
    \definition[种,个]{s.}{efeito; resultado | efeitos sonoros; vários sons ou fenômenos naturais criados para combinar com o enredo em dramas e filmes, como vento e chuva, tiros, fogo, neve, etc.}
  \end{phonetics}
\end{entry}

\begin{entry}{旁边}{10,5}[Radicais ⽅、⾡]
  \begin{phonetics}{旁边}{pang2bian1}[][HSK 1]
    \definition{adv.}{junto a | próximo de | ao lado}
  \end{phonetics}
\end{entry}

\begin{entry}{旅行}{10,6}[Radicais ⽅、⾏]
  \begin{phonetics}{旅行}{lv3xing2}[][HSK 2]
    \definition{v.}{viajar}
  \end{phonetics}
\end{entry}

\begin{entry}{旅行社}{10,6,7}[Radicais ⽅、⾏、⽰]
  \begin{phonetics}{旅行社}{lv3 xing2 she4}[][HSK 3]
    \definition[家]{s.}{agência de viagens}
  \end{phonetics}
\end{entry}

\begin{entry}{旅客}{10,9}[Radicais ⽅、⼧]
  \begin{phonetics}{旅客}{lv3 ke4}[][HSK 2]
    \definition{s.}{viajante | turista}
  \end{phonetics}
\end{entry}

\begin{entry}{旅馆}{10,11}[Radicais ⽅、⾷]
  \begin{phonetics}{旅馆}{lv3 guan3}[][HSK 3]
    \definition[家,个,所]{s.}{pousada; hotel}
  \end{phonetics}
\end{entry}

\begin{entry}{旅游}{10,12}[Radicais ⽅、⽔]
  \begin{phonetics}{旅游}{lv3you2}[][HSK 2]
    \definition[趟,次,个]{s.}{jornada | viagem}
    \definition{v.}{viajar}
  \end{phonetics}
\end{entry}

\begin{entry}{旅程}{10,12}[Radicais ⽅、⽲]
  \begin{phonetics}{旅程}{lv3cheng2}
    \definition{s.}{jornada | viagem}
  \end{phonetics}
\end{entry}

\begin{entry}{晒干}{10,3}[Radicais ⽇、⼲]
  \begin{phonetics}{晒干}{shai4gan1}
    \definition{v.}{secar ao sol}
  \end{phonetics}
\end{entry}

\begin{entry}{校}{10}[Radical ⽊]
  \begin{phonetics}{校}{jiao4}
    \definition{v.}{verificar | comparar | revisar}
  \end{phonetics}
  \begin{phonetics}{校}{xiao4}
    \definition[所]{s.}{oficial militar | escola}
  \end{phonetics}
\end{entry}

\begin{entry}{校长}{10,4}[Radicais ⽊、⾧]
  \begin{phonetics}{校长}{xiao4zhang3}[][HSK 2]
    \definition[个,位,名]{s.}{diretor de escola | reitor (universidade)}
  \end{phonetics}
\end{entry}

\begin{entry}{校园}{10,7}[Radicais ⽊、⼞]
  \begin{phonetics}{校园}{xiao4 yuan2}[][HSK 2]
    \definition{s.}{campus}
  \end{phonetics}
\end{entry}

\begin{entry}{校服}{10,8}[Radicais ⽊、⽉]
  \begin{phonetics}{校服}{xiao4fu2}
    \definition{s.}{uniforme escolar}
  \end{phonetics}
\end{entry}

\begin{entry}{校规}{10,8}[Radicais ⽊、⾒]
  \begin{phonetics}{校规}{xiao4gui1}
    \definition{s.}{regras e regulamentos escolares}
  \end{phonetics}
\end{entry}

\begin{entry}{校监}{10,10}[Radicais ⽊、⽫]
  \begin{phonetics}{校监}{xiao4jian1}
    \definition{s.}{diretor | supervisor (de escola)}
  \end{phonetics}
\end{entry}

\begin{entry}{样}{10}[Radical ⽊]
  \begin{phonetics}{样}{yang4}
    \definition{s.}{aparência | forma | modelo}
  \end{phonetics}
\end{entry}

\begin{entry}{样儿}{10,2}[Radicais ⽊、⼉]
  \begin{phonetics}{样儿}{yang4r5}
    \definition{s.}{aparência | forma | modelo}
  \seealsoref{样子}{yang4zi5}
  \end{phonetics}
\end{entry}

\begin{entry}{样子}{10,3}[Radicais ⽊、⼦]
  \begin{phonetics}{样子}{yang4zi5}[][HSK 2]
    \definition{s.}{aparência | forma | modelo}
  \seealsoref{样儿}{yang4r5}
  \end{phonetics}
\end{entry}

\begin{entry}{样品}{10,9}[Radicais ⽊、⼝]
  \begin{phonetics}{样品}{yang4pin3}
    \definition{s.}{amostra | espécime}
  \end{phonetics}
\end{entry}

\begin{entry}{样样}{10,10}[Radicais ⽊、⽊]
  \begin{phonetics}{样样}{yang4yang4}
    \definition{adv.}{todos os tipos}
  \end{phonetics}
\end{entry}

\begin{entry}{样章}{10,11}[Radicais ⽊、⾳]
  \begin{phonetics}{样章}{yang4zhang1}
    \definition{s.}{capítulo de amostra}
  \end{phonetics}
\end{entry}

\begin{entry}{核}{10}[Radical ⽊]
  \begin{phonetics}{核}{he2}
    \definition{adj.}{nuclear}
    \definition{s.}{poço | pedra | núcleo}
    \definition{v.}{examinar | checar | verificar}
  \end{phonetics}
\end{entry}

\begin{entry}{根本}{10,5}[Radicais ⽊、⽊]
  \begin{phonetics}{根本}{gen1ben3}[][HSK 3]
    \definition{adj.}{básico; essencial; fundamental}
    \definition{adv.}{sempre; simplesmente; absolutamente; de qualquer modo | radically; thoroughly}
    \definition[个]{s.}{base; raiz; fundação}
  \end{phonetics}
\end{entry}

\begin{entry}{根据}{10,11}[Radicais ⽊、⼿]
  \begin{phonetics}{根据}{gen1ju4}
    \definition{prep.}{de acordo com}
    \definition[个]{s.}{base | fundação}
  \end{phonetics}
\end{entry}

\begin{entry}{格兰菜}{10,5,11}[Radicais ⽊、⼋、⾋]
  \begin{phonetics}{格兰菜}{ge2lan2cai4}
    \definition{s.}{brócolis chinês | couve chinesa | mostarda}
    \seeref{芥蓝}{gai4lan2}
  \end{phonetics}
\end{entry}

\begin{entry}{格外}{10,5}[Radicais ⽊、⼣]
  \begin{phonetics}{格外}{ge2wai4}
    \definition{adv.}{especialmente | particularmente | adicionalmente | de outra forma}
  \end{phonetics}
\end{entry}

\begin{entry}{栽}{10}[Radical ⽊]
  \begin{phonetics}{栽}{zai1}
    \definition{v.}{cultivar | plantar}
  \end{phonetics}
\end{entry}

\begin{entry}{栽种}{10,9}[Radicais ⽊、⽲]
  \begin{phonetics}{栽种}{zai1zhong4}
    \definition{v.}{plantar}
  \end{phonetics}
\end{entry}

\begin{entry}{栽倒}{10,10}[Radicais ⽊、⼈]
  \begin{phonetics}{栽倒}{zai1dao3}
    \definition{v.}{cair | sofrer uma queda}
  \end{phonetics}
\end{entry}

\begin{entry}{栽赃}{10,10}[Radicais ⽊、⾙]
  \begin{phonetics}{栽赃}{zai1zang1}
    \definition{v.}{enquadrar alguém (plantar provas nele)}
  \end{phonetics}
\end{entry}

\begin{entry}{栽培}{10,11}[Radicais ⽊、⼟]
  \begin{phonetics}{栽培}{zai1pei2}
    \definition{v.}{cultivar | educar | patrocinar | treinar}
  \end{phonetics}
\end{entry}

\begin{entry}{栽培种}{10,11,9}[Radicais ⽊、⼟、⽲]
  \begin{phonetics}{栽培种}{zai1pei2 zhong3}
    \definition{s.}{espécies cultivadas}
  \end{phonetics}
\end{entry}

\begin{entry}{栽植}{10,12}[Radicais ⽊、⽊]
  \begin{phonetics}{栽植}{zai1zhi2}
    \definition{v.}{plantar | transplantar}
  \end{phonetics}
\end{entry}

\begin{entry}{桃}{10}[Radical ⽊]
  \begin{phonetics}{桃}{tao2}
    \definition{s.}{pêssego}
  \end{phonetics}
\end{entry}

\begin{entry}{桌}{10}[Radical ⽊]
  \begin{phonetics}{桌}{zhuo1}
    \definition{clas.}{para mesas de convidados em um banquete etc.}
    \definition{s.}{mesa}
  \end{phonetics}
\end{entry}

\begin{entry}{桌子}{10,3}[Radicais ⽊、⼦]
  \begin{phonetics}{桌子}{zhuo1zi5}[][HSK 1]
    \definition[张,套]{s.}{mesa}
  \end{phonetics}
\end{entry}

\begin{entry}{桌布}{10,5}[Radicais ⽊、⼱]
  \begin{phonetics}{桌布}{zhuo1bu4}
    \definition[条,块,张]{s.}{(computação) plano de fundo da área de trabalho | toalha de mesa | papel de parede}
  \end{phonetics}
\end{entry}

\begin{entry}{桌机}{10,6}[Radicais ⽊、⽊]
  \begin{phonetics}{桌机}{zhuo1ji1}
    \definition{s.}{computador \emph{desktop}}
  \end{phonetics}
\end{entry}

\begin{entry}{桌灯}{10,6}[Radicais ⽊、⽕]
  \begin{phonetics}{桌灯}{zhuo1deng1}
    \definition{s.}{luminária | lâmpada de mesa}
  \end{phonetics}
\end{entry}

\begin{entry}{桌面}{10,9}[Radicais ⽊、⾯]
  \begin{phonetics}{桌面}{zhuo1mian4}
    \definition{s.}{área de trabalho | mesa}
  \end{phonetics}
\end{entry}

\begin{entry}{桌球}{10,11}[Radicais ⽊、⽟]
  \begin{phonetics}{桌球}{zhuo1qiu2}
    \definition{s.}{bilhar | sinuca | mesa de ping-pong}
  \end{phonetics}
\end{entry}

\begin{entry}{桌游}{10,12}[Radicais ⽊、⽔]
  \begin{phonetics}{桌游}{zhuo1you2}
    \definition{s.}{jogo de tabuleiro}
  \end{phonetics}
\end{entry}

\begin{entry}{桑}{10}[Radical ⽊]
  \begin{phonetics}{桑}{sang1}
    \definition*{s.}{sobrenome Sang}
    \definition{s.}{amoreira}
  \end{phonetics}
\end{entry}

\begin{entry}{桑巴舞}{10,4,14}[Radicais ⽊、⼰、⾇]
  \begin{phonetics}{桑巴舞}{sang1ba1wu3}
    \definition{s.}{samba}
  \end{phonetics}
\end{entry}

\begin{entry}{桑树}{10,9}[Radicais ⽊、⽊]
  \begin{phonetics}{桑树}{sang1shu4}
    \definition{s.}{amoreira, suas folhas são utilizadas para alimentar bichos-da-seda}
  \end{phonetics}
\end{entry}

\begin{entry}{桥}{10}[Radical ⽊]
  \begin{phonetics}{桥}{qiao2}[][HSK 3]
    \definition*{s.}{sobrenome Qiao}
    \definition[座]{s.}{ponte}
  \end{phonetics}
\end{entry}

\begin{entry}{桩}{10}[Radical ⽊]
  \begin{phonetics}{桩}{zhuang1}
    \definition{clas.}{para eventos, casos, transações, assuntos, etc.}
    \definition{s.}{toco | estaca | pilha}
  \end{phonetics}
\end{entry}

\begin{entry}{欱}{10}[Radical ⽋]
  \begin{phonetics}{欱}{he1}
    \variantof{喝}
  \end{phonetics}
\end{entry}

\begin{entry}{氧}{10}[Radical ⽓]
  \begin{phonetics}{氧}{yang3}
    \definition{s.}{oxigênio}
  \end{phonetics}
\end{entry}

\begin{entry}{流}{10}[Radical ⽔]
  \begin{phonetics}{流}{liu2}[][HSK 2]
    \definition[名,个]{s.}{fluxo de água | correnteza | córrego | algo que se assemelha a um fluxo de água | corrente | fluxo | classe | grau | taxa (de variação)}
    \definition{v.}{fluir
deriva; mover; vagar
espalhar
degenerar; mudar para pior
enviar para o exílio; banir}
  \end{phonetics}
\end{entry}

\begin{entry}{流水}{10,4}[Radicais ⽔、⽔]
  \begin{phonetics}{流水}{liu2shui3}
    \definition{s.}{água corrente | (negócio) rotatividade}
  \end{phonetics}
\end{entry}

\begin{entry}{流行}{10,6}[Radicais ⽔、⾏]
  \begin{phonetics}{流行}{liu2xing2}[][HSK 2]
    \definition{adj.}{(estilo de roupa, música, etc.) popular, na moda}
    \definition{v.}{(doença contagiosa, etc.) espalhar | propagar}
  \end{phonetics}
\end{entry}

\begin{entry}{流利}{10,7}[Radicais ⽔、⼑]
  \begin{phonetics}{流利}{liu2li4}[][HSK 2]
    \definition{adj.}{fluente (em uma língua)}
  \end{phonetics}
\end{entry}

\begin{entry}{流星}{10,9}[Radicais ⽔、⽇]
  \begin{phonetics}{流星}{liu2xing1}
    \definition{s.}{meteoro | estrela cadente}
  \end{phonetics}
\end{entry}

\begin{entry}{浙江}{10,6}[Radicais ⽔、⽔]
  \begin{phonetics}{浙江}{zhe4jiang1}
    \definition*{s.}{Zhejiang}
  \end{phonetics}
\end{entry}

\begin{entry}{浪花}{10,7}[Radicais ⽔、⾋]
  \begin{phonetics}{浪花}{lang4hua1}
    \definition[朵]{s.}{\emph{spray} | \emph{spray} do oceano | (figurativo) acontecimentos de sua vida}
  \end{phonetics}
\end{entry}

\begin{entry}{浪费}{10,9}[Radicais ⽔、⾙]
  \begin{phonetics}{浪费}{lang4fei4}[][HSK 3]
    \definition{adj.}{desperdiçado}
    \definition{adv.}{extravagantemente}
    \definition{v.}{desperdiçar; dissipar; esbanjar; ser extravagante}
  \end{phonetics}
\end{entry}

\begin{entry}{浪漫}{10,14}[Radicais ⽔、⽔]
  \begin{phonetics}{浪漫}{lang4man4}
    \definition{adj.}{romântico}
  \end{phonetics}
\end{entry}

\begin{entry}{浮力}{10,2}[Radicais ⽔、⼒]
  \begin{phonetics}{浮力}{fu2li4}
    \definition{s.}{flutuabilidade}
  \end{phonetics}
\end{entry}

\begin{entry}{浮图}{10,8}[Radicais ⽔、⼞]
  \begin{phonetics}{浮图}{fu2tu2}
    \definition*{s.}{Termo alternativo para 佛陀}
    \variantof{浮屠}
  \seealsoref{佛陀}{fo2tuo2}
  \end{phonetics}
\end{entry}

\begin{entry}{浮屠}{10,11}[Radicais ⽔、⼫]
  \begin{phonetics}{浮屠}{fu2tu2}
    \definition*{s.}{Buda | Templo (Stupa) Budista (transliteração de Pali Thuo)}
  \end{phonetics}
\end{entry}

\begin{entry}{海}{10}[Radical ⽔]
  \begin{phonetics}{海}{hai3}[][HSK 2]
    \definition*{s.}{sobrenome Hai}
    \definition[个,片]{s.}{mar | oceano}
  \end{phonetics}
\end{entry}

\begin{entry}{海水}{10,4}[Radicais ⽔、⽔]
  \begin{phonetics}{海水}{hai3shui3}
    \definition{s.}{água do mar}
  \end{phonetics}
\end{entry}

\begin{entry}{海风}{10,4}[Radicais ⽔、⾵]
  \begin{phonetics}{海风}{hai3feng1}
    \definition{s.}{brisa do mar | vento que vem do mar}
  \end{phonetics}
\end{entry}

\begin{entry}{海边}{10,5}[Radicais ⽔、⾡]
  \begin{phonetics}{海边}{hai3 bian1}[][HSK 2]
    \definition{s.}{costa marítima | litoral | beira-mar | praia}
  \end{phonetics}
\end{entry}

\begin{entry}{海关}{10,6}[Radicais ⽔、⼋]
  \begin{phonetics}{海关}{hai3guan1}[][HSK 3]
    \definition{s.}{alfândega}
  \end{phonetics}
\end{entry}

\begin{entry}{海里}{10,7}[Radicais ⽔、⾥]
  \begin{phonetics}{海里}{hai3li3}
    \definition{s.}{milha náutica}
  \end{phonetics}
\end{entry}

\begin{entry}{海底}{10,8}[Radicais ⽔、⼴]
  \begin{phonetics}{海底}{hai3di3}
    \definition{adj.}{submarino}
    \definition{s.}{fundo do mar | solo oceânico | fundo do oceano}
  \end{phonetics}
\end{entry}

\begin{entry}{海鸥}{10,9}[Radicais ⽔、⿃]
  \begin{phonetics}{海鸥}{hai3'ou1}
    \definition{s.}{gaivota}
  \end{phonetics}
\end{entry}

\begin{entry}{海浪}{10,10}[Radicais ⽔、⽔]
  \begin{phonetics}{海浪}{hai3lang4}
    \definition{s.}{ondas do mar}
  \end{phonetics}
\end{entry}

\begin{entry}{海绵}{10,11}[Radicais ⽔、⽷]
  \begin{phonetics}{海绵}{hai3mian2}
    \definition{s.}{(zoologia) esponja do mar | esponja (feita de poliéster ou celulose, etc.) | espuma de borracha}
  \end{phonetics}
\end{entry}

\begin{entry}{海棠}{10,12}[Radicais ⽔、⽊]
  \begin{phonetics}{海棠}{hai3tang2}
    \definition{s.}{begônia}
  \end{phonetics}
\end{entry}

\begin{entry}{消失}{10,5}[Radicais ⽔、⼤]
  \begin{phonetics}{消失}{xiao1shi1}[][HSK 3]
    \definition{v.}{desaparecer; desvanecer; dissolver; dissipar; evaporar; sumir}
  \end{phonetics}
\end{entry}

\begin{entry}{消防}{10,6}[Radicais ⽔、⾩]
  \begin{phonetics}{消防}{xiao1fang2}
    \definition{s.}{combate a incêncios | controle de incêndios}
  \end{phonetics}
\end{entry}

\begin{entry}{消防员}{10,6,7}[Radicais ⽔、⾩、⼝]
  \begin{phonetics}{消防员}{xiao1fang2yuan2}
    \definition{s.}{bombeiro}
  \end{phonetics}
\end{entry}

\begin{entry}{消费}{10,9}[Radicais ⽔、⾙]
  \begin{phonetics}{消费}{xiao1fei4}[][HSK 3]
    \definition{v.}{gastar; consumir | consumir (recursos naturais)}
  \end{phonetics}
\end{entry}

\begin{entry}{消息}{10,10}[Radicais ⽔、⼼]
  \begin{phonetics}{消息}{xiao1xi5}[][HSK 3]
    \definition[个,条,篇]{s.}{notícias; informação}
  \end{phonetics}
\end{entry}

\begin{entry}{涨价}{10,6}[Radicais ⽔、⼈]
  \begin{phonetics}{涨价}{zhang3jia4}
    \definition{s.}{aumento de preços}
    \definition{v.+compl.}{avaliar (em valor) | dar preço | aumentar o preço}
  \end{phonetics}
\end{entry}

\begin{entry}{烈士}{10,3}[Radicais ⽕、⼠]
  \begin{phonetics}{烈士}{lie4shi4}
    \definition{s.}{mártir}
  \end{phonetics}
\end{entry}

\begin{entry}{烟}{10}[Radical ⽕]
  \begin{phonetics}{烟}{yan1}[][HSK 3]
    \definition*[竜]{s.}{sobrenome Yan}
    \definition[根]{s.}{fumaça; gases produzidos quando substâncias queimam | névoa; vapor; algo como fumaça | tabaco; planta de tabaco | fumo; cigarro; termo geral para cigarros, charutos, etc. | ópio}
    \definition{v.}{ficar irritado com a fumaça (olhos)}
  \end{phonetics}
\end{entry}

\begin{entry}{烟火}{10,4}[Radicais ⽕、⽕]
  \begin{phonetics}{烟火}{yan1huo3}
    \definition{s.}{fogo de artifício}
  \end{phonetics}
\end{entry}

\begin{entry}{烟叶}{10,5}[Radicais ⽕、⼝]
  \begin{phonetics}{烟叶}{yan1ye4}
    \definition{s.}{folha de tabaco}
  \end{phonetics}
\end{entry}

\begin{entry}{烟头}{10,5}[Radicais ⽕、⼤]
  \begin{phonetics}{烟头}{yan1tou2}
    \definition[根]{s.}{bituca de cigarro}
  \end{phonetics}
\end{entry}

\begin{entry}{烟囱}{10,7}[Radicais ⽕、⼞]
  \begin{phonetics}{烟囱}{yan1cong1}
    \definition{s.}{chaminé}
  \end{phonetics}
\end{entry}

\begin{entry}{烟花}{10,7}[Radicais ⽕、⾋]
  \begin{phonetics}{烟花}{yan1hua1}
    \definition{s.}{fogos de artifício}
  \end{phonetics}
\end{entry}

\begin{entry}{烟雨}{10,8}[Radicais ⽕、⾬]
  \begin{phonetics}{烟雨}{yan1yu3}
    \definition{s.}{chuvisco | garoa}
  \end{phonetics}
\end{entry}

\begin{entry}{烟草}{10,9}[Radicais ⽕、⾋]
  \begin{phonetics}{烟草}{yan1cao3}
    \definition{s.}{tabaco}
  \end{phonetics}
\end{entry}

\begin{entry}{烤}{10}[Radical ⽕]
  \begin{phonetics}{烤}{kao3}
    \definition{v.}{assar | grelhar}
  \end{phonetics}
\end{entry}

\begin{entry}{烤肉}{10,6}[Radicais ⽕、⾁]
  \begin{phonetics}{烤肉}{kao3rou4}
    \definition{s.}{churrasco}
  \end{phonetics}
\end{entry}

\begin{entry}{烧}{10}[Radical ⽕]
  \begin{phonetics}{烧}{shao1}
    \definition{s.}{febre}
    \definition{v.}{queimar | cozinhar | cozer | assar | aquecer | ferver (chá, água, etc.) | ter febre | (coloquial) deixar as coisas subirem à cabeça}
  \end{phonetics}
\end{entry}

\begin{entry}{烧烤}{10,10}[Radicais ⽕、⽕]
  \begin{phonetics}{烧烤}{shao1kao3}
    \definition{s.}{churrasco}
    \definition{v.}{assar}
  \end{phonetics}
\end{entry}

\begin{entry}{热}{10}[Radical ⽕]
  \begin{phonetics}{热}{re4}[][HSK 1]
    \definition{adj.}{quente (clima) | fervente | ardente | fervoroso}
    \definition{v.}{aquecer | ferver}
  \end{phonetics}
\end{entry}

\begin{entry}{热心}{10,4}[Radicais ⽕、⼼]
  \begin{phonetics}{热心}{re4xin1}
    \definition{adj.}{entusiasmado | ardente | zeloso}
  \end{phonetics}
\end{entry}

\begin{entry}{热血沸腾}{10,6,8,13}[Radicais ⽕、⾎、⽔、⾁]
  \begin{phonetics}{热血沸腾}{re4xue4fei4teng2}
    \definition{expr.}{ferver o sangue | apaixonar-se}
  \end{phonetics}
\end{entry}

\begin{entry}{热泪盈眶}{10,8,9,11}[Radicais ⽕、⽔、⽫、⽬]
  \begin{phonetics}{热泪盈眶}{re4lei4ying2kuang4}
    \definition{expr.}{olhos cheios de lágrimas de emoção | extremamente emocionado}
  \end{phonetics}
\end{entry}

\begin{entry}{热闹}{10,8}[Radicais ⽕、⾨]
  \begin{phonetics}{热闹}{re4nao5}
    \definition{adj.}{animado | movimentado com barulho e excitação}
  \end{phonetics}
\end{entry}

\begin{entry}{热烈}{10,10}[Radicais ⽕、⽕]
  \begin{phonetics}{热烈}{re4lie4}[][HSK 3]
    \definition{adj.}{caloroso; calorosamente; fervoroso; ardente; entusiasmado}
  \end{phonetics}
\end{entry}

\begin{entry}{热爱}{10,10}[Radicais ⽕、⽖]
  \begin{phonetics}{热爱}{re4'ai4}[][HSK 3]
    \definition{v.}{amar ardentemente; amar de coração; ter amor profundo por}
  \end{phonetics}
\end{entry}

\begin{entry}{热情}{10,11}[Radicais ⽕、⼼]
  \begin{phonetics}{热情}{re4qing2}[][HSK 2]
    \definition{adj.}{caloroso | fervoroso | entusiasmado}
    \definition{s.}{entusiasmo | ardor | devoção | calor | zelo}
  \end{phonetics}
\end{entry}

\begin{entry}{爱}{10}[Radical ⽖]
  \begin{phonetics}{爱}{ai4}[][HSK 1]
    \definition*{s.}{sobrenome Ai}
    \definition[个]{s.}{amor; afeição; afeição profunda; preocupação profunda; especialmente amor entre pessoas}
    \definition{v.}{amar; ter sentimentos profundos por pessoas ou coisas | gostar; gostar de; estar interessado em |  cuidar; valorizar; ter em alta estima; cuidar bem de | estar apto a; ter o hábito de}
  \end{phonetics}
\end{entry}

\begin{entry}{爱人}{10,2}[Radicais ⽖、⼈]
  \begin{phonetics}{爱人}{ai4ren5}[][HSK 2]
    \definition[个]{s.}{esposa | amor, amada}
  \end{phonetics}
\end{entry}

\begin{entry}{爱上}{10,3}[Radicais ⽖、⼀]
  \begin{phonetics}{爱上}{ai4shang4}
    \definition{v.}{apaixonar-se por | ser gentil com}
  \end{phonetics}
\end{entry}

\begin{entry}{爱心}{10,4}[Radicais ⽖、⼼]
  \begin{phonetics}{爱心}{ai4xin1}[][HSK 3]
    \definition[片]{s.}{amor | cuidado | compaixão}
  \end{phonetics}
\end{entry}

\begin{entry}{爱好}{10,6}[Radicais ⽖、⼥]
  \begin{phonetics}{爱好}{ai4 hao4}[][HSK 1]
    \definition[个,种]{s.}{passatempo; interesse; \emph{hobby}; sentimentos de interesse especial ou afeição por algo}
    \definition{v.}{estar interessado em; ter prazer em; ter um forte interesse em algo; ter sentimentos profundos por alguém ou algo}
  \end{phonetics}
\end{entry}

\begin{entry}{爱好者}{10,6,8}[Radicais ⽖、⼥、⽼]
  \begin{phonetics}{爱好者}{ai4 hao4 zhe3}
    \definition{s.}{amador | entusiasta | fã | amante de arte, esportes, etc.}
  \end{phonetics}
\end{entry}

\begin{entry}{爱抚}{10,7}[Radicais ⽖、⼿]
  \begin{phonetics}{爱抚}{ai4fu3}
    \definition{s.}{cuidado afetuoso | carinho}
    \definition{v.}{acariciar | cuidar (com ternura)}
  \end{phonetics}
\end{entry}

\begin{entry}{爱国}{10,8}[Radicais ⽖、⼞]
  \begin{phonetics}{爱国}{ai4guo2}
    \definition{adj.}{patriótico}
    \definition{v.}{amar o país | ser patriota}
  \end{phonetics}
\end{entry}

\begin{entry}{爱爱}{10,10}[Radicais ⽖、⽖]
  \begin{phonetics}{爱爱}{ai4'ai5}
    \definition{v.}{(coloquial) fazer amor}
  \end{phonetics}
\end{entry}

\begin{entry}{爱情}{10,11}[Radicais ⽖、⼼]
  \begin{phonetics}{爱情}{ai4qing2}[][HSK 2]
    \definition{s.}{amor (entre pessoas) | afeição}
  \end{phonetics}
\end{entry}

\begin{entry}{特地}{10,6}[Radicais ⽜、⼟]
  \begin{phonetics}{特地}{te4di4}
    \definition{adv.}{especialmente | propositalmente}
  \end{phonetics}
\end{entry}

\begin{entry}{特色}{10,6}[Radicais ⽜、⾊]
  \begin{phonetics}{特色}{te4se4}[][HSK 3]
    \definition{s.}{característica; característica distintiva | a cor única, estilo, etc. de um objeto}
  \end{phonetics}
\end{entry}

\begin{entry}{特别}{10,7}[Radicais ⽜、⼑]
  \begin{phonetics}{特别}{te4bie2}[][HSK 2]
    \definition{adj.}{especial | paricular | incomum}
    \definition{adv.}{especialmente | particularmente | propositalmente}
  \end{phonetics}
\end{entry}

\begin{entry}{特技}{10,7}[Radicais ⽜、⼿]
  \begin{phonetics}{特技}{te4ji4}
    \definition{s.}{efeito especial | dublê}
  \end{phonetics}
\end{entry}

\begin{entry}{特点}{10,9}[Radicais ⽜、⽕]
  \begin{phonetics}{特点}{te4dian3}[][HSK 2]
    \definition[个]{s.}{característica | peculiaridade | característica distintiva}
  \end{phonetics}
\end{entry}

\begin{entry}{牺牲}{10,9}[Radicais ⽜、⽜]
  \begin{phonetics}{牺牲}{xi1sheng1}
    \definition{s.}{abate de um animal como sacrifício}
    \definition{v.}{sacrificar a vida de alguém | sacrificar (algo de valor)}
  \end{phonetics}
\end{entry}

\begin{entry}{猃狁}{10,7}[Radicais ⽝、⽝]
  \begin{phonetics}{猃狁}{xian3yun3}
    \definition*{s.}{Termo da dinastia Zhou para uma tribo nômade do norte mais tarde chamou o Xiongnu (匈奴) nas dinastias Qin e Han}
  \seealsoref{匈奴}{xiong1nu2}
  \end{phonetics}
\end{entry}

\begin{entry}{珠子}{10,3}[Radicais ⽟、⼦]
  \begin{phonetics}{珠子}{zhu1zi5}
    \definition[粒,颗]{s.}{pérola | contas}
  \end{phonetics}
\end{entry}

\begin{entry}{班}{10}[Radical ⽟]
  \begin{phonetics}{班}{ban1}[][HSK 1]
    \definition*{s.}{sobrenome Ban}
    \definition{clas.}{para grupos}
    \definition[个]{s.}{equipe| time | esquadrão | turno de trabalho | classificação}
  \end{phonetics}
\end{entry}

\begin{entry}{班长}{10,4}[Radicais ⽟、⾧]
  \begin{phonetics}{班长}{ban1 zhang3}[][HSK 2]
    \definition[个]{s.}{monitor de classe | líder de equipe | líder de esquadrão}
  \end{phonetics}
\end{entry}

\begin{entry}{班级}{10,6}[Radicais ⽟、⽷]
  \begin{phonetics}{班级}{ban1 ji2}[][HSK 3]
    \definition[个]{s.}{classe | série (na escola)}
  \end{phonetics}
\end{entry}

\begin{entry}{瓶}{10}[Radical ⽡]
  \begin{phonetics}{瓶}{ping2}[][HSK 2]
    \definition{clas.}{para vinho ou líquidos}
    \definition[个]{s.}{garrafa | jarro| vaso}
  \end{phonetics}
\end{entry}

\begin{entry}{瓶子}{10,3}[Radicais ⽡、⼦]
  \begin{phonetics}{瓶子}{ping2zi5}[][HSK 2]
    \definition[个]{s.}{garrafa}
  \end{phonetics}
\end{entry}

\begin{entry}{瓶盖}{10,11}[Radicais ⽡、⽫]
  \begin{phonetics}{瓶盖}{ping2gai4}
    \definition{s.}{tampa de garrafa}
  \end{phonetics}
\end{entry}

\begin{entry}{瓶装}{10,12}[Radicais ⽡、⾐]
  \begin{phonetics}{瓶装}{ping2zhuang1}
    \definition{adj.}{engarrafado}
  \end{phonetics}
\end{entry}

\begin{entry}{瓷}{10}[Radical ⽡]
  \begin{phonetics}{瓷}{ci2}
    \definition{s.}{artigos de porcelana}
  \end{phonetics}
\end{entry}

\begin{entry}{留}{10}[Radical ⽥]
  \begin{phonetics}{留}{liu2}[][HSK 2]
    \definition{v.}{permanecer | ficar | pedir para alguém ficar | manter alguém onde ele está | concentrar-se em | reservar | manter | salvar | deixar crescer | crescer | vestir | aceitar | tomar | deixar para trás | estudar no exterior}
  \end{phonetics}
\end{entry}

\begin{entry}{留下}{10,3}[Radicais ⽥、⼀]
  \begin{phonetics}{留下}{liu2 xia4}[][HSK 2]
    \definition{v.}{deixar}
  \end{phonetics}
\end{entry}

\begin{entry}{留学}{10,8}[Radicais ⽥、⼦]
  \begin{phonetics}{留学}{liu2xue2}[][HSK 3]
    \definition{v.}{estudar no exterior}
  \end{phonetics}
\end{entry}

\begin{entry}{留学生}{10,8,5}[Radicais ⽥、⼦、⽣]
  \begin{phonetics}{留学生}{liu2 xue2 sheng1}[][HSK 2]
    \definition[个,位,名,批]{s.}{estudante estrangeiro | estudante estudando no exterior}
  \end{phonetics}
\end{entry}

\begin{entry}{留神}{10,9}[Radicais ⽥、⽰]
  \begin{phonetics}{留神}{liu2shen2}
    \definition{v.+compl.}{tomar cuidado | prestar atenção | manter os olhos abertos}
  \end{phonetics}
\end{entry}

\begin{entry}{畜}{10}[Radical ⽥]
  \begin{phonetics}{畜}{chu4}
    \definition{s.}{gado | animal domesticado | animal doméstico}
  \end{phonetics}
  \begin{phonetics}{畜}{xu4}
    \definition{v.}{criar (animais)}
  \end{phonetics}
\end{entry}

\begin{entry}{疼}{10}[Radical ⽧]
  \begin{phonetics}{疼}{teng2}[][HSK 2]
    \definition{adj.}{dolorido | doído}
    \definition{v.}{doer | amar ternamente}
  \end{phonetics}
\end{entry}

\begin{entry}{病}{10}[Radical ⽧]
  \begin{phonetics}{病}{bing4}[][HSK 1]
    \definition[场]{s.}{doença}
    \definition{v.}{adoecer | estar doente}
  \end{phonetics}
\end{entry}

\begin{entry}{病人}{10,2}[Radicais ⽧、⼈]
  \begin{phonetics}{病人}{bing4 ren2}[][HSK 1]
    \definition{s.}{doente | paciente}
  \end{phonetics}
\end{entry}

\begin{entry}{盏}{10}[Radical ⽫]
  \begin{phonetics}{盏}{zhan3}
    \definition{clas.}{para lâmpadas}
    \definition{s.}{copo pequeno}
  \end{phonetics}
\end{entry}

\begin{entry}{监狱}{10,9}[Radicais ⽫、⽝]
  \begin{phonetics}{监狱}{jian1yu4}
    \definition{s.}{prisão}
  \end{phonetics}
\end{entry}

\begin{entry}{眞}{10}[Radical ⽬]
  \begin{phonetics}{眞}{zhen1}
    \variantof{真}
  \end{phonetics}
\end{entry}

\begin{entry}{真}{10}[Radical ⼗]
  \begin{phonetics}{真}{zhen1}[][HSK 1]
    \definition{adj.}{genuíno}
    \definition{adv.}{que\dots tão\dots! | realmente}
  \end{phonetics}
\end{entry}

\begin{entry}{真切}{10,4}[Radicais ⼗、⼑]
  \begin{phonetics}{真切}{zhen1qie4}
    \definition{adj.}{claro | distinto | honesto | sincero | vívido}
  \end{phonetics}
\end{entry}

\begin{entry}{真心}{10,4}[Radicais ⼗、⼼]
  \begin{phonetics}{真心}{zhen1xin1}
    \definition{adj.}{sincero}
    \definition[片]{s.}{sinceridade}
  \end{phonetics}
\end{entry}

\begin{entry}{真牛}{10,4}[Radicais ⼗、⽜]
  \begin{phonetics}{真牛}{zhen1niu2}
    \definition{adj.}{(gíria) muito legal, incrível}
  \end{phonetics}
\end{entry}

\begin{entry}{真正}{10,5}[Radicais ⼗、⽌]
  \begin{phonetics}{真正}{zhen1zheng4}[][HSK 2]
    \definition{adj.}{verdadeiro | real | genuíno}
    \definition{adv.}{realmente | de ​​fato}
  \end{phonetics}
\end{entry}

\begin{entry}{真声}{10,7}[Radicais ⼗、⼠]
  \begin{phonetics}{真声}{zhen1sheng1}
    \definition{s.}{voz natural | voz verdadeira}
    \seeref{假声}{jia3sheng1}
  \end{phonetics}
\end{entry}

\begin{entry}{真实}{10,8}[Radicais ⼗、⼧]
  \begin{phonetics}{真实}{zhen1shi2}[][HSK 3]
    \definition{adj.}{verdadeiro; real; autêntico}
  \end{phonetics}
\end{entry}

\begin{entry}{真的}{10,8}[Radicais ⼗、⽩]
  \begin{phonetics}{真的}{zhen1 de5}[][HSK 1]
    \definition{adv.}{realmente | verdadeiramente}
  \end{phonetics}
\end{entry}

\begin{entry}{真珠}{10,10}[Radicais ⼗、⽟]
  \begin{phonetics}{真珠}{zhen1zhu1}
    \variantof{珍珠}
  \end{phonetics}
\end{entry}

\begin{entry}{真真}{10,10}[Radicais ⼗、⼗]
  \begin{phonetics}{真真}{zhen1zhen1}
    \definition{adv.}{genuinamente | realmente | escrupulosamente}
  \end{phonetics}
\end{entry}

\begin{entry}{真理}{10,11}[Radicais ⼗、⽟]
  \begin{phonetics}{真理}{zhen1li3}
    \definition[个]{s.}{verdade}
  \end{phonetics}
\end{entry}

\begin{entry}{真释}{10,12}[Radicais ⼗、⾤]
  \begin{phonetics}{真释}{zhen1shi4}
    \definition{s.}{razão genuína | explicação verdadeira}
  \end{phonetics}
\end{entry}

\begin{entry}{破}{10}[Radical ⽯]
  \begin{phonetics}{破}{po4}[][HSK 3]
    \definition{adj.}{quebrado; danificado; rasgado; desgastado | pobre; ruim; insignificante; péssimo; miserável}
    \definition{v.}{estar quebrado; estar danificado | quebrar; avariar; danificar | quebrar; dividir; cortar; cinzelar | trocar (dinheiro) | romper; quebrar (avanço) | livrar-se de; destruir; romper com
derrotar; capturar (uma cidade, etc.) | despender; gastar (dinheiro) | expor a verdade de; desnudar}
  \end{phonetics}
\end{entry}

\begin{entry}{破产}{10,6}[Radicais ⽯、⼇]
  \begin{phonetics}{破产}{po4chan3}
    \definition{v.+compl.}{falir | quebrar | tornar-se insolvente | ficar empobrecido | cair | ser destruído | ser arruinado}
  \end{phonetics}
\end{entry}

\begin{entry}{破坏}{10,7}[Radicais ⽯、⼟]
  \begin{phonetics}{破坏}{po4huai4}[][HSK 3]
    \definition{s.}{destruição | dano}
    \definition{v.}{demolir; naufragar; soçobrar; destruir; obliterar | quebrar; violar (um acordo, regulamento, etc.) | prejudicar; perturbar; sabotar; causar grande dano | reverter; mudar (um sistema social, costume, etc.) completamente ou violentamente | destruir; decompor}
  \end{phonetics}
\end{entry}

\begin{entry}{破坏性}{10,7,8}[Radicais ⽯、⼟、⼼]
  \begin{phonetics}{破坏性}{po4huai4xing4}
    \definition{adj.}{destrutivo}
    \definition{s.}{poder destrutivo}
  \end{phonetics}
\end{entry}

\begin{entry}{砸}{10}[Radical ⽯]
  \begin{phonetics}{砸}{za2}
    \definition{v.}{esmagar | bater | falhar | estragar}
  \end{phonetics}
\end{entry}

\begin{entry}{离}{10}[Radical ⼇]
  \begin{phonetics}{离}{li2}[][HSK 2]
    \definition*{s.}{sobrenome Li}
    \definition{prep.}{(ser longe) de\dots até\dots}
    \definition{v.}{ficar longe de | deixar | separar-se de}
  \end{phonetics}
\end{entry}

\begin{entry}{离开}{10,4}[Radicais ⼇、⼶]
  \begin{phonetics}{离开}{li2kai1}[][HSK 2]
    \definition{v.}{partir| deixar}
  \end{phonetics}
\end{entry}

\begin{entry}{离婚}{10,11}[Radicais ⼇、⼥]
  \begin{phonetics}{离婚}{li2hun1}[][HSK 3]
    \definition{v.+compl.}{divórciar; romper um casamento; obter o divórcio}
  \end{phonetics}
\end{entry}

\begin{entry}{租}{10}[Radical ⽲]
  \begin{phonetics}{租}{zu1}[][HSK 2]
    \definition{s.}{imposto sobre propriedade urbana ou rural}
    \definition{v.}{alugar | tomar de aluguel}
  \end{phonetics}
\end{entry}

\begin{entry}{租用}{10,5}[Radicais ⽲、⽤]
  \begin{phonetics}{租用}{zu1yong4}
    \definition{v.}{contratar | alugar | alugar (algo de alguém)}
  \end{phonetics}
\end{entry}

\begin{entry}{租让}{10,5}[Radicais ⽲、⾔]
  \begin{phonetics}{租让}{zu1rang4}
    \definition{v.}{alugar | alugar (a propriedade de alguém para outra pessoa)}
  \end{phonetics}
\end{entry}

\begin{entry}{租约}{10,6}[Radicais ⽲、⽷]
  \begin{phonetics}{租约}{zu1yue1}
    \definition{s.}{aluguel}
  \end{phonetics}
\end{entry}

\begin{entry}{租房}{10,8}[Radicais ⽲、⼾]
  \begin{phonetics}{租房}{zu1fang2}
    \definition{v.}{alugar um apartamento}
  \end{phonetics}
\end{entry}

\begin{entry}{租金}{10,8}[Radicais ⽲、⾦]
  \begin{phonetics}{租金}{zu1jin1}
    \definition{s.}{aluguel}
    \seeref{租钱}{zu1qian5}
  \end{phonetics}
\end{entry}

\begin{entry}{租赁}{10,10}[Radicais ⽲、⾙]
  \begin{phonetics}{租赁}{zu1lin4}
    \definition{v.}{contratar | alugar}
  \end{phonetics}
\end{entry}

\begin{entry}{租钱}{10,10}[Radicais ⽲、⾦]
  \begin{phonetics}{租钱}{zu1qian5}
    \definition{s.}{aluguel}
    \seeref{租金}{zu1jin1}
  \end{phonetics}
\end{entry}

\begin{entry}{租船}{10,11}[Radicais ⽲、⾈]
  \begin{phonetics}{租船}{zu1chuan2}
    \definition{v.}{fretar um navio | alugar um navio}
  \end{phonetics}
\end{entry}

\begin{entry}{积木}{10,4}[Radicais ⽲、⽊]
  \begin{phonetics}{积木}{ji1mu4}
    \definition{s.}{blocos de montar (brinquedo)}
  \end{phonetics}
\end{entry}

\begin{entry}{积极}{10,7}[Radicais ⽲、⽊]
  \begin{phonetics}{积极}{ji1ji2}[][HSK 3]
    \definition{adj.}{ativo | positivo}
  \end{phonetics}
\end{entry}

\begin{entry}{称}{10}[Radical ⽲]
  \begin{phonetics}{称}{chen4}
    \definition{v.}{ajustar | combinar}
  \end{phonetics}
  \begin{phonetics}{称}{cheng1}[][HSK 2]
    \definition*{s.}{sobrenome Cheng}
    \definition{s.}{nome}
    \definition{v.}{chamar | dizer | elogiar | louvar | pesar | levantar | começar}
  \end{phonetics}
\end{entry}

\begin{entry}{称为}{10,4}[Radicais ⽲、⼂]
  \begin{phonetics}{称为}{cheng1 wei2}[][HSK 3]
    \definition{v.}{chamar; ser chamado; ser conhecido como}
  \end{phonetics}
\end{entry}

\begin{entry}{站}{10}[Radical ⽴]
  \begin{phonetics}{站}{zhan4}[][HSK 1]
    \definition{s.}{estação | ponto | parada}
  \end{phonetics}
\end{entry}

\begin{entry}{站长}{10,4}[Radicais ⽴、⾧]
  \begin{phonetics}{站长}{zhan4zhang3}
    \definition{s.}{pessoa responsável pela estação de trem | chefe da estação | \emph{webmaster} | gerente de centro de voluntariado}
  \end{phonetics}
\end{entry}

\begin{entry}{站台}{10,5}[Radicais ⽴、⼝]
  \begin{phonetics}{站台}{zhan4tai2}
    \definition{s.}{plataforma (em uma estação ferroviária)}
  \end{phonetics}
\end{entry}

\begin{entry}{站住}{10,7}[Radicais ⽴、⼈]
  \begin{phonetics}{站住}{zhan4 zhu4}[][HSK 2]
    \definition{v.}{parar | deter | ficar firme em pé | manter os pés firmes | manter a própria posição | consolidar a própria posição | reter água | ser sustentável}
  \end{phonetics}
\end{entry}

\begin{entry}{站姿}{10,9}[Radicais ⽴、⼥]
  \begin{phonetics}{站姿}{zhan4zi1}
    \definition{s.}{postura}
  \end{phonetics}
\end{entry}

\begin{entry}{站点}{10,9}[Radicais ⽴、⽕]
  \begin{phonetics}{站点}{zhan4dian3}
    \definition{s.}{\emph{website}}
  \end{phonetics}
\end{entry}

\begin{entry}{竞赛}{10,14}[Radicais ⽴、⾙]
  \begin{phonetics}{竞赛}{jing4sai4}
    \definition{s.}{concurso | competição | partida | corrida}
    \definition{v.}{competir | correr}
  \end{phonetics}
\end{entry}

\begin{entry}{笋}{10}[Radical ⽵]
  \begin{phonetics}{笋}{sun3}
    \definition{s.}{broto de bambu}
  \end{phonetics}
\end{entry}

\begin{entry}{笑}{10}[Radical ⽵]
  \begin{phonetics}{笑}{xiao4}[][HSK 1]
    \definition{v.}{sorrir | rir | rir de}
  \end{phonetics}
\end{entry}

\begin{entry}{笑话}{10,8}[Radicais ⽵、⾔]
  \begin{phonetics}{笑话}{xiao4hua5}[][HSK 2]
    \definition{adj.}{absurdo | ridículo}
    \definition[个]{s.}{piada | brincadeira}
    \definition{v.}{rir de algo | zombar | ridicularizar}
  \end{phonetics}
\end{entry}

\begin{entry}{笑话儿}{10,8,2}[Radicais ⽵、⾔、⼉]
  \begin{phonetics}{笑话儿}{xiao4 hua4r5}[][HSK 2]
    \definition{s.}{piada | gracejo}
  \end{phonetics}
\end{entry}

\begin{entry}{笑容}{10,10}[Radicais ⽵、⼧]
  \begin{phonetics}{笑容}{xiao4rong2}
    \definition[副]{s.}{sorriso | expressão sorridente}
  \end{phonetics}
\end{entry}

\begin{entry}{笔}{10}[Radical ⽵]
  \begin{phonetics}{笔}{bi3}[][HSK 2]
    \definition{clas.}{para somas de dinheiro, negócios}
    \definition[支,枝]{s.}{caneta | lápis}
  \end{phonetics}
\end{entry}

\begin{entry}{笔记}{10,5}[Radicais ⽵、⾔]
  \begin{phonetics}{笔记}{bi3 ji4}[][HSK 2]
    \definition[篇,本,个]{s.}{notas | ensaios | esboços}
    \definition{v.}{tomar nota (por escrito)}
  \end{phonetics}
\end{entry}

\begin{entry}{笔记本}{10,5,5}[Radicais ⽵、⾔、⽊]
  \begin{phonetics}{笔记本}{bi3ji4ben3}[][HSK 2]
    \definition[本]{s.}{caderno}
    \definition{s.}{\emph{laptop}}
  \end{phonetics}
\end{entry}

\begin{entry}{粉}{10}[Radical ⽶]
  \begin{phonetics}{粉}{fen3}
    \definition{s.}{pó | pó cosmético facial | alimento preparado a partir de amido | macarrão feito de qualquer tipo de farinha}
    \definition{v.}{tornar algo em pó | ser um fã de}
  \end{phonetics}
\end{entry}

\begin{entry}{粉丝}{10,5}[Radicais ⽶、⼀]
  \begin{phonetics}{粉丝}{fen3si1}
    \definition{s.}{(empréstimo linguístico) fã | entusiasta de alguém ou alguma coisa}
    \definition[把]{s.}{aletria de amido de feijão | aletria chinesa | macarrão de celofane ou macarrão de vidro (transparente)}
  \end{phonetics}
\end{entry}

\begin{entry}{粉色}{10,6}[Radicais ⽶、⾊]
  \begin{phonetics}{粉色}{fen3 se4}
    \definition{s.}{cor-de-rosa}
  \end{phonetics}
\end{entry}

\begin{entry}{索性}{10,8}[Radicais ⽷、⼼]
  \begin{phonetics}{索性}{suo3xing4}
    \definition{adv.}{poderia muito bem | simplesmente | apenas}
  \end{phonetics}
\end{entry}

\begin{entry}{紧}{10}[Radical ⽷]
  \begin{phonetics}{紧}{jin3}[][HSK 3]
    \definition{adj.}{tenso; apertado | seguro; firme | cerrado; apertado | urgente; premente; tenso | rigoroso; rígido; severo | difícil; sem dinheiro}
    \definition{v.}{apertar}
  \end{phonetics}
\end{entry}

\begin{entry}{紧张}{10,7}[Radicais ⽷、⼸]
  \begin{phonetics}{紧张}{jin3zhang1}[][HSK 3]
    \definition{adj.}{nervoso; tenso | apertado; em falta | tenso; intenso; coado}
  \end{phonetics}
\end{entry}

\begin{entry}{紧急}{10,9}[Radicais ⽷、⼼]
  \begin{phonetics}{紧急}{jin3ji2}[][HSK 3]
    \definition{adj.}{urgente}
    \definition{adj.}{urgente; premente; crítico}
    \definition{s.}{emergência}
  \end{phonetics}
\end{entry}

\begin{entry}{绣}{10}[Radical ⽷]
  \begin{phonetics}{绣}{xiu4}
    \definition{s.}{bordado}
    \definition{v.}{bordar}
  \end{phonetics}
\end{entry}

\begin{entry}{继续}{10,11}[Radicais ⽷、⽷]
  \begin{phonetics}{继续}{ji4xu4}[][HSK 3]
    \definition{v.}{continuar; prosseguir}
  \end{phonetics}
\end{entry}

\begin{entry}{缺}{10}[Radical ⽸]
  \begin{phonetics}{缺}{que1}[][HSK 3]
    \definition{adj.}{incompleto; imperfeito}
    \definition[种]{s.}{vaga; abertura; falta}
    \definition{v.}{estar com falta de; faltar | estar faltando; estar incompleto | estar ausente}
  \end{phonetics}
\end{entry}

\begin{entry}{缺少}{10,4}[Radicais ⽸、⼩]
  \begin{phonetics}{缺少}{que1shao3}[][HSK 3]
    \definition{v.}{falta; estar com falta de; estar em falta de}
  \end{phonetics}
\end{entry}

\begin{entry}{缺点}{10,9}[Radicais ⽸、⽕]
  \begin{phonetics}{缺点}{que1dian3}[][HSK 3]
    \definition[个]{s.}{desvantagem; deficiência; inconveniência; ponto fraco}
  \end{phonetics}
\end{entry}

\begin{entry}{缺勤}{10,13}[Radicais ⽸、⼒]
  \begin{phonetics}{缺勤}{que1qin2}
    \definition{v.+compl.}{ausentar-se do dever (trabalho)}
  \end{phonetics}
\end{entry}

\begin{entry}{罢}{10}[Radical ⽹]
  \begin{phonetics}{罢}{ba4}
    \definition{v.}{parar | cessar | demitir | suspender | desistir | terminar}
  \end{phonetics}
  \begin{phonetics}{罢}{ba5}
    \definition{part.}{partícula final, a mesma que 吧}
  \seealsoref{吧}{ba5}
  \end{phonetics}
\end{entry}

\begin{entry}{耽心}{10,4}[Radicais ⽿、⼼]
  \begin{phonetics}{耽心}{dan1xin1}
    \variantof{担心}
  \end{phonetics}
\end{entry}

\begin{entry}{胶卷}{10,8}[Radicais ⾁、⼙]
  \begin{phonetics}{胶卷}{jiao1juan3}
    \definition{s.}{filme | rolo de filme}
  \end{phonetics}
\end{entry}

\begin{entry}{胸}{10}[Radical ⾁]
  \begin{phonetics}{胸}{xiong1}
    \definition{s.}{peito | tórax}
  \end{phonetics}
\end{entry}

\begin{entry}{能}{10}[Radical ⾁]
  \begin{phonetics}{能}{neng2}[][HSK 1]
    \definition*{s.}{sobrenome Neng}
    \definition{adv.}{talvez}
    \definition{s.}{(física)nenergia | habilidade}
    \definition{v.}{poder | ser capaz de}
  \end{phonetics}
\end{entry}

\begin{entry}{能力}{10,2}[Radicais ⾁、⼒]
  \begin{phonetics}{能力}{neng2li4}[][HSK 3]
    \definition{s.}{habilidade; capacidade; aptidão}
  \end{phonetics}
\end{entry}

\begin{entry}{能上能下}{10,3,10,3}[Radicais ⾁、⼀、⾁、⼀]
  \begin{phonetics}{能上能下}{neng2shang4neng2xia4}
    \definition{s.}{pronto para aceitar qualquer trabalho, alto ou baixo}
  \end{phonetics}
\end{entry}

\begin{entry}{能干}{10,3}[Radicais ⾁、⼲]
  \begin{phonetics}{能干}{neng2gan4}
    \definition{adj.}{capaz | competente}
  \end{phonetics}
\end{entry}

\begin{entry}{能不能}{10,4,10}[Radicais ⾁、⼀、⾁]
  \begin{phonetics}{能不能}{neng2 bu4 neng2}[][HSK 3]
    \definition{adv.}{pode ou não pode\dots?}
  \end{phonetics}
\end{entry}

\begin{entry}{能够}{10,11}[Radicais ⾁、⼣]
  \begin{phonetics}{能够}{neng2 gou4}[][HSK 2]
    \definition{v.}{ser capaz de}
  \end{phonetics}
\end{entry}

\begin{entry}{脂麻}{10,11}[Radicais ⾁、⿇]
  \begin{phonetics}{脂麻}{zhi1ma5}
    \variantof{芝麻}
  \end{phonetics}
\end{entry}

\begin{entry}{脏}{10}[Radical ⾁]
  \begin{phonetics}{脏}{zang1}[][HSK 2]
    \definition{adj.}{sujo | imundo}
  \end{phonetics}
  \begin{phonetics}{脏}{zang4}
    \definition{s.}{órgão (anatomia) | víscera}
  \end{phonetics}
\end{entry}

\begin{entry}{脏土}{10,3}[Radicais ⾁、⼟]
  \begin{phonetics}{脏土}{zang1tu3}
    \definition{s.}{solo sujo | lama | lixo}
  \end{phonetics}
\end{entry}

\begin{entry}{脏字}{10,6}[Radicais ⾁、⼦]
  \begin{phonetics}{脏字}{zang1zi4}
    \definition{s.}{obscenidade}
  \end{phonetics}
\end{entry}

\begin{entry}{脏病}{10,10}[Radicais ⾁、⽧]
  \begin{phonetics}{脏病}{zang1bing4}
    \definition{s.}{doença venérea}
  \end{phonetics}
\end{entry}

\begin{entry}{脏脏}{10,10}[Radicais ⾁、⾁]
  \begin{phonetics}{脏脏}{zang1zang1}
    \definition{adj.}{sujo}
  \end{phonetics}
\end{entry}

\begin{entry}{脏煤}{10,13}[Radicais ⾁、⽕]
  \begin{phonetics}{脏煤}{zang1mei2}
    \definition{s.}{carvão sujo | sujeira (de uma mina de carvão)}
  \end{phonetics}
\end{entry}

\begin{entry}{脏器}{10,16}[Radicais ⾁、⼝]
  \begin{phonetics}{脏器}{zang4qi4}
    \definition{s.}{órgãos internos}
  \end{phonetics}
\end{entry}

\begin{entry}{脏辫}{10,17}[Radicais ⾁、⾟]
  \begin{phonetics}{脏辫}{zang1bian4}
    \definition{s.}{\emph{dreadlocks}}
  \end{phonetics}
\end{entry}

\begin{entry}{脑瓜}{10,5}[Radicais ⾁、⽠]
  \begin{phonetics}{脑瓜}{nao3gua1}
    \definition{s.}{crânio | cérebro | cabeça | mente | mentalidade | ideia}
  \seealsoref{脑瓜子}{nao3gua1zi5}
  \end{phonetics}
\end{entry}

\begin{entry}{脑瓜子}{10,5,3}[Radicais ⾁、⽠、⼦]
  \begin{phonetics}{脑瓜子}{nao3gua1zi5}
    \definition{s.}{crânio | cérebro | cabeça | mente | mentalidade | ideia}
  \seealsoref{脑瓜}{nao3gua1}
  \end{phonetics}
\end{entry}

\begin{entry}{脑袋}{10,11}[Radicais ⾁、⾐]
  \begin{phonetics}{脑袋}{nao3dai5}
    \definition[颗,个]{s.}{cabeça | crânio | cérebro | capacidade mental}
  \end{phonetics}
\end{entry}

\begin{entry}{臭}{10}[Radical ⾃]
  \begin{phonetics}{臭}{chou4}
    \definition{adj.}{fétido | repulsivo | repugnante | malcheiroso}
    \definition{s.}{fedor}
    \definition{v.}{feder}
  \end{phonetics}
  \begin{phonetics}{臭}{xiu4}
    \definition{s.}{olfato | cheiro ruim}
  \end{phonetics}
\end{entry}

\begin{entry}{臭气}{10,4}[Radicais ⾃、⽓]
  \begin{phonetics}{臭气}{chou4qi4}
    \definition{s.}{fedor}
  \end{phonetics}
\end{entry}

\begin{entry}{致敬}{10,12}[Radicais ⾄、⽁]
  \begin{phonetics}{致敬}{zhi4jing4}
    \definition{v.}{saudar | prestar respeitos a | prestar homenagem a}
  \end{phonetics}
\end{entry}

\begin{entry}{航天员}{10,4,7}[Radicais ⾈、⼤、⼝]
  \begin{phonetics}{航天员}{hang2tian1yuan2}
    \definition{s.}{astronauta}
  \end{phonetics}
\end{entry}

\begin{entry}{航班}{10,10}[Radicais ⾈、⽟]
  \begin{phonetics}{航班}{hang2ban1}
    \definition{s.}{voo | número de voo}
  \end{phonetics}
\end{entry}

\begin{entry}{般}{10}[Radical ⾈]
  \begin{phonetics}{般}{ban1}
    \definition{s.}{espécie | tipo | classe | caminho | maneira}
  \end{phonetics}
  \begin{phonetics}{般}{bo1}
    \definition{s.}{utilizado em 般若 \dpy{bo1re3}}
    \seeref{般若}{bo1re3}
  \end{phonetics}
  \begin{phonetics}{般}{pan2}
    \definition{s.}{utilizado em 般乐 \dpy{pan2le4}}
    \seeref{般乐}{pan2le4}
  \end{phonetics}
\end{entry}

\begin{entry}{般乐}{10,5}[Radicais ⾈、⼃]
  \begin{phonetics}{般乐}{pan2le4}
    \definition{v.}{jogar | divertir-se}
  \end{phonetics}
\end{entry}

\begin{entry}{般若}{10,8}[Radicais ⾈、⾋]
  \begin{phonetics}{般若}{bo1re3}
    \definition*{s.}{Prajna (sânscrito), \emph{insight} sobre a verdadeira natureza da realidade | (Budismo) sabedoria}
  \end{phonetics}
\end{entry}

\begin{entry}{舱}{10}[Radical ⾈]
  \begin{phonetics}{舱}{cang1}
    \definition{s.}{cabine | porão (de carga) de um navio ou avião}
  \end{phonetics}
\end{entry}

\begin{entry}{荷}{10}[Radical ⾋]
  \begin{phonetics}{荷}{he2}
    \definition{s.}{lótus}
  \end{phonetics}
  \begin{phonetics}{荷}{he4}
    \definition{s.}{carga | responsabilidade}
    \definition{v.}{carregar no ombro ou nas costas}
  \end{phonetics}
\end{entry}

\begin{entry}{荷花}{10,7}[Radicais ⾋、⾋]
  \begin{phonetics}{荷花}{he2hua1}
    \definition{s.}{lótus}
  \end{phonetics}
\end{entry}

\begin{entry}{莎莎舞}{10,10,14}[Radicais ⾋、⾋、⾇]
  \begin{phonetics}{莎莎舞}{sha1sha1wu3}
    \definition{s.}{salsa (dança)}
  \end{phonetics}
\end{entry}

\begin{entry}{莫名其妙}{10,6,8,7}[Radicais ⾋、⼝、⼋、⼥]
  \begin{phonetics}{莫名其妙}{mo4ming2qi2miao4}
    \definition{adj.}{desconcertante | bizzaro | inexplicável | perplexo}
  \end{phonetics}
\end{entry}

\begin{entry}{莲花}{10,7}[Radicais ⾋、⾋]
  \begin{phonetics}{莲花}{lian2hua1}
    \definition{s.}{flor de lótus | lírio aquático}
  \end{phonetics}
\end{entry}

\begin{entry}{莲藕}{10,18}[Radicais ⾋、⾋]
  \begin{phonetics}{莲藕}{lian2'ou3}
    \definition{s.}{raiz de Lotus}
  \end{phonetics}
\end{entry}

\begin{entry}{蚊子}{10,3}[Radicais ⾍、⼦]
  \begin{phonetics}{蚊子}{wen2zi5}
    \definition{s.}{pernilongo}
  \end{phonetics}
\end{entry}

\begin{entry}{蚊香}{10,9}[Radicais ⾍、⾹]
  \begin{phonetics}{蚊香}{wen2xiang1}
    \definition{s.}{incenso ou espiral repelente de mosquitos}
  \end{phonetics}
\end{entry}

\begin{entry}{蚕纸}{10,7}[Radicais ⾍、⽷]
  \begin{phonetics}{蚕纸}{can2zhi3}
    \definition{s.}{papel onde o bicho-da-seda põe seus ovos}
  \end{phonetics}
\end{entry}

\begin{entry}{蚝}{10}[Radical ⾍]
  \begin{phonetics}{蚝}{hao2}
    \definition{s.}{ostra}
  \end{phonetics}
\end{entry}

\begin{entry}{袖}{10}[Radical ⾐]
  \begin{phonetics}{袖}{xiu4}
    \definition{s.}{manga (de camisa, de camiseta, etc.)}
  \end{phonetics}
\end{entry}

\begin{entry}{被}{10}[Radical ⾐]
  \begin{phonetics}{被}{bei4}[][HSK 3]
    \definition*{s.}{sobrenome Bei}
    \definition{part.}{usada antes de verbos para formar frases verbais passivas}
    \definition{prep.}{usado em uma frase para indicar que o sujeito é o receptor da ação}
    \definition{s.}{colcha}
    \definition{v.}{cobrir; espalhar
sofrer}
  \end{phonetics}
\end{entry}

\begin{entry}{被子}{10,3}[Radicais ⾐、⼦]
  \begin{phonetics}{被子}{bei4zi5}[][HSK 3]
    \definition[床]{s.}{colcha}
  \end{phonetics}
\end{entry}

\begin{entry}{被动}{10,6}[Radicais ⾐、⼒]
  \begin{phonetics}{被动}{bei4dong4}
    \definition{adj.}{passivo}
  \end{phonetics}
\end{entry}

\begin{entry}{被告}{10,7}[Radicais ⾐、⼝]
  \begin{phonetics}{被告}{bei4gao4}
    \definition{s.}{réu}
  \end{phonetics}
\end{entry}

\begin{entry}{被单}{10,8}[Radicais ⾐、⼗]
  \begin{phonetics}{被单}{bei4dan1}
    \definition[床]{s.}{lençol}
  \end{phonetics}
\end{entry}

\begin{entry}{被迫}{10,8}[Radicais ⾐、⾡]
  \begin{phonetics}{被迫}{bei4po4}
    \definition{v.}{ser compelido | ser forçado}
  \end{phonetics}
\end{entry}

\begin{entry}{被套}{10,10}[Radicais ⾐、⼤]
  \begin{phonetics}{被套}{bei4tao4}
    \definition{s.}{capa de \emph{edredon}}
    \definition{v.}{ter dinheiro preso (em ações, imóveis, etc.)}
  \end{phonetics}
\end{entry}

\begin{entry}{被窝}{10,12}[Radicais ⾐、⽳]
  \begin{phonetics}{被窝}{bei4wo1}
    \definition{s.}{colcha}
  \end{phonetics}
\end{entry}

\begin{entry}{请}{10}[Radical ⾔]
  \begin{phonetics}{请}{qing3}[][HSK 1]
    \definition{v.}{por favor (fazer alguma coisa) | perguntar | convidar | solicitar}
  \end{phonetics}
\end{entry}

\begin{entry}{请问}{10,6}[Radicais ⾔、⾨]
  \begin{phonetics}{请问}{qing3wen4}[][HSK 1]
    \definition{expr.}{Com licença, posso perguntar\dots? (para perguntar por qualquer coisa)}
  \end{phonetics}
\end{entry}

\begin{entry}{请坐}{10,7}[Radicais ⾔、⼟]
  \begin{phonetics}{请坐}{qing3 zuo4}[][HSK 1]
    \definition{v.}{por favor, sente-se}
  \end{phonetics}
\end{entry}

\begin{entry}{请求}{10,7}[Radicais ⾔、⽔]
  \begin{phonetics}{请求}{qing3qiu2}[][HSK 2]
    \definition[个]{s.}{solicitação}
    \definition{v.}{solicitar | perguntar}
  \end{phonetics}
\end{entry}

\begin{entry}{请进}{10,7}[Radicais ⾔、⾡]
  \begin{phonetics}{请进}{qing3 jin4}[][HSK 1]
    \definition{v.}{por favor entre}
  \end{phonetics}
\end{entry}

\begin{entry}{请客}{10,9}[Radicais ⾔、⼧]
  \begin{phonetics}{请客}{qing3ke4}[][HSK 2]
    \definition{v.+compl.}{entreter os convidados | dar um jantar | convidar para jantar}
  \end{phonetics}
\end{entry}

\begin{entry}{请假}{10,11}[Radicais ⾔、⼈]
  \begin{phonetics}{请假}{qing3 jia4}[][HSK 1]
    \definition{v.+compl.}{pedir licença para sair}
  \end{phonetics}
\end{entry}

\begin{entry}{请假条}{10,11,7}[Radicais ⾔、⼈、⽊]
  \begin{phonetics}{请假条}{qing3jia4tiao2}
    \definition{s.}{pedido de licença de ausência (do trabalho ou da escola)}
  \end{phonetics}
\end{entry}

\begin{entry}{请教}{10,11}[Radicais ⾔、⽁]
  \begin{phonetics}{请教}{qing3jiao4}[][HSK 3]
    \definition{v.}{consultar; pedir conselho}
  \end{phonetics}
\end{entry}

\begin{entry}{诺贝尔奖}{10,4,5,9}[Radicais ⾔、⾙、⼩、⼤]
  \begin{phonetics}{诺贝尔奖}{nuo4bei4'er3 jiang3}
    \definition*{s.}{Prêmio Nobel}
  \end{phonetics}
\end{entry}

\begin{entry}{诺奖}{10,9}[Radicais ⾔、⼤]
  \begin{phonetics}{诺奖}{nuo4jiang3}
    \definition*{s.}{Prêmio Nobel, abreviação de 诺贝尔奖}
    \seeref{诺贝尔奖}{nuo4bei4'er3 jiang3}
  \end{phonetics}
\end{entry}

\begin{entry}{读}{10}[Radical ⾔]
  \begin{phonetics}{读}{dou4}
    \definition{s.}{vírgula | frase marcada por pausa}
  \end{phonetics}
  \begin{phonetics}{读}{du2}[][HSK 1]
    \definition{v.}{ler em voz alta | ler | frequentar (escola) | estudar (uma matéria na escola) | pronunciar}
  \end{phonetics}
\end{entry}

\begin{entry}{读书}{10,4}[Radicais ⾔、⼄]
  \begin{phonetics}{读书}{du2 shu1}[][HSK 1]
    \definition{v.+compl.}{ler | estudar | frequentar a escola}
  \end{phonetics}
\end{entry}

\begin{entry}{读者}{10,8}[Radicais ⾔、⽼]
  \begin{phonetics}{读者}{du2 zhe3}[][HSK 3]
    \definition[个,位]{s.}{leitor}
  \end{phonetics}
\end{entry}

\begin{entry}{读音}{10,9}[Radicais ⾔、⾳]
  \begin{phonetics}{读音}{du2 yin1}[][HSK 2]
    \definition{s.}{pronúncia}
  \end{phonetics}
\end{entry}

\begin{entry}{课}{10}[Radical ⾔]
  \begin{phonetics}{课}{ke4}[][HSK 1]
    \definition{s.}{aula | curso | lição | imposto | taxa |seção}
  \end{phonetics}
\end{entry}

\begin{entry}{课文}{10,4}[Radicais ⾔、⽂]
  \begin{phonetics}{课文}{ke4 wen2}[][HSK 1]
    \definition{s.}{texto (de uma lição)}
  \end{phonetics}
\end{entry}

\begin{entry}{课本}{10,5}[Radicais ⾔、⽊]
  \begin{phonetics}{课本}{ke4 ben3}[][HSK 1]
    \definition[本]{s.}{livro do aluno | manual}
  \end{phonetics}
\end{entry}

\begin{entry}{课堂}{10,11}[Radicais ⾔、⼟]
  \begin{phonetics}{课堂}{ke4 tang2}[][HSK 2]
    \definition[间]{s.}{sala de aula}
  \end{phonetics}
\end{entry}

\begin{entry}{课程}{10,12}[Radicais ⾔、⽲]
  \begin{phonetics}{课程}{ke4cheng2}[][HSK 3]
    \definition[个,堂,节,门]{s.}{curso; currículo}
  \end{phonetics}
\end{entry}

\begin{entry}{谁}{10}[Radical ⾔]
  \begin{phonetics}{谁}{shei2}[][HSK 1]
    \definition{pron.}{quem?}
  \end{phonetics}
  \begin{phonetics}{谁}{shui2}[][HSK 1]
    \definition{pron.}{quem?}
  \end{phonetics}
\end{entry}

\begin{entry}{调}{10}[Radical ⾔]
  \begin{phonetics}{调}{diao4}[][HSK 3]
    \definition{s.}{sotaque | nota (musical) | melodia; tom}
    \definition{v.}{transferir; deslocar; mover | distribuir; alocar | trocar; permutar; comutar}
  \end{phonetics}
  \begin{phonetics}{调}{tiao2}[][HSK 3]
    \definition*{s.}{sobrenome Tiao}
    \definition{v.}{harmonizar; adequar-se bem; encaixar-se perfeitamente | mesclar; ajustar; ajustar; misturar | mediar |gracejar; tirar sarro de | provocar; alienar}
  \end{phonetics}
\end{entry}

\begin{entry}{调律}{10,9}[Radicais ⾔、⼻]
  \begin{phonetics}{调律}{tiao2lv4}
    \definition{v.}{afinar (por exemplo, um piano)}
  \end{phonetics}
\end{entry}

\begin{entry}{调查}{10,9}[Radicais ⾔、⽊]
  \begin{phonetics}{调查}{diao4cha2}[][HSK 3]
    \definition[项,个]{s.}{pesquisa; investigação}
    \definition{v.}{investigar; indagar; inquerir}
  \end{phonetics}
\end{entry}

\begin{entry}{调整}{10,16}[Radicais ⾔、⽁]
  \begin{phonetics}{调整}{tiao2zheng3}[][HSK 3]
    \definition{v.}{ajustar; revisar; regular}
  \end{phonetics}
\end{entry}

\begin{entry}{谈}{10}[Radical ⾔]
  \begin{phonetics}{谈}{tan2}[][HSK 3]
    \definition*{s.}{sobrenome Tan}
    \definition{s.}{o que é dito ou falado}
    \definition{v.}{falar; bater papo; discutir}
  \end{phonetics}
\end{entry}

\begin{entry}{谈判}{10,7}[Radicais ⾔、⼑]
  \begin{phonetics}{谈判}{tan2pan4}[][HSK 3]
    \definition{v.}{negociar; manter conversações}
  \end{phonetics}
\end{entry}

\begin{entry}{谈话}{10,8}[Radicais ⾔、⾔]
  \begin{phonetics}{谈话}{tan2 hua4}[][HSK 3]
    \definition[次]{s.}{declaração}
    \definition{v.+compl.}{conversar; discutir | falar}
  \end{phonetics}
\end{entry}

\begin{entry}{谈恋爱}{10,10,10}[Radicais ⾔、⼼、⽖]
  \begin{phonetics}{谈恋爱}{tan2lian4'ai4}
    \definition{v.}{namorar | apaixonar-se}
  \end{phonetics}
\end{entry}

\begin{entry}{豹子}{10,3}[Radicais ⾘、⼦]
  \begin{phonetics}{豹子}{bao4zi5}
    \definition[头]{s.}{leopardo}
  \end{phonetics}
\end{entry}

\begin{entry}{资}{10}[Radical ⾙]
  \begin{phonetics}{资}{zi1}
    \definition{s.}{recursos | capital | dinheiro | despesa}
    \definition{v.}{fornecer | suprir}
  \end{phonetics}
\end{entry}

\begin{entry}{资助}{10,7}[Radicais ⾙、⼒]
  \begin{phonetics}{资助}{zi1zhu4}
    \definition{s.}{subsídio}
    \definition{v.}{subsidiar | fornecer ajuda financeira}
  \end{phonetics}
\end{entry}

\begin{entry}{资金}{10,8}[Radicais ⾙、⾦]
  \begin{phonetics}{资金}{zi1jin1}[][HSK 3]
    \definition[笔]{s.}{fundo; capital; capital para atividades empresariais}
  \end{phonetics}
\end{entry}

\begin{entry}{资格}{10,10}[Radicais ⾙、⽊]
  \begin{phonetics}{资格}{zi1ge2}[][HSK 3]
    \definition{s.}{qualificação; as condições e identidades necessárias para exercer uma determinada atividade | senioridade; uma identidade formada pelo tempo gasto realizando um determinado trabalho ou atividade}
  \end{phonetics}
\end{entry}

\begin{entry}{赶}{10}[Radical ⾛]
  \begin{phonetics}{赶}{gan3}[][HSK 3]
    \definition*{s.}{sobrenome Gan}
    \definition{prep.}{por; até}
    \definition{v.}{ultrapassar; alcançar | perseguir; correr para; correr atrás; tentar pegar | dirigir | expulsar; afastar | encontrar; deparar-se com; esbarrar em; acontecer com; encontrar-se em (uma situação); aproveitar-se de (uma oportunidade) | ir para}
  \end{phonetics}
\end{entry}

\begin{entry}{赶上}{10,3}[Radicais ⾛、⼀]
  \begin{phonetics}{赶上}{gan3shang4}
    \definition{adv.}{a tempo para}
    \definition{v.}{alcançar | ultrapassar}
  \end{phonetics}
\end{entry}

\begin{entry}{赶忙}{10,6}[Radicais ⾛、⼼]
  \begin{phonetics}{赶忙}{gan3mang2}
    \definition{v.}{acelerar | apressar | se apressar}
  \end{phonetics}
\end{entry}

\begin{entry}{赶早}{10,6}[Radicais ⾛、⽇]
  \begin{phonetics}{赶早}{gan3zao3}
    \definition{adv.}{o mais breve possível | na primeira oportunidade | antes que seja tarde | quanto antes melhor}
  \end{phonetics}
\end{entry}

\begin{entry}{赶快}{10,7}[Radicais ⾛、⼼]
  \begin{phonetics}{赶快}{gan3kuai4}[][HSK 3]
    \definition{adv.}{rapidamente; imediatamente}
  \end{phonetics}
\end{entry}

\begin{entry}{赶走}{10,7}[Radicais ⾛、⾛]
  \begin{phonetics}{赶走}{gan3zou3}
    \definition{v.}{expulsar | voltar atrás}
  \end{phonetics}
\end{entry}

\begin{entry}{赶到}{10,8}[Radicais ⾛、⼑]
  \begin{phonetics}{赶到}{gan3 dao4}[][HSK 3]
    \definition{v.}{apressar (para algum lugar); avançar de súbito}
  \end{phonetics}
\end{entry}

\begin{entry}{赶赴}{10,9}[Radicais ⾛、⾛]
  \begin{phonetics}{赶赴}{gan3fu4}
    \definition{v.}{apressar}
  \end{phonetics}
\end{entry}

\begin{entry}{赶紧}{10,10}[Radicais ⾛、⽷]
  \begin{phonetics}{赶紧}{gan3jin3}[][HSK 3]
    \definition{adv.}{apressadamente; sem demora}
  \end{phonetics}
\end{entry}

\begin{entry}{赶脚}{10,11}[Radicais ⾛、⾁]
  \begin{phonetics}{赶脚}{gan3jiao3}
    \definition{v.}{transportar mercadorias para ganhar a vida (especialmente de burro) | trabalhar como carroceiro ou porteiro}
  \end{phonetics}
\end{entry}

\begin{entry}{赶跑}{10,12}[Radicais ⾛、⾜]
  \begin{phonetics}{赶跑}{gan3pao3}
    \definition{v.}{afastar | forçar a saída | repelir}
  \end{phonetics}
\end{entry}

\begin{entry}{赶集}{10,12}[Radicais ⾛、⾫]
  \begin{phonetics}{赶集}{gan3ji2}
    \definition{v.}{ir a uma feira | ir ao mercado}
  \end{phonetics}
\end{entry}

\begin{entry}{赶路}{10,13}[Radicais ⾛、⾜]
  \begin{phonetics}{赶路}{gan3lu4}
    \definition{v.}{apressar a jornada | apressar-se}
  \end{phonetics}
\end{entry}

\begin{entry}{起}{10}[Radical ⾛]
  \begin{phonetics}{起}{qi3}[][HSK 1]
    \definition*{s.}{sobrenome Qi}
    \definition{clas.}{caso; instância | lote; grupo}
    \definition{v.}{levantar | levantar-se | extrair| remover | puxar | aparecer | crescer | construir | configurar | começar | iniciar}
  \end{phonetics}
\end{entry}

\begin{entry}{起飞}{10,3}[Radicais ⾛、⾶]
  \begin{phonetics}{起飞}{qi3fei1}[][HSK 2]
    \definition{v.}{decolar}
  \end{phonetics}
\end{entry}

\begin{entry}{起床}{10,7}[Radicais ⾛、⼴]
  \begin{phonetics}{起床}{qi3 chuang2}[][HSK 1]
    \definition{v.+compl.}{sair da cama | levantar-se}
  \end{phonetics}
\end{entry}

\begin{entry}{起来}{10,7}[Radicais ⾛、⽊]
  \begin{phonetics}{起来}{qi3 lai2}[][HSK 1]
    \definition{v.+compl.}{levantar-se}
  \end{phonetics}
\end{entry}

\begin{entry}{起跳}{10,13}[Radicais ⾛、⾜]
  \begin{phonetics}{起跳}{qi3tiao4}
    \definition{v.}{(atletismo) decolar (no início de um salto) | (de preço, salário, etc.) começar (de um determinado nível)}
  \end{phonetics}
\end{entry}

\begin{entry}{较}{10}[Radical ⾞]
  \begin{phonetics}{较}{jiao4}[][HSK 3]
    \definition{adj.}{claro; óbvio; marcado}
    \definition{adv.}{comparativamente; relativamente; razoavelmente; bastante; bastante}
    \definition{prep.}{usado para comparar características e graus}
    \definition{v.}{comparar | disputar}
  \end{phonetics}
\end{entry}

\begin{entry}{辱骂}{10,9}[Radicais ⾠、⾺]
  \begin{phonetics}{辱骂}{ru3ma4}
    \definition{v.}{insultar | abusar}
  \end{phonetics}
\end{entry}

\begin{entry}{透}{10}[Radical ⾡]
  \begin{phonetics}{透}{tou4}
    \definition{adj.}{completo | total}
    \definition{adv.}{completamente | totalmente}
    \definition{v.}{aparecer | passar através | penetrar}
  \end{phonetics}
\end{entry}

\begin{entry}{透支}{10,4}[Radicais ⾡、⽀]
  \begin{phonetics}{透支}{tou4zhi1}
    \definition{v.}{cheque especial (bancário) | saque a descoberto}
  \end{phonetics}
\end{entry}

\begin{entry}{透气}{10,4}[Radicais ⾡、⽓]
  \begin{phonetics}{透气}{tou4qi4}
    \definition{v.}{respirar (sobre tecido, etc.) | fluir livremente (sobre ar) | respirar ar fresco | ventilar}
  \end{phonetics}
\end{entry}

\begin{entry}{透水}{10,4}[Radicais ⾡、⽔]
  \begin{phonetics}{透水}{tou4shui3}
    \definition{adj.}{permeável}
    \definition{s.}{vazamento de água}
  \end{phonetics}
\end{entry}

\begin{entry}{透过}{10,6}[Radicais ⾡、⾡]
  \begin{phonetics}{透过}{tou4guo4}
    \definition{v.}{passar através | penetrar}
  \end{phonetics}
\end{entry}

\begin{entry}{透彻}{10,7}[Radicais ⾡、⼻]
  \begin{phonetics}{透彻}{tou4che4}
    \definition{adj.}{minucioso | incisivo | penetrante}
  \end{phonetics}
\end{entry}

\begin{entry}{透明}{10,8}[Radicais ⾡、⽇]
  \begin{phonetics}{透明}{tou4ming2}
    \definition{adj.}{transparente | (figurativo) transparente, aberto a escrutínio}
  \end{phonetics}
\end{entry}

\begin{entry}{透顶}{10,8}[Radicais ⾡、⾴]
  \begin{phonetics}{透顶}{tou4ding3}
    \definition{adv.}{completamente}
  \end{phonetics}
\end{entry}

\begin{entry}{透亮}{10,9}[Radicais ⾡、⼇]
  \begin{phonetics}{透亮}{tou4liang4}
    \definition{adj.}{brilhante | claro como cristal}
  \end{phonetics}
\end{entry}

\begin{entry}{透辟}{10,13}[Radicais ⾡、⾟]
  \begin{phonetics}{透辟}{tou4pi4}
    \definition{adj.}{incisivo | penetrante}
  \end{phonetics}
\end{entry}

\begin{entry}{透澈}{10,15}[Radicais ⾡、⽔]
  \begin{phonetics}{透澈}{tou4che4}
    \variantof{透彻}
  \end{phonetics}
\end{entry}

\begin{entry}{透露}{10,21}[Radicais ⾡、⾬]
  \begin{phonetics}{透露}{tou4lu4}
    \definition{v.}{divulgar | vazar | revelar}
  \end{phonetics}
\end{entry}

\begin{entry}{逐步}{10,7}[Radicais ⾡、⽌]
  \begin{phonetics}{逐步}{zhu2bu4}
    \definition{adv.}{pouco a pouco; passo a passo; progressivamente}
  \end{phonetics}
\end{entry}

\begin{entry}{逐渐}{10,11}[Radicais ⾡、⽔]
  \begin{phonetics}{逐渐}{zhu2jian4}
    \definition{adv.}{pouco a pouco; passo a passo; progressivamente}
  \end{phonetics}
\end{entry}

\begin{entry}{通}{10}[Radical ⾡]
  \begin{phonetics}{通}{tong1}[][HSK 2]
    \definition{clas.}{para cartas, telegramas, telefonemas, etc.}
    \definition{suf.}{especialista}
    \definition{v.}{ligar para | conseguir a ligação}
  \end{phonetics}
  \begin{phonetics}{通}{tong4}
    \definition{clas.}{para uma atividade, tomada em sua totalidade (discurso de abuso, período de reprodução de música, bebedeira, etc.)}
  \end{phonetics}
\end{entry}

\begin{entry}{通观}{10,6}[Radicais ⾡、⾒]
  \begin{phonetics}{通观}{tong1guan1}
    \definition{v.}{ter uma visão geral de algo}
  \end{phonetics}
\end{entry}

\begin{entry}{通过}{10,6}[Radicais ⾡、⾡]
  \begin{phonetics}{通过}{tong1guo4}[][HSK 2]
    \definition{adv.}{por meio de | através de | via}
    \definition{v.}{passar por | adotar (uma resolução), aprovar (legislação) | passar (em um teste)}
  \end{phonetics}
\end{entry}

\begin{entry}{通识}{10,7}[Radicais ⾡、⾔]
  \begin{phonetics}{通识}{tong1shi2}
    \definition{s.}{conhecimento comum | erudição | conhecimento geral | amplamente conhecido}
  \end{phonetics}
\end{entry}

\begin{entry}{通知}{10,8}[Radicais ⾡、⽮]
  \begin{phonetics}{通知}{tong1zhi1}[][HSK 2]
    \definition[份,个,张]{s.}{aviso | circular}
    \definition{v.}{aconselhar | notificar | informar | dar aviso}
  \end{phonetics}
\end{entry}

\begin{entry}{通信}{10,9}[Radicais ⾡、⼈]
  \begin{phonetics}{通信}{tong1 xin4}[][HSK 3]
    \definition{v.+compl.}{corresponder; comunicar por carta | transmitir (ou transportar) mensagem; passar (ou transmitir) informação}
  \end{phonetics}
\end{entry}

\begin{entry}{通常}{10,11}[Radicais ⾡、⼱]
  \begin{phonetics}{通常}{tong1chang2}[][HSK 3]
    \definition{adj.}{usual; normal; geral}
    \definition{adv.}{habitualmente; usualmente; geralmente; ordinariamente}
  \end{phonetics}
\end{entry}

\begin{entry}{通牒}{10,13}[Radicais ⾡、⽚]
  \begin{phonetics}{通牒}{tong1die2}
    \definition{s.}{nota diplomática}
  \end{phonetics}
\end{entry}

\begin{entry}{速度}{10,9}[Radicais ⾡、⼴]
  \begin{phonetics}{速度}{su4du4}[][HSK 3]
    \definition[个,种]{s.}{velocidade; taxa; ritmo; andamento | velocidade; rapidez}
  \end{phonetics}
\end{entry}

\begin{entry}{造}{10}[Radical ⾡]
  \begin{phonetics}{造}{zao4}[][HSK 3]
    \definition*{s.}{sobrenome Zao}
    \definition{clas.}{para colheitas ou número de colheitas de safras}
    \definition{s.}{uma das duas partes em um acordo legal ou uma ação judicial | colheita; safra}
    \definition{v.}{fazer; construir; criar; produzir | cozinhar; fabricar; inventar | ir para; chegar a | alcançar; atingir | treinar; educar}
  \end{phonetics}
\end{entry}

\begin{entry}{造成}{10,6}[Radicais ⾡、⼽]
  \begin{phonetics}{造成}{zao4cheng2}[][HSK 3]
    \definition{v.}{criar; causar; acarretar; dar origem a; formar; levar a (principalmente resultados ruins)}
  \end{phonetics}
\end{entry}

\begin{entry}{部}{10}[Radical ⾢]
  \begin{phonetics}{部}{bu4}[][HSK 3]
    \definition{clas.}{para obras de literatura, filmes, máquinas etc.}
    \definition[根]{s.}{departamento | divisão | ministério | seção | parte | tropas}
  \end{phonetics}
\end{entry}

\begin{entry}{部下}{10,3}[Radicais ⾢、⼀]
  \begin{phonetics}{部下}{bu4xia4}
    \definition{s.}{subordinado | tropas sob comando de alguém}
  \end{phonetics}
\end{entry}

\begin{entry}{部门}{10,3}[Radicais ⾢、⾨]
  \begin{phonetics}{部门}{bu4men2}[][HSK 3]
    \definition[个]{s.}{filial | departamento | divisão | seção}
  \end{phonetics}
\end{entry}

\begin{entry}{部分}{10,4}[Radicais ⾢、⼑]
  \begin{phonetics}{部分}{bu4fen5}[][HSK 2]
    \definition[个]{s.}{parte | parte de | uma parte de | pedaço | secção}
  \end{phonetics}
\end{entry}

\begin{entry}{部长}{10,4}[Radicais ⾢、⾧]
  \begin{phonetics}{部长}{bu4 zhang3}[][HSK 3]
    \definition[个,位,名]{s.}{ministro | chefe de departamento | chefe de seção}
  \end{phonetics}
\end{entry}

\begin{entry}{部队}{10,4}[Radicais ⾢、⾩]
  \begin{phonetics}{部队}{bu4dui4}
    \definition[个]{s.}{exército | forças armadas | tropas | unidades}
  \end{phonetics}
\end{entry}

\begin{entry}{部族}{10,11}[Radicais ⾢、⽅]
  \begin{phonetics}{部族}{bu4zu2}
    \definition{adj.}{tribal}
    \definition{s.}{tribo}
  \end{phonetics}
\end{entry}

\begin{entry}{部属}{10,12}[Radicais ⾢、⼫]
  \begin{phonetics}{部属}{bu4shu3}
    \definition{s.}{afiliado a um ministério | subordinado | tropas sob comando de alguém}
  \end{phonetics}
\end{entry}

\begin{entry}{部署}{10,13}[Radicais ⾢、⽹]
  \begin{phonetics}{部署}{bu4shu3}
    \definition{s.}{implantação}
    \definition{v.}{implantar}
  \end{phonetics}
\end{entry}

\begin{entry}{都}{10}[Radical ⾢]
  \begin{phonetics}{都}{dou1}[][HSK 1]
    \definition{adv.}{todos | ambos | inteiramente | até | já (usado para dar ênfase) | (não) em tudo}
  \end{phonetics}
  \begin{phonetics}{都}{du1}
    \definition*{s.}{sobrenome Du}
    \definition{s.}{capital | metrópole}
  \end{phonetics}
\end{entry}

\begin{entry}{配}{10}[Radical ⾣]
  \begin{phonetics}{配}{pei4}[][HSK 3]
    \definition{s.}{esposa}
    \definition{v.}{unir-se em matrimônio | acasalar (animais) | compor; combinar; mesclar; amalgamar; misturar |distribuir de acordo com o plano; repartir | encontrar algo para encaixar ou substituir outra coisa | corresponder; combinar; equiparar | merecer; ser digno de; ser qualificado}
  \end{phonetics}
\end{entry}

\begin{entry}{配合}{10,6}[Radicais ⾣、⼝]
  \begin{phonetics}{配合}{pei4he2}[][HSK 3]
    \definition{s.}{coordenação}
    \definition{v.}{cooperar; coordenar}
  \end{phonetics}
\end{entry}

\begin{entry}{酒}{10}[Radical ⾣]
  \begin{phonetics}{酒}{jiu3}[][HSK 2]
    \definition[杯,瓶,罐,桶,缸]{s.}{bebida alcoólica | vinho (especialmente vinho de arroz) | aguardente | licor | espíritos}
  \end{phonetics}
\end{entry}

\begin{entry}{酒店}{10,8}[Radicais ⾣、⼴]
  \begin{phonetics}{酒店}{jiu3 dian4}[][HSK 2]
    \definition[家]{s.}{hotel | restaurante}
  \end{phonetics}
\end{entry}

\begin{entry}{酒鬼}{10,9}[Radicais ⾣、⿁]
  \begin{phonetics}{酒鬼}{jiu3gui3}
    \definition{adj.}{embriagado | ébrio}
    \definition{s.}{bêbado | alcoólatra | borracho}
  \end{phonetics}
\end{entry}

\begin{entry}{酒馆}{10,11}[Radicais ⾣、⾷]
  \begin{phonetics}{酒馆}{jiu3guan3}
    \definition{s.}{bar | taverna | adega}
  \end{phonetics}
\end{entry}

\begin{entry}{钱}{10}[Radical ⾦]
  \begin{phonetics}{钱}{qian2}[][HSK 1]
    \definition*{s.}{sobrenome Qian}
    \definition[笔]{s.}{moeda | dinheiro}
  \end{phonetics}
\end{entry}

\begin{entry}{钱包}{10,5}[Radicais ⾦、⼓]
  \begin{phonetics}{钱包}{qian2bao1}[][HSK 1]
    \definition{s.}{carteira | bolsa}
  \end{phonetics}
\end{entry}

\begin{entry}{钻石}{10,5}[Radicais ⾦、⽯]
  \begin{phonetics}{钻石}{zuan4shi2}
    \definition[颗]{s.}{diamante}
  \end{phonetics}
\end{entry}

\begin{entry}{钻戒}{10,7}[Radicais ⾦、⼽]
  \begin{phonetics}{钻戒}{zuan4jie4}
    \definition[只]{s.}{anel de diamante}
  \end{phonetics}
\end{entry}

\begin{entry}{钿}{10}[Radical ⾦]
  \begin{phonetics}{钿}{dian4}
    \definition{s.}{ornamento incrustado antigo em forma de flor}
    \definition{v.}{incrustar com ouro, prata, etc.}
  \end{phonetics}
  \begin{phonetics}{钿}{tian2}
    \definition{s.}{(dialeto) moeda, dinheiro}
  \end{phonetics}
\end{entry}

\begin{entry}{铁}{10}[Radical ⾦]
  \begin{phonetics}{铁}{tie3}[][HSK 3]
    \definition*{s.}{sobrenome Tie}
    \definition{adj.}{duro; forte; sólido como ferro | violento | inabalável; inalterável; determinado | (gíria) apertado}
    \definition{s.}{ferro (Fe) | arma; armamento}
    \definition{v.}{resolver; determinar}
  \end{phonetics}
\end{entry}

\begin{entry}{铁轨}{10,6}[Radicais ⾦、⾞]
  \begin{phonetics}{铁轨}{tie3gui3}
    \definition[根]{s.}{trilho | trilho ferroviário}
  \end{phonetics}
\end{entry}

\begin{entry}{铁路}{10,13}[Radicais ⾦、⾜]
  \begin{phonetics}{铁路}{tie3 lu4}[][HSK 3]
    \definition[条]{s.}{ferrovia; estrada de ferro}
  \end{phonetics}
\end{entry}

\begin{entry}{阅兵式}{10,7,6}[Radicais ⾨、⼋、⼷]
  \begin{phonetics}{阅兵式}{yue4bing1shi4}
    \definition{s.}{parada militar}
  \end{phonetics}
\end{entry}

\begin{entry}{阅览室}{10,9,9}[Radicais ⾨、⾒、⼧]
  \begin{phonetics}{阅览室}{yue4lan3shi4}
    \definition[间]{s.}{sala de leitura}
  \end{phonetics}
\end{entry}

\begin{entry}{阅读}{10,10}[Radicais ⾨、⾔]
  \begin{phonetics}{阅读}{yue4du2}
    \definition{s.}{leitura}
    \definition{v.}{ler}
  \end{phonetics}
\end{entry}

\begin{entry}{阅读广度}{10,10,3,9}[Radicais ⾨、⾔、⼴、⼴]
  \begin{phonetics}{阅读广度}{yue4du2guang3du4}
    \definition{s.}{intervalo de leitura}
  \end{phonetics}
\end{entry}

\begin{entry}{阅读时间}{10,10,7,7}[Radicais ⾨、⾔、⽇、⾨]
  \begin{phonetics}{阅读时间}{yue4du2shi2jian1}
    \definition{s.}{tempo de leitura}
  \end{phonetics}
\end{entry}

\begin{entry}{阅读理解}{10,10,11,13}[Radicais ⾨、⾔、⽟、⾓]
  \begin{phonetics}{阅读理解}{yue4du2li3jie3}
    \definition{s.}{compreensão de leitura}
  \end{phonetics}
\end{entry}

\begin{entry}{阅读装置}{10,10,12,13}[Radicais ⾨、⾔、⾐、⽹]
  \begin{phonetics}{阅读装置}{yue4du2zhuang1zhi4}
    \definition{s.}{dispositivo de leitura (por exemplo, para códigos de barras, etiquetas RFID, etc.)}
  \end{phonetics}
\end{entry}

\begin{entry}{阅读障碍}{10,10,13,13}[Radicais ⾨、⾔、⾩、⽯]
  \begin{phonetics}{阅读障碍}{yue4du2zhang4ai4}
    \definition{s.}{dislexia}
  \end{phonetics}
\end{entry}

\begin{entry}{阅读器}{10,10,16}[Radicais ⾨、⾔、⼝]
  \begin{phonetics}{阅读器}{yue4du2qi4}
    \definition{s.}{leitor (\emph{software})}
  \end{phonetics}
\end{entry}

\begin{entry}{陪}{10}[Radical ⾩]
  \begin{phonetics}{陪}{pei2}
    \definition{v.}{acompanhar | ajudar | fazer companhia a alguém}
  \end{phonetics}
\end{entry}

\begin{entry}{陵园}{10,7}[Radicais ⾩、⼞]
  \begin{phonetics}{陵园}{ling2yuan2}
    \definition{s.}{cemitério}
  \end{phonetics}
\end{entry}

\begin{entry}{陷入}{10,2}[Radicais ⾩、⼊]
  \begin{phonetics}{陷入}{xian4ru4}
    \definition{v.}{afundar | ser pego em | pousar (em uma situação)}
  \end{phonetics}
\end{entry}

\begin{entry}{难}{10}[Radical ⾫]
  \begin{phonetics}{难}{nan2}[][HSK 1]
    \definition{adj.}{difícil}
    \definition{s.}{dificuldade}
  \end{phonetics}
  \begin{phonetics}{难}{nan4}
    \definition{s.}{desastre}
    \definition{v.}{repreender}
  \end{phonetics}
\end{entry}

\begin{entry}{难过}{10,6}[Radicais ⾫、⾡]
  \begin{phonetics}{难过}{nan2guo4}[][HSK 2]
    \definition{adj.}{triste | ruim | pesaroso | arrependido | difícil}
  \end{phonetics}
\end{entry}

\begin{entry}{难听}{10,7}[Radicais ⾫、⼝]
  \begin{phonetics}{难听}{nan2 ting1}[][HSK 2]
    \definition{adj.}{desagradável de ouvir | ofensivo | grosseiro | escandaloso}
  \end{phonetics}
\end{entry}

\begin{entry}{难受}{10,8}[Radicais ⾫、⼜]
  \begin{phonetics}{难受}{nan2shou4}[][HSK 2]
    \definition{adj.}{sofrer dor | sentir-se mal | desconfortável | sentir-se infeliz}
  \end{phonetics}
\end{entry}

\begin{entry}{难度}{10,9}[Radicais ⾫、⼴]
  \begin{phonetics}{难度}{nan2 du4}[][HSK 3]
    \definition{s.}{dificuldade; grau de dificuldade}
  \end{phonetics}
\end{entry}

\begin{entry}{难看}{10,9}[Radicais ⾫、⽬]
  \begin{phonetics}{难看}{nan2 kan4}[][HSK 2]
    \definition{adj.}{feio | antiestético | vergonhoso | embaraçoso | vergonhoso}
  \end{phonetics}
\end{entry}

\begin{entry}{难道}{10,12}[Radicais ⾫、⾡]
  \begin{phonetics}{难道}{nan2dao4}[][HSK 3]
    \definition{adv.}{indica uma pergunta retórica | certamente não significa que\dots?; é possível que\dots?; não me diga\dots; poderia ser que\dots?}
  \end{phonetics}
\end{entry}

\begin{entry}{难题}{10,15}[Radicais ⾫、⾴]
  \begin{phonetics}{难题}{nan2 ti2}[][HSK 2]
    \definition[出]{s.}{desafio | problema difícil | pergunta difícil}
  \end{phonetics}
\end{entry}

\begin{entry}{顽强}{10,12}[Radicais ⾴、⼸]
  \begin{phonetics}{顽强}{wan2qiang2}
    \definition{adj.}{persistente | tenaz | difícil de derrotar}
  \end{phonetics}
\end{entry}

\begin{entry}{顾客}{10,9}[Radicais ⾴、⼧]
  \begin{phonetics}{顾客}{gu4ke4}[][HSK 2]
    \definition[位]{s.}{cliente}
  \end{phonetics}
\end{entry}

\begin{entry}{顿}{10}[Radical ⾴]
  \begin{phonetics}{顿}{dun4}[][HSK 3]
    \definition*{s.}{sobrenome Dun}
    \definition{adj.}{cansado; fatigado}
    \definition{adv.}{de repente; imediatamente}
    \definition{clas.}{para refeições | para surras, repreensões, etc.}
    \definition{s.}{um lugar para ficar}
    \definition{v.}{pausar
pausar na escrita para reforçar o início ou o fim de um traço
tocar o chão (com a cabeça)
pisar (o pé)
resolver; arranjar
montar acampamento; ficar temporariamente}
  \end{phonetics}
\end{entry}

\begin{entry}{预}{10}[Radical ⾴]
  \begin{phonetics}{预}{yu4}
    \definition{adv.}{antecipadamente}
    \definition{v.}{avançar | preparar}
  \end{phonetics}
\end{entry}

\begin{entry}{预习}{10,3}[Radicais ⾴、⼄]
  \begin{phonetics}{预习}{yu4xi2}[][HSK 3]
    \definition{v.}{pré-visualizar; preparar uma lição; estudar com antecedência as aulas que irá assistir}
  \end{phonetics}
\end{entry}

\begin{entry}{预见}{10,4}[Radicais ⾴、⾒]
  \begin{phonetics}{预见}{yu4jian4}
    \definition{s.}{previsão; intuição; vislumbre}
    \definition{v.}{prever}
  \end{phonetics}
\end{entry}

\begin{entry}{预计}{10,4}[Radicais ⾴、⾔]
  \begin{phonetics}{预计}{yu4 ji4}[][HSK 3]
    \definition{v.}{estimar; esperar; calcular com antecedência}
  \end{phonetics}
\end{entry}

\begin{entry}{预付}{10,5}[Radicais ⾴、⼈]
  \begin{phonetics}{预付}{yu4fu4}
    \definition{s.}{pré-pago}
    \definition{v.}{pagar antecipadamente}
  \end{phonetics}
\end{entry}

\begin{entry}{预约}{10,6}[Radicais ⾴、⽷]
  \begin{phonetics}{预约}{yu4yue1}
    \definition{s.}{reserva}
    \definition{v.}{agendar | marcar compromisso}
  \end{phonetics}
\end{entry}

\begin{entry}{预防}{10,6}[Radicais ⾴、⾩]
  \begin{phonetics}{预防}{yu4fang2}[][HSK 3]
    \definition{v.}{prevenir; proteger-se contra; tomar precauções contra}
  \end{phonetics}
\end{entry}

\begin{entry}{预判}{10,7}[Radicais ⾴、⼑]
  \begin{phonetics}{预判}{yu4pan4}
    \definition{v.}{prever | antecipar}
  \end{phonetics}
\end{entry}

\begin{entry}{预报}{10,7}[Radicais ⾴、⼿]
  \begin{phonetics}{预报}{yu4bao4}[][HSK 3]
    \definition[个]{s.}{boletim meteorológico; previsões meteorológicas antecipadas}
    \definition{v.}{prever (o tempo); relato de coisas antes que elas aconteçam, usado principalmente em clima, astronomia, desastres naturais, etc.}
  \end{phonetics}
\end{entry}

\begin{entry}{预定}{10,8}[Radicais ⾴、⼧]
  \begin{phonetics}{预定}{yu4ding4}
    \definition{v.}{agendar com antecedência}
  \end{phonetics}
\end{entry}

\begin{entry}{预购}{10,8}[Radicais ⾴、⾙]
  \begin{phonetics}{预购}{yu4gou4}
    \definition{s.}{compra antecipada}
    \definition{v.}{comprar antecipadamente}
  \end{phonetics}
\end{entry}

\begin{entry}{预祝}{10,9}[Radicais ⾴、⽰]
  \begin{phonetics}{预祝}{yu4zhu4}
    \definition{v.}{parabenizar de antemão | oferecer os melhores votos para}
  \end{phonetics}
\end{entry}

\begin{entry}{预览}{10,9}[Radicais ⾴、⾒]
  \begin{phonetics}{预览}{yu4lan3}
    \definition{s.}{visualização}
    \definition{v.}{visualizar}
  \end{phonetics}
\end{entry}

\begin{entry}{预留}{10,10}[Radicais ⾴、⽥]
  \begin{phonetics}{预留}{yu4liu2}
    \definition{v.}{separar | reservar}
  \end{phonetics}
\end{entry}

\begin{entry}{预配}{10,10}[Radicais ⾴、⾣]
  \begin{phonetics}{预配}{yu4pei4}
    \definition{s.}{pré-alocado | pré-cabeado}
    \definition{v.}{pré-alocar | pré-cabear}
  \end{phonetics}
\end{entry}

\begin{entry}{预谋}{10,11}[Radicais ⾴、⾔]
  \begin{phonetics}{预谋}{yu4mou2}
    \definition{adj.}{premeditado}
    \definition{v.}{planejar algo com antecedência (especialmente um crime)}
  \end{phonetics}
\end{entry}

\begin{entry}{预提}{10,12}[Radicais ⾴、⼿]
  \begin{phonetics}{预提}{yu4ti2}
    \definition{s.}{retenção}
    \definition{v.}{reter (imposto)}
  \end{phonetics}
\end{entry}

\begin{entry}{预感}{10,13}[Radicais ⾴、⼼]
  \begin{phonetics}{预感}{yu4gan3}
    \definition{s.}{premonição}
    \definition{v.}{ter uma premonição}
  \end{phonetics}
\end{entry}

\begin{entry}{预警}{10,19}[Radicais ⾴、⾔]
  \begin{phonetics}{预警}{yu4jing3}
    \definition{s.}{aviso | aviso antecipado}
  \end{phonetics}
\end{entry}

\begin{entry}{饿}{10}[Radical ⾷]
  \begin{phonetics}{饿}{e4}[][HSK 1]
    \definition{adj.}{faminto}
    \definition{s.}{fome}
    \definition{v.}{morrer de fome}
  \end{phonetics}
\end{entry}

\begin{entry}{高}{10}[Kangxi 189][Radical ⾼]
  \begin{phonetics}{高}{gao1}[][HSK 1]
    \definition*{s.}{sobrenome Gao}
    \definition{adj.}{alto | acima da média}
    \definition{pron.}{Seu (honorífico)}
  \end{phonetics}
\end{entry}

\begin{entry}{高中}{10,4}[Radicais ⾼、⼁]
  \begin{phonetics}{高中}{gao1 zhong1}[][HSK 2]
    \definition{s.}{escola secundária | escola de segundo grau}
  \end{phonetics}
\end{entry}

\begin{entry}{高手}{10,4}[Radicais ⾼、⼿]
  \begin{phonetics}{高手}{gao1shou3}
    \definition{s.}{\emph{expert} | mestre}
  \end{phonetics}
\end{entry}

\begin{entry}{高尔夫}{10,5,4}[Radicais ⾼、⼩、⼤]
  \begin{phonetics}{高尔夫}{gao1'er3fu1}
    \definition{s.}{(empréstimo linguístico) \emph{golf}}
  \end{phonetics}
\end{entry}

\begin{entry}{高兴}{10,6}[Radicais ⾼、⼋]
  \begin{phonetics}{高兴}{gao1xing4}[][HSK 1]
    \definition{adj.}{feliz | contente | disposto (a fazer alguma coisa) | de bom humor}
  \end{phonetics}
\end{entry}

\begin{entry}{高级}{10,6}[Radicais ⾼、⽷]
  \begin{phonetics}{高级}{gao1ji2}[][HSK 2]
    \definition{adj.}{sênior | alto escalão | alto nível | alto grau | grau superior | alta qualidade | avançado}
  \end{phonetics}
\end{entry}

\begin{entry}{高效}{10,10}[Radicais ⾼、⽁]
  \begin{phonetics}{高效}{gao1xiao4}
    \definition{adj.}{eficiente | altamente eficaz}
  \end{phonetics}
\end{entry}

\begin{entry}{高速}{10,10}[Radicais ⾼、⾡]
  \begin{phonetics}{高速}{gao1 su4}[][HSK 3]
    \definition{adj.}{alta velocidade}
    \definition{s.}{auto-estrada; via expressa}
  \end{phonetics}
\end{entry}

\begin{entry}{高速公路}{10,10,4,13}[Radicais ⾼、⾡、⼋、⾜]
  \begin{phonetics}{高速公路}{gao1su4gong1lu4}[][HSK 3]
    \definition[条]{s.}{via expressa; rodovia; auto-estrada}
  \end{phonetics}
\end{entry}

\begin{entry}{高楼}{10,13}[Radicais ⾼、⽊]
  \begin{phonetics}{高楼}{gao1lou2}
    \definition[座]{s.}{edifício alto | edifício de muitos andares | arranha-céu}
  \end{phonetics}
\end{entry}

\begin{entry}{高跟鞋}{10,13,15}[Radicais ⾼、⾜、⾰]
  \begin{phonetics}{高跟鞋}{gao1gen1xie2}
    \definition{s.}{sapatos de salto alto}
  \end{phonetics}
\end{entry}

\begin{entry}{鸭}{10}[Radical ⿃]
  \begin{phonetics}{鸭}{ya1}
    \definition[只]{s.}{pato | (gíria) prostituto}
  \end{phonetics}
\end{entry}

\begin{entry}{鸭子}{10,3}[Radicais ⿃、⼦]
  \begin{phonetics}{鸭子}{ya1zi5}
    \definition[只]{s.}{pato | (gíria) prostituto}
  \end{phonetics}
\end{entry}

\begin{entry}{鸵鸟}{10,5}[Radicais ⿃、⿃]
  \begin{phonetics}{鸵鸟}{tuo2niao3}
    \definition{s.}{avestruz}
  \end{phonetics}
\end{entry}

%%%%% EOF %%%%%

