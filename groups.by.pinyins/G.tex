%%%
%%% G
%%%
\section*{G}\addcontentsline{toc}{section}{G}\addcontentsline{loh}{figure}{\#\#\#\#\#\#\#\# G}

%%%%%%%%%% 夹 %%%%%%%%%%
\subsection*{夹}\addcontentsline{loh}{figure}{夹 \dpy{ga1}}

\begin{EntryWithPhonetic}{夹}{ga1}{6}{⼤}
  \definition{s.}{axila; sovaco; atualmente, costuma-se escrever 胳肢窝}
  \seeref{jia1}
  \seeref{jia2}
  \seealsoref{胳肢窝}{ga1 zhi1 wo1}
\end{EntryWithPhonetic}

%%%%%%%%%% 胳 %%%%%%%%%%
\subsection*{胳}\addcontentsline{loh}{figure}{胳 \dpy{ga1}}

\begin{EntryWithPhonetic}{胳}{ga1}{10}{⾁}
  \definition{s.}{usado em 胳肢窝}
  \seeref{ge1}
  \seeref{ge2}
  \seealsoref{胳肢窝}{ga1 zhi1 wo1}
\end{EntryWithPhonetic}

\begin{EntryWithPhonetic}{胳肢窝}{ga1 zhi1 wo1}{10,8,12}{⾁,⾁,⽳}
  \definition{s.}{axila; sovaco; também escrito 夹肢窝}
  \seealsoref{夹肢窝}{jia1 zhi1 wo1}
\end{EntryWithPhonetic}

%%%%%%%%%% 该 %%%%%%%%%%
\subsection*{该}\addcontentsline{loh}{figure}{该 \dpy{gai1}}

\begin{EntryWithPhonetic}{该}{gai1}{8}{⾔}[HSK 2,7-9]
  \definition{adj.}{completo; integral; abrangente; inclusivo; o mesmo que 赅}
  \definition{pron.}{isto; aquilo; o referido; o acima mencionado; indica a pessoa ou coisa mencionada acima, equivalente a 此, 这个, etc.}
  \definition{v.}{deveria ser; deveria ser assim | caber a alguém; ser a vez (ou dever) de alguém fazer algo | merecer; servir a alguém de direito; indica que algo deve ser feito | dever | deve; provavelmente irá; muito provavelmente; pode ser razoavelmente ou naturalmente esperado que; expressa uma conclusão lógica ou provável com base na razão ou na experiência}
  \definition{v.aux.}{usado em frases exclamativas, tem a função de reforçar o tom}
  \seealsoref{此}{ci3}
  \seealsoref{赅}{gai1}
  \seealsoref{这个}{zhe4ge5}
\end{EntryWithPhonetic}

%%%%%%%%%% 赅 %%%%%%%%%%
\subsection*{赅}\addcontentsline{loh}{figure}{赅 \dpy{gai1}}

\begin{EntryWithPhonetic}{赅}{gai1}{10}{⾙}
  \definition*{s.}{Sobrenome: Gai}
  \definition{adj.}{completo; integral; abrangente; inclusivo}
\end{EntryWithPhonetic}

%%%%%%%%%% 改 %%%%%%%%%%
\subsection*{改}\addcontentsline{loh}{figure}{改 \dpy{gai3}}

\begin{EntryWithPhonetic}{改}{gai3}{7}{⽁}[HSK 2]
  \definition{v.}{mudar; converter; transformar; alterar; substituir | alterar; revisar; aperfeiçoar; modificar | corrigir; retificar; remediar; consertar}
\end{EntryWithPhonetic}

\begin{EntryWithPhonetic}{改版}{gai3/ban3}{7,8}{⽁,⽚}[HSK 7-9]
  \definition{s.}{(programas de rádio ou TV) reformulação; ajuste | edição revisada}
  \definition{v.+compl.}{alterar o layout de uma folha impressa | alterar ou corrigir uma página definida | revisar a edição atual}
\end{EntryWithPhonetic}

\begin{EntryWithPhonetic}{改编}{gai3bian1}{7,12}{⽁,⽷}[HSK 7-9]
  \definition{v.}{adaptar; revisar; converter; reorganizar; transcrever; reescrever com base no trabalho original (geralmente em um gênero diferente) | reorganizar; redesignar; alterar a organização original (referindo"-se principalmente ao exército)}
\end{EntryWithPhonetic}

\begin{EntryWithPhonetic}{改变}{gai3bian4}{7,8}{⽁,⼜}[HSK 2]
  \definition{v.}{mudar; alterar; transformar; converter; moldar; modificar | causar mudanças; alterar}
\end{EntryWithPhonetic}

\begin{EntryWithPhonetic}{改动}{gai3dong4}{7,6}{⽁,⼒}[HSK 7-9]
  \definition{v.}{mudar; alterar; modificar; polir; melhorar | alterar (texto, itens, ordem, etc.)}
\end{EntryWithPhonetic}

\begin{EntryWithPhonetic}{改革}{gai3ge2}{7,9}{⽁,⾰}[HSK 5]
  \definition[项,次,种]{s.}{reforma; reformação; iniciativas para aprimorar a inovação}
  \definition{v.}{reformar; transformar as antigas partes irracionais das coisas em novas que possam ser adaptadas à situação objetiva}
\end{EntryWithPhonetic}

\begin{EntryWithPhonetic}{改革开放}{gai3ge2 kai1fang4}{7,9,4,8}{⽁,⾰,⼶,⽅}[HSK 7-9]
  \definition{v.}{reformar e abrir"-se ao mundo exterior (refere"-se às políticas de Deng Xiaoping por volta de 1980)}
\end{EntryWithPhonetic}

\begin{EntryWithPhonetic}{改进}{gai3jin4}{7,7}{⽁,⾡}[HSK 3]
  \definition[个,些]{s.}{melhoria}
  \definition{v.}{aprimorar; aperfeiçoar; melhorar; tornar melhor; mudar a situação antiga para melhorar | modificar (mudança mecânica)}
\end{EntryWithPhonetic}

\begin{EntryWithPhonetic}{改良}{gai3liang2}{7,7}{⽁,⾉}[HSK 7-9]
  \definition{v.}{melhorar; amenizar; remover as deficiências individuais das coisas para torná-las mais adequadas às necessidades | reformar; melhorar | Metalurgia: modificar}
\end{EntryWithPhonetic}

\begin{EntryWithPhonetic}{改名}{gai3ming2}{7,6}{⽁,⼝}[HSK 7-9]
  \definition{v.}{mudar o próprio nome; alterar nome}
\end{EntryWithPhonetic}

\begin{EntryWithPhonetic}{改日}{gai3ri4}{7,4}{⽁,⽇}[HSK 7-9]
  \definition{adv.}{algum outro dia; outro dia}
\end{EntryWithPhonetic}

\begin{EntryWithPhonetic}{改善}{gai3shan4}{7,12}{⽁,⼝}[HSK 4]
  \definition{v.}{melhorar; amenizar; mudar a situação original para torná-la melhor}
\end{EntryWithPhonetic}

\begin{EntryWithPhonetic}{改善关系}{gai3shan4guan1xi5}{7,12,6,7}{⽁,⼝,⼋,⽷}
  \definition{v.}{melhorar a relação}
\end{EntryWithPhonetic}

\begin{EntryWithPhonetic}{改善通讯}{gai3shan4tong1xun4}{7,12,10,5}{⽁,⼝,⾡,⾔}
  \definition{v.}{melhorar a comunicação}
\end{EntryWithPhonetic}

\begin{EntryWithPhonetic}{改为}{gai3wei2}{7,4}{⽁,⼂}[HSK 7-9]
  \definition{v.}{mudar para}[原计划改为明天开始。===O plano original foi mudado para começar amanhã.]
\end{EntryWithPhonetic}

\begin{EntryWithPhonetic}{改邪归正}{gai3xie2-gui1zheng4}{7,6,5,5}{⽁,⾢,⼹,⽌}[HSK 7-9]
  \definition{expr.}{abandonar os maus caminhos e retornar ao caminho certo; abandonar o vício e voltar-se para a virtude; virar uma nova página; retornar a um modo de vida cumpridor da lei}
\end{EntryWithPhonetic}

\begin{EntryWithPhonetic}{改造}{gai3zao4}{7,10}{⽁,⾡}[HSK 3]
  \definition{v.}{transformar; renovar; modificar o original para melhor se adequar às necessidades; usado principalmente para coisas específicas | remodelar; mudar radicalmente o que é velho e ruim; criar algo novo e bom, para se adaptar às novas circunstâncias e necessidades; usado principalmente para coisas abstratas}
\end{EntryWithPhonetic}

\begin{EntryWithPhonetic}{改正}{gai3zheng4}{7,5}{⽁,⽌}[HSK 4]
  \definition{v.}{corrigir; emendar; mudar o errado para o correto}
\end{EntryWithPhonetic}

\begin{EntryWithPhonetic}{改装}{gai3zhuang1}{7,12}{⽁,⾐}[HSK 6]
  \definition{v.}{mudar de traje ou vestido | reembalar | reequipar; reaparelhar | modificar; alterar o dispositivo original}
\end{EntryWithPhonetic}

%%%%%%%%%% 芥 %%%%%%%%%%
\subsection*{芥}\addcontentsline{loh}{figure}{芥 \dpy{gai4}}

\begin{EntryWithPhonetic}{芥}{gai4}{7}{⾋}
  \definition{s.}{mostarda}
  \seeref{jie4}
  \seealsoref{芥蓝}{gai4lan2}
\end{EntryWithPhonetic}

\begin{EntryWithPhonetic}{芥兰}{gai4lan2}{7,5}{⾋,⼋}
  \variantof{芥蓝}
\end{EntryWithPhonetic}

\begin{EntryWithPhonetic}{芥蓝}{gai4lan2}{7,13}{⾋,⾋}
  \definition{s.}{brócolis chinês; couve chinesa; mostarda}
  \seealsoref{格兰菜}{ge2lan2cai4}
\end{EntryWithPhonetic}

%%%%%%%%%% 钙 %%%%%%%%%%
\subsection*{钙}\addcontentsline{loh}{figure}{钙 \dpy{gai4}}

\begin{EntryWithPhonetic}{钙}{gai4}{9}{⾦}[HSK 7-9]
  \definition[克,毫克]{s.}{Ca, cálcio}
\end{EntryWithPhonetic}

%%%%%%%%%% 盖 %%%%%%%%%%
\subsection*{盖}\addcontentsline{loh}{figure}{盖 \dpy{gai4}}

\begin{EntryWithPhonetic}{盖}{gai4}{11}{⽫}[HSK 4]
  \definition*{s.}{Sobrenome: Gai}
  \definition{adj.}{excelente; soberbo; fantástico}
  \definition{adv.}{cerca de; ao redor; aproximadamente; expressa um julgamento especulativo sobre algo, ou uma explicação da causa, o que é equivalente a 大概 ou 原来}
  \definition{conj.}{para; porque; dando continuidade à frase anterior, afirmando a razão ou causa, com tom incerto}
  \definition{s.}{tampa; capa; cobertura; algo que cobre ou sela a parte superior de um objeto | carapaça; concha (de tartaruga, caranguejo, etc.); ossos em formato de crânio em certas partes do corpo humano; as conchas nas costas de certos animais | dossel; capota; toldo | nivelador (uma ferramenta agrícola usada para nivelar terras)}
  \definition{v.}{cobrir; proteger; colocar uma capa em; colocar uma tampa em um objeto | selar; afixar um selo em | superar; sobressair; sobrepujar; ultrapassar | construir; colocar para cima | esconder; ocultar; encobrir | nivelar o terreno com um nivelador (ferramenta agrícola)}
  \seeref{ge3}
  \seealsoref{大概}{da4gai4}
  \seealsoref{原来}{yuan2lai2}
\end{EntryWithPhonetic}

\begin{EntryWithPhonetic}{盖子}{gai4zi5}{11,3}{⽫,⼦}[HSK 7-9]
  \definition[个]{s.}{tampa; cobertura; capa; topo; algo que tem um efeito de proteção na parte superior de um objeto | casco (de tartaruga, etc.); conchas nas costas dos animais}
\end{EntryWithPhonetic}

%%%%%%%%%% 概 %%%%%%%%%%
\subsection*{概}\addcontentsline{loh}{figure}{概 \dpy{gai4}}

\begin{EntryWithPhonetic}{概}{gai4}{13}{⽊}
  \definition{adj.}{geral; aproximado}
  \definition{adv.}{sem exceção; categoricamente}
  \definition{s.}{ideia principal; esboço geral | maneira de se portar e conduzir; comportamento}
  \definition{v.}{generalizar; exemplificar; tipificar}
\end{EntryWithPhonetic}

\begin{EntryWithPhonetic}{概况}{gai4kuang4}{13,7}{⽊,⼎}[HSK 7-9]
  \definition{s.}{situação geral; levantamento; breve relato (de algo); fatos básicos}[个人概况===Perfil Pessoal]
\end{EntryWithPhonetic}

\begin{EntryWithPhonetic}{概括}{gai4kuo4}{13,9}{⽊,⼿}[HSK 4]
  \definition{adj.}{genérico; simples e claro, captando o conteúdo principal}
  \definition{s.}{generalização}
  \definition{v.}{generalizar; resumir}
\end{EntryWithPhonetic}

\begin{EntryWithPhonetic}{概论}{gai4lun4}{13,6}{⽊,⾔}[HSK 7-9]
  \definition{s.}{esboço; introdução; enquete; frequentemente usado em títulos de livros: um resumo da discussão}[«艺术历史概论»===«Introdução à História da Arte»]
\end{EntryWithPhonetic}

\begin{EntryWithPhonetic}{概率}{gai4lv4}{13,11}{⽊,⽞}[HSK 7-9]
  \definition{s.}{acaso; probabilidade; a probabilidade de um certo tipo de evento ocorrer nas mesmas condições}[成功的概率只有8\%。===A probabilidade de sucesso é de apenas 8\%.]
\end{EntryWithPhonetic}

\begin{EntryWithPhonetic}{概念}{gai4nian4}{13,8}{⽊,⼼}[HSK 3]
  \definition[个,种,项]{s.}{ideia; noção; conceito; concepção; uma forma de pensamento que resume as características comuns de algo em uma palavra}
\end{EntryWithPhonetic}

%%%%%%%%%% 干 %%%%%%%%%%
\subsection*{干}\addcontentsline{loh}{figure}{干 \dpy{gan1}}

\begin{EntryWithPhonetic}{干}{gan1}{3}{⼲}[Kangxi 51]
  \definition*{s.}{Sobrenome: Gan}
  \definition{adj.}{seco | vazio; oco; seco | sem substância; vazio | de parentesco nominal; (parentes) não ligados por laços sanguíneos | sem água; (água) esgotada; completamente vazia | assumido como parente nominal; relação familiar reconhecida por adoção | rude; grosseiro; mal-educado; descreve alguém que fala de forma muito direta e rude (sem delicadeza).}
  \definition{adv.}{em vão; fútil; sem propósito; para nada; sem resultado | apenas; sem nada mais | inutilmente; sem uso, sem aproveitamento | superficialmente; significa que não há conteúdo, apenas forma}
  \definition{s.}{Arcaico: escudo | margem; ribeira; margem das águas | alimentos desidratados | abreviação para os dez troncos celestiais}
  \definition{v.}{ofender; afrontar | ter a ver com; estar relacionado com; estar implicado em; interferir com | (antiquado) buscar (cargo público, remuneração, etc.) | (dialeto) deixar alguém de fora; tratar alguém com indiferença; desprezar | assediar; perturbar; criar confusão; causar estragos; bagunçar | solicitar; procurar; buscar (cargo, salário, etc.) | beber até o fim | tratar com indiferença; ignorar}
  \seeref{gan4}
  \seealsoref{干儿}{gan1 er2}
  \seealsoref{干儿}{gan1r5}
  \antonymref{湿}{shi1}
\end{EntryWithPhonetic}

\begin{EntryWithPhonetic}{干杯}{gan1/bei1}{3,8}{⼲,⽊}[HSK 2]
  \definition{interj.}{`Saúde!''}
  \definition{v.+compl.}{fazer um brinde;  brindar até a última gota}
\end{EntryWithPhonetic}

\begin{EntryWithPhonetic}{干脆}{gan1cui4}{3,10}{⼲,⾁}[HSK 5]
  \definition{adj.}{claro; direto; (falar, fazer coisas) sem hesitação; atitude clara}
  \definition{adv.}{justamente; diretamente; sem maiores considerações}
\end{EntryWithPhonetic}

\begin{EntryWithPhonetic}{干儿}{gan1 er2}{3,2}{⼲,⼉}
  \definition{s.}{filho adotivo (adoção tradicional, ou seja, sem implicações legais)}
  \seeref{gan1r5}
\end{EntryWithPhonetic}

\begin{EntryWithPhonetic}{干戈}{gan1ge1}{3,4}{⼲,⼽}[HSK 7-9]
  \definition{s.}{armas de guerra; armas; guerra}[化干戈为玉帛。===Transforme guerra em paz.]
  \antonymref{玉帛}{yu4bo2}
\end{EntryWithPhonetic}

\begin{EntryWithPhonetic}{干旱}{gan1han4}{3,7}{⼲,⽇}[HSK 7-9]
  \definition{adj.}{árido; seco}
\end{EntryWithPhonetic}

\begin{EntryWithPhonetic}{干净}{gan1jing4}{3,8}{⼲,⼎}[HSK 1]
  \definition{adj.}{limpo; limpo e arrumado; sem poeira, impurezas, etc.}
  \definition{adv.}{completamente; totalmente; sem deixar nada para trás}
\end{EntryWithPhonetic}

\begin{EntryWithPhonetic}{干你屁事}{gan1 ni3 pi4shi4}{3,7,7,8}{⼲,⼈,⼫,⼅}
  \definition{interj.}{`Foda-se!''}
\end{EntryWithPhonetic}

\begin{EntryWithPhonetic}{干儿}{gan1r5}{3,2}{⼲,⼉}
  \definition{s.}{alimentos secos, desidratados}
  \seeref{gan1 er2}
\end{EntryWithPhonetic}

\begin{EntryWithPhonetic}{干扰}{gan1rao3}{3,7}{⼲,⼿}[HSK 5]
  \definition{v.}{perturbar; incomodar | interferir; interromper o funcionamento adequado de equipamentos eletrônicos com sinais eletrônicos dispersos}
\end{EntryWithPhonetic}

\begin{EntryWithPhonetic}{干涉}{gan1she4}{3,10}{⼲,⽔}[HSK 6]
  \definition{s.}{interferência; refere"-se ao ato ou comportamento de interferir nos assuntos dos outros}
  \definition{v.}{interferir; intervir; intrometer"-se; pedir ou impedir algo geralmente significa interferir quando não se deve}
\end{EntryWithPhonetic}

\begin{EntryWithPhonetic}{干与}{gan1yu4}{3,3}{⼲,⼀}
  \variantof{干预}
\end{EntryWithPhonetic}

\begin{EntryWithPhonetic}{干预}{gan1yu4}{3,10}{⼲,⾴}[HSK 5]
  \definition{s.}{intromissão; intervenção}
  \definition{v.}{intrometer-se; intervir; interpor-se;}
\end{EntryWithPhonetic}

\begin{EntryWithPhonetic}{干燥}{gan1zao4}{3,17}{⼲,⽕}[HSK 7-9]
  \definition{adj.}{seco; árido; sem umidade ou muito pouca umidade | enfadonho; desinteressante; chato, sem graça}
\end{EntryWithPhonetic}

%%%%%%%%%% 甘 %%%%%%%%%%
\subsection*{甘}\addcontentsline{loh}{figure}{甘 \dpy{gan1}}

\begin{EntryWithPhonetic}{甘}{gan1}{5}{⽢}[Kangxi 99]
  \definition*{s.}{Província de Gansu, abreviação de 甘肃 | Sobrenome: Gan}
  \definition{adj.}{doce; agradável; satisfatório}
  \definition{v.}{estar disposto a; estar contente ou satisfeito com}
  \seealsoref{甘肃}{gan1su4}
\end{EntryWithPhonetic}

\begin{EntryWithPhonetic}{甘薯}{gan1shu3}{5,16}{⽢,⾋}
  \definition{s.}{batata doce}
\end{EntryWithPhonetic}

\begin{EntryWithPhonetic}{甘肃}{gan1su4}{5,8}{⽢,⾀}
  \definition*{s.}{Província de Gansu}
\end{EntryWithPhonetic}

\begin{EntryWithPhonetic}{甘心}{gan1xin1}{5,4}{⽢,⼼}[HSK 7-9]
  \definition{v.}{estar contente com; estar disposto a | reconciliar"-se com; resignar"-se com; contentar"-se com}
\end{EntryWithPhonetic}

%%%%%%%%%% 杆 %%%%%%%%%%
\subsection*{杆}\addcontentsline{loh}{figure}{杆 \dpy{gan1}}

\begin{EntryWithPhonetic}{杆}{gan1}{7}{⽊}
  \definition{s.}{poste; pólo; mastro}
  \seeref{gan3}
\end{EntryWithPhonetic}

%%%%%%%%%% 肝 %%%%%%%%%%
\subsection*{肝}\addcontentsline{loh}{figure}{肝 \dpy{gan1}}

\begin{EntryWithPhonetic}{肝}{gan1}{7}{⾁}[HSK 6]
  \definition[个]{s.}{fígado; um dos órgãos digestivos dos humanos e dos animais superiores}
\end{EntryWithPhonetic}

\begin{EntryWithPhonetic}{肝脏}{gan1zang4}{7,10}{⾁,⾁}[HSK 7-9]
  \definition{s.}{fígado}
\end{EntryWithPhonetic}

%%%%%%%%%% 尴 %%%%%%%%%%
\subsection*{尴}\addcontentsline{loh}{figure}{尴 \dpy{gan1}}

\begin{EntryWithPhonetic}{尴}{gan1}{13}{⼪}
  \definition{adj.}{envergonhado; uma situação ou assunto difícil de lidar | pouco à vontade; expressão não natural; envergonhado}
\end{EntryWithPhonetic}

\begin{EntryWithPhonetic}{尴尬}{gan1ga4}{13,7}{⼪,⼪}[HSK 7-9]
  \definition{adj.}{estranho; envergonhado; quando você se depara com algo difícil de lidar ou algo que o deixa envergonhado}
\end{EntryWithPhonetic}

%%%%%%%%%% 杆 %%%%%%%%%%
\subsection*{杆}\addcontentsline{loh}{figure}{杆 \dpy{gan3}}

\begin{EntryWithPhonetic}{杆}{gan3}{7}{⽊}[HSK 6]
  \definition{clas.}{usado para objetos semelhantes a hastes}
  \definition{s.}{eixo; braço | haste; barra; poste; a parte longa e fina de um objeto, semelhante a um bastão}
  \seeref{gan1}
\end{EntryWithPhonetic}

%%%%%%%%%% 赶 %%%%%%%%%%
\subsection*{赶}\addcontentsline{loh}{figure}{赶 \dpy{gan3}}

\begin{EntryWithPhonetic}{赶}{gan3}{10}{⾛}[HSK 3]
  \definition*{s.}{Sobrenome: Gan}
  \definition{prep.}{por; até; até que; até quando; introduzir o momento em que algo aconteceu, indicando que se espera até um determinado momento}
  \definition{v.}{ultrapassar; alcançar | perseguir; correr atrás; tentar alcançar; dar uma corrida; acelerar ou intensificar  | dirigir; conduzir | expulsar; afugentar; afastar | encontrar; deparar-se com; esbarrar em; acontecer; encontrar-se em (uma situação); aproveitar-se de (uma oportunidade) | ir para; participar (atividades com horário marcado)}
\end{EntryWithPhonetic}

\begin{EntryWithPhonetic}{赶不上}{gan3bu5shang4}{10,4,3}{⾛,⼀,⼀}[HSK 6]
  \definition{v.}{ficar para trás; ser incapaz de alcançar; não conseguir alcançar; não conseguir acompanhar | ser tarde demais (para fazer algo); (não) existir tempo suficiente (para fazer algo) |  deixar de ter; ser incapaz de encontrar ou ter a chance de encontrar; não encontrar; não encontrar (boa oportunidade) | não poder ser comparado a}
\end{EntryWithPhonetic}

\begin{EntryWithPhonetic}{赶到}{gan3dao4}{10,8}{⾛,⼑}[HSK 3]
  \definition{v.}{correr (para algum lugar); apressar-se}
\end{EntryWithPhonetic}

\begin{EntryWithPhonetic}{赶赴}{gan3fu4}{10,9}{⾛,⾛}[HSK 7-9]
  \definition{v.}{apressar-se para; correr para | apressar-se}
\end{EntryWithPhonetic}

\begin{EntryWithPhonetic}{赶集}{gan3ji2}{10,12}{⾛,⾫}
  \definition{v.}{ir a uma feira | ir ao mercado}
\end{EntryWithPhonetic}

\begin{EntryWithPhonetic}{赶脚}{gan3jiao3}{10,11}{⾛,⾁}
  \definition{v.}{transportar mercadorias para ganhar a vida (especialmente de burro) | trabalhar como carroceiro ou porteiro}
\end{EntryWithPhonetic}

\begin{EntryWithPhonetic}{赶紧}{gan3jin3}{10,10}{⾛,⽷}[HSK 3]
  \definition{adv.}{apressadamente; precipitadamente; às pressas; significa agir imediatamente, sem demora}
\end{EntryWithPhonetic}

\begin{EntryWithPhonetic}{赶快}{gan3kuai4}{10,7}{⾛,⼼}[HSK 3]
  \definition{adv.}{rapidamente; imediatamente; aproveite o momento e acelere o ritmo}
\end{EntryWithPhonetic}

\begin{EntryWithPhonetic}{赶路}{gan3lu4}{10,13}{⾛,⾜}
  \definition{v.}{apressar a jornada | apressar-se}
\end{EntryWithPhonetic}

\begin{EntryWithPhonetic}{赶忙}{gan3mang2}{10,6}{⾛,⼼}[HSK 6]
  \definition{adv.}{imediatamente; com pressa; às pressas; rapidamente}
\end{EntryWithPhonetic}

\begin{EntryWithPhonetic}{赶跑}{gan3pao3}{10,12}{⾛,⾜}
  \definition{v.}{afastar | forçar a saída | repelir}
\end{EntryWithPhonetic}

\begin{EntryWithPhonetic}{赶上}{gan3 shang4}{10,3}{⾛,⼀}[HSK 6]
  \definition{v.}{alcançar; manter o ritmo com; acompanhar alguém ou o padrão do planejador | chegar a tempo para; ter tempo suficiente; não ser tarde demais | encontrar; topar com; cruzar com; encontrar-se com; acontecer de encontrar; encontrar algo, em um determinado momento ou oportunidade}
\end{EntryWithPhonetic}

\begin{EntryWithPhonetic}{赶往}{gan3wang3}{10,8}{⾛,⼻}[HSK 7-9]
  \definition{v.}{apressar-se para (algum lugar)}
\end{EntryWithPhonetic}

\begin{EntryWithPhonetic}{赶早}{gan3zao3}{10,6}{⾛,⽇}
  \definition{adv.}{o mais breve possível | na primeira oportunidade | antes que seja tarde | quanto antes melhor}
\end{EntryWithPhonetic}

\begin{EntryWithPhonetic}{赶走}{gan3zou3}{10,7}{⾛,⾛}
  \definition{v.}{expulsar | voltar atrás}
\end{EntryWithPhonetic}

%%%%%%%%%% 敢 %%%%%%%%%%
\subsection*{敢}\addcontentsline{loh}{figure}{敢 \dpy{gan3}}

\begin{EntryWithPhonetic}{敢}{gan3}{11}{⽁}[HSK 3]
  \definition{adj.}{ousado; corajoso; audacioso; valente}
  \definition{adv.}{talvez; provavelmente}
  \definition{v.}{ser ousado o suficiente; atrever-se | ter confiança em; ter certeza; estar certo | aventurar-se; ter coragem de fazer algo | ser ousado; arriscar-se}
\end{EntryWithPhonetic}

\begin{EntryWithPhonetic}{敢情}{gan3qing5}{11,11}{⽁,⼼}[HSK 7-9]
  \definition{adv.}{por que; então; eu digo; indica a descoberta de algo que não foi descoberto anteriormente | claro; de fato; realmente; isso significa que a razão é óbvia e não há necessidade de duvidar dela}
\end{EntryWithPhonetic}

\begin{EntryWithPhonetic}{敢于}{gan3yu2}{11,3}{⽁,⼆}[HSK 6]
  \definition{v.}{ousar; ser ousado em; ter determinação; ter coragem (para fazer ou se esforçar para fazer)}
\end{EntryWithPhonetic}

%%%%%%%%%% 感 %%%%%%%%%%
\subsection*{感}\addcontentsline{loh}{figure}{感 \dpy{gan3}}

\begin{EntryWithPhonetic}{感}{gan3}{13}{⼼}[HSK 7-9]
  \definition{s.}{sentido; sensação; sentimento; impressão | emoção; sentimento}
  \definition{v.}{sentir; perceber; estar ciente | mover; tocar; afetar | ser grato; ser agradecido | ser afetado (pelo frio); pegar um resfriado | (fotografia) sensibilizar | ser grato; apreciar | ser afetado}
\end{EntryWithPhonetic}

\begin{EntryWithPhonetic}{感触}{gan3chu4}{13,13}{⼼,⾓}[HSK 7-9]
  \definition{s.}{sentimento; pensamentos e sentimentos; pensamentos e emoções causados por fatores externos}
\end{EntryWithPhonetic}

\begin{EntryWithPhonetic}{感到}{gan3dao4}{13,8}{⼼,⼑}[HSK 2]
  \definition{v.}{sentir; achar; perceber}
\end{EntryWithPhonetic}

\begin{EntryWithPhonetic}{感动}{gan3dong4}{13,6}{⼼,⼒}[HSK 2]
  \definition{v.}{mover (alguém) | tocar (alguém emocionalmente)}
\end{EntryWithPhonetic}

\begin{EntryWithPhonetic}{感恩}{gan3/en1}{13,10}{⼼,⼼}[HSK 7-9]
  \definition{v.+compl.}{sentir-se grato; ser grato; expressar gratidão pela ajuda dada por outros}
\end{EntryWithPhonetic}

\begin{EntryWithPhonetic}{感激}{gan3ji1}{13,16}{⼼,⽔}[HSK 7-9]
  \definition{v.}{apreciar; ser grato; sentir-se grato; sentir-se em dívida; desenvolver uma impressão favorável de alguém por causa de sua gentileza ou ajuda}
\end{EntryWithPhonetic}

\begin{EntryWithPhonetic}{感觉}{gan3jue2}{13,9}{⼼,⾒}[HSK 2]
  \definition[个]{s.}{sentimento; sensação; percepção sensorial;}
  \definition{v.}{sentir; perceber; tomar consciência; sentir no coração, acreditar}
\end{EntryWithPhonetic}

\begin{EntryWithPhonetic}{感慨}{gan3kai3}{13,12}{⼼,⼼}[HSK 7-9]
  \definition{v.}{suspirar de emoção; estar cheio de emoções; ficar profundamente comovido;  geralmente expresso em palavras}
\end{EntryWithPhonetic}

\begin{EntryWithPhonetic}{感冒}{gan3mao4}{13,9}{⼼,⽇}[HSK 3]
  \definition{adj.}{interessado em}
  \definition[场,次]{s.}{resfriado; gripe comum; \emph{influenza}; doença infecciosa causada por um vírus, que tende a causar sintomas como garganta seca, congestão nasal, tosse, espirros, dor de cabeça e febre quando o corpo está excessivamente cansado, resfriado ou com a imunidade enfraquecida}
  \definition{v.}{pegar (ter) um resfriado}
\end{EntryWithPhonetic}

\begin{EntryWithPhonetic}{感情}{gan3qing2}{13,11}{⼼,⼼}[HSK 3]
  \definition[份,个,种]{s.}{emoção; sentimento; reações psicológicas como amor, ódio, alegria, raiva, tristeza e felicidade, geradas por estímulos externos | amor; afeto; apego; preocupação e afeição por pessoas ou coisas}
\end{EntryWithPhonetic}

\begin{EntryWithPhonetic}{感染}{gan3ran3}{13,9}{⼼,⽊}[HSK 7-9]
  \definition{v.}{infectar; ser infectado com | infectar; afetar; influenciar; evocar os mesmos pensamentos e sentimentos}
\end{EntryWithPhonetic}

\begin{EntryWithPhonetic}{感染力}{gan3ran3li4}{13,9,2}{⼼,⽊,⼒}[HSK 7-9]
  \definition{s.}{poder de mover os sentimentos; apelo | infeccioso (entusiasmo) | inspiração}
\end{EntryWithPhonetic}

\begin{EntryWithPhonetic}{感人}{gan3ren2}{13,2}{⼼,⼈}[HSK 6]
  \definition{adj.}{comovente; tocante}
\end{EntryWithPhonetic}

\begin{EntryWithPhonetic}{感受}{gan3shou4}{13,8}{⼼,⼜}[HSK 3]
  \definition{s.}{percepção ; compreenção; sentimento; experiência; influência do contato com o mundo exterior}
  \definition{v.}{sentir; sentir (através dos sentidos); experimentar; ser afetado}
\end{EntryWithPhonetic}

\begin{EntryWithPhonetic}{感叹}{gan3tan4}{13,5}{⼼,⼝}[HSK 7-9]
  \definition{v.}{suspirar com sentimento; suspirar por causa de um sentimento}
\end{EntryWithPhonetic}

\begin{EntryWithPhonetic}{感想}{gan3xiang3}{13,13}{⼼,⼼}[HSK 5]
  \definition[个,条]{s.}{pensamentos; impressões; reflexões; resposta do pensamento decorrente da exposição ao mundo exterior}
\end{EntryWithPhonetic}

\begin{EntryWithPhonetic}{感谢}{gan3xie4}{13,12}{⼼,⾔}[HSK 2]
  \definition{v.}{agradecer; ser grato; expressar gratidão com palavras ou ações}
\end{EntryWithPhonetic}

\begin{EntryWithPhonetic}{感兴趣}{gan3/xing4qu4}{13,6,15}{⼼,⼋,⾛}[HSK 4]
  \definition{v.+compl.}{estar interessado}
  \seealsoref{对……感兴趣}{dui4 gan3xing4qu4}
\end{EntryWithPhonetic}

\begin{EntryWithPhonetic}{感性}{gan3xing4}{13,8}{⼼,⼼}[HSK 7-9]
  \definition{adj.}{perceptivo; sentimental; emocional; pertencente a formas intuitivas como sensação, percepção e representação}
  \definition{s.}{percepção; sensibilidade; refere"-se a pessoas que são emocionalmente ricas, sentimentais, capazes de ter empatia pelos outros, que têm grande sensibilidade e conseguem entender as mudanças emocionais de qualquer coisa}
\end{EntryWithPhonetic}

%%%%%%%%%% 橄 %%%%%%%%%%
\subsection*{橄}\addcontentsline{loh}{figure}{橄 \dpy{gan3}}

\begin{EntryWithPhonetic}{橄}{gan3}{15}{⽊}
  \definition*{s.}{Sobrenome: Gan}
\end{EntryWithPhonetic}

\begin{EntryWithPhonetic}{橄榄球}{gan3lan3qiu2}{15,13,11}{⽊,⽊,⽟}
  \definition{s.}{futebol jogado com bola oval (rúgbi, futebol americano, regras australianas, etc.)}
\end{EntryWithPhonetic}

%%%%%%%%%% 干 %%%%%%%%%%
\subsection*{干}\addcontentsline{loh}{figure}{干 \dpy{gan4}}

\begin{EntryWithPhonetic}{干}{gan4}{3}{⼲}[HSK 1][Kangxi 51]
  \definition{adj.}{capaz; competente; habilidoso}
  \definition{s.}{tronco; parte principal; corpo principal ou parte importante de algo | habilidade; capacidade; competência}
  \definition{v.}{fazer; trabalhar; cuidar; fazer coisas | ocupar o cargo de; estar envolvido em; assumir, exercer | lutar; golpear; esforçar-se}
  \seeref{gan1}
\end{EntryWithPhonetic}

\begin{EntryWithPhonetic}{干部}{gan4bu4}{3,10}{⼲,⾢}[HSK 7-9]
  \definition[名,个,位]{s.}{quadro; funcionário público; funcionário do governo; servidores públicos (excluindo soldados e pessoal geral) em órgãos estatais, militares e organizações populares; refere"-se àqueles que desempenham determinado trabalho de liderança ou gestão | líder; pessoal que ocupa determinados cargos de liderança}
\end{EntryWithPhonetic}

\begin{EntryWithPhonetic}{干活}{gan4/huo2}{3,9}{⼲,⽔}
  \definition{v.+compl.}{trabalhar | trabalhar em um emprego}
\end{EntryWithPhonetic}

\begin{EntryWithPhonetic}{干活儿}{gan4huo2r5}{3,9,2}{⼲,⽔,⼉}[HSK 2]
  \definition{v.}{trabalhar; gastar energia física ou mental para fazer algo, especialmente trabalho árduo ou esforçado.}
\end{EntryWithPhonetic}

\begin{EntryWithPhonetic}{干吗}{gan4ma2}{3,6}{⼲,⼝}[HSK 3]
  \definition{pron.}{por que?}
  \definition{v.}{o que fazer?}
\end{EntryWithPhonetic}

\begin{EntryWithPhonetic}{干什么}{gan4shen2me5}{3,4,3}{⼲,⼈,⼃}[HSK 1]
  \definition{adv.}{o que fazer; o que ele está fazendo?; o que você está fazendo?; perguntar a razão ou o objetivo}
\end{EntryWithPhonetic}

\begin{EntryWithPhonetic}{干事}{gan4shi5}{3,8}{⼲,⼅}[HSK 7-9]
  \definition{s.}{secretário (ou funcionário administrativo) encarregado de algo | administrador | secretária executiva}
\end{EntryWithPhonetic}

%%%%%%%%%% 刚 %%%%%%%%%%
\subsection*{刚}\addcontentsline{loh}{figure}{刚 \dpy{gang1}}

\begin{EntryWithPhonetic}{刚}{gang1}{6}{⼑}[HSK 2]
  \definition*{s.}{Sobrenome: Gang}
  \definition{adj.}{duro; firme; rígido; forte; (personalidade, atitude) forte; (vontade) determinada}
  \definition{adv.}{apenas; exatamente; justamente | apenas; apenas por pouco; significa atingir um certo nível com dificuldade | apenas; há pouco tempo; indica que a ação ou situação ocorreu há pouco tempo | assim que; somente neste momento; aconteceu que; use a palavra 就 para indicar que duas coisas estão intimamente relacionadas}
  \seealsoref{就}{jiu4}
\end{EntryWithPhonetic}

\begin{EntryWithPhonetic}{刚才}{gang1cai2}{6,3}{⼑,⼿}[HSK 2]
  \definition{s.}{agora mesmo; há pouco; há pouco tempo; referindo"-se ao período recente que acabou de passar}
\end{EntryWithPhonetic}

\begin{EntryWithPhonetic}{刚刚}{gang1gang1}{6,6}{⼑,⼑}[HSK 2]
  \definition{adv.}{apenas; somente; exatamente; refere"-se a algo que é adequado em termos de grau, quantidade, tempo, etc., nem mais nem menos, nem cedo nem tarde, atingindo um estado satisfatório ou que atende exatamente às necessidades | agora mesmo; há pouco; há um momento atrás; referindo"-se a um período de tempo muito curto no passado}
\end{EntryWithPhonetic}

\begin{EntryWithPhonetic}{刚好}{gang1hao3}{6,6}{⼑,⼥}[HSK 6]
  \definition{adj.}{apropriado; na medida certa}
  \definition{adv.}{apenas; acontece que; por acaso}
\end{EntryWithPhonetic}

\begin{EntryWithPhonetic}{刚毅}{gang1yi4}{6,15}{⼑,⽎}[HSK 7-9]
  \definition{adj.}{resoluto e firme | resoluto | robusto | firme}
\end{EntryWithPhonetic}

%%%%%%%%%% 扛 %%%%%%%%%%
\subsection*{扛}\addcontentsline{loh}{figure}{扛 \dpy{gang1}}

\begin{EntryWithPhonetic}{扛}{gang1}{6}{⼿}
  \definition{v.}{levantar com as duas mãos | carregar alguma coisa juntos (duas ou mais pessoas)}
  \seeref{kang2}
\end{EntryWithPhonetic}

%%%%%%%%%% 杠 %%%%%%%%%%
\subsection*{杠}\addcontentsline{loh}{figure}{杠 \dpy{gang1}}

\begin{EntryWithPhonetic}{杠}{gang1}{7}{⽊}
  \definition{s.}{pequena ponte | mastro de bandeira}
  \seeref{gang4}
\end{EntryWithPhonetic}

%%%%%%%%%% 纲 %%%%%%%%%%
\subsection*{纲}\addcontentsline{loh}{figure}{纲 \dpy{gang1}}

\begin{EntryWithPhonetic}{纲}{gang1}{7}{⽷}
  \definition[条,个]{s.}{a corda principal para içar uma rede, frequentemente usada como metáfora | elo-chave; princípio orientador; esboço; programa | Biologia: classe | Arcaico: transporte de mercadorias em comboio (na China feudal)}
  \definition{v.}{amarrar; comprender | corrigir; esclarecer o contorno do problema}
\end{EntryWithPhonetic}

\begin{EntryWithPhonetic}{纲领}{gang1ling3}{7,11}{⽷,⾴}[HSK 7-9]
  \definition{s.}{programa | princípios de orientação; normas diretivas; diretrizes | credo | esboço}
\end{EntryWithPhonetic}

\begin{EntryWithPhonetic}{纲要}{gang1yao4}{7,9}{⽷,⾑}[HSK 7-9]
  \definition{s.}{esboço; contorno | (usualmente em títulos de livros) essenciais; compêndio | fundamentos; princípios básicos}
\end{EntryWithPhonetic}

%%%%%%%%%% 缸 %%%%%%%%%%
\subsection*{缸}\addcontentsline{loh}{figure}{缸 \dpy{gang1}}

\begin{EntryWithPhonetic}{缸}{gang1}{9}{⽸}[HSK 7-9]
  \definition[口,个]{s.}{jarra; pote de barro; recipientes feitos de barro, porcelana, vidro, etc. geralmente têm uma boca grande e um fundo pequeno | composto de areia, argila, etc. para fazer louça de barro | vaso em forma de jarro; objetos em forma de potes}
\end{EntryWithPhonetic}

%%%%%%%%%% 钢 %%%%%%%%%%
\subsection*{钢}\addcontentsline{loh}{figure}{钢 \dpy{gang1}}

\begin{EntryWithPhonetic}{钢}{gang1}{9}{⾦}[HSK 7-9]
  \definition[吨,块,根]{s.}{aço; liga de ferro e carbono}
\end{EntryWithPhonetic}

\begin{EntryWithPhonetic}{钢笔}{gang1bi3}{9,10}{⾦,⽵}[HSK 5]
  \definition[支,杆]{s.}{caneta-tinteiro; canetas com ponta metálica}
\end{EntryWithPhonetic}

\begin{EntryWithPhonetic}{钢琴}{gang1qin2}{9,12}{⾦,⽟}[HSK 5]
  \definition[架,台]{s.}{piano}
\end{EntryWithPhonetic}

\begin{EntryWithPhonetic}{钢丝}{gang1si1}{9,5}{⾦,⼀}
  \definition{s.}{cabo de aço | corda bamba}
\end{EntryWithPhonetic}

%%%%%%%%%% 岗 %%%%%%%%%%
\subsection*{岗}\addcontentsline{loh}{figure}{岗 \dpy{gang3}}

\begin{EntryWithPhonetic}{岗}{gang3}{7}{⼭}
  \definition{s.}{outeiro; monte | crista; vergão (no rosto, pele, etc.) | sentinela; posto | trabalho | batida policial}
\end{EntryWithPhonetic}

\begin{EntryWithPhonetic}{岗位}{gang3wei4}{7,7}{⼭,⼈}[HSK 6]
  \definition[个,类]{s.}{posto; estação; originalmente se refere ao local guardado pelos militares e pela polícia, agora se refere a uma posição geral}
\end{EntryWithPhonetic}

%%%%%%%%%% 港 %%%%%%%%%%
\subsection*{港}\addcontentsline{loh}{figure}{港 \dpy{gang3}}

\begin{EntryWithPhonetic}{港}{gang3}{12}{⽔}[HSK 7-9]
  \definition*{s.}{Hong Kong, abreviação de 香港 | Sobrenome: Gang}
  \definition{s.}{porto; ancoradouro}
  \seealsoref{香港}{xiang1gang3}
\end{EntryWithPhonetic}

\begin{EntryWithPhonetic}{港口}{gang3kou3}{12,3}{⽔,⼝}[HSK 6]
  \definition[个,座]{s.}{porto; locais com certas condições naturais e instalações portuárias para atracação de navios, embarque e desembarque de passageiros e coleta e distribuição de cargas}
\end{EntryWithPhonetic}

%%%%%%%%%% 杠 %%%%%%%%%%
\subsection*{杠}\addcontentsline{loh}{figure}{杠 \dpy{gang4}}

\begin{EntryWithPhonetic}{杠}{gang4}{7}{⽊}
  \definition{s.}{vara grossa | (esportes) barra | peça sobressalente em forma de haste; peça sobressalente em forma de haste usada para máquinas-ferramentas | varas robustas usadas para carregar um caixão | (em um texto) linha grossa desenhada ao lado ou abaixo das palavras como uma marca | (coloquial) padrão; critério}
  \definition{v.}{marcar com uma linha grossa | afiar (faca, navalha, etc.)}
  \seeref{gang1}
\end{EntryWithPhonetic}

\begin{EntryWithPhonetic}{杠铃}{gang4ling2}{7,10}{⽊,⾦}[HSK 7-9]
  \definition{s.}{barra; levantamento de peso; equipamento de levantamento de peso, com placas metálicas em forma de disco instaladas em ambas as extremidades da barra horizontal}
\end{EntryWithPhonetic}

%%%%%%%%%% 高 %%%%%%%%%%
\subsection*{高}\addcontentsline{loh}{figure}{高 \dpy{gao1}}

\begin{EntryWithPhonetic}{高}{gao1}{10}{⾼}[HSK 1][Kangxi 189]
  \definition*{s.}{Sobrenome: Gao}
  \definition{adj.}{alto; elevado; grande distância de baixo para cima; longe do chão | barulhento | sofisticado; caro; de preço elevado; acima do valor real ou do preço de mercado | acima da média; de alto nível ou grau; acima do padrão geral ou da média; de nível superior}
  \definition{s.}{altura; altitude}
\end{EntryWithPhonetic}

\begin{EntryWithPhonetic}{高昂}{gao1'ang2}{10,8}{⾼,⽇}[HSK 7-9]
  \definition{adj.}{alto; eufórico; exaltado | caro; exorbitante}
  \definition{v.}{manter erguida (a cabeça, etc.)}
\end{EntryWithPhonetic}

\begin{EntryWithPhonetic}{高傲}{gao1'ao4}{10,12}{⾼,⼈}[HSK 7-9]
  \definition{adj.}{arrogante; altivo | orgulhoso; respeitoso; nobre}
\end{EntryWithPhonetic}

\begin{EntryWithPhonetic}{高层}{gao1ceng2}{10,7}{⾼,⼫}[HSK 6]
  \definition{adj.}{(de um edifício) arranha-céu | (de posição oficial) alto nível}
  \definition{s.}{nível superior; piso, camada, etc. | arranha-céus; um prédio de apartamentos alto}
\end{EntryWithPhonetic}

\begin{EntryWithPhonetic}{高超}{gao1chao1}{10,12}{⾼,⾛}[HSK 7-9]
  \definition{adj.}{soberbo; excelente; descreve um nível muito alto, excedendo a maioria dos níveis}
\end{EntryWithPhonetic}

\begin{EntryWithPhonetic}{高潮}{gao1chao2}{10,15}{⾼,⽔}[HSK 4]
  \definition[个,场]{s.}{maré alta; o nível mais alto da maré em um ciclo de maré | pico; aumento; maré alta; uma metáfora para o estágio mais próspero de desenvolvimento das coisas (diferente de 低潮) | (ficção, drama e filmes) clímax}
  \seealsoref{低潮}{di1chao2}
\end{EntryWithPhonetic}

\begin{EntryWithPhonetic}{高大}{gao1da4}{10,3}{⾼,⼤}[HSK 5]
  \definition{adj.}{alto e grande; alto | elevado; sublime; nobre}
\end{EntryWithPhonetic}

\begin{EntryWithPhonetic}{高档}{gao1dang4}{10,10}{⾼,⽊}[HSK 6]
  \definition{adj.}{grau superior; alta qualidade; alo grau; qualidade superior; boa qualidade, preço alto (produto)}
\end{EntryWithPhonetic}

\begin{EntryWithPhonetic}{高等}{gao1deng3}{10,12}{⾼,⽵}[HSK 6]
  \definition{adj.}{superior; avançado | alto nível}
  \antonymref{低等}{di1 deng3}
\end{EntryWithPhonetic}

\begin{EntryWithPhonetic}{高低}{gao1di1}{10,7}{⾼,⼈}[HSK 7-9]
  \definition{adv.}{apenas; simplesmente; em qualquer caso; de qualquer forma; em qualquer conta; indica que não importa o que | finalmente; no final, depois de tudo}
  \definition{s.}{inclinação; nível; altura | diferença de grau; superioridade ou inferioridade relativa | discrição; senso de propriedade | (falar ou fazer coisas) medida; profundidade, leveza e peso}
\end{EntryWithPhonetic}

\begin{EntryWithPhonetic}{高调}{gao1diao4}{10,10}{⾼,⾔}[HSK 7-9]
  \definition{adj.}{alto perfil; retaliação; significa agir de forma muito ostensiva e chamativa, deixando claro para todos; também pode significar opor-se deliberadamente aos outros, chegando até a provocar uma briga}
  \definition{s.}{tom elevado; palavras de alto som; sons mais agudos do que o normal ao cantar ou falar}
\end{EntryWithPhonetic}

\begin{EntryWithPhonetic}{高度}{gao1du4}{10,9}{⾼,⼴}[HSK 5]
  \definition{adj.}{alto; elevado; avançado; alto grau | alta concentração; intenso}
  \definition[个]{s.}{altura; altitude; elevação; distância de baixo para cima; o grau e o nível em que as coisas se desenvolveram}
\end{EntryWithPhonetic}

\begin{EntryWithPhonetic}{高额}{gao1'e2}{10,15}{⾼,⾴}[HSK 7-9]
  \definition{s.}{quantidade enorme; cota grande | grande quantidade}
\end{EntryWithPhonetic}

\begin{EntryWithPhonetic}{高尔夫}{gao1'er3fu1}{10,5,4}{⾼,⼩,⼤}
  \definition{s.}{(empréstimo linguístico) \emph{golf}}
\end{EntryWithPhonetic}

\begin{EntryWithPhonetic}{高尔夫球}{gao1'er3fu1qiu2}{10,5,4,11}{⾼,⼩,⼤,⽟}[HSK 7-9]
  \definition[个,只,场,些]{s.}{golfe | bola de golfe}
\end{EntryWithPhonetic}

\begin{EntryWithPhonetic}{高峰}{gao1feng1}{10,10}{⾼,⼭}[HSK 6]
  \definition[个,座]{s.}{cume; pináculo; pico da montanha | pico (de atividade, qualidade ou realização); uma metáfora para o ponto mais alto no desenvolvimento das coisas | cúpula; principais líderes; uma metáfora para o mais alto nível de liderança}
\end{EntryWithPhonetic}

\begin{EntryWithPhonetic}{高峰期}{gao1feng1qi1}{10,10,12}{⾼,⼭,⽉}[HSK 7-9]
  \definition[个]{s.}{período de pico; (de tráfego) horas de pico; o período em que ocorre com mais frequência ou se desenvolve mais próspero}
\end{EntryWithPhonetic}

\begin{EntryWithPhonetic}{高跟鞋}{gao1gen1xie2}{10,13,15}{⾼,⾜,⾰}[HSK 5]
  \definition[双]{s.}{salto alto; sapatos de salto alto; sapato feminino com salto mais alto e mais distante do chão}
\end{EntryWithPhonetic}

\begin{EntryWithPhonetic}{高贵}{gao1gui4}{10,9}{⾼,⾙}[HSK 7-9]
  \definition{adj.}{(caráter pessoal) nobre; honrado; moralmente elevado; magnânimo | grandeza; extremamente valioso | elitista; altamente privilegiado; refere"-se àqueles com \emph{status} elevado e vida superior}
\end{EntryWithPhonetic}

\begin{EntryWithPhonetic}{高级}{gao1ji2}{10,6}{⾼,⽷}[HSK 2]
  \definition{adj.}{sênior; de alto escalão; de alto nível; elevado; excelente; superior; estágio avançado | e alta qualidade; de primeira qualidade; avançado}
\end{EntryWithPhonetic}

\begin{EntryWithPhonetic}{高技术}{gao1ji4shu4}{10,7,5}{⾼,⼿,⽊}
  \definition{s.}{alta tecnologia; \emph{hight tech}}
  \seealsoref{高科技}{gao1ke1ji4}
\end{EntryWithPhonetic}

\begin{EntryWithPhonetic}{高价}{gao1jia4}{10,6}{⾼,⼈}[HSK 4]
  \definition{s.}{preço alto; bilhete caro; custo elevado; dispendioso}
\end{EntryWithPhonetic}

\begin{EntryWithPhonetic}{高考}{gao1kao3}{10,6}{⾼,⽼}[HSK 6]
  \definition[次,回,场]{s.}{vestibular; exame de admissão em instituições de ensino superior}
\end{EntryWithPhonetic}

\begin{EntryWithPhonetic}{高科技}{gao1ke1ji4}{10,9,7}{⾼,⽲,⼿}[HSK 6]
  \definition[种,类]{s.}{alta tecnologia; \emph{high tech}}
  \seealsoref{高技术}{gao1ji4shu4}
\end{EntryWithPhonetic}

\begin{EntryWithPhonetic}{高空}{gao1kong1}{10,8}{⾼,⽳}[HSK 7-9]
  \definition{s.}{alta altitude; ar superior}
  \antonymref{低空}{di1kong1}
\end{EntryWithPhonetic}

\begin{EntryWithPhonetic}{高龄}{gao1ling2}{10,13}{⾼,⿒}[HSK 7-9]
  \definition{adj.}{mais velho que o normal | avançado em anos}
  \definition{s.}{idade avançada; idade venerável}
\end{EntryWithPhonetic}

\begin{EntryWithPhonetic}{高楼}{gao1lou2}{10,13}{⾼,⽊}
  \definition[座]{s.}{edifício alto | edifício de muitos andares | arranha-céu}
\end{EntryWithPhonetic}

\begin{EntryWithPhonetic}{高明}{gao1ming2}{10,8}{⾼,⽇}[HSK 7-9]
  \definition{adj.}{sábio; brilhante; (percepção, habilidades) excelente}
  \definition{s.}{pessoa sábia; pessoa habilidosa}
\end{EntryWithPhonetic}

\begin{EntryWithPhonetic}{高山}{gao1shan1}{10,3}{⾼,⼭}[HSK 7-9]
  \definition[座]{s.}{alta montanha; alpes}
\end{EntryWithPhonetic}

\begin{EntryWithPhonetic}{高尚}{gao1shang4}{10,8}{⾼,⼩}[HSK 4]
  \definition{adj.}{nobre; elevado; descreve um alto padrão moral e uma boa qualidade de pensamento | significativo e não de mau gosto}
\end{EntryWithPhonetic}

\begin{EntryWithPhonetic}{高手}{gao1shou3}{10,4}{⾼,⼿}[HSK 6]
  \definition[位,个,名,些,群]{s.}{ás; mestre; especialista; \emph{expert}; uma pessoa com habilidades excepcionais}
\end{EntryWithPhonetic}

\begin{EntryWithPhonetic}{高速}{gao1su4}{10,10}{⾼,⾡}[HSK 3]
  \definition{adj.}{alta velocidade; veloz; rápido}
  \definition[条]{s.}{autoestrada; via expressa; rodovia}
\end{EntryWithPhonetic}

\begin{EntryWithPhonetic}{高速公路}{gao1su4gong1lu4}{10,10,4,13}{⾼,⾡,⼋,⾜}[HSK 3]
  \definition[条]{s.}{via expressa; rodovia; autoestrada; as rodovias destinadas exclusivamente ao tráfego de veículos em alta velocidade são retas e, quando cruzam outras vias, utilizam cruzamentos em nível}
\end{EntryWithPhonetic}

\begin{EntryWithPhonetic}{高铁}{gao1tie3}{10,10}{⾼,⾦}[HSK 4]
  \definition{s.}{trem de alta velocidade; trem bala}
\end{EntryWithPhonetic}

\begin{EntryWithPhonetic}{高温}{gao1wen1}{10,12}{⾼,⽔}[HSK 5]
  \definition{s.}{alta temperatura; temperatura elevada; hipertermia; megatemperatura; inferno}
  \antonymref{低温}{di1wen1}
\end{EntryWithPhonetic}

\begin{EntryWithPhonetic}{高效}{gao1xiao4}{10,10}{⾼,⽁}[HSK 7-9]
  \definition{adj.}{altamente eficiente | eficiente | altamente eficaz}
\end{EntryWithPhonetic}

\begin{EntryWithPhonetic}{高新技术}{gao1xin1-ji4shu4}{10,13,7,5}{⾼,⽄,⼿,⽊}[HSK 7-9]
  \definition[项,门,套]{s.}{nova e alta tecnologia; \emph{high‐tech}}
\end{EntryWithPhonetic}

\begin{EntryWithPhonetic}{高兴}{gao1xing4}{10,6}{⾼,⼋}[HSK 1]
  \definition{adj.}{contente; feliz; exultante; alegre; satisfeito; animado}
  \definition{v.}{estar contente; estar feliz; estar animado; estar de bom humor; fazer algo com alegria; gostar}
\end{EntryWithPhonetic}

\begin{EntryWithPhonetic}{高血压}{gao1xue4ya1}{10,6,6}{⾼,⾎,⼚}[HSK 7-9]
  \definition{adj.}{hipertenso}
  \definition[点儿]{s.}{hipertenção; pressão alta}
\end{EntryWithPhonetic}

\begin{EntryWithPhonetic}{高压}{gao1ya1}{10,6}{⾼,⼚}[HSK 7-9]
  \definition{s.}{Física, Meteorologia: alta pressão| Eletricidade: alta tensão; alta voltagem | Política: mão de ferro; arrogância | Medicina: pressão sistólica; pressão máxima | perseguição cruel; opressão extrema}
  \antonymref{低压}{di1ya1}
\end{EntryWithPhonetic}

\begin{EntryWithPhonetic}{高雅}{gao1ya3}{10,12}{⾼,⾫}[HSK 7-9]
  \definition{adj.}{delicado | elegante}
  \definition{s.}{delicadeza; decoro | elegância}
\end{EntryWithPhonetic}

\begin{EntryWithPhonetic}{高于}{gao1yu2}{10,3}{⾼,⼆}[HSK 5]
  \definition{v.}{ser mais alto do que; sobrepujar}
\end{EntryWithPhonetic}

\begin{EntryWithPhonetic}{高原}{gao1yuan2}{10,10}{⾼,⼚}[HSK 5]
  \definition[片]{s.}{platô; terras altas; planalto | planalto continental}
\end{EntryWithPhonetic}

\begin{EntryWithPhonetic}{高涨}{gao1zhang3}{10,10}{⾼,⽔}[HSK 7-9]
  \definition{v.}{ascender; subir alto; avançar; (preços, sentimento, etc.) subir rapidamente}
  \antonymref{低落}{di1luo4}
\end{EntryWithPhonetic}

\begin{EntryWithPhonetic}{高中}{gao1zhong1}{10,4}{⾼,⼁}[HSK 2]
  \definition[所,个]{s.}{ensino médio; escola secundária de ensino médio}
\end{EntryWithPhonetic}

%%%%%%%%%% 糕 %%%%%%%%%%
\subsection*{糕}\addcontentsline{loh}{figure}{糕 \dpy{gao1}}

\begin{EntryWithPhonetic}{糕}{gao1}{16}{⽶}
  \definition{s.}{bolo; alimentos feitos de farinha de arroz, farinha de trigo, etc.}
\end{EntryWithPhonetic}

\begin{EntryWithPhonetic}{糕点}{gao1dian3}{16,9}{⽶,⽕}
  \definition{s.}{bolos | pastéis}
\end{EntryWithPhonetic}

\begin{EntryWithPhonetic}{糕点店}{gao1dian3 dian4}{16,9,8}{⽶,⽕,⼴}
  \definition{s.}{confeitaria}
\end{EntryWithPhonetic}

\begin{EntryWithPhonetic}{糕点师}{gao1dian3 shi1}{16,9,6}{⽶,⽕,⼱}
  \definition{s.}{confeiteiro}
\end{EntryWithPhonetic}

%%%%%%%%%% 搞 %%%%%%%%%%
\subsection*{搞}\addcontentsline{loh}{figure}{搞 \dpy{gao3}}

\begin{EntryWithPhonetic}{搞}{gao3}{13}{⼿}[HSK 5]
  \definition{v.}{fazer; realizar; estar envolvido em; engajar-se em um estudo, fazer algo em relação a, etc. | fazer; produzir; gerar; trabalhar | iniciar; estabelecer; organizar; configurar | consertar (mudar) alguém; fazer alguém sofrer | obter; assegurar; agarrar |  (seguido de um complemento) fazer com que se torne; produzir um determinado efeito ou resultado}
\end{EntryWithPhonetic}

\begin{EntryWithPhonetic}{搞错}{gao3cuo4}{13,13}{⼿,⾦}
  \definition{v.}{cometer um erro}
\end{EntryWithPhonetic}

\begin{EntryWithPhonetic}{搞定}{gao3ding4}{13,8}{⼿,⼧}
  \definition{v.}{consertar | resolver}
\end{EntryWithPhonetic}

\begin{EntryWithPhonetic}{搞鬼}{gao3/gui3}{13,9}{⼿,⿁}[HSK 7-9]
  \definition{v.+compl.}{Coloquial: pregar peças; planejar em segredo; fazer alguma travessura}
\end{EntryWithPhonetic}

\begin{EntryWithPhonetic}{搞好}{gao3 hao3}{13,6}{⼿,⼥}[HSK 5]
  \definition{v.}{fazer um bom trabalho; fazer bem; suar; tornar submisso, tornar útil, por meio de solicitações e presentes amigáveis; amolecer}
\end{EntryWithPhonetic}

\begin{EntryWithPhonetic}{搞混}{gao3hun4}{13,11}{⼿,⽔}
  \definition{v.}{confundir; embaralhar}
\end{EntryWithPhonetic}

\begin{EntryWithPhonetic}{搞乱}{gao3luan4}{13,7}{⼿,⼄}
  \definition{v.}{estragar | confundir | bagunçar}
\end{EntryWithPhonetic}

\begin{EntryWithPhonetic}{搞钱}{gao3qian2}{13,10}{⼿,⾦}
  \definition{v.}{fazer dinheiro | acumular dinheiro}
\end{EntryWithPhonetic}

\begin{EntryWithPhonetic}{搞通}{gao3tong1}{13,10}{⼿,⾡}
  \definition{v.}{entender algo}
\end{EntryWithPhonetic}

\begin{EntryWithPhonetic}{搞笑}{gao3xiao4}{13,10}{⼿,⽵}[HSK 7-9]
  \definition{adj.}{engraçado; divertido; descreve um estado ou qualidade que é interessante, engraçado ou faz as pessoas rirem}
  \definition{v.}{fazer palhaçadas para provocar risos; fazer as pessoas rirem deliberadamente; criar piadas e fazer as pessoas rirem}
\end{EntryWithPhonetic}

%%%%%%%%%% 稿 %%%%%%%%%%
\subsection*{稿}\addcontentsline{loh}{figure}{稿 \dpy{gao3}}

\begin{EntryWithPhonetic}{稿}{gao3}{15}{⽲}
  \definition[篇]{s.}{(significado original) talo de grão; palha | rascunho; esboço; manuscrito | texto original}
\end{EntryWithPhonetic}

\begin{EntryWithPhonetic}{稿纸}{gao3zhi3}{15,7}{⽲,⽷}
  \definition{s.}{rascunho | manuscrito}
\end{EntryWithPhonetic}

\begin{EntryWithPhonetic}{稿子}{gao3zi5}{15,3}{⽲,⼦}[HSK 6]
  \definition[篇,份,堆,叠]{s.}{rascunho; esboço; rascunhos de poemas, ensaios, desenhos, etc. | rascunho; manuscrito; poemas escritos | ideia; plano; plano preliminar ou conceito de trabalho}
\end{EntryWithPhonetic}

%%%%%%%%%% 告 %%%%%%%%%%
\subsection*{告}\addcontentsline{loh}{figure}{告 \dpy{gao4}}

\begin{EntryWithPhonetic}{告}{gao4}{7}{⼝}[HSK 7-9]
  \definition*{s.}{Sobrenome: Gao}
  \definition{s.}{anúncio; declaração; notificação}
  \definition{v.}{informar; contar; notificar; explicar aos outros | acusar; processar; relatar | pedir; requisitar; solicitar | dar a conhecer; mostrar | anunciar; declarar; proclamar}
\end{EntryWithPhonetic}

\begin{EntryWithPhonetic}{告别}{gao4/bie2}{7,7}{⼝,⼑}[HSK 3]
  \definition{v.+compl.}{dizer adeus a; expressar a outros, por meio de palavras, que está prestes a partir | deixar; sair; partir de | prestar as últimas homenagens ao falecido}
\end{EntryWithPhonetic}

\begin{EntryWithPhonetic}{告辞}{gao4ci2}{7,13}{⼝,⾟}[HSK 7-9]
  \definition{v.}{despedir"-se (do anfitrião) | despedir"-se}
\end{EntryWithPhonetic}

\begin{EntryWithPhonetic}{告急}{gao4ji2}{7,9}{⼝,⼼}
  \definition{v.}{estar em estado de emergência; ser crítico | relatar uma emergência; pedir ajuda de emergência | estar em uma emergência; fazer uma solicitação urgente de ajuda em uma emergência}
\end{EntryWithPhonetic}

\begin{EntryWithPhonetic}{告诫}{gao4jie4}{7,9}{⼝,⾔}[HSK 7-9]
  \definition{v.}{avisar; advertir; exortar; admoestar}
\end{EntryWithPhonetic}

\begin{EntryWithPhonetic}{告示}{gao4shi5}{7,5}{⼝,⽰}[HSK 7-9]
  \definition[张,条,篇]{s.}{nota oficial; boletim; cartaz | Obsoleto: slogan; cartaz}
  \definition{v.}{notificar; anunciar}
\end{EntryWithPhonetic}

\begin{EntryWithPhonetic}{告诉}{gao4su4}{7,7}{⼝,⾔}
  \definition{v.}{dizer; informar (dar a conhecer); dizer aos outros, para que todos saibam}
  \seeref{gao4su5}
\end{EntryWithPhonetic}

\begin{EntryWithPhonetic}{告诉}{gao4su5}{7,7}{⼝,⾔}[HSK 1]
  \definition{v.}{dizer; informar (dar a conhecer)}
  \seeref{gao4su4}
\end{EntryWithPhonetic}

\begin{EntryWithPhonetic}{告知}{gao4zhi1}{7,8}{⼝,⽮}[HSK 7-9]
  \definition{v.}{contar; informar; transmitir; familiarizar}
\end{EntryWithPhonetic}

\begin{EntryWithPhonetic}{告状}{gao4/zhuang4}{7,7}{⼝,⽝}[HSK 7-9]
  \definition{v.+compl.}{ir à justiça contra alguém; apresentar uma queixa ao tribunal e solicitar que o caso seja aberto para julgamento | apresentar uma acusação ou queixa; informar aos pais ou superiores que você ou outras pessoas foram vítimas de \emph{bullying} ou injustiças}
\end{EntryWithPhonetic}

%%%%%%%%%% 戈 %%%%%%%%%%
\subsection*{戈}\addcontentsline{loh}{figure}{戈 \dpy{ge1}}

\begin{EntryWithPhonetic}{戈}{ge1}{4}{⼽}[Kangxi 62]
  \definition*{s.}{Sobrenome: Ge}
  \definition{s.}{Arcaico: machado"-adaga (com cabo longo e lâmina horizontal) | punhal"-machado; arma antiga, lâmina cruzada, feita de bronze ou ferro, com cabo longo}
\end{EntryWithPhonetic}

\begin{EntryWithPhonetic}{戈壁}{ge1bi4}{4,16}{⼽,⼟}[HSK 7-9]
  \definition*{s.}{Deserto de Gobi}
  \definition{s.}{deserto; refere"-se a uma área desértica onde o solo é quase coberto por areia grossa e cascalho e há poucas plantas}
\end{EntryWithPhonetic}

%%%%%%%%%% 哥 %%%%%%%%%%
\subsection*{哥}\addcontentsline{loh}{figure}{哥 \dpy{ge1}}

\begin{EntryWithPhonetic}{哥}{ge1}{10}{⼝}
  \definition[个,位,名,些]{s.}{irmão mais velho | forma de se dirigir a um parente masculino mais velho de sua geração | irmão; termo amigável para se dirigir a conhecidos mais velhos do sexo masculino}
  \seealsoref{哥哥}{ge1ge5}
\end{EntryWithPhonetic}

\begin{EntryWithPhonetic}{哥哥}{ge1ge5}{10,10}{⼝,⼝}[HSK 1]
  \definition[个,位]{s.}{irmão mais velho | primo}
\end{EntryWithPhonetic}

\begin{EntryWithPhonetic}{哥们}{ge1men5}{10,5}{⼝,⼈}
  \definition{expr.}{\emph{Brothers!}}
  \definition{s.}{(coloquial) cara | irmão (forma diminuta de tratamento entre homens)}
\end{EntryWithPhonetic}

\begin{EntryWithPhonetic}{哥斯拉}{ge1si1la1}{10,12,8}{⼝,⽄,⼿}
  \definition*{s.}{Godzilla}
  \seealsoref{酷斯拉}{ku4si1la1}
\end{EntryWithPhonetic}

%%%%%%%%%% 格 %%%%%%%%%%
\subsection*{格}\addcontentsline{loh}{figure}{格 \dpy{ge1}}

\begin{EntryWithPhonetic}{格}{ge1}{10}{⽊}
  \definition{s.}{Onomatopéia: estalo (som); riso zombeteiro}
  \seeref{ge2}
\end{EntryWithPhonetic}

%%%%%%%%%% 胳 %%%%%%%%%%
\subsection*{胳}\addcontentsline{loh}{figure}{胳 \dpy{ge1}}

\begin{EntryWithPhonetic}{胳}{ge1}{10}{⾁}
  \definition{s.}{axila; sovaco}
  \seeref{ga1}
  \seeref{ge2}
\end{EntryWithPhonetic}

\begin{EntryWithPhonetic}{胳膊}{ge1bo5}{10,14}{⾁,⾁}[HSK 7-9]
  \definition[条,双,只]{s.}{braço; a área abaixo do ombro e acima do pulso}
\end{EntryWithPhonetic}

%%%%%%%%%% 鸽 %%%%%%%%%%
\subsection*{鸽}\addcontentsline{loh}{figure}{鸽 \dpy{ge1}}

\begin{EntryWithPhonetic}{鸽}{ge1}{11}{⿃}
  \definition[只]{s.}{pombo}[和平鸽。===Pomba da Paz.]
\end{EntryWithPhonetic}

\begin{EntryWithPhonetic}{鸽子}{ge1zi5}{11,3}{⿃,⼦}[HSK 7-9]
  \definition[只,对,群]{s.}{pombo}
\end{EntryWithPhonetic}

%%%%%%%%%% 割 %%%%%%%%%%
\subsection*{割}\addcontentsline{loh}{figure}{割 \dpy{ge1}}

\begin{EntryWithPhonetic}{割}{ge1}{12}{⼑}[HSK 7-9]
  \definition{v.}{cortar; ceifar | dividir; cortar}
\end{EntryWithPhonetic}

%%%%%%%%%% 搁 %%%%%%%%%%
\subsection*{搁}\addcontentsline{loh}{figure}{搁 \dpy{ge1}}

\begin{EntryWithPhonetic}{搁}{ge1}{12}{⼿}[HSK 7-9]
  \definition{v.}{pôr; colocar | colocar à parte; deixar para trás; deixar para mais tarde| deixar de lado}
  \seeref{ge2}
\end{EntryWithPhonetic}

\begin{EntryWithPhonetic}{搁浅}{ge1/qian3}{12,8}{⼿,⽔}[HSK 7-9]
  \definition{v.+compl.}{ficar encalhado (navio); encalhar | ser retido; chegar a um impasse; metaforicamente, algo está bloqueado e não pode prosseguir}
\end{EntryWithPhonetic}

\begin{EntryWithPhonetic}{搁置}{ge1zhi4}{12,13}{⼿,⽹}[HSK 7-9]
  \definition{v.}{arquivar; deixar de lado; suspender; classificar; deitar; adiar; colocar na prateleira}
\end{EntryWithPhonetic}

%%%%%%%%%% 歌 %%%%%%%%%%
\subsection*{歌}\addcontentsline{loh}{figure}{歌 \dpy{ge1}}

\begin{EntryWithPhonetic}{歌}{ge1}{14}{⽋}[HSK 1]
  \definition[首,支,段]{s.}{canção; poesia cantável}
  \definition{v.}{cantar; entoar | louvar; exaltar; cantar louvores a}
\end{EntryWithPhonetic}

\begin{EntryWithPhonetic}{歌唱}{ge1chang4}{14,11}{⽋,⼝}[HSK 6]
  \definition{v.}{cantar | cantar em louvor de; louvor através de cânticos, recitações, etc.}
\end{EntryWithPhonetic}

\begin{EntryWithPhonetic}{歌词}{ge1ci2}{14,7}{⽋,⾔}[HSK 6]
  \definition{s.}{letra da música; libreto}
\end{EntryWithPhonetic}

\begin{EntryWithPhonetic}{歌剧}{ge1ju4}{14,10}{⽋,⼑}[HSK 7-9]
  \definition[场,出]{s.}{ópera | ópera ocidental; um drama que integra poesia, música, dança e outras artes, tendo o canto como principal característica}
\end{EntryWithPhonetic}

\begin{EntryWithPhonetic}{歌迷}{ge1mi2}{14,9}{⽋,⾡}[HSK 3]
  \definition{s.}{fã de um cantor; pessoas que gostam de ouvir música ou cantar e ficam fascinadas por isso}
\end{EntryWithPhonetic}

\begin{EntryWithPhonetic}{歌曲}{ge1qu3}{14,6}{⽋,⽈}[HSK 5]
  \definition[首,支]{s.}{música; obra para as pessoas cantarem, uma combinação de poesia e música}
\end{EntryWithPhonetic}

\begin{EntryWithPhonetic}{歌声}{ge1sheng1}{14,7}{⽋,⼠}[HSK 3]
  \definition{s.}{canto; voz cantada; som do canto}
\end{EntryWithPhonetic}

\begin{EntryWithPhonetic}{歌手}{ge1shou3}{14,4}{⽋,⼿}[HSK 3]
  \definition[个,位,名]{s.}{cantor; vocalista; pessoa com talento para cantar}
\end{EntryWithPhonetic}

\begin{EntryWithPhonetic}{歌颂}{ge1song4}{14,10}{⽋,⾴}[HSK 7-9]
  \definition{v.}{cantar louvores de; exaltar; elogiar; elogio com poesia, geralmente se refere a elogiar com palavras, etc.}
\end{EntryWithPhonetic}

\begin{EntryWithPhonetic}{歌舞}{ge1wu3}{14,14}{⽋,⾇}[HSK 7-9]
  \definition{s.}{canto e dança}
\end{EntryWithPhonetic}

\begin{EntryWithPhonetic}{歌星}{ge1xing1}{14,9}{⽋,⽇}[HSK 6]
  \definition[位,名]{s.}{cantor famoso; estrela da música}
\end{EntryWithPhonetic}

\begin{EntryWithPhonetic}{歌咏}{ge1yong3}{14,8}{⽋,⼝}[HSK 7-9]
  \definition{v.}{cantar; cantar canções}
\end{EntryWithPhonetic}

%%%%%%%%%% 阁 %%%%%%%%%%
\subsection*{阁}\addcontentsline{loh}{figure}{阁 \dpy{ge2}}

\begin{EntryWithPhonetic}{阁}{ge2}{9}{⾨}
  \definition{s.}{pavilhão (geralmente de dois andares) | gabinete (de um governo) | Obsoleto: quarto da mulher; \emph{boudoir} | prateleira}
\end{EntryWithPhonetic}

\begin{EntryWithPhonetic}{阁下}{ge2xia4}{9,3}{⾨,⼀}
  \definition{pron.}{Sua Excelência | Sua Majestade | \emph{Sire}}
\end{EntryWithPhonetic}

%%%%%%%%%% 革 %%%%%%%%%%
\subsection*{革}\addcontentsline{loh}{figure}{革 \dpy{ge2}}

\begin{EntryWithPhonetic}{革}{ge2}{9}{⾰}[Kangxi 177]
  \definition*{s.}{Sobrenome: Ge}
  \definition{s.}{couro; pele; peles de animais depiladas e processadas}
  \definition{v.}{mudar; transformar; reformar | demitir; remover do cargo; expulsar}
\end{EntryWithPhonetic}

\begin{EntryWithPhonetic}{革命}{ge2ming4}{9,8}{⾰,⼝}[HSK 7-9]
  \definition{adj.}{revolucionário}
  \definition[次,场]{s.}{revolução; a classe oprimida toma o poder pela violência, destrói o antigo sistema social decadente e estabelece um novo sistema social progressista; a revolução destrói as antigas relações de produção, estabelece novas relações de produção, libera as forças produtivas e promove o desenvolvimento social}
  \definition{v.}{participar da revolução; originalmente se referia à reforma do Mandato do Céu, ou seja, à mudança de dinastias; agora, refere"-se à classe oprimida usando a violência para tomar o poder, destruir o antigo sistema social, estabelecer um novo sistema social e promover o desenvolvimento social}
\end{EntryWithPhonetic}

\begin{EntryWithPhonetic}{革新}{ge2xin1}{9,13}{⾰,⽄}[HSK 6]
  \definition{v.}{inovar; renovar; livrar"-se do velho e criar o novo}
\end{EntryWithPhonetic}

%%%%%%%%%% 格 %%%%%%%%%%
\subsection*{格}\addcontentsline{loh}{figure}{格 \dpy{ge2}}

\begin{EntryWithPhonetic}{格}{ge2}{10}{⽊}[HSK 7-9]
  \definition*{s.}{Sobrenome: Ge}
  \definition{s.}{quadrados formados por linhas cruzadas; quadriculado; grade | divisão (horizontal ou não); treliça | padrão; forma; formato; estilo | caso; as categorias morfológicas de substantivos, pronomes e adjetivos em algumas línguas}
  \definition{v.}{resistir; dificultar; obstruir; impedir | estudar cuidadosamente; investigar | lutar; bater}
  \seeref{ge1}
\end{EntryWithPhonetic}

\begin{EntryWithPhonetic}{格格不入}{ge2ge2-bu2ru4}{10,10,4,2}{⽊,⽊,⼀,⼊}[HSK 7-9]
  \definition{expr.}{incompatível com; fora de sintonia com; estranho; fora do seu elemento; como uma estaca quadrada em um buraco redondo; desarmônico}
\end{EntryWithPhonetic}

\begin{EntryWithPhonetic}{格局}{ge2ju2}{10,7}{⽊,⼫}[HSK 7-9]
  \definition{s.}{padrão; configuração; estrutura; estilo; maneira; arranjo | a visão ou percepção de uma situação geral; a visão de uma pessoa, a altura e a profundidade da consideração do problema}
\end{EntryWithPhonetic}

\begin{EntryWithPhonetic}{格兰菜}{ge2lan2cai4}{10,5,11}{⽊,⼋,⾋}
  \definition{s.}{brócolis chinês | couve chinesa | mostarda}
  \seealsoref{芥蓝}{gai4lan2}
\end{EntryWithPhonetic}

\begin{EntryWithPhonetic}{格式}{ge2shi5}{10,6}{⽊,⼷}[HSK 7-9]
  \definition[种]{s.}{forma; estilo; \emph{layout}; padrão; formato; modo}
\end{EntryWithPhonetic}

\begin{EntryWithPhonetic}{格外}{ge2wai4}{10,5}{⽊,⼣}[HSK 4]
  \definition{adv.}{especialmente; particularmente; ainda mais; indica mais do que a média | adicionalmente; indica adicional ou extra}
\end{EntryWithPhonetic}

%%%%%%%%%% 胳 %%%%%%%%%%
\subsection*{胳}\addcontentsline{loh}{figure}{胳 \dpy{ge2}}

\begin{EntryWithPhonetic}{胳}{ge2}{10}{⾁}
  \definition{v.}{usado em 胳肢}
  \seeref{ga1}
  \seeref{ge1}
  \seealsoref{胳肢}{ge2zhi5}
\end{EntryWithPhonetic}

\begin{EntryWithPhonetic}{胳肢}{ge2zhi5}{10,8}{⾁,⾁}
  \definition{v.}{(dialeto) fazer cócegas}
\end{EntryWithPhonetic}

%%%%%%%%%% 鬲 %%%%%%%%%%
\subsection*{鬲}\addcontentsline{loh}{figure}{鬲 \dpy{ge2}}

\begin{EntryWithPhonetic}{鬲}{ge2}{10}{⿀}[Kangxi 193]
  \definition{s.}{um antigo utensílio de cozinha semelhante a um caldeirão; uma grande panela de barro | utilizado em nomes geográficos ou pessoais}
  \seeref{li4}
\end{EntryWithPhonetic}

%%%%%%%%%% 搁 %%%%%%%%%%
\subsection*{搁}\addcontentsline{loh}{figure}{搁 \dpy{ge2}}

\begin{EntryWithPhonetic}{搁}{ge2}{12}{⼿}
  \definition{v.}{suportar; resistir}
  \seeref{ge1}
\end{EntryWithPhonetic}

%%%%%%%%%% 隔 %%%%%%%%%%
\subsection*{隔}\addcontentsline{loh}{figure}{隔 \dpy{ge2}}

\begin{EntryWithPhonetic}{隔}{ge2}{12}{⾩}[HSK 4]
  \definition{adj.}{seguinte; vizinho}
  \definition{v.}{separar; cortar; dividir; particionar | estar a uma distância de, após ou em um intervalo de | ficar de pé ou deitar entre}
\end{EntryWithPhonetic}

\begin{EntryWithPhonetic}{隔壁}{ge2bi4}{12,16}{⾩,⼟}[HSK 5]
  \definition{s.}{vizinho; casas ou pessoas vizinhas | septo; distante (socialmente distante) | anteparo; partição}
\end{EntryWithPhonetic}

\begin{EntryWithPhonetic}{隔阂}{ge2he2}{12,9}{⾩,⾨}[HSK 7-9]
  \definition[层,种,点]{s.}{estranhamento; mal-entendido; há uma falta de conexão emocional e uma distância de pensamento entre eles}
\end{EntryWithPhonetic}

\begin{EntryWithPhonetic}{隔开}{ge2kai1}{12,4}{⾩,⼶}[HSK 4]
  \definition{v.}{separar; manter separado; barricar; separar completamente duas pessoas (ou coisas) ou duas partes de uma coisa que estão intimamente unidas}
\end{EntryWithPhonetic}

\begin{EntryWithPhonetic}{隔离}{ge2li2}{12,10}{⾩,⼇}[HSK 7-9]
  \definition{v.}{segregar; não permitir que as pessoas se reúnam, cortar o contato | isolar; colocar em quarentena; separar pessoas e animais com doenças infecciosas de pessoas e animais saudáveis para evitar o contato}
\end{EntryWithPhonetic}

%%%%%%%%%% 个 %%%%%%%%%%
\subsection*{个}\addcontentsline{loh}{figure}{个 \dpy{ge3}}

\begin{EntryWithPhonetic}{个}{ge3}{3}{⼈}
  \definition{pron.}{usado em 自个儿}
  \seeref{ge4}
  \seealsoref{自个儿}{zi4ge3r5}
\end{EntryWithPhonetic}

%%%%%%%%%% 盖 %%%%%%%%%%
\subsection*{盖}\addcontentsline{loh}{figure}{盖 \dpy{ge3}}

\begin{EntryWithPhonetic}{盖}{ge3}{11}{⽫}
  \definition*{s.}{Sobrenome: Ge}
  \seeref{gai4}
\end{EntryWithPhonetic}

%%%%%%%%%% 个 %%%%%%%%%%
\subsection*{个}\addcontentsline{loh}{figure}{个 \dpy{ge4}}

\begin{EntryWithPhonetic}{个}{ge4}{3}{⼈}[HSK 1]
  \definition{adj.}{individual}
  \definition{clas.}{usado antes de substantivos que não têm palavras de medida específicas | usado na frente do divisor; usado na frente do número aproximado | usado após verbos com objeto direto |  usado entre verbos e complementos}
  \definition{part.}{usado após pronomes demonstrativos | adicionado após certas palavras de tempo}
  \seeref{ge3}
\end{EntryWithPhonetic}

\begin{EntryWithPhonetic}{个案}{ge4'an4}{3,10}{⼈,⽊}[HSK 7-9]
  \definition[个,些]{s.}{caso individual (ou especial); caso; caso a caso}
\end{EntryWithPhonetic}

\begin{EntryWithPhonetic}{个别}{ge4bie2}{3,7}{⼈,⼑}[HSK 4]
  \definition{adj.}{muito poucos; excepcionais}
  \definition{adv.}{separadamente; individualmente; isoladamente}
\end{EntryWithPhonetic}

\begin{EntryWithPhonetic}{个儿}{ge4r5}{3,2}{⼈,⼉}[HSK 5]
  \definition{s.}{tamanho; altura; estatura; tamanho do corpo ou do objeto | pessoas ou coisas consideradas isoladamente; referir-se a uma pessoa ou coisa individualmente}
\end{EntryWithPhonetic}

\begin{EntryWithPhonetic}{个人}{ge4ren2}{3,2}{⼈,⼈}[HSK 3]
  \definition{pron.}{pessoal; si mesmo}
  \definition[个]{s.}{indivíduo; pessoa}
\end{EntryWithPhonetic}

\begin{EntryWithPhonetic}{个体}{ge4ti3}{3,7}{⼈,⼈}[HSK 4]
  \definition[个,位]{s.}{uma única pessoa ou organismo}
\end{EntryWithPhonetic}

\begin{EntryWithPhonetic}{个头儿}{ge4tou2er5}{3,5,2}{⼈,⼤,⼉}[HSK 7-9]
  \definition{s.}{tamanho; altura}
\end{EntryWithPhonetic}

\begin{EntryWithPhonetic}{个性}{ge4xing4}{3,8}{⼈,⼼}[HSK 3]
  \definition[种,点儿]{s.}{individualidade; personalidade; caráter individual; as características relativamente fixas de uma pessoa, formadas sob determinadas condições sociais e influências educacionais | propriedade específica; caráter específico; a propriedade ou característica especial que distingue uma coisa de outras coisas}
\end{EntryWithPhonetic}

\begin{EntryWithPhonetic}{个子}{ge4zi5}{3,3}{⼈,⼦}[HSK 2]
  \definition[个,种,些]{s.}{altura; estatura; refere"-se ao tamanho do corpo humano e também ao tamanho do corpo dos animais}
\end{EntryWithPhonetic}

%%%%%%%%%% 各 %%%%%%%%%%
\subsection*{各}\addcontentsline{loh}{figure}{各 \dpy{ge4}}

\begin{EntryWithPhonetic}{各}{ge4}{6}{⼝}[HSK 3]
  \definition{adv.}{de várias maneiras; de diversas formas; respectivamente; indica que algo é feito separadamente ou que possui uma determinada característica separadamente}
  \definition{pron.}{todo; todos; cada; refere"-se a todos os indivíduos dentro de um determinado intervalo, equivalente a 每个}
  \seealsoref{每个}{mei3ge4}
\end{EntryWithPhonetic}

\begin{EntryWithPhonetic}{各奔前程}{ge4ben4qian2cheng2}{6,8,9,12}{⼝,⼤,⼑,⽲}[HSK 7-9]
  \definition{expr.}{``Cada um segue seu próprio caminho.''; cada pessoa tem sua própria vida para viver; cada um deles desenvolve sua própria carreira ambiciosa; cada um segue seu próprio curso}
\end{EntryWithPhonetic}

\begin{EntryWithPhonetic}{各地}{ge4di4}{6,6}{⼝,⼟}[HSK 3]
  \definition{s.}{em todos os lugares; em vários locais}
\end{EntryWithPhonetic}

\begin{EntryWithPhonetic}{各个}{ge4ge4}{6,3}{⼝,⼈}[HSK 4]
  \definition{adv./pron.}{cada | um a um; um após o outro}
\end{EntryWithPhonetic}

\begin{EntryWithPhonetic}{各式各样}{ge4shi4-ge4yang4}{6,6,6,10}{⼝,⼷,⼝,⽊}[HSK 7-9]
  \definition{expr.}{todo tipo de\dots; todos os tipos de\dots; todos os tipos de; de várias maneiras; de todas as descrições; uma variedade de; uma variedade de variedades com cores diferentes}
\end{EntryWithPhonetic}

\begin{EntryWithPhonetic}{各位}{ge4wei4}{6,7}{⼝,⼈}[HSK 3]
  \definition{pron.}{todos; toda a gente; todo mundo | cada um}
\end{EntryWithPhonetic}

\begin{EntryWithPhonetic}{各种}{ge4zhong3}{6,9}{⼝,⽲}[HSK 3]
  \definition{adv.}{todos os tipos; vários tipos}
\end{EntryWithPhonetic}

\begin{EntryWithPhonetic}{各自}{ge4zi4}{6,6}{⼝,⾃}[HSK 3]
  \definition{pron.}{por si mesmo; por conta própria; cada um por si | cada um; indica cada uma das partes envolvidas}
\end{EntryWithPhonetic}

%%%%%%%%%% 给 %%%%%%%%%%
\subsection*{给}\addcontentsline{loh}{figure}{给 \dpy{gei3}}

\begin{EntryWithPhonetic}{给}{gei3}{9}{⽷}[HSK 1]
  \definition{prep.}{por; expressa significado passivo; tem o mesmo significado que 被, 叫; pode ser seguido pelo agente da ação; o agente da ação também pode não aparecer na frase | para; a; seguido por quem se beneficia da ação; igual a 为 | em direção a; seguido pelo destinatário da ação; o mesmo que 向 | indica transmissão}
  \definition{v.}{dar; conceder; fazer com que a outra parte obtenha algo | passar; pagar; indicar que a outra pessoa faça algo | deixar; permitir que alguém faça algo; autorizar alguém a fazer algo}
  \definition{v.aux.}{usado antes de verbos predicativos que expressam passividade, disposição, etc., para reforçar o tom}
  \seeref{ji3}
  \seealsoref{被}{bei4}
  \seealsoref{叫}{jiao4}
  \seealsoref{为}{wei4}
  \seealsoref{向}{xiang4}
\end{EntryWithPhonetic}

\begin{EntryWithPhonetic}{给……打电话}{gei3 da3 dian4hua4}{9,5,5,8}{⽷,⼿,⽥,⾔}
  \definition{expr.}{dar um telefonema para alguém}
  \seealsoref{打电话}{da3 dian4hua4}
\end{EntryWithPhonetic}

\begin{EntryWithPhonetic}{给……定向}{gei3 ding4xiang4}{9,8,6}{⽷,⼧,⼝}
  \definition{v.}{dar orientação para algo; orientar algo}
\end{EntryWithPhonetic}

%%%%%%%%%% 根 %%%%%%%%%%
\subsection*{根}\addcontentsline{loh}{figure}{根 \dpy{gen1}}

\begin{EntryWithPhonetic}{根}{gen1}{10}{⽊}[HSK 4]
  \definition*{s.}{Sobrenome: Gen}
  \definition{adv.}{completamente; minuciosamente; radicalmente}
  \definition{clas.}{usado para objetos finos, alongados}
  \definition{s.}{raiz (de uma planta) | descendentes; posteridade; analogia com as gerações futuras | raiz (abreviação de raiz quadrada) | radical (química, refere"-se a radicais carregados) | base; pé; raiz; parte inferior, base ou parte de um objeto que está presa a outra coisa | a parte de baixo das coisas; fonte; a origem  das coisas | base; fundamento}
\end{EntryWithPhonetic}

\begin{EntryWithPhonetic}{根本}{gen1ben3}{10,5}{⽊,⽊}[HSK 3]
  \definition{adj.}{básico; essencial; fundamental; importante; decisivo}
  \definition{adv.}{nunca; simplesmente; de forma alguma | radicalmente; completamente; nunca (mais usado em negativas)}
  \definition[个]{s.}{base; raiz; fundação; a origem, a base ou a parte mais importante das coisas}
\end{EntryWithPhonetic}

\begin{EntryWithPhonetic}{根基}{gen1ji1}{10,11}{⽊,⼟}[HSK 7-9]
  \definition{s.}{base; fundação; alicerce; parte subterrânea de um edifício | recursos; propriedade acumulada ao longo de um longo período}
\end{EntryWithPhonetic}

\begin{EntryWithPhonetic}{根据}{gen1ju4}{10,11}{⽊,⼿}[HSK 4]
  \definition{prep.}{com base em; de acordo com; à luz de}
  \definition[个]{s.}{base; fundamentos; razão; fundo; alicerce}
  \definition{v.}{basear; usar algo como premissa para uma conclusão ou como base para uma ação verbal}
\end{EntryWithPhonetic}

\begin{EntryWithPhonetic}{根深蒂固}{gen1shen1-di4gu4}{10,11,12,8}{⽊,⽔,⾋,⼞}[HSK 7-9]
  \definition{expr.}{arraigado; inveterado; tornar-se profundamente enraizado em; profundamente enraizado; profundamente enraizado; profundamente enraizado e firmemente plantado -- bem fundado; ter uma base firme; ter raízes profundas e uma base firme; bem estabelecido; significa que a fundação é sólida e não se abala facilmente}
\end{EntryWithPhonetic}

\begin{EntryWithPhonetic}{根源}{gen1yuan2}{10,13}{⽊,⽔}[HSK 7-9]
  \definition{s.}{fonte; origem; raiz | raízes da grama; fonte; nascente; raiz; fundo}
  \definition{v.}{originar-se; provir de}
\end{EntryWithPhonetic}

\begin{EntryWithPhonetic}{根治}{gen1zhi4}{10,8}{⽊,⽔}[HSK 7-9]
  \definition{v.}{efetuar uma cura radical; curar de uma vez por todas; colocar sob controle permanente; curar completamente (referindo"-se à erradicação de pragas ou doenças)}
\end{EntryWithPhonetic}

%%%%%%%%%% 跟 %%%%%%%%%%
\subsection*{跟}\addcontentsline{loh}{figure}{跟 \dpy{gen1}}

\begin{EntryWithPhonetic}{跟}{gen1}{13}{⾜}[HSK 1]
  \definition{conj.}{e; expressa uma relação de união; 和}
  \definition{prep.}{com; Introduzir objetos relacionados à mesma ação, equivalente a 同 | para; em direção a | de; introduzir o objeto de comparação; equivalente a 从, 由 | como; objetos que causam comparações e semelhanças}
  \definition[个]{s.}{calcanhar; parte posterior do pé ou parte posterior do sapato ou meia | base (de um objeto)}
  \definition{v.}{seguir; acompanhar; seguir imediatamente na mesma direção | (uma mulher) estar casada com; casar"-se com alguém}
  \seealsoref{从}{cong2}
  \seealsoref{和}{he2}
  \seealsoref{同}{tong2}
  \seealsoref{由}{you2}
\end{EntryWithPhonetic}

\begin{EntryWithPhonetic}{跟不上}{gen1 bu5 shang4}{13,4,3}{⾜,⼀,⼀}[HSK 7-9]
  \definition{v.}{não é capaz de acompanhar; não conseguir alcançar}
\end{EntryWithPhonetic}

\begin{EntryWithPhonetic}{跟前}{gen1qian2}{13,9}{⾜,⼑}[HSK 5]
  \definition{s.}{próximo; perto de; na frente de; (na ou para) a presença de alguém | o tempo imediatamente anterior a algum evento; tempo que se aproxima}
  \seeref{gen1qian5}
\end{EntryWithPhonetic}

\begin{EntryWithPhonetic}{跟前}{gen1qian5}{13,9}{⾜,⼑}
  \definition{v.}{(filhos de alguém) viver com alguém (exclusivamente com relação à presença ou ausência de crianças)}
  \seeref{gen1qian2}
\end{EntryWithPhonetic}

\begin{EntryWithPhonetic}{跟上}{gen1shang5}{13,3}{⾜,⼀}[HSK 7-9]
  \definition{v.}{acompanhar; alcançar; manter-se a par de}
\end{EntryWithPhonetic}

\begin{EntryWithPhonetic}{跟随}{gen1sui2}{13,11}{⾜,⾩}[HSK 5]
  \definition{s.}{seguidor; usado para se referir a alguém que seguiu}
  \definition{v.}{seguir; ir atrás; acompanhar}
\end{EntryWithPhonetic}

\begin{EntryWithPhonetic}{跟踪}{gen1zong1}{13,15}{⾜,⾜}[HSK 7-9]
  \definition{v.}{rastrear; alcançar; seguir; seguir atrás; seguir alguém; seguir os rastros de; seguir de perto}
\end{EntryWithPhonetic}

%%%%%%%%%% 更 %%%%%%%%%%
\subsection*{更}\addcontentsline{loh}{figure}{更 \dpy{geng1}}

\begin{EntryWithPhonetic}{更}{geng1}{7}{⽈}
  \definition*{s.}{Sobrenome: Geng}
  \definition{clas.}{um dos cinco períodos de duas horas em que a noite era anteriormente dividida; vigília; antigamente, a noite era dividida em cinco turnos, cada um com aproximadamente duas horas de duração}
  \definition{v.}{alterar; substituir | experimentar}
  \seeref{geng4}
\end{EntryWithPhonetic}

\begin{EntryWithPhonetic}{更改}{geng1gai3}{7,7}{⽈,⽁}[HSK 7-9]
  \definition{v.}{alterar; mudar}
\end{EntryWithPhonetic}

\begin{EntryWithPhonetic}{更换}{geng1huan4}{7,10}{⽈,⼿}[HSK 5]
  \definition{v.}{alterar; mudar; substituir; comutar}
\end{EntryWithPhonetic}

\begin{EntryWithPhonetic}{更新}{geng1xin1}{7,13}{⽈,⽄}[HSK 5]
  \definition{v.}{renovar; atualizar; substituir; remover o antigo e substituir pelo novo}
\end{EntryWithPhonetic}

\begin{EntryWithPhonetic}{更衣室}{geng1yi1shi4}{7,6,9}{⽈,⾐,⼧}[HSK 7-9]
  \definition{s.}{vestiário | camarim | \emph{toilet}; toucador}
\end{EntryWithPhonetic}

%%%%%%%%%% 耕 %%%%%%%%%%
\subsection*{耕}\addcontentsline{loh}{figure}{耕 \dpy{geng1}}

\begin{EntryWithPhonetic}{耕}{geng1}{10}{⽾}
  \definition{v.}{arar; cultivar | trabalhar; fazer | ganhar a vida}
\end{EntryWithPhonetic}

\begin{EntryWithPhonetic}{耕地}{geng1/di4}{10,6}{⽾,⼟}[HSK 7-9]
  \definition[块,公顷]{s.}{terra cultivada; terra para cultivo}
  \definition{v.+compl.}{lavrar; arar}
\end{EntryWithPhonetic}

%%%%%%%%%% 耿 %%%%%%%%%%
\subsection*{耿}\addcontentsline{loh}{figure}{耿 \dpy{geng3}}

\begin{EntryWithPhonetic}{耿}{geng3}{10}{⽿}
  \definition*{s.}{Sobrenome: Geng}
  \definition{adj.}{Literário: brilhante | honesto e justo; correto; íntegro | dedicado; leal}
\end{EntryWithPhonetic}

\begin{EntryWithPhonetic}{耿直}{geng3zhi2}{10,8}{⽿,⽬}[HSK 7-9]
  \definition{adj.}{íntregro; franco; correto; honesto e franco}
\end{EntryWithPhonetic}

%%%%%%%%%% 颈 %%%%%%%%%%
\subsection*{颈}\addcontentsline{loh}{figure}{颈 \dpy{geng3}}

\begin{EntryWithPhonetic}{颈}{geng3}{11}{⾴}
  \definition{s.}{nuca}
  \seeref{jing3}
\end{EntryWithPhonetic}

%%%%%%%%%% 更 %%%%%%%%%%
\subsection*{更}\addcontentsline{loh}{figure}{更 \dpy{geng4}}

\begin{EntryWithPhonetic}{更}{geng4}{7}{⽈}[HSK 2]
  \definition{adv.}{mais; ainda mais | além disso; além do mais; ainda mais}
  \seeref{geng1}
\end{EntryWithPhonetic}

\begin{EntryWithPhonetic}{更加}{geng4jia1}{7,5}{⽈,⼒}[HSK 3]
  \definition{adv.}{mais; ainda mais; em maior grau; indica um nível mais profundo ou um aumento ou diminuição quantitativa adicional}
\end{EntryWithPhonetic}

\begin{EntryWithPhonetic}{更是}{geng4shi4}{7,9}{⽈,⽇}[HSK 6]
  \definition{adv.}{ainda mais (assim)}
\end{EntryWithPhonetic}

%%%%%%%%%% 工 %%%%%%%%%%
\subsection*{工}\addcontentsline{loh}{figure}{工 \dpy{gong1}}

\begin{EntryWithPhonetic}{工}{gong1}{3}{⼯}[Kangxi 48]
  \definition*{s.}{Sobrenome: Gong}
  \definition{adj.}{fino; requintado; delicado}
  \definition{s.}{trabalhador; operário; artesão | trabalho; labor; trabalho produtivo | projeto; construção; refere"-se à engenharia | indústria; refere"-se à indústria | homem"-dia; a quantidade de trabalho que um trabalhador faz em um dia | uma nota da escala em Gongchepu (工尺谱), correspondente a 3 na notação musical numerada | engenheiro; refere"-se a engenheiros}
  \definition{v.}{ser versado em; ser bom em | trabalhar em; agora geralmente escrito como 功}
  \seealsoref{功}{gong1}
  \seealsoref{工尺谱}{gong1 che3 pu3}
\end{EntryWithPhonetic}

\begin{EntryWithPhonetic}{工厂}{gong1chang3}{3,2}{⼯,⼚}[HSK 3]
  \definition[个,家,座,间]{s.}{fábrica; moinho; planta; unidades que realizam atividades de produção industrial diretamente, geralmente incluindo diferentes oficinas}
\end{EntryWithPhonetic}

\begin{EntryWithPhonetic}{工尺谱}{gong1 che3 pu3}{3,4,14}{⼯,⼫,⾔}
  \definition*{s.}{Gongchepu, notação musical tradicional chinesa}
  \definition{s.}{notação musical tradicional chinesa que usa caracteres chineses para representar notas musicais}
\end{EntryWithPhonetic}

\begin{EntryWithPhonetic}{工程}{gong1cheng2}{3,12}{⼯,⽲}[HSK 4]
  \definition[个,项]{s.}{projeto; programa; trabalhos que utilizam equipamentos grandes e complexos, como projetos de reconstrução urbana e projetos de cestas de alimentos, etc. | engenharia; departamentos de produção e manufatura usam equipamentos grandes e complexos para realizar seu trabalho}
\end{EntryWithPhonetic}

\begin{EntryWithPhonetic}{工程师}{gong1cheng2shi1}{3,12,6}{⼯,⽲,⼱}[HSK 3]
  \definition[个,位,名,些]{s.}{engenheiro; um dos cargos técnicos é o de especialista capaz de realizar de forma independente o projeto e a execução de uma tarefa técnica específica}
\end{EntryWithPhonetic}

\begin{EntryWithPhonetic}{工地}{gong1di4}{3,6}{⼯,⼟}[HSK 7-9]
  \definition{s.}{canteiro de obras; locais onde são realizadas construções, desenvolvimentos, produções, etc.}
\end{EntryWithPhonetic}

\begin{EntryWithPhonetic}{工夫}{gong1fu1}{3,4}{⼯,⼤}
  \definition[个]{s.}{tempo | tempo livre; lazer}
  \seeref{gong1fu5}
\end{EntryWithPhonetic}

\begin{EntryWithPhonetic}{工夫}{gong1fu5}{3,4}{⼯,⼤}[HSK 3]
  \definition[个]{s.}{(um período de) tempo; o tempo ou energia gastos para realizar uma tarefa | tempo livre}
  \seeref{gong1fu1}
\end{EntryWithPhonetic}

\begin{EntryWithPhonetic}{工会}{gong1hui4}{3,6}{⼯,⼈}[HSK 7-9]
  \definition[个]{s.}{sindicato; sindicato trabalhista; organizações de massa criadas pelos trabalhadores para proteger os seus próprios interesses}
\end{EntryWithPhonetic}

\begin{EntryWithPhonetic}{工具}{gong1ju4}{3,8}{⼯,⼋}[HSK 3]
  \definition[个,件,套]{s.}{ferramenta; ferramentas utilizadas na produção | ferramenta; meio; instrumento; (metáfora) algo ou meio utilizado para atingir um determinado objetivo}
\end{EntryWithPhonetic}

\begin{EntryWithPhonetic}{工科}{gong1ke1}{3,9}{⼯,⽲}[HSK 7-9]
  \definition{s.}{curso de engenharia | engenharia como disciplina acadêmica; um termo geral para disciplinas de engenharia no ensino}
\end{EntryWithPhonetic}

\begin{EntryWithPhonetic}{工龄}{gong1ling2}{3,13}{⼯,⿒}
  \definition{s.}{tempo de serviço | senioridade}
\end{EntryWithPhonetic}

\begin{EntryWithPhonetic}{工人}{gong1ren5}{3,2}{⼯,⼈}[HSK 1]
  \definition[个,名]{s.}{trabalhador; operário; mão de obra; trabalhadores braçais que vivem do salário}
\end{EntryWithPhonetic}

\begin{EntryWithPhonetic}{工商}{gong1shang1}{3,11}{⼯,⼝}[HSK 6]
  \definition{s.}{indústria e comércio; um termo combinado para indústria e comércio}
\end{EntryWithPhonetic}

\begin{EntryWithPhonetic}{工商界}{gong1shang1jie4}{3,11,9}{⼯,⼝,⽥}[HSK 7-9]
  \definition{s.}{círculos industriais e comerciais; círculos de negócios | indústria | o mundo dos negócios}
\end{EntryWithPhonetic}

\begin{EntryWithPhonetic}{工序}{gong1xu4}{3,7}{⼯,⼴}[HSK 7-9]
  \definition[道]{s.}{processo; procedimento de trabalho; sequência do processo de produção}
\end{EntryWithPhonetic}

\begin{EntryWithPhonetic}{工业}{gong1ye4}{3,5}{⼯,⼀}[HSK 3]
  \definition{s.}{indústria; utilização de recursos naturais; fabricação de meios de produção; meios de subsistência; ou processamento de produtos agrícolas, produtos semiacabados, etc.}
\end{EntryWithPhonetic}

\begin{EntryWithPhonetic}{工艺}{gong1yi4}{3,4}{⼯,⾋}[HSK 5]
  \definition{s.}{técnica; tecnologia; arte industrial; técnicas ou métodos de fabricação e processamento de produtos | artesanato; arte artesanal}
\end{EntryWithPhonetic}

\begin{EntryWithPhonetic}{工艺流程}{gong1yi4 liu2cheng2}{3,4,10,12}{⼯,⾋,⽔,⽲}
  \definition{s.}{fluxograma do processo; fluxo do processo}
\end{EntryWithPhonetic}

\begin{EntryWithPhonetic}{工艺品}{gong1 yi4 pin3}{3,4,9}{⼯,⾋,⼝}
  \definition[个,件]{s.}{trabalho manual; artesanato; habilidade manual; artigo artesanal; itens delicados produzidos com técnicas artesanais. Por exemplo, esculturas em jade, esmaltes Jingtailan, bordados, etc.}
\end{EntryWithPhonetic}

\begin{EntryWithPhonetic}{工整}{gong1zheng3}{3,16}{⼯,⽁}[HSK 7-9]
  \definition{adj.}{limpo; organizado; meticuloso e organizado; não desleixado | fino; requintado}
\end{EntryWithPhonetic}

\begin{EntryWithPhonetic}{工资}{gong1zi1}{3,10}{⼯,⾙}[HSK 3]
  \definition[份,笔,月,天]{s.}{pagamento; salário; remuneração; vencimentos; o pagamento em dinheiro ou em espécie feito ao trabalhador como remuneração pelo trabalho realizado}
\end{EntryWithPhonetic}

\begin{EntryWithPhonetic}{工作}{gong1zuo4}{3,7}{⼯,⼈}[HSK 1]
  \definition[份,个,分,项]{s.}{trabalho; emprego | dever; tarefa; negócio}
  \definition{v.}{trabalhar; operar (uma máquina); envolver-se em trabalho físico ou intelectual, também se refere de maneira geral a máquinas e ferramentas operadas por pessoas para realizar funções produtivas}
\end{EntryWithPhonetic}

\begin{EntryWithPhonetic}{工作量}{gong1zuo4liang4}{3,7,12}{⼯,⼈,⾥}[HSK 7-9]
  \definition{s.}{quantidade de trabalho; volume de trabalho; carga de trabalho}
\end{EntryWithPhonetic}

\begin{EntryWithPhonetic}{工作日}{gong1zuo4ri4}{3,7,4}{⼯,⼈,⽇}[HSK 5]
  \definition{s.}{dia de trabalho; dia útil; dias em que você deveria estar trabalhando de acordo com as regras | horas de trabalho por dia; horas do dia para fazer o trabalho necessário}
\end{EntryWithPhonetic}

%%%%%%%%%% 弓 %%%%%%%%%%
\subsection*{弓}\addcontentsline{loh}{figure}{弓 \dpy{gong1}}

\begin{EntryWithPhonetic}{弓}{gong1}{3}{⼸}[HSK 7-9][Kangxi 57]
  \definition*{s.}{Sobrenome: Gong}
  \definition{clas.}{uma antiga unidade de comprimento para medir a terra, igual a cinco 尺}
  \definition[张]{s.}{arco | qualquer coisa em forma de arco | Obsoleto: ferramenta de madeira de medição de terreno; divisores de madeira para medição de terrenos; arco de medição; régua escalonada (1,5m)}
  \definition{v.}{dobrar; arquear; curvar; entortar}
  \seealsoref{尺}{chi3}
\end{EntryWithPhonetic}

%%%%%%%%%% 公 %%%%%%%%%%
\subsection*{公}\addcontentsline{loh}{figure}{公 \dpy{gong1}}

\begin{EntryWithPhonetic}{公}{gong1}{4}{⼋}[HSK 6]
  \definition*{s.}{Sobrenome: Gong}
  \definition{adj.}{público; estatal; coletivo | comum; geral | do mundo; internacional; universal; métrico | imparcial; justo; equitativo |  (de um animal) masculino}
  \definition{s.}{assuntos públicos; negócios oficiais (ou deveres) | autoridade; coletivo | duque | títulos respeitosos para homens idosos; uma saudação respeitosa | marido}
  \definition{v.}{tornar público; divulgar; abrir a todos; exibir}
  \antonymref{母}{mu3}
  \antonymref{私}{si1}
\end{EntryWithPhonetic}

\begin{EntryWithPhonetic}{公安}{gong1'an1}{4,6}{⼋,⼧}[HSK 6]
  \definition[名,位]{s.}{segurança pública; a segurança e estabilidade dos direitos dos cidadãos, da propriedade da segurança pública e da ordem social | agente de segurança pública; pessoal que mantém a segurança pública}
\end{EntryWithPhonetic}

\begin{EntryWithPhonetic}{公安局}{gong1'an1ju2}{4,6,7}{⼋,⼧,⼫}[HSK 7-9]
  \definition{s.}{departamento de polícia; departamento de segurança pública; o departamento responsável pelo trabalho de segurança pública do Governo Popular}
\end{EntryWithPhonetic}

\begin{EntryWithPhonetic}{公办}{gong1ban4}{4,4}{⼋,⼒}
  \definition{adj.}{público; estatal; administrado pelo governo}
\end{EntryWithPhonetic}

\begin{EntryWithPhonetic}{公布}{gong1bu4}{4,5}{⼋,⼱}[HSK 3]
  \definition{v.}{(leis, decretos, comunicados e avisos de órgãos governamentais) promulgar; anunciar; publicar; tornar público; divulgar publicamente}
\end{EntryWithPhonetic}

\begin{EntryWithPhonetic}{公车}{gong1che1}{4,4}{⼋,⾞}[HSK 7-9]
  \definition[辆]{s.}{ônibus, abreviação de 公共汽车 | carro pertencente a uma organização e usado por seus membros (carro do governo, carro de polícia, carro da empresa etc.), abreviação de 公务用车}
  \seealsoref{公共}{gong1gong4}
  \seealsoref{公共汽车}{gong1gong4 qi4che1}
  \seealsoref{公务用车}{gong1wu4yong4che1}
\end{EntryWithPhonetic}

\begin{EntryWithPhonetic}{公道}{gong1dao4}{4,12}{⼋,⾡}
  \definition{s.}{justiça; o princípio da justiça}
  \seeref{gong1dao5}
\end{EntryWithPhonetic}

\begin{EntryWithPhonetic}{公道}{gong1dao5}{4,12}{⼋,⾡}[HSK 7-9]
  \definition{adj.}{equitativo | justo}
  \seeref{gong1dao4}
\end{EntryWithPhonetic}

\begin{EntryWithPhonetic}{公费}{gong1fei4}{4,9}{⼋,⾙}[HSK 7-9]
  \definition{s.}{despesa pública; despesas fornecidas pelo estado ou grupo}
\end{EntryWithPhonetic}

\begin{EntryWithPhonetic}{公告}{gong1gao4}{4,7}{⼋,⼝}[HSK 5]
  \definition[张,份,项]{s.}{anúncio; notificação de assuntos importantes ao público em geral pelo governo ou por um órgão importante}
  \definition{v.}{anunciar; o governo ou órgão governamental informa publicamente às pessoas algo importante}
\end{EntryWithPhonetic}

\begin{EntryWithPhonetic}{公共}{gong1gong4}{4,6}{⼋,⼋}[HSK 3]
  \definition{adj.}{público; comum; comunal; comunitário; pertencente à sociedade}
  \definition[辆]{s.}{ônibus}
  \seealsoref{公车}{gong1che1}
  \seealsoref{公共汽车}{gong1gong4 qi4che1}
\end{EntryWithPhonetic}

\begin{EntryWithPhonetic}{公共场所}{gong1gong4 chang3suo3}{4,6,6,8}{⼋,⼋,⼟,⼾}[HSK 7-9]
  \definition{s.}{lugares públicos; locais onde o público pode ir}
\end{EntryWithPhonetic}

\begin{EntryWithPhonetic}{公共关系}{gong1gong4 guan1xi4}{4,6,6,7}{⼋,⼋,⼋,⽷}
  \definition{s.}{relações públicas; refere"-se à relação entre grupos, empresas ou indivíduos em atividades sociais, denominada relações públicas}
\end{EntryWithPhonetic}

\begin{EntryWithPhonetic}{公共汽车}{gong1gong4 qi4che1}{4,6,7,4}{⼋,⼋,⽔,⾞}[HSK 2]
  \definition[辆,个]{s.}{ônibus}
  \seealsoref{公车}{gong1che1}
  \seealsoref{公共}{gong1gong4}
\end{EntryWithPhonetic}

\begin{EntryWithPhonetic}{公关}{gong1guan1}{4,6}{⼋,⼋}[HSK 7-9]
  \definition{s.}{relações públicas, abreviação de 公共关系 | pessoa que trabalha em relações públicas}
  \seealsoref{公共关系}{gong1gong4 guan1xi4}
\end{EntryWithPhonetic}

\begin{EntryWithPhonetic}{公函}{gong1han2}{4,8}{⼋,⼐}[HSK 7-9]
  \definition{s.}{carta oficial; correspondência oficial entre departamentos paralelos e não relacionados}
  \antonymref{便函}{bian4han2}
  \antonymref{私函}{si1han2}
\end{EntryWithPhonetic}

\begin{EntryWithPhonetic}{公鸡}{gong1ji1}{4,7}{⼋,⿃}[HSK 6]
  \definition{s.}{galo; frango macho}
\end{EntryWithPhonetic}

\begin{EntryWithPhonetic}{公积金}{gong1ji1jin1}{4,10,8}{⼋,⽲,⾦}[HSK 7-9]
  \definition{s.}{fundo de acumulação (comum); fundo de reserva pública; fundo de reserva comum | fundo acumulado | reservas oficiais}
\end{EntryWithPhonetic}

\begin{EntryWithPhonetic}{公交车}{gong1jiao1che1}{4,6,4}{⼋,⼇,⾞}[HSK 2]
  \definition[辆]{s.}{ônibus urbano; veículo de transporte público}
\end{EntryWithPhonetic}

\begin{EntryWithPhonetic}{公斤}{gong1jin1}{4,4}{⼋,⽄}[HSK 2]
  \definition{clas.}{quilograma (kg)}
\end{EntryWithPhonetic}

\begin{EntryWithPhonetic}{公开}{gong1kai1}{4,4}{⼋,⼶}[HSK 3]
  \definition{adj.}{aberto; público; não oculto; exposto ao público}
  \definition{v.}{tornar público}
\end{EntryWithPhonetic}

\begin{EntryWithPhonetic}{公开信}{gong1kai1xin4}{4,4,9}{⼋,⼶,⼈}[HSK 7-9]
  \definition{s.}{carta aberta; cartas endereçadas a indivíduos ou grupos que o autor acredita serem necessárias para divulgação pública}
\end{EntryWithPhonetic}

\begin{EntryWithPhonetic}{公克}{gong1ke4}{4,7}{⼋,⼗}
  \definition{s.}{grama (medida de peso)}
\end{EntryWithPhonetic}

\begin{EntryWithPhonetic}{公款}{gong1kuan3}{4,12}{⼋,⽋}[HSK 7-9]
  \definition{s.}{dinheiro público (ou fundo); despesa do governo | fundos públicos}
\end{EntryWithPhonetic}

\begin{EntryWithPhonetic}{公里}{gong1li3}{4,7}{⼋,⾥}[HSK 2]
  \definition{s.}{quilômetro (km)}
\end{EntryWithPhonetic}

\begin{EntryWithPhonetic}{公立}{gong1li4}{4,5}{⼋,⽴}[HSK 7-9]
  \definition{adj.}{estabelecido e mantido pelo governo; público | financiado publicamente e administrado pelo governo; estabelecido pelo governo e operado com fundos governamentais para fornecer serviços ao público}
  \antonymref{私立}{si1li4}
\end{EntryWithPhonetic}

\begin{EntryWithPhonetic}{公路}{gong1lu4}{4,13}{⼋,⾜}[HSK 2]
  \definition[条,段]{s.}{rodovia; via de acesso; via de tráfego; estrada; estrada principal;}
\end{EntryWithPhonetic}

\begin{EntryWithPhonetic}{公民}{gong1min2}{4,5}{⼋,⽒}[HSK 3]
  \definition[个,位]{s.}{cidadão; civil; pessoa que possui a nacionalidade de um país, goza dos direitos e cumpre as obrigações previstos na Constituição e nas demais leis desse país}
\end{EntryWithPhonetic}

\begin{EntryWithPhonetic}{公募}{gong1mu4}{4,12}{⼋,⼒}
  \definition{s.}{financiamento público; arrecadação pública de fundos (investimento)}
\end{EntryWithPhonetic}

\begin{EntryWithPhonetic}{公墓}{gong1mu4}{4,13}{⼋,⼟}[HSK 7-9]
  \definition[顿]{s.}{cemitério; cemitério público | Arcaico: túmulos ou cemitérios reais ou aristocráticos | parque memorial}
\end{EntryWithPhonetic}

\begin{EntryWithPhonetic}{公平}{gong1ping2}{4,5}{⼋,⼲}[HSK 2]
  \definition{adj.}{justo; imparcial; equitativo; equidade}
\end{EntryWithPhonetic}

\begin{EntryWithPhonetic}{公仆}{gong1pu2}{4,4}{⼋,⼈}[HSK 7-9]
  \definition[个,位,名,些]{s.}{servidor público; oficial | funcionário público; pessoas que servem o público}
\end{EntryWithPhonetic}

\begin{EntryWithPhonetic}{公顷}{gong1qing3}{4,8}{⼋,⾴}[HSK 7-9]
  \definition{clas.}{hectare; é uma unidade de área terrestre no sistema métrico;  equivale a 10.000 metros quadrados, ou 15 市亩}
  \seealsoref{市亩}{shi4mu3}
\end{EntryWithPhonetic}

\begin{EntryWithPhonetic}{公然}{gong1ran2}{4,12}{⼋,⽕}[HSK 7-9]
  \definition{adv.}{Pejorativo: abertamente; publicamente; sem disfarces; descaradamente; flagrantemente}
  \antonymref{暗自}{an4zi4}
\end{EntryWithPhonetic}

\begin{EntryWithPhonetic}{公认}{gong1ren4}{4,4}{⼋,⾔}[HSK 5]
  \definition{v.}{(geralmente) reconhecer; (universalmente) aceitar}
\end{EntryWithPhonetic}

\begin{EntryWithPhonetic}{公示}{gong1shi4}{4,5}{⼋,⽰}[HSK 7-9]
  \definition{v.}{dar a conhecer ao público e pedir opiniões}
\end{EntryWithPhonetic}

\begin{EntryWithPhonetic}{公式}{gong1shi4}{4,6}{⼋,⼷}[HSK 5]
  \definition[个,些,种]{s.}{fórmula; expressão}
\end{EntryWithPhonetic}

\begin{EntryWithPhonetic}{公事}{gong1shi4}{4,8}{⼋,⼅}[HSK 7-9]
  \definition{s.}{assuntos públicos; negócios oficiais (ou deveres) | Coloquial: documento oficial}
  \antonymref{私事}{si1shi4}
\end{EntryWithPhonetic}

\begin{EntryWithPhonetic}{公司}{gong1si1}{4,5}{⼋,⼝}[HSK 2]
  \definition[个,家]{s.}{empresa; companhia; corporação; uma organização industrial e comercial que opera na produção de produtos, circulação de mercadorias ou certos empreendimentos de construção, etc.}
\end{EntryWithPhonetic}

\begin{EntryWithPhonetic}{公司治理}{gong1si1zhi4li3}{4,5,8,11}{⼋,⼝,⽔,⽟}
  \definition{s.}{governança corporativa}
\end{EntryWithPhonetic}

\begin{EntryWithPhonetic}{公务}{gong1wu4}{4,5}{⼋,⼒}[HSK 7-9]
  \definition{s.}{assuntos públicos; negócios oficiais; em relação a assuntos nacionais ou coletivos}
\end{EntryWithPhonetic}

\begin{EntryWithPhonetic}{公务用车}{gong1wu4yong4che1}{4,5,5,4}{⼋,⼒,⽤,⾞}
  \definition{s.}{veículos oficiais}
\end{EntryWithPhonetic}

\begin{EntryWithPhonetic}{公务员}{gong1wu4yuan2}{4,5,7}{⼋,⼒,⼝}[HSK 3]
  \definition[个,位,名,些]{s.}{funcionário público; funcionário de órgãos governamentais}
\end{EntryWithPhonetic}

\begin{EntryWithPhonetic}{公益}{gong1yi4}{4,10}{⼋,⽫}[HSK 7-9]
  \definition{s.}{bem público; comunitário; bem"-estar; interesse público (referindo"-se principalmente ao bem"-estar público, como saúde e assistência)}
\end{EntryWithPhonetic}

\begin{EntryWithPhonetic}{公益性}{gong1yi4xing4}{4,10,8}{⼋,⽫,⼼}[HSK 7-9]
  \definition{s.}{bem-estar público}
\end{EntryWithPhonetic}

\begin{EntryWithPhonetic}{公用}{gong1yong4}{4,5}{⼋,⽤}[HSK 7-9]
  \definition{adj.}{público; comunitário; para uso público, uso comum}
\end{EntryWithPhonetic}

\begin{EntryWithPhonetic}{公用电话}{gong1yong4dian4hua4}{4,5,5,8}{⼋,⽤,⽥,⾔}
  \definition[部]{s.}{telefone público}
\end{EntryWithPhonetic}

\begin{EntryWithPhonetic}{公寓}{gong1yu4}{4,12}{⼋,⼧}[HSK 7-9]
  \definition[套,座,栋,间]{s.}{apartamentos; moradias; colônias de férias; prédio de apartamentos; construções que podem acomodar várias famílias}
\end{EntryWithPhonetic}

\begin{EntryWithPhonetic}{公元}{gong1yuan2}{4,4}{⼋,⼉}[HSK 4]
  \definition{s.}{D.C. (Depois de~Cristo); a era cristã; um método internacionalmente aceito de registro de datas, o ano lendário do nascimento de Jesus é 1 d.C., também conhecido como o primeiro ano da Era Comum, e é denotado por D.C.}
  \seealsoref{前}{qian2}
\end{EntryWithPhonetic}

\begin{EntryWithPhonetic}{公园}{gong1yuan2}{4,7}{⼋,⼞}[HSK 2]
  \definition[个,座]{s.}{parque; jardim público; os jardins abertos ao público para passeios e descanso geralmente ficam nas cidades, têm muitas flores, árvores e, em alguns casos, lagos}
\end{EntryWithPhonetic}

\begin{EntryWithPhonetic}{公约}{gong1yue1}{4,6}{⼋,⽷}[HSK 7-9]
  \definition[项,条]{s.}{convenção; pacto; aliança; tratado | compromisso conjunto; regulamentos acordados coletivamente dentro de uma organização | regulamentos acordados coletivamente dentro de uma unidade de trabalho}
\end{EntryWithPhonetic}

\begin{EntryWithPhonetic}{公正}{gong1zheng4}{4,5}{⼋,⽌}[HSK 5]
  \definition{adj.}{justo; equitativo; imparcial; de mente justa; equidade e integridade sem favoritismo}
\end{EntryWithPhonetic}

\begin{EntryWithPhonetic}{公证}{gong1zheng4}{4,7}{⼋,⾔}[HSK 7-9]
  \definition{adj.}{autenticado em cartório}
  \definition{s.}{reconhecimento de firma}
  \definition{v.}{autenticar}
\end{EntryWithPhonetic}

\begin{EntryWithPhonetic}{公职}{gong1zhi2}{4,11}{⼋,⽿}[HSK 7-9]
  \definition{s.}{cargo público; emprego público; cargos ou patentes oficiais}
\end{EntryWithPhonetic}

\begin{EntryWithPhonetic}{公众}{gong1zhong4}{4,6}{⼋,⼈}[HSK 6]
  \definition[对]{s.}{o público; as massas; refere"-se à maioria das pessoas na sociedade}
\end{EntryWithPhonetic}

\begin{EntryWithPhonetic}{公主}{gong1zhu3}{4,5}{⼋,⼂}[HSK 6]
  \definition[个,位,名,些]{s.}{princesa; a filha do monarca}
\end{EntryWithPhonetic}

%%%%%%%%%% 功 %%%%%%%%%%
\subsection*{功}\addcontentsline{loh}{figure}{功 \dpy{gong1}}

\begin{EntryWithPhonetic}{功}{gong1}{5}{⼒}[HSK 7-9]
  \definition*{s.}{Sobrenome: Gong}
  \definition[次,大]{s.}{mérito; façanha; serviço meritório (ação) | resultado; eficácia; realização | habilidade; habilidade técnica; tecnologia e qualificação técnica | trabalho; uma força faz com que um objeto se desloque uma certa distância na direção da força}
\end{EntryWithPhonetic}

\begin{EntryWithPhonetic}{功臣}{gong1chen2}{5,6}{⼒,⾂}[HSK 7-9]
  \definition[个]{s.}{pessoa que prestou serviço excepcional | ministro que prestou serviço excepcional; um funcionário meritório se refere a alguém que fez contribuições significativas para uma determinada causa.}
  \antonymref{罪人}{zui4ren2}
\end{EntryWithPhonetic}

\begin{EntryWithPhonetic}{功底}{gong1di3}{5,8}{⼒,⼴}[HSK 7-9]
  \definition{s.}{fundação; base sólida; formação profunda; conhecimento dos fundamentos; boas habilidades básicas}
\end{EntryWithPhonetic}

\begin{EntryWithPhonetic}{功夫}{gong1fu5}{5,4}{⼒,⼤}[HSK 3]
  \definition*{s.}{Gongfu (Kung Fu), arte marcial}
  \definition[番]{s.}{habilidade; destreza; conhecimento | luta acrobática; habilidade em artes marciais | esforço; tempo e energia}
\end{EntryWithPhonetic}

\begin{EntryWithPhonetic}{功绩}{gong1ji4}{5,11}{⼒,⽷}
  \definition{s.}{mérito e realização; feito; contribuição}
  \antonymref{过失}{guo4shi1}
\end{EntryWithPhonetic}

\begin{EntryWithPhonetic}{功课}{gong1ke4}{5,10}{⼒,⾔}[HSK 3]
  \definition[份,门]{s.}{trabalho escolar; dever de casa; refere"-se aos trabalhos de casa atribuídos pelos professores aos alunos| tarefa; lições; lição escolar | preparações; preparação necessária antes de fazer algo}
\end{EntryWithPhonetic}

\begin{EntryWithPhonetic}{功劳}{gong1lao2}{5,7}{⼒,⼒}[HSK 7-9]
  \definition{s.}{contribuição; crédito; contribuição para a causa}
\end{EntryWithPhonetic}

\begin{EntryWithPhonetic}{功力}{gong1li4}{5,2}{⼒,⼒}[HSK 7-9]
  \definition{s.}{efeito; eficácia | habilidade}
\end{EntryWithPhonetic}

\begin{EntryWithPhonetic}{功率}{gong1lv4}{5,11}{⼒,⽞}[HSK 7-9]
  \definition[瓦,千瓦,兆瓦]{s.}{potência (W); uma grandeza física que indica a velocidade com que o trabalho é realizado; ou seja, o trabalho realizado ou consumido por unidade de tempo; a unidade é Watt}
\end{EntryWithPhonetic}

\begin{EntryWithPhonetic}{功能}{gong1neng2}{5,10}{⼒,⾁}[HSK 3]
  \definition[种,项]{s.}{função; os efeitos positivos produzidos por coisas ou métodos}
\end{EntryWithPhonetic}

\begin{EntryWithPhonetic}{功效}{gong1xiao4}{5,10}{⼒,⽁}[HSK 7-9]
  \definition{s.}{efeito; eficiência; comportamento; eficácia; o efeito, função ou eficiência de um medicamento, método ou outra coisa}
\end{EntryWithPhonetic}

%%%%%%%%%% 攻 %%%%%%%%%%
\subsection*{攻}\addcontentsline{loh}{figure}{攻 \dpy{gong1}}

\begin{EntryWithPhonetic}{攻}{gong1}{7}{⽁}[HSK 7-9]
  \definition*{s.}{Sobrenome: Gong}
  \definition{v.}{atacar; assaltar; tomar a ofensiva | acusar; cobrar | estudar; trabalhar em; especializar-se em}
\end{EntryWithPhonetic}

\begin{EntryWithPhonetic}{攻读}{gong1du2}{7,10}{⽁,⾔}[HSK 7-9]
  \definition{v.}{especializar-se em; trabalhar arduamente em uma matéria para obter um diploma ou certificado nessa matéria; estudar bastante; estudar ou aprofundar-se em um assunto}
\end{EntryWithPhonetic}

\begin{EntryWithPhonetic}{攻关}{gong1guan1}{7,6}{⽁,⼋}[HSK 7-9]
  \definition{v.}{ter que encarar; superar esseobstáculo; começar essa jornada; resolver esse problema; abordar os principais problemas}
\end{EntryWithPhonetic}

\begin{EntryWithPhonetic}{攻击}{gong1ji1}{7,5}{⽁,⼐}[HSK 6]
  \definition{v.}{atacar; assaltar; lançar uma ofensiva | difamar; caluniar; acusar; atacar (verbalmente)}
\end{EntryWithPhonetic}

%%%%%%%%%% 供 %%%%%%%%%%
\subsection*{供}\addcontentsline{loh}{figure}{供 \dpy{gong1}}

\begin{EntryWithPhonetic}{供}{gong1}{8}{⼈}[HSK 7-9]
  \definition*{s.}{Sobrenome: Gong}
  \definition{v.}{fornecer; alimentar |  fornecer algo (para uso ou conveniência de); fornecer algumas condições de exploração à outra parte}
  \seeref{gong4}
\end{EntryWithPhonetic}

\begin{EntryWithPhonetic}{供不应求}{gong1bu2ying4qiu2}{8,4,7,7}{⼈,⼀,⼴,⽔}[HSK 7-9]
  \definition{expr.}{``A oferta fica aquém da demanda.'' ou ``A demanda excede a oferta.''}
\end{EntryWithPhonetic}

\begin{EntryWithPhonetic}{供给}{gong1ji3}{8,9}{⼈,⽷}[HSK 6]
  \definition{s.}{fornecer; prover; fornecer produção e necessidades de vida, dinheiro, etc. para aqueles que precisam}
\end{EntryWithPhonetic}

\begin{EntryWithPhonetic}{供暖}{gong1nuan3}{8,13}{⼈,⽇}[HSK 7-9]
  \definition{s.}{fornecimento de aquecimento}
  \definition{v.}{fornecer aquecimento}
\end{EntryWithPhonetic}

\begin{EntryWithPhonetic}{供求}{gong1qiu2}{8,7}{⼈,⽔}[HSK 7-9]
  \definition{s.}{Economia: oferta e procura (principalmente de commodities)}
\end{EntryWithPhonetic}

\begin{EntryWithPhonetic}{供应}{gong1ying4}{8,7}{⼈,⼴}[HSK 4]
  \definition{v.}{fornecer; prover de}
\end{EntryWithPhonetic}

%%%%%%%%%% 宫 %%%%%%%%%%
\subsection*{宫}\addcontentsline{loh}{figure}{宫 \dpy{gong1}}

\begin{EntryWithPhonetic}{宫}{gong1}{9}{⼧}[HSK 6]
  \definition*{s.}{Sobrenome: Gong}
  \definition[座]{s.}{palácio imperial; palácio; casas onde o imperador, a imperatriz, o príncipe, etc. vivem | morada de seres sobrenaturais; palácio; paraíso; casas onde vivem os deuses na mitologia | templo (usado em um nome de templo) | local para atividades culturais e recreativas; um edifício para atividades culturais e recreativas; casas para fins culturais e de entretenimento | útero | uma nota da antiga escala chinesa de cinco tons, correspondente a 1 na notação musical numerada}
\end{EntryWithPhonetic}

\begin{EntryWithPhonetic}{宫殿}{gong1dian4}{9,13}{⼧,⽎}[HSK 7-9]
  \definition[座]{s.}{palácio; geralmente se refere às casas magníficas onde os imperadores vivem}
\end{EntryWithPhonetic}

\begin{EntryWithPhonetic}{宫廷}{gong1ting2}{9,6}{⼧,⼵}[HSK 7-9]
  \definition{s.}{palácio imperial; residência do imperador; palácio | corte real ou imperial; o monarca e seus funcionários; corte}
\end{EntryWithPhonetic}

%%%%%%%%%% 恭 %%%%%%%%%%
\subsection*{恭}\addcontentsline{loh}{figure}{恭 \dpy{gong1}}

\begin{EntryWithPhonetic}{恭}{gong1}{10}{⼼}
  \definition{adj.}{respeitoso; reverente | educado}
\end{EntryWithPhonetic}

\begin{EntryWithPhonetic}{恭维}{gong1wei2}{10,11}{⼼,⽷}[HSK 7-9]
  \definition{v.}{bajular; elogiar}
\end{EntryWithPhonetic}

\begin{EntryWithPhonetic}{恭喜}{gong1xi3}{10,12}{⼼,⼝}[HSK 7-9]
  \definition{v.}{parabenizar; uma maneira educada de parabenizar alguém por seu feliz evento}
\end{EntryWithPhonetic}

%%%%%%%%%% 巩 %%%%%%%%%%
\subsection*{巩}\addcontentsline{loh}{figure}{巩 \dpy{gong3}}

\begin{EntryWithPhonetic}{巩}{gong3}{6}{⼯}
  \definition*{s.}{Sobrenome: Gong}
  \definition{s.}{seguro | sólido}
  \definition{v.}{consolidar}
\end{EntryWithPhonetic}

\begin{EntryWithPhonetic}{巩固}{gong3gu4}{6,8}{⼯,⼞}[HSK 6]
  \definition{adj.}{sólido; estável; consolidado; não facilmente abalado (usado principalmente para coisas abstratas)}
  \definition{v.}{consolidar}
\end{EntryWithPhonetic}

%%%%%%%%%% 拱 %%%%%%%%%%
\subsection*{拱}\addcontentsline{loh}{figure}{拱 \dpy{gong3}}

\begin{EntryWithPhonetic}{拱}{gong3}{9}{⼿}[HSK 7-9]
  \definition*{s.}{Sobrenome: Gong}
  \definition{s.}{Arquitetura: arco}[游客们在拱门前留影。===Turistas tiram fotos em frente ao arco.]
  \definition{v.}{colocar uma mão na outra em frente ao peito (em saudação) | cercar | arquear-se | empurrar sem usar as mãos; bater em um objeto com seu corpo | (porcos, etc.) cavar a terra com o focinho; (minhocas, etc.) contorcer-se na terra | brotar através da terra}
\end{EntryWithPhonetic}

%%%%%%%%%% 共 %%%%%%%%%%
\subsection*{共}\addcontentsline{loh}{figure}{共 \dpy{gong4}}

\begin{EntryWithPhonetic}{共}{gong4}{6}{⼋}[HSK 4]
  \definition*{s.}{Partido Comunista, abreviação de 共产党 | Sobrenome: Gong}
  \definition{adj.}{conjunto; mútuo; geral; comum; o mesmo para todos}
  \definition{adv.}{juntos; juntamente; conjuntamente | em sua totalidade; em todos}
  \definition{v.}{compartilhar com; empreender ou realizar em conjunto}
  \seealsoref{共产党}{gong4chan3dang3}
\end{EntryWithPhonetic}

\begin{EntryWithPhonetic}{共产}{gong4chan3}{6,6}{⼋,⼇}
  \definition{adj.}{comunista}
  \definition{s.}{comunismo}
\end{EntryWithPhonetic}

\begin{EntryWithPhonetic}{共产党}{gong4chan3dang3}{6,6,10}{⼋,⼇,⼉}
  \definition*{s.}{Partido Comunista}
\end{EntryWithPhonetic}

\begin{EntryWithPhonetic}{共产主义}{gong4chan3 zhu3yi4}{6,6,5,3}{⼋,⼇,⼂,⼂}
  \definition*{s.}{Comunismo}
\end{EntryWithPhonetic}

\begin{EntryWithPhonetic}{共计}{gong4ji4}{6,4}{⼋,⾔}[HSK 5]
  \definition{s.}{total; total geral; agregado; montante}
  \definition{v.}{contar até; somar até; totalizar}
\end{EntryWithPhonetic}

\begin{EntryWithPhonetic}{共鸣}{gong4ming2}{6,8}{⼋,⿃}[HSK 7-9]
  \definition{s.}{ressonância; fenômeno que ocorre quando um objeto ressoa, por exemplo, quando dois diapasões com a mesma frequência são colocados próximos um do outro, quando um vibra e emite um som, o outro também emite um som | resposta simpática; uma metáfora para ter as mesmas emoções que outra pessoa}
\end{EntryWithPhonetic}

\begin{EntryWithPhonetic}{共时}{gong4shi2}{6,7}{⼋,⽇}
  \definition{adj.}{sincrônico; simultâneo}
  \antonymref{历时}{li4shi2}
\end{EntryWithPhonetic}

\begin{EntryWithPhonetic}{共识}{gong4shi2}{6,7}{⼋,⾔}[HSK 7-9]
  \definition{s.}{consenso; entendimento comum}
\end{EntryWithPhonetic}

\begin{EntryWithPhonetic}{共同}{gong4tong2}{6,6}{⼋,⼝}[HSK 3]
  \definition{adj.}{comum; compartilhado; colaborativo; todos têm}
  \definition{adv.}{juntos; conjuntamente; todos juntos (fazemos)}
\end{EntryWithPhonetic}

\begin{EntryWithPhonetic}{共同体}{gong4tong2ti3}{6,6,7}{⼋,⼝,⼈}[HSK 7-9]
  \definition[个]{s.}{comunidade}[欧洲经济共同体===Comunidade Econômica Europeia]
\end{EntryWithPhonetic}

\begin{EntryWithPhonetic}{共享}{gong4xiang3}{6,8}{⼋,⼇}[HSK 5]
  \definition{v.}{compartilhar; desfrutar juntos; aproveitar as coisas boas juntos}
\end{EntryWithPhonetic}

\begin{EntryWithPhonetic}{共性}{gong4xing4}{6,8}{⼋,⼼}[HSK 7-9]
  \definition{s.}{caráter geral (comum); natureza comum; generalidade; semelhança; universalidade}
\end{EntryWithPhonetic}

\begin{EntryWithPhonetic}{共有}{gong4you3}{6,6}{⼋,⽉}[HSK 3]
  \definition{v.}{compartilhar; possuir (por todos); possuir ou desfrutar em conjunto}
\end{EntryWithPhonetic}

%%%%%%%%%% 贡 %%%%%%%%%%
\subsection*{贡}\addcontentsline{loh}{figure}{贡 \dpy{gong4}}

\begin{EntryWithPhonetic}{贡}{gong4}{7}{⾙}
  \definition*{s.}{Sobrenome: Gong}
  \definition[个,批]{s.}{tributo}
  \definition{v.}{recomendar uma pessoa adequada à corte imperial; recomendar talentos à corte na era feudal | prestar homenagem; pagar tributos (à corte imperial)}
\end{EntryWithPhonetic}

\begin{EntryWithPhonetic}{贡献}{gong4xian4}{7,13}{⾙,⽝}[HSK 6]
  \definition[份]{s.}{contribuição; boas ações feitas para o país ou para o público}
  \definition{v.}{dedicar; contribuir; contribuir com materiais, força, experiência, etc. para o país ou para o público}
\end{EntryWithPhonetic}

%%%%%%%%%% 供 %%%%%%%%%%
\subsection*{供}\addcontentsline{loh}{figure}{供 \dpy{gong4}}

\begin{EntryWithPhonetic}{供}{gong4}{8}{⼈}
  \definition{s.}{oferendas | confissão}
  \definition{v.}{depositar (oferendas) | confessar}
  \seeref{gong1}
\end{EntryWithPhonetic}

\begin{EntryWithPhonetic}{供奉}{gong4feng4}{8,8}{⼈,⼤}[HSK 7-9]
  \definition{s.}{artista servindo ao imperador; uma pessoa que serve ao imperador com alguma habilidade; especialmente um ator que é convocado ao palácio para atuar}
  \definition{v.}{consagrar; consagrar e adorar; colocar incenso e velas em frente aos retratos ou tábuas de deuses, Budas ou ancestrais; colocar oferendas; mostrar respeito | prestar homenagem à corte imperial}
\end{EntryWithPhonetic}

%%%%%%%%%% 勾 %%%%%%%%%%
\subsection*{勾}\addcontentsline{loh}{figure}{勾 \dpy{gou1}}

\begin{EntryWithPhonetic}{勾}{gou1}{4}{⼓}[HSK 7-9]
  \definition*{s.}{Sobrenome: Gou}
  \definition{s.}{nos tempos antigos, referia"-se ao lado mais curto de um triângulo retângulo isósceles}
  \definition{v.}{cancelar; riscar; marcar | delinear; desenhar | preencher as juntas da alvenaria com argamassa ou cimento; apontar | adicionar algo para engrossar; engrossar | induzir; evocar; trazer à mente | conspirar com; unir"-se a | excluir; apagar | seduzir; atrair}
  \variantof{钩}
  \seeref{gou4}
\end{EntryWithPhonetic}

\begin{EntryWithPhonetic}{勾画}{gou1hua4}{4,8}{⼓,⽥}[HSK 7-9]
  \definition{v.}{esboçar; delinear; desenhar o contorno de; descrever em palavras curtas}
\end{EntryWithPhonetic}

\begin{EntryWithPhonetic}{勾结}{gou1jie2}{4,9}{⼓,⽷}[HSK 7-9]
  \definition{v.}{conspirar com; estar em aliança com; ser de mão e luva com; colaborar com; unir-se a; conspirar secretamente e combinar-se entre si para realizar atividades impróprias}
\end{EntryWithPhonetic}

%%%%%%%%%% 沟 %%%%%%%%%%
\subsection*{沟}\addcontentsline{loh}{figure}{沟 \dpy{gou1}}

\begin{EntryWithPhonetic}{沟}{gou1}{7}{⽔}[HSK 5]
  \definition[条,道,段]{s.}{canal; vala; sarjeta; trincheira; cursos d'água ou fortificações escavados | ranhura; sulco raso; uma depressão que se assemelha a uma vala | ravina; barranco; cursos d'água}
\end{EntryWithPhonetic}

\begin{EntryWithPhonetic}{沟通}{gou1tong1}{7,10}{⽔,⾡}[HSK 5]
  \definition{v.}{comunicar; comunicar-se para entender as ideias, opiniões, etc. | conectar; ligar; estabelecer um paralelo entre os dois}
\end{EntryWithPhonetic}

%%%%%%%%%% 钩 %%%%%%%%%%
\subsection*{钩}\addcontentsline{loh}{figure}{钩 \dpy{gou1}}

\begin{EntryWithPhonetic}{钩}{gou1}{9}{⾦}[HSK 7-9]
  \definition*{s.}{Sobrenome: Gou}
  \definition[只,个]{s.}{gancho | traço de gancho em caracteres chineses | marca de verificação; visto; \emph{tick}; \emph{check mark} | marca em forma de gancho | uma espada em forma de gancho | forma falada do numeral 9 em certas ocasiões}
  \definition{v.}{prender com um gancho; enganchar | fazer crochê | costurar com pontos grandes | costurar com pontos longos}
\end{EntryWithPhonetic}

\begin{EntryWithPhonetic}{钩子}{gou1zi5}{9,3}{⾦,⼦}[HSK 7-9]
  \definition[个]{s.}{gancho | coisa parecida com um gancho}
\end{EntryWithPhonetic}

%%%%%%%%%% 狗 %%%%%%%%%%
\subsection*{狗}\addcontentsline{loh}{figure}{狗 \dpy{gou3}}

\begin{EntryWithPhonetic}{狗}{gou3}{8}{⽝}[HSK 2]
  \definition[条,只,群]{s.}{cão; cachorro | palavrão usado para se referir a pessoas más ou seus capangas}
\end{EntryWithPhonetic}

%%%%%%%%%% 勾 %%%%%%%%%%
\subsection*{勾}\addcontentsline{loh}{figure}{勾 \dpy{gou4}}

\begin{EntryWithPhonetic}{勾}{gou4}{4}{⼓}
  \definition*{s.}{Sobrenome: Gou}
  \definition{s.}{usado em 勾当}
  \seeref{gou1}
  \seealsoref{勾当}{gou4dang4}
\end{EntryWithPhonetic}

\begin{EntryWithPhonetic}{勾当}{gou4dang4}{4,6}{⼓,⼹}
  \definition{s.}{(depreciativo, obscuro) negócio; acordo; esquema; atividade}
\end{EntryWithPhonetic}

%%%%%%%%%% 句 %%%%%%%%%%
\subsection*{句}\addcontentsline{loh}{figure}{句 \dpy{gou4}}

\begin{EntryWithPhonetic}{句}{gou4}{5}{⼝}
  \variantof{勾}
  \seeref{ju4}
\end{EntryWithPhonetic}

%%%%%%%%%% 构 %%%%%%%%%%
\subsection*{构}\addcontentsline{loh}{figure}{构 \dpy{gou4}}

\begin{EntryWithPhonetic}{构}{gou4}{8}{⽊}
  \definition{s.}{composição literária}
  \definition{v.}{construir; formar; compor | fabricar; inventar | construir; erguer uma casa}
  \variantof{够}
\end{EntryWithPhonetic}

\begin{EntryWithPhonetic}{构成}{gou4/cheng2}{8,6}{⽊,⼽}[HSK 4]
  \definition{s.}{parte; componente; composição; estrutura}
  \definition{v.+compl.}{formar; compor; constituir; compor; encaixar muitas partes para formar um todo | consistir; causar; formar (principalmente em termos jurídicos)}
\end{EntryWithPhonetic}

\begin{EntryWithPhonetic}{构建}{gou4jian4}{8,8}{⽊,⼵}[HSK 6]
  \definition{v.}{estabelecer (usado principalmente para coisas abstratas); montar; instalar}
\end{EntryWithPhonetic}

\begin{EntryWithPhonetic}{构思}{gou4si1}{8,9}{⽊,⼼}[HSK 7-9]
  \definition{s.}{concepção (ideia); o resultado da concepção}
  \definition{v.}{elaborar o enredo de uma obra literária ou a composição de uma pintura; pensar bem antes de escrever artigos ou criar obras literárias}
\end{EntryWithPhonetic}

\begin{EntryWithPhonetic}{构想}{gou4xiang3}{8,13}{⽊,⼼}[HSK 7-9]
  \definition{s.}{ideia; concepção; ideias formadas}
  \definition[种,个]{v.}{pensar (em um plano, projeto, etc.); conceber; usar a mente ao escrever ou criar arte}
\end{EntryWithPhonetic}

\begin{EntryWithPhonetic}{构造}{gou4zao4}{8,10}{⽊,⾡}[HSK 4]
  \definition[种]{s.}{estrutura; construção; disposição, organização e inter-relação dos componentes}
  \definition{v.}{formar; construir}
\end{EntryWithPhonetic}

%%%%%%%%%% 诟 %%%%%%%%%%
\subsection*{诟}\addcontentsline{loh}{figure}{诟 \dpy{gou4}}

\begin{EntryWithPhonetic}{诟}{gou4}{8}{⾔}
  \definition*{s.}{Sobrenome: Gou}
  \definition{s.}{vergonha; humilhação}
  \definition{v.}{insultar; xingar; falar de forma abusiva}
\end{EntryWithPhonetic}

\begin{EntryWithPhonetic}{诟骂}{gou4ma4}{8,9}{⾔,⾺}
  \definition{v.}{abusar verbalmente | insultar | criticar}
\end{EntryWithPhonetic}

%%%%%%%%%% 购 %%%%%%%%%%
\subsection*{购}\addcontentsline{loh}{figure}{购 \dpy{gou4}}

\begin{EntryWithPhonetic}{购}{gou4}{8}{⾙}[HSK 7-9]
  \definition{v.}{comprar}
\end{EntryWithPhonetic}

\begin{EntryWithPhonetic}{购买}{gou4mai3}{8,6}{⾙,⼄}[HSK 4]
  \definition{v.}{comprar; adquirir; usar dinheiro para obter itens}
\end{EntryWithPhonetic}

\begin{EntryWithPhonetic}{购物}{gou4wu4}{8,8}{⾙,⽜}[HSK 4]
  \definition{s.}{compras; itens comprados; \emph{shopping}}
  \definition{v.}{ir às compras; fazer compras}
\end{EntryWithPhonetic}

%%%%%%%%%% 够 %%%%%%%%%%
\subsection*{够}\addcontentsline{loh}{figure}{够 \dpy{gou4}}

\begin{EntryWithPhonetic}{够}{gou4}{11}{⼣}[HSK 2]
  \definition{adj.}{suficiente; adequado; apropriado; atingir e ultrapassar um determinado limite, difícil de suportar}
  \definition{adv.}{suficientemente; o suficiente (para atingir um determinado nível); indica que atingiu um determinado padrão ou nível elevado}
  \definition{v.}{alcançar (algo, esticando-se); (usando membros, etc.) esticar-se para alcançar ou tocar em locais de difícil acesso | atingir (um padrão ou nível); satisfazer ou atingir a quantidade, os padrões, etc. necessários}
\end{EntryWithPhonetic}

\begin{EntryWithPhonetic}{够本}{gou4ben3}{11,5}{⼣,⽊}
  \definition{v.}{empatar | fazer valer o dinheiro}
\end{EntryWithPhonetic}

\begin{EntryWithPhonetic}{够不着}{gou4bu5zhao2}{11,4,11}{⼣,⼀,⽬}
  \definition{v.}{ser incapaz de alcançar}
\end{EntryWithPhonetic}

\begin{EntryWithPhonetic}{够得着}{gou4de5zhao2}{11,11,11}{⼣,⼻,⽬}
  \definition{v.}{estar à altura | alcançar}
\end{EntryWithPhonetic}

\begin{EntryWithPhonetic}{够格}{gou4ge2}{11,10}{⼣,⽊}
  \definition{adj.}{apto | qualificado | apresentável}
\end{EntryWithPhonetic}

\begin{EntryWithPhonetic}{够朋友}{gou4peng2you5}{11,8,4}{⼣,⽉,⼜}
  \definition{v.}{ser um amigo verdadeiro}
\end{EntryWithPhonetic}

\begin{EntryWithPhonetic}{够呛}{gou4qiang4}{11,7}{⼣,⼝}[HSK 7-9]
  \definition{adj.}{terrível; insuportável; descreve uma situação extremamente grave e insuportável | improvável; bastante improvável; quase impossível; descreve como difícil de alcançar}
\end{EntryWithPhonetic}

\begin{EntryWithPhonetic}{够戗}{gou4qiang4}{11,8}{⼣,⼽}
  \variantof{够呛}
\end{EntryWithPhonetic}

\begin{EntryWithPhonetic}{够味}{gou4wei4}{11,8}{⼣,⼝}
  \definition{adj.}{excelente | na medida}
\end{EntryWithPhonetic}

%%%%%%%%%% 彀 %%%%%%%%%%
\subsection*{彀}\addcontentsline{loh}{figure}{彀 \dpy{gou4}}

\begin{EntryWithPhonetic}{彀}{gou4}{13}{⼸}
  \definition{adj.}{suficiente; adequado}
  \definition{v.}{puxar um arco ao máximo}
\end{EntryWithPhonetic}

%%%%%%%%%% 估 %%%%%%%%%%
\subsection*{估}\addcontentsline{loh}{figure}{估 \dpy{gu1}}

\begin{EntryWithPhonetic}{估}{gu1}{7}{⼈}
  \definition{v.}{estimar; avaliar; aferir}
  \seeref{gu4}
\end{EntryWithPhonetic}

\begin{EntryWithPhonetic}{估计}{gu1ji4}{7,4}{⼈,⾔}[HSK 5]
  \definition{v.}{fazer contas; estimar; calcular; julgar a natureza, quantidade, mudança, etc. de uma coisa em uma determinada situação | parecer; parecer como se; aparentar; fazer inferências aproximadas sobre a natureza, a quantidade e a mudança das coisas com base em determinadas circunstâncias}
\end{EntryWithPhonetic}

\begin{EntryWithPhonetic}{估算}{gu1suan4}{7,14}{⼈,⽵}[HSK 7-9]
  \definition{v.}{calcular; estimar}
\end{EntryWithPhonetic}

%%%%%%%%%% 姑 %%%%%%%%%%
\subsection*{姑}\addcontentsline{loh}{figure}{姑 \dpy{gu1}}

\begin{EntryWithPhonetic}{姑}{gu1}{8}{⼥}
  \definition{adv.}{provisoriamente; por enquanto}
  \definition[个,位,名,些]{s.}{irmã do pai; tia | irmã do marido; cunhada | mãe do marido; sogra | freira; mulher que exerce uma ocupação religiosa | a irmã do pai de alguém | mulheres jovens (no campo)}
\end{EntryWithPhonetic}

\begin{EntryWithPhonetic}{姑姑}{gu1gu5}{8,8}{⼥,⼥}[HSK 6]
  \definition[个,位,名]{s.}{tia; tia paterna}
\end{EntryWithPhonetic}

\begin{EntryWithPhonetic}{姑娘}{gu1niang5}{8,10}{⼥,⼥}[HSK 3]
  \definition[位,名,个,些]{s.}{menina; jovem senhora; mulher solteira | filha}
\end{EntryWithPhonetic}

\begin{EntryWithPhonetic}{姑且}{gu1qie3}{8,5}{⼥,⼀}
  \definition{adv.}{provisoriamente; por enquanto; temporariamente; indica temporário}
\end{EntryWithPhonetic}

%%%%%%%%%% 孤 %%%%%%%%%%
\subsection*{孤}\addcontentsline{loh}{figure}{孤 \dpy{gu1}}

\begin{EntryWithPhonetic}{孤}{gu1}{8}{⼦}
  \definition*{s.}{Sobrenome: Gu}
  \definition{adj.}{sozinho; solitário; isolado}
  \definition{pron.}{eu; meu humilde eu (usado por príncipes feudais); título autoproclamado dos príncipes feudais}
  \definition[个,名,位]{s.}{órfão}
\end{EntryWithPhonetic}

\begin{EntryWithPhonetic}{孤单}{gu1dan1}{8,8}{⼦,⼗}[HSK 7-9]
  \definition{adj.}{sozinho; solitário | fraco; inadequado; descreve um pequeno número de pessoas e poder fraco}
\end{EntryWithPhonetic}

\begin{EntryWithPhonetic}{孤独}{gu1du2}{8,9}{⼦,⽝}[HSK 6]
  \definition{adj.}{sozinho; solitário}
\end{EntryWithPhonetic}

\begin{EntryWithPhonetic}{孤儿}{gu1'er2}{8,2}{⼦,⼉}[HSK 6]
  \definition[个,名,位]{s.}{órfão; criança sem pais; crianças que perderam os pais}
\end{EntryWithPhonetic}

\begin{EntryWithPhonetic}{孤立}{gu1li4}{8,5}{⼦,⽴}[HSK 7-9]
  \definition{adj.}{isolado; condenado ao ostracismo; descreve a falta de ajuda e simpatia}
  \definition{v.}{isolar; ostracizar; privar uma pessoa de ajuda, apoio e confiança}
\end{EntryWithPhonetic}

\begin{EntryWithPhonetic}{孤零零}{gu1ling2ling2}{8,13,13}{⼦,⾬,⾬}[HSK 7-9]
  \definition{adj.}{solitário; sozinho; completamente sozinho; sem apoio ou companhia}
\end{EntryWithPhonetic}

\begin{EntryWithPhonetic}{孤陋寡闻}{gu1lou4-gua3wen2}{8,8,14,9}{⼦,⾩,⼧,⾨}[HSK 7-9]
  \definition{expr.}{ignorante e mal informado | ignorante e inexperiente | mal informado e tacanho}
\end{EntryWithPhonetic}

%%%%%%%%%% 沽 %%%%%%%%%%
\subsection*{沽}\addcontentsline{loh}{figure}{沽 \dpy{gu1}}

\begin{EntryWithPhonetic}{沽}{gu1}{8}{⽔}
  \definition*{s.}{Município de Tianjin; outro nome para Tianjin}
  \definition{v.}{comprar | vender}
\end{EntryWithPhonetic}

\begin{EntryWithPhonetic}{沽名钓誉}{gu1ming2-diao4yu4}{8,6,8,13}{⽔,⼝,⾦,⾔}[HSK 7-9]
  \definition{expr.}{``Buscando fama e reputação.''; pescar fama e elogios; tentar alcançar a fama}
\end{EntryWithPhonetic}

%%%%%%%%%% 辜 %%%%%%%%%%
\subsection*{辜}\addcontentsline{loh}{figure}{辜 \dpy{gu1}}

\begin{EntryWithPhonetic}{辜}{gu1}{12}{⾟}
  \definition*{s.}{Sobrenome: Gu}
  \definition{s.}{culpa; crime}
\end{EntryWithPhonetic}

\begin{EntryWithPhonetic}{辜负}{gu1fu4}{12,6}{⾟,⾙}[HSK 7-9]
  \definition{v.}{desapontar; decepcionar; ser indigno de; não corresponder a}
\end{EntryWithPhonetic}

%%%%%%%%%% 古 %%%%%%%%%%
\subsection*{古}\addcontentsline{loh}{figure}{古 \dpy{gu3}}

\begin{EntryWithPhonetic}{古}{gu3}{5}{⼝}[HSK 3]
  \definition*{s.}{Cuba, abreviação de 古巴 | Sobrenome: Gu}
  \definition{adj.}{antigo; milenar; ancestral; secular | simples e sincero | velho | arcaico}
  \definition{pref.}{(distante no tempo; antigo; primitivo) paleo-; arqueo-}
  \definition{s.}{tempos antigos | antiguidade; ancestralidade | livros ou ortodoxias dos sábios antigos, a tradição do Tao | uma forma de poesia pré-Tang}
  \seealsoref{古巴}{gu3ba1}
  \antonymref{今}{jin1}
\end{EntryWithPhonetic}

\begin{EntryWithPhonetic}{古巴}{gu3ba1}{5,4}{⼝,⼰}
  \definition*{s.}{Cuba}
\end{EntryWithPhonetic}

\begin{EntryWithPhonetic}{古板}{gu3ban3}{5,8}{⼝,⽊}
  \definition{adj.}{antiquado e inflexível | reto; intolerante; conservador; lento | (pensamentos e estilo de trabalho) teimoso e conservador; rígido e avesso a mudanças}
  \synonymref{痴呆}{chi1dai1}
  \synonymref{传统}{chuan2tong3}
  \synonymref{固执}{gu4zhi5}
  \synonymref{严肃}{yan2su4}
  \antonymref{风尚}{feng1shang4}
  \antonymref{活泼}{huo2po5}
  \antonymref{机灵}{ji1ling5}
  \antonymref{开通}{kai1tong5}
\end{EntryWithPhonetic}

\begin{EntryWithPhonetic}{古城}{gu3cheng2}{5,9}{⼝,⼟}
  \definition{s.}{cidade antiga}
\end{EntryWithPhonetic}

\begin{EntryWithPhonetic}{古代}{gu3dai4}{5,5}{⼝,⼈}[HSK 3]
  \definition{s.}{tempos antigos; o passado é um período muito distante do presente (diferentemente de 近代 e 现代); na periodização histórica chinesa, geralmente se refere ao período anterior a meados do século XIX | sociedade antiga; sociedade primitiva; refere"-se especificamente à era da sociedade escravista (às vezes também inclui a era comunal primitiva) |antigamente; tempos antigos; no passado}
  \seealsoref{近代}{jin4dai4}
  \seealsoref{现代}{xian4dai4}
\end{EntryWithPhonetic}

\begin{EntryWithPhonetic}{古典}{gu3dian3}{5,8}{⼝,⼋}[HSK 6]
  \definition{adj.}{clássico; descreve uma obra ou coisa como tendo características tradicionais ou exemplares}
  \definition{s.}{os clássicos}
\end{EntryWithPhonetic}

\begin{EntryWithPhonetic}{古董}{gu3dong3}{5,12}{⼝,⾋}[HSK 7-9]
  \definition{adj.}{antiquado; uma metáfora para coisas ultrapassadas ou pessoas teimosas}
  \definition[件,个]{s.}{objeto de arte; raridade; antiguidade; artefatos transmitidos desde os tempos antigos podem ser usados como referência para a compreensão da cultura antiga}
\end{EntryWithPhonetic}

\begin{EntryWithPhonetic}{古怪}{gu3guai4}{5,8}{⼝,⼼}[HSK 7-9]
  \definition{adj.}{pitoresco; excêntrico; esquisito; estranho; muito diferente do habitual, surpreendente; desconhecido e raro; raro e inovador}
\end{EntryWithPhonetic}

\begin{EntryWithPhonetic}{古迹}{gu3ji4}{5,9}{⼝,⾡}[HSK 7-9]
  \definition[处]{s.}{sítio histórico; local de interesse histórico; construções antigas ou outras relíquias de grande importância}
\end{EntryWithPhonetic}

\begin{EntryWithPhonetic}{古今中外}{gu3jin1-zhong1wai4}{5,4,4,5}{⼝,⼈,⼁,⼣}[HSK 7-9]
  \definition{expr.}{``Antigo e moderno, chinês e estrangeiro.''; em todos os tempos e em todas as terras}
\end{EntryWithPhonetic}

\begin{EntryWithPhonetic}{古柯碱}{gu3ke1jian3}{5,9,14}{⼝,⽊,⽯}
  \definition{s.}{cocaína}
\end{EntryWithPhonetic}

\begin{EntryWithPhonetic}{古老}{gu3lao3}{5,6}{⼝,⽼}[HSK 5]
  \definition{adj.}{antigo; antiquado; histórico}
\end{EntryWithPhonetic}

\begin{EntryWithPhonetic}{古朴}{gu3pu3}{5,6}{⼝,⽊}[HSK 7-9]
  \definition{adj.}{simples e pouco sofisticado (arte, arquitetura, etc.); descreve a aparência sem muita decoração ou modificação, dando às pessoas uma sensação antiga, e também descreve o comportamento das pessoas como simples e sincero}
\end{EntryWithPhonetic}

\begin{EntryWithPhonetic}{古人}{gu3ren2}{5,2}{⼝,⼈}[HSK 7-9]
  \definition{s.}{os antigos; antepassados | pessoas dos tempos antigos | espécies humanas extintas, como \emph{Homo erectus} ou \emph{Homo neanderthalensis} | Lliterário: pessoa falecida}
  \antonymref{今人}{jin1ren2}
\end{EntryWithPhonetic}

\begin{EntryWithPhonetic}{古铜色}{gu3tong2 se4}{5,11,6}{⼝,⾦,⾊}
  \definition{s.}{cor bronze}
\end{EntryWithPhonetic}

\begin{EntryWithPhonetic}{古装}{gu3 zhuang1}{5,12}{⼝,⾐}
  \definition[套]{s.}{traje antigo; roupas tradicionais; roupas de estilo antigo}
\end{EntryWithPhonetic}

%%%%%%%%%% 谷 %%%%%%%%%%
\subsection*{谷}\addcontentsline{loh}{figure}{谷 \dpy{gu3}}

\begin{EntryWithPhonetic}{谷}{gu3}{7}{⾕}[Kangxi 150]
  \definition*{s.}{Sobrenome: Gu}
  \definition{adj.}{bom; gentil}
  \definition{s.}{vale; ravina; desfiladeiro; garganta; faixa estreita de terra com uma saída no meio de duas colinas ou dois platôs | arroz não descascado | salário de funcionário (na época feudal) | calha; cocho; canal | fossa sob o cerebelo (anatomia); valécula | dificuldade; dilema}
  \definition{v.}{criar (filhos) | crescer}
\end{EntryWithPhonetic}

%%%%%%%%%% 股 %%%%%%%%%%
\subsection*{股}\addcontentsline{loh}{figure}{股 \dpy{gu3}}

\begin{EntryWithPhonetic}{股}{gu3}{8}{⾁}[HSK 6]
  \definition*{s.}{Sobrenome: Gu}
  \definition{clas.}{usado para coisas em tiras, longas e estreitas | usado para gás, odor, força, etc. | Pejorativo: usado para um grupo de pessoas}
  \definition{s.}{coxa; ancas | seção (de um escritório, empresa, etc.); unidades organizacionais em agências governamentais, empresas e grupos | fio; camada | uma das várias partes iguais de propriedade | ação; \emph{stock}; ação do capital social; uma parte igual de fundos ou propriedade | a perna mais longa de um triângulo retângulo}
\end{EntryWithPhonetic}

\begin{EntryWithPhonetic}{股东}{gu3dong1}{8,5}{⾁,⼀}[HSK 6]
  \definition[个,位,名,家]{s.}{acionista de uma sociedade anônima com direito a participar e votar nas assembleias gerais; refere"-se também a investidores em outras empresas industriais e comerciais administradas por sociedades}
\end{EntryWithPhonetic}

\begin{EntryWithPhonetic}{股份}{gu3fen4}{8,6}{⾁,⼈}[HSK 7-9]
  \definition{s.}{ação; unidade de distribuição de capital de uma sociedade anônima ou de uma empresa cooperativa, com uma parcela igual do capital total}
\end{EntryWithPhonetic}

\begin{EntryWithPhonetic}{股民}{gu3min2}{8,5}{⾁,⽒}[HSK 7-9]
  \definition{s.}{pessoa que compra e vende ações; acionista | corretor de ações | investidor em ações}
\end{EntryWithPhonetic}

\begin{EntryWithPhonetic}{股票}{gu3piao4}{8,11}{⾁,⽰}[HSK 6]
  \definition[只,股]{s.}{ação; quotas; certificado de ações; título de capital; capital social; títulos utilizados para representar ações}
\end{EntryWithPhonetic}

\begin{EntryWithPhonetic}{股市}{gu3shi4}{8,5}{⾁,⼱}[HSK 7-9]
  \definition{s.}{mercado de ações; mercado de compra e venda de ações | cotações na bolsa de valores}
\end{EntryWithPhonetic}

%%%%%%%%%% 骨 %%%%%%%%%%
\subsection*{骨}\addcontentsline{loh}{figure}{骨 \dpy{gu3}}

\begin{EntryWithPhonetic}{骨}{gu3}{9}{⾻}[Kangxi 188]
  \definition*{s.}{Sobrenome: Gu}
  \definition[根,块]{s.}{osso | esqueleto; estrutura | caráter; espírito | cadáver; corpo}
\end{EntryWithPhonetic}

\begin{EntryWithPhonetic}{骨干}{gu3gan4}{9,3}{⾻,⼲}[HSK 7-9]
  \definition[名,个,位]{s.}{diáfise; a parte central de um osso longo, conectada à epífise em ambas as extremidades, contém uma cavidade | espinha dorsal; esteio; metaforicamente falando, uma pessoa ou coisa que desempenha um papel importante}
\end{EntryWithPhonetic}

\begin{EntryWithPhonetic}{骨气}{gu3qi4}{9,4}{⾻,⽓}[HSK 7-9]
  \definition[些,种]{s.}{espinha dorsal; integridade moral; força de caráter | vigor dos traços caligráficos; refere"-se ao impulso forte e vertical expresso pela caligrafia}
\end{EntryWithPhonetic}

\begin{EntryWithPhonetic}{骨头}{gu3tou5}{9,5}{⾻,⼤}[HSK 4]
  \definition[根,块]{s.}{osso; tecidos mais duros no corpo de uma pessoa ou de alguns animais que sustentam o corpo ou protegem os órgãos do corpo | caráter de uma pessoa; refere"-se à qualidade do caráter de uma pessoa}
\end{EntryWithPhonetic}

\begin{EntryWithPhonetic}{骨折}{gu3zhe2}{9,7}{⾻,⼿}[HSK 7-9]
  \definition{v.}{sofrer uma fratura; quebrar (um osso)}
\end{EntryWithPhonetic}

%%%%%%%%%% 鼓 %%%%%%%%%%
\subsection*{鼓}\addcontentsline{loh}{figure}{鼓 \dpy{gu3}}

\begin{EntryWithPhonetic}{鼓}{gu3}{13}{⿎}[HSK 5][Kangxi 207]
  \definition*{s.}{Sobrenome: Gu}
  \definition{adj.}{abaulado; inchado; saliente; protuberante}
  \definition{clas.}{unidades antigas de cronometragem noturna; vigílias da noite}
  \definition[个,架,面,张]{s.}{tambor; instrumento de percussão | coisas semelhantes a tambores; formato, som e função semelhantes aos de um tambor}
  \definition{v.}{soar; bater; golpear; fazer um objeto soar | ventilar; soprar com fole | agitar; despertar; ativar; incitar; revigorar | bater asas | aumentar; fazer beicinho}
\end{EntryWithPhonetic}

\begin{EntryWithPhonetic}{鼓吹}{gu3chui1}{13,7}{⿎,⼝}
  \definition{v.}{defender (um réu) | Pejorativo: pregar; anunciar; exagerar gabar-se}
\end{EntryWithPhonetic}

\begin{EntryWithPhonetic}{鼓动}{gu3dong4}{13,6}{⿎,⼒}[HSK 7-9]
  \definition{v.}{promover; ativar; agitar; despertar; inspirar as pessoas a agir | instigar; incitar}
\end{EntryWithPhonetic}

\begin{EntryWithPhonetic}{鼓励}{gu3li4}{13,7}{⿎,⼒}[HSK 5]
  \definition{v.}{incitar; encorajar; provocar e incentivar}
\end{EntryWithPhonetic}

\begin{EntryWithPhonetic}{鼓舞}{gu3wu3}{13,14}{⿎,⾇}[HSK 7-9]
  \definition{adj.}{animado; elevado; inspirado; encorajado; motivado}
  \definition{v.}{animar; elevar; inspirar; encorajar; motivar}
\end{EntryWithPhonetic}

\begin{EntryWithPhonetic}{鼓掌}{gu3/zhang3}{13,12}{⿎,⼿}[HSK 5]
  \definition{v.+compl.}{aplaudir; bater palmas, principalmente para expressar felicidade, aprovação ou boas-vindas}
\end{EntryWithPhonetic}

%%%%%%%%%% 估 %%%%%%%%%%
\subsection*{估}\addcontentsline{loh}{figure}{估 \dpy{gu4}}

\begin{EntryWithPhonetic}{估}{gu4}{7}{⼈}
  \definition{adj.}{velho | roupas de segunda mão}
  \seeref{gu1}
\end{EntryWithPhonetic}

%%%%%%%%%% 固 %%%%%%%%%%
\subsection*{固}\addcontentsline{loh}{figure}{固 \dpy{gu4}}

\begin{EntryWithPhonetic}{固}{gu4}{8}{⼞}
  \definition*{s.}{Sobrenome: Gu}
  \definition{adj.}{sólido; firme; forte | duro; sólido | mal informado; superficial; ignorante}
  \definition{adv.}{firmemente; resolutamente | originalmente; em primeiro lugar | certamente; reconhecidamente; seguramente}
  \definition{conj.}{usado da mesma forma que 固然}
  \definition{v.}{solidificar; consolidar; fortalecer | defender; proteger}
  \seealsoref{固然}{gu4ran2}
\end{EntryWithPhonetic}

\begin{EntryWithPhonetic}{固定}{gu4ding4}{8,8}{⼞,⼧}[HSK 4]
  \definition{adj.}{fixo; regular; inalterado ou imóvel}
  \definition{v.}{consertar; tornar fixo, não mover novamente; colocar as coisas em ordem, não mudá-las novamente}
\end{EntryWithPhonetic}

\begin{EntryWithPhonetic}{固然}{gu4ran2}{8,12}{⼞,⽕}[HSK 7-9]
  \definition{conj.}{usado para introduzir uma cláusula adversativa admitindo primeiro um certo fato; quando usado na primeira metade de uma frase, a segunda metade geralmente tem 可是 ou 但是 para ecoá-lo, indicando que o fato A é reconhecido, mas o fato B não se torna inválido por causa do fato A | admitir um fato sem negar outro; indica o reconhecimento de um fato, levando a uma transição no texto seguinte; indica o reconhecimento do fato A e não nega o fato B}
  \seealsoref{但是}{dan4shi4}
  \seealsoref{可是}{ke3shi4}
\end{EntryWithPhonetic}

\begin{EntryWithPhonetic}{固执}{gu4zhi5}{8,6}{⼞,⼿}[HSK 7-9]
  \definition{adj.}{obstinado; teimoso; mantém suas próprias opiniões e não quer mudá-las, mesmo que estejam erradas}
\end{EntryWithPhonetic}

%%%%%%%%%% 故 %%%%%%%%%%
\subsection*{故}\addcontentsline{loh}{figure}{故 \dpy{gu4}}

\begin{EntryWithPhonetic}{故}{gu4}{9}{⽁}[HSK 7-9]
  \definition*{s.}{Sobrenome: Gu}
  \definition{adj.}{velho; antigo; original}
  \definition{adv.}{propositalmente; intencionalmente; deliberadamente}
  \definition{conj.}{assim; portanto; consequentemente; pelo contrário}
  \definition{s.}{evento; incidente; acontecimento; acidente | causa; razão | amigo; conhecido | o velho; refere"-se a coisas antigas e passadas}
  \definition{v.}{morrer}
\end{EntryWithPhonetic}

\begin{EntryWithPhonetic}{故宫}{gu4gong1}{9,9}{⽁,⼧}
  \definition*{s.}{O Palácio Imperial; O Museu do Palácio (em Pequim); A Cidade Proibida}
\end{EntryWithPhonetic}

\begin{EntryWithPhonetic}{故事}{gu4shi5}{9,8}{⽁,⼅}[HSK 2]
  \definition[个,段,篇,则]{s.}{história; conto; coisas reais ou fictícias usadas como objeto de narrativa, com coerência, atraentes e capazes de emocionar as pessoas | enredo; trama; enredo que consegue mostrar a personalidade dos personagens e refletir a ideia central da obra literária}
\end{EntryWithPhonetic}

\begin{EntryWithPhonetic}{故乡}{gu4xiang1}{9,3}{⽁,⼄}[HSK 3]
  \definition[个]{s.}{cidade natal; terra natal; local de nascimento ou onde viveu por muito tempo}
\end{EntryWithPhonetic}

\begin{EntryWithPhonetic}{故意}{gu4yi4}{9,13}{⽁,⼼}[HSK 2]
  \definition{adv.}{deliberadamente; intencionalmente; não é por descuido, mas sim conscientemente (geralmente coisas que não se devem fazer ou que não são necessárias)}
  \definition{s.}{intenção; um tipo de mentalidade, uma pessoa sabe claramente que seus atos podem causar danos a outras pessoas ou trazer consequências negativas para a sociedade, mas mesmo assim não faz nada para impedir isso}
\end{EntryWithPhonetic}

\begin{EntryWithPhonetic}{故障}{gu4zhang4}{9,13}{⽁,⾩}[HSK 6]
  \definition[出]{s.}{problema; falha; parada; mau funcionamento; avaria; situações em que máquinas, instrumentos, etc. não podem funcionar normalmente devido a problemas}
\end{EntryWithPhonetic}

%%%%%%%%%% 顾 %%%%%%%%%%
\subsection*{顾}\addcontentsline{loh}{figure}{顾 \dpy{gu4}}

\begin{EntryWithPhonetic}{顾}{gu4}{10}{⾴}[HSK 6]
  \definition*{s.}{Sobrenome: Gu}
  \definition{adv.}{em vez disso; pelo contrário; indica o oposto, equivalente a 却 ou 反而}
  \definition{conj.}{mas; no entanto}
  \definition{v.}{olhar para trás; olhar para; virar-se e olhar para | cuidar de; atender a; levar em conta ou consideração | visitar; chamar | sentir pena de}
  \seealsoref{反而}{fan3'er2}
  \seealsoref{却}{que4}
\end{EntryWithPhonetic}

\begin{EntryWithPhonetic}{顾不得}{gu4bu5de5}{10,4,11}{⾴,⼀,⼻}[HSK 7-9]
  \definition{v.}{incapaz de mudar algo | incapaz de lidar com}
\end{EntryWithPhonetic}

\begin{EntryWithPhonetic}{顾不上}{gu4bu5shang4}{10,4,3}{⾴,⼀,⼀}[HSK 7-9]
  \definition{v.}{não conseguir; não conseguir atender; incapaz de cuidar de (fazer algo)}
\end{EntryWithPhonetic}

\begin{EntryWithPhonetic}{顾及}{gu4ji2}{10,3}{⾴,⼃}[HSK 7-9]
  \definition{v.}{atender a; levar em conta; dar consideração a; cuidar de; notar}
\end{EntryWithPhonetic}

\begin{EntryWithPhonetic}{顾客}{gu4ke4}{10,9}{⾴,⼧}[HSK 2]
  \definition[个,位,名,些]{s.}{cliente; comprador; consumidor; paciente}
\end{EntryWithPhonetic}

\begin{EntryWithPhonetic}{顾虑}{gu4lv4}{10,10}{⾴,⾌}[HSK 7-9]
  \definition[丝,点]{s.}{preocupação; escrúpulo; receio; apreensão}
  \definition{v.}{estar apreensivo (sobre as consequências da própria ação)}
\end{EntryWithPhonetic}

\begin{EntryWithPhonetic}{顾全大局}{gu4quan2-da4ju2}{10,6,3,7}{⾴,⼊,⼤,⼫}[HSK 7-9]
  \definition{expr.}{``Considere a situação geral.''; levar em conta os interesses do todo; considerar a situação como um todo; levar em consideração o panorama geral; trabalhar para o benefício de todos}
\end{EntryWithPhonetic}

\begin{EntryWithPhonetic}{顾问}{gu4wen4}{10,6}{⾴,⾨}[HSK 5]
  \definition[个,位,名]{s.}{conselheiro; consultor; assessor; pessoas com conhecimento especializado ou experiência contratadas para prestar consultoria a organizações ou indivíduos}
\end{EntryWithPhonetic}

%%%%%%%%%% 雇 %%%%%%%%%%
\subsection*{雇}\addcontentsline{loh}{figure}{雇 \dpy{gu4}}

\begin{EntryWithPhonetic}{雇}{gu4}{12}{⾫}[HSK 7-9]
  \definition{v.}{contratar; empregar; pagar pessoas para fazerem coisas por você | contratar (transporte de aluguel)}
\end{EntryWithPhonetic}

\begin{EntryWithPhonetic}{雇佣}{gu4yong1}{12,7}{⾫,⼈}[HSK 7-9]
  \definition{v.}{contratar; empregar; comprar mão de obra com dinheiro}
\end{EntryWithPhonetic}

\begin{EntryWithPhonetic}{雇员}{gu4yuan2}{12,7}{⾫,⼝}[HSK 7-9]
  \definition[名,位,个]{s.}{empregado; servo; pessoal contratado ou temporário fora do estabelecimento}
\end{EntryWithPhonetic}

\begin{EntryWithPhonetic}{雇主}{gu4zhu3}{12,5}{⾫,⼂}[HSK 7-9]
  \definition[名]{s.}{empregador; uma pessoa que contrata trabalhadores, veículos ou barcos}
\end{EntryWithPhonetic}

%%%%%%%%%% 瓜 %%%%%%%%%%
\subsection*{瓜}\addcontentsline{loh}{figure}{瓜 \dpy{gua1}}

\begin{EntryWithPhonetic}{瓜}{gua1}{5}{⽠}[HSK 4][Kangxi 97]
  \definition*{s.}{Sobrenome: Gua}
  \definition[个]{s.}{qualquer tipo de melão ou cabaça | companheiro (termo depreciativo para uma pessoa)}
  \definition{v.}{fofocar}
\end{EntryWithPhonetic}

\begin{EntryWithPhonetic}{瓜分}{gua1fen1}{5,4}{⽠,⼑}[HSK 7-9]
  \definition{v.}{cortar um melão -- cortar; desmembrar; dividir; particionar}
\end{EntryWithPhonetic}

\begin{EntryWithPhonetic}{瓜子}{gua1zi3}{5,3}{⽠,⼦}[HSK 7-9]
  \definition[个,把,颗,粒,些]{s.}{sementes de melão; sementes de girassol; sementes de abóbora}
\end{EntryWithPhonetic}

%%%%%%%%%% 刮 %%%%%%%%%%
\subsection*{刮}\addcontentsline{loh}{figure}{刮 \dpy{gua1}}

\begin{EntryWithPhonetic}{刮}{gua1}{8}{⼑}[HSK 6]
  \definition{v.}{barbear; raspar; depilar | untar com (pasta, etc.)  | extorquir; pilhar; adquirir avidamente (propriedade) por vários meios | (do vento) soprar}
\end{EntryWithPhonetic}

\begin{EntryWithPhonetic}{刮风}{gua1/feng1}{8,4}{⼑,⾵}[HSK 7-9]
  \definition{v.+compl.}{ventar; fazer vento; soprar (vento)}
\end{EntryWithPhonetic}

%%%%%%%%%% 寡 %%%%%%%%%%
\subsection*{寡}\addcontentsline{loh}{figure}{寡 \dpy{gua3}}

\begin{EntryWithPhonetic}{寡}{gua3}{14}{⼧}
  \definition{adj.}{poucos; escassos | insípido; sem sabor | pouco; escasso | insípido; sem graça}
  \definition{pron.}{eu; título autoproclamado de um antigo monarca}
  \definition{s.}{viúva | viuvez; a natureza ou estado de uma mulher viúva que vive sozinha}
  \antonymref{多}{duo1}
  \antonymref{众}{zhong4}
\end{EntryWithPhonetic}

\begin{EntryWithPhonetic}{寡妇}{gua3fu5}{14,6}{⼧,⼥}[HSK 7-9]
  \definition[个]{s.}{viúva; uma mulher cujo marido morreu}
\end{EntryWithPhonetic}

%%%%%%%%%% 挂 %%%%%%%%%%
\subsection*{挂}\addcontentsline{loh}{figure}{挂 \dpy{gua4}}

\begin{EntryWithPhonetic}{挂}{gua4}{9}{⼿}[HSK 3]
  \definition{clas.}{usado principalmente para coisas que vêm em conjuntos ou séries}
  \definition{v.}{pendurar; colocar; suspender; usando cordas, ganchos, pregos e outros itens para prender objetos em um ou mais pontos específicos | interromper chamada (telefônica) | colocar alguém em contato com; ligar; telefonar; refere"-se a ligar o telefone, bem como a fazer uma chamada | falhar; fracassar | colocar em registro; registrar | pegar carona; ser pego | preocupar"-se com | ser revestido com; ser coberto com | estar pendente; deixar algo sem solução}
\end{EntryWithPhonetic}

\begin{EntryWithPhonetic}{挂钩}{gua4gou1}{9,9}{⼿,⾦}[HSK 7-9]
  \definition[个,种]{s.}{(vagões ferroviários) acoplamento; manilha; engate | gancho}
  \definition{v.}{acoplar (dois vagões ferroviários); articular | conectar-se com; estabelecer contato com; entrar em contato com; vincular-se a}
\end{EntryWithPhonetic}

\begin{EntryWithPhonetic}{挂号}{gua4/hao4}{9,5}{⼿,⼝}[HSK 7-9]
  \definition{v.+compl.}{registrar-se (em um hospital, etc.) | enviar através de carta registrada}
\end{EntryWithPhonetic}

\begin{EntryWithPhonetic}{挂号信}{gua4hao4xin4}{9,5,9}{⼿,⼝,⼈}
  \definition{s.}{carta registrada}
\end{EntryWithPhonetic}

\begin{EntryWithPhonetic}{挂念}{gua4nian4}{9,8}{⼿,⼼}[HSK 7-9]
  \definition{v.}{sentir falta; preocupar-se com alguém que está ausente}
\end{EntryWithPhonetic}

\begin{EntryWithPhonetic}{挂失}{gua4/shi1}{9,5}{⼿,⼤}[HSK 7-9]
  \definition{v.+compl.}{relatar a perda de algo; se perder uma nota ou certificado, você deve registrá-lo junto à autoridade emissora ou declará-lo inválido}
\end{EntryWithPhonetic}

%%%%%%%%%% 乖 %%%%%%%%%%
\subsection*{乖}\addcontentsline{loh}{figure}{乖 \dpy{guai1}}

\begin{EntryWithPhonetic}{乖}{guai1}{8}{⼃}[HSK 7-9]
  \definition{adj.}{(uma criança) bem comportado; bom; obediente | inteligente; astuto; esperto | (caráter, comportamento, etc.) estranho; anormal; irracional}
  \definition{v.}{perverter; ser contrário à razão; ir contra | (caráter, comportamento, etc.) ser anormal; ser estranho}
\end{EntryWithPhonetic}

\begin{EntryWithPhonetic}{乖乖}{guai1guai1}{8,8}{⼃,⼃}
  \definition{adj.}{bem-comportado; obediente}
  \definition{s.}{bebezinho; pequenino; querido; docinho (usado apenas para crianças)}
  \seeref{guai1guai5}
\end{EntryWithPhonetic}

\begin{EntryWithPhonetic}{乖乖}{guai1guai5}{8,8}{⼃,⼃}
  \definition{expr.}{Uau!; Nossa!; Meu Deus!; Oh meu Deus!}
  \seeref{guai1guai1}
\end{EntryWithPhonetic}

\begin{EntryWithPhonetic}{乖巧}{guai1qiao3}{8,5}{⼃,⼯}[HSK 7-9]
  \definition{adj.}{fofo; adorável; agradável; descreve crianças, pequenos animais, etc. como sendo obedientes, fofos e simpáticos | inteligente; engenhoso; descreve uma pessoa que sempre fala ou faz coisas de acordo com os desejos de outras pessoas e é querida por elas}
\end{EntryWithPhonetic}

\begin{EntryWithPhonetic}{乖张}{guai1zhang1}{8,7}{⼃,⼸}
  \definition{adj.}{excêntrico e irracional; perverso; recalcitrante | não suave; sem sucesso | irritadiço | irrazoável}
  \synonymref{荒诞}{huang1dan4}
  \synonymref{荒谬}{huang1miu4}
  \synonymref{荒唐}{huang1tang2}
  \antonymref{随和}{sui2he5}
  \antonymref{温和}{wen1he2}
  \antonymref{温柔}{wen1rou2}
  \antonymref{温顺}{wen1shun4}
\end{EntryWithPhonetic}

%%%%%%%%%% 拐 %%%%%%%%%%
\subsection*{拐}\addcontentsline{loh}{figure}{拐 \dpy{guai3}}

\begin{EntryWithPhonetic}{拐}{guai3}{8}{⼿}[HSK 6]
  \definition[支,根,副]{s.}{muleta; bengala; uma bengala com uma barra horizontal na parte superior, usada por pessoas com doenças ou deficiências nos membros inferiores para ajudá-las a caminhar |  sete; forma falada do numeral 七 | esquina; curva; canto}
  \definition{v.}{virar; girar; mudar de direção enquanto se move | enganar | mudar; transformar | mancar}
  \seealsoref{七}{qi1}
\end{EntryWithPhonetic}

\begin{EntryWithPhonetic}{拐弯}{guai3/wan1}{8,9}{⼿,⼸}[HSK 7-9]
  \definition[个]{s.}{esquina; curva; canto}
  \definition{v.}{virar; virar uma esquina; indica mudança de direção da viagem | dar meia-volta; seguir um novo curso; indica mudança de ideias, linguagem, etc.}
\end{EntryWithPhonetic}

\begin{EntryWithPhonetic}{拐杖}{guai3zhang4}{8,7}{⼿,⽊}[HSK 7-9]
  \definition[个,根,支,副]{s.}{muleta; bengala}
\end{EntryWithPhonetic}

%%%%%%%%%% 怪 %%%%%%%%%%
\subsection*{怪}\addcontentsline{loh}{figure}{怪 \dpy{guai4}}

\begin{EntryWithPhonetic}{怪}{guai4}{8}{⼼}[HSK 4,5]
  \definition*{s.}{Sobrenome: Guai}
  \definition{adj.}{estranho; esquisito; desconcertante | peculiar; excêntrico; pitoresco; monstruoso; anormal; incomum}
  \definition{adv.}{bastante; muito}
  \definition{s.}{monstro; demônio | diabo; ser maligno}
  \definition{v.}{culpar | achar algo estranho; maravilhar-se com; ficar surpreso | repreender; culpar; reclamar}
\end{EntryWithPhonetic}

\begin{EntryWithPhonetic}{怪不得}{guai4bu5de5}{8,4,11}{⼼,⼀,⼻}[HSK 7-9]
  \definition{adv.}{não é de admirar; então é por isso; isso explica por que; isso significa que você entende o motivo e não acha mais uma situação estranha}
  \definition{v.}{não culpar; não acusar; não poder culpar, não se ofender}[你做错了,怪不得别人。===Você cometeu um erro, então não culpe os outros.]
\end{EntryWithPhonetic}

\begin{EntryWithPhonetic}{怪癖}{guai4pi3}{8,18}{⼼,⽧}
  \definition{adj.}{peculiar}
  \definition{s.}{excentricidade | peculiaridade | hobby estranho}
\end{EntryWithPhonetic}

\begin{EntryWithPhonetic}{怪兽}{guai4shou4}{8,11}{⼼,⼋}
  \definition{s.}{animal raro | animal mítico | monstro}
\end{EntryWithPhonetic}

\begin{EntryWithPhonetic}{怪物}{guai4wu5}{8,8}{⼼,⽜}[HSK 7-9]
  \definition{s.}{monstro; aberração; coisas imaginárias que parecem estranhas, mas têm habilidades especiais | pessoa excêntrica; pássaro estranho; uma pessoa com temperamento excêntrico}
\end{EntryWithPhonetic}

\begin{EntryWithPhonetic}{怪异}{guai4yi4}{8,6}{⼼,⼶}[HSK 7-9]
  \definition{adj.}{monstruoso; estranho; incomum}
  \definition{s.}{fenômeno estranho; presságio; prodígio | monstruosidade}
\end{EntryWithPhonetic}

%%%%%%%%%% 关 %%%%%%%%%%
\subsection*{关}\addcontentsline{loh}{figure}{关 \dpy{guan1}}

\begin{EntryWithPhonetic}{关}{guan1}{6}{⼋}[HSK 1,4]
  \definition*{s.}{Sobrenome: Guan}
  \definition{s.}{passagem; ponto de controle | alfândega; escritórios de cobrança de impostos para exportação e importação de mercadorias | ponto de inflexão ou barreira; ponto de virada ou dificuldade | momento crítico; mecanismo}
  \definition{v.}{fechar; encerrar; amarrar algo | fechar; trancar | encerrar; sair do mercado; falir | conceder ou sacar o pagamento de um salário | desligar | envolver; preocupar-se; conectar-se}
\end{EntryWithPhonetic}

\begin{EntryWithPhonetic}{关爱}{guan1'ai4}{6,10}{⼋,⽖}[HSK 6]
  \definition{v.}{cuidar; cuidar e amar}
\end{EntryWithPhonetic}

\begin{EntryWithPhonetic}{关闭}{guan1bi4}{6,6}{⼋,⾨}[HSK 4]
  \definition{v.}{fechar | (empresa) falir}
\end{EntryWithPhonetic}

\begin{EntryWithPhonetic}{关掉}{guan1diao4}{6,11}{⼋,⼿}[HSK 7-9]
  \definition{v.}{desligar}
\end{EntryWithPhonetic}

\begin{EntryWithPhonetic}{关怀}{guan1huai2}{6,7}{⼋,⼼}[HSK 5]
  \definition{v.}{mostrar cuidado amoroso por; mostrar solicitude por; cuidar, amar, apoiar ou ajudar os fracos ou grupos em dificuldade | geralmente usado para superiores para subordinados, anciãos para juniores ou organizações para indivíduos}
\end{EntryWithPhonetic}

\begin{EntryWithPhonetic}{关机}{guan1 ji1}{6,6}{⼋,⽊}[HSK 2]
  \definition{v.}{encerrar; terminar; refere"-se especificamente à conclusão das filmagens de um filme ou série de TV | desligar; desligar a fonte de alimentação; parar o funcionamento da máquina}
\end{EntryWithPhonetic}

\begin{EntryWithPhonetic}{关键}{guan1jian4}{6,13}{⼋,⾦}[HSK 5]
  \definition{adj.}{crucial; decisivo; importante; que pode determinar o curso e o resultado dos eventos}
  \definition[个,点,些]{s.}{chave; ponto crucial; aspectos ou condições mais importantes que determinam o desenvolvimento e o resultado de algo}
\end{EntryWithPhonetic}

\begin{EntryWithPhonetic}{关节}{guan1jie2}{6,5}{⼋,⾋}[HSK 7-9]
  \definition{s.}{articulação; as partes onde os ossos se conectam e que possibilitam o movimento | suborno; relacionamentos que podem ajudar as pessoas a obter benefícios por meios impróprios | elo (ou ponto) chave (ou crucial)}
\end{EntryWithPhonetic}

\begin{EntryWithPhonetic}{关联}{guan1lian2}{6,12}{⼋,⽿}[HSK 6]
  \definition{s.}{conexão; inter-relação; a conexão entre as coisas}
  \definition{v.}{estar relacionado; estar conectado; as coisas estão envolvidas e influenciam umas às outras}
\end{EntryWithPhonetic}

\begin{EntryWithPhonetic}{关上}{guan1shang4}{6,3}{⼋,⼀}[HSK 1]
  \definition{v.}{fechar (uma porta); fechar um objeto | desligar (luz, equipamento elétrico etc.); parar ou encerrar (uma atividade, situação, etc.)}
\end{EntryWithPhonetic}

\begin{EntryWithPhonetic}{关税}{guan1shui4}{6,12}{⼋,⽲}[HSK 7-9]
  \definition{s.}{tarifa; taxa aduaneira; impostos cobrados pelo estado sobre mercadorias importadas e exportadas}
\end{EntryWithPhonetic}

\begin{EntryWithPhonetic}{关头}{guan1tou2}{6,5}{⼋,⼤}[HSK 7-9]
  \definition{s.}{conjuntura; momento; um momento decisivo ou ponto de virada}
\end{EntryWithPhonetic}

\begin{EntryWithPhonetic}{关系}{guan1xi5}{6,7}{⼋,⽷}[HSK 3]
  \definition[个,种]{s.}{relações; conexões; relacionamento; a interligação entre pessoas ou coisas | consequência; impacto; significado a influência ou importância de algo; algo digno de nota (geralmente usado com 没有, 有). | causa; razão (geralmente usado com 由于 ou 因为); refere"-se genericamente a causas, condições, etc. | credenciais que mostram filiação a uma organização; documento que comprova a existência de algum tipo de relação organizacional}
  \definition{v.}{preocupar; afetar; ter influência sobre; ter a ver com}
  \seealsoref{没有}{mei2you5}
  \seealsoref{因为}{yin1wei5}
  \seealsoref{由于}{you2yu2}
  \seealsoref{有}{you3}
\end{EntryWithPhonetic}

\begin{EntryWithPhonetic}{关心}{guan1xin1}{6,4}{⼋,⼼}[HSK 2]
  \definition{v.}{cuidar; preocupar-se com; manifestar interesse por; demonstrar solicitude por; (colocar uma pessoa ou coisa) sempre no coração; valorizar e cuidar}
\end{EntryWithPhonetic}

\begin{EntryWithPhonetic}{关于}{guan1yu2}{6,3}{⼋,⼆}[HSK 4]
  \definition{prep.}{sobre; relativo a; pertencente a; uma questão de; com relação a}
\end{EntryWithPhonetic}

\begin{EntryWithPhonetic}{关张}{guan1zhang1}{6,7}{⼋,⼸}
  \definition{v.}{Dialeto: (uma loja) fechar as portas; falir}
\end{EntryWithPhonetic}

\begin{EntryWithPhonetic}{关照}{guan1zhao4}{6,13}{⼋,⽕}[HSK 7-9]
  \definition{v.}{cuidar de; ficar de olho em; preocupar-se e cuidar de alguém e tomar a iniciativa de ajudar quando perceber que essa pessoa está com problemas | contar; notificar de boca em boca; notificação verbal para que as pessoas saibam ou se lembrem de algo}
\end{EntryWithPhonetic}

\begin{EntryWithPhonetic}{关注}{guan1zhu4}{6,8}{⼋,⽔}[HSK 3]
  \definition{v.}{prestar atenção em; seguir algo de perto; seguir (nas redes sociais)}
\end{EntryWithPhonetic}

%%%%%%%%%% 观 %%%%%%%%%%
\subsection*{观}\addcontentsline{loh}{figure}{观 \dpy{guan1}}

\begin{EntryWithPhonetic}{观}{guan1}{6}{⾒}
  \definition*{s.}{Templo taoísta; ``Koon''}
  \definition{s.}{visão; vista | perspectiva; visão; conceito | aparência; perspectiva | alcance de visão | noção; ideia; conhecimento ou visão das coisas | ponto de vista; postura; uma visão de uma coisa}
  \definition{v.}{olhar para; assistir; observar | contemplar}
  \seeref{guan4}
\end{EntryWithPhonetic}

\begin{EntryWithPhonetic}{观测}{guan1ce4}{6,9}{⾒,⽔}[HSK 7-9]
  \definition{v.}{pesquisar; observar e medir; observar e medir (astronomia, geografia, clima, direção, etc.) | observar; assistir e analisar; observar e medir (situação)}
\end{EntryWithPhonetic}

\begin{EntryWithPhonetic}{观察}{guan1cha2}{6,14}{⾒,⼧}[HSK 3]
  \definition{v.}{assistir; pesquisar; observar; examinar cuidadosamente coisas ou fenômenos}
\end{EntryWithPhonetic}

\begin{EntryWithPhonetic}{观点}{guan1dian3}{6,9}{⾒,⽕}[HSK 2]
  \definition[个,种]{s.}{ponto de vista; perspectiva; a visão ou atitude que se tem sobre algo a partir de uma determinada posição ou perspectiva | ponto de vista; perspectiva; a posição ou perspectiva adotada ao analisar uma questão}
\end{EntryWithPhonetic}

\begin{EntryWithPhonetic}{观感}{guan1gan3}{6,13}{⾒,⼼}[HSK 7-9]
  \definition{s.}{impressões; observações | impressões de alguém}
\end{EntryWithPhonetic}

\begin{EntryWithPhonetic}{观光}{guan1guang1}{6,6}{⾒,⼉}[HSK 6]
  \definition{v.}{visitar; passear; fazer turismo; fazer um passeio em um país ou lugar estrangeiro}
\end{EntryWithPhonetic}

\begin{EntryWithPhonetic}{观看}{guan1kan4}{6,9}{⾒,⽬}[HSK 3]
  \definition{v.}{assistir; ver propositadamente; observar}
\end{EntryWithPhonetic}

\begin{EntryWithPhonetic}{观摩}{guan1mo2}{6,15}{⾒,⼿}[HSK 7-9]
  \definition{v.}{inspecionar e aprender com o trabalho uns dos outros; visualizar e emular; observar, refere"-se principalmente a observar as conquistas uns dos outros, trocar experiências e aprender uns com os outros}
\end{EntryWithPhonetic}

\begin{EntryWithPhonetic}{观念}{guan1nian4}{6,8}{⾒,⼼}[HSK 3]
  \definition[种,个]{s.}{ideia; conceito; consciência ideológica}
\end{EntryWithPhonetic}

\begin{EntryWithPhonetic}{观赏}{guan1shang3}{6,12}{⾒,⾙}[HSK 7-9]
  \definition{v.}{ver e admirar; apreciar a vista de; assistir e aproveitar}
\end{EntryWithPhonetic}

\begin{EntryWithPhonetic}{观望}{guan1wang4}{6,11}{⾒,⽉}[HSK 7-9]
  \definition{v.}{esperar para ver; observar (de lado) | olhar ao redor}
\end{EntryWithPhonetic}

\begin{EntryWithPhonetic}{观众}{guan1zhong4}{6,6}{⾒,⼈}[HSK 3]
  \definition[位,名,批,个]{s.}{espectador; público; audiência; pessoas que assistem a espetáculos ou competições}
\end{EntryWithPhonetic}

%%%%%%%%%% 官 %%%%%%%%%%
\subsection*{官}\addcontentsline{loh}{figure}{官 \dpy{guan1}}

\begin{EntryWithPhonetic}{官}{guan1}{8}{⼧}[HSK 4]
  \definition*{s.}{Sobrenome: Guan}
  \definition{adj.}{propriedade do governo; pertencente ao governo ou ao público | público}
  \definition[个,位,名,些]{s.}{funcionário do governo; oficial; servidor público; titular de cargo; funcionário público nomeado acima de um determinado nível | órgão (parte do tecido do corpo)}
\end{EntryWithPhonetic}

\begin{EntryWithPhonetic}{官兵}{guan1bing1}{8,7}{⼧,⼋}[HSK 7-9]
  \definition{s.}{oficiais e soldados | Obsoleto: tropas governamentais}
\end{EntryWithPhonetic}

\begin{EntryWithPhonetic}{官方}{guan1fang1}{8,4}{⼧,⽅}[HSK 4]
  \definition{s.}{autoridade; (do ou pelo) governo | oficial (de uma organização ou instituição)}
\end{EntryWithPhonetic}

\begin{EntryWithPhonetic}{官桂}{guan1gui4}{8,10}{⼧,⽊}
  \definition{s.}{canela; também escrito como 肉桂}
  \seealsoref{肉桂}{rou4gui4}
\end{EntryWithPhonetic}

\begin{EntryWithPhonetic}{官吏}{guan1li4}{8,6}{⼧,⼝}[HSK 7-9]
  \definition{s.}{funcionários do governo | burocrata | oficial}
\end{EntryWithPhonetic}

\begin{EntryWithPhonetic}{官僚}{guan1liao2}{8,14}{⼧,⼈}[HSK 7-9]
  \definition{s.}{burocrata | burocracia | oficial}
\end{EntryWithPhonetic}

\begin{EntryWithPhonetic}{官僚主义}{guan1liao2 zhu3yi4}{8,14,5,3}{⼧,⼈,⼂,⼂}[HSK 7-9]
  \definition{s.}{burocracia; burocratismo}
\end{EntryWithPhonetic}

\begin{EntryWithPhonetic}{官司}{guan1si5}{8,5}{⼧,⼝}[HSK 6]
  \definition[场,个]{s.}{ação judicial}
\end{EntryWithPhonetic}

\begin{EntryWithPhonetic}{官员}{guan1yuan2}{8,7}{⼧,⼝}[HSK 7-9]
  \definition[名,位]{s.}{oficial; funcionários do governo de um determinado nível}
\end{EntryWithPhonetic}

%%%%%%%%%% 冠 %%%%%%%%%%
\subsection*{冠}\addcontentsline{loh}{figure}{冠 \dpy{guan1}}

\begin{EntryWithPhonetic}{冠}{guan1}{9}{⼍}
  \definition{s.}{chapéu | corona; coroa; copa | crista}
  \seeref{guan4}
\end{EntryWithPhonetic}

%%%%%%%%%% 棺 %%%%%%%%%%
\subsection*{棺}\addcontentsline{loh}{figure}{棺 \dpy{guan1}}

\begin{EntryWithPhonetic}{棺}{guan1}{12}{⽊}
  \definition[副]{s.}{caixão; esquife; ataúde}
\end{EntryWithPhonetic}

\begin{EntryWithPhonetic}{棺材}{guan1cai5}{12,7}{⽊,⽊}[HSK 7-9]
  \definition[具,口]{s.}{caixão; esquife; ataúde; féretro; urna funerária usada para enterrar os mortos, geralmente feito de madeira}
\end{EntryWithPhonetic}

%%%%%%%%%% 管 %%%%%%%%%%
\subsection*{管}\addcontentsline{loh}{figure}{管 \dpy{guan3}}

\begin{EntryWithPhonetic}{管}{guan3}{14}{⽵}[HSK 3]
  \definition*{s.}{Guan, um estado da dinastia Zhou | Sobrenome: Guan}
  \definition{adj.}{estreito; restrito; limitado; pequeno}
  \definition{clas.}{usado para objetos cilíndricos longos e finos}
  \definition{conj.}{não importa (quem, o quê, como, etc.)}
  \definition{prep.}{função semelhante a 把, usada especificamente em conjunto com 叫}
  \definition[根,条,排]{s.}{cano; tubo | instrumento musical de sopro | válvula; tubo | duto; canal; vasos}
  \definition{v.}{administrar; dirigir; controlar; cuidar; ser responsável por | ter jurisdição sobre; administrar | disciplinar (crianças ou alunos) | preocupar-se com; importar-se com; incomodar-se com; intervir | fornecer; garantir | supervisionar | governar | submeter alguém a disciplina | assumir; arcar com | incomodar; interferir | assegurar; garantir}
  \seealsoref{把}{ba3}
  \seealsoref{叫}{jiao4}
\end{EntryWithPhonetic}

\begin{EntryWithPhonetic}{管道}{guan3dao4}{14,12}{⽵,⾡}[HSK 6]
  \definition[根,千米,公里]{s.}{oleoduto; canal; túnel; tubulação; um tubo feito de metal ou outro material usado para transportar ou descarregar fluidos (como vapor, gás, óleo, água, etc.) | caminho; canal; abordagem}
\end{EntryWithPhonetic}

\begin{EntryWithPhonetic}{管家}{guan3jia1}{14,10}{⽵,⼧}[HSK 7-9]
  \definition[个]{s.}{mordomo; antigamente, referia"-se a alguém que administrava os negócios de uma família rica | governanta; alguém que gerencia as tarefas domésticas | gerente; governanta; uma pessoa que administra bens ou negócios familiares ou coletivos}
  \definition{v.}{administrar uma casa}
\end{EntryWithPhonetic}

\begin{EntryWithPhonetic}{管……叫……}{guan3 jiao4}{14,5}{⽵,⼝}
  \definition{expr.}{chamar alguém (ou algo) de alguém (ou algo)}
\end{EntryWithPhonetic}

\begin{EntryWithPhonetic}{管教}{guan3jiao4}{14,11}{⽵,⽁}[HSK 7-9]
  \definition{adv.}{Dialeto: certamente; seguramente}
  \definition{v.}{corrigir; disciplinar alguém júnior | responsabilizar"-se por | ensinar}
\end{EntryWithPhonetic}

\begin{EntryWithPhonetic}{管理}{guan3li3}{14,11}{⽵,⽟}[HSK 3]
  \definition{v.}{gerenciar; executar; administrar; governar; estar encarregado de; responsável por garantir o bom andamento de uma determinada tarefa | controlar; gerenciar; fazer com que pessoas e animais obedeçam ou se comportem de maneira ordeira | cuidar; zelar por; proteger; cuidar, organizar coisas}
\end{EntryWithPhonetic}

\begin{EntryWithPhonetic}{管理费}{guan3li3fei4}{14,11,9}{⽵,⽟,⾙}[HSK 7-9]
  \definition{s.}{despesas de gestão; custos de administração | taxa de administração}
\end{EntryWithPhonetic}

\begin{EntryWithPhonetic}{管辖}{guan3xia2}{14,14}{⽵,⾞}[HSK 7-9]
  \definition{v.}{gerenciar; governar (pessoal, assuntos, áreas, casos, etc.)}
\end{EntryWithPhonetic}

\begin{EntryWithPhonetic}{管用}{guan3yong4}{14,5}{⽵,⽤}[HSK 7-9]
  \definition{adj.}{eficaz; funcional}
\end{EntryWithPhonetic}

\begin{EntryWithPhonetic}{管仲}{guan3 zhong4}{14,6}{⽵,⼈}
  \definition*{s.}{uma visão restrita através de um tubo de bambu | conhecido como tubo de Guangzi 管子}
  \definition*{s.}{Guan Zhong (-645 aC), famoso político do Qi (齐国) do período da Primavera e Outono}
  \seealsoref{管子}{guan3zi5}
  \seealsoref{齐国}{qi2 guo2}
\end{EntryWithPhonetic}

\begin{EntryWithPhonetic}{管子}{guan3zi5}{14,3}{⽵,⼦}[HSK 7-9]
  \definition*{s.}{Guanzi ou Guan Zhong 管仲 (-645 a.C.), famoso político de Qi (齐国) do período da Primavera e do Outono | Guanzi, livro clássico contendo escritos de Guan Zhong e sua escola}
  \seealsoref{管仲}{guan3 zhong4}
  \seealsoref{齐国}{qi2 guo2}
\end{EntryWithPhonetic}

%%%%%%%%%% 观 %%%%%%%%%%
\subsection*{观}\addcontentsline{loh}{figure}{观 \dpy{guan4}}

\begin{EntryWithPhonetic}{观}{guan4}{6}{⾒}
  \definition*{s.}{Sobrenome: Guan}
  \definition{s.}{mosteiro taoísta | torre de vigia do portão do palácio | plataforma}
  \seeref{guan1}
\end{EntryWithPhonetic}

%%%%%%%%%% 贯 %%%%%%%%%%
\subsection*{贯}\addcontentsline{loh}{figure}{贯 \dpy{guan4}}

\begin{EntryWithPhonetic}{贯}{guan4}{8}{⾙}
  \definition*{s.}{Sobrenome: Guan}
  \definition{clas.}{uma sequência de 1.000 em dinheiro; antigamente, o dinheiro era amarrado com cordas, e cada mil moedas era uma corda.}
  \definition{s.}{lugar nativo; local de nascimento; lugar do lar ancestral; lugar onde gerações viveram | Literário: exemplo; instância; caso; precedente | Arcaico: guan (uma corda de 1.000 moedas de cobre); corda para amarrar dinheiro nos tempos antigos}
  \definition{v.}{passar através de; perfurar; enfiar; penetrar | estar ligados entre si; seguir em linha contínua; estar conectado | Literário: comparecer}
\end{EntryWithPhonetic}

\begin{EntryWithPhonetic}{贯彻}{guan4che4}{8,7}{⾙,⼻}[HSK 7-9]
  \definition{v.}{executar; implementar; pôr em prática; realizar ou incorporar completamente (diretrizes, políticas, espírito, etc.)}
\end{EntryWithPhonetic}

\begin{EntryWithPhonetic}{贯穿}{guan4chuan1}{8,9}{⾙,⽳}[HSK 7-9]
  \definition{v.}{cruzar; conectar; penetrar; correr através; passar através | permear; estar cheio de}
\end{EntryWithPhonetic}

\begin{EntryWithPhonetic}{贯通}{guan4tong1}{8,10}{⾙,⾡}[HSK 7-9]
  \definition{v.}{ter um conhecimento profundo de; ser bem versado (em); (acadêmico, ideológico, etc.) ter compreensão completa | ligar; encadear}
\end{EntryWithPhonetic}

%%%%%%%%%% 冠 %%%%%%%%%%
\subsection*{冠}\addcontentsline{loh}{figure}{冠 \dpy{guan4}}

\begin{EntryWithPhonetic}{冠}{guan4}{9}{⼍}
  \definition*{s.}{Sobrenome: Guan}
  \definition{s.}{primeiro lugar; o melhor; classificado em primeiro lugar}
  \definition{v.}{colocar um chapéu (boné) | preceder com (por); coroar com; adicionar um nome ou texto na frente}
  \seeref{guan1}
\end{EntryWithPhonetic}

\begin{EntryWithPhonetic}{冠军}{guan4jun1}{9,6}{⼍,⼍}[HSK 5]
  \definition[位,名,项,个]{s.}{campeão; medalhista de ouro; primeiro lugar em esportes e outras competições}
\end{EntryWithPhonetic}

%%%%%%%%%% 惯 %%%%%%%%%%
\subsection*{惯}\addcontentsline{loh}{figure}{惯 \dpy{guan4}}

\begin{EntryWithPhonetic}{惯}{guan4}{11}{⼼}[HSK 7-9]
  \definition{adj.}{habitual; costumeiro; usual | incorrigível; endurecido}
  \definition{v.}{estar acostumado a; ter o hábito de | mimar; estragar}
\end{EntryWithPhonetic}

\begin{EntryWithPhonetic}{惯例}{guan4li4}{11,8}{⼼,⼈}[HSK 7-9]
  \definition[个]{s.}{rotina; convenção; prática usual; prática habitual | precedente; embora não haja nenhuma disposição explícita na lei, há práticas que foram implementadas no passado e podem ser imitadas}
\end{EntryWithPhonetic}

\begin{EntryWithPhonetic}{惯性}{guan4xing4}{11,8}{⼼,⼼}[HSK 7-9]
  \definition{s.}{Física: inércia; a força da inércia}
\end{EntryWithPhonetic}

%%%%%%%%%% 灌 %%%%%%%%%%
\subsection*{灌}\addcontentsline{loh}{figure}{灌 \dpy{guan4}}

\begin{EntryWithPhonetic}{灌}{guan4}{20}{⽔}[HSK 7-9]
  \definition*{s.}{Sobrenome: Guan}
  \definition{s.}{arbusto; aglomerados de árvores baixas | irrigação}
  \definition{v.}{irrigar (rega e irrigação do solo) | encher; despejar; injetar | gravar; refere"-se à gravação (música)}
\end{EntryWithPhonetic}

\begin{EntryWithPhonetic}{灌溉}{guan4gai4}{20,12}{⽔,⽔}[HSK 7-9]
  \definition{v.}{regar; irrigar}
\end{EntryWithPhonetic}

\begin{EntryWithPhonetic}{灌输}{guan4shu1}{20,13}{⽔,⾞}[HSK 7-9]
  \definition{v.}{implantar; incutir em; inculcar; imbuir com (ideias, conhecimento); transmitir (ideias, conhecimento, etc.) | canalizar água; despejar água em; direcionar a água para onde ela é necessária}
\end{EntryWithPhonetic}

%%%%%%%%%% 罐 %%%%%%%%%%
\subsection*{罐}\addcontentsline{loh}{figure}{罐 \dpy{guan4}}

\begin{EntryWithPhonetic}{罐}{guan4}{23}{⽸}[HSK 7-9]
  \definition{clas.}{lata; jarra; gavetas e recipientes de água feitos de cerâmica ou metal}[我买了一罐可乐。===Comprei uma lata de Coca-Cola.]
  \definition{s.}{lata; jarra; jarro; pote; tanque | cuba de carvão; vagão de caçamba para carregamento de carvão em minas de carvão}
\end{EntryWithPhonetic}

\begin{EntryWithPhonetic}{罐头食品}{guan4tou2 shi2pin3}{23,5,9,9}{⽸,⼤,⾷,⼝}
  \definition{s.}{alimentos enlatados; produtos enlatados}
\end{EntryWithPhonetic}

\begin{EntryWithPhonetic}{罐头}{guan4tou5}{23,5}{⽸,⼤}[HSK 7-9]
  \definition[个,盒,瓶]{s.}{lata; jarra | enlatado; comida enlatada é a abreviação de 罐头食品, que é processada e embalada em latas de ferro seladas ou garrafas de vidro, e pode ser armazenada por um longo tempo}
  \seealsoref{罐头食品}{guan4tou2 shi2pin3}
\end{EntryWithPhonetic}

%%%%%%%%%% 光 %%%%%%%%%%
\subsection*{光}\addcontentsline{loh}{figure}{光 \dpy{guang1}}

\begin{EntryWithPhonetic}{光}{guang1}{6}{⼉}[HSK 3]
  \definition*{s.}{Sobrenome: Guang}
  \definition{adj.}{suave; liso; brilhante | esgotado; sem nada sobrando | brilhante}
  \definition{adv.}{somente; sozinho; meramente}
  \definition{s.}{luz; raio | cenário; paisagem | honra; glória; brilho | claridade | favor; graça | momento | corpo celeste; referindo"-se especificamente a corpos celestes, como o sol, a lua e as estrelas}
  \definition{v.}{glorificar; recuperar; reconquistar | estar nu; expor}
\end{EntryWithPhonetic}

\begin{EntryWithPhonetic}{光彩}{guang1cai3}{6,11}{⼉,⼺}[HSK 7-9]
  \definition{adj.}{glorioso; honroso; decente}
  \definition{s.}{brilho; esplendor; radiância}
\end{EntryWithPhonetic}

\begin{EntryWithPhonetic}{光碟}{guang1die2}{6,14}{⼉,⽯}[HSK 7-9]
  \definition[个,片,张]{s.}{disco compacto (CD); videodisco; CD; CD-ROM; disco ótico}
\end{EntryWithPhonetic}

\begin{EntryWithPhonetic}{光顾}{guang1gu4}{6,10}{⼉,⾴}[HSK 7-9]
  \definition{v.}{patrocinar; honrar com; uma palavra que demonstra respeito a alguém, referindo"-se à chegada de um convidado; restaurantes e lojas costumam usá"-la para dar as boas"-vindas aos clientes; também é usada de forma metafórica e irônica}
\end{EntryWithPhonetic}

\begin{EntryWithPhonetic}{光滑}{guang1hua2}{6,12}{⼉,⽔}[HSK 7-9]
  \definition{adj.}{liso; suave; brilhante}
\end{EntryWithPhonetic}

\begin{EntryWithPhonetic}{光环}{guang1huan2}{6,8}{⼉,⽟}[HSK 7-9]
  \definition[道]{s.}{um anel de luz; matéria brilhante ao redor de alguns planetas | halo; auréola; o halo anular na cabeça de uma divindade | um halo colorido que às vezes aparece ao redor do sol ou da lua | glória; distinção; esplendor; metáfora para fama e honra}
\end{EntryWithPhonetic}

\begin{EntryWithPhonetic}{光辉}{guang1hui1}{6,12}{⼉,⾞}[HSK 6]
  \definition{adj.}{brilhante; magnífico; glorioso}
  \definition{s.}{esplendor; brilho; glória | chama; brilho; halo; labareda; fulguração; lustre}
\end{EntryWithPhonetic}

\begin{EntryWithPhonetic}{光缆}{guang1lan3}{6,12}{⼉,⽷}[HSK 7-9]
  \definition[根,条]{s.}{cabo óptico; cabo de fibra óptica}
\end{EntryWithPhonetic}

\begin{EntryWithPhonetic}{光临}{guang1lin2}{6,9}{⼉,⼁}[HSK 4]
  \definition{v.}{honrar com sua presença, uma palavra de honra, usada para dizer que um convidado chegou}
\end{EntryWithPhonetic}

\begin{EntryWithPhonetic}{光芒}{guang1mang2}{6,6}{⼉,⾋}[HSK 7-9]
  \definition[道]{s.}{brilho; radiância; raios brilhantes; raios de luz; luz forte irradiando em todas as direções}
\end{EntryWithPhonetic}

\begin{EntryWithPhonetic}{光明}{guang1ming2}{6,8}{⼉,⽇}[HSK 3]
  \definition{adj.}{brilhante; luminoso | sincero; ingênuo; metáfora da justiça e da esperança | justo; honesto; franco}
  \definition{s.}{luz}
\end{EntryWithPhonetic}

\begin{EntryWithPhonetic}{光明磊落}{guang1ming2-lei3luo4}{6,8,15,12}{⼉,⽇,⽯,⾋}[HSK 7-9]
  \definition{expr.}{aberto e sincero; direto e honesto; descreve ser altruísta e de mente aberta; aberto e transparente}
\end{EntryWithPhonetic}

\begin{EntryWithPhonetic}{光盘}{guang1pan2}{6,11}{⼉,⽫}[HSK 4]
  \definition[张,套,片]{s.}{CD; disco compacto; um disco circular feito de plástico rígido composto que usa um laser para registrar e ler informações}
\end{EntryWithPhonetic}

\begin{EntryWithPhonetic}{光槃}{guang1pan2}{6,14}{⼉,⽊}
  \variantof{光盘}
\end{EntryWithPhonetic}

\begin{EntryWithPhonetic}{光荣}{guang1rong2}{6,9}{⼉,⾋}[HSK 5]
  \definition{adj.}{honroso; honrado; glorioso; por fazer algo que é benéfico para o país ou para a coletividade e que é considerado por todos como digno de respeito ou elogio}
  \definition{s.}{honra; glória; crédito; sentimento de honra decorrente do fato de ser respeitado ou elogiado por fazer algo importante ou grandioso}
\end{EntryWithPhonetic}

\begin{EntryWithPhonetic}{光污染}{guang1 wu1ran3}{6,6,9}{⼉,⽔,⽊}
  \definition{s.}{poluição luminosa}
\end{EntryWithPhonetic}

\begin{EntryWithPhonetic}{光线}{guang1xian4}{6,8}{⼉,⽷}[HSK 5]
  \definition[条,道]{s.}{luz; feixe luminoso; raio de luz}
\end{EntryWithPhonetic}

\begin{EntryWithPhonetic}{光泽}{guang1ze2}{6,8}{⼉,⽔}[HSK 7-9]
  \definition{s.}{brilho; lustro; fulgor; luz brilhante refletida de uma superfície; cor e brilho}
\end{EntryWithPhonetic}

%%%%%%%%%% 广 %%%%%%%%%%
\subsection*{广}\addcontentsline{loh}{figure}{广 \dpy{guang3}}

\begin{EntryWithPhonetic}{广}{guang3}{3}{⼴}[HSK 5][Kangxi 53]
  \definition*{s.}{Sobrenome: Guang}
  \definition{adj.}{largo; vasto; amplo; extenso | numeroso | comum; universal}
  \definition{s.}{Guangdong, 广东, e Guangxi, 广州}
  \definition{v.}{expandir; espalhar; ampliar}
  \seeref{an1}
  \seeref{yan3}
  \seealsoref{广东}{guang3dong1}
  \seealsoref{广州}{guang3zhou1}
  \antonymref{狭}{xia2}
\end{EntryWithPhonetic}

\begin{EntryWithPhonetic}{广播}{guang3bo1}{3,15}{⼴,⼿}[HSK 3]
  \definition[个,次,段,则,条]{s.}{programa de rádio; transmissão (de rádio); refere"-se a programas transmitidos por estações de rádio ou televisão a cabo}
  \definition{v.}{transmitir; estar no ar | espalhar"-se amplamente; ser conhecido em toda parte; divulgar amplamente}
\end{EntryWithPhonetic}

\begin{EntryWithPhonetic}{广场}{guang3chang3}{3,6}{⼴,⼟}[HSK 2]
  \definition{s.}{praça; praça pública; esplanada; área ampla, especificamente uma área ampla na cidade}
\end{EntryWithPhonetic}

\begin{EntryWithPhonetic}{广场舞}{guang3chang3wu3}{3,6,14}{⼴,⼟,⾇}
  \definition{s.}{quadrilha, uma rotina de exercícios tocada com música em quadrados públicos, parques e praças, popular especialmente entre mulheres de meia-idade e aposentados na China}
\end{EntryWithPhonetic}

\begin{EntryWithPhonetic}{广大}{guang3da4}{3,3}{⼴,⼤}[HSK 3]
  \definition{adj.}{muito difundido; enorme (alcance, escala) | (uma área ou espaço) vasto; extenso; em grande escala; amplo (área, espaço) | numeroso; muitos (número de pessoas)}
\end{EntryWithPhonetic}

\begin{EntryWithPhonetic}{广东}{guang3dong1}{3,5}{⼴,⼀}
  \definition*{s.}{Província de Guangdong}
  \seealsoref{粤}{yue4}
\end{EntryWithPhonetic}

\begin{EntryWithPhonetic}{广泛}{guang3fan4}{3,7}{⼴,⽔}[HSK 5]
  \definition{adj.}{amplo; extenso; de grande alcance; disseminado; escopo e cobertura amplos}
\end{EntryWithPhonetic}

\begin{EntryWithPhonetic}{广告}{guang3gao4}{3,7}{⼴,⼝}[HSK 2]
  \definition[则,条,段,项,个]{s.}{anúncio; propaganda; uma forma de divulgação ao público de produtos, serviços ou programas culturais e esportivos, geralmente realizada por meio de jornais, televisão, rádio, cartazes, etc.}
  \definition{v.}{anunciar; a ação ou ato de promover ou divulgar algo}
\end{EntryWithPhonetic}

\begin{EntryWithPhonetic}{广阔}{guang3kuo4}{3,12}{⼴,⾨}[HSK 6]
  \definition{adj.}{vasto; largo; amplo}
\end{EntryWithPhonetic}

\begin{EntryWithPhonetic}{广西}{guang3xi1}{3,6}{⼴,⾑}
  \definition*{s.}{Guangxi (Região Autônoma de Zhuang)}
  \seealsoref{壮}{zhuang4}
\end{EntryWithPhonetic}

\begin{EntryWithPhonetic}{广义}{guang3yi4}{3,3}{⼴,⼂}[HSK 7-9]
  \definition*{s.}{Província de Quang Ngai; nome de lugar vietnamita, uma das províncias do Vietnã Central}
  \definition{s.}{sentido amplo; sentido geral; definição mais ampla}
\end{EntryWithPhonetic}

\begin{EntryWithPhonetic}{广州}{guang3zhou1}{3,6}{⼴,⼮}
  \definition*{s.}{Guangzhou, antigamente Cantão; Capital da Província de Guangdong}
\end{EntryWithPhonetic}

%%%%%%%%%% 逛 %%%%%%%%%%
\subsection*{逛}\addcontentsline{loh}{figure}{逛 \dpy{guang4}}

\begin{EntryWithPhonetic}{逛}{guang4}{10}{⾡}[HSK 4]
  \definition{v.}{perambular; passear; vaguear}
\end{EntryWithPhonetic}

%%%%%%%%%% 归 %%%%%%%%%%
\subsection*{归}\addcontentsline{loh}{figure}{归 \dpy{gui1}}

\begin{EntryWithPhonetic}{归}{gui1}{5}{⼹}[HSK 4]
  \definition*{s.}{Sobrenome: Gui}
  \definition{s.}{divisão no ábaco com divisor de um dígito}
  \definition{v.}{retornar; voltar para; voltar (ou ir) | devolver algo a; dar de volta a | convergir; juntar-se | encarregar alguém de algo | atribuir a; pertencer a}
  \definition{v.aux.}{usado entre dois verbos idênticos, indicando que a ação não levou ao resultado correspondente}
\end{EntryWithPhonetic}

\begin{EntryWithPhonetic}{归根到底}{gui1gen1-dao4di3}{5,10,8,8}{⼹,⽊,⼑,⼴}[HSK 7-9]
  \definition{expr.}{``Em última análise.'', significa que, no final, as coisas acabarão de uma certa maneira; ela vem de 《何典》, de 张南庄, da Dinastia Qing (清); na análise final (última); no longo prazo; afinal; na análise final; em essência; fundamentalmente}
  \seealsoref{何典}{he2 dian3}
  \seealsoref{清}{qing1}
  \seealsoref{张南庄}{zhang1 nan2zhuang1}
\end{EntryWithPhonetic}

\begin{EntryWithPhonetic}{归还}{gui1huan2}{5,7}{⼹,⾡}[HSK 7-9]
  \definition{v.}{retornar; reverter; devolver dinheiro ou itens emprestados ao proprietário original}
  \antonymref{借用}{jie4yong4}
\end{EntryWithPhonetic}

\begin{EntryWithPhonetic}{归结}{gui1jie2}{5,9}{⼹,⽷}[HSK 7-9]
  \definition{s.}{fim; final (de uma história, etc.) | resolução}
  \definition{v.}{chegar a uma conclusão; resumir; colocar em poucas palavras}
\end{EntryWithPhonetic}

\begin{EntryWithPhonetic}{归来}{gui1lai2}{5,7}{⼹,⽊}[HSK 7-9]
  \definition{v.}{retornar; voltar ou estar de volta; retornar ao local de onde você começou ou partiu de outro lugar}
\end{EntryWithPhonetic}

\begin{EntryWithPhonetic}{归纳}{gui1na4}{5,7}{⼹,⽷}[HSK 7-9]
  \definition{s.}{indução; método indutivo}
  \definition{v.}{induzir; concluir; mesclar e classificar; resumir (usado principalmente para coisas abstratas)}
\end{EntryWithPhonetic}

\begin{EntryWithPhonetic}{归属}{gui1shu3}{5,12}{⼹,⼫}[HSK 7-9]
  \definition{v.}{pertencer a; estar sob a jurisdição de; definir afiliação}
\end{EntryWithPhonetic}

\begin{EntryWithPhonetic}{归宿}{gui1su4}{5,11}{⼹,⼧}[HSK 7-9]
  \definition{s.}{um lar para retornar; destino final; fim}
\end{EntryWithPhonetic}

%%%%%%%%%% 龟 %%%%%%%%%%
\subsection*{龟}\addcontentsline{loh}{figure}{龟 \dpy{gui1}}

\begin{EntryWithPhonetic}{龟}{gui1}{7}{⿔}[HSK 7-9][Kangxi 213]
  \definition[只]{s.}{tartaruga; cágado}
\end{EntryWithPhonetic}

\begin{EntryWithPhonetic}{龟速}{gui1su4}{7,10}{⿔,⾡}
  \definition{adv.}{tão lento quanto uma tartaruga}
\end{EntryWithPhonetic}

%%%%%%%%%% 规 %%%%%%%%%%
\subsection*{规}\addcontentsline{loh}{figure}{规 \dpy{gui1}}

\begin{EntryWithPhonetic}{规}{gui1}{8}{⾒}
  \definition*{s.}{Sobrenome: Gui}
  \definition[个,种]{s.}{bússola | regulamentação; regra | (mecânica) medidor | compasso; ferramenta para desenhar círculos}
  \definition{v.}{admoestar; aconselhar; advertir | planejar; fazer planos}
\end{EntryWithPhonetic}

\begin{EntryWithPhonetic}{规定}{gui1ding4}{8,8}{⾒,⼧}[HSK 3]
  \definition[个,条,项,款]{s.}{regra; regulamento; estipulação; tomar decisões sobre a forma, o método, a quantidade ou a qualidade de algo}
  \definition{v.}{estipular; prover; prescrever; estabelecer requisitos ou restrições em termos de métodos, qualidade, quantidade, tempo, etc.}
\end{EntryWithPhonetic}

\begin{EntryWithPhonetic}{规范}{gui1fan4}{8,9}{⾒,⾋}[HSK 3]
  \definition{adj.}{regular; normal; padrão; que atende às especificações; em conformidade com as normas}
  \definition{s.}{norma; padrão; diretriz}
  \definition{v.}{regular; padronizar; tornar conforme as normas}
\end{EntryWithPhonetic}

\begin{EntryWithPhonetic}{规格}{gui1ge2}{8,10}{⾒,⽊}[HSK 7-9]
  \definition[种]{s.}{normas; padrões; especificações; padrões de qualidade do produto, como determinados tamanho, peso, precisão, desempenho, etc. | formato; padrão; requisito; geralmente se refere a requisitos ou condições especificados}
\end{EntryWithPhonetic}

\begin{EntryWithPhonetic}{规划}{gui1hua4}{8,6}{⾒,⼑}[HSK 5]
  \definition[个,项]{s.}{plano; projeto; planejamento; programa; programação; esquematização; plano de desenvolvimento de longo prazo mais abrangente}
  \definition{v.}{planejar; programar}
\end{EntryWithPhonetic}

\begin{EntryWithPhonetic}{规矩}{gui1ju5}{8,9}{⾒,⽮}[HSK 7-9]
  \definition{adj.}{adequado; bem comportado; bem disciplinado; honesto e correto; de acordo com os padrões ou o senso comum}
  \definition[条,个,项]{s.}{regra; costume; prática estabelecida; certos padrões, regras ou costumes}
\end{EntryWithPhonetic}

\begin{EntryWithPhonetic}{规律}{gui1lv4}{8,9}{⾒,⼻}[HSK 4]
  \definition{adj.}{estável; regular; coisas, comportamentos, fenômenos, etc. que ocorrem em um determinado momento}
  \definition{s.}{lei; padrão regular; conexão essencial e recorrente entre as coisas}
\end{EntryWithPhonetic}

\begin{EntryWithPhonetic}{规模}{gui1mo2}{8,14}{⾒,⽊}[HSK 4]
  \definition[个,种]{s.}{escala; escopo; dimensões; padrão, forma ou escopo (de um empreendimento, instituição, projeto, movimento, etc.)}
\end{EntryWithPhonetic}

\begin{EntryWithPhonetic}{规则}{gui1ze2}{8,6}{⾒,⼑}[HSK 4]
  \definition{adj.}{ordenado; regular; descreve a forma, estrutura, arranjo, etc., que se conformam a uma determinada maneira organizada}
  \definition{s.}{regra; regulamento; sistema ou código de conduta prescrito para observância comum | lei; norma}
\end{EntryWithPhonetic}

%%%%%%%%%% 闺 %%%%%%%%%%
\subsection*{闺}\addcontentsline{loh}{figure}{闺 \dpy{gui1}}

\begin{EntryWithPhonetic}{闺}{gui1}{9}{⾨}
  \definition{s.}{Arcaico: (em uma casa) porta pequena; porta com arco | quarto da senhora; \emph{boudoir} | Literário: um pequeno portão; parte superior redonda e porta pequena na parte inferior}
\end{EntryWithPhonetic}

\begin{EntryWithPhonetic}{闺女}{gui1nv5}{9,3}{⾨,⼥}[HSK 7-9]
  \definition[个]{s.}{menina; donzela; mulher solteira | filha}
\end{EntryWithPhonetic}

%%%%%%%%%% 瑰 %%%%%%%%%%
\subsection*{瑰}\addcontentsline{loh}{figure}{瑰 \dpy{gui1}}

\begin{EntryWithPhonetic}{瑰}{gui1}{13}{⽟}
  \definition{adj.}{Literário: raro; maravilhoso; fabuloso}
  \definition[朵]{s.}{jaspe fino | Arcaico: uma espécie de pedra semelhante ao jade}
\end{EntryWithPhonetic}

\begin{EntryWithPhonetic}{瑰宝}{gui1bao3}{13,8}{⽟,⼧}[HSK 7-9]
  \definition{s.}{raridade; tesouro; joia; coisas muito preciosas}
\end{EntryWithPhonetic}

%%%%%%%%%% 轨 %%%%%%%%%%
\subsection*{轨}\addcontentsline{loh}{figure}{轨 \dpy{gui3}}

\begin{EntryWithPhonetic}{轨}{gui3}{6}{⾞}
  \definition{s.}{trilho; pista | curso; caminho | ordem; regulamento; regra | rotina; metaforicamente falando, métodos, regras, ordem, etc.}
  \definition{v.}{seguir | Literário: cumprir; aderir a}
\end{EntryWithPhonetic}

\begin{EntryWithPhonetic}{轨道}{gui3dao4}{6,12}{⾞,⾡}[HSK 6]
  \definition[条]{s.}{trilha; uma rota pavimentada com trilhos de aço para trens, bondes, etc. | órbita; trajetória; corpos celestes e objetos têm trajetórias de movimento regulares | caminho; curso; maneira adequada de fazer as coisas; curso adequado; uma metáfora para o desenvolvimento normal das coisas ou as normas e procedimentos que as pessoas devem seguir}
\end{EntryWithPhonetic}

\begin{EntryWithPhonetic}{轨迹}{gui3ji4}{6,9}{⾞,⾡}[HSK 7-9]
  \definition{s.}{trilha; caminho; trajetória | órbita; caminho | trilha; pegada; uma metáfora para experiências de vida ou o caminho de desenvolvimento das coisas}
\end{EntryWithPhonetic}

%%%%%%%%%% 鬼 %%%%%%%%%%
\subsection*{鬼}\addcontentsline{loh}{figure}{鬼 \dpy{gui3}}

\begin{EntryWithPhonetic}{鬼}{gui3}{9}{⿁}[HSK 5][Kangxi 194]
  \definition*{s.}{Gui, uma das mansões lunares | Gui, a vigésima terceira das vinte e oito constelações em que a esfera celeste foi dividida, consistindo de quatro estrelas em Câncer | Sobrenome: Gui}
  \definition{adj.}{evasivo; furtivo; sub"-reptício; ardiloso; enganoso, malicioso; obscuro | terrível; ruim; severo; vil | esperto; astuto; inteligente}
  \definition{s.}{espírito; fantasma; aparição; refere"-se à alma de uma pessoa após a morte | usado para formar um termo de abuso para caráter ignóbil; refere"-se a pessoas que têm maus hábitos ou cujo comportamento é repugnante | companheiro; pessoa que é considerada divertida}
\end{EntryWithPhonetic}

\begin{EntryWithPhonetic}{鬼怪}{gui3guai4}{9,8}{⿁,⼼}
  \definition{s.}{\emph{hobgoblin} | bicho-papão | fantasma}
\end{EntryWithPhonetic}

\begin{EntryWithPhonetic}{鬼火}{gui3huo3}{9,4}{⿁,⽕}
  \definition{s.}{fogo-fátuo | boitatá | fogo corredor | fogo de santelmo}
\end{EntryWithPhonetic}

%%%%%%%%%% 柜 %%%%%%%%%%
\subsection*{柜}\addcontentsline{loh}{figure}{柜 \dpy{gui4}}

\begin{EntryWithPhonetic}{柜}{gui4}{8}{⽊}
  \definition{s.}{baú; armário; gabinete | loja; balcão}
  \seeref{ju3}
\end{EntryWithPhonetic}

\begin{EntryWithPhonetic}{柜台}{gui4tai2}{8,5}{⽊,⼝}[HSK 7-9]
  \definition[个,排,组]{s.}{bar; balcão; uma longa área semelhante a uma mesa em uma loja ou banco usada para vender mercadorias ou conduzir negócios}
\end{EntryWithPhonetic}

\begin{EntryWithPhonetic}{柜子}{gui4zi5}{8,3}{⽊,⼦}[HSK 5]
  \definition[个]{s.}{gabinete; armário; dispositivo para guardar roupas, documentos, livros, etc.}
\end{EntryWithPhonetic}

%%%%%%%%%% 贵 %%%%%%%%%%
\subsection*{贵}\addcontentsline{loh}{figure}{贵 \dpy{gui4}}

\begin{EntryWithPhonetic}{贵}{gui4}{9}{⾙}[HSK 1]
  \definition*{s.}{Província de Guizhou, abreviação de 贵州 | Sobrenome: Gui}
  \definition{adj.}{caro; dispendioso | altamente valorizado; valioso | de alta patente; nobre | caro; preço ou valor elevado | digno de ser valorizado ou apreciado | nobre; honrado; posição social elevada}
  \definition{pron.}{Honrado: Seu}
  \seealsoref{贵州}{gui4zhou1}
  \antonymref{贱}{jian4}
\end{EntryWithPhonetic}

\begin{EntryWithPhonetic}{贵宾}{gui4bin1}{9,10}{⾙,⼧}[HSK 7-9]
  \definition[位]{s.}{convidado de honra; convidado distinto; um convidado de alto escalão, importante e respeitado}
\end{EntryWithPhonetic}

\begin{EntryWithPhonetic}{贵姓}{gui4xing4}{9,8}{⾙,⼥}
  \definition{expr.}{qual seu sobrenome?}
\end{EntryWithPhonetic}

\begin{EntryWithPhonetic}{贵重}{gui4zhong4}{9,9}{⾙,⾥}[HSK 7-9]
  \definition{adj.}{valioso; precioso; alto valor; digno de atenção}
\end{EntryWithPhonetic}

\begin{EntryWithPhonetic}{贵州}{gui4zhou1}{9,6}{⾙,⼮}
  \definition*{s.}{Província de Guizhou}
\end{EntryWithPhonetic}

\begin{EntryWithPhonetic}{贵族}{gui4zu2}{9,11}{⾙,⽅}[HSK 7-9]
  \definition{s.}{nobre; nobreza; aristocracia; a classe alta da classe dominante na sociedade escravista ou feudal e na monarquia moderna goza de privilégios}
\end{EntryWithPhonetic}

%%%%%%%%%% 桂 %%%%%%%%%%
\subsection*{桂}\addcontentsline{loh}{figure}{桂 \dpy{gui4}}

\begin{EntryWithPhonetic}{桂}{gui4}{10}{⽊}
  \definition*{s.}{outro nome para o rio Guijiang 桂江 (em Guangxi 广西) | outro nome para Guangxi 广西 (Região Autônoma de Zhuang) | Sobrenome: Gui}
  \definition[棵]{s.}{louro; loureiro | osmanthus de aroma doce | árvore de casca de cássia | canela; osmanthus}
  \seealsoref{广西}{guang3xi1}
  \seealsoref{桂江}{gui4jiang1}
\end{EntryWithPhonetic}

\begin{EntryWithPhonetic}{桂花}{gui4hua1}{10,7}{⽊,⾋}[HSK 7-9]
  \definition{s.}{jasmim do imperador; um arbusto perene ou pequena árvore, cujas flores também são chamadas de osmanthus, são muito perfumadas e podem ser usadas para extrair óleos aromáticos ou fazer especiarias. Variedades comuns incluem Jingui 金桂 (flores amarelo-alaranjadas), Dangui 丹桂 (flores vermelho-alaranjadas), Yingui 银桂 (flores branco-amareladas) e Sijigui 四季桂 (flores branco-amareladas).}
\end{EntryWithPhonetic}

\begin{EntryWithPhonetic}{桂江}{gui4jiang1}{10,6}{⽊,⽔}
  \definition*{s.}{Rio Guijiang}
\end{EntryWithPhonetic}

%%%%%%%%%% 跪 %%%%%%%%%%
\subsection*{跪}\addcontentsline{loh}{figure}{跪 \dpy{gui4}}

\begin{EntryWithPhonetic}{跪}{gui4}{13}{⾜}[HSK 6]
  \definition{v.}{ajoelhar-se; dobrar os joelhos de modo que um ou ambos os joelhos toquem o chão}
\end{EntryWithPhonetic}

\begin{EntryWithPhonetic}{跪拜}{gui4bai4}{13,9}{⾜,⼿}
  \definition{v.}{prostrar-se | ajoelhar-se e adorar}
\end{EntryWithPhonetic}

%%%%%%%%%% 滚 %%%%%%%%%%
\subsection*{滚}\addcontentsline{loh}{figure}{滚 \dpy{gun3}}

\begin{EntryWithPhonetic}{滚}{gun3}{13}{⽔}[HSK 5]
  \definition*{s.}{Sobrenome: Gun}
  \definition{adj.}{rolante | fervente | precipitado; torrencial}
  \definition{adv.}{muito; em um grau elevado}
  \definition{v.}{rolar; girar; virar | escapar; fugir; ir embora | ferver | amarrar; aparar; fazer bainha}
\end{EntryWithPhonetic}

\begin{EntryWithPhonetic}{滚动}{gun3dong4}{13,6}{⽔,⼒}[HSK 7-9]
  \definition{adv.}{(fazer algo) em intervalos regulares | (fazer algo) em um loop; onduladamente | expandir progressivamente (economia)}
  \definition{v.}{rolar; girar; fazer rodízio | (um trovão) fazer barulho; ribombar; estrondear | Computação: rolar}
\end{EntryWithPhonetic}

\begin{EntryWithPhonetic}{滚滚}{gun3 gun3}{13,13}{⽔,⽔}
  \definition{adj.}{ondulante | rolando continuamente}
  \definition{v.}{rolar; ondular; fluir}
\end{EntryWithPhonetic}

\begin{EntryWithPhonetic}{滚轮}{gun3lun2}{13,8}{⽔,⾞}
  \definition{s.}{pneu | dial rotativo | roda de rolagem (\emph{scroll})  (mouse de computador)}
\end{EntryWithPhonetic}

%%%%%%%%%% 棍 %%%%%%%%%%
\subsection*{棍}\addcontentsline{loh}{figure}{棍 \dpy{gun4}}

\begin{EntryWithPhonetic}{棍}{gun4}{12}{⽊}[HSK 7-9]
  \definition[根]{s.}{vara; bastão; porrete | canalha; patife; ladino; bandido}
\end{EntryWithPhonetic}

\begin{EntryWithPhonetic}{棍子}{gun4zi5}{12,3}{⽊,⼦}[HSK 7-9]
  \definition[根]{s.}{vara; bastão; um objeto longo e redondo feito de madeira, bambu ou metal}
\end{EntryWithPhonetic}

%%%%%%%%%% 过 %%%%%%%%%%
\subsection*{过}\addcontentsline{loh}{figure}{过 \dpy{guo1}}

\begin{EntryWithPhonetic}{过}{guo1}{6}{⾡}
  \definition*{s.}{Sobrenome: Guo}
  \seeref{guo4}
  \seeref{guo5}
\end{EntryWithPhonetic}

%%%%%%%%%% 锅 %%%%%%%%%%
\subsection*{锅}\addcontentsline{loh}{figure}{锅 \dpy{guo1}}

\begin{EntryWithPhonetic}{锅}{guo1}{12}{⾦}[HSK 5]
  \definition[口,个,只]{s.}{panela; frigideira; utensílios de cozinha, redondos e côncavos, feitos principalmente de ferro, alumínio, etc. | parte que se parece com um pote em alguns objetos}
\end{EntryWithPhonetic}

%%%%%%%%%% 囯 %%%%%%%%%%
\subsection*{囯}\addcontentsline{loh}{figure}{囯 \dpy{guo2}}

\begin{EntryWithPhonetic}{囯}{guo2}{7}{⼞}
  \definition*{s.}{Sobrenome: Guo}
  \definition{adj.}{do estado; nacional | do nosso país; Chinês | do país}
  \definition{s.}{país; nação; estado | o melhor da nação | o melhor; o mais bonito do país}
  \variantof{国}
\end{EntryWithPhonetic}

%%%%%%%%%% 国 %%%%%%%%%%
\subsection*{国}\addcontentsline{loh}{figure}{国 \dpy{guo2}}

\begin{EntryWithPhonetic}{国}{guo2}{8}{⼞}[HSK 1]
  \definition*{s.}{Sobrenome: Guo}
  \definition{adj.}{nacional; do estado; representante do país | o melhor de um país}
  \definition[个]{s.}{estado; nação; país}
\end{EntryWithPhonetic}

\begin{EntryWithPhonetic}{国宝}{guo2bao3}{8,8}{⼞,⼧}[HSK 7-9]
  \definition[件]{s.}{tesouro nacional}
\end{EntryWithPhonetic}

\begin{EntryWithPhonetic}{国宾馆}{guo2bin1guan3}{8,10,11}{⼞,⼧,⾷}
  \definition{s.}{pousada estadual}
\end{EntryWithPhonetic}

\begin{EntryWithPhonetic}{国产}{guo2chan3}{8,6}{⼞,⼇}[HSK 6]
  \definition{adj.}{doméstico; feito na China; produzido internamente, especificamente na China}
\end{EntryWithPhonetic}

\begin{EntryWithPhonetic}{国防}{guo2fang2}{8,6}{⼞,⾩}[HSK 7-9]
  \definition{s.}{defesa nacional; as instalações humanas, materiais e militares que um país possui para defender sua soberania territorial e impedir invasões estrangeiras}
\end{EntryWithPhonetic}

\begin{EntryWithPhonetic}{国歌}{guo2ge1}{8,14}{⼞,⽋}[HSK 6]
  \definition[首,支]{s.}{hino nacional; o hino nacional da China, oficialmente designado pelo estado como a música que representa o país, é ``Marcha dos Voluntários''}
\end{EntryWithPhonetic}

\begin{EntryWithPhonetic}{国画}{guo2hua4}{8,8}{⼞,⽥}[HSK 7-9]
  \definition[幅,张,卷]{s.}{pintura tradicional chinesa | arte chinesa | pintura nacional}
\end{EntryWithPhonetic}

\begin{EntryWithPhonetic}{国徽}{guo2hui1}{8,17}{⼞,⼻}[HSK 7-9]
  \definition{s.}{emblema nacional; o emblema nacional da China, oficialmente designado pelo estado para representar o país, apresenta a Praça da Paz Celestial sob o céu brilhante de cinco estrelas, cercada por espigas de grãos e engrenagens}
\end{EntryWithPhonetic}

\begin{EntryWithPhonetic}{国会}{guo2hui4}{8,6}{⼞,⼈}[HSK 6]
  \definition{s.}{parlamento; congresso}
\end{EntryWithPhonetic}

\begin{EntryWithPhonetic}{国籍}{guo2ji2}{8,20}{⼞,⽵}[HSK 5]
  \definition[个]{s.}{nacionalidade; cidadania; refere"-se à identidade de um indivíduo como pertencente a um Estado | identidade nacional (de um avião, navio, etc.)}
\end{EntryWithPhonetic}

\begin{EntryWithPhonetic}{国际}{guo2ji4}{8,7}{⼞,⾩}[HSK 2]
  \definition{adj.}{internacional; entre países; entre nações}
  \definition{s.}{internacional; o mundo; entre nações; entre países de todo o mundo}
\end{EntryWithPhonetic}

\begin{EntryWithPhonetic}{国际儿童节}{guo2ji4 er2tong2jie2}{8,7,2,12,5}{⼞,⾩,⼉,⽴,⾋}
  \definition*{s.}{Dia Internacional das Crianças (1 de junho)}
\end{EntryWithPhonetic}

\begin{EntryWithPhonetic}{国际妇女节}{guo2ji4 fu4nv3jie2}{8,7,6,3,5}{⼞,⾩,⼥,⼥,⾋}
  \definition*{s.}{Dia Internacional das Mulheres (8 de março)}
\end{EntryWithPhonetic}

\begin{EntryWithPhonetic}{国际劳动节}{guo2ji4 lao2dong4 jie2}{8,7,7,6,5}{⼞,⾩,⼒,⼒,⾋}
  \definition*{s.}{Dia Internacional dos Trabalhadores (1 de maio)}
\end{EntryWithPhonetic}

\begin{EntryWithPhonetic}{国家}{guo2jia1}{8,10}{⼞,⼧}[HSK 1]
  \definition[个]{s.}{país; estado; nação; um lugar reconhecido internacionalmente e com soberania independente, incluindo as pessoas e as instituições administrativas desse lugar}
\end{EntryWithPhonetic}

\begin{EntryWithPhonetic}{国民}{guo2min2}{8,5}{⼞,⽒}[HSK 5]
  \definition{adj.}{nacional}
  \definition[个]{s.}{membro de uma nação; povo de uma nação}
\end{EntryWithPhonetic}

\begin{EntryWithPhonetic}{国内}{guo2nei4}{8,4}{⼞,⼌}[HSK 3]
  \definition{s.}{interno (a um país); doméstico; lar; dentro de um determinado país}
\end{EntryWithPhonetic}

\begin{EntryWithPhonetic}{国旗}{guo2qi2}{8,14}{⼞,⽅}[HSK 6]
  \definition[面]{s.}{bandeira (de um país)}
\end{EntryWithPhonetic}

\begin{EntryWithPhonetic}{国情}{guo2qing2}{8,11}{⼞,⼼}[HSK 7-9]
  \definition{s.}{condição (ou estado) do país; condições nacionais; as condições e características básicas da natureza social, política, economia, cultura etc. de um país também se referem especificamente às condições e características básicas de um país em um determinado período de tempo}
\end{EntryWithPhonetic}

\begin{EntryWithPhonetic}{国庆}{guo2qing4}{8,6}{⼞,⼴}[HSK 3]
  \definition*{s.}{Dia Nacional, o dia em que um país comemora sua independência ou fundação}
\end{EntryWithPhonetic}

\begin{EntryWithPhonetic}{国庆节}{guo2qing4jie2}{8,6,5}{⼞,⼴,⾋}
  \definition*{s.}{Dia Nacional (1~de~outubro)}
\end{EntryWithPhonetic}

\begin{EntryWithPhonetic}{国人}{guo2ren2}{8,2}{⼞,⼈}
  \definition{s.}{compatriota}
\end{EntryWithPhonetic}

\begin{EntryWithPhonetic}{国土}{guo2tu3}{8,3}{⼞,⼟}[HSK 7-9]
  \definition{s.}{terra; território; território nacional}
\end{EntryWithPhonetic}

\begin{EntryWithPhonetic}{国外}{guo2wai4}{8,5}{⼞,⼣}[HSK 1]
  \definition{adj.}{externo; no exterior; fora do país; outros lugares fora do país; geralmente chamados de exterior;  exterior não é o mesmo que estrangeiro}
\end{EntryWithPhonetic}

\begin{EntryWithPhonetic}{国王}{guo2wang2}{8,4}{⼞,⽟}[HSK 6]
  \definition[位,名,个,些]{s.}{rei; soberanos; o governante supremo de algumas monarquias antigas; nos tempos modernos, refere"-se ao chefe de estado de algumas monarquias}
\end{EntryWithPhonetic}

\begin{EntryWithPhonetic}{国学}{guo2xue2}{8,8}{⼞,⼦}[HSK 7-9]
  \definition*{s.}{Arcaico: O Colégio Imperial}
  \definition{s.}{estudos da cultura clássica chinesa (história, filosofia, literatura, língua, etc.) | cultura nacional chinesa | estudos da antiga civilização chinesa}
\end{EntryWithPhonetic}

\begin{EntryWithPhonetic}{国有}{guo2you3}{8,6}{⼞,⽉}[HSK 7-9]
  \definition{v.}{pertencer ao estado; ser nacionalizado}
\end{EntryWithPhonetic}

\begin{EntryWithPhonetic}{国语}{guo2yu3}{8,9}{⼞,⾔}
  \definition*{s.}{Língua Chinesa (Mandarim), enfatizando sua natureza nacional}
\end{EntryWithPhonetic}

%%%%%%%%%% 果 %%%%%%%%%%
\subsection*{果}\addcontentsline{loh}{figure}{果 \dpy{guo3}}

\begin{EntryWithPhonetic}{果}{guo3}{8}{⽊}
  \definition*{s.}{Sobrenome: Guo}
  \definition{adj.}{resoluto; determinado; sem exitação}
  \definition{adv.}{realmente; como esperado; com certeza; isso significa que as coisas são consistentes com as expectativas, equivalente a 果然}
  \definition{conj.}{se realmente; se de fato}
  \definition[个,些,种]{s.}{fruta; fruto da planta | resultado; consequência; o resultado final de um assunto}
  \seealsoref{果然}{guo3ran2}
  \antonymref{因}{yin1}
\end{EntryWithPhonetic}

\begin{EntryWithPhonetic}{果断}{guo3duan4}{8,11}{⽊,⽄}[HSK 7-9]
  \definition{adj.}{resoluto; decisivo; agir decisivamente sem hesitação}
\end{EntryWithPhonetic}

\begin{EntryWithPhonetic}{果酱}{guo3jiang4}{8,13}{⽊,⾣}[HSK 6]
  \definition{s.}{geléia | compota ou doce (de frutas); fruta em conserva}
\end{EntryWithPhonetic}

\begin{EntryWithPhonetic}{果然}{guo3ran2}{8,12}{⽊,⽕}[HSK 3]
  \definition{adv.}{realmente; como esperado; com certeza; indica que os fatos correspondem ao que foi dito ou esperado}
  \definition{conj.}{se realmente; se de fato; suponha que os fatos correspondam ao que foi dito ou esperado}
\end{EntryWithPhonetic}

\begin{EntryWithPhonetic}{果实}{guo3shi2}{8,8}{⽊,⼧}[HSK 4]
  \definition[种]{s.}{fruta; o órgão que se desenvolve a partir do ovário ou com outras partes da flor após a fertilização da flor | ganhos; frutos;  uma metáfora para conquista ou recompensa por trabalho árduo}
\end{EntryWithPhonetic}

\begin{EntryWithPhonetic}{果树}{guo3shu4}{8,9}{⽊,⽊}[HSK 6]
  \definition[棵,个,片]{s.}{árvore frutífera; árvores cujos frutos são principalmente comestíveis, como pessegueiros e macieiras}
\end{EntryWithPhonetic}

\begin{EntryWithPhonetic}{果园}{guo3yuan2}{8,7}{⽊,⼞}[HSK 7-9]
  \definition[个,座]{s.}{pomar; um jardim onde são plantadas árvores frutíferas}
\end{EntryWithPhonetic}

\begin{EntryWithPhonetic}{果真}{guo3zhen1}{8,10}{⽊,⼗}[HSK 7-9]
  \definition{adv.}{realmente; como esperado; com certeza}
  \definition{conj.}{se de fato; se realmente; se for o caso}[果真如此, 我就放心了。===Se for esse o caso, então ficarei aliviado.]
\end{EntryWithPhonetic}

\begin{EntryWithPhonetic}{果汁}{guo3zhi1}{8,5}{⽊,⽔}[HSK 3]
  \definition[杯,瓶,种]{s.}{suco; suco de frutas frescas; também se refere a bebidas feitas com suco de frutas frescas}
\end{EntryWithPhonetic}

\begin{EntryWithPhonetic}{果子}{guo3zi5}{8,3}{⽊,⼦}
  \definition{s.}{fruta}
\end{EntryWithPhonetic}

%%%%%%%%%% 裹 %%%%%%%%%%
\subsection*{裹}\addcontentsline{loh}{figure}{裹 \dpy{guo3}}

\begin{EntryWithPhonetic}{裹}{guo3}{14}{⾐}[HSK 7-9]
  \definition{s.}{pacote; encomenda}
  \definition{v.}{amarrar; embrulhar; envolver | levar embora; varrer com violência | Dialeto: sugar (leite) | pressionar a servir; fugir com (algo)}
\end{EntryWithPhonetic}

%%%%%%%%%% 过 %%%%%%%%%%
\subsection*{过}\addcontentsline{loh}{figure}{过 \dpy{guo4}}

\begin{EntryWithPhonetic}{过}{guo4}{6}{⾡}[HSK 1,2]
  \definition{adv.}{excessivamente; em excesso}
  \definition{clas.}{tempo; número de vezes usado para a ação}
  \definition{s.}{falha; erro; demérito; equívoco; negligência}
  \definition{v.}{cruzar; passar; mudar-se de um lugar para outro; passar por | exceder; ir além; ultrapassar; usado após um adjetivo, significa ``mais do que'' | gastar (tempo); passar (tempo); exceder (um determinado limite ou limite) | celebrar; comemorar | mudar; transferir; transferir de um lado para o outro | passar por um processo; passar por; submeter a (algum tipo de tratamento) | visitar; fazer uma visita | falecer; morrer | infectar; ser contagioso; espalhar | exceder; ir além; usado após o verbo com o sufixo 得, significa ``superar'' ou ``passar'' | viver | revisar; examinar; usar os olhos para ver ou a mente para lembrar}
  \seeref{guo1}
  \seeref{guo5}
  \seealsoref{得}{de5}
  \antonymref{功}{gong1}
\end{EntryWithPhonetic}

\begin{EntryWithPhonetic}{过半}{guo4ban4}{6,5}{⾡,⼗}[HSK 7-9]
  \definition{s.}{maioria; mais da metade; mais de cinquenta por cento}
\end{EntryWithPhonetic}

\begin{EntryWithPhonetic}{过不惯}{guo4 bu5 guan4}{6,4,11}{⾡,⼀,⼼}
  \definition{v.}{não se acostumar; não se habituar}
  \seealsoref{过惯}{guo4guan4}
\end{EntryWithPhonetic}

\begin{EntryWithPhonetic}{过不去}{guo4bu5qu4}{6,4,5}{⾡,⼀,⼛}[HSK 7-9]
  \definition{v.}{não poder passar; ser incapaz de passar; ser ou estar bloqueado; ser intransitável | Coloquial: ser duro com; dificultar; envergonhar; colocar para fora | sentir pena; sentir-se mal | encontrar falhas em}
\end{EntryWithPhonetic}

\begin{EntryWithPhonetic}{过程}{guo4cheng2}{6,12}{⾡,⽲}[HSK 3]
  \definition[个,段]{s.}{curso dos eventos; processo; o processo pelo qual as coisas acontecem ou se desenvolvem.}
\end{EntryWithPhonetic}

\begin{EntryWithPhonetic}{过错}{guo4cuo4}{6,13}{⾡,⾦}[HSK 7-9]
  \definition{s.}{falha; erro; engano | ações ilícitas; no direito civil, refere"-se a atos ilegais que prejudicam outras pessoas intencionalmente ou negligentemente}
\end{EntryWithPhonetic}

\begin{EntryWithPhonetic}{过道}{guo4dao4}{6,12}{⾡,⾡}[HSK 7-9]
  \definition{s.}{corredor; caminho; passarela; passagem; o corredor da porta para cada cômodo da nova casa | passagem; uma passarela que conecta os pátios de uma casa antiga, especialmente o cômodo ou metade do cômodo onde o portão está localizado}
\end{EntryWithPhonetic}

\begin{EntryWithPhonetic}{过度}{guo4du4}{6,9}{⾡,⼴}[HSK 5]
  \definition{adj.}{excessivo; acima do limite; além do limite; além do que é apropriado}
\end{EntryWithPhonetic}

\begin{EntryWithPhonetic}{过渡}{guo4du4}{6,12}{⾡,⽔}[HSK 6]
  \definition{v.}{fazer a transição; estar em transição; estar em fase de transição; mudar de um estágio para outro | atravessar; cruzar}
\end{EntryWithPhonetic}

\begin{EntryWithPhonetic}{过分}{guo4fen4}{6,4}{⾡,⼑}[HSK 4]
  \definition{adj.}{excessivo; muito longe; demais; falar ou agir além dos limites ou graus adequados}
  \definition{adv.}{excessivamente; indevidamente; muito mesmo}
\end{EntryWithPhonetic}

\begin{EntryWithPhonetic}{过关}{guo4/guan1}{6,6}{⾡,⼋}[HSK 7-9]
  \definition{v.+compl.}{passar (um teste); alcançar (um padrão); cruzar uma barreira; superar (uma provação) ; passar por um posto de controle, frequentemente usado como metáfora}
\end{EntryWithPhonetic}

\begin{EntryWithPhonetic}{过惯}{guo4guan4}{6,11}{⾡,⼼}
  \definition{v.}{estar acostumado (a um certo estilo de vida, etc.)}
  \seealsoref{过不惯}{guo4 bu5 guan4}
\end{EntryWithPhonetic}

\begin{EntryWithPhonetic}{过后}{guo4hou4}{6,6}{⾡,⼝}[HSK 6]
  \definition[期]{s.}{depois; mais tarde}
\end{EntryWithPhonetic}

\begin{EntryWithPhonetic}{过奖}{guo4jiang3}{6,9}{⾡,⼤}[HSK 7-9]
  \definition{v.}{elogiar demais; bajular; dar elogios imerecidos}
\end{EntryWithPhonetic}

\begin{EntryWithPhonetic}{过节}{guo4/jie2}{6,5}{⾡,⾋}[HSK 7-9]
  \definition{v.+compl.}{celebrar um festival; passar as férias; comemorar durante as férias}[今年我们一起过节吧!===Vamos comemorar as festas juntos este ano!]
\end{EntryWithPhonetic}

\begin{EntryWithPhonetic}{过境}{guo4/jing4}{6,14}{⾡,⼟}[HSK 7-9]
  \definition{v.+compl.}{estar em trânsito; passar pelo território de um país}
\end{EntryWithPhonetic}

\begin{EntryWithPhonetic}{过来}{guo4/lai2}{6,7}{⾡,⽊}[HSK 2]
  \definition{v.+compl.}{vir até aqui | ser capaz de cuidar de | lidar com | administrar}
\end{EntryWithPhonetic}

\begin{EntryWithPhonetic}{过路人}{guo4lu4 ren2}{6,13,2}{⾡,⾜,⼈}
  \definition{s.}{transeunte}
\end{EntryWithPhonetic}

\begin{EntryWithPhonetic}{过滤}{guo4lv4}{6,13}{⾡,⽔}[HSK 7-9]
  \definition{v.}{filtrar; separar sólidos ou componentes nocivos de gases ou líquidos por meio de materiais porosos, como papel de filtro e pano de filtro}[所有饮用水必须经过过滤。===Toda água potável deve ser filtrada.]
\end{EntryWithPhonetic}

\begin{EntryWithPhonetic}{过敏}{guo4min3}{6,11}{⾡,⽁}[HSK 5]
  \definition{adj.}{sensível; excessivamente sensível; resposta acima do normal; ceticismo excessivo}
  \definition{v.}{ser alérgico a}
\end{EntryWithPhonetic}

\begin{EntryWithPhonetic}{过年}{guo4/nian2}{6,6}{⾡,⼲}[HSK 2]
  \definition{v.+compl.}{comemorar o Ano Novo; comemorar o Festival da Primavera; passar o Ano Novo; passar o Festival da Primavera; realizar atividades comemorativas durante o Ano Novo ou o Festival da Primavera}
\end{EntryWithPhonetic}

\begin{EntryWithPhonetic}{过期}{guo4/qi1}{6,12}{⾡,⽉}[HSK 7-9]
  \definition{v.+compl.}{expirar; estar vencido; exceder o limite de tempo; exceder o período prescrito ou acordado}
\end{EntryWithPhonetic}

\begin{EntryWithPhonetic}{过去}{guo4 qu5}{6,5}{⾡,⼛}[HSK 2,3]
  \definition{adv.}{(no) passado}
  \definition{s.}{o passado; refere"-se a um período anterior; também se refere a coisas anteriores}
  \definition{v.}{atravessar; passar; sair do local onde o interlocutor se encontra e deslocar"-se para outro local | acabar; passar; ficar para trás; indica que já passou por uma determinada fase | passar; indica que um determinado período ou situação já não existe mais | falecer | ir lá | passar por}
\end{EntryWithPhonetic}

\begin{EntryWithPhonetic}{过日子}{guo4 ri4zi5}{6,4,3}{⾡,⽇,⼦}[HSK 7-9]
  \definition{v.}{viver; conviver; passar/viver a própria vida}
\end{EntryWithPhonetic}

\begin{EntryWithPhonetic}{过剩}{guo4sheng4}{6,12}{⾡,⼑}[HSK 7-9]
  \definition{v.}{exceder; a quantidade excede em muito o limite necessário | saturar; oferecer em excesso; a oferta excede a demanda do mercado ou o poder de compra}
\end{EntryWithPhonetic}

\begin{EntryWithPhonetic}{过失}{guo4shi1}{6,5}{⾡,⼤}[HSK 7-9]
  \definition{s.}{falha; deslize; erro; erros cometidos por negligência | negligência; crime por negligência}
\end{EntryWithPhonetic}

\begin{EntryWithPhonetic}{过时}{guo4 shi2}{6,7}{⾡,⽇}[HSK 6]
  \definition{adj.}{fora de moda; obsoleto; antiquado; desatualizado; o que era popular no passado não é mais popular}
  \definition{v.}{passar do tempo marcado; estar após o tempo estipulado}
\end{EntryWithPhonetic}

\begin{EntryWithPhonetic}{过头}{guo4/tou2}{6,5}{⾡,⼤}[HSK 7-9]
  \definition{adv.}{excessivamente; acima da cabeça; por cima; ao alto}
  \definition{v.+compl.}{exagerar; ir além do limite; exceder o limite; ser excessivo}
\end{EntryWithPhonetic}

\begin{EntryWithPhonetic}{过屠门而大嚼}{guo4 tu2men2 er2 da4 jiao2}{6,11,3,6,3,20}{⾡,⼫,⾨,⽽,⼤,⼝}
  \definition{expr.}{``Passar pelo portão do açougueiro e comer com apetite.''; essa metáfora descreve alguém que admira algo, mas não pode tê-lo, e usa métodos irreais para se consolar; alimente-se de ilusões; tem significado negativo}
\end{EntryWithPhonetic}

\begin{EntryWithPhonetic}{过往}{guo4wang3}{6,8}{⾡,⼻}[HSK 7-9]
  \definition{v.}{ir e vir | ter relações amigáveis com; associar"-se a; lidar com}
\end{EntryWithPhonetic}

\begin{EntryWithPhonetic}{过意不去}{guo4yi4bu2qu4}{6,13,4,5}{⾡,⼼,⼀,⼛}[HSK 7-9]
  \definition{expr.}{sentir-se arrependido ou culpado; sentir-se mal ou envergonhado; sentir-se envergonhado ou arrependido; metáfora para aceitar um favor de alguém, mas não retribuí-lo, ou sentir pena de algo e não ser culpado, o que faz com que alguém se sinta arrependido e desconfortável}
\end{EntryWithPhonetic}

\begin{EntryWithPhonetic}{过瘾}{guo4/yin3}{6,16}{⾡,⽧}[HSK 7-9]
  \definition{adj.}{gratificante; imensamente agradável; satisfatório; realizador}
  \definition{v.+compl.}{satisfazer um desejo; divertir-se ao máximo; fazer algo à vontade}
\end{EntryWithPhonetic}

\begin{EntryWithPhonetic}{过硬}{guo4/ying4}{6,12}{⾡,⽯}[HSK 7-9]
  \definition{adj.}{perfeito; soberbo; à altura; verdadeiramente proficiente}
  \definition{v.+compl.}{ter domínio perfeito de algo; estar à altura; resistir a testes ou exames rigorosos}
\end{EntryWithPhonetic}

\begin{EntryWithPhonetic}{过于}{guo4yu2}{6,3}{⾡,⼆}[HSK 5]
  \definition{adv.}{demais; indevidamente; excessivamente; advérbios de grau ou quantidade excessiva}
\end{EntryWithPhonetic}

\begin{EntryWithPhonetic}{过早}{guo4 zao3}{6,6}{⾡,⽇}[HSK 7-9]
  \definition{adj.}{prematuro; inoportuno | Dialeto: café da manha}
  \definition{adv.}{muito cedo; prematuramente | Dialeto: tomar o café da manhã}
\end{EntryWithPhonetic}

\begin{EntryWithPhonetic}{过}{guo5}{6}{⾡}
  \definition{part.}{usado depois de um verbo para indicar conclusão | usado depois de um verbo para indicar que uma ação ou mudança ocorreu | usado depois de um adjetivo para indicar que algo já teve uma certa qualidade ou estado e para compará-lo com o presente}
  \seeref{guo1}
  \seeref{guo4}
\end{EntryWithPhonetic}

%%%%% EOF %%%%%

