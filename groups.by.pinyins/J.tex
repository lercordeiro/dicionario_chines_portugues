%%%
%%% J
%%%
\section*{J}\addcontentsline{toc}{section}{J}\addcontentsline{loh}{figure}{\#\#\#\#\#\#\#\# J}

%%%%%%%%%% 几 %%%%%%%%%%
\subsection*{几}\addcontentsline{loh}{figure}{几 \dpy{ji1}}

\begin{EntryWithPhonetic}{几}{ji1}{2}{⼏}[Kangxi 16]
  \definition{adv.}{quase; praticamente}
  \definition{s.}{uma mesa pequena}
  \seeref{ji3}
\end{EntryWithPhonetic}

\begin{EntryWithPhonetic}{几乎}{ji1hu1}{2,5}{⼏,⼃}[HSK 4]
  \definition{adv.}{quase; praticamente; próximo a | perto de; quase; à beira de}
\end{EntryWithPhonetic}

\begin{EntryWithPhonetic}{几率}{ji1lv4}{2,11}{⼏,⽞}[HSK 7-9]
  \definition{s.}{probabilidade; um evento pode ou não ocorrer sob as mesmas condições, a grandeza que indica a possibilidade de ocorrência é chamada de probabilidade}
\end{EntryWithPhonetic}

%%%%%%%%%% 讥 %%%%%%%%%%
\subsection*{讥}\addcontentsline{loh}{figure}{讥 \dpy{ji1}}

\begin{EntryWithPhonetic}{讥}{ji1}{4}{⾔}
  \definition{v.}{ridicularizar; zombar; satirizar}
\end{EntryWithPhonetic}

\begin{EntryWithPhonetic}{讥笑}{ji1xiao4}{4,10}{⾔,⽵}[HSK 7-9]
  \definition{v.}{ridicularizar; zombar; zombar de; tirar sarro de}
\end{EntryWithPhonetic}

%%%%%%%%%% 饥 %%%%%%%%%%
\subsection*{饥}\addcontentsline{loh}{figure}{饥 \dpy{ji1}}

\begin{EntryWithPhonetic}{饥}{ji1}{5}{⾷}
  \definition{adj.}{faminto; com fome}
  \definition[只]{s.}{fome; quebra de safra; colheita pobre ou inexistente}
\end{EntryWithPhonetic}

\begin{EntryWithPhonetic}{饥饿}{ji1'e4}{5,10}{⾷,⾷}[HSK 7-9]
  \definition{adj.}{faminto; esfomeado; estômago vazio; necessidade urgente de comer}[孩子们因饥饿哭泣。===As crianças choravam de fome.]
\end{EntryWithPhonetic}

%%%%%%%%%% 机 %%%%%%%%%%
\subsection*{机}\addcontentsline{loh}{figure}{机 \dpy{ji1}}

\begin{EntryWithPhonetic}{机}{ji1}{6}{⽊}
  \definition*{s.}{Sobrenome: Ji}
  \definition{adj.}{flexível; perspicaz; destreza; agilidade}
  \definition[台]{s.}{máquina; motor | avião; aeronave; aeroplano; refere"-se especificamente a aeronaves | ponto crucial; os fatores"-chave para a ocorrência e mudança das coisas | chance; ocasião; oportunidade; um momento crítico ou oportuno para o desenvolvimento e mudança das coisas | organismo; funções vitais dos organismos | besta; mecanismo de disparo de flechas de madeira em uma besta antiga | assuntos importantes; assuntos extremamente importantes e confidenciais | ideia; intenção}
\end{EntryWithPhonetic}

\begin{EntryWithPhonetic}{机舱}{ji1cang1}{6,10}{⽊,⾈}[HSK 7-9]
  \definition{s.}{sala de máquinas (de um navio) | compartimento de passageiros (de uma aeronave); cabine | espaço de máquinas; cabine de aeronave}
\end{EntryWithPhonetic}

\begin{EntryWithPhonetic}{机场}{ji1chang3}{6,6}{⽊,⼟}[HSK 1]
  \definition[个,家,处,座]{s.}{aeródromo; campo de aviação; aeroporto; campo de voo}
\end{EntryWithPhonetic}

\begin{EntryWithPhonetic}{机动}{ji1dong4}{6,6}{⽊,⼒}[HSK 7-9]
  \definition{adj.}{motorizado; movido a energia | flexível; manobrável; conveniente; móvel | em reserva; para uso emergencial}
\end{EntryWithPhonetic}

\begin{EntryWithPhonetic}{机动车}{ji1dong4che1}{6,6,4}{⽊,⼒,⾞}[HSK 6]
  \definition{s.}{veículo motorizado | veículo automotor; automóvel de passageiros: veículo comercial concebido e tecnicamente adequado para o transporte de passageiros e respetiva bagagem, incluindo o banco do condutor}
  \antonymref{人力车}{ren2li4che1}
\end{EntryWithPhonetic}

\begin{EntryWithPhonetic}{机构}{ji1gou4}{6,8}{⽊,⽊}[HSK 4]
  \definition[所]{s.}{órgão; organização; instituição; instalações; aparelhamento; configuração | mecanismo; funcionamento interno de uma máquina ou unidade | estrutura interna de uma organização}
\end{EntryWithPhonetic}

\begin{EntryWithPhonetic}{机关}{ji1guan1}{6,6}{⽊,⼋}[HSK 6]
  \definition{adj.}{operado por máquina | controlado mecanicamente}
  \definition[个]{s.}{engrenagem; mecanismo; Antigo: refere"-se a certos dispositivos controlados mecanicamente; também se refere às peças de frenagem de dispositivos mecânicos | escritório; órgão; corpo; instituição | esquema; maquinação; estratagema; um plano cuidadoso e inteligente}
\end{EntryWithPhonetic}

\begin{EntryWithPhonetic}{机会}{ji1hui5}{6,6}{⽊,⼈}[HSK 2]
  \definition[个,次,种,些]{s.}{chance; oportunidade; momento favorável raro}
\end{EntryWithPhonetic}

\begin{EntryWithPhonetic}{机甲}{ji1jia3}{6,5}{⽊,⽥}
  \definition{s.}{\emph{mecha} (robôs operados por humanos em mangá japonês)}
\end{EntryWithPhonetic}

\begin{EntryWithPhonetic}{机灵}{ji1ling5}{6,7}{⽊,⽕}[HSK 7-9]
  \definition{adj.}{inteligente; esperto; astuto; espirituoso}
\end{EntryWithPhonetic}

\begin{EntryWithPhonetic}{机密}{ji1mi4}{6,11}{⽊,⼧}[HSK 7-9]
  \definition{adj.}{secreto; classificado; privado; confidencial}
  \definition{s.}{segredo; assuntos confidenciais}
\end{EntryWithPhonetic}

\begin{EntryWithPhonetic}{机票}{ji1piao4}{6,11}{⽊,⽰}[HSK 1]
  \definition[张]{s.}{passagem aérea; passagem de avião}
  \seealsoref{飞机票}{fei1ji1piao4}
\end{EntryWithPhonetic}

\begin{EntryWithPhonetic}{机器}{ji1qi4}{6,16}{⽊,⼝}[HSK 3]
  \definition[台,部,个]{s.}{máquina; maquinário; motor; dispositivos e máquinas que são montados a partir de peças, podem funcionar, transformar energia ou produzir trabalho útil podem ser usados como ferramentas de produção, reduzindo a intensidade do trabalho humano e aumentando a produtividade | aparato; sistema político e econômico}
\end{EntryWithPhonetic}

\begin{EntryWithPhonetic}{机器人}{ji1qi4ren2}{6,16,2}{⽊,⼝,⼈}[HSK 5]
  \definition[个,些]{s.}{androide; golem | pessoa mecânica | robô}
\end{EntryWithPhonetic}

\begin{EntryWithPhonetic}{机械}{ji1xie4}{6,11}{⽊,⽊}[HSK 6]
  \definition{adj.}{rígido; mecânico; inflexível; uma metáfora para uma abordagem rígida e imutável}
  \definition[台,部,个]{s.}{máquina; maquinário; mecanismo; vários dispositivos compostos por princípios mecânicos}
\end{EntryWithPhonetic}

\begin{EntryWithPhonetic}{机遇}{ji1yu4}{6,12}{⽊,⾡}[HSK 4]
  \definition[个]{s.}{chance; oportunidade; circunstâncias favoráveis}
\end{EntryWithPhonetic}

\begin{EntryWithPhonetic}{机制}{ji1zhi4}{6,8}{⽊,⼑}[HSK 5]
  \definition{s.}{mecanismo; processado por máquina; feito por máquina}
\end{EntryWithPhonetic}

\begin{EntryWithPhonetic}{机智}{ji1zhi4}{6,12}{⽊,⽇}[HSK 7-9]
  \definition{adj.}{engenhoso; perspicaz; inteligente e adaptável}
\end{EntryWithPhonetic}

%%%%%%%%%% 肌 %%%%%%%%%%
\subsection*{肌}\addcontentsline{loh}{figure}{肌 \dpy{ji1}}

\begin{EntryWithPhonetic}{肌}{ji1}{6}{⾁}
  \definition[块,片]{s.}{músculo; carne | pele;}
\end{EntryWithPhonetic}

\begin{EntryWithPhonetic}{肌肤}{ji1fu1}{6,8}{⾁,⾁}[HSK 7-9]
  \definition{s.}{músculo e pele (humano)}
\end{EntryWithPhonetic}

\begin{EntryWithPhonetic}{肌肉}{ji1rou4}{6,6}{⾁,⾁}[HSK 5]
  \definition[身,块]{s.}{músculo; um dos tecidos básicos dos músculos humanos e de alguns animais, composto principalmente de células musculares fibrosas, pode se contrair, é o movimento do corpo e o corpo de digestão, respiração, circulação, excreção e outros processos fisiológicos da fonte de energia; pode ser dividido em três tipos: músculo liso, músculo esquelético e músculo cardíaco}
\end{EntryWithPhonetic}

%%%%%%%%%% 鸡 %%%%%%%%%%
\subsection*{鸡}\addcontentsline{loh}{figure}{鸡 \dpy{ji1}}

\begin{EntryWithPhonetic}{鸡}{ji1}{7}{⿃}[HSK 2]
  \definition*{s.}{Sobrenome: Ji}
  \definition[只]{s.}{galo, galinha, frango | palavra ofensiva para uma mulher que ganha dinheiro fazendo sexo com homens}
\end{EntryWithPhonetic}

\begin{EntryWithPhonetic}{鸡蛋}{ji1dan4}{7,11}{⿃,⾍}[HSK 1]
  \definition[个,枚,筐,箱,打]{s.}{ovo de galinha}
\end{EntryWithPhonetic}

\begin{EntryWithPhonetic}{鸡西}{ji1xi1}{7,6}{⿃,⾑}
  \definition*{s.}{Cidade no nível da prefeitura de Jixi, na província de Heilongjiang (黑龙江), no nordeste da China}
  \seealsoref{黑龙江}{hei1long2jiang1}
\end{EntryWithPhonetic}

%%%%%%%%%% 积 %%%%%%%%%%
\subsection*{积}\addcontentsline{loh}{figure}{积 \dpy{ji1}}

\begin{EntryWithPhonetic}{积}{ji1}{10}{⽲}[HSK 7-9]
  \definition{adj.}{de longa data; pendente há muito tempo | antiquíssimo; acumulado ao longo de um longo período de tempo}
  \definition{s.}{Medicina chinesa: indigestão (em bebês e crianças) | Matemática: produro, abreviação de 乘积}
  \definition{v.}{acumular; juntar; amontoar; reunir; coletar}
  \seealsoref{乘积}{cheng2ji1}
\end{EntryWithPhonetic}

\begin{EntryWithPhonetic}{积淀}{ji1dian4}{10,11}{⽲,⽔}[HSK 7-9]
  \definition{s.}{acumulação | depósitos acumulados ao longo de longos períodos | Figurativo: experiência valiosa, sabedoria acumulada}
  \definition{v.}{acumular}
\end{EntryWithPhonetic}

\begin{EntryWithPhonetic}{积极}{ji1ji2}{10,7}{⽲,⽊}[HSK 3]
  \definition{adj.}{ativo; descreve uma atitude proativa e esforçada | positivo; que tem um efeito positivo e ajuda no desenvolvimento das coisas}
\end{EntryWithPhonetic}

\begin{EntryWithPhonetic}{积累}{ji1lei3}{10,11}{⽲,⽷}[HSK 4]
  \definition{s.}{acúmulo; acumulação}
  \definition{v.}{acumular}
\end{EntryWithPhonetic}

\begin{EntryWithPhonetic}{积木}{ji1mu4}{10,4}{⽲,⽊}
  \definition{s.}{blocos de montar (brinquedo)}
\end{EntryWithPhonetic}

\begin{EntryWithPhonetic}{积蓄}{ji1xu4}{10,13}{⽲,⾋}[HSK 7-9]
  \definition[笔]{s.}{poupança; economias}
  \definition{v.}{acumular; poupar; economizar}
\end{EntryWithPhonetic}

%%%%%%%%%% 基 %%%%%%%%%%
\subsection*{基}\addcontentsline{loh}{figure}{基 \dpy{ji1}}

\begin{EntryWithPhonetic}{基}{ji1}{11}{⼟}
  \definition{adj.}{chave; básico; primário; cardinal; fundamental}
  \definition{s.}{base; fundação | base; grupo; radical; (química) uma parte dos átomos contidos na molécula de um composto, quando considerada como uma unidade, é chamada de base}
\end{EntryWithPhonetic}

\begin{EntryWithPhonetic}{基本}{ji1ben3}{11,5}{⼟,⽊}[HSK 3]
  \definition{adj.}{básico; fundamental; elementar | principal}
  \definition{adv.}{basicamente; em geral; no geral; em termos gerais}
  \definition{s.}{fundação}
\end{EntryWithPhonetic}

\begin{EntryWithPhonetic}{基本法}{ji1ben3fa3}{11,5,8}{⼟,⽊,⽔}
  \definition{s.}{lei básica (constituição)}
\end{EntryWithPhonetic}

\begin{EntryWithPhonetic}{基本功}{ji1ben3gong1}{11,5,5}{⼟,⽊,⼒}[HSK 7-9]
  \definition{s.}{treinamento básico; habilidade básica; técnica essencial}
\end{EntryWithPhonetic}

\begin{EntryWithPhonetic}{基本上}{ji1ben3shang5}{11,5,3}{⼟,⽊,⼀}[HSK 3]
  \definition{adv.}{basicamente; principalmente | em geral; de modo geral}
\end{EntryWithPhonetic}

\begin{EntryWithPhonetic}{基层}{ji1ceng2}{11,7}{⼟,⼫}[HSK 7-9]
  \definition{s.}{base; nível local; nível básico; o nível mais baixo de qualquer organização, tem a conexão mais direta com as massas}
\end{EntryWithPhonetic}

\begin{EntryWithPhonetic}{基础}{ji1chu3}{11,10}{⼟,⽯}[HSK 3]
  \definition[个,种,点,层]{s.}{base; fundamento; fundação; a essência ou o ponto de partida do desenvolvimento das coisas | básico; fundamental; refere"-se às condições mínimas | fundação do edifício; base do edifício}
\end{EntryWithPhonetic}

\begin{EntryWithPhonetic}{基地}{ji1di4}{11,6}{⼟,⼟}[HSK 5]
  \definition{s.}{base; como base para alguns negócios | base; um local dedicado à realização de um negócio}
\end{EntryWithPhonetic}

\begin{EntryWithPhonetic}{基督教}{ji1du1jiao4}{11,13,11}{⼟,⽬,⽁}[HSK 6]
  \definition*{s.}{Cristianismo; A Religião Cristã | Cristão}
\end{EntryWithPhonetic}

\begin{EntryWithPhonetic}{基金}{ji1jin1}{11,8}{⼟,⾦}[HSK 5]
  \definition[只,笔]{s.}{fundo; fundos reservados ou destinados ao estabelecimento ou desenvolvimento de uma empresa}
\end{EntryWithPhonetic}

\begin{EntryWithPhonetic}{基因}{ji1yin1}{11,6}{⼟,⼞}[HSK 7-9]
  \definition[段,个]{s.}{gene; unidade básica de um organismo que carrega e transmite informações genéticas; está localizado nos cromossomos do núcleo da célula}
\end{EntryWithPhonetic}

\begin{EntryWithPhonetic}{基于}{ji1yu2}{11,3}{⼟,⼆}[HSK 7-9]
  \definition{prep.}{devido a; por causa de; em vista de; apresentando a premissa ou base de uma ação}
\end{EntryWithPhonetic}

\begin{EntryWithPhonetic}{基准}{ji1zhun3}{11,10}{⼟,⼎}[HSK 7-9]
  \definition{s.}{referência; base; padrão inicial para medição | um critério; um padrão; uma referência; refere"-se a padrões básicos}
\end{EntryWithPhonetic}

%%%%%%%%%% 畸 %%%%%%%%%%
\subsection*{畸}\addcontentsline{loh}{figure}{畸 \dpy{ji1}}

\begin{EntryWithPhonetic}{畸}{ji1}{13}{⽥}
  \definition{adj.}{assimétrico; desequilibrado | irregular; anormal}
  \definition{s.}{Literário: uma quantidade fracionária (acima daquela mencionada em um número redondo); lotes ímpares | deformidade}
\end{EntryWithPhonetic}

\begin{EntryWithPhonetic}{畸形}{ji1xing2}{13,7}{⽥,⼺}[HSK 7-9]
  \definition{adj.}{Medicina: deformado; malformado | desequilibrado; torto; desigual; anormal | deformidade; dismorfia; dismorfose; malformação; disgenesia; monstruosidade; aberração; desenvolvimento anormal de uma parte}
\end{EntryWithPhonetic}

%%%%%%%%%% 激 %%%%%%%%%%
\subsection*{激}\addcontentsline{loh}{figure}{激 \dpy{ji1}}

\begin{EntryWithPhonetic}{激}{ji1}{16}{⽔}
  \definition*{s.}{Sobrenome: Ji}
  \definition{adj.}{afiado; feroz; violento | vívido}
  \definition{adv.}{bruscamente; ferozmente; violentamente}
  \definition{s.}{o impacto de ondas fortes contra a costa}
  \definition{v.}{bater; avançar; correr | despertar; estimular; incitar; excitar | ficar doente por se molhar | esfriar (colocando água gelada, etc.)}
\end{EntryWithPhonetic}

\begin{EntryWithPhonetic}{激动}{ji1dong4}{16,6}{⽔,⼒}[HSK 4]
  \definition{adj.}{animado; entusiasmado; empolgado}
  \definition{v.}{agitar; excitar; tornar fortes os sentimentos de alguém}
\end{EntryWithPhonetic}

\begin{EntryWithPhonetic}{激发}{ji1fa1}{16,5}{⽔,⼜}[HSK 7-9]
  \definition{v.}{despertar; desencadear; estimular; motivar; inspirar | excitar; mudar moléculas e átomos de um estado de energia mais baixo para um estado de energia mais alto}
\end{EntryWithPhonetic}

\begin{EntryWithPhonetic}{激光}{ji1guang1}{16,6}{⽔,⼉}[HSK 7-9]
  \definition*{s.}{LASER; \emph{Light Amplification by Stimulated Emission of Radiation}; amplificação de luz por emissão estimulada de radiação}
\end{EntryWithPhonetic}

\begin{EntryWithPhonetic}{激化}{ji1hua4}{16,4}{⽔,⼔}[HSK 7-9]
  \definition{v.}{aguçar; intensificar; tornar agudo}
\end{EntryWithPhonetic}

\begin{EntryWithPhonetic}{激活}{ji1huo2}{16,9}{⽔,⽔}[HSK 7-9]
  \definition{v.}{ativar; estimular certas substâncias no corpo para torná-las ativas | colocar em jogo; revigorar; estimular metaforicamente, influenciar algo, torná-lo ativo}
\end{EntryWithPhonetic}

\begin{EntryWithPhonetic}{激励}{ji1li4}{16,7}{⽔,⼒}[HSK 7-9]
  \definition{v.}{instar; impelir; inspirar; encorajar; usar palavras ou ações de outras pessoas para encorajar as pessoas a trabalhar mais e fazer melhor | dirigir; excitar; estimular ou excitar uma reação ou atividade}
\end{EntryWithPhonetic}

\begin{EntryWithPhonetic}{激烈}{ji1lie4}{16,10}{⽔,⽕}[HSK 4]
  \definition{adj.}{agudo; afiado; feroz; violento; intenso}
\end{EntryWithPhonetic}

\begin{EntryWithPhonetic}{激起}{ji1qi3}{16,10}{⽔,⾛}[HSK 7-9]
  \definition{v.}{excitar; agitar; despertar; evocar; desencadear}
\end{EntryWithPhonetic}

\begin{EntryWithPhonetic}{激情}{ji1qing2}{16,11}{⽔,⼼}[HSK 6]
  \definition{s.}{paixão; emoções fortes e explosivas, como êxtase, raiva, etc.}
\end{EntryWithPhonetic}

\begin{EntryWithPhonetic}{激素}{ji1su4}{16,10}{⽔,⽷}[HSK 7-9]
  \definition{s.}{Fisiológico: hormônio; uma substância química que regula a atividade celular}
\end{EntryWithPhonetic}

%%%%%%%%%% 鷄 %%%%%%%%%%
\subsection*{鷄}\addcontentsline{loh}{figure}{鷄 \dpy{ji1}}

\begin{EntryWithPhonetic}{鷄}{ji1}{21}{⿃}
  \variantof{鸡}
\end{EntryWithPhonetic}

%%%%%%%%%% 及 %%%%%%%%%%
\subsection*{及}\addcontentsline{loh}{figure}{及 \dpy{ji2}}

\begin{EntryWithPhonetic}{及}{ji2}{3}{⼃}[HSK 7-9]
  \definition*{s.}{Sobrenome: Ji}
  \definition{conj.}{e; bem como; conectando substantivos paralelos ou frases nominais}
  \definition{v.}{alcançar; chegar até | ser comparável a; alcançar (geralmente usado em termos negativos) | chegar a tempo para | estender-se a; cuidar de; envolver | dar}
\end{EntryWithPhonetic}

\begin{EntryWithPhonetic}{及格}{ji2/ge2}{3,10}{⼃,⽊}[HSK 4]
  \definition{v.+compl.}{passar; passar em um teste, exame, etc.}
  \synonymref{合格}{he2ge2}
\end{EntryWithPhonetic}

\begin{EntryWithPhonetic}{及其}{ji2 qi2}{3,8}{⼃,⼋}[HSK 7-9]
  \definition{conj.}{(conjunção que liga dois substantivos) e seu\dots.; e seus\dots.; e dele\dots.; e dela\dots; usado para conectar duas ou mais coisas para indicar que elas são de igual importância ou existem da mesma maneira}[文化及其发展影响社会。===A cultura e seu desenvolvimento influenciam a sociedade.]
\end{EntryWithPhonetic}

\begin{EntryWithPhonetic}{及时}{ji2shi2}{3,7}{⼃,⽇}[HSK 3]
  \definition{adj.}{oportuno; na hora certa; adequado; na ocasião certa}
  \definition{adv.}{prontamente; sem demora; imediatamente}
  \synonymref{随时}{sui2shi2}
  \antonymref{耽误}{dan1wu5}
\end{EntryWithPhonetic}

\begin{EntryWithPhonetic}{及早}{ji2zao3}{3,6}{⼃,⽇}[HSK 7-9]
  \definition{adv.}{o mais cedo possível; antes que seja tarde demais}
  \synonymref{趁早}{chen4zao3}
  \synonymref{赶早}{gan3zao3}
\end{EntryWithPhonetic}

%%%%%%%%%% 吉 %%%%%%%%%%
\subsection*{吉}\addcontentsline{loh}{figure}{吉 \dpy{ji2}}

\begin{EntryWithPhonetic}{吉}{ji2}{6}{⼝}
  \definition*{s.}{Província de Jilin, abreviação de 吉林 | Sobrenome: Ji}
  \definition{adj.}{sortudo; propício; auspicioso}
  \seealsoref{吉林}{ji2lin2}
  \antonymref{凶}{xiong1}
\end{EntryWithPhonetic}

\begin{EntryWithPhonetic}{吉利}{ji2li4}{6,7}{⼝,⼑}[HSK 6]
  \definition{adj.}{sortudo; auspicioso; propício}
\end{EntryWithPhonetic}

\begin{EntryWithPhonetic}{吉林}{ji2lin2}{6,8}{⼝,⽊}
  \definition*{s.}{Província de Jilin}
\end{EntryWithPhonetic}

\begin{EntryWithPhonetic}{吉普}{ji2pu3}{6,12}{⼝,⽇}[HSK 7-9]
  \definition*{s.}{Jeep (marca de carro)}
  \definition[辆]{s.}{Empéstimo linguístico: jipe}[他开着吉普车去沙漠旅行。===Ele fez uma viagem pelo deserto em seu jipe.]
  \seealsoref{吉普车}{ji2pu3che1}
\end{EntryWithPhonetic}

\begin{EntryWithPhonetic}{吉普车}{ji2pu3che1}{6,12,4}{⼝,⽇,⾞}
  \definition[辆]{s.}{Empréstimo linguístico: jipe (veículo militar)}
  \seealsoref{吉普}{ji2pu3}
\end{EntryWithPhonetic}

\begin{EntryWithPhonetic}{吉他}{ji2ta1}{6,5}{⼝,⼈}[HSK 7-9]
  \definition[把]{s.}{Empréstimo linguístico: violão}
\end{EntryWithPhonetic}

\begin{EntryWithPhonetic}{吉祥}{ji2xiang2}{6,10}{⼝,⽰}[HSK 6]
  \definition{adj.}{sortudo; auspicioso; propício}
  \definition[个,种]{s.}{sorte; auspiciosidade; propiciação; um sinal ou símbolo de boa sorte ou fortuna}
\end{EntryWithPhonetic}

\begin{EntryWithPhonetic}{吉祥物}{ji2xiang2wu4}{6,10,8}{⼝,⽰,⽜}[HSK 7-9]
  \definition{s.}{mascote}
\end{EntryWithPhonetic}

%%%%%%%%%% 级 %%%%%%%%%%
\subsection*{级}\addcontentsline{loh}{figure}{级 \dpy{ji2}}

\begin{EntryWithPhonetic}{级}{ji2}{6}{⽷}[HSK 2]
  \definition{clas.}{usado para degraus, escadas, pisos de torres, etc.}
  \definition[个,种]{s.}{nível; classificação; grau; classe | série; turma; qualquer uma das divisões anuais de um curso escolar | degrau}
\end{EntryWithPhonetic}

\begin{EntryWithPhonetic}{级别}{ji2bie2}{6,7}{⽷,⼑}[HSK 7-9]
  \definition[个]{s.}{classificação; nível; escala; a ordem da hierarquia devido às diferenças de identidade, \emph{status} e características}
\end{EntryWithPhonetic}

%%%%%%%%%% 即 %%%%%%%%%%
\subsection*{即}\addcontentsline{loh}{figure}{即 \dpy{ji2}}

\begin{EntryWithPhonetic}{即}{ji2}{7}{⼙}[HSK 7-9]
  \definition{adv.}{no presente; no futuro imediato | prontamente; imediatamente}
  \definition{conj.}{e | mesmo; mesmo que}[她瞟了一眼睡着的孩子,随即匆匆离开了。===Ela olhou para a criança adormecida e então saiu correndo. | 即使下雨我也去。===Eu irei mesmo que chova.]
  \definition{v.}{aproximar-se; alcançar; estar perto | assumir; ascender a; aceitar; começar a se envolver em | ser motivado pela ocasião | estar perto | Literário: ser; significar}
\end{EntryWithPhonetic}

\begin{EntryWithPhonetic}{即便}{ji2bian4}{7,9}{⼙,⼈}[HSK 7-9]
  \definition{conj.}{mesmo; mesmo que; embora; mesmo que isso signifique que, em uma situação hipotética ou extrema, o resultado não mudará}
\end{EntryWithPhonetic}

\begin{EntryWithPhonetic}{即便是}{ji2bian4shi4}{7,9,9}{⼙,⼈,⽇}
  \definition{conj.}{mesmo que seja}
\end{EntryWithPhonetic}

\begin{EntryWithPhonetic}{即或}{ji2huo4}{7,8}{⼙,⼽}
  \definition{conj.}{mesmo se/embora}
\end{EntryWithPhonetic}

\begin{EntryWithPhonetic}{即将}{ji2jiang1}{7,9}{⼙,⼨}[HSK 4]
  \definition{adv.}{em breve; estar prestes a; estar a ponto de}
\end{EntryWithPhonetic}

\begin{EntryWithPhonetic}{即可}{ji2ke3}{7,5}{⼙,⼝}[HSK 7-9]
  \definition{adv.}{equivalente a; pode então; pode imediatamente; e isso será suficiente (fazer algo); significa 就可以了}
  \seealsoref{就可以了}{jiu4 ke3yi3le5}
\end{EntryWithPhonetic}

\begin{EntryWithPhonetic}{即若}{ji2ruo4}{7,8}{⼙,⾋}
  \definition{conj.}{mesmo se/embora}
\end{EntryWithPhonetic}

\begin{EntryWithPhonetic}{即使}{ji2shi3}{7,8}{⼙,⼈}[HSK 5]
  \definition{conj.}{mesmo; mesmo que; mesmo se; apesar de; expressando uma concessão hipotética}
\end{EntryWithPhonetic}

\begin{EntryWithPhonetic}{即是}{ji2shi4}{7,9}{⼙,⽇}
  \definition{conj.}{aquilo é}
\end{EntryWithPhonetic}

%%%%%%%%%% 极 %%%%%%%%%%
\subsection*{极}\addcontentsline{loh}{figure}{极 \dpy{ji2}}

\begin{EntryWithPhonetic}{极}{ji2}{7}{⽊}[HSK 4]
  \definition*{s.}{Sobrenome: Ji}
  \definition{adj.}{máximo; extremo; final; supremo}
  \definition{adv.}{extremamente; excessivamente}
  \definition{s.}{o ponto máximo, mais alto; extremo; ápice; ponto culminante | pólo; as extremidades norte e sul da Terra; as extremidades de um ímã; a extremidade de uma fonte de alimentação ou de um aparelho elétrico onde a corrente entra ou sai do aparelho}
  \definition{v.}{chegar ao fim de; levar a extremos | Literário: fazer o máximo possível}
\end{EntryWithPhonetic}

\begin{EntryWithPhonetic}{极度}{ji2du4}{7,9}{⽊,⼴}[HSK 7-9]
  \definition{adv.}{extremamente; profundamente}
  \definition{s.}{extremo; excedente; o máximo; pólo}
\end{EntryWithPhonetic}

\begin{EntryWithPhonetic}{极端}{ji2duan1}{7,14}{⽊,⽴}[HSK 6]
  \definition{adj.}{extremo; absoluto; sem quaisquer restrições}
  \definition{adv.}{excessivamente; extremamente; alto grau de expressão}
  \definition{s.}{extremo; extremidade; o auge do desenvolvimento}
\end{EntryWithPhonetic}

\begin{EntryWithPhonetic}{……极了}{ji2le5}{7,2}{⽊,⼅}[HSK 3]
  \definition{expr.}{extremamente; alto grau de expressão}
\end{EntryWithPhonetic}

\begin{EntryWithPhonetic}{极力}{ji2li4}{7,2}{⽊,⼒}[HSK 7-9]
  \definition{v.}{fazer o máximo; não poupar esforços; tentar todos os meios possíveis}
\end{EntryWithPhonetic}

\begin{EntryWithPhonetic}{极其}{ji2qi2}{7,8}{⽊,⼋}[HSK 4]
  \definition{adv.}{mais; extremamente; excessivamente}
\end{EntryWithPhonetic}

\begin{EntryWithPhonetic}{极少数}{ji2 shao3shu4}{7,4,13}{⽊,⼩,⽁}[HSK 7-9]
  \definition{num.}{pequena minoria; apenas alguns; um punhado; extremamente poucos}
\end{EntryWithPhonetic}

\begin{EntryWithPhonetic}{极为}{ji2wei2}{7,4}{⽊,⼂}[HSK 7-9]
  \definition{adv.}{extremamente; excessivamente}
\end{EntryWithPhonetic}

\begin{EntryWithPhonetic}{极限}{ji2xian4}{7,8}{⽊,⾩}[HSK 7-9]
  \definition[个]{s.}{limite; extremidade; limite máximo}
\end{EntryWithPhonetic}

%%%%%%%%%% 急 %%%%%%%%%%
\subsection*{急}\addcontentsline{loh}{figure}{急 \dpy{ji2}}

\begin{EntryWithPhonetic}{急}{ji2}{9}{⼼}[HSK 2]
  \definition{adj.}{impaciente; ansioso | irritado; aborrecido; incomodado | rápido e intenso; veloz | urgente; premente}
  \definition{s.}{urgência; emergência; assunto urgente e grave}
  \definition{v.}{preocupar; deixar ansioso | estar ansioso para ajudar; tratar os problemas dos outros como se fossem urgentes e ajudar a resolvê-los imediatamente}
  \antonymref{缓}{huan3}
\end{EntryWithPhonetic}

\begin{EntryWithPhonetic}{急救}{ji2jiu4}{9,11}{⼼,⽁}[HSK 6]
  \definition{s.}{primeiros socorros; tratamento médico de emergência (para pessoas gravemente doentes ou gravemente feridas)}
  \definition{v.}{prestar primeiros socorros; dar tratamento de emergência}
\end{EntryWithPhonetic}

\begin{EntryWithPhonetic}{急剧}{ji2ju4}{9,10}{⼼,⼑}[HSK 7-9]
  \definition{adj.}{rápido; agudo; repentino}
  \definition{adv.}{Formal: rapidamente (geralmente mudando em uma direção ruim ou potencialmente levando a resultados ruins)}
\end{EntryWithPhonetic}

\begin{EntryWithPhonetic}{急忙}{ji2mang2}{9,6}{⼼,⼼}[HSK 4]
  \definition{adv.}{apressadamente; com pressa}
\end{EntryWithPhonetic}

\begin{EntryWithPhonetic}{急迫}{ji2po4}{9,8}{⼼,⾡}[HSK 7-9]
  \definition{adj.}{urgente; premente; imperativo}
\end{EntryWithPhonetic}

\begin{EntryWithPhonetic}{急性}{ji2xing4}{9,8}{⼼,⼼}[HSK 7-9]
  \definition{adj.}{aguda}
  \definition{s.}{pessoa impetuosa; cabeça quente}
  \seealsoref{急性儿}{ji2xing4r5}
  \antonymref{慢性}{man4xing4}
\end{EntryWithPhonetic}

\begin{EntryWithPhonetic}{急性儿}{ji2xing4r5}{9,8,2}{⼼,⼼,⼉}
  \definition{adj.}{impetuoso; temperamental; de temperamento explosivo; de disposição impaciente}
\end{EntryWithPhonetic}

\begin{EntryWithPhonetic}{急需}{ji2xu1}{9,14}{⼼,⾬}[HSK 7-9]
  \definition{v.}{estar em extrema necessidade de}
\end{EntryWithPhonetic}

\begin{EntryWithPhonetic}{急于}{ji2yu2}{9,3}{⼼,⼆}[HSK 7-9]
  \definition{v.}{estar (ser) ansioso; estar (ser) impaciente; estar ansioso para}
\end{EntryWithPhonetic}

\begin{EntryWithPhonetic}{急诊}{ji2zhen3}{9,7}{⼼,⾔}[HSK 7-9]
  \definition{s.}{pronto-socorro; emergência; tratamento de emergência; uma clínica ambulatorial especial em um hospital para pessoas com doenças agudas}
\end{EntryWithPhonetic}

\begin{EntryWithPhonetic}{急转弯}{ji2zhuan3wan1}{9,8,9}{⼼,⾞,⼸}[HSK 7-9]
  \definition{s.}{curva fechada; curva acetuada; cotovelo}
  \definition{v.}{Coloquial: (uma atitude, política, etc.) fazer uma mudança repentina | fazer uma curva repentina | fazer uma mudança radical}
  \seealsoref{急转弯儿}{ji2zhuan3wan1r5}
\end{EntryWithPhonetic}

\begin{EntryWithPhonetic}{急转弯儿}{ji2zhuan3wan1r5}{9,8,9,2}{⼼,⾞,⼸,⼉}
  \definition{s.}{curva fechada}
\end{EntryWithPhonetic}

%%%%%%%%%% 疾 %%%%%%%%%%
\subsection*{疾}\addcontentsline{loh}{figure}{疾 \dpy{ji2}}

\begin{EntryWithPhonetic}{疾}{ji2}{10}{⽧}
  \definition*{s.}{Sobrenome: Ji}
  \definition{s.}{doença; enfermidade; moléstia; padecimento | sofrimento; dor; dificuldade; mazela}
\end{EntryWithPhonetic}

\begin{EntryWithPhonetic}{疾病}{ji2bing4}{10,10}{⽧,⽧}[HSK 6]
  \definition[种]{s.}{doença; enfermidade; termo geral para doença}
\end{EntryWithPhonetic}

%%%%%%%%%% 脊 %%%%%%%%%%
\subsection*{脊}\addcontentsline{loh}{figure}{脊 \dpy{ji2}}

\begin{EntryWithPhonetic}{脊}{ji2}{10}{⾁}
  \definition{s.}{coluna vertebral (de humanos e animais) | espinha; costas; cumeeira; a parte de um objeto em forma de espinha}
  \seeref{ji3}
\end{EntryWithPhonetic}

%%%%%%%%%% 棘 %%%%%%%%%%
\subsection*{棘}\addcontentsline{loh}{figure}{棘 \dpy{ji2}}

\begin{EntryWithPhonetic}{棘}{ji2}{12}{⽊}
  \definition*{s.}{Sobrenome: Ji}
  \definition{s.}{árvore de jujuba | arbustos espinhosos; silvas | espinho}
\end{EntryWithPhonetic}

\begin{EntryWithPhonetic}{棘手}{ji2shou3}{12,4}{⽊,⼿}[HSK 7-9]
  \definition{adj.}{complicado; difícil; espinhoso; difícil de manusear}
\end{EntryWithPhonetic}

%%%%%%%%%% 集 %%%%%%%%%%
\subsection*{集}\addcontentsline{loh}{figure}{集 \dpy{ji2}}

\begin{EntryWithPhonetic}{集}{ji2}{12}{⾫}[HSK 6]
  \definition*{s.}{Sobrenome: Ji}
  \definition{clas.}{parte; volume}
  \definition[个,本]{s.}{mercado; feira rural | coleção; conjunto; antologia | Matemática: conjunto}
  \definition{v.}{reunir; coletar; montar}
\end{EntryWithPhonetic}

\begin{EntryWithPhonetic}{集合}{ji2he2}{12,6}{⾫,⼝}[HSK 4]
  \definition{s.}{conjunto; montagem; coleção; agregação}
  \definition{v.}{reunir-se; juntar-se | reunir, juntar, convocar}
\end{EntryWithPhonetic}

\begin{EntryWithPhonetic}{集会}{ji2hui4}{12,6}{⾫,⼈}[HSK 7-9]
  \definition[个,次]{s.}{assembleia; reunião}
  \definition{v.}{reunir; reunir-se}
\end{EntryWithPhonetic}

\begin{EntryWithPhonetic}{集结}{ji2jie2}{12,9}{⾫,⽷}[HSK 7-9]
  \definition{v.}{(especialmente tropas) reunir; concentrar; estabelecer; fortalecer}
\end{EntryWithPhonetic}

\begin{EntryWithPhonetic}{集体}{ji2ti3}{12,7}{⾫,⼈}[HSK 3]
  \definition{s.}{coletivo; comunidade; grupo; equipe; organizações ou grupos em que muitas pessoas trabalham, estudam e vivem juntas}
\end{EntryWithPhonetic}

\begin{EntryWithPhonetic}{集团}{ji2tuan2}{12,6}{⾫,⼞}[HSK 5]
  \definition[个,家,些]{s.}{anel; bloco; grupo; panelinha; círculo; grupo organizado para agir em conjunto com um determinado objetivo | grupo; entidade econômica com uma direção de negócios especializada, liderada por uma grande empresa com forte poder econômico e alta visibilidade, e formada pela combinação ou fusão de empresas relacionadas}
\end{EntryWithPhonetic}

\begin{EntryWithPhonetic}{集邮}{ji2/you2}{12,7}{⾫,⾢}[HSK 7-9]
  \definition[本]{s.}{filatelia; coleção de selos}
  \definition{v.+compl.}{colecionar selos}
\end{EntryWithPhonetic}

\begin{EntryWithPhonetic}{集中}{ji2zhong1}{12,4}{⾫,⼁}[HSK 3]
  \definition{adj.}{centralizado; concentrado}
  \definition{v.}{concentrar; centralizar; focar; acumular; reunir | reunir pessoas, coisas, forças, etc. dispersas; resumir opiniões, experiências, etc.}
  \antonymref{分散}{fen1san4}
\end{EntryWithPhonetic}

\begin{EntryWithPhonetic}{集装箱}{ji2zhuang1xiang1}{12,12,15}{⾫,⾐,⾋}[HSK 7-9]
  \definition{s.}{\emph{container}}
\end{EntryWithPhonetic}

\begin{EntryWithPhonetic}{集资}{ji2zi1}{12,10}{⾫,⾙}[HSK 7-9]
  \definition{v.}{angariar fundos; recolher dinheiro; reunir recursos | arrecadar (reunir) dinheiro; concentrar fundos; retirar dinheiro (de muitas fontes); arrecadar fundos; solicitar fundos}
\end{EntryWithPhonetic}

%%%%%%%%%% 嫉 %%%%%%%%%%
\subsection*{嫉}\addcontentsline{loh}{figure}{嫉 \dpy{ji2}}

\begin{EntryWithPhonetic}{嫉}{ji2}{13}{⼥}
  \definition{v.}{invejar | odiar | ter ciúmes; ter inveja}
\end{EntryWithPhonetic}

\begin{EntryWithPhonetic}{嫉妒}{ji2du4}{13,7}{⼥,⼥}[HSK 7-9]
  \definition{v.}{invejar; ter ciúmes de}
\end{EntryWithPhonetic}

%%%%%%%%%% 几 %%%%%%%%%%
\subsection*{几}\addcontentsline{loh}{figure}{几 \dpy{ji3}}

\begin{EntryWithPhonetic}{几}{ji3}{2}{⼏}[HSK 1][Kangxi 16]
  \definition{adv.}{quanto?, usado para perguntar sobre quantidade e tempo}
  \definition{num.}{alguns; vários; poucos; indica um número indeterminado maior que um e menor que dez}
  \seeref{ji1}
\end{EntryWithPhonetic}

\begin{EntryWithPhonetic}{几何}{ji3he2}{2,7}{⼏,⼈}
  \definition{s.}{geometria}
\end{EntryWithPhonetic}

%%%%%%%%%% 纪 %%%%%%%%%%
\subsection*{纪}\addcontentsline{loh}{figure}{纪 \dpy{ji3}}

\begin{EntryWithPhonetic}{纪}{ji3}{6}{⽷}
  \definition*{s.}{Sobrenome: Ji}
  \definition{s.}{disciplina | um período de doze anos (na China antiga); um período de anos | (geologia) subdivisão de uma era geológica; período}
  \definition{v.}{colocar por escrito; registrar; mesmo significado de 记, usado principalmente em 记录, 纪年, 纪元, 纪传, etc. | classificar (fios de seda)}
  \seeref{ji4}
  \seealsoref{记}{ji4}
  \seealsoref{纪传}{ji4 zhuan4}
  \seealsoref{记录}{ji4lu4}
  \seealsoref{纪年}{ji4nian2}
  \seealsoref{纪元}{ji4yuan2}
\end{EntryWithPhonetic}

%%%%%%%%%% 挤 %%%%%%%%%%
\subsection*{挤}\addcontentsline{loh}{figure}{挤 \dpy{ji3}}

\begin{EntryWithPhonetic}{挤}{ji3}{9}{⼿}[HSK 5]
  \definition{adj.}{lotado; congestionado; descreve um grande número de pessoas ou coisas e muito pouco espaço}
  \definition{v.}{empacotar; amontoar; aglomerar | sacudir; empurrar contra; empurrar alguém ou algo para longe com seu corpo com toda a força que puder| pressionar; apertar; expulsar por pressão}
\end{EntryWithPhonetic}

\begin{EntryWithPhonetic}{挤压}{ji3ya1}{9,6}{⼿,⼚}[HSK 7-9]
  \definition{v.}{pressionar; espremer; esmagar; aperta | Metalurgia: extrudar}
\end{EntryWithPhonetic}

%%%%%%%%%% 给 %%%%%%%%%%
\subsection*{给}\addcontentsline{loh}{figure}{给 \dpy{ji3}}

\begin{EntryWithPhonetic}{给}{ji3}{9}{⽷}
  \definition{adj.}{abundante; próspero; bem provido para}
  \definition{v.}{fornecer; prover}
  \seeref{gei3}
\end{EntryWithPhonetic}

\begin{EntryWithPhonetic}{给予}{ji3yu3}{9,4}{⽷,⼅}[HSK 6]
  \definition{v.}{dar; conceder; dar em troca}
\end{EntryWithPhonetic}

%%%%%%%%%% 脊 %%%%%%%%%%
\subsection*{脊}\addcontentsline{loh}{figure}{脊 \dpy{ji3}}

\begin{EntryWithPhonetic}{脊}{ji3}{10}{⾁}
  \definition{s.}{espinha dorsal; coluna vertebral | crista; cumeeira; espinhaço | vértebra}
  \seeref{ji2}
\end{EntryWithPhonetic}

\begin{EntryWithPhonetic}{脊梁}{ji3liang2}{10,11}{⾁,⽊}[HSK 7-9]
  \definition{s.}{espinha dorsal | coluna vertebral}
\end{EntryWithPhonetic}

%%%%%%%%%% 计 %%%%%%%%%%
\subsection*{计}\addcontentsline{loh}{figure}{计 \dpy{ji4}}

\begin{EntryWithPhonetic}{计}{ji4}{4}{⾔}[HSK 7-9]
  \definition*{s.}{Sobrenome: Ji}
  \definition{s.}{medidor; aferidor; indicador; um instrumento para medir ou calcular graus, tempo, etc. | ideia; ardil; estratagema; plano}
  \definition{v.}{contar; calcular; numerar | planejar; traçar; imaginar}
\end{EntryWithPhonetic}

\begin{EntryWithPhonetic}{计策}{ji4ce4}{4,12}{⾔,⽵}[HSK 7-9]
  \definition{s.}{estratagema; plano; manobra}
\end{EntryWithPhonetic}

\begin{EntryWithPhonetic}{计划}{ji4hua4}{4,6}{⾔,⼑}[HSK 2]
  \definition[个,项]{s.}{plano; projeto; programa; trabalho, ações, conteúdo e etapas previamente definidos}
  \definition{v.}{planejar; traçar um plano}
\end{EntryWithPhonetic}

\begin{EntryWithPhonetic}{计较}{ji4jiao4}{4,10}{⾔,⾞}[HSK 7-9]
  \definition{v.}{pechinchar; discutir sobre | argumentar; disputar | planejar; pensar sobre}
\end{EntryWithPhonetic}

\begin{EntryWithPhonetic}{计时}{ji4shi2}{4,7}{⾔,⽇}[HSK 7-9]
  \definition{v.}{contar pelo tempo; calcular com base em horas de trabalho e proficiência técnica}
\end{EntryWithPhonetic}

\begin{EntryWithPhonetic}{计算}{ji4suan4}{4,14}{⾔,⽵}[HSK 3]
  \definition{v.}{contar; calcular; computar; enumerar; encontrar a variável desconhecida | planejar; considerar | conspirar secretamente contra os outros; planejar secretamente prejudicar os outros}
\end{EntryWithPhonetic}

\begin{EntryWithPhonetic}{计算机}{ji4suan4ji1}{4,14,6}{⾔,⽵,⽊}[HSK 2]
  \definition[部,台]{s.}{computador; calculadora; máquinas capazes de realizar cálculos matemáticos são feitas com dispositivos mecânicos, como calculadoras manuais, outras são feitas com componentes eletrônicos, como computadores eletrônicos}
\end{EntryWithPhonetic}

\begin{EntryWithPhonetic}{计算机程序}{ji4suan4ji1 cheng2xu4}{4,14,6,12,7}{⾔,⽵,⽊,⽲,⼴}
  \definition{s.}{programa de computador}
\end{EntryWithPhonetic}

%%%%%%%%%% 记 %%%%%%%%%%
\subsection*{记}\addcontentsline{loh}{figure}{记 \dpy{ji4}}

\begin{EntryWithPhonetic}{记}{ji4}{5}{⾔}[HSK 1]
  \definition*{s.}{Sobrenome: Ji}
  \definition{clas.}{tapas, palmadas, bofetadas, etc.; usado para indicar o número de vezes que uma determinada ação é realizada}
  \definition{s.}{assinatura; bloco de notas; livro ou artigo que registra fatos | insígnia; indicação; \& comercial; símbolo | marca de nascença; manchas escuras presentes na pele desde o nascimento}
  \definition{v.}{lembrar; ter em mente; guardar na memória; manter a imagem na mente | escrever (anotar); registrar; inscrever}
\end{EntryWithPhonetic}

\begin{EntryWithPhonetic}{记得}{ji4 de5}{5,11}{⾔,⼻}[HSK 1]
  \definition{v.}{lembrar; recordar; lembrar-se; não esquecer | manter algo em mente; (informal) não se esquecer de fazer algo, usado para lembrar}
\end{EntryWithPhonetic}

\begin{EntryWithPhonetic}{记号}{ji4hao5}{5,5}{⾔,⼝}[HSK 7-9]
  \definition[个]{s.}{marca; sinal; marcas feitas para atrair atenção, auxiliar na identificação e na memória}
\end{EntryWithPhonetic}

\begin{EntryWithPhonetic}{记录}{ji4lu4}{5,8}{⾔,⼹}[HSK 3]
  \definition[份,名,位,个]{s.}{notas; registro | anotador; registrador; a pessoa que faz registros}
  \definition{v.}{tomar notas; registrar; escrever o que ouviu ou o que aconteceu; gravar o som ou a imagem com um gravador ou uma câmera de vídeo e transformar em algum tipo de obra}
\end{EntryWithPhonetic}

\begin{EntryWithPhonetic}{记性}{ji4xing5}{5,8}{⾔,⼼}
  \definition{s.}{memória (habilidade em reter informações)}
\end{EntryWithPhonetic}

\begin{EntryWithPhonetic}{记忆}{ji4yi4}{5,4}{⾔,⼼}[HSK 5]
  \definition[段]{s.}{memória; manter em sua mente uma imagem do passado}
  \definition{v.}{recordar; lembrar; lembrar-se ou recordar alguém ou algo do passado}
\end{EntryWithPhonetic}

\begin{EntryWithPhonetic}{记忆犹新}{ji4yi4-you2xin1}{5,4,7,13}{⾔,⼼,⽝,⽄}[HSK 7-9]
  \definition{expr.}{estar fresco na memória; lembrar vividamente; estar ainda muito vivo na memória; ainda poder lembrar vividamente\dots; permanecer fresco na memória; ainda ter lembranças frescas de\dots; ainda reter memórias de\dots; a memória ainda está fresca.; a coisa está fresca na memória}
\end{EntryWithPhonetic}

\begin{EntryWithPhonetic}{记载}{ji4zai3}{5,10}{⾔,⾞}[HSK 4]
  \definition[段,种,条]{s.}{registro; conta; artigos e materiais que registram eventos}
  \definition{v.}{registrar; colocar por escrito}
\end{EntryWithPhonetic}

\begin{EntryWithPhonetic}{记者}{ji4zhe3}{5,8}{⾔,⽼}[HSK 3]
  \definition[群,名,位]{s.}{repórter; correspondente; jornalista; profissionais dedicados a entrevistar e reportar notícias para a mídia}
\end{EntryWithPhonetic}

\begin{EntryWithPhonetic}{记住}{ji4 zhu5}{5,7}{⾔,⼈}[HSK 1]
  \definition{v.}{lembrar; aprender de cor; ter em mente; guardar na memória}
\end{EntryWithPhonetic}

%%%%%%%%%% 纪 %%%%%%%%%%
\subsection*{纪}\addcontentsline{loh}{figure}{纪 \dpy{ji4}}

\begin{EntryWithPhonetic}{纪}{ji4}{6}{⽷}
  \definition*{s.}{Sobrenome: Ji}
  \definition{s.}{disciplina | idade; época | Geologia: período | um período de doze anos (na China antiga); um período de anos | Geologia: subdivisão de uma era geológica}
  \definition{v.}{colocar por escrito; registrar | registrar, mesmo significado de 记, usado principalmente em 记录, 纪年, 纪元, 纪传, etc. | classificar (fios de seda)}
  \seeref{ji3}
  \seealsoref{记}{ji4}
  \seealsoref{纪传}{ji4 zhuan4}
  \seealsoref{记录}{ji4lu4}
  \seealsoref{纪年}{ji4nian2}
  \seealsoref{纪元}{ji4yuan2}
\end{EntryWithPhonetic}

\begin{EntryWithPhonetic}{纪录}{ji4lu4}{6,8}{⽷,⼹}[HSK 3]
  \definition[项,个]{s.}{recorde (esportes); o número mais alto ou mais baixo registrado em um determinado período de tempo}
\end{EntryWithPhonetic}

\begin{EntryWithPhonetic}{纪录片}{ji4lu4pian4}{6,8,4}{⽷,⼹,⽚}[HSK 7-9]
  \definition[个,部]{s.}{documentário; filme documentário}
\end{EntryWithPhonetic}

\begin{EntryWithPhonetic}{纪律}{ji4lv4}{6,9}{⽷,⼻}[HSK 4]
  \definition{s.}{disciplina; código de conduta que cada membro da vida coletiva deve observar}
\end{EntryWithPhonetic}

\begin{EntryWithPhonetic}{纪年}{ji4nian2}{6,6}{⽷,⼲}
  \definition{s.}{cronologia; uma maneira de numerar os anos | registro cronológico de eventos; anais; um dos gêneros de livros históricos é organizar fatos históricos em ordem cronológica}
\end{EntryWithPhonetic}

\begin{EntryWithPhonetic}{纪念}{ji4nian4}{6,8}{⽷,⼼}[HSK 3]
  \definition[个,次]{s.}{lembrança; recordação; usado para representar uma lembrança (objeto)}
  \definition{v.}{comemorar; expressar saudade por pessoas ou coisas através de objetos ou ações}
\end{EntryWithPhonetic}

\begin{EntryWithPhonetic}{纪念碑}{ji4nian4bei1}{6,8,13}{⽷,⼼,⽯}[HSK 7-9]
  \definition{s.}{monumento; memorial | cenotáfio; placa memorial}
\end{EntryWithPhonetic}

\begin{EntryWithPhonetic}{纪念馆}{ji4nian4guan3}{6,8,11}{⽷,⼼,⾷}[HSK 7-9]
  \definition{s.}{salão memorial; museu em memória de alguém | museu em memória de\dots}
\end{EntryWithPhonetic}

\begin{EntryWithPhonetic}{纪念日}{ji4nian4ri4}{6,8,4}{⽷,⼼,⽇}[HSK 7-9]
  \definition{s.}{dia de comemoração; um dia memorável}
\end{EntryWithPhonetic}

\begin{EntryWithPhonetic}{纪实}{ji4shi2}{6,8}{⽷,⼧}[HSK 7-9]
  \definition{s.}{registro de eventos reais; relatório no local; cobertura ao vivo de eventos ou incidentes}
\end{EntryWithPhonetic}

\begin{EntryWithPhonetic}{纪元}{ji4yuan2}{6,4}{⽷,⼉}
  \definition{s.}{o início de uma era (por exemplo, o reinado de um imperador) | época; era}
\end{EntryWithPhonetic}

\begin{EntryWithPhonetic}{纪传}{ji4 zhuan4}{6,6}{⽷,⼈}
  \definition{s.}{crônica; biografia}
\end{EntryWithPhonetic}

\begin{EntryWithPhonetic}{纪传体}{ji4 zhuan4 ti3}{6,6,7}{⽷,⼈,⼈}
  \definition{s.}{história apresentada em uma série de biografias | gênero histórico baseado em biografia}
\end{EntryWithPhonetic}

%%%%%%%%%% 忌 %%%%%%%%%%
\subsection*{忌}\addcontentsline{loh}{figure}{忌 \dpy{ji4}}

\begin{EntryWithPhonetic}{忌}{ji4}{7}{⼼}[HSK 7-9]
  \definition{s.}{medo; pavor; escrúpulo}
  \definition{v.}{ter ciúmes de; invejar | evitar; afastar-se de; esquivar-se de ; abster-se de | desistir; desistir}
\end{EntryWithPhonetic}

\begin{EntryWithPhonetic}{忌妒}{ji4du4}{7,7}{⼼,⼥}
  \definition{v.}{invejar; ter ciúmes de; estar infeliz ou até mesmo odiar ou ter ciúmes dos outros porque eles são melhores que eles em termos de talento, status, etc.}
\end{EntryWithPhonetic}

\begin{EntryWithPhonetic}{忌讳}{ji4hui4}{7,6}{⼼,⾔}[HSK 7-9]
  \definition[个,种]{s.}{tabu; certas palavras ou ações são tabu devido a costumes ou razões pessoais}
  \definition{v.}{evitar; ser tabu sobre; tentar evitar ou não querer que aconteça (algo que pode ter consequências negativas)}
\end{EntryWithPhonetic}

\begin{EntryWithPhonetic}{忌口}{ji4/kou3}{7,3}{⼼,⼝}[HSK 7-9]
  \definition{v.+compl.}{evitar certos alimentos (como quando se está doente); estar de dieta}
\end{EntryWithPhonetic}

%%%%%%%%%% 技 %%%%%%%%%%
\subsection*{技}\addcontentsline{loh}{figure}{技 \dpy{ji4}}

\begin{EntryWithPhonetic}{技}{ji4}{7}{⼿}
  \definition[门,项]{s.}{destreza; habilidade; estratagema | técnica; tecnologia}
\end{EntryWithPhonetic}

\begin{EntryWithPhonetic}{技俩}{ji4liang3}{7,9}{⼿,⼈}
  \definition{s.}{truque | estratagema | ardil | esquema | estratégia | tática}
\end{EntryWithPhonetic}

\begin{EntryWithPhonetic}{技能}{ji4neng2}{7,10}{⼿,⾁}[HSK 5]
  \definition[种,项]{s.}{habilidade técnica; domínio de uma habilidade ou técnica; capacidade de adquirir e aplicar conhecimento}
\end{EntryWithPhonetic}

\begin{EntryWithPhonetic}{技巧}{ji4qiao3}{7,5}{⼿,⼯}[HSK 4]
  \definition[个,些]{s.}{habilidade; técnica; habilidades engenhosas expressas em artes, artesanato, esportes, etc.}
\end{EntryWithPhonetic}

\begin{EntryWithPhonetic}{技术}{ji4shu4}{7,5}{⼿,⽊}[HSK 3]
  \definition[种,门,项]{s.}{habilidade; técnica; tecnologia; a experiência e o conhecimento acumulados pelo ser humano no processo de utilização e transformação da natureza, e refletidos no trabalho produtivo, também se referem, de maneira geral, a outras habilidades operacionais}
\end{EntryWithPhonetic}

\begin{EntryWithPhonetic}{技艺}{ji4yi4}{7,4}{⼿,⾋}[HSK 7-9]
  \definition{s.}{habilidade; arte; artes cênicas ou artesanato habilidosos}
\end{EntryWithPhonetic}

%%%%%%%%%% 系 %%%%%%%%%%
\subsection*{系}\addcontentsline{loh}{figure}{系 \dpy{ji4}}

\begin{EntryWithPhonetic}{系}{ji4}{7}{⽷}[HSK 4]
  \definition{v.}{amarrar; prender; abotoar; dar um nó}
  \seeref{xi4}
\end{EntryWithPhonetic}

%%%%%%%%%% 剂 %%%%%%%%%%
\subsection*{剂}\addcontentsline{loh}{figure}{剂 \dpy{ji4}}

\begin{EntryWithPhonetic}{剂}{ji4}{8}{⼑}[HSK 7-9]
  \definition*{s.}{Sobrenome: Ji}
  \definition{clas.}{dose}[一剂中药===uma dose de medicina chinesa]
  \definition{s.}{preparação (farmacêutica ou outra química) | pequeno pedaço de massa | agente; certas substâncias químicas}
  \definition{v.}{ajustar; regular}
\end{EntryWithPhonetic}

%%%%%%%%%% 季 %%%%%%%%%%
\subsection*{季}\addcontentsline{loh}{figure}{季 \dpy{ji4}}

\begin{EntryWithPhonetic}{季}{ji4}{8}{⼦}[HSK 4]
  \definition*{s.}{Sobrenome: Ji}
  \definition{s.}{estação; o ano é dividido em quatro estações, primavera, verão, outono e inverno, e uma estação dura três meses | temporada | o fim de uma era | o último mês de uma temporada | o quarto ou mais novo entre irmãos; último na ordem de precedência}
\end{EntryWithPhonetic}

\begin{EntryWithPhonetic}{季度}{ji4du4}{8,9}{⼦,⼴}[HSK 4]
  \definition[个]{s.}{trimestre; período de tempo trimestral}
\end{EntryWithPhonetic}

\begin{EntryWithPhonetic}{季节}{ji4jie2}{8,5}{⼦,⾋}[HSK 4]
  \definition[个]{s.}{estação (clima); um período característico do ano}
\end{EntryWithPhonetic}

%%%%%%%%%% 既 %%%%%%%%%%
\subsection*{既}\addcontentsline{loh}{figure}{既 \dpy{ji4}}

\begin{EntryWithPhonetic}{既}{ji4}{9}{⽆}[HSK 4]
  \definition*{s.}{Sobrenome: Ji}
  \definition{adv.}{já}
  \definition{conj.}{desde; como; agora que | assim como; e também; ambos\dots e\dots; usado em conjunto com advérbios como 且, 又, 也 para indicar uma combinação de ambas as situações}
  \seealsoref{且}{qie3}
  \seealsoref{也}{ye3}
  \seealsoref{又}{you4}
\end{EntryWithPhonetic}

\begin{EntryWithPhonetic}{既不……又不……}{ji4bu4 you4bu4}{9,4,2,4}{⽆,⼀,⼜,⼀}
  \definition{conj.}{nem\dots nem\dots}
\end{EntryWithPhonetic}

\begin{EntryWithPhonetic}{既然}{ji4ran2}{9,12}{⽆,⽕}[HSK 4]
  \definition{conj.}{como; desde; agora que; usado na primeira metade de uma frase, muitas vezes repetido na segunda metade pelos advérbios 就, 也, 还 para indicar que a premissa é primeiro declarada e depois inferida}
  \seealsoref{还}{hai2}
  \seealsoref{就}{jiu4}
  \seealsoref{也}{ye3}
\end{EntryWithPhonetic}

\begin{EntryWithPhonetic}{既又}{ji4you4}{9,2}{⽆,⼜}
  \definition{conj.}{desde | como | agora isso | os dois e | assim como}
\end{EntryWithPhonetic}

%%%%%%%%%% 迹 %%%%%%%%%%
\subsection*{迹}\addcontentsline{loh}{figure}{迹 \dpy{ji4}}

\begin{EntryWithPhonetic}{迹}{ji4}{9}{⾡}
  \definition[点,丝]{s.}{marca; traço; marca deixada para trás | restos; ruínas; vestígio; coisas deixadas por gerações anteriores (principalmente edifícios) | evento importante do passado; coisas feitas; feitos | aparência; ação; figura (escrita)
vestígios}
\end{EntryWithPhonetic}

\begin{EntryWithPhonetic}{迹象}{ji4xiang4}{9,11}{⾡,⾗}[HSK 7-9]
  \definition[种]{s.}{sinal; símbolo; indicação; refere"-se a vestígios e fenômenos que podem ser usados para inferir o passado ou o futuro das coisas}
\end{EntryWithPhonetic}

%%%%%%%%%% 继 %%%%%%%%%%
\subsection*{继}\addcontentsline{loh}{figure}{继 \dpy{ji4}}

\begin{EntryWithPhonetic}{继}{ji4}{10}{⽷}[HSK 7-9]
  \definition{adv.}{então; depois}
  \definition{s.}{filhos; prole}
  \definition{v.}{continuar; ter sucesso; seguir}
\end{EntryWithPhonetic}

\begin{EntryWithPhonetic}{继承}{ji4cheng2}{10,8}{⽷,⼿}[HSK 5]
  \definition{v.}{herdar (o patrimônio de uma pessoa falecida, etc.) de acordo com a lei | continuar; geralmente se refere à aceitação do estilo, da cultura, do conhecimento, etc., daqueles que nos precederam | continuar; os descendentes continuam o trabalho deixado por seus antecessores.}
\end{EntryWithPhonetic}

\begin{EntryWithPhonetic}{继而}{ji4'er2}{10,6}{⽷,⽽}[HSK 7-9]
  \definition{adv.}{então; depois; mais tarde}
\end{EntryWithPhonetic}

\begin{EntryWithPhonetic}{继父}{ji4fu4}{10,4}{⽷,⽗}[HSK 7-9]
  \definition{s.}{padrasto}
\end{EntryWithPhonetic}

\begin{EntryWithPhonetic}{继母}{ji4mu3}{10,5}{⽷,⽏}[HSK 7-9]
  \definition{s.}{madrasta}
\end{EntryWithPhonetic}

\begin{EntryWithPhonetic}{继续}{ji4xu4}{10,11}{⽷,⽷}[HSK 3]
  \definition{s.}{continuação}
  \definition{v.}{continuar; prosseguir | prosseguir; continuar; seguir em frente (com); (atividades, eventos, etc.) continuar após uma pausa ou um determinado período de tempo}
\end{EntryWithPhonetic}

%%%%%%%%%% 寂 %%%%%%%%%%
\subsection*{寂}\addcontentsline{loh}{figure}{寂 \dpy{ji4}}

\begin{EntryWithPhonetic}{寂}{ji4}{11}{⼧}
  \definition{adj.}{quieto; parado; silencioso | solitário}
\end{EntryWithPhonetic}

\begin{EntryWithPhonetic}{寂静}{ji4jing4}{11,14}{⼧,⾭}[HSK 7-9]
  \definition{adj.}{quieto; parado; silencioso; sem som; muito silencioso}
\end{EntryWithPhonetic}

\begin{EntryWithPhonetic}{寂寥}{ji4liao2}{11,14}{⼧,⼧}
  \definition{s.}{solidão | vasto e vazio | quieto e desolado (literário)}
\end{EntryWithPhonetic}

\begin{EntryWithPhonetic}{寂寞}{ji4mo4}{11,13}{⼧,⼧}[HSK 7-9]
  \definition{adj.}{solitário; só; isolado; deserto | quieto; parado; silencioso}
\end{EntryWithPhonetic}

%%%%%%%%%% 寄 %%%%%%%%%%
\subsection*{寄}\addcontentsline{loh}{figure}{寄 \dpy{ji4}}

\begin{EntryWithPhonetic}{寄}{ji4}{11}{⼧}[HSK 4]
  \definition*{s.}{Sobrenome: Ji}
  \definition{adj.}{adotado; fomentado; promovido}
  \definition{v.}{enviar; postar; remeter | confiar; depositar; colocar | depender de; apegar-se a}
\end{EntryWithPhonetic}

\begin{EntryWithPhonetic}{寄存}{ji4cun2}{11,6}{⼧,⼦}
  \definition{v.}{depositar | deixar algo com alguém | armazenar}
\end{EntryWithPhonetic}

\begin{EntryWithPhonetic}{寄递}{ji4di4}{11,10}{⼧,⾡}
  \definition{s.}{entrega de correspondência}
\end{EntryWithPhonetic}

\begin{EntryWithPhonetic}{寄放}{ji4fang4}{11,8}{⼧,⽅}
  \definition{v.}{deixar algo com alguém}
\end{EntryWithPhonetic}

\begin{EntryWithPhonetic}{寄居}{ji4ju1}{11,8}{⼧,⼫}
  \definition{s.}{morar longe de casa}
\end{EntryWithPhonetic}

\begin{EntryWithPhonetic}{寄卖}{ji4mai4}{11,8}{⼧,⼗}
  \definition{v.}{consignar para venda}
\end{EntryWithPhonetic}

\begin{EntryWithPhonetic}{寄生}{ji4sheng1}{11,5}{⼧,⽣}
  \definition{s.}{parasita | parasitismo}
  \definition{v.}{viver tirando vantagem dos outros | viver dentro ou sobre outro organismo como um parasita}
\end{EntryWithPhonetic}

\begin{EntryWithPhonetic}{寄生生活}{ji4sheng1sheng1huo2}{11,5,5,9}{⼧,⽣,⽣,⽔}
  \definition{s.}{parasitismo | vida parasitária}
\end{EntryWithPhonetic}

\begin{EntryWithPhonetic}{寄售}{ji4shou4}{11,11}{⼧,⼝}
  \definition{v.}{venda em consignação}
\end{EntryWithPhonetic}

\begin{EntryWithPhonetic}{寄送}{ji4song4}{11,9}{⼧,⾡}
  \definition{v.}{enviar | transmitir}
\end{EntryWithPhonetic}

\begin{EntryWithPhonetic}{寄宿}{ji4su4}{11,11}{⼧,⼧}
  \definition{s.}{embarque}
  \definition{v.}{embarcar}
\end{EntryWithPhonetic}

\begin{EntryWithPhonetic}{寄托}{ji4tuo1}{11,6}{⼧,⼿}[HSK 7-9]
  \definition{v.}{deixar com alguém; confiar aos cuidados de alguém; confiar | repousar; colocar (esperança, etc.) em; encontrar sustento em; depositar ideais, esperanças, sentimentos, etc. em (alguém ou alguma coisa)}
\end{EntryWithPhonetic}

\begin{EntryWithPhonetic}{寄望}{ji4wang4}{11,11}{⼧,⽉}
  \definition{v.}{depositar esperanças em}
\end{EntryWithPhonetic}

\begin{EntryWithPhonetic}{寄养}{ji4yang3}{11,9}{⼧,⼋}
  \definition{v.}{embarcar | promover | colocar sob os cuidados de alguém (uma criança, animal de estimação, etc.)}
\end{EntryWithPhonetic}

\begin{EntryWithPhonetic}{寄予}{ji4yu3}{11,4}{⼧,⼅}
  \definition{v.}{expressar | colocar (esperança, importância, etc.) em | mostrar}
\end{EntryWithPhonetic}

%%%%%%%%%% 旣 %%%%%%%%%%
\subsection*{旣}\addcontentsline{loh}{figure}{旣 \dpy{ji4}}

\begin{EntryWithPhonetic}{旣}{ji4}{11}{⽆}
  \variantof{既}
\end{EntryWithPhonetic}

%%%%%%%%%% 祭 %%%%%%%%%%
\subsection*{祭}\addcontentsline{loh}{figure}{祭 \dpy{ji4}}

\begin{EntryWithPhonetic}{祭}{ji4}{11}{⽰}[HSK 7-9]
  \definition{v.}{oferecer um sacrifício a | realizar uma cerimônia memorial para | empunhar; usar (arma mágica)}
  \seeref{zhai4}
\end{EntryWithPhonetic}

\begin{EntryWithPhonetic}{祭奠}{ji4dian4}{11,12}{⽰,⼤}[HSK 7-9]
  \definition{v.}{realizar uma cerimônia memorial para; lembrar e mostrar respeito por (os mortos) | realizar ou comparecer a um serviço memorial |  oferecer sacrifícios (aos ancestrais)}
\end{EntryWithPhonetic}

\begin{EntryWithPhonetic}{祭祀}{ji4si4}{11,7}{⽰,⽰}[HSK 7-9]
  \definition{v.}{oferecer sacrifícios aos deuses ou ancestrais; o antigo costume é preparar oferendas aos deuses, Budas ou ancestrais para mostrar respeito e buscar bênçãos; acreditar em (religião)}
\end{EntryWithPhonetic}

%%%%%%%%%% 加 %%%%%%%%%%
\subsection*{加}\addcontentsline{loh}{figure}{加 \dpy{jia1}}

\begin{EntryWithPhonetic}{加}{jia1}{5}{⼒}[HSK 2]
  \definition*{s.}{Canadá, abreviação de 加拿大 | Sobrenome: Jia}
  \definition{v.}{adicionar; somar | aumentar; incrementar; aumentar a quantidade ou o grau em relação ao original | inserir; adicionar; anexar; adicionar o que não existe; colocar no lugar | acrescentar; indica a realização de uma determinada ação | colocar uma coisa em cima da outra | impor ou aplicar algo a outra pessoa; atribuir um determinado comportamento a outra pessoa}
  \seealsoref{加拿大}{jia1na2da4}
\end{EntryWithPhonetic}

\begin{EntryWithPhonetic}{加班}{jia1/ban1}{5,10}{⼒,⽟}[HSK 4]
  \definition{v.+compl.}{fazer horas extras; trabalhar horas extras; aumentar o horário de trabalho ou os turnos além do limite de tempo prescrito}
\end{EntryWithPhonetic}

\begin{EntryWithPhonetic}{加工}{jia1/gong1}{5,3}{⼒,⼯}[HSK 3]
  \definition{s.}{processo | trabalho (de uma máquina)}
  \definition{v.+compl.}{processar; realizar diversos trabalhos em matérias"-primas e produtos semiacabados (como alterar dimensões, formas, propriedades, aumentar a precisão, pureza, etc.) para que atendam aos requisitos especificados | melhorar; polir; refere"-se a todos os tipos de trabalho que tornam o produto final mais perfeito e refinado}
\end{EntryWithPhonetic}

\begin{EntryWithPhonetic}{加紧}{jia1jin3}{5,10}{⼒,⽷}[HSK 7-9]
  \definition{v.}{intensificar; acelerar; aumentar a velocidade ou intensidade}
\end{EntryWithPhonetic}

\begin{EntryWithPhonetic}{加剧}{jia1ju4}{5,10}{⼒,⼑}[HSK 7-9]
  \definition{v.}{agravar; intensificar; exacerbar; tornar-se mais sério do que antes}
\end{EntryWithPhonetic}

\begin{EntryWithPhonetic}{加快}{jia1kuai4}{5,7}{⼒,⼼}[HSK 3]
  \definition{v.}{acelerar; aumentar a velocidade; agilizar}
\end{EntryWithPhonetic}

\begin{EntryWithPhonetic}{加盟}{jia1meng2}{5,13}{⼒,⽫}[HSK 6]
  \definition{v.}{aliar-se a; filiar-se a um sindicato; juntar-se a um grupo ou organização}
\end{EntryWithPhonetic}

\begin{EntryWithPhonetic}{加拿大}{jia1na2da4}{5,10,3}{⼒,⼿,⼤}
  \definition{s.}{Canadá}
\end{EntryWithPhonetic}

\begin{EntryWithPhonetic}{加拿大人}{jia1na2da4ren2}{5,10,3,2}{⼒,⼿,⼤,⼈}
  \definition{s.}{canadense | pessoa ou povo do Canadá}
\end{EntryWithPhonetic}

\begin{EntryWithPhonetic}{加强}{jia1qiang2}{5,12}{⼒,⼸}[HSK 3]
  \definition{v.}{fortalecer; engrandecer; reforçar; tornar mais forte ou mais eficaz}
\end{EntryWithPhonetic}

\begin{EntryWithPhonetic}{加热}{jia1 re4}{5,10}{⼒,⽕}[HSK 5]
  \definition{v.}{aquecer; esquentar; aumentar a temperatura de um objeto}
\end{EntryWithPhonetic}

\begin{EntryWithPhonetic}{加入}{jia1ru4}{5,2}{⼒,⼊}[HSK 4]
  \definition{v.}{juntar-se; unir-se; aderir a; tornar-se um membro de uma organização, grupo | adicionar; colocar em}
\end{EntryWithPhonetic}

\begin{EntryWithPhonetic}{加上}{jia1shang5}{5,3}{⼒,⼀}[HSK 5]
  \definition{conj.}{além disso; em adição}
  \definition{v.}{adicionar; acrescentar; dar; aumentar}
\end{EntryWithPhonetic}

\begin{EntryWithPhonetic}{加深}{jia1shen1}{5,11}{⼒,⽔}[HSK 7-9]
  \definition{v.}{aprofundar; aumentar a profundidade; torne-se mais profundo}
\end{EntryWithPhonetic}

\begin{EntryWithPhonetic}{加速}{jia1su4}{5,10}{⼒,⾡}[HSK 5]
  \definition{v.}{acelerar; agilizar}
\end{EntryWithPhonetic}

\begin{EntryWithPhonetic}{加速度}{jia1su4du4}{5,10,9}{⼒,⾡,⼴}
  \definition{s.}{velocidade acelerada; aceleração | Física: aceleração}
\end{EntryWithPhonetic}

\begin{EntryWithPhonetic}{加以}{jia1yi3}{5,4}{⼒,⼈}[HSK 5]
  \definition{conj.}{além disso; em adição; indica outras razões ou condições}
  \definition{v.aux.}{usado na frente de palavras dissilábicas para indicar como um objeto mencionado deve ser tratado ou descartado | usado antes de um verbo polifônico ou de um substantivo formado a partir de um verbo para indicar como tratar ou lidar com o que foi mencionado anteriormente}
\end{EntryWithPhonetic}

\begin{EntryWithPhonetic}{加油}{jia1/you2}{5,8}{⼒,⽔}[HSK 2]
  \definition{v.+compl.}{abastecer com óleo; reabastecer; adicionar combustível ou óleo lubrificante | fazer um esforço extra; dar o máximo; (Vamos lá!) metáfora para se esforçar ainda mais}
\end{EntryWithPhonetic}

\begin{EntryWithPhonetic}{加油工}{jia1 you2 gong1}{5,8,3}{⼒,⽔,⼯}
  \definition{s.}{frentista}
\end{EntryWithPhonetic}

\begin{EntryWithPhonetic}{加油站}{jia1you2zhan4}{5,8,10}{⼒,⽔,⽴}[HSK 4]
  \definition[个,座,家]{s.}{posto de gasolina; posto de combustível; postos de abastecimento para venda a varejo de gasolina e óleo para carros e outros veículos motorizados}
\end{EntryWithPhonetic}

\begin{EntryWithPhonetic}{加重}{jia1zhong4}{5,9}{⼒,⾥}[HSK 7-9]
  \definition{v.}{tornar mais pesado; aumentar o peso de; aumentar a quantidade de | tornar mais sério; agravar}
\end{EntryWithPhonetic}

%%%%%%%%%% 夹 %%%%%%%%%%
\subsection*{夹}\addcontentsline{loh}{figure}{夹 \dpy{jia1}}

\begin{EntryWithPhonetic}{夹}{jia1}{6}{⼤}[HSK 5]
  \definition{s.}{clipe, grampo, pasta, etc.}
  \definition{v.}{colocar no meio; pressionar de ambos os lados; aplicar força ou ação ao mesmo objeto de ambos os lados ao mesmo tempo | misturar; mesclar; intercalar}
  \seeref{ga1}
  \seeref{jia2}
\end{EntryWithPhonetic}

\begin{EntryWithPhonetic}{夹杂}{jia1 za2}{6,6}{⼤,⽊}
  \definition{v.}{ser misturado com; estar carregado de; adicionar (algo mais)}
\end{EntryWithPhonetic}

\begin{EntryWithPhonetic}{夹肢窝}{jia1 zhi1 wo1}{6,8,12}{⼤,⾁,⽳}
  \definition{s.}{axila; sovaco; também escrito como 胳肢窝}
  \seealsoref{胳肢窝}{ga1 zhi1 wo1}
\end{EntryWithPhonetic}

\begin{EntryWithPhonetic}{夹子}{jia1 zi5}{6,3}{⼤,⼦}
  \definition[个,堆,盒]{s.}{pasta; carteira; algo para guardar dinheiro, papel, etc. | clipe; grampo; pasta; pinça; ferramentas para prender coisas}
\end{EntryWithPhonetic}

%%%%%%%%%% 佳 %%%%%%%%%%
\subsection*{佳}\addcontentsline{loh}{figure}{佳 \dpy{jia1}}

\begin{EntryWithPhonetic}{佳}{jia1}{8}{⼈}
  \definition{adj.}{bom; ótimo; bonito; excelente | o melhor}
\end{EntryWithPhonetic}

\begin{EntryWithPhonetic}{佳节}{jia1jie2}{8,5}{⼈,⾋}[HSK 7-9]
  \definition{s.}{festival; época feliz de festival}[春节是中国人最重要的佳节。===O Festival da Primavera é o festival mais importante para o povo chinês.]
\end{EntryWithPhonetic}

%%%%%%%%%% 茄 %%%%%%%%%%
\subsection*{茄}\addcontentsline{loh}{figure}{茄 \dpy{jia1}}

\begin{EntryWithPhonetic}{茄}{jia1}{8}{⾋}
  \definition{s.}{caracter fonético usado em empréstimos linguísticos para o som ``jia'', embora 夹 seja mais comum}
  \seeref{qie2}
  \seealsoref{夹}{jia1}
\end{EntryWithPhonetic}

%%%%%%%%%% 家 %%%%%%%%%%
\subsection*{家}\addcontentsline{loh}{figure}{家 \dpy{jia1}}

\begin{EntryWithPhonetic}{家}{jia1}{10}{⼧}[HSK 1,2]
  \definition*{s.}{Sobrenome: Jia}
  \definition{adj.}{domado; domesticado; criado; alimentado | interno}
  \definition{clas.}{usado para famílias ou estabelecimentos comerciais; para uso doméstico; lojas; fábricas, etc.}
  \definition{pron.}{Educado: meu (irmã, tio, etc.)}
  \definition[个]{s.}{família; domicílio; clã | lar; casa; residência da família | pessoa ou família envolvida em um determinado comércio; pessoas que trabalham em determinada profissão ou que possuem determinada identidade | especialista em um determinado campo; pessoa que possui conhecimentos especializados ou se dedica a atividades específicas | escola de pensamento; rscola acadêmica | (em cartas de baralho, mah"-jong etc.) festa; lado; refere"-se a jogar xadrez ou cartas, em que uma das partes joga contra a outra | nacionalidade; referindo"-se à etnia | membros da família; parentes; pessoas ou famílias com quem você tem algum tipo de relação | membro do mesmo clã; pessoas com o mesmo sobrenome}
  \definition{suf.}{sufixo substantivo para designar um especialista em alguma atividade, como um músico ou revolucionário, para designar uma profissão como em -eiro, -ista, por exemplo 科学家}
  \seealsoref{科学家}{ke1xue2jia1}
\end{EntryWithPhonetic}

\begin{EntryWithPhonetic}{家电}{jia1dian4}{10,5}{⼧,⽥}[HSK 6]
  \definition[件,台]{s.}{eletrodomésticos, abreviação de 家用电器}
  \seealsoref{家用电器}{jia1yong4 dian4qi4}
\end{EntryWithPhonetic}

\begin{EntryWithPhonetic}{家伙}{jia1huo5}{10,6}{⼧,⼈}[HSK 7-9]
  \definition[些,个,群,帮]{s.}{ferramenta; utensílio; arma; refere"-se a ferramentas ou armas | cara; companheiro; refere"-se a pessoas (com desprezo ou humor)  | gado; animal doméstico}
\end{EntryWithPhonetic}

\begin{EntryWithPhonetic}{家家户户}{jia1jia1hu4hu4}{10,10,4,4}{⼧,⼧,⼾,⼾}[HSK 7-9]
  \definition{expr.}{cada família; cada lar}
\end{EntryWithPhonetic}

\begin{EntryWithPhonetic}{家教}{jia1jiao4}{10,11}{⼧,⽁}[HSK 7-9]
  \definition[个,名,位]{s.}{educação; educação familiar; disciplina doméstica; a educação de moral e etiqueta que os pais dão aos seus filhos |  tutor}
\end{EntryWithPhonetic}

\begin{EntryWithPhonetic}{家境}{jia1jing4}{10,14}{⼧,⼟}[HSK 7-9]
  \definition{s.}{situação financeira familiar; circunstâncias familiares}
\end{EntryWithPhonetic}

\begin{EntryWithPhonetic}{家俱}{jia1ju4}{10,10}{⼧,⼈}
  \definition{s.}{mobília}
\end{EntryWithPhonetic}

\begin{EntryWithPhonetic}{家具}{jia1ju5}{10,8}{⼧,⼋}[HSK 3]
  \definition[件,套,些,个]{s.}{móveis; mobiliário de casa; utensílios domésticos, incluem principalmente camas, mesas, cadeiras, armários, etc.}
\end{EntryWithPhonetic}

\begin{EntryWithPhonetic}{家里}{jia1li3}{10,7}{⼧,⾥}[HSK 1]
  \definition{s.}{(em) casa; (em sua) família | esposa}
\end{EntryWithPhonetic}

\begin{EntryWithPhonetic}{家禽}{jia1qin2}{10,12}{⼧,⽱}[HSK 7-9]
  \definition{s.}{aves domésticas | ave; pássaro doméstico}
\end{EntryWithPhonetic}

\begin{EntryWithPhonetic}{家人}{jia1ren2}{10,2}{⼧,⼈}[HSK 1]
  \definition{s.}{família (de alguém); membro da família; os membros de uma família}
\end{EntryWithPhonetic}

\begin{EntryWithPhonetic}{家属}{jia1shu3}{10,12}{⼧,⼫}[HSK 3]
  \definition{s.}{membros da família; dependentes (familiares); os membros da família que não sejam o próprio chefe da família, ou seja, os membros da família que não sejam o próprio trabalhador}
\end{EntryWithPhonetic}

\begin{EntryWithPhonetic}{家庭}{jia1ting2}{10,9}{⼧,⼴}[HSK 2]
  \definition[个,户]{s.}{família}
\end{EntryWithPhonetic}

\begin{EntryWithPhonetic}{家务}{jia1wu4}{10,5}{⼧,⼒}[HSK 4]
  \definition[堆,次,件]{s.}{trabalho doméstico; tarefas domésticas}
\end{EntryWithPhonetic}

\begin{EntryWithPhonetic}{家乡}{jia1xiang1}{10,3}{⼧,⼄}[HSK 3]
  \definition[片,座]{s.}{cidade natal; o lugar onde sua família vive há gerações}
\end{EntryWithPhonetic}

\begin{EntryWithPhonetic}{家用}{jia1yong4}{10,5}{⼧,⽤}[HSK 7-9]
  \definition{adj.}{doméstico; para uso doméstico}
  \definition[本]{s.}{despesas familiares; dinheiro para manutenção da casa}
  \definition{v.}{ser usado em casa; para uso doméstico}
\end{EntryWithPhonetic}

\begin{EntryWithPhonetic}{家用电器}{jia1yong4 dian4qi4}{10,5,5,16}{⼧,⽤,⽥,⼝}
  \definition{s.}{eletrodoméstico; refere"-se a diversos aparelhos elétricos utilizados na vida doméstica e coletiva}
\end{EntryWithPhonetic}

\begin{EntryWithPhonetic}{家喻户晓}{jia1yu4-hu4xiao3}{10,12,4,10}{⼧,⼝,⼾,⽇}[HSK 7-9]
  \definition{expr.}{nome familiar | conhecido por todas as famílias; amplamente conhecido; conhecido por todos}[他是家喻户晓的演员。===Ele é um ator famoso.]
\end{EntryWithPhonetic}

\begin{EntryWithPhonetic}{家园}{jia1yuan2}{10,7}{⼧,⼞}[HSK 6]
  \definition{s.}{casa; terra natal; um jardim em casa, geralmente referindo"-se à cidade natal ou à família}
\end{EntryWithPhonetic}

\begin{EntryWithPhonetic}{家长}{jia1zhang3}{10,4}{⼧,⾧}[HSK 2]
  \definition[位,名,个]{s.}{pais; patriarca; tutor; guardião; refere"-se aos pais ou outros responsáveis legais}
\end{EntryWithPhonetic}

\begin{EntryWithPhonetic}{家政}{jia1zheng4}{10,9}{⼧,⽁}[HSK 7-9]
  \definition{s.}{tarefas domésticas; trabalho de gestão doméstica}
\end{EntryWithPhonetic}

\begin{EntryWithPhonetic}{家族}{jia1zu2}{10,11}{⼧,⽅}[HSK 7-9]
  \definition[个]{s.}{clã; família; uma organização social baseada em relações de sangue, incluindo várias gerações do mesmo sangue}
\end{EntryWithPhonetic}

%%%%%%%%%% 傢 %%%%%%%%%%
\subsection*{傢}\addcontentsline{loh}{figure}{傢 \dpy{jia1}}

\begin{EntryWithPhonetic}{傢}{jia1}{12}{⼈}
  \definition{s.}{usado em 家伙  e 家俱}
  \variantof{家}
  \seealsoref{傢伙}{jia1huo5}
  \seealsoref{家俱}{jia1ju4}
\end{EntryWithPhonetic}

\begin{EntryWithPhonetic}{傢伙}{jia1huo5}{12,6}{⼈,⼈}
  \variantof{家伙}
\end{EntryWithPhonetic}

\begin{EntryWithPhonetic}{傢俱}{jia1ju4}{12,10}{⼈,⼈}
  \variantof{家俱}
\end{EntryWithPhonetic}

%%%%%%%%%% 嘉 %%%%%%%%%%
\subsection*{嘉}\addcontentsline{loh}{figure}{嘉 \dpy{jia1}}

\begin{EntryWithPhonetic}{嘉}{jia1}{14}{⼝}
  \definition*{s.}{Sobrenome: Jia}
  \definition{adj.}{bom; ótimo | auspicioso | excelente}
  \definition{v.}{elogiar; recomendar}
  \definition{v.}{elogiar}
\end{EntryWithPhonetic}

\begin{EntryWithPhonetic}{嘉宾}{jia1bin1}{14,10}{⼝,⼧}[HSK 6]
  \definition[个,位,名,些]{s.}{convidado}
\end{EntryWithPhonetic}

\begin{EntryWithPhonetic}{嘉年华}{jia1nian2hua2}{14,6,6}{⼝,⼲,⼗}[HSK 7-9]
  \definition{s.}{Empréstimo linguístico: carnaval}
\end{EntryWithPhonetic}

%%%%%%%%%% 夹 %%%%%%%%%%
\subsection*{夹}\addcontentsline{loh}{figure}{夹 \dpy{jia2}}

\begin{EntryWithPhonetic}{夹}{jia2}{6}{⼤}
  \definition{adj.}{forrado; com camada dupla; duas camadas (roupas, colchas, etc.) | pinçado; voz deliberadamente engraçada}
  \seeref{ga1}
  \seeref{jia1}
\end{EntryWithPhonetic}

%%%%%%%%%% 甲 %%%%%%%%%%
\subsection*{甲}\addcontentsline{loh}{figure}{甲 \dpy{jia3}}

\begin{EntryWithPhonetic}{甲}{jia3}{5}{⽥}[HSK 5]
  \definition*{s.}{Sobrenome: Jia}
  \definition{s.}{alfa; primeiro lugar; o primeiro dos caules celestiais, geralmente usado para indicar o primeiro em ordem ou classificação | concha; carapaça; crustáceos | unha; crostas queratinosas nos dedos das mãos e dos pés | armadura; equipamento de proteção feito de metal | Obsoleto: unidade de administração civil composta por 10 residências | uma palavra substituta para uma pessoa ou coisa indefinida; usado como pronome}
  \definition{v.}{ocupar o primeiro lugar; ser melhor do que}
\end{EntryWithPhonetic}

\begin{EntryWithPhonetic}{甲骨文}{jia3gu3wen2}{5,9,4}{⽥,⾻,⽂}
  \definition{s.}{escrituras de oráculos | inscrições em ossos de oráculos (forma original de escritura chinesa)}
\end{EntryWithPhonetic}

%%%%%%%%%% 假 %%%%%%%%%%
\subsection*{假}\addcontentsline{loh}{figure}{假 \dpy{jia3}}

\begin{EntryWithPhonetic}{假}{jia3}{11}{⼈}[HSK 2]
  \definition{adj.}{falso; artificial}
  \definition{conj.}{se; caso; no caso de; conecta frases, expressa relação hipotética, geralmente usada com 如, 若 e 使, equivalente a 如果}
  \definition[个,天]{s.}{falsificação; coisas falsas, irreais ou forjadas}
  \definition{v.}{emprestar | valer-se de; aproveitar; utilizar | supor; presumir; pressupor}
  \seeref{jia4}
  \seealsoref{如}{ru2}
  \seealsoref{如果}{ru2guo3}
  \seealsoref{若}{ruo4}
  \seealsoref{使}{shi3}
\end{EntryWithPhonetic}

\begin{EntryWithPhonetic}{假的}{jia3de5}{11,8}{⼈,⽩}
  \definition{adj.}{falso | substituto | simulado}
\end{EntryWithPhonetic}

\begin{EntryWithPhonetic}{假定}{jia3ding4}{11,8}{⼈,⼧}[HSK 7-9]
  \definition{adj.}{suposto; assim chamado}
  \definition[个,种]{s.}{hipótese; hipótese científica | suposição; postulação; presunção}
  \definition{v.}{supor; assumir; conceder; presumir}
\end{EntryWithPhonetic}

\begin{EntryWithPhonetic}{假冒}{jia3mao4}{11,9}{⼈,⽇}[HSK 7-9]
  \definition{v.}{passar-se por; passar por falso; fingir ser genuíno}
\end{EntryWithPhonetic}

\begin{EntryWithPhonetic}{假如}{jia3ru2}{11,6}{⼈,⼥}[HSK 4]
  \definition{conj.}{se; supondo; no caso}
\end{EntryWithPhonetic}

\begin{EntryWithPhonetic}{假设}{jia3she4}{11,6}{⼈,⾔}[HSK 7-9]
  \definition[个,种,些]{s.}{hipótese; na pesquisa científica, refere"-se à explicação ou conclusão que precisa ser comprovada com base em determinados fenômenos.}
  \definition{v.}{conceder; supor; assumir; presumir | fabricar; inventar; não ser baseado em fatos reais}
\end{EntryWithPhonetic}

\begin{EntryWithPhonetic}{假声}{jia3sheng1}{11,7}{⼈,⼠}
  \definition{s.}{falsete}
  \seealsoref{真声}{zhen1sheng1}
\end{EntryWithPhonetic}

\begin{EntryWithPhonetic}{假使}{jia3shi3}{11,8}{⼈,⼈}[HSK 7-9]
  \definition{conj.}{se; no caso de; supondo que}
\end{EntryWithPhonetic}

\begin{EntryWithPhonetic}{假证件}{jia3zheng4jian4}{11,7,6}{⼈,⾔,⼈}
  \definition{s.}{documentos falsos}
\end{EntryWithPhonetic}

\begin{EntryWithPhonetic}{假装}{jia3zhuang1}{11,12}{⼈,⾐}[HSK 7-9]
  \definition{v.}{fingir; assumir; simular; vestir; tentar fazer de conta; agir deliberadamente de uma forma diferente da situação real para fazer os outros acreditarem}
\end{EntryWithPhonetic}

%%%%%%%%%% 价 %%%%%%%%%%
\subsection*{价}\addcontentsline{loh}{figure}{价 \dpy{jia4}}

\begin{EntryWithPhonetic}{价}{jia4}{6}{⼈}[HSK 5]
  \definition{s.}{preço | valor; (figurativo) valores (éticos, culturais etc.) | Química: valência}
\end{EntryWithPhonetic}

\begin{EntryWithPhonetic}{价格}{jia4ge2}{6,10}{⼈,⽊}[HSK 3]
  \definition[个,种]{s.}{preço; tarifa; o valor monetário da mercadoria}
  \synonymref{代价}{dai4jia4}
  \synonymref{价值}{jia4zhi2}
  \antonymref{价钱}{jia4 qian2}
\end{EntryWithPhonetic}

\begin{EntryWithPhonetic}{价钱}{jia4 qian2}{6,10}{⼈,⾦}[HSK 3]
  \definition[个,种,笔]{s.}{preço}
  \synonymref{代价}{dai4jia4}
  \synonymref{价格}{jia4ge2}
  \synonymref{价值}{jia4zhi2}
\end{EntryWithPhonetic}

\begin{EntryWithPhonetic}{价位}{jia4wei4}{6,7}{⼈,⼈}[HSK 7-9]
  \definition{s.}{preço; nível de preço}
\end{EntryWithPhonetic}

\begin{EntryWithPhonetic}{价值}{jia4zhi2}{6,10}{⼈,⼈}[HSK 3]
  \definition{s.}{valor; o trabalho social necessário condensado nos produtos | valor; importância; efeitos positivos}
  \synonymref{代价}{dai4jia4}
  \synonymref{价钱}{jia4 qian2}
  \synonymref{价格}{jia4ge2}
\end{EntryWithPhonetic}

\begin{EntryWithPhonetic}{价值观}{jia4zhi2guan1}{6,10,6}{⼈,⼈,⾒}[HSK 7-9]
  \definition{s.}{valores; a visão geral sobre economia, política, moralidade, dinheiro, etc; as pessoas têm valores diferentes devido aos seus diferentes níveis sociais}
\end{EntryWithPhonetic}

%%%%%%%%%% 驾 %%%%%%%%%%
\subsection*{驾}\addcontentsline{loh}{figure}{驾 \dpy{jia4}}

\begin{EntryWithPhonetic}{驾}{jia4}{8}{⾺}[HSK 7-9]
  \definition*{s.}{Sobrenome: Jia}
  \definition{pron.}{Cortês: você; você mesmo}
  \definition[点]{s.}{carruagem do imperador; refere"-se especificamente ao carro do imperador, referindo"-se ao imperador | referindo"-se a um veículo, usado como um termo respeitoso para uma pessoa}
  \definition{v.}{atrelar; puxar (uma carroça, etc.) | dirigir (um veículo); pilotar (um avião); velejar (um barco) | montar; cavalgar}
\end{EntryWithPhonetic}

\begin{EntryWithPhonetic}{驾车}{jia4 che1}{8,4}{⾺,⾞}[HSK 7-9]
  \definition{v.}{dirigir um veículo}
\end{EntryWithPhonetic}

\begin{EntryWithPhonetic}{驾驶}{jia4shi3}{8,8}{⾺,⾺}[HSK 5]
  \definition{v.}{dirigir; pilotar; conduzir; guiar; operar (um carro, navio, avião, trator, etc.) para fazê-lo mover}
\end{EntryWithPhonetic}

\begin{EntryWithPhonetic}{驾驭}{jia4yu4}{8,5}{⾺,⾺}[HSK 7-9]
  \definition{v.}{dirigir; conduzir animais ou veículos para a frente | controlar; fazer algo agir de acordo com a vontade de alguém}
\end{EntryWithPhonetic}

\begin{EntryWithPhonetic}{驾照}{jia4zhao4}{8,13}{⾺,⽕}[HSK 5]
  \definition[本,张]{s.}{carteira de motorista}
\end{EntryWithPhonetic}

%%%%%%%%%% 架 %%%%%%%%%%
\subsection*{架}\addcontentsline{loh}{figure}{架 \dpy{jia4}}

\begin{EntryWithPhonetic}{架}{jia4}{9}{⽊}[HSK 3]
  \definition{clas.}{usado para coisas com pilares ou componentes mecânicos | quadrado (usado para montanhas)}
  \definition{s.}{estrutura; organização do corpo humano ou das coisas | prateleira; estante; suporte; componentes que sustentam objetos ou utensílios para colocar objetos, etc.}
  \definition{v.}{colocar para cima; erigir | brigar; discutir | resistir; repelir; afastar | sequestrar; levar alguém à força}
\end{EntryWithPhonetic}

\begin{EntryWithPhonetic}{架式}{jia4shi5}{9,6}{⽊,⼷}
  \variantof{架势}
\end{EntryWithPhonetic}

\begin{EntryWithPhonetic}{架势}{jia4shi5}{9,8}{⽊,⼒}[HSK 7-9]
  \definition{s.}{postura; atitude; posição (sobre um assunto, etc.)}
\end{EntryWithPhonetic}

\begin{EntryWithPhonetic}{架子}{jia4zi5}{9,3}{⽊,⼦}[HSK 7-9]
  \definition[个,种,套]{s.}{estrutura; suporte; um objeto feito de madeira, metal ou outros materiais que pode ser usado para armazenar ou pendurar coisas | esboço; estrutura; a organização e estrutura das coisas | ares; arrogância; maneiras altivas; pensar que você é melhor que os outros e fingir ser de uma certa maneira | postura; posição; pose}
\end{EntryWithPhonetic}

%%%%%%%%%% 假 %%%%%%%%%%
\subsection*{假}\addcontentsline{loh}{figure}{假 \dpy{jia4}}

\begin{EntryWithPhonetic}{假}{jia4}{11}{⼈}
  \definition[个,天]{s.}{feriado; férias; período de suspensão temporária do trabalho ou dos estudos, legal ou aprovado | licença; afastamento temporário; período de licença temporária do trabalho ou dos estudos, após aprovação}
  \seeref{jia3}
\end{EntryWithPhonetic}

\begin{EntryWithPhonetic}{假期}{jia4qi1}{11,12}{⼈,⽉}[HSK 2]
  \definition[个,段,次,种]{s.}{férias; feriados; período de licença}
\end{EntryWithPhonetic}

\begin{EntryWithPhonetic}{假日}{jia4ri4}{11,4}{⼈,⽇}[HSK 6]
  \definition[节]{s.}{feriado; dia de folga}
\end{EntryWithPhonetic}

%%%%%%%%%% 嫁 %%%%%%%%%%
\subsection*{嫁}\addcontentsline{loh}{figure}{嫁 \dpy{jia4}}

\begin{EntryWithPhonetic}{嫁}{jia4}{13}{⼥}[HSK 7-9]
  \definition{v.}{(uma mulher) casar | casar uma filha | transferir (uma culpa, perda, fardo, etc.)}
  \antonymref{娶}{qu3}
\end{EntryWithPhonetic}

\begin{EntryWithPhonetic}{嫁妆}{jia4zhuang5}{13,6}{⼥,⼥}[HSK 7-9]
  \definition{s.}{dote; enxoval}
\end{EntryWithPhonetic}

%%%%%%%%%% 奸 %%%%%%%%%%
\subsection*{奸}\addcontentsline{loh}{figure}{奸 \dpy{jian1}}

\begin{EntryWithPhonetic}{奸}{jian1}{6}{⼥}
  \definition{adj.}{perverso; maligno; traiçoeiro; malicioso}
  \definition{s.}{traidor; espião | pessoa perversa; pessoa traiçoeira | relações sexuais ilícitas; comportamento sexual impróprio}
  \definition{v.}{ter relações sexuais ilícitas}
\end{EntryWithPhonetic}

\begin{EntryWithPhonetic}{奸夫}{jian1fu1}{6,4}{⼥,⼤}
  \definition{s.}{homem adúltero}
\end{EntryWithPhonetic}

\begin{EntryWithPhonetic}{奸诈}{jian1zha4}{6,7}{⼥,⾔}[HSK 7-9]
  \definition{adj.}{fraudulento; astuto; enganoso; traiçoeiro; hipócrita e enganador, não confiável}
\end{EntryWithPhonetic}

%%%%%%%%%% 尖 %%%%%%%%%%
\subsection*{尖}\addcontentsline{loh}{figure}{尖 \dpy{jian1}}

\begin{EntryWithPhonetic}{尖}{jian1}{6}{⼩}[HSK 6]
  \definition{adj.}{pontiagudo; afilado; agudo | agudo; estridente; penetrante | mesquinho; pão-duro | mordaz; cáustico}
  \definition{s.}{ponto; ponta; topo | o melhor do seu tipo; a melhor escolha; a nata da safra; uma pessoa ou coisa notável}
  \definition{v.}{tornar (a voz, etc.) aguda; estridente}
\end{EntryWithPhonetic}

\begin{EntryWithPhonetic}{尖端}{jian1duan1}{6,14}{⼩,⽴}[HSK 7-9]
  \definition{adj.}{mais avançado; sofisticado; o mais alto nível de desenvolvimento (ciência e tecnologia, etc.)}
  \definition{s.}{ponta pontiaguda; pico; a ponta fina e pontiaguda de algo | topo; ápice; pico; uma metáfora para o mais alto nível de ciência e tecnologia}
\end{EntryWithPhonetic}

\begin{EntryWithPhonetic}{尖锐}{jian1rui4}{6,12}{⼩,⾦}[HSK 7-9]
  \definition{adj.}{pontiagudo; a ponta do objeto é fina e pequena, podendo facilmente entrar ou passar por outros objetos | incisivo; penetrante; conhecer e observar as coisas com muita precisão e profundidade | estridente; penetrante; o som é alto e fino, fazendo com que as pessoas se sintam desconfortáveis | afiado; intenso; muito intenso (discussão, luta, etc.)}
\end{EntryWithPhonetic}

%%%%%%%%%% 坚 %%%%%%%%%%
\subsection*{坚}\addcontentsline{loh}{figure}{坚 \dpy{jian1}}

\begin{EntryWithPhonetic}{坚}{jian1}{7}{⼟}
  \definition*{s.}{Sobrenome: Jian}
  \definition{adj.}{duro; firme; sólido; forte | firme; resoluto; constante}
  \definition{s.}{fortaleza; fortificação; um ponto fortemente fortificado; coisas sólidas, principalmente referindo"-se a posições | armadura}
\end{EntryWithPhonetic}

\begin{EntryWithPhonetic}{坚持}{jian1chi2}{7,9}{⼟,⼿}[HSK 3]
  \definition{v.}{persistir em; perseverar em; defender; insistir em; manter-se fiel a; aderir a; persistir com determinação e não desistir quando se depara com dificuldades | aderir a; insistir em; não alterar (os princípios, opiniões, pontos de vista originais, etc.)}
\end{EntryWithPhonetic}

\begin{EntryWithPhonetic}{坚持不懈}{jian1chi2-bu2xie4}{7,9,4,16}{⼟,⼿,⼀,⼼}[HSK 7-9]
  \definition{expr.}{aderir a algo incessantemente; manter-se consistente e persistentemente; incessante; perseverar incessantemente}
\end{EntryWithPhonetic}

\begin{EntryWithPhonetic}{坚定}{jian1ding4}{7,8}{⼟,⼧}[HSK 5]
  \definition{adj.}{firme; inabalável; inamovível; (posição, opinião, vontade, etc.) firme e estável, inabalável}
  \definition{v.}{fortalecer}
\end{EntryWithPhonetic}

\begin{EntryWithPhonetic}{坚固}{jian1gu4}{7,8}{⼟,⼞}[HSK 4]
  \definition{adj.}{firme; sólido; robusto; forte; durável; firmemente unidos e inquebráveis}
\end{EntryWithPhonetic}

\begin{EntryWithPhonetic}{坚决}{jian1jue2}{7,6}{⼟,⼎}[HSK 3]
  \definition{adj.}{firme; resoluto; (atitude, opinião, ação, etc.) determinado e inabalável}
\end{EntryWithPhonetic}

\begin{EntryWithPhonetic}{坚强}{jian1qiang2}{7,12}{⼟,⼸}[HSK 3]
  \definition{adj.}{forte; firme; convicto; (qualidades humanas, personalidade, determinação, etc.) firme e forte, não vacila diante das dificuldades}
  \definition{v.}{fortalecer; tornar forte; é a qualidade, a determinação, etc., que não vacilam}
\end{EntryWithPhonetic}

\begin{EntryWithPhonetic}{坚韧}{jian1ren4}{7,7}{⼟,⾱}[HSK 7-9]
  \definition{adj.}{resistente e elástico; forte e resiliente | firme e tenaz; não impaciente, não entediado}
\end{EntryWithPhonetic}

\begin{EntryWithPhonetic}{坚实}{jian1shi2}{7,8}{⼟,⼧}[HSK 7-9]
  \definition{adj.}{sólido; substancial; forte e resistente}
\end{EntryWithPhonetic}

\begin{EntryWithPhonetic}{坚守}{jian1shou3}{7,6}{⼟,⼧}[HSK 7-9]
  \definition{v.}{segurar; manter-se firme; manter-se firme em; aderir firmemente}
\end{EntryWithPhonetic}

\begin{EntryWithPhonetic}{坚信}{jian1xin4}{7,9}{⼟,⼈}[HSK 7-9]
  \definition{v.}{acreditar firmemente; estar firmemente convencido; estar totalmente confiante de}
\end{EntryWithPhonetic}

\begin{EntryWithPhonetic}{坚硬}{jian1ying4}{7,12}{⼟,⽯}[HSK 7-9]
  \definition{adj.}{duro; sólido; resistente; duro e forte}
\end{EntryWithPhonetic}

%%%%%%%%%% 歼 %%%%%%%%%%
\subsection*{歼}\addcontentsline{loh}{figure}{歼 \dpy{jian1}}

\begin{EntryWithPhonetic}{歼}{jian1}{7}{⽍}
  \definition{v.}{aniquilar; eliminar; destruir}
\end{EntryWithPhonetic}

\begin{EntryWithPhonetic}{歼灭}{jian1mie4}{7,5}{⽍,⽕}[HSK 7-9]
  \definition{v.}{aniquilar; eliminar; destruir}
\end{EntryWithPhonetic}

%%%%%%%%%% 间 %%%%%%%%%%
\subsection*{间}\addcontentsline{loh}{figure}{间 \dpy{jian1}}

\begin{EntryWithPhonetic}{间}{jian1}{7}{⾨}[HSK 1]
  \definition{clas.}{a menor unidade de uma casa; a menor unidade habitacional; cômodo}
  \definition{s.}{espaço entre duas partes  | (em um) tempo ou espaço definido | sala; quarto | uma seção de uma sala ou o espaço lateral entre dois pares de pilares | com um tempo ou espaço definido}
  \seeref{jian4}
\end{EntryWithPhonetic}

%%%%%%%%%% 浅 %%%%%%%%%%
\subsection*{浅}\addcontentsline{loh}{figure}{浅 \dpy{jian1}}

\begin{EntryWithPhonetic}{浅}{jian1}{8}{⽔}
  \definition{adj.}{murmurando, fluindo suavemente, gorgolejando suavemente}
  \definition{s.}{Onomatopéia: som de água em movimento}
  \seeref{qian3}
\end{EntryWithPhonetic}

%%%%%%%%%% 肩 %%%%%%%%%%
\subsection*{肩}\addcontentsline{loh}{figure}{肩 \dpy{jian1}}

\begin{EntryWithPhonetic}{肩}{jian1}{8}{⾁}[HSK 5]
  \definition*{s.}{Sobrenome: Jian}
  \definition{s.}{ombro; torso}
  \definition{v.}{assumir; empreender; carregar; suportar; suportar um fardo}
\end{EntryWithPhonetic}

\begin{EntryWithPhonetic}{肩膀}{jian1bang3}{8,14}{⾁,⾁}[HSK 7-9]
  \definition[个,副]{s.}{ombro}
\end{EntryWithPhonetic}

\begin{EntryWithPhonetic}{肩负}{jian1fu4}{8,6}{⾁,⾙}[HSK 7-9]
  \definition{v.}{assumir; empreender; carregar; suportar; ser confiado a}
\end{EntryWithPhonetic}

%%%%%%%%%% 艰 %%%%%%%%%%
\subsection*{艰}\addcontentsline{loh}{figure}{艰 \dpy{jian1}}

\begin{EntryWithPhonetic}{艰}{jian1}{8}{⾉}
  \definition{adj.}{difícil; duro}
\end{EntryWithPhonetic}

\begin{EntryWithPhonetic}{艰巨}{jian1ju4}{8,4}{⾉,⼯}[HSK 7-9]
  \definition{adj.}{árduo; oneroso; difícil; formidável}
\end{EntryWithPhonetic}

\begin{EntryWithPhonetic}{艰苦}{jian1ku3}{8,8}{⾉,⾋}[HSK 5]
  \definition{adj.}{duro; resistente; árduo; difícil; condições de trabalho ou de vida ruins que tornam as pessoas miseráveis}
\end{EntryWithPhonetic}

\begin{EntryWithPhonetic}{艰苦奋斗}{jian1ku3-fen4dou4}{8,8,8,4}{⾉,⾋,⼤,⽃}[HSK 7-9]
  \definition{expr.}{luta árdua; trabalho duro persistente}
  \definition{v.}{trabalhar duro e perseverantemente; lutar arduamente em meio às dificuldades; trabalhar diligentemente desafiando (apesar) das dificuldades; travar uma luta árdua}
\end{EntryWithPhonetic}

\begin{EntryWithPhonetic}{艰难}{jian1nan2}{8,10}{⾉,⾫}[HSK 5]
  \definition{adj.}{duro; árduo; difícil}
\end{EntryWithPhonetic}

\begin{EntryWithPhonetic}{艰险}{jian1xian3}{8,9}{⾉,⾩}[HSK 7-9]
  \definition{adj.}{perigoso}
  \definition{s.}{dificuldades e perigos}
\end{EntryWithPhonetic}

\begin{EntryWithPhonetic}{艰辛}{jian1xin1}{8,7}{⾉,⾟}[HSK 7-9]
  \definition{adj.}{duro; árduo; difícil}
\end{EntryWithPhonetic}

%%%%%%%%%% 兼 %%%%%%%%%%
\subsection*{兼}\addcontentsline{loh}{figure}{兼 \dpy{jian1}}

\begin{EntryWithPhonetic}{兼}{jian1}{10}{⼋}[HSK 7-9]
  \definition*{s.}{Sobrenome: Jian}
  \definition{adj.}{duplo; dobrado; duplicado | simultâneo; concomitante}
  \definition{adv.}{simultaneamente; concomitivamente; envolve várias coisas ao mesmo tempo.}
  \definition{v.}{ocupar um cargo simultâneo | ter dois ou mais empregos simultaneamente; fazer várias coisas ao mesmo tempo ou possuir várias coisas | Literário: reunir; unir em um só; anexar}
\end{EntryWithPhonetic}

\begin{EntryWithPhonetic}{兼顾}{jian1gu4}{10,10}{⼋,⾴}[HSK 7-9]
  \definition{v.}{levar em consideração duas ou mais coisas; dar atenção a duas ou mais coisas}
\end{EntryWithPhonetic}

\begin{EntryWithPhonetic}{兼任}{jian1ren4}{10,6}{⼋,⼈}[HSK 7-9]
  \definition{v.}{ocupar um cargo simultâneo; ter vários empregos ao mesmo tempo | realizar algo em tempo parcial; trabalhar em tempo parcial}
\end{EntryWithPhonetic}

\begin{EntryWithPhonetic}{兼容}{jian1rong2}{10,10}{⼋,⼧}[HSK 7-9]
  \definition{v.}{abranger a todos; ser compatível; aceitar e acomodar simultaneamente coisas ou aspectos diferentes.}
\end{EntryWithPhonetic}

\begin{EntryWithPhonetic}{兼职}{jian1zhi2}{10,11}{⼋,⽿}[HSK 7-9]
  \definition[份]{pron.}{vaga simultânea; emprego de meio período; cargos ocupados fora da função principal de emprego}
  \definition{v.}{ocupar dois ou mais cargos simultaneamente; exercer outras funções além do trabalho principal}
\end{EntryWithPhonetic}

%%%%%%%%%% 监 %%%%%%%%%%
\subsection*{监}\addcontentsline{loh}{figure}{监 \dpy{jian1}}

\begin{EntryWithPhonetic}{监}{jian1}{10}{⽫}
  \definition{s.}{prisão; cadeia}
  \definition{v.}{supervisionar; inspecionar; observar}
\end{EntryWithPhonetic}

\begin{EntryWithPhonetic}{监测}{jian1ce4}{10,9}{⽫,⽔}[HSK 6]
  \definition{v.}{monitorar; supervisionar e testar}
\end{EntryWithPhonetic}

\begin{EntryWithPhonetic}{监察}{jian1cha2}{10,14}{⽫,⼧}[HSK 7-9]
  \definition{v.}{supervisionar; controlar}
\end{EntryWithPhonetic}

\begin{EntryWithPhonetic}{监督}{jian1du1}{10,13}{⽫,⽬}[HSK 6]
  \definition[个,位,名]{s.}{monitoramento; supervisão; pessoas que supervisionam}
  \definition{v.}{controlar; supervisionar; superintender; monitorar e supervisionar de perto}
\end{EntryWithPhonetic}

\begin{EntryWithPhonetic}{监管}{jian1guan3}{10,14}{⽫,⽵}[HSK 7-9]
  \definition{v.}{monitorar; supervisionar}
\end{EntryWithPhonetic}

\begin{EntryWithPhonetic}{监护}{jian1hu4}{10,7}{⽫,⼿}[HSK 7-9]
  \definition{s.}{tutela}
  \definition{v.}{Lei: desempenhar as funções de tutor; agir como guardião; tutelar | Medicina: cuidar e zelar; vigiar}
\end{EntryWithPhonetic}

\begin{EntryWithPhonetic}{监控}{jian1kong4}{10,11}{⽫,⼿}[HSK 7-9]
  \definition{s.}{monitor}
  \definition{v.}{monitorar}
\end{EntryWithPhonetic}

\begin{EntryWithPhonetic}{监视}{jian1shi4}{10,8}{⽫,⾒}[HSK 7-9]
  \definition{v.}{manter vigilância; ficar de olho em; observar atentamente}
\end{EntryWithPhonetic}

\begin{EntryWithPhonetic}{监狱}{jian1yu4}{10,9}{⽫,⽝}[HSK 7-9]
  \definition[个,所,座]{s.}{cadeia; prisão; instituições estatais responsáveis pela aplicação de penas criminais; locais onde os presos são mantidos}
\end{EntryWithPhonetic}

%%%%%%%%%% 渐 %%%%%%%%%%
\subsection*{渐}\addcontentsline{loh}{figure}{渐 \dpy{jian1}}

\begin{EntryWithPhonetic}{渐}{jian1}{11}{⽔}
  \definition{v.}{encharcar; ficar saturado com | fluir para}
  \seeref{jian4}
\end{EntryWithPhonetic}

%%%%%%%%%% 煎 %%%%%%%%%%
\subsection*{煎}\addcontentsline{loh}{figure}{煎 \dpy{jian1}}

\begin{EntryWithPhonetic}{煎}{jian1}{13}{⽕}[HSK 7-9]
  \definition{s.}{Fitoterapia: decocção}
  \definition{v.}{fritar (em pouco óleo, sem mexer) | ferver (em água)}
\end{EntryWithPhonetic}

\begin{EntryWithPhonetic}{煎饼}{jian1bing3}{13,9}{⽕,⾷}
  \definition[张,块,摞]{s.}{uma panqueca fina feita de farinha de milho-miúdo}
\end{EntryWithPhonetic}

\begin{EntryWithPhonetic}{煎蛋}{jian1dan4}{13,11}{⽕,⾍}
  \definition{s.}{ovo frito}
\end{EntryWithPhonetic}

%%%%%%%%%% 拣 %%%%%%%%%%
\subsection*{拣}\addcontentsline{loh}{figure}{拣 \dpy{jian3}}

\begin{EntryWithPhonetic}{拣}{jian3}{8}{⼿}[HSK 7-9]
  \definition{v.}{escolher; selecionar | pegar; coletar; reunir | o mesmo que 捡}
  \seealsoref{捡}{jian3}
\end{EntryWithPhonetic}

%%%%%%%%%% 俭 %%%%%%%%%%
\subsection*{俭}\addcontentsline{loh}{figure}{俭 \dpy{jian3}}

\begin{EntryWithPhonetic}{俭}{jian3}{9}{⼈}
  \definition*{s.}{Sobrenome: Jian}
  \definition{adj.}{econômico; frugal | querendo; faltando; curto}
\end{EntryWithPhonetic}

\begin{EntryWithPhonetic}{俭省}{jian3sheng3}{9,9}{⼈,⽬}
  \definition{adj.}{econômico}
\end{EntryWithPhonetic}

%%%%%%%%%% 柬 %%%%%%%%%%
\subsection*{柬}\addcontentsline{loh}{figure}{柬 \dpy{jian3}}

\begin{EntryWithPhonetic}{柬}{jian3}{9}{⽊}
  \definition*{s.}{Sobrenome: Jian}
  \definition[张,封]{s.}{cartão; nota; carta; um termo geral para cartas, cartões de visita, postagens, etc.}
\end{EntryWithPhonetic}

\begin{EntryWithPhonetic}{柬埔寨}{jian3pu3zhai4}{9,10,14}{⽊,⼟,⼧}
  \definition*{s.}{Camboja}
\end{EntryWithPhonetic}

%%%%%%%%%% 捡 %%%%%%%%%%
\subsection*{捡}\addcontentsline{loh}{figure}{捡 \dpy{jian3}}

\begin{EntryWithPhonetic}{捡}{jian3}{10}{⼿}[HSK 6]
  \definition{v.}{coletar; reunir; apanhar; pegar coisas do chão}
\end{EntryWithPhonetic}

%%%%%%%%%% 减 %%%%%%%%%%
\subsection*{减}\addcontentsline{loh}{figure}{减 \dpy{jian3}}

\begin{EntryWithPhonetic}{减}{jian3}{11}{⼎}[HSK 4]
  \definition*{s.}{Sobrenome: Jian}
  \definition{v.}{subtrair; remover uma parte da quantidade original | reduzir; diminuir; cortar}
\end{EntryWithPhonetic}

\begin{EntryWithPhonetic}{减肥}{jian3/fei2}{11,8}{⼎,⾁}[HSK 4]
  \definition{v.+compl.}{perder peso; dieta, exercícios, medicamentos, massagem, cirurgia, etc., para reduzir o excesso de gordura corporal, de modo que o grau de obesidade seja reduzido}
\end{EntryWithPhonetic}

\begin{EntryWithPhonetic}{减免}{jian3mian3}{11,7}{⼎,⼉}[HSK 7-9]
  \definition{v.}{mitigar ou anular (uma punição) | reduzir ou isentar (impostos, etc.)}
\end{EntryWithPhonetic}

\begin{EntryWithPhonetic}{减轻}{jian3qing1}{11,9}{⼎,⾞}[HSK 5]
  \definition{v.}{aliviar; remeter; clarear; facilitar; mitigar}
\end{EntryWithPhonetic}

\begin{EntryWithPhonetic}{减弱}{jian3ruo4}{11,10}{⼎,⼸}[HSK 7-9]
  \definition{v.}{reduzir | enfraquecer}
\end{EntryWithPhonetic}

\begin{EntryWithPhonetic}{减少}{jian3shao3}{11,4}{⼎,⼩}[HSK 4]
  \definition{v.}{cair; reduzir; diminuir; subtrair uma parte}
\end{EntryWithPhonetic}

\begin{EntryWithPhonetic}{减速}{jian3/su4}{11,10}{⼎,⾡}[HSK 7-9]
  \definition{v.+compl.}{diminuir a velocidade; desacelerar; retardar | moderar; reduzir a velocidade; reduzir a marcha; atrasar; desacelerar; retardar}
  \antonymref{加速}{jia1su4}
\end{EntryWithPhonetic}

\begin{EntryWithPhonetic}{减压}{jian3ya1}{11,6}{⼎,⼚}[HSK 7-9]
  \definition{v.}{reduzir a pressão; descomprimir | reduzir o fardo | reduzir a pressão; despressurizar; descomprimir | relaxar}
\end{EntryWithPhonetic}

%%%%%%%%%% 剪 %%%%%%%%%%
\subsection*{剪}\addcontentsline{loh}{figure}{剪 \dpy{jian3}}

\begin{EntryWithPhonetic}{剪}{jian3}{11}{⼑}[HSK 5]
  \definition[把]{s.}{tesouras; tesouras de poda; cortadores | pinças; tenazes}
  \definition{v.}{cortar; aparar; tosquiar; cortar (com uma tesoura) | exterminar; eliminar; acabar com}
\end{EntryWithPhonetic}

\begin{EntryWithPhonetic}{剪刀}{jian3dao1}{11,2}{⼑,⼑}[HSK 5]
  \definition[把,个]{s.}{tesoura; tesoura de jardim; instrumento de ferro para cortar tecido, papel, barbante, etc., com duas lâminas interligadas que podem ser abertas e fechadas}
\end{EntryWithPhonetic}

\begin{EntryWithPhonetic}{剪子}{jian3zi5}{11,3}{⼑,⼦}[HSK 5]
  \definition[把]{s.}{tesouras; tesouras de podar; tosquiadeiras}
\end{EntryWithPhonetic}

%%%%%%%%%% 检 %%%%%%%%%%
\subsection*{检}\addcontentsline{loh}{figure}{检 \dpy{jian3}}

\begin{EntryWithPhonetic}{检}{jian3}{11}{⽊}
  \definition*{s.}{Sobrenome: Jian}
  \definition{v.}{verificar; inspecionar; examinar | conter-se; ter cuidado na conduta}
\end{EntryWithPhonetic}

\begin{EntryWithPhonetic}{检测}{jian3ce4}{11,9}{⽊,⽔}[HSK 4]
  \definition{v.}{testar; detectar; verificar}
\end{EntryWithPhonetic}

\begin{EntryWithPhonetic}{检查}{jian3cha2}{11,9}{⽊,⽊}[HSK 2]
  \definition[份,个,次]{s.}{autocrítica; reconhecer e criticar os próprios erros verbais ou escritos}
  \definition{v.}{verificar; inspecionar; examinar; verificar cuidadosamente para descobrir o problema | criticar a si mesmo; identificar seus pontos fracos e erros, e criticar seu próprio comportamento}
\end{EntryWithPhonetic}

\begin{EntryWithPhonetic}{检察}{jian3cha2}{11,14}{⽊,⼧}[HSK 7-9]
  \definition{v.}{realizar trabalho de procuradoria; refere"-se especificamente às atividades de supervisão legal realizadas pelas autoridades de supervisão legal do Estado, em conformidade com a lei}
\end{EntryWithPhonetic}

\begin{EntryWithPhonetic}{检讨}{jian3tao3}{11,5}{⽊,⾔}[HSK 7-9]
  \definition[份,个]{s.}{autocrítica}
  \definition{v.}{fazer uma autocrítica; identificar e reconhecer suas próprias falhas ou erros | examinar; inspecionar}
\end{EntryWithPhonetic}

\begin{EntryWithPhonetic}{检验}{jian3yan4}{11,10}{⽊,⾺}[HSK 5]
  \definition{v.}{testar; examinar; inspecionar}
\end{EntryWithPhonetic}

%%%%%%%%%% 简 %%%%%%%%%%
\subsection*{简}\addcontentsline{loh}{figure}{简 \dpy{jian3}}

\begin{EntryWithPhonetic}{简}{jian3}{13}{⽵}
  \definition*{s.}{Sobrenome: Jian}
  \definition{adj.}{simples; simplificado; breve | breve; em resumo; em poucas palavras}
  \definition{s.}{Arcaico: tiras de bambu (para escrever) | carta; correspondência}
  \definition{v.}{simplificar | (literário) selecionar; escolher}
  \antonymref{繁}{fan2}
\end{EntryWithPhonetic}

\begin{EntryWithPhonetic}{简称}{jian3cheng1}{13,10}{⽵,⽲}[HSK 7-9]
  \definition[个]{s.}{a forma abreviada de um nome; abreviação; forma mais curta; forma simplificada do nome}
  \definition{v.}{chamar algo abreviadamente; abreviar}[科技术语常常被简称。===Termos científicos e técnicos são frequentemente abreviados.]
\end{EntryWithPhonetic}

\begin{EntryWithPhonetic}{简单}{jian3dan1}{13,8}{⽵,⼗}[HSK 3]
  \definition{adj.}{simples; descomplicado; estrutura simples; poucas complicações; fácil de entender, usar ou lidar | comum; lugar-comum; (experiência, capacidade, etc.) comum (usado principalmente em frases negativas) | casual; simplificado; precipitado; pouco cuidadoso}
\end{EntryWithPhonetic}

\begin{EntryWithPhonetic}{简短}{jian3duan3}{13,12}{⽵,⽮}[HSK 7-9]
  \definition{adj.}{breve; curto; conciso; (palavras) não são longas}
\end{EntryWithPhonetic}

\begin{EntryWithPhonetic}{简化}{jian3hua4}{13,4}{⽵,⼔}[HSK 7-9]
  \definition{v.}{simplificar; transformar o complicado em simples}
\end{EntryWithPhonetic}

\begin{EntryWithPhonetic}{简洁}{jian3jie2}{13,9}{⽵,⽔}[HSK 7-9]
  \definition{adj.}{conciso; sucinto; breve; (oratória e escrita) conciso e direto ao ponto, sem palavras desnecessárias}
\end{EntryWithPhonetic}

\begin{EntryWithPhonetic}{简介}{jian3jie4}{13,4}{⽵,⼈}[HSK 6]
  \definition{s.}{breve introdução; sinopse; relato resumido}
  \definition{v.}{fazer um breve relato de (algo)}
\end{EntryWithPhonetic}

\begin{EntryWithPhonetic}{简历}{jian3li4}{13,4}{⽵,⼚}[HSK 4]
  \definition[个,份]{s.}{currículo; \emph{curriculum vitae} (CV); notas biográficas}
\end{EntryWithPhonetic}

\begin{EntryWithPhonetic}{简陋}{jian3lou4}{13,8}{⽵,⾩}[HSK 7-9]
  \definition{adj.}{simples e grosseiro}
\end{EntryWithPhonetic}

\begin{EntryWithPhonetic}{简体字}{jian3ti3zi4}{13,7,6}{⽵,⼈,⼦}[HSK 7-9]
  \definition{s.}{caracteres chineses simplificados (em oposição a 繁体字)}
  \seealsoref{繁体字}{fan2ti3zi4}
\end{EntryWithPhonetic}

\begin{EntryWithPhonetic}{简要}{jian3yao4}{13,9}{⽵,⾑}[HSK 7-9]
  \definition{adj.}{breve; conciso}
\end{EntryWithPhonetic}

\begin{EntryWithPhonetic}{简易}{jian3yi4}{13,8}{⽵,⽇}[HSK 7-9]
  \definition{adj.}{simples e fácil | de construção simples; com equipamentos simples; sem sofisticação}
\end{EntryWithPhonetic}

\begin{EntryWithPhonetic}{简直}{jian3zhi2}{13,8}{⽵,⽬}[HSK 3]
  \definition{adv.}{simplesmente; de forma alguma; praticamente; significa ``exatamente assim'' (tom exagerado)}
\end{EntryWithPhonetic}

%%%%%%%%%% 见 %%%%%%%%%%
\subsection*{见}\addcontentsline{loh}{figure}{见 \dpy{jian4}}

\begin{EntryWithPhonetic}{见}{jian4}{4}{⾒}[HSK 1][Kangxi 147]
  \definition*{s.}{Sobrenome: Jian}
  \definition{part.}{usado antes de um verbo para indicar voz passiva ou para expressar como isso me afeta}
  \definition{s.}{visão; ideia; opinião sobre algo; ponto de vista}
  \definition{v.}{ver; avistar | encontrar-se com; ser exposto a | parecer ser; mostrar evidência de | ver; referir-se a; indicar a fonte ou o local onde deve ser consultado | ver; encontrar; convocar}
  \seeref{xian4}
\end{EntryWithPhonetic}

\begin{EntryWithPhonetic}{见到}{jian4dao4}{4,8}{⾒,⼑}[HSK 2]
  \definition{v.}{ver | encontrar; esbarrar; deparar-se com}
\end{EntryWithPhonetic}

\begin{EntryWithPhonetic}{见过}{jian4guo4}{4,6}{⾒,⾡}[HSK 2]
  \definition{s.}{visto (ver); já viu alguém ou algo; indica um momento no passado; alguém já viu ou encontrou um determinado objeto}
\end{EntryWithPhonetic}

\begin{EntryWithPhonetic}{见解}{jian4jie3}{4,13}{⾒,⾓}[HSK 7-9]
  \definition[种,点,番]{s.}{ponto de vista; opinião; entendimento; compreensão e perspectivas sobre as coisas}
\end{EntryWithPhonetic}

\begin{EntryWithPhonetic}{见面}{jian4/mian4}{4,9}{⾒,⾯}[HSK 1]
  \definition{v.+compl.}{encontrar-se com alguém;  ver um ao outro; ver alguém face-a-face}
\end{EntryWithPhonetic}

\begin{EntryWithPhonetic}{见钱眼开}{jian4qian2-yan3kai1}{4,10,11,4}{⾒,⾦,⽬,⼶}[HSK 7-9]
  \definition{expr.}{os olhos se arregalam de alegria ao ver dinheiro; comover-se ao ver dinheiro; ser tentado pelo dinheiro; não se importar com nada além de dinheiro; arregalar os olhos ao ver dinheiro; ``Mostrar alegria ao ver dinheiro.'', descreve alguém que valoriza o dinheiro em excesso; ganancioso}
\end{EntryWithPhonetic}

\begin{EntryWithPhonetic}{见仁见智}{jian4ren2-jian4zhi4}{4,4,4,12}{⾒,⼈,⾒,⽇}[HSK 7-9]
  \definition{expr.}{opiniões divergentes; O Livro das Mutações, em seus Acréscimos, Parte I, afirma: ``Os benevolentes veem isso como benevolência, os sábios veem isso como sabedoria.'', isso significa que pessoas diferentes têm perspectivas diferentes sobre a mesma questão}
\end{EntryWithPhonetic}

\begin{EntryWithPhonetic}{见识}{jian4shi5}{4,7}{⾒,⾔}[HSK 7-9]
  \definition{s.}{experiência; conhecimento; bom senso; observações}
  \definition{v.}{ampliar o conhecimento; enriquecer a experiência; expor-se a novas experiências e ampliar seus horizontes}
\end{EntryWithPhonetic}

\begin{EntryWithPhonetic}{见外}{jian4wai4}{4,5}{⾒,⼣}[HSK 7-9]
  \definition{v.}{considerar alguém como um estranho; tratar alguém como um estranho; ser excessivamente educado}
\end{EntryWithPhonetic}

\begin{EntryWithPhonetic}{见效}{jian4xiao4}{4,10}{⾒,⽁}[HSK 7-9]
  \definition{v.}{tornar-se eficaz; produzir o resultado desejado; produzir um efeito; ver resultados, surtir efeito}
\end{EntryWithPhonetic}

\begin{EntryWithPhonetic}{见义勇为}{jian4yi4-yong3wei2}{4,3,9,4}{⾒,⼂,⼒,⼂}[HSK 7-9]
  \definition{expr.}{enxergar o que é certo e ter a coragem de fazê-lo; estar pronto para defender uma causa justa com unhas e dentes; faça com ousadia o que é justo; aja com coragem por uma causa justa; ajude um cão manco a passar por cima de uma cerca; nunca hesite quando o bem deve ser feito; nunca hesite em fazer o que é certo; esteja à altura da situação com bravura}
\end{EntryWithPhonetic}

\begin{EntryWithPhonetic}{见证}{jian4zheng4}{4,7}{⾒,⾔}[HSK 7-9]
  \definition{s.}{testemunha; depoimento; testemunhas ou itens que podem servir como prova}
  \definition{v.}{testemunhar; ver algo acontecer; quem presenciou o ocorrido pode confirmar}
\end{EntryWithPhonetic}

%%%%%%%%%% 件 %%%%%%%%%%
\subsection*{件}\addcontentsline{loh}{figure}{件 \dpy{jian4}}

\begin{EntryWithPhonetic}{件}{jian4}{6}{⼈}[HSK 2]
  \definition*{s.}{Sobrenome: Jian}
  \definition{clas.}{item; peça; artigo; usado para coisas individuais}
  \definition{s.}{refere"-se a coisas que podem ser contadas uma a uma | papel; carta; documento; correspondência}
\end{EntryWithPhonetic}

%%%%%%%%%% 间 %%%%%%%%%%
\subsection*{间}\addcontentsline{loh}{figure}{间 \dpy{jian4}}

\begin{EntryWithPhonetic}{间}{jian4}{7}{⾨}
  \definition{s.}{espaço entre as duas partes; abertura; lacuna}
  \definition{v.}{separar | semear a discórdia | desbastar (mudas); podar; remover ou arrancar as mudas em excesso}
  \seeref{jian1}
\end{EntryWithPhonetic}

\begin{EntryWithPhonetic}{间谍}{jian4die2}{7,11}{⾨,⾔}[HSK 7-9]
  \definition[个,名]{s.}{espião; agentes enviados ou recrutados por potências inimigas ou estrangeiras para espionar informações militares, segredos de Estado ou realizar atividades subversivas}
\end{EntryWithPhonetic}

\begin{EntryWithPhonetic}{间断}{jian4duan4}{7,11}{⾨,⽄}[HSK 7-9]
  \definition{s.}{intervalo; salto temporal; disjunção; hiato; interrupção; parada; pulsação; lacuna}
  \definition{v.}{ser desconectado; ser interrompido}
\end{EntryWithPhonetic}

\begin{EntryWithPhonetic}{间隔}{jian4ge2}{7,12}{⾨,⾩}[HSK 7-9]
  \definition{s.}{intervalo; distância das coisas no espaço ou no tempo}
\end{EntryWithPhonetic}

\begin{EntryWithPhonetic}{间或}{jian4huo4}{7,8}{⾨,⼽}
  \definition{adv.}{às vezes | ocasionalmente | de vez em quando}
\end{EntryWithPhonetic}

\begin{EntryWithPhonetic}{间接}{jian4jie1}{7,11}{⾨,⼿}[HSK 5]
  \definition{adj.}{indireto; de segunda mão}
  \antonymref{直接}{zhi2jie1}
\end{EntryWithPhonetic}

\begin{EntryWithPhonetic}{间隙}{jian4xi4}{7,12}{⾨,⾩}[HSK 7-9]
  \definition{s.}{lacuna; espaço; intervalo; tempo ou espaço não utilizado}
\end{EntryWithPhonetic}

%%%%%%%%%% 建 %%%%%%%%%%
\subsection*{建}\addcontentsline{loh}{figure}{建 \dpy{jian4}}

\begin{EntryWithPhonetic}{建}{jian4}{8}{⼵}[HSK 3]
  \definition*{s.}{Província de Fujian | Rio Jian Jiang (na província de Fujian) | Sobrenome: Jian}
  \definition{v.}{construir; construir; erigir | estabelecer; configurar; fundar | propor; defender; apresentar (suas próprias opiniões)}
\end{EntryWithPhonetic}

\begin{EntryWithPhonetic}{建成}{jian4cheng2}{8,6}{⼵,⼽}[HSK 3]
  \definition{v.}{terminar a construção}
\end{EntryWithPhonetic}

\begin{EntryWithPhonetic}{建交}{jian4/jiao1}{8,6}{⼵,⼇}[HSK 7-9]
  \definition{v.+compl.}{estabelecer relações diplomáticas}
\end{EntryWithPhonetic}

\begin{EntryWithPhonetic}{建立}{jian4li4}{8,5}{⼵,⽴}[HSK 3]
  \definition{v.}{estabelecer; construir; começar a construir | vir a ser; começar a surgir; começar a se formar}
\end{EntryWithPhonetic}

\begin{EntryWithPhonetic}{建立者}{jian4li4zhe3}{8,5,8}{⼵,⽴,⽼}
  \definition{s.}{fundador; construtor}
\end{EntryWithPhonetic}

\begin{EntryWithPhonetic}{建设}{jian4she4}{8,6}{⼵,⾔}[HSK 3]
  \definition{s.}{reconstrução; desenvolvimento; trabalhos relacionados com a construção}
  \definition{v.}{construir; edificar; (Estado ou coletividade) criar novos empreendimentos ou aumento de novas instalações}
\end{EntryWithPhonetic}

\begin{EntryWithPhonetic}{建设性}{jian4she4xing4}{8,6,8}{⼵,⾔,⼼}
  \definition{adj.}{construtivo}
  \definition{s.}{construtividade}
\end{EntryWithPhonetic}

\begin{EntryWithPhonetic}{建设者}{jian4she4zhe3}{8,6,8}{⼵,⾔,⽼}
  \definition{s.}{construtor}
\end{EntryWithPhonetic}

\begin{EntryWithPhonetic}{建树}{jian4shu4}{8,9}{⼵,⽊}[HSK 7-9]
  \definition{s.}{realização; conquista}
  \definition{v.}{dar uma contribuição; contribuir | Literário: alcançar uma conquista}
\end{EntryWithPhonetic}

\begin{EntryWithPhonetic}{建议}{jian4yi4}{8,5}{⼵,⾔}[HSK 3]
  \definition[个,点,条]{s.}{proposta; sugestão; recomendação; para que alguém ou alguma coisa evolua para melhor, para o coletivo; pontos de vista e opiniões apresentados pelos líderes, etc.}
  \definition{v.}{propor; sugerir; recomendar; em relação a determinada pessoa ou situação, apresentar seus pontos de vista e opiniões ao coletivo, aos líderes ou a indivíduos, para que as coisas evoluam para melhor}
\end{EntryWithPhonetic}

\begin{EntryWithPhonetic}{建造}{jian4zao4}{8,10}{⼵,⾡}[HSK 5]
  \definition{v.}{construir; edificar}
\end{EntryWithPhonetic}

\begin{EntryWithPhonetic}{建筑}{jian4zhu4}{8,12}{⼵,⽵}[HSK 5]
  \definition[座,幢,排]{s.}{construção; estrutura; edifício; prédio}
  \definition{v.}{construir; erguer; edificar; construir casas, estradas, pontes, etc.}
\end{EntryWithPhonetic}

\begin{EntryWithPhonetic}{建筑师}{jian4zhu4shi1}{8,12,6}{⼵,⽵,⼱}[HSK 7-9]
  \definition{s.}{arquiteto; profissionais de engenharia e técnicos atuantes na indústria da construção}
\end{EntryWithPhonetic}

\begin{EntryWithPhonetic}{建筑物}{jian4zhu4wu4}{8,12,8}{⼵,⽵,⽜}[HSK 7-9]
  \definition[栋]{s.}{edifício; estrutura; projetos de engenharia civil realizados pelo homem, como casas e pontes}
\end{EntryWithPhonetic}

%%%%%%%%%% 剑 %%%%%%%%%%
\subsection*{剑}\addcontentsline{loh}{figure}{剑 \dpy{jian4}}

\begin{EntryWithPhonetic}{剑}{jian4}{9}{⼑}[HSK 6]
  \definition[把,口]{s.}{espada; sabre; florete}
\end{EntryWithPhonetic}

\begin{EntryWithPhonetic}{剑客}{jian4ke4}{9,9}{⼑,⼧}
  \definition{s.}{espada | esgrimista, espadachim}
\end{EntryWithPhonetic}

%%%%%%%%%% 贱 %%%%%%%%%%
\subsection*{贱}\addcontentsline{loh}{figure}{贱 \dpy{jian4}}

\begin{EntryWithPhonetic}{贱}{jian4}{9}{⾙}[HSK 7-9]
  \definition*{s.}{Sobrenome: Jian}
  \definition{adj.}{baixo preço; barato | humilde | baixo; básico; desprezível | humilde; baixa posição social}
  \definition{pron.}{meu (autodepreciativo)}
  \antonymref{贵}{gui4}
\end{EntryWithPhonetic}

%%%%%%%%%% 健 %%%%%%%%%%
\subsection*{健}\addcontentsline{loh}{figure}{健 \dpy{jian4}}

\begin{EntryWithPhonetic}{健}{jian4}{10}{⼈}
  \definition{adj.}{forte; saudável; bem definido | ser forte em; ser bom em; apresentar um grau superior à média em determinado aspecto}
  \definition{v.}{fortalecer; endurecer; revigorar}
\end{EntryWithPhonetic}

\begin{EntryWithPhonetic}{健康}{jian4kang1}{10,11}{⼈,⼴}[HSK 2]
  \definition{adj.}{em forma; saudável; descreve que a pessoa está em ótimo estado físico ou mental, sem nenhum problema | sudável; tudo está normal, sem problemas | saudável; livre de doenças; bom para a saúde}
  \definition{s.}{saúde; físico; estado de saúde}
\end{EntryWithPhonetic}

\begin{EntryWithPhonetic}{健美}{jian4mei3}{10,9}{⼈,⽺}[HSK 7-9]
  \definition{adj.}{forte e bonito; vigoroso e gracioso; robusto e elegante}
  \definition[次]{s.}{fisiculturismo; exercícios que desenvolvem os músculos e o físico}
\end{EntryWithPhonetic}

\begin{EntryWithPhonetic}{健全}{jian4quan2}{10,6}{⼈,⼊}[HSK 5]
  \definition{adj.}{saudável; íntegro; capaz; apto; robusto e sem mácula | sólido; completo; perfeito}
  \definition{v.}{aperfeiçoar; melhorar; fortalecer; reforçar}
\end{EntryWithPhonetic}

\begin{EntryWithPhonetic}{健身}{jian4/shen1}{10,7}{⼈,⾝}[HSK 4]
  \definition{s.}{exercício físico | \emph{fitness}}
  \definition{v.+compl.}{exercitar-se; manter a forma; praticar um esporte, especialmente a ginástica, inclusive em aparelhos, para desenvolver força, flexibilidade, aumentar a resistência, melhorar a coordenação e o controle de todas as partes do corpo}
\end{EntryWithPhonetic}

\begin{EntryWithPhonetic}{健壮}{jian4zhuang4}{10,6}{⼈,⼠}[HSK 7-9]
  \definition{adj.}{robusto; saudável e forte}
\end{EntryWithPhonetic}

%%%%%%%%%% 渐 %%%%%%%%%%
\subsection*{渐}\addcontentsline{loh}{figure}{渐 \dpy{jian4}}

\begin{EntryWithPhonetic}{渐}{jian4}{11}{⽔}
  \definition{adv.}{gradualmente; por graus}
  \seeref{jian1}
\end{EntryWithPhonetic}

\begin{EntryWithPhonetic}{渐渐}{jian4jian4}{11,11}{⽔,⽔}[HSK 4]
  \definition{adv.}{gradualmente; pouco a pouco; passo a passo; indica um aumento ou diminuição gradual em grau ou quantidade}
\end{EntryWithPhonetic}

%%%%%%%%%% 溅 %%%%%%%%%%
\subsection*{溅}\addcontentsline{loh}{figure}{溅 \dpy{jian4}}

\begin{EntryWithPhonetic}{溅}{jian4}{12}{⽔}[HSK 7-9]
  \definition{s.}{respingo; o líquido foi ejetado em todas as direções devido ao impacto}
  \definition{v.}{respingar; salpicar}
\end{EntryWithPhonetic}

%%%%%%%%%% 鉴 %%%%%%%%%%
\subsection*{鉴}\addcontentsline{loh}{figure}{鉴 \dpy{jian4}}

\begin{EntryWithPhonetic}{鉴}{jian4}{13}{⾦}
  \definition*{s.}{Sobrenome: Jian}
  \definition{expr.}{uma expressão idiomática antiga usada para escrever cartas, depois da saudação inicial para pedir que alguém leia a carta}
  \definition{s.}{espelho (feito de bronze ou latão); espelho de bronze antigo | advertência; lição objetiva}
  \definition{v.}{Literário: refletir; espelhar | inspecionar; examinar; escrutinar; olhar cuidadosamente}
\end{EntryWithPhonetic}

\begin{EntryWithPhonetic}{鉴别}{jian4bie2}{13,7}{⾦,⼑}[HSK 7-9]
  \definition{v.}{distinguir; diferenciar; discernir; discriminar; distinguir entre o genuíno e o falso, o bom e o ruim}
\end{EntryWithPhonetic}

\begin{EntryWithPhonetic}{鉴定}{jian4ding4}{13,8}{⾦,⼧}[HSK 6]
  \definition{s.}{avaliação dos pontos fortes e fracos de uma pessoa; avaliação de pessoas ou coisas}
  \definition{v.}{avaliar; identificar; autenticar; determinar; identificar e determinar (a autenticidade e a qualidade das coisas) | conduzir uma avaliação; avaliar o desempenho de uma pessoa ao longo de um determinado período de tempo}
\end{EntryWithPhonetic}

\begin{EntryWithPhonetic}{鉴赏}{jian4shang3}{13,12}{⾦,⾙}[HSK 7-9]
  \definition{v.}{apreciar; avaliar e apreciar (obras de arte, relíquias culturais, etc.)}
\end{EntryWithPhonetic}

\begin{EntryWithPhonetic}{鉴于}{jian4yu2}{13,3}{⾦,⼆}[HSK 7-9]
  \definition{conj.}{considerando; com base em; tendo em vista; vendo que; usado no início da primeira oração de uma frase causal complexa para indicar causa ou fundamento}[鉴于群众反映,服务改善了。===Considerando o feedback do público, o serviço foi aprimorado.]
  \definition{prep.}{tendo em vista; à luz de; apresente uma situação observada ou considerada como base para a ação}
\end{EntryWithPhonetic}

%%%%%%%%%% 键 %%%%%%%%%%
\subsection*{键}\addcontentsline{loh}{figure}{键 \dpy{jian4}}

\begin{EntryWithPhonetic}{键}{jian4}{13}{⾦}[HSK 5]
  \definition[个]{s.}{chave | tecla (de uma máquina de escrever, piano, etc.) | Química: ligação | Literário: ferrolho (de uma porta) | pino (para máquinas)  | etapa crucial}
\end{EntryWithPhonetic}

\begin{EntryWithPhonetic}{键盘}{jian4pan2}{13,11}{⾦,⽫}[HSK 5]
  \definition[台,个]{s.}{teclado; cravo; painel de teclas}
\end{EntryWithPhonetic}

%%%%%%%%%% 箭 %%%%%%%%%%
\subsection*{箭}\addcontentsline{loh}{figure}{箭 \dpy{jian4}}

\begin{EntryWithPhonetic}{箭}{jian4}{15}{⽵}[HSK 6]
  \definition[支]{s.}{seta | distância percorrida por uma flecha}
\end{EntryWithPhonetic}

%%%%%%%%%% 江 %%%%%%%%%%
\subsection*{江}\addcontentsline{loh}{figure}{江 \dpy{jiang1}}

\begin{EntryWithPhonetic}{江}{jiang1}{6}{⽔}[HSK 4]
  \definition*{s.}{Rio Changjiang | Sobrenome: Jiang}
  \definition[条,道]{s.}{rio grande}
\end{EntryWithPhonetic}

\begin{EntryWithPhonetic}{江南水乡}{jiang1nan2shui3xiang1}{6,9,4,3}{⽔,⼗,⽔,⼄}
  \definition*{s.}{Vila Aquática de Jiangnan | Cidades Aquáticas}
\end{EntryWithPhonetic}

\begin{EntryWithPhonetic}{江水}{jiang1shui3}{6,4}{⽔,⽔}
  \definition{s.}{água do rio}
\end{EntryWithPhonetic}

\begin{EntryWithPhonetic}{江苏}{jiang1su1}{6,7}{⽔,⾋}
  \definition*{s.}{Província de Jiangsu}
\end{EntryWithPhonetic}

\begin{EntryWithPhonetic}{江西}{jiang1xi1}{6,6}{⽔,⾑}
  \definition*{s.}{Jiangxi}
\end{EntryWithPhonetic}

%%%%%%%%%% 姜 %%%%%%%%%%
\subsection*{姜}\addcontentsline{loh}{figure}{姜 \dpy{jiang1}}

\begin{EntryWithPhonetic}{姜}{jiang1}{9}{⼥}[HSK 7-9]
  \definition*{s.}{Sobrenome: Jiang}
  \definition[磅,斤,两]{s.}{gengibre; rizoma de gengibre}
\end{EntryWithPhonetic}

%%%%%%%%%% 将 %%%%%%%%%%
\subsection*{将}\addcontentsline{loh}{figure}{将 \dpy{jiang1}}

\begin{EntryWithPhonetic}{将}{jiang1}{9}{⼨}[HSK 5]
  \definition*{s.}{Sobrenome: Jiang}
  \definition{adv.}{estar indo para; parcialmente\dots parcialmente\dots}
  \definition{part.}{expressar uma direção, como 进来, 出去; usado no meio de verbos e complementos que indicam tendência, como 进来, 出去, etc.}
  \definition{prep.}{com; por meio de; por | usado da mesma forma que 把}
  \definition{v.}{fazer algo; lidar com (um assunto) | dar um cheque-mate | cuidar (da saúde) | incitar alguém a agir; desafiar; estimular | segurar; pegar | colocar; tirar | levar; trazer | dar suporte; dar apoio}
  \seeref{jiang4}
  \seeref{qiang1}
  \seealsoref{把}{ba3}
  \seealsoref{出去}{chu1 qu5}
  \seealsoref{进来}{jin4 lai5}
\end{EntryWithPhonetic}

\begin{EntryWithPhonetic}{将近}{jiang1jin4}{9,7}{⼨,⾡}[HSK 3]
  \definition{adv.}{quase}
\end{EntryWithPhonetic}

\begin{EntryWithPhonetic}{将军}{jiang1/jun1}{9,6}{⼨,⼍}[HSK 6]
  \definition[位,名]{s.}{general; geralmente se refere a generais seniores}
  \definition{v.+compl.}{dar xeque-mate; atacar o general ou rei do oponente no xadrez; colocar alguém em grandes apuros; metáfora para dar a alguém um problema difícil ou dificultar a tarefa para essa pessoa}
\end{EntryWithPhonetic}

\begin{EntryWithPhonetic}{将来}{jiang1lai2}{9,7}{⼨,⽊}[HSK 3]
  \definition[个]{s.}{no futuro (geralmente se refere a um período mais longo)}
\end{EntryWithPhonetic}

\begin{EntryWithPhonetic}{将要}{jiang1yao4}{9,9}{⼨,⾑}[HSK 5]
  \definition{adv.}{irá; deverá; estará prestes a; irá a; indica que um ato ou situação ocorre logo em seguida}
\end{EntryWithPhonetic}

%%%%%%%%%% 僵 %%%%%%%%%%
\subsection*{僵}\addcontentsline{loh}{figure}{僵 \dpy{jiang1}}

\begin{EntryWithPhonetic}{僵}{jiang1}{15}{⼈}[HSK 7-9]
  \definition{adj.}{rígido; paralisado; congelado | rígido; austero; duro | Dialeto: em impasse; tenso}
  \definition{v.}{Dialeto: parar de sorrir; ficar com uma expressão séria}
\end{EntryWithPhonetic}

\begin{EntryWithPhonetic}{僵化}{jiang1hua4}{15,4}{⼈,⼔}[HSK 7-9]
  \definition{v.}{tornar-se rígido; ossificar; tornar-se estereotipado; parar de se desenvolver; petrificar; inativar}
\end{EntryWithPhonetic}

\begin{EntryWithPhonetic}{僵局}{jiang1ju2}{15,7}{⼈,⼫}[HSK 7-9]
  \definition[个,种]{s.}{impasse; beco sem saída; a questão é difícil de resolver e o progresso está paralisado}
\end{EntryWithPhonetic}

%%%%%%%%%% 讲 %%%%%%%%%%
\subsection*{讲}\addcontentsline{loh}{figure}{讲 \dpy{jiang3}}

\begin{EntryWithPhonetic}{讲}{jiang3}{6}{⾔}[HSK 2]
  \definition[种]{s.}{palestra; discurso}
  \definition{v.}{contar; falar | explicar; transmitir oralmente; esclarecer | negociar; barganhar | ser exigente com; valorizar; dar importância}
\end{EntryWithPhonetic}

\begin{EntryWithPhonetic}{讲话}{jiang3hua4}{6,8}{⾔,⾔}[HSK 2]
  \definition[个]{s.}{discurso; palestra | guia; introdução}
  \definition{v.}{falar; conversar; dirigir-se a alguém | criticar}
\end{EntryWithPhonetic}

\begin{EntryWithPhonetic}{讲解}{jiang3jie3}{6,13}{⾔,⾓}[HSK 7-9]
  \definition{v.}{explicar; interpretar; expor}
\end{EntryWithPhonetic}

\begin{EntryWithPhonetic}{讲究}{jiang3jiu5}{6,7}{⾔,⽳}[HSK 4]
  \definition{adj.}{requintado; elegante; de bom gosto; exigente com a vida e com outros aspectos, buscando alto nível, qualidade e detalhes}
  \definition{s.}{estudo cuidadoso; algo que merece atenção; elementos e aspectos que merecem atenção especial}
  \definition{v.}{dar ênfase a; ser específico sobre; prestar atenção a}
\end{EntryWithPhonetic}

\begin{EntryWithPhonetic}{讲课}{jiang3 ke4}{6,10}{⾔,⾔}[HSK 6]
  \definition{v.}{ensinar; dar palestras; proferir uma palestra | dar uma lição (palestra)}
\end{EntryWithPhonetic}

\begin{EntryWithPhonetic}{讲述}{jiang3shu4}{6,8}{⾔,⾡}[HSK 7-9]
  \definition{v.}{narrar; relatar; contar sobre; dar um relato de}
\end{EntryWithPhonetic}

\begin{EntryWithPhonetic}{讲学}{jiang3/xue2}{6,8}{⾔,⼦}[HSK 7-9]
  \definition{v.+compl.}{ministrar palestras; discorrer sobre um assunto acadêmico}
\end{EntryWithPhonetic}

\begin{EntryWithPhonetic}{讲座}{jiang3zuo4}{6,10}{⾔,⼴}[HSK 4]
  \definition[场,次]{s.}{palestra; um curso de palestras; a forma de instrução usada para ensinar um determinado assunto ou tópico, geralmente por meio de palestras ao vivo, seriados de rádio ou televisão ou seriados de jornal.}
\end{EntryWithPhonetic}

%%%%%%%%%% 奖 %%%%%%%%%%
\subsection*{奖}\addcontentsline{loh}{figure}{奖 \dpy{jiang3}}

\begin{EntryWithPhonetic}{奖}{jiang3}{9}{⼤}[HSK 4]
  \definition[个,次]{s.}{prêmio; recompensa | elogio; loa}
  \definition{v.}{elogiar; recompensar; recomendar; incentivar}
\end{EntryWithPhonetic}

\begin{EntryWithPhonetic}{奖杯}{jiang3bei1}{9,8}{⼤,⽊}[HSK 7-9]
  \definition[个,座]{s.}{taça (como prêmio); copa; troféu; os prêmios em formato de taça, concedidos aos vencedores em competições esportivas, geralmente são feitos de ouro ou prata}
\end{EntryWithPhonetic}

\begin{EntryWithPhonetic}{奖金}{jiang3jin1}{9,8}{⼤,⾦}[HSK 4]
  \definition[个,笔]{s.}{bônus; recompensa; prêmio; prêmio em dinheiro; dinheiro de recompensa, dinheiro dado às pessoas para incentivá-las ou elogiá-las por terem se saído bem em alguma coisa}
\end{EntryWithPhonetic}

\begin{EntryWithPhonetic}{奖励}{jiang3li4}{9,7}{⼤,⼒}[HSK 5]
  \definition{s.}{prêmio; recompensa; dinheiro ou honras dadas em troca de elogios ou incentivos}
  \definition{v.}{recompensar; incentivar; encorajar}
\end{EntryWithPhonetic}

\begin{EntryWithPhonetic}{奖牌}{jiang3pai2}{9,12}{⼤,⽚}[HSK 7-9]
  \definition{s.}{medalha (concedida como prêmio); as medalhas são divididas em ouro, prata e bronze; os vencedores em competições esportivas recebem prêmios de acordo com esses três níveis}
\end{EntryWithPhonetic}

\begin{EntryWithPhonetic}{奖品}{jiang3pin3}{9,9}{⼤,⼝}[HSK 7-9]
  \definition[个,些,份]{s.}{prêmio; itens para recompensa}
\end{EntryWithPhonetic}

\begin{EntryWithPhonetic}{奖项}{jiang3xiang4}{9,9}{⼤,⾴}[HSK 7-9]
  \definition[项]{s.}{prêmio; projetos premiados}
\end{EntryWithPhonetic}

\begin{EntryWithPhonetic}{奖学金}{jiang3xue2jin1}{9,8,8}{⼤,⼦,⾦}[HSK 4]
  \definition[笔]{s.}{bolsa de estudos; exposição; prêmios concedidos por escolas, organizações ou indivíduos a alunos com bom desempenho acadêmico}
\end{EntryWithPhonetic}

%%%%%%%%%% 匠 %%%%%%%%%%
\subsection*{匠}\addcontentsline{loh}{figure}{匠 \dpy{jiang4}}

\begin{EntryWithPhonetic}{匠}{jiang4}{6}{⼕}
  \definition*{s.}{Sobrenome: Jiang}
  \definition{s.}{artesão | pessoa de realizações notáveis em um campo específico; mestre}
\end{EntryWithPhonetic}

%%%%%%%%%% 降 %%%%%%%%%%
\subsection*{降}\addcontentsline{loh}{figure}{降 \dpy{jiang4}}

\begin{EntryWithPhonetic}{降}{jiang4}{8}{⾩}[HSK 4]
  \definition*{s.}{Sobrenome: Jiang}
  \definition{v.}{cair; descer; quedar-se | diminuir; reduzir; cair; abaixar | nascer}
  \antonymref{升}{sheng1}
\end{EntryWithPhonetic}

\begin{EntryWithPhonetic}{降低}{jiang4di1}{8,7}{⾩,⼈}[HSK 4]
  \definition{v.}{reduzir; cortar; diminuir; rebaixar; cair; abaixar}
\end{EntryWithPhonetic}

\begin{EntryWithPhonetic}{降价}{jiang4 jia4}{8,6}{⾩,⼈}[HSK 4]
  \definition{v.}{ficar mais barato; cortar o preço; reduzir o preço}
\end{EntryWithPhonetic}

\begin{EntryWithPhonetic}{降临}{jiang4lin2}{8,9}{⾩,⼁}[HSK 7-9]
  \definition{v.}{acontecer; chegar; vir}[春天降临,万物复苏。===A primavera chega e tudo renasce.]
\end{EntryWithPhonetic}

\begin{EntryWithPhonetic}{降落}{jiang4luo4}{8,12}{⾩,⾋}[HSK 4]
  \definition{v.}{aterrissar; descer; descer do céu}
\end{EntryWithPhonetic}

\begin{EntryWithPhonetic}{降温}{jiang4 wen1}{8,12}{⾩,⽔}[HSK 4]
  \definition{v.}{baixar a temperatura (como em uma oficina);  recusar | cair a temperatura | esfriar; resfriar; metáfora para um declínio no entusiasmo ou uma diminuição no ímpeto de algo}
\end{EntryWithPhonetic}

%%%%%%%%%% 将 %%%%%%%%%%
\subsection*{将}\addcontentsline{loh}{figure}{将 \dpy{jiang4}}

\begin{EntryWithPhonetic}{将}{jiang4}{9}{⼨}
  \definition{s.}{general; nome do posto; abaixo de marechal de campo; acima de coronel}
  \definition{v.}{comandar; liderar}
  \seeref{jiang1}
  \seeref{qiang1}
\end{EntryWithPhonetic}

%%%%%%%%%% 强 %%%%%%%%%%
\subsection*{强}\addcontentsline{loh}{figure}{强 \dpy{jiang4}}

\begin{EntryWithPhonetic}{强}{jiang4}{12}{⼸}
  \definition{adj.}{teimoso; inflexível}
  \seeref{qiang2}
  \seeref{qiang3}
\end{EntryWithPhonetic}

%%%%%%%%%% 酱 %%%%%%%%%%
\subsection*{酱}\addcontentsline{loh}{figure}{酱 \dpy{jiang4}}

\begin{EntryWithPhonetic}{酱}{jiang4}{13}{⾣}[HSK 6]
  \definition{adj.}{marinado em molho de soja; cozido em molho de soja}
  \definition{s.}{molho espesso feito de soja, farinha, etc. | molho; pasta; geleia | um condimento pastoso feito de feijão, trigo fermentados e sal}
  \definition{v.}{cozinhar ou conservar em molho de soja}
\end{EntryWithPhonetic}

\begin{EntryWithPhonetic}{酱油}{jiang4you2}{13,8}{⾣,⽔}[HSK 6]
  \definition[袋,瓶,壶,桶]{s.}{molho de soja}
\end{EntryWithPhonetic}

%%%%%%%%%% 犟 %%%%%%%%%%
\subsection*{犟}\addcontentsline{loh}{figure}{犟 \dpy{jiang4}}

\begin{EntryWithPhonetic}{犟}{jiang4}{16}{⽜}
  \variantof{强}
\end{EntryWithPhonetic}

%%%%%%%%%% 交 %%%%%%%%%%
\subsection*{交}\addcontentsline{loh}{figure}{交 \dpy{jiao1}}

\begin{EntryWithPhonetic}{交}{jiao1}{6}{⼇}[HSK 2]
  \definition*{s.}{Sobrenome: Jiao}
  \definition{adv.}{mutuamente; recíprocamente; um ao outro | juntos; simultaneamente}
  \definition{s.}{amigo; conhecido; amizade; relacionamento | transação comercial; negócio; barganha | queda}
  \definition{v.}{entregar | (de lugares ou períodos de tempo) cruzar; encontrar; unir | chegar (a uma determinada hora ou estação); estabelecer-se; vir | cruzar; intersectar | associar-se a | ter relações sexuais | acasalar; reproduzir-se | transferir as coisas para as partes interessadas | unir (lugares ou períodos de tempo)}
  \antonymref{接}{jie1}
  \antonymref{收}{shou1}
\end{EntryWithPhonetic}

\begin{EntryWithPhonetic}{交班}{jiao1ban1}{6,10}{⼇,⽟}
  \definition{v.}{passar para o próximo turno de trabalho}
  \synonymref{接班}{jie1/ban1}
\end{EntryWithPhonetic}

\begin{EntryWithPhonetic}{交杯酒}{jiao1bei1jiu3}{6,8,10}{⼇,⽊,⾣}
  \definition{s.}{copo de vinho nupcial}
\end{EntryWithPhonetic}

\begin{EntryWithPhonetic}{交叉}{jiao1cha1}{6,3}{⼇,⼜}[HSK 7-9]
  \definition{v.}{cruzar; entrecruzar; interseccionar | coincidir | revezar-se}
\end{EntryWithPhonetic}

\begin{EntryWithPhonetic}{交叉点}{jiao1cha1dian3}{6,3,9}{⼇,⼜,⽕}
  \definition{s.}{interseção; cruzamento; encruzilhada; ponto de interseção; junção}
\end{EntryWithPhonetic}

\begin{EntryWithPhonetic}{交叉口}{jiao1cha1kou3}{6,3,3}{⼇,⼜,⼝}
  \definition{s.}{intersecção (rodovia)}
\end{EntryWithPhonetic}

\begin{EntryWithPhonetic}{交代}{jiao1dai4}{6,5}{⼇,⼈}[HSK 5]
  \definition{v.}{contar; entregar | ordenar; insistir; contar aos outros sobre suas intenções, instruções | contar; admitir}
  \synonymref{打发}{da3fa5}
  \synonymref{叮嘱}{ding1zhu3}
  \synonymref{吩咐}{fen1fu4}
  \synonymref{交接}{jiao1jie1}
  \synonymref{派遣}{pai4qian3}
  \synonymref{嘱托}{zhu3tuo1}
  \synonymref{嘱咐}{zhu3fu5}
  \antonymref{抗拒}{kang4ju4}
\end{EntryWithPhonetic}

\begin{EntryWithPhonetic}{交叠}{jiao1die2}{6,13}{⼇,⼜}
  \definition{s.}{sobreposição}
  \synonymref{重叠}{chong2die2}
  \synonymref{交集}{jiao1ji2}
\end{EntryWithPhonetic}

\begin{EntryWithPhonetic}{交费}{jiao1fei4}{6,9}{⼇,⾙}[HSK 3]
  \definition{v.}{pagar taxas ou impostos; pagar uma taxa ou imposto}
\end{EntryWithPhonetic}

\begin{EntryWithPhonetic}{交锋}{jiao1/feng1}{6,12}{⼇,⾦}[HSK 7-9]
  \definition{v.+compl.}{entrar em conflito; cruzar espadas; confrontar; participar de uma batalha ou disputa; ter um confronto (com alguém)}
  \synonymref{打仗}{da3/zhang4}
  \synonymref{接触}{jie1chu4}
  \synonymref{战争}{zhan4zheng1}
  \antonymref{逃避}{tao2bi4}
\end{EntryWithPhonetic}

\begin{EntryWithPhonetic}{交付}{jiao1fu4}{6,5}{⼇,⼈}[HSK 7-9]
  \definition{v.}{entregar; passar a responsabilidade; transferir a responsabilidade; pagar a}
\end{EntryWithPhonetic}

\begin{EntryWithPhonetic}{交给}{jiao1gei3}{6,9}{⼇,⽷}[HSK 2]
  \definition{v.}{entregar para | dar para}
\end{EntryWithPhonetic}

\begin{EntryWithPhonetic}{交媾}{jiao1gou4}{6,13}{⼇,⼥}
  \definition{v.}{copular | ter relações sexuais}
\end{EntryWithPhonetic}

\begin{EntryWithPhonetic}{交换}{jiao1huan4}{6,10}{⼇,⼿}[HSK 4]
  \definition{v.}{trocar; permutar; comutar; intercambiar}
  \synonymref{撤换}{che4huan4}
  \synonymref{兑换}{dui4huan4}
  \synonymref{交流}{jiao1liu2}
  \synonymref{替换}{ti4huan4}
  \antonymref{换取}{huan4qu3}
\end{EntryWithPhonetic}

\begin{EntryWithPhonetic}{交集}{jiao1ji2}{6,12}{⼇,⾫}[HSK 7-9]
  \definition{s.}{sobreposição; conexão; terreno comum; pontos em comum; intersecção; convergência}
  \definition{v.}{(diferentes sentimentos) estar misturado; ocorrer simultaneamente}
  \synonymref{暴躁}{bao4zao4}
  \synonymref{烦躁}{fan2zao4}
  \synonymref{慌张}{huang1zhang1}
  \synonymref{焦虑}{jiao1lv4}
  \synonymref{焦躁}{jiao1zao4}
  \synonymref{恐慌}{kong3huang1}
  \synonymref{着急}{zhao2/ji2}
  \antonymref{分散}{fen1san4}
\end{EntryWithPhonetic}

\begin{EntryWithPhonetic}{交际}{jiao1ji4}{6,7}{⼇,⾩}[HSK 4]
  \definition{s.}{contato; comunicação; relações sociais; contato interpessoal, socialização}
  \synonymref{社交}{she4jiao1}
  \synonymref{外交}{wai4jiao1}
\end{EntryWithPhonetic}

\begin{EntryWithPhonetic}{交接}{jiao1jie1}{6,11}{⼇,⼿}[HSK 7-9]
  \definition{v.}{juntar-se; conectar-se; conectar | entregar e assumir o controle; transferir | associar-se a; fazer amizade com; fazer amigos}
  \synonymref{交代}{jiao1dai4}
\end{EntryWithPhonetic}

\begin{EntryWithPhonetic}{交界}{jiao1jie4}{6,9}{⼇,⽥}[HSK 7-9]
  \definition{v.}{ter uma fronteira comum; ter um limite comum; fazer fronteira com}
\end{EntryWithPhonetic}

\begin{EntryWithPhonetic}{交警}{jiao1jing3}{6,19}{⼇,⾔}[HSK 3]
  \definition{s.}{policial de trânsito, abreviação de 交通警察}
  \seealsoref{交通警察}{jiao1tong1 jing3cha2}
\end{EntryWithPhonetic}

\begin{EntryWithPhonetic}{交流}{jiao1liu2}{6,10}{⼇,⽔}[HSK 3]
  \definition{v.}{trocar; interagir; comunicar-se; compartilhar o que cada um tem com o outro}
  \synonymref{对话}{dui4hua4}
  \synonymref{沟通}{gou1tong1}
  \synonymref{互动}{hu4dong4}
  \synonymref{换取}{huan4qu3}
  \synonymref{交换}{jiao1huan4}
  \antonymref{封闭}{feng1bi4}
\end{EntryWithPhonetic}

\begin{EntryWithPhonetic}{交纳}{jiao1na4}{6,7}{⼇,⽷}[HSK 7-9]
  \definition{v.}{pagar (ao estado ou a uma organização); entregar uma quantia predeterminada de dinheiro ou bens a um governo ou órgão público}
  \synonymref{缴纳}{jiao3na4}
\end{EntryWithPhonetic}

\begin{EntryWithPhonetic}{交朋友}{jiao1 peng2you3}{6,8,4}{⼇,⽉,⼜}[HSK 2]
  \definition{v.}{fazer amizade com alguém; fazer amigos}
  \synonymref{打交道}{da3 jiao1dao5}
\end{EntryWithPhonetic}

\begin{EntryWithPhonetic}{交情}{jiao1qing5}{6,11}{⼇,⼼}[HSK 7-9]
  \definition{s.}{amizade; relação amigável}
  \synonymref{情谊}{qing2yi4}
  \synonymref{友谊}{you3yi4}
\end{EntryWithPhonetic}

\begin{EntryWithPhonetic}{交涉}{jiao1she4}{6,10}{⼇,⽔}[HSK 7-9]
  \definition{v.}{negociar; entrar em contato com; fazer representações; discutir soluções para questões relacionadas com a outra parte}
  \synonymref{对话}{dui4hua4}
  \synonymref{谈判}{tan2pan4}
  \synonymref{协商}{xie2shang1}
\end{EntryWithPhonetic}

\begin{EntryWithPhonetic}{交谈}{jiao1tan2}{6,10}{⼇,⾔}[HSK 7-9]
  \definition{v.}{conversar; bater papo; falar um com o outro}
\end{EntryWithPhonetic}

\begin{EntryWithPhonetic}{交替}{jiao1ti4}{6,12}{⼇,⽈}[HSK 7-9]
  \definition{v.}{dar lugar a; substituir; suplantar; substituir coisas antigas por coisas novas | alternar; revezar-se}
  \synonymref{轮流}{lun2liu2}
  \antonymref{维持}{wei2chi2}
\end{EntryWithPhonetic}

\begin{EntryWithPhonetic}{交通}{jiao1tong1}{6,10}{⼇,⾡}[HSK 2]
  \definition{s.}{tráfego | ligação; conexão | transporte; termo genérico para todos os tipos de transporte, como ferroviário e rodoviário}
  \definition{v.}{conspirar; fazer amizades; conchavar | estar conectado; estar ligado; estar vinculado | associar-se a; conspirar com}
  \synonymref{通行}{tong1xing2}
  \synonymref{运输}{yun4shu1}
  \antonymref{堵塞}{du3se4}
\end{EntryWithPhonetic}

\begin{EntryWithPhonetic}{交通警察}{jiao1tong1 jing3cha2}{6,10,19,14}{⼇,⾡,⾔,⼧}
  \definition{s.}{policial de trânsito}
  \seealsoref{交警}{jiao1jing3}
\end{EntryWithPhonetic}

\begin{EntryWithPhonetic}{交头接耳}{jiao1tou2-jie1'er3}{6,5,11,6}{⼇,⼤,⼿,⽿}[HSK 7-9]
  \definition{expr.}{falar ao ouvido um do outro; sussurrar um para o outro; trocar sussurros confidenciais; sussurrar um ao ouvido do outro; cochichar um com o outro}
\end{EntryWithPhonetic}

\begin{EntryWithPhonetic}{交往}{jiao1wang3}{6,8}{⼇,⼻}[HSK 3]
  \definition{v.}{estar em contato com; associar-se a; interagir}
  \synonymref{交易}{jiao1yi4}
  \synonymref{来往}{lai2wang3}
  \synonymref{往来}{wang3lai2}
  \synonymref{相处}{xiang1chu3}
  \antonymref{断交}{duan4/jiao1}
\end{EntryWithPhonetic}

\begin{EntryWithPhonetic}{交响}{jiao1xiang3}{6,9}{⼇,⼝}
  \definition{s.}{sinfonia}
  \synonymref{共鸣}{gong4ming2}
\end{EntryWithPhonetic}

\begin{EntryWithPhonetic}{交响乐}{jiao1xiang3yue4}{6,9,5}{⼇,⼝,⼃}[HSK 7-9]
  \definition{s.}{sinfonia; música sinfônica; as peças musicais de grande escala executadas por uma orquestra normalmente consistem em quatro movimentos e são capazes de expressar pensamentos e sentimentos diversos e complexos}
\end{EntryWithPhonetic}

\begin{EntryWithPhonetic}{交易}{jiao1yi4}{6,8}{⼇,⽇}[HSK 3]
  \definition[笔,桩,个,场]{s.}{negócio; comércio; transação comercial; transação; atividades de compra e venda de mercadorias}
  \definition{v.}{negociar; comprar e vender mercadorias}
  \synonymref{成交}{cheng2/jiao1}
  \synonymref{交往}{jiao1wang3}
  \synonymref{来往}{lai2wang3}
  \synonymref{买卖}{mai3mai4}
  \synonymref{买卖}{mai3mai5}
  \synonymref{贸易}{mao4yi4}
  \synonymref{生意}{sheng1yi5}
  \synonymref{业务}{ye4wu4}
  \synonymref{营业}{ying2ye4}
  \antonymref{赠送}{zeng4song4}
\end{EntryWithPhonetic}

\begin{EntryWithPhonetic}{交运}{jiao1yun4}{6,7}{⼇,⾡}
  \definition{v.}{despachar (bagagem em um aeroporto, etc.) | entregar para transporte}
\end{EntryWithPhonetic}

%%%%%%%%%% 郊 %%%%%%%%%%
\subsection*{郊}\addcontentsline{loh}{figure}{郊 \dpy{jiao1}}

\begin{EntryWithPhonetic}{郊}{jiao1}{8}{⾢}
  \definition*{s.}{Sobrenome: Jiao}
  \definition{s.}{subúrbios; periferias; áreas ao redor da cidade}
\end{EntryWithPhonetic}

\begin{EntryWithPhonetic}{郊区}{jiao1qu1}{8,4}{⾢,⼖}[HSK 5]
  \definition[个,片,块]{s.}{subúrbios; arredores; periferia; área ao redor da cidade que está administrativamente sob a jurisdição da cidade}
\end{EntryWithPhonetic}

\begin{EntryWithPhonetic}{郊外}{jiao1wai4}{8,5}{⾢,⼣}[HSK 7-9]
  \definition{s.}{subúrbio; periferia; a zona rural ao redor de uma cidade; a área fora da cidade (referindo"-se a uma cidade específica)}
\end{EntryWithPhonetic}

\begin{EntryWithPhonetic}{郊游}{jiao1you2}{8,12}{⾢,⽔}[HSK 7-9]
  \definition{s.}{passeio; excursão}
\end{EntryWithPhonetic}

%%%%%%%%%% 娇 %%%%%%%%%%
\subsection*{娇}\addcontentsline{loh}{figure}{娇 \dpy{jiao1}}

\begin{EntryWithPhonetic}{娇}{jiao1}{9}{⼥}
  \definition{adj.}{terno; adorável; encantador | frágil; delicado | melindroso; exigente}
  \definition{v.}{mimar; estragar}
\end{EntryWithPhonetic}

\begin{EntryWithPhonetic}{娇惯}{jiao1guan4}{9,11}{⼥,⼼}[HSK 7-9]
  \definition{v.}{mimar; paparicar; estragar}
\end{EntryWithPhonetic}

\begin{EntryWithPhonetic}{娇气}{jiao1qi4}{9,4}{⼥,⽓}[HSK 7-9]
  \definition{adj.}{exigente; melindroso; com personalidade frágil, incapaz de suportar dificuldades ou injustiças | terno; frágil; delicado; (os itens) são facilmente danificados; (as plantas) não são fáceis de cultivar}
  \definition{s.}{delicadeza; fragilidade; personalidade e estilo frágeis}
\end{EntryWithPhonetic}

%%%%%%%%%% 浇 %%%%%%%%%%
\subsection*{浇}\addcontentsline{loh}{figure}{浇 \dpy{jiao1}}

\begin{EntryWithPhonetic}{浇}{jiao1}{9}{⽔}[HSK 7-9]
  \definition{adj.}{decadente; precipitado e pérfido}
  \definition{v.}{derramar líquido sobre; borrifar água sobre | aguar; regar; irrigar | injetar fluido no molde}
\end{EntryWithPhonetic}

%%%%%%%%%% 骄 %%%%%%%%%%
\subsection*{骄}\addcontentsline{loh}{figure}{骄 \dpy{jiao1}}

\begin{EntryWithPhonetic}{骄}{jiao1}{9}{⾺}
  \definition{adj.}{orgulhoso; arrogante; vaidoso | Literário: feroz; intenso; forte; violento}
\end{EntryWithPhonetic}

\begin{EntryWithPhonetic}{骄傲}{jiao1'ao4}{9,12}{⾺,⼈}[HSK 6]
  \definition{adj.}{arrogante; vaidoso; orgulhoso}
  \definition{s.}{orgulho; pessoas ou coisas das quais se orgulhar}
\end{EntryWithPhonetic}

%%%%%%%%%% 胶 %%%%%%%%%%
\subsection*{胶}\addcontentsline{loh}{figure}{胶 \dpy{jiao1}}

\begin{EntryWithPhonetic}{胶}{jiao1}{10}{⾁}
  \definition*{s.}{Sobrenome: Jiao}
  \definition{adj.}{pegajoso; viscoso; grudento}
  \definition{s.}{cola; goma; adesivo | borracha | gel; colóide}
  \definition{v.}{colar com cola | colar; grudar}
\end{EntryWithPhonetic}

\begin{EntryWithPhonetic}{胶带}{jiao1dai4}{10,9}{⾁,⼱}[HSK 5]
  \definition[卷,条,段]{s.}{fita de embalagem transparente; fita adesiva | fita magnética de plástico; fita de gravação | fita emborrachada; cinta de borracha}
\end{EntryWithPhonetic}

\begin{EntryWithPhonetic}{胶卷}{jiao1juan3}{10,8}{⾁,⼙}
  \definition{s.}{filme | rolo de filme}
\end{EntryWithPhonetic}

\begin{EntryWithPhonetic}{胶囊}{jiao1nang2}{10,22}{⾁,⾐}[HSK 7-9]
  \definition{s.}{Medicina: cápsula; refere"-se a uma cápsula de gelatina usada para encapsular medicamentos em pó ou granulados, facilitando a ingestão}
\end{EntryWithPhonetic}

\begin{EntryWithPhonetic}{胶片}{jiao1pian4}{10,4}{⾁,⽚}[HSK 7-9]
  \definition[卷]{s.}{filme; cartucho; filme fotográfico}
\end{EntryWithPhonetic}

\begin{EntryWithPhonetic}{胶水}{jiao1shui3}{10,4}{⾁,⽔}[HSK 5]
  \definition[瓶]{s.}{cola; mucilagem; cola líquida}
\end{EntryWithPhonetic}

%%%%%%%%%% 教 %%%%%%%%%%
\subsection*{教}\addcontentsline{loh}{figure}{教 \dpy{jiao1}}

\begin{EntryWithPhonetic}{教}{jiao1}{11}{⽁}
  \definition*{s.}{Sobrenome: Jiao}
  \definition{prep.}{em uma frase passiva para introduzir o executor da ação}
  \definition{s.}{religião | professor; referência à educação ou aos professores}
  \definition{v.}{ensinar; instruir |  pedir; ordenar; dizer | permitir; possibilitar}
  \seeref{jiao4}
  \synonymref{授}{shou4}
  \antonymref{学}{xue2}
\end{EntryWithPhonetic}

\begin{EntryWithPhonetic}{教会}{jiao1hui4}{11,6}{⽁,⼈}
  \definition{v.}{mostrar | ensinar}
  \seeref{jiao4hui4}
\end{EntryWithPhonetic}

%%%%%%%%%% 焦 %%%%%%%%%%
\subsection*{焦}\addcontentsline{loh}{figure}{焦 \dpy{jiao1}}

\begin{EntryWithPhonetic}{焦}{jiao1}{12}{⽕}[HSK 7-9]
  \definition*{s.}{Sobrenome: Jiao}
  \definition{adj.}{queimado; chamuscado; carbonizado | preocupado; ansioso}
  \definition{clas.}{J; Joule, abreviação}
  \definition{pref.}{(química) piro-}
  \definition{s.}{Metalurgia: coque}
\end{EntryWithPhonetic}

\begin{EntryWithPhonetic}{焦点}{jiao1dian3}{12,9}{⽕,⽕}[HSK 6]
  \definition{s.}{foco; ponto focal; Matemática: refere"-se a um ponto que tem uma relação especial com uma elipse, hipérbole, parábola, etc. | foco; ponto focal; Óptica: refere"-se à intersecção de feixes de luz paralelos após serem refratados por uma lente ou refletidos por um espelho curvo | foco; questão central; metaforicamente, uma coisa ou princípio que chama a atenção para o foco}
\end{EntryWithPhonetic}

\begin{EntryWithPhonetic}{焦急}{jiao1ji2}{12,9}{⽕,⼼}[HSK 7-9]
  \definition{adj.}{ansioso; preocupado; com pressa}
\end{EntryWithPhonetic}

\begin{EntryWithPhonetic}{焦距}{jiao1ju4}{12,11}{⽕,⾜}[HSK 7-9]
  \definition{s.}{Ótica: distância focal; comprimento focal}
\end{EntryWithPhonetic}

\begin{EntryWithPhonetic}{焦虑}{jiao1lv4}{12,10}{⽕,⾌}[HSK 7-9]
  \definition{adj.}{ansioso; preocupado; apreensivo}
\end{EntryWithPhonetic}

\begin{EntryWithPhonetic}{焦躁}{jiao1zao4}{12,20}{⽕,⾜}[HSK 7-9]
  \definition{adj.}{impaciente; inquieto de ansiedade; ansioso e irritadiço}
\end{EntryWithPhonetic}

%%%%%%%%%% 礁 %%%%%%%%%%
\subsection*{礁}\addcontentsline{loh}{figure}{礁 \dpy{jiao1}}

\begin{EntryWithPhonetic}{礁}{jiao1}{17}{⽯}
  \definition{s.}{recife; rocha}
\end{EntryWithPhonetic}

\begin{EntryWithPhonetic}{礁石}{jiao1shi2}{17,5}{⽯,⽯}[HSK 7-9]
  \definition{s.}{recife; rocha}
\end{EntryWithPhonetic}

%%%%%%%%%% 矫 %%%%%%%%%%
\subsection*{矫}\addcontentsline{loh}{figure}{矫 \dpy{jiao2}}

\begin{EntryWithPhonetic}{矫}{jiao2}{11}{⽮}
  \definition{s.}{usado em 矫情}
  \seeref{jiao3}
  \seealsoref{矫情}{jiao2qing5}
\end{EntryWithPhonetic}

\begin{EntryWithPhonetic}{矫情}{jiao2qing2}{11,11}{⽮,⼼}
  \definition{v.}{ser afetadamente não convencional; fingir ser incomum; ir deliberadamente contra o senso comum para demonstrar superioridade ou ser diferente dos outros}
  \seeref{jiao2qing5}
\end{EntryWithPhonetic}

\begin{EntryWithPhonetic}{矫情}{jiao2qing5}{11,11}{⽮,⼼}
  \definition{adj.}{briguento; contencioso; irracional; isso se refere a apresentar argumentos descabidos e causar problemas.}
  \seeref{jiao2qing2}
\end{EntryWithPhonetic}

%%%%%%%%%% 嚼 %%%%%%%%%%
\subsection*{嚼}\addcontentsline{loh}{figure}{嚼 \dpy{jiao2}}

\begin{EntryWithPhonetic}{嚼}{jiao2}{20}{⼝}[HSK 7-9]
  \definition{v.}{mastigar; mascar; limitado para uso em 过屠门而大嚼}
  \seeref{jiao4}
  \seeref{jue2}
  \seealsoref{过屠门而大嚼}{guo4 tu2men2 er2 da4 jiao2}
\end{EntryWithPhonetic}

%%%%%%%%%% 角 %%%%%%%%%%
\subsection*{角}\addcontentsline{loh}{figure}{角 \dpy{jiao3}}

\begin{EntryWithPhonetic}{角}{jiao3}{7}{⾓}[HSK 2][Kangxi 148]
  \definition*{s.}{Jiao, uma das mansões lunares}
  \definition{clas.}{uma unidade monetária fracionária na China (=1/10 de um yuan ou 10 fen)}
  \definition[个,只,对]{s.}{chifre; o objeto duro que cresce na cabeça de bovinos, ovinos, veados, etc. | buzina; corneta; instrumentos musicais tocados no exército antigo | algo com a forma de um chifre | cabo; promontório; península | esquina; canto; a junção entre duas arestas de um objeto | ângulo}
  \seeref{jue2}
\end{EntryWithPhonetic}

\begin{EntryWithPhonetic}{角度}{jiao3du4}{7,9}{⾓,⼴}[HSK 2]
  \definition[个,种]{s.}{perspectiva; ponto de vista; o ponto de partida para ver as coisas | ângulo; o tamanho do ângulo; normalmente expresso em graus ou radianos}
\end{EntryWithPhonetic}

\begin{EntryWithPhonetic}{角落}{jiao3luo4}{7,12}{⾓,⾋}[HSK 7-9]
  \definition[个,处]{s.}{canto; recanto; o ângulo côncavo na junção de duas paredes ou estruturas semelhantes | um lugar remoto}
\end{EntryWithPhonetic}

%%%%%%%%%% 狡 %%%%%%%%%%
\subsection*{狡}\addcontentsline{loh}{figure}{狡 \dpy{jiao3}}

\begin{EntryWithPhonetic}{狡}{jiao3}{9}{⽝}
  \definition{adj.}{astuto; esperto; ardiloso}
\end{EntryWithPhonetic}

\begin{EntryWithPhonetic}{狡猾}{jiao3hua2}{9,12}{⽝,⽜}[HSK 7-9]
  \definition{adj.}{astuto; ardiloso; manhoso; esperto; escorregadio; astuto e indigno de confiança}
\end{EntryWithPhonetic}

%%%%%%%%%% 绞 %%%%%%%%%%
\subsection*{绞}\addcontentsline{loh}{figure}{绞 \dpy{jiao3}}

\begin{EntryWithPhonetic}{绞}{jiao3}{9}{⽷}[HSK 7-9]
  \definition{clas.}{meada; novelo; utilizado para fios, lã, etc.}
  \definition{v.}{torcer; espremer; emaranhar | dar corda | picar; moer | pendurar pelo pescoço; estrangular | entrelaçar}
\end{EntryWithPhonetic}

%%%%%%%%%% 饺 %%%%%%%%%%
\subsection*{饺}\addcontentsline{loh}{figure}{饺 \dpy{jiao3}}

\begin{EntryWithPhonetic}{饺}{jiao3}{9}{⾷}
  \definition[盘,碗,顿,个]{s.}{bolinho de massa; \emph{dumpling}}
\end{EntryWithPhonetic}

\begin{EntryWithPhonetic}{饺子}{jiao3zi5}{9,3}{⾷,⼦}[HSK 2]
  \definition[个,盘,碗,锅]{s.}{jiaozi; bolinho chinês; bolinho de massa}
\end{EntryWithPhonetic}

%%%%%%%%%% 矫 %%%%%%%%%%
\subsection*{矫}\addcontentsline{loh}{figure}{矫 \dpy{jiao3}}

\begin{EntryWithPhonetic}{矫}{jiao3}{11}{⽮}
  \definition*{s.}{Sobrenome: Jiao}
  \definition{adj.}{forte; corajoso}
  \definition{v.}{retificar; corrigir; resolver | fingir; simular; dissimular}
  \seeref{jiao2}
\end{EntryWithPhonetic}

\begin{EntryWithPhonetic}{矫正}{jiao3zheng4}{11,5}{⽮,⽌}[HSK 7-9]
  \definition{v.}{corrigir; retificar}
\end{EntryWithPhonetic}

%%%%%%%%%% 脚 %%%%%%%%%%
\subsection*{脚}\addcontentsline{loh}{figure}{脚 \dpy{jiao3}}

\begin{EntryWithPhonetic}{脚}{jiao3}{11}{⾁}[HSK 2]
  \definition{clas.}{usado para chutes}
  \definition[只,双]{s.}{pé; a parte inferior das pernas de pessoas ou animais, que entra em contato com o solo | base; pé; a parte inferior do objeto | antigamente, referia"-se ao trabalho físico de transporte de cargas | resíduos; sobras}
\end{EntryWithPhonetic}

\begin{EntryWithPhonetic}{脚步}{jiao3bu4}{11,7}{⾁,⽌}[HSK 5]
  \definition{s.}{pé; passo; pisada; refere"-se ao movimento das pernas ao caminhar | ritmo; passo; distância entre os pés dianteiros e traseiros ao caminhar}
\end{EntryWithPhonetic}

\begin{EntryWithPhonetic}{脚印}{jiao3yin4}{11,5}{⾁,⼙}[HSK 6]
  \definition{s.}{trilha; pegada; marca de pé; os rastros deixados pelos passos}
\end{EntryWithPhonetic}

%%%%%%%%%% 搅 %%%%%%%%%%
\subsection*{搅}\addcontentsline{loh}{figure}{搅 \dpy{jiao3}}

\begin{EntryWithPhonetic}{搅}{jiao3}{12}{⼿}[HSK 7-9]
  \definition{v.}{mexer; misturar | perturbar; incomodar; interromper}
\end{EntryWithPhonetic}

\begin{EntryWithPhonetic}{搅拌}{jiao3ban4}{12,8}{⼿,⼿}[HSK 7-9]
  \definition{v.}{misturar; mexer; agitar; usar uma colher, um palito ou um utensílio semelhante para girar a mistura e homogeneizá-la}
\end{EntryWithPhonetic}

%%%%%%%%%% 缴 %%%%%%%%%%
\subsection*{缴}\addcontentsline{loh}{figure}{缴 \dpy{jiao3}}

\begin{EntryWithPhonetic}{缴}{jiao3}{16}{⽷}[HSK 7-9]
  \definition{v.}{pagar | capturar | entregar (em)}
  \seeref{zhuo2}
\end{EntryWithPhonetic}

\begin{EntryWithPhonetic}{缴费}{jiao3fei4}{16,9}{⽷,⾙}[HSK 7-9]
  \definition{v.}{pagar taxas; a taxa exigida para pagar por um serviço ou produto}
\end{EntryWithPhonetic}

\begin{EntryWithPhonetic}{缴纳}{jiao3na4}{16,7}{⽷,⽷}[HSK 7-9]
  \definition{v.}{pagar algo ao público; pagar; receber; colocar em}
\end{EntryWithPhonetic}

%%%%%%%%%% 叫 %%%%%%%%%%
\subsection*{叫}\addcontentsline{loh}{figure}{叫 \dpy{jiao4}}

\begin{EntryWithPhonetic}{叫}{jiao4}{5}{⼝}[HSK 1,3]
  \definition{adj.}{macho (animal)}
  \definition{prep.}{usado em frases passivas; introduz o agente da ação; equivalente a 被 | combinado com 看, 说; usado para expressar suas ideias e pontos de vista}
  \definition{v.}{chorar; gritar; berrar | nomear; chamar | chamar; chamar a atenção | cumprimentar; saudar; dizer olá | pedir; ordenar; licitar | permitir; concordar com algo; concordar em fazer algo | contratar; encomendar; comprar o que você precisa}
  \seealsoref{被}{bei4}
  \seealsoref{看}{kan4}
  \seealsoref{说}{shuo1}
\end{EntryWithPhonetic}

\begin{EntryWithPhonetic}{叫板}{jiao4/ban3}{5,8}{⼝,⽊}[HSK 7-9]
  \definition{v.+compl.}{Coloquial: desafiar | sinalizar aos músicos (na ópera chinesa, prolongando uma palavra falada antes de iniciar uma canção)}
\end{EntryWithPhonetic}

\begin{EntryWithPhonetic}{叫好}{jiao4/hao3}{5,6}{⼝,⼥}[HSK 7-9]
  \definition{v.+compl.}{aplaudir; gritar ``Bravo!''; gritar ``Muito bem!'' | aplaudir | torcer}
\end{EntryWithPhonetic}

\begin{EntryWithPhonetic}{叫作}{jiao4zuo4}{5,7}{⼝,⼈}[HSK 2]
  \definition{v.}{ser chamado de; ser conhecido como}
\end{EntryWithPhonetic}

%%%%%%%%%% 觉 %%%%%%%%%%
\subsection*{觉}\addcontentsline{loh}{figure}{觉 \dpy{jiao4}}

\begin{EntryWithPhonetic}{觉}{jiao4}{9}{⾒}[HSK 6]
  \definition[个]{s.}{sono; o processo desde adormecer até acordar}
  \seeref{jue2}
\end{EntryWithPhonetic}

%%%%%%%%%% 校 %%%%%%%%%%
\subsection*{校}\addcontentsline{loh}{figure}{校 \dpy{jiao4}}

\begin{EntryWithPhonetic}{校}{jiao4}{10}{⽊}
  \definition{v.}{verificar | comparar | revisar}
  \seeref{xiao4}
\end{EntryWithPhonetic}

%%%%%%%%%% 轿 %%%%%%%%%%
\subsection*{轿}\addcontentsline{loh}{figure}{轿 \dpy{jiao4}}

\begin{EntryWithPhonetic}{轿}{jiao4}{10}{⾞}
  \definition{s.}{liteira; palanquim; cadeira de arruar}
\end{EntryWithPhonetic}

\begin{EntryWithPhonetic}{轿车}{jiao4che1}{10,4}{⾞,⾞}[HSK 7-9]
  \definition[辆]{s.}{carruagem (puxada por cavalos); carruagens puxadas por animais com cortinas cobrindo os compartimentos de passageiros antigamente | ônibus; carro; sedã; um carro relativamente luxuoso e confortável, com teto e assentos para passageiros}
\end{EntryWithPhonetic}

%%%%%%%%%% 较 %%%%%%%%%%
\subsection*{较}\addcontentsline{loh}{figure}{较 \dpy{jiao4}}

\begin{EntryWithPhonetic}{较}{jiao4}{10}{⾞}[HSK 3]
  \definition{adj.}{claro; óbvio; evidente}
  \definition{adv.}{comparativamente; relativamente; razoavelmente; bastante; bastante}
  \definition{prep.}{usado para comparar características e graus; introduzir o objeto de comparação; equivalente a 比}
  \definition{v.}{comparar | disputar}
  \seealsoref{比}{bi3}
\end{EntryWithPhonetic}

\begin{EntryWithPhonetic}{较劲}{jiao4/jin4}{10,7}{⾞,⼒}[HSK 7-9]
  \definition{v.+compl.}{exigir esforço extra | adequar a própria força a (competição de força; disputa de habilidade) | colocar-se contra alguém; ir um contra o outro}
\end{EntryWithPhonetic}

\begin{EntryWithPhonetic}{较量}{jiao4liang4}{10,12}{⾞,⾥}[HSK 7-9]
  \definition{v.}{realizar uma competição; medir a própria força com; determinar quem é superior ou inferior através da competição, da luta ou de outros meios | regatear; discutir; disputar; calcular}
\end{EntryWithPhonetic}

%%%%%%%%%% 敎 %%%%%%%%%%
\subsection*{敎}\addcontentsline{loh}{figure}{敎 \dpy{jiao4}}

\begin{EntryWithPhonetic}{敎}{jiao4}{11}{⽁}
  \variantof{教}
\end{EntryWithPhonetic}

%%%%%%%%%% 教 %%%%%%%%%%
\subsection*{教}\addcontentsline{loh}{figure}{教 \dpy{jiao4}}

\begin{EntryWithPhonetic}{教}{jiao4}{11}{⽁}[HSK 1]
  \definition*{s.}{Sobrenome: Jiao}
  \definition{prep.}{em uma frase passiva para apresentar o autor da ação}
  \definition{s.}{religião | educação; professor}
  \definition{v.}{ensinar; instruir | perguntar; ordenar; contar | permitir; permitir}
  \seeref{jiao1}
  \synonymref{授}{shou4}
  \antonymref{学}{xue2}
\end{EntryWithPhonetic}

\begin{EntryWithPhonetic}{教材}{jiao4cai2}{11,7}{⽁,⽊}[HSK 3]
  \definition[本,套]{s.}{livro didático; materiais didáticos, incluindo livros didáticos, apostilas, materiais de referência, vídeos, imagens, etc.}
\end{EntryWithPhonetic}

\begin{EntryWithPhonetic}{教导}{jiao4dao3}{11,6}{⽁,⼨}
  \definition{s.}{instrução | orientação | ensino}
  \definition{v.}{instruir | orientar | ensinar}
\end{EntryWithPhonetic}

\begin{EntryWithPhonetic}{教官}{jiao4guan1}{11,8}{⽁,⼧}
  \definition{s.}{instrutor militar; um oficial que serviu como treinador no antigo exército ou escola}
\end{EntryWithPhonetic}

\begin{EntryWithPhonetic}{教会}{jiao4hui4}{11,6}{⽁,⼈}
  \definition{s.}{igreja cristã}
  \seeref{jiao1hui4}
\end{EntryWithPhonetic}

\begin{EntryWithPhonetic}{教科书}{jiao4ke1shu1}{11,9,4}{⽁,⽲,⼄}[HSK 7-9]
  \definition[本]{s.}{livro didático; livro do aluno | livro escolar; um livro escrito especialmente para alunos usarem em sala de aula e para revisão}
\end{EntryWithPhonetic}

\begin{EntryWithPhonetic}{教练}{jiao4lian4}{11,8}{⽁,⽷}[HSK 3]
  \definition[个,位,名]{s.}{instrutor; treinador (esportes); pessoas que trabalham como treinadores}
  \definition{v.}{treinar; treinar outras pessoas para dominarem uma determinada técnica (como esportes, dirigir carros, pilotar aviões, etc.)}
\end{EntryWithPhonetic}

\begin{EntryWithPhonetic}{教师}{jiao4shi1}{11,6}{⽁,⼱}[HSK 2]
  \definition[个,位,名]{s.}{professor; professor de escola}
\end{EntryWithPhonetic}

\begin{EntryWithPhonetic}{教室}{jiao4shi4}{11,9}{⽁,⼧}[HSK 2]
  \definition[间]{s.}{sala de aula}
\end{EntryWithPhonetic}

\begin{EntryWithPhonetic}{教授}{jiao4shou4}{11,11}{⽁,⼿}[HSK 4]
  \definition[个,位,名]{s.}{professor (universitário); o professor com a classificação mais alta em uma universidade}
  \definition{v.}{ensinar; instruir; dar aulas; dar palestras}
\end{EntryWithPhonetic}

\begin{EntryWithPhonetic}{教堂}{jiao4tang2}{11,11}{⽁,⼟}[HSK 6]
  \definition[座,所,间]{s.}{igreja; capela; catedral; casa de deus; um lugar onde os cristãos realizam cerimônias religiosas}
\end{EntryWithPhonetic}

\begin{EntryWithPhonetic}{教条}{jiao4tiao2}{11,7}{⽁,⽊}[HSK 7-9]
  \definition{adj.}{dogmático; opinativo}
  \definition{s.}{dogma; doutrina; credo | princípio; chavão}
\end{EntryWithPhonetic}

\begin{EntryWithPhonetic}{教学}{jiao4xue2}{11,8}{⽁,⼦}[HSK 2]
  \definition[个,门]{s.}{ensino; educação; o processo de transmissão de conhecimentos e habilidades}
\end{EntryWithPhonetic}

\begin{EntryWithPhonetic}{教学楼}{jiao4xue2lou2}{11,8,13}{⽁,⼦,⽊}[HSK 1]
  \definition{s.}{prédio da escola; bloco de ensino; edifícios utilizados para atividades educacionais, geralmente incluindo salas de aula, laboratórios, auditórios, etc.}
\end{EntryWithPhonetic}

\begin{EntryWithPhonetic}{教训}{jiao4xun5}{11,5}{⽁,⾔}[HSK 4]
  \definition[个,次,番,顿]{s.}{moral; lição}
  \definition{v.}{repreender; educar; ensinar uma lição a alguém; dar uma bronca em alguém; dar um sermão em alguém (por ter cometido um erro, etc.)}
\end{EntryWithPhonetic}

\begin{EntryWithPhonetic}{教养}{jiao4yang3}{11,9}{⽁,⼋}[HSK 7-9]
  \definition{s.}{criação; educação; formação}
\end{EntryWithPhonetic}

\begin{EntryWithPhonetic}{教育}{jiao4yu4}{11,8}{⽁,⾁}[HSK 2]
  \definition{s.}{educação; refere"-se a atividades sociais cujo objetivo direto é influenciar o desenvolvimento físico e mental das pessoas; refere"-se principalmente ao processo de formação dos alunos nas escolas}
  \definition{v.}{ensinar; educar; inspirar, fazer compreender a razão}
\end{EntryWithPhonetic}

\begin{EntryWithPhonetic}{教育部}{jiao4yu4bu4}{11,8,10}{⽁,⾁,⾢}[HSK 6]
  \definition*{s.}{Ministério da Educação}
\end{EntryWithPhonetic}

\begin{EntryWithPhonetic}{教长}{jiao4zhang3}{11,4}{⽁,⾧}
  \definition{s.}{imã (Islã) | mulá}
\end{EntryWithPhonetic}

%%%%%%%%%% 嚼 %%%%%%%%%%
\subsection*{嚼}\addcontentsline{loh}{figure}{嚼 \dpy{jiao4}}

\begin{EntryWithPhonetic}{嚼}{jiao4}{20}{⼝}
  \definition{v.}{mascar; ruminar}
\end{EntryWithPhonetic}

%%%%%%%%%% 节 %%%%%%%%%%
\subsection*{节}\addcontentsline{loh}{figure}{节 \dpy{jie1}}

\begin{EntryWithPhonetic}{节}{jie1}{5}{⾋}
  \definition{adj.}{momento crucial; momento crítico; momento decisivo; metáfora para algo importante, decisivo ou oportuno}
  \seeref{jie2}
\end{EntryWithPhonetic}

%%%%%%%%%% 阶 %%%%%%%%%%
\subsection*{阶}\addcontentsline{loh}{figure}{阶 \dpy{jie1}}

\begin{EntryWithPhonetic}{阶}{jie1}{6}{⾩}
  \definition{s.}{degrau; escada; escadaria | classificação | escala | ordem | estágio}
\end{EntryWithPhonetic}

\begin{EntryWithPhonetic}{阶层}{jie1ceng2}{6,7}{⾩,⼫}[HSK 7-9]
  \definition{s.}{posição; seção; estrato (social); isso se refere à estratificação dentro da mesma classe com base em diferentes status socioeconômicos, como a divisão da classe camponesa em camponeses pobres, camponeses médios, etc.}
\end{EntryWithPhonetic}

\begin{EntryWithPhonetic}{阶段}{jie1duan4}{6,9}{⾩,⽎}[HSK 4]
  \definition[个,段]{s.}{estágio; fase; período; bancada; gradação}
\end{EntryWithPhonetic}

\begin{EntryWithPhonetic}{阶级}{jie1ji2}{6,6}{⾩,⽷}[HSK 7-9]
  \definition[个,种]{s.}{classe (social); grupos sociais divididos de acordo com o \emph{status} socioeconômico das pessoas | degraus; escadas; passos | classificação; número de passos}
\end{EntryWithPhonetic}

\begin{EntryWithPhonetic}{阶梯}{jie1ti1}{6,11}{⾩,⽊}[HSK 7-9]
  \definition{s.}{lance de escadas; escada; degraus e escadas são metáforas para meios ou caminhos de ascensão social; equipamentos que funcionam de maneira semelhante a escadas}
\end{EntryWithPhonetic}

%%%%%%%%%% 皆 %%%%%%%%%%
\subsection*{皆}\addcontentsline{loh}{figure}{皆 \dpy{jie1}}

\begin{EntryWithPhonetic}{皆}{jie1}{9}{⽩}[HSK 7-9]
  \definition{adv.}{todos; em todos os casos; cada um e todos}
\end{EntryWithPhonetic}

%%%%%%%%%% 结 %%%%%%%%%%
\subsection*{结}\addcontentsline{loh}{figure}{结 \dpy{jie1}}

\begin{EntryWithPhonetic}{结}{jie1}{9}{⽷}[HSK 7-9]
  \definition{v.}{dar (frutos); formar (sementes); produzir frutos ou sementes (uma planta)}
  \seeref{jie2}
\end{EntryWithPhonetic}

\begin{EntryWithPhonetic}{结果}{jie1/guo3}{9,8}{⽷,⽊}[HSK 7-9]
  \definition{v.+compl.}{frutificar; dar frutos}
  \seeref{jie2/guo3}
\end{EntryWithPhonetic}

\begin{EntryWithPhonetic}{结实}{jie1shi5}{9,8}{⽷,⼧}[HSK 3]
  \definition{adj.}{sólido; resistente; durável | forte; resistente; robusto}
\end{EntryWithPhonetic}

%%%%%%%%%% 接 %%%%%%%%%%
\subsection*{接}\addcontentsline{loh}{figure}{接 \dpy{jie1}}

\begin{EntryWithPhonetic}{接}{jie1}{11}{⼿}[HSK 2]
  \definition*{s.}{Sobrenome: Jie}
  \definition{v.}{entrar em contato com; aproximar-se de | conectar; unir; juntar | continuar; prosseguir | assumir o controle; assumir o trabalho de outra pessoa e continuar a fazê-lo | pegar; agarrar; segurar ou sustentar com as mãos | receber; aceitar | encontrar; dar as boas-vindas}
\end{EntryWithPhonetic}

\begin{EntryWithPhonetic}{接班}{jie1/ban1}{11,10}{⼿,⽟}[HSK 7-9]
  \definition{v.}{assumir o turno de alguém; substituir alguém; assumir o lugar de; dar continuidade a; (sucessor) Assumir o trabalho do turno anterior | ter sucesso; dar continuidade a algo iniciado por seu antecessor}
  \seealsoref{接班儿}{jie1ban1r5}
\end{EntryWithPhonetic}

\begin{EntryWithPhonetic}{接班儿}{jie1ban1r5}{11,10,2}{⼿,⽟,⼉}
  \definition{v.}{assumir o turno de alguém; substituir alguém | ter sucesso; dar continuidade a algo iniciado por seu antecessor}
  \seealsoref{接班}{jie1/ban1}
\end{EntryWithPhonetic}

\begin{EntryWithPhonetic}{接班人}{jie1ban1ren2}{11,10,2}{⼿,⽟,⼈}[HSK 7-9]
  \definition{s.}{sucessor; a pessoa que assume o trabalho do turno anterior é frequentemente usada metaforicamente}
\end{EntryWithPhonetic}

\begin{EntryWithPhonetic}{接触}{jie1chu4}{11,13}{⼿,⾓}[HSK 5]
  \definition{v.}{entrar em contato com | entrar em contato; tocar; interagir | engajar; o termo militar refere"-se a fogo cruzado}
\end{EntryWithPhonetic}

\begin{EntryWithPhonetic}{接待}{jie1dai4}{11,9}{⼿,⼻}[HSK 3]
  \definition{v.}{receber (alguém); acolher; recepcionar; receber com cordialidade e generosidade}
\end{EntryWithPhonetic}

\begin{EntryWithPhonetic}{接到}{jie1dao4}{11,8}{⼿,⼑}[HSK 2]
  \definition{v.}{receber (carta, etc.)}
\end{EntryWithPhonetic}

\begin{EntryWithPhonetic}{接(电话)}{jie1(dian4hua4)}{11,5,8}{⼿,⽥,⾔}
  \definition{v.}{atender (o telefone) | receber (uma ligação telefônica)}
\end{EntryWithPhonetic}

\begin{EntryWithPhonetic}{接二连三}{jie1'er4-lian2san1}{11,2,7,3}{⼿,⼆,⾡,⼀}[HSK 7-9]
  \definition{expr.}{um após o outro; em rápida sucessão}
\end{EntryWithPhonetic}

\begin{EntryWithPhonetic}{接轨}{jie1/gui3}{11,6}{⼿,⾞}[HSK 7-9]
  \definition{s.}{junção; integração; ligação}
  \definition{v.+compl.}{ligar; juntar; conectar os trilhos | integrar; juntar-se a; mudar para; entrar na onda; alinhar-se; alinhar a; essa metáfora descreve como sistemas e métodos podem ser interconectados e consistentes}
\end{EntryWithPhonetic}

\begin{EntryWithPhonetic}{接济}{jie1ji4}{11,9}{⼿,⽔}[HSK 7-9]
  \definition{v.}{prestar assistência material a; dar ajuda financeira a; prestar auxílio material a}
\end{EntryWithPhonetic}

\begin{EntryWithPhonetic}{接见}{jie1jian4}{11,4}{⼿,⾒}[HSK 7-9]
  \definition{v.}{receber alguém; conceder uma entrevista a; reunir-se com as pessoas que vieram}
\end{EntryWithPhonetic}

\begin{EntryWithPhonetic}{接近}{jie1jin4}{11,7}{⼿,⾡}[HSK 3]
  \definition{adj.}{perto; próximo; a diferença entre os dois é mínima}
  \definition{v.}{estar perto de; aproximar; aproximar-se}
\end{EntryWithPhonetic}

\begin{EntryWithPhonetic}{接力}{jie1li4}{11,2}{⼿,⼒}[HSK 7-9]
  \definition{s.}{relé; trabalho por revezamento; revezamento}
\end{EntryWithPhonetic}

\begin{EntryWithPhonetic}{接连}{jie1lian2}{11,7}{⼿,⾡}[HSK 5]
  \definition{adv.}{no final; em sucessão; em uma fileira; um após o outro; seguindo o anterior; continuando}
\end{EntryWithPhonetic}

\begin{EntryWithPhonetic}{接纳}{jie1na4}{11,7}{⼿,⽷}[HSK 7-9]
  \definition{v.}{ser admitido (em uma organização); aceitar (como membro); incluir (indivíduos ou grupos que ingressam na organização) | adotar; aceitar; tomar}
\end{EntryWithPhonetic}

\begin{EntryWithPhonetic}{接收}{jie1shou1}{11,6}{⼿,⽁}[HSK 6]
  \definition{v.}{aceitar; receber | assumir; expropriar; tomar posse (de uma instituição, propriedade, etc.) de acordo com a lei | admitir; aceitar; absorver}
\end{EntryWithPhonetic}

\begin{EntryWithPhonetic}{接手}{jie1shou3}{11,4}{⼿,⼿}[HSK 7-9]
  \definition{v.}{assumir (responsabilidades, etc.); assumir problemas; assumir o trabalho de outra pessoa.}
\end{EntryWithPhonetic}

\begin{EntryWithPhonetic}{接受}{jie1shou4}{11,8}{⼿,⼜}[HSK 2]
  \definition{v.}{aceitar; não recusar (o que os outros oferecem) | concordar; não recusar (opiniões/sugestões/críticas/convites de outras pessoas, etc.)}
\end{EntryWithPhonetic}

\begin{EntryWithPhonetic}{接送}{jie1song4}{11,9}{⼿,⾡}[HSK 7-9]
  \definition{v.}{buscar e levar}
\end{EntryWithPhonetic}

\begin{EntryWithPhonetic}{接替}{jie1ti4}{11,12}{⼿,⽈}[HSK 7-9]
  \definition{v.}{assumir o controle; substituir | suceder; ocupar o lugar de}
\end{EntryWithPhonetic}

\begin{EntryWithPhonetic}{接听}{jie1ting1}{11,7}{⼿,⼝}[HSK 7-9]
  \definition{v.}{atender (o telefone)}
\end{EntryWithPhonetic}

\begin{EntryWithPhonetic}{接通}{jie1tong1}{11,10}{⼿,⾡}[HSK 7-9]
  \definition{v.}{transmitir; fazer ligação telefônica | conectar; completar a ligação; conseguir passar | fechar; encerrar; interromper; inserir; ativar; ligar; completar}
\end{EntryWithPhonetic}

\begin{EntryWithPhonetic}{接下来}{jie1xia4lai2}{11,3,7}{⼿,⼀,⽊}[HSK 2]
  \definition{expr.}{próximo; seguinte; indica uma sequência temporal subsequente}
\end{EntryWithPhonetic}

\begin{EntryWithPhonetic}{接着}{jie1zhe5}{11,11}{⼿,⽬}[HSK 2]
  \definition{adv.}{por sua vez; um após o outro; sucessivamente; conectado (à frase anterior); imediatamente após (a ação anterior)}
  \definition{v.}{seguir; prosseguir; continuar; seguir em frente; ficar ao lado | pegar com as mãos; apanhar}
\end{EntryWithPhonetic}

%%%%%%%%%% 揭 %%%%%%%%%%
\subsection*{揭}\addcontentsline{loh}{figure}{揭 \dpy{jie1}}

\begin{EntryWithPhonetic}{揭}{jie1}{12}{⼿}[HSK 6]
  \definition*{s.}{Sobrenome: Jie}
  \definition{v.}{rasgar; arrancar; tirar | descobrir; levantar (a tampa, etc.) | expor; mostrar; trazer à luz | (literário) levantar; içar}
\end{EntryWithPhonetic}

\begin{EntryWithPhonetic}{揭发}{jie1fa1}{12,5}{⼿,⼜}[HSK 7-9]
  \definition{v.}{expor; desmascarar; trazer à luz; expor e denunciar (pessoas más e más ações)}
\end{EntryWithPhonetic}

\begin{EntryWithPhonetic}{揭露}{jie1lu4}{12,21}{⼿,⾬}[HSK 7-9]
  \definition{v.}{expor; desmascarar; descobrir; revelar o que estava oculto}
\end{EntryWithPhonetic}

\begin{EntryWithPhonetic}{揭示}{jie1shi4}{12,5}{⼿,⽰}[HSK 7-9]
  \definition{v.}{anunciar; promulgar; exibir publicamente | revelar; desvendar; trazer à luz; apontar ou esclarecer a essência de coisas que não são facilmente visíveis}
\end{EntryWithPhonetic}

\begin{EntryWithPhonetic}{揭晓}{jie1xiao3}{12,10}{⼿,⽇}[HSK 7-9]
  \definition{v.}{revelar; anunciar; tornar conhecido; divulgar publicamente os resultados da investigação para que todos fiquem cientes}
\end{EntryWithPhonetic}

%%%%%%%%%% 街 %%%%%%%%%%
\subsection*{街}\addcontentsline{loh}{figure}{街 \dpy{jie1}}

\begin{EntryWithPhonetic}{街}{jie1}{12}{⾏}[HSK 2]
  \definition[条]{s.}{rua; avenida com prédios dos dois lados | mercado; feira rural}
\end{EntryWithPhonetic}

\begin{EntryWithPhonetic}{街道}{jie1dao4}{12,12}{⾏,⾡}[HSK 4]
  \definition[条]{s.}{caminho; rua; estrada; via pública com casas em ambos os lados, relativamente larga | escritório do subdistrito; tipo de organização responsável por gerenciar determinados aspectos da rua}
\end{EntryWithPhonetic}

\begin{EntryWithPhonetic}{街头}{jie1tou2}{12,5}{⾏,⼤}[HSK 6]
  \definition{s.}{rua; esquina da rua}
\end{EntryWithPhonetic}

\begin{EntryWithPhonetic}{街舞}{jie1wu3}{12,14}{⾏,⾇}
  \definition{s.}{dança de rua, \emph{street dance} (por exemplo, \emph{breakdance})}
\end{EntryWithPhonetic}

%%%%%%%%%% 楷 %%%%%%%%%%
\subsection*{楷}\addcontentsline{loh}{figure}{楷 \dpy{jie1}}

\begin{EntryWithPhonetic}{楷}{jie1}{13}{⽊}
  \definition{s.}{árvore de pistache chinês}
  \seeref{kai3}
\end{EntryWithPhonetic}

%%%%%%%%%% 节 %%%%%%%%%%
\subsection*{节}\addcontentsline{loh}{figure}{节 \dpy{jie2}}

\begin{EntryWithPhonetic}{节}{jie2}{5}{⾋}[HSK 2,6]
  \definition*{s.}{Sobrenome: Jie}
  \definition{clas.}{nó (kn), velocidade de um barco | para seções, comprimentos}
  \definition[个]{s.}{junta; botão; nó; geralmente se refere à parte da grama ou caule da grama onde as folhas crescem ou à parte onde os galhos e troncos das plantas são conectados | parte; divisão; um trecho de algo interligado; uma parte do todo | festival; feriado; dia memorável; um período de tempo ou um dia com características específicas | item; assunto | castidade; integridade ética e moral | articulação; o local onde os ossos humanos ou animais se conectam | etiqueta; cerimonial | batida; ritmo | registro; documento utilizado na antiguidade para comprovar a identidade | estação do ano | sílaba}
  \definition{v.}{economizar; conservar; poupar | resumir; extrair; retirar uma parte do todo | controlar; restringir; moderar}
  \seeref{jie1}
\end{EntryWithPhonetic}

\begin{EntryWithPhonetic}{节假日}{jie2jia4ri4}{5,11,4}{⾋,⼈,⽇}[HSK 6]
  \definition[个]{s.}{feriados; festivais e feriados}
\end{EntryWithPhonetic}

\begin{EntryWithPhonetic}{节俭}{jie2jian3}{5,9}{⾋,⼈}[HSK 7-9]
  \definition{adj.}{econômico; frugal}
\end{EntryWithPhonetic}

\begin{EntryWithPhonetic}{节目}{jie2mu4}{5,5}{⾋,⽬}[HSK 2]
  \definition[个,场,项,台]{s.}{programa; item (em um programa); programas artísticos ou projetos transmitidos por rádios e televisões}
\end{EntryWithPhonetic}

\begin{EntryWithPhonetic}{节能}{jie2 neng2}{5,10}{⾋,⾁}[HSK 6]
  \definition{v.}{economizar no consumo de energia; conservar energia}
\end{EntryWithPhonetic}

\begin{EntryWithPhonetic}{节气}{jie2qi4}{5,4}{⾋,⽓}[HSK 7-9]
  \definition{s.}{termo solar (é qualquer um dos 24 períodos nos calendários lunisolares tradicionais chineses); com base na duração do dia e da noite, na altura da sombra do meio-dia e em outros fatores, vários pontos são designados ao longo do ano, cada ponto é chamado de termo solar; os termos solares indicam a posição da Terra em sua órbita, ou seja, a posição do Sol na eclíptica; geralmente, também se referem ao dia da semana em que cada ponto ocorre}
\end{EntryWithPhonetic}

\begin{EntryWithPhonetic}{节日}{jie2ri4}{5,4}{⾋,⽇}[HSK 2]
  \definition[个,种,类]{s.}{festival; feriado; dia de comemoração tradicional; dia comemorativo estabelecido por lei}
\end{EntryWithPhonetic}

\begin{EntryWithPhonetic}{节省}{jie2sheng3}{5,9}{⾋,⽬}[HSK 4]
  \definition{adj.}{econômico; parcimonioso}
  \definition{v.}{economizar; conservar; usar com moderação; reduzir; eliminar ou minimizar o esgotamento de itens potencialmente esgotáveis}
\end{EntryWithPhonetic}

\begin{EntryWithPhonetic}{节水}{jie2shui3}{5,4}{⾋,⽔}[HSK 7-9]
  \definition{v.}{economizar água}
\end{EntryWithPhonetic}

\begin{EntryWithPhonetic}{节衣缩食}{jie2yi1-suo1shi2}{5,6,14,9}{⾋,⾐,⽷,⾷}[HSK 7-9]
  \definition{expr.}{``Reduza os gastos com comida e roupas.''; economizar em comida e roupas; ser mais econômico; viver frugalmente; praticar austeridade; praticar uma economia rigorosa}
\end{EntryWithPhonetic}

\begin{EntryWithPhonetic}{节约}{jie2yue1}{5,6}{⾋,⽷}[HSK 3]
  \definition{adj.}{econômico; sem luxo}
  \definition{v.}{guardar; economizar; usar com moderação; economizar gastos desnecessários}
\end{EntryWithPhonetic}

\begin{EntryWithPhonetic}{节奏}{jie2zou4}{5,9}{⾋,⼤}[HSK 6]
  \definition[个,种]{s.}{ritmo; o fenômeno da alternância regular de comprimento, força e fraqueza das notas na música | padrão regular; uma metáfora para um processo de ajuste adequado com tensão e relaxamento}
\end{EntryWithPhonetic}

%%%%%%%%%% 劫 %%%%%%%%%%
\subsection*{劫}\addcontentsline{loh}{figure}{劫 \dpy{jie2}}

\begin{EntryWithPhonetic}{劫}{jie2}{7}{⼒}[HSK 7-9]
  \definition{s.}{calamidade; desastre; infortúnio}
  \definition{v.}{roubar; saquear; invadir | coagir; compelir; intimidar}
\end{EntryWithPhonetic}

\begin{EntryWithPhonetic}{劫持}{jie2chi2}{7,9}{⼒,⼿}[HSK 7-9]
  \definition{v.}{sequestrar; manter sob coação; raptar; ameaçar; manter como refém}
\end{EntryWithPhonetic}

%%%%%%%%%% 杰 %%%%%%%%%%
\subsection*{杰}\addcontentsline{loh}{figure}{杰 \dpy{jie2}}

\begin{EntryWithPhonetic}{杰}{jie2}{8}{⽊}
  \definition{adj.}{notável; proeminente; fora do comum}
  \definition[位,名,个,些]{s.}{pessoa excepcional; herói; uma pessoa com talentos excepcionais}
\end{EntryWithPhonetic}

\begin{EntryWithPhonetic}{杰出}{jie2chu1}{8,5}{⽊,⼐}[HSK 6]
  \definition{adj.}{notável; proeminente; (talento, realização) excepcional}
\end{EntryWithPhonetic}

%%%%%%%%%% 拮 %%%%%%%%%%
\subsection*{拮}\addcontentsline{loh}{figure}{拮 \dpy{jie2}}

\begin{EntryWithPhonetic}{拮}{jie2}{9}{⼿}
  \definition{adj.}{trabalhoso | sem dinheiro | antagônico | trabalhando duro | pressionado}
\end{EntryWithPhonetic}

\begin{EntryWithPhonetic}{拮据}{jie2ju1}{9,11}{⼿,⼿}
  \definition{adj.}{em circunstâncias difíceis; sem dinheiro; em dificuldades}
\end{EntryWithPhonetic}

%%%%%%%%%% 洁 %%%%%%%%%%
\subsection*{洁}\addcontentsline{loh}{figure}{洁 \dpy{jie2}}

\begin{EntryWithPhonetic}{洁}{jie2}{9}{⽔}
  \definition{adj.}{limpo; arrumado | honesto; íntegro}
  \definition{v.}{limpar; purificar; tornar limpo | tornar inocente}
\end{EntryWithPhonetic}

\begin{EntryWithPhonetic}{洁净}{jie2jing4}{9,8}{⽔,⼎}[HSK 7-9]
  \definition{adj.}{limpo; impecável}
\end{EntryWithPhonetic}

%%%%%%%%%% 结 %%%%%%%%%%
\subsection*{结}\addcontentsline{loh}{figure}{结 \dpy{jie2}}

\begin{EntryWithPhonetic}{结}{jie2}{9}{⽷}[HSK 4]
  \definition*{s.}{Sobrenome: Jie}
  \definition{s.}{nó | declaração juramentada; garantia por escrito; documento que, antigamente, significava um reconhecimento de encerramento ou uma garantia de responsabilidade}
  \definition{v.}{amarrar; tricotar; dar nó; tecer | formar; forjar; cimentar; solidificar | resolver; concluir | combinar; formar um relacionamento}
  \seeref{jie1}
\end{EntryWithPhonetic}

\begin{EntryWithPhonetic}{结冰}{jie2bing1}{9,6}{⽷,⼎}[HSK 7-9]
  \definition{v.}{congelar; cobrir com gelo}
\end{EntryWithPhonetic}

\begin{EntryWithPhonetic}{结构}{jie2gou4}{9,8}{⽷,⽊}[HSK 4]
  \definition[个]{s.}{estrutura; composição; construção; formação; constituição; tecido; forma; sistematização; mecânica; organização | arquitetura; estrutura; construção; construção de partes de edifícios com suporte de carga ou com carga externa | Geologia: textura}[这些矿物质具有致密结构。===Esses minerais têm uma estrutura densa.]
\end{EntryWithPhonetic}

\begin{EntryWithPhonetic}{结果}{jie2/guo3}{9,8}{⽷,⽊}[HSK 2]
  \definition{conj.}{como resultado; no final}
  \definition{v.}{despachar | matar}
  \definition{v.+compl.}{resultado; conclusão; consequência}
  \seeref{jie1/guo3}
\end{EntryWithPhonetic}

\begin{EntryWithPhonetic}{结合}{jie2he2}{9,6}{⽷,⼝}[HSK 3]
  \definition{v.}{ligar; unir; combinar; integrar; formar uma relação estreita entre pessoas ou coisas | casar-se; unir-se em matrimônio; referir-se especificamente a casais}
\end{EntryWithPhonetic}

\begin{EntryWithPhonetic}{结婚}{jie2/hun1}{9,11}{⽷,⼥}[HSK 3]
  \definition{v.+compl.}{casar; casar"-se; casar"-se bem}
\end{EntryWithPhonetic}

\begin{EntryWithPhonetic}{结婚礼服}{jie2hun1 li3 fu2}{9,11,5,8}{⽷,⼥,⽰,⽉}
  \definition{s.}{vestido de casamento; vestido de noiva}
\end{EntryWithPhonetic}

\begin{EntryWithPhonetic}{结晶}{jie2jing1}{9,12}{⽷,⽇}[HSK 7-9]
  \definition{s.}{cristal; substâncias cristalinas | Figurativo: os frutos (do trabalho ou da atividade); metáfora para conquistas preciosas}
  \definition{v.}{cristalizar; as substâncias podem formar cristais a partir de um estado líquido (solução ou fundido) ou gasoso}
\end{EntryWithPhonetic}

\begin{EntryWithPhonetic}{结局}{jie2ju2}{9,7}{⽷,⼫}[HSK 7-9]
  \definition[个]{s.}{final; desfecho; resultado final; resultado final; conclusão}
\end{EntryWithPhonetic}

\begin{EntryWithPhonetic}{结论}{jie2lun4}{9,6}{⽷,⾔}[HSK 4]
  \definition[个]{s.}{conclusão; palavra final sobre uma pessoa ou coisa após investigação e pesquisa | veredito; julgamento deduzido de premissas também é chamado de conclusão}
\end{EntryWithPhonetic}

\begin{EntryWithPhonetic}{结社自由}{jie2she4zi4you2}{9,7,6,5}{⽷,⽰,⾃,⽥}
  \definition{s.}{(constitucional) liberdade de associação}
\end{EntryWithPhonetic}

\begin{EntryWithPhonetic}{结识}{jie2shi2}{9,7}{⽷,⾔}[HSK 7-9]
  \definition{v.}{conhecer alguém; fazer amizade com; familiarizar-se com alguém}
\end{EntryWithPhonetic}

\begin{EntryWithPhonetic}{结束}{jie2shu4}{9,7}{⽷,⽊}[HSK 3]
  \definition{v.}{finalizar; fechar; terminar; concluir; encerrar; desenvolver ou avançar até a fase final, sem continuidade}
\end{EntryWithPhonetic}

\begin{EntryWithPhonetic}{结束辩论}{jie2shu4 bian4 lun4}{9,7,16,6}{⽷,⽊,⾟,⾔}
  \definition{s.}{debate de encerramento}
\end{EntryWithPhonetic}

\begin{EntryWithPhonetic}{结束工作}{jie2shu4gong1zuo4}{9,7,3,7}{⽷,⽊,⼯,⼈}
  \definition{s.}{trabalho final}
  \definition{v.}{terminar o trabalho}
\end{EntryWithPhonetic}

\begin{EntryWithPhonetic}{结束剂}{jie2shu4 ji4}{9,7,8}{⽷,⽊,⼑}
  \definition{s.}{finalizador}
\end{EntryWithPhonetic}

\begin{EntryWithPhonetic}{结束区}{jie2shu4 qu1}{9,7,4}{⽷,⽊,⼖}
  \definition{s.}{zona final}
\end{EntryWithPhonetic}

\begin{EntryWithPhonetic}{结束文本}{jie2shu4 wen2ben3}{9,7,4,5}{⽷,⽊,⽂,⽊}
  \definition{s.}{texto final}
\end{EntryWithPhonetic}

\begin{EntryWithPhonetic}{结束语}{jie2shu4yu3}{9,7,9}{⽷,⽊,⾔}
  \definition{s.}{conclusões finais | considerações finais}
\end{EntryWithPhonetic}

\begin{EntryWithPhonetic}{结尾}{jie2wei3}{9,7}{⽷,⼫}[HSK 7-9]
  \definition{s.}{final; fase de encerramento; fim de fase ou parte}
  \definition{v.}{encerrar; concluir a etapa final}
\end{EntryWithPhonetic}

%%%%%%%%%% 捷 %%%%%%%%%%
\subsection*{捷}\addcontentsline{loh}{figure}{捷 \dpy{jie2}}

\begin{EntryWithPhonetic}{捷}{jie2}{11}{⼿}
  \definition*{s.}{Sobrenome: Jie}
  \definition{adj.}{rápido; ágil}
  \definition{s.}{vitória; triunfo; sucesso}
\end{EntryWithPhonetic}

\begin{EntryWithPhonetic}{捷径}{jie2jing4}{11,8}{⼿,⼻}
  \definition{s.}{atalho}
\end{EntryWithPhonetic}

%%%%%%%%%% 截 %%%%%%%%%%
\subsection*{截}\addcontentsline{loh}{figure}{截 \dpy{jie2}}

\begin{EntryWithPhonetic}{截}{jie2}{14}{⼽}[HSK 7-9]
  \definition{clas.}{seção; pedaço; comprimento}
  \definition{prep.}{por (um tempo especificado); até}
  \definition{v.}{cortar; romper | parar; verificar; interromper; interceptar}
\end{EntryWithPhonetic}

\begin{EntryWithPhonetic}{截然不同}{jie2ran2-bu4tong2}{14,12,4,6}{⼽,⽕,⼀,⼝}[HSK 7-9]
  \definition{expr.}{``Completamente diferente.''; tão diferente quanto preto e branco; polos opostos}
\end{EntryWithPhonetic}

\begin{EntryWithPhonetic}{截止}{jie2zhi3}{14,4}{⼽,⽌}[HSK 6]
  \definition{adv.}{até (um certo limite de tempo); por (um tempo especificado)}
\end{EntryWithPhonetic}

\begin{EntryWithPhonetic}{截至}{jie2zhi4}{14,6}{⼽,⾄}[HSK 6]
  \definition{adv.}{a partir de; até (um certo limite de tempo); por (um tempo especificado)}
\end{EntryWithPhonetic}

%%%%%%%%%% 竭 %%%%%%%%%%
\subsection*{竭}\addcontentsline{loh}{figure}{竭 \dpy{jie2}}

\begin{EntryWithPhonetic}{竭}{jie2}{14}{⽴}
  \definition*{s.}{Sobrenome: Jie}
  \definition{v.}{esgotar; consumir | Literário: secar; drenar}
\end{EntryWithPhonetic}

\begin{EntryWithPhonetic}{竭尽全力}{jie2jin4-quan2li4}{14,6,6,2}{⽴,⼫,⼊,⼒}[HSK 7-9]
  \definition{expr.}{``Dê o seu melhor.''; não poupar esforços; fazer o máximo possível; com todas as forças; usar todas as suas forças para descrever o ato de fazer o máximo esforço; fazer o máximo possível; fazer tudo o que estiver ao seu alcance}
\end{EntryWithPhonetic}

\begin{EntryWithPhonetic}{竭力}{jie2li4}{14,2}{⽴,⼒}[HSK 7-9]
  \definition{v.}{fazer o máximo; fazer o máximo; não poupar esforços; tentar por todos os meios possíveis; dar o melhor de si; usar todos os esforços do corpo e da mente para\dots; usar cada grama de sua energia}
\end{EntryWithPhonetic}

%%%%%%%%%% 姐 %%%%%%%%%%
\subsection*{姐}\addcontentsline{loh}{figure}{姐 \dpy{jie3}}

\begin{EntryWithPhonetic}{姐}{jie3}{8}{⼥}
  \definition[个,位]{s.}{irmã mais velha; irmã | termo genérico para mulheres jovens | mulheres da mesma geração que são mais velhas do que você (geralmente não inclui aquelas que podem ser chamadas de cunhadas) | um título respeitoso para mulheres jovens profissionais em determinados cargos}
  \seealsoref{姐姐}{jie3jie5}
\end{EntryWithPhonetic}

\begin{EntryWithPhonetic}{姐夫}{jie3fu5}{8,4}{⼥,⼤}
  \definition{s.}{marido da irmã mais velha}
\end{EntryWithPhonetic}

\begin{EntryWithPhonetic}{姐姐}{jie3jie5}{8,8}{⼥,⼥}[HSK 1]
  \definition[个]{s.}{irmã mais velha}
\end{EntryWithPhonetic}

\begin{EntryWithPhonetic}{姐妹}{jie3mei4}{8,8}{⼥,⼥}[HSK 4]
  \definition[个]{s.}{irmãs}
\end{EntryWithPhonetic}

%%%%%%%%%% 解 %%%%%%%%%%
\subsection*{解}\addcontentsline{loh}{figure}{解 \dpy{jie3}}

\begin{EntryWithPhonetic}{解}{jie3}{13}{⾓}[HSK 6]
  \definition{s.}{solução; o valor de uma variável desconhecida em uma equação algébrica}
  \definition{v.}{dividir; separar | desfazer; desatar; abrir algo que esteja amarrado ou encadernado | acalmar; dissipar; dispensar; eliminar | resolver; explicar; interpretar | entender; compreender | aliviar-se (excreção de urina e fezes) | dissolver; desintegrar | (cálculo analítico) resolver; solucionar}
\end{EntryWithPhonetic}

\begin{EntryWithPhonetic}{解除}{jie3chu2}{13,9}{⾓,⾩}[HSK 5]
  \definition{v.}{remover; aliviar; livrar"-se de; eliminar}
\end{EntryWithPhonetic}

\begin{EntryWithPhonetic}{解答}{jie3da2}{13,12}{⾓,⽵}[HSK 7-9]
  \definition{v.}{responder; explicar}
\end{EntryWithPhonetic}

\begin{EntryWithPhonetic}{解读}{jie3du2}{13,10}{⾓,⾔}[HSK 7-9]
  \definition{v.}{decodificar; interpretar; explicar; compreender por meio da análise}
\end{EntryWithPhonetic}

\begin{EntryWithPhonetic}{解放}{jie3fang4}{13,8}{⾓,⽅}[HSK 5]
  \definition*{s.}{Libertação (que significou o fim do domínio do regime reacionário Kuomintang em 1949 e ao estabelecimento da República Popular da China)}
  \definition{v.}{libertar; emancipar; eliminar as restrições para permitir o desenvolvimento da liberdade}
\end{EntryWithPhonetic}

\begin{EntryWithPhonetic}{解雇}{jie3gu4}{13,12}{⾓,⾫}[HSK 7-9]
  \definition{v.}{demitir; dispensar; exonerar}
\end{EntryWithPhonetic}

\begin{EntryWithPhonetic}{解救}{jie3jiu4}{13,11}{⾓,⽁}[HSK 7-9]
  \definition{v.}{salvar; resgatar; sair do perigo ou da dificuldade}
\end{EntryWithPhonetic}

\begin{EntryWithPhonetic}{解决}{jie3jue2}{13,6}{⾓,⼎}[HSK 3]
  \definition{v.}{solucionar; resolver; liquidar; resolver problemas com resultados | acabar com; descartar; eliminar (o inimigo)}
\end{EntryWithPhonetic}

\begin{EntryWithPhonetic}{解开}{jie3 kai1}{13,4}{⾓,⼶}[HSK 3]
  \definition{v.}{desatar; desamarrar; desabotoar; desamarrar ou desfazer nós}
\end{EntryWithPhonetic}

\begin{EntryWithPhonetic}{解剖}{jie3pou1}{13,10}{⾓,⼑}[HSK 7-9]
  \definition{v.}{dissecar | analisar; metaforicamente, refere"-se à observação e análise aprofundada das coisas}
\end{EntryWithPhonetic}

\begin{EntryWithPhonetic}{解散}{jie3san4}{13,12}{⾓,⽁}[HSK 7-9]
  \definition{v.}{dispensar | dissolver; desmantelar; cancelar}
\end{EntryWithPhonetic}

\begin{EntryWithPhonetic}{解释}{jie3shi4}{13,12}{⾓,⾤}[HSK 4]
  \definition{v.}{explicar; expor; interpretar | analisar; explicaro significado, razões, justificativas, etc.}
\end{EntryWithPhonetic}

\begin{EntryWithPhonetic}{解说}{jie3shuo1}{13,9}{⾓,⾔}[HSK 6]
  \definition{v.}{narrar; comentar; fazer um comentário; explicar oralmente}
\end{EntryWithPhonetic}

\begin{EntryWithPhonetic}{解体}{jie3ti3}{13,7}{⾓,⼈}[HSK 7-9]
  \definition{v.}{(corpo orgânico) decompor-se | (sistema social, organização, etc.) desintegrar-se | desmontar; desintegrar; quebrar; desmantelar}
\end{EntryWithPhonetic}

\begin{EntryWithPhonetic}{解脱}{jie3tuo1}{13,11}{⾓,⾁}[HSK 7-9]
  \definition{v.}{libertar"-se das preocupações mundanas; no budismo,  se refere à libertação do sofrimento e à conquista da liberdade | libertar"-se (ou desvencilhar"-se); livrar"-se de | absolver; exonerar}
\end{EntryWithPhonetic}

\begin{EntryWithPhonetic}{解围}{jie3/wei2}{13,7}{⾓,⼞}[HSK 7-9]
  \definition{v.+compl.}{forçar um inimigo a levantar um cerco; resgatar de um cerco; vir em socorro dos sitiados | ajudar a sair de uma situação difícil; evitar constrangimento; livrar alguém de um constrangimento; livrar alguém de uma enrascada; amenizar o constrangimento de alguém}
\end{EntryWithPhonetic}

\begin{EntryWithPhonetic}{解析}{jie3xi1}{13,8}{⾓,⽊}[HSK 7-9]
  \definition{v.}{analisar; examinar minuciosamente; resolver}
\end{EntryWithPhonetic}

\begin{EntryWithPhonetic}{解压}{jie3ya1}{13,6}{⾓,⼚}
  \definition{v.}{aliviar o estresse | Computação: descomprimir}
\end{EntryWithPhonetic}

%%%%%%%%%% 介 %%%%%%%%%%
\subsection*{介}\addcontentsline{loh}{figure}{介 \dpy{jie4}}

\begin{EntryWithPhonetic}{介}{jie4}{4}{⼈}
  \definition*{s.}{Sobrenome: Jie}
  \definition{adj.}{direto; honesto e franco; correto}
  \definition{s.}{armadura | concha (crustáceos e criaturas aquáticas) | preposição}
  \definition{v.}{estar situado entre; interpor | levar a sério; levar em conta; ter em mente}
\end{EntryWithPhonetic}

\begin{EntryWithPhonetic}{介入}{jie4ru4}{4,2}{⼈,⼊}[HSK 7-9]
  \definition{v.}{intervir; interpor-se; envolver-se}
  \synonymref{参与}{can1yu4}
  \antonymref{旁观}{pang2guan1}
\end{EntryWithPhonetic}

\begin{EntryWithPhonetic}{介绍}{jie4shao4}{4,8}{⼈,⽷}[HSK 1]
  \definition{s.}{introdução; apresentação}
  \definition{v.}{introduzir; apresentar | recomendar; sugerir | dar a conhecer; informar}
  \seealsoref{说明}{shuo1ming2}
\end{EntryWithPhonetic}

\begin{EntryWithPhonetic}{介意}{jie4/yi4}{4,13}{⼈,⼼}[HSK 7-9]
  \definition{v.+compl.}{(geralmente na forma negativa) incomodar; ofender-se; guardar coisas desagradáveis no coração; preocupar-se com elas}
  \synonymref{当心}{dang1xin1}
  \synonymref{留心}{liu2/xin1}
  \synonymref{留意}{liu2/yi4}
  \synonymref{小心}{xiao3xin5}
  \synonymref{在意}{zai4/yi4}
  \synonymref{在乎}{zai4hu5}
\end{EntryWithPhonetic}

\begin{EntryWithPhonetic}{介于}{jie4yu2}{4,3}{⼈,⼆}[HSK 7-9]
  \definition{v.}{estar no meio (entre os dois)}
\end{EntryWithPhonetic}

%%%%%%%%%% 戒 %%%%%%%%%%
\subsection*{戒}\addcontentsline{loh}{figure}{戒 \dpy{jie4}}

\begin{EntryWithPhonetic}{戒}{jie4}{7}{⼽}[HSK 5]
  \definition[个,枚]{s.}{advertência; exortação | disciplina monástica budista; preceitos budistas | anel (dedo)}
  \definition{v.}{proteger-se contra; estar preparado; estar atento | advertir; exortar; admoestar | abandonar; parar; desistir; desistir (de um hábito ruim)}
\end{EntryWithPhonetic}

\begin{EntryWithPhonetic}{戒备}{jie4bei4}{7,8}{⼽,⼡}[HSK 7-9]
  \definition{v.}{tomar precauções; estar em alerta; são muito cuidadosos em suas palavras e ações para evitar que algo de ruim aconteça temendo que outros possam lhes fazer mal | guardar; proteger-se de; os militares ou a polícia protegem um local para evitar que coisas ruins aconteçam}
\end{EntryWithPhonetic}

\begin{EntryWithPhonetic}{戒烟}{jie4 yan1}{7,10}{⼽,⽕}[HSK 7-9]
  \definition{v.}{deixar de fumar; parar de fumar; refere"-se a abandonar o hábito de fumar ópio.; refere"-se também a abandonar o hábito de fumar cigarros}
\end{EntryWithPhonetic}

\begin{EntryWithPhonetic}{戒指}{jie4zhi5}{7,9}{⼽,⼿}[HSK 7-9]
  \definition[个]{s.}{anel}
  \seealsoref{戒指儿}{jie4zhi5r5}
\end{EntryWithPhonetic}

\begin{EntryWithPhonetic}{戒指儿}{jie4zhi5r5}{7,9,2}{⼽,⼿,⼉}
  \definition{s.}{anel}
  \seealsoref{戒指}{jie4zhi5}
\end{EntryWithPhonetic}

%%%%%%%%%% 芥 %%%%%%%%%%
\subsection*{芥}\addcontentsline{loh}{figure}{芥 \dpy{jie4}}

\begin{EntryWithPhonetic}{芥}{jie4}{7}{⾋}
  \definition{s.}{mostarda}
  \seeref{gai4}
\end{EntryWithPhonetic}

\begin{EntryWithPhonetic}{芥兰}{jie4lan2}{7,5}{⾋,⼋}
  \definition{s.}{couve}
\end{EntryWithPhonetic}

%%%%%%%%%% 届 %%%%%%%%%%
\subsection*{届}\addcontentsline{loh}{figure}{届 \dpy{jie4}}

\begin{EntryWithPhonetic}{届}{jie4}{8}{⼫}[HSK 5]
  \definition{clas.}{sessões (de uma conferência); anos (de graduação); quantificador, ligeiramente equivalente a 次, usado para reuniões regulares ou turmas de formandos, etc.}
  \definition{v.}{vencer o prazo}
  \seealsoref{次}{ci4}
\end{EntryWithPhonetic}

\begin{EntryWithPhonetic}{届时}{jie4shi2}{8,7}{⼫,⽇}[HSK 7-9]
  \definition{adv.}{na ocasião; quando chegar a hora; no momento determinado; no horário combinado}
\end{EntryWithPhonetic}

%%%%%%%%%% 界 %%%%%%%%%%
\subsection*{界}\addcontentsline{loh}{figure}{界 \dpy{jie4}}

\begin{EntryWithPhonetic}{界}{jie4}{9}{⽥}[HSK 6]
  \definition{s.}{fronteira; limite | escopo; extensão | círculos | divisão primária; reino | era geológica | (matemática) limite | mundo; faixa dividida por ocupação, emprego ou gênero, etc. | grupo}
\end{EntryWithPhonetic}

\begin{EntryWithPhonetic}{界碑}{jie4bei1}{9,13}{⽥,⽯}
  \definition{s.}{marco de fronteira}
\end{EntryWithPhonetic}

\begin{EntryWithPhonetic}{界定}{jie4ding4}{9,8}{⽥,⼧}[HSK 7-9]
  \definition{v.}{definir; delimitar; especificar os limites; definir escopo | definir; dar uma definição}
\end{EntryWithPhonetic}

\begin{EntryWithPhonetic}{界线}{jie4xian4}{9,8}{⽥,⽷}[HSK 7-9]
  \definition{s.}{fronteira; limite; a fronteira entre coisas diferentes; a linha que divide duas regiões}
\end{EntryWithPhonetic}

\begin{EntryWithPhonetic}{界限}{jie4xian4}{9,8}{⽥,⾩}[HSK 7-9]
  \definition{s.}{linha; limites; fronteiras; demarcação (ou divisão); a fronteira entre coisas diferentes | limite; fim}
\end{EntryWithPhonetic}

%%%%%%%%%% 借 %%%%%%%%%%
\subsection*{借}\addcontentsline{loh}{figure}{借 \dpy{jie4}}

\begin{EntryWithPhonetic}{借}{jie4}{10}{⼈}[HSK 2]
  \definition{adv.}{por meio de}
  \definition{v.}{emprestar | pedir emprestado | usar como pretexto | aproveitar; tirar proveito (de uma oportunidade, etc.)}
\end{EntryWithPhonetic}

\begin{EntryWithPhonetic}{借鉴}{jie4jian4}{10,13}{⼈,⾦}[HSK 6]
  \definition{s.}{tirar lições de; aproveitar a experiência de; ganhar experiência e lições com o passado ou com as experiências de outras pessoas}
\end{EntryWithPhonetic}

\begin{EntryWithPhonetic}{借口}{jie4kou3}{10,3}{⼈,⼝}[HSK 7-9]
  \definition[个,种]{s.}{desculpa; pretexto; razões falsas apresentadas para atingir um objetivo}
  \definition{v.}{usar como desculpa; usar sob o pretexto de; usar com a justificativa de; usar (algo) como motivo (que não seja um motivo real)}
\end{EntryWithPhonetic}

\begin{EntryWithPhonetic}{借书证}{jie4shu1zheng4}{10,4,7}{⼈,⼄,⾔}
  \definition{s.}{cartão da biblioteca; comprovante de solicitação}
  \seealsoref{借书证卡}{jie4shu1zheng4 ka3}
\end{EntryWithPhonetic}

\begin{EntryWithPhonetic}{借书证卡}{jie4shu1zheng4 ka3}{10,4,7,5}{⼈,⼄,⾔,⼘}
  \definition{s.}{cartão da biblioteca}
  \seealsoref{借书证}{jie4shu1zheng4}
\end{EntryWithPhonetic}

\begin{EntryWithPhonetic}{借条}{jie4tiao2}{10,7}{⼈,⽊}[HSK 7-9]
  \definition{s.}{recibo de empréstimo; nota promissória}
  \seealsoref{借条儿}{jie4tiao2r5}
\end{EntryWithPhonetic}

\begin{EntryWithPhonetic}{借条儿}{jie4tiao2r5}{10,7,2}{⼈,⽊,⼉}
  \definition{s.}{nota promissória}
\end{EntryWithPhonetic}

\begin{EntryWithPhonetic}{借用}{jie4yong4}{10,5}{⼈,⽤}[HSK 7-9]
  \definition{v.}{tomar emprestado; ter o empréstimo de | usar algo para outro propósito}
\end{EntryWithPhonetic}

\begin{EntryWithPhonetic}{借助}{jie4zhu4}{10,7}{⼈,⼒}[HSK 7-9]
  \definition{v.}{contar com a ajuda de; obter apoio de}
\end{EntryWithPhonetic}

%%%%%%%%%% 今 %%%%%%%%%%
\subsection*{今}\addcontentsline{loh}{figure}{今 \dpy{jin1}}

\begin{EntryWithPhonetic}{今}{jin1}{4}{⼈}
  \definition*{s.}{Sobrenome: Jin}
  \definition{s.}{agora; o presente | moderno | de hoje; deste ano | isso; isto}
  \antonymref{古}{gu3}
  \antonymref{昔}{xi1}
\end{EntryWithPhonetic}

\begin{EntryWithPhonetic}{今后}{jin1hou4}{4,6}{⼈,⼝}[HSK 2]
  \definition{s.}{a partir de agora; doravante; no futuro; desde o momento em que falamos}
  \synonymref{此后}{ci3hou4}
  \synonymref{从此}{cong2ci3}
  \synonymref{往后}{wang3hou4}
  \synonymref{以后}{yi3hou4}
  \synonymref{以来}{yi3lai2}
  \antonymref{从来}{cong2lai2}
  \antonymref{当前}{dang1qian2}
  \antonymref{以前}{yi3qian2}
\end{EntryWithPhonetic}

\begin{EntryWithPhonetic}{今年}{jin1nian2}{4,6}{⼈,⼲}[HSK 1]
  \definition{adv.}{este ano}
  \synonymref{来年}{lai2nian2}
  \antonymref{去年}{qu4nian2}
  \antonymref{往年}{wang3nian2}
\end{EntryWithPhonetic}

\begin{EntryWithPhonetic}{今人}{jin1ren2}{4,2}{⼈,⼈}
  \definition{s.}{contemporâneos; pessoas da nossa era; pessoas de hoje | pessoas modernas; pessoas contemporâneas}
  \antonymref{古人}{gu3ren2}
\end{EntryWithPhonetic}

\begin{EntryWithPhonetic}{今日}{jin1ri4}{4,4}{⼈,⽇}[HSK 5]
  \definition{s.}{hoje}
  \antonymref{当初}{dang1chu1}
  \antonymref{往日}{wang3ri4}
  \antonymref{昔日}{xi1ri4}
\end{EntryWithPhonetic}

\begin{EntryWithPhonetic}{今天}{jin1tian1}{4,4}{⼈,⼤}[HSK 1]
  \definition{adv.}{hoje; neste dia | agora; o momento ou a época atual}
  \synonymref{今日}{jin1ri4}
  \antonymref{昨天}{zuo2tian1}
\end{EntryWithPhonetic}

%%%%%%%%%% 斤 %%%%%%%%%%
\subsection*{斤}\addcontentsline{loh}{figure}{斤 \dpy{jin1}}

\begin{EntryWithPhonetic}{斤}{jin1}{4}{⽄}[HSK 2][Kangxi 69]
  \definition{clas.}{uma unidade de peso (=500 gramas)}
  \definition{s.}{machado; cutelo; ferramentas antigas para cortar árvores}
\end{EntryWithPhonetic}

%%%%%%%%%% 金 %%%%%%%%%%
\subsection*{金}\addcontentsline{loh}{figure}{金 \dpy{jin1}}

\begin{EntryWithPhonetic}{金}{jin1}{8}{⾦}[HSK 3][Kangxi 167]
  \definition*{s.}{Dinastia Jin (1115-1234) | Sobrenome: Jin}
  \definition{adj.}{dourado | altamente respeitado; precioso. metáfora de nobreza}
  \definition[锭,块]{s.}{ouro | metal | dinheiro | instrumento antigo de percussão de metal}
\end{EntryWithPhonetic}

\begin{EntryWithPhonetic}{金额}{jin1'e2}{8,15}{⾦,⾴}[HSK 6]
  \definition[份,笔]{s.}{quantidade de dinheiro; soma de dinheiro}
\end{EntryWithPhonetic}

\begin{EntryWithPhonetic}{金刚石}{jin1gang1shi2}{8,6,5}{⾦,⼑,⽯}
  \definition{s.}{diamante, também chamado de 钻石}[金刚石比什么金属都硬。===O diamante é mais duro que qualquer metal.]
  \seealsoref{钻石}{zuan4shi2}
\end{EntryWithPhonetic}

\begin{EntryWithPhonetic}{金牌}{jin1pai2}{8,12}{⾦,⽚}[HSK 3]
  \definition[枚]{s.}{medalha de ouro; refere"-se à medalha conquistada pelo campeão em uma competição esportiva | ficha de ouro; placa de ouro usada como símbolo}
\end{EntryWithPhonetic}

\begin{EntryWithPhonetic}{金钱}{jin1qian2}{8,10}{⾦,⾦}[HSK 6]
  \definition[沓,笔,堆]{s.}{dinheiro; moeda}
\end{EntryWithPhonetic}

\begin{EntryWithPhonetic}{金融}{jin1rong2}{8,16}{⾦,⿀}[HSK 6]
  \definition{s.}{finanças; serviços bancários; refere"-se a atividades econômicas como a emissão, circulação e retirada de moeda, a concessão e retirada de empréstimos, o depósito e retirada de depósitos e transações de câmbio}
\end{EntryWithPhonetic}

\begin{EntryWithPhonetic}{金色}{jin1 se4}{8,6}{⾦,⾊}
  \definition{s.}{cor ouro; dourado}
\end{EntryWithPhonetic}

\begin{EntryWithPhonetic}{金属}{jin1shu3}{8,12}{⾦,⼫}[HSK 7-9]
  \definition[种,块,片]{s.}{metal; um tipo de substância com superfície relativamente lisa e brilhante, porém opaca, capaz de conduzir eletricidade e calor}
\end{EntryWithPhonetic}

\begin{EntryWithPhonetic}{金字塔}{jin1zi4ta3}{8,6,12}{⾦,⼦,⼟}[HSK 7-9]
  \definition[座]{s.}{pirâmide (edifício ou estrutura); as pirâmides egípcias, um tipo de estrutura utilizada por alguns povos antigos, são pirâmides de pedra com três ou mais lados, que, à distância, lembram o caractere chinês 金 (ouro); elas serviam de túmulo para antigos imperadores}
\end{EntryWithPhonetic}

\begin{EntryWithPhonetic}{金子}{jin1zi5}{8,3}{⾦,⼦}[HSK 7-9]
  \definition{s.}{ouro; elemento metálico, símbolo Au (aurum) amarelo-avermelhado, macio, dúctil, quimicamente estável é um metal precioso, usado para fabricar dinheiro, ornamentos etc.}
\end{EntryWithPhonetic}

%%%%%%%%%% 津 %%%%%%%%%%
\subsection*{津}\addcontentsline{loh}{figure}{津 \dpy{jin1}}

\begin{EntryWithPhonetic}{津}{jin1}{9}{⽔}
  \definition*{s.}{abreviação de Tianjin, 天津}
  \definition{adj.}{úmido; molhado; hidratado}
  \definition{s.}{suor | travessia de balsa; vau; balsa | metáfora para cargos importantes | saliva}
  \seealsoref{天津}{tian1jin1}
\end{EntryWithPhonetic}

\begin{EntryWithPhonetic}{津津有味}{jin1jin1-you3wei4}{9,9,6,8}{⽔,⽔,⽉,⼝}[HSK 7-9]
  \definition{expr.}{com grande prazer; o sabor é delicioso; com entusiasmo; com grande satisfação}
\end{EntryWithPhonetic}

\begin{EntryWithPhonetic}{津贴}{jin1tie1}{9,9}{⽔,⾙}[HSK 7-9]
  \definition{s.}{subsídio; auxílio financeiro; abono; pensão; além dos salários, os subsídios também se referem aos auxílios de custo de vida para funcionários do sistema de fornecimento}
  \definition{v.}{subsidiar; conceder auxílio financeiro; subsidiar pessoas com dinheiro ou bens}
\end{EntryWithPhonetic}

%%%%%%%%%% 矜 %%%%%%%%%%
\subsection*{矜}\addcontentsline{loh}{figure}{矜 \dpy{jin1}}

\begin{EntryWithPhonetic}{矜}{jin1}{9}{⽭}
  \definition{adj.}{presunçoso; vaidoso | contido; reservado; determinado}
  \definition{v.}{ter pena; simpatizar com; compadecer-se}
\end{EntryWithPhonetic}

%%%%%%%%%% 筋 %%%%%%%%%%
\subsection*{筋}\addcontentsline{loh}{figure}{筋 \dpy{jin1}}

\begin{EntryWithPhonetic}{筋}{jin1}{12}{⽵}[HSK 7-9]
  \definition*{s.}{Sobrenome: Jin}
  \definition[根,条]{s.}{músculo; Coloquial: tendão; ligamento; Coloquial: veias salientes sob a pele; qualquer coisa que se assemelhe a um tendão ou veia}
  \seealsoref{筋儿}{jin1r5}
\end{EntryWithPhonetic}

\begin{EntryWithPhonetic}{筋儿}{jin1r5}{12,2}{⽵,⼉}
  \definition{s.}{Coloquial: tendão; ligamento | qualquer coisa que se assemelhe a um tendão ou veia}
  \seealsoref{筋}{jin1}
\end{EntryWithPhonetic}

%%%%%%%%%% 禁 %%%%%%%%%%
\subsection*{禁}\addcontentsline{loh}{figure}{禁 \dpy{jin1}}

\begin{EntryWithPhonetic}{禁不住}{jin1bu5zhu4}{13,4,7}{⽰,⼀,⼈}[HSK 7-9]
  \definition{v.}{ser incapaz de resistir; ser incapaz de suportar ou aguentar; aplicado tanto a pessoas quanto a coisas | não consigo evitar (fazer algo); não consigo me conter; incapaz de suprimir, incontrolável; aplica-se apenas a humanos}
\end{EntryWithPhonetic}

%%%%%%%%%% 仅 %%%%%%%%%%
\subsection*{仅}\addcontentsline{loh}{figure}{仅 \dpy{jin3}}

\begin{EntryWithPhonetic}{仅}{jin3}{4}{⼈}[HSK 3]
  \definition{adv.}{somente; meramente; por muito pouco}
\end{EntryWithPhonetic}

\begin{EntryWithPhonetic}{仅此而已}{jin3ci3'er2yi3}{4,6,6,3}{⼈,⽌,⽽,⼰}
  \definition{adv.}{apenas isso e nada mais | isso é tudo}
  \antonymref{不计其数}{bu2 ji4 qi2 shu4}
\end{EntryWithPhonetic}

\begin{EntryWithPhonetic}{仅次于}{jin3 ci4 yu2}{4,6,3}{⼈,⽋,⼆}[HSK 7-9]
  \definition{adv.}{(em segundo lugar) precedido apenas por\dots; quanto menor o nível, mais tarde a ordem}
  \synonymref{差不多}{cha4bu5duo1}
\end{EntryWithPhonetic}

\begin{EntryWithPhonetic}{仅仅}{jin3jin3}{4,4}{⼈,⼈}[HSK 3]
  \definition{adv.}{somente; meramente; por muito pouco; indica que está limitado a um determinado âmbito}
  \synonymref{只是}{zhi3shi4}
  \antonymref{处处}{chu4chu4}
  \antonymref{甚至}{shen4zhi4}
  \antonymref{统统}{tong3tong3}
\end{EntryWithPhonetic}

%%%%%%%%%% 尽 %%%%%%%%%%
\subsection*{尽}\addcontentsline{loh}{figure}{尽 \dpy{jin3}}

\begin{EntryWithPhonetic}{尽}{jin3}{6}{⼫}[HSK 7-9]
  \definition{adv.}{na maior extensão possível | na extremidade mais distante de | usado antes de palavras que indicam direção, o mesmo que 最 | de agora em diante}
  \definition{prep.}{dentro dos limites de}
  \definition{v.}{dar prioridade a; deixar que certas pessoas ou coisas tenham precedência}
  \seeref{jin4}
  \seealsoref{最}{zui4}
\end{EntryWithPhonetic}

\begin{EntryWithPhonetic}{尽管}{jin3guan3}{6,14}{⼫,⽵}[HSK 5]
  \definition{adv.}{justo; livremente; faça o que quiser, não se preocupe, não há restrições de movimento ou comportamento}
  \definition{conj.}{no entanto; embora; apesar de ; normalmente usado no início de uma frase anterior para introduzir um fato, seguido de 但是, etc. para introduzir um resultado que o fato não deveria ter; às vezes, também pode ser usado no início de uma frase posterior.}
  \seealsoref{但是}{dan4shi4}
\end{EntryWithPhonetic}

\begin{EntryWithPhonetic}{尽可能}{jin3ke3neng2}{6,5,10}{⼫,⼝,⾁}[HSK 5]
  \definition{adv.}{na medida do possível; com o melhor de sua capacidade; tentar fazer algo, atingir um determinado nível ou extensão}
\end{EntryWithPhonetic}

\begin{EntryWithPhonetic}{尽快}{jin3kuai4}{6,7}{⼫,⼼}[HSK 4]
  \definition{adv.}{com toda a velocidade; o mais rápido possível; o mais breve possível}
\end{EntryWithPhonetic}

\begin{EntryWithPhonetic}{尽量}{jin3liang4}{6,12}{⼫,⾥}[HSK 3]
  \definition{adv.}{tanto quanto possível; da melhor maneira possível}
\end{EntryWithPhonetic}

\begin{EntryWithPhonetic}{尽早}{jin3zao3}{6,6}{⼫,⽇}[HSK 7-9]
  \definition{adv.}{o mais cedo possível; assim que possível; indica que deve ser feito o mais cedo possível}
\end{EntryWithPhonetic}

%%%%%%%%%% 紧 %%%%%%%%%%
\subsection*{紧}\addcontentsline{loh}{figure}{紧 \dpy{jin3}}

\begin{EntryWithPhonetic}{紧}{jin3}{10}{⽷}[HSK 3]
  \definition{adj.}{tenso; apertado; o estado em que um objeto se encontra após ser submetido a uma grande força de tração ou pressão.| seguro; firme | cerrado; apertado | urgente; premente; tenso | rigoroso; rígido; severo | difícil; sem dinheiro}
  \definition{v.}{apertar; tornar mais apertado}
\end{EntryWithPhonetic}

\begin{EntryWithPhonetic}{紧凑}{jin3cou4}{10,11}{⽷,⼎}[HSK 7-9]
  \definition{adj.}{compacto; conciso; bem estruturado; rígido; sucinto}
\end{EntryWithPhonetic}

\begin{EntryWithPhonetic}{紧急}{jin3ji2}{10,9}{⽷,⼼}[HSK 3]
  \definition{adj./adj.}{urgente; premente; crítico}
\end{EntryWithPhonetic}

\begin{EntryWithPhonetic}{紧接着}{jin3 jie1zhe5}{10,11,11}{⽷,⼿,⽬}[HSK 7-9]
  \definition{expr.}{imediatamente depois; uma coisa aconteceu após a outra}
\end{EntryWithPhonetic}

\begin{EntryWithPhonetic}{紧紧}{jin3jin3}{10,10}{⽷,⽷}[HSK 5]
  \definition{adv.}{firmemente; estreitamente; apertadamente; prestar muita atenção (em algo)}
\end{EntryWithPhonetic}

\begin{EntryWithPhonetic}{紧密}{jin3mi4}{10,11}{⽷,⼧}[HSK 4]
  \definition{adj.}{próximos; inseparáveis | incessante; rápido e intenso}
\end{EntryWithPhonetic}

\begin{EntryWithPhonetic}{紧迫}{jin3po4}{10,8}{⽷,⾡}[HSK 7-9]
  \definition{adj.}{urgente; premente; iminente; sem margem para manobras}
\end{EntryWithPhonetic}

\begin{EntryWithPhonetic}{紧缺}{jin3que1}{10,10}{⽷,⽸}[HSK 7-9]
  \definition{adj.}{em falta; extremamente necessário | escasso}
\end{EntryWithPhonetic}

\begin{EntryWithPhonetic}{紧缩}{jin3suo1}{10,14}{⽷,⽷}[HSK 7-9]
  \definition{v.}{reduzir; cortar; desmantelar; encolher}
\end{EntryWithPhonetic}

\begin{EntryWithPhonetic}{紧张}{jin3zhang1}{10,7}{⽷,⼸}[HSK 3]
  \definition{adj.}{nervoso; tenso; mentalmente em estado de alerta, excitado e inquieto | apertado; em falta; o que está disponível não satisfaz os requisitos| tenso; intenso; intenso ou urgente, causando tensão mental}
\end{EntryWithPhonetic}

%%%%%%%%%% 谨 %%%%%%%%%%
\subsection*{谨}\addcontentsline{loh}{figure}{谨 \dpy{jin3}}

\begin{EntryWithPhonetic}{谨}{jin3}{13}{⾔}
  \definition{adj.}{cuidadoso; cauteloso; circunspecto | solene; sincero; respeitoso}
\end{EntryWithPhonetic}

\begin{EntryWithPhonetic}{谨慎}{jin3shen4}{13,13}{⾔,⼼}[HSK 7-9]
  \definition{adj.}{prudente; cuidadoso; cauteloso; circunspecto}
\end{EntryWithPhonetic}

%%%%%%%%%% 锦 %%%%%%%%%%
\subsection*{锦}\addcontentsline{loh}{figure}{锦 \dpy{jin3}}

\begin{EntryWithPhonetic}{锦}{jin3}{13}{⾦}
  \definition*{s.}{Sobrenome: Jin}
  \definition{adj.}{brilhante e bonito (cores brilhantes e lindas)}
  \definition[块]{s.}{brocado; tecidos de seda com padrões coloridos}
\end{EntryWithPhonetic}

\begin{EntryWithPhonetic}{锦旗}{jin3qi2}{13,14}{⾦,⽅}[HSK 7-9]
  \definition{s.}{estandarte de seda (como prêmio ou presente) | flâmula; bandeiras feitas de seda colorida são concedidas aos vencedores de competições ou trabalhos produtivos, ou dadas a grupos ou indivíduos como sinal de respeito ou gratidão}
\end{EntryWithPhonetic}

\begin{EntryWithPhonetic}{锦上添花}{jin3 shang4 tian1 hua1}{13,3,11,7}{⾦,⼀,⽔,⾋}
  \definition{expr.}{adicionar flores ao brocado, tornar o que é bom ainda melhor; melhorar | dourando o lírio}
\end{EntryWithPhonetic}

%%%%%%%%%% 尽 %%%%%%%%%%
\subsection*{尽}\addcontentsline{loh}{figure}{尽 \dpy{jin4}}

\begin{EntryWithPhonetic}{尽}{jin4}{6}{⼫}[HSK 6]
  \definition*{s.}{Sobrenome: Jin}
  \definition{adj.}{exausto; acabado | ao máximo; ao limite | tudo; exaustivo}
  \definition{v.}{esgotar | tentar o seu melhor; fazer o melhor uso possível | morrer; falecer | terminar | chegar ao fim ao máximo; alcançar extremos}
  \seeref{jin3}
\end{EntryWithPhonetic}

\begin{EntryWithPhonetic}{尽力}{jin4/li4}{6,2}{⼫,⼒}[HSK 4]
  \definition{v.+compl.}{esforçar-se ao máximo; esforçar-se ao máximo; usar toda a sua força; fazer algo com seu melhor esforço}
\end{EntryWithPhonetic}

\begin{EntryWithPhonetic}{尽情}{jin4qing2}{6,11}{⼫,⼼}[HSK 7-9]
  \definition{v.}{expressar os próprios sentimentos de forma plena e livre; significa agir de acordo com os próprios sentimentos, na medida do possível}
\end{EntryWithPhonetic}

\begin{EntryWithPhonetic}{尽头}{jin4tou2}{6,5}{⼫,⼤}[HSK 7-9]
  \definition[台]{s.}{fim}[小路尽头是一片树林。===No fim do caminho havia um bosque de árvores.]
\end{EntryWithPhonetic}

%%%%%%%%%% 劲 %%%%%%%%%%
\subsection*{劲}\addcontentsline{loh}{figure}{劲 \dpy{jin4}}

\begin{EntryWithPhonetic}{劲}{jin4}{7}{⼒}
  \definition{s.}{força; energia | vigor; espírito; ímpeto; fervor | ar; modo; expressão | interesse; prazer; entusiasmo}
  \seeref{jing4}
  \seealsoref{劲儿}{jin4r5}
\end{EntryWithPhonetic}

\begin{EntryWithPhonetic}{劲儿}{jin4r5}{7,2}{⼒,⼉}
  \definition{s.}{energia; força}
  \seealsoref{劲}{jin4}
\end{EntryWithPhonetic}

\begin{EntryWithPhonetic}{劲头}{jin4tou2}{7,5}{⼒,⼤}[HSK 7-9]
  \definition{s.}{força; energia; poder | vigor; espírito; ímpeto; fervor}
\end{EntryWithPhonetic}

%%%%%%%%%% 近 %%%%%%%%%%
\subsection*{近}\addcontentsline{loh}{figure}{近 \dpy{jin4}}

\begin{EntryWithPhonetic}{近}{jin4}{7}{⾡}[HSK 2]
  \definition{adj.}{próximo; perto; distância espacial ou temporal curta | íntimo; intimamente relacionado; relação estreita | fácil de entender}
  \antonymref{远}{yuan3}
\end{EntryWithPhonetic}

\begin{EntryWithPhonetic}{近代}{jin4dai4}{7,5}{⾡,⼈}[HSK 4]
  \definition{s.}{tempos modernos; era passada relativamente próxima à era moderna, geralmente referida na história chinesa como 1840 a 1919 | o tempo ou era do capitalismo}
\end{EntryWithPhonetic}

\begin{EntryWithPhonetic}{近来}{jin4lai2}{7,7}{⾡,⽊}[HSK 5]
  \definition{adv.}{ultimamente; recentemente; de tarde; refere"-se a um período de tempo entre o passado imediato e o presente}
\end{EntryWithPhonetic}

\begin{EntryWithPhonetic}{近年来}{jin4nian2 lai2}{7,6,7}{⾡,⼲,⽊}[HSK 7-9]
  \definition{expr.}{nos últimos anos}
\end{EntryWithPhonetic}

\begin{EntryWithPhonetic}{近期}{jin4qi1}{7,12}{⾡,⽉}[HSK 3]
  \definition{adv.}{num futuro próximo; brevemente}
\end{EntryWithPhonetic}

\begin{EntryWithPhonetic}{近日}{jin4ri4}{7,4}{⾡,⽇}[HSK 6]
  \definition{s.}{recentemente; nos últimos dias; apontando para o passado | nos próximos dias; refere"-se ao futuro}
\end{EntryWithPhonetic}

\begin{EntryWithPhonetic}{近视}{jin4shi4}{7,8}{⾡,⾒}[HSK 6]
  \definition{adj.}{miopia; uma deficiência visual em que a visão próxima é clara, mas a visão distante é turva | míope (figurativo); metáfora para miopia}
\end{EntryWithPhonetic}

%%%%%%%%%% 进 %%%%%%%%%%
\subsection*{进}\addcontentsline{loh}{figure}{进 \dpy{jin4}}

\begin{EntryWithPhonetic}{进}{jin4}{7}{⾡}[HSK 1]
  \definition*{s.}{Sobrenome: Jin}
  \definition{clas.}{para seções em um edifício ou complexo residencial; qualquer uma das várias fileiras de casas em um complexo residencial de estilo antigo}
  \definition{s.}{Matemática: base de um sistema numérico}
  \definition{v.}{avançar; ir adiante; seguir em frente; oposto a 退 | entrar; entrar em; entrar ou sair; oposto a 出 | receber | comer; tomar; beber | submeter; apresentar | marcar um gol}
  \definition{v.aux.}{usado após um verbo, significa ``para dentro''}
  \seealsoref{出}{chu1}
  \seealsoref{退}{tui4}
\end{EntryWithPhonetic}

\begin{EntryWithPhonetic}{进步}{jin4bu4}{7,7}{⾡,⽌}[HSK 3]
  \definition{adj.}{progressivo; adequado às tendências da época; que impulsiona o desenvolvimento social}
  \definition{v.}{avançar; progredir; melhorar}
  \antonymref{落后}{luo4/hou4}
\end{EntryWithPhonetic}

\begin{EntryWithPhonetic}{进场}{jin4chang3}{7,6}{⾡,⼟}[HSK 7-9]
  \definition{v.}{entrar na arena; marchar | Aeroespacial: abordagem}
  \antonymref{退场}{tui4chang3}
\end{EntryWithPhonetic}

\begin{EntryWithPhonetic}{进程}{jin4cheng2}{7,12}{⾡,⽲}[HSK 7-9]
  \definition{s.}{processo; progresso; procedimento; o processo de mudança ou progresso das coisas}
\end{EntryWithPhonetic}

\begin{EntryWithPhonetic}{进出}{jin4chu1}{7,5}{⾡,⼐}[HSK 7-9]
  \definition{s.}{(faturamento) empresarial | entrada e saída | recibos e pagamentos}
  \definition{v.}{passar para dentro e para fora;  entrar e sair}
\end{EntryWithPhonetic}

\begin{EntryWithPhonetic}{进出口}{jin4-chu1kou3}{7,5,3}{⾡,⼐,⼝}[HSK 7-9]
  \definition{s.}{importação e exportação | saída; saídas e entradas}
\end{EntryWithPhonetic}

\begin{EntryWithPhonetic}{进度}{jin4du4}{7,9}{⾡,⼴}[HSK 7-9]
  \definition{s.}{taxa de progresso; taxa de avanço; a velocidade com que (o trabalho, o estudo, etc.) progride}
\end{EntryWithPhonetic}

\begin{EntryWithPhonetic}{进而}{jin4'er2}{7,6}{⾡,⽽}[HSK 7-9]
  \definition{conj.}{em seguida; além disso; e então; depois disso; continue em frente; vá mais longe}
\end{EntryWithPhonetic}

\begin{EntryWithPhonetic}{进攻}{jin4gong1}{7,7}{⾡,⽁}[HSK 6]
  \definition{s.}{ofensiva}
  \definition{v.}{atacar; assaltar; tomar a ofensiva}
  \antonymref{防守}{fang2shou3}
\end{EntryWithPhonetic}

\begin{EntryWithPhonetic}{进化}{jin4hua4}{7,4}{⾡,⼔}[HSK 5]
  \definition[个]{s.}{evolução; os organismos se desenvolvem e evoluem do simples para o complexo e de níveis baixos para altos}
  \definition{v.}{evoluir; um termo geral usado para descrever uma mudança gradual para melhor}
\end{EntryWithPhonetic}

\begin{EntryWithPhonetic}{进口}{jin4/kou3}{7,3}{⾡,⼝}[HSK 4]
  \definition{adj.}{importado}
  \definition{s.}{importação; entrada de um edifício ou local, também chamada de 入口}
  \definition{v.+compl.}{importar; comprar ou transportar mercadorias de outro país ou região | entrar no porto; navegar em direção a um porto}
  \seealsoref{入口}{ru4/kou3}
\end{EntryWithPhonetic}

\begin{EntryWithPhonetic}{进来}{jin4 lai5}{7,7}{⾡,⽊}[HSK 1]
  \definition{v.}{entrar (para a minha localização)}
\end{EntryWithPhonetic}

\begin{EntryWithPhonetic}{进去}{jin4 qu5}{7,5}{⾡,⼛}[HSK 1]
  \definition{v.}{entrar (a partir da minha localização)}
  \definition{v.aux.}{usado depois de um verbo, significa ``ir para dentro''; para um determinado intervalo ou período de tempo}
\end{EntryWithPhonetic}

\begin{EntryWithPhonetic}{进入}{jin4ru4}{7,2}{⾡,⼊}[HSK 2]
  \definition{v.}{entrar; entrar em}
\end{EntryWithPhonetic}

\begin{EntryWithPhonetic}{进行}{jin4xing2}{7,6}{⾡,⾏}[HSK 2]
  \definition{v.}{continuar; estar em andamento; estar em progresso | fazer; conduzir; realizar; executar | marchar; avançar; prosseguir; estar em marcha}
\end{EntryWithPhonetic}

\begin{EntryWithPhonetic}{进行编程}{jin4xing2bian1cheng2}{7,6,12,12}{⾡,⾏,⽷,⽲}
  \definition{s.}{programa de computador executável}
\end{EntryWithPhonetic}

\begin{EntryWithPhonetic}{进修}{jin4xiu1}{7,9}{⾡,⼈}[HSK 7-9]
  \definition{v.}{participar de estudos avançados; fazer um curso de atualização; envolver-se em estudos avançados; aprimorar as competências profissionais com formação complementar}
\end{EntryWithPhonetic}

\begin{EntryWithPhonetic}{进一步}{jin4yi2bu4}{7,1,7}{⾡,⼀,⽌}[HSK 3]
  \definition{adv.}{mais; dar um passo adiante; avançar um passo; indica que as coisas estão progredindo em um nível mais alto do que antes}
\end{EntryWithPhonetic}

\begin{EntryWithPhonetic}{进展}{jin4zhan3}{7,10}{⾡,⼫}[HSK 3]
  \definition{v.}{fazer progresso; progredir; avançar no desenvolvimento}
\end{EntryWithPhonetic}

%%%%%%%%%% 晋 %%%%%%%%%%
\subsection*{晋}\addcontentsline{loh}{figure}{晋 \dpy{jin4}}

\begin{EntryWithPhonetic}{晋}{jin4}{10}{⽇}
  \definition*{s.}{Estado da Dinastia Zhou (1046-256 a.C.), ocupando partes do que hoje são Shanxi, Shaanxi, Hebei e Henan | Dinastia Jin Ocidental (265-316), Dinastia Jin Oriental (317-420) e Dinastia Jin Posterior (936-946) | Nome abreviado da província de Shanxi: 山西 | Sobrenome: Jin}
  \definition{v.}{avançar | promover}
  \seealsoref{山西}{shan1xi1}
\end{EntryWithPhonetic}

\begin{EntryWithPhonetic}{晋升}{jin4sheng1}{10,4}{⽇,⼗}[HSK 7-9]
  \definition{v.}{elevar; promover (a um cargo superior)}
\end{EntryWithPhonetic}

%%%%%%%%%% 浸 %%%%%%%%%%
\subsection*{浸}\addcontentsline{loh}{figure}{浸 \dpy{jin4}}

\begin{EntryWithPhonetic}{浸}{jin4}{10}{⽔}
  \definition{adv.}{gradualmente; passo a passo; pouco a pouco | Literário: gradualmente; cada vez mais}
  \definition{v.}{deixar de molho; imergir; mergulhar | saturar}
\end{EntryWithPhonetic}

\begin{EntryWithPhonetic}{浸泡}{jin4pao4}{10,8}{⽔,⽔}[HSK 7-9]
  \definition{v.}{mergulhar; banhar; imergir; deixar de molho em líquido}
\end{EntryWithPhonetic}

%%%%%%%%%% 禁 %%%%%%%%%%
\subsection*{禁}\addcontentsline{loh}{figure}{禁 \dpy{jin4}}

\begin{EntryWithPhonetic}{禁}{jin4}{13}{⽰}
  \definition*{s.}{Sobrenome: Jin}
  \definition{s.}{um tabu; assuntos não permitidos por lei ou costume | área proibida | residência real; o lugar onde o imperador viveu nos tempos antigos}
  \definition{v.}{proibir; banir | aprisionar; deter}
\end{EntryWithPhonetic}

\begin{EntryWithPhonetic}{禁忌}{jin4ji4}{13,7}{⽰,⼼}[HSK 7-9]
  \definition[个]{s.}{tabu}
  \definition{v.}{evitar; abster-se de; contraindicar; evitar certos comportamentos geralmente se refere à abstinência de certos alimentos ou medicamentos}
\end{EntryWithPhonetic}

\begin{EntryWithPhonetic}{禁区}{jin4qu1}{13,4}{⽰,⼖}[HSK 7-9]
  \definition{s.}{zona proibida; zona restrita; faixa proibida | reserva (de vida selvagem ou vegetal); parque natural | (futebol) área de penalidade, grande área; (basquetebol) área restrita | área fora dos limites; áreas que são proibidas}
\end{EntryWithPhonetic}

\begin{EntryWithPhonetic}{禁止}{jin4zhi3}{13,4}{⽰,⽌}[HSK 4]
  \definition{v.}{banir; proibir; interditar}
\end{EntryWithPhonetic}

%%%%%%%%%% 京 %%%%%%%%%%
\subsection*{京}\addcontentsline{loh}{figure}{京 \dpy{jing1}}

\begin{EntryWithPhonetic}{京}{jing1}{8}{⼇}
  \definition*{s.}{Pequim (Beijing), abreviação de 北京 | Sobrenome: Jing}
  \definition{num.}{dez milhões (um numeral antigo); 10.000.000; 1000.0000}
  \definition{s.}{capital de um país}
  \seealsoref{北京}{bei3jing1}
\end{EntryWithPhonetic}

\begin{EntryWithPhonetic}{京二胡}{jing1'er4hu2}{8,2,9}{⼇,⼆,⾁}
  \definition{s.}{um tipo de violino chinês semelhante ao 二胡 de duas cordas, usado principalmente para acompanhamento do canto da ópera de Pequim | também chamado de 京胡 | jing'erhu, um violino de duas cordas, intermediário em tamanho e tom entre o 京胡 e o 二胡, usado para acompanhar a ópera chinesa}
  \seealsoref{二胡}{er4hu2}
  \seealsoref{京胡}{jing1hu2}
\end{EntryWithPhonetic}

\begin{EntryWithPhonetic}{京胡}{jing1hu2}{8,9}{⼇,⾁}
  \definition{s.}{jinghu, um instrumento de arco de duas cordas com registro agudo; violino da ópera de Pequim | também chamado de 京二胡 | jinghu, um 二胡 (violino de duas cordas) menor e mais agudo, usado para acompanhar a ópera chinesa}
  \seealsoref{二胡}{er4hu2}
  \seealsoref{胡琴}{hu2qin2}
  \seealsoref{京二胡}{jing1'er4hu2}
\end{EntryWithPhonetic}

\begin{EntryWithPhonetic}{京剧}{jing1ju4}{8,10}{⼇,⼑}[HSK 3]
  \definition*[场,段]{s.}{Ópera de Pequim}
  \synonymref{戏剧}{xi4ju4}
\end{EntryWithPhonetic}

%%%%%%%%%% 经 %%%%%%%%%%
\subsection*{经}\addcontentsline{loh}{figure}{经 \dpy{jing1}}

\begin{EntryWithPhonetic}{经}{jing1}{8}{⽷}[HSK 7-9]
  \definition*{s.}{Sobrenome: Jing}
  \definition{adj.}{constante; regular}
  \definition{prep.}{como resultado de; depois; através de}
  \definition{s.}{urdidura, os fios longitudinais de um tecido | Medicina chinesa: canais principais e colaterais | Geografia: longitude | escritura; sutra; cânone; clássico | menstruação}
  \definition{v.}{Literário: gerenciar; lidar com; envolver-se em | enforcar-se | suportar; ficar de pé; aguentar; resistir | passar por; sofrer; experimentar}
  \seeref{jing4}
  \antonymref{纬}{wei3}
\end{EntryWithPhonetic}

\begin{EntryWithPhonetic}{经常}{jing1chang2}{8,11}{⽷,⼱}[HSK 2]
  \definition{adj.}{habitual; cotidiano; diário; do dia a dia}
  \definition{adv.}{frequentemente; regularmente; constantemente; com frequência; indica que a ação ocorre repetidamente}
\end{EntryWithPhonetic}

\begin{EntryWithPhonetic}{经典}{jing1dian3}{8,8}{⽷,⼋}[HSK 4]
  \definition{adj.}{clássico; (escritos ou obras, etc.) que são típicos, autorizados}
  \definition{s.}{clássicos; escritos tradicionais e valiosos; os livros mais importantes e fundamentais da religião | escrituras; escritos de doutrinas religiosas}
\end{EntryWithPhonetic}

\begin{EntryWithPhonetic}{经度}{jing1du4}{8,9}{⽷,⼴}[HSK 7-9]
  \definition{s.}{longitude}
\end{EntryWithPhonetic}

\begin{EntryWithPhonetic}{经费}{jing1fei4}{8,9}{⽷,⾙}[HSK 5]
  \definition[笔]{s.}{fundos; desembolso; gastos | despesas; gastos}
\end{EntryWithPhonetic}

\begin{EntryWithPhonetic}{经过}{jing1guo4}{8,6}{⽷,⾡}[HSK 2]
  \definition{prep.}{depois; através; como resultado de; passar por uma atividade ou evento que traz novas mudanças para pessoas ou coisas}
  \definition[个,段,番]{s.}{processo; curso; experiência}
  \definition{v.}{passar; atravessar; passar por; através de (local, tempo, ação, etc.)}
\end{EntryWithPhonetic}

\begin{EntryWithPhonetic}{经济}{jing1ji4}{8,9}{⽷,⽔}[HSK 3]
  \definition{adj.}{econômico; parcimonioso; descreve algo que custa pouco e rende muito; preço acessível}
  \definition{s.}{economia; a soma das relações de produção social em um determinado período histórico|econômico; de valor industrial ou econômico; refere"-se à economia nacional; também se refere a um determinado setor da economia nacional | economia; refere"-se às atividades econômicas, incluindo produção, circulação, distribuição e consumo, bem como atividades ou processos financeiros, de seguros, etc. | renda; situação financeira; refere"-se à situação financeira de uma pessoa}
  \definition{v.}{governar o país e beneficiar o povo}
\end{EntryWithPhonetic}

\begin{EntryWithPhonetic}{经久不息}{jing1jiu3-bu4xi1}{8,3,4,10}{⽷,⼃,⼀,⼼}[HSK 7-9]
  \definition{expr.}{prolongado; duradouro}
\end{EntryWithPhonetic}

\begin{EntryWithPhonetic}{经理}{jing1li3}{8,11}{⽷,⽟}[HSK 2]
  \definition[个,位,名]{s.}{gerente; diretor; pessoas responsáveis pela gestão e administração de empresas ou corporações}
\end{EntryWithPhonetic}

\begin{EntryWithPhonetic}{经历}{jing1li4}{8,4}{⽷,⼚}[HSK 3]
  \definition[个,次,段,种]{s.}{experiência; coisas que você viu, fez ou sofreu pessoalmente}
  \definition{v.}{passar por; atravessar; ter visto, feito ou sofrido pessoalmente}
\end{EntryWithPhonetic}

\begin{EntryWithPhonetic}{经贸}{jing1mao4}{8,9}{⽷,⾙}[HSK 7-9]
  \definition{s.}{economia e comércio; o termo coletivo para economia e comércio}
\end{EntryWithPhonetic}

\begin{EntryWithPhonetic}{经商}{jing1/shang1}{8,11}{⽷,⼝}[HSK 7-9]
  \definition{v.+compl.}{dedicar-se ao comércio; estar em atividade comercial; envolver-se em atividades comerciais}
\end{EntryWithPhonetic}

\begin{EntryWithPhonetic}{经受}{jing1shou4}{8,8}{⽷,⼜}[HSK 7-9]
  \definition{v.}{experimentar; suportar; resistir; aguentar; passar por}
\end{EntryWithPhonetic}

\begin{EntryWithPhonetic}{经线}{jing1xian4}{8,8}{⽷,⽷}
  \definition{s.}{urdidura | Geografia: meridiano | linha de longitude}
\end{EntryWithPhonetic}

\begin{EntryWithPhonetic}{经验}{jing1yan4}{8,10}{⽷,⾺}[HSK 3]
  \definition[个,次,种]{s.}{experiência; conhecimento ou habilidades adquiridos através da prática}
  \definition{v.}{experimentar; passar por; ter visto, feito ou sofrido pessoalmente}
\end{EntryWithPhonetic}

\begin{EntryWithPhonetic}{经营}{jing1ying2}{8,11}{⽷,⾋}[HSK 3]
  \definition{v.}{executar; gerenciar; operar; envolver"-se em; planejar e gerenciar (empresas, etc.) | gerenciar; refere"-se a planos e organizações em geral}
\end{EntryWithPhonetic}

%%%%%%%%%% 茎 %%%%%%%%%%
\subsection*{茎}\addcontentsline{loh}{figure}{茎 \dpy{jing1}}

\begin{EntryWithPhonetic}{茎}{jing1}{8}{⾋}[HSK 7-9]
  \definition{clas.}{Literário: utilizado como classificador para indicar um objeto em forma de tira}[几茎小草。===Algumas folhas de grama.]
  \definition[根]{s.}{caule (de uma planta); talo; tronco | algo como um caule ou haste}
\end{EntryWithPhonetic}

%%%%%%%%%% 荆 %%%%%%%%%%
\subsection*{荆}\addcontentsline{loh}{figure}{荆 \dpy{jing1}}

\begin{EntryWithPhonetic}{荆}{jing1}{9}{⾋}
  \definition*{s.}{Sobrenome: Jing}
  \definition{s.}{árvore-da-castidade; vitex | uma vara para açoitar; varas de punição antigas feitas de galhos de espinho | a própria esposa; uma forma humilde de se referir à esposa antigamente}
\end{EntryWithPhonetic}

\begin{EntryWithPhonetic}{荆棘}{jing1ji2}{9,12}{⾋,⽊}[HSK 7-9]
  \definition{s.}{silvas; cardos e espinhos; vegetação rasteira espinhosa; geralmente se refere a arbustos espinhosos que crescem nas montanhas e nos campos}
\end{EntryWithPhonetic}

%%%%%%%%%% 惊 %%%%%%%%%%
\subsection*{惊}\addcontentsline{loh}{figure}{惊 \dpy{jing1}}

\begin{EntryWithPhonetic}{惊}{jing1}{11}{⼼}[HSK 7-9]
  \definition{v.}{assustar; ficar assustado; ficar nervoso devido a estímulo repentino; ficar com medo | surpreender; chocar; alarmar}
\end{EntryWithPhonetic}

\begin{EntryWithPhonetic}{惊诧}{jing1cha4}{11,8}{⼼,⾔}[HSK 7-9]
  \definition{adj.}{surpreso; admirado; estupefato | muito surpreso}
\end{EntryWithPhonetic}

\begin{EntryWithPhonetic}{惊呆}{jing1dai1}{11,7}{⼼,⼝}
  \definition{adj.}{atordoado; estupefato; chocado}
\end{EntryWithPhonetic}

\begin{EntryWithPhonetic}{惊慌}{jing1huang1}{11,12}{⼼,⼼}[HSK 7-9]
  \definition{adj.}{assustado; alarmado; amedrontado; atemorizado; em pânico}
\end{EntryWithPhonetic}

\begin{EntryWithPhonetic}{惊慌失措}{jing1huang1-shi1cuo4}{11,12,5,11}{⼼,⼼,⼤,⼿}[HSK 7-9]
  \definition{expr.}{apavorado; tomado pelo pânico; em pânico; perder a cabeça de medo; no Capítulo 11 da Parte 4 de ``O Oriente'', de Wei Wei: ``Guo Xiang e seus homens lançaram um ataque feroz, e o inimigo, em pânico, esqueceu-se de resistir, preocupando-se apenas em subir nos tanques.''; confusão aterrorizante; ficar apavorado; ficar em pânico}
\end{EntryWithPhonetic}

\begin{EntryWithPhonetic}{惊奇}{jing1qi2}{11,8}{⼼,⼤}[HSK 7-9]
  \definition{v.}{maravilhar-se; ficar surpreso; ficar boquiaberto}
\end{EntryWithPhonetic}

\begin{EntryWithPhonetic}{惊人}{jing1ren2}{11,2}{⼼,⼈}[HSK 6]
  \definition{adj.}{surpreso; espantado; atônito; surpreendente}
\end{EntryWithPhonetic}

\begin{EntryWithPhonetic}{惊叹}{jing1tan4}{11,5}{⼼,⼝}[HSK 7-9]
  \definition{v.}{maravilhar-se com; admirar-se com; exclamar (com admiração)}
\end{EntryWithPhonetic}

\begin{EntryWithPhonetic}{惊天动地}{jing1tian1-dong4di4}{11,4,6,6}{⼼,⼤,⼒,⼟}[HSK 7-9]
  \definition{expr.}{que abala os céus e a terra; que faz a terra tremer; assustar os céus e mover a terra; sacudir os céus e assustar a terra; que abala o mundo; de proporções sísmicas; de impacto mundial}
\end{EntryWithPhonetic}

\begin{EntryWithPhonetic}{惊喜}{jing1xi3}{11,12}{⼼,⼝}[HSK 6]
  \definition{s.}{boa surpresa; agradavelmente surpreso}
\end{EntryWithPhonetic}

\begin{EntryWithPhonetic}{惊险}{jing1xian3}{11,9}{⼼,⾩}[HSK 7-9]
  \definition{adj.}{emocionante; de tirar o fôlego; alarmantemente perigoso}
\end{EntryWithPhonetic}

\begin{EntryWithPhonetic}{惊心动魄}{jing1xin1-dong4po4}{11,4,6,14}{⼼,⼼,⼒,⿁}[HSK 7-9]
  \definition{expr.}{comovente; profundamente impactante; ficar apavorado (horror); de tirar o fôlego; de arrepiar os cabelos; emocionante; fazer o coração de alguém disparar; abalar alguém profundamente; deixar alguém sem fôlego}
\end{EntryWithPhonetic}

\begin{EntryWithPhonetic}{惊醒}{jing1xing3}{11,16}{⼼,⾣}[HSK 7-9]
  \definition{v.}{acordar sobressaltado; despertado pelo susto; despertado pelo choque}
\end{EntryWithPhonetic}

\begin{EntryWithPhonetic}{惊讶}{jing1ya4}{11,6}{⼼,⾔}[HSK 7-9]
  \definition{adj.}{surpreso; admirado; estupefato; atônito; sentindo-me surpreso e estranho}
\end{EntryWithPhonetic}

\begin{EntryWithPhonetic}{惊异}{jing1yi4}{11,6}{⼼,⼶}
  \definition{adj.}{surpreso; admirado; estupefato; atônito}
\end{EntryWithPhonetic}

%%%%%%%%%% 晶 %%%%%%%%%%
\subsection*{晶}\addcontentsline{loh}{figure}{晶 \dpy{jing1}}

\begin{EntryWithPhonetic}{晶}{jing1}{12}{⽇}
  \definition{adj.}{brilhante; cristalino}
  \definition{s.}{cristal de rocha; cristal; quartzo}
\end{EntryWithPhonetic}

\begin{EntryWithPhonetic}{晶莹}{jing1ying2}{12,10}{⽇,⾋}[HSK 7-9]
  \definition{adj.}{brilhante e cristalino; cintilante e translúcido; brilhante e transparente}
\end{EntryWithPhonetic}

%%%%%%%%%% 兢 %%%%%%%%%%
\subsection*{兢}\addcontentsline{loh}{figure}{兢 \dpy{jing1}}

\begin{EntryWithPhonetic}{兢}{jing1}{14}{⼉}
  \definition{adj.}{medroso | cauteloso | forte}
  \definition{v.}{mover}
\end{EntryWithPhonetic}

\begin{EntryWithPhonetic}{兢兢业业}{jing1jing1ye4ye4}{14,14,5,5}{⼉,⼉,⼀,⼀}[HSK 7-9]
  \definition{expr.}{cauteloso e consciencioso; zeloso; aplicado; descreve alguém que é muito cuidadoso, cauteloso e responsável ao fazer as coisas}
\end{EntryWithPhonetic}

%%%%%%%%%% 精 %%%%%%%%%%
\subsection*{精}\addcontentsline{loh}{figure}{精 \dpy{jing1}}

\begin{EntryWithPhonetic}{精}{jing1}{14}{⽶}[HSK 6]
  \definition{adj.}{refinado; escolhido; purificado ou selecionado | perfeito; excelente; melhor | fino; preciso; meticuloso | inteligente; astuto; esperto | habilidoso; versado; proficiente}
  \definition{adv.}{muito; extremamente; antes de certos adjetivos, significa 十分 ou 非常}
  \definition{s.}{extrato; essência; essência refinada ou selecionada; extraída | energia; espírito | semente; esperma; sêmen | \emph{goblin}; espírito; elfo; demônio}
  \seealsoref{非常}{fei1chang2}
  \seealsoref{十分}{shi2fen1}
  \antonymref{粗}{cu1}
\end{EntryWithPhonetic}

\begin{EntryWithPhonetic}{精彩}{jing1cai3}{14,11}{⽶,⼺}[HSK 3]
  \definition{adj.}{brilhante; esplêndido; maravilhoso}
\end{EntryWithPhonetic}

\begin{EntryWithPhonetic}{精打细算}{jing1da3-xi4suan4}{14,5,8,14}{⽶,⼿,⽷,⽵}[HSK 7-9]
  \definition{expr.}{seja muito cuidadoso nos cálculos; contagem precisa; seja preciso nos cálculos; faça um orçamento rigoroso; cálculo cuidadoso e detalhado (meticuloso); cálculo cuidadoso e orçamento rigoroso; conte cada centavo e faça cada centavo valer a pena; corte os gastos com precisão; planejar com meticulosidade e cuidado, isso significa calcular com precisão o uso de mão de obra, recursos materiais e recursos financeiros para evitar desperdícios}
\end{EntryWithPhonetic}

\begin{EntryWithPhonetic}{精华}{jing1hua2}{14,6}{⽶,⼗}[HSK 7-9]
  \definition{s.}{elite; creme; escolha; essência; quintessência; a melhor e mais refinada parte de tudo | glória; esplendor; brilho; luz (do Sol e da Lua)}
\end{EntryWithPhonetic}

\begin{EntryWithPhonetic}{精简}{jing1jian3}{14,13}{⽶,⽵}[HSK 7-9]
  \definition{v.}{reduzir; simplificar; cortar; simplificar; eliminar o desnecessário e conservar o necessário}
\end{EntryWithPhonetic}

\begin{EntryWithPhonetic}{精力}{jing1li4}{14,2}{⽶,⼒}[HSK 4]
  \definition[些]{s.}{energia; vigor; força mental e física}
\end{EntryWithPhonetic}

\begin{EntryWithPhonetic}{精练}{jing1lian4}{14,8}{⽶,⽷}[HSK 7-9]
  \definition{adj.}{conciso; sucinto; lacônico | refinado; palavras e frases redundantes eliminadas}
  \definition{v.}{praticar intensivamente}
  \synonymref{干脆}{gan1cui4}
  \synonymref{简单}{jian3dan1}
  \synonymref{简洁}{jian3jie2}
  \synonymref{精炼}{jing1lian4}
  \antonymref{冗长}{rong3chang2}
\end{EntryWithPhonetic}

\begin{EntryWithPhonetic}{精炼}{jing1lian4}{14,9}{⽶,⽕}
  \definition{adj.}{conciso; sucinto; lacônico}
  \definition{s.}{refino; remoção de impurezas}
  \definition{v.}{Metalurgia: refinar; purificar; fundir}
\end{EntryWithPhonetic}

\begin{EntryWithPhonetic}{精灵}{jing1ling2}{14,7}{⽶,⽕}
  \definition{s.}{espírito | fada | elfo | duende | gênio}
\end{EntryWithPhonetic}

\begin{EntryWithPhonetic}{精美}{jing1mei3}{14,9}{⽶,⽺}[HSK 6]
  \definition{adj.}{elegante; requintado}
\end{EntryWithPhonetic}

\begin{EntryWithPhonetic}{精密}{jing1mi4}{14,11}{⽶,⼧}
  \definition{adj.}{preciso; preciso e meticuloso}
\end{EntryWithPhonetic}

\begin{EntryWithPhonetic}{精妙}{jing1miao4}{14,7}{⽶,⼥}[HSK 7-9]
  \definition{adj.}{requintado | fino e delicado (geralmente de obras de arte)}
\end{EntryWithPhonetic}

\begin{EntryWithPhonetic}{精明}{jing1ming2}{14,8}{⽶,⽇}[HSK 7-9]
  \definition{adj.}{astuto; sagaz; perspicaz; inteligente e brilhante}
\end{EntryWithPhonetic}

\begin{EntryWithPhonetic}{精疲力竭}{jing1pi2-li4jie2}{14,10,2,14}{⽶,⽧,⼒,⽴}[HSK 7-9]
  \definition{expr.}{esgotado; exausto; desgastado; descrevendo fadiga extrema e completa falta de energia}
\end{EntryWithPhonetic}

\begin{EntryWithPhonetic}{精品}{jing1pin3}{14,9}{⽶,⼝}[HSK 6]
  \definition[个]{s.}{belas obras (de arte); objetos de arte | produtos de qualidade; artigos de excelente qualidade; produto \emph{premium}}
\end{EntryWithPhonetic}

\begin{EntryWithPhonetic}{精确}{jing1que4}{14,12}{⽶,⽯}[HSK 7-9]
  \definition{adj.}{exato; preciso; acurado; muito preciso e correto}
\end{EntryWithPhonetic}

\begin{EntryWithPhonetic}{精神}{jing1shen2}{14,9}{⽶,⽰}[HSK 3]
  \definition[种,个,类,股]{s.}{espírito; mente; estado mental; refere"-se à consciência, às atividades mentais e ao estado psicológico geral de uma pessoa | substância; espírito; essência; propósito; significado principal}
  \seeref{jing1shen5}
\end{EntryWithPhonetic}

\begin{EntryWithPhonetic}{精神病}{jing1shen2bing4}{14,9,10}{⽶,⽰,⽧}[HSK 7-9]
  \definition{s.}{doença mental; transtorno mental; psicose}[这是幻想型精神病的体现。===Isso é uma manifestação de psicose delirante.]
\end{EntryWithPhonetic}

\begin{EntryWithPhonetic}{精神}{jing1shen5}{14,9}{⽶,⽰}[HSK 3]
  \definition{adj.}{animado; espirituoso; vigoroso; descreve uma pessoa como cheia de energia | muito bonito; boa aparência, bom físico}
  \definition[种,个,类,股]{s.}{impulso; vigor; vitalidade}
  \seeref{jing1shen2}
\end{EntryWithPhonetic}

\begin{EntryWithPhonetic}{精髓}{jing1sui3}{14,21}{⽶,⾻}[HSK 7-9]
  \definition{s.}{medula; medula óssea; quintessência; essência metafórica das coisas}
\end{EntryWithPhonetic}

\begin{EntryWithPhonetic}{精通}{jing1tong1}{14,10}{⽶,⾡}[HSK 7-9]
  \definition{v.}{dominar; ser proficiente em; ter um bom domínio de; ter um profundo entendimento e conhecimento abrangente de uma área específica de estudo, tecnologia ou negócios}
\end{EntryWithPhonetic}

\begin{EntryWithPhonetic}{精细}{jing1xi4}{14,8}{⽶,⽷}[HSK 7-9]
  \definition{adj.}{fino; cuidadoso; meticuloso; muito delicado | astuto; perspicaz e cuidadoso; muito meticuloso}
\end{EntryWithPhonetic}

\begin{EntryWithPhonetic}{精心}{jing1xin1}{14,4}{⽶,⼼}[HSK 7-9]
  \definition{adv.}{meticulosamente; cuidadosamente; elaboradamente; preste muita atenção; concentre-se totalmente}[他们精心设计了这个项目。===Eles planejaram este projeto meticulosamente.]
\end{EntryWithPhonetic}

\begin{EntryWithPhonetic}{精益求精}{jing1yi4qiu2jing1}{14,10,7,14}{⽶,⽫,⽔,⽶}[HSK 7-9]
  \definition{expr.}{``Busque a excelência.''; esforçar-se pela perfeição; buscar a melhoria constante; perseguir a excelência; almejar a perfeição; já está muito bom, mas você ainda quer que fique ainda melhor; melhorar algo constantemente; continuar melhorando}
\end{EntryWithPhonetic}

\begin{EntryWithPhonetic}{精英}{jing1ying1}{14,8}{⽶,⾋}[HSK 7-9]
  \definition{s.}{creme; essência; quintessência | escolhido; elite; pessoa de habilidade excepcional}
\end{EntryWithPhonetic}

\begin{EntryWithPhonetic}{精致}{jing1zhi4}{14,10}{⽶,⾄}[HSK 7-9]
  \definition{adj.}{fino; requintado; delicado}[我们欣赏她精致的手工艺品。===Admiramos seu trabalho artesanal requintado.]
\end{EntryWithPhonetic}

\begin{EntryWithPhonetic}{精子}{jing1zi3}{14,3}{⽶,⼦}
  \definition{s.}{espermatozoide; célula germinativa}
\end{EntryWithPhonetic}

%%%%%%%%%% 鲸 %%%%%%%%%%
\subsection*{鲸}\addcontentsline{loh}{figure}{鲸 \dpy{jing1}}

\begin{EntryWithPhonetic}{鲸}{jing1}{16}{⿂}
  \definition[头,只,条]{s.}{baleia; cetáceo}
\end{EntryWithPhonetic}

\begin{EntryWithPhonetic}{鲸鲨}{jing1sha1}{16,15}{⿂,⿂}
  \definition{s.}{tubarão baleia}
\end{EntryWithPhonetic}

\begin{EntryWithPhonetic}{鲸鱼}{jing1yu2}{16,8}{⿂,⿂}
  \definition{s.}{baleia}
\end{EntryWithPhonetic}

%%%%%%%%%% 井 %%%%%%%%%%
\subsection*{井}\addcontentsline{loh}{figure}{井 \dpy{jing3}}

\begin{EntryWithPhonetic}{井}{jing3}{4}{⼆}[HSK 6]
  \definition*{s.}{Jing, uma das mansões lunares | Sobrenome: Jing}
  \definition{adj.}{limpo; organizado}
  \definition[口]{s.}{poço; um buraco profundo cavado no chão para tirar água | algo em forma de poço | vila natal ou cidade natal}
\end{EntryWithPhonetic}

%%%%%%%%%% 颈 %%%%%%%%%%
\subsection*{颈}\addcontentsline{loh}{figure}{颈 \dpy{jing3}}

\begin{EntryWithPhonetic}{颈}{jing3}{11}{⾴}
  \definition{s.}{pescoço}
  \seeref{geng3}
\end{EntryWithPhonetic}

\begin{EntryWithPhonetic}{颈部}{jing3bu4}{11,10}{⾴,⾢}[HSK 7-9]
  \definition{s.}{pescoço}
\end{EntryWithPhonetic}

%%%%%%%%%% 景 %%%%%%%%%%
\subsection*{景}\addcontentsline{loh}{figure}{景 \dpy{jing3}}

\begin{EntryWithPhonetic}{景}{jing3}{12}{⽇}[HSK 6]
  \definition*{s.}{Sobrenome: Jing}
  \definition{adj.}{grandioso; elevado; grande}
  \definition{s.}{vista; cenário; cena | situação; condição | cenário (de uma peça ou filme) | cena (de uma peça)}
  \definition{v.}{admirar; reverenciar; respeitar}
\end{EntryWithPhonetic}

\begin{EntryWithPhonetic}{景点}{jing3dian3}{12,9}{⽇,⽕}[HSK 6]
  \definition[个,处]{s.}{local cênico; atração turística; um lugar onde se concentram as atrações turísticas, incluindo atrações naturais e culturais}
\end{EntryWithPhonetic}

\begin{EntryWithPhonetic}{景观}{jing3guan1}{12,6}{⽇,⾒}[HSK 7-9]
  \definition{s.}{cenário; paisagem; paisagem formada naturalmente; também se refere a paisagem criada artificialmente}
\end{EntryWithPhonetic}

\begin{EntryWithPhonetic}{景区}{jing3qu1}{12,4}{⽇,⼖}[HSK 7-9]
  \definition{s.}{ponto turístico | área cênica}
\end{EntryWithPhonetic}

\begin{EntryWithPhonetic}{景色}{jing3se4}{12,6}{⽇,⾊}[HSK 3]
  \definition[片,幅,道,处]{s.}{vista; cena; cenário; paisagem}
\end{EntryWithPhonetic}

\begin{EntryWithPhonetic}{景象}{jing3xiang4}{12,11}{⽇,⾗}[HSK 5]
  \definition[个,种]{s.}{cena; visão; vista; quadro}
\end{EntryWithPhonetic}

%%%%%%%%%% 警 %%%%%%%%%%
\subsection*{警}\addcontentsline{loh}{figure}{警 \dpy{jing3}}

\begin{EntryWithPhonetic}{警}{jing3}{19}{⾔}
  \definition{s.}{policial}
  \definition{v.}{alertar | avisar}
\end{EntryWithPhonetic}

\begin{EntryWithPhonetic}{警察}{jing3cha2}{19,14}{⾔,⼧}[HSK 3]
  \definition[个,位,名,群,队]{s.}{polícia; policial; oficial de polícia; as forças armadas que mantêm a segurança social do país são uma parte importante do aparato estatal; também se refere aos membros dessas forças armadas}
\end{EntryWithPhonetic}

\begin{EntryWithPhonetic}{警车}{jing3che1}{19,4}{⾔,⾞}[HSK 7-9]
  \definition{s.}{carro (ou van) da polícia}
\end{EntryWithPhonetic}

\begin{EntryWithPhonetic}{警告}{jing3gao4}{19,7}{⾔,⼝}[HSK 5]
  \definition[个]{s.}{advertência (como medida disciplinar); uma forma de punição}
  \definition{v.}{avisar; advertir; admoestar}
\end{EntryWithPhonetic}

\begin{EntryWithPhonetic}{警官}{jing3guan1}{19,8}{⾔,⼧}[HSK 7-9]
  \definition[名]{s.}{policial; guarda; agente; oficial de polícia}
\end{EntryWithPhonetic}

\begin{EntryWithPhonetic}{警惕}{jing3ti4}{19,11}{⾔,⼼}[HSK 7-9]
  \definition{v.}{estar vigilante; ficar atento; estar em alerta; estar em guarda contra; estar muito atento aos perigos potenciais ou tendências errôneas}
\end{EntryWithPhonetic}

\begin{EntryWithPhonetic}{警钟}{jing3zhong1}{19,9}{⾔,⾦}[HSK 7-9]
  \definition{s.}{sino de alarme; toque de alarme; sirene}
\end{EntryWithPhonetic}

%%%%%%%%%% 劲 %%%%%%%%%%
\subsection*{劲}\addcontentsline{loh}{figure}{劲 \dpy{jing4}}

\begin{EntryWithPhonetic}{劲}{jing4}{7}{⼒}
  \definition{adj.}{forte; poderoso; resistente; resoluto}
\end{EntryWithPhonetic}

%%%%%%%%%% 净 %%%%%%%%%%
\subsection*{净}\addcontentsline{loh}{figure}{净 \dpy{jing4}}

\begin{EntryWithPhonetic}{净}{jing4}{8}{⼎}[HSK 6]
  \definition{adj.}{limpo | (depois de um verbo) terminado; sem nada sobrando | líquido | vazio; oco; nu}
  \definition{adv.}{todo; o tempo todo | somente; meramente; nada além de | inteiramente; indica puro e nada mais}
  \definition{s.}{o ``rosto pintado'', comumente conhecido como Hualian (花脸), um tipo de personagem da ópera de Pequim, etc.}
  \definition{v.}{tornar limpo | limpar; lavar; esfregar para limpar}
  \seealsoref{花脸}{hua1lian3}
\end{EntryWithPhonetic}

\begin{EntryWithPhonetic}{净化}{jing4hua4}{8,4}{⼎,⼔}[HSK 7-9]
  \definition{v.}{purificar; remover impurezas para purificar o objeto}
\end{EntryWithPhonetic}

%%%%%%%%%% 经 %%%%%%%%%%
\subsection*{经}\addcontentsline{loh}{figure}{经 \dpy{jing4}}

\begin{EntryWithPhonetic}{经}{jing4}{8}{⽷}
  \definition{s.}{fio de urdidura na tecelagem}
  \seeref{jing1}
\end{EntryWithPhonetic}

%%%%%%%%%% 竞 %%%%%%%%%%
\subsection*{竞}\addcontentsline{loh}{figure}{竞 \dpy{jing4}}

\begin{EntryWithPhonetic}{竞}{jing4}{10}{⽴}
  \definition{adj.}{forte; poderoso}
  \definition{v.}{competir; contender; disputar | contestar}
\end{EntryWithPhonetic}

\begin{EntryWithPhonetic}{竞技}{jing4ji4}{10,7}{⽴,⼿}[HSK 7-9]
  \definition{s.}{atletismo; provas de atletismo; esportes; pista e campo}
  \definition{v.}{competir; desafiar; geralmente referindo"-se a competições atléticas}
\end{EntryWithPhonetic}

\begin{EntryWithPhonetic}{竞赛}{jing4sai4}{10,14}{⽴,⾙}[HSK 5]
  \definition[个]{s.}{concurso; competição; partida; corrida}
  \definition{v.}{correr; competir; competir uns com os outros por superioridade; em esportes, produção e outras atividades, para comparar competência, habilidade etc., usado principalmente na linguagem falada}
\end{EntryWithPhonetic}

\begin{EntryWithPhonetic}{竞相}{jing4xiang1}{10,9}{⽴,⽬}[HSK 7-9]
  \definition{adv.}{ansiosamente}
  \definition{s.}{competição}
  \definition{v.}{competir; disputar}
\end{EntryWithPhonetic}

\begin{EntryWithPhonetic}{竞选}{jing4xuan3}{10,9}{⽴,⾡}[HSK 7-9]
  \definition{s.}{eleição; campanha eleitoral}
  \definition{v.}{participar de uma disputa eleitoral; fazer campanha para (um cargo); candidatar-se a}
\end{EntryWithPhonetic}

\begin{EntryWithPhonetic}{竞争}{jing4zheng1}{10,6}{⽴,⼑}[HSK 5]
  \definition{v.}{competir; disputar; lutar; entre duas ou mais partes; em prol de seus próprios interesses; lutar pela vitória por meio de uma disputa de sua própria força contra outra}
\end{EntryWithPhonetic}

%%%%%%%%%% 竟 %%%%%%%%%%
\subsection*{竟}\addcontentsline{loh}{figure}{竟 \dpy{jing4}}

\begin{EntryWithPhonetic}{竟}{jing4}{11}{⾳}[HSK 7-9]
  \definition{adj.}{todo; por toda parte; do começo ao fim}
  \definition{adv.}{no final; eventualmente | na verdade; inesperadamente; significa algo inesperado, equivalente a 居然}
  \definition{v.}{terminar; completar | investigar}
  \seealsoref{居然}{ju1ran2}
\end{EntryWithPhonetic}

\begin{EntryWithPhonetic}{竟敢}{jing4gan3}{11,11}{⾳,⽁}[HSK 7-9]
  \definition{v.}{ousar de fato; ter a audácia; ter a impertinência | ousar}
\end{EntryWithPhonetic}

\begin{EntryWithPhonetic}{竟然}{jing4ran2}{11,12}{⾳,⽕}[HSK 4]
  \definition{adv.}{de fato; inesperadamente; para surpresa de alguém; chegar ao ponto de; indica que algo é um pouco inesperado}
\end{EntryWithPhonetic}

%%%%%%%%%% 敬 %%%%%%%%%%
\subsection*{敬}\addcontentsline{loh}{figure}{敬 \dpy{jing4}}

\begin{EntryWithPhonetic}{敬}{jing4}{12}{⽁}[HSK 7-9]
  \definition*{s.}{Sobrenome: Jing}
  \definition{adj.}{respeitoso; reverente}
  \definition{adv.}{respeitosamente}
  \definition{v.}{respeitar; honrar; estimar | oferecer educadamente | envolver-se em; dedicar-se a}
\end{EntryWithPhonetic}

\begin{EntryWithPhonetic}{敬爱}{jing4'ai4}{12,10}{⽁,⽖}[HSK 7-9]
  \definition{v.}{amar; estimar; acarinhar; respeitar e amar}
\end{EntryWithPhonetic}

\begin{EntryWithPhonetic}{敬而远之}{jing4'er2yuan3zhi1}{12,6,7,3}{⽁,⽽,⾡,⼂}[HSK 7-9]
  \definition{expr.}{``Respeite, mas mantenha distância.''; mantenha uma distância respeitosa de alguém; afaste-se de; demonstrar respeito à distância}
\end{EntryWithPhonetic}

\begin{EntryWithPhonetic}{敬酒}{jing4/jiu3}{12,10}{⽁,⾣}[HSK 7-9]
  \definition{v.+compl.}{brindar; propor um brinde; levantar seu copo respeitosamente para convidar a outra pessoa a beber}
\end{EntryWithPhonetic}

\begin{EntryWithPhonetic}{敬礼}{jing4/li3}{12,5}{⽁,⽰}[HSK 7-9]
  \definition{s.}{honoríficos, usados no final de uma carta}[结尾都是``此致敬礼''。===Todas elas terminam com ``Atenciosamente''.]
  \definition{v.+compl.}{saudar; prestar continência; demonstrar respeito através de gestos como olhar para alguém, levantar a mão, ficar em posição de sentido e fazer uma reverência}
\end{EntryWithPhonetic}

\begin{EntryWithPhonetic}{敬佩}{jing4pei4}{12,8}{⽁,⼈}[HSK 7-9]
  \definition{v.}{estimar; admirar; respeitar e admirar}
\end{EntryWithPhonetic}

\begin{EntryWithPhonetic}{敬请}{jing4qing3}{12,10}{⽁,⾔}[HSK 7-9]
  \definition{v.}{solicitar; convidar respeitosamente; termos educados usados para convidar ou pedir (que alguém faça algo)}
\end{EntryWithPhonetic}

\begin{EntryWithPhonetic}{敬业}{jing4ye4}{12,5}{⽁,⼀}[HSK 7-9]
  \definition{v.}{ser dedicado ao próprio trabalho; refere"-se a um espírito louvável, dedicado aos estudos ou ao trabalho}
\end{EntryWithPhonetic}

\begin{EntryWithPhonetic}{敬意}{jing4yi4}{12,13}{⽁,⼼}[HSK 7-9]
  \definition{s.}{respeito; tributo; homenagem}
\end{EntryWithPhonetic}

\begin{EntryWithPhonetic}{敬重}{jing4zhong4}{12,9}{⽁,⾥}[HSK 7-9]
  \definition{v.}{reverenciar; honrar; estimar; respeitar profundamente}
\end{EntryWithPhonetic}

%%%%%%%%%% 靓 %%%%%%%%%%
\subsection*{靓}\addcontentsline{loh}{figure}{靓 \dpy{jing4}}

\begin{EntryWithPhonetic}{靓}{jing4}{12}{⾭}
  \definition{v.}{(referindo"-se a vestimenta de alguém) ficar bonito | vestir"-se | maquiar (o rosto)}
  \seeref{liang4}
\end{EntryWithPhonetic}

%%%%%%%%%% 境 %%%%%%%%%%
\subsection*{境}\addcontentsline{loh}{figure}{境 \dpy{jing4}}

\begin{EntryWithPhonetic}{境}{jing4}{14}{⼟}
  \definition{s.}{fronteira; limite | lugar; área; território; região | condição; situação; circunstâncias}
\end{EntryWithPhonetic}

\begin{EntryWithPhonetic}{境地}{jing4di4}{14,6}{⼟,⼟}[HSK 7-9]
  \definition{s.}{situação; circunstâncias; as circunstâncias ou a situação encontradas (geralmente usadas em um sentido negativo) | reino; estado}
\end{EntryWithPhonetic}

\begin{EntryWithPhonetic}{境界}{jing4jie4}{14,9}{⼟,⽥}[HSK 7-9]
  \definition{s.}{limite; limites de terra | estado; nível; extensão alcançada; o grau em que algo é alcançado ou o estado em que se manifesta}
\end{EntryWithPhonetic}

\begin{EntryWithPhonetic}{境内}{jing4nei4}{14,4}{⼟,⼌}[HSK 7-9]
  \definition{s.}{área dentro das fronteiras | doméstico | interno (para um país, província, cidade etc.) | dentro das fronteiras}
  \antonymref{境外}{jing4wai4}
\end{EntryWithPhonetic}

\begin{EntryWithPhonetic}{境外}{jing4wai4}{14,5}{⼟,⼣}[HSK 7-9]
  \definition{s.}{área fora das fronteiras (ou do território) de um país; além das fronteiras de um país ou região}
\end{EntryWithPhonetic}

\begin{EntryWithPhonetic}{境遇}{jing4yu4}{14,12}{⼟,⾡}[HSK 7-9]
  \definition{s.}{a sorte de alguém; as circunstâncias; circunstâncias e encontros}
\end{EntryWithPhonetic}

%%%%%%%%%% 静 %%%%%%%%%%
\subsection*{静}\addcontentsline{loh}{figure}{静 \dpy{jing4}}

\begin{EntryWithPhonetic}{静}{jing4}{14}{⾭}[HSK 3]
  \definition*{s.}{Sobrenome: Jing}
  \definition{adj.}{tranquilo;  sossegado; calmo; imóvel | silencioso; quieto; sem emitir nenhum som | calmo, sereno; serenidade; (interior) paz}
  \definition{v.}{acalmar-se; aquietar-se; tranquilizar (o coração)}
\end{EntryWithPhonetic}

\begin{EntryWithPhonetic}{静止}{jing4zhi3}{14,4}{⾭,⽌}[HSK 7-9]
  \definition{adj.}{estático; imóvel; parado; estacionário}
\end{EntryWithPhonetic}

%%%%%%%%%% 镜 %%%%%%%%%%
\subsection*{镜}\addcontentsline{loh}{figure}{镜 \dpy{jing4}}

\begin{EntryWithPhonetic}{镜}{jing4}{16}{⾦}
  \definition*{s.}{Sobrenome: Jing}
  \definition[面,副]{s.}{espelho | lente; vidro; dispositivos para auxiliar a visão ou conduzir experimentos ópticos}
  \definition{v.}{espelhar | perceber | usar como referência}
\end{EntryWithPhonetic}

\begin{EntryWithPhonetic}{镜头}{jing4tou2}{16,5}{⾦,⼤}[HSK 4]
  \definition[个,组]{s.}{lente de câmera; objetiva; combinação de várias lentes, usada para formar uma imagem | foto; cena}
\end{EntryWithPhonetic}

\begin{EntryWithPhonetic}{镜子}{jing4zi5}{16,3}{⾦,⼦}[HSK 4]
  \definition[面,个]{s.}{espelho; instrumento de reflexão de imagem liso e plano, antigamente esmerilhado a partir de um disco grosso de cobre fundido, atualmente feito de vidro plano revestido de prata ou alumínio | óculos; óculos de grau}
\end{EntryWithPhonetic}

%%%%%%%%%% 窘 %%%%%%%%%%
\subsection*{窘}\addcontentsline{loh}{figure}{窘 \dpy{jiong3}}

\begin{EntryWithPhonetic}{窘}{jiong3}{12}{⽳}
  \definition{adj.}{em situação financeira precária; sem dinheiro; pobre | desajeitado; constrangido; desconfortável; difícil}
  \definition{v.}{constranger; desconcertar; dificultar as coisas}
\end{EntryWithPhonetic}

\begin{EntryWithPhonetic}{窘迫}{jiong3po4}{12,8}{⽳,⾡}[HSK 7-9]
  \definition{adj.}{muito pobre; miserável; extremamente ruim | envergonhado; pressionado; em apuros; descreve uma situação que causa constrangimento}
\end{EntryWithPhonetic}

%%%%%%%%%% 纠 %%%%%%%%%%
\subsection*{纠}\addcontentsline{loh}{figure}{纠 \dpy{jiu1}}

\begin{EntryWithPhonetic}{纠}{jiu1}{5}{⽷}
  \definition*{s.}{Sobrenome: Jiu}
  \definition{v.}{emaranhar | reunir-se | corrigir; retificar | supervisionar; superintender}
\end{EntryWithPhonetic}

\begin{EntryWithPhonetic}{纠缠}{jiu1chan2}{5,13}{⽷,⽷}[HSK 7-9]
  \definition{v.}{enredar-se; estar em apuros; entrelaçar | importunar; preocupar; perturbar; causar problemas}
\end{EntryWithPhonetic}

\begin{EntryWithPhonetic}{纠纷}{jiu1fen1}{5,7}{⽷,⽷}[HSK 6]
  \definition[个,次]{s.}{questão; disputa; existem contradições ou conflitos de interesse entre as duas partes que precisam ser resolvidos}
\end{EntryWithPhonetic}

\begin{EntryWithPhonetic}{纠葛}{jiu1ge2}{5,12}{⽷,⾋}
  \definition{s.}{emaranhado | disputa}
\end{EntryWithPhonetic}

\begin{EntryWithPhonetic}{纠正}{jiu1zheng4}{5,5}{⽷,⽌}[HSK 6]
  \definition{v.}{fazer certo; corrigir (deficiências ou erros em pensamentos, ações, métodos, etc.)}
\end{EntryWithPhonetic}

%%%%%%%%%% 究 %%%%%%%%%%
\subsection*{究}\addcontentsline{loh}{figure}{究 \dpy{jiu1}}

\begin{EntryWithPhonetic}{究}{jiu1}{7}{⽳}
  \definition{adv.}{na verdade; realmente; afinal}
  \definition{v.}{estudar cuidadosamente; aprofundar; investigar; rastrear}
\end{EntryWithPhonetic}

\begin{EntryWithPhonetic}{究竟}{jiu1jing4}{7,11}{⽳,⾳}[HSK 4]
  \definition{adv.}{de fato; exatamente; usado em frases interrogativas para buscar | afinal de contas, no final; ênfase em fatos ou motivos}
  \definition{s.}{resultado; desfecho; a causa, o efeito ou a história completa do que aconteceu}
\end{EntryWithPhonetic}

%%%%%%%%%% 揪 %%%%%%%%%%
\subsection*{揪}\addcontentsline{loh}{figure}{揪 \dpy{jiu1}}

\begin{EntryWithPhonetic}{揪}{jiu1}{12}{⼿}[HSK 7-9]
  \definition{v.}{segurar com firmeza; agarrar e puxar}
\end{EntryWithPhonetic}

%%%%%%%%%% 九 %%%%%%%%%%
\subsection*{九}\addcontentsline{loh}{figure}{九 \dpy{jiu3}}

\begin{EntryWithPhonetic}{九}{jiu3}{2}{⼄}[HSK 1]
  \definition*{s.}{Sobrenome: Jiu}
  \definition{adj.}{muitos; numerosos; indica várias vezes ou a maioria das vezes}
  \definition{num.}{nove; 9}
  \definition{s.}{cada um dos nove períodos de nove dias começando no dia seguinte ao solstício de inverno}
\end{EntryWithPhonetic}

%%%%%%%%%% 久 %%%%%%%%%%
\subsection*{久}\addcontentsline{loh}{figure}{久 \dpy{jiu3}}

\begin{EntryWithPhonetic}{久}{jiu3}{3}{⼃}[HSK 3]
  \definition{adj.}{por muito tempo; longo período de tempo | duração de tempo especificada}
  \antonymref{暂}{zan4}
\end{EntryWithPhonetic}

\begin{EntryWithPhonetic}{久违}{jiu3wei2}{3,7}{⼃,⾡}[HSK 7-9]
  \definition{v.}{não ver há muito tempo; fazer muito tempo desde o último encontro; apenas um comentário educado, muito tempo sem ver}
  \synonymref{重逢}{chong2feng2}
  \antonymref{经常}{jing1chang2}
\end{EntryWithPhonetic}

\begin{EntryWithPhonetic}{久仰}{jiu3yang3}{3,6}{⼃,⼈}[HSK 7-9]
  \definition{expr.}{``É um prazer conhecê-lo(a).''; ``Há muito tempo que desejo conhecê-lo(a).''; ``Há muito tempo que aguardo ansiosamente o nosso encontro.''; ``Já ouvi falar muito de você.''}
\end{EntryWithPhonetic}

%%%%%%%%%% 韭 %%%%%%%%%%
\subsection*{韭}\addcontentsline{loh}{figure}{韭 \dpy{jiu3}}

\begin{EntryWithPhonetic}{韭}{jiu3}{9}{⾲}[Kangxi 179]
  \definition{s.}{alho de flor perfumada; cebolinha chinesa}
\end{EntryWithPhonetic}

\begin{EntryWithPhonetic}{韭菜}{jiu3cai4}{9,11}{⾲,⾋}
  \definition{s.}{cebolinha-de-alho; cebolinha chinesa | Coloquial: pessoas ingênuas ou facilmente exploráveis, especialmente pequenos investidores ou consumidores, comparadas à cebolinha, que pode ser colhida repetidamente para obter lucro}
\end{EntryWithPhonetic}

%%%%%%%%%% 酒 %%%%%%%%%%
\subsection*{酒}\addcontentsline{loh}{figure}{酒 \dpy{jiu3}}

\begin{EntryWithPhonetic}{酒}{jiu3}{10}{⾣}[HSK 2]
  \definition*{s.}{Sobrenome: Jiu}
  \definition[口,杯,瓶,罐,桶,缸]{s.}{bebida alcoólica; vinho; licor; bebidas destiladas}
\end{EntryWithPhonetic}

\begin{EntryWithPhonetic}{酒吧}{jiu3ba1}{10,7}{⾣,⼝}[HSK 4]
  \definition[家,间]{s.}{bar; \emph{pub}; um local onde são vendidas bebidas alcoólicas e onde as pessoas podem beber e conversar, referindo"-se principalmente a um restaurante ou hotel de estilo ocidental especializado na venda de bebidas alcoólicas}
\end{EntryWithPhonetic}

\begin{EntryWithPhonetic}{酒店}{jiu3dian4}{10,8}{⾣,⼴}[HSK 2]
  \definition[家,个]{s.}{hotel; Estabelecimento comercial que oferece hospedagem e alimentação aos hóspedes | restaurante}
\end{EntryWithPhonetic}

\begin{EntryWithPhonetic}{酒馆}{jiu3guan3}{10,11}{⾣,⾷}
  \definition{s.}{bar | taverna | adega}
\end{EntryWithPhonetic}

\begin{EntryWithPhonetic}{酒鬼}{jiu3gui3}{10,9}{⾣,⿁}[HSK 5]
  \definition[个]{s.}{bebedor de vinho; beberrão; ébrio | alcoólatra}
\end{EntryWithPhonetic}

\begin{EntryWithPhonetic}{酒精}{jiu3jing1}{10,14}{⾣,⽶}[HSK 7-9]
  \definition{s.}{álcool; álcool etílico; etanol}
\end{EntryWithPhonetic}

\begin{EntryWithPhonetic}{酒楼}{jiu3lou2}{10,13}{⾣,⽊}[HSK 7-9]
  \definition[座,家]{s.}{restaurante (em nomes de restaurantes)}[广东酒楼===Restaurante Guangdong]
\end{EntryWithPhonetic}

\begin{EntryWithPhonetic}{酒水}{jiu3shui3}{10,4}{⾣,⽔}[HSK 6]
  \definition{s.}{bebidas; bebidas e álcool | Dialeto: festa; banquete}
\end{EntryWithPhonetic}

%%%%%%%%%% 旧 %%%%%%%%%%
\subsection*{旧}\addcontentsline{loh}{figure}{旧 \dpy{jiu4}}

\begin{EntryWithPhonetic}{旧}{jiu4}{5}{⽇}[HSK 3]
  \definition{adj.}{passado; antigo; velho; ultrapassado | usado; desgastado; velho; descolorido ou deformado devido ao uso prolongado ou ao tempo | antigo; único; que já existiu; anterior}
  \definition{s.}{velha amizade; velho amigo}
  \antonymref{新}{xin1}
\end{EntryWithPhonetic}

%%%%%%%%%% 救 %%%%%%%%%%
\subsection*{救}\addcontentsline{loh}{figure}{救 \dpy{jiu4}}

\begin{EntryWithPhonetic}{救}{jiu4}{11}{⽁}[HSK 3]
  \definition*{s.}{Sobrenome: Jiu}
  \definition{v.}{resgatar; salvar | salvar de; aliviar (angústia, etc.) | resgatar; livrar alguém de um desastre ou perigo | ajudar; aliviar; socorrer; livrar pessoas e coisas de desastres e perigos}
\end{EntryWithPhonetic}

\begin{EntryWithPhonetic}{救出}{jiu4chu1}{11,5}{⽁,⼐}
  \definition{v.}{resgatar | tirar do perigo}
\end{EntryWithPhonetic}

\begin{EntryWithPhonetic}{救护车}{jiu4hu4che1}{11,7,4}{⽁,⼿,⾞}[HSK 7-9]
  \definition[辆]{s.}{ambulância; os veículos que transportam os feridos estão equipados com instalações que permitem à equipe médica prestar primeiros socorros temporários, cuidados médicos e serviços de enfermagem aos feridos}
\end{EntryWithPhonetic}

\begin{EntryWithPhonetic}{救济}{jiu4ji4}{11,9}{⽁,⽔}[HSK 7-9]
  \definition{v.}{fornecer ajuda com dinheiro ou bens; usar dinheiro e bens para ajudar vítimas de desastres ou outras pessoas que vivem em situação de pobreza}
\end{EntryWithPhonetic}

\begin{EntryWithPhonetic}{救命}{jiu4/ming4}{11,8}{⽁,⼝}[HSK 6]
  \definition{interj.}{``Socorro!''; ``Salve-me!''}
  \definition{v.+compl.}{ajudar; salvar a vida de alguém}
\end{EntryWithPhonetic}

\begin{EntryWithPhonetic}{救援}{jiu4yuan2}{11,12}{⽁,⼿}[HSK 6]
  \definition{v.}{resgatar; socorrer; vir em auxílio de alguém (resgate)}
\end{EntryWithPhonetic}

\begin{EntryWithPhonetic}{救灾}{jiu4 zai1}{11,7}{⽁,⽕}[HSK 5]
  \definition{v.}{ajudar as vítimas de desastres, aliviar o desastre; resgatar pessoas afetadas por desastres; recuperar danos causados por desastres}
\end{EntryWithPhonetic}

\begin{EntryWithPhonetic}{救治}{jiu4zhi4}{11,8}{⽁,⽔}[HSK 7-9]
  \definition{v.}{retirar um paciente do perigo; tratar e curar}
\end{EntryWithPhonetic}

\begin{EntryWithPhonetic}{救助}{jiu4zhu4}{11,7}{⽁,⼒}[HSK 6]
  \definition{v.}{ajudar alguém em perigo ou dificuldade; socorrer; resgatar e ajudar}
\end{EntryWithPhonetic}

%%%%%%%%%% 就 %%%%%%%%%%
\subsection*{就}\addcontentsline{loh}{figure}{就 \dpy{jiu4}}

\begin{EntryWithPhonetic}{就}{jiu4}{12}{⼪}[HSK 1]
  \definition{adv.}{de imediato; imediatamente; indica que algo ocorrerá em breve | tão cedo quanto; já; há muito tempo; indica que a ação ocorreu há muito tempo | assim que; logo depois; indica que os eventos se sucedem imediatamente | nesse caso; então; indica que, sob determinadas condições, ocorre naturalmente um determinado resultado | exatamente; precisamente; indica que é exatamente assim | apenas; meramente; somente | tantos quanto; enfatiza a quantidade | apenas; simplesmente; reforço da afirmação | colocado entre dois componentes idênticos, significa tolerância ou indiferença}
  \definition{prep.}{tirar proveito de alguém (algo); expressa condições, oportunidades, etc., equivalente a 趁 | quando se trata de alguém (algo); relativo a; com relação a; sobre; objeto ou escopo da introdução da ação |no local; introduz o local próximo ao qual a ação ocorreu}
  \definition{v.}{ser comido com; ir com; pratos, frutas, etc., acompanhados de alimentos básicos ou bebidas alcoólicas | aproximar-se; mover-se em direção a | ir para; assumir; empreender; envolver-se em; entrar em | realizar; fazer | tirar proveito de; acomodar-se a; adequar-se; encaixar-se | assumir; começar a entrar ou a exercer | seguir; acompanhar}
  \seealsoref{趁}{chen4}
\end{EntryWithPhonetic}

\begin{EntryWithPhonetic}{就餐}{jiu4can1}{12,16}{⼪,⾷}[HSK 7-9]
  \definition{v.}{comer; jantar; fazer uma refeição; ir comer}
\end{EntryWithPhonetic}

\begin{EntryWithPhonetic}{就地}{jiu4di4}{12,6}{⼪,⼟}[HSK 7-9]
  \definition{adv.}{no local; no próprio local | localmente}
\end{EntryWithPhonetic}

\begin{EntryWithPhonetic}{就读}{jiu4du2}{12,10}{⼪,⾔}[HSK 7-9]
  \definition{v.}{fazer um curso; frequentar a escola; ir à escola; ingressar na escola}
\end{EntryWithPhonetic}

\begin{EntryWithPhonetic}{就近}{jiu4jin4}{12,7}{⼪,⾡}[HSK 7-9]
  \definition{adv.}{(fazer ou obter algo) nas proximidades; na vizinhança; sem ter que ir longe; significa que está por perto}
\end{EntryWithPhonetic}

\begin{EntryWithPhonetic}{就可以了}{jiu4 ke3yi3le5}{12,5,4,2}{⼪,⼝,⼈,⼅}
  \definition{expr.}{é isso; é o suficiente}
\end{EntryWithPhonetic}

\begin{EntryWithPhonetic}{就任}{jiu4ren4}{12,6}{⼪,⼈}[HSK 7-9]
  \definition{v.}{assumir o cargo; tomar posse | assumir o próprio cargo}
\end{EntryWithPhonetic}

\begin{EntryWithPhonetic}{就是}{jiu4shi4}{12,9}{⼪,⽇}[HSK 3]
  \definition{adv.}{exatamente; precisamente; expressar concordância com a afirmação da outra pessoa ou confirmar que a afirmação da outra pessoa está correta | apenas; simplesmente; expressa afirmação, determinação ou ênfase, o significado específico deve ser determinado com base no contexto anterior ou posterior | usado para indicar escolha}
  \definition{conj.}{ainda que; mesmo que se reconheça que essa situação é verdadeira, a situação posterior não mudará}
  \definition{part.}{usado no final de uma frase para expressar afirmação}
\end{EntryWithPhonetic}

\begin{EntryWithPhonetic}{就是说}{jiu4shi4shuo1}{12,9,9}{⼪,⽇,⾔}[HSK 6]
  \definition{expr.}{ou seja; isto é; em outras palavras; é frequentemente usado como uma interjeição em uma frase para indicar que as palavras seguintes são uma explicação ou esclarecimento das anteriores}
\end{EntryWithPhonetic}

\begin{EntryWithPhonetic}{就算}{jiu4suan4}{12,14}{⼪,⽵}[HSK 6]
  \definition{conj.}{mesmo que; concedido que; expressam uma relação hipotética e concessiva, frequentemente usadas com 也, equivalente a 即使}
  \seealsoref{即使}{ji2shi3}
  \seealsoref{也}{ye3}
\end{EntryWithPhonetic}

\begin{EntryWithPhonetic}{就要}{jiu4yao4}{12,9}{⼪,⾑}[HSK 2]
  \definition{adv.}{estar prestes a; estar indo para; estar no ponto de}
\end{EntryWithPhonetic}

\begin{EntryWithPhonetic}{就业}{jiu4/ye4}{12,5}{⼪,⼀}[HSK 3]
  \definition{v.+compl.}{conseguir um emprego; obter emprego; assumir uma ocupação; começar a trabalhar}
\end{EntryWithPhonetic}

\begin{EntryWithPhonetic}{就医}{jiu4/yi1}{12,7}{⼪,⼖}[HSK 7-9]
  \definition{v.+compl.}{consultar um médico; ir ao médico; buscar aconselhamento médico}
\end{EntryWithPhonetic}

\begin{EntryWithPhonetic}{就诊}{jiu4/zhen3}{12,7}{⼪,⾔}[HSK 7-9]
  \definition{v.+compl.}{consultar um médico; procurar aconselhamento médico}
\end{EntryWithPhonetic}

\begin{EntryWithPhonetic}{就职}{jiu4/zhi2}{12,11}{⼪,⽿}
  \definition{v.+compl.}{assumir o cargo; assumir oficialmente um cargo (geralmente referindo"-se a uma posição de maior hierarquia)}
\end{EntryWithPhonetic}

\begin{EntryWithPhonetic}{就坐}{jiu4zuo4}{12,7}{⼪,⼟}
  \definition{v.}{sentar-se; estar sentado}
\end{EntryWithPhonetic}

\begin{EntryWithPhonetic}{就座}{jiu4/zuo4}{12,10}{⼪,⼴}[HSK 7-9]
  \definition{v.+compl.}{sentar-se; estar sentado | ocupar o próprio lugar (assento)}
  \seealsoref{就坐}{jiu4zuo4}
\end{EntryWithPhonetic}

%%%%%%%%%% 舅 %%%%%%%%%%
\subsection*{舅}\addcontentsline{loh}{figure}{舅 \dpy{jiu4}}

\begin{EntryWithPhonetic}{舅}{jiu4}{13}{⾅}
  \definition*{s.}{Sobrenome: Jiu}
  \definition[个,位,名,些]{s.}{irmão da mãe; tio materno | irmão da esposa; cunhado | Literário: pai do marido; sogro}
\end{EntryWithPhonetic}

\begin{EntryWithPhonetic}{舅舅}{jiu4jiu5}{13,13}{⾅,⾅}[HSK 7-9]
  \definition[个,位,名]{s.}{tio; irmão da mãe; é assim que você se dirige ao irmão mais velho ou mais novo de sua mãe; você também pode chamá-lo de 舅}
  \seealsoref{舅}{jiu4}
\end{EntryWithPhonetic}

%%%%%%%%%% 车 %%%%%%%%%%
\subsection*{车}\addcontentsline{loh}{figure}{车 \dpy{ju1}}

\begin{EntryWithPhonetic}{车}{ju1}{4}{⾞}[Kangxi 159]
  \definition{s.}{torre; castelo; carruagem, uma das peças do xadrez chinês}
  \seeref{che1}
\end{EntryWithPhonetic}

%%%%%%%%%% 居 %%%%%%%%%%
\subsection*{居}\addcontentsline{loh}{figure}{居 \dpy{ju1}}

\begin{EntryWithPhonetic}{居}{ju1}{8}{⼫}
  \definition*{s.}{Sobrenome: Ju}
  \definition{s.}{residência; casa | restaurante (em nomes de restaurantes)}
  \definition{v.}{residir; morar; viver | ocupar uma determinada posição; ocupar (um lugar); estar (em uma determinada posição) | reivindicar; afirmar | armazenar; guardar | ficar parado; estar parado}
\end{EntryWithPhonetic}

\begin{EntryWithPhonetic}{居高临下}{ju1gao1-lin2xia4}{8,10,9,3}{⼫,⾼,⼁,⼀}[HSK 7-9]
  \definition{expr.}{``Olhando para baixo.''; ocupar uma posição (ou altura) dominante; olhar de cima para baixo; viver no alto e olhar para baixo; ocupar o terreno elevado; ignorar; elevar-se acima | Figurativo: arrogância baseada na posição social de alguém}
\end{EntryWithPhonetic}

\begin{EntryWithPhonetic}{居民}{ju1min2}{8,5}{⼫,⽒}[HSK 4]
  \definition[个,户,位]{s.}{residente; habitante; pessoas que estão fixas em um único lugar}
\end{EntryWithPhonetic}

\begin{EntryWithPhonetic}{居民楼}{ju1min2lou2}{8,5,13}{⼫,⽒,⽊}[HSK 7-9]
  \definition{s.}{edifício residencial}
\end{EntryWithPhonetic}

\begin{EntryWithPhonetic}{居然}{ju1ran2}{8,12}{⼫,⽕}[HSK 5]
  \definition{adv.}{inesperadamente; para surpresa de alguém; além da expectativa (expressão idiomática)}
  \definition{v.}{ir tão longe a ponto de; ter a impudência de; ter o descaramento de}
\end{EntryWithPhonetic}

\begin{EntryWithPhonetic}{居住}{ju1zhu4}{8,7}{⼫,⼈}[HSK 4]
  \definition{v.}{viver; residir; morar; habitar}
\end{EntryWithPhonetic}

%%%%%%%%%% 拘 %%%%%%%%%%
\subsection*{拘}\addcontentsline{loh}{figure}{拘 \dpy{ju1}}

\begin{EntryWithPhonetic}{拘}{ju1}{8}{⼿}
  \definition{adj.}{inflexível; sem flexibilidade}
  \definition{v.}{prender; deter | restringir; limitar; constranger | aderir rigidamente; ser inflexível |limitar}
\end{EntryWithPhonetic}

\begin{EntryWithPhonetic}{拘留}{ju1liu2}{8,10}{⼿,⽥}[HSK 7-9]
  \definition{v.}{deter; manter sob custódia; colocar em prisão provisória; restringir a liberdade pessoal}
\end{EntryWithPhonetic}

\begin{EntryWithPhonetic}{拘束}{ju1shu4}{8,7}{⼿,⽊}[HSK 7-9]
  \definition{adj.}{desajeitado; desconfortável; constrangido; reservado; não natural}
  \definition{v.}{restringir; limitar; restringir excessivamente as palavras e ações de outras pessoas}
\end{EntryWithPhonetic}

%%%%%%%%%% 据 %%%%%%%%%%
\subsection*{据}\addcontentsline{loh}{figure}{据 \dpy{ju1}}

\begin{EntryWithPhonetic}{据}{ju1}{11}{⼿}
  \definition{part.}{elemento formador de palavras}
  \seeref{ju4}
  \seealsoref{拮据}{jie2ju1}
\end{EntryWithPhonetic}

%%%%%%%%%% 鞠 %%%%%%%%%%
\subsection*{鞠}\addcontentsline{loh}{figure}{鞠 \dpy{ju1}}

\begin{EntryWithPhonetic}{鞠}{ju1}{17}{⾰}
  \definition*{s.}{Sobrenome: Ju}
  \definition{s.}{arco | Arcaico: uma bola de futebol; um tipo antigo de bola; bola usada para jogar}
  \definition{v.}{Literátio: criar; educar; nutrir | obrar; curvar}
\end{EntryWithPhonetic}

\begin{EntryWithPhonetic}{鞠躬}{ju1/gong1}{17,10}{⾰,⾝}[HSK 7-9]
  \definition{v.+compl.}{curvar"-se; inclinar"-se}
\end{EntryWithPhonetic}

%%%%%%%%%% 局 %%%%%%%%%%
\subsection*{局}\addcontentsline{loh}{figure}{局 \dpy{ju2}}

\begin{EntryWithPhonetic}{局}{ju2}{7}{⼫}[HSK 4,6]
  \definition{adj.}{limitado; confinado}
  \definition{clas.}{\emph{set}; jogo; turno}
  \definition{s.}{tabuleiro de xadrez | situação; estado de coisas | generosidade de espírito; extensão da tolerância de alguém | festa; reunião; refere"-se a certas reuniões | ardil; armadilha | parte; porção; papel | escritório; agência; agências governamentais divididas por negócios | significa ``loja'' em nomes de lojas | departamento; agência; nomes de certas entidades empresariais | escritório; usado como nome de uma instituição ou outro local de negócios}
\end{EntryWithPhonetic}

\begin{EntryWithPhonetic}{局部}{ju2bu4}{7,10}{⼫,⾢}[HSK 7-9]
  \definition{s.}{parte; uma parte; não o todo}
\end{EntryWithPhonetic}

\begin{EntryWithPhonetic}{局面}{ju2mian4}{7,9}{⼫,⾯}[HSK 5]
  \definition[种]{s.}{aspecto; fase; situação; o estado das coisas em um período de tempo, em sua maior parte abstraído | escopo; escala}
\end{EntryWithPhonetic}

\begin{EntryWithPhonetic}{局势}{ju2shi4}{7,8}{⼫,⼒}[HSK 7-9]
  \definition{s.}{situação; estado (de coisas); (político, militar, etc.) desenvolvimentos ao longo de um período de tempo}
\end{EntryWithPhonetic}

\begin{EntryWithPhonetic}{局限}{ju2xian4}{7,8}{⼫,⾩}[HSK 7-9]
  \definition{v.}{limitar; confinar; limitar a um determinado intervalo}
\end{EntryWithPhonetic}

\begin{EntryWithPhonetic}{局长}{ju2zhang3}{7,4}{⼫,⾧}[HSK 5]
  \definition[位,名,个,些]{s.}{comissário; diretor; principais chefes de gabinete do governo}
\end{EntryWithPhonetic}

%%%%%%%%%% 菊 %%%%%%%%%%
\subsection*{菊}\addcontentsline{loh}{figure}{菊 \dpy{ju2}}

\begin{EntryWithPhonetic}{菊}{ju2}{11}{⾋}
  \definition*{s.}{Sobrenome: Ju}
  \definition[朵]{s.}{crisântemo}
\end{EntryWithPhonetic}

\begin{EntryWithPhonetic}{菊花}{ju2hua1}{11,7}{⾋,⾋}[HSK 7-9]
  \definition[朵,枝,把,束,棵,株]{s.}{crisântemo | Gíria: ânus; uma metáfora para o ânus (ou reto) humano}
\end{EntryWithPhonetic}

%%%%%%%%%% 橘 %%%%%%%%%%
\subsection*{橘}\addcontentsline{loh}{figure}{橘 \dpy{ju2}}

\begin{EntryWithPhonetic}{橘}{ju2}{16}{⽊}
  \definition[只,棵]{s.}{tangerina}
\end{EntryWithPhonetic}

\begin{EntryWithPhonetic}{橘子}{ju2zi5}{16,3}{⽊,⼦}[HSK 7-9]
  \definition[个,堆,箱,筐]{s.}{tangerina; laranja mandarina}
\end{EntryWithPhonetic}

\begin{EntryWithPhonetic}{橘子汁}{ju2zi5zhi1}{16,3,5}{⽊,⼦,⽔}
  \definition[瓶,杯,罐,盒]{s.}{suco de laranja}
  \seealsoref{橙汁}{cheng2zhi1}
  \seealsoref{柳橙汁}{liu3cheng2zhi1}
\end{EntryWithPhonetic}

%%%%%%%%%% 咀 %%%%%%%%%%
\subsection*{咀}\addcontentsline{loh}{figure}{咀 \dpy{ju3}}

\begin{EntryWithPhonetic}{咀}{ju3}{8}{⼝}
  \definition{v.}{mastigar; mascar}
  \seeref{zui3}
\end{EntryWithPhonetic}

\begin{EntryWithPhonetic}{咀嚼}{ju3jue2}{8,20}{⼝,⼝}
  \definition{v.}{mastigar; triturar alimentos com os dentes | refletir sobre; essa metáfora descreve o ato de ponderar e compreender algo repetidamente}
\end{EntryWithPhonetic}

%%%%%%%%%% 柜 %%%%%%%%%%
\subsection*{柜}\addcontentsline{loh}{figure}{柜 \dpy{ju3}}

\begin{EntryWithPhonetic}{柜}{ju3}{8}{⽊}
  \definition{s.}{faia; salgueiro}
  \seeref{gui4}
\end{EntryWithPhonetic}

%%%%%%%%%% 沮 %%%%%%%%%%
\subsection*{沮}\addcontentsline{loh}{figure}{沮 \dpy{ju3}}

\begin{EntryWithPhonetic}{沮}{ju3}{8}{⽔}
  \definition{v.}{Literário: parar; prevenir; evitar | ficar melancólico; ficar triste}
  \seeref{ju4}
\end{EntryWithPhonetic}

\begin{EntryWithPhonetic}{沮丧}{ju3sang4}{8,8}{⽔,⼗}[HSK 7-9]
  \definition{adj.}{desanimado; abatido; decepcionado}
\end{EntryWithPhonetic}

%%%%%%%%%% 举 %%%%%%%%%%
\subsection*{举}\addcontentsline{loh}{figure}{举 \dpy{ju3}}

\begin{EntryWithPhonetic}{举}{ju3}{9}{⼂}[HSK 2]
  \definition*{s.}{Sobrenome: Ju}
  \definition{adj.}{inteiro; completo}
  \definition{s.}{ato; ação; movimento; comportamento | (nas dinastias Ming e Qing) candidato aprovado nos exames imperiais a nível provincial}
  \definition{v.}{levantar; erguer; sustentar | começar; iniciar; surgir | eleger; escolher; recomendar; selecionar | citar; enumerar; propor; revelar}
\end{EntryWithPhonetic}

\begin{EntryWithPhonetic}{举办}{ju3ban4}{9,4}{⼂,⼒}[HSK 3]
  \definition{v.}{conduzir; organizar; realizar}
  \antonymref{进行}{jin4xing2}
  \antonymref{举行}{ju3xing2}
  \antonymref{召开}{zhao4kai1}
\end{EntryWithPhonetic}

\begin{EntryWithPhonetic}{举报}{ju3bao4}{9,7}{⼂,⼿}[HSK 7-9]
  \definition{v.}{relatar; denunciar}[我决定举报不法行为。===Decidi denunciar as atividades ilegais.]
  \synonymref{揭发}{jie1fa1}
  \synonymref{投诉}{tou2su4}
\end{EntryWithPhonetic}

\begin{EntryWithPhonetic}{举措}{ju3cuo4}{9,11}{⼂,⼿}[HSK 7-9]
  \definition{v.}{mover; agir; medir}
  \synonymref{办法}{ban4fa3}
  \synonymref{步骤}{bu4zhou4}
  \synonymref{措施}{cuo4shi1}
  \synonymref{动作}{dong4zuo4}
  \synonymref{方法}{fang1fa3}
  \synonymref{举动}{ju3dong4}
  \synonymref{设施}{she4shi1}
  \synonymref{行动}{xing2dong4}
\end{EntryWithPhonetic}

\begin{EntryWithPhonetic}{举动}{ju3dong4}{9,6}{⼂,⼒}[HSK 5]
  \definition{s.}{ato; atividade; movimento; ação}
  \synonymref{举措}{ju3cuo4}
  \synonymref{举止}{ju3zhi3}
  \synonymref{行为}{xing2wei2}
  \synonymref{作为}{zuo4wei2}
  \antonymref{内心}{nei4xin1}
\end{EntryWithPhonetic}

\begin{EntryWithPhonetic}{举例}{ju3/li4}{9,8}{⼂,⼈}[HSK 7-9]
  \definition{v.+compl.}{dar um exemplo; citar um caso}
  \synonymref{例如}{li4ru2}
\end{EntryWithPhonetic}

\begin{EntryWithPhonetic}{举世闻名}{ju3shi4-wen2ming2}{9,5,9,6}{⼂,⼀,⾨,⼝}[HSK 7-9]
  \definition{expr.}{``De renome mundial.''; mundialmente famoso}
  \synonymref{大名鼎鼎}{da4ming2-ding3ding3}
  \synonymref{举世瞩目}{ju3shi4-zhu3mu4}
  \antonymref{不为人知}{bu4wei2ren2zhi1}
  \antonymref{默默无闻}{mo4mo4-wu2wen2}
\end{EntryWithPhonetic}

\begin{EntryWithPhonetic}{举世无双}{ju3shi4-wu2shuang1}{9,5,4,4}{⼂,⼀,⽆,⼜}[HSK 7-9]
  \definition{expr.}{``Inigualável no mundo.''; inigualável; incomparável; sem igual; único; número um do mundo}
  \synonymref{独一无二}{du2yi1-wu2'er4}
\end{EntryWithPhonetic}

\begin{EntryWithPhonetic}{举世瞩目}{ju3shi4-zhu3mu4}{9,5,17,5}{⼂,⼀,⽬,⽬}[HSK 7-9]
  \definition{expr.}{``De renome mundial.''; atrair a atenção mundial; tornar-se o centro das atenções mundiais; o mundo inteiro está assistindo}
  \synonymref{举世闻名}{ju3shi4-wen2ming2}
  \antonymref{默默无闻}{mo4mo4-wu2wen2}
\end{EntryWithPhonetic}

\begin{EntryWithPhonetic}{举手}{ju3 shou3}{9,4}{⼂,⼿}[HSK 2]
  \definition{v.}{levantar a mão ou as mãos; levantar a mão para sinalizar ou responder a uma pergunta}
\end{EntryWithPhonetic}

\begin{EntryWithPhonetic}{举行}{ju3xing2}{9,6}{⼂,⾏}[HSK 2]
  \definition{v.}{realizar (uma reunião, cerimônia, etc.); realizar (atividades formais ou solenes)}
  \synonymref{进行}{jin4xing2}
  \synonymref{举办}{ju3ban4}
  \synonymref{实行}{shi2xing2}
  \synonymref{召开}{zhao4kai1}
  \antonymref{保留}{bao3liu2}
  \antonymref{取消}{qu3xiao1}
\end{EntryWithPhonetic}

\begin{EntryWithPhonetic}{举一反三}{ju3yi1-fan3san1}{9,1,4,3}{⼂,⼀,⼜,⼀}[HSK 7-9]
  \definition{expr.}{aprender por analogia; inferir outras coisas a partir de um fato; aprender muitas coisas por analogia a partir de uma única coisa}
\end{EntryWithPhonetic}

\begin{EntryWithPhonetic}{举止}{ju3zhi3}{9,4}{⼂,⽌}[HSK 7-9]
  \definition{s.}{maneira; comportamento; porte; postura; refere"-se à postura e ao comportamento}
  \synonymref{举动}{ju3dong4}
  \synonymref{行动}{xing2dong4}
  \synonymref{行为}{xing2wei2}
\end{EntryWithPhonetic}

\begin{EntryWithPhonetic}{举重}{ju3zhong4}{9,9}{⼂,⾥}[HSK 7-9]
  \definition{s.}{levantamento de peso}
  \definition{v.}{levantar pesos}
\end{EntryWithPhonetic}

%%%%%%%%%% 巨 %%%%%%%%%%
\subsection*{巨}\addcontentsline{loh}{figure}{巨 \dpy{ju4}}

\begin{EntryWithPhonetic}{巨}{ju4}{4}{⼯}
  \definition*{s.}{Sobrenome: Ju}
  \definition{adj.}{enorme; tremendo; gigantesco}
\end{EntryWithPhonetic}

\begin{EntryWithPhonetic}{巨大}{ju4da4}{4,3}{⼯,⼤}[HSK 4]
  \definition{adj.}{enorme; tremendo; enorme; gigantesco; imenso}
\end{EntryWithPhonetic}

\begin{EntryWithPhonetic}{巨额}{ju4'e2}{4,15}{⼯,⾴}[HSK 7-9]
  \definition{adj.}{enorme; imenso; soma gigantesca; grande quantidade; (dinheiro, etc.) uma grande quantia}
\end{EntryWithPhonetic}

\begin{EntryWithPhonetic}{巨人}{ju4ren2}{4,2}{⼯,⼈}[HSK 7-9]
  \definition{s.}{gigante; colosso | \emph{homo monstrosus}; arranha-céu}
\end{EntryWithPhonetic}

\begin{EntryWithPhonetic}{巨头}{ju4tou2}{4,5}{⼯,⼤}[HSK 7-9]
  \definition{s.}{magnata; pessoa ilustre; líderes com influência considerável nas esferas política e econômica}
\end{EntryWithPhonetic}

\begin{EntryWithPhonetic}{巨星}{ju4xing1}{4,9}{⼯,⽇}[HSK 7-9]
  \definition{s.}{Astronomia: estrela gigante | Figurativo: figura excepcional; superestrela; megaestrela (da ópera, do basquete etc.)}
\end{EntryWithPhonetic}

\begin{EntryWithPhonetic}{巨型}{ju4xing2}{4,9}{⼯,⼟}[HSK 7-9]
  \definition{adj.}{gigante; gigantesco; enorme; tamanho extra grande}
\end{EntryWithPhonetic}

%%%%%%%%%% 句 %%%%%%%%%%
\subsection*{句}\addcontentsline{loh}{figure}{句 \dpy{ju4}}

\begin{EntryWithPhonetic}{句}{ju4}{5}{⼝}[HSK 2]
  \definition{clas.}{para sentenças, frases ou linhas de versos}
  \definition{s.}{frase; sentença}
  \seeref{gou4}
\end{EntryWithPhonetic}

\begin{EntryWithPhonetic}{句子}{ju4zi5}{5,3}{⼝,⼦}[HSK 2]
  \definition[个,句]{s.}{sentença; uma unidade linguística composta por palavras ou frases que expressa um significado completo}
\end{EntryWithPhonetic}

%%%%%%%%%% 拒 %%%%%%%%%%
\subsection*{拒}\addcontentsline{loh}{figure}{拒 \dpy{ju4}}

\begin{EntryWithPhonetic}{拒}{ju4}{7}{⼿}
  \definition{v.}{resistir; repelir | recusar; rejeitar}
\end{EntryWithPhonetic}

\begin{EntryWithPhonetic}{拒绝}{ju4jue2}{7,9}{⼿,⽷}[HSK 5]
  \definition{v.}{recusar; rejeitar; declinar; não aceitar (pedidos, sugestões ou presentes)}
\end{EntryWithPhonetic}

%%%%%%%%%% 足 %%%%%%%%%%
\subsection*{足}\addcontentsline{loh}{figure}{足 \dpy{ju4}}

\begin{EntryWithPhonetic}{足}{ju4}{7}{⾜}[Kangxi 157]
  \definition{adj.}{excessivo}
  \seeref{zu2}
\end{EntryWithPhonetic}

%%%%%%%%%% 具 %%%%%%%%%%
\subsection*{具}\addcontentsline{loh}{figure}{具 \dpy{ju4}}

\begin{EntryWithPhonetic}{具}{ju4}{8}{⼋}
  \definition*{s.}{Sobrenome: Ju}
  \definition{clas.}{(literário) usado para caixões, cadáveres e certos objetos}
  \definition{s.}{utensílio; ferramenta; implemento | capacidade; habilidade}
  \definition{v.}{possuir; ter | fornecer; prover | declarar; enumerar}
\end{EntryWithPhonetic}

\begin{EntryWithPhonetic}{具备}{ju4bei4}{8,8}{⼋,⼡}[HSK 4]
  \definition{v.}{ter; possuir; ser provido de}
\end{EntryWithPhonetic}

\begin{EntryWithPhonetic}{具体}{ju4ti3}{8,7}{⼋,⼈}[HSK 3]
  \definition{adj.}{específico; particular | concreto; específico; mais detalhado; muito detalhado; muito claro | concreto; real; não é abstrato, tem uma forma definida; pode ser visto ou sentido}
  \definition{v.}{incorporar; objetivar; combinar teorias, princípios, padrões, etc. com pessoas ou coisas específicas}
\end{EntryWithPhonetic}

\begin{EntryWithPhonetic}{具有}{ju4you3}{8,6}{⼋,⽉}[HSK 3]
  \definition{v.}{ter; possuir; ser provido de}
\end{EntryWithPhonetic}

%%%%%%%%%% 沮 %%%%%%%%%%
\subsection*{沮}\addcontentsline{loh}{figure}{沮 \dpy{ju4}}

\begin{EntryWithPhonetic}{沮}{ju4}{8}{⽔}
  \definition{s.}{lama e folhas em decomposição}
\end{EntryWithPhonetic}

%%%%%%%%%% 俱 %%%%%%%%%%
\subsection*{俱}\addcontentsline{loh}{figure}{俱 \dpy{ju4}}

\begin{EntryWithPhonetic}{俱}{ju4}{10}{⼈}
  \definition{adv.}{(literário) tudo; completamente; inteiramente}
\end{EntryWithPhonetic}

\begin{EntryWithPhonetic}{俱乐部}{ju4le4bu4}{10,5,10}{⼈,⼃,⾢}[HSK 5]
  \definition[个,家,间]{s.}{clube; grupos e locais para atividades sociais, políticas, literárias, recreativas e outras}
\end{EntryWithPhonetic}

%%%%%%%%%% 剧 %%%%%%%%%%
\subsection*{剧}\addcontentsline{loh}{figure}{剧 \dpy{ju4}}

\begin{EntryWithPhonetic}{剧}{ju4}{10}{⼑}[HSK 6]
  \definition*{s.}{Sobrenome: Ju}
  \definition{adj.}{agudo; grave; intenso; violento}
  \definition[部,个,种]{s.}{obra teatral; drama; peça; ópera}
\end{EntryWithPhonetic}

\begin{EntryWithPhonetic}{剧本}{ju4ben3}{10,5}{⼑,⽊}[HSK 5]
  \definition[部,个]{s.}{cenário; roteiro (para drama, filme, etc.); gênero de obra literária que consiste em diálogos entre personagens (às vezes cantados) e indicações de palco}
\end{EntryWithPhonetic}

\begin{EntryWithPhonetic}{剧场}{ju4chang3}{10,6}{⼑,⼟}[HSK 3]
  \definition[个,坐]{s.}{teatro; local para apresentações teatrais, musicais, etc.}
\end{EntryWithPhonetic}

\begin{EntryWithPhonetic}{剧烈}{ju4lie4}{10,10}{⼑,⽕}[HSK 7-9]
  \definition{adj.}{violento; agudo; severo; feroz; rápido e intenso}
\end{EntryWithPhonetic}

\begin{EntryWithPhonetic}{剧目}{ju4mu4}{10,5}{⼑,⽬}[HSK 7-9]
  \definition{s.}{repertório; programa; lista de peças teatrais ou óperas}
\end{EntryWithPhonetic}

\begin{EntryWithPhonetic}{剧情}{ju4qing2}{10,11}{⼑,⼼}[HSK 7-9]
  \definition[个,段]{s.}{a história; o enredo (de uma peça ou ópera)}
\end{EntryWithPhonetic}

\begin{EntryWithPhonetic}{剧团}{ju4tuan2}{10,6}{⼑,⼞}[HSK 7-9]
  \definition[家,个]{s.}{companhia teatral; grupo de ópera; trupe | grupo de teatro}
\end{EntryWithPhonetic}

\begin{EntryWithPhonetic}{剧院}{ju4yuan4}{10,9}{⼑,⾩}[HSK 7-9]
  \definition[家,座]{s.}{teatro; casa de espetáculos | companhias teatrais maiores e de classe mais alta}
\end{EntryWithPhonetic}

\begin{EntryWithPhonetic}{剧组}{ju4zu3}{10,8}{⼑,⽷}[HSK 7-9]
  \definition{s.}{equipe de produção; elenco e equipe técnica; um grupo composto por diretores, atores e membros da equipe com o objetivo de apresentar uma peça teatral ou filmar um filme ou série de televisão}
\end{EntryWithPhonetic}

%%%%%%%%%% 据 %%%%%%%%%%
\subsection*{据}\addcontentsline{loh}{figure}{据 \dpy{ju4}}

\begin{EntryWithPhonetic}{据}{ju4}{11}{⼿}[HSK 6]
  \definition*{s.}{Sobrenome: Ju}
  \definition{prep.}{de acordo com; com base em}
  \definition{s.}{evidência; certificado; prova}
  \definition{v.}{ocupar; apreender | confiar em; depender de}
  \seeref{ju1}
\end{EntryWithPhonetic}

\begin{EntryWithPhonetic}{据此}{ju4ci3}{11,6}{⼿,⽌}[HSK 7-9]
  \definition{v.}{se basear nesses fundamentos; ter em vista o exposto acima; fazer algo em conformidade; se basear nas circunstâncias ou razões já mencionadas}
\end{EntryWithPhonetic}

\begin{EntryWithPhonetic}{据说}{ju4shuo1}{11,9}{⼿,⾔}[HSK 3]
  \definition{v.}{é o que dizem; é o que se diz}
\end{EntryWithPhonetic}

\begin{EntryWithPhonetic}{据悉}{ju4xi1}{11,11}{⼿,⼼}[HSK 7-9]
  \definition{adv.}{é relatado (que); de acordo com o que aprendi}
\end{EntryWithPhonetic}

%%%%%%%%%% 距 %%%%%%%%%%
\subsection*{距}\addcontentsline{loh}{figure}{距 \dpy{ju4}}

\begin{EntryWithPhonetic}{距}{ju4}{11}{⾜}[HSK 7-9]
  \definition{s.}{distância | espora (de um galo, etc.)}
  \definition{v.}{estar separado (longe) de; estar distante de}
\end{EntryWithPhonetic}

\begin{EntryWithPhonetic}{距离}{ju4li2}{11,10}{⾜,⼇}[HSK 4]
  \definition[个,段]{s.}{distância}
  \definition{v.}{estar distante de}
\end{EntryWithPhonetic}

%%%%%%%%%% 锯 %%%%%%%%%%
\subsection*{锯}\addcontentsline{loh}{figure}{锯 \dpy{ju4}}

\begin{EntryWithPhonetic}{锯}{ju4}{13}{⾦}[HSK 7-9]
  \definition[把,个]{s.}{serra; serrote; ferramentas para cortar madeira, etc.}
  \definition{v.}{cortar com uma serra; serrar}
\end{EntryWithPhonetic}

%%%%%%%%%% 聚 %%%%%%%%%%
\subsection*{聚}\addcontentsline{loh}{figure}{聚 \dpy{ju4}}

\begin{EntryWithPhonetic}{聚}{ju4}{14}{⽿}[HSK 4]
  \definition*{s.}{Sobrenome: Ju}
  \definition{v.}{reunir-se; juntar-se}
\end{EntryWithPhonetic}

\begin{EntryWithPhonetic}{聚会}{ju4hui4}{14,6}{⽿,⼈}[HSK 4]
  \definition[个,次]{s.}{reunião; encontro; confraternização; festa}
  \definition{v.}{encontrar-se; reunir-se}
\end{EntryWithPhonetic}

\begin{EntryWithPhonetic}{聚集}{ju4ji2}{14,12}{⽿,⾫}[HSK 7-9]
  \definition{v.}{reunir; juntar; coletar; reunir-se; juntar-se}
\end{EntryWithPhonetic}

\begin{EntryWithPhonetic}{聚精会神}{ju4jing1-hui4shen2}{14,14,6,9}{⽿,⽶,⼈,⽰}[HSK 7-9]
  \definition{expr.}{concentrado; concentrar a atenção; focar a mente; estar absorto em; estar profundamente concentrado; estar totalmente concentrado}
\end{EntryWithPhonetic}

\begin{EntryWithPhonetic}{聚散}{ju4san4}{14,12}{⽿,⽁}
  \definition{s.}{juntos e separados | agregação e dissipação}
\end{EntryWithPhonetic}

%%%%%%%%%% 捐 %%%%%%%%%%
\subsection*{捐}\addcontentsline{loh}{figure}{捐 \dpy{juan1}}

\begin{EntryWithPhonetic}{捐}{juan1}{10}{⼿}[HSK 6]
  \definition{s.}{imposto}
  \definition{v.}{renunciar; abandonar | contribuir; doar; assinar}
\end{EntryWithPhonetic}

\begin{EntryWithPhonetic}{捐款}{juan1/kuan3}{10,12}{⼿,⽋}[HSK 6]
  \definition[笔]{s.}{doação; contribuição (de dinheiro); valor doado}
  \definition{v.+compl.}{doar; contribuir com dinheiro}
\end{EntryWithPhonetic}

\begin{EntryWithPhonetic}{捐献}{juan1xian4}{10,13}{⼿,⽝}[HSK 7-9]
  \definition{v.}{doar; apresentar; contribuir (para uma organização); doar bens ao (estado, a uma cooperativa, etc.)}
\end{EntryWithPhonetic}

\begin{EntryWithPhonetic}{捐赠}{juan1zeng4}{10,16}{⼿,⾙}[HSK 6]
  \definition{v.}{apresentar; contribuir (como um presente); doar (itens para um país ou grupo)}
\end{EntryWithPhonetic}

\begin{EntryWithPhonetic}{捐助}{juan1zhu4}{10,7}{⼿,⼒}[HSK 6]
  \definition{v.}{oferecer (assistência financeira ou material); contribuir; doar}
\end{EntryWithPhonetic}

%%%%%%%%%% 圈 %%%%%%%%%%
\subsection*{圈}\addcontentsline{loh}{figure}{圈 \dpy{juan1}}

\begin{EntryWithPhonetic}{圈}{juan1}{11}{⼞}
  \definition{v.}{prender aves e animais de criação | Coloquial: prender os criminosos; colocar na cadeia, prisão | confinar; encarcerar}
  \seeref{juan4}
  \seeref{quan1}
\end{EntryWithPhonetic}

%%%%%%%%%% 卷 %%%%%%%%%%
\subsection*{卷}\addcontentsline{loh}{figure}{卷 \dpy{juan3}}

\begin{EntryWithPhonetic}{卷}{juan3}{8}{⼙}[HSK 4]
  \definition{clas.}{usado para pequenas coisas enroladas (maço de papel dinheiro, carretel de filme, etc.) | usado para rolos, carretéis, bobinas, etc.}
  \definition[张]{s.}{rolo; carretel; bobina}
  \definition{v.}{enrolar; dobrar algo em um cilindro ou semicírculo | varrer; carregar; levar junto | envolver-se; participar}
  \seeref{juan4}
\end{EntryWithPhonetic}

\begin{EntryWithPhonetic}{卷入}{juan3ru4}{8,2}{⼙,⼊}[HSK 7-9]
  \definition{v.}{ser envolvido; estar envolvido; ser atraído para; estar envolvido em}
\end{EntryWithPhonetic}

\begin{EntryWithPhonetic}{卷子}{juan3zi5}{8,3}{⼙,⼦}
  \definition{s.}{rolinho primavera; um tipo de prato de massa, feito amassando em folhas finas, cobrindo um lado com óleo e sal, enrolando-a e cozinhando-a no vapor}
  \seeref{juan4zi5}
\end{EntryWithPhonetic}

\begin{EntryWithPhonetic}{卷}{juan4}{8}{⼙}[HSK 4]
  \definition{clas.}{usado para capítulos, seções ou volumes; fascículos}
  \definition[大,小]{s.}{livro; livros e pinturas que são enrolados para coleção; geralmente se refere a pinturas e caligrafia | papel de exame | arquivo; dossiê}
  \seeref{juan3}
\end{EntryWithPhonetic}

\begin{EntryWithPhonetic}{卷子}{juan4zi5}{8,3}{⼙,⼦}[HSK 7-9]
  \definition{s.}{prova; prova de exame; um caderno fino ou uma única folha de papel para anotar as respostas durante as provas}
  \seeref{juan3zi5}
\end{EntryWithPhonetic}

%%%%%%%%%% 圈 %%%%%%%%%%
\subsection*{圈}\addcontentsline{loh}{figure}{圈 \dpy{juan4}}

\begin{EntryWithPhonetic}{圈}{juan4}{11}{⼞}[HSK 7-9]
  \definition*{s.}{Sobrenome: Juan}
  \definition{s.}{curral; local onde o gado ou as aves são mantidos, geralmente cercado ou murado, alguns com galpões}
  \seeref{juan1}
  \seeref{quan1}
\end{EntryWithPhonetic}

%%%%%%%%%% 决 %%%%%%%%%%
\subsection*{决}\addcontentsline{loh}{figure}{决 \dpy{jue2}}

\begin{EntryWithPhonetic}{决}{jue2}{6}{⼎}
  \definition{v.}{decidir; determinar | executar uma pessoa | (de um dique, etc.) romper; desabar}
\end{EntryWithPhonetic}

\begin{EntryWithPhonetic}{决不}{jue2bu4}{6,4}{⼎,⼀}[HSK 5]
  \definition{adv.}{em hipótese alguma; nunca | definitivamente não; certamente não; sob nenhuma circunstância; de forma alguma}
\end{EntryWithPhonetic}

\begin{EntryWithPhonetic}{决策}{jue2ce4}{6,12}{⼎,⽵}[HSK 6]
  \definition{s.}{decisão política; decisão de importância estratégica; estratégia ou método de decisão}
  \definition{v.}{formular políticas; tomar uma decisão estratégica; decidir sobre uma estratégia ou abordagem}
\end{EntryWithPhonetic}

\begin{EntryWithPhonetic}{决定}{jue2ding4}{6,8}{⼎,⼧}[HSK 3]
  \definition{adj.}{decisivo; as leis objetivas levam as coisas a se desenvolverem e mudarem em determinada direção}
  \definition[项,个]{s.}{decisão; resolução; assuntos decididos}
  \definition{v.}{decidir; determinar; algo se torna a base ou o pré-requisito para outra coisa; desempenha um papel dominante | decidir; resolver; tomar uma decisão; propor uma forma de agir}
\end{EntryWithPhonetic}

\begin{EntryWithPhonetic}{决赛}{jue2sai4}{6,14}{⼎,⾙}[HSK 3]
  \definition[场]{s.}{finais (de uma competição); em competições esportivas, a última partida ou rodada disputada para determinar a classificação}
\end{EntryWithPhonetic}

\begin{EntryWithPhonetic}{决心}{jue2xin1}{6,4}{⼎,⼼}[HSK 3]
  \definition{s.}{resolução; determinação; determinação inabalável}
  \definition{v.}{secidir-se; decidir fazer algo e não vacilar nem mudar de ideia}
\end{EntryWithPhonetic}

\begin{EntryWithPhonetic}{决议}{jue2yi4}{6,5}{⼎,⾔}[HSK 7-9]
  \definition[项]{s.}{resolução | resultado}
\end{EntryWithPhonetic}

%%%%%%%%%% 诀 %%%%%%%%%%
\subsection*{诀}\addcontentsline{loh}{figure}{诀 \dpy{jue2}}

\begin{EntryWithPhonetic}{诀}{jue2}{6}{⾔}
  \definition[条,个]{s.}{rimas; mnemônicos | jeito; truques do ofício; boas maneiras de resolver problemas}
  \definition{v.}{dar adeus; partir}
\end{EntryWithPhonetic}

\begin{EntryWithPhonetic}{诀别}{jue2bie2}{6,7}{⾔,⼑}[HSK 7-9]
  \definition{v.}{se despedir | separar-se (geralmente com pouca esperança de um novo encontro)}
\end{EntryWithPhonetic}

\begin{EntryWithPhonetic}{诀窍}{jue2qiao4}{6,10}{⾔,⽳}[HSK 7-9]
  \definition{s.}{talento; habilidade;  segredo do sucesso; truques do ofício}
  \seealsoref{诀窍儿}{jue2qiao4r5}
\end{EntryWithPhonetic}

\begin{EntryWithPhonetic}{诀窍儿}{jue2qiao4r5}{6,10,2}{⾔,⽳,⼉}
  \definition{s.}{segredo do sucesso; chave para o sucesso; truques do ofício; jeito para a coisa}
  \seealsoref{诀窍}{jue2qiao4}
\end{EntryWithPhonetic}

%%%%%%%%%% 角 %%%%%%%%%%
\subsection*{角}\addcontentsline{loh}{figure}{角 \dpy{jue2}}

\begin{EntryWithPhonetic}{角}{jue2}{7}{⾓}[Kangxi 148]
  \definition*{s.}{Sobrenome: Jue}
  \definition[个,只,对]{s.}{papel; parte; personagem | tipo de papel (no drama tradicional chinês); categorias de divisão profissional do trabalho entre atores de ópera | ator ou atriz | uma antiga taça de vinho com três pernas | uma nota da antiga escala chinesa de cinco tons, correspondente a 3 na notação musical numerada}
  \definition{v.}{competir; contender; lutar}
  \seeref{jiao3}
\end{EntryWithPhonetic}

\begin{EntryWithPhonetic}{角色}{jue2se4}{7,6}{⾓,⾊}[HSK 4]
  \definition{s.}{papel; personagem em uma peça; personagem representado por um ator | papel; função; parte; uma metáfora para um certo tipo de pessoas na vida social}
\end{EntryWithPhonetic}

\begin{EntryWithPhonetic}{角逐}{jue2zhu2}{7,10}{⾓,⾡}[HSK 7-9]
  \definition{v.}{contender; disputar; entrar em rivalidade; fazer malabarismos por}
\end{EntryWithPhonetic}

%%%%%%%%%% 绝 %%%%%%%%%%
\subsection*{绝}\addcontentsline{loh}{figure}{绝 \dpy{jue2}}

\begin{EntryWithPhonetic}{绝}{jue2}{9}{⽷}[HSK 6]
  \definition{adj.}{exausto; esgotado; acabado | desesperado; sem esperança | único; soberbo; incomparável | não deixar margem de manobra; não fazer concessões; intransigente}
  \definition{adv.}{extremamente; mais | (antes de uma negativa) absolutamente; no mínimo; por qualquer meio; em qualquer conta}
  \definition{s.}{(literário) jueju, um poema de quatro linhas}
  \definition{v.}{cortar; romper | parar de respirar; morrer}
\end{EntryWithPhonetic}

\begin{EntryWithPhonetic}{绝版}{jue2ban3}{9,8}{⽷,⽚}
  \definition{adj.}{esgotado | fora de catálogo}
\end{EntryWithPhonetic}

\begin{EntryWithPhonetic}{绝不}{jue2bu4}{9,4}{⽷,⼀}
  \definition{adv.}{definitivamente não | de forma alguma | sob nenhuma circunstância}
\end{EntryWithPhonetic}

\begin{EntryWithPhonetic}{绝大多数}{jue2da4duo1shu4}{9,3,6,13}{⽷,⼤,⼣,⽁}[HSK 6]
  \definition{expr.}{maioria absoluta | uma maioria esmagadora}
\end{EntryWithPhonetic}

\begin{EntryWithPhonetic}{绝对}{jue2dui4}{9,5}{⽷,⼨}[HSK 3]
  \definition{adj.}{absoluto; sem condições; sem restrições | absoluto; extremo; incompleto; sem margem para negociação ou alteração}
  \definition{adv.}{absolutamente; completamente; com certeza}
\end{EntryWithPhonetic}

\begin{EntryWithPhonetic}{绝技}{jue2ji4}{9,7}{⽷,⼿}[HSK 7-9]
  \definition{s.}{habilidade única; habilidade consumada | \emph{tour-de-force}; uma atuação ou conquista impressionante que foi realizada ou gerenciada com grande habilidade | façanha | feito supremo}
\end{EntryWithPhonetic}

\begin{EntryWithPhonetic}{绝望}{jue2/wang4}{9,11}{⽷,⽉}[HSK 5]
  \definition{v.+compl.}{desesperar; desistir de toda esperança; perder toda esperança de}
\end{EntryWithPhonetic}

\begin{EntryWithPhonetic}{绝缘}{jue2yuan2}{9,12}{⽷,⽷}[HSK 7-9]
  \definition{v.}{isolar; utilizar materiais como borracha ou madeira para bloquear a eletricidade, impedindo sua passagem | ser separado de; ser isolado de; completamente isolado do mundo exterior ou de um objeto específico}
\end{EntryWithPhonetic}

\begin{EntryWithPhonetic}{绝招}{jue2zhao1}{9,8}{⽷,⼿}[HSK 7-9]
  \definition{s.}{habilidade única; movimento delicado inesperado (como último recurso); habilidades únicas e magníficas; métodos engenhosos}
\end{EntryWithPhonetic}

%%%%%%%%%% 觉 %%%%%%%%%%
\subsection*{觉}\addcontentsline{loh}{figure}{觉 \dpy{jue2}}

\begin{EntryWithPhonetic}{觉}{jue2}{9}{⾒}
  \definition{s.}{sentimento; senso; percepção e discriminação de estímulos externos}
  \definition{v.}{sentir; perceber | acordar | tornar-se consciente; tornar-se desperto; despertar; entender}
  \seeref{jiao4}
\end{EntryWithPhonetic}

\begin{EntryWithPhonetic}{觉得}{jue2de5}{9,11}{⾒,⼻}[HSK 1]
  \definition{v.}{sentir; estar ciente; pressentir; causar uma sensação | pensar; sentir; encontrar; considerar (tom menos assertivo)}
\end{EntryWithPhonetic}

\begin{EntryWithPhonetic}{觉悟}{jue2wu4}{9,10}{⾒,⼼}[HSK 6]
  \definition{s.}{consciência; percepção; compreensão; nível de consciência}
  \definition{v.}{vir a compreender; tornar-se consciente de; tornar-se politicamente desperto; despertar}
\end{EntryWithPhonetic}

\begin{EntryWithPhonetic}{觉醒}{jue2xing3}{9,16}{⾒,⾣}[HSK 7-9]
  \definition{v.}{despertar; acordar; perceber; passar da confusão à clareza, do erro à correção, após reconhecer os erros e problemas}
\end{EntryWithPhonetic}

%%%%%%%%%% 倔 %%%%%%%%%%
\subsection*{倔}\addcontentsline{loh}{figure}{倔 \dpy{jue2}}

\begin{EntryWithPhonetic}{倔}{jue2}{10}{⼈}
  \definition{adj.}{rude; mal-humorado; abrupto; curto (uso limitado em 倔犟) | teimoso; direto e franco}
  \seeref{jue4}
\end{EntryWithPhonetic}

\begin{EntryWithPhonetic}{倔强}{jue2jiang4}{10,12}{⼈,⼸}[HSK 7-9]
  \definition{adj.}{teimoso; rígido; inflexível; de personalidade forte e teimosa}
\end{EntryWithPhonetic}

%%%%%%%%%% 崛 %%%%%%%%%%
\subsection*{崛}\addcontentsline{loh}{figure}{崛 \dpy{jue2}}

\begin{EntryWithPhonetic}{崛}{jue2}{11}{⼭}
  \definition{v.}{Literário: subir abruptamente; levantar-se abruptamente; empinar}
\end{EntryWithPhonetic}

\begin{EntryWithPhonetic}{崛起}{jue2qi3}{11,10}{⼭,⾛}[HSK 7-9]
  \definition{v.}{surgir abruptamente; subir repentinamente; aparecer de repente no horizonte | ganhar destaque; ascender}
\end{EntryWithPhonetic}

%%%%%%%%%% 脚 %%%%%%%%%%
\subsection*{脚}\addcontentsline{loh}{figure}{脚 \dpy{jue2}}

\begin{EntryWithPhonetic}{脚}{jue2}{11}{⾁}
  \variantof{角}
\end{EntryWithPhonetic}

%%%%%%%%%% 爵 %%%%%%%%%%
\subsection*{爵}\addcontentsline{loh}{figure}{爵 \dpy{jue2}}

\begin{EntryWithPhonetic}{爵}{jue2}{17}{⽖}
  \definition{s.}{grau de nobreza; aristocracia | o título de nobreza | Arcaico: um antigo recipiente para vinho com três pés e uma alça em forma de laço}
\end{EntryWithPhonetic}

\begin{EntryWithPhonetic}{爵士}{jue2shi4}{17,3}{⽖,⼠}[HSK 7-9]
  \definition*{s.}{\emph{Sir} (Senhor); Título honorífico para cavalheiros da Ordem do Império Britânico}
  \definition{s.}{cavaleiro; o título mais baixo de uma monarquia europeia | Empréstimo linguístico: jazz}[它赞助大型爵士音乐会。===A empresa patrocina concertos de jazz de grande escala.]
\end{EntryWithPhonetic}

%%%%%%%%%% 嚼 %%%%%%%%%%
\subsection*{嚼}\addcontentsline{loh}{figure}{嚼 \dpy{jue2}}

\begin{EntryWithPhonetic}{嚼}{jue2}{20}{⼝}
  \definition{v.}{mastigar; morder; mastigar completamente; é usado em algumas palavras compostas e expressões idiomáticas; usado em 咀嚼}
  \seeref{jiao2}
  \seeref{jiao4}
  \seealsoref{咀嚼}{ju3jue2}
\end{EntryWithPhonetic}

%%%%%%%%%% 倔 %%%%%%%%%%
\subsection*{倔}\addcontentsline{loh}{figure}{倔 \dpy{jue4}}

\begin{EntryWithPhonetic}{倔}{jue4}{10}{⼈}[HSK 7-9]
  \definition{adj.}{teimoso; direto; rude; grosseiro; de natureza direta, com uma atitude severa em relação aos outros}
  \seeref{jue2}
\end{EntryWithPhonetic}

%%%%%%%%%% 军 %%%%%%%%%%
\subsection*{军}\addcontentsline{loh}{figure}{军 \dpy{jun1}}

\begin{EntryWithPhonetic}{军}{jun1}{6}{⼍}
  \definition*{s.}{Sobrenome: Jun}
  \definition{s.}{forças armadas; exército; tropas | exército; contingente; muitas pessoas participando de uma atividade | exército; unidades militares}
\end{EntryWithPhonetic}

\begin{EntryWithPhonetic}{军队}{jun1dui4}{6,4}{⼍,⾩}[HSK 6]
  \definition[支,个]{s.}{forças armadas; exército; tropas}
\end{EntryWithPhonetic}

\begin{EntryWithPhonetic}{军官}{jun1guan1}{6,8}{⼍,⼧}[HSK 7-9]
  \definition{s.}{oficial; militares com patente igual ou superior a tenente, também se refere a oficiais com patente igual ou superior a comandante de pelotão nas forças armadas}
\end{EntryWithPhonetic}

\begin{EntryWithPhonetic}{军舰}{jun1jian4}{6,10}{⼍,⾈}[HSK 6]
  \definition[艘,只]{s.}{navio de guerra; embarcação naval | \emph{warcraft}; um termo geral para embarcações militares equipadas com armas e equipamentos que podem executar missões de combate, incluindo principalmente navios de guerra, cruzadores, contratorpedeiros, porta-aviões, submarinos, torpedeiros, etc.}
\end{EntryWithPhonetic}

\begin{EntryWithPhonetic}{军人}{jun1ren2}{6,2}{⼍,⼈}[HSK 5]
  \definition[名,位,个]{s.}{soldado; militar; pessoal militar; pessoas com status militar; pessoas servindo nas forças armadas}
\end{EntryWithPhonetic}

\begin{EntryWithPhonetic}{军事}{jun1shi4}{6,8}{⼍,⼅}[HSK 6]
  \definition{s.}{militar; assuntos militares; assuntos relativos aos militares e à guerra}
\end{EntryWithPhonetic}

\begin{EntryWithPhonetic}{军装}{jun1zhuang1}{6,12}{⼍,⾐}
  \definition{s.}{uniforme militar}
\end{EntryWithPhonetic}

%%%%%%%%%% 君 %%%%%%%%%%
\subsection*{君}\addcontentsline{loh}{figure}{君 \dpy{jun1}}

\begin{EntryWithPhonetic}{君}{jun1}{7}{⼝}
  \definition*{s.}{Sobrenome: Jun}
  \definition[个,位,名,些]{s.}{monarca; soberano; governante supremo | (como título) Senhor; Sr. | (literário) (em trato direto) você; senhor | cavalheiro | governante}
\end{EntryWithPhonetic}

\begin{EntryWithPhonetic}{君主立宪制}{jun1zhu3li4xian4zhi4}{7,5,5,9,8}{⼝,⼂,⽴,⼧,⼑}
  \definition{s.}{monarquia constitucional}
\end{EntryWithPhonetic}

\begin{EntryWithPhonetic}{君子}{jun1zi3}{7,3}{⼝,⼦}[HSK 7-9]
  \definition{s.}{cavalheiro; nobre; pessoa de caráter nobre; refere"-se a uma pessoa de elevado caráter moral}
\end{EntryWithPhonetic}

%%%%%%%%%% 均 %%%%%%%%%%
\subsection*{均}\addcontentsline{loh}{figure}{均 \dpy{jun1}}

\begin{EntryWithPhonetic}{均}{jun1}{7}{⼟}
  \definition{adj.}{igual; equilibrado; uniforme; igual em quantidade}
  \definition{adv.}{todos; sem exceção}
  \definition{v.}{ser igual a; igualar as quantidades; dividir igualmente}
\end{EntryWithPhonetic}

\begin{EntryWithPhonetic}{均衡}{jun1heng2}{7,16}{⼟,⾏}[HSK 7-9]
  \definition{adj.}{uniforme; equilibrado; proporcional}
\end{EntryWithPhonetic}

\begin{EntryWithPhonetic}{均匀}{jun1yun2}{7,4}{⼟,⼓}[HSK 7-9]
  \definition{adj.}{uniforme; homogêneo; regular; as quantidades são distribuídas ou alocadas igualmente em cada parte; os intervalos de tempo são iguais}
\end{EntryWithPhonetic}

%%%%%%%%%% 俊 %%%%%%%%%%
\subsection*{俊}\addcontentsline{loh}{figure}{俊 \dpy{jun4}}

\begin{EntryWithPhonetic}{俊}{jun4}{9}{⼈}[HSK 7-9]
  \definition*{s.}{Sobrenome: Jun}
  \definition{adj.}{bonito; lindo; encantador; atraente; delicado e bonito | talentoso; inteligente; brilhante; excepcional; excepcionalmente inteligente}
  \definition{s.}{uma pessoa de talento excepcional; pessoas com inteligência excepcional}
\end{EntryWithPhonetic}

\begin{EntryWithPhonetic}{俊俏}{jun4qiao4}{9,9}{⼈,⼈}[HSK 7-9]
  \definition{adj.}{Coloquial: bonito e encantador}
\end{EntryWithPhonetic}

%%%%%%%%%% 骏 %%%%%%%%%%
\subsection*{骏}\addcontentsline{loh}{figure}{骏 \dpy{jun4}}

\begin{EntryWithPhonetic}{骏}{jun4}{10}{⾺}
  \definition{s.}{belo cavalo; corcel; animal de montaria}
\end{EntryWithPhonetic}

\begin{EntryWithPhonetic}{骏马}{jun4ma3}{10,3}{⾺,⾺}[HSK 7-9]
  \definition[匹,群]{s.}{belo cavalo; corcel; animal de montaria}
\end{EntryWithPhonetic}

%%%%%%%%%% 竣 %%%%%%%%%%
\subsection*{竣}\addcontentsline{loh}{figure}{竣 \dpy{jun4}}

\begin{EntryWithPhonetic}{竣}{jun4}{12}{⽴}
  \definition{v.}{concluir; terminar; finalizar}
\end{EntryWithPhonetic}

\begin{EntryWithPhonetic}{竣工}{jun4gong1}{12,3}{⽴,⼯}[HSK 7-9]
  \definition{v.}{ser concluído, finalizado (projetos)}
\end{EntryWithPhonetic}

%%%%% EOF %%%%%

