%%%
%%% N
%%%
\section*{N}\addcontentsline{toc}{section}{N}\addcontentsline{loh}{figure}{\#\#\#\#\#\#\#\# N}

%%%%%%%%%% 那 %%%%%%%%%%
\subsection*{那}\addcontentsline{loh}{figure}{那 \dpy{na1}}

\begin{EntryWithPhonetic}{那}{na1}{6}{⾢}
  \definition*{s.}{Sobrenome: Na}
  \seeref{na3}
  \seeref{na4}
  \seeref{ne4}
  \seeref{nei4}
  \seeref{nuo2}
\end{EntryWithPhonetic}

%%%%%%%%%% 拿 %%%%%%%%%%
\subsection*{拿}\addcontentsline{loh}{figure}{拿 \dpy{na2}}

\begin{EntryWithPhonetic}{拿}{na2}{10}{⼿}[HSK 1]
  \definition{part.}{usado da mesma forma que 把: para marcar o seguinte substantivo seguinte como objeto direto}
  \definition{prep.}{ferramentas, materiais, métodos, etc. utilizados para a introdução | os objetos que estão sendo manipulados para introdução}
  \definition{v.}{segurar; pegar; pegar ou mover objetos com as mãos ou de outra forma | apreender; capturar; prender; usar força bruta para capturar | ter certeza de; ser capaz de fazer; ter uma compreensão firme de | tornar as coisas difíceis para alguém; colocar alguém em uma situação difícil; obstruir; chantagear; coagir; causar dificuldades intencionalmente | fingir ou fazer (algum tipo de postura ou aparência) | ter certeza de; tomar uma decisão | obter; ganhar; receber}
\end{EntryWithPhonetic}

\begin{EntryWithPhonetic}{拿出}{na2 chu1}{10,5}{⼿,⼐}[HSK 2]
  \definition{v.}{apresentar (evidências) | fornecer | apresentar (uma proposta) | oferecer; servir | retirar; tirar}
\end{EntryWithPhonetic}

\begin{EntryWithPhonetic}{拿到}{na2 dao4}{10,8}{⼿,⼑}[HSK 2]
  \definition{v.}{pegar; obter, conseguir}
\end{EntryWithPhonetic}

\begin{EntryWithPhonetic}{拿手}{na2shou3}{10,4}{⼿,⼿}[HSK 7-9]
  \definition{adj.}{hábil; especialista; bom em; proficiente em determinada tecnologia}
\end{EntryWithPhonetic}

\begin{EntryWithPhonetic}{拿走}{na2 zou3}{10,7}{⼿,⾛}[HSK 6]
  \definition{v.}{tirar; remover}
\end{EntryWithPhonetic}

%%%%%%%%%% 那 %%%%%%%%%%
\subsection*{那}\addcontentsline{loh}{figure}{那 \dpy{na3}}

\begin{EntryWithPhonetic}{那}{na3}{6}{⾢}
  \definition{adv.}{expressa negação em perguntas retóricas}
  \definition{pron.}{qual? | qualquer que seja; qualquer que; para expressar incerteza em uma declaração | variante de 哪}
  \seeref{na1}
  \seeref{na4}
  \seeref{ne4}
  \seeref{nei4}
  \seeref{nuo2}
  \seealsoref{哪}{na3}
\end{EntryWithPhonetic}

%%%%%%%%%% 哪 %%%%%%%%%%
\subsection*{哪}\addcontentsline{loh}{figure}{哪 \dpy{na3}}

\begin{EntryWithPhonetic}{哪}{na3}{9}{⼝}[HSK 1,4]
  \definition{adv.}{para expressar uma pergunta retórica, indicando que é impossível}
  \definition{pron.}{qual?; o que?; expressa a necessidade de determinar um entre várias pessoas ou coisas | qualquer; ser usado em um sentido geral | qual?; o que?; (usado sozinho, o mesmo que 什么, frequentemente usado de forma intercambiável com 什么) | qualquer; qualquer que seja; refere"-se a qualquer um, geralmente seguido por 都 ou 也, ou usando dois 哪 antes e depois | qual (indica algo incerto)}
  \seeref{na5}
  \seeref{nei3}
  \seealsoref{都}{dou1}
  \seealsoref{什么}{shen2me5}
  \seealsoref{也}{ye3}
\end{EntryWithPhonetic}

\begin{EntryWithPhonetic}{哪个}{na3ge5}{9,3}{⼝,⼈}
  \definition{pron.}{qual deles (pergunta sobre o objeto) | quem (perguntar a alguém ou indicar qualquer pessoa)}
\end{EntryWithPhonetic}

\begin{EntryWithPhonetic}{哪国人}{na3 guo2ren2}{9,8,2}{⼝,⼞,⼈}
  \definition{expr.}{de qual país?}
\end{EntryWithPhonetic}

\begin{EntryWithPhonetic}{哪里}{na3 li3}{9,7}{⼝,⾥}[HSK 1]
  \definition{adv.}{usado em perguntas retóricas para expressar um significado negativo}
  \definition{pron.}{onde?; em que lugar? | onde quer que seja; em qualquer lugar | usado como uma resposta educada a um elogio}
\end{EntryWithPhonetic}

\begin{EntryWithPhonetic}{哪怕}{na3pa4}{9,8}{⼝,⼼}[HSK 4]
  \definition{conj.}{mesmo; mesmo se; mesmo que; não importa o quão}
\end{EntryWithPhonetic}

\begin{EntryWithPhonetic}{哪儿}{na3r5}{9,2}{⼝,⼉}[HSK 1]
  \definition{adv.}{usado para perguntas retóricas, indicando negação}
  \definition{pron.}{onde? | onde quer que seja; em qualquer lugar | usado como uma resposta educada a um elogio}
\end{EntryWithPhonetic}

\begin{EntryWithPhonetic}{哪些}{na3xie1}{9,8}{⼝,⼆}[HSK 1]
  \definition{pron.}{quais?}
\end{EntryWithPhonetic}

\begin{EntryWithPhonetic}{哪知道}{na3 zhi1dao4}{9,8,12}{⼝,⽮,⾡}[HSK 7-9]
  \definition{expr.}{Quem sabe?}[我哪知道 你挑起来的。===Como eu ia saber que você tinha começado?]
\end{EntryWithPhonetic}

%%%%%%%%%% 那 %%%%%%%%%%
\subsection*{那}\addcontentsline{loh}{figure}{那 \dpy{na4}}

\begin{EntryWithPhonetic}{那}{na4}{6}{⾢}[HSK 1,2]
  \definition{conj.}{então; nessa situação; nesse caso; o mesmo que 那么}
  \definition{pron.}{aquele; aquilo; indica pessoas ou coisas distantes | aquele; aquilo; expressa muitas coisas, sem se referir especificamente a uma pessoa ou coisa, e é frequentemente usado em conjunto com 这}
  \seeref{na1}
  \seeref{na3}
  \seeref{ne4}
  \seeref{nei4}
  \seeref{nuo2}
  \seealsoref{那么}{na4 me5}
  \seealsoref{这}{zhe4}
\end{EntryWithPhonetic}

\begin{EntryWithPhonetic}{那边}{na4 bian5}{6,5}{⾢,⾡}[HSK 1]
  \definition{pron.}{ali; acolá; aquele lado}
\end{EntryWithPhonetic}

\begin{EntryWithPhonetic}{那个}{na4ge5}{6,3}{⾢,⼈}
  \definition{pron.}{aquele | usado antes de verbos e adjetivos para indicar exagero | para substituir o discurso direto inconveniente}
\end{EntryWithPhonetic}

\begin{EntryWithPhonetic}{那会儿}{na4 hui4r5}{6,6,2}{⾢,⼈,⼉}[HSK 2]
  \definition{pron.}{então; naquela época; refere"-se ao passado ou ao futuro}
\end{EntryWithPhonetic}

\begin{EntryWithPhonetic}{那里}{na4 li3}{6,7}{⾢,⾥}[HSK 1]
  \definition{pron./s.}{lá; ali; aquele lugar; indica um lugar distante}
\end{EntryWithPhonetic}

\begin{EntryWithPhonetic}{那么}{na4 me5}{6,3}{⾢,⼃}[HSK 2]
  \definition{conj.}{então; nesse caso; afirmar o resultado esperado ou fazer um julgamento}
  \definition{pron.}{assim; dessa maneira; indica a natureza, o estado, a forma, o grau, etc. | assim; sobre; colocado antes do numeral, indica uma estimativa}
\end{EntryWithPhonetic}

\begin{EntryWithPhonetic}{那麽}{na4 me5}{6,14}{⾢,⿇}
  \variantof{那么}
\end{EntryWithPhonetic}

\begin{EntryWithPhonetic}{那儿}{na4r5}{6,2}{⾢,⼉}[HSK 1]
  \definition{pron.}{lá; ali; naquele lugar | então; naquela época (usado após 打, 从 e 由)}
  \seealsoref{从}{cong2}
  \seealsoref{打}{da3}
  \seealsoref{由}{you2}
\end{EntryWithPhonetic}

\begin{EntryWithPhonetic}{那时}{na4 shi2}{6,7}{⾢,⽇}
  \definition{pron.}{então; naquela época; naqueles dias; geralmente se refere a um período de tempo distante do presente}
  \seealsoref{那时候}{na4 shi2 hou5}
\end{EntryWithPhonetic}

\begin{EntryWithPhonetic}{那时候}{na4 shi2 hou5}{6,7,10}{⾢,⽇,⼈}[HSK 2]
  \definition{adv.}{naquela hora; em algum momento no passado}
  \seealsoref{那时}{na4 shi2}
\end{EntryWithPhonetic}

\begin{EntryWithPhonetic}{那些}{na4 xie1}{6,8}{⾢,⼆}[HSK 1]
  \definition{pron.}{aqueles; indica duas ou mais pessoas ou coisas}
\end{EntryWithPhonetic}

\begin{EntryWithPhonetic}{那样}{na4 yang4}{6,10}{⾢,⽊}[HSK 2]
  \definition{pron.}{assim; tal; desse tipo; desse gênero; dessa natureza; desse tipo; indica a natureza, o estado, a maneira, o grau ou refere"-se a uma ação ou situação específica}
\end{EntryWithPhonetic}

\begin{EntryWithPhonetic}{那咱}{na4 zan5}{6,9}{⾢,⼝}
  \definition{s.}{(informal) naquela época; então | (antigo) naquela época}
\end{EntryWithPhonetic}

%%%%%%%%%% 呐 %%%%%%%%%%
\subsection*{呐}\addcontentsline{loh}{figure}{呐 \dpy{na4}}

\begin{EntryWithPhonetic}{呐}{na4}{7}{⼝}
  \definition{interj./v.}{elemento formador de palavras}
  \seeref{ne4}
\end{EntryWithPhonetic}

\begin{EntryWithPhonetic}{呐喊}{na4han3}{7,12}{⼝,⼝}[HSK 7-9]
  \definition{v.}{gritar bem alto; berrar bem alto}
\end{EntryWithPhonetic}

%%%%%%%%%% 纳 %%%%%%%%%%
\subsection*{纳}\addcontentsline{loh}{figure}{纳 \dpy{na4}}

\begin{EntryWithPhonetic}{纳}{na4}{7}{⽷}
  \definition*{s.}{Sobrenome: Na}
  \definition{v.}{receber; admitir; trazer para dentro; deixar entrar | aceitar | aproveitar | pagar impostos; pagar taxas | trazer para; colocar em | costurar com pontos próximos (sobre um remendo, etc.); remendar}
\end{EntryWithPhonetic}

\begin{EntryWithPhonetic}{纳闷儿}{na4/men4r5}{7,7,2}{⽷,⾨,⼉}[HSK 7-9]
  \definition{v.+compl.}{maravilhar-se; sentir-se perplexo; ficar confuso}
\end{EntryWithPhonetic}

\begin{EntryWithPhonetic}{纳入}{na4ru4}{7,2}{⽷,⼊}[HSK 7-9]
  \definition{v.}{incluir; incorporar; trazer para; inserir; classificar (geralmente usado para conceitos abstratos)}
\end{EntryWithPhonetic}

\begin{EntryWithPhonetic}{纳税}{na4/shui4}{7,12}{⽷,⽲}[HSK 7-9]
  \definition{v.+compl.}{pagar impostos; pagar impostos ao estado}
\end{EntryWithPhonetic}

\begin{EntryWithPhonetic}{纳税人}{na4shui4ren2}{7,12,2}{⽷,⽲,⼈}[HSK 7-9]
  \definition{s.}{contribuinte; pagador de impostos}[我们不能这样浪费纳税人的钱。===Não podemos desperdiçar o dinheiro dos contribuintes dessa forma.]
\end{EntryWithPhonetic}

%%%%%%%%%% 哪 %%%%%%%%%%
\subsection*{哪}\addcontentsline{loh}{figure}{哪 \dpy{na5}}

\begin{EntryWithPhonetic}{哪}{na5}{9}{⼝}
  \definition{part.}{usado depois de uma palavra com a terminação -n, é equivalente a 啊}
  \seeref{na3}
  \seeref{nei3}
  \seealsoref{啊}{a5}
\end{EntryWithPhonetic}

%%%%%%%%%% 乃 %%%%%%%%%%
\subsection*{乃}\addcontentsline{loh}{figure}{乃 \dpy{nai3}}

\begin{EntryWithPhonetic}{乃}{nai3}{2}{⼃}[HSK 7-9]
  \definition{adv.}{então; portanto | somente então}
  \definition{pron.}{você; seu}
  \definition{v.}{ser | ser realmente; ser de fato}[失败乃成功之母。===O fracasso é a mãe do sucesso.]
\end{EntryWithPhonetic}

\begin{EntryWithPhonetic}{乃至}{nai3zhi4}{2,6}{⼃,⾄}[HSK 7-9]
  \definition{conj.}{até mesmo; usado para enfatizar que algo excede o alcance ou a extensão esperada}
\end{EntryWithPhonetic}

%%%%%%%%%% 奶 %%%%%%%%%%
\subsection*{奶}\addcontentsline{loh}{figure}{奶 \dpy{nai3}}

\begin{EntryWithPhonetic}{奶}{nai3}{5}{⼥}[HSK 1]
  \definition{adj.}{bebê; infância; infantil}
  \definition[杯,滴,瓶,只,桶]{s.}{seios; mama | leite; produtos lácteos}
  \definition{v.}{amamentar; mamar}
\end{EntryWithPhonetic}

\begin{EntryWithPhonetic}{奶茶}{nai3 cha2}{5,9}{⼥,⾋}[HSK 3]
  \definition[杯]{s.}{chá com leite; chá com leite de vaca ou de ovelha}
\end{EntryWithPhonetic}

\begin{EntryWithPhonetic}{奶粉}{nai3 fen3}{5,10}{⼥,⽶}[HSK 6]
  \definition[袋,桶,罐,勺]{s.}{leite em pó}
\end{EntryWithPhonetic}

\begin{EntryWithPhonetic}{奶奶}{nai3nai5}{5,5}{⼥,⼥}[HSK 1]
  \definition[位]{s.}{avó (paterna) | vovó; avó; mulheres mais velhas | jovem senhora da casa}
\end{EntryWithPhonetic}

\begin{EntryWithPhonetic}{奶牛}{nai3 niu2}{5,4}{⼥,⽜}[HSK 6]
  \definition{s.}{vaca leiteira (ou leiteira); vaca}
\end{EntryWithPhonetic}

%%%%%%%%%% 耐 %%%%%%%%%%
\subsection*{耐}\addcontentsline{loh}{figure}{耐 \dpy{nai4}}

\begin{EntryWithPhonetic}{耐}{nai4}{9}{⽽}[HSK 7-9]
  \definition{v.}{ser capaz de suportar (ou tolerar); poder resistir; poder suportar | suportar; aguentar; resistir}
\end{EntryWithPhonetic}

\begin{EntryWithPhonetic}{耐人寻味}{nai4ren2xun2wei4}{9,2,6,8}{⽽,⼈,⼨,⼝}[HSK 7-9]
  \definition{expr.}{intrigante; instigante; é algo profundo e que merece uma reflexão cuidadosa}
\end{EntryWithPhonetic}

\begin{EntryWithPhonetic}{耐心}{nai4xin1}{9,4}{⽽,⼼}[HSK 5]
  \definition{adj.}{paciente}
  \definition[些]{s.}{paciência; uma pessoa que não se importa com problemas e é paciente}
\end{EntryWithPhonetic}

\begin{EntryWithPhonetic}{耐性}{nai4xing4}{9,8}{⽽,⼼}[HSK 7-9]
  \definition{s.}{paciência; resistência | tolerância; uma personalidade paciente e sem pressa}
\end{EntryWithPhonetic}

%%%%%%%%%% 男 %%%%%%%%%%
\subsection*{男}\addcontentsline{loh}{figure}{男 \dpy{nan2}}

\begin{EntryWithPhonetic}{男}{nan2}{7}{⽥}[HSK 1]
  \definition{adj.}{homem; macho; masculino}
  \definition[个,位]{s.}{filho; menino | homem | barão (o mais baixo de cinco ordens de nobreza)}
  \antonymref{女}{nv3}
\end{EntryWithPhonetic}

\begin{EntryWithPhonetic}{男孩儿}{nan2hai2r5}{7,9,2}{⽥,⼦,⼉}[HSK 1]
  \definition{s.}{menino; rapaz}
\end{EntryWithPhonetic}

\begin{EntryWithPhonetic}{男女}{nan2 nv3}{7,3}{⽥,⼥}[HSK 4]
  \definition{s.}{homens e mulheres; masculino e feminino}
\end{EntryWithPhonetic}

\begin{EntryWithPhonetic}{男朋友}{nan2 peng2 you5}{7,8,4}{⽥,⽉,⼜}[HSK 1]
  \definition{s.}{namorado}
\end{EntryWithPhonetic}

\begin{EntryWithPhonetic}{男人}{nan2 ren2}{7,2}{⽥,⼈}[HSK 1]
  \definition[个]{s.}{homem adulto; macho; cavalheiro | marido}
\end{EntryWithPhonetic}

\begin{EntryWithPhonetic}{男生}{nan2 sheng1}{7,5}{⽥,⽣}[HSK 1]
  \definition[个]{s.}{menino; estudante; estudante do sexo masculino; aluno do sexo masculino}
\end{EntryWithPhonetic}

\begin{EntryWithPhonetic}{男士}{nan2 shi4}{7,3}{⽥,⼠}[HSK 4]
  \definition{s.}{cavalheiro; \emph{gentleman}}
\end{EntryWithPhonetic}

\begin{EntryWithPhonetic}{男性}{nan2 xing4}{7,8}{⽥,⼼}[HSK 5]
  \definition{s.}{masculino; homem; masculinidade}
  \antonymref{女性}{nv3 xing4}
\end{EntryWithPhonetic}

\begin{EntryWithPhonetic}{男子}{nan2zi3}{7,3}{⽥,⼦}[HSK 3]
  \definition[个,位]{s.}{uma pessoa do sexo masculino; um homem}
\end{EntryWithPhonetic}

%%%%%%%%%% 南 %%%%%%%%%%
\subsection*{南}\addcontentsline{loh}{figure}{南 \dpy{nan2}}

\begin{EntryWithPhonetic}{南}{nan2}{9}{⼗}[HSK 1]
  \definition*{s.}{Sobrenome: Nan}
  \definition{s.}{sul; uma das quatro direções básicas, o lado direito quando se está de frente para o sol pela manhã | especificamente no sul da China}
  \antonymref{北}{bei3}
\end{EntryWithPhonetic}

\begin{EntryWithPhonetic}{南北}{nan2 bei3}{9,5}{⼗,⼔}[HSK 5]
  \definition{s.}{(território) norte e sul | (distância) de norte a sul}
\end{EntryWithPhonetic}

\begin{EntryWithPhonetic}{南边}{nan2 bian5}{9,5}{⼗,⾡}[HSK 1]
  \definition{s.}{sul; lado sul}
\end{EntryWithPhonetic}

\begin{EntryWithPhonetic}{南部}{nan2 bu4}{9,10}{⼗,⾢}[HSK 3]
  \definition{s.}{parte sul; sul | a parte sul}
\end{EntryWithPhonetic}

\begin{EntryWithPhonetic}{南方}{nan2 fang1}{9,4}{⼗,⽅}[HSK 2]
  \definition{s.}{sul; indica a direção sul | o sul; a região sul}
\end{EntryWithPhonetic}

\begin{EntryWithPhonetic}{南瓜}{nan2gua1}{9,5}{⼗,⽠}[HSK 7-9]
  \definition[个,斤,块]{s.}{abóbora}
\end{EntryWithPhonetic}

\begin{EntryWithPhonetic}{南极}{nan2ji2}{9,7}{⼗,⽊}[HSK 5]
  \definition*{s.}{Polo Sul; Polo Antártico | Polo sul magnético}
  \definition{s.}{polo sul magnético}
\end{EntryWithPhonetic}

\begin{EntryWithPhonetic}{南京}{nan2jing1}{9,8}{⼗,⼇}
  \definition*{s.}{Nanquim, capital da província de Jiangsu, 江苏}
  \seealsoref{江苏}{jiang1su1}
\end{EntryWithPhonetic}

\begin{EntryWithPhonetic}{南面}{nan2mian4}{9,9}{⼗,⾯}
  \definition{s.}{sul | lado sul}
\end{EntryWithPhonetic}

%%%%%%%%%% 难 %%%%%%%%%%
\subsection*{难}\addcontentsline{loh}{figure}{难 \dpy{nan2}}

\begin{EntryWithPhonetic}{难}{nan2}{10}{⾫}[HSK 1]
  \definition{adj.}{difícil; duro; problemático | dificilmente possível; inevitável | ruim; desagradável | problemático; improvável}
  \definition{s.}{dificuldade}
  \definition{v.}{colocar alguém em uma situação difícil}
  \seeref{nan4}
  \antonymref{易}{yi4}
\end{EntryWithPhonetic}

\begin{EntryWithPhonetic}{难处}{nan2chu3}{10,5}{⾫,⼡}
  \definition{adj.}{dificuldade de convivência; difícil de conviver; difícil de lidar}
  \seeref{nan2chu4}
  \seeref{nan2chu5}
\end{EntryWithPhonetic}

\begin{EntryWithPhonetic}{难处}{nan2chu4}{10,5}{⾫,⼡}[HSK 7-9]
  \definition{s.}{dificuldade; problema; assunto difícil}
  \seeref{nan2chu3}
  \seeref{nan2chu5}
\end{EntryWithPhonetic}

\begin{EntryWithPhonetic}{难处}{nan2chu5}{10,5}{⾫,⼡}
  \definition{s.}{dificuldade; problema; assunto difícil}
  \seeref{nan2chu3}
  \seeref{nan2chu4}
\end{EntryWithPhonetic}

\begin{EntryWithPhonetic}{难道}{nan2dao4}{10,12}{⾫,⾡}[HSK 3]
  \definition{adv.}{certamente não significa que\dots?; é possível que\dots?; não me diga\dots; poderia ser que\dots?; usado em frases interrogativas para reforçar o tom interrogativo; frequentemente usado com palavras como 吗 e 不成}
  \seealsoref{不成}{bu4 cheng2}
  \seealsoref{吗}{ma5}
\end{EntryWithPhonetic}

\begin{EntryWithPhonetic}{难得}{nan2de2}{10,11}{⾫,⼻}[HSK 5]
  \definition{adj.}{raro; difícil de encontrar; difícil de obter ou realizar, indicando que é valioso}
  \definition{adv.}{raramente; com pouca frequência}
\end{EntryWithPhonetic}

\begin{EntryWithPhonetic}{难得一见}{nan2de2 yi2 jian4}{10,11,1,4}{⾫,⼻,⼀,⾒}[HSK 7-9]
  \definition{expr.}{``Uma visão rara.''; raramente visto}
\end{EntryWithPhonetic}

\begin{EntryWithPhonetic}{难点}{nan2dian3}{10,9}{⾫,⽕}[HSK 7-9]
  \definition[个]{s.}{ponto difícil; osso duro de roer | dificuldade; áreas onde o problema não é fácil de resolver}
\end{EntryWithPhonetic}

\begin{EntryWithPhonetic}{难度}{nan2 du4}{10,9}{⾫,⼴}[HSK 3]
  \definition{s.}{dificuldade; grau de dificuldade}
\end{EntryWithPhonetic}

\begin{EntryWithPhonetic}{难怪}{nan2guai4}{10,8}{⾫,⼼}[HSK 7-9]
  \definition{adv.}{não é de admirar; não me admira}
  \definition{v.}{ser compreensível; ser perdoável}
\end{EntryWithPhonetic}

\begin{EntryWithPhonetic}{难关}{nan2guan1}{10,6}{⾫,⼋}[HSK 7-9]
  \definition[道,个]{s.}{crise; barreira; apertado; dificuldade; obstáculo; aperto; uma metáfora para uma dificuldade difícil de superar}
\end{EntryWithPhonetic}

\begin{EntryWithPhonetic}{难过}{nan2guo4}{10,6}{⾫,⾡}[HSK 2]
  \definition{adj.}{triste; ruim; psicologicamente desconfortável | difícil; árduo}
\end{EntryWithPhonetic}

\begin{EntryWithPhonetic}{难堪}{nan2kan1}{10,12}{⾫,⼟}[HSK 7-9]
  \definition{adj.}{constrangedor; sem graça}
  \definition{v.}{ser difícil de suportar}
\end{EntryWithPhonetic}

\begin{EntryWithPhonetic}{难看}{nan2 kan4}{10,9}{⾫,⽬}[HSK 2]
  \definition{adj.}{feio; desagradável à vista | vergonhoso; embaraçoso; desonroso; sem glória; sem dignidade}
\end{EntryWithPhonetic}

\begin{EntryWithPhonetic}{难受}{nan2shou4}{10,8}{⾫,⼜}[HSK 2]
  \definition{adj.}{sentir dor; sentir-se mal; sentir-se desconfortável | sentir-se mal; sentir-se infeliz; de mau humor; triste}
\end{EntryWithPhonetic}

\begin{EntryWithPhonetic}{难说}{nan2shuo1}{10,9}{⾫,⾔}[HSK 7-9]
  \definition{v.}{ser difícil dizer; nunca se saber ao certo}[他什么时候回来还很难说。===É difícil dizer quando ele voltará.]
\end{EntryWithPhonetic}

\begin{EntryWithPhonetic}{难题}{nan2 ti2}{10,15}{⾫,⾴}[HSK 2]
  \definition[个,道]{s.}{desafio; problema difícil; questão difícil; questões difíceis de responder ou resolver}
\end{EntryWithPhonetic}

\begin{EntryWithPhonetic}{难听}{nan2 ting1}{10,7}{⾫,⼝}[HSK 2]
  \definition{adj.}{desagradável de ouvir | ofensivo; grosseiro; vulgar e desagradável | escandaloso; indigno}
\end{EntryWithPhonetic}

\begin{EntryWithPhonetic}{难忘}{nan2 wang4}{10,7}{⾫,⼼}[HSK 6]
  \definition{adj.}{memorável; inesquecível}
\end{EntryWithPhonetic}

\begin{EntryWithPhonetic}{难为情}{nan2wei2qing2}{10,4,11}{⾫,⼂,⼼}[HSK 7-9]
  \definition{adj.}{tímido; envergonhado; constrangido; descreve o sentimento de constrangimento ou vergonha que alguém experimenta ao se deparar com uma situação embaraçosa ou constrangedora | desconcertante; embaraçoso; difícil de lidar; descreve a sensação de estar preso em uma situação difícil ou de se sentir impotente ao se deparar com um problema que é difícil de decidir, lidar ou enfrentar}
\end{EntryWithPhonetic}

\begin{EntryWithPhonetic}{难以}{nan2 yi3}{10,4}{⾫,⼈}[HSK 5]
  \definition{adj.}{difícil; complicado}
\end{EntryWithPhonetic}

\begin{EntryWithPhonetic}{难以想象}{nan2yi3-xiang3xiang4}{10,4,13,11}{⾫,⼈,⼼,⾗}[HSK 7-9]
  \definition{expr.}{inimaginável; difícil imaginar}
\end{EntryWithPhonetic}

\begin{EntryWithPhonetic}{难以置信}{nan2yi3-zhi4xin4}{10,4,13,9}{⾫,⼈,⽹,⼈}[HSK 7-9]
  \definition{expr.}{difícil de acreditar; incrível; inacreditável; ser quase inacreditável; ser inacreditável; ser difícil de acreditar; ser bastante difícil de engolir}
\end{EntryWithPhonetic}

\begin{EntryWithPhonetic}{难}{nan4}{10}{⾫}
  \definition{s.}{catástrofe; calamidade; desastre; adversidade; grande infortúnio}
  \definition{v.}{acusar; culpar}
  \seeref{nan2}
\end{EntryWithPhonetic}

\begin{EntryWithPhonetic}{难免}{nan4mian3}{10,7}{⾫,⼉}[HSK 4]
  \definition{adj.}{inevitável; difícil de evitar}
\end{EntryWithPhonetic}

%%%%%%%%%% 孬 %%%%%%%%%%
\subsection*{孬}\addcontentsline{loh}{figure}{孬 \dpy{nao1}}

\begin{EntryWithPhonetic}{孬}{nao1}{10}{⼥}
  \definition{adj.}{ruim | covarde | Dialeto: não (é) bom (contração de 不 + 好)}
  \seealsoref{不}{bu4}
  \seealsoref{好}{hao3}
\end{EntryWithPhonetic}

%%%%%%%%%% 呶 %%%%%%%%%%
\subsection*{呶}\addcontentsline{loh}{figure}{呶 \dpy{nao2}}

\begin{EntryWithPhonetic}{呶}{nao2}{8}{⼝}
  \definition{interj.}{Onomatopéia: ruído alto e contínuo}
  \definition{v.}{Literário: gritar; clamar; falar ruidosamente}
  \seealsoref{努}{nu3}
\end{EntryWithPhonetic}

%%%%%%%%%% 挠 %%%%%%%%%%
\subsection*{挠}\addcontentsline{loh}{figure}{挠 \dpy{nao2}}

\begin{EntryWithPhonetic}{挠}{nao2}{9}{⼿}[HSK 7-9]
  \definition{v.}{coçar; (usar os dedos) para segurar delicadamente | dificultar; obstruir; impedir que os outros façam as coisas sem problemas | recuar; ceder; dar a mão; dobrar, metaforicamente significando ceder}
\end{EntryWithPhonetic}

%%%%%%%%%% 恼 %%%%%%%%%%
\subsection*{恼}\addcontentsline{loh}{figure}{恼 \dpy{nao3}}

\begin{EntryWithPhonetic}{恼}{nao3}{9}{⼼}
  \definition{adj.}{infeliz; preocupado; aflito; angustiado}
  \definition{v.}{perturbar; irritar; incomodar | ficar com raiva; ficar irritado}
\end{EntryWithPhonetic}

\begin{EntryWithPhonetic}{恼羞成怒}{nao3xiu1-cheng2nu4}{9,10,6,9}{⼼,⽺,⼽,⼼}[HSK 7-9]
  \definition{expr.}{ficar furioso de vergonha; ser envergonhado a ponto de ficar com raiva; sentir vergonha a ponto de ficar com raiva; irritação; ficar furioso por causa da humilhação}
\end{EntryWithPhonetic}

%%%%%%%%%% 脑 %%%%%%%%%%
\subsection*{脑}\addcontentsline{loh}{figure}{脑 \dpy{nao3}}

\begin{EntryWithPhonetic}{脑}{nao3}{10}{⾁}
  \definition{s.}{(fisiologia) cérebro | tofu;  substância branca semelhante ao cérebro ou à medula espinhal cerebral | cabeça | a essência de um objeto}
\end{EntryWithPhonetic}

\begin{EntryWithPhonetic}{脑袋}{nao3dai5}{10,11}{⾁,⾐}[HSK 4]
  \definition[颗,个]{s.}{cabeça; a parte mais alta do corpo humano ou a parte mais alta de um animal que contém órgãos como a boca, o nariz, os olhos etc. | mente; cérebro; capacidade de pensar, lembrar, etc.}
\end{EntryWithPhonetic}

\begin{EntryWithPhonetic}{脑瓜}{nao3gua1}{10,5}{⾁,⽠}
  \definition{s.}{crânio | cérebro | cabeça | mente | mentalidade | ideia}
  \seealsoref{脑瓜子}{nao3gua1zi5}
\end{EntryWithPhonetic}

\begin{EntryWithPhonetic}{脑瓜子}{nao3gua1zi5}{10,5,3}{⾁,⽠,⼦}
  \definition{s.}{Coloquial: crânio; cérebro; cabeça; mente; mentalidade; ideia}
  \seealsoref{脑瓜}{nao3gua1}
\end{EntryWithPhonetic}

\begin{EntryWithPhonetic}{脑海}{nao3hai3}{10,10}{⾁,⽔}[HSK 7-9]
  \definition{s.}{cérebro (referindo"-se principalmente às suas funções de pensamento e memória); mente; olho da mente}
\end{EntryWithPhonetic}

\begin{EntryWithPhonetic}{脑筋}{nao3jin1}{10,12}{⾁,⽵}[HSK 7-9]
  \definition[个]{s.}{cérebro; mente; cabeça; refere"-se a habilidades como raciocínio e memória | ideias; conceitos; refere"-se a métodos ou hábitos de pensamento}
\end{EntryWithPhonetic}

\begin{EntryWithPhonetic}{脑子}{nao3 zi5}{10,3}{⾁,⼦}[HSK 5]
  \definition[个]{s.}{cérebro | mente; cabeça; cérebro; inteligência; poder mental; refere"-se à capacidade de pensar, memorizar, raciocinar, etc.; inteligência}
\end{EntryWithPhonetic}

%%%%%%%%%% 闹 %%%%%%%%%%
\subsection*{闹}\addcontentsline{loh}{figure}{闹 \dpy{nao4}}

\begin{EntryWithPhonetic}{闹}{nao4}{8}{⾾}[HSK 4]
  \definition{adj.}{barulhento}
  \definition{v.}{fazer barulho; provocar problemas | dar vazão (à sua raiva, ressentimento, etc.) | sofrer de; ser incomodado por; ocorrer (um desastre ou coisa ruim) | fazer;  entrar em ação | agitar; perturbar | brincar; fazer bagunça}
\end{EntryWithPhonetic}

\begin{EntryWithPhonetic}{闹事}{nao4/shi4}{8,8}{⾾,⼅}[HSK 7-9]
  \definition{v.+compl.}{criar perturbação; causar problemas}
\end{EntryWithPhonetic}

\begin{EntryWithPhonetic}{闹着玩儿}{nao4zhe5wan2r5}{8,11,8,2}{⾾,⽬,⽟,⼉}[HSK 7-9]
  \definition{expr.}{``Estou brincando.''; piada; brincar; fazer algo por diversão; estar brincando}
\end{EntryWithPhonetic}

\begin{EntryWithPhonetic}{闹钟}{nao4 zhong1}{8,9}{⾾,⾦}[HSK 4]
  \definition[个,台,只,款]{s.}{despertador; relógios capazes de tocar alarmes em horários predeterminados}
\end{EntryWithPhonetic}

%%%%%%%%%% 那 %%%%%%%%%%
\subsection*{那}\addcontentsline{loh}{figure}{那 \dpy{ne4}}

\begin{EntryWithPhonetic}{那}{ne4}{6}{⾢}
  \definition{conj.}{então; nesse caso; o mesmo que 那么}
  \definition{pron.}{aquele; aquilo; pronúncia coloquial de 那 (\dpy{na4})}
  \seeref{na1}
  \seeref{na3}
  \seeref{na4}
  \seeref{nei4}
  \seeref{nuo2}
  \seealsoref{那么}{na4 me5}
\end{EntryWithPhonetic}

%%%%%%%%%% 呐 %%%%%%%%%%
\subsection*{呐}\addcontentsline{loh}{figure}{呐 \dpy{ne4}}

\begin{EntryWithPhonetic}{呐}{ne4}{7}{⼝}
  \definition{adj.}{lento; de ritmo lento; hesitante em falar}
  \definition{v.}{sussurrar; falar baixinho}
\end{EntryWithPhonetic}

%%%%%%%%%% 呢 %%%%%%%%%%
\subsection*{呢}\addcontentsline{loh}{figure}{呢 \dpy{ne5}}

\begin{EntryWithPhonetic}{呢}{ne5}{8}{⼝}[HSK 1]
  \definition{part.}{usada no final de frases interrogativas (especificamente perguntas, perguntas de escolha e perguntas retóricas) para indicar um tom interrogativo | usada no final de uma frase declarativa, indica que uma ação ou situação está em andamento | usada em frases para indicar uma pausa (muitas vezes em pares) | usada no final de uma frase declarativa para confirmar um fato e convencer o interlocutor (com um tom de indicação e exagero)}
  \seeref{ni2}
\end{EntryWithPhonetic}

%%%%%%%%%% 哪 %%%%%%%%%%
\subsection*{哪}\addcontentsline{loh}{figure}{哪 \dpy{nei3}}

\begin{EntryWithPhonetic}{哪}{nei3}{9}{⼝}
  \definition{part.}{qual? (interrogativo, seguido de classificador ou numeral-classificador)}
  \seeref{na3}
  \seeref{na5}
\end{EntryWithPhonetic}

%%%%%%%%%% 内 %%%%%%%%%%
\subsection*{内}\addcontentsline{loh}{figure}{内 \dpy{nei4}}

\begin{EntryWithPhonetic}{内}{nei4}{4}{⼌}[HSK 3]
  \definition*{s.}{Sobrenome: Nei}
  \definition{s.}{dentro; interior; parte interna ou lateral |  (antes de um substantivo ou verbo na formação de uma palavra composta) interno | (depois de um substantivo para indicar lugar, tempo, escopo ou limites) dentro; em | coração; mente | esposa ou parentes da esposa}
  \antonymref{外}{wai4}
\end{EntryWithPhonetic}

\begin{EntryWithPhonetic}{内部}{nei4bu4}{4,10}{⼌,⾢}[HSK 4]
  \definition{s.}{interior; dentro; interno; dentro de um determinado intervalo}
\end{EntryWithPhonetic}

\begin{EntryWithPhonetic}{内存}{nei4cun2}{4,6}{⼌,⼦}[HSK 7-9]
  \definition{s.}{RAM (\emph{random access memory}) | capacidade da RAM | memória do computador; memória interna, abreviação de 内存储器 | armazenamento interno}
  \seealsoref{内存储器}{nei4cun2chu3qi4}
  \seealsoref{随机存取存储器}{sui2ji1cun2qu3cun2chu3qi4}
  \seealsoref{随机存取记忆体}{sui2ji1cun2qu3ji4yi4ti3}
\end{EntryWithPhonetic}

\begin{EntryWithPhonetic}{内存储器}{nei4cun2chu3qi4}{4,6,12,16}{⼌,⼦,⼈,⼝}
  \definition{s.}{memória interna}
\end{EntryWithPhonetic}

\begin{EntryWithPhonetic}{内地}{nei4 di4}{4,6}{⼌,⼟}[HSK 6]
  \definition{s.}{interior; sertão | China continental (RPC excluindo Hong Kong e Macau, mas incluindo ilhas como Hainan) | Japão (usado em Taiwan durante a colonização japonesa)}
\end{EntryWithPhonetic}

\begin{EntryWithPhonetic}{内阁}{nei4ge2}{4,9}{⼌,⾨}[HSK 7-9]
  \definition{s.}{gabinete; em alguns países, o órgão executivo máximo é composto pelo primeiro-ministro e vários membros do gabinete (ministros, chefes de gabinete ou outros funcionários de alto escalão)}
\end{EntryWithPhonetic}

\begin{EntryWithPhonetic}{内涵}{nei4han2}{4,11}{⼌,⽔}[HSK 7-9]
  \definition{s.}{intenção; conotação; os atributos essenciais das coisas objetivas refletidos por um conceito}
\end{EntryWithPhonetic}

\begin{EntryWithPhonetic}{内行}{nei4hang2}{4,6}{⼌,⾏}[HSK 7-9]
  \definition{adj.}{habilidoso; proficiente; especialista em; amplo conhecimento e experiência em determinado assunto ou função}
  \definition{s.}{especialista; perito; craque; profissional}
\end{EntryWithPhonetic}

\begin{EntryWithPhonetic}{内科}{nei4ke1}{4,9}{⼌,⽲}[HSK 4]
  \definition{s.}{medicina geral; clínica geral; clínica médica}
\end{EntryWithPhonetic}

\begin{EntryWithPhonetic}{内幕}{nei4mu4}{4,13}{⼌,⼱}[HSK 7-9]
  \definition{s.}{o que acontece nos bastidores; histórias internas}
\end{EntryWithPhonetic}

\begin{EntryWithPhonetic}{内燃机}{nei4ran2ji1}{4,16,6}{⼌,⽕,⽊}
  \definition{s.}{motor de combustão interna}
\end{EntryWithPhonetic}

\begin{EntryWithPhonetic}{内容}{nei4rong2}{4,10}{⼌,⼧}[HSK 3]
  \definition[份,个,项]{s.}{conteúdo; substância; a substância ou significado contido em algo}
\end{EntryWithPhonetic}

\begin{EntryWithPhonetic}{内外}{nei4 wai4}{4,5}{⼌,⼣}[HSK 6]
  \definition{s.}{dentro e fora; nacional e estrangeiro; interno e externo | ao redor; aproximadamente; número aproximado de exibição}
\end{EntryWithPhonetic}

\begin{EntryWithPhonetic}{内向}{nei4xiang4}{4,6}{⼌,⼝}[HSK 7-9]
  \definition{adj.}{(economia, etc.) voltado para o mercado interno; introvertido; descreve a personalidade de uma pessoa como sendo quieta e relutante em expressar seus pensamentos}
\end{EntryWithPhonetic}

\begin{EntryWithPhonetic}{内心}{nei4 xin1}{4,4}{⼌,⼼}[HSK 3]
  \definition{s.}{coração; interior; íntimo do ser}
\end{EntryWithPhonetic}

\begin{EntryWithPhonetic}{内省}{nei4xing3}{4,9}{⼌,⽬}
  \definition{s.}{introspecção}
  \definition{v.}{refletir sobre si mesmo}
\end{EntryWithPhonetic}

\begin{EntryWithPhonetic}{内需}{nei4xu1}{4,14}{⼌,⾬}[HSK 7-9]
  \definition{s.}{Economia: demanda interna}
  \antonymref{外需}{wai4xu1}
\end{EntryWithPhonetic}

\begin{EntryWithPhonetic}{内衣}{nei4 yi1}{4,6}{⼌,⾐}[HSK 6]
  \definition[件,个]{s.}{roupa íntima}
\end{EntryWithPhonetic}

\begin{EntryWithPhonetic}{内在}{nei4zai4}{4,6}{⼌,⼟}[HSK 5]
  \definition{adj.}{intrínseco; algo que existe em si mesmo, mas que não pode ser descoberto através da observação direta | interno; imanente; difícil de perceber}
\end{EntryWithPhonetic}

\begin{EntryWithPhonetic}{内资}{nei4 zi1}{4,10}{⼌,⾙}
  \definition{s.}{capital nacional; financiamento interno; investimento de fontes nacionais}
  \antonymref{外资}{wai4 zi1}
\end{EntryWithPhonetic}

%%%%%%%%%% 那 %%%%%%%%%%
\subsection*{那}\addcontentsline{loh}{figure}{那 \dpy{nei4}}

\begin{EntryWithPhonetic}{那}{nei4}{6}{⾢}
  \definition{conj.}{então; o mesmo que 那么}
  \definition{pron.}{aquele; aquilo; A pronúncia coloquial de 那 (\dpy{na4})}
  \seeref{na1}
  \seeref{na3}
  \seeref{na4}
  \seeref{ne4}
  \seeref{nuo2}
  \seealsoref{那么}{na4 me5}
\end{EntryWithPhonetic}

%%%%%%%%%% 嫩 %%%%%%%%%%
\subsection*{嫩}\addcontentsline{loh}{figure}{嫩 \dpy{nen4}}

\begin{EntryWithPhonetic}{嫩}{nen4}{14}{⼥}[HSK 7-9]
  \definition{adj.}{terno; delicado; recém-nascido e frágil | macio; malpassado; alguns alimentos são rápidos de cozinhar e fáceis de mastigar | claro; suave; algumas cores são claras | sem habilidade; inexperiente}
\end{EntryWithPhonetic}

%%%%%%%%%% 能 %%%%%%%%%%
\subsection*{能}\addcontentsline{loh}{figure}{能 \dpy{neng2}}

\begin{EntryWithPhonetic}{能}{neng2}{10}{⾁}[HSK 1]
  \definition*{s.}{Sobrenome: Neng}
  \definition{adv.}{talvez}
  \definition{s.}{habilidade; capacidade; competência | potência; energia; em física, refere"-se à energia}
  \definition{v.}{poder fazer; ser capaz de | ser possível | entre 不 \dots 不 para expressar obrigação, certeza ou grande probabilidade | poder; ter permissão para | ser bom em fazer algo | permitir}
\end{EntryWithPhonetic}

\begin{EntryWithPhonetic}{能不能}{neng2 bu4 neng2}{10,4,10}{⾁,⼀,⾁}[HSK 3]
  \definition{adv.}{pode ou não pode\dots?}
\end{EntryWithPhonetic}

\begin{EntryWithPhonetic}{能否}{neng2 fou3}{10,7}{⾁,⼝}[HSK 6]
  \definition{adv.}{é possível; se ou não; pode ou não pode; Você consegue?; expressa dúvida, frequentemente usado em perguntas de sim ou não}
\end{EntryWithPhonetic}

\begin{EntryWithPhonetic}{能干}{neng2gan4}{10,3}{⾁,⼲}[HSK 4]
  \definition{adj.}{apto; capaz; competente}
\end{EntryWithPhonetic}

\begin{EntryWithPhonetic}{能够}{neng2 gou4}{10,11}{⾁,⼣}[HSK 2]
  \definition{v.}{poder; ser capaz de; indica que possui uma determinada capacidade ou que atingiu um determinado nível de eficiência | poder; ser capaz de; indica que algo é permitido sob certas condições ou por motivos razoáveis}
\end{EntryWithPhonetic}

\begin{EntryWithPhonetic}{能耗}{neng2hao4}{10,10}{⾁,⽾}[HSK 7-9]
  \definition{s.}{consumo de energia}
\end{EntryWithPhonetic}

\begin{EntryWithPhonetic}{能力}{neng2li4}{10,2}{⾁,⼒}[HSK 3]
  \definition[个,种]{s.}{habilidade; capacidade; aptidão; as condições subjetivas para ser competente para uma tarefa}
\end{EntryWithPhonetic}

\begin{EntryWithPhonetic}{能量}{neng2liang4}{10,12}{⾁,⾥}[HSK 5]
  \definition[种]{s.}{energia; quantidade de energia; Uma grandeza física que mede a capacidade da matéria de realizar trabalho | capacidade; competências; capacidade e papel que uma pessoa pode desempenhar}
\end{EntryWithPhonetic}

\begin{EntryWithPhonetic}{能耐}{neng2nai5}{10,9}{⾁,⽽}[HSK 7-9]
  \definition{adj.}{habilidade; capacidade; destreza; competência | talento}
\end{EntryWithPhonetic}

\begin{EntryWithPhonetic}{能人}{neng2ren2}{10,2}{⾁,⼈}[HSK 7-9]
  \definition{s.}{pessoa capaz | mentes capazes | pessoa de grande calibre}
\end{EntryWithPhonetic}

\begin{EntryWithPhonetic}{能上能下}{neng2shang4neng2xia4}{10,3,10,3}{⾁,⼀,⾁,⼀}
  \definition{s.}{pronto para aceitar qualquer trabalho, alto ou baixo}
\end{EntryWithPhonetic}

\begin{EntryWithPhonetic}{能源}{neng2yuan2}{10,13}{⾁,⽔}[HSK 7-9]
  \definition[种,个]{s.}{energia; fonte de energia; recursos que geram diversas formas de energia, incluindo energia mecânica, térmica, luminosa, eletromagnética e química; exemplos incluem combustíveis, energia hidrelétrica, energia solar e energia eólica}
\end{EntryWithPhonetic}

%%%%%%%%%% 尼 %%%%%%%%%%
\subsection*{尼}\addcontentsline{loh}{figure}{尼 \dpy{ni2}}

\begin{EntryWithPhonetic}{尼}{ni2}{5}{⼫}
  \definition[个,名,位]{s.}{freira budista | freira; convento de freiras}
\end{EntryWithPhonetic}

\begin{EntryWithPhonetic}{尼龙}{ni2long2}{5,5}{⼫,⿓}[HSK 7-9]
  \definition{s.}{Empréstimo linguístico: \emph{nylon}; náilon; poliamida; resinas que contêm ligações amida em suas moléculas, incluindo plásticos feitos com essas resinas, apresentam-se em diversas variedades}
\end{EntryWithPhonetic}

%%%%%%%%%% 呢 %%%%%%%%%%
\subsection*{呢}\addcontentsline{loh}{figure}{呢 \dpy{ni2}}

\begin{EntryWithPhonetic}{呢}{ni2}{8}{⼝}
  \definition{s.}{(tecido feito de) lã; tecido de lã (para roupas pesadas); tecido de lã pesada; revestimento ou roupa de lã}
  \seeref{ne5}
\end{EntryWithPhonetic}

%%%%%%%%%% 泥 %%%%%%%%%%
\subsection*{泥}\addcontentsline{loh}{figure}{泥 \dpy{ni2}}

\begin{EntryWithPhonetic}{泥}{ni2}{8}{⽔}[HSK 6]
  \definition*{s.}{Sobrenome: Ni}
  \definition{s.}{lama; atoleiro | pasta ou polpa; amassado | qualquer matéria pastosa; purê de vegetais ou frutas}
  \seeref{ni4}
\end{EntryWithPhonetic}

\begin{EntryWithPhonetic}{泥潭}{ni2tan2}{8,15}{⽔,⽔}[HSK 7-9]
  \definition{s.}{pântano; charco; atoleiro | poça de lama}
\end{EntryWithPhonetic}

\begin{EntryWithPhonetic}{泥土}{ni2tu3}{8,3}{⽔,⼟}[HSK 7-9]
  \definition[块,堆]{s.}{lama; solo; argila; terra}
\end{EntryWithPhonetic}

%%%%%%%%%% 你 %%%%%%%%%%
\subsection*{你}\addcontentsline{loh}{figure}{你 \dpy{ni3}}

\begin{EntryWithPhonetic}{你}{ni3}{7}{⼈}[HSK 1]
  \definition{pron.}{você (segunda pessoa do singular); refere"-se à pessoa com quem se está conversando | (referindo"-se a qualquer pessoa) você; um; qualquer um | com 我 ou 你 em estruturas paralelas para indicar várias ou muitas pessoas se comportando da mesma maneira}
  \seealsoref{您}{nin2}
  \seealsoref{我}{wo3}
\end{EntryWithPhonetic}

\begin{EntryWithPhonetic}{你的}{ni3 de5}{7,8}{⼈,⽩}
  \definition{pron.}{seu}
\end{EntryWithPhonetic}

\begin{EntryWithPhonetic}{你好}{ni3hao3}{7,6}{⼈,⼥}
  \definition{interj.}{``Olá!''; ``Oi!''}
\end{EntryWithPhonetic}

\begin{EntryWithPhonetic}{你们}{ni3men5}{7,5}{⼈,⼈}[HSK 1]
  \definition{pron.}{você (segunda pessoa do plural); refere"-se a mais de uma pessoa ou a várias pessoas, incluindo a outra parte}
\end{EntryWithPhonetic}

\begin{EntryWithPhonetic}{你们的}{ni3men5 de5}{7,5,8}{⼈,⼈,⽩}
  \definition{pron.}{vossos}
\end{EntryWithPhonetic}

%%%%%%%%%% 拟 %%%%%%%%%%
\subsection*{拟}\addcontentsline{loh}{figure}{拟 \dpy{ni3}}

\begin{EntryWithPhonetic}{拟}{ni3}{7}{⼿}[HSK 7-9]
  \definition{v.}{elaborar; projetar; conceber; rascunhar | Literário: pretender; planejar | imitar; reproduzir; simular | comparar; traçar um paralelo; com a intenção de ser ilícito | conjecturar; imaginar}
\end{EntryWithPhonetic}

\begin{EntryWithPhonetic}{拟定}{ni3ding4}{7,8}{⼿,⼧}[HSK 7-9]
  \definition{v.}{elaborar; redigir; planejar; formular | adivinhar; conjecturar; especular}
\end{EntryWithPhonetic}

%%%%%%%%%% 伲 %%%%%%%%%%
\subsection*{伲}\addcontentsline{loh}{figure}{伲 \dpy{ni4}}

\begin{EntryWithPhonetic}{伲}{ni4}{7}{⼈}
  \definition{pron.}{Dialeto: eu; nós; meu; nosso}
  \seealsoref{你}{ni3}
\end{EntryWithPhonetic}

%%%%%%%%%% 泥 %%%%%%%%%%
\subsection*{泥}\addcontentsline{loh}{figure}{泥 \dpy{ni4}}

\begin{EntryWithPhonetic}{泥}{ni4}{8}{⽔}
  \definition{adj.}{fanático; teimoso; obstinado; cabeçudo}
  \definition{v.}{cobrir ou rebocar com gesso, massa de vidraceiro, etc.}
  \seeref{ni2}
\end{EntryWithPhonetic}

%%%%%%%%%% 逆 %%%%%%%%%%
\subsection*{逆}\addcontentsline{loh}{figure}{逆 \dpy{ni4}}

\begin{EntryWithPhonetic}{逆}{ni4}{9}{⾡}[HSK 7-9]
  \definition{adj.}{contrário; contra; oposto; inverso | traidor; rebelde}
  \definition{adv.}{antecipadamente; com antecedência}
  \definition{s.}{traidor; rebelde}
  \definition{v.}{ir contra; opor-se; desobedecer; resistir; desafiar | Lliterário: saudar; cumprimentar}
  \antonymref{顺}{shun4}
\end{EntryWithPhonetic}

\begin{EntryWithPhonetic}{逆境}{ni4jing4}{9,14}{⾡,⼟}
  \definition[对]{s.}{adversidade; tribulação; circunstâncias adversas; circunstâncias desfavoráveis}
\end{EntryWithPhonetic}

%%%%%%%%%% 匿 %%%%%%%%%%
\subsection*{匿}\addcontentsline{loh}{figure}{匿 \dpy{ni4}}

\begin{EntryWithPhonetic}{匿}{ni4}{10}{⼖}
  \definition{v.}{esconder; ocultar; manter em segredo}
\end{EntryWithPhonetic}

\begin{EntryWithPhonetic}{匿名}{ni4ming2}{10,6}{⼖,⼝}[HSK 7-9]
  \definition{v.}{permanecer anônimo; não revelar o próprio nome}
\end{EntryWithPhonetic}

%%%%%%%%%% 年 %%%%%%%%%%
\subsection*{年}\addcontentsline{loh}{figure}{年 \dpy{nian2}}

\begin{EntryWithPhonetic}{年}{nian2}{6}{⼲}[HSK 1]
  \definition*{s.}{Sobrenome: Nian}
  \definition{clas.}{ano; usado para calcular o número de anos}
  \definition{s.}{ano | idade | um período (época) da história | colheita anual | Ano Novo | artigos para o dia de Ano Novo | um período da vida de uma pessoa; fases da vida humana divididas por idade}
\end{EntryWithPhonetic}

\begin{EntryWithPhonetic}{年初}{nian2 chu1}{6,7}{⼲,⾐}[HSK 3]
  \definition{s.}{o começo do ano; os primeiros dias do ano}
\end{EntryWithPhonetic}

\begin{EntryWithPhonetic}{年代}{nian2dai4}{6,5}{⼲,⼈}[HSK 3]
  \definition[个]{s.}{idade; anos; tempo; um período de tempo com características distintas na história | uma década de um século; período de dez anos}
\end{EntryWithPhonetic}

\begin{EntryWithPhonetic}{年底}{nian2 di3}{6,8}{⼲,⼴}[HSK 3]
  \definition[个]{s.}{fim de ano; o fim do ano; geralmente os últimos dias de dezembro ou o fim do ano}
\end{EntryWithPhonetic}

\begin{EntryWithPhonetic}{年度}{nian2du4}{6,9}{⼲,⼴}[HSK 5]
  \definition{s.}{ano; de acordo com a natureza e as necessidades de um negócio, há um prazo de doze meses com data de início e término definidas}
\end{EntryWithPhonetic}

\begin{EntryWithPhonetic}{年画}{nian2hua4}{6,8}{⼲,⽥}[HSK 7-9]
  \definition{s.}{fotos de Ano Novo (ou Festival da Primavera); durante o Ano Novo Lunar, são afixadas imagens que retratam alegria e prosperidade}
  \seealsoref{年画儿}{nian2hua4r5}
\end{EntryWithPhonetic}

\begin{EntryWithPhonetic}{年画儿}{nian2hua4r5}{6,8,2}{⼲,⽥,⼉}
  \definition{s.}{foto de Ano Novo (Festival da Primavera)}
\end{EntryWithPhonetic}

\begin{EntryWithPhonetic}{年货}{nian2huo4}{6,8}{⼲,⾙}
  \definition{s.}{mercadorias vendidas no Ano Novo Chinês}
\end{EntryWithPhonetic}

\begin{EntryWithPhonetic}{年级}{nian2ji2}{6,6}{⼲,⽷}[HSK 2]
  \definition[个]{s.}{série; ano; níveis divididos de acordo com o tempo de estudo dos alunos na escola}
\end{EntryWithPhonetic}

\begin{EntryWithPhonetic}{年纪}{nian2ji4}{6,6}{⼲,⽷}[HSK 3]
  \definition[把,个]{s.}{idade (de uma pessoa)}
\end{EntryWithPhonetic}

\begin{EntryWithPhonetic}{年龄}{nian2ling2}{6,13}{⼲,⿒}[HSK 5]
  \definition[个,段]{s.}{idade; animais, plantas e outros seres vivos vivem e crescem no mundo durante um determinado número de anos}
\end{EntryWithPhonetic}

\begin{EntryWithPhonetic}{年迈}{nian2mai4}{6,6}{⼲,⾡}[HSK 7-9]
  \definition{adj.}{velho; idoso}
\end{EntryWithPhonetic}

\begin{EntryWithPhonetic}{年前}{nian2 qian2}{6,9}{⼲,⼑}[HSK 5]
  \definition{s.}{(pouco) antes da virada do ano | antes do final do ano | antes do ano novo}
\end{EntryWithPhonetic}

\begin{EntryWithPhonetic}{年轻}{nian2qing1}{6,9}{⼲,⾞}[HSK 2]
  \definition{adj.}{jovem; não muito velho (geralmente se refere a pessoas entre 10 e 20 anos)}
\end{EntryWithPhonetic}

\begin{EntryWithPhonetic}{年限}{nian2xian4}{6,8}{⼲,⾩}[HSK 7-9]
  \definition{s.}{limite de idade; número fixo de anos; o número de anos especificado ou usado como padrão geral}
\end{EntryWithPhonetic}

\begin{EntryWithPhonetic}{年薪}{nian2xin1}{6,16}{⼲,⾋}[HSK 7-9]
  \definition{s.}{salário anual; remuneração anual}[他的年薪已经达到了五十万元。===Seu salário anual chegou a 500.000 yuans.]
\end{EntryWithPhonetic}

\begin{EntryWithPhonetic}{年夜饭}{nian2ye4fan4}{6,8,7}{⼲,⼣,⾷}[HSK 7-9]
  \definition[顿,桌]{s.}{jantar de família na véspera de Ano Novo; um jantar especial realizado na véspera de Ano Novo (na noite de 31 de dezembro)}
\end{EntryWithPhonetic}

\begin{EntryWithPhonetic}{年终}{nian2zhong1}{6,8}{⼲,⽷}[HSK 7-9]
  \definition{s.}{fim de ano; fim do ano}
\end{EntryWithPhonetic}

%%%%%%%%%% 粘 %%%%%%%%%%
\subsection*{粘}\addcontentsline{loh}{figure}{粘 \dpy{nian2}}

\begin{EntryWithPhonetic}{粘}{nian2}{11}{⽶}
  \variantof{黏}
\end{EntryWithPhonetic}

%%%%%%%%%% 黏 %%%%%%%%%%
\subsection*{黏}\addcontentsline{loh}{figure}{黏 \dpy{nian2}}

\begin{EntryWithPhonetic}{黏}{nian2}{17}{⿉}[HSK 7-9]
  \definition{adj.}{pegajoso; adesivo; glutinoso}
\end{EntryWithPhonetic}

%%%%%%%%%% 碾 %%%%%%%%%%
\subsection*{碾}\addcontentsline{loh}{figure}{碾 \dpy{nian3}}

\begin{EntryWithPhonetic}{碾}{nian3}{15}{⽯}
  \definition[台,个]{s.}{rolo e mó; rolo de pedra | rolo compressor}
  \definition{v.}{moer ou descascar com um rolo; esmagar | (literário) cortar e polir (jade, vidro, etc.) | achatar | pisar; pisotear, 轧}
  \seealsoref{辗}{zhan3}
\end{EntryWithPhonetic}

\begin{EntryWithPhonetic}{碾碎}{nian3sui4}{15,13}{⽯,⽯}
  \definition{v.}{pulverizar | esmagar}
\end{EntryWithPhonetic}

%%%%%%%%%% 廿 %%%%%%%%%%
\subsection*{廿}\addcontentsline{loh}{figure}{廿 \dpy{nian4}}

\begin{EntryWithPhonetic}{廿}{nian4}{4}{⼶}
  \definition{num.}{(dialeto) vinte; 20}
\end{EntryWithPhonetic}

%%%%%%%%%% 念 %%%%%%%%%%
\subsection*{念}\addcontentsline{loh}{figure}{念 \dpy{nian4}}

\begin{EntryWithPhonetic}{念}{nian4}{8}{⼼}[HSK 3]
  \definition*{s.}{Sobrenome: Nian}
  \definition{num.}{vinte; 20; capitalização do número 廿}
  \definition{s.}{ideia; pensamento; pensamentos ou intenções internas}
  \definition{v.}{ler em voz alta | estudar; frequentar a escola | considerar; levar em conta | sentir falta; pensar em; pensar sobre; pensar frequentemente sobre}
  \seealsoref{廿}{nian4}
\end{EntryWithPhonetic}

\begin{EntryWithPhonetic}{念念不忘}{nian4nian4-bu2wang4}{8,8,4,7}{⼼,⼼,⼀,⼼}[HSK 7-9]
  \definition{expr.}{inesquecível; ter sempre em mente (expressão idiomática); nunca se esqueça}
\end{EntryWithPhonetic}

\begin{EntryWithPhonetic}{念书}{nian4/shu1}{8,4}{⼼,⼄}[HSK 7-9]
  \definition{v.+compl.}{estudar; ir à escola; frequentar a escola | ler livros}
\end{EntryWithPhonetic}

\begin{EntryWithPhonetic}{念头}{nian4tou5}{8,5}{⼼,⼤}[HSK 7-9]
  \definition[个,种]{s.}{ideia; pensamento; intenção; plano}
\end{EntryWithPhonetic}

%%%%%%%%%% 娘 %%%%%%%%%%
\subsection*{娘}\addcontentsline{loh}{figure}{娘 \dpy{niang2}}

\begin{EntryWithPhonetic}{娘}{niang2}{10}{⼥}[HSK 7-9]
  \definition{adj.}{efeminado; maricas}
  \definition{s.}{mãe | forma de tratamento para uma mulher idosa casada; dirigindo-se a uma mulher casada mais velha ou de idade mais avançada | jovem mulher}
\end{EntryWithPhonetic}

%%%%%%%%%% 酿 %%%%%%%%%%
\subsection*{酿}\addcontentsline{loh}{figure}{酿 \dpy{niang2}}

\begin{EntryWithPhonetic}{酿}{niang2}{14}{⾣}
  \definition{v.}{fermentar; preparar (vinho de arroz)}
  \seeref{niang4}
\end{EntryWithPhonetic}

\begin{EntryWithPhonetic}{酿}{niang4}{14}{⾣}
  \definition{s.}{vinho}
  \definition{v.}{fazer (vinho); fabricar (cerveja) | produzir (mel) | levar a; resultar em; formar gradualmente}
  \seeref{niang2}
\end{EntryWithPhonetic}

\begin{EntryWithPhonetic}{酿造}{niang4zao4}{14,10}{⾣,⾡}[HSK 7-9]
  \definition{v.}{fabricar (cerveja, etc.); produzir (vinho, vinagre, etc.); fabricar utilizando fermentação}
\end{EntryWithPhonetic}

%%%%%%%%%% 鸟 %%%%%%%%%%
\subsection*{鸟}\addcontentsline{loh}{figure}{鸟 \dpy{niao3}}

\begin{EntryWithPhonetic}{鸟}{niao3}{5}{⿃}[HSK 2][Kangxi 196]
  \definition*{s.}{Sobrenome: Niao}
  \definition[只,群]{s.}{pássaro; ave}
  \seeref{diao3}
\end{EntryWithPhonetic}

\begin{EntryWithPhonetic}{鸟巢}{niao3chao2}{5,11}{⿃,⼮}[HSK 7-9]
  \definition[个,些]{s.}{ninho de pássaro; ninhos usados por pássaros para pôr ovos e incubar seus filhotes}
\end{EntryWithPhonetic}

\begin{EntryWithPhonetic}{鸟儿}{niao3r5}{5,2}{⿃,⼉}
  \definition[只]{s.}{pássaro | ave}
\end{EntryWithPhonetic}

%%%%%%%%%% 尿 %%%%%%%%%%
\subsection*{尿}\addcontentsline{loh}{figure}{尿 \dpy{niao4}}

\begin{EntryWithPhonetic}{尿}{niao4}{7}{⼫}[HSK 7-9]
  \definition[泡]{s.}{urina}
  \definition{v.}{urinar; fazer xixi | Coloquial: fazer água; urinar; mijar}
  \seeref{sui1}
\end{EntryWithPhonetic}

%%%%%%%%%% 溺 %%%%%%%%%%
\subsection*{溺}\addcontentsline{loh}{figure}{溺 \dpy{niao4}}

\begin{EntryWithPhonetic}{溺}{niao4}{13}{⽔}
  \variantof{尿}
\end{EntryWithPhonetic}

%%%%%%%%%% 捏 %%%%%%%%%%
\subsection*{捏}\addcontentsline{loh}{figure}{捏 \dpy{nie1}}

\begin{EntryWithPhonetic}{捏}{nie1}{10}{⼿}[HSK 7-9]
  \definition{v.}{beliscar; segurar entre os dedos; usar o polegar e os outros dedos para pinçar | moldar; amassar com os dedos; usar os dedos para moldar o objeto macio | fabricar; compor; apresentar deliberadamente uma declaração falsa como se fosse um fato | juntar; unir; unir duas coisas ou duas pessoas | beliscar; usar as mãos para empurrar, pressionar, beliscar e amassar o corpo pode promover a circulação sanguínea, aumentar a resistência da pele e regular a função nervosa; pressionar firmemente com a palma da mão}
\end{EntryWithPhonetic}

%%%%%%%%%% 您 %%%%%%%%%%
\subsection*{您}\addcontentsline{loh}{figure}{您 \dpy{nin2}}

\begin{EntryWithPhonetic}{您}{nin2}{11}{⼼}[HSK 1]
  \definition{pron.}{você; a forma de tratamento respeitosa da segunda pessoa do singular 你}
  \seealsoref{你}{ni3}
\end{EntryWithPhonetic}

%%%%%%%%%% 宁 %%%%%%%%%%
\subsection*{宁}\addcontentsline{loh}{figure}{宁 \dpy{ning2}}

\begin{EntryWithPhonetic}{宁}{ning2}{5}{⼧}
  \definition*{s.}{Região Autônoma de Ningxia Hui, abreviação de 宁夏回族自治区 | outro nome para Nanquim, 南京 | Sobrenome: Ning}
  \definition{adj.}{calmo, pacífico, sereno; tranquilo}
  \definition{v.}{Literário: fazer uma visita (aos pais ou aos mais velhos); | Literário: pacificar; apaziguar}
  \seeref{ning4}
  \seealsoref{南京}{nan2jing1}
  \seealsoref{宁夏回族自治区}{ning2xia4 hui2zu2 zi4zhi4qu1}
\end{EntryWithPhonetic}

\begin{EntryWithPhonetic}{宁静}{ning2 jing4}{5,14}{⼧,⾭}[HSK 4]
  \definition{adj.}{calmo; tranquilo; pacífico}
\end{EntryWithPhonetic}

\begin{EntryWithPhonetic}{宁夏回族自治区}{ning2xia4 hui2zu2 zi4zhi4qu1}{5,10,6,11,6,8,4}{⼧,⼢,⼞,⽅,⾃,⽔,⼖}
  \definition*{s.}{Região Autônoma de Ningxia Hui}
\end{EntryWithPhonetic}

%%%%%%%%%% 拧 %%%%%%%%%%
\subsection*{拧}\addcontentsline{loh}{figure}{拧 \dpy{ning2}}

\begin{EntryWithPhonetic}{拧}{ning2}{8}{⼿}[HSK 7-9]
  \definition{v.}{torcer | beliscar; torcer a pele com os dedos e virá-la com força}
  \seeref{ning3}
  \seeref{ning4}
\end{EntryWithPhonetic}

%%%%%%%%%% 柠 %%%%%%%%%%
\subsection*{柠}\addcontentsline{loh}{figure}{柠 \dpy{ning2}}

\begin{EntryWithPhonetic}{柠}{ning2}{9}{⽊}
  \definition{s.}{limão}
\end{EntryWithPhonetic}

\begin{EntryWithPhonetic}{柠檬}{ning2meng2}{9,17}{⽊,⽊}
  \definition[个,片,只]{s.}{limão}
\end{EntryWithPhonetic}

%%%%%%%%%% 凝 %%%%%%%%%%
\subsection*{凝}\addcontentsline{loh}{figure}{凝 \dpy{ning2}}

\begin{EntryWithPhonetic}{凝}{ning2}{16}{⼎}
  \definition{adv.}{atentamente; atenção fixa}
  \definition{v.}{congelar; coagular; coalhar; condensar | contemplar; olhar pensativamente}
\end{EntryWithPhonetic}

\begin{EntryWithPhonetic}{凝固}{ning2gu4}{16,8}{⼎,⼞}[HSK 7-9]
  \definition{v.}{estagnar; parar, não se move mais nem muda de direção}
\end{EntryWithPhonetic}

\begin{EntryWithPhonetic}{凝聚}{ning2ju4}{16,14}{⼎,⽿}[HSK 7-9]
  \definition{v.}{condensar (o vapor); coagular; coalhar (os fluidos) | reunir; acumular; juntar}
\end{EntryWithPhonetic}

%%%%%%%%%% 拧 %%%%%%%%%%
\subsection*{拧}\addcontentsline{loh}{figure}{拧 \dpy{ning3}}

\begin{EntryWithPhonetic}{拧}{ning3}{8}{⼿}[HSK 7-9]
  \definition{adj.}{errado; equivocado; de cabeça para baixo; oposto}
  \definition{v.}{torcer; parafusar | divergir; discordar; estar em desacordo}
  \seeref{ning2}
  \seeref{ning4}
\end{EntryWithPhonetic}

\begin{EntryWithPhonetic}{拧开}{ning3kai1}{8,4}{⼿,⼶}
  \definition{v.}{ligar ou desligar (girando um botão) | girar (a maçaneta de uma porta) | abrir (uma torneira) | desenroscar (uma tampa) | desaparafusar | arrancar à força}
\end{EntryWithPhonetic}

%%%%%%%%%% 宁 %%%%%%%%%%
\subsection*{宁}\addcontentsline{loh}{figure}{宁 \dpy{ning4}}

\begin{EntryWithPhonetic}{宁}{ning4}{5}{⼧}
  \definition*{s.}{Sobrenome: Ning}
  \definition{adv.}{preferir; preferiria; melhor | Literário: como pôde ser; poderia haver}
  \seeref{ning2}
\end{EntryWithPhonetic}

\begin{EntryWithPhonetic}{宁可}{ning4ke3}{5,5}{⼧,⼝}[HSK 7-9]
  \definition{adv.}{melhor; preferiria; isso indica a escolha feita após comparar as vantagens e desvantagens de dois lados (frequentemente repetindo 与其 no texto anterior ou 也不 no texto seguinte)}
  \seealsoref{宁可…也不…}{ning4ke3 ye3bu4}
  \seealsoref{与其…宁可…}{yu3qi2 ning4ke3}
\end{EntryWithPhonetic}

\begin{EntryWithPhonetic}{宁可…也不…}{ning4ke3 ye3bu4}{5,5,3,4}{⼧,⼝,⼄,⼀}
  \definition{conj.}{preferiria\dots do que\dots}
\end{EntryWithPhonetic}

\begin{EntryWithPhonetic}{宁可…也要…}{ning4ke3 ye3yao4}{5,5,3,9}{⼧,⼝,⼄,⾑}
  \definition{conj.}{mesmo que tenhamos que\dots nós iremos\dots}
\end{EntryWithPhonetic}

\begin{EntryWithPhonetic}{宁肯}{ning4ken3}{5,8}{⼧,⾁}
  \definition{conj.}{mais\dots do que\dots, melhor\dots do que\dots}
\end{EntryWithPhonetic}

\begin{EntryWithPhonetic}{宁愿}{ning4yuan4}{5,14}{⼧,⽕}[HSK 7-9]
  \definition{adv.}{em vez disso; melhor; mais dispostos e mais ansiosos para escolher uma determinada situação ou abordagem}[他宁愿看书,也不看电视。===Ele prefere ler livros a assistir televisão.]
\end{EntryWithPhonetic}

%%%%%%%%%% 拧 %%%%%%%%%%
\subsection*{拧}\addcontentsline{loh}{figure}{拧 \dpy{ning4}}

\begin{EntryWithPhonetic}{拧}{ning4}{8}{⼿}
  \definition{adj.}{teimoso}
  \seeref{ning2}
  \seeref{ning3}
\end{EntryWithPhonetic}

%%%%%%%%%% 牛 %%%%%%%%%%
\subsection*{牛}\addcontentsline{loh}{figure}{牛 \dpy{niu2}}

\begin{EntryWithPhonetic}{牛}{niu2}{4}{⽜}[HSK 3,5][Kangxi 93]
  \definition*{s.}{Sobrenome: Niu}
  \definition{adj.}{muito capaz ou bom; descreve pessoas ou coisas como sendo muito capazes, muito competentes | teimoso; arrogante; descreve uma pessoa que é muito orgulhosa ou muito insistente em suas opiniões, difícil de mudar}
  \definition{clas.}{Newton (medida física de força)}
  \definition[头]{s.}{gado; boi | niu (nona das vinte e oito constelações em que a esfera celeste foi dividida, consistindo de seis estrelas, três em Áries e três em Sagitário)}
\end{EntryWithPhonetic}

\begin{EntryWithPhonetic}{牛顿}{niu2dun4}{4,10}{⽜,⾴}
  \definition*{s.}{Newton (nome) | N; Newton, unidade de força do SI}
\end{EntryWithPhonetic}

\begin{EntryWithPhonetic}{牛郎织女}{niu2 lang2 zhi1nv3}{4,8,8,3}{⽜,⾢,⽷,⼥}
  \definition*{s.}{Vaqueiro e Tecelã (personagens de contos folclóricos) | Altair e Vega (estrelas)}[我们这些牛郎织女都恨透了那条无情的``天河''。===Nós, o Vaqueiro e a Tecelã, odiamos a implacável ``Via Láctea''.]
  \definition{s.}{marido e mulher que vivem longe um do outro}
\end{EntryWithPhonetic}

\begin{EntryWithPhonetic}{牛奶}{niu2nai3}{4,5}{⽜,⼥}[HSK 1]
  \definition[杯,袋,瓶,盒,箱,桶]{s.}{leite}
\end{EntryWithPhonetic}

\begin{EntryWithPhonetic}{牛人}{niu2ren2}{4,2}{⽜,⼈}
  \definition{s.}{(coloquial) o cara | verdadeiro especialista | \emph{badass}}
\end{EntryWithPhonetic}

\begin{EntryWithPhonetic}{牛肉}{niu2rou4}{4,6}{⽜,⾁}
  \definition{s.}{carne de vaca | bife}
\end{EntryWithPhonetic}

\begin{EntryWithPhonetic}{牛仔裤}{niu2zai3ku4}{4,5,12}{⽜,⼈,⾐}[HSK 5]
  \definition[条]{s.}{calças jeans; calças geralmente feitas de tecido jeans azul grosso}
\end{EntryWithPhonetic}

%%%%%%%%%% 扭 %%%%%%%%%%
\subsection*{扭}\addcontentsline{loh}{figure}{扭 \dpy{niu3}}

\begin{EntryWithPhonetic}{扭}{niu3}{7}{⼿}[HSK 6]
  \definition{v.}{virar-se; girar | torcer; girar | torcer; luxar | rolar; balançar (ao caminhar) | agarrar; pegar;  lutar com}
\end{EntryWithPhonetic}

\begin{EntryWithPhonetic}{扭曲}{niu3qu1}{7,6}{⼿,⽈}[HSK 7-9]
  \definition{v.}{torcer; distorcer; deformação torsional | deformar; distorcer; deturpar; a distorção causa deformação e distorção}
\end{EntryWithPhonetic}

\begin{EntryWithPhonetic}{扭头}{niu3/tou2}{7,5}{⼿,⼤}[HSK 7-9]
  \definition{v.+compl.}{desviar o olhar; virar as costas | virar (em volta) | dar meia-volta | virar a cabeça}
\end{EntryWithPhonetic}

\begin{EntryWithPhonetic}{扭转}{niu3zhuan3}{7,8}{⼿,⾞}[HSK 7-9]
  \definition{v.}{virar-se; inverter a marcha | dar meia-volta; reverter; corrigir situações anormais ou alterar circunstâncias desfavoráveis}
\end{EntryWithPhonetic}

%%%%%%%%%% 纽 %%%%%%%%%%
\subsection*{纽}\addcontentsline{loh}{figure}{纽 \dpy{niu3}}

\begin{EntryWithPhonetic}{纽}{niu3}{7}{⽷}
  \definition*{s.}{Sobrenome: Niu}
  \definition[条]{s.}{alça; puxador | botão | vínculo; amarra | fruto ou melão recém-formado em uma trepadeira | Obsoleto: consoante inicial (linguística) | ligação; eixo}
  \definition{v.}{Dialeto: abotoar}
\end{EntryWithPhonetic}

\begin{EntryWithPhonetic}{纽带}{niu3dai4}{7,9}{⽷,⼱}[HSK 7-9]
  \definition[根]{s.}{laço; elo; vínculo; ligação; conexão; pessoas ou coisas que podem servir como elo}
\end{EntryWithPhonetic}

\begin{EntryWithPhonetic}{纽扣}{niu3kou4}{7,6}{⽷,⼿}[HSK 7-9]
  \definition[粒,颗,枚,个]{s.}{botão; objetos pequenos, redondos ou planos, que podem ser usados para fechar roupas, etc.}
\end{EntryWithPhonetic}

%%%%%%%%%% 农 %%%%%%%%%%
\subsection*{农}\addcontentsline{loh}{figure}{农 \dpy{nong2}}

\begin{EntryWithPhonetic}{农}{nong2}{6}{⼍}
  \definition*{s.}{Sobrenome: Nong}
  \definition{s.}{agricultura; criação de animais | camponês; fazendeiro}
\end{EntryWithPhonetic}

\begin{EntryWithPhonetic}{农产品}{nong2 chan3 pin3}{6,6,9}{⼍,⼇,⼝}[HSK 5]
  \definition[批]{s.}{produtos agrícolas}
\end{EntryWithPhonetic}

\begin{EntryWithPhonetic}{农场}{nong2chang3}{6,6}{⼍,⼟}[HSK 7-9]
  \definition[座,家]{s.}{fazenda; empresas que utilizam máquinas e se dedicam à produção agrícola em larga escala}
\end{EntryWithPhonetic}

\begin{EntryWithPhonetic}{农村}{nong2cun1}{6,7}{⼍,⽊}[HSK 3]
  \definition{s.}{aldeia; campo; área rural; locais onde vivem os trabalhadores principalmente dedicados à produção agrícola}
\end{EntryWithPhonetic}

\begin{EntryWithPhonetic}{农历}{nong2li4}{6,4}{⼍,⼚}[HSK 7-9]
  \definition{s.}{o calendário lunar; o calendário tradicional chinês}
\end{EntryWithPhonetic}

\begin{EntryWithPhonetic}{农民}{nong2min2}{6,5}{⼍,⽒}[HSK 3]
  \definition[个,位,名,些]{s.}{fazendeiro; camponês; campesinato; trabalhadores que participam da produção agrícola há muito tempo}
\end{EntryWithPhonetic}

\begin{EntryWithPhonetic}{农民工}{nong2min2gong1}{6,5,3}{⼍,⽒,⼯}[HSK 7-9]
  \definition{s.}{trabalhadores migrantes; trabalhadores com registro de domicílio agrícola que exercem atividades não agrícolas em áreas urbanas}
\end{EntryWithPhonetic}

\begin{EntryWithPhonetic}{农业}{nong2ye4}{6,5}{⼍,⼀}[HSK 3]
  \definition{s.}{agricultura}
\end{EntryWithPhonetic}

\begin{EntryWithPhonetic}{农作物}{nong2zuo4wu4}{6,7,8}{⼍,⼈,⽜}[HSK 7-9]
  \definition{s.}{culturas agrícolas; plantas agrícolas; termo genérico para diversas culturas agrícolas, como grãos, oleaginosas, hortaliças, algodão e tabaco}
\end{EntryWithPhonetic}

%%%%%%%%%% 浓 %%%%%%%%%%
\subsection*{浓}\addcontentsline{loh}{figure}{浓 \dpy{nong2}}

\begin{EntryWithPhonetic}{浓}{nong2}{9}{⽔}[HSK 4]
  \definition{adj.}{denso; espesso; concentrado; um líquido ou gás que contém mais de um determinado ingrediente | grande; forte; profundo (de grau ou extensão) | profundo; (algumas cores) escuro}
\end{EntryWithPhonetic}

\begin{EntryWithPhonetic}{浓厚}{nong2hou4}{9,9}{⽔,⼚}[HSK 7-9]
  \definition{adj.}{muito forte (de cor, interesse, intenção, atmosfera, etc.); descreve algo como sendo de grande interesse ou significativo em termos de cor, visibilidade, atmosfera, etc. | espesso; denso; descreve fumaça, neblina ou nuvens como abundantes e densas}
\end{EntryWithPhonetic}

\begin{EntryWithPhonetic}{浓缩}{nong2suo1}{9,14}{⽔,⽷}[HSK 7-9]
  \definition{v.}{concentrar; condensar; a concentração de uma solução é aumentada pela evaporação do solvente através de métodos como o aquecimento | enriquecer; geralmente se refere à redução das partes desnecessárias de algo, de modo que o conteúdo das partes necessárias aumente relativamente}
\end{EntryWithPhonetic}

\begin{EntryWithPhonetic}{浓郁}{nong2yu4}{9,8}{⽔,⾢}[HSK 7-9]
  \definition{adj.}{(perfume, fragrância, aroma, etc.) forte | denso; espesso; exuberante | (interesse, cor, etc.) forte; rico}
\end{EntryWithPhonetic}

\begin{EntryWithPhonetic}{浓重}{nong2zhong4}{9,9}{⽔,⾥}[HSK 7-9]
  \definition{adj.}{denso; espesso; forte; (fumaça, cheiro, cor, etc.) muito denso e pesado}
\end{EntryWithPhonetic}

%%%%%%%%%% 弄 %%%%%%%%%%
\subsection*{弄}\addcontentsline{loh}{figure}{弄 \dpy{nong4}}

\begin{EntryWithPhonetic}{弄}{nong4}{7}{⼶}[HSK 2]
  \definition{v.}{fazer, realizar; tratar; organizar | obter; buscar; tentar conseguir; encontrar uma maneira de conseguir | brincar com; enganar | pregar uma peça; brincar; manipular | mexer com; perturbar}
  \seeref{long4}
\end{EntryWithPhonetic}

\begin{EntryWithPhonetic}{弄虚作假}{nong4xu1-zuo4jia3}{7,11,7,11}{⼶,⾌,⼈,⼈}[HSK 7-9]
  \definition{expr.}{recorrer ao engano; empregar artimanhas; usar de artifícios; pregar peças; praticar fraude; humilhar-se; fazer truques e enganar as pessoas}
\end{EntryWithPhonetic}

%%%%%%%%%% 奴 %%%%%%%%%%
\subsection*{奴}\addcontentsline{loh}{figure}{奴 \dpy{nu2}}

\begin{EntryWithPhonetic}{奴}{nu2}{5}{⼥}
  \definition{pron.}{Obsoleto: eu, mim, sua humilde serva, esta garota (nome autoproclamado de uma garota, frequentemente encontrado na literatura vernácula antiga)}
  \definition[个,位]{s.}{servo; escravo; pessoas da sociedade antiga que eram oprimidas, exploradas e escravizadas, e que não tinham liberdade pessoal nem direitos políticos}
  \definition{v.}{escravizar; tornar alguém escravo | escravizar;  tratar e usar como escravo}
  \seealsoref{主}{zhu3}
  \antonymref{主}{zhu3}
\end{EntryWithPhonetic}

\begin{EntryWithPhonetic}{奴隶}{nu2li4}{5,8}{⼥,⾪}[HSK 7-9]
  \definition[个]{s.}{escravo; as pessoas que trabalhavam para donos de escravos e não tinham liberdade pessoal eram frequentemente compradas, vendidas ou mortas arbitrariamente por esses donos}
\end{EntryWithPhonetic}

%%%%%%%%%% 努 %%%%%%%%%%
\subsection*{努}\addcontentsline{loh}{figure}{努 \dpy{nu3}}

\begin{EntryWithPhonetic}{努}{nu3}{7}{⼒}
  \definition{v.}{(coloquial) aplicar (a força de alguém); exercer (o esforço de alguém) | (dialeto) machucar-se por esforço excessivo | projetar-se; inchar | aplicar (força); exercer (esforço); usar}
  \seealsoref{呶}{nao2}
\end{EntryWithPhonetic}

\begin{EntryWithPhonetic}{努力}{nu3li4}{7,2}{⼒,⼒}[HSK 2]
  \definition{adj.}{extenuante; árduo | diligente; trabalhador; quem faz as coisas com o máximo de capacidade ou esforço possível}
  \definition{s.}{esforço; tentativa; fazer o melhor possível}
  \definition{v.}{fazer grandes esforços; esforçar"-se; empenhar"-se | esforçar"-se; usar toda a força possível}
\end{EntryWithPhonetic}

%%%%%%%%%% 怒 %%%%%%%%%%
\subsection*{怒}\addcontentsline{loh}{figure}{怒 \dpy{nu4}}

\begin{EntryWithPhonetic}{怒}{nu4}{9}{⼼}
  \definition{adj.}{zangado; furioso | feroz; forte; descreve um forte impulso}
  \definition{adv.}{com força; vigorosamente; dinamicamente | com raiva}
  \definition{s.}{raiva; fúria}
  \definition{v.}{enfurecer-se; ficar com raiva}
\end{EntryWithPhonetic}

\begin{EntryWithPhonetic}{怒放}{nu4fang4}{9,8}{⼼,⽅}
  \definition{v.}{florescer em plena floração}
\end{EntryWithPhonetic}

\begin{EntryWithPhonetic}{怒骂}{nu4ma4}{9,9}{⼼,⾺}
  \definition{v.}{praguejar de raiva}
\end{EntryWithPhonetic}

%%%%%%%%%% 暖 %%%%%%%%%%
\subsection*{暖}\addcontentsline{loh}{figure}{暖 \dpy{nuan3}}

\begin{EntryWithPhonetic}{暖}{nuan3}{13}{⽇}[HSK 5]
  \definition{adj.}{caloroso; cordial}
  \definition{v.}{aquecer; esquentar; aquecer algo ou aquecer o corpo}
\end{EntryWithPhonetic}

\begin{EntryWithPhonetic}{暖烘烘}{nuan3hong1hong1}{13,10,10}{⽇,⽕,⽕}[HSK 7-9]
  \definition{adj.}{agradável e quentinho; aconchegante; acolhedor | emocionante}
\end{EntryWithPhonetic}

\begin{EntryWithPhonetic}{暖和}{nuan3huo5}{13,8}{⽇,⼝}[HSK 3]
  \definition{adj.}{morno; nem frio nem quente}
  \definition{v.}{aquecer; esquentar}
\end{EntryWithPhonetic}

\begin{EntryWithPhonetic}{暖气}{nuan3qi4}{13,4}{⽇,⽓}[HSK 4]
  \definition[个,种]{s.}{aquecedor; aquecimento; aquecimento central}
\end{EntryWithPhonetic}

%%%%%%%%%% 那 %%%%%%%%%%
\subsection*{那}\addcontentsline{loh}{figure}{那 \dpy{nuo2}}

\begin{EntryWithPhonetic}{那}{nuo2}{6}{⾢}
  \definition*{s.}{Sobrenome: Nuo}
  \seeref{na1}
  \seeref{na3}
  \seeref{na4}
  \seeref{ne4}
  \seeref{nei4}
\end{EntryWithPhonetic}

%%%%%%%%%% 挪 %%%%%%%%%%
\subsection*{挪}\addcontentsline{loh}{figure}{挪 \dpy{nuo2}}

\begin{EntryWithPhonetic}{挪}{nuo2}{9}{⼿}[HSK 7-9]
  \definition{v.}{mover; deslocar; transportar}
\end{EntryWithPhonetic}

%%%%%%%%%% 诺 %%%%%%%%%%
\subsection*{诺}\addcontentsline{loh}{figure}{诺 \dpy{nuo4}}

\begin{EntryWithPhonetic}{诺}{nuo4}{10}{⾔}
  \definition*{s.}{Sobrenome: Nuo}
  \definition{interj.}{`Sim!''}
  \definition{v.}{prometer}
\end{EntryWithPhonetic}

\begin{EntryWithPhonetic}{诺贝尔奖}{nuo4bei4'er3 jiang3}{10,4,5,9}{⾔,⾙,⼩,⼤}
  \definition*{s.}{Prêmio Nobel}
\end{EntryWithPhonetic}

\begin{EntryWithPhonetic}{诺奖}{nuo4jiang3}{10,9}{⾔,⼤}
  \definition*{s.}{Prêmio Nobel, abreviação de 诺贝尔奖}
  \seealsoref{诺贝尔奖}{nuo4bei4'er3 jiang3}
\end{EntryWithPhonetic}

\begin{EntryWithPhonetic}{诺言}{nuo4yan2}{10,7}{⾔,⾔}[HSK 7-9]
  \definition[个]{s.}{promessa}[他总是兑现他的诺言。===Ele sempre cumpre suas promessas.]
\end{EntryWithPhonetic}

%%%%%%%%%% 女 %%%%%%%%%%
\subsection*{女}\addcontentsline{loh}{figure}{女 \dpy{nv3}}

\begin{EntryWithPhonetic}{女}{nv3}{3}{⼥}[HSK 1][Kangxi 38]
  \definition{adj.}{mulher; feminino | fêmea (de certos animais)}
  \definition{s.}{menina; filha | nü, uma das mansões lunares | mulher}
  \antonymref{男}{nan2}
\end{EntryWithPhonetic}

\begin{EntryWithPhonetic}{女儿}{nv3'er2}{3,2}{⼥,⼉}[HSK 1]
  \definition[个]{s.}{menina; filha}
  \seealsoref{儿子}{er2zi5}
\end{EntryWithPhonetic}

\begin{EntryWithPhonetic}{女孩}{nv3hai2}{3,9}{⼥,⼦}
  \definition{s.}{menina | garota}
\end{EntryWithPhonetic}

\begin{EntryWithPhonetic}{女孩儿}{nv3 hai2r5}{3,9,2}{⼥,⼦,⼉}[HSK 1]
  \definition{s.}{garota; menina; atualmente também se refere a mulher adolescente | filha}
\end{EntryWithPhonetic}

\begin{EntryWithPhonetic}{女朋友}{nv3 peng2 you5}{3,8,4}{⼥,⽉,⼜}[HSK 1]
  \definition{s.}{namorada}
\end{EntryWithPhonetic}

\begin{EntryWithPhonetic}{女人}{nv3 ren2}{3,2}{⼥,⼈}[HSK 1]
  \definition[个,位]{s.}{mulher adulta}
\end{EntryWithPhonetic}

\begin{EntryWithPhonetic}{女生}{nv3 sheng1}{3,5}{⼥,⽣}[HSK 1]
  \definition[个]{s.}{estudante; aluna; estudante do sexo feminino | menina; jovem mulher}
\end{EntryWithPhonetic}

\begin{EntryWithPhonetic}{女士}{nv3shi4}{3,3}{⼥,⼠}[HSK 4]
  \definition{pron.}{Sra.; Senhorita; Senhora; título honorífico para mulheres (agora usado em contextos diplomáticos)}
  \definition[位,名,个,些]{s.}{senhora; madame}
\end{EntryWithPhonetic}

\begin{EntryWithPhonetic}{女王}{nv3wang2}{3,4}{⼥,⽟}
  \definition{s.}{rainha}
\end{EntryWithPhonetic}

\begin{EntryWithPhonetic}{女性}{nv3 xing4}{3,8}{⼥,⼼}[HSK 5]
  \definition[个,位,名]{s.}{mulher; feminino; feminilidade}
  \antonymref{男性}{nan2 xing4}
\end{EntryWithPhonetic}

\begin{EntryWithPhonetic}{女婿}{nv3xu5}{3,12}{⼥,⼥}[HSK 7-9]
  \definition[个,位]{s.}{genro; marido da filha | em algumas regiões, refere"-se ao marido}
\end{EntryWithPhonetic}

\begin{EntryWithPhonetic}{女子}{nv3 zi3}{3,3}{⼥,⼦}[HSK 3]
  \definition[位,名,个]{s.}{mulher; feminino; pessoa do sexo feminino}
\end{EntryWithPhonetic}

%%%%%%%%%% 虐 %%%%%%%%%%
\subsection*{虐}\addcontentsline{loh}{figure}{虐 \dpy{nve4}}

\begin{EntryWithPhonetic}{虐}{nve4}{9}{⾌}
  \definition{adj.}{cruel; tirânico; brutal e cruel}
\end{EntryWithPhonetic}

\begin{EntryWithPhonetic}{虐待}{nve4dai4}{9,9}{⾌,⼻}[HSK 7-9]
  \definition{v.}{maltratar; tiranizar; tratar com métodos cruéis e impiedosos}
\end{EntryWithPhonetic}

%%%%% EOF %%%%%

