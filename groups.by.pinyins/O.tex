%%%
%%% O
%%%
\section*{O}\addcontentsline{toc}{section}{O}\addcontentsline{loh}{figure}{\#\#\#\#\#\#\#\# O}

%%%%%%%%%% 喔 %%%%%%%%%%
\subsection*{喔}\addcontentsline{loh}{figure}{喔 \dpy{o1}}

\begin{EntryWithPhonetic}{喔}{o1}{12}{⼝}
  \definition{interj.}{``Oh!'', ``Entendi!'', usado para indicar realização, compreensão}
\end{EntryWithPhonetic}

%%%%%%%%%% 哦 %%%%%%%%%%
\subsection*{哦}\addcontentsline{loh}{figure}{哦 \dpy{o2}}

\begin{EntryWithPhonetic}{哦}{o2}{10}{⼝}
  \definition{interj.}{``Oh!''; usado para indicar dúvida ou surpresa}
  \seeref{e2}
  \seeref{o4}
  \seeref{o5}
\end{EntryWithPhonetic}

\begin{EntryWithPhonetic}{哦}{o4}{10}{⼝}[HSK 7-9]
  \definition{interj.}{``Oh!''; usado para indicar que acabou de aprender algo}
  \seeref{e2}
  \seeref{o2}
  \seeref{o5}
\end{EntryWithPhonetic}

\begin{EntryWithPhonetic}{哦}{o5}{10}{⼝}
  \definition{part.}{usado no final da frase para indicar que uma pessoa está afirmando um fato que a outra pessoa não sabe | usado no final da frase para transmitir informalidade, calor, simpatia ou intimidade}
  \seeref{e2}
  \seeref{o2}
  \seeref{o4}
\end{EntryWithPhonetic}

%%%%%%%%%% 区 %%%%%%%%%%
\subsection*{区}\addcontentsline{loh}{figure}{区 \dpy{ou1}}

\begin{EntryWithPhonetic}{区}{ou1}{4}{⼖}
  \definition*{s.}{Sobrenome: Ou}
  \seeref{qu1}
\end{EntryWithPhonetic}

%%%%%%%%%% 欧 %%%%%%%%%%
\subsection*{欧}\addcontentsline{loh}{figure}{欧 \dpy{ou1}}

\begin{EntryWithPhonetic}{欧}{ou1}{8}{⽋}
  \definition*{s.}{Europa, abreviação de 欧洲 | Sobrenome: Ou}
  \seealsoref{欧洲}{ou1zhou1}
\end{EntryWithPhonetic}

\begin{EntryWithPhonetic}{欧盟}{ou1meng2}{8,13}{⽋,⽫}
  \definition*{s.}{União Europeia (EU)}
\end{EntryWithPhonetic}

\begin{EntryWithPhonetic}{欧阳询}{ou1yang2 xun2}{8,6,8}{⽋,⾩,⾔}
  \definition*{s.}{Ouyang Xun (557--641), um dos quatro grandes calígrafos do início da dinastia Tang (唐初四大家)}
  \seealsoref{唐初四大家}{tang2 chu1 si4 da4jia1}
\end{EntryWithPhonetic}

\begin{EntryWithPhonetic}{欧洲}{ou1zhou1}{8,9}{⽋,⽔}
  \definition*{s.}{Europa}
\end{EntryWithPhonetic}

\begin{EntryWithPhonetic}{欧洲共同体}{ou1zhou1 gong4tong2ti3}{8,9,6,6,7}{⽋,⽔,⼋,⼝,⼈}
  \definition*{s.}{Comunidade Europeia}
\end{EntryWithPhonetic}

\begin{EntryWithPhonetic}{欧洲人}{ou1zhou1ren2}{8,9,2}{⽋,⽔,⼈}
  \definition{s.}{europeu | pessoa ou povo da Europa}
\end{EntryWithPhonetic}

%%%%%%%%%% 殴 %%%%%%%%%%
\subsection*{殴}\addcontentsline{loh}{figure}{殴 \dpy{ou1}}

\begin{EntryWithPhonetic}{殴}{ou1}{8}{⽎}
  \definition{v.}{espancar; bater; acertar}
\end{EntryWithPhonetic}

\begin{EntryWithPhonetic}{殴打}{ou1da3}{8,5}{⽎,⼿}[HSK 7-9]
  \definition{v.}{espancar; bater; bater em alguém}
\end{EntryWithPhonetic}

%%%%%%%%%% 呕 %%%%%%%%%%
\subsection*{呕}\addcontentsline{loh}{figure}{呕 \dpy{ou3}}

\begin{EntryWithPhonetic}{呕}{ou3}{7}{⼝}
  \definition{v.}{vomitar}
\end{EntryWithPhonetic}

\begin{EntryWithPhonetic}{呕吐}{ou3tu4}{7,6}{⼝,⼝}[HSK 7-9]
  \definition{v.}{vomitar; enjoar; jorrar pela boca a comida proveniente do estômago}
\end{EntryWithPhonetic}

%%%%%%%%%% 偶 %%%%%%%%%%
\subsection*{偶}\addcontentsline{loh}{figure}{偶 \dpy{ou3}}

\begin{EntryWithPhonetic}{偶}{ou3}{11}{⼈}
  \definition{adv.}{por acaso; por acidente; de vez em quando; ocasionalmente | par; número par; pareado}
  \definition{s.}{imagem; ídolo; figuras feitas de madeira, barro, etc. | companheiro; cônjuge; parceiro; refere"-se a um casal ou a um dos casais}
  \antonymref{奇}{qi2}
\end{EntryWithPhonetic}

\begin{EntryWithPhonetic}{偶尔}{ou3'er3}{11,5}{⼈,⼩}[HSK 5]
  \definition{adj.}{ocasional}
  \definition{adv.}{ocasionalmente; de vez em quando; às vezes}
\end{EntryWithPhonetic}

\begin{EntryWithPhonetic}{偶然}{ou3ran2}{11,12}{⼈,⽕}[HSK 5]
  \definition{adj.}{acidental; ocasional}
  \definition{adv.}{por acaso; acidentalmente; sem querer; inesperadamente | ocasionalmente; de vez em quando; às vezes}
\end{EntryWithPhonetic}

\begin{EntryWithPhonetic}{偶像}{ou3xiang4}{11,13}{⼈,⼈}[HSK 5]
  \definition[位,个,名]{s.}{ídolo; pessoa amada pelas pessoas; refere"-se a uma pessoa que é apreciada por todos e que, em certos aspectos, é digna de admiração e respeito}
\end{EntryWithPhonetic}

%%%%% EOF %%%%%

