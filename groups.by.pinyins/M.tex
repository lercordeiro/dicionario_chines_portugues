%%%
%%% M
%%%
\section*{M}\addcontentsline{toc}{section}{M}\addcontentsline{loh}{figure}{\#\#\#\#\#\#\#\# M}

%%%%%%%%%% 妈 %%%%%%%%%%
\subsection*{妈}\addcontentsline{loh}{figure}{妈 \dpy{ma1}}

\begin{EntryWithPhonetic}{妈}{ma1}{6}{⼥}[HSK 1]
  \definition[个,位]{s.}{mãe; mamãe | uma forma de tratamento para uma mulher casada uma geração mais velha | (antigo) uma forma de tratamento para uma empregada doméstica de meia-idade ou velha}
  \seealsoref{妈妈}{ma1 ma5}
\end{EntryWithPhonetic}

\begin{EntryWithPhonetic}{妈妈}{ma1 ma5}{6,6}{⼥,⼥}[HSK 1]
  \definition[个,位]{s.}{mamãe; mãe | uma forma de chamar uma mulher de meia-idade; títulos de respeito para mulheres mais velhas}
\end{EntryWithPhonetic}

%%%%%%%%%% 抹 %%%%%%%%%%
\subsection*{抹}\addcontentsline{loh}{figure}{抹 \dpy{ma1}}

\begin{EntryWithPhonetic}{抹}{ma1}{8}{⼿}
  \definition{v.}{esfregar; limpar | deslizar algo para fora; tirar}
  \seeref{mo3}
  \seeref{mo4}
\end{EntryWithPhonetic}

%%%%%%%%%% 蚂 %%%%%%%%%%
\subsection*{蚂}\addcontentsline{loh}{figure}{蚂 \dpy{ma1}}

\begin{EntryWithPhonetic}{蚂}{ma1}{9}{⾍}
  \definition{part.}{caracter formador de palavras}
  \definition[只]{s.}{libélula}
  \seeref{ma3}
  \seeref{ma4}
\end{EntryWithPhonetic}

%%%%%%%%%% 麻 %%%%%%%%%%
\subsection*{麻}\addcontentsline{loh}{figure}{麻 \dpy{ma1}}

\begin{EntryWithPhonetic}{麻}{ma1}{11}{⿇}[Kangxi 200]
  \definition{adj.}{sombrio; escuro; completamente escuro}
  \seeref{ma2}
\end{EntryWithPhonetic}

%%%%%%%%%% 吗 %%%%%%%%%%
\subsection*{吗}\addcontentsline{loh}{figure}{吗 \dpy{ma2}}

\begin{EntryWithPhonetic}{吗}{ma2}{6}{⼝}
  \definition{adv.}{(coloquial) que?}
  \seeref{ma3}
  \seeref{ma5}
\end{EntryWithPhonetic}

%%%%%%%%%% 麻 %%%%%%%%%%
\subsection*{麻}\addcontentsline{loh}{figure}{麻 \dpy{ma2}}

\begin{EntryWithPhonetic}{麻}{ma2}{11}{⿇}[HSK 7-9][Kangxi 200]
  \definition*{s.}{Sobrenome: Ma}
  \definition{adj.}{áspero; grosseiro | marcado; manchado | espinhas; manchas ásperas; cicatrizes deixadas após a varíola}
  \definition[棵,株]{s.}{nome geral para cânhamo, linho, etc. | fibra de cânhamo, linho, etc. para têxteis | sésamo; gergelim | marcas de varíola; um rosto com marcas de varíola}
  \definition{v.}{anestesiar | corromper (a mente de alguém); envenenar}
  \seeref{ma1}
\end{EntryWithPhonetic}

\begin{EntryWithPhonetic}{麻痹}{ma2bi4}{11,13}{⿇,⽧}[HSK 7-9]
  \definition{adj.}{insensível; descuidado; negligente; essa metáfora descreve um estado de entorpecimento mental e perda de vigilância}
  \definition{v.}{ficar paralisado; ficar insensível; perder total ou parcialmente as funções sensoriais e motoras em uma parte do corpo | estar insensível; baixar a guarda; entorpecer a mente}
\end{EntryWithPhonetic}

\begin{EntryWithPhonetic}{麻烦}{ma2fan5}{11,10}{⿇,⽕}[HSK 3]
  \definition{adj.}{incômodo; inconveniente; complicado; trabalhoso; burocrático | incômodo; inconveniente; (a situação) é confusa e complicada}
  \definition[个,些,点,堆]{s.}{problema; inconveniência; assuntos complicados e difíceis de resolver}
  \definition{v.}{incomodar; perturbar; incomodar alguém; irritar; aborrecer; causar incômodo ou sobrecarregar outras pessoas}
\end{EntryWithPhonetic}

\begin{EntryWithPhonetic}{麻将}{ma2jiang4}{11,9}{⿇,⼨}[HSK 7-9]
  \definition*[副]{s.}{Mahjong}
\end{EntryWithPhonetic}

\begin{EntryWithPhonetic}{麻辣}{ma2la4}{11,14}{⿇,⾟}[HSK 7-9]
  \definition{adj.}{picante; quente e dormente; descreve um sabor como dormente e picante, semelhante à pimenta-de-Sichuan ou à pimenta malagueta}
\end{EntryWithPhonetic}

\begin{EntryWithPhonetic}{麻辣豆腐}{ma2la4 dou4fu5}{11,14,7,14}{⿇,⾟,⾖,⾁}
  \definition{s.}{tofú guisado em molho picante (prato)}
\end{EntryWithPhonetic}

\begin{EntryWithPhonetic}{麻木}{ma2mu4}{11,4}{⿇,⽊}[HSK 7-9]
  \definition{adj.}{dormente; descreve uma sensação de dormência ou perda de sensibilidade em uma parte do corpo devido ao frio ou à inatividade prolongada | entorpecido; insensível; sem vida; apático; de raciocínio lento}
\end{EntryWithPhonetic}

\begin{EntryWithPhonetic}{麻醉}{ma2zui4}{11,15}{⿇,⾣}[HSK 7-9]
  \definition{v.}{narcotizar; anestesiar; utilizar drogas, acupuntura ou outros métodos para deixar temporariamente todo ou parte de um organismo inconsciente | corromper (a mente de alguém); desgastar (a força de vontade de alguém); essa metáfora descreve o uso de certos métodos para desgastar a vontade de uma pessoa, fazendo com que ela perca a capacidade de distinguir o certo do errado}
\end{EntryWithPhonetic}

%%%%%%%%%% 马 %%%%%%%%%%
\subsection*{马}\addcontentsline{loh}{figure}{马 \dpy{ma3}}

\begin{EntryWithPhonetic}{马}{ma3}{3}{⾺}[HSK 3][Kangxi 187]
  \definition*{s.}{Sobrenome: Ma}
  \definition{adj.}{grande; extenso; amplo}
  \definition[匹,头,只,群]{s.}{cavalo | a peça do cavalo no xadrez chinês}
\end{EntryWithPhonetic}

\begin{EntryWithPhonetic}{马车}{ma3 che1}{3,4}{⾺,⾞}[HSK 6]
  \definition[辆]{s.}{carruagem (puxada por cavalo); carroça; charrete}
\end{EntryWithPhonetic}

\begin{EntryWithPhonetic}{马耳他}{ma3'er3ta1}{3,6,5}{⾺,⽿,⼈}
  \definition*{s.}{Malta}
\end{EntryWithPhonetic}

\begin{EntryWithPhonetic}{马后炮}{ma3hou4pao4}{3,6,9}{⾺,⼝,⽕}[HSK 7-9]
  \definition{s.}{(termo do xadrez chinês) ação ou conselho tardio; esforço tardio; tarde demais | Figurativo: ação tardia | visão retrospectiva}
\end{EntryWithPhonetic}

\begin{EntryWithPhonetic}{马虎}{ma3hu5}{3,8}{⾺,⾌}[HSK 7-9]
  \definition{adj.}{descuidado; casual | superficial; apressado; descuidado}
  \definition{v.}{encarar de forma leviana; fazer um trabalho malfeito}
\end{EntryWithPhonetic}

\begin{EntryWithPhonetic}{马克思列宁主义}{ma3ke4si1 lie4ning2 zhu3yi4}{3,7,9,6,5,5,3}{⾺,⼗,⼼,⼑,⼧,⼂,⼂}
  \definition*{s.}{Marxismo-Leninismo}
\end{EntryWithPhonetic}

\begin{EntryWithPhonetic}{马力}{ma3li4}{3,2}{⾺,⼒}[HSK 7-9]
  \definition{clas.}{Física: cavalos de potência, cavalo-vapor (cv)}
\end{EntryWithPhonetic}

\begin{EntryWithPhonetic}{马列主义}{ma3 lie4 zhu3yi4}{3,6,5,3}{⾺,⼑,⼂,⼂}
  \definition*{s.}{Marxismo-Leninismo}
\end{EntryWithPhonetic}

\begin{EntryWithPhonetic}{马路}{ma3lu4}{3,13}{⾺,⾜}[HSK 1]
  \definition[条]{s.}{estrada; rua; avenida; estradas largas e planas para o tráfego de carros e cavalos nas cidades ou nos subúrbios}
\end{EntryWithPhonetic}

\begin{EntryWithPhonetic}{马马虎虎}{ma3ma3hu3hu3}{3,3,8,8}{⾺,⾺,⾌,⾌}
  \definition{adj.}{tolerável; aceitável; mais ou menos; razoável; não tão ruim | descuidado; casual; vago; descuidado e negligente na execução das tarefas}
\end{EntryWithPhonetic}

\begin{EntryWithPhonetic}{马上}{ma3shang4}{3,3}{⾺,⼀}[HSK 1]
  \definition{adv.}{imediatamente; de uma só vez; em um piscar de olhos | em breve; em um futuro próximo; em um curto espaço de tempo}
\end{EntryWithPhonetic}

\begin{EntryWithPhonetic}{马桶}{ma3tong3}{3,11}{⾺,⽊}[HSK 7-9]
  \definition[个,台]{s.}{vaso sanitário}[小心别把手机掉进马桶。===Tenha cuidado para não deixar seu celular cair no vaso sanitário.]
\end{EntryWithPhonetic}

\begin{EntryWithPhonetic}{马尾}{ma3wei3}{3,7}{⾺,⼫}
  \definition{s.}{(penteado) rabo de cavalo | cauda de cavalo}
\end{EntryWithPhonetic}

\begin{EntryWithPhonetic}{马戏}{ma3xi4}{3,6}{⾺,⼽}[HSK 7-9]
  \definition[场,次]{s.}{circo | espetáculo de circo (apresentação)}
\end{EntryWithPhonetic}

%%%%%%%%%% 吗 %%%%%%%%%%
\subsection*{吗}\addcontentsline{loh}{figure}{吗 \dpy{ma3}}

\begin{EntryWithPhonetic}{吗}{ma3}{6}{⼝}
  \definition{s.}{usada em 吗啡, morfina}
  \seeref{ma2}
  \seeref{ma5}
  \seealsoref{吗啡}{ma3fei1}
\end{EntryWithPhonetic}

\begin{EntryWithPhonetic}{吗啡}{ma3fei1}{6,11}{⼝,⼝}
  \definition{s.}{morfina (empréstimo linguístico)}
\end{EntryWithPhonetic}

%%%%%%%%%% 码 %%%%%%%%%%
\subsection*{码}\addcontentsline{loh}{figure}{码 \dpy{ma3}}

\begin{EntryWithPhonetic}{码}{ma3}{8}{⽯}[HSK 7-9]
  \definition{clas.}{refere-se a um assunto específico ou a uma categoria de assuntos; refere-se a uma coisa ou a uma classe de coisas | jarda; unidade de comprimento britânica e americana}
  \definition{s.}{um sinal ou objeto que indica número; código; símbolos ou ferramentas que indicam números, como códigos de barras ou \emph{QR code}}
  \definition{v.}{empilhar; acumular}
\end{EntryWithPhonetic}

\begin{EntryWithPhonetic}{码头}{ma3tou2}{8,5}{⽯,⼤}[HSK 5]
  \definition[个,座]{s.}{doca; cais; píer; molhe; edifícios à beira-mar ou à beira do rio destinados exclusivamente à atracação de embarcações, embarque e desembarque de passageiros e carga e descarga de mercadorias | cidade portuária; centro comercial e de transportes; refere-se a uma cidade comercial com transporte terrestre e marítimo bem desenvolvido.}
\end{EntryWithPhonetic}

%%%%%%%%%% 蚂 %%%%%%%%%%
\subsection*{蚂}\addcontentsline{loh}{figure}{蚂 \dpy{ma3}}

\begin{EntryWithPhonetic}{蚂}{ma3}{9}{⾍}
  \definition{part.}{caracter formador de palavras}
  \seeref{ma1}
  \seeref{ma4}
\end{EntryWithPhonetic}

\begin{EntryWithPhonetic}{蚂蚁}{ma3yi3}{9,9}{⾍,⾍}
  \definition{s.}{formiga}
\end{EntryWithPhonetic}

\begin{EntryWithPhonetic}{蚂}{ma4}{9}{⾍}
  \definition{part.}{caracter formador de palavras}
  \seeref{ma1}
  \seeref{ma3}
\end{EntryWithPhonetic}

%%%%%%%%%% 骂 %%%%%%%%%%
\subsection*{骂}\addcontentsline{loh}{figure}{骂 \dpy{ma4}}

\begin{EntryWithPhonetic}{骂}{ma4}{9}{⾺}[HSK 5]
  \definition{v.}{abusar; xingar; insultar; insultar alguém com palavras grosseiras ou maliciosas | repreender; censurar; condenar}
\end{EntryWithPhonetic}

\begin{EntryWithPhonetic}{骂街}{ma4jie1}{9,12}{⾺,⾏}
  \definition{v.}{gritar abusos na rua}
\end{EntryWithPhonetic}

\begin{EntryWithPhonetic}{骂名}{ma4ming2}{9,6}{⾺,⼝}
  \definition{s.}{infâmia}
\end{EntryWithPhonetic}

%%%%%%%%%% 吗 %%%%%%%%%%
\subsection*{吗}\addcontentsline{loh}{figure}{吗 \dpy{ma5}}

\begin{EntryWithPhonetic}{吗}{ma5}{6}{⼝}[HSK 1]
  \definition{part.}{usado no final de uma pergunta | como uma pausa em uma frase antes de introduzir o ponto principal | usado no final de uma pergunta retórica}
  \seeref{ma2}
  \seeref{ma3}
\end{EntryWithPhonetic}

%%%%%%%%%% 嘛 %%%%%%%%%%
\subsection*{嘛}\addcontentsline{loh}{figure}{嘛 \dpy{ma5}}

\begin{EntryWithPhonetic}{嘛}{ma5}{14}{⼝}[HSK 6]
  \definition{part.}{usado no final de uma declaração para expressar que é claro que é verdade que é óbvio | usado no final de uma frase imperativa para expressar expectativa ou dissuasão | usado em uma frase para indicar uma pausa e chamar a atenção da outra pessoa}
\end{EntryWithPhonetic}

%%%%%%%%%% 埋 %%%%%%%%%%
\subsection*{埋}\addcontentsline{loh}{figure}{埋 \dpy{mai2}}

\begin{EntryWithPhonetic}{埋}{mai2}{10}{⼟}[HSK 6]
  \definition{v.}{cobrir (com terra, neve, etc.); enterrar | esconder | enterrar (uma pessoa morta)}
  \seeref{man2}
\end{EntryWithPhonetic}

\begin{EntryWithPhonetic}{埋藏}{mai2cang2}{10,17}{⼟,⾋}[HSK 7-9]
  \definition{v.}{ocultar os próprios sentimentos e pensamentos; esconder as próprias emoções; não deixar que seus pensamentos e sentimentos transpareçam | enterrar algo no chão para que outros não consigam encontrar; esconder o objeto debaixo da terra para que ninguém o encontre | (minerais) estar oculto no solo ou nas montanhas; (minerais, etc.) enterrados no subsolo ou em montanhas}
\end{EntryWithPhonetic}

\begin{EntryWithPhonetic}{埋伏}{mai2fu2}{10,6}{⼟,⼈}[HSK 7-9]
  \definition{v.}{emboscar; armar uma emboscada; desdobrar tropas secretamente em áreas por onde se espera que o inimigo passe e aguarde o momento oportuno para atacar | esconder-se; ficar debaixo de uma cobertura; espreitar}
\end{EntryWithPhonetic}

\begin{EntryWithPhonetic}{埋没}{mai2mo4}{10,7}{⼟,⽔}[HSK 7-9]
  \definition{v.}{enterrar; cobrir (com terra, neve, etc.) | negligenciar; sufocar; suprimir; tornar invisível}
\end{EntryWithPhonetic}

%%%%%%%%%% 买 %%%%%%%%%%
\subsection*{买}\addcontentsline{loh}{figure}{买 \dpy{mai3}}

\begin{EntryWithPhonetic}{买}{mai3}{6}{⼄}[HSK 1]
  \definition*{s.}{Sobrenome: Mai}
  \definition{v.}{comprar; adquirir | comprar; subornar; usar dinheiro ou outros meios para angariar apoio| pedir; obter; trocar dinheiro por coisas}
\end{EntryWithPhonetic}

\begin{EntryWithPhonetic}{买不起}{mai3 bu5 qi3}{6,4,10}{⼄,⼀,⾛}[HSK 7-9]
  \definition{v.}{não ter condições de comprar | não ter condições de pagar}
\end{EntryWithPhonetic}

\begin{EntryWithPhonetic}{买东西}{mai3 dong1xi5}{6,5,6}{⼄,⼀,⾑}
  \definition{v.}{fazer compras; comprar bens ou serviços}
\end{EntryWithPhonetic}

\begin{EntryWithPhonetic}{买卖}{mai3 mai4}{6,8}{⼄,⼗}[HSK 5]
  \definition[笔,桩,宗,家]{s.}{negócio; compra e venda; transação | Privado: loja; armazém}
\end{EntryWithPhonetic}

%%%%%%%%%% 迈 %%%%%%%%%%
\subsection*{迈}\addcontentsline{loh}{figure}{迈 \dpy{mai4}}

\begin{EntryWithPhonetic}{迈}{mai4}{6}{⾡}[HSK 7-9]
  \definition{adj.}{velho; idoso}
  \definition{clas.}{milha}
  \definition{v.}{dar um passo; passar; avançar; levantar o pé e caminhar para a frente; dar um passo largo}
\end{EntryWithPhonetic}

\begin{EntryWithPhonetic}{迈进}{mai4jin4}{6,7}{⾡,⾡}[HSK 7-9]
  \definition{v.}{avançar; seguir em frente; prosseguir com passos largos}
\end{EntryWithPhonetic}

%%%%%%%%%% 麦 %%%%%%%%%%
\subsection*{麦}\addcontentsline{loh}{figure}{麦 \dpy{mai4}}

\begin{EntryWithPhonetic}{麦}{mai4}{7}{⿆}[Kangxi 199]
  \definition*{s.}{Sobrenome: Mai}
  \definition[袋,筐,车]{s.}{um termo geral para trigo, cevada, etc.}
\end{EntryWithPhonetic}

\begin{EntryWithPhonetic}{麦当劳}{mai4dang1lao2}{7,6,7}{⿆,⼹,⼒}
  \definition*{s.}{McDonald's, restaurante de \emph{fast-food}}
  \seealsoref{麦当劳叔叔}{mai4dang1lao2 shu1shu5}
\end{EntryWithPhonetic}

\begin{EntryWithPhonetic}{麦当劳叔叔}{mai4dang1lao2 shu1shu5}{7,6,7,8,8}{⿆,⼹,⼒,⼜,⼜}
  \definition*{s.}{Ronald McDonald}
  \seealsoref{麦当劳}{mai4dang1lao2}
\end{EntryWithPhonetic}

\begin{EntryWithPhonetic}{麦淇淋}{mai4qi2lin2}{7,11,11}{⿆,⽔,⽔}
  \definition{s.}{Empréstimo linguístico: margarina}
\end{EntryWithPhonetic}

%%%%%%%%%% 卖 %%%%%%%%%%
\subsection*{卖}\addcontentsline{loh}{figure}{卖 \dpy{mai4}}

\begin{EntryWithPhonetic}{卖}{mai4}{8}{⼗}[HSK 2]
  \definition*{s.}{Sobrenome: Mai}
  \definition{clas.}{um prato (nos tempos antigos); antigamente, os restaurantes chamavam cada prato vendido de 一卖 (uma porção)}
  \definition{v.}{vender (oposto de 买) | trair (o próprio país ou amigos); alcançar objetivos pessoais à custa dos interesses do país, da nação e dos outros | não poupar esforços; esforçar-se ao máximo; tentar fazer o máximo possível | mostrar-se intencionalmente; exibir-se | vender o próprio trabalho; trabalhar em troca de dinheiro}
  \seealsoref{买}{mai3}
\end{EntryWithPhonetic}

\begin{EntryWithPhonetic}{卖弄}{mai4nong5}{8,7}{⼗,⼶}[HSK 7-9]
  \definition{v.}{exibir-se; desfilar; exibir ou ostentar intencionalmente (as próprias habilidades)}
\end{EntryWithPhonetic}

%%%%%%%%%% 脉 %%%%%%%%%%
\subsection*{脉}\addcontentsline{loh}{figure}{脉 \dpy{mai4}}

\begin{EntryWithPhonetic}{脉}{mai4}{9}{⾁}
  \definition{s.}{artérias e veias | pulso | nervura (de uma folha, asa de inseto, etc.) | rede; sistema; malha | Coloquial: milhas}
  \seeref{mo4}
\end{EntryWithPhonetic}

\begin{EntryWithPhonetic}{脉搏}{mai4bo2}{9,13}{⾁,⼿}[HSK 7-9]
  \definition{s.}{pulso; o fenômeno das artérias pulsarem ritmicamente à medida que o sangue bombeado impacta as paredes arteriais durante a contração cardíaca | uma metáfora para o desenvolvimento, as mudanças ou as tendências da sociedade, da vida, etc.}
\end{EntryWithPhonetic}

\begin{EntryWithPhonetic}{脉络}{mai4luo4}{9,9}{⾁,⽷}[HSK 7-9]
  \definition{s.}{termo geral para artérias, veias e canais por onde circula a energia vital; a medicina tradicional chinesa se refere aos vasos sanguíneos e meridianos por todo o corpo | uma linha de raciocínio; uma sequência de ideias; metaforicamente, refere-se à ordem ou estrutura das coisas ou da escrita | as nervuras (de uma folha, etc.); veios nas folhas das plantas ou em outras estruturas}
\end{EntryWithPhonetic}

%%%%%%%%%% 嫚 %%%%%%%%%%
\subsection*{嫚}\addcontentsline{loh}{figure}{嫚 \dpy{man1}}

\begin{EntryWithPhonetic}{嫚}{man1}{14}{⼥}
  \definition{s.}{menina bem-comportada}
  \seealsoref{嫚子}{man1zi5}
\end{EntryWithPhonetic}

\begin{EntryWithPhonetic}{嫚子}{man1zi5}{14,3}{⼥,⼦}
  \definition{s.}{Dialeto: menina}
\end{EntryWithPhonetic}

%%%%%%%%%% 埋 %%%%%%%%%%
\subsection*{埋}\addcontentsline{loh}{figure}{埋 \dpy{man2}}

\begin{EntryWithPhonetic}{埋}{man2}{10}{⼟}
  \definition{part.}{caracter formador de palavras}
  \seeref{mai2}
\end{EntryWithPhonetic}

\begin{EntryWithPhonetic}{埋怨}{man2yuan4}{10,9}{⼟,⼼}[HSK 7-9]
  \definition{v.}{reclamar; culpar; resmungar; estar insatisfeiro com alguém ou algo que se acredita ser a causa do desconforto da situação}
\end{EntryWithPhonetic}

%%%%%%%%%% 蛮 %%%%%%%%%%
\subsection*{蛮}\addcontentsline{loh}{figure}{蛮 \dpy{man2}}

\begin{EntryWithPhonetic}{蛮}{man2}{12}{⾍}[HSK 7-9]
  \definition{adj.}{grosseiro; rude; feroz; irracional; cruel | imprudente; implacável}
  \definition{adv.}{muito; bastante; razoavelmente}
  \definition{s.}{um nome antigo para os grupos étnicos do sul}
\end{EntryWithPhonetic}

%%%%%%%%%% 谩 %%%%%%%%%%
\subsection*{谩}\addcontentsline{loh}{figure}{谩 \dpy{man2}}

\begin{EntryWithPhonetic}{谩}{man2}{13}{⾔}
  \definition{v.}{enganar; ludibriar; iludir}
  \seeref{man4}
\end{EntryWithPhonetic}

%%%%%%%%%% 蔓 %%%%%%%%%%
\subsection*{蔓}\addcontentsline{loh}{figure}{蔓 \dpy{man2}}

\begin{EntryWithPhonetic}{蔓}{man2}{14}{⾋}
  \definition{s.}{couve-chinesa | nabo}
  \seeref{man4}
  \seeref{wan4}
\end{EntryWithPhonetic}

%%%%%%%%%% 馒 %%%%%%%%%%
\subsection*{馒}\addcontentsline{loh}{figure}{馒 \dpy{man2}}

\begin{EntryWithPhonetic}{馒}{man2}{14}{⾷}
  \definition{s.}{pão cozido no vapor}
\end{EntryWithPhonetic}

\begin{EntryWithPhonetic}{馒头}{man2tou5}{14,5}{⾷,⼤}[HSK 6]
  \definition[个,锅,屉,筐]{s.}{pão cozido no vapor; um alimento cozido no vapor feito de farinha fermentada, geralmente redondo na parte superior e plano na parte inferior, sem recheio}
\end{EntryWithPhonetic}

%%%%%%%%%% 瞒 %%%%%%%%%%
\subsection*{瞒}\addcontentsline{loh}{figure}{瞒 \dpy{man2}}

\begin{EntryWithPhonetic}{瞒}{man2}{15}{⽬}[HSK 7-9]
  \definition*{s.}{Sobrenome: Man}
  \definition{v.}{ocultar a verdade de; esconder; esconder a verdade de alguém}
\end{EntryWithPhonetic}

%%%%%%%%%% 满 %%%%%%%%%%
\subsection*{满}\addcontentsline{loh}{figure}{满 \dpy{man3}}

\begin{EntryWithPhonetic}{满}{man3}{13}{⽔}[HSK 2]
  \definition*{s.}{Etnia Manchu | Sobrenome: Man}
  \definition{adj.}{cheio; repleto; lotado; totalmente cheio; atingindo o limite da capacidade | tudo; inteiro; completo | presunçoso; complacente; orgulhoso}
  \definition{adv.}{muito; um tanto; bastante | completamente; inteiramente; perfeitamente}
  \definition{v.}{encher | sentir-se satisfeito; sentir que já é o suficiente | expirar; atingir o limite; atingir um determinado prazo ou limite}
\end{EntryWithPhonetic}

\begin{EntryWithPhonetic}{满分}{man3fen1}{13,4}{⽔,⼑}
  \definition{s.}{pontuação completa}
\end{EntryWithPhonetic}

\begin{EntryWithPhonetic}{满怀}{man3huai2}{13,7}{⽔,⼼}[HSK 7-9]
  \definition{s.}{peito; refere-se a toda a área frontal do tórax}
  \definition{v.}{estar cheio de; estar impregnado de}
\end{EntryWithPhonetic}

\begin{EntryWithPhonetic}{满满}{man3man3}{13,13}{⽔,⽔}
  \definition{adj.}{completo | densamente empacotado}
\end{EntryWithPhonetic}

\begin{EntryWithPhonetic}{满意}{man3yi4}{13,13}{⽔,⼼}[HSK 2]
  \definition{adj.}{satisfeito; contente; gratificado}
  \definition{v.}{estar satisfeito; sentir-se contente; satisfazer os seus desejos; estar de acordo com os seus desejos}
\end{EntryWithPhonetic}

\begin{EntryWithPhonetic}{满足}{man3zu2}{13,7}{⽔,⾜}[HSK 3]
  \definition{v.}{estar satisfeito; contentar-se; sentir-se satisfeito | satisfazer}
\end{EntryWithPhonetic}

%%%%%%%%%% 谩 %%%%%%%%%%
\subsection*{谩}\addcontentsline{loh}{figure}{谩 \dpy{man4}}

\begin{EntryWithPhonetic}{谩}{man4}{13}{⾔}
  \definition{v.}{ser desrespeitoso | caluniar | desconsiderar}
  \seeref{man2}
\end{EntryWithPhonetic}

\begin{EntryWithPhonetic}{谩骂}{man4ma4}{13,9}{⾔,⾺}
  \definition{v.}{proferir injúrias; insultar; criticar; xingar | lançar abusos; xingar aleatoriamente}
  \variantof{嫚骂}
\end{EntryWithPhonetic}

%%%%%%%%%% 嫚 %%%%%%%%%%
\subsection*{嫚}\addcontentsline{loh}{figure}{嫚 \dpy{man4}}

\begin{EntryWithPhonetic}{嫚}{man4}{14}{⼥}
  \definition*{s.}{Sobrenome: Man}
  \definition{s.}{Dialeto: menina}
  \definition{v.}{Literário: desprezar; menosprezar; insultar; humilhar}
  \seeref{man1}
\end{EntryWithPhonetic}

\begin{EntryWithPhonetic}{嫚骂}{man4ma4}{14,9}{⼥,⾺}
  \definition{s.}{insultar; repreender; xingar}
\end{EntryWithPhonetic}

%%%%%%%%%% 慢 %%%%%%%%%%
\subsection*{慢}\addcontentsline{loh}{figure}{慢 \dpy{man4}}

\begin{EntryWithPhonetic}{慢}{man4}{14}{⼼}[HSK 1]
  \definition*{s.}{Sobrenome: Man}
  \definition{adj.}{lento; devagar; baixa velocidade; longa duração (em oposição a 快) | rude; arrogante; sem educação com as pessoas | frouxo; lento}
  \definition{adv.}{lentamente}
  \seealsoref{快}{kuai4}
\end{EntryWithPhonetic}

\begin{EntryWithPhonetic}{慢车}{man4 che1}{14,4}{⼼,⾞}[HSK 6]
  \definition{s.}{trem lento com muitas paradas (oposto a 快车) | ônibus ou trem local; parada do trem}
  \seealsoref{快车}{kuai4 che1}
\end{EntryWithPhonetic}

\begin{EntryWithPhonetic}{慢动作}{man4dong4zuo4}{14,6,7}{⼼,⼒,⼈}
  \definition{s.}{(cinema) câmera lenta}
\end{EntryWithPhonetic}

\begin{EntryWithPhonetic}{慢慢}{man4 man4}{14,14}{⼼,⼼}[HSK 3]
  \definition{adv.}{lentamente; vagarosamente; gradualmente | lentamente; vagarosamente; gradualmente; depois de um longo período de tempo}
\end{EntryWithPhonetic}

\begin{EntryWithPhonetic}{慢慢来}{man4man4 lai2}{14,14,7}{⼼,⼼,⽊}[HSK 7-9]
  \definition{v.}{ir com calma; não ter pressa; significa não ter impaciência ao fazer as coisas e prosseguir no seu próprio ritmo}
\end{EntryWithPhonetic}

\begin{EntryWithPhonetic}{慢性}{man4xing4}{14,8}{⼼,⼼}[HSK 7-9]
  \definition{adj.}{crônico; duradouro | lento (em fazer efeito)}
\end{EntryWithPhonetic}

%%%%%%%%%% 漫 %%%%%%%%%%
\subsection*{漫}\addcontentsline{loh}{figure}{漫 \dpy{man4}}

\begin{EntryWithPhonetic}{漫}{man4}{14}{⽔}[HSK 7-9]
  \definition{adj.}{livre; desimpedido; casual; sem restrições; arbitrário | em todo lugar; por toda parte | longo; extenso; distante}
  \definition{adv.}{não; não há necessidade de; expressa negação, equivalente a 不要}
  \definition{v.}{transbordar; inundar; alagar | estar em todo lugar; estar em todos os lugares}
  \seealsoref{不要}{bu2 yao4}
\end{EntryWithPhonetic}

\begin{EntryWithPhonetic}{漫长}{man4chang2}{14,4}{⽔,⾧}[HSK 5]
  \definition{adj.}{muito longo; interminável; (tempo, espaço) dura muito tempo}
\end{EntryWithPhonetic}

\begin{EntryWithPhonetic}{漫画}{man4hua4}{14,8}{⽔,⽥}[HSK 5]
  \definition[幅,本,张,套]{s.}{desenho animado; caricatura; \emph{cartoon}}
\end{EntryWithPhonetic}

\begin{EntryWithPhonetic}{漫骂}{man4ma4}{14,9}{⽔,⾺}
  \definition{v.}{usar linguagem ofensiva contra; insultar; difamar}
  \variantof{谩骂}
\end{EntryWithPhonetic}

\begin{EntryWithPhonetic}{漫游}{man4you2}{14,12}{⽔,⽔}[HSK 7-9]
  \definition{v.}{vagar; perambular; dar voltas; fazer uma viagem de lazer | vagar; navegar; isso se refere à capacidade de telefones celulares e outros dispositivos se conectarem a qualquer terminal em outra área de serviço por meio da rede, após entrarem em uma área de serviço não registrada | (peixes) mover-se livremente; nadar livremente na água}
\end{EntryWithPhonetic}

%%%%%%%%%% 蔓 %%%%%%%%%%
\subsection*{蔓}\addcontentsline{loh}{figure}{蔓 \dpy{man4}}

\begin{EntryWithPhonetic}{蔓}{man4}{14}{⾋}
  \definition{s.}{uma videira com gavinhas; caule fino que não consegue ficar em pé}
  \definition{v.}{rastejar; espalhar; estender}
  \seeref{man2}
  \seeref{wan4}
\end{EntryWithPhonetic}

\begin{EntryWithPhonetic}{蔓草}{man4cao3}{14,9}{⾋,⾋}
  \definition{s.}{videira | trepadeira}
\end{EntryWithPhonetic}

\begin{EntryWithPhonetic}{蔓延}{man4yan2}{14,6}{⾋,⼵}[HSK 7-9]
  \definition{v.}{espalhar; esticar; estender | infestar; espalhar; essa metáfora descreve coisas que se estendem e se expandem para fora como trepadeiras rastejantes}
\end{EntryWithPhonetic}

%%%%%%%%%% 忙 %%%%%%%%%%
\subsection*{忙}\addcontentsline{loh}{figure}{忙 \dpy{mang2}}

\begin{EntryWithPhonetic}{忙}{mang2}{6}{⼼}[HSK 1]
  \definition*{s.}{Sobrenome: Mang}
  \definition{adj.}{ocupado; movimentado; totalmente ocupado; muitas coisas para fazer, sem tempo livre (oposto de 闲) | imperativo; ansioso; urgente}
  \definition{v.}{apressar-se; agitar-se; fazer algo com urgência e constantemente | trabalhar; fazer}
  \seealsoref{闲}{xian2}
\end{EntryWithPhonetic}

\begin{EntryWithPhonetic}{忙得}{mang2de2}{6,11}{⼼,⼻}
  \definition{adj.}{muito ocupado}
\end{EntryWithPhonetic}

\begin{EntryWithPhonetic}{忙活}{mang2huo2}{6,9}{⼼,⽔}
  \definition{s.}{tarefa urgente}
  \definition{v.}{estar ocupado com o trabalho; estar envolvido com tarefas; estar com pressa para terminar as coisas}
  \seeref{mang2huo5}
\end{EntryWithPhonetic}

\begin{EntryWithPhonetic}{忙活}{mang2huo5}{6,9}{⼼,⽔}[HSK 7-9]
  \definition{adj.}{Coloquial: ocupado; movimentado}
  \definition{v.}{estar ocupado; movimentar-se freneticamente}
  \seeref{mang2huo2}
\end{EntryWithPhonetic}

\begin{EntryWithPhonetic}{忙碌}{mang2lu4}{6,13}{⼼,⽯}[HSK 7-9]
  \definition{adj.}{ocupado; atarefado; agitado; movimentado; ocupado com várias coisas, sem tempo livre}
\end{EntryWithPhonetic}

\begin{EntryWithPhonetic}{忙乱}{mang2luan4}{6,7}{⼼,⼄}[HSK 7-9]
  \definition{adj.}{apressado e desordenado; às pressas e em meio à confusão}
  \definition{v.}{estar com pressa e desorganizado; realizar uma tarefa de forma apressada e desordenada; agir de forma apressada e desordenada}
\end{EntryWithPhonetic}

%%%%%%%%%% 盲 %%%%%%%%%%
\subsection*{盲}\addcontentsline{loh}{figure}{盲 \dpy{mang2}}

\begin{EntryWithPhonetic}{盲}{mang2}{8}{⽬}
  \definition{adj.}{cego | incapaz de distinguir coisas | totalmente incompetente}
  \definition{adv.}{cegamente}
\end{EntryWithPhonetic}

\begin{EntryWithPhonetic}{盲目}{mang2mu4}{8,5}{⽬,⽬}[HSK 7-9]
  \definition{adj.}{cego; sem rumo; essa metáfora descreve a falta de compreensão clara; consideração incompleta ou descuidada; objetivos pouco claros}
\end{EntryWithPhonetic}

\begin{EntryWithPhonetic}{盲人}{mang2 ren2}{8,2}{⽬,⼈}[HSK 6]
  \definition[个,位,名]{s.}{cego; pessoa cega; pessoas com deficiência visual}
\end{EntryWithPhonetic}

%%%%%%%%%% 茫 %%%%%%%%%%
\subsection*{茫}\addcontentsline{loh}{figure}{茫 \dpy{mang2}}

\begin{EntryWithPhonetic}{茫}{mang2}{9}{⾋}
  \definition{adj.}{ilimitado e indistinto | ignorante; no escuro | disseminado e pouco claro; descreve a água ou outras coisas como ilimitadas e incertas}
\end{EntryWithPhonetic}

\begin{EntryWithPhonetic}{茫然}{mang2ran2}{9,12}{⾋,⽕}[HSK 7-9]
  \definition{adj.}{vazio; vago; ignorante; no escuro; descreve um estado de completa ignorância ou perplexidade | frustrado; decepcionado; descreve uma aparência atordoada ou distraída devido à decepção}
\end{EntryWithPhonetic}

%%%%%%%%%% 猫 %%%%%%%%%%
\subsection*{猫}\addcontentsline{loh}{figure}{猫 \dpy{mao1}}

\begin{EntryWithPhonetic}{猫}{mao1}{11}{⽝}[HSK 2]
  \definition*[只,种,群,窝,个]{s.}{gato |  Empréstimo linguístico: MODEM}
  \definition{v.}{esconder-se; entrar em esconderijo | inclinar-se para a frente; curvar-se}
  \seeref{mao2}
\end{EntryWithPhonetic}

\begin{EntryWithPhonetic}{猫熊}{mao1xiong2}{11,14}{⽝,⽕}
  \definition[把,只]{s.}{panda gigante}
  \seealsoref{熊猫}{xiong2mao1}
\end{EntryWithPhonetic}

%%%%%%%%%% 毛 %%%%%%%%%%
\subsection*{毛}\addcontentsline{loh}{figure}{毛 \dpy{mao2}}

\begin{EntryWithPhonetic}{毛}{mao2}{4}{⽑}[HSK 1,3][Kangxi 82]
  \definition*{s.}{Sobrenome: Mao}
  \definition{adj.}{bruto; semiacabado | grosseiro | pequeno | fino | descuidado; rude; precipitado | assustado; nervoso; em pânico | impetuoso | rústico; sem acabamento | impuro | (de moeda) que não vale mais seu valor nominal; depreciado}
  \definition{clas.}{mao, uma unidade fracionária de dinheiro na China; dez centavos; uma peça de dez centavos}
  \definition[根]{s.}{(de um animal, planta, etc.) cabelo; pena; penugem | (de humanos) cabelo; barba | planta; colheita | lã | mofo; bolor}
  \definition{v.}{depreciar; desvalorizar; refere-se à desvalorização da moeda | (de cavalos, gado, etc.) assustar-se; sentir medo}
\end{EntryWithPhonetic}

\begin{EntryWithPhonetic}{毛笔}{mao2 bi3}{4,10}{⽑,⽵}[HSK 5]
  \definition[支,枝,根,管]{s.}{pincel para escrever; pincel chinês; canetas feitas com pelos de coelho, carneiro, doninha, etc., são materiais tradicionais utilizados para escrever caracteres chineses e pintar pinturas tradicionais chinesas}
\end{EntryWithPhonetic}

\begin{EntryWithPhonetic}{毛病}{mao2bing4}{4,10}{⽑,⽧}[HSK 3]
  \definition[个,点,种,些]{s.}{doença ou deficiência; condição de saúde precária ou deficiência física | problema; fracasso; onde o produto está com defeito ou não funciona corretamente | mau hábito; deficiência; falhas no comportamento humano}
\end{EntryWithPhonetic}

\begin{EntryWithPhonetic}{毛巾}{mao2jin1}{4,3}{⽑,⼱}[HSK 4]
  \definition[条,块]{s.}{toalha; toalha de banho}
\end{EntryWithPhonetic}

\begin{EntryWithPhonetic}{毛衣}{mao2 yi1}{4,6}{⽑,⾐}[HSK 4]
  \definition[件,个]{s.}{suéter; blusa feita de lã}
\end{EntryWithPhonetic}

%%%%%%%%%% 矛 %%%%%%%%%%
\subsection*{矛}\addcontentsline{loh}{figure}{矛 \dpy{mao2}}

\begin{EntryWithPhonetic}{矛}{mao2}{5}{⽭}[Kangxi 110]
  \definition{s.}{Arcaico: lança; lanceta}
\end{EntryWithPhonetic}

\begin{EntryWithPhonetic}{矛盾}{mao2dun4}{5,9}{⽭,⽬}[HSK 5]
  \definition{adj.}{contraditório; descreve pessoas ou coisas que se opõem ou se repelem mutuamente}
  \definition[对,个,种]{s.}{problema; contradição; discrepância; inconsistência | disputas e conflitos; relacionamento de oposição entre as duas partes devido a diferenças de opinião ou abordagem}
  \definition{v.}{opor-se; entrar em conflito; contradizer; nesta situação, apenas uma das opções está correta ou é verdadeira; não é possível que ambas estejam corretas ao mesmo tempo}
\end{EntryWithPhonetic}

\begin{EntryWithPhonetic}{矛头}{mao2tou2}{5,5}{⽭,⼤}[HSK 7-9]
  \definition{s.}{ponta de lança; lança}
\end{EntryWithPhonetic}

%%%%%%%%%% 牦 %%%%%%%%%%
\subsection*{牦}\addcontentsline{loh}{figure}{牦 \dpy{mao2}}

\begin{EntryWithPhonetic}{牦}{mao2}{8}{⽜}
  \definition[头]{s.}{iaque; boi da Tartária}
\end{EntryWithPhonetic}

\begin{EntryWithPhonetic}{牦牛}{mao2niu2}{8,4}{⽜,⽜}
  \definition{s.}{iaque}
\end{EntryWithPhonetic}

%%%%%%%%%% 茅 %%%%%%%%%%
\subsection*{茅}\addcontentsline{loh}{figure}{茅 \dpy{mao2}}

\begin{EntryWithPhonetic}{茅}{mao2}{8}{⾋}
  \definition*{s.}{Sobrenome: Mao}
  \definition[座]{s.}{capim-cogon | planta semelhante ao capim-cogon (como palha)}
\end{EntryWithPhonetic}

\begin{EntryWithPhonetic}{茅厕}{mao2ce4}{8,8}{⾋,⼚}
  \definition{s.}{(dialeto) latrina}
\end{EntryWithPhonetic}

\begin{EntryWithPhonetic}{茅台}{mao2tai2}{8,5}{⾋,⼝}
  \definition*{s.}{Moutai (bebida alcoólica)}
  \seealsoref{茅台酒}{mao2tai2 jiu3}
\end{EntryWithPhonetic}

\begin{EntryWithPhonetic}{茅台酒}{mao2tai2 jiu3}{8,5,10}{⾋,⼝,⾣}[HSK 7-9]
  \definition*[瓶,斤,箱,口,杯]{s.}{Maotai; Moutai (espírito forte) | Maotai (um licor chinês); Mao Tai}
\end{EntryWithPhonetic}

%%%%%%%%%% 猫 %%%%%%%%%%
\subsection*{猫}\addcontentsline{loh}{figure}{猫 \dpy{mao2}}

\begin{EntryWithPhonetic}{猫}{mao2}{11}{⽝}
  \definition{v.}{utilizado em 猫腰 \dpy{mao2yao1}}
  \seeref{mao1}
  \seealsoref{猫腰}{mao2yao1}
\end{EntryWithPhonetic}

\begin{EntryWithPhonetic}{猫腰}{mao2yao1}{11,13}{⽝,⾁}
  \definition{v.}{curvar-se}
\end{EntryWithPhonetic}

%%%%%%%%%% 茂 %%%%%%%%%%
\subsection*{茂}\addcontentsline{loh}{figure}{茂 \dpy{mao4}}

\begin{EntryWithPhonetic}{茂}{mao4}{8}{⾋}
  \definition*{s.}{Sobrenome: Mao}
  \definition{adj.}{luxuriante; exuberante; profuso | rico e esplêndido; rico e requintado}
  \definition[种]{s.}{ciclopentadieno; composto orgânico, fórmula molecular $C_5H_6$, líquido incolor, usado na fabricação de pesticidas, plásticos, etc.}
\end{EntryWithPhonetic}

\begin{EntryWithPhonetic}{茂密}{mao4mi4}{8,11}{⾋,⼧}[HSK 7-9]
  \definition{adj.}{(grama ou árvores) denso; espesso; exuberante e denso}
\end{EntryWithPhonetic}

\begin{EntryWithPhonetic}{茂盛}{mao4sheng4}{8,11}{⾋,⽫}[HSK 7-9]
  \definition{adj.}{exuberante; viçoso; abundante; florescente | exuberante; florescente; próspero}
\end{EntryWithPhonetic}

%%%%%%%%%% 冒 %%%%%%%%%%
\subsection*{冒}\addcontentsline{loh}{figure}{冒 \dpy{mao4}}

\begin{EntryWithPhonetic}{冒}{mao4}{9}{⽇}[HSK 5]
  \definition*{s.}{Sobrenome: Mao}
  \definition{adv.}{com ousadia; precipitadamente | fingidamente; falsamente; fraudulentamente}
  \definition{v.}{emitir; liberar; enviar (para cima) | arriscar; ser corajoso}
\end{EntryWithPhonetic}

\begin{EntryWithPhonetic}{冒充}{mao4chong1}{9,6}{⽇,⼉}[HSK 7-9]
  \definition{v.}{fingir ser; fazer passar alguém/algo por; confundir o falso com o verdadeiro; confundir o mau com o bom}
\end{EntryWithPhonetic}

\begin{EntryWithPhonetic}{冒犯}{mao4fan4}{9,5}{⽇,⽝}[HSK 7-9]
  \definition{v.}{ofender; afrontar}
\end{EntryWithPhonetic}

\begin{EntryWithPhonetic}{冒昧}{mao4mei4}{9,9}{⽇,⽇}[HSK 7-9]
  \definition{v.}{ter a ousadia de fazer algo; tomar a liberdade de; descreve as palavras e ações de alguém como frívolas, desconsiderando seu status, posição ou a ocasião (frequentemente usado como uma expressão de humildade)}
\end{EntryWithPhonetic}

\begin{EntryWithPhonetic}{冒险}{mao4/xian3}{9,9}{⽇,⾩}[HSK 7-9]
  \definition{v.+compl.}{arriscar; aventurar-se; correr um risco; assumir um risco}
\end{EntryWithPhonetic}

%%%%%%%%%% 贸 %%%%%%%%%%
\subsection*{贸}\addcontentsline{loh}{figure}{贸 \dpy{mao4}}

\begin{EntryWithPhonetic}{贸}{mao4}{9}{⾙}
  \definition*{s.}{Sobrenome: Mao}
  \definition{s.}{comércio; negociação}
\end{EntryWithPhonetic}

\begin{EntryWithPhonetic}{贸易}{mao4yi4}{9,8}{⾙,⽇}[HSK 5]
  \definition[笔,宗,项,个]{s.}{comércio; troca; negócios; refere-se a atividades comerciais, como a troca de mercadorias}
  \definition{v.}{fazer uma transação comercial}
\end{EntryWithPhonetic}

%%%%%%%%%% 帽 %%%%%%%%%%
\subsection*{帽}\addcontentsline{loh}{figure}{帽 \dpy{mao4}}

\begin{EntryWithPhonetic}{帽}{mao4}{12}{⼱}
  \definition[个,顶]{s.}{chapéu; boné | capa; uma coisa que cobre um objeto e tem a função ou formato de um chapéu | elmo; capacete}
\end{EntryWithPhonetic}

\begin{EntryWithPhonetic}{帽子}{mao4zi5}{12,3}{⼱,⼦}[HSK 4]
  \definition[顶,个,种]{s.}{boné; chapéu; capacete | etiqueta; rótulo; marca}
\end{EntryWithPhonetic}

%%%%%%%%%% 没 %%%%%%%%%%
\subsection*{没}\addcontentsline{loh}{figure}{没 \dpy{mei2}}

\begin{EntryWithPhonetic}{没}{mei2}{7}{⽔}[HSK 1]
  \definition{adv.}{não; nunca; negar que uma ação ou situação tenha ocorrido, com o significado de 不曾}
  \definition{pref.}{não (prefixo negativo para verbos, traduzido para outras línguas com verbos no pretérito)}
  \definition{v.}{não possuir; não ter | não existe; não há | ninguém; usado antes de 谁, 什么, 哪个, significa 全都不 | não ser tão bom quanto; ser inferior a; não chega a; não é tão bom quanto | menor que; insuficiente}
  \seeref{mo4}
  \seealsoref{不曾}{bu4 ceng2}
  \seealsoref{哪个}{na3ge5}
  \seealsoref{全都不}{quan2dou1 bu4}
  \seealsoref{谁}{shei2}
  \seealsoref{什么}{shen2me5}
\end{EntryWithPhonetic}

\begin{EntryWithPhonetic}{没错}{mei2 cuo4}{7,13}{⽔,⾦}[HSK 4]
  \definition{adv.}{está certo; é isso mesmo; não há como errar}
\end{EntryWithPhonetic}

\begin{EntryWithPhonetic}{没法儿}{mei2 fa3r5}{7,8,2}{⽔,⽔,⼉}[HSK 4]
  \definition{adv.}{não pode; sem chance}
\end{EntryWithPhonetic}

\begin{EntryWithPhonetic}{没关系}{mei2guan1xi5}{7,6,7}{⽔,⼋,⽷}[HSK 1]
  \definition{v.}{está tudo bem; não é nada; não importa; não se preocupe}
  \seealsoref{没有关系}{mei2you3guan1xi5}
\end{EntryWithPhonetic}

\begin{EntryWithPhonetic}{没劲}{mei2jin4}{7,7}{⽔,⼒}[HSK 7-9]
  \definition{adj.}{chato; sem graça; entediante; desinteressante (expressando insatisfação)}
  \definition{v.}{cansar; entediar-se; não se sentir bem}
\end{EntryWithPhonetic}

\begin{EntryWithPhonetic}{没了}{mei2le5}{7,2}{⽔,⼅}
  \definition{v.}{estar morto | deixar de existir}
\end{EntryWithPhonetic}

\begin{EntryWithPhonetic}{没什么}{mei2 shen2 me5}{7,4,3}{⽔,⼈,⼃}[HSK 1]
  \definition{expr.}{não é nada; está tudo bem; não importa}
\end{EntryWithPhonetic}

\begin{EntryWithPhonetic}{没事儿}{mei2 shi4r5}{7,8,2}{⽔,⼅,⼉}[HSK 1]
  \definition{expr.}{fora de perigo; nada sério | não importa; não é nada; está tudo bem; não importa | está tudo bem; sem problemas; não se preocupe com isso; não é grande coisa; não há nada errado}
  \definition{v.}{não ter nada para fazer; ser livre; estar perdido | estar desempregado; estar sem trabalho | não ter responsabilidade}
\end{EntryWithPhonetic}

\begin{EntryWithPhonetic}{没说的}{mei2shuo1de5}{7,9,8}{⽔,⾔,⽩}[HSK 7-9]
  \definition{adj.}{impecável; perfeito | óbvio; sem dúvida alguma; não há necessidade de dizer mais nada sobre isso | naturalmente; é claro | inquestionável; realmente bom; sem defeitos a criticar | sem problemas}
\end{EntryWithPhonetic}

\begin{EntryWithPhonetic}{没完没了}{mei2wan2-mei2liao3}{7,7,7,2}{⽔,⼧,⽔,⼅}[HSK 7-9]
  \definition{expr.}{``Isso nunca acaba.''; interminável; sem fim; ininterrupto; que continua indefinidamente; descreve alguém que fala ou age sem parar}
\end{EntryWithPhonetic}

\begin{EntryWithPhonetic}{没想到}{mei2 xiang3 dao4}{7,13,8}{⽔,⼼,⼑}[HSK 4]
  \definition{expr.}{não esperava; inesperado}
\end{EntryWithPhonetic}

\begin{EntryWithPhonetic}{没意思}{mei2 yi4si5}{7,13,9}{⽔,⼼,⼼}[HSK 7-9]
  \definition{adj.}{entediante; tedioso; sem graça nenhuma}
\end{EntryWithPhonetic}

\begin{EntryWithPhonetic}{没用}{mei2 yong4}{7,5}{⽔,⽤}[HSK 3]
  \definition{adj.}{inútil; imprestável; sem valor; sem préstimo; vão; que não serve para nada}
\end{EntryWithPhonetic}

\begin{EntryWithPhonetic}{没有}{mei2 you3}{7,6}{⽔,⽉}[HSK 1]
  \definition{adv.}{ainda não; (usado com o pretérito) não; ação ou estado negativo ocorreu}
  \definition{v.}{não há; não tem; não existe}
\end{EntryWithPhonetic}

\begin{EntryWithPhonetic}{没有次序}{mei2you3 ci4xu4}{7,6,6,7}{⽔,⽉,⽋,⼴}
  \definition{adj.}{sem ordem; nenhuma ordem}
\end{EntryWithPhonetic}

\begin{EntryWithPhonetic}{没有关系}{mei2you3guan1xi5}{7,6,6,7}{⽔,⽉,⼋,⽷}
  \definition{expr.}{Está tudo bem; sem problemas}
  \seealsoref{没关系}{mei2guan1xi5}
\end{EntryWithPhonetic}

\begin{EntryWithPhonetic}{没有哪一种东西}{mei2you3 na3 yi4 zhong3 dong1xi1}{7,6,9,1,9,5,6}{⽔,⽉,⼝,⼀,⽲,⼀,⾑}
  \definition{pron.}{nada; não existe tal coisa}
\end{EntryWithPhonetic}

\begin{EntryWithPhonetic}{没有谁}{mei2you3 shei2}{7,6,10}{⽔,⽉,⾔}
  \definition{pron.}{ninguém}
\end{EntryWithPhonetic}

\begin{EntryWithPhonetic}{没有意思}{mei2you3yi4si5}{7,6,13,9}{⽔,⽉,⼼,⼼}
  \definition{adj.}{entediante; desinteressante}
\end{EntryWithPhonetic}

\begin{EntryWithPhonetic}{没辙}{mei2/zhe2}{7,16}{⽔,⾞}[HSK 7-9]
  \definition{v.+compl.}{sem saber o que fazer; ser incapaz de encontrar uma saída; não ter jeito}
\end{EntryWithPhonetic}

\begin{EntryWithPhonetic}{没准儿}{mei2/zhun3r5}{7,10,2}{⽔,⼎,⼉}[HSK 7-9]
  \definition{adj.}{não confiável}
  \definition{adv.}{talvez; quem sabe}
  \definition{v.+compl.}{não ter certeza; estar incerto; não se saber ao certo}
\end{EntryWithPhonetic}

%%%%%%%%%% 枚 %%%%%%%%%%
\subsection*{枚}\addcontentsline{loh}{figure}{枚 \dpy{mei2}}

\begin{EntryWithPhonetic}{枚}{mei2}{8}{⽊}[HSK 7-9]
  \definition*{s.}{Sobrenome: Mei}
  \definition{clas.}{utilizado para objetos pequenos | utilizado para moedas, anéis, distintivos, pérolas, medalhas esportivas, foguetes, satélites etc.}
  \definition{s.}{Arcaico: tronco de árvore (significado original) | Arcaico: chicote | Arcaico: pino de madeira, usado como mordaça para soldados em marcha}
\end{EntryWithPhonetic}

%%%%%%%%%% 玫 %%%%%%%%%%
\subsection*{玫}\addcontentsline{loh}{figure}{玫 \dpy{mei2}}

\begin{EntryWithPhonetic}{玫}{mei2}{8}{⽟}
  \definition[朵]{s.}{rosa | Arcaico: um tipo de jade bonito}
\end{EntryWithPhonetic}

\begin{EntryWithPhonetic}{玫瑰}{mei2gui5}{8,13}{⽟,⽟}[HSK 7-9]
  \definition[束,朵,棵,株]{s.}{rosa; um arbusto decíduo, seus ramos são espinhosos, suas folhas são ovais e suas flores, que podem ser vermelho-púrpura, brancas e de outras cores, são perfumadas e ornamentais}
\end{EntryWithPhonetic}

%%%%%%%%%% 眉 %%%%%%%%%%
\subsection*{眉}\addcontentsline{loh}{figure}{眉 \dpy{mei2}}

\begin{EntryWithPhonetic}{眉}{mei2}{9}{⽬}
  \definition*{s.}{Sobrenome: Mei}
  \definition[个]{s.}{sobrancelha | a margem superior de uma página; o espaço em branco na parte superior da página}
\end{EntryWithPhonetic}

\begin{EntryWithPhonetic}{眉开眼笑}{mei2kai1-yan3xiao4}{9,4,11,10}{⽬,⼶,⽬,⽵}[HSK 7-9]
  \definition{expr.}{parecer alegre; um semblante radiante; estar sempre sorrindo; irradiar alegria; estar cheio de felicidade; rosto transbordando de sorrisos; sentir-se feliz e sorrir; sorrir de orelha a orelha; os olhos brilhando de alegria; o rosto se iluminando de sorrisos; os olhos brilhantes dançando de alegria; sorrir alegremente (de orelha a orelha); muito feliz}
\end{EntryWithPhonetic}

\begin{EntryWithPhonetic}{眉毛}{mei2mao5}{9,4}{⽬,⽑}[HSK 7-9]
  \definition[根]{s.}{sobrancelha; pelos que crescem ao longo da borda superior da órbita ocular humana}
\end{EntryWithPhonetic}

\begin{EntryWithPhonetic}{眉头}{mei2tou2}{9,5}{⽬,⼤}
  \definition{s.}{testa; área próxima às sobrancelhas}
\end{EntryWithPhonetic}

%%%%%%%%%% 梅 %%%%%%%%%%
\subsection*{梅}\addcontentsline{loh}{figure}{梅 \dpy{mei2}}

\begin{EntryWithPhonetic}{梅}{mei2}{11}{⽊}
  \definition*{s.}{Sobrenome: Mei}
  \definition{s.}{ameixa | flor de ameixa | ameixeira | estação chuvosa}
\end{EntryWithPhonetic}

\begin{EntryWithPhonetic}{梅花}{mei2 hua1}{11,7}{⽊,⾋}[HSK 6]
  \definition[朵,枝,片,瓣,束,株]{s.}{paus ♣ (um naipe em jogos de cartas) | flor de ameixa | doçura-de-inverno; refere-se especificamente à flor-de-inverno ; também se refere a algo que se parece com esta flor}
  \seealsoref{方片}{fang1 pian4}
  \seealsoref{黑桃}{hei1 tao2}
  \seealsoref{红心}{hong2 xin1}
\end{EntryWithPhonetic}

\begin{EntryWithPhonetic}{梅赛德斯-奔驰}{mei2sai4de2si1-ben1chi2}{11,14,15,12,8,6}{⽊,⾙,⼻,⽄,⼤,⾺}
  \definition*{s.}{Mercedes-Benz}
\end{EntryWithPhonetic}

%%%%%%%%%% 媒 %%%%%%%%%%
\subsection*{媒}\addcontentsline{loh}{figure}{媒 \dpy{mei2}}

\begin{EntryWithPhonetic}{媒}{mei2}{12}{⼥}
  \definition{s.}{casamenteiro; intermediário | intermediário; médio}
  \definition{v.}{fazer uma combinação}
\end{EntryWithPhonetic}

\begin{EntryWithPhonetic}{媒体}{mei2ti3}{12,7}{⼥,⼈}[HSK 3]
  \definition[家,个,种]{s.}{mídia; mídia de massa; vários meios de comunicação e transmissão de informações, como televisão, rádio, jornais, etc.}
\end{EntryWithPhonetic}

%%%%%%%%%% 煤 %%%%%%%%%%
\subsection*{煤}\addcontentsline{loh}{figure}{煤 \dpy{mei2}}

\begin{EntryWithPhonetic}{煤}{mei2}{13}{⽕}[HSK 5]
  \definition[块,吨,斤,堆]{s.}{carvão; carvão vegetal; minério sólido preto}
\end{EntryWithPhonetic}

\begin{EntryWithPhonetic}{煤矿}{mei2kuang4}{13,8}{⽕,⽯}[HSK 7-9]
  \definition{s.}{mina de carvão}
\end{EntryWithPhonetic}

\begin{EntryWithPhonetic}{煤气}{mei2 qi4}{13,4}{⽕,⽓}[HSK 5]
  \definition[罐,瓶]{s.}{gás; gás de carvão; gás obtido a partir do processamento do carvão não tem cor nem odor, é tóxico e pode ser queimado ou utilizado como matéria-prima na indústria química | envenenamento por monóxido de carbono}
\end{EntryWithPhonetic}

\begin{EntryWithPhonetic}{煤炭}{mei2tan4}{13,9}{⽕,⽕}[HSK 7-9]
  \definition{s.}{carvão}
\end{EntryWithPhonetic}

%%%%%%%%%% 每 %%%%%%%%%%
\subsection*{每}\addcontentsline{loh}{figure}{每 \dpy{mei3}}

\begin{EntryWithPhonetic}{每}{mei3}{7}{⽏}[HSK 3]
  \definition{adv.}{cada um; cada qual; indica qualquer uma das repetições ou um conjunto de repetições de um movimento}
  \definition{pron.}{cada; cada um; cada qual; refere-se a qualquer indivíduo do grupo, enfatizando as semelhanças entre os indivíduos}
\end{EntryWithPhonetic}

\begin{EntryWithPhonetic}{每次}{mei3ci4}{7,6}{⽏,⽋}
  \definition{adv.}{toda vez | cada vez}
\end{EntryWithPhonetic}

\begin{EntryWithPhonetic}{每当}{mei3dang1}{7,6}{⽏,⼹}[HSK 7-9]
  \definition{adv.}{sempre que; todas as vezes; toda vez que isso acontece; em qualquer momento}
\end{EntryWithPhonetic}

\begin{EntryWithPhonetic}{每逢}{mei3feng2}{7,10}{⽏,⾡}[HSK 7-9]
  \definition{adv.}{sempre que; em todas as ocasiões; em todas as situações; toda vez que me deparo com isso; toda vez que chego}
\end{EntryWithPhonetic}

\begin{EntryWithPhonetic}{每个}{mei3ge4}{7,3}{⽏,⼈}
  \definition{pron.}{cada; cada um}
\end{EntryWithPhonetic}

\begin{EntryWithPhonetic}{每个人}{mei3ge5ren2}{7,3,2}{⽏,⼈,⼈}
  \definition{pron.}{todo mundo | todos}
\end{EntryWithPhonetic}

\begin{EntryWithPhonetic}{每天}{mei3tian1}{7,4}{⽏,⼤}
  \definition{adv.}{todo dia | cada dia}
\end{EntryWithPhonetic}

%%%%%%%%%% 美 %%%%%%%%%%
\subsection*{美}\addcontentsline{loh}{figure}{美 \dpy{mei3}}

\begin{EntryWithPhonetic}{美}{mei3}{9}{⽺}[HSK 3]
  \definition*{s.}{Abreviatura de América, 美洲 | Abreviatura de Estados Unidos da América, 美国 | As Américas, 美洲}
  \definition{adj.}{belo; bonito (oposto de 丑) | muito satisfatório; bom; agradável}
  \definition{s.}{beleza (oposto de 丑)}
  \definition{v.}{embelezar; tornar mais bonito | estar satisfeito consigo mesmo; orgulhar-se; sentir-se presunçoso}
  \seealsoref{丑}{chou3}
  \seealsoref{美国}{mei3guo2}
  \seealsoref{美洲}{mei3zhou1}
\end{EntryWithPhonetic}

\begin{EntryWithPhonetic}{美德}{mei3de2}{9,15}{⽺,⼻}[HSK 7-9]
  \definition*{s.}{My Duc District (Hanoi)}
  \definition[种,些]{s.}{virtude; excelência moral; bom caráter}
\end{EntryWithPhonetic}

\begin{EntryWithPhonetic}{美观}{mei3guan1}{9,6}{⽺,⾒}[HSK 7-9]
  \definition{adj.}{artístico; belo; agradável à vista; que agrada aos olhos}
\end{EntryWithPhonetic}

\begin{EntryWithPhonetic}{美国}{mei3guo2}{9,8}{⽺,⼞}
  \definition*{s.}{Estados Unidos da América}
\end{EntryWithPhonetic}

\begin{EntryWithPhonetic}{美国人}{mei3guo2ren2}{9,8,2}{⽺,⼞,⼈}
  \definition{s.}{americano | pessoa ou povo dos Estados Unidos da América}
\end{EntryWithPhonetic}

\begin{EntryWithPhonetic}{美好}{mei3 hao3}{9,6}{⽺,⼥}[HSK 3]
  \definition{adj.}{bem; feliz; glorioso; descreve a vida, os desejos, etc. como sendo muito bons e satisfatórios}
\end{EntryWithPhonetic}

\begin{EntryWithPhonetic}{美化}{mei3hua4}{9,4}{⽺,⼔}[HSK 7-9]
  \definition{v.}{embelezar; enfeitar; adornar; estética através da decoração e do embelezamento | embelezar; retratar o mal como bem e o feio como belo}
\end{EntryWithPhonetic}

\begin{EntryWithPhonetic}{美甲}{mei3jia3}{9,5}{⽺,⽥}
  \definition{s.}{manicure e/ou pedicure}
\end{EntryWithPhonetic}

\begin{EntryWithPhonetic}{美金}{mei3 jin1}{9,8}{⽺,⾦}[HSK 4]
  \definition{s.}{USD; dólar americano: a moeda local dos Estados Unidos}
\end{EntryWithPhonetic}

\begin{EntryWithPhonetic}{美景}{mei3jing3}{9,12}{⽺,⽇}[HSK 7-9]
  \definition{s.}{paisagem deslumbrante; paisagens belíssimas (como o mar, a terra ou o céu)}
\end{EntryWithPhonetic}

\begin{EntryWithPhonetic}{美丽}{mei3li4}{9,7}{⽺,⼀}[HSK 3]
  \definition{adj.}{bonito; lindo; capaz de proporcionar uma sensação de beleza}
\end{EntryWithPhonetic}

\begin{EntryWithPhonetic}{美满}{mei3man3}{9,13}{⽺,⽔}[HSK 7-9]
  \definition{adj.}{feliz; perfeitamente satisfatório; lindo e perfeito}
\end{EntryWithPhonetic}

\begin{EntryWithPhonetic}{美妙}{mei3miao4}{9,7}{⽺,⼥}[HSK 7-9]
  \definition{adj.}{esplêndido; belo; maravilhoso; descreve obras literárias, música, sentimentos, etc., como belos, únicos ou engenhosos}
\end{EntryWithPhonetic}

\begin{EntryWithPhonetic}{美女}{mei3 nv3}{9,3}{⽺,⼥}[HSK 4]
  \definition[个,位,名,些]{s.}{beldade; mulher bonita; uma jovem linda}
\end{EntryWithPhonetic}

\begin{EntryWithPhonetic}{美人}{mei3ren2}{9,2}{⽺,⼈}[HSK 7-9]
  \definition[个,位,名]{s.}{beleza; mulher bonita}
\end{EntryWithPhonetic}

\begin{EntryWithPhonetic}{美容}{mei3 rong2}{9,10}{⽺,⼧}[HSK 6]
  \definition{v.}{embelezar; melhorar a aparência de alguém; deixar seu rosto bonito retocando, cuidando, etc.}
\end{EntryWithPhonetic}

\begin{EntryWithPhonetic}{美食}{mei3 shi2}{9,9}{⽺,⾷}[HSK 3]
  \definition[种,道,桌]{s.}{iguaria; (gastronomia) comida saborosa}
\end{EntryWithPhonetic}

\begin{EntryWithPhonetic}{美术}{mei3shu4}{9,5}{⽺,⽊}[HSK 3]
  \definition[种]{s.}{arte; artes plásticas: arte que ocupa um determinado espaço, compõe imagens estéticas e permite que as pessoas apreciem visualmente, incluindo pintura, escultura, arquitetura, etc. | pintura; pintura tradicional chinesa}
\end{EntryWithPhonetic}

\begin{EntryWithPhonetic}{美味}{mei3wei4}{9,8}{⽺,⼝}[HSK 7-9]
  \definition[顿]{s.}{comida deliciosa; iguaria (\emph{delicacy}); sabor delicioso}
\end{EntryWithPhonetic}

\begin{EntryWithPhonetic}{美学}{mei3xue2}{9,8}{⽺,⼦}
  \definition{s.}{estética; a ciência que estuda as leis e os princípios gerais da beleza na natureza, na sociedade e na arte explora principalmente a natureza da beleza, a relação entre arte e realidade e as leis gerais da criação artística}
\end{EntryWithPhonetic}

\begin{EntryWithPhonetic}{美元}{mei3yuan2}{9,4}{⽺,⼉}[HSK 3]
  \definition*[元,笔,沓]{s.}{Dólar Americano; a moeda dos Estados Unidos}
\end{EntryWithPhonetic}

\begin{EntryWithPhonetic}{美中不足}{mei3zhong1-bu4zu2}{9,4,4,7}{⽺,⼁,⼀,⾜}[HSK 7-9]
  \definition{expr.}{belo, porém incompleto (que carece de perfeição); uma imperfeição em algo perfeito; uma falha em algo aparentemente perfeito; um problema; a parte desagradável de algo agradável; algumas imperfeições em algo aparentemente perfeito; alguma pequena falta de perfeição; algo que não atinge a perfeição; há uma falha no ato; é bom, mas ainda tem falhas; um obstáculo; uma falha que prejudica a beleza; um defeito em algo aparentemente perfeito}
\end{EntryWithPhonetic}

\begin{EntryWithPhonetic}{美洲}{mei3zhou1}{9,9}{⽺,⽔}
  \definition*{s.}{América (incluindo Norte, Central e Sul)}
\end{EntryWithPhonetic}

\begin{EntryWithPhonetic}{美洲人}{mei3zhou1ren2}{9,9,2}{⽺,⽔,⼈}
  \definition{s.}{americano | pessoa ou povo do continente Americano}
\end{EntryWithPhonetic}

\begin{EntryWithPhonetic}{美滋滋}{mei3zi1zi1}{9,12,12}{⽺,⽔,⽔}[HSK 7-9]
  \definition{interj.}{Me sentindo ótimo!; exultante; muito feliz; muito satisfeito consigo mesmo}
\end{EntryWithPhonetic}

%%%%%%%%%% 妹 %%%%%%%%%%
\subsection*{妹}\addcontentsline{loh}{figure}{妹 \dpy{mei4}}

\begin{EntryWithPhonetic}{妹}{mei4}{8}{⼥}[HSK 1]
  \definition*{s.}{Sobrenome: Mei}
  \definition[个]{s.}{irmã mais nova | parente do sexo feminino da mesma geração | jovem garota; jovem mulher ou menina}
  \seealsoref{妹妹}{mei4 mei5}
\end{EntryWithPhonetic}

\begin{EntryWithPhonetic}{妹夫}{mei4fu5}{8,4}{⼥,⼤}
  \definition{s.}{marido da irmã mais nova}
\end{EntryWithPhonetic}

\begin{EntryWithPhonetic}{妹妹}{mei4 mei5}{8,8}{⼥,⼥}[HSK 1]
  \definition[个]{s.}{irmã mais nova}
\end{EntryWithPhonetic}

%%%%%%%%%% 谜 %%%%%%%%%%
\subsection*{谜}\addcontentsline{loh}{figure}{谜 \dpy{mei4}}

\begin{EntryWithPhonetic}{谜}{mei4}{11}{⾔}
  \definition[个]{s.}{enigma}
  \seeref{mi2}
  \seealsoref{谜儿}{mei4r5}
\end{EntryWithPhonetic}

\begin{EntryWithPhonetic}{谜儿}{mei4r5}{11,2}{⾔,⼉}
  \definition{s.}{enigma; mistério}
\end{EntryWithPhonetic}

%%%%%%%%%% 魅 %%%%%%%%%%
\subsection*{魅}\addcontentsline{loh}{figure}{魅 \dpy{mei4}}

\begin{EntryWithPhonetic}{魅}{mei4}{14}{⿁}
  \definition{s.}{espírito maligno; demônio | \emph{goblin}; trasgo; gnomo; duende maléfico}
  \definition{v.}{atormentar; cativar}
\end{EntryWithPhonetic}

\begin{EntryWithPhonetic}{魅力}{mei4li4}{14,2}{⿁,⼒}[HSK 7-9]
  \definition[种]{s.}{charme; feitiço; glamour; bruxaria; carisma; feitiçaria; encanto; fascínio; encantamento; o poder de atrair e motivar pessoas}
\end{EntryWithPhonetic}

%%%%%%%%%% 闷 %%%%%%%%%%
\subsection*{闷}\addcontentsline{loh}{figure}{闷 \dpy{men1}}

\begin{EntryWithPhonetic}{闷}{men1}{7}{⾨}[HSK 7-9]
  \definition{adj.}{abafado; fechado; sufocante; baixa pressão de ar ou má circulação de ar | abafado; som baixo ou opaco}
  \definition{v.}{cubrir bem; fazer algo hermético | ficar sem fala; parar de falar | fechar a si mesmo ou alguém dentro de casa; ficar em casa e não sair}
  \seeref{men4}
\end{EntryWithPhonetic}

\begin{EntryWithPhonetic}{闷热}{men1re4}{7,10}{⾨,⽕}
  \definition{adj.}{abafado | quente e abafado | sufocantemente quente | quente e sensual}
\end{EntryWithPhonetic}

%%%%%%%%%% 门 %%%%%%%%%%
\subsection*{门}\addcontentsline{loh}{figure}{门 \dpy{men2}}

\begin{EntryWithPhonetic}{门}{men2}{3}{⾨}[HSK 1][Kangxi 169]
  \definition*{s.}{Sobrenome: Men}
  \definition{clas.}{para equipamentos de artilharia (por exemplo: canhões) | para trabalhos escolares, ciência e tecnologia, etc. | para idiomas | para casamentos | para parentes}
  \definition[个,把,道,扇]{s.}{entradas e saídas de edifícios, veículos, navios, aviões, etc. | válvula; interruptor; algo que funciona como um interruptor ou como uma porta | habilidade; método; acesso; maneira de fazer algo | família; ramo de uma família ou clã | seita (religiosa); escola (de pensamento); faculdades acadêmicas, ideológicas ou religiosas | classe; categoria; ramo de estudo; refere-se à categoria geral de coisas | filo; segundo nível da classificação biológica | (computador) \emph{gate}; porta (lógica) | porta; portão; entrada; refere-se a uma porta que pode ser aberta e fechada, instalada na entrada e saída | qualquer abertura; partes de objetos que podem ser abertas e fechadas | orifício no corpo humano; refere-se especificamente aos orifícios do corpo humano | estudar com o mesmo professor; refere-se especificamente ao professor ou mestre | posição em um jogo de apostas (em relação ao local onde se senta ou onde se faz uma aposta)}
\end{EntryWithPhonetic}

\begin{EntryWithPhonetic}{门当户对}{men2dang1-hu4dui4}{3,6,4,5}{⾨,⼹,⼾,⼨}[HSK 7-9]
  \definition{expr.}{``Casar com alguém de posição social equivalente.''; compatibilidade social e econômica adequada (para casamento); (um possível parceiro para casamento) uma combinação adequada; as famílias são bem compatíveis em termos de status social}
\end{EntryWithPhonetic}

\begin{EntryWithPhonetic}{门道}{men2dao4}{3,12}{⾨,⾡}
  \definition{s.}{porta | portal}
  \seeref{men2dao5}
\end{EntryWithPhonetic}

\begin{EntryWithPhonetic}{门道}{men2dao5}{3,12}{⾨,⾡}
  \definition{s.}{talento | a maneira de fazer algo}
  \seeref{men2dao4}
\end{EntryWithPhonetic}

\begin{EntryWithPhonetic}{门槛}{men2kan3}{3,14}{⾨,⽊}[HSK 7-9]
  \definition{s.}{soleira; batente da porta; a viga horizontal ou faixa de pedra na parte inferior do batente da porta, próxima ao chão, etc. | o requisito para entrar em um determinado campo; metaforicamente, refere-se aos padrões ou condições para entrar em um determinado intervalo}
\end{EntryWithPhonetic}

\begin{EntryWithPhonetic}{门口}{men2 kou3}{3,3}{⾨,⼝}[HSK 1]
  \definition[个]{s.}{porta; portão; entrada; porta de entrada}
\end{EntryWithPhonetic}

\begin{EntryWithPhonetic}{门铃}{men2ling2}{3,10}{⾨,⾦}[HSK 7-9]
  \definition{s.}{campainha; uma campainha ou campainha elétrica para bater à porta}
\end{EntryWithPhonetic}

\begin{EntryWithPhonetic}{门路}{men2lu5}{3,13}{⾨,⾜}[HSK 7-9]
  \definition{s.}{maneira de fazer algo; jeito | conexões sociais (para conseguir empregos, etc.) | jeito; saber fazer; truque do ofício; o segredo para fazer as coisas acontecerem}
  \seealsoref{门道}{men2dao5}
\end{EntryWithPhonetic}

\begin{EntryWithPhonetic}{门票}{men2 piao4}{3,11}{⾨,⽰}[HSK 1]
  \definition{s.}{bilhete de entrada; bilhete de admissão; ingressos para locais de turismo, entretenimento, etc.}
\end{EntryWithPhonetic}

\begin{EntryWithPhonetic}{门诊}{men2 zhen3}{3,7}{⾨,⾔}[HSK 5]
  \definition{s.}{(no hospital) clínica ambulatorial; seção para pacientes ambulatoriais; local onde os médicos atendem pacientes que não estão internados no hospital}
\end{EntryWithPhonetic}

%%%%%%%%%% 闷 %%%%%%%%%%
\subsection*{闷}\addcontentsline{loh}{figure}{闷 \dpy{men4}}

\begin{EntryWithPhonetic}{闷}{men4}{7}{⾨}[HSK 7-9]
  \definition{adj.}{entediado; deprimido; irritado; desanimado | hermeticamente fechado; selado | triste e silencioso; chateado | hermético}
  \definition{s.}{desânimo}
  \seeref{men1}
\end{EntryWithPhonetic}

%%%%%%%%%% 们 %%%%%%%%%%
\subsection*{们}\addcontentsline{loh}{figure}{们 \dpy{men5}}

\begin{EntryWithPhonetic}{们}{men5}{5}{⼈}[HSK 1]
  \definition{suf.}{usado após pronomes ou substantivos que se referem a pessoas para indicar pluralidade}
\end{EntryWithPhonetic}

%%%%%%%%%% 蒙 %%%%%%%%%%
\subsection*{蒙}\addcontentsline{loh}{figure}{蒙 \dpy{meng1}}

\begin{EntryWithPhonetic}{蒙}{meng1}{13}{⾋}[HSK 6]
  \definition{adj.}{inconsciente; sem sentido;  em coma | confuso; perplexo}
  \definition{v.}{enganar; enganar; trapacear; iludir; trair | fazer um palpite ousado; dar um palpite ousado; arriscar-se}
  \seeref{meng2}
  \seeref{meng3}
\end{EntryWithPhonetic}

%%%%%%%%%% 萌 %%%%%%%%%%
\subsection*{萌}\addcontentsline{loh}{figure}{萌 \dpy{meng2}}

\begin{EntryWithPhonetic}{萌}{meng2}{11}{⾋}
  \definition*{s.}{Sobrenome: Meng}
  \definition{s.}{broto; rebento | Arcaico: o povo comum}
  \definition{v.}{(plantas) brotar; surgir; brotar; germinar | começar; surgir; ocorrer; emergir}
\end{EntryWithPhonetic}

\begin{EntryWithPhonetic}{萌发}{meng2fa1}{11,5}{⾋,⼜}[HSK 7-9]
  \definition{v.}{brotar; germinar; germinação de sementes ou esporos | emergir; vir à tona; isso se refere metaforicamente ao início de um determinado pensamento, ideia ou sentimento}
\end{EntryWithPhonetic}

\begin{EntryWithPhonetic}{萌芽}{meng2ya2}{11,7}{⾋,⾋}[HSK 7-9]
  \definition{s.}{semente; germe; rudimento; uma metáfora para algo novo e ainda não totalmente desenvolvido}
  \definition{v.}{brotar; germinar; desabrochar; o brotar de plantas é uma metáfora para algo que está apenas começando a acontecer}
\end{EntryWithPhonetic}

%%%%%%%%%% 盟 %%%%%%%%%%
\subsection*{盟}\addcontentsline{loh}{figure}{盟 \dpy{meng2}}

\begin{EntryWithPhonetic}{盟}{meng2}{13}{⽫}
  \definition{adj.}{jurados (irmãos); aliados | jurado; antigamente, referia-se a uma irmandade jurada}
  \definition{s.}{aliança; coligação | liga (uma divisão administrativa da Região Autônoma da Mongólia Interior, correspondente a uma prefeitura)}
  \definition{v.}{aliar-se | fazer um juramento; jurar}
  \seeref{ming2}
\end{EntryWithPhonetic}

\begin{EntryWithPhonetic}{盟友}{meng2you3}{13,4}{⽫,⼜}[HSK 7-9]
  \definition{s.}{aliado; amigo leal | país (ou estado) aliado}
\end{EntryWithPhonetic}

%%%%%%%%%% 蒙 %%%%%%%%%%
\subsection*{蒙}\addcontentsline{loh}{figure}{蒙 \dpy{meng2}}

\begin{EntryWithPhonetic}{蒙}{meng2}{13}{⾋}[HSK 6]
  \definition*{s.}{Sobrenome: Meng}
  \definition{adj.}{ignorância; analfabetismo; falta de instrução | nebuloso; aparência pequena e pouco clara, como chuva ou neblina}
  \definition{s.}{aberto; inicial}
  \definition{v.}{cobrir; espalhar | receber apoio | receber; encontrar-se com; encontrar-se; palavras respeitosas; expressam os benefícios recebidos de outros | sofrer; incorrer}
  \seeref{meng1}
  \seeref{meng3}
\end{EntryWithPhonetic}

\begin{EntryWithPhonetic}{蒙面}{meng2mian4}{13,9}{⾋,⾯}
  \definition{adj.}{descarado | desavergonhado | mascarado}
  \definition{v.}{cobrir o rosto | usar uma máscara}
\end{EntryWithPhonetic}

%%%%%%%%%% 朦 %%%%%%%%%%
\subsection*{朦}\addcontentsline{loh}{figure}{朦 \dpy{meng2}}

\begin{EntryWithPhonetic}{朦}{meng2}{17}{⽉}
  \definition{adj.}{indistinto | pouco claro}
  \definition{v.}{enganar}
\end{EntryWithPhonetic}

\begin{EntryWithPhonetic}{朦胧}{meng2long2}{17,9}{⽉,⾁}[HSK 7-9]
  \definition{adj.}{fraco; nebuloso; obscuro; pouco claro; vago}
\end{EntryWithPhonetic}

%%%%%%%%%% 猛 %%%%%%%%%%
\subsection*{猛}\addcontentsline{loh}{figure}{猛 \dpy{meng3}}

\begin{EntryWithPhonetic}{猛}{meng3}{11}{⽝}[HSK 6]
  \definition*{s.}{Sobrenome: Meng}
  \definition{adj.}{feroz; violento | enérgico; vigoroso | valente}
  \definition{adv.}{de repente; abruptamente | vigorosamente; com força repentina | (coloquial) ao contentamento do coração; de todo o coração | ferozmente; violentamente}
\end{EntryWithPhonetic}

\begin{EntryWithPhonetic}{猛烈}{meng3lie4}{11,10}{⽝,⽕}[HSK 7-9]
  \definition{adj.}{feroz; violento; vigoroso}
\end{EntryWithPhonetic}

\begin{EntryWithPhonetic}{猛然}{meng3ran2}{11,12}{⽝,⽕}[HSK 7-9]
  \definition{adv.}{de repente; abruptamente; indica ação repentina e rápida}
\end{EntryWithPhonetic}

%%%%%%%%%% 蒙 %%%%%%%%%%
\subsection*{蒙}\addcontentsline{loh}{figure}{蒙 \dpy{meng3}}

\begin{EntryWithPhonetic}{蒙}{meng3}{13}{⾋}
  \definition{s.}{grupo étnico mongol; mongol}
  \seeref{meng1}
  \seeref{meng2}
\end{EntryWithPhonetic}

%%%%%%%%%% 懵 %%%%%%%%%%
\subsection*{懵}\addcontentsline{loh}{figure}{懵 \dpy{meng3}}

\begin{EntryWithPhonetic}{懵}{meng3}{18}{⼼}
  \definition{adj.}{confuso; ignorante; irracional | inconsciente; entorpecido}
\end{EntryWithPhonetic}

\begin{EntryWithPhonetic}{懵懂}{meng3dong3}{18,15}{⼼,⼼}
  \definition{adj.}{confuso | ignorante}
\end{EntryWithPhonetic}

%%%%%%%%%% 梦 %%%%%%%%%%
\subsection*{梦}\addcontentsline{loh}{figure}{梦 \dpy{meng4}}

\begin{EntryWithPhonetic}{梦}{meng4}{11}{⼣}[HSK 4]
  \definition*{s.}{Sobrenome: Meng}
  \definition[个,场]{s.}{sonho; atividade de representação no cérebro durante o sono}
  \definition{v.}{sonhar; ter um sonho}
\end{EntryWithPhonetic}

\begin{EntryWithPhonetic}{梦幻}{meng4huan4}{11,4}{⼣,⼳}[HSK 7-9]
  \definition[种,片]{s.}{ilusão; sonho; devaneio; cenas e situações estranhas que aparecem nos sonhos}
\end{EntryWithPhonetic}

\begin{EntryWithPhonetic}{梦见}{meng4 jian4}{11,4}{⼣,⾒}[HSK 4]
  \definition{v.}{sonhar; sonhar com; ver em um sonho}
\end{EntryWithPhonetic}

\begin{EntryWithPhonetic}{梦想}{meng4xiang3}{11,13}{⼣,⼼}[HSK 4]
  \definition[个,种,些,番]{s.}{sonho; esperança vã; sonho irreal; divagação; um desejo ou ideia que você espera particularmente realizar}
  \definition{v.}{sonhar; desejar sinceramente; ansiar}
\end{EntryWithPhonetic}

%%%%%%%%%% 眯 %%%%%%%%%%
\subsection*{眯}\addcontentsline{loh}{figure}{眯 \dpy{mi1}}

\begin{EntryWithPhonetic}{眯}{mi1}{11}{⽬}
  \definition{v.}{estreitar os olhos | esmagar | (dialeto) tirar uma soneca}
  \seeref{mi2}
\end{EntryWithPhonetic}

%%%%%%%%%% 弥 %%%%%%%%%%
\subsection*{弥}\addcontentsline{loh}{figure}{弥 \dpy{mi2}}

\begin{EntryWithPhonetic}{弥}{mi2}{8}{⼸}
  \definition*{s.}{Sobrenome: Mi}
  \definition{adj.}{cheio; inteiro}
  \definition{adv.}{Literário: mais; ainda mais}
  \definition{v.}{transbordar; encher | cobrir; encher}
\end{EntryWithPhonetic}

\begin{EntryWithPhonetic}{弥补}{mi2bu3}{8,7}{⼸,⾐}[HSK 7-9]
  \definition{v.}{remediar; compensar; reparar; restituir; consertar; preencher; inventar}
\end{EntryWithPhonetic}

\begin{EntryWithPhonetic}{弥漫}{mi2man4}{8,14}{⼸,⽔}[HSK 7-9]
  \definition{v.}{encher; transbordar; preencher o ar; espalhar-se por toda parte}
\end{EntryWithPhonetic}

%%%%%%%%%% 迷 %%%%%%%%%%
\subsection*{迷}\addcontentsline{loh}{figure}{迷 \dpy{mi2}}

\begin{EntryWithPhonetic}{迷}{mi2}{9}{⾡}[HSK 3]
  \definition[个]{s.}{fã; entusiasta; aficionado; pessoa que gosta excessivamente de algo}
  \definition{v.}{estar confuso; perder o rumo; se perder-se; perda da capacidade de discernimento e julgamento | ficar fascinado por; entregar-se a; ficar encantado com (por); ser louco por | confundir; desorientar; fascinar; encantar; tornar indistinto; deixar encantado e fascinado}
\end{EntryWithPhonetic}

\begin{EntryWithPhonetic}{迷宫}{mi2gong1}{9,9}{⾡,⼧}
  \definition{s.}{labirinto}
\end{EntryWithPhonetic}

\begin{EntryWithPhonetic}{迷惑}{mi2huo4}{9,12}{⾡,⼼}[HSK 7-9]
  \definition{adj.}{perplexo; confuso; desnorteado; pouco claro; incompreesível}
  \definition{v.}{intrigar; confundir; deixar perplexo; desconcertar}
\end{EntryWithPhonetic}

\begin{EntryWithPhonetic}{迷惑不解}{mi2huo4-bu4jie3}{9,12,4,13}{⾡,⼼,⼀,⾓}[HSK 7-9]
  \definition{expr.}{sentir-se perplexo; estar confuso; ficar intrigado}
\end{EntryWithPhonetic}

\begin{EntryWithPhonetic}{迷恋}{mi2lian4}{9,10}{⾡,⼼}[HSK 7-9]
  \definition{v.}{ser obcecado por; estar apaixonado por; agarrar-se loucamente a; ter um carinho excessivo por algo e achar difícil abrir mão disso}
\end{EntryWithPhonetic}

\begin{EntryWithPhonetic}{迷路}{mi2/lu4}{9,13}{⾡,⾜}[HSK 7-9]
  \definition{v.+compl.}{desviar-se; perder-se; errar o caminho; perder a noção de direção; ir pelo caminho errado; não conseguir encontrar o caminho | perder-se; estar perdido na vida; essa metáfora descreve a perda da direção correta}
\end{EntryWithPhonetic}

\begin{EntryWithPhonetic}{迷你}{mi2ni3}{9,7}{⾡,⼈}
  \definition{adj.}{(empréstimo linguístico) mini, como em minissaia ou \emph{Mini Cooper}}
\end{EntryWithPhonetic}

\begin{EntryWithPhonetic}{迷人}{mi2ren2}{9,2}{⾡,⼈}[HSK 5]
  \definition{adj.}{encantador; fascinante; sedutor; hipnotizante}
  \definition{v.}{confundir; intrigar; enganar}
\end{EntryWithPhonetic}

\begin{EntryWithPhonetic}{迷失}{mi2shi1}{9,5}{⾡,⼤}[HSK 7-9]
  \definition{v.}{perder-se; ser incapaz de distinguir (direções, estradas, etc.)}
\end{EntryWithPhonetic}

\begin{EntryWithPhonetic}{迷信}{mi2xin4}{9,9}{⾡,⼈}[HSK 5]
  \definition{s.}{superstição; crença supersticiosa | fé cega; adoração cega}
  \definition{v.}{ter fé cega em; ter um fetiche de}
\end{EntryWithPhonetic}

%%%%%%%%%% 眯 %%%%%%%%%%
\subsection*{眯}\addcontentsline{loh}{figure}{眯 \dpy{mi2}}

\begin{EntryWithPhonetic}{眯}{mi2}{11}{⽬}
  \definition{v.}{cegar (como com poeira)}
  \seeref{mi1}
\end{EntryWithPhonetic}

%%%%%%%%%% 谜 %%%%%%%%%%
\subsection*{谜}\addcontentsline{loh}{figure}{谜 \dpy{mi2}}

\begin{EntryWithPhonetic}{谜}{mi2}{11}{⾔}[HSK 7-9]
  \definition[个]{s.}{enigma; charada | enigma; mistério; quebra-cabeça}
  \seeref{mei4}
\end{EntryWithPhonetic}

\begin{EntryWithPhonetic}{谜底}{mi2di3}{11,8}{⾔,⼴}[HSK 7-9]
  \definition[个]{s.}{resposta a um enigma | verdade; metáfora para a verdade dos fatos}
\end{EntryWithPhonetic}

\begin{EntryWithPhonetic}{谜团}{mi2tuan2}{11,6}{⾔,⼞}[HSK 7-9]
  \definition{s.}{dúvidas e suspeitas | assuntos elusivos | enigma | situação imprevisível}
\end{EntryWithPhonetic}

\begin{EntryWithPhonetic}{谜语}{mi2yu3}{11,9}{⾔,⾔}[HSK 7-9]
  \definition[条,则]{s.}{enigma; charada; um enigma é uma mensagem críptica que alude a coisas ou palavras, deixando ao leitor a tarefa de adivinhar; consiste principalmente em duas partes: o próprio enigma e a resposta}
\end{EntryWithPhonetic}

%%%%%%%%%% 米 %%%%%%%%%%
\subsection*{米}\addcontentsline{loh}{figure}{米 \dpy{mi3}}

\begin{EntryWithPhonetic}{米}{mi3}{6}{⽶}[HSK 2,3][Kangxi 119]
  \definition*{s.}{Sobrenome: Mi}
  \definition{clas.}{m, metro; unidade principal de comprimento do sistema métrico}
  \definition[粒,斤]{s.}{arroz | sementes descascadas; refere-se a sementes comestíveis descascadas ou sem casca | qualquer coisa que se assemelhe a um grão de arroz}
\end{EntryWithPhonetic}

\begin{EntryWithPhonetic}{米饭}{mi3fan4}{6,7}{⽶,⾷}[HSK 1]
  \definition{s.}{arroz (cozido)}
\end{EntryWithPhonetic}

%%%%%%%%%% 秘 %%%%%%%%%%
\subsection*{秘}\addcontentsline{loh}{figure}{秘 \dpy{mi4}}

\begin{EntryWithPhonetic}{秘}{mi4}{10}{⽲}
  \definition{adj.}{secreto; misterioso | raro; raramente visto; estranho}
  \definition{adv.}{secretamente; privadamente}
  \definition{s.}{secretário}
  \definition{v.}{manter algo em segredo; esconder algo; guardar segredos | bloquear; obstruir; ter dificuldade para defecar}
  \seeref{bi4}
\end{EntryWithPhonetic}

\begin{EntryWithPhonetic}{秘方}{mi4fang1}{10,4}{⽲,⽅}[HSK 7-9]
  \definition{s.}{receita secreta; prescrição secreta; prescrições não divulgadas com efeitos médicos significativos}
\end{EntryWithPhonetic}

\begin{EntryWithPhonetic}{秘诀}{mi4jue2}{10,6}{⽲,⾔}[HSK 7-9]
  \definition[个,条]{s.}{código mágico; segredo (do sucesso); uma boa maneira de resolver o problema sem torná-lo público}
\end{EntryWithPhonetic}

\begin{EntryWithPhonetic}{秘密}{mi4mi4}{10,11}{⽲,⼧}[HSK 4]
  \definition{adj.}{secreto}
  \definition[个,条,些]{s.}{segredo; algo secreto; coisas que você não quer que as pessoas saibam}
\end{EntryWithPhonetic}

\begin{EntryWithPhonetic}{秘书}{mi4shu1}{10,4}{⽲,⼄}[HSK 4]
  \definition[个,位,名]{s.}{o cargo de secretário; funções de secretariado | secretário; pessoas encarregadas da correspondência e que auxiliam o chefe do órgão ou departamento na condução diária de seu trabalho}
\end{EntryWithPhonetic}

\begin{EntryWithPhonetic}{秘书长}{mi4 shu1 zhang3}{10,4,4}{⽲,⼄,⾧}[HSK 6]
  \definition{s.}{secretário-geral}
\end{EntryWithPhonetic}

%%%%%%%%%% 密 %%%%%%%%%%
\subsection*{密}\addcontentsline{loh}{figure}{密 \dpy{mi4}}

\begin{EntryWithPhonetic}{密}{mi4}{11}{⼧}[HSK 4]
  \definition*{s.}{Sobrenome: Mi}
  \definition{adj.}{fechado; denso; espesso | íntimo; próximo; afetuoso | delicado; fino; cuidadoso; meticuloso}
  \definition{adv.}{secretamente}
  \definition{s.}{segredo | densidade | senha; \emph{password}}
\end{EntryWithPhonetic}

\begin{EntryWithPhonetic}{密不可分}{mi4bu4ke3fen1}{11,4,5,4}{⼧,⼀,⼝,⼑}[HSK 7-9]
  \definition{expr.}{inextricavelmente ligados (expressão idiomática) | inseparáveis}
\end{EntryWithPhonetic}

\begin{EntryWithPhonetic}{密度}{mi4du4}{11,9}{⼧,⼴}[HSK 7-9]
  \definition[口]{s.}{densidade; espessura | Física: densidade; a gravidade específica é a razão entre a massa de um objeto e seu volume; anteriormente, era conhecida como densidade específica}
\end{EntryWithPhonetic}

\begin{EntryWithPhonetic}{密封}{mi4feng1}{11,9}{⼧,⼨}[HSK 7-9]
  \definition{v.}{selar; vedar; selar hermeticamente; selar completamente; lacrar rigorosamente}
\end{EntryWithPhonetic}

\begin{EntryWithPhonetic}{密集}{mi4ji2}{11,12}{⼧,⾫}[HSK 7-9]
  \definition{adj.}{denso; intensivo; concentrado}
  \definition{v.}{concentrar-se; aglomerar-se}
\end{EntryWithPhonetic}

\begin{EntryWithPhonetic}{密码}{mi4ma3}{11,8}{⼧,⽯}[HSK 4]
  \definition[个,种]{s.}{código; senha; um código secreto especialmente formulado usado entre as partes acordadas (diferente do 明码)}
  \seealsoref{明码}{ming2ma3}
\end{EntryWithPhonetic}

\begin{EntryWithPhonetic}{密切}{mi4qie4}{11,4}{⼧,⼑}[HSK 4]
  \definition{adj.}{próximo; íntimo; relacionamento próximo}
  \definition{adv.}{cuidadosamente; atentamente; intimamente}
  \definition{v.}{tornar-se próximo; tornar-se íntimo; conectar-se}
\end{EntryWithPhonetic}

%%%%%%%%%% 蜜 %%%%%%%%%%
\subsection*{蜜}\addcontentsline{loh}{figure}{蜜 \dpy{mi4}}

\begin{EntryWithPhonetic}{蜜}{mi4}{14}{⾍}[HSK 7-9]
  \definition{adj.}{melado; doce}
  \definition{s.}{mel | semelhante ao mel | coisas parecidas com mel; melaço}
\end{EntryWithPhonetic}

\begin{EntryWithPhonetic}{蜜蜂}{mi4feng1}{14,13}{⾍,⾍}[HSK 7-9]
  \definition[只,群,箱,窝]{s.}{abelha; abelha-melífera}
\end{EntryWithPhonetic}

\begin{EntryWithPhonetic}{蜜桃}{mi4tao2}{14,10}{⾍,⽊}
  \definition{s.}{nectarina | pêssego | pêssego suculento}
\end{EntryWithPhonetic}

\begin{EntryWithPhonetic}{蜜月}{mi4yue4}{14,4}{⾍,⽉}[HSK 7-9]
  \definition{s.}{lua de mel; o primeiro mês após o casamento}
\end{EntryWithPhonetic}

%%%%%%%%%% 棉 %%%%%%%%%%
\subsection*{棉}\addcontentsline{loh}{figure}{棉 \dpy{mian2}}

\begin{EntryWithPhonetic}{棉}{mian2}{12}{⽊}
  \definition{adj.}{almofadado com algodão; acolchoado}
  \definition[些,种,类]{s.}{termo genérico para algodão ou paina | algodão | material semelhante ao algodão | acolchoado ou estofado de algodão}
\end{EntryWithPhonetic}

\begin{EntryWithPhonetic}{棉花}{mian2hua1}{12,7}{⽊,⾋}[HSK 7-9]
  \definition[团,斤,种]{s.}{algodão; as fibras dos caroços de algodão são utilizadas para fiar fios, enchimento de roupas e roupas de cama, etc. | algodão; o nome comum para o capim-algodão}
\end{EntryWithPhonetic}

%%%%%%%%%% 免 %%%%%%%%%%
\subsection*{免}\addcontentsline{loh}{figure}{免 \dpy{mian3}}

\begin{EntryWithPhonetic}{免}{mian3}{7}{⼉}[HSK 7-9]
  \definition*{s.}{Sobrenome: Mian}
  \definition{v.}{desculpar alguém de algo; isentar; dispensar; renunciar | remover do cargo; demitir | evitar; desviar; escapar | não deveria ser permitido; não precisar fazer algo | remover; livrar-se de | isentar; dispensar | não permitir}
\end{EntryWithPhonetic}

\begin{EntryWithPhonetic}{免不了}{mian3bu5liao3}{7,4,2}{⼉,⼀,⼅}[HSK 7-9]
  \definition{v.}{ser inevitável; ser fadado a ser}
\end{EntryWithPhonetic}

\begin{EntryWithPhonetic}{免除}{mian3chu2}{7,9}{⼉,⾩}[HSK 7-9]
  \definition{v.}{prevenir; evitar; afugentar | dispensar; desculpar; isentar; aliviar | imunizar; isentar}
\end{EntryWithPhonetic}

\begin{EntryWithPhonetic}{免得}{mian3de5}{7,11}{⼉,⼻}[HSK 6]
  \definition{conj.}{de modo a não; para evitar; para que não; indica evitar uma situação que não é desejável e é frequentemente usado no início da oração seguinte}
\end{EntryWithPhonetic}

\begin{EntryWithPhonetic}{免费}{mian3/fei4}{7,9}{⼉,⾙}[HSK 4]
  \definition{v.+compl.}{isentar de taxas; tonar grátis}
\end{EntryWithPhonetic}

\begin{EntryWithPhonetic}{免税}{mian3/shui4}{7,12}{⼉,⽲}
  \definition{adj.}{isento de impostos (tributação)}
  \definition{s.}{livre de impostos | isenção de impostos}
  \definition{v.+compl.}{isentar impostos}
\end{EntryWithPhonetic}

\begin{EntryWithPhonetic}{免疫}{mian3yi4}{7,9}{⼉,⽧}[HSK 7-9]
  \definition{adj.}{imune; imunológico}
  \definition{s.}{imunidade; por possuírem resistência a certas doenças infecciosas, são imunes a elas; existem dois tipos: imunidade inata e imunidade adquirida}
\end{EntryWithPhonetic}

\begin{EntryWithPhonetic}{免职}{mian3/zhi2}{7,11}{⼉,⽿}[HSK 7-9]
  \definition{v.+compl.}{remover alguém do cargo | destituir alguém do seu cargo | demitir alguém do seu cargo}
\end{EntryWithPhonetic}

%%%%%%%%%% 勉 %%%%%%%%%%
\subsection*{勉}\addcontentsline{loh}{figure}{勉 \dpy{mian3}}

\begin{EntryWithPhonetic}{勉}{mian3}{9}{⼒}
  \definition{s.}{Sobrenome: Mian}
  \definition{v.}{esforçar-se; esforçar-se para; lutar | encorajar; instar; exortar | forçar-se a fazer algo; esforçar-se para fazer o que está além de suas capacidades; fazer algo contra a vontade; esforçar-se para trabalhar arduamente; empenhar-se em realizar trabalhos árduos; dar o seu melhor, mesmo que não tenha forças suficientes}
\end{EntryWithPhonetic}

\begin{EntryWithPhonetic}{勉强}{mian3qiang3}{9,12}{⼒,⼸}[HSK 7-9]
  \definition{adj.}{relutante; indisposto; a contragosto; não muito disposto | inadequado; pouco convincente; forçado; inverossímil | mal o suficiente; mal dá para\dots}[这些钱勉强够用一周的。===Esse dinheiro mal dá para uma semana.]
  \definition{v.}{empurrar; forçar alguém a fazer algo; obrigar alguém a fazer algo que não quer fazer}
\end{EntryWithPhonetic}

%%%%%%%%%% 缅 %%%%%%%%%%
\subsection*{缅}\addcontentsline{loh}{figure}{缅 \dpy{mian3}}

\begin{EntryWithPhonetic}{缅}{mian3}{12}{⽷}
  \definition*{s.}{Mianmar (antiga Birmânia), abreviação de 缅甸}
  \definition{adj.}{remoto; muito distante | detalhado}
  \definition{s.}{Literário: filamento fino}
  \definition{v.}{Dialeto: enrolar; virar | ponderar; refletir}
  \seealsoref{缅甸}{mian3dian4}
\end{EntryWithPhonetic}

\begin{EntryWithPhonetic}{缅甸}{mian3dian4}{12,7}{⽷,⽥}
  \definition*{s.}{Birmânia; Mianmar}
\end{EntryWithPhonetic}

\begin{EntryWithPhonetic}{缅怀}{mian3huai2}{12,7}{⽷,⼼}[HSK 7-9]
  \definition{v.}{recordar (atos passados); guardar com carinho a memória de}
\end{EntryWithPhonetic}

%%%%%%%%%% 靣 %%%%%%%%%%
\subsection*{靣}\addcontentsline{loh}{figure}{靣 \dpy{mian4}}

\begin{EntryWithPhonetic}{靣}{mian4}{8}{⼀}[Kangxi 176]
  \variantof{面}
\end{EntryWithPhonetic}

%%%%%%%%%% 面 %%%%%%%%%%
\subsection*{面}\addcontentsline{loh}{figure}{面 \dpy{mian4}}

\begin{EntryWithPhonetic}{面}{mian4}{9}{⾯}[HSK 2][Kangxi 176]
  \definition*{s.}{Sobrenome: Mian}
  \definition{adj.}{macio e farinhento; descreve algo que é muito macio ao comer | superficial}
  \definition{adv.}{diretamente; pessoalmente; na frente de alguém; cara a cara}
  \definition{clas.}{usado para objetos planos | usado para indicar o número de vezes que as pessoas se encontram}
  \definition[斤,两,碗]{s.}{face; parte frontal da cabeça; rosto | topo; superfície | capa; exterior; a parte externa de um objeto ou a face frontal de um tecido (em oposição à 里) | (matemática) superfície | cara; sentimento; emoção | geral; área total; abrangente; toda a região | lado; aspecto | escopo; escala; extensão; alcance; âmbito | farinha; farinha de trigo | pó; algo em pó | macarrão; \emph{noodle}}
  \definition{suf.}{sufixo para localização ou direção; anexado ao final de palavras que indicam localização, equivalente a 边}
  \definition{v.}{encarar algo | encontrar; revelar-se}
  \seealsoref{边}{bian1}
  \seealsoref{里}{li3}
\end{EntryWithPhonetic}

\begin{EntryWithPhonetic}{面包}{mian4bao1}{9,5}{⾯,⼓}[HSK 1]
  \definition[个,片,袋,块]{s.}{pão}[我买八个面包了。===Comprei oito pães. | 他在吃两片面包。===Ele está comendo duas fatias de pão. | 我在家里带了一袋面包。===Trouxe um saco de pão para casa. | 我拿了一块面包。===Peguei um pedaço de pão.]
\end{EntryWithPhonetic}

\begin{EntryWithPhonetic}{面部}{mian4bu4}{9,10}{⾯,⾢}[HSK 7-9]
  \definition{s.}{rosto; face}
\end{EntryWithPhonetic}

\begin{EntryWithPhonetic}{面对}{mian4dui4}{9,5}{⾯,⼨}[HSK 3]
  \definition{v.}{enfrentar; defrontar; olhar para (uma pessoa ou um objeto específico) | confrontar (problema); problemas, dificuldades e outras questões que precisam ser resolvidas e que merecem atenção}
\end{EntryWithPhonetic}

\begin{EntryWithPhonetic}{面对面}{mian4 dui4 mian4}{9,5,9}{⾯,⼨,⾯}[HSK 6]
  \definition{adj./expr.}{frente a frente; cara a cara; vis"-à"-vis}
\end{EntryWithPhonetic}

\begin{EntryWithPhonetic}{面对面吃面}{mian4dui4mian4 chi1 mian4}{9,5,9,6,9}{⾯,⼨,⾯,⼝,⾯}
  \definition{expr.}{Comer macarrão cara a cara; indica que o seu estado atual, ou algumas das posições em que você está, ou algumas das coisas que você fez são muito claras}
\end{EntryWithPhonetic}

\begin{EntryWithPhonetic}{面粉}{mian4fen3}{9,10}{⾯,⽶}[HSK 7-9]
  \definition[份]{s.}{farinha; farinha de trigo}
\end{EntryWithPhonetic}

\begin{EntryWithPhonetic}{面红耳赤}{mian4hong2-er3chi4}{9,6,6,7}{⾯,⽷,⽿,⾚}[HSK 7-9]
  \definition{expr.}{ficar corado; ficar ruborizado de raiva; descreve um rosto corado devido à impaciência ou timidez}
\end{EntryWithPhonetic}

\begin{EntryWithPhonetic}{面积}{mian4ji1}{9,10}{⾯,⽲}[HSK 3]
  \definition{s.}{área (de um andar, pedaço de terreno, etc.); área de uma superfície; o tamanho de uma superfície plana ou da superfície de um objeto}
\end{EntryWithPhonetic}

\begin{EntryWithPhonetic}{面临}{mian4lin2}{9,9}{⾯,⼁}[HSK 4]
  \definition{v.}{ser confrontado com; encontrar (uma situação) na frente de}
\end{EntryWithPhonetic}

\begin{EntryWithPhonetic}{面貌}{mian4mao4}{9,14}{⾯,⾘}[HSK 5]
  \definition[种,个]{s.}{rosto; traços faciais; formato do rosto; aparência | aparência; aspecto; aparência (das coisas)}
\end{EntryWithPhonetic}

\begin{EntryWithPhonetic}{面面俱到}{mian4mian4-ju4dao4}{9,9,10,8}{⾯,⾯,⼈,⼑}[HSK 7-9]
  \definition{expr.}{cuidar de tudo; resolver tudo; contemplar todos os aspectos e não deixa nada de fora; dar atenção a todos os aspectos de uma questão}
\end{EntryWithPhonetic}

\begin{EntryWithPhonetic}{面目全非}{mian4mu4-quan2fei1}{9,5,6,8}{⾯,⽬,⼊,⾮}[HSK 7-9]
  \definition{expr.}{perder a própria identidade; uma mudança completa; tudo parece errado ou diferente; ser alterado (distorcido) a ponto de ficar irreconhecível; não ser mais como era antes; ser muito diferente do original; os originais não existem mais; a aparência das coisas mudou drasticamente (frequentemente com uma conotação negativa); mudança além do reconhecimento (frequentemente com uma conotação pejorativa)}
\end{EntryWithPhonetic}

\begin{EntryWithPhonetic}{面前}{mian4 qian2}{9,9}{⾯,⼑}[HSK 2]
  \definition{s.}{antes; na frente de; diante de}
\end{EntryWithPhonetic}

\begin{EntryWithPhonetic}{面试}{mian4 shi4}{9,8}{⾯,⾔}[HSK 4]
  \definition{v.}{entrevistar (é realizado na forma de perguntas e respostas orais presenciais)}
\end{EntryWithPhonetic}

\begin{EntryWithPhonetic}{面条}{mian4tiao2}{9,7}{⾯,⽊}
  \definition{s.}{macarrão | espaguete}
\end{EntryWithPhonetic}

\begin{EntryWithPhonetic}{面条儿}{mian4 tiao2r5}{9,7,2}{⾯,⽊,⼉}[HSK 1]
  \definition{s.}{macarrão; \emph{noodles}}
\end{EntryWithPhonetic}

\begin{EntryWithPhonetic}{面团}{mian4tuan2}{9,6}{⾯,⼞}
  \definition{s.}{massa | pasta}
\end{EntryWithPhonetic}

\begin{EntryWithPhonetic}{面向}{mian4 xiang4}{9,6}{⾯,⼝}[HSK 6]
  \definition{v.}{virar o rosto para; virar na direção de; defrontar; voltado para algum lugar | estar orientado para as necessidades de; atender a; principalmente para um certo tipo de pessoas}
\end{EntryWithPhonetic}

\begin{EntryWithPhonetic}{面子}{mian4zi5}{9,3}{⾯,⼦}[HSK 5]
  \definition{s.}{face; exterior; parte externa; superfície do objeto | imagem; reputação; prestígio; decência; vaidade superficial | sentimentos; sensibilidades | pó}
\end{EntryWithPhonetic}

%%%%%%%%%% 糆 %%%%%%%%%%
\subsection*{糆}\addcontentsline{loh}{figure}{糆 \dpy{mian4}}

\begin{EntryWithPhonetic}{糆}{mian4}{15}{⽶}
  \variantof{面}
\end{EntryWithPhonetic}

%%%%%%%%%% 麫 %%%%%%%%%%
\subsection*{麫}\addcontentsline{loh}{figure}{麫 \dpy{mian4}}

\begin{EntryWithPhonetic}{麫}{mian4}{15}{⿆}
  \variantof{面}
\end{EntryWithPhonetic}

%%%%%%%%%% 苗 %%%%%%%%%%
\subsection*{苗}\addcontentsline{loh}{figure}{苗 \dpy{miao2}}

\begin{EntryWithPhonetic}{苗}{miao2}{8}{⾋}[HSK 7-9]
  \definition*{s.}{Miao, grupo étnico, abreviação de 苗族 | Sobrenome: Miao}
  \definition[棵,株,尾,头,些]{s.}{planta jovem; muda; broto; plantas recém-germinadas | descendente; filho | filhotes de alguns animais; animais domésticos recém-nascidos | vacina | algo que se assemelha a uma planta jovem}
  \seealsoref{苗族}{miao2zu2}
\end{EntryWithPhonetic}

\begin{EntryWithPhonetic}{苗条}{miao2tiao5}{8,7}{⾋,⽊}[HSK 7-9]
  \definition{adj.}{(figura feminina) magra; esbelta; esbelta e graciosa}
\end{EntryWithPhonetic}

\begin{EntryWithPhonetic}{苗头}{miao2tou5}{8,5}{⾋,⼤}[HSK 7-9]
  \definition{s.}{sintoma de uma tendência; indício de um novo desenvolvimento; uma tendência ou situação de desenvolvimento ligeiramente emergente}
\end{EntryWithPhonetic}

\begin{EntryWithPhonetic}{苗族}{miao2zu2}{8,11}{⾋,⽅}
  \definition*{s.}{Grupo étnico Hmong ou Miao do sudoeste da China; uma das minorias étnicas da China, distribuída por Guizhou 贵州, Hunan 湖南, Yunnan 云南, Guangxi 广西, Sichuan 四川, Guangdong 广东 e Hubei 湖北}
  \seealsoref{广东}{guang3dong1}
  \seealsoref{广西}{guang3xi1}
  \seealsoref{贵州}{gui4zhou1}
  \seealsoref{湖北}{hu2bei3}
  \seealsoref{湖南}{hu2nan2}
  \seealsoref{四川}{si4chuan1}
  \seealsoref{云南}{yun2nan2}
\end{EntryWithPhonetic}

%%%%%%%%%% 描 %%%%%%%%%%
\subsection*{描}\addcontentsline{loh}{figure}{描 \dpy{miao2}}

\begin{EntryWithPhonetic}{描}{miao2}{11}{⼿}
  \definition{v.}{traçar; copiar | retocar; retocar | traçar um desenho | retratar | esboçar}
\end{EntryWithPhonetic}

\begin{EntryWithPhonetic}{描绘}{miao2hui4}{11,9}{⼿,⽷}[HSK 7-9]
  \definition{v.}{descrever; retratar; representar; desenhar}
\end{EntryWithPhonetic}

\begin{EntryWithPhonetic}{描述}{miao2 shu4}{11,8}{⼿,⾡}[HSK 4]
  \definition[段,种]{s.}{descrição; trecho que descreve um evento ou uma cena}
  \definition{v.}{descrever; representar}
\end{EntryWithPhonetic}

\begin{EntryWithPhonetic}{描写}{miao2xie3}{11,5}{⼿,⼍}[HSK 4]
  \definition{v.}{representar; retratar; descrever; usar a linguagem e as palavras para transmitir uma imagem concreta de uma pessoa, evento ou situação}
\end{EntryWithPhonetic}

%%%%%%%%%% 瞄 %%%%%%%%%%
\subsection*{瞄}\addcontentsline{loh}{figure}{瞄 \dpy{miao2}}

\begin{EntryWithPhonetic}{瞄}{miao2}{13}{⽬}
  \definition{v.}{concentrar o olhar em; mirar | olhar fixamente para; mirar em; prestar atenção em | olhar; dar uma olhada rápida}
\end{EntryWithPhonetic}

\begin{EntryWithPhonetic}{瞄准}{miao2/zhun3}{13,10}{⽬,⼎}[HSK 7-9]
  \definition{v.+compl.}{mirar; apontar para}
\end{EntryWithPhonetic}

%%%%%%%%%% 秒 %%%%%%%%%%
\subsection*{秒}\addcontentsline{loh}{figure}{秒 \dpy{miao3}}

\begin{EntryWithPhonetic}{秒}{miao3}{9}{⽲}[HSK 5]
  \definition{adv.}{instantaneamente}
  \definition{s.}{segundo (unidade de tempo) | segundo (unidade de medida angular)}
\end{EntryWithPhonetic}

%%%%%%%%%% 渺 %%%%%%%%%%
\subsection*{渺}\addcontentsline{loh}{figure}{渺 \dpy{miao3}}

\begin{EntryWithPhonetic}{渺}{miao3}{12}{⽔}
  \definition{adj.}{(uma vasta extensão de água) vasta; que se estende ao longe | distante e indistinto; vago | minúsculo; insignificante}
\end{EntryWithPhonetic}

\begin{EntryWithPhonetic}{渺小}{miao3xiao3}{12,3}{⽔,⼩}[HSK 7-9]
  \definition{adj.}{minúsculo; reles; desprezível; insignificante}
\end{EntryWithPhonetic}

%%%%%%%%%% 妙 %%%%%%%%%%
\subsection*{妙}\addcontentsline{loh}{figure}{妙 \dpy{miao4}}

\begin{EntryWithPhonetic}{妙}{miao4}{7}{⼥}[HSK 6]
  \definition*{s.}{Sobrenome: Miao}
  \definition{adj.}{maravilhoso; excelente; bom | engenhoso; esperto; sutil | extraordinário | requintado; mágico; engenhoso; misterioso}
\end{EntryWithPhonetic}

\begin{EntryWithPhonetic}{妙招}{miao4zhao1}{7,8}{⼥,⼿}
  \definition{adj.}{escorregadio}
  \definition{s.}{movimento inteligente | maneira inteligente de fazer algo}
\end{EntryWithPhonetic}

%%%%%%%%%% 庙 %%%%%%%%%%
\subsection*{庙}\addcontentsline{loh}{figure}{庙 \dpy{miao4}}

\begin{EntryWithPhonetic}{庙}{miao4}{8}{⼴}[HSK 7-9]
  \definition[座,个,间]{s.}{templo; santuário | feira do templo | Literário: corte imperial; corte real | Literário: imperador falecido | casa de incenso; locais onde, no passado, foram consagradas tábuas ancestrais, divindades ou figuras históricas}
\end{EntryWithPhonetic}

\begin{EntryWithPhonetic}{庙会}{miao4hui4}{8,6}{⼴,⼈}[HSK 7-9]
  \definition{s.}{feira; feira do templo; festival feira do templo; mercados montados em templos ou próximos a eles; geralmente realizados em festivais ou dias específicos}
\end{EntryWithPhonetic}

%%%%%%%%%% 灭 %%%%%%%%%%
\subsection*{灭}\addcontentsline{loh}{figure}{灭 \dpy{mie4}}

\begin{EntryWithPhonetic}{灭}{mie4}{5}{⽕}[HSK 6]
  \definition{v.}{extinguir-se | extinguir; apagar; desligar | afogar; inundar; submergir | perecer; destruir | exterminar; apagar; acabar com; tornar inexistente}
\end{EntryWithPhonetic}

\begin{EntryWithPhonetic}{灭火}{mie4huo3}{5,4}{⽕,⽕}
  \definition{s.}{combate a incêndios}
  \definition{v.}{extinguir um incêndio}
\end{EntryWithPhonetic}

\begin{EntryWithPhonetic}{灭绝}{mie4jue2}{5,9}{⽕,⽷}[HSK 7-9]
  \definition{v.}{extinguir-se; ser extinto; eliminar completamente | perder completamente; perder totalmente}
\end{EntryWithPhonetic}

\begin{EntryWithPhonetic}{灭亡}{mie4wang2}{5,3}{⽕,⼇}[HSK 7-9]
  \definition{v.}{ser destruído; extinguir-se; perecer; desaparecer; isso se refere à eliminação de uma nação, grupo étnico ou grupo político, que deixa de existir | destruir; exterminar}
\end{EntryWithPhonetic}

%%%%%%%%%% 民 %%%%%%%%%%
\subsection*{民}\addcontentsline{loh}{figure}{民 \dpy{min2}}

\begin{EntryWithPhonetic}{民}{min2}{5}{⽒}
  \definition*{s.}{Sobrenome: Min}
  \definition{adj.}{folclórico ; civil (não militar)}
  \definition{s.}{pessoa | membro de um grupo étnico | uma pessoa de uma determinada ocupação | do povo; folclore | civil; cidadão | o povo | um membro de uma nacionalidade}
\end{EntryWithPhonetic}

\begin{EntryWithPhonetic}{民办}{min2ban4}{5,4}{⽒,⼒}[HSK 7-9]
  \definition{adj.}{administrado pela população local; administrado por civis (escola) | privado; administrado de forma privada (oposto de 公办)}
  \seealsoref{公办}{gong1ban4}
\end{EntryWithPhonetic}

\begin{EntryWithPhonetic}{民歌}{min2 ge1}{5,14}{⽒,⽋}[HSK 6]
  \definition[支,首]{s.}{canção folclórica; os nomes dos autores das canções transmitidas oralmente são muitas vezes desconhecidos}
\end{EntryWithPhonetic}

\begin{EntryWithPhonetic}{民工}{min2 gong1}{5,3}{⽒,⼯}[HSK 6]
  \definition{s.}{trabalhador trabalhando em um projeto público | trabalhador temporário alistado em um projeto público | agricultor que trabalha em empregos temporários na cidade | trabalhador migrante}
\end{EntryWithPhonetic}

\begin{EntryWithPhonetic}{民间}{min2jian1}{5,7}{⽒,⾨}[HSK 3]
  \definition{s.}{entre o povo | não governamental; de pessoa para pessoa}
\end{EntryWithPhonetic}

\begin{EntryWithPhonetic}{民警}{min2 jing3}{5,19}{⽒,⾔}[HSK 6]
  \definition{s.}{polícia; policial}
\end{EntryWithPhonetic}

\begin{EntryWithPhonetic}{民俗}{min2su2}{5,9}{⽒,⼈}[HSK 7-9]
  \definition{s.}{costumes populares; tradição popular}
\end{EntryWithPhonetic}

\begin{EntryWithPhonetic}{民意}{min2 yi4}{5,13}{⽒,⼼}[HSK 6]
  \definition{s.}{vontade do povo; vontade popular | opinião pública}
\end{EntryWithPhonetic}

\begin{EntryWithPhonetic}{民用}{min2yong4}{5,5}{⽒,⽤}[HSK 7-9]
  \definition{adj.}{para uso civil; civil | usado no dia a dia das pessoas}
\end{EntryWithPhonetic}

\begin{EntryWithPhonetic}{民众}{min2zhong4}{5,6}{⽒,⼈}[HSK 7-9]
  \definition{s.}{a população; o povo comum; as massas populares}
\end{EntryWithPhonetic}

\begin{EntryWithPhonetic}{民主}{min2zhu3}{5,5}{⽒,⼂}[HSK 6]
  \definition{adj.}{democrático; em consonância com os princípios democráticos}
  \definition[个]{s.}{democracia; direitos democráticos; refere-se ao direito do povo de participar da vida política e dos assuntos do Estado e de expressar livremente suas opiniões}
\end{EntryWithPhonetic}

\begin{EntryWithPhonetic}{民族}{min2zu2}{5,11}{⽒,⽅}[HSK 3]
  \definition[个]{s.}{nação; uma comunidade estável formada ao longo da história pela humanidade, com uma língua comum, uma região comum, uma vida econômica comum e uma mentalidade comum expressa em uma cultura comum | grupo étnico; refere-se, de maneira geral, às comunidades formadas ao longo da história por pessoas em diferentes estágios de desenvolvimento social}
\end{EntryWithPhonetic}

%%%%%%%%%% 敏 %%%%%%%%%%
\subsection*{敏}\addcontentsline{loh}{figure}{敏 \dpy{min3}}

\begin{EntryWithPhonetic}{敏}{min3}{11}{⽁}
  \definition*{s.}{Sobrenome: Min}
  \definition{adj.}{rápido; ágil | perspicaz; inteligente; rápido | inteligente; esperto}
\end{EntryWithPhonetic}

\begin{EntryWithPhonetic}{敏感}{min3gan3}{11,13}{⽁,⼼}[HSK 5]
  \definition{adj.}{sensível; descreve pessoas ou animais que rapidamente percebem mudanças ou estímulos externos | reativo; sensível; fácil de causar reações intensas}
\end{EntryWithPhonetic}

\begin{EntryWithPhonetic}{敏捷}{min3jie2}{11,11}{⽁,⼿}[HSK 7-9]
  \definition{adj.}{ágil; rápido; descreve reações rápidas em ações, pensamentos, etc.}
\end{EntryWithPhonetic}

\begin{EntryWithPhonetic}{敏锐}{min3rui4}{11,12}{⽁,⾦}[HSK 7-9]
  \definition{adj.}{agudo; perspicaz; aguçado; (pensamento) rápido de raciocínio, (intuição) aguçado}
\end{EntryWithPhonetic}

%%%%%%%%%% 名 %%%%%%%%%%
\subsection*{名}\addcontentsline{loh}{figure}{名 \dpy{ming2}}

\begin{EntryWithPhonetic}{名}{ming2}{6}{⼝}[HSK 2]
  \definition*{s.}{Sobrenome: Ming}
  \definition{adj.}{notável; famoso; conhecido; renomado}
  \definition{clas.}{usado para pessoas | usado para classificação por ordem}
  \definition{s.}{nome; denominação | desculpa; pretexto | fama; reputação}
  \definition{v.}{nome próprio (é) | expressar; descrever | possuir; tomar; ter}
\end{EntryWithPhonetic}

\begin{EntryWithPhonetic}{名称}{ming2 cheng1}{6,10}{⼝,⽲}[HSK 2]
  \definition[个,种]{s.}{nomes, apelidos e formas de se referir a pessoas ou coisas}
\end{EntryWithPhonetic}

\begin{EntryWithPhonetic}{名单}{ming2 dan1}{6,8}{⼝,⼗}[HSK 2]
  \definition[个,份]{s.}{lista com nomes de pessoas ou nomes de organizações}
\end{EntryWithPhonetic}

\begin{EntryWithPhonetic}{名额}{ming2'e2}{6,15}{⼝,⾴}[HSK 6]
  \definition[个]{s.}{cota de pessoas; número de pessoas designadas ou permitidas; número necessário de pessoal}
\end{EntryWithPhonetic}

\begin{EntryWithPhonetic}{名副其实}{ming2fu4qi2shi2}{6,11,8,8}{⼝,⼑,⼋,⼧}[HSK 7-9]
  \definition{expr.}{``Faz jus ao seu nome.''; ser digno do nome; ser algo na realidade, bem como no nome; ser digno da própria reputação; em nome e de fato; no verdadeiro sentido do termo; a reputação de alguém é justificada; o nome corresponde à realidade; merecer verdadeiramente o seu nome; fiel ao próprio nome}
\end{EntryWithPhonetic}

\begin{EntryWithPhonetic}{名贵}{ming2gui4}{6,9}{⼝,⾙}[HSK 7-9]
  \definition{adj.}{famoso e precioso; raro}
\end{EntryWithPhonetic}

\begin{EntryWithPhonetic}{名利}{ming2li4}{6,7}{⼝,⼑}[HSK 7-9]
  \definition{s.}{fama e ganho; fama e riqueza | fama e dinheiro; refere-se ao status e aos interesses de um indivíduo}
\end{EntryWithPhonetic}

\begin{EntryWithPhonetic}{名牌儿}{ming2 pai2r5}{6,12,2}{⼝,⽚,⼉}[HSK 4]
  \definition{s.}{marca famosa}
\end{EntryWithPhonetic}

\begin{EntryWithPhonetic}{名片}{ming2pian4}{6,4}{⼝,⽚}[HSK 4]
  \definition[张,盒,叠]{s.}{cartão de visita; um pedaço de papel retangular com o nome, o cargo, o endereço etc. impressos}
\end{EntryWithPhonetic}

\begin{EntryWithPhonetic}{名气}{ming2qi5}{6,4}{⼝,⽓}[HSK 7-9]
  \definition{s.}{nome; fama; reputação}
\end{EntryWithPhonetic}

\begin{EntryWithPhonetic}{名人}{ming2 ren2}{6,2}{⼝,⼈}[HSK 4]
  \definition[位,个]{s.}{celebridade; pessoa famosa}
\end{EntryWithPhonetic}

\begin{EntryWithPhonetic}{名声}{ming2sheng1}{6,7}{⼝,⼠}[HSK 7-9]
  \definition{s.}{reputação; renome; prestígio; comentários amplamente divulgados na sociedade}
\end{EntryWithPhonetic}

\begin{EntryWithPhonetic}{名胜}{ming2 sheng4}{6,9}{⼝,⾁}[HSK 6]
  \definition[处,个]{s.}{pontos turísticos; atrações famosas; lugares famosos com locais históricos ou belas paisagens}
\end{EntryWithPhonetic}

\begin{EntryWithPhonetic}{名言}{ming2yan2}{6,7}{⼝,⾔}[HSK 7-9]
  \definition{s.}{ditados; comentário famoso; frases ou expressões famosas}
\end{EntryWithPhonetic}

\begin{EntryWithPhonetic}{名义}{ming2 yi4}{6,3}{⼝,⼂}[HSK 6]
  \definition{s.}{nominal; em nome (geralmente seguido por 上); um nome ou título usado como base para fazer algo}[有人盗用我名义申请贷款。===Alguém solicitou um empréstimo em meu nome. | 他们只是名义上的夫妻。===Eles são marido e mulher apenas no nome.]
  \seealsoref{上}{shang4}
\end{EntryWithPhonetic}

\begin{EntryWithPhonetic}{名誉}{ming2yu4}{6,13}{⼝,⾔}[HSK 6]
  \definition{adj.}{honorário; nominal (geralmente se refere ao nome de um presente, com um sentido de respeito)}[他是学校的名誉教授。===Ele é professor honorário da escola.]
  \definition{s.}{fama; reputação; honra}[名誉才是最神圣的。===Reputação é a coisa mais sagrada. | 我用自己的名誉发誓。===Juro pela minha honra.]
\end{EntryWithPhonetic}

\begin{EntryWithPhonetic}{名著}{ming2zhu4}{6,11}{⼝,⽬}[HSK 7-9]
  \definition{s.}{clássico; livro famoso; obra famosa; obra-prima; obras valiosas e famosas}
\end{EntryWithPhonetic}

\begin{EntryWithPhonetic}{名字}{ming2zi5}{6,6}{⼝,⼦}[HSK 1]
  \definition[个]{s.}{nome; nome próprio | nome (de uma coisa)}
\end{EntryWithPhonetic}

%%%%%%%%%% 明 %%%%%%%%%%
\subsection*{明}\addcontentsline{loh}{figure}{明 \dpy{ming2}}

\begin{EntryWithPhonetic}{明}{ming2}{8}{⽇}
  \definition*{s.}{Dinastia Ming (1368-1644) | Sobrenome: Ming}
  \definition{adj.}{claro; brilhante; brilhante | claro; distinto; de fácil entendimento | aberto; evidente; explícito; exposto | de ​​olhos aguçados; boa visão; visão nítida | honesto}
  \definition{adv.}{claramente; definitivamente; aparentemente; de fato}
  \definition{s.}{imediatamente a seguir no tempo; ao lado deste ano e hoje; visão}
  \definition{v.}{mostrar; revelar; tornar conhecido; deixar claro | entender; compreender}
\end{EntryWithPhonetic}

\begin{EntryWithPhonetic}{明白}{ming2bai5}{8,5}{⽇,⽩}[HSK 1]
  \definition{adj.}{claro; óbvio; evidente; inequívoco | sensato; razoável | aberto; franco; inequívoco; explícito}
  \definition{v.}{entender; compreender; saber}
\end{EntryWithPhonetic}

\begin{EntryWithPhonetic}{明朗}{ming2lang3}{8,10}{⽇,⽉}[HSK 7-9]
  \definition{adj.}{brilhante e claro; bem iluminado (geralmente se referindo a ambientes externos) | claro; óbvio; inequívoco | franco; direto; alegre e otimista; íntegro e honesto; (pensamentos, mentalidade, caráter, etc.) otimista, alegre e não melancólico ou deprimido}
\end{EntryWithPhonetic}

\begin{EntryWithPhonetic}{明亮}{ming2 liang4}{8,9}{⽇,⼇}[HSK 5]
  \definition{adj.}{claro; bem iluminado | brilhante; resplandecente | claro; simples; compreensível}
\end{EntryWithPhonetic}

\begin{EntryWithPhonetic}{明码}{ming2ma3}{8,8}{⽇,⽯}
  \definition{s.}{código simples, em claro (oposto a 密码) | preço claramente marcado}
  \seealsoref{密码}{mi4ma3}
\end{EntryWithPhonetic}

\begin{EntryWithPhonetic}{明媚}{ming2mei4}{8,12}{⽇,⼥}[HSK 7-9]
  \definition{adj.}{brilhante e belo; radiante e encantador}
\end{EntryWithPhonetic}

\begin{EntryWithPhonetic}{明明}{ming2ming2}{8,8}{⽇,⽇}[HSK 5]
  \definition{adv.}{obviamente; claramente; sem dúvida; indica que o fenômeno ou princípio é evidente}
\end{EntryWithPhonetic}

\begin{EntryWithPhonetic}{明年}{ming2 nian2}{8,6}{⽇,⼲}[HSK 1]
  \definition{s.}{próximo ano}
\end{EntryWithPhonetic}

\begin{EntryWithPhonetic}{明确}{ming2que4}{8,12}{⽇,⽯}[HSK 3]
  \definition{adj.}{claro; definido; específico}
  \definition{v.}{deixar claro; tornar definitivo; tornar um ponto de vista, uma tarefa, etc. claro, compreensível e definitivo}
\end{EntryWithPhonetic}

\begin{EntryWithPhonetic}{明日}{ming2 ri4}{8,4}{⽇,⽇}[HSK 6]
  \definition{s.}{amanhã}
  \seealsoref{明天}{ming2tian1}
\end{EntryWithPhonetic}

\begin{EntryWithPhonetic}{明天}{ming2tian1}{8,4}{⽇,⼤}[HSK 1]
  \definition{s.}{amanhã | futuro próximo}
\end{EntryWithPhonetic}

\begin{EntryWithPhonetic}{明显}{ming2xian3}{8,9}{⽇,⽇}[HSK 3]
  \definition{adj.}{claro; óbvio; distinto; claramente visível}
\end{EntryWithPhonetic}

\begin{EntryWithPhonetic}{明星}{ming2xing1}{8,9}{⽇,⽇}[HSK 2]
  \definition[个,位,颗,名]{s.}{estrela; ator, atleta, cantor famosos, etc. | talento de ponta; profissional de destaque; também é usado como metáfora para pessoas ou grupos que se destacam pelo seu bom desempenho ou excelência | estrela brilhante; estrela resplandecente; referindo-se a estrelas muito brilhantes}
\end{EntryWithPhonetic}

\begin{EntryWithPhonetic}{明智}{ming2zhi4}{8,12}{⽇,⽇}[HSK 7-9]
  \definition{adj.}{sábio; sensato; sagaz; previdente; ponderado}
\end{EntryWithPhonetic}

\begin{EntryWithPhonetic}{明珠}{ming2zhu1}{8,10}{⽇,⽟}
  \definition{s.}{pérola | jóia (de grande valor)}
\end{EntryWithPhonetic}

%%%%%%%%%% 鸣 %%%%%%%%%%
\subsection*{鸣}\addcontentsline{loh}{figure}{鸣 \dpy{ming2}}

\begin{EntryWithPhonetic}{鸣}{ming2}{8}{⿃}
  \definition{v.}{chorar (pássaros, animais e insetos) | fazer um som | dar voz (gratidão, queixas, etc.)}
\end{EntryWithPhonetic}

%%%%%%%%%% 铭 %%%%%%%%%%
\subsection*{铭}\addcontentsline{loh}{figure}{铭 \dpy{ming2}}

\begin{EntryWithPhonetic}{铭}{ming2}{11}{⾦}
  \definition*{s.}{Sobrenome: Ming}
  \definition{s.}{inscrição; textos antigos fundidos ou gravados em objetos e estelas para registrar fatos, realizações ou para servir de advertência}
  \definition{v.}{gravar; lembrar; inscrever; inscrever textos comemorativos em objetos; uma metáfora para recordar profundamente}
\end{EntryWithPhonetic}

\begin{EntryWithPhonetic}{铭记}{ming2ji4}{11,5}{⾦,⾔}[HSK 7-9]
  \definition{v.}{lembrar sempre; gravar na mente; guardar algo na memória com muito carinho}
\end{EntryWithPhonetic}

%%%%%%%%%% 盟 %%%%%%%%%%
\subsection*{盟}\addcontentsline{loh}{figure}{盟 \dpy{ming2}}

\begin{EntryWithPhonetic}{盟}{ming2}{13}{⽫}
  \definition{v.}{jurar; prometer; fazer um juramento}
  \seeref{meng2}
\end{EntryWithPhonetic}

%%%%%%%%%% 命 %%%%%%%%%%
\subsection*{命}\addcontentsline{loh}{figure}{命 \dpy{ming4}}

\begin{EntryWithPhonetic}{命}{ming4}{8}{⼝}[HSK 6-9]
  \definition[条]{s.}{vida | sorte; destino; fado | ordem; comando; instrução | atribuição de um nome, título etc.}
  \definition{v.}{ordenar; nomear | atribuir (um nome etc.)}
\end{EntryWithPhonetic}

\begin{EntryWithPhonetic}{命令}{ming4ling4}{8,5}{⼝,⼈}[HSK 5]
  \definition[条,项,道,个]{s.}{ordem; comando; instruções emitidas pelos superiores aos subordinados}
  \definition{v.}{ordenar; comandar}
\end{EntryWithPhonetic}

\begin{EntryWithPhonetic}{命名}{ming4/ming2}{8,6}{⼝,⼝}[HSK 7-9]
  \definition{v.+compl.}{dar nome a alguém ou a alguma coisa; atribuir um nome a; dar nome; conferir um nome; geralmente não usado para fins pessoais}
\end{EntryWithPhonetic}

\begin{EntryWithPhonetic}{命题}{ming4/ti2}{8,15}{⼝,⾴}[HSK 7-9]
  \definition[个]{s.}{proposição; afirmação; tese; em lógica, refere-se à forma linguística usada para expressar juízos}['地球是圆的' 是一个命题。===A afirmação de que ``a Terra é redonda'' é uma proposição.]
  \definition{v.}{atribuir um tema; formular uma pergunta}
\end{EntryWithPhonetic}

\begin{EntryWithPhonetic}{命运}{ming4yun4}{8,7}{⼝,⾡}[HSK 3]
  \definition[个]{s.}{tendência de desenvolvimento; tendência de futuro; metáfora para a direção e tendência do desenvolvimento e das mudanças | destino; sina; sorte; refere-se à vida e à morte, à riqueza e à pobreza e a todas as experiências da vida}
\end{EntryWithPhonetic}

%%%%%%%%%% 摸 %%%%%%%%%%
\subsection*{摸}\addcontentsline{loh}{figure}{摸 \dpy{mo1}}

\begin{EntryWithPhonetic}{摸}{mo1}{13}{⼿}[HSK 4]
  \definition{v.}{sentir; acariciar; tocar; tocar (um objeto) levemente com a mão e depois removê-lo ou mover a mão suavemente sobre a superfície do objeto | sentir para; tatear para; sentir algo com as mãos | descobrir; sentir; sondar; explorar; tentar fazer ou entender | sentir o caminho; tatear no escuro; andar por estradas que você não consegue reconhecer | furtar; roubar}
\end{EntryWithPhonetic}

\begin{EntryWithPhonetic}{摸索}{mo1suo3}{13,10}{⼿,⽷}[HSK 7-9]
  \definition{v.}{tatear; apalpar; explorar; sentir; procurar | investigar; estudar; explorar; descobrir; procurar; buscar; pesquisar}
\end{EntryWithPhonetic}

%%%%%%%%%% 模 %%%%%%%%%%
\subsection*{模}\addcontentsline{loh}{figure}{模 \dpy{mo2}}

\begin{EntryWithPhonetic}{模}{mo2}{14}{⽊}
  \definition{s.}{padrão | modelo; exemplo | modelo (pessoa) | exame simulado | módulo}
  \definition{v.}{imitar | copiar; emular}
  \seeref{mu2}
\end{EntryWithPhonetic}

\begin{EntryWithPhonetic}{模范}{mo2fan4}{14,9}{⽊,⾋}[HSK 5]
  \definition{adj.}{exemplar}
  \definition{s.}{modelo; exemplo excelente; pessoa exemplar; coisa exemplar; pessoas ou coisas exemplares que servem de modelo}
\end{EntryWithPhonetic}

\begin{EntryWithPhonetic}{模仿}{mo2fang3}{14,6}{⽊,⼈}[HSK 5]
  \definition{v.}{copiar; imitar; aprender a fazer algo seguindo um modelo pronto}
\end{EntryWithPhonetic}

\begin{EntryWithPhonetic}{模糊}{mo2hu5}{14,15}{⽊,⽶}[HSK 5]
  \definition{adj.}{vago; confuso; indistinto}
  \definition{v.}{confundir; desorientar}
\end{EntryWithPhonetic}

\begin{EntryWithPhonetic}{模拟}{mo2ni3}{14,7}{⽊,⼿}[HSK 7-9]
  \definition{v.}{ser análogo; imitar; simular; fazer de maneira formal}
\end{EntryWithPhonetic}

\begin{EntryWithPhonetic}{模式}{mo2shi4}{14,6}{⽊,⼷}[HSK 5]
  \definition{s.}{modelo; modo; padrão; a forma padrão de algo ou o modelo padrão que as pessoas podem seguir}
\end{EntryWithPhonetic}

\begin{EntryWithPhonetic}{模特儿}{mo2 te4r5}{14,10,2}{⽊,⽜,⼉}[HSK 4]
  \definition[个,名,位]{s.}{modelo (pessoa que posa para um fotógrafo ou pintor ou escultor); objeto de representação ou referência usado por artistas para esboços e esculturas, como o corpo humano, objetos, modelos etc.; também se refere aos arquétipos que os estudiosos da literatura usam para retratar seus personagens | modelo (uma pessoa que usa roupas para exibir modas); pessoa ou manequim usado para exibir estilos de roupas}
\end{EntryWithPhonetic}

\begin{EntryWithPhonetic}{模型}{mo2xing2}{14,9}{⽊,⼟}[HSK 4]
  \definition[个]{s.}{modelo; padrão; itens feitos em escala com base em objetos ou desenhos | molde; padrão; molde para fundir máquinas, objetos, etc.}
\end{EntryWithPhonetic}

%%%%%%%%%% 膜 %%%%%%%%%%
\subsection*{膜}\addcontentsline{loh}{figure}{膜 \dpy{mo2}}

\begin{EntryWithPhonetic}{膜}{mo2}{14}{⾁}[HSK 6]
  \definition[张]{s.}{membrana | filme; revestimento fino}
\end{EntryWithPhonetic}

\begin{EntryWithPhonetic}{膜拜}{mo2bai4}{14,9}{⾁,⼿}
  \definition{v.}{ajoelhar-se e se curvar com as mãos unidas no nível da testa | ter ou mostrar sentimentos fortes de respeito e admiração por um deus}
\end{EntryWithPhonetic}

%%%%%%%%%% 摩 %%%%%%%%%%
\subsection*{摩}\addcontentsline{loh}{figure}{摩 \dpy{mo2}}

\begin{EntryWithPhonetic}{摩}{mo2}{15}{⼿}
  \definition{v.}{esfregar; raspar; tocar | refletir; estudar | afagar}
\end{EntryWithPhonetic}

\begin{EntryWithPhonetic}{摩擦}{mo2ca1}{15,17}{⼿,⼿}[HSK 5]
  \definition{s.}{atrito; desacordo; conflito (entre duas partes); a ação de impedir o movimento relativo entre dois objetos em contato, produzida na superfície de contato | atrito; metáfora para o conflito entre as duas partes}
  \definition{v.}{esfregar}
\end{EntryWithPhonetic}

\begin{EntryWithPhonetic}{摩托}{mo2 tuo1}{15,6}{⼿,⼿}[HSK 5]
  \definition[辆]{s.}{Empréstimo linguístico: motor; motor de combustão interna | Empréstimo linguístico: motocicleta, abreviação de 摩托车}
  \seealsoref{摩托车}{mo2tuo1che1}
\end{EntryWithPhonetic}

\begin{EntryWithPhonetic}{摩托车}{mo2tuo1che1}{15,6,4}{⼿,⼿,⾞}
  \definition[辆,部]{s.}{(empréstimo linguístico) motocicleta}
\end{EntryWithPhonetic}

%%%%%%%%%% 磨 %%%%%%%%%%
\subsection*{磨}\addcontentsline{loh}{figure}{磨 \dpy{mo2}}

\begin{EntryWithPhonetic}{磨}{mo2}{16}{⽯}[HSK 6]
  \definition{v.}{esfregar; desgastar | moer; refletir; polir | desgastar; esgotar; cansar; exaurir | incomodar; causar problemas | destruir; obliterar; extinguir-se | ficar ocioso; perder tempo; perder tempo; procrastinar}
  \seeref{mo4}
\end{EntryWithPhonetic}

\begin{EntryWithPhonetic}{磨菇}{mo2gu5}{16,11}{⽯,⾋}
  \variantof{蘑菇}
\end{EntryWithPhonetic}

\begin{EntryWithPhonetic}{磨合}{mo2he2}{16,6}{⽯,⼝}[HSK 7-9]
  \definition{v.}{amaciar; máquinas e veículos novos ou reformados, após um período de operação, têm suas marcas de usinagem suavizadas, tornando as superfícies de fricção mais bem vedadas | (pessoas) conviver em harmonia; aprender a se dar bem; acomodar-se mutuamente}
\end{EntryWithPhonetic}

\begin{EntryWithPhonetic}{磨难}{mo2nan4}{16,10}{⽯,⾫}[HSK 7-9]
  \definition{s.}{tribulação; dificuldade; sofrimento; o tormento sofrido em circunstâncias difíceis também é chamado de tribulação}
\end{EntryWithPhonetic}

\begin{EntryWithPhonetic}{磨损}{mo2sun3}{16,10}{⽯,⼿}[HSK 7-9]
  \definition{v.}{desgastar; causar abrasão; desgastar por fricção e uso}
\end{EntryWithPhonetic}

%%%%%%%%%% 蘑 %%%%%%%%%%
\subsection*{蘑}\addcontentsline{loh}{figure}{蘑 \dpy{mo2}}

\begin{EntryWithPhonetic}{蘑}{mo2}{19}{⾋}
  \definition{s.}{cogumelo}
\end{EntryWithPhonetic}

\begin{EntryWithPhonetic}{蘑菇}{mo2gu5}{19,11}{⾋,⾋}[HSK 7-9]
  \definition[个,朵,斤,种]{s.}{cogumelo; termo genérico para fungos em forma de guarda-chuva; referindo-se especificamente a cogumelos champignon ou cogumelos shiitake}
  \definition{v.}{afligir; importunar; insistir em | demorar; enrolar; movimentar-se lenta e arrastadamente}
\end{EntryWithPhonetic}

%%%%%%%%%% 魔 %%%%%%%%%%
\subsection*{魔}\addcontentsline{loh}{figure}{魔 \dpy{mo2}}

\begin{EntryWithPhonetic}{魔}{mo2}{20}{⿁}
  \definition{adj.}{místico; misterioso; mágico}
  \definition{s.}{espírito maligno; demônio; diabo; monstro | mágico; místico}
\end{EntryWithPhonetic}

\begin{EntryWithPhonetic}{魔鬼}{mo2gui3}{20,9}{⿁,⿁}[HSK 7-9]
  \definition[个,些,群]{s.}{diabo; demônio; monstro; na religião ou mitologia, refere-se a fantasmas ou monstros malignos; metaforicamente, também pode se referir a pessoas perversas que cometem atos malignos}
\end{EntryWithPhonetic}

\begin{EntryWithPhonetic}{魔术}{mo2shu4}{20,5}{⿁,⽊}[HSK 7-9]
  \definition[个,场]{s.}{magia; ilusionismo; prestidigitação; truques; utilizando princípios físicos e químicos ou dispositivos especiais, os objetos podem aparecer, desaparecer ou sofrer mudanças maravilhosas de maneira sutil e imperceptível}
\end{EntryWithPhonetic}

\begin{EntryWithPhonetic}{魔头}{mo2tou2}{20,5}{⿁,⼤}
  \definition[个,些]{s.}{diabo; demônio; monstro; espírito maligno}
\end{EntryWithPhonetic}

%%%%%%%%%% 抹 %%%%%%%%%%
\subsection*{抹}\addcontentsline{loh}{figure}{抹 \dpy{mo3}}

\begin{EntryWithPhonetic}{抹}{mo3}{8}{⼿}[HSK 7-9]
  \definition{v.}{colocar; aplicar; untar; engessar | limpar | anular; apagar | (para nuvem, etc.) irradiar; raiar; riscar; traçar | riscar; cancelar; marcar; remover; excluir}
  \seeref{ma1}
  \seeref{mo4}
\end{EntryWithPhonetic}

\begin{EntryWithPhonetic}{抹泪}{mo3lei4}{8,8}{⼿,⽔}
  \definition{v.}{limpar as lágrimas | Figurativo: derramar lágrimas}
\end{EntryWithPhonetic}

%%%%%%%%%% 末 %%%%%%%%%%
\subsection*{末}\addcontentsline{loh}{figure}{末 \dpy{mo4}}

\begin{EntryWithPhonetic}{末}{mo4}{5}{⽊}[HSK 4]
  \definition{adj.}{último; final}
  \definition{s.}{ponta; terminal; extremidade; o final de algo | não essenciais; detalhes secundários | fim; final | pó; poeira | um papel na ópera tradicional}
\end{EntryWithPhonetic}

\begin{EntryWithPhonetic}{末日}{mo4ri4}{5,4}{⽊,⽇}[HSK 7-9]
  \definition{s.}{último dia; dia do juízo final; no cristianismo, refere-se ao último dia do mundo, geralmente significando o dia da morte ou da destruição}
\end{EntryWithPhonetic}

%%%%%%%%%% 没 %%%%%%%%%%
\subsection*{没}\addcontentsline{loh}{figure}{没 \dpy{mo4}}

\begin{EntryWithPhonetic}{没}{mo4}{7}{⽔}
  \definition{adj.}{último; final}
  \definition{v.}{afundar na água; submergir | transbordar; subir além; exceder ou ultrapassar | esconder-se; desaparecer; sumir; ocultar-se | confiscar; expropriar | morrer}
  \variantof{没}
  \seeref{mei2}
\end{EntryWithPhonetic}

\begin{EntryWithPhonetic}{没落}{mo4luo4}{7,12}{⽔,⾋}[HSK 7-9]
  \definition{v.}{declinar; diminuir; estar em declínio; afundar}
\end{EntryWithPhonetic}

\begin{EntryWithPhonetic}{没收}{mo4 shou1}{7,6}{⽔,⽁}[HSK 6]
  \definition{v.}{confiscar; expropriar; os bens e pertences de pessoas ou grupos que violem leis ou proibições serão tornados propriedade pública, de acordo com a lei}
\end{EntryWithPhonetic}

%%%%%%%%%% 抹 %%%%%%%%%%
\subsection*{抹}\addcontentsline{loh}{figure}{抹 \dpy{mo4}}

\begin{EntryWithPhonetic}{抹}{mo4}{8}{⼿}
  \definition{v.}{rebocar; engessar; alisar a massa ou o gesso com uma espátula | virar; contornar; dar uma volta de perto}
  \seeref{ma1}
  \seeref{mo3}
\end{EntryWithPhonetic}

%%%%%%%%%% 陌 %%%%%%%%%%
\subsection*{陌}\addcontentsline{loh}{figure}{陌 \dpy{mo4}}

\begin{EntryWithPhonetic}{陌}{mo4}{8}{⾩}
  \definition[个]{s.}{Literário: caminho entre campos (indo de leste a oeste); trilhas entre campos que correm de leste a oeste; geralmente se refere a estradas nos campos | Obsoleto: estrada}
\end{EntryWithPhonetic}

\begin{EntryWithPhonetic}{陌生}{mo4sheng1}{8,5}{⾩,⽣}[HSK 7-9]
  \definition{adj.}{estranho; desconhecido; inexperiente; indica algo desconhecido ou não familiar e é frequentemente usado como um atributo ou predicado}
\end{EntryWithPhonetic}

%%%%%%%%%% 脉 %%%%%%%%%%
\subsection*{脉}\addcontentsline{loh}{figure}{脉 \dpy{mo4}}

\begin{EntryWithPhonetic}{脉}{mo4}{9}{⾁}
  \definition{adv.}{afetuosamente; amorosamente; carinhosamente; expressar afeto silenciosamente através dos olhos ou ações}
  \seeref{mai4}
\end{EntryWithPhonetic}

%%%%%%%%%% 莫 %%%%%%%%%%
\subsection*{莫}\addcontentsline{loh}{figure}{莫 \dpy{mo4}}

\begin{EntryWithPhonetic}{莫}{mo4}{10}{⾋}
  \definition*{s.}{Sobrenome: Mo}
  \definition{adv.}{não, frequentemente usado em frases imperativas | não; não pode | pode ser que; não pode ser que; é possível que}
  \definition{pron.}{nenhum; nada; ninguém; significa 没有谁 ou 没有哪一种东西}
  \seealsoref{没有哪一种东西}{mei2you3 na3 yi4 zhong3 dong1xi1}
  \seealsoref{没有谁}{mei2you3 shei2}
\end{EntryWithPhonetic}

\begin{EntryWithPhonetic}{莫非}{mo4fei1}{10,8}{⾋,⾮}[HSK 7-9]
  \definition{adv.}{pode ser que; é possível que; será que; frequentemente usado em conjunto com 不成}
  \seealsoref{不成}{bu4 cheng2}
\end{EntryWithPhonetic}

\begin{EntryWithPhonetic}{莫过于}{mo4guo4yu2}{10,6,3}{⾋,⾡,⼆}[HSK 7-9]
  \definition{conj.}{nada é mais\dots do que; nada é melhor do que; nada pode superar}
\end{EntryWithPhonetic}

\begin{EntryWithPhonetic}{莫名其妙}{mo4ming2qi2miao4}{10,6,8,7}{⾋,⼝,⼋,⼥}[HSK 7-9]
  \definition{adj.}{desconcertante; enigmático; bizarro; sem lógica; inexplicável; ninguém consegue explicar seu mistério (razão), indicando que as coisas são estranhas e incompreensíveis; o caractere 名 também é escrito como 明}
\end{EntryWithPhonetic}

%%%%%%%%%% 漠 %%%%%%%%%%
\subsection*{漠}\addcontentsline{loh}{figure}{漠 \dpy{mo4}}

\begin{EntryWithPhonetic}{漠}{mo4}{13}{⽔}
  \definition{adj.}{indiferente; desinteressado | distante; frio; indiferente; despreocupado}
  \definition{s.}{deserto}
\end{EntryWithPhonetic}

\begin{EntryWithPhonetic}{漠然}{mo4ran2}{13,12}{⽔,⽕}[HSK 7-9]
  \definition{adj.}{indiferente; apático; desinteressado; sem qualquer preocupação}
  \definition{adv.}{indiferentemente; apaticamente}
\end{EntryWithPhonetic}

%%%%%%%%%% 嘿 %%%%%%%%%%
\subsection*{嘿}\addcontentsline{loh}{figure}{嘿 \dpy{mo4}}

\begin{EntryWithPhonetic}{嘿}{mo4}{15}{⼝}
  \definition{adj.}{quieto; silencioso; tácito}
  \seeref{hei1}
\end{EntryWithPhonetic}

%%%%%%%%%% 墨 %%%%%%%%%%
\subsection*{墨}\addcontentsline{loh}{figure}{墨 \dpy{mo4}}

\begin{EntryWithPhonetic}{墨}{mo4}{15}{⿊}[HSK 7-9]
  \definition*{s.}{Escola Moísta; Moísmo | México, abreviação de 墨西哥}
  \definition{adj.}{preto; escuro como breu | corrupto | escuro}
  \definition{s.}{tinta chinesa; bastão de tinta | pigmento; tinta | caligrafia ou pintura | aprendizagem; alfabetização | marcador de linha de carpinteiro; marcador de tinta | tatuar o rosto (um castigo); uma punição na China antiga | corrupção; peculato; fraude}
  \seealsoref{墨西哥}{mo4xi1ge1}
\end{EntryWithPhonetic}

\begin{EntryWithPhonetic}{墨镜}{mo4jing4}{15,16}{⿊,⾦}
  \definition[只,双,副]{s.}{óculos escuros}
\end{EntryWithPhonetic}

\begin{EntryWithPhonetic}{墨水}{mo4 shui3}{15,4}{⿊,⽔}[HSK 6]
  \definition[瓶]{s.}{tinta chinesa preparada; tinta (para caneta-tinteiro) | aprendizagem; alfabetização; uma metáfora para o conhecimento ou a capacidade de ler e escrever}
\end{EntryWithPhonetic}

\begin{EntryWithPhonetic}{墨西哥}{mo4xi1ge1}{15,6,10}{⿊,⾑,⼝}
  \definition*{s.}{México; Planalto no México}
\end{EntryWithPhonetic}

%%%%%%%%%% 磨 %%%%%%%%%%
\subsection*{磨}\addcontentsline{loh}{figure}{磨 \dpy{mo4}}

\begin{EntryWithPhonetic}{磨}{mo4}{16}{⽯}
  \definition[盘]{s.}{mó (pedra pesada e redonda para moinho)}
  \definition{v.}{moer; esfarelar; triturar | virar; inverter a marcha}
  \seeref{mo2}
\end{EntryWithPhonetic}

%%%%%%%%%% 默 %%%%%%%%%%
\subsection*{默}\addcontentsline{loh}{figure}{默 \dpy{mo4}}

\begin{EntryWithPhonetic}{默}{mo4}{16}{⿊}
  \definition*{s.}{Sobrenome: Mo}
  \definition{adj.}{taciturno; reservado | silencioso}
  \definition{v.}{escrever de memória}
\end{EntryWithPhonetic}

\begin{EntryWithPhonetic}{默读}{mo4du2}{16,10}{⿊,⾔}[HSK 7-9]
  \definition{v.}{ler em silêncio (oposto de 朗读) | subvocalizar}
  \seealsoref{朗读}{lang3du2}
\end{EntryWithPhonetic}

\begin{EntryWithPhonetic}{默默}{mo4mo4}{16,16}{⿊,⿊}[HSK 4]
  \definition{adj.}{mudo; quieto; silencioso}
  \definition{adv.}{silenciosamente}
\end{EntryWithPhonetic}

\begin{EntryWithPhonetic}{默默无闻}{mo4mo4-wu2wen2}{16,16,4,9}{⿊,⿊,⽆,⾨}[HSK 7-9]
  \definition{expr.}{obscuro; quieto e desconhecido; desconhecido}
\end{EntryWithPhonetic}

\begin{EntryWithPhonetic}{默契}{mo4qi4}{16,9}{⿊,⼤}[HSK 7-9]
  \definition{adj.}{bem coordenado; mutuamente e tacitamente compreendido/acordado; descreve uma conexão profunda entre duas pessoas que transcende as palavras}
  \definition[些,种,份,点]{s.}{acordo ou contrato secreto; entendimento tácito}
\end{EntryWithPhonetic}

%%%%%%%%%% 谋 %%%%%%%%%%
\subsection*{谋}\addcontentsline{loh}{figure}{谋 \dpy{mou2}}

\begin{EntryWithPhonetic}{谋}{mou2}{11}{⾔}
  \definition*{s.}{Sobrenome: Mou}
  \definition[个]{s.}{estratagema; plano; esquema | estratégia; ideia; esquema; plano}
  \definition{v.}{trabalhar para; buscar | consultar | planejar; traçar | conferir; discutir}
\end{EntryWithPhonetic}

\begin{EntryWithPhonetic}{谋害}{mou2hai4}{11,10}{⾔,⼧}[HSK 7-9]
  \definition{v.}{planejar um assassinato; conspirar para matar; tramar para matar | planejar uma conspiração contra; conspirar contra alguém; tramar para incriminar}
\end{EntryWithPhonetic}

\begin{EntryWithPhonetic}{谋求}{mou2qiu2}{11,7}{⾔,⽔}[HSK 7-9]
  \definition{v.}{buscar; esforçar-se por; estar em questão de; tentar obter}
\end{EntryWithPhonetic}

\begin{EntryWithPhonetic}{谋生}{mou2sheng1}{11,5}{⾔,⽣}[HSK 7-9]
  \definition{v.}{ganhar a vida; obter renda; tentar encontrar uma maneira de ganhar a vida}
\end{EntryWithPhonetic}

%%%%%%%%%% 某 %%%%%%%%%%
\subsection*{某}\addcontentsline{loh}{figure}{某 \dpy{mou3}}

\begin{EntryWithPhonetic}{某}{mou3}{9}{⽊}[HSK 3]
  \definition{pron.}{alguém ou algo indefinido; refere-se a pessoas ou coisas incertas | referindo-se a si mesmo; em vez do seu próprio nome | alguns; certos; refere-se a uma pessoa ou coisa específica cujo nome não se sabe ou não se pode revelar | tal e tal; substituir o nome de outra pessoa (geralmente com um tom rude)}
\end{EntryWithPhonetic}

\begin{EntryWithPhonetic}{某些}{mou3 xie1}{9,8}{⽊,⼆}
  \definition{pron.}{certos; alguns; uns poucos; refere-se a pessoas ou coisas que são conhecidas, mas das quais não se fala}
\end{EntryWithPhonetic}

%%%%%%%%%% 模 %%%%%%%%%%
\subsection*{模}\addcontentsline{loh}{figure}{模 \dpy{mu2}}

\begin{EntryWithPhonetic}{模}{mu2}{14}{⽊}
  \definition*{s.}{Sobrenome: Mu}
  \definition{s.}{molde; padrão; matriz}
  \seeref{mo2}
\end{EntryWithPhonetic}

\begin{EntryWithPhonetic}{模具}{mu2ju4}{14,8}{⽊,⼋}
  \definition{s.}{molde | matriz | padrão}
\end{EntryWithPhonetic}

\begin{EntryWithPhonetic}{模样}{mu2yang4}{14,10}{⽊,⽊}[HSK 5]
  \definition[副,种]{s.}{aparência; a aparência ou o estilo de vestir de uma pessoa | indicando uma estimativa aproximada de tempo ou idade; expressão de estimativas relativas a tempo, idade, etc. | tendência; situação; inclinação}
\end{EntryWithPhonetic}

%%%%%%%%%% 母 %%%%%%%%%%
\subsection*{母}\addcontentsline{loh}{figure}{母 \dpy{mu3}}

\begin{EntryWithPhonetic}{母}{mu3}{5}{⽏}[HSK 6][Kangxi 80]
  \definition*{s.}{Sobrenome: Mu}
  \definition{adj.}{fêmea}
  \definition[位,名,个,些]{s.}{mãe | fêmea (animal) (oposto a 公) | origem; pais | parentes idosas; geralmente se refere a mulheres idosas | côncavo | fonte; algo que tem a capacidade ou função de produzir outras coisas}
  \seealsoref{公}{gong1}
\end{EntryWithPhonetic}

\begin{EntryWithPhonetic}{母鸡}{mu3ji1}{5,7}{⽏,⿃}[HSK 6]
  \definition{s.}{galinha}
\end{EntryWithPhonetic}

\begin{EntryWithPhonetic}{母女}{mu3 nv3}{5,3}{⽏,⼥}[HSK 6]
  \definition{s.}{mãe e filha}
\end{EntryWithPhonetic}

\begin{EntryWithPhonetic}{母亲}{mu3qin1}{5,9}{⽏,⼇}[HSK 3]
  \definition[位,名,个,些]{s.}{mãe}
\end{EntryWithPhonetic}

\begin{EntryWithPhonetic}{母语}{mu3yu3}{5,9}{⽏,⾔}
  \definition{s.}{língua materna | língua nativa}
\end{EntryWithPhonetic}

\begin{EntryWithPhonetic}{母子}{mu3 zi3}{5,3}{⽏,⼦}[HSK 6]
  \definition{s.}{mãe e filho}
\end{EntryWithPhonetic}

%%%%%%%%%% 亩 %%%%%%%%%%
\subsection*{亩}\addcontentsline{loh}{figure}{亩 \dpy{mu3}}

\begin{EntryWithPhonetic}{亩}{mu3}{7}{⼇}[HSK 7-9]
  \definition{clas.}{usado para campos | acre; unidade de área igual a um décimo quinto de um hectare}
\end{EntryWithPhonetic}

%%%%%%%%%% 牡 %%%%%%%%%%
\subsection*{牡}\addcontentsline{loh}{figure}{牡 \dpy{mu3}}

\begin{EntryWithPhonetic}{牡}{mu3}{7}{⽜}
  \definition[些]{s.}{macho (de certas aves e animais) (oposto de 牝) | colinas | peônia}
  \seealsoref{牝}{pin4}
\end{EntryWithPhonetic}

\begin{EntryWithPhonetic}{牡丹江}{mu3dan1jiang1}{7,4,6}{⽜,⼂,⽔}
  \definition*{s.}{Cidade de Mudanjiang na província de Heilongjiang, 黑龙江 no nordeste da China}
  \seealsoref{黑龙江}{hei1long2jiang1}
\end{EntryWithPhonetic}

\begin{EntryWithPhonetic}{牡丹}{mu3dan5}{7,4}{⽜,⼂}[HSK 7-9]
  \definition{s.}{peônia; peônia arbórea}
\end{EntryWithPhonetic}

%%%%%%%%%% 姥 %%%%%%%%%%
\subsection*{姥}\addcontentsline{loh}{figure}{姥 \dpy{mu3}}

\begin{EntryWithPhonetic}{姥}{mu3}{9}{⼥}
  \definition*{s.}{Sobrenome: Mu}
  \definition{s.}{mulher idosa; Literário: velha senhora}
  \seeref{lao3}
\end{EntryWithPhonetic}

%%%%%%%%%% 木 %%%%%%%%%%
\subsection*{木}\addcontentsline{loh}{figure}{木 \dpy{mu4}}

\begin{EntryWithPhonetic}{木}{mu4}{4}{⽊}[Kangxi 75]
  \definition{adj.}{de madeira; feito de madeira | estúpido; de raciocínio lento; atordoado; lento para reagir | simplório; chato | entorpecido; de madeira; dormência localizada ou perda de sensibilidade}
  \definition{s.}{árvore | madeira; madeiramento | caixão}
\end{EntryWithPhonetic}

\begin{EntryWithPhonetic}{木板}{mu4ban3}{4,8}{⽊,⽊}[HSK 7-9]
  \definition[张,块]{s.}{prancha; tábua; ripa}
\end{EntryWithPhonetic}

\begin{EntryWithPhonetic}{木材}{mu4cai2}{4,7}{⽊,⽊}[HSK 7-9]
  \definition[批,根]{s.}{madeira; tábuas; madeira serrada; materiais após processamento preliminar de árvores abatidas}
\end{EntryWithPhonetic}

\begin{EntryWithPhonetic}{木匠}{mu4jiang4}{4,6}{⽊,⼕}[HSK 7-9]
  \definition[名,位,个]{s.}{carpinteiro; trabalhadores que fabricam ou reparam peças de madeira, e que produzem e instalam componentes de madeira para casas}
\end{EntryWithPhonetic}

\begin{EntryWithPhonetic}{木偶}{mu4'ou3}{4,11}{⽊,⼈}[HSK 7-9]
  \definition{s.}{fantoche, marionete | imagem de madeira; figura esculpida}
\end{EntryWithPhonetic}

\begin{EntryWithPhonetic}{木头}{mu4tou5}{4,5}{⽊,⼤}[HSK 3]
  \definition[根,块,堆,截]{s.}{tronco; madeira; lenha; denominação genérica para madeira e materiais de madeira}
\end{EntryWithPhonetic}

%%%%%%%%%% 目 %%%%%%%%%%
\subsection*{目}\addcontentsline{loh}{figure}{目 \dpy{mu4}}

\begin{EntryWithPhonetic}{目}{mu4}{5}{⽬}[Kangxi 109]
  \definition*{s.}{Sobrenome: Mu}
  \definition{s.}{olho | item | (biologia) ordem | lista de coisas; catálogo; sumário | buraco em uma rede; malha (abertura)  | (de documentos, teses, etc.) nome; título | ponto; ponto de território, um termo do Go; refere-se à intersecção das linhas verticais e horizontais no tabuleiro, uma intersecção é chamada de 一目, \dpy{yi2 mu4}}
  \definition{v.}{(literário) olhar; considerar}
\end{EntryWithPhonetic}

\begin{EntryWithPhonetic}{目标}{mu4biao1}{5,9}{⽬,⽊}[HSK 3]
  \definition[个,项]{s.}{alvo; objetivo; objeto de tiro, ataque ou busca | objetivo; meta; destino; a situação ou padrão que se deseja alcançar}
\end{EntryWithPhonetic}

\begin{EntryWithPhonetic}{目不转睛}{mu4bu4zhuan3jing1}{5,4,8,13}{⽬,⼀,⾞,⽬}[HSK 7-9]
  \definition{expr.}{``Olhos fixos.''; as pupilas não se movem; um olhar fixo; observar com a máxima concentração; estar totalmente atento a\dots; concentrar o olhar em; contemplar\dots; manter o olhar fixo em\dots; olhar para algo (ou alguém) atentamente; olhar com olhos fixos; não desviar o olhar de\dots; fixar os olhos em\dots; encarar\dots fixamente; olhar para\dots sem piscar; observar\dots sem pestanejar; com todos os olhos; com toda a atenção voltada para a pessoa; descreve alguém que está muito concentrado e absorto naquilo que está observando}
\end{EntryWithPhonetic}

\begin{EntryWithPhonetic}{目瞪口呆}{mu4deng4-kou3dai1}{5,17,3,7}{⽬,⽬,⼝,⼝}[HSK 7-9]
  \definition{expr.}{estupefato; atônito; atordoado; descreve a sensação de estar assustado e atordoado}
\end{EntryWithPhonetic}

\begin{EntryWithPhonetic}{目的}{mu4di4}{5,8}{⽬,⽩}[HSK 2]
  \definition[个,些,种]{s.}{objetivo; meta; alvo; finalidade; propósito; o lugar ou situação que se deseja alcançar; o resultado que se deseja obter; o centro do alvo}
\end{EntryWithPhonetic}

\begin{EntryWithPhonetic}{目的地}{mu4di4di4}{5,8,6}{⽬,⽩,⼟}[HSK 7-9]
  \definition{s.}{destino; o lugar aonde quero chegar}
\end{EntryWithPhonetic}

\begin{EntryWithPhonetic}{目睹}{mu4du3}{5,13}{⽬,⽬}[HSK 7-9]
  \definition{v.}{testemunhar; ver com os próprios olhos}
\end{EntryWithPhonetic}

\begin{EntryWithPhonetic}{目光}{mu4guang1}{5,6}{⽬,⼉}[HSK 5]
  \definition[道,束,种]{s.}{olhar fixo; a expressão e atitude reveladas pelos olhos | visão; vista; percepção visual; a linha imaginária formada entre os olhos e o objeto quando se olha para ele | perspicácia (capacidade de observar e reconhecer coisas); conhecimento adquirido através do contato com as coisas, capacidade de observar as coisas}
\end{EntryWithPhonetic}

\begin{EntryWithPhonetic}{目录}{mu4lu4}{5,8}{⽬,⼹}[HSK 7-9]
  \definition[个]{s.}{lista; conteúdo; catálogo; listar os itens em uma determinada ordem para referência | conteúdo; sumário; os títulos dos capítulos listados em livros e periódicos (geralmente colocados antes do texto principal)}
\end{EntryWithPhonetic}

\begin{EntryWithPhonetic}{目前}{mu4qian2}{5,9}{⽬,⼑}[HSK 3]
  \definition{adv.}{agora; recentemente; no momento; no presente}
\end{EntryWithPhonetic}

\begin{EntryWithPhonetic}{目中无人}{mu4zhong1-wu2ren2}{5,4,4,2}{⽬,⼁,⽆,⼈}[HSK 7-9]
  \definition{expr.}{considerar todos inferiores a si; ​​arrogante; presunçoso; aos seus olhos, não há outro; não se importar com ninguém; tratar os outros com desprezo; se achar superior e menosprezar os outros; não ter respeito por ninguém; olhar para todos com desdém; ser arrogante; ser presunçoso; ficar convencido demais; orgulhoso demais; com o nariz empinado; orgulhoso e altivo}
\end{EntryWithPhonetic}

%%%%%%%%%% 沐 %%%%%%%%%%
\subsection*{沐}\addcontentsline{loh}{figure}{沐 \dpy{mu4}}

\begin{EntryWithPhonetic}{沐}{mu4}{7}{⽔}
  \definition*{s.}{Sobrenome: Mu}
  \definition{v.}{Significado original: lavar o cabelo | Literário: receber (ou ser dado) (bondade, favor, etc.)}
  \definition{v.}{Transliteração: banhar-se}
\end{EntryWithPhonetic}

\begin{EntryWithPhonetic}{沐浴露}{mu4yu4lu4}{7,10,21}{⽔,⽔,⾬}[HSK 7-9]
  \definition[瓶]{s.}{gel de banho; sabonete líquido; creme de banho}
\end{EntryWithPhonetic}

%%%%%%%%%% 牧 %%%%%%%%%%
\subsection*{牧}\addcontentsline{loh}{figure}{牧 \dpy{mu4}}

\begin{EntryWithPhonetic}{牧}{mu4}{8}{⽜}
  \definition*{s.}{Sobrenome: Mu}
  \definition{v.}{cuidar (de ovelhas, gado, etc.); pastorear}
\end{EntryWithPhonetic}

\begin{EntryWithPhonetic}{牧场}{mu4chang3}{8,6}{⽜,⼟}[HSK 7-9]
  \definition[个,片]{s.}{pastagem; campo de pastoreio; área de pastagem; pastos}
\end{EntryWithPhonetic}

\begin{EntryWithPhonetic}{牧民}{mu4min2}{8,5}{⽜,⽒}[HSK 7-9]
  \definition[个]{s.}{pastor; pessoas em áreas pastoris que ganham a vida criando gado}
\end{EntryWithPhonetic}

%%%%%%%%%% 募 %%%%%%%%%%
\subsection*{募}\addcontentsline{loh}{figure}{募 \dpy{mu4}}

\begin{EntryWithPhonetic}{募}{mu4}{12}{⼒}
  \definition*{s.}{Sobrenome: Mu}
  \definition{v.}{arrecadar; coletar | alistar; recrutar}
\end{EntryWithPhonetic}

\begin{EntryWithPhonetic}{募捐}{mu4/juan1}{12,10}{⼒,⼿}[HSK 7-9]
  \definition[场,次]{v.+compl.}{solicitar contribuições; arrecadar doações; passar o chapéu}
\end{EntryWithPhonetic}

%%%%%%%%%% 墓 %%%%%%%%%%
\subsection*{墓}\addcontentsline{loh}{figure}{墓 \dpy{mu4}}

\begin{EntryWithPhonetic}{墓}{mu4}{13}{⼟}
  \definition[座,个,号]{s.}{sepultura; túmulo; mausoléu}
\end{EntryWithPhonetic}

\begin{EntryWithPhonetic}{墓碑}{mu4bei1}{13,13}{⼟,⽯}[HSK 7-9]
  \definition[块,座]{s.}{lápide; túmulo}
\end{EntryWithPhonetic}

\begin{EntryWithPhonetic}{墓地}{mu4di4}{13,6}{⼟,⼟}[HSK 7-9]
  \definition{s.}{cemitério; local de sepultamento}
\end{EntryWithPhonetic}

%%%%%%%%%% 幕 %%%%%%%%%%
\subsection*{幕}\addcontentsline{loh}{figure}{幕 \dpy{mu4}}

\begin{EntryWithPhonetic}{幕}{mu4}{13}{⼱}[HSK 7-9]
  \definition{s.}{tenda | cortina; tela | ato (de uma peça); cena; um trecho mais completo da peça | Obsoleto: gabinete de um governador ou de um general | materiais de cobertura como tecido, seda e feltro}
\end{EntryWithPhonetic}

\begin{EntryWithPhonetic}{幕后}{mu4hou4}{13,6}{⼱,⼝}[HSK 7-9]
  \definition{s.}{nos bastidores; por trás das cenas; por trás da cortina do palco; uma metáfora para aqueles que permanecem ocultos e seus lugares escondidos}
\end{EntryWithPhonetic}

%%%%%%%%%% 穆 %%%%%%%%%%
\subsection*{穆}\addcontentsline{loh}{figure}{穆 \dpy{mu4}}

\begin{EntryWithPhonetic}{穆}{mu4}{16}{⽲}
  \definition*{s.}{Sobrenome: Mu}
  \definition{adj.}{solene; reverente | respeitoso}
\end{EntryWithPhonetic}

\begin{EntryWithPhonetic}{穆棱}{mu4ling2}{16,12}{⽲,⽊}
  \definition*{s.}{Cidade no nível do condado de Muling, na província de Mudanjiang 牡丹江,Heilongjiang}
  \seealsoref{牡丹江}{mu3dan1jiang1}
\end{EntryWithPhonetic}

\begin{EntryWithPhonetic}{穆斯林}{mu4si1lin2}{16,12,8}{⽲,⽄,⽊}[HSK 7-9]
  \definition[个,位,名]{s.}{muçulmano}
\end{EntryWithPhonetic}

%%%%% EOF %%%%%

