%%%
%%% K
%%%
\section*{K}\addcontentsline{toc}{section}{K}\addcontentsline{loh}{figure}{\#\#\#\#\#\#\#\# K}

%%%%%%%%%% 咖 %%%%%%%%%%
\subsection*{咖}\addcontentsline{loh}{figure}{咖 \dpy{ka1}}

\begin{EntryWithPhonetic}{咖}{ka1}{8}{⼝}
  \definition[杯]{s.}{classe | café | graduação}
\end{EntryWithPhonetic}

\begin{EntryWithPhonetic}{咖啡}{ka1fei1}{8,11}{⼝,⼝}[HSK 3]
  \definition[杯,瓶,罐,壶,包,袋,盒]{s.}{(empréstimo linguístico) café}
\end{EntryWithPhonetic}

\begin{EntryWithPhonetic}{咖啡馆}{ka1fei1guan3}{8,11,11}{⼝,⼝,⾷}
  \definition[家]{s.}{cafeteria}
\end{EntryWithPhonetic}

\begin{EntryWithPhonetic}{咖啡色}{ka1fei1 se4}{8,11,6}{⼝,⼝,⾊}
  \definition{s.}{cor café}
\end{EntryWithPhonetic}

%%%%%%%%%% 卡 %%%%%%%%%%
\subsection*{卡}\addcontentsline{loh}{figure}{卡 \dpy{ka3}}

\begin{EntryWithPhonetic}{卡}{ka3}{5}{⼘}[HSK 2]
  \definition{clas.}{calorias (cal)}
  \definition[张,片]{s.}{cartão; documento semelhante a um cartão | cassete; dispositivo tipo compartimento para colocar fitas cassete no gravador | caminhão}
  \seeref{qia3}
\end{EntryWithPhonetic}

\begin{EntryWithPhonetic}{卡车}{ka3che1}{5,4}{⼘,⾞}[HSK 7-9]
  \definition[辆]{s.}{caminhão; caminhões pesados para transporte de mercadorias, equipamentos, etc.}
\end{EntryWithPhonetic}

\begin{EntryWithPhonetic}{卡车司机}{ka3che1 si1ji1}{5,4,5,6}{⼘,⾞,⼝,⽊}
  \definition{s.}{motorista de caminhão}
\end{EntryWithPhonetic}

\begin{EntryWithPhonetic}{卡片}{ka3pian4}{5,4}{⼘,⽚}[HSK 7-9]
  \definition[张,盒,套]{s.}{cartão; pedaços de papel usados para registrar diversas informações para comparação, verificação e referência}
\end{EntryWithPhonetic}

\begin{EntryWithPhonetic}{卡片游戏}{ka3pian4 you2xi4}{5,4,12,6}{⼘,⽚,⽔,⼽}
  \definition{s.}{carta de baralho; jogos de cartas}
\end{EntryWithPhonetic}

\begin{EntryWithPhonetic}{卡通}{ka3tong1}{5,10}{⼘,⾡}[HSK 7-9]
  \definition[本]{s.}{Empréstimo linguístico: \emph{cartoon}; desenho animado}
\end{EntryWithPhonetic}

%%%%%%%%%% 开 %%%%%%%%%%
\subsection*{开}\addcontentsline{loh}{figure}{开 \dpy{kai1}}

\begin{EntryWithPhonetic}{开}{kai1}{4}{⼶}[HSK 1]
  \definition*{s.}{Sobrenome: Kai}
  \definition{clas.}{divisão do papel de impressão de tamanho padrão (uma parte da folha inteira) | quilate; unidade de cálculo da quantidade de ouro puro contida no ouro}
  \definition{s.}{porcentagem; percentual}
  \definition{v.}{abrir; estar ligado; ligar | recuperar; abrir; fazer uma abertura; escavar; abrir caminho; desbravar | abrir para fora; soltar-se | descongelar (rios); tornar-se navegável | levantar; libertar | iniciar; operar; manobrar | mover; estabelecer | executar; configurar | começar; iniciar | manter | escrever; fazer uma lista de | pagamento (salários, tarifas, etc.) | ferver}
  \definition{v.aux.}{usado após um verbo, indica ampliação ou expansão | usado após um verbo, indica o início e a continuidade}
  \seealsoref{开尔文}{kai1'er3wen2}
\end{EntryWithPhonetic}

\begin{EntryWithPhonetic}{开办}{kai1ban4}{4,4}{⼶,⼒}[HSK 7-9]
  \definition{v.}{abrir; iniciar; criar; configurar; estabelecer (uma empresa ou entidade comercial)}
\end{EntryWithPhonetic}

\begin{EntryWithPhonetic}{开采}{kai1cai3}{4,8}{⼶,⾤}[HSK 7-9]
  \definition{v.}{minerar; extrair; explorar; recuperar}
\end{EntryWithPhonetic}

\begin{EntryWithPhonetic}{开场}{kai1/chang3}{4,6}{⼶,⼟}[HSK 7-9]
  \definition{v.+compl.}{(apresentação, etc.) começar; iniciar; estrear; o início de uma apresentação teatral ou de uma apresentação cultural em geral também pode ser usado metaforicamente para descrever o início de uma atividade geral}
\end{EntryWithPhonetic}

\begin{EntryWithPhonetic}{开场白}{kai1chang3bai2}{4,6,5}{⼶,⼟,⽩}[HSK 7-9]
  \definition{s.}{prólogo (de uma peça); introdução; discurso de abertura; comentários iniciais (ou introdutórios); as linhas iniciais de uma peça teatral que introduzem o tema; metaforicamente, a seção inicial de um artigo ou discurso que apresenta a ideia principal}
\end{EntryWithPhonetic}

\begin{EntryWithPhonetic}{开车}{kai1/che1}{4,4}{⼶,⾞}[HSK 1]
  \definition{v.+compl.}{dirigir um carro, trem, etc. | colocar uma máquina em funcionamento | (de um trem, etc.) partida | dirigir veículos motorizados}
\end{EntryWithPhonetic}

\begin{EntryWithPhonetic}{开除}{kai1chu2}{4,9}{⼶,⾩}[HSK 7-9]
  \definition{v.}{expulsar; demitir; dispensar; despedir}
\end{EntryWithPhonetic}

\begin{EntryWithPhonetic}{开创}{kai1chuang4}{4,6}{⼶,⼑}[HSK 6]
  \definition{v.}{começar; iniciar; fundar; ser pioneiro; estabelecer; criar}
\end{EntryWithPhonetic}

\begin{EntryWithPhonetic}{开动}{kai1dong4}{4,6}{⼶,⼒}[HSK 7-9]
  \definition{v.}{iniciar; pôr em movimento; pôr em funcionamento; iniciar operação | mover-se; marchar; estar em movimento; zarpar}
\end{EntryWithPhonetic}

\begin{EntryWithPhonetic}{开尔文}{kai1'er3wen2}{4,5,4}{⼶,⼩,⽂}
  \definition{s.}{Kelvin, temperatura absoluta | K, escala de temperatura}
\end{EntryWithPhonetic}

\begin{EntryWithPhonetic}{开发区}{kai1fa1qu1}{4,5,4}{⼶,⼜,⼖}[HSK 7-9]
  \definition*{s.}{Zona Econômica Aberta; Zona Econômica Especial}
  \definition{s.}{zona de desenvolvimento}
\end{EntryWithPhonetic}

\begin{EntryWithPhonetic}{开发商}{kai1fa1shang1}{4,5,11}{⼶,⼜,⼝}[HSK 7-9]
  \definition{s.}{incorporador (de imóveis, de um produto comercial, etc.)}
\end{EntryWithPhonetic}

\begin{EntryWithPhonetic}{开发}{kai1fa5}{4,5}{⼶,⼜}[HSK 3]
  \definition{v.}{explorar; trabalhar com recursos naturais como terras baldias, minas, florestas e energia hidráulica para fins de aproveitamento | tornar acessível; descobrir ou explorar talentos, tecnologias, etc. para aproveitamento}
\end{EntryWithPhonetic}

\begin{EntryWithPhonetic}{开放}{kai1fang4}{4,8}{⼶,⽅}[HSK 3]
  \definition{adj.}{de mente aberta; sem restrições por convenções; pensamento e ambiente não conservadores, disposto a aceitar coisas novas e novas ideias; personalidade alegre}
  \definition{v.}{florescer | abrir (para o público); levantar bloqueios, proibições, restrições, etc. | diminuir uma proibição, restrição, etc. (de política); (economia) reduzir as restrições políticas, com justificativas específicas}
\end{EntryWithPhonetic}

\begin{EntryWithPhonetic}{开工}{kai1/gong1}{4,3}{⼶,⼯}[HSK 7-9]
  \definition{v.+compl.}{(fábrica, etc.) entrar em operação; iniciar a construção}
\end{EntryWithPhonetic}

\begin{EntryWithPhonetic}{开关}{kai1guan1}{4,6}{⼶,⼋}[HSK 6]
  \definition[个,种,些]{s.}{interruptor; um dispositivo que conecta e desconecta o circuito de um dispositivo elétrico | registro; um dispositivo instalado em uma tubulação de fluido para controlar o fluxo}
\end{EntryWithPhonetic}

\begin{EntryWithPhonetic}{开花}{kai1/hua1}{4,7}{⼶,⾋}[HSK 4]
  \definition{v.+compl.}{florescer; desabrochar; estar em flor; entrar em flor;  metáfora para um coração feliz ou um rosto sorridente | explodir; quebrar; dividir | sentir-se feliz ou sorrir alegremente | (experiência) espalhar-se; (empreendimento) surgir; surgir | (cabeça) ser ferido e sangrar profusamente}
\end{EntryWithPhonetic}

\begin{EntryWithPhonetic}{开会}{kai1/hui4}{4,6}{⼶,⼈}[HSK 1]
  \definition{v.+compl.}{realizar uma reunião; ter uma reunião; participar de uma reunião (conferência)}
\end{EntryWithPhonetic}

\begin{EntryWithPhonetic}{开机}{kai1 ji1}{4,6}{⼶,⽊}[HSK 2]
  \definition{v.}{começar a filmar um filme ou programa de TV; refere"-se ao início das filmagens (de filmes, séries de TV, etc.) | ligar uma máquina}
\end{EntryWithPhonetic}

\begin{EntryWithPhonetic}{开垦}{kai1ken3}{4,9}{⼶,⼟}[HSK 7-9]
  \definition{v.}{abrir (ou recuperar) terrenos baldios; trazer para o cultivo; abrir terrenos baldios para a agricultura; preparar o terreno; cultivar}
\end{EntryWithPhonetic}

\begin{EntryWithPhonetic}{开口}{kai1/kou3}{4,3}{⼶,⼝}[HSK 7-9]
  \definition{s.}{abertura; um corte ou rachadura}
  \definition{v.+compl.}{abrir a boca; começar a falar; abrir a boca e falar | pedir algo a alguém; fazer um pedido ou uma exigência |  afiar (uma faca nova ou uma tesoura nova) | cavar uma brecha; encontrar uma brecha; abrir uma fenda}
\end{EntryWithPhonetic}

\begin{EntryWithPhonetic}{开阔}{kai1kuo4}{4,12}{⼶,⾨}[HSK 7-9]
  \definition{adj.}{aberto; amplo; largo | de mente aberta}
  \definition{v.}{ampliar; abrir}
\end{EntryWithPhonetic}

\begin{EntryWithPhonetic}{开朗}{kai1lang3}{4,10}{⼶,⽉}[HSK 7-9]
  \definition{adj.}{otimista; alegre; despreocupado; (pensamentos, personalidade e mentalidade) otimista, alegre e não melancólico ou deprimido | espaçoso; bem iluminado; aberto e desimpedido; aberto e luminoso}
\end{EntryWithPhonetic}

\begin{EntryWithPhonetic}{开幕}{kai1 mu4}{4,13}{⼶,⼱}[HSK 5]
  \definition{v.}{começar a apresentação; iniciar o espetáculo; levantar das cortinas | abrir; inaugurar; iniciar (uma conferência, exposição, etc.)}
\end{EntryWithPhonetic}

\begin{EntryWithPhonetic}{开幕式}{kai1mu4shi4}{4,13,6}{⼶,⼱,⼷}[HSK 5]
  \definition[场,次,届]{s.}{cerimônia de abertura; cerimônias e apresentações antes de eventos esportivos ou grandes eventos}
\end{EntryWithPhonetic}

\begin{EntryWithPhonetic}{开辟}{kai1pi4}{4,13}{⼶,⾟}[HSK 7-9]
  \definition{v.}{abrir; talhar; romper; abrir passagem | abrir; iniciar; desenvolver; explorar; ampliar; expandir | abrir; iniciar; fundar; estabelecer; criar}
\end{EntryWithPhonetic}

\begin{EntryWithPhonetic}{开启}{kai1qi3}{4,7}{⼶,⼝}[HSK 7-9]
  \definition{v.}{abrir | iniciar | Computação: ativar}
\end{EntryWithPhonetic}

\begin{EntryWithPhonetic}{开枪}{kai1 qiang1}{4,8}{⼶,⽊}[HSK 7-9]
  \definition{v.}{disparar com um rifle, pistola, etc.; disparar um tiro; atirar}
\end{EntryWithPhonetic}

\begin{EntryWithPhonetic}{开设}{kai1she4}{4,6}{⼶,⾔}[HSK 6]
  \definition{v.}{montar; estabelecer; abrir (uma loja, fábrica, etc.); estabelecer novas instituições ou campos | oferecer (um curso na faculdade, etc.)}
\end{EntryWithPhonetic}

\begin{EntryWithPhonetic}{开始}{kai1shi3}{4,8}{⼶,⼥}[HSK 3]
  \definition[个]{s.}{começo; início; estágio inicial}
  \definition{v.}{começar; iniciar; começar a fazer algo}
\end{EntryWithPhonetic}

\begin{EntryWithPhonetic}{开水}{kai1shui3}{4,4}{⼶,⽔}[HSK 4]
  \definition[杯,瓶]{s.}{água fervida; água fervente}
\end{EntryWithPhonetic}

\begin{EntryWithPhonetic}{开锁}{kai1suo3}{4,12}{⼶,⾦}
  \definition{v.}{desbloquear | destravar}
\end{EntryWithPhonetic}

\begin{EntryWithPhonetic}{开天辟地}{kai1tian1-pi4di4}{4,4,13,6}{⼶,⼤,⾟,⼟}[HSK 7-9]
  \definition{expr.}{``Criação do Céu e da Terra.''; a criação do céu e da terra; quando o céu se separou da terra; a criação do mundo; no princípio do céu e da terra (Gênesis) | desde o alvorecer da história; desde o início da história | que marca época; inovador, revolucionário, que inaugura uma nova era}
\end{EntryWithPhonetic}

\begin{EntryWithPhonetic}{开通}{kai1tong1}{4,10}{⼶,⾡}[HSK 6]
  \definition{v.}{limpar; dragar; remover obstáculos de; abrir o canal; desbloquear}
  \seeref{kai1tong5}
\end{EntryWithPhonetic}

\begin{EntryWithPhonetic}{开通}{kai1tong5}{4,10}{⼶,⾡}
  \definition{adj.}{liberal; mente aberta; mente moderna; mente liberal; sábio e sensato; não conservador ou teimoso}
  \seeref{kai1tong1}
\end{EntryWithPhonetic}

\begin{EntryWithPhonetic}{开头}{kai1/tou2}{4,5}{⼶,⼤}[HSK 6]
  \definition{s.}{início; começo; o momento ou estágio do início; antecedente no tempo}
  \definition{v.+compl.}{começar, iniciar; a primeira ocorrência de um evento, ação, fenômeno, etc. | pôr-se a pé; começar}
\end{EntryWithPhonetic}

\begin{EntryWithPhonetic}{开拓}{kai1tuo4}{4,8}{⼶,⼿}[HSK 7-9]
  \definition{v.}{desenvolver; ser pioneiro; abrir caminho; expandir; abrir}
\end{EntryWithPhonetic}

\begin{EntryWithPhonetic}{开玩笑}{kai1 wan2xiao4}{4,8,10}{⼶,⽟,⽵}[HSK 1]
  \definition{v.}{fazer (ou brincar, fazer) uma piada; gracejar; zombar de; provocar; fazer uma brincadeira; zombar de alguém | tratar casualmente; dar pouca importância a; considerar como um assunto insignificante; insignificante | fazer uma brincadeira; pregar uma peça; brincar; em tom de brincadeira}
\end{EntryWithPhonetic}

\begin{EntryWithPhonetic}{开销}{kai1xiao1}{4,12}{⼶,⾦}[HSK 7-9]
  \definition[笔,项]{s.}{despesas; taxas pagas}
  \definition{v.}{pagar despesas (taxas)}
\end{EntryWithPhonetic}

\begin{EntryWithPhonetic}{开心}{kai1/xin1}{4,4}{⼶,⼼}[HSK 2]
  \definition{adj.}{feliz; alegre; exultante; encantado}
  \definition{v.+compl.}{provocar; brincar; tirar sarro de alguém; zombar; divertir-se}
\end{EntryWithPhonetic}

\begin{EntryWithPhonetic}{开学}{kai1 xue2}{4,8}{⼶,⼦}[HSK 2]
  \definition{v.}{iniciar as aulas; iniciar o semestre; começar as aulas}
\end{EntryWithPhonetic}

\begin{EntryWithPhonetic}{开业}{kai1 ye4}{4,5}{⼶,⼀}[HSK 3]
  \definition[场]{v.}{iniciar um negócio; abrir para negócios | abrir um consultório particular}
\end{EntryWithPhonetic}

\begin{EntryWithPhonetic}{开夜车}{kai1/ye4che1}{4,8,4}{⼶,⼣,⾞}[HSK 6]
  \definition{v.+compl.}{``Dirigir à noite.''; ``Conduzir carro à noite.''; trabalhar até tarde da noite; ficar acordado até tarde da noite estudando ou trabalhando para cumprir prazos}
\end{EntryWithPhonetic}

\begin{EntryWithPhonetic}{开展}{kai1zhan3}{4,10}{⼶,⼫}[HSK 3]
  \definition{v.}{lançar; desenvolver | abrir; inaugurar}
\end{EntryWithPhonetic}

\begin{EntryWithPhonetic}{开张}{kai1/zhang1}{4,7}{⼶,⼸}[HSK 7-9]
  \definition{v.+compl.}{estrear; abrir um negócio; começar a fazer negócios; lojas, hotéis, etc., recém-construídos, começam a abrir as portas | fazer a primeira transação do dia; para empresários, isso se refere à primeira transação do dia}
  \antonymref{关张}{guan1zhang1}
\end{EntryWithPhonetic}

\begin{EntryWithPhonetic}{开支}{kai1zhi1}{4,4}{⼶,⽀}[HSK 7-9]
  \definition{v.}{pagar (despesas); gastar | pagar salários; receber o pagamento}
\end{EntryWithPhonetic}

%%%%%%%%%% 凯 %%%%%%%%%%
\subsection*{凯}\addcontentsline{loh}{figure}{凯 \dpy{kai3}}

\begin{EntryWithPhonetic}{凯}{kai3}{8}{⼏}
  \definition*{s.}{Sobrenome: Kai}
  \definition{adj.}{vitorioso; triunfante}
  \definition{s.}{canção da vitória; canção triunfal | vitória; triunfo}
\end{EntryWithPhonetic}

\begin{EntryWithPhonetic}{凯歌}{kai3ge1}{8,14}{⼏,⽋}[HSK 7-9]
  \definition{s.}{canção de triunfo (ou vitória); hino; canções cantadas após uma vitória}
\end{EntryWithPhonetic}

%%%%%%%%%% 楷 %%%%%%%%%%
\subsection*{楷}\addcontentsline{loh}{figure}{楷 \dpy{kai3}}

\begin{EntryWithPhonetic}{楷}{kai3}{13}{⽊}
  \definition*{s.}{Sobrenome: Kai}
  \definition{s.}{modelo; padrão | escrita regular (em caligrafia chinesa)}
\end{EntryWithPhonetic}

\begin{EntryWithPhonetic}{楷模}{kai3mo2}{13,14}{⽊,⽊}[HSK 7-9]
  \definition{s.}{modelo; exemplo a seguir; modelo a ser seguido; exemplos notáveis}
\end{EntryWithPhonetic}

%%%%%%%%%% 刊 %%%%%%%%%%
\subsection*{刊}\addcontentsline{loh}{figure}{刊 \dpy{kan1}}

\begin{EntryWithPhonetic}{刊}{kan1}{5}{⼑}
  \definition{s.}{periódico; publicação (jornais, revistas, etc., excluindo livros) | (geralmente em um jornal) coluna especial}
  \definition{v.}{Literário: cortar; picar | Literário: esculpir; gravar | apagar ou corrigir | imprimir; publicar; publicar em um jornal ou revista}
\end{EntryWithPhonetic}

\begin{EntryWithPhonetic}{刊登}{kan1deng1}{5,12}{⼑,⽨}[HSK 7-9]
  \definition{v.}{lançar; publicar}
\end{EntryWithPhonetic}

\begin{EntryWithPhonetic}{刊物}{kan1wu4}{5,8}{⼑,⽜}[HSK 7-9]
  \definition[份,本,家,期]{s.}{jornal; periódico; publicação; revistas publicadas regularmente ou irregularmente geralmente contêm artigos, imagens, etc.}
\end{EntryWithPhonetic}

%%%%%%%%%% 看 %%%%%%%%%%
\subsection*{看}\addcontentsline{loh}{figure}{看 \dpy{kan1}}

\begin{EntryWithPhonetic}{看}{kan1}{9}{⽬}[HSK 6]
  \definition{v.}{cuidar de; tomar conta de; cuidar de; proteger | manter sob vigilância}
  \seeref{kan4}
\end{EntryWithPhonetic}

\begin{EntryWithPhonetic}{看管}{kan1guan3}{9,14}{⽬,⽵}[HSK 6]
  \definition{v.}{cuidar; atender | guardar; vigiar; ficar de olho em | assumir o comando; estar no comando}
\end{EntryWithPhonetic}

\begin{EntryWithPhonetic}{看护}{kan1hu4}{9,7}{⽬,⼿}[HSK 7-9]
  \definition{s.}{Obsoleto: enfermeira hospitalar}
  \definition{v.}{cuidar; zelar por; tratar de}
\end{EntryWithPhonetic}

%%%%%%%%%% 勘 %%%%%%%%%%
\subsection*{勘}\addcontentsline{loh}{figure}{勘 \dpy{kan1}}

\begin{EntryWithPhonetic}{勘}{kan1}{11}{⼒}
  \definition{v.}{ler e corrigir; conferir | investigar; realizar levantamento | ler e corrigir o texto de; revisar}
\end{EntryWithPhonetic}

\begin{EntryWithPhonetic}{勘探}{kan1tan4}{11,11}{⼒,⼿}[HSK 7-9]
  \definition{v.}{examinar; prospectar; investigar a distribuição de depósitos minerais e determinar a localização, forma, tamanho, regularidade metalogenética, propriedades das rochas e estrutura geológica dos corpos de minério}
\end{EntryWithPhonetic}

%%%%%%%%%% 堪 %%%%%%%%%%
\subsection*{堪}\addcontentsline{loh}{figure}{堪 \dpy{kan1}}

\begin{EntryWithPhonetic}{堪}{kan1}{12}{⼟}
  \definition*{s.}{Sobrenome: Kan}
  \definition{v.}{pode; consegue | suportar; resistir; aguentar}
\end{EntryWithPhonetic}

\begin{EntryWithPhonetic}{堪称}{kan1cheng1}{12,10}{⼟,⽲}[HSK 7-9]
  \definition{v.}{pode ser classificado como; pode ser chamado assim; merece ser chamado assim}
\end{EntryWithPhonetic}

%%%%%%%%%% 侃 %%%%%%%%%%
\subsection*{侃}\addcontentsline{loh}{figure}{侃 \dpy{kan3}}

\begin{EntryWithPhonetic}{侃}{kan3}{8}{⼈}
  \definition{adj.}{íntegro e honesto; reto e franco; direto | amável; agradável | animado; alegre}
  \definition{v.}{Coloquial: bater papo ociosamente; conversar à toa; fofocar | gabar-se | conversar fluentemente}
\end{EntryWithPhonetic}

\begin{EntryWithPhonetic}{侃大山}{kan3 da4shan1}{8,3,3}{⼈,⼤,⼭}[HSK 7-9]
  \definition{v.}{fofocar; dedurar; bater papo; bater um papo; conversar fiado}
\end{EntryWithPhonetic}

%%%%%%%%%% 砍 %%%%%%%%%%
\subsection*{砍}\addcontentsline{loh}{figure}{砍 \dpy{kan3}}

\begin{EntryWithPhonetic}{砍}{kan3}{9}{⽯}[HSK 7-9]
  \definition{v.}{cortar; picar; talhar; desbastar; derrubar; podar | reduzir; diminuir; remover | Dialeto: atirar algo em | Coloquial: conversar à toa; fofocar}
\end{EntryWithPhonetic}

\begin{EntryWithPhonetic}{砍刀}{kan3dao1}{9,2}{⽯,⼑}
  \definition{s.}{facão | machete}
\end{EntryWithPhonetic}

\begin{EntryWithPhonetic}{砍掉}{kan3diao4}{9,11}{⽯,⼿}
  \definition{v.}{amputar}
\end{EntryWithPhonetic}

\begin{EntryWithPhonetic}{砍断}{kan3duan4}{9,11}{⽯,⽄}
  \definition{v.}{cortar}
\end{EntryWithPhonetic}

\begin{EntryWithPhonetic}{砍价}{kan3jia4}{9,6}{⽯,⼈}
  \definition{v.}{barganhar | cortar ou derrubar um preço}
\end{EntryWithPhonetic}

\begin{EntryWithPhonetic}{砍杀}{kan3sha1}{9,6}{⽯,⽊}
  \definition{v.}{atacar com arma branca}
\end{EntryWithPhonetic}

\begin{EntryWithPhonetic}{砍伤}{kan3shang1}{9,6}{⽯,⼈}
  \definition{v.}{ferir com lâmina ou machado}
\end{EntryWithPhonetic}

\begin{EntryWithPhonetic}{砍树}{kan3shu4}{9,9}{⽯,⽊}
  \definition{v.}{derrubar árvores}
\end{EntryWithPhonetic}

\begin{EntryWithPhonetic}{砍死}{kan3si3}{9,6}{⽯,⽍}
  \definition{v.}{matar com um machado}
\end{EntryWithPhonetic}

\begin{EntryWithPhonetic}{砍头}{kan3tou2}{9,5}{⽯,⼤}
  \definition{v.}{decapitar}
\end{EntryWithPhonetic}

%%%%%%%%%% 看 %%%%%%%%%%
\subsection*{看}\addcontentsline{loh}{figure}{看 \dpy{kan4}}

\begin{EntryWithPhonetic}{看}{kan4}{9}{⽬}[HSK 1]
  \definition{interj.}{``Cuidado!'' (para um perigo)}
  \definition{part.}{tentar, usado depois de outros verbos}
  \definition{v.}{ver; olhar para; observar; fazer contato visual com pessoas ou objetos | pensar; considerar; observar; julgar; observar e analisar | visitar; ver; fazer uma visita | olhar para; considerar; tratar | tratar (um paciente ou uma doença) | cuidar | ficar atento; ficar de olho | depender de; ser dependente de | ler}
  \seeref{kan1}
\end{EntryWithPhonetic}

\begin{EntryWithPhonetic}{看病}{kan4/bing4}{9,10}{⽬,⽧}[HSK 1]
  \definition{v.+compl.}{(de um médico) ver um paciente | (de um paciente) ver (consultar) um médico}
\end{EntryWithPhonetic}

\begin{EntryWithPhonetic}{看不起}{kan4bu5qi3}{9,4,10}{⽬,⼀,⾛}[HSK 4]
  \definition{v.}{desprezar; desdenhar; menosprezar; ter desprezo; olhar de cima para baixo}
\end{EntryWithPhonetic}

\begin{EntryWithPhonetic}{看成}{kan4cheng2}{9,6}{⽬,⼽}[HSK 5]
  \definition{v.}{ser capaz de ver ou assistir | tomar como; olhar como; considerar como | tratar como; considerar como; pensar como; ter como}
\end{EntryWithPhonetic}

\begin{EntryWithPhonetic}{看出}{kan4 chu1}{9,5}{⽬,⼐}[HSK 5]
  \definition{v.}{decifrar; ver; sondar; encontrar; discernir; perceber | descobrir; estar ciente de}
\end{EntryWithPhonetic}

\begin{EntryWithPhonetic}{看待}{kan4dai4}{9,9}{⽬,⼻}[HSK 5]
  \definition{v.}{tratar; considerar; olhar com atenção; ter uma certa atitude ou visão em relação a alguém ou alguma coisa}
\end{EntryWithPhonetic}

\begin{EntryWithPhonetic}{看淡}{kan4dan4}{9,11}{⽬,⽔}
  \definition{v.}{considerar sem importância | ser indiferente a (fama, riqueza, etc.) | (de uma economia ou mercado) enfraquecer, ficar mais lento, diminuir a velocidade}
\end{EntryWithPhonetic}

\begin{EntryWithPhonetic}{看到}{kan4 dao4}{9,8}{⽬,⼑}[HSK 1]
  \definition{v.}{ver; avistar}
\end{EntryWithPhonetic}

\begin{EntryWithPhonetic}{看得出}{kan4de5chu1}{9,11,5}{⽬,⼻,⼐}[HSK 7-9]
  \definition{expr.}{ser evidente; poder ver; poder ser visto}
\end{EntryWithPhonetic}

\begin{EntryWithPhonetic}{看得见}{kan4de5jian4}{9,11,4}{⽬,⼻,⾒}[HSK 6]
  \definition{adj.}{perceptível; visível; tangível}
\end{EntryWithPhonetic}

\begin{EntryWithPhonetic}{看得起}{kan4de5qi3}{9,11,10}{⽬,⼻,⾛}[HSK 6]
  \definition{v.}{ter uma boa opinião sobre; pensar muito (ou muito) sobre}
\end{EntryWithPhonetic}

\begin{EntryWithPhonetic}{看法}{kan4fa5}{9,8}{⽬,⽔}[HSK 2]
  \definition[个,种,点]{s.}{opinião; perspectiva; (ponto de) vista; uma maneira de ver uma coisa | opinião desfavorável (ou crítica) sobre alguém}
\end{EntryWithPhonetic}

\begin{EntryWithPhonetic}{看好}{kan4hao3}{9,6}{⽬,⼥}[HSK 6]
  \definition{v.}{elogiar; apreciar; encorajar; acreditar que pessoas ou coisas terão uma boa tendência | estar prestes a surgir uma boa tendência}
\end{EntryWithPhonetic}

\begin{EntryWithPhonetic}{看见}{kan4 jian5}{9,4}{⽬,⾒}[HSK 1]
  \definition{v.}{ver; avistar; ao olhar, descobrir alguém ou algo}
\end{EntryWithPhonetic}

\begin{EntryWithPhonetic}{看来}{kan4lai5}{9,7}{⽬,⽊}[HSK 4]
  \definition{adv.}{parecer; parecer como se (ou embora); refere"-se a um julgamento aproximado; expressa um julgamento por observação}
  \definition{v.}{ser considerado; na visão de alguém; na opinião de alguém; expressar a ideia aproximada que o locutor tem da situação}
\end{EntryWithPhonetic}

\begin{EntryWithPhonetic}{看起来}{kan4qi3lai5}{9,10,7}{⽬,⾛,⽊}[HSK 3]
  \definition{v.}{parecer; aparentar; dar a impressão de (ou como se)}
\end{EntryWithPhonetic}

\begin{EntryWithPhonetic}{看热闹}{kan4 re4nao5}{9,10,8}{⽬,⽕,⾾}[HSK 7-9]
  \definition{v.}{observar a emoção (ou a diversão) | regozijar"-se com (ou sobre); observar de braços cruzados; ficar de fora | observar a cena movimentada; ser um espectador; observar; ver (assistir) a diversão}
\end{EntryWithPhonetic}

\begin{EntryWithPhonetic}{看上去}{kan4shang4qu5}{9,3,5}{⽬,⼀,⼛}[HSK 3]
  \definition{adv.}{parece que}
\end{EntryWithPhonetic}

\begin{EntryWithPhonetic}{看似}{kan4si4}{9,6}{⽬,⼈}[HSK 7-9]
  \definition{v.}{parecer; dar a impressão de ser}
\end{EntryWithPhonetic}

\begin{EntryWithPhonetic}{看台}{kan4tai2}{9,5}{⽬,⼝}[HSK 7-9]
  \definition{s.}{arquibancada | arquibancada para espectadores | terraço | plataforma de visualização}
\end{EntryWithPhonetic}

\begin{EntryWithPhonetic}{看望}{kan4wang5}{9,11}{⽬,⽉}[HSK 4]
  \definition{v.}{ver; visitar; ligar; dar uma olhada; ir até os pais, idosos, professores ou amigos para cumprimentá-los}
\end{EntryWithPhonetic}

\begin{EntryWithPhonetic}{看样子}{kan4 yang4zi5}{9,10,3}{⽬,⽊,⼦}[HSK 7-9]
  \definition{v.}{parecer; aparentar; dar a impressão de ser; estimar com base na situação (frequentemente usado como elemento inserido em uma frase)}
\end{EntryWithPhonetic}

\begin{EntryWithPhonetic}{看中}{kan4/zhong4}{9,4}{⽬,⼁}[HSK 7-9]
  \definition{v.+compl.}{optar por; gostar de; sentir-se satisfeito com}
\end{EntryWithPhonetic}

\begin{EntryWithPhonetic}{看重}{kan4zhong4}{9,9}{⽬,⾥}[HSK 7-9]
  \definition{v.}{ter em alta consideração; considerar importante; atribuir importância a}
\end{EntryWithPhonetic}

\begin{EntryWithPhonetic}{看作}{kan4zuo4}{9,7}{⽬,⼈}[HSK 6]
  \definition{v.}{considerar como; olhar como}
\end{EntryWithPhonetic}

%%%%%%%%%% 康 %%%%%%%%%%
\subsection*{康}\addcontentsline{loh}{figure}{康 \dpy{kang1}}

\begin{EntryWithPhonetic}{康}{kang1}{11}{⼴}
  \definition*{s.}{Sobrenome: Kang}
  \definition{adj.}{saudável |  fácil; pacífico; abundante | amplo; largo | Dialeto: de baixa qualidade; inferior}
  \definition{s.}{bem-estar; saúde | palha; farelo; casca}
  \definition{v.}{(normalmente de um rabanete) tornar-se esponjoso}
\end{EntryWithPhonetic}

\begin{EntryWithPhonetic}{康复}{kang1fu4}{11,9}{⼴,⼢}[HSK 6]
  \definition{v.}{Saúde: estaurar; recuperar; reabilitar}
\end{EntryWithPhonetic}

%%%%%%%%%% 慷 %%%%%%%%%%
\subsection*{慷}\addcontentsline{loh}{figure}{慷 \dpy{kang1}}

\begin{EntryWithPhonetic}{慷}{kang1}{14}{⼼}
  \definition{adj.}{generoso | magnânimo}
\end{EntryWithPhonetic}

\begin{EntryWithPhonetic}{慷慨}{kang1kai3}{14,12}{⼼,⼼}[HSK 7-9]
  \definition{adj.}{generoso; descreve alguém como alguém que não é mesquinho; disposto a ajudar os outros com dinheiro ou bens | veemente; fervoroso; apaixonado; descreve alguém como alguém repleto de senso de justiça e emocionalmente intenso}
\end{EntryWithPhonetic}

%%%%%%%%%% 扛 %%%%%%%%%%
\subsection*{扛}\addcontentsline{loh}{figure}{扛 \dpy{kang2}}

\begin{EntryWithPhonetic}{扛}{kang2}{6}{⼿}[HSK 7-9]
  \definition{v.}{carregar objetos nos ombros |  suportar; aguentar | lidar; assumir}
  \seeref{gang1}
\end{EntryWithPhonetic}

%%%%%%%%%% 抗 %%%%%%%%%%
\subsection*{抗}\addcontentsline{loh}{figure}{抗 \dpy{kang4}}

\begin{EntryWithPhonetic}{抗}{kang4}{7}{⼿}
  \definition*{s.}{Sobrenome: Kang}
  \definition{pref.}{anti-}
  \definition{v.}{resistir; combater; lutar | recusar; desafiar}
\end{EntryWithPhonetic}

\begin{EntryWithPhonetic}{抗衡}{kang4heng2}{7,16}{⼿,⾏}[HSK 7-9]
  \definition{v.}{competir com; desafiar; rivalizar | servir de contrapeso a; competir com; rivalizar com; estar em pé de igualdade com}
\end{EntryWithPhonetic}

\begin{EntryWithPhonetic}{抗拒}{kang4ju4}{7,7}{⼿,⼿}[HSK 7-9]
  \definition{v.}{resistir; suportar}
\end{EntryWithPhonetic}

\begin{EntryWithPhonetic}{抗生素}{kang4sheng1su4}{7,5,10}{⼿,⽣,⽷}[HSK 7-9]
  \definition{s.}{antibiótico}
\end{EntryWithPhonetic}

\begin{EntryWithPhonetic}{抗议}{kang4yi4}{7,5}{⼿,⾔}[HSK 6]
  \definition{v.}{protestar; reconsiderar; levantar objeções fortes}
\end{EntryWithPhonetic}

\begin{EntryWithPhonetic}{抗争}{kang4zheng1}{7,6}{⼿,⼑}[HSK 7-9]
  \definition{v.}{resistir; opor-se; posicionar-se contra; confrontar; lutar}
\end{EntryWithPhonetic}

%%%%%%%%%% 考 %%%%%%%%%%
\subsection*{考}\addcontentsline{loh}{figure}{考 \dpy{kao3}}

\begin{EntryWithPhonetic}{考}{kao3}{6}{⽼}[HSK 1]
  \definition*{s.}{Sobrenome: Kao}
  \definition{adj.}{antigo; velho; com idade avançada}
  \definition{s.}{o pai falecido de alguém}
  \definition{v.}{examinar; dar (fazer) um exame, teste ou questionário | verificar; inspecionar | estudar; verificar; investigar | perguntar; testar; fazer perguntas para que o outro responda, a fim de testar suas habilidades em determinada área}
\end{EntryWithPhonetic}

\begin{EntryWithPhonetic}{考察}{kao3cha2}{6,14}{⽼,⼧}[HSK 4]
  \definition{v.}{inspecionar; investigar; observar e estudar}
\end{EntryWithPhonetic}

\begin{EntryWithPhonetic}{考场}{kao3chang3}{6,6}{⽼,⼟}[HSK 6]
  \definition{s.}{sala de exames}
\end{EntryWithPhonetic}

\begin{EntryWithPhonetic}{考核}{kao3he2}{6,10}{⽼,⽊}[HSK 5]
  \definition{v.}{examinar; checar; avaliar; avaliar (a proficiência de alguém)}
\end{EntryWithPhonetic}

\begin{EntryWithPhonetic}{考量}{kao3liang2}{6,12}{⽼,⾥}[HSK 7-9]
  \definition{v.}{considerar; examinar e medir}
\end{EntryWithPhonetic}

\begin{EntryWithPhonetic}{考虑}{kao3lv4}{6,10}{⽼,⾌}[HSK 4]
  \definition{v.}{considerar; refletir sobre; levar em conta}
\end{EntryWithPhonetic}

\begin{EntryWithPhonetic}{考生}{kao3sheng1}{6,5}{⽼,⽣}[HSK 2]
  \definition{s.}{candidato a exame; alunos inscritos para o exame de admissão}
\end{EntryWithPhonetic}

\begin{EntryWithPhonetic}{考试}{kao3/shi4}{6,8}{⽼,⾔}[HSK 1]
  \definition[次]{s.}{teste; exame; prova; atividades realizadas para verificar conhecimentos ou habilidades}
  \definition{v.+compl.}{testar; avaliar; avaliar conhecimentos e habilidades por meio de perguntas escritas ou orais.}
\end{EntryWithPhonetic}

\begin{EntryWithPhonetic}{考题}{kao3ti2}{6,15}{⽼,⾴}[HSK 6]
  \definition{s.}{questões de exame; prova de exame; tópicos de exame}
\end{EntryWithPhonetic}

\begin{EntryWithPhonetic}{考验}{kao3yan4}{6,10}{⽼,⾺}[HSK 3]
  \definition[场,个,种]{s.}{teste; julgamento; atividade realizada para verificar se as habilidades, ideias, moral e qualidades de uma pessoa atendem aos requisitos}
  \definition{v.}{testar; testar as capacidades, ideias, moral e qualidades de uma pessoa através de situações, ações ou ambientes difíceis, para verificar se elas atendem aos requisitos}
\end{EntryWithPhonetic}

%%%%%%%%%% 拷 %%%%%%%%%%
\subsection*{拷}\addcontentsline{loh}{figure}{拷 \dpy{kao3}}

\begin{EntryWithPhonetic}{拷}{kao3}{9}{⼿}
  \definition{v.}{açoitar; bater; torturar | copiar | Dialeto, Empréstimo linguístico: \emph{call}, ligar | vencer | interrogar sob tortura}
\end{EntryWithPhonetic}

%%%%%%%%%% 烤 %%%%%%%%%%
\subsection*{烤}\addcontentsline{loh}{figure}{烤 \dpy{kao3}}

\begin{EntryWithPhonetic}{烤}{kao3}{10}{⽕}
  \definition{v.}{assar | grelhar}
\end{EntryWithPhonetic}

\begin{EntryWithPhonetic}{烤肉}{kao3rou4}{10,6}{⽕,⾁}[HSK 5]
  \definition[块,串,片,盘]{s.}{churrasco (literalmente carne assada)}
\end{EntryWithPhonetic}

\begin{EntryWithPhonetic}{烤鸭}{kao3ya1}{10,10}{⽕,⿃}[HSK 5]
  \definition[只,盘]{s.}{pato assado; pato recheado e assado em um forno especial após ser abatido}
\end{EntryWithPhonetic}

%%%%%%%%%% 靠 %%%%%%%%%%
\subsection*{靠}\addcontentsline{loh}{figure}{靠 \dpy{kao4}}

\begin{EntryWithPhonetic}{靠}{kao4}{15}{⾮}[HSK 2]
  \definition{prep.}{manter (em); aproximar"-se (de); ao longo de | por; graças a; com base em; de acordo com}
  \definition{s.}{armadura de palco (feita de seda bordada); armadura usada pelos generais militares antigos nas peças teatrais}
  \definition{v.}{inclinar"-se; sentado ou em pé, deixar parte do peso do corpo ser suportado por outra pessoa ou objeto (pessoa) | encostar"-se (em); apoiar"-se ou levantar-se com a ajuda de alguma coisa | aproximar"-se; estar perto de | confiar em; depender de | confiar}
\end{EntryWithPhonetic}

\begin{EntryWithPhonetic}{靠近}{kao4jin4}{15,7}{⾮,⾡}[HSK 5]
  \definition{adv.}{próximo; perto de; ao lado de}
  \definition{v.}{aproximar"-se; chegar perto; avançar em direção a um determinado objetivo de modo que a distância fique cada vez menor}
\end{EntryWithPhonetic}

\begin{EntryWithPhonetic}{靠拢}{kao4long3}{15,8}{⾮,⼿}[HSK 7-9]
  \definition{v.}{aproximar-se de; encostar-se; reunir-se; aconchegar-se}
\end{EntryWithPhonetic}

%%%%%%%%%% 苛 %%%%%%%%%%
\subsection*{苛}\addcontentsline{loh}{figure}{苛 \dpy{ke1}}

\begin{EntryWithPhonetic}{苛}{ke1}{8}{⾋}
  \definition{adj.}{duro; severo; exigente; opressivo | excessivamente elaborado; exorbitante; diverso; variado}
\end{EntryWithPhonetic}

\begin{EntryWithPhonetic}{苛刻}{ke1ke4}{8,8}{⾋,⼑}[HSK 7-9]
  \definition{adj.}{severo; rigoroso; os requisitos são muito rigorosos ou as condições são muito elevadas}
\end{EntryWithPhonetic}

%%%%%%%%%% 科 %%%%%%%%%%
\subsection*{科}\addcontentsline{loh}{figure}{科 \dpy{ke1}}

\begin{EntryWithPhonetic}{科}{ke1}{9}{⽲}[HSK 2]
  \definition*{s.}{Sobrenome: Ke}
  \definition{s.}{um ramo de estudo acadêmico ou profissional |uma divisão ou subdivisão de uma unidade administrativa | família | instruções de palco no drama chinês clássico; nos roteiros de peças clássicas, termos usados para indicar as ações dos personagens | nível; classificação; categoria | sessão de exames; refere"-se às disciplinas, notas e anos das provas para a seleção de candidatos a cargos públicos militares e civis na antiguidade | tecnológico | assunto | lei; regulamento; decreto | penalidade; pena; punição | treinamento profissional ou formal; curso profissionalizante}
  \definition{v.}{proferir uma sentença (penal)}
\end{EntryWithPhonetic}

\begin{EntryWithPhonetic}{科幻}{ke1huan4}{9,4}{⽲,⼳}[HSK 7-9]
  \definition[个]{s.}{ficção científica}
\end{EntryWithPhonetic}

\begin{EntryWithPhonetic}{科技}{ke1ji4}{9,7}{⽲,⼿}[HSK 3]
  \definition{s.}{ciência e tecnologia}
\end{EntryWithPhonetic}

\begin{EntryWithPhonetic}{科目}{ke1mu4}{9,5}{⽲,⽬}[HSK 7-9]
  \definition[个]{s.}{curso; disciplina (em um currículo); categorias como disciplinas acadêmicas, classificadas de acordo com suas diferentes naturezas | cabeçalhos em um livro de contas; registros contábeis}
\end{EntryWithPhonetic}

\begin{EntryWithPhonetic}{科普}{ke1pu3}{9,12}{⽲,⽇}[HSK 7-9]
  \definition{v.}{popularizar a ciência}
\end{EntryWithPhonetic}

\begin{EntryWithPhonetic}{科学}{ke1xue2}{9,8}{⽲,⼦}[HSK 2]
  \definition{adj.}{científico; em conformidade com as leis da ciência}
  \definition[门,个,种]{s.}{ciência; um conjunto de conhecimentos que reflete as leis objetivas da natureza, da sociedade, do pensamento, etc.}
\end{EntryWithPhonetic}

\begin{EntryWithPhonetic}{科学家}{ke1xue2jia1}{9,8,10}{⽲,⼦,⼧}
  \definition[位,名,个]{s.}{cientista; pessoas com realizações significativas no campo da pesquisa científica}
\end{EntryWithPhonetic}

\begin{EntryWithPhonetic}{科研}{ke1yan2}{9,9}{⽲,⽯}[HSK 6]
  \definition{s.}{pesquisa científica}
  \definition{v.}{envolver-se em pesquisa científica}
\end{EntryWithPhonetic}

%%%%%%%%%% 棵 %%%%%%%%%%
\subsection*{棵}\addcontentsline{loh}{figure}{棵 \dpy{ke1}}

\begin{EntryWithPhonetic}{棵}{ke1}{12}{⽊}[HSK 4]
  \definition{clas.}{usado para plantas, árvores}
\end{EntryWithPhonetic}

%%%%%%%%%% 颗 %%%%%%%%%%
\subsection*{颗}\addcontentsline{loh}{figure}{颗 \dpy{ke1}}

\begin{EntryWithPhonetic}{颗}{ke1}{14}{⾴}[HSK 5]
  \definition{clas.}{usado para grãos, pérolas, dentes, corações, satelites, pequenas esferas, etc.}
  \definition{s.}{grão; partícula; pequenas coisas redondas}
\end{EntryWithPhonetic}

%%%%%%%%%% 磕 %%%%%%%%%%
\subsection*{磕}\addcontentsline{loh}{figure}{磕 \dpy{ke1}}

\begin{EntryWithPhonetic}{磕}{ke1}{15}{⽯}[HSK 7-9]
  \definition{v.}{bater (com força em algo); bater em algo duro | derrubar algo de um recipiente, vaso, etc.}
\end{EntryWithPhonetic}

%%%%%%%%%% 蝌 %%%%%%%%%%
\subsection*{蝌}\addcontentsline{loh}{figure}{蝌 \dpy{ke1}}

\begin{EntryWithPhonetic}{蝌}{ke1}{15}{⾍}
  \definition[只]{s.}{girino}
\end{EntryWithPhonetic}

\begin{EntryWithPhonetic}{蝌蚪}{ke1dou3}{15,10}{⾍,⾍}
  \definition{s.}{girino}
\end{EntryWithPhonetic}

%%%%%%%%%% 壳 %%%%%%%%%%
\subsection*{壳}\addcontentsline{loh}{figure}{壳 \dpy{ke2}}

\begin{EntryWithPhonetic}{壳}{ke2}{7}{⼠}[HSK 7-9]
  \definition[层,个]{s.}{casca, concha; significa o mesmo que 壳 | concha; revestimento externo | empresa de fachada}
  \seeref{qiao4}
  \seealsoref{壳儿}{ke2r5}
\end{EntryWithPhonetic}

\begin{EntryWithPhonetic}{壳儿}{ke2r5}{7,2}{⼠,⼉}
  \definition{s.}{crosta | concha}
  \seealsoref{壳}{ke2}
\end{EntryWithPhonetic}

%%%%%%%%%% 咳 %%%%%%%%%%
\subsection*{咳}\addcontentsline{loh}{figure}{咳 \dpy{ke2}}

\begin{EntryWithPhonetic}{咳}{ke2}{9}{⼝}[HSK 5]
  \definition{v.}{tossir}
  \seeref{hai1}
\end{EntryWithPhonetic}

\begin{EntryWithPhonetic}{咳嗽}{ke2sou5}{9,14}{⼝,⼝}[HSK 7-9]
  \definition[次,声,阵,回]{s.}{tosse}
  \definition{v.}{ter tosse; tossir}
\end{EntryWithPhonetic}

%%%%%%%%%% 可 %%%%%%%%%%
\subsection*{可}\addcontentsline{loh}{figure}{可 \dpy{ke3}}

\begin{EntryWithPhonetic}{可}{ke3}{5}{⼝}[HSK 5]
  \definition*{s.}{Sobrenome: Ke}
  \definition{adv.}{indica ênfase | indica o fortalecimento de perguntas retóricas | indica um tom de questionamento mais forte | sobre; a respeito de}
  \definition{conj.}{mas; ainda}
  \definition{v.}{aprovar; concordar com | poder; permitir; ser capaz de | precisar (fazer); valer a pena (fazer); merecer | ajustar; adequar | estar pronto para; estar disposto a; pretender}
  \seeref{ke4}
\end{EntryWithPhonetic}

\begin{EntryWithPhonetic}{可爱}{ke3'ai4}{5,10}{⼝,⽖}[HSK 2]
  \definition{adj.}{adorável; simpático; encantador | bonitinho; adorável | amado; querido; encantador; cativante; relacionamento próximo, sentimentos profundos | fofo; bonito}
\end{EntryWithPhonetic}

\begin{EntryWithPhonetic}{可悲}{ke3bei1}{5,12}{⼝,⽕}[HSK 7-9]
  \definition{adj.}{triste; lamentável; deplorável; de partir o coração}
\end{EntryWithPhonetic}

\begin{EntryWithPhonetic}{可编程}{ke3bian1cheng2}{5,12,12}{⼝,⽷,⽲}
  \definition{adj.}{programável}
\end{EntryWithPhonetic}

\begin{EntryWithPhonetic}{可不是}{ke3bu2shi4}{5,4,9}{⼝,⼀,⽇}[HSK 7-9]
  \definition{adv.}{certamente; exatamente; é assim mesmo; para expressar concordância, é frequentemente usado como uma frase independente, mas também pode ser expresso como 可不}
  \seealsoref{可不}{ke3bu4}
\end{EntryWithPhonetic}

\begin{EntryWithPhonetic}{可不}{ke3bu4}{5,4}{⼝,⼀}
  \definition{adv.}{certamente; exatamente; em uma conversa, significa concordar plenamente com o que a outra pessoa diz}
  \seealsoref{可不是}{ke3bu2shi4}
\end{EntryWithPhonetic}

\begin{EntryWithPhonetic}{可擦写可编程只读存储器}{ke3 ca1 xie3 ke3 bian1cheng2 zhi1 du2 cun2chu3qi4}{5,17,5,5,12,12,5,10,6,12,16}{⼝,⼿,⼍,⼝,⽷,⽲,⼝,⾔,⼦,⼈,⼝}
  \definition{s.}{EPROM (\emph{erasable programmable read-only memory})}
\end{EntryWithPhonetic}

\begin{EntryWithPhonetic}{可乘之机}{ke3cheng2zhi1ji1}{5,10,3,6}{⼝,⽲,⼂,⽊}[HSK 7-9]
  \definition{expr.}{uma oportunidade para aproveitar; oportunidade a explorar; abertura}
\end{EntryWithPhonetic}

\begin{EntryWithPhonetic}{可耻}{ke3chi3}{5,10}{⼝,⽿}[HSK 7-9]
  \definition{adj.}{vergonhoso; ignominioso; desonroso}
\end{EntryWithPhonetic}

\begin{EntryWithPhonetic}{可歌可泣}{ke3ge1-ke3qi4}{5,14,5,8}{⼝,⽋,⼝,⽔}[HSK 7-9]
  \definition{expr.}{``Uma história digna de elogios e lágrimas.''; capaz de evocar elogios e lágrimas; tanto alegre quanto trágico; comovente e digno de uma canção; tocante e digno de uma canção; digno de elogios, comovente até às lágrimas refere"-se a feitos trágicos e heroicos que tocam profundamente as pessoas; digno de ser louvado e de comover até às lágrimas}
\end{EntryWithPhonetic}

\begin{EntryWithPhonetic}{可观}{ke3guan1}{5,6}{⼝,⾒}[HSK 7-9]
  \definition{adj.}{que vale a pena ver; que vale a pena assistir | considerável; impressionante; substancial; de nível alcançado; de grau relativamente alto}
\end{EntryWithPhonetic}

\begin{EntryWithPhonetic}{可贵}{ke3gui4}{5,9}{⼝,⾙}[HSK 7-9]
  \definition{adj.}{estimado; valioso; valorizado; louvável; recomendável}
\end{EntryWithPhonetic}

\begin{EntryWithPhonetic}{可见}{ke3jian4}{5,4}{⼝,⾒}[HSK 4]
  \definition{adj.}{visível; concebível; algo que é óbvio ou evidente}
  \definition{conj.}{isso mostra; isto prova; é, portanto, claro (ou evidente, óbvio) que}
  \definition{v.}{ser ou estar visível ; ser ou estar claro}
\end{EntryWithPhonetic}

\begin{EntryWithPhonetic}{可卡因}{ke3ka3yin1}{5,5,6}{⼝,⼘,⼞}
  \definition{s.}{Empréstimo linguístico: cocaína; eritroxilon}
\end{EntryWithPhonetic}

\begin{EntryWithPhonetic}{可靠}{ke3kao4}{5,15}{⼝,⾮}[HSK 3]
  \definition{adj.}{confiável; digno de confiança | verdadeiro; autêntico; descrever notícias e outras informações como verdadeiras, de modo que as pessoas possam acreditar nelas}
\end{EntryWithPhonetic}

\begin{EntryWithPhonetic}{可口}{ke3kou3}{5,3}{⼝,⼝}[HSK 7-9]
  \definition{adj.}{bom; saboroso; apetitoso; gostoso de comer; (alimento ou bebida) tem uma textura agradável e um sabor bom}
\end{EntryWithPhonetic}

\begin{EntryWithPhonetic}{可口可乐}{ke3kou3ke3le4}{5,3,5,5}{⼝,⼝,⼝,⼃}
  \definition*{s.}{Empréstimo linguístico: Coca-Cola}
\end{EntryWithPhonetic}

\begin{EntryWithPhonetic}{可乐}{ke3le4}{5,5}{⼝,⼃}[HSK 3]
  \definition*[罐,杯,瓶,听,口]{s.}{\emph{coke}; coca; coca-cola}
  \definition{adj.}{engraçado; divertido; risível}
\end{EntryWithPhonetic}

\begin{EntryWithPhonetic}{可怜}{ke3lian2}{5,8}{⼝,⼼}[HSK 5]
  \definition{adj.}{pobre; lamentável; lastimável | miserável (de quantidade ou qualidade); descreve um número pequeno ou um lugar tão pequeno que não vale a pena falar sobre ele}
  \definition{v.}{ter pena; ter piedade de; ter simpatia por pessoas que tiveram coisas muito ruins acontecendo com elas}
\end{EntryWithPhonetic}

\begin{EntryWithPhonetic}{可能}{ke3neng2}{5,10}{⼝,⾁}[HSK 2]
  \definition{adj.}{possível}
  \definition{adv.}{possivelmente}
  \definition[种]{s.}{possibilidade; tendências ou oportunidades que podem se tornar realidade}
\end{EntryWithPhonetic}

\begin{EntryWithPhonetic}{可怕}{ke3pa4}{5,8}{⼝,⼼}[HSK 2]
  \definition{adj.}{assustador; terrível; hediondo; medonho; horrível; aterrorizante}
  \definition{adv.}{terrivelmente}
\end{EntryWithPhonetic}

\begin{EntryWithPhonetic}{可是}{ke3shi4}{5,9}{⼝,⽇}[HSK 2]
  \definition{adv.}{de fato (usado para dar ênfase), equivalente a 的确}
  \definition{conj.}{mas; no entanto; contudo; conecta frases, expressa uma relação de transição, equivalente a 但是}
  \seealsoref{但是}{dan4shi4}
  \seealsoref{的确}{di2que4}
\end{EntryWithPhonetic}

\begin{EntryWithPhonetic}{可谓}{ke3wei4}{5,11}{⼝,⾔}[HSK 7-9]
  \definition{s.}{poder ser dito; poder ser considerado; poder ser chamado de}[这机会可谓是千载难逢。===Pode-se dizer que esta é a oportunidade da sua vida.]
\end{EntryWithPhonetic}

\begin{EntryWithPhonetic}{可恶}{ke3wu4}{5,10}{⼝,⼼}[HSK 7-9]
  \definition{adj.}{odioso; abominável; detestável; repugnante; extremamente irritante}
\end{EntryWithPhonetic}

\begin{EntryWithPhonetic}{可惜}{ke3xi1}{5,11}{⼝,⼼}[HSK 5]
  \definition{adj.}{é uma pena; é muito ruim; é lamentável}
  \definition{adv.}{infelizmente}
\end{EntryWithPhonetic}

\begin{EntryWithPhonetic}{可想而知}{ke3xiang3'er2zhi1}{5,13,6,8}{⼝,⼼,⽽,⽮}[HSK 7-9]
  \definition{expr.}{é fácil de imaginar; é óbvio; claramente; como você pode imaginar; você pode imaginar isso sem precisar de explicações; pode-se muito bem imaginar}
\end{EntryWithPhonetic}

\begin{EntryWithPhonetic}{可笑}{ke3xiao4}{5,10}{⼝,⽵}[HSK 7-9]
  \definition{adj.}{absurdo; risível; ridículo; hilário; engraçado}
\end{EntryWithPhonetic}

\begin{EntryWithPhonetic}{可信}{ke3xin4}{5,9}{⼝,⼈}[HSK 7-9]
  \definition{adj.}{confiável; em que (quem) se pode acreditar}
\end{EntryWithPhonetic}

\begin{EntryWithPhonetic}{可行}{ke3xing2}{5,6}{⼝,⾏}[HSK 7-9]
  \definition{adj.}{viável; praticável; funcional}
\end{EntryWithPhonetic}

\begin{EntryWithPhonetic}{可疑}{ke3yi2}{5,14}{⼝,⽦}[HSK 7-9]
  \definition{adj.}{duvidoso; suspeito; questionável}
\end{EntryWithPhonetic}

\begin{EntryWithPhonetic}{可以}{ke3yi3}{5,4}{⼝,⼈}[HSK 2]
  \definition{adj.}{aceitável; nada mal; muito bom | impressionante; espantoso; tremendo}
  \definition{v.}{poder; ter condições, capacidade e tempo para fazer algo ou ter alguma utilidade | permitir; poder | valer a pena fazer; considerar que vale a pena, recomendar fazer algo}
\end{EntryWithPhonetic}

%%%%%%%%%% 渴 %%%%%%%%%%
\subsection*{渴}\addcontentsline{loh}{figure}{渴 \dpy{ke3}}

\begin{EntryWithPhonetic}{渴}{ke3}{12}{⽔}[HSK 1]
  \definition{adj.}{sedento}
  \definition{adv.}{ansiosamente; metáfora de urgência}
  \definition{v.}{desejar; ansiar por}
\end{EntryWithPhonetic}

\begin{EntryWithPhonetic}{渴望}{ke3wang4}{12,11}{⽔,⽉}[HSK 5]
  \definition{v.}{aspirar; (ter sede, ansiar, desejar) por}
\end{EntryWithPhonetic}

%%%%%%%%%% 可 %%%%%%%%%%
\subsection*{可}\addcontentsline{loh}{figure}{可 \dpy{ke4}}

\begin{EntryWithPhonetic}{可}{ke4}{5}{⼝}
  \definition{s.}{governante supremo de uma tribo nômade do norte; Khan (可汗), título do governante supremo dos antigos grupos étnicos xianbei, turco, uigur e mongol}
  \seeref{ke3}
  \seealsoref{可汗}{ke4han2}
\end{EntryWithPhonetic}

\begin{EntryWithPhonetic}{可汗}{ke4han2}{5,6}{⼝,⽔}
  \definition{s.}{khan (empréstimo linguístico); cham}
\end{EntryWithPhonetic}

%%%%%%%%%% 克 %%%%%%%%%%
\subsection*{克}\addcontentsline{loh}{figure}{克 \dpy{ke4}}

\begin{EntryWithPhonetic}{克}{ke4}{7}{⼗}[HSK 2]
  \definition*{s.}{Sobrenome: Ke}
  \definition{clas.}{g, grama, unidade de peso | unidade tibetana de volume ou medida seca (com capacidade para cerca de 25 斤, de cevada) | unidade tibetana de área de terra equivalente a cerca de 1 亩}
  \definition{v.}{poder; ser capaz de | tolerar; conter; restringir; suprimir| subjugar; capturar; conquistar (uma cidade, etc.) | digerir (alimentos) | reduzir; diminuir | definir um limite de tempo}
  \seealsoref{斤}{jin1}
  \seealsoref{亩}{mu3}
\end{EntryWithPhonetic}

\begin{EntryWithPhonetic}{克服}{ke4fu2}{7,8}{⼗,⽉}[HSK 3]
  \definition{v.}{sobrepujar; superar; conquistar; vencer com força de vontade e determinação (deficiências, erros, fenômenos negativos, condições desfavoráveis, etc.) | aguentar; suportar (dificuldades, inconveniências, etc.)}
\end{EntryWithPhonetic}

\begin{EntryWithPhonetic}{克隆}{ke4long2}{7,11}{⼗,⾩}[HSK 7-9]
  \definition{v.}{clonar; geralmente, refere"-se à reprodução assexuada induzida artificialmente | copiar; imitar; metaforicamente, significa copiar (frequentemente usado em um sentido humorístico ou depreciativo)}
\end{EntryWithPhonetic}

\begin{EntryWithPhonetic}{克制}{ke4zhi4}{7,8}{⼗,⼑}[HSK 7-9]
  \definition{v.}{restringir; exercer contenção; exercer autocontrole nas emoções, palavras e ações}
\end{EntryWithPhonetic}

%%%%%%%%%% 刻 %%%%%%%%%%
\subsection*{刻}\addcontentsline{loh}{figure}{刻 \dpy{ke4}}

\begin{EntryWithPhonetic}{刻}{ke4}{8}{⼑}[HSK 2,5]
  \definition{adj.}{cruel; severo; rude; indelicado | no mais alto grau}
  \definition{clas.}{um quarto (de uma hora, 15min)}
  \definition[件]{s.}{quarto (de hora); momento}
  \definition{v.}{esculpir; inscrever; gravar; talhar com uma faca (padrões, texto, etc.) | definir um limite de tempo | imprimir (CD)}
\end{EntryWithPhonetic}

\begin{EntryWithPhonetic}{刻画}{ke4hua4}{8,8}{⼑,⽥}
  \definition{v.}{retratar | tirar um retrato}
\end{EntryWithPhonetic}

\begin{EntryWithPhonetic}{刻苦}{ke4ku3}{8,8}{⼑,⾋}[HSK 7-9]
  \definition{adj.}{assíduo; trabalhador; meticuloso; diligente e trabalhador, capaz de se dedicar ao trabalho árduo | simples e econômico}
\end{EntryWithPhonetic}

\begin{EntryWithPhonetic}{刻意}{ke4yi4}{8,13}{⼑,⼼}[HSK 7-9]
  \definition{adv.}{diligentemente; assiduamente; fazendo tudo o que se pode; estar completamente absorto; dedicar-se inteiramente a algo; isso enfatiza ações tomadas para atrair a atenção dos outros}
\end{EntryWithPhonetic}

\begin{EntryWithPhonetic}{刻钟}{ke4 zhong1}{8,9}{⼑,⾦}
  \definition{s.}{um quarto de hora}
\end{EntryWithPhonetic}

\begin{EntryWithPhonetic}{刻舟求剑}{ke4zhou1-qiu2jian4}{8,6,7,9}{⼑,⾈,⽔,⼑}[HSK 7-9]
  \definition{expr.}{``Marcando o barco para encontrar a espada.''; um entalhe na lateral de um barco para localizar uma espada que caiu ao mar; Figurativo: uma ação que se torna inútil devido a circunstâncias alteradas; tomar medidas sem levar em conta mudanças nas circunstâncias; ``Um homem do estado de Chu deixou cair sua espada no rio enquanto o atravessava. Ele marcou o local onde a espada havia caído na lateral do barco. Quando o barco parou, ele entrou na água a partir do ponto marcado para procurar sua espada, mas, naturalmente, não a encontrou.'' de Lüshi Chunqiu (吕氏春秋), Observando o Presente (察今)}
\end{EntryWithPhonetic}

%%%%%%%%%% 客 %%%%%%%%%%
\subsection*{客}\addcontentsline{loh}{figure}{客 \dpy{ke4}}

\begin{EntryWithPhonetic}{客}{ke4}{9}{⼧}
  \definition*{s.}{Sobrenome: Ke}
  \definition{adj.}{objetivo; independente da consciência humana | estrangeiro; não desta região, unidade ou indústria}
  \definition{clas.}{porção (de comida, bebida, etc.); em algumas áreas, é usado para vender alimentos e bebidas em porções}
  \definition[个,位,名,些]{s.}{convidado; visitante; aquele que é convidado; aquele que vem visitar | viajante; passageiro | comerciante viajante; refere"-se especificamente a comerciantes que transportam mercadorias de um lugar para o outro | cliente; patrono; consumidor | uma pessoa envolvida em alguma atividade específica; pessoas que viajam fazendo algum tipo de atividade}
  \definition{v.}{ser um estranho; estabelecer-se (ou viver) em um lugar estranho; estar longe de casa ou morar no exterior}
  \antonymref{主}{zhu3}
\end{EntryWithPhonetic}

\begin{EntryWithPhonetic}{客车}{ke4che1}{9,4}{⼧,⾞}[HSK 6]
  \definition[辆,列,次,趟]{s.}{ônibus; veículo de passageiros; veículos que transportam passageiros em ferrovias e estradas}
\end{EntryWithPhonetic}

\begin{EntryWithPhonetic}{客房}{ke4fang2}{9,8}{⼧,⼾}[HSK 7-9]
  \definition{s.}{quarto de hóspedes; quartos para viajantes ou hóspedes}
\end{EntryWithPhonetic}

\begin{EntryWithPhonetic}{客观}{ke4guan1}{9,6}{⼧,⾒}[HSK 3]
  \definition{adj.}{objetivo; justo e razoável; imparcial; com base na situação real, sem preconceitos pessoais}
  \definition{s.}{objetivo; existe fora da consciência, sem depender da consciência subjetiva}
\end{EntryWithPhonetic}

\begin{EntryWithPhonetic}{客户}{ke4hu4}{9,4}{⼧,⼾}[HSK 5]
  \definition[位,个,家,批]{s.}{cliente; consumidor}
\end{EntryWithPhonetic}

\begin{EntryWithPhonetic}{客机}{ke4ji1}{9,6}{⼧,⽊}[HSK 7-9]
  \definition[架]{s.}{avião de passageiros; avião comercial}
  \antonymref{货机}{huo4ji1}
\end{EntryWithPhonetic}

\begin{EntryWithPhonetic}{客流}{ke4liu2}{9,10}{⼧,⽔}[HSK 7-9]
  \definition{s.}{fluxo de passageiros; o setor de transportes refere"-se ao fluxo de passageiros em uma determinada direção dentro de um determinado período de tempo | fluxo de clientes (frequentando uma loja, etc.)}
\end{EntryWithPhonetic}

\begin{EntryWithPhonetic}{客气}{ke4qi5}{9,4}{⼧,⽓}[HSK 5]
  \definition{adj.}{educado; modesto; cortês}
  \definition{v.}{ser educado; ser cortês; fazer comentários educados ou agir educadamente}
\end{EntryWithPhonetic}

\begin{EntryWithPhonetic}{客人}{ke4ren5}{9,2}{⼧,⼈}[HSK 2]
  \definition[位,个,桌,拨,批]{s.}{visitante; convidado | cliente; passageiro; hóspede; viajante}
\end{EntryWithPhonetic}

\begin{EntryWithPhonetic}{客厅}{ke4ting1}{9,4}{⼧,⼚}[HSK 5]
  \definition[间,个]{s.}{sala de estar; sala de visitas; sala para receber convidados}
\end{EntryWithPhonetic}

\begin{EntryWithPhonetic}{客运}{ke4yun4}{9,7}{⼧,⾡}[HSK 7-9]
  \definition{s.}{transporte de passageiros; (setor de transporte) o negócio de transporte de passageiros}
\end{EntryWithPhonetic}

%%%%%%%%%% 课 %%%%%%%%%%
\subsection*{课}\addcontentsline{loh}{figure}{课 \dpy{ke4}}

\begin{EntryWithPhonetic}{课}{ke4}{10}{⾔}[HSK 1]
  \definition{clas.}{aula; lição; unidade de tempo de ensino; parágrafo do material didático}
  \definition[门,节]{s.}{classe; aula; ensino por etapas planejado | disciplina; curso | imposto; antiga referência a impostos | seção; departamentos de escritório criados no antigo governo}
  \definition{v.}{cobrar; impor; taxar}
\end{EntryWithPhonetic}

\begin{EntryWithPhonetic}{课本}{ke4ben3}{10,5}{⾔,⽊}[HSK 1]
  \definition[本]{s.}{livro didático; livro-texto}
\end{EntryWithPhonetic}

\begin{EntryWithPhonetic}{课程}{ke4cheng2}{10,12}{⾔,⽲}[HSK 3]
  \definition[个,堂,节,门]{s.}{curso; currículo; as disciplinas e o programa letivo da escola}
\end{EntryWithPhonetic}

\begin{EntryWithPhonetic}{课堂}{ke4tang2}{10,11}{⾔,⼟}[HSK 2]
  \definition[间,节,个]{s.}{sala de aula; local onde se realizam as aulas; local onde se realizam as atividades de ensino}
\end{EntryWithPhonetic}

\begin{EntryWithPhonetic}{课题}{ke4ti2}{10,15}{⾔,⾴}[HSK 5]
  \definition[组]{s.}{uma questão para estudo ou discussão; principais questões a serem pesquisadas ou discutidas, ou assuntos importantes que precisam ser resolvidos com urgência | tarefa; problema; questões a serem resolvidas}
\end{EntryWithPhonetic}

\begin{EntryWithPhonetic}{课文}{ke4wen2}{10,4}{⾔,⽂}[HSK 1]
  \definition[篇,段]{s.}{texto (de uma lição); texto principal do livro didático (diferente das notas de rodapé, exercícios, etc.)}
\end{EntryWithPhonetic}

%%%%%%%%%% 肯 %%%%%%%%%%
\subsection*{肯}\addcontentsline{loh}{figure}{肯 \dpy{ken3}}

\begin{EntryWithPhonetic}{肯}{ken3}{8}{⾁}[HSK 6]
  \definition{s.}{carne presa ao osso}
  \definition{v.}{concordar; consentir}
  \definition{v.aux.}{estar disposto a; estar pronto para; para expressar vontade subjetiva; vontade de aceitar}
\end{EntryWithPhonetic}

\begin{EntryWithPhonetic}{肯定}{ken3ding4}{8,8}{⾁,⼧}[HSK 5]
  \definition{adj.}{certo; definitivo; positivo; afirmativo | positivo; afirmativo; aceitável}
  \definition{adv.}{certamente; definitivamente; sem dúvida; sem dúvida alguma}
  \definition{v.}{afirmar; aprovar; confirmar; considerar positivo; reconhecer a existência de algo ou sua autenticidade ou racionalidade}
  \antonymref{否定}{fou3ding4}
\end{EntryWithPhonetic}

%%%%%%%%%% 恳 %%%%%%%%%%
\subsection*{恳}\addcontentsline{loh}{figure}{恳 \dpy{ken3}}

\begin{EntryWithPhonetic}{恳}{ken3}{10}{⼼}
  \definition{adj.}{sério; sincero | cordial; honesto}
  \definition{v.}{pedir; suplicar; implorar; rogar}
\end{EntryWithPhonetic}

\begin{EntryWithPhonetic}{恳求}{ken3qiu2}{10,7}{⼼,⽔}[HSK 7-9]
  \definition{v.}{implorar; suplicar; rogar; solicitar encarecidamente}
\end{EntryWithPhonetic}

%%%%%%%%%% 啃 %%%%%%%%%%
\subsection*{啃}\addcontentsline{loh}{figure}{啃 \dpy{ken3}}

\begin{EntryWithPhonetic}{啃}{ken3}{11}{⼝}[HSK 7-9]
  \definition{v.}{roer; mordiscar | Figurativo: estudar}
\end{EntryWithPhonetic}

%%%%%%%%%% 坑 %%%%%%%%%%
\subsection*{坑}\addcontentsline{loh}{figure}{坑 \dpy{keng1}}

\begin{EntryWithPhonetic}{坑}{keng1}{7}{⼟}[HSK 7-9]
  \definition[个]{s.}{poço; buraco; cavidade | poço; túnel; caverna subterrânea}
  \definition{v.}{enredar; enganar; trapacear | nos tempos antigos, significava enterrar as pessoas vivas}
\end{EntryWithPhonetic}

\begin{EntryWithPhonetic}{坑人}{keng1/ren2}{7,2}{⼟,⼈}
  \definition{v.+compl.}{enganar; ludibriar | Dialeto: ficar chateado (com uma grande perda) | enganar alguém}
\end{EntryWithPhonetic}

%%%%%%%%%% 空 %%%%%%%%%%
\subsection*{空}\addcontentsline{loh}{figure}{空 \dpy{kong1}}

\begin{EntryWithPhonetic}{空}{kong1}{8}{⽳}[HSK 3]
  \definition*{s.}{Sobrenome: Kong}
  \definition{adj.}{vazio; oco; nulo; não inclui nada; não contém nada ou não tem conteúdo; irrealista}
  \definition{adv.}{por nada; em vão; sem efeito}
  \definition{s.}{céu; ar | vazio; vazio do mundo dos sentidos}
  \seeref{kong4}
\end{EntryWithPhonetic}

\begin{EntryWithPhonetic}{空荡荡}{kong1dang4dang4}{8,9,9}{⽳,⾋,⾋}[HSK 7-9]
  \definition{adj.}{vazio; deserto; descreve uma casa, terreno, etc., como estando muito vazio | vazio; desolado; descreve um estado de vazio espiritual e falta de plenitude}
\end{EntryWithPhonetic}

\begin{EntryWithPhonetic}{空地}{kong1di4}{8,6}{⽳,⼟}
  \definition{s.}{abertura; espaço vazio; área; gramado; terreno baldio; espaço aberto}
  \seeref{kong4di4}
\end{EntryWithPhonetic}

\begin{EntryWithPhonetic}{空间}{kong1jian1}{8,7}{⽳,⾨}[HSK 4]
  \definition[个]{s.}{espaço; recinto; cômodo; espaço em branco; interespaço}
\end{EntryWithPhonetic}

\begin{EntryWithPhonetic}{空间站}{kong1jian1zhan4}{8,7,10}{⽳,⾨,⽴}
  \definition{s.}{estação espacial}
\end{EntryWithPhonetic}

\begin{EntryWithPhonetic}{空姐}{kong1jie3}{8,8}{⽳,⼥}
  \definition[名,位,个]{s.}{aeromoça; comissária de bordo; abreviação de 空中小姐}
  \seealsoref{空中小姐}{kong1zhong1xiao3jie3}
\end{EntryWithPhonetic}

\begin{EntryWithPhonetic}{空军}{kong1jun1}{8,6}{⽳,⼍}[HSK 6]
  \definition[名,位,个,支]{s.}{força aérea; um exército que luta no ar, geralmente composto por várias unidades de aviação e unidades terrestres da força aérea}
\end{EntryWithPhonetic}

\begin{EntryWithPhonetic}{空难}{kong1nan4}{8,10}{⽳,⾫}[HSK 7-9]
  \definition{s.}{desastre aéreo; acidente aéreo; incidente aéreo; acidente de aviação}
\end{EntryWithPhonetic}

\begin{EntryWithPhonetic}{空气}{kong1qi4}{8,4}{⽳,⽓}[HSK 2]
  \definition[缕,股,份,阵]{s.}{ar; gases que compõe a atmosfera terrestre | atmosfera}
\end{EntryWithPhonetic}

\begin{EntryWithPhonetic}{空前}{kong1qian2}{8,9}{⽳,⼑}[HSK 7-9]
  \definition{adj.}{sem precedentes; nunca antes}
\end{EntryWithPhonetic}

\begin{EntryWithPhonetic}{空调}{kong1tiao2}{8,10}{⽳,⾔}[HSK 3]
  \definition[台,个]{s.}{ar-condicionado;  condicionador de ar}
\end{EntryWithPhonetic}

\begin{EntryWithPhonetic}{空想}{kong1xiang3}{8,13}{⽳,⼼}[HSK 7-9]
  \definition{s.}{pensamento irrealista; fantasia; devaneio | fantasia; sonho vão; esperança vã}
  \definition{v.}{entregar-se à fantasia; sonhar acordado}
\end{EntryWithPhonetic}

\begin{EntryWithPhonetic}{空心菜}{kong1xin1cai4}{8,4,11}{⽳,⼼,⾋}
  \definition{s.}{espinafre aquático | \emph{ong choy} | repolho do pântano | convolvulus aquático | glória-da-manhã aquática}
  \seealsoref{蕹菜}{weng4cai4}
\end{EntryWithPhonetic}

\begin{EntryWithPhonetic}{空虚}{kong1xu1}{8,11}{⽳,⾌}[HSK 7-9]
  \definition{adj.}{vazio; oco; não contém nada de substancial; não é substancial}
\end{EntryWithPhonetic}

\begin{EntryWithPhonetic}{空中}{kong1zhong1}{8,4}{⽳,⼁}[HSK 5]
  \definition{adj.}{aéreo; aerotransportado; refere"-se à transmissão de sinais de rádio}
  \definition{s.}{no céu; no ar}
\end{EntryWithPhonetic}

\begin{EntryWithPhonetic}{空中小姐}{kong1zhong1xiao3jie3}{8,4,3,8}{⽳,⼁,⼩,⼥}
  \definition{s.}{aeromoça}
\end{EntryWithPhonetic}

%%%%%%%%%% 孔 %%%%%%%%%%
\subsection*{孔}\addcontentsline{loh}{figure}{孔 \dpy{kong3}}

\begin{EntryWithPhonetic}{孔}{kong3}{4}{⼦}
  \definition*{s.}{Abreviação de Confúcio, 孔子 | Sobrenome: Kong}
  \definition{adj.}{Clássico: muito; bastante; razoavelmente}
  \definition{adj.}{aberto; desimpedido; claro; desobstruído}
  \definition{clas.}{usado para habitações em cavernas}
  \definition[个,排]{s.}{buraco; abertura; poro}
  \seealsoref{孔子}{kong3zi3}
\end{EntryWithPhonetic}

\begin{EntryWithPhonetic}{孔夫子}{kong3fu1zi3}{4,4,3}{⼦,⼤,⼦}
  \definition*{s.}{Confúcio (551-479 aC), pensador e filósofo social chinês}
  \seealsoref{孔子}{kong3zi3}
\end{EntryWithPhonetic}

\begin{EntryWithPhonetic}{孔雀}{kong3que4}{4,11}{⼦,⾫}
  \definition{s.}{pavão}
\end{EntryWithPhonetic}

\begin{EntryWithPhonetic}{孔子}{kong3zi3}{4,3}{⼦,⼦}
  \definition*{s.}{Confúcio (551-479 aC), pensador e filósofo social chinês}
  \seealsoref{孔夫子}{kong3fu1zi3}
\end{EntryWithPhonetic}

\begin{EntryWithPhonetic}{孔子学院}{kong3zi3 xue2yuan4}{4,3,8,9}{⼦,⼦,⼦,⾩}
  \definition*{s.}{Instituto Confúcio, organização estabelecida internacionalmente pela República Popular da China, que promove a língua e a cultura chinesas}
\end{EntryWithPhonetic}

%%%%%%%%%% 恐 %%%%%%%%%%
\subsection*{恐}\addcontentsline{loh}{figure}{恐 \dpy{kong3}}

\begin{EntryWithPhonetic}{恐}{kong3}{10}{⼼}
  \definition{adv.}{talvez; provavelmente}
  \definition{v.}{temer; recear; ter medo de | ameaçar; aterrorizar; intimidar}
\end{EntryWithPhonetic}

\begin{EntryWithPhonetic}{恐怖}{kong3bu4}{10,8}{⼼,⼼}[HSK 7-9]
  \definition[部]{adj.}{terrível; aterrador; horripilante; medo causado por ameaças à vida ou por presenciar violência ou derramamento de sangue | assustador; aterrorizante | terroristas; o comportamento ou os métodos utilizados são extremamente cruéis e perversos, causando choque e medo}
\end{EntryWithPhonetic}

\begin{EntryWithPhonetic}{恐怖主义}{kong3bu4zhu3yi4}{10,8,5,3}{⼼,⼼,⼂,⼂}
  \definition{adj.}{terrorista}
  \definition{s.}{terrorismo}
\end{EntryWithPhonetic}

\begin{EntryWithPhonetic}{恐吓}{kong3he4}{10,6}{⼼,⼝}[HSK 7-9]
  \definition{v.}{ameaçar; assustar; intimidar; ameaçar alguém com palavras ou meios ameaçadores}
\end{EntryWithPhonetic}

\begin{EntryWithPhonetic}{恐慌}{kong3huang1}{10,12}{⼼,⼼}[HSK 7-9]
  \definition{adj.}{pânico; em pânico; pânico devido ao medo}
  \definition{s.}{pânico; medo}
\end{EntryWithPhonetic}

\begin{EntryWithPhonetic}{恐惧}{kong3ju4}{10,11}{⼼,⼼}[HSK 7-9]
  \definition{adj.}{assustado; com medo; muito assustado}
\end{EntryWithPhonetic}

\begin{EntryWithPhonetic}{恐龙}{kong3long2}{10,5}{⼼,⿓}[HSK 7-9]
  \definition[只,头]{s.}{dinossauro | garota feia (gíria da \emph{Internet}, ofensiva)}
\end{EntryWithPhonetic}

\begin{EntryWithPhonetic}{恐怕}{kong3pa4}{10,8}{⼼,⼼}[HSK 3]
  \definition{adv.}{talvez; provavelmente; pode ser; expressa suposição; estimativa. | por medo de; expressar estimativa e preocupação}
  \definition{v.}{ter medo de; temer; recear}
\end{EntryWithPhonetic}

%%%%%%%%%% 空 %%%%%%%%%%
\subsection*{空}\addcontentsline{loh}{figure}{空 \dpy{kong4}}

\begin{EntryWithPhonetic}{空}{kong4}{8}{⽳}[HSK 4]
  \definition*{s.}{Sobrenome: Kong}
  \definition{adj.}{vazio; oco; nulo; que não contém nada; que não tem nada ou nenhum conteúdo; impraticável}
  \definition{adv.}{para nada; em vão; sem efeito}
  \definition{s.}{céu; ar | vazio; ausência do mundo dos sentidos}
  \seeref{kong1}
\end{EntryWithPhonetic}

\begin{EntryWithPhonetic}{空白}{kong4bai2}{8,5}{⽳,⽩}[HSK 7-9]
  \definition[块,片,个]{s.}{espaço; margem; espaço em branco; (na diagramação da página, páginas do livro, ilustrações, etc.) partes vazias, não preenchidas ou não utilizadas}
\end{EntryWithPhonetic}

\begin{EntryWithPhonetic}{空地}{kong4di4}{8,6}{⽳,⼟}[HSK 7-9]
  \definition{s.}{abertura; espaço vazio; área; gramado; terreno baldio; espaço aberto}
  \seeref{kong1di4}
\end{EntryWithPhonetic}

\begin{EntryWithPhonetic}{空儿}{kong4r5}{8,2}{⽳,⼉}[HSK 3]
  \definition[个]{s.}{tempo livre; sem horário específico | sala; espaço (não utilizado); área ainda não utilizada}
  \definition{v.}{ter tempo livre}
\end{EntryWithPhonetic}

\begin{EntryWithPhonetic}{空隙}{kong4xi4}{8,12}{⽳,⾩}[HSK 7-9]
  \definition{s.}{lacuna; vazio; espaço; folga; o espaço vazio no meio | intervalo; interstício; tempo livre não utilizado | chance; ocasião; oportunidade; lacunas; oportunidades a explorar}
\end{EntryWithPhonetic}

%%%%%%%%%% 控 %%%%%%%%%%
\subsection*{控}\addcontentsline{loh}{figure}{控 \dpy{kong4}}

\begin{EntryWithPhonetic}{控}{kong4}{11}{⼿}
  \definition{v.}{acusar; cobrar | controlar; dominar | manter (parte do corpo em uma determinada posição) sem apoio | virar (um recipiente) de cabeça para baixo para deixar o líquido escorrer}
\end{EntryWithPhonetic}

\begin{EntryWithPhonetic}{控告}{kong4gao4}{11,7}{⼿,⼝}[HSK 7-9]
  \definition{v.}{acusar; denunciar; incriminar; indiciar; processar alguém; apresentar uma queixa legal contra alguém}
\end{EntryWithPhonetic}

\begin{EntryWithPhonetic}{控制}{kong4zhi4}{11,8}{⼿,⼑}[HSK 5]
  \definition{v.}{controlar; restringir; dominar; fazer com que não ultrapasse um determinado limite | controlar; dominar; comandar; ocupar, fazer com que não se perca}
\end{EntryWithPhonetic}

%%%%%%%%%% 抠 %%%%%%%%%%
\subsection*{抠}\addcontentsline{loh}{figure}{抠 \dpy{kou1}}

\begin{EntryWithPhonetic}{抠}{kou1}{7}{⼿}[HSK 7-9]
  \definition{adj.}{Dialeto: mesquinho; avarento}
  \definition{v.}{escolher; cavar ou desenterrar com o dedo ou algo pontiagudo; arranhar | esculpir; cortar | aprofundar-se em; estudar meticulosamente; ir desnecessariamente ao âmago de}
\end{EntryWithPhonetic}

%%%%%%%%%% 口 %%%%%%%%%%
\subsection*{口}\addcontentsline{loh}{figure}{口 \dpy{kou3}}

\begin{EntryWithPhonetic}{口}{kou3}{3}{⼝}[HSK 1][Kangxi 30]
  \definition*{s.}{Sobrenome: Kou}
  \definition{clas.}{usado para coisas com bocas (pessoas, animais domésticos, canhões, etc.) | usado para mordidas ou bocados | usado para idiomas}
  \definition{s.}{boca | borda; boca; o espaço externo ao recipiente | saída; entrada; local de entrada e saída | o gosto de alguém | corte; buraco; ferida |  a borda de uma faca; lâminas de facas, espadas, tesouras, etc. | a idade de um animal de tração | seção; departamento; sistema integrado de departamentos relacionados | conversa, discurso; pronunciamento; referência à fala | um portão da Grande Muralha (frequentemente usado em nomes de lugares)}
\end{EntryWithPhonetic}

\begin{EntryWithPhonetic}{口碑}{kou3bei1}{3,13}{⼝,⽯}[HSK 7-9]
  \definition{s.}{elogio público; isso se refere à avaliação verbal que as pessoas fazem de alguém (antigamente, elogios a uma pessoa eram frequentemente gravados em tábuas de pedra)}
\end{EntryWithPhonetic}

\begin{EntryWithPhonetic}{口才}{kou3cai2}{3,3}{⼝,⼿}[HSK 7-9]
  \definition{s.}{eloquência; persuasão; a capacidade de se expressar verbalmente; o talento para a fala}
\end{EntryWithPhonetic}

\begin{EntryWithPhonetic}{口吃}{kou3chi1}{3,6}{⼝,⼝}[HSK 7-9]
  \definition{s.}{gagueira; espasmofemia; balbucinato; mogilalia; battarismo; battarismo; iscnofonia; pselismo; o fenômeno de repetir palavras ou interromper frases ao falar é um defeito habitual de linguagem comumente conhecido como gagueira}
\end{EntryWithPhonetic}

\begin{EntryWithPhonetic}{口吃病}{kou3chi1 bing4}{3,6,10}{⼝,⼝,⽧}
  \definition{s.}{doença da gagueira}
\end{EntryWithPhonetic}

\begin{EntryWithPhonetic}{口袋妖怪}{kou3dai4 yao1guai4}{3,11,7,8}{⼝,⾐,⼥,⼼}
  \definition*{s.}{Pokémon (franquia de mídia japonesa)}
\end{EntryWithPhonetic}

\begin{EntryWithPhonetic}{口袋}{kou3dai5}{3,11}{⼝,⾐}[HSK 4]
  \definition[个,只]{s.}{bolso | saco; sacola; artigos de tecido ou couro}
\end{EntryWithPhonetic}

\begin{EntryWithPhonetic}{口感}{kou3gan3}{3,13}{⼝,⼼}[HSK 7-9]
  \definition{s.}{textura (dos alimentos); sabor; sensação que o alimento proporciona na boca}
\end{EntryWithPhonetic}

\begin{EntryWithPhonetic}{口号}{kou3hao4}{3,5}{⼝,⼝}[HSK 5]
  \definition[个,条,些]{s.}{\emph{slogan}; palavra de ordem; lema}
\end{EntryWithPhonetic}

\begin{EntryWithPhonetic}{口径}{kou3jing4}{3,8}{⼝,⼻}[HSK 7-9]
  \definition{s.}{calibre; diâmetro; diâmetro da boca circular do vaso | requisitos; especificações; geralmente se refere às especificações, desempenho, etc., exigidos | declaração; pensamento; linha de ação ou fala; conteúdo metafórico da fala}
\end{EntryWithPhonetic}

\begin{EntryWithPhonetic}{口令}{kou3ling4}{3,5}{⼝,⼈}[HSK 7-9]
  \definition[串]{s.}{palavra de comando | senha; palavra-chave; contra-senha | palavra; comando; lema; comando verbal | códigos verbais para identificar amigo ou inimigo}
\end{EntryWithPhonetic}

\begin{EntryWithPhonetic}{口气}{kou3qi4}{3,4}{⼝,⽓}[HSK 7-9]
  \definition{s.}{maneira de falar; a força e o ritmo do tom; a dinâmica do discurso | implicação; o que realmente se quer dizer; o significado não dito nas palavras | tom; nota; tom emocional na fala}
\end{EntryWithPhonetic}

\begin{EntryWithPhonetic}{口腔}{kou3qiang1}{3,12}{⼝,⾁}[HSK 7-9]
  \definition{s.}{cavidade oral; a cavidade oral é um espaço oco composto pelos lábios, bochechas, palato duro e palato mole; contém órgãos como dentes, língua e glândulas salivares}
\end{EntryWithPhonetic}

\begin{EntryWithPhonetic}{口哨}{kou3shao4}{3,10}{⼝,⼝}[HSK 7-9]
  \definition{s.}{apito; assobio}
  \definition{v.}{assobiar}
\end{EntryWithPhonetic}

\begin{EntryWithPhonetic}{口试}{kou3shi4}{3,8}{⼝,⾔}[HSK 6]
  \definition{s.}{exame oral (ou teste); um tipo de exame que exige que os candidatos respondam a perguntas oralmente}
  \definition{v.}{examinar oralmente}
  \antonymref{笔试}{bi3shi4}
\end{EntryWithPhonetic}

\begin{EntryWithPhonetic}{口水}{kou3shui3}{3,4}{⼝,⽔}[HSK 7-9]
  \definition{s.}{saliva; baba; o termo geral para saliva}
\end{EntryWithPhonetic}

\begin{EntryWithPhonetic}{口头}{kou3tou2}{3,5}{⼝,⼤}[HSK 7-9]
  \definition{s.}{oral; verbal; expressa-se através da fala | boca (referindo"-se à boca ao falar); lábios}
  \synonymref{表面}{biao3mian4}
  \antonymref{思想}{si1xiang3}
  \antonymref{行动}{xing2dong4}
  \antonymref{行为}{xing2wei2}
\end{EntryWithPhonetic}

\begin{EntryWithPhonetic}{口味}{kou3wei4}{3,8}{⼝,⼝}[HSK 7-9]
  \definition[个,种]{s.}{sabor da comida; o gosto da comida | o gosto de uma pessoa; preferência de cada um em termos de sabor | gostos; metáfora para interesses e hobbies pessoais}
\end{EntryWithPhonetic}

\begin{EntryWithPhonetic}{口香糖}{kou3xiang1tang2}{3,9,16}{⼝,⾹,⽶}[HSK 7-9]
  \definition{s.}{goma de mascar; chiclete; um tipo de doce feito da seiva viscosa que escorre do tronco da sapotilha (uma árvore perene cujos frutos têm o tamanho de peras e formato de coração, daí o nome), juntamente com açúcar e aromatizantes; é para ser mastigado apenas e não deve ser engolido}
\end{EntryWithPhonetic}

\begin{EntryWithPhonetic}{口音}{kou3yin1}{3,9}{⼝,⾳}[HSK 7-9]
  \definition[种]{s.}{sotaques; sons da fala oral (linguística); os hábitos de fala de uma pessoa, especialmente seus hábitos de pronúncia, podem revelar sua origem linguística}
  \seeref{kou3yin5}
\end{EntryWithPhonetic}

\begin{EntryWithPhonetic}{口音}{kou3yin5}{3,9}{⼝,⾳}
  \definition[种]{s.}{voz | sotaque; dialeto}
  \seeref{kou3yin1}
\end{EntryWithPhonetic}

\begin{EntryWithPhonetic}{口语}{kou3yu3}{3,9}{⼝,⾔}[HSK 4]
  \definition[门]{s.}{linguagem oral; linguagem falada; linguagem coloquial; linguagem usada em conversas}
\end{EntryWithPhonetic}

\begin{EntryWithPhonetic}{口罩}{kou3zhao4}{3,13}{⼝,⽹}[HSK 7-9]
  \definition[副]{s.}{máscara cirúrgica; produtos de higiene; feitos de gaze, etc.; usados sobre a boca e o nariz para evitar a entrada de poeira e germes}
\end{EntryWithPhonetic}

\begin{EntryWithPhonetic}{口子}{kou3zi5}{3,3}{⼝,⼦}[HSK 7-9]
  \definition{clas.}{referente a pessoas}[你家有几口子?===Quantas pessoas há na sua família?]
  \definition{s.}{Coloquial: pessoa | marido ou esposa | abertura; buraco; corte; rasgo; uma grande lacuna; uma fenda | Figurativo: abertura; oportunidade}
\end{EntryWithPhonetic}

%%%%%%%%%% 扣 %%%%%%%%%%
\subsection*{扣}\addcontentsline{loh}{figure}{扣 \dpy{kou4}}

\begin{EntryWithPhonetic}{扣}{kou4}{6}{⼿}[HSK 6]
  \definition*{s.}{Sobrenome: Kou}
  \definition{clas.}{giro; volta; uma volta de uma rosca}
  \definition[个,颗,粒]{s.}{nó | fivela; botão | círculo de rosca (em um parafuso)}
  \definition{v.}{fivela; abotoar; amarrar ou prender com um laço ou anel | colocar uma xícara, tigela etc. de cabeça para baixo; cobrir com uma xícara, tigela etc. invertida; colocar a boca do recipiente para baixo | deter; prender; levar sob custódia | cravar; esmagar (a bola); arremessar ou bater (em uma bola) com força de cima para baixo | atracar; deduzir; descontar; subtrair uma parte do valor original | puxar; pressionar | impor; marcar sem fundamento; acusar injustamente; impor ou atribuir (um crime ou má fama) a alguém}
\end{EntryWithPhonetic}

\begin{EntryWithPhonetic}{扣除}{kou4chu2}{6,9}{⼿,⾩}[HSK 7-9]
  \definition{v.}{deduzir; tirar; subtrair do total}
\end{EntryWithPhonetic}

\begin{EntryWithPhonetic}{扣留}{kou4liu2}{6,10}{⼿,⽥}[HSK 7-9]
  \definition{s.}{apreensão; detenção}
  \definition{v.}{deter; manter sob custódia; pôr em prisão domiciliar; prender}
\end{EntryWithPhonetic}

\begin{EntryWithPhonetic}{扣人心弦}{kou4ren2xin1xian2}{6,2,4,8}{⼿,⼈,⼼,⼸}[HSK 7-9]
  \definition{expr.}{emocionante; cativar alguém; conquistar o coração de; comovente; tocar os sentimentos de alguém; tocar o coração de alguém; tocar profundamente; muito tocante; descrever poesia, prosa, performances, etc., como tendo uma qualidade contagiante que desperta emoções; eletrizante; de tirar o fôlego}
\end{EntryWithPhonetic}

\begin{EntryWithPhonetic}{扣押}{kou4ya1}{6,8}{⼿,⼿}[HSK 7-9]
  \definition{s.}{detenção}
  \definition{v.}{deter; apreender; manter sob custódia | apreender; confiscar; embargar | sequestrar (ou tomar, reter) alguém como refém}
\end{EntryWithPhonetic}

%%%%%%%%%% 枯 %%%%%%%%%%
\subsection*{枯}\addcontentsline{loh}{figure}{枯 \dpy{ku1}}

\begin{EntryWithPhonetic}{枯}{ku1}{9}{⽊}
  \definition{adj.}{murcho | (de um poço, rio, etc.) seco | chato; desinteressante | magro e abatido; emaciado}
  \definition[片]{s.}{borra; resíduo}
\end{EntryWithPhonetic}

\begin{EntryWithPhonetic}{枯木}{ku1mu4}{9,4}{⽊,⽊}
  \definition{s.}{árvore morta | madeira morta}
\end{EntryWithPhonetic}

\begin{EntryWithPhonetic}{枯燥}{ku1zao4}{9,17}{⽊,⽕}[HSK 7-9]
  \definition{adj.}{sem graça; monótono e sem vida; monótono; desinteressante}
\end{EntryWithPhonetic}

%%%%%%%%%% 哭 %%%%%%%%%%
\subsection*{哭}\addcontentsline{loh}{figure}{哭 \dpy{ku1}}

\begin{EntryWithPhonetic}{哭}{ku1}{10}{⼝}[HSK 2]
  \definition{v.}{chorar; soluçar; lamentar-se; chorar de dor ou de emoção}
\end{EntryWithPhonetic}

\begin{EntryWithPhonetic}{哭泣}{ku1qi4}{10,8}{⼝,⽔}[HSK 7-9]
  \definition{v.}{chorar; soluçar; chorar copiosamente}
\end{EntryWithPhonetic}

\begin{EntryWithPhonetic}{哭墙}{ku1qiang2}{10,14}{⼝,⼟}
  \definition*{s.}{Muro das Lamentações (Jerusalém)}
\end{EntryWithPhonetic}

\begin{EntryWithPhonetic}{哭笑不得}{ku1xiao4-bu4de2}{10,10,4,11}{⼝,⽵,⼀,⼻}[HSK 7-9]
  \definition{expr.}{``Sem saber se ria ou chorava.''; a incapacidade de chorar ou rir descreve uma situação constrangedora em que a pessoa não sabe o que fazer}
\end{EntryWithPhonetic}

%%%%%%%%%% 窟 %%%%%%%%%%
\subsection*{窟}\addcontentsline{loh}{figure}{窟 \dpy{ku1}}

\begin{EntryWithPhonetic}{窟}{ku1}{13}{⽳}
  \definition{s.}{buraco; caverna; cavidade; toca; gruta}
\end{EntryWithPhonetic}

\begin{EntryWithPhonetic}{窟窿}{ku1long5}{13,16}{⽳,⽳}[HSK 7-9]
  \definition[个]{s.}{furo; cavidade; buraco | déficit (figurativo); débito; dívida; essa metáfora se refere a déficits financeiros ou brechas no trabalho}
\end{EntryWithPhonetic}

%%%%%%%%%% 苦 %%%%%%%%%%
\subsection*{苦}\addcontentsline{loh}{figure}{苦 \dpy{ku3}}

\begin{EntryWithPhonetic}{苦}{ku3}{8}{⾋}[HSK 4]
  \definition{adj.}{amargo; descreve um sabor parecido com o de melão amargo ou raiz de coptis | difícil; doloroso; sofrido}
  \definition{adv.}{meticulosamente; diligentemente; pacientemente}
  \definition{v.}{causar sofrimento a alguém; dificultar a vida de alguém; causar dor; tornar desconfortável | sofrer de; ser incomodado por; sentir-se angustiado com uma situação | estar desgastado; cortar demais; descrever a superação de um certo nível em algum aspecto}
  \antonymref{甘}{gan1}
  \antonymref{甜}{tian2}
\end{EntryWithPhonetic}

\begin{EntryWithPhonetic}{苦瓜}{ku3gua1}{8,5}{⾋,⽠}
  \definition{s.}{melão amargo (cabaça amarga, pêra bálsamo, maçã bálsamo, pepino amargo)}
\end{EntryWithPhonetic}

\begin{EntryWithPhonetic}{苦力}{ku3li4}{8,2}{⾋,⼒}[HSK 7-9]
  \definition{s.}{trabalhador hindu ou chinês; carregador | Empréstimo linguístico: \emph{coolie}, trabalhador chinês não qualificado nos tempos coloniais | trabalho amargo | trabalho árduo}
\end{EntryWithPhonetic}

\begin{EntryWithPhonetic}{苦练}{ku3 lian4}{8,8}{⾋,⽷}[HSK 7-9]
  \definition{v.}{treinar diligentemente; praticar bastante}
\end{EntryWithPhonetic}

\begin{EntryWithPhonetic}{苦难}{ku3nan4}{8,10}{⾋,⾫}[HSK 7-9]
  \definition{adj.}{sofrimento; dificuldade}
  \definition{s.}{sofrimento; miséria; angústia; tribulação; dor e desastre}
\end{EntryWithPhonetic}

\begin{EntryWithPhonetic}{苦恼}{ku3nao3}{8,9}{⾋,⼼}[HSK 7-9]
  \definition{adj.}{aflito; preocupado; atormentado}
  \definition{v.}{atormentar; causar dor e sofrimento}
\end{EntryWithPhonetic}

\begin{EntryWithPhonetic}{苦笑}{ku3xiao4}{8,10}{⾋,⽵}[HSK 7-9]
  \definition{s.}{sorriso forçado; sorriso irônico; sorriso amargo}
  \definition{v.}{forçar um sorriso; esboçar um sorriso irônico; dar uma risada amarga; produzir um sorriso forçado; forçar um sorriso quando você não está de bom humor}
\end{EntryWithPhonetic}

\begin{EntryWithPhonetic}{苦心}{ku3xin1}{8,4}{⾋,⼼}[HSK 7-9]
  \definition{adv.}{com esmero; assiduamente; meticulosamente; com esforço ou dedicação; isso demonstra que você dedicou muito esforço e energia}
  \definition{s.}{dores; dificuldade tomada; o esforço e a energia investidos em trabalhar arduamente por algo}
\end{EntryWithPhonetic}

%%%%%%%%%% 库 %%%%%%%%%%
\subsection*{库}\addcontentsline{loh}{figure}{库 \dpy{ku4}}

\begin{EntryWithPhonetic}{库}{ku4}{7}{⼴}[HSK 5]
  \definition{s.}{depósito; tesouraria; armazém; almoxarifado; edifícios e equipamentos para armazenamento de mercadorias | Computação: banco de dados}
\end{EntryWithPhonetic}

%%%%%%%%%% 裤 %%%%%%%%%%
\subsection*{裤}\addcontentsline{loh}{figure}{裤 \dpy{ku4}}

\begin{EntryWithPhonetic}{裤}{ku4}{12}{⾐}
  \definition[条]{s.}{calças}
\end{EntryWithPhonetic}

\begin{EntryWithPhonetic}{裤子}{ku4zi5}{12,3}{⾐,⼦}[HSK 3]
  \definition[条]{s.}{calças; calções; roupas usadas abaixo da cintura, com cós, virilha e duas pernas}
\end{EntryWithPhonetic}

%%%%%%%%%% 酷 %%%%%%%%%%
\subsection*{酷}\addcontentsline{loh}{figure}{酷 \dpy{ku4}}

\begin{EntryWithPhonetic}{酷}{ku4}{14}{⾣}[HSK 6]
  \definition{adj.}{cruel; opressivo | feroz; escaldante | brutal | Empréstimo linguístico: \emph{cool}; legal; excelente; moderno; ótimo | elegante e sóbrio; gracioso e severo}
  \definition{adv.}{muito; extremamente}
\end{EntryWithPhonetic}

\begin{EntryWithPhonetic}{酷斯拉}{ku4si1la1}{14,12,8}{⾣,⽄,⼿}
  \definition*{s.}{Godzilla. do Japonês Gojira, ゴジラ}
  \seealsoref{哥斯拉}{ge1si1la1}
\end{EntryWithPhonetic}

\begin{EntryWithPhonetic}{酷似}{ku4si4}{14,6}{⾣,⼈}[HSK 7-9]
  \definition{v.}{ser a própria imagem de; ser exatamente igual a; apresentar forte semelhança com}
\end{EntryWithPhonetic}

%%%%%%%%%% 夸 %%%%%%%%%%
\subsection*{夸}\addcontentsline{loh}{figure}{夸 \dpy{kua1}}

\begin{EntryWithPhonetic}{夸}{kua1}{6}{⼤}[HSK 7-9]
  \definition{v.}{exagerar; superestimar; vangloriar-se | elogiar; destacar as qualidades positivas de uma pessoa ou coisa}
\end{EntryWithPhonetic}

\begin{EntryWithPhonetic}{夸大}{kua1da4}{6,3}{⼤,⼤}[HSK 7-9]
  \definition{v.}{exagerar; superestimar; magnificar; engrandecer}
\end{EntryWithPhonetic}

\begin{EntryWithPhonetic}{夸奖}{kua1jiang3}{6,9}{⼤,⼤}[HSK 7-9]
  \definition{v.}{louvar; elogiar; cumprimentar; demonstrar apreço e incentivar alguém por suas qualidades ou boas ações}
\end{EntryWithPhonetic}

\begin{EntryWithPhonetic}{夸夸其谈}{kua1kua1-qi2tan2}{6,6,8,10}{⼤,⼤,⼋,⾔}[HSK 7-9]
  \definition{expr.}{entregar-se à verborragia; conversa empolgante e jactanciosa sem muito significado; cheio de ar quente; um longo discurso cheio de pompa; tagarelice; falar demais; uma grande conversa; discursar; conversa pomposa; conversa ou escrita pomposa, mas sem sentido; discurso pomposo; exagerar; desabafar; exagerar (ou usar verborragia); tagarelar; falar pelos cotovelos}
\end{EntryWithPhonetic}

\begin{EntryWithPhonetic}{夸耀}{kua1yao4}{6,20}{⼤,⽻}[HSK 7-9]
  \definition{v.}{ostentar; gabar"-se de; vangloriar"-se de; exibir (as próprias habilidades, conquistas, status e poder)}
\end{EntryWithPhonetic}

\begin{EntryWithPhonetic}{夸张}{kua1zhang1}{6,7}{⼤,⼸}[HSK 7-9]
  \definition{adj.}{exagerado; superestimado; pomposo}
  \definition{s.}{hipérbole; uma figura de linguagem que utiliza palavras exageradas para descrever coisas}
\end{EntryWithPhonetic}

%%%%%%%%%% 垮 %%%%%%%%%%
\subsection*{垮}\addcontentsline{loh}{figure}{垮 \dpy{kua3}}

\begin{EntryWithPhonetic}{垮}{kua3}{9}{⼟}[HSK 7-9]
  \definition{v.}{colapsar; cair (quebrar); desmoronar | desmoronar (declínio mental e físico)}
\end{EntryWithPhonetic}

%%%%%%%%%% 挎 %%%%%%%%%%
\subsection*{挎}\addcontentsline{loh}{figure}{挎 \dpy{kua4}}

\begin{EntryWithPhonetic}{挎}{kua4}{9}{⼿}[HSK 7-9]
  \definition{v.}{carregar no braço | carregar algo sobre o ombro, ao redor do pescoço ou ao lado do corpo | pendurar coisas no ombro, pescoço ou cintura}
\end{EntryWithPhonetic}

%%%%%%%%%% 跨 %%%%%%%%%%
\subsection*{跨}\addcontentsline{loh}{figure}{跨 \dpy{kua4}}

\begin{EntryWithPhonetic}{跨}{kua4}{13}{⾜}[HSK 6]
  \definition{adj.}{localizado ao lado de; anexo a}
  \definition{v.}{dar um passo; andar a passos largos | disputar; ficar de pernas abertas | atravessar; ir além (dos limites de uma certa quantidade, tempo, região, etc.)}
\end{EntryWithPhonetic}

\begin{EntryWithPhonetic}{跨国}{kua4guo2}{13,8}{⾜,⼞}[HSK 7-9]
  \definition{adj.}{transnacional; que transcende fronteiras nacionais, envolvendo dois ou mais países}
\end{EntryWithPhonetic}

\begin{EntryWithPhonetic}{跨越}{kua4yue4}{13,12}{⾜,⾛}[HSK 7-9]
  \definition{v.}{passar por cima; saltar por cima; atravessar; dar um passo largo; cruzar fronteiras regionais ou de período}
\end{EntryWithPhonetic}

%%%%%%%%%% 会 %%%%%%%%%%
\subsection*{会}\addcontentsline{loh}{figure}{会 \dpy{kuai4}}

\begin{EntryWithPhonetic}{会}{kuai4}{6}{⼈}
  \definition[个,场,次]{s.}{contabilidade}
  \definition{v.}{computar; calcular; equilibrar uma conta}
  \seeref{hui4}
\end{EntryWithPhonetic}

\begin{EntryWithPhonetic}{会计}{kuai4ji5}{6,4}{⼈,⾔}[HSK 4]
  \definition[个,位,名]{s.}{contabilidade | contador; contabilista; guarda-livros; pessoal que trabalha como contador}
\end{EntryWithPhonetic}

%%%%%%%%%% 块 %%%%%%%%%%
\subsection*{块}\addcontentsline{loh}{figure}{块 \dpy{kuai4}}

\begin{EntryWithPhonetic}{块}{kuai4}{7}{⼟}[HSK 1]
  \definition{clas.}{usado para coisas em pedaços | usado para coisas em pedaços ou em algumas formas de folhas | usado para moedas de prata ou notas de papel equivalentes a 圆}
  \definition{s.}{pedaço; pedaço (de terra); peça; algo que forma um pedaço ou massa}
  \seealsoref{圆}{yuan2}
\end{EntryWithPhonetic}

%%%%%%%%%% 快 %%%%%%%%%%
\subsection*{快}\addcontentsline{loh}{figure}{快 \dpy{kuai4}}

\begin{EntryWithPhonetic}{快}{kuai4}{7}{⼼}[HSK 1]
  \definition*{s.}{Sobrenome: Kuai}
  \definition{adj.}{rápido; veloz | apressado | perspicaz; ágil; inteligente; de mente rápida | (faca, espada, etc.) afiado | direto; franco; sem rodeios | satisfeito; feliz; gratificado | rápido; veloz; alta velocidade; tempo de execução curto | satisfeito; feliz; contente | engenhoso; ágil | afiado; facas, tesouras, machados e outros objetos afiados | sincero}
  \definition{adv.}{em breve; antes de muito tempo; estar prestes a | rapidamente}
  \definition{s.}{policial; polícia | (antigo) oficial encarregado de efetuar prisões}
  \antonymref{钝}{dun4}
  \antonymref{慢}{man4}
\end{EntryWithPhonetic}

\begin{EntryWithPhonetic}{快餐}{kuai4can1}{7,16}{⼼,⾷}[HSK 2]
  \definition[份,顿]{s.}{pedido (comida) rápido; \emph{fast food}; refere"-se a refeições simples preparadas com antecedência e que podem ser servidas rapidamente}
\end{EntryWithPhonetic}

\begin{EntryWithPhonetic}{快车}{kuai4che1}{7,4}{⼼,⾞}[HSK 6]
  \definition{s.}{trem ou ônibus expresso; um trem ou ônibus com menos paradas e tempos de viagem mais curtos (usado principalmente para transporte de passageiros)}
  \antonymref{慢车}{man4che1}
\end{EntryWithPhonetic}

\begin{EntryWithPhonetic}{快递}{kuai4di4}{7,10}{⼼,⾡}[HSK 4]
  \definition[个,件,批]{s.}{correio rápido; entrega expressa; entrega rápida}
  \definition{v.}{entregar (serviço de entrega rápida por transportadoras especializadas)}
\end{EntryWithPhonetic}

\begin{EntryWithPhonetic}{快点儿}{kuai4dian3r5}{7,9,2}{⼼,⽕,⼉}[HSK 2]
  \definition{v.}{apressar-se}
\end{EntryWithPhonetic}

\begin{EntryWithPhonetic}{快活}{kuai4huo5}{7,9}{⼼,⽔}[HSK 5]
  \definition{adj.}{feliz; alegre; contente; animado}
\end{EntryWithPhonetic}

\begin{EntryWithPhonetic}{快捷}{kuai4jie2}{7,11}{⼼,⼿}[HSK 7-9]
  \definition{adj.}{rápido; veloz (em relação à velocidade); ágil (em relação a pessoas)}
\end{EntryWithPhonetic}

\begin{EntryWithPhonetic}{快乐}{kuai4le4}{7,5}{⼼,⼃}[HSK 2]
  \definition{adj.}{feliz; alegre; animado; prazeiroso}
  \definition{s.}{felicidade | alegria}
\end{EntryWithPhonetic}

\begin{EntryWithPhonetic}{快速}{kuai4su4}{7,10}{⼼,⾡}[HSK 3]
  \definition{adj.}{rápido; veloz; de alta velocidade; descreve o tempo curto gasto para caminhar, fazer algo, etc.}
\end{EntryWithPhonetic}

\begin{EntryWithPhonetic}{快要}{kuai4yao4}{7,9}{⼼,⾑}[HSK 2]
  \definition{adv.}{estar prestes a; estar indo para; estar à beira de; em breve; em pouco tempo; indica que a situação está prestes a ocorrer}
\end{EntryWithPhonetic}

%%%%%%%%%% 筷 %%%%%%%%%%
\subsection*{筷}\addcontentsline{loh}{figure}{筷 \dpy{kuai4}}

\begin{EntryWithPhonetic}{筷}{kuai4}{13}{⽵}
  \definition[双,根,个]{s.}{pauzinhos para comer}
\end{EntryWithPhonetic}

\begin{EntryWithPhonetic}{筷子}{kuai4zi5}{13,3}{⽵,⼦}[HSK 2]
  \definition[根,双,副,把,对]{s.}{pauzinhos; \emph{chopsticks}; dois bastôes finos feitos de bambu, madeira, metal ou outro material, usados para segurar comida ou outros objetos}
\end{EntryWithPhonetic}

%%%%%%%%%% 宽 %%%%%%%%%%
\subsection*{宽}\addcontentsline{loh}{figure}{宽 \dpy{kuan1}}

\begin{EntryWithPhonetic}{宽}{kuan1}{10}{⼧}[HSK 4]
  \definition*{s.}{Sobrenome: Kuan}
  \definition{adj.}{largo; amplo; espaçoso; grandes distâncias horizontais | leniente; generoso; indulgente | bem de vida; rico; confortável}
  \definition[米]{s.}{largura; amplitude}[桌子有一米宽。===A mesa tem um metro de largura.]
  \definition{v.}{relaxar; aliviar}
  \antonymref{窄}{zhai3}
\end{EntryWithPhonetic}

\begin{EntryWithPhonetic}{宽敞}{kuan1chang5}{10,12}{⼧,⽁}[HSK 7-9]
  \definition{adj.}{amplo; espaçoso; descreve um espaço ou área grande}
\end{EntryWithPhonetic}

\begin{EntryWithPhonetic}{宽度}{kuan1du4}{10,9}{⼧,⼴}[HSK 5]
  \definition{s.}{largura; amplitude; duração; o grau de largura e estreiteza; a distância horizontal (no caso de um retângulo, a distância entre os dois lados mais longos)}
\end{EntryWithPhonetic}

\begin{EntryWithPhonetic}{宽泛}{kuan1fan4}{10,7}{⼧,⽔}[HSK 7-9]
  \definition{adj.}{abrangente; (conteúdo, significado) abrange uma ampla gama de aspectos}
\end{EntryWithPhonetic}

\begin{EntryWithPhonetic}{宽广}{kuan1guang3}{10,3}{⼧,⼴}[HSK 4]
  \definition{adj.}{vasto; amplo; espaçoso; extenso}
\end{EntryWithPhonetic}

\begin{EntryWithPhonetic}{宽厚}{kuan1hou4}{10,9}{⼧,⼚}[HSK 7-9]
  \definition{adj.}{largo e espesso; amplo e sólido | tolerante; gentil e generoso; tolerância e bondade | simples; sincero; (voz) profunda e ressonante}
\end{EntryWithPhonetic}

\begin{EntryWithPhonetic}{宽阔}{kuan1kuo4}{10,12}{⼧,⾨}[HSK 6]
  \definition{adj.}{amplo; largo; espaçoso | tolerante; mente aberta; descreve uma mente alegre e ampla}
\end{EntryWithPhonetic}

\begin{EntryWithPhonetic}{宽容}{kuan1rong2}{10,10}{⼧,⼧}[HSK 7-9]
  \definition{adj.}{tolerante; generoso e magnânimo, não mesquinho ou oportunista}
  \definition{v.}{tolerar; ter paciência com; ser tolerante com os outros, não guardar rancor nem insistir no assunto}
\end{EntryWithPhonetic}

\begin{EntryWithPhonetic}{宽恕}{kuan1shu4}{10,10}{⼧,⼼}[HSK 7-9]
  \definition{v.}{perdoar; desculpar; absolver}
\end{EntryWithPhonetic}

\begin{EntryWithPhonetic}{宽松}{kuan1song1}{10,8}{⼧,⽊}[HSK 7-9]
  \definition{adj.}{(roupas) folgado e confortável; espaçoso e sem aglomeração; relaxante e sem aperto | (estado mental, atmosfera, etc.) relaxado; aliviado; sem tensão; livre de preocupações; descontraído; não há tensão | Economia: abundante; que tem dinheiro suficiente para viver bem e sem problemas, mas sem ser excessivamente rico}
\end{EntryWithPhonetic}

\begin{EntryWithPhonetic}{宽影片}{kuan1ying3pian4}{10,15,4}{⼧,⼺,⽚}
  \definition{s.}{filme \emph{widescreen}}
\end{EntryWithPhonetic}

%%%%%%%%%% 款 %%%%%%%%%%
\subsection*{款}\addcontentsline{loh}{figure}{款 \dpy{kuan3}}

\begin{EntryWithPhonetic}{款}{kuan3}{12}{⽋}
  \definition{adj.}{sincero | lento; sem pressa | vazio; oco; irreal; é intercambiável com 窾}
  \definition{s.}{parágrafo; seção (de um artigo em um documento legal, etc.); os itens listados de acordo com as disposições de leis, regulamentos, tratados, etc. | dinheiro; fundo; uma quantia de dinheiro | o nome do remetente ou destinatário inscrito em uma pintura ou obra de caligrafia oferecida como presente; inscrições fundidas em recipientes de bronze, como sinos e tripés; títulos em pinturas e caligrafia | forma; estilo; especificações}
  \definition[笔,个]{v.}{entreter; receber com hospitalidade | Literário: bater; prostrar-se}
  \seealsoref{窾}{kuan3}
\end{EntryWithPhonetic}

\begin{EntryWithPhonetic}{款式}{kuan3shi4}{12,6}{⽋,⼷}[HSK 7-9]
  \definition[种,个,款,类]{s.}{modelo; estilo; design; moda; padrão; formato}
\end{EntryWithPhonetic}

\begin{EntryWithPhonetic}{款项}{kuan3xiang4}{12,9}{⽋,⾴}[HSK 7-9]
  \definition[宗]{s.}{soma de dinheiro; refere"-se a uma grande quantia de dinheiro com um propósito específico | cláusula (ordem, regra, tratado); artigos (em leis, regulamentos, tratados, etc.); itens dentro de artigos de leis, regulamentos, tratados, etc., são geralmente divididos em cláusulas, e cláusulas em subitens}
\end{EntryWithPhonetic}

%%%%%%%%%% 窾 %%%%%%%%%%
\subsection*{窾}\addcontentsline{loh}{figure}{窾 \dpy{kuan3}}

\begin{EntryWithPhonetic}{窾}{kuan3}{17}{⽳}
  \definition{adj.}{oco}
  \definition{s.}{rachadura; cavidade | (onomatopéia) água batendo na rocha}
  \definition{v.}{escavar um buraco}
  \seeref{cuan4}
\end{EntryWithPhonetic}

%%%%%%%%%% 筐 %%%%%%%%%%
\subsection*{筐}\addcontentsline{loh}{figure}{筐 \dpy{kuang1}}

\begin{EntryWithPhonetic}{筐}{kuang1}{12}{⽵}[HSK 7-9]
  \definition[个,只]{s.}{cesto; cestaria; caixa; recipientes trançados com tiras de bambu, galhos de salgueiro e sarças}
\end{EntryWithPhonetic}

%%%%%%%%%% 狂 %%%%%%%%%%
\subsection*{狂}\addcontentsline{loh}{figure}{狂 \dpy{kuang2}}

\begin{EntryWithPhonetic}{狂}{kuang2}{7}{⽝}[HSK 5]
  \definition*{s.}{Sobrenome: Kuang}
  \definition{adj.}{louco; maluco | violento; selvagem | selvagem; delirante; furioso; desenfreado; desinibido; sem restrições | arrogante; autoritário}
\end{EntryWithPhonetic}

\begin{EntryWithPhonetic}{狂欢}{kuang2huan1}{7,6}{⽝,⽋}[HSK 7-9]
  \definition{s.}{festejar; fazer folia; carnavalear; aproveitar a diversão}
  \definition{s.}{folia; carnaval}
\end{EntryWithPhonetic}

\begin{EntryWithPhonetic}{狂欢节}{kuang2huan1jie2}{7,6,5}{⽝,⽋,⾋}[HSK 7-9]
  \definition*{s.}{Carnaval}
\end{EntryWithPhonetic}

\begin{EntryWithPhonetic}{狂热}{kuang2re4}{7,10}{⽝,⽕}[HSK 7-9]
  \definition{adj.}{fanático; febril; o entusiasmo extremo que foi despertado em um determinado momento}
  \definition{s.}{febre; emoções extremamente intensas}
\end{EntryWithPhonetic}

%%%%%%%%%% 况 %%%%%%%%%%
\subsection*{况}\addcontentsline{loh}{figure}{况 \dpy{kuang4}}

\begin{EntryWithPhonetic}{况}{kuang4}{7}{⼎}
  \definition*{s.}{Sobrenome: Kuang}
  \definition{conj.}{além disso | mesmo; muito menos; sem mencionar}
  \definition{s.}{condição; situação}
  \definition{v.}{comparar}
\end{EntryWithPhonetic}

\begin{EntryWithPhonetic}{况且}{kuang4qie3}{7,5}{⼎,⼀}
  \definition{conj.}{além disso; além do mais; orações de conexão para expressar uma relação progressiva}
\end{EntryWithPhonetic}

%%%%%%%%%% 旷 %%%%%%%%%%
\subsection*{旷}\addcontentsline{loh}{figure}{旷 \dpy{kuang4}}

\begin{EntryWithPhonetic}{旷}{kuang4}{7}{⽇}
  \definition*{s.}{Sobrenome: Kuang}
  \definition{adj.}{vasto; espaçoso | livre de preocupações e ideias mesquinhas | folgado}
  \definition{v.}{negligenciar ou desperdiçar | estar ausente de | desperdiçar; abandonar; negligenciar}
\end{EntryWithPhonetic}

\begin{EntryWithPhonetic}{旷课}{kuang4/ke4}{7,10}{⽇,⾔}[HSK 7-9]
  \definition{v.+compl.}{matar aula; cabular aula; faltar às aulas}
\end{EntryWithPhonetic}

\begin{EntryWithPhonetic}{旷野}{kuang4ye3}{7,11}{⽇,⾥}
  \definition{s.}{região selvagem; as planícies abertas}
\end{EntryWithPhonetic}

%%%%%%%%%% 矿 %%%%%%%%%%
\subsection*{矿}\addcontentsline{loh}{figure}{矿 \dpy{kuang4}}

\begin{EntryWithPhonetic}{矿}{kuang4}{8}{⽯}[HSK 6]
  \definition[个,座]{s.}{depósito de minério | minério | mina}
\end{EntryWithPhonetic}

\begin{EntryWithPhonetic}{矿藏}{kuang4cang2}{8,17}{⽯,⾋}[HSK 7-9]
  \definition{s.}{depósito mineral; recurso mineral; termo genérico para todos os tipos de minerais enterrados no subsolo}
\end{EntryWithPhonetic}

\begin{EntryWithPhonetic}{矿泉水}{kuang4quan2shui3}{8,9,4}{⽯,⽔,⽔}[HSK 4]
  \definition[瓶,杯,口]{s.}{água mineral de nascente}
\end{EntryWithPhonetic}

%%%%%%%%%% 框 %%%%%%%%%%
\subsection*{框}\addcontentsline{loh}{figure}{框 \dpy{kuang4}}

\begin{EntryWithPhonetic}{框}{kuang4}{10}{⽊}[HSK 7-9]
  \definition{s.}{moldura; estojo | caixa; bloco}
  \definition{v.}{Obsoleto: desenhar uma moldura ao redor; adicionar linhas ao redor do texto e das imagens | Obsoleto: restringir; confinar; conter; amarrar; colocar em uma camisa de força}
\end{EntryWithPhonetic}

\begin{EntryWithPhonetic}{框架}{kuang4jia4}{10,9}{⽊,⽊}[HSK 7-9]
  \definition[个,副,种,套]{s.}{moldura; estrutura; na construção civil, as estruturas são formadas por conexões como vigas e colunas | estrutura; estrutura básica de um sistema, texto, etc.; metáfora para a organização e estrutura das coisas}
\end{EntryWithPhonetic}

%%%%%%%%%% 亏 %%%%%%%%%%
\subsection*{亏}\addcontentsline{loh}{figure}{亏 \dpy{kui1}}

\begin{EntryWithPhonetic}{亏}{kui1}{3}{⼆}[HSK 5]
  \definition{adv.}{felizmente; por sorte; graças a | contrariamente, expressando sarcasmo}
  \definition{s.}{prejuízo; perda; déficit | perda; dano; ferida}
  \definition{v.}{perder dinheiro, etc.; ter um déficit; ter prejuízo | ter falta de; ser deficiente; carecer de | tratar injustamente; causar prejuízo; trair a confiança}
\end{EntryWithPhonetic}

\begin{EntryWithPhonetic}{亏本}{kui1/ben3}{3,5}{⼆,⽊}[HSK 7-9]
  \definition{v.+compl.}{estar no vermelho; perder o capital; perder dinheiro (nos negócios)}
\end{EntryWithPhonetic}

\begin{EntryWithPhonetic}{亏损}{kui1sun3}{3,10}{⼆,⼿}[HSK 7-9]
  \definition{v.}{perder; esgotar; ter déficit; as despesas excedem as receitas | desidratar; enfraquecer; o corpo está enfraquecido devido a danos ou falta de nutrição}
\end{EntryWithPhonetic}

%%%%%%%%%% 葵 %%%%%%%%%%
\subsection*{葵}\addcontentsline{loh}{figure}{葵 \dpy{kui2}}

\begin{EntryWithPhonetic}{葵}{kui2}{12}{⾋}
  \definition*{s.}{Sobrenome: Kui}
  \definition[朵]{s.}{certas ervas de flores grandes}
\end{EntryWithPhonetic}

\begin{EntryWithPhonetic}{葵花}{kui2hua1}{12,7}{⾋,⾋}
  \definition{s.}{girassol (flor)}
\end{EntryWithPhonetic}

%%%%%%%%%% 昆 %%%%%%%%%%
\subsection*{昆}\addcontentsline{loh}{figure}{昆 \dpy{kun1}}

\begin{EntryWithPhonetic}{昆}{kun1}{8}{⽇}
  \definition*{s.}{Sobrenome: Kun}
  \definition{s.}{irmão mais velho | descendentes; filhos}
\end{EntryWithPhonetic}

\begin{EntryWithPhonetic}{昆虫}{kun1chong2}{8,6}{⽇,⾍}[HSK 7-9]
  \definition[只,种,个,群,堆]{s.}{inseto; uma classe de artrópodes}
\end{EntryWithPhonetic}

%%%%%%%%%% 捆 %%%%%%%%%%
\subsection*{捆}\addcontentsline{loh}{figure}{捆 \dpy{kun3}}

\begin{EntryWithPhonetic}{捆}{kun3}{10}{⼿}[HSK 7-9]
  \definition{clas.}{feixe; maço; materiais usados para amarrar}
  \definition{s.}{coisas que estão agrupadas}
  \definition{v.}{amarrar; prender; agrupar | amarrar; acorrentar; algemar | agrupar; enfardar}
  \seealsoref{捆儿}{kun3r5}
\end{EntryWithPhonetic}

\begin{EntryWithPhonetic}{捆儿}{kun3r5}{10,2}{⼿,⼉}
  \definition{s.}{coisas que estão agrupadas}
  \seealsoref{捆}{kun3}
\end{EntryWithPhonetic}

%%%%%%%%%% 困 %%%%%%%%%%
\subsection*{困}\addcontentsline{loh}{figure}{困 \dpy{kun4}}

\begin{EntryWithPhonetic}{困}{kun4}{7}{⼞}[HSK 3]
  \definition{adj.}{cansado; exausto; fatigado | difícil; complicado; difícil e penoso; pobre e miserável | sonolento; com sono; cansado, com vontade de dormir}
  \definition{v.}{ficar encalhado; estar em apuros; preso em dificuldades e sofrimentos ou limitado por circunstâncias e condições que não pode escapar | cercar; envolver; imobilizar; controlar dentro de um determinado limite | dormir}
\end{EntryWithPhonetic}

\begin{EntryWithPhonetic}{困惑}{kun4huo4}{7,12}{⼞,⼼}[HSK 7-9]
  \definition{adj.}{em um quebra-cabeça; me sinto confuso e sem saber o que fazer}
  \definition{v.}{sentir-se perplexo; ser intrigante o porquê, e eu não saber o que fazer}
\end{EntryWithPhonetic}

\begin{EntryWithPhonetic}{困境}{kun4jing4}{7,14}{⼞,⼟}[HSK 7-9]
  \definition{s.}{dilema; situação difícil; apuro}
\end{EntryWithPhonetic}

\begin{EntryWithPhonetic}{困难}{kun4nan5}{7,10}{⼞,⾫}[HSK 3]
  \definition{adj.}{dificuldades financeiras; circunstâncias difíceis | complicado; complexo; difícil; árduo; a situação é complexa e há muitos obstáculos}
  \definition[种]{s.}{dificuldade; situação difícil; problemas ou situações difíceis de resolver no trabalho e na vida}
\end{EntryWithPhonetic}

\begin{EntryWithPhonetic}{困扰}{kun4rao3}{7,7}{⼞,⼿}[HSK 5]
  \definition{v.}{perturbar; deixar perplexo; perseguir}
\end{EntryWithPhonetic}

%%%%%%%%%% 扩 %%%%%%%%%%
\subsection*{扩}\addcontentsline{loh}{figure}{扩 \dpy{kuo4}}

\begin{EntryWithPhonetic}{扩}{kuo4}{6}{⼿}[HSK 7-9]
  \definition{v.}{expandir; ampliar; estender; alargar}
\end{EntryWithPhonetic}

\begin{EntryWithPhonetic}{扩大}{kuo4da4}{6,3}{⼿,⼤}[HSK 4]
  \definition{v.}{ampliar; expandir; estender; alargar}
\end{EntryWithPhonetic}

\begin{EntryWithPhonetic}{扩建}{kuo4jian4}{6,8}{⼿,⼵}[HSK 7-9]
  \definition{v.}{expandir; ampliar; ampliar a escala original do edifício ou a escala da área}
\end{EntryWithPhonetic}

\begin{EntryWithPhonetic}{扩散}{kuo4san4}{6,12}{⼿,⽁}[HSK 7-9]
  \definition{v.}{espalhar; difundir; dispersar; proliferar}
\end{EntryWithPhonetic}

\begin{EntryWithPhonetic}{扩展}{kuo4zhan3}{6,10}{⼿,⼫}[HSK 4]
  \definition{v.}{esticar; expandir; estender; espalhar}
\end{EntryWithPhonetic}

\begin{EntryWithPhonetic}{扩张}{kuo4zhang1}{6,7}{⼿,⼸}[HSK 7-9]
  \definition{v.}{expandir; aumentar; estender; espalhar; engrandecer | dilatar (dilatação vascular)}
\end{EntryWithPhonetic}

%%%%%%%%%% 括 %%%%%%%%%%
\subsection*{括}\addcontentsline{loh}{figure}{括 \dpy{kuo4}}

\begin{EntryWithPhonetic}{括}{kuo4}{9}{⼿}
  \definition{v.}{unir (músculos, etc.); contrair | incluir | adicionar colchetes a | amarrar; empacotar}
\end{EntryWithPhonetic}

\begin{EntryWithPhonetic}{括号}{kuo4hao4}{9,5}{⼿,⼝}[HSK 4]
  \definition{s.}{chaves, colchetes e parênteses (em fórmulas aritméticas ou algébricas, os símbolos que indicam a combinação e a ordem de vários números ou termos) | colchetes e parênteses usados como um tipo de sinal de pontuação para mostrar a parte explicativa de uma passagem em um texto}
\end{EntryWithPhonetic}

\begin{EntryWithPhonetic}{括弧}{kuo4hu2}{9,8}{⼿,⼸}[HSK 7-9]
  \definition{s.}{parênteses; também podem se referir a indicadores}
\end{EntryWithPhonetic}

%%%%%%%%%% 阔 %%%%%%%%%%
\subsection*{阔}\addcontentsline{loh}{figure}{阔 \dpy{kuo4}}

\begin{EntryWithPhonetic}{阔}{kuo4}{12}{⾨}[HSK 6]
  \definition{adj.}{amplo; amplo; vasto | rico | longo, no sentido de ``há muito tempo'' | vazio; impraticável}
\end{EntryWithPhonetic}

\begin{EntryWithPhonetic}{阔绰}{kuo4chuo4}{12,11}{⾨,⽷}[HSK 7-9]
  \definition{adj.}{ostentoso; generoso com dinheiro; extravagante; luxuoso}
\end{EntryWithPhonetic}

%%%%% EOF %%%%%

