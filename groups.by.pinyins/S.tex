%%%
%%% S
%%%
\section*{S}\addcontentsline{toc}{section}{S}\addcontentsline{loh}{figure}{\#\#\#\#\#\#\#\# S}

%%%%%%%%%% 撒 %%%%%%%%%%
\subsection*{撒}\addcontentsline{loh}{figure}{撒 \dpy{sa1}}

\begin{EntryWithPhonetic}{撒}{sa1}{15}{⼿}[HSK 7-9]
  \definition{v.}{lançar; soltar; deixar escapar; liberar | abandonar todas as restrições; deixar"-se levar; tentar usá"-lo ou exibi"-lo o máximo possível}
  \seeref{sa3}
\end{EntryWithPhonetic}

\begin{EntryWithPhonetic}{撒旦}{sa1dan4}{15,5}{⼿,⽇}
  \definition*{s.}{Satanás, que significa 抵挡, sinônimo do diabo nas histórias bíblicas; um termo cristão para alguém que se opõe especificamente a Deus e é um inimigo Dele | Satã; Diabo}
  \seealsoref{抵挡}{di3dang3}
\end{EntryWithPhonetic}

\begin{EntryWithPhonetic}{撒旦主义}{sa1dan4 zhu3yi4}{15,5,5,3}{⼿,⽇,⼂,⼂}
  \definition*{s.}{Satanismo}
\end{EntryWithPhonetic}

\begin{EntryWithPhonetic}{撒但}{sa1dan4}{15,7}{⼿,⼈}
  \variantof{撒旦}
  \seealsoref{撒旦}{sa1dan4}
\end{EntryWithPhonetic}

\begin{EntryWithPhonetic}{撒谎}{sa1/huang3}{15,11}{⼿,⾔}[HSK 7-9]
  \definition{v.+compl.}{mentir; contar uma mentira}
\end{EntryWithPhonetic}

%%%%%%%%%% 洒 %%%%%%%%%%
\subsection*{洒}\addcontentsline{loh}{figure}{洒 \dpy{sa3}}

\begin{EntryWithPhonetic}{洒}{sa3}{9}{⽔}[HSK 5]
  \definition{adj.}{natural e sem restrições; confortável (sem restrições)}
  \definition{v.}{derramar; espalhar; borrifar; salpicar; fazer com que (água ou outra coisa) caia de forma dispersa | derramar; cair de forma dispersa}
\end{EntryWithPhonetic}

\begin{EntryWithPhonetic}{洒水}{sa3shui3}{9,4}{⽔,⽔}
  \definition{v.}{borrifar}
\end{EntryWithPhonetic}

%%%%%%%%%% 撒 %%%%%%%%%%
\subsection*{撒}\addcontentsline{loh}{figure}{撒 \dpy{sa3}}

\begin{EntryWithPhonetic}{撒}{sa3}{15}{⼿}
  \definition*{s.}{Sobrenome: Sa}
  \definition{v.}{espalhar; polvilhar; difundir; lançar | derramar; deixar cair}
\end{EntryWithPhonetic}

%%%%%%%%%% 飒 %%%%%%%%%%
\subsection*{飒}\addcontentsline{loh}{figure}{飒 \dpy{sa4}}

\begin{EntryWithPhonetic}{飒}{sa4}{9}{⾵}
  \definition{adj.}{(mulheres) natural e desenfreada; elegante; valente}
  \definition{interj.}{Onomatopéia: farfalhar; sussurrar | Onomatopéia: som do vento}
  \definition{v.}{murchar}
\end{EntryWithPhonetic}

\begin{EntryWithPhonetic}{飒飒}{sa4sa4}{9,9}{⾵,⾵}
  \definition{s.}{o farfalhar | sussurro | murmúrio (do vento nas árvores, o mar, etc.)}
\end{EntryWithPhonetic}

%%%%%%%%%% 塞 %%%%%%%%%%
\subsection*{塞}\addcontentsline{loh}{figure}{塞 \dpy{sai1}}

\begin{EntryWithPhonetic}{塞}{sai1}{13}{⼟}[HSK 6]
  \definition{s.}{rolha; plugue}
  \definition{v.}{encher; conectar; preencher; espremer; bloquear | superar (para comparação)}
\end{EntryWithPhonetic}

%%%%%%%%%% 赛 %%%%%%%%%%
\subsection*{赛}\addcontentsline{loh}{figure}{赛 \dpy{sai4}}

\begin{EntryWithPhonetic}{赛}{sai4}{14}{⾙}[HSK 6]
  \definition*{s.}{Sobrenome: Sai}
  \definition{s.}{jogo; partida; competição | sacrifício; cerimônia de sacrifício; antigamente, sacrifícios eram feitos para agradecer aos deuses por suas dádivas}
  \definition{v.}{ter uma competição (comparando alto e baixo, forte e fraco) | superar; ser comparável a; comparar com}
\end{EntryWithPhonetic}

\begin{EntryWithPhonetic}{赛场}{sai4chang3}{14,6}{⾙,⼟}[HSK 6]
  \definition{s.}{local de competição; arena; ringue; terreno | campo (para competição de atletismo) | pista de corrida}
\end{EntryWithPhonetic}

\begin{EntryWithPhonetic}{赛车}{sai4che1}{14,4}{⾙,⾞}[HSK 7-9]
  \definition{s.}{veículo de corrida; carro de corrida; bicicletas, motocicletas ou carros de corrida | corridas de carros}
  \definition{v.}{correr; disputar uma corrida}
\end{EntryWithPhonetic}

\begin{EntryWithPhonetic}{赛跑}{sai4pao3}{14,12}{⾙,⾜}[HSK 7-9]
  \definition{v.}{correr; disputar uma corrida; esportes que testam a velocidade de corrida incluem provas de curta, média, longa e ultra"-longa distância, além de corridas com barreiras, revezamentos, corridas com obstáculos e corridas de \emph{cross"-country}}
\end{EntryWithPhonetic}

%%%%%%%%%% 三 %%%%%%%%%%
\subsection*{三}\addcontentsline{loh}{figure}{三 \dpy{san1}}

\begin{EntryWithPhonetic}{三}{san1}{3}{⼀}[HSK 1]
  \definition*{s.}{Sobrenome: San}
  \definition{num.}{três; 3 | muitos; vários; mais de dois; referindo"-se a muitos ou à maioria | alguns; poucos; menos; não muitos}
\end{EntryWithPhonetic}

\begin{EntryWithPhonetic}{三番五次}{san1fan1-wu3ci4}{3,12,4,6}{⼀,⽥,⼆,⽋}[HSK 7-9]
  \definition{expr.}{repetidamente; de novo e de novo; várias e várias vezes; diversas vezes}
  \synonymref{翻来覆去}{fan1lai2-fu4qu4}
  \synonymref{接二连三}{jie1'er4-lian2san1}
\end{EntryWithPhonetic}

\begin{EntryWithPhonetic}{三角}{san1jiao3}{3,7}{⼀,⾓}[HSK 7-9]
  \definition{adj.}{tripartido; que constitui uma relação tripartite}
  \definition[个,些]{s.}{triângulo; coisas triangulares | trigonometria, abreviação de 三角学}
  \seealsoref{三角学}{san1jiao3 xue2}
\end{EntryWithPhonetic}

\begin{EntryWithPhonetic}{三角恋爱}{san1jiao3lian4'ai4}{3,7,10,10}{⼀,⾓,⼼,⽖}
  \definition[场]{s.}{triângulo amoroso | triângulo eterno}
\end{EntryWithPhonetic}

\begin{EntryWithPhonetic}{三角学}{san1jiao3 xue2}{3,7,8}{⼀,⾓,⼦}
  \definition{s.}{trigonometria; um ramo da matemática que estuda principalmente as funções trigonométricas e suas propriedades, bem como suas aplicações em geometria}[我三角学学得很好。===Sou muito bom em trigonometria.]
\end{EntryWithPhonetic}

\begin{EntryWithPhonetic}{三轮车}{san1lun2che1}{3,8,4}{⼀,⾞,⾞}
  \definition{s.}{triciclo}
\end{EntryWithPhonetic}

\begin{EntryWithPhonetic}{三明治}{san1ming2zhi4}{3,8,8}{⼀,⽇,⽔}[HSK 6]
  \definition[个,些,块]{s.}{Empréstimo linguístico: sanduíche, \emph{sandwich}}
\end{EntryWithPhonetic}

\begin{EntryWithPhonetic}{三维}{san1wei2}{3,11}{⼀,⽷}[HSK 7-9]
  \definition{s.}{três dimensões; 3D; tridimensional}[我们生活在三维空间。===Vivemos em um espaço tridimensional.]
\end{EntryWithPhonetic}

%%%%%%%%%% 伞 %%%%%%%%%%
\subsection*{伞}\addcontentsline{loh}{figure}{伞 \dpy{san3}}

\begin{EntryWithPhonetic}{伞}{san3}{6}{⼈}[HSK 4]
  \definition*{s.}{Sobrenome: San}
  \definition[把]{s.}{guarda"-chuva; proteção contra chuva ou sol | algo que tem o formato de um guarda"-chuva}
\end{EntryWithPhonetic}

%%%%%%%%%% 散 %%%%%%%%%%
\subsection*{散}\addcontentsline{loh}{figure}{散 \dpy{san3}}

\begin{EntryWithPhonetic}{散}{san3}{12}{⽁}[HSK 5]
  \definition{adj.}{disperso; fragmentado; não integrado}
  \definition{s.}{medicamento em forma de pó}
  \definition{v.}{divergir; espalhar"-se; separar"-se; soltar"-se; não se manter unido;  desintegrar}
  \seeref{san4}
\end{EntryWithPhonetic}

\begin{EntryWithPhonetic}{散文}{san3wen2}{12,4}{⽁,⽂}[HSK 5]
  \definition[篇,种]{s.}{ensaio; prosa; gênero literário, na antiguidade, referia"-se a textos em prosa, em oposição à poesia e à prosa paralela; atualmente, refere"-se a obras literárias que não sejam poesia, teatro ou romance, incluindo ensaios, contos, crônicas, relatos de viagem, etc.}
\end{EntryWithPhonetic}

\begin{EntryWithPhonetic}{散}{san4}{12}{⽁}[HSK 4]
  \definition{v.}{quebrar; fragmentar; dispersar | dar; distribuir; disseminar; divulgar | dissipar; deixar sai  | terminar um acordo ou contrato; demitir}
  \seeref{san3}
\end{EntryWithPhonetic}

\begin{EntryWithPhonetic}{散布}{san4bu4}{12,5}{⽁,⼱}[HSK 7-9]
  \definition{v.}{espalhar; distribuir; disseminar | espalhar; propagar}
\end{EntryWithPhonetic}

\begin{EntryWithPhonetic}{散步}{san4/bu4}{12,7}{⽁,⽌}[HSK 3]
  \definition{v.+compl.}{dar uma volta; dar um passeio; dar uma caminhada}
\end{EntryWithPhonetic}

\begin{EntryWithPhonetic}{散发}{san4fa4}{12,5}{⽁,⼜}[HSK 7-9]
  \definition{v.}{emitir; difundir; enviar; divulgar | emitir; distribuir; dar}
\end{EntryWithPhonetic}

\begin{EntryWithPhonetic}{散心}{san4/xin1}{12,4}{⽁,⼼}
  \definition{v.+compl.}{aliviar o tédio | desfrutar de uma diversão | estar despreocupado}
\end{EntryWithPhonetic}

%%%%%%%%%% 丧 %%%%%%%%%%
\subsection*{丧}\addcontentsline{loh}{figure}{丧 \dpy{sang1}}

\begin{EntryWithPhonetic}{丧}{sang1}{8}{⼗}
  \definition{adj.}{decepcionado; deprimido; desanimado}
  \definition{v.}{perder | desanimar; frustrar}
  \seeref{sang4}
\end{EntryWithPhonetic}

\begin{EntryWithPhonetic}{丧钟}{sang1zhong1}{8,9}{⼗,⾦}
  \definition{s.}{sentença de morte}
\end{EntryWithPhonetic}

%%%%%%%%%% 桑 %%%%%%%%%%
\subsection*{桑}\addcontentsline{loh}{figure}{桑 \dpy{sang1}}

\begin{EntryWithPhonetic}{桑}{sang1}{10}{⽊}
  \definition*{s.}{Sobrenome: Sang}
  \definition[棵]{s.}{amoreira}
\end{EntryWithPhonetic}

\begin{EntryWithPhonetic}{桑巴舞}{sang1ba1wu3}{10,4,14}{⽊,⼰,⾇}
  \definition{s.}{samba}
\end{EntryWithPhonetic}

\begin{EntryWithPhonetic}{桑拿}{sang1na2}{10,10}{⽊,⼿}[HSK 7-9]
  \definition{s.}{Empréstimo Linguístico: sauna; banho turco}
\end{EntryWithPhonetic}

\begin{EntryWithPhonetic}{桑树}{sang1shu4}{10,9}{⽊,⽊}
  \definition{s.}{amoreira, suas folhas são utilizadas para alimentar bichos-da-seda}
\end{EntryWithPhonetic}

%%%%%%%%%% 嗓 %%%%%%%%%%
\subsection*{嗓}\addcontentsline{loh}{figure}{嗓 \dpy{sang3}}

\begin{EntryWithPhonetic}{嗓}{sang3}{13}{⼝}
  \definition{s.}{garganta; laringe | voz}
  \seealsoref{嗓儿}{sang3r5}
\end{EntryWithPhonetic}

\begin{EntryWithPhonetic}{嗓儿}{sang3r5}{13,2}{⼝,⼉}
  \definition{s.}{garganta}
\end{EntryWithPhonetic}

\begin{EntryWithPhonetic}{嗓子}{sang3zi5}{13,3}{⼝,⼦}[HSK 7-9]
  \definition[副,个]{s.}{garganta; laringe | voz}
\end{EntryWithPhonetic}

%%%%%%%%%% 丧 %%%%%%%%%%
\subsection*{丧}\addcontentsline{loh}{figure}{丧 \dpy{sang4}}

\begin{EntryWithPhonetic}{丧}{sang4}{8}{⼗}
  \definition{adj.}{decepcionado | desanimado}
  \definition{v.}{estar enlutado (do cônjuge etc.) | morrer}
  \seeref{sang1}
\end{EntryWithPhonetic}

\begin{EntryWithPhonetic}{丧身}{sang4shen1}{8,7}{⼗,⾝}
  \definition{v.}{morrer; perder a vida}
  \seealsoref{丧生}{sang4/sheng1}
\end{EntryWithPhonetic}

\begin{EntryWithPhonetic}{丧生}{sang4/sheng1}{8,5}{⼗,⽣}[HSK 7-9]
  \definition{v.+compl.}{morrer; encontrar a morte; perder a vida; ser morto}
  \seealsoref{丧身}{sang4shen1}
\end{EntryWithPhonetic}

\begin{EntryWithPhonetic}{丧失}{sang4shi1}{8,5}{⼗,⼤}[HSK 6]
  \definition{v.}{perder (algo que se tem)}
\end{EntryWithPhonetic}

%%%%%%%%%% 骚 %%%%%%%%%%
\subsection*{骚}\addcontentsline{loh}{figure}{骚 \dpy{sao1}}

\begin{EntryWithPhonetic}{骚}{sao1}{12}{⾺}
  \definition*{s.}{Abreviação de Li Sao (Encontrando a Tristeza), um poema do poeta e estadista do século IV a.C. Qu Yuan (屈原)}
  \definition{adj.}{coquete; (de uma mulher) lasciva | masculino (de alguns animais domésticos)}
  \definition{s.}{escritos literários; geralmente se refere à poesia | o cheiro de urina; mau cheiro}
  \definition{v.}{perturbar}
  \seealsoref{屈原}{qu1yuan2}
\end{EntryWithPhonetic}

\begin{EntryWithPhonetic}{骚乱}{sao1luan4}{12,7}{⾺,⼄}[HSK 7-9]
  \definition{s.}{rebelião; perturbação; motim; confusão}
  \definition{v.}{criar perturbação; estar em meio a uma turbulência; causar problemas}
\end{EntryWithPhonetic}

\begin{EntryWithPhonetic}{骚扰}{sao1rao3}{12,7}{⾺,⼿}[HSK 7-9]
  \definition{v.}{assediar; molestar; perturbar}
\end{EntryWithPhonetic}

%%%%%%%%%% 扫 %%%%%%%%%%
\subsection*{扫}\addcontentsline{loh}{figure}{扫 \dpy{sao3}}

\begin{EntryWithPhonetic}{扫}{sao3}{6}{⼿}[HSK 4]
  \definition{v.}{varrer; limpar | passar rapidamente ao longo ou sobre; varrer | juntar tudo | Computação: scanear}
  \seeref{sao4}
\end{EntryWithPhonetic}

\begin{EntryWithPhonetic}{扫除}{sao3chu2}{6,9}{⼿,⾩}[HSK 7-9]
  \definition{s.}{limpeza; arrumação}
  \definition{v.}{limpar; arrumar | limpar; remover; eliminar}
\end{EntryWithPhonetic}

\begin{EntryWithPhonetic}{扫描}{sao3miao2}{6,11}{⼿,⼿}[HSK 7-9]
  \definition{v.}{digitalizar | dar uma olhada rápida; percorrer (o olhar, etc.) | utilizar software especializado para inspecionar e pesquisar (dados, vírus, etc. em computadores)}
\end{EntryWithPhonetic}

\begin{EntryWithPhonetic}{扫墓}{sao3/mu4}{6,13}{⼿,⼟}[HSK 7-9]
  \definition{v.+compl.}{limpar sepulturas e prestar homenagens aos mortos; também se refere à realização de atividades comemorativas nos túmulos dos mártires}
\end{EntryWithPhonetic}

\begin{EntryWithPhonetic}{扫兴}{sao3/xing4}{6,6}{⼿,⼋}[HSK 7-9]
  \definition{v.+compl.}{sentir"-se desapontado; ter o ânimo abalado; quando você está se sentindo feliz, algo desagradável pode abalar seu ânimo}
\end{EntryWithPhonetic}

%%%%%%%%%% 嫂 %%%%%%%%%%
\subsection*{嫂}\addcontentsline{loh}{figure}{嫂 \dpy{sao3}}

\begin{EntryWithPhonetic}{嫂}{sao3}{12}{⼥}
  \definition[个,位,名,些]{s.}{esposa do irmão mais velho; cunhada | irmã (uma forma de tratamento para uma mulher casada, mais ou menos da mesma idade)}
\end{EntryWithPhonetic}

\begin{EntryWithPhonetic}{嫂子}{sao3zi5}{12,3}{⼥,⼦}[HSK 7-9]
  \definition[个,名,位]{s.}{esposa do irmão mais velho; cunhada}
\end{EntryWithPhonetic}

%%%%%%%%%% 扫 %%%%%%%%%%
\subsection*{扫}\addcontentsline{loh}{figure}{扫 \dpy{sao4}}

\begin{EntryWithPhonetic}{扫}{sao4}{6}{⼿}
  \definition{s.}{elemento formadore de palavra}
  \seeref{sao3}
  \seealsoref{扫帚}{sao4zhou5}
\end{EntryWithPhonetic}

\begin{EntryWithPhonetic}{扫帚}{sao4zhou5}{6,8}{⼿,⼱}
  \definition[把,个]{s.}{vassoura; ferramenta de varredura feita de varas de bambu, etc., maior que uma vassora}
\end{EntryWithPhonetic}

%%%%%%%%%% 色 %%%%%%%%%%
\subsection*{色}\addcontentsline{loh}{figure}{色 \dpy{se4}}

\begin{EntryWithPhonetic}{色}{se4}{6}{⾊}[HSK 4][Kangxi 139]
  \definition*{s.}{Sobrenome: Se}
  \definition[种]{s.}{cor | aparência; semblante; expressão | tipo; gênero; descrição | cena; cenário;  paisagem | qualidade (de metais preciosos, mercadorias, etc.) | aparência feminina; beleza feminina | erotismo; apetite sexual; luxúria; desejo sexual}
  \seeref{shai3}
\end{EntryWithPhonetic}

\begin{EntryWithPhonetic}{色彩}{se4cai3}{6,11}{⾊,⼺}[HSK 4]
  \definition[种,丝]{s.}{cor; matiz; tonalidade | cor; sabor; característica; metáfora para um determinado estado de espírito ou tendência de pensamento}
\end{EntryWithPhonetic}

\begin{EntryWithPhonetic}{色狼}{se4lang2}{6,10}{⾊,⽝}
  \definition*{s.}{Sátiro}
  \definition{adj.}{lascivo; lobo; pervertido; isso se refere a uma pessoa que persegue mulheres com ganância e as agride sexualmente de forma brutal}
\end{EntryWithPhonetic}

%%%%%%%%%% 森 %%%%%%%%%%
\subsection*{森}\addcontentsline{loh}{figure}{森 \dpy{sen1}}

\begin{EntryWithPhonetic}{森}{sen1}{12}{⽊}
  \definition{adj.}{cheio de árvores | multitudinário; em multidões | escuro; sombrio}
\end{EntryWithPhonetic}

\begin{EntryWithPhonetic}{森林}{sen1lin2}{12,8}{⽊,⽊}[HSK 4]
  \definition[片,座,处]{s.}{floresta; bosque; normalmente, refere"-se a uma grande área de árvores em crescimento; na silvicultura, refere"-se a um grande número de árvores que crescem em uma área razoavelmente grande de terra, juntamente com os animais e outras plantas}
\end{EntryWithPhonetic}

%%%%%%%%%% 僧 %%%%%%%%%%
\subsection*{僧}\addcontentsline{loh}{figure}{僧 \dpy{seng1}}

\begin{EntryWithPhonetic}{僧}{seng1}{14}{⼈}
  \definition*{s.}{Sobrenome: Seng}
  \definition[位,名,个]{s.}{monge Budista, abreviação de 僧伽}
  \seealsoref{僧伽}{seng1qie2}
\end{EntryWithPhonetic}

\begin{EntryWithPhonetic}{僧伽}{seng1qie2}{14,7}{⼈,⼈}
  \definition{s.}{sangha ou sanga (Budismo) | a comunidade monástica | monge}
\end{EntryWithPhonetic}

\begin{EntryWithPhonetic}{僧人}{seng1ren2}{14,2}{⼈,⼈}[HSK 7-9]
  \definition{s.}{monge budista | monge}
\end{EntryWithPhonetic}

%%%%%%%%%% 杀 %%%%%%%%%%
\subsection*{杀}\addcontentsline{loh}{figure}{杀 \dpy{sha1}}

\begin{EntryWithPhonetic}{杀}{sha1}{6}{⽊}[HSK 5]
  \definition{adv.}{em extremo; excessivamente; usado após um verbo, indica grau intenso}
  \definition{v.}{matar; abater; esquartejar | lutar; entrar em batalha | enfraquecer; reduzir; diminuir | decolar; neutralizar}
\end{EntryWithPhonetic}

\begin{EntryWithPhonetic}{杀毒}{sha1 du2}{6,9}{⽊,⽏}[HSK 5]
  \definition{s.}{Computação: antivírus}
  \definition{v.}{esterilizar; desinfetar | Computação: eliminar um vírus}
\end{EntryWithPhonetic}

\begin{EntryWithPhonetic}{杀害}{sha1hai4}{6,10}{⽊,⼧}[HSK 7-9]
  \definition{v.}{assassinar; massacrar; trucidar; chacinar; matar (uma pessoa) por motivos ilegítimos}
\end{EntryWithPhonetic}

\begin{EntryWithPhonetic}{杀气}{sha1qi4}{6,4}{⽊,⽓}
  \definition{s.}{espírito assassino | aura de morte}
  \definition{v.}{desabafar a raiva de alguém}
\end{EntryWithPhonetic}

\begin{EntryWithPhonetic}{杀手}{sha1shou3}{6,4}{⽊,⼿}[HSK 7-9]
  \definition{s.}{assassino; homicida | pessoa magistral (de certo tipo) | Esporte: jogador formidável | assassino de aluguel}
\end{EntryWithPhonetic}

%%%%%%%%%% 沙 %%%%%%%%%%
\subsection*{沙}\addcontentsline{loh}{figure}{沙 \dpy{sha1}}

\begin{EntryWithPhonetic}{沙}{sha1}{7}{⽔}
  \definition*{s.}{Sobrenome: Sha}
  \definition{adj.}{granulado; em pó | rouco}[我今天感冒了,嗓音有点沙哑。===Estou resfriado hoje e minha voz está um pouco rouca.]
  \definition[车,把,袋,吨]{s.}{areia; cascalho; grânulo; pó}
\end{EntryWithPhonetic}

\begin{EntryWithPhonetic}{沙发}{sha1fa1}{7,5}{⽔,⼜}[HSK 3]
  \definition[套,组,个,张]{s.}{sofá; assentos com molas ou espuma plástica espessa, etc., com apoios de braços em ambos os lados}
\end{EntryWithPhonetic}

\begin{EntryWithPhonetic}{沙龙}{sha1long2}{7,5}{⽔,⿓}[HSK 7-9]
  \definition{s.}{Empréstimo linguístico: \emph{salon}; salão; sala de estar}
\end{EntryWithPhonetic}

\begin{EntryWithPhonetic}{沙漠}{sha1mo4}{7,13}{⽔,⽔}[HSK 5]
  \definition[个,片]{s.}{deserto; superfície totalmente coberta por areia, sem água corrente, clima seco e vegetação escassa}
\end{EntryWithPhonetic}

\begin{EntryWithPhonetic}{沙滩}{sha1tan1}{7,13}{⽔,⽔}[HSK 7-9]
  \definition[片,个]{s.}{areia; praia; praia arenosa; terreno formado por depósitos de areia dentro ou perto da água}
\end{EntryWithPhonetic}

\begin{EntryWithPhonetic}{沙糖}{sha1tang2}{7,16}{⽔,⽶}
  \variantof{砂糖}
\end{EntryWithPhonetic}

\begin{EntryWithPhonetic}{沙特}{sha1te4}{7,10}{⽔,⽜}
  \definition*{s.}{Saudita | Arábia Saudita, abreviação de 沙特阿拉伯}
  \seealsoref{沙特阿拉伯}{sha1te4 a1la1bo2}
\end{EntryWithPhonetic}

\begin{EntryWithPhonetic}{沙特阿拉伯}{sha1te4 a1la1bo2}{7,10,7,8,7}{⽔,⽜,⾩,⼿,⼈}
  \definition*{s.}{Arábia Saudita}
\end{EntryWithPhonetic}

\begin{EntryWithPhonetic}{沙鱼}{sha1yu2}{7,8}{⽔,⿂}
  \variantof{鲨鱼}
\end{EntryWithPhonetic}

\begin{EntryWithPhonetic}{沙子}{sha1zi5}{7,3}{⽔,⼦}[HSK 3]
  \definition[粒,把,堆,袋,车]{s.}{areia; grão; pequenas pedras | \emph{pellets}; grãos pequenos; coisas parecidas com areia}
\end{EntryWithPhonetic}

%%%%%%%%%% 纱 %%%%%%%%%%
\subsection*{纱}\addcontentsline{loh}{figure}{纱 \dpy{sha1}}

\begin{EntryWithPhonetic}{纱}{sha1}{7}{⽷}[HSK 7-9]
  \definition[层,块儿]{s.}{fio; filamentos finos e soltos, fiados a partir de algodão, cânhamo, etc., podem ser torcidos para formar fios ou tecidos para produzir tecido | gaze; tecidos confeccionados com fios de urdidura e trama muito finos | produtos têxteis; produtos como telas de janela | nomes de classes de certos tecidos}
\end{EntryWithPhonetic}

%%%%%%%%%% 刹 %%%%%%%%%%
\subsection*{刹}\addcontentsline{loh}{figure}{刹 \dpy{sha1}}

\begin{EntryWithPhonetic}{刹}{sha1}{8}{⼑}
  \definition{v.}{acionar o(s) freio(s); frear; brecar}
  \seeref{cha4}
\end{EntryWithPhonetic}

\begin{EntryWithPhonetic}{刹车}{sha1/che1}{8,4}{⼑,⾞}[HSK 7-9]
  \definition{s.}{freios; o mecanismo que impede o veículo de se mover}
  \definition{v.+compl.}{frear; pisar nos freios; utilizar os freios ou outros mecanismos para parar o movimento do veículo ou interromper o funcionamento da máquina | desligar uma máquina; parar uma máquina cortando a energia | desligue ou desconectar a fonte de alimentação para interromper o funcionamento da máquina | interromper (um projeto, etc.); uma metáfora para interromper algo imediatamente}
\end{EntryWithPhonetic}

\begin{EntryWithPhonetic}{刹多罗}{sha1duo1luo2}{8,6,8}{⼑,⼣,⽹}
  \definition*{s.}{Kshatara, sânscrito ``ksetra''}
\end{EntryWithPhonetic}

%%%%%%%%%% 砂 %%%%%%%%%%
\subsection*{砂}\addcontentsline{loh}{figure}{砂 \dpy{sha1}}

\begin{EntryWithPhonetic}{砂}{sha1}{9}{⽯}
  \variantof{沙}
\end{EntryWithPhonetic}

\begin{EntryWithPhonetic}{砂糖}{sha1tang2}{9,16}{⽯,⽶}[HSK 7-9]
  \definition{s.}{açúcar granulado}
\end{EntryWithPhonetic}

%%%%%%%%%% 莎 %%%%%%%%%%
\subsection*{莎}\addcontentsline{loh}{figure}{莎 \dpy{sha1}}

\begin{EntryWithPhonetic}{莎}{sha1}{10}{⾋}
  \definition{s.}{em nomes pessoais e de lugares | cigarra | fonético ``sha'' usado na transliteração}
  \seeref{suo1}
\end{EntryWithPhonetic}

\begin{EntryWithPhonetic}{莎莎舞}{sha1sha1wu3}{10,10,14}{⾋,⾋,⾇}
  \definition{s.}{salsa (dança)}
\end{EntryWithPhonetic}

%%%%%%%%%% 鲨 %%%%%%%%%%
\subsection*{鲨}\addcontentsline{loh}{figure}{鲨 \dpy{sha1}}

\begin{EntryWithPhonetic}{鲨}{sha1}{15}{⿂}
  \definition[只,条]{s.}{tubarão}
\end{EntryWithPhonetic}

\begin{EntryWithPhonetic}{鲨鱼}{sha1yu2}{15,8}{⿂,⿂}[HSK 7-9]
  \definition[只,群,条]{s.}{tubarão}
\end{EntryWithPhonetic}

%%%%%%%%%% 啥 %%%%%%%%%%
\subsection*{啥}\addcontentsline{loh}{figure}{啥 \dpy{sha2}}

\begin{EntryWithPhonetic}{啥}{sha2}{11}{⼝}
  \definition{pron.}{Dialeto: O que?; equivalente a 什么}
\end{EntryWithPhonetic}

%%%%%%%%%% 傻 %%%%%%%%%%
\subsection*{傻}\addcontentsline{loh}{figure}{傻 \dpy{sha3}}

\begin{EntryWithPhonetic}{傻}{sha3}{13}{⼈}[HSK 5]
  \definition{adj.}{estúpido; confuso; burro; idiota; inflexível; (ação ou pensamento) mecânico}
\end{EntryWithPhonetic}

\begin{EntryWithPhonetic}{傻瓜}{sha3gua1}{13,5}{⼈,⽠}[HSK 7-9]
  \definition[个,群]{adj.}{tolo; idiota; cabeça-dura; simplório; uma pessoa estúpida ou lenta para reagir; termo frequentemente usado como insulto ou piada}
  \definition{v.}{enganar | iludir | lograr}
\end{EntryWithPhonetic}

\begin{EntryWithPhonetic}{傻眼}{sha3yan3}{13,11}{⼈,⽬}
  \definition{adj.}{estupefato | atordoado}
\end{EntryWithPhonetic}

%%%%%%%%%% 嗄 %%%%%%%%%%
\subsection*{嗄}\addcontentsline{loh}{figure}{嗄 \dpy{sha4}}

\begin{EntryWithPhonetic}{嗄}{sha4}{13}{⼝}
  \definition{adj.}{rouco}
  \seeref{a2}
\end{EntryWithPhonetic}

%%%%%%%%%% 筛 %%%%%%%%%%
\subsection*{筛}\addcontentsline{loh}{figure}{筛 \dpy{shai1}}

\begin{EntryWithPhonetic}{筛}{shai1}{12}{⽵}[HSK 7-9]
  \definition{s.}{peneira; coador; tela}
  \definition{v.}{peneirar; crivar; triar | aquecer o vinho sobre uma fogueira; aquecer uma panela de vinho em fogo baixo | servir vinho, chá, etc. | bater; golpear | Dialeto: bater (o gongo)}
\end{EntryWithPhonetic}

\begin{EntryWithPhonetic}{筛选}{shai1xuan3}{12,9}{⽵,⾡}[HSK 7-9]
  \definition{v.}{selecionar; filtrar; eliminar por seleção; eliminar o que é ruim e selecionar o que é bom}
\end{EntryWithPhonetic}

%%%%%%%%%% 色 %%%%%%%%%%
\subsection*{色}\addcontentsline{loh}{figure}{色 \dpy{shai3}}

\begin{EntryWithPhonetic}{色}{shai3}{6}{⾊}[Kangxi 139]
  \definition[4]{s.}{cor; (~儿) tem o mesmo significado que 色, usado em algumas palavras faladas}
  \seeref{se4}
\end{EntryWithPhonetic}

%%%%%%%%%% 晒 %%%%%%%%%%
\subsection*{晒}\addcontentsline{loh}{figure}{晒 \dpy{shai4}}

\begin{EntryWithPhonetic}{晒}{shai4}{10}{⽇}[HSK 4]
  \definition{v.}{(sol) brilhar sobre | aquecer"-se; secar ao sol; colocar algo sob a luz do sol para secar | ignorar (alguém) | mostrar; divulgar o conteúdo de sua vida privada na Internet}
\end{EntryWithPhonetic}

\begin{EntryWithPhonetic}{晒干}{shai4gan1}{10,3}{⽇,⼲}
  \definition{v.}{secar ao sol}
\end{EntryWithPhonetic}

\begin{EntryWithPhonetic}{晒太阳}{shai4 tai4yang2}{10,4,6}{⽇,⼤,⾩}[HSK 7-9]
  \definition{v.}{colocar algo ao sol (por exemplo, para secar); estar ao sol (aquecer"-se ou tomar banho de sol, etc.); absorver luz e calor sob a luz solar}
\end{EntryWithPhonetic}

%%%%%%%%%% 山 %%%%%%%%%%
\subsection*{山}\addcontentsline{loh}{figure}{山 \dpy{shan1}}

\begin{EntryWithPhonetic}{山}{shan1}{3}{⼭}[HSK 1][Kangxi 46]
  \definition*{s.}{Sobrenome: Shan}
  \definition[座]{s.}{colina; maciço; montanha | qualquer coisa que se assemelhe a uma montanha | arbustos nos quais os bichos"-da"-seda tecem seus casulos; referindo"-se a casulos de bicho"-da"-seda | eco; metáfora para um som muito alto}
\end{EntryWithPhonetic}

\begin{EntryWithPhonetic}{山川}{shan1chuan1}{3,3}{⼭,⼮}[HSK 7-9]
  \definition{s.}{montanhas e rios; paisagem}
\end{EntryWithPhonetic}

\begin{EntryWithPhonetic}{山顶}{shan1ding3}{3,8}{⼭,⾴}[HSK 7-9]
  \definition{s.}{topo de uma colina; cume de uma montanha; o pico de uma montanha; o topo da montanha}[他们距山顶还有100米远。===Eles ainda estavam a 100 metros do cume.]
\end{EntryWithPhonetic}

\begin{EntryWithPhonetic}{山东}{shan1dong1}{3,5}{⼭,⼀}
  \definition*{s.}{Província de Shandong (Shantung) no nordeste da China}
\end{EntryWithPhonetic}

\begin{EntryWithPhonetic}{山峰}{shan1feng1}{3,10}{⼭,⼭}[HSK 6]
  \definition[座,个]{s.}{pico (montanha); topo alto e pontudo da montanha}
\end{EntryWithPhonetic}

\begin{EntryWithPhonetic}{山冈}{shan1gang1}{3,4}{⼭,⼌}[HSK 7-9]
  \definition[座]{s.}{colina baixa; pequeno morro}
\end{EntryWithPhonetic}

\begin{EntryWithPhonetic}{山谷}{shan1gu3}{3,7}{⼭,⾕}[HSK 6]
  \definition[条,个]{s.}{vale; desfiladeiro; ravina; a área baixa e estreita entre duas montanhas geralmente tem riachos no meio}
\end{EntryWithPhonetic}

\begin{EntryWithPhonetic}{山岭}{shan1ling3}{3,8}{⼭,⼭}[HSK 7-9]
  \definition[座,片,条]{s.}{cadeia de montanhas; cordilheira | crista; montanhas altas contínuas}
\end{EntryWithPhonetic}

\begin{EntryWithPhonetic}{山路}{shan1lu4}{3,13}{⼭,⾜}[HSK 7-9]
  \definition{s.}{trilha de montanha | estrada de montanha}
\end{EntryWithPhonetic}

\begin{EntryWithPhonetic}{山坡}{shan1po1}{3,8}{⼭,⼟}[HSK 6]
  \definition[个,座,片]{s.}{encosta; encosta da montanha; a inclinação entre o topo da montanha e o terreno plano}
\end{EntryWithPhonetic}

\begin{EntryWithPhonetic}{山区}{shan1qu1}{3,4}{⼭,⼖}[HSK 5]
  \definition[片]{s.}{área montanhosa; região montanhosa | colina; serra; montanha | distrito montanhoso}
\end{EntryWithPhonetic}

\begin{EntryWithPhonetic}{山体}{shan1ti3}{3,7}{⼭,⼈}
  \definition{s.}{forma de uma montanha}
\end{EntryWithPhonetic}

\begin{EntryWithPhonetic}{山西}{shan1xi1}{3,6}{⼭,⾑}
  \definition*{s.}{Província de Shanxi (Shansi) no norte da China entre Hebei e Shaanxi, abreviada como 晋 (Shansi)}
  \seealsoref{晋}{jin4}
\end{EntryWithPhonetic}

\begin{EntryWithPhonetic}{山羊}{shan1yang2}{3,6}{⼭,⽺}
  \definition{s.}{cabra | (ginástica) cavalo de salto de pequeno porte}
\end{EntryWithPhonetic}

\begin{EntryWithPhonetic}{山阴}{shan1yin1}{3,6}{⼭,⾩}
  \definition*{s.}{Condado de Shanyin em Shuozhou, Shanxi}
  \definition{s.}{lado norte (ou sombreado) de uma montanha}
\end{EntryWithPhonetic}

\begin{EntryWithPhonetic}{山寨}{shan1zhai4}{3,14}{⼭,⼧}[HSK 7-9]
  \definition{s.}{fortaleza na montanha (especialmente de bandidos); aldeia fortificada na montanha; vila fortificada na colina | Humorístico: “bandido”; imitador | Figurativo: imitação (de produtos); falsificação}
\end{EntryWithPhonetic}

%%%%%%%%%% 删 %%%%%%%%%%
\subsection*{删}\addcontentsline{loh}{figure}{删 \dpy{shan1}}

\begin{EntryWithPhonetic}{删}{shan1}{7}{⼑}[HSK 7-9]
  \definition{v.}{excluir; omitir; remover}
\end{EntryWithPhonetic}

\begin{EntryWithPhonetic}{删除}{shan1chu2}{7,9}{⼑,⾩}[HSK 7-9]
  \definition{v.}{apagar; cortar; eliminar; excluir; riscar; descartar}
\end{EntryWithPhonetic}

%%%%%%%%%% 扇 %%%%%%%%%%
\subsection*{扇}\addcontentsline{loh}{figure}{扇 \dpy{shan1}}

\begin{EntryWithPhonetic}{扇}{shan1}{10}{⼾}[HSK 5]
  \definition{s.}{ventilar; agitar um leque para fazer o ar circular | dar um tapa; bater com a palma da mão | bater asas; esvoaçar | incitar; instigar; estimular; agitar}
  \seeref{shan4}
\end{EntryWithPhonetic}

%%%%%%%%%% 煽 %%%%%%%%%%
\subsection*{煽}\addcontentsline{loh}{figure}{煽 \dpy{shan1}}

\begin{EntryWithPhonetic}{煽}{shan1}{14}{⽕}
  \definition{v.}{abanar (fogo); agitar um leque ou outra folha | incitar; instigar; agitar | vangloriar"-se de; esbanjar prêmios em}
\end{EntryWithPhonetic}

\begin{EntryWithPhonetic}{煽动}{shan1dong4}{14,6}{⽕,⼒}[HSK 7-9]
  \definition{v.}{instigar; incitar (alguém a fazer coisas ruins); agitar; inflamar}
\end{EntryWithPhonetic}

%%%%%%%%%% 闪 %%%%%%%%%%
\subsection*{闪}\addcontentsline{loh}{figure}{闪 \dpy{shan3}}

\begin{EntryWithPhonetic}{闪}{shan3}{5}{⾨}[HSK 4]
  \definition*{s.}{Sobrenome: Shan}
  \definition{s.}{relâmpago}
  \definition{v.}{esquivar"-se; desviar; sair do caminho | torcer; distender | surgir de repente | cintilar; brilhar | deixar para trás; abandonar | (corpo) oscilar dramaticamente}
\end{EntryWithPhonetic}

\begin{EntryWithPhonetic}{闪存盘}{shan3cun2pan2}{5,6,11}{⾨,⼦,⽫}
  \definition{s.}{unidade de memória \emph{USB} | cartão de memória}
  \seealsoref{优盘}{you1pan2}
\end{EntryWithPhonetic}

\begin{EntryWithPhonetic}{闪电}{shan3dian4}{5,5}{⾨,⽥}[HSK 4]
  \definition[道]{s.}{relâmpago; descargas elétricas entre nuvens ou entre nuvens e o solo}
  \seealsoref{雷电}{lei2dian4}
\end{EntryWithPhonetic}

\begin{EntryWithPhonetic}{闪烁}{shan3shuo4}{5,9}{⾨,⽕}[HSK 7-9]
  \definition{adj.}{vago; evasivo; não comprometido}
  \definition{v.}{cintilar; brilhar; reluzir; tremeluzir}
\end{EntryWithPhonetic}

\begin{EntryWithPhonetic}{闪耀}{shan3yao4}{5,20}{⾨,⽻}
  \definition{v.}{brilhar; cintilar; resplandecer; irradiar; a luz oscila, às vezes brilhante, às vezes fraca | brilhar; irradiar luz deslumbrante}
\end{EntryWithPhonetic}

%%%%%%%%%% 掺 %%%%%%%%%%
\subsection*{掺}\addcontentsline{loh}{figure}{掺 \dpy{shan3}}

\begin{EntryWithPhonetic}{掺}{shan3}{11}{⼿}
  \definition{v.}{misturar; mesclar | conter; reter}
  \seeref{can4}
  \seeref{chan1}
\end{EntryWithPhonetic}

%%%%%%%%%% 单 %%%%%%%%%%
\subsection*{单}\addcontentsline{loh}{figure}{单 \dpy{shan4}}

\begin{EntryWithPhonetic}{单}{shan4}{8}{⼗}
  \definition*{s.}{Sobrenome: Shan}
  \definition{s.}{material de tecido de largura simples (dupla) | número singular (plural)}
  \seeref{chan2}
  \seeref{dan1}
\end{EntryWithPhonetic}

%%%%%%%%%% 扇 %%%%%%%%%%
\subsection*{扇}\addcontentsline{loh}{figure}{扇 \dpy{shan4}}

\begin{EntryWithPhonetic}{扇}{shan4}{10}{⼾}[HSK 5]
  \definition{clas.}{usado para portas, janelas, etc.}
  \definition[把]{s.}{leque | folha; algo em forma de placa ou folha}
  \seeref{shan1}
\end{EntryWithPhonetic}

\begin{EntryWithPhonetic}{扇子}{shan4zi5}{10,3}{⼾,⼦}[HSK 5]
  \definition[把,个]{s.}{leque; abano; abanador; utensílios que produzem vento ao serem agitados}
\end{EntryWithPhonetic}

%%%%%%%%%% 善 %%%%%%%%%%
\subsection*{善}\addcontentsline{loh}{figure}{善 \dpy{shan4}}

\begin{EntryWithPhonetic}{善}{shan4}{12}{⼝}[HSK 7-9]
  \definition*{s.}{Sobrenome: Shan}
  \definition{adj.}{bom; bem | bom; satisfatório | gentil; amigável | familiar}
  \definition{adv.}{bom; bem}
  \definition{s.}{boa ação; ato benevolente; coisas boas}
  \definition{v.}{fazer sucesso; fazer bem; fazer acontecer | ser bom em; ser especialista (versado) em | ser apto a}
  \antonymref{恶}{e4}
\end{EntryWithPhonetic}

\begin{EntryWithPhonetic}{善良}{shan4liang2}{12,7}{⼝,⾉}[HSK 4]
  \definition{adj.}{de bom coração; bom e honesto; de bom coração e cheio de boa vontade}
\end{EntryWithPhonetic}

\begin{EntryWithPhonetic}{善意}{shan4yi4}{12,13}{⼝,⼼}[HSK 7-9]
  \definition[片]{s.}{boa vontade;  boa intenção;  benevolência; bondade}
\end{EntryWithPhonetic}

\begin{EntryWithPhonetic}{善于}{shan4yu2}{12,3}{⼝,⼆}[HSK 4]
  \definition{adv./v.}{ser bom em; ser hábil em}
\end{EntryWithPhonetic}

%%%%%%%%%% 禅 %%%%%%%%%%
\subsection*{禅}\addcontentsline{loh}{figure}{禅 \dpy{shan4}}

\begin{EntryWithPhonetic}{禅}{shan4}{12}{⽰}
  \definition{v.}{abdicar e entregar a coroa a outra pessoa}
  \seeref{chan2}
\end{EntryWithPhonetic}

%%%%%%%%%% 擅 %%%%%%%%%%
\subsection*{擅}\addcontentsline{loh}{figure}{擅 \dpy{shan4}}

\begin{EntryWithPhonetic}{擅}{shan4}{16}{⼿}
  \definition{adv.}{sem autorização; arbitrariamente | fazer algo por conta própria}
  \definition{v.}{ser bom em; ser especialista em | arrogar-se a si mesmo; fazer algo por conta própria | reivindicar arbitrariamente; ir além do escopo e ajir arbitrariamente}
\end{EntryWithPhonetic}

\begin{EntryWithPhonetic}{擅长}{shan4chang2}{16,4}{⼿,⾧}[HSK 7-9]
  \definition{v.}{ser bom em; ser especialista em; ser habilidoso em; ter um talento especial em determinada área}
\end{EntryWithPhonetic}

\begin{EntryWithPhonetic}{擅自}{shan4zi4}{16,6}{⼿,⾃}[HSK 7-9]
  \definition{adv.}{arbitrariamente; sem permissão ou autorização; agir por iniciativa própria em assuntos que estão fora da sua alçada}
\end{EntryWithPhonetic}

%%%%%%%%%% 膳 %%%%%%%%%%
\subsection*{膳}\addcontentsline{loh}{figure}{膳 \dpy{shan4}}

\begin{EntryWithPhonetic}{膳}{shan4}{16}{⾁}
  \definition{s.}{refeições; comida; alimentação}
\end{EntryWithPhonetic}

\begin{EntryWithPhonetic}{膳食}{shan4shi2}{16,9}{⾁,⾷}[HSK 7-9]
  \definition{s.}{comida; refeições; refeições consumidas todos os dias}
\end{EntryWithPhonetic}

%%%%%%%%%% 赡 %%%%%%%%%%
\subsection*{赡}\addcontentsline{loh}{figure}{赡 \dpy{shan4}}

\begin{EntryWithPhonetic}{赡}{shan4}{17}{⾙}[HSK 7-9]
  \definition*{s.}{Sobrenome: Shan}
  \definition{adj.}{Literário: suficiente; abundante; adequado | Literário: refinado, rico}
  \definition{v.}{apoiar; prover para}
\end{EntryWithPhonetic}

\begin{EntryWithPhonetic}{赡养}{shan4yang3}{17,9}{⾙,⼋}[HSK 7-9]
  \definition{v.}{apoiar; prover para; prover as necessidades básicas refere"-se especificamente à ajuda que os filhos dão aos pais, tanto materialmente quanto no dia a dia}
  \synonymref{抚养}{fu3yang3}
\end{EntryWithPhonetic}

%%%%%%%%%% 伤 %%%%%%%%%%
\subsection*{伤}\addcontentsline{loh}{figure}{伤 \dpy{shang1}}

\begin{EntryWithPhonetic}{伤}{shang1}{6}{⼈}[HSK 3]
  \definition*{s.}{Sobrenome: Shang}
  \definition[处]{s.}{ferida; ferimento}
  \definition{v.}{ferir; machucar | ter os sentimentos feridos | estar angustiado | enjoar de algo; desenvolver aversão a algo | ser prejudicial a; entravar}
\end{EntryWithPhonetic}

\begin{EntryWithPhonetic}{伤残}{shang1can2}{6,9}{⼈,⽍}[HSK 7-9]
  \definition{adj.}{ferido e incapacitado; deficiente; mutilado; desfigurado; aleijado | (um produto, etc.) defeituoso; falho; danificado}
  \synonymref{残疾}{can2ji5}
  \antonymref{完好}{wan2hao3}
\end{EntryWithPhonetic}

\begin{EntryWithPhonetic}{伤感}{shang1gan3}{6,13}{⼈,⼼}[HSK 7-9]
  \definition{adj.}{doente de coração; sentimental; piegas; triste}
\end{EntryWithPhonetic}

\begin{EntryWithPhonetic}{伤害}{shang1hai4}{6,10}{⼈,⼧}[HSK 4]
  \definition[种]{v.}{ferir; prejudicar; machucar; magoar; causar danos físicos ou mentais}
  \synonymref{摧毁}{cui1hui3}
  \synonymref{虐待}{nve4dai4}
  \synonymref{破坏}{po4huai4}
  \synonymref{受伤}{shou4/shang1}
  \synonymref{损害}{sun3hai4}
  \synonymref{损伤}{sun3shang1}
  \synonymref{危害}{wei1hai4}
  \synonymref{危险}{wei1xian3}
  \antonymref{爱护}{ai4hu4}
  \antonymref{保护}{bao3hu4}
  \antonymref{关怀}{guan1huai2}
  \antonymref{救援}{jiu4yuan2}
  \antonymref{治病}{zhi4bing4}
  \antonymref{治疗}{zhi4liao2}
\end{EntryWithPhonetic}

\begin{EntryWithPhonetic}{伤痕}{shang1hen2}{6,11}{⼈,⽧}[HSK 7-9]
  \definition[道,处]{s.}{cicatriz; ferida; também se refere a uma marca deixada após um objeto ter sido danificado | cicatriz; ferida; metáfora para trauma psicológico}
\end{EntryWithPhonetic}

\begin{EntryWithPhonetic}{伤口}{shang1kou3}{6,3}{⼈,⼝}[HSK 6]
  \definition[处]{s.}{corte; ferida; onde a pele, os músculos, etc. são feridos, rompidos ou onde são realizadas aberturas cirúrgicas}
\end{EntryWithPhonetic}

\begin{EntryWithPhonetic}{伤脑筋}{shang1 nao3jin1}{6,10,12}{⼈,⾁,⽵}[HSK 7-9]
  \definition{adj.}{complicado; problemático; incômodo; que causa dor de cabeça em alguém; descreve uma situação como difícil e que exige muita reflexão}
\end{EntryWithPhonetic}

\begin{EntryWithPhonetic}{伤势}{shang1shi4}{6,8}{⼈,⼒}[HSK 7-9]
  \definition{s.}{condição de uma lesão (ou ferida) | o estado de uma lesão (ou ferida)}
  \synonymref{伤害}{shang1hai4}
  \synonymref{伤口}{shang1kou3}
\end{EntryWithPhonetic}

\begin{EntryWithPhonetic}{伤亡}{shang1wang2}{6,3}{⼈,⼇}[HSK 6]
  \definition{s.}{ferimentos e mortes; feridos e mortos; pessoas feridas e mortas; baixas}
  \definition{v.}{ser ferido e morto}
  \synonymref{伤害}{shang1hai4}
  \synonymref{死亡}{si3wang2}
  \synonymref{英勇}{ying1yong3}
\end{EntryWithPhonetic}

\begin{EntryWithPhonetic}{伤心}{shang1/xin1}{6,4}{⼈,⼼}[HSK 3]
  \definition{v.+compl.}{estar triste; lamentar; estar com o coração partido; sentir"-se triste por causa de infortúnio ou decepção}
  \antonymref{快乐}{kuai4le4}
\end{EntryWithPhonetic}

\begin{EntryWithPhonetic}{伤员}{shang1yuan2}{6,7}{⼈,⼝}[HSK 6]
  \definition[名,位,个]{s.}{Exército: pessoal ferido; os feridos}
  \synonymref{病人}{bing4ren2}
\end{EntryWithPhonetic}

%%%%%%%%%% 汤 %%%%%%%%%%
\subsection*{汤}\addcontentsline{loh}{figure}{汤 \dpy{shang1}}

\begin{EntryWithPhonetic}{汤}{shang1}{6}{⽔}
  \definition{s.}{correnteza forte}
  \seeref{tang1}
\end{EntryWithPhonetic}

%%%%%%%%%% 商 %%%%%%%%%%
\subsection*{商}\addcontentsline{loh}{figure}{商 \dpy{shang1}}

\begin{EntryWithPhonetic}{商}{shang1}{11}{⼝}
  \definition*{s.}{Dinastia Shang (1600--1046 a.C.) | Shang, nome da estrela da constelação do coração entre as vinte e oito constelações | Sobrenome: Shang}
  \definition{s.}{comércio; negócio; a atividade econômica de compra e venda de mercadorias | comerciante; negociante; comerciante; empresário; pessoas que compram e vendem mercadorias | (matemática) quociente;  o resultado de uma operação de divisão em aritmética | uma nota da antiga escala chinesa de cinco tons, correspondente a 2 na notação musical numerada}
  \definition{v.}{discutir; consultar; trocar ideias}
\end{EntryWithPhonetic}

\begin{EntryWithPhonetic}{商标}{shang1biao1}{11,9}{⼝,⽊}[HSK 5]
  \definition[个]{s.}{marca; marca registrada; \emph{trademark}; marca ou símbolo (desenho, padrão, texto, etc.) gravado ou impresso na superfície ou embalagem de um produto, para diferenciá-lo de outros produtos semelhantes}
\end{EntryWithPhonetic}

\begin{EntryWithPhonetic}{商场}{shang1chang3}{11,6}{⼝,⼟}[HSK 1]
  \definition[家]{s.}{mercado; shopping center; loja de departamentos; loja de grande área com uma variedade completa de produtos | o mundo dos negócios; referindo"-se ao mundo dos negócios | mercado; mercado composto por várias lojas reunidas em um ou vários edifícios interligados}
\end{EntryWithPhonetic}

\begin{EntryWithPhonetic}{商城}{shang1cheng2}{11,9}{⼝,⼟}[HSK 6]
  \definition{s.}{um mercado; um centro comercial; um \emph{shopping center}; refere"-se a um complexo comercial contíguo com um grande espaço de construção}
\end{EntryWithPhonetic}

\begin{EntryWithPhonetic}{商店}{shang1dian4}{11,8}{⼝,⼴}[HSK 1]
  \definition[间,家,个]{s.}{loja; armazém; local de venda de mercadorias em recinto fechado}
\end{EntryWithPhonetic}

\begin{EntryWithPhonetic}{商贩}{shang1fan4}{11,8}{⼝,⾙}[HSK 7-9]
  \definition{s.}{varejista; vendedor ambulante; comerciante; pequenos comerciantes que vendem ou comercializam produtos}
\end{EntryWithPhonetic}

\begin{EntryWithPhonetic}{商贾}{shang1gu3}{11,10}{⼝,⾙}[HSK 7-9]
  \definition{s.}{Literário: comerciante}
\end{EntryWithPhonetic}

\begin{EntryWithPhonetic}{商量}{shang1liang5}{11,12}{⼝,⾥}[HSK 2]
  \definition{v.}{consultar; discutir; conversar sobre; discutir e trocar opiniões}
\end{EntryWithPhonetic}

\begin{EntryWithPhonetic}{商贸}{shang1mao4}{11,9}{⼝,⾙}
  \definition{s.}{comércio}
\end{EntryWithPhonetic}

\begin{EntryWithPhonetic}{商品}{shang1pin3}{11,9}{⼝,⼝}[HSK 3]
  \definition[种,个,件,批]{s.}{bens; mercadoria; \emph{merchande}; os produtos do trabalho produzidos para troca têm a dupla natureza de valor de uso e valor; as mercadorias incorporam diferentes relações de produção em diferentes sistemas sociais}
\end{EntryWithPhonetic}

\begin{EntryWithPhonetic}{商人}{shang1ren2}{11,2}{⼝,⼈}[HSK 2]
  \definition[位,名]{s.}{comerciante; mercador; empresário; homem de negócios; pessoas que trabalham com a distribuição de mercadorias}
\end{EntryWithPhonetic}

\begin{EntryWithPhonetic}{商讨}{shang1tao3}{11,5}{⼝,⾔}[HSK 7-9]
  \definition{v.}{discutir; deliberar sobre; trocar ideias e discutir para resolver problemas maiores e mais complexos}
\end{EntryWithPhonetic}

\begin{EntryWithPhonetic}{商务}{shang1wu4}{11,5}{⼝,⼒}[HSK 4]
  \definition[种,类,项]{s.}{negócios; assuntos de negócios; assuntos comerciais}
\end{EntryWithPhonetic}

\begin{EntryWithPhonetic}{商业}{shang1ye4}{11,5}{⼝,⼀}[HSK 3]
  \definition[个,种]{s.}{barganha; negócio; comércio; atividade econômica que circula mercadorias por meio de compra e venda}
\end{EntryWithPhonetic}

%%%%%%%%%% 上 %%%%%%%%%%
\subsection*{上}\addcontentsline{loh}{figure}{上 \dpy{shang3}}

\begin{EntryWithPhonetic}{上}{shang3}{3}{⼀}
  \definition{s.}{tom descendente-ascendente; significa o segundo tom dos quatro tons do mandarim, e também se refere ao terceiro tom do mandarim padrão}
  \seeref{shang4}
\end{EntryWithPhonetic}

\begin{EntryWithPhonetic}{上声}{shang3sheng1}{3,7}{⼀,⼠}
  \definition{s.}{tom descendente e ascendente | terceiro tom no mandarim moderno}
\end{EntryWithPhonetic}

%%%%%%%%%% 赏 %%%%%%%%%%
\subsection*{赏}\addcontentsline{loh}{figure}{赏 \dpy{shang3}}

\begin{EntryWithPhonetic}{赏}{shang3}{12}{⾙}[HSK 4]
  \definition*{s.}{Sobrenome: Shang}
  \definition{s.}{recompensa; prêmio}
  \definition{v.}{conceder (outorgar) uma recompensa; recompensar; premiar | admirar; desfrutar; apreciar; valorizar}
\end{EntryWithPhonetic}

\begin{EntryWithPhonetic}{赏赐}{shang3ci4}{12,12}{⾙,⾙}
  \definition{s.}{recompensa | prêmio}
  \definition{v.}{recompensar | premiar}
\end{EntryWithPhonetic}

\begin{EntryWithPhonetic}{赏心悦目}{shang3xin1yue4mu4}{12,4,10,5}{⾙,⼼,⼼,⽬}
  \definition{expr.}{``Aquece o coração e encanta os olhos.''; achar a paisagem agradável tanto aos olhos quanto à mente}
\end{EntryWithPhonetic}

%%%%%%%%%% 上 %%%%%%%%%%
\subsection*{上}\addcontentsline{loh}{figure}{上 \dpy{shang4}}

\begin{EntryWithPhonetic}{上}{shang4}{3}{⼀}[HSK 1]
  \definition{adj.}{mais recente; último; anterior; tempo ou a sequência anterior | superior; mais alto; melhor; indica uma posição elevada em termos de qualidade, nível, etc. | lugar elevado; posição superior}
  \definition{s.}{superior; acima; para cima; um lugar alto ou mais alto do que um determinado local | na superfície de um objeto; usado após um substantivo, indica a superfície de um objeto | indica estar dentro do escopo de algo; usado após um substantivo, indica que algo está dentro do âmbito de determinada coisa | indica um aspecto específico | antigamente, referia"-se ao imperador | usado após palavras que indicam idade, equivale a ``\dots 的时候'' | o primeiro nível da escala da música folclórica chinesa, usado como um símbolo de nota na notação musical, equivalente ao ``1'' na notação simplificada.}
  \definition{v.}{subir; montar; embarcar; entrar | ir para; partir para | estar ocupado (com trabalho, estudos, etc.) em um horário fixo; começar a trabalhar ou estudar na hora marcada, etc. | seguir em frente; prosseguir | encher; abastecer; servir; melhorar; aumentar | aparecer no palco; entrar | colocar algo em posição; ajustar; fixar; montar as duas partes de algo | aplicar; pintar; espalhar | ser registrado; ser publicado (em uma publicação) | atingir; ser suficiente (uma determinada quantidade ou grau) | submeter; enviar; apresentar; submeter à aprovação superior | ventilar; apertar; torcer | trazer; servir; colocar comida, pratos, chá e outras coisas na mesa para os convidados | indicar que uma ação tem um resultado | pesquisar na \emph{Internet} | emaranhar"-se; ficar emaranhado; enredar"-se}
  \definition{v.aux.}{usado após um verbo para indicar início e continuidade}
  \seeref{shang3}
  \seealsoref{的时候}{de5 shi2hou4}
  \antonymref{下}{xia4}
\end{EntryWithPhonetic}

\begin{EntryWithPhonetic}{上班}{shang4/ban1}{3,10}{⼀,⽟}[HSK 1]
  \definition{v.+compl.}{ir trabalhar; começar a trabalhar; estar de plantão; ir trabalhar no local de trabalho regular no horário especificado}
  \synonymref{加班}{jia1/ban1}
  \antonymref{下班}{xia4/ban1}
\end{EntryWithPhonetic}

\begin{EntryWithPhonetic}{上班族}{shang4 ban1 zu2}{3,10,11}{⼀,⽟,⽅}
  \definition[本]{s.}{trabalhadores de escritório (como grupo social)}
\end{EntryWithPhonetic}

\begin{EntryWithPhonetic}{上报}{shang4bao4}{3,7}{⼀,⼿}[HSK 7-9]
  \definition{v.}{aparecer nos jornais; ser publicado | reportar a um órgão superior; reportar à liderança; reportar"-se aos superiores}
  \synonymref{报到}{bao4/dao4}
\end{EntryWithPhonetic}

\begin{EntryWithPhonetic}{上边}{shang4bian5}{3,5}{⼀,⾡}[HSK 1]
  \definition{s.}{topo; acima; sobre; superior}
  \synonymref{上方}{shang4fang1}
  \antonymref{下边}{xia4bian5}
\end{EntryWithPhonetic}

\begin{EntryWithPhonetic}{上场}{shang4/chang3}{3,6}{⼀,⼟}[HSK 7-9]
  \definition{v.+compl.}{Esporte: entrar na quadra (ou campo); participar de uma competição | aparecer no palco; subir no palco; entrar em cena}
  \antonymref{退场}{tui4chang3}
\end{EntryWithPhonetic}

\begin{EntryWithPhonetic}{上车}{shang4che1}{3,4}{⼀,⾞}[HSK 1]
  \definition{v.}{entrar; subir (em um ônibus, trem, carro etc.)}
  \antonymref{下车}{xia4che1}
\end{EntryWithPhonetic}

\begin{EntryWithPhonetic}{上次}{shang4ci4}{3,6}{⼀,⽋}[HSK 1]
  \definition{adv.}{última vez}
\end{EntryWithPhonetic}

\begin{EntryWithPhonetic}{上当}{shang4/dang4}{3,6}{⼀,⼹}[HSK 6]
  \definition{v.+compl.}{ser enganado; ser ludibriado; morder a isca; cair nas mãos de alguém}
  \synonymref{受骗}{shou4/pian4}
  \antonymref{精明}{jing1ming2}
  \antonymref{警惕}{jing3ti4}
\end{EntryWithPhonetic}

\begin{EntryWithPhonetic}{上帝}{shang4di4}{3,9}{⼀,⼱}[HSK 6]
  \definition*{s.}{Deus; O Deus Supremo no Cristianismo | O Imperador do Céu; um deus na antiga crença chinesa que pode controlar tudo no mundo}
  \definition[个]{s.}{(figurado) cliente; metáfora para consumidores}
\end{EntryWithPhonetic}

\begin{EntryWithPhonetic}{上方}{shang4fang1}{3,4}{⼀,⽅}[HSK 7-9]
  \definition{s.}{acima; sobre; em cima de | superjacente}
  \seealsoref{下方}{xia4fang1}
\end{EntryWithPhonetic}

\begin{EntryWithPhonetic}{上访}{shang4fang3}{3,6}{⼀,⾔}
  \definition{v.}{buscar uma audiência com superiores (especialmente funcionários do governo) para fazer uma petição por algo}
\end{EntryWithPhonetic}

\begin{EntryWithPhonetic}{上岗}{shang4/gang3}{3,7}{⼀,⼭}[HSK 7-9]
  \definition{v.+compl.}{estar em período probatório | começar a trabalhar; assumir um cargo}
\end{EntryWithPhonetic}

\begin{EntryWithPhonetic}{上个月}{shang4ge4yue4}{3,3,4}{⼀,⼈,⽉}[HSK 4]
  \definition{s.}{mês passado; refere"-se à hora de um mês atrás, ou seja, o mês passado mais próximo da hora atual}
\end{EntryWithPhonetic}

\begin{EntryWithPhonetic}{上古}{shang4gu3}{3,5}{⼀,⼝}
  \definition{s.}{tempos antigos; eras remotas | antiguidade | tempos históricos antigos | o passado distante}
  \antonymref{现代}{xian4dai4}
\end{EntryWithPhonetic}

\begin{EntryWithPhonetic}{上海}{shang4hai3}{3,10}{⼀,⽔}
  \definition*{s.}{Município de Xangai (Shanghai), centro-leste da China}
\end{EntryWithPhonetic}

\begin{EntryWithPhonetic}{上火}{shang4/huo3}{3,4}{⼀,⽕}[HSK 7-9]
  \definition{v.+compl.}{ter dor de garganta | ter excesso de calor interno; na medicina tradicional chinesa, sintomas como prisão de ventre ou inflamação da mucosa nasal, da mucosa oral ou da conjuntiva são classificados como ``calor interno'' | ficar com raiva}
  \seealsoref{上火儿}{shang4huo3r5}
\end{EntryWithPhonetic}

\begin{EntryWithPhonetic}{上火儿}{shang4huo3r5}{3,4,2}{⼀,⽕,⼉}
  \definition{v.}{Dialeto: ficar com raiva; explodir}
\end{EntryWithPhonetic}

\begin{EntryWithPhonetic}{上级}{shang4ji2}{3,6}{⼀,⽷}[HSK 5]
  \definition[个,位]{s.}{nível superior; organização ou pessoa em nível superior; organizações ou pessoas de nível superior dentro do mesmo sistema organizacional}
  \synonymref{部属}{bu4shu3}
  \synonymref{上司}{shang4si5}
  \antonymref{部下}{bu4xia4}
\end{EntryWithPhonetic}

\begin{EntryWithPhonetic}{上课}{shang4/ke4}{3,10}{⼀,⾔}[HSK 1]
  \definition{v.+compl.}{frequentar aulas; ir às aulas; dar uma aula}
  \synonymref{上学}{shang4 xue2}
  \antonymref{下课}{xia4/ke4}
\end{EntryWithPhonetic}

\begin{EntryWithPhonetic}{上空}{shang4kong1}{3,8}{⼀,⽳}[HSK 7-9]
  \definition{s.}{no céu; acima da cabeça; no alto, no ar}
\end{EntryWithPhonetic}

\begin{EntryWithPhonetic}{上来}{shang4 lai5}{3,7}{⼀,⽊}[HSK 3]
  \definition{v.}{subir (para a minha localização) | estar no começo; começar; iniciar | surgir; de um lugar baixo para um lugar alto (o interlocutor está em um lugar alto) | usado após o verbo, indica que algo foi concluído com sucesso}
  \synonymref{上去}{shang4 qu5}
  \antonymref{下去}{xia4 qu5}
\end{EntryWithPhonetic}

\begin{EntryWithPhonetic}{上流}{shang4liu2}{3,10}{⼀,⽔}[HSK 7-9]
  \definition{adj.}{da classe alta; refinado | pertencente aos círculos superiores; anteriormente, referia"-se a pessoas de alto \emph{status} social}
  \definition{s.}{trecho superior (de um rio); a montante}
  \synonymref{崇高}{chong2gao1}
  \synonymref{高超}{gao1chao1}
  \synonymref{高贵}{gao1gui4}
  \synonymref{高尚}{gao1shang4}
\end{EntryWithPhonetic}

\begin{EntryWithPhonetic}{上楼}{shang4lou2}{3,13}{⼀,⽊}[HSK 4]
  \definition{v.}{subir as escadas; ir para o andar de cima}
\end{EntryWithPhonetic}

\begin{EntryWithPhonetic}{上门}{shang4 men2}{3,3}{⼀,⾨}[HSK 4]
  \definition{v.}{chamar; visitar; aparecer; ir ou vir para ver alguém; ir até a porta; ir até a casa de alguém | trancar a porta; fechar a porta durante a noite | casar"-se e morar com a família da noiva}
  \synonymref{拜访}{bai4fang3}
\end{EntryWithPhonetic}

\begin{EntryWithPhonetic}{上面}{shang4mian5}{3,9}{⼀,⾯}[HSK 3]
  \definition{s.}{uma posição mais alta que algo; uma posição acima/acima de algo | superfície do objeto | aspecto | a parte acima mencionada; a parte que vem primeiro na ordem; a parte de um artigo ou discurso que vem antes da presente | autoridades superiores | os mais velhos; a geração mais velha da família}
  \synonymref{方面}{fang1mian4}
  \synonymref{上方}{shang4fang1}
  \synonymref{上头}{shang4tou5}
  \antonymref{底下}{di3xia5}
  \antonymref{下面}{xia4mian5}
\end{EntryWithPhonetic}

\begin{EntryWithPhonetic}{上坡路}{shang4po1lu4}{3,8,13}{⼀,⼟,⾜}
  \definition{s.}{aclive | progresso | (fig.) tendência ascendente}
\end{EntryWithPhonetic}

\begin{EntryWithPhonetic}{上期}{shang4 qi1}{3,12}{⼀,⽉}[HSK 7-9]
  \definition{s.}{período anterior}
\end{EntryWithPhonetic}

\begin{EntryWithPhonetic}{上去}{shang4 qu5}{3,5}{⼀,⼛}[HSK 3]
  \definition{v.}{subir (a partir da minha localização) | ascender a um lugar (ou estado) considerado mais elevado (ou acima); usado depois de um verbo para indicar movimento, de baixo para cima ou de perto para longe}
  \synonymref{上来}{shang4 lai5}
  \antonymref{下来}{xia4 lai5}
\end{EntryWithPhonetic}

\begin{EntryWithPhonetic}{上任}{shang4/ren4}{3,6}{⼀,⼈}[HSK 7-9]
  \definition[班]{s.}{predecessor; ex"-funcionário}
  \definition{v.+compl.}{assumir o cargo; ocupar um cargo oficial; refere"-se à posse de autoridades}
  \synonymref{出任}{chu1ren4}
  \synonymref{就职}{jiu4/zhi2}
  \synonymref{就任}{jiu4ren4}
  \synonymref{上台}{shang4 tai2}
  \antonymref{辞职}{ci2/zhi2}
  \antonymref{离职}{li2/zhi2}
\end{EntryWithPhonetic}

\begin{EntryWithPhonetic}{上升}{shang4sheng1}{3,4}{⼀,⼗}[HSK 3]
  \definition{v.}{elevar; subir; mover"-se para cima; mover de baixo para cima; aumentar em nível, grau, quantidade, etc.}
  \synonymref{上涨}{shang4zhang3}
  \antonymref{回升}{hui2sheng1}
  \antonymref{降落}{jiang4luo4}
  \antonymref{落下}{luo4xia4}
  \antonymref{下降}{xia4jiang4}
\end{EntryWithPhonetic}

\begin{EntryWithPhonetic}{上市}{shang4 shi4}{3,5}{⼀,⼱}[HSK 6]
  \definition{v.}{listar; abrir o capital; ser listado (na bolsa de valores) | estar na estação; estar (aparecer) no mercado | ir ao mercado (para fazer compras)}
  \synonymref{问世}{wen4shi4}
\end{EntryWithPhonetic}

\begin{EntryWithPhonetic}{上述}{shang4shu4}{3,8}{⼀,⾡}[HSK 7-9]
  \definition{adj.}{mencionado anteriormente; supracitado; acima citado; conforme dito ou narrado acima}
  \synonymref{描述}{miao2shu4}
  \synonymref{摸索}{mo1suo3}
  \synonymref{试验}{shi4yan4}
\end{EntryWithPhonetic}

\begin{EntryWithPhonetic}{上司}{shang4si5}{3,5}{⼀,⼝}[HSK 7-9]
  \definition[位,名,个]{s.}{chefe; superior}
  \synonymref{上级}{shang4ji2}
  \antonymref{部属}{bu4shu3}
  \antonymref{部下}{bu4xia4}
\end{EntryWithPhonetic}

\begin{EntryWithPhonetic}{上诉}{shang4su4}{3,7}{⼀,⾔}[HSK 7-9]
  \definition{s.}{apelação (para um tribunal superior)}
  \definition{v.}{apresentar um recurso; instaurar um recurso}
\end{EntryWithPhonetic}

\begin{EntryWithPhonetic}{上台}{shang4 tai2}{3,5}{⼀,⼝}[HSK 6]
  \definition{v.}{aparecer no palco; subir na plataforma; ir para o palco ou pódio | assumir o poder; chegar (subir) ao poder; começar a assumir papéis de liderança ou a ganhar algum tipo de poder}
  \synonymref{上任}{shang4/ren4}
\end{EntryWithPhonetic}

\begin{EntryWithPhonetic}{上调}{shang4tiao2}{3,10}{⼀,⾔}[HSK 7-9]
  \definition{v.}{transferir (alguém) para um cargo de nível superior | transferir bens, fundos, etc. para uma unidade de nível superior | ajustar para cima | aumentar (os preços)}
  \antonymref{下调}{xia4tiao2}
\end{EntryWithPhonetic}

\begin{EntryWithPhonetic}{上头}{shang4tou2}{3,5}{⼀,⼤}
  \definition{v.}{(álcool, amor etc.) subir à cabeça; (uma ideia, uma música etc.) entrar na cabeça de alguém; capturar a atenção de alguém | Obsoleto: (uma garota no dia do seu casamento) começar a usar o cabelo preso em um coque (em vez de uma trança)}
  \seeref{shang4tou5}
  \synonymref{方面}{fang1mian4}
  \synonymref{上面}{shang4mian5}
\end{EntryWithPhonetic}

\begin{EntryWithPhonetic}{上头}{shang4tou5}{3,5}{⼀,⼤}[HSK 7-9]
  \definition{s.}{acima; em cima de; na superfície de; superior}
  \seeref{shang4tou2}
\end{EntryWithPhonetic}

\begin{EntryWithPhonetic}{上网}{shang4/wang3}{3,6}{⼀,⽹}[HSK 1]
  \definition{v.+compl.}{conectar"-se à \emph{Internet}; acessar a \emph{Internet}; entrar na \emph{Internet}; acessar a rede; refere"-se especificamente ao computador do usuário conectado à \emph{Internet} para pesquisar e consultar informações, etc.}
\end{EntryWithPhonetic}

\begin{EntryWithPhonetic}{上午}{shang4wu3}{3,4}{⼀,⼗}[HSK 1]
  \definition[个]{s.}{manhã; \emph{ante meridiem} (a.m.); geralmente refere"-se ao período entre a manhã e o meio"-dia}
  \antonymref{下午}{xia4wu3}
\end{EntryWithPhonetic}

\begin{EntryWithPhonetic}{上下}{shang4xia4}{3,3}{⼀,⼀}[HSK 5]
  \definition{adv.}{para cima e para baixo}
  \definition[顶]{s.}{alto e baixo | de cima para baixo; para cima e para baixo | superioridade ou inferioridade relativa | (após números redondos) aproximadamente; mais ou menos; por aí | velhos e jovens; hierarquia em termos de cargo e posição social}
  \definition{v.}{subir ou descer | subir e descer; da alta para a baixa ou da baixa para a alta}
  \synonymref{高低}{gao1di1}
  \antonymref{左右}{zuo3you4}
\end{EntryWithPhonetic}

\begin{EntryWithPhonetic}{上限}{shang4xian4}{3,8}{⼀,⾩}[HSK 7-9]
  \definition[个]{s.}{teto; limite superior; refere"-se ao primeiro ou mais alto limite dentro de um determinado conjunto de limites}
  \antonymref{下限}{xia4xian4}
\end{EntryWithPhonetic}

\begin{EntryWithPhonetic}{上学}{shang4 xue2}{3,8}{⼀,⼦}[HSK 1]
  \definition{v.}{ir à escola; frequentar a escola; estar na escola; ir à escola para estudar | começar a escola; começar a estudar no ensino fundamental}
  \synonymref{入学}{ru4/xue2}
  \synonymref{上课}{shang4/ke4}
  \antonymref{放学}{fang4/xue2}
\end{EntryWithPhonetic}

\begin{EntryWithPhonetic}{上旬}{shang4xun2}{3,6}{⼀,⽇}[HSK 7-9]
  \definition{s.}{primeiro terço do mês; os dez dias do dia 1 ao dia 10 de cada mês}
  \seealsoref{中旬}{zhong1xun2}
  \antonymref{下旬}{xia4xun2}
\end{EntryWithPhonetic}

\begin{EntryWithPhonetic}{上演}{shang4yan3}{3,14}{⼀,⽔}[HSK 6]
  \definition{s.}{exibição | encenação}
  \definition{v.}{exibir (um filme); encenar (uma peça); atuar; colocar no palco}
  \synonymref{演出}{yan3chu1}
\end{EntryWithPhonetic}

\begin{EntryWithPhonetic}{上衣}{shang4yi1}{3,6}{⼀,⾐}[HSK 3]
  \definition[件]{s.}{jaqueta; roupas para a parte superior do corpo}
\end{EntryWithPhonetic}

\begin{EntryWithPhonetic}{上瘾}{shang4/yin3}{3,16}{⼀,⽧}[HSK 7-9]
  \definition{v.+compl.}{ser viciado (em algo); adquirir o hábito (de fazer algo); gostar muito de algo, a ponto de não conseguir viver sem}
  \synonymref{沉迷}{chen2mi2}
  \synonymref{痴迷}{chi1mi2}
  \synonymref{陶醉}{tao2zui4}
\end{EntryWithPhonetic}

\begin{EntryWithPhonetic}{上映}{shang4ying4}{3,9}{⼀,⽇}[HSK 7-9]
  \definition{v.}{executar; exibir; mostrar (um filme); (novo filme) lançar para exibição}
  \synonymref{放映}{fang4ying4}
\end{EntryWithPhonetic}

\begin{EntryWithPhonetic}{上游}{shang4you2}{3,12}{⼀,⽔}[HSK 7-9]
  \definition{s.}{a montante; trecho superior de um rio; o trecho de um rio próximo à sua nascente; também se refere à área por onde esse trecho flui | posição avançada; metaforicamente, refere"-se a um \emph{status} ou nível avançado}
  \synonymref{领先}{ling3/xian1}
  \synonymref{先进}{xian1jin4}
  \antonymref{下游}{xia4you2}
\end{EntryWithPhonetic}

\begin{EntryWithPhonetic}{上涨}{shang4zhang3}{3,10}{⼀,⽔}[HSK 5]
  \definition{v.}{subir; ir para cima; ascender}
  \synonymref{高潮}{gao1chao2}
  \synonymref{高涨}{gao1zhang3}
  \synonymref{上升}{shang4sheng1}
  \antonymref{回落}{hui2luo4}
  \antonymref{下降}{xia4jiang4}
\end{EntryWithPhonetic}

\begin{EntryWithPhonetic}{上周}{shang4zhou1}{3,8}{⼀,⼝}[HSK 2]
  \definition{s.}{semana passada}
  \antonymref{下周}{xia4zhou1}
\end{EntryWithPhonetic}

%%%%%%%%%% 尚 %%%%%%%%%%
\subsection*{尚}\addcontentsline{loh}{figure}{尚 \dpy{shang4}}

\begin{EntryWithPhonetic}{尚}{shang4}{8}{⼩}[HSK 7-9]
  \definition*{s.}{Sobrenome: Shang}
  \definition{adv.}{ainda}
  \definition{s.}{costume predominante; refere"-se à tendência predominante na sociedade; coisas que geralmente são admiradas pelas pessoas}
  \definition{v.}{valorizar; estimar; dar grande importância a; respeitar}
\end{EntryWithPhonetic}

\begin{EntryWithPhonetic}{尚且}{shang4 qie3}{8,5}{⼩,⼀}
  \definition{conj.}{nem\dots; muito menos\dots; é usado antes do verbo da primeira oração de uma frase complexa para apresentar alguns exemplos óbvios para comparação, a segunda oração frequentemente usa 何况 ou 更 para ecoar e tirar conclusões inevitáveis sobre exemplos semelhantes com diferentes graus de gravidade}
  \seealsoref{更}{geng4}
  \seealsoref{何况}{he2kuang4}
\end{EntryWithPhonetic}

\begin{EntryWithPhonetic}{尚且……何况……}{shang4qie3 he2kuang4}{8,5,7,7}{⼩,⼀,⼈,⼎}
  \definition{conj.}{ainda que\dots, \dots; além do mais\dots e muito menos\dots}
\end{EntryWithPhonetic}

\begin{EntryWithPhonetic}{尚未}{shang4wei4}{8,5}{⼩,⽊}[HSK 7-9]
  \definition{adv.}{ainda não}[问题尚未解决。===O problema ainda não foi resolvido.]
\end{EntryWithPhonetic}

%%%%%%%%%% 捎 %%%%%%%%%%
\subsection*{捎}\addcontentsline{loh}{figure}{捎 \dpy{shao1}}

\begin{EntryWithPhonetic}{捎}{shao1}{10}{⼿}[HSK 7-9]
  \definition{v.}{trazer para alguém; levar algo para alguém ou em nome de alguém; levar consigo}
  \seeref{shao4}
\end{EntryWithPhonetic}

%%%%%%%%%% 烧 %%%%%%%%%%
\subsection*{烧}\addcontentsline{loh}{figure}{烧 \dpy{shao1}}

\begin{EntryWithPhonetic}{烧}{shao1}{10}{⽕}[HSK 4]
  \definition[次]{s.}{febre; temperatura corporal mais alta do que o normal}
  \definition{v.}{queimar; pegar fogo | cozinhar; aquecer; assar | guisar depois de fritar ou fritar depois de guisar | assar; grelhar os ingredientes dos alimentos diretamente sobre o fogo | ter febre; estar com febre | danificar (matar ou murchar) as plantas pelo uso excessivo (ou inadequado) de fertilizantes | tornar"-se arrogante ou presunçoso; metáfora de estar em uma boa posição e se deixar levar}
\end{EntryWithPhonetic}

\begin{EntryWithPhonetic}{烧毁}{shao1hui3}{10,13}{⽕,⽎}[HSK 7-9]
  \definition{v.}{queimar; destruir pelo fogo}
\end{EntryWithPhonetic}

\begin{EntryWithPhonetic}{烧烤}{shao1kao3}{10,10}{⽕,⽕}[HSK 7-9]
  \definition[顿,次,场]{s.}{churrasco; carne assada ou grelhada}
  \definition{v.}{assar; grelhar; fazer churrasco; grelhar ou assar (carnes) sobre carvão}
\end{EntryWithPhonetic}

%%%%%%%%%% 稍 %%%%%%%%%%
\subsection*{稍}\addcontentsline{loh}{figure}{稍 \dpy{shao1}}

\begin{EntryWithPhonetic}{稍}{shao1}{12}{⽲}[HSK 5]
  \definition{adv.}{ligeiramente; um pouco; um pouquinho}
  \seeref{shao4}
  \seealsoref{稍稍}{shao1shao1}
\end{EntryWithPhonetic}

\begin{EntryWithPhonetic}{稍后}{shao1hou4}{12,6}{⽲,⼝}[HSK 7-9]
  \definition{s.}{um pouco mais tarde (no tempo ou no espaço)}
\end{EntryWithPhonetic}

\begin{EntryWithPhonetic}{稍候}{shao1hou4}{12,10}{⽲,⼈}[HSK 7-9]
  \definition{v.}{aguardar um momento}
\end{EntryWithPhonetic}

\begin{EntryWithPhonetic}{稍稍}{shao1shao1}{12,12}{⽲,⽲}[HSK 7-9]
  \definition{adv.}{um pouco; ligeiramente}
\end{EntryWithPhonetic}

\begin{EntryWithPhonetic}{稍微}{shao1wei1}{12,13}{⽲,⼻}[HSK 5]
  \definition{adv.}{um pouco; um pouquinho; uma ninharia; indica que a quantidade é pequena ou o grau é superficial}
\end{EntryWithPhonetic}

%%%%%%%%%% 勺 %%%%%%%%%%
\subsection*{勺}\addcontentsline{loh}{figure}{勺 \dpy{shao2}}

\begin{EntryWithPhonetic}{勺}{shao2}{3}{⼓}[HSK 6]
  \definition{clas.}{shao; uma unidade tradicional de volume, igual a 0,01 市升, e equivalente a 1 centilitro ou 0,018 \emph{pint}}
  \definition{s.}{colher; concha}
  \seealsoref{市升}{shi4sheng1}
\end{EntryWithPhonetic}

%%%%%%%%%% 少 %%%%%%%%%%
\subsection*{少}\addcontentsline{loh}{figure}{少 \dpy{shao3}}

\begin{EntryWithPhonetic}{少}{shao3}{4}{⼩}[HSK 1]
  \definition{adj.}{menos; pouco; escasso; não atingir a quantidade original ou esperada}
  \definition{adv.}{um momento; um instante; provisoriamente; ligeiramente}
  \definition{v.}{faltar; ser insuficiente | dever | perder; desaparecer; extraviar | parar; desistir}
  \seeref{shao4}
  \antonymref{多}{duo1}
\end{EntryWithPhonetic}

\begin{EntryWithPhonetic}{少不了}{shao3bu5liao3}{4,4,2}{⼩,⼀,⼅}[HSK 7-9]
  \definition{v.}{não poder prescindir de; não poder dispensar | estar obrigado a; ser inevitável | não poder ser apenas alguns (ou pouco); não poder ser menos}
\end{EntryWithPhonetic}

\begin{EntryWithPhonetic}{少见}{shao3jian4}{4,4}{⼩,⾒}[HSK 7-9]
  \definition{adj.}{raramente visto; infrequente; raro}
  \definition{expr.}{``Não te vejo há muito tempo.'', ``Tenho te visto muito pouco ultimamente.'' ou ``Estou muito feliz em te ver novamente.''}
  \definition{v.}{difícil de ver | não ser familiar (para o falante) | ser raro (algo)}
\end{EntryWithPhonetic}

\begin{EntryWithPhonetic}{少量}{shao3liang4}{4,12}{⼩,⾥}[HSK 7-9]
  \definition{s.}{uma pequena quantidade; um pouco; alguns; picada de pulga; toque; gosto; tapa; estalo; ninharia; gole}
\end{EntryWithPhonetic}

\begin{EntryWithPhonetic}{少数}{shao3shu4}{4,13}{⼩,⽁}[HSK 2]
  \definition{s.}{número pequeno; poucos; minoria}
\end{EntryWithPhonetic}

\begin{EntryWithPhonetic}{少有}{shao3you3}{4,6}{⼩,⽉}[HSK 7-9]
  \definition{adj.}{raro; excepcional; escasso}
  \definition{adv.}{raramente}
\end{EntryWithPhonetic}

\begin{EntryWithPhonetic}{少}{shao4}{4}{⼩}
  \definition*{s.}{Sobrenome: Shao}
  \definition{s.}{jovem}
  \definition{s.}{jovem mestre; filho de uma família rica}
  \seeref{shao3}
  \antonymref{老}{lao3}
\end{EntryWithPhonetic}

\begin{EntryWithPhonetic}{少儿}{shao4'er2}{4,2}{⼩,⼉}[HSK 6]
  \definition{s.}{criança}
\end{EntryWithPhonetic}

\begin{EntryWithPhonetic}{少林寺}{shao4lin2 si4}{4,8,6}{⼩,⽊,⼨}[HSK 7-9]
  \definition*{s.}{Templo (ou Mosteiro) Shaolin, ao pé do Monte Song (嵩山) na província de Henan, onde o kung"-fu Shaolin foi desenvolvido}
  \seealsoref{嵩山}{song1shan1}
\end{EntryWithPhonetic}

\begin{EntryWithPhonetic}{少年}{shao4nian2}{4,6}{⼩,⼲}[HSK 2]
  \definition[个,名,位]{s.}{adolescente; juventude; atualmente, a faixa etária geralmente referida é de 10 anos ou mais a 18 anos ou mais | menor; jovem; juvenil; refere"-se a menores na faixa etária anterior | jovem; adolescente; rapaz}
\end{EntryWithPhonetic}

\begin{EntryWithPhonetic}{少女}{shao4nv3}{4,3}{⼩,⼥}[HSK 7-9]
  \definition[位,名]{s.}{empregada doméstica; jovem; moça; mulheres jovens solteiras}
\end{EntryWithPhonetic}

%%%%%%%%%% 召 %%%%%%%%%%
\subsection*{召}\addcontentsline{loh}{figure}{召 \dpy{shao4}}

\begin{EntryWithPhonetic}{召}{shao4}{5}{⼝}
  \definition*{s.}{Sobrenome: Shao}
  \definition{s.}{(frequentemente em nomes de lugares mongóis) templo; mosteiro}
  \definition{v.}{convocar; intimar; invocar}
  \seeref{zhao4}
\end{EntryWithPhonetic}

%%%%%%%%%% 绍 %%%%%%%%%%
\subsection*{绍}\addcontentsline{loh}{figure}{绍 \dpy{shao4}}

\begin{EntryWithPhonetic}{绍}{shao4}{8}{⽷}
  \definition*{s.}{Shaoxing, abreviação de 绍兴 | Sobrenome: Shao}
  \definition{v.}{continuar; herdar}
  \seealsoref{绍兴}{shao4xing1}
\end{EntryWithPhonetic}

\begin{EntryWithPhonetic}{绍兴}{shao4xing1}{8,6}{⽷,⼋}
  \definition*{s.}{Shaoxing, anteriormente conhecida como Kuaiji, é uma cidade de nível de prefeitura na província de Zhejiang, na China; é uma grande cidade localizada na parte centro"-norte da província de Zhejiang}
\end{EntryWithPhonetic}

%%%%%%%%%% 捎 %%%%%%%%%%
\subsection*{捎}\addcontentsline{loh}{figure}{捎 \dpy{shao4}}

\begin{EntryWithPhonetic}{捎}{shao4}{10}{⼿}
  \definition{adj.}{desbotado (cor)}
  \definition{v.}{recuar; puxar para trás (cavalo, burro); mover"-se ligeiramente para trás (geralmente referindo"-se a mulas, cavalos, etc.) | desbotar (cor)}
  \seeref{shao1}
\end{EntryWithPhonetic}

%%%%%%%%%% 稍 %%%%%%%%%%
\subsection*{稍}\addcontentsline{loh}{figure}{稍 \dpy{shao4}}

\begin{EntryWithPhonetic}{稍}{shao4}{12}{⽲}
  \definition{adv.}{à vontade}
\end{EntryWithPhonetic}

%%%%%%%%%% 奢 %%%%%%%%%%
\subsection*{奢}\addcontentsline{loh}{figure}{奢 \dpy{she1}}

\begin{EntryWithPhonetic}{奢}{she1}{11}{⼤}
  \definition{adj.}{luxuoso; extravagante | excessivo; desmedido; extravagante}
\end{EntryWithPhonetic}

\begin{EntryWithPhonetic}{奢侈}{she1chi3}{11,8}{⼤,⼈}[HSK 7-9]
  \definition{adj.}{luxuoso; de luxo}
\end{EntryWithPhonetic}

\begin{EntryWithPhonetic}{奢望}{she1wang4}{11,11}{⼤,⽉}[HSK 7-9]
  \definition{s.}{desejos desmedidos; esperanças extravagantes; esperança excessiva}
  \definition{v.}{esperar demais de alguém; nutrir expectativas excessivas}
\end{EntryWithPhonetic}

%%%%%%%%%% 舌 %%%%%%%%%%
\subsection*{舌}\addcontentsline{loh}{figure}{舌 \dpy{she2}}

\begin{EntryWithPhonetic}{舌}{she2}{6}{⾆}[Kangxi 135]
  \definition*{s.}{Sobrenome: She}
  \definition[片,条]{s.}{língua (de um ser humano ou animal); glossa | algo em forma de língua | língua de sino; badalo}
\end{EntryWithPhonetic}

\begin{EntryWithPhonetic}{舌头}{she2tou5}{6,5}{⾆,⼤}[HSK 6]
  \definition[个]{s.}{língua; órgão que auxilia no paladar, na mastigação e na pronúncia | espião}
\end{EntryWithPhonetic}

%%%%%%%%%% 折 %%%%%%%%%%
\subsection*{折}\addcontentsline{loh}{figure}{折 \dpy{she2}}

\begin{EntryWithPhonetic}{折}{she2}{7}{⼿}
  \definition{clas.}{um ato de zaju | um parágrafo em um drama da Dinastia Yuan, aproximadamente equivalente a uma cena ou ato em uma ópera moderna}
  \definition[张,个,些]{s.}{abatimento; desconto | os traços dos caracteres chineses têm o formato de 𠃍 e 乚 | pasta; livreto}
  \definition{v.}{estalar; quebrar; fazer quebrar | perder; sofrer a perda de | dobrar; torcer; curvar-se | voltar; mudar de direção; retornar | estar convencido; estar cheio de admiração | equivaler a; converter em}
  \seeref{zhe1}
  \seeref{zhe2}
\end{EntryWithPhonetic}

%%%%%%%%%% 蛇 %%%%%%%%%%
\subsection*{蛇}\addcontentsline{loh}{figure}{蛇 \dpy{she2}}

\begin{EntryWithPhonetic}{蛇}{she2}{11}{⾍}[HSK 5]
  \definition[条]{s.}{cobra; serpente; répteis}
\end{EntryWithPhonetic}

%%%%%%%%%% 舍 %%%%%%%%%%
\subsection*{舍}\addcontentsline{loh}{figure}{舍 \dpy{she3}}

\begin{EntryWithPhonetic}{舍}{she3}{8}{⾆}
  \definition{v.}{abandonar; desistir; descartar; jogar fora | dar esmola; dispensar caridade}
  \seeref{she4}
\end{EntryWithPhonetic}

\begin{EntryWithPhonetic}{舍不得}{she3bu5de5}{8,4,11}{⾆,⼀,⼻}[HSK 5]
  \definition{v.}{não se pode abandonar ou deixar, não se quer usar ou descartar; detestar separar-me ou usar}
\end{EntryWithPhonetic}

\begin{EntryWithPhonetic}{舍得}{she3 de5}{8,11}{⾆,⼻}[HSK 5]
  \definition{v.}{não guardar rancor; estar disposto a abrir mão de algo; estar disposto a gastar dinheiro, tempo, etc.; estar disposto a abrir mão de pessoas, oportunidades, coisas, etc. que são importantes para você}
\end{EntryWithPhonetic}

%%%%%%%%%% 设 %%%%%%%%%%
\subsection*{设}\addcontentsline{loh}{figure}{设 \dpy{she4}}

\begin{EntryWithPhonetic}{设}{she4}{6}{⾔}[HSK 7-9]
  \definition*{s.}{Sobrenome: She}
  \definition{conj.}{se; no caso | Matemática: dado; suponha; se}
  \definition{v.}{configurar; estabelecer; encontrar; colocar em prática}
\end{EntryWithPhonetic}

\begin{EntryWithPhonetic}{设备}{she4bei4}{6,8}{⾔,⼡}[HSK 3]
  \definition[台,套]{s.}{instalação; equipamento; montagem; um conjunto de edifícios ou equipamentos necessários para executar uma determinada tarefa ou suprir uma determinada necessidade}
\end{EntryWithPhonetic}

\begin{EntryWithPhonetic}{设定}{she4ding4}{6,8}{⾔,⼧}[HSK 7-9]
  \definition{v.}{definir; configurar; instalar}
\end{EntryWithPhonetic}

\begin{EntryWithPhonetic}{设法}{she4fa3}{6,8}{⾔,⽔}[HSK 7-9]
  \definition{v.}{encontrar um jeito de; conseguir; fazer esforços para; significa tentar encontrar uma maneira (de fazer algo)}
\end{EntryWithPhonetic}

\begin{EntryWithPhonetic}{设计}{she4ji4}{6,4}{⾔,⾔}[HSK 3]
  \definition[份]{s.}{plano; esquema; refere"-se a um plano de design ou a um projeto para um plano, etc.}
  \definition{v.}{planejar; projetar; formular métodos, desenhos, etc. com antecedência, de acordo com determinados requisitos de finalidade, antes de iniciar oficialmente um trabalho | arquitetar; idear; tramar; fazer um plano}
\end{EntryWithPhonetic}

\begin{EntryWithPhonetic}{设计师}{she4ji4shi1}{6,4,6}{⾔,⾔,⼱}[HSK 6]
  \definition[个,位,名,些]{s.}{planejador de projeto; designer | arquiteto}
\end{EntryWithPhonetic}

\begin{EntryWithPhonetic}{设立}{she4li4}{6,5}{⾔,⽴}[HSK 3]
  \definition{v.}{fundar; estabelecer; começar}
\end{EntryWithPhonetic}

\begin{EntryWithPhonetic}{设施}{she4shi1}{6,9}{⾔,⽅}[HSK 4]
  \definition{s.}{facilidade; instalação; instituições, sistemas, organizações, edifícios, etc., estabelecidos para realizar um trabalho ou atender a uma necessidade}
\end{EntryWithPhonetic}

\begin{EntryWithPhonetic}{设想}{she4xiang3}{6,13}{⾔,⼼}[HSK 5]
  \definition[个,种]{s.}{plano provisório (ou ideia); (item, tipo) refere"-se a algo hipotético ou imaginário}
  \definition{v.}{imaginar; prever; conceber; supor | ter consideração por}
\end{EntryWithPhonetic}

\begin{EntryWithPhonetic}{设置}{she4zhi4}{6,13}{⾔,⽹}[HSK 4]
  \definition{v.}{estabelecer; colocar em prática; estabelecer ou criar instituições, empregos, profissões ou códigos, etc. | encaixar; ajustar; instalar; configurar; colocar}
\end{EntryWithPhonetic}

%%%%%%%%%% 社 %%%%%%%%%%
\subsection*{社}\addcontentsline{loh}{figure}{社 \dpy{she4}}

\begin{EntryWithPhonetic}{社}{she4}{7}{⽰}[HSK 5]
  \definition[个,家]{s.}{agência; sociedade; órgão organizado; organização; comunidade | comuna popular | o deus da terra, sacrifícios a ele ou altares para tais sacrifícios; na antiguidade, o deus da terra, o local onde ele era venerado, o dia da veneração e o ritual eram chamados de 社 | agência de notícias |  imprensa}
\end{EntryWithPhonetic}

\begin{EntryWithPhonetic}{社会}{she4hui4}{7,6}{⽰,⼈}[HSK 3]
  \definition[个,种]{s.}{sociedade; em um determinado estágio do desenvolvimento histórico, a relação geral entre as pessoas nas atividades de produção | comunidade; geralmente se refere a um grupo de pessoas que estão conectadas por atividades comuns}
\end{EntryWithPhonetic}

\begin{EntryWithPhonetic}{社会主义}{she4hui4 zhu3yi4}{7,6,5,3}{⽰,⼈,⼂,⼂}[HSK 7-9]
  \definition*{s.}{Socialismo}
\end{EntryWithPhonetic}

\begin{EntryWithPhonetic}{社交}{she4jiao1}{7,6}{⽰,⼇}[HSK 7-9]
  \definition{s.}{contato social; interação social; refere"-se às interações interpessoais na sociedade}
\end{EntryWithPhonetic}

\begin{EntryWithPhonetic}{社论}{she4lun4}{7,6}{⽰,⾔}[HSK 7-9]
  \definition[篇]{s.}{editorial; artigo principal}
\end{EntryWithPhonetic}

\begin{EntryWithPhonetic}{社区}{she4qu1}{7,4}{⽰,⼖}[HSK 5]
  \definition[个]{s.}{bairro; comunidade residencial; bairros da cidade, divididos de acordo com a localização geográfica | distrito; comunidade (para pessoas da mesma classe social, etc.) ; lugar onde pessoas com características comuns, como classe social, vivem juntas}
\end{EntryWithPhonetic}

\begin{EntryWithPhonetic}{社团}{she4tuan2}{7,6}{⽰,⼞}[HSK 7-9]
  \definition[个]{s.}{clube, associação; sociedade; organização social; termo genérico para diversas organizações de massa, como sindicatos, federações de mulheres e grêmios estudantis}
\end{EntryWithPhonetic}

%%%%%%%%%% 舍 %%%%%%%%%%
\subsection*{舍}\addcontentsline{loh}{figure}{舍 \dpy{she4}}

\begin{EntryWithPhonetic}{舍}{she4}{8}{⾆}
  \definition*{s.}{Sobrenome: She}
  \definition{clas.}{uma unidade antiga de distância igual a 30 li, 里}
  \definition{pron.}{meu, uma palavra humilde usada para se referir aos parentes mais jovens ou de geração inferior}
  \definition{s.}{cabana; casa | minha casa; minha humilde morada | chiqueiro; galpão; curral de gado}
  \seeref{she3}
  \seealsoref{里}{li3}
\end{EntryWithPhonetic}

%%%%%%%%%% 拾 %%%%%%%%%%
\subsection*{拾}\addcontentsline{loh}{figure}{拾 \dpy{she4}}

\begin{EntryWithPhonetic}{拾}{she4}{9}{⼿}
  \definition{v.}{subir em passos leves}
  \seeref{shi2}
\end{EntryWithPhonetic}

%%%%%%%%%% 射 %%%%%%%%%%
\subsection*{射}\addcontentsline{loh}{figure}{射 \dpy{she4}}

\begin{EntryWithPhonetic}{射}{she4}{10}{⼨}[HSK 5]
  \definition*{s.}{Sobrenome: She}
  \definition{v.}{atirar; disparar | descarregar em jato; jorrar | emitir (luz, calor, etc.) | irradiar | aludir a algo ou alguém; insinuar}
\end{EntryWithPhonetic}

\begin{EntryWithPhonetic}{射击}{she4ji1}{10,5}{⼨,⼐}[HSK 5]
  \definition{s.}{tiro; tiro ao alvo}
  \definition{v.}{disparar; atirar}
\end{EntryWithPhonetic}

%%%%%%%%%% 涉 %%%%%%%%%%
\subsection*{涉}\addcontentsline{loh}{figure}{涉 \dpy{she4}}

\begin{EntryWithPhonetic}{涉}{she4}{10}{⽔}
  \definition*{s.}{Sobrenome: She}
  \definition{v.}{vadear; atravessar ou passar um rio ou um obstáculo | passar por; experimentar | envolver; implicar}
\end{EntryWithPhonetic}

\begin{EntryWithPhonetic}{涉及}{she4ji2}{10,3}{⽔,⼃}[HSK 6]
  \definition{v.}{envolver; relacionar"-se com; referir"-se a; tocar em}
\end{EntryWithPhonetic}

\begin{EntryWithPhonetic}{涉嫌}{she4xian2}{10,13}{⽔,⼥}[HSK 7-9]
  \definition{v.}{ser suspeito; ser suspeito de estar envolvido; ser suspeito de envolvimento em determinado assunto}
\end{EntryWithPhonetic}

%%%%%%%%%% 摄 %%%%%%%%%%
\subsection*{摄}\addcontentsline{loh}{figure}{摄 \dpy{she4}}

\begin{EntryWithPhonetic}{摄}{she4}{13}{⼿}
  \definition*{s.}{Sobrenome: She}
  \definition{v.}{absorver; assimilar | tirar uma fotografia de; fotografar | conservar (a saúde) | atuar}
\end{EntryWithPhonetic}

\begin{EntryWithPhonetic}{摄氏}{she4shi4}{13,4}{⼿,⽒}
  \definition{s.}{graus Celsius (°C), centígrado}
\end{EntryWithPhonetic}

\begin{EntryWithPhonetic}{摄氏度}{she4shi4du4}{13,4,9}{⼿,⽒,⼴}[HSK 7-9]
  \definition{s.}{centígrado; grau Celsius}[水在100摄氏度时沸腾。===A água ferve a 100 graus Celsius.]
\end{EntryWithPhonetic}

\begin{EntryWithPhonetic}{摄像}{she4xiang4}{13,13}{⼿,⼈}[HSK 5]
  \definition{v.}{gravar; filmar; filmar com câmera; fazer uma gravação de vídeo (com uma câmera de vídeo ou TV)}
\end{EntryWithPhonetic}

\begin{EntryWithPhonetic}{摄像机}{she4xiang4ji1}{13,13,6}{⼿,⼈,⽊}[HSK 5]
  \definition[个,部,台]{s.}{câmera de vídeo; dispositivo que pode ser usado para converter imagens captadas em sinais de imagem de televisão}
\end{EntryWithPhonetic}

\begin{EntryWithPhonetic}{摄影}{she4ying3}{13,15}{⼿,⼺}[HSK 5]
  \definition{v.}{fotografar; tirar uma foto; tirar fotos ou filmar}
\end{EntryWithPhonetic}

\begin{EntryWithPhonetic}{摄影师}{she4ying3shi1}{13,15,6}{⼿,⼺,⼱}[HSK 5]
  \definition[个,名,位]{s.}{fotógrafo; cinegrafista; operador de câmera; técnico de fotografia em estúdio fotográfico}
\end{EntryWithPhonetic}

%%%%%%%%%% 谁 %%%%%%%%%%
\subsection*{谁}\addcontentsline{loh}{figure}{谁 \dpy{shei2}}

\begin{EntryWithPhonetic}{谁}{shei2}{10}{⾔}[HSK 1]
  \definition{pron.}{quem? | (em pergunta retórica) quem?; usado em perguntas retóricas, para indicar que não há ninguém | refere"-se a pessoas que não têm certeza, incluindo aquelas que não sabem | alguém; qualquer pessoa; indica qualquer pessoa ou qualquer um | repetido em uma frase para se referir a uma pessoa | (repetido em duas frases) quem quer que seja; fazer com que o sujeito e o objeto se refiram a duas pessoas diferentes}
  \seeref{shui2}
\end{EntryWithPhonetic}

\begin{EntryWithPhonetic}{谁知道}{shei2 zhi1dao4}{10,8,12}{⾔,⽮,⾡}[HSK 7-9]
  \definition{interj.}{``Quem sabe?''; ``Só Deus sabe\dots''; ``Quem diria?''; ``Quem poderia imaginar?''}
\end{EntryWithPhonetic}

%%%%%%%%%% 申 %%%%%%%%%%
\subsection*{申}\addcontentsline{loh}{figure}{申 \dpy{shen1}}

\begin{EntryWithPhonetic}{申}{shen1}{5}{⽥}
  \definition*{s.}{O nono dos doze Ramos Terrestres | Outro nome para Xangai, 上海 | Sobrenome: Shen}
  \definition{v.}{declarar; explicar; expressar}
  \seealsoref{上海}{shang4hai3}
\end{EntryWithPhonetic}

\begin{EntryWithPhonetic}{申办}{shen1ban4}{5,4}{⽥,⼒}[HSK 7-9]
  \definition{v.}{candidatar"-se a sediar; candidatar"-se a}
\end{EntryWithPhonetic}

\begin{EntryWithPhonetic}{申报}{shen1bao4}{5,7}{⽥,⼿}[HSK 7-9]
  \definition{v.}{reportar a um órgão superior; reportar aos superiores ou departamentos relevantes por escrito (frequentemente usado em documentos legais); reportar aos superiores em documentos oficiais | declarar algo (à Alfândega)}
\end{EntryWithPhonetic}

\begin{EntryWithPhonetic}{申领}{shen1ling3}{5,11}{⽥,⾴}[HSK 7-9]
  \definition{v.}{candidatar"-se a; obter mediante candidatura a | solicitar (licença, visto etc.)}
\end{EntryWithPhonetic}

\begin{EntryWithPhonetic}{申请}{shen1qing3}{5,10}{⽥,⾔}[HSK 4]
  \definition[份,批,项]{s.}{a solicitação para; o requerimento para; um pedido para ser visto pelos superiores ou departamentos relevantes}
  \definition{v.}{solicitar; apresentar uma solicitação; apresentar os motivos e fazer o pedido aos superiores ou aos departamentos competentes}
\end{EntryWithPhonetic}

%%%%%%%%%% 伸 %%%%%%%%%%
\subsection*{伸}\addcontentsline{loh}{figure}{伸 \dpy{shen1}}

\begin{EntryWithPhonetic}{伸}{shen1}{7}{⼈}[HSK 5]
  \definition{v.}{alongar; esticar; estender}
  \synonymref{开}{kai1}
  \synonymref{展}{zhan3}
  \synonymref{长}{zhang3}
  \antonymref{屈}{qu1}
  \antonymref{缩}{suo1}
\end{EntryWithPhonetic}

\begin{EntryWithPhonetic}{伸手}{shen1/shou3}{7,4}{⼈,⼿}[HSK 7-9]
  \definition{v.+compl.}{estender a mão; metaforicamente, pedir a alguém ou a uma organização (algo, honra, etc.) | pedir ajuda, dinheiro, presentes, etc. | ter participação em (geralmente pejorativo); interferir (com conotação negativa)}
  \synonymref{绅士}{shen1shi4}
  \antonymref{缩手}{suo1shou3}
\end{EntryWithPhonetic}

\begin{EntryWithPhonetic}{伸缩}{shen1suo1}{7,14}{⼈,⽷}[HSK 7-9]
  \definition{adj.}{flexible; elastic; adjustable}
  \definition{s.}{ampliação; dilatação; expansão}
  \definition{v.}{esticar e retrair; expandir e contrair; alongar e encurtar | ser flexível; ser adaptável}
  \synonymref{步行}{bu4xing2}
  \synonymref{乘坐}{cheng2zuo4}
  \antonymref{膨胀}{peng2zhang4}
\end{EntryWithPhonetic}

\begin{EntryWithPhonetic}{伸张}{shen1zhang1}{7,7}{⼈,⼸}[HSK 7-9]
  \definition{v.}{defender; promover}
  \synonymref{扩大}{kuo4da4}
  \synonymref{扩展}{kuo4zhan3}
  \synonymref{扩张}{kuo4zhang1}
  \synonymref{蔓延}{man4yan2}
  \antonymref{收缩}{shou1suo1}
\end{EntryWithPhonetic}

%%%%%%%%%% 身 %%%%%%%%%%
\subsection*{身}\addcontentsline{loh}{figure}{身 \dpy{shen1}}

\begin{EntryWithPhonetic}{身}{shen1}{7}{⾝}[Kangxi 158]
  \definition*{s.}{Sobrenome: Shen}
  \definition{adv.}{eu mesmo; a si mesmo; pessoalmente}
  \definition{s.}{corpo humano ou animal | vida | o caráter moral e a conduta de alguém; cultivo moral | corpo; a parte principal de uma estrutura; o corpo principal ou tronco de um objeto |  uma vida inteira; a vida inteira de alguém | \emph{status} social; identidade}
\end{EntryWithPhonetic}

\begin{EntryWithPhonetic}{身边}{shen1bian1}{7,5}{⾝,⾡}[HSK 2]
  \definition{adv.}{ao redor; ao lado de alguém; perto do corpo | carregar consigo (transportar); à mão}
\end{EntryWithPhonetic}

\begin{EntryWithPhonetic}{身不由己}{shen1bu4you2ji3}{7,4,5,3}{⾝,⼀,⽥,⼰}[HSK 7-9]
  \definition{expr.}{involuntariamente; sob compulsão; apesar de si mesmo; não por vontade própria; sem liberdade para agir de forma independente; involuntário}
  \synonymref{不由自主}{bu4you2zi4zhu3}
  \synonymref{情不自禁}{qing2bu2zi4jin1}
  \antonymref{独立自主}{du2li4-zi4zhu3}
\end{EntryWithPhonetic}

\begin{EntryWithPhonetic}{身材}{shen1cai2}{7,7}{⾝,⽊}[HSK 4]
  \definition[种,个,具]{s.}{figura; estatura; altura e peso corporal}
\end{EntryWithPhonetic}

\begin{EntryWithPhonetic}{身份证}{shen1fen4zheng4}{7,6,7}{⾝,⼈,⾔}[HSK 3]
  \definition[张]{s.}{ID; bilhete de identidade; carteira de identidade}
\end{EntryWithPhonetic}

\begin{EntryWithPhonetic}{身份}{shen1fen5}{7,6}{⾝,⼈}[HSK 4]
  \definition[种]{s.}{\emph{status}; capacidade; identidade; refere"-se à origem, ao \emph{status} e às qualificações de uma pessoa | dignidade; posição honrada; referência especial ao \emph{status} respeitável}
\end{EntryWithPhonetic}

\begin{EntryWithPhonetic}{身高}{shen1gao1}{7,10}{⾝,⾼}[HSK 4]
  \definition[个,种,段]{s.}{estatura; altura (de uma pessoa)}
\end{EntryWithPhonetic}

\begin{EntryWithPhonetic}{身价}{shen1jia4}{7,6}{⾝,⼈}[HSK 7-9]
  \definition{adj.}{Literário: suficiente; adequado; abundante; em grande quantidade | Literário: refinado; rico}
  \definition{v.}{apoiar; prover para}
\end{EntryWithPhonetic}

\begin{EntryWithPhonetic}{身躯}{shen1qu1}{7,11}{⾝,⾝}[HSK 7-9]
  \definition{s.}{corpo; estatura}
  \synonymref{依然}{yi1ran2}
  \antonymref{灵魂}{ling2hun2}
\end{EntryWithPhonetic}

\begin{EntryWithPhonetic}{身上}{shen1shang5}{7,3}{⾝,⼀}[HSK 1]
  \definition{s.}{no corpo de alguém | em um;  com um}
\end{EntryWithPhonetic}

\begin{EntryWithPhonetic}{身体}{shen1ti3}{7,7}{⾝,⼈}[HSK 1]
  \definition[具,个]{s.}{corpo | saúde; saúde das pessoas}
\end{EntryWithPhonetic}

\begin{EntryWithPhonetic}{身体能力}{shen1ti3 neng2li4}{7,7,10,2}{⾝,⼈,⾁,⼒}
  \definition{s.}{habilidade física}
\end{EntryWithPhonetic}

\begin{EntryWithPhonetic}{身体乳}{shen1ti3 ru3}{7,7,8}{⾝,⼈,⼄}
  \definition{s.}{loção corporal}
\end{EntryWithPhonetic}

\begin{EntryWithPhonetic}{身亡}{shen1wang2}{7,3}{⾝,⼇}
  \definition{v.}{morrer}
\end{EntryWithPhonetic}

\begin{EntryWithPhonetic}{身心}{shen1xin1}{7,4}{⾝,⼼}[HSK 7-9]
  \definition{s.}{corpo e mente}
\end{EntryWithPhonetic}

\begin{EntryWithPhonetic}{身影}{shen1ying3}{7,15}{⾝,⼺}[HSK 7-9]
  \definition{s.}{forma; figura; a silhueta de uma pessoa; imagem desfocada de um corpo vista à distância}
  \synonymref{影子}{ying3zi5}
\end{EntryWithPhonetic}

\begin{EntryWithPhonetic}{身子}{shen1zi5}{7,3}{⾝,⼦}[HSK 7-9]
  \definition{s.}{corpo | gravidez | saúde humana}
  \synonymref{身躯}{shen1qu1}
\end{EntryWithPhonetic}

%%%%%%%%%% 绅 %%%%%%%%%%
\subsection*{绅}\addcontentsline{loh}{figure}{绅 \dpy{shen1}}

\begin{EntryWithPhonetic}{绅}{shen1}{8}{⽷}
  \definition[个]{s.}{Arcaico: cinto (usado por funcionários e homens de letras) | nobreza}
\end{EntryWithPhonetic}

\begin{EntryWithPhonetic}{绅士}{shen1shi4}{8,3}{⽷,⼠}[HSK 7-9]
  \definition{adj.}{cavalheiro; bem"-educado}
  \definition[位,名]{s.}{cavalheiro; antigamente, isso se referia a indivíduos influentes e bem"-sucedidos em uma área local, geralmente proprietários de terras ou funcionários aposentados}
  \synonymref{名流}{ming2liu2}
  \synonymref{伸手}{shen1/shou3}
  \antonymref{恶霸}{e4ba4}
  \antonymref{混蛋}{hun4dan4}
\end{EntryWithPhonetic}

%%%%%%%%%% 深 %%%%%%%%%%
\subsection*{深}\addcontentsline{loh}{figure}{深 \dpy{shen1}}

\begin{EntryWithPhonetic}{深}{shen1}{11}{⽔}[HSK 3]
  \definition*{s.}{Sobrenome: Shen}
  \definition{adj.}{profundo | difícil; intenso; profundo | completo; penetrante; intenso; profundo | próximo; íntimo; afeição profunda; relacionamento próximo | escuro; profundo | tardio}
  \definition{adv.}{muito; grandemente; profundamente}
  \definition{s.}{profundidade}
  \seealsoref{浅}{qian3}
\end{EntryWithPhonetic}

\begin{EntryWithPhonetic}{深奥}{shen1'ao4}{11,12}{⽔,⼤}[HSK 7-9]
  \definition{adj.}{profundo; abstruso; (os princípios e significados) são profundos e difíceis de compreender.}
  \synonymref{深厚}{shen1hou4}
\end{EntryWithPhonetic}

\begin{EntryWithPhonetic}{深处}{shen1chu4}{11,5}{⽔,⼡}[HSK 5]
  \definition{s.}{profundidades; recantos; recessos | profundezas}
\end{EntryWithPhonetic}

\begin{EntryWithPhonetic}{深度}{shen1du4}{11,9}{⽔,⼴}[HSK 5]
  \definition{adj.}{(em grau ou extensão) profundo; sério; grave}
  \definition{s.}{profundidade; grau de profundidade; | profundidade; rigor; meticulosidade; grau de contato com a essência das coisas | estágio avançado (ou em deterioração) de desenvolvimento; grau de crescimento e desenvolvimento das coisas}
\end{EntryWithPhonetic}

\begin{EntryWithPhonetic}{深厚}{shen1hou4}{11,9}{⽔,⼚}[HSK 4]
  \definition{adj.}{profundo; sentimentos fortes | sólido; profundamente enraizado; fundação sólida}
\end{EntryWithPhonetic}

\begin{EntryWithPhonetic}{深化}{shen1hua4}{11,4}{⽔,⼔}[HSK 6]
  \definition{v.}{aprofundar; avançar; intensificar; tornar"-se mais profundo; tornar mais profundo}
\end{EntryWithPhonetic}

\begin{EntryWithPhonetic}{深刻}{shen1ke4}{11,8}{⽔,⼑}[HSK 3]
  \definition{adj.}{profundo; instenso; chegar à essência de um assunto ou problema}
\end{EntryWithPhonetic}

\begin{EntryWithPhonetic}{深切}{shen1qie4}{11,4}{⽔,⼑}[HSK 7-9]
  \definition{adj.}{sincero; profundo; descreve um afeto muito profundo ou um relacionamento muito próximo | aguçado; penetrante; minucioso; descreve um nível profundo e genuíno de compreensão ou sentimento}
  \synonymref{深刻}{shen1ke4}
  \synonymref{深入}{shen1ru4}
  \synonymref{深深}{shen1shen1}
  \synonymref{真切}{zhen1qie4}
\end{EntryWithPhonetic}

\begin{EntryWithPhonetic}{深情}{shen1qing2}{11,11}{⽔,⼼}[HSK 7-9]
  \definition{adj.}{afetuoso; bondoso}
  \definition{s.}{amor profundo; sentimento profundo; profundo afeto}
  \synonymref{情谊}{qing2yi4}
  \antonymref{仇恨}{chou2hen4}
\end{EntryWithPhonetic}

\begin{EntryWithPhonetic}{深入}{shen1ru4}{11,2}{⽔,⼊}[HSK 3]
  \definition{adj.}{profundo; completo}
  \definition{v.}{ir fundo em; penetrar em; penetrar o exterior; alcançar o interior ou o centro de algo}
\end{EntryWithPhonetic}

\begin{EntryWithPhonetic}{深入人心}{shen1ru4-ren2xin1}{11,2,2,4}{⽔,⼊,⼈,⼼}[HSK 7-9]
  \definition{expr.}{``Profundamente enraizado nos corações das pessoas.''; criar raízes nos corações das pessoas; com profundidade e uma abordagem fácil de entender; penetrar profundamente no coração das pessoas; ter um impacto real nas pessoas}
  \synonymref{家喻户晓}{jia1yu4-hu4xiao3}
\end{EntryWithPhonetic}

\begin{EntryWithPhonetic}{深深}{shen1shen1}{11,11}{⽔,⽔}[HSK 6]
  \definition{adj.}{profundo; intenso}
  \definition{adv.}{profundamente; intensamente; descreve um grau profundo ou forte}
\end{EntryWithPhonetic}

\begin{EntryWithPhonetic}{深受}{shen1shou4}{11,8}{⽔,⼜}[HSK 7-9]
  \definition{adv.}{profundamente (amado; apreciado; recebido com carinho)}
\end{EntryWithPhonetic}

\begin{EntryWithPhonetic}{深思}{shen1si1}{11,9}{⽔,⼼}[HSK 7-9]
  \definition{v.}{ponderar; meditar; estar imerso em pensamentos; refletir profundamente sobre}
  \synonymref{沉思}{chen2si1}
  \synonymref{反思}{fan3si1}
\end{EntryWithPhonetic}

\begin{EntryWithPhonetic}{深信}{shen1xin4}{11,9}{⽔,⼈}[HSK 7-9]
  \definition{v.}{estar profundamente convencido; acreditar firmemente; ter muita fé}
  \synonymref{坚信}{jian1xin4}
  \synonymref{确信}{que4xin4}
  \antonymref{怀疑}{huai2yi2}
\end{EntryWithPhonetic}

\begin{EntryWithPhonetic}{深夜}{shen1ye4}{11,8}{⽔,⼣}[HSK 7-9]
  \definition[个]{s.}{tarde da noite; depois da meia-noite; as primeiras horas da manhã}
  \synonymref{半夜}{ban4ye4}
  \synonymref{午夜}{wu3ye4}
  \antonymref{傍晚}{bang4wan3}
  \antonymref{清晨}{qing1chen2}
\end{EntryWithPhonetic}

\begin{EntryWithPhonetic}{深远}{shen1yuan3}{11,7}{⽔,⾡}[HSK 7-9]
  \definition{adj.}{de longo alcance; profundo e duradouro; descreve algo como tendo um impacto profundo ou um significado que perdura por muito tempo}
  \synonymref{长久}{chang2jiu3}
  \synonymref{长远}{chang2yuan3}
  \synonymref{深厚}{shen1hou4}
  \synonymref{深刻}{shen1ke4}
  \synonymref{深切}{shen1qie4}
  \synonymref{深入}{shen1ru4}
  \synonymref{永远}{yong3yuan3}
\end{EntryWithPhonetic}

%%%%%%%%%% 什 %%%%%%%%%%
\subsection*{什}\addcontentsline{loh}{figure}{什 \dpy{shen2}}

\begin{EntryWithPhonetic}{什}{shen2}{4}{⼈}
  \definition{pron.}{o que; qualquer coisa}
  \seeref{shi2}
  \seealsoref{什么}{shen2me5}
\end{EntryWithPhonetic}

\begin{EntryWithPhonetic}{什么}{shen2me5}{4,3}{⼈,⼃}[HSK 1]
  \definition{pron.}{o que?; expressar dúvida, perguntar sobre o mundo, locais, pessoas ou coisas | usado para se referir a algo indefinido; expressar incerteza | qualquer; todos; refere"-se a todas as pessoas ou coisas | dois 什么 são usados juntos, indicando que o primeiro determina o segundo | usado para expressar surpresa ou insatisfação | usado para expressar discordância com o que foi dito; expressar negação | usado antes de elementos paralelos para indicar que a lista é infinita}
  \synonymref{哪儿}{na3r5}
\end{EntryWithPhonetic}

\begin{EntryWithPhonetic}{什么时候}{shen2me5shi2hou5}{4,3,7,10}{⼈,⼃,⽇,⼈}
  \definition{adv.}{quando? | a que horas?}
\end{EntryWithPhonetic}

\begin{EntryWithPhonetic}{什么样}{shen2me5yang4}{4,3,10}{⼈,⼃,⽊}[HSK 2]
  \definition{pron.}{que tipo?; usado para perguntar sobre a natureza, características ou aparência de algo |  o quê?; de que tipo?; usado para perguntar sobre a situação ou o estado de alguém ou algo}
\end{EntryWithPhonetic}

%%%%%%%%%% 神 %%%%%%%%%%
\subsection*{神}\addcontentsline{loh}{figure}{神 \dpy{shen2}}

\begin{EntryWithPhonetic}{神}{shen2}{9}{⽰}[HSK 5]
  \definition*{s.}{Deus | Sobrenome: Shen}
  \definition{adj.}{inteligente; esperto | mágico; sobrenatural}
  \definition[个,位,尊,名]{s.}{divindade; deidade | espírito; mente; refere"-se ao espírito, energia ou atenção de uma pessoa | olhar; expressão; expressões que refletem o estado interior}
\end{EntryWithPhonetic}

\begin{EntryWithPhonetic}{神话}{shen2hua4}{9,8}{⽰,⾔}[HSK 4]
  \definition[段,篇]{s.}{mito; mitologia; conto de fadas; refere"-se a deuses e deusas lendários e histórias de heróis antigos deificados | lorota; refere"-se a alegações ridículas e infundadas}
\end{EntryWithPhonetic}

\begin{EntryWithPhonetic}{神经}{shen2jing1}{9,8}{⽰,⽷}[HSK 5]
  \definition{adj.}{excêntrico; estranho; peculiar; descreve anormalidade neurológica}
  \definition[根,条]{s.}{nervo; um tipo de tecido presente no corpo humano ou animal que conecta o cérebro aos órgãos, transmitindo as sensações ao cérebro e as informações do cérebro aos órgãos}
\end{EntryWithPhonetic}

\begin{EntryWithPhonetic}{神经病的}{shen2jing1bing4 de5}{9,8,10,8}{⽰,⽷,⽧,⽩}
  \definition{adj.}{neuropático; neurótico}
\end{EntryWithPhonetic}

\begin{EntryWithPhonetic}{神经病学}{shen2jing1bing4 xue2}{9,8,10,8}{⽰,⽷,⽧,⼦}
  \definition{s.}{neurologia}
\end{EntryWithPhonetic}

\begin{EntryWithPhonetic}{神秘}{shen2mi4}{9,10}{⽰,⽲}[HSK 4]
  \definition{adj.}{místico; misterioso}
\end{EntryWithPhonetic}

\begin{EntryWithPhonetic}{神明}{shen2ming2}{9,8}{⽰,⽇}
  \definition{s.}{divindades | deuses}
\end{EntryWithPhonetic}

\begin{EntryWithPhonetic}{神奇}{shen2qi2}{9,8}{⽰,⼤}[HSK 5]
  \definition{adj.}{mágico; peculiar; místico; milagroso; faz as pessoas se sentirem muito revigoradas; é completamente inesperado e geralmente traz boas influências}
  \definition{adj.}{mágico; peculiar; místico; milagroso; algo que parece muito novo; algo que ninguém imaginaria, mas que geralmente traz bons resultados}
\end{EntryWithPhonetic}

\begin{EntryWithPhonetic}{神气}{shen2qi4}{9,4}{⽰,⽓}[HSK 7-9]
  \definition{adj.}{enérgico; vigoroso; cheio de energia | arrogante; que se acha; presunçoso; exibe um ar de superioridade ou arrogância}
  \definition{s.}{ar; modo; expressão; olhar}
  \synonymref{脸色}{lian3se4}
  \synonymref{心情}{xin1qing2}
  \synonymref{样子}{yang4zi5}
\end{EntryWithPhonetic}

\begin{EntryWithPhonetic}{神器}{shen2qi4}{9,16}{⽰,⼝}
  \definition{s.}{objeto mágico | objeto simbólico do poder imperial | arma fina | ferramenta muito útil}
\end{EntryWithPhonetic}

\begin{EntryWithPhonetic}{神情}{shen2qing2}{9,11}{⽰,⼼}[HSK 5]
  \definition{s.}{aparência; expressão; atividades internas reveladas no rosto das pessoas}
\end{EntryWithPhonetic}

\begin{EntryWithPhonetic}{神圣}{shen2sheng4}{9,5}{⽰,⼟}[HSK 7-9]
  \definition{adj.}{santo; sagrado; extremamente sublime e solene; inviolável}
\end{EntryWithPhonetic}

\begin{EntryWithPhonetic}{神兽}{shen2shou4}{9,11}{⽰,⼋}
  \definition{s.}{animal mitológico | fera}
\end{EntryWithPhonetic}

\begin{EntryWithPhonetic}{神态}{shen2tai4}{9,8}{⽰,⼼}[HSK 7-9]
  \definition{s.}{semblante; maneira; porte; expressão; expressão e atitude}
  \synonymref{表情}{biao3qing2}
  \synonymref{脸色}{lian3se4}
  \synonymref{模样}{mu2yang4}
  \synonymref{容貌}{rong2mao4}
  \synonymref{神情}{shen2qing2}
  \synonymref{心情}{xin1qing2}
  \synonymref{形状}{xing2zhuang4}
  \synonymref{样子}{yang4zi5}
\end{EntryWithPhonetic}

\begin{EntryWithPhonetic}{神仙}{shen2xian1}{9,5}{⽰,⼈}[HSK 7-9]
  \definition{s.}{pessoa livre das preocupações mundanas; monge iluminado; uma metáfora para uma pessoa despreocupada, sem restrições e sem preocupações | (na ficção moderna) fada, elfo, duende etc.; figuras míticas possuem habilidades sobre"-humanas, transcendendo o reino mortal e alcançando a imortalidade; imortal | profeta; vidente; uma metáfora para alguém que consegue prever ou compreender as coisas}
\end{EntryWithPhonetic}

%%%%%%%%%% 审 %%%%%%%%%%
\subsection*{审}\addcontentsline{loh}{figure}{审 \dpy{shen3}}

\begin{EntryWithPhonetic}{审}{shen3}{8}{⼧}[HSK 7-9]
  \definition*{s.}{Sobrenome: Shen}
  \definition{adj.}{cuidadoso; detalhado; completo}
  \definition{adv.}{Literário: realmente; de fato; como esperado}
  \definition{v.}{examinar; analizar | julgar; interrogar | Literário: saber}
\end{EntryWithPhonetic}

\begin{EntryWithPhonetic}{审查}{shen3cha2}{8,9}{⼧,⽊}[HSK 6]
  \definition{v.}{examinar; investigar; verificar se algo está correto e apropriado (geralmente referindo"-se a planos, propostas, escritos, qualificações pessoais, etc.); ler e avaliar (provas ou trabalhos de exame)}
\end{EntryWithPhonetic}

\begin{EntryWithPhonetic}{审定}{shen3ding4}{8,8}{⼧,⼧}[HSK 7-9]
  \definition{v.}{examinar e aprovar; examinar e finalizar | autorizar; verificar e decidir}
  \synonymref{鉴定}{jian4ding4}
\end{EntryWithPhonetic}

\begin{EntryWithPhonetic}{审核}{shen3he2}{8,10}{⼧,⽊}[HSK 7-9]
  \definition{v.}{verificar; examinar e confirmar; revisar e aprovar (geralmente referindo"-se a materiais escritos ou digitais)}
  \synonymref{考核}{kao3he2}
  \antonymref{放任}{fang4ren4}
\end{EntryWithPhonetic}

\begin{EntryWithPhonetic}{审美}{shen3mei3}{8,9}{⼧,⽺}[HSK 7-9]
  \definition{v.}{apreciar a beleza; apreciar, discernir e avaliar a beleza das coisas ou obras de arte}
\end{EntryWithPhonetic}

\begin{EntryWithPhonetic}{审判}{shen3pan4}{8,7}{⼧,⼑}[HSK 7-9]
  \definition{v.}{julgar; levar a julgamento; realizar um julgamento}
\end{EntryWithPhonetic}

\begin{EntryWithPhonetic}{审批}{shen3pi1}{8,7}{⼧,⼿}[HSK 7-9]
  \definition{v.}{examinar e aprovar; examinar e dar instruções; revisar e aprovar (planos escritos, relatórios, etc., submetidos por subordinados a superiores)}
\end{EntryWithPhonetic}

\begin{EntryWithPhonetic}{审视}{shen3shi4}{8,8}{⼧,⾒}[HSK 7-9]
  \definition{v.}{examinar; observar; olhar atentamente para algo/alguém; examinar com atenção}
  \synonymref{注视}{zhu4shi4}
\end{EntryWithPhonetic}

%%%%%%%%%% 肾 %%%%%%%%%%
\subsection*{肾}\addcontentsline{loh}{figure}{肾 \dpy{shen4}}

\begin{EntryWithPhonetic}{肾}{shen4}{8}{⾁}[HSK 7-9]
  \definition[个,只]{s.}{rim}
\end{EntryWithPhonetic}

%%%%%%%%%% 甚 %%%%%%%%%%
\subsection*{甚}\addcontentsline{loh}{figure}{甚 \dpy{shen4}}

\begin{EntryWithPhonetic}{甚}{shen4}{9}{⽢}
  \definition{adv.}{muito; extremamente}
  \definition{pron.}{o que}
  \definition{v.}{exceder; superar}
  \seealsoref{什么}{shen2me5}
\end{EntryWithPhonetic}

\begin{EntryWithPhonetic}{甚而}{shen4'er2}{9,6}{⽢,⽽}
  \definition{conj.}{(ir) tão longe quanto | tanto que}
\end{EntryWithPhonetic}

\begin{EntryWithPhonetic}{甚或}{shen4huo4}{9,8}{⽢,⼽}
  \definition{conj.}{(ir) tão longe quanto | tanto que}
\end{EntryWithPhonetic}

\begin{EntryWithPhonetic}{甚至}{shen4zhi4}{9,6}{⽢,⾄}[HSK 4]
  \definition{conj.}{e até mesmo; nem mesmo; para apresentar uma situação típica e especial, para enfatizar a profundidade e a seriedade de uma situação}
\end{EntryWithPhonetic}

\begin{EntryWithPhonetic}{甚至于}{shen4zhi4yu2}{9,6,3}{⽢,⾄,⼆}[HSK 7-9]
  \definition{adv.}{até (na medida em que)}
  \definition{conj.}{(ir) tão longe a ponto de; tanto que; ainda mais}
\end{EntryWithPhonetic}

%%%%%%%%%% 渗 %%%%%%%%%%
\subsection*{渗}\addcontentsline{loh}{figure}{渗 \dpy{shen4}}

\begin{EntryWithPhonetic}{渗}{shen4}{11}{⽔}[HSK 7-9]
  \definition{v.}{infiltrar; permear; infiltrar"-se em}
\end{EntryWithPhonetic}

\begin{EntryWithPhonetic}{渗透}{shen4tou4}{11,10}{⽔,⾡}[HSK 7-9]
  \definition{s.}{osmose; dois gases ou dois líquidos miscíveis são misturados ao passarem por uma membrana porosa}
  \definition{v.}{infiltrar"-se; permear | infiltrar; essa metáfora descreve como algo ou alguma força entra gradualmente em outras áreas}
  \synonymref{分泌}{fen1mi4}
\end{EntryWithPhonetic}

%%%%%%%%%% 慎 %%%%%%%%%%
\subsection*{慎}\addcontentsline{loh}{figure}{慎 \dpy{shen4}}

\begin{EntryWithPhonetic}{慎}{shen4}{13}{⼼}
  \definition*{s.}{Sobrenome: Shen}
  \definition{adj.}{cuidadoso; cauteloso}
\end{EntryWithPhonetic}

\begin{EntryWithPhonetic}{慎重}{shen4zhong4}{13,9}{⼼,⾥}[HSK 7-9]
  \definition{adj.}{cuidadoso; cauteloso; discreto; prudente}
  \synonymref{把稳}{ba3wen3}
  \synonymref{谨慎}{jin3shen4}
  \synonymref{留心}{liu2/xin1}
  \synonymref{留意}{liu2/yi4}
  \synonymref{小心}{xiao3xin5}
\end{EntryWithPhonetic}

%%%%%%%%%% 升 %%%%%%%%%%
\subsection*{升}\addcontentsline{loh}{figure}{升 \dpy{sheng1}}

\begin{EntryWithPhonetic}{升}{sheng1}{4}{⼗}[HSK 3]
  \definition*{s.}{Sobrenome: Sheng}
  \definition{clas.}{litro (l)}
  \definition{s.}{sheng, uma unidade de medida seca para grãos (= 1 litro), um décimo de 斗}
  \definition{v.}{elevar; içar; subir; ascender; subir ou subir mais alto | promover; melhorar (nível)}
  \seealsoref{斗}{dou4}
  \antonymref{降}{jiang4}
\end{EntryWithPhonetic}

\begin{EntryWithPhonetic}{升高}{sheng1gao1}{4,10}{⼗,⾼}[HSK 5]
  \definition{v.}{subir; ascender | promover; elevar; intensificar; potencializar; melhorar}
\end{EntryWithPhonetic}

\begin{EntryWithPhonetic}{升级}{sheng1/ji2}{4,6}{⼗,⽷}[HSK 6]
  \definition{v.+compl.}{atualizar (software) | (guerra) escalar; (tensão) aprofundar | subir um ou mais níveis; passar de uma série ou classe inferior para uma série ou classe superior}
\end{EntryWithPhonetic}

\begin{EntryWithPhonetic}{升起}{sheng1qi3}{4,10}{⼗,⾛}
  \definition{v.}{levantar | içar | subir}
\end{EntryWithPhonetic}

\begin{EntryWithPhonetic}{升温}{sheng1wen1}{4,12}{⼗,⽔}[HSK 7-9]
  \definition{v.}{aquecer; aumentar (temperatura); metaforicamente, isso também se refere a um aumento no nível de atenção que algo recebe}
  \synonymref{加热}{jia1 re4}
\end{EntryWithPhonetic}

\begin{EntryWithPhonetic}{升学}{sheng1 xue2}{4,8}{⼗,⼦}[HSK 6]
  \definition{v.}{ir para uma universidade, faculdade; entrar em uma universidade, faculdade}
\end{EntryWithPhonetic}

\begin{EntryWithPhonetic}{升值}{sheng1zhi2}{4,10}{⼗,⼈}[HSK 6]
  \definition{v.}{Economia: reavaliar; apreciar | Figurativo: aumento de valor | valorização; apreciação; aumentar o valor; aumentar os preços}
\end{EntryWithPhonetic}

%%%%%%%%%% 生 %%%%%%%%%%
\subsection*{生}\addcontentsline{loh}{figure}{生 \dpy{sheng1}}

\begin{EntryWithPhonetic}{生}{sheng1}{5}{⽣}[HSK 2,3][Kangxi 100]
  \definition*{s.}{Sobrenome: Sheng}
  \definition{adj.}{vivo; vital | verde; não maduro | cru; não cozido; mal cozido | bruto; não refinado; não processado | estranho; desconhecido; não familiarizado | rígido; mecânico; forçado}
  \definition{adv.}{muito; usado antes de certas palavras que expressam emoções e sentimentos | verdadeiramente; realmente; forçosamente}
  \definition{s.}{vida | meio de subsistência | aluno; estudante | estudioso; antigamente chamados de eruditos | o tipo de personagem masculino na ópera de Pequim, etc.}
  \definition{suf.}{certos sufixos substantivos que se referem a pessoas (学生) | sufixos de certos advérbios (好生)}
  \definition{v.}{dar à luz; ter um filho | nascer | crescer; cultivar | viver; existir; sobreviver | favorecer; gerar; ocorrer | acender (uma fogueira); fazer o combustível queimar}
  \seealsoref{好生}{hao3sheng1}
  \seealsoref{学生}{xue2sheng5}
\end{EntryWithPhonetic}

\begin{EntryWithPhonetic}{生病}{sheng1/bing4}{5,10}{⽣,⽧}[HSK 1]
  \definition{v.+compl.}{adoecer; ficar doente; ficar mal; contrair uma doença}
\end{EntryWithPhonetic}

\begin{EntryWithPhonetic}{生菜}{sheng1cai4}{5,11}{⽣,⾋}
  \definition{s.}{alface}
\end{EntryWithPhonetic}

\begin{EntryWithPhonetic}{生产}{sheng1chan3}{5,6}{⽣,⼇}[HSK 3]
  \definition{v.}{produzir; fabricar; utilizar ferramentas para mudar o objeto de trabalho e criar meios de produção e meios de subsistência | dar à luz uma criança; ter filhos}
\end{EntryWithPhonetic}

\begin{EntryWithPhonetic}{生成}{sheng1cheng2}{5,6}{⽣,⼽}[HSK 5]
  \definition{v.}{formar; gerar; produzir | ter por natureza; nascer com}
\end{EntryWithPhonetic}

\begin{EntryWithPhonetic}{生词}{sheng1ci2}{5,7}{⽣,⾔}[HSK 2]
  \definition[个,组,堆,条]{s.}{nova palavra; palavras que não aprendi, não conheço ou não entendo}
\end{EntryWithPhonetic}

\begin{EntryWithPhonetic}{生存}{sheng1cun2}{5,6}{⽣,⼦}[HSK 3]
  \definition{v.}{viver; sobreviver; subsistir; manter a vida; estar vivo}
\end{EntryWithPhonetic}

\begin{EntryWithPhonetic}{生的}{sheng1de5}{5,8}{⽣,⽩}
  \definition{conj.}{para evitar isso | para que\dots não\dots}
\end{EntryWithPhonetic}

\begin{EntryWithPhonetic}{生动}{sheng1dong4}{5,6}{⽣,⼒}[HSK 3]
  \definition{adj.}{vívido; animado; descreve a linguagem e as formas de expressão como sendo ativas e em movimento}
\end{EntryWithPhonetic}

\begin{EntryWithPhonetic}{生活}{sheng1huo2}{5,9}{⽣,⽔}[HSK 2]
  \definition[个,段,种]{s.}{vida; subsistência; as diversas atividades realizadas por pessoas ou seres vivos para sobreviver e se desenvolver | estilo de vida; condições de vida; situação em termos de vestuário, alimentação, habitação e transporte | trabalho (principalmente nas áreas industrial, agrícola e artesanal)}
  \definition{v.}{viver; realizar várias atividades | sobreviver}
\end{EntryWithPhonetic}

\begin{EntryWithPhonetic}{生活费}{sheng1huo2fei4}{5,9,9}{⽣,⽔,⾙}[HSK 6]
  \definition{s.}{subsídio; despesas de subsistência; despesas necessárias para manter a vida diária}
\end{EntryWithPhonetic}

\begin{EntryWithPhonetic}{生活垃圾}{sheng1huo2la1ji1}{5,9,8,6}{⽣,⽔,⼟,⼟}
  \definition{s.}{lixo doméstico}
\end{EntryWithPhonetic}

\begin{EntryWithPhonetic}{生活型}{sheng1huo2 xing2}{5,9,9}{⽣,⽔,⼟}
  \definition{s.}{forma de vida}
\end{EntryWithPhonetic}

\begin{EntryWithPhonetic}{生机}{sheng1ji1}{5,6}{⽣,⽊}[HSK 7-9]
  \definition{s.}{chance de sobrevivência | vitalidade}
  \synonymref{活力}{huo2li4}
  \synonymref{期望}{qi1wang4}
  \synonymref{希望}{xi1wang4}
\end{EntryWithPhonetic}

\begin{EntryWithPhonetic}{生理}{sheng1li3}{5,11}{⽣,⽟}[HSK 7-9]
  \definition{adj.}{fisiológico}
  \definition{s.}{fisiologia; as atividades vitais do corpo e as funções de seus diversos órgãos}
\end{EntryWithPhonetic}

\begin{EntryWithPhonetic}{生命}{sheng1ming4}{5,8}{⽣,⼝}[HSK 3]
  \definition{s.}{vida; não envolve apenas a existência e as atividades dos organismos, mas também inclui experiências de vida humana, valores e elementos-chave da sobrevivência e do desenvolvimento de várias coisas}
\end{EntryWithPhonetic}

\begin{EntryWithPhonetic}{生命线}{sheng1ming4xian4}{5,8,8}{⽣,⼝,⽷}[HSK 7-9]
  \definition{s.}{linha da vida; força vital}
\end{EntryWithPhonetic}

\begin{EntryWithPhonetic}{生怕}{sheng1pa4}{5,8}{⽣,⼼}[HSK 7-9]
  \definition{v.}{termer; recear; estar com medo de; estar com receio de; estar preocupado}
  \synonymref{恐怕}{kong3pa4}
\end{EntryWithPhonetic}

\begin{EntryWithPhonetic}{生平}{sheng1ping2}{5,5}{⽣,⼲}[HSK 7-9]
  \definition{s.}{biografia | toda a vida; vida inteira; todo o processo da vida de uma pessoa | desde o nascimento; por toda a minha vida}
  \synonymref{从来}{cong2lai2}
  \synonymref{一生}{yi4sheng1}
\end{EntryWithPhonetic}

\begin{EntryWithPhonetic}{生气}{sheng1/qi4}{5,4}{⽣,⽓}[HSK 1]
  \definition{s.}{vitalidade; vigor; energia da vida}
  \definition{v.+compl.}{ficar com raiva; ficar ofendido; ficar zangado; encontrar algo que não é do seu agrado e sentir"-se descontente}
\end{EntryWithPhonetic}

\begin{EntryWithPhonetic}{生前}{sheng1qian2}{5,9}{⽣,⼑}[HSK 7-9]
  \definition[出]{s.}{antes da morte; durante a vida}
\end{EntryWithPhonetic}

\begin{EntryWithPhonetic}{生日}{sheng1ri5}{5,4}{⽣,⽇}[HSK 1]
  \definition[个,次]{s.}{aniversário; dia de nascimento, também se refere ao dia em que se completa um ano de idade a cada ano}
\end{EntryWithPhonetic}

\begin{EntryWithPhonetic}{生疏}{sheng1shu1}{5,12}{⽣,⽦}
  \definition{adj.}{inexperiência; desconhecimento de assuntos ou situações, falta de experiência suficiente | falta de prática; habilidades e ofícios enferrujam devido à falta de uso ao longo de um longo período de tempo | não tão perto como antes; as relações entre as pessoas tornaram"-se menos íntimas e mais distantes do que antes}
\end{EntryWithPhonetic}

\begin{EntryWithPhonetic}{生死}{sheng1si3}{5,6}{⽣,⽍}[HSK 7-9]
  \definition{s.}{vida e morte}
\end{EntryWithPhonetic}

\begin{EntryWithPhonetic}{生态}{sheng1tai4}{5,8}{⽣,⼼}[HSK 7-9]
  \definition[种]{s.}{ecologia; ecossistema; habitat do organismo; refere"-se às condições de vida e às inter"-relações de vários organismos em um determinado ambiente natural; refere"-se também às características fisiológicas e aos hábitos de vida dos organismos}
  \synonymref{天然}{tian1ran2}
  \synonymref{自然}{zi4ran5}
\end{EntryWithPhonetic}

\begin{EntryWithPhonetic}{生物}{sheng1wu4}{5,8}{⽣,⽜}[HSK 7-9]
  \definition{adj.}{biológico}
  \definition[种,个]{s.}{organismo; ser vivo; todos os seres vivos na natureza, incluindo animais, plantas e microrganismos | biologia; a disciplina que estuda diversos organismos}
  \synonymref{动物}{dong4wu4}
  \synonymref{生命}{sheng1ming4}
\end{EntryWithPhonetic}

\begin{EntryWithPhonetic}{生效}{sheng1/xiao4}{5,10}{⽣,⽁}[HSK 7-9]
  \definition{v.+compl.}{entrar em vigor; tornar"-se efetivo; produzir efeitos}
\end{EntryWithPhonetic}

\begin{EntryWithPhonetic}{生涯}{sheng1ya2}{5,11}{⽣,⽔}[HSK 7-9]
  \definition{s.}{carreira; profissão; refere"-se à vida dedicada a uma determinada atividade ou profissão}
  \synonymref{生活}{sheng1huo2}
\end{EntryWithPhonetic}

\begin{EntryWithPhonetic}{生意}{sheng1yi4}{5,13}{⽣,⼼}
  \definition[笔,种,次]{s.}{tendência a crescer; vitalidade; vigor; energia}
  \seeref{sheng1yi5}
\end{EntryWithPhonetic}

\begin{EntryWithPhonetic}{生意}{sheng1yi5}{5,13}{⽣,⼼}[HSK 3]
  \definition[笔,种,次]{s.}{comércio, compra e venda; negócios; indústria; colegas do mesmo setor}
  \seeref{sheng1yi4}
\end{EntryWithPhonetic}

\begin{EntryWithPhonetic}{生硬}{sheng1ying4}{5,12}{⽣,⽯}[HSK 7-9]
  \definition{adj.}{grosseiro; artificial; não suave (na escrita); foi feito com relutância; foi antinatural; não foi habilidoso | rombudo; rígido; grosseiro; duro; não é gentil; não é meticuloso}
  \synonymref{生疏}{sheng1shu1}
  \antonymref{流利}{liu2li4}
  \antonymref{柔和}{rou2he2}
  \antonymref{熟练}{shu2lian4}
  \antonymref{自然}{zi4ran5}
\end{EntryWithPhonetic}

\begin{EntryWithPhonetic}{生鱼片}{sheng1yu2pian4}{5,8,4}{⽣,⿂,⽚}
  \definition{s.}{fatias de peixe cru, \emph{sashimi}}
\end{EntryWithPhonetic}

\begin{EntryWithPhonetic}{生育}{sheng1yu4}{5,8}{⽣,⾁}[HSK 7-9]
  \definition{v.}{dar à luz; ter um bebê; parir}
\end{EntryWithPhonetic}

\begin{EntryWithPhonetic}{生长}{sheng1zhang3}{5,4}{⽣,⾧}[HSK 3]
  \definition{v.}{cresçer; sob certas condições de vida, o volume e o peso dos organismos aumentam gradualmente | nascer e crescer}
\end{EntryWithPhonetic}

%%%%%%%%%% 声 %%%%%%%%%%
\subsection*{声}\addcontentsline{loh}{figure}{声 \dpy{sheng1}}

\begin{EntryWithPhonetic}{声}{sheng1}{7}{⼠}[HSK 5]
  \definition{clas.}{indica o número de vezes que um som é emitido}
  \definition{s.}{som; voz | reputação | consoante inicial (de uma sílaba chinesa) | tom; tom de voz | informação; notícia}
  \definition{v.}{declarar; anunciar; emitir um som}
\end{EntryWithPhonetic}

\begin{EntryWithPhonetic}{声称}{sheng1cheng1}{7,10}{⼠,⽲}[HSK 7-9]
  \definition{v.}{alegar; declarar publicamente a própria atitude ou situação, mas isso pode não ser ecessariamente verdade}
\end{EntryWithPhonetic}

\begin{EntryWithPhonetic}{声明}{sheng1ming2}{7,8}{⼠,⽇}[HSK 3]
  \definition[项,份]{s.}{declaração}
  \definition{v.}{declarar; anunciar; expressar publicamente a sua atitude ou dizer a verdade}
\end{EntryWithPhonetic}

\begin{EntryWithPhonetic}{声望}{sheng1wang4}{7,11}{⼠,⽉}[HSK 7-9]
  \definition{s.}{renome; prestígio; popularidade; reputação; uma reputação admirada pelas pessoas na sociedade}
  \synonymref{名誉}{ming2yu4}
  \synonymref{名气}{ming2qi5}
  \synonymref{荣誉}{rong2yu4}
  \synonymref{声誉}{sheng1yu4}
\end{EntryWithPhonetic}

\begin{EntryWithPhonetic}{声音}{sheng1yin1}{7,9}{⼠,⾳}[HSK 2]
  \definition[个,种]{s.}{som; voz; a percepção auditiva das ondas sonoras}
\end{EntryWithPhonetic}

\begin{EntryWithPhonetic}{声誉}{sheng1yu4}{7,13}{⼠,⾔}[HSK 7-9]
  \definition{s.}{fama; prestígio; reputação}
  \synonymref{光荣}{guang1rong2}
  \synonymref{口碑}{kou3bei1}
  \synonymref{名誉}{ming2yu4}
  \synonymref{荣誉}{rong2yu4}
  \synonymref{声望}{sheng1wang4}
\end{EntryWithPhonetic}

%%%%%%%%%% 牲 %%%%%%%%%%
\subsection*{牲}\addcontentsline{loh}{figure}{牲 \dpy{sheng1}}

\begin{EntryWithPhonetic}{牲}{sheng1}{9}{⽜}
  \definition[头]{s.}{gado (para sacrifício) | sacrifício de animais | animal doméstico}
\end{EntryWithPhonetic}

\begin{EntryWithPhonetic}{牲畜}{sheng1chu4}{9,10}{⽜,⽥}[HSK 7-9]
  \definition[种,群]{s.}{gado; animais domésticos}
\end{EntryWithPhonetic}

%%%%%%%%%% 绳 %%%%%%%%%%
\subsection*{绳}\addcontentsline{loh}{figure}{绳 \dpy{sheng2}}

\begin{EntryWithPhonetic}{绳}{sheng2}{11}{⽷}
  \definition*{s.}{Sobrenome: Sheng}
  \definition[根]{s.}{corda; cordão; barbante | a linha no marcador de tinta de carpinteiro}
  \definition{v.}{restringir; corrigir; sancionar | medir | continuar}
\end{EntryWithPhonetic}

\begin{EntryWithPhonetic}{绳子}{sheng2zi5}{11,3}{⽷,⼦}[HSK 7-9]
  \definition[条,根]{s.}{corda; barbante; fio; um objeto longo e estreito feito torcendo dois ou mais fios de linha, palha ou cânhamo, frequentemente usado para amarrar ou puxar coisas}
\end{EntryWithPhonetic}

%%%%%%%%%% 省 %%%%%%%%%%
\subsection*{省}\addcontentsline{loh}{figure}{省 \dpy{sheng3}}

\begin{EntryWithPhonetic}{省}{sheng3}{9}{⽬}[HSK 2]
  \definition*{s.}{Sobrenome: Sheng}
  \definition{s.}{província; unidade administrativa, subordinada diretamente ao governo central | capital provincial; refere"-se à capital da província, localização da administração provincial | abreviação (de palavras)}
  \definition{v.}{economizar; poupar; reduzir o consumo | omitir; deixar de fora}
  \seeref{xing3}
  \antonymref{费}{fei4}
\end{EntryWithPhonetic}

\begin{EntryWithPhonetic}{省城}{sheng3cheng2}{9,9}{⽬,⼟}
  \definition{s.}{capital da província}
  \synonymref{省会}{sheng3hui4}
  \synonymref{首府}{shou3fu3}
\end{EntryWithPhonetic}

\begin{EntryWithPhonetic}{省会}{sheng3hui4}{9,6}{⽬,⼈}
  \definition{s.}{capital da província}
  \synonymref{省城}{sheng3cheng2}
\end{EntryWithPhonetic}

\begin{EntryWithPhonetic}{省俭}{sheng3jian3}{9,9}{⽬,⼈}
  \definition{s.}{econômico | frugal}
  \definition{v.}{economizar}
\end{EntryWithPhonetic}

\begin{EntryWithPhonetic}{省力}{sheng3li4}{9,2}{⽬,⼒}
  \definition{v.}{economizar esforço ou trabalho}
\end{EntryWithPhonetic}

\begin{EntryWithPhonetic}{省略}{sheng3lve4}{9,11}{⽬,⽥}[HSK 7-9]
  \definition{v.}{omitir; excluir; apagar; eliminar linguagem e procedimentos desnecessários}
  \synonymref{减少}{jian3shao3}
  \synonymref{删除}{shan1chu2}
\end{EntryWithPhonetic}

\begin{EntryWithPhonetic}{省钱}{sheng3qian2}{9,10}{⽬,⾦}[HSK 6]
  \definition{adj.}{barato; não caro}
  \definition{v.}{economizar dinheiro}
\end{EntryWithPhonetic}

\begin{EntryWithPhonetic}{省却}{sheng3que4}{9,7}{⽬,⼙}
  \definition{v.}{livrar"-se (para economizar espaço) | salvar}
\end{EntryWithPhonetic}

\begin{EntryWithPhonetic}{省事}{sheng3/shi4}{9,8}{⽬,⼅}[HSK 7-9]
  \definition{adj.}{conveniente; prático; sem problemas}
  \definition{v.+compl.}{evitar problemas; simplificar as coisas}
\end{EntryWithPhonetic}

\begin{EntryWithPhonetic}{省心}{sheng3xin1}{9,4}{⽬,⼼}
  \definition{adj.}{despreocupado}
  \definition{v.}{ser poupado de preocupações | despreocupar"-se}
\end{EntryWithPhonetic}

\begin{EntryWithPhonetic}{省长}{sheng3zhang3}{9,4}{⽬,⾧}
  \definition[位,任]{s.}{governador; governador de uma província}
\end{EntryWithPhonetic}

%%%%%%%%%% 圣 %%%%%%%%%%
\subsection*{圣}\addcontentsline{loh}{figure}{圣 \dpy{sheng4}}

\begin{EntryWithPhonetic}{圣}{sheng4}{5}{⼟}
  \definition*{s.}{Sobrenome: Sheng}
  \definition{adj.}{santo; sagrado | imperial}
  \definition{s.}{santo; sábio | imperador | o maior mestre de uma determinada arte ou habilidade}
\end{EntryWithPhonetic}

\begin{EntryWithPhonetic}{圣诞节}{sheng4dan4 jie2}{5,8,5}{⼟,⾔,⾋}[HSK 6]
  \definition*{s.}{Natal; Nascimento de Jesus Cristo em 25 de dezembro}
\end{EntryWithPhonetic}

\begin{EntryWithPhonetic}{圣地}{sheng4di4}{5,6}{⼟,⼟}
  \definition{s.}{terra santa (de uma religião) | lugar sagrado | santuário | cidade santa (como Jerusalém, Meca, etc.) | centro de interesse histórico}
\end{EntryWithPhonetic}

\begin{EntryWithPhonetic}{圣贤}{sheng4xian2}{5,8}{⼟,⾙}[HSK 7-9]
  \definition{s.}{sábios e homens de virtude}
\end{EntryWithPhonetic}

%%%%%%%%%% 胜 %%%%%%%%%%
\subsection*{胜}\addcontentsline{loh}{figure}{胜 \dpy{sheng4}}

\begin{EntryWithPhonetic}{胜}{sheng4}{9}{⾁}[HSK 3]
  \definition{adj.}{soberbo; maravilhoso; adorável}
  \definition[场]{s.}{vitória; sucesso | penteado de mulher; joias usadas pelas mulheres na antiguidade}
  \definition{v.}{vencer | derrotar | (frequentemente seguido por 于, etc.) superar; ser superior a; levar a melhor sobre | vencer; ter sucesso; derrotar o adversário | ultrapassar; ser superior ao outro | suportar; ser capaz de suportar ou aguentar}
  \seealsoref{于}{yu2}
  \antonymref{败}{bai4}
  \antonymref{负}{fu4}
\end{EntryWithPhonetic}

\begin{EntryWithPhonetic}{胜出}{sheng4chu1}{9,5}{⾁,⼐}[HSK 7-9]
  \definition{v.}{(em um jogo, competição, etc.) superar; vencer; derrotar um oponente | ganhar}
  \antonymref{出局}{chu1/ju2}
\end{EntryWithPhonetic}

\begin{EntryWithPhonetic}{胜负}{sheng4-fu4}{9,6}{⾁,⾙}[HSK 5]
  \definition{s.}{vitória ou derrota; sucesso ou fracasso}
\end{EntryWithPhonetic}

\begin{EntryWithPhonetic}{胜利}{sheng4li4}{9,7}{⾁,⼑}[HSK 3]
  \definition{adv.}{com sucesso; triunfantemente; atingir o objetivo previsto}
  \definition{v.}{ganhar; vencer; triunfar; ter sucesso}
\end{EntryWithPhonetic}

\begin{EntryWithPhonetic}{胜任}{sheng4ren4}{9,6}{⾁,⼈}[HSK 7-9]
  \definition{v.}{ser competente; ser qualificado; ser igual a; estar à altura de; possuir capacidade suficiente para desempenhar (o trabalho, a função, etc.)}
  \synonymref{担当}{dan1dang1}
  \synonymref{担任}{dan1ren4}
\end{EntryWithPhonetic}

\begin{EntryWithPhonetic}{胜算}{sheng4suan4}{9,14}{⾁,⽵}
  \definition{s.}{probabilidade de sucesso | estratégia que garante o sucesso}
  \definition{v.}{ter certeza do sucesso}
\end{EntryWithPhonetic}

%%%%%%%%%% 乘 %%%%%%%%%%
\subsection*{乘}\addcontentsline{loh}{figure}{乘 \dpy{sheng4}}

\begin{EntryWithPhonetic}{乘}{sheng4}{10}{⽲}
  \definition{clas.}{usado para carruagens de guerra puxada por quatro cavalos}
  \definition{s.}{obras históricas; livros de história geral | antigamente, uma carruagem puxada por quatro cavalos}
  \seeref{cheng2}
\end{EntryWithPhonetic}

%%%%%%%%%% 盛 %%%%%%%%%%
\subsection*{盛}\addcontentsline{loh}{figure}{盛 \dpy{sheng4}}

\begin{EntryWithPhonetic}{盛}{sheng4}{11}{⽫}
  \definition*{s.}{Sobrenome: Sheng}
  \definition{adj.}{florescente; próspero | vigoroso; enérgico | grandioso; magnífico | abundante; profundo | popular; comum; difundido; universal | amplo; generoso; abundante; suficiente | ótimo}
  \definition{adv.}{muito; profundamente}
  \seeref{cheng2}
\end{EntryWithPhonetic}

\begin{EntryWithPhonetic}{盛大}{sheng4da4}{11,3}{⽫,⼤}[HSK 7-9]
  \definition{adj.}{grandioso; magnífico; em grande escala; solene (atividade em grupo)}
  \synonymref{广大}{guang3da4}
  \synonymref{广泛}{guang3fan4}
  \synonymref{广阔}{guang3kuo4}
  \synonymref{隆重}{long2zhong4}
  \synonymref{无边}{wu2bian1}
  \antonymref{简陋}{jian3lou4}
\end{EntryWithPhonetic}

\begin{EntryWithPhonetic}{盛会}{sheng4hui4}{11,6}{⽫,⼈}[HSK 7-9]
  \definition{s.}{grande evento; reunião ilustre; encontro importante}
  \synonymref{佳节}{jia1jie2}
\end{EntryWithPhonetic}

\begin{EntryWithPhonetic}{盛开}{sheng4kai1}{11,4}{⽫,⼶}[HSK 7-9]
  \definition{v.}{estar em plena floração; (flores) desabrochar em abundância}
  \synonymref{开放}{kai1fang4}
  \synonymref{怒放}{nu4fang4}
\end{EntryWithPhonetic}

\begin{EntryWithPhonetic}{盛气凌人}{sheng4qi4-ling2ren2}{11,4,10,2}{⽫,⽓,⼎,⼈}[HSK 7-9]
  \definition{expr.}{dominador; arrogante; autoritário e prepotente; valentão arrogante; dominante; autoritário}
  \synonymref{目中无人}{mu4zhong1-wu2ren2}
\end{EntryWithPhonetic}

\begin{EntryWithPhonetic}{盛行}{sheng4xing2}{11,6}{⽫,⾏}[HSK 6]
  \definition{v.}{predominar; estar atual; estar na moda; ser amplamente popular}
\end{EntryWithPhonetic}

\begin{EntryWithPhonetic}{盛宴}{sheng4yan4}{11,10}{⽫,⼧}
  \definition{s.}{celebração}
\end{EntryWithPhonetic}

%%%%%%%%%% 剩 %%%%%%%%%%
\subsection*{剩}\addcontentsline{loh}{figure}{剩 \dpy{sheng4}}

\begin{EntryWithPhonetic}{剩}{sheng4}{12}{⼑}[HSK 5]
  \definition*{s.}{Sobrenome: Sheng}
  \definition{v.}{permanecer; ser deixado (para trás)}
\end{EntryWithPhonetic}

\begin{EntryWithPhonetic}{剩下}{sheng4/xia5}{12,3}{⼑,⼀}[HSK 5]
  \definition{v.+compl.}{permanecer; ser deixado (para trás); consumir e utilizar, restando apenas os resíduos}
\end{EntryWithPhonetic}

\begin{EntryWithPhonetic}{剩余}{sheng4yu2}{12,7}{⼑,⼈}[HSK 7-9]
  \definition{s.}{excedente; restante; o que sobra depois de subtrair uma porção de uma determinada quantidade}
  \definition{v.}{sobrar; restar; ser excedente}
  \antonymref{短缺}{duan3que1}
  \antonymref{缺少}{que1shao3}
\end{EntryWithPhonetic}

%%%%%%%%%% 尸 %%%%%%%%%%
\subsection*{尸}\addcontentsline{loh}{figure}{尸 \dpy{shi1}}

\begin{EntryWithPhonetic}{尸}{shi1}{3}{⼫}[Kangxi 44]
  \definition*{s.}{Sobrenome: Shi}
  \definition[具]{s.}{cadáver; corpo morto; restos mortais | Arcaico: pessoa que se sentava atrás do altar, representando o falecido durante a realização de ritos sacrificiais}
  \definition{v.}{manter um emprego sem fazer nada (como um cadáver) | dirigir; agir como responsável | dispor}
\end{EntryWithPhonetic}

\begin{EntryWithPhonetic}{尸体}{shi1ti3}{3,7}{⼫,⼈}[HSK 7-9]
  \definition[具,个]{s.}{cadáver; restos mortais; corpo morto; os corpos de humanos e animais após a morte}
\end{EntryWithPhonetic}

%%%%%%%%%% 失 %%%%%%%%%%
\subsection*{失}\addcontentsline{loh}{figure}{失 \dpy{shi1}}

\begin{EntryWithPhonetic}{失}{shi1}{5}{⼤}
  \definition{s.}{deslize; erro; defeito; acidente}
  \definition{v.}{perder | perder; deixar escapar | não agir de acordo com; negligenciar; violar | perder o controle de | errar; cometer um deslize; apresentar defeito em | não consiguir encontrar | não conseguir atingir o objetivo | desviar-se do normal | quebrar (uma promessa); voltar atrás (na palavra dada) | não conseguir obter | se perder}
  \antonymref{得}{de2}
\end{EntryWithPhonetic}

\begin{EntryWithPhonetic}{失败}{shi1bai4}{5,8}{⼤,⾒}[HSK 4]
  \definition{adj.}{insatisfatório; a maneira como as coisas aconteceram deixou muito a desejar; o resultado final deixou muito a desejar}
  \definition{v.}{perder; ser derrotado; não vencer em uma guerra ou competição | falhar; fracassar; não dar em nada; falhar em atingir um objetivo ou meta desejada (trabalho, carreira, etc.)}
\end{EntryWithPhonetic}

\begin{EntryWithPhonetic}{失传}{shi1chuan2}{5,6}{⼤,⼈}[HSK 7-9]
  \definition{v.}{não ser transmitido de geração em geração; ser perdido; estar perdido}
  \synonymref{消失}{xiao1shi1}
  \antonymref{流传}{liu2chuan2}
\end{EntryWithPhonetic}

\begin{EntryWithPhonetic}{失控}{shi1kong4}{5,11}{⼤,⼿}[HSK 7-9]
  \definition{v.}{perder o controle; ficar fora de controle}
  \synonymref{放纵}{fang4zong4}
  \antonymref{控制}{kong4zhi4}
  \antonymref{掌握}{zhang3wo4}
\end{EntryWithPhonetic}

\begin{EntryWithPhonetic}{失利}{shi1/li4}{5,7}{⼤,⼑}[HSK 7-9]
  \definition{v.+compl.}{sofrer um revés (ou derrota); perder uma batalha; ser derrotado; perder uma competição}
  \synonymref{腐败}{fu3bai4}
  \synonymref{失败}{shi1bai4}
  \antonymref{取胜}{qu3sheng4}
  \antonymref{顺利}{shun4li4}
\end{EntryWithPhonetic}

\begin{EntryWithPhonetic}{失恋}{shi1/lian4}{5,10}{⼤,⼼}[HSK 7-9]
  \definition{v.+compl.}{ser rejeitado(a); ficar desapontado(a) em um relacionamento amoroso}
\end{EntryWithPhonetic}

\begin{EntryWithPhonetic}{失灵}{shi1ling2}{5,7}{⼤,⽕}[HSK 7-9]
  \definition{v.+compl.}{(máquina, instrumento, etc.) não funcionar; não funcionar corretamente; estar fora de serviço; estar travado}
  \synonymref{故障}{gu4zhang4}
  \antonymref{见效}{jian4xiao4}
\end{EntryWithPhonetic}

\begin{EntryWithPhonetic}{失落}{shi1luo4}{5,12}{⼤,⾋}[HSK 7-9]
  \definition{adj.}{perdido; chateado; frustrado}
  \definition{v.}{perder (algo); sentir uma sensação de perda}
  \synonymref{丢失}{diu1shi1}
  \synonymref{沮丧}{ju3sang4}
  \synonymref{难受}{nan2shou4}
  \synonymref{丧失}{sang4shi1}
  \synonymref{失去}{shi1qu4}
  \synonymref{失意}{shi1yi4}
  \antonymref{存在}{cun2zai4}
  \antonymref{得到}{de2 dao4}
  \antonymref{兴奋}{xing1fen4}
\end{EntryWithPhonetic}

\begin{EntryWithPhonetic}{失眠}{shi1/mian2}{5,10}{⼤,⽬}[HSK 7-9]
  \definition{s.}{ficar sem dormir; perder o sono; sofrer de insônia; ter dificuldade para adormecer à noite ou não conseguir voltar a dormir depois de acordar}
  \definition{s.}{insônia; dificuldade para dormir; um estado de distúrbio do sono}
\end{EntryWithPhonetic}

\begin{EntryWithPhonetic}{失明}{shi1/ming2}{5,8}{⼤,⽇}[HSK 7-9]
  \definition{v.+compl.}{perder a visão; ficar cego}
\end{EntryWithPhonetic}

\begin{EntryWithPhonetic}{失去}{shi1qu4}{5,5}{⼤,⼛}[HSK 3]
  \definition{v.}{perder}
\end{EntryWithPhonetic}

\begin{EntryWithPhonetic}{失望}{shi1wang4}{5,11}{⼤,⽉}[HSK 4]
  \definition{adj.}{desapontado; decepcionado}
  \definition{v.}{ficar desapontado; ficar decepcionado; estar desapontado; sentir"-se sem esperança; perder a confiança}
\end{EntryWithPhonetic}

\begin{EntryWithPhonetic}{失误}{shi1wu4}{5,9}{⼤,⾔}[HSK 5]
  \definition[个]{s.}{erro; engano; equívoco; erros causados por negligência ou medidas inadequadas}
  \definition{v.}{cometer um erro; cometer um equívoco}
\end{EntryWithPhonetic}

\begin{EntryWithPhonetic}{失效}{shi1/xiao4}{5,10}{⼤,⽁}[HSK 7-9]
  \definition{v.+compl.}{perder a eficácia; perder a efetividade; deixar de ser eficaz; perder o efeito | tornar"-se inválido; deixar de estar em vigor (um tratado, um acordo, etc.)}
  \antonymref{见效}{jian4xiao4}
  \antonymref{生效}{sheng1/xiao4}
  \antonymref{效果}{xiao4guo3}
  \antonymref{有效}{you3xiao4}
  \antonymref{奏效}{zou4xiao4}
\end{EntryWithPhonetic}

\begin{EntryWithPhonetic}{失业}{shi1/ye4}{5,5}{⼤,⼀}[HSK 4]
  \definition{v.+compl.}{não ter emprego; estar desempregado; estar sem trabalho; refere"-se àqueles que estão dentro da idade legal para trabalhar, têm capacidade para trabalhar, estão desempregados e querem encontrar um emprego, mas não conseguem; embora se envolvam em certos trabalhos sociais, sua remuneração é menor do que o padrão mínimo de vida urbano local e são considerados desempregados}
\end{EntryWithPhonetic}

\begin{EntryWithPhonetic}{失业率}{shi1ye4lv4}{5,5,11}{⼤,⼀,⽞}[HSK 7-9]
  \definition{s.}{taxa de desemprego; taxa de desempregados}
\end{EntryWithPhonetic}

\begin{EntryWithPhonetic}{失意}{shi1yi4}{5,13}{⼤,⼼}
  \definition{adj.}{desapontado | frustrado}
\end{EntryWithPhonetic}

\begin{EntryWithPhonetic}{失踪}{shi1/zong1}{5,15}{⼤,⾜}[HSK 7-9]
  \definition{v.+compl.}{desaparecer; estar ausente; significa que o paradeiro é desconhecido e nenhum vestígio pode ser encontrado}
  \synonymref{失落}{shi1luo4}
\end{EntryWithPhonetic}

%%%%%%%%%% 师 %%%%%%%%%%
\subsection*{师}\addcontentsline{loh}{figure}{师 \dpy{shi1}}

\begin{EntryWithPhonetic}{师}{shi1}{6}{⼱}
  \definition*{s.}{Sobrenome: Shi}
  \definition[位,名,个]{s.}{professor; tutor; mestre | exemplo; modelo a seguir | título honorífico para um monge budista; (termo de respeito para um monge ou freira) mestre; mãe | do seu mestre ou professor | divisão; tropas; exército}
  \definition{suf.}{pessoa qualificada em determinada profissão}
  \definition{v.}{Literário: imitar; aprender}
\end{EntryWithPhonetic}

\begin{EntryWithPhonetic}{师范}{shi1fan4}{6,9}{⼱,⾋}[HSK 7-9]
  \definition{s.}{pedagogia; formação de professores; escola normal; escola especializada na formação de professores, abreviação de 师范学校, 师范学院 ou 师范大学 | modelo; uma pessoa de virtude exemplar}
  \seealsoref{师范大学}{shi1fan4 da4xue2}
  \seealsoref{师范学校}{shi1fan4 xue2xiao4}
  \seealsoref{师范学院}{shi1fan4 xue2yuan4}
\end{EntryWithPhonetic}

\begin{EntryWithPhonetic}{师范大学}{shi1fan4 da4xue2}{6,9,3,8}{⼱,⾋,⼤,⼦}
  \definition{s.}{universidade normal; universidade (normal) de professores; universidade de pedagogia}
\end{EntryWithPhonetic}

\begin{EntryWithPhonetic}{师范学校}{shi1fan4 xue2xiao4}{6,9,8,10}{⼱,⾋,⼦,⽊}
  \definition{s.}{escola normal; escolas especializadas na formação de professores}
\end{EntryWithPhonetic}

\begin{EntryWithPhonetic}{师范学院}{shi1fan4 xue2yuan4}{6,9,8,9}{⼱,⾋,⼦,⾩}
  \definition{s.}{faculdade de formação de professores; faculdade de pedagogia}
\end{EntryWithPhonetic}

\begin{EntryWithPhonetic}{师父}{shi1fu5}{6,4}{⼱,⽗}[HSK 6]
  \definition[个,位,名,些]{s.}{mestre; mestre trabalhador; um título respeitoso dado por um aprendiz ao seu mestre | um título respeitoso para monges, freiras e sacerdotes taoístas}
\end{EntryWithPhonetic}

\begin{EntryWithPhonetic}{师傅}{shi1fu5}{6,12}{⼱,⼈}[HSK 5]
  \definition[个,位,名]{s.}{mestre; um trabalhador qualificado; título honorífico para pessoas habilidosas | mestre; professor (em certos ofícios); pessoas que ensinam técnicas em áreas como engenharia, comércio e teatro}
\end{EntryWithPhonetic}

\begin{EntryWithPhonetic}{师生}{shi1sheng1}{6,5}{⼱,⽣}[HSK 6]
  \definition{s.}{mestre e discípulo; professores e alunos; um nome combinado para professores e alunos}
\end{EntryWithPhonetic}

\begin{EntryWithPhonetic}{师长}{shi1zhang3}{6,4}{⼱,⾧}[HSK 7-9]
  \definition{s.}{Cortês: professor | Militar: comandante de divisão}
  \synonymref{教授}{jiao4shou4}
  \synonymref{老师}{lao3shi1}
  \synonymref{先生}{xian1sheng5}
  \antonymref{学生}{xue2sheng5}
\end{EntryWithPhonetic}

\begin{EntryWithPhonetic}{师资}{shi1zi1}{6,10}{⼱,⾙}[HSK 7-9]
  \definition{s.}{pessoas qualificadas para ensinar; indivíduos talentosos e qualificados para serem professores; professores | corpo docente; qualificações dos professores; recursos para o corpo docente}
\end{EntryWithPhonetic}

%%%%%%%%%% 诗 %%%%%%%%%%
\subsection*{诗}\addcontentsline{loh}{figure}{诗 \dpy{shi1}}

\begin{EntryWithPhonetic}{诗}{shi1}{8}{⾔}[HSK 4]
  \definition[首,句,行]{s.}{poesia; verso; poema; um gênero literário que reflete a vida e expressa emoções por meio de uma linguagem rítmica e rimada}
  \seealsoref{诗经}{shi1jing1}
\end{EntryWithPhonetic}

\begin{EntryWithPhonetic}{诗词}{shi1ci2}{8,7}{⾔,⾔}
  \definition{s.}{verso}
\end{EntryWithPhonetic}

\begin{EntryWithPhonetic}{诗歌}{shi1ge1}{8,14}{⾔,⽋}[HSK 5]
  \definition[本,首,段]{s.}{poesia; poemas e canções; refere"-se a todos os tipos de poesia}
\end{EntryWithPhonetic}

\begin{EntryWithPhonetic}{诗经}{shi1jing1}{8,8}{⾔,⽷}
  \definition*{s.}{Shijing, o Livro das Canções, antiga coleção de poemas chineses e um dos Cinco Clássicos do Confucionismo}
\end{EntryWithPhonetic}

\begin{EntryWithPhonetic}{诗句}{shi1ju4}{8,5}{⾔,⼝}
  \definition[行]{s.}{verso | versículo}
\end{EntryWithPhonetic}

\begin{EntryWithPhonetic}{诗人}{shi1ren2}{8,2}{⾔,⼈}[HSK 4]
  \definition[个,位,名,些]{s.}{poeta; escritor de poesia}
\end{EntryWithPhonetic}

\begin{EntryWithPhonetic}{诗意}{shi1yi4}{8,13}{⾔,⼼}
  \definition{adj.}{poético}
  \definition{s.}{poesia}
\end{EntryWithPhonetic}

%%%%%%%%%% 施 %%%%%%%%%%
\subsection*{施}\addcontentsline{loh}{figure}{施 \dpy{shi1}}

\begin{EntryWithPhonetic}{施}{shi1}{9}{⽅}
  \definition*{s.}{Sobrenome: Shi}
  \definition{v.}{pôr em prática; executar; realizar | outorgar; conceder; distribuir | exercer; impor | usar; aplicar}
\end{EntryWithPhonetic}

\begin{EntryWithPhonetic}{施工}{shi1/gong1}{9,3}{⽅,⼯}[HSK 7-9]
  \definition{v.+compl.}{construir; realizar obras (ou grandes reparos); construir casas, pontes, estradas, projetos de conservação de água, etc., de acordo com as especificações e requisitos do projeto}
  \synonymref{动工}{dong4/gong1}
  \synonymref{开工}{kai1/gong1}
  \antonymref{竣工}{jun4gong1}
\end{EntryWithPhonetic}

\begin{EntryWithPhonetic}{施加}{shi1jia1}{9,5}{⽅,⼒}[HSK 7-9]
  \definition{v.}{exercer (pressão, influência, etc.); aplicar sobre}
  \antonymref{承受}{cheng2shou4}
  \antonymref{遭受}{zao1shou4}
\end{EntryWithPhonetic}

\begin{EntryWithPhonetic}{施行}{shi1xing2}{9,6}{⽅,⾏}[HSK 7-9]
  \definition{v.}{aplicar; implementar; implementar de uma determinada maneira ou forma | executar; entrar em vigor; leis, regulamentos e normas entrarão em vigor após a sua promulgação}
  \synonymref{实施}{shi2shi1}
  \synonymref{执行}{zhi2xing2}
  \antonymref{废除}{fei4chu2}
\end{EntryWithPhonetic}

\begin{EntryWithPhonetic}{施压}{shi1ya1}{9,6}{⽅,⼚}[HSK 7-9]
  \definition{v.}{aplicar pressão; pressionar |exercer pressão sobre}
\end{EntryWithPhonetic}

%%%%%%%%%% 狮 %%%%%%%%%%
\subsection*{狮}\addcontentsline{loh}{figure}{狮 \dpy{shi1}}

\begin{EntryWithPhonetic}{狮}{shi1}{9}{⽝}
  \definition[只,头]{s.}{leão}
\end{EntryWithPhonetic}

\begin{EntryWithPhonetic}{狮子}{shi1zi5}{9,3}{⽝,⼦}[HSK 7-9]
  \definition[头,只,群]{s.}{leão}
\end{EntryWithPhonetic}

%%%%%%%%%% 湿 %%%%%%%%%%
\subsection*{湿}\addcontentsline{loh}{figure}{湿 \dpy{shi1}}

\begin{EntryWithPhonetic}{湿}{shi1}{12}{⽔}[HSK 4]
  \definition{adj.}{molhado; úmido; algo com água ou com muita água dentro}
\end{EntryWithPhonetic}

\begin{EntryWithPhonetic}{湿度}{shi1du4}{12,9}{⽔,⼴}[HSK 7-9]
  \definition{s.}{umidade; a quantidade de água contida em uma substância; especificamente, a quantidade de umidade contida no ar}
\end{EntryWithPhonetic}

\begin{EntryWithPhonetic}{湿润}{shi1run4}{12,10}{⽔,⽔}[HSK 7-9]
  \definition{adj.}{úmido; molhado; (solo, ar, etc.) úmido e hidratado}
  \synonymref{潮湿}{chao2shi1}
  \antonymref{干燥}{gan1zao4}
\end{EntryWithPhonetic}

%%%%%%%%%% 十 %%%%%%%%%%
\subsection*{十}\addcontentsline{loh}{figure}{十 \dpy{shi2}}

\begin{EntryWithPhonetic}{十}{shi2}{2}{⼗}[HSK 1][Kangxi 24]
  \definition*{s.}{Sobrenome: Shi}
  \definition{num.}{dez; 10 | dezena | completo; no topo; máximo; referindo"-se a algo que atingiu o ápice da perfeição ou plenitude | um monte de; indica que há muitos}
\end{EntryWithPhonetic}

\begin{EntryWithPhonetic}{十分}{shi2fen1}{2,4}{⼗,⼑}[HSK 2]
  \definition{adv.}{muito; totalmente; completamente; extremamente; indica um nível muito alto}
\end{EntryWithPhonetic}

\begin{EntryWithPhonetic}{十字路口}{shi2zi4 lu4kou3}{2,6,13,3}{⼗,⼦,⾜,⼝}[HSK 7-9]
  \definition{expr.}{encruzilhada; cruzamento; interseção; ponto de virada; uma encruzilhada, um lugar onde duas estradas se cruzam, é uma metáfora para uma situação em que se deve escolher um caminho a seguir em uma questão crucial}
\end{EntryWithPhonetic}

\begin{EntryWithPhonetic}{十足}{shi2zu2}{2,7}{⼗,⾜}[HSK 5]
  \definition{adj.}{puro e simples; apenas este componente ou esta característica é muito evidente | 100\%; completo; total; muito satisfatório; muito adequado}
\end{EntryWithPhonetic}

%%%%%%%%%% 什 %%%%%%%%%%
\subsection*{什}\addcontentsline{loh}{figure}{什 \dpy{shi2}}

\begin{EntryWithPhonetic}{什}{shi2}{4}{⼈}
  \definition*{s.}{Sobrenome: Shi}
  \definition{adj.}{variado; sortido; diverso; vários; misturados}
  \definition{num.}{(em frações ou múltiplos) dez}
  \definition{s.}{várias coisas; artigos diversos}
  \seeref{shen2}
\end{EntryWithPhonetic}

%%%%%%%%%% 石 %%%%%%%%%%
\subsection*{石}\addcontentsline{loh}{figure}{石 \dpy{shi2}}

\begin{EntryWithPhonetic}{石}{shi2}{5}{⽯}[Kangxi 112]
  \definition*{s.}{Sobrenome: Shi}
  \definition{s.}{pedra; rocha; o material duro que constitui a crosta terrestre é composto por uma coleção de minerais | inscrição em pedra; esculturas em pedra}
  \seeref{dan4}
\end{EntryWithPhonetic}

\begin{EntryWithPhonetic}{石头}{shi2tou5}{5,5}{⽯,⼤}[HSK 3]
  \definition[块,堆,些]{s.}{rocha; pedra; uma substância muito dura que é o principal material da superfície da Terra}
\end{EntryWithPhonetic}

\begin{EntryWithPhonetic}{石油}{shi2you2}{5,8}{⽯,⽔}[HSK 3]
  \definition[桶,吨,升]{s.}{óleo; óleo fóssil; petróleo; um líquido inflamável extraído do solo, geralmente marrom escuro, preto ou verde escuro, do qual gasolina e outras substâncias podem ser obtidas}
\end{EntryWithPhonetic}

%%%%%%%%%% 时 %%%%%%%%%%
\subsection*{时}\addcontentsline{loh}{figure}{时 \dpy{shi2}}

\begin{EntryWithPhonetic}{时}{shi2}{7}{⽇}[HSK 3]
  \definition*{s.}{Sobrenome: Shi}
  \definition{adj.}{atual; presente | temporário; oportuno}
  \definition{adv.}{de vez em quando; ocasionalmente; de tempos em tempos; equivalente a 常常 ou 经常 | às vezes\dots às vezes\dots; dois caracteres 时 usados juntos são equivalentes a ``有时…有时…'' e ``一会儿…一会儿…''}
  \definition{clas.}{hora, cada uma das 24 partes iguais de um dia e uma noite; também usada como unidade legal de tempo}
  \definition{s.}{dias; tempos; longo período de tempo; refere"-se a um período de tempo | tempo; tempo fixo; refere"-se ao tempo especificado | hora; hora do dia | temporada | chance; oportunidade; momento oportuno | atual; presente | tempo verbal; uma categoria gramatical que utiliza certas formas gramaticais para indicar o momento em que uma ação ocorre; geralmente é dividida em presente, pretérito e futuro}
  \seealsoref{常常}{chang2chang2}
  \seealsoref{经常}{jing1chang2}
  \seealsoref{一会儿……一会儿……}{yi1hui4r5 yi1hui4r5}
  \seealsoref{有时……有时……}{you3shi2 you3shi2}
\end{EntryWithPhonetic}

\begin{EntryWithPhonetic}{时不时}{shi2bu5shi2}{7,4,7}{⽇,⼀,⽇}[HSK 7-9]
  \definition{adv.}{frequentemente; muitas vezes; de tempos em tempos}
  \synonymref{动不动}{dong4bu5dong4}
\end{EntryWithPhonetic}

\begin{EntryWithPhonetic}{时差}{shi2cha1}{7,9}{⽇,⼯}
  \definition{s.}{diferença de tempo | \emph{jet lag}}
\end{EntryWithPhonetic}

\begin{EntryWithPhonetic}{时常}{shi2chang2}{7,11}{⽇,⼱}[HSK 5]
  \definition{adv.}{frequentemente; com frequência}
\end{EntryWithPhonetic}

\begin{EntryWithPhonetic}{时代}{shi2dai4}{7,5}{⽇,⼈}[HSK 3]
  \definition[个]{s.}{idade; era; tempos; época; períodos e fases históricas divididas de acordo com condições econômicas, políticas, culturais e outras | um período na vida de alguém; uma fase na vida de uma pessoa}
\end{EntryWithPhonetic}

\begin{EntryWithPhonetic}{时段}{shi2duan4}{7,9}{⽇,⽎}[HSK 7-9]
  \definition{s.}{um período de tempo; refere"-se a um período de tempo específico}
  \synonymref{阶段}{jie1duan4}
\end{EntryWithPhonetic}

\begin{EntryWithPhonetic}{时而}{shi2'er2}{7,6}{⽇,⽽}[HSK 6]
  \definition{adv.}{às vezes; de tempos em tempos; indica que algo acontece repetidamente em intervalos irregulares}
\end{EntryWithPhonetic}

\begin{EntryWithPhonetic}{时而……,时而……}{shi2'er2 shi2'er2}{7,6,7,6}{⽇,⽽,⽇,⽽}
  \definition{adv.}{agora\dots, agora\dots; às vezes\dots, às vezes\dots; usado antes e depois; indica que diferentes fenômenos ou coisas ocorrem alternadamente ou mudam continuamente dentro de um determinado período de tempo}[\underline{时而}下雨,\underline{时而}晴天。===Às vezes chove, às vezes faz sol. | 这个地方\underline{时而}热,\underline{时而}冷。===Este lugar às vezes é quente e às vezes frio.]
\end{EntryWithPhonetic}

\begin{EntryWithPhonetic}{时隔}{shi2ge2}{7,12}{⽇,⾩}[HSK 7-9]
  \definition{s.}{após um intervalo; separado no tempo; intervalo de tempo: um período de tempo fixo e bem definido}
\end{EntryWithPhonetic}

\begin{EntryWithPhonetic}{时光}{shi2guang1}{7,6}{⽇,⼉}[HSK 5]
  \definition[台]{s.}{tempo; passagem do tempo | dias; horas; anos; épocas; períodos}
\end{EntryWithPhonetic}

\begin{EntryWithPhonetic}{时好时坏}{shi2hao3-shi2huai4}{7,6,7,7}{⽇,⼥,⽇,⼟}[HSK 7-9]
  \definition{expr.}{``Às vezes bom, às vezes ruim.''}
\end{EntryWithPhonetic}

\begin{EntryWithPhonetic}{时候}{shi2hou5}{7,10}{⽇,⼈}[HSK 1]
  \definition[个]{s.}{(um ponto no) tempo; momento; um determinado momento no tempo | (a duração do) tempo; um período de tempo com início e fim}
\end{EntryWithPhonetic}

\begin{EntryWithPhonetic}{时机}{shi2ji1}{7,6}{⽇,⽊}[HSK 5]
  \definition[个]{s.}{oportunidade; momento oportuno}
\end{EntryWithPhonetic}

\begin{EntryWithPhonetic}{时间}{shi2jian1}{7,7}{⽇,⾨}[HSK 1]
  \definition[段]{s.}{tempo; refere"-se à forma de existência do movimento da matéria, um sistema contínuo composto pelo passado, presente e futuro | tempo; período (duração); um período de tempo com início e fim | tempo (um ponto); em algum momento do tempo}
\end{EntryWithPhonetic}

\begin{EntryWithPhonetic}{时间表}{shi2jian1biao3}{7,7,8}{⽇,⾨,⾐}[HSK 7-9]
  \definition{s.}{um cronograma; uma planilha de horários}
\end{EntryWithPhonetic}

\begin{EntryWithPhonetic}{时节}{shi2jie2}{7,5}{⽇,⾋}[HSK 6]
  \definition{s.}{temporada; um período de tempo em um ano com certas características, geralmente relacionadas à estação ou ao termo solar | época; tempo}
\end{EntryWithPhonetic}

\begin{EntryWithPhonetic}{时刻}{shi2ke4}{7,8}{⽇,⼑}[HSK 3]
  \definition{adv.}{constantemente; sempre; a cada momento; frequentemente}
  \definition[个,段]{s.}{tempo; hora; momento; conjuntura; um ponto no tempo}
\end{EntryWithPhonetic}

\begin{EntryWithPhonetic}{时空}{shi2kong1}{7,8}{⽇,⽳}[HSK 7-9]
  \definition{s.}{tempo e espaço; espaço"-tempo; relações espaciais e temporais}
  \synonymref{穿越}{chuan1yue4}
  \synonymref{空间}{kong1jian1}
\end{EntryWithPhonetic}

\begin{EntryWithPhonetic}{时髦}{shi2mao2}{7,14}{⽇,⾽}[HSK 7-9]
  \definition{adj.}{elegante; na moda; em voga; novíssimo, atualmente em alta}
  \synonymref{标志}{biao1zhi4}
  \synonymref{大方}{da4fang5}
  \synonymref{流行}{liu2xing2}
  \synonymref{美丽}{mei3li4}
  \synonymref{文雅}{wen2ya3}
  \antonymref{古朴}{gu3pu3}
  \antonymref{过时}{guo4 shi2}
  \antonymref{过期}{guo4/qi1}
  \antonymref{落后}{luo4/hou4}
\end{EntryWithPhonetic}

\begin{EntryWithPhonetic}{时期}{shi2qi1}{7,12}{⽇,⽉}[HSK 6]
  \definition[个,段]{s.}{um período específico; um período de tempo com uma certa característica}
\end{EntryWithPhonetic}

\begin{EntryWithPhonetic}{时尚}{shi2shang4}{7,8}{⽇,⼩}[HSK 7-9]
  \definition{adj.}{na moda; em voga}
  \definition[种,股]{s.}{moda; tendência; estilo; um estilo de vida e hábito popular de um período de tempo}
  \antonymref{传统}{chuan2tong3}
  \antonymref{风俗}{feng1su2}
\end{EntryWithPhonetic}

\begin{EntryWithPhonetic}{时时}{shi2shi2}{7,7}{⽇,⽇}[HSK 6]
  \definition{adv.}{frequentemente; sempre; constantemente; indica que algo acontece várias vezes dentro de um determinado período de tempo}
\end{EntryWithPhonetic}

\begin{EntryWithPhonetic}{时事}{shi2shi4}{7,8}{⽇,⼅}[HSK 5]
  \definition{s.}{acontecimentos atuais; assuntos atuais; eventos atuais | tendências atuais | como as coisas estão indo | a situação atual}
\end{EntryWithPhonetic}

\begin{EntryWithPhonetic}{时速}{shi2su4}{7,10}{⽇,⾡}[HSK 7-9]
  \definition{s.}{velocidade (por hora)}
\end{EntryWithPhonetic}

\begin{EntryWithPhonetic}{时装}{shi2zhuang1}{7,12}{⽇,⾐}[HSK 6]
  \definition{s.}{vestido da moda; a última moda; os últimos estilos de roupas | roupas contemporâneas}
  \antonymref{古装}{gu3 zhuang1}
\end{EntryWithPhonetic}

%%%%%%%%%% 识 %%%%%%%%%%
\subsection*{识}\addcontentsline{loh}{figure}{识 \dpy{shi2}}

\begin{EntryWithPhonetic}{识}{shi2}{7}{⾔}[HSK 6]
  \definition{s.}{percepção; conhecimento}
  \definition{v.}{saber; reconhecer | saber; entender}
  \seeref{zhi4}
\end{EntryWithPhonetic}

\begin{EntryWithPhonetic}{识别}{shi2bie2}{7,7}{⾔,⼑}[HSK 7-9]
  \definition{v.}{identificar; discernir; escolher; distinguir}
  \synonymref{辨别}{bian4bie2}
  \synonymref{辨认}{bian4ren4}
  \synonymref{鉴别}{jian4bie2}
  \synonymref{区别}{qu1bie2}
\end{EntryWithPhonetic}

\begin{EntryWithPhonetic}{识字}{shi2 zi4}{7,6}{⾔,⼦}[HSK 6]
  \definition{v.}{aprender a ler; tornar"-se alfabetizado; reconhecer caracteres}
\end{EntryWithPhonetic}

%%%%%%%%%% 实 %%%%%%%%%%
\subsection*{实}\addcontentsline{loh}{figure}{实 \dpy{shi2}}

\begin{EntryWithPhonetic}{实}{shi2}{8}{⼧}
  \definition{adj.}{sólido; cheio por dentro; sem espaços vazios | verdadeiro; real; atual; sincero | forte; eficaz; concreto; real}
  \definition{adv.}{verdadeiramente; realmente; de fato; originalmente}
  \definition{s.}{fato; realidade | semente; fruto}
  \definition{v.}{preencher}
  \antonymref{虚}{xu1}
\end{EntryWithPhonetic}

\begin{EntryWithPhonetic}{实地}{shi2di4}{8,6}{⼧,⼟}[HSK 7-9]
  \definition{adv.}{de fato; realmente; realmente muito sério}
  \definition{s.}{campo; no local}
\end{EntryWithPhonetic}

\begin{EntryWithPhonetic}{实话}{shi2hua4}{8,8}{⼧,⾔}[HSK 7-9]
  \definition[句]{s.}{verdade}
  \antonymref{谎话}{huang3hua4}
  \antonymref{谎言}{huang3yan2}
\end{EntryWithPhonetic}

\begin{EntryWithPhonetic}{实话实说}{shi2hua4-shi2shuo1}{8,8,8,9}{⼧,⾔,⼧,⾔}[HSK 7-9]
  \definition{expr.}{``Para dizer a verdade.''; falar francamente; falar sem rodeios; dizer as coisas como elas são; dizer a verdade}
\end{EntryWithPhonetic}

\begin{EntryWithPhonetic}{实惠}{shi2hui4}{8,12}{⼧,⼼}[HSK 5]
  \definition{adj.}{sólido; substancial; benefícios práticos}
  \definition{s.}{benefício material; benefícios tangíveis; benefícios reais}
\end{EntryWithPhonetic}

\begin{EntryWithPhonetic}{实际}{shi2ji4}{8,7}{⼧,⾩}[HSK 2]
  \definition{adj.}{real; efetivo; concreto; prático | factual; prático; realista; de acordo com os fatos}
  \definition{s.}{realidade; prática; coisas e situações que existem objetivamente}
\end{EntryWithPhonetic}

\begin{EntryWithPhonetic}{实际上}{shi2ji4shang5}{8,7,3}{⼧,⾩,⼀}[HSK 3]
  \definition{adv.}{de fato; na verdade}
\end{EntryWithPhonetic}

\begin{EntryWithPhonetic}{实践}{shi2jian4}{8,12}{⼧,⾜}[HSK 6]
  \definition{s.}{prática; filosoficamente, refere"-se às ações conscientes das pessoas para transformar a natureza e a sociedade; as atividades de produção são as atividades práticas mais básicas e também incluem atividades políticas, experimentos científicos, educação cultural, etc.}
  \definition{v.}{praticar; realizar; implementar planos e intenções em ações específicas}
\end{EntryWithPhonetic}

\begin{EntryWithPhonetic}{实况}{shi2kuang4}{8,7}{⼧,⼎}[HSK 7-9]
  \definition{s.}{o que está realmente acontecendo; evento atual; ao vivo | ao vivo (ex.: transmissão ou gravação) | cena | a situação real}
  \synonymref{实际}{shi2ji4}
\end{EntryWithPhonetic}

\begin{EntryWithPhonetic}{实力}{shi2li4}{8,2}{⼧,⼒}[HSK 3]
  \definition{s.}{força real; geralmente se refere à força militar e econômica de um país, grupo ou indivíduo, e também se refere à capacidade de um indivíduo ou grupo em uma competição}
\end{EntryWithPhonetic}

\begin{EntryWithPhonetic}{实施}{shi2shi1}{8,9}{⼧,⽅}[HSK 4]
  \definition{v.}{colocar em vigor; implementar (leis, políticas, etc.); executar; trazer (colocar) algo em vigor; fazer cumprir; colocar algo em (prática)}
\end{EntryWithPhonetic}

\begin{EntryWithPhonetic}{实事求是}{shi2shi4-qiu2shi4}{8,8,7,9}{⼧,⼅,⽔,⽇}[HSK 7-9]
  \definition{expr.}{``Buscando a verdade nos fatos.''; ser prático e realista; devemos partir da situação real, sem exagerar nem minimizar, e abordar e lidar com os problemas de forma correta}
  \synonymref{恰如其分}{qia4ru2-qi2fen4}
  \antonymref{弄虚作假}{nong4xu1-zuo4jia3}
  \antonymref{有名无实}{you3ming2wu2shi2}
\end{EntryWithPhonetic}

\begin{EntryWithPhonetic}{实体}{shi2ti3}{8,7}{⼧,⼈}[HSK 7-9]
  \definition[种,个]{s.}{substância; um conceito da filosofia pré-marxista, que sustenta que a substância é o fundamento e a origem imutáveis de todas as coisas; o ``espírito'' dos idealistas e a ``matéria'' dos materialistas metafísicos são ambos tais substâncias | entidade; geralmente se refere a coisas objetivas que existem independentemente e possuem certos atributos}
\end{EntryWithPhonetic}

\begin{EntryWithPhonetic}{实物}{shi2wu4}{8,8}{⼧,⽜}[HSK 7-9]
  \definition[件]{s.}{entidade; matéria; coisas reais e concretas | objeto material; objeto físico; aplicações práticas}
\end{EntryWithPhonetic}

\begin{EntryWithPhonetic}{实习}{shi2xi2}{8,3}{⼧,⼄}[HSK 2]
  \definition{s.}{estagiário; prática; estágio}
  \definition{v.}{aplicar e testar os conhecimentos teóricos aprendidos no trabalho prático, a fim de exercitar a capacidade profissional}
\end{EntryWithPhonetic}

\begin{EntryWithPhonetic}{实现}{shi2xian4}{8,8}{⼧,⾒}[HSK 2]
  \definition{v.}{alcançar; atingir; realizar; concretizar; tornar (ideais, planos, etc.) realidade}
\end{EntryWithPhonetic}

\begin{EntryWithPhonetic}{实行}{shi2xing2}{8,6}{⼧,⾏}[HSK 3]
  \definition{v.}{praticar; implementar; executar; colocar em prática; realizar (programa, política, plano, etc.) por meio de ação}
\end{EntryWithPhonetic}

\begin{EntryWithPhonetic}{实验}{shi2yan4}{8,10}{⼧,⾺}[HSK 3]
  \definition[个,次]{s.}{teste; experimento; trabalho de laboratório}
  \definition{v.}{testar; experimentar; realizar uma operação ou se envolver em uma atividade para testar uma teoria ou hipótese científica}
\end{EntryWithPhonetic}

\begin{EntryWithPhonetic}{实验室}{shi2yan4shi4}{8,10,9}{⼧,⾺,⼧}[HSK 3]
  \definition[个,间]{s.}{laboratório; salas especiais para experimentos científicos}
\end{EntryWithPhonetic}

\begin{EntryWithPhonetic}{实用}{shi2yong4}{8,5}{⼧,⽤}[HSK 4]
  \definition{adj.}{prático; pragmático; funcional; atende aos requisitos reais da aplicação}
  \definition{v.}{colocar em uso prático}
\end{EntryWithPhonetic}

\begin{EntryWithPhonetic}{实在}{shi2zai5}{8,6}{⼧,⼟}[HSK 2]
  \definition{adj.}{honesto; sincero | verdadeiro; honesto; realista; não é falso, não é enganador}
  \definition{adv.}{verdadeiramente; de fato; na verdade; usado para reforçar o tom afirmativo, enfatizando que a situação é realmente assim}
\end{EntryWithPhonetic}

\begin{EntryWithPhonetic}{实质}{shi2zhi4}{8,8}{⼧,⾙}[HSK 7-9]
  \definition{s.}{essência; substância; natureza}
  \synonymref{本色}{ben3se4}
  \synonymref{本质}{ben3zhi4}
  \synonymref{内容}{nei4rong2}
  \synonymref{内心}{nei4xin1}
  \synonymref{实际}{shi2ji4}
  \antonymref{表面}{biao3mian4}
  \antonymref{面子}{mian4zi5}
  \antonymref{名义}{ming2yi4}
  \antonymref{现象}{xian4xiang4}
  \antonymref{形式}{xing2shi4}
\end{EntryWithPhonetic}

%%%%%%%%%% 拾 %%%%%%%%%%
\subsection*{拾}\addcontentsline{loh}{figure}{拾 \dpy{shi2}}

\begin{EntryWithPhonetic}{拾}{shi2}{9}{⼿}[HSK 5]
  \definition{num.}{dez (usado no lugar do numeral 十 em cheques, notas bancárias, etc., para evitar erros ou alterações)}
  \definition{v.}{pegar (do chão); recolher}
  \seeref{she4}
  \synonymref{捡}{jian3}
  \antonymref{丢}{diu1}
\end{EntryWithPhonetic}

%%%%%%%%%% 食 %%%%%%%%%%
\subsection*{食}\addcontentsline{loh}{figure}{食 \dpy{shi2}}

\begin{EntryWithPhonetic}{食}{shi2}{9}{⾷}[Kangxi 184]
  \definition{adj.}{para cozinhar; comestível}
  \definition{s.}{refeição; comida; o que as pessoas e os animais comem | alimentação; alimento para animais; ração | eclipse solar; eclipse lunar}
  \definition{v.}{comer}
  \seeref{si4}
\end{EntryWithPhonetic}

\begin{EntryWithPhonetic}{食品}{shi2pin3}{9,9}{⾷,⼝}[HSK 3]
  \definition[种]{s.}{comida; gêneros alimentícios; provisões; alimentos vendidos em lojas que passaram por algum processamento}
\end{EntryWithPhonetic}

\begin{EntryWithPhonetic}{食宿}{shi2su4}{9,11}{⾷,⼧}[HSK 7-9]
  \definition{s.}{pensão e alojamento; alimentação e acomodação | alojamento e alimentação}
\end{EntryWithPhonetic}

\begin{EntryWithPhonetic}{食堂}{shi2tang2}{9,11}{⾷,⼟}[HSK 4]
  \definition[个,间]{s.}{cantina; refeitório}
\end{EntryWithPhonetic}

\begin{EntryWithPhonetic}{食物}{shi2wu4}{9,8}{⾷,⽜}[HSK 2]
  \definition[种]{s.}{comida; alimentos; comestíveis}
\end{EntryWithPhonetic}

\begin{EntryWithPhonetic}{食用}{shi2yong4}{9,5}{⾷,⽤}[HSK 7-9]
  \definition{adj.}{comestível; que pode ser usado como alimento.}
  \definition{v.}{usar como alimento; ser comestível}
\end{EntryWithPhonetic}

\begin{EntryWithPhonetic}{食欲}{shi2yu4}{9,11}{⾷,⽋}[HSK 6]
  \definition{adj.}{apetitoso}
  \definition{s.}{apetite; desejo humano de comer}
\end{EntryWithPhonetic}

%%%%%%%%%% 史 %%%%%%%%%%
\subsection*{史}\addcontentsline{loh}{figure}{史 \dpy{shi3}}

\begin{EntryWithPhonetic}{史}{shi3}{5}{⼝}
  \definition{s.}{história (como disciplina escolar) | história | historiador oficial; cronista real | pré-histórico}
\end{EntryWithPhonetic}

\begin{EntryWithPhonetic}{史无前例}{shi3wu2qian2li4}{5,4,9,8}{⼝,⽆,⼑,⼈}[HSK 7-9]
  \definition{expr.}{``Sem precedentes.''; não houve paralelo na história; ser sem precedentes na história da nação; sem precedentes na história; sem paralelo na história}
  \synonymref{开天辟地}{kai1tian1-pi4di4}
  \synonymref{前所未有}{qian2suo3wei4you3}
\end{EntryWithPhonetic}

%%%%%%%%%% 使 %%%%%%%%%%
\subsection*{使}\addcontentsline{loh}{figure}{使 \dpy{shi3}}

\begin{EntryWithPhonetic}{使}{shi3}{8}{⼈}[HSK 3]
  \definition{conj.}{se; supondo; usado como a primeira cláusula de uma frase complexa; indica uma relação hipotética; equivalente a 假如}
  \definition{s.}{enviado; mensageiro; pessoas em uma missão}
  \definition{v.}{enviar; despachar; dizer a alguém para fazer algo | usar; empregar; aplicar | deixar; chamar; habilitar}
  \seealsoref{假如}{jia3ru2}
\end{EntryWithPhonetic}

\begin{EntryWithPhonetic}{使得}{shi3de5}{8,11}{⼈,⼻}[HSK 5]
  \definition{v.}{ser utilizável; poder ser usado | ser viável; ser exequível; ser possível;  poder fazer | fazer; tornar; causar um determinado resultado (intenção, plano, coisa)}
\end{EntryWithPhonetic}

\begin{EntryWithPhonetic}{使唤}{shi3huan5}{8,10}{⼈,⼝}[HSK 7-9]
  \definition{v.}{ordenar sobre; pedir a alguém para fazer coisas por você | Coloquial: usar (ferramentas, animais, etc.) ; manusear}
  \synonymref{吩咐}{fen1fu4}
  \antonymref{听从}{ting1cong2}
\end{EntryWithPhonetic}

\begin{EntryWithPhonetic}{使劲}{shi3/jin4}{8,7}{⼈,⼒}[HSK 4]
  \definition{v.+compl.}{colocar energia; exercer toda a sua força | esforçar"-se para ajudar; colocar energia para ajudar}
\end{EntryWithPhonetic}

\begin{EntryWithPhonetic}{使命}{shi3ming4}{8,8}{⼈,⼝}[HSK 7-9]
  \definition{s.}{dever; missão; propósito (uma grande responsabilidade ou dever que uma pessoa assume); uma metáfora para a grande responsabilidade que se carrega | missão (uma tarefa ou dever atribuído a alguém que deve ser cumprido); as ordens recebidas pelo instigador}
  \synonymref{任务}{ren4wu5}
\end{EntryWithPhonetic}

\begin{EntryWithPhonetic}{使用}{shi3yong4}{8,5}{⼈,⽤}[HSK 2]
  \definition{v.}{usar; empregar; aplicar; fazer com que pessoas, equipamentos, fundos, etc. sirvam a um determinado propósito}
\end{EntryWithPhonetic}

\begin{EntryWithPhonetic}{使者}{shi3zhe3}{8,8}{⼈,⽼}[HSK 7-9]
  \definition[名,位]{s.}{enviado; mensageiro; emissário; uma pessoa que está cumprindo uma missão (atualmente, geralmente se referindo a pessoal diplomático)}
  \synonymref{大使}{da4shi3}
\end{EntryWithPhonetic}

%%%%%%%%%% 始 %%%%%%%%%%
\subsection*{始}\addcontentsline{loh}{figure}{始 \dpy{shi3}}

\begin{EntryWithPhonetic}{始}{shi3}{8}{⼥}
  \definition*{s.}{Sobrenome: Shi}
  \definition{adv.}{somente então; não\dots até}
  \definition{s.}{começo; início}
  \definition{v.}{começar; iniciar}
\end{EntryWithPhonetic}

\begin{EntryWithPhonetic}{始终}{shi3zhong1}{8,8}{⼥,⽷}[HSK 3]
  \definition{adv.}{sempre; o tempo todo; durante todo; do começo ao fim; indica continuidade do início ao fim}
  \definition{s.}{todo o processo do começo ao fim}
\end{EntryWithPhonetic}

%%%%%%%%%% 屎 %%%%%%%%%%
\subsection*{屎}\addcontentsline{loh}{figure}{屎 \dpy{shi3}}

\begin{EntryWithPhonetic}{屎}{shi3}{9}{⼫}
  \definition{s.}{fezes | excrementos | (forma ligada) secreção (do ouvido, olho, etc.)}
\end{EntryWithPhonetic}

%%%%%%%%%% 士 %%%%%%%%%%
\subsection*{士}\addcontentsline{loh}{figure}{士 \dpy{shi4}}

\begin{EntryWithPhonetic}{士}{shi4}{3}{⼠}[Kangxi 33]
  \definition*{s.}{Sobrenome: Shi}
  \definition[位,名,个]{s.}{soldado; militar | oficial não comissionado; primeira classe de soldados | pessoa treinada em uma determinada área; algum tipo de técnico | pessoa (louvável) | bacharel (na China antiga) | classe social, entre os oficiais, 大夫, e o povo comum, 庶民 | estudioso | guarda-costas, uma das peças do xadrez chinês}
  \seealsoref{大夫}{da4fu1}
  \seealsoref{庶民}{shu4min2}
\end{EntryWithPhonetic}

\begin{EntryWithPhonetic}{士兵}{shi4bing1}{3,7}{⼠,⼋}[HSK 4]
  \definition[个,名,位,批,群]{s.}{soldado; militar; termo coletivo para oficiais não comissionados e soldados; os membros mais jovens do exército}
\end{EntryWithPhonetic}

\begin{EntryWithPhonetic}{士气}{shi4qi4}{3,4}{⼠,⽓}[HSK 7-9]
  \definition{s.}{moral; o espírito de luta do exército e, de forma mais ampla, o espírito de luta das massas}
  \synonymref{斗志}{dou4zhi4}
  \synonymref{气魄}{qi4po4}
  \synonymref{气势}{qi4shi4}
  \synonymref{勇气}{yong3qi4}
\end{EntryWithPhonetic}

%%%%%%%%%% 世 %%%%%%%%%%
\subsection*{世}\addcontentsline{loh}{figure}{世 \dpy{shi4}}

\begin{EntryWithPhonetic}{世}{shi4}{5}{⼀}
  \definition*{s.}{Sobrenome: Shi}
  \definition{s.}{vida; tempo de vida; vida humana | geração; geração após geração | idade; era | o mundo; sociedade | Geologia: época, abaixo de período}
\end{EntryWithPhonetic}

\begin{EntryWithPhonetic}{世代}{shi4dai4}{5,5}{⼀,⼈}[HSK 7-9]
  \definition{s.}{anos; idades | por gerações; de uma geração para a seguinte; geração após geração | de geração em geração | uma época ou era | geração}
\end{EntryWithPhonetic}

\begin{EntryWithPhonetic}{世故}{shi4gu4}{5,9}{⼀,⽁}
  \definition{s.}{os caminhos do mundo; a arte de lidar com as pessoas; experiência de vida}
  \seeref{shi4gu5}
\end{EntryWithPhonetic}

\begin{EntryWithPhonetic}{世故}{shi4gu5}{5,9}{⼀,⽁}[HSK 7-9]
  \definition{adj.}{sofisticado; experiente; (ao lidar com pessoas) ser diplomático e evitar ofender alguém}
  \seeref{shi4gu4}
\end{EntryWithPhonetic}

\begin{EntryWithPhonetic}{世纪}{shi4ji4}{5,6}{⼀,⽷}[HSK 3]
  \definition[个,段]{s.}{século; uma unidade para calcular anos, cem anos é um século}
  \synonymref{时代}{shi2dai4}
\end{EntryWithPhonetic}

\begin{EntryWithPhonetic}{世界}{shi4jie4}{5,9}{⼀,⽥}[HSK 3]
  \definition[个,片,种]{s.}{mundo; todos os lugares da Terra | a soma da natureza e da sociedade humana; refere"-se à soma de toda a existência objetiva na natureza e na sociedade humana | campo; refere"-se a uma determinada área ou campo | o universo sem limites; costumava ser um termo budista, mas agora também se refere ao mundo natural ilimitado e à sociedade humana | situação social; a situação ou atmosfera social de um determinado período}
  \synonymref{地球}{di4qiu2}
  \synonymref{全国}{quan2guo2}
  \synonymref{天下}{tian1xia4}
  \synonymref{宇宙}{yu3zhou4}
\end{EntryWithPhonetic}

\begin{EntryWithPhonetic}{世界杯}{shi4jie4bei1}{5,9,8}{⼀,⽥,⽊}[HSK 3]
  \definition*{s.}{Copa do Mundo; Troféu da Copa do Mundo}
\end{EntryWithPhonetic}

\begin{EntryWithPhonetic}{世界级}{shi4jie4 ji2}{5,9,6}{⼀,⽥,⽷}[HSK 7-9]
  \definition{adj.}{de classe mundial}
  \definition{s.}{classe mundial}
\end{EntryWithPhonetic}

\begin{EntryWithPhonetic}{世锦赛}{shi4jin3sai4}{5,13,14}{⼀,⾦,⾙}
  \definition*{s.}{Campeonato Mundial}
\end{EntryWithPhonetic}

\begin{EntryWithPhonetic}{世袭}{shi4xi2}{5,11}{⼀,⾐}[HSK 7-9]
  \definition{v.}{obter por herança | herdar; isso se refere à sucessão hereditária de tronos imperiais, títulos de nobreza, etc.}
  \antonymref{选举}{xuan3ju3}
\end{EntryWithPhonetic}

%%%%%%%%%% 市 %%%%%%%%%%
\subsection*{市}\addcontentsline{loh}{figure}{市 \dpy{shi4}}

\begin{EntryWithPhonetic}{市}{shi4}{5}{⼱}[HSK 2]
  \definition{s.}{mercado; lugar onde se concentra o comércio | cidade; município; áreas densamente povoadas, com indústrias, comércio e cultura desenvolvidos | relativo ao sistema tradicional chinês de pesos e medidas; unidades administrativas, incluindo cidades sob jurisdição direta e cidades sob jurisdição provincial (ou autônoma) | unidade padrão de mercado; pertencente ao sistema municipal (unidades de medida) | preço de transação no mercado}
  \definition{v.}{comprar ou vender; fazer transações}
\end{EntryWithPhonetic}

\begin{EntryWithPhonetic}{市场}{shi4chang3}{5,6}{⼱,⼟}[HSK 3]
  \definition[家]{s.}{mercado (também no abstrato); um lugar fixo onde as pessoas compram e vendem coisas juntas | área de \emph{marketing}; região onde o produto é vendido | âmbito de influência (figurado); uma metáfora para o escopo e o grau em que uma determinada ideia ou comportamento é aceito por outros}
\end{EntryWithPhonetic}

\begin{EntryWithPhonetic}{市场经济}{shi4chang3 jing1ji4}{5,6,8,9}{⼱,⼟,⽷,⽔}[HSK 7-9]
  \definition{expr.}{``Economia de mercado.''; economia orientada para o mercado}
\end{EntryWithPhonetic}

\begin{EntryWithPhonetic}{市尺}{shi4 chi3}{5,4}{⼱,⼫}
  \definition{clas.}{chi, uma unidade tradicional de comprimento, equivalente a 0,333 metros ou 1,094 pés}
\end{EntryWithPhonetic}

\begin{EntryWithPhonetic}{市斤}{shi4jin1}{5,4}{⼱,⽄}
  \definition{clas.}{jin, uma unidade tradicional de peso, cada uma contendo 10 liang (市两)  e equivalente a 0,5 quilogramas ou 1,102 libras}
  \seealsoref{市两}{shi4liang3}
\end{EntryWithPhonetic}

\begin{EntryWithPhonetic}{市两}{shi4liang3}{5,7}{⼱,⼀}
  \definition{clas.}{liang, uma unidade tradicional de peso, igual a 0,1 jin (市斤), e equivalente a 50 gramas ou 1,763 onças}
  \seealsoref{市斤}{shi4jin1}
\end{EntryWithPhonetic}

\begin{EntryWithPhonetic}{市民}{shi4min2}{5,5}{⼱,⽒}[HSK 6]
  \definition[位,名]{s.}{habitantes da cidade; residente da cidade; moradores da cidade | cidadão; refere"-se especificamente aos artesãos e comerciantes de pequeno e médio porte nas cidades da sociedade feudal tardia}
\end{EntryWithPhonetic}

\begin{EntryWithPhonetic}{市亩}{shi4mu3}{5,7}{⼱,⼇}
  \definition{clas.}{mu, uma unidade tradicional de área, igual a 60 zhang quadrados (平方市丈) e equivalente a 6,667 ares ou 0,165 acre}
  \seealsoref{平方市丈}{ping2fang1 shi4 zhang4}
\end{EntryWithPhonetic}

\begin{EntryWithPhonetic}{市区}{shi4qu1}{5,4}{⼱,⼖}[HSK 4]
  \definition[个]{s.}{\emph{downtown}; centro da cidade; distrito urbano; áreas que ficam dentro dos limites da cidade e geralmente têm uma alta concentração de população e estoque de moradias}
\end{EntryWithPhonetic}

\begin{EntryWithPhonetic}{市升}{shi4sheng1}{5,4}{⼱,⼗}
  \definition{clas.}{sheng; uma unidade tradicional de volume, equivalente a 1 litro ou 1,76 \emph{pints} ou 0,22 galão}
\end{EntryWithPhonetic}

\begin{EntryWithPhonetic}{市长}{shi4zhang3}{5,4}{⼱,⾧}[HSK 2]
  \definition[个,位,名]{s.}{prefeito; chefe administrativo responsável pela administração de uma cidade}
\end{EntryWithPhonetic}

\begin{EntryWithPhonetic}{市中心}{shi4zhong1xin1}{5,4,4}{⼱,⼁,⼼}
  \definition{s.}{centro da cidade}
\end{EntryWithPhonetic}

%%%%%%%%%% 示 %%%%%%%%%%
\subsection*{示}\addcontentsline{loh}{figure}{示 \dpy{shi4}}

\begin{EntryWithPhonetic}{示}{shi4}{5}{⽰}[Kangxi 113]
  \definition*{s.}{Sobrenome: Shi}
  \definition{s.}{(sua) carta  | missiva; instruções; palavras ou escritos para subordinados ou gerações mais jovens}
  \definition{v.}{mostrar; notificar; instruir | indicar; significar; mostrar ou apontar, fazer conhecido}
\end{EntryWithPhonetic}

\begin{EntryWithPhonetic}{示范}{shi4fan4}{5,9}{⽰,⾋}[HSK 5]
  \definition{v.}{demonstrar; dar o exemplo; criar um modelo que todos possam aprender}
\end{EntryWithPhonetic}

\begin{EntryWithPhonetic}{示威}{shi4wei1}{5,9}{⽰,⼥}[HSK 7-9]
  \definition{v.}{demonstrar; realizar uma demonstração; uma atividade coletiva realizada para expressar uma insatisfação específica ou para manifestar reivindicações não atendidas | fazer uma demonstração de força; exibir a própria força; exibir a própria força para intimidar os outros}
  \synonymref{游行}{you2xing2}
\end{EntryWithPhonetic}

\begin{EntryWithPhonetic}{示意}{shi4yi4}{5,13}{⽰,⼼}[HSK 7-9]
  \definition{v.}{insinuar; sinalizar; dar um sinal; (utilizando expressões faciais, gestos, palavras-código, gráficos, etc.) para expressar um determinado significado}
  \synonymref{暗示}{an4shi4}
  \synonymref{表示}{biao3shi4}
  \synonymref{提醒}{ti2/xing3}
\end{EntryWithPhonetic}

%%%%%%%%%% 似 %%%%%%%%%%
\subsection*{似}\addcontentsline{loh}{figure}{似 \dpy{shi4}}

\begin{EntryWithPhonetic}{似}{shi4}{6}{⼈}
  \definition{v.}{ver; parecer}
  \seeref{si4}
  \synonymref{类}{lei4}
  \synonymref{如}{ru2}
  \synonymref{像}{xiang4}
\end{EntryWithPhonetic}

\begin{EntryWithPhonetic}{似的}{shi4de5}{6,8}{⼈,⽩}[HSK 4]
  \definition{part.}{como; como\dots como; como se (embora); usada após uma palavra ou frase para indicar uma semelhança com algo ou uma situação | usada para indicar alto grau}
\end{EntryWithPhonetic}

%%%%%%%%%% 式 %%%%%%%%%%
\subsection*{式}\addcontentsline{loh}{figure}{式 \dpy{shi4}}

\begin{EntryWithPhonetic}{式}{shi4}{6}{⼷}[HSK 5]
  \definition*{s.}{Sobrenome: Shi}
  \definition{s.}{tipo; estilo | forma; padrão | ritual; cerimônia | fórmula; conjunto de símbolos que expressam uma lei natural na ciência natural | humor; modo; categoria gramatical que expressa a atitude subjetiva do falante em relação ao que está sendo dito, como narrativa, imperativa e condicional}
\end{EntryWithPhonetic}

%%%%%%%%%% 事 %%%%%%%%%%
\subsection*{事}\addcontentsline{loh}{figure}{事 \dpy{shi4}}

\begin{EntryWithPhonetic}{事}{shi4}{8}{⼅}[HSK 1]
  \definition[件,桩,回]{s.}{assunto; questão; coisa; negócio | problema; acidente | emprego; trabalho | responsabilidade; envolvimento | caso, coisa; o que aconteceu}
  \definition{v.}{servir; atender | estar envolvido em; dedicar"-se a}
\end{EntryWithPhonetic}

\begin{EntryWithPhonetic}{事故}{shi4gu4}{8,9}{⼅,⽁}[HSK 3]
  \definition[起,桩,次,场]{s.}{acidente; perdas ou desastres repentinos, muitas vezes relacionados ao transporte, produção, trabalho e segurança pessoal}
  \synonymref{故障}{gu4zhang4}
  \synonymref{意外}{yi4wai4}
\end{EntryWithPhonetic}

\begin{EntryWithPhonetic}{事后}{shi4hou4}{8,6}{⼅,⼝}[HSK 6]
  \definition{s.}{depois; depois do evento; após o incidente ocorrer ou o problema ser resolvido}
  \synonymref{过后}{guo4hou4}
  \synonymref{之后}{zhi1hou4}
  \antonymref{当时}{dang1shi2}
  \antonymref{当时}{dang4shi2}
  \antonymref{事先}{shi4xian1}
\end{EntryWithPhonetic}

\begin{EntryWithPhonetic}{事迹}{shi4ji4}{8,9}{⼅,⾡}[HSK 7-9]
  \definition{s.}{feito; conquista; coisas importantes que um indivíduo ou grupo fez no passado}
  \synonymref{古迹}{gu3ji4}
  \synonymref{奇迹}{qi2ji4}
  \synonymref{事业}{shi4ye4}
  \synonymref{遗迹}{yi2ji4}
\end{EntryWithPhonetic}

\begin{EntryWithPhonetic}{事件}{shi4jian4}{8,6}{⼅,⼈}[HSK 3]
  \definition[个,件,次]{s.}{evento; incidente; grandes eventos na história ou na sociedade}
  \synonymref{事故}{shi4gu4}
  \synonymref{事务}{shi4wu4}
  \synonymref{事项}{shi4xiang4}
  \synonymref{事宜}{shi4yi2}
  \synonymref{事情}{shi4qing5}
\end{EntryWithPhonetic}

\begin{EntryWithPhonetic}{事情}{shi4qing5}{8,11}{⼅,⼼}[HSK 2]
  \definition[件,个,些,种]{s.}{assunto; questão; coisa; negócio | erro; acidente; infortúnio | (coloquial) emprego; trabalho}
  \synonymref{工作}{gong1zuo4}
  \synonymref{活儿}{huo2r5}
  \synonymref{事故}{shi4gu4}
  \synonymref{事件}{shi4jian4}
  \synonymref{事务}{shi4wu4}
  \synonymref{事项}{shi4xiang4}
  \synonymref{事宜}{shi4yi2}
\end{EntryWithPhonetic}

\begin{EntryWithPhonetic}{事儿}{shi4r5}{8,2}{⼅,⼉}
  \definition[件,桩]{s.}{o emprego | negócio | afazeres | assunto que precisa ser resolvido | matéria}
\end{EntryWithPhonetic}

\begin{EntryWithPhonetic}{事实}{shi4shi2}{8,8}{⼅,⼧}[HSK 3]
  \definition[个,件]{s.}{mito; lenda; uma narrativa sobre alguém ou algo que foi transmitida oralmente}
  \definition{v.}{dizer; contar; ser dito; contar a história}
  \synonymref{毕竟}{bi4jing4}
  \synonymref{到底}{dao4di3}
  \synonymref{结果}{jie2/guo3}
  \synonymref{究竟}{jiu1jing4}
  \synonymref{客观}{ke4guan1}
  \synonymref{真相}{zhen1xiang4}
  \antonymref{传奇}{chuan2qi2}
  \antonymref{据说}{ju4shuo1}
  \antonymref{梦想}{meng4xiang3}
  \antonymref{神话}{shen2hua4}
  \antonymref{虚伪}{xu1wei3}
\end{EntryWithPhonetic}

\begin{EntryWithPhonetic}{事实上}{shi4shi2shang5}{8,8,3}{⼅,⼧,⼀}[HSK 3]
  \definition{adv.}{realmente; de fato; na realidade; na verdade; de fato}
  \synonymref{实际上}{shi2ji4shang5}
\end{EntryWithPhonetic}

\begin{EntryWithPhonetic}{事态}{shi4tai4}{8,8}{⼅,⼼}[HSK 7-9]
  \definition{s.}{estado de coisas; situação; circunstâncias (na maioria das vezes ruins)}
  \synonymref{大局}{da4ju2}
  \synonymref{局面}{ju2mian4}
  \synonymref{形势}{xing2shi4}
\end{EntryWithPhonetic}

\begin{EntryWithPhonetic}{事务}{shi4wu4}{8,5}{⼅,⼒}[HSK 7-9]
  \definition{s.}{trabalho; rotina; assuntos; refere"-se a um negócio específico | assuntos gerais | casos; tarefas específicas; trabalho diário}
  \synonymref{工作}{gong1zuo4}
  \synonymref{事件}{shi4jian4}
  \synonymref{事宜}{shi4yi2}
  \synonymref{事情}{shi4qing5}
\end{EntryWithPhonetic}

\begin{EntryWithPhonetic}{事务所}{shi4wu4suo3}{8,5,8}{⼅,⼒,⼾}[HSK 7-9]
  \definition{s.}{escritório; firma; empresa}
\end{EntryWithPhonetic}

\begin{EntryWithPhonetic}{事物}{shi4wu4}{8,8}{⼅,⽜}[HSK 4]
  \definition[种,类,个]{s.}{coisa; objeto; todos os objetos e fenômenos que existem objetivamente}
\end{EntryWithPhonetic}

\begin{EntryWithPhonetic}{事先}{shi4xian1}{8,6}{⼅,⼉}[HSK 4]
  \definition{adv.}{antes; de antemão; com antecedência; antecipadamente}
  \synonymref{早就}{zao3jiu4}
  \antonymref{当时}{dang1shi2}
  \antonymref{当时}{dang4shi2}
  \antonymref{事后}{shi4hou4}
\end{EntryWithPhonetic}

\begin{EntryWithPhonetic}{事项}{shi4xiang4}{8,9}{⼅,⾴}[HSK 7-9]
  \definition{s.}{item; matéria; projeto}
  \synonymref{事故}{shi4gu4}
  \synonymref{事件}{shi4jian4}
  \synonymref{事情}{shi4qing5}
\end{EntryWithPhonetic}

\begin{EntryWithPhonetic}{事业}{shi4ye4}{8,5}{⼅,⼀}[HSK 3]
  \definition[个]{s.}{causa; carreira; empreendimento; atividades regulares realizadas por pessoas com um determinado objetivo, escala e sistema que têm impacto no desenvolvimento social | instituição; instalações; unidade de trabalho apoiada financeiramente pelo governo; refere"-se especificamente a empresas que não têm rendimentos de produção, são financiadas pelo Estado e não realizam contabilidade económica}
  \synonymref{工作}{gong1zuo4}
  \synonymref{事迹}{shi4ji4}
  \synonymref{职业}{zhi2ye4}
\end{EntryWithPhonetic}

\begin{EntryWithPhonetic}{事宜}{shi4yi2}{8,8}{⼅,⼧}[HSK 7-9]
  \definition[个,项]{s.}{acordos; assuntos em questão; assuntos ou tarefas que precisam ser organizados ou tratados (frequentemente usado em documentos oficiais, leis, etc.)}
  \synonymref{符合}{fu2he2}
  \synonymref{合适}{he2shi4}
  \synonymref{恰当}{qia4dang4}
  \synonymref{适合}{shi4he2}
  \synonymref{事件}{shi4jian4}
  \synonymref{事务}{shi4wu4}
  \synonymref{适应}{shi4ying4}
  \synonymref{事情}{shi4qing5}
  \synonymref{妥当}{tuo3dang4}
  \synonymref{妥善}{tuo3shan4}
\end{EntryWithPhonetic}

%%%%%%%%%% 侍 %%%%%%%%%%
\subsection*{侍}\addcontentsline{loh}{figure}{侍 \dpy{shi4}}

\begin{EntryWithPhonetic}{侍}{shi4}{8}{⼈}
  \definition{v.}{atender a; servir a; acompanhar e servir}
\end{EntryWithPhonetic}

\begin{EntryWithPhonetic}{侍候}{shi4hou4}{8,10}{⼈,⼈}[HSK 7-9]
  \definition{v.}{atender; cuidar de; prestar assistência | atender a; estar presente em}
  \synonymref{伺候}{ci4hou5}
\end{EntryWithPhonetic}

%%%%%%%%%% 势 %%%%%%%%%%
\subsection*{势}\addcontentsline{loh}{figure}{势 \dpy{shi4}}

\begin{EntryWithPhonetic}{势}{shi4}{8}{⼒}
  \definition{s.}{poder; força; influência | momentum; tendência | aparência externa de um objeto natural; fenômenos ou situações naturais | situação; estado de coisas; circunstâncias | sinal; gesto | genitais masculinos}
\end{EntryWithPhonetic}

\begin{EntryWithPhonetic}{势必}{shi4bi4}{8,5}{⼒,⼼}[HSK 7-9]
  \definition{adv.}{inevitavelmente; certamente irá; estará obrigado a; isso indica que, com base na situação, infere"-se que uma determinada situação ocorrerá inevitavelmente}
  \synonymref{必然}{bi4ran2}
  \antonymref{也许}{ye3xu3}
\end{EntryWithPhonetic}

\begin{EntryWithPhonetic}{势不可当}{shi4bu4ke3dang1}{8,4,5,6}{⼒,⼀,⼝,⼹}[HSK 7-9]
  \definition{expr.}{impossível de resistir (expressão idiomática); uma força irresistível | irresistível; implacável; imparável}
\end{EntryWithPhonetic}

\begin{EntryWithPhonetic}{势力}{shi4li5}{8,2}{⼒,⼒}[HSK 5]
  \definition[股]{s.}{força; poder; influência; forças políticas, econômicas, militares, etc.}
\end{EntryWithPhonetic}

\begin{EntryWithPhonetic}{势头}{shi4tou5}{8,5}{⼒,⼤}[HSK 7-9]
  \definition{s.}{ímpeto; impulso; momento; o estado das coisas; a situação}
\end{EntryWithPhonetic}

%%%%%%%%%% 视 %%%%%%%%%%
\subsection*{视}\addcontentsline{loh}{figure}{视 \dpy{shi4}}

\begin{EntryWithPhonetic}{视}{shi4}{8}{⾒}
  \definition{v.}{olhar para | considerar; olhar para | inspecionar; observar}
\end{EntryWithPhonetic}

\begin{EntryWithPhonetic}{视察}{shi4cha2}{8,14}{⾒,⼧}[HSK 7-9]
  \definition{s.}{visitação; revisão; visita de inspeção; o pessoal superior inspeciona o trabalho das organizações subordinadas}
  \definition{v.}{inspecionar | observar; assistir}
  \synonymref{参观}{can1guan1}
  \synonymref{调查}{diao4cha2}
  \synonymref{调研}{diao4yan2}
  \synonymref{观测}{guan1ce4}
  \synonymref{观察}{guan1cha2}
  \synonymref{考察}{kao3cha2}
  \synonymref{考核}{kao3he2}
\end{EntryWithPhonetic}

\begin{EntryWithPhonetic}{视角}{shi4jiao3}{8,7}{⾒,⾓}[HSK 7-9]
  \definition{s.}{ângulo de visão; ângulo visual; em física, o ângulo de visão é o ângulo entre dois raios de luz que atingem o olho vindos de extremidades opostas de um objeto | abordagem; ponto de vista; perspectiva; ângulo a partir do qual se analisa um problema; um ângulo para observar e examinar problemas | perspectiva}
\end{EntryWithPhonetic}

\begin{EntryWithPhonetic}{视觉}{shi4jue2}{8,9}{⾒,⾒}[HSK 7-9]
  \definition{s.}{sentido da visão; a sensação produzida pelos raios de luz, sejam eles provenientes diretamente de uma fonte de luz ou refletidos por um objeto, ao atuarem sobre a retina}
  \antonymref{触觉}{chu4jue2}
\end{EntryWithPhonetic}

\begin{EntryWithPhonetic}{视力}{shi4li4}{8,2}{⾒,⼒}[HSK 7-9]
  \definition{s.}{visão; vista; a capacidade do olho de distinguir a forma de um objeto a uma certa distância}
  \synonymref{见识}{jian4shi5}
  \synonymref{眼光}{yan3guang1}
  \synonymref{眼睛}{yan3jing5}
\end{EntryWithPhonetic}

\begin{EntryWithPhonetic}{视频}{shi4pin2}{8,13}{⾒,⾴}[HSK 5]
  \definition[个,段,条]{s.}{vídeo; videoclipe}
\end{EntryWithPhonetic}

\begin{EntryWithPhonetic}{视为}{shi4wei2}{8,4}{⾒,⼂}[HSK 5]
  \definition{v.}{considerar; ver como; considerar como; considerar ser; achar que é}
\end{EntryWithPhonetic}

\begin{EntryWithPhonetic}{视线}{shi4xian4}{8,8}{⾒,⽷}[HSK 7-9]
  \definition[道]{s.}{vista; linha de visão; campo de visão (em topografia); a linha reta imaginária entre o olho e o objeto, ou a área do campo de visão ao olhar para um objeto | atenção; direção e objetivo metafóricos}
  \synonymref{视野}{shi4ye3}
\end{EntryWithPhonetic}

\begin{EntryWithPhonetic}{视野}{shi4ye3}{8,11}{⾒,⾥}[HSK 7-9]
  \definition{s.}{campo visual; a extensão do espaço vista pelos olhos | campo de visão; domínio metafórico do pensamento ou do conhecimento}
  \synonymref{视线}{shi4xian4}
\end{EntryWithPhonetic}

%%%%%%%%%% 试 %%%%%%%%%%
\subsection*{试}\addcontentsline{loh}{figure}{试 \dpy{shi4}}

\begin{EntryWithPhonetic}{试}{shi4}{8}{⾔}[HSK 1]
  \definition{s.}{teste; exame; avaliação de conhecimentos ou habilidades através de métodos específicos}
  \definition{v.}{tentar; investigar resultados ou verificar a natureza, não se envolver formalmente (em determinada atividade)}
\end{EntryWithPhonetic}

\begin{EntryWithPhonetic}{试点}{shi4dian3}{8,9}{⾔,⽕}[HSK 6]
  \definition[个]{s.}{local onde um experimento é conduzido; unidade experimental; local de teste; um lugar para pequenos experimentos}
  \definition{v.}{experimentar; fazer experimentos; realizar testes em pontos selecionados; lançar um projeto piloto}
\end{EntryWithPhonetic}

\begin{EntryWithPhonetic}{试卷}{shi4juan4}{8,8}{⾔,⼙}[HSK 4]
  \definition[分,张]{s.}{folha de teste; folha de exame; papel usado para escrever as respostas nos exames}
\end{EntryWithPhonetic}

\begin{EntryWithPhonetic}{试探}{shi4tan4}{8,11}{⾔,⼿}
  \definition{v.}{investigar; explorar; descobrir (uma questão); tentar explorar (um problema ou uma situação)}
  \seeref{shi4tan5}
  \synonymref{尝试}{chang2shi4}
  \synonymref{摸索}{mo1suo3}
  \synonymref{试验}{shi4yan4}
  \synonymref{探索}{tan4suo3}
\end{EntryWithPhonetic}

\begin{EntryWithPhonetic}{试探}{shi4tan5}{8,11}{⾔,⼿}[HSK 7-9]
  \definition{v.}{sondar; testar (a reação de alguém); utilizar palavras ou ações vagas para obter uma resposta da outra parte, compreendendo assim o seu significado}
  \seeref{shi4tan4}
  \synonymref{尝试}{chang2shi4}
  \synonymref{摸索}{mo1suo3}
  \synonymref{试验}{shi4yan4}
  \synonymref{探索}{tan4suo3}
\end{EntryWithPhonetic}

\begin{EntryWithPhonetic}{试题}{shi4ti2}{8,15}{⾔,⾴}[HSK 3]
  \definition[道]{s.}{questões de um exame}
\end{EntryWithPhonetic}

\begin{EntryWithPhonetic}{试图}{shi4tu2}{8,8}{⾔,⼞}[HSK 5]
  \definition{v.}{tentar; pretender, fazer o possível para realizar algo}
\end{EntryWithPhonetic}

\begin{EntryWithPhonetic}{试行}{shi4xing2}{8,6}{⾔,⾏}[HSK 7-9]
  \definition{v.}{experimentar; testar | colocar em uso experimental; realizar uma implementação experimental}
\end{EntryWithPhonetic}

\begin{EntryWithPhonetic}{试验}{shi4yan4}{8,10}{⾔,⾺}[HSK 3]
  \definition{v.}{testar; fazer um teste; fazer um experimento; para examinar o efeito ou desempenho de algo, primeiro experimente em um laboratório ou em uma escala menor}
\end{EntryWithPhonetic}

\begin{EntryWithPhonetic}{试用}{shi4yong4}{8,5}{⾔,⽤}[HSK 7-9]
  \definition{v.}{fazer um teste; estar em período probatório; antes de usar formalmente, experimentar primeiro para ver se atende aos requisitos}
  \synonymref{尝试}{chang2shi4}
\end{EntryWithPhonetic}

\begin{EntryWithPhonetic}{试用期}{shi4yong4qi1}{8,5,12}{⾔,⽤,⽉}[HSK 7-9]
  \definition{s.}{período probatório; período de experiência; a Lei do Contrato de Trabalho estabeleceu disposições específicas para abordar os problemas de abuso dos períodos de experiência e de períodos de experiência excessivamente longos}
\end{EntryWithPhonetic}

%%%%%%%%%% 室 %%%%%%%%%%
\subsection*{室}\addcontentsline{loh}{figure}{室 \dpy{shi4}}

\begin{EntryWithPhonetic}{室}{shi4}{9}{⼧}[HSK 3]
  \definition*{s.}{Shi, a décima terceira das vinte e oito constelações da esfera celeste, composta por duas estrelas em linha reta na constelação de Pégaso | Sobrenome: Shi}
  \definition{s.}{sala; quarto; casa | departamento; sala como unidade administrativa ou de trabalho; órgãos públicos, fábricas, escolas e outras unidades de trabalho internas | esposa; familiares ou esposa | família; clã | cavidade; órgão com forma semelhante a uma câmara}
\end{EntryWithPhonetic}

%%%%%%%%%% 是 %%%%%%%%%%
\subsection*{是}\addcontentsline{loh}{figure}{是 \dpy{shi4}}

\begin{EntryWithPhonetic}{是}{shi4}{9}{⽇}[HSK 1]
  \definition*{s.}{Sobrenome: Shi}
  \definition{adj.}{correto; certo | verdadeiro}
  \definition{adv.}{(expressar afirmação firme) de fato; realmente}
  \definition{pron.}{isso; isto |  todos; qualquer um; usado antes de substantivos, tem o significado de 凡是}
  \definition{s.}{assuntos (importantes); grandes planos}
  \definition{v.}{usado como ``ser'' antes de substantivos ou pronomes para identificar, descrever ou ampliar o sujeito; indica que duas coisas são iguais, ou que a segunda explica a primeira | usado entre duas palavras idênticas; relacionar duas palavras semelhantes |  (usado antes de substantivos) ser exatamente; ser corretamente; usado antes de substantivos, tem o significado de 适合 | elogiar; justificar | expressar afirmação ou concordância (frequentemente usado sozinho) | usado para escolher perguntas, perguntas sim/não ou perguntas retóricas | (usado no início de uma frase) enfatizar uma determinada parte de uma frase | usado em perguntas sim-não}
  \seealsoref{凡是}{fan2shi4}
  \seealsoref{适合}{shi4he2}
\end{EntryWithPhonetic}

\begin{EntryWithPhonetic}{是不是}{shi4bu5shi4}{9,4,9}{⽇,⼀,⽇}[HSK 1]
  \definition{expr.}{sim ou não; é ou não é; se ou não; questões levantadas sobre a confirmação e a negação dos fatos}
\end{EntryWithPhonetic}

\begin{EntryWithPhonetic}{是的}{shi4de5}{9,8}{⽇,⽩}
  \definition{adv.}{sim | está certo}
\end{EntryWithPhonetic}

\begin{EntryWithPhonetic}{是非}{shi4fei1}{9,8}{⽇,⾮}[HSK 7-9]
  \definition{s.}{certo e errado; correto e incorreto | briga; disputa}
  \synonymref{长短}{chang2duan3}
  \synonymref{吵嘴}{chao3/zui3}
  \synonymref{黑白}{hei1bai2}
  \synonymref{利害}{li4hai4}
  \synonymref{辱骂}{ru3ma4}
\end{EntryWithPhonetic}

\begin{EntryWithPhonetic}{是否}{shi4fou3}{9,7}{⽇,⼝}[HSK 4]
  \definition{adv.}{se; se ou não; sim ou não}
\end{EntryWithPhonetic}

%%%%%%%%%% 柿 %%%%%%%%%%
\subsection*{柿}\addcontentsline{loh}{figure}{柿 \dpy{shi4}}

\begin{EntryWithPhonetic}{柿}{shi4}{9}{⽊}
  \definition{s.}{caqui | árvore de caqui}
\end{EntryWithPhonetic}

\begin{EntryWithPhonetic}{柿子}{shi4zi5}{9,3}{⽊,⼦}[HSK 7-9]
  \definition[个]{s.}{caqui | caquizeiro}
\end{EntryWithPhonetic}

%%%%%%%%%% 适 %%%%%%%%%%
\subsection*{适}\addcontentsline{loh}{figure}{适 \dpy{shi4}}

\begin{EntryWithPhonetic}{适}{shi4}{9}{⾡}
  \definition*{s.}{Sobrenome: Shi}
  \definition{adj.}{confortável; bem | adequado; apropriado | certo; oportuno}
  \definition{v.}{ser apto; ser adequado; ser apropriado | ir; seguir; perseguir | (de uma mulher) casar}
\end{EntryWithPhonetic}

\begin{EntryWithPhonetic}{适当}{shi4dang4}{9,6}{⾡,⼹}[HSK 6]
  \definition{s.}{adequado; apropriado}
\end{EntryWithPhonetic}

\begin{EntryWithPhonetic}{适度}{shi4du4}{9,9}{⾡,⼴}[HSK 7-9]
  \definition{adj.}{adequado; moderado; apropriado}
  \synonymref{截至}{jie2zhi4}
  \synonymref{戒指}{jie4zhi5}
  \antonymref{过度}{guo4du4}
  \antonymref{极度}{ji2du4}
\end{EntryWithPhonetic}

\begin{EntryWithPhonetic}{适合}{shi4he2}{9,6}{⾡,⼝}[HSK 3]
  \definition{v.}{servir; caber; se adequar; atender às necessidades de uma determinada situação ou pessoa}
\end{EntryWithPhonetic}

\begin{EntryWithPhonetic}{适量}{shi4liang4}{9,12}{⾡,⾥}[HSK 7-9]
  \definition{adj.}{moderado; quantidade adequada; apropriado; a quantidade é ideal, nem muita, nem pouca}
  \synonymref{适当}{shi4dang4}
  \synonymref{些许}{xie1xu3}
\end{EntryWithPhonetic}

\begin{EntryWithPhonetic}{适时}{shi4shi2}{9,7}{⾡,⽇}[HSK 7-9]
  \definition{adj.}{oportuno; em boa hora; no momento certo; momento apropriado; nem muito cedo, nem muito tarde}
\end{EntryWithPhonetic}

\begin{EntryWithPhonetic}{适宜}{shi4yi2}{9,8}{⾡,⼧}[HSK 7-9]
  \definition{adj.}{adequado; apropriado; conveniente}
  \synonymref{得当}{de2dang4}
  \synonymref{符合}{fu2he2}
  \synonymref{合适}{he2shi4}
  \synonymref{恰当}{qia4dang4}
  \synonymref{适当}{shi4dang4}
  \synonymref{适合}{shi4he2}
  \synonymref{适应}{shi4ying4}
  \synonymref{妥当}{tuo3dang4}
  \synonymref{妥善}{tuo3shan4}
  \antonymref{不适}{bu2shi4}
  \antonymref{不宜}{bu4yi2}
\end{EntryWithPhonetic}

\begin{EntryWithPhonetic}{适应}{shi4ying4}{9,7}{⾡,⼴}[HSK 3]
  \definition{v.}{ajustar"-se; adequar"-se; adaptar"-se; fazer as alterações correspondentes para se adequar à medida que as condições mudam}
\end{EntryWithPhonetic}

\begin{EntryWithPhonetic}{适用}{shi4yong4}{9,5}{⾡,⽤}[HSK 3]
  \definition{adj.}{adequado; aplicável}
\end{EntryWithPhonetic}

%%%%%%%%%% 逝 %%%%%%%%%%
\subsection*{逝}\addcontentsline{loh}{figure}{逝 \dpy{shi4}}

\begin{EntryWithPhonetic}{逝}{shi4}{10}{⾡}
  \definition{v.}{passar | morrer; falecer | decorrer}
\end{EntryWithPhonetic}

\begin{EntryWithPhonetic}{逝世}{shi4shi4}{10,5}{⾡,⼀}[HSK 7-9]
  \definition{v.}{morrer; falecer; deixar este mundo (com uma conotação solene)}
  \synonymref{去世}{qu4shi4}
  \synonymref{丧生}{sang4/sheng1}
  \synonymref{死亡}{si3wang2}
  \synonymref{牺牲}{xi1sheng1}
  \antonymref{诞生}{dan4sheng1}
\end{EntryWithPhonetic}

%%%%%%%%%% 释 %%%%%%%%%%
\subsection*{释}\addcontentsline{loh}{figure}{释 \dpy{shi4}}

\begin{EntryWithPhonetic}{释}{shi4}{12}{⾤}
  \definition*{s.}{Sakyamuni; refere"-se a Siddhartha Gautama, o fundador do budismo; também se refere ao próprio budismo}
  \definition{s.}{budismo}
  \definition{v.}{explicar; elucidar | esclarecer; dissipar; deixar ir; aliviar | soltar; ser aliviado de; aliviar; deixar ir; colocar no chão | libertar; pôr em liberdade}
\end{EntryWithPhonetic}

\begin{EntryWithPhonetic}{释放}{shi4fang4}{12,8}{⾤,⽅}[HSK 7-9]
  \definition{v.}{libertar; pôr em liberdade; restaurar a liberdade pessoal dos detidos | libertar; deixar sair; liberar algo interno, como matéria ou energia; metaforicamente, liberar as emoções internas}
  \synonymref{放走}{fang4zou3}
  \synonymref{排放}{pai2fang4}
  \antonymref{捕捉}{bu3zhuo1}
  \antonymref{逮捕}{dai4bu3}
  \antonymref{拘留}{ju1liu2}
  \antonymref{扣押}{kou4ya1}
  \antonymref{收集}{shou1ji2}
  \antonymref{吸收}{xi1shou1}
\end{EntryWithPhonetic}

%%%%%%%%%% 嗜 %%%%%%%%%%
\subsection*{嗜}\addcontentsline{loh}{figure}{嗜 \dpy{shi4}}

\begin{EntryWithPhonetic}{嗜}{shi4}{13}{⼝}
  \definition{v.}{ter apreço por; ser viciado em}
\end{EntryWithPhonetic}

\begin{EntryWithPhonetic}{嗜好}{shi4hao4}{13,6}{⼝,⼥}[HSK 7-9]
  \definition[个]{s.}{hobby; vício; passatempos peculiares (frequentemente referindo"-se a passatempos indesejáveis)}
  \synonymref{爱好}{ai4hao4}
  \synonymref{喜爱}{xi3'ai4}
  \synonymref{喜欢}{xi3huan5}
\end{EntryWithPhonetic}

%%%%%%%%%% 收 %%%%%%%%%%
\subsection*{收}\addcontentsline{loh}{figure}{收 \dpy{shou1}}

\begin{EntryWithPhonetic}{收}{shou1}{6}{⽁}[HSK 2]
  \definition{expr.}{aos cuidados de (usado na linha de endereço após o nome)}
  \definition{v.}{recolocar; juntar; reunir e juntar coisas espalhadas ou dispersas | recolher; cobrar | ganhar; obter (benefícios econômicos) | colher; recolher; colher ou cortar frutas, legumes, cereais maduros, etc. | aceitar; receber; acolher | controlar; restringir; restringir, controlar os sentimentos ou ações, para voltar ao estado normal | finalizar; parar; concluir; encerrar | prender; deter; colocar sob custódia}
\end{EntryWithPhonetic}

\begin{EntryWithPhonetic}{收藏}{shou1cang2}{6,17}{⽁,⾋}[HSK 6]
  \definition{v.}{coletar; armazenar; consagrar}
\end{EntryWithPhonetic}

\begin{EntryWithPhonetic}{收到}{shou1 dao4}{6,8}{⽁,⼑}[HSK 2]
  \definition{v.}{conseguir; obter; receber; alcançar}
\end{EntryWithPhonetic}

\begin{EntryWithPhonetic}{收费}{shou1 fei4}{6,9}{⽁,⾙}[HSK 3]
  \definition{v.}{cobrar; cobrar taxas}
\end{EntryWithPhonetic}

\begin{EntryWithPhonetic}{收复}{shou1fu4}{6,9}{⽁,⼢}[HSK 7-9]
  \definition{v.}{recuperar; recapturar | retomar; recuperar | recuperar o próprio território}
  \synonymref{复原}{fu4/yuan2}
  \synonymref{复兴}{fu4xing1}
  \synonymref{恢复}{hui1fu4}
\end{EntryWithPhonetic}

\begin{EntryWithPhonetic}{收购}{shou1gou4}{6,8}{⽁,⾙}[HSK 5]
  \definition{v.}{comprar; adquirir; comprar muito em vários lugares | adquirir uma empresa; obter o controle efetivo de uma empresa por meio de dinheiro, transações de ações, etc.}
\end{EntryWithPhonetic}

\begin{EntryWithPhonetic}{收回}{shou1 hui2}{6,6}{⽁,⼞}[HSK 4]
  \definition{v.}{retomar; recuperar; relembrar; recordar; receber de volta o que foi enviado ou emprestado, ou o dinheiro que foi emprestado ou usado | sacar; retirar; recolher; rescindir; cancelar (uma opinião, ordem, etc.)}
\end{EntryWithPhonetic}

\begin{EntryWithPhonetic}{收获}{shou1huo4}{6,10}{⽁,⾋}[HSK 4]
  \definition[次,番,份]{s.}{resultados; ganhos; metaforicamente falando, conhecimento, experiência, etc. obtidos em estudo ou trabalho; os resultados obtidos por meio de trabalho árduo | colheita; colheita de safras}
  \definition{v.}{colher; juntar as colheitas}
\end{EntryWithPhonetic}

\begin{EntryWithPhonetic}{收集}{shou1ji2}{6,12}{⽁,⾫}[HSK 5]
  \definition{v.}{coletar; reunir; recolher}
\end{EntryWithPhonetic}

\begin{EntryWithPhonetic}{收据}{shou1ju4}{6,11}{⽁,⼿}[HSK 7-9]
  \definition[张]{s.}{recibo; quitação; uma declaração escrita entregue à outra parte como prova após o recebimento de dinheiro ou bens}
\end{EntryWithPhonetic}

\begin{EntryWithPhonetic}{收看}{shou1kan4}{6,9}{⽁,⽬}[HSK 3]
  \definition{v.}{assistir (a um programa de TV)}
\end{EntryWithPhonetic}

\begin{EntryWithPhonetic}{收敛}{shou1lian3}{6,11}{⽁,⽁}[HSK 7-9]
  \definition{v.}{enfraquecer; desaparecer; diminuir ou desaparecer (sorriso, luz, etc.) | conter"-se; restringir e controlar (fala e comportamento sem restrições) | adstringir; provocar a contração do corpo ou reduzir a secreção glandular}
  \synonymref{约束}{yue1shu4}
  \antonymref{放肆}{fang4si4}
  \antonymref{放纵}{fang4zong4}
  \antonymref{展开}{zhan3kai1}
\end{EntryWithPhonetic}

\begin{EntryWithPhonetic}{收留}{shou1liu2}{6,10}{⽁,⽥}[HSK 7-9]
  \definition{v.}{acolher alguém; ter alguém sob seus cuidados; oferecer abrigo}
  \synonymref{放走}{fang4zou3}
  \synonymref{收养}{shou1yang3}
  \antonymref{赶走}{gan3zou3}
  \antonymref{抛弃}{pao1qi4}
  \antonymref{驱逐}{qu1zhu2}
\end{EntryWithPhonetic}

\begin{EntryWithPhonetic}{收买}{shou1mai3}{6,6}{⽁,⼄}[HSK 7-9]
  \definition{v.}{comprar; adquirir | comprar; subornar; conquistar corações e mentes}
  \synonymref{拉拢}{la1long3}
  \synonymref{收购}{shou1gou4}
  \antonymref{出卖}{chu1mai4}
  \antonymref{出售}{chu1shou4}
\end{EntryWithPhonetic}

\begin{EntryWithPhonetic}{收取}{shou1qu3}{6,8}{⽁,⼜}[HSK 6]
  \definition{v.}{obter; coletar; receber; aceitar o dinheiro pago pela outra parte}
\end{EntryWithPhonetic}

\begin{EntryWithPhonetic}{收入}{shou1ru4}{6,2}{⽁,⼊}[HSK 2]
  \definition[笔,个]{s.}{renda; salário; dinheiro recebido}
  \definition{v.}{receber dinheiro | coletar; receber}
\end{EntryWithPhonetic}

\begin{EntryWithPhonetic}{收视率}{shou1shi4lv4}{6,8,11}{⽁,⾒,⽞}[HSK 7-9]
  \definition{s.}{classificações (de um programa de TV); a audiência televisiva refere"-se à porcentagem de pessoas (ou domicílios) que assistem a um determinado canal de televisão (ou programa de televisão) durante um período específico, em relação ao número total de telespectadores (ou domicílios)}
\end{EntryWithPhonetic}

\begin{EntryWithPhonetic}{收拾}{shou1shi5}{6,9}{⽁,⼿}[HSK 5]
  \definition{v.}{arrumar; empacotar; limpar; organizar, policiar, restaurar a normalidade em situações adversas | consertar; reparar; restaurar algo que está danificado ao seu estado ou função original |  punir; punir alguém, geralmente com medidas mais severas | matar}
\end{EntryWithPhonetic}

\begin{EntryWithPhonetic}{收缩}{shou1suo1}{6,14}{⽁,⽷}[HSK 7-9]
  \definition{v.}{contrair; encolher; (objeto) mudar de grande para pequeno ou de comprido para curto | recuar; concentrar as forças; apertar; mudar de dispersão para concentração}
  \synonymref{关上}{guan1shang4}
  \synonymref{减弱}{jian3ruo4}
  \synonymref{减少}{jian3shao3}
  \synonymref{紧缩}{jin3suo1}
  \synonymref{缩短}{suo1/duan3}
  \synonymref{缩小}{suo1/xiao3}
  \synonymref{萎缩}{wei3suo1}
  \synonymref{中断}{zhong1duan4}
  \antonymref{打开}{da3 kai1}
  \antonymref{发展}{fa1zhan3}
  \antonymref{开展}{kai1zhan3}
  \antonymref{扩大}{kuo4da4}
  \antonymref{扩散}{kuo4san4}
  \antonymref{扩张}{kuo4zhang1}
  \antonymref{膨胀}{peng2zhang4}
  \antonymref{拓宽}{tuo4kuan1}
  \antonymref{展开}{zhan3kai1}
\end{EntryWithPhonetic}

\begin{EntryWithPhonetic}{收听}{shou1ting1}{6,7}{⽁,⼝}[HSK 3]
  \definition{v.}{ouvir (rádio)}
\end{EntryWithPhonetic}

\begin{EntryWithPhonetic}{收养}{shou1yang3}{6,9}{⽁,⼋}[HSK 6]
  \definition{v.}{acolher e criar; adotar; acolher os filhos dos outros e criá-los como se fossem da sua própria família}
\end{EntryWithPhonetic}

\begin{EntryWithPhonetic}{收益}{shou1yi4}{6,10}{⽁,⽫}[HSK 4]
  \definition{s.}{lucro; renda; benefício; ganhos; vantagens ou benefícios obtidos}
\end{EntryWithPhonetic}

\begin{EntryWithPhonetic}{收音机}{shou1yin1ji1}{6,9,6}{⽁,⾳,⽊}[HSK 3]
  \definition[部,台]{s.}{rádio; sem fio; um termo geral para receptores de rádio}
\end{EntryWithPhonetic}

\begin{EntryWithPhonetic}{收支}{shou1zhi1}{6,4}{⽁,⽀}[HSK 7-9]
  \definition{s.}{receitas e despesas; rendimentos e despesas}
  \synonymref{出入}{chu1ru4}
  \synonymref{进出}{jin4chu1}
\end{EntryWithPhonetic}

%%%%%%%%%% 手 %%%%%%%%%%
\subsection*{手}\addcontentsline{loh}{figure}{手 \dpy{shou3}}

\begin{EntryWithPhonetic}{手}{shou3}{4}{⼿}[HSK 1][Kangxi 64]
  \definition{adj.}{prático; conveniente}
  \definition{adv.}{pessoalmente | para habilidade ou destreza}
  \definition{clas.}{usado para habilidades e competências | usado para indicar o número de vezes em que algo foi feito}
  \definition[双,只]{s.}{mão | pessoa proficiente em determinada atividade | habilidade; meios; referência a habilidades, técnicas ou meios | uma pessoa que faz ou é boa em determinado trabalho}
  \definition{v.}{ter na mão; segurar}
\end{EntryWithPhonetic}

\begin{EntryWithPhonetic}{手臂}{shou3bi4}{4,17}{⼿,⾁}[HSK 7-9]
  \definition[只,条]{s.}{braço; membros superiores humanos | Figurativo: ajudante; assistente}
\end{EntryWithPhonetic}

\begin{EntryWithPhonetic}{手边}{shou3bian1}{4,5}{⼿,⾡}
  \definition{adv.}{à mão | na mão}
\end{EntryWithPhonetic}

\begin{EntryWithPhonetic}{手表}{shou3biao3}{4,8}{⼿,⾐}[HSK 2]
  \definition[块,只,个]{s.}{relógio de pulso}
\end{EntryWithPhonetic}

\begin{EntryWithPhonetic}{手册}{shou3ce4}{4,5}{⼿,⼌}[HSK 7-9]
  \definition{s.}{manual; guia; diretório; brochura; uma coletânea de informações básicas de uso comum, um livro de referência para consulta rápida (frequentemente usado no título do livro) | diário de bordo; livro de registros; caderno de exercícios; um pequeno caderno mantido sempre à mão para fazer anotações}
\end{EntryWithPhonetic}

\begin{EntryWithPhonetic}{手动}{shou3dong4}{4,6}{⼿,⼒}[HSK 7-9]
  \definition{adj.}{operado manualmente; manual}
  \definition{s.}{mudança de marcha manual}
\end{EntryWithPhonetic}

\begin{EntryWithPhonetic}{手段}{shou3duan4}{4,9}{⼿,⽎}[HSK 5]
  \definition[种,个]{s.}{meios; meio; medida; método; métodos e técnicas utilizados para atingir um determinado objetivo | truque; artifício; métodos inadequados de lidar com as pessoas | habilidade; capacidade; delicadeza; sutileza; técnica}
\end{EntryWithPhonetic}

\begin{EntryWithPhonetic}{手法}{shou3fa3}{4,8}{⼿,⽔}[HSK 5]
  \definition[种,个]{s.}{habilidade; técnica; técnicas de criação (de obras literárias e artísticas) | truque; artifício; artimanha; refere"-se a métodos inadequados usados para lidar com as pessoas}
\end{EntryWithPhonetic}

\begin{EntryWithPhonetic}{手工}{shou3gong1}{4,3}{⼿,⼯}[HSK 4]
  \definition{s.}{trabalho manual; trabalho feito à mão | método de operação manual; método manual, sem máquina | remuneração por trabalho manual, braçal; custo de mão de obra braçal}
\end{EntryWithPhonetic}

\begin{EntryWithPhonetic}{手工艺人}{shou3gong1 yi4ren2}{4,3,4,2}{⼿,⼯,⾋,⼈}
  \definition{s.}{artesão}
\end{EntryWithPhonetic}

\begin{EntryWithPhonetic}{手机}{shou3ji1}{4,6}{⼿,⽊}[HSK 1]
  \definition[部,台,个]{s.}{celular; telefone celular; telefone móvel}
\end{EntryWithPhonetic}

\begin{EntryWithPhonetic}{手脚}{shou3jiao3}{4,11}{⼿,⾁}[HSK 7-9]
  \definition{s.}{movimento das mãos (ou pés); movimento | Dialeto: (pejorativo) método desonesto; artifício ardiloso; truque}
  \synonymref{动作}{dong4zuo4}
  \synonymref{行动}{xing2dong4}
  \synonymref{行为}{xing2wei2}
  \synonymref{作为}{zuo4wei2}
\end{EntryWithPhonetic}

\begin{EntryWithPhonetic}{手里}{shou3li5}{4,7}{⼿,⾥}[HSK 4]
  \definition[个]{s.}{(uma situação está) nas mãos de alguém | em mãos}
\end{EntryWithPhonetic}

\begin{EntryWithPhonetic}{手帕}{shou3pa4}{4,8}{⼿,⼱}[HSK 7-9]
  \definition[条,块,方]{s.}{lenço; um pano quadrado que você carrega consigo pode ser usado para enxugar suor, as mãos, etc.}
\end{EntryWithPhonetic}

\begin{EntryWithPhonetic}{手枪}{shou3qiang1}{4,8}{⼿,⽊}[HSK 7-9]
  \definition[支,把]{s.}{pistola; revólver; as pistolas de disparo com uma só mão podem ser classificadas em automáticas e semiautomáticas com base em sua construção, e são usadas para tiros a curta distância}
\end{EntryWithPhonetic}

\begin{EntryWithPhonetic}{手刹}{shou3sha1}{4,8}{⼿,⼑}
  \definition{s.}{freio de mão}
\end{EntryWithPhonetic}

\begin{EntryWithPhonetic}{手势}{shou3shi4}{4,8}{⼿,⼒}[HSK 7-9]
  \definition[个]{s.}{sinal; gesto; gestos com as mãos usados para expressar significado}
\end{EntryWithPhonetic}

\begin{EntryWithPhonetic}{手术}{shou3shu4}{4,5}{⼿,⽊}[HSK 4]
  \definition[个,次]{s.}{cirurgia; operação (cirúrgica); método de tratamento no qual o médico usa uma faca, tesoura etc. para fazer uma incisão em uma parte do corpo do paciente}
  \definition{v.}{realizar uma cirurgia}
\end{EntryWithPhonetic}

\begin{EntryWithPhonetic}{手术室}{shou3shu4shi4}{4,5,9}{⼿,⽊,⼧}[HSK 7-9]
  \definition{s.}{sala de cirurgia}
\end{EntryWithPhonetic}

\begin{EntryWithPhonetic}{手套}{shou3tao4}{4,10}{⼿,⼤}[HSK 4]
  \definition[副,套,双,种]{s.}{luvas; itens usados nas mãos, feitos de algodão, lã, couro, etc., para proteger as mãos ou manter o frio longe}
\end{EntryWithPhonetic}

\begin{EntryWithPhonetic}{手头}{shou3tou2}{4,5}{⼿,⼤}[HSK 7-9]
  \definition{s.}{à mão; disponível; bem ao lado; em posse de alguém | a situação financeira de alguém no momento; refere"-se à situação econômica pessoal | habilidades de escrita ou outras habilidades; a capacidade de escrever ou fazer outras coisas}
  \synonymref{肩膀}{jian1bang3}
\end{EntryWithPhonetic}

\begin{EntryWithPhonetic}{手腕}{shou3wan4}{4,12}{⼿,⾁}[HSK 7-9]
  \definition{s.}{pulso | truque; estratagema | artifício; habilidade; sutileza}
  \synonymref{办法}{ban4fa3}
  \synonymref{本领}{ben3ling3}
  \synonymref{本事}{ben3shi4}
  \synonymref{措施}{cuo4shi1}
  \synonymref{方法}{fang1fa3}
  \synonymref{技巧}{ji4qiao3}
  \synonymref{手段}{shou3duan4}
\end{EntryWithPhonetic}

\begin{EntryWithPhonetic}{手续}{shou3xu4}{4,11}{⼿,⽷}[HSK 3]
  \definition[项]{s.}{processo; formalidade; procedimento; procedimentos realizados de acordo com os regulamentos}
\end{EntryWithPhonetic}

\begin{EntryWithPhonetic}{手续费}{shou3xu4fei4}{4,11,9}{⼿,⽷,⾙}[HSK 6]
  \definition{s.}{comissão; corretagem; taxa de serviço; taxas a pagar pelos procedimentos de manuseio}
\end{EntryWithPhonetic}

\begin{EntryWithPhonetic}{手艺}{shou3yi4}{4,4}{⼿,⾋}[HSK 7-9]
  \definition[门]{s.}{habilidade; perícia; as habilidades dos artesãos}
  \synonymref{工夫}{gong1fu5}
  \synonymref{技能}{ji4neng2}
  \synonymref{技巧}{ji4qiao3}
  \synonymref{技术}{ji4shu4}
  \synonymref{技艺}{ji4yi4}
\end{EntryWithPhonetic}

\begin{EntryWithPhonetic}{手掌}{shou3zhang3}{4,12}{⼿,⼿}[HSK 7-9]
  \definition[个,张]{s.}{palma (da mão); a lateral da mão que as pontas dos dedos tocam quando a mão está fechada}
\end{EntryWithPhonetic}

\begin{EntryWithPhonetic}{手指}{shou3zhi3}{4,9}{⼿,⼿}[HSK 3]
  \definition[个,根,只]{s.}{dedo da mão}
\end{EntryWithPhonetic}

%%%%%%%%%% 守 %%%%%%%%%%
\subsection*{守}\addcontentsline{loh}{figure}{守 \dpy{shou3}}

\begin{EntryWithPhonetic}{守}{shou3}{6}{⼧}[HSK 4]
  \definition*{s.}{Sobrenome: Shou}
  \definition{adv.}{próximo; perto de; perto de algum lugar em posição, perto de algum lugar}
  \definition{v.}{guardar; defender; estar presente para cuidar; não ir embora | manter vigilância; defender do ataque do oponente em uma luta ou confronto | observar; cumprir; respeitar; fazer as coisas como elas devem ser feitas | manter, observar a integridade; honrar a palavra de alguém; manter a palavra de alguém}
\end{EntryWithPhonetic}

\begin{EntryWithPhonetic}{守候}{shou3hou4}{6,10}{⼧,⼈}[HSK 7-9]
  \definition{v.}{esperar; esperar por | continuar observando; cuidar de}
  \synonymref{等待}{deng3dai4}
  \synonymref{等候}{deng3hou4}
  \synonymref{期待}{qi1dai4}
  \antonymref{放弃}{fang4qi4}
\end{EntryWithPhonetic}

\begin{EntryWithPhonetic}{守护}{shou3hu4}{6,7}{⼧,⼿}[HSK 7-9]
  \definition{v.}{guardar; defender; aparar; vigiar}
  \synonymref{保护}{bao3hu4}
  \synonymref{保卫}{bao3wei4}
  \synonymref{防守}{fang2shou3}
  \antonymref{破坏}{po4huai4}
\end{EntryWithPhonetic}

\begin{EntryWithPhonetic}{守门员}{shou3men2yuan2}{6,3,7}{⼧,⾨,⼝}
  \definition{s.}{goleiro}
\end{EntryWithPhonetic}

\begin{EntryWithPhonetic}{守株待兔}{shou3zhu1-dai4tu4}{6,10,9,8}{⼧,⽊,⼻,⼉}[HSK 7-9]
  \definition{expr.}{``Esperando que um coelho bata num toco de árvore.''; guardar um toco de árvore; esperar ociosamente por oportunidades; esperar por coelhos; conta a lenda que, durante o período dos Reinos Combatentes, um agricultor do Estado de Song viu um coelho morrer após se chocar contra um toco de árvore, ele então largou suas ferramentas agrícolas e esperou ali, na esperança de pegar outro coelho que tivesse morrido da mesma forma (de ``Han Feizi'', ``Cinco Pragas''); essa história é usada para descrever alguém que, em vez de se esforçar, nutre uma mentalidade de apostador e espera por um ganho inesperado}
  \synonymref{刻舟求剑}{ke4zhou1-qiu2jian4}
\end{EntryWithPhonetic}

%%%%%%%%%% 首 %%%%%%%%%%
\subsection*{首}\addcontentsline{loh}{figure}{首 \dpy{shou3}}

\begin{EntryWithPhonetic}{首}{shou3}{9}{⾸}[HSK 4,6][Kangxi 185]
  \definition*{s.}{Sobrenome: Shou}
  \definition{adj.}{primeiro}
  \definition{adv.}{inicialmente; como o primeiro; em primeiro lugar}
  \definition{clas.}{usado para canções e poemas}
  \definition{s.}{cabeça | cabeça; chefe; líder | capital (cidade)}
  \definition{v.}{apresentar acusações contra alguém}
\end{EntryWithPhonetic}

\begin{EntryWithPhonetic}{首创}{shou3chuang4}{9,6}{⾸,⼑}[HSK 7-9]
  \definition{v.}{originar; iniciar; ser pioneiro}
  \synonymref{创办}{chuang4ban4}
  \synonymref{开创}{kai1chuang4}
  \antonymref{模仿}{mo2fang3}
\end{EntryWithPhonetic}

\begin{EntryWithPhonetic}{首次}{shou3ci4}{9,6}{⾸,⽋}[HSK 6]
  \definition{s.}{o primeiro; pela primeira vez}
\end{EntryWithPhonetic}

\begin{EntryWithPhonetic}{首都}{shou3du1}{9,10}{⾸,⾢}[HSK 3]
  \definition[个,座]{s.}{capital (cidade); a sede do mais alto poder político do país e o centro político do país}
\end{EntryWithPhonetic}

\begin{EntryWithPhonetic}{首府}{shou3fu3}{9,8}{⾸,⼴}[HSK 7-9]
  \definition{s.}{prefeitura principal (a prefeitura onde se localizava a capital da província); anteriormente se referia à localização da capital provincial | capital de uma região autônoma ou prefeitura; hoje em dia, geralmente se refere à sede do governo de uma região autônoma ou prefeitura autônoma | capital de uma dependência ou colônia; a sede do governo supremo de um estado dependente, colônia ou território sob tutela}
  \synonymref{省城}{sheng3cheng2}
\end{EntryWithPhonetic}

\begin{EntryWithPhonetic}{首脑}{shou3nao3}{9,10}{⾸,⾁}[HSK 6]
  \definition[位]{s.}{cabeça; líder; chefe}
\end{EntryWithPhonetic}

\begin{EntryWithPhonetic}{首批}{shou3pi1}{9,7}{⾸,⼿}[HSK 7-9]
  \definition{adj./adv.}{primeiro lote}
\end{EntryWithPhonetic}

\begin{EntryWithPhonetic}{首饰}{shou3shi4}{9,8}{⾸,⾷}[HSK 7-9]
  \definition[件,套,串]{s.}{joias; adorno de cabeça; originalmente referindo"-se a ornamentos usados na cabeça, agora geralmente se refere a brincos, colares, anéis, pulseiras, etc.}
\end{EntryWithPhonetic}

\begin{EntryWithPhonetic}{首席}{shou3xi2}{9,10}{⾸,⼱}[HSK 6]
  \definition{adj.}{chefe; a primeira; a posição mais alta}
  \definition{s.}{assento de honra; o assento mais honroso}
\end{EntryWithPhonetic}

\begin{EntryWithPhonetic}{首席执行官}{shou3xi2 zhi2xing2 guan1}{9,10,6,6,8}{⾸,⼱,⼿,⾏,⼧}
  \definition{s.}{\emph{chief executive officer}, CEO}
\end{EntryWithPhonetic}

\begin{EntryWithPhonetic}{首先}{shou3xian1}{9,6}{⾸,⼉}[HSK 3]
  \definition{adv.}{primeiramente; antes de todos os outros}
  \definition{conj.}{acima de tudo; primeiramente; em primeiro lugar}
\end{EntryWithPhonetic}

\begin{EntryWithPhonetic}{首相}{shou3xiang4}{9,9}{⾸,⽬}[HSK 6]
  \definition*[个,名,位]{s.}{Primeiro-Ministro (Japão, UK, etc.); o mais alto cargo oficial no gabinete de uma monarquia; o chefe do governo central de alguns países não monárquicos às vezes usa esse nome}
\end{EntryWithPhonetic}

\begin{EntryWithPhonetic}{首要}{shou3yao4}{9,9}{⾸,⾑}[HSK 7-9]
  \definition{adj.}{principal; primordial; de suma importância; prioridade máxima | líder}
  \synonymref{严重}{yan2zhong4}
  \synonymref{重要}{zhong4yao4}
  \synonymref{主体}{zhu3ti3}
  \synonymref{主要}{zhu3yao4}
\end{EntryWithPhonetic}

%%%%%%%%%% 掱 %%%%%%%%%%
\subsection*{掱}\addcontentsline{loh}{figure}{掱 \dpy{shou3}}

\begin{EntryWithPhonetic}{掱}{shou3}{12}{⼿}
  \variantof{手}
\end{EntryWithPhonetic}

%%%%%%%%%% 寿 %%%%%%%%%%
\subsection*{寿}\addcontentsline{loh}{figure}{寿 \dpy{shou4}}

\begin{EntryWithPhonetic}{寿}{shou4}{7}{⼨}
  \definition[个,份]{s.}{vida longa; velhice | vida; idade | aniversário | (eufenismo) funerário; preparado antes da morte | longevidade}
\end{EntryWithPhonetic}

\begin{EntryWithPhonetic}{寿命}{shou4ming4}{7,8}{⼨,⼝}[HSK 7-9]
  \definition{s.}{vida; tempo de vida; longevidade; expectativa de vida | vida útil; tempo de vida; expectativa de vida; metaforicamente, refere"-se ao período de uso ou validade de um item}
  \synonymref{年纪}{nian2ji4}
  \synonymref{年龄}{nian2ling2}
  \synonymref{年限}{nian2xian4}
  \synonymref{生命}{sheng1ming4}
\end{EntryWithPhonetic}

\begin{EntryWithPhonetic}{寿司}{shou4si1}{7,5}{⼨,⼝}[HSK 5]
  \definition[份]{s.}{\emph{sushi}; iguaria tradicional japonesa}
\end{EntryWithPhonetic}

%%%%%%%%%% 受 %%%%%%%%%%
\subsection*{受}\addcontentsline{loh}{figure}{受 \dpy{shou4}}

\begin{EntryWithPhonetic}{受}{shou4}{8}{⼜}[HSK 3]
  \definition{v.}{receber; aceitar | sofrer; ser submetido a | aguentar; suportar; tolerar | ser agradável}
\end{EntryWithPhonetic}

\begin{EntryWithPhonetic}{受不了}{shou4bu5liao3}{8,4,2}{⼜,⼀,⼅}[HSK 4]
  \definition{v.}{ser insuportável; não poder suportar algo; não suportar algo}
\end{EntryWithPhonetic}

\begin{EntryWithPhonetic}{受到}{shou4dao4}{8,8}{⼜,⼑}[HSK 2]
  \definition{v.}{receber; receber itens, mensagens, instruções, etc. fornecidos por outras pessoas}
\end{EntryWithPhonetic}

\begin{EntryWithPhonetic}{受得了}{shou4de5liao3}{8,11,2}{⼜,⼻,⼅}
  \definition{v.}{suportar | aguentar}
\end{EntryWithPhonetic}

\begin{EntryWithPhonetic}{受过}{shou4/guo4}{8,6}{⼜,⾡}[HSK 7-9]
  \definition{v.+compl.}{assumir a culpa (por outra pessoa) | sofrer}
\end{EntryWithPhonetic}

\begin{EntryWithPhonetic}{受害}{shou4/hai4}{8,10}{⼜,⼧}[HSK 7-9]
  \definition{v.+compl.}{sofrer lesão; ser vítima de queda; ser afetado por | ser afetado; ser afligido}
  \synonymref{受伤}{shou4/shang1}
  \antonymref{受益}{shou4yi4}
\end{EntryWithPhonetic}

\begin{EntryWithPhonetic}{受害人}{shou4hai4ren2}{8,10,2}{⼜,⼧,⼈}[HSK 7-9]
  \definition{s.}{vítima | sofredor}
\end{EntryWithPhonetic}

\begin{EntryWithPhonetic}{受贿}{shou4/hui4}{8,10}{⼜,⾙}[HSK 7-9]
  \definition{v.+compl.}{aceitar (ou receber) subornos}
\end{EntryWithPhonetic}

\begin{EntryWithPhonetic}{受惊}{shou4/jing1}{8,11}{⼜,⼼}[HSK 7-9]
  \definition{adj.}{assustado; sobressaltado; medo causado por estímulos ou ameaças repentinas}
  \definition{v.+compl.}{ficar assustado; levar um susto}
  \synonymref{吃惊}{chi1/jing1}
  \antonymref{坦然}{tan3ran2}
\end{EntryWithPhonetic}

\begin{EntryWithPhonetic}{受苦}{shou4/ku3}{8,8}{⼜,⾋}[HSK 7-9]
  \definition{v.+compl.}{sofrer (dificuldades); passar por momentos difíceis}
  \synonymref{吃苦}{chi1/ku3}
  \synonymref{刻苦}{ke4ku3}
  \antonymref{舒服}{shu1fu5}
\end{EntryWithPhonetic}

\begin{EntryWithPhonetic}{受理}{shou4li3}{8,11}{⼜,⽟}[HSK 7-9]
  \definition{v.}{aceitar e tratar um caso (autoridades judiciais) | aceitar e lidar com; aceitar e processar}
  \synonymref{办理}{ban4li3}
  \synonymref{承办}{cheng2ban4}
  \synonymref{处理}{chu3li3}
  \synonymref{处置}{chu3zhi4}
  \antonymref{驳回}{bo2hui2}
\end{EntryWithPhonetic}

\begin{EntryWithPhonetic}{受骗}{shou4/pian4}{8,12}{⼜,⾺}[HSK 7-9]
  \definition{v.+compl.}{ser enganado; ser iludido; ser ludibriado}
  \synonymref{上当}{shang4/dang4}
\end{EntryWithPhonetic}

\begin{EntryWithPhonetic}{受伤}{shou4/shang1}{8,6}{⼜,⼈}[HSK 3]
  \definition{v.+compl.}{ser ferido; sofrer uma lesão}
\end{EntryWithPhonetic}

\begin{EntryWithPhonetic}{受限}{shou4xian4}{8,8}{⼜,⾩}
  \definition{v.}{ser limitado | ser restrito | ser constrangido}
\end{EntryWithPhonetic}

\begin{EntryWithPhonetic}{受益}{shou4yi4}{8,10}{⼜,⽫}[HSK 7-9]
  \definition{v.}{lucrar com; beneficiar"-se de; ser beneficiado; receber benefícios; obter vantagens}
  \synonymref{利益}{li4yi4}
  \synonymref{收益}{shou1yi4}
  \antonymref{受害}{shou4/hai4}
\end{EntryWithPhonetic}

\begin{EntryWithPhonetic}{受灾}{shou4 zai1}{8,7}{⼜,⽕}[HSK 5]
  \definition{v.}{ser atingido por um desastre natural (ou calamidade) | ser atingido por uma adversidade natural}
\end{EntryWithPhonetic}

%%%%%%%%%% 兽 %%%%%%%%%%
\subsection*{兽}\addcontentsline{loh}{figure}{兽 \dpy{shou4}}

\begin{EntryWithPhonetic}{兽}{shou4}{11}{⼋}
  \definition{adj.}{bestial; brutal}
  \definition{s.}{besta; animal}
\end{EntryWithPhonetic}

\begin{EntryWithPhonetic}{兽力车}{shou4li4che1}{11,2,4}{⼋,⼒,⾞}
  \definition{s.}{veículo puxado por animais | carruagem; carroça}
  \antonymref{人力车}{ren2li4che1}
\end{EntryWithPhonetic}

\begin{EntryWithPhonetic}{兽行}{shou4xing2}{11,6}{⼋,⾏}
  \definition{s.}{ato brutal; brutalidade | bestialidade}
\end{EntryWithPhonetic}

%%%%%%%%%% 售 %%%%%%%%%%
\subsection*{售}\addcontentsline{loh}{figure}{售 \dpy{shou4}}

\begin{EntryWithPhonetic}{售}{shou4}{11}{⼝}
  \definition{v.}{vender | fazer (o plano, truque, etc.) funcionar; continuar (as intrigas) | realizar (intrigas)}
\end{EntryWithPhonetic}

\begin{EntryWithPhonetic}{售货员}{shou4huo4yuan2}{11,8,7}{⼝,⾙,⼝}[HSK 4]
  \definition[名,位]{s.}{vendedor; balconista; assistente de loja; equipe que vende produtos em lojas}
\end{EntryWithPhonetic}

\begin{EntryWithPhonetic}{售价}{shou4jia4}{11,6}{⼝,⼈}[HSK 7-9]
  \definition{s.}{preço; preço de venda}
\end{EntryWithPhonetic}

\begin{EntryWithPhonetic}{售票}{shou4piao4}{11,11}{⼝,⽰}[HSK 7-9]
  \definition{s.}{venda (vendedor) de ingressos}
  \definition{v.}{vender ingressos}
  \synonymref{售货员}{shou4huo4yuan2}
\end{EntryWithPhonetic}

%%%%%%%%%% 授 %%%%%%%%%%
\subsection*{授}\addcontentsline{loh}{figure}{授 \dpy{shou4}}

\begin{EntryWithPhonetic}{授}{shou4}{11}{⼿}
  \definition*{s.}{Sobrenome: Shou}
  \definition{v.}{premiar; conferir; conceder; dar; entregar | ensinar; instruir; transmitir; ensinar}
  \synonymref{教}{jiao1}
  \synonymref{教}{jiao4}
  \antonymref{受}{shou4}
\end{EntryWithPhonetic}

\begin{EntryWithPhonetic}{授权}{shou4quan2}{11,6}{⼿,⽊}[HSK 7-9]
  \definition{v.}{capacitar; autorizar; garantir; delegar poder a indivíduos ou organizações para execução}
  \synonymref{授予}{shou4yu3}
  \synonymref{委托}{wei3tuo1}
\end{EntryWithPhonetic}

\begin{EntryWithPhonetic}{授予}{shou4yu3}{11,4}{⼿,⼅}[HSK 7-9]
  \definition{v.}{conferir; premiar; conceder; atribuir (medalhas, graus, títulos, etc.)}
  \synonymref{赋予}{fu4yu3}
  \synonymref{授权}{shou4quan2}
  \antonymref{剥夺}{bo1duo2}
\end{EntryWithPhonetic}

%%%%%%%%%% 瘦 %%%%%%%%%%
\subsection*{瘦}\addcontentsline{loh}{figure}{瘦 \dpy{shou4}}

\begin{EntryWithPhonetic}{瘦}{shou4}{14}{⽧}[HSK 5]
  \definition{adj.}{magro; esquelético | magro | apertado | infértil; pobre | esquelético; pouca gordura; pouca carne | (roupas, sapatos, meias, etc.) apertado |magra; (carne comestível) com baixo teor de gordura}
  \definition{v.}{perder peso}
  \antonymref{肥}{fei2}
  \antonymref{或}{huo4}
  \antonymref{胖}{pang4}
\end{EntryWithPhonetic}

%%%%%%%%%% 书 %%%%%%%%%%
\subsection*{书}\addcontentsline{loh}{figure}{书 \dpy{shu1}}

\begin{EntryWithPhonetic}{书}{shu1}{4}{⼄}[HSK 1]
  \definition*{s.}{Sobrenome: Shu}
  \definition[本,册,部,套,卷]{s.}{livro; obras encadernadas | carta; carta especial | documento | estilo de caligrafia; escrita}
  \definition{v.}{escrever; registrar}
\end{EntryWithPhonetic}

\begin{EntryWithPhonetic}{书包}{shu1bao1}{4,5}{⼄,⼓}[HSK 1]
  \definition[个,款]{s.}{mochila para guardar livros e materiais escolares}
  \synonymref{背包}{bei1bao1}
\end{EntryWithPhonetic}

\begin{EntryWithPhonetic}{书橱}{shu1chu2}{4,16}{⼄,⽊}[HSK 7-9]
  \definition{s.}{estante (geralmente com portas de vidro); armário de livros | estante de livros}
  \synonymref{书柜}{shu1gui4}
\end{EntryWithPhonetic}

\begin{EntryWithPhonetic}{书店}{shu1dian4}{4,8}{⼄,⼴}[HSK 1]
  \definition[个,家]{s.}{livraria; lojas que vendem livros}
\end{EntryWithPhonetic}

\begin{EntryWithPhonetic}{书法}{shu1fa3}{4,8}{⼄,⽔}[HSK 5]
  \definition[幅,卷,种,派]{s.}{caligrafia; arte de escrever caracteres, especialmente arte de escrever caracteres chineses com um pincel}
\end{EntryWithPhonetic}

\begin{EntryWithPhonetic}{书房}{shu1fang2}{4,8}{⼄,⼾}[HSK 6]
  \definition[间]{s.}{uma biblioteca (em uma residência privada); espaço para leitura e escrita}
\end{EntryWithPhonetic}

\begin{EntryWithPhonetic}{书柜}{shu1gui4}{4,8}{⼄,⽊}[HSK 5]
  \definition{s.}{estante; armário de livros}
  \synonymref{书橱}{shu1chu2}
\end{EntryWithPhonetic}

\begin{EntryWithPhonetic}{书籍}{shu1ji2}{4,20}{⼄,⽵}[HSK 7-9]
  \definition[本,册,堆,套]{s.}{livros; obras; literatura; o termo geral para livros}
\end{EntryWithPhonetic}

\begin{EntryWithPhonetic}{书记}{shu1ji4}{4,5}{⼄,⾔}[HSK 7-9]
  \definition[位]{s.}{secretário; principais líderes do Partido, da Liga da Juventude e de outras organizações em todos os níveis | escriturário; antigamente, isso se referia ao pessoal que lidava com documentos e os registrava por escrito}
\end{EntryWithPhonetic}

\begin{EntryWithPhonetic}{书架}{shu1jia4}{4,9}{⼄,⽊}[HSK 3]
  \definition[个,种,套]{s.}{estante de livros}
  \synonymref{书橱}{shu1chu2}
  \synonymref{书柜}{shu1gui4}
\end{EntryWithPhonetic}

\begin{EntryWithPhonetic}{书面}{shu1mian4}{4,9}{⼄,⾯}[HSK 7-9]
  \definition{s.}{escrito; em forma escrita (em oposição à 口头)}
  \seealsoref{口头}{kou3tou2}
\end{EntryWithPhonetic}

\begin{EntryWithPhonetic}{书写}{shu1xie3}{4,5}{⼄,⼍}[HSK 7-9]
  \definition{v.}{escrever (usado principalmente na forma escrita)}
  \synonymref{抄写}{chao1xie3}
\end{EntryWithPhonetic}

\begin{EntryWithPhonetic}{书桌}{shu1zhuo1}{4,10}{⼄,⽊}[HSK 5]
  \definition[个,张]{s.}{escrivaninha; mesa para ler e escrever}
\end{EntryWithPhonetic}

%%%%%%%%%% 抒 %%%%%%%%%%
\subsection*{抒}\addcontentsline{loh}{figure}{抒 \dpy{shu1}}

\begin{EntryWithPhonetic}{抒}{shu1}{7}{⼿}
  \definition{v.}{expressar; dar expressão a; transmitir | expressar; publicar | explicar; liberar}
\end{EntryWithPhonetic}

\begin{EntryWithPhonetic}{抒情}{shu1qing2}{7,11}{⼿,⼼}[HSK 7-9]
  \definition{v.}{expressar as próprias emoções}
\end{EntryWithPhonetic}

%%%%%%%%%% 叔 %%%%%%%%%%
\subsection*{叔}\addcontentsline{loh}{figure}{叔 \dpy{shu1}}

\begin{EntryWithPhonetic}{叔}{shu1}{8}{⼜}
  \definition*{s.}{Sobrenome: Shu}
  \definition{s.}{irmão mais novo do pai; tio (por parte de pai)| irmão mais novo do marido | terceiro entre irmãos | tio | uma forma de tratamento para um homem um pouco mais jovem que o pai; tio | terceiro tio (de quatro irmãos) | primo mais novo da mãe}
\end{EntryWithPhonetic}

\begin{EntryWithPhonetic}{叔叔}{shu1shu5}{8,8}{⼜,⼜}[HSK 4]
  \definition[个,位,名]{s.}{tio; irmão mais novo do pai | tio, dirigindo"-se a um homem da mesma geração que o pai e mais jovem em idade}
\end{EntryWithPhonetic}

%%%%%%%%%% 枢 %%%%%%%%%%
\subsection*{枢}\addcontentsline{loh}{figure}{枢 \dpy{shu1}}

\begin{EntryWithPhonetic}{枢}{shu1}{8}{⽊}
  \definition[户]{s.}{dobradiça de porta | pivô; eixo; centro; parte importante ou central}
\end{EntryWithPhonetic}

\begin{EntryWithPhonetic}{枢纽}{shu1niu3}{8,7}{⽊,⽷}[HSK 7-9]
  \definition{s.}{eixo; pivô; posição chave; a chave para tudo, o elo central na interconexão das coisas}
  \synonymref{关键}{guan1jian4}
  \synonymref{关节}{guan1jie2}
  \synonymref{环节}{huan2jie2}
  \synonymref{纽带}{niu3dai4}
\end{EntryWithPhonetic}

%%%%%%%%%% 梳 %%%%%%%%%%
\subsection*{梳}\addcontentsline{loh}{figure}{梳 \dpy{shu1}}

\begin{EntryWithPhonetic}{梳}{shu1}{11}{⽊}[HSK 7-9]
  \definition{s.}{pente}
  \definition{v.}{pentear (o cabelo, etc.)}
\end{EntryWithPhonetic}

\begin{EntryWithPhonetic}{梳理}{shu1li3}{11,11}{⽊,⽟}[HSK 7-9]
  \definition{v.}{pentear; na fabricação têxtil, o uso de agulhas ou dentes em uma máquina para alinhar as fibras e remover fibras curtas e impurezas é um processo | pentear (o cabelo); desembaraçar (o cabelo); usar um pente para pentear (barba, cabelo, etc.) | organizar; resolver assuntos, problemas, etc.}
  \synonymref{整顿}{zheng3dun4}
  \synonymref{整理}{zheng3li3}
  \synonymref{整治}{zheng3zhi4}
  \synonymref{治理}{zhi4li3}
\end{EntryWithPhonetic}

\begin{EntryWithPhonetic}{梳子}{shu1zi5}{11,3}{⽊,⼦}[HSK 7-9]
  \definition[把,个]{s.}{pente; ferramentas para cuidar do cabelo e da barba}
\end{EntryWithPhonetic}

%%%%%%%%%% 疏 %%%%%%%%%%
\subsection*{疏}\addcontentsline{loh}{figure}{疏 \dpy{shu1}}

\begin{EntryWithPhonetic}{疏}{shu1}{12}{⽦}
  \definition*{s.}{Sobrenome: Shu}
  \definition{adj.}{fino; esparso; disperso | espalhado; disperso; difuso; a distância entre as coisas é grande; as lacunas entre as partes das coisas são grandes | distante; relacionamento distante; não próximo (de relações familiares ou sociais) | não familiarizado com; desconhecido | escasso; vazio}
  \definition{s.}{memorial; memorial ao trono; um texto em que um ministro na era feudal apresentava seus assuntos ao monarca em detalhes | comentário; anotações mais detalhadas de livros antigos do que 注}
  \definition{v.}{dragar (um rio, etc.) | negligenciar | dispersar; espalhar}
  \seealsoref{注}{zhu4}
  \antonymref{密}{mi4}
\end{EntryWithPhonetic}

\begin{EntryWithPhonetic}{疏导}{shu1dao3}{12,6}{⽦,⼨}[HSK 7-9]
  \definition{v.}{drenar; a analogia de guiar a água para fluir em um rio também é usada para descrever o ato de fazer com que pessoas e veículos presos no trânsito comecem a se mover | aconselhar; melhorar estados psicológicos negativos através de orientação}
  \synonymref{沟通}{gou1tong1}
  \synonymref{疏通}{shu1tong1}
  \synonymref{引导}{yin3dao3}
  \antonymref{堵塞}{du3se4}
\end{EntryWithPhonetic}

\begin{EntryWithPhonetic}{疏忽}{shu1hu5}{12,8}{⽦,⼼}[HSK 7-9]
  \definition{v.}{cometer um deslize; negligenciar; ser descuidado; não perceber por descuido}
  \synonymref{粗心}{cu1xin1}
  \synonymref{大意}{da4yi5}
  \synonymref{怠慢}{dai4man4}
  \synonymref{忽略}{hu1lve4}
  \synonymref{忽视}{hu1shi4}
  \synonymref{马虎}{ma3hu5}
  \synonymref{松弛}{song1chi2}
  \synonymref{无视}{wu2shi4}
  \antonymref{谨慎}{jin3shen4}
  \antonymref{精细}{jing1xi4}
  \antonymref{精心}{jing1xin1}
  \antonymref{警惕}{jing3ti4}
  \antonymref{留意}{liu2/yi4}
  \antonymref{慎重}{shen4zhong4}
  \antonymref{注意}{zhu4/yi4}
  \antonymref{仔细}{zi3xi4}
\end{EntryWithPhonetic}

\begin{EntryWithPhonetic}{疏散}{shu1san4}{12,12}{⽦,⽁}[HSK 7-9]
  \definition{adj.}{esparso; disperso; espalhado; escasso}
  \definition{v.}{desocupar; dispersar; evacuar; dispersar as pessoas ou coisas que estão reunidas}
  \synonymref{分散}{fen1san4}
  \antonymref{集结}{ji2jie2}
  \antonymref{集中}{ji2zhong1}
  \antonymref{紧紧}{jin3jin3}
  \antonymref{密集}{mi4ji2}
\end{EntryWithPhonetic}

\begin{EntryWithPhonetic}{疏通}{shu1tong1}{12,10}{⽦,⾡}[HSK 7-9]
  \definition{v.}{dragar | mediar entre duas partes}
  \synonymref{畅通}{chang4tong1}
  \synonymref{沟通}{gou1tong1}
  \synonymref{疏导}{shu1dao3}
  \synonymref{运动}{yun4dong5}
  \antonymref{堵塞}{du3se4}
\end{EntryWithPhonetic}

%%%%%%%%%% 舒 %%%%%%%%%%
\subsection*{舒}\addcontentsline{loh}{figure}{舒 \dpy{shu1}}

\begin{EntryWithPhonetic}{舒}{shu1}{12}{⾆}
  \definition*{s.}{Sobrenome: Shu}
  \definition{adj.}{lento; vagaroso; sem pressa | confortável; relaxado e feliz}
  \definition{v.}{esticar; desdobrar | alongar; relaxar}
\end{EntryWithPhonetic}

\begin{EntryWithPhonetic}{舒畅}{shu1chang4}{12,8}{⾆,⽥}[HSK 7-9]
  \definition{adj.}{feliz; confortável; completamente livre de preocupações}
  \synonymref{安逸}{an1yi4}
  \synonymref{高兴}{gao1xing4}
  \synonymref{舒适}{shu1shi4}
  \synonymref{舒服}{shu1fu5}
  \synonymref{痛快}{tong4kuai5}
  \synonymref{写意}{xie4yi4}
  \antonymref{沉闷}{chen2men4}
  \antonymref{烦闷}{fan2men4}
  \antonymref{难过}{nan2guo4}
\end{EntryWithPhonetic}

\begin{EntryWithPhonetic}{舒服}{shu1fu5}{12,8}{⾆,⽉}[HSK 2]
  \definition{adj.}{confortável; sentir"-se relaxado e feliz, tanto física quanto mentalmente}
\end{EntryWithPhonetic}

\begin{EntryWithPhonetic}{舒适}{shu1shi4}{12,9}{⾆,⾡}[HSK 4]
  \definition{adj.}{aconchegante; confortável; acolhedor; cômodo}
\end{EntryWithPhonetic}

%%%%%%%%%% 输 %%%%%%%%%%
\subsection*{输}\addcontentsline{loh}{figure}{输 \dpy{shu1}}

\begin{EntryWithPhonetic}{输}{shu1}{13}{⾞}[HSK 3]
  \definition{v.}{transportar; entregar | contribuir com dinheiro; doar | perder; falhar; ser batido; ser derrotado}
\end{EntryWithPhonetic}

\begin{EntryWithPhonetic}{输出}{shu1chu1}{13,5}{⾞,⼐}[HSK 5]
  \definition{v.}{exportar (de dentro para fora); transportar (de dentro) para fora | exportar; vender ou distribuir no exterior ou fora do país | emitir informações, programas, dados, sinais, etc. a partir de uma máquina; enviar por uma determinada instituição ou dispositivo (energia, sinal, etc.)}
\end{EntryWithPhonetic}

\begin{EntryWithPhonetic}{输家}{shu1jia1}{13,10}{⾞,⼧}[HSK 7-9]
  \definition{s.}{perdedor (especialmente em jogos de azar)}
\end{EntryWithPhonetic}

\begin{EntryWithPhonetic}{输入}{shu1ru4}{13,2}{⾞,⼊}[HSK 3]
  \definition{v.}{introduzir; importar; comprar bens, introduzir tecnologia, contratar mão de obra, introduzir capital, etc. | inserir informações, programas, dados, sinais, etc. em uma máquina}
\end{EntryWithPhonetic}

\begin{EntryWithPhonetic}{输送}{shu1song4}{13,9}{⾞,⾡}[HSK 7-9]
  \definition{v.}{transferir; transportar; fornecer}
  \synonymref{传输}{chuan2shu1}
  \synonymref{配送}{pei4song4}
  \synonymref{运输}{yun4shu1}
  \antonymref{回收}{hui2shou1}
\end{EntryWithPhonetic}

\begin{EntryWithPhonetic}{输血}{shu1/xue4}{13,6}{⾞,⾎}[HSK 7-9]
  \definition{v.+compl.}{prestar auxílio e apoio | transfundir sangue}
\end{EntryWithPhonetic}

\begin{EntryWithPhonetic}{输液}{shu1/ye4}{13,11}{⾞,⽔}[HSK 7-9]
  \definition{v.+compl.}{transfundir; receber soro intravenoso}
  \antonymref{输出}{shu1chu1}
\end{EntryWithPhonetic}

%%%%%%%%%% 蔬 %%%%%%%%%%
\subsection*{蔬}\addcontentsline{loh}{figure}{蔬 \dpy{shu1}}

\begin{EntryWithPhonetic}{蔬}{shu1}{15}{⾋}
  \definition{s.}{vegetais}
\end{EntryWithPhonetic}

\begin{EntryWithPhonetic}{蔬菜}{shu1cai4}{15,11}{⾋,⾋}[HSK 5]
  \definition[样,种]{s.}{verduras; legumes; vegetais; ervas que podem ser usadas na culinária}
\end{EntryWithPhonetic}

%%%%%%%%%% 赎 %%%%%%%%%%
\subsection*{赎}\addcontentsline{loh}{figure}{赎 \dpy{shu2}}

\begin{EntryWithPhonetic}{赎}{shu2}{12}{⾙}[HSK 7-9]
  \definition*{s.}{Sobrenome: Shu}
  \definition{v.}{resgatar; pedir resgate; usar dinheiro para trocar por garantias | expiar (um crime); desviar; inventar}
  \antonymref{当}{dang4}
\end{EntryWithPhonetic}

%%%%%%%%%% 熟 %%%%%%%%%%
\subsection*{熟}\addcontentsline{loh}{figure}{熟 \dpy{shu2}}

\begin{EntryWithPhonetic}{熟}{shu2}{15}{⽕}[HSK 2]
  \definition{adj.}{maduro (frutos) | pronto; cozido | processado, fabricado ou exercitado | familiar, bem conhecido; conhecido por ser comum ou frequentemente utilizado | habilidoso;  (trabalho, tecnologia) experiente; não é novato | profundo; sólido}
\end{EntryWithPhonetic}

\begin{EntryWithPhonetic}{熟练}{shu2lian4}{15,8}{⽕,⽷}[HSK 4]
  \definition{adj.}{especializado; proficiente; qualificado; habilidoso}
\end{EntryWithPhonetic}

\begin{EntryWithPhonetic}{熟人}{shu2ren2}{15,2}{⽕,⼈}[HSK 3]
  \definition[位,名,个,些]{s.}{amigo; conhecido; pessoas que se conhecem há muito tempo; pessoas que são muito familiares}
\end{EntryWithPhonetic}

\begin{EntryWithPhonetic}{熟悉}{shu2xi5}{15,11}{⽕,⼼}[HSK 5]
  \definition{adj.}{familiarizado com; não ser estranho}
  \definition{v.}{estar familiarizado com; saber claramente que | conhecer bem algo ou alguém; compreender e dominar (a situação) através da observação ou da experiência}
\end{EntryWithPhonetic}

%%%%%%%%%% 属 %%%%%%%%%%
\subsection*{属}\addcontentsline{loh}{figure}{属 \dpy{shu3}}

\begin{EntryWithPhonetic}{属}{shu3}{12}{⼫}[HSK 3]
  \definition{s.}{categoria | gênero | membros da família; dependentes; familiares; parentes}
  \definition{v.}{estar sob; subordinado a | pertencer a | nascer no ano de (um dos doze animais do zodíaco)}
  \seeref{zhu3}
\end{EntryWithPhonetic}

\begin{EntryWithPhonetic}{属性}{shu3xing4}{12,8}{⼫,⼼}[HSK 7-9]
  \definition{s.}{um atributo; uma propriedade; as propriedades inerentes das coisas; tudo possui múltiplos atributos, que podem ser categorizados em atributos essenciais e atributos não essenciais}
  \synonymref{本质}{ben3zhi4}
  \synonymref{特性}{te4xing4}
  \synonymref{特质}{te4zhi4}
\end{EntryWithPhonetic}

\begin{EntryWithPhonetic}{属于}{shu3yu2}{12,3}{⼫,⼆}[HSK 3]
  \definition{v.}{pertencer a; fazer parte de; pertencer ou ser propriedade de uma determinada parte}
\end{EntryWithPhonetic}

%%%%%%%%%% 暑 %%%%%%%%%%
\subsection*{暑}\addcontentsline{loh}{figure}{暑 \dpy{shu3}}

\begin{EntryWithPhonetic}{暑}{shu3}{12}{⽇}
  \definition{adj.}{calor; clima quente; quente}
  \definition{s.}{verão}
  \antonymref{寒}{han2}
\end{EntryWithPhonetic}

\begin{EntryWithPhonetic}{暑假}{shu3jia4}{12,11}{⽇,⼈}[HSK 4]
  \definition[个]{s.}{férias de verão; feriado de verão; férias escolares de verão, na China, durante o sétimo e o oitavo meses do calendário gregoriano}
\end{EntryWithPhonetic}

\begin{EntryWithPhonetic}{暑期}{shu3qi1}{12,12}{⽇,⽉}[HSK 7-9]
  \definition[个]{s.}{época de férias de verão | durante as férias de verão}
  \synonymref{暑假}{shu3jia4}
  \antonymref{寒假}{han2jia4}
\end{EntryWithPhonetic}

%%%%%%%%%% 黍 %%%%%%%%%%
\subsection*{黍}\addcontentsline{loh}{figure}{黍 \dpy{shu3}}

\begin{EntryWithPhonetic}{黍}{shu3}{12}{⿉}[Kangxi 202]
  \definition{s.}{painço}
\end{EntryWithPhonetic}

%%%%%%%%%% 数 %%%%%%%%%%
\subsection*{数}\addcontentsline{loh}{figure}{数 \dpy{shu3}}

\begin{EntryWithPhonetic}{数}{shu3}{13}{⽁}[HSK 2]
  \definition{v.}{contar (número); contar (número) um a um | ser considerado excepcionalmente (bom, ruim, etc.) | enumerar; listar}
  \seeref{shu4}
  \seeref{shuo4}
\end{EntryWithPhonetic}

%%%%%%%%%% 鼠 %%%%%%%%%%
\subsection*{鼠}\addcontentsline{loh}{figure}{鼠 \dpy{shu3}}

\begin{EntryWithPhonetic}{鼠}{shu3}{13}{⿏}[HSK 5][Kangxi 208]
  \definition[只]{s.}{rato; camundongo}
\end{EntryWithPhonetic}

\begin{EntryWithPhonetic}{鼠标}{shu3biao1}{13,9}{⿏,⽊}[HSK 5]
  \definition[个,只]{s.}{\emph{mouse} (de computador); dispositivo de entrada externo para computadores, usado para controlar o movimento do cursor na tela do computador, selecionar objetos de operação, executar vários comandos, etc.}
\end{EntryWithPhonetic}

%%%%%%%%%% 薯 %%%%%%%%%%
\subsection*{薯}\addcontentsline{loh}{figure}{薯 \dpy{shu3}}

\begin{EntryWithPhonetic}{薯}{shu3}{16}{⾋}
  \definition{s.}{batata | inhame}
\end{EntryWithPhonetic}

\begin{EntryWithPhonetic}{薯片}{shu3pian4}{16,4}{⾋,⽚}[HSK 6]
  \definition{s.}{batatas fritas (\emph{chips}); batatas fritas crocantes ; flocos finos feitos de batatas}
\end{EntryWithPhonetic}

\begin{EntryWithPhonetic}{薯条}{shu3tiao2}{16,7}{⾋,⽊}[HSK 6]
  \definition{s.}{batatas fritas (palito)}
\end{EntryWithPhonetic}

%%%%%%%%%% 曙 %%%%%%%%%%
\subsection*{曙}\addcontentsline{loh}{figure}{曙 \dpy{shu3}}

\begin{EntryWithPhonetic}{曙}{shu3}{17}{⽇}
  \definition{s.}{Literário: amanhecer; alvorecer | Literário: o alvorecer de uma nova época (metáfora)}
\end{EntryWithPhonetic}

\begin{EntryWithPhonetic}{曙光}{shu3guang1}{17,6}{⽇,⼉}[HSK 7-9]
  \definition{s.}{aurora; amanhecer; crepúsculo; primeira luz da manhã; a primeira luz do dia; crepúsculo matutino; luz do sol surgindo no horizonte | amanhecer; uma metáfora para um futuro brilhante que já está à vista}
\end{EntryWithPhonetic}

%%%%%%%%%% 术 %%%%%%%%%%
\subsection*{术}\addcontentsline{loh}{figure}{术 \dpy{shu4}}

\begin{EntryWithPhonetic}{术}{shu4}{5}{⽊}
  \definition*{s.}{Sobrenome: Shu}
  \definition{s.}{arte; habilidade; técnica; tecnologia; acadêmico | método; tática; estratégia}
  \seeref{zhu2}
\end{EntryWithPhonetic}

\begin{EntryWithPhonetic}{术科}{shu4ke1}{5,9}{⽊,⽲}
  \definition{s.}{cursos técnicos oferecidos em treinamento militar ou físico}
  \antonymref{学科}{xue2ke1}
\end{EntryWithPhonetic}

%%%%%%%%%% 束 %%%%%%%%%%
\subsection*{束}\addcontentsline{loh}{figure}{束 \dpy{shu4}}

\begin{EntryWithPhonetic}{束}{shu4}{7}{⽊}[HSK 3]
  \definition*{s.}{Sobrenome: Shu}
  \definition{clas.}{usado para cachos, molhos, feixes, feixes de luz, etc.}
  \definition{s.}{monte; pacote; maço; feixe; cacho; coisas agrupadas ou reunidas em tiras}
  \definition{v.}{atar; amarrar; vincular | controlar; restringir}
\end{EntryWithPhonetic}

\begin{EntryWithPhonetic}{束缚}{shu4fu4}{7,13}{⽊,⽷}[HSK 7-9]
  \definition{v.}{amarrar; acorrentar; prender; restringir ou limitar; confinar a um escopo restrito}
  \synonymref{管理}{guan3li3}
  \synonymref{拘束}{ju1shu4}
  \synonymref{牵制}{qian1zhi4}
  \synonymref{限制}{xian4zhi4}
  \synonymref{约束}{yue1shu4}
  \antonymref{解放}{jie3fang4}
  \antonymref{解脱}{jie3tuo1}
  \antonymref{任凭}{ren4 ping2}
  \antonymref{自由}{zi4you2}
\end{EntryWithPhonetic}

\begin{EntryWithPhonetic}{束腰}{shu4yao1}{7,13}{⽊,⾁}
  \definition{s.}{cinto | cinta | cinturão}
\end{EntryWithPhonetic}

%%%%%%%%%% 树 %%%%%%%%%%
\subsection*{树}\addcontentsline{loh}{figure}{树 \dpy{shu4}}

\begin{EntryWithPhonetic}{树}{shu4}{9}{⽊}[HSK 1]
  \definition*{s.}{Sobrenome: Shu}
  \definition[棵,株]{s.}{árvore; nome comum das plantas lenhosas}
  \definition{v.}{plantar; cultivar | configurar; manter; estabelecer}
\end{EntryWithPhonetic}

\begin{EntryWithPhonetic}{树立}{shu4li4}{9,5}{⽊,⽴}[HSK 7-9]
  \definition{v.}{estabelecer; configurar; possibilitar a formação e o estabelecimento de algo bom e com impacto positivo, um conceito abstrato}
  \synonymref{成立}{cheng2li4}
  \synonymref{创办}{chuang4ban4}
  \synonymref{创建}{chuang4jian4}
  \synonymref{创立}{chuang4li4}
  \synonymref{建立}{jian4li4}
  \synonymref{建设}{jian4she4}
  \synonymref{确立}{que4li4}
  \synonymref{设立}{she4li4}
  \synonymref{设置}{she4zhi4}
  \synonymref{塑造}{su4zao4}
  \antonymref{倒闭}{dao3bi4}
\end{EntryWithPhonetic}

\begin{EntryWithPhonetic}{树林}{shu4lin2}{9,8}{⽊,⽊}[HSK 4]
  \definition[片,座]{s.}{bosque; muitas árvores que crescem em fragmentos, menores que as florestas}
\end{EntryWithPhonetic}

\begin{EntryWithPhonetic}{树莓}{shu4mei2}{9,10}{⽊,⾋}
  \definition{s.}{framboesa}
\end{EntryWithPhonetic}

\begin{EntryWithPhonetic}{树木}{shu4mu4}{9,4}{⽊,⽊}[HSK 7-9]
  \definition[棵,株]{s.}{árvore}
  \synonymref{树林}{shu4lin2}
  \synonymref{植物}{zhi2wu4}
\end{EntryWithPhonetic}

\begin{EntryWithPhonetic}{树梢}{shu4shao1}{9,11}{⽊,⽊}[HSK 7-9]
  \definition[根]{s.}{ponta de uma árvore; copa da árvore}
  \seealsoref{树梢儿}{shu4shao1r5}
  \synonymref{树枝}{shu4zhi1}
\end{EntryWithPhonetic}

\begin{EntryWithPhonetic}{树梢儿}{shu4shao1r5}{9,11,2}{⽊,⽊,⼉}
  \definition[根]{s.}{ponta de uma árvore; copa da árvore}
\end{EntryWithPhonetic}

\begin{EntryWithPhonetic}{树叶}{shu4ye4}{9,5}{⽊,⼝}[HSK 4]
  \definition[片,枚,堆]{s.}{folha; folhagem}
\end{EntryWithPhonetic}

\begin{EntryWithPhonetic}{树阴儿}{shu4yin1r5}{9,6,2}{⽊,⾩,⼉}
  \definition{s.}{sombra da árvore}
\end{EntryWithPhonetic}

\begin{EntryWithPhonetic}{树荫儿}{shu4yin1r5}{9,9,2}{⽊,⾋,⼉}
  \definition{s.}{sombra}
\end{EntryWithPhonetic}

\begin{EntryWithPhonetic}{树荫}{shu4yin4}{9,9}{⽊,⾋}[HSK 7-9]
  \definition[片]{s.}{sombra (de uma árvore)}
  \seealsoref{树阴儿}{shu4yin1r5}
  \seealsoref{树荫儿}{shu4yin1r5}
\end{EntryWithPhonetic}

\begin{EntryWithPhonetic}{树枝}{shu4zhi1}{9,8}{⽊,⽊}[HSK 7-9]
  \definition[根]{s.}{galho; ramo}
  \synonymref{树梢}{shu4shao1}
\end{EntryWithPhonetic}

%%%%%%%%%% 竖 %%%%%%%%%%
\subsection*{竖}\addcontentsline{loh}{figure}{竖 \dpy{shu4}}

\begin{EntryWithPhonetic}{竖}{shu4}{9}{⽴}[HSK 7-9]
  \definition*{s.}{Sobrenome: Shu}
  \definition{adj.}{vertical; ereto; perpendicular ao solo}
  \definition{s.}{traço vertical (em caracteres chineses) | empregados domésticos; jovens criados}
  \definition{v.}{colocar em pé; erguer; ficar de pé; colocar o objeto perpendicular ao solo}
  \antonymref{横}{heng2}
\end{EntryWithPhonetic}

\begin{EntryWithPhonetic}{竖向}{shu4xiang4}{9,6}{⽴,⼝}
  \definition{adj.}{vertical}
\end{EntryWithPhonetic}

%%%%%%%%%% 庶 %%%%%%%%%%
\subsection*{庶}\addcontentsline{loh}{figure}{庶 \dpy{shu4}}

\begin{EntryWithPhonetic}{庶}{shu4}{11}{⼴}
  \definition*{s.}{Sobrenome: Shu}
  \definition{adj.}{multitudinário; numeroso}
  \definition{conj.}{para que; de modo a}
  \definition{s.}{da ou pela concubina (diferentemente da esposa legal); no sistema patriarcal, refere"-se ao ramo lateral da família}
\end{EntryWithPhonetic}

\begin{EntryWithPhonetic}{庶民}{shu4min2}{11,5}{⼴,⽒}
  \definition{s.}{(antigo) pessoas comuns | (antigo) plebeu; plebeus | (antigo) a multidão de pessoas comuns (na literatura erudita)}
\end{EntryWithPhonetic}

%%%%%%%%%% 数 %%%%%%%%%%
\subsection*{数}\addcontentsline{loh}{figure}{数 \dpy{shu4}}

\begin{EntryWithPhonetic}{数}{shu4}{13}{⽁}
  \definition{num.}{vários; alguns}
  \definition{s.}{número; cifra; figura | número (conceito matemático básico que representa a quantidade de coisas) | número; indica a quantidade de coisas a que se referem os substantivos ou pronomes | destino; sorte}
  \seeref{shu3}
  \seeref{shuo4}
\end{EntryWithPhonetic}

\begin{EntryWithPhonetic}{数额}{shu4'e2}{13,15}{⽁,⾴}[HSK 7-9]
  \definition{s.}{cota; número; quantidade; um certo número}
\end{EntryWithPhonetic}

\begin{EntryWithPhonetic}{数据}{shu4ju4}{13,11}{⽁,⼿}[HSK 4]
  \definition[组,个,条]{s.}{dados; valores com base nos quais são realizadas estatísticas, cálculos, pesquisas científicas ou projetos técnicos}
\end{EntryWithPhonetic}

\begin{EntryWithPhonetic}{数据库}{shu4ju4ku4}{13,11,7}{⽁,⼿,⼴}[HSK 7-9]
  \definition{s.}{banco de dados}
\end{EntryWithPhonetic}

\begin{EntryWithPhonetic}{数量}{shu4liang4}{13,12}{⽁,⾥}[HSK 3]
  \definition[个,种]{s.}{quantidade; quantum; quantia; magnitude; número}
\end{EntryWithPhonetic}

\begin{EntryWithPhonetic}{数码}{shu4ma3}{13,8}{⽁,⽯}[HSK 4]
  \definition{s.}{dígito; numeral; algarismo | número; quantidade (usado principalmente na linguagem falada)}
  \definition{v.}{digitalizar}
\end{EntryWithPhonetic}

\begin{EntryWithPhonetic}{数目}{shu4mu4}{13,5}{⽁,⽬}[HSK 5]
  \definition{s.}{número; quantidade; quantidade de algo expressa em uma determinada medida padrão (como unidades de medida, etc.)}
\end{EntryWithPhonetic}

\begin{EntryWithPhonetic}{数学}{shu4xue2}{13,8}{⽁,⼦}
  \definition{s.}{matemática; a ciência que estuda as formas espaciais e as relações quantitativas do mundo real, incluindo matemática elementar e matemática superior}
\end{EntryWithPhonetic}

\begin{EntryWithPhonetic}{数字}{shu4zi4}{13,6}{⽁,⼦}[HSK 2]
  \definition{adj.}{digital; usando tecnologia digital}
  \definition[个,串]{s.}{dígito; número; um caractere que representa um número | numeral; símbolos que representam números, como algarismos arábicos, algarismos romanos, etc. | quantidade; montante}
\end{EntryWithPhonetic}

%%%%%%%%%% 刷 %%%%%%%%%%
\subsection*{刷}\addcontentsline{loh}{figure}{刷 \dpy{shua1}}

\begin{EntryWithPhonetic}{刷}{shua1}{8}{⼑}[HSK 4]
  \definition{s.}{escova; pincel | (onomatopéia) farfalhar; descreve o som de uma passagem rápida}
  \definition{v.}{escovar; esfregar; remover com uma escova | borrar; colar; aplicar com um pincel | eliminar; remover; limpar | rolar; navegar; visualizar grandes quantidades de informações muito rapidamente em um curto período de tempo online ou em dispositivos móveis | deslizar (passar o cartão magnético)}
  \seeref{shua4}
\end{EntryWithPhonetic}

\begin{EntryWithPhonetic}{刷新}{shua1xin1}{8,13}{⼑,⽄}[HSK 7-9]
  \definition{v.}{renovar; reformar; atualizar; criar novo | atualizar (um site, uma página da \emph{web}) | quebrar (um recorde); essa metáfora descreve a substituição de conquistas existentes por novas e melhores}
  \synonymref{改革}{gai3ge2}
  \synonymref{改进}{gai3jin4}
  \synonymref{改良}{gai3liang2}
  \synonymref{改善}{gai3shan4}
  \synonymref{改正}{gai3zheng4}
  \synonymref{革新}{ge2xin1}
\end{EntryWithPhonetic}

\begin{EntryWithPhonetic}{刷牙}{shua1/ya2}{8,4}{⼑,⽛}[HSK 4]
  \definition{v.+compl.}{escovar os dentes}
\end{EntryWithPhonetic}

\begin{EntryWithPhonetic}{刷子}{shua1zi5}{8,3}{⼑,⼦}[HSK 4]
  \definition[把,个]{s.}{escova; escovão; utensílio feito de lã, fio de plástico, fio de metal, etc., para remover sujeira ou aplicar óleo de unção, etc., geralmente longo ou oval, alguns com alças}
\end{EntryWithPhonetic}

%%%%%%%%%% 耍 %%%%%%%%%%
\subsection*{耍}\addcontentsline{loh}{figure}{耍 \dpy{shua3}}

\begin{EntryWithPhonetic}{耍}{shua3}{9}{⽽}[HSK 7-9]
  \definition*{s.}{Sobrenome: Shua}
  \definition{v.}{brincar (com truques); se divertir | brincar com; agitar; provocar}
\end{EntryWithPhonetic}

\begin{EntryWithPhonetic}{耍赖}{shua3lai4}{9,13}{⽽,⾙}[HSK 7-9]
  \definition{v.}{agir sem vergonha; trapacear; agir desonestamente; comportar"-se de forma teimosa; usar métodos inescrupulosos}
\end{EntryWithPhonetic}

%%%%%%%%%% 刷 %%%%%%%%%%
\subsection*{刷}\addcontentsline{loh}{figure}{刷 \dpy{shua4}}

\begin{EntryWithPhonetic}{刷}{shua4}{8}{⼑}
  \definition{adj.}{pálido ou branco-azulado}
  \definition{adv.}{bastante; completamente; extremamente; descreve movimentos ágeis}
  \seeref{shua1}
\end{EntryWithPhonetic}

%%%%%%%%%% 衰 %%%%%%%%%%
\subsection*{衰}\addcontentsline{loh}{figure}{衰 \dpy{shuai1}}

\begin{EntryWithPhonetic}{衰}{shuai1}{10}{⾐}
  \definition{s.}{declínio; definhamento}
  \definition{v.}{diminuir; declinar}
  \synonymref{败}{bai4}
  \antonymref{盛}{sheng4}
  \antonymref{兴}{xing1}
\end{EntryWithPhonetic}

\begin{EntryWithPhonetic}{衰减}{shuai1jian3}{10,11}{⾐,⼎}[HSK 7-9]
  \definition{v.}{enfraquecer; falhar; diminuir | Eletrônica: atenuar; reduzir}
  \synonymref{放大}{fang4/da4}
  \synonymref{减弱}{jian3ruo4}
  \synonymref{减少}{jian3shao3}
  \synonymref{增强}{zeng1qiang2}
\end{EntryWithPhonetic}

\begin{EntryWithPhonetic}{衰竭}{shuai1jie2}{10,14}{⾐,⽴}[HSK 7-9]
  \definition{s.}{Medicina: exaustão; prostração}
  \definition{v.}{entrar em colapso; falhar; chegar à exaustão; prostrar"-se}
  \antonymref{充沛}{chong1pei4}
\end{EntryWithPhonetic}

\begin{EntryWithPhonetic}{衰老}{shuai1lao3}{10,6}{⾐,⽼}[HSK 7-9]
  \definition{adj.}{senil; decrépito; senescente; velho e fraco; velhice e declínio da saúde física e mental}
\end{EntryWithPhonetic}

\begin{EntryWithPhonetic}{衰弱}{shuai1ruo4}{10,10}{⾐,⼸}[HSK 7-9]
  \definition{adj.}{fraco; débil; (o corpo) perdeu sua energia e função plenas | frágil; fraco; as coisas mudam de fortes para fracas}
  \synonymref{薄弱}{bo2ruo4}
  \synonymref{脆弱}{cui4ruo4}
  \synonymref{腐败}{fu3bai4}
  \synonymref{让步}{rang4/bu4}
  \synonymref{软弱}{ruan3ruo4}
  \synonymref{失败}{shi1bai4}
  \synonymref{衰老}{shuai1lao3}
  \antonymref{昌盛}{chang1sheng4}
  \antonymref{健康}{jian4kang1}
  \antonymref{健壮}{jian4zhuang4}
  \antonymref{强壮}{qiang2zhuang4}
\end{EntryWithPhonetic}

\begin{EntryWithPhonetic}{衰退}{shuai1tui4}{10,9}{⾐,⾡}[HSK 7-9]
  \definition{v.}{falhar; declinar; (físico, mental, força de vontade, habilidades, etc.) tender ao declínio; (a situação política e econômica de um país) estar em declínio}
  \synonymref{没落}{mo4luo4}
  \synonymref{衰弱}{shuai1ruo4}
  \antonymref{旺盛}{wang4sheng4}
\end{EntryWithPhonetic}

%%%%%%%%%% 摔 %%%%%%%%%%
\subsection*{摔}\addcontentsline{loh}{figure}{摔 \dpy{shuai1}}

\begin{EntryWithPhonetic}{摔}{shuai1}{14}{⼿}[HSK 5]
  \definition{v.}{cair; tropeçar; perder o equilíbrio | mergulhar; precipitar-se; cair de uma altura elevada | quebrar; fazer cair e quebrar | lançar; atirar; arremessar; joguar coisas com força e para baixo | bater; golpear; bater com força para que o que está grudado cair}
\end{EntryWithPhonetic}

\begin{EntryWithPhonetic}{摔倒}{shuai1dao3}{14,10}{⼿,⼈}[HSK 5]
  \definition{v.}{cair; tropeçar; perder o equilíbrio e cair}
\end{EntryWithPhonetic}

\begin{EntryWithPhonetic}{摔跤}{shuai1/jiao1}{14,13}{⼿,⾜}[HSK 7-9]
  \definition[场,次]{s.}{luta livre; um esporte em que duas pessoas lutam corpo a corpo, usando diversas técnicas, habilidades e métodos para derrubar o oponente no chão, de acordo com certas regras}
  \definition{v.}{tropeçar; cair; tropeçar e cair}
  \synonymref{摔倒}{shuai1dao3}
\end{EntryWithPhonetic}

%%%%%%%%%% 甩 %%%%%%%%%%
\subsection*{甩}\addcontentsline{loh}{figure}{甩 \dpy{shuai3}}

\begin{EntryWithPhonetic}{甩}{shuai3}{5}{⽤}[HSK 7-9]
  \definition{v.}{balançar; mover"-se para trás e para a frente | atirar; lançar; arremessar | descartar; abandonar alguém}
\end{EntryWithPhonetic}

%%%%%%%%%% 帅 %%%%%%%%%%
\subsection*{帅}\addcontentsline{loh}{figure}{帅 \dpy{shuai4}}

\begin{EntryWithPhonetic}{帅}{shuai4}{5}{⼱}[HSK 4]
  \definition*{s.}{Sobrenome: Shuai}
  \definition{adj.}{bonito; arrojado; elegante; inteligente}
  \definition{interj.}{``Legal!''}
  \definition[位,名,个,些]{s.}{comandante em chefe; o mais alto comandante do exército | comandante em chefe, a peça principal no xadrez chinês}
\end{EntryWithPhonetic}

\begin{EntryWithPhonetic}{帅哥}{shuai4ge1}{5,10}{⼱,⼝}[HSK 4]
  \definition[个,位,名,些]{s.}{rapaz bonito; um garoto que é bonito e atraente na aparência}
\end{EntryWithPhonetic}

%%%%%%%%%% 率 %%%%%%%%%%
\subsection*{率}\addcontentsline{loh}{figure}{率 \dpy{shuai4}}

\begin{EntryWithPhonetic}{率}{shuai4}{11}{⽞}[HSK 7-9]
  \definition*{s.}{Sobrenome: Shuai}
  \definition{adj.}{precipitado; não cuidadoso; não cauteloso | franco; direto | elegante; bonito; o mesmo que 帅}
  \definition{adv.}{geralmente; expressa uma estimativa incerta, equivalente a 大约 e 大抵}
  \definition{s.}{modelo; exemplo}
  \definition{v.}{liderar; comandar | obedecer; seguir}
  \seeref{lv4}
  \seealsoref{大抵}{da4di3}
  \seealsoref{大约}{da4yue1}
  \seealsoref{帅}{shuai4}
\end{EntryWithPhonetic}

\begin{EntryWithPhonetic}{率领}{shuai4ling3}{11,11}{⽞,⾴}[HSK 5]
  \definition{v.}{liderar (equipe ou grupo); chefiar; comandar}
\end{EntryWithPhonetic}

\begin{EntryWithPhonetic}{率先}{shuai4xian1}{11,6}{⽞,⼉}[HSK 4]
  \definition{v.}{tomar a iniciativa de fazer algo; ser o primeiro a fazer algo; assumir a liderança}
\end{EntryWithPhonetic}

%%%%%%%%%% 拴 %%%%%%%%%%
\subsection*{拴}\addcontentsline{loh}{figure}{拴 \dpy{shuan1}}

\begin{EntryWithPhonetic}{拴}{shuan1}{9}{⼿}[HSK 7-9]
  \definition{v.}{amarrar; prender; enrolar uma corda ou objeto similar em volta do objeto e dar um nó | estar preso a algo; estar atolado; restringir e privar as pessoas de sua liberdade}
\end{EntryWithPhonetic}

%%%%%%%%%% 涮 %%%%%%%%%%
\subsection*{涮}\addcontentsline{loh}{figure}{涮 \dpy{shuan4}}

\begin{EntryWithPhonetic}{涮}{shuan4}{11}{⽔}[HSK 7-9]
  \definition{v.}{lavar; balançar a mão ou algo parecido na água | enxaguar; colcar água dentro do recipiente e agitar para enxaguar | mergulhar; escaldar as fatias de carne rapidamente em água fervente, depois retirar e mergulhar no molho antes de servir | enganar; ludibriar; perturbar}
\end{EntryWithPhonetic}

%%%%%%%%%% 双 %%%%%%%%%%
\subsection*{双}\addcontentsline{loh}{figure}{双 \dpy{shuang1}}

\begin{EntryWithPhonetic}{双}{shuang1}{4}{⼜}[HSK 3]
  \definition*{s.}{Sobrenome: Shuang}
  \definition{adj.}{dois; gêmeo; par; dual | números pares | duplo; dobro}
  \definition{clas.}{usado para certos membros, órgãos ou coisas pareadas que são bilateralmente simétricas, por exemplo, sapatos, meias, pauzinhos, etc.}
  \antonymref{单}{dan1}
\end{EntryWithPhonetic}

\begin{EntryWithPhonetic}{双胞胎}{shuang1bao1tai1}{4,9,9}{⼜,⾁,⾁}[HSK 7-9]
  \definition[对]{s.}{gêmeos; dois bebês na mesma gravidez; duas pessoas nascidas na mesma gravidez}
\end{EntryWithPhonetic}

\begin{EntryWithPhonetic}{双边}{shuang1bian1}{4,5}{⼜,⾡}[HSK 7-9]
  \definition{adj.}{bilateral; participação de ambas as partes; especificamente, participação de dois países | duplicado}
\end{EntryWithPhonetic}

\begin{EntryWithPhonetic}{双层床}{shuang1ceng2chuang2}{4,7,7}{⼜,⼫,⼴}
  \definition{s.}{beliche}
\end{EntryWithPhonetic}

\begin{EntryWithPhonetic}{双重}{shuang1chong2}{4,9}{⼜,⾥}[HSK 7-9]
  \definition{adj.}{dobro; dual; duplo; duas camadas; dois aspectos (frequentemente usado para conceitos abstratos)}
\end{EntryWithPhonetic}

\begin{EntryWithPhonetic}{双打}{shuang1da3}{4,5}{⼜,⼿}[HSK 6]
  \definition[场,局,次]{s.}{duplas (em esportes)}
\end{EntryWithPhonetic}

\begin{EntryWithPhonetic}{双方}{shuang1fang1}{4,4}{⼜,⽅}[HSK 3]
  \definition{s.}{ambos os lados; as duas partes; duas pessoas ou dois grupos frente a frente em um determinado relacionamento ou situação}
\end{EntryWithPhonetic}

\begin{EntryWithPhonetic}{双方同意}{shuang1fang1tong2yi4}{4,4,6,13}{⼜,⽅,⼝,⼼}
  \definition{s.}{acordo bilateral}
\end{EntryWithPhonetic}

\begin{EntryWithPhonetic}{双手}{shuang1shou3}{4,4}{⼜,⼿}[HSK 5]
  \definition{s.}{com as duas mãos; ambas as mãos; par de mãos}
\end{EntryWithPhonetic}

\begin{EntryWithPhonetic}{双向}{shuang1xiang4}{4,6}{⼜,⼝}[HSK 7-9]
  \definition{s.}{mão"-dupla; bidirecional (oposto de 单向) | interativo}
  \seealsoref{单向}{dan1xiang4}
\end{EntryWithPhonetic}

\begin{EntryWithPhonetic}{双赢}{shuang1ying2}{4,17}{⼜,⾙}[HSK 7-9]
  \definition{v.}{ganhar\-ganhar; alcançar um resultado em uma situação em que todos ganham; ser mutuamente vantajoso}
  \synonymref{互利}{hu4li4}
\end{EntryWithPhonetic}

%%%%%%%%%% 霜 %%%%%%%%%%
\subsection*{霜}\addcontentsline{loh}{figure}{霜 \dpy{shuang1}}

\begin{EntryWithPhonetic}{霜}{shuang1}{17}{⾬}[HSK 7-9]
  \definition*{s.}{Sobrenome: Shuang}
  \definition{adj.}{branco; grisalho}
  \definition{s.}{geada | pó branco ou creme espalhado por uma superfície | glacê | creme para os olhos}
\end{EntryWithPhonetic}

%%%%%%%%%% 爽 %%%%%%%%%%
\subsection*{爽}\addcontentsline{loh}{figure}{爽 \dpy{shuang3}}

\begin{EntryWithPhonetic}{爽}{shuang3}{11}{⽘}[HSK 6]
  \definition{adj.}{claro; nítido; brilhante | franco; de coração aberto; direto | relaxado; confortável}
  \definition{v.}{desviar; afastar | tornar confortável; ficar confortável}
\end{EntryWithPhonetic}

\begin{EntryWithPhonetic}{爽快}{shuang3kuai5}{11,7}{⽘,⼼}[HSK 7-9]
  \definition{adj.}{relaxado; confortável; devido a fatores ambientais e climáticos, sinto"-me relaxado e confortável tanto física quanto mentalmente | direto; franco; isso descreve alguém que consegue tomar decisões rapidamente, seja falando ou agindo, sem hesitação ou indecisão}
  \synonymref{干脆}{gan1cui4}
  \synonymref{坦率}{tan3shuai4}
  \antonymref{迟疑}{chi2yi2}
  \antonymref{犹豫}{you2yu4}
\end{EntryWithPhonetic}

%%%%%%%%%% 谁 %%%%%%%%%%
\subsection*{谁}\addcontentsline{loh}{figure}{谁 \dpy{shui2}}

\begin{EntryWithPhonetic}{谁}{shui2}{10}{⾔}
  \seeref{shei2}
\end{EntryWithPhonetic}

%%%%%%%%%% 水 %%%%%%%%%%
\subsection*{水}\addcontentsline{loh}{figure}{水 \dpy{shui3}}

\begin{EntryWithPhonetic}{水}{shui3}{4}{⽔}[HSK 1][Kangxi 85]
  \definition*{s.}{Etnia Shui, que vive principalmente em Guizhou | Sobrenome: Shui}
  \definition{adj.}{de má qualidade; mal feito; de baixa qualidade e conteúdo}
  \definition{clas.}{usado para número de lavagens}
  \definition[条,杯]{s.}{água | rio | termo geral para rios, lagos, mares, etc.; água | corrente; fluxo de água | um líquido; suco ralo | teor de prata nas moedas | encargos adicionais ou receitas | água, um dos cinco elementos}
\end{EntryWithPhonetic}

\begin{EntryWithPhonetic}{水边}{shui3bian1}{4,5}{⽔,⾡}
  \definition{s.}{beira d'água | beira"-mar | costa (de mar, lago ou rio)}
\end{EntryWithPhonetic}

\begin{EntryWithPhonetic}{水波}{shui3bo1}{4,8}{⽔,⽔}
  \definition{s.}{ondulação (na água) | onda}
\end{EntryWithPhonetic}

\begin{EntryWithPhonetic}{水槽}{shui3cao2}{4,15}{⽔,⽊}[HSK 7-9]
  \definition{s.}{sarjeta; canal; canal de água; conduto de água | pia; calha}
\end{EntryWithPhonetic}

\begin{EntryWithPhonetic}{水产品}{shui3chan3pin3}{4,6,9}{⽔,⼇,⼝}[HSK 5]
  \definition{s.}{produto aquático (peixes, camarões, etc.)}
\end{EntryWithPhonetic}

\begin{EntryWithPhonetic}{水稻}{shui3dao4}{4,15}{⽔,⽲}[HSK 7-9]
  \definition[亩,棵,株]{s.}{arroz; arrozal}
\end{EntryWithPhonetic}

\begin{EntryWithPhonetic}{水分}{shui3fen4}{4,4}{⽔,⼑}[HSK 5]
  \definition{s.}{teor de umidade; água contida em um objeto | exagero; metáfora de algo falso}
\end{EntryWithPhonetic}

\begin{EntryWithPhonetic}{水管}{shui3guan3}{4,14}{⽔,⽵}[HSK 7-9]
  \definition{s.}{cano de água; mangueira de água; tubulações para transporte de água}
\end{EntryWithPhonetic}

\begin{EntryWithPhonetic}{水果}{shui3guo3}{4,8}{⽔,⽊}[HSK 1]
  \definition[个]{s.}{fruta; um nome genérico para frutas com alto teor de água que podem ser consumidas, como peras, pêssegos, maçãs, etc.}
\end{EntryWithPhonetic}

\begin{EntryWithPhonetic}{水壶}{shui3hu2}{4,10}{⽔,⼠}[HSK 7-9]
  \definition{s.}{chaleira; jarro de água; recipientes para armazenar água}
\end{EntryWithPhonetic}

\begin{EntryWithPhonetic}{水货}{shui3huo4}{4,8}{⽔,⾙}[HSK 7-9]
  \definition{s.}{mercadorias contrabandeadas; mercadorias ilegais; importações de origem duvidosa | produtos de má qualidade | mercadorias não autorizadas}
  \synonymref{山寨}{shan1zhai4}
\end{EntryWithPhonetic}

\begin{EntryWithPhonetic}{水饺}{shui3jiao3}{4,9}{⽔,⾷}
  \definition{s.}{\emph{dumplings} | pastéis chineses cozidos}
\end{EntryWithPhonetic}

\begin{EntryWithPhonetic}{水晶}{shui3jing1}{4,12}{⽔,⽇}[HSK 7-9]
  \definition{s.}{cristal; cristal de rocha; cristais de quartzo incolores e transparentes}
\end{EntryWithPhonetic}

\begin{EntryWithPhonetic}{水库}{shui3ku4}{4,7}{⽔,⼴}[HSK 5]
  \definition[座]{s.}{reservatório; lago artificial construído pelo homem, que utiliza barragens e outras estruturas para represar a água e regular o fluxo, podendo ser utilizado para armazenamento de água, geração de energia e piscicultura, entre outros fins}
\end{EntryWithPhonetic}

\begin{EntryWithPhonetic}{水利}{shui3li4}{4,7}{⽔,⼑}[HSK 7-9]
  \definition{s.}{conservação de água; utilização dos recursos hídricos | obras de irrigação; projeto de conservação de água}
\end{EntryWithPhonetic}

\begin{EntryWithPhonetic}{水灵}{shui3ling2}{4,7}{⽔,⽕}
  \definition{adj.}{cheio de vida (sobre uma pessoa, etc.) | úmido e brilhante (sobre os olhos) | fresco (sobre frutas, etc.) | brilhante | aparência saudável}
\end{EntryWithPhonetic}

\begin{EntryWithPhonetic}{水灵灵}{shui3ling2ling2}{4,7,7}{⽔,⽕,⽕}[HSK 7-9]
  \definition{adj.}{fresco e suculento; rechonchudo e vibrante. aquoso}
\end{EntryWithPhonetic}

\begin{EntryWithPhonetic}{水龙头}{shui3long2tou2}{4,5,5}{⽔,⿓,⼤}[HSK 7-9]
  \definition[个]{s.}{torneira; torneira de água}
\end{EntryWithPhonetic}

\begin{EntryWithPhonetic}{水路}{shui3lu4}{4,13}{⽔,⾜}
  \definition{s.}{hidrovia}
\end{EntryWithPhonetic}

\begin{EntryWithPhonetic}{水落石出}{shui3luo4shi2chu1}{4,12,5,5}{⽔,⾋,⽯,⼐}[HSK 7-9]
  \definition{expr.}{``A verdade virá à tona.''; quando a água baixar, as rochas emergirão; as dúvidas se dissiparão quando os fatos forem conhecidos; chegar ao fundo de algo; chegará o momento de revelar toda a verdade; tudo virá à tona; tudo fica à mostra}
\end{EntryWithPhonetic}

\begin{EntryWithPhonetic}{水面}{shui3mian4}{4,9}{⽔,⾯}[HSK 7-9]
  \definition{s.}{superfície da água | área de um corpo d'água}
  \synonymref{海面}{hai3mian4}
\end{EntryWithPhonetic}

\begin{EntryWithPhonetic}{水泥}{shui3ni2}{4,8}{⽔,⽔}[HSK 6]
  \definition[袋,层]{s.}{cimento; um tipo de material mineral em pó que pode endurecer gradualmente no ar e na água após a mistura com água}
\end{EntryWithPhonetic}

\begin{EntryWithPhonetic}{水培}{shui3pei2}{4,11}{⽔,⼟}
  \definition{v.}{cultivar plantas hidroponicamente}
\end{EntryWithPhonetic}

\begin{EntryWithPhonetic}{水平}{shui3ping2}{4,5}{⽔,⼲}[HSK 2]
  \definition{adj.}{horizontal; nivelado; paralelo à superfície da água}
  \definition{s.}{padrão; nível; o nível alcançado em determinado aspecto}
\end{EntryWithPhonetic}

\begin{EntryWithPhonetic}{水平尺}{shui3ping2chi3}{4,5,4}{⽔,⼲,⼫}
  \definition{s.}{nível espiritual}
\end{EntryWithPhonetic}

\begin{EntryWithPhonetic}{水平度}{shui3ping2 du4}{4,5,9}{⽔,⼲,⼴}
  \definition{s.}{nivelamento}
\end{EntryWithPhonetic}

\begin{EntryWithPhonetic}{水平面}{shui3ping2mian4}{4,5,9}{⽔,⼲,⾯}
  \definition{s.}{superfície nivelada; nível | plano horizontal; nível da água; superfície horizontal | nível da água}
\end{EntryWithPhonetic}

\begin{EntryWithPhonetic}{水平视差}{shui3ping2 shi4cha1}{4,5,8,9}{⽔,⼲,⾒,⼯}
  \definition{s.}{paralaxe horizontal}
\end{EntryWithPhonetic}

\begin{EntryWithPhonetic}{水平仪}{shui3ping2yi2}{4,5,5}{⽔,⼲,⼈}
  \definition{s.}{nível (dispositivo para determinar horizontal) | nível espiritual | nível de topógrafo}
\end{EntryWithPhonetic}

\begin{EntryWithPhonetic}{水平以下}{shui3ping2 yi3xia4}{4,5,4,3}{⽔,⼲,⼈,⼀}
  \definition{s.}{sub-nível}
\end{EntryWithPhonetic}

\begin{EntryWithPhonetic}{水平轴}{shui3ping2zhou2}{4,5,9}{⽔,⼲,⾞}
  \definition{s.}{eixo horizontal}
\end{EntryWithPhonetic}

\begin{EntryWithPhonetic}{水瓶}{shui3 ping2}{4,10}{⽔,⽡}
  \definition{s.}{garrada de água}
\end{EntryWithPhonetic}

\begin{EntryWithPhonetic}{水手}{shui3shou3}{4,4}{⽔,⼿}[HSK 7-9]
  \definition{s.}{marinheiro}
\end{EntryWithPhonetic}

\begin{EntryWithPhonetic}{水豚}{shui3tun2}{4,11}{⽔,⾗}
  \definition{s.}{capivara}
\end{EntryWithPhonetic}

\begin{EntryWithPhonetic}{水温}{shui3wen1}{4,12}{⽔,⽔}[HSK 7-9]
  \definition{s.}{temperatura da água}
\end{EntryWithPhonetic}

\begin{EntryWithPhonetic}{水污染}{shui3wu1ran3}{4,6,9}{⽔,⽔,⽊}
  \definition{s.}{poluição da água}
\end{EntryWithPhonetic}

\begin{EntryWithPhonetic}{水域}{shui3yu4}{4,11}{⽔,⼟}[HSK 7-9]
  \definition{s.}{águas; área aquática; corpo d'água | águas territoriais; uma determinada área (da superfície até o fundo) do mar, rio ou lago}
\end{EntryWithPhonetic}

\begin{EntryWithPhonetic}{水源}{shui3yuan2}{4,13}{⽔,⽔}[HSK 7-9]
  \definition{s.}{fonte de água; fontes de água para uso doméstico, industrial e agrícola | nascente do rio; o lugar onde o rio começa}
\end{EntryWithPhonetic}

\begin{EntryWithPhonetic}{水灾}{shui3zai1}{4,7}{⽔,⽕}[HSK 5]
  \definition[场,次]{s.}{inundação; desastres causados por excesso de chuvas, entre outros motivos}
\end{EntryWithPhonetic}

\begin{EntryWithPhonetic}{水涨船高}{shui3zhang3-chuan2gao1}{4,10,11,10}{⽔,⽔,⾈,⾼}[HSK 7-9]
  \definition{expr.}{``Com a maré alta, todos os barcos sobem junto.''; quando o rio sobe, o barco flutua alto; a maré sobe, o barco flutua; figurativamente, mudar conforme a tendência geral; desenvolver de acordo com a situação}
\end{EntryWithPhonetic}

\begin{EntryWithPhonetic}{水准}{shui3zhun3}{4,10}{⽔,⼎}[HSK 7-9]
  \definition{s.}{calibres; nível (de realização etc.) | nível (topografia); planos horizontais de diferentes partes da Terra}
  \synonymref{程度}{cheng2du4}
  \synonymref{水平}{shui3ping2}
\end{EntryWithPhonetic}

%%%%%%%%%% 说 %%%%%%%%%%
\subsection*{说}\addcontentsline{loh}{figure}{说 \dpy{shui4}}

\begin{EntryWithPhonetic}{说}{shui4}{9}{⾔}
  \definition{v.}{persuadir}
  \seeref{shuo1}
\end{EntryWithPhonetic}

%%%%%%%%%% 税 %%%%%%%%%%
\subsection*{税}\addcontentsline{loh}{figure}{税 \dpy{shui4}}

\begin{EntryWithPhonetic}{税}{shui4}{12}{⽲}[HSK 6]
  \definition*{s.}{Sobrenome: Shui}
  \definition{s.}{imposto; taxa; tarifa}
\end{EntryWithPhonetic}

\begin{EntryWithPhonetic}{税收}{shui4shou1}{12,6}{⽲,⽁}[HSK 7-9]
  \definition{s.}{receita tributária; imposto; o Estado obtém receita através da cobrança de impostos de acordo com a lei}
\end{EntryWithPhonetic}

\begin{EntryWithPhonetic}{税务}{shui4wu4}{12,5}{⽲,⼒}[HSK 7-9]
  \definition{s.}{tributação; administração tributária | serviço de receita estadual | serviços de tributação}
\end{EntryWithPhonetic}

%%%%%%%%%% 睡 %%%%%%%%%%
\subsection*{睡}\addcontentsline{loh}{figure}{睡 \dpy{shui4}}

\begin{EntryWithPhonetic}{睡}{shui4}{13}{⽬}[HSK 1]
  \definition{v.}{dormir | deitar-se}
\end{EntryWithPhonetic}

\begin{EntryWithPhonetic}{睡袋}{shui4dai4}{13,11}{⽬,⾐}[HSK 7-9]
  \definition{s.}{saco de dormir}
\end{EntryWithPhonetic}

\begin{EntryWithPhonetic}{睡觉}{shui4/jiao4}{13,9}{⽬,⾒}[HSK 1]
  \definition{v.+compl.}{dormir; ir para a cama; entrar em estado de sono}
\end{EntryWithPhonetic}

\begin{EntryWithPhonetic}{睡懒觉}{shui4lan3jiao4}{13,16,9}{⽬,⼼,⾒}
  \definition{v.}{levantar"-se tarde | passar o tempo a dormir}
\end{EntryWithPhonetic}

\begin{EntryWithPhonetic}{睡眠}{shui4mian2}{13,10}{⽬,⽬}[HSK 5]
  \definition{s.}{sono; \emph{somnus}; sonolência}
\end{EntryWithPhonetic}

\begin{EntryWithPhonetic}{睡衣}{shui4yi1}{13,6}{⽬,⾐}
  \definition{s.}{pijamas | roupas de dormir}
\end{EntryWithPhonetic}

\begin{EntryWithPhonetic}{睡着}{shui4zhao2}{13,11}{⽬,⽬}[HSK 4]
  \definition{v.}{dormir; adormecer; cair no sono}
\end{EntryWithPhonetic}

%%%%%%%%%% 顺 %%%%%%%%%%
\subsection*{顺}\addcontentsline{loh}{figure}{顺 \dpy{shun4}}

\begin{EntryWithPhonetic}{顺}{shun4}{9}{⾴}[HSK 6]
  \definition{adj.}{(de escritos) legível; claro e bem escrito; organizado | favorável; harmonioso | favorável; bem"-sucedido}
  \definition{prep.}{conforme a conveniência de alguém | ao longo; a introdução da rota, situação ou oportunidade que a ação segue pode ser seguida por 着 | com a corrente; na mesma direção | com; na mesma direção que}
  \definition{v.}{organizar; colocar em ordem; tornar as coisas organizadas ou ordenadas | obedecer; ceder a; agir em submissão a | ser adequado; ser agradável}
  \seealsoref{着}{zhe5}
\end{EntryWithPhonetic}

\begin{EntryWithPhonetic}{顺便}{shun4bian4}{9,9}{⾴,⼈}[HSK 7-9]
  \definition{adv.}{convenientemente; de passagem; incidentalmente; a propósito; (a propósito) enquanto faz outra coisa (faça outra coisa)}
  \synonymref{趁机}{chen4ji1}
  \synonymref{附带}{fu4dai4}
  \synonymref{随时}{sui2shi2}
  \antonymref{特别}{te4bie2}
  \antonymref{特地}{te4di4}
  \antonymref{特意}{te4yi4}
  \antonymref{专门}{zhuan1men2}
\end{EntryWithPhonetic}

\begin{EntryWithPhonetic}{顺差}{shun4cha1}{9,9}{⾴,⼯}[HSK 7-9]
  \definition{s.}{um excedente; um saldo favorável; a balança comercial no comércio internacional refere"-se à diferença entre o valor total das exportações e o valor total das importações}
\end{EntryWithPhonetic}

\begin{EntryWithPhonetic}{顺畅}{shun4chang4}{9,8}{⾴,⽥}[HSK 7-9]
  \definition{adj.}{suave; desimpedido; suave e desobstruído}
  \synonymref{畅通}{chang4tong1}
  \synonymref{流畅}{liu2chang4}
  \synonymref{流利}{liu2li4}
  \synonymref{通顺}{tong1shun4}
  \antonymref{堵塞}{du3se4}
\end{EntryWithPhonetic}

\begin{EntryWithPhonetic}{顺从}{shun4cong2}{9,4}{⾴,⼈}[HSK 7-9]
  \definition{v.}{obedecer; cumprir com; agir de acordo com os desejos dos outros}
  \synonymref{服从}{fu2cong2}
  \synonymref{听从}{ting1cong2}
  \synonymref{投降}{tou2xiang2}
  \antonymref{抵制}{di3zhi4}
  \antonymref{对峙}{dui4zhi4}
  \antonymref{反抗}{fan3kang4}
  \antonymref{固执}{gu4zhi5}
  \antonymref{抗拒}{kang4ju4}
  \antonymref{克服}{ke4fu2}
  \antonymref{迁就}{qian1jiu4}
  \antonymref{挑衅}{tiao3xin4}
  \antonymref{违背}{wei2bei4}
  \antonymref{挣扎}{zheng1zha2}
\end{EntryWithPhonetic}

\begin{EntryWithPhonetic}{顺当}{shun4dang5}{9,6}{⾴,⼹}
  \definition{adv.}{suavemente}
\end{EntryWithPhonetic}

\begin{EntryWithPhonetic}{顺耳}{shun4'er3}{9,6}{⾴,⽿}
  \definition{adj.}{agradável ao ouvido}
\end{EntryWithPhonetic}

\begin{EntryWithPhonetic}{顺境}{shun4jing4}{9,14}{⾴,⼟}
  \definition{s.}{circunstâncias fáceis (ou favoráveis)}
  \antonymref{逆境}{ni4jing4}
\end{EntryWithPhonetic}

\begin{EntryWithPhonetic}{顺理成章}{shun4li3-cheng2zhang1}{9,11,6,11}{⾴,⽟,⼽,⾳}[HSK 7-9]
  \definition{expr.}{desenvolve"-se naturalmente; faz todo o sentido; descreve a escrita ou a execução de tarefas de forma clara e organizada}
  \synonymref{理所当然}{li3suo3dang1ran2}
  \synonymref{理直气壮}{li3zhi2-qi4zhuang4}
\end{EntryWithPhonetic}

\begin{EntryWithPhonetic}{顺利}{shun4li4}{9,7}{⾴,⼑}[HSK 2]
  \definition{adj.}{sem problemas; com sucesso; sem dificuldades; sem contratempos; sem obstáculos; sem obstáculos ou dificuldades significativas no desempenho das tarefas}
\end{EntryWithPhonetic}

\begin{EntryWithPhonetic}{顺路}{shun4lu4}{9,13}{⾴,⾜}[HSK 7-9]
  \definition{adj.}{ser uma rota direta; significa que a estrada é lisa e desobstruída; também se diz que é conveniente caminhar por ela, ou ``estar a caminho''}
  \definition{adv.}{a caminho; (a caminho) seguindo a rota que fizemos (para outro lugar)}
  \antonymref{迷路}{mi2/lu4}
\end{EntryWithPhonetic}

\begin{EntryWithPhonetic}{顺其自然}{shun4qi2zi4ran2}{9,8,6,12}{⾴,⼋,⾃,⽕}[HSK 7-9]
  \definition{expr.}{``Deixe a natureza seguir seu curso.''; de acordo com sua tendência natural; significa seguir o curso natural das coisas sem interferência ou coerção}
\end{EntryWithPhonetic}

\begin{EntryWithPhonetic}{顺势}{shun4shi4}{9,8}{⾴,⼒}[HSK 7-9]
  \definition{adv.}{aproveitar uma oportunidade (proporcionada por uma jogada imprudente do oponente); deixe"-se levar; aproveite a oportunidade | convenientemente; de passagem | conforme a conveniência de alguém; (fazer algo) sem se esforçar muito; aliás; por acaso}
  \synonymref{惯性}{guan4xing4}
\end{EntryWithPhonetic}

\begin{EntryWithPhonetic}{顺手}{shun4shou3}{9,4}{⾴,⼿}[HSK 7-9]
  \definition{adj.}{suave; sem dificuldade; tudo correu bem, sem obstáculos | prático; conveniente e fácil de usar}
  \definition{adv.}{suavemente; a propósito; aliás; juntamente com}
  \synonymref{顺利}{shun4li4}
  \synonymref{随手}{sui2shou3}
  \antonymref{棘手}{ji2shou3}
\end{EntryWithPhonetic}

\begin{EntryWithPhonetic}{顺水}{shun4shui3}{9,4}{⾴,⽔}
  \definition{v.}{ir com o fluxo}
\end{EntryWithPhonetic}

\begin{EntryWithPhonetic}{顺心}{shun4/xin1}{9,4}{⾴,⼼}[HSK 7-9]
  \definition{v.+compl.}{estar satisfeito; estar feliz}
  \synonymref{如意}{ru2/yi4}
  \synonymref{舒服}{shu1fu5}
  \synonymref{顺眼}{shun4yan3}
  \synonymref{通顺}{tong1shun4}
  \antonymref{别扭}{bie4niu5}
\end{EntryWithPhonetic}

\begin{EntryWithPhonetic}{顺序}{shun4xu4}{9,7}{⾴,⼴}[HSK 4]
  \definition{adv.}{por sua vez; na ordem correta; na devida ordem; na ordem adequada; na ordem apropriada}
  \definition[个]{s.}{ordem; sequência; sucessão; subsequência; sequência simples; ordem de prioridade}
\end{EntryWithPhonetic}

\begin{EntryWithPhonetic}{顺叙}{shun4xu4}{9,9}{⾴,⼜}
  \definition{s.}{narrativa cronológica}
\end{EntryWithPhonetic}

\begin{EntryWithPhonetic}{顺延}{shun4yan2}{9,6}{⾴,⼵}
  \definition{v.}{adiar | procrastinar}
\end{EntryWithPhonetic}

\begin{EntryWithPhonetic}{顺眼}{shun4yan3}{9,11}{⾴,⽬}
  \definition{adj.}{agradável aos olhos}
\end{EntryWithPhonetic}

\begin{EntryWithPhonetic}{顺应}{shun4ying4}{9,7}{⾴,⼴}[HSK 7-9]
  \definition{v.}{cumprir com; estar em conformidade com | ajustar; obedecer; adaptar}
  \synonymref{适合}{shi4he2}
  \synonymref{适应}{shi4ying4}
\end{EntryWithPhonetic}

\begin{EntryWithPhonetic}{顺着}{shun4zhe5}{9,11}{⾴,⽬}[HSK 7-9]
  \definition{v.}{deslocar-se ao longo de uma determinada rota ou direção | falar e agir de acordo com os desejos dos outros}
\end{EntryWithPhonetic}

\begin{EntryWithPhonetic}{顺嘴}{shun4zui3}{9,16}{⾴,⼝}
  \definition{v.}{deixar escapar (sem pensar) | ler suavemente (texto) | adequar-se  ao gosto (comida)}
\end{EntryWithPhonetic}

%%%%%%%%%% 舜 %%%%%%%%%%
\subsection*{舜}\addcontentsline{loh}{figure}{舜 \dpy{shun4}}

\begin{EntryWithPhonetic}{舜}{shun4}{12}{⾇}
  \definition*{s.}{Shun, o nome de um monarca lendário da China antiga | Sobrenome: Shun}
\end{EntryWithPhonetic}

%%%%%%%%%% 瞬 %%%%%%%%%%
\subsection*{瞬}\addcontentsline{loh}{figure}{瞬 \dpy{shun4}}

\begin{EntryWithPhonetic}{瞬}{shun4}{17}{⽬}
  \definition{s.}{piscadela; cintilação; num piscar de olhos}
  \definition{v.}{piscar; cintilar}
\end{EntryWithPhonetic}

\begin{EntryWithPhonetic}{瞬间}{shun4jian1}{17,7}{⽬,⾨}[HSK 7-9]
  \definition{s.}{momento; instante; minuto; num piscar de olhos; num abrir e fechar de olhos; em um tempo extremamente curto}
  \antonymref{长期}{chang2qi1}
  \antonymref{漫长}{man4chang2}
  \antonymref{永远}{yong3yuan3}
\end{EntryWithPhonetic}

%%%%%%%%%% 说 %%%%%%%%%%
\subsection*{说}\addcontentsline{loh}{figure}{说 \dpy{shuo1}}

\begin{EntryWithPhonetic}{说}{shuo1}{9}{⾔}[HSK 1]
  \definition{s.}{uma teoria (normalmente o último caractere, como em 日心说, teoria heliocêntrica); ensinamentos; doutrina}
  \definition{v.}{falar; conversar; dizer | explicar | repreender | atuar como casamenteiro | referir"-se a; indicar | criticar; aconselhar | fazer uma combinação; conciliar; mediar | discutir; falar sobre; conversar sobre | uma forma de expressão linguística da arte cênica}
  \seeref{shui4}
  \seealsoref{日心说}{ri4 xin1 shuo1}
\end{EntryWithPhonetic}

\begin{EntryWithPhonetic}{说白了}{shuo1bai2le5}{9,5,2}{⾔,⽩,⼅}[HSK 7-9]
  \definition{v.}{falar francamente; para ser franco; para ser honesto}[说白了,他就是不想帮忙。===Para ser franco, ele simplesmente não queria ajudar.]
\end{EntryWithPhonetic}

\begin{EntryWithPhonetic}{说不定}{shuo1bu5ding4}{9,4,8}{⾔,⼀,⼧}[HSK 4]
  \definition{adv.}{talvez; indica uma estimativa, possivelmente, provavelmente}
  \definition{v.}{não ter certeza; não estar certo; ser impreciso}
\end{EntryWithPhonetic}

\begin{EntryWithPhonetic}{说不上}{shuo1bu5shang4}{9,4,3}{⾔,⼀,⼀}[HSK 7-9]
  \definition{v.}{não posso dizer; não posso afirmar; não sei dizer; não consigo explicar | não vale a pena mencionar; não é digno de ser chamado; não é bom o suficiente para ser chamado}
\end{EntryWithPhonetic}

\begin{EntryWithPhonetic}{说到底}{shuo1dao4di3}{9,8,8}{⾔,⼑,⼴}[HSK 7-9]
  \definition{adv.}{em última análise; resumindo; no fundo}[偏差和片面性的产生, 说到底, 是主观认识脱离了客观实际。===Em última análise, os preconceitos e a parcialidade surgem da dissociação entre a compreensão subjetiva e a realidade objetiva.]
\end{EntryWithPhonetic}

\begin{EntryWithPhonetic}{说道}{shuo1dao4}{9,12}{⾔,⾡}
  \definition{v.}{dizer  (usado frequentemente em romances para introduzir diretamente as falas de um personagem, e ainda em uso hoje em dia)}[少先队员说道:``我们要向雷锋叔叔学习。''===Os Jovens Pioneiros disseram: ``Devemos aprender com o tio Lei Feng.'']
  \seeref{shuo1dao5}
\end{EntryWithPhonetic}

\begin{EntryWithPhonetic}{说道}{shuo1dao5}{9,12}{⾔,⾡}[HSK 7-9]
  \definition{s.}{aquilo que está por trás de algo; razão}
  \definition{v.}{dizer; contar | conversar; discutir}
  \seeref{shuo1dao4}
  \seealsoref{说道儿}{shuo1dao5r5}
\end{EntryWithPhonetic}

\begin{EntryWithPhonetic}{说道儿}{shuo1dao5r5}{9,12,2}{⾔,⾡,⼉}
  \definition{v.}{falar sobre algo}
\end{EntryWithPhonetic}

\begin{EntryWithPhonetic}{说法}{shuo1fa5}{9,8}{⾔,⽔}[HSK 5]
  \definition[种,个]{s.}{formulação; maneira de dizer uma coisa; formas de expressar opiniões | versão; argumento; declaração; opinião | explicação; acordo; palavras justas; razões ou fundamentos legítimos}
\end{EntryWithPhonetic}

\begin{EntryWithPhonetic}{说服}{shuo1/fu2}{9,8}{⾔,⽉}[HSK 4]
  \definition{v.+compl.}{persuadir; convencer; convencer a outra parte com palavras bem fundamentadas}
\end{EntryWithPhonetic}

\begin{EntryWithPhonetic}{说干就干}{shuo1 gan4 jiu4 gan4}{9,3,12,3}{⾔,⼲,⼪,⼲}[HSK 7-9]
  \definition{expr.}{``Vamos fazê-lo!''; simplesmente faça; aja sem demora}[说干就干,别犹豫!===Vamos fazer isso sem hesitar!]
\end{EntryWithPhonetic}

\begin{EntryWithPhonetic}{说好}{shuo1hao3}{9,6}{⾔,⼥}
  \definition{v.}{chegar a um acordo | concluir negociações}
\end{EntryWithPhonetic}

\begin{EntryWithPhonetic}{说话}{shuo1 hua4}{9,8}{⾔,⾔}[HSK 1]
  \definition{adv.}{imediatamente; em um minuto; refere"-se ao tempo que leva para falar, indicando um período muito curto}
  \definition{v.}{falar; conversar; dizer; expressar o significado através da linguagem | conversar (conversa fiada); bater papo | fofocar; conversar; criticar; censurar}
\end{EntryWithPhonetic}

\begin{EntryWithPhonetic}{说谎}{shuo1/huang3}{9,11}{⾔,⾔}[HSK 7-9]
  \definition{v.+compl.}{mentir; contar uma mentira; dizer coisas falsas intencionalmente}
  \synonymref{撒谎}{sa1/huang3}
  \antonymref{诚实}{cheng2shi5}
\end{EntryWithPhonetic}

\begin{EntryWithPhonetic}{说老实话}{shuo1 lao3shi5 hua4}{9,6,8,8}{⾔,⽼,⼧,⾔}[HSK 7-9]
  \definition{expr.}{para ser honesto; diga a verdade}[说老实话,我有点累。===Para ser sincero, estou um pouco cansado.]
\end{EntryWithPhonetic}

\begin{EntryWithPhonetic}{说理}{shuo1li3}{9,11}{⾔,⽟}
  \definition{v.}{racionalizar | discutir logicamente}
\end{EntryWithPhonetic}

\begin{EntryWithPhonetic}{说明}{shuo1ming2}{9,8}{⾔,⽇}[HSK 2]
  \definition[本,个]{s.}{legenda; instrução; explicação}
  \definition{v.}{mostrar; explicar; ilustrar | indicar; mostrar; provar; demonstrar; usar materiais confiáveis para demonstrar ou determinar a autenticidade de pessoas ou coisas}
\end{EntryWithPhonetic}

\begin{EntryWithPhonetic}{说明书}{shuo1ming2shu1}{9,8,4}{⾔,⽇,⼄}[HSK 6]
  \definition[本]{s.}{manual; livro de instruções; descrições textuais da finalidade, especificações, desempenho e uso de itens, bem como enredos de peças e filmes, etc.}
\end{EntryWithPhonetic}

\begin{EntryWithPhonetic}{说起来}{shuo1 qi3lai2}{9,10,7}{⾔,⾛,⽊}[HSK 7-9]
  \definition{adv.}{por falar nisso; falar sobre; comentar sobre isso; mencionar (algo)}[说起来,真有趣!===Por falar nisso, é realmente interessante!]
\end{EntryWithPhonetic}

\begin{EntryWithPhonetic}{说情}{shuo1/qing2}{9,11}{⾔,⼼}[HSK 7-9]
  \definition{v.+compl.}{suplicar por misericórdia para alguém; interceder por alguém}
  \seealsoref{说情儿}{shuo1qing2r5}
\end{EntryWithPhonetic}

\begin{EntryWithPhonetic}{说情儿}{shuo1qing2r5}{9,11,2}{⾔,⼼,⼉}
  \definition{v.}{suplicar por misericórdia para alguém; interceder por alguém}
\end{EntryWithPhonetic}

\begin{EntryWithPhonetic}{说实话}{shuo1/shi2hua4}{9,8,8}{⾔,⼧,⾔}[HSK 6]
  \definition{v.+compl.}{falar a verdade; dizer a verdade sobre (os próprios erros ou crimes)}
\end{EntryWithPhonetic}

\begin{EntryWithPhonetic}{说完}{shuo1-wan2}{9,7}{⾔,⼧}
  \definition{expr.}{acabar/terminar palavras}
\end{EntryWithPhonetic}

\begin{EntryWithPhonetic}{说闲话}{shuo1 xian2hua4}{9,7,8}{⾔,⾨,⾔}[HSK 7-9]
  \definition{s.}{fofoca; resmungo | bate"-papo; conversa à toa | conversa fiada; falar pelas costas de alguém}
  \seealsoref{说闲话儿}{shuo1xian2hua4 er5}
\end{EntryWithPhonetic}

\begin{EntryWithPhonetic}{说闲话儿}{shuo1xian2hua4 er5}{9,7,8,2}{⾔,⾨,⾔,⼉}
  \definition{s.}{fofoca}
\end{EntryWithPhonetic}

\begin{EntryWithPhonetic}{说真的}{shuo1 zhen1de5}{9,10,8}{⾔,⼗,⽩}[HSK 7-9]
  \definition{adv.}{sem brincadeira; sinceramente; honestamente}[说真的,现在要买台万能仪可不容易。===Sinceramente, não é fácil comprar um instrumento multifuncional hoje em dia.]
  \synonymref{说实话}{shuo1/shi2hua4}
\end{EntryWithPhonetic}

%%%%%%%%%% 硕 %%%%%%%%%%
\subsection*{硕}\addcontentsline{loh}{figure}{硕 \dpy{shuo4}}

\begin{EntryWithPhonetic}{硕}{shuo4}{11}{⽯}
  \definition*{s.}{Sobrenome: Shuo}
  \definition{adj.}{grande; enorme}
  \definition{s.}{mestrado (MBA)}
\end{EntryWithPhonetic}

\begin{EntryWithPhonetic}{硕果}{shuo4guo3}{11,8}{⽯,⽊}[HSK 7-9]
  \definition{s.}{resultados frutíferos; frutos grandes e maduros | ótimo trabalho | grande conquista | sucesso triunfante}
  \synonymref{成果}{cheng2guo3}
  \synonymref{果实}{guo3shi2}
  \synonymref{收获}{shou1huo4}
\end{EntryWithPhonetic}

\begin{EntryWithPhonetic}{硕士}{shuo4shi4}{11,3}{⽯,⼠}[HSK 5]
  \definition[个,位,名]{s.}{mestrado; um diploma concedido por uma universidade ou faculdade a um aluno após um ou dois anos de estudo adicional após o bacharelado}
\end{EntryWithPhonetic}

%%%%%%%%%% 数 %%%%%%%%%%
\subsection*{数}\addcontentsline{loh}{figure}{数 \dpy{shuo4}}

\begin{EntryWithPhonetic}{数}{shuo4}{13}{⽁}
  \definition{adv.}{com frequência; repetidamente; indica uma ação frequente, equivalente a 屡次}
  \seeref{shu3}
  \seeref{shu4}
  \seealsoref{屡次}{lv3ci4}
\end{EntryWithPhonetic}

%%%%%%%%%% 丝 %%%%%%%%%%
\subsection*{丝}\addcontentsline{loh}{figure}{丝 \dpy{si1}}

\begin{EntryWithPhonetic}{丝}{si1}{5}{⼀}[HSK 7-9]
  \definition{clas.}{si, uma unidade de peso (=0,0005 gramas) | usado para expressar a aparência ou expressão de uma pessoa | um décimo de milésimo de certas unidades de medida (medida de comprimento) | usado para representar coisas abstratas}
  \definition[些,种,类,跟,缕]{s.}{seda | uma coisa semelhante a um fio; itens semelhantes à seda | cordas; instrumentos de corda}
\end{EntryWithPhonetic}

\begin{EntryWithPhonetic}{丝绸}{si1chou2}{5,11}{⼀,⽷}[HSK 7-9]
  \definition[段,米,面]{s.}{seda; tecido de seda; termo genérico para tecidos feitos de seda}
\end{EntryWithPhonetic}

\begin{EntryWithPhonetic}{丝毫}{si1hao2}{5,11}{⼀,⽊}[HSK 7-9]
  \definition{adj.}{um pouco; muito pouco; uma ninharia; um pedacinho; uma partícula; a menor quantidade ou grau}
  \antonymref{全部}{quan2bu4}
\end{EntryWithPhonetic}

%%%%%%%%%% 司 %%%%%%%%%%
\subsection*{司}\addcontentsline{loh}{figure}{司 \dpy{si1}}

\begin{EntryWithPhonetic}{司}{si1}{5}{⼝}
  \definition*{s.}{Sobrenome: Si}
  \definition{s.}{departamento (sob um ministério); um departamento dentro de uma agência de nível ministerial}
  \definition{v.}{assumir o comando de; atender; administrar; operar; gerenciar}
\end{EntryWithPhonetic}

\begin{EntryWithPhonetic}{司法}{si1fa3}{5,8}{⼝,⽔}[HSK 7-9]
  \definition{s.}{(administração da) justiça; juízes; isso se refere à investigação e ao julgamento pelo Ministério Público ou tribunal, de acordo com a legislação de processos cíveis e criminais}
  \synonymref{法律}{fa3lv4}
  \antonymref{犯法}{fan4fa3}
  \antonymref{混乱}{hun4luan4}
\end{EntryWithPhonetic}

\begin{EntryWithPhonetic}{司机}{si1ji1}{5,6}{⼝,⽊}[HSK 2]
  \definition[个,名,位]{s.}{motorista; motorista particular; chofer; motoristas de veículos de transporte público, como trens, ônibus e bondes}
\end{EntryWithPhonetic}

\begin{EntryWithPhonetic}{司空见惯}{si1kong1-jian4guan4}{5,8,4,11}{⼝,⽳,⾒,⼼}[HSK 7-9]
  \definition{expr.}{ocorrência comum; nada de incomum; é uma expressão idiomática chinesa que significa algo comum e banal; tem origem no poema ``A História da Poesia: Emoções''《本事诗·情感》}
\end{EntryWithPhonetic}

\begin{EntryWithPhonetic}{司令}{si1ling4}{5,5}{⼝,⼈}[HSK 7-9]
  \definition[位]{s.}{comandante; oficial comandante}
\end{EntryWithPhonetic}

\begin{EntryWithPhonetic}{司长}{si1zhang3}{5,4}{⼝,⾧}[HSK 6]
  \definition[位,名]{s.}{diretor-geral | chefe de gabinete}
\end{EntryWithPhonetic}

%%%%%%%%%% 私 %%%%%%%%%%
\subsection*{私}\addcontentsline{loh}{figure}{私 \dpy{si1}}

\begin{EntryWithPhonetic}{私}{si1}{7}{⽲}
  \definition*{s.}{Sobrenome: Si}
  \definition{adj.}{pessoal; privado | egoísta | secreto; privado | ilícito; ilegal}
  \definition{s.}{interesse privado (ou egoísta); motivo (ou ideia) egoísta | contrabando; mercadorias contrabandeadas | propriedade privada | interesses privados; ganho pessoal}
  \antonymref{公}{gong1}
\end{EntryWithPhonetic}

\begin{EntryWithPhonetic}{私房钱}{si1fang2qian2}{7,8,10}{⽲,⼾,⾦}[HSK 7-9]
  \definition{s.}{poupança privada (de um membro da família) | bolsa secreta | esconderijo secreto de dinheiro}
\end{EntryWithPhonetic}

\begin{EntryWithPhonetic}{私函}{si1han2}{7,8}{⽲,⼐}
  \definition{s.}{carta privada}
\end{EntryWithPhonetic}

\begin{EntryWithPhonetic}{私家车}{si1jia1che1}{7,10,4}{⽲,⼧,⾞}[HSK 7-9]
  \definition[辆]{s.}{carro particular}
  \seealsoref{公车}{gong1che1}
\end{EntryWithPhonetic}

\begin{EntryWithPhonetic}{私立}{si1li4}{7,5}{⽲,⽴}[HSK 7-9]
  \definition{s.}{privado; estabelecido privadamente}[这是一所私立学校。===Esta é uma escola particular.]
  \definition{v.}{estabelecer"-se ilegalmente}
  \antonymref{公立}{gong1li4}
\end{EntryWithPhonetic}

\begin{EntryWithPhonetic}{私人}{si1ren2}{7,2}{⽲,⼈}[HSK 5]
  \definition{adj.}{privado; pertencente a um indivíduo ou exercido a título individual; não público | pessoal; entre indivíduos}
  \definition[个]{s.}{algo privado; pessoas que se aproximam de você por motivos pessoais ou interesses próprios}
\end{EntryWithPhonetic}

\begin{EntryWithPhonetic}{私人信件}{si1ren2 xin4jian4}{7,2,9,6}{⽲,⼈,⼈,⼈}
  \definition{s.}{carta pessoal}
\end{EntryWithPhonetic}

\begin{EntryWithPhonetic}{私人钥匙}{si1ren2yao4shi5}{7,2,9,11}{⽲,⼈,⾦,⼔}
  \definition{s.}{(criptografia) chave privada}
\end{EntryWithPhonetic}

\begin{EntryWithPhonetic}{私人诊所}{si1ren2 zhen3suo3}{7,2,7,8}{⽲,⼈,⾔,⼾}
  \definition[些]{s.}{clínica privada}
\end{EntryWithPhonetic}

\begin{EntryWithPhonetic}{私生活}{si1sheng1huo2}{7,5,9}{⽲,⽣,⽔}
  \definition{s.}{vida privada}
\end{EntryWithPhonetic}

\begin{EntryWithPhonetic}{私事}{si1shi4}{7,8}{⽲,⼅}[HSK 7-9]
  \definition[件,桩]{s.}{privacidade; assuntos privados; assuntos pessoais}
  \antonymref{公事}{gong1shi4}
  \antonymref{公务}{gong1wu4}
\end{EntryWithPhonetic}

\begin{EntryWithPhonetic}{私下}{si1xia4}{7,3}{⽲,⼀}[HSK 7-9]
  \definition{adv.}{em particular; em segredo; realizada em caráter privado, sem o consentimento dos departamentos competentes ou do público | sem autorização; sem aprovação; nos bastidores}
  \synonymref{私自}{si1zi4}
  \antonymref{公开}{gong1kai1}
\end{EntryWithPhonetic}

\begin{EntryWithPhonetic}{私营}{si1ying2}{7,11}{⽲,⾋}[HSK 7-9]
  \definition{adj.}{de propriedade privada; administrado (ou operado) de forma privada; privado}
  \definition{s.}{propriedade privada}
\end{EntryWithPhonetic}

\begin{EntryWithPhonetic}{私有}{si1you3}{7,6}{⽲,⽉}[HSK 7-9]
  \definition{adj.}{de propriedade privada; privado}
  \antonymref{共有}{gong4you3}
  \antonymref{国有}{guo2you3}
\end{EntryWithPhonetic}

\begin{EntryWithPhonetic}{私自}{si1zi4}{7,6}{⽲,⾃}[HSK 7-9]
  \definition{adv.}{em particular; secretamente; fazer algo que viole leis ou regulamentos pelas costas da organização ou do pessoal relevante}
  \synonymref{擅自}{shan4zi4}
  \synonymref{私下}{si1xia4}
  \antonymref{奉献}{feng4xian4}
\end{EntryWithPhonetic}

%%%%%%%%%% 思 %%%%%%%%%%
\subsection*{思}\addcontentsline{loh}{figure}{思 \dpy{si1}}

\begin{EntryWithPhonetic}{思}{si1}{9}{⼼}
  \definition*{s.}{Sobrenome: Si}
  \definition{s.}{pensamento; ideias | pensamentos; emoções; humor}
  \definition{v.}{pensar; considerar; deliberar | pensar em; ansiar por}
\end{EntryWithPhonetic}

\begin{EntryWithPhonetic}{思考}{si1kao3}{9,6}{⼼,⽼}[HSK 4]
  \definition{v.}{pensar; ponderar; considerar; deliberar; envolver"-se em atividades de pensamento, como análise, síntese, julgamento, raciocínio e generalização}
\end{EntryWithPhonetic}

\begin{EntryWithPhonetic}{思路}{si1lu4}{9,13}{⼼,⾜}[HSK 7-9]
  \definition[种]{s.}{ideias; pensamento; mentalidade; linha de raciocínio}
  \synonymref{灵感}{ling2gan3}
  \synonymref{线索}{xian4suo3}
\end{EntryWithPhonetic}

\begin{EntryWithPhonetic}{思念}{si1nian4}{9,8}{⼼,⼼}[HSK 7-9]
  \definition{v.}{sentir falta; pensar em; ansiar por; sentir saudades}
  \synonymref{怀念}{huai2nian4}
\end{EntryWithPhonetic}

\begin{EntryWithPhonetic}{思前想后}{si1qian2-xiang3hou4}{9,9,13,6}{⼼,⼑,⼼,⼝}[HSK 7-9]
  \definition{expr.}{``Depois de refletir bastante sobre o assunto.''; refletir repetidamente; ponderar sobre a mesma coisa; considerar a causa passada e o efeito futuro; refletir sobre o passado e o futuro; refletir sobre as razões e a conexão}
\end{EntryWithPhonetic}

\begin{EntryWithPhonetic}{思索}{si1suo3}{9,10}{⼼,⽷}[HSK 7-9]
  \definition{v.}{ponderar; considerar; especular; deliberar; pensar profundamente}
  \synonymref{考虑}{kao3lv4}
  \synonymref{思考}{si1kao3}
  \synonymref{思念}{si1nian4}
  \synonymref{推敲}{tui1qiao1}
  \synonymref{研究}{yan2jiu1}
  \antonymref{行动}{xing2dong4}
\end{EntryWithPhonetic}

\begin{EntryWithPhonetic}{思维}{si1wei2}{9,11}{⼼,⽷}[HSK 5]
  \definition[种]{s.}{pensamento; reflexão; organizar e transformar os materiais obtidos através do conhecimento sensorial para formar conceitos, julgamentos e raciocínios}
  \definition{v.}{pensar}
\end{EntryWithPhonetic}

\begin{EntryWithPhonetic}{思想}{si1xiang3}{9,13}{⼼,⼼}[HSK 3]
  \definition[个,种]{s.}{reflexão; pensamento; ideologia; a existência objetiva é refletida na consciência das pessoas por meio de atividades de pensamento, que pertencem à cognição racional | ideia; pensamento}
\end{EntryWithPhonetic}

%%%%%%%%%% 斯 %%%%%%%%%%
\subsection*{斯}\addcontentsline{loh}{figure}{斯 \dpy{si1}}

\begin{EntryWithPhonetic}{斯}{si1}{12}{⽄}
  \definition*{s.}{Sobrenome: Si}
  \definition{adv.}{então; assim}
  \definition{pron.}{isto; aqui}
\end{EntryWithPhonetic}

\begin{EntryWithPhonetic}{斯巴达}{si1ba1da2}{12,4,6}{⽄,⼰,⾡}
  \definition*{s.}{Esparta}
\end{EntryWithPhonetic}

%%%%%%%%%% 撕 %%%%%%%%%%
\subsection*{撕}\addcontentsline{loh}{figure}{撕 \dpy{si1}}

\begin{EntryWithPhonetic}{撕}{si1}{15}{⼿}[HSK 7-9]
  \definition{v.}{rasgar; romper | rasgar; despedaçar; puxar algo em direções opostas para que se quebre em pedaços menores, geralmente papel}
\end{EntryWithPhonetic}

%%%%%%%%%% 死 %%%%%%%%%%
\subsection*{死}\addcontentsline{loh}{figure}{死 \dpy{si3}}

\begin{EntryWithPhonetic}{死}{si3}{6}{⽍}[HSK 3]
  \definition{adj.}{até a morte | implacável; mortal | fixo; rígido; inflexível | intransitável; fechado | (expressando raiva, reclamação, etc., às vezes jocosamente) maldito}
  \definition{adv.}{(frequentemente no negativo) teimosamente; inflexivelmente}
  \definition{v.}{morrer; estar morto}
  \antonymref{活}{huo2}
  \antonymref{生}{sheng1}
\end{EntryWithPhonetic}

\begin{EntryWithPhonetic}{死亡}{si3wang2}{6,3}{⽍,⼇}[HSK 6]
  \definition{s.}{morte; condenação; dar o último suspiro; refere"-se ao estado de vida desaparecendo}
  \definition{v.}{morrer; estar morto; perder a vida (em oposição à 生存)}
  \seealsoref{生存}{sheng1cun2}
\end{EntryWithPhonetic}

\begin{EntryWithPhonetic}{死心}{si3xin1}{6,4}{⽍,⼼}[HSK 7-9]
  \definition{v.}{abandonar a ideia para sempre; não alimentar mais ilusões sobre o assunto; desistir da ideia e parar de ter esperança}
  \antonymref{留恋}{liu2lian4}
  \antonymref{迷恋}{mi2lian4}
  \antonymref{牵挂}{qian1gua4}
\end{EntryWithPhonetic}

\begin{EntryWithPhonetic}{死心塌地}{si3xin1-ta1di4}{6,4,13,6}{⽍,⼼,⼟,⼟}[HSK 7-9]
  \definition{expr.}{estar obstinado em; estar irremediavelmente decidido a; comprometer"-se desesperadamente e irremediavelmente com um caminho maligno; desistir de qualquer ideia de uma alternativa; descreve uma decisão que está firmemente tomada e não será alterada}
\end{EntryWithPhonetic}

%%%%%%%%%% 四 %%%%%%%%%%
\subsection*{四}\addcontentsline{loh}{figure}{四 \dpy{si4}}

\begin{EntryWithPhonetic}{四}{si4}{5}{⼞}[HSK 1]
  \definition*{s.}{Sobrenome: Si}
  \definition{num.}{quatro; 4}
  \definition{s.}{uma nota da escala em Gongchepu (工尺谱), correspondente ao 6 na notação musical numerada}
  \seealsoref{工尺谱}{gong1 che3 pu3}
\end{EntryWithPhonetic}

\begin{EntryWithPhonetic}{四处}{si4chu4}{5,5}{⼞,⼡}[HSK 6]
  \definition{adv.}{em volta; ao redor; em todos os lugares; em todas as direções}
\end{EntryWithPhonetic}

\begin{EntryWithPhonetic}{四川}{si4chuan1}{5,3}{⼞,⼮}
  \definition*{s.}{Província de Sichuan}
\end{EntryWithPhonetic}

\begin{EntryWithPhonetic}{四合院}{si4he2yuan4}{5,6,9}{⼞,⼝,⾩}[HSK 7-9]
  \definition[座,所,幢,套]{s.}{pátio; habitações quadrangulares; um conjunto de casas em torno de um pátio quadrado; o estilo residencial tradicional em Pequim consiste em casas nos quatro lados e um pátio no meio}
\end{EntryWithPhonetic}

\begin{EntryWithPhonetic}{四季}{si4ji4}{5,8}{⼞,⼦}[HSK 7-9]
  \definition{s.}{quatro estações; primavera, verão, outono e inverno, cada uma com duração de três meses}
\end{EntryWithPhonetic}

\begin{EntryWithPhonetic}{四季分明}{si4ji4-fen1ming2}{5,8,4,8}{⼞,⼦,⼑,⽇}
  \definition{expr.}{as quatro estações são muito distintas}
\end{EntryWithPhonetic}

\begin{EntryWithPhonetic}{四季如春}{si4ji4-ru2chun1}{5,8,6,9}{⼞,⼦,⼥,⽇}
  \definition{expr.}{é primavera todo o ano | clima favorável durante todo o ano | quatro estações como a primavera}
\end{EntryWithPhonetic}

\begin{EntryWithPhonetic}{四面八方}{si4mian4-ba1fang1}{5,9,2,4}{⼞,⾯,⼋,⽅}[HSK 7-9]
  \definition{expr.}{em todas as direções; em todos os ângulos; ao redor; de perto e de longe}
  \synonymref{大街小巷}{da4jie1-xiao3xiang4}
\end{EntryWithPhonetic}

\begin{EntryWithPhonetic}{四周}{si4zhou1}{5,8}{⼞,⼝}[HSK 5]
  \definition{s.}{ao redor; por todos os lados; a parte que circunda o centro}
\end{EntryWithPhonetic}

%%%%%%%%%% 似 %%%%%%%%%%
\subsection*{似}\addcontentsline{loh}{figure}{似 \dpy{si4}}

\begin{EntryWithPhonetic}{似}{si4}{6}{⼈}
  \definition*{s.}{Sobrenome: Si}
  \definition{adv.}{parece; como se}
  \definition{v.}{ser semelhante; parecer-se com | parecer; aparecer | exceder}
  \seeref{shi4}
  \synonymref{类}{lei4}
  \synonymref{如}{ru2}
  \synonymref{像}{xiang4}
\end{EntryWithPhonetic}

\begin{EntryWithPhonetic}{似曾相识}{si4ceng2-xiang1shi2}{6,12,9,7}{⼈,⽈,⽬,⾔}[HSK 7-9]
  \definition{s.}{parecem já ter se encontrado antes; aparentemente já se conhecia; \emph{déjà vu} (a experiência de ver exatamente a mesma situação uma segunda vez); aparentemente familiar}
\end{EntryWithPhonetic}

\begin{EntryWithPhonetic}{似乎}{si4hu1}{6,5}{⼈,⼃}[HSK 4]
  \definition{adv.}{como se; aparentemente; se parece como}
  \synonymref{仿佛}{fang3fu2}
  \synonymref{好似}{hao3si4}
  \synonymref{好像}{hao3xiang4}
  \synonymref{如同}{ru2tong2}
  \synonymref{相似}{xiang1si4}
  \antonymref{肯定}{ken3ding4}
  \antonymref{确定}{que4ding4}
\end{EntryWithPhonetic}

\begin{EntryWithPhonetic}{似是而非}{si4shi4-er2fei1}{6,9,6,8}{⼈,⽇,⽽,⾮}[HSK 7-9]
  \definition{expr.}{``Aparentemente verdade, mas na realidade falso.''; especioso; plausível; parecer certo, mas na verdade errado; aparentemente verdadeiro, mas na realidade errado; parecer o que realmente não é; ser como água e vinho; semelhante à realidade, mas não ser como ela é}
  \antonymref{天经地义}{tian1jing1-di4yi4}
\end{EntryWithPhonetic}

%%%%%%%%%% 寺 %%%%%%%%%%
\subsection*{寺}\addcontentsline{loh}{figure}{寺 \dpy{si4}}

\begin{EntryWithPhonetic}{寺}{si4}{6}{⼨}[HSK 6]
  \definition*{s.}{Sobrenome: Si}
  \definition[座]{s.}{templo | (Islã) mesquita | Obsoleto: ministério; agência governamental na China antiga}
\end{EntryWithPhonetic}

\begin{EntryWithPhonetic}{寺庙}{si4miao4}{6,8}{⼨,⼴}[HSK 7-9]
  \definition[座]{s.}{mosteiro; casa de Deus; templo; templos budistas}
\end{EntryWithPhonetic}

%%%%%%%%%% 伺 %%%%%%%%%%
\subsection*{伺}\addcontentsline{loh}{figure}{伺 \dpy{si4}}

\begin{EntryWithPhonetic}{伺}{si4}{7}{⼈}
  \definition{v.}{aguardar; observar; esperar por}
  \seeref{ci4}
\end{EntryWithPhonetic}

\begin{EntryWithPhonetic}{伺机}{si4ji1}{7,6}{⼈,⽊}[HSK 7-9]
  \definition{v.}{ficar atento às oportunidades | aguardar uma oportunidade}
\end{EntryWithPhonetic}

%%%%%%%%%% 饲 %%%%%%%%%%
\subsection*{饲}\addcontentsline{loh}{figure}{饲 \dpy{si4}}

\begin{EntryWithPhonetic}{饲}{si4}{8}{⾷}
  \definition{s.}{forragem; alimento; ração}
  \definition{v.}{criar; manter; reproduzir}
\end{EntryWithPhonetic}

\begin{EntryWithPhonetic}{饲料}{si4liao4}{8,10}{⾷,⽃}[HSK 7-9]
  \definition[袋]{s.}{alimento; forragem; alimento para gado ou aves}
\end{EntryWithPhonetic}

\begin{EntryWithPhonetic}{饲养}{si4yang3}{8,9}{⾷,⼋}[HSK 7-9]
  \definition{v.}{criar; alimentar aves, gado e outros animais}
  \synonymref{放养}{fang4yang3}
  \synonymref{喂养}{wei4yang3}
\end{EntryWithPhonetic}

%%%%%%%%%% 食 %%%%%%%%%%
\subsection*{食}\addcontentsline{loh}{figure}{食 \dpy{si4}}

\begin{EntryWithPhonetic}{食}{si4}{9}{⾷}[Kangxi 184]
  \definition{v.}{alimentar; dar comida a}
  \seeref{shi2}
\end{EntryWithPhonetic}

%%%%%%%%%% 肆 %%%%%%%%%%
\subsection*{肆}\addcontentsline{loh}{figure}{肆 \dpy{si4}}

\begin{EntryWithPhonetic}{肆}{si4}{13}{⾀}
  \definition*{s.}{Sobrenome: Si}
  \definition{adj.}{desenfreado; sem limites; descuidado; imprudente}
  \definition{num.}{quatro (usado para o numeral 四 em cheques, etc., para evitar erros ou alterações)}
  \definition{s.}{Literário: loja; armazém}
  \seealsoref{四}{si4}
\end{EntryWithPhonetic}

%%%%%%%%%% 厕 %%%%%%%%%%
\subsection*{厕}\addcontentsline{loh}{figure}{厕 \dpy{si5}}

\begin{EntryWithPhonetic}{厕}{si5}{8}{⼚}
  \definition{s.}{componente formador de palavras | latrina; fossa sanitária}
  \seeref{ce4}
  \seealsoref{茅厕}{mao2ce4}
\end{EntryWithPhonetic}

%%%%%%%%%% 松 %%%%%%%%%%
\subsection*{松}\addcontentsline{loh}{figure}{松 \dpy{song1}}

\begin{EntryWithPhonetic}{松}{song1}{8}{⽊}[HSK 4]
  \definition*{s.}{Sobrenome: Song}
  \definition{adj.}{solto; frouxo; folgado | leve e crocante; macio | relaxado; confortável}
  \definition[棵]{s.}{pinheiro | fio de carne seca; carne moída seca; alimentos macios ou quebradiços}
  \definition{v.}{afrouxar; relaxar; abrandar | desamarrar; desatar; liberar}
\end{EntryWithPhonetic}

\begin{EntryWithPhonetic}{松绑}{song1/bang3}{8,9}{⽊,⽷}[HSK 7-9]
  \definition{v.+compl.}{desatar; desamarrar uma pessoa | Figurativo: relaxar as restrições; flexibilizar as restrições; desatar; libertar}
\end{EntryWithPhonetic}

\begin{EntryWithPhonetic}{松弛}{song1chi2}{8,6}{⽊,⼸}[HSK 7-9]
  \definition{adj.}{frouxo; solto; relaxado; não tenso | frouxo; (regras, regulamentos, etc.) não são rigorosamente aplicados}
  \synonymref{缓和}{huan3he2}
  \synonymref{宽容}{kuan1rong2}
  \synonymref{宽松}{kuan1song1}
  \synonymref{马虎}{ma3hu5}
  \synonymref{轻松}{qing1song1}
  \synonymref{疏忽}{shu1hu5}
  \synonymref{随便}{sui2/bian4}
  \antonymref{坚实}{jian1shi2}
  \antonymref{紧急}{jin3ji2}
  \antonymref{紧张}{jin3zhang1}
  \antonymref{牢牢}{lao2lao2}
  \antonymref{严格}{yan2ge2}
\end{EntryWithPhonetic}

\begin{EntryWithPhonetic}{松木}{song1mu4}{8,4}{⽊,⽊}
  \definition{s.}{pinheiro}
\end{EntryWithPhonetic}

\begin{EntryWithPhonetic}{松树}{song1shu4}{8,9}{⽊,⽊}[HSK 4]
  \definition[棵]{s.}{pinheiro; conífera comum, geralmente com folhas longas e pontiagudas e cones lenhosos}
\end{EntryWithPhonetic}

%%%%%%%%%% 嵩 %%%%%%%%%%
\subsection*{嵩}\addcontentsline{loh}{figure}{嵩 \dpy{song1}}

\begin{EntryWithPhonetic}{嵩}{song1}{13}{⼭}
  \definition*{s.}{Sobrenome: Song}
  \definition{adj.}{Arcaico: (montanhas) alto; elevado}
\end{EntryWithPhonetic}

\begin{EntryWithPhonetic}{嵩山}{song1shan1}{13,3}{⼭,⼭}
  \definition*{s.}{Monte Song em Henan, montanha central das Cinco Montanhas Sagradas (五岳)}
  \seealsoref{五岳}{wu3yue4}
\end{EntryWithPhonetic}

%%%%%%%%%% 耸 %%%%%%%%%%
\subsection*{耸}\addcontentsline{loh}{figure}{耸 \dpy{song3}}

\begin{EntryWithPhonetic}{耸}{song3}{10}{⽿}
  \definition{v.}{ser imponente; ser elevado | alarmar; chocar | encolher os ombros; levantar (os ombros, etc.) | elevar"-se; erguer"-se verticalmente | chocar; atrair a atenção; surpreender}
\end{EntryWithPhonetic}

\begin{EntryWithPhonetic}{耸立}{song3li4}{10,5}{⽿,⽴}[HSK 7-9]
  \definition{v.}{elevar"-se no alto; erguer"-se bem alto; manter"-se ereto}
  \synonymref{操场}{cao1chang3}
  \synonymref{改为}{gai3wei2}
  \synonymref{挺拔}{ting3ba2}
  \synonymref{挺立}{ting3li4}
\end{EntryWithPhonetic}

%%%%%%%%%% 宋 %%%%%%%%%%
\subsection*{宋}\addcontentsline{loh}{figure}{宋 \dpy{song4}}

\begin{EntryWithPhonetic}{宋}{song4}{7}{⼧}
  \definition*{s.}{Dinastia Song (960--1279) | Song das dinastias do sul (420--479) | Sobrenome: Song}
  \definition{clas.}{sone; unidade de intensidade sonora}
\end{EntryWithPhonetic}

%%%%%%%%%% 送 %%%%%%%%%%
\subsection*{送}\addcontentsline{loh}{figure}{送 \dpy{song4}}

\begin{EntryWithPhonetic}{送}{song4}{9}{⾡}[HSK 1]
  \definition*{s.}{Sobrenome: Song}
  \definition{v.}{transportar; entregar | dar; dar como presente; presentear | acompanhar; despedir-se de alguém (ao sair); acompanhar a pessoa que está partindo até o destino ou caminhar um trecho com ela | escoltar}
\end{EntryWithPhonetic}

\begin{EntryWithPhonetic}{送别}{song4/bie2}{9,7}{⾡,⼑}[HSK 7-9]
  \definition{v.+compl.}{despedir"-se; dar adeus}
\end{EntryWithPhonetic}

\begin{EntryWithPhonetic}{送到}{song4 dao4}{9,8}{⾡,⼑}[HSK 2]
  \definition{v.}{enviar para (lugar)}
\end{EntryWithPhonetic}

\begin{EntryWithPhonetic}{送给}{song4gei3}{9,9}{⾡,⽷}[HSK 2]
  \definition{v.}{dar a (alguém ou organização); dar como algo gratuito; dar como presente}
\end{EntryWithPhonetic}

\begin{EntryWithPhonetic}{送礼}{song4 li3}{9,5}{⾡,⽰}[HSK 6]
  \definition{v.}{dar um presente a alguém; presentear alguém com um presente | enviar presentes (para obter favores) | dar um presente; enviar um presente}
\end{EntryWithPhonetic}

\begin{EntryWithPhonetic}{送行}{song4 xing2}{9,6}{⾡,⾏}[HSK 6]
  \definition{v.}{ver alguém partir; ir até o local onde o viajante iniciou sua jornada, despedir-se dele e observar ele partir | dar uma festa de despedida; realizar uma festa de despedida | despedir-se do falecido}
\end{EntryWithPhonetic}

%%%%%%%%%% 㮸 %%%%%%%%%%
\subsection*{㮸}\addcontentsline{loh}{figure}{㮸 \dpy{song4}}

\begin{EntryWithPhonetic}{㮸}{song4}{14}{⽊}
  \variantof{送}
\end{EntryWithPhonetic}

%%%%%%%%%% 搜 %%%%%%%%%%
\subsection*{搜}\addcontentsline{loh}{figure}{搜 \dpy{sou1}}

\begin{EntryWithPhonetic}{搜}{sou1}{12}{⼿}[HSK 5]
  \definition{v.}{procurar | pesquisar | coletar; reunir | procurar ou revistar um lugar de forma completa e desordenada}
  \seealsoref{搜查}{sou1cha2}
\end{EntryWithPhonetic}

\begin{EntryWithPhonetic}{搜查}{sou1cha2}{12,9}{⼿,⽊}[HSK 7-9]
  \definition{v.}{procurar; vasculhar; revirar | buscar e inspecionar (criminosos ou contrabando); examinar minuciosamente para encontrar problemas}
  \seealsoref{搜}{sou1}
  \synonymref{搜索}{sou1suo3}
  \synonymref{搜寻}{sou1xun2}
  \antonymref{隐藏}{yin3cang2}
\end{EntryWithPhonetic}

\begin{EntryWithPhonetic}{搜集}{sou1ji2}{12,12}{⼿,⾫}[HSK 7-9]
  \definition{v.}{coletar; reunir; pesquisar coleção}
  \synonymref{采集}{cai3ji2}
  \synonymref{汇集}{hui4ji2}
  \synonymref{收集}{shou1ji2}
  \synonymref{网络}{wang3luo4}
  \synonymref{征求}{zheng1qiu2}
\end{EntryWithPhonetic}

\begin{EntryWithPhonetic}{搜救}{sou1jiu4}{12,11}{⼿,⽁}[HSK 7-9]
  \definition{v.}{buscar e resgatar}
  \synonymref{救助}{jiu4zhu4}
  \synonymref{抢救}{qiang3jiu4}
  \synonymref{搜寻}{sou1xun2}
  \synonymref{援助}{yuan2zhu4}
\end{EntryWithPhonetic}

\begin{EntryWithPhonetic}{搜索}{sou1suo3}{12,10}{⼿,⽷}[HSK 5]
  \definition{v.}{procurar; caçar; explorar; pesquisar cuidadosamente; refere"-se especificamente à busca militar para identificar situações suspeitas em determinada região, área marítima ou aérea}
\end{EntryWithPhonetic}

\begin{EntryWithPhonetic}{搜寻}{sou1xun2}{12,6}{⼿,⼨}[HSK 7-9]
  \definition{v.}{procurar; caçar; buscar}
  \synonymref{搜查}{sou1cha2}
  \synonymref{搜救}{sou1jiu4}
  \synonymref{搜索}{sou1suo3}
\end{EntryWithPhonetic}

%%%%%%%%%% 艘 %%%%%%%%%%
\subsection*{艘}\addcontentsline{loh}{figure}{艘 \dpy{sou1}}

\begin{EntryWithPhonetic}{艘}{sou1}{15}{⾈}[HSK 7-9]
  \definition{clas.}{utilizado para barcos grandes, navios}
\end{EntryWithPhonetic}

%%%%%%%%%% 苏 %%%%%%%%%%
\subsection*{苏}\addcontentsline{loh}{figure}{苏 \dpy{su1}}

\begin{EntryWithPhonetic}{苏}{su1}{7}{⾋}
  \definition*{s.}{Suzhou, abreviação de 苏州 | Província de Jiangsu, abreviação de 江苏 | União Soviética, abreviação de 苏联 | Sobrenome: Su}
  \definition{s.}{perilla planta da família das mentas}
  \definition{v.}{reviver; vir a; acordar}
  \seealsoref{江苏}{jiang1su1}
  \seealsoref{苏联}{su1lian2}
  \seealsoref{苏州}{su1zhou1}
\end{EntryWithPhonetic}

\begin{EntryWithPhonetic}{苏格兰}{su1ge2lan2}{7,10,5}{⾋,⽊,⼋}
  \definition*{s.}{Escócia}
\end{EntryWithPhonetic}

\begin{EntryWithPhonetic}{苏联}{su1lian2}{7,12}{⾋,⽿}
  \definition*{s.}{União das Repúblicas Socialistas Soviéticas (1922--1991)}
\end{EntryWithPhonetic}

\begin{EntryWithPhonetic}{苏醒}{su1xing3}{7,16}{⾋,⾣}[HSK 7-9]
  \definition{v.}{voltar a si; recuperar a consciência; acordar do coma e recuperar a consciência normal}
  \synonymref{复苏}{fu4su1}
  \synonymref{惊醒}{jing1xing3}
  \synonymref{觉醒}{jue2xing3}
  \synonymref{清醒}{qing1xing3}
  \antonymref{昏迷}{hun1mi2}
  \antonymref{睡眠}{shui4mian2}
\end{EntryWithPhonetic}

\begin{EntryWithPhonetic}{苏州}{su1zhou1}{7,6}{⾋,⼮}
  \definition*{s.}{Suzhou, cidade na Província de Jiangsu}
\end{EntryWithPhonetic}

%%%%%%%%%% 酥 %%%%%%%%%%
\subsection*{酥}\addcontentsline{loh}{figure}{酥 \dpy{su1}}

\begin{EntryWithPhonetic}{酥}{su1}{12}{⾣}[HSK 7-9]
  \definition{adj.}{crocante; (alimento) solto e frágil | macio; derretido}
  \definition{s.}{Arcaico: ghee (tipo de manteiga clarificada) | biscoito amanteigado; uma massa leve e quebradiça feita com farinha, óleo e açúcar}
\end{EntryWithPhonetic}

%%%%%%%%%% 俗 %%%%%%%%%%
\subsection*{俗}\addcontentsline{loh}{figure}{俗 \dpy{su2}}

\begin{EntryWithPhonetic}{俗}{su2}{9}{⼈}[HSK 7-9]
  \definition{adj.}{secular; leigo | popular; comum | vulgar; sem gosto | comum; ordinário | grosseiro; abrutalhado}
  \definition[个,种]{s.}{costume; convenção; culto}
  \antonymref{僧}{seng1}
\end{EntryWithPhonetic}

\begin{EntryWithPhonetic}{俗话}{su2hua4}{9,8}{⼈,⾔}[HSK 7-9]
  \definition[句]{s.}{provérbio; ditado popular}
  \synonymref{俗语}{su2yu3}
\end{EntryWithPhonetic}

\begin{EntryWithPhonetic}{俗话说}{su2hua4 shuo1}{9,8,9}{⼈,⾔,⾔}[HSK 7-9]
  \definition{expr.}{como diz o ditado; como diz o provérbio}[俗话说,百闻不如一见。===Como diz o ditado, ver para crer.]
\end{EntryWithPhonetic}

\begin{EntryWithPhonetic}{俗语}{su2yu3}{9,9}{⼈,⾔}[HSK 7-9]
  \definition{s.}{ditado popular; provérbio; máxima popular; são expressões fixas comuns e amplamente utilizadas, concisas e vívidas, criadas principalmente pela classe trabalhadora, refletindo suas experiências de vida e aspirações}
\end{EntryWithPhonetic}

%%%%%%%%%% 诉 %%%%%%%%%%
\subsection*{诉}\addcontentsline{loh}{figure}{诉 \dpy{su4}}

\begin{EntryWithPhonetic}{诉}{su4}{7}{⾔}
  \definition{v.}{contar; relatar; informar; falar para que outros saibam; declarar | apelar para; recorrer a; apresentar os fatos do caso ao tribunal; registrar uma queixa | reclamar}
\end{EntryWithPhonetic}

\begin{EntryWithPhonetic}{诉苦}{su4/ku3}{7,8}{⾔,⾋}[HSK 7-9]
  \definition{v.+compl.}{reclamar; desabafar; expressar as próprias queixas; expor os próprios problemas; contar aos outros sobre o sofrimento que você suportou}
  \synonymref{抱怨}{bao4yuan5}
\end{EntryWithPhonetic}

\begin{EntryWithPhonetic}{诉说}{su4shuo1}{7,9}{⾔,⾔}[HSK 7-9]
  \definition{v.}{contar; relatar; narrar; declarar com emoção}
\end{EntryWithPhonetic}

\begin{EntryWithPhonetic}{诉讼}{su4song4}{7,6}{⾔,⾔}[HSK 7-9]
  \definition[场,起]{s.}{processo judicial; litígio; atividades realizadas por promotores, tribunais, partes em processos cíveis e promotores privados em processos criminais na resolução de casos}
\end{EntryWithPhonetic}

%%%%%%%%%% 素 %%%%%%%%%%
\subsection*{素}\addcontentsline{loh}{figure}{素 \dpy{su4}}

\begin{EntryWithPhonetic}{素}{su4}{10}{⽷}[HSK 7-9]
  \definition{adj.}{branco; de cor natural | simples; natural; singelo; de cor simples | nativo; original | normal; usual; geral}
  \definition{adv.}{geralmente; sempre; habitualmente}
  \definition{s.}{vegetais, frutas e outros alimentos | matéria-prima; matéria-prima básico; tecidos de seda naturais e não processados | elemento; os componentes básicos de algo}
  \seealsoref{荤}{hun1}
  \antonymref{荤}{hun1}
\end{EntryWithPhonetic}

\begin{EntryWithPhonetic}{素不相识}{su4bu4xiang1shi2}{10,4,9,7}{⽷,⼀,⽬,⾔}[HSK 7-9]
  \definition{expr.}{ser completamente estranhos um para o outro; ser um completo estranho para\dots; não conhecerz\dots antes; nunca ter se encontrado (visto) antes; não se conhecer de forma alguma; nunca se encontraram; não se conhecem}
\end{EntryWithPhonetic}

\begin{EntryWithPhonetic}{素材}{su4cai2}{10,7}{⽷,⽊}[HSK 7-9]
  \definition{s.}{material; material de origem; os materiais originais coletados da vida real são a base para a criação literária e artística}
  \synonymref{材料}{cai2liao4}
\end{EntryWithPhonetic}

\begin{EntryWithPhonetic}{素描}{su4miao2}{10,11}{⽷,⼿}[HSK 7-9]
  \definition{s.}{esboço; pintura sem cor | esboço literário; descrição sem floreios}
\end{EntryWithPhonetic}

\begin{EntryWithPhonetic}{素食}{su4shi2}{10,9}{⽷,⾷}[HSK 7-9]
  \definition[顿]{s.}{comida vegetariana; dieta vegetariana; alimentos sem carne}
  \definition{v.}{ser vegetariano}
\end{EntryWithPhonetic}

\begin{EntryWithPhonetic}{素养}{su4yang3}{10,9}{⽷,⼋}[HSK 7-9]
  \definition{s.}{conquista; realização; autocultivo habitual}
  \synonymref{教养}{jiao4yang3}
  \synonymref{素质}{su4zhi4}
  \synonymref{修养}{xiu1yang3}
\end{EntryWithPhonetic}

\begin{EntryWithPhonetic}{素质}{su4zhi4}{10,8}{⽷,⾙}[HSK 6]
  \definition[个,种]{s.}{qualidade; características; caráter; o nível físico, moral, mental, intelectual e cultural de uma pessoa}
\end{EntryWithPhonetic}

%%%%%%%%%% 速 %%%%%%%%%%
\subsection*{速}\addcontentsline{loh}{figure}{速 \dpy{su4}}

\begin{EntryWithPhonetic}{速}{su4}{10}{⾡}
  \definition{adj.}{rápido; veloz}
  \definition{s.}{velocidade}
  \definition{v.aux.}{convidar}
\end{EntryWithPhonetic}

\begin{EntryWithPhonetic}{速度}{su4du4}{10,9}{⾡,⼴}[HSK 3]
  \definition[个,种]{s.}{velocidade; taxa; ritmo; andamento; uma quantidade física que indica a velocidade e a direção do movimento de um objeto, ou seja, a distância que um objeto percorre em uma direção por unidade de tempo | velocidade; rapidez; geralmente se refere ao grau de velocidade}
\end{EntryWithPhonetic}

%%%%%%%%%% 宿 %%%%%%%%%%
\subsection*{宿}\addcontentsline{loh}{figure}{宿 \dpy{su4}}

\begin{EntryWithPhonetic}{宿}{su4}{11}{⼧}
  \definition*{s.}{Sobrenome: Su}
  \definition{adj.}{de longa data; antigo; velho | veterano; velho; experiente}
  \definition{v.}{hospedar-se para passar a noite; passar a noite}
  \seeref{xiu3}
  \seeref{xiu4}
\end{EntryWithPhonetic}

\begin{EntryWithPhonetic}{宿舍}{su4she4}{11,8}{⼧,⾆}[HSK 5]
  \definition[间,幢]{s.}{alojamento; dormitório; república; albergue; casas onde escolas, empresas, etc. acomodam seus alunos ou funcionários}
\end{EntryWithPhonetic}

%%%%%%%%%% 塑 %%%%%%%%%%
\subsection*{塑}\addcontentsline{loh}{figure}{塑 \dpy{su4}}

\begin{EntryWithPhonetic}{塑}{su4}{13}{⼟}
  \definition{s.}{plástico; material plástico}
  \definition{v.}{modelo; molde; forma}
\end{EntryWithPhonetic}

\begin{EntryWithPhonetic}{塑料}{su4liao4}{13,10}{⼟,⽃}[HSK 4]
  \definition[块,种]{s.}{plástico; compostos de polímeros feitos de resinas naturais ou sintéticas como componente principal}
\end{EntryWithPhonetic}

\begin{EntryWithPhonetic}{塑料袋}{su4liao4dai4}{13,10,11}{⼟,⽃,⾐}[HSK 4]
  \definition[个,只]{s.}{saco plástico; sacola de plástico}
\end{EntryWithPhonetic}

\begin{EntryWithPhonetic}{塑造}{su4zao4}{13,10}{⼟,⾡}[HSK 7-9]
  \definition{s.}{modelar; dar forma; utilizar materiais como argila para criar imagens de pessoas e objetos | retratar; nas obras literárias e artísticas, os personagens são retratados utilizando a linguagem ou outros recursos artísticos | dar forma; cultivar e transformar pessoas ou coisas de acordo com objetivos predefinidos}
  \synonymref{打造}{da3zao4}
  \synonymref{刻画}{ke4hua4}
  \synonymref{树立}{shu4li4}
  \antonymref{破坏}{po4huai4}
\end{EntryWithPhonetic}

%%%%%%%%%% 缩 %%%%%%%%%%
\subsection*{缩}\addcontentsline{loh}{figure}{缩 \dpy{su4}}

\begin{EntryWithPhonetic}{缩}{su4}{14}{⽷}
  \definition{s.}{lírio anão}[缩花朵能做药材。===As flores do lírio anão podem ser usadas como material medicinal.]
  \seeref{suo1}
\end{EntryWithPhonetic}

%%%%%%%%%% 痠 %%%%%%%%%%
\subsection*{痠}\addcontentsline{loh}{figure}{痠 \dpy{suan1}}

\begin{EntryWithPhonetic}{痠}{suan1}{12}{⽧}
  \definition{v.}{doer | estar dolorido}
  \variantof{酸}
\end{EntryWithPhonetic}

%%%%%%%%%% 酸 %%%%%%%%%%
\subsection*{酸}\addcontentsline{loh}{figure}{酸 \dpy{suan1}}

\begin{EntryWithPhonetic}{酸}{suan1}{14}{⾣}[HSK 4]
  \definition{adj.}{azedo; ácido | aflito; angustiado; doente do coração | pedante; descreve uma pessoa que finge ser culta e também descreve uma pessoa que é muito inflexível com suas próprias ideias e não está disposta a mudá-las para atender às exigências da época, é usado principalmente para satirizar intelectuais que fingem ser capazes de escrever poemas e artigos | ciumento; invejoso; sentimentos desconfortáveis porque outra pessoa é melhor do que você e, em geral, também apresenta comportamento hostil}
  \definition{s.}{ácido; produto químico que tem um sabor ácido quando misturado com água}
  \definition{v.}{estar dolorido (devido à fadiga ou doença); descreve a sensação de não ter força muscular e um pouco de dor por estar doente ou muito cansado}
\end{EntryWithPhonetic}

\begin{EntryWithPhonetic}{酸辣汤}{suan1la4tang1}{14,14,6}{⾣,⾟,⽔}
  \definition{s.}{sopa avinagrada e picante (prato)}
\end{EntryWithPhonetic}

\begin{EntryWithPhonetic}{酸奶}{suan1nai3}{14,5}{⾣,⼥}[HSK 4]
  \definition[瓶,杯,盒,袋]{s.}{iogurte; produto lácteo fermentado por bactérias de ácido láctico}
\end{EntryWithPhonetic}

\begin{EntryWithPhonetic}{酸甜苦辣}{suan1-tian2-ku3-la4}{14,11,8,14}{⾣,⽢,⾋,⾟}[HSK 5]
  \definition{expr.}{os altos e baixos da vida; as experiências agridoces da vida; os aspectos doces, azedos, amargos e picantes da vida; refere"-se a todos os tipos de sabores, como metáfora para experiências diversas, como felicidade, sofrimento, etc.; azedo, doce, amargo, picante (alegrias e tristezas da vida)}
\end{EntryWithPhonetic}

%%%%%%%%%% 蒜 %%%%%%%%%%
\subsection*{蒜}\addcontentsline{loh}{figure}{蒜 \dpy{suan4}}

\begin{EntryWithPhonetic}{蒜}{suan4}{13}{⾋}[HSK 7-9]
  \definition[瓣,头]{s.}{alho}
\end{EntryWithPhonetic}

%%%%%%%%%% 算 %%%%%%%%%%
\subsection*{算}\addcontentsline{loh}{figure}{算 \dpy{suan4}}

\begin{EntryWithPhonetic}{算}{suan4}{14}{⽵}[HSK 2]
  \definition{adv.}{finalmente; por fim; no final; significa que, após um longo período de tempo ou muitas dificuldades, finalmente se alcançou o objetivo, equivalente a 总算}
  \definition{v.}{calcular; estimar; computar | contar; incluir | planejar; calcular; projetar | pensar; supor; especular | considerar; considerar como; contar como; reconhecer como | (aritmética) contar; ter peso | deixe estar; deixe passar; seguido por 了: desistir, não se importar mais}
  \seealsoref{了}{le5}
  \seealsoref{总算}{zong3suan4}
\end{EntryWithPhonetic}

\begin{EntryWithPhonetic}{算计}{suan4ji4}{14,4}{⽵,⾔}[HSK 7-9]
  \definition{v.}{calcular; determinar; descobrir | planejar; pensar; considerar | esperar; calcular; estimar | planejar; esquematizar; tramar em segredo}
  \synonymref{估计}{gu1ji4}
  \synonymref{合计}{he2ji4}
  \synonymref{计算}{ji4suan4}
  \synonymref{盘算}{pan2suan5}
  \synonymref{推算}{tui1suan4}
  \synonymref{阴谋}{yin1mou2}
\end{EntryWithPhonetic}

\begin{EntryWithPhonetic}{算了}{suan4le5}{14,2}{⽵,⼅}[HSK 6]
  \definition{part.}{deixe estar; deixe passar; usado no final de uma frase para expressar imperativo, término, etc.}
  \definition{v.}{deixar; deixe estar; deixe passar; esquecer isso; não querer continuar; é usado para persuadir os outros ou para expressar que posso aceitar a situação atual, para encerrar o assunto ou assunto atual, ou para dizer ``esqueça''}
\end{EntryWithPhonetic}

\begin{EntryWithPhonetic}{算命}{suan4ming4}{14,8}{⽵,⼝}
  \definition{s.}{cartomante}
  \definition{v.}{ler a sorte | fazer advinhações}
\end{EntryWithPhonetic}

\begin{EntryWithPhonetic}{算盘}{suan4pan2}{14,11}{⽵,⽫}[HSK 7-9]
  \definition[把]{s.}{ábaco; uma ferramenta para realizar cálculos | plano; esquema; cálculo; uma metáfora para um plano que é benéfico para si mesmo ou que está alinhado com os seus próprios desejos}
  \synonymref{打算}{da3suan5}
  \synonymref{计划}{ji4hua4}
  \synonymref{盘算}{pan2suan5}
\end{EntryWithPhonetic}

\begin{EntryWithPhonetic}{算是}{suan4shi4}{14,9}{⽵,⽇}[HSK 6]
  \definition{adv.}{finalmente; por fim; depois de muito tempo, o objetivo foi finalmente alcançado}
  \definition{v.}{contar como; pensar que; ser considerado}
\end{EntryWithPhonetic}

\begin{EntryWithPhonetic}{算账}{suan4/zhang4}{14,8}{⽵,⾙}[HSK 7-9]
  \definition{v.+compl.}{fazer contas; equilibrar os livros; emitir faturas; calcular contas | acertar as contas com alguém; acertar as contas com alguém; discutir ou brigar com outras pessoas após sofrer uma perda ou fracasso}
\end{EntryWithPhonetic}

%%%%%%%%%% 尿 %%%%%%%%%%
\subsection*{尿}\addcontentsline{loh}{figure}{尿 \dpy{sui1}}

\begin{EntryWithPhonetic}{尿}{sui1}{7}{⼫}
  \definition[泡]{s.}{urina}
  \definition{v.}{urinar}
  \seeref{niao4}
\end{EntryWithPhonetic}

%%%%%%%%%% 虽 %%%%%%%%%%
\subsection*{虽}\addcontentsline{loh}{figure}{虽 \dpy{sui1}}

\begin{EntryWithPhonetic}{虽}{sui1}{9}{⾍}[HSK 6]
  \definition{conj.}{no entanto; embora | mesmo se}
\end{EntryWithPhonetic}

\begin{EntryWithPhonetic}{虽然}{sui1ran2}{9,12}{⾍,⽕}[HSK 2]
  \definition{conj.}{apesar de; embora (frequentemente usado correlativamente com 可是, 但是, etc); geralmente é usado no início de uma frase para indicar que o fato anterior foi reconhecido, mas não mudará o que acontecerá em seguida}
  \seealsoref{但是}{dan4shi4}
  \seealsoref{可是}{ke3shi4}
\end{EntryWithPhonetic}

\begin{EntryWithPhonetic}{虽说}{sui1shuo1}{9,9}{⾍,⾔}[HSK 7-9]
  \definition{conj.}{embora; apesar de}
  \synonymref{虽然}{sui1ran2}
\end{EntryWithPhonetic}

%%%%%%%%%% 随 %%%%%%%%%%
\subsection*{随}\addcontentsline{loh}{figure}{随 \dpy{sui2}}

\begin{EntryWithPhonetic}{随}{sui2}{11}{⾩}[HSK 3]
  \definition*{s.}{Sobrenome: Sui}
  \definition{adv.}{fazer algo imediatamente assim que ocorre, sem demora ou hesitação; usado antes de dois verbos ou frases verbais para indicar que a última ação segue a anterior}
  \definition{prep.}{junto com (alguma outra ação) | apresentando as condições das quais a ação depende}
  \definition{v.}{seguir; vir (ou ir) junto com | concordar com; adaptar"-se a | deixar (alguém fazer o que quiser) | (dialeto) parecer"-se com; assemelhar"-se a | seguir ou agir de acordo com a condição ou circunstância da qual a ação depende}
\end{EntryWithPhonetic}

\begin{EntryWithPhonetic}{随便}{sui2/bian4}{11,9}{⾩,⼈}[HSK 2]
  \definition{adj.}{relaxado; descontraído; sem restrições; sem limitações | aleatório; casual; descuidado; indiferente; distraído, não pensa bem antes de falar ou agir | casual; informal; não dá importância aos detalhes}
  \definition{conj.}{qualquer; qualquer que seja; não importa}
  \definition{v.+compl.}{deixar alguém à vontade}
\end{EntryWithPhonetic}

\begin{EntryWithPhonetic}{随处}{sui2chu4}{11,5}{⾩,⼡}
  \definition{adv.}{em qualquer lugar}
\end{EntryWithPhonetic}

\begin{EntryWithPhonetic}{随处可见}{sui2chu4 ke3 jian4}{11,5,5,4}{⾩,⼡,⼝,⾒}[HSK 7-9]
  \definition{expr.}{em todos os lugares; por toda parte; onipresente}
\end{EntryWithPhonetic}

\begin{EntryWithPhonetic}{随大流}{sui2 da4liu2}{11,3,10}{⾩,⼤,⽔}[HSK 5]
  \definition{v.}{deixar"-se levar (ou nadar) pela correnteza; seguir a tendência geral; fazer como os outros fazem | seguir a maré | seguir a multidão}
  \synonymref{随大溜}{sui2 da4liu4}
\end{EntryWithPhonetic}

\begin{EntryWithPhonetic}{随大溜}{sui2 da4liu4}{11,3,13}{⾩,⼤,⽔}[HSK 7-9]
  \definition{v.}{seguir a multidão; deixar"-se levar (ou nadar) pela correnteza; seguir a tendência geral; fazer como os outros fazem}
  \synonymref{随大流}{sui2 da4liu2}
\end{EntryWithPhonetic}

\begin{EntryWithPhonetic}{随地}{sui2di4}{11,6}{⾩,⼟}
  \definition{adv.}{qualquer lugar | todo lugar}
\end{EntryWithPhonetic}

\begin{EntryWithPhonetic}{随和}{sui2he5}{11,8}{⾩,⼝}
  \definition{adj.}{afável; prestativo; tranquilo; fácil de conviver; gentil}
  \synonymref{和蔼}{he2'ai3}
  \synonymref{温和}{wen1he2}
  \antonymref{反对}{fan3dui4}
  \antonymref{固执}{gu4zhi5}
  \antonymref{乖张}{guai1zhang1}
  \antonymref{矫情}{jiao2qing5}
  \antonymref{严肃}{yan2su4}
\end{EntryWithPhonetic}

\begin{EntryWithPhonetic}{随后}{sui2hou4}{11,6}{⾩,⼝}[HSK 5]
  \definition{adv.}{logo em seguida; logo depois; indica que segue imediatamente após a ação ou situação anterior (geralmente usado em conjunto com 就)}
  \seealsoref{就}{jiu4}
\end{EntryWithPhonetic}

\begin{EntryWithPhonetic}{随机}{sui2ji1}{11,6}{⾩,⽊}[HSK 7-9]
  \definition{adj.}{aleatório; indica fazer algo em conformidade com as mudanças no tempo e nas circunstâncias}
  \definition{adv.}{de acordo com a situação; sem quaisquer condições; fazer algo arbitrariamente}
  \synonymref{立即}{li4ji2}
  \synonymref{立刻}{li4ke4}
\end{EntryWithPhonetic}

\begin{EntryWithPhonetic}{随机存取存储器}{sui2ji1cun2qu3cun2chu3qi4}{11,6,6,8,6,12,16}{⾩,⽊,⼦,⼜,⼦,⼈,⼝}
  \definition{s.}{RAM (\emph{random access memory})}
  \seealsoref{内存}{nei4cun2}
  \seealsoref{随机存取记忆体}{sui2ji1cun2qu3ji4yi4ti3}
\end{EntryWithPhonetic}

\begin{EntryWithPhonetic}{随机存取记忆体}{sui2ji1cun2qu3ji4yi4ti3}{11,6,6,8,5,4,7}{⾩,⽊,⼦,⼜,⾔,⼼,⼈}
  \definition{s.}{RAM (\emph{random access memory})}
  \seealsoref{内存}{nei4cun2}
  \seealsoref{随机存取存储器}{sui2ji1cun2qu3cun2chu3qi4}
\end{EntryWithPhonetic}

\begin{EntryWithPhonetic}{随即}{sui2ji2}{11,7}{⾩,⼙}[HSK 7-9]
  \definition{adv.}{imediatamente; atualmente; indica que algo acontece imediatamente após uma ação ou situação anterior}
  \synonymref{理科}{li3ke1}
  \synonymref{立即}{li4ji2}
  \synonymref{立刻}{li4ke4}
  \antonymref{永远}{yong3yuan3}
\end{EntryWithPhonetic}

\begin{EntryWithPhonetic}{随身}{sui2shen1}{11,7}{⾩,⾝}[HSK 7-9]
  \definition{adj.}{significa ``carregar consigo'' ou ``estar ao lado'', indicando que algo é mantido junto ao corpo ou próximo a ele, levado para todos os lugares; é frequentemente usada para descrever pertences pessoais, bagagem ou equipamentos, como: bagagem de mão (随身行李), itens de mão (随身携带)}
  \seealsoref{随身行李}{sui2shen1 hang2li3}
  \seealsoref{随身携带}{sui2shen1 xie2dai4}
\end{EntryWithPhonetic}

\begin{EntryWithPhonetic}{随身行李}{sui2shen1 hang2li3}{11,7,6,7}{⾩,⾝,⾏,⽊}[HSK 7-9]
  \definition{s.}{bagagem de mão; itens para bagagem de mão}
\end{EntryWithPhonetic}

\begin{EntryWithPhonetic}{随身携带}{sui2shen1 xie2dai4}{11,7,13,9}{⾩,⾝,⼿,⼱}
  \definition{s.}{itens de mão (leve"-o consigo)}
\end{EntryWithPhonetic}

\begin{EntryWithPhonetic}{随时}{sui2shi2}{11,7}{⾩,⽇}[HSK 2]
  \definition{adv.}{a qualquer momento; em todos os momentos}
\end{EntryWithPhonetic}

\begin{EntryWithPhonetic}{随时随地}{sui2shi2-sui2di4}{11,7,11,6}{⾩,⽇,⾩,⼟}[HSK 7-9]
  \definition{expr.}{``A qualquer hora, em qualquer lugar.''; sempre que possível; em todos os lugares; em todos os momentos e lugares; quando e onde}
\end{EntryWithPhonetic}

\begin{EntryWithPhonetic}{随手}{sui2shou3}{11,4}{⾩,⼿}[HSK 4]
  \definition{adv.}{convenientemente; sem problemas adicionais; casualmente}
\end{EntryWithPhonetic}

\begin{EntryWithPhonetic}{随心所欲}{sui2xin1suo3yu4}{11,4,8,11}{⾩,⼼,⼾,⽋}[HSK 7-9]
  \definition{expr.}{faça do seu jeito; fazer o que se quer, segundo a própria vontade (mais tarde, passou a significar fazer tudo o que se deseja); seguir as próprias inclinações; fazer do seu jeito; agir como bem entender}
  \antonymref{力不从心}{li4bu4cong2xin1}
  \antonymref{力所能及}{li4suo3neng2ji2}
\end{EntryWithPhonetic}

\begin{EntryWithPhonetic}{随意}{sui2/yi4}{11,13}{⾩,⼼}[HSK 5]
  \definition{adj.}{aleatório; casual; à vontade; como se deseja}
\end{EntryWithPhonetic}

\begin{EntryWithPhonetic}{随着}{sui2zhe5}{11,11}{⾩,⽬}[HSK 5]
  \definition{prep.}{junto com; na esteira de; em sintonia com; usado no início da frase ou antes do verbo, indica as condições necessárias para que uma ação, comportamento ou evento ocorra}
\end{EntryWithPhonetic}

%%%%%%%%%% 遂 %%%%%%%%%%
\subsection*{遂}\addcontentsline{loh}{figure}{遂 \dpy{sui2}}

\begin{EntryWithPhonetic}{遂}{sui2}{12}{⾡}
  \definition{adv.}{então; portanto; assim; como resultado; posteriormente}
  \definition{s.}{hemiplegia (paralisia de um lado do corpo)}
  \definition{v.}{ser como desejado; cumprir; satisfazer | ter sucesso; realizar; alcançar; concluir; ser bem"-sucedido}
  \seeref{sui4}
\end{EntryWithPhonetic}

%%%%%%%%%% 岁 %%%%%%%%%%
\subsection*{岁}\addcontentsline{loh}{figure}{岁 \dpy{sui4}}

\begin{EntryWithPhonetic}{岁}{sui4}{6}{⼭}[HSK 1]
  \definition{clas.}{usado para anos (de idade)}
  \definition{s.}{ano (literário) | colheita do ano (literário) | idade | tempo (literário) | ano (de idade) | ano (para as colheitas)}
\end{EntryWithPhonetic}

\begin{EntryWithPhonetic}{岁数}{sui4shu4}{6,13}{⼭,⽁}[HSK 6]
  \definition{s.}{idade; anos; a idade de uma pessoa}
\end{EntryWithPhonetic}

\begin{EntryWithPhonetic}{岁月}{sui4yue4}{6,4}{⼭,⽉}[HSK 5]
  \definition[段,番]{s.}{anos; ano e mês; refere"-se a tempo em geral}
\end{EntryWithPhonetic}

%%%%%%%%%% 遂 %%%%%%%%%%
\subsection*{遂}\addcontentsline{loh}{figure}{遂 \dpy{sui4}}

\begin{EntryWithPhonetic}{遂}{sui4}{12}{⾡}
  \definition*{s.}{Sobrenome: Sui}
  \definition{adv.}{Literário: então; em seguida}
  \definition{v.}{satisfazer; realizar | Literário: ter sucesso | cumprir}
\end{EntryWithPhonetic}

\begin{EntryWithPhonetic}{遂心}{sui4/xin1}{12,4}{⾡,⼼}[HSK 7-9]
  \definition{v.+compl.}{ter o desejo do coração realizado; ter um anseio atendido; satisfazer o próprio desejo}
\end{EntryWithPhonetic}

\begin{EntryWithPhonetic}{遂意}{sui4yi4}{12,13}{⾡,⼼}
  \definition{adj.}{de acordo com o próprio gosto; do agrado de alguém}
\end{EntryWithPhonetic}

%%%%%%%%%% 碎 %%%%%%%%%%
\subsection*{碎}\addcontentsline{loh}{figure}{碎 \dpy{sui4}}

\begin{EntryWithPhonetic}{碎}{sui4}{13}{⽯}[HSK 5]
  \definition*{s.}{Sobrenome: Sui}
  \definition{adj.}{quebrado; fragmentado | tagarela; falante}
  \definition{v.}{quebrar em pedaços; esmagar}
\end{EntryWithPhonetic}

%%%%%%%%%% 隧 %%%%%%%%%%
\subsection*{隧}\addcontentsline{loh}{figure}{隧 \dpy{sui4}}

\begin{EntryWithPhonetic}{隧}{sui4}{14}{⾩}
  \definition{s.}{túnel; passagem subterrânea | estrada | subúrbios; áreas suburbanas}
  \definition{v.}{virar}
\end{EntryWithPhonetic}

\begin{EntryWithPhonetic}{隧道}{sui4dao4}{14,12}{⾩,⾡}[HSK 7-9]
  \definition[条,段,孔,段,处]{s.}{túnel; uma passagem escavada em uma montanha ou no subsolo}
  \synonymref{地道}{di4dao4}
  \synonymref{轨道}{gui3dao4}
\end{EntryWithPhonetic}

%%%%%%%%%% 孙 %%%%%%%%%%
\subsection*{孙}\addcontentsline{loh}{figure}{孙 \dpy{sun1}}

\begin{EntryWithPhonetic}{孙}{sun1}{6}{⼦}
  \definition*{s.}{Sobrenome: Sun}
  \definition{s.}{neto; neta | gerações abaixo da do neto | parentes pertencentes à geração do neto | segundo crescimento das plantas}
\end{EntryWithPhonetic}

\begin{EntryWithPhonetic}{孙女}{sun1nv5}{6,3}{⼦,⼥}[HSK 4]
  \definition[个]{s.}{filha do filho; neta}
\end{EntryWithPhonetic}

\begin{EntryWithPhonetic}{孙武}{sun1wu3}{6,8}{⼦,⽌}
  \definition*{s.}{Sun Wu, também conhecido por Sun Tzu (孙子) general, estrategista e filósofo autor do ``Arte da Guerra''《孙子兵法》}
  \seealsoref{孙子}{sun1zi3}
  \seealsoref{孙子兵法}{sun1zi3 bing1fa3}
\end{EntryWithPhonetic}

\begin{EntryWithPhonetic}{孙子}{sun1zi3}{6,3}{⼦,⼦}
  \definition*{s.}{Sun Tzu, também conhecido por Sun Wu (孙武), general, estrategista e filósofo autor do ``Arte da Guerra''《孙子兵法》}
  \seeref{sun1zi5}
  \seealsoref{孙武}{sun1wu3}
  \seealsoref{孙子兵法}{sun1zi3 bing1fa3}
\end{EntryWithPhonetic}

\begin{EntryWithPhonetic}{孙子兵法}{sun1zi3 bing1fa3}{6,3,7,8}{⼦,⼦,⼋,⽔}
  \definition*{s.}{``Arte da Guerra'', o antigo clássico chinês sobre estratégia militar, escrito por Sun Tzu (孫子)}
  \seealsoref{孙武}{sun1wu3}
  \seealsoref{孙子}{sun1zi3}
\end{EntryWithPhonetic}

\begin{EntryWithPhonetic}{孙子}{sun1zi5}{6,3}{⼦,⼦}[HSK 4]
  \definition[个]{s.}{filho do filho; neto}
  \seeref{sun1zi3}
\end{EntryWithPhonetic}

%%%%%%%%%% 损 %%%%%%%%%%
\subsection*{损}\addcontentsline{loh}{figure}{损 \dpy{sun3}}

\begin{EntryWithPhonetic}{损}{sun3}{10}{⼿}[HSK 7-9]
  \definition{adj.}{sarcástico; cortante; de língua afiada; maldoso; mau; cruel}
  \definition{v.}{diminuir; perder; reduzir | prejudicar; danificar; degradar; destruir; arruinar; destruir o estado original ou fazê-lo perder sua eficácia original | ser sarcástico; ser cáustico; ser cortante; ferir; insultar; usar palavras duras para zombar de alguém}
  \synonymref{亏}{kui1}
  \antonymref{益}{yi4}
  \antonymref{增}{zeng1}
\end{EntryWithPhonetic}

\begin{EntryWithPhonetic}{损害}{sun3hai4}{10,10}{⼿,⼧}[HSK 5]
  \definition{v.}{prejudicar; danificar; ferir; causar danos; causar perdas}
\end{EntryWithPhonetic}

\begin{EntryWithPhonetic}{损坏}{sun3huai4}{10,7}{⼿,⼟}[HSK 7-9]
  \definition{v.}{estragar; prejudicar; danificar (uma causa, interesses, saúde, reputação, etc.); tornar incompleto; causar dano}
  \synonymref{摧毁}{cui1hui3}
  \synonymref{毁坏}{hui3huai4}
  \synonymref{磨损}{mo2sun3}
  \synonymref{破坏}{po4huai4}
  \synonymref{损害}{sun3hai4}
  \antonymref{爱护}{ai4hu4}
  \antonymref{保护}{bao3hu4}
  \antonymref{补救}{bu3jiu4}
  \antonymref{防护}{fang2hu4}
  \antonymref{维修}{wei2xiu1}
  \antonymref{修理}{xiu1li3}
\end{EntryWithPhonetic}

\begin{EntryWithPhonetic}{损人利己}{sun3ren2-li4ji3}{10,2,7,3}{⼿,⼈,⼑,⼰}[HSK 7-9]
  \definition{expr.}{``Prejudicar os outros para obter vantagens pessoais.''; beneficiar"-se às custas dos outros; enriquecer"-se às custas dos outros; ganhar às custas dos outros; ferir os outros para obter ganho próprio; prejudicar alguém para benefício próprio; ferir os outros para obter vantagem própria; prejudicar os outros para obter lucro próprio; ferir os outros para benefício próprio; lucrar às custas dos outros; buscar fins pessoais às custas dos outros; buscar satisfação pessoal às custas dos outros}
\end{EntryWithPhonetic}

\begin{EntryWithPhonetic}{损伤}{sun3shang1}{10,6}{⼿,⼈}[HSK 7-9]
  \definition{s.}{perda; lesão; danos, destruição ou lesão a um objeto ou corpo}
  \definition{v.}{ferir; causar dano; prejudicar; lesionar; ferir; causar lesão | perder; causar perda a}
  \synonymref{磨损}{mo2sun3}
  \synonymref{伤害}{shang1hai4}
  \synonymref{受伤}{shou4/shang1}
  \synonymref{损害}{sun3hai4}
  \antonymref{保护}{bao3hu4}
  \antonymref{保养}{bao3yang3}
\end{EntryWithPhonetic}

\begin{EntryWithPhonetic}{损失}{sun3shi1}{10,5}{⼿,⼤}[HSK 5]
  \definition{s.}{perda; desperdício; algo que se consome ou se perde sem custo algum}
  \definition{v.}{perder; consumir ou perder}
\end{EntryWithPhonetic}

%%%%%%%%%% 笋 %%%%%%%%%%
\subsection*{笋}\addcontentsline{loh}{figure}{笋 \dpy{sun3}}

\begin{EntryWithPhonetic}{笋}{sun3}{10}{⽵}
  \definition{s.}{broto de bambu}
\end{EntryWithPhonetic}

%%%%%%%%%% 莎 %%%%%%%%%%
\subsection*{莎}\addcontentsline{loh}{figure}{莎 \dpy{suo1}}

\begin{EntryWithPhonetic}{莎}{suo1}{10}{⾋}
  \seeref{sha1}
\end{EntryWithPhonetic}

%%%%%%%%%% 缩 %%%%%%%%%%
\subsection*{缩}\addcontentsline{loh}{figure}{缩 \dpy{suo1}}

\begin{EntryWithPhonetic}{缩}{suo1}{14}{⽷}
  \definition*{s.}{Sobrenome: Suo}
  \definition{v.}{contrair; encolher | recuar; retirar"-se | economizar}
  \seeref{su4}
  \synonymref{收}{shou1}
  \synonymref{退}{tui4}
  \synonymref{萎}{wei3}
  \synonymref{约}{yue1}
  \antonymref{伸}{shen1}
\end{EntryWithPhonetic}

\begin{EntryWithPhonetic}{缩短}{suo1/duan3}{14,12}{⽷,⽮}[HSK 4]
  \definition{v.+compl.}{encurtar; reduzir; diminuir}
\end{EntryWithPhonetic}

\begin{EntryWithPhonetic}{缩手}{suo1shou3}{14,4}{⽷,⼿}
  \definition{v.}{retirar a mão}
\end{EntryWithPhonetic}

\begin{EntryWithPhonetic}{缩水}{suo1/shui3}{14,4}{⽷,⽔}[HSK 7-9]
  \definition{v.+compl.}{encolher (tecidos, após lavagem) | encolher; metaforicamente, refere"-se a uma redução, diminuição ou diminuição de escala, preço, quantidade, etc.}
  \antonymref{膨胀}{peng2zhang4}
\end{EntryWithPhonetic}

\begin{EntryWithPhonetic}{缩小}{suo1/xiao3}{14,3}{⽷,⼩}[HSK 4]
  \definition{v.+compl.}{reduzir, estreitar, encolher;  tornar menor}
  \antonymref{放大}{fang4/da4}
\end{EntryWithPhonetic}

\begin{EntryWithPhonetic}{缩影}{suo1ying3}{14,15}{⽷,⼺}[HSK 7-9]
  \definition{s.}{epítome; miniatura; microcosmo; refere"-se a uma pessoa ou coisa que pode representar ou refletir o mesmo tipo de pessoa ou coisa}
  \synonymref{蓝图}{lan2tu2}
  \synonymref{写照}{xie3zhao4}
\end{EntryWithPhonetic}

\begin{EntryWithPhonetic}{缩影卡片}{suo1ying3 ka3pian4}{14,15,5,4}{⽷,⼺,⼘,⽚}
  \definition{s.}{cartão em miniatura; microcartão}
\end{EntryWithPhonetic}

%%%%%%%%%% 所 %%%%%%%%%%
\subsection*{所}\addcontentsline{loh}{figure}{所 \dpy{suo3}}

\begin{EntryWithPhonetic}{所}{suo3}{8}{⼾}[HSK 3,6]
  \definition*{s.}{Sobrenome: Suo}
  \definition{clas.}{usado para casas, etc.}
  \definition{part.}{usado com 为 ou com 被 para indicar voz passiva | usado antes do verbo para formar um substantivo ou para qualificar um substantivo | usado antes do verbo na estrutura sujeito"-predicado usada como complemento, indica que o termo central é o objeto}
  \definition{s.}{lugar | usado como nome de órgãos governamentais ou outros locais de trabalho}
  \seealsoref{被}{bei4}
  \seealsoref{为}{wei4}
\end{EntryWithPhonetic}

\begin{EntryWithPhonetic}{所长}{suo3chang2}{8,4}{⼾,⾧}
  \definition{s.}{aquilo em que alguém é bom; o ponto forte de alguém; o forte de alguém}
  \seeref{suo3zhang3}
\end{EntryWithPhonetic}

\begin{EntryWithPhonetic}{所属}{suo3shu3}{8,12}{⼾,⼫}[HSK 7-9]
  \definition{s.}{os subordinados, aqueles que estão sob o controle ou comando de alguém; aquilo a que alguém pertence ou com o que está afiliado}
\end{EntryWithPhonetic}

\begin{EntryWithPhonetic}{所谓}{suo3wei4}{8,11}{⼾,⾔}[HSK 7-9]
  \definition{adj.}{o que é chamado; o que é conhecido como; o que significa | o chamado; o que (algumas pessoas) disseram (implicando uma falta de reconhecimento)}
  \synonymref{竟然}{jing4ran2}
  \synonymref{显然}{xian3ran2}
  \synonymref{因为}{yin1wei5}
  \synonymref{终究}{zhong1jiu1}
\end{EntryWithPhonetic}

\begin{EntryWithPhonetic}{所以}{suo3yi3}{8,4}{⼾,⼈}[HSK 2]
  \definition{conj.}{assim; portanto; como resultado; conecta frases, expressa resultados e costuma corresponder a expressões como 因为 e 由于}
  \definition[个]{s.}{motivo real; causa real; comportamento adequado}
  \seealsoref{因为}{yin1wei5}
  \seealsoref{由于}{you2yu2}
\end{EntryWithPhonetic}

\begin{EntryWithPhonetic}{所有}{suo3you3}{8,6}{⼾,⽉}[HSK 2]
  \definition{adj.}{todo | tudo}
  \definition{adj.}{tudo}
  \definition{s.}{bens; posses;}
  \definition{v.}{possuir; ter}
\end{EntryWithPhonetic}

\begin{EntryWithPhonetic}{所在}{suo3zai4}{8,6}{⼾,⼟}[HSK 5]
  \definition[个]{s.}{lugar; local; localização | o lugar onde alguém ou algo está}
\end{EntryWithPhonetic}

\begin{EntryWithPhonetic}{所长}{suo3zhang3}{8,4}{⼾,⾧}[HSK 3]
  \definition{s.}{chefe de um instituto, etc. | superintendente}
  \seeref{suo3chang2}
\end{EntryWithPhonetic}

\begin{EntryWithPhonetic}{所作所为}{suo3zuo4-suo3wei2}{8,7,8,4}{⼾,⼈,⼾,⼂}[HSK 7-9]
  \definition{expr.}{as ações (de alguém); todos os atos de alguém; o que alguém faz; o que alguém faz e como se comporta; o comportamento ou conduta de alguém}
\end{EntryWithPhonetic}

%%%%%%%%%% 索 %%%%%%%%%%
\subsection*{索}\addcontentsline{loh}{figure}{索 \dpy{suo3}}

\begin{EntryWithPhonetic}{索}{suo3}{10}{⽷}
  \definition*{s.}{Sobrenome: Suo}
  \definition{adj.}{completamente sozinho; sozinho | maçante; insípido; sem significado}
  \definition[根]{s.}{corda; cabo; cordão; corrente | uma corda grande}
  \definition{v.}{(literário) pesquisar | exigir; pedir}
\end{EntryWithPhonetic}

\begin{EntryWithPhonetic}{索赔}{suo3pei2}{10,12}{⽷,⾙}[HSK 7-9]
  \definition{v.}{reivindicar indenização por danos; exigir compensação}
  \synonymref{赔偿}{pei2chang2}
\end{EntryWithPhonetic}

\begin{EntryWithPhonetic}{索取}{suo3qu3}{10,8}{⽷,⼜}[HSK 7-9]
  \definition{v.}{solicitar; exigir; cobrar; pedir; isso ocorre devido ao desejo de ganhar ou recuperar algo}
  \antonymref{奉献}{feng4xian4}
  \antonymref{赋予}{fu4yu3}
  \antonymref{贡献}{gong4xian4}
  \antonymref{回报}{hui2bao4}
  \antonymref{贿赂}{hui4lu4}
  \antonymref{给予}{ji3yu3}
  \antonymref{退回}{tui4hui2}
  \antonymref{赠送}{zeng4song4}
\end{EntryWithPhonetic}

\begin{EntryWithPhonetic}{索性}{suo3xing4}{10,8}{⽷,⼼}[HSK 7-9]
  \definition{adv.}{simplesmente; tanto faz; igualmente bem; tomar uma atitude ou uma decisão sem hesitar}
  \synonymref{干脆}{gan1cui4}
  \synonymref{痛快}{tong4kuai5}
  \antonymref{迟疑}{chi2yi2}
  \antonymref{犹豫}{you2yu4}
\end{EntryWithPhonetic}

%%%%%%%%%% 锁 %%%%%%%%%%
\subsection*{锁}\addcontentsline{loh}{figure}{锁 \dpy{suo3}}

\begin{EntryWithPhonetic}{锁}{suo3}{12}{⾦}[HSK 5]
  \definition[把]{s.}{fechadura; dispositivo que impede que as pessoas abram facilmente a parte que se abre e fecha | correntes; cadeado e correntes | qualquer coisa com a forma de um cadeado antigo}
  \definition{v.}{trancar; trancar com chave | costurar com ponto fixo | tricotar}
\end{EntryWithPhonetic}

\begin{EntryWithPhonetic}{锁定}{suo3ding4}{12,8}{⾦,⼧}[HSK 7-9]
  \definition{v.}{fixar; trancar | assegurar; confirmar; estabelecer; garantir | travar em; acompanhar atentamente}
  \synonymref{冻结}{dong4jie2}
  \synonymref{固定}{gu4ding4}
\end{EntryWithPhonetic}

%%%%% EOF %%%%%

