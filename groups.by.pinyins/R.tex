%%%
%%% R
%%%
\section*{R}\addcontentsline{toc}{section}{R}\addcontentsline{loh}{figure}{\#\#\#\#\#\#\#\# R}

%%%%%%%%%% 儿 %%%%%%%%%%
\subsection*{儿}\addcontentsline{loh}{figure}{儿 \dpy{r5}}

\begin{EntryWithPhonetic}{儿}{r5}{2}{⼉}[Kangxi 10]
  \definition{suf.}{sufixo diminutivo não silábico | final retroflexo, pronunciado como ``r'' | adicionado a substantivos para expressar pequenez  | adicionado a verbos, adjetivos e classificadores para formar substantivos | adicionado a substantivos para formar substantivos com significados diferentes | sufixos de alguns verbos | anexado após adjetivos duplicados}
  \seeref{er2}
\end{EntryWithPhonetic}

%%%%%%%%%% 然 %%%%%%%%%%
\subsection*{然}\addcontentsline{loh}{figure}{然 \dpy{ran2}}

\begin{EntryWithPhonetic}{然}{ran2}{12}{⽕}
  \definition{conj.}{mas | no entanto}
\end{EntryWithPhonetic}

\begin{EntryWithPhonetic}{然而}{ran2'er2}{12,6}{⽕,⽽}[HSK 4]
  \definition{conj.}{ainda; mas; contudo; todavia; usado no início de uma frase para indicar uma transição; para indicar uma transição, geralmente é precedido por uma conjunção como 虽然 para indicar concessão}
  \seealsoref{虽然}{sui1ran2}
\end{EntryWithPhonetic}

\begin{EntryWithPhonetic}{然后}{ran2hou4}{12,6}{⽕,⼝}[HSK 2]
  \definition{conj.}{então; depois disso; posteriormente; indica que algo segue após uma ação ou situação}
\end{EntryWithPhonetic}

%%%%%%%%%% 燃 %%%%%%%%%%
\subsection*{燃}\addcontentsline{loh}{figure}{燃 \dpy{ran2}}

\begin{EntryWithPhonetic}{燃}{ran2}{16}{⽕}
  \definition{v.}{queimar | acender; inflamar}
\end{EntryWithPhonetic}

\begin{EntryWithPhonetic}{燃放}{ran2fang4}{16,8}{⽕,⽅}[HSK 7-9]
  \definition{v.}{acender (fogos de artifício, etc.); acender fogos de artifício, etc., para causar uma explosão}
\end{EntryWithPhonetic}

\begin{EntryWithPhonetic}{燃料}{ran2liao4}{16,10}{⽕,⽃}[HSK 4]
  \definition[种]{s.}{combustível; carburante; substâncias que podem gerar calor e energia luminosa quando queimadas podem ser divididas em três tipos de acordo com sua forma: combustível sólido (como carvão, carvão vegetal, madeira), combustível líquido (como gasolina, querosene) e combustível gasoso (como gás de carvão, biogás); também se refere a substâncias que podem gerar energia nuclear, como urânio, plutônio, etc.}
\end{EntryWithPhonetic}

\begin{EntryWithPhonetic}{燃气}{ran2qi4}{16,4}{⽕,⽓}[HSK 7-9]
  \definition{s.}{combustível gasoso, como gás de carvão, biogás, gás natural e gás liquefeito de petróleo}
\end{EntryWithPhonetic}

\begin{EntryWithPhonetic}{燃烧}{ran2shao1}{16,10}{⽕,⽕}[HSK 4]
  \definition{v.}{queimar; acender | arder; inflamar; ferver; metáfora para as emoções de uma pessoa serem muito fortes, como um fogo ardente}
\end{EntryWithPhonetic}

\begin{EntryWithPhonetic}{燃油}{ran2you2}{16,8}{⽕,⽔}[HSK 7-9]
  \definition{s.}{óleo combustível}[燃油价格持续上涨了。===Os preços dos combustíveis continuaram a subir.]
\end{EntryWithPhonetic}

%%%%%%%%%% 染 %%%%%%%%%%
\subsection*{染}\addcontentsline{loh}{figure}{染 \dpy{ran3}}

\begin{EntryWithPhonetic}{染}{ran3}{9}{⽊}[HSK 5]
  \definition*{s.}{Sobrenome: Ran}
  \definition{s.}{soja fermentada e temperada em forma de pasta}
  \definition{v.}{tingir; pintar | pegar (uma doença); cair em (um mau hábito, etc.) | sujar; contaminar | pegar (contrair) (uma doença) | adquirir (um mau hábito, etc.); contaminar}
\end{EntryWithPhonetic}

%%%%%%%%%% 嚷 %%%%%%%%%%
\subsection*{嚷}\addcontentsline{loh}{figure}{嚷 \dpy{rang1}}

\begin{EntryWithPhonetic}{嚷}{rang1}{20}{⼝}
  \definition{v.}{gritar; berrar; fazer barulho | gritar; fazer barulho; revelar; falar em voz alta; dizer algo}
  \seeref{rang3}
\end{EntryWithPhonetic}

\begin{EntryWithPhonetic}{嚷}{rang3}{20}{⼝}[HSK 7-9]
  \definition{v.}{gritar; berrar; causar alvoroço | Coloquial: fazer barulho; causar alvoroço | Dialeto: repreender; humilhar; culpar}
  \seeref{rang1}
\end{EntryWithPhonetic}

%%%%%%%%%% 壤 %%%%%%%%%%
\subsection*{壤}\addcontentsline{loh}{figure}{壤 \dpy{rang3}}

\begin{EntryWithPhonetic}{壤}{rang3}{20}{⼟}
  \definition{s.}{solo | terra | (literário) a terra (em contraste com o céu 天)}
\end{EntryWithPhonetic}

%%%%%%%%%% 让 %%%%%%%%%%
\subsection*{让}\addcontentsline{loh}{figure}{让 \dpy{rang4}}

\begin{EntryWithPhonetic}{让}{rang4}{5}{⾔}[HSK 2]
  \definition*{s.}{Sobrenome: Rang}
  \definition{prep.}{em uma frase passiva para introduzir o executor da ação | de acordo com; em conformidade com; à luz de; com base em; usado para expressar a opinião subjetiva de alguém}
  \definition{v.}{ceder; recuar; render"-se; desistir; admitir | convidar; oferecer | deixar; permitir; fazer | deixar alguém ter algo por um preço justo | ser inferior a; não ser tão bom quanto | ceder; afastar"-se | expressar desejos | esquivar"-se; evitar; fugir | usado antes de 我们, indica uma ordem ou sugestão para que todos façam algo juntos}
  \seealsoref{我们}{wo3men5}
\end{EntryWithPhonetic}

\begin{EntryWithPhonetic}{让步}{rang4/bu4}{5,7}{⾔,⽌}[HSK 7-9]
  \definition{v.+compl.}{ceder; fazer uma concessão; comprometer; em uma disputa, significa renunciar parcial ou totalmente às próprias opiniões ou interesses}
\end{EntryWithPhonetic}

\begin{EntryWithPhonetic}{让座}{rang4 zuo4}{5,10}{⾔,⼴}[HSK 6]
  \definition{v.}{oferecer seu lugar a alguém; ceder seu lugar a alguém | convidar os convidados para se sentarem}
\end{EntryWithPhonetic}

%%%%%%%%%% 饶 %%%%%%%%%%
\subsection*{饶}\addcontentsline{loh}{figure}{饶 \dpy{rao2}}

\begin{EntryWithPhonetic}{饶}{rao2}{9}{⾷}[HSK 7-9]
  \definition{adj.}{rico; abundante; farto}
  \definition{conj.}{embora; apesar do fato de que; indica concessão, com significado semelhante a 虽然 ou 尽管}
  \definition{v.}{ter misericórdia de; absolver alguém; perdoar; absolver | dar algo a mais; permitir que alguém receba algo em troca | desculpar; perdoar; tolerar}
  \seealsoref{尽管}{jin3guan3}
  \seealsoref{虽然}{sui1ran2}
\end{EntryWithPhonetic}

\begin{EntryWithPhonetic}{饶恕}{rao2shu4}{9,10}{⾷,⼼}[HSK 7-9]
  \definition{v.}{desculpar; absolver; perdoar; não punir quando a punição é devida}
\end{EntryWithPhonetic}

%%%%%%%%%% 扰 %%%%%%%%%%
\subsection*{扰}\addcontentsline{loh}{figure}{扰 \dpy{rao3}}

\begin{EntryWithPhonetic}{扰}{rao3}{7}{⼿}
  \definition*{s.}{Sobrenome: Rao}
  \definition{adj.}{desordenado; bagunçado}
  \definition{v.}{perturbar; importunar; causar problemas | abusar da hospitalidade de alguém}
\end{EntryWithPhonetic}

\begin{EntryWithPhonetic}{扰乱}{rao3luan4}{7,7}{⼿,⼄}[HSK 7-9]
  \definition{v.}{importunar; perturbar; causar confusão; usar palavras ou ações para interromper ou causar caos em um processo em andamento}
\end{EntryWithPhonetic}

%%%%%%%%%% 绕 %%%%%%%%%%
\subsection*{绕}\addcontentsline{loh}{figure}{绕 \dpy{rao4}}

\begin{EntryWithPhonetic}{绕}{rao4}{9}{⽷}[HSK 5]
  \definition*{s.}{Sobrenome: Rao}
  \definition{v.}{enrolar; bobinar; rebobinar | mover"-se em círculo; girar; revolver | fazer um desvio; contornar; dar a volta | confundir; desorientar}
\end{EntryWithPhonetic}

\begin{EntryWithPhonetic}{绕行}{rao4xing2}{9,6}{⽷,⾏}[HSK 7-9]
  \definition{v.}{desviar; contornar | mover"-se em círculo; circular}
\end{EntryWithPhonetic}

%%%%%%%%%% 惹 %%%%%%%%%%
\subsection*{惹}\addcontentsline{loh}{figure}{惹 \dpy{re3}}

\begin{EntryWithPhonetic}{惹}{re3}{12}{⼼}[HSK 7-9]
  \definition{v.}{provocar; convidar ou pedir (algo indesejável); (características de uma pessoa ou coisa) evocar sentimentos de amor ou ódio | ofender; provocar; instigar; (palavras e ações) tocar na outra pessoa | atrair; causar (algo ruim)}
\end{EntryWithPhonetic}

%%%%%%%%%% 热 %%%%%%%%%%
\subsection*{热}\addcontentsline{loh}{figure}{热 \dpy{re4}}

\begin{EntryWithPhonetic}{热}{re4}{10}{⽕}[HSK 1]
  \definition{adj.}{quente; temperatura elevada | ardente; caloroso; profundamente afetuoso | ansioso; invejoso; descreve inveja e desejo de possuir algo | térmico; altamente radioativo | popular; muito procurado; muito apreciado por muitas pessoas}
  \definition{s.}{calor; energia liberada pelo movimento irregular das moléculas dentro de um objeto | febre; febre alta causada por doença | moda passageira; mania; febre}
  \definition{v.}{aquecer (geralmente se refere a alimentos)}
\end{EntryWithPhonetic}

\begin{EntryWithPhonetic}{热爱}{re4'ai4}{10,10}{⽕,⽖}[HSK 3]
  \definition{v.}{amar ardentemente; amar de coração; ter amor profundo por; amar apaixonadamente}
\end{EntryWithPhonetic}

\begin{EntryWithPhonetic}{热潮}{re4chao2}{10,15}{⽕,⽔}[HSK 7-9]
  \definition{s.}{surto; mania; onda de entusiasmo; descreve uma situação próspera e agitada}
\end{EntryWithPhonetic}

\begin{EntryWithPhonetic}{热带}{re4dai4}{10,9}{⽕,⼱}[HSK 7-9]
  \definition{s.}{zona tropical; os trópicos; localizada em ambos os lados da linha do Equador, entre o Trópico de Câncer e o Trópico de Capricórnio, a região apresenta condições climáticas onde a duração do dia e da noite varia pouco com as estações do ano, o clima é quente durante todo o ano e as chuvas são abundantes}
\end{EntryWithPhonetic}

\begin{EntryWithPhonetic}{热点}{re4dian3}{10,9}{⽕,⽕}[HSK 6]
  \definition{s.}{ponto de acesso; \emph{hotspot}}
\end{EntryWithPhonetic}

\begin{EntryWithPhonetic}{热泪盈眶}{re4lei4ying2kuang4}{10,8,9,11}{⽕,⽔,⽫,⽬}
  \definition{expr.}{olhos cheios de lágrimas de emoção | extremamente emocionado}
\end{EntryWithPhonetic}

\begin{EntryWithPhonetic}{热量}{re4liang4}{10,12}{⽕,⾥}[HSK 5]
  \definition{s.}{calor; quantidade de calor; calorias; em física, refere"-se à energia transferida entre objetos com temperaturas diferentes, do objeto com temperatura mais alta para o objeto com temperatura mais baixa}
\end{EntryWithPhonetic}

\begin{EntryWithPhonetic}{热烈}{re4lie4}{10,10}{⽕,⽕}[HSK 3]
  \definition{adj.}{caloroso; fervoroso; ardente; entusiasmado; excitado}
\end{EntryWithPhonetic}

\begin{EntryWithPhonetic}{热门}{re4men2}{10,3}{⽕,⾨}[HSK 5]
  \definition{adj.}{popular; durante um período de tempo, foi algo que interessava a todos}
  \definition{s.}{algo que desperta o interesse popular; metáfora para algo que está na moda e recebe a atenção de todos (em contraste com 冷门)}
  \seealsoref{冷门}{leng3men2}
\end{EntryWithPhonetic}

\begin{EntryWithPhonetic}{热闹}{re4nao5}{10,8}{⽕,⾾}[HSK 4]
  \definition{adj.}{animado; agitado; movimentado com barulho e excitação; descreve uma cena animada com uma atmosfera calorosa}
  \definition{s.}{uma vista emocionante; uma cena de agitação e excitação; atmosfera acolhedora}
  \definition{v.}{animar; divertir"-se}
\end{EntryWithPhonetic}

\begin{EntryWithPhonetic}{热气}{re4qi4}{10,4}{⽕,⽓}[HSK 7-9]
  \definition[股]{s.}{vapor; calor | Coloquial: entusiasmo | gás quente}
\end{EntryWithPhonetic}

\begin{EntryWithPhonetic}{热气球}{re4qi4qiu2}{10,4,11}{⽕,⽓,⽟}[HSK 7-9]
  \definition{s.}{balão de ar quente}[我们坐了热气球升空。===Fizemos um passeio de balão de ar quente.]
\end{EntryWithPhonetic}

\begin{EntryWithPhonetic}{热情}{re4qing2}{10,11}{⽕,⼼}[HSK 2]
  \definition{adj.}{caloroso; fervoroso; entusiasmado; cordial; descreve sentimentos calorosos por alguém}
  \definition{s.}{entusiasmo; ardor; devoção; calor humano; zelo; sentimentos calorosos}
\end{EntryWithPhonetic}

\begin{EntryWithPhonetic}{热水}{re4shui3}{10,4}{⽕,⽔}[HSK 6]
  \definition{s.}{água quente; água em temperatura mais alta}
\end{EntryWithPhonetic}

\begin{EntryWithPhonetic}{热水器}{re4shui3qi4}{10,4,16}{⽕,⽔,⼝}[HSK 6]
  \definition[台]{s.}{aquecedor de água; aparelhos que aquecem água usando eletricidade, gás natural, gás liquefeito de petróleo ou energia solar}
\end{EntryWithPhonetic}

\begin{EntryWithPhonetic}{热腾腾}{re4teng2teng2}{10,13,13}{⽕,⾁,⾁}[HSK 7-9]
  \definition{adj.}{bem quente; fumegando}[桌上放着一碗热腾腾的汤。===Uma tigela de sopa fumegante está sobre a mesa.]
\end{EntryWithPhonetic}

\begin{EntryWithPhonetic}{热线}{re4xian4}{10,8}{⽕,⽷}[HSK 6]
  \definition[条]{s.}{raio infravermelho | linha direta; \emph{hot line}; uma linha telefônica ou telegráfica direta; uma linha para um ponto de acesso | rota quente (ou movimentada, popular) | raio de calor}
\end{EntryWithPhonetic}

\begin{EntryWithPhonetic}{热心}{re4xin1}{10,4}{⽕,⼼}[HSK 4]
  \definition{adj.}{ardente; sincero; entusiasmado; afetuoso; apaixonado; interessado}
  \definition{v.}{ser entusiasmado com alguma coisa}
\end{EntryWithPhonetic}

\begin{EntryWithPhonetic}{热血沸腾}{re4xue4-fei4teng2}{10,6,8,13}{⽕,⾎,⽔,⾁}
  \definition{expr.}{estar animado; ter o sangue correndo}
\end{EntryWithPhonetic}

\begin{EntryWithPhonetic}{热衷}{re4zhong1}{10,10}{⽕,⾐}[HSK 7-9]
  \definition{v.}{gostar de; ter grande apreço por; gostar muito de (uma determinada atividade) | ansiar; desejar ardentemente; buscar avidamente por (fama, fortuna e poder)}
\end{EntryWithPhonetic}

%%%%%%%%%% 人 %%%%%%%%%%
\subsection*{人}\addcontentsline{loh}{figure}{人 \dpy{ren2}}

\begin{EntryWithPhonetic}{人}{ren2}{2}{⼈}[HSK 1][Kangxi 9]
  \definition*{s.}{Sobrenome: Ren}
  \definition[个,名,位]{s.}{homem; pessoa; pessoas; ser humano | todos; cada um; todo mundo | adulto; crescido | uma pessoa envolvida em uma atividade específica | pessoas; outras pessoas | caráter; personalidade; qualidade, caráter ou reputação de uma pessoa | como alguém se sente; estado de saúde de alguém | mão de obra; força de trabalho}
  \synonymref{样}{yang4}
  \antonymref{神}{shen2}
  \antonymref{我}{wo3}
\end{EntryWithPhonetic}

\begin{EntryWithPhonetic}{人才}{ren2cai2}{2,3}{⼈,⼿}[HSK 3]
  \definition{adj.}{aparência bonita, elegante}
  \definition[个,些,位]{s.}{talento; pessoal qualificado; pessoa com capacidade; uma pessoa com capacidade e integridade política; uma pessoa com talentos especiais | aparência bonita; refere"-se à aparência; especialmente à aparência bonita}
\end{EntryWithPhonetic}

\begin{EntryWithPhonetic}{人材}{ren2cai2}{2,7}{⼈,⽊}
  \variantof{人才}
\end{EntryWithPhonetic}

\begin{EntryWithPhonetic}{人次}{ren2ci4}{2,6}{⼈,⽋}[HSK 7-9]
  \definition{clas.}{visitantes; utilizado para o número de total participantes em várias visitas}[参观展览的总共二十万人次。===A exposição atraiu um total de 200.000 visitantes.]
\end{EntryWithPhonetic}

\begin{EntryWithPhonetic}{人道}{ren2dao4}{2,12}{⼈,⾡}[HSK 7-9]
  \definition{s.}{solidariedade humana; humanitarismo | humano | Budismo: ``a maneira humana'', um dos estágios do ciclo de reencarnação | relação sexual}
  \synonymref{人性}{ren2xing4}
\end{EntryWithPhonetic}

\begin{EntryWithPhonetic}{人格}{ren2ge2}{2,10}{⼈,⽊}[HSK 7-9]
  \definition{s.}{caráter; personalidade; individualidade; a soma do caráter, temperamento, habilidades e outras características de uma pessoa | qualidade moral; caráter moral pessoal | entidade jurídica; dignidade humana; as qualificações de uma pessoa para agir como sujeito de direitos e obrigações}
  \synonymref{品德}{pin3de2}
  \synonymref{品行}{pin3xing2}
  \synonymref{人品}{ren2pin3}
\end{EntryWithPhonetic}

\begin{EntryWithPhonetic}{人工}{ren2gong1}{2,3}{⼈,⼯}[HSK 3]
  \definition{adj.}{feito pelo homem; artificial}
  \definition[个]{s.}{trabalho manual; trabalho feito à mão | mão de obra; homem-dia; uma unidade de cálculo da quantidade de trabalho realizado}
  \synonymref{人力}{ren2li4}
  \synonymref{人为}{ren2wei2}
  \synonymref{人造}{ren2zao4}
  \antonymref{华山}{hua4shan1}
  \antonymref{天然}{tian1ran2}
  \antonymref{野生}{ye3sheng1}
\end{EntryWithPhonetic}

\begin{EntryWithPhonetic}{人工智能}{ren2gong1-zhi4neng2}{2,3,12,10}{⼈,⼯,⽇,⾁}[HSK 7-9]
  \definition*{s.}{Inteligência Artificial (IA)}
\end{EntryWithPhonetic}

\begin{EntryWithPhonetic}{人海}{ren2hai3}{2,10}{⼈,⽔}
  \definition{s.}{uma multidão | um mar de pessoas}
\end{EntryWithPhonetic}

\begin{EntryWithPhonetic}{人家}{ren2jia1}{2,10}{⼈,⼧}
  \definition[户,个]{s.}{lar; família; família do noivo; casa do futuro marido}
  \seeref{ren2jia5}
\end{EntryWithPhonetic}

\begin{EntryWithPhonetic}{人家}{ren2jia5}{2,10}{⼈,⼧}[HSK 4]
  \definition{pron.}{outros; uma pessoa ou pessoas diferentes do falante ou ouvinte; refere"-se a alguém diferente de si mesmo ou de outra pessoa | certa pessoa ou pessoas (a pessoa ou pessoas mencionadas em um contexto próximo, aproximadamente equivalente ao pronome de terceira pessoa);  refere"-se a uma pessoa ou algumas pessoas, com significado semelhante a 他 | eu; mim (usado retoricamente no lugar do primeiro pronome pessoal, muitas vezes expressando descontentamento de forma jocosa; geralmente usado quando se fala com pessoas próximas, para significar 自己, usado principamente por meninas)}
  \seeref{ren2jia1}
  \seealsoref{他}{ta1}
  \seealsoref{自己}{zi4ji3}
\end{EntryWithPhonetic}

\begin{EntryWithPhonetic}{人间}{ren2jian1}{2,7}{⼈,⾨}[HSK 5]
  \definition{s.}{o mundo humano; o Mundo; a Terra}
  \antonymref{地狱}{di4yu4}
  \antonymref{天堂}{tian1tang2}
\end{EntryWithPhonetic}

\begin{EntryWithPhonetic}{人均}{ren2jun1}{2,7}{⼈,⼟}[HSK 7-9]
  \definition{adj.}{per capita (ou por pessoa, cabeça)}[人均收入今年有所增长。===A renda per capita aumentou este ano.]
\end{EntryWithPhonetic}

\begin{EntryWithPhonetic}{人口}{ren2kou3}{2,3}{⼈,⼝}[HSK 2]
  \definition[个,群]{s.}{população; o número total de pessoas que vivem em uma determinada região durante um determinado período de tempo | número de membros da família; o número total de pessoas em uma família | pessoas; público; população; referência geral a pessoas | rumores do povo; referindo"-se à opinião pública}
  \synonymref{人手}{ren2shou3}
\end{EntryWithPhonetic}

\begin{EntryWithPhonetic}{人类}{ren2lei4}{2,9}{⼈,⽶}[HSK 3]
  \definition[种]{s.}{humano; humanidade; raça humana; um termo geral para pessoas}
  \synonymref{动物}{dong4wu4}
  \synonymref{人们}{ren2men5}
\end{EntryWithPhonetic}

\begin{EntryWithPhonetic}{人力}{ren2li4}{2,2}{⼈,⼒}[HSK 5]
  \definition{s.}{mão de obra; trabalho manual; força de trabalho}
  \synonymref{人工}{ren2gong1}
\end{EntryWithPhonetic}

\begin{EntryWithPhonetic}{人力车}{ren2li4che1}{2,2,4}{⼈,⼒,⾞}
  \definition{s.}{veículo de duas rodas puxado ou empurrado por um homem | Obsoleto: riquixá | uma carroça puxada ou empurrada por humanos}
  \synonymref{自行车}{zi4xing2che1}
  \antonymref{机动车}{ji1dong4che1}
  \antonymref{兽力车}{shou4li4che1}
\end{EntryWithPhonetic}

\begin{EntryWithPhonetic}{人们}{ren2men5}{2,5}{⼈,⼈}[HSK 2]
  \definition{s.}{homens; pessoas; o público; referindo"-se a muitas pessoas; todos}
  \synonymref{大家}{da4jia1}
  \synonymref{人家}{ren2jia5}
  \synonymref{人类}{ren2lei4}
  \synonymref{人群}{ren2qun2}
\end{EntryWithPhonetic}

\begin{EntryWithPhonetic}{人民}{ren2min2}{2,5}{⼈,⽒}[HSK 3]
  \definition[群,批,个,国]{s.}{o povo; refere"-se a um certo tipo de pessoas; membros básicos da sociedade com as massas trabalhadoras como o corpo principal}
  \synonymref{群众}{qun2zhong4}
  \antonymref{敌人}{di2ren2}
\end{EntryWithPhonetic}

\begin{EntryWithPhonetic}{人民币}{ren2min2bi4}{2,5,4}{⼈,⽒,⼱}[HSK 3]
  \definition*[块,张,元]{s.}{Renminbi (RMB); Yuan Chinês (CYN); nome da moeda chinesa}
\end{EntryWithPhonetic}

\begin{EntryWithPhonetic}{人品}{ren2pin3}{2,9}{⼈,⼝}[HSK 7-9]
  \definition{s.}{caráter; força moral; integridade | Coloquial: aparência; porte; atitude}
  \synonymref{品德}{pin3de2}
  \synonymref{品行}{pin3xing2}
  \synonymref{人格}{ren2ge2}
  \synonymref{为人}{wei2ren2}
\end{EntryWithPhonetic}

\begin{EntryWithPhonetic}{人气}{ren2qi4}{2,4}{⼈,⽓}[HSK 7-9]
  \definition[股,点,些]{s.}{fama; humor; popularidade; sentimento público/aclamação/apoio/confiança/entusiasmo; o grau em que uma pessoa ou coisa é popular | atmosfera animada | qualidade e estilo excelentes; boa personalidade; refere"-se ao caráter de uma pessoa}
\end{EntryWithPhonetic}

\begin{EntryWithPhonetic}{人情}{ren2qing2}{2,11}{⼈,⼼}[HSK 7-9]
  \definition{s.}{sensibilidades; compaixão humana; emoções humanas; as emoções que as pessoas deveriam ter em circunstâncias normais | sentimentos; sensibilidades; relacionamento | etiqueta; costume; cortesia e costumes nas interações interpessoais | bondade; favor | presente; dádiva; um presente oferecido para expressar um determinado sentimento}
\end{EntryWithPhonetic}

\begin{EntryWithPhonetic}{人权}{ren2quan2}{2,6}{⼈,⽊}[HSK 6]
  \definition{s.}{direitos humanos}[最基本的人权是生存权。===O direito humano mais básico é o direito à vida.]
  \seealsoref{人权法}{ren2quan2fa3}
\end{EntryWithPhonetic}

\begin{EntryWithPhonetic}{人权法}{ren2quan2fa3}{2,6,8}{⼈,⽊,⽔}
  \definition*{s.}{Direitos Humanos}
  \seealsoref{人权}{ren2quan2}
\end{EntryWithPhonetic}

\begin{EntryWithPhonetic}{人群}{ren2qun2}{2,13}{⼈,⽺}[HSK 3]
  \definition[个,类]{s.}{multidão; ajuntamento; torpel; aglomeração; um grupo de pessoas}
  \synonymref{人们}{ren2men5}
\end{EntryWithPhonetic}

\begin{EntryWithPhonetic}{人身}{ren2shen1}{2,7}{⼈,⾝}[HSK 7-9]
  \definition[个]{s.}{corpo vivo de um ser humano; pessoa | corpo humano | pessoal}
\end{EntryWithPhonetic}

\begin{EntryWithPhonetic}{人生}{ren2sheng1}{2,5}{⼈,⽣}[HSK 3]
  \definition{s.}{vida; sobrevivência e vida humana}
  \synonymref{存在}{cun2zai4}
  \synonymref{经历}{jing1li4}
  \synonymref{历程}{li4cheng2}
  \synonymref{生命}{sheng1ming4}
  \antonymref{灭亡}{mie4wang2}
  \antonymref{没落}{mo4luo4}
  \antonymref{死亡}{si3wang2}
\end{EntryWithPhonetic}

\begin{EntryWithPhonetic}{人士}{ren2shi4}{2,3}{⼈,⼠}[HSK 5]
  \definition{s.}{pessoa; figura; personalidade; figura pública; pessoas com certa influência social}
\end{EntryWithPhonetic}

\begin{EntryWithPhonetic}{人事}{ren2shi4}{2,8}{⼈,⼅}[HSK 7-9]
  \definition{s.}{assuntos humanos; acontecimentos na vida humana; separação, reencontro, circunstâncias, sobrevivência e morte dos seres humanos | assuntos pessoais; questões relativas a alterações de pessoal dentro de uma unidade, como recrutamento, demissão, promoção, rebaixamento, recompensas e punições, treinamento e transferências | modos de vida; relações humanas e princípios | consciência do mundo exterior; o objeto da consciência humana | o que é humanamente possível; o que os humanos podem fazer | relações de recursos humanos; refere"-se às relações entre pessoas | pessoal}
\end{EntryWithPhonetic}

\begin{EntryWithPhonetic}{人手}{ren2shou3}{2,4}{⼈,⼿}[HSK 7-9]
  \definition{s.}{mão de obra; mão; pessoas que fazem coisas}
  \synonymref{人口}{ren2kou3}
  \synonymref{忍耐}{ren3nai4}
  \synonymref{容忍}{rong2ren3}
\end{EntryWithPhonetic}

\begin{EntryWithPhonetic}{人数}{ren2shu4}{2,13}{⼈,⽁}[HSK 2]
  \definition{s.}{número de pessoas; significa o número total de pessoas, uma quantidade de pessoas; normalmente, usa-se números para fazer estatísticas específicas, mas às vezes também se usa um intervalo aproximado para fazer estimativas}
\end{EntryWithPhonetic}

\begin{EntryWithPhonetic}{人体}{ren2ti3}{2,7}{⼈,⼈}[HSK 7-9]
  \definition{s.}{corpo humano}
  \seealsoref{人身}{ren2shen1}
  \antonymref{雕塑}{diao1su4}
\end{EntryWithPhonetic}

\begin{EntryWithPhonetic}{人为}{ren2wei2}{2,4}{⼈,⼂}[HSK 7-9]
  \definition{adj.}{artificial; feito pelo homem; causado por pessoas (usado para descrever coisas desagradáveis)}
  \definition{v.}{fazer pelo homem; fazer esforço humano; fazer isso com força humana}
  \synonymref{报酬}{bao4chou5}
  \synonymref{人工}{ren2gong1}
  \synonymref{人造}{ren2zao4}
  \antonymref{天然}{tian1ran2}
\end{EntryWithPhonetic}

\begin{EntryWithPhonetic}{人文}{ren2wen2}{2,4}{⼈,⽂}[HSK 7-9]
  \definition{s.}{humanidades; atividades culturais na sociedade humana; originalmente referindo"-se à poesia, livros, ritos e música, posteriormente passou a se referir a vários fenômenos culturais na sociedade humana}
  \synonymref{文化}{wen2hua4}
  \synonymref{文明}{wen2ming2}
  \antonymref{地理}{di4li3}
\end{EntryWithPhonetic}

\begin{EntryWithPhonetic}{人物}{ren2wu4}{2,8}{⼈,⽜}[HSK 5]
  \definition[个,位,名]{s.}{personagem; personagens criados em obras literárias e artísticas | figura; personalidade; homem influente; refere"-se a pessoas com grande talento e status; também se refere a pessoas com certas características ou que são representativas em algum aspecto | pintura figurativa; um tipo de pintura tradicional chinesa com personagens como tema}
  \synonymref{角色}{jue2se4}
  \synonymref{形象}{xing2xiang4}
\end{EntryWithPhonetic}

\begin{EntryWithPhonetic}{人像}{ren2xiang4}{2,13}{⼈,⼈}
  \definition{s.}{``retrato'' de uma pessoa (esboço, foto, escultura, etc.)}
\end{EntryWithPhonetic}

\begin{EntryWithPhonetic}{人行道}{ren2xing2dao4}{2,6,12}{⼈,⾏,⾡}[HSK 7-9]
  \definition{s.}{calçada; as calçadas em ambos os lados da rua são exclusivas para pedestres}
\end{EntryWithPhonetic}

\begin{EntryWithPhonetic}{人性}{ren2xing4}{2,8}{⼈,⼼}[HSK 7-9]
  \definition{s.}{humanidade; natureza humana; as emoções e a razão normais que os seres humanos possuem}
  \synonymref{本性}{ben3xing4}
  \synonymref{人道}{ren2dao4}
\end{EntryWithPhonetic}

\begin{EntryWithPhonetic}{人选}{ren2xuan3}{2,9}{⼈,⾡}[HSK 7-9]
  \definition{s.}{candidato; pessoa selecionada de acordo com determinados critérios}
\end{EntryWithPhonetic}

\begin{EntryWithPhonetic}{人鱼}{ren2yu2}{2,8}{⼈,⿂}
  \definition{s.}{sereia | peixe-boi | salamandra gigante}
\end{EntryWithPhonetic}

\begin{EntryWithPhonetic}{人员}{ren2yuan2}{2,7}{⼈,⼝}[HSK 3]
  \definition[个,位,名]{s.}{funcionários ; uma pessoa que ocupa uma determinada posição | pessoal; membros de um grupo}
  \synonymref{职员}{zhi2yuan2}
\end{EntryWithPhonetic}

\begin{EntryWithPhonetic}{人缘儿}{ren2yuan2r5}{2,12,2}{⼈,⽷,⼉}[HSK 7-9]
  \definition{s.}{relações com as pessoas; popularidade}
\end{EntryWithPhonetic}

\begin{EntryWithPhonetic}{人造}{ren2zao4}{2,10}{⼈,⾡}[HSK 7-9]
  \definition{adj.}{feito pelo homem; artificial; imitação | sintético}
  \synonymref{人工}{ren2gong1}
  \synonymref{人为}{ren2wei2}
  \antonymref{天然}{tian1ran2}
\end{EntryWithPhonetic}

\begin{EntryWithPhonetic}{人质}{ren2zhi4}{2,8}{⼈,⾙}[HSK 7-9]
  \definition[个,名]{s.}{refém; uma das partes detém os pertences da outra parte para obrigá-la a cumprir uma promessa ou aceitar uma condição}
\end{EntryWithPhonetic}

%%%%%%%%%% 仁 %%%%%%%%%%
\subsection*{仁}\addcontentsline{loh}{figure}{仁 \dpy{ren2}}

\begin{EntryWithPhonetic}{仁}{ren2}{4}{⼈}
  \definition*{s.}{Sobrenome: Ren}
  \definition{adj.}{sensível}
  \definition{s.}{benevolência; bondade; generosidade; humanidade | ideal | amêndoa; caroço | carne de camarão | amor; bondade}
  \seealsoref{仁儿}{ren2r5}
  \synonymref{慈}{ci2}
\end{EntryWithPhonetic}

\begin{EntryWithPhonetic}{仁慈}{ren2ci2}{4,13}{⼈,⼼}[HSK 7-9]
  \definition{adj.}{bondoso; misericordioso; benevolente; descreve alguém como muito educado, compassivo e capaz de cuidar e ajudar os outros}
  \synonymref{慈善}{ci2shan4}
  \synonymref{慈祥}{ci2xiang2}
  \synonymref{善良}{shan4liang2}
  \antonymref{残酷}{can2ku4}
  \antonymref{残忍}{can2ren3}
  \antonymref{冷酷}{leng3ku4}
\end{EntryWithPhonetic}

\begin{EntryWithPhonetic}{仁儿}{ren2r5}{4,2}{⼈,⼉}
  \definition{s.}{amêndoa; caroço | carne de camarão}
\end{EntryWithPhonetic}

%%%%%%%%%% 任 %%%%%%%%%%
\subsection*{任}\addcontentsline{loh}{figure}{任 \dpy{ren2}}

\begin{EntryWithPhonetic}{任}{ren2}{6}{⼈}
  \definition*{s.}{Condado de Ren em Hebei (河北) | usado em nomes de lugares, por exemplo, Renxian (任县) e Renqi (任丘) ficam na província de Hebei (河北) | Sobrenome: Ren}
  \seeref{ren4}
\end{EntryWithPhonetic}

%%%%%%%%%% 忍 %%%%%%%%%%
\subsection*{忍}\addcontentsline{loh}{figure}{忍 \dpy{ren3}}

\begin{EntryWithPhonetic}{忍}{ren3}{7}{⼼}[HSK 5]
  \definition{v.}{suportar; aguentar; tolerar; aturar | ter coragem para; ser insensível o suficiente para; ser capaz de endurecer o coração e fazer coisas que não se devem fazer por uma questão de razão}
\end{EntryWithPhonetic}

\begin{EntryWithPhonetic}{忍不住}{ren3bu5zhu4}{7,4,7}{⼼,⼀,⼈}[HSK 5]
  \definition{v.}{incapaz de suportar; não conseguir evitar fazer algo; não conseguir se controlar}
\end{EntryWithPhonetic}

\begin{EntryWithPhonetic}{忍饥挨饿}{ren3ji1-ai2'e4}{7,5,10,10}{⼼,⾷,⼿,⾷}[HSK 7-9]
  \definition{expr.}{``Morrendo de fome.''; suportar os tormentos da fome; famintos}
\end{EntryWithPhonetic}

\begin{EntryWithPhonetic}{忍耐}{ren3nai4}{7,9}{⼼,⽽}[HSK 7-9]
  \definition{v.}{conter-se; exercer paciência (ou contenção); suprimir um determinado sentimento ou emoção para impedir que ele seja expresso}
\end{EntryWithPhonetic}

\begin{EntryWithPhonetic}{忍受}{ren3shou4}{7,8}{⼼,⼜}[HSK 5]
  \definition{v.}{suportar; sofrer; aguentar; tolerar; suportar com dificuldade o sofrimento, as dificuldades e as adversidades da vida}
\end{EntryWithPhonetic}

\begin{EntryWithPhonetic}{忍心}{ren3/xin1}{7,4}{⼼,⼼}[HSK 7-9]
  \definition{v.+compl.}{ter a coragem de; ser suficientemente insensível para; ser capaz de endurecer o coração (fazer coisas que não se suporta fazer)}
\end{EntryWithPhonetic}

%%%%%%%%%% 认 %%%%%%%%%%
\subsection*{认}\addcontentsline{loh}{figure}{认 \dpy{ren4}}

\begin{EntryWithPhonetic}{认}{ren4}{4}{⾔}[HSK 5]
  \definition{v.}{reconhecer; saber; distinguir; identificar | estabelecer uma determinada relação com; adotar | admitir; reconhecer; assumir | comprometer-se a fazer algo | (frequentemente seguido por 了) aceitar como inevitável; resignar-se}
  \seealsoref{了}{le5}
\end{EntryWithPhonetic}

\begin{EntryWithPhonetic}{认出}{ren4 chu1}{4,5}{⾔,⼐}[HSK 3]
  \definition{v.}{reconhecer; identificar; reconhecer alguém ou algo pela observação ou memória}
\end{EntryWithPhonetic}

\begin{EntryWithPhonetic}{认错}{ren4/cuo4}{4,13}{⾔,⾦}[HSK 7-9]
  \definition{v.+compl.}{admitir uma falha; pedir desculpas; reconhecer um erro}
\end{EntryWithPhonetic}

\begin{EntryWithPhonetic}{认得}{ren4 de5}{4,11}{⾔,⼻}[HSK 3]
  \definition{v.}{saber; reconhecer; capacidade de confirmar a pessoa ou coisa que você vê}
\end{EntryWithPhonetic}

\begin{EntryWithPhonetic}{认定}{ren4ding4}{4,8}{⾔,⼧}[HSK 5]
  \definition{v.}{afirmar; manter; acreditar firmemente; considerar com certeza | decidir-se por algo; confirmar; chegar a uma conclusão afirmativa}
\end{EntryWithPhonetic}

\begin{EntryWithPhonetic}{认可}{ren4ke3}{4,5}{⾔,⼝}[HSK 3]
  \definition{v.}{aceitar; aprovar; confirmar; dar força legal a | permitir; concordar}
\end{EntryWithPhonetic}

\begin{EntryWithPhonetic}{认识}{ren4shi5}{4,7}{⾔,⾔}[HSK 1]
  \definition[份]{s.}{cognição; conhecimento; compreensão; refere"-se à reflexão da mente humana sobre o mundo objetivo}
  \definition{v.}{saber; compreender; reconhecer}
\end{EntryWithPhonetic}

\begin{EntryWithPhonetic}{认同}{ren4tong2}{4,6}{⾔,⼝}[HSK 6]
  \definition{v.}{identificar; pensar que a outra pessoa tem algo em comum com você | aprovar; reconhecer}
\end{EntryWithPhonetic}

\begin{EntryWithPhonetic}{认为}{ren4wei2}{4,4}{⾔,⼂}[HSK 2]
  \definition{v.}{pensar; considerar; manter; julgar; formar uma opinião sobre uma pessoa ou coisa, fazer um julgamento}
\end{EntryWithPhonetic}

\begin{EntryWithPhonetic}{认真}{ren4zhen1}{4,10}{⾔,⼗}[HSK 1]
  \definition{adj.}{sério; sério e meticuloso}
  \definition{adv.}{seriamente}
  \definition{v.}{levar algo a sério; considerar como verdadeiro; levar a sério}
\end{EntryWithPhonetic}

\begin{EntryWithPhonetic}{认证}{ren4zheng4}{4,7}{⾔,⾔}[HSK 7-9]
  \definition{v.}{certificar; autenticar; o cartório emitirá uma certidão após verificar a autenticidade dos documentos apresentados pelas partes}
\end{EntryWithPhonetic}

\begin{EntryWithPhonetic}{认知}{ren4zhi1}{4,8}{⾔,⽮}[HSK 7-9]
  \definition{v.}{ter conhecimento; reconhecer; estar ciente de}
\end{EntryWithPhonetic}

%%%%%%%%%% 任 %%%%%%%%%%
\subsection*{任}\addcontentsline{loh}{figure}{任 \dpy{ren4}}

\begin{EntryWithPhonetic}{任}{ren4}{6}{⼈}[HSK 3]
  \definition{clas.}{usado para o número de mandatos cumpridos em um cargo oficial}
  \definition{conj.}{não importa (como, o que, etc.); orações de conexão, ou usadas antes de pronomes interrogativos, para expressar incondicionalidade, equivalente a 不管 ou 无论}
  \definition{s.}{escritório; posto oficial; cargo | dever; fardo; responsabilidade}
  \definition{v.}{nomear; designar alguém para um cargo | assumir um emprego; assumir um posto; assumir uma posição | deixar; permitir; dar rédea solta a | suportar; empreender | ceder; permitir sem restrições; deixar (alguém) fazer o que quiser}
  \seeref{ren2}
  \seealsoref{不管}{bu4guan3}
  \seealsoref{无论}{wu2lun4}
  \antonymref{免}{mian3}
\end{EntryWithPhonetic}

\begin{EntryWithPhonetic}{任何}{ren4he2}{6,7}{⼈,⼈}[HSK 3]
  \definition{pron.}{qualquer; qualquer que seja; o que for; não importa o que}
  \synonymref{所以}{suo3yi3}
  \synonymref{无论}{wu2lun4}
  \synonymref{一切}{yi2qie4}
  \antonymref{唯一}{wei2yi1}
\end{EntryWithPhonetic}

\begin{EntryWithPhonetic}{任命}{ren4ming4}{6,8}{⼈,⼝}[HSK 7-9]
  \definition{v.}{nomear; designar; incumbir; comissionar}
  \synonymref{任职}{ren4/zhi2}
  \antonymref{免职}{mian3/zhi2}
  \antonymref{免除}{mian3chu2}
\end{EntryWithPhonetic}

\begin{EntryWithPhonetic}{任凭}{ren4 ping2}{6,8}{⼈,⼏}
  \definition{conj.}{não importa (como, o quê, etc.) | mesmo que; embora}
  \definition{v.}{permitir; deixar (algo como: fazer o que lhe agrada); conforme a conveniência de alguém}
  \synonymref{听凭}{ting1ping2}
  \antonymref{束缚}{shu4fu4}
\end{EntryWithPhonetic}

\begin{EntryWithPhonetic}{任期}{ren4qi1}{6,12}{⼈,⽉}[HSK 7-9]
  \definition[个,届]{s.}{mandato; duração do mandato; mandato legal}
\end{EntryWithPhonetic}

\begin{EntryWithPhonetic}{任人宰割}{ren4ren2-zai3ge1}{6,2,10,12}{⼈,⼈,⼧,⼑}[HSK 7-9]
  \definition{expr.}{(não pode deixar de) permitir que lhe pisoteiem; ser explorado; ser pisoteado}
\end{EntryWithPhonetic}

\begin{EntryWithPhonetic}{任务}{ren4wu5}{6,5}{⼈,⼒}[HSK 3]
  \definition[项,个,种,些]{s.}{tarefa; dever; missão; designação; trabalho designado; responsabilidades designadas}
  \synonymref{工作}{gong1zuo4}
  \synonymref{使命}{shi3ming4}
  \synonymref{团队}{tuan2dui4}
  \synonymref{义务}{yi4wu4}
  \synonymref{职业}{zhi2ye4}
  \synonymref{职责}{zhi2ze2}
\end{EntryWithPhonetic}

\begin{EntryWithPhonetic}{任意}{ren4yi4}{6,13}{⼈,⼼}[HSK 7-9]
  \definition{adj.}{sem reservas; sem quaisquer condições}
  \definition{adv.}{arbitrariamente; sem restrições, sem limitações, faça o que quiser}
  \synonymref{大肆}{da4si4}
  \synonymref{放肆}{fang4si4}
  \synonymref{随便}{sui2/bian4}
  \synonymref{随意}{sui2/yi4}
  \antonymref{拘束}{ju1shu4}
\end{EntryWithPhonetic}

\begin{EntryWithPhonetic}{任职}{ren4/zhi2}{6,11}{⼈,⽿}[HSK 7-9]
  \definition{v.+compl.}{ocupar um cargo; estar em um cargo de chefia}
  \synonymref{服务}{fu2wu4}
  \synonymref{任命}{ren4ming4}
  \antonymref{离职}{li2/zhi2}
  \antonymref{免职}{mian3/zhi2}
\end{EntryWithPhonetic}

%%%%%%%%%% 韧 %%%%%%%%%%
\subsection*{韧}\addcontentsline{loh}{figure}{韧 \dpy{ren4}}

\begin{EntryWithPhonetic}{韧}{ren4}{7}{⾱}
  \definition{adj.}{flexível, mas forte; tenaz; resistente | resistente; macio e forte, não quebra facilmente}
  \antonymref{脆}{cui4}
\end{EntryWithPhonetic}

\begin{EntryWithPhonetic}{韧性}{ren4xing4}{7,8}{⾱,⼼}[HSK 7-9]
  \definition{s.}{ductilidade; tenacidade; resistência; propriedades de um objeto: macio, porém resistente, e não quebra facilmente | tenacidade; refere"-se a um espírito de perseverança e tenacidade}
\end{EntryWithPhonetic}

%%%%%%%%%% 扔 %%%%%%%%%%
\subsection*{扔}\addcontentsline{loh}{figure}{扔 \dpy{reng1}}

\begin{EntryWithPhonetic}{扔}{reng1}{5}{⼿}[HSK 5]
  \definition{v.}{arremessar; lançar; atirar; jogar | esquecer; jogar fora; descartar | colocar casualmente; deixar as pessoas ou as coisas de lado, não se importar}
\end{EntryWithPhonetic}

\begin{EntryWithPhonetic}{扔掉}{reng1diao4}{5,11}{⼿,⼿}
  \definition{v.}{jogar fora}
\end{EntryWithPhonetic}

\begin{EntryWithPhonetic}{扔弃}{reng1qi4}{5,7}{⼿,⼶}
  \definition{v.}{abandonar | descartar | jogar fora}
\end{EntryWithPhonetic}

\begin{EntryWithPhonetic}{扔下}{reng1xia4}{5,3}{⼿,⼀}
  \definition{v.}{lançar (uma bomba) | derrubar}
\end{EntryWithPhonetic}

%%%%%%%%%% 仍 %%%%%%%%%%
\subsection*{仍}\addcontentsline{loh}{figure}{仍 \dpy{reng2}}

\begin{EntryWithPhonetic}{仍}{reng2}{4}{⼈}[HSK 3]
  \definition{adv.}{ainda; repetidamente; frequentemente; continuamente}
  \definition{v.}{permanecer}
  \synonymref{仍旧}{reng2jiu4}
  \synonymref{仍然}{reng2ran2}
\end{EntryWithPhonetic}

\begin{EntryWithPhonetic}{仍旧}{reng2jiu4}{4,5}{⼈,⽇}[HSK 5]
  \definition{adv.}{ainda; ainda assim; contudo}
  \definition{v.}{permanecer igual; continuar sendo}
  \synonymref{保持}{bao3chi2}
  \synonymref{宝石}{bao3shi2}
  \synonymref{还是}{hai2shi5}
  \synonymref{仍然}{reng2ran2}
  \synonymref{依旧}{yi1jiu4}
  \synonymref{依然}{yi1ran2}
  \synonymref{已经}{yi3jing1}
  \synonymref{照样}{zhao4yang4}
  \antonymref{不再}{bu2zai4}
  \antonymref{改变}{gai3bian4}
\end{EntryWithPhonetic}

\begin{EntryWithPhonetic}{仍然}{reng2ran2}{4,12}{⼈,⽕}[HSK 3]
  \definition{adv.}{ainda; contudo; como antes; indica que a situação continua inalterada ou retorna ao seu estado original}
  \synonymref{还是}{hai2shi5}
  \synonymref{依然}{yi1ran2}
  \synonymref{照样}{zhao4yang4}
  \antonymref{不再}{bu2zai4}
  \antonymref{尚未}{shang4wei4}
\end{EntryWithPhonetic}

%%%%%%%%%% 日 %%%%%%%%%%
\subsection*{日}\addcontentsline{loh}{figure}{日 \dpy{ri4}}

\begin{EntryWithPhonetic}{日}{ri4}{4}{⽇}[HSK 1][Kangxi 72]
  \definition*{s.}{Japão, abreviação de 日本}
  \definition{clas.}{usado para contar o número de dias}
  \definition{s.}{sol | dia; período diurno | diariamente; todos os dias; a cada dia que passa | um dia específico; dia especial | tempo; refere"-se a um período de tempo | dia; uma rotação da Terra}
  \seealsoref{日本}{ri4ben3}
  \antonymref{夜}{ye4}
\end{EntryWithPhonetic}

\begin{EntryWithPhonetic}{日报}{ri4bao4}{4,7}{⽇,⼿}[HSK 2]
  \definition[份,种]{s.}{diário; jornais diários; jornal publicado todas as manhãs}
\end{EntryWithPhonetic}

\begin{EntryWithPhonetic}{日本}{ri4ben3}{4,5}{⽇,⽊}
  \definition*{s.}{Japão}
\end{EntryWithPhonetic}

\begin{EntryWithPhonetic}{日本人}{ri4ben3ren2}{4,5,2}{⽇,⽊,⼈}
  \definition{s.}{japonês | pessoa ou povo do Japão}
\end{EntryWithPhonetic}

\begin{EntryWithPhonetic}{日常}{ri4chang2}{4,11}{⽇,⼱}[HSK 3]
  \definition{adj.}{usual; diário; cotidiano; dia a dia; pertencem ao habitual}
\end{EntryWithPhonetic}

\begin{EntryWithPhonetic}{日程}{ri4cheng2}{4,12}{⽇,⽲}[HSK 7-9]
  \definition[个]{s.}{cronograma; programa; agenda do dia; cronograma diário de tarefas ou atividades}
\end{EntryWithPhonetic}

\begin{EntryWithPhonetic}{日出}{ri4chu1}{4,5}{⽇,⼐}
  \definition{s.}{nascer do sol}
  \seealsoref{夕阳}{xi1yang2}
\end{EntryWithPhonetic}

\begin{EntryWithPhonetic}{日复一日}{ri4fu4yi2ri4}{4,9,1,4}{⽇,⼢,⼀,⽇}[HSK 7-9]
  \definition{expr.}{``Dia a dia.''; dia após dia; realizar uma determinada atividade por vários dias consecutivos}
\end{EntryWithPhonetic}

\begin{EntryWithPhonetic}{日光灯}{ri4guang1deng1}{4,6,6}{⽇,⼉,⽕}
  \definition{s.}{lâmpada fluorescente}
\end{EntryWithPhonetic}

\begin{EntryWithPhonetic}{日后}{ri4hou4}{4,6}{⽇,⼝}[HSK 7-9]
  \definition{s.}{no futuro; nos próximos dias}
\end{EntryWithPhonetic}

\begin{EntryWithPhonetic}{日记}{ri4ji4}{4,5}{⽇,⾔}[HSK 4]
  \definition[本,篇,册]{s.}{diário; artigo que registra eventos e pensamentos diários}
\end{EntryWithPhonetic}

\begin{EntryWithPhonetic}{日历}{ri4li4}{4,4}{⽇,⼚}[HSK 4]
  \definition[张,本]{s.}{caledário; livro com o ano, mês, dia, semana, termo solar, aniversário, etc. registrados, um livro por ano, uma página por dia, aberto diariamente}
\end{EntryWithPhonetic}

\begin{EntryWithPhonetic}{日期}{ri4qi1}{4,12}{⽇,⽉}[HSK 1]
  \definition[个,段]{s.}{data; a data ou período específico em que algo aconteceu}
\end{EntryWithPhonetic}

\begin{EntryWithPhonetic}{日前}{ri4qian2}{4,9}{⽇,⼑}[HSK 7-9]
  \definition{s.}{recentemente; há alguns dias}
\end{EntryWithPhonetic}

\begin{EntryWithPhonetic}{日趋}{ri4qu1}{4,12}{⽇,⾛}[HSK 7-9]
  \definition{adv.}{gradualmente; dia após dia; a cada dia que passa; isso indica que as coisas estão mudando em uma determinada direção a cada dia}[环境污染问题日趋严重。===A poluição ambiental está se tornando cada vez mais grave.]
\end{EntryWithPhonetic}

\begin{EntryWithPhonetic}{日心说}{ri4 xin1 shuo1}{4,4,9}{⽇,⼼,⾔}
  \definition{s.}{teoria heliocêntrica | a teoria de que o sol está no centro do universo}
\end{EntryWithPhonetic}

\begin{EntryWithPhonetic}{日新月异}{ri4xin1-yue4yi4}{4,13,4,6}{⽇,⽄,⽉,⼶}[HSK 7-9]
  \definition{expr.}{``Mudanças rápidas.''; mudar rapidamente; alterar"-se de um dia para o outro; provocar novas mudanças; melhorar a cada dia e a cada mês; prosperar (mudar) a cada dia que passa; mudanças incessantes em; mudanças e melhorias sem fim; mudanças constantes, tanto diárias quanto mensais, o que demonstra um progresso e desenvolvimento rápidos}
\end{EntryWithPhonetic}

\begin{EntryWithPhonetic}{日夜}{ri4ye4}{4,8}{⽇,⼣}[HSK 6]
  \definition{s.}{dia e noite; noite e dia; 24 horas por dia}
\end{EntryWithPhonetic}

\begin{EntryWithPhonetic}{日益}{ri4yi4}{4,10}{⽇,⽫}[HSK 7-9]
  \definition{adv.}{cada vez mais; de modo crescente; dia após dia}
\end{EntryWithPhonetic}

\begin{EntryWithPhonetic}{日语}{ri4yu3}{4,9}{⽇,⾔}[HSK 6]
  \definition{s.}{japonês; língua japonesa}
\end{EntryWithPhonetic}

\begin{EntryWithPhonetic}{日子}{ri4zi5}{4,3}{⽇,⼦}[HSK 2]
  \definition[个,段,些,番]{s.}{dia; data; referência a uma data específica | dias; tempo; referência ao número de dias e horas | vida; subsistência; refere"-se à vida ou ao sustento}
\end{EntryWithPhonetic}

%%%%%%%%%% 荣 %%%%%%%%%%
\subsection*{荣}\addcontentsline{loh}{figure}{荣 \dpy{rong2}}

\begin{EntryWithPhonetic}{荣}{rong2}{9}{⾋}
  \definition*{s.}{Sobrenome: Rong}
  \definition{adj.}{próspero; florescente | exuberante | glorioso}
  \definition{s.}{honra; glória | guarda-sol chinês | flor; flor de planta herbácea | beirais virados para cima}
  \definition{v.}{glorificar; luxuriar; crescer abundantemente; florescer | florescer | lançar}
  \antonymref{辱}{ru3}
\end{EntryWithPhonetic}

\begin{EntryWithPhonetic}{荣获}{rong2huo4}{9,10}{⾋,⾋}[HSK 7-9]
  \definition{v.}{ganhar; ser homenageado com; receber honras}
\end{EntryWithPhonetic}

\begin{EntryWithPhonetic}{荣幸}{rong2xing4}{9,8}{⾋,⼲}[HSK 7-9]
  \definition{adj.}{ser honrado; honroso; glorioso e afortunado}
  \definition[种]{s.}{honra}
\end{EntryWithPhonetic}

\begin{EntryWithPhonetic}{荣誉}{rong2yu4}{9,13}{⾋,⾔}[HSK 7-9]
  \definition[种,份]{s.}{honra; glória; crédito; reputação honrosa; reputação gloriosa}
\end{EntryWithPhonetic}

%%%%%%%%%% 容 %%%%%%%%%%
\subsection*{容}\addcontentsline{loh}{figure}{容 \dpy{rong2}}

\begin{EntryWithPhonetic}{容}{rong2}{10}{⼧}
  \definition*{s.}{Sobrenome: Rong}
  \definition{adv.}{talvez; provavelmente; possivelmente}
  \definition{s.}{expressão facial e tez | aparência; o estado ou condição das coisas}
  \definition{v.}{permitir; quando os outros querem fazer algo, deixe-os fazer | tolerar; ser capaz de aceitar pessoas ou coisas com as quais você não está satisfeito | conter (número de pessoas ou coisas que podem ser colocadas em um determinado espaço)}
\end{EntryWithPhonetic}

\begin{EntryWithPhonetic}{容光焕发}{rong2guang1-huan4fa1}{10,6,11,5}{⼧,⼉,⽕,⼜}[HSK 7-9]
  \definition{expr.}{o rosto radiante de saúde; de bom humor; alegre; só sorrisos; aparência luminosa}
\end{EntryWithPhonetic}

\begin{EntryWithPhonetic}{容量}{rong2liang4}{10,12}{⼧,⾥}[HSK 7-9]
  \definition{s.}{capacidade; a dimensão de um volume é chamada de capacidade; a principal unidade de capacidade no sistema métrico é o litro}
\end{EntryWithPhonetic}

\begin{EntryWithPhonetic}{容貌}{rong2mao4}{10,14}{⼧,⾘}
  \definition{s.}{aparência | aspecto | características}
\end{EntryWithPhonetic}

\begin{EntryWithPhonetic}{容纳}{rong2na4}{10,7}{⼧,⽷}[HSK 7-9]
  \definition{v.}{acomodar; conter; possuir | aceitar (opiniões, etc.); tolerar}
\end{EntryWithPhonetic}

\begin{EntryWithPhonetic}{容忍}{rong2ren3}{10,7}{⼧,⼼}[HSK 7-9]
  \definition{v.}{tolerar; aceitar; suportar}
\end{EntryWithPhonetic}

\begin{EntryWithPhonetic}{容许}{rong2xu3}{10,6}{⼧,⾔}[HSK 7-9]
  \definition{adv.}{Literário: talvez; possivelmente}
  \definition{v.}{tolerar; permitir; autorizar | Literário: admitir; tolerar}
\end{EntryWithPhonetic}

\begin{EntryWithPhonetic}{容颜}{rong2yan2}{10,15}{⼧,⾴}[HSK 7-9]
  \definition{s.}{Literário: aparência; visual; tez}
\end{EntryWithPhonetic}

\begin{EntryWithPhonetic}{容易}{rong2yi4}{10,8}{⼧,⽇}[HSK 3]
  \definition{adj.}{fácil; simples; sem complicações | provável; passível; inclinado; indica uma alta probabilidade de algo acontecer}
\end{EntryWithPhonetic}

%%%%%%%%%% 溶 %%%%%%%%%%
\subsection*{溶}\addcontentsline{loh}{figure}{溶 \dpy{rong2}}

\begin{EntryWithPhonetic}{溶}{rong2}{13}{⽔}
  \definition{v.}{dissolver; solver; solucionar}
\end{EntryWithPhonetic}

\begin{EntryWithPhonetic}{溶解}{rong2jie3}{13,13}{⽔,⾓}[HSK 7-9]
  \definition{v.}{dissolver}[慢慢加热直到糖溶解为止。===Aqueça lentamente até o açúcar dissolver.]
\end{EntryWithPhonetic}

%%%%%%%%%% 融 %%%%%%%%%%
\subsection*{融}\addcontentsline{loh}{figure}{融 \dpy{rong2}}

\begin{EntryWithPhonetic}{融}{rong2}{16}{⿀}[HSK 7-9]
  \definition*{s.}{Sobrenome: Rong}
  \definition{adj.}{permanente; longo prazo; duradouro | muito brilhante | circulante; corrente}
  \definition{s.}{fogo | plena luz do dia}
  \definition{v.}{derreter; descongelar | misturar; fundir; estar em harmonia | circular (dinheiro, etc.)}
\end{EntryWithPhonetic}

\begin{EntryWithPhonetic}{融合}{rong2he2}{16,6}{⿀,⼝}[HSK 6]
  \definition{v.}{fundir; mesclar; misturar; combinar várias coisas diferentes em uma}
\end{EntryWithPhonetic}

\begin{EntryWithPhonetic}{融化}{rong2hua4}{16,4}{⿀,⼔}[HSK 7-9]
  \definition{v.}{derreter; descongelar}
\end{EntryWithPhonetic}

\begin{EntryWithPhonetic}{融洽}{rong2qia4}{16,9}{⿀,⽔}[HSK 7-9]
  \definition{adj.}{harmonioso; em termos amigáveis}
\end{EntryWithPhonetic}

\begin{EntryWithPhonetic}{融入}{rong2ru4}{16,2}{⿀,⼊}[HSK 6]
  \definition{v.}{integrar em; juntar-se, integrar-se ao grupo | misturar-se; enfatizar a mistura e a combinação com o ambiente circundante para se tornar harmonioso e consistente | encher com (um certo sentimento); imbuir com (uma certa qualidade); preparar (chá, ervas, etc.); imergir; infundir (drogas, etc.)}
\end{EntryWithPhonetic}

%%%%%%%%%% 冗 %%%%%%%%%%
\subsection*{冗}\addcontentsline{loh}{figure}{冗 \dpy{rong3}}

\begin{EntryWithPhonetic}{冗}{rong3}{4}{⼍}
  \definition{adj.}{supérfluo; redundante | cheio de detalhes triviais | incapaz}
  \definition{s.}{ocupação; negócio}
\end{EntryWithPhonetic}

\begin{EntryWithPhonetic}{冗长}{rong3chang2}{4,4}{⼍,⾧}[HSK 7-9]
  \definition{adj.}{tediosamente longo; extenso; prolixo; extenso | incômodo}
\end{EntryWithPhonetic}

%%%%%%%%%% 柔 %%%%%%%%%%
\subsection*{柔}\addcontentsline{loh}{figure}{柔 \dpy{rou2}}

\begin{EntryWithPhonetic}{柔}{rou2}{9}{⽊}
  \definition*{s.}{Sobrenome: Rou}
  \definition{adj.}{macio; flexível; maleável | gentil; flexível; brando}
  \definition{v.}{tornar macio; amolecer | apaziguar}
\end{EntryWithPhonetic}

\begin{EntryWithPhonetic}{柔和}{rou2he2}{9,8}{⽊,⼝}[HSK 7-9]
  \definition{adj.}{suave; delicado; ameno; macio}
\end{EntryWithPhonetic}

\begin{EntryWithPhonetic}{柔软}{rou2ruan3}{9,8}{⽊,⾞}[HSK 7-9]
  \definition{adj.}{macio; flexível; maleável}
\end{EntryWithPhonetic}

%%%%%%%%%% 揉 %%%%%%%%%%
\subsection*{揉}\addcontentsline{loh}{figure}{揉 \dpy{rou2}}

\begin{EntryWithPhonetic}{揉}{rou2}{12}{⼿}[HSK 7-9]
  \definition{v.}{esfregar; esfregar ou friccionar com as mãos | amassar; enrolar | Literário: dobrar; torcer}
\end{EntryWithPhonetic}

\begin{EntryWithPhonetic}{揉碎}{rou2sui4}{12,13}{⼿,⽯}
  \definition{v.}{desfazer"-se em pedaços | esmagar}
\end{EntryWithPhonetic}

%%%%%%%%%% 肉 %%%%%%%%%%
\subsection*{肉}\addcontentsline{loh}{figure}{肉 \dpy{rou4}}

\begin{EntryWithPhonetic}{肉}{rou4}{6}{⾁}[HSK 1][Kangxi 130]
  \definition{adj.}{não crocante; mole | lento (em movimento); preguiçoso | carnal; erótico}
  \definition[块]{s.}{carne (especialmente carne de porco) | carne | polpa (da fruta)}
\end{EntryWithPhonetic}

\begin{EntryWithPhonetic}{肉桂}{rou4gui4}{6,10}{⾁,⽊}
  \definition{s.}{canela (árvore) | casca seca desta árvore; canela (uma especiaria aromática) | canela chinesa; cássia}
  \seealsoref{官桂}{guan1gui4}
\end{EntryWithPhonetic}

%%%%%%%%%% 如 %%%%%%%%%%
\subsection*{如}\addcontentsline{loh}{figure}{如 \dpy{ru2}}

\begin{EntryWithPhonetic}{如}{ru2}{6}{⼥}[HSK 6]
  \definition{adv.}{por exemplo; tal como; como}
  \definition{conj.}{se; no caso (de); no caso de; como se; como}
  \definition{prep.}{em conformidade com; de acordo com}
  \definition{v.}{estar em conformidade (ou de acordo) com | (geralmente no negativo) pode ser comparado com; ser comparável a; ser tão bom quanto | superar; exceder | (literário) ir para}
\end{EntryWithPhonetic}

\begin{EntryWithPhonetic}{如此}{ru2ci3}{6,6}{⼥,⽌}[HSK 5]
  \definition{adv.}{assim; tal; dessa forma; dessa maneira; refere"-se a uma situação mencionada anteriormente, equivalente a 这样}
  \seealsoref{这样}{zhe4yang4}
\end{EntryWithPhonetic}

\begin{EntryWithPhonetic}{如果}{ru2guo3}{6,8}{⼥,⽊}[HSK 2]
  \definition{conj.}{se; no caso de; na eventualidade de; supondo que; para expressar suposições, pode"-se usar 要是 na linguagem falada.}
  \seealsoref{要是}{yao4shi5}
\end{EntryWithPhonetic}

\begin{EntryWithPhonetic}{如果说}{ru2guo3 shuo1}{6,8,9}{⼥,⽊,⾔}[HSK 7-9]
  \definition{conj.}{se}[如果说今天没空,就明天见。===Se você estiver ocupado hoje, nos vemos amanhã.]
\end{EntryWithPhonetic}

\begin{EntryWithPhonetic}{如何}{ru2he2}{6,7}{⼥,⼈}[HSK 3]
  \definition{pron.}{como?; o que?; usado para perguntar como resolver um problema | como?; o que?; usado para perguntar sobre a situação ou obter a opinião de outras pessoas}
\end{EntryWithPhonetic}

\begin{EntryWithPhonetic}{如画}{ru2hua4}{6,8}{⼥,⽥}
  \definition{adj.}{pitoresco}
\end{EntryWithPhonetic}

\begin{EntryWithPhonetic}{如今}{ru2jin1}{6,4}{⼥,⼈}[HSK 4]
  \definition{s.}{agora; hoje em dia; atualmente; no presente}
\end{EntryWithPhonetic}

\begin{EntryWithPhonetic}{如实}{ru2shi2}{6,8}{⼥,⼧}[HSK 7-9]
  \definition{adv.}{factualmente; veridicamente; de acordo com a situação real}
\end{EntryWithPhonetic}

\begin{EntryWithPhonetic}{如同}{ru2tong2}{6,6}{⼥,⼝}[HSK 5]
  \definition{v.}{parecer que; usado principalmente em metáforas}
\end{EntryWithPhonetic}

\begin{EntryWithPhonetic}{如下}{ru2xia4}{6,3}{⼥,⼀}[HSK 5]
  \definition{adv.}{como descrito ou listado abaixo; conforme segue; conforme abaixo}
\end{EntryWithPhonetic}

\begin{EntryWithPhonetic}{如一}{ru2yi1}{6,1}{⼥,⼀}[HSK 6]
  \definition{adj.}{consistente; coerente}
\end{EntryWithPhonetic}

\begin{EntryWithPhonetic}{如意}{ru2/yi4}{6,13}{⼥,⼼}[HSK 7-9]
  \definition{adj.}{satisfeito; contente; descreve algo como a realização do desejo do coração}
  \definition{v.+compl.}{estar satisfeito; atender aos desejos de alguém; combinar com o meu gosto}
\end{EntryWithPhonetic}

\begin{EntryWithPhonetic}{如愿以偿}{ru2yuan4yi3chang2}{6,14,4,11}{⼥,⽕,⼈,⼈}[HSK 7-9]
  \definition{expr.}{``Como eu desejava.''; ter um desejo realizado; alcançar (ou obter) o que se deseja; o desejo foi atendido conforme o esperado, ou seja, a aspiração foi realizada}
  \definition{s.}{favorabilidade}
\end{EntryWithPhonetic}

\begin{EntryWithPhonetic}{如醉如痴}{ru2zui4-ru2chi1}{6,15,6,13}{⼥,⾣,⼥,⽧}[HSK 7-9]
  \definition{expr.}{embriagado; como se estivesse embriagado e atordoado; intoxicado por alguma coisa; louco por alguma coisa; obcecado por}
\end{EntryWithPhonetic}

%%%%%%%%%% 儒 %%%%%%%%%%
\subsection*{儒}\addcontentsline{loh}{figure}{儒 \dpy{ru2}}

\begin{EntryWithPhonetic}{儒}{ru2}{16}{⼈}
  \definition*{s.}{Confucionismo; Confucionista | Sobrenome: Ru}
  \definition{s.}{Obsoleto: erudito; homem culto | confucionismo}
\end{EntryWithPhonetic}

\begin{EntryWithPhonetic}{儒家}{ru2jia1}{16,10}{⼈,⼧}[HSK 7-9]
  \definition*{s.}{Confucionismo; Escola Confucionista; Confucionistas; uma corrente de pensamento do período pré"-Qin, representada por Confúcio, que defendia o governo por meio de ritos e enfatizava as relações éticas tradicionais}
\end{EntryWithPhonetic}

\begin{EntryWithPhonetic}{儒教}{ru2jiao4}{16,11}{⼈,⽁}
  \definition{s.}{confucionismo; o confucionismo, a partir das Dinastias do Norte e do Sul, era chamado de confucionismo enquanto religião e era mencionado juntamente com o budismo e o taoísmo}
  \seealsoref{儒家}{ru2jia1}
\end{EntryWithPhonetic}

\begin{EntryWithPhonetic}{儒学}{ru2xue2}{16,8}{⼈,⼦}[HSK 7-9]
  \definition{s.}{confucionismo; ensinamentos confucionistas | Obsoleto: escola administrada pelo governo em nível provincial, de prefeitura ou de condado durante as dinastias Yuan, Ming e Qing}
\end{EntryWithPhonetic}

%%%%%%%%%% 乳 %%%%%%%%%%
\subsection*{乳}\addcontentsline{loh}{figure}{乳 \dpy{ru3}}

\begin{EntryWithPhonetic}{乳}{ru3}{8}{⼄}
  \definition{adj.}{recém-nascido (animal); lactente}
  \definition{s.}{mama; peito | leite (em geral) | qualquer líquido semelhante ao leite}
  \definition{v.}{dar à luz}
\end{EntryWithPhonetic}

\begin{EntryWithPhonetic}{乳房}{ru3fang2}{8,8}{⼄,⼾}
  \definition{s.}{seio | mama | úbere}
\end{EntryWithPhonetic}

\begin{EntryWithPhonetic}{乳制品}{ru3zhi4pin3}{8,8,9}{⼄,⼑,⼝}[HSK 6]
  \definition{s.}{produtos lácteos}
\end{EntryWithPhonetic}

%%%%%%%%%% 辱 %%%%%%%%%%
\subsection*{辱}\addcontentsline{loh}{figure}{辱 \dpy{ru3}}

\begin{EntryWithPhonetic}{辱}{ru3}{10}{⾠}
  \definition*{s.}{Sobrenome: Ru}
  \definition{s.}{desgraça; desonra}
  \definition{v.}{trazer desgraça (ou humilhação) para | trazer desgraça; ser uma desgraça para | estar em dívida (com alguém por uma gentileza) | humilhar; insultar}
  \antonymref{荣}{rong2}
\end{EntryWithPhonetic}

\begin{EntryWithPhonetic}{辱骂}{ru3ma4}{10,9}{⾠,⾺}
  \definition{v.}{insultar | abusar}
\end{EntryWithPhonetic}

%%%%%%%%%% 入 %%%%%%%%%%
\subsection*{入}\addcontentsline{loh}{figure}{入 \dpy{ru4}}

\begin{EntryWithPhonetic}{入}{ru4}{2}{⼊}[HSK 6][Kangxi 11]
  \definition{s.}{renda | tom de entrada}
  \definition{v.}{entrar; entrar | juntar"-se; ser admitido em; tornar"-se membro de | conformar"-se com; concordar com | alcançar; atingir; entrar em (um certo nível ou estado) | fazer entrar; fazer algo entrar; fazer entrada}
  \antonymref{出}{chu1}
\end{EntryWithPhonetic}

\begin{EntryWithPhonetic}{入场}{ru4/chang3}{2,6}{⼊,⼟}[HSK 7-9]
  \definition{v.+compl.}{entrar; ser admitido; entrar no local}
\end{EntryWithPhonetic}

\begin{EntryWithPhonetic}{入场券}{ru4chang3quan4}{2,6,8}{⼊,⼟,⼑}[HSK 7-9]
  \definition{s.}{bilhete (de entrada) | pré-requisito para atingir um objetivo; qualificação para entrar em uma partida | ingresso; admissões}
\end{EntryWithPhonetic}

\begin{EntryWithPhonetic}{入党}{ru4dang3}{2,10}{⼊,⼉}
  \definition{v.}{ingressar em um partido político (especialmente o partido comunista)}
\end{EntryWithPhonetic}

\begin{EntryWithPhonetic}{入境}{ru4/jing4}{2,14}{⼊,⼟}[HSK 7-9]
  \definition{v.+compl.}{entrar em um país; imigrar}
\end{EntryWithPhonetic}

\begin{EntryWithPhonetic}{入口}{ru4/kou3}{2,3}{⼊,⼝}[HSK 2]
  \definition[个]{s.}{entrada; entrada em locais, edifícios, estradas, etc., através de portões ou portas}
  \definition{v.+compl.}{entrar na boca | importar; mercadorias estrangeiras importadas, às vezes também se refere a mercadorias de outras regiões importadas para esta região}
\end{EntryWithPhonetic}

\begin{EntryWithPhonetic}{入门}{ru4/men2}{2,3}{⼊,⾨}[HSK 5]
  \definition{s.}{(geralmente em títulos de livros) curso básico; manual introdutório | ABC; guia; refere"-se a leituras básicas; conhecimentos básicos}
  \definition{v.+compl.}{ultrapassar o limiar; aprender os rudimentos de um assunto | aprender o ABC de; ser introduzido a um assunto; aprender o básico}
\end{EntryWithPhonetic}

\begin{EntryWithPhonetic}{入侵}{ru4qin1}{2,9}{⼊,⼈}[HSK 7-9]
  \definition{v.}{invadir; intrometer"-se; fazer uma incursão; abrir caminho}
\end{EntryWithPhonetic}

\begin{EntryWithPhonetic}{入手}{ru4shou3}{2,4}{⼊,⼿}
  \definition{v.}{começar com; proceder a partir de; tomar como ponto de partida | obter; apoderar"-se  | começar; para dar início}
  \antonymref{出手}{chu1/shou3}
\end{EntryWithPhonetic}

\begin{EntryWithPhonetic}{入乡随俗}{ru4xiang1-sui2su2}{2,3,11,9}{⼊,⼄,⾩,⼈}
  \definition{expr.}{``Em roma, faça como os romanos!''}
\end{EntryWithPhonetic}

\begin{EntryWithPhonetic}{入选}{ru4xuan3}{2,9}{⼊,⾡}[HSK 7-9]
  \definition{v.}{ser escolhido; ser selecionado}
\end{EntryWithPhonetic}

\begin{EntryWithPhonetic}{入学}{ru4/xue2}{2,8}{⼊,⼦}[HSK 6]
  \definition{v.+compl.}{(uma criança) começar a escola; começar a escola primária | entrar em uma escola; matricular-se em uma escola}
\end{EntryWithPhonetic}

%%%%%%%%%% 软 %%%%%%%%%%
\subsection*{软}\addcontentsline{loh}{figure}{软 \dpy{ruan3}}

\begin{EntryWithPhonetic}{软}{ruan3}{8}{⾞}[HSK 5]
  \definition*{s.}{Sobrenome: Ruan}
  \definition{adj.}{macio; flexível; maleável; maleável | suave; brando; delicado | fraco; débil | de baixa qualidade, capacidade, etc. | facilmente movido (ou influenciado) | de maneira suave (ou gentil) | indulgente; tolerante | maleável; flexível | fácil de se emocionar ou abalar}
  \antonymref{硬}{ying4}
\end{EntryWithPhonetic}

\begin{EntryWithPhonetic}{软件}{ruan3jian4}{8,6}{⾞,⼈}[HSK 5]
  \definition[款,个]{s.}{\emph{software}; programas de computador, procedimentos, regras e quaisquer arquivos, documentos e dados relacionados à operação do sistema de computador}
\end{EntryWithPhonetic}

\begin{EntryWithPhonetic}{软弱}{ruan3ruo4}{8,10}{⾞,⼸}[HSK 7-9]
  \definition{adj.}{fraco; descreve o eu interior, a personalidade, etc. de alguém como fraco ou sem força | fraco; débil; flácido; descreve a falta de força física}
\end{EntryWithPhonetic}

\begin{EntryWithPhonetic}{软实力}{ruan3shi2li4}{8,8,2}{⾞,⼧,⼒}[HSK 7-9]
  \definition{s.}{\emph{soft power} (nas relações internacionais)}
\end{EntryWithPhonetic}

%%%%%%%%%% 锐 %%%%%%%%%%
\subsection*{锐}\addcontentsline{loh}{figure}{锐 \dpy{rui4}}

\begin{EntryWithPhonetic}{锐}{rui4}{12}{⾦}
  \definition*{s.}{Sobrenome: Rui}
  \definition{adj.}{afiado; aguçado | agudo; perspicaz | rápido; ágil; veloz}
  \definition{adv.}{rapidamente; de repente}
  \definition{s.}{vigor; espírito de luta | armas afiadas}
  \antonymref{钝}{dun4}
\end{EntryWithPhonetic}

%%%%%%%%%% 瑞 %%%%%%%%%%
\subsection*{瑞}\addcontentsline{loh}{figure}{瑞 \dpy{rui4}}

\begin{EntryWithPhonetic}{瑞}{rui4}{13}{⽟}
  \definition*{s.}{Sobrenome: Rui}
  \definition{adj.}{sortudo; auspicioso}
  \definition{s.}{ficha feita de jade; uma placa de jade usada como símbolo de autoridade e boa fé nos tempos antigos | sinal auspicioso; bom presságio; sorte}
\end{EntryWithPhonetic}

\begin{EntryWithPhonetic}{瑞雪}{rui4xue3}{13,11}{⽟,⾬}[HSK 7-9]
  \definition{s.}{neve oportuna (ou auspiciosa); neve boa e oportuna}
\end{EntryWithPhonetic}

%%%%%%%%%% 润 %%%%%%%%%%
\subsection*{润}\addcontentsline{loh}{figure}{润 \dpy{run4}}

\begin{EntryWithPhonetic}{润}{run4}{10}{⽔}[HSK 7-9]
  \definition{adj.}{úmido; molhado | liso; elegante}
  \definition{s.}{lucro; benefício}
  \definition{v.}{umedecer; lubrificar | embelezar; retocar | retocar; fazer com que fique radiante}
\end{EntryWithPhonetic}

%%%%%%%%%% 若 %%%%%%%%%%
\subsection*{若}\addcontentsline{loh}{figure}{若 \dpy{ruo4}}

\begin{EntryWithPhonetic}{若}{ruo4}{8}{⾋}[HSK 6]
  \definition*{s.}{Sobrenome: Ruo}
  \definition{adv.}{como se; como se fosse; usado antes do verbo para indicar que o que foi dito é mais ou menos assim, equivalente a 好像}
  \definition{conj.}{se; usado na primeira parte de uma frase composta, expressa uma relação hipotética, equivalente a 如果}
  \definition{pron.}{você; referir"-se ao interlocutor como 你 ou 你的}
  \definition{v.}{parecer}
  \seealsoref{好像}{hao3xiang4}
  \seealsoref{你}{ni3}
  \seealsoref{你的}{ni3 de5}
  \seealsoref{如果}{ru2guo3}
\end{EntryWithPhonetic}

\begin{EntryWithPhonetic}{若干}{ruo4gan1}{8,3}{⾋,⼲}[HSK 7-9]
  \definition{pron.}{alguns; vários; um certo número ou quantidade; significam 某些, 有些 ou 一些 | quantos?; quanto custa?}
  \seealsoref{某些}{mou3 xie1}
  \seealsoref{一些}{yi4xie1}
  \seealsoref{有些}{you3xie1}
\end{EntryWithPhonetic}

%%%%%%%%%% 弱 %%%%%%%%%%
\subsection*{弱}\addcontentsline{loh}{figure}{弱 \dpy{ruo4}}

\begin{EntryWithPhonetic}{弱}{ruo4}{10}{⼸}[HSK 4]
  \definition{adj.}{fraco; debilitado | jovem | inferior; pior | colocado depois de uma fração ou decimal para indicar que é um pouco menor que esse número}
  \definition{v.}{perder (através da morte)}
  \antonymref{强}{qiang2}
\end{EntryWithPhonetic}

\begin{EntryWithPhonetic}{弱点}{ruo4dian3}{10,9}{⼸,⽕}[HSK 7-9]
  \definition[个,种]{s.}{fraqueza; ponto fraco; áreas inadequadas; áreas frágeis}
\end{EntryWithPhonetic}

\begin{EntryWithPhonetic}{弱势}{ruo4shi4}{10,8}{⼸,⼒}[HSK 7-9]
  \definition{s.}{uma tendência de queda; o ímpeto está diminuindo ou enfraquecendo | fraqueza; forças fracas}
\end{EntryWithPhonetic}

%%%%% EOF %%%%%

