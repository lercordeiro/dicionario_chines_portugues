%%%
%%% T
%%%
\section*{T}\addcontentsline{toc}{section}{T}\addcontentsline{loh}{figure}{\#\#\#\#\#\#\#\# T}

\begin{EntryWithPhonetic}{T-恤}{t5 xu4}{∅,9}{∅,⼼}
  \definition{s.}{camiseta | pulôver | suéter}
\end{EntryWithPhonetic}

%%%%%%%%%% 他 %%%%%%%%%%
\subsection*{他}\addcontentsline{loh}{figure}{他 \dpy{ta1}}

\begin{EntryWithPhonetic}{他}{ta1}{5}{⼈}[HSK 1]
  \definition{pron.}{ele | outro; referindo"-se a outro; diferente | usado após o verbo, indica referência vaga | alguém; todos; usado em conjunto com 你, significa qualquer pessoa ou muitas pessoas | em outro lugar; outro lugar}
  \seealsoref{你}{ni3}
  \seealsoref{怹}{tan1}
  \antonymref{她}{ta1}
\end{EntryWithPhonetic}

\begin{EntryWithPhonetic}{他的}{ta1 de5}{5,8}{⼈,⽩}
  \definition{pron.}{dele}
\end{EntryWithPhonetic}

\begin{EntryWithPhonetic}{他妈的}{ta1ma1de5}{5,6,8}{⼈,⼥,⽩}
  \definition{interj.}{Palavrão tabu: ``Dane-se!''; ``Droga!''; ``Que se dane!''}
\end{EntryWithPhonetic}

\begin{EntryWithPhonetic}{他们}{ta1men5}{5,5}{⼈,⼈}[HSK 1]
  \definition{pron.}{eles}
  \synonymref{常常}{chang2chang2}
  \synonymref{你们}{ni3men5}
  \synonymref{她们}{ta1men5}
  \synonymref{我们}{wo3men5}
  \synonymref{相信}{xiang1xin4}
\end{EntryWithPhonetic}

\begin{EntryWithPhonetic}{他们的}{ta1men5 de5}{5,5,8}{⼈,⼈,⽩}
  \definition{pron.}{deles}
\end{EntryWithPhonetic}

\begin{EntryWithPhonetic}{他人}{ta1ren2}{5,2}{⼈,⼈}[HSK 7-9]
  \definition[任]{pron.}{outros; outras pessoas; outra pessoa}
  \synonymref{别人}{bie2ren5}
  \synonymref{对方}{dui4fang1}
  \antonymref{本人}{ben3ren2}
  \antonymref{自己}{zi4ji3}
  \antonymref{自我}{zi4wo3}
\end{EntryWithPhonetic}

%%%%%%%%%% 它 %%%%%%%%%%
\subsection*{它}\addcontentsline{loh}{figure}{它 \dpy{ta1}}

\begin{EntryWithPhonetic}{它}{ta1}{5}{⼧}[HSK 2]
  \definition*{s.}{Sobrenome: Ta}
  \definition{pron.}{ele; referência a algo além da pessoa (para objetos inanimados) | ele; usado após o verbo, indica referência vaga}
\end{EntryWithPhonetic}

\begin{EntryWithPhonetic}{它们}{ta1men5}{5,5}{⼧,⼈}[HSK 2]
  \definition{pron.}{eles; usado para se referir a mais de uma coisa não humana; geralmente se refere a animais, objetos ou conceitos abstratos}
\end{EntryWithPhonetic}

%%%%%%%%%% 她 %%%%%%%%%%
\subsection*{她}\addcontentsline{loh}{figure}{她 \dpy{ta1}}

\begin{EntryWithPhonetic}{她}{ta1}{6}{⼥}[HSK 1]
  \definition{pron.}{ela | ela; referir-se a coisas que se ama ou aprecia, como a pátria, a bandeira nacional, etc.}
\end{EntryWithPhonetic}

\begin{EntryWithPhonetic}{她的}{ta1 de5}{6,8}{⼥,⽩}
  \definition{pron.}{dela}
\end{EntryWithPhonetic}

\begin{EntryWithPhonetic}{她们}{ta1men5}{6,5}{⼥,⼈}[HSK 1]
  \definition{pron.}{elas; referindo"-se a várias mulheres: em textos escritos, use 她们 quando todas as pessoas forem mulheres e 他们 quando houver homens e mulheres}
  \seealsoref{他们}{ta1men5}
\end{EntryWithPhonetic}

\begin{EntryWithPhonetic}{她们的}{ta1men5 de5}{6,5,8}{⼥,⼈,⽩}
  \definition{pron.}{delas}
\end{EntryWithPhonetic}

%%%%%%%%%% 塌 %%%%%%%%%%
\subsection*{塌}\addcontentsline{loh}{figure}{塌 \dpy{ta1}}

\begin{EntryWithPhonetic}{塌}{ta1}{13}{⼟}[HSK 7-9]
  \definition{v.}{colapsar; desabar; cair; ruir | afundar; tombar; cair ou afundar (aquilo que estava sendo sustentado) | acalmar"-se; tranquilizar"-se; estabilizar | Dialeto: (moral, etc.) desmoronar; afundar | murchar; curvar; cair de cara no chão}
\end{EntryWithPhonetic}

%%%%%%%%%% 踏 %%%%%%%%%%
\subsection*{踏}\addcontentsline{loh}{figure}{踏 \dpy{ta1}}

\begin{EntryWithPhonetic}{踏}{ta1}{15}{⾜}
  \definition{part.}{Caracter formador de palavras}
  \seeref{ta4}
\end{EntryWithPhonetic}

\begin{EntryWithPhonetic}{踏上}{ta1shang5}{15,3}{⾜,⼀}[HSK 7-9]
  \definition{v.}{pôr os pés em; pisar; pisar em | pisar em ou entrar; começar a entrar ou se envolver em uma área}
\end{EntryWithPhonetic}

\begin{EntryWithPhonetic}{踏实}{ta1shi5}{15,8}{⾜,⼧}[HSK 6]
  \definition{adj.}{confiável; sério; estável e seguro; descreve uma atitude séria em relação ao trabalho ou estudo | à vontade; livre de ansiedade; descreve uma mente ou sentimento estável, sem qualquer preocupação ou ansiedade}
\end{EntryWithPhonetic}

%%%%%%%%%% 塔 %%%%%%%%%%
\subsection*{塔}\addcontentsline{loh}{figure}{塔 \dpy{ta3}}

\begin{EntryWithPhonetic}{塔}{ta3}{12}{⼟}[HSK 6]
  \definition*{s.}{Sobrenome: Ta}
  \definition[个,座]{s.}{pagode budista; pagode | torre | (química) coluna; torre}[蒸馏塔===torre de destilação]
\end{EntryWithPhonetic}

%%%%%%%%%% 拓 %%%%%%%%%%
\subsection*{拓}\addcontentsline{loh}{figure}{拓 \dpy{ta4}}

\begin{EntryWithPhonetic}{拓}{ta4}{8}{⼿}
  \definition{v.}{fazer decalques de formas, textos e gráficos em inscrições em pedra, artefatos de bronze, etc.}
  \seeref{tuo4}
\end{EntryWithPhonetic}

%%%%%%%%%% 踏 %%%%%%%%%%
\subsection*{踏}\addcontentsline{loh}{figure}{踏 \dpy{ta4}}

\begin{EntryWithPhonetic}{踏}{ta4}{15}{⾜}[HSK 6]
  \definition{v.}{por os pés em; pisar em; esmagar com o pé | fazer uma investigação ou levantamento no local}
  \seeref{ta1}
\end{EntryWithPhonetic}

\begin{EntryWithPhonetic}{踏板}{ta4ban3}{15,8}{⾜,⽊}
  \definition{s.}{pedal (em um carro, em um piano, etc.) |  apoio para os pés | estribo}
\end{EntryWithPhonetic}

%%%%%%%%%% 胎 %%%%%%%%%%
\subsection*{胎}\addcontentsline{loh}{figure}{胎 \dpy{tai1}}

\begin{EntryWithPhonetic}{胎}{tai1}{9}{⾁}[HSK 7-9]
  \definition{s.}{feto; embrião | nascimento | enchimento; acolchoamento; estofamento | reboco (na fabricação de porcelana, cloisonné, etc.) | pneu | base; reboco; os espaços em branco de certos objetos}
  \seealsoref{胎儿}{tai1'er2}
\end{EntryWithPhonetic}

\begin{EntryWithPhonetic}{胎儿}{tai1'er2}{9,2}{⾁,⼉}[HSK 7-9]
  \definition{s.}{embrião; feto; criança não nascida}
\end{EntryWithPhonetic}

%%%%%%%%%% 台 %%%%%%%%%%
\subsection*{台}\addcontentsline{loh}{figure}{台 \dpy{tai2}}

\begin{EntryWithPhonetic}{台}{tai2}{5}{⼝}[HSK 3]
  \definition*{s.}{Sobrenome: Tai}
  \definition{clas.}{usado para certas máquinas, aparelhos, instrumentos, etc | usado para uma performance completa, como drama, música e dança}
  \definition{s.}{torre | plataforma; palco | suporte; pedestal | qualquer coisa em forma de plataforma ou palco | mesa; escrivaninha | estação de transmissão; refere"-se a estações de rádio | um serviço telefônico especial; refere"-se à estação telefônica | ``seu'', um termo respeitoso usado antigamente para se dirigir a alguém | tufão}
\end{EntryWithPhonetic}

\begin{EntryWithPhonetic}{台灯}{tai2deng1}{5,6}{⼝,⽕}[HSK 6]
  \definition[个,盏]{s.}{luminária de mesa; luminária de leitura; uma luminária com base para uso sobre uma mesa}
\end{EntryWithPhonetic}

\begin{EntryWithPhonetic}{台风}{tai2feng1}{5,4}{⼝,⾵}[HSK 5]
  \definition[场,阵,级]{s.}{tufão; classificação de um ciclone tropical ocorrido no oeste do Pacífico Norte | postura; presença de palco; comportamento ou estilo que os atores demonstram no palco}
\end{EntryWithPhonetic}

\begin{EntryWithPhonetic}{台阶}{tai2jie1}{5,6}{⼝,⾩}[HSK 4]
  \definition[个,级]{s.}{escada; escadaria | passos; metáfora para uma maneira ou oportunidade de evitar constrangimentos causados por um impasse | nova fase; novo nível; novo patamar; metáfora para novas conquistas ou novos patamares alcançados no estudo ou no trabalho}
\end{EntryWithPhonetic}

\begin{EntryWithPhonetic}{台球}{tai2qiu2}{5,11}{⼝,⽟}[HSK 7-9]
  \definition[场,个,颗]{s.}{bilhar; um esporte com bola praticado em uma mesa especialmente projetada, utilizando bastões de madeira dura para golpear uma bola | bola de bilhar; a bola sólida usada no bilhar é feita de materiais resistentes, como plástico, e tem cerca de sete centímetros de diâmetro}
\end{EntryWithPhonetic}

\begin{EntryWithPhonetic}{台上}{tai2shang4}{5,3}{⼝,⼀}[HSK 4]
  \definition{s.}{no palco}
\end{EntryWithPhonetic}

\begin{EntryWithPhonetic}{台下}{tai2xia4}{5,3}{⼝,⼀}
  \definition{s.}{platéia | fora do palco}
\end{EntryWithPhonetic}

%%%%%%%%%% 抬 %%%%%%%%%%
\subsection*{抬}\addcontentsline{loh}{figure}{抬 \dpy{tai2}}

\begin{EntryWithPhonetic}{抬}{tai2}{8}{⼿}[HSK 5]
  \definition{clas.}{usado para objetos que precisam ser carregados por pessoas quando transportados (por exemplo, móveis)}
  \definition{v.}{levantar; elevar; puxar para cima | (por duas ou mais pessoas) carregar; transportar; duas ou mais pessoas carregando algo com as mãos ou nos ombros | discutir, debater (geralmente sem sentido ou sem importância)}
\end{EntryWithPhonetic}

\begin{EntryWithPhonetic}{抬杠}{tai2/gang4}{8,7}{⼿,⽊}
  \definition{v.+compl.}{brigar; discutir; discutir por discutir; discutir sobre o certo e o errado (geralmente sem princípios) | Arcaico: carregar um caixão em barras resistentes}
\end{EntryWithPhonetic}

\begin{EntryWithPhonetic}{抬头}{tai2 tou2}{8,5}{⼿,⼤}[HSK 5]
  \definition{s.}{(em recibos, contas, etc.) nome do comprador ou beneficiário, o local no documento onde o nome do beneficiário ou destinatário é escrito}
  \definition{v.}{levantar a cabeça}
\end{EntryWithPhonetic}

%%%%%%%%%% 太 %%%%%%%%%%
\subsection*{太}\addcontentsline{loh}{figure}{太 \dpy{tai4}}

\begin{EntryWithPhonetic}{太}{tai4}{4}{⼤}[HSK 1]
  \definition*{s.}{Sobrenome: Tai}
  \definition{adj.}{mais alto; maior; mais distante | maior; extremo | bisavô; mais velho ou mais antigo; o de maior posição social ou hierarquia}
  \definition{adv.}{demais; expressa um grau excessivo (usado principalmente para coisas indesejáveis) | muito; extremamente; excessivamente; indica um grau extremamente elevado | muito; usado após o advérbio negativo 不, enfraquece o grau de negação e contém um tom diplomático}
\end{EntryWithPhonetic}

\begin{EntryWithPhonetic}{太极}{tai4ji2}{4,7}{⼤,⽊}[HSK 7-9]
  \definition*{s.}{Filosofia: Tai Chi; Supremo Último, o Absoluto na antiga cosmologia chinesa, apresentado como a fonte primária de todas as coisas criadas (万物)}
  \seealsoref{万物}{wan4wu4}
\end{EntryWithPhonetic}

\begin{EntryWithPhonetic}{太极拳}{tai4ji2quan2}{4,7,10}{⼤,⽊,⼿}[HSK 7-9]
  \definition*[套]{s.}{Tai Chi Chuan, Taiji, T'aichi ou T'aichichuan; uma arte marcial chinesa originária do início da Dinastia Qing; seus movimentos são suaves, lentos e fluidos; acredita"-se que ela aprimore o condicionamento físico e promova a saúde}
\end{EntryWithPhonetic}

\begin{EntryWithPhonetic}{太空}{tai4kong1}{4,8}{⼤,⽳}[HSK 5]
  \definition[把]{s.}{firmamento; espaço sideral; espaço além da atmosfera terrestre; o céu vasto e infinito}
\end{EntryWithPhonetic}

\begin{EntryWithPhonetic}{太平}{tai4ping2}{4,5}{⼤,⼲}[HSK 7-9]
  \definition*{s.}{Província de Thai Binh, nome de um local no norte do Vietnã, uma das províncias do Vietnã}
  \definition{adj.}{seguro e protegido; pacífico e tranquilo; boa ordem social sem guerra; estabilidade (social, nacional); paz}
  \synonymref{安定}{an1ding4}
\end{EntryWithPhonetic}

\begin{EntryWithPhonetic}{太平洋}{tai4ping2 yang2}{4,5,9}{⼤,⼲,⽔}
  \definition*{s.}{Oceano Pacífico}
\end{EntryWithPhonetic}

\begin{EntryWithPhonetic}{太太}{tai4tai5}{4,4}{⼤,⼤}[HSK 2]
  \definition[位,名,个,些]{s.}{senhora; madame; títulos para mulheres casadas | esposa; senhora; madame; referir-se à própria esposa ou à esposa de outra pessoa}
\end{EntryWithPhonetic}

\begin{EntryWithPhonetic}{太阳}{tai4yang2}{4,6}{⼤,⾩}[HSK 2]
  \definition[个,轮,枚,颗,盏]{s.}{o Sol | luz do sol; luz solar}
\end{EntryWithPhonetic}

\begin{EntryWithPhonetic}{太阳窗}{tai4yang2chuang1}{4,6,12}{⼤,⾩,⽳}
  \definition{s.}{teto solar (de veículos)}
\end{EntryWithPhonetic}

\begin{EntryWithPhonetic}{太阳灯}{tai4yang2deng1}{4,6,6}{⼤,⾩,⽕}
  \definition{s.}{lâmpada solar (com células fotovoltaicas)}
\end{EntryWithPhonetic}

\begin{EntryWithPhonetic}{太阳风}{tai4yang2feng1}{4,6,4}{⼤,⾩,⾵}
  \definition{s.}{vento solar}
\end{EntryWithPhonetic}

\begin{EntryWithPhonetic}{太阳镜}{tai4yang2jing4}{4,6,16}{⼤,⾩,⾦}
  \definition{s.}{óculos de sol}
\end{EntryWithPhonetic}

\begin{EntryWithPhonetic}{太阳能}{tai4yang2neng2}{4,6,10}{⼤,⾩,⾁}[HSK 6]
  \definition{s.}{energia solar; a energia de radiação emitida pelo Sol é a energia solar produzida pela reação de fusão dos núcleos de hidrogênio no Sol, é a fonte de luz e calor na Terra}
\end{EntryWithPhonetic}

\begin{EntryWithPhonetic}{太阳日}{tai4yang2ri4}{4,6,4}{⼤,⾩,⽇}
  \definition{s.}{dia solar}
\end{EntryWithPhonetic}

\begin{EntryWithPhonetic}{太阳穴}{tai4yang2xue2}{4,6,5}{⼤,⾩,⽳}
  \definition{s.}{têmpora (nas laterais da cabeça humana)}
\end{EntryWithPhonetic}

\begin{EntryWithPhonetic}{太阳翼}{tai4yang2yi4}{4,6,17}{⼤,⾩,⽻}
  \definition{s.}{painel solar}
\end{EntryWithPhonetic}

\begin{EntryWithPhonetic}{太阳雨}{tai4yang2yu3}{4,6,8}{⼤,⾩,⾬}
  \definition{s.}{banho de sol}
\end{EntryWithPhonetic}

%%%%%%%%%% 态 %%%%%%%%%%
\subsection*{态}\addcontentsline{loh}{figure}{态 \dpy{tai4}}

\begin{EntryWithPhonetic}{态}{tai4}{8}{⼼}
  \definition{s.}{forma; aparência; condição | (física) estado | (linguística) voz}[气态===estado gasoso | 被动态===voz passiva]
\end{EntryWithPhonetic}

\begin{EntryWithPhonetic}{态度}{tai4du5}{8,9}{⼼,⼴}[HSK 2]
  \definition[种,个]{s.}{maneira; comportamento; atitude; comportamento e expressão facial das pessoas | atitude; abordagem; opinião sobre o assunto e medidas tomadas}
\end{EntryWithPhonetic}

%%%%%%%%%% 泰 %%%%%%%%%%
\subsection*{泰}\addcontentsline{loh}{figure}{泰 \dpy{tai4}}

\begin{EntryWithPhonetic}{泰}{tai4}{10}{⽔}
  \definition*{s.}{Tai, um dos Sessenta Diagramas | Tailândia, abreviação de 泰国 | Sobrenome: Tai}
  \definition{adj.}{seguro; pacífico | Literário: extremo; demasiado | calmo | arrogante}
  \definition{adv.}{extremamente; o mais; demais}
  \seealsoref{泰国}{tai4guo2}
\end{EntryWithPhonetic}

\begin{EntryWithPhonetic}{泰斗}{tai4dou3}{10,4}{⽔,⽃}[HSK 7-9]
  \definition[位]{s.}{grão-mestre; uma autoridade líder; eminente estudioso, músico, artista, etc.}
\end{EntryWithPhonetic}

\begin{EntryWithPhonetic}{泰国}{tai4guo2}{10,8}{⽔,⼞}
  \definition*{s.}{Tailândia}
\end{EntryWithPhonetic}

\begin{EntryWithPhonetic}{泰山}{tai4shan1}{10,3}{⽔,⼭}
  \definition*{s.}{Monte Tai na província de Shandong, montanha oriental das Cinco Montanhas Sagradas (五岳); na antiguidade, era considerado um representante das altas montanhas e frequentemente usado para simbolizar pessoas veneradas e coisas de grande importância e valor}
  \definition{s.}{filha da esposa; sogro}
  \seealsoref{五岳}{wu3yue4}
\end{EntryWithPhonetic}

%%%%%%%%%% 贪 %%%%%%%%%%
\subsection*{贪}\addcontentsline{loh}{figure}{贪 \dpy{tan1}}

\begin{EntryWithPhonetic}{贪}{tan1}{8}{⾙}[HSK 7-9]
  \definition{v.}{apropriar-se indevidamente; desviar fundos; praticar corrupção; ser corrupto; originalmente, referia-se ao amor ao dinheiro; posteriormente, passou a ser associado à corrupção | ter um desejo insaciável por; ter um desejo voraz por | cobiçar; ansiar por; ser ganancioso por}
\end{EntryWithPhonetic}

\begin{EntryWithPhonetic}{贪婪}{tan1lan2}{8,11}{⾙,⼥}[HSK 7-9]
  \definition{adj.}{ganancioso; avarento; descreve uma pessoa ou animal como alguém que nunca está satisfeito | ganancioso; ávido; descreve alguém que nunca está satisfeito com coisas boas como conhecimento e ar puro}
  \definition{s.}{ganância}
  \synonymref{海量}{hai3liang4}
  \antonymref{满足}{man3zu2}
\end{EntryWithPhonetic}

\begin{EntryWithPhonetic}{贪玩儿}{tan1wan2r5}{8,8,2}{⾙,⽟,⼉}[HSK 7-9]
  \definition{s.}{brincalhão}
  \definition{v.}{ser brincalhão}
\end{EntryWithPhonetic}

\begin{EntryWithPhonetic}{贪污}{tan1wu1}{8,6}{⾙,⽔}[HSK 7-9]
  \definition{v.}{desviar fundos; obter ilegalmente terras, dinheiro ou propriedade estatal, coletiva ou unitária, abusando do poder ou da posição}
  \synonymref{腐败}{fu3bai4}
  \antonymref{廉洁}{lian2jie2}
\end{EntryWithPhonetic}

%%%%%%%%%% 怹 %%%%%%%%%%
\subsection*{怹}\addcontentsline{loh}{figure}{怹 \dpy{tan1}}

\begin{EntryWithPhonetic}{怹}{tan1}{9}{⼼}
  \definition{pron.}{ele, ela (cortês)}
  \synonymref{他}{ta1}
\end{EntryWithPhonetic}

%%%%%%%%%% 摊 %%%%%%%%%%
\subsection*{摊}\addcontentsline{loh}{figure}{摊 \dpy{tan1}}

\begin{EntryWithPhonetic}{摊}{tan1}{13}{⼿}[HSK 7-9]
  \definition{clas.}{utilizado para pastas ou líquidos espalhados}[路边是一摊泥。===Havia uma poça de lama à beira da estrada.]
  \definition{s.}{banca de vendedor; barraca; quiosque; pontos de venda instalados ao longo das estradas ou em praças}
  \definition{v.}{espalhar; estender | cozinhar em uma camada fina e uniforme | partilhar; dividir; repartir; compartilhar | (algo desagradável) acontecer; ocorrer a; encontrar-se com; deparar-se com}
\end{EntryWithPhonetic}

%%%%%%%%%% 瘫 %%%%%%%%%%
\subsection*{瘫}\addcontentsline{loh}{figure}{瘫 \dpy{tan1}}

\begin{EntryWithPhonetic}{瘫}{tan1}{15}{⽧}[HSK 7-9]
  \definition{adj.}{paralisado}
  \definition{s.}{paralisia}
  \definition{v.}{ficar fisicamente paralisado}
\end{EntryWithPhonetic}

\begin{EntryWithPhonetic}{瘫痪}{tan1huan4}{15,12}{⽧,⽧}[HSK 7-9]
  \definition{v.}{ficar paralisado; devido a disfunções neurológicas, uma parte do corpo pode perder total ou parcialmente a capacidade de se mover | entrar em colapso; parar completamente; essa metáfora descreve uma organização desorganizada e incapaz de funcionar adequadamente}
  \synonymref{残疾}{can2ji5}
  \antonymref{健康}{jian4kang1}
\end{EntryWithPhonetic}

%%%%%%%%%% 坛 %%%%%%%%%%
\subsection*{坛}\addcontentsline{loh}{figure}{坛 \dpy{tan2}}

\begin{EntryWithPhonetic}{坛}{tan2}{7}{⼟}[HSK 7-9]
  \definition{s.}{altar | terreno elevado para plantio de flores, etc. | plataforma; fórum | círculos; mundo | Obsoleto: organização criada por uma sociedade secreta para adorar deuses em uma manifestação | jarro de barro; cântaro; tonel | garrafão; projetado especialmente para armazenar líquidos corrosivos, como o ácido sulfúrico}
  \seealsoref{坛儿}{tan2r5}
\end{EntryWithPhonetic}

\begin{EntryWithPhonetic}{坛儿}{tan2r5}{7,2}{⼟,⼉}
  \definition{s.}{jarro de barro; cântaro; tonel}
\end{EntryWithPhonetic}

%%%%%%%%%% 谈 %%%%%%%%%%
\subsection*{谈}\addcontentsline{loh}{figure}{谈 \dpy{tan2}}

\begin{EntryWithPhonetic}{谈}{tan2}{10}{⾔}[HSK 3]
  \definition*{s.}{Sobrenome: Tan}
  \definition{s.}{o que é dito ou falado; discurso}
  \definition{v.}{falar; bater papo; discutir}
\end{EntryWithPhonetic}

\begin{EntryWithPhonetic}{谈不上}{tan2 bu5 shang4}{10,4,3}{⾔,⼀,⼀}[HSK 7-9]
  \definition{expr.}{nem pensar; não conta, longe de atingir um certo nível; não merecedor de; não digno de ser chamado}
\end{EntryWithPhonetic}

\begin{EntryWithPhonetic}{谈到}{tan2dao4}{10,8}{⾔,⼑}[HSK 7-9]
  \definition{v.}{falar de (sobre); comentar sobre (algo relacionado a); referir"-se a}
\end{EntryWithPhonetic}

\begin{EntryWithPhonetic}{谈话}{tan2/hua4}{10,8}{⾔,⾔}[HSK 3]
  \definition[次]{s.}{declaração; opiniões (principalmente políticas) expressas na forma de conversas}
  \definition{v.+compl.}{conversar; discutir | falar; refere"-se especificamente ao uso da conversa para entender a situação, fazer trabalho ideológico, etc. (usado principalmente por superiores para subordinados)}
\end{EntryWithPhonetic}

\begin{EntryWithPhonetic}{谈恋爱}{tan2lian4'ai4}{10,10,10}{⾔,⼼,⽖}
  \definition{v.}{namorar | apaixonar-se}
\end{EntryWithPhonetic}

\begin{EntryWithPhonetic}{谈论}{tan2lun4}{10,6}{⾔,⾔}[HSK 7-9]
  \definition{v.}{debater; discutir; falar sobre; conversar sobre}
  \synonymref{辩论}{bian4lun4}
  \synonymref{带来}{dai4 lai2}
  \synonymref{评论}{ping2lun4}
  \synonymref{讨论}{tao3lun4}
  \synonymref{议论}{yi4lun4}
  \antonymref{沉默}{chen2mo4}
\end{EntryWithPhonetic}

\begin{EntryWithPhonetic}{谈判}{tan2pan4}{10,7}{⾔,⼑}[HSK 3]
  \definition{v.}{negociar; manter conversações; para resolver um grande problema, as partes relevantes trocaram opiniões entre si, na esperança de encontrar uma solução com a qual todos pudessem concordar}
\end{EntryWithPhonetic}

\begin{EntryWithPhonetic}{谈起}{tan2qi3}{10,10}{⾔,⾛}[HSK 7-9]
  \definition{v.}{mencionar; falar de}
\end{EntryWithPhonetic}

%%%%%%%%%% 弹 %%%%%%%%%%
\subsection*{弹}\addcontentsline{loh}{figure}{弹 \dpy{tan2}}

\begin{EntryWithPhonetic}{弹}{tan2}{11}{⼸}[HSK 5]
  \definition{v.}{enviar; atirar (como com uma catapulta, etc.); usar a elasticidade de um objeto para lançar outro objeto | afofar; preparar fibras; usar um dispositivo elástico para amolecer as fibras | virar; sacudir | dedilhar; tocar (um instrumento musical de cordas) | acusar; atacar; criticar; relatar | saltar; quicar}
  \seeref{dan4}
\end{EntryWithPhonetic}

\begin{EntryWithPhonetic}{弹性}{tan2xing4}{11,8}{⼸,⼼}[HSK 7-9]
  \definition{adj.}{elástico; flexível; essa metáfora descreve a natureza das coisas, que são ajustáveis ​​e adaptáveis ​​de acordo com as necessidades reais}
  \definition{s.}{elasticidade; resiliência; a propriedade de um objeto se deformar sob a ação de uma força externa e retornar à sua forma original após a remoção dessa força}
  \synonymref{韧性}{ren4xing4}
\end{EntryWithPhonetic}

%%%%%%%%%% 痰 %%%%%%%%%%
\subsection*{痰}\addcontentsline{loh}{figure}{痰 \dpy{tan2}}

\begin{EntryWithPhonetic}{痰}{tan2}{13}{⽧}[HSK 7-9]
  \definition[口]{s.}{catarro; escarro; muco secretado pelos alvéolos e pela traqueia}
\end{EntryWithPhonetic}

%%%%%%%%%% 忐 %%%%%%%%%%
\subsection*{忐}\addcontentsline{loh}{figure}{忐 \dpy{tan3}}

\begin{EntryWithPhonetic}{忐}{tan3}{7}{⼼}
  \definition{adj.}{nervoso; inquieto; agitado}
\end{EntryWithPhonetic}

%%%%%%%%%% 坦 %%%%%%%%%%
\subsection*{坦}\addcontentsline{loh}{figure}{坦 \dpy{tan3}}

\begin{EntryWithPhonetic}{坦}{tan3}{8}{⼟}
  \definition*{s.}{Sobrenome: Tan}
  \definition{adj.}{nivelado; suave; plano | calmo; composto | aberto; sincero; franco}
\end{EntryWithPhonetic}

\begin{EntryWithPhonetic}{坦白}{tan3bai2}{8,5}{⼟,⽩}[HSK 7-9]
  \definition{adj.}{honesto; franco; sincero; puros de coração}
  \definition{v.}{ser franco; confessar; dizer a verdade sobre erros ou crimes}
  \synonymref{坦率}{tan3shuai4}
  \antonymref{抗拒}{kang4ju4}
  \antonymref{通知}{tong1zhi1}
\end{EntryWithPhonetic}

\begin{EntryWithPhonetic}{坦诚}{tan3cheng2}{8,8}{⼟,⾔}[HSK 7-9]
  \definition{adj.}{franco e sincero; franco e aberto}
  \synonymref{诚恳}{cheng2ken3}
  \synonymref{坦率}{tan3shuai4}
  \synonymref{真诚}{zhen1cheng2}
  \antonymref{撒谎}{sa1/huang3}
\end{EntryWithPhonetic}

\begin{EntryWithPhonetic}{坦克}{tan3ke4}{8,7}{⼟,⼗}[HSK 7-9]
  \definition[辆]{s.}{Empréstimo linguístico: tanque (veículo militar); veículos de combate blindados sobre esteiras, equipados com canhões, metralhadoras e torres giratórias}
\end{EntryWithPhonetic}

\begin{EntryWithPhonetic}{坦然}{tan3ran2}{8,12}{⼟,⽕}[HSK 7-9]
  \definition{adj.}{calmo; sereno; imperturbável; sem receios; descreve um estado de espírito como calmo e sem preocupações}
  \synonymref{安心}{an1xin1}
  \synonymref{平静}{ping2jing4}
  \antonymref{狼狈}{lang2bei4}
  \antonymref{受惊}{shou4/jing1}
\end{EntryWithPhonetic}

\begin{EntryWithPhonetic}{坦率}{tan3shuai4}{8,11}{⼟,⽞}[HSK 7-9]
  \definition{adj.}{1. sincero; franco; direto}
  \synonymref{爽快}{shuang3kuai5}
  \synonymref{坦白}{tan3bai2}
  \synonymref{坦诚}{tan3cheng2}
  \antonymref{通知}{tong1zhi1}
  \antonymref{委婉}{wei3wan3}
\end{EntryWithPhonetic}

%%%%%%%%%% 毯 %%%%%%%%%%
\subsection*{毯}\addcontentsline{loh}{figure}{毯 \dpy{tan3}}

\begin{EntryWithPhonetic}{毯}{tan3}{12}{⽑}
  \definition[条]{s.}{cobertor; tapete; carpete}
\end{EntryWithPhonetic}

\begin{EntryWithPhonetic}{毯子}{tan3zi5}{12,3}{⽑,⼦}[HSK 7-9]
  \definition[条,张,床,面]{s.}{cobertor; tecidos grossos e macios podem ser usados ​​como cortinas, capas ou decorações}
  \synonymref{被子}{bei4zi5}
  \synonymref{垫子}{dian4zi5}
\end{EntryWithPhonetic}

%%%%%%%%%% 叹 %%%%%%%%%%
\subsection*{叹}\addcontentsline{loh}{figure}{叹 \dpy{tan4}}

\begin{EntryWithPhonetic}{叹}{tan4}{5}{⼝}
  \definition{v.}{suspirar | exclamar com admiração; aclamar; louvar |recitar com cadência; entoar cântico; entoar}
\end{EntryWithPhonetic}

\begin{EntryWithPhonetic}{叹气}{tan4/qi4}{5,4}{⼝,⽓}[HSK 6]
  \definition{v.+compl.}{suspirar; soltar um suspiro; soltar um longo suspiro e fazer um som devido à insatisfação ou desamparo}
\end{EntryWithPhonetic}

%%%%%%%%%% 炭 %%%%%%%%%%
\subsection*{炭}\addcontentsline{loh}{figure}{炭 \dpy{tan4}}

\begin{EntryWithPhonetic}{炭}{tan4}{9}{⽕}[HSK 7-9]
  \definition*{s.}{Sobrenome: Tan}
  \definition[块,吨]{s.}{carvão; algo semelhante a carvão}
  \antonymref{冰}{bing1}
\end{EntryWithPhonetic}

%%%%%%%%%% 探 %%%%%%%%%%
\subsection*{探}\addcontentsline{loh}{figure}{探 \dpy{tan4}}

\begin{EntryWithPhonetic}{探}{tan4}{11}{⼿}[HSK 7-9]
  \definition[个,位,名]{s.}{batedor; espião; detetive}
  \definition{v.}{tentar descobrir; explorar; soar | explorar; espionar | visitar; fazer uma visita em | se destacar | preocupar-se com; envolver-se em | ver; invocar}
\end{EntryWithPhonetic}

\begin{EntryWithPhonetic}{探测}{tan4ce4}{11,9}{⼿,⽔}[HSK 7-9]
  \definition{v.}{sondar; examinar; explorar; utilizar ferramentas para observar ou medir coisas que não podem ser observadas ou medidas diretamente}
\end{EntryWithPhonetic}

\begin{EntryWithPhonetic}{探亲}{tan4/qin1}{11,9}{⼿,⼇}[HSK 7-9]
  \definition{v.+compl.}{ir para casa para visitar a família; visitar parentes em outra cidade (geralmente referindo-se à visita aos pais ou cônjuge)}
\end{EntryWithPhonetic}

\begin{EntryWithPhonetic}{探求}{tan4qiu2}{11,7}{⼿,⽔}[HSK 7-9]
  \definition{v.}{procurar; perseguir; buscar; explorar e buscar}
  \synonymref{考虑}{kao3lv4}
  \synonymref{商量}{shang1liang5}
  \synonymref{搜索}{sou1suo3}
  \synonymref{探索}{tan4suo3}
  \synonymref{推测}{tui1ce4}
  \synonymref{寻求}{xun2qiu2}
  \synonymref{寻找}{xun2zhao3}
  \synonymref{研究}{yan2jiu1}
  \synonymref{追求}{zhui1qiu2}
\end{EntryWithPhonetic}

\begin{EntryWithPhonetic}{探索}{tan4suo3}{11,10}{⼿,⽷}[HSK 6]
  \definition{v.}{sondar; explorar; procurar respostas de várias fontes para resolver dúvidas}
\end{EntryWithPhonetic}

\begin{EntryWithPhonetic}{探讨}{tan4tao3}{11,5}{⼿,⾔}[HSK 6]
  \definition{v.}{examinar; indagar; investigar; discutir}
\end{EntryWithPhonetic}

\begin{EntryWithPhonetic}{探望}{tan4wang4}{11,11}{⼿,⽉}[HSK 7-9]
  \definition{v.}{olhar ao redor; observar (tentar descobrir o que está acontecendo) | ver; visitar; fazer uma visita a alguém (geralmente de longe)}
  \synonymref{拜访}{bai4fang3}
  \synonymref{访问}{fang3wen4}
  \synonymref{看望}{kan4wang5}
  \antonymref{回避}{hui2bi4}
\end{EntryWithPhonetic}

\begin{EntryWithPhonetic}{探险}{tan4/xian3}{11,9}{⼿,⾩}[HSK 7-9]
  \definition{v.+compl.}{explorar; fazer explorações; aventurar"-se no desconhecido; investigar lugares onde ninguém jamais esteve ou onde pouquíssimas pessoas estiveram (o mundo natural)}
  \synonymref{冒险}{mao4/xian3}
  \synonymref{探求}{tan4qiu2}
\end{EntryWithPhonetic}

%%%%%%%%%% 碳 %%%%%%%%%%
\subsection*{碳}\addcontentsline{loh}{figure}{碳 \dpy{tan4}}

\begin{EntryWithPhonetic}{碳}{tan4}{14}{⽯}[HSK 7-9]
  \definition[块]{s.}{C; carbono (elemento químico)}
\end{EntryWithPhonetic}

\begin{EntryWithPhonetic}{碳足迹}{tan4 zu2ji4}{14,7,9}{⽯,⾜,⾡}
  \definition{s.}{pegada de carbono}
\end{EntryWithPhonetic}

%%%%%%%%%% 汤 %%%%%%%%%%
\subsection*{汤}\addcontentsline{loh}{figure}{汤 \dpy{tang1}}

\begin{EntryWithPhonetic}{汤}{tang1}{6}{⽔}[HSK 3]
  \definition*{s.}{Sobrenome: Tang}
  \definition[勺,碗,杯,锅]{s.}{água quente; água fervente | fontes termais | água utilizada para ferver algo| sopa; caldo | uma preparação líquida de ervas medicinais; decocção}
  \seeref{shang1}
\end{EntryWithPhonetic}

\begin{EntryWithPhonetic}{汤圆}{tang1yuan2}{6,10}{⽔,⼞}[HSK 7-9]
  \definition{s.}{bolinho doce; (geralmente bolinhos recheados) feitos de farinha de arroz glutinoso servidos em sopa}
  \synonymref{元宵}{yuan2xiao1}
\end{EntryWithPhonetic}

%%%%%%%%%% 趟 %%%%%%%%%%
\subsection*{趟}\addcontentsline{loh}{figure}{趟 \dpy{tang1}}

\begin{EntryWithPhonetic}{趟}{tang1}{15}{⾛}
  \definition{v.}{atravessar; andar na grama ou onde não haja caminho | usar arados, capinadores, etc. para virar o solo e remover ervas daninhas | vadear; atravessar a vau; caminhar por águas rasas}[我们趟水去那小岛。===Nós vadeamos até a ilha.]
  \seeref{tang4}
\end{EntryWithPhonetic}

%%%%%%%%%% 唐 %%%%%%%%%%
\subsection*{唐}\addcontentsline{loh}{figure}{唐 \dpy{tang2}}

\begin{EntryWithPhonetic}{唐}{tang2}{10}{⼝}
  \definition*{s.}{Dinastia estabelecida pelo Imperador Yao, 尧, no período lendário da história chinesa | Dinastia Tang (618-907) | Dinastia Tang posterior (923-936), uma das cinco dinastias | Sobrenome: Tang}
  \definition{adj.}{exagerado; bombástico; orgulhoso | em vão; por nada}
  \seealsoref{尧}{yao2}
\end{EntryWithPhonetic}

\begin{EntryWithPhonetic}{唐初四大家}{tang2 chu1 si4 da4jia1}{10,7,5,3,10}{⼝,⾐,⼞,⼤,⼧}
  \definition*{s.}{Quatro grandes calígrafos do início da dinastia Tang; refere"-se a Yu Shi'nan (虞世南), Ouyang Xun (欧阳询), Chu Suiliang (褚遂良) e Xue Ji (薛稷)}
  \seealsoref{褚遂良}{chu3 sui4liang2}
  \seealsoref{欧阳询}{ou1yang2 xun2}
  \seealsoref{薛稷}{xue1 ji4}
  \seealsoref{虞世南}{yu2 shi4nan2}
\end{EntryWithPhonetic}

\begin{EntryWithPhonetic}{唐人街}{tang2ren2 jie1}{10,2,12}{⼝,⼈,⾏}
  \definition*[条,座]{s.}{Bairro Chinês; Chinatown; refere"-se ao mercado de rua onde os chineses do exterior vivem e abrem muitas lojas com características chinesas}
  \seealsoref{中国城}{zhong1guo2cheng2}
\end{EntryWithPhonetic}

%%%%%%%%%% 堂 %%%%%%%%%%
\subsection*{堂}\addcontentsline{loh}{figure}{堂 \dpy{tang2}}

\begin{EntryWithPhonetic}{堂}{tang2}{11}{⼟}[HSK 7-9]
  \definition*{s.}{Sobrenome: Tang}
  \definition{clas.}{classe; uma turma é dividida em seções; uma seção é chamada de classe | caso; um julgamento de cada vez é chamado de sessão judicial | conjunto; conjuntos de móveis | utilizado para cenas; murais, etc.}
  \definition[节,门]{s.}{sala; cômodos principais; hall como símbolo das casas principais no sistema familiar | tribunal; antigamente, era um local onde se realizavam cerimônias em repartições públicas; um local para audiências judiciais | placa da loja; o nome de uma loja; utilizado para a identidade visual da loja | mãe; pais; salão interno; metaforicamente referindo-se à mãe | do mesmo clã; parentesco entre primos, etc., do mesmo avô paterno ou bisavô | salão; casas projetadas especificamente para uma determinada atividade}[我参观了三槐堂。===Visitei o Salão Sanhuaitang.]
\end{EntryWithPhonetic}

%%%%%%%%%% 糖 %%%%%%%%%%
\subsection*{糖}\addcontentsline{loh}{figure}{糖 \dpy{tang2}}

\begin{EntryWithPhonetic}{糖}{tang2}{16}{⽶}[HSK 3]
  \definition[包,斤,勺,袋,块]{s.}{açúcar; um tipo de açúcar; um tipo de composto orgânico, que pode ser dividido em três tipos: monossacarídeos, dissacarídeos e polissacarídeos; é a principal substância que produz energia térmica no corpo humano, como glicose, sacarose, lactose, amido, etc. | açúcar; açúcar comestível; termo geral para açúcar | doces; balas | carboidrato; algo doce e calórico}
\end{EntryWithPhonetic}

\begin{EntryWithPhonetic}{糖醋鱼}{tang2cu4yu2}{16,15,8}{⽶,⾣,⿂}
  \definition{s.}{peixe guisado em molho agridoce (prato)}
\end{EntryWithPhonetic}

\begin{EntryWithPhonetic}{糖果}{tang2guo3}{16,8}{⽶,⽊}[HSK 7-9]
  \definition[个,颗,包,袋]{s.}{doce; alimentos açucarados, que geralmente contêm suco de frutas, especiarias, leite ou café, etc.}
\end{EntryWithPhonetic}

\begin{EntryWithPhonetic}{糖尿病}{tang2niao4bing4}{16,7,10}{⽶,⼫,⽧}[HSK 7-9]
  \definition{s.}{diabetes; diabetes mellitus; doença crônica causada pela secreção insuficiente de insulina, levando a distúrbios no metabolismo da glicose e níveis elevados de açúcar no sangue}[糖尿病会带来严重的问题。===O diabetes pode causar problemas graves.]
\end{EntryWithPhonetic}

%%%%%%%%%% 倘 %%%%%%%%%%
\subsection*{倘}\addcontentsline{loh}{figure}{倘 \dpy{tang3}}

\begin{EntryWithPhonetic}{倘}{tang3}{10}{⼈}
  \definition{conj.}{se; supondo; no caso}
  \seeref{chang2}
\end{EntryWithPhonetic}

\begin{EntryWithPhonetic}{倘或}{tang3huo4}{10,8}{⼈,⼽}
  \definition{conj.}{se | supondo que | no caso}
\end{EntryWithPhonetic}

\begin{EntryWithPhonetic}{倘若}{tang3ruo4}{10,8}{⼈,⾋}[HSK 7-9]
  \definition{conj.}{se; caso; supondo que; indica uma hipótese e geralmente corresponde a palavras como 那, 那么 e 就}[倘若她不同意呢?===E se ela discordar?]
  \seealsoref{就}{jiu4}
  \seealsoref{那}{na4}
  \seealsoref{那么}{na4me5}
  \synonymref{假使}{jia3shi3}
\end{EntryWithPhonetic}

\begin{EntryWithPhonetic}{倘使}{tang3shi3}{10,8}{⼈,⼈}
  \definition{conj.}{se | supondo que | no caso}
\end{EntryWithPhonetic}

%%%%%%%%%% 淌 %%%%%%%%%%
\subsection*{淌}\addcontentsline{loh}{figure}{淌 \dpy{tang3}}

\begin{EntryWithPhonetic}{淌}{tang3}{11}{⽔}[HSK 7-9]
  \definition{v.}{gotejar; escorrer; pingar | fluir para baixo}
  \synonymref{流}{liu2}
\end{EntryWithPhonetic}

%%%%%%%%%% 躺 %%%%%%%%%%
\subsection*{躺}\addcontentsline{loh}{figure}{躺 \dpy{tang3}}

\begin{EntryWithPhonetic}{躺}{tang3}{15}{⾝}[HSK 4]
  \definition{v.}{deitar; reclinar; cair no chão ou sobre um objeto}
\end{EntryWithPhonetic}

%%%%%%%%%% 烫 %%%%%%%%%%
\subsection*{烫}\addcontentsline{loh}{figure}{烫 \dpy{tang4}}

\begin{EntryWithPhonetic}{烫}{tang4}{10}{⽕}[HSK 7-9]
  \definition{adj.}{escaldante; muito quente; alta temperatura do objeto}
  \definition{v.}{queimar; escaldar; objetos em alta temperatura causam dor ao entrarem em contato com a pele | passar a ferro; prensar; aquecer; aquecer em água quente; utilizar um objeto mais quente para aumentar a temperatura de outro objeto ou provocar outras alterações | fazer permanente (cabelo)}
\end{EntryWithPhonetic}

%%%%%%%%%% 趟 %%%%%%%%%%
\subsection*{趟}\addcontentsline{loh}{figure}{趟 \dpy{tang4}}

\begin{EntryWithPhonetic}{趟}{tang4}{15}{⾛}[HSK 6]
  \definition{clas.}{usado para o número de vezes de viagens de ida e volta |  usado para coisas dispostas em fileiras ou tiras | usado para a programação de veículos, navios, etc. que circulam em uma determinada ordem | usado em conjuntos de movimentos de artes marciais}
  \definition{s.}{marcha; procissão; jornada; viagem}
  \seeref{tang1}
\end{EntryWithPhonetic}

%%%%%%%%%% 掏 %%%%%%%%%%
\subsection*{掏}\addcontentsline{loh}{figure}{掏 \dpy{tao1}}

\begin{EntryWithPhonetic}{掏}{tao1}{11}{⼿}[HSK 6]
  \definition{v.}{extrair; retirar; pescar | cavar (um buraco, etc.); escavar; retirar | (coloquial) roubar do bolso de alguém | tirar}
\end{EntryWithPhonetic}

\begin{EntryWithPhonetic}{掏钱}{tao1 qian2}{11,10}{⼿,⾦}[HSK 7-9]
  \definition{v.}{pagar; pagar dinheiro; gastar dinheiro}
\end{EntryWithPhonetic}

%%%%%%%%%% 滔 %%%%%%%%%%
\subsection*{滔}\addcontentsline{loh}{figure}{滔 \dpy{tao1}}

\begin{EntryWithPhonetic}{滔}{tao1}{13}{⽔}
  \definition{adj.}{(de água) transbordando | arrogante | turbulento | largo e longo; grande}
  \definition{v.}{inundar; alagar}
\end{EntryWithPhonetic}

\begin{EntryWithPhonetic}{滔滔不绝}{tao1tao1-bu4jue2}{13,13,4,9}{⽔,⽔,⼀,⽷}[HSK 7-9]
  \definition{expr.}{tagarelando sem parar; falando sem parar; como um fluxo contínuo de água, continua sem parar.; frequentemente usado para descrever alguém que fala muito e nunca para}
  \synonymref{夸夸其谈}{kua1kua1-qi2tan2}
\end{EntryWithPhonetic}

\begin{EntryWithPhonetic}{滔天}{tao1tian1}{13,4}{⽔,⼤}
  \definition{adj.}{(ondas, raiva, desastres, crimes, etc.) imponente, avassalador, imenso}
\end{EntryWithPhonetic}

%%%%%%%%%% 逃 %%%%%%%%%%
\subsection*{逃}\addcontentsline{loh}{figure}{逃 \dpy{tao2}}

\begin{EntryWithPhonetic}{逃}{tao2}{9}{⾡}[HSK 5]
  \definition{v.}{fugir; escapar; correr; dar no pé | evadir; esquivar-se; escapar}
\end{EntryWithPhonetic}

\begin{EntryWithPhonetic}{逃避}{tao2bi4}{9,16}{⾡,⾌}[HSK 7-9]
  \definition{v.}{esquivar-se; evadir-se; escapar; evitar coisas que você não quer ou não se atreve a tocar}
  \synonymref{躲避}{duo3bi4}
  \synonymref{躲藏}{duo3cang2}
  \synonymref{回避}{hui2bi4}
  \synonymref{隐藏}{yin3cang2}
  \antonymref{尝试}{chang2shi4}
  \antonymref{交锋}{jiao1/feng1}
  \antonymref{经受}{jing1shou4}
  \antonymref{面对}{mian4dui4}
\end{EntryWithPhonetic}

\begin{EntryWithPhonetic}{逃跑}{tao2pao3}{9,12}{⾡,⾜}[HSK 5]
  \definition{v.}{fugir; escapar; correr; partir para fugir de um ambiente ou de coisas que não lhe são favoráveis}
\end{EntryWithPhonetic}

\begin{EntryWithPhonetic}{逃生}{tao2sheng1}{9,5}{⾡,⽣}[HSK 7-9]
  \definition{v.}{escapar; fugir para salvar a vida; escapar com vida; escapar de um ambiente perigoso para sobreviver}
  \synonymref{脱离}{tuo1li2}
\end{EntryWithPhonetic}

\begin{EntryWithPhonetic}{逃亡}{tao2wang2}{9,3}{⾡,⼇}[HSK 7-9]
  \definition{v.}{fugir de casa; ir para o exílio; tornar"-se um fugitivo}
  \synonymref{避难}{bi4/nan4}
  \synonymref{流浪}{liu2lang4}
\end{EntryWithPhonetic}

\begin{EntryWithPhonetic}{逃走}{tao2 zou3}{9,7}{⾡,⾛}[HSK 5]
  \definition{v.}{escapar; afastar-se de pessoas, coisas ou lugares que não são bons para você ou que você não gosta}
\end{EntryWithPhonetic}

%%%%%%%%%% 桃 %%%%%%%%%%
\subsection*{桃}\addcontentsline{loh}{figure}{桃 \dpy{tao2}}

\begin{EntryWithPhonetic}{桃}{tao2}{10}{⽊}[HSK 5]
  \definition*{s.}{Sobrenome: Tao}
  \definition[个,箱,袋,斤,棵,种]{s.}{pêssego | em forma de pêssego | pessegueiro}
\end{EntryWithPhonetic}

\begin{EntryWithPhonetic}{桃花}{tao2hua1}{10,7}{⽊,⾋}[HSK 5]
  \definition[朵,枝,株]{s.}{Figurativo: caso amoroso | flor de pessegueiro}
\end{EntryWithPhonetic}

\begin{EntryWithPhonetic}{桃树}{tao2shu4}{10,9}{⽊,⽊}[HSK 5]
  \definition[棵,株]{s.}{pêssego (árvore) | pessegueiro; pêssegos}
\end{EntryWithPhonetic}

%%%%%%%%%% 陶 %%%%%%%%%%
\subsection*{陶}\addcontentsline{loh}{figure}{陶 \dpy{tao2}}

\begin{EntryWithPhonetic}{陶}{tao2}{10}{⾩}
  \definition*{s.}{Sobrenome: Tao}
  \definition{adj.}{satisfeito; feliz | de barro; feito de argila}
  \definition{s.}{cerâmica; louça}
  \definition{v.}{fazer cerâmica (louça de barro) | cultivar; moldar; educar}
  \seeref{yao2}
\end{EntryWithPhonetic}

\begin{EntryWithPhonetic}{陶瓷}{tao2ci2}{10,10}{⾩,⽡}[HSK 7-9]
  \definition[件,套,个]{s.}{cerâmica; louça e porcelana; termo geral para cerâmica e porcelana}
\end{EntryWithPhonetic}

\begin{EntryWithPhonetic}{陶冶}{tao2ye3}{10,7}{⾩,⼎}[HSK 7-9]
  \definition{v.}{fazer cerâmica e fundir metal | exercer uma influência favorável (no caráter de uma pessoa, etc.); moldar}
  \synonymref{锻炼}{duan4lian4}
  \synonymref{训练}{xun4lian4}
\end{EntryWithPhonetic}

\begin{EntryWithPhonetic}{陶醉}{tao2zui4}{10,15}{⾩,⾣}[HSK 7-9]
  \definition{v.}{deleitar"-se com; embriagar"-se (de felicidade, etc.); estar imerso em um determinado estado ou sentimento}
  \synonymref{沉迷}{chen2mi2}
  \synonymref{迷恋}{mi2lian4}
\end{EntryWithPhonetic}

%%%%%%%%%% 淘 %%%%%%%%%%
\subsection*{淘}\addcontentsline{loh}{figure}{淘 \dpy{tao2}}

\begin{EntryWithPhonetic}{淘}{tao2}{11}{⽔}[HSK 7-9]
  \definition{adj.}{travesso}
  \definition{v.}{lavar em uma bacia ou cesto; enxaguar | limpar; dragar; retirar com concha; recolher; retirar esgoto, etc. | causar problemas; ser um fardo para a mente; perturbar o cérebro com algo}
  \seealsoref{掏}{tao1}
\end{EntryWithPhonetic}

\begin{EntryWithPhonetic}{淘气}{tao2/qi4}{11,4}{⽔,⽓}[HSK 7-9]
  \definition*{v.+compl.}{Dialeto: ficar com raiva; perder a paciência}
  \definition{adj.}{travesso; malicioso; descreve uma criança como particularmente brincalhona e travessa}
  \synonymref{调皮}{tiao2pi2}
  \synonymref{顽皮}{wan2pi2}
  \antonymref{老实}{lao3shi5}
  \antonymref{听话}{ting1/hua4}
\end{EntryWithPhonetic}

\begin{EntryWithPhonetic}{淘汰}{tao2tai4}{11,7}{⽔,⽔}[HSK 7-9]
  \definition{v.}{1. Eliminar por meio de seleção ou competição
Ao selecionar, você pode eliminar os itens ruins ou inadequados.}
  \synonymref{减少}{jian3shao3}
  \synonymref{剔除}{ti1chu2}
  \antonymref{达标}{da2biao1}
  \antonymref{合格}{he2ge2}
  \antonymref{挑选}{tiao1xuan3}
  \antonymref{通过}{tong1guo4}
  \antonymref{选择}{xuan3ze2}
  \antonymref{引进}{yin3jin4}
\end{EntryWithPhonetic}

%%%%%%%%%% 讨 %%%%%%%%%%
\subsection*{讨}\addcontentsline{loh}{figure}{讨 \dpy{tao3}}

\begin{EntryWithPhonetic}{讨}{tao3}{5}{⾔}[HSK 7-9]
  \definition{v.}{enviar forças armadas para suprimir; enviar uma expedição punitiva contra; enviar exército ou despachar tropas para suprimir ou atacar | denunciar; condenar; censurar | exigir; pedir; implorar por | casar (com uma mulher) | incorrer; convidar | discutir; estudar | provocar; cortejar}
\end{EntryWithPhonetic}

\begin{EntryWithPhonetic}{讨好}{tao3/hao3}{5,6}{⾔,⼥}[HSK 7-9]
  \definition{v.+compl.}{bajular; tentar agradar; bajular; tentar ganhar a simpatia de; (agradar aos outros) lisonjear os outros e obter seu favor ou elogio | obter um bom resultado; ser recompensado com um resultado frutífero; (bajulação) produzir bons resultados (frequentemente usada em contextos negativos)}
  \synonymref{取悦}{qu3yue4}
\end{EntryWithPhonetic}

\begin{EntryWithPhonetic}{讨价还价}{tao3jia4-huan2jia4}{5,6,7,6}{⾔,⼈,⾡,⼈}[HSK 7-9]
  \definition{expr.}{regatear o preço; barganhar; essa metáfora descreve o ato de fazer várias exigências e pechinchar sobre cada pequeno detalhe ao aceitar uma tarefa ou negociar; também pode ser descrita como barganha}
\end{EntryWithPhonetic}

\begin{EntryWithPhonetic}{讨论}{tao3lun4}{5,6}{⾔,⾔}[HSK 2]
  \definition{v.}{discutir; conversar sobre; trocar opiniões ou debater as questões levantadas}
\end{EntryWithPhonetic}

\begin{EntryWithPhonetic}{讨人喜欢}{tao3 ren2 xi3huan5}{5,2,12,6}{⾔,⼈,⼝,⽋}[HSK 7-9]
  \definition{adj.}{encantador}
  \definition{v.}{atrair o afeto das pessoas}
\end{EntryWithPhonetic}

\begin{EntryWithPhonetic}{讨生活}{tao3sheng1huo2}{5,5,9}{⾔,⽣,⽔}
  \definition{v.}{ganhar a vida}
\end{EntryWithPhonetic}

\begin{EntryWithPhonetic}{讨厌}{tao3/yan4}{5,6}{⾔,⼚}[HSK 5]
  \definition{adj.}{desagradável; repugnante; repulsivo; irritante; incômodo}
  \definition{v.+compl.}{odiar; não gostar; sentir repulsa por}
\end{EntryWithPhonetic}

%%%%%%%%%% 套 %%%%%%%%%%
\subsection*{套}\addcontentsline{loh}{figure}{套 \dpy{tao4}}

\begin{EntryWithPhonetic}{套}{tao4}{10}{⼤}[HSK 2]
  \definition{clas.}{usado para coisas agrupadas: conjuntos, coleções, séries, etc.}
  \definition{s.}{estojo; capa; bainha | local onde o rio ou a cordilheira faz uma curva (usado principalmente em nomes de lugares) | enchimento de algodão em roupas e edredons | arreios; corda para amarrar animais | nó; laço; um objeto circular feito com corda ou algo semelhante | cortersia; convenção; conversa fiada; métodos repetitivos | armadilha; truque; conspiração}
  \definition{v.}{sobrepor; interligar | deslizar sobre; cobrir por fora | atrelar; engatar; usar um cinto de segurança | copiar; imitar; seguir o modelo de | extrair; induzir a falar; persuadir alguém a revelar um segredo; induzir; provocar | tentar vencer; aproximar-se de; aproximar-se intencionalmente de outras pessoas para algum propósito | fazer a rosca de um parafuso; usar um macho de rosca ou uma chave de rosca para fazer roscas}
\end{EntryWithPhonetic}

\begin{EntryWithPhonetic}{套餐}{tao4can1}{10,16}{⼤,⾷}[HSK 4]
  \definition{s.}{combo; pacote de produtos; pacote de serviços; metaforicamente, bens ou projetos que são combinados e levados ao mercado | refeição preparada; pacotes de refeições completos}
\end{EntryWithPhonetic}

\begin{EntryWithPhonetic}{套问}{tao4wen4}{10,6}{⼤,⾨}
  \definition{s.}{retórica}
  \definition{v.}{descobrir por meio de questionamento indireto diplomático}
\end{EntryWithPhonetic}

%%%%%%%%%% 特 %%%%%%%%%%
\subsection*{特}\addcontentsline{loh}{figure}{特 \dpy{te4}}

\begin{EntryWithPhonetic}{特}{te4}{10}{⽜}[HSK 6]
  \definition{adj.}{especial; incomum; particular; excepcional; diferente do geral | especial; solteiro; solitário}
  \definition{adv.}{muito; extremamente | especialmente; para um propósito especial |mas; somente}
  \definition{clas.}{TEX; abreviação para unidades de medida como TEX; a unidade de medida TEX indica a espessura de um fio têxtil através do seu peso}
  \definition{s.}{espião; agente secreto}
\end{EntryWithPhonetic}

\begin{EntryWithPhonetic}{特别}{te4bie2}{10,7}{⽜,⼑}[HSK 2]
  \definition{adj.}{especial; particular; fora do comum; diferente dos outros, com características próprias}
  \definition{adv.}{especialmente; particularmente | ainda mais; em particular; frequentemente usado com 是 | especialmente; deliberadamente; para um propósito específico}
  \seealsoref{是}{shi4}
\end{EntryWithPhonetic}

\begin{EntryWithPhonetic}{特别快车}{te4bie2 kuai4che1}{10,7,7,4}{⽜,⼑,⼼,⾞}
  \definition{s.}{trem expresso; expresso; expresso especial; refere"-se a trens de passageiros que param em menos estações e têm menor tempo de viagem do que trens expressos diretos}
\end{EntryWithPhonetic}

\begin{EntryWithPhonetic}{特产}{te4chan3}{10,6}{⽜,⼇}[HSK 7-9]
  \definition{s.}{produto nativo; especialidade; produto local especial; produtos que são exclusivos de um determinado local ou país, ou que são particularmente famosos}
\end{EntryWithPhonetic}

\begin{EntryWithPhonetic}{特长}{te4chang2}{10,4}{⽜,⾧}[HSK 7-9]
  \definition[项,个]{s.}{talento; ponto forte; especialidade; aquilo em que alguém é habilidoso; habilidades especiais}
  \synonymref{爱好}{ai4hao4}
  \synonymref{拿手}{na2shou3}
  \synonymref{擅长}{shan4chang2}
  \synonymref{善于}{shan4yu2}
  \synonymref{特技}{te4ji4}
  \antonymref{缺陷}{que1xian4}
\end{EntryWithPhonetic}

\begin{EntryWithPhonetic}{特大}{te4da4}{10,3}{⽜,⼤}[HSK 6]
  \definition{adj.}{especialmente (excepcionalmente) grande; o mais}
\end{EntryWithPhonetic}

\begin{EntryWithPhonetic}{特地}{te4di4}{10,6}{⽜,⼟}[HSK 6]
  \definition{adv.}{especialmente; propositalmente; para um propósito especial}
\end{EntryWithPhonetic}

\begin{EntryWithPhonetic}{特点}{te4dian3}{10,9}{⽜,⽕}[HSK 2]
  \definition[个,大]{s.}{característica; peculiaridade; traço distintivo; a singularidade de uma pessoa ou coisa}
\end{EntryWithPhonetic}

\begin{EntryWithPhonetic}{特定}{te4ding4}{10,8}{⽜,⼧}[HSK 5]
  \definition{adj.}{particular; específico; especialmente designado | dado; especificado; específico; uma pessoa específica, um determinado momento, lugar, ambiente, etc.}
\end{EntryWithPhonetic}

\begin{EntryWithPhonetic}{特技}{te4ji4}{10,7}{⽜,⼿}
  \definition{s.}{efeito especial | dublê}
\end{EntryWithPhonetic}

\begin{EntryWithPhonetic}{特价}{te4jia4}{10,6}{⽜,⼈}[HSK 4]
  \definition{s.}{oferta especial; preço de barganha; preço especial reduzido}
\end{EntryWithPhonetic}

\begin{EntryWithPhonetic}{特快}{te4kuai4}{10,7}{⽜,⼼}[HSK 6]
  \definition{adj.}{expresso (trem, entrega etc.)}
  \definition{s.}{trem expresso; abreviação de 特别快车}
  \seealsoref{特别快车}{te4bie2 kuai4che1}
  \antonymref{普快}{pu3 kuai4}
\end{EntryWithPhonetic}

\begin{EntryWithPhonetic}{特例}{te4li4}{10,8}{⽜,⼈}[HSK 7-9]
  \definition{s.}{exemplo isolado | caso especial}
  \synonymref{特制}{te4zhi4}
  \antonymref{惯例}{guan4li4}
\end{EntryWithPhonetic}

\begin{EntryWithPhonetic}{特权}{te4quan2}{10,6}{⽜,⽊}[HSK 7-9]
  \definition[种]{s.}{privilégio; prerrogativa | direito e privilégio especiais}
\end{EntryWithPhonetic}

\begin{EntryWithPhonetic}{特色}{te4se4}{10,6}{⽜,⾊}[HSK 3]
  \definition{s.}{característica; característica distintiva; a cor única, estilo, etc. de um objeto}
\end{EntryWithPhonetic}

\begin{EntryWithPhonetic}{特殊}{te4shu1}{10,10}{⽜,⽍}[HSK 4]
  \definition{adj.}{especial; particular; peculiar; excepcional; incomum}
\end{EntryWithPhonetic}

\begin{EntryWithPhonetic}{特性}{te4xing4}{10,8}{⽜,⼼}[HSK 5]
  \definition[种,个]{s.}{propriedade específica (ou característica) | característica; sabores | propriedade}
\end{EntryWithPhonetic}

\begin{EntryWithPhonetic}{特邀}{te4yao1}{10,16}{⽜,⾡}[HSK 7-9]
  \definition{v.}{convidar especialmente (um convidado para um evento)}
  \synonymref{敬请}{jing4qing3}
\end{EntryWithPhonetic}

\begin{EntryWithPhonetic}{特意}{te4yi4}{10,13}{⽜,⼼}[HSK 6]
  \definition{adv.}{especialmente; para um propósito especial}
\end{EntryWithPhonetic}

\begin{EntryWithPhonetic}{特有}{te4you3}{10,6}{⽜,⽉}[HSK 5]
  \definition{adj.}{específico; peculiar; característico; único; exclusivo; especial}
\end{EntryWithPhonetic}

\begin{EntryWithPhonetic}{特征}{te4zheng1}{10,8}{⽜,⼻}[HSK 4]
  \definition[个,种]{s.}{característica; aparência ou o fenômeno característico de uma pessoa ou coisa que pode ser visto de fora}
\end{EntryWithPhonetic}

\begin{EntryWithPhonetic}{特制}{te4zhi4}{10,8}{⽜,⼑}[HSK 7-9]
  \definition{v.}{fabricar especialmente; fazer sob encomenda}
  \synonymref{特别}{te4bie2}
  \synonymref{特例}{te4li4}
  \synonymref{特殊}{te4shu1}
  \antonymref{通用}{tong1yong4}
\end{EntryWithPhonetic}

\begin{EntryWithPhonetic}{特质}{te4zhi4}{10,8}{⽜,⾙}[HSK 7-9]
  \definition[种]{s.}{qualidade especial; característica; qualidades ou propriedades únicas}
  \synonymref{属性}{shu3xing4}
  \synonymref{特点}{te4dian3}
  \synonymref{特色}{te4se4}
  \synonymref{特性}{te4xing4}
  \synonymref{特征}{te4zheng1}
  \antonymref{共性}{gong4xing4}
\end{EntryWithPhonetic}

%%%%%%%%%% 疼 %%%%%%%%%%
\subsection*{疼}\addcontentsline{loh}{figure}{疼 \dpy{teng2}}

\begin{EntryWithPhonetic}{疼}{teng2}{10}{⽧}[HSK 2]
  \definition{adj.}{dolorido; doído; sensação de extremo desconforto causada por ferimentos, doenças, etc.}
  \definition{v.}{ferir; machucar | adorar; amar profundamente; gostar muito; cuidar}
\end{EntryWithPhonetic}

\begin{EntryWithPhonetic}{疼痛}{teng2tong4}{10,12}{⽧,⽧}[HSK 6]
  \definition[阵,种]{s.}{dor; sofrimento; ferimento; descreve a sensação de dor causada por lesão ou doença}
\end{EntryWithPhonetic}

%%%%%%%%%% 腾 %%%%%%%%%%
\subsection*{腾}\addcontentsline{loh}{figure}{腾 \dpy{teng2}}

\begin{EntryWithPhonetic}{腾}{teng2}{13}{⾁}[HSK 7-9]
  \definition*{s.}{Sobrenome: Teng}
  \definition{v.}{galopar; saltar; trotar | subir; planar | abrir espaço; desocupar; esvaziar | ascender; subir aos céus | excitar; agitar; mexer}
  \definition{v.aux.}{usado após certos verbos para indicar repetição}
\end{EntryWithPhonetic}

%%%%%%%%%% 藤 %%%%%%%%%%
\subsection*{藤}\addcontentsline{loh}{figure}{藤 \dpy{teng2}}

\begin{EntryWithPhonetic}{藤}{teng2}{18}{⾋}
  \definition*{s.}{Teng}
  \definition[根,条]{s.}{cana; vime | videira}
\end{EntryWithPhonetic}

\begin{EntryWithPhonetic}{藤椅}{teng2yi3}{18,12}{⾋,⽊}[HSK 7-9]
  \definition{s.}{cadeira de vime}
\end{EntryWithPhonetic}

%%%%%%%%%% 剔 %%%%%%%%%%
\subsection*{剔}\addcontentsline{loh}{figure}{剔 \dpy{ti1}}

\begin{EntryWithPhonetic}{剔}{ti1}{10}{⼑}
  \definition{s.}{traço ascendente (em caracteres chineses)}
  \definition{v.}{limpar com um instrumento pontiagudo; cutucar | selecionar e descartar; rejeitar; eliminar | raspar a carne do osso | escolher; selecionar de dentro para fora}
\end{EntryWithPhonetic}

\begin{EntryWithPhonetic}{剔除}{ti1chu2}{10,9}{⼑,⾩}[HSK 7-9]
  \definition{v.}{rejeitar; excluir; remover; livrar"-se de; significa remover, excluir ou limpar}
  \synonymref{淘汰}{tao2tai4}
  \antonymref{吸收}{xi1shou1}
\end{EntryWithPhonetic}

%%%%%%%%%% 梯 %%%%%%%%%%
\subsection*{梯}\addcontentsline{loh}{figure}{梯 \dpy{ti1}}

\begin{EntryWithPhonetic}{梯}{ti1}{11}{⽊}
  \definition*{s.}{Sobrenome: Ti}
  \definition{adj.}{em forma de escada; em socalcos}
  \definition[个]{s.}{escada; degrau; socalco (são plataformas niveladas, semelhantes a degraus, cortadas em encostas de morros para permitir o cultivo agrícola e evitar a erosão do solo)}
\end{EntryWithPhonetic}

\begin{EntryWithPhonetic}{梯恩梯}{ti1'en1ti1}{11,10,11}{⽊,⼼,⽊}
  \definition{s.}{Empréstimo linguístico: TNT, trinitrotolueno}
\end{EntryWithPhonetic}

\begin{EntryWithPhonetic}{梯子}{ti1zi5}{11,3}{⽊,⼦}[HSK 7-9]
  \definition[个,把]{s.}{escada; escada de mão; ferramentas para facilitar o acesso das pessoas}
\end{EntryWithPhonetic}

%%%%%%%%%% 踢 %%%%%%%%%%
\subsection*{踢}\addcontentsline{loh}{figure}{踢 \dpy{ti1}}

\begin{EntryWithPhonetic}{踢}{ti1}{15}{⾜}[HSK 6]
  \definition{v.}{chutar | jogar (por exemplo, futebol)}
\end{EntryWithPhonetic}

\begin{EntryWithPhonetic}{踢爆}{ti1bao4}{15,19}{⾜,⽕}
  \definition{v.}{expor | revelar}
\end{EntryWithPhonetic}

\begin{EntryWithPhonetic}{踢蹋舞}{ti1ta4wu3}{15,17,14}{⾜,⾜,⾇}
  \definition{s.}{sapateado | passo de dança}
\end{EntryWithPhonetic}

%%%%%%%%%% 提 %%%%%%%%%%
\subsection*{提}\addcontentsline{loh}{figure}{提 \dpy{ti2}}

\begin{EntryWithPhonetic}{提}{ti2}{12}{⼿}[HSK 2]
  \definition*{s.}{Sobrenome: Ti}
  \definition{s.}{concha; utensílio para servir óleo ou vinho | traço ascendente (em caracteres chineses)}
  \definition{v.}{carregar (na mão, com o braço para baixo) ; segurar com as mãos para baixo | elevar; levantar; promover | avançar; antecipar uma data; mudar para uma data anterior; adiar o prazo previsto | levantar; apresentar; indicar ou citar | extrair; retirar (tirar) | (prisioneiros) trazer; entregar | mencionar; referir-se a; abordar}
\end{EntryWithPhonetic}

\begin{EntryWithPhonetic}{提拔}{ti2ba2}{12,8}{⼿,⼿}[HSK 7-9]
  \definition{v.}{1. promover; favorecer
Selecionar pessoal para assumir cargos mais importantes.}
  \synonymref{教育}{jiao4yu4}
  \synonymref{培育}{pei2yu4}
  \synonymref{提升}{ti2sheng1}
  \synonymref{选拔}{xuan3ba2}
\end{EntryWithPhonetic}

\begin{EntryWithPhonetic}{提倡}{ti2chang4}{12,10}{⼿,⼈}[HSK 5]
  \definition{v.}{promover; incentivar; recomendar; apresentar as vantagens de algo para incentivar as pessoas a usá-lo ou implementá-lo}
\end{EntryWithPhonetic}

\begin{EntryWithPhonetic}{提出}{ti2 chu1}{12,5}{⼿,⼐}[HSK 2]
  \definition{v.}{levantar; propor; apresentar; expressar seus desejos, ideias, sugestões, etc. por meio de palavras ou textos}
\end{EntryWithPhonetic}

\begin{EntryWithPhonetic}{提到}{ti2dao4}{12,8}{⼿,⼑}[HSK 2]
  \definition{v.}{mencionar; referir-se a; levantar (assunto)}
\end{EntryWithPhonetic}

\begin{EntryWithPhonetic}{提高}{ti2/gao1}{12,10}{⼿,⾼}[HSK 2]
  \definition{v.+compl.}{elevar; aprimorar; aumentar; melhorar a posição, o nível, a quantidade, a qualidade e outros aspectos em relação ao estado original}
\end{EntryWithPhonetic}

\begin{EntryWithPhonetic}{提供}{ti2gong1}{12,8}{⼿,⼈}[HSK 4]
  \definition{v.}{oferecer; fornecer; suprir; prover; proporcionar}
\end{EntryWithPhonetic}

\begin{EntryWithPhonetic}{提及}{ti2ji2}{12,3}{⼿,⼃}
  \definition{v.}{mencionar | levantar (um assunto) | chamar a atenção de alguém}
\end{EntryWithPhonetic}

\begin{EntryWithPhonetic}{提交}{ti2jiao1}{12,6}{⼿,⼇}[HSK 6]
  \definition{v.}{referir-se a; submeter (um problema, etc.) a; enviar questões que precisam ser discutidas, decididas ou tratadas para agências ou reuniões relevantes}
\end{EntryWithPhonetic}

\begin{EntryWithPhonetic}{提炼}{ti2lian4}{12,9}{⼿,⽕}[HSK 7-9]
  \definition{v.}{extrair e purificar; abstrair; refinar | extrair}
  \antonymref{融合}{rong2he2}
\end{EntryWithPhonetic}

\begin{EntryWithPhonetic}{提名}{ti2/ming2}{12,6}{⼿,⼝}[HSK 7-9]
  \definition{v.+compl.}{indicar; nomear; propor pessoas ou coisas que provavelmente serão eleitas em uma seleção ou eleição}
\end{EntryWithPhonetic}

\begin{EntryWithPhonetic}{提起}{ti2qi3}{12,10}{⼿,⾛}[HSK 5]
  \definition{v.}{mencionar; falar sobre; abordar | levantar; despertar; estimular; revigorar; alegrar/animar | iniciar; instituir; propor | levantar; pegar}
\end{EntryWithPhonetic}

\begin{EntryWithPhonetic}{提前}{ti2qian2}{12,9}{⼿,⼑}[HSK 3]
  \definition{adv.}{antecipadamente; faça uma coisa antes de fazer outra}
  \definition{v.}{avançar; adiantar; mudar para uma data anterior; trazer para frente}
\end{EntryWithPhonetic}

\begin{EntryWithPhonetic}{提升}{ti2sheng1}{12,4}{⼿,⼗}[HSK 6]
  \definition{v.}{promover; avançar; melhorar (posição, grau, qualidade, etc.) | içar; elevar; transportar (minerais, materiais, etc.) para um local mais alto usando um guincho, etc.}
\end{EntryWithPhonetic}

\begin{EntryWithPhonetic}{提示}{ti2shi4}{12,5}{⼿,⽰}[HSK 5]
  \definition[个]{s.}{dica; lembrete; pistas ou informações fornecidas para chamar a atenção, fazer com que a outra pessoa pense ou compreenda}
  \definition{v.}{solicitar; lembrar; indicar; alertar; levantar questões que o outro não tenha pensado ou não tenha imaginado, para chamar a atenção dele}
\end{EntryWithPhonetic}

\begin{EntryWithPhonetic}{提速}{ti2/su4}{12,10}{⼿,⾡}[HSK 7-9]
  \definition{v.+compl.}{acelerar; aumentar a velocidade | para aumentar a velocidade de cruzeiro especificada | ganhar velocidade}
  \synonymref{加速}{jia1su4}
\end{EntryWithPhonetic}

\begin{EntryWithPhonetic}{提问}{ti2wen4}{12,6}{⼿,⾨}[HSK 3]
  \definition{v.}{\emph{quiz}; fazer uma pergunta; colocar questões para}
\end{EntryWithPhonetic}

\begin{EntryWithPhonetic}{提心吊胆}{ti2xin1-diao4dan3}{12,4,6,9}{⼿,⼼,⼝,⾁}[HSK 7-9]
  \definition{expr.}{nervoso; com o coração na boca; em constante medo; extremamente preocupado; em estado de ansiedade; estar em suspense; descreve um estado de grande preocupação ou medo}
  \seealsoref{悬心吊胆}{xuan2xin1-diao4dan3}
\end{EntryWithPhonetic}

\begin{EntryWithPhonetic}{提醒}{ti2/xing3}{12,16}{⼿,⾣}[HSK 4]
  \definition{v.+compl.}{alertar; avisar; advertir; lembrar; apontar para ou chamar a atenção para}
\end{EntryWithPhonetic}

\begin{EntryWithPhonetic}{提议}{ti2yi4}{12,5}{⼿,⾔}[HSK 7-9]
  \definition[项,个]{s.}{proposta; moção}
  \definition{v.}{propor; sugerir; ao discutir assuntos, propor ideias para que todos possam debatê-las}
  \synonymref{倡导}{chang4dao3}
  \synonymref{倡议}{chang4yi4}
  \synonymref{发起}{fa1qi3}
  \synonymref{建议}{jian4yi4}
  \synonymref{提出}{ti2 chu1}
  \synonymref{提倡}{ti2chang4}
\end{EntryWithPhonetic}

\begin{EntryWithPhonetic}{提早}{ti2zao3}{12,6}{⼿,⽇}[HSK 7-9]
  \definition{v.}{antecipar"-se ao horário previsto; chegar mais cedo do que o planejado ou esperado}
  \synonymref{赶早}{gan3zao3}
  \synonymref{尽快}{jin3kuai4}
  \synonymref{尽早}{jin3zao3}
  \synonymref{提前}{ti2qian2}
  \antonymref{顺延}{shun4yan2}
  \antonymref{推迟}{tui1chi2}
\end{EntryWithPhonetic}

%%%%%%%%%% 题 %%%%%%%%%%
\subsection*{题}\addcontentsline{loh}{figure}{题 \dpy{ti2}}

\begin{EntryWithPhonetic}{题}{ti2}{15}{⾴}[HSK 2]
  \definition*{s.}{Sobrenome: Ti}
  \definition[个,道]{s.}{tópico; título; assunto; problema; frases que indicam o conteúdo de poemas ou discursos | questão; questões que devem ser respondidas durante os exercícios ou exames | antigamente, referia"-se à testa}
  \definition{v.}{inscrever; escrever; assinar}
\end{EntryWithPhonetic}

\begin{EntryWithPhonetic}{题材}{ti2cai2}{15,7}{⾴,⽊}[HSK 5]
  \definition{s.}{tema; assunto; material que compõe as obras literárias e artísticas, ou seja, os eventos ou fenômenos da vida descritos concretamente nas obras}
\end{EntryWithPhonetic}

\begin{EntryWithPhonetic}{题目}{ti2mu4}{15,5}{⾴,⽬}[HSK 3]
  \definition[个,道]{s.}{título; assunto; tópico; o título de um poema ou discurso | quebra-cabeça; problema de exercício; questões a serem respondidas em exercícios ou provas}
\end{EntryWithPhonetic}

%%%%%%%%%% 体 %%%%%%%%%%
\subsection*{体}\addcontentsline{loh}{figure}{体 \dpy{ti3}}

\begin{EntryWithPhonetic}{体}{ti3}{7}{⼈}
  \definition{s.}{corpo; parte do corpo | substância; objeto; estado de uma substância | estilo; forma | sistema | estilo de caligrafia | tipo de letra; fonte | Linguística: aspecto (de um verbo) | estrutura; a forma escrita do texto; o gênero da obra}
  \definition{v.}{fazer ou vivenciar algo pessoalmente | colocar"-se na posição de outro; colocar"-se mentalmente na posição do outro; colocar"-se no lugar do outro}
\end{EntryWithPhonetic}

\begin{EntryWithPhonetic}{体操}{ti3cao1}{7,16}{⼈,⼿}[HSK 4]
  \definition{s.}{ginástica; esportes, exercícios ou performances de vários movimentos, sem armas ou com o auxílio de determinados equipamentos}
\end{EntryWithPhonetic}

\begin{EntryWithPhonetic}{体会}{ti3hui4}{7,6}{⼈,⼈}[HSK 3]
  \definition[个,些,种]{s.}{conhecimento; compreensão; experiência pessoal}
  \definition{v.}{perceber; saber (ou aprender) com a experiência}
\end{EntryWithPhonetic}

\begin{EntryWithPhonetic}{体积}{ti3ji1}{7,10}{⼈,⽲}[HSK 5]
  \definition[个]{s.}{volume; quantidade; o tamanho do espaço ocupado pelo objeto}
\end{EntryWithPhonetic}

\begin{EntryWithPhonetic}{体检}{ti3jian3}{7,11}{⼈,⽊}[HSK 4]
  \definition{v.}{fazer um exame médico}
\end{EntryWithPhonetic}

\begin{EntryWithPhonetic}{体力}{ti3li4}{7,2}{⼈,⼒}[HSK 5]
  \definition{s.}{força física; vigor físico (ou corporal); a força do corpo humano para sustentar suas próprias atividades}
\end{EntryWithPhonetic}

\begin{EntryWithPhonetic}{体谅}{ti3liang4}{7,10}{⼈,⾔}[HSK 7-9]
  \definition{v.}{levar em consideração; demonstrar compreensão e simpatia por; colocar-se no lugar deles e ser compreensivo}
  \synonymref{宽容}{kuan1rong2}
  \synonymref{谅解}{liang4jie3}
  \synonymref{体贴}{ti3tie1}
  \synonymref{原谅}{yuan2liang4}
\end{EntryWithPhonetic}

\begin{EntryWithPhonetic}{体面}{ti3mian4}{7,9}{⼈,⾯}[HSK 7-9]
  \definition{adj.}{honroso; digno de crédito; respeitável | bonito; atraente}
  \definition{s.}{face; dignidade; decoro; conduta adequada; \emph{status}}
  \synonymref{场合}{chang3he2}
  \synonymref{场面}{chang3mian4}
  \synonymref{得体}{de2ti3}
  \synonymref{好看}{hao3kan4}
  \synonymref{合适}{he2shi4}
  \synonymref{局面}{ju2mian4}
  \synonymref{美观}{mei3guan1}
  \synonymref{面子}{mian4zi5}
  \antonymref{难堪}{nan2kan1}
  \antonymref{难看}{nan2kan4}
  \antonymref{难听}{nan2ting1}
\end{EntryWithPhonetic}

\begin{EntryWithPhonetic}{体内}{ti3nei4}{7,4}{⼈,⼌}
  \definition{adj.}{dentro do corpo | \emph{in vivo} (versus \emph{in vitro} | interno a}
\end{EntryWithPhonetic}

\begin{EntryWithPhonetic}{体能}{ti3neng2}{7,10}{⼈,⾁}[HSK 7-9]
  \definition{s.}{resistência (física) | capacidade física}
\end{EntryWithPhonetic}

\begin{EntryWithPhonetic}{体贴}{ti3tie1}{7,9}{⼈,⾙}[HSK 7-9]
  \definition{adj.}{atencioso; considere com atenção os sentimentos e as circunstâncias das outras pessoas e ofereça"-lhes cuidado e consideração}
  \definition{v.}{cuidar de; demonstrar consideração por; dedicar toda a atenção a}
  \synonymref{爱护}{ai4hu4}
  \synonymref{关爱}{guan1'ai4}
  \synonymref{关怀}{guan1huai2}
  \synonymref{关心}{guan1xin1}
  \synonymref{关注}{guan1zhu4}
  \synonymref{谅解}{liang4jie3}
  \synonymref{体谅}{ti3liang4}
  \synonymref{温柔}{wen1rou2}
  \synonymref{照顾}{zhao4gu5}
  \antonymref{冷淡}{leng3dan4}
  \antonymref{虐待}{nve4dai4}
\end{EntryWithPhonetic}

\begin{EntryWithPhonetic}{体温}{ti3wen1}{7,12}{⼈,⽔}[HSK 7-9]
  \definition{s.}{temperatura corporal}
  \synonymref{温度}{wen1du4}
\end{EntryWithPhonetic}

\begin{EntryWithPhonetic}{体系}{ti3xi4}{7,7}{⼈,⽷}[HSK 7-9]
  \definition[个,套]{s.}{configuração; sistema; um todo formado pela interconexão de muitas coisas ou ideias relacionadas}
  \synonymref{编制}{bian1zhi4}
  \synonymref{机制}{ji1zhi4}
  \synonymref{体制}{ti3zhi4}
  \synonymref{系列}{xi4lie4}
  \synonymref{系统}{xi4tong3}
\end{EntryWithPhonetic}

\begin{EntryWithPhonetic}{体现}{ti3xian4}{7,8}{⼈,⾒}[HSK 3]
  \definition{v.}{refletir; incorporar; encarnar; uma certa qualidade ou fenômeno se manifesta especificamente em uma determinada coisa}
\end{EntryWithPhonetic}

\begin{EntryWithPhonetic}{体验}{ti3yan4}{7,10}{⼈,⾺}[HSK 3]
  \definition[种]{s.}{experiência; a sensação adquirida pela experiência pessoal}
  \definition{v.}{aprender através da prática; aprender através da experiência pessoal; entender as coisas através da experiência pessoal}
\end{EntryWithPhonetic}

\begin{EntryWithPhonetic}{体育}{ti3yu4}{7,8}{⼈,⾁}[HSK 2]
  \definition{s.}{cultura física; treinamento físico; educação cuja principal tarefa é desenvolver a capacidade física e fortalecer a constituição física, alcançada através da participação em várias atividades esportivas | esportes; atividades esportivas; refere"-se a esportes}
\end{EntryWithPhonetic}

\begin{EntryWithPhonetic}{体育场}{ti3yu4chang3}{7,8,6}{⼈,⾁,⼟}[HSK 2]
  \definition[个,座]{s.}{estádio; campo esportivo; espaço ao ar livre para a prática de exercícios físicos ou competições esportivas}
\end{EntryWithPhonetic}

\begin{EntryWithPhonetic}{体育馆}{ti3yu4guan3}{7,8,11}{⼈,⾁,⾷}[HSK 2]
  \definition[个,座,家]{s.}{ginásio; locais esportivos ou competições em ambientes fechados geralmente têm arquibancadas fixas}
\end{EntryWithPhonetic}

\begin{EntryWithPhonetic}{体制}{ti3zhi4}{7,8}{⼈,⼑}[HSK 7-9]
  \definition{s.}{estrutura; sistema (de organização); sistemas organizacionais de agências governamentais, empresas, instituições públicas, etc. | forma; estilo (de escrita literária); o gênero e a estrutura das obras de arte}
  \synonymref{机制}{ji1zhi4}
  \synonymref{体系}{ti3xi4}
\end{EntryWithPhonetic}

\begin{EntryWithPhonetic}{体质}{ti3zhi4}{7,8}{⼈,⾙}[HSK 7-9]
  \definition{s.}{compleição física; constituição; nível de saúde humana e adaptabilidade ao ambiente externo}
  \synonymref{身体}{shen1ti3}
  \synonymref{提高}{ti2/gao1}
  \synonymref{增强}{zeng1qiang2}
\end{EntryWithPhonetic}

\begin{EntryWithPhonetic}{体重}{ti3zhong4}{7,9}{⼈,⾥}[HSK 4]
  \definition{s.}{peso corporal}
\end{EntryWithPhonetic}

%%%%%%%%%% 剃 %%%%%%%%%%
\subsection*{剃}\addcontentsline{loh}{figure}{剃 \dpy{ti4}}

\begin{EntryWithPhonetic}{剃}{ti4}{9}{⼑}[HSK 7-9]
  \definition{v.}{depilar; raspar; cortar; usar uma lâmina especial para raspar (cabelo, barba, etc.)}
\end{EntryWithPhonetic}

%%%%%%%%%% 替 %%%%%%%%%%
\subsection*{替}\addcontentsline{loh}{figure}{替 \dpy{ti4}}

\begin{EntryWithPhonetic}{替}{ti4}{12}{⽈}[HSK 4]
  \definition{prep.}{para; em nome de}
  \definition{s.}{decadência; declínio; enfraquecimento}
  \definition{v.}{substituir; substituir por; tomar o lugar de}
\end{EntryWithPhonetic}

\begin{EntryWithPhonetic}{替代}{ti4dai4}{12,5}{⽈,⼈}[HSK 4]
  \definition{v.}{substituir; suplantar}
\end{EntryWithPhonetic}

\begin{EntryWithPhonetic}{替换}{ti4huan4}{12,10}{⽈,⼿}[HSK 7-9]
  \definition{v.}{substituir; substituir por; deslocar; tomar o lugar de; alternar}
  \synonymref{撤换}{che4huan4}
  \synonymref{更换}{geng1huan4}
  \synonymref{交换}{jiao1huan4}
  \synonymref{替代}{ti4dai4}
  \antonymref{代替}{dai4ti4}
\end{EntryWithPhonetic}

\begin{EntryWithPhonetic}{替身}{ti4shen1}{12,7}{⽈,⾝}[HSK 7-9]
  \definition{s.}{bode expiatório | dublê de corpo | dublê}
  \definition{v.}{substituir; substituir alguém; ocupar o lugar de}
\end{EntryWithPhonetic}

%%%%%%%%%% 天 %%%%%%%%%%
\subsection*{天}\addcontentsline{loh}{figure}{天 \dpy{tian1}}

\begin{EntryWithPhonetic}{天}{tian1}{4}{⼤}[HSK 1]
  \definition*{s.}{Sobrenome: Tian}
  \definition{adj.}{localizado no topo; suspenso no ar | inato; natural}
  \definition{clas.}{usado para contar dias}
  \definition{s.}{céu; paraíso; espaço onde se encontram o sol, a lua e as estrelas | dia; as 24 horas do dia, às vezes referindo"-se especificamente ao período diurno | um período de tempo em um dia; em algum momento do dia | temporada; estação do ano | clima | natureza | Deus; céu; o criador | paraíso; refere"-se ao local onde residem os deuses, budas e imortais}
\end{EntryWithPhonetic}

\begin{EntryWithPhonetic}{天才}{tian1cai2}{4,3}{⼤,⼿}[HSK 5]
  \definition{adj.}{talentoso | superdotado | genial}
  \definition[个,位,名]{s.}{dom; genialidade; talento natural; inteligência e sabedoria acima da média}
\end{EntryWithPhonetic}

\begin{EntryWithPhonetic}{天长地久}{tian1chang2-di4jiu3}{4,4,6,3}{⼤,⾧,⼟,⼃}[HSK 7-9]
  \definition{expr.}{eterno; perpétuo; que dura tanto quanto o céu e a terra; enquanto houver céu e terra, descrevendo algo que é eterno e imutável (frequentemente referindo-se ao amor)}
\end{EntryWithPhonetic}

\begin{EntryWithPhonetic}{天秤座}{tian1cheng4zuo4}{4,10,10}{⼤,⽲,⼴}
  \definition*{s.}{Libra (signo do zodíaco) | Astronomia: Constelação de Libra}
\end{EntryWithPhonetic}

\begin{EntryWithPhonetic}{天地}{tian1di4}{4,6}{⼤,⼟}[HSK 7-9]
  \definition{s.}{mundo; universo; céu e terra; Céu e Terra, o mundo natural e a sociedade | campo de atividade; âmbito de atuação; âmbito das atividades | situação difícil; situação ruim | 4. título de um periódico; palavra; coluna (em jornais, etc.); nomes usados ​​em jornais, revistas ou colunas}
\end{EntryWithPhonetic}

\begin{EntryWithPhonetic}{天鹅}{tian1'e2}{4,12}{⼤,⿃}[HSK 7-9]
  \definition{s.}{cisne}
\end{EntryWithPhonetic}

\begin{EntryWithPhonetic}{天分}{tian1fen4}{4,4}{⼤,⼑}[HSK 7-9]
  \definition{s.}{talento; dom natural; dons especiais}
  \synonymref{本性}{ben3xing4}
  \synonymref{天才}{tian1cai2}
  \synonymref{天赋}{tian1fu4}
  \synonymref{天生}{tian1sheng1}
  \synonymref{天性}{tian1xing4}
  \synonymref{先天}{xian1tian1}
  \synonymref{性格}{xing4ge2}
\end{EntryWithPhonetic}

\begin{EntryWithPhonetic}{天赋}{tian1fu4}{4,12}{⼤,⾙}[HSK 7-9]
  \definition[项,种]{adj.}{talento; habilidade inata; dom natural; dons; dotado naturalmente}
  \synonymref{天才}{tian1cai2}
  \synonymref{天分}{tian1fen4}
  \synonymref{天生}{tian1sheng1}
  \synonymref{天性}{tian1xing4}
  \synonymref{先天}{xian1tian1}
\end{EntryWithPhonetic}

\begin{EntryWithPhonetic}{天公}{tian1gong1}{4,4}{⼤,⼋}
  \definition{s.}{céu, paraíso | senhor do céu}
\end{EntryWithPhonetic}

\begin{EntryWithPhonetic}{天花板}{tian1hua1ban3}{4,7,8}{⼤,⾋,⽊}
  \definition{s.}{teto}
\end{EntryWithPhonetic}

\begin{EntryWithPhonetic}{天津}{tian1jin1}{4,9}{⼤,⽔}
  \definition*{s.}{Tianjin, um município no nordeste da China}
\end{EntryWithPhonetic}

\begin{EntryWithPhonetic}{天经地义}{tian1jing1-di4yi4}{4,8,6,3}{⼤,⽷,⼟,⼂}[HSK 7-9]
  \definition{expr.}{(em conformidade com) os princípios do céu e da terra; correto e apropriado; perfeitamente justificado; algo natural; lei do céu e princípio da terra; Figurativo: correto e próprio}
  \synonymref{理所当然}{li3suo3dang1ran2}
\end{EntryWithPhonetic}

\begin{EntryWithPhonetic}{天空}{tian1kong1}{4,8}{⼤,⽳}[HSK 3]
  \definition{s.}{o céu; o firmamento}
\end{EntryWithPhonetic}

\begin{EntryWithPhonetic}{天平}{tian1ping2}{4,5}{⼤,⼲}[HSK 7-9]
  \definition{s.}{balança; balanço analítico}
  \definition{s.}{Signo Libra (天秤座)}
  \seealsoref{天秤座}{tian1cheng4zuo4}
\end{EntryWithPhonetic}

\begin{EntryWithPhonetic}{天气}{tian1qi4}{4,4}{⼤,⽓}[HSK 1]
  \definition{s.}{clima, tempo; mudanças meteorológicas que ocorrem na atmosfera em uma determinada área e durante um determinado período de tempo, tais como temperatura, umidade, pressão atmosférica, precipitação, vento, nuvens, etc.}
\end{EntryWithPhonetic}

\begin{EntryWithPhonetic}{天桥}{tian1qiao2}{4,10}{⼤,⽊}[HSK 7-9]
  \definition{s.}{ponte elevada; ponte plataforma; passarela suspensa; faixa de pedestres elevada; pontes erguidas sobre ferrovias, rodovias, ruas, etc., para facilitar a circulação de pedestres | ponte (um instrumento semelhante a uma escada); um equipamento esportivo em formato de ponte de madeira, é alto e comprido, com escadas em ambas as extremidades}
\end{EntryWithPhonetic}

\begin{EntryWithPhonetic}{天然}{tian1ran2}{4,12}{⼤,⽕}[HSK 6]
  \definition{adj.}{natural; produzido ou ocorrido narturalmente}
\end{EntryWithPhonetic}

\begin{EntryWithPhonetic}{天然气}{tian1ran2qi4}{4,12,4}{⼤,⽕,⽓}[HSK 5]
  \definition{s.}{gás; gás natural; gás combustível produzido em campos petrolíferos, carboníferos e pântanos}
\end{EntryWithPhonetic}

\begin{EntryWithPhonetic}{天上}{tian1shang4}{4,3}{⼤,⼀}[HSK 2]
  \definition[期]{s.}{o céu; o paraíso}
\end{EntryWithPhonetic}

\begin{EntryWithPhonetic}{天生}{tian1sheng1}{4,5}{⼤,⽣}[HSK 7-9]
  \definition{adj.}{inato; inerente; congênito; nascido; descreve algo presente desde o nascimento; que ocorre naturalmente}
  \synonymref{生成}{sheng1cheng2}
  \synonymref{天才}{tian1cai2}
  \synonymref{天分}{tian1fen4}
  \synonymref{天赋}{tian1fu4}
  \synonymref{先天}{xian1tian1}
  \antonymref{后天}{hou4tian1}
  \antonymref{遗传}{yi2chuan2}
\end{EntryWithPhonetic}

\begin{EntryWithPhonetic}{天使}{tian1shi3}{4,8}{⼤,⼈}[HSK 7-9]
  \definition[个,位,名]{s.}{anjo; em religiões como o judaísmo, o cristianismo e o islamismo, os anjos são frequentemente representados como meninas ou crianças aladas, e hoje em dia o termo é comumente usado para descrever pessoas inocentes e adoráveis ​​(geralmente referindo-se a mulheres ou crianças) | enviado imperial; mensageiro do imperador}
\end{EntryWithPhonetic}

\begin{EntryWithPhonetic}{天堂}{tian1tang2}{4,11}{⼤,⼟}[HSK 6]
  \definition[间]{s.}{paraíso, céu; em algumas religiões, refere"-se ao paraíso para onde as almas das pessoas boas retornam após a morte (diferente do 地狱) | lugar perfeito; ambiente de vida extremamente feliz e bonito; uma metáfora para um ambiente de vida feliz e bonito}
  \seealsoref{地狱}{di4yu4}
\end{EntryWithPhonetic}

\begin{EntryWithPhonetic}{天天}{tian1tian1}{4,4}{⼤,⼤}
  \definition{adv.}{todo dia}
\end{EntryWithPhonetic}

\begin{EntryWithPhonetic}{天文}{tian1wen2}{4,4}{⼤,⽂}[HSK 5]
  \definition[对]{s.}{astronomia; a distribuição e o movimento dos corpos celestes, como o sol, a lua e as estrelas, no universo}
\end{EntryWithPhonetic}

\begin{EntryWithPhonetic}{天下}{tian1xia4}{4,3}{⼤,⼀}[HSK 6]
  \definition[期]{s.}{China ou o mundo; refere"-se à China ou ao mundo | dominação; o poder dominante de um país | situação; um determinado campo; uma metáfora para uma determinada área ou situação}
\end{EntryWithPhonetic}

\begin{EntryWithPhonetic}{天线}{tian1xian4}{4,8}{⼤,⽷}[HSK 7-9]
  \definition{s.}{antena | conexão com altos funcionários | mastro}
\end{EntryWithPhonetic}

\begin{EntryWithPhonetic}{天性}{tian1xing4}{4,8}{⼤,⼼}[HSK 7-9]
  \definition{s.}{natureza; instintos naturais; refere"-se às qualidades inatas ou ao temperamento de uma pessoa |  inteligência; qualidades humanas}
  \synonymref{本性}{ben3xing4}
  \synonymref{个性}{ge4xing4}
  \synonymref{天才}{tian1cai2}
  \synonymref{天分}{tian1fen4}
  \synonymref{天赋}{tian1fu4}
  \synonymref{性格}{xing4ge2}
  \antonymref{后天}{hou4tian1}
\end{EntryWithPhonetic}

\begin{EntryWithPhonetic}{天择}{tian1ze2}{4,8}{⼤,⼿}
  \definition{s.}{seleção natural}
\end{EntryWithPhonetic}

\begin{EntryWithPhonetic}{天真}{tian1zhen1}{4,10}{⼤,⼗}[HSK 4]
  \definition{adj.}{ingênuo; inocente; ignorante; simples de coração, direto por natureza, livre de fingimento e hipocrisia}
\end{EntryWithPhonetic}

\begin{EntryWithPhonetic}{天主教}{tian1zhu3jiao4}{4,5,11}{⼤,⼂,⽁}[HSK 7-9]
  \definition*{s.}{Catolicismo; uma das antigas igrejas do cristianismo, após a queda do Império Romano do Ocidente em 476 d.C., viu uma divisão entre os ramos oriental e ocidental do cristianismo; suas características incluem a unidade, a santidade e o catolicismo; a adoração a Deus e a Jeová, e a veneração de Maria como a Virgem Maria; A Igreja Católica é um sistema hierárquico, que enfatiza a obediência dos fiéis à autoridade eclesiástica; também é conhecida como ``Igreja Católica Romana'' ou ``Igreja Romana''}
\end{EntryWithPhonetic}

\begin{EntryWithPhonetic}{天柱}{tian1zhu4}{4,9}{⼤,⽊}
  \definition{s.}{pilar celestial, que sustenta o céu}
\end{EntryWithPhonetic}

%%%%%%%%%% 兲 %%%%%%%%%%
\subsection*{兲}\addcontentsline{loh}{figure}{兲 \dpy{tian1}}

\begin{EntryWithPhonetic}{兲}{tian1}{6}{⼋}
  \variantof{天}
\end{EntryWithPhonetic}

%%%%%%%%%% 添 %%%%%%%%%%
\subsection*{添}\addcontentsline{loh}{figure}{添 \dpy{tian1}}

\begin{EntryWithPhonetic}{添}{tian1}{11}{⽔}[HSK 6]
  \definition{v.}{adicionar; aumentar | dar à luz}
\end{EntryWithPhonetic}

\begin{EntryWithPhonetic}{添加}{tian1jia1}{11,5}{⽔,⼒}[HSK 7-9]
  \definition{v.}{adicionar à; aumentar; adicionar}
  \synonymref{补充}{bu3chong1}
  \synonymref{加上}{jia1shang5}
  \synonymref{增加}{zeng1jia1}
  \antonymref{扣除}{kou4chu2}
  \antonymref{去除}{qu4chu2}
  \antonymref{删除}{shan1chu2}
\end{EntryWithPhonetic}

%%%%%%%%%% 田 %%%%%%%%%%
\subsection*{田}\addcontentsline{loh}{figure}{田 \dpy{tian2}}

\begin{EntryWithPhonetic}{田}{tian2}{5}{⽥}[HSK 6][Kangxi 102]
  \definition*{s.}{Sobrenome: Tian}
  \definition[亩,块,片]{s.}{campo; terra; terra de cultivo | área aberta rica em algum produto natural; campo}
  \definition{v.}{(arcaico) caçar}
\end{EntryWithPhonetic}

\begin{EntryWithPhonetic}{田径}{tian2jing4}{5,8}{⽥,⼻}[HSK 6]
  \definition{s.}{Esporte: atletismo}[他参加了这次的田径赛。===Ele participou da competição de atletismo.]
\end{EntryWithPhonetic}

\begin{EntryWithPhonetic}{田园}{tian2yuan2}{5,7}{⽥,⼞}
  \definition{adj.}{bucólico}
  \definition{s.}{campo | interior | rural}
\end{EntryWithPhonetic}

%%%%%%%%%% 钿 %%%%%%%%%%
\subsection*{钿}\addcontentsline{loh}{figure}{钿 \dpy{tian2}}

\begin{EntryWithPhonetic}{钿}{tian2}{10}{⾦}
  \definition{s.}{(dialeto) moeda | dinheiro; moeda | uma quantia de dinheiro}
  \seeref{dian4}
\end{EntryWithPhonetic}

%%%%%%%%%% 甜 %%%%%%%%%%
\subsection*{甜}\addcontentsline{loh}{figure}{甜 \dpy{tian2}}

\begin{EntryWithPhonetic}{甜}{tian2}{11}{⽢}[HSK 3]
  \definition{adj.}{doce; melado | agradável; confortável; fazer as pessoas se sentirem confortáveis e felizes | (sono) profundo | feliz; descreve o sentimento de felicidade}
\end{EntryWithPhonetic}

\begin{EntryWithPhonetic}{甜酒}{tian2jiu3}{11,10}{⽢,⾣}
  \definition{s.}{licor doce}
\end{EntryWithPhonetic}

\begin{EntryWithPhonetic}{甜菊}{tian2ju2}{11,11}{⽢,⾋}
  \definition{s.}{estévia, arbusto cujas folhas produzem um substituto para o açúcar}
\end{EntryWithPhonetic}

\begin{EntryWithPhonetic}{甜美}{tian2mei3}{11,9}{⽢,⽺}[HSK 7-9]
  \definition{adj.}{doce; melado; adocicado | agradável; refrescante; descreve uma sensação de prazer, conforto ou beleza}
  \synonymref{甜蜜}{tian2mi4}
  \antonymref{苦恼}{ku3nao3}
\end{EntryWithPhonetic}

\begin{EntryWithPhonetic}{甜蜜}{tian2mi4}{11,14}{⽢,⾍}[HSK 7-9]
  \definition{adj.}{doce; feliz; descrevendo a sensação de felicidade, alegria e conforto}
  \seealsoref{甜}{tian2}
  \antonymref{痛苦}{tong4ku3}
\end{EntryWithPhonetic}

\begin{EntryWithPhonetic}{甜品}{tian2pin3}{11,9}{⽢,⼝}
  \definition{s.}{sobremesa}
\end{EntryWithPhonetic}

\begin{EntryWithPhonetic}{甜食}{tian2shi2}{11,9}{⽢,⾷}
  \definition{s.}{doces | sobremesa}
\end{EntryWithPhonetic}

\begin{EntryWithPhonetic}{甜酸}{tian2suan1}{11,14}{⽢,⾣}
  \definition{adj.}{agridoce}
\end{EntryWithPhonetic}

\begin{EntryWithPhonetic}{甜甜圈}{tian2tian2quan1}{11,11,11}{⽢,⽢,⼞}
  \definition{s.}{rosquinha | \emph{doughnut}}
\end{EntryWithPhonetic}

\begin{EntryWithPhonetic}{甜筒}{tian2tong3}{11,12}{⽢,⽵}
  \definition{s.}{sorvete de casquinha}
\end{EntryWithPhonetic}

\begin{EntryWithPhonetic}{甜头}{tian2tou5}{11,5}{⽢,⼤}[HSK 7-9]
  \definition{s.}{sabor doce; sabor agradável; (doçura) Um sabor levemente adocicado, geralmente referindo-se a um sabor delicioso | bom; benefício (como incentivo); (doçura) benefícios; vantagens (frequentemente referindo-se a algo tentador)}
  \synonymref{便宜}{bian4yi2}
  \synonymref{长处}{chang2chu4}
  \synonymref{好处}{hao3chu5}
  \synonymref{利益}{li4yi4}
  \synonymref{优点}{you1dian3}
\end{EntryWithPhonetic}

\begin{EntryWithPhonetic}{甜心}{tian2xin1}{11,4}{⽢,⼼}
  \definition{s.}{querido}
\end{EntryWithPhonetic}

\begin{EntryWithPhonetic}{甜言}{tian2yan2}{11,7}{⽢,⾔}
  \definition{s.}{boa conversa | palavras amáveis}
\end{EntryWithPhonetic}

\begin{EntryWithPhonetic}{甜玉米}{tian2 yu4mi3}{11,5,6}{⽢,⽟,⽶}
  \definition{s.}{milho doce}
\end{EntryWithPhonetic}

\begin{EntryWithPhonetic}{甜稚}{tian2zhi4}{11,13}{⽢,⽲}
  \definition{s.}{doce e inocente}
\end{EntryWithPhonetic}

%%%%%%%%%% 填 %%%%%%%%%%
\subsection*{填}\addcontentsline{loh}{figure}{填 \dpy{tian2}}

\begin{EntryWithPhonetic}{填}{tian2}{13}{⼟}[HSK 4]
  \definition{v.}{encher; rechear | reabastecer; suplementar; complementar | preencher; escrever dados em uma caixa (em um questionário ou formulário da \emph{Web})}
\end{EntryWithPhonetic}

\begin{EntryWithPhonetic}{填补}{tian2bu3}{13,7}{⼟,⾐}[HSK 7-9]
  \definition{v.}{preencher | preencher (uma vaga, lacuna, etc.); assumir a responsabilidade}
  \synonymref{补充}{bu3chong1}
  \synonymref{弥补}{mi2bu3}
  \synonymref{填充}{tian2chong1}
  \synonymref{增加}{zeng1jia1}
\end{EntryWithPhonetic}

\begin{EntryWithPhonetic}{填充}{tian2chong1}{13,6}{⼟,⼉}[HSK 7-9]
  \definition{v.}{encher; rechear; preencher | preencher os espaços em branco (em uma prova) | preencher; empacotar; estofar; acolchoar}
  \synonymref{补充}{bu3chong1}
  \synonymref{弥补}{mi2bu3}
  \synonymref{填补}{tian2bu3}
  \synonymref{增加}{zeng1jia1}
\end{EntryWithPhonetic}

\begin{EntryWithPhonetic}{填空}{tian2/kong4}{13,8}{⼟,⽳}[HSK 4]
  \definition{v.+compl.}{preencher o espaço em branco (por exemplo, em um teste)}
\end{EntryWithPhonetic}

\begin{EntryWithPhonetic}{填写}{tian2xie3}{13,5}{⼟,⼍}[HSK 7-9]
  \definition{v.}{escrever; preencher; completar; escrever (texto ou números) nos espaços em branco de formulários, documentos, etc., conforme necessário}
  \synonymref{填}{tian2}
\end{EntryWithPhonetic}

%%%%%%%%%% 舔 %%%%%%%%%%
\subsection*{舔}\addcontentsline{loh}{figure}{舔 \dpy{tian3}}

\begin{EntryWithPhonetic}{舔}{tian3}{14}{⾆}[HSK 7-9]
  \definition{v.}{lamber; dar uma lambida; tocar com a língua}
\end{EntryWithPhonetic}

%%%%%%%%%% 挑 %%%%%%%%%%
\subsection*{挑}\addcontentsline{loh}{figure}{挑 \dpy{tiao1}}

\begin{EntryWithPhonetic}{挑}{tiao1}{9}{⼿}[HSK 4]
  \definition{clas.}{usado para coisas que são escolhidas ou selecionadas | usado para coisas que podem ser usadas como palhetas}
  \definition{s.}{vara comprida com algo pendurado nas pontas; haste de transporte}
  \definition{v.}{escolher; selecionar | fazer picuinhas; ser hipercrítico; ser meticuloso; ser excessivamente rigoroso nos detalhes | carregar com uma haste de transporte; carregar no ombro; pendurar coisas nas pontas de varas longas e carregá-las em seus ombros}
  \seeref{tiao3}
\end{EntryWithPhonetic}

\begin{EntryWithPhonetic}{挑剔}{tiao1ti5}{9,10}{⼿,⼑}[HSK 7-9]
  \definition{v.}{ser meticuloso; ser excessivamente crítico; ser exigente; ser excessivamente crítico em relação aos detalhes}
  \synonymref{批判}{pi1pan4}
  \synonymref{评论}{ping2lun4}
  \synonymref{指责}{zhi3ze2}
\end{EntryWithPhonetic}

\begin{EntryWithPhonetic}{挑选}{tiao1xuan3}{9,9}{⼿,⾡}[HSK 4]
  \definition{v.}{escolher; optar; selecionar; escolher a pessoa ou coisa certa para o trabalho}
\end{EntryWithPhonetic}

%%%%%%%%%% 条 %%%%%%%%%%
\subsection*{条}\addcontentsline{loh}{figure}{条 \dpy{tiao2}}

\begin{EntryWithPhonetic}{条}{tiao2}{7}{⽊}[HSK 2]
  \definition*{s.}{Sobrenome: Tiao}
  \definition{clas.}{usado para objetos longos e finos; usado para sintetizar certas coisas longas e retangulares em quantidades fixas | usado para itemização | aplicado ao corpo humano}
  \definition{s.}{galho; galhos finos e longos | tira; faixa | item; artigo | ordem; método | nota; anotação em papel}
\end{EntryWithPhonetic}

\begin{EntryWithPhonetic}{条幅}{tiao2fu2}{7,12}{⽊,⼱}
  \definition{s.}{faixa | banner | pergaminho de parede (para pintura ou caligrafia)}
\end{EntryWithPhonetic}

\begin{EntryWithPhonetic}{条贯}{tiao2guan4}{7,8}{⽊,⾙}
  \definition{s.}{ordem | procedimentos | sequência | sistema}
\end{EntryWithPhonetic}

\begin{EntryWithPhonetic}{条件}{tiao2jian4}{7,6}{⽊,⼈}[HSK 2]
  \definition[个,项,些]{s.}{condição; termo; fator; fatores que restringem a ocorrência, existência ou desenvolvimento das coisas | requisito; pré-requisito; qualificação; requisitos ou padrões estabelecidos para determinadas coisas | situação; estado; condição}
\end{EntryWithPhonetic}

\begin{EntryWithPhonetic}{条款}{tiao2kuan3}{7,12}{⽊,⽋}[HSK 7-9]
  \definition[项,个]{s.}{artigo; disposição; cláusula (em um documento formal); itens em leis, tratados, estatutos sociais e outros documentos e contratos}
  \synonymref{条件}{tiao2jian4}
  \synonymref{条目}{tiao2mu4}
\end{EntryWithPhonetic}

\begin{EntryWithPhonetic}{条例}{tiao2li4}{7,8}{⽊,⼈}[HSK 7-9]
  \definition[项]{s.}{regras; decretos; imperativo; regulamentos; regras e regulamentos formulados por departamentos e organizações administrativas nacionais}
  \synonymref{规则}{gui1ze2}
\end{EntryWithPhonetic}

\begin{EntryWithPhonetic}{条目}{tiao2mu4}{7,5}{⽊,⽬}
  \definition{s.}{cláusulas e subcláusulas (em documento formal) | verbete (em um dicionário, enciclopédia, etc.)}
\end{EntryWithPhonetic}

\begin{EntryWithPhonetic}{条约}{tiao2yue1}{7,6}{⽊,⽷}[HSK 7-9]
  \definition[个,项,些]{s.}{tratado; pacto; instrumentos assinados entre Estados referentes a direitos e obrigações em matéria política, militar, econômica ou cultural}
  \synonymref{公约}{gong1yue1}
  \synonymref{合同}{he2tong5}
  \synonymref{契约}{qi4yue1}
  \synonymref{协议}{xie2yi4}
\end{EntryWithPhonetic}

%%%%%%%%%% 调 %%%%%%%%%%
\subsection*{调}\addcontentsline{loh}{figure}{调 \dpy{tiao2}}

\begin{EntryWithPhonetic}{调}{tiao2}{10}{⾔}[HSK 3]
  \definition{adj.}{harmonioso; boa coordenação}
  \definition{v.}{misturar; ajustar; fazer o ajuste uniforme e apropriado | provocar; importunar; fazer pouco de | incitar; instigar; provocar; semear discórdia | mediar; trazer harmonia}
  \seeref{diao4}
\end{EntryWithPhonetic}

\begin{EntryWithPhonetic}{调节}{tiao2jie2}{10,5}{⾔,⾋}[HSK 5]
  \definition{v.}{regular; ajustar; ajustar e controlar de várias maneiras para atender aos requisitos}
\end{EntryWithPhonetic}

\begin{EntryWithPhonetic}{调解}{tiao2jie3}{10,13}{⾔,⾓}[HSK 5]
  \definition{v.}{mediar; fazer as pazes; resolver conflitos através da persuasão}
\end{EntryWithPhonetic}

\begin{EntryWithPhonetic}{调侃}{tiao2kan3}{10,8}{⾔,⼈}[HSK 7-9]
  \definition{v.}{ridicularizar; zombar; caçoar; brincar; provocar ou zombar com humor}
  \synonymref{嘲弄}{chao2nong4}
  \synonymref{嘲笑}{chao2xiao4}
  \synonymref{讥笑}{ji1xiao4}
  \synonymref{戏弄}{xi4nong4}
\end{EntryWithPhonetic}

\begin{EntryWithPhonetic}{调控}{tiao2kong4}{10,11}{⾔,⼿}[HSK 7-9]
  \definition{v.}{regular e controlar}
  \synonymref{操作}{cao1zuo4}
  \synonymref{调节}{tiao2jie2}
  \synonymref{调整}{tiao2zheng3}
\end{EntryWithPhonetic}

\begin{EntryWithPhonetic}{调料}{tiao2liao4}{10,10}{⾔,⽃}[HSK 7-9]
  \definition[种]{s.}{tempero; condimento; aromatizante; no preparo dos pratos, os ingredientes usados ​​para temperar os alimentos incluem óleo, sal, molho de soja, vinagre, cebolinha, gengibre e alho}
\end{EntryWithPhonetic}

\begin{EntryWithPhonetic}{调律}{tiao2lv4}{10,9}{⾔,⼻}
  \definition{v.}{afinar (por exemplo, um piano)}
\end{EntryWithPhonetic}

\begin{EntryWithPhonetic}{调皮}{tiao2pi2}{10,5}{⾔,⽪}[HSK 4]
  \definition{adj.}{travesso; malicioso; malandro | indisciplinado; desordeiro; indomável; astuto | inteligente e desonesto}
\end{EntryWithPhonetic}

\begin{EntryWithPhonetic}{调试}{tiao2shi4}{10,8}{⾔,⾔}[HSK 7-9]
  \definition{v.}{depurar; testar estabilidade; testar e ajustar (máquinas, instrumentos, etc.)}
\end{EntryWithPhonetic}

\begin{EntryWithPhonetic}{调整}{tiao2zheng3}{10,16}{⾔,⽁}[HSK 3]
  \definition{v.}{ajustar; revisar; regularizar; fazer as alterações apropriadas no estado original para se adaptar à nova situação}
\end{EntryWithPhonetic}

%%%%%%%%%% 挑 %%%%%%%%%%
\subsection*{挑}\addcontentsline{loh}{figure}{挑 \dpy{tiao3}}

\begin{EntryWithPhonetic}{挑}{tiao3}{9}{⼿}[HSK 4]
  \definition{s.}{um dos traços dos caracteres chineses; inclinado para cima da esquerda para a direita}
  \definition{v.}{levantar; elevar; erguer | levantar ou apoiar com uma extremidade de uma vara ou objeto semelhante; segurar ou apoiar com a ponta de uma vara etc. | causar conflitos deliberadamente; provocar deliberadamente um conflito | (método de bordado) usar uma agulha para levantar os fios de urdidura ou trama, com a agulha e a linha passando por baixo para formar padrões e desenhos}
  \seeref{tiao1}
\end{EntryWithPhonetic}

\begin{EntryWithPhonetic}{挑起}{tiao3qi3}{9,10}{⼿,⾛}[HSK 7-9]
  \definition{v.}{provocar; incitar; instigar}
\end{EntryWithPhonetic}

\begin{EntryWithPhonetic}{挑衅}{tiao3xin4}{9,11}{⼿,⾎}[HSK 7-9]
  \definition{s.}{provocação}
  \definition{v.}{provocar; causar problemas; tentar causar conflito ou guerra}
  \synonymref{搬弄}{ban1nong4}
  \synonymref{挑战}{tiao3/zhan4}
  \antonymref{顺从}{shun4cong2}
  \antonymref{妥协}{tuo3xie2}
  \antonymref{友好}{you3hao3}
\end{EntryWithPhonetic}

\begin{EntryWithPhonetic}{挑战}{tiao3/zhan4}{9,9}{⼿,⼽}[HSK 4]
  \definition{v.+compl.}{desafiar; deixar um oponente deliberadamente irritado e sair para lutar ou lutar consigo mesmo; estimular um oponente a lutar consigo mesmo}
\end{EntryWithPhonetic}

%%%%%%%%%% 跳 %%%%%%%%%%
\subsection*{跳}\addcontentsline{loh}{figure}{跳 \dpy{tiao4}}

\begin{EntryWithPhonetic}{跳}{tiao4}{13}{⾜}[HSK 3]
  \definition{v.}{pular; saltar | mover para cima e para baixo | pular (por cima); fazer omissões | quicar; a força elástica faz com que o objeto se mova repentinamente para cima | pulsar; palpitar; contrair-se | pular sobre;  saltar sobre; cruzar}
\end{EntryWithPhonetic}

\begin{EntryWithPhonetic}{跳槽}{tiao4/cao2}{13,15}{⾜,⽊}[HSK 7-9]
  \definition{v.+compl.}{mudar de emprego; abandonar uma ocupação em favor de outra; trocar de emprego com frequência; mudar de emprego constantemente; essa metáfora descreve alguém que deixa seu emprego ou local de trabalho original para trabalhar em outra organização ou mudar de profissão | pular o cocho; o gado saiu do cocho que lhe fora designado para comer em outros cochos}
  \synonymref{离职}{li2/zhi2}
  \antonymref{坚守}{jian1shou3}
\end{EntryWithPhonetic}

\begin{EntryWithPhonetic}{跳挡}{tiao4dang3}{13,9}{⾜,⼿}
  \definition{v.}{pular marcha (de um carro) | perder a marcha}
\end{EntryWithPhonetic}

\begin{EntryWithPhonetic}{跳电}{tiao4dian4}{13,5}{⾜,⽥}
  \definition{v.}{desarmar (um disjuntor ou interruptor)}
\end{EntryWithPhonetic}

\begin{EntryWithPhonetic}{跳动}{tiao4dong4}{13,6}{⾜,⼒}[HSK 7-9]
  \definition{v.}{bater; piscar; pular}
  \synonymref{跳跃}{tiao4yue4}
  \antonymref{静止}{jing4zhi3}
\end{EntryWithPhonetic}

\begin{EntryWithPhonetic}{跳高}{tiao4gao1}{13,10}{⾜,⾼}[HSK 3]
  \definition{s.}{salto em altura (atletismo)}
  \definition{v.}{saltar em altura}
\end{EntryWithPhonetic}

\begin{EntryWithPhonetic}{跳频}{tiao4pin2}{13,13}{⾜,⾴}
  \definition{s.}{FHSS, \emph{Frequency-Hopping Spread Spectrum}, método de transmissão de sinais de rádio}
\end{EntryWithPhonetic}

\begin{EntryWithPhonetic}{跳伞}{tiao4/san3}{13,6}{⾜,⼈}[HSK 7-9]
  \definition{s.}{salto de paraquedas}
  \definition{v.+compl.}{saltar de paraquedas; ejetar"-se}
\end{EntryWithPhonetic}

\begin{EntryWithPhonetic}{跳绳}{tiao4sheng2}{13,11}{⾜,⽷}
  \definition{v.}{pular corda}
\end{EntryWithPhonetic}

\begin{EntryWithPhonetic}{跳水}{tiao4shui3}{13,4}{⾜,⽔}[HSK 6]
  \definition{s.}{Esporte: mergulho}
  \definition{v.}{mergulhar | Figurativo: (preços, lucros, etc.) cair drasticamente; cair repentinamente; mergulhar; despencar | cometer suicídio pulando na água | mergulhar (na água)}
\end{EntryWithPhonetic}

\begin{EntryWithPhonetic}{跳跳糖}{tiao4tiao4tang2}{13,13,16}{⾜,⾜,⽶}
  \definition{s.}{\emph{Pop Rocks}, \emph{popping candy}}
\end{EntryWithPhonetic}

\begin{EntryWithPhonetic}{跳舞}{tiao4/wu3}{13,14}{⾜,⾇}[HSK 3]
  \definition{v.+compl.}{dançar (como performance); executar dança, especialmente dança de salão}
\end{EntryWithPhonetic}

\begin{EntryWithPhonetic}{跳远}{tiao4yuan3}{13,7}{⾜,⾡}[HSK 3]
  \definition{s.}{salto em distância (atletismo)}
\end{EntryWithPhonetic}

\begin{EntryWithPhonetic}{跳跃}{tiao4yue4}{13,11}{⾜,⾜}[HSK 7-9]
  \definition{v.}{pular; saltar; dar um pulo}
  \synonymref{跨越}{kua4yue4}
  \synonymref{跳动}{tiao4dong4}
\end{EntryWithPhonetic}

\begin{EntryWithPhonetic}{跳蚤}{tiao4zao5}{13,9}{⾜,⾍}
  \definition{s.}{pulga}
\end{EntryWithPhonetic}

%%%%%%%%%% 帖 %%%%%%%%%%
\subsection*{帖}\addcontentsline{loh}{figure}{帖 \dpy{tie1}}

\begin{EntryWithPhonetic}{帖}{tie1}{8}{⼱}
  \definition{adj.}{apropriado; adequado; seguro}
  \definition{v.}{obedecer; cumprir; seguir}
  \seeref{tie3}
  \seeref{tie4}
\end{EntryWithPhonetic}

%%%%%%%%%% 贴 %%%%%%%%%%
\subsection*{贴}\addcontentsline{loh}{figure}{贴 \dpy{tie1}}

\begin{EntryWithPhonetic}{贴}{tie1}{9}{⾙}[HSK 4]
  \definition{adj.}{submisso; obediente | apropriado}
  \definition{clas.}{usado em gessos, emplastros}
  \definition{s.}{subsídio; subvenção}
  \definition{v.}{grudar; colar | aninhar-se a; aconchegar-se a; aconchegar-se em | subsidiar; ajudar financeiramente}
\end{EntryWithPhonetic}

\begin{EntryWithPhonetic}{贴近}{tie1jin4}{9,7}{⾙,⾡}[HSK 7-9]
  \definition{adj.}{próximo; íntimo}
  \definition{v.}{apertar"-se contra; aconchegar"-se contra; encostar"-se a; aproximar}
  \synonymref{逼近}{bi1jin4}
  \synonymref{接近}{jie1jin4}
  \synonymref{靠近}{kao4jin4}
  \synonymref{靠拢}{kao4long3}
  \synonymref{亲切}{qin1qie4}
  \synonymref{贴切}{tie1qie4}
  \antonymref{远离}{yuan3li2}
\end{EntryWithPhonetic}

\begin{EntryWithPhonetic}{贴切}{tie1qie4}{9,4}{⾙,⼑}[HSK 7-9]
  \definition{adj.}{(palavras) apto; adequado; apropriado; conveniente}
  \synonymref{恰当}{qia4dang4}
  \synonymref{贴近}{tie1jin4}
\end{EntryWithPhonetic}

%%%%%%%%%% 帖 %%%%%%%%%%
\subsection*{帖}\addcontentsline{loh}{figure}{帖 \dpy{tie3}}

\begin{EntryWithPhonetic}{帖}{tie3}{8}{⼱}
  \definition{clas.}{prescrição (uma combinação de vários ingredientes medicinais); fórmula (usada para se referir a vários ingredientes em uma decocção); Dialeto: para fitoterapia}
  \definition{s.}{convite | nota; cartão | \emph{Internet}: \emph{post}; postagem; publicação; tópico}
  \seeref{tie1}
  \seeref{tie4}
  \seealsoref{帖儿}{tie3r5}
  \seealsoref{帖子}{tie3zi5}
\end{EntryWithPhonetic}

\begin{EntryWithPhonetic}{帖儿}{tie3r5}{8,2}{⼱,⼉}
  \definition{s.}{\emph{Internet}: \emph{post}; postagem; publicação}
\end{EntryWithPhonetic}

\begin{EntryWithPhonetic}{帖子}{tie3zi5}{8,3}{⼱,⼦}[HSK 7-9]
  \definition[个,张]{s.}{convite; notificação de convite para convidado | cartão; nota; pequenos pedaços de papel com escrita neles | postagens; tópicos; isso se refere a textos, imagens, etc., publicados na \emph{Internet} sobre um tópico específico}
\end{EntryWithPhonetic}

%%%%%%%%%% 铁 %%%%%%%%%%
\subsection*{铁}\addcontentsline{loh}{figure}{铁 \dpy{tie3}}

\begin{EntryWithPhonetic}{铁}{tie3}{10}{⾦}[HSK 3]
  \definition*{s.}{Sobrenome: Tie}
  \definition{adj.}{duro; forte; sólido como ferro; metáfora para natureza dura; vontade forte | violento | inabalável; inalterável; determinado; metáfora para violência ou crueldade}
  \definition{s.}{Fe; ferro | arma; armamento; refere"-se a facas, armas de fogo, etc.}
  \definition{v.}{resolver; determinar}
\end{EntryWithPhonetic}

\begin{EntryWithPhonetic}{铁轨}{tie3gui3}{10,6}{⾦,⾞}
  \definition[根]{s.}{trilho | trilho ferroviário}
\end{EntryWithPhonetic}

\begin{EntryWithPhonetic}{铁路}{tie3lu4}{10,13}{⾦,⾜}[HSK 3]
  \definition[条,公里]{s.}{ferrovia; estrada de ferro; uma estrada com trilhos de aço dispostos no leito da estrada para a circulação de trens}
\end{EntryWithPhonetic}

%%%%%%%%%% 帖 %%%%%%%%%%
\subsection*{帖}\addcontentsline{loh}{figure}{帖 \dpy{tie4}}

\begin{EntryWithPhonetic}{帖}{tie4}{8}{⼱}
  \definition{s.}{um livro contendo modelos de caligrafia ou pintura para os alunos copiarem; exemplos para copiar}
  \seeref{tie1}
  \seeref{tie3}
\end{EntryWithPhonetic}

%%%%%%%%%% 厅 %%%%%%%%%%
\subsection*{厅}\addcontentsline{loh}{figure}{厅 \dpy{ting1}}

\begin{EntryWithPhonetic}{厅}{ting1}{4}{⼚}[HSK 5]
  \definition{s.}{salão; sala grande para reuniões ou receber convidados | escritório; nome de um departamento administrativo de uma grande organização | departamento governamental a nível provincial; nomes de alguns órgãos estaduais}
\end{EntryWithPhonetic}

%%%%%%%%%% 听 %%%%%%%%%%
\subsection*{听}\addcontentsline{loh}{figure}{听 \dpy{ting1}}

\begin{EntryWithPhonetic}{听}{ting1}{7}{⼝}[HSK 1]
  \definition{clas.}{latas; usado para bebidas e alimentos para levar consigo}
  \definition{s.}{lata; embalagem metálica; recipiente cilíndrico utilizado para armazenar bebidas, alimentos, etc.}
  \definition{v.}{ouvir; escutar | obedecer; dar ouvidos; aceitar | supervisionar; administrar; tratar (assuntos políticos); julgar (casos) | permitir; deixar ser; deixar fazer}
  \seeref{yin3}
\end{EntryWithPhonetic}

\begin{EntryWithPhonetic}{听从}{ting1cong2}{7,4}{⼝,⼈}[HSK 7-9]
  \definition{v.}{obedecer; acatar; cumprir com; fazer as coisas de acordo com os desejos dos outros}
  \synonymref{服从}{fu2cong2}
  \synonymref{顺从}{shun4cong2}
  \synonymref{听命}{ting1ming4}
  \synonymref{听取}{ting1qu3}
  \synonymref{遵守}{zun1shou3}
  \antonymref{使唤}{shi3huan5}
  \antonymref{违反}{wei2fan3}
\end{EntryWithPhonetic}

\begin{EntryWithPhonetic}{听到}{ting1dao4}{7,8}{⼝,⼑}[HSK 1]
  \definition{v.}{ouvir, escutar; ouvir atentamente, escutar atentamente}
\end{EntryWithPhonetic}

\begin{EntryWithPhonetic}{听断}{ting1duan4}{7,11}{⼝,⽄}
  \definition{v.}{ouvir e decidir | julgar (ou seja, ouvir e julgar em um tribunal)}
\end{EntryWithPhonetic}

\begin{EntryWithPhonetic}{听骨}{ting1gu3}{7,9}{⼝,⾻}
  \definition{s.}{ossículos (do ouvido médio)}
  \seealsoref{听小骨}{ting1xiao3gu3}
\end{EntryWithPhonetic}

\begin{EntryWithPhonetic}{听话}{ting1/hua4}{7,8}{⼝,⾔}[HSK 7-9]
  \definition{v.+compl.}{ser obediente; escutar as palavras dos mais velhos ou dos líderes | ouvir; escutar; ouvir as pessoas falarem | aguardar a resposta}
  \synonymref{懂事}{dong3shi4}
  \synonymref{乖乖}{guai1guai1}
  \synonymref{乖巧}{guai1qiao3}
  \antonymref{淘气}{tao2/qi4}
  \antonymref{调皮}{tiao2pi2}
\end{EntryWithPhonetic}

\begin{EntryWithPhonetic}{听会}{ting1hui4}{7,6}{⼝,⼈}
  \definition{v.}{participar de uma reunião (e ouvir o que é discutido)}
\end{EntryWithPhonetic}

\begin{EntryWithPhonetic}{听见}{ting1/jian5}{7,4}{⼝,⾒}[HSK 1]
  \definition{v.+compl.}{ouvir; conseguir ouvir}
\end{EntryWithPhonetic}

\begin{EntryWithPhonetic}{听讲}{ting1/jiang3}{7,6}{⼝,⾔}[HSK 2]
  \definition{v.+compl.}{assistir a uma palestra; ouvir palestras ou discursos}
\end{EntryWithPhonetic}

\begin{EntryWithPhonetic}{听来}{ting1lai2}{7,7}{⼝,⽊}
  \definition{v.}{ouvir de algum lugar | soar (antigo, estrangeiro, excitante, certo, etc.) | soar como se (ou seja, dar uma impressão ao ouvinte)}
\end{EntryWithPhonetic}

\begin{EntryWithPhonetic}{听力}{ting1li4}{7,2}{⼝,⼒}[HSK 3]
  \definition{s.}{audição; capacidade auditiva | compreensão auditiva (na aprendizagem de línguas)}
\end{EntryWithPhonetic}

\begin{EntryWithPhonetic}{听力理解}{ting1li4li3jie3}{7,2,11,13}{⼝,⼒,⽟,⾓}
  \definition{s.}{compreensão auditiva}
\end{EntryWithPhonetic}

\begin{EntryWithPhonetic}{听命}{ting1ming4}{7,8}{⼝,⼝}
  \definition{v.}{obedecer ordens | receber ordens}
\end{EntryWithPhonetic}

\begin{EntryWithPhonetic}{听凭}{ting1ping2}{7,8}{⼝,⼏}
  \definition{v.}{permitir (alguém a fazer o que desejar)}
\end{EntryWithPhonetic}

\begin{EntryWithPhonetic}{听取}{ting1qu3}{7,8}{⼝,⼜}[HSK 6]
  \definition{v.}{ouvir (opiniões, reflexões, relatórios, etc.)}
\end{EntryWithPhonetic}

\begin{EntryWithPhonetic}{听说}{ting1shuo1}{7,9}{⼝,⾔}[HSK 2]
  \definition{v.}{ser informado; ouvir falar de; ouvir dizer | ouvir e falar}
\end{EntryWithPhonetic}

\begin{EntryWithPhonetic}{听随}{ting1sui2}{7,11}{⼝,⾩}
  \definition{v.}{permitir | obedecer}
\end{EntryWithPhonetic}

\begin{EntryWithPhonetic}{听戏}{ting1xi4}{7,6}{⼝,⼽}
  \definition{v.}{assistir a uma ópera | ver uma ópera}
\end{EntryWithPhonetic}

\begin{EntryWithPhonetic}{听小骨}{ting1xiao3gu3}{7,3,9}{⼝,⼩,⾻}
  \definition{s.}{ossículos (do ouvido médio)}
  \seealsoref{听骨}{ting1gu3}
\end{EntryWithPhonetic}

\begin{EntryWithPhonetic}{听写}{ting1xie3}{7,5}{⼝,⼍}[HSK 1]
  \definition{s.}{ditado}
  \definition{v.}{ouvir e escrever}
\end{EntryWithPhonetic}

\begin{EntryWithPhonetic}{听众}{ting1zhong4}{7,6}{⼝,⼈}[HSK 3]
  \definition[位,名,个]{s.}{audiência; ouvintes; pessoas que ouvem palestras, música ou transmissões}
\end{EntryWithPhonetic}

%%%%%%%%%% 聼 %%%%%%%%%%
\subsection*{聼}\addcontentsline{loh}{figure}{聼 \dpy{ting1}}

\begin{EntryWithPhonetic}{聼}{ting1}{19}{⼼}
  \variantof{听}
\end{EntryWithPhonetic}

%%%%%%%%%% 亭 %%%%%%%%%%
\subsection*{亭}\addcontentsline{loh}{figure}{亭 \dpy{ting2}}

\begin{EntryWithPhonetic}{亭}{ting2}{9}{⼇}
  \definition{s.}{pavilhão | cabine | quiosque}
\end{EntryWithPhonetic}

%%%%%%%%%% 停 %%%%%%%%%%
\subsection*{停}\addcontentsline{loh}{figure}{停 \dpy{ting2}}

\begin{EntryWithPhonetic}{停}{ting2}{11}{⼈}[HSK 2]
  \definition{adj.}{pronto; resolvido; bem organizado}
  \definition{clas.}{usado para partes (de um total); porções}
  \definition{v.}{parar; interromper; cessar; fazer uma pausa | permanecer; ficar; fazer uma parada (para descansar) | estacionar; ancorar; atracar}
\end{EntryWithPhonetic}

\begin{EntryWithPhonetic}{停办}{ting2ban4}{11,4}{⼈,⼒}
  \definition{v.}{cancelar | sair do negócio | desligar | terminar}
\end{EntryWithPhonetic}

\begin{EntryWithPhonetic}{停泊}{ting2bo2}{11,8}{⼈,⽔}[HSK 7-9]
  \definition{v.}{(navios) atracar; fundear; ancorar}
\end{EntryWithPhonetic}

\begin{EntryWithPhonetic}{停车}{ting2 che1}{11,4}{⼈,⾞}[HSK 2]
  \definition{v.}{(veículo) parar; frear | estacionar o veículo | parar; deixar de funcionar}
\end{EntryWithPhonetic}

\begin{EntryWithPhonetic}{停车场}{ting2che1chang3}{11,4,6}{⼈,⾞,⼟}[HSK 2]
  \definition[个]{s.}{estacionamento; área de estacionamento; local para estacionamento de veículos}
\end{EntryWithPhonetic}

\begin{EntryWithPhonetic}{停车位}{ting2che1wei4}{11,4,7}{⼈,⾞,⼈}[HSK 7-9]
  \definition[个]{s.}{vagas de estacionamento; um espaço onde carros ou outros veículos podem ser estacionados}
\end{EntryWithPhonetic}

\begin{EntryWithPhonetic}{停当}{ting2dang5}{11,6}{⼈,⼹}
  \definition{adj.}{realizado | preparado | assentado}
\end{EntryWithPhonetic}

\begin{EntryWithPhonetic}{停电}{ting2dian4}{11,5}{⼈,⽥}[HSK 7-9]
  \definition{v.}{cortar o fornecimento de energia; ter uma falha de energia}
  \antonymref{来电}{lai2dian4}
\end{EntryWithPhonetic}

\begin{EntryWithPhonetic}{停顿}{ting2dun4}{11,10}{⼈,⾴}[HSK 7-9]
  \definition{v.}{parar; interromper; pausar; o assunto foi suspenso ou interrompido | dar pausas (na fala)}
  \synonymref{堵塞}{du3se4}
  \synonymref{平息}{ping2xi1}
  \synonymref{停留}{ting2liu2}
  \synonymref{停止}{ting2zhi3}
  \synonymref{休息}{xiu1xi5}
  \synonymref{中断}{zhong1duan4}
  \antonymref{畅通}{chang4tong1}
  \antonymref{持续}{chi2xu4}
\end{EntryWithPhonetic}

\begin{EntryWithPhonetic}{停放}{ting2fang4}{11,8}{⼈,⽅}[HSK 7-9]
  \definition{v.}{estacionar (um veículo) | colocar (um caixão) | deixar algo (em um lugar) | atracar (um barco, etc.)}
\end{EntryWithPhonetic}

\begin{EntryWithPhonetic}{停工}{ting2gong1}{11,3}{⼈,⼯}
  \definition{v.}{parar de trabalhar | parar a produção}
\end{EntryWithPhonetic}

\begin{EntryWithPhonetic}{停火}{ting2/huo3}{11,4}{⼈,⽕}
  \definition{s.}{cessar-fogo}
  \definition{v.+compl.}{cessar fogo}
\end{EntryWithPhonetic}

\begin{EntryWithPhonetic}{停课}{ting2ke4}{11,10}{⼈,⾔}
  \definition{v.}{fechar (escola) | parar as aulas}
\end{EntryWithPhonetic}

\begin{EntryWithPhonetic}{停留}{ting2liu2}{11,10}{⼈,⽥}[HSK 5]
  \definition{v.}{permanecer; ficar por muito tempo; parar temporariamente em algum lugar, sem continuar avançando | permanecer; parar por um longo tempo; parar em um determinado estágio ou nível, sem evoluir}
\end{EntryWithPhonetic}

\begin{EntryWithPhonetic}{停息}{ting2xi1}{11,10}{⼈,⼼}
  \definition{v.}{cessar | parar}
\end{EntryWithPhonetic}

\begin{EntryWithPhonetic}{停下}{ting2xia4}{11,3}{⼈,⼀}[HSK 4]
  \definition{v.}{encerrar; desligar; parar}
\end{EntryWithPhonetic}

\begin{EntryWithPhonetic}{停歇}{ting2xie1}{11,13}{⼈,⽋}
  \definition{v.}{parar para descansar}
\end{EntryWithPhonetic}

\begin{EntryWithPhonetic}{停业}{ting2/ye4}{11,5}{⼈,⼀}[HSK 7-9]
  \definition{v.+compl.}{cessar as atividades comerciais; encerramento das atividades comerciais; finalizar os negócios; suspensão das atividades comerciais; fechar as portas}[修理内部,停业5天。===O estabelecimento ficará fechado por 5 dias para reparos internos.]
  \synonymref{倒闭}{dao3bi4}
  \synonymref{破产}{po4/chan3}
  \antonymref{开业}{kai1 ye4}
  \antonymref{开张}{kai1/zhang1}
  \antonymref{营业}{ying2ye4}
\end{EntryWithPhonetic}

\begin{EntryWithPhonetic}{停用}{ting2yong4}{11,5}{⼈,⽤}
  \definition{v.}{desabilitar | descontinuar | parar de usar | suspender}
\end{EntryWithPhonetic}

\begin{EntryWithPhonetic}{停止}{ting2zhi3}{11,4}{⼈,⽌}[HSK 3]
  \definition{v.}{parar; suspender; cessar; cancelar}
\end{EntryWithPhonetic}

%%%%%%%%%% 挺 %%%%%%%%%%
\subsection*{挺}\addcontentsline{loh}{figure}{挺 \dpy{ting3}}

\begin{EntryWithPhonetic}{挺}{ting3}{9}{⼿}[HSK 2,4]
  \definition{adj.}{rígido; ereto; vertical; reto | notável; destacado; distinto}
  \definition{adv.}{muito; bastante}
  \definition{clas.}{usado para metralhadoras}
  \definition{v.}{sobressair; endireitar-se; protrudir (protuberância ou saliência) | suportar; aguentar; resistir; perseverar}
\end{EntryWithPhonetic}

\begin{EntryWithPhonetic}{挺拔}{ting3ba2}{9,8}{⼿,⼿}
  \definition{adj.}{alto e reto}
\end{EntryWithPhonetic}

\begin{EntryWithPhonetic}{挺杆}{ting3gan3}{9,7}{⼿,⽊}
  \definition{s.}{tucho (peça de máquina)}
\end{EntryWithPhonetic}

\begin{EntryWithPhonetic}{挺过}{ting3guo4}{9,6}{⼿,⾡}
  \definition{s.}{sobreviver}
\end{EntryWithPhonetic}

\begin{EntryWithPhonetic}{挺好}{ting3hao3}{9,6}{⼿,⼥}[HSK 2]
  \definition{adj.}{nada mal; surpreendentemente bom}
\end{EntryWithPhonetic}

\begin{EntryWithPhonetic}{挺进}{ting3jin4}{9,7}{⼿,⾡}
  \definition{s.}{progresso | avanço}
  \definition{v.}{progredir | avançar}
\end{EntryWithPhonetic}

\begin{EntryWithPhonetic}{挺立}{ting3li4}{9,5}{⼿,⽴}
  \definition{v.}{ficar ereto | ficar de pé}
\end{EntryWithPhonetic}

\begin{EntryWithPhonetic}{挺身}{ting3shen1}{9,7}{⼿,⾝}
  \definition{v.}{endireitar as costas}
\end{EntryWithPhonetic}

\begin{EntryWithPhonetic}{挺尸}{ting3shi1}{9,3}{⼿,⼫}
  \definition{v.}{(coloquial) dormir | (literalmente) ficar deitado duro como um cadáver}
\end{EntryWithPhonetic}

\begin{EntryWithPhonetic}{挺腰}{ting3yao1}{9,13}{⼿,⾁}
  \definition{v.}{arquear as costas | endireitar as costas}
\end{EntryWithPhonetic}

\begin{EntryWithPhonetic}{挺住}{ting3zhu4}{9,7}{⼿,⼈}
  \definition{v.}{permanecer firme | manter-se firme (diante da adversidade ou da dor)}
\end{EntryWithPhonetic}

%%%%%%%%%% 通 %%%%%%%%%%
\subsection*{通}\addcontentsline{loh}{figure}{通 \dpy{tong1}}

\begin{EntryWithPhonetic}{通}{tong1}{10}{⾡}[HSK 2]
  \definition*{s.}{Sobrenome: Tong}
  \definition{adj.}{lógico; coerente | geral; comum | tudo; inteiro | aberto; através de | total}
  \definition{clas.}{(antigo) usado para cartas, telegramas, documentos oficiais, etc.}
  \definition{s.}{autoridade; especialista}
  \definition{suf.}{especialista}
  \definition{v.}{abrir; atravessar | abrir ou limpar cutucando ou espetando | levar a; ir a | conectar; comunicar | notificar; informar | compreender; saber | cutucar; dar uma pancada | transmitir; conectar; interagir | dominar; compreender; entender}
  \seeref{tong4}
\end{EntryWithPhonetic}

\begin{EntryWithPhonetic}{通报}{tong1bao4}{10,7}{⾡,⼿}[HSK 6]
  \definition[份]{s.}{circular | boletim; jornal; publicação | sumário; notificação para informações gerais}
  \definition{v.}{circular um aviso (aviso por escrito) | notificar; dar informações com; compartilhar informações com}
\end{EntryWithPhonetic}

\begin{EntryWithPhonetic}{通常}{tong1chang2}{10,11}{⾡,⼱}[HSK 3]
  \definition{adj.}{usual; normal; geral}
  \definition{adv.}{habitualmente; usualmente; geralmente; ordinariamente}
\end{EntryWithPhonetic}

\begin{EntryWithPhonetic}{通畅}{tong1chang4}{10,8}{⾡,⽥}[HSK 7-9]
  \definition{adj.}{claro; livre; aberto; desimpedido; desobstruído | fácil e suave; (pensamentos e escrita) suave}
  \synonymref{畅通}{chang4tong1}
  \synonymref{灵通}{ling2tong1}
  \synonymref{流畅}{liu2chang4}
  \synonymref{流利}{liu2li4}
  \synonymref{流通}{liu2tong1}
  \synonymref{通顺}{tong1shun4}
  \synonymref{通行}{tong1xing2}
  \antonymref{堵塞}{du3se4}
  \antonymref{障碍}{zhang4'ai4}
\end{EntryWithPhonetic}

\begin{EntryWithPhonetic}{通车}{tong1/che1}{10,4}{⾡,⾞}[HSK 7-9]
  \definition{v.+compl.}{(uma ferrovia, rodovia, ponte, etc.) abrir para o tráfego | prestar serviço de transporte | comutar | (uma localidade) possuir serviço de transporte}
\end{EntryWithPhonetic}

\begin{EntryWithPhonetic}{通道}{tong1dao4}{10,12}{⾡,⾡}[HSK 6]
  \definition[条,个]{s.}{acesso; corredor; passagem; caminhos que levam ao exterior de teatros, minas, etc. | passagem; via pública}
\end{EntryWithPhonetic}

\begin{EntryWithPhonetic}{通牒}{tong1die2}{10,13}{⾡,⽚}
  \definition{s.}{nota diplomática}
\end{EntryWithPhonetic}

\begin{EntryWithPhonetic}{通风}{tong1/feng1}{10,4}{⾡,⾵}[HSK 7-9]
  \definition{v.}{arejar; ventilar; permitir a circulação de ar | ser bem ventilado; ter circulação de ar; ser respirável | divulgar informações; vazar de informações}
  \synonymref{透气}{tou4/qi4}
  \antonymref{封闭}{feng1bi4}
\end{EntryWithPhonetic}

\begin{EntryWithPhonetic}{通告}{tong1gao4}{10,7}{⾡,⼝}[HSK 7-9]
  \definition[张,个,则]{s.}{aviso público; anúncio; circular; proclamação}
  \definition{v.}{dar publicidade; anunciar; proclamar | informar}
  \seealsoref{通报}{tong1bao4}
  \synonymref{告诉}{gao4su4}
  \synonymref{告示}{gao4shi5}
  \synonymref{公布}{gong1bu4}
  \synonymref{公告}{gong1gao4}
  \synonymref{通知}{tong1zhi1}
  \synonymref{宣布}{xuan1bu4}
\end{EntryWithPhonetic}

\begin{EntryWithPhonetic}{通观}{tong1guan1}{10,6}{⾡,⾒}
  \definition{v.}{ter uma visão geral de algo}
\end{EntryWithPhonetic}

\begin{EntryWithPhonetic}{通过}{tong1guo4}{10,6}{⾡,⾡}[HSK 2]
  \definition{prep.}{por; através de; por meio de; por meio de; meios, métodos, etc. para introduzir ações}
  \definition{v.}{atravessar; passar por; transitar | aprovar; adotar | solicitar o consentimento ou aprovação de}
\end{EntryWithPhonetic}

\begin{EntryWithPhonetic}{通红}{tong1hong2}{10,6}{⾡,⽷}[HSK 6]
  \definition{adj.}{muito vermelho; vermelho por completo}
\end{EntryWithPhonetic}

\begin{EntryWithPhonetic}{通话}{tong1hua4}{10,8}{⾡,⾔}[HSK 6]
  \definition{v.}{comunicar por telefone | conversar; comunicar; falar em uma língua que ambos possam entender}
\end{EntryWithPhonetic}

\begin{EntryWithPhonetic}{通缉}{tong1ji1}{10,12}{⾡,⽷}[HSK 7-9]
  \definition{v.}{incluir na lista de procurados; ordenar a prisão de um criminoso foragido; as autoridades de segurança pública ou judiciais emitem ordens em suas jurisdições para prender criminosos foragidos}
\end{EntryWithPhonetic}

\begin{EntryWithPhonetic}{通识}{tong1shi2}{10,7}{⾡,⾔}
  \definition{s.}{conhecimento comum | erudição | conhecimento geral | amplamente conhecido}
\end{EntryWithPhonetic}

\begin{EntryWithPhonetic}{通顺}{tong1shun4}{10,9}{⾡,⾴}[HSK 7-9]
  \definition{adj.}{suave; claro e coerente (na escrita); o artigo não apresenta erros lógicos ou gramaticais; a linguagem é fluida e natural}
  \synonymref{畅通}{chang4tong1}
  \synonymref{顺心}{shun4/xin1}
  \synonymref{顺畅}{shun4chang4}
  \synonymref{通畅}{tong1chang4}
  \antonymref{别扭}{bie4niu5}
\end{EntryWithPhonetic}

\begin{EntryWithPhonetic}{通俗}{tong1su2}{10,9}{⾡,⼈}[HSK 7-9]
  \definition{adj.}{popular; comum; simples e fácil de entender; adequado ao nível e às necessidades da pessoa média}
  \synonymref{广泛}{guang3fan4}
  \synonymref{平常}{ping2chang2}
  \synonymref{平凡}{ping2fan2}
  \synonymref{平方}{ping2fang1}
  \synonymref{普通}{pu3tong1}
  \synonymref{通常}{tong1chang2}
  \antonymref{高贵}{gao1gui4}
  \antonymref{深奥}{shen1'ao4}
  \antonymref{诗意}{shi1yi4}
\end{EntryWithPhonetic}

\begin{EntryWithPhonetic}{通通}{tong1tong1}{10,10}{⾡,⾡}[HSK 7-9]
  \definition{adv.}{todo; tudo; inteiramente; completamente; indica sumarização}
  \synonymref{统统}{tong3tong3}
\end{EntryWithPhonetic}

\begin{EntryWithPhonetic}{通往}{tong1wang3}{10,8}{⾡,⼻}[HSK 7-9]
  \definition{v.}{levar a (um determinado lugar)}
\end{EntryWithPhonetic}

\begin{EntryWithPhonetic}{通宵}{tong1xiao1}{10,10}{⾡,⼧}[HSK 7-9]
  \definition{s.}{a noite toda; a noite inteira; durante toda a noite}
  \synonymref{熬夜}{ao2/ye4}
\end{EntryWithPhonetic}

\begin{EntryWithPhonetic}{通信}{tong1/xin4}{10,9}{⾡,⼈}[HSK 3]
  \definition{v.+compl.}{corresponder; comunicar por carta; comunicar situações e informações escrevendo cartas | transmitir (ou transportar) mensagem; passar (ou transmitir) informação; usar ondas de rádio e outros sinais para transmitir texto, imagens, etc.}
\end{EntryWithPhonetic}

\begin{EntryWithPhonetic}{通行}{tong1xing2}{10,6}{⾡,⾏}[HSK 6]
  \definition{adj.}{atual; geral}
  \definition{v.}{passar (ou ir) através; passar por; atravessar | prevalecer; predominar; ser corrente | (pedestres, veículos, etc.) passar na linha de trânsito}
\end{EntryWithPhonetic}

\begin{EntryWithPhonetic}{通行证}{tong1xing2zheng4}{10,6,7}{⾡,⾏,⾔}[HSK 7-9]
  \definition{v.}{passar; permitir; conceder salvo-conduto; dar carta branca; permissões que autorizam a entrada e a saída de áreas designadas}
\end{EntryWithPhonetic}

\begin{EntryWithPhonetic}{通讯}{tong1xun4}{10,5}{⾡,⾔}[HSK 6]
  \definition[个,种]{s.}{relatório; comunicação; boletim informativo; correspondência; reportagem; despacho de notícias; artigos que relatam fatos objetivos ou números típicos de forma detalhada e vívida}
  \definition{v.}{usar equipamentos de telecomunicações para transmitir mensagens}
\end{EntryWithPhonetic}

\begin{EntryWithPhonetic}{通用}{tong1yong4}{10,5}{⾡,⽤}[HSK 5]
  \definition[家]{adj.}{de uso comum; universal; (em um determinado âmbito) de uso generalizado | intercambiável; alguns caracteres chineses com grafia diferente, mas pronúncia igual, podem ser usados indistintamente (alguns limitados a um determinado significado)}
\end{EntryWithPhonetic}

\begin{EntryWithPhonetic}{通知}{tong1zhi1}{10,8}{⾡,⽮}[HSK 2]
  \definition[份,个,张]{s.}{aviso; circular; notificação por escrito ou verbal}
  \definition{v.}{aconselhar; notificar; informar; dar aviso prévio}
\end{EntryWithPhonetic}

\begin{EntryWithPhonetic}{通知书}{tong1zhi1shu1}{10,8,4}{⾡,⽮,⼄}[HSK 4]
  \definition[份]{s.}{aviso; observação; notificação}
\end{EntryWithPhonetic}

%%%%%%%%%% 同 %%%%%%%%%%
\subsection*{同}\addcontentsline{loh}{figure}{同 \dpy{tong2}}

\begin{EntryWithPhonetic}{同}{tong2}{6}{⼝}[HSK 6]
  \definition{adj.}{como; igual; parecido; similar; o mesmo; sem diferença}
  \definition{adv.}{juntos; em comum; indica que diferentes atores realizam uma determinada ação juntos ou estão na mesma situação, o que equivale a 一同 ou 一起}
  \definition{v.}{ser o mesmo que}
  \seeref{tong4}
  \seealsoref{一起}{yi4qi3}
  \seealsoref{一同}{yi4tong2}
\end{EntryWithPhonetic}

\begin{EntryWithPhonetic}{同伴}{tong2ban4}{6,7}{⼝,⼈}[HSK 7-9]
  \definition[名,个]{s.}{companheiro; uma pessoa com quem você trabalha, mora ou realiza alguma atividade específica}
  \synonymref{伴侣}{ban4lv3}
  \synonymref{差错}{cha1cuo4}
  \synonymref{错误}{cuo4wu4}
  \synonymref{搭档}{da1dang4}
  \synonymref{过错}{guo4cuo4}
  \synonymref{伙伴}{huo3ban4}
  \synonymref{朋友}{peng2you5}
  \synonymref{同伙}{tong2huo3}
\end{EntryWithPhonetic}

\begin{EntryWithPhonetic}{同胞}{tong2bao1}{6,9}{⼝,⾁}[HSK 6]
  \definition{s.}{nascidos dos mesmos pais | compatriota; conterrâneo; pessoas do mesmo país ou etnia}
\end{EntryWithPhonetic}

\begin{EntryWithPhonetic}{同步}{tong2bu4}{6,7}{⼝,⽌}[HSK 7-9]
  \definition{s.}{sincronizar; sincronizar com; coordenar em tempo de progresso; geralmente se refere ao ritmo de ação coordenado e consistente de coisas interconectadas}
  \definition{s.}{sincronismo; sincronização; em ciência e tecnologia, refere-se a duas ou mais grandezas que se alteram ao longo do tempo, mantendo uma determinada relação relativa durante o processo de mudança}
  \synonymref{协调}{xie2tiao2}
  \synonymref{一起}{yi4qi3}
\end{EntryWithPhonetic}

\begin{EntryWithPhonetic}{同等}{tong2deng3}{6,12}{⼝,⽵}[HSK 7-9]
  \definition{adj.}{igual; que pertence à mesma classe social ou posição social; pessoas da mesma posição ou \emph{status}}
  \synonymref{平等}{ping2deng3}
  \synonymref{一律}{yi2lv4}
\end{EntryWithPhonetic}

\begin{EntryWithPhonetic}{同感}{tong2gan3}{6,13}{⼝,⼼}[HSK 7-9]
  \definition{s.}{consenso; simpatia; os mesmos pensamentos ou sentimentos}
\end{EntryWithPhonetic}

\begin{EntryWithPhonetic}{同行}{tong2hang2}{6,6}{⼝,⾏}[HSK 6]
  \definition{s.}{do mesmo ofício ou ocupação; pessoas no mesmo setor}
  \definition{v.}{ser do mesmo ofício ou ocupação; trabalhar no mesmo setor}
\end{EntryWithPhonetic}

\begin{EntryWithPhonetic}{同伙}{tong2huo3}{6,6}{⼝,⼈}[HSK 7-9]
  \definition[个]{s.}{parceiro; cúmplice; aliado}
  \definition{v.}{trabalhar em parceria; conspirar (para praticar o mal) | conspirar com; participar conjuntamente em uma organização ou envolver-se em determinada atividade | associar"-se}
  \synonymref{伴侣}{ban4lv3}
  \synonymref{伙伴}{huo3ban4}
  \synonymref{朋友}{peng2you5}
  \synonymref{同伴}{tong2ban4}
  \synonymref{同盟}{tong2meng2}
  \antonymref{对手}{dui4shou3}
\end{EntryWithPhonetic}

\begin{EntryWithPhonetic}{同类}{tong2lei4}{6,9}{⼝,⽶}[HSK 7-9]
  \definition{adj.}{similar; semelhante; do mesmo tipo; pessoas ou coisas do mesmo tipo}
  \synonymref{同样}{tong2yang4}
\end{EntryWithPhonetic}

\begin{EntryWithPhonetic}{同流合污}{tong2liu2he2wu1}{6,10,6,6}{⼝,⽔,⼝,⽔}
  \definition{expr.}{chafurdar na lama com alguém | seguir o mau exemplo dos outros}
\end{EntryWithPhonetic}

\begin{EntryWithPhonetic}{同盟}{tong2meng2}{6,13}{⼝,⽫}[HSK 7-9]
  \definition[个]{s.}{aliança; liga}
  \synonymref{合作}{he2zuo4}
  \synonymref{联盟}{lian2meng2}
  \synonymref{同伙}{tong2huo3}
  \synonymref{战友}{zhan4you3}
\end{EntryWithPhonetic}

\begin{EntryWithPhonetic}{同年}{tong2nian2}{6,6}{⼝,⼲}[HSK 7-9]
  \definition{adj.}{Dialeto: da mesma idade; no mesmo ano}
  \definition{s.}{mesmo ano | Obsoleto: candidatos que passaram nos exames imperiais no mesmo ano}
\end{EntryWithPhonetic}

\begin{EntryWithPhonetic}{同期}{tong2qi1}{6,12}{⼝,⽉}[HSK 6]
  \definition{s.}{o período correspondente; o mesmo período; no mesmo tempo}
\end{EntryWithPhonetic}

\begin{EntryWithPhonetic}{同情}{tong2qing2}{6,11}{⼝,⼼}[HSK 4]
  \definition{s.}{simpatia}
  \definition{v.}{simpatizar com; solidarizar-se; compadecer-se; ter empatia emocional pelo que os outros estão passando}
\end{EntryWithPhonetic}

\begin{EntryWithPhonetic}{同人}{tong2ren2}{6,2}{⼝,⼈}[HSK 7-9]
  \definition{s.}{colega de trabalho; pessoas do mesmo local de trabalho ou profissão | colega | entusiastas da cultura pop que criam fanfics etc.}
\end{EntryWithPhonetic}

\begin{EntryWithPhonetic}{同时}{tong2shi2}{6,7}{⼝,⽇}[HSK 2]
  \definition{conj.}{além disso; além do mais; ainda mais; indica uma relação de equivalência, geralmente com um significado mais profundo}
  \definition{s.}{enquanto isso; ao mesmo tempo}
\end{EntryWithPhonetic}

\begin{EntryWithPhonetic}{同事}{tong2shi4}{6,8}{⼝,⼅}[HSK 2]
  \definition[个,位,名]{s.}{companheiro; colega; colega de trabalho; pessoas que trabalham juntas}
  \definition{v.}{trabalhar no mesmo lugar; trabalhar juntos; trabalhar na mesma unidade}
\end{EntryWithPhonetic}

\begin{EntryWithPhonetic}{同屋}{tong2wu1}{6,9}{⼝,⼫}
  \definition[个]{s.}{companheiro de quarto | colega de quarto}
\end{EntryWithPhonetic}

\begin{EntryWithPhonetic}{同性恋}{tong2xing4lian4}{6,8,10}{⼝,⼼,⼼}
  \definition{s.}{homossexualidade | pessoa gay | amor gay}
\end{EntryWithPhonetic}

\begin{EntryWithPhonetic}{同学}{tong2xue2}{6,8}{⼝,⼦}[HSK 1]
  \definition[位,个,些]{s.}{colega de escola; colega de turma; colega de estudos; pessoas que estudam na mesma escola}
\end{EntryWithPhonetic}

\begin{EntryWithPhonetic}{同砚}{tong2yan4}{6,9}{⼝,⽯}
  \definition[位,个]{s.}{colega de classe | colega estudante}
\end{EntryWithPhonetic}

\begin{EntryWithPhonetic}{同样}{tong2yang4}{6,10}{⼝,⽊}[HSK 2]
  \definition{adj.}{igual; semelhante; similar; idêntico; sem diferença}
\end{EntryWithPhonetic}

\begin{EntryWithPhonetic}{同一}{tong2yi1}{6,1}{⼝,⼀}[HSK 6]
  \definition{adj.}{mesmo; idêntico}
  \definition[讲]{s.}{identidade; unidade}
\end{EntryWithPhonetic}

\begin{EntryWithPhonetic}{同意}{tong2yi4}{6,13}{⼝,⼼}[HSK 3]
  \definition{v.}{concordar; consentir; aprovar; concordar com; dizer sim}
\end{EntryWithPhonetic}

\begin{EntryWithPhonetic}{同志}{tong2zhi4}{6,7}{⼝,⼼}[HSK 7-9]
  \definition[个,位,名,些]{s.}{camarada; pessoas que compartilham objetivos comuns; especificamente, membros do mesmo partido político | título habitual usado em ocasiões formais; a forma como as pessoas se tratam atualmente é geralmente usada em contextos formais e é menos comum na linguagem falada}
  \antonymref{敌人}{di2ren2}
\end{EntryWithPhonetic}

\begin{EntryWithPhonetic}{同舟共济}{tong2zhou1-gong4ji4}{6,6,6,9}{⼝,⾈,⼋,⽔}[HSK 7-9]
  \definition{expr.}{atravessar um rio no mesmo barco; remar juntos em tempos difíceis; unir-se em momentos de adversidade; pessoas no mesmo barco se ajudam mutuamente; unem forças para superar dificuldades}
  \antonymref{各奔前程}{ge4ben4qian2cheng2}
\end{EntryWithPhonetic}

%%%%%%%%%% 铜 %%%%%%%%%%
\subsection*{铜}\addcontentsline{loh}{figure}{铜 \dpy{tong2}}

\begin{EntryWithPhonetic}{铜}{tong2}{11}{⾦}[HSK 7-9]
  \definition[块]{s.}{Cu; cobre}
\end{EntryWithPhonetic}

\begin{EntryWithPhonetic}{铜牌}{tong2pai2}{11,12}{⾦,⽚}[HSK 6]
  \definition[枚]{s.}{medalha de bronze; o bronze | placa de bronze com nome ou logotipo comercial, etc.}
\end{EntryWithPhonetic}

%%%%%%%%%% 童 %%%%%%%%%%
\subsection*{童}\addcontentsline{loh}{figure}{童 \dpy{tong2}}

\begin{EntryWithPhonetic}{童}{tong2}{12}{⽴}
  \definition*{s.}{Sobrenome: Tong}
  \definition{adj.}{virgem; solteira | nu; careca | árido; estéril}
  \definition{s.}{criança | jovem servo; antigamente, referia"-se a um servo menor de idade}
\end{EntryWithPhonetic}

\begin{EntryWithPhonetic}{童话}{tong2hua4}{12,8}{⽴,⾔}[HSK 4]
  \definition[个,部]{s.}{conto de fadas; gênero de literatura infantil no qual as histórias adequadas para a diversão das crianças são escritas com muita imaginação, fantasia e exagero}
\end{EntryWithPhonetic}

\begin{EntryWithPhonetic}{童年}{tong2nian2}{12,6}{⽴,⼲}[HSK 4]
  \definition[对]{s.}{infância; primeiros anos de vida}
\end{EntryWithPhonetic}

%%%%%%%%%% 僮 %%%%%%%%%%
\subsection*{僮}\addcontentsline{loh}{figure}{僮 \dpy{tong2}}

\begin{EntryWithPhonetic}{僮}{tong2}{14}{⼈}
  \definition*{s.}{Sobrenome: Tong}
  \seeref{zhuang4}
\end{EntryWithPhonetic}

%%%%%%%%%% 獞 %%%%%%%%%%
\subsection*{獞}\addcontentsline{loh}{figure}{獞 \dpy{tong2}}

\begin{EntryWithPhonetic}{獞}{tong2}{15}{⽝}
  \definition{s.}{nome de uma variedade de cão | tribos selvagens no sul da China}
  \seeref{zhuang4}
\end{EntryWithPhonetic}

%%%%%%%%%% 统 %%%%%%%%%%
\subsection*{统}\addcontentsline{loh}{figure}{统 \dpy{tong3}}

\begin{EntryWithPhonetic}{统}{tong3}{9}{⽷}
  \definition{adv.}{todos; juntos; de forma unificada | inteiramente; totalmente}
  \definition{s.}{interligado; inter-relacionado | sistema interconectado | qualquer parte em forma de tubo de uma peça de roupa, etc.; o mesmo que 筒}
  \definition{v.}{reunir em um; unir | unir; liderar; comandar}
  \seealsoref{筒}{tong3}
\end{EntryWithPhonetic}

\begin{EntryWithPhonetic}{统筹}{tong3chou2}{9,13}{⽷,⽵}[HSK 7-9]
  \definition{v.}{planejar como um todo; elaborar planos gerais}
  \synonymref{兼顾}{jian1gu4}
\end{EntryWithPhonetic}

\begin{EntryWithPhonetic}{统计}{tong3ji4}{9,4}{⽷,⾔}[HSK 4]
  \definition{v.}{compilar estatísticas; refere"-se à realização de trabalho estatístico, ou seja, coletar, reunir, analisar e extrapolar dados sobre um fenômeno | somar; adicionar; contar}
\end{EntryWithPhonetic}

\begin{EntryWithPhonetic}{统统}{tong3tong3}{9,9}{⽷,⽷}[HSK 7-9]
  \definition{adv.}{todos; completamente; inteiramente}
  \seealsoref{一律}{yi2lv4}
  \synonymref{绝对}{jue2dui4}
  \synonymref{全部}{quan2bu4}
  \synonymref{十足}{shi2zu2}
  \synonymref{所有}{suo3you3}
  \synonymref{完全}{wan2quan2}
  \synonymref{完整}{wan2zheng3}
  \synonymref{一共}{yi2gong4}
  \synonymref{一切}{yi2qie4}
  \synonymref{整个}{zheng3ge4}
  \synonymref{总共}{zong3gong4}
  \antonymref{仅仅}{jin3jin3}
  \antonymref{少数}{shao3shu4}
\end{EntryWithPhonetic}

\begin{EntryWithPhonetic}{统一}{tong3yi1}{9,1}{⽷,⼀}[HSK 4]
  \definition{adj.}{unificado; unitário; centralizado; consistente}
  \definition{v.}{unificar; unir; integrar; padronizar}
\end{EntryWithPhonetic}

\begin{EntryWithPhonetic}{统治}{tong3zhi4}{9,8}{⽷,⽔}[HSK 7-9]
  \definition{v.}{governar; dominar; controlar e governar um país ou região através do poder político}
  \synonymref{办理}{ban4li3}
  \synonymref{处理}{chu3li3}
  \synonymref{管理}{guan3li3}
  \synonymref{管辖}{guan3xia2}
  \synonymref{争霸}{zheng1ba4}
  \synonymref{治理}{zhi4li3}
\end{EntryWithPhonetic}

%%%%%%%%%% 捅 %%%%%%%%%%
\subsection*{捅}\addcontentsline{loh}{figure}{捅 \dpy{tong3}}

\begin{EntryWithPhonetic}{捅}{tong3}{10}{⼿}[HSK 7-9]
  \definition{v.}{esfaquear; dar uma estocada; cutucar | cutucar; esbarrar; bater | expor; revelar}
\end{EntryWithPhonetic}

%%%%%%%%%% 桶 %%%%%%%%%%
\subsection*{桶}\addcontentsline{loh}{figure}{桶 \dpy{tong3}}

\begin{EntryWithPhonetic}{桶}{tong3}{11}{⽊}[HSK 7-9]
  \definition{clas.}{barril; uma unidade de capacidade}
  \definition[只,个]{s.}{barril; tina; tonel; balde; vaso sanitário}
\end{EntryWithPhonetic}

%%%%%%%%%% 筒 %%%%%%%%%%
\subsection*{筒}\addcontentsline{loh}{figure}{筒 \dpy{tong3}}

\begin{EntryWithPhonetic}{筒}{tong3}{12}{⽵}[HSK 7-9]
  \definition[个]{s.}{seção de bambu grosso; tubo grosso de bambu | objeto em forma de tubo largo | a parte em forma de tubo das roupas etc.}
  \definition{v.}{colocar dentro de (um objeto cilíndrico)}
\end{EntryWithPhonetic}

%%%%%%%%%% 同 %%%%%%%%%%
\subsection*{同}\addcontentsline{loh}{figure}{同 \dpy{tong4}}

\begin{EntryWithPhonetic}{同}{tong4}{6}{⼝}
  \definition[条,处]{s.}{beco; rua estreita}
  \seeref{tong2}
  \seealsoref{胡同}{hu2tong5}
\end{EntryWithPhonetic}

%%%%%%%%%% 通 %%%%%%%%%%
\subsection*{通}\addcontentsline{loh}{figure}{通 \dpy{tong4}}

\begin{EntryWithPhonetic}{通}{tong4}{10}{⾡}
  \definition{clas.}{usado para uma atividade, tomada em sua totalidade (discurso de abuso, período de reprodução de música, bebedeira, etc.)}
  \seeref{tong1}
\end{EntryWithPhonetic}

%%%%%%%%%% 痛 %%%%%%%%%%
\subsection*{痛}\addcontentsline{loh}{figure}{痛 \dpy{tong4}}

\begin{EntryWithPhonetic}{痛}{tong4}{12}{⽧}[HSK 3,7-9]
  \definition{adv.}{extremamente; profundamente; amargamente}
  \definition{s.}{dor; sofrimento | tristeza; pesar}
  \definition{v.}{sentir dor; ter dor | sentir tristeza; sentir aflição}
\end{EntryWithPhonetic}

\begin{EntryWithPhonetic}{痛苦}{tong4ku3}{12,8}{⽧,⾋}[HSK 3]
  \definition{adj.}{doloroso; angustiado; sentindo"-se muito desconfortável física ou mentalmente}
  \definition[降,种]{s.}{dor; agonia; sofrimento; refere"-se a um estado ou sentimento de extremo desconforto físico ou mental}
\end{EntryWithPhonetic}

\begin{EntryWithPhonetic}{痛快}{tong4kuai5}{12,7}{⽧,⼼}[HSK 4]
  \definition{adj.}{encantado; alegre; muito feliz; confortável | franco; direto; simples e direto}
\end{EntryWithPhonetic}

\begin{EntryWithPhonetic}{痛骂}{tong4ma4}{12,9}{⽧,⾺}
  \definition{v.}{repreender severamente}
\end{EntryWithPhonetic}

\begin{EntryWithPhonetic}{痛心}{tong4xin1}{12,4}{⽧,⼼}[HSK 7-9]
  \definition{adj.}{de partir o coração; dolorido; angustiado; aflito; enlutado; extrema tristeza}
\end{EntryWithPhonetic}

%%%%%%%%%% 偷 %%%%%%%%%%
\subsection*{偷}\addcontentsline{loh}{figure}{偷 \dpy{tou1}}

\begin{EntryWithPhonetic}{偷}{tou1}{11}{⼈}[HSK 5]
  \definition{adv.}{furtivamente; secretamente; às escondidas}
  \definition{s.}{ladrão; furtador}
  \definition{v.}{roubar; furtar; levar sem pagar; roubar os bens alheios às escondidas | encontrar (tempo) | deixar-se levar; viver apenas para o presente, sem se preocupar com o futuro}
\end{EntryWithPhonetic}

\begin{EntryWithPhonetic}{偷安}{tou1'an1}{11,6}{⼈,⼧}
  \definition{v.}{buscar facilidade temporária}
\end{EntryWithPhonetic}

\begin{EntryWithPhonetic}{偷渡}{tou1du4}{11,12}{⼈,⽔}
  \definition{s.}{contrabando | imigração ilegal | clandestino (em um navio)}
  \definition{v.}{executar um bloqueio | roubar através da fronteira internacional}
\end{EntryWithPhonetic}

\begin{EntryWithPhonetic}{偷看}{tou1kan4}{11,9}{⼈,⽬}[HSK 7-9]
  \definition{v.}{dar uma olhada rápida; espiar; dar uma olhadinha | espiar; espreitar}
\end{EntryWithPhonetic}

\begin{EntryWithPhonetic}{偷窥}{tou1kui1}{11,13}{⼈,⽳}[HSK 7-9]
  \definition{v.}{espiar; espionar; bisbilhotar; observar secretamente (especialmente para prazer sexual)}
\end{EntryWithPhonetic}

\begin{EntryWithPhonetic}{偷懒}{tou1/lan3}{11,16}{⼈,⼼}[HSK 7-9]
  \definition{v.+compl.}{ser preguiçoso; vadiar no trabalho; buscar conforto e conveniência, esquivar"-se das responsabilidades}
  \synonymref{怠慢}{dai4man4}
  \synonymref{懒惰}{lan3duo4}
  \synonymref{懒散}{lan3san3}
  \antonymref{练习}{lian4xi2}
\end{EntryWithPhonetic}

\begin{EntryWithPhonetic}{偷窃}{tou1qie4}{11,9}{⼈,⽳}
  \definition{v.}{furtar | roubar}
\end{EntryWithPhonetic}

\begin{EntryWithPhonetic}{偷情}{tou1qing2}{11,11}{⼈,⼼}
  \definition{v.}{manter um caso amoroso clandestino; anteriormente usado para se referir a ter um relacionamento romântico secreto, agora geralmente se refere a ter um relacionamento impróprio entre um homem e uma mulher}
\end{EntryWithPhonetic}

\begin{EntryWithPhonetic}{偷税}{tou1shui4}{11,12}{⼈,⽲}
  \definition{s.}{evasão fiscal}
\end{EntryWithPhonetic}

\begin{EntryWithPhonetic}{偷听}{tou1ting1}{11,7}{⼈,⼝}
  \definition{v.}{bisbilhotar; monitorar (secretamente)}
\end{EntryWithPhonetic}

\begin{EntryWithPhonetic}{偷偷}{tou1tou1}{11,11}{⼈,⼈}[HSK 5]
  \definition{adv.}{secretamente; dissimuladamente; furtivamente; às escondidas; descreve uma ação que não é notada pelos outros; em segredo ou em privado, não revelada}
\end{EntryWithPhonetic}

\begin{EntryWithPhonetic}{偷袭}{tou1xi2}{11,11}{⼈,⾐}
  \definition{s.}{ataque surpresa}
  \definition{v.}{montar um ataque furtivo | invadir}
\end{EntryWithPhonetic}

%%%%%%%%%% 偸 %%%%%%%%%%
\subsection*{偸}\addcontentsline{loh}{figure}{偸 \dpy{tou1}}

\begin{EntryWithPhonetic}{偸}{tou1}{11}{⼈}
  \variantof{偷}
\end{EntryWithPhonetic}

%%%%%%%%%% 头 %%%%%%%%%%
\subsection*{头}\addcontentsline{loh}{figure}{头 \dpy{tou2}}

\begin{EntryWithPhonetic}{头}{tou2}{5}{⼤}[HSK 2,3]
  \definition{adj.}{(antes de um numeral) primeiro | (antes de 年 ou 天) último; anterior}
  \definition{clas.}{usado para suínos ou gado (animais de criação) | usado para cabeças de alho ou coisas com formato de cabeça}
  \definition{num.}{primeiro}
  \definition{prep.}{antes de; perto de; introduz o tempo de uma ação, equivalente a  在……之前 ou 临近 | (entre dois algarismos, indicando um número aproximado) cerca de}
  \definition[个,颗]{s.}{cabeça; a parte do corpo humano ou animal que possui órgãos como boca, nariz, olhos e ouvidos | cabelo ou penteado | topo; fim; a parte superior ou final de um objeto | começo ou fim; o ponto inicial ou final de algo | fim; remanescente; os restos de algo | cabeça; chefe; líder | lado; aspecto}
  \seeref{tou5}
  \seealsoref{临近}{lin2jin4}
  \seealsoref{年}{nian2}
  \seealsoref{天}{tian1}
  \seealsoref{在}{zai4}
  \seealsoref{之前}{zhi1qian2}
\end{EntryWithPhonetic}

\begin{EntryWithPhonetic}{头部}{tou2bu4}{5,10}{⼤,⾢}[HSK 7-9]
  \definition{s.}{cabeça; cefalossoma}
\end{EntryWithPhonetic}

\begin{EntryWithPhonetic}{头顶}{tou2ding3}{5,8}{⼤,⾴}[HSK 7-9]
  \definition{s.}{topo da cabeça; cocuruto}
\end{EntryWithPhonetic}

\begin{EntryWithPhonetic}{头发}{tou2fa5}{5,5}{⼤,⼜}[HSK 2]
  \definition[根,缕,头]{s.}{cabelo}
\end{EntryWithPhonetic}

\begin{EntryWithPhonetic}{头号}{tou2hao4}{5,5}{⼤,⼝}[HSK 7-9]
  \definition{adj.}{\emph{top rank}; número um; tamanho um | de primeira classe; de ​​qualidade superior; o melhor}
\end{EntryWithPhonetic}

\begin{EntryWithPhonetic}{头脑}{tou2nao3}{5,10}{⼤,⾁}[HSK 3]
  \definition{s.}{inteligência; mente | pista; tópicos principais | chefe; líder; capitão}
\end{EntryWithPhonetic}

\begin{EntryWithPhonetic}{头脑风暴}{tou2nao3feng1bao4}{5,10,4,15}{⼤,⾁,⾵,⽇}
  \definition{s.}{\emph{brainstorm}}
\end{EntryWithPhonetic}

\begin{EntryWithPhonetic}{头疼}{tou2teng2}{5,10}{⼤,⽧}[HSK 6]
  \definition{s.}{dor de cabeça}
  \definition{v.}{estar preocupado ou incomodado por alguém ou algo}
\end{EntryWithPhonetic}

\begin{EntryWithPhonetic}{头条}{tou2tiao2}{5,7}{⼤,⽊}[HSK 7-9]
  \definition{s.}{manchete; notícia principal; notícia importante}
\end{EntryWithPhonetic}

\begin{EntryWithPhonetic}{头头}{tou2tou2}{5,5}{⼤,⼤}
  \definition{s.}{chefe | o cabeça}
\end{EntryWithPhonetic}

\begin{EntryWithPhonetic}{头头是道}{tou2tou2shi4dao4}{5,5,9,12}{⼤,⼤,⽇,⾡}[HSK 7-9]
  \definition{expr.}{bem fundamentado e argumentado; parecer impressionante; claro e lógico; fazer algo para a perfeita satisfação de alguém; (soar) plausível; (parecer) bem fundamentado e racional; com razão; sistemático e ordenado; coerente e convincente}
  \antonymref{乱七八糟}{luan4qi1ba1zao1}
\end{EntryWithPhonetic}

\begin{EntryWithPhonetic}{头衔}{tou2xian2}{5,11}{⼤,⾏}[HSK 7-9]
  \definition{s.}{título (um sinal de posição, profissão, etc.); refere"-se a títulos como títulos oficiais, títulos acadêmicos e títulos profissionais}
  \synonymref{称号}{cheng1hao4}
\end{EntryWithPhonetic}

\begin{EntryWithPhonetic}{头像}{tou2xiang4}{5,13}{⼤,⼈}
  \definition{s.}{retrato | busto | avatar | imagem de perfil (computação)}
\end{EntryWithPhonetic}

\begin{EntryWithPhonetic}{头晕}{tou2yun1}{5,10}{⼤,⽇}[HSK 7-9]
  \definition{s.}{fraco; tonto; vertiginoso (em grandes alturas); sentindo tontura}
  \synonymref{头疼}{tou2teng2}
  \antonymref{清醒}{qing1xing3}
\end{EntryWithPhonetic}

%%%%%%%%%% 投 %%%%%%%%%%
\subsection*{投}\addcontentsline{loh}{figure}{投 \dpy{tou2}}

\begin{EntryWithPhonetic}{投}{tou2}{7}{⼿}[HSK 4]
  \definition*{s.}{Sobrenome: Tou}
  \definition{pron.}{para; indica tempo, equivalente a 到, 临 | para; em direção a; indica orientação, direção, equivalente a 朝 ou 向}
  \definition{s.}{um jogo durante uma festa em que o vencedor era decidido pelo número de flechas lançadas em um pote distante | jogo de dados}
  \definition{v.}{lançar; arremessar; atirar | deixar cair; colocar em; lançar | mergulhar em; lançar-se em; pular dentro | lançar; projetar; sombrear | entregar; postar; enviar | ir até; ir para; buscar; juntar-se | sentir-se atraído por; adaptar-se a; concordar com; atender a}
  \seealsoref{朝}{chao2}
  \seealsoref{到}{dao4}
  \seealsoref{临}{lin2}
  \seealsoref{向}{xiang4}
\end{EntryWithPhonetic}

\begin{EntryWithPhonetic}{投奔}{tou2ben4}{7,8}{⼿,⼤}[HSK 7-9]
  \definition{v.}{ir para (um amigo ou um lugar) em busca de abrigo; recorrer a (alguém ou organização) e confiar em}
\end{EntryWithPhonetic}

\begin{EntryWithPhonetic}{投递}{tou2di4}{7,10}{⼿,⾡}
  \definition{v.}{despachar | enviar}
\end{EntryWithPhonetic}

\begin{EntryWithPhonetic}{投稿}{tou2/gao3}{7,15}{⼿,⽲}[HSK 7-9]
  \definition{v.+compl.}{submeter um texto para publicação; contribuir (para um jornal, revista, etc.); envar o manuscrito para redações de jornais, editoras, etc., solicitando a publicação}
\end{EntryWithPhonetic}

\begin{EntryWithPhonetic}{投机}{tou2ji1}{7,6}{⼿,⽊}[HSK 7-9]
  \definition{adj.}{afável; agradável; opiniões compartilhadas}
  \definition{v.}{especular; ser oportunista; aproveitar uma oportunidade para obter ganho pessoal}
  \synonymref{炒股}{chao3/gu3}
  \antonymref{本分}{ben3fen4}
\end{EntryWithPhonetic}

\begin{EntryWithPhonetic}{投票}{tou2/piao4}{7,11}{⼿,⽰}[HSK 6]
  \definition{v.+compl.}{votar; dar um voto; um método de eleição no qual os eleitores escrevem o nome da pessoa que querem eleger na cédula, ou marcam a cédula com o nome do candidato impresso e depois a colocam na urna para votar na resolução}
\end{EntryWithPhonetic}

\begin{EntryWithPhonetic}{投入}{tou2ru4}{7,2}{⼿,⼊}[HSK 4]
  \definition{adj.}{sisudo; dedicado; devotado; absorto}
  \definition{s.}{investimento; insumo; refere"-se à aplicação de recursos}
  \definition{v.}{lançar em; colocar em; jogar em; por em | entrar em uma situação; participar de | aplicar; investir; colocar fundos em}
\end{EntryWithPhonetic}

\begin{EntryWithPhonetic}{投射}{tou2she4}{7,10}{⼿,⼨}[HSK 7-9]
  \definition{v.}{lançar; arremessar; atirar | lançar; projetar (um raio de luz, etc.)}
\end{EntryWithPhonetic}

\begin{EntryWithPhonetic}{投身}{tou2shen1}{7,7}{⼿,⾝}[HSK 7-9]
  \definition{v.}{dedicar"-se; entregar"-se de corpo e alma a algo}
  \synonymref{奔赴}{ben1fu4}
  \synonymref{投入}{tou2ru4}
\end{EntryWithPhonetic}

\begin{EntryWithPhonetic}{投诉}{tou2su4}{7,7}{⼿,⾔}[HSK 4]
  \definition{v.}{reclamar; queixar-se; reclamar às autoridades ou pessoas envolvidas}
\end{EntryWithPhonetic}

\begin{EntryWithPhonetic}{投降}{tou2xiang2}{7,8}{⼿,⾩}[HSK 7-9]
  \definition{v.}{render"-se; capitular; cessar a resistência; render"-se ao outro lado}
  \synonymref{背叛}{bei4pan4}
  \synonymref{俘虏}{fu2lu3}
  \synonymref{屈服}{qu1fu2}
  \synonymref{顺从}{shun4cong2}
  \synonymref{征服}{zheng1fu2}
  \antonymref{抵抗}{di3kang4}
  \antonymref{抵御}{di3yu4}
  \antonymref{对抗}{dui4kang4}
  \antonymref{反抗}{fan3kang4}
  \antonymref{抗拒}{kang4ju4}
  \antonymref{侵略}{qin1lve4}
\end{EntryWithPhonetic}

\begin{EntryWithPhonetic}{投资}{tou2zi1}{7,10}{⼿,⾙}[HSK 4]
  \definition[笔]{s.}{investimento}
  \definition{v.}{investir; aplicar dinheiro; investir dinheiro em negócios}
\end{EntryWithPhonetic}

\begin{EntryWithPhonetic}{投资风险}{tou2zi1 feng1xian3}{7,10,4,9}{⼿,⾙,⾵,⾩}
  \definition*{s.}{risco de investimento}
\end{EntryWithPhonetic}

\begin{EntryWithPhonetic}{投资回报率}{tou2zi1 hui2bao4 lv4}{7,10,6,7,11}{⼿,⾙,⼞,⼿,⽞}
  \definition{s.}{retorno sobre o investimento (ROI)}
\end{EntryWithPhonetic}

\begin{EntryWithPhonetic}{投资家}{tou2zi1jia1}{7,10,10}{⼿,⾙,⼧}
  \definition{s.}{investidor}
  \seealsoref{投资人}{tou2zi1ren2}
  \seealsoref{投资者}{tou2zi1zhe3}
\end{EntryWithPhonetic}

\begin{EntryWithPhonetic}{投资人}{tou2zi1ren2}{7,10,2}{⼿,⾙,⼈}
  \definition{s.}{investidor}
  \seealsoref{投资家}{tou2zi1jia1}
  \seealsoref{投资者}{tou2zi1zhe3}
\end{EntryWithPhonetic}

\begin{EntryWithPhonetic}{投资者}{tou2zi1zhe3}{7,10,8}{⼿,⾙,⽼}
  \definition{s.}{investidor}
  \seealsoref{投资家}{tou2zi1jia1}
  \seealsoref{投资人}{tou2zi1ren2}
\end{EntryWithPhonetic}

%%%%%%%%%% 透 %%%%%%%%%%
\subsection*{透}\addcontentsline{loh}{figure}{透 \dpy{tou4}}

\begin{EntryWithPhonetic}{透}{tou4}{10}{⾡}[HSK 4]
  \definition{adv.}{totalmente; completamente; minuciosamente | profundamente; extremamente}
  \definition{v.}{penetrar; passar através de; infiltrar-se através de | revelar; deixar transparecer; contar secretamente |mostrar; aparecer}
\end{EntryWithPhonetic}

\begin{EntryWithPhonetic}{透彻}{tou4che4}{10,7}{⾡,⼻}[HSK 7-9]
  \definition{adj.}{penetrante; minucioso; incisivo; (compreensão da situação e análise das razões) detalhado e aprofundado}
  \synonymref{彻底}{che4di3}
  \synonymref{透顶}{tou4ding3}
  \synonymref{透辟}{tou4pi4}
  \antonymref{朦胧}{meng2long2}
\end{EntryWithPhonetic}

\begin{EntryWithPhonetic}{透澈}{tou4che4}{10,15}{⾡,⽔}
  \variantof{透彻}
\end{EntryWithPhonetic}

\begin{EntryWithPhonetic}{透顶}{tou4ding3}{10,8}{⾡,⾴}
  \definition{adv.}{completamente}
\end{EntryWithPhonetic}

\begin{EntryWithPhonetic}{透过}{tou4guo4}{10,6}{⾡,⾡}[HSK 7-9]
  \definition{v.}{passar por (algo ou espaço) | penetrar; infiltrar}
  \synonymref{穿过}{chuan1guo4}
\end{EntryWithPhonetic}

\begin{EntryWithPhonetic}{透亮}{tou4liang4}{10,9}{⾡,⼇}
  \definition{adj.}{brilhante | claro como cristal}
\end{EntryWithPhonetic}

\begin{EntryWithPhonetic}{透露}{tou4lu4}{10,21}{⾡,⾬}[HSK 6]
  \definition{v.}{vazar; revelar; expor; divulgar; contar deliberadamente um segredo a alguém; revelar um certo significado}
\end{EntryWithPhonetic}

\begin{EntryWithPhonetic}{透明}{tou4ming2}{10,8}{⾡,⽇}[HSK 4]
  \definition{adj.}{transparente; diáfano; capaz de transmitir luz | evidente; transparente; situação ou assunto que seja aberto e não oculto | transparente; diáfano; indica pureza, ausência de impurezas}
\end{EntryWithPhonetic}

\begin{EntryWithPhonetic}{透辟}{tou4pi4}{10,13}{⾡,⾟}
  \definition{adj.}{incisivo | penetrante}
\end{EntryWithPhonetic}

\begin{EntryWithPhonetic}{透气}{tou4/qi4}{10,4}{⾡,⽓}[HSK 7-9]
  \definition{v.+compl.}{ventilar; poder passar o ar | respirar livremente; respirar ar puro | vazar (ou revelar) informações; dar uma dica; alertar}
  \synonymref{通风}{tong1/feng1}
\end{EntryWithPhonetic}

\begin{EntryWithPhonetic}{透水}{tou4shui3}{10,4}{⾡,⽔}
  \definition{adj.}{permeável}
  \definition{s.}{vazamento de água}
\end{EntryWithPhonetic}

\begin{EntryWithPhonetic}{透支}{tou4zhi1}{10,4}{⾡,⽀}[HSK 7-9]
  \definition{v.}{fazer um descoberto bancário | exceder as despesas sobre as receitas; gastar em excesso | receber o salário antecipadamente}
  \synonymref{过度}{guo4du4}
  \antonymref{充沛}{chong1pei4}
\end{EntryWithPhonetic}

%%%%%%%%%% 头 %%%%%%%%%%
\subsection*{头}\addcontentsline{loh}{figure}{头 \dpy{tou5}}

\begin{EntryWithPhonetic}{头}{tou5}{5}{⼤}
  \definition{suf.}{adicionado após componentes nominais comuns | adicionado após o componente verbal, forma um substantivo abstrato, geralmente indicando que vale a pena realizar essa ação | adicionado após um componente adjetival, forma um substantivo, geralmente indicando algo abstrato | adicionado após o componente substantivo que indica a direção}
  \seeref{tou2}
\end{EntryWithPhonetic}

%%%%%%%%%% 凸 %%%%%%%%%%
\subsection*{凸}\addcontentsline{loh}{figure}{凸 \dpy{tu1}}

\begin{EntryWithPhonetic}{凸}{tu1}{5}{⼐}[HSK 7-9]
  \definition{adj.}{saliente; elevado | levantado; mais alto que o entorno}
  \definition{s.}{protuberância; saliência}
  \antonymref{凹}{ao1}
  \antonymref{平}{ping2}
\end{EntryWithPhonetic}

\begin{EntryWithPhonetic}{凸显}{tu1xian3}{5,9}{⼐,⽇}[HSK 7-9]
  \definition{v.}{apresentar com clareza; dar destaque a; enfatizar; revelar com clareza}
  \synonymref{显出}{xian3 chu1}
  \antonymref{隐藏}{yin3cang2}
\end{EntryWithPhonetic}

%%%%%%%%%% 秃 %%%%%%%%%%
\subsection*{秃}\addcontentsline{loh}{figure}{秃 \dpy{tu1}}

\begin{EntryWithPhonetic}{秃}{tu1}{7}{⽲}[HSK 7-9]
  \definition{adj.}{careca | rombudo; sem ponta | Coloquial: incompleto; insatisfatório | estéril; sem galhos ou folhas; sem árvores}
  \antonymref{尖}{jian1}
\end{EntryWithPhonetic}

%%%%%%%%%% 突 %%%%%%%%%%
\subsection*{突}\addcontentsline{loh}{figure}{突 \dpy{tu1}}

\begin{EntryWithPhonetic}{突}{tu1}{9}{⽳}
  \definition{adv.}{de repente; abruptamente; inesperadamente}
  \definition{s.}{chaminé}
  \definition{v.}{avançar rapidamente; atacar | projetar; destacar-se | romper | projetar-se; inchar; fazer bojo}
\end{EntryWithPhonetic}

\begin{EntryWithPhonetic}{突出}{tu1/chu1}{9,5}{⽳,⼐}[HSK 3]
  \definition{adj.}{proeminente; excelente; mais que a média}
  \definition{v.+compl.}{romper | enfatizar; destacar; dar destaque a | sobressair; projetar-se; destacar-se}
\end{EntryWithPhonetic}

\begin{EntryWithPhonetic}{突发}{tu1fa1}{9,5}{⽳,⼜}[HSK 7-9]
  \definition{v.}{surgir de repente; aparecer inesperadamente | explodir repentinamente}
\end{EntryWithPhonetic}

\begin{EntryWithPhonetic}{突击}{tu1ji1}{9,5}{⽳,⼐}[HSK 7-9]
  \definition[方]{s.}{ataque repentino e violento; agressão | Figurativo: trabalho apressado; esforço concentrado para terminar um trabalho rapidamente}
  \definition{v.}{fazer um ataque repentino e violento; agredir | fazer um esforço concentrado para terminar um trabalho rapidamente; fazer um trabalho às pressas}
\end{EntryWithPhonetic}

\begin{EntryWithPhonetic}{突破}{tu1/po4}{9,10}{⽳,⽯}[HSK 5]
  \definition{v.+compl.}{romper; fazer uma descoberta revolucionária; concentrar esforços em um único ponto de ataque, reunir o sucesso | quebrar (limite); superar (dificuldade); superar dificuldades; ultrapassar os números ou limites anteriores, superar recordes anteriores, etc.; romper com as limitações e restrições anteriores}
\end{EntryWithPhonetic}

\begin{EntryWithPhonetic}{突破口}{tu1po4kou3}{9,10,3}{⽳,⽯,⼝}[HSK 7-9]
  \definition{s.}{ponto de virada; avanço; solução inovadora | brecha; lacuna}
\end{EntryWithPhonetic}

\begin{EntryWithPhonetic}{突然}{tu1ran2}{9,12}{⽳,⽕}[HSK 3]
  \definition{adj.}{repentino; abrupto; inesperado}
  \definition{adv.}{de repente; abruptamente; inesperadamente; subitamente}
\end{EntryWithPhonetic}

\begin{EntryWithPhonetic}{突如其来}{tu1ru2-qi2lai2}{9,6,8,7}{⽳,⼥,⼋,⽊}[HSK 7-9]
  \definition{expr.}{surgir repentinamente; aparecer de repente; surgir do nada; aconteceu de forma inesperada e repentina}
\end{EntryWithPhonetic}

%%%%%%%%%% 图 %%%%%%%%%%
\subsection*{图}\addcontentsline{loh}{figure}{图 \dpy{tu2}}

\begin{EntryWithPhonetic}{图}{tu2}{8}{⼞}[HSK 3]
  \definition*{s.}{Sobrenome: Tu}
  \definition[张]{s.}{mapa; gráfico; imagem; desenho | plano; esquema; tentativa}
  \definition{v.}{procurar; perseguir; esperar obter| desenhar; retratar; pintar | imaginar; planejar; pensar; maquinar}
\end{EntryWithPhonetic}

\begin{EntryWithPhonetic}{图案}{tu2'an4}{8,10}{⼞,⽊}[HSK 4]
  \definition[种,个]{s.}{padrão; desenho; padrões e gráficos usados para decoração de edifícios, tecidos, artes e artesanato, etc.}
\end{EntryWithPhonetic}

\begin{EntryWithPhonetic}{图标}{tu2biao1}{8,9}{⼞,⽊}
  \definition{s.}{ícone (informática)}[这个图标代表设置。===Este ícone representa as configurações.]
\end{EntryWithPhonetic}

\begin{EntryWithPhonetic}{图表}{tu2biao3}{8,8}{⼞,⾐}[HSK 7-9]
  \definition[张,个]{s.}{carta; diagrama; gráfico | figura; gráfico; diagrama; pictograma; pictografia; cronograma; folha; tabela; termo geral para diagramas e tabelas que representam diversas situações e indicam vários números, como diagramas esquemáticos e tabelas estatísticas}
\end{EntryWithPhonetic}

\begin{EntryWithPhonetic}{图画}{tu2hua4}{8,8}{⼞,⽥}[HSK 3]
  \definition[幅,张,套]{s.}{desenho; imagem; pintura}
\end{EntryWithPhonetic}

\begin{EntryWithPhonetic}{图片}{tu2pian4}{8,4}{⼞,⽚}[HSK 2]
  \definition[张,幅]{s.}{imagem; fotografia; um termo geral para imagens, fotografias, decalques, etc. usados para ilustrar algo}
\end{EntryWithPhonetic}

\begin{EntryWithPhonetic}{图书}{tu2shu1}{8,4}{⼞,⼄}[HSK 6]
  \definition{s.}{livros; um termo geral para publicações como livros e álbuns de imagens}[这些图书都可以借阅。===Esses livros estão disponíveis para empréstimo.]
\end{EntryWithPhonetic}

\begin{EntryWithPhonetic}{图书馆}{tu2shu1guan3}{8,4,11}{⼞,⼄,⾷}[HSK 1]
  \definition[个,家]{s.}{biblioteca; instituição que coleta, organiza e armazena livros e materiais para leitura e consulta}
\end{EntryWithPhonetic}

\begin{EntryWithPhonetic}{图像}{tu2xiang4}{8,13}{⼞,⼈}[HSK 7-9]
  \definition{s.}{imagem; figura; imagens desenhadas, fotografadas ou impressas}
\end{EntryWithPhonetic}

\begin{EntryWithPhonetic}{图形}{tu2xing2}{8,7}{⼞,⼺}[HSK 7-9]
  \definition[种,个]{s.}{figura; gráfico | sinal; carta; desenho; diagrama}
  \synonymref{图案}{tu2'an4}
\end{EntryWithPhonetic}

\begin{EntryWithPhonetic}{图纸}{tu2zhi3}{8,7}{⼞,⽷}[HSK 7-9]
  \definition[张]{s.}{desenho; planta; folha de desenho; \emph{blueprint}; um documento técnico que utiliza desenhos e texto para descrever a estrutura, forma, dimensões e outros requisitos de estruturas de engenharia, máquinas, equipamentos, etc.}
\end{EntryWithPhonetic}

%%%%%%%%%% 徒 %%%%%%%%%%
\subsection*{徒}\addcontentsline{loh}{figure}{徒 \dpy{tu2}}

\begin{EntryWithPhonetic}{徒}{tu2}{10}{⼻}
  \definition{adj.}{vazio; nu}
  \definition{adv.}{somente; meramente; apenas | a pé | em vão; sem sucesso; sem sucesso}
  \definition{s.}{aprendiz; aluno | seguidor; crente | (pejorativo) pessoas da mesma facção | (pejorativo) pessoa; companheiro | (prisão) pena; prisão; sentença | estudante}
  \definition{v.}{estar a pé | andar}
\end{EntryWithPhonetic}

\begin{EntryWithPhonetic}{徒步}{tu2bu4}{10,7}{⼻,⽌}[HSK 7-9]
  \definition{adv.}{a pé}
  \synonymref{步行}{bu4xing2}
  \synonymref{散步}{san4/bu4}
\end{EntryWithPhonetic}

\begin{EntryWithPhonetic}{徒弟}{tu2di5}{10,7}{⼻,⼸}[HSK 6]
  \definition[位,名,个]{s.}{discípulo; aprendiz; uma pessoa que aprende com um mestre; geralmente se refere a uma pessoa que aprende com um especialista}[他是我的徒弟。===Ele é meu aprendiz.]
\end{EntryWithPhonetic}

\begin{EntryWithPhonetic}{徒手}{tu2shou3}{10,4}{⼻,⼿}
  \definition{adj.}{com as mãos vazias | desarmado | mão livre (desenho) | lutando mão-a-mão}
\end{EntryWithPhonetic}

%%%%%%%%%% 涂 %%%%%%%%%%
\subsection*{涂}\addcontentsline{loh}{figure}{涂 \dpy{tu2}}

\begin{EntryWithPhonetic}{涂}{tu2}{10}{⽔}[HSK 7-9]
  \definition*{s.}{Sobrenome: Tu}
  \definition{s.}{Literário: lama; pântano | praia; praia marítima; planícies de maré | estrada; caminho}
  \definition{v.}{aplicar; espalhar; aplicar sobre; permitir que tinta, corantes, cosméticos, medicamentos, etc., adiram a um objeto | rabiscar; escrever à mão; rabiscar ou desenhar aleatoriamente; escrever ou desenhar o que se quiser | riscar; anular; apagar}
  \synonymref{擦}{ca1}
  \synonymref{抹}{mo3}
\end{EntryWithPhonetic}

%%%%%%%%%% 途 %%%%%%%%%%
\subsection*{途}\addcontentsline{loh}{figure}{途 \dpy{tu2}}

\begin{EntryWithPhonetic}{途}{tu2}{10}{⾡}
  \definition[条]{s.}{caminho; estrada; rota | jornada; caminho}
\end{EntryWithPhonetic}

\begin{EntryWithPhonetic}{途径}{tu2jing4}{10,8}{⾡,⼻}[HSK 6]
  \definition[种,条,个]{s.}{caminho; canal; metaforicamente falando, uma maneira ou método de resolver um problema ou fazer algo}
\end{EntryWithPhonetic}

\begin{EntryWithPhonetic}{途中}{tu2zhong1}{10,4}{⾡,⼁}[HSK 4]
  \definition[家]{adv.}{no caminho; ao longo do caminho}
\end{EntryWithPhonetic}

%%%%%%%%%% 屠 %%%%%%%%%%
\subsection*{屠}\addcontentsline{loh}{figure}{屠 \dpy{tu2}}

\begin{EntryWithPhonetic}{屠}{tu2}{11}{⼫}
  \definition*{s.}{Sobrenome: Tu}
  \definition[个,位,名,些]{v.}{abater (animais para alimentação) | massacre; carnificina}
\end{EntryWithPhonetic}

\begin{EntryWithPhonetic}{屠杀}{tu2sha1}{11,6}{⼫,⽊}[HSK 7-9]
  \definition{s.}{massacre; carnificina; matança; assassinatos em massa}
\end{EntryWithPhonetic}

%%%%%%%%%% 土 %%%%%%%%%%
\subsection*{土}\addcontentsline{loh}{figure}{土 \dpy{tu3}}

\begin{EntryWithPhonetic}{土}{tu3}{3}{⼟}[HSK 3,6][Kangxi 32]
  \definition*{s.}{Sobrenome: Tu}
  \definition{adj.}{local; nativo; local com características regionais| caseiro; indígena; o que é tradicional no país; popular | não refinado; não esclarecido; não está na moda; não é popular}
  \definition[堆,捧,层]{s.}{solo; terra | terra; território | ópio bruto | cidade natal; terra natal; pátria}
\end{EntryWithPhonetic}

\begin{EntryWithPhonetic}{土地}{tu3di4}{3,6}{⼟,⼟}[HSK 4]
  \definition[片,块,顷]{s.}{terra; solo; chão; superfície terrestre da Terra usada para cultivar, construir edifícios e viver | território; território de um país}
  \seeref{tu3di5}
\end{EntryWithPhonetic}

\begin{EntryWithPhonetic}{土地}{tu3di5}{3,6}{⼟,⼟}
  \definition[片,块,顷]{s.}{deus da audeia; deus local; \emph{genius loci} deidade protetora de um local; Superstição: refere"-se ao deus da terra que governa uma pequena área}
  \seeref{tu3di4}
\end{EntryWithPhonetic}

\begin{EntryWithPhonetic}{土豆}{tu3dou4}{3,7}{⼟,⾖}[HSK 5]
  \definition[颗,斤,个,棵]{s.}{batata; denominação comum da batata}
\end{EntryWithPhonetic}

\begin{EntryWithPhonetic}{土豆泥}{tu3dou4ni2}{3,7,8}{⼟,⾖,⽔}
  \definition{s.}{purê de batata}[她的土豆泥确实不错。===O purê de batatas dela estava realmente muito bom.]
\end{EntryWithPhonetic}

\begin{EntryWithPhonetic}{土匪}{tu3fei3}{3,10}{⼟,⼕}[HSK 7-9]
  \definition{s.}{bandido; salteador}
  \seealsoref{胡匪}{hu2fei3}
  \synonymref{强盗}{qiang2dao4}
\end{EntryWithPhonetic}

\begin{EntryWithPhonetic}{土鸡}{tu3ji1}{3,7}{⼟,⿃}
  \definition{s.}{galinha caipira}
\end{EntryWithPhonetic}

\begin{EntryWithPhonetic}{土壤}{tu3rang3}{3,20}{⼟,⼟}[HSK 7-9]
  \definition{s.}{solo; uma camada solta de material na superfície terrestre, composta por diversos minerais granulares, matéria orgânica, água, ar, microrganismos, etc., que permite o crescimento de plantas}
  \synonymref{泥土}{ni2tu3}
\end{EntryWithPhonetic}

\begin{EntryWithPhonetic}{土生土长}{tu3sheng1-tu3zhang3}{3,5,3,4}{⼟,⽣,⼟,⾧}[HSK 7-9]
  \definition{expr.}{1. nativo; nascido e criado
Cultivado localmente}
\end{EntryWithPhonetic}

%%%%%%%%%% 吐 %%%%%%%%%%
\subsection*{吐}\addcontentsline{loh}{figure}{吐 \dpy{tu3}}

\begin{EntryWithPhonetic}{吐}{tu3}{6}{⼝}[HSK 5]
  \definition{v.}{cuspir; sair pela boca | surgir ou aparecer pela boca ou por uma fenda | dizer; contar; falar abertamente}
  \seeref{tu4}
\end{EntryWithPhonetic}

\begin{EntryWithPhonetic}{吐}{tu4}{6}{⼝}[HSK 5]
  \definition{v.}{vomitar; sair pela boca | vomitar; expelir; metáfora para ser forçado a devolver bens usurpados}
  \seeref{tu3}
\end{EntryWithPhonetic}

%%%%%%%%%% 兔 %%%%%%%%%%
\subsection*{兔}\addcontentsline{loh}{figure}{兔 \dpy{tu4}}

\begin{EntryWithPhonetic}{兔}{tu4}{8}{⼉}[HSK 5]
  \definition[只]{s.}{lebre; coelho}
\end{EntryWithPhonetic}

\begin{EntryWithPhonetic}{兔子}{tu4zi5}{8,3}{⼉,⼦}
  \definition[只]{s.}{coelho | lebre}
\end{EntryWithPhonetic}

%%%%%%%%%% 团 %%%%%%%%%%
\subsection*{团}\addcontentsline{loh}{figure}{团 \dpy{tuan2}}

\begin{EntryWithPhonetic}{团}{tuan2}{6}{⼞}[HSK 3]
  \definition*{s.}{Liga da Juventude Comunista da China; Liga}
  \definition{adj.}{redondo; circular}
  \definition{clas.}{usado para algo em forma de bola}
  \definition[个]{s.}{bolinho de massa; comida em forma de bola feita de arroz ou farinha | algo em forma de bola | grupo; corpo; sociedade; organização; um grupo envolvido em um determinado trabalho ou atividade | regimento; unidade organizacional militar, geralmente abaixo do nível de divisão e acima do nível de batalhão}
  \definition{v.}{enrolar algo para formar uma bola; rolar | reunir; unir; conglomerar}
\end{EntryWithPhonetic}

\begin{EntryWithPhonetic}{团队}{tuan2dui4}{6,4}{⼞,⾩}[HSK 6]
  \definition[个,支,种]{s.}{equipe; time; grupo; um grupo de alguma natureza}
\end{EntryWithPhonetic}

\begin{EntryWithPhonetic}{团伙}{tuan2huo3}{6,6}{⼞,⼈}[HSK 7-9]
  \definition[个]{s.}{gangue; bando; panelinha | gangue (criminosa) | cúmplice; comparsa; membro de gangue}
\end{EntryWithPhonetic}

\begin{EntryWithPhonetic}{团结}{tuan2jie2}{6,9}{⼞,⽷}[HSK 3]
  \definition{adj.}{unido; amigável; harmonioso; relação harmoniosa e coexistência harmoniosa}
  \definition{v.}{unir; reunir}
\end{EntryWithPhonetic}

\begin{EntryWithPhonetic}{团聚}{tuan2ju4}{6,14}{⼞,⽿}[HSK 7-9]
  \definition{v.}{reunir; reencontrar familiares ou amigos muito próximos após a separação | unir; reunir; congregar}
  \synonymref{重逢}{chong2feng2}
  \synonymref{欢聚}{huan1ju4}
  \synonymref{聚会}{ju4hui4}
  \synonymref{团圆}{tuan2yuan2}
  \synonymref{团员}{tuan2yuan2}
  \antonymref{分开}{fen1/kai1}
  \antonymref{分别}{fen1bie2}
  \antonymref{分离}{fen1li2}
  \antonymref{分散}{fen1san4}
\end{EntryWithPhonetic}

\begin{EntryWithPhonetic}{团体}{tuan2ti3}{6,7}{⼞,⼈}[HSK 3]
  \definition[种,个]{s.}{equipe; grupo; organização; um grupo de pessoas com objetivos e interesses comuns}
\end{EntryWithPhonetic}

\begin{EntryWithPhonetic}{团员}{tuan2yuan2}{6,7}{⼞,⼝}[HSK 7-9]
  \definition[名,位,个]{s.}{membro; membro de um grupo específico | membro da Liga; membro da Liga da Juventude Comunista da China; refere"-se especificamente aos membros da Liga da Juventude Comunista da China}
  \synonymref{会员}{hui4yuan2}
  \synonymref{团聚}{tuan2ju4}
\end{EntryWithPhonetic}

\begin{EntryWithPhonetic}{团圆}{tuan2yuan2}{6,10}{⼞,⼞}[HSK 7-9]
  \definition{adj.}{redondo; indica que a forma é circular}
  \definition{v.}{reunir; reencontrar membros da família após um período de separação}
  \synonymref{团聚}{tuan2ju4}
  \antonymref{分离}{fen1li2}
\end{EntryWithPhonetic}

\begin{EntryWithPhonetic}{团长}{tuan2zhang3}{6,4}{⼞,⾧}[HSK 5]
  \definition[位,名]{s.}{comandante do regimento | chefe (ou presidente) de uma delegação, trupe, etc. | líder de uma delegação}
\end{EntryWithPhonetic}

%%%%%%%%%% 推 %%%%%%%%%%
\subsection*{推}\addcontentsline{loh}{figure}{推 \dpy{tui1}}

\begin{EntryWithPhonetic}{推}{tui1}{11}{⼿}[HSK 2]
  \definition{v.}{empurrar; dar um encontrão | girar um moinho ou uma pedra de amolar; moer | cortar; aparar | impulsionar; promover; avançar | inferir; deduzir | afastar; fugir; deslocar | adiar | eleger; escolher | ter em alta estima; elogiar muito | declinar | selecionar | elogiar muito}
\end{EntryWithPhonetic}

\begin{EntryWithPhonetic}{推测}{tui1ce4}{11,9}{⼿,⽔}[HSK 7-9]
  \definition{v.}{inferir; supor; conjecturar; especular; estimar ou imaginar o desconhecido com base no conhecido}
  \synonymref{猜测}{cai1ce4}
  \synonymref{猜想}{cai1xiang3}
  \synonymref{揣测}{chuai3ce4}
  \synonymref{揣摩}{chuai3mo2}
  \synonymref{估计}{gu1ji4}
  \synonymref{探求}{tan4qiu2}
  \synonymref{推断}{tui1duan4}
  \antonymref{断定}{duan4ding4}
\end{EntryWithPhonetic}

\begin{EntryWithPhonetic}{推迟}{tui1chi2}{11,7}{⼿,⾡}[HSK 4]
  \definition{v.}{adiar; postergar; tardar; deixar para mais tarde}
\end{EntryWithPhonetic}

\begin{EntryWithPhonetic}{推出}{tui1chu1}{11,5}{⼿,⼐}[HSK 6]
  \definition{v.}{lançar; apresentar; fazer com que apareça diante do público | deduzir; tirar conclusões da análise}
\end{EntryWithPhonetic}

\begin{EntryWithPhonetic}{推辞}{tui1ci2}{11,13}{⼿,⾟}[HSK 7-9]
  \definition{s.}{indicar recusa (de compromissos, convites, presentes, etc.); recusar (pedidos, opiniões ou presentes)}
  \synonymref{不要}{bu2yao4}
  \synonymref{拒绝}{ju4jue2}
  \synonymref{推卸}{tui1xie4}
  \synonymref{退却}{tui4que4}
  \antonymref{承诺}{cheng2nuo4}
  \antonymref{答应}{da1ying5}
  \antonymref{接纳}{jie1na4}
  \antonymref{接收}{jie1shou1}
  \antonymref{接受}{jie1shou4}
  \antonymref{提出}{ti2 chu1}
\end{EntryWithPhonetic}

\begin{EntryWithPhonetic}{推动}{tui1 dong4}{11,6}{⼿,⼒}[HSK 3]
  \definition{v.}{promover; atuar; impulsionar; empurrar para a frente; dar ímpeto a; começar ou avançar algo (com alguma força); começar a trabalhar}
\end{EntryWithPhonetic}

\begin{EntryWithPhonetic}{推断}{tui1duan4}{11,11}{⼿,⽄}[HSK 7-9]
  \definition{v.}{inferir; deduzir; especular e concluir}
  \synonymref{猜测}{cai1ce4}
  \synonymref{猜想}{cai1xiang3}
  \synonymref{揣测}{chuai3ce4}
  \synonymref{估计}{gu1ji4}
  \synonymref{判断}{pan4duan4}
  \synonymref{推测}{tui1ce4}
  \synonymref{推理}{tui1li3}
  \antonymref{断定}{duan4ding4}
\end{EntryWithPhonetic}

\begin{EntryWithPhonetic}{推翻}{tui1/fan1}{11,18}{⼿,⽻}[HSK 7-9]
  \definition{v.+compl.}{tombar; virar; derrubar; derrubar o regime original ou mudar o sistema social | cancelar; reverter; repudiar; negar completamente as declarações, conclusões, decisões, etc., existentes}
  \synonymref{撤销}{che4xiao1}
  \synonymref{摧毁}{cui1hui3}
  \synonymref{打倒}{da3/dao3}
  \synonymref{颠覆}{dian1fu4}
  \antonymref{创建}{chuang4jian4}
  \antonymref{创立}{chuang4li4}
  \antonymref{建立}{jian4li4}
  \antonymref{证明}{zheng4ming2}
\end{EntryWithPhonetic}

\begin{EntryWithPhonetic}{推广}{tui1guang3}{11,3}{⼿,⼴}[HSK 3]
  \definition{v.}{espalhar; estender; promover; popularizar; expandir o escopo de uso ou função de algo}
\end{EntryWithPhonetic}

\begin{EntryWithPhonetic}{推荐}{tui1jian4}{11,9}{⼿,⾋}[HSK 7-9]
  \definition[份]{s.}{recomendação}
  \definition{v.}{recomendar; apresentar pessoas ou coisas boas a pessoas ou organizações, na esperança de empregá-las ou aceitá-las}
  \seealsoref{介绍}{jie4shao4}
  \synonymref{推出}{tui1chu1}
  \synonymref{推选}{tui1xuan3}
\end{EntryWithPhonetic}

\begin{EntryWithPhonetic}{推介}{tui1jie4}{11,4}{⼿,⼈}
  \definition{s.}{promoção}
  \definition{v.}{promover | introduzir e recomendar}
\end{EntryWithPhonetic}

\begin{EntryWithPhonetic}{推进}{tui1jin4}{11,7}{⼿,⾡}[HSK 3]
  \definition{v.}{avançar; empurrar; levar adiante; dar ímpeto a; promover o trabalho e fazê-lo avançar | empurrar; dirigir; avançar; seguir em frente; seguir em frente}
\end{EntryWithPhonetic}

\begin{EntryWithPhonetic}{推开}{tui1kai1}{11,4}{⼿,⼶}[HSK 3]
  \definition{v.}{declinar; rejeitar | empurrar para longe; aplicar força em uma determinada direção para mover uma pessoa ou objeto para longe de seu lugar original | empurrar para abrir (um portão, etc.); empurrar para fora para abrir algo que está fechado | estender; popularizar; promover para um alcance mais amplo e realizar em uma escala mais ampla}
\end{EntryWithPhonetic}

\begin{EntryWithPhonetic}{推理}{tui1li3}{11,11}{⼿,⽟}[HSK 7-9]
  \definition{s.}{inferência; raciocínio; o processo de tirar novas conclusões com base em informações existentes}
  \synonymref{推断}{tui1duan4}
\end{EntryWithPhonetic}

\begin{EntryWithPhonetic}{推敲}{tui1qiao1}{11,14}{⼿,⽁}[HSK 7-9]
  \definition{v.}{pesar; deliberar; pensar repetidamente ao escrever ou realizar tarefas}
  \synonymref{揣摩}{chuai3mo2}
  \synonymref{考虑}{kao3lv4}
  \synonymref{商量}{shang1liang5}
  \synonymref{思考}{si1kao3}
  \synonymref{思索}{si1suo3}
  \synonymref{研究}{yan2jiu1}
\end{EntryWithPhonetic}

\begin{EntryWithPhonetic}{推算}{tui1suan4}{11,14}{⼿,⽵}[HSK 7-9]
  \definition{v.}{calcular; estimar; calcular os valores relevantes com base nos dados existentes}
  \seealsoref{推测}{tui1ce4}
  \synonymref{计算}{ji4suan4}
  \synonymref{算计}{suan4ji4}
  \synonymref{阴谋}{yin1mou2}
  \antonymref{断定}{duan4ding4}
\end{EntryWithPhonetic}

\begin{EntryWithPhonetic}{推销}{tui1xiao1}{11,12}{⼿,⾦}[HSK 4]
  \definition{v.}{vender; comercializar; promover vendas; promover a comercialização de mercadorias}
\end{EntryWithPhonetic}

\begin{EntryWithPhonetic}{推卸}{tui1xie4}{11,9}{⼿,⼙}[HSK 7-9]
  \definition{v.}{esquivar"-se (da responsabilidade); recusar"-se a assumir (responsabilidades e obrigações, etc.)}
  \synonymref{推辞}{tui1ci2}
  \synonymref{退却}{tui4que4}
  \antonymref{承担}{cheng2dan1}
  \antonymref{承受}{cheng2shou4}
  \antonymref{担负}{dan1fu4}
\end{EntryWithPhonetic}

\begin{EntryWithPhonetic}{推行}{tui1xing2}{11,6}{⼿,⾏}[HSK 5]
  \definition{v.}{realizar; prosseguir; praticar | implementar; praticar; implementação generalizada; divulgar (experiências, métodos, etc.)}
\end{EntryWithPhonetic}

\begin{EntryWithPhonetic}{推选}{tui1xuan3}{11,9}{⼿,⾡}[HSK 7-9]
  \definition{v.}{eleger; escolher; nomear}
  \seealsoref{选}{xuan3}
  \synonymref{竞选}{jing4xuan3}
  \synonymref{推荐}{tui1jian4}
  \synonymref{选举}{xuan3ju3}
  \antonymref{指定}{zhi3ding4}
\end{EntryWithPhonetic}

\begin{EntryWithPhonetic}{推移}{tui1yi2}{11,11}{⼿,⽲}[HSK 7-9]
  \definition{v.}{passar; decorrer; movimento, mudança ou desenvolvimento}
  \synonymref{发展}{fa1zhan3}
  \synonymref{推动}{tui1 dong4}
  \synonymref{推进}{tui1jin4}
  \synonymref{移动}{yi2dong4}
\end{EntryWithPhonetic}

%%%%%%%%%% 颓 %%%%%%%%%%
\subsection*{颓}\addcontentsline{loh}{figure}{颓 \dpy{tui2}}

\begin{EntryWithPhonetic}{颓}{tui2}{13}{⾴}
  \definition{adj.}{arruinado; dilapidado | em declínio; decadente | abatido; desanimado; apático}
  \definition{s.}{Literário: vento de tempestade}
  \definition{v.}{colapsar | deteriorar; declinar | desanimar; deprimir | fluir para baixo}
\end{EntryWithPhonetic}

\begin{EntryWithPhonetic}{颓废}{tui2fei4}{13,8}{⾴,⼴}[HSK 7-9]
  \definition{adj.}{decadente; desanimado e apático}
  \definition{s.}{decadência}
  \synonymref{悲哀}{bei1'ai1}
  \synonymref{悲观}{bei1guan1}
  \synonymref{悲伤}{bei1shang1}
  \synonymref{灰心}{hui1/xin1}
  \synonymref{沮丧}{ju3sang4}
  \synonymref{失恋}{shi1/lian4}
  \synonymref{失望}{shi1wang4}
  \synonymref{消极}{xiao1ji2}
  \antonymref{鼓舞}{gu3wu3}
  \antonymref{积极}{ji1ji2}
  \antonymref{激情}{ji1qing2}
  \antonymref{励志}{li4zhi4}
  \antonymref{拼搏}{pin1bo2}
\end{EntryWithPhonetic}

%%%%%%%%%% 腿 %%%%%%%%%%
\subsection*{腿}\addcontentsline{loh}{figure}{腿 \dpy{tui3}}

\begin{EntryWithPhonetic}{腿}{tui3}{13}{⾁}[HSK 2]
  \definition[条,双]{s.}{perna; as partes dos humanos e dos animais que sustentam o corpo e permitem caminhar | um suporte em forma de perna; a parte inferior de um objeto que atua como uma perna e serve de suporte | presunto}
\end{EntryWithPhonetic}

\begin{EntryWithPhonetic}{腿号}{tui3hao4}{13,5}{⾁,⼝}
  \definition{s.}{anilha numerada (por exemplo, usada para identificar pássaros)}
  \seealsoref{腿号箍}{tui3hao4gu1}
\end{EntryWithPhonetic}

\begin{EntryWithPhonetic}{腿号箍}{tui3hao4gu1}{13,5,14}{⾁,⼝,⽵}
  \definition{s.}{anilha numerada (por exemplo, usada para identificar pássaros)}
  \seealsoref{腿号}{tui3hao4}
\end{EntryWithPhonetic}

%%%%%%%%%% 退 %%%%%%%%%%
\subsection*{退}\addcontentsline{loh}{figure}{退 \dpy{tui4}}

\begin{EntryWithPhonetic}{退}{tui4}{9}{⾡}[HSK 3]
  \definition{v.}{recuar; mover"-se para trás | remover; retirar; fazer recuar; mover para trás | desistir; retirar"-se de | refluir; declinar; retroceder | aposentar"-se; deixar o emprego por atingir a idade estipulada ou por problemas de saúde | retornar; reembolsar; devolver | romper; cancelar o que foi decidido}
  \antonymref{进}{jin4}
\end{EntryWithPhonetic}

\begin{EntryWithPhonetic}{退场}{tui4chang3}{9,6}{⾡,⼟}
  \definition{v.}{(atletas) retirar-se da arena (como após a cerimônia de abertura); sair da arena em marcha; atletas deixando o campo após a competição | (uma plateia) sair do teatro (como quando uma peça termina)}
\end{EntryWithPhonetic}

\begin{EntryWithPhonetic}{退出}{tui4 chu1}{9,5}{⾡,⼐}[HSK 3]
  \definition{v.}{desistir; retirar-se; separar-se; retirar-se de; abandonar o local ou outro lugar e parar de participar; abandonaar o grupo ou organização}
\end{EntryWithPhonetic}

\begin{EntryWithPhonetic}{退回}{tui4hui2}{9,6}{⾡,⼞}[HSK 7-9]
  \definition{v.}{devolver; retornar; enviar de volta | voltar (ou retornar); retornar ao lugar original}
  \synonymref{归还}{gui1huan2}
  \antonymref{索取}{suo3qu3}
\end{EntryWithPhonetic}

\begin{EntryWithPhonetic}{退票}{tui4piao4}{9,11}{⾡,⽰}[HSK 6]
  \definition{s.}{bilhete devolvido (ou não utilizado) | reembolso do bilhete}
  \definition{v.}{devolver um bilhete; obter um reembolso por um bilhete | devolver (um cheque)}
\end{EntryWithPhonetic}

\begin{EntryWithPhonetic}{退却}{tui4que4}{9,7}{⾡,⼙}[HSK 7-9]
  \definition{v.}{recuar; retirar"-se | recuar; encolher"-se; estremecer | voltar}
  \synonymref{后退}{hou4tui4}
  \synonymref{推辞}{tui1ci2}
  \synonymref{推卸}{tui1xie4}
  \synonymref{退缩}{tui4suo1}
  \synonymref{畏惧}{wei4ju4}
  \synonymref{畏缩}{wei4suo1}
  \antonymref{冲锋}{chong1feng1}
  \antonymref{坚守}{jian1shou3}
  \antonymref{进攻}{jin4gong1}
  \antonymref{前进}{qian2jin4}
  \antonymref{挺身}{ting3shen1}
\end{EntryWithPhonetic}

\begin{EntryWithPhonetic}{退让}{tui4rang4}{9,5}{⾡,⾔}[HSK 7-9]
  \definition{v.}{ceder; fazer uma concessão; aceitar | recuar; afastar"-e}
  \synonymref{让步}{rang4/bu4}
  \synonymref{退却}{tui4que4}
  \synonymref{妥协}{tuo3xie2}
\end{EntryWithPhonetic}

\begin{EntryWithPhonetic}{退缩}{tui4suo1}{9,14}{⾡,⽷}[HSK 7-9]
  \definition{v.}{encolher"-se; recuar; acovardar"-se}
  \synonymref{退却}{tui4que4}
  \synonymref{畏缩}{wei4suo1}
  \antonymref{进展}{jin4zhan3}
  \antonymref{扩张}{kuo4zhang1}
  \antonymref{前进}{qian2jin4}
\end{EntryWithPhonetic}

\begin{EntryWithPhonetic}{退休}{tui4/xiu1}{9,6}{⾡,⼈}[HSK 3]
  \definition{v.+compl.}{aposentar-se; os trabalhadores que deixarem o emprego por velhice ou invalidez causada pelo trabalho receberão as despesas de subsistência conforme o cronograma}
\end{EntryWithPhonetic}

\begin{EntryWithPhonetic}{退休金}{tui4xiu1jin1}{9,6,8}{⾡,⼈,⾦}[HSK 7-9]
  \definition{s.}{aposentadoria; pensão}
\end{EntryWithPhonetic}

\begin{EntryWithPhonetic}{退学}{tui4/xue2}{9,8}{⾡,⼦}[HSK 7-9]
  \definition{v.+compl.}{abandonar a escola; interromper os estudos; a matrícula de um aluno pode ser cancelada caso ele não consiga continuar seus estudos por qualquer motivo, ou caso seja desqualificado de prosseguir com os estudos devido a graves infrações disciplinares}
  \antonymref{入学}{ru4/xue2}
\end{EntryWithPhonetic}

\begin{EntryWithPhonetic}{退役}{tui4/yi4}{9,7}{⾡,⼻}[HSK 7-9]
  \definition{v.+compl.}{aposentar-se das forças armadas; os soldados deixam as forças armadas após servirem por um determinado número de anos | aposentar-se da carreira profissional; os atletas já não encaram as competições desportivas como a sua profissão | ser retirado de serviço; equipamentos militares, instalações de transporte, etc., são desativados após um certo número de anos de uso}
\end{EntryWithPhonetic}

%%%%%%%%%% 吞 %%%%%%%%%%
\subsection*{吞}\addcontentsline{loh}{figure}{吞 \dpy{tun1}}

\begin{EntryWithPhonetic}{吞}{tun1}{7}{⼝}[HSK 6]
  \definition*{s.}{Sobrenome: Tun}
  \definition{v.}{engolir; engolir em seco | tomar posse de; anexar | engolir; tragar; devorar; engolir inteiro ou em pedaços | absorver; engolir; engolfar}
\end{EntryWithPhonetic}

%%%%%%%%%% 屯 %%%%%%%%%%
\subsection*{屯}\addcontentsline{loh}{figure}{屯 \dpy{tun2}}

\begin{EntryWithPhonetic}{屯}{tun2}{4}{⼬}[HSK 7-9]
  \definition*{s.}{Sobrenome: Tun}
  \definition{s.}{vila (geralmente usado em nomes de vilas); vilarejos; aldeias; povoados}
  \definition{v.}{coletar; estocar; armazenar; acumular | estacionar (tropas); aquartelar}
  \seeref{zhun1}
\end{EntryWithPhonetic}

%%%%%%%%%% 托 %%%%%%%%%%
\subsection*{托}\addcontentsline{loh}{figure}{托 \dpy{tuo1}}

\begin{EntryWithPhonetic}{托}{tuo1}{6}{⼿}[HSK 6]
  \definition{clas.}{torr, uma unidade de pressão, 1 torr é igual à pressão de 1 mmHg, ou 133,322 Pa}
  \definition{s.}{algo servindo como suporte | fantoche; cúmplice; pessoas que ajudam golpistas a enganar outras pessoas}
  \definition{v.}{segurar na palma; apoiar com a mão ou palma; suportar (um objeto) com um objeto ou com a palma da mão | destacar; servir como contraste | pedir; confiar | implorar; dar como pretexto | dever a; confiar em}
\end{EntryWithPhonetic}

\begin{EntryWithPhonetic}{托付}{tuo1fu4}{6,5}{⼿,⼈}[HSK 7-9]
  \definition{v.}{confiar; entregar algo aos cuidados de alguém; pedir a alguém que cuide de você ou que faça algo por você}
  \synonymref{拜托}{bai4tuo1}
  \synonymref{吩咐}{fen1fu4}
  \synonymref{寄托}{ji4tuo1}
  \synonymref{交付}{jiao1fu4}
  \synonymref{委托}{wei3tuo1}
  \synonymref{嘱托}{zhu3tuo1}
\end{EntryWithPhonetic}

%%%%%%%%%% 拖 %%%%%%%%%%
\subsection*{拖}\addcontentsline{loh}{figure}{拖 \dpy{tuo1}}

\begin{EntryWithPhonetic}{拖}{tuo1}{8}{⼿}[HSK 6]
  \definition{v.}{puxar; arrastar; transportar; puxar um objeto para movê-lo contra o solo ou outra superfície | esfregar; limpar o chão com uma ferramenta especial para esfregar | atrasar; prolongar; procrastinar; arrastar; coisas que deveriam ser feitas nunca são iniciadas ou concluídas; uma certa nota é prolongada por um longo tempo | atrasar; conter; segurar; restringir}
\end{EntryWithPhonetic}

\begin{EntryWithPhonetic}{拖拉机}{tuo1la1ji1}{8,8,6}{⼿,⼿,⽊}
  \definition[台]{s.}{trator}
\end{EntryWithPhonetic}

\begin{EntryWithPhonetic}{拖累}{tuo1lei3}{8,11}{⼿,⽷}[HSK 7-9]
  \definition{v.}{onerar; ser um fardo para; sobrecarregar outras pessoas ou coisas, impedindo seu desenvolvimento harmonioso}
  \antonymref{轻松}{qing1song1}
\end{EntryWithPhonetic}

\begin{EntryWithPhonetic}{拖欠}{tuo1qian4}{8,4}{⼿,⽋}[HSK 7-9]
  \definition{v.}{estar em atraso; estar com pagamentos atrasados; atrasar a devolução ou reter o pagamento}
  \synonymref{拖延}{tuo1yan2}
  \antonymref{偿还}{chang2huan2}
\end{EntryWithPhonetic}

\begin{EntryWithPhonetic}{拖鞋}{tuo1xie2}{8,15}{⼿,⾰}[HSK 6]
  \definition[双,只]{s.}{chinelos; sandálias; babouche; sapatos sem cabedal geralmente são usados em ambientes fechados}
\end{EntryWithPhonetic}

\begin{EntryWithPhonetic}{拖延}{tuo1yan2}{8,6}{⼿,⼵}[HSK 7-9]
  \definition{v.}{adiar; postergar; procrastinar; pendurar; prolongar o tempo, não processar rapidamente}
  \synonymref{迟到}{chi2/dao4}
  \synonymref{耽搁}{dan1ge5}
  \synonymref{耽误}{dan1wu5}
  \synonymref{缓慢}{huan3man4}
  \synonymref{拖欠}{tuo1qian4}
\end{EntryWithPhonetic}

%%%%%%%%%% 脱 %%%%%%%%%%
\subsection*{脱}\addcontentsline{loh}{figure}{脱 \dpy{tuo1}}

\begin{EntryWithPhonetic}{脱}{tuo1}{11}{⾁}[HSK 4]
  \definition{conj.}{se; no caso}
  \definition{v.}{(cabelo, pele) soltar-se; desprender-se; cair | retirar peça de roupa do corpo | sair de; escapar de | perder (palavras) | livrar-se de algo}
\end{EntryWithPhonetic}

\begin{EntryWithPhonetic}{脱节}{tuo1/jie2}{11,5}{⾁,⾋}[HSK 7-9]
  \definition{v.+compl.}{desfazer-se; estar desalinhado; estar em desacordo com | desengatar | estar fora de articulação}
  \synonymref{摆脱}{bai3/tuo1}
  \synonymref{离开}{li2/kai1}
  \synonymref{脱离}{tuo1li2}
  \antonymref{连接}{lian2jie1}
  \antonymref{联系}{lian2xi4}
\end{EntryWithPhonetic}

\begin{EntryWithPhonetic}{脱口而出}{tuo1kou3'er2chu1}{11,3,6,5}{⾁,⼝,⽽,⼐}[HSK 7-9]
  \definition{expr.}{deixar escapar; dizer algo sem querer; sem pensar, simplesmente deixando escapar}
  \synonymref{不假思索}{bu4jia3-si1suo3}
\end{EntryWithPhonetic}

\begin{EntryWithPhonetic}{脱离}{tuo1li2}{11,10}{⾁,⼇}[HSK 5]
  \definition{v.}{separar-se; divorciar-se; afastar-se; sair (de um determinado ambiente ou situação); romper (uma determinada relação)}
\end{EntryWithPhonetic}

\begin{EntryWithPhonetic}{脱落}{tuo1luo4}{11,12}{⾁,⾋}[HSK 7-9]
  \definition{v.}{cair; despencar; desprender; descascar | omitir (um caractere ao escrever); omitir texto}
\end{EntryWithPhonetic}

\begin{EntryWithPhonetic}{脱毛}{tuo1mao2}{11,4}{⾁,⽑}
  \definition{s.}{depilação}
  \definition{v.}{perder cabelo ou penas | depilar | fazer a barba}
\end{EntryWithPhonetic}

\begin{EntryWithPhonetic}{脱身}{tuo1/shen1}{11,7}{⾁,⾝}[HSK 7-9]
  \definition{v.+compl.}{escapar; libertar"-se; livrar"-se; sair de um determinado lugar; livrar"-se de algo}
\end{EntryWithPhonetic}

\begin{EntryWithPhonetic}{脱险}{tuo1xian3}{11,9}{⾁,⾩}
  \definition{v.}{sair do perigo}
\end{EntryWithPhonetic}

\begin{EntryWithPhonetic}{脱颖而出}{tuo1ying3'er2chu1}{11,13,6,5}{⾁,⾴,⽽,⼐}[HSK 7-9]
  \definition{expr.}{``A ponta inteira da sovela é visível através do saco de pano.''; isso significa, metaforicamente, que os talentos de uma pessoa talentosa acabarão sendo revelados; destacar"-se; sobressair"-se da multidão; sobressair"-se; emergir de forma proeminente; distinguir"-se}
\end{EntryWithPhonetic}

%%%%%%%%%% 驮 %%%%%%%%%%
\subsection*{驮}\addcontentsline{loh}{figure}{驮 \dpy{tuo2}}

\begin{EntryWithPhonetic}{驮}{tuo2}{6}{⾺}[HSK 7-9]
  \definition{v.}{carregar nas costas; apoiar objetos com as costas; suportar nas costas}
  \seeref{duo4}
\end{EntryWithPhonetic}

%%%%%%%%%% 柁 %%%%%%%%%%
\subsection*{柁}\addcontentsline{loh}{figure}{柁 \dpy{tuo2}}

\begin{EntryWithPhonetic}{柁}{tuo2}{9}{⽊}
  \definition{s.}{leme; leme | viga; uma grande viga horizontal em uma treliça de telhado de madeira}
  \seealsoref{舵}{duo4}
\end{EntryWithPhonetic}

%%%%%%%%%% 鸵 %%%%%%%%%%
\subsection*{鸵}\addcontentsline{loh}{figure}{鸵 \dpy{tuo2}}

\begin{EntryWithPhonetic}{鸵}{tuo2}{10}{⿃}
  \definition[只]{s.}{avestruz}
\end{EntryWithPhonetic}

\begin{EntryWithPhonetic}{鸵鸟}{tuo2niao3}{10,5}{⿃,⿃}
  \definition{s.}{avestruz}
\end{EntryWithPhonetic}

%%%%%%%%%% 妥 %%%%%%%%%%
\subsection*{妥}\addcontentsline{loh}{figure}{妥 \dpy{tuo3}}

\begin{EntryWithPhonetic}{妥}{tuo3}{7}{⼥}[HSK 7-9]
  \definition*{s.}{Sobrenome: Tuo}
  \definition{adj.}{apropriado; adequado | (geralmente após um verbo) pronto; resolvido; terminado}
\end{EntryWithPhonetic}

\begin{EntryWithPhonetic}{妥当}{tuo3dang4}{7,6}{⼥,⼹}[HSK 7-9]
  \definition{adj.}{adequado; apropriado; conveniente; prudente e apropriado}
  \synonymref{得当}{de2dang4}
  \synonymref{恰当}{qia4dang4}
  \synonymref{适当}{shi4dang4}
  \synonymref{事宜}{shi4yi2}
  \synonymref{停当}{ting2dang5}
  \synonymref{妥善}{tuo3shan4}
\end{EntryWithPhonetic}

\begin{EntryWithPhonetic}{妥善}{tuo3shan4}{7,12}{⼥,⼝}[HSK 7-9]
  \definition{adj.}{adequado; apropriado; bem organizado; adequado e completo}
  \synonymref{得当}{de2dang4}
  \synonymref{恰当}{qia4dang4}
  \synonymref{适宜}{shi4yi2}
  \synonymref{事宜}{shi4yi2}
  \synonymref{停当}{ting2dang5}
  \synonymref{妥当}{tuo3dang4}
\end{EntryWithPhonetic}

\begin{EntryWithPhonetic}{妥协}{tuo3xie2}{7,6}{⼥,⼗}[HSK 7-9]
  \definition{v.}{fazer concessões; chegar a um acordo; evitar conflitos ou disputas fazendo concessões}
  \synonymref{和解}{he2jie3}
  \synonymref{和睦}{he2mu4}
  \synonymref{迁就}{qian1jiu4}
  \synonymref{让步}{rang4/bu4}
  \synonymref{退让}{tui4rang4}
  \synonymref{协调}{xie2tiao2}
  \antonymref{斗争}{dou4zheng1}
  \antonymref{对立}{dui4li4}
  \antonymref{对峙}{dui4zhi4}
  \antonymref{挑衅}{tiao3xin4}
  \antonymref{争议}{zheng1yi4}
\end{EntryWithPhonetic}

%%%%%%%%%% 拓 %%%%%%%%%%
\subsection*{拓}\addcontentsline{loh}{figure}{拓 \dpy{tuo4}}

\begin{EntryWithPhonetic}{拓}{tuo4}{8}{⼿}
  \definition*{s.}{Sobrenome: Tuo}
  \definition{v.}{abrir; desenvolver | expandir}
  \seeref{ta4}
\end{EntryWithPhonetic}

\begin{EntryWithPhonetic}{拓宽}{tuo4kuan1}{8,10}{⼿,⼧}[HSK 7-9]
  \definition{v.}{ampliar; expandir}
  \synonymref{开拓}{kai1tuo4}
  \synonymref{拓展}{tuo4zhan3}
  \antonymref{收缩}{shou1suo1}
\end{EntryWithPhonetic}

\begin{EntryWithPhonetic}{拓展}{tuo4zhan3}{8,10}{⼿,⼫}[HSK 7-9]
  \definition{v.}{expandir; desenvolver; disseminar; ampliar o escopo}
  \synonymref{开阔}{kai1kuo4}
  \synonymref{开拓}{kai1tuo4}
  \synonymref{扩展}{kuo4zhan3}
  \synonymref{拓宽}{tuo4kuan1}
  \synonymref{延伸}{yan2shen1}
  \antonymref{缩小}{suo1/xiao3}
\end{EntryWithPhonetic}

%%%%%%%%%% 唾 %%%%%%%%%%
\subsection*{唾}\addcontentsline{loh}{figure}{唾 \dpy{tuo4}}

\begin{EntryWithPhonetic}{唾}{tuo4}{11}{⼝}
  \definition[口]{s.}{saliva; cuspe}
  \definition{v.}{cuspir (mostrar desprezo) | rejeitar}
\end{EntryWithPhonetic}

\begin{EntryWithPhonetic}{唾骂}{tuo4ma4}{11,9}{⼝,⾺}
  \definition{v.}{insultar | amaldiçoar}
\end{EntryWithPhonetic}

\begin{EntryWithPhonetic}{唾液}{tuo4ye4}{11,11}{⼝,⽔}[HSK 7-9]
  \definition[口]{s.}{saliva; cuspe}
  \synonymref{口水}{kou3shui3}
\end{EntryWithPhonetic}

%%%%%%%%%% 魄 %%%%%%%%%%
\subsection*{魄}\addcontentsline{loh}{figure}{魄 \dpy{tuo4}}

\begin{EntryWithPhonetic}{魄}{tuo4}{14}{⿁}
  \definition{adj.}{desanimado; sem ânimo; mentalmente abatido; outra pronúncia de 魄 em 落魄}
  \seealsoref{落魄}{luo4po4}
\end{EntryWithPhonetic}

%%%%% EOF %%%%%

