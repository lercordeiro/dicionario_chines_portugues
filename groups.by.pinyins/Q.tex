%%%
%%% Q
%%%
\section*{Q}\addcontentsline{toc}{section}{Q}\addcontentsline{loh}{figure}{\#\#\#\#\#\#\#\# Q}

%%%%%%%%%% 七 %%%%%%%%%%
\subsection*{七}\addcontentsline{loh}{figure}{七 \dpy{qi1}}

\begin{EntryWithPhonetic}{七}{qi1}{2}{⼀}[HSK 1]
  \definition*{s.}{Sobrenome: Qi}
  \definition{num.}{sete; 7}
  \definition{s.}{antigamente, os mortos eram homenageados a cada sete dias, chamados de 七, até o quadragésimo nono dia, num total de sete 七}
\end{EntryWithPhonetic}

\begin{EntryWithPhonetic}{七夕}{qi1xi1}{2,3}{⼀,⼣}
  \definition*{s.}{Dia dos Namorados Chinês, quando o vaqueiro e a tecelã (牛郎织女) têm permissão para se reunirem anualmente | Festival das Meninas | Festival Duplo Sete, noite do sétimo mês lunar}
  \seealsoref{牛郎织女}{niu2 lang2 zhi1nv3}
\end{EntryWithPhonetic}

\begin{EntryWithPhonetic}{七嘴八舌}{qi1zui3-ba1she2}{2,16,2,6}{⼀,⼝,⼋,⾆}[HSK 7-9]
  \definition{expr.}{``Uma cacofonia de vozes.''; todos falando ao mesmo tempo; falando uns por cima dos outros; isso descreve uma situação em que muitas pessoas estão falando ao mesmo tempo, com opiniões conflitantes; também descreve alguém que é falante e fofoqueiro}
\end{EntryWithPhonetic}

%%%%%%%%%% 沏 %%%%%%%%%%
\subsection*{沏}\addcontentsline{loh}{figure}{沏 \dpy{qi1}}

\begin{EntryWithPhonetic}{沏}{qi1}{7}{⽔}[HSK 7-9]
  \definition{v.}{infundir (com água fervente); dissolver (em água fervente)}[妈妈喜欢沏一壶绿茶。===Minha mãe gosta de infundir um bule de chá verde.]
\end{EntryWithPhonetic}

%%%%%%%%%% 妻 %%%%%%%%%%
\subsection*{妻}\addcontentsline{loh}{figure}{妻 \dpy{qi1}}

\begin{EntryWithPhonetic}{妻}{qi1}{8}{⼥}
  \definition{s.}{esposa}
  \seeref{qi4}
\end{EntryWithPhonetic}

\begin{EntryWithPhonetic}{妻子}{qi1zi3}{8,3}{⼥,⼦}
  \definition[个]{s.}{esposa e filhos; (chinês antigo) refere-se a esposas, filhos e filhas}
  \seeref{qi1zi5}
\end{EntryWithPhonetic}

\begin{EntryWithPhonetic}{妻子}{qi1zi5}{8,3}{⼥,⼦}[HSK 4]
  \definition[个]{s.}{esposa (não é usado como um termo carinhoso)}
  \seeref{qi1zi3}
\end{EntryWithPhonetic}

%%%%%%%%%% 凄 %%%%%%%%%%
\subsection*{凄}\addcontentsline{loh}{figure}{凄 \dpy{qi1}}

\begin{EntryWithPhonetic}{凄}{qi1}{10}{⼎}
  \definition{adj.}{frio; gelado | sombrio e desolado | triste; miserável; infeliz}
\end{EntryWithPhonetic}

\begin{EntryWithPhonetic}{凄凉}{qi1liang2}{10,10}{⼎,⼎}[HSK 7-9]
  \definition{adj.}{sombrio; desolador; solitário e desolado; miserável (frequentemente usado para descrever um ambiente ou cena) | miserável; sombrio e desolado; desolado e trágico}[夜晚的街道显得很凄凉。===As ruas estavam sombrias à noite.|他的眼神看起来很凄凉。===Seus olhos pareciam muito desolados.]
\end{EntryWithPhonetic}

%%%%%%%%%% 期 %%%%%%%%%%
\subsection*{期}\addcontentsline{loh}{figure}{期 \dpy{qi1}}

\begin{EntryWithPhonetic}{期}{qi1}{12}{⽉}[HSK 3]
  \definition{clas.}{questão; número; termo; coisas usadas para parcelamento}
  \definition{s.}{um período de tempo; fase; estágio | horário agendado; data agendada | tempo designado (programado)}
  \definition{v.}{marcar uma consulta | esperar; aguardar | esperar; ter esperança}
\end{EntryWithPhonetic}

\begin{EntryWithPhonetic}{期待}{qi1dai4}{12,9}{⽉,⼻}[HSK 4]
  \definition{v.}{aguardar; esperar; aguardar ansiosamente; ter em mente a realização de um determinado fim ou a ocorrência de uma determinada situação}
\end{EntryWithPhonetic}

\begin{EntryWithPhonetic}{期间}{qi1jian1}{12,7}{⽉,⾨}[HSK 4]
  \definition{s.}{prazo; tempo; período}
\end{EntryWithPhonetic}

\begin{EntryWithPhonetic}{期末}{qi1 mo4}{12,5}{⽉,⽊}[HSK 4]
  \definition{s.}{terminal; final do prazo; fim do período}
\end{EntryWithPhonetic}

\begin{EntryWithPhonetic}{期盼}{qi1pan4}{12,9}{⽉,⽬}[HSK 7-9]
  \definition{v.}{esperar; aguardar}
\end{EntryWithPhonetic}

\begin{EntryWithPhonetic}{期望}{qi1wang4}{12,11}{⽉,⽉}[HSK 5]
  \definition{s.}{esperança; expectativa}
  \definition{v.}{esperar; ter esperança}
\end{EntryWithPhonetic}

\begin{EntryWithPhonetic}{期限}{qi1xian4}{12,8}{⽉,⾩}[HSK 4]
  \definition{s.}{prazo; limite de tempo; tempo alocado; período de tempo limitado, também o limite final do limite de tempo; \emph{deadline}}
\end{EntryWithPhonetic}

\begin{EntryWithPhonetic}{期中}{qi1 zhong1}{12,4}{⽉,⼁}[HSK 4]
  \definition{adj.}{provisório; interino; intermediário}
\end{EntryWithPhonetic}

%%%%%%%%%% 欺 %%%%%%%%%%
\subsection*{欺}\addcontentsline{loh}{figure}{欺 \dpy{qi1}}

\begin{EntryWithPhonetic}{欺}{qi1}{12}{⽋}
  \definition{v.}{enganar; trapacear | intimidar; tirar vantagem de alguém; tirar vantagem da fraqueza de (alguém, etc.)}
\end{EntryWithPhonetic}

\begin{EntryWithPhonetic}{欺负}{qi1fu5}{12,6}{⽋,⾙}[HSK 6]
  \definition{v.}{violar, oprimir ou insultar com meios irracionais; \emph{bully}}
\end{EntryWithPhonetic}

\begin{EntryWithPhonetic}{欺骗}{qi1pian4}{12,12}{⽋,⾺}[HSK 7-9]
  \definition{v.}{enganar; ludibriar; usar palavras ou ações falsas para encobrir a verdade e enganar as pessoas}
\end{EntryWithPhonetic}

\begin{EntryWithPhonetic}{欺诈}{qi1zha4}{12,7}{⽋,⾔}[HSK 7-9]
  \definition{v.}{trapacear; enganar; usar métodos astutos para enganar as pessoas para obter lucro}
\end{EntryWithPhonetic}

%%%%%%%%%% 漆 %%%%%%%%%%
\subsection*{漆}\addcontentsline{loh}{figure}{漆 \dpy{qi1}}

\begin{EntryWithPhonetic}{漆}{qi1}{14}{⽔}[HSK 7-9]
  \definition*{s.}{Sobrenome: Qi}
  \definition[桶,种,层]{s.}{laca; tinta}
  \definition{v.}{aplicar verniz; pintar}
\end{EntryWithPhonetic}

%%%%%%%%%% 齐 %%%%%%%%%%
\subsection*{齐}\addcontentsline{loh}{figure}{齐 \dpy{qi2}}

\begin{EntryWithPhonetic}{齐}{qi2}{6}{⿑}[HSK 3][Kangxi 210]
  \definition*{s.}{Qi, um estado da Dinastia Zhou | Dinastia Qi do Sul (479-502), uma das Dinastias do Sul | Dinastia Qi do Norte (550-577), uma das Dinastias do Norte | Sobrenome: Qi}
  \definition{adj.}{arrumado; uniforme; regular; comprimento, tamanho, etc. são praticamente iguais; uniformes | semelhante; similar; da mesma forma; de acordo| tudo pronto; todos presentes; completo; perfeito}
  \definition{adv.}{juntos; simultaneamente; ao mesmo tempo}
  \definition{v.}{estar no mesmo nível que; alcançar o mesmo nível | estar nivelado em um ponto ou ao longo de uma linha; tornar consistente; harmonizar}
\end{EntryWithPhonetic}

\begin{EntryWithPhonetic}{齐国}{qi2 guo2}{6,8}{⿑,⼞}
  \definition*{s.}{Estado Qi de Zhou Ocidental e os Estados Combatentes (1122-265 a.C.), centrado em Shandong}
\end{EntryWithPhonetic}

\begin{EntryWithPhonetic}{齐全}{qi2quan2}{6,6}{⿑,⼊}[HSK 5]
  \definition{adj.}{completo; tudo pronto}
\end{EntryWithPhonetic}

\begin{EntryWithPhonetic}{齐心协力}{qi2xin1-xie2li4}{6,4,6,2}{⿑,⼼,⼗,⼒}[HSK 7-9]
  \definition{expr.}{trabalhar em conjunto; fazer esforços concertados; reunir; trabalhar como um; trabalhar com um propósito comum; fazer esforços conjuntos}
\end{EntryWithPhonetic}

%%%%%%%%%% 其 %%%%%%%%%%
\subsection*{其}\addcontentsline{loh}{figure}{其 \dpy{qi2}}

\begin{EntryWithPhonetic}{其}{qi2}{8}{⼋}[HSK 5]
  \definition*{s.}{Sobrenome: Qi}
  \definition{adv.}{fazer uma suposição ou uma réplica | expressar comando, ordem}
  \definition{pron.}{dele (dela, deles, delas) | ele, ela, isso, eles; elas | isso; tal | isso (referindo-se a nenhuma pessoa ou coisa específica)}
  \definition{suf.}{sufixo de palavra, anexado ao advérbio}
\end{EntryWithPhonetic}

\begin{EntryWithPhonetic}{其次}{qi2ci4}{8,6}{⼋,⽋}[HSK 3]
  \definition{adj.}{secundário}
  \definition{conj.}{próximo; então; em segundo lugar; mais tarde na ordem}
\end{EntryWithPhonetic}

\begin{EntryWithPhonetic}{其后}{qi2hou4}{8,6}{⼋,⼝}[HSK 7-9]
  \definition{adv.}{mais tarde; depois; posteriormente | depois disso | próximo}
\end{EntryWithPhonetic}

\begin{EntryWithPhonetic}{其间}{qi2jian1}{8,7}{⼋,⾨}[HSK 7-9]
  \definition{s.}{nele; deles; entre eles; no meio | durante este (ou aquele) período; dentro de um determinado período de tempo}
\end{EntryWithPhonetic}

\begin{EntryWithPhonetic}{其实}{qi2shi2}{8,8}{⼋,⼧}[HSK 3]
  \definition{adv.}{na verdade; na realidade; a primeira parte é a situação aparente, e 其实 é usado para introduzir a situação real}
\end{EntryWithPhonetic}

\begin{EntryWithPhonetic}{其他}{qi2ta1}{8,5}{⼋,⼈}[HSK 2]
  \definition{pron.}{outra pessoa/outra coisa | outras coisas; outras pessoas; em substituição de outras pessoas ou coisas}
\end{EntryWithPhonetic}

\begin{EntryWithPhonetic}{其余}{qi2yu2}{8,7}{⼋,⼈}[HSK 4]
  \definition{pron.}{o resto; os outros; o restante}
\end{EntryWithPhonetic}

\begin{EntryWithPhonetic}{其中}{qi2zhong1}{8,4}{⼋,⼁}[HSK 2]
  \definition{pron.}{dentro; entre (os quais, eles, etc.); em (o qual, ele, etc.); nas pessoas ou coisas mencionadas anteriormente}
\end{EntryWithPhonetic}

%%%%%%%%%% 奇 %%%%%%%%%%
\subsection*{奇}\addcontentsline{loh}{figure}{奇 \dpy{qi2}}

\begin{EntryWithPhonetic}{奇}{qi2}{8}{⼤}
  \definition{adj.}{ímpar (número); singular; solteiro; não em pares (ao contrário de 偶)}
  \definition{s.}{lotes ímpares; quantidade fracionária (acima daquela mencionada em um número redondo)}
  \seealsoref{偶}{ou3}
\end{EntryWithPhonetic}

\begin{EntryWithPhonetic}{奇怪}{qi2guai4}{8,8}{⼤,⼼}[HSK 3]
  \definition{adj.}{estranho; diferente do habitual; raramente visto, até um pouco irracional | estranho; esquisito; a descrição é diferente do imaginado e é difícil de entender}
  \definition{v.}{ficar perplexo; maravilhar-se; sentir-se surpreso; sentir-se estranho; sentir-se incompreensível}
\end{EntryWithPhonetic}

\begin{EntryWithPhonetic}{奇花异草}{qi2hua1-yi4cao3}{8,7,6,9}{⼤,⾋,⼶,⾋}[HSK 7-9]
  \definition{expr.}{``Flores exóticas e ervas raras.''; vista espetacular do mundo botânico; muito raramente visto, incomum}
\end{EntryWithPhonetic}

\begin{EntryWithPhonetic}{奇迹}{qi2ji4}{8,9}{⼤,⾡}[HSK 7-9]
  \definition[个,种]{s.}{milagre; maravilha; coisas extraordinárias inimagináveis}
\end{EntryWithPhonetic}

\begin{EntryWithPhonetic}{奇妙}{qi2miao4}{8,7}{⼤,⼥}[HSK 6]
  \definition{adj.}{maravilhoso; milagroso; intrigante; muito inteligente e engenhoso (usado principalmente para descrever coisas interessantes e novas)}
\end{EntryWithPhonetic}

\begin{EntryWithPhonetic}{奇特}{qi2te4}{8,10}{⼤,⽜}[HSK 7-9]
  \definition{adj.}{estranho; peculiar; singular; incomum; extraordinário}
\end{EntryWithPhonetic}

%%%%%%%%%% 歧 %%%%%%%%%%
\subsection*{歧}\addcontentsline{loh}{figure}{歧 \dpy{qi2}}

\begin{EntryWithPhonetic}{歧}{qi2}{8}{⽌}
  \definition{adj.}{divergente; diferente; ambíguo; inconsistente}
  \definition{s.}{bifurcação; ramificação; bifurcação da estrada; uma estrada que se ramifica de uma estrada principal}
\end{EntryWithPhonetic}

\begin{EntryWithPhonetic}{歧视}{qi2shi4}{8,8}{⽌,⾒}[HSK 7-9]
  \definition[种,些,点]{s.}{discriminação}
  \definition{v.}{discriminar contra; discriminar; tratar alguém ou um grupo de forma desigual, com uma atitude injusta ou desproporcional}
\end{EntryWithPhonetic}

%%%%%%%%%% 祈 %%%%%%%%%%
\subsection*{祈}\addcontentsline{loh}{figure}{祈 \dpy{qi2}}

\begin{EntryWithPhonetic}{祈}{qi2}{8}{⽰}
  \definition{v.}{orar; rezar}
\end{EntryWithPhonetic}

\begin{EntryWithPhonetic}{祈祷}{qi2dao3}{8,11}{⽰,⽰}[HSK 7-9]
  \definition{v.}{orar; realizar um ritual religioso; expressar silenciosamente seus desejos a Deus}
\end{EntryWithPhonetic}

%%%%%%%%%% 骑 %%%%%%%%%%
\subsection*{骑}\addcontentsline{loh}{figure}{骑 \dpy{qi2}}

\begin{EntryWithPhonetic}{骑}{qi2}{11}{⾺}[HSK 2]
  \definition{s.}{cavalos ou outros animais para montaria | cavalaria; cavaleiro, também se refere genericamente a qualquer pessoa que monta a cavalo}
  \definition{v.}{montar (um animal ou bicicleta); sentar-se na parte de trás de | montar; abranger ambos os lados}
\end{EntryWithPhonetic}

\begin{EntryWithPhonetic}{骑车}{qi2 che1}{11,4}{⾺,⾞}[HSK 2]
  \definition{v.}{andar de bicicleta; pedalar}
\end{EntryWithPhonetic}

%%%%%%%%%% 棋 %%%%%%%%%%
\subsection*{棋}\addcontentsline{loh}{figure}{棋 \dpy{qi2}}

\begin{EntryWithPhonetic}{棋}{qi2}{12}{⽊}[HSK 7-9]
  \definition[盘]{s.}{xadrez | jogo semelhante ao xadrez | uma partida de xadrez}
  \definition[个,颗]{s.}{peça de xadrez}
\end{EntryWithPhonetic}

\begin{EntryWithPhonetic}{棋子儿}{qi2zi3r5}{12,3,2}{⽊,⼦,⼉}
  \definition{s.}{peça de xadrez}
\end{EntryWithPhonetic}

\begin{EntryWithPhonetic}{棋子}{qi2zi5}{12,3}{⽊,⼦}[HSK 7-9]
  \definition[个,颗]{s.}{peça (em um jogo de tabuleiro); peça de xadrez}
  \seealsoref{棋子儿}{qi2zi3r5}
\end{EntryWithPhonetic}

%%%%%%%%%% 旗 %%%%%%%%%%
\subsection*{旗}\addcontentsline{loh}{figure}{旗 \dpy{qi2}}

\begin{EntryWithPhonetic}{旗}{qi2}{14}{⽅}
  \definition[面]{s.}{bandeira}
\end{EntryWithPhonetic}

\begin{EntryWithPhonetic}{旗袍}{qi2pao2}{14,10}{⽅,⾐}[HSK 7-9]
  \definition[件,个]{s.}{qipao; cheongsam; uma túnica longa usada por mulheres, originalmente usada por mulheres manchus}
\end{EntryWithPhonetic}

\begin{EntryWithPhonetic}{旗帜}{qi2zhi4}{14,8}{⽅,⼱}[HSK 7-9]
  \definition[面]{s.}{bandeira; estandarte | modelo; bom exemplo; metáfora para modelo ou exemplo a seguir | bandeira (de um pensamento representativo ou posição política); essa metáfora se refere a uma ideologia, doutrina ou força política representativa ou influente}
\end{EntryWithPhonetic}

%%%%%%%%%% 乞 %%%%%%%%%%
\subsection*{乞}\addcontentsline{loh}{figure}{乞 \dpy{qi3}}

\begin{EntryWithPhonetic}{乞}{qi3}{3}{⼄}
  \definition*{s.}{Sobrenome: Qi}
  \definition{v.}{implorar (por esmolas, etc.); suplicar}
\end{EntryWithPhonetic}

\begin{EntryWithPhonetic}{乞丐}{qi3gai4}{3,4}{⼄,⼀}[HSK 7-9]
  \definition[个,位,群]{s.}{mendigo; pessoas que não têm meios de subsistência e dependem exclusivamente da mendicância para conseguir comida e dinheiro para sobreviver}
\end{EntryWithPhonetic}

\begin{EntryWithPhonetic}{乞求}{qi3qiu2}{3,7}{⼄,⽔}[HSK 7-9]
  \definition{v.}{implorar; suplicar; mendigar | cair de joelhos}
\end{EntryWithPhonetic}

\begin{EntryWithPhonetic}{乞讨}{qi3tao3}{3,5}{⼄,⾔}[HSK 7-9]
  \definition{v.}{implorar; pedir dinheiro, pedir comida, etc.}
\end{EntryWithPhonetic}

%%%%%%%%%% 企 %%%%%%%%%%
\subsection*{企}\addcontentsline{loh}{figure}{企 \dpy{qi3}}

\begin{EntryWithPhonetic}{企}{qi3}{6}{⼈}
  \definition{v.}{ficar na ponta dos pés | esperar ansiosamente por algo; ansiar por | planejar um projeto}
\end{EntryWithPhonetic}

\begin{EntryWithPhonetic}{企图}{qi3tu2}{6,8}{⼈,⼞}[HSK 6]
  \definition[种]{s.}{plano; tentativa; intenção (principalmente negativa)}
  \definition{v.}{procurar; tentar; pretender}
\end{EntryWithPhonetic}

\begin{EntryWithPhonetic}{企业}{qi3ye4}{6,5}{⼈,⼀}[HSK 4]
  \definition[家,个]{s.}{empresa; estabelecimento; empreendimento; negócio; setores envolvidos em atividades econômicas como produção, transporte, comércio, etc., como fábricas, minas, ferrovias, empresas comerciais, etc.}
\end{EntryWithPhonetic}

%%%%%%%%%% 岂 %%%%%%%%%%
\subsection*{岂}\addcontentsline{loh}{figure}{岂 \dpy{qi3}}

\begin{EntryWithPhonetic}{岂}{qi3}{6}{⼭}
  \definition*{s.}{Sobrenome: Qi}
  \definition{adv.}{Litarário: expressa uma pergunta retórica, equivalente a 哪里, 怎么 e 难道}
  \seealsoref{哪里}{na3 li3}
  \seealsoref{难道}{nan2dao4}
  \seealsoref{怎么}{zen3me5}
\end{EntryWithPhonetic}

\begin{EntryWithPhonetic}{岂有此理}{qi3you3ci3li3}{6,6,6,11}{⼭,⽉,⽌,⽟}[HSK 7-9]
  \definition{expr./interj.}{Isso é um absurdo!; Que exorbitante!; Absurdo!; Como isso pode ser assim?; Ridículo!}
\end{EntryWithPhonetic}

%%%%%%%%%% 启 %%%%%%%%%%
\subsection*{启}\addcontentsline{loh}{figure}{启 \dpy{qi3}}

\begin{EntryWithPhonetic}{启}{qi3}{7}{⼝}
  \definition*{s.}{Sobrenome: Qi}
  \definition{s.}{nota; carta; um dos antigos estilos literários, uma carta relativamente curta}
  \definition{v.}{abrir | despertar; iluminar | começar; iniciar | declarar; informar}
\end{EntryWithPhonetic}

\begin{EntryWithPhonetic}{启程}{qi3cheng2}{7,12}{⼝,⽲}
  \definition{v.}{partir; iniciar uma viagem; começar sua longa jornada}
\end{EntryWithPhonetic}

\begin{EntryWithPhonetic}{启迪}{qi3di2}{7,8}{⼝,⾡}[HSK 7-9]
  \definition{v.}{inspirar; iluminar; incentivar;}
\end{EntryWithPhonetic}

\begin{EntryWithPhonetic}{启动}{qi3 dong4}{7,6}{⼝,⼒}[HSK 5]
  \definition{v.}{ligar (uma máquina); acionar; ligar máquinas, equipamentos elétricos, etc., para começar a trabalhar | entrar em vigor; começar a vigorar e a ser implementados planos, projetos, documentos jurídicos, etc.}
\end{EntryWithPhonetic}

\begin{EntryWithPhonetic}{启发}{qi3fa1}{7,5}{⼝,⼜}[HSK 5]
  \definition{s.}{iluminação; esclarecimento; fenômenos e princípios que levam as pessoas a refletir e a abrir suas mentes}
  \definition{v.}{despertar; inspirar; esclarecer; orientar, fazer com que compreendam}
\end{EntryWithPhonetic}

\begin{EntryWithPhonetic}{启蒙}{qi3meng2}{7,13}{⼝,⾋}[HSK 7-9]
  \definition{v.}{iniciar; transmitir conhecimento rudimentar a iniciantes; ensinar conhecimentos ou habilidades básicas a iniciantes | esclarecer; libertar alguém de preconceitos ou superstições; por meio da publicidade e da educação, a sociedade pode aceitar coisas novas e se livrar da ignorância e do atraso}
\end{EntryWithPhonetic}

\begin{EntryWithPhonetic}{启示}{qi3shi4}{7,5}{⼝,⽰}[HSK 7-9]
  \definition[个,条,种]{s.}{revelação; inspiração; iluminação; orientação inspiradora leva à compreensão; dicas e sugestões para ajudar você a entender}
  \definition{v.}{revelar; inspirar}
\end{EntryWithPhonetic}

\begin{EntryWithPhonetic}{启事}{qi3shi4}{7,8}{⼝,⼅}[HSK 5]
  \definition[个,则,份,张,条]{s.}{aviso; anúncio; texto publicado em jornais ou afixado em paredes com o objetivo de divulgar publicamente algo}
\end{EntryWithPhonetic}

%%%%%%%%%% 起 %%%%%%%%%%
\subsection*{起}\addcontentsline{loh}{figure}{起 \dpy{qi3}}

\begin{EntryWithPhonetic}{起}{qi3}{10}{⾛}[HSK 1]
  \definition{clas.}{caso; instância | lote; grupo}
  \definition{prep.}{de; colocado antes de uma palavra de tempo ou lugar, indica um ponto de partida | por; colocado antes de uma palavra de lugar, indica um lugar por onde passou}
  \definition{v.}{levantar-se; ficar de pé| iniciar; lançar; deicar a posição original | subir; ascender | aparecer; levantar; crescer (bolhas, protuberâncias, brotoeja) | puxar para cima; puxar para fora; tirar o que está guardado ou incorporado | crescer; aumentar | esboçar; elaborar | construir; montar; estabelecer | receber (comprovante) | começar; iniciar; combina com 从 e 由; indica quando, onde e quem começou | buscar; pegar; usado após um verbo, indica movimento para cima | indicar se alguém tem força suficiente ou não; usado após um verbo, indica que a força é suficiente ou insuficiente | indicar que a ação envolve alguém ou algo; equivalente a 及 ou 到 | começar; iniciar; usado depois de um verbo, indica o início de uma ação | juntar; implodir; (informal) usado depois de um verbo, para unir coisas ou fechá-las}
  \seealsoref{从}{cong2}
  \seealsoref{到}{dao4}
  \seealsoref{及}{ji2}
  \seealsoref{由}{you2}
\end{EntryWithPhonetic}

\begin{EntryWithPhonetic}{起步}{qi3bu4}{10,7}{⾛,⽌}[HSK 7-9]
  \definition{v.}{começar; mover-se; começar a dar os primeiros passos | começar; dar início; metaforicamente, significa o início de (uma carreira, emprego, etc.)}
\end{EntryWithPhonetic}

\begin{EntryWithPhonetic}{起草}{qi3/cao3}{10,9}{⾛,⾋}[HSK 7-9]
  \definition{v.+compl.}{elaborar; redigir; preparar um rascunho}
\end{EntryWithPhonetic}

\begin{EntryWithPhonetic}{起程}{qi3cheng2}{10,12}{⾛,⽲}[HSK 7-9]
  \definition{v.}{partir; sair; iniciar uma viagem}
  \seealsoref{启程}{qi3cheng2}
\end{EntryWithPhonetic}

\begin{EntryWithPhonetic}{起初}{qi3chu1}{10,7}{⾛,⾐}[HSK 7-9]
  \definition{adv.}{princípio; início; inicial (frequentemente contrastado com 后来 ou 现在)}
  \seealsoref{后来}{hou4lai2}
  \seealsoref{现在}{xian4zai4}
\end{EntryWithPhonetic}

\begin{EntryWithPhonetic}{起床}{qi3/chuang2}{10,7}{⾛,⼴}[HSK 1]
  \definition{v.+compl.}{levantar-se; sair da cama; acordar e sair da cama (geralmente pela manhã); levantar-se da posição sentada, deitada ou deitada de bruços, ou sentar-se a partir da posição deitada}
\end{EntryWithPhonetic}

\begin{EntryWithPhonetic}{起到}{qi3 dao4}{10,8}{⾛,⼑}[HSK 5]
  \definition{v.}{ter (um efeito motivador, etc.); desempenhar (um papel estabilizador, etc.)}
\end{EntryWithPhonetic}

\begin{EntryWithPhonetic}{起点}{qi3 dian3}{10,9}{⾛,⽕}[HSK 6]
  \definition[个]{s.}{ponto de partida (para o tempo ou local do início de algo); o lugar ou hora de início | ponto de partida (para o nível ou base de algo feito inicialmente); refere-se especificamente ao ponto de partida designado em um evento de pista}
\end{EntryWithPhonetic}

\begin{EntryWithPhonetic}{起飞}{qi3fei1}{10,3}{⾛,⾶}[HSK 2]
  \definition{v.}{decolar; levantar voo | crescer rapidamente; decolar; disparar; metáfora para o rápido desenvolvimento de negócios, economia, etc.}
\end{EntryWithPhonetic}

\begin{EntryWithPhonetic}{起伏}{qi3fu2}{10,6}{⾛,⼈}[HSK 7-9]
  \definition{v.}{subir e descer; ondular;  subida e descida contínuas | (emoções, relações, etc.) flutuar; oscilar; aumentar e diminuir; essa metáfora descreve a oscilação de emoções, relacionamentos, etc.}
\end{EntryWithPhonetic}

\begin{EntryWithPhonetic}{起劲}{qi3jin4}{10,7}{⾛,⼒}[HSK 7-9]
  \definition{adj.}{vigoroso; animado; entusiasmado; enérgico; descreve um alto nível de entusiasmo por fazer algo}
  \seealsoref{起劲儿}{qi3jin4r5}
\end{EntryWithPhonetic}

\begin{EntryWithPhonetic}{起劲儿}{qi3jin4r5}{10,7,2}{⾛,⼒,⼉}
  \definition{v.}{ser enérgico}
\end{EntryWithPhonetic}

\begin{EntryWithPhonetic}{起来}{qi3/lai2}{10,7}{⾛,⽊}[HSK 1]
  \definition{v.+compl.}{levantar-se; passar de posições como deitado, sentado ou ajoelhado para ficar em pé | levantar-se; sair da cama | levantar-se; revoltar-se; rebelar-se; refere-se a ascensão, surgimento, levantamento, etc.}
  \seeref{qi3lai5}
  \seeref{qi5lai2}
\end{EntryWithPhonetic}

\begin{EntryWithPhonetic}{起来}{qi3lai5}{10,7}{⾛,⽊}
  \definition{v.aux.}{usado depois de um verbo para indicar movimento ascendente}
  \seeref{qi3/lai2}
  \seeref{qi5lai2}
\end{EntryWithPhonetic}

\begin{EntryWithPhonetic}{起码}{qi3ma3}{10,8}{⾛,⽯}[HSK 5]
  \definition{adj.}{mínimo; elementar; rudimentar}
  \definition{adv.}{mínimamente; pelo menos;}
\end{EntryWithPhonetic}

\begin{EntryWithPhonetic}{起跑线}{qi3pao3xian4}{10,12,8}{⾛,⾜,⽷}[HSK 7-9]
  \definition{s.}{linha de partida (em uma corrida de revezamento); a linha de largada (de uma corrida); a linha de partida em provas de atletismo; metaforicamente, o ponto de partida para o trabalho, estudo, etc.}
\end{EntryWithPhonetic}

\begin{EntryWithPhonetic}{起诉}{qi3 su4}{10,7}{⾛,⾔}[HSK 6]
  \definition{v.}{processar; entrar com uma ação judicial}
\end{EntryWithPhonetic}

\begin{EntryWithPhonetic}{起跳}{qi3tiao4}{10,13}{⾛,⾜}
  \definition{v.}{(atletismo) decolar (no início de um salto) | (de preço, salário, etc.) começar (de um determinado nível)}
\end{EntryWithPhonetic}

\begin{EntryWithPhonetic}{起源}{qi3yuan2}{10,13}{⾛,⽔}[HSK 7-9]
  \definition{s.}{a origem; a origem mais remota das coisas}
  \definition{v.}{originar-se; ter origem em; começar a acontecer; começar a aparecer}
\end{EntryWithPhonetic}

%%%%%%%%%% 气 %%%%%%%%%%
\subsection*{气}\addcontentsline{loh}{figure}{气 \dpy{qi4}}

\begin{EntryWithPhonetic}{气}{qi4}{4}{⽓}[HSK 2][Kangxi 84]
  \definition*{s.}{Sobrenome: Qi}
  \definition[口]{s.}{gás; gás em geral | ar; especificamente, o ar | respiração | clima; refere-se a fenômenos naturais como sol, chuva, frio e calor | cheiro; odor; o cheiro que o nariz sente | ânimo; moral; estado mental | ares; estilo; maneiras; refere-se ao estilo e aos hábitos de uma pessoa | raiva; irritação; aborrecimento; sentimento de irritação | energia vital; energia da vida; na medicina tradicional chinesa refere-se às substâncias sutis que circulam no corpo humano e permitem que os vários órgãos funcionem normalmente | certos sintomas (de doenças); na medicina tradicional chinesa refere-se a um determinado quadro clínico}
  \definition{v.}{ficar com raiva; ficar furioso; ficar irritado | irritar; enfurecer; deixar com raiva | ser intimidado; sofrer injustiça; intimidar}
\end{EntryWithPhonetic}

\begin{EntryWithPhonetic}{气氛}{qi4fen1}{4,8}{⽓,⽓}[HSK 6]
  \definition{s.}{atmosfera; sensação circundante; uma certa emoção ou cena que existe em um determinado ambiente e pode fazer as pessoas sentirem}
\end{EntryWithPhonetic}

\begin{EntryWithPhonetic}{气愤}{qi4fen4}{4,12}{⽓,⼼}[HSK 7-9]
  \definition{adj.}{furioso; indignado; irritado e ressentido}
\end{EntryWithPhonetic}

\begin{EntryWithPhonetic}{气管}{qi4guan3}{4,14}{⽓,⽵}[HSK 7-9]
  \definition[个,条,根]{s.}{traqueia; tubo respiratório}
\end{EntryWithPhonetic}

\begin{EntryWithPhonetic}{气候}{qi4hou4}{4,10}{⽓,⼈}[HSK 3]
  \definition[种]{s.}{clima; tempo; condições meteorológicas gerais obtidas após muitos anos de observação em uma determinada região, estão relacionadas com correntes de ar, latitude, altitude acima do nível do mar, relevo, etc. | tendência; situação; metáfora do ambiente social, de uma determinada tendência | resultado; influência; conquista; realização; metáfora para algum tipo de resultado, conquista, influência significativa ou potencial de desenvolvimento}
\end{EntryWithPhonetic}

\begin{EntryWithPhonetic}{气馁}{qi4nei3}{4,10}{⽓,⾷}[HSK 7-9]
  \definition{adj.}{desencorajado}
\end{EntryWithPhonetic}

\begin{EntryWithPhonetic}{气派}{qi4pai4}{4,9}{⽓,⽔}[HSK 7-9]
  \definition{adj.}{de maneira marcante; de ​​aparência impressionante; presunçoso; arrogante}
  \definition{s.}{ar digno; maneira imponente; refere-se à atitude de uma pessoa, ao ímpeto ou à aura emanada por certas coisas}
\end{EntryWithPhonetic}

\begin{EntryWithPhonetic}{气泡}{qi4pao4}{4,8}{⽓,⽔}[HSK 7-9]
  \definition{s.}{bolha; bolha de ar; um corpo esférico ou hemisférico formado por um gás que ocupa um determinado espaço dentro ou abaixo da superfície de um sólido ou líquido}
\end{EntryWithPhonetic}

\begin{EntryWithPhonetic}{气魄}{qi4po4}{4,14}{⽓,⿁}[HSK 7-9]
  \definition{s.}{ousadia de visão; amplitude de espírito; audácia; coragem para fazer as coisas | maneira imponente; impulso}
\end{EntryWithPhonetic}

\begin{EntryWithPhonetic}{气球}{qi4qiu2}{4,11}{⽓,⽟}[HSK 4]
  \definition[个,只]{s.}{balão; bolas feitas de borracha, plástico, etc., que podem ser aumentadas soprando ar nelas e podem ser usadas como brinquedos, decorações ou meios de transporte}
\end{EntryWithPhonetic}

\begin{EntryWithPhonetic}{气势}{qi4shi4}{4,8}{⽓,⼒}[HSK 7-9]
  \definition{s.}{ímpeto; maneira imponente; um certo poder e posição exibidos por (uma pessoa ou coisa)}
\end{EntryWithPhonetic}

\begin{EntryWithPhonetic}{气体}{qi4 ti3}{4,7}{⽓,⼈}[HSK 5]
  \definition[种,瓶,升]{s.}{gás; não têm forma nem volume definidos e podem fluir.; o ar, o oxigênio, o gás metano e outros são gases}
\end{EntryWithPhonetic}

\begin{EntryWithPhonetic}{气味}{qi4wei4}{4,8}{⽓,⼝}[HSK 7-9]
  \definition[种,股]{s.}{cheiro; o cheiro que impregna o ar | caráter e hábito (geralmente ruins); metaforicamente, refere-se ao temperamento; gosto (geralmente usado em sentido pejorativo)}
\end{EntryWithPhonetic}

\begin{EntryWithPhonetic}{气温}{qi4 wen1}{4,12}{⽓,⽔}[HSK 2]
  \definition[个]{s.}{temperatura do ar}
\end{EntryWithPhonetic}

\begin{EntryWithPhonetic}{气息}{qi4xi1}{4,10}{⽓,⼼}[HSK 7-9]
  \definition[种,丝,缕,股]{s.}{respiração; o ar que entra e sai quando respiramos | cheiro; estilo; metaforicamente, refere-se a uma forte sensação ou característica que algo transmite}
\end{EntryWithPhonetic}

\begin{EntryWithPhonetic}{气象}{qi4xiang4}{4,11}{⽓,⾗}[HSK 5]
  \definition[种,派]{s.}{fenômenos meteorológicos; condições e fenômenos atmosféricos, como vento, relâmpagos, trovões, geadas, neve, etc. | meteorologia | situação; atmosfera; cena; circunstância | maneira imponente}
\end{EntryWithPhonetic}

\begin{EntryWithPhonetic}{气质}{qi4zhi4}{4,8}{⽓,⾙}[HSK 7-9]
  \definition{s.}{temperamento; disposição; traços de personalidade | qualidades; componentes; refere-se aos traços de personalidade e estilo relativamente estáveis ​​de uma pessoa}
\end{EntryWithPhonetic}

%%%%%%%%%% 迄 %%%%%%%%%%
\subsection*{迄}\addcontentsline{loh}{figure}{迄 \dpy{qi4}}

\begin{EntryWithPhonetic}{迄}{qi4}{6}{⾡}
  \definition{adv.}{até agora; ao longo de todo o processo; sempre; constantemente; usado antes de 未 ou 无}
  \definition{prep.}{até; para}
  \seealsoref{未}{wei4}
  \seealsoref{无}{wu2}
\end{EntryWithPhonetic}

\begin{EntryWithPhonetic}{迄今}{qi4jin1}{6,4}{⾡,⼈}[HSK 7-9]
  \definition{adv.}{até agora; até o momento; até hoje}
\end{EntryWithPhonetic}

\begin{EntryWithPhonetic}{迄今为止}{qi4jin1-wei2zhi3}{6,4,4,4}{⾡,⼈,⼂,⽌}[HSK 7-9]
  \definition{adv.}{até hoje; até agora; até o momento}
\end{EntryWithPhonetic}

%%%%%%%%%% 汽 %%%%%%%%%%
\subsection*{汽}\addcontentsline{loh}{figure}{汽 \dpy{qi4}}

\begin{EntryWithPhonetic}{汽}{qi4}{7}{⽔}
  \definition{s.}{vapor | vaporizador}
\end{EntryWithPhonetic}

\begin{EntryWithPhonetic}{汽车}{qi4 che1}{7,4}{⽔,⾞}[HSK 1]
  \definition[辆,种,款]{s.}{automóvel; carro; veículo motorizado; veículo movido a motor de combustão interna, que circula principalmente em rodovias ou ruas, geralmente com quatro ou mais pneus de borracha, usado para transportar pessoas ou mercadorias}
\end{EntryWithPhonetic}

\begin{EntryWithPhonetic}{汽水}{qi4 shui3}{7,4}{⽔,⽔}[HSK 4]
  \definition[罐,杯,瓶,听,口]{s.}{refrigerante; refrigerante gaseificado; bebida refrescante, feita com a pressão de dióxido de carbono para dissolver na água e adicionar açúcar, suco de frutas, especiarias etc.}
\end{EntryWithPhonetic}

\begin{EntryWithPhonetic}{汽油}{qi4you2}{7,8}{⽔,⽔}[HSK 4]
  \definition[桶,升,吨]{s.}{gasolina; mistura líquida de hidrocarbonetos com volatilidade e combustibilidade, que é usada como combustível a partir do fracionamento ou craqueamento do petróleo}
\end{EntryWithPhonetic}

%%%%%%%%%% 妻 %%%%%%%%%%
\subsection*{妻}\addcontentsline{loh}{figure}{妻 \dpy{qi4}}

\begin{EntryWithPhonetic}{妻}{qi4}{8}{⼥}
  \definition{v.}{casar uma mulher com (alguém)}
  \seeref{qi1}
\end{EntryWithPhonetic}

%%%%%%%%%% 契 %%%%%%%%%%
\subsection*{契}\addcontentsline{loh}{figure}{契 \dpy{qi4}}

\begin{EntryWithPhonetic}{契}{qi4}{9}{⼤}
  \definition{s.}{contrato; escritura | Arcaico: personagens esculpidos}
  \definition{v.}{Literário: gravar; esculpir | concordar; dar-se bem}
  \seeref{xie4}
\end{EntryWithPhonetic}

\begin{EntryWithPhonetic}{契机}{qi4ji1}{9,6}{⼤,⽊}[HSK 7-9]
  \definition{s.}{oportunidade; ponto de virada; a chave para mudar as coisas numa direção favorável}
\end{EntryWithPhonetic}

\begin{EntryWithPhonetic}{契约}{qi4yue1}{9,6}{⼤,⽷}[HSK 7-9]
  \definition{s.}{escritura; carta; contrato; documentos comprovativos de vendas, hipotecas, arrendamentos, etc.}
\end{EntryWithPhonetic}

%%%%%%%%%% 器 %%%%%%%%%%
\subsection*{器}\addcontentsline{loh}{figure}{器 \dpy{qi4}}

\begin{EntryWithPhonetic}{器}{qi4}{16}{⼝}
  \definition[台]{s.}{dispositivo | ferramenta | utensílio}
\end{EntryWithPhonetic}

\begin{EntryWithPhonetic}{器材}{qi4cai2}{16,7}{⼝,⽊}[HSK 7-9]
  \definition[种,批,套,件]{s.}{material; aparelho; equipamento; ferramentas e materiais}
\end{EntryWithPhonetic}

\begin{EntryWithPhonetic}{器官}{qi4guan1}{16,8}{⼝,⼧}[HSK 4]
  \definition[个,种]{s.}{órgão; aparelho; parte de um organismo que consiste em vários tipos de tecidos celulares que podem desempenhar uma função fisiológica separada}
\end{EntryWithPhonetic}

\begin{EntryWithPhonetic}{器械}{qi4xie4}{16,11}{⼝,⽊}[HSK 7-9]
  \definition[种,批,些]{s.}{aparelho; dispositivo; instrumento; equipamento; instrumentos com finalidades especiais ou com construção relativamente precisa | arma; armamento; instrumentos e dispositivos usados ​​diretamente para matar pessoal inimigo e destruir instalações de combate inimigas, como facas, armas de fogo, artilharia e mísseis}
\end{EntryWithPhonetic}

%%%%%%%%%% 起 %%%%%%%%%%
\subsection*{起}\addcontentsline{loh}{figure}{起 \dpy{qi5}}

\begin{EntryWithPhonetic}{起来}{qi5lai2}{10,7}{⾛,⽊}
  \definition{v.}{descrever resultados, retratar comportamentos, transmitir movimento}
  \seeref{qi3/lai2}
  \seeref{qi3lai5}
\end{EntryWithPhonetic}

%%%%%%%%%% 掐 %%%%%%%%%%
\subsection*{掐}\addcontentsline{loh}{figure}{掐 \dpy{qia1}}

\begin{EntryWithPhonetic}{掐}{qia1}{11}{⼿}[HSK 7-9]
  \definition{s.}{Dialeto: um punhado, maço, pitada, etc. de}
  \definition{v.}{beliscar; dar uma mordidinha | agarrar}
  \seealsoref{掐儿}{qia1r5}
\end{EntryWithPhonetic}

\begin{EntryWithPhonetic}{掐儿}{qia1r5}{11,2}{⼿,⼉}
  \definition{s.}{Dialeto: um punhado, maço, pitada, etc. de}
\end{EntryWithPhonetic}

%%%%%%%%%% 卡 %%%%%%%%%%
\subsection*{卡}\addcontentsline{loh}{figure}{卡 \dpy{qia3}}

\begin{EntryWithPhonetic}{卡}{qia3}{5}{⼘}[HSK 7-9]
  \definition*{s.}{Sobrenome: Qia}
  \definition[张,片]{s.}{clipe; prendedor; pinça; utensílio para prender objetos | posto de controle; posto de guarda ou posto de controle localizado em vias de comunicação importantes ou em locais com terreno acidentado}
  \definition{v.}{encravar; ficar preso; impedir de se mover | parar; controlar; impedir | pressionar firmemente com a palma da mão}
  \seeref{ka3}
\end{EntryWithPhonetic}

\begin{EntryWithPhonetic}{卡子}{qia3zi5}{5,3}{⼘,⼦}[HSK 7-9]
  \definition[个,种,把]{s.}{presilha; grampo de cabelo; prendedor; ferramenta de fixação | posto de controle; postos de controle ou áreas de fiscalização estabelecidos para fins de arrecadação de impostos ou segurança}
\end{EntryWithPhonetic}

%%%%%%%%%% 恰 %%%%%%%%%%
\subsection*{恰}\addcontentsline{loh}{figure}{恰 \dpy{qia4}}

\begin{EntryWithPhonetic}{恰}{qia4}{9}{⼼}
  \definition{adv.}{exatamente | apenas}
\end{EntryWithPhonetic}

\begin{EntryWithPhonetic}{恰当}{qia4dang4}{9,6}{⼼,⼹}[HSK 6]
  \definition{adj.}{adequado; apropriado; conveniente; apropriado; a linguagem ou abordagem é muito apropriada}
\end{EntryWithPhonetic}

\begin{EntryWithPhonetic}{恰到好处}{qia4dao4-hao3chu4}{9,8,6,5}{⼼,⼑,⼥,⼡}[HSK 7-9]
  \definition{expr.}{``Na medida certa.''; perfeito (para o propósito ou ocasião); significa que as palavras e ações de alguém atingiram o ponto mais apropriado}
\end{EntryWithPhonetic}

\begin{EntryWithPhonetic}{恰好}{qia4 hao3}{9,6}{⼼,⼥}[HSK 6]
  \definition{adv.}{na medida certa; como a sorte quis}
\end{EntryWithPhonetic}

\begin{EntryWithPhonetic}{恰恰}{qia4 qia4}{9,9}{⼼,⼼}[HSK 6]
  \definition{adv.}{justamente; exatamente; precisamente; bem na hora}
\end{EntryWithPhonetic}

\begin{EntryWithPhonetic}{恰恰相反}{qia4qia4 xiang1fan3}{9,9,9,4}{⼼,⼼,⽬,⼜}[HSK 7-9]
  \definition{expr.}{pelo contrário}
\end{EntryWithPhonetic}

\begin{EntryWithPhonetic}{恰巧}{qia4qiao3}{9,5}{⼼,⼯}[HSK 7-9]
  \definition{adv.}{por acaso; felizmente ou infelizmente; perfeitamente; por coincidência}
\end{EntryWithPhonetic}

\begin{EntryWithPhonetic}{恰如其分}{qia4ru2-qi2fen4}{9,6,8,4}{⼼,⼥,⼋,⼑}[HSK 7-9]
  \definition{expr.}{``Na medida certa.''; apropriado; agir ou falar de maneira diplomática}
\end{EntryWithPhonetic}

%%%%%%%%%% 洽 %%%%%%%%%%
\subsection*{洽}\addcontentsline{loh}{figure}{洽 \dpy{qia4}}

\begin{EntryWithPhonetic}{洽}{qia4}{9}{⽔}
  \definition{adj.}{em harmonia; em acordo | extenso; amplo}
  \definition{v.}{consultar; combinar com}
\end{EntryWithPhonetic}

\begin{EntryWithPhonetic}{洽谈}{qia4tan2}{9,10}{⽔,⾔}[HSK 7-9]
  \definition{v.}{negociar; negociação e consulta geralmente se referem às conversas ou discussões realizadas em atividades comerciais relacionadas a negócios, transações de mercadorias e compra e venda}
\end{EntryWithPhonetic}

%%%%%%%%%% 千 %%%%%%%%%%
\subsection*{千}\addcontentsline{loh}{figure}{千 \dpy{qian1}}

\begin{EntryWithPhonetic}{千}{qian1}{3}{⼗}[HSK 2]
  \definition*{s.}{Sobrenome: Qian}
  \definition{num.}{mil; 1.000; 1000 | a grande quantidade de; um grande número de}
\end{EntryWithPhonetic}

\begin{EntryWithPhonetic}{千变万化}{qian1bian4-wan4hua4}{3,8,3,4}{⼗,⼜,⼀,⼔}[HSK 7-9]
  \definition{expr.}{``Sempre em mudança.''; as miríades de mudanças; mudança caleidoscópica; mudanças intermináveis; em constante transformação; ser infinito em variedade; mudanças infinitas; em constante mudança}
\end{EntryWithPhonetic}

\begin{EntryWithPhonetic}{千方百计}{qian1fang1-bai3ji4}{3,4,6,4}{⼗,⽅,⽩,⾔}[HSK 7-9]
  \definition{expr.}{por todos os meios; fazer tudo o que for possível; descreve alguém que esgotou todos os meios ou métodos}
\end{EntryWithPhonetic}

\begin{EntryWithPhonetic}{千古}{qian1gu3}{3,5}{⼗,⼝}
  \definition{adv.}{por toda a eternidade | em todas as idades}
  \definition{s.}{eternidade (usada em um dístico elegíaco, coroa de flores, etc., dedicada aos mortos)}
\end{EntryWithPhonetic}

\begin{EntryWithPhonetic}{千家万户}{qian1jia1-wan4hu4}{3,10,3,4}{⼗,⼧,⼀,⼾}[HSK 7-9]
  \definition{expr.}{``Milhares de famílias.''; inúmeras famílias; todas as famílias}
\end{EntryWithPhonetic}

\begin{EntryWithPhonetic}{千军万马}{qian1jun1-wan4ma3}{3,6,3,3}{⼗,⼍,⼀,⾺}[HSK 7-9]
  \definition{expr.}{``Milhares de soldados.''; milhares e milhares de homens e cavalos; um exército poderoso; uma força imensa; todos os cavalos do rei e todos os homens do rei; exército magnífico com milhares de homens e cavalos; demonstração impressionante de força humana}
\end{EntryWithPhonetic}

\begin{EntryWithPhonetic}{千钧一发}{qian1jun1-yi1fa4}{3,9,1,5}{⼗,⾦,⼀,⼜}[HSK 7-9]
  \definition{expr.}{``Por pouco não deu certo.''; cem pesos pendurados por um fio; em perigo iminente; uma questão de vida ou morte}
\end{EntryWithPhonetic}

\begin{EntryWithPhonetic}{千克}{qian1 ke4}{3,7}{⼗,⼗}[HSK 2]
  \definition{clas.}{kg; quilo; quilograma; 1 quilograma equivale a 1.000 gramas, ou 2 jin (斤)}
  \seealsoref{斤}{jin1}
\end{EntryWithPhonetic}

\begin{EntryWithPhonetic}{千年}{qian1nian2}{3,6}{⼗,⼲}
  \definition{s.}{milênio}
\end{EntryWithPhonetic}

\begin{EntryWithPhonetic}{千千万万}{qian1qian1wan4wan4}{3,3,3,3}{⼗,⼗,⼀,⼀}
  \definition{num.}{inumerável | números incontáveis | milhares e milhares}
\end{EntryWithPhonetic}

\begin{EntryWithPhonetic}{千万}{qian1wan4}{3,3}{⼗,⼀}[HSK 3]
  \definition{adv.}{(usado para indicar desejos fortes) por todos os meios; sob quaisquer circunstâncias; expressa uma exortação sincera, equivalente a 务必}
  \definition{num.}{dez milhões; 10.000.000; 1000.0000; milhões e milhões; um número aproximado, indicando um grande número}
  \seealsoref{务必}{wu4bi4}
\end{EntryWithPhonetic}

%%%%%%%%%% 迁 %%%%%%%%%%
\subsection*{迁}\addcontentsline{loh}{figure}{迁 \dpy{qian1}}

\begin{EntryWithPhonetic}{迁}{qian1}{6}{⾡}[HSK 7-9]
  \definition{v.}{mover algo para algum lugar; migrar | mudar}
\end{EntryWithPhonetic}

\begin{EntryWithPhonetic}{迁就}{qian1jiu4}{6,12}{⾡,⼪}[HSK 7-9]
  \definition{v.}{ceder a; acomodar-se a; atender aos interesses dos outros}
\end{EntryWithPhonetic}

\begin{EntryWithPhonetic}{迁移}{qian1yi2}{6,11}{⾡,⽲}[HSK 7-9]
  \definition{v.}{mover; migrar; mudar-se de sua localização original para outro lugar}
\end{EntryWithPhonetic}

%%%%%%%%%% 牵 %%%%%%%%%%
\subsection*{牵}\addcontentsline{loh}{figure}{牵 \dpy{qian1}}

\begin{EntryWithPhonetic}{牵}{qian1}{9}{⽜}[HSK 6]
  \definition{v.}{conduzir (segurando a mão, o cabresto, etc.); puxar | envolver-se | sentir falta; preocupar-se com | controlar; restringir; ser retido; ser constrangido}
\end{EntryWithPhonetic}

\begin{EntryWithPhonetic}{牵扯}{qian1che3}{9,7}{⽜,⼿}[HSK 7-9]
  \definition{v.}{envolver; arrastar para}
\end{EntryWithPhonetic}

\begin{EntryWithPhonetic}{牵挂}{qian1gua4}{9,9}{⽜,⼿}[HSK 7-9]
  \definition{v.}{preocupar-se; estar preocupado; perder}
\end{EntryWithPhonetic}

\begin{EntryWithPhonetic}{牵涉}{qian1she4}{9,10}{⽜,⽔}[HSK 7-9]
  \definition{v.}{preocupar-se com; envolver; arrastar para; uma coisa está relacionada a outras coisas ou pessoas}
\end{EntryWithPhonetic}

\begin{EntryWithPhonetic}{牵头}{qian1/tou2}{9,5}{⽜,⼤}[HSK 7-9]
  \definition{v.+compl.}{intermediar (por exemplo: casamenteiro) | coordenar (uma operação combinada) | conduzir (um animal pela cabeça) | mediar | assumir a liderança}
\end{EntryWithPhonetic}

\begin{EntryWithPhonetic}{牵制}{qian1zhi4}{9,8}{⽜,⼑}[HSK 7-9]
  \definition{v.}{conter; imobilizar; amarrar; restringir ou impedir a livre circulação (frequentemente usado em contextos militares)}
\end{EntryWithPhonetic}

%%%%%%%%%% 铅 %%%%%%%%%%
\subsection*{铅}\addcontentsline{loh}{figure}{铅 \dpy{qian1}}

\begin{EntryWithPhonetic}{铅}{qian1}{10}{⾦}[HSK 7-9]
  \definition[根,盒]{s.}{chumbo (Pb) | grafite (em um lápis); grafite preta}
\end{EntryWithPhonetic}

\begin{EntryWithPhonetic}{铅笔}{qian1bi3}{10,10}{⾦,⽵}[HSK 6]
  \definition[支,盒,种,枝,杆]{s.}{lápis; canetas com pontas de grafite ou argila pigmentada}
\end{EntryWithPhonetic}

%%%%%%%%%% 谦 %%%%%%%%%%
\subsection*{谦}\addcontentsline{loh}{figure}{谦 \dpy{qian1}}

\begin{EntryWithPhonetic}{谦}{qian1}{12}{⾔}
  \definition*{s.}{Sobrenome: Qian}
  \definition{adj.}{modesto}
  \definition{s.}{modéstia}
\end{EntryWithPhonetic}

\begin{EntryWithPhonetic}{谦虚}{qian1xu1}{12,11}{⾔,⾌}[HSK 6]
  \definition{adj.}{modesto; não se orgulhe de suas próprias conquistas e esteja disposto a aceitar críticas e opiniões de outras pessoas}
  \definition{v.}{falar modestamente; quando recebo elogios e cumprimentos de outras pessoas, sinto que não sou tão bom}
\end{EntryWithPhonetic}

\begin{EntryWithPhonetic}{谦逊}{qian1xun4}{12,9}{⾔,⾡}[HSK 7-9]
  \definition{adj.}{humilde; modesto; despretensioso; sem afetação}
\end{EntryWithPhonetic}

%%%%%%%%%% 签 %%%%%%%%%%
\subsection*{签}\addcontentsline{loh}{figure}{签 \dpy{qian1}}

\begin{EntryWithPhonetic}{签}{qian1}{13}{⽵}[HSK 5,7-9]
  \definition[个,根,支]{s.}{tiras de bambu usadas para adivinhação ou sorteio; pPequenas tiras de bambu ou varas finas com caracteres e símbolos gravados, usadas para adivinhação, jogos de azar ou como fichas para contagem, etc. | etiqueta; adesivo; pequena tira usada como marca | um pedaço fino e pontiagudo de bambu ou madeira; pequeno bastão pontiagudo}
  \definition{v.}{assinar; autografar; escrever o nome, palavras ou fazer marcas em documentos ou recibos | fazer comentários breves em um documento; escrever brevemente (pontos principais ou opiniões) | (em costura) alinhavar; costura grosseira}
\end{EntryWithPhonetic}

\begin{EntryWithPhonetic}{签订}{qian1 ding4}{13,4}{⽵,⾔}[HSK 5]
  \definition{v.}{concluir e assinar (um tratado, etc.)}
\end{EntryWithPhonetic}

\begin{EntryWithPhonetic}{签名}{qian1/ming2}{13,6}{⽵,⼝}[HSK 5]
  \definition[个,次]{s.}{assinatura; autógrafo}
  \definition{v.+compl.}{assinar o próprio nome; autografar; escrever seu nome para indicar concordância, apoio ou homenagem, etc.}
\end{EntryWithPhonetic}

\begin{EntryWithPhonetic}{签署}{qian1shu3}{13,13}{⽵,⽹}[HSK 7-9]
  \definition{v.}{assinar; assinar formalmente e apor o próprio nome em documentos e tratados importantes}
\end{EntryWithPhonetic}

\begin{EntryWithPhonetic}{签约}{qian1 yue1}{13,6}{⽵,⽷}[HSK 5]
  \definition{v.}{assinar um contrato; assinar contratos e tratados, frequentemente utilizado no trabalho e em cooperações comerciais}
\end{EntryWithPhonetic}

\begin{EntryWithPhonetic}{签证}{qian1zheng4}{13,7}{⽵,⾔}[HSK 5]
  \definition[张,个,份]{s.}{visto; visto de entrada em um país}
\end{EntryWithPhonetic}

\begin{EntryWithPhonetic}{签字}{qian1 zi4}{13,6}{⽵,⼦}[HSK 5]
  \definition{v.}{assinar; colocar a assinatura; escrever seu nome à mão em documentos, recibos, etc., para demonstrar responsabilidade}
\end{EntryWithPhonetic}

%%%%%%%%%% 前 %%%%%%%%%%
\subsection*{前}\addcontentsline{loh}{figure}{前 \dpy{qian2}}

\begin{EntryWithPhonetic}{前}{qian2}{9}{⼑}[HSK 1]
  \definition*{s.}{Sobrenome: Qian}
  \definition{s.}{frente | futuro; perspectiva | atrás; antes; mais cedo do que uma coisa ou um momento | à frente; para a frente; na parte frontal (referindo-se ao espaço, em oposição a 后) | precedente; antes que algo aconteça | antigo; antigamente | topo; primeiro; primeiro na ordem | frente; campo de batalha | A.C. (Antes de~Cristo)}[前293年===293 a.C.]
  \definition{v.}{seguir em frente; ir em frente}
  \seealsoref{公元}{gong1yuan2}
  \seealsoref{后}{hou4}
\end{EntryWithPhonetic}

\begin{EntryWithPhonetic}{前辈}{qian2bei4}{9,12}{⼑,⾞}[HSK 7-9]
  \definition{s.}{idoso; sênior; geração mais velha; refere-se a pessoas mais velhas ou com mais experiência na mesma indústria ou área de atuação | predecessor; antepassados; ancestrais; a geração anterior ou gerações anteriores, também se referindo aos ancestrais}
\end{EntryWithPhonetic}

\begin{EntryWithPhonetic}{前边}{qian2 bian5}{9,5}{⼑,⾡}[HSK 1]
  \definition{adv.}{à frente; na frente}
\end{EntryWithPhonetic}

\begin{EntryWithPhonetic}{前不久}{qian2bu4jiu3}{9,4,3}{⼑,⼀,⼃}[HSK 7-9]
  \definition{adv.}{não faz muito tempo | não muito tempo antes}
\end{EntryWithPhonetic}

\begin{EntryWithPhonetic}{前方}{qian2 fang1}{9,4}{⼑,⽅}[HSK 6]
  \definition{s.}{frente; o espaço à frente; a direção voltada para a frente; a frente (em oposição à 后方) | linha de frente; frente de batalha; áreas onde os exércitos de ambos os lados estão se aproximando ou lutando}
  \seealsoref{后方}{hou4 fang1}
\end{EntryWithPhonetic}

\begin{EntryWithPhonetic}{前赴后继}{qian2fu4-hou4ji4}{9,9,6,10}{⼑,⾛,⼝,⽷}[HSK 7-9]
  \definition{expr.}{``Um após o outro.''; avançar destemidamente em onda após onda; os que estão na frente sobem, e os que estão atrás os seguem, o que demonstra um espírito de progresso entusiasmado e contínuo}
\end{EntryWithPhonetic}

\begin{EntryWithPhonetic}{前后}{qian2 hou4}{9,6}{⼑,⼝}[HSK 3]
  \definition{s.}{em volta; sobre; um período de tempo ligeiramente anterior ou posterior a um horário específico| do início ao fim; refere-se ao período de tempo do início ao fim de algo | frente e verso; na frente e atrás de algo}
\end{EntryWithPhonetic}

\begin{EntryWithPhonetic}{前进}{qian2 jin4}{9,7}{⼑,⾡}[HSK 3]
  \definition{v.}{marchar; avançar; para ir em frente; seguir em frente; geralmente se refere ao desenvolvimento futuro}
\end{EntryWithPhonetic}

\begin{EntryWithPhonetic}{前景}{qian2jing3}{9,12}{⼑,⽇}[HSK 5]
  \definition{s.}{primeiro plano (de uma vista, imagem, foto, etc.); as imagens que parecem mais próximas do espectador em pinturas, palcos e telas | vista; perspectiva; prospecto; ponto de vista; situações que podem ocorrer no trabalho, na carreira, etc.}
\end{EntryWithPhonetic}

\begin{EntryWithPhonetic}{前来}{qian2 lai2}{9,7}{⼑,⽊}[HSK 6]
  \definition{v.}{vir; em direção à localização e direção do falante}
\end{EntryWithPhonetic}

\begin{EntryWithPhonetic}{前面}{qian2mian4}{9,9}{⼑,⾯}[HSK 3]
  \definition{s.}{frente; a parte frontal do espaço ou posição | parte anterior; acima; a parte que vem primeiro na ordem; a parte de um artigo ou discurso que precede a narração atual}
\end{EntryWithPhonetic}

\begin{EntryWithPhonetic}{前年}{qian2 nian2}{9,6}{⼑,⼲}[HSK 2]
  \definition{adv.}{há dois anos; dois anos atrás}
\end{EntryWithPhonetic}

\begin{EntryWithPhonetic}{前期}{qian2qi1}{9,12}{⼑,⽉}[HSK 7-9]
  \definition{s.}{estágio inicial; primeiros dias; prófase; a etapa anterior de um determinado período}
\end{EntryWithPhonetic}

\begin{EntryWithPhonetic}{前任}{qian2ren4}{9,6}{⼑,⼈}[HSK 7-9]
  \definition[个,位,名]{s.}{predecessor; a pessoa que ocupava este cargo anteriormente}
\end{EntryWithPhonetic}

\begin{EntryWithPhonetic}{前所未有}{qian2suo3wei4you3}{9,8,5,6}{⼑,⼾,⽊,⽉}[HSK 7-9]
  \definition{expr.}{``Sem precedentes.''; nunca existiu antes; até então desconhecido; nunca antes na história}
\end{EntryWithPhonetic}

\begin{EntryWithPhonetic}{前台}{qian2tai2}{9,5}{⼑,⼝}[HSK 7-9]
  \definition[个]{s.}{proscênio; trabalho diverso para uma apresentação; refere-se a diversas tarefas administrativas relacionadas ao desempenho | palco; frente do palco; a parte da frente do palco, voltada para a plateia, é onde os atores se apresentam; geralmente, ela é separada da área dos bastidores por cortinas ou outras barreiras | (em um hotel) balcão de recepção; balcões de atendimento em restaurantes, casas noturnas, hotéis, etc., responsáveis ​​pela recepção, cadastro e pagamento | lugar público; referindo-se metaforicamente a uma ocasião pública (frequentemente usado em sentido pejorativo)}
\end{EntryWithPhonetic}

\begin{EntryWithPhonetic}{前提}{qian2ti2}{9,12}{⼑,⼿}[HSK 5]
  \definition[个,项]{s.}{premissa; pressuposto | pré-requisito; pressuposição; condições prévias para que algo aconteça ou se desenvolva}
\end{EntryWithPhonetic}

\begin{EntryWithPhonetic}{前天}{qian2 tian1}{9,4}{⼑,⼤}[HSK 1]
  \definition{adv.}{anteontem; dia anterior a ontem}
\end{EntryWithPhonetic}

\begin{EntryWithPhonetic}{前头}{qian2 tou5}{9,5}{⼑,⼤}[HSK 4]
  \definition{s.}{à frente; na frente; adiante}
\end{EntryWithPhonetic}

\begin{EntryWithPhonetic}{前途}{qian2tu2}{9,10}{⼑,⾡}[HSK 4]
  \definition[片,段,种]{s.}{futuro; perspectiva; prospecto; originalmente, refere-se à jornada à frente, mas, metaforicamente, refere-se ao futuro.}
\end{EntryWithPhonetic}

\begin{EntryWithPhonetic}{前往}{qian2 wang3}{9,8}{⼑,⼻}[HSK 3]
  \definition{v.}{ir para; prosseguir para; partir para; ir em frente}
\end{EntryWithPhonetic}

\begin{EntryWithPhonetic}{前无古人}{qian2wu2gu3ren2}{9,4,5,2}{⼑,⽆,⼝,⼈}[HSK 7-9]
  \definition[位,名,个,些]{expr.}{``Sem precedentes.''; refere-se a algo que nunca foi possuído ou alcançado antes; algo sem precedentes; sem paralelo na história}
\end{EntryWithPhonetic}

\begin{EntryWithPhonetic}{前夕}{qian2xi1}{9,3}{⼑,⼣}[HSK 7-9]
  \definition{s.}{véspera; na noite anterior | a véspera; geralmente se refere ao período de tempo imediatamente anterior à ocorrência de um evento ou ao momento em que um evento está prestes a ocorrer}
\end{EntryWithPhonetic}

\begin{EntryWithPhonetic}{前线}{qian2xian4}{9,8}{⼑,⽷}[HSK 7-9]
  \definition{s.}{linha de frente; frente (oposto à 后方) | frente de batalha; a área onde os dois exércitos se aproximam durante uma batalha (em oposição à 后方)}
  \seealsoref{后方}{hou4 fang1}
\end{EntryWithPhonetic}

\begin{EntryWithPhonetic}{前沿}{qian2yan2}{9,8}{⼑,⽔}[HSK 7-9]
  \definition{s.}{Militar: posição avançada | Figurativo: fronteira na pesquisa científica | fronteira (da ciência, tecnologia etc.) | posto avançado}
\end{EntryWithPhonetic}

\begin{EntryWithPhonetic}{前仰后合}{qian2yang3-hou4he2}{9,6,6,6}{⼑,⼈,⼝,⼝}[HSK 7-9]
  \definition{expr.}{``Inclinando-se para a frente e para trás.''; balançar (com risos); balançar para frente e para trás}
\end{EntryWithPhonetic}

\begin{EntryWithPhonetic}{前者}{qian2zhe3}{9,8}{⼑,⽼}[HSK 7-9]
  \definition{adj.}{antigo; anterior; primeiro; precedente; refere-se à primeira das duas coisas ou pessoas listadas acima (distinguindo-se de 后者)}
  \seealsoref{后者}{hou4zhe3}
\end{EntryWithPhonetic}

%%%%%%%%%% 虔 %%%%%%%%%%
\subsection*{虔}\addcontentsline{loh}{figure}{虔 \dpy{qian2}}

\begin{EntryWithPhonetic}{虔}{qian2}{10}{⾌}
  \definition*{s.}{Sobrenome: Qian}
  \definition{adj.}{piedoso; sincero}
\end{EntryWithPhonetic}

\begin{EntryWithPhonetic}{虔诚}{qian2cheng2}{10,8}{⾌,⾔}[HSK 7-9]
  \definition{adj.}{piedoso; devoto; devotado; respeitoso e sincero (frequentemente referindo-se à religião ou à fé)}
\end{EntryWithPhonetic}

%%%%%%%%%% 钱 %%%%%%%%%%
\subsection*{钱}\addcontentsline{loh}{figure}{钱 \dpy{qian2}}

\begin{EntryWithPhonetic}{钱}{qian2}{10}{⾦}[HSK 1]
  \definition*{s.}{Sobrenome: Qian}
  \definition{clas.}{qian, uma unidade de peso (=5 gramas) | qian, uma unidade de peso (um décimo de um tael 两)}
  \definition[笔]{s.}{dinheiro; riqueza; bens | moeda de cobre; dinheiro | objeto em forma de moeda de cobre | fundo; montante | dinheiro guardado ou gasto para algum fim específico (geralmente se refere a quantias significativas de dinheiro que entram e saem de órgãos públicos, organizações, etc.)}
  \seealsoref{两}{liang3}
\end{EntryWithPhonetic}

\begin{EntryWithPhonetic}{钱包}{qian2 bao1}{10,5}{⾦,⼓}[HSK 1]
  \definition[个]{s.}{carteira; bolsa; bolsa de dinheiro}
\end{EntryWithPhonetic}

\begin{EntryWithPhonetic}{钱财}{qian2cai2}{10,7}{⾦,⾙}[HSK 7-9]
  \definition{s.}{riqueza; dinheiro}
\end{EntryWithPhonetic}

%%%%%%%%%% 钳 %%%%%%%%%%
\subsection*{钳}\addcontentsline{loh}{figure}{钳 \dpy{qian2}}

\begin{EntryWithPhonetic}{钳}{qian2}{10}{⾦}
  \definition{s.}{pinças; alicates; tenazes}
  \definition{v.}{agarrar (com pinças); prender | restringir; limitar}
\end{EntryWithPhonetic}

\begin{EntryWithPhonetic}{钳子}{qian2zi5}{10,3}{⾦,⼦}[HSK 7-9]
  \definition[把,个]{s.}{alicate; tenaz; ferramentas usadas para prender ou cortar coisas}
\end{EntryWithPhonetic}

%%%%%%%%%% 潜 %%%%%%%%%%
\subsection*{潜}\addcontentsline{loh}{figure}{潜 \dpy{qian2}}

\begin{EntryWithPhonetic}{潜}{qian2}{15}{⽔}
  \definition*{s.}{Sobrenome: Qian}
  \definition{adj.}{latente; oculto}
  \definition{adv.}{furtivamente; secretamente; às escondidas}
  \definition{v.}{ir para debaixo d'água; esconder-se debaixo d'água; mergulhar | esconder | vadear (atravessar) na água | enterrar | fugir de casa}
\end{EntryWithPhonetic}

\begin{EntryWithPhonetic}{潜力}{qian2li4}{15,2}{⽔,⼒}[HSK 6]
  \definition{s.}{potencial; potencialidade; capacidade latente; as habilidades e possibilidades de desenvolvimento que as pessoas e as coisas ainda não demonstraram}
\end{EntryWithPhonetic}

\begin{EntryWithPhonetic}{潜能}{qian2neng2}{15,10}{⽔,⾁}[HSK 7-9]
  \definition{s.}{proficiência; potencial de habilidades ou energia inexploradas}
\end{EntryWithPhonetic}

\begin{EntryWithPhonetic}{潜水}{qian2/shui3}{15,4}{⽔,⽔}[HSK 7-9]
  \definition{s.}{água freática; lençol freático; água subterrânea escondida na primeira camada impermeável abaixo do solo}
  \definition{v.+compl.}{mergulhar; ir para debaixo d'água; entrar abaixo da superfície da água}
\end{EntryWithPhonetic}

\begin{EntryWithPhonetic}{潜艇}{qian2ting3}{15,12}{⽔,⾈}[HSK 7-9]
  \definition{s.}{submarino; navios de guerra que se dedicam principalmente a operações de combate subaquáticas, usando torpedos ou mísseis para atacar navios inimigos e alvos costeiros, e que também servem como embarcações de reconhecimento}
\end{EntryWithPhonetic}

\begin{EntryWithPhonetic}{潜移默化}{qian2yi2-mo4hua4}{15,11,16,4}{⽔,⽲,⿊,⼔}[HSK 7-9]
  \definition{expr.}{``Influência sutil.''; de maneiras sutis; influência gradual e imperceptível; influência imperceptível; influência lenta e despercebida; isso se refere a uma mudança nos pensamentos ou no caráter de uma pessoa causada por influência ou influência inconsciente; exercer uma influência sutil no caráter, pensamento, etc. de alguém; influenciar imperceptivelmente}
\end{EntryWithPhonetic}

\begin{EntryWithPhonetic}{潜在}{qian2zai4}{15,6}{⽔,⼟}[HSK 7-9]
  \definition{adj.}{latente; oculto; potencial; ela existe dentro das coisas e não é facilmente descoberta ou detectada}[这个问题有潜在风险。===Essa questão apresenta riscos potenciais.]
\end{EntryWithPhonetic}

%%%%%%%%%% 浅 %%%%%%%%%%
\subsection*{浅}\addcontentsline{loh}{figure}{浅 \dpy{qian3}}

\begin{EntryWithPhonetic}{浅}{qian3}{8}{⽔}[HSK 4]
  \definition{adj.}{raso; superficial;  (em oposição a 深) | fácil; simples; redação, conteúdo, etc. simples e fáceis de entender | superficial; não é profundo em aprendizado, percepção e sabedoria | não próximo; não íntimo; sentimentos não profundos | (cor) claro; pálido;  cor pouco intensa; leve |experiência breve; duração de tempo breve | baixo grau; peso leve; nível baixo}
  \seeref{jian1}
  \seealsoref{深}{shen1}
\end{EntryWithPhonetic}

%%%%%%%%%% 谴 %%%%%%%%%%
\subsection*{谴}\addcontentsline{loh}{figure}{谴 \dpy{qian3}}

\begin{EntryWithPhonetic}{谴}{qian3}{15}{⾔}
  \definition*{s.}{Sobrenome: Qian}
  \definition{s.}{falha | Literário: pecado}
  \definition{v.}{condenar; denunciar; censurar | rebaixar; antigamente, os funcionários eram rebaixados ou exilados}
\end{EntryWithPhonetic}

\begin{EntryWithPhonetic}{谴责}{qian3ze2}{15,8}{⾔,⾙}[HSK 7-9]
  \definition{v.}{culpar; condenar; censurar; denunciar}
\end{EntryWithPhonetic}

%%%%%%%%%% 欠 %%%%%%%%%%
\subsection*{欠}\addcontentsline{loh}{figure}{欠 \dpy{qian4}}

\begin{EntryWithPhonetic}{欠}{qian4}{4}{⽋}[HSK 5][Kangxi 76]
  \definition{v.}{bocejar | levantar ligeiramente (uma parte do corpo) | estar em dívida; estar atrasado com; não devolver o que pediu emprestado a outra pessoa, ou não dar o que deveria ter dado a outra pessoa | faltar; não ser suficiente}
\end{EntryWithPhonetic}

\begin{EntryWithPhonetic}{欠缺}{qian4que1}{4,10}{⽋,⽸}[HSK 7-9]
  \definition{s.}{deficiência; falha}
  \definition{v.}{ter falta de; ser deficiente/carente/inadequado em}
\end{EntryWithPhonetic}

\begin{EntryWithPhonetic}{欠条}{qian4tiao2}{4,7}{⽋,⽊}[HSK 7-9]
  \definition{s.}{uma fatura assinada como reconhecimento de dívida; recibo de dívida | fichas; recibos | certificado de dívida}
\end{EntryWithPhonetic}

%%%%%%%%%% 歉 %%%%%%%%%%
\subsection*{歉}\addcontentsline{loh}{figure}{歉 \dpy{qian4}}

\begin{EntryWithPhonetic}{歉}{qian4}{14}{⽋}
  \definition{adj.}{pobre, ruim (colheita) ruim; baixa produtividade (agrícola)}
  \definition{s.}{pedido de desculpas; apologia | quebra de safra}
  \definition{v.}{pedir desculpas; sentir pena dos outros}
\end{EntryWithPhonetic}

\begin{EntryWithPhonetic}{歉意}{qian4yi4}{14,13}{⽋,⼼}[HSK 7-9]
  \definition{s.}{desculpas; arrependimento; pedido de desculpas}
\end{EntryWithPhonetic}

%%%%%%%%%% 呛 %%%%%%%%%%
\subsection*{呛}\addcontentsline{loh}{figure}{呛 \dpy{qiang1}}

\begin{EntryWithPhonetic}{呛}{qiang1}{7}{⼝}[HSK 7-9]
  \definition{v.}{sufocar; engasgar}
  \seeref{qiang4}
\end{EntryWithPhonetic}

%%%%%%%%%% 抢 %%%%%%%%%%
\subsection*{抢}\addcontentsline{loh}{figure}{抢 \dpy{qiang1}}

\begin{EntryWithPhonetic}{抢}{qiang1}{7}{⼿}
  \definition{prep.}{contra; direção relativa inversa}
  \definition{v.}{bater; tocar}
  \seeref{qiang3}
\end{EntryWithPhonetic}

%%%%%%%%%% 枪 %%%%%%%%%%
\subsection*{枪}\addcontentsline{loh}{figure}{枪 \dpy{qiang1}}

\begin{EntryWithPhonetic}{枪}{qiang1}{8}{⽊}[HSK 5]
  \definition*{s.}{Sobrenome: Qiang}
  \definition[把,杆,支,挺]{s.}{lança | arma; rifle; arma de fogo | uma coisa em forma de arma | enxada; ferramenta para cavar a terra}
  \definition{v.}{escrever artigos ou responder perguntas para outras pessoas}
\end{EntryWithPhonetic}

\begin{EntryWithPhonetic}{枪毙}{qiang1bi4}{8,10}{⽊,⽐}[HSK 7-9]
  \definition{v.}{executar por disparo; matar | Humor figurativo: rejeitar; vetar; recusar (uma proposta, manuscrito, etc.) | matar a tiros}
\end{EntryWithPhonetic}

%%%%%%%%%% 将 %%%%%%%%%%
\subsection*{将}\addcontentsline{loh}{figure}{将 \dpy{qiang1}}

\begin{EntryWithPhonetic}{将}{qiang1}{9}{⼨}
  \definition{v.}{pedir; apelar para}
  \seeref{jiang1}
  \seeref{jiang4}
\end{EntryWithPhonetic}

%%%%%%%%%% 腔 %%%%%%%%%%
\subsection*{腔}\addcontentsline{loh}{figure}{腔 \dpy{qiang1}}

\begin{EntryWithPhonetic}{腔}{qiang1}{12}{⾁}[HSK 7-9]
  \definition{clas.}{Arcaico: utilizado para carcaças de animais abatidos}
  \definition{s.}{cavidade; câmara (em corpos humanos ou animais) | afinação; tom; tom de voz | sotaque (na fala) | discurso | (geralmente no vernáculo antigo para cabra ou ovelha abatida) carcaça}
  \seealsoref{腔儿}{qiang1r5}
\end{EntryWithPhonetic}

\begin{EntryWithPhonetic}{腔儿}{qiang1r5}{12,2}{⾁,⼉}
  \definition{s.}{afinação; tom | sotaque | fala}
\end{EntryWithPhonetic}

%%%%%%%%%% 强 %%%%%%%%%%
\subsection*{强}\addcontentsline{loh}{figure}{强 \dpy{qiang2}}

\begin{EntryWithPhonetic}{强}{qiang2}{12}{⼸}[HSK 3]
  \definition*{s.}{Sobrenome: Qiang}
  \definition{adj.}{forte; poderoso  (em oposição a 弱) | melhor; superior | mais; extra; adicional; um pouco mais que; usado após uma fração ou decimal para indicar que é um pouco maior que o número | resoluto; firme | violento | alto padrão}
  \definition{v.}{fortalecer; tornar forte; tornar poderoso}
  \seeref{jiang4}
  \seeref{qiang3}
  \seealsoref{弱}{ruo4}
\end{EntryWithPhonetic}

\begin{EntryWithPhonetic}{强大}{qiang2 da4}{12,3}{⼸,⼤}[HSK 3]
  \definition{adj.}{forte; poderoso; potente; possante; descreve força forte e grande poder}
\end{EntryWithPhonetic}

\begin{EntryWithPhonetic}{强盗}{qiang2 dao4}{12,11}{⼸,⽫}[HSK 6]
  \definition[个,群,伙,帮]{s.}{ladrão; bandido; uma pessoa que usa violência para confiscar a propriedade de outros; também se refere a uma pessoa ou força que se envolve em comportamento semelhante}
\end{EntryWithPhonetic}

\begin{EntryWithPhonetic}{强调}{qiang2diao4}{12,10}{⼸,⾔}[HSK 3]
  \definition{v.}{salientar; sublinhar; enfatizar; dar ênfase a; vincar}
\end{EntryWithPhonetic}

\begin{EntryWithPhonetic}{强度}{qiang2 du4}{12,9}{⼸,⼴}[HSK 5]
  \definition[个,种]{s.}{intensidade; força | magnitude; rigor; avidez}
\end{EntryWithPhonetic}

\begin{EntryWithPhonetic}{强化}{qiang2 hua4}{12,4}{⼸,⼔}[HSK 6]
  \definition{v.}{intensificar; fortalecer; consolidar; tornar mais forte, melhorar sua habilidade e nível}
\end{EntryWithPhonetic}

\begin{EntryWithPhonetic}{强加}{qiang2jia1}{12,5}{⼸,⼒}[HSK 7-9]
  \definition{v.}{forçar; impor; forçar alguém a aceitar uma determinada opinião ou prática}
\end{EntryWithPhonetic}

\begin{EntryWithPhonetic}{强劲}{qiang2jing4}{12,7}{⼸,⼒}[HSK 7-9]
  \definition{adj.}{poderoso}
\end{EntryWithPhonetic}

\begin{EntryWithPhonetic}{强烈}{qiang2lie4}{12,10}{⼸,⽕}[HSK 3]
  \definition{adj.}{muito forte; intenso; poderoso | violento; impetuoso; nível muito alto; atitude muito firme, sem espaço para mudanças | afiado; marcante; mostrado em contraste; muito claro}
\end{EntryWithPhonetic}

\begin{EntryWithPhonetic}{强势}{qiang2 shi4}{12,8}{⼸,⼒}[HSK 6]
  \definition*{adj.}{forte; poderoso; dominante}
  \definition{s.}{momento; ímpeto; grande impulso; forte impulso | força; influência dominante; forças poderosas}
\end{EntryWithPhonetic}

\begin{EntryWithPhonetic}{强项}{qiang2xiang4}{12,9}{⼸,⾴}[HSK 7-9]
  \definition{adj.}{Literário: resoluto e inflexível; reto e inabalável}
  \definition{s.}{ponto forte (de um atleta ou de uma equipe) | jogo, evento ou assunto no qual alguém é forte | principal ponto forte | especialidade}
\end{EntryWithPhonetic}

\begin{EntryWithPhonetic}{强行}{qiang2xing2}{12,6}{⼸,⾏}[HSK 7-9]
  \definition{adv.}{à força; com força}
\end{EntryWithPhonetic}

\begin{EntryWithPhonetic}{强硬}{qiang2ying4}{12,12}{⼸,⽯}[HSK 7-9]
  \definition{adj.}{forte; resistente; inflexível; poderoso; não disposto a recuar}
\end{EntryWithPhonetic}

\begin{EntryWithPhonetic}{强占}{qiang2zhan4}{12,5}{⼸,⼘}[HSK 7-9]
  \definition{v.}{ocupar à força; apoderar-se | ocupar à força; confiscar; anexar}
\end{EntryWithPhonetic}

\begin{EntryWithPhonetic}{强制}{qiang2zhi4}{12,8}{⼸,⼑}[HSK 7-9]
  \definition{v.}{forçar; compelir; coagir; utilizar o poder legal, político e econômico para forçar}
\end{EntryWithPhonetic}

\begin{EntryWithPhonetic}{强壮}{qiang2 zhuang4}{12,6}{⼸,⼠}[HSK 6]
  \definition{s.}{(corpo) forte, poderoso, robusto, resistente}
  \definition{v.}{fortalecer; construir}
\end{EntryWithPhonetic}

%%%%%%%%%% 墙 %%%%%%%%%%
\subsection*{墙}\addcontentsline{loh}{figure}{墙 \dpy{qiang2}}

\begin{EntryWithPhonetic}{墙}{qiang2}{14}{⼟}[HSK 2]
  \definition[面,堵,道]{s.}{parede; barreira ou perímetro construído com tijolos, pedras, etc. | qualquer coisa com a forma ou função de uma parede; a parte de um objeto que funciona como parede ou divisória}
  \definition{v.}{(gíria) bloquear (um website) (usado geralmente na voz passiva: 被墙)}
\end{EntryWithPhonetic}

\begin{EntryWithPhonetic}{墙壁}{qiang2 bi4}{14,16}{⼟,⼟}[HSK 5]
  \definition[面,堵,道]{s.}{parede; barreira ou perímetro construído com tijolos, pedras ou terra}
\end{EntryWithPhonetic}

\begin{EntryWithPhonetic}{墙纸}{qiang2zhi3}{14,7}{⼟,⽷}
  \definition{s.}{papel de parede}
\end{EntryWithPhonetic}

%%%%%%%%%% 抢 %%%%%%%%%%
\subsection*{抢}\addcontentsline{loh}{figure}{抢 \dpy{qiang3}}

\begin{EntryWithPhonetic}{抢}{qiang3}{7}{⼿}[HSK 5]
  \definition{v.}{roubar; saquear | agarrar; apanhar; arrebatar | disputar; lutar por; ser o primeiro; competir para ser o primeiro | correr; apressar-se; fazer uma incursão | raspar; arranhar; raspar ou esfregar uma camada da superfície de um objeto}
  \seeref{qiang1}
\end{EntryWithPhonetic}

\begin{EntryWithPhonetic}{抢夺}{qiang3duo2}{7,6}{⼿,⼤}[HSK 7-9]
  \definition{v.}{agarrar; arrebatar; tomar}
\end{EntryWithPhonetic}

\begin{EntryWithPhonetic}{抢劫}{qiang3jie2}{7,7}{⼿,⼒}[HSK 7-9]
  \definition{v.}{roubar; saquear; pilhar; usar violência ilegalmente para se apropriar da propriedade alheia}
\end{EntryWithPhonetic}

\begin{EntryWithPhonetic}{抢救}{qiang3jiu4}{7,11}{⼿,⽁}[HSK 5]
  \definition{v.}{salvar; resgatar; prestar de socorro ou assistência rápidos em situações de emergência | salvar; tomar medidas rápidas para evitar ou minimizar perdas iminentes.}
\end{EntryWithPhonetic}

\begin{EntryWithPhonetic}{抢掠}{qiang3lve4}{7,11}{⼿,⼿}
  \definition{s.}{saque | pilhagem}
  \definition{v.}{saquear | pilhar}
\end{EntryWithPhonetic}

\begin{EntryWithPhonetic}{抢眼}{qiang3yan3}{7,11}{⼿,⽬}[HSK 7-9]
  \definition{adj.}{chamativo; significa ser muito chamativo e atrair a atenção do público}
\end{EntryWithPhonetic}

%%%%%%%%%% 强 %%%%%%%%%%
\subsection*{强}\addcontentsline{loh}{figure}{强 \dpy{qiang3}}

\begin{EntryWithPhonetic}{强}{qiang3}{12}{⼸}
  \definition{v.}{fazer um esforço; esforçar-se}
  \seeref{jiang4}
  \seeref{qiang2}
\end{EntryWithPhonetic}

\begin{EntryWithPhonetic}{强迫}{qiang3po4}{12,8}{⼸,⾡}[HSK 5]
  \definition{v.}{impelir; forçar; impor; compelir; aplicar pessão para obedecer}
\end{EntryWithPhonetic}

%%%%%%%%%% 呛 %%%%%%%%%%
\subsection*{呛}\addcontentsline{loh}{figure}{呛 \dpy{qiang4}}

\begin{EntryWithPhonetic}{呛}{qiang4}{7}{⼝}
  \definition{v.}{causar asfixia ou sufocamento; irritar; sensação de mal-estar devido à entrada de gases irritantes no sistema respiratório}
  \seeref{qiang1}
\end{EntryWithPhonetic}

%%%%%%%%%% 悄 %%%%%%%%%%
\subsection*{悄}\addcontentsline{loh}{figure}{悄 \dpy{qiao1}}

\begin{EntryWithPhonetic}{悄}{qiao1}{10}{⼼}
  \definition{adj.}{quieto; silencioso}
  \seeref{qiao3}
\end{EntryWithPhonetic}

\begin{EntryWithPhonetic}{悄悄}{qiao1qiao1}{10,10}{⼼,⼼}[HSK 5]
  \definition{adv.}{silenciosamente; em silêncio; aos sussuros; sem som ou em voz baixa; com o mínimo de ruído possível}
\end{EntryWithPhonetic}

%%%%%%%%%% 敲 %%%%%%%%%%
\subsection*{敲}\addcontentsline{loh}{figure}{敲 \dpy{qiao1}}

\begin{EntryWithPhonetic}{敲}{qiao1}{14}{⽁}[HSK 5]
  \definition{v.}{bater; dar uma pancada; golpear | explorar alguém; cobrar a mais; extorquir; chantagear | lembrar; criticar; alertar; advertir}
\end{EntryWithPhonetic}

\begin{EntryWithPhonetic}{敲边鼓}{qiao1 bian1gu3}{14,5,13}{⽁,⾡,⿎}[HSK 7-9]
  \definition{v.}{``Tocar trompa.'' | Coloquial: falar ou agir para ajudar alguém à margem; apoiar alguém; apoiar alguém em uma discussão}
\end{EntryWithPhonetic}

\begin{EntryWithPhonetic}{敲门}{qiao1 men2}{14,3}{⽁,⾨}[HSK 5]
  \definition{v.}{bater na porta}
\end{EntryWithPhonetic}

\begin{EntryWithPhonetic}{敲诈}{qiao1zha4}{14,7}{⽁,⾔}[HSK 7-9]
  \definition{v.}{extorquir; chantagear; usar o poder, a intimidação e as ameaças para extorquir dinheiro}
\end{EntryWithPhonetic}

%%%%%%%%%% 乔 %%%%%%%%%%
\subsection*{乔}\addcontentsline{loh}{figure}{乔 \dpy{qiao2}}

\begin{EntryWithPhonetic}{乔}{qiao2}{6}{⼃}
  \definition*{s.}{Sobrenome: Qiao}
  \definition{adj.}{alto, imponente; orgulhoso, imponente}
\end{EntryWithPhonetic}

\begin{EntryWithPhonetic}{乔装}{qiao2zhuang1}{6,12}{⼃,⾐}[HSK 7-9]
  \definition{v.}{disfarçar; vestir-se}
\end{EntryWithPhonetic}

%%%%%%%%%% 桥 %%%%%%%%%%
\subsection*{桥}\addcontentsline{loh}{figure}{桥 \dpy{qiao2}}

\begin{EntryWithPhonetic}{桥}{qiao2}{10}{⽊}[HSK 3]
  \definition*{s.}{Sobrenome: Qiao}
  \definition[座]{s.}{ponte; construção que atravessa a água conectando as duas margens}
\end{EntryWithPhonetic}

\begin{EntryWithPhonetic}{桥梁}{qiao2liang2}{10,11}{⽊,⽊}[HSK 6]
  \definition[座]{s.}{ponte; acesso; uma obra construída na superfície do rio, conectando as duas margens | ponte; metáfora para pessoas ou coisas que podem se comunicar}
\end{EntryWithPhonetic}

%%%%%%%%%% 翘 %%%%%%%%%%
\subsection*{翘}\addcontentsline{loh}{figure}{翘 \dpy{qiao2}}

\begin{EntryWithPhonetic}{翘}{qiao2}{12}{⽻}
  \definition{v.}{levantar (a cabeça) | empenar; tornar"-se deformado}
  \seeref{qiao4}
\end{EntryWithPhonetic}

%%%%%%%%%% 瞧 %%%%%%%%%%
\subsection*{瞧}\addcontentsline{loh}{figure}{瞧 \dpy{qiao2}}

\begin{EntryWithPhonetic}{瞧}{qiao2}{17}{⽬}[HSK 5]
  \definition{v.}{ver; olhar | tratar; diagnosticar e tratar | ver; visitar; fazer uma visita}
\end{EntryWithPhonetic}

\begin{EntryWithPhonetic}{瞧不起}{qiao2bu5qi3}{17,4,10}{⽬,⼀,⾛}[HSK 7-9]
  \definition{v.}{olhar com desdém para alguém; menosprezar alguém; torcer o nariz para alguém; desprezar}
\end{EntryWithPhonetic}

%%%%%%%%%% 巧 %%%%%%%%%%
\subsection*{巧}\addcontentsline{loh}{figure}{巧 \dpy{qiao3}}

\begin{EntryWithPhonetic}{巧}{qiao3}{5}{⼯}[HSK 3]
  \definition{adj.}{habilidoso; engenhoso; esperto | oportuno; coincidente; fortuito | astuto; enganoso; enganador; traiçoeiro; ardiloso | (de mão, língua) hábil; loquaz}
  \definition{s.}{(tecnologia, artesanato) habilidade; destreza}
\end{EntryWithPhonetic}

\begin{EntryWithPhonetic}{巧合}{qiao3he2}{5,6}{⼯,⼝}[HSK 7-9]
  \definition{s.}{coincidência; (coisas) coincidentes ou idênticas}
\end{EntryWithPhonetic}

\begin{EntryWithPhonetic}{巧克力}{qiao3ke4li4}{5,7,2}{⼯,⼗,⼒}[HSK 4]
  \definition[块,颗,盒,包]{s.}{Empréstimo linguístico: chocolate; alimentos feitos com cacau em pó como principal matéria-prima, açúcar e especiarias}
\end{EntryWithPhonetic}

\begin{EntryWithPhonetic}{巧妙}{qiao3miao4}{5,7}{⼯,⼥}[HSK 6]
  \definition{adj.}{inteligente; engenhoso; (método ou técnica, etc.) inteligente, além do comum}
\end{EntryWithPhonetic}

%%%%%%%%%% 悄 %%%%%%%%%%
\subsection*{悄}\addcontentsline{loh}{figure}{悄 \dpy{qiao3}}

\begin{EntryWithPhonetic}{悄}{qiao3}{10}{⼼}
  \definition{adj.}{quieto; silencioso | triste; preocupado; aflito}
  \seeref{qiao1}
\end{EntryWithPhonetic}

%%%%%%%%%% 壳 %%%%%%%%%%
\subsection*{壳}\addcontentsline{loh}{figure}{壳 \dpy{qiao4}}

\begin{EntryWithPhonetic}{壳}{qiao4}{7}{⼠}
  \definition[层,个]{s.}{Coloquial: concha | invólucro; caixa; carapaça | empresa de fachada (ou corporação) | superfície dura}
  \seeref{ke2}
\end{EntryWithPhonetic}

%%%%%%%%%% 窍 %%%%%%%%%%
\subsection*{窍}\addcontentsline{loh}{figure}{窍 \dpy{qiao4}}

\begin{EntryWithPhonetic}{窍}{qiao4}{10}{⽳}
  \definition{s.}{abertura; furo | chave para algo; a chave para a questão}
\end{EntryWithPhonetic}

\begin{EntryWithPhonetic}{窍门}{qiao4men2}{10,3}{⽳,⾨}[HSK 7-9]
  \definition[个]{s.}{habilidade; chave para um problema; segredo para fazer algo; um método inteligente que resolve o problema e é simples e fácil de implementar}
\end{EntryWithPhonetic}

%%%%%%%%%% 翘 %%%%%%%%%%
\subsection*{翘}\addcontentsline{loh}{figure}{翘 \dpy{qiao4}}

\begin{EntryWithPhonetic}{翘}{qiao4}{12}{⽻}[HSK 7-9]
  \definition{v.}{manter (segurar) erguido; dobrar (virar) para cima; enrolar-se}
  \seeref{qiao2}
\end{EntryWithPhonetic}

%%%%%%%%%% 撬 %%%%%%%%%%
\subsection*{撬}\addcontentsline{loh}{figure}{撬 \dpy{qiao4}}

\begin{EntryWithPhonetic}{撬}{qiao4}{15}{⼿}[HSK 7-9]
  \definition{v.}{arrombar; arrancar; forçar com alavanca}[钥匙丢了,他只好把门撬开。===Ele perdeu a chave, então teve que arrombar a porta.]
\end{EntryWithPhonetic}

%%%%%%%%%% 切 %%%%%%%%%%
\subsection*{切}\addcontentsline{loh}{figure}{切 \dpy{qie1}}

\begin{EntryWithPhonetic}{切}{qie1}{4}{⼑}[HSK 4]
  \definition{v.}{cortar; fatiar; separar itens com uma faca | cortar ou romper; truncar | Geometria: refere-se a quando uma linha, círculo ou superfície intercepta um círculo, arco ou esfera em apenas um ponto}
  \seeref{qie4}
\end{EntryWithPhonetic}

\begin{EntryWithPhonetic}{切除}{qie1chu2}{4,9}{⼑,⾩}[HSK 7-9]
  \definition[次]{s.}{excisão; ressecção; abscisão; corte; seccionamento}
  \definition{v.}{excisar; ressectar; remover cirurgicamente uma parte de uma estrutura ou órgão do corpo}
\end{EntryWithPhonetic}

\begin{EntryWithPhonetic}{切断}{qie1duan4}{4,11}{⼑,⽄}[HSK 7-9]
  \definition{v.}{cortar; cortar algo ao meio com uma faca; uma metáfora para usar a força para separar coisas que estão conectadas}
\end{EntryWithPhonetic}

\begin{EntryWithPhonetic}{切割}{qie1ge1}{4,12}{⼑,⼑}[HSK 7-9]
  \definition{v.}{esculpir; cortar algo com uma faca | cortar algo com máquina, fogo, arco voltaico; refere-se especificamente ao corte de materiais metálicos com máquinas-ferramentas ou à queima deles com arco elétrico, laser, etc.}
\end{EntryWithPhonetic}

%%%%%%%%%% 茄 %%%%%%%%%%
\subsection*{茄}\addcontentsline{loh}{figure}{茄 \dpy{qie2}}

\begin{EntryWithPhonetic}{茄}{qie2}{8}{⾋}
  \definition[只]{s.}{berinjela}
  \seeref{jia1}
\end{EntryWithPhonetic}

\begin{EntryWithPhonetic}{茄子}{qie2 zi5}{8,3}{⾋,⼦}[HSK 6]
  \definition{interj.}{Onomatopéia: ``xis'' fonético (ao ser fotografado), equivale ao ``diga xis''}
  \definition[个,根]{s.}{berinjela (fruto e planta)}
\end{EntryWithPhonetic}

%%%%%%%%%% 且 %%%%%%%%%%
\subsection*{且}\addcontentsline{loh}{figure}{且 \dpy{qie3}}

\begin{EntryWithPhonetic}{且}{qie3}{5}{⼀}[HSK 7-9]
  \definition*{s.}{Sobrenome: Qie}
  \definition{adv.}{apenas; por enquanto | por um longo tempo; indica algo duradouro e resistente}
  \definition{conj.}{mesmo; até; até mesmo; usado na primeira cláusula de uma frase complexa para expressar concessão, equivalente a 尚且 | ambos\dots e\dots; conecta adjetivos ou verbos para expressar relacionamento paralelo, equivalente a 而且 e 又…又…}
  \seealsoref{而且}{er2 qie3}
  \seealsoref{尚且}{shang4 qie3}
  \seealsoref{又…又…}{you4 you4}
\end{EntryWithPhonetic}

%%%%%%%%%% 切 %%%%%%%%%%
\subsection*{切}\addcontentsline{loh}{figure}{切 \dpy{qie4}}

\begin{EntryWithPhonetic}{切}{qie4}{4}{⼑}
  \definition{adj.}{ansioso; sério | duro; severo; rude; áspero}
  \definition{adv.}{com certeza; certamente}
  \definition{s.}{limiar; degrau}
  \definition{v.}{ser prático ou realista | ajustar-se ou corresponder | ser próximo ou íntimo | cortar algo em pedaços com uma faca | tomar o pulso (medicina tradicional chinesa)}
  \seeref{qie1}
\end{EntryWithPhonetic}

\begin{EntryWithPhonetic}{切身}{qie4shen1}{4,7}{⼑,⾝}[HSK 7-9]
  \definition{adj.}{de interesse imediato para si próprio ou para alguém | pessoal; de primeira mão | de preocupação imediata para si mesmo; estreitamente relacionado a si mesmo}
\end{EntryWithPhonetic}

\begin{EntryWithPhonetic}{切实}{qie4shi2}{4,8}{⼑,⼧}[HSK 6]
  \definition{adj.}{prático; viável; realista}
\end{EntryWithPhonetic}

%%%%%%%%%% 窃 %%%%%%%%%%
\subsection*{窃}\addcontentsline{loh}{figure}{窃 \dpy{qie4}}

\begin{EntryWithPhonetic}{窃}{qie4}{9}{⽳}
  \definition{adv.}{secretamente; sorrateiramente; furtivamente | usado antes de verbos para demonstrar modéstia, frequentemente significando ``Eu humildemente penso\dots'' ou ``Eu acredito em particular\dots''}[臣窃谓此策虽妙,实难实行。===Acredito humildemente que, embora essa estratégia seja engenhosa, na realidade é difícil de implementar.]
  \definition{pron.}{Literário: (referindo-se às próprias opiniões) meu; minha}
  \definition{v.}{roubar; furtar | apoderar-se ou ocupar ilegitimamente; tomar posse sem direito}
\end{EntryWithPhonetic}

\begin{EntryWithPhonetic}{窃取}{qie4qu3}{9,8}{⽳,⼜}[HSK 7-9]
  \definition{v.}{usurpar; apoderar-se; roubar (frequentemente usado metaforicamente)}
\end{EntryWithPhonetic}

%%%%%%%%%% 亲 %%%%%%%%%%
\subsection*{亲}\addcontentsline{loh}{figure}{亲 \dpy{qin1}}

\begin{EntryWithPhonetic}{亲}{qin1}{9}{⼇}[HSK 3]
  \definition{adj.}{parente próximo; relacionado por sangue; de ​​parentesco consanguíneo; parente consanguíneo mais próximo | querido; próximo; íntimo; relações próximas entre pessoas; sentimentos profundos (em oposição a 疏) | em si mesmo; pessoalmente}
  \definition[位]{s.}{pais; refere-se aos pais; também se refere apenas ao pai ou à mãe | parente; refere-se a pessoas que são relacionadas por sangue ou casamento| casal; casamento; refere-se ao casamento ou relacionamento conjugal | noiva; refere-se especificamente à noiva}
  \definition{v.}{beijar | (de países, partidos, etc.) a favor de; apoiar; estar perto de}
  \seeref{qing4}
  \seealsoref{疏}{shu1}
\end{EntryWithPhonetic}

\begin{EntryWithPhonetic}{亲爱}{qin1'ai4}{9,10}{⼇,⽖}[HSK 4]
  \definition{adj.}{querido; amado; termo carinhoso que expressa intimidade e afeto}
\end{EntryWithPhonetic}

\begin{EntryWithPhonetic}{亲和力}{qin1he2li4}{9,8,2}{⼇,⼝,⼒}[HSK 7-9]
  \definition{s.}{afinidade; as forças que interagem quando duas ou mais substâncias se combinam para formar um composto | amabilidade; sociabilidade; forte atração; metaforicamente falando, refere-se a uma força que faz as pessoas se sentirem confortáveis, amigáveis ​​e dispostas a se aproximar}
\end{EntryWithPhonetic}

\begin{EntryWithPhonetic}{亲近}{qin1jin4}{9,7}{⼇,⾡}[HSK 7-9]
  \definition{adj.}{íntimo e próximo}
  \definition{v.}{ser próximo de; ter intimidade com}
\end{EntryWithPhonetic}

\begin{EntryWithPhonetic}{亲密}{qin1mi4}{9,11}{⼇,⼧}[HSK 4]
  \definition{adj.}{próximo; íntimo; relacionamento afetuoso e próximo}
\end{EntryWithPhonetic}

\begin{EntryWithPhonetic}{亲朋好友}{qin1peng2-hao3you3}{9,8,6,4}{⼇,⽉,⼥,⼜}[HSK 7-9]
  \definition{s.}{``Amigos e familiares.''; amigos e família; parentes e amigos}
\end{EntryWithPhonetic}

\begin{EntryWithPhonetic}{亲戚}{qin1qi5}{9,11}{⼇,⼽}[HSK 7-9]
  \definition[门,个,位]{s.}{parentes; pessoas com laços matrimoniais ou consanguíneos em sua própria família}
\end{EntryWithPhonetic}

\begin{EntryWithPhonetic}{亲切}{qin1qie4}{9,4}{⼇,⼑}[HSK 3]
  \definition{adj.}{gentil; cordial; cheio de sinceridade e cuidado, fazendo com que as pessoas se sintam acolhidas e acessíveis | próximo; íntimo; por familiaridade e afeição}
\end{EntryWithPhonetic}

\begin{EntryWithPhonetic}{亲情}{qin1qing2}{9,11}{⼇,⼼}[HSK 7-9]
  \definition{s.}{afeto; laços familiares; vínculo emocional entre membros da família; o afeto entre membros da família}
\end{EntryWithPhonetic}

\begin{EntryWithPhonetic}{亲热}{qin1re4}{9,10}{⼇,⽕}[HSK 7-9]
  \definition{adj.}{afetuoso; íntimo; caloroso; íntimo e acolhedor}
  \definition{v.}{comportar-se afetuosamente; demonstrar intimidade e entusiasmo}
\end{EntryWithPhonetic}

\begin{EntryWithPhonetic}{亲人}{qin1 ren2}{9,2}{⼇,⼈}[HSK 3]
  \definition[个,位]{s.}{um membro da família; os pais, o cônjuge, os filhos, etc.; refere-se a parentes ou cônjuges | queridos; entes queridos; aqueles queridos para alguém; uma metáfora para pessoas que têm um relacionamento próximo e sentimentos profundos}
\end{EntryWithPhonetic}

\begin{EntryWithPhonetic}{亲身}{qin1shen1}{9,7}{⼇,⾝}[HSK 7-9]
  \definition{adj.}{pessoal; em primeira mão}
  \definition{adv.}{pessoalmente}
\end{EntryWithPhonetic}

\begin{EntryWithPhonetic}{亲生}{qin1sheng1}{9,5}{⼇,⽣}[HSK 7-9]
  \definition{adj.}{próprio; biológico (filhos, pais); aquelas que dão à luz a si mesmas ou têm filhos próprios}
  \definition{v.}{ser filho biológico de alguém (ou seja, não adotado)}
\end{EntryWithPhonetic}

\begin{EntryWithPhonetic}{亲手}{qin1shou3}{9,4}{⼇,⼿}[HSK 7-9]
  \definition{adv.}{si mesmo; pessoalmente; com as próprias mãos}
\end{EntryWithPhonetic}

\begin{EntryWithPhonetic}{亲属}{qin1 shu3}{9,12}{⼇,⼫}[HSK 6]
  \definition{s.}{parentes; cognatos}
\end{EntryWithPhonetic}

\begin{EntryWithPhonetic}{亲眼}{qin1 yan3}{9,11}{⼇,⽬}[HSK 6]
  \definition{adv.}{pessoalmente; com os próprios olhos}
\end{EntryWithPhonetic}

\begin{EntryWithPhonetic}{亲友}{qin1you3}{9,4}{⼇,⼜}[HSK 7-9]
  \definition[位,个]{s.}{amigos e parentes; parentes próximos}
\end{EntryWithPhonetic}

\begin{EntryWithPhonetic}{亲自}{qin1zi4}{9,6}{⼇,⾃}[HSK 3]
  \definition{adv.}{pessoalmente; em pessoa; si mesmo; fazer algo diretamente por si mesmo}
\end{EntryWithPhonetic}

%%%%%%%%%% 侵 %%%%%%%%%%
\subsection*{侵}\addcontentsline{loh}{figure}{侵 \dpy{qin1}}

\begin{EntryWithPhonetic}{侵}{qin1}{9}{⼈}
  \definition*{s.}{Sobrenome: Qin}
  \definition{prep.}{aproximando-se; aproximar}
  \definition{v.}{invadir; intrometer-se em; infringir | aproximar-se (amanhecer)}
\end{EntryWithPhonetic}

\begin{EntryWithPhonetic}{侵犯}{qin1fan4}{9,5}{⼈,⽝}[HSK 6]
  \definition{v.}{violar; invadir; infringir; interferência ilegal com terceiros e violação de seus direitos | violar; fazer incursões; invadir o território de outro país}
\end{EntryWithPhonetic}

\begin{EntryWithPhonetic}{侵害}{qin1hai4}{9,10}{⼈,⼧}[HSK 7-9]
  \definition{v.}{prejudicar; violar; infringir}
\end{EntryWithPhonetic}

\begin{EntryWithPhonetic}{侵略}{qin1lve4}{9,11}{⼈,⽥}[HSK 7-9]
  \definition{v.}{invadir; agredir; violar o território e a soberania de outro país por meio de invasão armada, interferência política ou infiltração econômica e cultural, prejudicando assim os interesses desse outro país}
\end{EntryWithPhonetic}

\begin{EntryWithPhonetic}{侵权}{qin1quan2}{9,6}{⼈,⽊}[HSK 7-9]
  \definition{s.}{infração}
  \definition{v.}{infringir os direitos de}
\end{EntryWithPhonetic}

\begin{EntryWithPhonetic}{侵入}{qin1ru4}{9,2}{⼈,⼊}
  \definition{v.}{invadir; intrometer-se em; fazer incursões em; (o inimigo) entra no território; (coisas estranhas ou nocivas) entram no interior}
\end{EntryWithPhonetic}

\begin{EntryWithPhonetic}{侵占}{qin1zhan4}{9,5}{⼈,⼘}[HSK 7-9]
  \definition{v.}{ocupar à força; tomar posse ilegal da propriedade alheia | invadir e ocupar o território de outro país; ocupar o território de outro país por meio de agressão}
\end{EntryWithPhonetic}

%%%%%%%%%% 钦 %%%%%%%%%%
\subsection*{钦}\addcontentsline{loh}{figure}{钦 \dpy{qin1}}

\begin{EntryWithPhonetic}{钦}{qin1}{9}{⾦}
  \definition*{s.}{Sobrenome: Qin}
  \definition{adv.}{pelo próprio imperador}
  \definition{v.}{admirar; respeitar}
\end{EntryWithPhonetic}

\begin{EntryWithPhonetic}{钦佩}{qin1pei4}{9,8}{⾦,⼈}[HSK 7-9]
  \definition{v.}{admirar; respeitar; ter em alta consideração alguém; sentir respeito e afeto por alguém}
\end{EntryWithPhonetic}

%%%%%%%%%% 芹 %%%%%%%%%%
\subsection*{芹}\addcontentsline{loh}{figure}{芹 \dpy{qin2}}

\begin{EntryWithPhonetic}{芹}{qin2}{7}{⾋}
  \definition[把,棵]{s.}{aipo | aipo chinês}
\end{EntryWithPhonetic}

\begin{EntryWithPhonetic}{芹菜}{qin2cai4}{7,11}{⾋,⾋}
  \definition{s.}{salsão}
\end{EntryWithPhonetic}

%%%%%%%%%% 琴 %%%%%%%%%%
\subsection*{琴}\addcontentsline{loh}{figure}{琴 \dpy{qin2}}

\begin{EntryWithPhonetic}{琴}{qin2}{12}{⽟}[HSK 5]
  \definition*{s.}{Sobrenome: Qin}
  \definition[架,台]{s.}{cítara; qin; guqin (um instrumento de cordas dedilhadas com sete cordas, em alguns aspectos semelhante à cítara)  | nome genérico para certos instrumentos musicais}
\end{EntryWithPhonetic}

\begin{EntryWithPhonetic}{琴键}{qin2jian4}{12,13}{⽟,⾦}
  \definition{s.}{tecla de piano}
\end{EntryWithPhonetic}

%%%%%%%%%% 禽 %%%%%%%%%%
\subsection*{禽}\addcontentsline{loh}{figure}{禽 \dpy{qin2}}

\begin{EntryWithPhonetic}{禽}{qin2}{12}{⽱}
  \definition*{s.}{Sobrenome: Qin}
  \definition[只]{s.}{aves; pássaros | termo genérico para aves e animais}
\end{EntryWithPhonetic}

%%%%%%%%%% 勤 %%%%%%%%%%
\subsection*{勤}\addcontentsline{loh}{figure}{勤 \dpy{qin2}}

\begin{EntryWithPhonetic}{勤}{qin2}{13}{⼒}
  \definition*{s.}{Sobrenome: Qin}
  \definition{adj.}{diligente; industrial; trabalhador}
  \definition{adv.}{frequentemente}
  \definition{s.}{dever; serviço | presença; trabalhadores que chegam ao trabalho no horário especificado}
\end{EntryWithPhonetic}

\begin{EntryWithPhonetic}{勤奋}{qin2fen4}{13,8}{⼒,⼤}[HSK 5]
  \definition{adj.}{diligente; assíduo; trabalhador; descreve alguém que se esforça continuamente nos estudos ou no trabalho}
\end{EntryWithPhonetic}

\begin{EntryWithPhonetic}{勤工俭学}{qin2gong1-jian3xue2}{13,3,9,8}{⼒,⼯,⼈,⼦}[HSK 7-9]
  \definition{expr.}{``Programa de trabalho e estudo.''; estudar em um programa de trabalho e estudo; trabalho em tempo parcial e estudo em tempo parcial; administrar uma escola com base na autossustentabilidade por meio de trabalho árduo; estudar em regime de trabalho e estudo; a prática de trabalhar enquanto estuda}
\end{EntryWithPhonetic}

\begin{EntryWithPhonetic}{勤快}{qin2kuai5}{13,7}{⼒,⼼}[HSK 7-9]
  \definition{adj.}{diligente; trabalhador; gosta de fazer coisas e não tem medo de se cansar}
\end{EntryWithPhonetic}

\begin{EntryWithPhonetic}{勤劳}{qin2lao2}{13,7}{⼒,⼒}[HSK 7-9]
  \definition{adj.}{diligente; trabalhador; esforçado; trabalhador e corajoso; trabalhe duro e não tenha medo das dificuldades}
\end{EntryWithPhonetic}

%%%%%%%%%% 擒 %%%%%%%%%%
\subsection*{擒}\addcontentsline{loh}{figure}{擒 \dpy{qin2}}

\begin{EntryWithPhonetic}{擒}{qin2}{15}{⼿}
  \definition{v.}{capturar; pegar; apreender}
\end{EntryWithPhonetic}

\begin{EntryWithPhonetic}{擒获}{qin2huo4}{15,10}{⼿,⾋}
  \definition{v.}{apreender | capturar}
\end{EntryWithPhonetic}

%%%%%%%%%% 寝 %%%%%%%%%%
\subsection*{寝}\addcontentsline{loh}{figure}{寝 \dpy{qin3}}

\begin{EntryWithPhonetic}{寝}{qin3}{13}{⼧}
  \definition{s.}{quarto | túmulo; tumba}
  \definition{v.}{dormir | parar; terminar}
\end{EntryWithPhonetic}

\begin{EntryWithPhonetic}{寝室}{qin3shi4}{13,9}{⼧,⼧}[HSK 7-9]
  \definition[间]{s.}{quarto (em um dormitório); quarto de dormir}
\end{EntryWithPhonetic}

%%%%%%%%%% 青 %%%%%%%%%%
\subsection*{青}\addcontentsline{loh}{figure}{青 \dpy{qing1}}

\begin{EntryWithPhonetic}{青}{qing1}{8}{⾭}[HSK 5][Kangxi 174]
  \definition*{s.}{Província de Qinghai, abreviação de 青海 | Sobrenome: Qing}
  \definition{adj.}{azul ou verde | preto | jovens (pessoas)}
  \definition{s.}{grama verde | colheitas jovens (não maduras) | tiras de bambu verde}
  \seealsoref{青海}{qing1hai3}
\end{EntryWithPhonetic}

\begin{EntryWithPhonetic}{青菜}{qing1cai4}{8,11}{⾭,⾋}
  \definition{s.}{verduras}
\end{EntryWithPhonetic}

\begin{EntryWithPhonetic}{青春}{qing1chun1}{8,9}{⾭,⽇}[HSK 4]
  \definition[个]{s.}{juventude; jovialidade}
\end{EntryWithPhonetic}

\begin{EntryWithPhonetic}{青春期}{qing1chun1qi1}{8,9,12}{⾭,⽇,⽉}[HSK 7-9]
  \definition{s.}{puberdade; adolescência; refere-se ao período em que os órgãos sexuais masculinos e femininos se desenvolvem rapidamente até a maturidade completa, tipicamente entre os 14 e 16 anos para os meninos e entre os 13 e 14 anos para as meninas}
\end{EntryWithPhonetic}

\begin{EntryWithPhonetic}{青海}{qing1hai3}{8,10}{⾭,⽔}
  \definition*{s.}{Província de Qinghai}
\end{EntryWithPhonetic}

\begin{EntryWithPhonetic}{青椒}{qing1jiao1}{8,12}{⾭,⽊}
  \definition{s.}{pimenta verde}
\end{EntryWithPhonetic}

\begin{EntryWithPhonetic}{青年}{qing1 nian2}{8,6}{⾭,⼲}[HSK 2]
  \definition[个,位,名,些]{s.}{juventude; jovem; refere-se ao período entre os 15 e os 30 anos de idade.}
\end{EntryWithPhonetic}

\begin{EntryWithPhonetic}{青年节}{qing1nian2jie2}{8,6,5}{⾭,⼲,⾋}
  \definition*{s.}{Dia da Juventude (4 de maio)}
\end{EntryWithPhonetic}

\begin{EntryWithPhonetic}{青少年}{qing1shao4nian2}{8,4,6}{⾭,⼩,⼲}[HSK 2]
  \definition[位,名,个,些]{s.}{adolescentes}
\end{EntryWithPhonetic}

\begin{EntryWithPhonetic}{青天}{qing1tian1}{8,4}{⾭,⼤}
  \definition{s.}{céu claro, limpo ou azul}
\end{EntryWithPhonetic}

\begin{EntryWithPhonetic}{青铜}{qing1tong2}{8,11}{⾭,⾦}
  \definition{s.}{bronze (liga de cobre, 銅, e estanho, 锡)}
\end{EntryWithPhonetic}

\begin{EntryWithPhonetic}{青蛙}{qing1wa1}{8,12}{⾭,⾍}[HSK 7-9]
  \definition[只]{s.}{sapo}
\end{EntryWithPhonetic}

\begin{EntryWithPhonetic}{青玉米}{qing1yu4mi3}{8,5,6}{⾭,⽟,⽶}
  \definition{s.}{milho verde}
\end{EntryWithPhonetic}

%%%%%%%%%% 轻 %%%%%%%%%%
\subsection*{轻}\addcontentsline{loh}{figure}{轻 \dpy{qing1}}

\begin{EntryWithPhonetic}{轻}{qing1}{9}{⾞}[HSK 2]
  \definition{adj.}{de pouco peso; leve (oposto de 重) | (de carga, equipamento, etc.) pequeno; simples | pequeno em número, grau, etc. | não sério; relaxante; leve | sem importância | suave; delicado | levianos, crédulos | leve; peso leve; densidade baixa | leve; descontraído; fácil | imprudente; descuidado | inconstante; frívolo}
  \definition{v.}{menosprezar; subestimar}
  \seealsoref{重}{zhong4}
\end{EntryWithPhonetic}

\begin{EntryWithPhonetic}{轻而易举}{qing1'er2yi4ju3}{9,6,8,9}{⾞,⽽,⽇,⼂}[HSK 7-9]
  \definition{expr.}{pode ser feito de forma descuidada; com facilidade; leve e fácil de levantar; descreve algo como fácil de fazer, que exige pouco esforço}
\end{EntryWithPhonetic}

\begin{EntryWithPhonetic}{轻蔑}{qing1mie4}{9,14}{⾞,⾋}[HSK 7-9]
  \definition{v.}{desprezar; menosprezar; ignorar}
\end{EntryWithPhonetic}

\begin{EntryWithPhonetic}{轻松}{qing1song1}{9,8}{⾞,⽊}[HSK 4]
  \definition{adj.}{leve; relaxado; livre de fardos; não nervoso; não cansado}
  \definition{v.}{sentir-se livre de fardos; não se sentir nervoso ou cansado}
\end{EntryWithPhonetic}

\begin{EntryWithPhonetic}{轻微}{qing1wei1}{9,13}{⾞,⼻}[HSK 7-9]
  \definition{adj.}{leve; insignificante; banal; trivial}
\end{EntryWithPhonetic}

\begin{EntryWithPhonetic}{轻型}{qing1xing2}{9,9}{⾞,⼟}[HSK 7-9]
  \definition{adj.}{leve (máquinas, aeronaves etc.); leve (oposto a 重型)}
  \seealsoref{重型}{zhong4xing2}
\end{EntryWithPhonetic}

\begin{EntryWithPhonetic}{轻易}{qing1yi4}{9,8}{⾞,⽇}[HSK 4]
  \definition{adv.}{facilmente; prontamente | facilmente; precipitadamente; indica que uma ação é realizada casualmente, geralmente usado em frases negativas}
\end{EntryWithPhonetic}

%%%%%%%%%% 倾 %%%%%%%%%%
\subsection*{倾}\addcontentsline{loh}{figure}{倾 \dpy{qing1}}

\begin{EntryWithPhonetic}{倾}{qing1}{10}{⼈}
  \definition{s.}{desvio; tendência}
  \definition{v.}{inclinar; inclinar-se; dobrar-se | colapsar | virar e despejar; esvaziar | fazer tudo o que puder; usar todos os recursos | sobrecarregar; dominar; dominar | admirar | superar}
\end{EntryWithPhonetic}

\begin{EntryWithPhonetic}{倾城}{qing1cheng2}{10,9}{⼈,⼟}
  \definition{adj.}{sedutora (mulher)}
  \definition{adv.}{de todo o lugar | vindo de todos os lugares}
  \definition{v.}{arruinar e derrubar o estado}
\end{EntryWithPhonetic}

\begin{EntryWithPhonetic}{倾家荡产}{qing1jia1-dang4chan3}{10,10,9,6}{⼈,⼧,⾋,⼇}[HSK 7-9]
  \definition{expr.}{``Perder a fortuna da família.''; todos os bens da família foram perdidos; ser reduzido à pobreza e à ruína}
\end{EntryWithPhonetic}

\begin{EntryWithPhonetic}{倾诉}{qing1su4}{10,7}{⼈,⾔}[HSK 7-9]
  \definition{v.}{confessar; desabafar; confidenciar}
\end{EntryWithPhonetic}

\begin{EntryWithPhonetic}{倾听}{qing1ting1}{10,7}{⼈,⼝}[HSK 7-9]
  \definition{v.}{escutar atentamente; dar ouvidos com atenção; é frequentemente usado quando um superior se dirige a um subordinado}
\end{EntryWithPhonetic}

\begin{EntryWithPhonetic}{倾向}{qing1xiang4}{10,6}{⼈,⼝}[HSK 6]
  \definition{s.}{tendência; desvio; inclinação; direção do desenvolvimento}
  \definition{v.}{preferir; estar inclinado a; concordar com uma determinada opinião}
\end{EntryWithPhonetic}

\begin{EntryWithPhonetic}{倾销}{qing1xiao1}{10,12}{⼈,⾦}[HSK 7-9]
  \definition{v.}{despejar; praticar dumping; os capitalistas monopolistas vendem grandes quantidades de mercadorias a preços abaixo do mercado para derrotar os concorrentes, conquistar participação de mercado, monopolizar os preços das commodities e obter lucros exorbitantes}
\end{EntryWithPhonetic}

\begin{EntryWithPhonetic}{倾斜}{qing1xie2}{10,11}{⼈,⽃}[HSK 7-9]
  \definition{v.}{inclinar; inclinar-se; pender | ser a favor de; dar tratamento preferencial a; a metáfora descreve uma orientação política que enfatiza um aspecto específico}
\end{EntryWithPhonetic}

%%%%%%%%%% 清 %%%%%%%%%%
\subsection*{清}\addcontentsline{loh}{figure}{清 \dpy{qing1}}

\begin{EntryWithPhonetic}{清}{qing1}{11}{⽔}[HSK 6]
  \definition*{s.}{Dinastia Qing (1644-1911) | Sobrenome: Qing}
  \definition{adj.}{claro; não misturado; (líquido ou gasoso) puro e sem mistura (em oposição a 浊) | silencioso; quieto | justo e honesto | distinto; claro; esclarecido | simples; puro, sem qualquer adulteração ou combinação | limpo; puro}
  \definition{v.}{limpar; tornar limpo | resolver; esclarecer; pagar; liquidar | contar; inspecionar}
  \seealsoref{浊}{zhuo2}
\end{EntryWithPhonetic}

\begin{EntryWithPhonetic}{清唱}{qing1chang4}{11,11}{⽔,⼝}
  \definition{v.}{cantar à capela}
\end{EntryWithPhonetic}

\begin{EntryWithPhonetic}{清彻}{qing1che4}{11,7}{⽔,⼻}
  \variantof{清澈}
\end{EntryWithPhonetic}

\begin{EntryWithPhonetic}{清澈}{qing1che4}{11,15}{⽔,⽔}
  \definition{adj.}{claro | límpido}
\end{EntryWithPhonetic}

\begin{EntryWithPhonetic}{清晨}{qing1chen2}{11,11}{⽔,⽇}[HSK 5]
  \definition{s.}{matinal; manhã cedo; geralmente se refere ao período do amanhecer até logo após o nascer do sol}
\end{EntryWithPhonetic}

\begin{EntryWithPhonetic}{清除}{qing1chu2}{11,9}{⽔,⾩}[HSK 7-9]
  \definition{v.}{eliminar; remover; livrar-se de; remover completamente}
\end{EntryWithPhonetic}

\begin{EntryWithPhonetic}{清楚}{qing1chu5}{11,13}{⽔,⽊}[HSK 2]
  \definition{adj.}{claro; distinto; compreensível; organizado; fácil de identificar e entender | plenamente consciente de; claro sobre}
  \definition{v.}{ter clareza sobre; compreender; ação que expressa compreensão e conhecimento}
\end{EntryWithPhonetic}

\begin{EntryWithPhonetic}{清脆}{qing1cui4}{11,10}{⽔,⾁}[HSK 7-9]
  \definition{adj.}{claro e melodioso; nítido e agradável ao ouvido, não abafado | (comida) crocante e refrescante}
\end{EntryWithPhonetic}

\begin{EntryWithPhonetic}{清单}{qing1dan1}{11,8}{⽔,⼗}[HSK 7-9]
  \definition[张]{s.}{lista detalhada; relato detalhado; um catálogo; um inventário; formulário de inscrição detalhado para projetos relevantes}
\end{EntryWithPhonetic}

\begin{EntryWithPhonetic}{清淡}{qing1dan4}{11,11}{⽔,⽔}[HSK 7-9]
  \definition{adj.}{leve; fraco; suave; delicado; (cor, cheiro) leve e suave; não forte | leve; não gorduroso ou com sabor forte; (alimento) com baixo teor de gordura | suave; simples; descreve uma vida ou ritmo de vida como simples e descomplicado}
\end{EntryWithPhonetic}

\begin{EntryWithPhonetic}{清洁}{qing1jie2}{11,9}{⽔,⽔}[HSK 6]
  \definition{adj.}{limpo; sem poeira, gordura, etc.}
  \definition{v.}{limpar}
\end{EntryWithPhonetic}

\begin{EntryWithPhonetic}{清洁工}{qing1 jie2 gong1}{11,9,3}{⽔,⽔,⼯}[HSK 6]
  \definition{s.}{coletor de lixo; trabalhador de saneamento; limpador de rua; trabalhadores envolvidos na limpeza do ambiente, remoção de lixo e fezes, etc.}
\end{EntryWithPhonetic}

\begin{EntryWithPhonetic}{清静}{qing1jing4}{11,14}{⽔,⾭}[HSK 7-9]
  \definition{adj.}{(ambiente) calmo; pacífico; tranquilo; isolado; silencioso}
\end{EntryWithPhonetic}

\begin{EntryWithPhonetic}{清理}{qing1li3}{11,11}{⽔,⽟}[HSK 5]
  \definition{v.}{esclarecer; resolver; verificar; colocar em ordem; organizar tudo e jogar fora o que não for útil}
\end{EntryWithPhonetic}

\begin{EntryWithPhonetic}{清凉}{qing1liang2}{11,10}{⽔,⼎}[HSK 7-9]
  \definition{adj.}{fresco e refrescante; agradavelmente fresco; fresco e agradável; refrescante e gelado}
\end{EntryWithPhonetic}

\begin{EntryWithPhonetic}{清明}{qing1ming2}{11,8}{⽔,⽇}[HSK 7-9]
  \definition*{s.}{Festival de Qingming (nos dias 4, 5 e 6 de abril); é costume limpar os túmulos nos dias 4, 5 ou 6 de abril, de acordo com a tradição popular}
  \definition{adj.}{limpo e em pé; (política) é legal e ordeira | calmo; sereno; (mental) claro e calmo | claro; brilhante}
\end{EntryWithPhonetic}

\begin{EntryWithPhonetic}{清明节}{qing1 ming2 jie2}{11,8,5}{⽔,⽇,⾋}[HSK 6]
  \definition*{s.}{Qingming ou Festival do Brilho Puro ou Dia da Varredura de Túmulos, Dia dos Finados (uma das 24~divisões do ano solar no calendário lunar chinês:~dia~4 ou 5~de~abril solar)}
\end{EntryWithPhonetic}

\begin{EntryWithPhonetic}{清爽}{qing1shuang3}{11,11}{⽔,⽘}
  \definition{adj.}{refrescante | relaxado}
\end{EntryWithPhonetic}

\begin{EntryWithPhonetic}{清晰}{qing1xi1}{11,12}{⽔,⽇}[HSK 7-9]
  \definition{adj.}{claro; distinto; consegue ver e ouvir com clareza, compreender com clareza; tem um processo de pensamento claro}
\end{EntryWithPhonetic}

\begin{EntryWithPhonetic}{清洗}{qing1 xi3}{11,9}{⽔,⽔}[HSK 6]
  \definition{v.}{enxaguar; lavar; limpar | purgar; limpar | eliminar}
\end{EntryWithPhonetic}

\begin{EntryWithPhonetic}{清新}{qing1xin1}{11,13}{⽔,⽄}[HSK 7-9]
  \definition{adj.}{fresco; puro e fresco; fresco e limpo; fresco e refrescante | (estilo) romance; original; inovador e único}
\end{EntryWithPhonetic}

\begin{EntryWithPhonetic}{清醒}{qing1xing3}{11,16}{⽔,⾣}[HSK 4]
  \definition{adj.}{sóbrio; lúcido}
  \definition{v.}{recuperar a consciência; recuperar-se de um coma}
\end{EntryWithPhonetic}

\begin{EntryWithPhonetic}{清真寺}{qing1zhen1si4}{11,10,6}{⽔,⼗,⼨}[HSK 7-9]
  \definition[所,座,个]{s.}{mesquita; as mesquitas islâmicas também são chamadas de salas de oração}
\end{EntryWithPhonetic}

%%%%%%%%%% 蜻 %%%%%%%%%%
\subsection*{蜻}\addcontentsline{loh}{figure}{蜻 \dpy{qing1}}

\begin{EntryWithPhonetic}{蜻}{qing1}{14}{⾍}
  \definition[只]{s.}{libélula, 蜻蜓}
  \seealsoref{蜻蜓}{qing1ting2}
\end{EntryWithPhonetic}

\begin{EntryWithPhonetic}{蜻蜓}{qing1ting2}{14,12}{⾍,⾍}
  \definition{s.}{libélula}
\end{EntryWithPhonetic}

\begin{EntryWithPhonetic}{蜻蝏}{qing1ting2}{14,15}{⾍,⾍}
  \variantof{蜻蜓}
\end{EntryWithPhonetic}

%%%%%%%%%% 情 %%%%%%%%%%
\subsection*{情}\addcontentsline{loh}{figure}{情 \dpy{qing2}}

\begin{EntryWithPhonetic}{情}{qing2}{11}{⼼}[HSK 7-9]
  \definition{s.}{sentimento; afeição | amor; paixão | paixão sexual; luxúria | favor; gentileza | situação; circunstâncias; condição | razão; sentido | sensibilidades; sentimentos}
\end{EntryWithPhonetic}

\begin{EntryWithPhonetic}{情报}{qing2bao4}{11,7}{⼼,⼿}[HSK 7-9]
  \definition[个,份]{s.}{inteligência; informação; notícias e reportagens sobre determinada situação são frequentemente classificadas como confidenciais}
\end{EntryWithPhonetic}

\begin{EntryWithPhonetic}{情不自禁}{qing2bu2zi4jin1}{11,4,6,13}{⼼,⼀,⾃,⽰}[HSK 7-9]
  \definition{expr.}{``Não consigo ajudar.''; não conseguir se conter; não conseguir evitar (fazer algo); ser tomado por um impulso repentino de; dominado pela emoção; enfatizando o controle total sobre as próprias emoções}
\end{EntryWithPhonetic}

\begin{EntryWithPhonetic}{情调}{qing2diao4}{11,10}{⼼,⾔}[HSK 7-9]
  \definition[出]{s.}{sentimento; apelo emocional; tom afetivo; o estilo expresso através de pensamentos e sentimentos; a natureza das coisas que podem evocar diversas emoções diferentes nas pessoas}
\end{EntryWithPhonetic}

\begin{EntryWithPhonetic}{情感}{qing2 gan3}{11,13}{⼼,⼼}[HSK 3]
  \definition[份]{s.}{emoção; sentimento | afeição; apego; reações psicológicas positivas ou negativas a estímulos externos, como gosto, raiva, tristeza, medo, amor, nojo, etc.}
\end{EntryWithPhonetic}

\begin{EntryWithPhonetic}{情怀}{qing2huai2}{11,7}{⼼,⼼}[HSK 7-9]
  \definition{s.}{sentimentos; um estado de espírito que contém uma determinada emoção}
\end{EntryWithPhonetic}

\begin{EntryWithPhonetic}{情节}{qing2jie2}{11,5}{⼼,⾋}[HSK 5]
  \definition[个,段]{s.}{enredo; trama; desenrolar específico dos acontecimentos | circunstância; detalhes do crime ou erro | enredo; roteiro; refere-se especificamente ao processo de desenvolvimento e evolução dos conflitos e contradições em obras literárias narrativas}
\end{EntryWithPhonetic}

\begin{EntryWithPhonetic}{情结}{qing2jie2}{11,9}{⼼,⽷}[HSK 7-9]
  \definition{s.}{complexidade; a turbulência emocional em meu coração; um certo sentimento que frequentemente persiste em minha mente}
\end{EntryWithPhonetic}

\begin{EntryWithPhonetic}{情景}{qing2jing3}{11,12}{⼼,⽇}[HSK 4]
  \definition[个,幕,种]{s.}{cena; vista; circunstâncias}
\end{EntryWithPhonetic}

\begin{EntryWithPhonetic}{情况}{qing2kuang4}{11,7}{⼼,⼎}[HSK 3]
  \definition[种,个,些]{s.}{condição; situação; circunstâncias; estado das coisas | mudanças notáveis e impactantes}
\end{EntryWithPhonetic}

\begin{EntryWithPhonetic}{情侣}{qing2lv3}{11,8}{⼼,⼈}[HSK 7-9]
  \definition[对,双,群]{s.}{amantes; namorados; um casal apaixonado ou um deles}
\end{EntryWithPhonetic}

\begin{EntryWithPhonetic}{情人}{qing2ren2}{11,2}{⼼,⼈}[HSK 7-9]
  \definition[对,个,位]{s.}{amante; namorado(a) | concubina; amante}
\end{EntryWithPhonetic}

\begin{EntryWithPhonetic}{情形}{qing2xing2}{11,7}{⼼,⼺}[HSK 5]
  \definition[个,种]{s.}{situação; condição; circunstâncias; estado de coisas; a situação específica das coisas}
\end{EntryWithPhonetic}

\begin{EntryWithPhonetic}{情绪}{qing2xu4}{11,11}{⼼,⽷}[HSK 6]
  \definition[种,片,股,丝]{s.}{mau humor; depressão; um sentimento ruim no coração, especialmente um estado mental desagradável quando se sente injusto | emoção; humor; moral; sentimento; o estado mental de uma pessoa ao longo de um período de tempo}
\end{EntryWithPhonetic}

\begin{EntryWithPhonetic}{情谊}{qing2yi4}{11,10}{⼼,⾔}[HSK 7-9]
  \definition{s.}{amizade; sentimentos amigáveis; emoções amistosas; os sentimentos de carinho e amor entre pessoas}
\end{EntryWithPhonetic}

\begin{EntryWithPhonetic}{情愿}{qing2yuan4}{11,14}{⼼,⽕}[HSK 7-9]
  \definition{v.}{estar disposto a; estar genuinamente disposto a fazer algo que outros não estão dispostos a fazer}
\end{EntryWithPhonetic}

%%%%%%%%%% 晴 %%%%%%%%%%
\subsection*{晴}\addcontentsline{loh}{figure}{晴 \dpy{qing2}}

\begin{EntryWithPhonetic}{晴}{qing2}{12}{⽇}[HSK 2]
  \definition{adj.}{ensolarado; bom; claro; não há nuvens no céu ou há poucas nuvens}
\end{EntryWithPhonetic}

\begin{EntryWithPhonetic}{晴朗}{qing2lang3}{12,10}{⽇,⽉}[HSK 5]
  \definition{adj.}{bom; claro; ensolarado; céu limpo e sem nuvens}
\end{EntryWithPhonetic}

\begin{EntryWithPhonetic}{晴天}{qing2 tian1}{12,4}{⽇,⼤}[HSK 2]
  \definition[个]{s.}{dia ensolarado; tempo sem nuvens ou com poucas nuvens; em meteorologia, refere-se a um tempo em que a cobertura de nuvens no céu é inferior a 10\%}
\end{EntryWithPhonetic}

%%%%%%%%%% 请 %%%%%%%%%%
\subsection*{请}\addcontentsline{loh}{figure}{请 \dpy{qing3}}

\begin{EntryWithPhonetic}{请}{qing3}{10}{⾔}[HSK 1]
  \definition*{s.}{Sobrenome: Qing}
  \definition{v.}{solicitar; perguntar | convidar; envolver | por favor; uma expressão educada usada quando você quer que alguém faça algo | comprar coisas sagradas para sacrifício, como incenso, velas, cavalos de papel e santuários de Buda; superstição se refere à compra de estátuas de Buda, santuários, etc. | entreter}
\end{EntryWithPhonetic}

\begin{EntryWithPhonetic}{请假}{qing3/jia4}{10,11}{⾔,⼈}[HSK 1]
  \definition{v.+compl.}{pedir licença para sair; solicitar permissão para não trabalhar ou estudar por um determinado período de tempo devido a doença ou outros motivos}
\end{EntryWithPhonetic}

\begin{EntryWithPhonetic}{请假条}{qing3jia4tiao2}{10,11,7}{⾔,⼈,⽊}
  \definition{s.}{pedido de licença de ausência (do trabalho ou da escola)}
\end{EntryWithPhonetic}

\begin{EntryWithPhonetic}{请柬}{qing3jian3}{10,9}{⾔,⽊}[HSK 7-9]
  \definition[封,张,份]{s.}{cartão de convite; convite}
\end{EntryWithPhonetic}

\begin{EntryWithPhonetic}{请教}{qing3jiao4}{10,11}{⾔,⽁}[HSK 3]
  \definition{v.}{consultar; pedir conselho}
\end{EntryWithPhonetic}

\begin{EntryWithPhonetic}{请进}{qing3 jin4}{10,7}{⾔,⾡}[HSK 1]
  \definition{v.}{por favor entre; convidar alguém para um espaço ou lugar}
\end{EntryWithPhonetic}

\begin{EntryWithPhonetic}{请客}{qing3/ke4}{10,9}{⾔,⼧}[HSK 2]
  \definition{v.+compl.}{receber convidados; hospedar convidados | oferecer; convidar; pagar a conta; arcar com os custos; convidar alguém para comer, tomar chá, etc.}
\end{EntryWithPhonetic}

\begin{EntryWithPhonetic}{请求}{qing3qiu2}{10,7}{⾔,⽔}[HSK 2]
  \definition[个,次]{s.}{pedido; petição; solicitação; refere-se à exigência apresentada}
  \definition{v.}{pedir; solicitar; requerer; peticionar; fazer uma solicitação e pedir que a outra parte concorde com ela}
\end{EntryWithPhonetic}

\begin{EntryWithPhonetic}{请帖}{qing3tie3}{10,8}{⾔,⼱}[HSK 7-9]
  \definition[张,份]{s.}{cartão; convite; cartão de convite; notificação enviada ao convidar convidados}
\end{EntryWithPhonetic}

\begin{EntryWithPhonetic}{请问}{qing3 wen4}{10,6}{⾔,⾨}[HSK 1]
  \definition{expr.}{Com licença, posso perguntar\dots? (para perguntar por qualquer coisa); uma maneira educada de pedir para alguém responder a uma pergunta}
\end{EntryWithPhonetic}

\begin{EntryWithPhonetic}{请坐}{qing3 zuo4}{10,7}{⾔,⼟}[HSK 1]
  \definition{v.}{por favor, sente-se; convidar outras pessoas para sentar ou descansar}
\end{EntryWithPhonetic}

%%%%%%%%%% 庆 %%%%%%%%%%
\subsection*{庆}\addcontentsline{loh}{figure}{庆 \dpy{qing4}}

\begin{EntryWithPhonetic}{庆}{qing4}{6}{⼴}
  \definition*{s.}{Sobrenome: Qing}
  \definition{s.}{celebração | ocasião para celebração; um aniversário que vale a pena comemorar}
  \definition{v.}{celebrar; felicitar; comemorar}
\end{EntryWithPhonetic}

\begin{EntryWithPhonetic}{庆典}{qing4dian3}{6,8}{⼴,⼋}[HSK 7-9]
  \definition{s.}{cerimônia; celebração; uma cerimônia de celebração muito grandiosa}
\end{EntryWithPhonetic}

\begin{EntryWithPhonetic}{庆贺}{qing4he4}{6,9}{⼴,⾙}[HSK 7-9]
  \definition{v.}{parabenizar; celebrar; celebrar uma ocasião alegre compartilhada ou parabenizar alguém que está recebendo boas notícias}
\end{EntryWithPhonetic}

\begin{EntryWithPhonetic}{庆幸}{qing4xing4}{6,8}{⼴,⼲}[HSK 7-9]
  \definition{v.}{alegrar-se; ficar contente; ficar feliz por uma situação inesperadamente boa}
\end{EntryWithPhonetic}

\begin{EntryWithPhonetic}{庆祝}{qing4zhu4}{6,9}{⼴,⽰}[HSK 3]
  \definition{v.}{celebrar; comemorar; festejar; realizar atividades para comemorar ou celebrar festivais comuns e eventos felizes}
\end{EntryWithPhonetic}

%%%%%%%%%% 亲 %%%%%%%%%%
\subsection*{亲}\addcontentsline{loh}{figure}{亲 \dpy{qing4}}

\begin{EntryWithPhonetic}{亲}{qing4}{9}{⼇}
  \definition{s.}{parentes por afinidade; parentes por casamento}
  \seeref{qin1}
\end{EntryWithPhonetic}

%%%%%%%%%% 穷 %%%%%%%%%%
\subsection*{穷}\addcontentsline{loh}{figure}{穷 \dpy{qiong2}}

\begin{EntryWithPhonetic}{穷}{qiong2}{7}{⽳}[HSK 4]
  \definition{adj.}{remoto; isolado; de difícil acesso | pobre; atingido pela pobreza | situação difícil, sem saída}
  \definition{adv.}{completamente | extremamente}
  \definition{v.}{exaurir; esgotar; consmir | ir até o fim; perseguir completamente perseguido; sondar profundamente | gastar}
\end{EntryWithPhonetic}

\begin{EntryWithPhonetic}{穷人}{qiong2 ren2}{7,2}{⽳,⼈}[HSK 4]
  \definition[个]{s.}{os pobres; pessoas pobres}
\end{EntryWithPhonetic}

%%%%%%%%%% 丘 %%%%%%%%%%
\subsection*{丘}\addcontentsline{loh}{figure}{丘 \dpy{qiu1}}

\begin{EntryWithPhonetic}{丘}{qiu1}{5}{⼀}
  \definition*{s.}{Sobrenome: Qiu}
  \definition[个]{s.}{monte; outeiro | (literário) sepultura}
\end{EntryWithPhonetic}

\begin{EntryWithPhonetic}{丘陵}{qiu1ling2}{5,10}{⼀,⾩}[HSK 7-9]
  \definition[个,片]{s.}{colinas; colinas baixas contínuas}
\end{EntryWithPhonetic}

%%%%%%%%%% 秋 %%%%%%%%%%
\subsection*{秋}\addcontentsline{loh}{figure}{秋 \dpy{qiu1}}

\begin{EntryWithPhonetic}{秋}{qiu1}{9}{⽲}
  \definition*{s.}{Sobrenome: Qiu}
  \definition{s.}{outono | época da colheita; a estação em que as colheitas amadurecem; colheitas maduras no outono | ano; refere-se a um ano | um período de tempo (geralmente conturbado)}
\end{EntryWithPhonetic}

\begin{EntryWithPhonetic}{秋季}{qiu1 ji4}{9,8}{⽲,⼦}[HSK 4]
  \definition[个]{s.}{outono; terceiro trimestre do ano, segundo o costume chinês, refere-se ao período de três meses entre o outono e o inverno, também se refere aos sétimo, oitavo e nono meses do calendário lunar}
\end{EntryWithPhonetic}

\begin{EntryWithPhonetic}{秋天}{qiu1 tian1}{9,4}{⽲,⼤}[HSK 2]
  \definition[个,段,季,番]{s.}{outono}
\end{EntryWithPhonetic}

%%%%%%%%%% 仇 %%%%%%%%%%
\subsection*{仇}\addcontentsline{loh}{figure}{仇 \dpy{qiu2}}

\begin{EntryWithPhonetic}{仇}{qiu2}{4}{⼈}
  \definition*{s.}{Sobrenome: Qiu}
  \definition{s.}{Literário: cônjuge; esposa; companheira}
  \seeref{chou2}
\end{EntryWithPhonetic}

%%%%%%%%%% 囚 %%%%%%%%%%
\subsection*{囚}\addcontentsline{loh}{figure}{囚 \dpy{qiu2}}

\begin{EntryWithPhonetic}{囚}{qiu2}{5}{⼞}
  \definition[个,群,位,名,些,批]{s.}{prisioneiro; condenado}
  \definition{v.}{aprisionar}
\end{EntryWithPhonetic}

\begin{EntryWithPhonetic}{囚犯}{qiu2fan4}{5,5}{⼞,⽝}[HSK 7-9]
  \definition[名]{s.}{prisioneiro; condenado}
\end{EntryWithPhonetic}

%%%%%%%%%% 求 %%%%%%%%%%
\subsection*{求}\addcontentsline{loh}{figure}{求 \dpy{qiu2}}

\begin{EntryWithPhonetic}{求}{qiu2}{7}{⽔}[HSK 2]
  \definition*{s.}{Sobrenome: Qiu}
  \definition{v.}{implorar; solicitar; suplicar; rogar | lutar por; buscar; investigar | tentar; procurar; tentar obter | demandar}
\end{EntryWithPhonetic}

\begin{EntryWithPhonetic}{求婚}{qiu2/hun1}{7,11}{⽔,⼥}[HSK 7-9]
  \definition{v.+compl.}{propor; fazer uma oferta de casamento; pedir em casamento}
\end{EntryWithPhonetic}

\begin{EntryWithPhonetic}{求救}{qiu2jiu4}{7,11}{⽔,⽁}[HSK 7-9]
  \definition{v.}{pedir socorro; solicitar que alguém venha em socorro; solicitar ajuda (geralmente usado em situações de desastre e perigo)}
\end{EntryWithPhonetic}

\begin{EntryWithPhonetic}{求学}{qiu2xue2}{7,8}{⽔,⼦}[HSK 7-9]
  \definition{v.}{estudar; estudar na escola | buscar conhecimento; dedicar-se aos estudos; explorar o conhecimento}
\end{EntryWithPhonetic}

\begin{EntryWithPhonetic}{求医}{qiu2yi1}{7,7}{⽔,⼖}[HSK 7-9]
  \definition{v.}{consultar um médico | procurar tratamento médico}
\end{EntryWithPhonetic}

\begin{EntryWithPhonetic}{求证}{qiu2zheng4}{7,7}{⽔,⾔}[HSK 7-9]
  \definition{v.}{procurar provar; procurar evidências (ou verificação) | buscar confirmação | buscar provas}
\end{EntryWithPhonetic}

\begin{EntryWithPhonetic}{求职}{qiu2 zhi2}{7,11}{⽔,⽿}[HSK 6]
  \definition{v.}{procurar emprego; candidatar-se a um emprego; encontrar um emprego}
\end{EntryWithPhonetic}

\begin{EntryWithPhonetic}{求助}{qiu2zhu4}{7,7}{⽔,⼒}[HSK 7-9]
  \definition{v.}{recorrer a alguém em busca de ajuda; pedir ajuda; solicitar assistência}
\end{EntryWithPhonetic}

%%%%%%%%%% 球 %%%%%%%%%%
\subsection*{球}\addcontentsline{loh}{figure}{球 \dpy{qiu2}}

\begin{EntryWithPhonetic}{球}{qiu2}{11}{⽟}[HSK 1]
  \definition[个,颗,筐]{s.}{esfera; globo; equipamento de jogo antigo, objeto tridimensional circular, feito de couro, recheado com penas, para ser chutado com os pés ou batido com um bastão | qualquer coisa com formato de bola; algo esférico ou quase esférico | bola; refere-se a certos artigos esportivos (geralmente redondos e tridimensionais) | jogo; partida; referência a esportes com bola | o Globo; a Terra; referindo-se especificamente à Terra}
\end{EntryWithPhonetic}

\begin{EntryWithPhonetic}{球场}{qiu2 chang3}{11,6}{⽟,⼟}[HSK 2]
  \definition[个,座]{s.}{quadra; campo; terreno para jogos com bola; campos para a prática de esportes com bola, como basquete, futebol, tênis e vôlei, cuja forma, tamanho e equipamentos variam de acordo com as exigências de cada esporte}
\end{EntryWithPhonetic}

\begin{EntryWithPhonetic}{球队}{qiu2 dui4}{11,4}{⽟,⾩}[HSK 2]
  \definition[个,支]{s.}{equipe (basquete, futebol, etc.); equipe de atletas formada para competições esportivas com bola, como times de basquete, futebol, etc.}
\end{EntryWithPhonetic}

\begin{EntryWithPhonetic}{球迷}{qiu2mi2}{11,9}{⽟,⾡}[HSK 3]
  \definition[个,位,名,些]{s.}{fã (de esportes de bola); pessoas obcecadas por jogar ou assistir jogos de bola}
\end{EntryWithPhonetic}

\begin{EntryWithPhonetic}{球拍}{qiu2 pai1}{11,8}{⽟,⼿}[HSK 6]
  \definition[支]{s.}{(tênis, badminton, etc.) raquete}
\end{EntryWithPhonetic}

\begin{EntryWithPhonetic}{球鞋}{qiu2 xie2}{11,15}{⽟,⾰}[HSK 2]
  \definition[双,只,款]{s.}{tênis de ginástica; tênis de tênis; tênis esportivos}
\end{EntryWithPhonetic}

\begin{EntryWithPhonetic}{球星}{qiu2 xing1}{11,9}{⽟,⽇}[HSK 6]
  \definition[位,名]{s.}{estrela do esporte (esporte com bola)}
\end{EntryWithPhonetic}

\begin{EntryWithPhonetic}{球衣}{qiu2yi1}{11,6}{⽟,⾐}
  \definition{s.}{uniforme, roupa (de uma equipe específica); camisa; camisa polo}
\end{EntryWithPhonetic}

\begin{EntryWithPhonetic}{球员}{qiu2 yuan2}{11,7}{⽟,⼝}[HSK 6]
  \definition[名,位,个]{s.}{Esporte: jogador | membro do clube esportivo}
\end{EntryWithPhonetic}

%%%%%%%%%% 区 %%%%%%%%%%
\subsection*{区}\addcontentsline{loh}{figure}{区 \dpy{qu1}}

\begin{EntryWithPhonetic}{区}{qu1}{4}{⼖}[HSK 3]
  \definition{s.}{área; distrito; região; zona; uma determinada área em terra, água ou ar | uma divisão administrativa; as divisões administrativas incluem regiões autônomas étnicas de nível provincial, distritos municipais e de condado e distritos de condado; grandes regiões administrativas, regiões, zonas especiais e regiões administrativas especiais}
  \definition{v.}{classificar; subdividir; distinguir}
  \seeref{ou1}
\end{EntryWithPhonetic}

\begin{EntryWithPhonetic}{区别}{qu1bie2}{4,7}{⼖,⼑}[HSK 3]
  \definition[种,个]{s.}{diferença; distinção; discriminação}
  \definition{v.}{distinguir; diferenciar; fazer distinção entre}
\end{EntryWithPhonetic}

\begin{EntryWithPhonetic}{区分}{qu1fen1}{4,4}{⼖,⼑}[HSK 6]
  \definition{v.}{discriminar; diferenciar; distinguir; comparar dois ou mais objetos; reconhecer suas diferenças}
\end{EntryWithPhonetic}

\begin{EntryWithPhonetic}{区域}{qu1yu4}{4,11}{⼖,⼟}[HSK 5]
  \definition[片,块,个]{s.}{área; setor; região; faixa; inclui áreas regionais com condições naturais, culturais, administrativas, etc.}
\end{EntryWithPhonetic}

%%%%%%%%%% 曲 %%%%%%%%%%
\subsection*{曲}\addcontentsline{loh}{figure}{曲 \dpy{qu1}}

\begin{EntryWithPhonetic}{曲}{qu1}{6}{⽈}
  \definition*{s.}{Sobrenome: Qu}
  \definition{adj.}{dobrado; curvo; sinuoso; oposto a 直 | errado; injustificável}
  \definition{s.}{curva (de um rio, etc.) | fermento; levedura}
  \definition{v.}{dobrar; torcer}
  \seeref{qu3}
  \seealsoref{直}{zhi2}
\end{EntryWithPhonetic}

\begin{EntryWithPhonetic}{曲棍球}{qu1gun4qiu2}{6,12,11}{⽈,⽊,⽟}
  \definition{s.}{hóquei em campo; hóquei | bola de hóquei}
\end{EntryWithPhonetic}

\begin{EntryWithPhonetic}{曲线}{qu1xian4}{6,8}{⽈,⽷}[HSK 7-9]
  \definition[条]{s.}{curva; em geometria, refere-se à trajetória de um ponto que se move sob certas condições em um plano ou no espaço | corpo curvado; linhas onduladas; também se refere às linhas do corpo humano}
\end{EntryWithPhonetic}

\begin{EntryWithPhonetic}{曲折}{qu1zhe2}{6,7}{⽈,⼿}[HSK 7-9]
  \definition{adj.}{sinuoso; tortuoso; não reto | complicado; intrincado; a situação e o enredo são complexos}
  \definition{s.}{complicações; enredo complexo e frustrante}
\end{EntryWithPhonetic}

%%%%%%%%%% 驱 %%%%%%%%%%
\subsection*{驱}\addcontentsline{loh}{figure}{驱 \dpy{qu1}}

\begin{EntryWithPhonetic}{驱}{qu1}{7}{⾺}
  \definition{v.}{dirigir (um cavalo, um carro, etc.) | expulsar; dispersar | correr rápido}
\end{EntryWithPhonetic}

\begin{EntryWithPhonetic}{驱动}{qu1dong4}{7,6}{⾺,⼒}[HSK 7-9]
  \definition{v.}{acionar; alimentar; ser movido | acionar; atuar; dirigir; ser motivado}
\end{EntryWithPhonetic}

\begin{EntryWithPhonetic}{驱逐}{qu1zhu2}{7,10}{⾺,⾡}[HSK 7-9]
  \definition{v.}{expulsar; deportar; banir; expelir}
\end{EntryWithPhonetic}

%%%%%%%%%% 屈 %%%%%%%%%%
\subsection*{屈}\addcontentsline{loh}{figure}{屈 \dpy{qu1}}

\begin{EntryWithPhonetic}{屈}{qu1}{8}{⼫}
  \definition*{s.}{Sobrenome: Qu}
  \definition[个]{s.}{injustiça; tratamento injusto | erro; queixa; injustiça}
  \definition{v.}{dobrar; curvar; encurvar | subjugar; submeter | tratar mal; tratar injustamente (ou deslealmente) | estar errado}
\end{EntryWithPhonetic}

\begin{EntryWithPhonetic}{屈服}{qu1fu2}{8,8}{⼫,⽉}[HSK 7-9]
  \definition{v.}{subjugar; submeter-se; ceder; dobrar-se; ceder e recuar diante da pressão externa, desistir da luta}
\end{EntryWithPhonetic}

\begin{EntryWithPhonetic}{屈原}{qu1yuan2}{8,10}{⼫,⼚}
  \definition*{s.}{Qu Yuan, poeta, é uma figura histórica famosa na cultura chinesa que viveu durante o Período dos Reinos Combatentes (340-278 a.C.).}
\end{EntryWithPhonetic}

%%%%%%%%%% 趋 %%%%%%%%%%
\subsection*{趋}\addcontentsline{loh}{figure}{趋 \dpy{qu1}}

\begin{EntryWithPhonetic}{趋}{qu1}{12}{⾛}
  \definition{v.}{apressar-se | tender para; tender a se tornar | (ganso, cobra, etc.) estalar a cabeça e morder as pessoas}
\end{EntryWithPhonetic}

\begin{EntryWithPhonetic}{趋势}{qu1shi4}{12,8}{⾛,⼒}[HSK 4]
  \definition{s.}{rumo; tendência; direção; impulso das coisas que se movem em uma direção ou outra}
\end{EntryWithPhonetic}

\begin{EntryWithPhonetic}{趋于}{qu1yu2}{12,3}{⾛,⼆}[HSK 7-9]
  \definition{v.}{tender a}
\end{EntryWithPhonetic}

%%%%%%%%%% 渠 %%%%%%%%%%
\subsection*{渠}\addcontentsline{loh}{figure}{渠 \dpy{qu2}}

\begin{EntryWithPhonetic}{渠}{qu2}{11}{⽊}
  \definition*{s.}{Sobrenome: Qu}
  \definition{adj.}{Literário: grande}
  \definition{pron.}{Dialeto: ele; ela}
  \definition[条]{s.}{canal; vala; fosso; trincheira | borda externa da roda | escudo}
\end{EntryWithPhonetic}

\begin{EntryWithPhonetic}{渠道}{qu2dao4}{11,12}{⽊,⾡}[HSK 6]
  \definition[条,个,种]{s.}{vala de irrigação; os cursos de água escavados pelos trabalhadores para drenagem e irrigação | maneira; meio; caminho}
\end{EntryWithPhonetic}

%%%%%%%%%% 曲 %%%%%%%%%%
\subsection*{曲}\addcontentsline{loh}{figure}{曲 \dpy{qu3}}

\begin{EntryWithPhonetic}{曲}{qu3}{6}{⽈}[HSK 7-9]
  \definition{s.}{canção; melodia; partitura}
  \seeref{qu1}
\end{EntryWithPhonetic}

%%%%%%%%%% 取 %%%%%%%%%%
\subsection*{取}\addcontentsline{loh}{figure}{取 \dpy{qu3}}

\begin{EntryWithPhonetic}{取}{qu3}{8}{⼜}[HSK 2]
  \definition{v.}{pegar; obter; buscar; pegar de um lugar; pegar nas mãos | visar; procurar; obter; provocar | adotar; assumir; escolher; selecionar}
\end{EntryWithPhonetic}

\begin{EntryWithPhonetic}{取代}{qu3dai4}{8,5}{⼜,⼈}[HSK 7-9]
  \definition{v.}{deslocar; substituir; suplantar; substituir por; assumir o controle; tomar o lugar de}
\end{EntryWithPhonetic}

\begin{EntryWithPhonetic}{取得}{qu3 de2}{8,11}{⼜,⼻}[HSK 2]
  \definition{v.}{ganhar; adquirir; obter; ser o primeiro a conseguir}
\end{EntryWithPhonetic}

\begin{EntryWithPhonetic}{取缔}{qu3di4}{8,12}{⼜,⽷}[HSK 7-9]
  \definition{v.}{proibir; criminalizar; suprimir; cancelar, encerrar ou proibir explicitamente}
\end{EntryWithPhonetic}

\begin{EntryWithPhonetic}{取而代之}{qu3'er2dai4zhi1}{8,6,5,3}{⼜,⽽,⼈,⼂}[HSK 7-9]
  \definition{expr.}{substituir alguém; suplantar alguém; tomar o lugar de alguém ou de algo; assumir o controle}
\end{EntryWithPhonetic}

\begin{EntryWithPhonetic}{取经}{qu3/jing1}{8,8}{⼜,⽷}[HSK 7-9]
  \definition{v.+compl.}{fazer uma peregrinação em busca de escrituras budistas | buscar experiência; aprender com a experiência de outra pessoa}
\end{EntryWithPhonetic}

\begin{EntryWithPhonetic}{取决于}{qu3jue2 yu2}{8,6,3}{⼜,⼎,⼆}[HSK 7-9]
  \definition{v.}{depender de; ser determinado por (algo)}
\end{EntryWithPhonetic}

\begin{EntryWithPhonetic}{取款}{qu3kuan3}{8,12}{⼜,⽋}[HSK 6]
  \definition{v.}{sacar dinheiro (de um banco); retirar o dinheiro que você depositou (geralmente se refere a retirar dinheiro do banco)}
\end{EntryWithPhonetic}

\begin{EntryWithPhonetic}{取款机}{qu3 kuan3 ji1}{8,12,6}{⼜,⽋,⽊}[HSK 6]
  \definition{s.}{ATM; caixa eletrônico; um caixa eletrônico é uma máquina que pode concluir automaticamente operações bancárias, como saques e consultas de saldo}
\end{EntryWithPhonetic}

\begin{EntryWithPhonetic}{取暖}{qu3nuan3}{8,13}{⼜,⽇}[HSK 7-9]
  \definition{v.}{aquecer-se; utilizar a energia térmica para aquecer o corpo}
\end{EntryWithPhonetic}

\begin{EntryWithPhonetic}{取胜}{qu3sheng4}{8,9}{⼜,⾁}[HSK 7-9]
  \definition{v.}{obter a vitória; alcançar o sucesso; alcançar a vitória}
\end{EntryWithPhonetic}

\begin{EntryWithPhonetic}{取水}{qu3shui3}{8,4}{⼜,⽔}
  \definition{v.}{obter água (de um poço, etc.)}
\end{EntryWithPhonetic}

\begin{EntryWithPhonetic}{取现}{qu3xian4}{8,8}{⼜,⾒}
  \definition{v.}{sacar dinheiro}
\end{EntryWithPhonetic}

\begin{EntryWithPhonetic}{取消}{qu3xiao1}{8,10}{⼜,⽔}[HSK 3]
  \definition{v.}{cancelar; suspender; anular; abolir; revogar; rescindir; tornar o sistema original, regulamentos, qualificações, direitos, etc. inválidos}
\end{EntryWithPhonetic}

\begin{EntryWithPhonetic}{取笑}{qu3xiao4}{8,10}{⼜,⽵}[HSK 7-9]
  \definition{v.}{ridicularizar; zombar de; fazer alarde de; buscar diversão}
\end{EntryWithPhonetic}

\begin{EntryWithPhonetic}{取悦}{qu3yue4}{8,10}{⼜,⼼}
  \definition{v.}{tentar agradar}
\end{EntryWithPhonetic}

%%%%%%%%%% 娶 %%%%%%%%%%
\subsection*{娶}\addcontentsline{loh}{figure}{娶 \dpy{qu3}}

\begin{EntryWithPhonetic}{娶}{qu3}{11}{⼥}[HSK 7-9]
  \definition{v.}{casar (com uma mulher); tomar por esposa}
\end{EntryWithPhonetic}

%%%%%%%%%% 厺 %%%%%%%%%%
\subsection*{厺}\addcontentsline{loh}{figure}{厺 \dpy{qu4}}

\begin{EntryWithPhonetic}{厺}{qu4}{5}{⼤}
  \variantof{去}
\end{EntryWithPhonetic}

%%%%%%%%%% 去 %%%%%%%%%%
\subsection*{去}\addcontentsline{loh}{figure}{去 \dpy{qu4}}

\begin{EntryWithPhonetic}{去}{qu4}{5}{⼛}[HSK 1]
  \definition{adj.}{passado; último; refere-se ao tempo passado (um ano)}
  \definition{adv.}{muito; extremamente; usado depois de adjetivos como 大, 多 e 远, significa 极 ou 非常}
  \definition{s.}{tom descendente, um dos quatro tons do chinês clássico e o quarto tom na pronúncia padrão do chinês moderno}
  \definition{v.}{ir; partir; sair | estar separado de | perder | remover; livrar-se de | ir (a algum lugar) para fazer algo; sair do local onde o interlocutor se encontra para outro lugar (oposto a 来) | ir para; estar indo para (fazer algo lá); usado antes de outro verbo para indicar fazer algo | desempenhar o papel de; representar o papel de; interpretar papéis em óperas | enviar; fazer ir; despachar}
  \definition{v.aux.}{usado entre uma frase verbal (ou frase preposicional) e um verbo para indicar que o primeiro é um método ou atitude e o último é um propósito | usado depois de um verbo para indicar que a ação está longe da localização do falante}
  \seealsoref{大}{da4}
  \seealsoref{多}{duo1}
  \seealsoref{非常}{fei1chang2}
  \seealsoref{极}{ji2}
  \seealsoref{来}{lai2}
  \seealsoref{远}{yuan3}
\end{EntryWithPhonetic}

\begin{EntryWithPhonetic}{去除}{qu4chu2}{5,9}{⼛,⾩}[HSK 7-9]
  \definition{v.}{remover; eliminar; livrar-se de | desalojar; expulsar}
\end{EntryWithPhonetic}

\begin{EntryWithPhonetic}{去处}{qu4chu4}{5,5}{⼛,⼡}[HSK 7-9]
  \definition{s.}{lugar para ir; paradeiro | lugar; local; localização}
\end{EntryWithPhonetic}

\begin{EntryWithPhonetic}{去掉}{qu4 diao4}{5,11}{⼛,⼿}[HSK 6]
  \definition{v.}{livrar-se de; tirar; acabar com; abandonar; erradicar}
\end{EntryWithPhonetic}

\begin{EntryWithPhonetic}{去年}{qu4nian2}{5,6}{⼛,⼲}[HSK 1]
  \definition{s.}{ano passado}
\end{EntryWithPhonetic}

\begin{EntryWithPhonetic}{去世}{qu4shi4}{5,5}{⼛,⼀}[HSK 3]
  \definition{v.}{(usado apenas para adultos, com conotações solenes) morrer; falecer; deixar este mundo}
\end{EntryWithPhonetic}

\begin{EntryWithPhonetic}{去死}{qu4si3}{5,6}{⼛,⽍}
  \definition{interj.}{Caia morto! | Vá para o Inferno!}
\end{EntryWithPhonetic}

\begin{EntryWithPhonetic}{去向}{qu4xiang4}{5,6}{⼛,⼝}[HSK 7-9]
  \definition{s.}{direção em que alguém ou algo se moveu}
\end{EntryWithPhonetic}

%%%%%%%%%% 趣 %%%%%%%%%%
\subsection*{趣}\addcontentsline{loh}{figure}{趣 \dpy{qu4}}

\begin{EntryWithPhonetic}{趣}{qu4}{15}{⾛}
  \definition{adj.}{interessante}
  \definition{s.}{interesse; deleite; diversão; passatempo | propósito; inclinação}
\end{EntryWithPhonetic}

\begin{EntryWithPhonetic}{趣味}{qu4wei4}{15,8}{⾛,⼝}[HSK 7-9]
  \definition{adj.}{agradável; interessante; atraente; simpático}
  \definition[种]{s.}{interesse; deleite; passatempo}
\end{EntryWithPhonetic}

%%%%%%%%%% 圈 %%%%%%%%%%
\subsection*{圈}\addcontentsline{loh}{figure}{圈 \dpy{quan1}}

\begin{EntryWithPhonetic}{圈}{quan1}{11}{⼞}[HSK 4]
  \definition[个]{s.}{anel; círculo; refere-se a algo em forma de anel | domínio; grupo; escopo; círculo(s)}
  \definition{v.}{cercar; rodear; circundar | marcar com um círculo}
  \seeref{juan1}
  \seeref{juan4}
\end{EntryWithPhonetic}

\begin{EntryWithPhonetic}{圈粉}{quan1fen3}{11,10}{⼞,⽶}
  \definition{s.}{(neologismo, coloquial) ganhar alguém como fã, obter novos fãs}
\end{EntryWithPhonetic}

\begin{EntryWithPhonetic}{圈套}{quan1tao4}{11,10}{⼞,⼤}[HSK 7-9]
  \definition[个]{s.}{malha; armadilha; laço; esquemas para enganar pessoas}
\end{EntryWithPhonetic}

\begin{EntryWithPhonetic}{圈子}{quan1zi5}{11,3}{⼞,⼦}[HSK 7-9]
  \definition[个]{s.}{anel; círculo; uma figura plana, redonda e oca; um objeto em forma de anel | círculo; panelinha; grupo; âmbito; refere-se ao âmbito das atividades humanas ou à área de um grupo}
\end{EntryWithPhonetic}

%%%%%%%%%% 全 %%%%%%%%%%
\subsection*{全}\addcontentsline{loh}{figure}{全 \dpy{quan2}}

\begin{EntryWithPhonetic}{全}{quan2}{6}{⼊}[HSK 2]
  \definition*{s.}{Sobrenome: Quan}
  \definition{adj.}{completo; total; inteiro}
  \definition{adv.}{inteiramente; totalmente; completamente; significa 100\%; equivalente a 完全 ou 全然}
  \definition{v.}{manter intacto; tornar perfeito ou completo; completar}
  \seealsoref{全然}{quan2ran2}
  \seealsoref{完全}{wan2quan2}
\end{EntryWithPhonetic}

\begin{EntryWithPhonetic}{全部}{quan2bu4}{6,10}{⼊,⾢}[HSK 2]
  \definition{adv.}{tudo; total; inteiro; completo; aplica-se a todos, sem exceção}
  \definition{s.}{totalidade; total; integridade; a soma de todas as partes; o todo}
\end{EntryWithPhonetic}

\begin{EntryWithPhonetic}{全长}{quan2chang2}{6,4}{⼊,⾧}[HSK 7-9]
  \definition{s.}{comprimento total | extensão; alcance}
\end{EntryWithPhonetic}

\begin{EntryWithPhonetic}{全场}{quan2 chang3}{6,6}{⼊,⼟}[HSK 3]
  \definition{s.}{toda a audiência; todos os presentes; todo o público}
\end{EntryWithPhonetic}

\begin{EntryWithPhonetic}{全称特命全权大使}{quan2cheng1 te4ming4 quan2quan2 da4shi3}{6,10,10,8,6,6,3,8}{⼊,⽲,⽜,⼝,⼊,⽊,⼤,⼈}
  \definition*{s.}{Embaixador Extraordinário e Plenipotenciário}
\end{EntryWithPhonetic}

\begin{EntryWithPhonetic}{全程}{quan2cheng2}{6,12}{⼊,⽲}[HSK 7-9]
  \definition{s.}{toda a jornada; todo o percurso}
\end{EntryWithPhonetic}

\begin{EntryWithPhonetic}{全都}{quan2 dou1}{6,10}{⼊,⾢}[HSK 5]
  \definition{adv.}{tudo; todos; sem exceção}
\end{EntryWithPhonetic}

\begin{EntryWithPhonetic}{全都不}{quan2dou1 bu4}{6,10,4}{⼊,⾢,⼀}
  \definition{adj.}{nada; nenhum; nenhum deles; nada disso}
\end{EntryWithPhonetic}

\begin{EntryWithPhonetic}{全方位}{quan2fang1wei4}{6,4,7}{⼊,⽅,⼈}[HSK 7-9]
  \definition{adj.}{versátil | em todo o redor | completo | abrangente | holístico | omnidirecional}
\end{EntryWithPhonetic}

\begin{EntryWithPhonetic}{全国}{quan2 guo2}{6,8}{⼊,⼞}[HSK 2]
  \definition{s.}{toda a nação (ou país); em todo o país; em todo o território nacional | toda a nação; todo o país}
\end{EntryWithPhonetic}

\begin{EntryWithPhonetic}{全家}{quan2 jia1}{6,10}{⼊,⼧}[HSK 2]
  \definition{s.}{toda a família; a família inteira}
\end{EntryWithPhonetic}

\begin{EntryWithPhonetic}{全局}{quan2ju2}{6,7}{⼊,⼫}[HSK 7-9]
  \definition{s.}{situação geral; situação como um todo}
\end{EntryWithPhonetic}

\begin{EntryWithPhonetic}{全力}{quan2 li4}{6,2}{⼊,⼒}[HSK 6]
  \definition{s.}{exercendo todos os seus esforços; energia ou força total; toda força ou energia}
\end{EntryWithPhonetic}

\begin{EntryWithPhonetic}{全力以赴}{quan2li4yi3fu4}{6,2,4,9}{⼊,⼒,⼈,⾛}[HSK 7-9]
  \definition{expr.}{``Dê tudo de si.''; fazer a todo custo; dar o máximo de si; prosseguir; dedicar todas as suas forças a algo}
\end{EntryWithPhonetic}

\begin{EntryWithPhonetic}{全面}{quan2mian4}{6,9}{⼊,⾯}[HSK 3]
  \definition{adj.}{geral; completo; abrangente; onipotente}
  \definition{s.}{todos os aspectos; cada aspecto}
  \seealsoref{片面}{pian4mian4}
\end{EntryWithPhonetic}

\begin{EntryWithPhonetic}{全能}{quan2neng2}{6,10}{⼊,⾁}[HSK 7-9]
  \definition{adj.}{todo-poderoso; onipotente | Esporte: versátil | pluripotente}
\end{EntryWithPhonetic}

\begin{EntryWithPhonetic}{全年}{quan2 nian2}{6,6}{⼊,⼲}[HSK 2]
  \definition{s.}{ano inteiro | anual; todo ano}
\end{EntryWithPhonetic}

\begin{EntryWithPhonetic}{全球}{quan2 qiu2}{6,11}{⼊,⽟}[HSK 3]
  \definition[门]{s.}{o mundo inteiro; a Terra inteira}
\end{EntryWithPhonetic}

\begin{EntryWithPhonetic}{全然}{quan2ran2}{6,12}{⼊,⽕}
  \definition{adv.}{completamente; inteiramente}
\end{EntryWithPhonetic}

\begin{EntryWithPhonetic}{全身}{quan2 shen1}{6,7}{⼊,⾝}[HSK 2]
  \definition{s.}{corpo inteiro; por todo o corpo; todo o corpo}
\end{EntryWithPhonetic}

\begin{EntryWithPhonetic}{全世界}{quan2 shi4 jie4}{6,5,9}{⼊,⼀,⽥}[HSK 5]
  \definition[种]{s.}{mundo inteiro; mundo todo | em todo o mundo}
\end{EntryWithPhonetic}

\begin{EntryWithPhonetic}{全体}{quan2 ti3}{6,7}{⼊,⼈}[HSK 2]
  \definition{s.}{(frequentemente referido a pessoas) todos; número total; todos | por todo o corpo | todos; inteiro; a soma de todas as partes; a soma de todos os indivíduos (geralmente se refere a pessoas)}
\end{EntryWithPhonetic}

\begin{EntryWithPhonetic}{全文}{quan2wen2}{6,4}{⼊,⽂}[HSK 7-9]
  \definition{s.}{texto completo}
\end{EntryWithPhonetic}

\begin{EntryWithPhonetic}{全心全意}{quan2xin1-quan2yi4}{6,4,6,13}{⼊,⼼,⼊,⼼}[HSK 7-9]
  \definition{expr.}{``De todo o coração.''; dedicar-se de corpo e alma a; de corpo e alma; com todo o coração}
\end{EntryWithPhonetic}

\begin{EntryWithPhonetic}{全新}{quan2 xin1}{6,13}{⼊,⽄}[HSK 6]
  \definition{adj.}{totalmente novo; inteiramente/completamente novo; refere-se a algo completamente novo, especialmente algo que não foi usado}
\end{EntryWithPhonetic}

\begin{EntryWithPhonetic}{全职}{quan2zhi2}{6,11}{⼊,⽿}
  \definition{s.}{período integral | tempo inteiro | (trabalho) \emph{full-time}}
\end{EntryWithPhonetic}

%%%%%%%%%% 权 %%%%%%%%%%
\subsection*{权}\addcontentsline{loh}{figure}{权 \dpy{quan2}}

\begin{EntryWithPhonetic}{权}{quan2}{6}{⽊}[HSK 6]
  \definition*{s.}{Sobrenome: Quan}
  \definition{adv.}{provisoriamente; por enquanto}
  \definition{s.}{Lliterário: contrapeso; peso deslizante de uma balança romana | poder; autoridade | direito | posição vantajosa | conveniência}
  \definition{v.}{pesar; medir o peso}
\end{EntryWithPhonetic}

\begin{EntryWithPhonetic}{权衡}{quan2heng2}{6,16}{⽊,⾏}[HSK 7-9]
  \definition{s.}{peso; peso de pesagem e balança de pesagem}
  \definition{v.}{pesar; equilibrar; calcular; refere-se a pesar, comparar e considerar}
\end{EntryWithPhonetic}

\begin{EntryWithPhonetic}{权力}{quan2li4}{6,2}{⽊,⼒}[HSK 6]
  \definition[种]{s.}{poder; autoridade; o poder de liderança no âmbito da responsabilidade | poder; coerção política; o poder coercitivo do status social e político}
\end{EntryWithPhonetic}

\begin{EntryWithPhonetic}{权利}{quan2li4}{6,7}{⽊,⼑}[HSK 4]
  \definition[项,种,个,条,份]{s.}{direito; interesse; os poderes e benefícios (em oposição a 义务) exercidos por um cidadão ou pessoa jurídica de acordo com a lei}
  \seealsoref{义务}{yi4wu4}
\end{EntryWithPhonetic}

\begin{EntryWithPhonetic}{权威}{quan2wei1}{6,9}{⽊,⼥}[HSK 7-9]
  \definition{adj.}{autoritativo; tem o poder e o prestígio para convencer as pessoas}
  \definition{s.}{autoridade; poder de decisão; o poder e o prestígio para inspirar a fé | autoridade; pessoa de autoridade; a pessoa ou coisa mais influente e relevante em um determinado âmbito ou área}
\end{EntryWithPhonetic}

\begin{EntryWithPhonetic}{权益}{quan2yi4}{6,10}{⽊,⽫}[HSK 7-9]
  \definition{s.}{direitos; interesses; direito legal; direitos e interesses; os direitos invioláveis ​​que devem ser desfrutados}
\end{EntryWithPhonetic}

%%%%%%%%%% 泉 %%%%%%%%%%
\subsection*{泉}\addcontentsline{loh}{figure}{泉 \dpy{quan2}}

\begin{EntryWithPhonetic}{泉}{quan2}{9}{⽔}[HSK 5]
  \definition*{s.}{Sobrenome: Quan}
  \definition[股,眼,汪]{s.}{fonte (de água mineral) | a nascente de um rio | termo antigo para moeda}
\end{EntryWithPhonetic}

%%%%%%%%%% 拳 %%%%%%%%%%
\subsection*{拳}\addcontentsline{loh}{figure}{拳 \dpy{quan2}}

\begin{EntryWithPhonetic}{拳}{quan2}{10}{⼿}[HSK 7-9]
  \definition*{s.}{Sobrenome: Quan}
  \definition[个,记,套]{s.}{punho | boxe; pugilismo}
  \definition{v.}{enrolar}
\end{EntryWithPhonetic}

\begin{EntryWithPhonetic}{拳法}{quan2fa3}{10,8}{⼿,⽔}
  \definition{s.}{boxe | luta}
\end{EntryWithPhonetic}

\begin{EntryWithPhonetic}{拳头}{quan2tou2}{10,5}{⼿,⼤}[HSK 7-9]
  \definition{adj.}{nocaute; (de produtos) de boa qualidade e competitividade; uma metáfora para ter uma vantagem e uma forte competitividade}
  \definition[个]{s.}{punho; mãos com os dedos dobrados para dentro e entrelaçados}
\end{EntryWithPhonetic}

\begin{EntryWithPhonetic}{拳王}{quan2wang2}{10,4}{⼿,⽟}
  \definition{s.}{pugilista | boxeador}
\end{EntryWithPhonetic}

%%%%%%%%%% 犬 %%%%%%%%%%
\subsection*{犬}\addcontentsline{loh}{figure}{犬 \dpy{quan3}}

\begin{EntryWithPhonetic}{犬}{quan3}{4}{⽝}[Kangxi 94]
  \definition{s.}{cachorro}
\end{EntryWithPhonetic}

%%%%%%%%%% 劝 %%%%%%%%%%
\subsection*{劝}\addcontentsline{loh}{figure}{劝 \dpy{quan4}}

\begin{EntryWithPhonetic}{劝}{quan4}{4}{⼒}[HSK 5]
  \definition*{s.}{Sobrenome: Quan}
  \definition{v.}{insistir; aconselhar; tentar persuadir; persuadir, argumentar para que as pessoas obedeçam | incentivar; encorajar}
\end{EntryWithPhonetic}

\begin{EntryWithPhonetic}{劝告}{quan4gao4}{4,7}{⼒,⼝}[HSK 7-9]
  \definition[席]{s.}{conselho; advertência; exortação; palavras ditas na esperança de que alguém corrija seus erros ou aceite conselhos}
  \definition{v.}{aconselhar; exortar; insistir; persuadir as pessoas com argumentos racionais; ajudá-las a corrigir seus erros ou a aceitar conselhos}
\end{EntryWithPhonetic}

\begin{EntryWithPhonetic}{劝说}{quan4shuo1}{4,9}{⼒,⾔}[HSK 7-9]
  \definition{v.}{aconselhar; persuadir; persuadir alguém a fazer algo ou fazer com que alguém concorde com algo}
\end{EntryWithPhonetic}

\begin{EntryWithPhonetic}{劝阻}{quan4zu3}{4,7}{⼒,⾩}[HSK 7-9]
  \definition{v.}{dissuadir alguém de; convencer alguém a não fazer algo; persuadir e parar}
\end{EntryWithPhonetic}

%%%%%%%%%% 券 %%%%%%%%%%
\subsection*{券}\addcontentsline{loh}{figure}{券 \dpy{quan4}}

\begin{EntryWithPhonetic}{券}{quan4}{8}{⼑}[HSK 6]
  \definition[张]{s.}{certificado; bilhete; ingresso; uma conta ou pedaço de papel que serve como recibo}
\end{EntryWithPhonetic}

%%%%%%%%%% 缺 %%%%%%%%%%
\subsection*{缺}\addcontentsline{loh}{figure}{缺 \dpy{que1}}

\begin{EntryWithPhonetic}{缺}{que1}{10}{⽸}[HSK 3]
  \definition{adj.}{incompleto; imperfeito}
  \definition[种]{s.}{vaga; abertura; falta}
  \definition{v.}{estar com falta de; faltar | estar ausente}
\end{EntryWithPhonetic}

\begin{EntryWithPhonetic}{缺点}{que1dian3}{10,9}{⽸,⽕}[HSK 3]
  \definition[个,些]{s.}{desvantagem; deficiência; inconveniência; ponto fraco; uma deficiência ou imperfeição (em oposição a 优点)}
  \seealsoref{优点}{you1dian3}
\end{EntryWithPhonetic}

\begin{EntryWithPhonetic}{缺乏}{que1fa2}{10,4}{⽸,⼃}[HSK 5]
  \definition{v.}{faltar; estar em falta de; não ter ou não ter totalmente (algo que deveria possuir ou é desejaria possuir)}
\end{EntryWithPhonetic}

\begin{EntryWithPhonetic}{缺口}{que1kou3}{10,3}{⽸,⼝}[HSK 7-9]
  \definition[个]{s.}{fenda; entalhe; brecha; reentrância; uma lacuna formada pela falta de uma parte de um objeto | déficit; falta de fundos, materiais, etc.; suprimentos, fundos, etc. insuficientes}
  \seealsoref{缺口儿}{que1kou3r5}
\end{EntryWithPhonetic}

\begin{EntryWithPhonetic}{缺口儿}{que1kou3r5}{10,3,2}{⽸,⼝,⼉}
  \definition{s.}{brecha; lacuna}
\end{EntryWithPhonetic}

\begin{EntryWithPhonetic}{缺勤}{que1/qin2}{10,13}{⽸,⼒}
  \definition{v.+compl.}{ausentar-se do dever (trabalho)}
\end{EntryWithPhonetic}

\begin{EntryWithPhonetic}{缺少}{que1shao3}{10,4}{⽸,⼩}[HSK 3]
  \definition{v.}{falta; estar com falta de; estar em falta de; geralmente se refere à falta de pessoas ou coisas}
\end{EntryWithPhonetic}

\begin{EntryWithPhonetic}{缺失}{que1shi1}{10,5}{⽸,⼤}[HSK 7-9]
  \definition{s.}{defeito; desvantagem; deficiência}
  \definition{v.}{perder; faltar; ter pouca}
\end{EntryWithPhonetic}

\begin{EntryWithPhonetic}{缺席}{que1/xi2}{10,10}{⽸,⼱}[HSK 7-9]
  \definition{v.+compl.}{ausentar-se; estar ausente (de uma reunião, etc.)}
\end{EntryWithPhonetic}

\begin{EntryWithPhonetic}{缺陷}{que1xian4}{10,10}{⽸,⾩}[HSK 6]
  \definition[个,处,项]{pron.}{defeito; falha; inconveniência; mancha; um lugar onde uma pessoa ou coisa está incompleta ou tem falhas porque algo está faltando}
\end{EntryWithPhonetic}

%%%%%%%%%% 却 %%%%%%%%%%
\subsection*{却}\addcontentsline{loh}{figure}{却 \dpy{que4}}

\begin{EntryWithPhonetic}{却}{que4}{7}{⼙}[HSK 4]
  \definition{adv.}{mas; contudo; no entanto; enquanto; indica um ponto de virada}
  \definition{v.}{recuar; retroceder | afastar; repelir; desencorajar | declinar; recusar; rejeitar}
  \definition{v.aux.}{usado depois de certos verbos para indicar a conclusão de uma ação, resultado, equivalente a 去 ou 掉}
  \seealsoref{掉}{diao4}
  \seealsoref{去}{qu4}
\end{EntryWithPhonetic}

\begin{EntryWithPhonetic}{却是}{que4 shi4}{7,9}{⼙,⽇}[HSK 6]
  \definition{conj.}{na verdade; no entanto; o fato é\dots; indica um ponto de virada, contrário às suas expectativas anteriores}
\end{EntryWithPhonetic}

%%%%%%%%%% 确 %%%%%%%%%%
\subsection*{确}\addcontentsline{loh}{figure}{确 \dpy{que4}}

\begin{EntryWithPhonetic}{确}{que4}{12}{⽯}
  \definition{adj.}{autenticado | sólido | firme | real | verdadeiro}
\end{EntryWithPhonetic}

\begin{EntryWithPhonetic}{确保}{que4bao3}{12,9}{⽯,⼈}[HSK 3]
  \definition{v.}{assegurar; garantir; manter ou garantir com certeza}
\end{EntryWithPhonetic}

\begin{EntryWithPhonetic}{确定}{que4ding4}{12,8}{⽯,⼧}[HSK 3]
  \definition{adj.}{definido; certo; claro}
  \definition{v.}{firmar; definir; determinar; tomar uma decisão clara e não mudar}
\end{EntryWithPhonetic}

\begin{EntryWithPhonetic}{确立}{que4li4}{12,5}{⽯,⽴}[HSK 5]
  \definition{v.}{estabelecer; criar; construir; estabelecer ou consolidar firmemente}
\end{EntryWithPhonetic}

\begin{EntryWithPhonetic}{确切}{que4qie4}{12,4}{⽯,⼑}[HSK 7-9]
  \definition{adj.}{verdadeiro; exato; definido; preciso; apropriado | verdadeiro; seguro; confiável; credível; digno de confiança}
\end{EntryWithPhonetic}

\begin{EntryWithPhonetic}{确认}{que4ren4}{12,4}{⽯,⾔}[HSK 4]
  \definition{v.}{afirmar; confirmar; reconhecer; confirmar explicitamente (fatos, princípios, etc.)}
\end{EntryWithPhonetic}

\begin{EntryWithPhonetic}{确实}{que4shi2}{12,8}{⽯,⼧}[HSK 3]
  \definition{adj.}{verdadeiro; confiável; autêntico}
  \definition{adv.}{verdadeiramente; realmente; de ​​fato; afirmar a autenticidade de fatos objetivos}
\end{EntryWithPhonetic}

\begin{EntryWithPhonetic}{确信}{que4xin4}{12,9}{⽯,⼈}[HSK 7-9]
  \definition{s.}{confirmação; informações autênticas e confiáveis}
  \definition{v.}{ter certeza; acreditar firmemente; estar convencido; acreditar plenamente, sem dúvida alguma}
\end{EntryWithPhonetic}

\begin{EntryWithPhonetic}{确凿}{que4zao2}{12,12}{⽯,⼐}[HSK 7-9]
  \definition{adj.}{autêntico; irrefutável; inegável; absolutamente verdadeiro; inegavelmente verdadeiro}
\end{EntryWithPhonetic}

\begin{EntryWithPhonetic}{确诊}{que4zhen3}{12,7}{⽯,⾔}[HSK 7-9]
  \definition{v.}{diagnosticar; fazer um diagnóstico definitivo; fazer um diagnóstico preciso (da doença)}
\end{EntryWithPhonetic}

%%%%%%%%%% 裙 %%%%%%%%%%
\subsection*{裙}\addcontentsline{loh}{figure}{裙 \dpy{qun2}}

\begin{EntryWithPhonetic}{裙}{qun2}{12}{⾐}
  \definition[条]{s.}{saia | avental | algo como uma saia}
\end{EntryWithPhonetic}

\begin{EntryWithPhonetic}{裙子}{qun2zi5}{12,3}{⾐,⼦}[HSK 3]
  \definition[条,件]{s.}{saia (peça de vestuário); uma vestimenta usada abaixo da cintura}
\end{EntryWithPhonetic}

%%%%%%%%%% 群 %%%%%%%%%%
\subsection*{群}\addcontentsline{loh}{figure}{群 \dpy{qun2}}

\begin{EntryWithPhonetic}{群}{qun2}{13}{⽺}[HSK 3]
  \definition*{s.}{Sobrenome: Qun}
  \definition{adj.}{em grupos; numerosos}
  \definition{clas.}{usado para grupos de pessoas ou coisas; grupo; rebanho; manada}
  \definition{s.}{multidão; grupo; muitas pessoas ou coisas reunidas | as massas; um grupo de pessoas; refere-se a um grande número de pessoas}
\end{EntryWithPhonetic}

\begin{EntryWithPhonetic}{群山}{qun2shan1}{13,3}{⽺,⼭}
  \definition{s.}{montanhas | uma cadeia de colinas}
\end{EntryWithPhonetic}

\begin{EntryWithPhonetic}{群体}{qun2 ti3}{13,7}{⽺,⼈}[HSK 5]
  \definition[个]{s.}{colônia; um conjunto composto por muitos indivíduos da mesma espécie que estão fisicamente conectados, exemplos incluem corais entre os animais e certas algas entre as plantas | grupos; refere-se, de maneira geral, ao conjunto formado por muitos indivíduos interligados que compartilham características essenciais em comum}
\end{EntryWithPhonetic}

\begin{EntryWithPhonetic}{群众}{qun2zhong4}{13,6}{⽺,⼈}[HSK 5]
  \definition[个,名,位]{s.}{as massas; refere-se ao povo em geral | não filiado; apartidário; refere-se a pessoas que não são membros do Partido Comunista Chinês nem da Liga da Juventude Comunista | alguém que não ocupa uma posição de liderança}
\end{EntryWithPhonetic}

%%%%% EOF %%%%%

