%%%
%%% W
%%%
\section*{W}\addcontentsline{toc}{section}{W}\addcontentsline{loh}{figure}{\#\#\#\#\#\#\#\# W}

%%%%%%%%%% 凹 %%%%%%%%%%
\subsection*{凹}\addcontentsline{loh}{figure}{凹 \dpy{wa1}}

\begin{EntryWithPhonetic}{凹}{wa1}{5}{⼐}
  \variantof{洼}
  \seeref{ao1}
\end{EntryWithPhonetic}

%%%%%%%%%% 哇 %%%%%%%%%%
\subsection*{哇}\addcontentsline{loh}{figure}{哇 \dpy{wa1}}

\begin{EntryWithPhonetic}{哇}{wa1}{9}{⼝}
  \definition{interj.}{Onomatopéia: som de choro ou vômito | ``Uau!''; expressa surpresa}
  \seeref{wa5}
\end{EntryWithPhonetic}

\begin{EntryWithPhonetic}{哇塞}{wa1sai1}{9,13}{⼝,⼟}
  \definition{interj.}{``Uau!'; exclamação de espanto, admiração, etc.}
\end{EntryWithPhonetic}

\begin{EntryWithPhonetic}{哇噻}{wa1sai1}{9,16}{⼝,⼝}
  \variantof{哇塞}
\end{EntryWithPhonetic}

%%%%%%%%%% 挖 %%%%%%%%%%
\subsection*{挖}\addcontentsline{loh}{figure}{挖 \dpy{wa1}}

\begin{EntryWithPhonetic}{挖}{wa1}{9}{⼿}[HSK 6]
  \definition{v.}{cavar; escavar; arrancar | explorar; sondar | (dialeto) arranhar | escavar a superfície de um objeto com ferramentas ou mãos}
\end{EntryWithPhonetic}

\begin{EntryWithPhonetic}{挖掘}{wa1jue2}{9,11}{⼿,⼿}[HSK 7-9]
  \definition{v.}{desenterrar; escavar; aprofundar"-se | sondar; conectar"-se a; metaforicamente, significa desenvolver em profundidade e buscar}
  \synonymref{发掘}{fa1jue2}
  \synonymref{发现}{fa1xian4}
  \synonymref{开采}{kai1cai3}
  \antonymref{埋藏}{mai2cang2}
  \antonymref{埋没}{mai2mo4}
\end{EntryWithPhonetic}

\begin{EntryWithPhonetic}{挖掘机}{wa1jue2ji1}{9,11,6}{⼿,⼿,⽊}
  \definition{s.}{escavadeira; máquina de escavar; pá mecânica}
  \seealsoref{挖土机}{wa1tu3ji1}
\end{EntryWithPhonetic}

\begin{EntryWithPhonetic}{挖苦}{wa1ku5}{9,8}{⼿,⾋}[HSK 7-9]
  \definition{v.}{falar sarcasticamente; ridicularizar alguém com palavras ácidas e sarcásticas}
  \synonymref{嘲弄}{chao2nong4}
  \synonymref{嘲笑}{chao2xiao4}
  \synonymref{讽刺}{feng3ci4}
  \synonymref{讥笑}{ji1xiao4}
  \antonymref{称赞}{cheng1zan4}
  \antonymref{恭维}{gong1wei2}
  \antonymref{赞赏}{zan4shang3}
  \antonymref{赞扬}{zan4yang2}
\end{EntryWithPhonetic}

\begin{EntryWithPhonetic}{挖土机}{wa1tu3ji1}{9,3,6}{⼿,⼟,⽊}
  \definition{s.}{escavadeira; máquina de escavar | retroescavadeira}
\end{EntryWithPhonetic}

%%%%%%%%%% 洼 %%%%%%%%%%
\subsection*{洼}\addcontentsline{loh}{figure}{洼 \dpy{wa1}}

\begin{EntryWithPhonetic}{洼}{wa1}{9}{⽔}
  \definition{adj.}{oco; baixo}
  \definition{s.}{área baixa; depressão; oco}
\end{EntryWithPhonetic}

%%%%%%%%%% 娃 %%%%%%%%%%
\subsection*{娃}\addcontentsline{loh}{figure}{娃 \dpy{wa2}}

\begin{EntryWithPhonetic}{娃}{wa2}{9}{⼥}
  \definition[个,名,位,只]{s.}{bebê; criança | filho ou filha; criança | Dialeto: animal recém-nascido | Literário: menina; jovem mulher | Literário: menina bonita}
\end{EntryWithPhonetic}

\begin{EntryWithPhonetic}{娃娃}{wa2wa5}{9,9}{⼥,⼥}[HSK 6]
  \definition[个,名,位]{s.}{bebê; criança; criança pequena | boneca; brinquedos em forma de crianças}
\end{EntryWithPhonetic}

%%%%%%%%%% 瓦 %%%%%%%%%%
\subsection*{瓦}\addcontentsline{loh}{figure}{瓦 \dpy{wa3}}

\begin{EntryWithPhonetic}{瓦}{wa3}{4}{⽡}[HSK 7-9][Kangxi 98]
  \definition{clas.}{W, watt; medida de potência elétrica}
  \definition[片,块]{s.}{telha; os materiais para telhados são geralmente feitos de barro cozido, mas também podem ser feitos de cimento ou outros materiai; eles vêm em vários formatos, incluindo arqueados, planos ou semicilíndricos | cerâmica; algo feito de argila ou barro cozido}
  \seeref{wa4}
  \seealsoref{瓦特}{wa3te4}
\end{EntryWithPhonetic}

\begin{EntryWithPhonetic}{瓦努阿图}{wa3nu3'a1tu2}{4,7,7,8}{⽡,⼒,⾩,⼞}
  \definition*{s.}{Vanuatu, país do sudoeste do Oceano Pacífico}
\end{EntryWithPhonetic}

\begin{EntryWithPhonetic}{瓦特}{wa3te4}{4,10}{⽡,⽜}
  \definition{s.}{Empréstimo Linguístico: watt | medida de potência}
\end{EntryWithPhonetic}

\begin{EntryWithPhonetic}{瓦}{wa4}{4}{⽡}[Kangxi 98]
  \definition{v.}{colocar telhas (cobertura de telhados)}
  \seeref{wa3}
\end{EntryWithPhonetic}

%%%%%%%%%% 袜 %%%%%%%%%%
\subsection*{袜}\addcontentsline{loh}{figure}{袜 \dpy{wa4}}

\begin{EntryWithPhonetic}{袜}{wa4}{10}{⾐}
  \definition[只,双,打]{s.}{meias; meias-calças}
\end{EntryWithPhonetic}

\begin{EntryWithPhonetic}{袜子}{wa4zi5}{10,3}{⾐,⼦}[HSK 4]
  \definition[双,只,对]{s.}{meias; peúgas; meias-calças}
\end{EntryWithPhonetic}

%%%%%%%%%% 哇 %%%%%%%%%%
\subsection*{哇}\addcontentsline{loh}{figure}{哇 \dpy{wa5}}

\begin{EntryWithPhonetic}{哇}{wa5}{9}{⼝}[HSK 6]
  \definition{part.}{a mudança do som de 啊 devido à influência do som final da palavra anterior, ``u'' ou ``ao''}
  \seeref{wa1}
  \seealsoref{啊}{a5}
\end{EntryWithPhonetic}

%%%%%%%%%% 歪 %%%%%%%%%%
\subsection*{歪}\addcontentsline{loh}{figure}{歪 \dpy{wai1}}

\begin{EntryWithPhonetic}{歪}{wai1}{9}{⽌}[HSK 7-9]
  \definition{adj.}{torto; desalinhado; oblíquo; inclinado; não reto; enviesado | tortuoso; ardiloso; indecente; impróprio; desonesto}
  \definition{v.}{estar inclinado | reclinar-se para descansar}
  \antonymref{偏}{pian1}
  \antonymref{斜}{xie2}
  \antonymref{正}{zheng4}
\end{EntryWithPhonetic}

\begin{EntryWithPhonetic}{歪果仁}{wai1 guo3 ren2}{9,8,4}{⽌,⽊,⼈}
  \definition{s.}{gíria na \emph{Internet} para estrangeiro (外国人)}
  \seealsoref{外国人}{wai4 guo2 ren2}
\end{EntryWithPhonetic}

\begin{EntryWithPhonetic}{歪曲}{wai1qu1}{9,6}{⽌,⽈}[HSK 7-9]
  \definition{v.}{distorcer as coisas intencionalmente (geralmente se referindo a retratar coisas boas como ruins)}
  \synonymref{扭曲}{niu3qu1}
  \synonymref{误解}{wu4jie3}
\end{EntryWithPhonetic}

%%%%%%%%%% 外 %%%%%%%%%%
\subsection*{外}\addcontentsline{loh}{figure}{外 \dpy{wai4}}

\begin{EntryWithPhonetic}{外}{wai4}{5}{⼣}[HSK 1]
  \definition{adj.}{outro (que não o próprio) | não íntimo; não intimamente relacionado | não oficial | exterior; externo; do lado de fora | outros; referindo"-se a um local fora da sua localização atual | do lado da mãe, da irmã ou da filha; referir"-se a parentes do lado materno, irmãs ou filhas | informal; não oficial}
  \definition{adv.}{adicionalmente; além disso | para fora; para o exterior; fora | extra; além disso}
  \definition{s.}{fora; externo; exterior | outro local; outro lugar | estrangeiro; país estrangeiro | lado externo | parentes de sua mãe, irmãs ou filhas}
  \antonymref{里}{li3}
  \antonymref{内}{nei4}
\end{EntryWithPhonetic}

\begin{EntryWithPhonetic}{外币}{wai4bi4}{5,4}{⼣,⼱}[HSK 6]
  \definition[种]{s.}{moeda estrangeira}
\end{EntryWithPhonetic}

\begin{EntryWithPhonetic}{外边}{wai4bian5}{5,5}{⼣,⾡}[HSK 1]
  \definition{s.}{fora; exterior; externo; além de um determinado limite | local diferente de onde se vive ou trabalha; referindo"-se a lugares distantes | exterior; externo; superfície}
\end{EntryWithPhonetic}

\begin{EntryWithPhonetic}{外表}{wai4biao3}{5,8}{⼣,⾐}[HSK 7-9]
  \definition[个]{s.}{exterior; superfície; aparência}
  \synonymref{表面}{biao3mian4}
  \synonymref{概况}{gai4kuang4}
  \synonymref{轮廓}{lun2kuo4}
  \synonymref{外观}{wai4guan1}
  \synonymref{外貌}{wai4mao4}
  \synonymref{外形}{wai4xing2}
  \synonymref{外面}{wai4mian5}
  \antonymref{内涵}{nei4han2}
  \antonymref{内心}{nei4xin1}
  \antonymref{内在}{nei4zai4}
\end{EntryWithPhonetic}

\begin{EntryWithPhonetic}{外部}{wai4bu4}{5,10}{⼣,⾢}[HSK 6]
  \definition{s.}{fora; externo; fora de um certo intervalo | exterior; superfície}
\end{EntryWithPhonetic}

\begin{EntryWithPhonetic}{外插}{wai4cha1}{5,12}{⼣,⼿}
  \definition{v.}{extrapolar | (computação) conectar (um dispositivo periférico, etc.)}
\end{EntryWithPhonetic}

\begin{EntryWithPhonetic}{外出}{wai4chu1}{5,5}{⼣,⼐}[HSK 6]
  \definition{v.}{sair, especialmente para ir a outro lugar a negócios}
\end{EntryWithPhonetic}

\begin{EntryWithPhonetic}{外地}{wai4di4}{5,6}{⼣,⼟}[HSK 2]
  \definition{s.}{não local; outros lugares; locais fora da área local}
\end{EntryWithPhonetic}

\begin{EntryWithPhonetic}{外公}{wai4gong1}{5,4}{⼣,⼋}[HSK 7-9]
  \definition[位,名,个]{s.}{avô materno; pai da mãe}
  \antonymref{孙子}{sun1zi5}
\end{EntryWithPhonetic}

\begin{EntryWithPhonetic}{外观}{wai4guan1}{5,6}{⼣,⾒}[HSK 6]
  \definition{s.}{aspecto; semblante; aparência; aparência exterior; a aparência de um objeto}
\end{EntryWithPhonetic}

\begin{EntryWithPhonetic}{外国}{wai4guo2}{5,8}{⼣,⼞}[HSK 1]
  \definition[个]{s.}{país estrangeiro}
\end{EntryWithPhonetic}

\begin{EntryWithPhonetic}{外国人}{wai4 guo2 ren2}{5,8,2}{⼣,⼞,⼈}
  \definition[个]{s.}{estrangeiro | alienígena}
\end{EntryWithPhonetic}

\begin{EntryWithPhonetic}{外海}{wai4hai3}{5,10}{⼣,⽔}
  \definition{s.}{mar aberto}
\end{EntryWithPhonetic}

\begin{EntryWithPhonetic}{外行}{wai4hang2}{5,6}{⼣,⾏}[HSK 7-9]
  \definition{adj.}{sem habilidade, inexperiente, não especialista; descreve alguém que não possui absolutamente nenhum conhecimento ou experiência em determinada profissão ou tecnologia}
  \definition[个]{s.}{leigo; forasteiro; pessoas que não possuem conhecimento ou experiência em determinada profissão ou tecnologia}
  \antonymref{行家}{hang2jia5}
  \antonymref{内行}{nei4hang2}
  \antonymref{在行}{zai4hang2}
  \antonymref{专家}{zhuan1jia1}
\end{EntryWithPhonetic}

\begin{EntryWithPhonetic}{外号}{wai4hao4}{5,5}{⼣,⼝}[HSK 7-9]
  \definition[个]{s.}{apelido; nomes dados por outras pessoas, diferentes do nome real, muitas vezes têm conotações de carinho, brincadeira, elogio ou ódio}[她的外号是“胖胖”。===Seu apelido é "Gordinha".]
  \synonymref{绰号}{chuo4hao4}
\end{EntryWithPhonetic}

\begin{EntryWithPhonetic}{外汇}{wai4hui4}{5,5}{⼣,⽔}[HSK 4]
  \definition{s.}{câmbio estrangeiro; moeda estrangeira; moedas estrangeiras e títulos, como cheques, letras de câmbio, notas promissórias, etc., conversíveis em moedas estrangeiras, usados na compensação do comércio internacional}
\end{EntryWithPhonetic}

\begin{EntryWithPhonetic}{外积}{wai4ji1}{5,10}{⼣,⽲}
  \definition{s.}{produto exterior | (matemática) o produto vetorial de dois vetores}
\end{EntryWithPhonetic}

\begin{EntryWithPhonetic}{外籍}{wai4ji2}{5,20}{⼣,⽵}[HSK 7-9]
  \definition[个]{s.}{nacionalidade estrangeira | registro de residência permanente não local; \emph{status} de visitante}
  \synonymref{国外}{guo2wai4}
\end{EntryWithPhonetic}

\begin{EntryWithPhonetic}{外交}{wai4jiao1}{5,6}{⼣,⼇}[HSK 3]
  \definition[个]{s.}{diplomacia; relações exteriores; atividades de um país nas relações internacionais, como participar de organizações e conferências internacionais, trocar enviados com outros países, conduzir negociações, assinar tratados e acordos, etc.}
\end{EntryWithPhonetic}

\begin{EntryWithPhonetic}{外交官}{wai4jiao1guan1}{5,6,8}{⼣,⼇,⼧}[HSK 4]
  \definition[位,名]{s.}{diplomata}
\end{EntryWithPhonetic}

\begin{EntryWithPhonetic}{外交家}{wai4jiao1jia1}{5,6,10}{⼣,⼇,⼧}
  \definition{s.}{diplomata}[老资格的外交家===diplomata veterano]
\end{EntryWithPhonetic}

\begin{EntryWithPhonetic}{外交学}{wai4jiao1 xue2}{5,6,8}{⼣,⼇,⼦}
  \definition{s.}{diplomacia}
\end{EntryWithPhonetic}

\begin{EntryWithPhonetic}{外界}{wai4jie4}{5,9}{⼣,⽥}[HSK 5]
  \definition{s.}{o exterior; o mundo externo; área fora de um determinado âmbito; sociedade externa}
\end{EntryWithPhonetic}

\begin{EntryWithPhonetic}{外科}{wai4ke1}{5,9}{⼣,⽲}[HSK 6]
  \definition[名]{s.}{cirurgia; departamento cirúrgico; um departamento em uma instituição médica que usa principalmente cirurgia para tratar doenças internas e externas}
\end{EntryWithPhonetic}

\begin{EntryWithPhonetic}{外来}{wai4lai2}{5,7}{⼣,⽊}[HSK 6]
  \definition{adj.}{de fora; externo; estrangeiro}
\end{EntryWithPhonetic}

\begin{EntryWithPhonetic}{外卖}{wai4mai4}{5,8}{⼣,⼗}[HSK 2]
  \definition[份,单,盒]{s.}{comida para viagem; levar para viagem}
  \definition{v.}{entregar; oferecer; refere"-se à ação do comerciante entregar alimentos no local especificado pelo cliente}
\end{EntryWithPhonetic}

\begin{EntryWithPhonetic}{外贸}{wai4mao4}{5,9}{⼣,⾙}[HSK 7-9]
  \definition{s.}{comércio exterior, abreviação de 对外贸易}
  \seealsoref{对外贸易}{dui4wai4 mao4yi4}
  \synonymref{概念}{gai4nian4}
  \synonymref{金钱}{jin1qian2}
  \synonymref{贸易}{mao4yi4}
\end{EntryWithPhonetic}

\begin{EntryWithPhonetic}{外贸协会}{wai4mao4xie2hui4}{5,9,6,6}{⼣,⾙,⼗,⼈}
  \definition*{s.}{Associação de Comércio Exterior}
\end{EntryWithPhonetic}

\begin{EntryWithPhonetic}{外貌}{wai4mao4}{5,14}{⼣,⾘}[HSK 7-9]
  \definition{s.}{aparência; exterior; aspecto; a aparência de uma pessoa ou coisa}
  \synonymref{模样}{mu2yang4}
  \synonymref{容貌}{rong2mao4}
  \synonymref{容颜}{rong2yan2}
  \synonymref{外表}{wai4biao3}
  \antonymref{内涵}{nei4han2}
  \antonymref{内心}{nei4xin1}
  \antonymref{内在}{nei4zai4}
  \antonymref{心灵}{xin1ling2}
\end{EntryWithPhonetic}

\begin{EntryWithPhonetic}{外貌协会}{wai4mao4xie2hui4}{5,14,6,6}{⼣,⾘,⼗,⼈}
  \definition{s.}{o ``clube da boa aparência'': pessoas que atribuem grande importância à aparência de uma pessoa (trocadilho com a Associação de Comércio Exterior, 外贸协会)}
  \seealsoref{外贸协会}{wai4mao4xie2hui4}
  \seealsoref{外协}{wai4xie2}
\end{EntryWithPhonetic}

\begin{EntryWithPhonetic}{外面}{wai4mian5}{5,9}{⼣,⾯}[HSK 3]
  \definition{s.}{o lado de fora; fora de um certo intervalo | exterior; aparência externa; a superfície de um objeto}
\end{EntryWithPhonetic}

\begin{EntryWithPhonetic}{外婆}{wai4po2}{5,11}{⼣,⼥}[HSK 7-9]
  \definition{s.}{avó materna; a mãe da minha mãe; no Sul, ela geralmente é chamada de 外婆, enquanto no Norte, ela geralmente é chamada de 姥姥}
  \synonymref{姥姥}{lao3lao5}
\end{EntryWithPhonetic}

\begin{EntryWithPhonetic}{外企}{wai4qi3}{5,6}{⼣,⼈}[HSK 7-9]
  \definition[家]{s.}{empresa estrangeira; empresas com investimento estrangeiro}
\end{EntryWithPhonetic}

\begin{EntryWithPhonetic}{外事}{wai4shi4}{5,8}{⼣,⼅}
  \definition{s.}{assuntos ou relações exteriores}
\end{EntryWithPhonetic}

\begin{EntryWithPhonetic}{外水}{wai4shui3}{5,4}{⼣,⽔}
  \definition{s.}{renda extra}
\end{EntryWithPhonetic}

\begin{EntryWithPhonetic}{外孙}{wai4sun1}{5,6}{⼣,⼦}
  \definition{s.}{filho da filha}
\end{EntryWithPhonetic}

\begin{EntryWithPhonetic}{外孙女}{wai4sun1nv3}{5,6,3}{⼣,⼦,⼥}
  \definition{s.}{filha da filha}
\end{EntryWithPhonetic}

\begin{EntryWithPhonetic}{外套}{wai4tao4}{5,10}{⼣,⼤}[HSK 4]
  \definition[件,套,个]{s.}{casaco; jaqueta; paletó; sobretudo}
\end{EntryWithPhonetic}

\begin{EntryWithPhonetic}{外头}{wai4tou5}{5,5}{⼣,⼤}[HSK 6]
  \definition{s.}{Coloquial: fora; ao ar livre}
  \antonymref{里头}{li3tou5}
\end{EntryWithPhonetic}

\begin{EntryWithPhonetic}{外围}{wai4wei2}{5,7}{⼣,⼞}
  \definition{adv.}{arredores}
\end{EntryWithPhonetic}

\begin{EntryWithPhonetic}{外文}{wai4wen2}{5,4}{⼣,⽂}[HSK 3]
  \definition[种,门]{s.}{língua ou escrita estrangeira}
\end{EntryWithPhonetic}

\begin{EntryWithPhonetic}{外协}{wai4xie2}{5,6}{⼣,⼗}
  \definition{s.}{terceirização | pessoas que julgam os outros pela aparência}
  \seealsoref{外貌协会}{wai4mao4xie2hui4}
\end{EntryWithPhonetic}

\begin{EntryWithPhonetic}{外星人}{wai4xing1ren2}{5,9,2}{⼣,⽇,⼈}[HSK 7-9]
  \definition{s.}{extraterrestre; alienígena | alienígena espacial}[我们真的遇到过外星人吗?===Alguma vez já nos deparamos realmente com extraterrestres?]
\end{EntryWithPhonetic}

\begin{EntryWithPhonetic}{外形}{wai4xing2}{5,7}{⼣,⼺}[HSK 7-9]
  \definition{s.}{forma; exterior; aparência; formato externo}
  \synonymref{轮廓}{lun2kuo4}
  \synonymref{外表}{wai4biao3}
  \synonymref{外观}{wai4guan1}
  \synonymref{外貌}{wai4mao4}
\end{EntryWithPhonetic}

\begin{EntryWithPhonetic}{外需}{wai4xu1}{5,14}{⼣,⾬}
  \definition{s.}{Economia: demanda externa}
  \seealsoref{内需}{nei4xu1}
\end{EntryWithPhonetic}

\begin{EntryWithPhonetic}{外衣}{wai4yi1}{5,6}{⼣,⾐}[HSK 6]
  \definition[件]{s.}{casaco; jaqueta; colete; sobreveste; envoltório; roupa externa (ou vestimenta); capa externa; vestido externo | semblante; aparência; feição}
\end{EntryWithPhonetic}

\begin{EntryWithPhonetic}{外语}{wai4yu3}{5,9}{⼣,⾔}[HSK 1]
  \definition[种,门]{s.}{língua estrangeira}
\end{EntryWithPhonetic}

\begin{EntryWithPhonetic}{外援}{wai4yuan2}{5,12}{⼣,⼿}[HSK 7-9]
  \definition[个,名]{s.}{ajuda externa; auxílio externo; assistência externa | Esporte: jogador estrangeiro}
\end{EntryWithPhonetic}

\begin{EntryWithPhonetic}{外资}{wai4zi1}{5,10}{⼣,⾙}[HSK 6]
  \definition{s.}{capital estrangeiro; investimento estrangeiro; fundos estrangeiros; capital investido por países estrangeiros}
  \antonymref{内资}{nei4 zi1}
\end{EntryWithPhonetic}

%%%%%%%%%% 弯 %%%%%%%%%%
\subsection*{弯}\addcontentsline{loh}{figure}{弯 \dpy{wan1}}

\begin{EntryWithPhonetic}{弯}{wan1}{9}{⼸}[HSK 4]
  \definition{adj.}{curvo; tortuoso; torto | para algo curvo, como a lua, etc. | dobrado; flexível}
  \definition[个,道]{s.}{curva; dobra; volta}
  \definition{v.}{dobrar; flexionar; curvar | Literário: desenhar}
\end{EntryWithPhonetic}

\begin{EntryWithPhonetic}{弯曲}{wan1qu1}{9,6}{⼸,⽈}[HSK 6]
  \definition{s.}{torto; curvo; sinuoso; tortuoso; não reto}
  \definition{v.}{dobrar; curvar; flexionar}
\end{EntryWithPhonetic}

%%%%%%%%%% 豌 %%%%%%%%%%
\subsection*{豌}\addcontentsline{loh}{figure}{豌 \dpy{wan1}}

\begin{EntryWithPhonetic}{豌}{wan1}{15}{⾖}
  \definition[粒]{s.}{ervilhas}
\end{EntryWithPhonetic}

\begin{EntryWithPhonetic}{豌豆}{wan1dou4}{15,7}{⾖,⾖}
  \definition{s.}{ervilha}
\end{EntryWithPhonetic}

%%%%%%%%%% 丸 %%%%%%%%%%
\subsection*{丸}\addcontentsline{loh}{figure}{丸 \dpy{wan2}}

\begin{EntryWithPhonetic}{丸}{wan2}{3}{⼂}[HSK 7-9]
  \definition{clas.}{utilizado para medicamentos em comprimido}
  \definition[个]{s.}{bola; grânulo | comprimido; bolo (como em bolo alimentar); pílula}
  \seealsoref{丸儿}{wan2r5}
\end{EntryWithPhonetic}

\begin{EntryWithPhonetic}{丸儿}{wan2r5}{3,2}{⼂,⼉}
  \definition{s.}{bola; grânulo}
\end{EntryWithPhonetic}

%%%%%%%%%% 完 %%%%%%%%%%
\subsection*{完}\addcontentsline{loh}{figure}{完 \dpy{wan2}}

\begin{EntryWithPhonetic}{完}{wan2}{7}{⼧}[HSK 2]
  \definition*{s.}{Sobrenome: Wan}
  \definition{adj.}{inteiro; intacto; completo}
  \definition{v.}{acabar; terminar; completar | pagar | estar terminado; estar pronto para | esgotar; ser usado}
\end{EntryWithPhonetic}

\begin{EntryWithPhonetic}{完备}{wan2bei4}{7,8}{⼧,⼡}[HSK 7-9]
  \definition{adj.}{perfeito; completo; impecável; descreve ter tudo o que se deve ter, sem lhe faltar nada}
  \definition{v.}{não deixar nada a desejar}
  \synonymref{具备}{ju4bei4}
  \synonymref{齐全}{qi2quan2}
  \synonymref{完好}{wan2hao3}
  \synonymref{完满}{wan2man3}
  \synonymref{完美}{wan2mei3}
  \synonymref{完全}{wan2quan2}
  \synonymref{完善}{wan2shan4}
  \synonymref{完整}{wan2zheng3}
  \synonymref{圆满}{yuan2man3}
  \antonymref{简陋}{jian3lou4}
  \antonymref{欠缺}{qian4que1}
\end{EntryWithPhonetic}

\begin{EntryWithPhonetic}{完毕}{wan2bi4}{7,6}{⼧,⽐}[HSK 7-9]
  \definition{v.}{terminar; concluir; finalizar; estar concluído}
  \seealsoref{结束}{jie2shu4}
  \synonymref{了结}{liao3jie2}
  \synonymref{完成}{wan2/cheng2}
  \synonymref{完了}{wan2le5}
  \synonymref{完了}{wan2liao3}
  \antonymref{进行}{jin4xing2}
  \antonymref{开始}{kai1shi3}
\end{EntryWithPhonetic}

\begin{EntryWithPhonetic}{完成}{wan2/cheng2}{7,6}{⼧,⼽}[HSK 2]
  \definition{v.+compl.}{realizar; completar; terminar; cumprir; levar ao sucesso}
\end{EntryWithPhonetic}

\begin{EntryWithPhonetic}{完蛋}{wan2/dan4}{7,11}{⼧,⾍}[HSK 7-9]
  \definition{v.+compl.}{estar terminado; estar concluído para; estar destruído ou colapsado}
  \antonymref{成功}{cheng2gong1}
\end{EntryWithPhonetic}

\begin{EntryWithPhonetic}{完好}{wan2hao3}{7,6}{⼧,⼥}[HSK 7-9]
  \definition{adj.}{intacto; inteiro; em bom estado}
  \synonymref{齐全}{qi2quan2}
  \synonymref{完备}{wan2bei4}
  \synonymref{完满}{wan2man3}
  \synonymref{完美}{wan2mei3}
  \synonymref{完全}{wan2quan2}
  \synonymref{完善}{wan2shan4}
  \synonymref{完整}{wan2zheng3}
  \synonymref{圆满}{yuan2man3}
  \antonymref{残缺}{can2que1}
  \antonymref{粉碎}{fen3sui4}
  \antonymref{漏洞}{lou4dong4}
  \antonymref{破碎}{po4sui4}
\end{EntryWithPhonetic}

\begin{EntryWithPhonetic}{完了}{wan2le5}{7,2}{⼧,⼅}[HSK 5]
  \definition{v.}{acabar; terminar; concluir; chegar ao fim}
  \seeref{wan2liao3}
\end{EntryWithPhonetic}

\begin{EntryWithPhonetic}{完了}{wan2liao3}{7,2}{⼧,⼅}
  \definition{v.}{estar terminado; estar concluído (uma tarefa)}
\end{EntryWithPhonetic}

\begin{EntryWithPhonetic}{完满}{wan2man3}{7,13}{⼧,⽔}
  \definition{adj.}{satisfatório; bem"-sucedido; perfeito}
  \synonymref{美满}{mei3man3}
  \synonymref{齐全}{qi2quan2}
  \synonymref{完备}{wan2bei4}
  \synonymref{完好}{wan2hao3}
  \synonymref{完美}{wan2mei3}
  \synonymref{完善}{wan2shan4}
  \synonymref{圆满}{yuan2man3}
\end{EntryWithPhonetic}

\begin{EntryWithPhonetic}{完美}{wan2mei3}{7,9}{⼧,⽺}[HSK 3]
  \definition{adj.}{perfeito; impecável; consumado}
\end{EntryWithPhonetic}

\begin{EntryWithPhonetic}{完全}{wan2quan2}{7,6}{⼧,⼊}[HSK 2]
  \definition{adj.}{inteiro; completo; não falta nada, está tudo completo}
  \definition{adv.}{completamente; representa tudo}
\end{EntryWithPhonetic}

\begin{EntryWithPhonetic}{完人}{wan2ren2}{7,2}{⼧,⼈}
  \definition{s.}{pessoa perfeita}
\end{EntryWithPhonetic}

\begin{EntryWithPhonetic}{完善}{wan2shan4}{7,12}{⼧,⼝}[HSK 3]
  \definition{adj.}{perfeito; consumado}
  \definition{v.}{refinar; melhorar; tornar perfeito}
\end{EntryWithPhonetic}

\begin{EntryWithPhonetic}{完税}{wan2shui4}{7,12}{⼧,⽲}
  \definition{v.}{pagar imposto}
\end{EntryWithPhonetic}

\begin{EntryWithPhonetic}{完完全全}{wan2wan2quan2quan2}{7,7,6,6}{⼧,⼧,⼊,⼊}
  \definition{adv.}{completamente}
\end{EntryWithPhonetic}

\begin{EntryWithPhonetic}{完整}{wan2zheng3}{7,16}{⼧,⽁}[HSK 3]
  \definition{adj.}{intacto; inteiro; completo; integrado; nenhum dano ou mutilação}
\end{EntryWithPhonetic}

%%%%%%%%%% 玩 %%%%%%%%%%
\subsection*{玩}\addcontentsline{loh}{figure}{玩 \dpy{wan2}}

\begin{EntryWithPhonetic}{玩}{wan2}{8}{⽟}
  \definition*{s.}{Sobrenome: Wan}
  \definition{s.}{objeto de apreciação; coisas para assistir}
  \definition{v.}{(~儿) divertir"-se; entreter"-se; fazer atividades que te deixem feliz | jogar; praticar algum tipo de atividade cultural, de entretenimento ou esportiva | recorrer a; usar métodos e meios impróprios para atingir o objetivo | provocar; subestimar; tratar com uma atitude frívola; desprezar | desfrutar; apreciar; observar | (~儿) envolver"-se em; tomar parte em; perseguir ou expressar deliberadamente um certo sentimento | ponderar; pensar cuidadosamente; apreciar}
\end{EntryWithPhonetic}

\begin{EntryWithPhonetic}{玩伴}{wan2ban4}{8,7}{⽟,⼈}
  \definition{s.}{parceiro de brincadeira}
\end{EntryWithPhonetic}

\begin{EntryWithPhonetic}{玩遍}{wan2bian4}{8,12}{⽟,⾡}
  \definition{v.}{passear (todo o país, toda a cidade, etc.) | visitar (um grande número de lugares)}
\end{EntryWithPhonetic}

\begin{EntryWithPhonetic}{玩家}{wan2jia1}{8,10}{⽟,⼧}
  \definition{s.}{entusiasta (áudio, modelos de aviões, etc.) | jogador (de um jogo)}
\end{EntryWithPhonetic}

\begin{EntryWithPhonetic}{玩具}{wan2ju4}{8,8}{⽟,⼋}[HSK 3]
  \definition[个,件,套]{s.}{brinquedo; coisas para brincar}
\end{EntryWithPhonetic}

\begin{EntryWithPhonetic}{玩具厂}{wan2ju4chang3}{8,8,2}{⽟,⼋,⼚}
  \definition{s.}{fábrica de brinquedos}
\end{EntryWithPhonetic}

\begin{EntryWithPhonetic}{玩具车}{wan2ju4 che1}{8,8,4}{⽟,⼋,⾞}
  \definition{s.}{carrinho de brinquedo}
\end{EntryWithPhonetic}

\begin{EntryWithPhonetic}{玩偶}{wan2'ou3}{8,11}{⽟,⼈}
  \definition{s.}{estatueta de brinquedo | boneco de ação | bicho de pelúcia | boneca}
\end{EntryWithPhonetic}

\begin{EntryWithPhonetic}{玩儿}{wan2r5}{8,2}{⽟,⼉}[HSK 1]
  \definition{v.}{divertir"-se; (entretenimento) relaxar ou experimentar alguma atividade}
\end{EntryWithPhonetic}

\begin{EntryWithPhonetic}{玩耍}{wan2shua3}{8,9}{⽟,⽽}[HSK 7-9]
  \definition{v.}{brincar; divertir"-se; entreter"-se; envolver"-se em atividades que lhe tragam alegria; jogar jogos}
  \synonymref{游玩}{you2wan2}
  \synonymref{游戏}{you2xi4}
  \antonymref{休息}{xiu1xi5}
\end{EntryWithPhonetic}

\begin{EntryWithPhonetic}{玩味}{wan2wei4}{8,8}{⽟,⼝}
  \definition{v.}{ponderar sutilezas | ruminar (pensamentos)}
\end{EntryWithPhonetic}

\begin{EntryWithPhonetic}{玩艺}{wan2yi4}{8,4}{⽟,⾋}
  \variantof{玩意}
\end{EntryWithPhonetic}

\begin{EntryWithPhonetic}{玩意}{wan2yi4}{8,13}{⽟,⼼}
  \definition{s.}{ato | brinquedo | coisa | truque (em uma performance, show de palco, acrobacias, etc.)}
\end{EntryWithPhonetic}

\begin{EntryWithPhonetic}{玩意儿}{wan2yi4r5}{8,13,2}{⽟,⼼,⼉}[HSK 7-9]
  \definition[个]{s.}{brinquedo; objeto de diversão | mágica, acrobacias, diálogos cruzados, canto de baladas, etc.; refere"-se às artes folclóricas, acrobacias, artes marciais, etc. | coisa; objeto | usado de forma desdenhosa; termos pejorativos para pessoas ou coisas}[他就一个没用的玩意儿。===Ele não passa de um lixo inútil.]
\end{EntryWithPhonetic}

\begin{EntryWithPhonetic}{玩者}{wan2zhe3}{8,8}{⽟,⽼}
  \definition{s.}{jogador}
\end{EntryWithPhonetic}

%%%%%%%%%% 顽 %%%%%%%%%%
\subsection*{顽}\addcontentsline{loh}{figure}{顽 \dpy{wan2}}

\begin{EntryWithPhonetic}{顽}{wan2}{10}{⾴}
  \definition*{s.}{Sobrenome: Wan}
  \definition{adj.}{estúpido; denso; insensível | teimoso; obstinado; não é facilmente persuadido ou subjugado | travesso; pernicioso | cabeça dura; estúpido e ignorante}
  \definition{v.}{brincar; divertir"-se | empregar; recorrer a | envolver"-se em; tomar parte em}
\end{EntryWithPhonetic}

\begin{EntryWithPhonetic}{顽固}{wan2gu4}{10,8}{⾴,⼞}[HSK 7-9]
  \definition{adj.}{obstinado; teimoso; cabeçudo; pensamento conservador, sem vontade de mudar | intransigente; veementemente contrário à mudança; postura reacionária | crônico; profundamente fixado; refere"-se a um estado difícil de mudar}
  \synonymref{保守}{bao3shou3}
  \synonymref{固执}{gu4zhi5}
  \synonymref{坚定}{jian1ding4}
  \synonymref{坚决}{jian1jue2}
  \synonymref{坚强}{jian1qiang2}
  \synonymref{鉴定}{jian4ding4}
  \synonymref{倔强}{jue2jiang4}
  \synonymref{顽强}{wan2qiang2}
  \antonymref{开通}{kai1tong5}
\end{EntryWithPhonetic}

\begin{EntryWithPhonetic}{顽皮}{wan2pi2}{10,5}{⾴,⽪}[HSK 6]
  \definition{adj.}{atrevido; travesso; arteiro; levado; (crianças, adolescentes, etc.) adoram brincar e causar problemas e não dão ouvidos a conselhos}
\end{EntryWithPhonetic}

\begin{EntryWithPhonetic}{顽强}{wan2qiang2}{10,12}{⾴,⼸}[HSK 6]
  \definition{adj.}{firme; tenaz; indomável; forte; resistente}
\end{EntryWithPhonetic}

%%%%%%%%%% 挽 %%%%%%%%%%
\subsection*{挽}\addcontentsline{loh}{figure}{挽 \dpy{wan3}}

\begin{EntryWithPhonetic}{挽}{wan3}{10}{⼿}[HSK 7-9]
  \definition{s.}{canção fúnebre; canção memorial; elegia}
  \definition{v.}{puxar | enrolar (as roupas) | rebocar (veículos) | enganchar o braço | lamentar a morte de alguém}
\end{EntryWithPhonetic}

\begin{EntryWithPhonetic}{挽回}{wan3hui2}{10,6}{⼿,⼞}[HSK 7-9]
  \definition{v.}{recuperar; resgatar; reaver; reconquistar; recuperar o que foi perdido | transformar (o mal em bem); reverter a situação desfavorável estabelecida}
  \synonymref{补救}{bu3jiu4}
  \synonymref{回旋}{hui2xuan2}
  \synonymref{解救}{jie3jiu4}
  \synonymref{扭转}{niu3zhuan3}
  \synonymref{抢救}{qiang3jiu4}
  \synonymref{挽救}{wan3jiu4}
  \synonymref{旋转}{xuan2zhuan3}
  \antonymref{剥夺}{bo1duo2}
\end{EntryWithPhonetic}

\begin{EntryWithPhonetic}{挽救}{wan3jiu4}{10,11}{⼿,⽁}[HSK 7-9]
  \definition{v.}{salvar; resgatar; remediar; resgatar do perigo}
  \synonymref{补救}{bu3jiu4}
  \synonymref{急救}{ji2jiu4}
  \synonymref{救济}{jiu4ji4}
  \synonymref{名言}{ming2yan2}
  \synonymref{扭转}{niu3zhuan3}
  \synonymref{抢救}{qiang3jiu4}
  \synonymref{调解}{tiao2jie3}
  \synonymref{挽回}{wan3hui2}
  \synonymref{旋转}{xuan2zhuan3}
\end{EntryWithPhonetic}

%%%%%%%%%% 埦 %%%%%%%%%%
\subsection*{埦}\addcontentsline{loh}{figure}{埦 \dpy{wan3}}

\begin{EntryWithPhonetic}{埦}{wan3}{11}{⼟}
  \variantof{碗}
\end{EntryWithPhonetic}

%%%%%%%%%% 惋 %%%%%%%%%%
\subsection*{惋}\addcontentsline{loh}{figure}{惋 \dpy{wan3}}

\begin{EntryWithPhonetic}{惋}{wan3}{11}{⼼}
  \definition{s.}{Literário: suspiro}
  \definition{v.}{Literário: suspirar}
\end{EntryWithPhonetic}

\begin{EntryWithPhonetic}{惋惜}{wan3xi1}{11,11}{⼼,⼼}[HSK 7-9]
  \definition{v.}{lamentar; sentir que é uma pena; sentir pena de alguém; estar arrependido; expressar simpatia e pena pelos infortúnios das pessoas ou por mudanças insatisfatórias nas coisas}
  \synonymref{可惜}{ke3xi1}
  \synonymref{怜惜}{lian2xi1}
  \synonymref{无奈}{wu2nai4}
  \synonymref{遗憾}{yi2han4}
  \antonymref{庆幸}{qing4xing4}
\end{EntryWithPhonetic}

%%%%%%%%%% 晚 %%%%%%%%%%
\subsection*{晚}\addcontentsline{loh}{figure}{晚 \dpy{wan3}}

\begin{EntryWithPhonetic}{晚}{wan3}{11}{⽇}[HSK 1]
  \definition*{s.}{Sobrenome: Wan}
  \definition{adj.}{tarde; tardio; passado o prazo acordado | júnior; mais jovem | mais tarde no tempo}
  \definition{s.}{noite; à noite; após o pôr do sol | últimos anos; última vida; um período posterior; refere"-se especificamente à velhice de uma pessoa | pôr do sol; ao pôr do sol}
\end{EntryWithPhonetic}

\begin{EntryWithPhonetic}{晚安}{wan3'an1}{11,6}{⽇,⼧}[HSK 2]
  \definition{expr.}{Tenha uma boa noite; uma frase educada usada para se despedir ou cumprimentar as pessoas à noite}
\end{EntryWithPhonetic}

\begin{EntryWithPhonetic}{晚报}{wan3bao4}{11,7}{⽇,⼿}[HSK 2]
  \definition[份,张]{s.}{jornal vespertino; um jornal publicado todas as tardes}
\end{EntryWithPhonetic}

\begin{EntryWithPhonetic}{晚餐}{wan3can1}{11,16}{⽇,⾷}[HSK 2]
  \definition[份,顿,次]{s.}{ceia; jantar}
\end{EntryWithPhonetic}

\begin{EntryWithPhonetic}{晚点}{wan3 dian3}{11,9}{⽇,⽕}[HSK 4]
  \definition{v.}{atrasar; adiar; (veículo, navio ou avião) partir, operar ou chegar mais tarde do que o horário especificado}
\end{EntryWithPhonetic}

\begin{EntryWithPhonetic}{晚饭}{wan3fan4}{11,7}{⽇,⾷}[HSK 1]
  \definition[顿]{s.}{jantar}
\end{EntryWithPhonetic}

\begin{EntryWithPhonetic}{晚会}{wan3hui4}{11,6}{⽇,⼈}[HSK 2]
  \definition[场,个,次]{s.}{festa noturna; entretenimento noturno}
\end{EntryWithPhonetic}

\begin{EntryWithPhonetic}{晚间}{wan3jian1}{11,7}{⽇,⾨}[HSK 7-9]
  \definition{s.}{(à) noite; (ao) anoitecer}
  \synonymref{晚上}{wan3shang5}
  \antonymref{早晨}{zao3chen5}
\end{EntryWithPhonetic}

\begin{EntryWithPhonetic}{晚近}{wan3jin4}{11,7}{⽇,⾡}
  \definition{s.}{nos últimos anos; durante os últimos anos | tarde | mais recente no passado | recentemente}
\end{EntryWithPhonetic}

\begin{EntryWithPhonetic}{晚景}{wan3jing3}{11,12}{⽇,⽇}
  \definition{s.}{circunstâncias dos anos de declínio de alguém | cena noturna}
\end{EntryWithPhonetic}

\begin{EntryWithPhonetic}{晚年}{wan3nian2}{11,6}{⽇,⼲}[HSK 7-9]
  \definition{s.}{velhice; ocaso; os últimos anos (restantes) da vida; crepúsculo; os estágios finais da vida}
\end{EntryWithPhonetic}

\begin{EntryWithPhonetic}{晚期}{wan3qi1}{11,12}{⽇,⽉}[HSK 7-9]
  \definition{s.}{estágio tardio; (da doença) estágio terminal | período posterior | fase final | terminal}
  \antonymref{早期}{zao3qi1}
\end{EntryWithPhonetic}

\begin{EntryWithPhonetic}{晚上}{wan3shang5}{11,3}{⽇,⼀}[HSK 1]
  \definition[个]{s.}{noite; o período entre o pôr do sol e a madrugada}
\end{EntryWithPhonetic}

\begin{EntryWithPhonetic}{晚育}{wan3yu4}{11,8}{⽇,⾁}
  \definition{s.}{parto tardio}
  \definition{v.}{ter um filho mais tarde}
\end{EntryWithPhonetic}

%%%%%%%%%% 碗 %%%%%%%%%%
\subsection*{碗}\addcontentsline{loh}{figure}{碗 \dpy{wan3}}

\begin{EntryWithPhonetic}{碗}{wan3}{13}{⽯}[HSK 2]
  \definition*{s.}{Sobrenome: Wan}
  \definition{clas.}{usado para medição de alimentos e bebidas}
  \definition[只,个]{s.}{tigela | objeto em forma de tigela}
\end{EntryWithPhonetic}

\begin{EntryWithPhonetic}{碗柜}{wan3gui4}{13,8}{⽯,⽊}
  \definition{s.}{armário}
\end{EntryWithPhonetic}

\begin{EntryWithPhonetic}{碗子}{wan3zi5}{13,3}{⽯,⼦}
  \definition{s.}{tigela}
\end{EntryWithPhonetic}

%%%%%%%%%% 万 %%%%%%%%%%
\subsection*{万}\addcontentsline{loh}{figure}{万 \dpy{wan4}}

\begin{EntryWithPhonetic}{万}{wan4}{3}{⼀}[HSK 2]
  \definition*{s.}{Sobrenome: Wan}
  \definition{adv.}{absolutamente; indica um grau extremamente alto, equivalente a 完全, 绝对 e 极}
  \definition{num.}{dez mil; 10.000; 1.0000 | miríade; um número muito grande}
  \seealsoref{极}{ji2}
  \seealsoref{绝对}{jue2dui4}
  \seealsoref{完全}{wan2quan2}
\end{EntryWithPhonetic}

\begin{EntryWithPhonetic}{万分}{wan4fen1}{3,4}{⼀,⼑}[HSK 7-9]
  \definition{adv.}{extremamente; muito}
  \synonymref{非常}{fei1chang2}
  \synonymref{极度}{ji2du4}
  \synonymref{极端}{ji2duan1}
  \synonymref{十分}{shi2fen1}
  \synonymref{特别}{te4bie2}
  \synonymref{异常}{yi4chang2}
\end{EntryWithPhonetic}

\begin{EntryWithPhonetic}{万福}{wan4fu2}{3,13}{⼀,⽰}
  \definition{s.}{(antigo) reverência feminina; reverência}
\end{EntryWithPhonetic}

\begin{EntryWithPhonetic}{万古长青}{wan4gu3-chang2qing1}{3,5,4,8}{⼀,⼝,⾧,⾭}[HSK 7-9]
  \definition{expr.}{seja perene; seja sempre verde; perene e eterno; sempre vivo; florescer para sempre; durar para sempre; permanecer fresco para sempre; sempre será verde como pinheiros e ciprestes por milhares de gerações; uma metáfora para um espírito nobre ou uma amizade profunda que nunca desaparecerá}
\end{EntryWithPhonetic}

\begin{EntryWithPhonetic}{万能}{wan4neng2}{3,10}{⼀,⾁}[HSK 7-9]
  \definition{adj.}{onipotente; todo"-poderoso | universal; multifacetado; de múltiplos usos}
  \synonymref{全能}{quan2neng2}
\end{EntryWithPhonetic}

\begin{EntryWithPhonetic}{万圣节}{wan4 sheng4 jie2}{3,5,5}{⼀,⼟,⾋}
  \definition*{s.}{Dia de Todos os Santos}
  \seealsoref{万圣节前夕}{wan4sheng4 jie2 qian2xi1}
\end{EntryWithPhonetic}

\begin{EntryWithPhonetic}{万圣节前夕}{wan4sheng4 jie2 qian2xi1}{3,5,5,9,3}{⼀,⼟,⾋,⼑,⼣}
  \definition*{s.}{Véspera do Dia de Todos os Santos | Halloween}
  \seealsoref{万圣节}{wan4 sheng4 jie2}
\end{EntryWithPhonetic}

\begin{EntryWithPhonetic}{万万}{wan4wan4}{3,3}{⼀,⼀}[HSK 7-9]
  \definition{adv.}{totalmente; absolutamente; em qualquer caso; não importa o que aconteça}
  \definition{num.}{cem milhões; 100.000.000; 1.0000.0000}
  \synonymref{绝对}{jue2dui4}
  \synonymref{千万}{qian1wan4}
  \synonymref{完全}{wan2quan2}
\end{EntryWithPhonetic}

\begin{EntryWithPhonetic}{万无一失}{wan4wu2-yi1shi1}{3,4,1,5}{⼀,⽆,⼀,⼤}[HSK 7-9]
  \definition{expr.}{não há perigo de algo dar errado; esteja do lado seguro; \dots não pode falhar em nenhuma circunstância; garantir sucesso total; nenhum risco; não há chance de erro; (faça com que seja mais do que provável que) nada dê errado; perfeitamente seguro; infalível; as chances são de mil para uma, não falharemos}
\end{EntryWithPhonetic}

\begin{EntryWithPhonetic}{万物}{wan4wu4}{3,8}{⼀,⽜}
  \definition{s.}{toda a criação; todas as coisas na Terra; todos os seres vivos; tudo no universo}
  \synonymref{生物}{sheng1wu4}
\end{EntryWithPhonetic}

\begin{EntryWithPhonetic}{万一}{wan4yi1}{3,1}{⼀,⼀}[HSK 4]
  \definition{conj.}{por via das dúvidas; se por acaso; só por precaução; expressa uma suposição muito improvável (usado para coisas desagradáveis)}
  \definition{num.}{um décimo milionésimo; uma porcentagem muito pequena}
  \definition{s.}{contingência; eventualidade; contingências muito improváveis}
  \synonymref{如果}{ru2guo3}
  \synonymref{要是}{yao4shi5}
  \synonymref{一旦}{yi2dan4}
\end{EntryWithPhonetic}

%%%%%%%%%% 蔓 %%%%%%%%%%
\subsection*{蔓}\addcontentsline{loh}{figure}{蔓 \dpy{wan4}}

\begin{EntryWithPhonetic}{蔓}{wan4}{14}{⾋}
  \definition*{s.}{Sobrenome: Wan}
  \definition{s.}{uma videira com gavinhas; caule fino que não consegue ficar em pé}
  \seeref{man2}
  \seeref{man4}
\end{EntryWithPhonetic}

%%%%%%%%%% 汪 %%%%%%%%%%
\subsection*{汪}\addcontentsline{loh}{figure}{汪 \dpy{wang1}}

\begin{EntryWithPhonetic}{汪}{wang1}{7}{⽔}
  \definition*{s.}{Sobrenome: Wang}
  \definition{adj.}{(um corpo de água) profundo e vasto}
  \definition{clas.}{utilizado para líquidos: piscina, poça}
  \definition{interj.}{Onomatopéia: latido}
  \definition{s.}{Dialeto: lagoa; piscina}
  \definition{v.}{(líquido) coletar; acumular | (líquido) exsudar; escorrer}
\end{EntryWithPhonetic}

\begin{EntryWithPhonetic}{汪洋}{wang1yang2}{7,9}{⽔,⽔}[HSK 7-9]
  \definition{adj.}{(um corpo de água) vasto; ilimitado; descreve a aparência de uma vastidão de água}
  \synonymref{海洋}{hai3yang2}
\end{EntryWithPhonetic}

%%%%%%%%%% 亡 %%%%%%%%%%
\subsection*{亡}\addcontentsline{loh}{figure}{亡 \dpy{wang2}}

\begin{EntryWithPhonetic}{亡}{wang2}{3}{⼇}
  \definition{adj.}{falecido}
  \definition{v.}{fugir; escapar | perder; ir embora; jogar fora | morrer; perecer; falecer | conquistar; subjugar | ser destruído; morrer}
  \synonymref{灭}{mie4}
  \synonymref{死}{si3}
  \synonymref{卒}{zu2}
  \antonymref{存}{cun2}
  \antonymref{兴}{xing1}
\end{EntryWithPhonetic}

\begin{EntryWithPhonetic}{亡羊补牢}{wang2yang2-bu3lao2}{3,6,7,7}{⼇,⽺,⾐,⼧}[HSK 7-9]
  \definition{expr.}{consertar a situação antes que seja tarde demais; agir tardiamente após a ocorrência de um acidente; reparar o curral depois que uma ovelha se perde é uma metáfora para encontrar maneiras de fazer as pazes depois de sofrer uma perda, de modo a evitar sofrer perdas novamente no futuro}
\end{EntryWithPhonetic}

%%%%%%%%%% 王 %%%%%%%%%%
\subsection*{王}\addcontentsline{loh}{figure}{王 \dpy{wang2}}

\begin{EntryWithPhonetic}{王}{wang2}{4}{⽟}[HSK 4][Kangxi 96]
  \definition*{s.}{Sobrenome: Wang}
  \definition{adj.}{grande; ótimo; honoríficos antigos para avós}
  \definition{s.}{rei; monarca; imperador; governante supremo de uma monarquia | cabeça; chefe; líder | o primeiro, maior ou mais forte de seu tipo | duque; príncipe; o título mais alto da sociedade feudal após a dinastia Han}
  \seeref{wang4}
\end{EntryWithPhonetic}

\begin{EntryWithPhonetic}{王八蛋}{wang2 ba1 dan4}{4,2,11}{⽟,⼋,⾍}
  \definition{s.}{bastardo; filho da puta; miserável}
  \synonymref{混蛋}{hun4dan4}
\end{EntryWithPhonetic}

\begin{EntryWithPhonetic}{王朝}{wang2chao2}{4,12}{⽟,⽉}
  \definition{s.}{dinastia}
\end{EntryWithPhonetic}

\begin{EntryWithPhonetic}{王国}{wang2guo2}{4,8}{⽟,⼞}[HSK 7-9]
  \definition[个]{s.}{reino; nação | reino; domínio; campo}
\end{EntryWithPhonetic}

\begin{EntryWithPhonetic}{王后}{wang2hou4}{4,6}{⽟,⼝}[HSK 6]
  \definition[个,位,名,些]{s.}{rainha consorte; rainha}
\end{EntryWithPhonetic}

\begin{EntryWithPhonetic}{王牌}{wang2pai2}{4,12}{⽟,⽚}[HSK 7-9]
  \definition[张,大]{s.}{trunfo; ás}
  \synonymref{绝招}{jue2zhao1}
  \synonymref{首席}{shou3xi2}
\end{EntryWithPhonetic}

\begin{EntryWithPhonetic}{王五}{wang2wu3}{4,4}{⽟,⼆}
  \definition{s.}{Wang Wu | Zé Ninguém | nome para uma pessoa não especificada, 3 de 3}
  \seealsoref{李四}{li3si4}
  \seealsoref{张三}{zhang1san1}
\end{EntryWithPhonetic}

\begin{EntryWithPhonetic}{王子}{wang2zi3}{4,3}{⽟,⼦}[HSK 6]
  \definition[位]{s.}{príncipe; filho do rei}
\end{EntryWithPhonetic}

%%%%%%%%%% 网 %%%%%%%%%%
\subsection*{网}\addcontentsline{loh}{figure}{网 \dpy{wang3}}

\begin{EntryWithPhonetic}{网}{wang3}{6}{⽹}[HSK 2][Kangxi 122]
  \definition[张]{s.}{rede; um dispositivo feito de corda ou barbante para capturar peixes e pássaros | algo que parece uma rede | rede; uma rede de organizações; um sistema}
  \definition{v.}{pegar com uma rede | cobrir como com uma rede}
\end{EntryWithPhonetic}

\begin{EntryWithPhonetic}{网吧}{wang3ba1}{6,7}{⽹,⼝}[HSK 6]
  \definition[家,间]{s.}{cybercafé; \emph{Internet} café; refere"-se a um local comercial aberto ao público que utiliza redes de computadores para fornecer serviços de navegação, consulta e outras informações}
\end{EntryWithPhonetic}

\begin{EntryWithPhonetic}{网点}{wang3dian3}{6,9}{⽹,⽕}[HSK 7-9]
  \definition{s.}{pontos de venda; rede (de estabelecimento comercial)}
\end{EntryWithPhonetic}

\begin{EntryWithPhonetic}{网罟}{wang3gu3}{6,10}{⽹,⽹}
  \definition{s.}{(fig.) a rede da justiça | rede usada para capturar peixes (ou outros animais, como pássaros)}
\end{EntryWithPhonetic}

\begin{EntryWithPhonetic}{网际网路}{wang3 ji4 wang3 lu4}{6,7,6,13}{⽹,⾩,⽹,⾜}
  \definition*{s.}{Internet}
  \seealsoref{互联网}{hu4lian2wang3}
  \seealsoref{网际网络}{wang3 ji4 wang3 luo4}
  \seealsoref{网路}{wang3 lu4}
\end{EntryWithPhonetic}

\begin{EntryWithPhonetic}{网际网络}{wang3 ji4 wang3 luo4}{6,7,6,9}{⽹,⾩,⽹,⽷}
  \definition*{s.}{Internet}
  \seealsoref{互联网}{hu4lian2wang3}
  \seealsoref{网际网路}{wang3 ji4 wang3 lu4}
  \seealsoref{网路}{wang3 lu4}
\end{EntryWithPhonetic}

\begin{EntryWithPhonetic}{网路}{wang3 lu4}{6,13}{⽹,⾜}
  \definition*{s.}{Internet}
  \seealsoref{互联网}{hu4lian2wang3}
  \seealsoref{网际网路}{wang3 ji4 wang3 lu4}
  \seealsoref{网际网络}{wang3 ji4 wang3 luo4}
\end{EntryWithPhonetic}

\begin{EntryWithPhonetic}{网络}{wang3luo4}{6,9}{⽹,⽷}[HSK 4]
  \definition{s.}{rede; um sistema que consiste em ramificações interconectadas; em um sistema elétrico, um circuito ou parte de um circuito que consiste em vários elementos que permitem a transmissão de sinais elétricos de acordo com determinados requisitos | rede; rede de computadores}
\end{EntryWithPhonetic}

\begin{EntryWithPhonetic}{网民}{wang3min2}{6,5}{⽹,⽒}[HSK 7-9]
  \definition[位,名]{s.}{internauta; geralmente se refere a um usuário de rede de computadores}
\end{EntryWithPhonetic}

\begin{EntryWithPhonetic}{网球}{wang3qiu2}{6,11}{⽹,⽟}[HSK 2]
  \definition[个,颗,些]{s.}{tênis (esporte) | bola de tênis}
\end{EntryWithPhonetic}

\begin{EntryWithPhonetic}{网上}{wang3shang4}{6,3}{⽹,⼀}[HSK 1]
  \definition{s.}{\emph{online}; refere"-se a acessar a \emph{Internet} através de um computador ou celular para pesquisar e consultar informações na rede}
\end{EntryWithPhonetic}

\begin{EntryWithPhonetic}{网上银行}{wang3shang4yin2hang2}{6,3,11,6}{⽹,⼀,⾦,⾏}
  \definition[个]{s.}{banco \emph{online} | acesso a operações bancárias via \emph{Internet}}
  \seealsoref{网银}{wang3yin2}
\end{EntryWithPhonetic}

\begin{EntryWithPhonetic}{网页}{wang3ye4}{6,6}{⽹,⾴}[HSK 6]
  \definition[个]{s.}{site; página da web; \emph{website}; \emph{web page}}
\end{EntryWithPhonetic}

\begin{EntryWithPhonetic}{网银}{wang3yin2}{6,11}{⽹,⾦}
  \definition{s.}{banco \emph{online} | acesso a operações bancárias via \emph{Internet}}
  \seealsoref{网上银行}{wang3shang4yin2hang2}
\end{EntryWithPhonetic}

\begin{EntryWithPhonetic}{网友}{wang3you3}{6,4}{⽹,⼜}[HSK 1]
  \definition{s.}{internauta; usuário da \emph{Internet}; amigos que se conhecem pela Internet; também usado como forma de tratamento entre internautas}
\end{EntryWithPhonetic}

\begin{EntryWithPhonetic}{网站}{wang3zhan4}{6,10}{⽹,⽴}[HSK 2]
  \definition[个,家]{s.}{\emph{web}; \emph{website}; um site virtual na Internet para uma organização ou indivíduo, geralmente consistindo em uma página inicial e muitas páginas da web}
\end{EntryWithPhonetic}

\begin{EntryWithPhonetic}{网址}{wang3zhi3}{6,7}{⽹,⼟}[HSK 4]
  \definition[个]{s.}{\emph{website}; endereço da \emph{web}; endereço de um \emph{site} na \emph{Internet}, que os usuários podem acessar, consultar e obter recursos de informações nesse \emph{site} clicando nele}
\end{EntryWithPhonetic}

%%%%%%%%%% 往 %%%%%%%%%%
\subsection*{往}\addcontentsline{loh}{figure}{往 \dpy{wang3}}

\begin{EntryWithPhonetic}{往}{wang3}{8}{⼻}[HSK 2]
  \definition{adj.}{passado; anterior}
  \definition{prep.}{para; em direção a; na direção de}
  \definition{v.}{ir}
\end{EntryWithPhonetic}

\begin{EntryWithPhonetic}{往常}{wang3chang2}{8,11}{⼻,⼱}[HSK 7-9]
  \definition{s.}{habitual; normal; como sempre}
  \synonymref{平常}{ping2chang2}
  \synonymref{平时}{ping2shi2}
  \synonymref{以前}{yi3qian2}
  \antonymref{如今}{ru2jin1}
\end{EntryWithPhonetic}

\begin{EntryWithPhonetic}{往程}{wang3cheng2}{8,12}{⼻,⽲}
  \definition{s.}{saída (de uma viagem de ônibus ou trem, etc.)}
\end{EntryWithPhonetic}

\begin{EntryWithPhonetic}{往返}{wang3fan3}{8,7}{⼻,⾡}[HSK 7-9]
  \definition{v.}{transportar; ir e voltar; viajar de e para}
  \synonymref{来回}{lai2hui2}
  \synonymref{往复}{wang3fu4}
  \synonymref{往来}{wang3lai2}
\end{EntryWithPhonetic}

\begin{EntryWithPhonetic}{往复}{wang3fu4}{8,9}{⼻,⼢}
  \definition{s.}{para trás e para frente (por exemplo, da ação do pistão ou da bomba)}
  \definition{v.}{ir e voltar | fazer uma viagem de volta}
\end{EntryWithPhonetic}

\begin{EntryWithPhonetic}{往后}{wang3hou4}{8,6}{⼻,⼝}[HSK 6]
  \definition{s.}{de agora em diante; mais tarde; no futuro | na parte traseira; na parte de trás | para trás; depois; à ré}
\end{EntryWithPhonetic}

\begin{EntryWithPhonetic}{往迹}{wang3ji4}{8,9}{⼻,⾡}
  \definition{s.}{Literário: eventos passados; coisa do passado; tempos antigos}
\end{EntryWithPhonetic}

\begin{EntryWithPhonetic}{往来}{wang3lai2}{8,7}{⼻,⽊}[HSK 6]
  \definition{s.}{contatos comerciais; relações comerciais; relações diplomáticas | negociações; visitas mútuas; comunicação}
  \definition{v.}{ir e vir | contatar; ter relações}
\end{EntryWithPhonetic}

\begin{EntryWithPhonetic}{往例}{wang3li4}{8,8}{⼻,⼈}
  \definition{s.}{prática (habitual) do passado | precedente}
\end{EntryWithPhonetic}

\begin{EntryWithPhonetic}{往年}{wang3nian2}{8,6}{⼻,⼲}[HSK 6]
  \definition{s.}{(em) anos anteriores}
\end{EntryWithPhonetic}

\begin{EntryWithPhonetic}{往日}{wang3ri4}{8,4}{⼻,⽇}[HSK 7-9]
  \definition{adv.}{(nos) dias anteriores; (nos) últimos dias; (em) dias passados}
  \definition{s.}{o passado}
  \synonymref{昔日}{xi1ri4}
\end{EntryWithPhonetic}

\begin{EntryWithPhonetic}{往生}{wang3sheng1}{8,5}{⼻,⽣}
  \definition{v.}{renascer | morrer | (Budismo) viver no paraíso}
\end{EntryWithPhonetic}

\begin{EntryWithPhonetic}{往事}{wang3shi4}{8,8}{⼻,⼅}[HSK 7-9]
  \definition[件,段]{s.}{eventos passados; o passado; coisas passadas}
\end{EntryWithPhonetic}

\begin{EntryWithPhonetic}{往往}{wang3wang3}{8,8}{⼻,⼻}[HSK 3]
  \definition{adv.}{frequentemente; muitas vezes; mais frequentemente do que não; indica que uma situação existe ou ocorre com frequência}
\end{EntryWithPhonetic}

\begin{EntryWithPhonetic}{往昔}{wang3xi1}{8,8}{⼻,⽇}
  \definition{s.}{o passado}
\end{EntryWithPhonetic}

%%%%%%%%%% 罔 %%%%%%%%%%
\subsection*{罔}\addcontentsline{loh}{figure}{罔 \dpy{wang3}}

\begin{EntryWithPhonetic}{罔}{wang3}{8}{⼌}
  \definition{v.}{enganar}
\end{EntryWithPhonetic}

%%%%%%%%%% 王 %%%%%%%%%%
\subsection*{王}\addcontentsline{loh}{figure}{王 \dpy{wang4}}

\begin{EntryWithPhonetic}{王}{wang4}{4}{⽟}[Kangxi 96]
  \definition{v.}{reger; governar; reinar; dominar}
  \seeref{wang2}
\end{EntryWithPhonetic}

%%%%%%%%%% 妄 %%%%%%%%%%
\subsection*{妄}\addcontentsline{loh}{figure}{妄 \dpy{wang4}}

\begin{EntryWithPhonetic}{妄}{wang4}{6}{⼥}
  \definition{adj.}{absurdo; absurdo | ultrajante; ridículo e irracional | precipitado; irresponsável; presunçoso; irracional; fora da rotina; aleatório}
\end{EntryWithPhonetic}

\begin{EntryWithPhonetic}{妄想}{wang4xiang3}{6,13}{⼥,⼼}[HSK 7-9]
  \definition{s.}{ilusão; vã esperança; idéias falsas e irrealizáveis}
  \definition{v.}{fazer uma tentativa vã de; esperar em vão fazer algo; planos que não podem ser realizados}
  \synonymref{梦想}{meng4xiang3}
\end{EntryWithPhonetic}

%%%%%%%%%% 忘 %%%%%%%%%%
\subsection*{忘}\addcontentsline{loh}{figure}{忘 \dpy{wang4}}

\begin{EntryWithPhonetic}{忘}{wang4}{7}{⼼}[HSK 1]
  \definition{v.}{esquecer | ignorar; negligenciar}
\end{EntryWithPhonetic}

\begin{EntryWithPhonetic}{忘本}{wang4ben3}{7,5}{⼼,⽊}
  \definition{v.}{esquecer as próprias raízes}
\end{EntryWithPhonetic}

\begin{EntryWithPhonetic}{忘不了}{wang4 bu5 liao3}{7,4,2}{⼼,⼀,⼅}[HSK 7-9]
  \definition{v.}{não poder esquecer}
\end{EntryWithPhonetic}

\begin{EntryWithPhonetic}{忘餐}{wang4can1}{7,16}{⼼,⾷}
  \definition{v.}{esquecer as refeições}
\end{EntryWithPhonetic}

\begin{EntryWithPhonetic}{忘掉}{wang4/diao4}{7,11}{⼼,⼿}[HSK 7-9]
  \definition{v.+compl.}{esquecer; deixar escapar da mente; não se lembrar}
  \synonymref{忘怀}{wang4huai2}
  \synonymref{忘记}{wang4ji4}
  \synonymref{忘却}{wang4que4}
  \antonymref{记住}{ji4 zhu5}
\end{EntryWithPhonetic}

\begin{EntryWithPhonetic}{忘恩}{wang4'en1}{7,10}{⼼,⼼}
  \definition{v.}{ser ingrato}
\end{EntryWithPhonetic}

\begin{EntryWithPhonetic}{忘怀}{wang4huai2}{7,7}{⼼,⼼}
  \definition{v.}{esquecer}
\end{EntryWithPhonetic}

\begin{EntryWithPhonetic}{忘记}{wang4ji4}{7,5}{⼼,⾔}[HSK 1]
  \definition{v.}{esquecer | ignorar; negligenciar | sair da memória de alguém; não ser lembrado | descartar da mente; ignorar}
\end{EntryWithPhonetic}

\begin{EntryWithPhonetic}{忘却}{wang4que4}{7,7}{⼼,⼙}
  \definition{v.}{esquecer}
\end{EntryWithPhonetic}

%%%%%%%%%% 旺 %%%%%%%%%%
\subsection*{旺}\addcontentsline{loh}{figure}{旺 \dpy{wang4}}

\begin{EntryWithPhonetic}{旺}{wang4}{8}{⽇}[HSK 7-9]
  \definition{adj.}{próspero; florescente; vigoroso | abundante; numeroso}
\end{EntryWithPhonetic}

\begin{EntryWithPhonetic}{旺季}{wang4ji4}{8,8}{⽇,⼦}[HSK 7-9]
  \definition{s.}{alta temporada; período de pico; temporada movimentada; a estação em que um determinado produto é produzido em grandes quantidades ou quando os negócios estão crescendo}
  \seealsoref{淡季}{dan4ji4}
  \antonymref{淡季}{dan4ji4}
\end{EntryWithPhonetic}

\begin{EntryWithPhonetic}{旺盛}{wang4sheng4}{8,11}{⽇,⽫}[HSK 7-9]
  \definition{adj.}{vigoroso; exuberante; de forte vitalidade; de bom humor}
  \synonymref{繁华}{fan2hua2}
  \synonymref{繁荣}{fan2rong2}
  \synonymref{红火}{hong2huo5}
  \synonymref{焕发}{huan4fa1}
  \synonymref{茂盛}{mao4sheng4}
  \synonymref{蓬勃}{peng2bo2}
  \synonymref{兴旺}{xing1wang4}
  \antonymref{衰退}{shuai1tui4}
\end{EntryWithPhonetic}

%%%%%%%%%% 望 %%%%%%%%%%
\subsection*{望}\addcontentsline{loh}{figure}{望 \dpy{wang4}}

\begin{EntryWithPhonetic}{望}{wang4}{11}{⽉}[HSK 7-9]
  \definition*{s.}{Sobrenome: Wang}
  \definition{prep.}{para; em direção a; em ``olhando para frente (望前看)'', ``caminhando para o leste (望东走)'', etc.; 望 é frequentemente escrito como 往}
  \definition{s.}{prestígio; reputação; fama | lua cheia | o 15º dia de um mês lunar}
  \definition{v.}{olhar por cima; olhar para a distância; olhar para longe na distância | visitar; ligar para | ter esperança; esperar | odiar; ressentir"-se | pensar em atingir um determinado objetivo ou uma determinada situação em mente}
  \seealsoref{往}{wang3}
  \synonymref{看}{kan4}
  \synonymref{瞧}{qiao2}
  \synonymref{视}{shi4}
\end{EntryWithPhonetic}

\begin{EntryWithPhonetic}{望见}{wang4jian4}{11,4}{⽉,⾒}[HSK 6]
  \definition{v.}{espiar; ver; pôr os olhos em | detectar}
\end{EntryWithPhonetic}

\begin{EntryWithPhonetic}{望远镜}{wang4yuan3jing4}{11,7,16}{⽉,⾡,⾦}[HSK 7-9]
  \definition[架,个,付,副,部]{s.}{telescópio; o telescópio refrator mais simples consiste em dois conjuntos de lentes | binóculos; um instrumento óptico para observar objetos distantes}
\end{EntryWithPhonetic}

%%%%%%%%%% 危 %%%%%%%%%%
\subsection*{危}\addcontentsline{loh}{figure}{危 \dpy{wei1}}

\begin{EntryWithPhonetic}{危}{wei1}{6}{⼙}
  \definition*{s.}{Wei, a décima segunda das vinte e oito constelações em que a esfera celeste foi dividida, consistindo de três estrelas em forma de triângulo obtuso, uma em Aquário e duas em Pégaso | Wei, uma das mansões lunares | Sobrenome: Wei}
  \definition{adj.}{arriscado; inseguro; perigoso | estar gravemente doente; estar morrendo | alto; íngreme}
  \definition{s.}{perigo | cumeeira (de um telhado)}
  \definition{v.}{pôr em perigo; colocar em perigo; comprometer}
  \antonymref{安}{an1}
\end{EntryWithPhonetic}

\begin{EntryWithPhonetic}{危害}{wei1hai4}{6,10}{⼙,⼧}[HSK 3]
  \definition[个,种]{s.}{prejuízo; perigo; dano}
  \definition{v.}{destruir; prejudicar; pôr em perigo; pôr em risco}
\end{EntryWithPhonetic}

\begin{EntryWithPhonetic}{危机}{wei1ji1}{6,6}{⼙,⽊}[HSK 6]
  \definition[个,次]{s.}{crise}
\end{EntryWithPhonetic}

\begin{EntryWithPhonetic}{危及}{wei1ji2}{6,3}{⼙,⼃}[HSK 7-9]
  \definition{v.}{prejudicar; colocar em perigo; comprometer; ameaçar}
\end{EntryWithPhonetic}

\begin{EntryWithPhonetic}{危急}{wei1ji2}{6,9}{⼙,⼼}[HSK 7-9]
  \definition{adj.}{crítico; em perigo iminente; em uma situação desesperadora; perigoso e urgente}
  \synonymref{风险}{feng1xian3}
  \synonymref{紧急}{jin3ji2}
  \synonymref{紧迫}{jin3po4}
  \synonymref{紧张}{jin3zhang1}
  \synonymref{迫切}{po4qie4}
  \synonymref{危害}{wei1hai4}
  \synonymref{危机}{wei1ji1}
  \synonymref{危险}{wei1xian3}
  \synonymref{严重}{yan2zhong4}
  \antonymref{安全}{an1quan2}
  \antonymref{安稳}{an1wen3}
\end{EntryWithPhonetic}

\begin{EntryWithPhonetic}{危难}{wei1nan4}{6,10}{⼙,⾫}
  \definition{s.}{calamidade}
\end{EntryWithPhonetic}

\begin{EntryWithPhonetic}{危险}{wei1xian3}{6,9}{⼙,⾩}[HSK 3]
  \definition{adj.}{arriscado; perigoso}
\end{EntryWithPhonetic}

%%%%%%%%%% 委 %%%%%%%%%%
\subsection*{委}\addcontentsline{loh}{figure}{委 \dpy{wei1}}

\begin{EntryWithPhonetic}{委}{wei1}{8}{⼥}
  \definition{adj./adv.}{o mesmo que 逶 em 逶迤 sinuoso, curvo}
  \seeref{wei3}
  \seealsoref{逶}{wei1}
  \seealsoref{逶迤}{wei1yi2}
\end{EntryWithPhonetic}

%%%%%%%%%% 威 %%%%%%%%%%
\subsection*{威}\addcontentsline{loh}{figure}{威 \dpy{wei1}}

\begin{EntryWithPhonetic}{威}{wei1}{9}{⼥}
  \definition*{s.}{Sobrenome: Wei}
  \definition{adj.}{forte; poderoso}
  \definition{s.}{força impressionante; poder; força}
  \definition{v.}{ameaçar pela força; intimidar com força}
\end{EntryWithPhonetic}

\begin{EntryWithPhonetic}{威风}{wei1feng1}{9,4}{⼥,⾵}[HSK 7-9]
  \definition{adj.}{imponente; impressionante; inspirador | de influência dominadora; de alto prestígio apoiado em poder; de momento ou estilo inspirador | de aparência majestosa; majestoso}
  \definition{s.}{poder; prestígio}
  \synonymref{威信}{wei1xin4}
\end{EntryWithPhonetic}

\begin{EntryWithPhonetic}{威力}{wei1li4}{9,2}{⼥,⼒}[HSK 7-9]
  \definition{s.}{poder; vigor; força}
  \synonymref{魄力}{po4li4}
  \synonymref{气魄}{qi4po4}
  \synonymref{气势}{qi4shi4}
\end{EntryWithPhonetic}

\begin{EntryWithPhonetic}{威慑}{wei1she4}{9,13}{⼥,⼼}[HSK 7-9]
  \definition{v.}{aterrorizar com força; dissuadir | acovardar"-se pela força militar; deter}
  \synonymref{恐吓}{kong3he4}
  \synonymref{威胁}{wei1xie2}
\end{EntryWithPhonetic}

\begin{EntryWithPhonetic}{威胁}{wei1xie2}{9,8}{⼥,⾁}[HSK 6]
  \definition{v.}{pôr em perigo; ameaçar; intimidar}
\end{EntryWithPhonetic}

\begin{EntryWithPhonetic}{威信}{wei1xin4}{9,9}{⼥,⼈}[HSK 7-9]
  \definition{s.}{prestígio; confiança popular; prestígio e credibilidade}
  \synonymref{威风}{wei1feng1}
\end{EntryWithPhonetic}

%%%%%%%%%% 萎 %%%%%%%%%%
\subsection*{萎}\addcontentsline{loh}{figure}{萎 \dpy{wei1}}

\begin{EntryWithPhonetic}{萎}{wei1}{11}{⾋}
  \definition{v.}{murchar; cair}
  \seeref{wei3}
  \synonymref{缩}{suo1}
  \synonymref{谢}{xie4}
\end{EntryWithPhonetic}

%%%%%%%%%% 逶 %%%%%%%%%%
\subsection*{逶}\addcontentsline{loh}{figure}{逶 \dpy{wei1}}

\begin{EntryWithPhonetic}{逶}{wei1}{11}{⾡}
  \definition{adj.}{sinuoso; tortuoso}
\end{EntryWithPhonetic}

\begin{EntryWithPhonetic}{逶迤}{wei1yi2}{11,8}{⾡,⾡}
  \definition{adj.}{sinuoso; tortuoso; descreve a aparência sinuosa e contínua de estradas, montanhas, rios, etc.}
\end{EntryWithPhonetic}

%%%%%%%%%% 微 %%%%%%%%%%
\subsection*{微}\addcontentsline{loh}{figure}{微 \dpy{wei1}}

\begin{EntryWithPhonetic}{微}{wei1}{13}{⼻}
  \definition{adj.}{minúsculo; leve | profundo; abstruso | humilde; tendo pouca influência; baixo \emph{status}}
  \definition{adv.}{pouco; ligeiramente; indica um grau menor, equivalente a 稍 ou 略}
  \definition{num.}{um milionésimo de uma determinada unidade de medida}
  \definition{suf.}{micro-}
  \seealsoref{略}{lve4}
  \seealsoref{稍}{shao1}
\end{EntryWithPhonetic}

\begin{EntryWithPhonetic}{微波炉}{wei1bo1lu2}{13,8,8}{⼻,⽔,⽕}[HSK 6]
  \definition[台,个]{s.}{forno de micro-ondas}
\end{EntryWithPhonetic}

\begin{EntryWithPhonetic}{微博}{wei1bo2}{13,12}{⼻,⼗}[HSK 5]
  \definition*{s.}{Weibo (um aplicativo de mídia social chinês)}
  \definition[条]{s.}{\emph{microblog}; abreviação de 微型博客}
  \seealsoref{微型博客}{wei1xing2 bo2ke4}
\end{EntryWithPhonetic}

\begin{EntryWithPhonetic}{微不足道}{wei1bu4zu2dao4}{13,4,7,12}{⼻,⼀,⾜,⾡}[HSK 7-9]
  \definition{expr.}{muito trivial ou insignificante para ser mencionado; insignificante; muito trivial, não vale a pena comentar}
\end{EntryWithPhonetic}

\begin{EntryWithPhonetic}{微风}{wei1feng1}{13,4}{⼻,⾵}
  \definition{s.}{brisa | vento leve}
\end{EntryWithPhonetic}

\begin{EntryWithPhonetic}{微观}{wei1guan1}{13,6}{⼻,⾒}[HSK 7-9]
  \definition{adj.}{microscópico; microcósmico}
  \antonymref{宏观}{hong2guan1}
\end{EntryWithPhonetic}

\begin{EntryWithPhonetic}{微妙}{wei1miao4}{13,7}{⼻,⼥}[HSK 7-9]
  \definition{adj.}{delicado; sutil; esotérico; indescritível}
  \synonymref{奥秘}{ao4mi4}
  \synonymref{奇妙}{qi2miao4}
  \antonymref{豁达}{huo4da2}
  \antonymref{突出}{tu1/chu1}
  \antonymref{显著}{xian3zhu4}
\end{EntryWithPhonetic}

\begin{EntryWithPhonetic}{微软}{wei1ruan3}{13,8}{⼻,⾞}
  \definition*{s.}{Microsoft Corporation}
\end{EntryWithPhonetic}

\begin{EntryWithPhonetic}{微弱}{wei1ruo4}{13,10}{⼻,⼸}[HSK 7-9]
  \definition{adj.}{fraco}
  \synonymref{薄弱}{bo2ruo4}
  \synonymref{单薄}{dan1bo2}
  \synonymref{轻微}{qing1wei1}
  \synonymref{衰弱}{shuai1ruo4}
  \synonymref{微小}{wei1xiao3}
  \antonymref{猛烈}{meng3lie4}
  \antonymref{强大}{qiang2da4}
  \antonymref{强劲}{qiang2jing4}
  \antonymref{强烈}{qiang2lie4}
\end{EntryWithPhonetic}

\begin{EntryWithPhonetic}{微小}{wei1xiao3}{13,3}{⼻,⼩}
  \definition{adj.}{pequeno; minúsculo; diminuto}
  \definition{s.}{vírus de RNA}
\end{EntryWithPhonetic}

\begin{EntryWithPhonetic}{微笑}{wei1xiao4}{13,10}{⼻,⽵}[HSK 4]
  \definition[个,丝]{s.}{sorriso; sorriso sutil}
  \definition{v.}{sorrir; rir baixinho e sutilmente}
\end{EntryWithPhonetic}

\begin{EntryWithPhonetic}{微信}{wei1xin4}{13,9}{⼻,⼈}[HSK 4]
  \definition*[个,条]{s.}{WeChat, aplicativo gratuito lançado pela Tencent em 21 de janeiro de 2011 para fornecer serviços de mensagens instantâneas para terminais inteligentes}
\end{EntryWithPhonetic}

\begin{EntryWithPhonetic}{微型}{wei1xing2}{13,9}{⼻,⼟}[HSK 7-9]
  \definition{adj.}{minúsculo}
  \definition{pref.}{micro-; mini-}
  \definition{s.}{miniatura; microescala}
  \synonymref{小型}{xiao3xing2}
  \antonymref{大型}{da4xing2}
  \antonymref{巨型}{ju4xing2}
\end{EntryWithPhonetic}

\begin{EntryWithPhonetic}{微型博客}{wei1xing2 bo2ke4}{13,9,12,9}{⼻,⼟,⼗,⼧}
  \definition{s.}{\emph{microblog}}
\end{EntryWithPhonetic}

%%%%%%%%%% 为 %%%%%%%%%%
\subsection*{为}\addcontentsline{loh}{figure}{为 \dpy{wei2}}

\begin{EntryWithPhonetic}{为}{wei2}{4}{⼂}[HSK 3]
  \definition*{s.}{Sobrenome: Wei}
  \definition{part.}{frequentemente usado com 何 em uma pergunta retórica}
  \definition{prep.}{por; usado em frases passivas para introduzir o agente da ação, equivalente a 被 (frequentemente usado com 所)}
  \definition{suf.}{é anexado a alguns adjetivos ou advérbios monossilábicos para formar advérbios dissilábicos que expressam grau ou amplitude, geralmente modificando adjetivos ou verbos dissilábicos}
  \definition{v.}{fazer; agir | tornar"-se; transformar"-se em | ser; significar | servir como; agir como; desempenhar o papel de | fazer; trabalhar; indica certas ações e comportamentos, incluindo os significados de governança, engajamento, cenário e pesquisa}
  \seeref{wei4}
  \seealsoref{被}{bei4}
  \seealsoref{何}{he2}
  \seealsoref{所}{suo3}
  \synonymref{替}{ti4}
\end{EntryWithPhonetic}

\begin{EntryWithPhonetic}{为难}{wei2nan2}{4,10}{⼂,⾫}[HSK 5]
  \definition{adj.}{envergonhado; sentir"-se constrangido; sentir"-se sobrecarregado; sentir"-se incapaz de lidar com algo}
  \definition{v.}{dificultar as coisas para; dificultar; contrariar}
  \synonymref{刁难}{diao1nan4}
  \synonymref{尴尬}{gan1ga4}
  \antonymref{乐意}{le4yi4}
  \antonymref{愿意}{yuan4yi5}
\end{EntryWithPhonetic}

\begin{EntryWithPhonetic}{为期}{wei2qi1}{4,12}{⼂,⽉}[HSK 5]
  \definition{s.}{Literário: tempo restante}
  \definition{v.}{Literário: a ser concluído (até uma data definida, por um determinado período de tempo)}
\end{EntryWithPhonetic}

\begin{EntryWithPhonetic}{为人}{wei2ren2}{4,2}{⼂,⼈}[HSK 7-9]
  \definition[方]{s.}{comportamento; conduta; atitude em relação aos outros e às coisas}
  \definition{v.}{comportar"-se; conduzir"-se}
  \synonymref{君子}{jun1zi3}
  \synonymref{人品}{ren2pin3}
\end{EntryWithPhonetic}

\begin{EntryWithPhonetic}{为止}{wei2zhi3}{4,4}{⼂,⽌}[HSK 5]
  \definition{adv.}{até; até um determinado momento}
  \synonymref{截止}{jie2zhi3}
  \synonymref{截至}{jie2zhi4}
\end{EntryWithPhonetic}

\begin{EntryWithPhonetic}{为主}{wei2zhu3}{4,5}{⼂,⼂}[HSK 5]
  \definition{v.}{dar prioridade a; dar preferência a; dar importância a}
\end{EntryWithPhonetic}

%%%%%%%%%% 围 %%%%%%%%%%
\subsection*{围}\addcontentsline{loh}{figure}{围 \dpy{wei2}}

\begin{EntryWithPhonetic}{围}{wei2}{7}{⼞}[HSK 3]
  \definition*{s.}{Sobrenome: Wei}
  \definition{clas.}{o comprimento das duas mãos com os polegares e os dedos indicadores juntos ou dos dois braços juntos}
  \definition{s.}{em volta de tudo; ao redor}
  \definition{v.}{cercar; rodear; circundar; encurralar; cercar tudo, impedindo a passagem entre o interior e o exterior | envolver; contornar}
\end{EntryWithPhonetic}

\begin{EntryWithPhonetic}{围巾}{wei2jin1}{7,3}{⼞,⼱}[HSK 4]
  \definition[条]{s.}{lenço; cachecol; echarpe; gravata; tiras longas de malha ou tecido usadas ao redor do pescoço para aquecimento, proteção do colarinho ou decoração}
\end{EntryWithPhonetic}

\begin{EntryWithPhonetic}{围墙}{wei2qiang2}{7,14}{⼞,⼟}[HSK 7-9]
  \definition[堵,道,面]{s.}{recinto; parede envolvente; construído em torno de uma casa, jardim, pátio, etc.; uma parede que serve de barreira}
\end{EntryWithPhonetic}

\begin{EntryWithPhonetic}{围绕}{wei2rao4}{7,9}{⼞,⽷}[HSK 5]
  \definition{v.}{girar; circundar; dar voltas; girar em torno de algo; cercar | concentrar-se em; centrar-se em; centrar-se em uma questão ou evento (para realizar atividades)}
\end{EntryWithPhonetic}

%%%%%%%%%% 违 %%%%%%%%%%
\subsection*{违}\addcontentsline{loh}{figure}{违 \dpy{wei2}}

\begin{EntryWithPhonetic}{违}{wei2}{7}{⾡}
  \definition{v.}{desobedecer; violar | ser separado; separar"-se de | desafiar; não cumprir; não obedecer}
\end{EntryWithPhonetic}

\begin{EntryWithPhonetic}{违背}{wei2bei4}{7,9}{⾡,⾁}[HSK 7-9]
  \definition{v.}{violar; infringir; ir contra; correr contra; desviar}
  \synonymref{违反}{wei2fan3}
  \antonymref{按照}{an4zhao4}
  \antonymref{服从}{fu2cong2}
  \antonymref{符合}{fu2he2}
  \antonymref{顺从}{shun4cong2}
  \antonymref{遵守}{zun1shou3}
\end{EntryWithPhonetic}

\begin{EntryWithPhonetic}{违法}{wei2 fa3}{7,8}{⾡,⽔}[HSK 5]
  \definition{v.}{ser ilegal; infringir a lei; violar a lei ou os regulamentos}
\end{EntryWithPhonetic}

\begin{EntryWithPhonetic}{违反}{wei2fan3}{7,4}{⾡,⼜}[HSK 5]
  \definition{v.}{violar; transgredir; contrariar; não estar em conformidade (com as regras, regulamentos, etc.)}
\end{EntryWithPhonetic}

\begin{EntryWithPhonetic}{违规}{wei2 gui1}{7,8}{⾡,⾒}[HSK 5]
  \definition{v.}{violar (regras); infringir as regras e regulamentos}
\end{EntryWithPhonetic}

\begin{EntryWithPhonetic}{违宪}{wei2xian4}{7,9}{⾡,⼧}
  \definition{adj.}{inconstitucional}
\end{EntryWithPhonetic}

\begin{EntryWithPhonetic}{违约}{wei2/yue1}{7,6}{⾡,⽷}[HSK 7-9]
  \definition{v.+compl.}{quebrar uma promessa; violar um acordo; ter inadimplência (em um empréstimo ou contrato); violar ou deixar de cumprir um acordo mútuo}
  \antonymref{契约}{qi4yue1}
\end{EntryWithPhonetic}

\begin{EntryWithPhonetic}{违章}{wei2/zhang1}{7,11}{⾡,⾳}[HSK 7-9]
  \definition{v.+compl.}{quebrar regras e regulamentos}
\end{EntryWithPhonetic}

%%%%%%%%%% 唯 %%%%%%%%%%
\subsection*{唯}\addcontentsline{loh}{figure}{唯 \dpy{wei2}}

\begin{EntryWithPhonetic}{唯}{wei2}{11}{⼝}[HSK 7-9]
  \definition{adv.}{somente; sozinho | ainda; somente; exceto que}
  \seeref{wei3}
\end{EntryWithPhonetic}

\begin{EntryWithPhonetic}{唯独}{wei2du2}{11,9}{⼝,⽝}[HSK 7-9]
  \definition{adv.}{apenas; sozinho; pode ser usado antes de frases verbais, significando 只 ou 仅仅; também pode ser usado antes de uma frase sujeito"-predicado, geralmente no início de uma frase, significando 只有; às vezes pode ser colocado diretamente antes do substantivo ou pronome, mas precisa ser seguido por um verbo e uma cláusula sexual}
  \seealsoref{仅仅}{jin3jin3}
  \seealsoref{只}{zhi3}
  \synonymref{唯一}{wei2yi1}
  \synonymref{只有}{zhi3you3}
  \antonymref{众多}{zhong4duo1}
\end{EntryWithPhonetic}

\begin{EntryWithPhonetic}{唯一}{wei2yi1}{11,1}{⼝,⼀}[HSK 5]
  \definition{adj.}{único; exclusivo; singular; apenas um; nenhum outro}
\end{EntryWithPhonetic}

%%%%%%%%%% 维 %%%%%%%%%%
\subsection*{维}\addcontentsline{loh}{figure}{维 \dpy{wei2}}

\begin{EntryWithPhonetic}{维}{wei2}{11}{⽷}
  \definition*{s.}{Sobrenome: Wei}
  \definition{s.}{pensamento | dimensão; conceitos básicos de geometria e teoria do espaço}
  \definition{v.}{ligar; amarrar; manter unido; conectar | manter; manter; salvaguardar; preservar}
\end{EntryWithPhonetic}

\begin{EntryWithPhonetic}{维持}{wei2chi2}{11,9}{⽷,⼿}[HSK 4]
  \definition{v.}{manter; conservar; guardar; manter vivo}
\end{EntryWithPhonetic}

\begin{EntryWithPhonetic}{维护}{wei2hu4}{11,7}{⽷,⼿}[HSK 4]
  \definition{v.}{defender; proteger; manter; preservar}
\end{EntryWithPhonetic}

\begin{EntryWithPhonetic}{维生素}{wei2sheng1su4}{11,5,10}{⽷,⽣,⽷}[HSK 6]
  \definition[点]{s.}{vitamina}[西瓜中含丰富的维生素。===A melancia é rica em vitaminas.]
\end{EntryWithPhonetic}

\begin{EntryWithPhonetic}{维吾尔}{wei2wu2'er3}{11,7,5}{⽷,⼝,⼩}
  \definition*{s.}{Etnia Uigur de Xinjiang}
\end{EntryWithPhonetic}

\begin{EntryWithPhonetic}{维修}{wei2xiu1}{11,9}{⽷,⼈}[HSK 4]
  \definition{v.}{prestar serviços; manter; reparar; manter em (bom) estado de conservação}
\end{EntryWithPhonetic}

%%%%%%%%%% 伟 %%%%%%%%%%
\subsection*{伟}\addcontentsline{loh}{figure}{伟 \dpy{wei3}}

\begin{EntryWithPhonetic}{伟}{wei3}{6}{⼈}
  \definition*{s.}{Sobrenome: Wei}
  \definition{adj.}{grande; ótimo; poderoso | Literário: grande}
\end{EntryWithPhonetic}

\begin{EntryWithPhonetic}{伟大}{wei3da4}{6,3}{⼈,⼤}[HSK 3]
  \definition{adj.}{ótimo; excelente; extrovertido; descreve uma pessoa com moral e qualidades excelentes, habilidades e realizações excepcionais, que inspira grande respeito | ótimo; magnífico; descreve algo de grande importância, com impacto significativo, acima do normal, algo notável}
  \synonymref{崇高}{chong2gao1}
  \synonymref{广大}{guang3da4}
  \synonymref{宏大}{hong2da4}
  \synonymref{宏伟}{hong2wei3}
  \synonymref{巨大}{ju4da4}
  \synonymref{庞大}{pang2da4}
  \synonymref{雄伟}{xiong2wei3}
  \antonymref{渺小}{miao3xiao3}
  \antonymref{平凡}{ping2fan2}
\end{EntryWithPhonetic}

%%%%%%%%%% 伪 %%%%%%%%%%
\subsection*{伪}\addcontentsline{loh}{figure}{伪 \dpy{wei3}}

\begin{EntryWithPhonetic}{伪}{wei3}{6}{⼈}
  \definition{adj.}{falso; falsificado | fantoche; colaboracionista; ilegal | forjado; falso}
  \definition{pref.}{pseudo-; quasi-; quase-}
  \synonymref{假}{jia3}
  \antonymref{真}{zhen1}
\end{EntryWithPhonetic}

\begin{EntryWithPhonetic}{伪造}{wei3zao4}{6,10}{⼈,⾡}[HSK 7-9]
  \definition{v.}{forjar; falsificar; fabricar; fingir}
\end{EntryWithPhonetic}

\begin{EntryWithPhonetic}{伪装}{wei3zhuang1}{6,12}{⼈,⾐}[HSK 7-9]
  \definition[种,个]{s.}{máscara; disfarce; camuflagem; aparência fingida; algo usado como disfarce}
  \definition{v.}{ser falso; fingir; camuflar; usar certos meios secretos em assuntos militares para enganar e confundir o inimigo}
  \synonymref{假装}{jia3zhuang1}
  \antonymref{真诚}{zhen1cheng2}
\end{EntryWithPhonetic}

%%%%%%%%%% 尾 %%%%%%%%%%
\subsection*{尾}\addcontentsline{loh}{figure}{尾 \dpy{wei3}}

\begin{EntryWithPhonetic}{尾}{wei3}{7}{⼫}
  \definition*{s.}{Wei, sexta das vinte e oito constelações nas quais a esfera celeste foi dividida, consistindo de nove estrelas em forma de gancho em Escorpião| Wei, uma das mansões lunares | Sobrenome: Wei}
  \definition{clas.}{usado para peixes}
  \definition{s.}{cauda; traseira | parte semelhante a uma cauda | fim | parte restante (ou inacabada); remanescente; a parte fora da parte principal; negócio inacabado}
  \seeref{yi3}
\end{EntryWithPhonetic}

\begin{EntryWithPhonetic}{尾巴}{wei3ba5}{7,4}{⼫,⼰}[HSK 4]
  \definition[条,根]{s.}{cauda; projeções na extremidade do corpo de certos animais | parte semelhante a uma cauda; refere"-se, em geral, ao final de algo | apêndice; anexo; adepto servil; pessoa que segue ou concorda com outra pessoa | Figura de linguagem: alguém que faz sombra a outro | fim; remanescente; parte restante (ou inacabada)}
\end{EntryWithPhonetic}

\begin{EntryWithPhonetic}{尾气}{wei3qi4}{7,4}{⼫,⽓}[HSK 7-9]
  \definition{s.}{gás residual; escape; emissões; gases inúteis emitidos quando carros ou máquinas estão funcionando}
\end{EntryWithPhonetic}

\begin{EntryWithPhonetic}{尾声}{wei3sheng1}{7,7}{⼫,⼠}[HSK 7-9]
  \definition[个]{s.}{fim; coda; epílogo; a última parte de uma atividade, evento, obra literária e artística, etc.}
  \synonymref{结束}{jie2shu4}
  \synonymref{结尾}{jie2wei3}
\end{EntryWithPhonetic}

%%%%%%%%%% 纬 %%%%%%%%%%
\subsection*{纬}\addcontentsline{loh}{figure}{纬 \dpy{wei3}}

\begin{EntryWithPhonetic}{纬}{wei3}{7}{⽷}
  \definition*{s.}{Sobrenome: Wei}
  \definition[本]{s.}{trama; o fio ou linha horizontal no tecido | latitude}
  \antonymref{经}{jing1}
\end{EntryWithPhonetic}

\begin{EntryWithPhonetic}{纬度}{wei3du4}{7,9}{⽷,⼴}[HSK 7-9]
  \definition{s.}{latitude}
\end{EntryWithPhonetic}

\begin{EntryWithPhonetic}{纬线}{wei3xian4}{7,8}{⽷,⽷}
  \definition{s.}{trama (têxtil); tecido | paralelo; linha de latitude}
\end{EntryWithPhonetic}

%%%%%%%%%% 委 %%%%%%%%%%
\subsection*{委}\addcontentsline{loh}{figure}{委 \dpy{wei3}}

\begin{EntryWithPhonetic}{委}{wei3}{8}{⼥}
  \definition*{s.}{Sobrenome: Wei}
  \definition{adj.}{indireto; desviado | apático; abatido | sinuoso; tortuoso | desanimado; apático; sem inspiração}
  \definition{adv.}{realmente; certamente; na verdade}
  \definition{s.}{membro do comitê | comitê; comissão; conselho}
  \definition{v.}{confiar; nomear |  jogar fora; deixar de lado | culpar os outros | confiar | descartar; abandonar | mudar; empurrar | acumular}
  \seeref{wei1}
\end{EntryWithPhonetic}

\begin{EntryWithPhonetic}{委内瑞拉}{wei3nei4rui4la1}{8,4,13,8}{⼥,⼌,⽟,⼿}
  \definition*{s.}{Venezuela}
\end{EntryWithPhonetic}

\begin{EntryWithPhonetic}{委屈}{wei3qu5}{8,8}{⼥,⼫}[HSK 7-9]
  \definition{adj.}{injustiçado; ofendido}
  \definition[点,次]{s.}{reclamação; tratamento injusto, injustiça ou maus tratos, enfatizando uma experiência ou estado negativo, triste e injusto}
  \definition{v.}{sentir"-se injustiçado; cuidar de uma queixa; sofrer com a injustiça; fazer alguém se sentir injustiçado; tratar alguém imerecidamente}
  \synonymref{无辜}{wu2gu1}
\end{EntryWithPhonetic}

\begin{EntryWithPhonetic}{委托}{wei3tuo1}{8,6}{⼥,⼿}[HSK 5]
  \definition{v.}{confiar; confiar uma tarefa a outra pessoa ou instituição (para que seja realizada)}
\end{EntryWithPhonetic}

\begin{EntryWithPhonetic}{委婉}{wei3wan3}{8,11}{⼥,⼥}[HSK 7-9]
  \definition{adj.}{diplomático; indireto (de palavras); rítmico (de voz); a linguagem usada para descrevê"-lo não é muito direta ou o som é alto ou baixo, muito bonito}
  \synonymref{含蓄}{han2xu4}
  \antonymref{坦白}{tan3bai2}
\end{EntryWithPhonetic}

\begin{EntryWithPhonetic}{委员}{wei3yuan2}{8,7}{⼥,⼝}[HSK 7-9]
  \definition[个,位,名]{s.}{membro; membro do comitê; membro de uma comissão; membros de organizações de liderança coletiva, tais como instituições, grupos ou escolas; membros de organizações especializadas criadas para realizar determinadas tarefas | enviado com comissão especial; uma pessoa com responsabilidade atribuída a uma tarefa específica}
\end{EntryWithPhonetic}

\begin{EntryWithPhonetic}{委员会}{wei3yuan2hui4}{8,7,6}{⼥,⼝,⼈}[HSK 7-9]
  \definition[个]{s.}{conselho; comitê; comissão; organizações de liderança coletiva em partidos políticos, grupos, agências e escolas | comitê; nome do departamento ou agência governamental | comitê; organizações especializadas estabelecidas por agências, grupos, escolas, etc. para completar certas tarefas}
  \synonymref{部门}{bu4men2}
  \synonymref{机构}{ji1gou4}
\end{EntryWithPhonetic}

%%%%%%%%%% 唯 %%%%%%%%%%
\subsection*{唯}\addcontentsline{loh}{figure}{唯 \dpy{wei3}}

\begin{EntryWithPhonetic}{唯}{wei3}{11}{⼝}
  \definition{interj.}{Onomatopéia: ``Sim!''; ``Yea!''; significa uma palavra que indica acordo}
  \seeref{wei2}
\end{EntryWithPhonetic}

%%%%%%%%%% 萎 %%%%%%%%%%
\subsection*{萎}\addcontentsline{loh}{figure}{萎 \dpy{wei3}}

\begin{EntryWithPhonetic}{萎}{wei3}{11}{⾋}
  \definition{adj.}{em declínio; decadente | sem ânimo; abatido}
  \definition{v.}{murchar; definhar; tombar; deixar cair}
  \seeref{wei1}
  \synonymref{缩}{suo1}
  \synonymref{谢}{xie4}
\end{EntryWithPhonetic}

\begin{EntryWithPhonetic}{萎缩}{wei3suo1}{11,14}{⾋,⽷}[HSK 7-9]
  \definition{v.}{murchar; (corpo, vegetação, etc.) secar | ceder; encolher; contrair}
  \synonymref{收缩}{shou1suo1}
  \antonymref{发达}{fa1da2}
  \antonymref{蔓延}{man4yan2}
  \antonymref{蓬勃}{peng2bo2}
  \antonymref{膨胀}{peng2zhang4}
\end{EntryWithPhonetic}

%%%%%%%%%% 卫 %%%%%%%%%%
\subsection*{卫}\addcontentsline{loh}{figure}{卫 \dpy{wei4}}

\begin{EntryWithPhonetic}{卫}{wei4}{3}{⼙}
  \definition*{s.}{Wei, um estado da Dinastia Zhou | Sobrenome: Wei}
  \definition{s.}{uma palavra usada no nome do lugar | outro nome para um burro}
  \definition{v.}{defender; guardar; proteger}
\end{EntryWithPhonetic}

\begin{EntryWithPhonetic}{卫生}{wei4sheng1}{3,5}{⼙,⽣}[HSK 3]
  \definition{adj.}{bom para a saúde; higiênico; limpo; capaz de prevenir doenças e benéfico para a saúde}
  \definition{s.}{higiene; saneamento; situação limpa}
\end{EntryWithPhonetic}

\begin{EntryWithPhonetic}{卫生部}{wei4sheng1bu4}{3,5,10}{⼙,⽣,⾢}
  \definition*{s.}{Ministério da Saúde}
\end{EntryWithPhonetic}

\begin{EntryWithPhonetic}{卫生防疫}{wei4sheng1 fang2yi4}{3,5,6,9}{⼙,⽣,⾩,⽧}
  \definition{s.}{prevenção contra a epidemia}
\end{EntryWithPhonetic}

\begin{EntryWithPhonetic}{卫生间}{wei4sheng1jian1}{3,5,7}{⼙,⽣,⾨}[HSK 3]
  \definition[间,个]{s.}{banheiro; sanitário; \emph{toilette}; quartos com instalações sanitárias em hotéis ou residências}
\end{EntryWithPhonetic}

\begin{EntryWithPhonetic}{卫生巾}{wei4sheng1jin1}{3,5,3}{⼙,⽣,⼱}
  \definition{s.}{absorvente higiênico}
\end{EntryWithPhonetic}

\begin{EntryWithPhonetic}{卫生局}{wei4sheng1ju2}{3,5,7}{⼙,⽣,⼫}
  \definition*{s.}{Departamento de Saúde | Escritório de Saúde}
\end{EntryWithPhonetic}

\begin{EntryWithPhonetic}{卫生棉}{wei4sheng1mian2}{3,5,12}{⼙,⽣,⽊}
  \definition{s.}{absorvente | algodão absorvente esterilizado (usado para curativos ou limpeza de feridas) | absorvente tampão}
\end{EntryWithPhonetic}

\begin{EntryWithPhonetic}{卫生球}{wei4sheng1qiu2}{3,5,11}{⼙,⽣,⽟}
  \definition{s.}{naftalina}
\end{EntryWithPhonetic}

\begin{EntryWithPhonetic}{卫生署}{wei4sheng1shu3}{3,5,13}{⼙,⽣,⽹}
  \definition*{s.}{Agência de Saúde (ou Escritório, ou Departamento)}
\end{EntryWithPhonetic}

\begin{EntryWithPhonetic}{卫生套}{wei4sheng1tao4}{3,5,10}{⼙,⽣,⼤}
  \definition[只]{s.}{preservativo | camisinha}
\end{EntryWithPhonetic}

\begin{EntryWithPhonetic}{卫生厅}{wei4 sheng1 ting1}{3,5,4}{⼙,⽣,⼚}
  \definition*{s.}{Departamento de Saúde (da Província)}
\end{EntryWithPhonetic}

\begin{EntryWithPhonetic}{卫生纸}{wei4sheng1zhi3}{3,5,7}{⼙,⽣,⽷}
  \definition{s.}{papel higiênico}
\end{EntryWithPhonetic}

\begin{EntryWithPhonetic}{卫视}{wei4shi4}{3,8}{⼙,⾒}[HSK 7-9]
  \definition[大,家]{s.}{televisão por satélite; abreviação de 卫星电视}
  \seealsoref{卫星电视}{wei4xing1 dian4shi4}
\end{EntryWithPhonetic}

\begin{EntryWithPhonetic}{卫星}{wei4xing1}{3,9}{⼙,⽇}[HSK 5]
  \definition[个,颗]{s.}{satélite; lua; corpos celestes orbitando planetas | satélite artificial | algo que gira em torno de um centro}
\end{EntryWithPhonetic}

\begin{EntryWithPhonetic}{卫星电视}{wei4xing1 dian4shi4}{3,9,5,8}{⼙,⽇,⽥,⾒}
  \definition{s.}{TV por satélite; televisão por satélite}
\end{EntryWithPhonetic}

%%%%%%%%%% 为 %%%%%%%%%%
\subsection*{为}\addcontentsline{loh}{figure}{为 \dpy{wei4}}

\begin{EntryWithPhonetic}{为}{wei4}{4}{⼂}[HSK 2]
  \definition*{s.}{Sobrenome: Wei}
  \definition{part.}{com 何 em uma pergunta retórica para expressar dúvida}
  \definition{prep.}{por; usado em frases passivas para introduzir o agente da ação, equivalente a 被 (frequentemente usado com 所)}
  \definition{suf.}{é anexado a alguns adjetivos ou advérbios monossilábicos para formar advérbios dissilábicos que expressam grau ou amplitude, geralmente modificando adjetivos ou verbos dissilábicos}
  \definition{v.}{fazer; agir | tornar"-se; transformar-se em | ser;  significar | servir como; agir como; desempenhar o papel de | fazer; trabalhar; indica certas ações e comportamentos, incluindo os significados de governança, engajamento, cenário e pesquisa}
  \seeref{wei2}
  \seealsoref{被}{bei4}
  \seealsoref{何}{he2}
  \seealsoref{所}{suo3}
  \synonymref{替}{ti4}
\end{EntryWithPhonetic}

\begin{EntryWithPhonetic}{为此}{wei4ci3}{4,6}{⼂,⽌}[HSK 6]
  \definition{conj.}{portanto; para este fim; por esta razão; para este propósito; nesta conexão; contexto de conexão, indicando que o comportamento descrito é devido aos motivos mencionados anteriormente}
  \synonymref{所以}{suo3yi3}
  \synonymref{因而}{yin1'er2}
  \synonymref{因此}{yin1ci3}
\end{EntryWithPhonetic}

\begin{EntryWithPhonetic}{为何}{wei4he2}{4,7}{⼂,⼈}[HSK 6]
  \definition{adv.}{por que?; por qual razão?}
  \seealsoref{为什么}{wei4shen2me5}
  \synonymref{何故}{he2gu4}
\end{EntryWithPhonetic}

\begin{EntryWithPhonetic}{为了}{wei4le5}{4,2}{⼂,⼅}[HSK 3]
  \definition{prep.}{para; por causa de; a fim de; o objetivo da introdução de ações comportamentais}
  \synonymref{使得}{shi3de5}
  \synonymref{为}{wei4}
  \synonymref{因为}{yin1wei5}
  \synonymref{着想}{zhuo2xiang3}
\end{EntryWithPhonetic}

\begin{EntryWithPhonetic}{为什么}{wei4shen2me5}{4,4,3}{⼂,⼈,⼃}[HSK 2]
  \definition{adv.}{por que?; por que é que?; como é que?;  nota: 为什么不 geralmente tem o significado de conselho, o mesmo que 何不}
  \seealsoref{何不}{he2bu4}
\end{EntryWithPhonetic}

%%%%%%%%%% 未 %%%%%%%%%%
\subsection*{未}\addcontentsline{loh}{figure}{未 \dpy{wei4}}

\begin{EntryWithPhonetic}{未}{wei4}{5}{⽊}[HSK 7-9]
  \definition*{s.}{Sobrenome: Wei}
  \definition{adv.}{Literário: não tem; não fez | Literário: não}
  \definition{part.}{ou não; no final das perguntas, indicando dúvida}[今可以言未?===Posso falar agora?]
  \definition{s.}{wei (oitavo dos doze Ramos Terrestres)}
  \antonymref{已}{yi3}
\end{EntryWithPhonetic}

\begin{EntryWithPhonetic}{未必}{wei4bi4}{5,5}{⽊,⼼}[HSK 4]
  \definition{adv.}{não tenho certeza; talvez não; não necessariamente}
\end{EntryWithPhonetic}

\begin{EntryWithPhonetic}{未成年人}{wei4cheng2nian2ren2}{5,6,6,2}{⽊,⼽,⼲,⼈}[HSK 7-9]
  \definition[名,个]{s.}{menores; juvenis; uma pessoa que não atingiu a maioridade e é legalmente incapaz de exercer direitos privados, e que necessita da administração de terceiros}
\end{EntryWithPhonetic}

\begin{EntryWithPhonetic}{未经}{wei4jing1}{5,8}{⽊,⽷}[HSK 7-9]
  \definition{v.}{não ter sido submetido a; sem ter passado por um determinado processo; não passar por (um determinado processo)}
\end{EntryWithPhonetic}

\begin{EntryWithPhonetic}{未来}{wei4lai2}{5,7}{⽊,⽊}[HSK 4]
  \definition{adj.}{próximo (refere"-se ao tempo)}
  \definition[个,段,种]{s.}{futuro; o amanhã}
\end{EntryWithPhonetic}

\begin{EntryWithPhonetic}{未免}{wei4mian3}{5,7}{⽊,⼉}[HSK 7-9]
  \definition{adv.}{bastante; verdadeiramente; um pouco demais (implicando discordância); isso é absolutamente verdade; chegamos ao ponto em que precisamos dizer isso; isso indica que o falante sente que não deveria ser feito dessa maneira e está rejeitando uma determinada abordagem | naturalmente; inevitavelmente; é inevitável que certas situações ocorram}
  \synonymref{不免}{bu4mian3}
  \synonymref{难免}{nan2mian3}
\end{EntryWithPhonetic}

\begin{EntryWithPhonetic}{未知数}{wei4zhi1shu4}{5,8,13}{⽊,⽮,⽁}[HSK 7-9]
  \definition[个]{s.}{Matemática: número desconhecido | algo desconhecido; incerteza | desconhecido; incerto}
\end{EntryWithPhonetic}

%%%%%%%%%% 位 %%%%%%%%%%
\subsection*{位}\addcontentsline{loh}{figure}{位 \dpy{wei4}}

\begin{EntryWithPhonetic}{位}{wei4}{7}{⼈}[HSK 2]
  \definition*{s.}{Sobrenome: Wei}
  \definition{clas.}{usado para pessoas (com cortesia, respeito) | usado para bits binários}[十六位===16 bits]
  \definition{s.}{lugar; localização; o lugar onde ou onde alguém está localizado | posto; \emph{status}; posição; a posição de uma pessoa em uma determinada área da vida social | trono; refere"-se especificamente ao status do imperador | lugar; dígito; a posição de cada dígito em um número}
\end{EntryWithPhonetic}

\begin{EntryWithPhonetic}{位居}{wei4ju1}{7,8}{⼈,⼫}
  \definition{v.}{estar localizado em}
\end{EntryWithPhonetic}

\begin{EntryWithPhonetic}{位于}{wei4yu2}{7,3}{⼈,⼆}[HSK 4]
  \definition{v.}{estar localizado; estar situado}
  \synonymref{处于}{chu3yu2}
\end{EntryWithPhonetic}

\begin{EntryWithPhonetic}{位置}{wei4zhi5}{7,13}{⼈,⽹}[HSK 4]
  \definition[个,处]{s.}{assento; lugar; localização | lugar; posição; \emph{status} | posição (por exemplo: cargo no escritório)}
  \synonymref{场所}{chang3suo3}
  \synonymref{地点}{di4dian3}
  \synonymref{地位}{di4wei4}
  \synonymref{地址}{di4zhi3}
  \synonymref{地方}{di4fang5}
  \synonymref{对称}{dui4chen4}
  \synonymref{方向}{fang1xiang5}
  \synonymref{位子}{wei4zi5}
\end{EntryWithPhonetic}

\begin{EntryWithPhonetic}{位子}{wei4zi5}{7,3}{⼈,⼦}[HSK 7-9]
  \definition[个]{s.}{assento; lugar; posição; o espaço ocupado por pessoas}
  \synonymref{地位}{di4wei4}
  \synonymref{位置}{wei4zhi5}
\end{EntryWithPhonetic}

%%%%%%%%%% 味 %%%%%%%%%%
\subsection*{味}\addcontentsline{loh}{figure}{味 \dpy{wei4}}

\begin{EntryWithPhonetic}{味}{wei4}{8}{⼝}
  \definition{clas.}{usado para ingredientes (de uma receita de medicina chinesa)}
  \definition{s.}{gosto; sabor | cheiro; odor | interesse; deleite | acepipe; \emph{delicacy} | significância; significado}
  \definition{v.}{distinguir (provar) o sabor de; saborear}
\end{EntryWithPhonetic}

\begin{EntryWithPhonetic}{味道}{wei4dao5}{8,12}{⼝,⾡}[HSK 2]
  \definition[个,股,种]{s.}{gosto; sabor | sensação; gosto; experiência | interesse; deleite | cheiro; odor}
\end{EntryWithPhonetic}

\begin{EntryWithPhonetic}{味精}{wei4jing1}{8,14}{⼝,⽶}[HSK 7-9]
  \definition[勺,袋,克]{s.}{glutamato monossódico (MSG); pó gourmet; temperos, em pó branco ou cristais granulados, adicionados a sopas e pratos para realçar o sabor umami}
\end{EntryWithPhonetic}

\begin{EntryWithPhonetic}{味儿}{wei4r5}{8,2}{⼝,⼉}[HSK 4]
  \definition{s.}{gosto; sabor; propriedade de uma substância que dá à língua uma determinada sensação de sabor | cheiro; odor; propriedade de uma substância que dá ao nariz um determinado sentido de cheiro | interesse; significado; deleite}
\end{EntryWithPhonetic}

%%%%%%%%%% 畏 %%%%%%%%%%
\subsection*{畏}\addcontentsline{loh}{figure}{畏 \dpy{wei4}}

\begin{EntryWithPhonetic}{畏}{wei4}{9}{⽥}
  \definition{s.}{medo; temor}
  \definition{v.}{ter medo; temer | respeitar; admirar}
\end{EntryWithPhonetic}

\begin{EntryWithPhonetic}{畏惧}{wei4ju4}{9,11}{⽥,⼼}[HSK 7-9]
  \definition{v.}{temer; recear; estar muito assustado}
  \synonymref{胆怯}{dan3qie4}
  \synonymref{害怕}{hai4pa4}
  \synonymref{恐惧}{kong3ju4}
  \synonymref{恐怕}{kong3pa4}
  \synonymref{畏缩}{wei4suo1}
  \antonymref{勇敢}{yong3gan3}
\end{EntryWithPhonetic}

\begin{EntryWithPhonetic}{畏缩}{wei4suo1}{9,14}{⽥,⽷}[HSK 7-9]
  \definition{v.}{recuar; encolher; sobressaltar; o medo impede o avanço}
  \synonymref{胆怯}{dan3qie4}
  \synonymref{害怕}{hai4pa4}
  \synonymref{后退}{hou4tui4}
  \synonymref{恐惧}{kong3ju4}
  \synonymref{退却}{tui4que4}
  \synonymref{退缩}{tui4suo1}
  \synonymref{畏惧}{wei4ju4}
  \antonymref{挺身}{ting3shen1}
\end{EntryWithPhonetic}

%%%%%%%%%% 胃 %%%%%%%%%%
\subsection*{胃}\addcontentsline{loh}{figure}{胃 \dpy{wei4}}

\begin{EntryWithPhonetic}{胃}{wei4}{9}{⾁}[HSK 5]
  \definition*{s.}{Wei, uma das mansões lunares | Wei, uma das vinte e oito constelações}
  \definition{s.}{estômago; parte do aparelho digestivo}
\end{EntryWithPhonetic}

\begin{EntryWithPhonetic}{胃口}{wei4kou3}{9,3}{⾁,⼝}[HSK 7-9]
  \definition{s.}{apetite; a sensação de querer comer | gostar; ter interesse em algo ou em atividades; metaforicamente, refere"-se ao interesse por coisas ou atividades | ambição; apetite}
\end{EntryWithPhonetic}

%%%%%%%%%% 喂 %%%%%%%%%%
\subsection*{喂}\addcontentsline{loh}{figure}{喂 \dpy{wei4}}

\begin{EntryWithPhonetic}{喂}{wei4}{12}{⼝}[HSK 2,4]
  \definition{interj.}{``Ei!'', ``Olá!'', para chamar atenção | ``Alô?'' (quando respondendo uma chamada telefônica, pronuncia"-se como \dpy{wei2})}
  \definition{v.}{criar; alimentar (animais); dar comida a um animal | alimentar (pessoas); colocar alimentos, medicamentos, etc. na boca de alguém}
\end{EntryWithPhonetic}

\begin{EntryWithPhonetic}{喂哺}{wei4bu3}{12,10}{⼝,⼝}
  \definition{v.}{alimentar (um bebê)}
\end{EntryWithPhonetic}

\begin{EntryWithPhonetic}{喂料}{wei4liao4}{12,10}{⼝,⽃}
  \definition{v.}{alimentar (também no sentido figurativo)}
\end{EntryWithPhonetic}

\begin{EntryWithPhonetic}{喂母乳}{wei4mu3ru3}{12,5,8}{⼝,⽏,⼄}
  \definition{s.}{amamentação}
\end{EntryWithPhonetic}

\begin{EntryWithPhonetic}{喂奶}{wei4nai3}{12,5}{⼝,⼥}
  \definition{v.}{amamentar}
\end{EntryWithPhonetic}

\begin{EntryWithPhonetic}{喂食}{wei4shi2}{12,9}{⼝,⾷}
  \definition{v.}{alimentar}
\end{EntryWithPhonetic}

\begin{EntryWithPhonetic}{喂养}{wei4yang3}{12,9}{⼝,⼋}[HSK 7-9]
  \definition{v.}{alimentar (uma criança, animal doméstico, etc.); manter; criar (um animal); alimentar as crianças pequenas ou os animais e cuidar deles para ajudá-los a crescer}
  \synonymref{放养}{fang4yang3}
  \synonymref{饲养}{si4yang3}
  \synonymref{喂食}{wei4shi2}
\end{EntryWithPhonetic}

%%%%%%%%%% 慰 %%%%%%%%%%
\subsection*{慰}\addcontentsline{loh}{figure}{慰 \dpy{wei4}}

\begin{EntryWithPhonetic}{慰}{wei4}{15}{⼼}
  \definition{adj.}{aliviado; em paz; confortável}
  \definition{v.}{consolar; confortar | ser (ficar) aliviado}
\end{EntryWithPhonetic}

\begin{EntryWithPhonetic}{慰劳}{wei4lao2}{15,7}{⼼,⼒}[HSK 7-9]
  \definition{v.}{levar presentes ou enviar votos de felicidades em reconhecimento aos serviços prestados | confortar | demonstrar apreço (através de palavras gentis, pequenos presentes, etc.)}
  \synonymref{安慰}{an1wei4}
  \synonymref{慰问}{wei4wen4}
  \synonymref{问候}{wen4hou4}
  \antonymref{打击}{da3ji1}
\end{EntryWithPhonetic}

\begin{EntryWithPhonetic}{慰问}{wei4wen4}{15,6}{⼼,⾨}[HSK 5]
  \definition{v.}{visitar; consolar; expressar simpatia por; confortar e cumprimentar com palavras e presentes;  enfatizar o conforto e o cumprimento, frequentemente usado por superiores para subordinados}
\end{EntryWithPhonetic}

%%%%%%%%%% 温 %%%%%%%%%%
\subsection*{温}\addcontentsline{loh}{figure}{温 \dpy{wen1}}

\begin{EntryWithPhonetic}{温}{wen1}{12}{⽔}
  \definition{adj.}{morno; quente; suave}
  \definition{s.}{temperatura | doenças transmissíveis agudas; praga}
  \definition{v.}{aquecer; reaquecer; aquecer ligeiramente | revisar; repassar}
\end{EntryWithPhonetic}

\begin{EntryWithPhonetic}{温度}{wen1du4}{12,9}{⽔,⼴}[HSK 2]
  \definition[度,级,档,个]{s.}{temperatura}
\end{EntryWithPhonetic}

\begin{EntryWithPhonetic}{温度表}{wen1du4biao3}{12,9,8}{⽔,⼴,⾐}
  \definition{s.}{termômetro}
\end{EntryWithPhonetic}

\begin{EntryWithPhonetic}{温度计}{wen1du4ji4}{12,9,4}{⽔,⼴,⾔}[HSK 7-9]
  \definition[支,个]{s.}{termômetro; termógrafo}
  \synonymref{温度表}{wen1du4biao3}
\end{EntryWithPhonetic}

\begin{EntryWithPhonetic}{温度梯度}{wen1du4ti1du4}{12,9,11,9}{⽔,⼴,⽊,⼴}
  \definition{s.}{gradiente de temperatura}
\end{EntryWithPhonetic}

\begin{EntryWithPhonetic}{温和}{wen1he2}{12,8}{⽔,⼝}[HSK 5]
  \definition{adj.}{gentil; suave; moderado}
\end{EntryWithPhonetic}

\begin{EntryWithPhonetic}{温暖}{wen1nuan3}{12,13}{⽔,⽇}[HSK 3]
  \definition{adj.}{caloroso; gentil; amigável | caloroso; quente}
  \definition{v.}{aquecer; fazer com que se sinta calor}
\end{EntryWithPhonetic}

\begin{EntryWithPhonetic}{温泉}{wen1quan2}{12,9}{⽔,⽔}[HSK 7-9]
  \definition[处,座,眼]{s.}{fonte termal; água de nascente com temperatura superior à temperatura média anual local}
\end{EntryWithPhonetic}

\begin{EntryWithPhonetic}{温柔}{wen1rou2}{12,9}{⽔,⽊}[HSK 7-9]
  \definition{adj.}{manso; gentil; agradavelmente afetuoso; delicado e suave; gentil e submissa (geralmente usado para descrever mulheres)}
  \seealsoref{温和}{wen1he2}
  \synonymref{温顺}{wen1shun4}
  \antonymref{粗暴}{cu1bao4}
\end{EntryWithPhonetic}

\begin{EntryWithPhonetic}{温室}{wen1shi4}{12,9}{⽔,⼧}[HSK 7-9]
  \definition[个,座,间]{s.}{estufa; jardim de inverno; originalmente referindo"-se a um ambiente interno aquecido, o termo é agora usado principalmente no contexto do aquecimento global | um ambiente criado propositadamente para que as pessoas cresçam sem desvantagens; uma metáfora para um ambiente de vida confortável}
\end{EntryWithPhonetic}

\begin{EntryWithPhonetic}{温顺}{wen1shun4}{12,9}{⽔,⾴}
  \definition{adj.}{dócil; manso; gentil e obediente}
  \synonymref{和气}{he2qi5}
  \synonymref{暖和}{nuan3huo5}
  \synonymref{平和}{ping2he2}
  \synonymref{温和}{wen1he2}
  \synonymref{温暖}{wen1nuan3}
  \synonymref{温柔}{wen1rou2}
  \antonymref{暴躁}{bao4zao4}
  \antonymref{乖张}{guai1zhang1}
\end{EntryWithPhonetic}

\begin{EntryWithPhonetic}{温习}{wen1xi2}{12,3}{⽔,⼄}[HSK 7-9]
  \definition{v.}{revisar; analisar; reaprender o que você já aprendeu ajuda a consolidar seu entendimento}
  \synonymref{复习}{fu4xi2}
\end{EntryWithPhonetic}

\begin{EntryWithPhonetic}{温馨}{wen1xin1}{12,20}{⽔,⾹}[HSK 7-9]
  \definition{adj.}{quente; macio; doce; descreve um ambiente ou atmosfera que faz as pessoas se sentirem acolhidas e confortáveis}
  \synonymref{和睦}{he2mu4}
  \synonymref{融洽}{rong2qia4}
  \antonymref{冷漠}{leng3mo4}
\end{EntryWithPhonetic}

%%%%%%%%%% 瘟 %%%%%%%%%%
\subsection*{瘟}\addcontentsline{loh}{figure}{瘟 \dpy{wen1}}

\begin{EntryWithPhonetic}{瘟}{wen1}{14}{⽧}
  \definition{adj.}{(ópera tradicional) enfadonho e insípido; (ópera tradicional) deprimente e entediante; (ópera tradicional) monótona e entediante}
  \definition[场]{s.}{doenças transmissíveis agudas}
\end{EntryWithPhonetic}

\begin{EntryWithPhonetic}{瘟疫}{wen1yi4}{14,9}{⽧,⽧}[HSK 7-9]
  \definition{s.}{praga; pestilência; refere"-se a doenças infecciosas agudas epidêmicas}
\end{EntryWithPhonetic}

%%%%%%%%%% 文 %%%%%%%%%%
\subsection*{文}\addcontentsline{loh}{figure}{文 \dpy{wen2}}

\begin{EntryWithPhonetic}{文}{wen2}{4}{⽂}[HSK 7-9][Kangxi 67]
  \definition*{s.}{Sobrenome: Wen}
  \definition{adj.}{civil; não militar | suave; refinado | refinado; literário; descreve que o conteúdo de um artigo ou discurso é difícil de entender}
  \definition{clas.}{usado para moedas de cobre antigas (moedas de cobre com palavras gravadas em um lado)}
  \definition{s.}{roteiro; escrita; personagem | escrita; composição literária; artigo | linguagem literária; redação | cultura; refere"-se ao estado manifestado quando a sociedade atinge um estágio superior de desenvolvimento | ritual; refere"-se ao antigo sistema ritual e musical | certos fenômenos naturais; refere"-se a certos fenômenos na natureza ou na sociedade humana | literatos; coisas não militares | arte liberal; refere"-se às ciências humanas e sociais | documento; refere"-se a documentos oficiais | padrão; textura}
  \definition{v.}{tatuar padrões ou palavras no corpo ou no rosto | cobrir; pintar por cima}
  \antonymref{武}{wu3}
\end{EntryWithPhonetic}

\begin{EntryWithPhonetic}{文化}{wen2hua4}{4,4}{⽂,⼔}[HSK 3]
  \definition[个,种]{s.}{cultura; civilização; tudo o que os seres humanos criaram em termos materiais e espirituais ao longo da história social | cultura; alfabetização; escolaridade; educação; o nível de conhecimento das pessoas e a capacidade de usar a linguagem escrita}
\end{EntryWithPhonetic}

\begin{EntryWithPhonetic}{文化层}{wen2hua4ceng2}{4,4,7}{⽂,⼔,⼫}
  \definition{s.}{nível de cultura (em sítio arqueológico)}
\end{EntryWithPhonetic}

\begin{EntryWithPhonetic}{文化宫}{wen2hua4gong1}{4,4,9}{⽂,⼔,⼧}
  \definition{s.}{palácio cultural}
\end{EntryWithPhonetic}

\begin{EntryWithPhonetic}{文化圈}{wen2hua4quan1}{4,4,11}{⽂,⼔,⼞}
  \definition{s.}{esfera de influência cultural}
\end{EntryWithPhonetic}

\begin{EntryWithPhonetic}{文化热}{wen2hua4re4}{4,4,10}{⽂,⼔,⽕}
  \definition{s.}{mania cultural | febre cultural}
\end{EntryWithPhonetic}

\begin{EntryWithPhonetic}{文化史}{wen2hua4shi3}{4,4,5}{⽂,⼔,⼝}
  \definition*{s.}{História Cultural}
\end{EntryWithPhonetic}

\begin{EntryWithPhonetic}{文化水平}{wen2hua4 shui3ping2}{4,4,4,5}{⽂,⼔,⽔,⼲}
  \definition{s.}{nível educacional}
\end{EntryWithPhonetic}

\begin{EntryWithPhonetic}{文化障碍}{wen2hua4 zhang4'ai4}{4,4,13,13}{⽂,⼔,⾩,⽯}
  \definition{s.}{barreira cultural}
\end{EntryWithPhonetic}

\begin{EntryWithPhonetic}{文件}{wen2jian4}{4,6}{⽂,⼈}[HSK 3]
  \definition[份,堆,叠]{s.}{documentos oficiais; papéis; instrumentos; termo genérico para documentos oficiais, cartas, etc. | os arquivos no computador; informações registradas no celular ou computador | artigos ou trabalhos sobre teorias políticas, atualidades, pesquisas acadêmicas, etc.; textos ou artigos sobre teoria política, políticas, etc.}
\end{EntryWithPhonetic}

\begin{EntryWithPhonetic}{文具}{wen2ju4}{4,8}{⽂,⼋}[HSK 7-9]
  \definition[件,套]{s.}{artigos de papelaria; materiais de escrita; termo genérico para instrumentos de escrita, como pincéis, tinta, papel e tinteiros}
\end{EntryWithPhonetic}

\begin{EntryWithPhonetic}{文科}{wen2ke1}{4,9}{⽂,⽲}[HSK 7-9]
  \definition{s.}{artes liberais; uma disciplina que engloba matérias como literatura, línguas, história, economia, filosofia e direito, em oposição a matérias científicas como matemática, física e química}
  \antonymref{理科}{li3ke1}
\end{EntryWithPhonetic}

\begin{EntryWithPhonetic}{文盲}{wen2mang2}{4,8}{⽂,⽬}[HSK 7-9]
  \definition[个,批]{s.}{analfabetismo; pessoa analfabeta; adultos analfabetos ou cujas habilidades de alfabetização não atendem aos padrões nacionais e que não possuem habilidades básicas de leitura e escrita}
  \antonymref{文明}{wen2ming2}
\end{EntryWithPhonetic}

\begin{EntryWithPhonetic}{文明}{wen2ming2}{4,8}{⽂,⽇}[HSK 3]
  \definition{adj.}{civilizado; sociedade desenvolvida e com alto nível cultural}
  \definition[个,种]{s.}{cultura; civilização}
\end{EntryWithPhonetic}

\begin{EntryWithPhonetic}{文凭}{wen2ping2}{4,8}{⽂,⼏}[HSK 7-9]
  \definition[张]{s.}{diploma; documento oficial usado como comprovante da escolaridade recebida; anteriormente referente a documentos oficiais usados como comprovante, agora se refere especificamente a certificados de conclusão de curso}
  \synonymref{证书}{zheng4shu1}
\end{EntryWithPhonetic}

\begin{EntryWithPhonetic}{文人}{wen2ren2}{4,2}{⽂,⼈}[HSK 7-9]
  \definition[位,个]{s.}{homem de letras; erudito; literato; um acadêmico que escreve bem}
\end{EntryWithPhonetic}

\begin{EntryWithPhonetic}{文物}{wen2wu4}{4,8}{⽂,⽜}[HSK 7-9]
  \definition[件,个,批]{s.}{relíquia cultural; relíquia histórica; objetos valiosos deixados pela história}
\end{EntryWithPhonetic}

\begin{EntryWithPhonetic}{文献}{wen2xian4}{4,13}{⽂,⽝}[HSK 7-9]
  \definition[篇,种,份]{s.}{documento; literatura; livros e materiais com valor histórico ou de referência}
  \synonymref{文件}{wen2jian4}
\end{EntryWithPhonetic}

\begin{EntryWithPhonetic}{文学}{wen2xue2}{4,8}{⽂,⼦}[HSK 3]
  \definition[个,种]{s.}{literatura; a arte de moldar imagens e refletir a vida social através da linguagem e da escrita, incluindo romances, poesia, prosa, teatro, etc.}
\end{EntryWithPhonetic}

\begin{EntryWithPhonetic}{文学系}{wen2xue2 xi4}{4,8,7}{⽂,⼦,⽷}
  \definition*{s.}{Departamento de Literatura}
\end{EntryWithPhonetic}

\begin{EntryWithPhonetic}{文雅}{wen2ya3}{4,12}{⽂,⾫}[HSK 7-9]
  \definition{adj.}{refinado; elegante (na fala e nos modos); descreve alguém como alguém que fala e se comporta de maneira gentil, educada e sem vulgaridades}
  \synonymref{大方}{da4fang5}
  \synonymref{高雅}{gao1ya3}
  \synonymref{美丽}{mei3li4}
  \synonymref{时髦}{shi2mao2}
  \synonymref{文化}{wen2hua4}
  \synonymref{文明}{wen2ming2}
  \antonymref{粗鲁}{cu1lu3}
  \antonymref{难听}{nan2ting1}
\end{EntryWithPhonetic}

\begin{EntryWithPhonetic}{文艺}{wen2yi4}{4,4}{⽂,⾋}[HSK 5]
  \definition{s.}{termo genérico para literatura e arte | performance (arte); refere"-se especificamente às artes performativas, como música e dança}
\end{EntryWithPhonetic}

\begin{EntryWithPhonetic}{文艺界}{wen2 yi4 jie4}{4,4,9}{⽂,⾋,⽥}
  \definition{s.}{círculos literários e artísticos; o mundo da literatura e da arte}
\end{EntryWithPhonetic}

\begin{EntryWithPhonetic}{文娱}{wen2yu2}{4,10}{⽂,⼥}[HSK 6]
  \definition{s.}{recreação cultural; entretenimento}
\end{EntryWithPhonetic}

\begin{EntryWithPhonetic}{文章}{wen2zhang1}{4,11}{⽂,⾳}[HSK 3]
  \definition[篇,段,页,系列]{s.}{ensaio; artigo; texto independente; também se refere a obras literárias em geral | significado oculto; significado implícito | trabalho; coisas que podem ser feitas}
\end{EntryWithPhonetic}

\begin{EntryWithPhonetic}{文字}{wen2zi4}{4,6}{⽂,⼦}[HSK 3]
  \definition[种,类,段,行,篇]{s.}{caracteres; caligrafia; escrita; símbolos escritos para registrar a linguagem | linguagem escrita; a forma escrita da língua}
\end{EntryWithPhonetic}

%%%%%%%%%% 纹 %%%%%%%%%%
\subsection*{纹}\addcontentsline{loh}{figure}{纹 \dpy{wen2}}

\begin{EntryWithPhonetic}{纹}{wen2}{7}{⽷}
  \definition[个]{s.}{linhas; veios; grãos; rugas na pele | padrão; desenho em tecido de seda; listras ou padrões em tecidos de seda; geralmente se refere a padrões lineares na superfície de um objeto}
\end{EntryWithPhonetic}

\begin{EntryWithPhonetic}{纹路}{wen2lu4}{7,13}{⽷,⾜}
  \definition{s.}{padrão de linhas | rugas | veias | veias (em mármore ou impressão digital) | grãos (em madeira, etc.)}
\end{EntryWithPhonetic}

%%%%%%%%%% 闻 %%%%%%%%%%
\subsection*{闻}\addcontentsline{loh}{figure}{闻 \dpy{wen2}}

\begin{EntryWithPhonetic}{闻}{wen2}{9}{⾨}[HSK 2]
  \definition*{s.}{Sobrenome: Wen}
  \definition{adj.}{bem conhecido; famoso}
  \definition{s.}{notícia; história | reputação | boato; rumor}
  \definition{v.}{cheirar | ouvir}
\end{EntryWithPhonetic}

\begin{EntryWithPhonetic}{闻名}{wen2ming2}{9,6}{⾨,⼝}[HSK 7-9]
  \definition{adj.}{famoso; renomado; conhecido}
  \definition{v.}{conhecer alguém por reputação; estar familiarizado com o nome de alguém; ouvir falar da reputação}
  \synonymref{驰名}{chi2ming2}
  \synonymref{出名}{chu1 ming2}
  \synonymref{有名}{you3 ming2}
  \synonymref{知名}{zhi1ming2}
  \synonymref{著名}{zhu4ming2}
  \antonymref{普通}{pu3tong1}
\end{EntryWithPhonetic}

%%%%%%%%%% 蚊 %%%%%%%%%%
\subsection*{蚊}\addcontentsline{loh}{figure}{蚊 \dpy{wen2}}

\begin{EntryWithPhonetic}{蚊}{wen2}{10}{⾍}
  \definition{s.}{mosquito; pernilongo}
\end{EntryWithPhonetic}

\begin{EntryWithPhonetic}{蚊香}{wen2xiang1}{10,9}{⾍,⾹}
  \definition{s.}{incenso ou espiral repelente de mosquitos}
\end{EntryWithPhonetic}

\begin{EntryWithPhonetic}{蚊帐}{wen2zhang4}{10,7}{⾍,⼱}[HSK 7-9]
  \definition[顶]{s.}{mosquiteiro; as redes mosquiteiras, penduradas acima e ao redor da cama para bloquear os mosquitos, vêm em formatos de guarda"-chuva e retangulares}
\end{EntryWithPhonetic}

\begin{EntryWithPhonetic}{蚊子}{wen2zi5}{10,3}{⾍,⼦}[HSK 7-9]
  \definition[只]{s.}{pernilongo; mosquito}
\end{EntryWithPhonetic}

%%%%%%%%%% 吻 %%%%%%%%%%
\subsection*{吻}\addcontentsline{loh}{figure}{吻 \dpy{wen3}}

\begin{EntryWithPhonetic}{吻}{wen3}{7}{⼝}[HSK 7-9]
  \definition{s.}{tom; voz | lábios | bocas de animais}
  \definition{v.}{beijar; tocar uma pessoa ou objeto com os lábios | tocar; estar perto de}[他轻轻地吻了她一下。===Ele a beijou delicadamente. | 几乎吻着地面。===Quase tocou o chão.]
\end{EntryWithPhonetic}

\begin{EntryWithPhonetic}{吻合}{wen3he2}{7,6}{⼝,⼝}[HSK 7-9]
  \definition{s.}{correspondente; consistente; idêntico; totalmente compatível}
  \definition{v.}{conectar por anastomose; Medicina: refere"-se à união de duas superfícies fraturadas de um órgão}
  \synonymref{符合}{fu2he2}
  \synonymref{适合}{shi4he2}
  \antonymref{出入}{chu1ru4}
  \antonymref{抵触}{di3chu4}
\end{EntryWithPhonetic}

%%%%%%%%%% 紊 %%%%%%%%%%
\subsection*{紊}\addcontentsline{loh}{figure}{紊 \dpy{wen3}}

\begin{EntryWithPhonetic}{紊}{wen3}{10}{⽷}
  \definition{adj.}{desordenado; confuso}
\end{EntryWithPhonetic}

\begin{EntryWithPhonetic}{紊乱}{wen3luan4}{10,7}{⽷,⼄}[HSK 7-9]
  \definition{adj.}{caótico; desordenado; confuso; sem ordem ou sequência}
  \synonymref{混乱}{hun4luan4}
  \antonymref{整齐}{zheng3qi2}
\end{EntryWithPhonetic}

%%%%%%%%%% 稳 %%%%%%%%%%
\subsection*{稳}\addcontentsline{loh}{figure}{稳 \dpy{wen3}}

\begin{EntryWithPhonetic}{稳}{wen3}{14}{⽲}[HSK 4]
  \definition{adj.}{constante; estável; firme | estável; estático; sedado | seguro; confiável; certo}
  \definition{adv.}{certamente; com certeza; seguramente; sem dúvida}
  \definition{v.}{estabilizar, manter estável; acalmar}
\end{EntryWithPhonetic}

\begin{EntryWithPhonetic}{稳定}{wen3ding4}{14,8}{⽲,⼧}[HSK 4]
  \definition{adj.}{estável; firme; descreve uma natureza, um estado, etc. relativamente fixo; não muda significativamente}
  \definition{v.}{manter estável; estabilizar}
\end{EntryWithPhonetic}

\begin{EntryWithPhonetic}{稳固}{wen3gu4}{14,8}{⽲,⼞}[HSK 7-9]
  \definition{adj.}{firme; constante; estável; estável e sólido}
  \definition{v.}{estabilizar; o estado de manter as coisas inalteradas}
  \synonymref{安稳}{an1wen3}
  \synonymref{巩固}{gong3gu4}
  \synonymref{坚固}{jian1gu4}
  \synonymref{坚韧}{jian1ren4}
  \synonymref{坚实}{jian1shi2}
  \synonymref{结识}{jie2shi2}
  \synonymref{牢固}{lao2gu4}
  \synonymref{平稳}{ping2wen3}
  \synonymref{稳定}{wen3ding4}
  \antonymref{动摇}{dong4yao2}
  \antonymref{摇晃}{yao2huang4}
\end{EntryWithPhonetic}

\begin{EntryWithPhonetic}{稳健}{wen3jian4}{14,10}{⽲,⼈}[HSK 7-9]
  \definition{adj.}{firme; constante; estável; descreve um modo de andar ou o desenvolvimento de algo como suave e poderoso | seguro; constante; confiável; descreve alguém que age com maturidade e ponderação; que permanece calmo e lida bem com as situações}
  \synonymref{保守}{bao3shou3}
  \synonymref{妥当}{tuo3dang4}
  \synonymref{稳妥}{wen3tuo3}
  \synonymref{稳重}{wen3zhong4}
\end{EntryWithPhonetic}

\begin{EntryWithPhonetic}{稳妥}{wen3tuo3}{14,7}{⽲,⼥}[HSK 7-9]
  \definition{adj.}{seguro; confiável; prudente; estável; constante e adequado}
  \synonymref{恰当}{qia4dang4}
  \synonymref{停当}{ting2dang5}
  \synonymref{妥当}{tuo3dang4}
  \synonymref{妥善}{tuo3shan4}
  \synonymref{稳健}{wen3jian4}
  \antonymref{冒险}{mao4/xian3}
\end{EntryWithPhonetic}

\begin{EntryWithPhonetic}{稳重}{wen3zhong4}{14,9}{⽲,⾥}[HSK 7-9]
  \definition{adj.}{calmo; estável; sereno (na fala ou no comportamento); descreve alguém que fala e age com maturidade; não é descuidado; permanece calmo sob pressão e lida bem com as situações, transmitindo tranquilidade aos outros}
  \synonymref{安宁}{an1ning2}
  \synonymref{沉着}{chen2zhuo2}
  \synonymref{从容}{cong2rong2}
  \synonymref{谨慎}{jin3shen4}
  \synonymref{耐心}{nai4xin1}
  \synonymref{慎重}{shen4zhong4}
  \synonymref{严肃}{yan2su4}
  \synonymref{自在}{zi4zai4}
  \antonymref{暴躁}{bao4zao4}
  \antonymref{浮躁}{fu2zao4}
  \antonymref{慌乱}{huang1luan4}
  \antonymref{急忙}{ji2mang2}
  \antonymref{鲁莽}{lu3mang3}
\end{EntryWithPhonetic}

%%%%%%%%%% 问 %%%%%%%%%%
\subsection*{问}\addcontentsline{loh}{figure}{问 \dpy{wen4}}

\begin{EntryWithPhonetic}{问}{wen4}{6}{⾨}[HSK 1]
  \definition*{s.}{Sobrenome: Wen}
  \definition{prep.}{de; introduzir o objeto da ação, equivalente a 向 e 跟}
  \definition{v.}{perguntar; indagar; fazer com que as pessoas respondam ou esclareçam coisas que não sabem ou não têm certeza | perguntar (ou indagar) sobre | examinar; interrogar | intervir; responsabilizar; investigar | cuidar; preocupar"-se; gerenciar; interferir}
  \seealsoref{跟}{gen1}
  \seealsoref{向}{xiang4}
\end{EntryWithPhonetic}

\begin{EntryWithPhonetic}{问安}{wen4'an1}{6,6}{⾨,⼧}
  \definition{s.}{saudações}
  \definition{v.}{dar cumprimentos a | prestar homenagem}
\end{EntryWithPhonetic}

\begin{EntryWithPhonetic}{问鼎}{wen4ding3}{6,12}{⾨,⿍}
  \definition{v.}{visar (o primeiro lugar, etc.) | aspirar ao trono}
\end{EntryWithPhonetic}

\begin{EntryWithPhonetic}{问候}{wen4hou4}{6,10}{⾨,⼈}[HSK 4]
  \definition{v.}{prestar homenagem; enviar uma saudação;  dar os respeitos (cumprimentos) a alguém}
\end{EntryWithPhonetic}

\begin{EntryWithPhonetic}{问卷}{wen4juan4}{6,8}{⾨,⼙}[HSK 7-9]
  \definition[份]{s.}{questionário; um questionário utilizado para realizar pesquisas ou solicitar opiniões, que lista diversas perguntas para as pessoas responderem}
\end{EntryWithPhonetic}

\begin{EntryWithPhonetic}{问路}{wen4 lu4}{6,13}{⾨,⾜}[HSK 2]
  \definition{v.}{perguntar o caminho; pedir direções}
\end{EntryWithPhonetic}

\begin{EntryWithPhonetic}{问世}{wen4shi4}{6,5}{⾨,⼀}[HSK 7-9]
  \definition{v.}{sair; ser publicado; (obras, invenções, novos produtos, etc.) ser apresentados ao mundo}
  \synonymref{诞生}{dan4sheng1}
  \synonymref{上市}{shang4 shi4}
  \synonymref{研制}{yan2zhi4}
  \antonymref{下线}{xia4xian4}
\end{EntryWithPhonetic}

\begin{EntryWithPhonetic}{问市}{wen4shi4}{6,5}{⾨,⼱}
  \definition{v.}{chegar ao mercado | bater o mercado | atingir o mercado}
\end{EntryWithPhonetic}

\begin{EntryWithPhonetic}{问题}{wen4ti2}{6,15}{⾨,⾴}[HSK 2]
  \definition{adj.}{desqualificado; indesejável; anormal, não atende aos requisitos}
  \definition[个,种,类,串]{s.}{pergunta; problema; perguntas a serem respondidas | problema; questão; contradições que precisam ser estudadas e resolvidas | problema; acidente; incidente | chave; ponto crucial; pontos importantes}
\end{EntryWithPhonetic}

%%%%%%%%%% 嗡 %%%%%%%%%%
\subsection*{嗡}\addcontentsline{loh}{figure}{嗡 \dpy{weng1}}

\begin{EntryWithPhonetic}{嗡}{weng1}{13}{⼝}
  \definition[出]{part.}{(onomatopéia) zumbido; zunido; zunzum; descreve o som do bater de asas de um inseto}
\end{EntryWithPhonetic}

\begin{EntryWithPhonetic}{嗡嗡}{weng1weng1}{13,13}{⼝,⼝}
  \definition{s.}{zumbido}
  \definition{v.}{zumbir}
\end{EntryWithPhonetic}

%%%%%%%%%% 蕹 %%%%%%%%%%
\subsection*{蕹}\addcontentsline{loh}{figure}{蕹 \dpy{weng4}}

\begin{EntryWithPhonetic}{蕹}{weng4}{16}{⾋}
  \definition{s.}{espinafre-d’água ou \emph{ong choy}, usado como vegetal no sul da China e no sudeste da Ásia}
\end{EntryWithPhonetic}

\begin{EntryWithPhonetic}{蕹菜}{weng4cai4}{16,11}{⾋,⾋}
  \definition{s.}{espinafre aquático | \emph{ong choy} | repolho do pântano | convolvulus aquático | glória-da-manhã aquática}
  \seealsoref{空心菜}{kong1xin1cai4}
\end{EntryWithPhonetic}

%%%%%%%%%% 窝 %%%%%%%%%%
\subsection*{窝}\addcontentsline{loh}{figure}{窝 \dpy{wo1}}

\begin{EntryWithPhonetic}{窝}{wo1}{12}{⽳}[HSK 7-9]
  \definition{clas.}{ninhada; cria; usado para animais}
  \definition[个,只,块]{s.}{ninho; habitat de aves, animais e insetos | toca; covil; ninho; uma metáfora para um lugar onde pessoas más se reúnem | lugar; metaforicamente, a posição ocupada por um corpo humano ou objeto | poço; cavidade; área rebaixada}
  \definition{v.}{abrigar; dar abrigo | reprimir; suprimir; reter (voltar); emoções reprimidas não podem ser expressas ou manifestadas | dobrar; flexionar; torcer}
\end{EntryWithPhonetic}

%%%%%%%%%% 我 %%%%%%%%%%
\subsection*{我}\addcontentsline{loh}{figure}{我 \dpy{wo3}}

\begin{EntryWithPhonetic}{我}{wo3}{7}{⼽}[HSK 1]
  \definition{pron.}{eu; mim | um; qualquer um; usado para contrastar 他 e 我; refere"-se a muitas pessoas em geral}
  \seealsoref{他}{ta1}
\end{EntryWithPhonetic}

\begin{EntryWithPhonetic}{我的}{wo3 de5}{7,8}{⼽,⽩}
  \definition{pron.}{meu, meus}
\end{EntryWithPhonetic}

\begin{EntryWithPhonetic}{我们}{wo3men5}{7,5}{⼽,⼈}[HSK 1]
  \definition{pron.}{nós; nos}
\end{EntryWithPhonetic}

\begin{EntryWithPhonetic}{我们的}{wo3men5 de5}{7,5,8}{⼽,⼈,⽩}
  \definition{pron.}{nosso, nossos}
\end{EntryWithPhonetic}

\begin{EntryWithPhonetic}{我去}{wo3qu4}{7,5}{⼽,⼛}
  \definition{interj.}{Gíria: ``O que\dots!!'' | ``Oh meu Deus!'' | ``Isso é insano!''}
\end{EntryWithPhonetic}

%%%%%%%%%% 卧 %%%%%%%%%%
\subsection*{卧}\addcontentsline{loh}{figure}{卧 \dpy{wo4}}

\begin{EntryWithPhonetic}{卧}{wo4}{8}{⾂}[HSK 7-9]
  \definition{adj.}{para dormir}
  \definition{s.}{vagão-leito (ou carruagem); leito | beliche | quarto | Dialeto: pochê (ovos)}
  \definition{v.}{deitar | Dialeto: deitar um bebê | (animais ou pássaros) agachar"-se; sentar"-se; empoleirar"-se | Figurativo: viver em reclusão}
  \antonymref{坐}{zuo4}
\end{EntryWithPhonetic}

\begin{EntryWithPhonetic}{卧病}{wo4bing4}{8,10}{⾂,⽧}
  \definition{s.}{acamado | doente na cama}
\end{EntryWithPhonetic}

\begin{EntryWithPhonetic}{卧舱}{wo4cang1}{8,10}{⾂,⾈}
  \definition{s.}{cabine de dormir em um barco ou trem}
\end{EntryWithPhonetic}

\begin{EntryWithPhonetic}{卧车}{wo4che1}{8,4}{⾂,⾞}
  \definition{s.}{um carro-leito | vagão-leito}
\end{EntryWithPhonetic}

\begin{EntryWithPhonetic}{卧床}{wo4chuang2}{8,7}{⾂,⼴}
  \definition{adj.}{acamado}
  \definition{s.}{cama}
  \definition{v.}{deitar na cama}
\end{EntryWithPhonetic}

\begin{EntryWithPhonetic}{卧倒}{wo4dao3}{8,10}{⾂,⼈}
  \definition{v.}{cair no chão | deitar-se}
\end{EntryWithPhonetic}

\begin{EntryWithPhonetic}{卧铺}{wo4pu4}{8,12}{⾂,⾦}[HSK 6]
  \definition[个,排]{s.}{beliche para dormir; um beliche em um trem ou ônibus de longa distância}
\end{EntryWithPhonetic}

\begin{EntryWithPhonetic}{卧式}{wo4shi4}{8,6}{⾂,⼷}
  \definition{adj.}{horizontal}
\end{EntryWithPhonetic}

\begin{EntryWithPhonetic}{卧室}{wo4shi4}{8,9}{⾂,⼧}[HSK 5]
  \definition[间,个]{s.}{quarto de dormir; quarto de uma casa usado para dormir}
\end{EntryWithPhonetic}

\begin{EntryWithPhonetic}{卧榻}{wo4ta4}{8,14}{⾂,⽊}
  \definition{s.}{um sofá | uma cama estreita}
\end{EntryWithPhonetic}

\begin{EntryWithPhonetic}{卧推}{wo4tui1}{8,11}{⾂,⼿}
  \definition{s.}{supino}
\end{EntryWithPhonetic}

%%%%%%%%%% 握 %%%%%%%%%%
\subsection*{握}\addcontentsline{loh}{figure}{握 \dpy{wo4}}

\begin{EntryWithPhonetic}{握}{wo4}{12}{⼿}[HSK 5]
  \definition{v.}{segurar; agarrar | agarrar; segurar; empunhar; controlar | pegar pela mão}
\end{EntryWithPhonetic}

\begin{EntryWithPhonetic}{握手}{wo4/shou3}{12,4}{⼿,⼿}[HSK 3]
  \definition{v.+compl.}{apertar as mãos; dar um aperto de mão; estender a mão e apertar a mão do outro é uma forma de saudação ao se encontrar ou se despedir, e também é usado para expressar felicitações ou condolências}
\end{EntryWithPhonetic}

%%%%%%%%%% 斡 %%%%%%%%%%
\subsection*{斡}\addcontentsline{loh}{figure}{斡 \dpy{wo4}}

\begin{EntryWithPhonetic}{斡}{wo4}{14}{⽃}
  \definition{v.}{virar"-se}
\end{EntryWithPhonetic}

\begin{EntryWithPhonetic}{斡旋}{wo4xuan2}{14,11}{⽃,⽅}
  \definition{v.}{mediar (um conflito, etc.)}
\end{EntryWithPhonetic}

%%%%%%%%%% 乌 %%%%%%%%%%
\subsection*{乌}\addcontentsline{loh}{figure}{乌 \dpy{wu1}}

\begin{EntryWithPhonetic}{乌}{wu1}{4}{⼃}
  \definition*{s.}{Sobrenome: Wu}
  \definition{adj.}{preto; escuro}
  \definition{pron.}{como; o que é}
  \definition{s.}{corvo; gralha}
  \seeref{wu4}
  \synonymref{黑}{hei1}
\end{EntryWithPhonetic}

\begin{EntryWithPhonetic}{乌龟}{wu1gui1}{4,7}{⼃,⿔}
  \definition[头,只]{s.}{tartaruga}
\end{EntryWithPhonetic}

\begin{EntryWithPhonetic}{乌克兰}{wu1ke4lan2}{4,7,5}{⼃,⼗,⼋}
  \definition*{s.}{Ucrânia}
\end{EntryWithPhonetic}

\begin{EntryWithPhonetic}{乌云}{wu1yun2}{4,4}{⼃,⼆}[HSK 6]
  \definition[片]{s.}{nuvens negras; nuvens escuras | cabelo preto (de mulher); uma metáfora para o cabelo preto brilhante de uma mulher}
  \antonymref{彩虹}{cai3hong2}
\end{EntryWithPhonetic}

%%%%%%%%%% 污 %%%%%%%%%%
\subsection*{污}\addcontentsline{loh}{figure}{污 \dpy{wu1}}

\begin{EntryWithPhonetic}{污}{wu1}{6}{⽔}
  \definition{adj.}{sujo; imundo; imundo | corrupto}
  \definition{s.}{sujeira; imundície | esgoto; água suja; coisas sujas}
  \definition{v.}{contaminar; sujar | manchar}
\end{EntryWithPhonetic}

\begin{EntryWithPhonetic}{污秽}{wu1hui4}{6,11}{⽔,⽲}[HSK 7-9]
  \definition{adj.}{Literário: imundo; repugnante}
  \definition{s.}{sujeira; imundície}
  \antonymref{干净}{gan1jing4}
  \antonymref{洁净}{jie2jing4}
  \antonymref{清洁}{qing1jie2}
\end{EntryWithPhonetic}

\begin{EntryWithPhonetic}{污染}{wu1ran3}{6,9}{⽔,⽊}[HSK 5]
  \definition{v.}{poluir; contaminar com substâncias nocivas e prejudiciais; refere"-se especificamente à destruição do ambiente natural causada por substâncias nocivas, tais como gases, líquidos e resíduos emitidos por indústrias, minas, veículos, etc. | contaminar; metáfora de que pensamentos prejudiciais causam efeitos negativos nas pessoas}
\end{EntryWithPhonetic}

\begin{EntryWithPhonetic}{污染区}{wu1ran3qu1}{6,9,4}{⽔,⽊,⼖}
  \definition{s.}{área contaminada}
\end{EntryWithPhonetic}

\begin{EntryWithPhonetic}{污染物}{wu1ran3wu4}{6,9,8}{⽔,⽊,⽜}
  \definition{s.}{poluente}
  \seealsoref{污染物质}{wu1ran3 wu4zhi4}
\end{EntryWithPhonetic}

\begin{EntryWithPhonetic}{污染物质}{wu1ran3 wu4zhi4}{6,9,8,8}{⽔,⽊,⽜,⾙}
  \definition{s.}{poluente}
  \seealsoref{污染物}{wu1ran3wu4}
\end{EntryWithPhonetic}

\begin{EntryWithPhonetic}{污水}{wu1shui3}{6,4}{⽔,⽔}[HSK 5]
  \definition[桶,滩]{s.}{água suja (ou poluída, residual); esgoto; lodo | efluente; drenagem; água suja; água poluída; água residual}
\end{EntryWithPhonetic}

%%%%%%%%%% 呜 %%%%%%%%%%
\subsection*{呜}\addcontentsline{loh}{figure}{呜 \dpy{wu1}}

\begin{EntryWithPhonetic}{呜}{wu1}{7}{⼝}
  \definition{s.}{Onomatopéia: buzina; pio; ``zoom''}
\end{EntryWithPhonetic}

\begin{EntryWithPhonetic}{呜咽}{wu1ye4}{7,9}{⼝,⼝}[HSK 7-9]
  \definition{v.}{soluçar; choramingar | (água, vento, instrumentos de corda, etc.) chorar; lamentar; prantear |  emitir sons melancólicos}
  \synonymref{哭泣}{ku1qi4}
\end{EntryWithPhonetic}

%%%%%%%%%% 巫 %%%%%%%%%%
\subsection*{巫}\addcontentsline{loh}{figure}{巫 \dpy{wu1}}

\begin{EntryWithPhonetic}{巫}{wu1}{7}{⼯}
  \definition*{s.}{Sobrenome: Wu}
  \definition[位,名,个,些]{s.}{xamã; bruxa; mago | bruxas, feiticeiros}
\end{EntryWithPhonetic}

\begin{EntryWithPhonetic}{巫婆}{wu1po2}{7,11}{⼯,⼥}[HSK 7-9]
  \definition[个,位,名]{s.}{bruxa; feiticeira; xamã | xamã feminina}
\end{EntryWithPhonetic}

%%%%%%%%%% 屋 %%%%%%%%%%
\subsection*{屋}\addcontentsline{loh}{figure}{屋 \dpy{wu1}}

\begin{EntryWithPhonetic}{屋}{wu1}{9}{⼫}[HSK 5]
  \definition[间,座]{s.}{casa | quarto}
\end{EntryWithPhonetic}

\begin{EntryWithPhonetic}{屋顶}{wu1ding3}{9,8}{⼫,⾴}[HSK 7-9]
  \definition[个]{s.}{teto; telhado; cobertura; telhado de uma casa; o topo da casa}
  \antonymref{地板}{di4ban3}
\end{EntryWithPhonetic}

\begin{EntryWithPhonetic}{屋子}{wu1zi5}{9,3}{⼫,⼦}[HSK 3]
  \definition[间,座,栋]{s.}{quarto; sala}
\end{EntryWithPhonetic}

%%%%%%%%%% 恶 %%%%%%%%%%
\subsection*{恶}\addcontentsline{loh}{figure}{恶 \dpy{wu1}}

\begin{EntryWithPhonetic}{恶}{wu1}{10}{⼼}
  \definition{interj.}{``Droga!''; ``Ah não!''; expressa surpresa}
  \definition{pron.}{como?; por que?; refere"-se a um lugar ou coisa; expressa uma pergunta retórica; equivalente a 何 ou 怎么}
  \seeref{e3}
  \seeref{e4}
  \seeref{wu4}
  \seealsoref{何}{he2}
  \seealsoref{怎么}{zen3me5}
\end{EntryWithPhonetic}

%%%%%%%%%% 无 %%%%%%%%%%
\subsection*{无}\addcontentsline{loh}{figure}{无 \dpy{wu2}}

\begin{EntryWithPhonetic}{无}{wu2}{4}{⽆}[HSK 4][Kangxi 71]
  \definition{adv.}{não (em posição a 有); não ter algo; não há\dots}
  \definition{conj.}{independentemente de; não importa se, o que, etc.}
  \definition{v.}{não ter; estar sem; não existir}
  \seealsoref{有}{you3}
\end{EntryWithPhonetic}

\begin{EntryWithPhonetic}{无比}{wu2bi3}{4,4}{⽆,⽐}[HSK 7-9]
  \definition{adj.}{inigualável; incomparável; sem paralelo; nada se compara (frequentemente usado por razões positivas)}
  \synonymref{非常}{fei1chang2}
  \synonymref{格外}{ge2wai4}
  \synonymref{极其}{ji2qi2}
  \synonymref{十分}{shi2fen1}
  \antonymref{一般}{yi4ban1}
\end{EntryWithPhonetic}

\begin{EntryWithPhonetic}{无边}{wu2bian1}{4,5}{⽆,⾡}[HSK 6]
  \definition{adj.}{ilimitado; vasto; sem limites; sem abas; sem bordas}
\end{EntryWithPhonetic}

\begin{EntryWithPhonetic}{无不}{wu2bu4}{4,4}{⽆,⼀}[HSK 7-9]
  \definition{adv.}{invariavelmente; todos sem exceção; nenhum (nem tanto);  significa que não há exceções}
  \synonymref{全都}{quan2dou1}
\end{EntryWithPhonetic}

\begin{EntryWithPhonetic}{无偿}{wu2chang2}{4,11}{⽆,⼈}[HSK 7-9]
  \definition{adj.}{grátis; gratuito; sem custo; não pago; que não exige pagamento da outra parte}
  \synonymref{免费}{mian3/fei4}
  \synonymref{义务}{yi4wu4}
\end{EntryWithPhonetic}

\begin{EntryWithPhonetic}{无敌}{wu2di2}{4,10}{⽆,⾆}[HSK 7-9]
  \definition{adj.}{inigualável; invencível; imbatível; sem rivais}
\end{EntryWithPhonetic}

\begin{EntryWithPhonetic}{无恶不作}{wu2'e4-bu2zuo4}{4,10,4,7}{⽆,⼼,⼀,⼈}[HSK 7-9]
  \definition{expr.}{``Não se furtar a nenhum crime.'' (expressão idiomática); cometer qualquer delito imaginável; essa expressão descreve alguém extremamente perverso e capaz de praticar qualquer tipo de maldade; não hesitar em praticar o mal; não se deter diante de nenhum mal; cometer todo tipo de crime}
\end{EntryWithPhonetic}

\begin{EntryWithPhonetic}{无法}{wu2fa3}{4,8}{⽆,⽔}[HSK 4]
  \definition{adj.}{incapaz; incapacitado; não tem jeito}
\end{EntryWithPhonetic}

\begin{EntryWithPhonetic}{无非}{wu2fei1}{4,8}{⽆,⾮}[HSK 7-9]
  \definition{adv.}{somente; simplesmente; nada além de; não mais que; nada mais do que; significando tudo dentro de um certo intervalo}
  \synonymref{并非}{bing4fei1}
\end{EntryWithPhonetic}

\begin{EntryWithPhonetic}{无辜}{wu2gu1}{4,12}{⽆,⾟}[HSK 7-9]
  \definition{adj.}{não culpado; inocente}
  \definition[个,名,位]{s.}{pessoa inocente}
  \synonymref{可怜}{ke3lian2}
  \synonymref{委屈}{wei3qu5}
  \antonymref{罪恶}{zui4'e4}
\end{EntryWithPhonetic}

\begin{EntryWithPhonetic}{无骨}{wu2 gu3}{4,9}{⽆,⾻}
  \definition{adj.}{desossado}
\end{EntryWithPhonetic}

\begin{EntryWithPhonetic}{无故}{wu2gu4}{4,9}{⽆,⽁}[HSK 7-9]
  \definition{adv.}{sem motivo; por nenhuma razão}
  \synonymref{无理}{wu2li3}
\end{EntryWithPhonetic}

\begin{EntryWithPhonetic}{无关}{wu2guan1}{4,6}{⽆,⼋}[HSK 6]
  \definition{v.}{não ter nada a ver com; nada a fazer | não envolver; ser irrelevante; não ter efeito sobre}
\end{EntryWithPhonetic}

\begin{EntryWithPhonetic}{无关紧要}{wu2guan1-jin3yao4}{4,6,10,9}{⽆,⼋,⽷,⾑}[HSK 7-9]
  \definition{expr.}{sem importância; irrelevante; indiferente; fragilidade; insignificante}
  \synonymref{无足轻重}{wu2zu2-qing1zhong4}
\end{EntryWithPhonetic}

\begin{EntryWithPhonetic}{无话可说}{wu2hua4-ke3shuo1}{4,8,5,9}{⽆,⾔,⼝,⾔}[HSK 7-9]
  \definition{expr.}{``Não ter nada a dizer.''; não havia nada a dizer, nenhuma opinião ou razão a apresentar}
  \synonymref{无可奉告}{wu2ke3feng4gao4}
\end{EntryWithPhonetic}

\begin{EntryWithPhonetic}{无济于事}{wu2ji4yu2shi4}{4,9,3,8}{⽆,⽔,⼆,⼅}[HSK 7-9]
  \definition{expr.}{não adiantou nada; não ajuda em nada; inútil; sem efeito; sem utilidade; sem sucesso}
  \antonymref{潜移默化}{qian2yi2-mo4hua4}
\end{EntryWithPhonetic}

\begin{EntryWithPhonetic}{无家可归}{wu2jia1-ke3gui1}{4,10,5,5}{⽆,⼧,⼝,⼹}[HSK 7-9]
  \definition{expr.}{sem"-teto; morador de rua; significa não ter um lar para onde voltar, referindo"-se a estar deslocado e sem"-teto; vagar sem rumo}
\end{EntryWithPhonetic}

\begin{EntryWithPhonetic}{无精打采}{wu2jing1-da3cai3}{4,14,5,8}{⽆,⽶,⼿,⾤}[HSK 7-9]
  \definition{expr.}{``Desanimado e abatido.''; apático; deprimido; descreve alguém que está apático, sem energia, infeliz ou desanimado; indisposto; desleixado}
  \synonymref{垂头丧气}{chui2tou2-sang4qi4}
  \antonymref{容光焕发}{rong2guang1-huan4fa1}
\end{EntryWithPhonetic}

\begin{EntryWithPhonetic}{无可奉告}{wu2ke3feng4gao4}{4,5,8,7}{⽆,⼝,⼤,⼝}[HSK 7-9]
  \definition{expr.}{``Sem comentários.''}
  \synonymref{无话可说}{wu2hua4-ke3shuo1}
\end{EntryWithPhonetic}

\begin{EntryWithPhonetic}{无可厚非}{wu2ke3hou4fei1}{4,5,9,8}{⽆,⼝,⼚,⾮}[HSK 7-9]
  \definition{expr.}{desculpável; compreensível; inegavelmente; é importante não ser excessivamente crítico; embora possam existir falhas, elas são perdoáveis e compreensíveis}
\end{EntryWithPhonetic}

\begin{EntryWithPhonetic}{无可奈何}{wu2ke3nai4he2}{4,5,8,7}{⽆,⼝,⼤,⼈}[HSK 7-9]
  \definition{expr.}{desamparado; não ter saída; estar completamente impotente; não ter alternativa; não há a mínima possibilidade; não há a menor ideia; não há absolutamente nenhuma possibilidade}
\end{EntryWithPhonetic}

\begin{EntryWithPhonetic}{无理}{wu2li3}{4,11}{⽆,⽟}[HSK 7-9]
  \definition{adj.}{irracional; injustificável}
  \definition{s.}{Matemática: irracional (número)}
  \synonymref{荒谬}{huang1miu4}
  \synonymref{畸形}{ji1xing2}
  \synonymref{无故}{wu2gu4}
  \antonymref{合理}{he2li3}
\end{EntryWithPhonetic}

\begin{EntryWithPhonetic}{无力}{wu2li4}{4,2}{⽆,⼒}[HSK 7-9]
  \definition{v.}{sentir"-se fraco; não ter força | ser incapaz; estar impossibilitado; não ter poder para}
  \seealsoref{无法}{wu2fa3}
  \synonymref{没劲}{mei2jin4}
  \synonymref{疲惫}{pi2bei4}
  \synonymref{疲劳}{pi2lao2}
  \antonymref{强劲}{qiang2jing4}
  \antonymref{有力}{you3li4}
\end{EntryWithPhonetic}

\begin{EntryWithPhonetic}{无聊}{wu2liao2}{4,11}{⽆,⽿}[HSK 4]
  \definition{adj.}{entediado; aborrecido; sentir"-se desinteressado porque não há nada para fazer | tolo; bobo; sem sentido; descreve palavras ou coisas ditas ou feitas como sem sentido e irritantes; descreve pessoas ou coisas como sem sentido e pouco atraentes}
\end{EntryWithPhonetic}

\begin{EntryWithPhonetic}{无论}{wu2lun4}{4,6}{⽆,⾔}[HSK 4]
  \definition{conj.}{não importa o quê; não importa como; independentemente de; indica que as condições são diferentes, mas resultado é o mesmo}
  \seealsoref{无论……也……}{wu2lun4 ye3}
\end{EntryWithPhonetic}

\begin{EntryWithPhonetic}{无论如何}{wu2lun4-ru2he2}{4,6,6,7}{⽆,⾔,⼥,⼈}[HSK 7-9]
  \definition{adv.}{em qualquer caso; de qualquer forma; como puder; em todo caso; pelo menos; por todos os meios possíveis; independentemente das circunstâncias, o resultado será o mesmo}
  \seealsoref{不管}{bu4guan3}
  \seealsoref{无论}{wu2lun4}
  \synonymref{不论}{bu2lun4}
\end{EntryWithPhonetic}

\begin{EntryWithPhonetic}{无论……也……}{wu2lun4 ye3}{4,6,3}{⽆,⾔,⼄}
  \definition{conj.}{não apenas\dots, (o que, quem, como, etc.), \dots}
\end{EntryWithPhonetic}

\begin{EntryWithPhonetic}{无奈}{wu2nai4}{4,8}{⽆,⼤}[HSK 5]
  \definition{conj.}{mas (infelizmente); no entanto}
  \definition{v.}{não poder evitar; não ter alternativa; não ter escolha; não haver nada a fazer}
\end{EntryWithPhonetic}

\begin{EntryWithPhonetic}{无能}{wu2neng2}{4,10}{⽆,⾁}[HSK 7-9]
  \definition{adj.}{incompetente; incapaz; incapaz de fazer qualquer coisa}
  \antonymref{才能}{cai2neng2}
  \antonymref{技能}{ji4neng2}
  \antonymref{能干}{neng2gan4}
  \antonymref{作用}{zuo4yong4}
\end{EntryWithPhonetic}

\begin{EntryWithPhonetic}{无能为力}{wu2neng2wei2li4}{4,10,4,2}{⽆,⾁,⼂,⼒}[HSK 7-9]
  \definition{expr.}{impotente; indefeso; incapaz de agir; incapaz de usar a força; com falta de força ou com força fraca}
  \synonymref{力不从心}{li4bu4cong2xin1}
  \synonymref{无可奈何}{wu2ke3nai4he2}
  \antonymref{力所能及}{li4suo3neng2ji2}
\end{EntryWithPhonetic}

\begin{EntryWithPhonetic}{无情}{wu2qing2}{4,11}{⽆,⼼}[HSK 7-9]
  \definition{adj.}{insensível; desalmado; sem emoção | impiedoso; intransigente; descreve uma atitude resoluta que não se preocupa em demonstrar respeito pelos outros | inexorável; descreve como as leis e a realidade não podem mudar de acordo com os desejos individuais}
  \synonymref{冷酷}{leng3ku4}
  \antonymref{深情}{shen1qing2}
\end{EntryWithPhonetic}

\begin{EntryWithPhonetic}{无情无义}{wu2qing2-wu2yi4}{4,11,4,3}{⽆,⼼,⽆,⼂}[HSK 7-9]
  \definition{expr.}{``Completamente desprovido de qualquer sentimento ou senso de justiça.''; frio e implacável; sem coração e ingrato; sem coração e sem fé}
  \synonymref{冷酷无情}{leng3ku4-wu2qing2}
\end{EntryWithPhonetic}

\begin{EntryWithPhonetic}{无穷}{wu2qiong2}{4,7}{⽆,⽳}[HSK 7-9]
  \definition{adj.}{infinito; sem fim; ilimitado; inesgotável; sem limites}
  \seealsoref{无数}{wu2shu4}
  \synonymref{无限}{wu2xian4}
  \antonymref{有限}{you3xian4}
\end{EntryWithPhonetic}

\begin{EntryWithPhonetic}{无人}{wu2ren2}{4,2}{⽆,⼈}
  \definition{adj.}{não tripulado | desabitado}
\end{EntryWithPhonetic}

\begin{EntryWithPhonetic}{无人机}{wu2ren2ji1}{4,2,6}{⽆,⼈,⽊}
  \definition{s.}{\emph{drone} | veículo aéreo não tripulado}
\end{EntryWithPhonetic}

\begin{EntryWithPhonetic}{无日}{wu2ri4}{4,4}{⽆,⽇}
  \definition{adv.}{Literário: o tempo todo; nenhum dia sequer | Literário: em breve; antes de muito tempo}
\end{EntryWithPhonetic}

\begin{EntryWithPhonetic}{无视}{wu2shi4}{4,8}{⽆,⾒}
  \definition{v.}{ignorar | desconsiderar}
\end{EntryWithPhonetic}

\begin{EntryWithPhonetic}{无数}{wu2shu4}{4,13}{⽆,⽁}[HSK 4]
  \definition{adj.}{incontável; inumerável | inseguro; incerto; não conhecer a história ou os detalhes internos; não ter certeza}
\end{EntryWithPhonetic}

\begin{EntryWithPhonetic}{无私}{wu2si1}{4,7}{⽆,⽲}[HSK 7-9]
  \definition{adj.}{altruísta; generoso; desinteresseiro; sem egoísmo}
\end{EntryWithPhonetic}

\begin{EntryWithPhonetic}{无所事事}{wu2suo3shi4shi4}{4,8,8,8}{⽆,⼾,⼅,⼅}[HSK 7-9]
  \definition{expr.}{não fazer nada; passar o tempo ocioso}
  \synonymref{无所作为}{wu2suo3zuo4wei2}
  \antonymref{废寝忘食}{fei4qin3-wang4shi2}
\end{EntryWithPhonetic}

\begin{EntryWithPhonetic}{无所谓}{wu2suo3wei4}{4,8,11}{⽆,⼾,⾔}[HSK 4]
  \definition{v.}{não pode ser designado como; não merece o nome de; ser incapaz de dizer ou contar | não ter importância; ser indiferente}
\end{EntryWithPhonetic}

\begin{EntryWithPhonetic}{无所作为}{wu2suo3zuo4wei2}{4,8,7,4}{⽆,⼾,⼈,⼂}[HSK 7-9]
  \definition{expr.}{``Sem tentar nada e sem conseguir nada.''; não tentar nada e não realizar nada; passivo; sem qualquer iniciativa ou motivação; irresponsável}
  \synonymref{无所事事}{wu2suo3shi4shi4}
  \antonymref{大有可为}{da4you3-ke3wei2}
  \antonymref{发愤图强}{fa1fen4-tu2qiang2}
  \antonymref{发奋图强}{fa1fen4-tu2qiang2}
\end{EntryWithPhonetic}

\begin{EntryWithPhonetic}{无条件}{wu2tiao2jian4}{4,7,6}{⽆,⽊,⼈}[HSK 7-9]
  \definition{adj.}{incondicional; sem pré-condições; irrestrito; nenhuma condição foi estabelecida; nenhuma condição foi proposta}
\end{EntryWithPhonetic}

\begin{EntryWithPhonetic}{无微不至}{wu2wei1-bu2zhi4}{4,13,4,6}{⽆,⼻,⼀,⾄}[HSK 7-9]
  \definition{expr.}{``De todas as maneiras possíveis.''; meticuloso; em todos os sentidos possíveis; com grande cuidado; significa ser muito atencioso e cuidadoso ao lidar com as pessoas}
\end{EntryWithPhonetic}

\begin{EntryWithPhonetic}{无线}{wu2xian4}{4,8}{⽆,⽷}[HSK 7-9]
  \definition{adj.}{sem fio; sem cabos}
\end{EntryWithPhonetic}

\begin{EntryWithPhonetic}{无线电}{wu2xian4dian4}{4,8,5}{⽆,⽷,⽥}[HSK 7-9]
  \definition[台]{s.}{rádio | sem fio; comunicação sem fio; comunicação por rádio; um tipo de comunicação onde o sinal não é transmitido por fios, mas sim pelo ar na forma de ondas eletromagnéticas}
  \synonymref{收音机}{shou1yin1ji1}
\end{EntryWithPhonetic}

\begin{EntryWithPhonetic}{无限}{wu2xian4}{4,8}{⽆,⾩}[HSK 4]
  \definition{adj.}{infinito; ilimitado; sem limites; sem fim à vista}
\end{EntryWithPhonetic}

\begin{EntryWithPhonetic}{无效}{wu2xiao4}{4,10}{⽆,⽁}[HSK 6]
  \definition{adj.}{sem efeito; de (ou sem) utilidade; em vão; (uma certa abordagem) é inútil e não pode resolver o problema ou atingir o objetivo | inválido; nulo e sem efeito; (um determinado comportamento) não é reconhecido por lei e os direitos relevantes não serão protegidos}
\end{EntryWithPhonetic}

\begin{EntryWithPhonetic}{无形}{wu2xing2}{4,7}{⽆,⼺}[HSK 7-9]
  \definition{adj.}{invisível; intangível; aquilo que tem função semelhante, mas não possui a forma ou o nome de determinada coisa; aquilo que não pode ser percebido pelos sentidos}
  \definition{adv.}{sutilmente; inconscientemente; imperceptivelmente}
\end{EntryWithPhonetic}

\begin{EntryWithPhonetic}{无形中}{wu2xing2zhong1}{4,7,4}{⽆,⼺,⼁}[HSK 7-9]
  \definition{adv.}{invisivelmente; imperceptivelmente; virtualmente; inconscientemente; numa situação em que algo tem substância, mas não tem nome, também se pode dizer que é intangível}
\end{EntryWithPhonetic}

\begin{EntryWithPhonetic}{无须}{wu2xu1}{4,9}{⽆,⾴}[HSK 7-9]
  \definition{adv.}{desnecessariamente; sem necessidade}
  \synonymref{不必}{bu2bi4}
  \synonymref{不用}{bu2yong4}
  \antonymref{必须}{bi4xu1}
\end{EntryWithPhonetic}

\begin{EntryWithPhonetic}{无氧}{wu2yang3}{4,10}{⽆,⽓}
  \definition{adj.}{anaeróbico}
\end{EntryWithPhonetic}

\begin{EntryWithPhonetic}{无疑}{wu2yi2}{4,14}{⽆,⽦}[HSK 5]
  \definition{adv.}{indubitavelmente; sem dúvida; sem sombra de dúvida}
\end{EntryWithPhonetic}

\begin{EntryWithPhonetic}{无意}{wu2yi4}{4,13}{⽆,⼼}[HSK 7-9]
  \definition{adv.}{involuntariamente; por acaso; conscientemente; não intencionalmente}
  \definition{v.}{não ter inclinação para; não ter intenção (de fazer algo)}
  \antonymref{存心}{cun2xin1}
  \antonymref{故意}{gu4yi4}
\end{EntryWithPhonetic}

\begin{EntryWithPhonetic}{无忧无虑}{wu2you1-wu2lv4}{4,7,4,10}{⽆,⼼,⽆,⾌}[HSK 7-9]
  \definition{expr.}{despreocupado; sem ansiedade; sem preocupações, a pessoa vive em estado de contentamento}
\end{EntryWithPhonetic}

\begin{EntryWithPhonetic}{无缘}{wu2yuan2}{4,12}{⽆,⽷}[HSK 7-9]
  \definition{v.}{não ter tido sorte (para fazer algo); sem ter sorte ou tribulação que atraem as pessoas}
\end{EntryWithPhonetic}

\begin{EntryWithPhonetic}{无知}{wu2zhi1}{4,8}{⽆,⽮}[HSK 7-9]
  \definition{adj.}{ignorante; falta de conhecimento; ignorância da razão}
  \antonymref{经验}{jing1yan4}
  \antonymref{知识}{zhi1shi5}
\end{EntryWithPhonetic}

\begin{EntryWithPhonetic}{无足轻重}{wu2zu2-qing1zhong4}{4,7,9,9}{⽆,⾜,⾞,⾥}[HSK 7-9]
  \definition{expr.}{insignificante; até mesmo coisas sem importância são descritas como insignificantes}
  \synonymref{无关紧要}{wu2guan1-jin3yao4}
\end{EntryWithPhonetic}

%%%%%%%%%% 吾 %%%%%%%%%%
\subsection*{吾}\addcontentsline{loh}{figure}{吾 \dpy{wu2}}

\begin{EntryWithPhonetic}{吾}{wu2}{7}{⼝}
  \definition*{s.}{Sobrenome: Wu}
  \definition{pron.}{eu; nós}
\end{EntryWithPhonetic}

%%%%%%%%%% 捂 %%%%%%%%%%
\subsection*{捂}\addcontentsline{loh}{figure}{捂 \dpy{wu2}}

\begin{EntryWithPhonetic}{捂}{wu2}{10}{⼿}
  \definition{v.}{encobrir; esconder; evitar; falar de forma vaga ou evasiva: ignorar}
  \seeref{wu3}
  \antonymref{揭}{jie1}
\end{EntryWithPhonetic}

%%%%%%%%%% 五 %%%%%%%%%%
\subsection*{五}\addcontentsline{loh}{figure}{五 \dpy{wu3}}

\begin{EntryWithPhonetic}{五}{wu3}{4}{⼆}[HSK 1]
  \definition*{s.}{Sobrenome: Wu}
  \definition{num.}{cinco; 5}
  \definition{s.}{uma nota da escala em Gongchepu (工尺谱), correspondente a 6 na notação musical numerada}
  \seealsoref{工尺谱}{gong1 che3 pu3}
\end{EntryWithPhonetic}

\begin{EntryWithPhonetic}{五花八门}{wu3hua1-ba1men2}{4,7,2,3}{⼆,⾋,⼋,⾨}[HSK 7-9]
  \definition{expr.}{de todos os tipos; uma metáfora para uma grande variedade de estilos ou infinitas variações; multifacetado; de uma grande (ou rica) variedade}
  \synonymref{丰富多彩}{feng1fu4-duo1cai3}
  \synonymref{各式各样}{ge4shi4-ge4yang4}
  \synonymref{千变万化}{qian1bian4-wan4hua4}
  \synonymref{五颜六色}{wu3yan2liu4se4}
\end{EntryWithPhonetic}

\begin{EntryWithPhonetic}{五体投地}{wu3ti3tou2di4}{4,7,7,6}{⼆,⼈,⼿,⼟}
  \definition{expr.}{prostrar-se em admiração | adular alguém}
  \antonymref{不以为然}{bu4yi3wei2ran2}
\end{EntryWithPhonetic}

\begin{EntryWithPhonetic}{五五}{wu3wu3}{4,4}{⼆,⼆}
  \definition{num.}{50--50}
  \definition{s.}{igual (partilha, parceria, etc.)}
\end{EntryWithPhonetic}

\begin{EntryWithPhonetic}{五星级}{wu3xing1ji2}{4,9,6}{⼆,⽇,⽷}[HSK 7-9]
  \definition{adj.}{cinco estrelas}
  \definition{s.}{cinco estrelas (hotel, restautante)}
\end{EntryWithPhonetic}

\begin{EntryWithPhonetic}{五颜六色}{wu3yan2liu4se4}{4,15,4,6}{⼆,⾴,⼋,⾊}[HSK 4]
  \definition{adj.}{de várias (ou todas) cores; multicolorido; colorido}
  \synonymref{五花八门}{wu3hua1-ba1men2}
\end{EntryWithPhonetic}

\begin{EntryWithPhonetic}{五岳}{wu3yue4}{4,8}{⼆,⼭}
  \definition*{s.}{Cinco Montanhas Sagradas dos taoístas, a saber: Monte Tai (泰山) na província de Shandong (Pico Oriental), Monte Hua (华山) na província de Shaanxi (Pico Ocidental), Monte Heng (衡山) na província de Hunan (Pico Meridional), Monte Heng (恒山) na província de Shanxi (Pico Setentrional) e Monte Song (嵩山) na província de Henan (Pico Central) são as cinco montanhas mais famosas da história da China}
  \seealsoref{衡山}{heng2shan1}
  \seealsoref{恒山}{heng2shan1}
  \seealsoref{华山}{hua4shan1}
  \seealsoref{嵩山}{song1shan1}
  \seealsoref{泰山}{tai4shan1}
\end{EntryWithPhonetic}

%%%%%%%%%% 午 %%%%%%%%%%
\subsection*{午}\addcontentsline{loh}{figure}{午 \dpy{wu3}}

\begin{EntryWithPhonetic}{午}{wu3}{4}{⼗}
  \definition{s.}{meio"-dia; período entre 11h00 e 13h00 | wu (sétimo dos doze Ramos Terrestres)}
\end{EntryWithPhonetic}

\begin{EntryWithPhonetic}{午餐}{wu3can1}{4,16}{⼗,⾷}[HSK 2]
  \definition[份,顿,次]{s.}{almoço}
  \seealsoref{午饭}{wu3fan4}
\end{EntryWithPhonetic}

\begin{EntryWithPhonetic}{午饭}{wu3fan4}{4,7}{⼗,⾷}[HSK 1]
  \definition[顿]{s.}{almoço}
  \seealsoref{午餐}{wu3can1}
\end{EntryWithPhonetic}

\begin{EntryWithPhonetic}{午后}{wu3hou4}{4,6}{⼗,⼝}
  \definition{s.}{tarde | período da tarde}
\end{EntryWithPhonetic}

\begin{EntryWithPhonetic}{午前}{wu3qian2}{4,9}{⼗,⼑}
  \definition{s.}{\emph{A.M.} | manhã | período da manhã}
\end{EntryWithPhonetic}

\begin{EntryWithPhonetic}{午睡}{wu3shui4}{4,13}{⼗,⽬}[HSK 2]
  \definition{s.}{\emph{siesta}; cochilo da tarde; soneca do meio"-dia}
  \definition{v.}{tirar uma soneca depois do almoço}
\end{EntryWithPhonetic}

\begin{EntryWithPhonetic}{午休}{wu3xiu1}{4,6}{⼗,⼈}
  \definition{s.}{pausa para almoço | cochilo na hora do almoço | intervalo do meio"-dia}
\end{EntryWithPhonetic}

\begin{EntryWithPhonetic}{午宴}{wu3yan4}{4,10}{⼗,⼧}
  \definition{s.}{banquete de almoço}
\end{EntryWithPhonetic}

\begin{EntryWithPhonetic}{午夜}{wu3ye4}{4,8}{⼗,⼣}
  \definition{s.}{meia-noite}
\end{EntryWithPhonetic}

%%%%%%%%%% 武 %%%%%%%%%%
\subsection*{武}\addcontentsline{loh}{figure}{武 \dpy{wu3}}

\begin{EntryWithPhonetic}{武}{wu3}{8}{⽌}
  \definition*{s.}{Sobrenome: Wu}
  \definition{adj.}{valente; corajoso}
  \definition{s.}{militar; atividades e comportamentos relacionados a habilidades militares e de combate | arte marcial | passo; meio passo; pegadas}
  \antonymref{文}{wen2}
\end{EntryWithPhonetic}

\begin{EntryWithPhonetic}{武大戏}{wu3 da4xi4}{8,3,6}{⽌,⼤,⼽}
  \definition*{s.}{Drama de Luta Acrobática | Drama Wu}
\end{EntryWithPhonetic}

\begin{EntryWithPhonetic}{武断}{wu3duan4}{8,11}{⽌,⽄}
  \definition{adj.}{arbitrário | dogmático | subjetivo}
\end{EntryWithPhonetic}

\begin{EntryWithPhonetic}{武官}{wu3guan1}{8,8}{⽌,⼧}
  \definition{s.}{oficial militar | adido militar}
\end{EntryWithPhonetic}

\begin{EntryWithPhonetic}{武力}{wu3li4}{8,2}{⽌,⼒}[HSK 7-9]
  \definition{s.}{força | força militar; poderio armado; poderio bélico; força das armas | força armada (poder)}[霸道政策依靠武力。===Políticas autoritárias dependem da força.]
  \synonymref{暴力}{bao4li4}
\end{EntryWithPhonetic}

\begin{EntryWithPhonetic}{武器}{wu3qi4}{8,16}{⽌,⼝}[HSK 3]
  \definition[批,个,件,种]{s.}{arma; equipamentos e dispositivos utilizados diretamente para matar inimigos ou destruir suas instalações defensivas e ofensivas | armas; armamento; metáfora usada como ferramenta de luta}
\end{EntryWithPhonetic}

\begin{EntryWithPhonetic}{武士}{wu3shi4}{8,3}{⽌,⼠}
  \definition{s.}{samurai | guerreiro}
\end{EntryWithPhonetic}

\begin{EntryWithPhonetic}{武术}{wu3shu4}{8,5}{⽌,⽊}[HSK 3]
  \definition[种,套,门]{s.}{arte marcial; autodefesa; \emph{wushu}; um esporte tradicional chinês que utiliza técnicas com os punhos, pernas, pés ou armas como facas e espadas}
\end{EntryWithPhonetic}

\begin{EntryWithPhonetic}{武艺}{wu3yi4}{8,4}{⽌,⾋}
  \definition{s.}{arte marcial | habilidade militar}
\end{EntryWithPhonetic}

\begin{EntryWithPhonetic}{武装}{wu3zhuang1}{8,12}{⽌,⾐}[HSK 7-9]
  \definition{s.}{armas; equipamento militar; uniforme de combate | forças armadas}
  \definition{v.}{armar; equipar com armas; fornecer armas}
\end{EntryWithPhonetic}

%%%%%%%%%% 侮 %%%%%%%%%%
\subsection*{侮}\addcontentsline{loh}{figure}{侮 \dpy{wu3}}

\begin{EntryWithPhonetic}{侮}{wu3}{9}{⼈}
  \definition{v.}{insultar; intimidar | humilhar; caluniar; assediar (assédio moral); menosprezar}
\end{EntryWithPhonetic}

\begin{EntryWithPhonetic}{侮辱}{wu3ru3}{9,10}{⼈,⾠}[HSK 7-9]
  \definition{v.}{insultar; humilhar; submeter alguém a indignidades; a dignidade e a reputação de uma pessoa são prejudicadas por palavras ou ações | molestar; assediar (uma mulher)}
  \synonymref{耻辱}{chi3ru3}
  \synonymref{欺负}{qi1fu5}
  \antonymref{敬重}{jing4zhong4}
  \antonymref{尊敬}{zun1jing4}
  \antonymref{尊重}{zun1zhong4}
\end{EntryWithPhonetic}

%%%%%%%%%% 捂 %%%%%%%%%%
\subsection*{捂}\addcontentsline{loh}{figure}{捂 \dpy{wu3}}

\begin{EntryWithPhonetic}{捂}{wu3}{10}{⼿}[HSK 7-9]
  \definition{v.}{selar; cobrir; abafar; embrulhar; fechar}
  \seeref{wu2}
  \antonymref{揭}{jie1}
\end{EntryWithPhonetic}

%%%%%%%%%% 舞 %%%%%%%%%%
\subsection*{舞}\addcontentsline{loh}{figure}{舞 \dpy{wu3}}

\begin{EntryWithPhonetic}{舞}{wu3}{14}{⾇}[HSK 5]
  \definition[支,段,个]{s.}{dança | palco; metáfora do domínio das atividades sociais}
  \definition{v.}{mover"-se como numa dança | dançar com algo nas mãos; brincar com | florescer; empunhar; brandir | esvoaçar | fazer malabarismos; brincar com}
\end{EntryWithPhonetic}

\begin{EntryWithPhonetic}{舞抃}{wu3bian4}{14,7}{⾇,⼿}
  \definition{s.}{dançar por prazer}
\end{EntryWithPhonetic}

\begin{EntryWithPhonetic}{舞蹈}{wu3dao3}{14,17}{⾇,⾜}[HSK 6]
  \definition[段,支,场,个]{s.}{dança; uma forma de arte que usa movimentos rítmicos como principal meio de expressão, podendo expressar a vida, os pensamentos e os sentimentos das pessoas, geralmente acompanhada de música}
  \definition{v.}{dançar}
\end{EntryWithPhonetic}

\begin{EntryWithPhonetic}{舞会}{wu3hui4}{14,6}{⾇,⼈}
  \definition{s.}{baile}
\end{EntryWithPhonetic}

\begin{EntryWithPhonetic}{舞会舞}{wu3hui4wu3}{14,6,14}{⾇,⼈,⾇}
  \definition{s.}{baile}
\end{EntryWithPhonetic}

\begin{EntryWithPhonetic}{舞台}{wu3tai2}{14,5}{⾇,⼝}[HSK 3]
  \definition[个]{s.}{palco; plataforma elevada usada exclusivamente para apresentações artísticas, geralmente localizada na parte frontal de teatros e auditórios | palco; metáfora do campo das atividades sociais}
\end{EntryWithPhonetic}

\begin{EntryWithPhonetic}{舞厅}{wu3ting1}{14,4}{⾇,⼚}[HSK 7-9]
  \definition[家,间]{s.}{salão de dança; salão de baile}
\end{EntryWithPhonetic}

\begin{EntryWithPhonetic}{舞厅舞}{wu3ting1wu3}{14,4,14}{⾇,⼚,⾇}
  \definition{s.}{dança de salão}
\end{EntryWithPhonetic}

%%%%%%%%%% 乌 %%%%%%%%%%
\subsection*{乌}\addcontentsline{loh}{figure}{乌 \dpy{wu4}}

\begin{EntryWithPhonetic}{乌}{wu4}{4}{⼃}
  \definition{s.}{sapatos u-la (sapatos com forro de grama para aquecimento) | um tipo de grama chamada u-la}
  \seeref{wu1}
\end{EntryWithPhonetic}

%%%%%%%%%% 勿 %%%%%%%%%%
\subsection*{勿}\addcontentsline{loh}{figure}{勿 \dpy{wu4}}

\begin{EntryWithPhonetic}{勿}{wu4}{4}{⼓}[HSK 7-9]
  \definition{adv.}{não; indica proibição ou dissuasão, como 不要}
  \seealsoref{不要}{bu2yao4}
\end{EntryWithPhonetic}

%%%%%%%%%% 务 %%%%%%%%%%
\subsection*{务}\addcontentsline{loh}{figure}{务 \dpy{wu4}}

\begin{EntryWithPhonetic}{务}{wu4}{5}{⼒}
  \definition*{s.}{Sobrenome: Wu}
  \definition{s.}{caso; negócio | usado em nomes de lugares}
  \definition{v.}{engajar"-se em; dedicar seus esforços a | procurar; perseguir; ir atrás | estar envolvido em; dedicar"-se a; envolver"-se em; comprometer"-se com | deve; deveria; ter certeza de}
\end{EntryWithPhonetic}

\begin{EntryWithPhonetic}{务必}{wu4bi4}{5,5}{⼒,⼼}[HSK 7-9]
  \definition{adv.}{deve; ter certeza de; necessariamente; usado principalmente em frases afirmativas}
  \seealsoref{一定}{yi2ding4}
  \synonymref{必须}{bi4xu1}
  \synonymref{必需}{bi4xu1}
\end{EntryWithPhonetic}

\begin{EntryWithPhonetic}{务实}{wu4shi2}{5,8}{⼒,⼧}[HSK 7-9]
  \definition{adj.}{pragmático; prático; pé-no-chão; focado em resultados práticos, evitando superficialidades}
  \definition{v.}{lidar com assuntos concretos relacionados ao trabalho; discutir e estudar problemas específicos; executar tarefas específicas}
\end{EntryWithPhonetic}

%%%%%%%%%% 物 %%%%%%%%%%
\subsection*{物}\addcontentsline{loh}{figure}{物 \dpy{wu4}}

\begin{EntryWithPhonetic}{物}{wu4}{8}{⽜}
  \definition{s.}{coisa; matéria; objeto | mundo exterior distinto de si mesmo; outras pessoas; refere"-se a outras pessoas além de si mesmo ou ao ambiente em relação a si mesmo | essência; conteúdo; substância | criatura; criação}
\end{EntryWithPhonetic}

\begin{EntryWithPhonetic}{物价}{wu4jia4}{8,6}{⽜,⼈}[HSK 5]
  \definition[个]{s.}{preços das commodities; preços das matérias-primas; preço das mercadorias}
\end{EntryWithPhonetic}

\begin{EntryWithPhonetic}{物理}{wu4li3}{8,11}{⽜,⽟}
  \definition{s.}{física (disciplina)}
\end{EntryWithPhonetic}

\begin{EntryWithPhonetic}{物流}{wu4liu2}{8,10}{⽜,⽔}[HSK 7-9]
  \definition{s.}{logística; a movimentação de mercadorias de um lugar para outro, incluindo embalagem, armazenamento e transporte}
  \synonymref{货运}{huo4yun4}
\end{EntryWithPhonetic}

\begin{EntryWithPhonetic}{物品}{wu4pin3}{8,9}{⽜,⼝}[HSK 6]
  \definition[件,个]{s.}{artigos; itens; bens}
\end{EntryWithPhonetic}

\begin{EntryWithPhonetic}{物体}{wu4ti3}{8,7}{⽜,⼈}[HSK 7-9]
  \definition[个]{s.}{corpo; substância; objeto; coisa}
  \synonymref{东西}{dong1xi5}
\end{EntryWithPhonetic}

\begin{EntryWithPhonetic}{物业}{wu4ye4}{8,5}{⽜,⼀}[HSK 5]
  \definition[处]{s.}{propriedade; gestão de propriedades; gestão patrimonial; administração de imóveis | empresa de administração de imóveis; empresa de gestão imobiliária; empresa de administração de bens imóveis}
\end{EntryWithPhonetic}

\begin{EntryWithPhonetic}{物证}{wu4zheng4}{8,7}{⽜,⾔}[HSK 7-9]
  \definition[件]{s.}{provas materiais | evidências físicas | prova}
\end{EntryWithPhonetic}

\begin{EntryWithPhonetic}{物质}{wu4zhi4}{8,8}{⽜,⾙}[HSK 5]
  \definition[种,类,个]{s.}{matéria; substância; algo que existe além do espírito, que pode ser visto, tocado, cheirado ou detectado por instrumentos científicos | material; meios de subsistência; coisas que permitem às pessoas viver ou viver melhor, como comida, roupas, casas, dinheiro, etc.}
\end{EntryWithPhonetic}

\begin{EntryWithPhonetic}{物资}{wu4zi1}{8,10}{⽜,⾙}[HSK 7-9]
  \definition[份]{s.}{suprimentos; bens e materiais; recursos materiais necessários para a produção e a vida diária}
  \synonymref{物质}{wu4zhi4}
  \synonymref{资源}{zi1yuan2}
\end{EntryWithPhonetic}

%%%%%%%%%% 误 %%%%%%%%%%
\subsection*{误}\addcontentsline{loh}{figure}{误 \dpy{wu4}}

\begin{EntryWithPhonetic}{误}{wu4}{9}{⾔}[HSK 6]
  \definition{adj.}{errado; falso; impreciso | acidental}
  \definition{adv.}{por engano; por acidente; não intencional}
  \definition{s.}{engano; erro}
  \definition{v.}{perder | dificultar; impedir; prejudicar | confundir; entender mal; cometer um erro | causar desvantagem a. causar dano}
\end{EntryWithPhonetic}

\begin{EntryWithPhonetic}{误差}{wu4cha1}{9,9}{⾔,⼯}[HSK 7-9]
  \definition{s.}{erro; a diferença entre o valor medido ou outra aproximação e o valor verdadeiro é chamada de erro | erro; inconsistência}
  \synonymref{差错}{cha1cuo4}
  \synonymref{过错}{guo4cuo4}
  \synonymref{过失}{guo4shi1}
  \synonymref{偏差}{pian1cha1}
  \synonymref{缺点}{que1dian3}
\end{EntryWithPhonetic}

\begin{EntryWithPhonetic}{误导}{wu4dao3}{9,6}{⾔,⼨}[HSK 7-9]
  \definition{v.}{enganar; orientar incorretamente}
  \antonymref{提醒}{ti2/xing3}
\end{EntryWithPhonetic}

\begin{EntryWithPhonetic}{误点}{wu4/dian3}{9,9}{⾔,⽕}
  \definition{v.+compl.}{atrasar | chegar tarde}
\end{EntryWithPhonetic}

\begin{EntryWithPhonetic}{误会}{wu4hui4}{9,6}{⾔,⼈}[HSK 4]
  \definition[场]{s.}{mal"-entendido; desentendimentos ou conflitos decorrentes de mal"-entendidos}
  \definition{v.}{entender mal; entender errado; interpretar mal; não entender; não entender corretamente o significado}
\end{EntryWithPhonetic}

\begin{EntryWithPhonetic}{误解}{wu4jie3}{9,13}{⾔,⾓}[HSK 5]
  \definition[个,种]{s.}{equívoco; mal"-entendido; desentendimento}
  \definition{v.}{interpretar mal; interpretar erroneamente; não compreender corretamente}
\end{EntryWithPhonetic}

\begin{EntryWithPhonetic}{误区}{wu4qu1}{9,4}{⾔,⼖}[HSK 7-9]
  \definition[个]{s.}{equívoco; a área de erro; zona errônea; ideia equivocada persistente; refere"-se a equívocos ou práticas errôneas de longa data}
  \synonymref{埋伏}{mai2fu2}
  \synonymref{圈套}{quan1tao4}
  \synonymref{阴谋}{yin1mou2}
\end{EntryWithPhonetic}

%%%%%%%%%% 恶 %%%%%%%%%%
\subsection*{恶}\addcontentsline{loh}{figure}{恶 \dpy{wu4}}

\begin{EntryWithPhonetic}{恶}{wu4}{10}{⼼}
  \definition{v.}{não gostar; odiar; detestar; repugnar}
  \seeref{e3}
  \seeref{e4}
  \seeref{wu1}
\end{EntryWithPhonetic}

%%%%%%%%%% 雾 %%%%%%%%%%
\subsection*{雾}\addcontentsline{loh}{figure}{雾 \dpy{wu4}}

\begin{EntryWithPhonetic}{雾}{wu4}{13}{⾬}[HSK 7-9]
  \definition[层,场,阵]{s.}{neblina; pequenas gotas de água condensadas do vapor de água | pulverização fina; como muitas pequenas gotas de água na neblina}
\end{EntryWithPhonetic}

\begin{EntryWithPhonetic}{雾气}{wu4qi4}{13,4}{⾬,⽓}
  \definition{s.}{nevoeiro; névoa; neblina; vapor}
\end{EntryWithPhonetic}

%%%%% EOF %%%%%

