%%%
%%% Radical "⼫"
%%%

\section*{Radical 44: ``⼫''}\addcontentsline{toc}{section}{Radical 44: ⼫}

\begin{entry}{尺}{4}{⼫}
  \begin{phonetics}{尺}{che3}
    \definition{s.}{(tom) uma nota da escala em Gongchepu (工尺谱), correspondente a 2 na notação musical numerada}
  \seealsoref{工尺谱}{gong1 che3 pu3}
  \end{phonetics}
  \begin{phonetics}{尺}{chi3}[][HSK 4]
    \definition{clas.}{chi, uma unidade de comprimento (=13 metros)}
    \definition[支,把]{s.}{régua; instrumentos de medição | um instrumento no formato de uma régua}
  \end{phonetics}
\end{entry}

\begin{entry}{尺子}{4,3}{⼫、⼦}
  \begin{phonetics}{尺子}{chi3zi5}[][HSK 4]
    \definition[把]{s.}{régua de madeira ou metal para orientar a caneta ou o lápis para desenhar linhas ou fazer medições}
  \end{phonetics}
\end{entry}

\begin{entry}{尺寸}{4,3}{⼫、⼨}
  \begin{phonetics}{尺寸}{chi3 cun4}[][HSK 4]
    \definition{s.}{tamanho; medida; dimensão}
  \end{phonetics}
\end{entry}

\begin{entry}{尽}{6}{⼫}
  \begin{phonetics}{尽}{jin3}
    \definition{adv.}{na maior extensão possível | na extremidade mais distante de | usado antes de palavras que indicam direção, o mesmo que 最 | de agora em diante}
    \definition{prep.}{dentro dos limites de}
    \definition{v.}{dar prioridade a | deixar que certas pessoas ou coisas tenham precedência}
  \seealsoref{最}{zui4}
  \end{phonetics}
  \begin{phonetics}{尽}{jin4}[][HSK 6]
    \definition*{s.}{Sobrenome Jin}
    \definition{adj.}{exausto; acabado | ao máximo; ao limite | tudo; exaustivo}
    \definition{v.}{esgotar | tentar o seu melhor; fazer o melhor uso possível | morrer; falecer | terminar | chegar ao fim ao máximo; alcançar extremos}
  \end{phonetics}
\end{entry}

\begin{entry}{尽力}{6,2}{⼫、⼒}
  \begin{phonetics}{尽力}{jin4li4}[][HSK 4]
    \definition{v.+compl.}{esforçar-se ao máximo; esforçar-se ao máximo; usar toda a sua força; fazer algo com seu melhor esforço}
  \end{phonetics}
\end{entry}

\begin{entry}{尽可能}{6,5,10}{⼫、⼝、⾁}
  \begin{phonetics}{尽可能}{jin3 ke3 neng2}[][HSK 5]
    \definition{adv.}{na medida do possível; com o melhor de sua capacidade; tentar fazer algo, atingir um determinado nível ou extensão}
  \end{phonetics}
\end{entry}

\begin{entry}{尽快}{6,7}{⼫、⼼}
  \begin{phonetics}{尽快}{jin3kuai4}[][HSK 4]
    \definition{adv.}{com toda a velocidade; o mais rápido possível; o mais breve possível}
  \end{phonetics}
\end{entry}

\begin{entry}{尽量}{6,12}{⼫、⾥}
  \begin{phonetics}{尽量}{jin3liang4}[][HSK 3]
    \definition{adv.}{tanto quanto possível; da melhor maneira possível}
  \end{phonetics}
\end{entry}

\begin{entry}{尽管}{6,14}{⼫、⽵}
  \begin{phonetics}{尽管}{jin3guan3}[][HSK 5]
    \definition{adv.}{justo; livremente; faça o que quiser, não se preocupe, não há restrições de movimento ou comportamento}
    \definition{conj.}{no entanto; embora; apesar de ; normalmente usado no início de uma frase anterior para introduzir um fato, seguido de 但是, etc. para introduzir um resultado que o fato não deveria ter; às vezes, também pode ser usado no início de uma frase posterior.}
  \seealsoref{但是}{dan4 shi4}
  \end{phonetics}
\end{entry}

\begin{entry}{尾}{7}{⼫}
  \begin{phonetics}{尾}{wei3}
    \definition*{s.}{Wei, sexta das vinte e oito constelações nas quais a esfera celeste foi dividida, consistindo de nove estrelas em forma de gancho em Escorpião| Wei, uma das mansões lunares | Sobrenome Wei}
    \definition{clas.}{usado para peixes}
    \definition{s.}{cauda; traseira | parte semelhante a uma cauda | fim | parte restante (ou inacabada); remanescente; a parte fora da parte principal; negócio inacabado}
  \end{phonetics}
  \begin{phonetics}{尾}{yi3}
    \definition{s.}{rabo do cavalo | parte posterior pontiaguda de um gafanhoto etc.}
  \end{phonetics}
\end{entry}

\begin{entry}{尾巴}{7,4}{⼫、⼰}
  \begin{phonetics}{尾巴}{wei3ba5}[][HSK 4]
    \definition{s.}{cauda; projeções na extremidade do corpo de certos animais | parte semelhante a uma cauda; refere-se, em geral, ao final de algo | apêndice; anexo; adepto servil; pessoa que segue ou concorda com outra pessoa | (figura de linguagem) alguém que faz sombra a outro | fim; remanescente; parte restante (ou inacabada)}
  \end{phonetics}
\end{entry}

\begin{entry}{尿}{7}{⼫}
  \begin{phonetics}{尿}{niao4}
    \definition[泡]{s.}{urina}
    \definition{v.}{urinar}
  \end{phonetics}
  \begin{phonetics}{尿}{sui1}
    \definition{s.}{(coloquial) urina}
  \end{phonetics}
\end{entry}

\begin{entry}{局}{7}{⼫}
  \begin{phonetics}{局}{ju2}[][HSK 4,6]
    \definition{adj.}{limitado; confinado}
    \definition{clas.}{\emph{set}; jogo; turno}
    \definition{s.}{tabuleiro de xadrez | situação; estado de coisas | generosidade de espírito; extensão da tolerância de alguém | festa; reunião; refere-se a certas reuniões | ardil; armadilha | parte; porção; papel | escritório; agência; agências governamentais divididas por negócios | significa ``loja'' em nomes de lojas | departamento; agência; nomes de certas entidades empresariais | escritório; usado como nome de uma instituição ou outro local de negócios}
  \end{phonetics}
\end{entry}

\begin{entry}{局长}{7,4}{⼫、⾧}
  \begin{phonetics}{局长}{ju2 zhang3}[][HSK 5]
    \definition[位,个]{s.}{comissário; diretor; principais chefes de gabinete do governo}
  \end{phonetics}
\end{entry}

\begin{entry}{局面}{7,9}{⼫、⾯}
  \begin{phonetics}{局面}{ju2mian4}[][HSK 5]
    \definition[种]{s.}{aspecto; fase; situação; o estado das coisas em um período de tempo, em sua maior parte abstraído | escopo; escala}
  \end{phonetics}
\end{entry}

\begin{entry}{屁}{7}{⼫}
  \begin{phonetics}{屁}{pi4}
    \definition{s.}{vento (ou gás) (dos intestinos); peido | (vulgar) bobagem; merda; lixo | quadril; bunda}
  \end{phonetics}
\end{entry}

\begin{entry}{屁股}{7,8}{⼫、⾁}
  \begin{phonetics}{屁股}{pi4gu5}
    \definition{s.}{nádega | quadris}
  \end{phonetics}
\end{entry}

\begin{entry}{屁话}{7,8}{⼫、⾔}
  \begin{phonetics}{屁话}{pi4hua4}
    \definition{s.}{absurdo | tolice | besteira}
  \end{phonetics}
\end{entry}

\begin{entry}{层}{7}{⼫}
  \begin{phonetics}{层}{ceng2}[][HSK 2]
    \definition{clas.}{usado para coisas que se sobrepõem e se acumulam, como andares, camadas e estratos | usado para coisas que podem ser divididas em itens e etapas | usado para coisas que podem ser removidas ou apagadas da superfície de um objeto}
    \definition{s.}{camada; nível; coisas que se sobrepõem | nível; classificação; camada}
    \definition{v.}{sobrepor; empilhar camada sobre camada}
  \end{phonetics}
\end{entry}

\begin{entry}{层次}{7,6}{⼫、⽋}
  \begin{phonetics}{层次}{ceng2ci4}[][HSK 5]
    \definition{s.}{disposição ordenada do conteúdo (de um discurso ou texto) | nível ou estrutura administrativa; distinções entre a mesma coisa devido a diferenças de tamanho, altura, etc. | nível; níveis de afiliação}
  \end{phonetics}
\end{entry}

\begin{entry}{层层}{7,7}{⼫、⼫}
  \begin{phonetics}{层层}{ceng2ceng2}
    \definition{s.}{camada sobre camada}
  \end{phonetics}
\end{entry}

\begin{entry}{层面}{7,9}{⼫、⾯}
  \begin{phonetics}{层面}{ceng2 mian4}[][HSK 6]
    \definition[个]{s.}{escopo; alcance | aspecto; campo}
  \end{phonetics}
\end{entry}

\begin{entry}{居}{8}{⼫}
  \begin{phonetics}{居}{ju1}
    \definition*{s.}{Sobrenome Ju}
    \definition{s.}{residência; casa | restaurante (em nomes de restaurantes)}
    \definition{v.}{residir; morar; viver | ocupar uma determinada posição; ocupar (um lugar); estar (em uma determinada posição) | reivindicar; afirmar | armazenar; guardar | ficar parado; estar parado}
  \end{phonetics}
\end{entry}

\begin{entry}{居民}{8,5}{⼫、⽒}
  \begin{phonetics}{居民}{ju1min2}[][HSK 4]
    \definition[个,户,位]{s.}{residente; habitante; pessoas que estão fixas em um único lugar}
  \end{phonetics}
\end{entry}

\begin{entry}{居住}{8,7}{⼫、⼈}
  \begin{phonetics}{居住}{ju1zhu4}[][HSK 4]
    \definition{v.}{viver; residir; morar; habitar}
  \end{phonetics}
\end{entry}

\begin{entry}{居然}{8,12}{⼫、⽕}
  \begin{phonetics}{居然}{ju1ran2}[][HSK 5]
    \definition{adv.}{inesperadamente; para surpresa de alguém; além da expectativa (expressão idiomática) |}
    \definition{v.}{ir tão longe a ponto de; ter a impudência de; ter o descaramento de;}
  \end{phonetics}
\end{entry}

\begin{entry}{屈}{8}{⼫}
  \begin{phonetics}{屈}{qu1}
    \definition*{s.}{Sobrenome Qu}
    \definition[个]{s.}{injustiça; tratamento injusto | erro; queixa; injustiça}
    \definition{v.}{dobrar; curvar; encurvar | subjugar; submeter | tratar mal; tratar injustamente (ou deslealmente) | estar errado}
  \end{phonetics}
\end{entry}

\begin{entry}{屈原}{8,10}{⼫、⼚}
  \begin{phonetics}{屈原}{qu1yuan2}
    \definition*{s.}{Qu Yuan, poeta, é uma figura histórica famosa na cultura chinesa que viveu durante o Período dos Reinos Combatentes (340-278 a.C.).}
  \end{phonetics}
\end{entry}

\begin{entry}{届}{8}{⼫}
  \begin{phonetics}{届}{jie4}[][HSK 5]
    \definition{clas.}{sessões (de uma conferência); anos (de graduação); quantificador, ligeiramente equivalente a 次, usado para reuniões regulares ou turmas de formandos, etc.}
    \definition{v.}{vencer o prazo}
  \seealsoref{次}{ci4}
  \end{phonetics}
\end{entry}

\begin{entry}{屋}{9}{⼫}
  \begin{phonetics}{屋}{wu1}[][HSK 5]
    \definition[间,座]{s.}{casa | quarto}
  \end{phonetics}
\end{entry}

\begin{entry}{屋子}{9,3}{⼫、⼦}
  \begin{phonetics}{屋子}{wu1zi5}[][HSK 3]
    \definition[间,座,栋]{s.}{quarto; sala}
  \end{phonetics}
\end{entry}

\begin{entry}{屌}{9}{⼫}
  \begin{phonetics}{屌}{diao3}
    \definition{adj.}{(gíria) legal ou extraordinário}
    \definition{s.}{órgão genital masculino; pênis}
    \definition{v.}{(cantonês) foder}
  \end{phonetics}
\end{entry}

\begin{entry}{屌丝}{9,5}{⼫、⼀}
  \begin{phonetics}{屌丝}{diao3si1}
    \definition{adj.}{panaca | zé-ninguém | (gíria de \emph{Internet}) \emph{looser}}
  \end{phonetics}
\end{entry}

\begin{entry}{屎}{9}{⼫}
  \begin{phonetics}{屎}{shi3}
    \definition{s.}{fezes | excrementos | (forma ligada) secreção (do ouvido, olho, etc.)}
  \end{phonetics}
\end{entry}

\begin{entry}{屏}{9}{⼫}
  \begin{phonetics}{屏}{bing1}
    \definition{s.}{antigamente, referia-se à pequena parede de tela em frente ao portão de um antigo palácio; no chinês moderno, também é usado como uma palavra humilde para expressar o significado de 惶恐}
  \seealsoref{惶恐}{huang2kong3}
  \end{phonetics}
  \begin{phonetics}{屏}{bing3}
    \definition*{s.}{Sobrenome Bing}
    \definition{v.}{prender (a respiração); conter a respiração | rejeitar; livrar-se de; remover; pôr (colocar) de lado; abandonar; descartar}
  \end{phonetics}
  \begin{phonetics}{屏}{ping2}
    \definition{s.}{tela | um conjunto de pergaminhos; tiras de tela}
    \definition{v.}{proteger alguém ou algo; resguardar}
  \end{phonetics}
\end{entry}

\begin{entry}{屏幕}{9,13}{⼫、⼱}
  \begin{phonetics}{屏幕}{ping2 mu4}[][HSK 6]
    \definition[个,块]{s.}{tela; a parte dos computadores, televisores, celulares, etc. que exibe texto, imagens, etc.}
  \end{phonetics}
\end{entry}

\begin{entry}{展}{10}{⼫}
  \begin{phonetics}{展}{zhan3}
    \definition*{s.}{Sobrenome Zhan}
    \definition{s.}{exposição}
    \definition{v.}{abrir; espalhar; desdobrar | fazer bom uso; dar liberdade para | adiar; estender; prolongar | expandir | abrir; deixar ir | exibir; mostrar}
  \end{phonetics}
\end{entry}

\begin{entry}{展开}{10,4}{⼫、⼶}
  \begin{phonetics}{展开}{zhan3kai1}[][HSK 3]
    \definition{s.}{desenvolvimento; expansão; explosão; evolução}
    \definition{v.}{espalhar; desdobrar; abrir | lançar; desdobrar; desenvolver; realizar em grande escala | espalhar; desenrolar; amplificar; desenvolver; expandir; explodir; evoluir; alongar}
  \end{phonetics}
\end{entry}

\begin{entry}{展示}{10,5}{⼫、⽰}
  \begin{phonetics}{展示}{zhan3shi4}[][HSK 5]
    \definition{v.}{mostrar; revelar; pôr a nu; abrir diante de alguém; expor claramente; manifestar de forma evidente}
  \end{phonetics}
\end{entry}

\begin{entry}{展现}{10,8}{⼫、⾒}
  \begin{phonetics}{展现}{zhan3xian4}[][HSK 5]
    \definition{v.}{mostrar; surgir; manifestar}
  \end{phonetics}
\end{entry}

\begin{entry}{展览}{10,9}{⼫、⾒}
  \begin{phonetics}{展览}{zhan3lan3}[][HSK 5]
    \definition[个,次,场]{s.}{exposição; exibição; atividades expostas; itens expostos}
    \definition{v.}{mostrar; exibir; expor; expor algo para que as pessoas vejam}
  \end{phonetics}
\end{entry}

\begin{entry}{属}{12}{⼫}
  \begin{phonetics}{属}{shu3}[][HSK 3]
    \definition{s.}{categoria | gênero | membros da família; dependentes; familiares; parentes}
    \definition{v.}{estar sob; subordinado a | pertencer a | nascer no ano de (um dos doze animais do zodíaco)}
  \end{phonetics}
  \begin{phonetics}{属}{zhu3}
    \definition{v.}{juntar; combinar | fixar (a mente) em; centrar (a atenção, etc.) em}
  \end{phonetics}
\end{entry}

\begin{entry}{属于}{12,3}{⼫、⼆}
  \begin{phonetics}{属于}{shu3yu2}[][HSK 3]
    \definition{v.}{pertencer a; fazer parte de; pertencer ou ser propriedade de uma determinada parte}
  \end{phonetics}
\end{entry}

\begin{entry}{屡}{12}{⼫}
  \begin{phonetics}{屡}{lv3}
    \definition{adv.}{uma e outra vez; repetidamente | frequentemente}
  \end{phonetics}
\end{entry}

\begin{entry}{屡次}{12,6}{⼫、⽋}
  \begin{phonetics}{屡次}{lv3ci4}
    \definition{adv.}{repetidamente | uma e outra vez | muitas vezes}
  \end{phonetics}
\end{entry}

%%%%% EOF %%%%%

