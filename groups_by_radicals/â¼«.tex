%%%
%%% Radical "⼫"
%%%

\section*{Radical 44: ``⼫''}\addcontentsline{toc}{section}{Radical 44: ⼫}

\begin{entry}{尺}{4}{⼫}
  \begin{phonetics}{尺}{che3}
    \definition{s.}{(tom) uma nota da escala em gongchepu (工尺谱), correspondente a 2 na notação musical numerada}
  \seealsoref{工尺谱}{gong1 che3 pu3}
  \end{phonetics}
  \begin{phonetics}{尺}{chi3}[][HSK 4]
    \definition{clas.}{chi, uma unidade de comprimento (=13 metros)}
    \definition[把]{s.}{régua; instrumentos de medição | um instrumento no formato de uma régua}
  \end{phonetics}
\end{entry}

\begin{entry}{尺子}{4,3}{⼫、⼦}
  \begin{phonetics}{尺子}{chi3zi5}[][HSK 4]
    \definition[把]{s.}{régua de madeira ou metal para orientar a caneta ou o lápis para desenhar linhas ou fazer medições}
  \end{phonetics}
\end{entry}

\begin{entry}{尺寸}{4,3}{⼫、⼨}
  \begin{phonetics}{尺寸}{chi3 cun4}[][HSK 4]
    \definition{s.}{tamanho; medida; dimensão}
  \end{phonetics}
\end{entry}

\begin{entry}{尽力}{6,2}{⼫、⼒}
  \begin{phonetics}{尽力}{jin4li4}[][HSK 4]
    \definition{v.+compl.}{esforçar-se ao máximo; esforçar-se ao máximo; usar toda a sua força; fazer algo com seu melhor esforço}
  \end{phonetics}
\end{entry}

\begin{entry}{尽快}{6,7}{⼫、⼼}
  \begin{phonetics}{尽快}{jin3kuai4}[][HSK 4]
    \definition{adv.}{com toda a velocidade; o mais rápido possível; o mais breve possível}
  \end{phonetics}
\end{entry}

\begin{entry}{尽量}{6,12}{⼫、⾥}
  \begin{phonetics}{尽量}{jin3liang4}[][HSK 3]
    \definition{adv.}{tanto quanto possível; da melhor maneira possível}
  \end{phonetics}
\end{entry}

\begin{entry}{尽管}{6,14}{⼫、⽵}
  \begin{phonetics}{尽管}{jin3guan3}
    \definition{conj.}{no entanto | embora | apesar de}
  \end{phonetics}
\end{entry}

\begin{entry}{尾巴}{7,4}{⼫、⼰}
  \begin{phonetics}{尾巴}{wei3ba5}[][HSK 4]
    \definition{s.}{cauda; projeções na extremidade do corpo de certos animais | parte semelhante a uma cauda; refere-se, em geral, ao final de algo | apêndice; anexo; adepto servil; pessoa que segue ou concorda com outra pessoa | (figura de linguagem) alguém que faz sombra a outro | fim; remanescente; parte restante (ou inacabada)}
  \end{phonetics}
\end{entry}

\begin{entry}{尿}{7}{⼫}
  \begin{phonetics}{尿}{niao4}
    \definition[泡]{s.}{urina}
    \definition{v.}{urinar}
  \end{phonetics}
  \begin{phonetics}{尿}{sui1}
    \definition{s.}{(coloquial) urina}
  \end{phonetics}
\end{entry}

\begin{entry}{局}{7}{⼫}
  \begin{phonetics}{局}{ju2}[][HSK 4]
    \definition{s.}{tabuleiro de xadrez | jogo; turno; \emph{set} | situação; estado das coisas | tolerância; grandeza ou pequenez da mente; grau de tolerância de uma pessoa em relação às outras | reunião de pessoas em festas | ardil; artidício; estratagema; armadilha | parte; porção; parcela | nome de determinadas lojas}
  \end{phonetics}
\end{entry}

\begin{entry}{屁股}{7,8}{⼫、⾁}
  \begin{phonetics}{屁股}{pi4gu5}
    \definition{s.}{nádega | quadris}
  \end{phonetics}
\end{entry}

\begin{entry}{屁话}{7,8}{⼫、⾔}
  \begin{phonetics}{屁话}{pi4hua4}
    \definition{s.}{absurdo | tolice | besteira}
  \end{phonetics}
\end{entry}

\begin{entry}{层}{7}{⼫}
  \begin{phonetics}{层}{ceng2}[][HSK 2]
    \definition{clas.}{para andar, piso}
  \end{phonetics}
\end{entry}

\begin{entry}{层次}{7,6}{⼫、⽋}
  \begin{phonetics}{层次}{ceng2ci4}
    \definition{s.}{camada | nível | graduação | arranjo de ideias}
  \end{phonetics}
\end{entry}

\begin{entry}{层层}{7,7}{⼫、⼫}
  \begin{phonetics}{层层}{ceng2ceng2}
    \definition{s.}{camada sobre camada}
  \end{phonetics}
\end{entry}

\begin{entry}{居民}{8,5}{⼫、⽒}
  \begin{phonetics}{居民}{ju1min2}[][HSK 4]
    \definition[个,户,位]{s.}{residente; habitante; pessoas que estão fixas em um único lugar}
  \end{phonetics}
\end{entry}

\begin{entry}{居住}{8,7}{⼫、⼈}
  \begin{phonetics}{居住}{ju1zhu4}[][HSK 4]
    \definition{v.}{viver; residir; morar; habitar}
  \end{phonetics}
\end{entry}

\begin{entry}{居然}{8,12}{⼫、⽕}
  \begin{phonetics}{居然}{ju1ran2}
    \definition{adv.}{inesperadamente | na verdade | para surpresa de alguém}
  \end{phonetics}
\end{entry}

\begin{entry}{屋子}{9,3}{⼫、⼦}
  \begin{phonetics}{屋子}{wu1zi5}[][HSK 3]
    \definition[间,座,栋]{s.}{casa}
  \end{phonetics}
\end{entry}

\begin{entry}{屌丝}{9,5}{⼫、⼀}
  \begin{phonetics}{屌丝}{diao3si1}
    \definition{adj.}{panaca | zé-ninguém | (gíria de \emph{Internet}) \emph{looser}}
  \end{phonetics}
\end{entry}

\begin{entry}{屎}{9}{⼫}
  \begin{phonetics}{屎}{shi3}
    \definition{s.}{fezes | excrementos | (forma ligada) secreção (do ouvido, olho, etc.)}
  \end{phonetics}
\end{entry}

\begin{entry}{展开}{10,4}{⼫、⼶}
  \begin{phonetics}{展开}{zhan3kai1}[][HSK 3]
    \definition{s.}{desenvolvimento; expansão; explosão; evolução}
    \definition{v.}{desenvolver; espalhar; desdobrar; abrir; desenrolar; amplificar; esticar; ventilar | lançar; desdobrar; desenvolver; executar}
  \end{phonetics}
\end{entry}

\begin{entry}{展示}{10,5}{⼫、⽰}
  \begin{phonetics}{展示}{zhan3shi4}
    \definition{v.}{revelar | mostrar | exibir}
  \end{phonetics}
\end{entry}

\begin{entry}{属}{12}{⼫}
  \begin{phonetics}{属}{shu3}[][HSK 3]
    \definition{s.}{categoria
gênero
membros da família; dependentes}
    \definition{v.}{estar sob; subordinado a | pertencer a | nascer no ano de (um dos doze animais do zodíaco)}
  \end{phonetics}
  \begin{phonetics}{属}{zhu3}
    \definition{v.}{juntar; combinar | fixar (a mente) em; centrar (a atenção, etc.) em}
  \end{phonetics}
\end{entry}

\begin{entry}{属于}{12,3}{⼫、⼆}
  \begin{phonetics}{属于}{shu3yu2}[][HSK 3]
    \definition{v.}{pertencer a; fazer parte de; ser classificado como}
  \end{phonetics}
\end{entry}

\begin{entry}{屡次}{12,6}{⼫、⽋}
  \begin{phonetics}{屡次}{lv3ci4}
    \definition{adv.}{repetidamente | uma e outra vez | muitas vezes}
  \end{phonetics}
\end{entry}

%%%%% EOF %%%%%

