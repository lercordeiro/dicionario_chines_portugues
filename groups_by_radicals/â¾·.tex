%%%
%%% Radical "⾷"
%%%

\section*{Radical 184: ``⾷'' (饣、飠)}\addcontentsline{toc}{section}{Radical 184: ⾷、饣、飠}

\begin{entry}{饭}{7}{⾷}
  \begin{phonetics}{饭}{fan4}[][HSK 1]
    \definition{s.}{(empréstimo linguístico) fã, devoto}
    \definition[顿,份,碗,口,锅]{s.}{cereais cozidos; grãos cozidos | refeição; alimentos consumidos diariamente em horários regulares | trabalho; meio de subsistência; meio de vida}
  \end{phonetics}
\end{entry}

\begin{entry}{饭店}{7,8}{⾷、⼴}
  \begin{phonetics}{饭店}{fan4dian4}[][HSK 1]
    \definition[家,个]{s.}{restaurante | hotel; hotel grande e bem equipado}
  \end{phonetics}
\end{entry}

\begin{entry}{饭馆}{7,11}{⾷、⾷}
  \begin{phonetics}{饭馆}{fan4 guan3}[][HSK 2]
    \definition[家,个]{s.}{restaurante; lanchonete}
  \end{phonetics}
\end{entry}

\begin{entry}{饮食}{7,9}{⾷、⾷}
  \begin{phonetics}{饮食}{yin3shi2}[][HSK 5]
    \definition{s.}{dieta; alimentos e bebidas}
    \definition{v.}{comer; beber}
  \end{phonetics}
\end{entry}

\begin{entry}{饮料}{7,10}{⾷、⽃}
  \begin{phonetics}{饮料}{yin3liao4}[][HSK 5]
    \definition[杯,瓶,种]{s.}{bebida; drinque; líquidos processados e fabricados para consumo, como vinho, chá, refrigerantes, suco de laranja, etc.}
  \end{phonetics}
\end{entry}

\begin{entry}{饱}{8}{⾷}
  \begin{phonetics}{饱}{bao3}[][HSK 2]
    \definition{adj.}{cheio; comer até ficar satisfeito | cheio; rechonchudo}
    \definition{adv.}{totalmente; completamente; plenamente}
    \definition{v.}{satisfazer}
  \end{phonetics}
\end{entry}

\begin{entry}{食}{9}{⾷}
  \begin{phonetics}{食}{shi2}
    \definition{adj.}{para cozinhar; comestível}
    \definition{s.}{refeição; comida; o que as pessoas e os animais comem | alimentação; alimento para animais; ração | eclipse solar; eclipse lunar}
    \definition{v.}{comer}
  \end{phonetics}
  \begin{phonetics}{食}{si4}
    \definition{v.}{alimentar; dar comida a}
  \end{phonetics}
\end{entry}

\begin{entry}{食物}{9,8}{⾷、⽜}
  \begin{phonetics}{食物}{shi2wu4}[][HSK 2]
    \definition[种]{s.}{comida; alimentos; comestíveis}
  \end{phonetics}
\end{entry}

\begin{entry}{食品}{9,9}{⾷、⼝}
  \begin{phonetics}{食品}{shi2 pin3}[][HSK 3]
    \definition[种]{s.}{comida; gêneros alimentícios; provisões; alimentos vendidos em lojas que passaram por algum processamento}
  \end{phonetics}
\end{entry}

\begin{entry}{食堂}{9,11}{⾷、⼟}
  \begin{phonetics}{食堂}{shi2 tang2}[][HSK 4]
    \definition[个,间]{s.}{cantina; refeitório}
  \end{phonetics}
\end{entry}

\begin{entry}{饺}{9}{⾷}
  \begin{phonetics}{饺}{jiao3}
    \definition[盘,碗,顿,个]{s.}{bolinho de massa; \emph{dumpling}}
  \end{phonetics}
\end{entry}

\begin{entry}{饺子}{9,3}{⾷、⼦}
  \begin{phonetics}{饺子}{jiao3zi5}[][HSK 2]
    \definition[个,盘,碗,锅]{s.}{jiaozi; bolinho chinês; bolinho de massa}
  \end{phonetics}
\end{entry}

\begin{entry}{饼}{9}{⾷}
  \begin{phonetics}{饼}{bing3}[][HSK 5]
    \definition[张]{s.}{um bolo redondo e plano; massa assada ou cozida no vapor | algo que tem o formato de um bolo; semelhante a uma torta}
  \end{phonetics}
\end{entry}

\begin{entry}{饼干}{9,3}{⾷、⼲}
  \begin{phonetics}{饼干}{bing3gan1}[][HSK 5]
    \definition[块,片,包,盒,袋]{s.}{biscoito; bolacha; \emph{cookie}; alimentos, pedaços pequenos e finos cozidos em farinha com açúcar, ovos, leite, etc.}
  \end{phonetics}
\end{entry}

\begin{entry}{饿}{10}{⾷}
  \begin{phonetics}{饿}{e4}[][HSK 1]
    \definition{adj.}{faminto}
    \definition{v.}{passar fome; causar fome}
  \end{phonetics}
\end{entry}

\begin{entry}{馒}{14}{⾷}
  \begin{phonetics}{馒}{man2}
    \definition{s.}{pão cozido no vapor}
  \end{phonetics}
\end{entry}

\begin{entry}{馒头}{14,5}{⾷、⼤}
  \begin{phonetics}{馒头}{man2tou5}
    \definition{s.}{pão cozido no vapor}
  \end{phonetics}
\end{entry}

\begin{entry}{餐}{16}{⾷}
  \begin{phonetics}{餐}{can1}
    \definition{clas.}{comer; fazer uma refeição}
    \definition{clas.}{usado para refeições}
    \definition{s.}{comida; refeição}
  \end{phonetics}
\end{entry}

\begin{entry}{餐厅}{16,4}{⾷、⼚}
  \begin{phonetics}{餐厅}{can1ting1}[][HSK 5]
    \definition[家]{s.}{restaurante; refeitório em um hotel | cantina, refeitório; também é chamado de 食堂}
    \definition[间]{s.}{sala de jantar}
  \seealsoref{食堂}{shi2 tang2}
  \end{phonetics}
\end{entry}

\begin{entry}{餐饮}{16,7}{⾷、⾷}
  \begin{phonetics}{餐饮}{can1 yin3}[][HSK 5]
    \definition{s.}{comidas e bebidas; refere-se a atividades de bufê em restaurantes e hotéis}
  \end{phonetics}
\end{entry}

\begin{entry}{餐馆}{16,11}{⾷、⾷}
  \begin{phonetics}{餐馆}{can1 guan3}[][HSK 5]
    \definition[家]{s.}{restaurante;}
  \end{phonetics}
\end{entry}

%%%%% EOF %%%%%

