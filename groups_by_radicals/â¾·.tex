%%%
%%% Radical "⾷"
%%%

\section*{Radical 184: ``⾷'' (饣、飠)}\addcontentsline{toc}{section}{Radical 184: ⾷、饣、飠}

\begin{Entry}{饭}{7}{⾷}
  \begin{Phonetics}{饭}{fan4}[][HSK 1]
    \definition{s.}{(empréstimo linguístico) fã, devoto}
    \definition[顿,份,碗,口,锅]{s.}{cereais cozidos; grãos cozidos | refeição; alimentos consumidos diariamente em horários regulares | trabalho; meio de subsistência; meio de vida}
  \end{Phonetics}
\end{Entry}

\begin{Entry}{饭店}{7,8}{⾷、⼴}
  \begin{Phonetics}{饭店}{fan4dian4}[][HSK 1]
    \definition[家,个]{s.}{restaurante | hotel; hotel grande e bem equipado}
  \end{Phonetics}
\end{Entry}

\begin{Entry}{饭馆}{7,11}{⾷、⾷}
  \begin{Phonetics}{饭馆}{fan4 guan3}[][HSK 2]
    \definition[家,个]{s.}{restaurante; lanchonete}
  \end{Phonetics}
\end{Entry}

\begin{Entry}{饭碗}{7,13}{⾷、⽯}
  \begin{Phonetics}{饭碗}{fan4wan3}[][HSK 7-9]
    \definition[个,只]{s.}{tigela de arroz | Coloquial: emprego; meio de subsistência | Figurativo: meio de subsistência; maneira de ganhar a vida}
  \end{Phonetics}
\end{Entry}

\begin{Entry}{饮}{7}{⾷}
  \begin{Phonetics}{饮}{yin3}
    \definition{s.}{bebidas; \emph{drinks}; algo para beber | uma decocção da medicina chinesa para ser tomada fria | fluido retido}
    \definition{v.}{beber | cuidar; engolir a pílula amarga}
  \end{Phonetics}
  \begin{Phonetics}{饮}{yin4}
    \definition{v.}{dar (aos animais) água para beber}
  \end{Phonetics}
\end{Entry}

\begin{Entry}{饮食}{7,9}{⾷、⾷}
  \begin{Phonetics}{饮食}{yin3shi2}[][HSK 5]
    \definition{s.}{dieta; alimentos e bebidas}
    \definition{v.}{comer; beber}
  \end{Phonetics}
\end{Entry}

\begin{Entry}{饮料}{7,10}{⾷、⽃}
  \begin{Phonetics}{饮料}{yin3liao4}[][HSK 5]
    \definition[杯,瓶,种]{s.}{bebida; drinque; líquidos processados e fabricados para consumo, como vinho, chá, refrigerantes, suco de laranja, etc.}
  \end{Phonetics}
\end{Entry}

\begin{Entry}{饱}{8}{⾷}
  \begin{Phonetics}{饱}{bao3}[][HSK 2]
    \definition{adj.}{cheio; comer até ficar satisfeito | cheio; rechonchudo}
    \definition{adv.}{totalmente; completamente; plenamente}
    \definition{v.}{satisfazer}
  \end{Phonetics}
\end{Entry}

\begin{Entry}{饱和}{8,8}{⾷、⼝}
  \begin{Phonetics}{饱和}{bao3he2}[][HSK 7-9]
    \definition{v.}{estar saturado; a uma certa temperatura ou pressão, a quantidade de soluto contida na solução atinge seu limite máximo e não consegue mais se dissolver | estar saturado; metaforicamente, a quantidade de algo atinge um máximo dentro de um certo intervalo}
  \end{Phonetics}
\end{Entry}

\begin{Entry}{饱满}{8,13}{⾷、⽔}
  \begin{Phonetics}{饱满}{bao3man3}[][HSK 7-9]
    \definition{adj.}{cheio; rechonchudo; bem empilhado; preenchido | robusto; abundante; pleno; vigoroso}
  \end{Phonetics}
\end{Entry}

\begin{Entry}{食}{9}{⾷}[Kangxi 184]
  \begin{Phonetics}{食}{shi2}
    \definition{adj.}{para cozinhar; comestível}
    \definition{s.}{refeição; comida; o que as pessoas e os animais comem | alimentação; alimento para animais; ração | eclipse solar; eclipse lunar}
    \definition{v.}{comer}
  \end{Phonetics}
  \begin{Phonetics}{食}{si4}
    \definition{v.}{alimentar; dar comida a}
  \end{Phonetics}
\end{Entry}

\begin{Entry}{食物}{9,8}{⾷、⽜}
  \begin{Phonetics}{食物}{shi2wu4}[][HSK 2]
    \definition[种]{s.}{comida; alimentos; comestíveis}
  \end{Phonetics}
\end{Entry}

\begin{Entry}{食品}{9,9}{⾷、⼝}
  \begin{Phonetics}{食品}{shi2 pin3}[][HSK 3]
    \definition[种]{s.}{comida; gêneros alimentícios; provisões; alimentos vendidos em lojas que passaram por algum processamento}
  \end{Phonetics}
\end{Entry}

\begin{Entry}{食堂}{9,11}{⾷、⼟}
  \begin{Phonetics}{食堂}{shi2 tang2}[][HSK 4]
    \definition[个,间]{s.}{cantina; refeitório}
  \end{Phonetics}
\end{Entry}

\begin{Entry}{食欲}{9,11}{⾷、⽋}
  \begin{Phonetics}{食欲}{shi2 yu4}[][HSK 6]
    \definition{adj.}{apetitoso}
    \definition{s.}{apetite; desejo humano de comer}
  \end{Phonetics}
\end{Entry}

\begin{Entry}{饺}{9}{⾷}
  \begin{Phonetics}{饺}{jiao3}
    \definition[盘,碗,顿,个]{s.}{bolinho de massa; \emph{dumpling}}
  \end{Phonetics}
\end{Entry}

\begin{Entry}{饺子}{9,3}{⾷、⼦}
  \begin{Phonetics}{饺子}{jiao3zi5}[][HSK 2]
    \definition[个,盘,碗,锅]{s.}{jiaozi; bolinho chinês; bolinho de massa}
  \end{Phonetics}
\end{Entry}

\begin{Entry}{饼}{9}{⾷}
  \begin{Phonetics}{饼}{bing3}[][HSK 5]
    \definition[张]{s.}{um bolo redondo e plano; massa assada ou cozida no vapor | algo que tem o formato de um bolo; semelhante a uma torta}
  \end{Phonetics}
\end{Entry}

\begin{Entry}{饼干}{9,3}{⾷、⼲}
  \begin{Phonetics}{饼干}{bing3gan1}[][HSK 5]
    \definition[块,片,包,盒,袋]{s.}{biscoito; bolacha; \emph{cookie}; alimentos, pedaços pequenos e finos cozidos em farinha com açúcar, ovos, leite, etc.}
  \end{Phonetics}
\end{Entry}

\begin{Entry}{饿}{10}{⾷}
  \begin{Phonetics}{饿}{e4}[][HSK 1]
    \definition{adj.}{faminto}
    \definition{v.}{passar fome; causar fome}
  \end{Phonetics}
\end{Entry}

\begin{Entry}{馋}{12}{⾷}
  \begin{Phonetics}{馋}{chan2}[][HSK 7-9]
    \definition{adj.}{cobiçoso; invejoso | guloso; comilão; glutão}
    \definition{v.}{desejar comida; querer comer (alguma coisa)}
  \end{Phonetics}
\end{Entry}

\begin{Entry}{馒}{14}{⾷}
  \begin{Phonetics}{馒}{man2}
    \definition{s.}{pão cozido no vapor}
  \end{Phonetics}
\end{Entry}

\begin{Entry}{馒头}{14,5}{⾷、⼤}
  \begin{Phonetics}{馒头}{man2tou5}[][HSK 6]
    \definition[个,锅,屉,筐]{s.}{pão cozido no vapor; um alimento cozido no vapor feito de farinha fermentada, geralmente redondo na parte superior e plano na parte inferior, sem recheio}
  \end{Phonetics}
\end{Entry}

\begin{Entry}{餐}{16}{⾷}
  \begin{Phonetics}{餐}{can1}[][HSK 6]
    \definition{clas.}{comer; fazer uma refeição}
    \definition{clas.}{usado para refeições}
    \definition{s.}{comida; refeição}
  \end{Phonetics}
\end{Entry}

\begin{Entry}{餐厅}{16,4}{⾷、⼚}
  \begin{Phonetics}{餐厅}{can1ting1}[][HSK 5]
    \definition[间]{s.}{sala de jantar}
    \definition[间,家,个]{s.}{restaurante; refeitório em um hotel | cantina, refeitório; também é chamado de 食堂}
  \seealsoref{食堂}{shi2 tang2}
  \end{Phonetics}
\end{Entry}

\begin{Entry}{餐饮}{16,7}{⾷、⾷}
  \begin{Phonetics}{餐饮}{can1 yin3}[][HSK 5]
    \definition[个]{s.}{comidas e bebidas; refere-se a atividades de bufê em restaurantes e hotéis}
  \end{Phonetics}
\end{Entry}

\begin{Entry}{餐桌}{16,10}{⾷、⽊}
  \begin{Phonetics}{餐桌}{can1zhuo1}[][HSK 7-9]
    \definition[张]{s.}{mesa de jantar}
  \end{Phonetics}
\end{Entry}

\begin{Entry}{餐馆}{16,11}{⾷、⾷}
  \begin{Phonetics}{餐馆}{can1 guan3}[][HSK 5]
    \definition[家,个]{s.}{restaurante}
  \end{Phonetics}
\end{Entry}

%%%%% EOF %%%%%

