%%%
%%% Radical "⼀"
%%%

\section*{Radical 1: ``⼀''}\addcontentsline{toc}{section}{Radical 1: ⼀}

\begin{entry}{一}{1}{⼀}[Kangxi 1]
  \begin{phonetics}{一}{yi1}[(quando usado sozinho)][HSK 1]
    \definition{num.}{um; 1 | pronunciado como \dpy{yao1} quando dito número a número}
  \end{phonetics}
  \begin{phonetics}{一}{yi2}[(antes de quarto tom)][HSK 1]
    \definition{num.}{um; 1 | um (artigo)}
  \end{phonetics}
  \begin{phonetics}{一}{yi4}[][HSK 1]
    \definition{adv.}{uma vez | assim que | ao}
    \definition{num.}{um; 1 | um (artigo)}
  \end{phonetics}
\end{entry}

\begin{entry}{一下}{1,3}{⼀、⼀}
  \begin{phonetics}{一下}{yi2xia4}
    \definition{adv.}{em um curto tempo | rapidamente}
  \end{phonetics}
\end{entry}

\begin{entry}{一下儿}{1,3,2}{⼀、⼀、⼉}
  \begin{phonetics}{一下儿}{yi2xia4r5}[][HSK 1]
    \definition{adv.}{um pouco}
  \end{phonetics}
\end{entry}

\begin{entry}{一个样}{1,3,10}{⼀、⼈、⽊}
  \begin{phonetics}{一个样}{yi2ge5yang4}
    \definition{adj.}{igual | mesmo}
    \seeref{一样}{yi2yang4}
  \end{phonetics}
\end{entry}

\begin{entry}{一切}{1,4}{⼀、⼑}
  \begin{phonetics}{一切}{yi2qie4}[][HSK 3]
    \definition{pron.}{tudo; todo; todas as coisas}
  \end{phonetics}
\end{entry}

\begin{entry}{一方面}{1,4,9}{⼀、⽅、⾯}
  \begin{phonetics}{一方面}{yi4 fang1 mian4}[][HSK 3]
    \definition{s.}{um lado; um dos dois aspectos opostos ou um lado de algo que está relacionado a outro}
  \end{phonetics}
\end{entry}

\begin{entry}{一方面……,一方面……}{1,4,9,1,4,9}{⼀、⽅、⾯、⼀、⽅、⾯}
  \begin{phonetics}{一方面……,一方面……}{yi4 fang1 mian4 yi4 fang1 mian4}[][HSK 3]
    \definition{conj.}{por um lado\dots, por outro lado\dots; conecta duas orações paralelas (devem ser usadas juntas)}
  \end{phonetics}
\end{entry}

\begin{entry}{一半}{1,5}{⼀、⼗}
  \begin{phonetics}{一半}{yi2ban4}[][HSK 1]
    \definition{num.}{meio | metade}
  \end{phonetics}
\end{entry}

\begin{entry}{一生}{1,5}{⼀、⽣}
  \begin{phonetics}{一生}{yi4 sheng1}[][HSK 2]
    \definition{s.}{toda a vida | ao longo da vida | a vida de alguém}
  \end{phonetics}
\end{entry}

\begin{entry}{一边}{1,5}{⼀、⾡}
  \begin{phonetics}{一边}{yi4bian1}[][HSK 1]
    \definition{adj.}{mesmo | igual}
    \definition{adv.}{enquanto | como | ao mesmo tempo | simultaneamente}
    \definition{s.}{lado | um lado | cada lado | ao lado de}
  \end{phonetics}
\end{entry}

\begin{entry}{一会儿}{1,6,2}{⼀、⼈、⼉}
  \begin{phonetics}{一会儿}{yi2 hui4r5}[][HSK 1,2]
    \definition{adv.}{daqui a pouco tempo | pouco tempo}
  \end{phonetics}
\end{entry}

\begin{entry}{一共}{1,6}{⼀、⼋}
  \begin{phonetics}{一共}{yi2gong4}[][HSK 2]
    \definition{adv.}{completamente | no total | no todo | em suma}
  \end{phonetics}
\end{entry}

\begin{entry}{一再}{1,6}{⼀、⼌}
  \begin{phonetics}{一再}{yi2zai4}[][HSK 4]
    \definition{adv.}{repetidamente; de novo e de novo; repetidas vezes}
  \end{phonetics}
\end{entry}

\begin{entry}{一同}{1,6}{⼀、⼝}
  \begin{phonetics}{一同}{yi4tong2}
    \definition{adv.}{juntos, ao mesmo tempo}
  \end{phonetics}
\end{entry}

\begin{entry}{一行}{1,6}{⼀、⾏}
  \begin{phonetics}{一行}{yi1xing2}
    \definition{s.}{festa | delegação}
  \end{phonetics}
\end{entry}

\begin{entry}{一齐}{1,6}{⼀、⿑}
  \begin{phonetics}{一齐}{yi4qi2}
    \definition{adv.}{tudo ao mesmo tempo | em uníssono | junto}
  \end{phonetics}
\end{entry}

\begin{entry}{一块}{1,7}{⼀、⼟}
  \begin{phonetics}{一块}{yi2kuai4}
    \definition{adv.}{(principalmente mandarim) juntos}
  \end{phonetics}
\end{entry}

\begin{entry}{一块儿}{1,7,2}{⼀、⼟、⼉}
  \begin{phonetics}{一块儿}{yi2kuai4r5}[][HSK 1]
    \definition{adv.}{juntos}
  \end{phonetics}
\end{entry}

\begin{entry}{一时}{1,7}{⼀、⽇}
  \begin{phonetics}{一时}{yi4shi2}
    \definition{adv.}{por pouco tempo | por um tempo | temporariamente | momentaneamente | uma vez | de tempos em tempos | ocasionalmente}
  \end{phonetics}
\end{entry}

\begin{entry}{一些}{1,8}{⼀、⼆}
  \begin{phonetics}{一些}{yi4xie1}[][HSK 1]
    \definition{pron.}{uns | alguns}
  \end{phonetics}
\end{entry}

\begin{entry}{一定}{1,8}{⼀、⼧}
  \begin{phonetics}{一定}{yi2ding4}[][HSK 2]
    \definition{adv.}{certamente | definitivamente}
  \end{phonetics}
\end{entry}

\begin{entry}{一直}{1,8}{⼀、⽬}
  \begin{phonetics}{一直}{yi4zhi2}[][HSK 2]
    \definition{adv.}{diretamente | sempre em frente | o tempo todo | sempre | constantemente}
  \end{phonetics}
\end{entry}

\begin{entry}{一律}{1,9}{⼀、⼻}
  \begin{phonetics}{一律}{yi2lv4}[][HSK 4]
    \definition{adj.}{igual; semelhante; uniforme; parecido; idêntico}
    \definition{adv.}{todos; tudo; sem exceção; enfatiza que todos devem ser assim, sem exceção, e é usado principalmente em regulamentos ou requisitos}
  \end{phonetics}
\end{entry}

\begin{entry}{一战}{1,9}{⼀、⼽}
  \begin{phonetics}{一战}{yi2zhan4}
    \definition*{s.}{Primeira Guerra Mundial}
  \end{phonetics}
\end{entry}

\begin{entry}{一点儿}{1,9,2}{⼀、⽕、⼉}
  \begin{phonetics}{一点儿}{yi4dian3r5}[][HSK 1]
    \definition{adv.}{um pouco (``{adj.}+一点儿'' ou ``一点儿+{s.}'') | um ponto}
  \end{phonetics}
\end{entry}

\begin{entry}{一点点}{1,9,9}{⼀、⽕、⽕}
  \begin{phonetics}{一点点}{yi4 dian3 dian3}[][HSK 2]
    \definition{adj.}{um pouco}
  \end{phonetics}
\end{entry}

\begin{entry}{一样}{1,10}{⼀、⽊}
  \begin{phonetics}{一样}{yi2yang4}[][HSK 1]
    \definition{adj.}{igual | mesmo}
  \end{phonetics}
\end{entry}

\begin{entry}{一致}{1,10}{⼀、⾄}
  \begin{phonetics}{一致}{yi2zhi4}[][HSK 4]
    \definition{adj.}{equado; idêntico; uniforme; unânime; nenhuma diferença (de opinião ou ação)}
    \definition{adv.}{juntos; em conjunto}
  \end{phonetics}
\end{entry}

\begin{entry}{一般}{1,10}{⼀、⾈}
  \begin{phonetics}{一般}{yi4ban1}[][HSK 2]
    \definition{adj.}{geral | comum | normal}
    \definition{adv.}{normalmente}
  \end{phonetics}
\end{entry}

\begin{entry}{一般来说}{1,10,7,9}{⼀、⾈、⽊、⾔}
  \begin{phonetics}{一般来说}{yi4 ban1 lai2 shuo1}[][HSK 4]
    \definition{expr.}{de modo geral; na média; no caso usual; a declaração usual}
  \end{phonetics}
\end{entry}

\begin{entry}{一起}{1,10}{⼀、⾛}
  \begin{phonetics}{一起}{yi4qi3}[][HSK 1]
    \definition{adv.}{juntamente | em conjunto | no mesmo lugar | completamente | em todos}
  \end{phonetics}
\end{entry}

\begin{entry}{一部分}{1,10,4}{⼀、⾢、⼑}
  \begin{phonetics}{一部分}{yi2 bu4 fen4}[][HSK 2]
    \definition{adv.}{parcialmente}
    \definition{num.}{parte | porção | seção | fração}
    \definition[把]{s.}{parcial}
  \end{phonetics}
\end{entry}

\begin{entry}{一……就……}{1,12}{⼀、⼪}
  \begin{phonetics}{一……就……}{yi1 jiu4}
    \definition{expr.}{logo que |  uma vez que}
  \end{phonetics}
\end{entry}

\begin{entry}{一道}{1,12}{⼀、⾡}
  \begin{phonetics}{一道}{yi2dao4}
    \definition{adv.}{juntos | ao lado}
  \end{phonetics}
\end{entry}

\begin{entry}{一路平安}{1,13,5,6}{⼀、⾜、⼲、⼧}
  \begin{phonetics}{一路平安}{yi2 lu4 ping2 an1}[][HSK 2]
    \definition{expr.}{Boa viagem!}
    \definition{v.}{ter uma viagem agradável}
  \end{phonetics}
\end{entry}

\begin{entry}{一路顺风}{1,13,9,4}{⼀、⾜、⾴、⾵}
  \begin{phonetics}{一路顺风}{yi2 lu4 shun4 feng1}[][HSK 2]
    \definition{expr.}{ter uma viagem agradável}
  \end{phonetics}
\end{entry}

\begin{entry}{七}{2}{⼀}
  \begin{phonetics}{七}{qi1}[][HSK 1]
    \definition{num.}{sete; 7}
  \end{phonetics}
\end{entry}

\begin{entry}{七夕}{2,3}{⼀、⼣}
  \begin{phonetics}{七夕}{qi1xi1}
    \definition*{s.}{Dia dos Namorados Chinês, quando o vaqueiro e a tecelã (牛郎织女) têm permissão para se reunirem anualmente | Festival das Meninas | Festival Duplo Sete, noite do sétimo mês lunar}
    \seeref{牛郎织女}{niu2lang2zhi1nv3}
  \end{phonetics}
\end{entry}

\begin{entry}{万}{3}{⼀}
  \begin{phonetics}{万}{wan4}[][HSK 2]
    \definition*{s.}{sobrenome Wan}
    \definition{adj.}{um grande número}
    \definition{num.}{dez mil; 10.000; 1.0000}
  \end{phonetics}
\end{entry}

\begin{entry}{万一}{3,1}{⼀、⼀}
  \begin{phonetics}{万一}{wan4yi1}[][HSK 4]
    \definition{conj.}{por via das dúvidas; se por acaso; só por precaução; expressa uma suposição muito improvável (usado para coisas desagradáveis)}
    \definition{num.}{um décimo milionésimo; uma porcentagem muito pequena}
    \definition{s.}{contingência; eventualidade; contingências muito improváveis}
  \end{phonetics}
\end{entry}

\begin{entry}{万万}{3,3}{⼀、⼀}
  \begin{phonetics}{万万}{wan4wan4}
    \definition{adv.}{absolutamente | totalmente}
  \end{phonetics}
\end{entry}

\begin{entry}{万圣节}{3,5,5}{⼀、⼟、⾋}
  \begin{phonetics}{万圣节}{wan4sheng4jie2}
    \definition*{s.}{Dia de Todos os Santos}
  \seealsoref{万圣节前夕}{wan4sheng4jie2qian2xi1}
  \end{phonetics}
\end{entry}

\begin{entry}{万圣节前夕}{3,5,5,9,3}{⼀、⼟、⾋、⼑、⼣}
  \begin{phonetics}{万圣节前夕}{wan4sheng4jie2qian2xi1}
    \definition*{s.}{Véspera do Dia de Todos os Santos | \emph{Halloween}}
  \seealsoref{万圣节}{wan4sheng4jie2}
  \end{phonetics}
\end{entry}

\begin{entry}{丈夫}{3,4}{⼀、⼤}
  \begin{phonetics}{丈夫}{zhang4fu5}[][HSK 4]
    \definition[个,位,名]{s.}{marido; esposo}
  \end{phonetics}
\end{entry}

\begin{entry}{三}{3}{⼀}
  \begin{phonetics}{三}{san1}[][HSK 1]
    \definition*{s.}{sobrenome San}
    \definition{num.}{três; 3}
  \end{phonetics}
\end{entry}

\begin{entry}{三角}{3,7}{⼀、⾓}
  \begin{phonetics}{三角}{san1jiao3}
    \definition{s.}{triângulo}
  \end{phonetics}
\end{entry}

\begin{entry}{三角恋爱}{3,7,10,10}{⼀、⾓、⼼、⽖}
  \begin{phonetics}{三角恋爱}{san1jiao3lian4'ai4}
    \definition{s.}{triângulo amoroso}
  \end{phonetics}
\end{entry}

\begin{entry}{三明治}{3,8,8}{⼀、⽇、⽔}
  \begin{phonetics}{三明治}{san1ming2zhi4}
    \definition{s.}{(empréstimo linguístico) sanduíche}
  \end{phonetics}
\end{entry}

\begin{entry}{三轮车}{3,8,4}{⼀、⾞、⾞}
  \begin{phonetics}{三轮车}{san1lun2che1}
    \definition{s.}{triciclo}
  \end{phonetics}
\end{entry}

\begin{entry}{上}{3}{⼀}
  \begin{phonetics}{上}{shang4}[][HSK 1]
    \definition{adv.}{acima | em cima | sobre}
    \definition{v.}{subir | entrar em | frequentar (aula ou universidade)}
  \end{phonetics}
\end{entry}

\begin{entry}{上个月}{3,3,4}{⼀、⼈、⽉}
  \begin{phonetics}{上个月}{shang4 ge4 yue4}[][HSK 4]
    \definition{s.}{mês passado; refere-se à hora de um mês atrás, ou seja, o mês passado mais próximo da hora atual}
  \end{phonetics}
\end{entry}

\begin{entry}{上门}{3,3}{⼀、⾨}
  \begin{phonetics}{上门}{shang4 men2}[][HSK 4]
    \definition{v.}{chamar; visitar; aparecer; ir ou vir para ver alguém; ir até a porta; ir até a casa de alguém | trancar a porta; fechar a porta durante a noite | casar-se e morar com a família da noiva}
  \end{phonetics}
\end{entry}

\begin{entry}{上升}{3,4}{⼀、⼗}
  \begin{phonetics}{上升}{shang4 sheng1}[][HSK 3]
    \definition{v.}{elevar; subir; mover-se para cima}
  \end{phonetics}
\end{entry}

\begin{entry}{上午}{3,4}{⼀、⼗}
  \begin{phonetics}{上午}{shang4wu3}[][HSK 1]
    \definition{adv.}{manhã | de manhã | período antes do meio-dia}
  \end{phonetics}
\end{entry}

\begin{entry}{上车}{3,4}{⼀、⾞}
  \begin{phonetics}{上车}{shang4 che1}[][HSK 1]
    \definition{v.}{entrar (em ônibus, trem, carro, etc.)}
  \end{phonetics}
\end{entry}

\begin{entry}{上去}{3,5}{⼀、⼛}
  \begin{phonetics}{上去}{shang4 qu4}[][HSK 3]
    \definition{v.}{subir (a partir da minha localização) | ascender a um lugar (ou estado) considerado mais elevado (ou acima)}
  \end{phonetics}
\end{entry}

\begin{entry}{上古}{3,5}{⼀、⼝}
  \begin{phonetics}{上古}{shang4gu3}
    \definition{s.}{o passado distante | tempos antigos | antiguidade}
  \end{phonetics}
\end{entry}

\begin{entry}{上边}{3,5}{⼀、⾡}
  \begin{phonetics}{上边}{shang4bian5}[][HSK 1]
    \definition{adv.}{acima de | parte de cima | por cima}
  \end{phonetics}
\end{entry}

\begin{entry}{上当}{3,6}{⼀、⼹}
  \begin{phonetics}{上当}{shang4dang4}
    \definition{v.+compl.}{ser enganado | morder uma isca | ser manipulado | ser joguete nas mãos de alguém}
  \end{phonetics}
\end{entry}

\begin{entry}{上次}{3,6}{⼀、⽋}
  \begin{phonetics}{上次}{shang4 ci4}[][HSK 1]
    \definition{adv.}{última vez}
  \end{phonetics}
\end{entry}

\begin{entry}{上网}{3,6}{⼀、⽹}
  \begin{phonetics}{上网}{shang4 wang3}[][HSK 1]
    \definition{v.}{conectar à \emph{Internet} | fazer \emph{upload} | ficar \emph{online}}
  \end{phonetics}
\end{entry}

\begin{entry}{上衣}{3,6}{⼀、⾐}
  \begin{phonetics}{上衣}{shang4 yi1}[][HSK 3]
    \definition{s.}{jaqueta; vestimenta externa superior}
  \end{phonetics}
\end{entry}

\begin{entry}{上访}{3,6}{⼀、⾔}
  \begin{phonetics}{上访}{shang4fang3}
    \definition{v.}{buscar uma audiência com superiores (especialmente funcionários do governo) para fazer uma petição por algo}
  \end{phonetics}
\end{entry}

\begin{entry}{上声}{3,7}{⼀、⼠}
  \begin{phonetics}{上声}{shang3sheng1}
    \definition{s.}{tom descendente e ascendente | terceiro tom no mandarim moderno}
  \end{phonetics}
\end{entry}

\begin{entry}{上来}{3,7}{⼀、⽊}
  \begin{phonetics}{上来}{shang4 lai2}[][HSK 3]
    \definition{v.}{subir (para a minha localização) | estar no começo | vir à tona | usado depois de um verbo para indicar sucesso em fazer algo}
  \end{phonetics}
\end{entry}

\begin{entry}{上周}{3,8}{⼀、⼝}
  \begin{phonetics}{上周}{shang4 zhou1}[][HSK 2]
    \definition{s.}{semana passada}
  \end{phonetics}
\end{entry}

\begin{entry}{上坡路}{3,8,13}{⼀、⼟、⾜}
  \begin{phonetics}{上坡路}{shang4po1lu4}
    \definition{s.}{aclive | progresso | (fig.) tendência ascendente}
  \end{phonetics}
\end{entry}

\begin{entry}{上学}{3,8}{⼀、⼦}
  \begin{phonetics}{上学}{shang4 xue2}[][HSK 1]
    \definition{v.}{ir à escola | frequentar a escola | estar na escola | iniciar as aulas}
  \end{phonetics}
\end{entry}

\begin{entry}{上询}{3,8}{⼀、⾔}
  \begin{phonetics}{上询}{shang4 xun2}
    \definition{adv.}{primeira dezena do mês}
  \end{phonetics}
\end{entry}

\begin{entry}{上面}{3,9}{⼀、⾯}
  \begin{phonetics}{上面}{shang4 mian4}[][HSK 3]
    \definition{s.}{uma posição mais alta que algo; uma posição acima/acima de algo | superfície do objeto | aspecto | a parte acima mencionada | autoridades superiores | os mais velhos; a geração mais velha da família}
  \end{phonetics}
\end{entry}

\begin{entry}{上海}{3,10}{⼀、⽔}
  \begin{phonetics}{上海}{shang4hai3}
    \definition*{s.}{Shangai (Xangai)}
  \end{phonetics}
\end{entry}

\begin{entry}{上班}{3,10}{⼀、⽟}
  \begin{phonetics}{上班}{shang4 ban1}[][HSK 1]
    \definition{v.+compl.}{ir para o trabalho | ir para o emprego | estar de plantão}
  \end{phonetics}
\end{entry}

\begin{entry}{上课}{3,10}{⼀、⾔}
  \begin{phonetics}{上课}{shang4 ke4}[][HSK 1]
    \definition{v.}{assistir à aula | ir para a aula | ir dar uma aula}
  \end{phonetics}
\end{entry}

\begin{entry}{上楼}{3,13}{⼀、⽊}
  \begin{phonetics}{上楼}{shang4 lou2}[][HSK 4]
    \definition{v.}{subir as escadas; ir para o andar de cima}
  \end{phonetics}
\end{entry}

\begin{entry}{上演}{3,14}{⼀、⽔}
  \begin{phonetics}{上演}{shang4yan3}
    \definition{s.}{exibição | encenação}
    \definition{v.}{exibir (um filme) | encenar (uma peça)}
  \end{phonetics}
\end{entry}

\begin{entry}{下}{3}{⼀}
  \begin{phonetics}{下}{xia4}[][HSK 1,2]
    \definition{adv.}{abaixo | em baixo de}
    \definition{clas.}{para número de vezes para ações}
    \definition{v.}{descer | chegar a (uma decisão, conclusão, etc.) | recusar}
  \end{phonetics}
\end{entry}

\begin{entry}{下个月}{3,3,4}{⼀、⼈、⽉}
  \begin{phonetics}{下个月}{xia4 ge4 yue4}[][HSK 4]
    \definition{s.}{próximo mês; mês que vem; refere-se ao próximo mês do mês atual}
  \end{phonetics}
\end{entry}

\begin{entry}{下午}{3,4}{⼀、⼗}
  \begin{phonetics}{下午}{xia4wu3}[][HSK 1]
    \definition{adv.}{tarde | à tarde | período logo após o meio-dia}
  \end{phonetics}
\end{entry}

\begin{entry}{下午茶}{3,4,9}{⼀、⼗、⾋}
  \begin{phonetics}{下午茶}{xia4wu3cha2}
    \definition{s.}{chá da tarde (normalmente chás com doces)}
  \end{phonetics}
\end{entry}

\begin{entry}{下巴}{3,4}{⼀、⼰}
  \begin{phonetics}{下巴}{xia4ba5}
    \definition[个]{s.}{queixo}
  \end{phonetics}
\end{entry}

\begin{entry}{下水道}{3,4,12}{⼀、⽔、⾡}
  \begin{phonetics}{下水道}{xia4shui3dao4}
    \definition{s.}{esgoto}
  \end{phonetics}
\end{entry}

\begin{entry}{下车}{3,4}{⼀、⾞}
  \begin{phonetics}{下车}{xia4 che1}[][HSK 1]
    \definition{v.}{descer ou sair (de ônibus, carro, etc.)}
  \end{phonetics}
\end{entry}

\begin{entry}{下去}{3,5}{⼀、⼛}
  \begin{phonetics}{下去}{xia4 qu4}[][HSK 3]
    \definition{part.}{usado depois de verbos para indicar de alto a baixo | usado depois de um verbo para indicar continuação}
    \definition{v.}{descer (a partir da minha localização)
continuar
obter; crescer; tornar-se}
  \end{phonetics}
\end{entry}

\begin{entry}{下边}{3,5}{⼀、⾡}
  \begin{phonetics}{下边}{xia4bian5}[][HSK 1]
    \definition{adv.}{em baixo | abaixo | parte de baixo}
  \end{phonetics}
\end{entry}

\begin{entry}{下旬}{3,6}{⼀、⽇}
  \begin{phonetics}{下旬}{xia4xun2}
    \definition{adv.}{última dezena do mês}
  \end{phonetics}
\end{entry}

\begin{entry}{下次}{3,6}{⼀、⽋}
  \begin{phonetics}{下次}{xia4 ci4}[][HSK 1]
    \definition{s.}{próxima vez}
  \end{phonetics}
\end{entry}

\begin{entry}{下来}{3,7}{⼀、⽊}
  \begin{phonetics}{下来}{xia4 lai5}[][HSK 3]
    \definition{part.}{usado depois de um verbo para indicar que uma ação ou comportamento está se movendo em direção ao falante ou que a ação está continuando ou sendo concluída | usado depois de um adjetivo para indicar que um certo estado começou a aparecer e continuará a se desenvolver.}
    \definition{v.}{descer (para a minha localização) | (colheitas/frutas/vegetais, etc.) ser colhido; estar maduro o suficiente para ser colhido | (período de tempo) acabar; passar; chegar ao fim}
  \end{phonetics}
\end{entry}

\begin{entry}{下周}{3,8}{⼀、⼝}
  \begin{phonetics}{下周}{xia4 zhou1}[][HSK 2]
    \definition{s.}{próxima semana}
  \end{phonetics}
\end{entry}

\begin{entry}{下线}{3,8}{⼀、⽷}
  \begin{phonetics}{下线}{xia4xian4}
    \definition{v.}{ficar \emph{offline} | (um produto) sair da linha de montagem | pessoa abaixo de si em um esquema de pirâmide}
  \end{phonetics}
\end{entry}

\begin{entry}{下降}{3,8}{⼀、⾩}
  \begin{phonetics}{下降}{xia4 jiang4}[][HSK 4]
    \definition{v.}{cair; despencar; declinar; descer; diminuir; ir para baixo}
  \end{phonetics}
\end{entry}

\begin{entry}{下雨}{3,8}{⼀、⾬}
  \begin{phonetics}{下雨}{xia4 yu3}[][HSK 1]
    \definition{v.+compl.}{chover}
  \end{phonetics}
\end{entry}

\begin{entry}{下面}{3,9}{⼀、⾯}
  \begin{phonetics}{下面}{xia4 mian4}[][HSK 3]
    \definition{s.}{em baixo; abaixo; parte de baixo | próximo; seguinte | subordinado; o nível inferior; homens nos níveis inferiores}
    \definition{v.}{cozinhar macarrão}
  \end{phonetics}
\end{entry}

\begin{entry}{下海}{3,10}{⼀、⽔}
  \begin{phonetics}{下海}{xia4hai3}
    \definition{v.+compl.}{ir para o mar; (barco) deixar o porto e iniciar uma jornada | ir pescar no mar | tornar-se ator profissional}
  \end{phonetics}
\end{entry}

\begin{entry}{下班}{3,10}{⼀、⽟}
  \begin{phonetics}{下班}{xia4 ban1}[][HSK 1]
    \definition{v.+compl.}{sair do trabalho}
  \end{phonetics}
\end{entry}

\begin{entry}{下课}{3,10}{⼀、⾔}
  \begin{phonetics}{下课}{xia4 ke4}[][HSK 1]
    \definition{v.+compl.}{acabar a aula | terminar a aula}
  \end{phonetics}
\end{entry}

\begin{entry}{下载}{3,10}{⼀、⾞}
  \begin{phonetics}{下载}{xia4zai3}[][HSK 4]
    \definition{v.}{\emph{download}; baixar; salvar informações da \emph{Web} em um dispositivo, como um computador}
  \end{phonetics}
\end{entry}

\begin{entry}{下蛋}{3,11}{⼀、⾍}
  \begin{phonetics}{下蛋}{xia4dan4}
    \definition{v.}{botar ovos}
  \end{phonetics}
\end{entry}

\begin{entry}{下雪}{3,11}{⼀、⾬}
  \begin{phonetics}{下雪}{xia4 xue3}[][HSK 2]
    \definition[场,次]{s.}{neve}
    \definition{v.+compl.}{nevar}
  \end{phonetics}
\end{entry}

\begin{entry}{下崽}{3,12}{⼀、⼭}
  \begin{phonetics}{下崽}{xia4zai3}
    \definition{v.}{dar à luz (animais) | parir}
  \end{phonetics}
\end{entry}

\begin{entry}{下楼}{3,13}{⼀、⽊}
  \begin{phonetics}{下楼}{xia4 lou2}[][HSK 4]
    \definition{v.}{descer as escadas}
  \end{phonetics}
\end{entry}

\begin{entry}{与}{3}{⼀}
  \begin{phonetics}{与}{yu3}
    \definition{conj.}{e, com}
  \end{phonetics}
  \begin{phonetics}{与}{yu4}
    \definition{v.}{fazer parte de}
  \end{phonetics}
\end{entry}

\begin{entry}{与其}{3,8}{⼀、⼋}
  \begin{phonetics}{与其}{yu3qi2}
    \definition{conj.}{mais do que}
  \end{phonetics}
\end{entry}

\begin{entry}{与其……不如……}{3,8,4,6}{⼀、⼋、⼀、⼥}
  \begin{phonetics}{与其……不如……}{yu3qi2 bu4ru2}
    \definition{conj.}{ao invés de\dots melhor que\dots}
  \end{phonetics}
\end{entry}

\begin{entry}{与其……宁可……}{3,8,5,5}{⼀、⼋、⼧、⼝}
  \begin{phonetics}{与其……宁可……}{yu3qi2 ning4ke3}
    \definition{conj.}{ao invés de\dots melhor que\dots}
  \end{phonetics}
\end{entry}

\begin{entry}{不}{4}{⼀}
  \begin{phonetics}{不}{bu2}[(antes de quarto tom)][HSK 1]
    \definition{adv.}{não}
    \definition{pref.}{prefixo negativo}
  \end{phonetics}
  \begin{phonetics}{不}{bu4}[][HSK 1]
    \definition{adv.}{não}
    \definition{pref.}{prefixo negativo}
  \end{phonetics}
  \begin{phonetics}{不}{bu5}[][HSK 1]
    \definition{adv.}{não (em expressões ``v.+不+v.'')}
  \end{phonetics}
\end{entry}

\begin{entry}{不一会儿}{4,1,6,2}{⼀、⼀、⼈、⼉}
  \begin{phonetics}{不一会儿}{bu4 yi2 hui4r5}[][HSK 2]
    \definition{expr.}{em um momento | em pouco tempo |em breve}
  \end{phonetics}
\end{entry}

\begin{entry}{不一定}{4,1,8}{⼀、⼀、⼧}
  \begin{phonetics}{不一定}{bu4 yi2 ding4}[][HSK 2]
    \definition{adv.}{talvez | incerto | não tenho certeza | não necessariamente}
  \end{phonetics}
\end{entry}

\begin{entry}{不久}{4,3}{⼀、⼃}
  \begin{phonetics}{不久}{bu4 jiu3}[][HSK 2]
    \definition{adj.}{em breve | futuro próximo | logo depois | não muito depois | não muito tempo (antes ou depois de algo)}
  \end{phonetics}
\end{entry}

\begin{entry}{不大}{4,3}{⼀、⼤}
  \begin{phonetics}{不大}{bu2 da4}[][HSK 1]
    \definition{adv.}{não muito | não frequentemente | raramente |dificilmente | escassamente}
  \end{phonetics}
\end{entry}

\begin{entry}{不大离}{4,3,10}{⼀、⼤、⼇}
  \begin{phonetics}{不大离}{bu2da4li2}
    \definition{adj.}{bem perto | quase certo | nada mal}
  \end{phonetics}
\end{entry}

\begin{entry}{不仅}{4,4}{⼀、⼈}
  \begin{phonetics}{不仅}{bu4jin3}[][HSK 3]
    \definition{adv.}{não apenas (em número, quantidade ou extensão)}
    \definition{conj.}{não somente}
  \end{phonetics}
\end{entry}

\begin{entry}{不公}{4,4}{⼀、⼋}
  \begin{phonetics}{不公}{bu4gong1}
    \definition{adj.}{injusto}
  \end{phonetics}
\end{entry}

\begin{entry}{不太}{4,4}{⼀、⼤}
  \begin{phonetics}{不太}{bu2 tai4}[][HSK 2]
    \definition{adv.}{não bastante | não muito}
  \end{phonetics}
\end{entry}

\begin{entry}{不少}{4,4}{⼀、⼩}
  \begin{phonetics}{不少}{bu4 shao3}[][HSK 2]
    \definition{adj.}{muitos | bastante | não poucos}
  \end{phonetics}
\end{entry}

\begin{entry}{不日}{4,4}{⼀、⽇}
  \begin{phonetics}{不日}{bu2ri4}
    \definition{adv.}{em alguns dias}
  \end{phonetics}
\end{entry}

\begin{entry}{不止}{4,4}{⼀、⽌}
  \begin{phonetics}{不止}{bu4zhi3}
    \definition{adv.}{incessantemente | sem fim | mais que | não limitado a}
  \end{phonetics}
\end{entry}

\begin{entry}{不可避免}{4,5,16,7}{⼀、⼝、⾌、⼉}
  \begin{phonetics}{不可避免}{bu4ke3bi4mian3}
    \definition{adj./adv.}{inevitável}
  \end{phonetics}
\end{entry}

\begin{entry}{不对}{4,5}{⼀、⼨}
  \begin{phonetics}{不对}{bu2 dui4}[][HSK 1]
    \definition{adj.}{incorreto | errado | anormal | estranho | estar em desacordo com | ser difícil de conviver}
  \end{phonetics}
\end{entry}

\begin{entry}{不必}{4,5}{⼀、⼼}
  \begin{phonetics}{不必}{bu2 bi4}[][HSK 3]
    \definition{adv.}{não precisa | não tem que}
  \end{phonetics}
\end{entry}

\begin{entry}{不用}{4,5}{⼀、⽤}
  \begin{phonetics}{不用}{bu2 yong4}[][HSK 1]
    \definition{v.}{não precisar}
    \seeref{甭}{beng2}
  \end{phonetics}
\end{entry}

\begin{entry}{不光}{4,6}{⼀、⼉}
  \begin{phonetics}{不光}{bu4 guang1}[][HSK 3]
    \definition{adv.}{não é o único}
    \definition{conj.}{não somente}
  \end{phonetics}
\end{entry}

\begin{entry}{不同}{4,6}{⼀、⼝}
  \begin{phonetics}{不同}{bu4 tong2}[][HSK 2]
    \definition{adj.}{diferente | distinto}
  \end{phonetics}
\end{entry}

\begin{entry}{不在乎}{4,6,5}{⼀、⼟、⼃}
  \begin{phonetics}{不在乎}{bu2 zai4 hu1}[][HSK 4]
    \definition{v.}{não se importar; não dar a mínima; não dar atenção}
  \end{phonetics}
\end{entry}

\begin{entry}{不好意思}{4,6,13,9}{⼀、⼥、⼼、⼼}
  \begin{phonetics}{不好意思}{bu4 hao3 yi4 si5}[][HSK 2]
    \definition{adj.}{pedir desculpas (por incomodar alguém) | sentir-se envergonhado | achar isso embaraçoso}
  \end{phonetics}
\end{entry}

\begin{entry}{不如}{4,6}{⼀、⼥}
  \begin{phonetics}{不如}{bu4ru2}[][HSK 2]
    \definition{conj.}{em vez de | melhor que | seria melhor}
    \definition{v.}{ser inferior a | não ser igual a | não ser tão bom quanto | não poder fazer melhor que}
  \end{phonetics}
\end{entry}

\begin{entry}{不安}{4,6}{⼀、⼧}
  \begin{phonetics}{不安}{bu4'an1}[][HSK 3]
    \definition{adj.}{inquieto | instável | intranquilo | pesaroso}
  \end{phonetics}
\end{entry}

\begin{entry}{不成话}{4,6,8}{⼀、⼽、⾔}
  \begin{phonetics}{不成话}{bu4cheng2hua4}
    \definition{expr.}{sem razão | demasiado irracionável}
    \seeref{不是话}{bu2shi4hua4}
    \seeref{不像话}{bu2xiang4hua4}
  \end{phonetics}
\end{entry}

\begin{entry}{不行}{4,6}{⼀、⾏}
  \begin{phonetics}{不行}{bu4 xing2}[][HSK 2]
    \definition{adj.}{não funciona | não é bom}
    \definition{adv.}{profundamente | terrivelmente | extremamente}
    \definition{v.}{não fazer | não ser permitido | estar fora de questão | estar à beira da morte}
  \end{phonetics}
\end{entry}

\begin{entry}{不论}{4,6}{⼀、⾔}
  \begin{phonetics}{不论}{bu2 lun4}[][HSK 3]
    \definition{conj.}{não importa (o que, quem, como, etc.) | se \dots ou \dots}
  \end{phonetics}
\end{entry}

\begin{entry}{不论……也……}{4,6,3}{⼀、⾔、⼄}
  \begin{phonetics}{不论……也……}{bu2lun4 ye3}
    \definition{conj.}{não apenas\dots, (o que, quem, como, etc.), \dots}
  \end{phonetics}
\end{entry}

\begin{entry}{不论……都……}{4,6,10}{⼀、⾔、⾢}
  \begin{phonetics}{不论……都……}{bu2lun4 dou1}
    \definition{conj.}{não apenas\dots, (o que, quem, como, etc.), \dots}
  \end{phonetics}
\end{entry}

\begin{entry}{不过}{4,6}{⼀、⾡}
  \begin{phonetics}{不过}{bu2guo4}[][HSK 2]
    \definition{conj.}{mas | contudo | no entanto}
  \end{phonetics}
\end{entry}

\begin{entry}{不但}{4,7}{⼀、⼈}
  \begin{phonetics}{不但}{bu2 dan4}[][HSK 2]
    \definition{conj.}{não somente}
  \end{phonetics}
\end{entry}

\begin{entry}{不但……而且……}{4,7,6,5}{⼀、⼈、⽽、⼀}
  \begin{phonetics}{不但……而且……}{bu2 dan4 er2qie3}[][HSK 2]
    \definition{conj.}{não só\dots mas também\dots}
  \end{phonetics}
\end{entry}

\begin{entry}{不到}{4,8}{⼀、⼑}
  \begin{phonetics}{不到}{bu2dao4}
    \definition{adj.}{insuficiente}
    \definition{adv.}{menos que}
    \definition{v.}{não chegar}
  \end{phonetics}
\end{entry}

\begin{entry}{不注意}{4,8,13}{⼀、⽔、⼼}
  \begin{phonetics}{不注意}{bu2zhu4yi4}
    \definition{adj.}{impensado | distraído}
    \definition{s.}{descuido | distração}
  \end{phonetics}
\end{entry}

\begin{entry}{不客气}{4,9,4}{⼀、⼧、⽓}
  \begin{phonetics}{不客气}{bu2 ke4 qi5}[][HSK 1]
    \definition{adj.}{indelicado | rude | brusco}
    \definition{expr.}{de nada | não há de que | não mencione isso}
  \end{phonetics}
\end{entry}

\begin{entry}{不是……而是}{4,9,6,9}{⼀、⽇、⽽、⽇}
  \begin{phonetics}{不是……而是}{bu4shi4 er2 shi4}
    \definition{conj.}{não somente\dots mas também\dots, expressam um relacionamento mais profundo e avançado em significado, mas as orações antes e depois são consistentes em expressar significados negativos e afirmativos, entretanto, a primeira metade da frase expressa negação, e a segunda metade expressa afirmação, e o significado das orações anteriores e seguintes não pode ser de um nível mais alto}
  \end{phonetics}
\end{entry}

\begin{entry}{不是话}{4,9,8}{⼀、⽇、⾔}
  \begin{phonetics}{不是话}{bu2shi4hua4}
    \definition{expr.}{sem razão | demasiado irracionável}
    \seeref{不像话}{bu2xiang4hua4}
    \seeref{不成话}{bu4cheng2hua4}
  \end{phonetics}
\end{entry}

\begin{entry}{不要}{4,9}{⼀、⾑}
  \begin{phonetics}{不要}{bu2 yao4}[][HSK 2]
    \definition{adv.}{nada de (pedir a alguém para não fazer) | não}
  \end{phonetics}
\end{entry}

\begin{entry}{不要紧}{4,9,10}{⼀、⾑、⽷}
  \begin{phonetics}{不要紧}{bu2yao4jin3}[][HSK 4]
    \definition{adj.}{sem importância; sem seriedade; não problemático | não importa; não é um obstáculo | parece estar tudo bem, mas | à primeira vista, isso não parece atrapalhar}
  \end{phonetics}
\end{entry}

\begin{entry}{不够}{4,11}{⼀、⼣}
  \begin{phonetics}{不够}{bu2 gou4}[][HSK 2]
    \definition{adv.}{insuficiente}
    \definition{v.}{não ser suficiente}
  \end{phonetics}
\end{entry}

\begin{entry}{不得不}{4,11,4}{⼀、⼻、⼀}
  \begin{phonetics}{不得不}{bu4de2bu4}[][HSK 3]
    \definition{adv.}{tem que | não tem escolha a não ser}
  \end{phonetics}
\end{entry}

\begin{entry}{不断}{4,11}{⼀、⽄}
  \begin{phonetics}{不断}{bu2duan4}[][HSK 3]
    \definition{adv.}{continuamente | sem fim}
  \end{phonetics}
\end{entry}

\begin{entry}{不然}{4,12}{⼀、⽕}
  \begin{phonetics}{不然}{bu4ran2}[][HSK 4]
    \definition{adj.}{não é assim; não é o caso}
    \definition{conj.}{se não; caso contrário; indica outra consequência ou circunstância que teria ocorrido se não fosse}
  \end{phonetics}
\end{entry}

\begin{entry}{不像话}{4,13,8}{⼀、⼈、⾔}
  \begin{phonetics}{不像话}{bu2xiang4hua4}
    \definition{expr.}{sem razão | demasiado irracionável}
    \seeref{不是话}{bu2shi4hua4}
    \seeref{不成话}{bu4cheng2hua4}
  \end{phonetics}
\end{entry}

\begin{entry}{不满}{4,13}{⼀、⽔}
  \begin{phonetics}{不满}{bu4 man3}[][HSK 2]
    \definition{adj.}{ressentido | insatisfeito | descontente}
    \definition{v.}{estar descontente com |ser menor que}
  \end{phonetics}
\end{entry}

\begin{entry}{不错}{4,13}{⼀、⾦}
  \begin{phonetics}{不错}{bu2 cuo4}[][HSK 2]
    \definition{adj.}{correto | não (é) mau | bastante bom | certo}
  \end{phonetics}
\end{entry}

\begin{entry}{不管}{4,14}{⼀、⽵}
  \begin{phonetics}{不管}{bu4guan3}[][HSK 4]
    \definition{conj.}{não importa (o que, como, etc.); independentemente de; indica que, embora as condições ou circunstâncias tenham mudado, o resultado permanece o mesmo}
    \seeref{不管……都……}{bu4guan3 dou1}
    \seeref{不管……也……}{bu4guan3 ye3}
  \end{phonetics}
\end{entry}

\begin{entry}{不管……也……}{4,14,3}{⼀、⽵、⼄}
  \begin{phonetics}{不管……也……}{bu4guan3 ye3}
    \definition{conj.}{não apenas\dots, (o que, quem, como, etc.), \dots}
  \end{phonetics}
\end{entry}

\begin{entry}{不管……都……}{4,14,10}{⼀、⽵、⾢}
  \begin{phonetics}{不管……都……}{bu4guan3 dou1}
    \definition{conj.}{não apenas\dots, (o que, quem, como, etc.), \dots}
  \end{phonetics}
\end{entry}

\begin{entry}{专门}{4,3}{⼀、⾨}
  \begin{phonetics}{专门}{zhuan1men2}[][HSK 3]
    \definition{adj.}{especializado}
    \definition{adv.}{especialmente}
  \end{phonetics}
\end{entry}

\begin{entry}{专心}{4,4}{⼀、⼼}
  \begin{phonetics}{专心}{zhuan1xin1}[][HSK 4]
    \definition{adj.}{absorto; concentrado}
  \end{phonetics}
\end{entry}

\begin{entry}{专业}{4,5}{⼀、⼀}
  \begin{phonetics}{专业}{zhuan1ye4}[][HSK 3]
    \definition{adj.}{profissional; descreve uma pessoa que tem um alto nível ou rico conhecimento em uma determinada área}
    \definition[个,门]{s.}{profissão; linha especial; comércio especializado; unidades de negócios no departamento de produção | especialidade; disciplina; assunto especializado; campo especial de estudo; um departamento em uma faculdade ou escola profissional secundária}
  \end{phonetics}
\end{entry}

\begin{entry}{专业人士}{4,5,2,3}{⼀、⼀、⼈、⼠}
  \begin{phonetics}{专业人士}{zhuan1ye4ren2shi4}
    \definition{s.}{profissional}
  \end{phonetics}
\end{entry}

\begin{entry}{专业人才}{4,5,2,3}{⼀、⼀、⼈、⼿}
  \begin{phonetics}{专业人才}{zhuan1ye4ren2cai2}
    \definition{s.}{especialista (em uma área)}
  \end{phonetics}
\end{entry}

\begin{entry}{专业化}{4,5,4}{⼀、⼀、⼔}
  \begin{phonetics}{专业化}{zhuan1ye4hua4}
    \definition{s.}{especialização}
  \end{phonetics}
\end{entry}

\begin{entry}{专业户}{4,5,4}{⼀、⼀、⼾}
  \begin{phonetics}{专业户}{zhuan1ye4hu4}
    \definition{s.}{indústria caseira | empresa familiar produzindo um produto especial}
  \end{phonetics}
\end{entry}

\begin{entry}{专业性}{4,5,8}{⼀、⼀、⼼}
  \begin{phonetics}{专业性}{zhuan1ye4xing4}
    \definition{s.}{profissionalismo | expertise}
  \end{phonetics}
\end{entry}

\begin{entry}{专业教育}{4,5,11,8}{⼀、⼀、⽁、⾁}
  \begin{phonetics}{专业教育}{zhuan1ye4jiao4yu4}
    \definition{s.}{educação especializada | escola técnica}
  \end{phonetics}
\end{entry}

\begin{entry}{专家}{4,10}{⼀、⼧}
  \begin{phonetics}{专家}{zhuan1jia1}[][HSK 3]
    \definition[个]{s.}{perito; especialista; proficiente; uma pessoa que é especialista em um determinado assunto; uma pessoa que é boa em uma determinada tecnologia}
  \end{phonetics}
\end{entry}

\begin{entry}{专题}{4,15}{⼀、⾴}
  \begin{phonetics}{专题}{zhuan1ti2}[][HSK 3]
    \definition{s.}{assunto especial; tópico especial}
  \end{phonetics}
\end{entry}

\begin{entry}{且}{5}{⼀}
  \begin{phonetics}{且}{qie3}
    \definition*{s.}{sobrenome Qie}
    \definition{adv.}{apenas; por enquanto | por um longo tempo}
    \definition{conj.}{mesmo; até; até mesmo | ambos\dots e\dots}
  \end{phonetics}
\end{entry}

\begin{entry}{世代}{5,5}{⼀、⼈}
  \begin{phonetics}{世代}{shi4dai4}
    \definition{adv.}{por muitas gerações, eras}
    \definition{s.}{geração | era}
  \end{phonetics}
\end{entry}

\begin{entry}{世纪}{5,6}{⼀、⽷}
  \begin{phonetics}{世纪}{shi4ji4}[][HSK 3]
    \definition[个]{s.}{século}
  \end{phonetics}
\end{entry}

\begin{entry}{世界}{5,9}{⼀、⽥}
  \begin{phonetics}{世界}{shi4jie4}[][HSK 3]
    \definition[个]{s.}{mundo | a soma da natureza e da sociedade humana | o universo sem limites | situação social}
  \end{phonetics}
\end{entry}

\begin{entry}{世界杯}{5,9,8}{⼀、⽥、⽊}
  \begin{phonetics}{世界杯}{shi4jie4bei1}[][HSK 3]
    \definition*{s.}{Copa do Mundo}
  \end{phonetics}
\end{entry}

\begin{entry}{世锦赛}{5,13,14}{⼀、⾦、⾙}
  \begin{phonetics}{世锦赛}{shi4jin3sai4}
    \definition*{s.}{Campeonato Mundial}
  \end{phonetics}
\end{entry}

\begin{entry}{丘陵}{5,10}{⼀、⾩}
  \begin{phonetics}{丘陵}{qiu1ling2}
    \definition{s.}{colinas}
  \end{phonetics}
\end{entry}

\begin{entry}{业余}{5,7}{⼀、⼈}
  \begin{phonetics}{业余}{ye4yu2}[][HSK 4]
    \definition{adj.}{tempo livre; depois do expediente; fora do horário de trabalho | amador; não profissional}
  \end{phonetics}
\end{entry}

\begin{entry}{东}{5}{⼀}
  \begin{phonetics}{东}{dong1}[][HSK 1]
    \definition*{s.}{sobrenome Dong}
    \definition{s.}{leste}
  \end{phonetics}
\end{entry}

\begin{entry}{东方}{5,4}{⼀、⽅}
  \begin{phonetics}{东方}{dong1 fang1}[][HSK 2]
    \definition*{s.}{sobrenome Dongfang}
    \definition{s.}{leste | oriente}
  \end{phonetics}
\end{entry}

\begin{entry}{东方学院}{5,4,8,9}{⼀、⽅、⼦、⾩}
  \begin{phonetics}{东方学院}{dong1fang1 xue2yuan4}
    \definition*{s.}{Instituto Oriental}
  \end{phonetics}
\end{entry}

\begin{entry}{东北}{5,5}{⼀、⼔}
  \begin{phonetics}{东北}{dong1 bei3}[][HSK 2]
    \definition*{s.}{Nordeste da China | Manchúria}
    \definition{s.}{nordeste}
  \end{phonetics}
\end{entry}

\begin{entry}{东半球}{5,5,11}{⼀、⼗、⽟}
  \begin{phonetics}{东半球}{dong1ban4qiu2}
    \definition*{s.}{Hemisfério Oriental}
  \end{phonetics}
\end{entry}

\begin{entry}{东边}{5,5}{⼀、⾡}
  \begin{phonetics}{东边}{dong1 bian5}[][HSK 1]
    \definition{s.}{este | leste | lado leste | oriente}
  \end{phonetics}
\end{entry}

\begin{entry}{东西}{5,6}{⼀、⾑}
  \begin{phonetics}{东西}{dong1xi1}
    \definition{s.}{leste e oeste}
  \end{phonetics}
  \begin{phonetics}{东西}{dong1xi5}[][HSK 1]
    \definition[个,件]{s.}{coisa | material | pessoa}
  \end{phonetics}
\end{entry}

\begin{entry}{东南}{5,9}{⼀、⼗}
  \begin{phonetics}{东南}{dong1 nan2}[][HSK 2]
    \definition{s.}{sudeste | sudeste da China | o Sudeste}
  \end{phonetics}
\end{entry}

\begin{entry}{东面}{5,9}{⼀、⾯}
  \begin{phonetics}{东面}{dong1mian4}
    \definition{s.}{lado leste (de algo)}
  \end{phonetics}
\end{entry}

\begin{entry}{东部}{5,10}{⼀、⾢}
  \begin{phonetics}{东部}{dong1 bu4}[][HSK 3]
    \definition{s.}{o leste; parte oriental}
  \end{phonetics}
\end{entry}

\begin{entry}{丝}{5}{⼀}
  \begin{phonetics}{丝}{si1}
    \definition{adj.}{filiforme | delgado como um fio | que se assemelha a um fio}
    \definition{clas.}{um traço (de fumaça, etc.) | um pouquinho, etc.}
    \definition{s.}{seda | (cozinha) pedaços ou tiras de julienne, tiras cortadas finas}
  \end{phonetics}
\end{entry}

\begin{entry}{两}{7}{⼀}
  \begin{phonetics}{两}{liang3}[][HSK 1,2]
    \definition{adv.}{ambos (lados) | cada (lado)}
    \definition{clas.}{liang, uma unidade de peso (=50 gramas)}
    \definition{num.}{dois (sempre usado antes de classificadores) | poucos; alguns}
  \end{phonetics}
\end{entry}

\begin{entry}{两边}{7,5}{⼀、⾡}
  \begin{phonetics}{两边}{liang3 bian1}[][HSK 4]
    \definition{s.}{ambos os lados; ambas as direções; ambos os lugares | ambas as partes; ambos os lados}
  \end{phonetics}
\end{entry}

\begin{entry}{两码事}{7,8,8}{⼀、⽯、⼅}
  \begin{phonetics}{两码事}{liang3ma3shi4}
    \definition{expr.}{duas coisas completamente diferentes}
  \end{phonetics}
\end{entry}

\begin{entry}{严}{7}{⼀}
  \begin{phonetics}{严}{yan2}[][HSK 4]
    \definition*{s.}{sobrenome Yan}
    \definition{adj.}{rígido; rigoroso; estrito; severo}
    \definition{s.}{pai; refere-se ao pai}
  \end{phonetics}
\end{entry}

\begin{entry}{严重}{7,9}{⼀、⾥}
  \begin{phonetics}{严重}{yan2zhong4}[][HSK 4]
    \definition{adj.}{sério; grave; crítico; severo}
  \end{phonetics}
\end{entry}

\begin{entry}{严重打伤}{7,9,5,6}{⼀、⾥、⼿、⼈}
  \begin{phonetics}{严重打伤}{yan2zhong4 da3 shang1}
    \definition{s.}{gravemente ferido}
  \end{phonetics}
\end{entry}

\begin{entry}{严重伤害}{7,9,6,10}{⼀、⾥、⼈、⼧}
  \begin{phonetics}{严重伤害}{yan2zhong4 shang1hai4}
    \definition{s.}{ferimento grave}
  \end{phonetics}
\end{entry}

\begin{entry}{严重关切}{7,9,6,4}{⼀、⾥、⼋、⼑}
  \begin{phonetics}{严重关切}{yan2zhong4guan1qie4}
    \definition{s.}{preocupação séria}
  \end{phonetics}
\end{entry}

\begin{entry}{严重危害}{7,9,6,10}{⼀、⾥、⼙、⼧}
  \begin{phonetics}{严重危害}{yan2zhong4wei1hai4}
    \definition{s.}{danos graves}
  \end{phonetics}
\end{entry}

\begin{entry}{严重后果}{7,9,6,8}{⼀、⾥、⼝、⽊}
  \begin{phonetics}{严重后果}{yan2zhong4hou4guo3}
    \definition{s.}{consequências sérias | repercursões graves}
  \end{phonetics}
\end{entry}

\begin{entry}{严重地}{7,9,6}{⼀、⾥、⼟}
  \begin{phonetics}{严重地}{yan2zhong4 di4}
    \definition{adv.}{seriamente | gravemente}
  \end{phonetics}
\end{entry}

\begin{entry}{严重问题}{7,9,6,15}{⼀、⾥、⾨、⾴}
  \begin{phonetics}{严重问题}{yan2zhong4wen4ti2}
    \definition{s.}{problema sério}
  \end{phonetics}
\end{entry}

\begin{entry}{严重性}{7,9,8}{⼀、⾥、⼼}
  \begin{phonetics}{严重性}{yan2zhong4xing4}
    \definition{s.}{seriedade | gravidade}
  \end{phonetics}
\end{entry}

\begin{entry}{严重破坏}{7,9,10,7}{⼀、⾥、⽯、⼟}
  \begin{phonetics}{严重破坏}{yan2zhong4 po4huai4}
    \definition{s.}{destruição grave}
  \end{phonetics}
\end{entry}

\begin{entry}{严格}{7,10}{⼀、⽊}
  \begin{phonetics}{严格}{yan2ge2}[][HSK 4]
    \definition{adj.}{rígido; estrito; rigoroso; muito consciente e meticuloso na implementação de sistemas e no domínio de padrões}
    \definition{v.}{tornar (sistemas, provisões, etc.) rigorosos;}
  \end{phonetics}
\end{entry}

\begin{entry}{靣}{8}{⼀}
  \begin{phonetics}{靣}{mian4}
    \variantof{面}
  \end{phonetics}
\end{entry}

%%%%% EOF %%%%%

