%%%
%%% Radical "⼀"
%%%

\section*{Radical 1: ``⼀''}\addcontentsline{toc}{section}{Radical 1: ⼀}

\begin{Entry}{一}{1}{⼀}[Kangxi 1]
  \begin{Phonetics}{一}{yi1}[(quando usado sozinho)][HSK 1]
    \definition{adv.}{uma vez; assim que; indica que duas ações ocorreram em um intervalo de tempo muito curto, uma terminando e a outra começando imediatamente em seguida | indica que primeiro se realiza uma ação e, em seguida, o resultado dessa ação  | indica uma ação única, indicando que a ação é muito curta ou apenas uma tentativa}
    \definition{num.}{um; 1 | pronunciado como \dpy{yao1} quando dito número a número | igual; refere-se ao mesmo ou igual | inteiro; todo; por toda parte | exclusivo ou único | refere-se a algo específico | também; caso contrário; referindo-se a outro ou mais um}
    \definition{part.}{antes de certas palavras para dar ênfase}
    \definition{prep.}{cada; por; toda vez}
    \definition{s.}{uma nota da escala em Gongchepu (工尺谱), correspondente ao 17 na notação musical numerada}
  \seealsoref{工尺谱}{gong1 che3 pu3}
  \end{Phonetics}
  \begin{Phonetics}{一}{yi2}[(antes de quarto tom)][HSK 1]
    \definition{num.}{um; 1 | um (artigo)}
  \end{Phonetics}
  \begin{Phonetics}{一}{yi4}[][HSK 1]
    \definition{adv.}{uma vez | assim que | ao}
    \definition{num.}{um; 1 | um (artigo)}
  \end{Phonetics}
\end{Entry}

\begin{Entry}{一下}{1,3}{⼀、⼀}
  \begin{Phonetics}{一下}{yi2xia4}
    \definition{adv.}{em um curto tempo | rapidamente}
  \end{Phonetics}
\end{Entry}

\begin{Entry}{一下儿}{1,3,2}{⼀、⼀、⼉}
  \begin{Phonetics}{一下儿}{yi2 xia4r5}[][HSK 1,5]
    \definition{s.}{um tempo; um momento}
  \end{Phonetics}
\end{Entry}

\begin{Entry}{一下子}{1,3,3}{⼀、⼀、⼦}
  \begin{Phonetics}{一下子}{yi2 xia4 zi5}[][HSK 5]
    \definition{adv.}{tudo de uma vez; de repente; em pouco tempo; em um curto espaço de tempo}
  \end{Phonetics}
\end{Entry}

\begin{Entry}{一个样}{1,3,10}{⼀、⼈、⽊}
  \begin{Phonetics}{一个样}{yi2ge5yang4}
    \definition{s.}{o mesmo}
  \seealsoref{一样}{yi2yang4}
  \end{Phonetics}
\end{Entry}

\begin{Entry}{一口气}{1,3,4}{⼀、⼝、⽓}
  \begin{Phonetics}{一口气}{yi4 kou3 qi4}[][HSK 5]
    \definition{adv.}{em um só fôlego; sem pausa; fazer algo continuamente}
  \end{Phonetics}
\end{Entry}

\begin{Entry}{一切}{1,4}{⼀、⼑}
  \begin{Phonetics}{一切}{yi2qie4}[][HSK 3]
    \definition{pron.}{tudo; todo; todas as coisas}
  \end{Phonetics}
\end{Entry}

\begin{Entry}{一方面}{1,4,9}{⼀、⽅、⾯}
  \begin{Phonetics}{一方面}{yi4 fang1 mian4}[][HSK 3]
    \definition{s.}{um lado; um dos dois aspectos opostos ou um lado de algo que está relacionado a outro}
  \seealsoref{一方面……,一方面……}{yi4 fang1 mian4 yi4 fang1 mian4}
  \end{Phonetics}
\end{Entry}

\begin{Entry}{一方面……,一方面……}{1,4,9,1,4,9}{⼀、⽅、⾯、⼀、⽅、⾯}
  \begin{Phonetics}{一方面……,一方面……}{yi4 fang1 mian4 yi4 fang1 mian4}[][HSK 3]
    \definition{conj.}{por um lado\dots, por outro lado\dots; conecta duas orações paralelas (devem ser usadas juntas)}[\underline{一方}面觉得兴奋,\underline{一方面}又害怕。===Por um lado, sinto-me entusiasmado, mas, por outro, também sinto medo.]
  \end{Phonetics}
\end{Entry}

\begin{Entry}{一代}{1,5}{⼀、⼈}
  \begin{Phonetics}{一代}{yi2 dai4}[][HSK 6]
    \definition{s.}{uma dinastia | era; época atual | vida; geração; toda a vida de uma pessoa}
  \end{Phonetics}
\end{Entry}

\begin{Entry}{一半}{1,5}{⼀、⼗}
  \begin{Phonetics}{一半}{yi2ban4}[][HSK 1]
    \definition{num.}{metade; em parte; uma metade}
  \end{Phonetics}
\end{Entry}

\begin{Entry}{一句话}{1,5,8}{⼀、⼝、⾔}
  \begin{Phonetics}{一句话}{yi2 ju4 hua4}[][HSK 5]
    \definition{s.}{em resumo; em uma palavra; expressar um conteúdo complexo de forma sucinta | trabalho fácil; fácil de fazer; descrever uma tarefa ou trabalho como muito simples e fácil de realizar}
  \end{Phonetics}
\end{Entry}

\begin{Entry}{一旦}{1,5}{⼀、⽇}
  \begin{Phonetics}{一旦}{yi2dan4}[][HSK 5]
    \definition{adv.}{uma vez; no caso; agora que | de repente; uma vez}
    \definition{s.}{em um único dia; em um tempo muito curto;}
  \end{Phonetics}
\end{Entry}

\begin{Entry}{一生}{1,5}{⼀、⽣}
  \begin{Phonetics}{一生}{yi4 sheng1}[][HSK 2]
    \definition{s.}{vida inteira; toda a vida; ao longo da vida; todo o tempo desde o nascimento até a morte; às vezes exagerado para indicar um longo período de tempo no curso da vida}
  \end{Phonetics}
\end{Entry}

\begin{Entry}{一边}{1,5}{⼀、⾡}
  \begin{Phonetics}{一边}{yi4bian1}[][HSK 1]
    \definition{adj.}{igual; idêntico; da mesma forma}
    \definition{adv.}{enquanto; ao mesmo tempo; simultaneamente; indica que uma ação ocorre simultaneamente a outra ação}
    \definition{s.}{lado; um lado; um aspecto | ambos os lados; ao lado de}
  \end{Phonetics}
\end{Entry}

\begin{Entry}{一会儿}{1,6,2}{⼀、⼈、⼉}
  \begin{Phonetics}{一会儿}{yi2 hui4r5}[][HSK 1,2]
    \definition{adv.}{agora\dots agora\dots; um momento\dots o próximo\dots; usado antes de dois antônimos, indica a alternância de situações}
    \definition{s.}{um pouquinho de tempo; muito pouco tempo}
  \end{Phonetics}
\end{Entry}

\begin{Entry}{一会儿……一会儿……}{1,6,2,1,6,2}{⼀、⼈、⼉、⼀、⼈、⼉}
  \begin{Phonetics}{一会儿……一会儿……}{yi1hui4r5 yi1hui4r5}
    \definition{adv.}{um tempo\dots um tempo\dots}
  \end{Phonetics}
\end{Entry}

\begin{Entry}{一共}{1,6}{⼀、⼋}
  \begin{Phonetics}{一共}{yi2gong4}[][HSK 2]
    \definition{adv.}{completamente; em tudo; no todo}
  \end{Phonetics}
\end{Entry}

\begin{Entry}{一再}{1,6}{⼀、⼌}
  \begin{Phonetics}{一再}{yi2zai4}[][HSK 4]
    \definition{adv.}{repetidamente; de novo e de novo; repetidas vezes; uma e outra vez}
  \end{Phonetics}
\end{Entry}

\begin{Entry}{一同}{1,6}{⼀、⼝}
  \begin{Phonetics}{一同}{yi4tong2}[][HSK 6]
    \definition{adv.}{juntos; ao mesmo tempo e lugar}
  \end{Phonetics}
\end{Entry}

\begin{Entry}{一向}{1,6}{⼀、⼝}
  \begin{Phonetics}{一向}{yi2xiang4}[][HSK 5]
    \definition{adv.}{desde o início; indica do passado até o presente}
  \end{Phonetics}
\end{Entry}

\begin{Entry}{一次性}{1,6,8}{⼀、⽋、⼼}
  \begin{Phonetics}{一次性}{yi2 ci4 xing4}[][HSK 6]
    \definition{adj.}{único; uso único; descartável (produtos); apenas uma vez, sem necessidade ou necessidade de fazer novamente}
  \end{Phonetics}
\end{Entry}

\begin{Entry}{一行}{1,6}{⼀、⾏}
  \begin{Phonetics}{一行}{yi1 xing2}[][HSK 6]
    \definition{s.}{delegação; um grupo viajando junto; festa}
  \end{Phonetics}
\end{Entry}

\begin{Entry}{一齐}{1,6}{⼀、⿑}
  \begin{Phonetics}{一齐}{yi4 qi2}[][HSK 6]
    \definition{adv.}{juntos; em uníssono; simultaneamente; ao mesmo tempo; indica que diferentes sujeitos emitem simultaneamente o mesmo comportamento ou o mesmo sujeito emite vários comportamentos diferentes ao mesmo tempo}
  \end{Phonetics}
\end{Entry}

\begin{Entry}{一块}{1,7}{⼀、⼟}
  \begin{Phonetics}{一块}{yi2kuai4}
    \definition{adv.}{(principalmente mandarim) juntos}
  \end{Phonetics}
\end{Entry}

\begin{Entry}{一块儿}{1,7,2}{⼀、⼟、⼉}
  \begin{Phonetics}{一块儿}{yi2 kuai4r5}[][HSK 1]
    \definition{adv.}{juntos; em conjunto}
    \definition{s.}{no mesmo lugar; no mesmo local}
  \end{Phonetics}
\end{Entry}

\begin{Entry}{一时}{1,7}{⼀、⽇}
  \begin{Phonetics}{一时}{yi4 shi2}[][HSK 6]
    \definition{adv.}{por um curto período; temporário | (usado em pares) agora\dots, agora\dots; este momento\dots, e o próximo\dots; o mesmo que 时而}
    \definition{s.}{um período de tempo | um momento; um breve momento; um tempo muito curto}
  \seealsoref{时而}{shi2'er2}
  \seealsoref{一时……,一时……}{yi4 shi2 yi4 shi2}
  \end{Phonetics}
\end{Entry}

\begin{Entry}{一时……,一时……}{1,7,1,7}{⼀、⽇、⼀、⽇}
  \begin{Phonetics}{一时……,一时……}{yi4 shi2 yi4 shi2}[][HSK 6]
    \definition{adv.}{por um tempo\dots, por um tempo\dots}
  \seealsoref{一时}{yi4 shi2}
  \end{Phonetics}
\end{Entry}

\begin{Entry}{一身}{1,7}{⼀、⾝}
  \begin{Phonetics}{一身}{yi4 shen1}[][HSK 5]
    \definition{s.}{o corpo inteiro; em todo o corpo | um terno; (um conjunto completo de) roupas | sozinho; uma única pessoa; relativo a uma única pessoa}
  \end{Phonetics}
\end{Entry}

\begin{Entry}{一些}{1,8}{⼀、⼆}
  \begin{Phonetics}{一些}{yi4 xie1}[][HSK 1]
    \definition{clas.}{alguns; um número de; quantidade indeterminada | um pouco; uma pequena quantidade | mais de um; mais de uma vez; indica mais de um ou mais de uma vez, etc. | uma ligeira mudança no grau, intensidade; usado após certos verbos, adjetivos, etc., para indicar uma quantidade muito pequena}
    \definition{pron.}{uns; alguns}
  \end{Phonetics}
\end{Entry}

\begin{Entry}{一定}{1,8}{⼀、⼧}
  \begin{Phonetics}{一定}{yi2ding4}[][HSK 2]
    \definition{adj.}{certo; particular; tendo um certo nível de especificidade; (objeto, situação) determinado em um ou mais | devido; certo; sempre foi assim, não vai mudar | fixo; especificado; há requisitos claros quanto à maneira, método, quantidade, etc.}
    \definition{adv.}{certamente; necessariamente; expressando determinação ou certeza | certamente; indica especulação ou avaliação de que um evento ou situação definitivamente acontecerá ou realmente existirá}
  \end{Phonetics}
\end{Entry}

\begin{Entry}{一直}{1,8}{⼀、⽬}
  \begin{Phonetics}{一直}{yi4zhi2}[][HSK 2]
    \definition{adv.}{direto; indica que permanece inalterado em uma direção | sempre; continuamente; o tempo todo; o tempo todo; indica que a ação é sempre ininterrupta ou o estado é sempre inalterado | de um ponto a outro sem enfatizar nenhuma exceção}
  \end{Phonetics}
\end{Entry}

\begin{Entry}{一贯}{1,8}{⼀、⾙}
  \begin{Phonetics}{一贯}{yi2guan4}[][HSK 6]
    \definition{adj./adv.}{do começo ao fim; inabalável; consistente; persistente; o tempo todo}
  \end{Phonetics}
\end{Entry}

\begin{Entry}{一带}{1,9}{⼀、⼱}
  \begin{Phonetics}{一带}{yi2 dai4}[][HSK 5]
    \definition{s.}{a área em torno de um determinado local; refere-se a um determinado local e suas proximidades}
  \end{Phonetics}
\end{Entry}

\begin{Entry}{一律}{1,9}{⼀、⼻}
  \begin{Phonetics}{一律}{yi2lv4}[][HSK 4]
    \definition{adj.}{igual; semelhante; uniforme; parecido; idêntico}
    \definition{adv.}{todos; tudo; sem exceção; enfatiza que todos devem ser assim, sem exceção, e é usado principalmente em regulamentos ou requisitos}
  \end{Phonetics}
\end{Entry}

\begin{Entry}{一战}{1,9}{⼀、⼽}
  \begin{Phonetics}{一战}{yi2zhan4}
    \definition*{s.}{Primeira Guerra Mundial}
  \end{Phonetics}
\end{Entry}

\begin{Entry}{一点儿}{1,9,2}{⼀、⽕、⼉}
  \begin{Phonetics}{一点儿}{yi4dian3r5}[][HSK 1]
    \definition{adv.}{um pouco; uma pitada; uma gota; uma amostra; uma pequena quantidade; ({adj.} + (一)点儿, 一点儿 + {s.} ou 有 + (一)点儿 + {s.})}
  \end{Phonetics}
\end{Entry}

\begin{Entry}{一点点}{1,9,9}{⼀、⽕、⽕}
  \begin{Phonetics}{一点点}{yi4 dian3 dian3}[][HSK 2]
    \definition{adj.}{um pouco; muito pouco ou um pouquinho}
  \end{Phonetics}
\end{Entry}

\begin{Entry}{一样}{1,10}{⼀、⽊}
  \begin{Phonetics}{一样}{yi2yang4}[][HSK 1]
    \definition{adj.}{o mesmo; igualmente; semelhante; tão\dots quanto\dots}
    \definition{part.}{na mesma medida; anexado a verbos ou palavras nominais, indica uma comparação ou semelhança, equivalente a 似的}
  \seealsoref{似的}{shi4de5}
  \end{Phonetics}
\end{Entry}

\begin{Entry}{一流}{1,10}{⼀、⽔}
  \begin{Phonetics}{一流}{yi4liu2}[][HSK 5]
    \definition{adj.}{clássico; de primeira linha; de primeira classe; o melhor}
    \definition[些]{s.}{tipo; mesmo tipo; da mesma classe; da mesma categoria; uma categoria}
  \end{Phonetics}
\end{Entry}

\begin{Entry}{一致}{1,10}{⼀、⾄}
  \begin{Phonetics}{一致}{yi2zhi4}[][HSK 4]
    \definition{adj.}{equado; idêntico; uniforme; unânime; nenhuma diferença (de opinião ou ação)}
    \definition{adv.}{juntos; em conjunto}
  \end{Phonetics}
\end{Entry}

\begin{Entry}{一般}{1,10}{⼀、⾈}
  \begin{Phonetics}{一般}{yi4ban1}[][HSK 2]
    \definition{adj.}{o mesmo que; exatamente como | geral; ordinário; comum | médio; medíocre; o grau ou nível não é muito alto}
    \definition{adv.}{frequentemente; geralmente}
  \end{Phonetics}
\end{Entry}

\begin{Entry}{一般来说}{1,10,7,9}{⼀、⾈、⽊、⾔}
  \begin{Phonetics}{一般来说}{yi4 ban1 lai2 shuo1}[][HSK 4]
    \definition{expr.}{de modo geral; na média; no caso usual; a declaração usual}
  \end{Phonetics}
\end{Entry}

\begin{Entry}{一起}{1,10}{⼀、⾛}
  \begin{Phonetics}{一起}{yi4qi3}[][HSK 1]
    \definition{adv.}{juntos; em companhia; indica o mesmo local, ao mesmo tempo que se faz algo | no total; em todos; no conjunto}
    \definition{s.}{no mesmo lugar}
  \end{Phonetics}
\end{Entry}

\begin{Entry}{一部分}{1,10,4}{⼀、⾢、⼑}
  \begin{Phonetics}{一部分}{yi2 bu4 fen4}[][HSK 2]
    \definition{adj.}{parcial}
    \definition{adv.}{parcialmente}
    \definition{num.}{parte; porção; seção; fração}
  \end{Phonetics}
\end{Entry}

\begin{Entry}{一……就……}{1,12}{⼀、⼪}
  \begin{Phonetics}{一……就……}{yi1 jiu4}
    \definition{expr.}{logo que |  uma vez que}
  \end{Phonetics}
\end{Entry}

\begin{Entry}{一番}{1,12}{⼀、⽥}
  \begin{Phonetics}{一番}{yi4 fan1}[][HSK 6]
    \definition{adv.}{uma demonstração de, uma dose de, um pedaço de (conversa, investigação, pensamento)}
  \end{Phonetics}
\end{Entry}

\begin{Entry}{一辈子}{1,12,3}{⼀、⾞、⼦}
  \begin{Phonetics}{一辈子}{yi2bei4zi5}[][HSK 5]
    \definition{s.}{uma vida inteira; vida inteira; toda a vida; durante toda a vida; enquanto se vive; todo o tempo entre o nascimento e a morte}
  \end{Phonetics}
\end{Entry}

\begin{Entry}{一道}{1,12}{⼀、⾡}
  \begin{Phonetics}{一道}{yi2 dao4}[][HSK 6]
    \definition{adv.}{juntos; lado a lado; junto com}
  \end{Phonetics}
\end{Entry}

\begin{Entry}{一路}{1,13}{⼀、⾜}
  \begin{Phonetics}{一路}{yi2 lu4}[][HSK 5]
    \definition{adv.}{o tempo todo; persistentemente; continuamente | juntos; sem parar; continuamente}
    \definition{s.}{o mesmo caminho; a mesma rota; ao longo de toda a viagem, ao longo do caminho | do mesmo tipo; da mesma categoria}
  \end{Phonetics}
\end{Entry}

\begin{Entry}{一路上}{1,13,3}{⼀、⾜、⼀}
  \begin{Phonetics}{一路上}{yi2 lu4 shang4}[][HSK 6]
    \definition{s.}{ao longo do caminho; todo o caminho}
  \end{Phonetics}
\end{Entry}

\begin{Entry}{一路平安}{1,13,5,6}{⼀、⾜、⼲、⼧}
  \begin{Phonetics}{一路平安}{yi2 lu4 ping2 an1}[][HSK 2]
    \definition{expr.}{Boa viagem!; Tenha uma boa viagem!}
    \definition{v.}{ter uma viagem agradável}
  \end{Phonetics}
\end{Entry}

\begin{Entry}{一路顺风}{1,13,9,4}{⼀、⾜、⾴、⾵}
  \begin{Phonetics}{一路顺风}{yi2 lu4 shun4 feng1}[][HSK 2]
    \definition{expr.}{ter uma viagem agradável; toda a viagem foi segura e tranquila; é uma metáfora para cada etapa do processo de lidar com algo que ocorre sem problemas | Tenha uma boa viagem!; Boa viagem!}
  \end{Phonetics}
\end{Entry}

\begin{Entry}{一模一样}{1,14,1,10}{⼀、⽊、⼀、⽊}
  \begin{Phonetics}{一模一样}{yi4 mu2 yi2 yang4}[][HSK 6]
    \definition{expr.}{tão parecidos quanto duas ervilhas; ser exatamente iguais; muito parecido, a mesma aparência}
  \end{Phonetics}
\end{Entry}

\begin{Entry}{七}{2}{⼀}
  \begin{Phonetics}{七}{qi1}[][HSK 1]
    \definition*{s.}{Sobrenome Qi}
    \definition{num.}{sete; 7}
    \definition{s.}{antigamente, os mortos eram homenageados a cada sete dias, chamados de 七, até o quadragésimo nono dia, num total de sete 七}
  \end{Phonetics}
\end{Entry}

\begin{Entry}{七夕}{2,3}{⼀、⼣}
  \begin{Phonetics}{七夕}{qi1xi1}
    \definition*{s.}{Dia dos Namorados Chinês, quando o vaqueiro e a tecelã (牛郎织女) têm permissão para se reunirem anualmente | Festival das Meninas | Festival Duplo Sete, noite do sétimo mês lunar}
  \seealsoref{牛郎织女}{niu2 lang2 zhi1nv3}
  \end{Phonetics}
\end{Entry}

\begin{Entry}{万}{3}{⼀}
  \begin{Phonetics}{万}{wan4}[][HSK 2]
    \definition*{s.}{Sobrenome Wan}
    \definition{adv.}{absolutamente; indica um grau extremamente alto, equivalente a 完全, 绝对 e 极}
    \definition{num.}{dez mil; 10.000; 1.0000 | miríade; um número muito grande}
  \seealsoref{极}{ji2}
  \seealsoref{绝对}{jue2dui4}
  \seealsoref{完全}{wan2quan2}
  \end{Phonetics}
\end{Entry}

\begin{Entry}{万一}{3,1}{⼀、⼀}
  \begin{Phonetics}{万一}{wan4yi1}[][HSK 4]
    \definition{conj.}{por via das dúvidas; se por acaso; só por precaução; expressa uma suposição muito improvável (usado para coisas desagradáveis)}
    \definition{num.}{um décimo milionésimo; uma porcentagem muito pequena}
    \definition{s.}{contingência; eventualidade; contingências muito improváveis}
  \end{Phonetics}
\end{Entry}

\begin{Entry}{万万}{3,3}{⼀、⼀}
  \begin{Phonetics}{万万}{wan4wan4}
    \definition{adv.}{absolutamente | totalmente}
  \end{Phonetics}
\end{Entry}

\begin{Entry}{万圣节}{3,5,5}{⼀、⼟、⾋}
  \begin{Phonetics}{万圣节}{wan4 sheng4 jie2}
    \definition*{s.}{Dia de Todos os Santos}
  \seealsoref{万圣节前夕}{wan4sheng4 jie2 qian2xi1}
  \end{Phonetics}
\end{Entry}

\begin{Entry}{万圣节前夕}{3,5,5,9,3}{⼀、⼟、⾋、⼑、⼣}
  \begin{Phonetics}{万圣节前夕}{wan4sheng4 jie2 qian2xi1}
    \definition*{s.}{Véspera do Dia de Todos os Santos | Halloween}
  \seealsoref{万圣节}{wan4 sheng4 jie2}
  \end{Phonetics}
\end{Entry}

\begin{Entry}{万福}{3,13}{⼀、⽰}
  \begin{Phonetics}{万福}{wan4fu2}
    \definition{s.}{(antigo) reverência feminina; reverência}
  \end{Phonetics}
\end{Entry}

\begin{Entry}{丈}{3}{⼀}
  \begin{Phonetics}{丈}{zhang4}
    \definition{clas.}{zhang, uma unidade tradicional de comprimento, igual a 10 市尺 e equivalente a 3,333 metros ou 3,65 jardas}
    \definition{s.}{zhang, uma unidade de comprimento (= 3,333 metros)}
    \definition{s.}{sênior; ancião | marido (em certos termos de parentesco) | tratamento respeitoso ao idoso na China antiga; um título respeitoso para homens idosos nos tempos antigos | uma forma de tratamento para certos parentes do sexo masculino por casamento}
  \seealsoref{市尺}{shi4 chi3}
  \end{Phonetics}
\end{Entry}

\begin{Entry}{丈夫}{3,4}{⼀、⼤}
  \begin{Phonetics}{丈夫}{zhang4fu5}[][HSK 4]
    \definition[个,位,名]{s.}{marido; esposo}
  \end{Phonetics}
\end{Entry}

\begin{Entry}{三}{3}{⼀}
  \begin{Phonetics}{三}{san1}[][HSK 1]
    \definition*{s.}{Sobrenome San}
    \definition{num.}{três; 3 | muitos; vários; mais de dois; referindo-se a muitos ou à maioria | alguns; poucos; menos; não muitos}
  \end{Phonetics}
\end{Entry}

\begin{Entry}{三角}{3,7}{⼀、⾓}
  \begin{Phonetics}{三角}{san1jiao3}
    \definition{s.}{triângulo}
  \end{Phonetics}
\end{Entry}

\begin{Entry}{三角恋爱}{3,7,10,10}{⼀、⾓、⼼、⽖}
  \begin{Phonetics}{三角恋爱}{san1jiao3lian4'ai4}
    \definition{s.}{triângulo amoroso}
  \end{Phonetics}
\end{Entry}

\begin{Entry}{三明治}{3,8,8}{⼀、⽇、⽔}
  \begin{Phonetics}{三明治}{san1 ming2 zhi4}[][HSK 6]
    \definition[个,些,块]{s.}{Empréstimo linguístico: sanduíche, \emph{sandwich}}
  \end{Phonetics}
\end{Entry}

\begin{Entry}{三轮车}{3,8,4}{⼀、⾞、⾞}
  \begin{Phonetics}{三轮车}{san1lun2che1}
    \definition{s.}{triciclo}
  \end{Phonetics}
\end{Entry}

\begin{Entry}{上}{3}{⼀}
  \begin{Phonetics}{上}{shang3}
    \definition{s.}{tom descendente-ascendente; significa o segundo tom dos quatro tons do mandarim, e também se refere ao terceiro tom do mandarim padrão}
  \end{Phonetics}
  \begin{Phonetics}{上}{shang4}[][HSK 1]
    \definition{adj.}{mais recente; último; anterior; tempo ou a sequência anterior | superior; mais alto; melhor; indica uma posição elevada em termos de qualidade, nível, etc. | lugar elevado; posição superior (em oposição a 下)}
    \definition{s.}{superior; acima; para cima; um lugar alto ou mais alto do que um determinado local | na superfície de um objeto; usado após um substantivo, indica a superfície de um objeto | indica estar dentro do escopo de algo; usado após um substantivo, indica que algo está dentro do âmbito de determinada coisa | indica um aspecto específico | antigamente, referia-se ao imperador | usado após palavras que indicam idade, equivale a ``\dots 的时候'' | o primeiro nível da escala da música folclórica chinesa, usado como um símbolo de nota na notação musical, equivalente ao '1' na notação simplificada.}
    \definition{v.}{subir; montar; embarcar; entrar | ir para; partir para | estar ocupado (com trabalho, estudos, etc.) em um horário fixo; começar a trabalhar ou estudar na hora marcada, etc. | seguir em frente; prosseguir | encher; abastecer; servir; melhorar; aumentar | aparecer no palco; entrar | colocar algo em posição; ajustar; fixar; montar as duas partes de algo | aplicar; pintar; espalhar | ser registrado; ser publicado (em uma publicação) | atingir; ser suficiente (uma determinada quantidade ou grau) | submeter; enviar; apresentar; submeter à aprovação superior | ventilar; apertar; torcer | trazer; servir; colocar comida, pratos, chá e outras coisas na mesa para os convidados | indicar que uma ação tem um resultado | pesquisar na \emph{Internet} | emaranhar-se; ficar emaranhado; enredar-se}
    \definition{v.aux.}{usado após um verbo para indicar início e continuidade}
  \seealsoref{的时候}{de5 shi2hou4}
  \seealsoref{下}{xia4}
  \end{Phonetics}
\end{Entry}

\begin{Entry}{上下}{3,3}{⼀、⼀}
  \begin{Phonetics}{上下}{shang4 xia4}[][HSK 5]
    \definition{adv.}{para cima e para baixo}
    \definition[顶]{s.}{alto e baixo | de cima para baixo; para cima e para baixo | superioridade ou inferioridade relativa | (após números redondos) aproximadamente; mais ou menos; por aí | velhos e jovens; hierarquia em termos de cargo e posição social}
    \definition{v.}{subir ou descer | subir e descer; da alta para a baixa ou da baixa para a alta}
  \end{Phonetics}
\end{Entry}

\begin{Entry}{上个月}{3,3,4}{⼀、⼈、⽉}
  \begin{Phonetics}{上个月}{shang4 ge4 yue4}[][HSK 4]
    \definition{s.}{mês passado; refere-se à hora de um mês atrás, ou seja, o mês passado mais próximo da hora atual}
  \end{Phonetics}
\end{Entry}

\begin{Entry}{上门}{3,3}{⼀、⾨}
  \begin{Phonetics}{上门}{shang4 men2}[][HSK 4]
    \definition{v.}{chamar; visitar; aparecer; ir ou vir para ver alguém; ir até a porta; ir até a casa de alguém | trancar a porta; fechar a porta durante a noite | casar-se e morar com a família da noiva}
  \end{Phonetics}
\end{Entry}

\begin{Entry}{上升}{3,4}{⼀、⼗}
  \begin{Phonetics}{上升}{shang4 sheng1}[][HSK 3]
    \definition{v.}{elevar; subir; mover-se para cima; mover de baixo para cima; aumentar em nível, grau, quantidade, etc.}
  \end{Phonetics}
\end{Entry}

\begin{Entry}{上午}{3,4}{⼀、⼗}
  \begin{Phonetics}{上午}{shang4wu3}[][HSK 1]
    \definition[个]{s.}{manhã; \emph{ante meridiem} (a.m.); geralmente refere-se ao período entre a manhã e o meio-dia}
  \end{Phonetics}
\end{Entry}

\begin{Entry}{上车}{3,4}{⼀、⾞}
  \begin{Phonetics}{上车}{shang4 che1}[][HSK 1]
    \definition{v.}{entrar; subir (em um ônibus, trem, carro etc.)}
  \end{Phonetics}
\end{Entry}

\begin{Entry}{上去}{3,5}{⼀、⼛}
  \begin{Phonetics}{上去}{shang4 qu4}[][HSK 3]
    \definition{v.}{subir (a partir da minha localização) | ascender a um lugar (ou estado) considerado mais elevado (ou acima); usado depois de um verbo para indicar movimento, de baixo para cima ou de perto para longe}
  \end{Phonetics}
\end{Entry}

\begin{Entry}{上古}{3,5}{⼀、⼝}
  \begin{Phonetics}{上古}{shang4gu3}
    \definition{s.}{o passado distante | tempos antigos | antiguidade}
  \end{Phonetics}
\end{Entry}

\begin{Entry}{上台}{3,5}{⼀、⼝}
  \begin{Phonetics}{上台}{shang4 tai2}[][HSK 6]
    \definition{v.}{aparecer no palco; subir na plataforma; ir para o palco ou pódio | assumir o poder; chegar (subir) ao poder; começar a assumir papéis de liderança ou a ganhar algum tipo de poder}
  \end{Phonetics}
\end{Entry}

\begin{Entry}{上市}{3,5}{⼀、⼱}
  \begin{Phonetics}{上市}{shang4 shi4}[][HSK 6]
    \definition{v.}{listar; abrir o capital; ser listado (na bolsa de valores) | estar na estação; estar (aparecer) no mercado | ir ao mercado (para fazer compras)}
  \end{Phonetics}
\end{Entry}

\begin{Entry}{上边}{3,5}{⼀、⾡}
  \begin{Phonetics}{上边}{shang4 bian5}[][HSK 1]
    \definition{s.}{topo; acima; sobre; superior}
  \end{Phonetics}
\end{Entry}

\begin{Entry}{上当}{3,6}{⼀、⼹}
  \begin{Phonetics}{上当}{shang4dang4}[][HSK 6]
    \definition{v.+compl.}{ser enganado; ser ludibriado; morder a isca; cair nas mãos de alguém}
  \end{Phonetics}
\end{Entry}

\begin{Entry}{上次}{3,6}{⼀、⽋}
  \begin{Phonetics}{上次}{shang4 ci4}[][HSK 1]
    \definition{adv.}{última vez}
  \end{Phonetics}
\end{Entry}

\begin{Entry}{上级}{3,6}{⼀、⽷}
  \begin{Phonetics}{上级}{shang4ji2}[][HSK 5]
    \definition[个,位]{s.}{nível superior; organização ou pessoa em nível superior; organizações ou pessoas de nível superior dentro do mesmo sistema organizacional}
  \end{Phonetics}
\end{Entry}

\begin{Entry}{上网}{3,6}{⼀、⽹}
  \begin{Phonetics}{上网}{shang4wang3}[][HSK 1]
    \definition{v.}{conectar-se à \emph{Internet}; acessar a \emph{Internet}; entrar na \emph{Internet}; acessar a rede; refere-se especificamente ao computador do usuário conectado à Internet para pesquisar e consultar informações, etc.}
  \end{Phonetics}
\end{Entry}

\begin{Entry}{上衣}{3,6}{⼀、⾐}
  \begin{Phonetics}{上衣}{shang4 yi1}[][HSK 3]
    \definition[件]{s.}{jaqueta; roupas para a parte superior do corpo}
  \end{Phonetics}
\end{Entry}

\begin{Entry}{上访}{3,6}{⼀、⾔}
  \begin{Phonetics}{上访}{shang4fang3}
    \definition{v.}{buscar uma audiência com superiores (especialmente funcionários do governo) para fazer uma petição por algo}
  \end{Phonetics}
\end{Entry}

\begin{Entry}{上声}{3,7}{⼀、⼠}
  \begin{Phonetics}{上声}{shang3sheng1}
    \definition{s.}{tom descendente e ascendente | terceiro tom no mandarim moderno}
  \end{Phonetics}
\end{Entry}

\begin{Entry}{上来}{3,7}{⼀、⽊}
  \begin{Phonetics}{上来}{shang4 lai2}[][HSK 3]
    \definition{v.}{subir (para a minha localização) | estar no começo; começar; iniciar | surgir; de um lugar baixo para um lugar alto (o interlocutor está em um lugar alto) | usado após o verbo, indica que algo foi concluído com sucesso}
  \end{Phonetics}
\end{Entry}

\begin{Entry}{上周}{3,8}{⼀、⼝}
  \begin{Phonetics}{上周}{shang4 zhou1}[][HSK 2]
    \definition{s.}{semana passada}
  \end{Phonetics}
\end{Entry}

\begin{Entry}{上坡路}{3,8,13}{⼀、⼟、⾜}
  \begin{Phonetics}{上坡路}{shang4po1lu4}
    \definition{s.}{aclive | progresso | (fig.) tendência ascendente}
  \end{Phonetics}
\end{Entry}

\begin{Entry}{上学}{3,8}{⼀、⼦}
  \begin{Phonetics}{上学}{shang4 xue2}[][HSK 1]
    \definition{v.}{ir à escola; frequentar a escola; estar na escola; ir à escola para estudar | começar a escola; começar a estudar no ensino fundamental}
  \end{Phonetics}
\end{Entry}

\begin{Entry}{上询}{3,8}{⼀、⾔}
  \begin{Phonetics}{上询}{shang4 xun2}
    \definition{adv.}{primeira dezena do mês}
  \end{Phonetics}
\end{Entry}

\begin{Entry}{上帝}{3,9}{⼀、⼱}
  \begin{Phonetics}{上帝}{shang4 di4}[][HSK 6]
    \definition*{s.}{Deus; O Deus Supremo no Cristianismo | O Imperador do Céu; um deus na antiga crença chinesa que pode controlar tudo no mundo}
    \definition[个]{s.}{(figurado) cliente; metáfora para consumidores}
  \end{Phonetics}
\end{Entry}

\begin{Entry}{上面}{3,9}{⼀、⾯}
  \begin{Phonetics}{上面}{shang4 mian4}[][HSK 3]
    \definition{s.}{uma posição mais alta que algo; uma posição acima/acima de algo | superfície do objeto | aspecto | a parte acima mencionada; a parte que vem primeiro na ordem; a parte de um artigo ou discurso que vem antes da presente | autoridades superiores | os mais velhos; a geração mais velha da família}
  \end{Phonetics}
\end{Entry}

\begin{Entry}{上海}{3,10}{⼀、⽔}
  \begin{Phonetics}{上海}{shang4hai3}
    \definition*{s.}{Município de Xangai (Shanghai), centro-leste da China}
  \end{Phonetics}
\end{Entry}

\begin{Entry}{上涨}{3,10}{⼀、⽔}
  \begin{Phonetics}{上涨}{shang4 zhang3}[][HSK 5]
    \definition{v.}{subir; ir para cima; ascender}
  \end{Phonetics}
\end{Entry}

\begin{Entry}{上班}{3,10}{⼀、⽟}
  \begin{Phonetics}{上班}{shang4ban1}[][HSK 1]
    \definition{v.+compl.}{ir trabalhar; começar a trabalhar; estar de plantão; ir trabalhar no local de trabalho regular no horário especificado}
  \end{Phonetics}
\end{Entry}

\begin{Entry}{上班族}{3,10,11}{⼀、⽟、⽅}
  \begin{Phonetics}{上班族}{shang4 ban1 zu2}
    \definition[本]{s.}{trabalhadores de escritório (como grupo social)}
  \end{Phonetics}
\end{Entry}

\begin{Entry}{上课}{3,10}{⼀、⾔}
  \begin{Phonetics}{上课}{shang4 ke4}[][HSK 1]
    \definition{v.+compl.}{frequentar aulas; ir às aulas; dar uma aula}
  \end{Phonetics}
\end{Entry}

\begin{Entry}{上楼}{3,13}{⼀、⽊}
  \begin{Phonetics}{上楼}{shang4 lou2}[][HSK 4]
    \definition{v.}{subir as escadas; ir para o andar de cima}
  \end{Phonetics}
\end{Entry}

\begin{Entry}{上演}{3,14}{⼀、⽔}
  \begin{Phonetics}{上演}{shang4 yan3}[][HSK 6]
    \definition{s.}{exibição | encenação}
    \definition{v.}{exibir (um filme); encenar (uma peça); atuar; colocar no palco}
  \end{Phonetics}
\end{Entry}

\begin{Entry}{下}{3}{⼀}
  \begin{Phonetics}{下}{xia4}[][HSK 1,2]
    \definition{clas.}{número de vezes usado para a ação | volume de um contêiner; quantidade de objetos que cabem em um utensílio | usado depois de 两 e 几 para expressar habilidade, capacidade, destreza}
    \definition{s.}{abaixo | próximo; último; segundo; referindo-se ao que está por vir ou ao que vem em seguida | mais baixo; inferior; de baixo nível ou grau | próximo; último; segundo; em ordem ou em ordem cronológica | indica pertencer a uma determinada faixa, situação, condição, etc. | indica uma determinada época ou estação | usado após um número para indicar posição ou direção | para baixo (após uma preposição) | sob (depois de um substantivo) | para baixo (antes de um verbo)}
    \definition{v.}{desembarcar; descer; sair | cair (chuva, neve, etc.) | enviar; emitir; entregar | ir para | sair; partir; retirar-se | lançar; colocar | descarregar; desmontar; tirar (fora) | formar (uma opinião, ideia, etc.); tomar decisões, fazer julgamentos, etc. | usar; aplicar | dar à luz (animais) | tomar; capturar; conquistar | ceder | terminar; deixar de lado; terminar o trabalho ou os estudos diários na hora prevista | para negação; ser inferior a; ser menor que}
  \seealsoref{几}{ji3}
  \seealsoref{两}{liang3}
  \end{Phonetics}
\end{Entry}

\begin{Entry}{下个月}{3,3,4}{⼀、⼈、⽉}
  \begin{Phonetics}{下个月}{xia4 ge4 yue4}[][HSK 4]
    \definition{s.}{próximo mês; mês que vem; refere-se ao próximo mês do mês atual}
  \end{Phonetics}
\end{Entry}

\begin{Entry}{下午}{3,4}{⼀、⼗}
  \begin{Phonetics}{下午}{xia4wu3}[][HSK 1]
    \definition[个]{s.}{tarde; \emph{post meridiem} (p.m.); refere-se ao período entre o meio-dia e o pôr do sol}
  \end{Phonetics}
\end{Entry}

\begin{Entry}{下午茶}{3,4,9}{⼀、⼗、⾋}
  \begin{Phonetics}{下午茶}{xia4wu3cha2}
    \definition{s.}{chá da tarde (normalmente chás com doces)}
  \end{Phonetics}
\end{Entry}

\begin{Entry}{下巴}{3,4}{⼀、⼰}
  \begin{Phonetics}{下巴}{xia4ba5}
    \definition[个]{s.}{queixo}
  \end{Phonetics}
\end{Entry}

\begin{Entry}{下水道}{3,4,12}{⼀、⽔、⾡}
  \begin{Phonetics}{下水道}{xia4shui3dao4}
    \definition{s.}{esgoto}
  \end{Phonetics}
\end{Entry}

\begin{Entry}{下车}{3,4}{⼀、⾞}
  \begin{Phonetics}{下车}{xia4 che1}[][HSK 1]
    \definition{v.}{descer ou sair de (um ônibus, trem, carro etc.)}
  \end{Phonetics}
\end{Entry}

\begin{Entry}{下去}{3,5}{⼀、⼛}
  \begin{Phonetics}{下去}{xia4 qu4}[][HSK 3]
    \definition{part.}{usado depois de verbos para indicar de alto a baixo | usado depois de um verbo para indicar continuação}
    \definition{v.}{descer; baixar (a partir da minha localização) | (após um verbo) continuar (fazendo algo); prosseguir | usado após o verbo, indica uma descida de um ponto alto para um ponto baixo | usado após o verbo, indica continuidade | usado após um adjetivo, indica que o grau continua aumentando}
  \end{Phonetics}
\end{Entry}

\begin{Entry}{下边}{3,5}{⼀、⾡}
  \begin{Phonetics}{下边}{xia4 bian5}[][HSK 1]
    \definition{s.}{abaixo; sob; por baixo | próximo em ordem; seguinte | nível inferior; subordinado | a parte inferior}
  \end{Phonetics}
\end{Entry}

\begin{Entry}{下旬}{3,6}{⼀、⽇}
  \begin{Phonetics}{下旬}{xia4xun2}
    \definition{adv.}{última dezena do mês}
  \end{Phonetics}
\end{Entry}

\begin{Entry}{下次}{3,6}{⼀、⽋}
  \begin{Phonetics}{下次}{xia4 ci4}[][HSK 1]
    \definition{s.}{na próxima vez; na próxima oportunidade ou no próximo evento}
  \end{Phonetics}
\end{Entry}

\begin{Entry}{下来}{3,7}{⼀、⽊}
  \begin{Phonetics}{下来}{xia4 lai5}[][HSK 3]
    \definition{part.}{usado após o verbo, indica que a ação ou o comportamento se dirige para a posição do falante ou que a ação é contínua ou concluída | usado após um adjetivo, indica que uma determinada situação começou a ocorrer e continuará a se desenvolver}
    \definition{v.}{descer (para a minha localização) | (colheitas/frutas/vegetais, etc.) ser colhido; estar maduro o suficiente para ser colhido | (período de tempo) acabar; passar; chegar ao fim; indicar o fim de um período de tempo}
  \end{Phonetics}
\end{Entry}

\begin{Entry}{下周}{3,8}{⼀、⼝}
  \begin{Phonetics}{下周}{xia4 zhou1}[][HSK 2]
    \definition{s.}{próxima semana}
  \end{Phonetics}
\end{Entry}

\begin{Entry}{下线}{3,8}{⼀、⽷}
  \begin{Phonetics}{下线}{xia4xian4}
    \definition{v.}{ficar \emph{offline} | (um produto) sair da linha de montagem | pessoa abaixo de si em um esquema de pirâmide}
  \end{Phonetics}
\end{Entry}

\begin{Entry}{下降}{3,8}{⼀、⾩}
  \begin{Phonetics}{下降}{xia4 jiang4}[][HSK 4]
    \definition{v.}{cair; despencar; declinar; descer; diminuir; ir para baixo}
  \end{Phonetics}
\end{Entry}

\begin{Entry}{下雨}{3,8}{⼀、⾬}
  \begin{Phonetics}{下雨}{xia4 yu3}[][HSK 1]
    \definition{v.+compl.}{chover}
  \end{Phonetics}
\end{Entry}

\begin{Entry}{下面}{3,9}{⼀、⾯}
  \begin{Phonetics}{下面}{xia4 mian4}[][HSK 3]
    \definition{s.}{em baixo; abaixo; parte de baixo | próximo; seguinte; a parte posterior; a parte posterior de um artigo ou discurso em relação ao que está sendo narrado no momento | subordinado; o nível inferior; homens nos níveis inferiores | por baixo}
  \end{Phonetics}
\end{Entry}

\begin{Entry}{下海}{3,10}{⼀、⽔}
  \begin{Phonetics}{下海}{xia4hai3}
    \definition{v.+compl.}{ir para o mar; (barco) deixar o porto e iniciar uma jornada | ir pescar no mar | tornar-se ator profissional}
  \end{Phonetics}
\end{Entry}

\begin{Entry}{下班}{3,10}{⼀、⽟}
  \begin{Phonetics}{下班}{xia4 ban1}[][HSK 1]
    \definition{v.+compl.}{sair do trabalho; bater ponto; terminar o trabalho na hora prevista e sair do local de trabalho}
  \end{Phonetics}
\end{Entry}

\begin{Entry}{下课}{3,10}{⼀、⾔}
  \begin{Phonetics}{下课}{xia4 ke4}[][HSK 1]
    \definition{v.+compl.}{terminar a aula; sair da aula}
  \end{Phonetics}
\end{Entry}

\begin{Entry}{下载}{3,10}{⼀、⾞}
  \begin{Phonetics}{下载}{xia4zai3}[][HSK 4]
    \definition{v.}{\emph{download}; baixar; salvar informações da \emph{Web} em um dispositivo, como um computador}
  \end{Phonetics}
\end{Entry}

\begin{Entry}{下蛋}{3,11}{⼀、⾍}
  \begin{Phonetics}{下蛋}{xia4dan4}
    \definition{v.}{botar ovos}
  \end{Phonetics}
\end{Entry}

\begin{Entry}{下雪}{3,11}{⼀、⾬}
  \begin{Phonetics}{下雪}{xia4 xue3}[][HSK 2]
    \definition{v.+compl.}{nevar}
  \end{Phonetics}
\end{Entry}

\begin{Entry}{下崽}{3,12}{⼀、⼭}
  \begin{Phonetics}{下崽}{xia4zai3}
    \definition{v.}{dar à luz (animais) | parir}
  \end{Phonetics}
\end{Entry}

\begin{Entry}{下楼}{3,13}{⼀、⽊}
  \begin{Phonetics}{下楼}{xia4 lou2}[][HSK 4]
    \definition{v.}{descer as escadas}
  \end{Phonetics}
\end{Entry}

\begin{Entry}{与}{3}{⼀}
  \begin{Phonetics}{与}{yu3}[][HSK 6]
    \definition*{s.}{Sobrenome Yu}
    \definition{conj.}{e; junto com}
    \definition{prep.}{com}
    \definition{v.}{dar; oferecer; conceder | conviver com; estar em bons termos com; socializar; ser amigável | ajudar; apoiar; patrocinar | Literário: esperar}
  \end{Phonetics}
  \begin{Phonetics}{与}{yu4}
    \definition{v.}{participar de; tomar parte em}
  \end{Phonetics}
\end{Entry}

\begin{Entry}{与其}{3,8}{⼀、⼋}
  \begin{Phonetics}{与其}{yu3qi2}
    \definition{conj.}{mais do que}
  \end{Phonetics}
\end{Entry}

\begin{Entry}{与其……不如……}{3,8,4,6}{⼀、⼋、⼀、⼥}
  \begin{Phonetics}{与其……不如……}{yu3qi2 bu4ru2}
    \definition{conj.}{ao invés de\dots melhor que\dots}
  \end{Phonetics}
\end{Entry}

\begin{Entry}{与其……宁可……}{3,8,5,5}{⼀、⼋、⼧、⼝}
  \begin{Phonetics}{与其……宁可……}{yu3qi2 ning4ke3}
    \definition{conj.}{ao invés de\dots melhor que\dots}
  \end{Phonetics}
\end{Entry}

\begin{Entry}{不}{4}{⼀}
  \begin{Phonetics}{不}{bu2}[(antes de quarto tom)][HSK 1]
  \end{Phonetics}
  \begin{Phonetics}{不}{bu4}[][HSK 1]
    \definition{adv.}{(antes de verbos, adjetivos e outros advérbios; nunca antes do verbo 有) não; não vai; não quer | em algumas expressões educadas, significa que não é necessário fazer isso, o que equivale a 不用 ou 不要 | | (entre um verbo e seu complemento) não pode | usado com 就 para indicar escolha}
    \definition{part.}{no final da frase para indicar uma pergunta; (usar sozinho ou com uma partícula nas respostas) não}
    \definition{pref.}{(antes de certos substantivos para formar um adjetivo) un-; in-}
  \seealsoref{不要}{bu2 yao4}
  \seealsoref{不用}{bu2 yong4}
  \seealsoref{就}{jiu4}
  \seealsoref{有}{you3}
  \end{Phonetics}
  \begin{Phonetics}{不}{bu5}[][HSK 1]
    \definition{adv.}{não (em expressões \{v.\} + 不 + \{v.\})}
  \end{Phonetics}
\end{Entry}

\begin{Entry}{不一会儿}{4,1,6,2}{⼀、⼀、⼈、⼉}
  \begin{Phonetics}{不一会儿}{bu4 yi2 hui4r5}[][HSK 2]
    \definition{expr.}{em um momento; em pouco tempo; em breve; depois de algum tempo}
  \end{Phonetics}
\end{Entry}

\begin{Entry}{不一定}{4,1,8}{⼀、⼀、⼧}
  \begin{Phonetics}{不一定}{bu4 yi2 ding4}[][HSK 2]
    \definition{adv.}{talvez; incerto; não tenho certeza; não necessariamente assim; refere-se a algo que não pode ser determinado}
  \end{Phonetics}
\end{Entry}

\begin{Entry}{不久}{4,3}{⼀、⼃}
  \begin{Phonetics}{不久}{bu4 jiu3}[][HSK 2]
    \definition{adv.}{em breve; dentro em breve; num futuro próximo | logo depois; pouco tempo depois | não muito tempo (antes ou depois de algo)}
  \end{Phonetics}
\end{Entry}

\begin{Entry}{不大}{4,3}{⼀、⼤}
  \begin{Phonetics}{不大}{bu2 da4}[][HSK 1]
    \definition{adv.}{não muito (indicando um grau baixo); não demasiado | não com frequência; raramente; dificilmente}
  \end{Phonetics}
\end{Entry}

\begin{Entry}{不大离}{4,3,10}{⼀、⼤、⼇}
  \begin{Phonetics}{不大离}{bu2da4li2}
    \definition{adj.}{bem perto | quase certo | nada mal}
  \end{Phonetics}
\end{Entry}

\begin{Entry}{不仅}{4,4}{⼀、⼈}
  \begin{Phonetics}{不仅}{bu4jin3}[][HSK 3]
    \definition{adv.}{não apenas (em número, quantidade ou extensão); costuma-se dizer 不仅仅}
    \definition{conj.}{não somente}
  \seealsoref{不仅仅}{bu4 jin3 jin3}
  \end{Phonetics}
\end{Entry}

\begin{Entry}{不仅仅}{4,4,4}{⼀、⼈、⼈}
  \begin{Phonetics}{不仅仅}{bu4 jin3 jin3}[][HSK 6]
    \definition{adv.}{não só; não apenas}
  \end{Phonetics}
\end{Entry}

\begin{Entry}{不公}{4,4}{⼀、⼋}
  \begin{Phonetics}{不公}{bu4gong1}
    \definition{adj.}{injusto}
  \end{Phonetics}
\end{Entry}

\begin{Entry}{不太}{4,4}{⼀、⼤}
  \begin{Phonetics}{不太}{bu2 tai4}[][HSK 2]
    \definition{adv.}{não exatamente | não muito bom}
  \end{Phonetics}
\end{Entry}

\begin{Entry}{不少}{4,4}{⼀、⼩}
  \begin{Phonetics}{不少}{bu4 shao3}[][HSK 2]
    \definition{adj.}{muitos; bastante; não poucos; indica uma quantidade considerável, equivalente a muitos ou bastante}
  \end{Phonetics}
\end{Entry}

\begin{Entry}{不日}{4,4}{⼀、⽇}
  \begin{Phonetics}{不日}{bu2ri4}
    \definition{adv.}{em alguns dias}
  \end{Phonetics}
\end{Entry}

\begin{Entry}{不止}{4,4}{⼀、⽌}
  \begin{Phonetics}{不止}{bu4zhi3}[][HSK 5]
    \definition{adv.}{mais do que; não limitado a; indica mais do que esse valor ou intervalo}
    \definition{v.}{exceder; superar; não ser possível interromper a ação}
  \end{Phonetics}
\end{Entry}

\begin{Entry}{不见}{4,4}{⼀、⾒}
  \begin{Phonetics}{不见}{bu2 jian4}[][HSK 6]
    \definition{v.}{não ver; não conhecer; não encontrar | estar desaparecido; desaparecer; não consiguir encontrar algo}
  \end{Phonetics}
\end{Entry}

\begin{Entry}{不计其数}{4,4,8,13}{⼀、⾔、⼋、⽁}
  \begin{Phonetics}{不计其数}{bu2 ji4 qi2 shu4}
    \definition{expr.}{seu número não pode ser contado; incontáveis; inumeráveis}
  \end{Phonetics}
\end{Entry}

\begin{Entry}{不可避免}{4,5,16,7}{⼀、⼝、⾌、⼉}
  \begin{Phonetics}{不可避免}{bu4ke3bi4mian3}
    \definition{adj./adv.}{inevitável}
  \end{Phonetics}
\end{Entry}

\begin{Entry}{不对}{4,5}{⼀、⼨}
  \begin{Phonetics}{不对}{bu2 dui4}[][HSK 1]
    \definition{adj.}{incorreto; errado | anormal; anômalo; estranho | desarmonia; incompatibilidade; discórdia}
  \end{Phonetics}
\end{Entry}

\begin{Entry}{不必}{4,5}{⼀、⼼}
  \begin{Phonetics}{不必}{bu2 bi4}[][HSK 3]
    \definition{adv.}{não precisa; não tem que; indica que não é necessário em termos de razão ou emoção}
  \end{Phonetics}
\end{Entry}

\begin{Entry}{不用}{4,5}{⼀、⽤}
  \begin{Phonetics}{不用}{bu2 yong4}[][HSK 1]
    \definition{v.}{não precisar; não ter necessidade; indicar que, na verdade, não é necessário}
  \end{Phonetics}
\end{Entry}

\begin{Entry}{不光}{4,6}{⼀、⼉}
  \begin{Phonetics}{不光}{bu4 guang1}[][HSK 3]
    \definition{adv.}{não é o único; não apenas; não só; indica que excede uma determinada quantidade ou faixa}
    \definition{conj.}{não somente; não só}
  \end{Phonetics}
\end{Entry}

\begin{Entry}{不再}{4,6}{⼀、⼌}
  \begin{Phonetics}{不再}{bu2 zai4}[][HSK 6]
    \definition{adv.}{não mais; não repita uma segunda vez}
    \definition{v.}{ter ido embora; não retornar; não aparecer; não existir mais}
  \end{Phonetics}
\end{Entry}

\begin{Entry}{不同}{4,6}{⼀、⼝}
  \begin{Phonetics}{不同}{bu4 tong2}[][HSK 2]
    \definition{adj.}{diferente; distinto; não semelhante;}
  \end{Phonetics}
\end{Entry}

\begin{Entry}{不在乎}{4,6,5}{⼀、⼟、⼃}
  \begin{Phonetics}{不在乎}{bu2 zai4 hu1}[][HSK 4]
    \definition{v.}{não se importar; não dar a mínima; não dar atenção}
  \end{Phonetics}
\end{Entry}

\begin{Entry}{不好意思}{4,6,13,9}{⼀、⼥、⼼、⼼}
  \begin{Phonetics}{不好意思}{bu4 hao3 yi4 si5}[][HSK 2]
    \definition{adj.}{envergonhado; desconfortável; constrangido; sem jeito}
    \definition{interj.}{com licença; peço desculpas; desculpe-me}
    \definition{v.}{achar constrangedor (fazer algo) | pedir desculpas (por incomodar alguém) | sentir-se envergonhado | achar algo embaraçoso}
  \end{Phonetics}
\end{Entry}

\begin{Entry}{不如}{4,6}{⼀、⼥}
  \begin{Phonetics}{不如}{bu4ru2}[][HSK 2]
    \definition{conj.}{em vez de; melhor do que; seria melhor; preferiria; seria melhor; usado no início da segunda parte da frase, indica uma escolha feita após comparação (geralmente em correspondência com o termo 与其 no texto anterior)}
    \definition{v.}{ser inferior a; não ser igual a; não ser tão bom quanto;  não poder fazer melhor que}
  \seealsoref{与其}{yu3qi2}
  \end{Phonetics}
\end{Entry}

\begin{Entry}{不安}{4,6}{⼀、⼧}
  \begin{Phonetics}{不安}{bu4'an1}[][HSK 3]
    \definition{adj.}{(humor) inquieto; (ambiente, etc.)  instável; intranquilo; perturbado; sem paz | desculpe; frases de cortesia, expressões de desculpas, equivalentes a 不好意思}
  \seealsoref{不好意思}{bu4 hao3 yi4 si5}
  \end{Phonetics}
\end{Entry}

\begin{Entry}{不成}{4,6}{⼀、⼽}
  \begin{Phonetics}{不成}{bu4 cheng2}[][HSK 6]
    \definition{adj.}{não é bom; não funciona; impraticável}
    \definition{part.}{usada no final de uma frase para expressar especulação ou tom contraintuitivo, geralmente precedido por palavras como 嘛 ou 莫非}
    \definition{v.}{não ser permitido; não ser permissível; ser impossível}
  \seealsoref{莫非}{mo4fei1}
  \seealsoref{难道}{nan2dao4}
  \end{Phonetics}
\end{Entry}

\begin{Entry}{不成话}{4,6,8}{⼀、⼽、⾔}
  \begin{Phonetics}{不成话}{bu4cheng2hua4}
    \definition{expr.}{irracional | chocante; ultrajante; inapropriado}
  \seealsoref{不是话}{bu2shi4hua4}
  \seealsoref{不像话}{bu2xiang4hua4}
  \end{Phonetics}
\end{Entry}

\begin{Entry}{不至于}{4,6,3}{⼀、⾄、⼆}
  \begin{Phonetics}{不至于}{bu2 zhi4 yu2}[][HSK 6]
    \definition{adv.}{não pode ir tão longe a ponto de; não tanto\dots. a ponto de\dots; não a ponto de}
  \end{Phonetics}
\end{Entry}

\begin{Entry}{不行}{4,6}{⼀、⾏}
  \begin{Phonetics}{不行}{bu4 xing2}[][HSK 2]
    \definition{adj.}{não funciona; não é bom; falta de capacidade e habilidade; nível baixo}
    \definition{adv.}{profundamente; terrivelmente; extremamente; expressa um grau muito profundo; incrível (usado após o caractere 得 como complemento)}
    \definition{v.}{não servir; não ser permitido; estar fora de questão | estar à beira da morte}
  \seealsoref{得}{de5}
  \end{Phonetics}
\end{Entry}

\begin{Entry}{不许}{4,6}{⼀、⾔}
  \begin{Phonetics}{不许}{bu4 xu3}[][HSK 5]
    \definition{v.}{não permitir; ser proibido; proibir firmemente | não pode (usado em perguntas retóricas)}
  \end{Phonetics}
\end{Entry}

\begin{Entry}{不论}{4,6}{⼀、⾔}
  \begin{Phonetics}{不论}{bu2 lun4}[][HSK 3]
    \definition{conj.}{não importa (o que, quem, como, etc.); se \dots ou \dots; significa que as condições ou situações são diferentes, mas os resultados permanecem os mesmos; geralmente é seguido por palavras paralelas ou pronomes interrogativos; geralmente é seguido por advérbios como 都 e 总}
    \definition{v.}{não discutir nem argumentar; não discutir; não debater}
  \seealsoref{都}{dou1}
  \seealsoref{总}{zong3}
  \end{Phonetics}
\end{Entry}

\begin{Entry}{不论……也……}{4,6,3}{⼀、⾔、⼄}
  \begin{Phonetics}{不论……也……}{bu2lun4 ye3}
    \definition{conj.}{não apenas\dots, (o que, quem, como, etc.), \dots}
  \end{Phonetics}
\end{Entry}

\begin{Entry}{不论……都……}{4,6,10}{⼀、⾔、⾢}
  \begin{Phonetics}{不论……都……}{bu2lun4 dou1}
    \definition{conj.}{não apenas\dots, (o que, quem, como, etc.), \dots}
  \end{Phonetics}
\end{Entry}

\begin{Entry}{不过}{4,6}{⼀、⾡}
  \begin{Phonetics}{不过}{bu2guo4}[][HSK 2]
    \definition{adv.}{apenas; meramente; nada mais do que; indica que não excede um determinado limite, equivalente a 仅 ou 只 | como intensificador após certos adjetivos}
    \definition{conj.}{mas; no entanto; apenas; usado no início da segunda parte da frase, indica o contrário do sentido anterior e modifica ou complementa o significado anterior}
  \end{Phonetics}
\end{Entry}

\begin{Entry}{不但}{4,7}{⼀、⼈}
  \begin{Phonetics}{不但}{bu2 dan4}[][HSK 2]
    \definition{conj.}{não só\dots mas também; usado na primeira parte de uma frase composta que expressa progressão, a segunda parte geralmente contém conjunções como 而且,  并且 ou advérbios como 也, 还 que correspondem à primeira parte}
  \seealsoref{并且}{bing4qie3}
  \seealsoref{而且}{er2 qie3}
  \seealsoref{还}{hai2}
  \seealsoref{也}{ye3}
  \end{Phonetics}
\end{Entry}

\begin{Entry}{不但……而且……}{4,7,6,5}{⼀、⼈、⽽、⼀}
  \begin{Phonetics}{不但……而且……}{bu2 dan4 er2qie3}[][HSK 2]
    \definition{conj.}{não só\dots mas também\dots}
  \end{Phonetics}
\end{Entry}

\begin{Entry}{不免}{4,7}{⼀、⼉}
  \begin{Phonetics}{不免}{bu4mian3}[][HSK 5]
    \definition{adv.}{inevitavelmente; inexoravelmente}
  \end{Phonetics}
\end{Entry}

\begin{Entry}{不利}{4,7}{⼀、⼑}
  \begin{Phonetics}{不利}{bu2 li4}[][HSK 5]
    \definition{adj.}{desfavorável; desvantajoso; nocivo; prejudicial | malsucedido}
  \end{Phonetics}
\end{Entry}

\begin{Entry}{不时}{4,7}{⼀、⽇}
  \begin{Phonetics}{不时}{bu4shi2}[][HSK 5]
    \definition{adv.}{frequentemente; de tempos em tempos | a qualquer momento}
  \end{Phonetics}
\end{Entry}

\begin{Entry}{不良}{4,7}{⼀、⾉}
  \begin{Phonetics}{不良}{bu4 liang2}[][HSK 5]
    \definition{adj.}{ruim; prejudicial; nocivo; insalubre}
  \end{Phonetics}
\end{Entry}

\begin{Entry}{不足}{4,7}{⼀、⾜}
  \begin{Phonetics}{不足}{bu4zu2}[][HSK 5]
    \definition{adj.}{não o bastante; inadequado; insuficiente}
    \definition{s.}{deficiência; inadequação; desvantagens, não é bom o suficiente}
    \definition{v.}{não exceder um determinado número | não valer a pena; ser inferior; não merecer | não pode; não deveria}
  \end{Phonetics}
\end{Entry}

\begin{Entry}{不到}{4,8}{⼀、⼑}
  \begin{Phonetics}{不到}{bu2dao4}
    \definition{adj.}{insuficiente}
    \definition{adv.}{menos que}
    \definition{v.}{não chegar}
  \end{Phonetics}
\end{Entry}

\begin{Entry}{不幸}{4,8}{⼀、⼲}
  \begin{Phonetics}{不幸}{bu2 xing4}[][HSK 5]
    \definition{adj.}{triste; infeliz; lamentável; azarado | infeliz; indica o mais indesejável (que aconteceu)}
    \definition[些]{s.}{morte; desastre; infortúnio; adversidade; calamidade}
  \end{Phonetics}
\end{Entry}

\begin{Entry}{不易}{4,8}{⼀、⽇}
  \begin{Phonetics}{不易}{bu2 yi4}[][HSK 5]
    \definition{adj.}{não é fácil; difícil | imutável}
    \definition{v.}{não é fácil fazer algo | não pode ser alterado}
  \end{Phonetics}
\end{Entry}

\begin{Entry}{不注意}{4,8,13}{⼀、⽔、⼼}
  \begin{Phonetics}{不注意}{bu2zhu4yi4}
    \definition{adj.}{impensado | distraído}
    \definition{s.}{descuido | distração}
  \end{Phonetics}
\end{Entry}

\begin{Entry}{不便}{4,9}{⼀、⼈}
  \begin{Phonetics}{不便}{bu2 bian4}[][HSK 6]
    \definition{adj.}{inconveniente; inapropriado | não ter dinheiro em mãos; estar com pouco dinheiro}
    \definition{v.}{inadequado fazer algo; indica que fazer algo é inapropriado ou inconveniente}
  \end{Phonetics}
\end{Entry}

\begin{Entry}{不客气}{4,9,4}{⼀、⼧、⽓}
  \begin{Phonetics}{不客气}{bu2 ke4 qi5}[][HSK 1]
    \definition{adj.}{rude; indelicado; duro | franco; sincero; direto}
    \definition{expr.}{de modo algum; não mencione isso; de nada}
    \definition{v.}{dizer palavras ou fazer gestos indelicados}
  \end{Phonetics}
\end{Entry}

\begin{Entry}{不怎么}{4,9,3}{⼀、⼼、⼃}
  \begin{Phonetics}{不怎么}{bu4 zen3 me5}[][HSK 6]
    \definition{adv.}{não muito; não particularmente; não exatamente}
  \end{Phonetics}
\end{Entry}

\begin{Entry}{不怎么样}{4,9,3,10}{⼀、⼼、⼃、⽊}
  \begin{Phonetics}{不怎么样}{bu4 zen3 me5 yang4}[][HSK 6]
    \definition{adj.}{não muito bom; não particularmente bom | muito indiferente; mais ou menos}
  \end{Phonetics}
\end{Entry}

\begin{Entry}{不是……而是}{4,9,6,9}{⼀、⽇、⽽、⽇}
  \begin{Phonetics}{不是……而是}{bu4shi4 er2 shi4}
    \definition{conj.}{não somente\dots mas também\dots, expressam um relacionamento mais profundo e avançado em significado, mas as orações antes e depois são consistentes em expressar significados negativos e afirmativos, entretanto, a primeira metade da frase expressa negação, e a segunda metade expressa afirmação, e o significado das orações anteriores e seguintes não pode ser de um nível mais alto}
  \end{Phonetics}
\end{Entry}

\begin{Entry}{不是话}{4,9,8}{⼀、⽇、⾔}
  \begin{Phonetics}{不是话}{bu2shi4hua4}
    \definition{expr.}{inacreditável; além das palavras; (palavras) não fazem sentido}
  \seealsoref{不像话}{bu2xiang4hua4}
  \seealsoref{不成话}{bu4cheng2hua4}
  \end{Phonetics}
\end{Entry}

\begin{Entry}{不耐烦}{4,9,10}{⼀、⽽、⽕}
  \begin{Phonetics}{不耐烦}{bu2nai4fan2}[][HSK 5]
    \definition{adj.}{impaciente; significa não ser capaz de suportar coisas tediosas ou que causam distração}
  \end{Phonetics}
\end{Entry}

\begin{Entry}{不要}{4,9}{⼀、⾑}
  \begin{Phonetics}{不要}{bu2 yao4}[][HSK 2]
    \definition{adv.}{nada de (pedir a alguém para não fazer) | não; expressa proibição e dissuasão}
  \end{Phonetics}
\end{Entry}

\begin{Entry}{不要紧}{4,9,10}{⼀、⾑、⽷}
  \begin{Phonetics}{不要紧}{bu2yao4jin3}[][HSK 4]
    \definition{adj.}{não é sério; não importa; não importa; não é um problema; nenhum obstáculo; nenhum problema; parece estar tudo bem; à primeira vista, não parece haver nenhum obstáculo}
  \end{Phonetics}
\end{Entry}

\begin{Entry}{不值}{4,10}{⼀、⼈}
  \begin{Phonetics}{不值}{bu4 zhi2}[][HSK 6]
    \definition{v.}{não valer a pena}
  \end{Phonetics}
\end{Entry}

\begin{Entry}{不料}{4,10}{⼀、⽃}
  \begin{Phonetics}{不料}{bu2liao4}[][HSK 6]
    \definition{conj.}{inesperadamente; para surpresa de alguém}
  \end{Phonetics}
\end{Entry}

\begin{Entry}{不能不}{4,10,4}{⼀、⾁、⼀}
  \begin{Phonetics}{不能不}{bu4 neng2 bu4}[][HSK 5]
    \definition{adv.}{tem que; não pode, mas; necessariamente; definitivamente}
  \end{Phonetics}
\end{Entry}

\begin{Entry}{不通}{4,10}{⼀、⾡}
  \begin{Phonetics}{不通}{bu4 tong1}[][HSK 6]
    \definition{adj.}{sem sentido; ilógico; agramatical | usado para se referir a coisas abstratas}
    \definition{v.}{obstruir; bloquear; estar obstruído; estar bloqueado; ser intransitável | não saber; não entender; não poder aceitar}
  \end{Phonetics}
\end{Entry}

\begin{Entry}{不顾}{4,10}{⼀、⾴}
  \begin{Phonetics}{不顾}{bu2gu4}[][HSK 5]
    \definition{v.}{não considerar; desconsiderar | desconsiderar; não levar em consideração; ignorar; não se preocupar com}
  \end{Phonetics}
\end{Entry}

\begin{Entry}{不停}{4,11}{⼀、⼈}
  \begin{Phonetics}{不停}{bu4 ting2}[][HSK 5]
    \definition{adv.}{sem parar; sem interrupção; continuamente}
  \end{Phonetics}
\end{Entry}

\begin{Entry}{不够}{4,11}{⼀、⼣}
  \begin{Phonetics}{不够}{bu2 gou4}[][HSK 2]
    \definition{adv.}{insuficiente; indica que não atingiu o nível esperado}
    \definition{v.}{não ser suficiente; indica que é inferior ao exigido em quantidade ou grau}
  \end{Phonetics}
\end{Entry}

\begin{Entry}{不得了}{4,11,2}{⼀、⼻、⼅}
  \begin{Phonetics}{不得了}{bu4de2liao3}[][HSK 5]
    \definition{adj.}{terrível; horrível; extremamente sério; indica uma situação grave}
    \definition{adv.}{muito; extremamente; excessivamente; indica um grau profundo}
  \end{Phonetics}
\end{Entry}

\begin{Entry}{不得不}{4,11,4}{⼀、⼻、⼀}
  \begin{Phonetics}{不得不}{bu4de2bu4}[][HSK 3]
    \definition{adv.}{ter que; não ter outra escolha a não ser; como obrigação ou necessidade}
  \end{Phonetics}
\end{Entry}

\begin{Entry}{不敢当}{4,11,6}{⼀、⽁、⼹}
  \begin{Phonetics}{不敢当}{bu4gan3dang1}[][HSK 5]
    \definition{expr.}{Eu realmente não mereço isso.; Eu não sou digno de tais elogios.; Não estou à altura da honra.; Você me lisonjeia.; palavra de humildade, para mostrar que você não pode pagar (hospitalidade, elogios, etc.)}
  \end{Phonetics}
\end{Entry}

\begin{Entry}{不断}{4,11}{⼀、⽄}
  \begin{Phonetics}{不断}{bu2duan4}[][HSK 3]
    \definition{adv.}{incessantemente; ininterruptamente; continuamente; constantemente}
    \definition{v.}{continuar; enfatiza a continuação da ação}
  \end{Phonetics}
\end{Entry}

\begin{Entry}{不曾}{4,12}{⼀、⽈}
  \begin{Phonetics}{不曾}{bu4 ceng2}[][HSK 5]
    \definition{adv.}{nunca (ter feito algo); indica que não aconteceu (negação de 曾经)}
  \seealsoref{曾经}{ceng2jing1}
  \end{Phonetics}
\end{Entry}

\begin{Entry}{不然}{4,12}{⼀、⽕}
  \begin{Phonetics}{不然}{bu4ran2}[][HSK 4]
    \definition{adj.}{não é assim; não é o caso}
    \definition{conj.}{se não; caso contrário; indica outra consequência ou circunstância que teria ocorrido se não fosse}
  \end{Phonetics}
\end{Entry}

\begin{Entry}{不像话}{4,13,8}{⼀、⼈、⾔}
  \begin{Phonetics}{不像话}{bu2xiang4hua4}
    \definition{expr.}{absurdo; sem sentido; irracional; uma determinada prática ou afirmação não está de acordo com o senso comum ou a razão e parece irracional | chocante; ultrajante; uma ação, palavra ou situação tão extrema que não pode ser aceita ou tolerada}
  \seealsoref{不是话}{bu2shi4hua4}
  \seealsoref{不成话}{bu4cheng2hua4}
  \end{Phonetics}
\end{Entry}

\begin{Entry}{不满}{4,13}{⼀、⽔}
  \begin{Phonetics}{不满}{bu4 man3}[][HSK 2]
    \definition{adj.}{ressentido; insatisfeito; descontente}
    \definition{v.}{estar descontente com; insatisfação ou descontentamento com alguém ou alguma coisa |ser menor que; quantidade ou tempo insuficientes ou inadequados}
  \end{Phonetics}
\end{Entry}

\begin{Entry}{不禁}{4,13}{⼀、⽰}
  \begin{Phonetics}{不禁}{bu4jin1}[][HSK 6]
    \definition{adv.}{não pode evitar (fazer algo); não pode se abster de; incapaz de conter (produzir certas emoções, realizar certas ações)}
  \end{Phonetics}
\end{Entry}

\begin{Entry}{不错}{4,13}{⼀、⾦}
  \begin{Phonetics}{不错}{bu2 cuo4}[][HSK 2]
    \definition{adj.}{certo; correto | nada mal; muito bom}
  \end{Phonetics}
\end{Entry}

\begin{Entry}{不管}{4,14}{⼀、⽵}
  \begin{Phonetics}{不管}{bu4guan3}[][HSK 4]
    \definition{conj.}{não importa (o que, como, etc.); independentemente de; indica que, embora as condições ou circunstâncias tenham mudado, o resultado permanece o mesmo; 不管 deve ser seguido por algo incerto}
  \seealsoref{不管……都……}{bu4guan3 dou1}
  \seealsoref{不管……也……}{bu4guan3 ye3}
  \end{Phonetics}
\end{Entry}

\begin{Entry}{不管……也……}{4,14,3}{⼀、⽵、⼄}
  \begin{Phonetics}{不管……也……}{bu4guan3 ye3}
    \definition{conj.}{não apenas\dots, (o que, quem, como, etc.), \dots}
  \end{Phonetics}
\end{Entry}

\begin{Entry}{不管……都……}{4,14,10}{⼀、⽵、⾢}
  \begin{Phonetics}{不管……都……}{bu4guan3 dou1}
    \definition{conj.}{não apenas\dots, (o que, quem, como, etc.), \dots}
  \end{Phonetics}
\end{Entry}

\begin{Entry}{丑}{4}{⼀}
  \begin{Phonetics}{丑}{chou3}[][HSK 5]
    \definition*{s.}{Sobrenome Chou}
    \definition{adj.}{feio, sem atrativos; em oposição a 美 | vergonhoso; desavergonhado; escandaloso; censurável; questionável}
    \definition{s.}{palhaço na ópera de Pequim, etc. | o segundo dos Doze Ramos Terrestres}
  \seealsoref{美}{mei3}
  \end{Phonetics}
\end{Entry}

\begin{Entry}{专}{4}{⼀}
  \begin{Phonetics}{专}{zhuan1}
    \definition{adj.}{específico para; dedicado a um uso específico; dedicado a; especial}
    \definition{adv.}{especialmente; especificamente}
    \definition{v.}{monopolizar}
  \end{Phonetics}
\end{Entry}

\begin{Entry}{专门}{4,3}{⼀、⾨}
  \begin{Phonetics}{专门}{zhuan1men2}[][HSK 3]
    \definition{adj.}{especializado; dedicar-se exclusivamente a uma determinada tarefa; expressa ênfase em fazer frequentemente um determinado tipo de coisa}
    \definition{adv.}{especialmente}
  \end{Phonetics}
\end{Entry}

\begin{Entry}{专心}{4,4}{⼀、⼼}
  \begin{Phonetics}{专心}{zhuan1xin1}[][HSK 4]
    \definition{adj.}{absorto; concentrado}
  \end{Phonetics}
\end{Entry}

\begin{Entry}{专业}{4,5}{⼀、⼀}
  \begin{Phonetics}{专业}{zhuan1ye4}[][HSK 3]
    \definition{adj.}{profissional; descreve uma pessoa que possui um alto nível ou conhecimento profundo em determinada área}
    \definition[个,门]{s.}{profissão; área específica; comércio especializado; departamentos operacionais da divisão de produção | especialidade; disciplina; matéria especializada; área de estudo específica; em um departamento de uma instituição de ensino superior ou em uma escola profissionalizante de nível médio}
  \end{Phonetics}
\end{Entry}

\begin{Entry}{专业人士}{4,5,2,3}{⼀、⼀、⼈、⼠}
  \begin{Phonetics}{专业人士}{zhuan1ye4ren2shi4}
    \definition{s.}{profissional}
  \end{Phonetics}
\end{Entry}

\begin{Entry}{专业人才}{4,5,2,3}{⼀、⼀、⼈、⼿}
  \begin{Phonetics}{专业人才}{zhuan1ye4ren2cai2}
    \definition{s.}{especialista (em uma área)}
  \end{Phonetics}
\end{Entry}

\begin{Entry}{专业化}{4,5,4}{⼀、⼀、⼔}
  \begin{Phonetics}{专业化}{zhuan1ye4hua4}
    \definition{s.}{especialização}
  \end{Phonetics}
\end{Entry}

\begin{Entry}{专业户}{4,5,4}{⼀、⼀、⼾}
  \begin{Phonetics}{专业户}{zhuan1ye4hu4}
    \definition{s.}{indústria caseira | empresa familiar produzindo um produto especial}
  \end{Phonetics}
\end{Entry}

\begin{Entry}{专业性}{4,5,8}{⼀、⼀、⼼}
  \begin{Phonetics}{专业性}{zhuan1ye4xing4}
    \definition{s.}{profissionalismo | expertise}
  \end{Phonetics}
\end{Entry}

\begin{Entry}{专业教育}{4,5,11,8}{⼀、⼀、⽁、⾁}
  \begin{Phonetics}{专业教育}{zhuan1ye4jiao4yu4}
    \definition{s.}{educação especializada | escola técnica}
  \end{Phonetics}
\end{Entry}

\begin{Entry}{专用}{4,5}{⼀、⽤}
  \begin{Phonetics}{专用}{zhuan1 yong4}[][HSK 6]
    \definition{adj.}{(reservado para) uso especial; para um propósito especial; dedicado a uma determinada necessidade ou a uma determinada pessoa}[他需要一个专用的工作空间。===Ele precisava de um espaço de trabalho dedicado.]
  \end{Phonetics}
\end{Entry}

\begin{Entry}{专利}{4,7}{⼀、⼑}
  \begin{Phonetics}{专利}{zhuan1li4}[][HSK 5]
    \definition[个,项,些]{s.}{patente; a garantia de que os criadores e inventores desfrutem exclusivamente dos benefícios decorrentes de suas criações e invenções durante um determinado período | direitos de patente; referência à patente}
  \end{Phonetics}
\end{Entry}

\begin{Entry}{专家}{4,10}{⼀、⼧}
  \begin{Phonetics}{专家}{zhuan1jia1}[][HSK 3]
    \definition[个,位]{s.}{perito; especialista; profissional; pessoa que se dedica ao estudo aprofundado de uma determinada disciplina; pessoa especializada em uma determinada técnica}
  \end{Phonetics}
\end{Entry}

\begin{Entry}{专辑}{4,13}{⼀、⾞}
  \begin{Phonetics}{专辑}{zhuan1 ji2}[][HSK 5]
    \definition[张]{s.}{álbum (música) | registro (música) | coleção especial de material impresso ou transmitido}
  \end{Phonetics}
\end{Entry}

\begin{Entry}{专题}{4,15}{⼀、⾴}
  \begin{Phonetics}{专题}{zhuan1ti2}[][HSK 3]
    \definition[个,些,种]{s.}{assunto especial; tópico especial; questões específicas}
  \end{Phonetics}
\end{Entry}

\begin{Entry}{且}{5}{⼀}
  \begin{Phonetics}{且}{qie3}
    \definition*{s.}{Sobrenome Qie}
    \definition{adv.}{apenas; por enquanto | por um longo tempo}
    \definition{conj.}{mesmo; até; até mesmo; usado na primeira cláusula de uma frase complexa para expressar concessão, equivalente a 尚且 | ambos\dots e\dots; conecta adjetivos ou verbos para expressar relacionamento paralelo, equivalente a 而且 e 又……又……}
  \seealsoref{而且}{er2 qie3}
  \seealsoref{尚且}{shang4 qie3}
  \seealsoref{又……又……}{you4 you4}
  \end{Phonetics}
\end{Entry}

\begin{Entry}{世}{5}{⼀}
  \begin{Phonetics}{世}{shi4}
    \definition*{s.}{Sobrenome Shi}
    \definition{s.}{vida; tempo de vida; vida humana | geração; geração após geração | idade; era | o mundo; sociedade | (geologia) época, abaixo de ``período''}
  \end{Phonetics}
\end{Entry}

\begin{Entry}{世代}{5,5}{⼀、⼈}
  \begin{Phonetics}{世代}{shi4dai4}
    \definition{adv.}{por muitas gerações, eras}
    \definition{s.}{geração | era}
  \end{Phonetics}
\end{Entry}

\begin{Entry}{世纪}{5,6}{⼀、⽷}
  \begin{Phonetics}{世纪}{shi4ji4}[][HSK 3]
    \definition[个,段]{s.}{século; uma unidade para calcular anos, cem anos é um século}
  \end{Phonetics}
\end{Entry}

\begin{Entry}{世界}{5,9}{⼀、⽥}
  \begin{Phonetics}{世界}{shi4jie4}[][HSK 3]
    \definition[个,片,种]{s.}{mundo; todos os lugares da Terra | a soma da natureza e da sociedade humana; refere-se à soma de toda a existência objetiva na natureza e na sociedade humana | campo; refere-se a uma determinada área ou campo | o universo sem limites; costumava ser um termo budista, mas agora também se refere ao mundo natural ilimitado e à sociedade humana | situação social; a situação ou atmosfera social de um determinado período}
  \end{Phonetics}
\end{Entry}

\begin{Entry}{世界杯}{5,9,8}{⼀、⽥、⽊}
  \begin{Phonetics}{世界杯}{shi4jie4bei1}[][HSK 3]
    \definition*{s.}{Copa do Mundo; Troféu da Copa do Mundo}
  \end{Phonetics}
\end{Entry}

\begin{Entry}{世锦赛}{5,13,14}{⼀、⾦、⾙}
  \begin{Phonetics}{世锦赛}{shi4jin3sai4}
    \definition*{s.}{Campeonato Mundial}
  \end{Phonetics}
\end{Entry}

\begin{Entry}{丘}{5}{⼀}
  \begin{Phonetics}{丘}{qiu1}
    \definition*{s.}{Sobrenome Qiu}
    \definition[个]{s.}{monte; outeiro | (literário) sepultura}
  \end{Phonetics}
\end{Entry}

\begin{Entry}{丘陵}{5,10}{⼀、⾩}
  \begin{Phonetics}{丘陵}{qiu1ling2}
    \definition{s.}{colinas}
  \end{Phonetics}
\end{Entry}

\begin{Entry}{业}{5}{⼀}
  \begin{Phonetics}{业}{ye4}
    \definition*{s.}{Sobrenome Ye}
    \definition{adv.}{já; indica que a ação foi concluída, equivalente a 已经}
    \definition{s.}{comércio; indústria; ramo de negócios | emprego; ocupação; profissão | curso de estudo | causa; empreendimento | propriedade | carma; o budismo se refere a todas as ações, palavras e pensamentos humanos como carma, que são chamados de carma corporal, carma da fala e carma mental; o carma inclui aspectos bons e ruins, geralmente referindo-se ao destino ou ao pecado}
    \definition{v.}{envolver-se em; exercer uma determinada profissão}
  \seealsoref{已经}{yi3jing1}
  \end{Phonetics}
\end{Entry}

\begin{Entry}{业务}{5,5}{⼀、⼒}
  \begin{Phonetics}{业务}{ye4wu4}[][HSK 5]
    \definition[项,笔,个,类,种]{s.}{negócios; trabalho vocacional; trabalho profissional}
  \end{Phonetics}
\end{Entry}

\begin{Entry}{业余}{5,7}{⼀、⼈}
  \begin{Phonetics}{业余}{ye4yu2}[][HSK 4]
    \definition{adj.}{tempo livre; depois do expediente; fora do horário de trabalho | amador; não profissional}
  \end{Phonetics}
\end{Entry}

\begin{Entry}{东}{5}{⼀}
  \begin{Phonetics}{东}{dong1}[][HSK 1]
    \definition*{s.}{Sobrenome Dong}
    \definition{s.}{leste; uma das quatro direções básicas, o lado onde o sol nasce | proprietário; dono | anfitrião (antigamente, o anfitrião ficava a leste e os convidados a oeste)}
  \end{Phonetics}
\end{Entry}

\begin{Entry}{东方}{5,4}{⼀、⽅}
  \begin{Phonetics}{东方}{dong1 fang1}[][HSK 2]
    \definition*{s.}{Sobrenome Dongfang}
    \definition{s.}{leste | oriente; o leste; o Oriente}
  \end{Phonetics}
\end{Entry}

\begin{Entry}{东方学院}{5,4,8,9}{⼀、⽅、⼦、⾩}
  \begin{Phonetics}{东方学院}{dong1fang1 xue2yuan4}
    \definition*{s.}{Instituto Oriental}
  \end{Phonetics}
\end{Entry}

\begin{Entry}{东北}{5,5}{⼀、⼔}
  \begin{Phonetics}{东北}{dong1 bei3}[][HSK 2]
    \definition*{s.}{Nordeste da China; O Nordeste | Manchúria}
    \definition{s.}{nordeste}
  \end{Phonetics}
\end{Entry}

\begin{Entry}{东半球}{5,5,11}{⼀、⼗、⽟}
  \begin{Phonetics}{东半球}{dong1ban4qiu2}
    \definition*{s.}{Hemisfério Oriental}
  \end{Phonetics}
\end{Entry}

\begin{Entry}{东边}{5,5}{⼀、⾡}
  \begin{Phonetics}{东边}{dong1 bian5}[][HSK 1]
    \definition{s.}{leste; o lado leste; refere-se à fronteira oriental}
  \end{Phonetics}
\end{Entry}

\begin{Entry}{东西}{5,6}{⼀、⾑}
  \begin{Phonetics}{东西}{dong1xi1}
    \definition{s.}{leste e oeste | de leste a oeste; a distância de um lugar de leste a oeste}
  \end{Phonetics}
  \begin{Phonetics}{东西}{dong1xi5}[][HSK 1]
    \definition[个,件]{s.}{coisa; refere-se a todos os tipos de coisas concretas ou abstratas | coisa; criatura; refere-se especificamente a pessoas ou coisas que causam repulsa ou simpatia}
  \end{Phonetics}
\end{Entry}

\begin{Entry}{东南}{5,9}{⼀、⼗}
  \begin{Phonetics}{东南}{dong1 nan2}[][HSK 2]
    \definition*{s.}{Sudeste da China; O Sudeste; refere-se à região costeira sudeste da China, incluindo as províncias e cidades de Xangai, Jiangsu, Zhejiang, Fujian, Taiwan, etc.}
    \definition{s.}{sudeste}
  \end{Phonetics}
\end{Entry}

\begin{Entry}{东面}{5,9}{⼀、⾯}
  \begin{Phonetics}{东面}{dong1mian4}
    \definition{s.}{lado leste (de algo)}
  \end{Phonetics}
\end{Entry}

\begin{Entry}{东部}{5,10}{⼀、⾢}
  \begin{Phonetics}{东部}{dong1 bu4}[][HSK 3]
    \definition{s.}{o leste; parte oriental; a parte oriental de uma determinada região}
  \end{Phonetics}
\end{Entry}

\begin{Entry}{丝}{5}{⼀}
  \begin{Phonetics}{丝}{si1}
    \definition{clas.}{si, uma unidade de peso (=0,0005 gramas) | usado para expressar a aparência ou expressão de uma pessoa | um décimo de milésimo de certas unidades de medida (medida de comprimento) | usado para representar coisas abstratas}
    \definition[些,种,类,跟,缕]{s.}{seda | uma coisa semelhante a um fio; itens semelhantes à seda | cordas; instrumentos de corda}
  \end{Phonetics}
\end{Entry}

\begin{Entry}{两}{7}{⼀}
  \begin{Phonetics}{两}{liang3}[][HSK 1,2]
    \definition*{s.}{Sobrenome Liang}
    \definition{clas.}{liang, uma unidade de peso (=50 gramas)}
    \definition{num.}{dois (sempre usado antes de classificadores) | poucos; alguns; indica um número indeterminado}
    \definition{s.}{ambos (lados); qualquer (lado)}
  \end{Phonetics}
\end{Entry}

\begin{Entry}{两手}{7,4}{⼀、⼿}
  \begin{Phonetics}{两手}{liang3 shou3}[][HSK 6]
    \definition{s.}{ambas as mãos | ambos os aspectos; táticas duplas | Coloquial: habilidade; capacidade}
  \end{Phonetics}
\end{Entry}

\begin{Entry}{两边}{7,5}{⼀、⾡}
  \begin{Phonetics}{两边}{liang3 bian1}[][HSK 4]
    \definition{s.}{ambos os lados; ambas as direções; ambos os lugares | ambas as partes; ambos os lados}
  \end{Phonetics}
\end{Entry}

\begin{Entry}{两侧}{7,8}{⼀、⼈}
  \begin{Phonetics}{两侧}{liang3 ce4}[][HSK 6]
    \definition{s.}{dois flancos; dois (ambos) lados; ambos}
  \end{Phonetics}
\end{Entry}

\begin{Entry}{两岸}{7,8}{⼀、⼭}
  \begin{Phonetics}{两岸}{liang3 an4}[][HSK 5]
    \definition{s.}{ambos os lados; ambas as margens; ambas as costas; entre os dois lados do estreito; bilateral}
  \end{Phonetics}
\end{Entry}

\begin{Entry}{两码事}{7,8,8}{⼀、⽯、⼅}
  \begin{Phonetics}{两码事}{liang3ma3shi4}
    \definition{expr.}{duas coisas completamente diferentes; dois assuntos diferentes}
  \end{Phonetics}
\end{Entry}

\begin{Entry}{严}{7}{⼀}
  \begin{Phonetics}{严}{yan2}[][HSK 4]
    \definition*{s.}{Sobrenome Yan}
    \definition{adj.}{apertado; próximo | rigoroso; severo; duro; áspero; rigoroso; austero | severo; extremo; difícil}
    \definition{s.}{pai; refere-se ao pai}
  \end{Phonetics}
\end{Entry}

\begin{Entry}{严厉}{7,5}{⼀、⼚}
  \begin{Phonetics}{严厉}{yan2li4}[][HSK 5]
    \definition{adj.}{severo; rigoroso; as palavras e atitudes de crítica ou punição são muito sérias e severas}
  \end{Phonetics}
\end{Entry}

\begin{Entry}{严肃}{7,8}{⼀、⾀}
  \begin{Phonetics}{严肃}{yan2su4}[][HSK 5]
    \definition{adj.}{sério; solene; sincero; (expressão, atmosfera, etc.) faz as pessoas se sentirem admiradas e desconfortáveis | sóbrio; grave; sério; sincero}
    \definition{v.}{aplicar rigorosamente; fazer algo sério}
  \end{Phonetics}
\end{Entry}

\begin{Entry}{严重}{7,9}{⼀、⾥}
  \begin{Phonetics}{严重}{yan2zhong4}[][HSK 4]
    \definition{adj.}{sério; grave; crítico; severo}
  \end{Phonetics}
\end{Entry}

\begin{Entry}{严重打伤}{7,9,5,6}{⼀、⾥、⼿、⼈}
  \begin{Phonetics}{严重打伤}{yan2zhong4 da3 shang1}
    \definition{s.}{gravemente ferido}
  \end{Phonetics}
\end{Entry}

\begin{Entry}{严重伤害}{7,9,6,10}{⼀、⾥、⼈、⼧}
  \begin{Phonetics}{严重伤害}{yan2zhong4 shang1hai4}
    \definition{s.}{ferimento grave; lesão grave}
  \end{Phonetics}
\end{Entry}

\begin{Entry}{严重关切}{7,9,6,4}{⼀、⾥、⼋、⼑}
  \begin{Phonetics}{严重关切}{yan2zhong4guan1qie4}
    \definition{s.}{preocupação séria}
  \end{Phonetics}
\end{Entry}

\begin{Entry}{严重危害}{7,9,6,10}{⼀、⾥、⼙、⼧}
  \begin{Phonetics}{严重危害}{yan2zhong4 wei1hai4}
    \definition{s.}{perigo crítico | dano grave}
  \end{Phonetics}
\end{Entry}

\begin{Entry}{严重后果}{7,9,6,8}{⼀、⾥、⼝、⽊}
  \begin{Phonetics}{严重后果}{yan2zhong4hou4guo3}
    \definition{s.}{consequências sérias | repercursões graves}
  \end{Phonetics}
\end{Entry}

\begin{Entry}{严重地}{7,9,6}{⼀、⾥、⼟}
  \begin{Phonetics}{严重地}{yan2zhong4 di4}
    \definition{adv.}{seriamente | gravemente}
  \end{Phonetics}
\end{Entry}

\begin{Entry}{严重问题}{7,9,6,15}{⼀、⾥、⾨、⾴}
  \begin{Phonetics}{严重问题}{yan2zhong4 wen4ti2}
    \definition{s.}{problema sério}
  \end{Phonetics}
\end{Entry}

\begin{Entry}{严重性}{7,9,8}{⼀、⾥、⼼}
  \begin{Phonetics}{严重性}{yan2zhong4xing4}
    \definition{s.}{seriedade | gravidade}
  \end{Phonetics}
\end{Entry}

\begin{Entry}{严重破坏}{7,9,10,7}{⼀、⾥、⽯、⼟}
  \begin{Phonetics}{严重破坏}{yan2zhong4 po4huai4}
    \definition{s.}{destruição grave}
  \end{Phonetics}
\end{Entry}

\begin{Entry}{严格}{7,10}{⼀、⽊}
  \begin{Phonetics}{严格}{yan2ge2}[][HSK 4]
    \definition{adj.}{rígido; estrito; rigoroso; muito consciente e meticuloso na implementação de sistemas e no domínio de padrões}
    \definition{v.}{tornar (sistemas, provisões, etc.) rigorosos}
  \end{Phonetics}
\end{Entry}

\begin{Entry}{靣}{8}{⼀}[Kangxi 176]
  \begin{Phonetics}{靣}{mian4}
    \variantof{面}
  \end{Phonetics}
\end{Entry}

%%%%% EOF %%%%%

