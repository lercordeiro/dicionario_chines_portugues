%%%
%%% Radical "⾖"
%%%

\section*{Radical 151: ``⾖''}\addcontentsline{toc}{section}{Radical 151: ⾖}

\begin{Entry}{豆}{7}{⾖}[Kangxi 151]
  \begin{Phonetics}{豆}{dou4}
    \definition*{s.}{Sobrenome Dou}
    \definition{s.}{planta que produz vagens ou suas sementes | coisa em forma de feijão | leguminosas ou sementes de leguminosas; feijões; ervilhas | uma xícara ou tigela antiga com haste}
  \end{Phonetics}
\end{Entry}

\begin{Entry}{豆子}{7,3}{⾖、⼦}
  \begin{Phonetics}{豆子}{dou4zi5}[][HSK 7-9]
    \definition[颗,粒,把,袋]{s.}{planta que produz vagens ou suas sementes | algo em forma de feijão | leguminosas; feijões}
  \end{Phonetics}
\end{Entry}

\begin{Entry}{豆角}{7,7}{⾖、⾓}
  \begin{Phonetics}{豆角}{dou4jiao3}
    \definition{s.}{feijão verde}
  \end{Phonetics}
\end{Entry}

\begin{Entry}{豆制品}{7,8,9}{⾖、⼑、⼝}
  \begin{Phonetics}{豆制品}{dou4 zhi4 pin3}[][HSK 5]
    \definition{s.}{produtos de soja}
  \end{Phonetics}
\end{Entry}

\begin{Entry}{豆荚}{7,9}{⾖、⾋}
  \begin{Phonetics}{豆荚}{dou4jia2}
    \definition{s.}{vagem (de legumes)}
  \end{Phonetics}
\end{Entry}

\begin{Entry}{豆浆}{7,10}{⾖、⽔}
  \begin{Phonetics}{豆浆}{dou4jiang1}[][HSK 7-9]
    \definition[杯,碗]{s.}{leite de soja}
  \end{Phonetics}
\end{Entry}

\begin{Entry}{豆腐}{7,14}{⾖、⾁}
  \begin{Phonetics}{豆腐}{dou4fu5}[][HSK 4]
    \definition[块,盒,斤,盘]{s.}{\emph{tofu}}
  \end{Phonetics}
\end{Entry}

\begin{Entry}{豌}{15}{⾖}
  \begin{Phonetics}{豌}{wan1}
    \definition[粒]{s.}{ervilhas}
  \end{Phonetics}
\end{Entry}

\begin{Entry}{豌豆}{15,7}{⾖、⾖}
  \begin{Phonetics}{豌豆}{wan1dou4}
    \definition{s.}{ervilha}
  \end{Phonetics}
\end{Entry}

%%%%% EOF %%%%%

