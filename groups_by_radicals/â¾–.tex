%%%
%%% Radical "⾖"
%%%

\section*{Radical 151: ``⾖''}\addcontentsline{toc}{section}{Radical 151: ⾖}

\begin{entry}{豆}{7}{⾖}
  \begin{phonetics}{豆}{dou4}
    \definition*{s.}{sobrenome Dou}
    \definition{s.}{planta que produz vagens ou suas sementes | coisa em forma de feijão | leguminosas ou sementes de leguminosas; feijões; ervilhas | uma xícara ou tigela antiga com haste}
  \end{phonetics}
\end{entry}

\begin{entry}{豆角}{7,7}{⾖、⾓}
  \begin{phonetics}{豆角}{dou4jiao3}
    \definition{s.}{feijão verde}
  \end{phonetics}
\end{entry}

\begin{entry}{豆制品}{7,8,9}{⾖、⼑、⼝}
  \begin{phonetics}{豆制品}{dou4 zhi4 pin3}[][HSK 5]
    \definition{s.}{produtos de soja}
  \end{phonetics}
\end{entry}

\begin{entry}{豆荚}{7,9}{⾖、⾋}
  \begin{phonetics}{豆荚}{dou4jia2}
    \definition{s.}{vagem (de legumes)}
  \end{phonetics}
\end{entry}

\begin{entry}{豆腐}{7,14}{⾖、⾁}
  \begin{phonetics}{豆腐}{dou4fu5}[][HSK 4]
    \definition[块,盒,斤,盘,锅]{s.}{\emph{tofu}}
  \end{phonetics}
\end{entry}

\begin{entry}{豌豆}{15,7}{⾖、⾖}
  \begin{phonetics}{豌豆}{wan1dou4}
    \definition{s.}{ervilha}
  \end{phonetics}
\end{entry}

%%%%% EOF %%%%%

