%%%
%%% Radical "⼲"
%%%

\section*{Radical 51: ``⼲''}\addcontentsline{toc}{section}{Radical 51: ⼲}

\begin{entry}{干}{3}{⼲}
  \begin{phonetics}{干}{gan1}[][HSK 1]
    \definition*{s.}{sobrenome Gan}
    \definition{v.}{preocupar | ignorar | interferir}
  \end{phonetics}
  \begin{phonetics}{干}{gan4}[][HSK 1]
    \definition{v.}{fazer | gerenciar | trabalhar | (gíria) matar | (vulgar) foder}
  \end{phonetics}
\end{entry}

\begin{entry}{干与}{3,3}{⼲、⼀}
  \begin{phonetics}{干与}{gan1yu4}
    \variantof{干预}
  \end{phonetics}
\end{entry}

\begin{entry}{干什么}{3,4,3}{⼲、⼈、⼃}
  \begin{phonetics}{干什么}{gan4 shen2 me5}[][HSK 1]
    \definition{v.}{o que fazer? | o que está fazendo?}
  \end{phonetics}
\end{entry}

\begin{entry}{干吗}{3,6}{⼲、⼝}
  \begin{phonetics}{干吗}{gan4 ma2}[][HSK 3]
    \definition{pron.}{por que?}
    \definition{v.}{o que fazer?}
  \end{phonetics}
\end{entry}

\begin{entry}{干你屁事}{3,7,7,8}{⼲、⼈、⼫、⼅}
  \begin{phonetics}{干你屁事}{gan1 ni3 pi4shi4}
    \definition{interj.}{Foda-se!}
  \end{phonetics}
\end{entry}

\begin{entry}{干净}{3,8}{⼲、⼎}
  \begin{phonetics}{干净}{gan1jing4}[][HSK 1]
    \definition{adj.}{limpo | arrumado}
  \end{phonetics}
\end{entry}

\begin{entry}{干杯}{3,8}{⼲、⽊}
  \begin{phonetics}{干杯}{gan1bei1}[][HSK 2]
    \definition{interj.}{Saúde!}
    \definition{v.+compl.}{fazer um brinde | brindar até a última gota}
  \end{phonetics}
\end{entry}

\begin{entry}{干活}{3,9}{⼲、⽔}
  \begin{phonetics}{干活}{gan4huo2}
    \definition{v.+compl.}{trabalhar | trabalhar em um emprego}
  \end{phonetics}
\end{entry}

\begin{entry}{干活儿}{3,9,2}{⼲、⽔、⼉}
  \begin{phonetics}{干活儿}{gan4huo2r5}[][HSK 2]
    \definition{v.}{trabalhar em um emprego}
  \end{phonetics}
\end{entry}

\begin{entry}{干预}{3,10}{⼲、⾴}
  \begin{phonetics}{干预}{gan1yu4}
    \definition{s.}{intervenção}
    \definition{v.}{intervir | intrometer-se}
  \end{phonetics}
\end{entry}

\begin{entry}{平}{5}{⼲}
  \begin{phonetics}{平}{ping2}[][HSK 2]
    \definition*{s.}{sobrenome Ping}
    \definition{adj.}{calmo | pacífico}
    \definition{s.}{plano | nível}
    \definition{v.}{fazer a mesma pontuação | marcar uma pontuação}
  \end{phonetics}
\end{entry}

\begin{entry}{平方}{5,4}{⼲、⽅}
  \begin{phonetics}{平方}{ping2fang1}[][HSK 4]
    \definition{s.}{quadrado}
  \end{phonetics}
\end{entry}

\begin{entry}{平方米}{5,4,6}{⼲、⽅、⽶}
  \begin{phonetics}{平方米}{ping2fang1 mi3}
    \definition{clas.}{unidade de medida de área, 1 metro quadrado equivale a 10.000 centímetros quadrados}
  \end{phonetics}
\end{entry}

\begin{entry}{平台}{5,5}{⼲、⼝}
  \begin{phonetics}{平台}{ping2tai2}
    \definition{s.}{plataforma | terraço | edifício de telhado plano}
  \end{phonetics}
\end{entry}

\begin{entry}{平地}{5,6}{⼲、⼟}
  \begin{phonetics}{平地}{ping2di4}
    \definition{v.}{nivelar a terra | aplanar}
  \end{phonetics}
\end{entry}

\begin{entry}{平安}{5,6}{⼲、⼧}
  \begin{phonetics}{平安}{ping2'an1}[][HSK 2]
    \definition{s.}{seguro | bem | sem contratempos | são e salvo}
  \end{phonetics}
\end{entry}

\begin{entry}{平均}{5,7}{⼲、⼟}
  \begin{phonetics}{平均}{ping2jun1}[][HSK 4]
    \definition{adj.}{igual; médio}
    \definition{s.}{média}
    \definition{v.}{calcular a média de um conjunto de números}
  \end{phonetics}
\end{entry}

\begin{entry}{平时}{5,7}{⼲、⽇}
  \begin{phonetics}{平时}{ping2shi2}[][HSK 2]
    \definition{adv.}{normalmente | em tempos normais | em tempos de paz}
  \end{phonetics}
\end{entry}

\begin{entry}{平常}{5,11}{⼲、⼱}
  \begin{phonetics}{平常}{ping2chang2}[][HSK 2]
    \definition{adj.}{comum | ordinário | usual}
    \definition{adv.}{usualmente | geralmente | ordinariamente | como regra}
  \end{phonetics}
\end{entry}

\begin{entry}{平等}{5,12}{⼲、⽵}
  \begin{phonetics}{平等}{ping2deng3}[][HSK 2]
    \definition{adj.}{igual | igualdade}
  \end{phonetics}
\end{entry}

\begin{entry}{平稳}{5,14}{⼲、⽲}
  \begin{phonetics}{平稳}{ping2 wen3}[][HSK 4]
    \definition{adj.}{firme; estável; suave e constante; sem oscilações ou flutuações}
  \end{phonetics}
\end{entry}

\begin{entry}{平静}{5,14}{⼲、⾭}
  \begin{phonetics}{平静}{ping2jing4}[][HSK 4]
    \definition{adj.}{(humor, ambiente, etc.) calmo; quieto; pacífico; tranquilo}
  \end{phonetics}
\end{entry}

\begin{entry}{年}{6}{⼲}
  \begin{phonetics}{年}{nian2}[][HSK 1]
    \definition*{s.}{sobrenome Nian}
    \definition[个]{clas./s.}{ano}
  \end{phonetics}
\end{entry}

\begin{entry}{年代}{6,5}{⼲、⼈}
  \begin{phonetics}{年代}{nian2dai4}[][HSK 3]
    \definition[个]{s.}{idade; anos; tempo | uma década de um século}
  \end{phonetics}
\end{entry}

\begin{entry}{年级}{6,6}{⼲、⽷}
  \begin{phonetics}{年级}{nian2ji2}[][HSK 2]
    \definition[个]{s.}{classe | ano (escola)}
  \end{phonetics}
\end{entry}

\begin{entry}{年纪}{6,6}{⼲、⽷}
  \begin{phonetics}{年纪}{nian2ji4}[][HSK 3]
    \definition{s.}{era; época; idade}
  \end{phonetics}
\end{entry}

\begin{entry}{年初}{6,7}{⼲、⾐}
  \begin{phonetics}{年初}{nian2 chu1}[][HSK 3]
    \definition{s.}{o começo do ano}
  \end{phonetics}
\end{entry}

\begin{entry}{年底}{6,8}{⼲、⼴}
  \begin{phonetics}{年底}{nian2 di3}[][HSK 3]
    \definition[个]{s.}{fim de ano; o fim do ano}
  \end{phonetics}
\end{entry}

\begin{entry}{年货}{6,8}{⼲、⾙}
  \begin{phonetics}{年货}{nian2huo4}
    \definition{s.}{mercadorias vendidas no Ano Novo Chinês}
  \end{phonetics}
\end{entry}

\begin{entry}{年轻}{6,9}{⼲、⾞}
  \begin{phonetics}{年轻}{nian2qing1}[][HSK 2]
    \definition{adj.}{jovem}
  \end{phonetics}
\end{entry}

\begin{entry}{并}{6}{⼲}
  \begin{phonetics}{并}{bing4}[][HSK 3,4]
    \definition{adv.}{igualmente; simultaneamente; lado a lado; coisas diferentes existem ao mesmo tempo; coisas diferentes estão acontecendo ao mesmo tempo | em absoluto (usado antes de uma negativa para dar ênfase);  usado antes de uma palavra negativa para reforçar o tom e refutá-la ligeiramente}
    \definition{conj.}{além de; e}
    \definition{v.}{combinar; fundir; incorporar; anexar; juntar}
  \end{phonetics}
\end{entry}

\begin{entry}{并且}{6,5}{⼲、⼀}
  \begin{phonetics}{并且}{bing4qie3}[][HSK 3]
    \definition{conj.}{além disso | o que é mais | e}
  \end{phonetics}
\end{entry}

\begin{entry}{并排}{6,11}{⼲、⼿}
  \begin{phonetics}{并排}{bing4pai2}
    \definition{adv.}{lado a lado}
  \end{phonetics}
\end{entry}

\begin{entry}{幷}{8}{⼲}
  \begin{phonetics}{幷}{bing4}
    \variantof{并}
  \end{phonetics}
\end{entry}

\begin{entry}{幸亏}{8,3}{⼲、⼆}
  \begin{phonetics}{幸亏}{xing4kui1}
    \definition{adv.}{felizmente}
  \end{phonetics}
\end{entry}

\begin{entry}{幸运}{8,7}{⼲、⾡}
  \begin{phonetics}{幸运}{xing4yun4}[][HSK 3]
    \definition{adj.}{sortudo; feliz; afortunado}
    \definition[个]{s.}{boa sorte; boa fortuna}
  \end{phonetics}
\end{entry}

\begin{entry}{幸运儿}{8,7,2}{⼲、⾡、⼉}
  \begin{phonetics}{幸运儿}{xing4yun4'er2}
    \definition{s.}{pessoa de sorte}
  \end{phonetics}
\end{entry}

\begin{entry}{幸运抽奖}{8,7,8,9}{⼲、⾡、⼿、⼤}
  \begin{phonetics}{幸运抽奖}{xing4yun4chou1jiang3}
    \definition{s.}{loteria | sorteio}
  \end{phonetics}
\end{entry}

\begin{entry}{幸福}{8,13}{⼲、⽰}
  \begin{phonetics}{幸福}{xing4fu2}[][HSK 3]
    \definition{adj.}{feliz | a vida, a família, etc. deixam as pessoas satisfeitas e felizes}
    \definition{s.}{felicidade; bem estar | uma sensação ou experiência satisfatória e feliz}
  \end{phonetics}
\end{entry}

%%%%% EOF %%%%%

