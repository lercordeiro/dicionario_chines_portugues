%%%
%%% Radical "⼲"
%%%

\section*{Radical 51: ``⼲''}\addcontentsline{toc}{section}{Radical 51: ⼲}

\begin{entry}{干}{3}{⼲}[Kangxi 51]
  \begin{phonetics}{干}{gan1}[][HSK 1]
    \definition*{s.}{Sobrenome Gan}
    \definition{adj.}{seco (oposto a 湿) | vazio; oco; seco | sem substância; vazio | de parentesco nominal; (parentes) não ligados por laços sanguíneos | sem água; (água) esgotada; completamente vazia | assumido como parente nominal; relação familiar reconhecida por adoção | rude; grosseiro; mal-educado; descreve alguém que fala de forma muito direta e rude (sem delicadeza).}
    \definition{adv.}{em vão; fútil; sem propósito; para nada; sem resultado | apenas; sem nada mais | inutilmente; sem uso, sem aproveitamento | superficialmente; significa que não há conteúdo, apenas forma}
    \definition{s.}{(arcaico) escudo | margem; ribeira; margem das águas | alimentos desidratados | abreviação para os dez troncos celestiais}
    \definition{v.}{ofender; afrontar | ter a ver com; estar relacionado com; estar implicado em; interferir com | (antiquado) buscar (cargo público, remuneração, etc.) | (dialeto) deixar alguém de fora; tratar alguém com indiferença; desprezar | assediar; perturbar; criar confusão; causar estragos; bagunçar | solicitar; procurar; buscar (cargo, salário, etc.) | beber até o fim | tratar com indiferença; ignorar}
  \seealsoref{干儿}{gan1 er2}
  \seealsoref{干儿}{gan1r5}
  \seealsoref{湿}{shi1}
  \end{phonetics}
  \begin{phonetics}{干}{gan4}[][HSK 1]
    \definition{adj.}{capaz; competente; habilidoso}
    \definition{s.}{tronco; parte principal; corpo principal ou parte importante de algo | habilidade; capacidade; competência}
    \definition{v.}{fazer; trabalhar; cuidar; fazer coisas | ocupar o cargo de; estar envolvido em; assumir, exercer | lutar; golpear; esforçar-se}
  \end{phonetics}
\end{entry}

\begin{entry}{干儿}{3,2}{⼲、⼉}
  \begin{phonetics}{干儿}{gan1 er2}
    \definition{s.}{filho adotivo (adoção tradicional, ou seja, sem implicações legais)}
  \end{phonetics}
  \begin{phonetics}{干儿}{gan1r5}
    \definition{s.}{alimentos secos, desidratados}
  \end{phonetics}
\end{entry}

\begin{entry}{干与}{3,3}{⼲、⼀}
  \begin{phonetics}{干与}{gan1yu4}
    \variantof{干预}
  \end{phonetics}
\end{entry}

\begin{entry}{干什么}{3,4,3}{⼲、⼈、⼃}
  \begin{phonetics}{干什么}{gan4 shen2 me5}[][HSK 1]
    \definition{adv.}{o que fazer; o que ele está fazendo?; o que você está fazendo?; perguntar a razão ou o objetivo}
  \end{phonetics}
\end{entry}

\begin{entry}{干吗}{3,6}{⼲、⼝}
  \begin{phonetics}{干吗}{gan4 ma2}[][HSK 3]
    \definition{pron.}{por que?}
    \definition{v.}{o que fazer?}
  \end{phonetics}
\end{entry}

\begin{entry}{干你屁事}{3,7,7,8}{⼲、⼈、⼫、⼅}
  \begin{phonetics}{干你屁事}{gan1 ni3 pi4shi4}
    \definition{interj.}{Foda-se!}
  \end{phonetics}
\end{entry}

\begin{entry}{干扰}{3,7}{⼲、⼿}
  \begin{phonetics}{干扰}{gan1rao3}[][HSK 5]
    \definition{v.}{perturbar; incomodar | interferir; interromper o funcionamento adequado de equipamentos eletrônicos com sinais eletrônicos dispersos}
  \end{phonetics}
\end{entry}

\begin{entry}{干净}{3,8}{⼲、⼎}
  \begin{phonetics}{干净}{gan1jing4}[][HSK 1]
    \definition{adj.}{limpo; limpo e arrumado; sem poeira, impurezas, etc. |}
    \definition{adv.}{completely; totally; sem deixar nada para trás}
  \end{phonetics}
\end{entry}

\begin{entry}{干杯}{3,8}{⼲、⽊}
  \begin{phonetics}{干杯}{gan1bei1}[][HSK 2]
    \definition{interj.}{Saúde!}
    \definition{v.+compl.}{fazer um brinde;  brindar até a última gota}
  \end{phonetics}
\end{entry}

\begin{entry}{干活}{3,9}{⼲、⽔}
  \begin{phonetics}{干活}{gan4huo2}
    \definition{v.+compl.}{trabalhar | trabalhar em um emprego}
  \end{phonetics}
\end{entry}

\begin{entry}{干活儿}{3,9,2}{⼲、⽔、⼉}
  \begin{phonetics}{干活儿}{gan4huo2r5}[][HSK 2]
    \definition{v.}{trabalhar; gastar energia física ou mental para fazer algo, especialmente trabalho árduo ou esforçado.}
  \end{phonetics}
\end{entry}

\begin{entry}{干涉}{3,10}{⼲、⽔}
  \begin{phonetics}{干涉}{gan1she4}[][HSK 6]
    \definition{s.}{interferência; refere-se ao ato ou comportamento de interferir nos assuntos dos outros}
    \definition{v.}{interferir; intervir; intrometer-se; pedir ou impedir algo geralmente significa interferir quando não se deve}
  \end{phonetics}
\end{entry}

\begin{entry}{干脆}{3,10}{⼲、⾁}
  \begin{phonetics}{干脆}{gan1cui4}[][HSK 5]
    \definition{adj.}{claro; direto; (falar, fazer coisas) sem hesitação; atitude clara}
    \definition{adv.}{justamente; diretamente; sem maiores considerações}
  \end{phonetics}
\end{entry}

\begin{entry}{干预}{3,10}{⼲、⾴}
  \begin{phonetics}{干预}{gan1yu4}[][HSK 5]
    \definition{s.}{intromissão; intervenção}
    \definition{v.}{intrometer-se; intervir; interpor-se;}
  \end{phonetics}
\end{entry}

\begin{entry}{平}{5}{⼲}
  \begin{phonetics}{平}{ping2}[][HSK 2]
    \definition*{s.}{Sobrenome Ping}
    \definition{adj.}{plano; nivelado; uniforme; liso | igual; justo | mesma pontuação; empatado | médio; comum | silencioso; tranquilo | no mesmo nível; altura igual; sem diferença | imparcial; médio; equitativo | calmo; estável; tranquilo | comum;  frequente}
    \definition{s.}{no mesmo nível; em pé de igualdade com; igual | tom nivelado, um dos quatro tons do chinês clássico}
    \definition{v.}{tornar nivelado ou uniforme; nivelar | reprimir; suprimir | acalmar; tornar pacífico; silenciar (acalmar); conter a raiva | estar no mesmo nível | acalmar; amenizar; controlar a raiva}
  \end{phonetics}
\end{entry}

\begin{entry}{平凡}{5,3}{⼲、⼏}
  \begin{phonetics}{平凡}{ping2fan2}[][HSK 6]
    \definition{adj.}{comum; ordinário; normal; não surpreendente}
  \end{phonetics}
\end{entry}

\begin{entry}{平方}{5,4}{⼲、⽅}
  \begin{phonetics}{平方}{ping2fang1}[][HSK 4]
    \definition{s.}{quadrado}
  \end{phonetics}
\end{entry}

\begin{entry}{平方市丈}{5,4,5,3}{⼲、⽅、⼱、⼀}
  \begin{phonetics}{平方市丈}{ping2fang1 shi4 zhang4}
    \definition{clas.}{pés quadrados}
  \end{phonetics}
\end{entry}

\begin{entry}{平方米}{5,4,6}{⼲、⽅、⽶}
  \begin{phonetics}{平方米}{ping2 fang1 mi3}[][HSK 6]
    \definition{s.}{metro quadrado; a unidade legal de medida de área, 1 metro quadrado é igual a 10.000 centímetros quadrados}
  \end{phonetics}
  \begin{phonetics}{平方米}{ping2fang1 mi3}
    \definition{clas.}{unidade de medida de área, 1 metro quadrado equivale a 10.000 centímetros quadrados}
  \end{phonetics}
\end{entry}

\begin{entry}{平台}{5,5}{⼲、⼝}
  \begin{phonetics}{平台}{ping2 tai2}[][HSK 6]
    \definition[个]{s.}{casa com telhado plano rebocado | terraço | plataforma móvel; metaforicamente, refere-se às áreas, oportunidades, ambientes, espaços, etc. que fornecem suporte e garantia para algo | plataforma; um sistema em um computador eletrônico que consiste em software e hardware básicos; tal sistema pode suportar a execução de programas aplicativos e softwares aplicativos podem ser desenvolvidos nesse sistema | plataforma; lugar; falando metaforicamente, o mesmo nível ou grau}
  \end{phonetics}
\end{entry}

\begin{entry}{平地}{5,6}{⼲、⼟}
  \begin{phonetics}{平地}{ping2di4}
    \definition{v.}{nivelar a terra | aplanar}
  \end{phonetics}
\end{entry}

\begin{entry}{平安}{5,6}{⼲、⼧}
  \begin{phonetics}{平安}{ping2'an1}[][HSK 2]
    \definition{s.}{seguro; bem; sem contratempos; sem acidentes; são e salvo}
  \end{phonetics}
\end{entry}

\begin{entry}{平均}{5,7}{⼲、⼟}
  \begin{phonetics}{平均}{ping2jun1}[][HSK 4]
    \definition{adj.}{igual; médio}
    \definition{s.}{média}
    \definition{v.}{calcular a média de um conjunto de números}
  \end{phonetics}
\end{entry}

\begin{entry}{平时}{5,7}{⼲、⽇}
  \begin{phonetics}{平时}{ping2shi2}[][HSK 2]
    \definition{s.}{em tempos normais; em tempos comuns | em tempo de paz; refere-se a períodos normais}
  \end{phonetics}
\end{entry}

\begin{entry}{平坦}{5,8}{⼲、⼟}
  \begin{phonetics}{平坦}{ping2tan3}[][HSK 5]
    \definition{adj.}{plano; uniforme; nivelado; liso; sem elevações ou depressões (referindo-se principalmente ao relevo)}
  \end{phonetics}
\end{entry}

\begin{entry}{平原}{5,10}{⼲、⼚}
  \begin{phonetics}{平原}{ping2yuan2}[][HSK 5]
    \definition[片]{s.}{campo; planície; terreno plano e extenso}
  \end{phonetics}
\end{entry}

\begin{entry}{平常}{5,11}{⼲、⼱}
  \begin{phonetics}{平常}{ping2chang2}[][HSK 2]
    \definition{adj.}{comum; normal; ordinário; nada de especial}
    \definition{adv.}{normalmente; geralmente; como regra geral}
  \end{phonetics}
\end{entry}

\begin{entry}{平等}{5,12}{⼲、⽵}
  \begin{phonetics}{平等}{ping2deng3}[][HSK 2]
    \definition{adj.}{igual; igualdade; refere-se ao fato de as pessoas gozarem de tratamento igualitário nos aspectos sociais, políticos, econômicos e jurídicos}
  \end{phonetics}
\end{entry}

\begin{entry}{平稳}{5,14}{⼲、⽲}
  \begin{phonetics}{平稳}{ping2 wen3}[][HSK 4]
    \definition{adj.}{firme; estável; suave e constante; sem oscilações ou flutuações}
  \end{phonetics}
\end{entry}

\begin{entry}{平静}{5,14}{⼲、⾭}
  \begin{phonetics}{平静}{ping2jing4}[][HSK 4]
    \definition{adj.}{(humor, ambiente, etc.) calmo; quieto; pacífico; tranquilo}
  \end{phonetics}
\end{entry}

\begin{entry}{平衡}{5,16}{⼲、⾏}
  \begin{phonetics}{平衡}{ping2 heng2}[][HSK 6]
    \definition{adj.}{balanceado; equilibrado; os aspectos opostos são iguais ou compensados ​​em quantidade ou qualidade | equilibrado; várias forças atuam sobre um objeto com magnitude igual e direções opostas para manter o objeto estável}
    \definition{v.}{equilibrar; trazer ou manter em equilíbrio; tornar as coisas ou alimentos iguais em quantidade, qualidade ou força}
  \end{phonetics}
\end{entry}

\begin{entry}{年}{6}{⼲}
  \begin{phonetics}{年}{nian2}[][HSK 1]
    \definition*{s.}{Sobrenome Nian}
    \definition{clas.}{ano; usado para calcular o número de anos}
    \definition{s.}{ano | idade | um período (época) da história | colheita anual | Ano Novo | artigos para o dia de Ano Novo | um período da vida de uma pessoa; fases da vida humana divididas por idade}
  \end{phonetics}
\end{entry}

\begin{entry}{年代}{6,5}{⼲、⼈}
  \begin{phonetics}{年代}{nian2dai4}[][HSK 3]
    \definition[个]{s.}{idade; anos; tempo; um período de tempo com características distintas na história | uma década de um século; período de dez anos}
  \end{phonetics}
\end{entry}

\begin{entry}{年级}{6,6}{⼲、⽷}
  \begin{phonetics}{年级}{nian2ji2}[][HSK 2]
    \definition[个]{s.}{série; ano; níveis divididos de acordo com o tempo de estudo dos alunos na escola}
  \end{phonetics}
\end{entry}

\begin{entry}{年纪}{6,6}{⼲、⽷}
  \begin{phonetics}{年纪}{nian2ji4}[][HSK 3]
    \definition[把,个]{s.}{idade (de uma pessoa)}
  \end{phonetics}
\end{entry}

\begin{entry}{年初}{6,7}{⼲、⾐}
  \begin{phonetics}{年初}{nian2 chu1}[][HSK 3]
    \definition{s.}{o começo do ano; os primeiros dias do ano}
  \end{phonetics}
\end{entry}

\begin{entry}{年底}{6,8}{⼲、⼴}
  \begin{phonetics}{年底}{nian2 di3}[][HSK 3]
    \definition[个]{s.}{fim de ano; o fim do ano; geralmente os últimos dias de dezembro ou o fim do ano}
  \end{phonetics}
\end{entry}

\begin{entry}{年货}{6,8}{⼲、⾙}
  \begin{phonetics}{年货}{nian2huo4}
    \definition{s.}{mercadorias vendidas no Ano Novo Chinês}
  \end{phonetics}
\end{entry}

\begin{entry}{年前}{6,9}{⼲、⼑}
  \begin{phonetics}{年前}{nian2 qian2}[][HSK 5]
    \definition{s.}{antes do final do ano; antes do ano novo}
  \end{phonetics}
\end{entry}

\begin{entry}{年度}{6,9}{⼲、⼴}
  \begin{phonetics}{年度}{nian2du4}[][HSK 5]
    \definition{s.}{ano; de acordo com a natureza e as necessidades de um negócio, há um prazo de doze meses com data de início e término definidas}
  \end{phonetics}
\end{entry}

\begin{entry}{年轻}{6,9}{⼲、⾞}
  \begin{phonetics}{年轻}{nian2qing1}[][HSK 2]
    \definition{adj.}{jovem; não muito velho (geralmente se refere a pessoas entre 10 e 20 anos)}
  \end{phonetics}
\end{entry}

\begin{entry}{年龄}{6,13}{⼲、⿒}
  \begin{phonetics}{年龄}{nian2ling2}[][HSK 5]
    \definition[个]{s.}{idade; animais, plantas e outros seres vivos vivem e crescem no mundo durante um determinado número de anos}
  \end{phonetics}
\end{entry}

\begin{entry}{并}{6}{⼲}
  \begin{phonetics}{并}{bing4}[][HSK 3,4]
    \definition{adv.}{lado a lado; igualmente; simultaneamente | (usado para reforçar uma negação) na verdade; definitivamente | mesmo assim | (usado para reforçar uma negação) na verdade; de forma alguma | todos; indica o conjunto completo, equivalente a 全部}
    \definition{conj.}{e; além disso}
    \definition{v.}{combinar; fundir; incorporar | ficar (ou colocar) lado a lado | estar paralelo a | anexar; juntar}
  \seealsoref{全部}{quan2bu4}
  \end{phonetics}
\end{entry}

\begin{entry}{并且}{6,5}{⼲、⼀}
  \begin{phonetics}{并且}{bing4qie3}[][HSK 3]
    \definition{conj.}{e; bem como; usado entre verbos, adjetivos ou frases paralelas para indicar que várias ações são realizadas ao mesmo tempo ou que propriedades existem ao mesmo tempo | além disso; além do mais; ademais; usado na segunda metade de uma frase complexa para expressar um significado adicional}
  \end{phonetics}
\end{entry}

\begin{entry}{并排}{6,11}{⼲、⼿}
  \begin{phonetics}{并排}{bing4pai2}
    \definition{adv.}{lado a lado}
  \end{phonetics}
\end{entry}

\begin{entry}{幷}{8}{⼲}
  \begin{phonetics}{幷}{bing4}
    \variantof{并}
  \end{phonetics}
\end{entry}

\begin{entry}{幸}{8}{⼲}
  \begin{phonetics}{幸}{xing4}
    \definition*{s.}{Sobrenome Xing}
    \definition{adj.}{feliz}
    \definition{adv.}{afortunadamente; felizmente}
    \definition{s.}{felicidade}
    \definition{v.}{alegrar-se; sentir-se feliz e contente | favorecer; patrocinar | vir; chegar; antigamente, referia-se à chegada de um monarca a um determinado lugar}
  \end{phonetics}
\end{entry}

\begin{entry}{幸亏}{8,3}{⼲、⼆}
  \begin{phonetics}{幸亏}{xing4kui1}
    \definition{adv.}{felizmente}
  \end{phonetics}
\end{entry}

\begin{entry}{幸运}{8,7}{⼲、⾡}
  \begin{phonetics}{幸运}{xing4yun4}[][HSK 3]
    \definition{adj.}{sortudo; feliz; afortunado}
    \definition[个,点,丝]{s.}{boa sorte; boa fortuna}
  \end{phonetics}
\end{entry}

\begin{entry}{幸运儿}{8,7,2}{⼲、⾡、⼉}
  \begin{phonetics}{幸运儿}{xing4yun4'er2}
    \definition{s.}{pessoa de sorte}
  \end{phonetics}
\end{entry}

\begin{entry}{幸运抽奖}{8,7,8,9}{⼲、⾡、⼿、⼤}
  \begin{phonetics}{幸运抽奖}{xing4yun4chou1jiang3}
    \definition{s.}{loteria | sorteio}
  \end{phonetics}
\end{entry}

\begin{entry}{幸福}{8,13}{⼲、⽰}
  \begin{phonetics}{幸福}{xing4fu2}[][HSK 3]
    \definition{adj.}{feliz; a vida, a família e outras circunstâncias deixam as pessoas satisfeitas e felizes}
    \definition{s.}{felicidade; bem estar; sensação ou experiência satisfatória e feliz, etc.}
  \end{phonetics}
\end{entry}

%%%%% EOF %%%%%

