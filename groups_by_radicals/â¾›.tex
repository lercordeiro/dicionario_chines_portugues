%%%
%%% Radical "⾛"
%%%

\section*{Radical 156: ``⾛''}\addcontentsline{toc}{section}{Radical 156: ⾛}

\begin{Entry}{走}{7}{⾛}[Kangxi 156]
  \begin{Phonetics}{走}{zou3}[][HSK 1]
    \definition{v.}{andar; caminhar | correr | mover; movimentar; deslocar | sair; partir; ir embora | visitar; fazer uma visita; (entre amigos e familiares) troca de visitas | passar por; atravessar; ultrapassar | vazar; revelar; divulgar | afastar-se do original; alterar ou perder a forma, o sabor, a cor, etc. originais}
  \end{Phonetics}
\end{Entry}

\begin{Entry}{走开}{7,4}{⾛、⼶}
  \begin{Phonetics}{走开}{zou3 kai1}[][HSK 2]
    \definition{v.}{ir embora; fugir; ir para outro lugar}
  \end{Phonetics}
\end{Entry}

\begin{Entry}{走去}{7,5}{⾛、⼛}
  \begin{Phonetics}{走去}{zou3qu4}
    \definition{v.}{caminhar até (para)}
  \end{Phonetics}
\end{Entry}

\begin{Entry}{走过}{7,6}{⾛、⾡}
  \begin{Phonetics}{走过}{zou3 guo4}[][HSK 2]
    \definition{v.}{passar por; perambular}
  \end{Phonetics}
\end{Entry}

\begin{Entry}{走秀}{7,7}{⾛、⽲}
  \begin{Phonetics}{走秀}{zou3xiu4}
    \definition{s.}{desfile de moda}
    \definition{v.}{andar na passarela (em um desfile de moda)}
  \end{Phonetics}
\end{Entry}

\begin{Entry}{走私}{7,7}{⾛、⽲}
  \begin{Phonetics}{走私}{zou3si1}[][HSK 6]
    \definition{v.}{contrabandear; ter um caso ilícito; violar regulamentos alfandegários; evadir inspeções alfandegárias; transportar mercadorias ilegalmente para dentro e para fora do país}
  \end{Phonetics}
\end{Entry}

\begin{Entry}{走进}{7,7}{⾛、⾡}
  \begin{Phonetics}{走进}{zou3 jin4}[][HSK 2]
    \definition{v.}{entrar}
  \end{Phonetics}
\end{Entry}

\begin{Entry}{走势}{7,8}{⾛、⼒}
  \begin{Phonetics}{走势}{zou3shi4}
    \definition{s.}{caminho | tendência}
  \end{Phonetics}
\end{Entry}

\begin{Entry}{走卒}{7,8}{⾛、⼗}
  \begin{Phonetics}{走卒}{zou3zu2}
    \definition{s.}{lacaio (masculino) | peão (isto é, soldado de infantaria) | servo}
  \end{Phonetics}
\end{Entry}

\begin{Entry}{走鬼}{7,9}{⾛、⿁}
  \begin{Phonetics}{走鬼}{zou3gui3}
    \definition{s.}{vendedor ambulante sem licença}
  \end{Phonetics}
\end{Entry}

\begin{Entry}{走索}{7,10}{⾛、⽷}
  \begin{Phonetics}{走索}{zou3suo3}
    \definition{v.}{Acrobacia: andar na corda bamba; dança da corda; caminhada na corda}
  \seealsoref{走绳}{zou3sheng2}
  \end{Phonetics}
\end{Entry}

\begin{Entry}{走绳}{7,11}{⾛、⽷}
  \begin{Phonetics}{走绳}{zou3sheng2}
    \definition{v.}{andar na corda bamba; dança da corda; caminhada na corda; um tipo de acrobacia em que atores andam para frente e para trás em uma corda suspensa e realizam diversas ações}
  \seealsoref{走索}{zou3suo3}
  \end{Phonetics}
\end{Entry}

\begin{Entry}{走路}{7,13}{⾛、⾜}
  \begin{Phonetics}{走路}{zou3 lu4}[][HSK 1]
    \definition{v.}{caminhar; ir a pé; andar em pé sobre a terra | sair; ir embora; partir}
  \end{Phonetics}
\end{Entry}

\begin{Entry}{赶}{10}{⾛}
  \begin{Phonetics}{赶}{gan3}[][HSK 3]
    \definition*{s.}{Sobrenome Gan}
    \definition{prep.}{por; até; até que; até quando; introduzir o momento em que algo aconteceu, indicando que se espera até um determinado momento}
    \definition{v.}{ultrapassar; alcançar | perseguir; correr atrás; tentar alcançar; dar uma corrida; acelerar ou intensificar  | dirigir; conduzir | expulsar; afugentar; afastar | encontrar; deparar-se com; esbarrar em; acontecer; encontrar-se em (uma situação); aproveitar-se de (uma oportunidade) | ir para; participar (atividades com horário marcado)}
  \end{Phonetics}
\end{Entry}

\begin{Entry}{赶上}{10,3}{⾛、⼀}
  \begin{Phonetics}{赶上}{gan3 shang4}[][HSK 6]
    \definition{v.}{alcançar; manter o ritmo com; acompanhar alguém ou o padrão do planejador | chegar a tempo para; ter tempo suficiente; não ser tarde demais | encontrar; topar com; cruzar com; encontrar-se com; acontecer de encontrar; encontrar algo, em um determinado momento ou oportunidade}
  \end{Phonetics}
\end{Entry}

\begin{Entry}{赶不上}{10,4,3}{⾛、⼀、⼀}
  \begin{Phonetics}{赶不上}{gan3 bu5 shang4}[][HSK 6]
    \definition{v.}{ficar para trás; ser incapaz de alcançar; não conseguir alcançar; não conseguir acompanhar | ser tarde demais (para fazer algo); (não) existir tempo suficiente (para fazer algo) |  deixar de ter; ser incapaz de encontrar ou ter a chance de encontrar; não encontrar; não encontrar (boa oportunidade) | não poder ser comparado a}
  \end{Phonetics}
\end{Entry}

\begin{Entry}{赶忙}{10,6}{⾛、⼼}
  \begin{Phonetics}{赶忙}{gan3 mang2}[][HSK 6]
    \definition{adv.}{imediatamente; com pressa; às pressas; rapidamente}
  \end{Phonetics}
\end{Entry}

\begin{Entry}{赶早}{10,6}{⾛、⽇}
  \begin{Phonetics}{赶早}{gan3zao3}
    \definition{adv.}{o mais breve possível | na primeira oportunidade | antes que seja tarde | quanto antes melhor}
  \end{Phonetics}
\end{Entry}

\begin{Entry}{赶快}{10,7}{⾛、⼼}
  \begin{Phonetics}{赶快}{gan3kuai4}[][HSK 3]
    \definition{adv.}{rapidamente; imediatamente; aproveite o momento e acelere o ritmo}
  \end{Phonetics}
\end{Entry}

\begin{Entry}{赶走}{10,7}{⾛、⾛}
  \begin{Phonetics}{赶走}{gan3zou3}
    \definition{v.}{expulsar | voltar atrás}
  \end{Phonetics}
\end{Entry}

\begin{Entry}{赶到}{10,8}{⾛、⼑}
  \begin{Phonetics}{赶到}{gan3 dao4}[][HSK 3]
    \definition{v.}{correr (para algum lugar); apressar-se}
  \end{Phonetics}
\end{Entry}

\begin{Entry}{赶赴}{10,9}{⾛、⾛}
  \begin{Phonetics}{赶赴}{gan3fu4}
    \definition{v.}{apressar}
  \end{Phonetics}
\end{Entry}

\begin{Entry}{赶紧}{10,10}{⾛、⽷}
  \begin{Phonetics}{赶紧}{gan3jin3}[][HSK 3]
    \definition{adv.}{apressadamente; precipitadamente; às pressas; significa agir imediatamente, sem demora}
  \end{Phonetics}
\end{Entry}

\begin{Entry}{赶脚}{10,11}{⾛、⾁}
  \begin{Phonetics}{赶脚}{gan3jiao3}
    \definition{v.}{transportar mercadorias para ganhar a vida (especialmente de burro) | trabalhar como carroceiro ou porteiro}
  \end{Phonetics}
\end{Entry}

\begin{Entry}{赶跑}{10,12}{⾛、⾜}
  \begin{Phonetics}{赶跑}{gan3pao3}
    \definition{v.}{afastar | forçar a saída | repelir}
  \end{Phonetics}
\end{Entry}

\begin{Entry}{赶集}{10,12}{⾛、⾫}
  \begin{Phonetics}{赶集}{gan3ji2}
    \definition{v.}{ir a uma feira | ir ao mercado}
  \end{Phonetics}
\end{Entry}

\begin{Entry}{赶路}{10,13}{⾛、⾜}
  \begin{Phonetics}{赶路}{gan3lu4}
    \definition{v.}{apressar a jornada | apressar-se}
  \end{Phonetics}
\end{Entry}

\begin{Entry}{起}{10}{⾛}
  \begin{Phonetics}{起}{qi3}[][HSK 1]
    \definition{clas.}{caso; instância | lote; grupo}
    \definition{prep.}{de; colocado antes de uma palavra de tempo ou lugar, indica um ponto de partida | por; colocado antes de uma palavra de lugar, indica um lugar por onde passou}
    \definition{v.}{levantar-se; ficar de pé| iniciar; lançar; deicar a posição original | subir; ascender | aparecer; levantar; crescer (bolhas, protuberâncias, brotoeja) | puxar para cima; puxar para fora; tirar o que está guardado ou incorporado | crescer; aumentar | esboçar; elaborar | construir; montar; estabelecer | receber (comprovante) | começar; iniciar; combina com 从 e 由; indica quando, onde e quem começou | buscar; pegar; usado após um verbo, indica movimento para cima | indicar se alguém tem força suficiente ou não; usado após um verbo, indica que a força é suficiente ou insuficiente | indicar que a ação envolve alguém ou algo; equivalente a 及 ou 到 | começar; iniciar; usado depois de um verbo, indica o início de uma ação | juntar; implodir; (informal) usado depois de um verbo, para unir coisas ou fechá-las}
  \seealsoref{从}{cong2}
  \seealsoref{到}{dao4}
  \seealsoref{及}{ji2}
  \seealsoref{由}{you2}
  \end{Phonetics}
\end{Entry}

\begin{Entry}{起飞}{10,3}{⾛、⾶}
  \begin{Phonetics}{起飞}{qi3fei1}[][HSK 2]
    \definition{v.}{decolar; levantar voo | crescer rapidamente; decolar; disparar; metáfora para o rápido desenvolvimento de negócios, economia, etc.}
  \end{Phonetics}
\end{Entry}

\begin{Entry}{起床}{10,7}{⾛、⼴}
  \begin{Phonetics}{起床}{qi3chuang2}[][HSK 1]
    \definition{v.+compl.}{levantar-se; sair da cama; acordar e sair da cama (geralmente pela manhã); levantar-se da posição sentada, deitada ou deitada de bruços, ou sentar-se a partir da posição deitada}
  \end{Phonetics}
\end{Entry}

\begin{Entry}{起来}{10,7}{⾛、⽊}
  \begin{Phonetics}{起来}{qi3 lai2}[][HSK 1]
    \definition{v.+compl.}{levantar-se; passar de posições como deitado, sentado ou ajoelhado para ficar em pé | levantar-se; sair da cama | levantar-se; revoltar-se; rebelar-se; refere-se a ascensão, surgimento, levantamento, etc.}
  \end{Phonetics}
  \begin{Phonetics}{起来}{qi5lai2}
    \definition{v.}{descrever resultados, retratar comportamentos, transmitir movimento}
  \end{Phonetics}
\end{Entry}

\begin{Entry}{起诉}{10,7}{⾛、⾔}
  \begin{Phonetics}{起诉}{qi3 su4}[][HSK 6]
    \definition{v.}{processar; entrar com uma ação judicial}
  \end{Phonetics}
\end{Entry}

\begin{Entry}{起到}{10,8}{⾛、⼑}
  \begin{Phonetics}{起到}{qi3 dao4}[][HSK 5]
    \definition{v.}{ter (um efeito motivador, etc.); desempenhar (um papel estabilizador, etc.)}
  \end{Phonetics}
\end{Entry}

\begin{Entry}{起码}{10,8}{⾛、⽯}
  \begin{Phonetics}{起码}{qi3ma3}[][HSK 5]
    \definition{adj.}{mínimo; elementar; rudimentar}
    \definition{adv.}{mínimamente; pelo menos;}
  \end{Phonetics}
\end{Entry}

\begin{Entry}{起点}{10,9}{⾛、⽕}
  \begin{Phonetics}{起点}{qi3 dian3}[][HSK 6]
    \definition[个]{s.}{ponto de partida (para o tempo ou local do início de algo); o lugar ou hora de início | ponto de partida (para o nível ou base de algo feito inicialmente); refere-se especificamente ao ponto de partida designado em um evento de pista}
  \end{Phonetics}
\end{Entry}

\begin{Entry}{起跳}{10,13}{⾛、⾜}
  \begin{Phonetics}{起跳}{qi3tiao4}
    \definition{v.}{(atletismo) decolar (no início de um salto) | (de preço, salário, etc.) começar (de um determinado nível)}
  \end{Phonetics}
\end{Entry}

\begin{Entry}{趁}{12}{⾛}
  \begin{Phonetics}{趁}{chen4}
    \definition{prep.}{aproveitar-se de; tirar vantagem de (tempo, oportunidade, etc.); indica o tempo e as condições de uso}
    \definition{v.}{ser rico em; possuir}
  \end{Phonetics}
\end{Entry}

\begin{Entry}{超}{12}{⾛}
  \begin{Phonetics}{超}{chao1}[][HSK 6]
    \definition{adj.}{super; extremamente; maior (ou menor) que o padrão geral}
    \definition{v.}{exceder; ultrapassar; vir para a frente por trás; prevalecer | transcender; ir além; não ser sujeito a certas restrições; ir além de um certo intervalo | exceder; superar; exceder o limite prescrito}
  \end{Phonetics}
\end{Entry}

\begin{Entry}{超出}{12,5}{⾛、⼐}
  \begin{Phonetics}{超出}{chao1 chu1}[][HSK 6]
    \definition{v.}{exceder; ultrapassar; ir além (de uma certa quantidade ou intervalo)}
  \end{Phonetics}
\end{Entry}

\begin{Entry}{超市}{12,5}{⾛、⼱}
  \begin{Phonetics}{超市}{chao1shi4}[][HSK 2]
    \definition[家]{s.}{supermercado; abreviação de 超级市场}
  \seealsoref{超级市场}{chao1 ji2 shi4 chang3}
  \end{Phonetics}
\end{Entry}

\begin{Entry}{超级}{12,6}{⾛、⽷}
  \begin{Phonetics}{超级}{chao1ji2}[][HSK 3]
    \definition{adj.}{super; além do nível geral}
    \definition{pref.}{super-; ultra-; hiper-}
  \end{Phonetics}
\end{Entry}

\begin{Entry}{超级市场}{12,6,5,6}{⾛、⽷、⼱、⼟}
  \begin{Phonetics}{超级市场}{chao1 ji2 shi4 chang3}
    \definition[个,间,所,家]{s.}{supermercado; hipermercado}
  \end{Phonetics}
\end{Entry}

\begin{Entry}{超过}{12,6}{⾛、⾡}
  \begin{Phonetics}{超过}{chao1guo4}[][HSK 2]
    \definition{v.}{ultrapassar; superar (algo ou alguém); passar de trás para a frente de alguém ou algo | exceder; ser mais do que; ultrapassar (um padrão)}
  \end{Phonetics}
\end{Entry}

\begin{Entry}{超声}{12,7}{⾛、⼠}
  \begin{Phonetics}{超声}{chao1sheng1}
    \definition{adj.}{ultrasônico}
    \definition{s.}{ultrasom}
  \end{Phonetics}
\end{Entry}

\begin{Entry}{超越}{12,12}{⾛、⾛}
  \begin{Phonetics}{超越}{chao1yue4}[][HSK 5]
    \definition{v.}{ultrapassar; superar; passar por cima; transcender}
  \end{Phonetics}
\end{Entry}

\begin{Entry}{越}{12}{⾛}
  \begin{Phonetics}{越}{yue4}[][HSK 2]
    \definition{adj.}{superior; excede ou ultrapassa o ordinário}
    \definition{adv.}{quanto mais\dots mais; sados juntos, eles formam o formato de "越……越……" para indicar que o grau de uma situação se torna mais sério à medida que se desenvolve; "成年……" para indicar que o grau de uma situação se torna mais sério à medida que o tempo passa}
    \definition{v.}{passar por cima; pular; cruzar | exceder; ultrapassar | estar em um tom alto; estar animado | saquear; pilhar; expoliar; apreender; roubar | passar; passar através; atravessar}
  \seealsoref{越来越……}{yue4 lai2 yue4}
  \seealsoref{越……越……}{yue4 yue4}
  \end{Phonetics}
\end{Entry}

\begin{Entry}{越来越……}{12,7,12}{⾛、⽊、⾛}
  \begin{Phonetics}{越来越……}{yue4 lai2 yue4}[][HSK 2]
    \definition{adv.}{cada vez mais\dots; isso significa que o grau de algo se aprofunda à medida que o tempo passa}
  \end{Phonetics}
\end{Entry}

\begin{Entry}{越……越……}{12,12}{⾛、⾛}
  \begin{Phonetics}{越……越……}{yue4 yue4}[][HSK 2]
    \definition{expr.}{quanto mais\dots tanto mais\dots}
  \end{Phonetics}
\end{Entry}

\begin{Entry}{越障}{12,13}{⾛、⾩}
  \begin{Phonetics}{越障}{yue4zhang4}
    \definition{s.}{curso com obstáculos para treinamento de tropas}
    \definition{v.}{superar obstáculos}
  \end{Phonetics}
\end{Entry}

\begin{Entry}{越境}{12,14}{⾛、⼟}
  \begin{Phonetics}{越境}{yue4jing4}
    \definition{v.}{cruzar uma fronteira (geralmente ilegalmente) | entrar ou sair furtivamente de um país}
  \end{Phonetics}
\end{Entry}

\begin{Entry}{趋}{12}{⾛}
  \begin{Phonetics}{趋}{qu1}
    \definition{v.}{apressar-se | tender para; tender a se tornar | (de um ganso, cobra, etc.) estalar a cabeça e morder as pessoas}
  \end{Phonetics}
\end{Entry}

\begin{Entry}{趋势}{12,8}{⾛、⼒}
  \begin{Phonetics}{趋势}{qu1shi4}[][HSK 4]
    \definition{s.}{tendência; tendência; direção; impulso das coisas que se movem em uma direção ou outra}
  \end{Phonetics}
\end{Entry}

\begin{Entry}{趟}{15}{⾛}
  \begin{Phonetics}{趟}{tang1}
    \definition{v.}{atravessar; andar na grama ou onde não haja caminho | usar arados, capinadores, etc. para virar o solo e remover ervas daninhas | vadear; atravessar a vau; caminhar por águas rasas}[我们趟水去那小岛。===Nós vadeamos até a ilha.]
  \end{Phonetics}
  \begin{Phonetics}{趟}{tang4}[][HSK 6]
    \definition{clas.}{usado para o número de vezes de viagens de ida e volta |  usado para coisas dispostas em fileiras ou tiras | usado para a programação de veículos, navios, etc. que circulam em uma determinada ordem | usado em conjuntos de movimentos de artes marciais}
    \definition{s.}{marcha; procissão; jornada; viagem}
  \end{Phonetics}
\end{Entry}

%%%%% EOF %%%%%

