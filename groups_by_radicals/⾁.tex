%%%
%%% Radical "⾁"
%%%

\section*{Radical 130: ``⾁'' (月、⺼)}\addcontentsline{toc}{section}{Radical 130: ⾁、月、⺼}

\begin{entry}{肉}{6}{⾁}[Kangxi 130]
  \begin{phonetics}{肉}{rou4}[][HSK 1]
    \definition{adj.}{não crocante; mole | lento (em movimento); preguiçoso | carnal; erótico}
    \definition[块]{s.}{carne (especialmente carne de porco) | carne | polpa (da fruta)}
  \end{phonetics}
\end{entry}

\begin{entry}{肉桂}{6,10}{⾁、⽊}
  \begin{phonetics}{肉桂}{rou4gui4}
    \definition{s.}{canela}
  \seealsoref{官桂}{guan1gui4}
  \end{phonetics}
\end{entry}

\begin{entry}{肌}{6}{⾁}
  \begin{phonetics}{肌}{ji1}
    \definition[块,片]{s.}{músculo; carne | pele;}
  \end{phonetics}
\end{entry}

\begin{entry}{肌肉}{6,6}{⾁、⾁}
  \begin{phonetics}{肌肉}{ji1rou4}[][HSK 5]
    \definition[身,块]{s.}{músculo; um dos tecidos básicos dos músculos humanos e de alguns animais, composto principalmente de células musculares fibrosas, pode se contrair, é o movimento do corpo e o corpo de digestão, respiração, circulação, excreção e outros processos fisiológicos da fonte de energia; pode ser dividido em três tipos: músculo liso, músculo esquelético e músculo cardíaco}
  \end{phonetics}
\end{entry}

\begin{entry}{肚}{7}{⾁}
  \begin{phonetics}{肚}{du3}
    \definition{s.}{tripas; entranhas}
  \end{phonetics}
  \begin{phonetics}{肚}{du4}
    \definition{s.}{barriga; abdômen; estômago | tolerância}
  \end{phonetics}
\end{entry}

\begin{entry}{肚子}{7,3}{⾁、⼦}
  \begin{phonetics}{肚子}{du4zi5}[][HSK 4]
    \definition[个,只]{s.}{abdômen; barriguinha; ventre; barriga}
  \end{phonetics}
\end{entry}

\begin{entry}{肝}{7}{⾁}
  \begin{phonetics}{肝}{gan1}[][HSK 6]
    \definition[个]{s.}{fígado; um dos órgãos digestivos dos humanos e dos animais superiores}
  \end{phonetics}
\end{entry}

\begin{entry}{肠}{7}{⾁}
  \begin{phonetics}{肠}{chang2}[][HSK 5]
    \definition[根,段,片]{s.}{intestinos; parte do sistema digestivo | salsicha; linguiça; alimentos de tripas recheadas com carne, peixe, etc. | sentimentos; emoções; humor}
  \end{phonetics}
\end{entry}

\begin{entry}{股}{8}{⾁}
  \begin{phonetics}{股}{gu3}[][HSK 6]
    \definition*{s.}{Sobrenome Gu}
    \definition{clas.}{usado para coisas em tiras, longas e estreitas | usado para gás, odor, força, etc. | Pejorativo: usado para um grupo de pessoas}
    \definition{s.}{coxa; ancas | seção (de um escritório, empresa, etc.); unidades organizacionais em agências governamentais, empresas e grupos | fio; camada | uma das várias partes iguais de propriedade | ação; \emph{stock}; ação do capital social; uma parte igual de fundos ou propriedade | a perna mais longa de um triângulo retângulo}
  \end{phonetics}
\end{entry}

\begin{entry}{肥}{8}{⾁}
  \begin{phonetics}{肥}{fei2}[][HSK 4]
    \definition{adj.}{gordo; gorduroso; contém muita gordura (o oposto de 瘦, geralmente não usado para descrever pessoas) | fértil; rico | solto; largo; folgado; (roupas, etc.) largas (em oposição a 瘦) | lucrativo; rendendo bons lucros}
    \definition{s.}{fertilizante; esterco}
    \definition{v.}{fertilizar; tornar fértil ou obeso | enriquecer com renda ilegal, ilícita}
  \seealsoref{瘦}{shou4}
  \end{phonetics}
\end{entry}

\begin{entry}{肩}{8}{⾁}
  \begin{phonetics}{肩}{jian1}[][HSK 5]
    \definition*{s.}{Sobrenome Jian}
    \definition{s.}{ombro; torso}
    \definition{v.}{assumir; empreender; carregar; suportar; suportar um fardo}
  \end{phonetics}
\end{entry}

\begin{entry}{肩膀}{8,14}{⾁、⾁}
  \begin{phonetics}{肩膀}{jian1bang3}
    \definition{s.}{ombro}
  \end{phonetics}
\end{entry}

\begin{entry}{肯}{8}{⾁}
  \begin{phonetics}{肯}{ken3}[][HSK 6]
    \definition{s.}{carne presa ao osso}
    \definition{v.}{concordar; consentir}
    \definition{v.aux.}{estar disposto a; estar pronto para; para expressar vontade subjetiva; vontade de aceitar}
  \end{phonetics}
\end{entry}

\begin{entry}{肯定}{8,8}{⾁、⼧}
  \begin{phonetics}{肯定}{ken3ding4}[][HSK 5]
    \definition{adj.}{certo; definitivo; positivo; afirmativo | positivo; afirmativo; aceitável}
    \definition{adv.}{com certeza | certamente | definitivamente | afirmativo (resposta)}
    \definition{adv.}{certamente; definitivamente; sem dúvida; sem dúvida alguma}
    \definition{v.}{afirmar; aprovar; confirmar; considerar positivo; reconhecer a existência de algo ou sua autenticidade ou racionalidade (em oposição à 否定)}
  \seealsoref{否定}{fou3ding4}
  \end{phonetics}
\end{entry}

\begin{entry}{肺}{8}{⾁}
  \begin{phonetics}{肺}{fei4}[][HSK 6]
    \definition[叶]{s.}{pulmão | pulmões; órgãos respiratórios de humanos e animais superiores}
  \end{phonetics}
\end{entry}

\begin{entry}{肿}{8}{⾁}
  \begin{phonetics}{肿}{zhong3}[][HSK 6]
    \definition{s.}{inchaço; protuberância}
    \definition{v.}{inchar; estar inchado}
  \end{phonetics}
\end{entry}

\begin{entry}{胃}{9}{⾁}
  \begin{phonetics}{胃}{wei4}[][HSK 5]
    \definition*{s.}{Wei, uma das mansões lunares | Wei, uma das vinte e oito constelações}
    \definition{s.}{estômago; parte do aparelho digestivo}
  \end{phonetics}
\end{entry}

\begin{entry}{胃口}{9,3}{⾁、⼝}
  \begin{phonetics}{胃口}{wei4kou3}
    \definition{s.}{apetite}
  \end{phonetics}
\end{entry}

\begin{entry}{胆}{9}{⾁}
  \begin{phonetics}{胆}{dan3}[][HSK 5]
    \definition[个,颗]{s.}{vesícula biliar | coragem; bravura | um recipiente interno semelhante a uma bexiga; algo que se encaixa dentro de um objeto e pode conter água, ar, etc.}
  \end{phonetics}
\end{entry}

\begin{entry}{胆小}{9,3}{⾁、⼩}
  \begin{phonetics}{胆小}{dan3 xiao3}[][HSK 5]
    \definition{adj.}{tímido; covarde}
  \end{phonetics}
\end{entry}

\begin{entry}{胆小鬼}{9,3,9}{⾁、⼩、⿁}
  \begin{phonetics}{胆小鬼}{dan3xiao3gui3}
    \definition{adj.}{covarde | medroso}
  \end{phonetics}
\end{entry}

\begin{entry}{背}{9}{⾁}
  \begin{phonetics}{背}{bei1}[][HSK 2]
    \definition{clas.}{carga; pacote; para transportar coisas nas costas}
    \definition{v.}{carregar nas costas | suportar; carregar}
  \end{phonetics}
  \begin{phonetics}{背}{bei4}[][HSK 3]
    \definition{adj.}{azarado | fora do caminho; um lugar muito distante do centro movimentado, onde poucas pessoas aparecem | deficiente auditivo}
    \definition{s.}{parte posterior do corpo; costas; coluna vertebral; parte do tronco entre os ombros e a região lombar | parte de trás de um objeto}
    \definition{v.}{afastar-se; virar as costas | decorar; memorizar; recitar de memória | esconder algo de; fazer algo em segredo | sair, ir embora; partir; abandonar | quebrar; violar; agir de forma contrária a}
  \end{phonetics}
\end{entry}

\begin{entry}{背心}{9,4}{⾁、⼼}
  \begin{phonetics}{背心}{bei4 xin1}[][HSK 6]
    \definition[件]{s.}{colete; vestimenta sem mangas; \emph{tops} sem gola e sem mangas}
  \end{phonetics}
\end{entry}

\begin{entry}{背包}{9,5}{⾁、⼓}
  \begin{phonetics}{背包}{bei1 bao1}[][HSK 5]
    \definition[个]{s.}{mochila; mochila de ataque; mochila de infantaria; pacotes de roupas carregados nas costas quando marcham}
  \end{phonetics}
\end{entry}

\begin{entry}{背后}{9,6}{⾁、⼝}
  \begin{phonetics}{背后}{bei4 hou4}[][HSK 3]
    \definition{s.}{parte posterior; parte de trás; traseira | pelas costas de alguém}
  \end{phonetics}
\end{entry}

\begin{entry}{背着}{9,11}{⾁、⽬}
  \begin{phonetics}{背着}{bei4 zhe5}[][HSK 6]
    \definition{adv.}{pelas costas; atrás de alguém}
    \definition{v.}{carregar nas costas}
  \end{phonetics}
\end{entry}

\begin{entry}{背景}{9,12}{⾁、⽇}
  \begin{phonetics}{背景}{bei4jing3}[][HSK 4]
    \definition[种]{s.}{pano de fundo; fundo; cenário de teatro, filme ou drama de TV | fundo; cenário que permeia a imagem principal na tela | condições sociais; ambientes históricos (significativamente influentes para algo ou alguém) | poder que dá suporte a alguém}
  \end{phonetics}
\end{entry}

\begin{entry}{胖}{9}{⾁}
  \begin{phonetics}{胖}{pan2}
    \definition{adj.}{saudável}
  \end{phonetics}
  \begin{phonetics}{胖}{pang4}[][HSK 3]
    \definition{adj.}{gordo; robusto; rechonchudo; (corpo humano) com muita gordura ou carne (em oposição a 瘦)}
  \seealsoref{瘦}{shou4}
  \end{phonetics}
\end{entry}

\begin{entry}{胖子}{9,3}{⾁、⼦}
  \begin{phonetics}{胖子}{pang4 zi5}[][HSK 4]
    \definition{s.}{obeso; gordo; pessoa gorda}
  \end{phonetics}
\end{entry}

\begin{entry}{胚}{9}{⾁}
  \begin{phonetics}{胚}{pei1}
    \definition{s.}{embrião}
  \end{phonetics}
\end{entry}

\begin{entry}{胜}{9}{⾁}
  \begin{phonetics}{胜}{sheng4}[][HSK 3]
    \definition{adj.}{soberbo; maravilhoso; adorável}
    \definition[场]{s.}{vitória; sucesso | penteado de mulher; joias usadas pelas mulheres na antiguidade}
    \definition{v.}{vencer (oposto de 负, 败) | derrotar | (frequentemente seguido por 于, etc.) superar; ser superior a; levar a melhor sobre | vencer; ter sucesso; derrotar o adversário | ultrapassar; ser superior ao outro | suportar; ser capaz de suportar ou aguentar}
  \seealsoref{败}{bai4}
  \seealsoref{负}{fu4}
  \seealsoref{于}{yu2}
  \end{phonetics}
\end{entry}

\begin{entry}{胜负}{9,6}{⾁、⾙}
  \begin{phonetics}{胜负}{sheng4fu4}[][HSK 5]
    \definition{s.}{vitória ou derrota; sucesso ou fracasso}
  \end{phonetics}
\end{entry}

\begin{entry}{胜利}{9,7}{⾁、⼑}
  \begin{phonetics}{胜利}{sheng4li4}[][HSK 3]
    \definition{adv.}{com sucesso; triunfantemente; atingir o objetivo previsto}
    \definition{v.}{ganhar; vencer; triunfar; ter sucesso}
  \end{phonetics}
\end{entry}

\begin{entry}{胜算}{9,14}{⾁、⽵}
  \begin{phonetics}{胜算}{sheng4suan4}
    \definition{s.}{probabilidade de sucesso | estratégia que garante o sucesso}
    \definition{v.}{ter certeza do sucesso}
  \end{phonetics}
\end{entry}

\begin{entry}{胡}{9}{⾁}
  \begin{phonetics}{胡}{hu2}
    \definition*{s.}{Sobrenome Hu}
    \definition{adj.}{introduzidos de nacionalidades do norte e do oeste ou do exterior | nos tempos antigos, o termo "Oriente e Ocidente" se referia às minorias étnicas do norte e do oeste, e também, de modo geral, às pessoas do exterior}
    \definition{adv.}{imprudentemente; desenfreadamente; escandalosamente; sem lei, ordem ou razão}
    \definition{pron.}{Por que?; palavras interrogativas: 为什么, 何故}
    \definition{s.}{nos tempos antigos, geralmente se referia às minorias étnicas do norte e do oeste | violino chinês | barba; bigode}
  \seealsoref{何故}{he2gu4}
  \seealsoref{为什么}{wei4shen2me5}
  \end{phonetics}
\end{entry}

\begin{entry}{胡子}{9,3}{⾁、⼦}
  \begin{phonetics}{胡子}{hu2 zi5}[][HSK 5]
    \definition[团,根,个]{s.}{barba; bigode | bandido; salteador}
  \end{phonetics}
\end{entry}

\begin{entry}{胡同}{9,6}{⾁、⼝}
  \begin{phonetics}{胡同}{hu2tong5}
    \definition[条,个]{s.}{beco; rua pequena}
  \end{phonetics}
\end{entry}

\begin{entry}{胡同儿}{9,6,2}{⾁、⼝、⼉}
  \begin{phonetics}{胡同儿}{hu2 tong4r5}[][HSK 5]
    \definition{s.}{beco}
  \end{phonetics}
\end{entry}

\begin{entry}{胡萝卜}{9,11,2}{⾁、⾋、⼘}
  \begin{phonetics}{胡萝卜}{hu2luo2bo5}
    \definition{s.}{cenoura}
  \end{phonetics}
\end{entry}

\begin{entry}{胡琴}{9,12}{⾁、⽟}
  \begin{phonetics}{胡琴}{hu2qin2}
    \definition{s.}{huqin, um termo geral para certos instrumentos de arco de duas cordas, como 二胡, 京胡, etc. | família de violinos chineses de duas cordas, com caixa de ressonância de madeira revestida de pele de cobra e arco de bambu com corda de crina de cavalo}
  \seealsoref{二胡}{er4hu2}
  \seealsoref{京胡}{jing1hu2}
  \end{phonetics}
\end{entry}

\begin{entry}{胳}{10}{⾁}
  \begin{phonetics}{胳}{ga1}
    \definition{s.}{usado em 胳肢窝}
  \seealsoref{胳肢窝}{ga1 zhi1 wo1}
  \end{phonetics}
  \begin{phonetics}{胳}{ge1}
    \definition{s.}{axila; sovaco}
  \end{phonetics}
  \begin{phonetics}{胳}{ge2}
    \definition{v.}{usado em 胳肢}
  \seealsoref{胳肢}{ge2zhi5}
  \end{phonetics}
\end{entry}

\begin{entry}{胳肢}{10,8}{⾁、⾁}
  \begin{phonetics}{胳肢}{ge2zhi5}
    \definition{v.}{(dialeto) fazer cócegas}
  \end{phonetics}
\end{entry}

\begin{entry}{胳肢窝}{10,8,12}{⾁、⾁、⽳}
  \begin{phonetics}{胳肢窝}{ga1 zhi1 wo1}
    \definition{s.}{axila; sovaco; também escrito 夹肢窝}
  \seealsoref{夹肢窝}{jia1 zhi1 wo1}
  \end{phonetics}
\end{entry}

\begin{entry}{胶}{10}{⾁}
  \begin{phonetics}{胶}{jiao1}
    \definition*{s.}{Sobrenome Jiao}
    \definition{adj.}{pegajoso; viscoso; grudento}
    \definition{s.}{cola; goma; adesivo | borracha | gel; colóide}
    \definition{v.}{colar com cola | colar; grudar}
  \end{phonetics}
\end{entry}

\begin{entry}{胶水}{10,4}{⾁、⽔}
  \begin{phonetics}{胶水}{jiao1shui3}[][HSK 5]
    \definition[瓶]{s.}{cola; mucilagem; cola líquida}
  \end{phonetics}
\end{entry}

\begin{entry}{胶卷}{10,8}{⾁、⼙}
  \begin{phonetics}{胶卷}{jiao1juan3}
    \definition{s.}{filme | rolo de filme}
  \end{phonetics}
\end{entry}

\begin{entry}{胶带}{10,9}{⾁、⼱}
  \begin{phonetics}{胶带}{jiao1 dai4}[][HSK 5]
    \definition[卷]{s.}{fita adesiva | fita de gravação | correia de borracha}
  \end{phonetics}
\end{entry}

\begin{entry}{胸}{10}{⾁}
  \begin{phonetics}{胸}{xiong1}
    \definition{s.}{peito | tórax}
  \end{phonetics}
\end{entry}

\begin{entry}{胸部}{10,10}{⾁、⾢}
  \begin{phonetics}{胸部}{xiong1 bu4}[][HSK 4]
    \definition{s.}{peito; tórax; seios}
  \end{phonetics}
\end{entry}

\begin{entry}{能}{10}{⾁}
  \begin{phonetics}{能}{neng2}[][HSK 1]
    \definition*{s.}{Sobrenome Neng}
    \definition{adv.}{talvez}
    \definition{s.}{habilidade; capacidade; competência | potência; energia; em física, refere-se à energia}
    \definition{v.}{poder fazer; ser capaz de | ser possível | entre 不 \dots 不 para expressar obrigação, certeza ou grande probabilidade | poder; ter permissão para | ser bom em fazer algo | permitir}
  \end{phonetics}
\end{entry}

\begin{entry}{能力}{10,2}{⾁、⼒}
  \begin{phonetics}{能力}{neng2li4}[][HSK 3]
    \definition[个,种]{s.}{habilidade; capacidade; aptidão; as condições subjetivas para ser competente para uma tarefa}
  \end{phonetics}
\end{entry}

\begin{entry}{能上能下}{10,3,10,3}{⾁、⼀、⾁、⼀}
  \begin{phonetics}{能上能下}{neng2shang4neng2xia4}
    \definition{s.}{pronto para aceitar qualquer trabalho, alto ou baixo}
  \end{phonetics}
\end{entry}

\begin{entry}{能干}{10,3}{⾁、⼲}
  \begin{phonetics}{能干}{neng2gan4}[][HSK 4]
    \definition{adj.}{apto; capaz; competente}
  \end{phonetics}
\end{entry}

\begin{entry}{能不能}{10,4,10}{⾁、⼀、⾁}
  \begin{phonetics}{能不能}{neng2 bu4 neng2}[][HSK 3]
    \definition{adv.}{pode ou não pode\dots?}
  \end{phonetics}
\end{entry}

\begin{entry}{能够}{10,11}{⾁、⼣}
  \begin{phonetics}{能够}{neng2 gou4}[][HSK 2]
    \definition{v.}{poder; ser capaz de; indica que possui uma determinada capacidade ou que atingiu um determinado nível de eficiência | poder; ser capaz de; indica que algo é permitido sob certas condições ou por motivos razoáveis}
  \end{phonetics}
\end{entry}

\begin{entry}{能量}{10,12}{⾁、⾥}
  \begin{phonetics}{能量}{neng2liang4}[][HSK 5]
    \definition[种]{s.}{energia; quantidade de energia; Uma grandeza física que mede a capacidade da matéria de realizar trabalho | capacidade; competências; capacidade e papel que uma pessoa pode desempenhar}
  \end{phonetics}
\end{entry}

\begin{entry}{脂}{10}{⾁}
  \begin{phonetics}{脂}{zhi1}
    \definition*{s.}{Sobrenome Zhi}
    \definition{s.}{gordura; graxa; sebo | (cosméticos) rouge | (cosméticos) baton; protetor labial}
  \end{phonetics}
\end{entry}

\begin{entry}{脂麻}{10,11}{⾁、⿇}
  \begin{phonetics}{脂麻}{zhi1ma5}
    \variantof{芝麻}
  \end{phonetics}
\end{entry}

\begin{entry}{脆}{10}{⾁}
  \begin{phonetics}{脆}{cui4}[][HSK 5]
    \definition{adj.}{frágil; quebradiço (oposto a 韧) | crocante | (voz) clara; nítida | puro}
  \seealsoref{韧}{ren4}
  \end{phonetics}
\end{entry}

\begin{entry}{脏}{10}{⾁}
  \begin{phonetics}{脏}{zang1}[][HSK 2]
    \definition{adj.}{sujo; imundo | imundo; metáfora para vulgaridade e obscenidade}
    \definition{v.}{tornar algo sujo ou impuro}
  \end{phonetics}
  \begin{phonetics}{脏}{zang4}
    \definition[处]{s.}{vísceras; órgãos internos do corpo, geralmente o coração, o fígado, o baço, os pulmões e os rins; um termo geral para órgãos nas cavidades torácica e abdominal de humanos ou animais | (anatomia) órgão; a medicina tradicional chinesa chama o coração, o fígado, o baço, os pulmões e os rins de órgãos internos}
  \end{phonetics}
\end{entry}

\begin{entry}{脏土}{10,3}{⾁、⼟}
  \begin{phonetics}{脏土}{zang1tu3}
    \definition{s.}{solo sujo | lama | lixo}
  \end{phonetics}
\end{entry}

\begin{entry}{脏字}{10,6}{⾁、⼦}
  \begin{phonetics}{脏字}{zang1zi4}
    \definition{s.}{obscenidade}
  \end{phonetics}
\end{entry}

\begin{entry}{脏病}{10,10}{⾁、⽧}
  \begin{phonetics}{脏病}{zang1bing4}
    \definition{s.}{doença venérea}
  \end{phonetics}
\end{entry}

\begin{entry}{脏脏}{10,10}{⾁、⾁}
  \begin{phonetics}{脏脏}{zang1zang1}
    \definition{adj.}{sujo}
  \end{phonetics}
\end{entry}

\begin{entry}{脏煤}{10,13}{⾁、⽕}
  \begin{phonetics}{脏煤}{zang1mei2}
    \definition{s.}{carvão sujo | sujeira (de uma mina de carvão)}
  \end{phonetics}
\end{entry}

\begin{entry}{脏器}{10,16}{⾁、⼝}
  \begin{phonetics}{脏器}{zang4qi4}
    \definition{s.}{órgãos internos}
  \end{phonetics}
\end{entry}

\begin{entry}{脏辫}{10,17}{⾁、⾟}
  \begin{phonetics}{脏辫}{zang1bian4}
    \definition{s.}{\emph{dreadlocks}}
  \end{phonetics}
\end{entry}

\begin{entry}{脑}{10}{⾁}
  \begin{phonetics}{脑}{nao3}
    \definition{s.}{(fisiologia) cérebro | tofu;  substância branca semelhante ao cérebro ou à medula espinhal cerebral | cabeça | a essência de um objeto}
  \end{phonetics}
\end{entry}

\begin{entry}{脑子}{10,3}{⾁、⼦}
  \begin{phonetics}{脑子}{nao3 zi5}[][HSK 5]
    \definition[个]{s.}{cérebro | mente; cabeça; cérebro; inteligência; poder mental; refere-se à capacidade de pensar, memorizar, raciocinar, etc.; inteligência}
  \end{phonetics}
\end{entry}

\begin{entry}{脑瓜}{10,5}{⾁、⽠}
  \begin{phonetics}{脑瓜}{nao3gua1}
    \definition{s.}{crânio | cérebro | cabeça | mente | mentalidade | ideia}
  \seealsoref{脑瓜子}{nao3gua1zi5}
  \end{phonetics}
\end{entry}

\begin{entry}{脑瓜子}{10,5,3}{⾁、⽠、⼦}
  \begin{phonetics}{脑瓜子}{nao3gua1zi5}
    \definition{s.}{crânio | cérebro | cabeça | mente | mentalidade | ideia}
  \seealsoref{脑瓜}{nao3gua1}
  \end{phonetics}
\end{entry}

\begin{entry}{脑袋}{10,11}{⾁、⾐}
  \begin{phonetics}{脑袋}{nao3dai5}[][HSK 4]
    \definition[颗,个]{s.}{cabeça; a parte mais alta do corpo humano ou a parte mais alta de um animal que contém órgãos como a boca, o nariz, os olhos etc. | mente; cérebro; capacidade de pensar, lembrar, etc.}
  \end{phonetics}
\end{entry}

\begin{entry}{脖}{11}{⾁}
  \begin{phonetics}{脖}{bo2}
    \definition[个]{s.}{pescoço | em forma de pescoço | parte semelhante ao pescoço}
  \end{phonetics}
\end{entry}

\begin{entry}{脖子}{11,3}{⾁、⼦}
  \begin{phonetics}{脖子}{bo2zi5}
    \definition[个]{s.}{pescoço}
  \end{phonetics}
\end{entry}

\begin{entry}{脚}{11}{⾁}
  \begin{phonetics}{脚}{jiao3}[][HSK 2]
    \definition{clas.}{usado para chutes}
    \definition[只,双]{s.}{pé; a parte inferior das pernas de pessoas ou animais, que entra em contato com o solo | base; pé; a parte inferior do objeto | antigamente, referia-se ao trabalho físico de transporte de cargas | resíduos; sobras}
  \end{phonetics}
  \begin{phonetics}{脚}{jue2}
    \variantof{角}
  \end{phonetics}
\end{entry}

\begin{entry}{脚步}{11,7}{⾁、⽌}
  \begin{phonetics}{脚步}{jiao3 bu4}[][HSK 5]
    \definition{s.}{pé; passo; pisada; refere-se ao movimento das pernas ao caminhar | ritmo; passo; distância entre os pés dianteiros e traseiros ao caminhar}
  \end{phonetics}
\end{entry}

\begin{entry}{脱}{11}{⾁}
  \begin{phonetics}{脱}{tuo1}[][HSK 4]
    \definition{conj.}{se; no caso;}
    \definition{v.}{(cabelo, pele) soltar-se; desprender-se; cair | retirar peça de roupa do corpo | sair de; escapar de | perder (palavras) | livrar-se de algo}
  \end{phonetics}
\end{entry}

\begin{entry}{脱毛}{11,4}{⾁、⽑}
  \begin{phonetics}{脱毛}{tuo1mao2}
    \definition{s.}{depilação}
    \definition{v.}{perder cabelo ou penas | depilar | fazer a barba}
  \end{phonetics}
\end{entry}

\begin{entry}{脱险}{11,9}{⾁、⾩}
  \begin{phonetics}{脱险}{tuo1xian3}
    \definition{v.}{sair do perigo}
  \end{phonetics}
\end{entry}

\begin{entry}{脱离}{11,10}{⾁、⼇}
  \begin{phonetics}{脱离}{tuo1li2}[][HSK 5]
    \definition{v.}{separar-se; divorciar-se; afastar-se; sair (de um determinado ambiente ou situação); romper (uma determinada relação)}
  \end{phonetics}
\end{entry}

\begin{entry}{脸}{11}{⾁}
  \begin{phonetics}{脸}{lian3}[][HSK 2]
    \definition[张,个]{s.}{rosto (de pessoas ou animais); a parte frontal da cabeça, da testa ao queixo | parte frontal de algo | cara; autoestima; aparência | rosto; expressões faciais}
  \end{phonetics}
\end{entry}

\begin{entry}{脸色}{11,6}{⾁、⾊}
  \begin{phonetics}{脸色}{lian3 se4}[][HSK 5]
    \definition{s.}{aparência; tez; cor da pele | aparência; expressão facial | (indicando a condição física de alguém) aparência; cor}
  \end{phonetics}
\end{entry}

\begin{entry}{脸盆}{11,9}{⾁、⽫}
  \begin{phonetics}{脸盆}{lian3 pen2}[][HSK 5]
    \definition[个]{s.}{lavatório; bacia para lavar as mãos e o rosto}
  \end{phonetics}
\end{entry}

\begin{entry}{脾}{12}{⾁}
  \begin{phonetics}{脾}{pi2}
    \definition{s.}{baço}
  \end{phonetics}
\end{entry}

\begin{entry}{脾气}{12,4}{⾁、⽓}
  \begin{phonetics}{脾气}{pi2qi5}[][HSK 5]
    \definition[发,个]{s.}{temperamento; disposição; referindo-se ao caráter de uma pessoa | mau humor; temperamento irascível}
  \end{phonetics}
\end{entry}

\begin{entry}{腰}{13}{⾁}
  \begin{phonetics}{腰}{yao1}[][HSK 4]
    \definition*{s.}{Sobrenome Yao}
    \definition[个]{s.}{cintura; região lombar | cós | bolso | parte do meio das coisas | lombo}
  \end{phonetics}
\end{entry}

\begin{entry}{腰包}{13,5}{⾁、⼓}
  \begin{phonetics}{腰包}{yao1bao1}
    \definition{s.}{pochete | bolso}
  \end{phonetics}
\end{entry}

\begin{entry}{腰椎}{13,12}{⾁、⽊}
  \begin{phonetics}{腰椎}{yao1zhui1}
    \definition{s.}{vértebra lombar (espinha dorsal inferior)}
  \end{phonetics}
\end{entry}

\begin{entry}{腿}{13}{⾁}
  \begin{phonetics}{腿}{tui3}[][HSK 2]
    \definition[条,双]{s.}{perna; as partes dos humanos e dos animais que sustentam o corpo e permitem caminhar | um suporte em forma de perna; a parte inferior de um objeto que atua como uma perna e serve de suporte | presunto}
  \end{phonetics}
\end{entry}

\begin{entry}{腿号}{13,5}{⾁、⼝}
  \begin{phonetics}{腿号}{tui3hao4}
    \definition{s.}{anilha numerada (por exemplo, usada para identificar pássaros)}
  \seealsoref{腿号箍}{tui3hao4gu1}
  \end{phonetics}
\end{entry}

\begin{entry}{腿号箍}{13,5,14}{⾁、⼝、⽵}
  \begin{phonetics}{腿号箍}{tui3hao4gu1}
    \definition{s.}{anilha numerada (por exemplo, usada para identificar pássaros)}
  \seealsoref{腿号}{tui3hao4}
  \end{phonetics}
\end{entry}

\begin{entry}{膜}{14}{⾁}
  \begin{phonetics}{膜}{mo2}[][HSK 6]
    \definition[张]{s.}{membrana | filme; revestimento fino}
  \end{phonetics}
\end{entry}

\begin{entry}{膜拜}{14,9}{⾁、⼿}
  \begin{phonetics}{膜拜}{mo2bai4}
    \definition{v.}{ajoelhar-se e se curvar com as mãos unidas no nível da testa | ter ou mostrar sentimentos fortes de respeito e admiração por um deus}
  \end{phonetics}
\end{entry}

\begin{entry}{膨}{16}{⾁}
  \begin{phonetics}{膨}{peng2}
    \definition{v.}{inchar; inflar | expandir; aumentar o comprimento ou o volume de um objeto}
  \end{phonetics}
\end{entry}

\begin{entry}{膨胀}{16,8}{⾁、⾁}
  \begin{phonetics}{膨胀}{peng2zhang4}
    \definition{v.}{expandir | inflar | inchar}
  \end{phonetics}
\end{entry}

%%%%% EOF %%%%%

