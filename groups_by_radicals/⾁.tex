%%%
%%% Radical "⾁"
%%%

\section*{Radical 130: ``⾁'' (月、⺼)}\addcontentsline{toc}{section}{Radical 130: ⾁、月、⺼}

\begin{entry}{肉}{6}{⾁}[Kangxi 130]
  \begin{phonetics}{肉}{rou4}[][HSK 1]
    \definition{s.}{carne | polpa de uma fruta}
  \end{phonetics}
\end{entry}

\begin{entry}{肉桂}{6,10}{⾁、⽊}
  \begin{phonetics}{肉桂}{rou4gui4}
    \definition{s.}{canela}
  \seealsoref{官桂}{guan1gui4}
  \end{phonetics}
\end{entry}

\begin{entry}{肌肉}{6,6}{⾁、⾁}
  \begin{phonetics}{肌肉}{ji1rou4}[][HSK 5]
    \definition[身,块]{s.}{músculo; um dos tecidos básicos dos músculos humanos e de alguns animais, composto principalmente de células musculares fibrosas, pode se contrair, é o movimento do corpo e o corpo de digestão, respiração, circulação, excreção e outros processos fisiológicos da fonte de energia; pode ser dividido em três tipos: músculo liso, músculo esquelético e músculo cardíaco}
  \end{phonetics}
\end{entry}

\begin{entry}{肚}{7}{⾁}
  \begin{phonetics}{肚}{du3}
    \definition{s.}{tripas | entranhas}
  \end{phonetics}
  \begin{phonetics}{肚}{du4}
    \definition{s.}{barriga}
  \end{phonetics}
\end{entry}

\begin{entry}{肚子}{7,3}{⾁、⼦}
  \begin{phonetics}{肚子}{du4zi5}[][HSK 4]
    \definition[个,只]{s.}{abdômen; barriguinha; ventre; barriga}
  \end{phonetics}
\end{entry}

\begin{entry}{肠}{7}{⾁}
  \begin{phonetics}{肠}{chang2}[][HSK 5]
    \definition{s.}{intestinos | salsicha; linguiça | coração; sentimentos; emoções}
  \end{phonetics}
\end{entry}

\begin{entry}{肥}{8}{⾁}
  \begin{phonetics}{肥}{fei2}[][HSK 4]
    \definition*{s.}{sobrenome Fei}
    \definition{adj.}{gordo; gorduroso | fértil; rico | solto e largo; folgado | inchaço (de órgão interno)}
    \definition{s.}{fertilizante; esterco}
    \definition{v.}{fertilizar; adubar | enriquecer com renda ilegal, ilícita}
  \end{phonetics}
\end{entry}

\begin{entry}{肩}{8}{⾁}
  \begin{phonetics}{肩}{jian1}[][HSK 5]
    \definition*{s.}{sobrenome Jian}
    \definition{s.}{ombro; torso}
    \definition{v.}{assumir; empreender; carregar; suportar; suportar um fardo}
  \end{phonetics}
\end{entry}

\begin{entry}{肩膀}{8,14}{⾁、⾁}
  \begin{phonetics}{肩膀}{jian1bang3}
    \definition{s.}{ombro}
  \end{phonetics}
\end{entry}

\begin{entry}{肯定}{8,8}{⾁、⼧}
  \begin{phonetics}{肯定}{ken3ding4}[][HSK 5]
    \definition{adj.}{certo; definitivo; positivo; afirmativo | positivo; afirmativo; aceitável}
    \definition{adv.}{com certeza | certamente | definitivamente | afirmativo (resposta)}
    \definition{adv.}{certamente; definitivamente; sem dúvida; sem dúvida alguma}
    \definition{v.}{afirmar; aprovar; confirmar; considerar positivo; reconhecer a existência de algo ou sua autenticidade ou racionalidade (em oposição à ``否定'')}
  \seealsoref{否定}{fou3ding4}
  \end{phonetics}
\end{entry}

\begin{entry}{胃口}{9,3}{⾁、⼝}
  \begin{phonetics}{胃口}{wei4kou3}
    \definition{s.}{apetite}
  \end{phonetics}
\end{entry}

\begin{entry}{胆}{9}{⾁}
  \begin{phonetics}{胆}{dan3}[][HSK 5]
    \definition[个,颗]{s.}{vesícula biliar | coragem; bravura | um recipiente interno semelhante a uma bexiga; algo que se encaixa dentro de um objeto e pode conter água, ar, etc.}
  \end{phonetics}
\end{entry}

\begin{entry}{胆小}{9,3}{⾁、⼩}
  \begin{phonetics}{胆小}{dan3 xiao3}[][HSK 5]
    \definition{adj.}{tímido; covarde}
  \end{phonetics}
\end{entry}

\begin{entry}{胆小鬼}{9,3,9}{⾁、⼩、⿁}
  \begin{phonetics}{胆小鬼}{dan3xiao3gui3}
    \definition{adj.}{covarde | medroso}
  \end{phonetics}
\end{entry}

\begin{entry}{背}{9}{⾁}
  \begin{phonetics}{背}{bei1}[][HSK 2]
    \definition{v.}{estar sobrecarregado | carregar nas costas ou no ombro}
  \end{phonetics}
  \begin{phonetics}{背}{bei4}[][HSK 3]
    \definition{adv.}{a parte de trás de um corpo ou objeto}
    \definition{s.}{costas | (gíria) azarado}
    \definition{v.}{esconder algo de | decorar | recitar de memória | virar as costas}
  \end{phonetics}
\end{entry}

\begin{entry}{背包}{9,5}{⾁、⼓}
  \begin{phonetics}{背包}{bei1 bao1}[][HSK 5]
    \definition[个]{s.}{mochila; mochila de ataque; mochila de infantaria; pacotes de roupas carregados nas costas quando marcham}
  \end{phonetics}
\end{entry}

\begin{entry}{背后}{9,6}{⾁、⼝}
  \begin{phonetics}{背后}{bei4 hou4}[][HSK 3]
    \definition{s.}{parte de trás | traseira | nas costas de alguém}
  \end{phonetics}
\end{entry}

\begin{entry}{背景}{9,12}{⾁、⽇}
  \begin{phonetics}{背景}{bei4jing3}[][HSK 4]
    \definition[种]{s.}{pano de fundo; fundo; cenário de teatro, filme ou drama de TV | fundo; cenário que permeia a imagem principal na tela | condições sociais; ambientes históricos (significativamente influentes para algo ou alguém) | poder que dá suporte a alguém}
  \end{phonetics}
\end{entry}

\begin{entry}{胖}{9}{⾁}
  \begin{phonetics}{胖}{pan2}
    \definition{adj.}{saudável}
  \end{phonetics}
  \begin{phonetics}{胖}{pang4}[][HSK 3]
    \definition{adj.}{gordo; robusto; rechonchudo}
  \end{phonetics}
\end{entry}

\begin{entry}{胖子}{9,3}{⾁、⼦}
  \begin{phonetics}{胖子}{pang4 zi5}[][HSK 4]
    \definition{s.}{obeso; gordo; pessoa gorda}
  \end{phonetics}
\end{entry}

\begin{entry}{胚}{9}{⾁}
  \begin{phonetics}{胚}{pei1}
    \definition{s.}{embrião}
  \end{phonetics}
\end{entry}

\begin{entry}{胜}{9}{⾁}
  \begin{phonetics}{胜}{sheng4}[][HSK 3]
    \definition{adj.}{soberbo; maravilhoso; adorável}
    \definition[场]{s.}{vitória; sucesso | penteado de mulher}
    \definition{v.}{ganhar; derrotar; vencer; ter sucesso | superar; ser superior a; levar a melhor sobre | ser igual a; poder suportar}
  \end{phonetics}
\end{entry}

\begin{entry}{胜利}{9,7}{⾁、⼑}
  \begin{phonetics}{胜利}{sheng4li4}[][HSK 3]
    \definition{adv.}{com sucesso; triunfantemente}
    \definition[场,个]{s.}{vitória; triunfo; sucesso}
    \definition{v.}{ganhar; vencer; triunfar; ter sucesso}
  \end{phonetics}
\end{entry}

\begin{entry}{胜算}{9,14}{⾁、⽵}
  \begin{phonetics}{胜算}{sheng4suan4}
    \definition{s.}{probabilidade de sucesso | estratégia que garante o sucesso}
    \definition{v.}{ter certeza do sucesso}
  \end{phonetics}
\end{entry}

\begin{entry}{胡子}{9,3}{⾁、⼦}
  \begin{phonetics}{胡子}{hu2 zi5}[][HSK 5]
    \definition[团,根,个]{s.}{barba; bigode | bandido; salteador}
  \end{phonetics}
\end{entry}

\begin{entry}{胡同儿}{9,6,2}{⾁、⼝、⼉}
  \begin{phonetics}{胡同儿}{hu2 tong4r5}[][HSK 5]
    \definition{s.}{beco; via; rua}
  \end{phonetics}
\end{entry}

\begin{entry}{胡萝卜}{9,11,2}{⾁、⾋、⼘}
  \begin{phonetics}{胡萝卜}{hu2luo2bo5}
    \definition{s.}{cenoura}
  \end{phonetics}
\end{entry}

\begin{entry}{胳肢窝}{10,8,12}{⾁、⾁、⽳}
  \begin{phonetics}{胳肢窝}{ga1 zhi1 wo1}
    \definition{s.}{axila; sovaco; também escrito ``夹肢窝''}
  \seealsoref{夹肢窝}{jia1 zhi1 wo1}
  \end{phonetics}
\end{entry}

\begin{entry}{胶水}{10,4}{⾁、⽔}
  \begin{phonetics}{胶水}{jiao1shui3}[][HSK 5]
    \definition[瓶]{s.}{cola; mucilagem; cola líquida}
  \end{phonetics}
\end{entry}

\begin{entry}{胶卷}{10,8}{⾁、⼙}
  \begin{phonetics}{胶卷}{jiao1juan3}
    \definition{s.}{filme | rolo de filme}
  \end{phonetics}
\end{entry}

\begin{entry}{胶带}{10,9}{⾁、⼱}
  \begin{phonetics}{胶带}{jiao1 dai4}[][HSK 5]
    \definition[卷]{s.}{fita adesiva | fita de gravação | correia de borracha}
  \end{phonetics}
\end{entry}

\begin{entry}{胸}{10}{⾁}
  \begin{phonetics}{胸}{xiong1}
    \definition{s.}{peito | tórax}
  \end{phonetics}
\end{entry}

\begin{entry}{胸部}{10,10}{⾁、⾢}
  \begin{phonetics}{胸部}{xiong1 bu4}[][HSK 4]
    \definition{s.}{peito; tórax; seios}
  \end{phonetics}
\end{entry}

\begin{entry}{能}{10}{⾁}
  \begin{phonetics}{能}{neng2}[][HSK 1]
    \definition*{s.}{sobrenome Neng}
    \definition{adv.}{talvez}
    \definition{s.}{(física)nenergia | habilidade}
    \definition{v.}{poder | ser capaz de}
  \end{phonetics}
\end{entry}

\begin{entry}{能力}{10,2}{⾁、⼒}
  \begin{phonetics}{能力}{neng2li4}[][HSK 3]
    \definition{s.}{habilidade; capacidade; aptidão}
  \end{phonetics}
\end{entry}

\begin{entry}{能上能下}{10,3,10,3}{⾁、⼀、⾁、⼀}
  \begin{phonetics}{能上能下}{neng2shang4neng2xia4}
    \definition{s.}{pronto para aceitar qualquer trabalho, alto ou baixo}
  \end{phonetics}
\end{entry}

\begin{entry}{能干}{10,3}{⾁、⼲}
  \begin{phonetics}{能干}{neng2gan4}[][HSK 4]
    \definition{adj.}{apto; capaz; competente}
  \end{phonetics}
\end{entry}

\begin{entry}{能不能}{10,4,10}{⾁、⼀、⾁}
  \begin{phonetics}{能不能}{neng2 bu4 neng2}[][HSK 3]
    \definition{adv.}{pode ou não pode\dots?}
  \end{phonetics}
\end{entry}

\begin{entry}{能够}{10,11}{⾁、⼣}
  \begin{phonetics}{能够}{neng2 gou4}[][HSK 2]
    \definition{v.}{ser capaz de}
  \end{phonetics}
\end{entry}

\begin{entry}{能量}{10,12}{⾁、⾥}
  \begin{phonetics}{能量}{neng2liang4}[][HSK 5]
    \definition[种]{s.}{energia; quantidade de energia; Uma grandeza física que mede a capacidade da matéria de realizar trabalho | capacidade; competências; capacidade e papel que uma pessoa pode desempenhar}
  \end{phonetics}
\end{entry}

\begin{entry}{脂麻}{10,11}{⾁、⿇}
  \begin{phonetics}{脂麻}{zhi1ma5}
    \variantof{芝麻}
  \end{phonetics}
\end{entry}

\begin{entry}{脆}{10}{⾁}
  \begin{phonetics}{脆}{cui4}[][HSK 5]
    \definition{adj.}{frágil; quebradiço | crocante | clara; nítida; (voz)| puro}
  \end{phonetics}
\end{entry}

\begin{entry}{脏}{10}{⾁}
  \begin{phonetics}{脏}{zang1}[][HSK 2]
    \definition{adj.}{sujo | imundo}
  \end{phonetics}
  \begin{phonetics}{脏}{zang4}
    \definition{s.}{órgão (anatomia) | víscera}
  \end{phonetics}
\end{entry}

\begin{entry}{脏土}{10,3}{⾁、⼟}
  \begin{phonetics}{脏土}{zang1tu3}
    \definition{s.}{solo sujo | lama | lixo}
  \end{phonetics}
\end{entry}

\begin{entry}{脏字}{10,6}{⾁、⼦}
  \begin{phonetics}{脏字}{zang1zi4}
    \definition{s.}{obscenidade}
  \end{phonetics}
\end{entry}

\begin{entry}{脏病}{10,10}{⾁、⽧}
  \begin{phonetics}{脏病}{zang1bing4}
    \definition{s.}{doença venérea}
  \end{phonetics}
\end{entry}

\begin{entry}{脏脏}{10,10}{⾁、⾁}
  \begin{phonetics}{脏脏}{zang1zang1}
    \definition{adj.}{sujo}
  \end{phonetics}
\end{entry}

\begin{entry}{脏煤}{10,13}{⾁、⽕}
  \begin{phonetics}{脏煤}{zang1mei2}
    \definition{s.}{carvão sujo | sujeira (de uma mina de carvão)}
  \end{phonetics}
\end{entry}

\begin{entry}{脏器}{10,16}{⾁、⼝}
  \begin{phonetics}{脏器}{zang4qi4}
    \definition{s.}{órgãos internos}
  \end{phonetics}
\end{entry}

\begin{entry}{脏辫}{10,17}{⾁、⾟}
  \begin{phonetics}{脏辫}{zang1bian4}
    \definition{s.}{\emph{dreadlocks}}
  \end{phonetics}
\end{entry}

\begin{entry}{脑子}{10,3}{⾁、⼦}
  \begin{phonetics}{脑子}{nao3 zi5}[][HSK 5]
    \definition[个]{s.}{cérebro | mente; cabeça; cérebro; inteligência; poder mental; refere-se à capacidade de pensar, memorizar, raciocinar, etc.; inteligência}
  \end{phonetics}
\end{entry}

\begin{entry}{脑瓜}{10,5}{⾁、⽠}
  \begin{phonetics}{脑瓜}{nao3gua1}
    \definition{s.}{crânio | cérebro | cabeça | mente | mentalidade | ideia}
  \seealsoref{脑瓜子}{nao3gua1zi5}
  \end{phonetics}
\end{entry}

\begin{entry}{脑瓜子}{10,5,3}{⾁、⽠、⼦}
  \begin{phonetics}{脑瓜子}{nao3gua1zi5}
    \definition{s.}{crânio | cérebro | cabeça | mente | mentalidade | ideia}
  \seealsoref{脑瓜}{nao3gua1}
  \end{phonetics}
\end{entry}

\begin{entry}{脑袋}{10,11}{⾁、⾐}
  \begin{phonetics}{脑袋}{nao3dai5}[][HSK 4]
    \definition[颗,个]{s.}{cabeça; a parte mais alta do corpo humano ou a parte mais alta de um animal que contém órgãos como a boca, o nariz, os olhos etc. | mente; cérebro; capacidade de pensar, lembrar, etc.}
  \end{phonetics}
\end{entry}

\begin{entry}{脖子}{11,3}{⾁、⼦}
  \begin{phonetics}{脖子}{bo2zi5}
    \definition[个]{s.}{pescoço}
  \end{phonetics}
\end{entry}

\begin{entry}{脚}{11}{⾁}
  \begin{phonetics}{脚}{jiao3}[][HSK 2]
    \definition{clas.}{para chutes}
    \definition[双,只]{s.}{pé | base (de um objeto) | perna (de um animal ou objeto)}
  \end{phonetics}
  \begin{phonetics}{脚}{jue2}
    \variantof{角}
  \end{phonetics}
\end{entry}

\begin{entry}{脚步}{11,7}{⾁、⽌}
  \begin{phonetics}{脚步}{jiao3 bu4}[][HSK 5]
    \definition{s.}{pé; passo; pisada; refere-se ao movimento das pernas ao caminhar | ritmo; passo; distância entre os pés dianteiros e traseiros ao caminhar}
  \end{phonetics}
\end{entry}

\begin{entry}{脱}{11}{⾁}
  \begin{phonetics}{脱}{tuo1}[][HSK 4]
    \definition{conj.}{se; no caso;}
    \definition{v.}{(cabelo, pele) soltar-se; desprender-se; cair | retirar peça de roupa do corpo | sair de; escapar de | perder (palavras) | livrar-se de algo}
  \end{phonetics}
\end{entry}

\begin{entry}{脱毛}{11,4}{⾁、⽑}
  \begin{phonetics}{脱毛}{tuo1mao2}
    \definition{s.}{depilação}
    \definition{v.}{perder cabelo ou penas | depilar | fazer a barba}
  \end{phonetics}
\end{entry}

\begin{entry}{脱险}{11,9}{⾁、⾩}
  \begin{phonetics}{脱险}{tuo1xian3}
    \definition{v.}{sair do perigo}
  \end{phonetics}
\end{entry}

\begin{entry}{脸}{11}{⾁}
  \begin{phonetics}{脸}{lian3}[][HSK 2]
    \definition[张,个]{s.}{cara | rosto | face}
  \end{phonetics}
\end{entry}

\begin{entry}{脸色}{11,6}{⾁、⾊}
  \begin{phonetics}{脸色}{lian3 se4}[][HSK 5]
    \definition{s.}{aparência; tez; cor da pele | aparência; expressão facial | (indicando a condição física de alguém) aparência; cor}
  \end{phonetics}
\end{entry}

\begin{entry}{脸盆}{11,9}{⾁、⽫}
  \begin{phonetics}{脸盆}{lian3 pen2}[][HSK 5]
    \definition[个]{s.}{lavatório; bacia para lavar as mãos e o rosto}
  \end{phonetics}
\end{entry}

\begin{entry}{脾气}{12,4}{⾁、⽓}
  \begin{phonetics}{脾气}{pi2qi5}[][HSK 5]
    \definition[发,个]{s.}{temperamento; disposição; referindo-se ao caráter de uma pessoa | mau humor; temperamento irascível}
  \end{phonetics}
\end{entry}

\begin{entry}{腰}{13}{⾁}
  \begin{phonetics}{腰}{yao1}[][HSK 4]
    \definition*{s.}{sobrenome Yao}
    \definition[个]{s.}{cintura; região lombar | cós | bolso | parte do meio das coisas | lombo}
  \end{phonetics}
\end{entry}

\begin{entry}{腰包}{13,5}{⾁、⼓}
  \begin{phonetics}{腰包}{yao1bao1}
    \definition{s.}{pochete | bolso}
  \end{phonetics}
\end{entry}

\begin{entry}{腰椎}{13,12}{⾁、⽊}
  \begin{phonetics}{腰椎}{yao1zhui1}
    \definition{s.}{vértebra lombar (espinha dorsal inferior)}
  \end{phonetics}
\end{entry}

\begin{entry}{腿}{13}{⾁}
  \begin{phonetics}{腿}{tui3}[][HSK 2]
    \definition[条]{s.}{perna | osso do quadril}
  \end{phonetics}
\end{entry}

\begin{entry}{腿号}{13,5}{⾁、⼝}
  \begin{phonetics}{腿号}{tui3hao4}
    \definition{s.}{anilha numerada (por exemplo, usada para identificar pássaros)}
  \seealsoref{腿号箍}{tui3hao4gu1}
  \end{phonetics}
\end{entry}

\begin{entry}{腿号箍}{13,5,14}{⾁、⼝、⽵}
  \begin{phonetics}{腿号箍}{tui3hao4gu1}
    \definition{s.}{anilha numerada (por exemplo, usada para identificar pássaros)}
  \seealsoref{腿号}{tui3hao4}
  \end{phonetics}
\end{entry}

\begin{entry}{膜拜}{14,9}{⾁、⼿}
  \begin{phonetics}{膜拜}{mo2bai4}
    \definition{v.}{ajoelhar-se e se curvar com as mãos unidas no nível da testa | ter ou mostrar sentimentos fortes de respeito e admiração por um deus}
  \end{phonetics}
\end{entry}

\begin{entry}{膨胀}{16,8}{⾁、⾁}
  \begin{phonetics}{膨胀}{peng2zhang4}
    \definition{v.}{expandir | inflar | inchar}
  \end{phonetics}
\end{entry}

%%%%% EOF %%%%%

