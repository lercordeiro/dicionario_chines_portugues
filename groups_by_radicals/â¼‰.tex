%%%
%%% Radical "⼉"
%%%

\section*{Radical 10: ``⼉''}\addcontentsline{toc}{section}{Radical 10: ⼉}

\begin{Entry}{儿}{2}{⼉}[Kangxi 10]
  \begin{Phonetics}{儿}{er2}
    \definition{adj.}{macho}
    \definition{s.}{criança | jovem; juventude | filho}
    \definition{suf.}{adicionado a substantivos para expressar pequenez  | adicionado a verbos, adjetivos e classificadores para formar substantivos | adicionado a substantivos para formar substantivos com significados diferentes | sufixos de alguns verbos | anexado após adjetivos duplicados}
  \end{Phonetics}
  \begin{Phonetics}{儿}{r5}
    \definition{suf.}{sufixo diminutivo não silábico | final retroflexo, pronunciado como ``r'' | adicionado a substantivos para expressar pequenez  | adicionado a verbos, adjetivos e classificadores para formar substantivos | adicionado a substantivos para formar substantivos com significados diferentes | sufixos de alguns verbos | anexado após adjetivos duplicados}
  \end{Phonetics}
\end{Entry}

\begin{Entry}{儿女}{2,3}{⼉、⼥}
  \begin{Phonetics}{儿女}{er2 nv3}[][HSK 5]
    \definition{s.}{crianças; filhos e filhas | homem e mulher jovens (apaixonados)}
  \end{Phonetics}
\end{Entry}

\begin{Entry}{儿子}{2,3}{⼉、⼦}
  \begin{Phonetics}{儿子}{er2zi5}[][HSK 1]
    \definition[个]{s.}{filho}
  \seealsoref{女儿}{nv3'er2}
  \end{Phonetics}
\end{Entry}

\begin{Entry}{儿科}{2,9}{⼉、⽲}
  \begin{Phonetics}{儿科}{er2 ke1}[][HSK 6]
    \definition{s.}{(departamento de) pediatria | pediatria; o ramo da medicina que trata do desenvolvimento, cuidado e doença das crianças}
  \end{Phonetics}
\end{Entry}

\begin{Entry}{儿童}{2,12}{⼉、⽴}
  \begin{Phonetics}{儿童}{er2tong2}[][HSK 4]
    \definition[个,群]{s.}{criança; menor de idade (mais jovem do que 少年)}
  \seealsoref{少年}{shao4 nian2}
  \end{Phonetics}
\end{Entry}

\begin{Entry}{儿媳}{2,13}{⼉、⼥}
  \begin{Phonetics}{儿媳}{er2xi2}
    \definition{s.}{esposa do filho}
  \end{Phonetics}
\end{Entry}

\begin{Entry}{允}{4}{⼉}
  \begin{Phonetics}{允}{yun3}
    \definition*{s.}{Sobrenome Yun}
    \definition{adj.}{justo; imparcial}
    \definition{v.}{permitir; deixar; consentir}
  \end{Phonetics}
\end{Entry}

\begin{Entry}{允许}{4,6}{⼉、⾔}
  \begin{Phonetics}{允许}{yun3xu3}[][HSK 6]
    \definition{s.}{permitido; permissão}
    \definition{v.}{permitir; deixar; concordar com alguém para fazer algo}
  \end{Phonetics}
\end{Entry}

\begin{Entry}{元}{4}{⼉}
  \begin{Phonetics}{元}{yuan2}[][HSK 1]
    \definition*{s.}{Dinastia Yuan (1271-1368) | Sobrenome Yuan}
    \definition{adj.}{primeiro; principal; primário | chefe; diretor; líder | básico; fundamental; principal | unidade; componente; formando um todo}
    \definition{clas.}{yuan, a unidade monetária da China}
    \definition{s.}{moeda com valor e peso fixos | origem; elemento}
  \end{Phonetics}
\end{Entry}

\begin{Entry}{元气}{4,4}{⼉、⽓}
  \begin{Phonetics}{元气}{yuan2qi4}
    \definition{s.}{força | vigor | vitalidade | energial vital}
  \end{Phonetics}
\end{Entry}

\begin{Entry}{元旦}{4,5}{⼉、⽇}
  \begin{Phonetics}{元旦}{yuan2dan4}[][HSK 5]
    \definition*{s.}{Dia de Ano Novo (1 de janeiro)}
  \end{Phonetics}
\end{Entry}

\begin{Entry}{元来}{4,7}{⼉、⽊}
  \begin{Phonetics}{元来}{yuan2lai2}
    \variantof{原来}
  \end{Phonetics}
\end{Entry}

\begin{Entry}{元夜}{4,8}{⼉、⼣}
  \begin{Phonetics}{元夜}{yuan2ye4}
    \definition*{s.}{Festival das Lanternas}
  \seealsoref{元宵}{yuan2xiao1}
  \seealsoref{元宵节}{yuan2xiao1jie2}
  \end{Phonetics}
\end{Entry}

\begin{Entry}{元宵}{4,10}{⼉、⼧}
  \begin{Phonetics}{元宵}{yuan2xiao1}
    \definition*{s.}{Festival das Lanternas}
  \seealsoref{元宵节}{yuan2xiao1jie2}
  \seealsoref{元夜}{yuan2ye4}
  \end{Phonetics}
\end{Entry}

\begin{Entry}{元宵节}{4,10,5}{⼉、⼧、⾋}
  \begin{Phonetics}{元宵节}{yuan2xiao1jie2}
    \definition*{s.}{Festival das Lanternas (15º~dia do primeiro mês lunar)}
  \seealsoref{元宵}{yuan2xiao1}
  \seealsoref{元夜}{yuan2ye4}
  \end{Phonetics}
\end{Entry}

\begin{Entry}{元素}{4,10}{⼉、⽷}
  \begin{Phonetics}{元素}{yuan2su4}[][HSK 6]
    \definition{s.}{elemento; fator essencial | elemento químico | Geometria: as partes que compõem uma figura, como os lados e ângulos de um triângulo | Algebra: os elementos algébricos incluem números e símbolos}
  \end{Phonetics}
\end{Entry}

\begin{Entry}{兄}{5}{⼉}
  \begin{Phonetics}{兄}{xiong1}
    \definition{s.}{irmão mais velho | parente mais velho do sexo masculino da mesma geração | uma forma cortês de tratamento entre amigos homens; um título respeitoso para amigos homens}
  \end{Phonetics}
\end{Entry}

\begin{Entry}{兄弟}{5,7}{⼉、⼸}
  \begin{Phonetics}{兄弟}{xiong1di4}[][HSK 4]
    \definition{adj.}{fraternal}
    \definition{pron.}{eu, me (termo de uso humilde por homens em discurso público)}
    \definition[个,位]{s.}{irmãos; irmão}
  \end{Phonetics}
\end{Entry}

\begin{Entry}{充}{6}{⼉}
  \begin{Phonetics}{充}{chong1}[][HSK 7-9]
    \definition*{s.}{Sobrenome Chong}
    \definition{adj.}{suficiente; completo; amplo; cheio}
    \definition{v.}{encher; carregar; atulhar | servir como; agir como | fingir ser; posar como; passar algo como}
  \end{Phonetics}
\end{Entry}

\begin{Entry}{充分}{6,4}{⼉、⼑}
  \begin{Phonetics}{充分}{chong1fen4}[][HSK 4]
    \definition{adj.}{cheio; amplo; abundante; suficiente; adequado}
    \definition{adv.}{totalmente; até o fim}
  \end{Phonetics}
\end{Entry}

\begin{Entry}{充电}{6,5}{⼉、⽥}
  \begin{Phonetics}{充电}{chong1 dian4}[][HSK 4]
    \definition{v.}{carregar (uma bateria); conectar uma fonte de alimentação CC aos terminais da bateria para recarregar a bateria | relaxar; passar o tempo livre; ``recarregar as baterias''; estudar para adquirir mais conhecimento; reabastecer (ou ampliar) o conhecimento; metaforicamente falando, para reabastecer a força física e a energia por meio do descanso e da recreação; também metaforicamente falando, para reabastecer novos conhecimentos e desenvolver novas habilidades por meio do reaprendizado}
  \end{Phonetics}
\end{Entry}

\begin{Entry}{充电器}{6,5,16}{⼉、⽥、⼝}
  \begin{Phonetics}{充电器}{chong1dian4qi4}[][HSK 4]
    \definition[个,台]{s.}{carregador de bateria; dispositivo para alimentar uma bateria com energia, forçando uma corrente através dela}
  \end{Phonetics}
\end{Entry}

\begin{Entry}{充当}{6,6}{⼉、⼹}
  \begin{Phonetics}{充当}{chong1dang1}[][HSK 7-9]
    \definition{v.}{agir como; servir como; desempenhar o papel de; assumir o comando de}
  \end{Phonetics}
\end{Entry}

\begin{Entry}{充沛}{6,7}{⼉、⽔}
  \begin{Phonetics}{充沛}{chong1pei4}[][HSK 7-9]
    \definition{adj.}{abundante; cheio de}
  \end{Phonetics}
\end{Entry}

\begin{Entry}{充足}{6,7}{⼉、⾜}
  \begin{Phonetics}{充足}{chong1zu2}[][HSK 5]
    \definition{adj.}{bastante; adequado; suficiente; mais do que suficiente para atender às necessidades (usado principalmente para coisas mais específicas)}
  \end{Phonetics}
\end{Entry}

\begin{Entry}{充实}{6,8}{⼉、⼧}
  \begin{Phonetics}{充实}{chong1shi2}[][HSK 7-9]
    \definition{adj.}{rico; cheio; substancial; gratificante}
    \definition{v.}{enriquecer; aumentar; substanciar (um argumento)}
  \end{Phonetics}
\end{Entry}

\begin{Entry}{充满}{6,13}{⼉、⽔}
  \begin{Phonetics}{充满}{chong1man3}[][HSK 3]
    \definition{v.}{preencher; encher; cobrir completamente| estar cheio de; estar repleto de; estar transbordando de; estar impregnado de}
  \end{Phonetics}
\end{Entry}

\begin{Entry}{兆}{6}{⼉}
  \begin{Phonetics}{兆}{zhao4}
    \definition{num.}{trilhão}
  \end{Phonetics}
\end{Entry}

\begin{Entry}{先}{6}{⼉}
  \begin{Phonetics}{先}{xian1}[][HSK 1]
    \definition*{s.}{Sobrenome Xian}
    \definition{adv.}{primeiro; antes; mais cedo; com antecedência | no momento; por enquanto; em um curto espaço de tempo; temporariamente}
    \definition{s.}{início; começo; em ordem cronológica ou de precedência | ancestral; geração mais velha; antepassado | tardio; falecido; morto (honrar os mortos)}
  \end{Phonetics}
\end{Entry}

\begin{Entry}{先不先}{6,4,6}{⼉、⼀、⼉}
  \begin{Phonetics}{先不先}{xian1bu4xian1}
    \definition{adv.}{(dialeto) antes de tudo | em primeiro lugar}
  \end{Phonetics}
\end{Entry}

\begin{Entry}{先天}{6,4}{⼉、⼤}
  \begin{Phonetics}{先天}{xian1tian1}
    \definition{adj.}{congênito | inato | natural}
    \definition{s.}{período embrionário}
  \end{Phonetics}
\end{Entry}

\begin{Entry}{先生}{6,5}{⼉、⽣}
  \begin{Phonetics}{先生}{xian1sheng5}[][HSK 1]
    \definition[个,位]{s.}{professor; títulos honoríficos para professores, médicos, etc. | marido; antigamente, referia-se ao marido de outra pessoa ou ao próprio marido (ambos com pronomes pessoais como determinantes) | médico; títulos usados para se referir aos médicos no passado | refere-se a pessoas cuja profissão envolve contar histórias, adivinhação, etc.; antigamente, era chamado de contador | senhor; \emph{sir}; título dado aos intelectuais}
  \end{Phonetics}
\end{Entry}

\begin{Entry}{先后}{6,6}{⼉、⼝}
  \begin{Phonetics}{先后}{xian1 hou4}[][HSK 5]
    \definition{adv.}{sucessivamente; um após o outro}
    \definition{s.}{prioridade; ordem; cedo ou tarde; primeiro e último}
  \end{Phonetics}
\end{Entry}

\begin{Entry}{先有}{6,6}{⼉、⽉}
  \begin{Phonetics}{先有}{xian1you3}
    \definition{adj.}{preexistente | anterior}
  \end{Phonetics}
\end{Entry}

\begin{Entry}{先进}{6,7}{⼉、⾡}
  \begin{Phonetics}{先进}{xian1jin4}[][HSK 3]
    \definition{adj.}{avançado; progressos rápidos e nível elevado, podendo servir de exemplo a seguir}
    \definition{s.}{indivíduos ou grupos avançados}
  \end{Phonetics}
\end{Entry}

\begin{Entry}{先到先得}{6,8,6,11}{⼉、⼑、⼉、⼻}
  \begin{Phonetics}{先到先得}{xian1dao4xian1de2}
    \definition{expr.}{primeiro a chegar | primeiro a ser servido}
  \end{Phonetics}
\end{Entry}

\begin{Entry}{先前}{6,9}{⼉、⼑}
  \begin{Phonetics}{先前}{xian1qian2}[][HSK 5]
    \definition[出]{s.}{antes; anteriormente; refere-se ao passado ou a um certo tempo anterior}
  \end{Phonetics}
\end{Entry}

\begin{Entry}{先烈}{6,10}{⼉、⽕}
  \begin{Phonetics}{先烈}{xian1lie4}
    \definition{s.}{mártir}
  \end{Phonetics}
\end{Entry}

\begin{Entry}{先验}{6,10}{⼉、⾺}
  \begin{Phonetics}{先验}{xian1yan4}
    \definition{adj.}{(filosofia) a priori}
  \end{Phonetics}
\end{Entry}

\begin{Entry}{先期}{6,12}{⼉、⽉}
  \begin{Phonetics}{先期}{xian1qi1}
    \definition{adv.}{antecipadamente}
    \definition{s.}{prematuro | \emph{front-end}}
  \end{Phonetics}
\end{Entry}

\begin{Entry}{先锋}{6,12}{⼉、⾦}
  \begin{Phonetics}{先锋}{xian1 feng1}[][HSK 6]
    \definition{s.}{pioneiro; vanguarda; a vanguarda de uma batalha ou marcha; geralmente se refere a uma pessoa ou grupo que desempenha um papel de vanguarda}
  \end{Phonetics}
\end{Entry}

\begin{Entry}{光}{6}{⼉}
  \begin{Phonetics}{光}{guang1}[][HSK 3]
    \definition*{s.}{Sobrenome Guang}
    \definition{adj.}{suave; liso; brilhante | esgotado; sem nada sobrando | brilhante}
    \definition{adv.}{somente; sozinho; meramente}
    \definition{s.}{luz; raio | cenário; paisagem | honra; glória; brilho | claridade | favor; graça | momento | corpo celeste; referindo-se especificamente a corpos celestes, como o sol, a lua e as estrelas}
    \definition{v.}{glorificar; recuperar; reconquistar | estar nu; expor}
  \end{Phonetics}
\end{Entry}

\begin{Entry}{光污染}{6,6,9}{⼉、⽔、⽊}
  \begin{Phonetics}{光污染}{guang1 wu1ran3}
    \definition{s.}{poluição luminosa}
  \end{Phonetics}
\end{Entry}

\begin{Entry}{光芒}{6,6}{⼉、⾋}
  \begin{Phonetics}{光芒}{guang1mang2}[][HSK 7-9]
    \definition[道]{s.}{brilho; radiância; raios brilhantes; raios de luz; luz forte irradiando em todas as direções}
  \end{Phonetics}
\end{Entry}

\begin{Entry}{光明}{6,8}{⼉、⽇}
  \begin{Phonetics}{光明}{guang1ming2}[][HSK 3]
    \definition{adj.}{brilhante; luminoso | sincero; ingênuo; metáfora da justiça e da esperança | justo; honesto; franco}
    \definition{s.}{luz}
  \end{Phonetics}
\end{Entry}

\begin{Entry}{光明磊落}{6,8,15,12}{⼉、⽇、⽯、⾋}
  \begin{Phonetics}{光明磊落}{guang1ming2-lei3luo4}[][HSK 7-9]
    \definition{expr.}{aberto e sincero; direto e honesto; descreve ser altruísta e de mente aberta; aberto e transparente}
  \end{Phonetics}
\end{Entry}

\begin{Entry}{光泽}{6,8}{⼉、⽔}
  \begin{Phonetics}{光泽}{guang1ze2}[][HSK 7-9]
    \definition{s.}{brilho; lustro; fulgor; luz brilhante refletida de uma superfície; cor e brilho}
  \end{Phonetics}
\end{Entry}

\begin{Entry}{光环}{6,8}{⼉、⽟}
  \begin{Phonetics}{光环}{guang1huan2}[][HSK 7-9]
    \definition[道]{s.}{um anel de luz; matéria brilhante ao redor de alguns planetas | halo; auréola; o halo anular na cabeça de uma divindade | um halo colorido que às vezes aparece ao redor do sol ou da lua | glória; distinção; esplendor; metáfora para fama e honra}
  \end{Phonetics}
\end{Entry}

\begin{Entry}{光线}{6,8}{⼉、⽷}
  \begin{Phonetics}{光线}{guang1 xian4}[][HSK 5]
    \definition[条,道]{s.}{luz; feixe luminoso; raio de luz}
  \end{Phonetics}
\end{Entry}

\begin{Entry}{光临}{6,9}{⼉、⼁}
  \begin{Phonetics}{光临}{guang1lin2}[][HSK 4]
    \definition{v.}{honrar com sua presença, uma palavra de honra, usada para dizer que um convidado chegou}
  \end{Phonetics}
\end{Entry}

\begin{Entry}{光荣}{6,9}{⼉、⾋}
  \begin{Phonetics}{光荣}{guang1rong2}[][HSK 5]
    \definition{adj.}{honroso; honrado; glorioso; por fazer algo que é benéfico para o país ou para a coletividade e que é considerado por todos como digno de respeito ou elogio}
    \definition{s.}{honra; glória; crédito; sentimento de honra decorrente do fato de ser respeitado ou elogiado por fazer algo importante ou grandioso}
  \end{Phonetics}
\end{Entry}

\begin{Entry}{光顾}{6,10}{⼉、⾴}
  \begin{Phonetics}{光顾}{guang1gu4}[][HSK 7-9]
    \definition{v.}{patrocinar; honrar com; uma palavra que demonstra respeito a alguém, referindo-se à chegada de um convidado; restaurantes e lojas costumam usá-la para dar as boas-vindas aos clientes; também é usada de forma metafórica e irônica}
  \end{Phonetics}
\end{Entry}

\begin{Entry}{光彩}{6,11}{⼉、⼺}
  \begin{Phonetics}{光彩}{guang1cai3}[][HSK 7-9]
    \definition{adj.}{glorioso; honroso; decente}
    \definition{s.}{brilho; esplendor; radiância}
  \end{Phonetics}
\end{Entry}

\begin{Entry}{光盘}{6,11}{⼉、⽫}
  \begin{Phonetics}{光盘}{guang1pan2}[][HSK 4]
    \definition[张,套,片]{s.}{CD; disco compacto; um disco circular feito de plástico rígido composto que usa um laser para registrar e ler informações}
  \end{Phonetics}
\end{Entry}

\begin{Entry}{光滑}{6,12}{⼉、⽔}
  \begin{Phonetics}{光滑}{guang1hua2}[][HSK 7-9]
    \definition{adj.}{liso; suave; brilhante}
  \end{Phonetics}
\end{Entry}

\begin{Entry}{光缆}{6,12}{⼉、⽷}
  \begin{Phonetics}{光缆}{guang1lan3}[][HSK 7-9]
    \definition[根,条]{s.}{cabo óptico; cabo de fibra óptica}
  \end{Phonetics}
\end{Entry}

\begin{Entry}{光辉}{6,12}{⼉、⾞}
  \begin{Phonetics}{光辉}{guang1hui1}[][HSK 6]
    \definition{adj.}{brilhante; magnífico; glorioso}
    \definition{s.}{esplendor; brilho; glória | chama; brilho; halo; labareda; fulguração; lustre}
  \end{Phonetics}
\end{Entry}

\begin{Entry}{光槃}{6,14}{⼉、⽊}
  \begin{Phonetics}{光槃}{guang1pan2}
    \variantof{光盘}
  \end{Phonetics}
\end{Entry}

\begin{Entry}{光碟}{6,14}{⼉、⽯}
  \begin{Phonetics}{光碟}{guang1die2}[][HSK 7-9]
    \definition[个,片,张]{s.}{disco compacto (CD); videodisco; CD; CD-ROM; disco ótico}
  \end{Phonetics}
\end{Entry}

\begin{Entry}{免}{7}{⼉}
  \begin{Phonetics}{免}{mian3}
    \definition*{s.}{Sobrenome Mian}
    \definition{v.}{desculpar alguém de algo; isentar; dispensar; renunciar | remover do cargo; demitir | evitar; desviar; escapar | não deveria ser permitido; não precisar fazer algo | remover; livrar-se de | isentar; dispensar | não permitir}
  \end{Phonetics}
\end{Entry}

\begin{Entry}{免费}{7,9}{⼉、⾙}
  \begin{Phonetics}{免费}{mian3/fei4}[][HSK 4]
    \definition{v.+compl.}{isentar de taxas; tonar grátis}
  \end{Phonetics}
\end{Entry}

\begin{Entry}{免得}{7,11}{⼉、⼻}
  \begin{Phonetics}{免得}{mian3de5}[][HSK 6]
    \definition{conj.}{de modo a não; para evitar; para que não; indica evitar uma situação que não é desejável e é frequentemente usado no início da oração seguinte}
  \end{Phonetics}
\end{Entry}

\begin{Entry}{免税}{7,12}{⼉、⽲}
  \begin{Phonetics}{免税}{mian3/shui4}
    \definition{adj.}{isento de impostos (tributação)}
    \definition{s.}{livre de impostos | isenção de impostos}
    \definition{v.+compl.}{isentar impostos}
  \end{Phonetics}
\end{Entry}

\begin{Entry}{兑}{7}{⼉}
  \begin{Phonetics}{兑}{dui4}
    \definition*{s.}{Sobrenome Dui}
    \definition{s.}{dui, um dos Oito Trigramas que representa pântano}
    \definition{v.}{trocar; converter | adicionar (água, etc.) | sacar; pagar ou receber dinheiro por fatura}
  \end{Phonetics}
\end{Entry}

\begin{Entry}{兑现}{7,8}{⼉、⾒}
  \begin{Phonetics}{兑现}{dui4xian4}[][HSK 7-9]
    \definition{v.}{sacar (dinheiro, cheques, etc.) | cumprir; fazer o bem; honrar (um compromisso, etc.); metáfora para cumprir uma promessa}
  \end{Phonetics}
\end{Entry}

\begin{Entry}{兑换}{7,10}{⼉、⼿}
  \begin{Phonetics}{兑换}{dui4huan4}[][HSK 7-9]
    \definition{v.}{converter; trocar; trocar uma moeda por outra; trocar um cheque, etc., por dinheiro}
  \end{Phonetics}
\end{Entry}

\begin{Entry}{兔}{8}{⼉}
  \begin{Phonetics}{兔}{tu4}[][HSK 5]
    \definition[只]{s.}{lebre; coelho}
  \end{Phonetics}
\end{Entry}

\begin{Entry}{兔子}{8,3}{⼉、⼦}
  \begin{Phonetics}{兔子}{tu4zi5}
    \definition[只]{s.}{coelho | lebre}
  \end{Phonetics}
\end{Entry}

\begin{Entry}{党}{10}{⼉}
  \begin{Phonetics}{党}{dang3}[][HSK 6]
    \definition*{s.}{O Partido (Partido Comunista da China) | Sobrenome Dang}
    \definition{s.}{partido político; partido | camarilha; facção; gangue | Datado: parentes}
    \definition{v.}{ser parcial; tomar partido de}
  \end{Phonetics}
\end{Entry}

\begin{Entry}{兜}{11}{⼉}
  \begin{Phonetics}{兜}{dou1}[][HSK 7-9]
    \definition{s.}{bolsa; bolso; coisas tipo bolso}
    \definition{v.}{embrulhar em um pedaço de pano, etc.; fazer um formato de bolso para guardar coisas | mover-se; dar uma volta; fazer um desvio | solicitar; sondar; recrutamentar | assumir a responsabilidade por algo; assumir o controle de}
  \end{Phonetics}
\end{Entry}

\begin{Entry}{兜儿}{11,2}{⼉、⼉}
  \begin{Phonetics}{兜儿}{dou1r5}[][HSK 7-9]
    \definition{s.}{bolso}
  \end{Phonetics}
\end{Entry}

\begin{Entry}{兜售}{11,11}{⼉、⼝}
  \begin{Phonetics}{兜售}{dou1shou4}[][HSK 7-9]
    \definition{v.}{vender; apregoar | vender; fazer uma venda de}
  \end{Phonetics}
\end{Entry}

%%%%% EOF %%%%%

