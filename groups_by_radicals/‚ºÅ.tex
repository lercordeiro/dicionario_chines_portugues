%%%
%%% Radical "⼁"
%%%

\section*{Radical 2: ``⼁''}\addcontentsline{toc}{section}{Radical 2: ⼁}

\begin{entry}{中}{4}{⼁}
  \begin{phonetics}{中}{zhong1}[][HSK 1]
    \definition*{s.}{China; referindo-se à China | Sobrenome Zhong}
    \definition{adj.}{\emph{O.K.}; tudo bem, ótimo, adequado}
    \definition{s.}{centro; meio; a parte que está à mesma distância de todos os lados, acima e abaixo ou nas duas extremidades | em; entre (usado para indicar coisas dentro de um determinado intervalo) | meio; centro; localizado entre os dois extremos | médio; intermediário; classificação entre os dois extremos | médio; a meio caminho entre dois extremos; imparcial | intermediário | enquanto; durante (use após um verbo para mostrar que a ação está em andamento)}
    \definition{v.}{ser adequado para; ser compatível com}
  \seealsoref{中国}{zhong1guo2}
  \end{phonetics}
  \begin{phonetics}{中}{zhong4}
    \definition{v.}{acertar | encaixar exatamente |ser atingido por | cair em | ser afetado por | sofrer | sustentar}
  \end{phonetics}
\end{entry}

\begin{entry}{中土}{4,3}{⼁、⼟}
  \begin{phonetics}{中土}{zhong1 tu3}
    \definition*{s.}{Datado: China (Terra Média) | Sino-turco}
  \end{phonetics}
\end{entry}

\begin{entry}{中小学}{4,3,8}{⼁、⼩、⼦}
  \begin{phonetics}{中小学}{zhong1 xiao3 xue2}[][HSK 2]
    \definition{s.}{escolas primárias e secundárias}
  \end{phonetics}
\end{entry}

\begin{entry}{中介}{4,4}{⼁、⼈}
  \begin{phonetics}{中介}{zhong1jie4}[][HSK 4]
    \definition[个]{s.}{agente; intermediário}
  \end{phonetics}
\end{entry}

\begin{entry}{中午}{4,4}{⼁、⼗}
  \begin{phonetics}{中午}{zhong1wu3}[][HSK 1]
    \definition[个]{s.}{meio-dia; refere-se ao período entre 12 e 13 horas}
  \end{phonetics}
\end{entry}

\begin{entry}{中心}{4,4}{⼁、⼼}
  \begin{phonetics}{中心}{zhong1xin1}[][HSK 2]
    \definition[个]{s.}{núcleo; coração; meio; centro; posições com distâncias iguais de todas as áreas circundantes | chave; coração; a parte principal de algo; a pessoa ou coisa que desempenha um papel importante | centro; uma cidade ou lugar que tem um impacto significativo ou desempenha um papel importante em uma determinada área | centro; uma instituição com equipamentos e tecnologia relativamente completos e avançados em uma determinada área}
  \end{phonetics}
\end{entry}

\begin{entry}{中文}{4,4}{⼁、⽂}
  \begin{phonetics}{中文}{zhong1wen2}[][HSK 1]
    \definition{s.}{a língua chinesa; chinês; a língua e a escrita da China; especificamente, o chinês e os caracteres chineses}
  \end{phonetics}
\end{entry}

\begin{entry}{中东}{4,5}{⼁、⼀}
  \begin{phonetics}{中东}{zhong1dong1}
    \definition*{s.}{Oriente Médio}
  \end{phonetics}
\end{entry}

\begin{entry}{中央}{4,5}{⼁、⼤}
  \begin{phonetics}{中央}{zhong1yang1}[][HSK 5]
    \definition{s.}{centro; meio; localização central | autoridades centrais; refere-se especificamente ao órgão máximo de liderança de um país ou partido político}
  \end{phonetics}
\end{entry}

\begin{entry}{中央情报局}{4,5,11,7,7}{⼁、⼤、⼼、⼿、⼫}
  \begin{phonetics}{中央情报局}{zhong1yang1 qing2bao4ju2}
    \definition*{s.}{Agência Central de Inteligência dos EUA, CIA}
  \end{phonetics}
\end{entry}

\begin{entry}{中华}{4,6}{⼁、⼗}
  \begin{phonetics}{中华}{zhong1hua2}
    \definition{s.}{China; na antiguidade, a região do rio Amarelo era chamada de Zhonghua, sendo o local onde a etnia Han surgiu inicialmente. Posteriormente, passou a designar a China}
  \seealsoref{中国}{zhong1guo2}
  \end{phonetics}
\end{entry}

\begin{entry}{中华民族}{4,6,5,11}{⼁、⼗、⽒、⽅}
  \begin{phonetics}{中华民族}{zhong1 hua2 min2 zu2}[][HSK 3]
    \definition*{s.}{O Povo Chinês; o nome genérico para todas as etnias da China, incluindo 56 etnias, com uma longa história, um rico patrimônio cultural e uma gloriosa tradição revolucionária | A Nação Chinesa}
  \end{phonetics}
\end{entry}

\begin{entry}{中州}{4,6}{⼁、⼮}
  \begin{phonetics}{中州}{zhong1zhou1}
    \definition*{s.}{Região Central, ou seja, província de Henan, devido à sua localização central no país}
  \end{phonetics}
\end{entry}

\begin{entry}{中年}{4,6}{⼁、⼲}
  \begin{phonetics}{中年}{zhong1 nian2}[][HSK 2]
    \definition{s.}{meia-idade; na casa dos quarenta ou cinquenta anos}
  \end{phonetics}
\end{entry}

\begin{entry}{中级}{4,6}{⼁、⽷}
  \begin{phonetics}{中级}{zhong1 ji2}[][HSK 2]
    \definition{adj.}{nível médio; nível intermediário; entre avançado e iniciante}
  \end{phonetics}
\end{entry}

\begin{entry}{中医}{4,7}{⼁、⼖}
  \begin{phonetics}{中医}{zhong1 yi1}[][HSK 2]
    \definition[位,个,名,些,群]{s.}{ciência médica tradicional chinesa; medicina chinesa | médico de medicina tradicional chinesa; praticante de medicina chinesa}
  \end{phonetics}
\end{entry}

\begin{entry}{中间}{4,7}{⼁、⾨}
  \begin{phonetics}{中间}{zhong1jian1}[][HSK 1]
    \definition[本]{s.}{em meio a; entre; dentro de um determinado intervalo | meio; centro; posição entre os dois extremos de uma coisa ou entre duas coisas | posição intermediária; espaço entre duas extremidades; posição entre os dois extremos de uma coisa, dois momentos, duas coisas, etc.}
  \end{phonetics}
\end{entry}

\begin{entry}{中国}{4,8}{⼁、⼞}
  \begin{phonetics}{中国}{zhong1guo2}[][HSK 1]
    \definition*{s.}{China; os povos Huaxia e Han estabeleceram suas capitais principalmente ao sul e ao norte do rio Huang He, e por isso chamaram essa região de 中国, com o mesmo significado de 中土, 中原l, 中州 e 中华}
  \seealsoref{中土}{zhong1 tu3}
  \seealsoref{中华}{zhong1hua2}
  \seealsoref{中原}{zhong1yuan2}
  \seealsoref{中州}{zhong1zhou1}
  \end{phonetics}
\end{entry}

\begin{entry}{中国人}{4,8,2}{⼁、⼞、⼈}
  \begin{phonetics}{中国人}{zhong1guo2ren2}
    \definition{s.}{chinês | pessoa ou povo da China}
  \end{phonetics}
\end{entry}

\begin{entry}{中国城}{4,8,9}{⼁、⼞、⼟}
  \begin{phonetics}{中国城}{zhong1guo2cheng2}
    \definition*[座]{s.}{Bairro Chinês, Chinatown}
  \seealsoref{唐人街}{tang2ren2 jie1}
  \end{phonetics}
\end{entry}

\begin{entry}{中国科学院}{4,8,9,8,9}{⼁、⼞、⽲、⼦、⾩}
  \begin{phonetics}{中国科学院}{zhong1guo2 ke1xue2yuan4}
    \definition*{s.}{Academia Chinesa de Ciências}
  \end{phonetics}
\end{entry}

\begin{entry}{中国通}{4,8,10}{⼁、⼞、⾡}
  \begin{phonetics}{中国通}{zhong1guo2tong1}
    \definition*{s.}{Sinólogo | Conhecedor da China, especialista em tudo sobre a China; refere-se a estrangeiros que estão familiarizados com a China}
  \end{phonetics}
\end{entry}

\begin{entry}{中学}{4,8}{⼁、⼦}
  \begin{phonetics}{中学}{zhong1 xue2}[][HSK 1]
    \definition[个]{s.}{escola ensino médio; escolas onde os jovens recebem educação secundária geralmente incluem o ensino fundamental II e o ensino médio | aprendizagem chinesa (um termo da dinastia Qing tardia para a aprendizagem tradicional chinesa); antigamente, referia-se à academia tradicional da China, como filosofia, linguística, medicina tradicional chinesa, etc.}
  \end{phonetics}
\end{entry}

\begin{entry}{中学生}{4,8,5}{⼁、⼦、⽣}
  \begin{phonetics}{中学生}{zhong1 xue2 sheng1}[][HSK 1]
    \definition{s.}{aluno, estudante do ensino médio; alunos matriculados no ensino médio. Inclui alunos do ensino fundamental II e do ensino médio}
  \end{phonetics}
\end{entry}

\begin{entry}{中性}{4,8}{⼁、⼼}
  \begin{phonetics}{中性}{zhong1xing4}
    \definition{adj.}{neutro}
  \end{phonetics}
\end{entry}

\begin{entry}{中询}{4,8}{⼁、⾔}
  \begin{phonetics}{中询}{zhong1 xun2}
    \definition{adv.}{segunda dezena do mês | meio do mês | em meados do mês}
  \end{phonetics}
\end{entry}

\begin{entry}{中奖}{4,9}{⼁、⼤}
  \begin{phonetics}{中奖}{zhong4 jiang3}[][HSK 4]
    \definition{v.}{ganhar um prêmio (em uma loteria, etc.)}
  \end{phonetics}
\end{entry}

\begin{entry}{中毒}{4,9}{⼁、⽏}
  \begin{phonetics}{中毒}{zhong4 du2}[][HSK 5]
    \definition{v.}{envenenar; intoxicar | ser envenenado; ideia de que o pensamento foi contaminado}
  \end{phonetics}
\end{entry}

\begin{entry}{中秋节}{4,9,5}{⼁、⽲、⾋}
  \begin{phonetics}{中秋节}{zhong1 qiu1 jie2}[][HSK 5]
    \definition*{s.}{Festival do Meio-Outono | Festival do Bolo Lunar (15º dia do oitavo mês lunar)}
  \end{phonetics}
\end{entry}

\begin{entry}{中药}{4,9}{⼁、⾋}
  \begin{phonetics}{中药}{zhong1 yao4}[][HSK 5]
    \definition[服,种]{s.}{medicina herbal; medicina tradicional chinesa; fitoterapia; medicamentos utilizados na medicina tradicional chinesa; incluem medicamentos naturais e seus produtos processados (em contraste com 西药)}
  \seealsoref{西药}{xi1 yao4}
  \end{phonetics}
\end{entry}

\begin{entry}{中原}{4,10}{⼁、⼚}
  \begin{phonetics}{中原}{zhong1yuan2}
    \definition*{s.}{Planícies Centrais (compreendendo os trechos médio e inferior do rio Huang He) | Planície Central, regiões média e baixa do rio Amarelo, incluindo Henan, oeste de Shandong, sul de Shanxi e Hebei}
  \end{phonetics}
\end{entry}

\begin{entry}{中部}{4,10}{⼁、⾢}
  \begin{phonetics}{中部}{zhong1 bu4}[][HSK 3]
    \definition{s.}{parte do meio; região central; seção central; região ou parte intermediária | parte do meio; parte central; refere-se à parte intermediária de uma série de três partes, como romances, obras cinematográficas e televisivas}
  \end{phonetics}
\end{entry}

\begin{entry}{中情局}{4,11,7}{⼁、⼼、⼫}
  \begin{phonetics}{中情局}{zhong1qing2ju2}
    \definition*{s.}{Agência Central de Inteligência dos EUA, CIA (abreviação de 中央情报局)}
  \seealsoref{中央情报局}{zhong1yang1 qing2bao4ju2}
  \end{phonetics}
\end{entry}

\begin{entry}{中断}{4,11}{⼁、⽄}
  \begin{phonetics}{中断}{zhong1duan4}[][HSK 5]
    \definition{v.}{suspender; romper; descontinuar; interromper; quebrar | dividir; quebrar; ser quebrado}
  \end{phonetics}
\end{entry}

\begin{entry}{中意}{4,13}{⼁、⼼}
  \begin{phonetics}{中意}{zhong4yi4}
    \definition{s.}{ser do seu agrado | começar a gostar muito de algo ou de alguém}
  \end{phonetics}
\end{entry}

\begin{entry}{中餐}{4,16}{⼁、⾷}
  \begin{phonetics}{中餐}{zhong1 can1}[][HSK 2]
    \definition[份,顿,桌]{s.}{refeição chinesa; comida chinesa; comida de estilo chinês (diferente de 西餐) | almoço}
  \seealsoref{西餐}{xi1 can1}
  \end{phonetics}
\end{entry}

\begin{entry}{丰}{4}{⼁}
  \begin{phonetics}{丰}{feng1}
    \definition*{s.}{Sobrenome Feng}
    \definition[阵,丝]{adj.}{cheio; rico; abundante | ótimo | bonito; de boa aparência; cheio e redondo}
  \end{phonetics}
\end{entry}

\begin{entry}{丰收}{4,6}{⼁、⽁}
  \begin{phonetics}{丰收}{feng1shou1}[][HSK 5]
    \definition{v.}{ter uma boa colheita; obter uma colheita boa e abundante; obter bons resultados}
  \end{phonetics}
\end{entry}

\begin{entry}{丰富}{4,12}{⼁、⼧}
  \begin{phonetics}{丰富}{feng1fu4}[][HSK 3]
    \definition{adj.}{rico; abundante; pleno; (riqueza material, conhecimento, experiência, etc.) variedade ou quantidade}
    \definition{v.}{enriquecer}
  \end{phonetics}
\end{entry}

\begin{entry}{串}{7}{⼁}
  \begin{phonetics}{串}{chuan4}[][HSK 6]
    \definition{clas.}{corda; grupo; aglomerado; usado para amarrar coisas}
    \definition{s.}{espeto}
    \definition{v.}{atuar; desempenhar um papel (em uma peça) | misturar as coisas (de maneira caótica) | vagar; correr; ir de um lugar para outro | conspirar (conluio, depreciativo) | encadear juntos; conectar as coisas uma a uma para formar um todo | misturar; refere-se à mistura de coisas diferentes e à alteração de suas características originais}
  \end{phonetics}
\end{entry}

\begin{entry}{临}{9}{⼁}
  \begin{phonetics}{临}{lin2}
    \definition*{s.}{Sobrenome Lin}
    \definition{adv.}{pouco antes; prestes a; no ponto de}
    \definition{v.}{encarar; enfrentar; aproximar-se | chegar; estar presente | copiar (um modelo de caligrafia ou pintura); traçar sobre as palavras ou figuras | olhar de cima para baixo | ir de cima para baixo}
  \end{phonetics}
\end{entry}

\begin{entry}{临时}{9,7}{⼁、⽇}
  \begin{phonetics}{临时}{lin2shi2}[][HSK 4]
    \definition{adj.}{temporário; provisório; por um breve período}
    \definition{adv.}{no momento em que algo acontece (quando as coisas dão errado)}
  \end{phonetics}
\end{entry}

\begin{entry}{临近}{9,7}{⼁、⾡}
  \begin{phonetics}{临近}{lin2jin4}
    \definition{v.}{aproximar-se; estar perto de}
  \end{phonetics}
\end{entry}

%%%%% EOF %%%%%

