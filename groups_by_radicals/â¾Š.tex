%%%
%%% Radical "⾊"
%%%

\section*{Radical 139: ``⾊''}\addcontentsline{toc}{section}{Radical 139: ⾊}

\begin{Entry}{色}{6}{⾊}[Kangxi 139]
  \begin{Phonetics}{色}{se4}[][HSK 4]
    \definition*{s.}{Sobrenome Se}
    \definition[种]{s.}{cor | aparência; semblante; expressão | tipo; gênero; descrição | cena; cenário;  paisagem | qualidade (de metais preciosos, mercadorias, etc.) | aparência feminina; beleza feminina | erotismo; apetite sexual; luxúria; desejo sexual}
  \end{Phonetics}
  \begin{Phonetics}{色}{shai3}
    \definition[4]{s.}{cor; (~儿) tem o mesmo significado que "色", usado em algumas palavras faladas}
  \end{Phonetics}
\end{Entry}

\begin{Entry}{色狼}{6,10}{⾊、⽝}
  \begin{Phonetics}{色狼}{se4lang2}
    \definition*{s.}{Sátiro}
    \definition{adj.}{depravado | tarado}
  \end{Phonetics}
\end{Entry}

\begin{Entry}{色彩}{6,11}{⾊、⼺}
  \begin{Phonetics}{色彩}{se4cai3}[][HSK 4]
    \definition[种,丝]{s.}{cor; matiz; tonalidade | cor; sabor; característica; metáfora para um determinado estado de espírito ou tendência de pensamento}
  \end{Phonetics}
\end{Entry}

%%%%% EOF %%%%%

