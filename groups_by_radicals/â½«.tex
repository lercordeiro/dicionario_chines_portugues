%%%
%%% Radical "⽫"
%%%

\section*{Radical 108: ``⽫''}\addcontentsline{toc}{section}{Radical 108: ⽫}

\begin{entry}{盆}{9}{⽫}
  \begin{phonetics}{盆}{pen2}[][HSK 5]
    \definition*{s.}{Sobrenome Pen}
    \definition[个]{s.}{bacia; banheira; panela; utensílios para guardar ou lavar coisas}
  \end{phonetics}
\end{entry}

\begin{entry}{盆友}{9,4}{⽫、⼜}
  \begin{phonetics}{盆友}{pen2you3}
    \definition{s.}{(gíria na \emph{Internet}) amigo (trocadilho com 朋友)}
  \seealsoref{朋友}{peng2you5}
  \end{phonetics}
\end{entry}

\begin{entry}{盏}{10}{⽫}
  \begin{phonetics}{盏}{zhan3}
    \definition{clas.}{para lâmpadas}
    \definition{s.}{copo pequeno}
  \end{phonetics}
\end{entry}

\begin{entry}{盐}{10}{⽫}
  \begin{phonetics}{盐}{yan2}[][HSK 4]
    \definition*{s.}{Sobrenome Yan}
    \definition{s.}{sal; sais}
  \end{phonetics}
\end{entry}

\begin{entry}{监}{10}{⽫}
  \begin{phonetics}{监}{jian1}
    \definition{s.}{prisão; cadeia}
    \definition{v.}{supervisionar; inspecionar; observar}
  \end{phonetics}
\end{entry}

\begin{entry}{监测}{10,9}{⽫、⽔}
  \begin{phonetics}{监测}{jian1 ce4}[][HSK 6]
    \definition{v.}{monitorar; supervisionar e testar}
  \end{phonetics}
\end{entry}

\begin{entry}{监狱}{10,9}{⽫、⽝}
  \begin{phonetics}{监狱}{jian1yu4}
    \definition{s.}{prisão}
  \end{phonetics}
\end{entry}

\begin{entry}{监督}{10,13}{⽫、⽬}
  \begin{phonetics}{监督}{jian1du1}[][HSK 6]
    \definition[个,位,名]{s.}{monitoramento; supervisão; pessoas que supervisionam}
    \definition{v.}{controlar; supervisionar; superintender; monitorar e supervisionar de perto}
  \end{phonetics}
\end{entry}

\begin{entry}{盒}{11}{⽫}
  \begin{phonetics}{盒}{he2}[][HSK 5]
    \definition{clas.}{caixa (de pequena dimensão)}
    \definition[个]{s.}{caixa; estojo; recipiente; receptáculo}
  \end{phonetics}
\end{entry}

\begin{entry}{盒子}{11,3}{⽫、⼦}
  \begin{phonetics}{盒子}{he2zi5}[][HSK 5]
    \definition[个]{s.}{caixa; recipiente que têm tampas na parte superior e podem conter coisas dentro, geralmente é pequeno e plano}
  \end{phonetics}
\end{entry}

\begin{entry}{盒饭}{11,7}{⽫、⾷}
  \begin{phonetics}{盒饭}{he2 fan4}[][HSK 5]
    \definition{s.}{refeição embalada; marmita; \emph{fast-food} vendida em caixas}
  \end{phonetics}
\end{entry}

\begin{entry}{盖}{11}{⽫}
  \begin{phonetics}{盖}{gai4}[][HSK 4]
    \definition*{s.}{Sobrenome Gai}
    \definition{adj.}{excelente; soberbo; fantástico}
    \definition{adv.}{cerca de; ao redor; aproximadamente; expressa um julgamento especulativo sobre algo, ou uma explicação da causa, o que é equivalente a 大概 ou 原来}
    \definition{conj.}{para; porque; dando continuidade à frase anterior, afirmando a razão ou causa, com tom incerto}
    \definition{s.}{tampa; capa; cobertura; algo que cobre ou sela a parte superior de um objeto | carapaça; concha (de tartaruga, caranguejo, etc.); ossos em formato de crânio em certas partes do corpo humano; as conchas nas costas de certos animais | dossel; capota; toldo | nivelador (uma ferramenta agrícola usada para nivelar terras)}
    \definition{v.}{cobrir; proteger; colocar uma capa em; colocar uma tampa em um objeto | selar; afixar um selo em | superar; sobressair; sobrepujar; ultrapassar | construir; colocar para cima | esconder; ocultar; encobrir | nivelar o terreno com um nivelador (ferramenta agrícola)}
  \seealsoref{大概}{da4gai4}
  \seealsoref{原来}{yuan2lai2}
  \end{phonetics}
  \begin{phonetics}{盖}{ge3}
    \definition*{s.}{Sobrenome Ge}
  \end{phonetics}
\end{entry}

\begin{entry}{盗}{11}{⽫}
  \begin{phonetics}{盗}{dao4}
    \definition[个,伙,帮,窝]{s.}{ladrão; assaltante}
    \definition{v.}{roubar; saquear | usurpar; buscar ganho pessoal ou ganho por meios impróprios}
  \end{phonetics}
\end{entry}

\begin{entry}{盗版}{11,8}{⽫、⽚}
  \begin{phonetics}{盗版}{dao4 ban3}[][HSK 6]
    \definition{s.}{cópia ilegal; cópia pirata; refere-se a livros, periódicos e produtos audiovisuais pirateados (diferentes dos 正版)}
    \definition{v.}{piratear; copiar ou vender ilegalmente; para obter lucros enormes, reimprimir ou copiar livros, periódicos ou produtos audiovisuais em grandes quantidades sem o consentimento do detentor dos direitos autorais}
  \seealsoref{正版}{zheng4 ban3}
  \end{phonetics}
\end{entry}

\begin{entry}{盘}{11}{⽫}
  \begin{phonetics}{盘}{pan2}[][HSK 4]
    \definition*{s.}{Sobrenome Pan}
    \definition{clas.}{para pratos, pedras de moer, etc. | para jogos de xadrez e de bola | para as coisas que estão entrelaçadas, emaranhadas}
    \definition{s.}{bandeja; tabuleiro | recipiente plano e raso, como uma bandeja, prato, travessa etc.  | preço atual; cotação de mercado; refere-se ao preço básico pelo qual as commodities são negociadas}
    \definition{v.}{enrolar; torcer; enrolar (para cima); entrelaçar; cercar | construir (assentando tijolos, pedras, etc.) | checar; examinar; interrogar; verificar um por um ou repetidamente (quantidade, situação, etc.) | transferir a propriedade de; passar para outra pessoa | carregar; transportar}
  \end{phonetics}
\end{entry}

\begin{entry}{盘子}{11,3}{⽫、⼦}
  \begin{phonetics}{盘子}{pan2zi5}[][HSK 4]
    \definition[个,叠,套,只]{s.}{prato; utensílio de fundo raso para guardar objetos, maior do que um pires, geralmente de formato redondo | situação de mercado; taxa de mercado; transação comercial}
  \end{phonetics}
\end{entry}

\begin{entry}{盛}{11}{⽫}
  \begin{phonetics}{盛}{cheng2}
    \definition{v.}{encher; encher com uma concha; colocar as coisas em recipientes; especialmente colocar alimentos em tigelas, pratos e outros recipientes | segurar; conter; acomodar}
  \end{phonetics}
  \begin{phonetics}{盛}{sheng4}
    \definition*{s.}{Sobrenome Sheng}
    \definition{adj.}{florescente; próspero | vigoroso; enérgico | grandioso; magnífico | abundante; profundo | popular; comum; difundido; universal | amplo; generoso; abundante; suficiente | ótimo}
    \definition{adv.}{muito; profundamente}
  \end{phonetics}
\end{entry}

\begin{entry}{盛行}{11,6}{⽫、⾏}
  \begin{phonetics}{盛行}{sheng4xing2}[][HSK 6]
    \definition{v.}{predominar; estar atual; estar na moda; ser amplamente popular}
  \end{phonetics}
\end{entry}

\begin{entry}{盛宴}{11,10}{⽫、⼧}
  \begin{phonetics}{盛宴}{sheng4yan4}
    \definition{s.}{celebração}
  \end{phonetics}
\end{entry}

%%%%% EOF %%%%%

