%%%
%%% Radical "⾡"
%%%

\section*{Radical 162: ``⾡'' (⻌、⻍、⻎)}\addcontentsline{toc}{section}{Radical 162: ⾡、⻌、⻍、⻎}

\begin{Entry}{边}{5}{⾡}
  \begin{Phonetics}{边}{bian1}[][HSK 2]
    \definition*{s.}{Sobrenome Bian}
    \definition{adv.}{dois ou mais 边 são usados separadamente antes de diferentes verbos, indicando que diferentes ações ocorrem simultaneamente}
    \definition[条,个]{s.}{lado (de uma figura geométrica) | borda; lado; margem; aba; rebordo | fronteira; limite | ao lado de; lugar próximo a; perto de um objeto; lateral | aro; aba; borda; decoração em forma de faixa incrustada ou pintada na borda de um objeto}
    \definition{suf.}{lado; anexado a palavras de localização monossilábicas, formando palavras de localização dissílabas}
  \end{Phonetics}
  \begin{Phonetics}{边}{bian5}
    \definition{suf.}{sufixo de uma palavra de localidade (lado); indica posição e direção, usado após palavras que indicam direção, como 上, 下, 前, 后, 左, 右}
  \end{Phonetics}
\end{Entry}

\begin{Entry}{边关}{5,6}{⾡、⼋}
  \begin{Phonetics}{边关}{bian1guan1}
    \definition{s.}{posto de fronteira | posição defensiva estratégica na fronteira}
  \end{Phonetics}
\end{Entry}

\begin{Entry}{边防}{5,6}{⾡、⾩}
  \begin{Phonetics}{边防}{bian1fang2}
    \definition{s.}{defesa da fronteira}
  \end{Phonetics}
\end{Entry}

\begin{Entry}{边远}{5,7}{⾡、⾡}
  \begin{Phonetics}{边远}{bian1yuan3}[][HSK 7-9]
    \definition{adj.}{longe do centro; remoto; periférico; perto da fronteira; longe da área central}
  \end{Phonetics}
\end{Entry}

\begin{Entry}{边界}{5,9}{⾡、⽥}
  \begin{Phonetics}{边界}{bian1jie4}[][HSK 7-9]
    \definition{s.}{limite; linha de fronteira; fronteiras entre países ou regiões}
  \end{Phonetics}
\end{Entry}

\begin{Entry}{边缘}{5,12}{⾡、⽷}
  \begin{Phonetics}{边缘}{bian1yuan2}[][HSK 6]
    \definition{s.}{borda; beira; franja; uma área ou objeto próximo ao extremo |  borda; beira; algo está muito próximo de uma situação perigosa | interdisciplinar; relacionado a muitas coisas}
  \end{Phonetics}
\end{Entry}

\begin{Entry}{边境}{5,14}{⾡、⼟}
  \begin{Phonetics}{边境}{bian1jing4}[][HSK 5]
    \definition{s.}{fronteira; borda; perto da fronteira}
  \end{Phonetics}
\end{Entry}

\begin{Entry}{边疆}{5,19}{⾡、⼸}
  \begin{Phonetics}{边疆}{bian1jiang1}[][HSK 7-9]
    \definition{s.}{fronteira; área de fronteira; região de fronteira; território próximo à fronteira}
  \end{Phonetics}
\end{Entry}

\begin{Entry}{巡}{6}{⾡}
  \begin{Phonetics}{巡}{xun2}
    \definition{clas.}{rodada de bebidas | usado para servir vinho a todos}
    \definition{v.}{patrulhar; fazer rondas; fazer uma excursão de inspeção}
  \end{Phonetics}
\end{Entry}

\begin{Entry}{巡逻}{6,11}{⾡、⾡}
  \begin{Phonetics}{巡逻}{xun2luo2}
    \definition{s.}{patrulha}
    \definition{v.}{patrulhar (polícia, exército ou marinha)}
  \end{Phonetics}
\end{Entry}

\begin{Entry}{达}{6}{⾡}
  \begin{Phonetics}{达}{da2}
    \definition*{s.}{Sobrenome Da}
    \definition{adj.}{eminente; distinto; refere-se a um funcionário distinto; \emph{status} elevado | otimista; de mente aberta}
    \definition{v.}{prolongar | alcançar; atingir; equivaler a | entender completamente; compreender (assuntos) | expressar; comunicar}
  \end{Phonetics}
\end{Entry}

\begin{Entry}{达成}{6,6}{⾡、⼽}
  \begin{Phonetics}{达成}{da2cheng2}[][HSK 5]
    \definition{v.}{concluir; chegar (a um acordo); conseguir; obter (principalmente como resultado de uma negociação)}
  \end{Phonetics}
\end{Entry}

\begin{Entry}{达到}{6,8}{⾡、⼑}
  \begin{Phonetics}{达到}{da2dao4}[][HSK 3]
    \definition{v.}{alcançar; atender o padrão; atingir (refere-se principalmente a coisas abstratas ou graus); chegar a um determinado ponto ou grau}
  \end{Phonetics}
\end{Entry}

\begin{Entry}{达标}{6,9}{⾡、⽊}
  \begin{Phonetics}{达标}{da2biao1}[][HSK 7-9]
    \definition{v.}{atingir um padrão definido (até o padrão) | atingir um padrão definido (qualifica)}
  \end{Phonetics}
\end{Entry}

\begin{Entry}{迅}{6}{⾡}
  \begin{Phonetics}{迅}{xun4}
    \definition{adj.}{rápido; veloz}
    \definition{adv.}{rapidamente; velozmente}
  \end{Phonetics}
\end{Entry}

\begin{Entry}{迅速}{6,10}{⾡、⾡}
  \begin{Phonetics}{迅速}{xun4su4}[][HSK 4]
    \definition{adv.}{rapidamente; velozmente; prontamente}
  \end{Phonetics}
\end{Entry}

\begin{Entry}{过}{6}{⾡}
  \begin{Phonetics}{过}{guo1}
    \definition*{s.}{Sobrenome Guo}
  \end{Phonetics}
  \begin{Phonetics}{过}{guo4}[][HSK 1,2]
    \definition{adv.}{excessivamente; em excesso}
    \definition{clas.}{tempo; número de vezes usado para a ação}
    \definition{s.}{falha; erro; demérito; equívoco; negligência; (oposto a 功)}
    \definition{v.}{cruzar; passar; mudar-se de um lugar para outro; passar por | exceder; ir além; ultrapassar; usado após um adjetivo, significa ``mais do que'' | gastar (tempo); passar (tempo); exceder (um determinado limite ou limite) | celebrar; comemorar | mudar; transferir; transferir de um lado para o outro | passar por um processo; passar por; submeter a (algum tipo de tratamento) | visitar; fazer uma visita | falecer; morrer | infectar; ser contagioso; espalhar | exceder; ir além; usado após o verbo com o sufixo 得, significa ``superar'' ou ``passar'' | viver | revisar; examinar; usar os olhos para ver ou a mente para lembrar}
  \seealsoref{得}{de5}
  \seealsoref{功}{gong1}
  \end{Phonetics}
  \begin{Phonetics}{过}{guo5}
    \definition{part.}{usado depois de um verbo para indicar conclusão | usado depois de um verbo para indicar que uma ação ou mudança ocorreu | usado depois de um adjetivo para indicar que algo já teve uma certa qualidade ou estado e para compará-lo com o presente}
  \end{Phonetics}
\end{Entry}

\begin{Entry}{过于}{6,3}{⾡、⼆}
  \begin{Phonetics}{过于}{guo4yu2}[][HSK 5]
    \definition{adv.}{demais; indevidamente; excessivamente; advérbios de grau ou quantidade excessiva}
  \end{Phonetics}
\end{Entry}

\begin{Entry}{过不去}{6,4,5}{⾡、⼀、⼛}
  \begin{Phonetics}{过不去}{guo4bu5qu4}[][HSK 7-9]
    \definition{v.}{não poder passar; ser incapaz de passar; ser ou estar bloqueado; ser intransitável | Coloquial: ser duro com; dificultar; envergonhar; colocar para fora | sentir pena; sentir-se mal | encontrar falhas em}
  \end{Phonetics}
\end{Entry}

\begin{Entry}{过不惯}{6,4,11}{⾡、⼀、⼼}
  \begin{Phonetics}{过不惯}{guo4 bu5 guan4}
    \definition{v.}{não se acostumar; não se habituar}
  \seealsoref{过惯}{guo4guan4}
  \end{Phonetics}
\end{Entry}

\begin{Entry}{过分}{6,4}{⾡、⼑}
  \begin{Phonetics}{过分}{guo4fen4}[][HSK 4]
    \definition{adj.}{excessivo; muito longe; demais; falar ou agir além dos limites ou graus adequados}
    \definition{adv.}{excessivamente; indevidamente; muito mesmo}
  \end{Phonetics}
\end{Entry}

\begin{Entry}{过日子}{6,4,3}{⾡、⽇、⼦}
  \begin{Phonetics}{过日子}{guo4 ri4zi5}[][HSK 7-9]
    \definition{v.}{viver; conviver; passar/viver a própria vida}
  \end{Phonetics}
\end{Entry}

\begin{Entry}{过半}{6,5}{⾡、⼗}
  \begin{Phonetics}{过半}{guo4ban4}[][HSK 7-9]
    \definition{s.}{maioria; mais da metade; mais de cinquenta por cento}
  \end{Phonetics}
\end{Entry}

\begin{Entry}{过去}{6,5}{⾡、⼛}
  \begin{Phonetics}{过去}{guo4 qu4}[][HSK 2,3]
    \definition{adv.}{(no) passado}
    \definition{s.}{o passado; refere-se a um período anterior; também se refere a coisas anteriores}
    \definition{v.}{atravessar; passar; sair do local onde o interlocutor se encontra e deslocar-se para outro local | acabar; passar; ficar para trás; indica que já passou por uma determinada fase | passar; indica que um determinado período ou situação já não existe mais | falecer | ir lá | passar por}
  \end{Phonetics}
\end{Entry}

\begin{Entry}{过失}{6,5}{⾡、⼤}
  \begin{Phonetics}{过失}{guo4shi1}[][HSK 7-9]
    \definition{s.}{falha; deslize; erro; erros cometidos por negligência | negligência; crime por negligência}
  \end{Phonetics}
\end{Entry}

\begin{Entry}{过头}{6,5}{⾡、⼤}
  \begin{Phonetics}{过头}{guo4/tou2}[][HSK 7-9]
    \definition{adv.}{excessivamente; acima da cabeça; por cima; ao alto}
    \definition{v.+compl.}{exagerar; ir além do limite; exceder o limite; ser excessivo}
  \end{Phonetics}
\end{Entry}

\begin{Entry}{过节}{6,5}{⾡、⾋}
  \begin{Phonetics}{过节}{guo4/jie2}[][HSK 7-9]
    \definition{v.+compl.}{celebrar um festival; passar as férias; comemorar durante as férias}[今年我们一起过节吧!===Vamos comemorar as festas juntos este ano!]
  \end{Phonetics}
\end{Entry}

\begin{Entry}{过关}{6,6}{⾡、⼋}
  \begin{Phonetics}{过关}{guo4/guan1}[][HSK 7-9]
    \definition{v.+compl.}{passar (um teste); alcançar (um padrão); cruzar uma barreira; superar (uma provação) ; passar por um posto de controle, frequentemente usado como metáfora}
  \end{Phonetics}
\end{Entry}

\begin{Entry}{过后}{6,6}{⾡、⼝}
  \begin{Phonetics}{过后}{guo4 hou4}[][HSK 6]
    \definition[期]{s.}{depois; mais tarde}
  \end{Phonetics}
\end{Entry}

\begin{Entry}{过年}{6,6}{⾡、⼲}
  \begin{Phonetics}{过年}{guo4/nian2}[][HSK 2]
    \definition{v.+compl.}{comemorar o Ano Novo; comemorar o Festival da Primavera; passar o Ano Novo; passar o Festival da Primavera; realizar atividades comemorativas durante o Ano Novo ou o Festival da Primavera}
  \end{Phonetics}
\end{Entry}

\begin{Entry}{过早}{6,6}{⾡、⽇}
  \begin{Phonetics}{过早}{guo4 zao3}[][HSK 7-9]
    \definition{adj.}{prematuro; inoportuno | Dialeto: café da manha}
    \definition{adv.}{muito cedo; prematuramente | Dialeto: tomar o café da manhã}
  \end{Phonetics}
\end{Entry}

\begin{Entry}{过时}{6,7}{⾡、⽇}
  \begin{Phonetics}{过时}{guo4 shi2}[][HSK 6]
    \definition{adj.}{fora de moda; obsoleto; antiquado; desatualizado; o que era popular no passado não é mais popular}
    \definition{v.}{passar do tempo marcado; estar após o tempo estipulado}
  \end{Phonetics}
\end{Entry}

\begin{Entry}{过来}{6,7}{⾡、⽊}
  \begin{Phonetics}{过来}{guo4 lai2}[][HSK 2]
    \definition{v.}{vir até aqui | ser capaz de cuidar de | lidar com | administrar}
  \end{Phonetics}
\end{Entry}

\begin{Entry}{过往}{6,8}{⾡、⼻}
  \begin{Phonetics}{过往}{guo4wang3}[][HSK 7-9]
    \definition{v.}{ir e vir | ter relações amigáveis ​​com; associar-se a; lidar com}
  \end{Phonetics}
\end{Entry}

\begin{Entry}{过奖}{6,9}{⾡、⼤}
  \begin{Phonetics}{过奖}{guo4jiang3}[][HSK 7-9]
    \definition{v.}{elogiar demais; bajular; dar elogios imerecidos}
  \end{Phonetics}
\end{Entry}

\begin{Entry}{过度}{6,9}{⾡、⼴}
  \begin{Phonetics}{过度}{guo4du4}[][HSK 5]
    \definition{adj.}{excessivo; acima do limite; além do limite; além do que é apropriado}
  \end{Phonetics}
\end{Entry}

\begin{Entry}{过惯}{6,11}{⾡、⼼}
  \begin{Phonetics}{过惯}{guo4guan4}
    \definition{v.}{estar acostumado (a um certo estilo de vida, etc.)}
  \seealsoref{过不惯}{guo4 bu5 guan4}
  \end{Phonetics}
\end{Entry}

\begin{Entry}{过敏}{6,11}{⾡、⽁}
  \begin{Phonetics}{过敏}{guo4min3}[][HSK 5]
    \definition{adj.}{sensível; excessivamente sensível; resposta acima do normal; ceticismo excessivo}
    \definition{v.}{ser alérgico a}
  \end{Phonetics}
\end{Entry}

\begin{Entry}{过剩}{6,12}{⾡、⼑}
  \begin{Phonetics}{过剩}{guo4sheng4}[][HSK 7-9]
    \definition{v.}{exceder; a quantidade excede em muito o limite necessário | saturar; oferecer em excesso; a oferta excede a demanda do mercado ou o poder de compra}
  \end{Phonetics}
\end{Entry}

\begin{Entry}{过期}{6,12}{⾡、⽉}
  \begin{Phonetics}{过期}{guo4/qi1}[][HSK 7-9]
    \definition{v.+compl.}{expirar; estar vencido; exceder o limite de tempo; exceder o período prescrito ou acordado}
  \end{Phonetics}
\end{Entry}

\begin{Entry}{过渡}{6,12}{⾡、⽔}
  \begin{Phonetics}{过渡}{guo4du4}[][HSK 6]
    \definition{v.}{fazer a transição; estar em transição; estar em fase de transição; mudar de um estágio para outro | atravessar; cruzar}
  \end{Phonetics}
\end{Entry}

\begin{Entry}{过硬}{6,12}{⾡、⽯}
  \begin{Phonetics}{过硬}{guo4/ying4}[][HSK 7-9]
    \definition{adj.}{perfeito; soberbo; à altura; verdadeiramente proficiente}
    \definition{v.+compl.}{ter domínio perfeito de algo; estar à altura; resistir a testes ou exames rigorosos}
  \end{Phonetics}
\end{Entry}

\begin{Entry}{过程}{6,12}{⾡、⽲}
  \begin{Phonetics}{过程}{guo4cheng2}[][HSK 3]
    \definition[个,段]{s.}{curso dos eventos; processo; o processo pelo qual as coisas acontecem ou se desenvolvem.}
  \end{Phonetics}
\end{Entry}

\begin{Entry}{过道}{6,12}{⾡、⾡}
  \begin{Phonetics}{过道}{guo4dao4}[][HSK 7-9]
    \definition{s.}{corredor; caminho; passarela; passagem; o corredor da porta para cada cômodo da nova casa | passagem; uma passarela que conecta os pátios de uma casa antiga, especialmente o cômodo ou metade do cômodo onde o portão está localizado}
  \end{Phonetics}
\end{Entry}

\begin{Entry}{过意不去}{6,13,4,5}{⾡、⼼、⼀、⼛}
  \begin{Phonetics}{过意不去}{guo4yi4bu2qu4}[][HSK 7-9]
    \definition{expr.}{sentir-se arrependido ou culpado; sentir-se mal ou envergonhado; sentir-se envergonhado ou arrependido; metáfora para aceitar um favor de alguém, mas não retribuí-lo, ou sentir pena de algo e não ser culpado, o que faz com que alguém se sinta arrependido e desconfortável}
  \end{Phonetics}
\end{Entry}

\begin{Entry}{过滤}{6,13}{⾡、⽔}
  \begin{Phonetics}{过滤}{guo4lv4}[][HSK 7-9]
    \definition{v.}{filtrar; separar sólidos ou componentes nocivos de gases ou líquidos por meio de materiais porosos, como papel de filtro e pano de filtro}[所有饮用水必须经过过滤。===Toda água potável deve ser filtrada.]
  \end{Phonetics}
\end{Entry}

\begin{Entry}{过路人}{6,13,2}{⾡、⾜、⼈}
  \begin{Phonetics}{过路人}{guo4lu4 ren2}
    \definition{s.}{transeunte}
  \end{Phonetics}
\end{Entry}

\begin{Entry}{过错}{6,13}{⾡、⾦}
  \begin{Phonetics}{过错}{guo4cuo4}[][HSK 7-9]
    \definition{s.}{falha; erro; engano | ações ilícitas; no direito civil, refere-se a atos ilegais que prejudicam outras pessoas intencionalmente ou negligentemente}
  \end{Phonetics}
\end{Entry}

\begin{Entry}{过境}{6,14}{⾡、⼟}
  \begin{Phonetics}{过境}{guo4/jing4}[][HSK 7-9]
    \definition{v.+compl.}{estar em trânsito; passar pelo território de um país}
  \end{Phonetics}
\end{Entry}

\begin{Entry}{过瘾}{6,16}{⾡、⽧}
  \begin{Phonetics}{过瘾}{guo4/yin3}[][HSK 7-9]
    \definition{adj.}{gratificante; imensamente agradável; satisfatório; realizador}
    \definition{v.+compl.}{satisfazer um desejo; divertir-se ao máximo; fazer algo à vontade}
  \end{Phonetics}
\end{Entry}

\begin{Entry}{迎}{7}{⾡}
  \begin{Phonetics}{迎}{ying2}
    \definition{v.}{ir ao encontro; cumprimentar; acolher; receber | mover-se em direção a; encontrar-se cara a cara}
  \end{Phonetics}
\end{Entry}

\begin{Entry}{迎来}{7,7}{⾡、⽊}
  \begin{Phonetics}{迎来}{ying2 lai2}[][HSK 6]
    \definition{v.}{dar boas-vindas; cumprimentar | introduzir}
  \end{Phonetics}
\end{Entry}

\begin{Entry}{迎接}{7,11}{⾡、⼿}
  \begin{Phonetics}{迎接}{ying2jie1}[][HSK 3]
    \definition{v.}{conhecer; cumprimentar; felicitar; dar as boas-vindas | cumprimentar; felicitar; dar as boas-vindas; preparar-se; aguardar a chegada de um determinado momento ou evento}
  \end{Phonetics}
\end{Entry}

\begin{Entry}{运}{7}{⾡}
  \begin{Phonetics}{运}{yun4}[][HSK 5]
    \definition*{s.}{Sobrenome Yun}
    \definition{s.}{sorte; destino; fortuna}
    \definition{v.}{mover; deslocar | transportar; levar | usar; empunhar; utilizar}
  \end{Phonetics}
\end{Entry}

\begin{Entry}{运气}{7,4}{⾡、⽓}
  \begin{Phonetics}{运气}{yun4/qi4}
    \definition{v.+compl.}{tentar a sorte | concentrar a energia em uma parte do corpo}[他们地运一口气。===Eles respiraram fundo.]
  \end{Phonetics}
  \begin{Phonetics}{运气}{yun4qi5}[][HSK 4]
    \definition{adj.}{sortudo; afortunado}
    \definition{s.}{sorte; fortuna}
  \end{Phonetics}
\end{Entry}

\begin{Entry}{运用}{7,5}{⾡、⽤}
  \begin{Phonetics}{运用}{yun4yong4}[][HSK 4]
    \definition{v.}{usar; utilizar; manejar; aplicar; explorar as coisas de acordo com suas características}
  \end{Phonetics}
\end{Entry}

\begin{Entry}{运动}{7,6}{⾡、⼒}
  \begin{Phonetics}{运动}{yun4dong4}[][HSK 2]
    \definition[项,种,场,次]{s.}{esportes; atletismo; exercício; atividades esportivas | movimento; campanha (política); atividades de massa organizadas, intencionais e de alto nível na política, cultura, produção, etc. | movimento; refere-se a todas as mudanças}
    \definition{v.}{exercitar; fazer atividade física | mover-se; refere-se à mudança na posição de um objeto}
  \end{Phonetics}
\end{Entry}

\begin{Entry}{运动会}{7,6,6}{⾡、⼒、⼈}
  \begin{Phonetics}{运动会}{yun4 dong4 hui4}[][HSK 4]
    \definition[届,场,次,个]{s.}{jogos; encontro esportivo; dia de esportes; encontro atlético; competição esportiva abrangente}
  \end{Phonetics}
\end{Entry}

\begin{Entry}{运动场}{7,6,6}{⾡、⼒、⼟}
  \begin{Phonetics}{运动场}{yun4dong4chang3}
    \definition{s.}{campo desportivo | campo de jogos}
  \end{Phonetics}
\end{Entry}

\begin{Entry}{运动员}{7,6,7}{⾡、⼒、⼝}
  \begin{Phonetics}{运动员}{yun4 dong4 yuan2}[][HSK 4]
    \definition[名,个,班]{s.}{jogador; atleta; esportista; pessoas que participam de competições esportivas}
  \end{Phonetics}
\end{Entry}

\begin{Entry}{运动学}{7,6,8}{⾡、⼒、⼦}
  \begin{Phonetics}{运动学}{yun4dong4xue2}
    \definition{s.}{cinemática; um ramo da ciência do esporte que usa a anatomia e a mecânica humanas para explicar várias atividades esportivas}
  \end{Phonetics}
\end{Entry}

\begin{Entry}{运动服}{7,6,8}{⾡、⼒、⽉}
  \begin{Phonetics}{运动服}{yun4dong4fu2}
    \definition{s.}{roupa para prática de esporte}
  \end{Phonetics}
\end{Entry}

\begin{Entry}{运动衫}{7,6,8}{⾡、⼒、⾐}
  \begin{Phonetics}{运动衫}{yun4dong4shan1}
    \definition[件]{s.}{moletom | camisa esportiva}
  \end{Phonetics}
\end{Entry}

\begin{Entry}{运动家}{7,6,10}{⾡、⼒、⼧}
  \begin{Phonetics}{运动家}{yun4dong4jia1}
    \definition{s.}{ativista | atleta | esportista}
  \end{Phonetics}
\end{Entry}

\begin{Entry}{运动病}{7,6,10}{⾡、⼒、⽧}
  \begin{Phonetics}{运动病}{yun4dong4bing4}
    \definition{s.}{enjôo (movimento, carro, etc.)}
  \end{Phonetics}
\end{Entry}

\begin{Entry}{运动鞋}{7,6,15}{⾡、⼒、⾰}
  \begin{Phonetics}{运动鞋}{yun4dong4xie2}
    \definition{s.}{tênis | sapatos esportivos}
  \end{Phonetics}
\end{Entry}

\begin{Entry}{运行}{7,6}{⾡、⾏}
  \begin{Phonetics}{运行}{yun4xing2}[][HSK 5]
    \definition{v.}{correr; mover; trabalhar; estar em movimento; (veículo, nave, planeta, etc.) mover-se em um ciclo repetitivo; avançar de maneira regular e direcional}
  \end{Phonetics}
\end{Entry}

\begin{Entry}{运作}{7,7}{⾡、⼈}
  \begin{Phonetics}{运作}{yun4 zuo4}[][HSK 6]
    \definition{v.}{trabalhar; operar; (uma instituição, organização, etc.) realizar trabalho; realizar atividades}
  \end{Phonetics}
\end{Entry}

\begin{Entry}{运河}{7,8}{⾡、⽔}
  \begin{Phonetics}{运河}{yun4he2}
    \definition{s.}{canal (em um rio)}
  \end{Phonetics}
\end{Entry}

\begin{Entry}{运输}{7,13}{⾡、⾞}
  \begin{Phonetics}{运输}{yun4shu1}[][HSK 3]
    \definition{v.}{enviar; transportar; transportar pessoas ou coisas de um lugar para outro usando carros, barcos, aviões, etc.}
  \end{Phonetics}
\end{Entry}

\begin{Entry}{近}{7}{⾡}
  \begin{Phonetics}{近}{jin4}[][HSK 2]
    \definition{adj.}{próximo; perto; distância espacial ou temporal curta (oposto de 远) | íntimo; intimamente relacionado; relação estreita | fácil de entender}
  \seealsoref{远}{yuan3}
  \end{Phonetics}
\end{Entry}

\begin{Entry}{近日}{7,4}{⾡、⽇}
  \begin{Phonetics}{近日}{jin4 ri4}[][HSK 6]
    \definition{s.}{recentemente; nos últimos dias; apontando para o passado | nos próximos dias; refere-se ao futuro}
  \end{Phonetics}
\end{Entry}

\begin{Entry}{近代}{7,5}{⾡、⼈}
  \begin{Phonetics}{近代}{jin4dai4}[][HSK 4]
    \definition{s.}{tempos modernos; era passada relativamente próxima à era moderna, geralmente referida na história chinesa como 1840 a 1919 | o tempo ou era do capitalismo}
  \end{Phonetics}
\end{Entry}

\begin{Entry}{近来}{7,7}{⾡、⽊}
  \begin{Phonetics}{近来}{jin4lai2}[][HSK 5]
    \definition{adv.}{ultimamente; recentemente; de ​​tarde; refere-se a um período de tempo entre o passado imediato e o presente}
  \end{Phonetics}
\end{Entry}

\begin{Entry}{近视}{7,8}{⾡、⾒}
  \begin{Phonetics}{近视}{jin4 shi4}[][HSK 6]
    \definition{adj.}{miopia; uma deficiência visual em que a visão próxima é clara, mas a visão distante é turva | míope (figurativo); metáfora para miopia}
  \end{Phonetics}
\end{Entry}

\begin{Entry}{近期}{7,12}{⾡、⽉}
  \begin{Phonetics}{近期}{jin4 qi1}[][HSK 3]
    \definition{adv.}{num futuro próximo; brevemente}
  \end{Phonetics}
\end{Entry}

\begin{Entry}{返}{7}{⾡}
  \begin{Phonetics}{返}{fan3}
    \definition{v.}{retornar; vir ou voltar}
  \end{Phonetics}
\end{Entry}

\begin{Entry}{返回}{7,6}{⾡、⼞}
  \begin{Phonetics}{返回}{fan3 hui2}[][HSK 5]
    \definition{v.}{retornar; ir (voltar); reverter; recorrer; retroceder; voltar para (o lugar original)}
  \end{Phonetics}
\end{Entry}

\begin{Entry}{返还}{7,7}{⾡、⾡}
  \begin{Phonetics}{返还}{fan3huan2}[][HSK 7-9]
    \definition{s.}{remessa | restituição | devolução de algo ao seu dono original}
  \end{Phonetics}
\end{Entry}

\begin{Entry}{还}{7}{⾡}
  \begin{Phonetics}{还}{hai2}[][HSK 1]
    \definition{adv.}{ainda; indica que a ação ou estado permanece inalterado, equivalente a 仍然 | também; além disso; em adição; indica que há um aumento ou suplemento além do escopo já indicado | ainda mais; usado com 比 para indicar que as características e o grau das coisas comparadas aumentaram, o que é equivalente a 更加 razoavelmente; medianamente; usado antes de um adjetivo, indica que algo atinge apenas o nível mínimo exigido | mesmo; usado na primeira parte da frase como complemento, e na segunda parte como conclusão, equivalente a 尚且 | que expressa realização ou descoberta; expressa surpresa por algo que não se esperava, mas que acabou acontecendo | tão cedo quanto; por um curto período de tempo; indica que já era assim há muito tempo | para dar ênfase; para reforçar o tom}
  \seealsoref{比}{bi3}
  \seealsoref{更加}{geng4 jia1}
  \seealsoref{仍然}{reng2ran2}
  \seealsoref{尚且}{shang4 qie3}
  \end{Phonetics}
  \begin{Phonetics}{还}{huan2}[][HSK 1]
    \definition*{s.}{Sobrenome Huan}
    \definition{v.}{voltar; retornar; voltar ao lugar original ou restaurar o estado original | retribuir; devolver; reembolsar; devolver o dinheiro ou os bens emprestados ao seu proprietário | dar ou fazer algo em troca; retribuir as ações dos outros}
  \end{Phonetics}
\end{Entry}

\begin{Entry}{还有}{7,6}{⾡、⽉}
  \begin{Phonetics}{还有}{hai2 you3}[][HSK 1]
    \definition{adv.}{também; ainda; além disso; então novamente; enfatizar as partes complementares, excedentes ou não mencionadas além do que já é conhecido}
  \end{Phonetics}
\end{Entry}

\begin{Entry}{还是}{7,9}{⾡、⽇}
  \begin{Phonetics}{还是}{hai2shi5}[][HSK 1]
    \definition{adv.}{ainda; ainda assim; não é a continuação de um determinado estado, fenômeno ou ação; o resultado é o mesmo de antes, sem mudanças  |que expressa uma preferência por uma alternativa; expressa comparação ou escolha feita após consideração cuidadosa, frequentemente usado para fazer sugestões | que expressa realização ou descoberta; indica que o resultado final foi inesperado}
    \definition{conj.}{ou (somente para frases interrogativas); indica várias opções, geralmente usado em perguntas | tudo; se; não importa; independentemente de; significa que, independentemente das mudanças que ocorram, o resultado permanecerá o mesmo}
  \end{Phonetics}
\end{Entry}

\begin{Entry}{还原}{7,10}{⾡、⼚}
  \begin{Phonetics}{还原}{huan2/yuan2}[][HSK 7-9]
    \definition{v.+compl.}{restaurar ao estado ou forma original | desoxidar; reduzir; refere-se à privação de oxigênio de substâncias que o contêm; também se refere, de modo geral, ao processo pelo qual uma substância ganha elétrons ou pares de elétrons em uma reação química}
  \end{Phonetics}
\end{Entry}

\begin{Entry}{还款}{7,12}{⾡、⽋}
  \begin{Phonetics}{还款}{huan2 kuan3}[][HSK 7-9]
    \definition{v.}{reembolsar | devolver dinheiro}
  \end{Phonetics}
\end{Entry}

\begin{Entry}{这}{7}{⾡}
  \begin{Phonetics}{这}{zhe4}[][HSK 1]
    \definition{pron.}{este, isto; substitui pessoas ou coisas que estão mais próximas | agora; em vez de 这时候, tem o efeito de reforçar a ênfase}
  \seealsoref{这时候}{zhe4 shi2 hou5}
  \end{Phonetics}
  \begin{Phonetics}{这}{zhei4}
    \definition{pron.}{(coloquial) este}
  \end{Phonetics}
\end{Entry}

\begin{Entry}{这儿}{7,2}{⾡、⼉}
  \begin{Phonetics}{这儿}{zhe4r5}[][HSK 1]
    \definition{pron.}{aqui | agora; neste momento (utilizado apenas após 打, 从, 由)}
  \seealsoref{从}{cong2}
  \seealsoref{打}{da3}
  \seealsoref{由}{you2}
  \end{Phonetics}
\end{Entry}

\begin{Entry}{这个}{7,3}{⾡、⼈}
  \begin{Phonetics}{这个}{zhe4ge5}
    \definition{pron.}{isto; este | isso; em vez das coisas mencionadas anteriormente | assim; tal; usado antes de verbos e adjetivos, indica um grau muito profundo, com um sentido exagerado | usado junto com 那个 para indicar pessoas ou objetos indefinidos}
  \seealsoref{那个}{na4ge5}
  \end{Phonetics}
\end{Entry}

\begin{Entry}{这么}{7,3}{⾡、⼃}
  \begin{Phonetics}{这么}{zhe4 me5}[][HSK 2]
    \definition{pron.}{tal (usado para mostrar o grau) | então (usado para mostrar exagero e exclamação) | desta forma; assim; formas de expressar ações | tal; indica quantidade}
  \end{Phonetics}
\end{Entry}

\begin{Entry}{这边}{7,5}{⾡、⾡}
  \begin{Phonetics}{这边}{zhe4 bian1}[][HSK 1]
    \definition{pron.}{aqui; deste lado; refere-se a um lugar próximo}
  \end{Phonetics}
\end{Entry}

\begin{Entry}{这会儿}{7,6,2}{⾡、⼈、⼉}
  \begin{Phonetics}{这会儿}{zhe4 hui4r5}
    \definition{adv./pron./s.}{agora; no momento; no presente}
  \end{Phonetics}
\end{Entry}

\begin{Entry}{这时}{7,7}{⾡、⽇}
  \begin{Phonetics}{这时}{zhe4 shi2}[][HSK 2]
    \definition{adv.}{neste momento}
  \end{Phonetics}
\end{Entry}

\begin{Entry}{这时候}{7,7,10}{⾡、⽇、⼈}
  \begin{Phonetics}{这时候}{zhe4 shi2 hou5}[][HSK 2]
    \definition{adv.}{neste momento}
  \end{Phonetics}
\end{Entry}

\begin{Entry}{这里}{7,7}{⾡、⾥}
  \begin{Phonetics}{这里}{zhe4 li3}[][HSK 1]
    \definition{pron.}{aqui; pronomes demonstrativo, indicando locais próximos}
  \end{Phonetics}
\end{Entry}

\begin{Entry}{这些}{7,8}{⾡、⼆}
  \begin{Phonetics}{这些}{zhe4 xie1}[][HSK 1]
    \definition{pron.}{estes; pronome demonstrativo, que indicam duas ou mais pessoas ou coisas que estão próximas}
  \end{Phonetics}
\end{Entry}

\begin{Entry}{这咱}{7,9}{⾡、⼝}
  \begin{Phonetics}{这咱}{zhe4 zan5}
    \definition{s.}{agora; no momento; no presente | neste momento}
  \end{Phonetics}
\end{Entry}

\begin{Entry}{这样}{7,10}{⾡、⽊}
  \begin{Phonetics}{这样}{zhe4 yang4}[][HSK 2]
    \definition{pron.}{assim; tal; assim; deste jeito; pronome demonstrativo, que indica a natureza, estado, maneira, grau, etc.}
  \end{Phonetics}
\end{Entry}

\begin{Entry}{这就是说}{7,12,9,9}{⾡、⼪、⽇、⾔}
  \begin{Phonetics}{这就是说}{zhe4 jiu4 shi4 shuo1}[][HSK 6]
    \definition{expr.}{isto significa que; isto é dizer}
  \end{Phonetics}
\end{Entry}

\begin{Entry}{这麽}{7,14}{⾡、⿇}
  \begin{Phonetics}{这麽}{zhe4 me5}
    \variantof{这么}
  \end{Phonetics}
\end{Entry}

\begin{Entry}{进}{7}{⾡}
  \begin{Phonetics}{进}{jin4}[][HSK 1]
    \definition*{s.}{Sobrenome Jin}
    \definition{clas.}{para seções em um edifício ou complexo residencial; qualquer uma das várias fileiras de casas em um complexo residencial de estilo antigo}
    \definition{s.}{(matemática) base de um sistema numérico}
    \definition{v.}{avançar; ir adiante; seguir em frente; (oposto a 退) | entrar; entrar em; entrar ou sair; (oposto a 出) | receber | comer; tomar; beber | submeter; apresentar | marcar um gol}
    \definition{v.aux.}{usado após um verbo, significa ``para dentro''}
  \seealsoref{出}{chu1}
  \seealsoref{退}{tui4}
  \end{Phonetics}
\end{Entry}

\begin{Entry}{进一步}{7,1,7}{⾡、⼀、⽌}
  \begin{Phonetics}{进一步}{jin4 yi2 bu4}[][HSK 3]
    \definition{adv.}{mais; dar um passo adiante; avançar um passo; indica que as coisas estão progredindo em um nível mais alto do que antes}
  \end{Phonetics}
\end{Entry}

\begin{Entry}{进入}{7,2}{⾡、⼊}
  \begin{Phonetics}{进入}{jin4 ru4}[][HSK 2]
    \definition{v.}{entrar; entrar em}
  \end{Phonetics}
\end{Entry}

\begin{Entry}{进口}{7,3}{⾡、⼝}
  \begin{Phonetics}{进口}{jin4/kou3}[][HSK 4]
    \definition{adj.}{importado}
    \definition{s.}{importação; entrada de um edifício ou local, também chamada de 入口}
    \definition{v.+compl.}{importar; comprar ou transportar mercadorias de outro país ou região | entrar no porto; navegar em direção a um porto}
  \seealsoref{入口}{ru4kou3}
  \end{Phonetics}
\end{Entry}

\begin{Entry}{进化}{7,4}{⾡、⼔}
  \begin{Phonetics}{进化}{jin4hua4}[][HSK 5]
    \definition[个]{s.}{evolução; os organismos se desenvolvem e evoluem do simples para o complexo e de níveis baixos para altos}
    \definition{v.}{evoluir; um termo geral usado para descrever uma mudança gradual para melhor}
  \end{Phonetics}
\end{Entry}

\begin{Entry}{进出口}{7,5,3}{⾡、⼐、⼝}
  \begin{Phonetics}{进出口}{jin4chu1kou3}
    \definition{s.}{importação e exportação}
    \definition{v.}{importar e exportar}
  \end{Phonetics}
\end{Entry}

\begin{Entry}{进去}{7,5}{⾡、⼛}
  \begin{Phonetics}{进去}{jin4 qu4}[][HSK 1]
    \definition{v.}{entrar (a partir da minha localização)}
    \definition{v.aux.}{usado depois de um verbo, significa ``ir para dentro''; para um determinado intervalo ou período de tempo}
  \end{Phonetics}
\end{Entry}

\begin{Entry}{进行}{7,6}{⾡、⾏}
  \begin{Phonetics}{进行}{jin4xing2}[][HSK 2]
    \definition{v.}{continuar; estar em andamento; estar em progresso | fazer; conduzir; realizar; executar | marchar; avançar; prosseguir; estar em marcha}
  \end{Phonetics}
\end{Entry}

\begin{Entry}{进行编程}{7,6,12,12}{⾡、⾏、⽷、⽲}
  \begin{Phonetics}{进行编程}{jin4xing2bian1cheng2}
    \definition{s.}{programa de computador executável}
  \end{Phonetics}
\end{Entry}

\begin{Entry}{进攻}{7,7}{⾡、⽁}
  \begin{Phonetics}{进攻}{jin4gong1}[][HSK 6]
    \definition{s.}{ofensiva}
    \definition{v.}{atacar; assaltar; tomar a ofensiva (oposto à 防守)}
  \seealsoref{防守}{fang2shou3}
  \end{Phonetics}
\end{Entry}

\begin{Entry}{进来}{7,7}{⾡、⽊}
  \begin{Phonetics}{进来}{jin4 lai2}[][HSK 1]
    \definition{v.}{entrar (para a minha localização)}
  \end{Phonetics}
\end{Entry}

\begin{Entry}{进步}{7,7}{⾡、⽌}
  \begin{Phonetics}{进步}{jin4bu4}[][HSK 3]
    \definition{adj.}{progressivo; adequado às tendências da época; que impulsiona o desenvolvimento social (em oposição a 落后)}
    \definition{v.}{avançar; progredir; melhorar}
  \seealsoref{落后}{luo4hou4}
  \end{Phonetics}
\end{Entry}

\begin{Entry}{进展}{7,10}{⾡、⼫}
  \begin{Phonetics}{进展}{jin4zhan3}[][HSK 3]
    \definition{v.}{fazer progresso; progredir; avançar no desenvolvimento}
  \end{Phonetics}
\end{Entry}

\begin{Entry}{远}{7}{⾡}
  \begin{Phonetics}{远}{yuan3}[][HSK 1]
    \definition*{s.}{Sobrenome Yuan}
    \definition{adj.}{distante (no tempo ou no espaço); longe; remoto; Longa distância espacial ou temporal (em oposição a 近) | (relações de parentesco) distante | com grande diferença}
    \definition{v.}{manter-se afastado de; não se aproximar}
  \seealsoref{近}{jin4}
  \end{Phonetics}
\end{Entry}

\begin{Entry}{远天}{7,4}{⾡、⼤}
  \begin{Phonetics}{远天}{yuan3tian1}
    \definition{s.}{paraíso | o céu distante}
  \end{Phonetics}
\end{Entry}

\begin{Entry}{远方}{7,4}{⾡、⽅}
  \begin{Phonetics}{远方}{yuan3 fang1}[][HSK 6]
    \definition{s.}{distância; de longe; lugar distante}
  \end{Phonetics}
\end{Entry}

\begin{Entry}{远处}{7,5}{⾡、⼡}
  \begin{Phonetics}{远处}{yuan3 chu4}[][HSK 5]
    \definition{s.}{distância; lugar distante}
  \end{Phonetics}
\end{Entry}

\begin{Entry}{远远}{7,7}{⾡、⾡}
  \begin{Phonetics}{远远}{yuan3 yuan3}[][HSK 6]
    \definition{adv.}{de longe; em grande medida; para descrever um alto grau ou uma grande quantidade}
  \end{Phonetics}
\end{Entry}

\begin{Entry}{远征}{7,8}{⾡、⼻}
  \begin{Phonetics}{远征}{yuan3zheng1}
    \definition{s.}{uma expedição militar | marcha para regiões remotas}
  \end{Phonetics}
\end{Entry}

\begin{Entry}{远离}{7,10}{⾡、⼇}
  \begin{Phonetics}{远离}{yuan3 li2}[][HSK 6]
    \definition{adj.}{afastado; distante}
    \definition{v.}{partir para; ficar longe}
  \end{Phonetics}
\end{Entry}

\begin{Entry}{违}{7}{⾡}
  \begin{Phonetics}{违}{wei2}
    \definition{v.}{desobedecer; violar | ser separado; separar-se de | desafiar; não cumprir; não obedecer}
  \end{Phonetics}
\end{Entry}

\begin{Entry}{违反}{7,4}{⾡、⼜}
  \begin{Phonetics}{违反}{wei2fan3}[][HSK 5]
    \definition{v.}{violar; transgredir; contrariar; não estar em conformidade (com as regras, regulamentos, etc.)}
  \end{Phonetics}
\end{Entry}

\begin{Entry}{违法}{7,8}{⾡、⽔}
  \begin{Phonetics}{违法}{wei2 fa3}[][HSK 5]
    \definition{v.}{ser ilegal; infringir a lei; violar a lei ou os regulamentos}
  \end{Phonetics}
\end{Entry}

\begin{Entry}{违规}{7,8}{⾡、⾒}
  \begin{Phonetics}{违规}{wei2 gui1}[][HSK 5]
    \definition{v.}{violar (regras); infringir as regras e regulamentos}
  \end{Phonetics}
\end{Entry}

\begin{Entry}{违宪}{7,9}{⾡、⼧}
  \begin{Phonetics}{违宪}{wei2xian4}
    \definition{adj.}{inconstitucional}
  \end{Phonetics}
\end{Entry}

\begin{Entry}{连}{7}{⾡}
  \begin{Phonetics}{连}{lian2}[][HSK 3]
    \definition*{s.}{Sobrenome Lian}
    \definition{adv.}{em sucessão; um após o outro; repetidamente}
    \definition{prep.}{incluindo; incluido | até mesmo}
    \definition[个]{s.}{companhia; unidades organizacionais das forças armadas}
    \definition{v.}{ligar; juntar; conectar | envolver-se (em problemas); implicar; incriminar | costurar; coser}
  \end{Phonetics}
\end{Entry}

\begin{Entry}{连忙}{7,6}{⾡、⼼}
  \begin{Phonetics}{连忙}{lian2mang2}[][HSK 3]
    \definition{adv.}{imediatamente; de imediato; com pressa; apressadamente}
  \end{Phonetics}
\end{Entry}

\begin{Entry}{连接}{7,11}{⾡、⼿}
  \begin{Phonetics}{连接}{lian2 jie1}[][HSK 5]
    \definition[条]{s.}{conexão}
    \definition{v.}{ligar; unir; relacionar, conectar; anexar}
  \end{Phonetics}
\end{Entry}

\begin{Entry}{连续}{7,11}{⾡、⽷}
  \begin{Phonetics}{连续}{lian2xu4}[][HSK 3]
    \definition{adv.}{continuamente; sucessivamente; em uma fileira; um após o outro}
  \end{Phonetics}
\end{Entry}

\begin{Entry}{连续剧}{7,11,10}{⾡、⽷、⼑}
  \begin{Phonetics}{连续剧}{lian2 xu4 ju4}[][HSK 3]
    \definition[部,集]{s.}{série; novela; drama dividido em vários episódios, transmitido continuamente pela rádio ou televisão, com enredo contínuo}
  \end{Phonetics}
\end{Entry}

\begin{Entry}{连锁反应}{7,12,4,7}{⾡、⾦、⼜、⼴}
  \begin{Phonetics}{连锁反应}{lian2suo3fan3ying4}
    \definition{s.}{reação em cadeia}
  \end{Phonetics}
\end{Entry}

\begin{Entry}{迟}{7}{⾡}
  \begin{Phonetics}{迟}{chi2}[][HSK 5]
    \definition*{s.}{Sobrenome Chi}
    \definition{adj.}{lento; tardio; demorado | atrasado | lento; obtuso}
  \end{Phonetics}
\end{Entry}

\begin{Entry}{迟早}{7,6}{⾡、⽇}
  \begin{Phonetics}{迟早}{chi2zao3}[][HSK 7-9]
    \definition{adv.}{mais cedo ou mais tarde; cedo ou tarde}[我们迟早会成功的。===Teremos sucesso mais cedo ou mais tarde.]
  \end{Phonetics}
\end{Entry}

\begin{Entry}{迟迟}{7,7}{⾡、⾡}
  \begin{Phonetics}{迟迟}{chi2chi2}[][HSK 7-9]
    \definition{adv.}{lentamente; tardiamente}
  \end{Phonetics}
\end{Entry}

\begin{Entry}{迟到}{7,8}{⾡、⼑}
  \begin{Phonetics}{迟到}{chi2dao4}[][HSK 4]
    \definition{v.}{chegar atrasado; atrasar-se; chegar depois do horário estipulado, geralmente usado para aulas, reuniões ou encontros em horário combinado, etc.}
  \end{Phonetics}
\end{Entry}

\begin{Entry}{迟疑}{7,14}{⾡、⽦}
  \begin{Phonetics}{迟疑}{chi2yi2}[][HSK 7-9]
    \definition{v.}{hesitar; titubear}
  \end{Phonetics}
\end{Entry}

\begin{Entry}{迫}{8}{⾡}
  \begin{Phonetics}{迫}{po4}
    \definition{adj.}{urgente; premente}
    \definition{s.}{morteiro; artilharia}
    \definition{v.}{compelir; forçar; pressionar | aproximar-se; ir em direção a (ou perto de)}
  \end{Phonetics}
\end{Entry}

\begin{Entry}{迫切}{8,4}{⾡、⼑}
  \begin{Phonetics}{迫切}{po4qie4}[][HSK 4]
    \definition{adj.}{urgente; premente; muito ansiosamente, a ponto de ser difícil esperar}
  \end{Phonetics}
\end{Entry}

\begin{Entry}{迭}{8}{⾡}
  \begin{Phonetics}{迭}{die2}
    \definition*{s.}{Sobrenome Die}
    \definition{adv.}{repetidamente; de ​​novo e de novo | a tempo para}
    \definition{v.}{alternar; mudar; revezar-se; substituir}
  \end{Phonetics}
\end{Entry}

\begin{Entry}{迭起}{8,10}{⾡、⾛}
  \begin{Phonetics}{迭起}{die2qi3}[][HSK 7-9]
    \definition{v.}{ocorrer repetidamente; acontecer com frequência | surgir repetidamente}
  \end{Phonetics}
\end{Entry}

\begin{Entry}{迷}{9}{⾡}
  \begin{Phonetics}{迷}{mi2}[][HSK 3]
    \definition[个]{s.}{fã; entusiasta; aficionado; pessoa que gosta excessivamente de algo}
    \definition{v.}{estar confuso; perder o rumo; se perder-se; perda da capacidade de discernimento e julgamento | ficar fascinado por; entregar-se a; ficar encantado com (por); ser louco por | confundir; desorientar; fascinar; encantar; tornar indistinto; deixar encantado e fascinado}
  \end{Phonetics}
\end{Entry}

\begin{Entry}{迷人}{9,2}{⾡、⼈}
  \begin{Phonetics}{迷人}{mi2ren2}[][HSK 5]
    \definition{adj.}{encantador; fascinante; sedutor; hipnotizante}
    \definition{v.}{confundir; intrigar; enganar}
  \end{Phonetics}
\end{Entry}

\begin{Entry}{迷你}{9,7}{⾡、⼈}
  \begin{Phonetics}{迷你}{mi2ni3}
    \definition{adj.}{(empréstimo linguístico) mini, como em minissaia ou \emph{Mini Cooper}}
  \end{Phonetics}
\end{Entry}

\begin{Entry}{迷信}{9,9}{⾡、⼈}
  \begin{Phonetics}{迷信}{mi2xin4}[][HSK 5]
    \definition{s.}{superstição; crença supersticiosa | fé cega; adoração cega}
    \definition{v.}{ter fé cega em; ter um fetiche de}
  \end{Phonetics}
\end{Entry}

\begin{Entry}{迷宫}{9,9}{⾡、⼧}
  \begin{Phonetics}{迷宫}{mi2gong1}
    \definition{s.}{labirinto}
  \end{Phonetics}
\end{Entry}

\begin{Entry}{迷恋}{9,10}{⾡、⼼}
  \begin{Phonetics}{迷恋}{mi2lian4}
    \definition{adj.}{obcecado}
    \definition{v.}{estar/ser apaixonado por | ficar encantado por | estar/ser obcecado por}
  \end{Phonetics}
\end{Entry}

\begin{Entry}{迷路}{9,13}{⾡、⾜}
  \begin{Phonetics}{迷路}{mi2/lu4}
    \definition{s.}{labirinto | ouvido interno}
    \definition{v.+compl.}{perder o caminho | perder-se | seguir pelo caminho errado | não conseguir encontrar o caminho}
  \end{Phonetics}
\end{Entry}

\begin{Entry}{迹}{9}{⾡}
  \begin{Phonetics}{迹}{ji4}
    \definition[点,丝]{s.}{marca; traço; marca deixada para trás | restos; ruínas; vestígio; coisas deixadas por gerações anteriores (principalmente edifícios) | evento importante do passado; coisas feitas; feitos | aparência; ação; figura (escrita)
vestígios}
  \end{Phonetics}
\end{Entry}

\begin{Entry}{迹象}{9,11}{⾡、⾗}
  \begin{Phonetics}{迹象}{ji4xiang4}[][HSK 7-9]
    \definition[种]{s.}{sinal; símbolo; indicação; refere-se a vestígios e fenômenos que podem ser usados ​​para inferir o passado ou o futuro das coisas}
  \end{Phonetics}
\end{Entry}

\begin{Entry}{追}{9}{⾡}
  \begin{Phonetics}{追}{zhui1}[][HSK 3]
    \definition{v.}{perseguir; correr atrás; seguir de perto | rastrear; investigar; chegar ao fundo de | procurar; ir atrás; esforçar-se para alcançar um determinado objetivo | recordar; relembrar | fazer depois do ocorrido; retrabalhar | cortejar (uma mulher)}
  \end{Phonetics}
\end{Entry}

\begin{Entry}{追求}{9,7}{⾡、⽔}
  \begin{Phonetics}{追求}{zhui1qiu2}[][HSK 4]
    \definition{s.}{perseguição (ações e metas positivas)}[她的追求是获得成功。===Sua meta é alcançar o sucesso.]
    \definition{v.}{procurar; aspirar; perseguir | cortejar; refere-se especificamente ao namoro}
  \end{Phonetics}
\end{Entry}

\begin{Entry}{追究}{9,7}{⾡、⽳}
  \begin{Phonetics}{追究}{zhui1jiu1}[][HSK 6]
    \definition{v.}{descobrir; investigar}
  \end{Phonetics}
\end{Entry}

\begin{Entry}{追赶}{9,10}{⾡、⾛}
  \begin{Phonetics}{追赶}{zhui1gan3}
    \definition{v.}{perseguir | acelerar | alcançar | ultrapassar}
  \end{Phonetics}
\end{Entry}

\begin{Entry}{退}{9}{⾡}
  \begin{Phonetics}{退}{tui4}[][HSK 3]
    \definition{v.}{recuar; mover-se para trás  (oposto de 進) | remover; retirar; fazer recuar; mover para trás | desistir; retirar-se de | refluir; declinar; retroceder | aposentar-se; deixar o emprego por atingir a idade estipulada ou por problemas de saúde | retornar; reembolsar; devolver | romper; cancelar o que foi decidido}
  \seealsoref{进}{jin4}
  \end{Phonetics}
\end{Entry}

\begin{Entry}{退出}{9,5}{⾡、⼐}
  \begin{Phonetics}{退出}{tui4 chu1}[][HSK 3]
    \definition{v.}{desistir; retirar-se; separar-se; retirar-se de; abandonar o local ou outro lugar e parar de participar; abandonaar o grupo ou organização}
  \end{Phonetics}
\end{Entry}

\begin{Entry}{退休}{9,6}{⾡、⼈}
  \begin{Phonetics}{退休}{tui4/xiu1}[][HSK 3]
    \definition{v.+compl.}{aposentar-se; os trabalhadores que deixarem o emprego por velhice ou invalidez causada pelo trabalho receberão as despesas de subsistência conforme o cronograma}
  \end{Phonetics}
\end{Entry}

\begin{Entry}{退票}{9,11}{⾡、⽰}
  \begin{Phonetics}{退票}{tui4 piao4}[][HSK 6]
    \definition{s.}{bilhete devolvido (ou não utilizado) | reembolso do bilhete}
    \definition{v.}{devolver um bilhete; obter um reembolso por um bilhete | devolver (um cheque)}
  \end{Phonetics}
\end{Entry}

\begin{Entry}{送}{9}{⾡}
  \begin{Phonetics}{送}{song4}[][HSK 1]
    \definition*{s.}{Sobrenome Song}
    \definition{v.}{transportar; entregar | dar; dar como presente; presentear | acompanhar; despedir-se de alguém (ao sair); acompanhar a pessoa que está partindo até o destino ou caminhar um trecho com ela | escoltar}
  \end{Phonetics}
\end{Entry}

\begin{Entry}{送礼}{9,5}{⾡、⽰}
  \begin{Phonetics}{送礼}{song4 li3}[][HSK 6]
    \definition{v.}{dar um presente a alguém; presentear alguém com um presente | enviar presentes (para obter favores) | dar um presente; enviar um presente}
  \end{Phonetics}
\end{Entry}

\begin{Entry}{送行}{9,6}{⾡、⾏}
  \begin{Phonetics}{送行}{song4 xing2}[][HSK 6]
    \definition{v.}{ver alguém partir; ir até o local onde o viajante iniciou sua jornada, despedir-se dele e observar ele partir | dar uma festa de despedida; realizar uma festa de despedida | despedir-se do falecido}
  \end{Phonetics}
\end{Entry}

\begin{Entry}{送到}{9,8}{⾡、⼑}
  \begin{Phonetics}{送到}{song4 dao4}[][HSK 2]
    \definition{v.}{enviar para (lugar)}
  \end{Phonetics}
\end{Entry}

\begin{Entry}{送给}{9,9}{⾡、⽷}
  \begin{Phonetics}{送给}{song4 gei3}[][HSK 2]
    \definition{v.}{dar a (alguém ou organização); dar como algo gratuito; dar como presente}
  \end{Phonetics}
\end{Entry}

\begin{Entry}{适}{9}{⾡}
  \begin{Phonetics}{适}{shi4}
    \definition*{s.}{Sobrenome Shi}
    \definition{adj.}{confortável; bem | adequado; apropriado | certo; oportuno}
    \definition{v.}{ser apto; ser adequado; ser apropriado | ir; seguir; perseguir | (de uma mulher) casar}
  \end{Phonetics}
\end{Entry}

\begin{Entry}{适用}{9,5}{⾡、⽤}
  \begin{Phonetics}{适用}{shi4 yong4}[][HSK 3]
    \definition{adj.}{adequado; aplicável}
  \end{Phonetics}
\end{Entry}

\begin{Entry}{适合}{9,6}{⾡、⼝}
  \begin{Phonetics}{适合}{shi4he2}[][HSK 3]
    \definition{v.}{servir; caber; se adequar; atender às necessidades de uma determinada situação ou pessoa}
  \end{Phonetics}
\end{Entry}

\begin{Entry}{适当}{9,6}{⾡、⼹}
  \begin{Phonetics}{适当}{shi4 dang4}[][HSK 6]
    \definition{s.}{adequado; apropriado}
  \end{Phonetics}
\end{Entry}

\begin{Entry}{适应}{9,7}{⾡、⼴}
  \begin{Phonetics}{适应}{shi4ying4}[][HSK 3]
    \definition{v.}{ajustar-se; adequar-se; adaptar-se; fazer as alterações correspondentes para se adequar à medida que as condições mudam}
  \end{Phonetics}
\end{Entry}

\begin{Entry}{逃}{9}{⾡}
  \begin{Phonetics}{逃}{tao2}[][HSK 5]
    \definition{v.}{fugir; escapar; correr; dar no pé | evadir; esquivar-se; escapar}
  \end{Phonetics}
\end{Entry}

\begin{Entry}{逃走}{9,7}{⾡、⾛}
  \begin{Phonetics}{逃走}{tao2 zou3}[][HSK 5]
    \definition{v.}{escapar; afastar-se de pessoas, coisas ou lugares que não são bons para você ou que você não gosta}
  \end{Phonetics}
\end{Entry}

\begin{Entry}{逃跑}{9,12}{⾡、⾜}
  \begin{Phonetics}{逃跑}{tao2 pao3}[][HSK 5]
    \definition{v.}{fugir; escapar; correr; partir para fugir de um ambiente ou de coisas que não lhe são favoráveis}
  \end{Phonetics}
\end{Entry}

\begin{Entry}{逆}{9}{⾡}
  \begin{Phonetics}{逆}{ni4}
    \definition{adj.}{contrário (oposto a 顺) ; contra; oposto; inverso | traidor; rebelde}
    \definition{adv.}{antecipadamente; com antecedência}
    \definition{s.}{traidor; rebelde}
    \definition{v.}{ir contra; opor-se; desobedecer; resistir; desafiar (oposto a 顺) | (literário) saudar; cumprimentar}
  \seealsoref{顺}{shun4}
  \end{Phonetics}
\end{Entry}

\begin{Entry}{逆境}{9,14}{⾡、⼟}
  \begin{Phonetics}{逆境}{ni4jing4}
    \definition{s.}{adversidade | tribulação}
  \end{Phonetics}
\end{Entry}

\begin{Entry}{选}{9}{⾡}
  \begin{Phonetics}{选}{xuan3}[][HSK 2]
    \definition{s.}{pessoa ou coisa selecionada | seleções; antologia; trabalhos selecionados e compilados}
    \definition{v.}{selecionar; escolher | eleger}
  \end{Phonetics}
\end{Entry}

\begin{Entry}{选手}{9,4}{⾡、⼿}
  \begin{Phonetics}{选手}{xuan3shou3}[][HSK 3]
    \definition[位,名,个,些]{s.}{jogador; (selecionado) competidor; atleta selecionado para uma competição esportiva; participantes selecionados entre um grande número de candidatos}
  \end{Phonetics}
\end{Entry}

\begin{Entry}{选拔}{9,8}{⾡、⼿}
  \begin{Phonetics}{选拔}{xuan3ba2}[][HSK 6]
    \definition{v.}{selecionar; escolher}
  \end{Phonetics}
\end{Entry}

\begin{Entry}{选择}{9,8}{⾡、⼿}
  \begin{Phonetics}{选择}{xuan3ze2}[][HSK 4]
    \definition[个,种,次]{s.}{escolha; opção; resultado da escolha; possibilidade de escolha}
    \definition{v.}{selecionar; escolher}
  \end{Phonetics}
\end{Entry}

\begin{Entry}{选举}{9,9}{⾡、⼂}
  \begin{Phonetics}{选举}{xuan3ju3}[][HSK 6]
    \definition[次,个]{s.}{eleição; as eleições são o processo pelo qual os cidadãos escolhem os seus representantes ou líderes através do voto}
    \definition{v.}{votar; eleger; eleger representantes ou responsáveis ​​votando ou levantando as mãos}
  \end{Phonetics}
\end{Entry}

\begin{Entry}{选修}{9,9}{⾡、⼈}
  \begin{Phonetics}{选修}{xuan3 xiu1}[][HSK 5]
    \definition{v.}{fazer como disciplina eletiva; escolher entre uma seleção de cursos disponíveis}
  \end{Phonetics}
\end{Entry}

\begin{Entry}{透}{10}{⾡}
  \begin{Phonetics}{透}{tou4}[][HSK 4]
    \definition{adv.}{totalmente; completamente; minuciosamente | profundamente; extremamente}
    \definition{v.}{penetrar; passar através de; infiltrar-se através de | revelar; deixar transparecer; contar secretamente |mostrar; aparecer}
  \end{Phonetics}
\end{Entry}

\begin{Entry}{透支}{10,4}{⾡、⽀}
  \begin{Phonetics}{透支}{tou4zhi1}
    \definition{v.}{cheque especial (bancário) | saque a descoberto}
  \end{Phonetics}
\end{Entry}

\begin{Entry}{透气}{10,4}{⾡、⽓}
  \begin{Phonetics}{透气}{tou4qi4}
    \definition{v.}{respirar (sobre tecido, etc.) | fluir livremente (sobre ar) | respirar ar fresco | ventilar}
  \end{Phonetics}
\end{Entry}

\begin{Entry}{透水}{10,4}{⾡、⽔}
  \begin{Phonetics}{透水}{tou4shui3}
    \definition{adj.}{permeável}
    \definition{s.}{vazamento de água}
  \end{Phonetics}
\end{Entry}

\begin{Entry}{透过}{10,6}{⾡、⾡}
  \begin{Phonetics}{透过}{tou4guo4}
    \definition{v.}{passar através | penetrar}
  \end{Phonetics}
\end{Entry}

\begin{Entry}{透彻}{10,7}{⾡、⼻}
  \begin{Phonetics}{透彻}{tou4che4}
    \definition{adj.}{minucioso | incisivo | penetrante}
  \end{Phonetics}
\end{Entry}

\begin{Entry}{透明}{10,8}{⾡、⽇}
  \begin{Phonetics}{透明}{tou4ming2}[][HSK 4]
    \definition{adj.}{transparente; diáfano; capaz de transmitir luz | evidente; transparente; situação ou assunto que seja aberto e não oculto | transparente; diáfano; indica pureza, ausência de impurezas}
  \end{Phonetics}
\end{Entry}

\begin{Entry}{透顶}{10,8}{⾡、⾴}
  \begin{Phonetics}{透顶}{tou4ding3}
    \definition{adv.}{completamente}
  \end{Phonetics}
\end{Entry}

\begin{Entry}{透亮}{10,9}{⾡、⼇}
  \begin{Phonetics}{透亮}{tou4liang4}
    \definition{adj.}{brilhante | claro como cristal}
  \end{Phonetics}
\end{Entry}

\begin{Entry}{透辟}{10,13}{⾡、⾟}
  \begin{Phonetics}{透辟}{tou4pi4}
    \definition{adj.}{incisivo | penetrante}
  \end{Phonetics}
\end{Entry}

\begin{Entry}{透澈}{10,15}{⾡、⽔}
  \begin{Phonetics}{透澈}{tou4che4}
    \variantof{透彻}
  \end{Phonetics}
\end{Entry}

\begin{Entry}{透露}{10,21}{⾡、⾬}
  \begin{Phonetics}{透露}{tou4lu4}[][HSK 6]
    \definition{v.}{vazar; revelar; expor; divulgar; contar deliberadamente um segredo a alguém; revelar um certo significado}
  \end{Phonetics}
\end{Entry}

\begin{Entry}{逐}{10}{⾡}
  \begin{Phonetics}{逐}{zhu2}
    \definition{prep.}{um por um; um a um}[逐月===mês a mês]
    \definition{v.}{ir atrás de; perseguir | expulsar; banir | correr atrás; alcançar}
  \end{Phonetics}
\end{Entry}

\begin{Entry}{逐步}{10,7}{⾡、⽌}
  \begin{Phonetics}{逐步}{zhu2bu4}[][HSK 4]
    \definition{adv.}{gradualmente; passo a passo; progressivamente}
  \end{Phonetics}
\end{Entry}

\begin{Entry}{逐渐}{10,11}{⾡、⽔}
  \begin{Phonetics}{逐渐}{zhu2jian4}[][HSK 4]
    \definition{adv.}{gradualmente; aos poucos; por etapas; indica mudanças lentas e ordenadas no grau, na quantidade, etc.}
  \end{Phonetics}
\end{Entry}

\begin{Entry}{递}{10}{⾡}
  \begin{Phonetics}{递}{di4}[][HSK 5]
    \definition{adv.}{na ordem correta; sucessivamente}
    \definition{v.}{entregar; passar; dar; transmitir}
  \end{Phonetics}
\end{Entry}

\begin{Entry}{递交}{10,6}{⾡、⼇}
  \begin{Phonetics}{递交}{di4jiao1}[][HSK 7-9]
    \definition{v.}{apresentar; submeter; entregar; entregar pessoalmente}
  \end{Phonetics}
\end{Entry}

\begin{Entry}{递给}{10,9}{⾡、⽷}
  \begin{Phonetics}{递给}{di4 gei3}[][HSK 5]
    \definition{v.}{entregar algo a alguém; passar itens ou coisas para outras pessoas}
  \end{Phonetics}
\end{Entry}

\begin{Entry}{途}{10}{⾡}
  \begin{Phonetics}{途}{tu2}
    \definition[条]{s.}{caminho; estrada; rota | jornada; caminho}
  \end{Phonetics}
\end{Entry}

\begin{Entry}{途中}{10,4}{⾡、⼁}
  \begin{Phonetics}{途中}{tu2 zhong1}[][HSK 4]
    \definition[家]{adv.}{no caminho; ao longo do caminho}
  \end{Phonetics}
\end{Entry}

\begin{Entry}{途径}{10,8}{⾡、⼻}
  \begin{Phonetics}{途径}{tu2jing4}[][HSK 6]
    \definition[种,条,个]{s.}{caminho; canal; metaforicamente falando, uma maneira ou método de resolver um problema ou fazer algo}
  \end{Phonetics}
\end{Entry}

\begin{Entry}{逗}{10}{⾡}
  \begin{Phonetics}{逗}{dou4}[][HSK 7-9]
    \definition{adj.}{engraçado; divertido}
    \definition{s.}{ligeira pausa na leitura; antigamente, referia-se ao lugar em um artigo onde o significado de uma frase não era completado e uma pausa era necessária durante a leitura}
    \definition{v.}{provocar; brincar com | divertir; provocar (risos, etc.) | ficar; parar}
  \end{Phonetics}
\end{Entry}

\begin{Entry}{通}{10}{⾡}
  \begin{Phonetics}{通}{tong1}[][HSK 2]
    \definition*{s.}{Sobrenome Tong}
    \definition{adj.}{lógico; coerente | geral; comum | tudo; inteiro | aberto; através de | total}
    \definition{clas.}{(antigo) usado para cartas, telegramas, documentos oficiais, etc.}
    \definition{s.}{autoridade; especialista}
    \definition{suf.}{especialista}
    \definition{v.}{abrir; atravessar | abrir ou limpar cutucando ou espetando | levar a; ir a | conectar; comunicar | notificar; informar | compreender; saber | cutucar; dar uma pancada | transmitir; conectar; interagir | dominar; compreender; entender}
  \end{Phonetics}
  \begin{Phonetics}{通}{tong4}
    \definition{clas.}{usado para uma atividade, tomada em sua totalidade (discurso de abuso, período de reprodução de música, bebedeira, etc.)}
  \end{Phonetics}
\end{Entry}

\begin{Entry}{通用}{10,5}{⾡、⽤}
  \begin{Phonetics}{通用}{tong1yong4}[][HSK 5]
    \definition[家]{adj.}{de uso comum; universal; (em um determinado âmbito) de uso generalizado | intercambiável; alguns caracteres chineses com grafia diferente, mas pronúncia igual, podem ser usados indistintamente (alguns limitados a um determinado significado)}
  \end{Phonetics}
\end{Entry}

\begin{Entry}{通讯}{10,5}{⾡、⾔}
  \begin{Phonetics}{通讯}{tong1xun4}[][HSK 6]
    \definition[个,种]{s.}{relatório; comunicação; boletim informativo; correspondência; reportagem; despacho de notícias; artigos que relatam fatos objetivos ou números típicos de forma detalhada e vívida}
    \definition{v.}{usar equipamentos de telecomunicações para transmitir mensagens}
  \end{Phonetics}
\end{Entry}

\begin{Entry}{通红}{10,6}{⾡、⽷}
  \begin{Phonetics}{通红}{tong1 hong2}[][HSK 6]
    \definition{adj.}{muito vermelho; vermelho por completo}
  \end{Phonetics}
\end{Entry}

\begin{Entry}{通行}{10,6}{⾡、⾏}
  \begin{Phonetics}{通行}{tong1 xing2}[][HSK 6]
    \definition{adj.}{atual; geral}
    \definition{v.}{passar (ou ir) através; passar por; atravessar | prevalecer; predominar; ser corrente | (pedestres, veículos, etc.) passar na linha de trânsito}
  \end{Phonetics}
\end{Entry}

\begin{Entry}{通观}{10,6}{⾡、⾒}
  \begin{Phonetics}{通观}{tong1guan1}
    \definition{v.}{ter uma visão geral de algo}
  \end{Phonetics}
\end{Entry}

\begin{Entry}{通过}{10,6}{⾡、⾡}
  \begin{Phonetics}{通过}{tong1guo4}[][HSK 2]
    \definition{prep.}{por; através de; por meio de; por meio de; meios, métodos, etc. para introduzir ações}
    \definition{v.}{atravessar; passar por; transitar | aprovar; adotar | solicitar o consentimento ou aprovação de}
  \end{Phonetics}
\end{Entry}

\begin{Entry}{通报}{10,7}{⾡、⼿}
  \begin{Phonetics}{通报}{tong1 bao4}[][HSK 6]
    \definition[份]{s.}{circular | boletim; jornal; publicação | sumário; notificação para informações gerais}
    \definition{v.}{circular um aviso (aviso por escrito) | notificar; dar informações com; compartilhar informações com}
  \end{Phonetics}
\end{Entry}

\begin{Entry}{通识}{10,7}{⾡、⾔}
  \begin{Phonetics}{通识}{tong1shi2}
    \definition{s.}{conhecimento comum | erudição | conhecimento geral | amplamente conhecido}
  \end{Phonetics}
\end{Entry}

\begin{Entry}{通知}{10,8}{⾡、⽮}
  \begin{Phonetics}{通知}{tong1zhi1}[][HSK 2]
    \definition[份,个,张]{s.}{aviso; circular; notificação por escrito ou verbal}
    \definition{v.}{aconselhar; notificar; informar; dar aviso prévio}
  \end{Phonetics}
\end{Entry}

\begin{Entry}{通知书}{10,8,4}{⾡、⽮、⼄}
  \begin{Phonetics}{通知书}{tong1 zhi1 shu1}[][HSK 4]
    \definition[份]{s.}{aviso; observação; notificação}
  \end{Phonetics}
\end{Entry}

\begin{Entry}{通话}{10,8}{⾡、⾔}
  \begin{Phonetics}{通话}{tong1 hua4}[][HSK 6]
    \definition{v.}{comunicar por telefone | conversar; comunicar; falar em uma língua que ambos possam entender}
  \end{Phonetics}
\end{Entry}

\begin{Entry}{通信}{10,9}{⾡、⼈}
  \begin{Phonetics}{通信}{tong1/xin4}[][HSK 3]
    \definition{v.+compl.}{corresponder; comunicar por carta; comunicar situações e informações escrevendo cartas | transmitir (ou transportar) mensagem; passar (ou transmitir) informação; usar ondas de rádio e outros sinais para transmitir texto, imagens, etc.}
  \end{Phonetics}
\end{Entry}

\begin{Entry}{通常}{10,11}{⾡、⼱}
  \begin{Phonetics}{通常}{tong1chang2}[][HSK 3]
    \definition{adj.}{usual; normal; geral}
    \definition{adv.}{habitualmente; usualmente; geralmente; ordinariamente}
  \end{Phonetics}
\end{Entry}

\begin{Entry}{通道}{10,12}{⾡、⾡}
  \begin{Phonetics}{通道}{tong1 dao4}[][HSK 6]
    \definition[条,个]{s.}{acesso; corredor; passagem; caminhos que levam ao exterior de teatros, minas, etc. | passagem; via pública}
  \end{Phonetics}
\end{Entry}

\begin{Entry}{通牒}{10,13}{⾡、⽚}
  \begin{Phonetics}{通牒}{tong1die2}
    \definition{s.}{nota diplomática}
  \end{Phonetics}
\end{Entry}

\begin{Entry}{逛}{10}{⾡}
  \begin{Phonetics}{逛}{guang4}[][HSK 4]
    \definition{v.}{perambular; passear; vaguear}
  \end{Phonetics}
\end{Entry}

\begin{Entry}{逞}{10}{⾡}
  \begin{Phonetics}{逞}{cheng3}
    \definition*{s.}{Sobrenome Cheng}
    \definition{v.}{exibir-se; ostentar; gabar-se | executar (um plano maligno); ter sucesso (em um esquema) | saciar; satisfazer; dar rédea solta a; deliciar-se}
  \end{Phonetics}
\end{Entry}

\begin{Entry}{逞能}{10,10}{⾡、⾁}
  \begin{Phonetics}{逞能}{cheng3/neng2}[][HSK 7-9]
    \definition{v.+compl.}{exibir a própria habilidade (ou capacidade); exibir a própria capacidade | mostrar sua habilidade ou capacidade}
  \end{Phonetics}
\end{Entry}

\begin{Entry}{逞强}{10,12}{⾡、⼸}
  \begin{Phonetics}{逞强}{cheng3/qiang2}[][HSK 7-9]
    \definition{v.+compl.}{exibir-se; ser orgulhoso; ser teimoso; ostentar a própria superioridade}
  \end{Phonetics}
\end{Entry}

\begin{Entry}{速}{10}{⾡}
  \begin{Phonetics}{速}{su4}
    \definition{adj.}{rápido; veloz}
    \definition{s.}{velocidade}
    \definition{v.aux.}{convidar}
  \end{Phonetics}
\end{Entry}

\begin{Entry}{速度}{10,9}{⾡、⼴}
  \begin{Phonetics}{速度}{su4du4}[][HSK 3]
    \definition[个,种]{s.}{velocidade; taxa; ritmo; andamento; uma quantidade física que indica a velocidade e a direção do movimento de um objeto, ou seja, a distância que um objeto percorre em uma direção por unidade de tempo | velocidade; rapidez; geralmente se refere ao grau de velocidade}
  \end{Phonetics}
\end{Entry}

\begin{Entry}{造}{10}{⾡}
  \begin{Phonetics}{造}{zao4}[][HSK 3]
    \definition*{s.}{Sobrenome Zao}
    \definition{clas.}{para colheitas ou número de colheitas de safras}
    \definition{s.}{uma das duas partes em um acordo legal ou um processo judicial | (dialeto) colheita; safra | realizações; conquistas |}
    \definition{v.}{fazer; construir; criar; produzir | forjar; inventar | correr solto; bagunçar as coisas | expor sem restrições |  treinar; educar | fabricar | alcançar; atingir}
  \end{Phonetics}
\end{Entry}

\begin{Entry}{造成}{10,6}{⾡、⼽}
  \begin{Phonetics}{造成}{zao4cheng2}[][HSK 3]
    \definition{v.}{criar; dar origem a; provocar; causar (geralmente se refere a resultados negativos)}
  \end{Phonetics}
\end{Entry}

\begin{Entry}{造型}{10,9}{⾡、⼟}
  \begin{Phonetics}{造型}{zao4xing2}[][HSK 4]
    \definition[个,种]{s.}{molde; modelo; formato; forma; moldagem}
    \definition{v.}{modelar; moldar}
  \end{Phonetics}
\end{Entry}

\begin{Entry}{逢}{10}{⾡}
  \begin{Phonetics}{逢}{feng2}[][HSK 7-9]
    \definition*{s.}{Sobrenome Feng}
    \definition{v.}{encontrar; vir até; encontrar-se por acaso}
  \end{Phonetics}
\end{Entry}

\begin{Entry}{逮}{11}{⾡}
  \begin{Phonetics}{逮}{dai3}[][HSK 7-9]
    \definition{v.}{capturar; pegar}[猫逮老鼠。===Gatos pegam ratos.]
  \end{Phonetics}
  \begin{Phonetics}{逮}{dai4}
    \definition*{s.}{Sobrenome Dai}
    \definition{v.}{alcançar | prender, usado em 逮捕}
  \seealsoref{逮捕}{dai4bu3}
  \end{Phonetics}
\end{Entry}

\begin{Entry}{逮捕}{11,10}{⾡、⼿}
  \begin{Phonetics}{逮捕}{dai4bu3}[][HSK 7-9]
    \definition{v.}{prender; apreender; levar sob custódia}
  \end{Phonetics}
\end{Entry}

\begin{Entry}{逶}{11}{⾡}
  \begin{Phonetics}{逶}{wei1}
    \definition{adj.}{sinuoso; tortuoso}
  \end{Phonetics}
\end{Entry}

\begin{Entry}{逶迤}{11,8}{⾡、⾡}
  \begin{Phonetics}{逶迤}{wei1yi2}
    \definition{adj.}{sinuoso; tortuoso; descreve a aparência sinuosa e contínua de estradas, montanhas, rios, etc.}
  \end{Phonetics}
\end{Entry}

\begin{Entry}{逻}{11}{⾡}
  \begin{Phonetics}{逻}{luo2}
    \definition{s.}{patrulha | (literário) a beira de um riacho de montanha}
    \definition{v.}{patrulhar; fazer rondas}
  \end{Phonetics}
\end{Entry}

\begin{Entry}{逻辑}{11,13}{⾡、⾞}
  \begin{Phonetics}{逻辑}{luo2ji5}[][HSK 5]
    \definition[套,条,种]{s.}{lógica; lei objetiva; a objetividade das leis que regem o desenvolvimento das coisas | lógica; razão; regras para o pensamento | lógica como ciência do raciocínio, do pensamento; disciplina que estuda a lógica}
  \end{Phonetics}
\end{Entry}

\begin{Entry}{逼}{12}{⾡}
  \begin{Phonetics}{逼}{bi1}[][HSK 6]
    \definition{adj.}{estreito}
    \definition{v.}{forçar; pressionar; compelir | extorquir; pressionar por | fechar em; pressionar em direção a; aproximar-se}
  \end{Phonetics}
\end{Entry}

\begin{Entry}{逼近}{12,7}{⾡、⾡}
  \begin{Phonetics}{逼近}{bi1jin4}[][HSK 7-9]
    \definition{adj.}{aproximado (valor númerico)}
    \definition{s.}{aproximação (função matemática mais simples)}
    \definition{v.}{avançar em direção a; aproximar-se de; aproximar-se; aproximar-se | ganhar em (sobre); aglomerar-se em}
  \end{Phonetics}
\end{Entry}

\begin{Entry}{逼迫}{12,8}{⾡、⾡}
  \begin{Phonetics}{逼迫}{bi1po4}[][HSK 7-9]
    \definition{v.}{forçar; compelir; coagir | restringir; exercer pressão para induzir; forçar}
  \end{Phonetics}
\end{Entry}

\begin{Entry}{逼真}{12,10}{⾡、⼗}
  \begin{Phonetics}{逼真}{bi1zhen1}[][HSK 7-9]
    \definition{adj.}{fiel à realidade; realista; muito semelhante à coisa real | claro; distinto; verdadeiro}
  \end{Phonetics}
\end{Entry}

\begin{Entry}{遇}{12}{⾡}
  \begin{Phonetics}{遇}{yu4}[][HSK 4]
    \definition*{s.}{Sobrenome Yu}
    \definition{s.}{chance; oportunidade}
    \definition{v.}{encontrar; deparar-se com; encontrar-se | tratar; receber}
  \end{Phonetics}
\end{Entry}

\begin{Entry}{遇见}{12,4}{⾡、⾒}
  \begin{Phonetics}{遇见}{yu4 jian4}[][HSK 4]
    \definition{v.}{encontrar; deparar-se com}
  \end{Phonetics}
\end{Entry}

\begin{Entry}{遇到}{12,8}{⾡、⼑}
  \begin{Phonetics}{遇到}{yu4dao4}[][HSK 4]
    \definition{v.}{esbarrar em; encontrar; deparar-se com; conhecer alguém ou algo (inesperado)}
  \end{Phonetics}
\end{Entry}

\begin{Entry}{遍}{12}{⾡}
  \begin{Phonetics}{遍}{bian4}[][HSK 2]
    \definition{adv.}{por toda parte; em toda parte; em todos os lugares}
    \definition{clas.}{usado para a repetição de ações de leitura, fala ou escrita}
  \end{Phonetics}
\end{Entry}

\begin{Entry}{遍布}{12,5}{⾡、⼱}
  \begin{Phonetics}{遍布}{bian4bu4}[][HSK 7-9]
    \definition{v.}{encontrar em todos os lugares; espalhar por toda parte; distribuir em todos os lugares}
  \end{Phonetics}
\end{Entry}

\begin{Entry}{遍地}{12,6}{⾡、⼟}
  \begin{Phonetics}{遍地}{bian4 di4}[][HSK 6]
    \definition{adv.}{em todos os lugares; em toda parte; por toda parte}
  \end{Phonetics}
\end{Entry}

\begin{Entry}{遏}{12}{⾡}
  \begin{Phonetics}{遏}{e4}
    \definition{v.}{reprimir; restringir; reter; impedir; proibir}
  \end{Phonetics}
\end{Entry}

\begin{Entry}{遏制}{12,8}{⾡、⼑}
  \begin{Phonetics}{遏制}{e4zhi4}[][HSK 7-9]
    \definition{v.}{conter; restringir; controlar e prevenir ativamente o desenvolvimento de coisas que possam trazer perigo; usado principalmente para discutir tópicos formais}
  \end{Phonetics}
\end{Entry}

\begin{Entry}{道}{12}{⾡}
  \begin{Phonetics}{道}{dao4}[][HSK 2]
    \definition*{s.}{Taoismo;  Taoista | Sobrenome Dao}
    \definition{clas.}{usado para pratos em refeições, etapas em um procedimento, etc. | usado para certos objetos longos e estreitos; tira | usado para portas, paredes, etc.; pesado | usado para comandos, títulos, etc.}
    \definition[条]{s.}{estrada; caminho; trilha | curso; canal; o caminho percorrido pelo fluxo da água | maneira; método; princípio; raciocínio | moral; moralidade | habilidade; técnica | doutrina; princípio; sistema de pensamento acadêmico ou religioso; origem de todas as coisas no universo | taoísta; taoísmo; pertencente ao taoísmo | seita supersticiosa; certas organizações reacionárias e supersticiosas | linha; traços finos e alongados | trato; os canais dentro do corpo}
    \definition{v.}{dizer; falar; expressar-se | pensar; supor; considerar; acreditar que}
  \end{Phonetics}
\end{Entry}

\begin{Entry}{道行}{12,6}{⾡、⾏}
  \begin{Phonetics}{道行}{dao4 heng2}
    \definition{s.}{realizações de um monge budista ou sacerdote taoísta | habilidades; capacidades; aptidões | (figurativo) habilidade | habilidades adquiridas através da prática religiosa}
  \end{Phonetics}
\end{Entry}

\begin{Entry}{道具}{12,8}{⾡、⼋}
  \begin{Phonetics}{道具}{dao4ju4}[][HSK 7-9]
    \definition{s.}{adereços; objetos de cena; artigos de palco; objetos usados ​​em apresentações, como mesas e cadeiras, são chamados de grandes adereços, enquanto cigarros e xícaras de chá são chamados de pequenos adereços}
  \end{Phonetics}
\end{Entry}

\begin{Entry}{道教}{12,11}{⾡、⽁}
  \begin{Phonetics}{道教}{dao4 jiao4}[][HSK 6]
    \definition*{s.}{Taoísmo (sistema de crenças chinês)}
    \definition{s.}{a religião taoísta; taoísmo}
  \end{Phonetics}
\end{Entry}

\begin{Entry}{道理}{12,11}{⾡、⽟}
  \begin{Phonetics}{道理}{dao4li5}[][HSK 2]
    \definition[个,种]{s.}{verdade; princípio; a lei das coisas | sentido; razão}
  \end{Phonetics}
\end{Entry}

\begin{Entry}{道路}{12,13}{⾡、⾜}
  \begin{Phonetics}{道路}{dao4 lu4}[][HSK 2]
    \definition[条,段]{s.}{estrada; caminho; os canais de comunicação entre os dois lugares, incluindo terrestres e aquáticos | caminho; processo; refere-se à vida, à existência (significado abstrato)}
  \end{Phonetics}
\end{Entry}

\begin{Entry}{道歉}{12,14}{⾡、⽋}
  \begin{Phonetics}{道歉}{dao4/qian4}[][HSK 6]
    \definition{v.+compl.}{pedir desculpas; fazer um pedido de desculpas; dizer aos outros que você estava errado e pedir perdão}
  \end{Phonetics}
\end{Entry}

\begin{Entry}{道德}{12,15}{⾡、⼻}
  \begin{Phonetics}{道德}{dao4de2}[][HSK 5]
    \definition{adj.}{moral; descreve uma pessoa ou comportamento que atende aos requisitos morais; mais usado em situações negativas}
    \definition[种]{s.}{moral; ética; moralidade; regras e normas para que as pessoas vivam juntas e se comportem em comum}
  \end{Phonetics}
\end{Entry}

\begin{Entry}{遗}{12}{⾡}
  \begin{Phonetics}{遗}{yi2}
    \definition*{s.}{Sobrenome Yi}
    \definition{s.}{descarga involuntária de urina, etc. | algo perdido}
    \definition{v.}{perder | omitir | deixar para trás; guardar; não dar | deixar para trás após a morte; legar; transmitir}
  \end{Phonetics}
\end{Entry}

\begin{Entry}{遗产}{12,6}{⾡、⼇}
  \begin{Phonetics}{遗产}{yi2chan3}[][HSK 4]
    \definition[笔,份]{s.}{legado; herança; patrimônio; propriedade deixada pelo falecido | patrimônio; riqueza cultural ou riqueza material transmitida pela história}
  \end{Phonetics}
\end{Entry}

\begin{Entry}{遗传}{12,6}{⾡、⼈}
  \begin{Phonetics}{遗传}{yi2chuan2}[][HSK 4]
    \definition{v.}{herdar, descender, transmitir, passar adiante}
  \end{Phonetics}
\end{Entry}

\begin{Entry}{遗男}{12,7}{⾡、⽥}
  \begin{Phonetics}{遗男}{yi2nan2}
    \definition{s.}{órfão | filho póstumo}
  \end{Phonetics}
\end{Entry}

\begin{Entry}{遗迹}{12,9}{⾡、⾡}
  \begin{Phonetics}{遗迹}{yi2ji4}
    \definition{s.}{vestígio histórico; sítio; vestígio; traço; ruína; vestígios deixados por tempos antigos ou eras passadas}
  \end{Phonetics}
\end{Entry}

\begin{Entry}{遗案}{12,10}{⾡、⽊}
  \begin{Phonetics}{遗案}{yi2'an4}
    \definition{s.}{(lei) caso não resolvido}
  \end{Phonetics}
\end{Entry}

\begin{Entry}{遗落}{12,12}{⾡、⾋}
  \begin{Phonetics}{遗落}{yi2luo4}
    \definition{v.}{esquecer | deixar para trás (inadvertidamente) | deixar de fora | omitir}
  \end{Phonetics}
\end{Entry}

\begin{Entry}{遗嘱}{12,15}{⾡、⼝}
  \begin{Phonetics}{遗嘱}{yi2zhu3}
    \definition{s.}{testamento}
  \end{Phonetics}
\end{Entry}

\begin{Entry}{遗骸}{12,15}{⾡、⾻}
  \begin{Phonetics}{遗骸}{yi2hai2}
    \definition{v.}{restos mortais}
  \end{Phonetics}
\end{Entry}

\begin{Entry}{遗憾}{12,16}{⾡、⼼}
  \begin{Phonetics}{遗憾}{yi2han4}[][HSK 6]
    \definition{adj.}{triste; arrependido; contrito; sentir pena de situações que estão fora de controle ou são insatisfatórias}
    \definition{s.}{pena; arrependimento; sentindo pena que os desejos não se realizaram}
  \end{Phonetics}
\end{Entry}

\begin{Entry}{遛}{13}{⾡}
  \begin{Phonetics}{遛}{liu4}
    \definition{v.}{passear | andar (um animal) | caminhar conduzindo um animal doméstico}
  \end{Phonetics}
\end{Entry}

\begin{Entry}{遛狗}{13,8}{⾡、⽝}
  \begin{Phonetics}{遛狗}{liu4/gou3}
    \definition{v.+compl.}{passear com um cachorro}
  \end{Phonetics}
\end{Entry}

\begin{Entry}{遥}{13}{⾡}
  \begin{Phonetics}{遥}{yao2}
    \definition{adj.}{distante; remoto; longe}
  \end{Phonetics}
\end{Entry}

\begin{Entry}{遥控}{13,11}{⾡、⼿}
  \begin{Phonetics}{遥控}{yao2kong4}
    \definition{s.}{controle remoto}
    \definition{v.}{dirigir operações de um local remoto | controlar remotamente}
  \end{Phonetics}
\end{Entry}

\begin{Entry}{遭}{14}{⾡}
  \begin{Phonetics}{遭}{zao1}
    \definition{clas.}{tempo; vez; ocasião | rodadas}
    \definition{v.}{encontrar-se com (desastre, infortúnio, etc.); sofrer}
  \end{Phonetics}
\end{Entry}

\begin{Entry}{遭到}{14,8}{⾡、⼑}
  \begin{Phonetics}{遭到}{zao1 dao4}[][HSK 6]
    \definition{v.}{sofrer; ser rejeitado; receber crítica; significa sofrer infortúnio ou dano}[我们遭到意外事故。===Nós sofremos um acidente.]
  \end{Phonetics}
\end{Entry}

\begin{Entry}{遭受}{14,8}{⾡、⼜}
  \begin{Phonetics}{遭受}{zao1shou4}[][HSK 6]
    \definition{v.}{sofrer; aguentar; ser submetido a; encontrar ou vivenciar coisas dolorosas que você não quer que aconteçam}
  \end{Phonetics}
\end{Entry}

\begin{Entry}{遭遇}{14,12}{⾡、⾡}
  \begin{Phonetics}{遭遇}{zao1yu4}[][HSK 6]
    \definition[场,次,种,段]{s.}{sorte (difícil); experiência (amarga); encontrando coisas ruins}
    \definition{v.}{encontrar; encontrar-se com; esbarrar em; encontros inesperados com pessoas ou coisas que não são boas para você}
  \end{Phonetics}
\end{Entry}

\begin{Entry}{遵}{15}{⾡}
  \begin{Phonetics}{遵}{zun1}
    \definition{v.}{cumprir; obedecer; observar; seguir}
  \end{Phonetics}
\end{Entry}

\begin{Entry}{遵守}{15,6}{⾡、⼧}
  \begin{Phonetics}{遵守}{zun1shou3}[][HSK 5]
    \definition{v.}{obedecer; observar; cumprir; respeitar; atuar de acordo com as regras; não infringir}
  \end{Phonetics}
\end{Entry}

\begin{Entry}{邀}{16}{⾡}
  \begin{Phonetics}{邀}{yao1}
    \definition{v.}{convidar; requerer | (literário)  buscar aprovação; pedir permissão | interceptar}
  \end{Phonetics}
\end{Entry}

\begin{Entry}{邀请}{16,10}{⾡、⾔}
  \begin{Phonetics}{邀请}{yao1qing3}[][HSK 5]
    \definition[份,个]{s.}{convite}
    \definition{v.}{convidar; solicitar; convidar pessoas para irem à sua casa ou a um local combinado}
  \end{Phonetics}
\end{Entry}

\begin{Entry}{邉}{17}{⾡}
  \begin{Phonetics}{邉}{bian1}
    \variantof{边}
  \end{Phonetics}
\end{Entry}

%%%%% EOF %%%%%

