%%%
%%% Radical "⾡"
%%%

\section*{Radical 162: ``⾡'' (⻌、⻍、⻎)}\addcontentsline{toc}{section}{Radical 162: ⾡、⻌、⻍、⻎}

\begin{entry}{边}{5}{⾡}
  \begin{phonetics}{边}{bian1}[][HSK 2]
    \definition{adv.}{simultaneamente}
    \definition[个]{s.}{fronteira | limite | borda | margem | lado}
  \end{phonetics}
  \begin{phonetics}{边}{bian5}
    \definition{suf.}{sufixo de uma palavra de localidade}
  \end{phonetics}
\end{entry}

\begin{entry}{边关}{5,6}{⾡、⼋}
  \begin{phonetics}{边关}{bian1guan1}
    \definition{s.}{posto de fronteira | posição defensiva estratégica na fronteira}
  \end{phonetics}
\end{entry}

\begin{entry}{边防}{5,6}{⾡、⾩}
  \begin{phonetics}{边防}{bian1fang2}
    \definition{s.}{defesa da fronteira}
  \end{phonetics}
\end{entry}

\begin{entry}{边境}{5,14}{⾡、⼟}
  \begin{phonetics}{边境}{bian1jing4}[][HSK 5]
    \definition{s.}{fronteira}
  \end{phonetics}
\end{entry}

\begin{entry}{巡逻}{6,11}{⾡、⾡}
  \begin{phonetics}{巡逻}{xun2luo2}
    \definition{s.}{patrulha}
    \definition{v.}{patrulhar (polícia, exército ou marinha)}
  \end{phonetics}
\end{entry}

\begin{entry}{达成}{6,6}{⾡、⼽}
  \begin{phonetics}{达成}{da2cheng2}[][HSK 5]
    \definition{v.}{concluir; chegar (a um acordo); conseguir; obter (principalmente como resultado de uma negociação)}
  \end{phonetics}
\end{entry}

\begin{entry}{达到}{6,8}{⾡、⼑}
  \begin{phonetics}{达到}{da2dao4}[][HSK 3]
    \definition{v.}{alcançar; atingir; atender o padrão}
  \end{phonetics}
\end{entry}

\begin{entry}{迅速}{6,10}{⾡、⾡}
  \begin{phonetics}{迅速}{xun4su4}[][HSK 4]
    \definition{adv.}{rapidamente; velozmente; prontamente}
  \end{phonetics}
\end{entry}

\begin{entry}{过}{6}{⾡}
  \begin{phonetics}{过}{guo1}
    \definition*{s.}{sobrenome Guo}
  \end{phonetics}
  \begin{phonetics}{过}{guo4}[][HSK 1,2]
    \definition{part.}{passado}
    \definition{v.}{atravessar | passar (tempo)}
  \end{phonetics}
  \begin{phonetics}{过}{guo5}
    \definition{part.}{(marcador de ação experiente)}
  \end{phonetics}
\end{entry}

\begin{entry}{过于}{6,3}{⾡、⼆}
  \begin{phonetics}{过于}{guo4yu2}[][HSK 5]
    \definition{adv.}{demais; indevidamente; excessivamente; advérbios de grau ou quantidade excessiva}
  \end{phonetics}
\end{entry}

\begin{entry}{过不惯}{6,4,11}{⾡、⼀、⼼}
  \begin{phonetics}{过不惯}{guo4 bu5 guan4}
    \definition{v.}{não se acostumar | não se habituar}
    \seeref{过惯}{guo4guan4}
  \end{phonetics}
\end{entry}

\begin{entry}{过分}{6,4}{⾡、⼑}
  \begin{phonetics}{过分}{guo4fen4}[][HSK 4]
    \definition{adj.}{excessivo; muito longe; demais; falar ou agir além dos limites ou graus adequados}
    \definition{adv.}{excessivamente; indevidamente; muito mesmo}
  \end{phonetics}
\end{entry}

\begin{entry}{过去}{6,5}{⾡、⼛}
  \begin{phonetics}{过去}{guo4 qu4}[][HSK 2,3]
    \definition{v.}{atravessar, passar por (a partir da minha localização) | falecer}
  \end{phonetics}
\end{entry}

\begin{entry}{过节}{6,5}{⾡、⾋}
  \begin{phonetics}{过节}{guo4jie2}
    \definition{v.+compl.}{celebrar festividades | comemorar um festival}
  \end{phonetics}
\end{entry}

\begin{entry}{过关}{6,6}{⾡、⼋}
  \begin{phonetics}{过关}{guo4guan1}
    \definition{v.+compl.}{passar uma barreira | passar por uma provação | passar em um teste | atingir um padrão | passar pela alfândega}
  \end{phonetics}
\end{entry}

\begin{entry}{过年}{6,6}{⾡、⼲}
  \begin{phonetics}{过年}{guo4 nian2}[][HSK 2]
    \definition{v.+compl.}{comemorar o Ano Novo | comemorar o Festival da Primavera | passar o Ano Novo | passar o Festival da Primavera}
  \end{phonetics}
\end{entry}

\begin{entry}{过来}{6,7}{⾡、⽊}
  \begin{phonetics}{过来}{guo4 lai2}[][HSK 2]
    \definition{v.}{atravessar (para a minha localização) | vir até aqui}
  \end{phonetics}
\end{entry}

\begin{entry}{过度}{6,9}{⾡、⼴}
  \begin{phonetics}{过度}{guo4du4}[][HSK 5]
    \definition{adj.}{excessivo; acima do limite; além do limite; além do que é apropriado}
  \end{phonetics}
\end{entry}

\begin{entry}{过惯}{6,11}{⾡、⼼}
  \begin{phonetics}{过惯}{guo4guan4}
    \definition{v.}{estar acostumado (a um certo estilo de vida, etc.)}
    \seeref{过不惯}{guo4 bu5 guan4}
  \end{phonetics}
\end{entry}

\begin{entry}{过敏}{6,11}{⾡、⽁}
  \begin{phonetics}{过敏}{guo4min3}[][HSK 5]
    \definition{adj.}{sensível; excessivamente sensível; resposta acima do normal; ceticismo excessivo}
    \definition{v.}{ser alérgico a}
  \end{phonetics}
\end{entry}

\begin{entry}{过期}{6,12}{⾡、⽉}
  \begin{phonetics}{过期}{guo4qi1}
    \definition{v.+compl.}{exceder a data | passar a data | expirar (passar a data de expiração)}
  \end{phonetics}
\end{entry}

\begin{entry}{过程}{6,12}{⾡、⽲}
  \begin{phonetics}{过程}{guo4cheng2}[][HSK 3]
    \definition[个]{s.}{curso dos eventos; processo}
  \end{phonetics}
\end{entry}

\begin{entry}{过瘾}{6,16}{⾡、⽧}
  \begin{phonetics}{过瘾}{guo4yin3}
    \definition{adj.}{gratificante | imensamente agradável | satisfatório}
    \definition{v.+compl.}{satisfazer um desejo | se divertir com algo}
  \end{phonetics}
\end{entry}

\begin{entry}{迎接}{7,11}{⾡、⼿}
  \begin{phonetics}{迎接}{ying2jie1}[][HSK 3]
    \definition{v.}{conhecer; cumprimentar; dar as boas-vindas}
  \end{phonetics}
\end{entry}

\begin{entry}{运气}{7,4}{⾡、⽓}
  \begin{phonetics}{运气}{yun4qi5}[][HSK 4]
    \definition{adj.}{sortudo; afortunado}
    \definition{s.}{sorte; fortuna}
  \end{phonetics}
\end{entry}

\begin{entry}{运用}{7,5}{⾡、⽤}
  \begin{phonetics}{运用}{yun4yong4}[][HSK 4]
    \definition{v.}{usar; utilizar; manejar; aplicar; explorar as coisas de acordo com suas características}
  \end{phonetics}
\end{entry}

\begin{entry}{运动}{7,6}{⾡、⼒}
  \begin{phonetics}{运动}{yun4dong4}[][HSK 2]
    \definition[场]{s.}{esporte | desporto}
    \definition{v.}{exercitar | mover-se}
  \end{phonetics}
\end{entry}

\begin{entry}{运动会}{7,6,6}{⾡、⼒、⼈}
  \begin{phonetics}{运动会}{yun4 dong4 hui4}[][HSK 4]
    \definition[个]{s.}{jogos; encontro esportivo; dia de esportes; reunião atlética}
  \end{phonetics}
\end{entry}

\begin{entry}{运动场}{7,6,6}{⾡、⼒、⼟}
  \begin{phonetics}{运动场}{yun4dong4chang3}
    \definition{s.}{campo desportivo | campo de jogos}
  \end{phonetics}
\end{entry}

\begin{entry}{运动员}{7,6,7}{⾡、⼒、⼝}
  \begin{phonetics}{运动员}{yun4 dong4 yuan2}[][HSK 4]
    \definition[名,个]{s.}{jogador; atleta; esportista; pessoas que participam de competições esportivas}
  \end{phonetics}
\end{entry}

\begin{entry}{运动学}{7,6,8}{⾡、⼒、⼦}
  \begin{phonetics}{运动学}{yun4dong4xue2}
    \definition{s.}{cinemática}
  \end{phonetics}
\end{entry}

\begin{entry}{运动服}{7,6,8}{⾡、⼒、⽉}
  \begin{phonetics}{运动服}{yun4dong4fu2}
    \definition{s.}{roupa para prática de esporte}
  \end{phonetics}
\end{entry}

\begin{entry}{运动衫}{7,6,8}{⾡、⼒、⾐}
  \begin{phonetics}{运动衫}{yun4dong4shan1}
    \definition[件]{s.}{moletom | camisa esportiva}
  \end{phonetics}
\end{entry}

\begin{entry}{运动家}{7,6,10}{⾡、⼒、⼧}
  \begin{phonetics}{运动家}{yun4dong4jia1}
    \definition{s.}{ativista | atleta | esportista}
  \end{phonetics}
\end{entry}

\begin{entry}{运动病}{7,6,10}{⾡、⼒、⽧}
  \begin{phonetics}{运动病}{yun4dong4bing4}
    \definition{s.}{enjôo (movimento, carro, etc.)}
  \end{phonetics}
\end{entry}

\begin{entry}{运动鞋}{7,6,15}{⾡、⼒、⾰}
  \begin{phonetics}{运动鞋}{yun4dong4xie2}
    \definition{s.}{tênis | sapatos esportivos}
  \end{phonetics}
\end{entry}

\begin{entry}{运行}{7,6}{⾡、⾏}
  \begin{phonetics}{运行}{yun4xing2}
    \definition{v.}{(corpos celestes, etc.) mover-se ao longo do curso | (figurativo) funcionar, estar em operação | (serviço de trem, etc.) operar | (computador) executar um programa}
  \end{phonetics}
\end{entry}

\begin{entry}{运河}{7,8}{⾡、⽔}
  \begin{phonetics}{运河}{yun4he2}
    \definition{s.}{canal (em um rio)}
  \end{phonetics}
\end{entry}

\begin{entry}{运输}{7,13}{⾡、⾞}
  \begin{phonetics}{运输}{yun4shu1}[][HSK 3]
    \definition{v.}{enviar; transportar; usar um carro, navio, avião, etc. para transportar pessoas ou coisas de um lugar para outro}
  \end{phonetics}
\end{entry}

\begin{entry}{近}{7}{⾡}
  \begin{phonetics}{近}{jin4}[][HSK 2]
    \definition{adj.}{perto | próximo}
  \end{phonetics}
\end{entry}

\begin{entry}{近代}{7,5}{⾡、⼈}
  \begin{phonetics}{近代}{jin4dai4}[][HSK 4]
    \definition{s.}{tempos modernos; era passada relativamente próxima à era moderna, geralmente referida na história chinesa como 1840 a 1919 | na história mundial, geralmente se refere à era capitalista}
  \end{phonetics}
\end{entry}

\begin{entry}{近来}{7,7}{⾡、⽊}
  \begin{phonetics}{近来}{jin4lai2}[][HSK 5]
    \definition{adv.}{ultimamente; recentemente; de ​​tarde; refere-se a um período de tempo entre o passado imediato e o presente}
  \end{phonetics}
\end{entry}

\begin{entry}{近期}{7,12}{⾡、⽉}
  \begin{phonetics}{近期}{jin4 qi1}[][HSK 3]
    \definition{adv.}{num futuro próximo; brevemente}
  \end{phonetics}
\end{entry}

\begin{entry}{返回}{7,6}{⾡、⼞}
  \begin{phonetics}{返回}{fan3 hui2}[][HSK 5]
    \definition{v.}{retornar; ir (voltar); reverter; recorrer; retroceder; voltar para (o lugar original)}
  \end{phonetics}
\end{entry}

\begin{entry}{还}{7}{⾡}
  \begin{phonetics}{还}{hai2}[][HSK 1]
    \definition{adv.}{ainda | também | ainda mais | razoavelmente | bastante}
  \end{phonetics}
  \begin{phonetics}{还}{huan2}[][HSK 1]
    \definition*{s.}{sobrenome Huan}
    \definition{v.}{devolver | restituir | pagar de volta}
  \end{phonetics}
\end{entry}

\begin{entry}{还有}{7,6}{⾡、⽉}
  \begin{phonetics}{还有}{hai2 you3}[][HSK 1]
    \definition{adv.}{além do mais | além disso | ainda permanece | ainda há}
  \end{phonetics}
\end{entry}

\begin{entry}{还是}{7,9}{⾡、⽇}
  \begin{phonetics}{还是}{hai2shi5}[][HSK 1]
    \definition{adv.}{ainda (como antes) | inesperadamente | teve melhor}
    \definition{conj.}{ou (somente para frases interrogativas)}
  \end{phonetics}
\end{entry}

\begin{entry}{这}{7}{⾡}
  \begin{phonetics}{这}{zhe4}[][HSK 1]
    \definition{pron.}{este, isto}
  \end{phonetics}
  \begin{phonetics}{这}{zhei4}
    \definition{pron.}{(coloquial) este}
  \end{phonetics}
\end{entry}

\begin{entry}{这儿}{7,2}{⾡、⼉}
  \begin{phonetics}{这儿}{zhe4r5}[][HSK 1]
    \definition{pron.}{aqui}
  \end{phonetics}
\end{entry}

\begin{entry}{这么}{7,3}{⾡、⼃}
  \begin{phonetics}{这么}{zhe4 me5}[][HSK 2]
    \definition{adv.}{como este | desta maneira}
  \end{phonetics}
\end{entry}

\begin{entry}{这边}{7,5}{⾡、⾡}
  \begin{phonetics}{这边}{zhe4bian5}[][HSK 1]
    \definition{pron.}{aqui | este lado}
  \end{phonetics}
\end{entry}

\begin{entry}{这时}{7,7}{⾡、⽇}
  \begin{phonetics}{这时}{zhe4 shi2}[][HSK 2]
    \definition{adv.}{neste momento}
  \end{phonetics}
\end{entry}

\begin{entry}{这时候}{7,7,10}{⾡、⽇、⼈}
  \begin{phonetics}{这时候}{zhe4 shi2 hou5}[][HSK 2]
    \definition{adv.}{neste momento}
  \end{phonetics}
\end{entry}

\begin{entry}{这里}{7,7}{⾡、⾥}
  \begin{phonetics}{这里}{zhe4li3}[][HSK 1]
    \definition{pron.}{aqui}
  \end{phonetics}
\end{entry}

\begin{entry}{这些}{7,8}{⾡、⼆}
  \begin{phonetics}{这些}{zhe4xie1}[][HSK 1]
    \definition{pron.}{estes}
  \end{phonetics}
\end{entry}

\begin{entry}{这样}{7,10}{⾡、⽊}
  \begin{phonetics}{这样}{zhe4 yang4}[][HSK 2]
    \definition{adv.}{assim | dessa maneira | deste modo}
  \end{phonetics}
\end{entry}

\begin{entry}{这麽}{7,14}{⾡、⿇}
  \begin{phonetics}{这麽}{zhe4 me5}
    \variantof{这么}
  \end{phonetics}
\end{entry}

\begin{entry}{进}{7}{⾡}
  \begin{phonetics}{进}{jin4}[][HSK 1]
    \definition{clas.}{para seções em um edifício ou complexo residencial}
    \definition{s.}{(matemática) base de um sistema numérico}
    \definition{v.}{entrar}
  \end{phonetics}
\end{entry}

\begin{entry}{进一步}{7,1,7}{⾡、⼀、⽌}
  \begin{phonetics}{进一步}{jin4 yi2 bu4}[][HSK 3]
    \definition{adv.}{mais; dar um passo adiante; avançar um passo}
  \end{phonetics}
\end{entry}

\begin{entry}{进入}{7,2}{⾡、⼊}
  \begin{phonetics}{进入}{jin4 ru4}[][HSK 2]
    \definition{v.}{entrar | juntar-se}
  \end{phonetics}
\end{entry}

\begin{entry}{进口}{7,3}{⾡、⼝}
  \begin{phonetics}{进口}{jin4kou3}[][HSK 4]
    \definition{adj.}{importado}
    \definition{s.}{importação; entrada de um edifício ou local, também chamada de `` 入口''}
    \definition{v.+compl.}{importar; comprar ou transportar mercadorias de outro país ou região | entrar no porto; navegar em direção a um porto}
  \seealsoref{入口}{ru4kou3}
  \end{phonetics}
\end{entry}

\begin{entry}{进化}{7,4}{⾡、⼔}
  \begin{phonetics}{进化}{jin4hua4}[][HSK 5]
    \definition[个]{s.}{evolução; os organismos se desenvolvem e evoluem do simples para o complexo e de níveis baixos para altos}
    \definition{v.}{evoluir; um termo geral usado para descrever uma mudança gradual para melhor}
  \end{phonetics}
\end{entry}

\begin{entry}{进出口}{7,5,3}{⾡、⼐、⼝}
  \begin{phonetics}{进出口}{jin4chu1kou3}
    \definition{s.}{importação e exportação}
    \definition{v.}{importar e exportar}
  \end{phonetics}
\end{entry}

\begin{entry}{进去}{7,5}{⾡、⼛}
  \begin{phonetics}{进去}{jin4 qu4}[][HSK 1]
    \definition{v.}{entrar (a partir da minha localização)}
  \end{phonetics}
\end{entry}

\begin{entry}{进行}{7,6}{⾡、⾏}
  \begin{phonetics}{进行}{jin4xing2}[][HSK 2]
    \definition{v.}{continuar | estar em andamento | fazer | conduzir | continuar | executar | marchar | avançar | prosseguir | estar em marcha}
  \end{phonetics}
\end{entry}

\begin{entry}{进行编程}{7,6,12,12}{⾡、⾏、⽷、⽲}
  \begin{phonetics}{进行编程}{jin4xing2bian1cheng2}
    \definition{s.}{programa de computador executável}
  \end{phonetics}
\end{entry}

\begin{entry}{进来}{7,7}{⾡、⽊}
  \begin{phonetics}{进来}{jin4 lai2}[][HSK 1]
    \definition{v.}{entrar (para a minha localização)}
  \end{phonetics}
\end{entry}

\begin{entry}{进步}{7,7}{⾡、⽌}
  \begin{phonetics}{进步}{jin4bu4}[][HSK 3]
    \definition{adj.}{progressivo}
    \definition[个]{s.}{avanço; progresso; melhora}
    \definition{v.}{avançar; progredir; melhorar}
  \end{phonetics}
\end{entry}

\begin{entry}{进展}{7,10}{⾡、⼫}
  \begin{phonetics}{进展}{jin4zhan3}[][HSK 3]
    \definition{v.}{fazer progresso; progredir}
  \end{phonetics}
\end{entry}

\begin{entry}{远}{7}{⾡}
  \begin{phonetics}{远}{yuan3}[][HSK 1]
    \definition{adj.}{longe | distante | remoto}
  \end{phonetics}
\end{entry}

\begin{entry}{远天}{7,4}{⾡、⼤}
  \begin{phonetics}{远天}{yuan3tian1}
    \definition{s.}{paraíso | o céu distante}
  \end{phonetics}
\end{entry}

\begin{entry}{远方}{7,4}{⾡、⽅}
  \begin{phonetics}{远方}{yuan3fang1}
    \definition{s.}{longe | um local distante}
  \end{phonetics}
\end{entry}

\begin{entry}{远远}{7,7}{⾡、⾡}
  \begin{phonetics}{远远}{yuan3yuan3}
    \definition{adv.}{de longe}
  \end{phonetics}
\end{entry}

\begin{entry}{远征}{7,8}{⾡、⼻}
  \begin{phonetics}{远征}{yuan3zheng1}
    \definition{s.}{uma expedição militar | marcha para regiões remotas}
  \end{phonetics}
\end{entry}

\begin{entry}{违反}{7,4}{⾡、⼜}
  \begin{phonetics}{违反}{wei2fan3}[][HSK 5]
    \definition{v.}{violar; transgredir; contrariar; não estar em conformidade (com as regras, regulamentos, etc.)}
  \end{phonetics}
\end{entry}

\begin{entry}{违法}{7,8}{⾡、⽔}
  \begin{phonetics}{违法}{wei2 fa3}[][HSK 5]
    \definition{v.}{ser ilegal; infringir a lei; violar a lei ou os regulamentos}
  \end{phonetics}
\end{entry}

\begin{entry}{违规}{7,8}{⾡、⾒}
  \begin{phonetics}{违规}{wei2 gui1}[][HSK 5]
    \definition{v.}{violar (regras); infringir as regras e regulamentos}
  \end{phonetics}
\end{entry}

\begin{entry}{违宪}{7,9}{⾡、⼧}
  \begin{phonetics}{违宪}{wei2xian4}
    \definition{adj.}{inconstitucional}
  \end{phonetics}
\end{entry}

\begin{entry}{连}{7}{⾡}
  \begin{phonetics}{连}{lian2}[][HSK 3]
    \definition*{s.}{sobrenome Lian}
    \definition{adv.}{em sucessão; um após o outro; repetidamente | até}
    \definition{prep.}{incluindo}
    \definition{s.}{companhia | conjunção}
    \definition{v.}{ligar; juntar; conectar | envolver (em problemas); implicar | costurar; coser}
  \end{phonetics}
\end{entry}

\begin{entry}{连忙}{7,6}{⾡、⼼}
  \begin{phonetics}{连忙}{lian2mang2}[][HSK 3]
    \definition{adv.}{prontamente; imediatamente; apressadamente}
  \end{phonetics}
\end{entry}

\begin{entry}{连接}{7,11}{⾡、⼿}
  \begin{phonetics}{连接}{lian2 jie1}[][HSK 5]
    \definition[条]{s.}{conexão}
    \definition{v.}{ligar; unir; relacionar, conectar; anexar}
  \end{phonetics}
\end{entry}

\begin{entry}{连续}{7,11}{⾡、⽷}
  \begin{phonetics}{连续}{lian2xu4}[][HSK 3]
    \definition{adv.}{continuamente; sucessivamente; em uma fileira}
  \end{phonetics}
\end{entry}

\begin{entry}{连续剧}{7,11,10}{⾡、⽷、⼑}
  \begin{phonetics}{连续剧}{lian2 xu4 ju4}[][HSK 3]
    \definition{s.}{série; novela}
  \end{phonetics}
\end{entry}

\begin{entry}{连锁反应}{7,12,4,7}{⾡、⾦、⼜、⼴}
  \begin{phonetics}{连锁反应}{lian2suo3fan3ying4}
    \definition{s.}{reação em cadeia}
  \end{phonetics}
\end{entry}

\begin{entry}{迟}{7}{⾡}
  \begin{phonetics}{迟}{chi2}[][HSK 5]
    \definition*{s.}{sobrenome Chi}
    \definition{adj.}{lento; tardio; demorado | atrasado | lento; obtuso}
  \end{phonetics}
\end{entry}

\begin{entry}{迟到}{7,8}{⾡、⼑}
  \begin{phonetics}{迟到}{chi2dao4}[][HSK 4]
    \definition{v.}{chegar atrasado; atrasar-se}
  \end{phonetics}
\end{entry}

\begin{entry}{迫切}{8,4}{⾡、⼑}
  \begin{phonetics}{迫切}{po4qie4}[][HSK 4]
    \definition{adj.}{urgente; premente; muito ansiosamente, a ponto de ser difícil esperar}
  \end{phonetics}
\end{entry}

\begin{entry}{迷}{9}{⾡}
  \begin{phonetics}{迷}{mi2}[][HSK 3]
    \definition*{s.}{sobrenome Mi}
    \definition{adj.}{perdido; confuso}
    \definition{s.}{fã; entusiasta; fanático}
    \definition{v.}{estar confuso; perder o rumo; se perder-se | ficar fascinado por; entregar-se a; ficar encantado com (por); ser louco por | confundir; desorientar; fascinar; encantar}
  \end{phonetics}
\end{entry}

\begin{entry}{迷人}{9,2}{⾡、⼈}
  \begin{phonetics}{迷人}{mi2ren2}[][HSK 5]
    \definition{adj.}{encantador; fascinante; sedutor; hipnotizante}
    \definition{v.}{confundir; intrigar; enganar}
  \end{phonetics}
\end{entry}

\begin{entry}{迷你}{9,7}{⾡、⼈}
  \begin{phonetics}{迷你}{mi2ni3}
    \definition{adj.}{(empréstimo linguístico) mini, como em minissaia ou \emph{Mini Cooper}}
  \end{phonetics}
\end{entry}

\begin{entry}{迷信}{9,9}{⾡、⼈}
  \begin{phonetics}{迷信}{mi2xin4}[][HSK 5]
    \definition{s.}{superstição; crença supersticiosa; fé cega; adoração cega; crença em deuses, espíritos e fantasmas}
    \definition{v.}{ter fé cega em; fazer um fetiche de}
  \end{phonetics}
\end{entry}

\begin{entry}{迷宫}{9,9}{⾡、⼧}
  \begin{phonetics}{迷宫}{mi2gong1}
    \definition{s.}{labirinto}
  \end{phonetics}
\end{entry}

\begin{entry}{迷恋}{9,10}{⾡、⼼}
  \begin{phonetics}{迷恋}{mi2lian4}
    \definition{adj.}{obcecado}
    \definition{v.}{estar/ser apaixonado por | ficar encantado por | estar/ser obcecado por}
  \end{phonetics}
\end{entry}

\begin{entry}{迷路}{9,13}{⾡、⾜}
  \begin{phonetics}{迷路}{mi2lu4}
    \definition{s.}{labirinto | ouvido interno}
    \definition{v.+compl.}{perder o caminho | perder-se | seguir pelo caminho errado | não conseguir encontrar o caminho}
  \end{phonetics}
\end{entry}

\begin{entry}{追}{9}{⾡}
  \begin{phonetics}{追}{zhui1}[][HSK 3]
    \definition*{s.}{sobrenome Zhui}
    \definition{v.}{perseguir; correr atrás; ir atrás de; alcançar | rastrear; investigar; chegar ao fundo de | ansiar por (depois); ir atrás; procurar | recordar; relembrar; lembrar | agir retroativamente; fazer postumamente}
  \end{phonetics}
\end{entry}

\begin{entry}{追求}{9,7}{⾡、⽔}
  \begin{phonetics}{追求}{zhui1qiu2}[][HSK 4]
    \definition{s.}{perseguição (ações e metas positivas)}[她的追求是获得成功。(Sua meta é alcançar o sucesso.)]
    \definition{v.}{buscar; aspirar; perseguir | cortejar, uma referência especial ao namoro}
  \end{phonetics}
\end{entry}

\begin{entry}{追赶}{9,10}{⾡、⾛}
  \begin{phonetics}{追赶}{zhui1gan3}
    \definition{v.}{perseguir | acelerar | alcançar | ultrapassar}
  \end{phonetics}
\end{entry}

\begin{entry}{退}{9}{⾡}
  \begin{phonetics}{退}{tui4}[][HSK 3]
    \definition{v.}{recuar; mover-se para trás | fazer recuar; remover; retirar | desistir; retirar-se de | retroceder; refluir; declinar | desaparecer; desvanecer | devolver; retornar | cancelar; rescindir; romper}
  \end{phonetics}
\end{entry}

\begin{entry}{退出}{9,5}{⾡、⼐}
  \begin{phonetics}{退出}{tui4 chu1}[][HSK 3]
    \definition{v.}{desistir; retirar-se; separar-se; retirar-se de}
  \end{phonetics}
\end{entry}

\begin{entry}{退休}{9,6}{⾡、⼈}
  \begin{phonetics}{退休}{tui4xiu1}[][HSK 3]
    \definition{v.+compl.}{aposentar-se}
  \end{phonetics}
\end{entry}

\begin{entry}{送}{9}{⾡}
  \begin{phonetics}{送}{song4}[][HSK 1]
    \definition{v.}{distribuir | entregar | dar | oferecer (alguma coisa como presente) | enviar | remeter}
  \end{phonetics}
\end{entry}

\begin{entry}{送到}{9,8}{⾡、⼑}
  \begin{phonetics}{送到}{song4 dao4}[][HSK 2]
    \definition{v.}{enviar para (lugar)}
  \end{phonetics}
\end{entry}

\begin{entry}{送给}{9,9}{⾡、⽷}
  \begin{phonetics}{送给}{song4 gei3}[][HSK 2]
    \definition{v.}{dar a (alguém ou organização)}
  \end{phonetics}
\end{entry}

\begin{entry}{适用}{9,5}{⾡、⽤}
  \begin{phonetics}{适用}{shi4 yong4}[][HSK 3]
    \definition{adj.}{adequado; aplicável}
    \definition{v.}{ser aplicável; ser adequado}
  \end{phonetics}
\end{entry}

\begin{entry}{适合}{9,6}{⾡、⼝}
  \begin{phonetics}{适合}{shi4he2}[][HSK 3]
    \definition{v.}{servir (uma roupa); caber; se adequar}
  \end{phonetics}
\end{entry}

\begin{entry}{适应}{9,7}{⾡、⼴}
  \begin{phonetics}{适应}{shi4ying4}[][HSK 3]
    \definition{v.}{ajustar-se; adequar-se; adaptar-se}
  \end{phonetics}
\end{entry}

\begin{entry}{逃}{9}{⾡}
  \begin{phonetics}{逃}{tao2}[][HSK 5]
    \definition{v.}{fugir; escapar; correr; dar no pé | evadir; esquivar-se; escapar}
  \end{phonetics}
\end{entry}

\begin{entry}{逃走}{9,7}{⾡、⾛}
  \begin{phonetics}{逃走}{tao2 zou3}[][HSK 5]
    \definition{v.}{escapar; afastar-se de pessoas, coisas ou lugares que não são bons para você ou que você não gosta}
  \end{phonetics}
\end{entry}

\begin{entry}{逃跑}{9,12}{⾡、⾜}
  \begin{phonetics}{逃跑}{tao2 pao3}[][HSK 5]
    \definition{v.}{fugir; escapar; correr; partir para fugir de um ambiente ou de coisas que não lhe são favoráveis}
  \end{phonetics}
\end{entry}

\begin{entry}{逆境}{9,14}{⾡、⼟}
  \begin{phonetics}{逆境}{ni4jing4}
    \definition{s.}{adversidade | tribulação}
  \end{phonetics}
\end{entry}

\begin{entry}{选}{9}{⾡}
  \begin{phonetics}{选}{xuan3}[][HSK 2]
    \definition{s.}{seleções | antologia}
    \definition{v.}{selecionar | escolher | eleger}
  \end{phonetics}
\end{entry}

\begin{entry}{选手}{9,4}{⾡、⼿}
  \begin{phonetics}{选手}{xuan3shou3}[][HSK 3]
    \definition[位]{s.}{jogador; competidor (selecionado); atleta selecionado para uma competição esportiva}
  \end{phonetics}
\end{entry}

\begin{entry}{选择}{9,8}{⾡、⼿}
  \begin{phonetics}{选择}{xuan3ze2}[][HSK 4]
    \definition[个,种,次]{s.}{escolha; opção; resultado da escolha; possibilidade de escolha}
    \definition{v.}{selecionar; escolher}
  \end{phonetics}
\end{entry}

\begin{entry}{透}{10}{⾡}
  \begin{phonetics}{透}{tou4}[][HSK 4]
    \definition{adv.}{totalmente; completamente; minuciosamente | profundamente; extremamente}
    \definition{v.}{penetrar; passar através de; infiltrar-se através de | revelar; deixar transparecer; contar secretamente |mostrar; aparecer}
  \end{phonetics}
\end{entry}

\begin{entry}{透支}{10,4}{⾡、⽀}
  \begin{phonetics}{透支}{tou4zhi1}
    \definition{v.}{cheque especial (bancário) | saque a descoberto}
  \end{phonetics}
\end{entry}

\begin{entry}{透气}{10,4}{⾡、⽓}
  \begin{phonetics}{透气}{tou4qi4}
    \definition{v.}{respirar (sobre tecido, etc.) | fluir livremente (sobre ar) | respirar ar fresco | ventilar}
  \end{phonetics}
\end{entry}

\begin{entry}{透水}{10,4}{⾡、⽔}
  \begin{phonetics}{透水}{tou4shui3}
    \definition{adj.}{permeável}
    \definition{s.}{vazamento de água}
  \end{phonetics}
\end{entry}

\begin{entry}{透过}{10,6}{⾡、⾡}
  \begin{phonetics}{透过}{tou4guo4}
    \definition{v.}{passar através | penetrar}
  \end{phonetics}
\end{entry}

\begin{entry}{透彻}{10,7}{⾡、⼻}
  \begin{phonetics}{透彻}{tou4che4}
    \definition{adj.}{minucioso | incisivo | penetrante}
  \end{phonetics}
\end{entry}

\begin{entry}{透明}{10,8}{⾡、⽇}
  \begin{phonetics}{透明}{tou4ming2}[][HSK 4]
    \definition{adj.}{transparente; diáfano; capaz de transmitir luz | evidente; transparente; situação ou assunto que seja aberto e não oculto | transparente; diáfano; indica pureza, ausência de impurezas}
  \end{phonetics}
\end{entry}

\begin{entry}{透顶}{10,8}{⾡、⾴}
  \begin{phonetics}{透顶}{tou4ding3}
    \definition{adv.}{completamente}
  \end{phonetics}
\end{entry}

\begin{entry}{透亮}{10,9}{⾡、⼇}
  \begin{phonetics}{透亮}{tou4liang4}
    \definition{adj.}{brilhante | claro como cristal}
  \end{phonetics}
\end{entry}

\begin{entry}{透辟}{10,13}{⾡、⾟}
  \begin{phonetics}{透辟}{tou4pi4}
    \definition{adj.}{incisivo | penetrante}
  \end{phonetics}
\end{entry}

\begin{entry}{透澈}{10,15}{⾡、⽔}
  \begin{phonetics}{透澈}{tou4che4}
    \variantof{透彻}
  \end{phonetics}
\end{entry}

\begin{entry}{透露}{10,21}{⾡、⾬}
  \begin{phonetics}{透露}{tou4lu4}
    \definition{v.}{divulgar | vazar | revelar}
  \end{phonetics}
\end{entry}

\begin{entry}{逐步}{10,7}{⾡、⽌}
  \begin{phonetics}{逐步}{zhu2bu4}[][HSK 4]
    \definition{adv.}{gradualmente; passo a passo; progressivamente}
  \end{phonetics}
\end{entry}

\begin{entry}{逐渐}{10,11}{⾡、⽔}
  \begin{phonetics}{逐渐}{zhu2jian4}[][HSK 4]
    \definition{adv.}{gradualmente; aos poucos; por etapas; indica mudanças lentas e ordenadas no grau, na quantidade, etc.}
  \end{phonetics}
\end{entry}

\begin{entry}{递}{10}{⾡}
  \begin{phonetics}{递}{di4}[][HSK 5]
    \definition{adv.}{na ordem correta; sucessivamente}
    \definition{v.}{entregar; passar; dar; transmitir}
  \end{phonetics}
\end{entry}

\begin{entry}{递给}{10,9}{⾡、⽷}
  \begin{phonetics}{递给}{di4 gei3}[][HSK 5]
    \definition{v.}{entregar algo a alguém; passar itens ou coisas para outras pessoas}
  \end{phonetics}
\end{entry}

\begin{entry}{途中}{10,4}{⾡、⼁}
  \begin{phonetics}{途中}{tu2 zhong1}[][HSK 4]
    \definition{adv.}{no caminho; ao longo do caminho}
  \end{phonetics}
\end{entry}

\begin{entry}{通}{10}{⾡}
  \begin{phonetics}{通}{tong1}[][HSK 2]
    \definition{clas.}{para cartas, telegramas, telefonemas, etc.}
    \definition{suf.}{especialista}
    \definition{v.}{ligar para | conseguir a ligação}
  \end{phonetics}
  \begin{phonetics}{通}{tong4}
    \definition{clas.}{para uma atividade, tomada em sua totalidade (discurso de abuso, período de reprodução de música, bebedeira, etc.)}
  \end{phonetics}
\end{entry}

\begin{entry}{通用}{10,5}{⾡、⽤}
  \begin{phonetics}{通用}{tong1yong4}[][HSK 5]
    \definition{adj.}{de uso comum; universal; (em um determinado âmbito) de uso generalizado | intercambiável; alguns caracteres chineses com grafia diferente, mas pronúncia igual, podem ser usados indistintamente (alguns limitados a um determinado significado)}
  \end{phonetics}
\end{entry}

\begin{entry}{通观}{10,6}{⾡、⾒}
  \begin{phonetics}{通观}{tong1guan1}
    \definition{v.}{ter uma visão geral de algo}
  \end{phonetics}
\end{entry}

\begin{entry}{通过}{10,6}{⾡、⾡}
  \begin{phonetics}{通过}{tong1guo4}[][HSK 2]
    \definition{adv.}{por meio de | através de | via}
    \definition{v.}{passar por | adotar (uma resolução), aprovar (legislação) | passar (em um teste)}
  \end{phonetics}
\end{entry}

\begin{entry}{通识}{10,7}{⾡、⾔}
  \begin{phonetics}{通识}{tong1shi2}
    \definition{s.}{conhecimento comum | erudição | conhecimento geral | amplamente conhecido}
  \end{phonetics}
\end{entry}

\begin{entry}{通知}{10,8}{⾡、⽮}
  \begin{phonetics}{通知}{tong1zhi1}[][HSK 2]
    \definition[份,个,张]{s.}{aviso | circular}
    \definition{v.}{aconselhar | notificar | informar | dar aviso}
  \end{phonetics}
\end{entry}

\begin{entry}{通知书}{10,8,4}{⾡、⽮、⼄}
  \begin{phonetics}{通知书}{tong1 zhi1 shu1}[][HSK 4]
    \definition{s.}{aviso; observação; notificação}
  \end{phonetics}
\end{entry}

\begin{entry}{通信}{10,9}{⾡、⼈}
  \begin{phonetics}{通信}{tong1 xin4}[][HSK 3]
    \definition{v.+compl.}{corresponder; comunicar por carta | transmitir (ou transportar) mensagem; passar (ou transmitir) informação}
  \end{phonetics}
\end{entry}

\begin{entry}{通常}{10,11}{⾡、⼱}
  \begin{phonetics}{通常}{tong1chang2}[][HSK 3]
    \definition{adj.}{usual; normal; geral}
    \definition{adv.}{habitualmente; usualmente; geralmente; ordinariamente}
  \end{phonetics}
\end{entry}

\begin{entry}{通牒}{10,13}{⾡、⽚}
  \begin{phonetics}{通牒}{tong1die2}
    \definition{s.}{nota diplomática}
  \end{phonetics}
\end{entry}

\begin{entry}{逛}{10}{⾡}
  \begin{phonetics}{逛}{guang4}[][HSK 4]
    \definition{v.}{perambular; passear; vaguear}
  \end{phonetics}
\end{entry}

\begin{entry}{速度}{10,9}{⾡、⼴}
  \begin{phonetics}{速度}{su4du4}[][HSK 3]
    \definition[个,种]{s.}{velocidade; taxa; ritmo; andamento | velocidade; rapidez}
  \end{phonetics}
\end{entry}

\begin{entry}{造}{10}{⾡}
  \begin{phonetics}{造}{zao4}[][HSK 3]
    \definition*{s.}{sobrenome Zao}
    \definition{clas.}{para colheitas ou número de colheitas de safras}
    \definition{s.}{uma das duas partes em um acordo legal ou uma ação judicial | colheita; safra}
    \definition{v.}{fazer; construir; criar; produzir | cozinhar; fabricar; inventar | ir para; chegar a | alcançar; atingir | treinar; educar}
  \end{phonetics}
\end{entry}

\begin{entry}{造成}{10,6}{⾡、⼽}
  \begin{phonetics}{造成}{zao4cheng2}[][HSK 3]
    \definition{v.}{criar; causar; acarretar; dar origem a; formar; levar a (principalmente resultados ruins)}
  \end{phonetics}
\end{entry}

\begin{entry}{造型}{10,9}{⾡、⼟}
  \begin{phonetics}{造型}{zao4xing2}[][HSK 4]
    \definition{s.}{modelo; formato; forma; moldagem}
    \definition{v.}{modelar; moldar}
  \end{phonetics}
\end{entry}

\begin{entry}{逮}{11}{⾡}
  \begin{phonetics}{逮}{dai3}
    \definition{v.}{(coloquial) pegar, aproveitar, capturar}
  \end{phonetics}
  \begin{phonetics}{逮}{dai4}
    \definition{v.}{(literário) alcançar, usado em 逮捕}
  \seealsoref{逮捕}{dai4bu3}
  \end{phonetics}
\end{entry}

\begin{entry}{逮捕}{11,10}{⾡、⼿}
  \begin{phonetics}{逮捕}{dai4bu3}
    \definition{v.}{prender | apreender | levar sob custódia}
  \end{phonetics}
\end{entry}

\begin{entry}{逻辑}{11,13}{⾡、⾞}
  \begin{phonetics}{逻辑}{luo2ji5}[][HSK 5]
    \definition{s.}{lógica; lei objetiva; a objetividade das leis que regem o desenvolvimento das coisas | lógica; razão; regras para o pensamento | lógica como ciência do raciocínio, do pensamento; disciplina que estuda a lógica}
  \end{phonetics}
\end{entry}

\begin{entry}{遇}{12}{⾡}
  \begin{phonetics}{遇}{yu4}[][HSK 4]
    \definition*{s.}{sobrenome Yu}
    \definition{s.}{chance; oportunidade}
    \definition{v.}{encontrar; deparar-se com; encontrar-se | tratar; receber}
  \end{phonetics}
\end{entry}

\begin{entry}{遇见}{12,4}{⾡、⾒}
  \begin{phonetics}{遇见}{yu4 jian4}[][HSK 4]
    \definition{v.}{encontrar; deparar-se com}
  \end{phonetics}
\end{entry}

\begin{entry}{遇到}{12,8}{⾡、⼑}
  \begin{phonetics}{遇到}{yu4dao4}[][HSK 4]
    \definition{v.}{esbarrar em; encontrar; deparar-se com; conhecer alguém ou algo (inesperado)}
  \end{phonetics}
\end{entry}

\begin{entry}{遍}{12}{⾡}
  \begin{phonetics}{遍}{bian4}[][HSK 2]
    \definition{adv.}{em todos os lugares | por toda parte}
    \definition{clas.}{para a repetição de ações de leitura, fala ou escrita}
  \end{phonetics}
\end{entry}

\begin{entry}{道}{12}{⾡}
  \begin{phonetics}{道}{dao4}[][HSK 2]
    \definition*{s.}{Taoism | Taoist}
    \definition*{s.}{sobrenome Dao}
    \definition{s.}{estrada | caminho | rota | caminho | canal | curso | maneira | método | moral | moralidade | doutrina | corpo de ensinamentos morais | o Caminho da Natureza que não pode receber um nome | princípio | seita supersticiosa | linha | trato | habilidade}
  \end{phonetics}
\end{entry}

\begin{entry}{道理}{12,11}{⾡、⽟}
  \begin{phonetics}{道理}{dao4li5}[][HSK 2]
    \definition[个]{s.}{razão | argumento | sentido | princípio | base | justificativa}
  \end{phonetics}
\end{entry}

\begin{entry}{道路}{12,13}{⾡、⾜}
  \begin{phonetics}{道路}{dao4 lu4}[][HSK 2]
    \definition{s.}{estrada | caminho | processo}
  \end{phonetics}
\end{entry}

\begin{entry}{道歉}{12,14}{⾡、⽋}
  \begin{phonetics}{道歉}{dao4qian4}
    \definition{v.+compl.}{desculpar-se | fazer um pedido de desculpas}
  \end{phonetics}
\end{entry}

\begin{entry}{道德}{12,15}{⾡、⼻}
  \begin{phonetics}{道德}{dao4de2}[][HSK 5]
    \definition{adj.}{moral; descreve uma pessoa ou comportamento que atende aos requisitos morais; mais usado em situações negativas}
    \definition{s.}{moral; ética; moralidade; regras e normas para que as pessoas vivam juntas e se comportem em comum}
  \end{phonetics}
\end{entry}

\begin{entry}{遗产}{12,6}{⾡、⼇}
  \begin{phonetics}{遗产}{yi2chan3}[][HSK 4]
    \definition[笔,份]{s.}{legado; herança; patrimônio; propriedade deixada pelo falecido | patrimônio; riqueza cultural ou riqueza material transmitida pela história}
  \end{phonetics}
\end{entry}

\begin{entry}{遗传}{12,6}{⾡、⼈}
  \begin{phonetics}{遗传}{yi2chuan2}[][HSK 4]
    \definition{v.}{herdar, descender, transmitir, passar adiante}
  \end{phonetics}
\end{entry}

\begin{entry}{遗男}{12,7}{⾡、⽥}
  \begin{phonetics}{遗男}{yi2nan2}
    \definition{s.}{órfão | filho póstumo}
  \end{phonetics}
\end{entry}

\begin{entry}{遗迹}{12,9}{⾡、⾡}
  \begin{phonetics}{遗迹}{yi2ji4}
    \definition{s.}{vestígios históricos | remanescente | vestígio}
  \end{phonetics}
\end{entry}

\begin{entry}{遗案}{12,10}{⾡、⽊}
  \begin{phonetics}{遗案}{yi2'an4}
    \definition{s.}{(lei) caso não resolvido}
  \end{phonetics}
\end{entry}

\begin{entry}{遗落}{12,12}{⾡、⾋}
  \begin{phonetics}{遗落}{yi2luo4}
    \definition{v.}{esquecer | deixar para trás (inadvertidamente) | deixar de fora | omitir}
  \end{phonetics}
\end{entry}

\begin{entry}{遗嘱}{12,15}{⾡、⼝}
  \begin{phonetics}{遗嘱}{yi2zhu3}
    \definition{s.}{testamento}
  \end{phonetics}
\end{entry}

\begin{entry}{遗骸}{12,15}{⾡、⾻}
  \begin{phonetics}{遗骸}{yi2hai2}
    \definition{v.}{restos mortais}
  \end{phonetics}
\end{entry}

\begin{entry}{遗憾}{12,16}{⾡、⼼}
  \begin{phonetics}{遗憾}{yi2han4}
    \definition{v.}{ter pena de | lamentar}
  \end{phonetics}
\end{entry}

\begin{entry}{遛狗}{13,8}{⾡、⽝}
  \begin{phonetics}{遛狗}{liu4gou3}
    \definition{v.+compl.}{passear com um cachorro}
  \end{phonetics}
\end{entry}

\begin{entry}{遥控}{13,11}{⾡、⼿}
  \begin{phonetics}{遥控}{yao2kong4}
    \definition{s.}{controle remoto}
    \definition{v.}{dirigir operações de um local remoto | controlar remotamente}
  \end{phonetics}
\end{entry}

\begin{entry}{遭到}{14,8}{⾡、⼑}
  \begin{phonetics}{遭到}{zao1dao4}
    \definition{v.}{sofrer | encontrar-se com (algo infeliz)}
  \end{phonetics}
\end{entry}

\begin{entry}{遭受}{14,8}{⾡、⼜}
  \begin{phonetics}{遭受}{zao1shou4}
    \definition{v.}{sofrer | suportar (perda, infornúnio)}
  \end{phonetics}
\end{entry}

\begin{entry}{遭遇}{14,12}{⾡、⾡}
  \begin{phonetics}{遭遇}{zao1yu4}
    \definition{s.}{experiência (amarga)}
    \definition{v.}{encontrar-se com (algo infeliz)}
  \end{phonetics}
\end{entry}

\begin{entry}{邉}{17}{⾡}
  \begin{phonetics}{邉}{bian1}
    \variantof{边}
  \end{phonetics}
\end{entry}

%%%%% EOF %%%%%

