%%%
%%% Radical "⼴"
%%%

\section*{Radical 53: ``⼴''}\addcontentsline{toc}{section}{Radical 53: ⼴}

\begin{entry}{广}{3}{⼴}
  \begin{phonetics}{广}{an1}
    \definition{s.}{mais comum em nomes de pessoas; o mesmo que 庵}[广安是我的朋友。___An'an é meu amigo.]
  \seealsoref{庵}{an1}
  \end{phonetics}
  \begin{phonetics}{广}{guang3}[][HSK 5]
    \definition*{s.}{Sobrenome Guang}
    \definition{adj.}{largo; vasto; amplo; extenso (oposto a 狭) | numeroso | comum; universal}
    \definition{s.}{Guangdong, 广东, e Guangxi, 广州}
    \definition{v.}{expandir; espalhar; ampliar}
  \seealsoref{广东}{guang3dong1}
  \seealsoref{广州}{guang3zhou1}
  \seealsoref{狭}{xia2}
  \end{phonetics}
  \begin{phonetics}{广}{yan3}
    \definition[家]{s.}{casa ou edifício construído contra ou ao longo da encosta de uma montanha ou penhasco}
  \end{phonetics}
\end{entry}

\begin{entry}{广大}{3,3}{⼴、⼤}
  \begin{phonetics}{广大}{guang3da4}[][HSK 3]
    \definition{adj.}{muito difundido; enorme (alcance, escala) | (uma área ou espaço) vasto; extenso; em grande escala; amplo (área, espaço) | numeroso; muitos (número de pessoas)}
  \end{phonetics}
\end{entry}

\begin{entry}{广东}{3,5}{⼴、⼀}
  \begin{phonetics}{广东}{guang3dong1}
    \definition*{s.}{Província de Guangdong}
  \seealsoref{粤}{yue4}
  \end{phonetics}
\end{entry}

\begin{entry}{广场}{3,6}{⼴、⼟}
  \begin{phonetics}{广场}{guang3chang3}[][HSK 2]
    \definition{s.}{praça; praça pública; esplanada; área ampla, especificamente uma área ampla na cidade}
  \end{phonetics}
\end{entry}

\begin{entry}{广场舞}{3,6,14}{⼴、⼟、⾇}
  \begin{phonetics}{广场舞}{guang3chang3wu3}
    \definition{s.}{quadrilha, uma rotina de exercícios tocada com música em quadrados públicos, parques e praças, popular especialmente entre mulheres de meia-idade e aposentados na China}
  \end{phonetics}
\end{entry}

\begin{entry}{广州}{3,6}{⼴、⼮}
  \begin{phonetics}{广州}{guang3zhou1}
    \definition*{s.}{Guangzhou, antigamente Cantão; Capital da Província de Guangdong}
  \end{phonetics}
\end{entry}

\begin{entry}{广告}{3,7}{⼴、⼝}
  \begin{phonetics}{广告}{guang3gao4}[][HSK 2]
    \definition[则,条,段,项,个]{s.}{anúncio; propaganda; uma forma de divulgação ao público de produtos, serviços ou programas culturais e esportivos, geralmente realizada por meio de jornais, televisão, rádio, cartazes, etc.}
    \definition{v.}{anunciar; a ação ou ato de promover ou divulgar algo}
  \end{phonetics}
\end{entry}

\begin{entry}{广泛}{3,7}{⼴、⽔}
  \begin{phonetics}{广泛}{guang3fan4}[][HSK 5]
    \definition{adj.}{amplo; extenso; de grande alcance; disseminado; escopo e cobertura amplos}
  \end{phonetics}
\end{entry}

\begin{entry}{广播}{3,15}{⼴、⼿}
  \begin{phonetics}{广播}{guang3bo1}[][HSK 3]
    \definition[个,次,段,则,条]{s.}{programa de rádio; transmissão (de rádio); refere-se a programas transmitidos por estações de rádio ou televisão a cabo}
    \definition{v.}{transmitir; estar no ar | espalhar-se amplamente; ser conhecido em toda parte; divulgar amplamente}
  \end{phonetics}
\end{entry}

\begin{entry}{庆}{6}{⼴}
  \begin{phonetics}{庆}{qing4}
    \definition*{s.}{Sobrenome Qing}
    \definition{s.}{celebração | ocasião para celebração; um aniversário que vale a pena comemorar}
    \definition{v.}{celebrar; felicitar; comemorar}
  \end{phonetics}
\end{entry}

\begin{entry}{庆祝}{6,9}{⼴、⽰}
  \begin{phonetics}{庆祝}{qing4zhu4}[][HSK 3]
    \definition{v.}{celebrar; comemorar; festejar; realizar atividades para comemorar ou celebrar festivais comuns e eventos felizes}
  \end{phonetics}
\end{entry}

\begin{entry}{床}{7}{⼴}
  \begin{phonetics}{床}{chuang2}[][HSK 1]
    \definition{clas.}{usado para colchas, roupas de cama, etc.}
    \definition[张]{s.}{cama; sofá; móveis para dormir | algo com o formato de uma cama}
  \end{phonetics}
\end{entry}

\begin{entry}{库}{7}{⼴}
  \begin{phonetics}{库}{ku4}[][HSK 5]
    \definition{s.}{depósito; tesouraria; armazém; almoxarifado; edifícios e equipamentos para armazenamento de mercadorias | banco de dados}
  \end{phonetics}
\end{entry}

\begin{entry}{应}{7}{⼴}
  \begin{phonetics}{应}{ying1}[][HSK 4,5]
    \definition{v.}{ecoar; responder; responder a; responder às chamadas, saudações, perguntas, etc. de outras pessoas | conceder; cumprir | adequar; adaptar; responder a | lidar com; enfrentar; abordar | tornar-se realidade; ser cumprido}
  \end{phonetics}
\end{entry}

\begin{entry}{应对}{7,5}{⼴、⼨}
  \begin{phonetics}{应对}{ying4dui4}
    \definition{v.}{responder | manusear | lidar}
  \end{phonetics}
\end{entry}

\begin{entry}{应用}{7,5}{⼴、⽤}
  \begin{phonetics}{应用}{ying4yong4}[][HSK 3]
    \definition{adj.}{aplicado (na vida ou na produção); usado diretamente na vida ou na produção}
    \definition{v.}{usar; aplicar}
  \end{phonetics}
\end{entry}

\begin{entry}{应用程序}{7,5,12,7}{⼴、⽤、⽲、⼴}
  \begin{phonetics}{应用程序}{ying4yong4 cheng2xu4}
    \definition{s.}{programa aplicativo; principais categorias de \emph{software}}
  \end{phonetics}
\end{entry}

\begin{entry}{应用程序接口}{7,5,12,7,11,3}{⼴、⽤、⽲、⼴、⼿、⼝}
  \begin{phonetics}{应用程序接口}{ying4yong4cheng2xu4jie1kou3}
    \definition{s.}{API (\emph{application programming interface})}
  \seealsoref{应用程序编程接口}{ying4yong4cheng2xu4bian1cheng2jie1kou3}
  \end{phonetics}
\end{entry}

\begin{entry*}{应用程序编程接口}{7,5,12,7,12,12,11,3}{⼴、⽤、⽲、⼴、⽷、⽲、⼿、⼝}
  \begin{phonetics}{应用程序编程接口}{ying4yong4cheng2xu4bian1cheng2jie1kou3}
    \definition{s.}{API (\emph{application programming interface})}
  \seealsoref{应用程序接口}{ying4yong4cheng2xu4jie1kou3}
  \end{phonetics}
\end{entry*}

\begin{entry}{应当}{7,6}{⼴、⼹}
  \begin{phonetics}{应当}{ying1 dang1}[][HSK 3]
    \definition{v.}{dever}[学生们应当努力学习。___Os alunos devem se esforçar nos estudos.]
  \end{phonetics}
\end{entry}

\begin{entry}{应该}{7,8}{⼴、⾔}
  \begin{phonetics}{应该}{ying1gai1}[][HSK 2]
    \definition{v.}{deveria; deve ser assim | deveria; acho que deve ser esse o caso}
  \end{phonetics}
\end{entry}

\begin{entry}{底}{8}{⼴}
  \begin{phonetics}{底}{de5}
    \definition{part.}{usada após uma palavra ou frase que é usada como determinante para indicar subordinação à palavra central}
  \end{phonetics}
  \begin{phonetics}{底}{di3}[][HSK 4]
    \definition*{s.}{Sobrenome Di}
    \definition{pron.}{o que? |  isto; isso; aqui | assim; tal}
    \definition{s.}{base; fundo; parte inferior de um objeto | detalhes; o cerne da questão; base, fonte ou contexto de uma coisa | rascunho; cópia mantida como registro; rascunho que pode ser usado como base | final de um ano ou mês | chão; fundo; fundação | a última parte de algo}
  \end{phonetics}
\end{entry}

\begin{entry}{底下}{8,3}{⼴、⼀}
  \begin{phonetics}{底下}{di3 xia4}[][HSK 3]
    \definition{adv.}{em baixo; abaixo; sob | próximo; mais tarde; depois; daqui para a frente}
  \end{phonetics}
\end{entry}

\begin{entry}{底气}{8,4}{⼴、⽓}
  \begin{phonetics}{底气}{di3qi4}
    \definition{s.}{capacidade pulmonar | ousadia | confiança | autoconfiança | vigor}
  \end{phonetics}
\end{entry}

\begin{entry}{店}{8}{⼴}
  \begin{phonetics}{店}{dian4}[][HSK 2]
    \definition[家,间,个]{s.}{loja; armazém; loja de venda de mercadorias | pousada; pequena pousada com instalações simples | usado para nomes de lugares}
  \end{phonetics}
\end{entry}

\begin{entry}{店主}{8,5}{⼴、⼂}
  \begin{phonetics}{店主}{dian4zhu3}
    \definition{s.}{lojista | dono de loja}
  \end{phonetics}
\end{entry}

\begin{entry}{店员}{8,7}{⼴、⼝}
  \begin{phonetics}{店员}{dian4yuan2}
    \definition{s.}{assistente de loja | balconista | vendedor}
  \end{phonetics}
\end{entry}

\begin{entry}{度}{9}{⼴}
  \begin{phonetics}{度}{du4}[][HSK 2]
    \definition*{s.}{Sobrenome Du}
    \definition{clas.}{grau; unidade de medida para ângulos, temperatura, etc. | quilowatt-hora (kWh) | usado para indicar a quantidade de álcool presente no vinho | usado para arcos e ângulos | usado para indicar o grau de curvatura da lente dos óculos ou o grau de miopia | tempo; número de vezes | usado para longitude e latitude, localização geográfica}
    \definition{s.}{medida linear; padrões e instrumentos para medir comprimentos | grau de intensidade; refere-se especificamente ao grau alcançado por uma determinada propriedade de uma coisa | limite; extensão; grau; quota | regras; código de conduta; diretrizes | tolerância; magnanimidade; refere-se especificamente ao grau de tolerância | maneira; temperamento; disposição; a personalidade ou aparência de uma pessoa | indicador de grau, nível alcançado por algo | tempo ou espaço limitado; um determinado período de tempo ou espaço}
    \definition{v.}{passar; atravessar; passar por cima | (em termos de tempo) passar; passar por | (de monges ou monjas budistas, ou sacerdotes taoístas) pregar; converter; proselitar}
  \end{phonetics}
  \begin{phonetics}{度}{duo2}
    \definition{v.}{supor; estimar; especular}
  \end{phonetics}
\end{entry}

\begin{entry}{度过}{9,6}{⼴、⾡}
  \begin{phonetics}{度过}{du4guo4}[][HSK 4]
    \definition{s.}{passar o tempo; fazer o tempo desaparecer no trabalho, na vida, no lazer e no descanso}
  \end{phonetics}
\end{entry}

\begin{entry}{座}{10}{⼴}
  \begin{phonetics}{座}{zuo4}[][HSK 2]
    \definition{clas.}{usado para montanhas, edifícios e objetos imóveis semelhantes}
    \definition{s.}{assento; lugar | suporte; pedestal; base | (astronomia) constalação | (antigo) forma de tratamento a altos funcionários |}
  \end{phonetics}
\end{entry}

\begin{entry}{座子}{10,3}{⼴、⼦}
  \begin{phonetics}{座子}{zuo4zi5}
    \definition{s.}{soquete | pedestal | sela}
  \end{phonetics}
\end{entry}

\begin{entry}{座位}{10,7}{⼴、⼈}
  \begin{phonetics}{座位}{zuo4wei4}[][HSK 2]
    \definition[个,排]{s.}{assento; lugar}
  \end{phonetics}
\end{entry}

\begin{entry}{座标}{10,9}{⼴、⽊}
  \begin{phonetics}{座标}{zuo4biao1}
    \variantof{坐标}
  \end{phonetics}
\end{entry}

\begin{entry}{庵}{11}{⼴}
  \begin{phonetics}{庵}{an1}
    \definition*{s.}{Sobrenome An}
    \definition[个,座]{s.}{cabana | convento de freiras; templos budistas, principalmente onde vivem as freiras}
  \end{phonetics}
\end{entry}

\begin{entry}{庶}{11}{⼴}
  \begin{phonetics}{庶}{shu4}
    \definition*{s.}{Sobrenome Shu}
    \definition{adj.}{multitudinário; numeroso}
    \definition{conj.}{para que; de ​​modo a}
    \definition{s.}{da ou pela concubina (diferentemente da esposa legal); no sistema patriarcal, refere-se ao ramo lateral da família}
  \end{phonetics}
\end{entry}

\begin{entry}{庶民}{11,5}{⼴、⽒}
  \begin{phonetics}{庶民}{shu4min2}
    \definition{s.}{(antigo) pessoas comuns | (antigo) plebeu; plebeus | (antigo) a multidão de pessoas comuns (na literatura erudita)}
  \end{phonetics}
\end{entry}

\begin{entry}{廊}{11}{⼴}
  \begin{phonetics}{廊}{lang2}
    \definition[个]{s.}{varanda; corredor}
  \end{phonetics}
\end{entry}

\begin{entry}{廊坊}{11,7}{⼴、⼟}
  \begin{phonetics}{廊坊}{lang2fang2}
    \definition*{s.}{Cidade de Langfang em Hebei}
  \end{phonetics}
\end{entry}

%%%%% EOF %%%%%

