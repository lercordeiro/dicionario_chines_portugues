%%%
%%% Radical "⽧"
%%%

\section*{Radical 104: ``⽧''}\addcontentsline{toc}{section}{Radical 104: ⽧}

\begin{entry}{疗养}{7,9}{⽧、⼋}
  \begin{phonetics}{疗养}{liao2 yang3}[][HSK 4]
    \definition{v.}{recuperar; convalescer; tratar pessoas com doenças crônicas ou debilitantes em instituições médicas especializadas com foco na recuperação}
  \end{phonetics}
\end{entry}

\begin{entry}{疯}{9}{⽧}
  \begin{phonetics}{疯}{feng1}[][HSK 5]
    \definition{adj.}{louco; doido; maluco; insano | termo usado para se referir ao crescimento vigoroso de plantas e culturas que não dão frutos}
  \end{phonetics}
\end{entry}

\begin{entry}{疯狂}{9,7}{⽧、⽝}
  \begin{phonetics}{疯狂}{feng1kuang2}[][HSK 5]
    \definition{adj.}{louco; insano; frenético; desenfreado}
  \end{phonetics}
\end{entry}

\begin{entry}{疼}{10}{⽧}
  \begin{phonetics}{疼}{teng2}[][HSK 2]
    \definition{adj.}{dolorido; doído; sensação de extremo desconforto causada por ferimentos, doenças, etc.}
    \definition{v.}{ferir; machucar | adorar; amar profundamente; gostar muito; cuidar}
  \end{phonetics}
\end{entry}

\begin{entry}{病}{10}{⽧}
  \begin{phonetics}{病}{bing4}[][HSK 1]
    \definition[种]{s.}{doença; enfermidade | doença; males | falha; defeito; desvantagem; erro}
    \definition{v.}{adoecer; ficar doente | ferir; causar danos a | angustiar; desaprovar}
  \end{phonetics}
\end{entry}

\begin{entry}{病人}{10,2}{⽧、⼈}
  \begin{phonetics}{病人}{bing4 ren2}[][HSK 1]
    \definition[个,位]{s.}{doente; paciente; pessoas doentes; pessoas em tratamento}
  \end{phonetics}
\end{entry}

\begin{entry}{病毒}{10,9}{⽧、⽏}
  \begin{phonetics}{病毒}{bing4du2}[][HSK 5]
    \definition[种,株,类]{s.}{vírus; patógenos que são menores que os germes e visíveis somente com um microscópio eletrônico | vírus de computador}
  \end{phonetics}
\end{entry}

\begin{entry}{痛}{12}{⽧}
  \begin{phonetics}{痛}{tong4}[][HSK 3]
    \definition*{s.}{sobrenome Tong}
    \definition{s.}{pesar; angústia; aflição; tristeza}
    \definition{v.}{doer; causar dor}
  \end{phonetics}
\end{entry}

\begin{entry}{痛快}{12,7}{⽧、⼼}
  \begin{phonetics}{痛快}{tong4kuai4}[][HSK 4]
    \definition{adj.}{encantado; alegre; muito feliz; confortável | franco; direto; simples e direto}
  \end{phonetics}
\end{entry}

\begin{entry}{痛苦}{12,8}{⽧、⾋}
  \begin{phonetics}{痛苦}{tong4ku3}[][HSK 3]
    \definition{adj.}{doloroso; angustiado}
    \definition[降,种]{s.}{dor; agonia; sofrimento}
  \end{phonetics}
\end{entry}

\begin{entry}{痛骂}{12,9}{⽧、⾺}
  \begin{phonetics}{痛骂}{tong4ma4}
    \definition{v.}{repreender severamente}
  \end{phonetics}
\end{entry}

\begin{entry}{痠}{12}{⽧}
  \begin{phonetics}{痠}{suan1}
    \definition{v.}{doer | estar dolorido}
    \variantof{酸}
  \end{phonetics}
\end{entry}

\begin{entry}{瘦}{14}{⽧}
  \begin{phonetics}{瘦}{shou4}[][HSK 5]
    \definition{adj.}{magro; esquelético (oposto de 胖, 肥) | magro (oposto de 肥) | apertado (oposto de 肥) | infértil; pobre | esquelético; pouca gordura; pouca carne (em oposição a 或 ou 肥) | (roupas, sapatos, meias, etc.) apertado (em oposição a 肥) |magra; (carne comestível) com baixo teor de gordura (em oposição a 肥)}
    \definition{v.}{perder peso}
  \seealsoref{肥}{fei2}
  \seealsoref{或}{huo4}
  \seealsoref{胖}{pang4}
  \end{phonetics}
\end{entry}

%%%%% EOF %%%%%

