%%%
%%% Radical "⽧"
%%%

\section*{Radical 104: ``⽧''}\addcontentsline{toc}{section}{Radical 104: ⽧}

\begin{Entry}{疗}{7}{⽧}
  \begin{Phonetics}{疗}{liao2}
    \definition{v.}{tratar; curar | recuperar}
  \end{Phonetics}
\end{Entry}

\begin{Entry}{疗养}{7,9}{⽧、⼋}
  \begin{Phonetics}{疗养}{liao2 yang3}[][HSK 4]
    \definition{v.}{recuperar; convalescer; tratar pessoas com doenças crônicas ou debilitantes em instituições médicas especializadas com foco na recuperação}
  \end{Phonetics}
\end{Entry}

\begin{Entry}{疯}{9}{⽧}
  \begin{Phonetics}{疯}{feng1}[][HSK 5]
    \definition{adj.}{louco; insano | tolo; leviano | (de uma planta, safra de grãos, etc.) esguia; refere-se ao crescimento vigoroso das plantações, mas sem frutos | com todas as forças; fazer o máximo possível}
    \definition{v.}{jogar sem restrições}
  \end{Phonetics}
\end{Entry}

\begin{Entry}{疯子}{9,3}{⽧、⼦}
  \begin{Phonetics}{疯子}{feng1zi5}[][HSK 7-9]
    \definition[个,些,种]{s.}{maníaco; lunático; louco; pessoas com doenças mentais graves}[别把我当成疯子!===Não me trate como um louco!]
  \end{Phonetics}
\end{Entry}

\begin{Entry}{疯狂}{9,7}{⽧、⽝}
  \begin{Phonetics}{疯狂}{feng1kuang2}[][HSK 5]
    \definition{adj.}{louco; insano; frenético; desenfreado}
  \end{Phonetics}
\end{Entry}

\begin{Entry}{疼}{10}{⽧}
  \begin{Phonetics}{疼}{teng2}[][HSK 2]
    \definition{adj.}{dolorido; doído; sensação de extremo desconforto causada por ferimentos, doenças, etc.}
    \definition{v.}{ferir; machucar | adorar; amar profundamente; gostar muito; cuidar}
  \end{Phonetics}
\end{Entry}

\begin{Entry}{疼痛}{10,12}{⽧、⽧}
  \begin{Phonetics}{疼痛}{teng2 tong4}[][HSK 6]
    \definition[阵,种]{s.}{dor; sofrimento; ferimento; descreve a sensação de dor causada por lesão ou doença}
  \end{Phonetics}
\end{Entry}

\begin{Entry}{疾}{10}{⽧}
  \begin{Phonetics}{疾}{ji2}
    \definition*{s.}{Sobrenome Ji}
    \definition{s.}{doença; enfermidade; moléstia; padecimento | sofrimento; dor; dificuldade; mazela}
  \end{Phonetics}
\end{Entry}

\begin{Entry}{疾病}{10,10}{⽧、⽧}
  \begin{Phonetics}{疾病}{ji2bing4}[][HSK 6]
    \definition[种]{s.}{doença; enfermidade; termo geral para doença}
  \end{Phonetics}
\end{Entry}

\begin{Entry}{病}{10}{⽧}
  \begin{Phonetics}{病}{bing4}[][HSK 1]
    \definition[种]{s.}{doença; enfermidade | doença; males | falha; defeito; desvantagem; erro}
    \definition{v.}{adoecer; ficar doente | ferir; causar danos a | angustiar; desaprovar}
  \end{Phonetics}
\end{Entry}

\begin{Entry}{病人}{10,2}{⽧、⼈}
  \begin{Phonetics}{病人}{bing4 ren2}[][HSK 1]
    \definition[个,位]{s.}{doente; paciente; pessoas doentes; pessoas em tratamento}
  \end{Phonetics}
\end{Entry}

\begin{Entry}{病床}{10,7}{⽧、⼴}
  \begin{Phonetics}{病床}{bing4chuang2}[][HSK 7-9]
    \definition[号,张]{s.}{cama de hospital | leito de doente}
  \end{Phonetics}
\end{Entry}

\begin{Entry}{病房}{10,8}{⽧、⼾}
  \begin{Phonetics}{病房}{bing4 fang2}[][HSK 6]
    \definition[个,间]{s.}{enfermaria de um hospital; quartos onde ficam os pacientes em hospitais e onde vivem em casas de repouso}
  \end{Phonetics}
\end{Entry}

\begin{Entry}{病毒}{10,9}{⽧、⽏}
  \begin{Phonetics}{病毒}{bing4du2}[][HSK 5]
    \definition[种,株,类]{s.}{vírus; patógenos que são menores que os germes e visíveis somente com um microscópio eletrônico | Computação: vírus de computador}
  \end{Phonetics}
\end{Entry}

\begin{Entry}{病症}{10,10}{⽧、⽧}
  \begin{Phonetics}{病症}{bing4zheng4}[][HSK 7-9]
    \definition[种]{s.}{doença; enfermidade}
  \end{Phonetics}
\end{Entry}

\begin{Entry}{病情}{10,11}{⽧、⼼}
  \begin{Phonetics}{病情}{bing4 qing2}[][HSK 6]
    \definition{s.}{estado de uma doença; condição do paciente; mudanças na doença}
  \end{Phonetics}
\end{Entry}

\begin{Entry}{症}{10}{⽧}
  \begin{Phonetics}{症}{zheng1}
    \definition{s.}{doença; enfermidade | (figurativo) ponto de atrito | tumor abdominal | obstrução intestinal}
  \end{Phonetics}
  \begin{Phonetics}{症}{zheng4}
    \definition{s.}{doença; enfermidade}
  \end{Phonetics}
\end{Entry}

\begin{Entry}{症状}{10,7}{⽧、⽝}
  \begin{Phonetics}{症状}{zheng4zhuang4}[][HSK 6]
    \definition[种,些]{s.}{sintoma; estado anormal de um organismo devido a uma doença, como tosse, febre, etc.}
  \end{Phonetics}
\end{Entry}

\begin{Entry}{痛}{12}{⽧}
  \begin{Phonetics}{痛}{tong4}[][HSK 3]
    \definition{adv.}{extremamente; profundamente; amargamente}
    \definition{s.}{dor; sofrimento | tristeza; pesar}
  \end{Phonetics}
\end{Entry}

\begin{Entry}{痛快}{12,7}{⽧、⼼}
  \begin{Phonetics}{痛快}{tong4kuai4}[][HSK 4]
    \definition{adj.}{encantado; alegre; muito feliz; confortável | franco; direto; simples e direto}
  \end{Phonetics}
\end{Entry}

\begin{Entry}{痛苦}{12,8}{⽧、⾋}
  \begin{Phonetics}{痛苦}{tong4ku3}[][HSK 3]
    \definition{adj.}{doloroso; angustiado; sentindo-se muito desconfortável física ou mentalmente}
    \definition[降,种]{s.}{dor; agonia; sofrimento; refere-se a um estado ou sentimento de extremo desconforto físico ou mental}
  \end{Phonetics}
\end{Entry}

\begin{Entry}{痛骂}{12,9}{⽧、⾺}
  \begin{Phonetics}{痛骂}{tong4ma4}
    \definition{v.}{repreender severamente}
  \end{Phonetics}
\end{Entry}

\begin{Entry}{痠}{12}{⽧}
  \begin{Phonetics}{痠}{suan1}
    \definition{v.}{doer | estar dolorido}
    \variantof{酸}
  \end{Phonetics}
\end{Entry}

\begin{Entry}{痴}{13}{⽧}
  \begin{Phonetics}{痴}{chi1}
    \definition{adj.}{bobo; idiota; estúpido | louco por alguém (ou algo); extremamente obcecado por alguém ou alguma coisa | Dialeto: louco; insano; mentalmente perturbado}
  \end{Phonetics}
\end{Entry}

\begin{Entry}{痴心}{13,4}{⽧、⼼}
  \begin{Phonetics}{痴心}{chi1xin1}[][HSK 7-9]
    \definition{adj.}{apaixonado; obcecado por alguém ou alguma coisa}
    \definition{s.}{desejo tolo; amor cego; paixão cega}
  \end{Phonetics}
\end{Entry}

\begin{Entry}{痴呆}{13,7}{⽧、⼝}
  \begin{Phonetics}{痴呆}{chi1dai1}[][HSK 7-9]
    \definition{adj.}{estúpido; tolo; expressão ou comportamento enfadonho}
    \definition{s.}{demência}
  \end{Phonetics}
\end{Entry}

\begin{Entry}{痴迷}{13,9}{⽧、⾡}
  \begin{Phonetics}{痴迷}{chi1mi2}[][HSK 7-9]
    \definition{v.}{ser ou estar apaixonado; ser ou estar obcecado; ser ou estar louco por}
  \end{Phonetics}
\end{Entry}

\begin{Entry}{瘦}{14}{⽧}
  \begin{Phonetics}{瘦}{shou4}[][HSK 5]
    \definition{adj.}{magro; esquelético (oposto de 胖, 肥) | magro (oposto de 肥) | apertado (oposto de 肥) | infértil; pobre | esquelético; pouca gordura; pouca carne (em oposição a 或 ou 肥) | (roupas, sapatos, meias, etc.) apertado (em oposição a 肥) |magra; (carne comestível) com baixo teor de gordura (em oposição a 肥)}
    \definition{v.}{perder peso}
  \seealsoref{肥}{fei2}
  \seealsoref{或}{huo4}
  \seealsoref{胖}{pang4}
  \end{Phonetics}
\end{Entry}

\begin{Entry}{癌}{17}{⽧}
  \begin{Phonetics}{癌}{ai2}[][HSK 7-9]
    \definition{s.}{câncer; carcinoma; tumor maligno}
  \end{Phonetics}
\end{Entry}

\begin{Entry}{癌症}{17,10}{⽧、⽧}
  \begin{Phonetics}{癌症}{ai2zheng4}[][HSK 7-9]
    \definition[种]{s.}{câncer; tumores malignos no corpo}
  \end{Phonetics}
\end{Entry}

\begin{Entry}{癫}{21}{⽧}
  \begin{Phonetics}{癫}{dian1}
    \definition{adj.}{mentalmente perturbado; insano; louco}
  \end{Phonetics}
\end{Entry}

%%%%% EOF %%%%%

