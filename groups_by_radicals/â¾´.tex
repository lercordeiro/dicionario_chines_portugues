%%%
%%% Radical "⾴"
%%%

\section*{Radical 181: ``⾴'' (页)}\addcontentsline{toc}{section}{Radical 181: ⾴、页}

\begin{entry}{页}{6}{⾴}[Kangxi 181]
  \begin{phonetics}{页}{ye4}[][HSK 1]
    \definition{clas.}{página; folha de papel; lâmina; antigamente, referia-se a uma folha de um livro encadernado; atualmente, refere-se a uma das faces de um livro impresso em ambos os lados}
    \definition{s.}{página; folha de papel; folhas soltas de um livro}
  \end{phonetics}
\end{entry}

\begin{entry}{顶}{8}{⾴}
  \begin{phonetics}{顶}{ding3}[][HSK 4]
    \definition{adv.}{muito (linguagem falada); a maioria; extremamente; expressa o grau mais alto, equivalente a 最 e 极}
    \definition{clas.}{para coisas que têm um topo}
    \definition{prep.}{até}
    \definition{s.}{coroa da cabeça; parte mais alta do corpo ou objeto | topo; limite superior; ponto mais alto}
    \definition{v.}{carregar na cabeça; carregar em sua cabeça | empurrar (ou apoiar) para cima; empurrar por baixo (ou por trás) |
dar cabeçadas; dar uma coronhada | sustentar; apoiar; suportar | resistir; ir contra; enfrentar | rebater; retorquir; responder de volta | cooperar; enfrentar; apoiar; dar suporte | igualar; ser equivalente a | substituir; tomar o lugar de | assumir o controle; transferir ou adquirir o direito de administrar um negócio ou alugar uma casa ou terreno}
  \seealsoref{极}{ji2}
  \seealsoref{最}{zui4}
  \end{phonetics}
\end{entry}

\begin{entry}{项}{9}{⾴}
  \begin{phonetics}{项}{xiang4}[][HSK 4]
    \definition*{s.}{sobrenome Xiang}
    \definition{clas.}{para itens discriminados; taxonomia}
    \definition{s.}{nuca (do pescoço); a parte de trás do pescoço |
soma (de dinheiro); fundos para fins especiais |
termo; em álgebra, significa uma única fórmula que não é unida por um sinal de mais ou de menos | item}
  \end{phonetics}
\end{entry}

\begin{entry}{项目}{9,5}{⾴、⽬}
  \begin{phonetics}{项目}{xiang4mu4}[][HSK 4]
    \definition{s.}{evento | item; projeto; trabalhos de engenharia, acadêmicos, etc., de conteúdo específico}
  \end{phonetics}
\end{entry}

\begin{entry}{顺}{9}{⾴}
  \begin{phonetics}{顺}{shun4}
    \definition{adj.}{correr bem | favorável}
  \end{phonetics}
\end{entry}

\begin{entry}{顺从}{9,4}{⾴、⼈}
  \begin{phonetics}{顺从}{shun4cong2}
    \definition{v.}{obedecer | submeter-se}
  \end{phonetics}
\end{entry}

\begin{entry}{顺心}{9,4}{⾴、⼼}
  \begin{phonetics}{顺心}{shun4xin1}
    \definition{adj.}{satisfatório | satisfeito}
  \end{phonetics}
\end{entry}

\begin{entry}{顺水}{9,4}{⾴、⽔}
  \begin{phonetics}{顺水}{shun4shui3}
    \definition{v.}{ir com o fluxo}
  \end{phonetics}
\end{entry}

\begin{entry}{顺延}{9,6}{⾴、⼵}
  \begin{phonetics}{顺延}{shun4yan2}
    \definition{v.}{adiar | procrastinar}
  \end{phonetics}
\end{entry}

\begin{entry}{顺当}{9,6}{⾴、⼹}
  \begin{phonetics}{顺当}{shun4dang5}
    \definition{adv.}{suavemente}
  \end{phonetics}
\end{entry}

\begin{entry}{顺耳}{9,6}{⾴、⽿}
  \begin{phonetics}{顺耳}{shun4'er3}
    \definition{adj.}{agradável ao ouvido}
  \end{phonetics}
\end{entry}

\begin{entry}{顺利}{9,7}{⾴、⼑}
  \begin{phonetics}{顺利}{shun4li4}[][HSK 2]
    \definition{adj.}{sem problemas; com sucesso; sem dificuldades; sem contratempos; sem obstáculos; sem obstáculos ou dificuldades significativas no desempenho das tarefas}
  \end{phonetics}
\end{entry}

\begin{entry}{顺序}{9,7}{⾴、⼴}
  \begin{phonetics}{顺序}{shun4xu4}[][HSK 4]
    \definition{adv.}{por sua vez; na ordem correta; na devida ordem; na ordem adequada; na ordem apropriada}
    \definition[个]{s.}{ordem; sequência; sucessão; subsequência; sequência simples; ordem de prioridade}
  \end{phonetics}
\end{entry}

\begin{entry}{顺畅}{9,8}{⾴、⽥}
  \begin{phonetics}{顺畅}{shun4chang4}
    \definition{adj.}{liso e sem obstáculos | fluente}
  \end{phonetics}
\end{entry}

\begin{entry}{顺便}{9,9}{⾴、⼈}
  \begin{phonetics}{顺便}{shun4bian4}
    \definition{adv.}{convenientemente | de passagem | sem muito esforço extra}
  \end{phonetics}
\end{entry}

\begin{entry}{顺叙}{9,9}{⾴、⼜}
  \begin{phonetics}{顺叙}{shun4xu4}
    \definition{s.}{narrativa cronológica}
  \end{phonetics}
\end{entry}

\begin{entry}{顺眼}{9,11}{⾴、⽬}
  \begin{phonetics}{顺眼}{shun4yan3}
    \definition{adj.}{agradável aos olhos}
  \end{phonetics}
\end{entry}

\begin{entry}{顺境}{9,14}{⾴、⼟}
  \begin{phonetics}{顺境}{shun4jing4}
    \definition{s.}{circunstâncias favoráveis}
  \end{phonetics}
\end{entry}

\begin{entry}{顺嘴}{9,16}{⾴、⼝}
  \begin{phonetics}{顺嘴}{shun4zui3}
    \definition{v.}{deixar escapar (sem pensar) | ler suavemente (texto) | adequar-se  ao gosto (comida)}
  \end{phonetics}
\end{entry}

\begin{entry}{顽强}{10,12}{⾴、⼸}
  \begin{phonetics}{顽强}{wan2qiang2}
    \definition{adj.}{persistente | tenaz | difícil de derrotar}
  \end{phonetics}
\end{entry}

\begin{entry}{顾问}{10,6}{⾴、⾨}
  \begin{phonetics}{顾问}{gu4wen4}[][HSK 5]
    \definition{s.}{conselheiro; consultor; assessor; pessoas com conhecimento especializado ou experiência contratadas para prestar consultoria a organizações ou indivíduos}
  \end{phonetics}
\end{entry}

\begin{entry}{顾客}{10,9}{⾴、⼧}
  \begin{phonetics}{顾客}{gu4ke4}[][HSK 2]
    \definition[个,位,名,些]{s.}{cliente; comprador; consumidor; paciente}
  \end{phonetics}
\end{entry}

\begin{entry}{顿}{10}{⾴}
  \begin{phonetics}{顿}{dun4}[][HSK 3]
    \definition*{s.}{sobrenome Dun}
    \definition{adj.}{cansado; fatigado}
    \definition{adv.}{de repente; imediatamente}
    \definition{clas.}{para refeições | para surras, repreensões, etc.}
    \definition{s.}{um lugar para ficar}
    \definition{v.}{pausar
pausar na escrita para reforçar o início ou o fim de um traço
tocar o chão (com a cabeça)
pisar (o pé)
resolver; arranjar
montar acampamento; ficar temporariamente}
  \end{phonetics}
\end{entry}

\begin{entry}{预}{10}{⾴}
  \begin{phonetics}{预}{yu4}
    \definition{adv.}{antecipadamente}
    \definition{v.}{avançar | preparar}
  \end{phonetics}
\end{entry}

\begin{entry}{预习}{10,3}{⾴、⼄}
  \begin{phonetics}{预习}{yu4xi2}[][HSK 3]
    \definition{v.}{pré-visualizar; preparar uma lição; estudar com antecedência as aulas que irá assistir}
  \end{phonetics}
\end{entry}

\begin{entry}{预见}{10,4}{⾴、⾒}
  \begin{phonetics}{预见}{yu4jian4}
    \definition{s.}{previsão; intuição; vislumbre}
    \definition{v.}{prever}
  \end{phonetics}
\end{entry}

\begin{entry}{预计}{10,4}{⾴、⾔}
  \begin{phonetics}{预计}{yu4 ji4}[][HSK 3]
    \definition{v.}{estimar; esperar; calcular com antecedência}
  \end{phonetics}
\end{entry}

\begin{entry}{预订}{10,4}{⾴、⾔}
  \begin{phonetics}{预订}{yu4ding4}[][HSK 4]
    \definition{v.}{reservar; fazer uma reserva}
  \end{phonetics}
\end{entry}

\begin{entry}{预付}{10,5}{⾴、⼈}
  \begin{phonetics}{预付}{yu4fu4}
    \definition{s.}{pré-pago}
    \definition{v.}{pagar antecipadamente}
  \end{phonetics}
\end{entry}

\begin{entry}{预约}{10,6}{⾴、⽷}
  \begin{phonetics}{预约}{yu4yue1}
    \definition{s.}{reserva}
    \definition{v.}{agendar | marcar compromisso}
  \end{phonetics}
\end{entry}

\begin{entry}{预防}{10,6}{⾴、⾩}
  \begin{phonetics}{预防}{yu4fang2}[][HSK 3]
    \definition{v.}{prevenir; proteger-se contra; tomar precauções contra}
  \end{phonetics}
\end{entry}

\begin{entry}{预判}{10,7}{⾴、⼑}
  \begin{phonetics}{预判}{yu4pan4}
    \definition{v.}{prever | antecipar}
  \end{phonetics}
\end{entry}

\begin{entry}{预报}{10,7}{⾴、⼿}
  \begin{phonetics}{预报}{yu4bao4}[][HSK 3]
    \definition[个]{s.}{boletim meteorológico; previsões meteorológicas antecipadas}
    \definition{v.}{prever (o tempo); relato de coisas antes que elas aconteçam, usado principalmente em clima, astronomia, desastres naturais, etc.}
  \end{phonetics}
\end{entry}

\begin{entry}{预备}{10,8}{⾴、⼡}
  \begin{phonetics}{预备}{yu4 bei4}[][HSK 5]
    \definition{v.}{preparar-se; ficar pronto}
  \end{phonetics}
\end{entry}

\begin{entry}{预定}{10,8}{⾴、⼧}
  \begin{phonetics}{预定}{yu4ding4}
    \definition{v.}{agendar com antecedência}
  \end{phonetics}
\end{entry}

\begin{entry}{预购}{10,8}{⾴、⾙}
  \begin{phonetics}{预购}{yu4gou4}
    \definition{s.}{compra antecipada}
    \definition{v.}{comprar antecipadamente}
  \end{phonetics}
\end{entry}

\begin{entry}{预测}{10,9}{⾴、⽔}
  \begin{phonetics}{预测}{yu4 ce4}[][HSK 4]
    \definition{v.}{prever; prognosticar; predizer}
  \end{phonetics}
\end{entry}

\begin{entry}{预祝}{10,9}{⾴、⽰}
  \begin{phonetics}{预祝}{yu4zhu4}
    \definition{v.}{parabenizar de antemão | oferecer os melhores votos para}
  \end{phonetics}
\end{entry}

\begin{entry}{预览}{10,9}{⾴、⾒}
  \begin{phonetics}{预览}{yu4lan3}
    \definition{s.}{visualização}
    \definition{v.}{visualizar}
  \end{phonetics}
\end{entry}

\begin{entry}{预留}{10,10}{⾴、⽥}
  \begin{phonetics}{预留}{yu4liu2}
    \definition{v.}{separar | reservar}
  \end{phonetics}
\end{entry}

\begin{entry}{预配}{10,10}{⾴、⾣}
  \begin{phonetics}{预配}{yu4pei4}
    \definition{s.}{pré-alocado | pré-cabeado}
    \definition{v.}{pré-alocar | pré-cabear}
  \end{phonetics}
\end{entry}

\begin{entry}{预谋}{10,11}{⾴、⾔}
  \begin{phonetics}{预谋}{yu4mou2}
    \definition{adj.}{premeditado}
    \definition{v.}{planejar algo com antecedência (especialmente um crime)}
  \end{phonetics}
\end{entry}

\begin{entry}{预提}{10,12}{⾴、⼿}
  \begin{phonetics}{预提}{yu4ti2}
    \definition{s.}{retenção}
    \definition{v.}{reter (imposto)}
  \end{phonetics}
\end{entry}

\begin{entry}{预期}{10,12}{⾴、⽉}
  \begin{phonetics}{预期}{yu4qi1}[][HSK 5]
    \definition{v.}{esperar; antecipar; imaginar; antecipar com expectativa}
  \end{phonetics}
\end{entry}

\begin{entry}{预感}{10,13}{⾴、⼼}
  \begin{phonetics}{预感}{yu4gan3}
    \definition{s.}{premonição}
    \definition{v.}{ter uma premonição}
  \end{phonetics}
\end{entry}

\begin{entry}{预警}{10,19}{⾴、⾔}
  \begin{phonetics}{预警}{yu4jing3}
    \definition{s.}{aviso | aviso antecipado}
  \end{phonetics}
\end{entry}

\begin{entry}{领}{11}{⾴}
  \begin{phonetics}{领}{ling3}[][HSK 3]
    \definition{adj.}{territorial (sob jurisdição de; em posse de)}
    \definition{clas.}{para roupas, tapetes, telas, etc.}
    \definition{s.}{pescoço; gargalo | colarinho; faixa de pescoço | esboço; ponto principal; essência}
    \definition{v.}{encabeçar; liderar; conduzir | possuir; ser o possuidor de | receber; obter; conseguir | aceitar; tomar |entender; compreender | adotar}
  \end{phonetics}
\end{entry}

\begin{entry}{领先}{11,6}{⾴、⼉}
  \begin{phonetics}{领先}{ling3xian1}[][HSK 3]
    \definition{v.}{liderar; assumir a liderança; estar na liderança}
  \end{phonetics}
\end{entry}

\begin{entry}{领导}{11,6}{⾴、⼨}
  \begin{phonetics}{领导}{ling3dao3}[][HSK 3]
    \definition[个,位]{s.}{líder; liderança}
    \definition{v.}{liderar; exercer liderança}
  \end{phonetics}
\end{entry}

\begin{entry}{领带}{11,9}{⾴、⼱}
  \begin{phonetics}{领带}{ling3 dai4}[][HSK 5]
    \definition[条]{s.}{colar; gargantilha; gravata}
  \end{phonetics}
\end{entry}

\begin{entry}{领情}{11,11}{⾴、⼼}
  \begin{phonetics}{领情}{ling3qing2}
    \definition{v.+compl.}{sentir-se grato a alguém}
  \end{phonetics}
\end{entry}

\begin{entry}{颐和园}{13,8,7}{⾴、⼝、⼞}
  \begin{phonetics}{颐和园}{yi2he2yuan2}
    \definition*{s.}{Palácio de Verão}
  \end{phonetics}
\end{entry}

\begin{entry}{频道}{13,12}{⾴、⾡}
  \begin{phonetics}{频道}{pin2dao4}[][HSK 5]
    \definition[个]{s.}{canal; canal de frequência; televisão e rádio, os sinais de som e imagem ocupam um determinado canal de frequência}
  \end{phonetics}
\end{entry}

\begin{entry}{频繁}{13,17}{⾴、⽷}
  \begin{phonetics}{频繁}{pin2fan2}[][HSK 5]
    \definition{adj.}{frequentemente}
    \definition{adj.}{frequente}
  \end{phonetics}
\end{entry}

\begin{entry}{颗}{14}{⾴}
  \begin{phonetics}{颗}{ke1}[][HSK 5]
    \definition{clas.}{para grãos, pérolas, dentes, corações, satelites, pequenas esferas, etc.}
    \definition{s.}{grão; partícula; pequenas coisas redondas}
  \end{phonetics}
\end{entry}

\begin{entry}{题}{15}{⾴}
  \begin{phonetics}{题}{ti2}[][HSK 2]
    \definition*{s.}{sobrenome Ti}
    \definition[个,道]{s.}{tópico; título; assunto; problema; frases que indicam o conteúdo de poemas ou discursos | questão; questões que devem ser respondidas durante os exercícios ou exames | antigamente, referia-se à testa}
    \definition{v.}{inscrever; escrever; assinar}
  \end{phonetics}
\end{entry}

\begin{entry}{题目}{15,5}{⾴、⽬}
  \begin{phonetics}{题目}{ti2mu4}[][HSK 3]
    \definition[道,个]{s.}{título; assunto; tópico | quebra-cabeça; problema de exercício; questão de exame}
  \end{phonetics}
\end{entry}

\begin{entry}{题材}{15,7}{⾴、⽊}
  \begin{phonetics}{题材}{ti2cai2}[][HSK 5]
    \definition{s.}{tema; assunto; material que compõe as obras literárias e artísticas, ou seja, os eventos ou fenômenos da vida descritos concretamente nas obras}
  \end{phonetics}
\end{entry}

\begin{entry}{颜}{15}{⾴}
  \begin{phonetics}{颜}{yan2}
    \definition*{s.}{sobrenome Yan}
    \definition{s.}{cor | face | semblante}
  \end{phonetics}
\end{entry}

\begin{entry}{颜色}{15,6}{⾴、⾊}
  \begin{phonetics}{颜色}{yan2 se4}[][HSK 2]
    \definition{s.}{cor | pigmento | tintura}
  \end{phonetics}
\end{entry}

%%%%% EOF %%%%%

