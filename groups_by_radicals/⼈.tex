%%%
%%% Radical "⼈"
%%%

\section*{Radical 9: ``⼈'' (亻、𠆢)}\addcontentsline{toc}{section}{Radical 9: ⼈、亻、𠆢}

\begin{entry}{人}{2}{⼈}[Kangxi 9]
  \begin{phonetics}{人}{ren2}[][HSK 1]
    \definition[个,位]{s.}{pessoa | gente}
  \end{phonetics}
\end{entry}

\begin{entry}{人口}{2,3}{⼈、⼝}
  \begin{phonetics}{人口}{ren2kou3}[][HSK 2]
    \definition{s.}{pessoas | população}
  \end{phonetics}
\end{entry}

\begin{entry}{人工}{2,3}{⼈、⼯}
  \begin{phonetics}{人工}{ren2gong1}[][HSK 3]
    \definition{adj.}{feito pelo homem; artificial}
    \definition[个]{s.}{trabalho manual; trabalho feito à mão | mão de obra; homem-dia; uma unidade de cálculo da quantidade de trabalho realizado}
  \end{phonetics}
\end{entry}

\begin{entry}{人才}{2,3}{⼈、⼿}
  \begin{phonetics}{人才}{ren2cai2}[][HSK 3]
    \definition{adj.}{aparência bonita, elegante}
    \definition[个]{s.}{talento; pessoal qualificado; pessoa com capacidade}
  \end{phonetics}
\end{entry}

\begin{entry}{人们}{2,5}{⼈、⼈}
  \begin{phonetics}{人们}{ren2 men5}[][HSK 2]
    \definition{s.}{homens |  pessoas | o público}
  \end{phonetics}
\end{entry}

\begin{entry}{人民}{2,5}{⼈、⽒}
  \begin{phonetics}{人民}{ren2 min2}[][HSK 3]
    \definition[群,批,个]{s.}{o povo}
  \end{phonetics}
\end{entry}

\begin{entry}{人民币}{2,5,4}{⼈、⽒、⼱}
  \begin{phonetics}{人民币}{ren2min2bi4}[][HSK 3]
    \definition*[块,张,元]{s.}{Renminbi (RMB); Yuan Chinês (CYN); nome da moeda chinesa}
  \end{phonetics}
\end{entry}

\begin{entry}{人生}{2,5}{⼈、⽣}
  \begin{phonetics}{人生}{ren2sheng1}[][HSK 3]
    \definition{s.}{vida (tempo de alguém na Terra)}
  \end{phonetics}
\end{entry}

\begin{entry}{人权}{2,6}{⼈、⽊}
  \begin{phonetics}{人权}{ren2quan2}
    \definition*{s.}{Direitos Humanos}
  \seealsoref{人权法}{ren2quan2fa3}
  \end{phonetics}
\end{entry}

\begin{entry}{人权法}{2,6,8}{⼈、⽊、⽔}
  \begin{phonetics}{人权法}{ren2quan2fa3}
    \definition*{s.}{Direitos Humanos}
  \seealsoref{人权}{ren2quan2}
  \end{phonetics}
\end{entry}

\begin{entry}{人行道}{2,6,12}{⼈、⾏、⾡}
  \begin{phonetics}{人行道}{ren2xing2dao4}
    \definition{s.}{calçada}
  \end{phonetics}
\end{entry}

\begin{entry}{人员}{2,7}{⼈、⼝}
  \begin{phonetics}{人员}{ren2yuan2}[][HSK 3]
    \definition[个,位,名]{s.}{funcionários | pessoal}
  \end{phonetics}
\end{entry}

\begin{entry}{人材}{2,7}{⼈、⽊}
  \begin{phonetics}{人材}{ren2cai2}
    \variantof{人才}
  \end{phonetics}
\end{entry}

\begin{entry}{人间}{2,7}{⼈、⾨}
  \begin{phonetics}{人间}{ren2jian1}
    \definition{s.}{o mundo humano | a Terra}
  \end{phonetics}
\end{entry}

\begin{entry}{人鱼}{2,8}{⼈、⿂}
  \begin{phonetics}{人鱼}{ren2yu2}
    \definition{s.}{sereia | peixe-boi | salamandra gigante}
  \end{phonetics}
\end{entry}

\begin{entry}{人类}{2,9}{⼈、⽶}
  \begin{phonetics}{人类}{ren2lei4}[][HSK 3]
    \definition[种]{s.}{humano; humanidade; raça humana}
  \end{phonetics}
\end{entry}

\begin{entry}{人家}{2,10}{⼈、⼧}
  \begin{phonetics}{人家}{ren2jia1}[][HSK 4]
    \definition[对]{s.}{lar; família; família do noivo; casa do futuro marido}
  \end{phonetics}
  \begin{phonetics}{人家}{ren2jia5}
    \definition{pron.}{outros; uma pessoa ou pessoas diferentes do falante ou ouvinte; refere-se a alguém diferente de si mesmo ou de outra pessoa | certa pessoa ou pessoas (a pessoa ou pessoas mencionadas em um contexto próximo, aproximadamente equivalente ao pronome de terceira pessoa);  refere-se a uma pessoa ou algumas pessoas, com significado semelhante a ``他'' | eu; mim (usado retoricamente no lugar do primeiro pronome pessoal, muitas vezes expressando descontentamento de forma jocosa; geralmente usado quando se fala com pessoas próximas, para significar ``自己'', usado principamente por meninas)}
  \seealsoref{他}{ta1}
  \seealsoref{自己}{zi4ji3}
  \end{phonetics}
\end{entry}

\begin{entry}{人海}{2,10}{⼈、⽔}
  \begin{phonetics}{人海}{ren2hai3}
    \definition{s.}{uma multidão | um mar de pessoas}
  \end{phonetics}
\end{entry}

\begin{entry}{人道}{2,12}{⼈、⾡}
  \begin{phonetics}{人道}{ren2dao4}
    \definition{s.}{solidariedade humana | humanitarismo | humano | a ``maneira humana'', um dos estágios do ciclo de reencarnação (budismo) | relação sexual}
  \end{phonetics}
\end{entry}

\begin{entry}{人像}{2,13}{⼈、⼈}
  \begin{phonetics}{人像}{ren2xiang4}
    \definition{s.}{``retrato'' de uma pessoa (esboço, foto, escultura, etc.)}
  \end{phonetics}
\end{entry}

\begin{entry}{人数}{2,13}{⼈、⽁}
  \begin{phonetics}{人数}{ren2 shu4}[][HSK 2]
    \definition{s.}{número de pessoas}
  \end{phonetics}
\end{entry}

\begin{entry}{人群}{2,13}{⼈、⽺}
  \begin{phonetics}{人群}{ren2 qun2}[][HSK 3]
    \definition{s.}{multidão; ajuntamento; torpel; aglomeração; um grupo de pessoas}
  \end{phonetics}
\end{entry}

\begin{entry}{个}{3}{⼈}
  \begin{phonetics}{个}{ge3}
    \definition{pron.}{usado em 自个儿}
    \seeref{自个儿}{zi4ge3r5}
  \end{phonetics}
  \begin{phonetics}{个}{ge4}[][HSK 1]
    \definition{clas.}{para objetos e pessoas em geral}
    \definition{pron.}{isto | aquilo}
    \definition{s.}{indivíduo | tamanho}
  \end{phonetics}
\end{entry}

\begin{entry}{个人}{3,2}{⼈、⼈}
  \begin{phonetics}{个人}{ge4ren2}[][HSK 3]
    \definition{pron.}{pessoal; si mesmo}
    \definition[个]{s.}{indivíduo}
  \end{phonetics}
\end{entry}

\begin{entry}{个儿}{3,2}{⼈、⼉}
  \begin{phonetics}{个儿}{ge4r5}[][HSK 5]
    \definition{s.}{tamanho; altura; estatura; tamanho do corpo ou do objeto |
pessoas ou coisas consideradas isoladamente; referir-se a uma pessoa ou coisa individualmente}
  \end{phonetics}
\end{entry}

\begin{entry}{个子}{3,3}{⼈、⼦}
  \begin{phonetics}{个子}{ge4zi5}[][HSK 2]
    \definition{s.}{altura | estatura}
  \end{phonetics}
\end{entry}

\begin{entry}{个体}{3,7}{⼈、⼈}
  \begin{phonetics}{个体}{ge4ti3}[][HSK 4]
    \definition{s.}{pessoa ou organismo individual}
  \end{phonetics}
\end{entry}

\begin{entry}{个别}{3,7}{⼈、⼑}
  \begin{phonetics}{个别}{ge4bie2}[][HSK 4]
    \definition{adj.}{muito poucos; excepcionais}
    \definition{adv.}{separadamente; individualmente; isoladamente}
  \end{phonetics}
\end{entry}

\begin{entry}{个性}{3,8}{⼈、⼼}
  \begin{phonetics}{个性}{ge4xing4}[][HSK 3]
    \definition{s.}{caráter individual; individualidade; personalidade}
  \end{phonetics}
\end{entry}

\begin{entry}{亿}{3}{⼈}
  \begin{phonetics}{亿}{yi4}[][HSK 2]
    \definition{num.}{cem milhões; 100.000.000; 1.0000.0000}
  \end{phonetics}
\end{entry}

\begin{entry}{什么}{4,3}{⼈、⼃}
  \begin{phonetics}{什么}{shen2me5}[][HSK 1]
    \definition{pron.}{que? | o que?}
    \definition{pron.}{algo | qualquer coisa}
  \end{phonetics}
\end{entry}

\begin{entry}{什么时候}{4,3,7,10}{⼈、⼃、⽇、⼈}
  \begin{phonetics}{什么时候}{shen2me5shi2hou5}
    \definition{adv.}{quando? | a que horas?}
  \end{phonetics}
\end{entry}

\begin{entry}{什么样}{4,3,10}{⼈、⼃、⽊}
  \begin{phonetics}{什么样}{shen2 me5 yang4}[][HSK 2]
    \definition{pron.}{que tipo? | o quê? | que tipo?}
  \end{phonetics}
\end{entry}

\begin{entry}{仅}{4}{⼈}
  \begin{phonetics}{仅}{jin3}[][HSK 3]
    \definition{adv.}{somente; meramente; por muito pouco}
  \end{phonetics}
\end{entry}

\begin{entry}{仅仅}{4,4}{⼈、⼈}
  \begin{phonetics}{仅仅}{jin3 jin3}[][HSK 3]
    \definition{adv.}{somente; meramente; por muito pouco}
  \end{phonetics}
\end{entry}

\begin{entry}{今天}{4,4}{⼈、⼤}
  \begin{phonetics}{今天}{jin1tian1}[][HSK 1]
    \definition{adv.}{hoje | no presente | agora}
  \end{phonetics}
\end{entry}

\begin{entry}{今后}{4,6}{⼈、⼝}
  \begin{phonetics}{今后}{jin1 hou4}[][HSK 2]
    \definition{s.}{de agora em diante | daqui em diante | no futuro}
  \end{phonetics}
\end{entry}

\begin{entry}{今年}{4,6}{⼈、⼲}
  \begin{phonetics}{今年}{jin1 nian2}[][HSK 1]
    \definition{adv.}{este ano}
  \end{phonetics}
\end{entry}

\begin{entry}{介绍}{4,8}{⼈、⽷}
  \begin{phonetics}{介绍}{jie4shao4}[][HSK 1]
    \definition{s.}{introdução | apresentação}
    \definition{v.}{fazer uma apresentação | apresentar (alguém para alguém) | apresentar (alguém para um emprego, etc.)}
  \end{phonetics}
\end{entry}

\begin{entry}{仍}{4}{⼈}
  \begin{phonetics}{仍}{reng2}[][HSK 3]
    \definition*{s.}{sobrenome Reng}
    \definition{adv.}{ainda}
    \definition{conj.}{por isso}
    \definition{v.}{permanecer}
  \end{phonetics}
\end{entry}

\begin{entry}{仍然}{4,12}{⼈、⽕}
  \begin{phonetics}{仍然}{reng2ran2}[][HSK 3]
    \definition{adv.}{ainda; como antes}
  \end{phonetics}
\end{entry}

\begin{entry}{从}{4}{⼈}
  \begin{phonetics}{从}{cong2}[][HSK 1]
    \definition*{s.}{sobrenome Cong}
    \definition{prep.}{de | desde | a partir de}
  \end{phonetics}
\end{entry}

\begin{entry}{从小}{4,3}{⼈、⼩}
  \begin{phonetics}{从小}{cong2 xiao3}[][HSK 2]
    \definition{adv.}{desde a infância | desde muito jovem | quando criança}
  \end{phonetics}
\end{entry}

\begin{entry}{从不}{4,4}{⼈、⼀}
  \begin{phonetics}{从不}{cong2bu4}
    \definition{adv.}{nunca}
  \end{phonetics}
\end{entry}

\begin{entry}{从中}{4,4}{⼈、⼁}
  \begin{phonetics}{从中}{cong2 zhong1}[][HSK 5]
    \definition{adv.}{de; dentre; daí}
  \end{phonetics}
\end{entry}

\begin{entry}{从未}{4,5}{⼈、⽊}
  \begin{phonetics}{从未}{cong2wei4}
    \definition{adv.}{nunca}
  \end{phonetics}
\end{entry}

\begin{entry}{从此}{4,6}{⼈、⽌}
  \begin{phonetics}{从此}{cong2ci3}[][HSK 4]
    \definition{conj.}{doravante; portanto; a partir deste momento; de agora em diante; a partir de então}
  \end{phonetics}
\end{entry}

\begin{entry}{从而}{4,6}{⼈、⽽}
  \begin{phonetics}{从而}{cong2'er2}[][HSK 5]
    \definition{conj.}{assim; por isso; portanto; desse modo; por esse motivo; conjunção usada no início do texto seguinte para expressar o resultado, propósito ou ação posterior, o que é equivalente a ``因此就''}
  \seealsoref{因此就}{yin1ci3 jiu4}
  \end{phonetics}
\end{entry}

\begin{entry}{从来}{4,7}{⼈、⽊}
  \begin{phonetics}{从来}{cong2lai2}[][HSK 3]
    \definition{adv.}{sempre; o tempo todo; em todos os momentos}
  \end{phonetics}
\end{entry}

\begin{entry}{从事}{4,8}{⼈、⼅}
  \begin{phonetics}{从事}{cong2shi4}[][HSK 3]
    \definition{v.}{trabalhar; empreender; empenhar-se em; envolver-se em | lidar com; manusear}
  \end{phonetics}
\end{entry}

\begin{entry}{从前}{4,9}{⼈、⼑}
  \begin{phonetics}{从前}{cong2qian2}[][HSK 3]
    \definition{s.}{antes; antigamente; no passado | era uma vez; há muito tempo atrás}
  \end{phonetics}
\end{entry}

\begin{entry}{以上}{4,3}{⼈、⼀}
  \begin{phonetics}{以上}{yi3 shang4}[][HSK 2]
    \definition{s.}{mais que | sobre | acima | o acima | o precedente | o acima mencionado}
  \end{phonetics}
\end{entry}

\begin{entry}{以下}{4,3}{⼈、⼀}
  \begin{phonetics}{以下}{yi3 xia4}[][HSK 2]
    \definition[所]{s.}{abaixo | sob | seguinte}
  \end{phonetics}
\end{entry}

\begin{entry}{以及}{4,3}{⼈、⼃}
  \begin{phonetics}{以及}{yi3ji2}[][HSK 4]
    \definition{conj.}{assim como; juntamente como; bem como; também}
  \end{phonetics}
\end{entry}

\begin{entry}{以为}{4,4}{⼈、⼂}
  \begin{phonetics}{以为}{yi3wei2}[][HSK 2]
    \definition{v.}{pensar, ou seja, considerar que\dots (geralmente há uma implicação de que a noção está errada --- exceto ao expressar a própria opnião atual)}
  \end{phonetics}
\end{entry}

\begin{entry}{以内}{4,4}{⼈、⼌}
  \begin{phonetics}{以内}{yi3 nei4}[][HSK 4]
    \definition{adv.}{dentro de; menos que; não mais que; dentro de certos limites de tempo, premissas, quantidade e escopo}
  \end{phonetics}
\end{entry}

\begin{entry}{以外}{4,5}{⼈、⼣}
  \begin{phonetics}{以外}{yi3 wai4}[][HSK 2]
    \definition{s.}{além | exceto | fora | diferente de}
  \end{phonetics}
\end{entry}

\begin{entry}{以后}{4,6}{⼈、⼝}
  \begin{phonetics}{以后}{yi3 hou4}[][HSK 2]
    \definition{adv.}{depois de | depois | após}
  \end{phonetics}
\end{entry}

\begin{entry}{以此}{4,6}{⼈、⽌}
  \begin{phonetics}{以此}{yi3ci3}
    \definition{adv.}{devido a esta | deste modo | por isso | com isso}
  \end{phonetics}
\end{entry}

\begin{entry}{以至}{4,6}{⼈、⾄}
  \begin{phonetics}{以至}{yi3zhi4}
    \definition{adv.}{até}
    \definition{conj.}{a tal ponto que\dots}
  \seealsoref{以至于}{yi3zhi4yu2}
  \end{phonetics}
\end{entry}

\begin{entry}{以至于}{4,6,3}{⼈、⾄、⼆}
  \begin{phonetics}{以至于}{yi3zhi4yu2}
    \definition{adv.}{até}
    \definition{conj.}{na medida em que\dots}
  \seealsoref{以至}{yi3zhi4}
  \end{phonetics}
\end{entry}

\begin{entry}{以色列}{4,6,6}{⼈、⾊、⼑}
  \begin{phonetics}{以色列}{yi3se4lie4}
    \definition*{s.}{Israel}
  \end{phonetics}
\end{entry}

\begin{entry}{以免}{4,7}{⼈、⼉}
  \begin{phonetics}{以免}{yi3mian3}
    \definition{conj.}{para evitar isso}
  \end{phonetics}
\end{entry}

\begin{entry}{以来}{4,7}{⼈、⽊}
  \begin{phonetics}{以来}{yi3lai2}[][HSK 3]
    \definition{prep.}{desde (um evento anterior); indica um período de um certo tempo no passado até o presente}
  \end{phonetics}
\end{entry}

\begin{entry}{以求}{4,7}{⼈、⽔}
  \begin{phonetics}{以求}{yi3qiu2}
    \definition{conj.}{a fim de}
  \end{phonetics}
\end{entry}

\begin{entry}{以便}{4,9}{⼈、⼈}
  \begin{phonetics}{以便}{yi3bian4}
    \definition{conj.}{a fim de | para que | assim como}
  \end{phonetics}
\end{entry}

\begin{entry}{以前}{4,9}{⼈、⼑}
  \begin{phonetics}{以前}{yi3qian2}[][HSK 2]
    \definition{adv.}{antes de | antes}
  \end{phonetics}
\end{entry}

\begin{entry}{以期}{4,12}{⼈、⽉}
  \begin{phonetics}{以期}{yi3qi1}
    \definition{v.}{tentando | esperando | esperando por}
  \end{phonetics}
\end{entry}

\begin{entry}{他}{5}{⼈}
  \begin{phonetics}{他}{ta1}[][HSK 1]
    \definition{pron.}{ele | se, o, lhe | si, consigo, ele}
    \seeref{怹}{tan1}
  \end{phonetics}
\end{entry}

\begin{entry}{他们}{5,5}{⼈、⼈}
  \begin{phonetics}{他们}{ta1men5}[][HSK 1]
    \definition{pron.}{eles | se, os, lhes | si, consigo, eles}
  \end{phonetics}
\end{entry}

\begin{entry}{他们的}{5,5,8}{⼈、⼈、⽩}
  \begin{phonetics}{他们的}{ta1men5 de5}
    \definition{pron.}{deles}
  \end{phonetics}
\end{entry}

\begin{entry}{他妈的}{5,6,8}{⼈、⼥、⽩}
  \begin{phonetics}{他妈的}{ta1ma1de5}
    \definition{interj.}{Dane-se! | Foda-se!}
  \end{phonetics}
\end{entry}

\begin{entry}{他的}{5,8}{⼈、⽩}
  \begin{phonetics}{他的}{ta1 de5}
    \definition{pron.}{dele}
  \end{phonetics}
\end{entry}

\begin{entry}{付}{5}{⼈}
  \begin{phonetics}{付}{fu4}[][HSK 3]
    \definition*{s.}{sobrenome Fu}
    \definition{clas.}{para pares ou conjuntos de coisas | para expressões faciais}
    \definition{v.}{comprometer-se a; entregar (entregar) a; entregar | pagar}
  \end{phonetics}
\end{entry}

\begin{entry}{付出}{5,5}{⼈、⼐}
  \begin{phonetics}{付出}{fu4 chu1}[][HSK 4]
    \definition{v.}{pagar; gastar; entregar (dinheiro, consideração, etc.)}
  \end{phonetics}
\end{entry}

\begin{entry}{付款}{5,12}{⼈、⽋}
  \begin{phonetics}{付款}{fu4kuan3}
    \definition{s.}{pagamento}
    \definition{v.+compl.}{pagar uma quantia em dinheiro}
  \end{phonetics}
\end{entry}

\begin{entry}{仙}{5}{⼈}
  \begin{phonetics}{仙}{xian1}
    \definition{s.}{imortal}
  \end{phonetics}
\end{entry}

\begin{entry}{代}{5}{⼈}
  \begin{phonetics}{代}{dai4}[][HSK 3]
    \definition*{s.}{sobrenome Dai}
    \definition{s.}{período histórico | dinastia | geração | era}
    \definition{v.}{tomar o lugar de; estar no lugar de
agir em nome de; exercer}
  \end{phonetics}
\end{entry}

\begin{entry}{代价}{5,6}{⼈、⼈}
  \begin{phonetics}{代价}{dai4jia4}[][HSK 5]
    \definition[种,个]{s.}{preço; material, energia gasta ou sacrifícios feitos para atingir um objetivo | custo; preço; dinheiro pago para obter algo}
  \end{phonetics}
\end{entry}

\begin{entry}{代言}{5,7}{⼈、⾔}
  \begin{phonetics}{代言}{dai4yan2}
    \definition{v.}{ser um porta-voz | ser um embaixador (para uma marca) | endossar}
  \end{phonetics}
\end{entry}

\begin{entry}{代表}{5,8}{⼈、⾐}
  \begin{phonetics}{代表}{dai4biao3}[][HSK 3]
    \definition[位,个,名]{s.}{deputado; delegado; representante | representante oficial}
    \definition{v.}{representar; defender}
  \end{phonetics}
\end{entry}

\begin{entry}{代表团}{5,8,6}{⼈、⾐、⼞}
  \begin{phonetics}{代表团}{dai4 biao3 tuan2}[][HSK 3]
    \definition[个]{s.}{delegação; contingente}
  \end{phonetics}
\end{entry}

\begin{entry}{代称}{5,10}{⼈、⽲}
  \begin{phonetics}{代称}{dai4cheng1}
    \definition{s.}{nome alternativo | antonomásia}
    \definition{v.}{referir-se a algo ou alguém por outro nome}
  \end{phonetics}
\end{entry}

\begin{entry}{代理}{5,11}{⼈、⽟}
  \begin{phonetics}{代理}{dai4li3}[][HSK 5]
    \definition{v.}{agir em nome de alguém em uma posição de responsabilidade; substituir alguém | agir como procurador; agir como agente; ser encarregado pelas partes de realizar atividades e conduzir assuntos em seu nome dentro do escopo de sua autorização}
  \end{phonetics}
\end{entry}

\begin{entry}{代替}{5,12}{⼈、⽈}
  \begin{phonetics}{代替}{dai4ti4}[][HSK 4]
    \definition{v.}{substituir; substituir por; tomar o lugar de}
  \end{phonetics}
\end{entry}

\begin{entry}{令人}{5,2}{⼈、⼈}
  \begin{phonetics}{令人}{ling4ren2}
    \definition{v.}{causar alguém (a fazer alguma coisa) | fazer alguém ficar zangado, encantado, etc.}
  \end{phonetics}
\end{entry}

\begin{entry}{仪式}{5,6}{⼈、⼷}
  \begin{phonetics}{仪式}{yi2shi4}
    \definition{s.}{cerimônia}
  \end{phonetics}
\end{entry}

\begin{entry}{们}{5}{⼈}
  \begin{phonetics}{们}{men5}[][HSK 1]
    \definition{part.}{sufixo para plural de pronomes e substantivos referentes a indivíduos}
  \end{phonetics}
\end{entry}

\begin{entry}{件}{6}{⼈}
  \begin{phonetics}{件}{jian4}[][HSK 2]
    \definition{clas.}{para eventos, coisas, roupas etc.}
    \definition{s.}{item | componente}
  \end{phonetics}
\end{entry}

\begin{entry}{价值}{6,10}{⼈、⼈}
  \begin{phonetics}{价值}{jia4zhi2}[][HSK 3]
    \definition{s.}{valor}
  \end{phonetics}
\end{entry}

\begin{entry}{价格}{6,10}{⼈、⽊}
  \begin{phonetics}{价格}{jia4ge2}[][HSK 3]
    \definition[个]{s.}{preço; tarifa}
  \end{phonetics}
\end{entry}

\begin{entry}{价钱}{6,10}{⼈、⾦}
  \begin{phonetics}{价钱}{jia4 qian2}[][HSK 3]
    \definition[些]{s.}{preço}
  \end{phonetics}
\end{entry}

\begin{entry}{任}{6}{⼈}
  \begin{phonetics}{任}{ren4}[][HSK 3]
    \definition{clas.}{número de vezes que serviu em uma posição}
    \definition{conj.}{não importa (como, o que, etc.)}
    \definition{s.}{correio oficial; escritório}
    \definition{v.}{nomear; designar | assumir um posto; assumir um emprego | deixar; permitir; dar rédea solta a}
  \end{phonetics}
\end{entry}

\begin{entry}{任务}{6,5}{⼈、⼒}
  \begin{phonetics}{任务}{ren4wu5}[][HSK 3]
    \definition[项,个]{s.}{tarefa; dever; missão; designação}
  \end{phonetics}
\end{entry}

\begin{entry}{任何}{6,7}{⼈、⼈}
  \begin{phonetics}{任何}{ren4he2}[][HSK 3]
    \definition{pron.}{qualquer; qualquer que seja; o que for}
  \end{phonetics}
\end{entry}

\begin{entry}{份}{6}{⼈}
  \begin{phonetics}{份}{fen4}[][HSK 2]
    \definition{clas.}{para presentes, jornais, revistas, papéis, relatórios, contratos, etc. ou pratos (refeição)}
  \end{phonetics}
\end{entry}

\begin{entry}{企业}{6,5}{⼈、⼀}
  \begin{phonetics}{企业}{qi3ye4}[][HSK 4]
    \definition[家,个]{s.}{empresa; estabelecimento; empreendimento; negócio; setores envolvidos em atividades econômicas como produção, transporte, comércio, etc., como fábricas, minas, ferrovias, empresas comerciais, etc.}
  \end{phonetics}
\end{entry}

\begin{entry}{伊马姆}{6,3,8}{⼈、⾺、⼥}
  \begin{phonetics}{伊马姆}{yi1ma3mu3}
    \definition*{s.}{Islã}
  \seealsoref{伊玛目}{yi1ma3mu4}
  \seealsoref{伊曼}{yi1man4}
  \seealsoref{伊斯兰}{yi1si1lan2}
  \end{phonetics}
\end{entry}

\begin{entry}{伊玛目}{6,7,5}{⼈、⽟、⽬}
  \begin{phonetics}{伊玛目}{yi1ma3mu4}
    \definition*{s.}{Islã}
  \seealsoref{伊马姆}{yi1ma3mu3}
  \seealsoref{伊曼}{yi1man4}
  \seealsoref{伊斯兰}{yi1si1lan2}
  \end{phonetics}
\end{entry}

\begin{entry}{伊朗}{6,10}{⼈、⽉}
  \begin{phonetics}{伊朗}{yi1lang3}
    \definition*{s.}{Irã}
  \end{phonetics}
\end{entry}

\begin{entry}{伊曼}{6,11}{⼈、⽈}
  \begin{phonetics}{伊曼}{yi1man4}
    \definition*{s.}{Islã}
  \seealsoref{伊马姆}{yi1ma3mu3}
  \seealsoref{伊玛目}{yi1ma3mu4}
  \seealsoref{伊斯兰}{yi1si1lan2}
  \end{phonetics}
\end{entry}

\begin{entry}{伊斯兰}{6,12,5}{⼈、⽄、⼋}
  \begin{phonetics}{伊斯兰}{yi1si1lan2}
    \definition*{s.}{Islã}
  \seealsoref{伊马姆}{yi1ma3mu3}
  \seealsoref{伊玛目}{yi1ma3mu4}
  \seealsoref{伊曼}{yi1man4}
  \end{phonetics}
\end{entry}

\begin{entry}{休兵}{6,7}{⼈、⼋}
  \begin{phonetics}{休兵}{xiu1bing1}
    \definition{s.}{armistício}
    \definition{v.}{cessar fogo}
  \end{phonetics}
\end{entry}

\begin{entry}{休闲}{6,7}{⼈、⾨}
  \begin{phonetics}{休闲}{xiu1xian2}
    \definition{s.}{ócio | lazer}
    \definition{v.}{desfrutar do lazer}
  \end{phonetics}
\end{entry}

\begin{entry}{休息}{6,10}{⼈、⼼}
  \begin{phonetics}{休息}{xiu1xi5}[][HSK 1]
    \definition{s.}{descanço}
    \definition{v.}{descansar}
  \end{phonetics}
\end{entry}

\begin{entry}{休息室}{6,10,9}{⼈、⼼、⼧}
  \begin{phonetics}{休息室}{xiu1xi1shi4}
    \definition{s.}{saguão | salão}
  \end{phonetics}
\end{entry}

\begin{entry}{休假}{6,11}{⼈、⼈}
  \begin{phonetics}{休假}{xiu1 jia4}[][HSK 2]
    \definition{v.+compl.}{ter um feriado | tirar férias | sair de férias}
  \end{phonetics}
\end{entry}

\begin{entry}{休憩}{6,16}{⼈、⼼}
  \begin{phonetics}{休憩}{xiu1qi4}
    \definition{v.}{relaxar | descansar | dar um tempo}
  \end{phonetics}
\end{entry}

\begin{entry}{休整}{6,16}{⼈、⽁}
  \begin{phonetics}{休整}{xiu1zheng3}
    \definition{v.}{(militar) descansar e reorganizar}
  \end{phonetics}
\end{entry}

\begin{entry}{众}{6}{⼈}
  \begin{phonetics}{众}{zhong4}
    \definition*{s.}{Câmara dos Deputados, abreviação de 众议院}
    \definition{adj.}{numeroso}
    \definition{adv.}{muitos}
    \definition{s.}{multidão}
    \seeref{众议院}{zhong4yi4yuan4}
  \end{phonetics}
\end{entry}

\begin{entry}{众议院}{6,5,9}{⼈、⾔、⾩}
  \begin{phonetics}{众议院}{zhong4yi4yuan4}
    \definition*{s.}{Casa baixa da Assembléia Bicameral | Câmara dos Deputados}
  \end{phonetics}
\end{entry}

\begin{entry}{优}{6}{⼈}
  \begin{phonetics}{优}{you1}
    \definition{adj.}{excelente | superior}
  \end{phonetics}
\end{entry}

\begin{entry}{优于}{6,3}{⼈、⼆}
  \begin{phonetics}{优于}{you1yu2}
    \definition{v.}{superar}
  \end{phonetics}
\end{entry}

\begin{entry}{优先}{6,6}{⼈、⼉}
  \begin{phonetics}{优先}{you1xian1}
    \definition{v.}{ter prioridade | ter precedência}
  \end{phonetics}
\end{entry}

\begin{entry}{优伶}{6,7}{⼈、⼈}
  \begin{phonetics}{优伶}{you1ling2}
    \definition{s.}{ator}
  \end{phonetics}
\end{entry}

\begin{entry}{优秀}{6,7}{⼈、⽲}
  \begin{phonetics}{优秀}{you1xiu4}[][HSK 4]
    \definition{adj.}{esplêndido; excelente; extraordinário; excepcional; notável; descreve moral, qualidades, realizações, aprendizado, etc. muito bons.}
  \end{phonetics}
\end{entry}

\begin{entry}{优良}{6,7}{⼈、⾉}
  \begin{phonetics}{优良}{you1 liang2}[][HSK 4]
    \definition{adj.}{ótimo; bom; excelente; (variedade, qualidade, desempenho, estilo, etc.) muito bom}
  \end{phonetics}
\end{entry}

\begin{entry}{优势}{6,8}{⼈、⼒}
  \begin{phonetics}{优势}{you1shi4}[][HSK 3]
    \definition[种,个]{s.}{vantagem; superioridade; preponderância; posição dominante}
  \end{phonetics}
\end{entry}

\begin{entry}{优质}{6,8}{⼈、⾙}
  \begin{phonetics}{优质}{you1zhi4}
    \definition{adj.}{excelente qualidade}
  \end{phonetics}
\end{entry}

\begin{entry}{优厚}{6,9}{⼈、⼚}
  \begin{phonetics}{优厚}{you1hou4}
    \definition{adj.}{generoso}
  \end{phonetics}
\end{entry}

\begin{entry}{优点}{6,9}{⼈、⽕}
  \begin{phonetics}{优点}{you1dian3}[][HSK 3]
    \definition[个,项,种,些]{s.}{mérito; virtude; vantagem; ponto forte}
  \end{phonetics}
\end{entry}

\begin{entry}{优美}{6,9}{⼈、⽺}
  \begin{phonetics}{优美}{you1mei3}[][HSK 4]
    \definition{adj.}{fino; elegante; gracioso; bonito}
  \end{phonetics}
\end{entry}

\begin{entry}{优选}{6,9}{⼈、⾡}
  \begin{phonetics}{优选}{you1xuan3}
    \definition{v.}{otimizar}
  \end{phonetics}
\end{entry}

\begin{entry}{优格}{6,10}{⼈、⽊}
  \begin{phonetics}{优格}{you1ge2}
    \definition{s.}{iogurte}
  \end{phonetics}
\end{entry}

\begin{entry}{优盘}{6,11}{⼈、⽫}
  \begin{phonetics}{优盘}{you1pan2}
    \definition{s.}{unidade de memória USB}
  \seealsoref{闪存盘}{shan3cun2pan2}
  \end{phonetics}
\end{entry}

\begin{entry}{优等}{6,12}{⼈、⽵}
  \begin{phonetics}{优等}{you1deng3}
    \definition{adj.}{excelente | de primeira linha | alta classe | da mais alta ordem, superior}
  \end{phonetics}
\end{entry}

\begin{entry}{优裕}{6,12}{⼈、⾐}
  \begin{phonetics}{优裕}{you1yu4}
    \definition{adj.}{abundante | bastante}
    \definition{s.}{abundância}
  \end{phonetics}
\end{entry}

\begin{entry}{伙}{6}{⼈}
  \begin{phonetics}{伙}{huo3}[][HSK 4]
    \definition{clas.}{grupo; multidão; banda}
    \definition{s.}{iguaria; alimentação; refeições | parceiro; companheiro | coletivo de colegas}
    \definition{v.}{combinar; unir}
  \end{phonetics}
\end{entry}

\begin{entry}{伙伴}{6,7}{⼈、⼈}
  \begin{phonetics}{伙伴}{huo3ban4}[][HSK 4]
    \definition[个,位,群]{s.}{parceiro; companheiro; antigo sistema militar de dez pessoas para uma fogueira, o chefe da fogueira, uma pessoa encarregada de cozinhar, com a fogueira é chamado de parceiro da fogueira, agora se refere à participação comum em uma determinada organização ou engajada em certas atividades}
  \end{phonetics}
\end{entry}

\begin{entry}{会}{6}{⼈}
  \begin{phonetics}{会}{hui4}[][HSK 1,2]
    \definition{adv.}{um momento}
    \definition{s.}{encontro | reunião}
    \definition{suf.}{união | grupo | associação}
    \definition{v.}{poder (ter a habilidade, saber como fazer) | saber | ter habilidade | saber como fazer | ser provável | ter certeza de | encontrar-se | reunir-se}
  \end{phonetics}
  \begin{phonetics}{会}{kuai4}
    \definition{s.}{contabilidade}
    \definition{v.}{equilibrar uma conta}
  \end{phonetics}
\end{entry}

\begin{entry}{会计}{6,4}{⼈、⾔}
  \begin{phonetics}{会计}{kuai4ji4}[][HSK 4]
    \definition[个,位,名]{s.}{contabilidade | contador; contabilista; guarda-livros; pessoal que trabalha como contador}
  \end{phonetics}
\end{entry}

\begin{entry}{会议}{6,5}{⼈、⾔}
  \begin{phonetics}{会议}{hui4yi4}[][HSK 3]
    \definition[场,届,个]{s.}{reunião; conferência | conselho; congresso}
  \end{phonetics}
\end{entry}

\begin{entry}{会员}{6,7}{⼈、⼝}
  \begin{phonetics}{会员}{hui4 yuan2}[][HSK 3]
    \definition[位]{s.}{membro; associado | filiação}
  \end{phonetics}
\end{entry}

\begin{entry}{会首}{6,9}{⼈、⾸}
  \begin{phonetics}{会首}{hui4shou3}
    \definition{s.}{chefe de uma sociedade | patrocinador de uma organização}
  \end{phonetics}
\end{entry}

\begin{entry}{伞}{6}{⼈}
  \begin{phonetics}{伞}{san3}[][HSK 4]
    \definition*{s.}{sobrenome San}
    \definition[把]{s.}{guarda-chuva; proteção contra chuva ou sol | algo que tem o formato de um guarda-chuva}
  \end{phonetics}
\end{entry}

\begin{entry}{伟}{6}{⼈}
  \begin{phonetics}{伟}{wei3}
    \definition{adj.}{grande | ótimo}
  \end{phonetics}
\end{entry}

\begin{entry}{伟大}{6,3}{⼈、⼤}
  \begin{phonetics}{伟大}{wei3da4}[][HSK 3]
    \definition{adj.}{ótimo; importante (contribuição, etc.) | ótimo; magnífico; digno da maior admiração}
  \end{phonetics}
\end{entry}

\begin{entry}{传}{6}{⼈}
  \begin{phonetics}{传}{chuan2}[][HSK 3]
    \definition{v.}{passar; passar adiante | passar adiante; legar; passar de \dots para \dots | transmitir (conhecimento, habilidade, etc.); comunicar; ensinar | espalhar; propagar | transmitir; conduzir; transferir | transmitir; expressar |convocar | infectar; ser contagioso}
  \end{phonetics}
  \begin{phonetics}{传}{zhuan4}
    \definition{s.}{comentários sobre clássicos | biografia | romances sobre eventos históricos}
  \end{phonetics}
\end{entry}

\begin{entry}{传达}{6,6}{⼈、⾡}
  \begin{phonetics}{传达}{chuan2da2}[][HSK 5]
    \definition{s.}{recepção e registro de chamadas em um estabelecimento público | zelador; recepcionista}
    \definition{v.}{passar adiante (informações, etc.); transmitir; retransmitir; comunicar}
  \end{phonetics}
\end{entry}

\begin{entry}{传来}{6,7}{⼈、⽊}
  \begin{phonetics}{传来}{chuan2 lai2}[][HSK 3]
    \definition{v.}{(um som) passar | (notícias) chegar}
  \end{phonetics}
\end{entry}

\begin{entry}{传承}{6,8}{⼈、⼿}
  \begin{phonetics}{传承}{chuan2cheng2}
    \definition{s.}{herança | tradição continuada}
    \definition{v.}{transmitir (para as gerações futuras) | passar adiante (desde os tempos antigos)}
  \end{phonetics}
\end{entry}

\begin{entry}{传给}{6,9}{⼈、⽷}
  \begin{phonetics}{传给}{chuan2gei3}
    \definition{v.}{passar para | transferir para | entregar a}
  \end{phonetics}
\end{entry}

\begin{entry}{传统}{6,9}{⼈、⽷}
  \begin{phonetics}{传统}{chuan2tong3}[][HSK 4]
    \definition{adj.}{tradicional; histórico; transmitido de geração em geração | antiquado, conservador e fora de sintonia com os tempos}
    \definition[个]{s.}{tradição; costume; fatores sociais, como costumes, moral, ideias, estilos, artes, instituições etc., que são transmitidos de uma geração para outra e que são característicos da sociedade}
  \end{phonetics}
\end{entry}

\begin{entry}{传说}{6,9}{⼈、⾔}
  \begin{phonetics}{传说}{chuan2shuo1}[][HSK 3]
    \definition{s.}{lenda | conto popular | folclore}
    \definition{v.}{dizer que; ser dito; passar de boca em boca}
  \end{phonetics}
\end{entry}

\begin{entry}{传真}{6,10}{⼈、⼗}
  \begin{phonetics}{传真}{chuan2zhen1}[][HSK 5]
    \definition[台,部,份]{s.}{\emph{FAX}, facsímile; texto, diagramas, fotografias, etc., transmitidos por aparelho de fax}
    \definition{v.}{enviar um fax}
  \end{phonetics}
\end{entry}

\begin{entry}{传递}{6,10}{⼈、⾡}
  \begin{phonetics}{传递}{chuan2 di4}[][HSK 5]
    \definition{v.}{transmitir; entregar; transferir; passar adiante}
  \end{phonetics}
\end{entry}

\begin{entry}{传播}{6,15}{⼈、⼿}
  \begin{phonetics}{传播}{chuan2bo1}[][HSK 3]
    \definition{v.}{espalhar; difundir; propagar; disseminar}
  \end{phonetics}
\end{entry}

\begin{entry}{伤}{6}{⼈}
  \begin{phonetics}{伤}{shang1}[][HSK 3]
    \definition*{s.}{sobrenome Shang}
    \definition{s.}{ferida; ferimento}
    \definition{v.}{ferir; machucar | estar angustiado | enjoar de algo; desenvolver aversão a algo. |ser prejudicial a; entravar}
  \end{phonetics}
\end{entry}

\begin{entry}{伤心}{6,4}{⼈、⼼}
  \begin{phonetics}{伤心}{shang1xin1}[][HSK 3]
    \definition{v.+compl.}{estar triste; lamentar; estar com o coração partido}
  \end{phonetics}
\end{entry}

\begin{entry}{伤害}{6,10}{⼈、⼧}
  \begin{phonetics}{伤害}{shang1hai4}[][HSK 4]
    \definition{v.}{ferir; prejudicar; machucar; magoar; causar danos físicos ou mentais}
  \end{phonetics}
\end{entry}

\begin{entry}{伦敦}{6,12}{⼈、⽁}
  \begin{phonetics}{伦敦}{lun2dun1}
    \definition*{s.}{Londres}
  \end{phonetics}
\end{entry}

\begin{entry}{似乎}{6,5}{⼈、⼃}
  \begin{phonetics}{似乎}{si4hu1}[][HSK 4]
    \definition{adv.}{como se; aparentemente; se parece como}
  \end{phonetics}
\end{entry}

\begin{entry}{似的}{6,8}{⼈、⽩}
  \begin{phonetics}{似的}{shi4de5}[][HSK 4]
    \definition{part.}{como; como\dots como; como se (embora); usada após uma palavra ou frase para indicar uma semelhança com algo ou uma situação | usada para indicar alto grau}
  \end{phonetics}
\end{entry}

\begin{entry}{似曾相识}{6,12,9,7}{⼈、⽈、⽬、⾔}
  \begin{phonetics}{似曾相识}{si4ceng2xiang1shi2}
    \definition{s.}{\emph{déjà vu} (a experiência de ver exatamente a mesma situação pela segunda vez) | situação aparentemente familiar}
  \end{phonetics}
\end{entry}

\begin{entry}{估计}{7,4}{⼈、⾔}
  \begin{phonetics}{估计}{gu1ji4}
    \definition{v.}{estimar | avaliar | calcular}
  \end{phonetics}
\end{entry}

\begin{entry}{伲}{7}{⼈}
  \begin{phonetics}{伲}{ni4}
    \definition{pron.}{(dialeto) eu | meu | nosso | nós}
    \seeref{你}{ni3}
  \end{phonetics}
\end{entry}

\begin{entry}{伴侣}{7,8}{⼈、⼈}
  \begin{phonetics}{伴侣}{ban4lv3}
    \definition{s.}{companheiro | parceiro}
  \end{phonetics}
\end{entry}

\begin{entry}{但}{7}{⼈}
  \begin{phonetics}{但}{dan4}[][HSK 2]
    \definition{conj.}{mas | ainda | no entanto | apenas}
  \end{phonetics}
\end{entry}

\begin{entry}{但是}{7,9}{⼈、⽇}
  \begin{phonetics}{但是}{dan4 shi4}[][HSK 2]
    \definition{conj.}{mas | ainda | no entanto}
  \end{phonetics}
\end{entry}

\begin{entry}{位}{7}{⼈}
  \begin{phonetics}{位}{wei4}[][HSK 2]
    \definition{clas.}{para pessoas (com cortesia) | para bits binários}[十六位 (16 bits)]
    \definition{s.}{(física) potencial | localização | lugar | posição | assento}
  \end{phonetics}
\end{entry}

\begin{entry}{位于}{7,3}{⼈、⼆}
  \begin{phonetics}{位于}{wei4yu2}[][HSK 4]
    \definition{v.}{estar localizado; estar situado}
  \end{phonetics}
\end{entry}

\begin{entry}{位子}{7,3}{⼈、⼦}
  \begin{phonetics}{位子}{wei4zi5}
    \definition{s.}{lugar | assento}
  \end{phonetics}
\end{entry}

\begin{entry}{位居}{7,8}{⼈、⼫}
  \begin{phonetics}{位居}{wei4ju1}
    \definition{v.}{estar localizado em}
  \end{phonetics}
\end{entry}

\begin{entry}{位置}{7,13}{⼈、⽹}
  \begin{phonetics}{位置}{wei4zhi4}[][HSK 4]
    \definition[通,个]{s.}{assento; lugar; localização | lugar; posição; \emph{status} | posição (por exemplo: cargo no escritório)}
  \end{phonetics}
\end{entry}

\begin{entry}{低}{7}{⼈}
  \begin{phonetics}{低}{di1}[][HSK 2]
    \definition{adj.}{baixo}
    \definition{adv.}{abaixo}
    \definition{v.}{abaixar (a cabeça) | deixar cair | pendurar | inclinar}
  \end{phonetics}
\end{entry}

\begin{entry}{低于}{7,3}{⼈、⼆}
  \begin{phonetics}{低于}{di1 yu2}[][HSK 5]
    \definition{v.}{ser inferior a; algo ou fenômeno é, de alguma forma, inferior ou pior do que outra coisa}
  \end{phonetics}
\end{entry}

\begin{entry}{低潮}{7,15}{⼈、⽔}
  \begin{phonetics}{低潮}{di1chao2}
    \definition{s.}{maré baixa/vazante; o nível mais baixo da maré durante um ciclo de maré (distinto da ``高潮'') | vazante baixa; o ponto mais baixo; uma metáfora para o baixo estágio de desenvolvimento das coisas}
  \seealsoref{高潮}{gao1chao2}
  \end{phonetics}
\end{entry}

\begin{entry}{住}{7}{⼈}
  \begin{phonetics}{住}{zhu4}[][HSK 1]
    \definition{v.}{habitar | residir | morar | alojar-se}
  \end{phonetics}
\end{entry}

\begin{entry}{住处}{7,5}{⼈、⼡}
  \begin{phonetics}{住处}{zhu4chu4}
    \definition{s.}{morada | habitação | residência}
  \end{phonetics}
\end{entry}

\begin{entry}{住宅}{7,6}{⼈、⼧}
  \begin{phonetics}{住宅}{zhu4zhai2}
    \definition{s.}{residência}
  \end{phonetics}
\end{entry}

\begin{entry}{住房}{7,8}{⼈、⼾}
  \begin{phonetics}{住房}{zhu4fang2}[][HSK 2]
    \definition{s.}{habitação}
  \end{phonetics}
\end{entry}

\begin{entry}{住所}{7,8}{⼈、⼾}
  \begin{phonetics}{住所}{zhu4suo3}
    \definition[处]{s.}{morada | habitação | residência}
  \end{phonetics}
\end{entry}

\begin{entry}{住院}{7,9}{⼈、⾩}
  \begin{phonetics}{住院}{zhu4 yuan4}[][HSK 2]
    \definition{v.}{estar hospitalizado | estar no hospital}
  \end{phonetics}
\end{entry}

\begin{entry}{住嘴}{7,16}{⼈、⼝}
  \begin{phonetics}{住嘴}{zhu4zui3}
    \definition{interj.}{Cale-se!}
    \definition{v.}{calar | calar-se}
  \end{phonetics}
\end{entry}

\begin{entry}{体内}{7,4}{⼈、⼌}
  \begin{phonetics}{体内}{ti3nei4}
    \definition{adj.}{dentro do corpo | \emph{in vivo} (versus \emph{in vitro} | interno a}
  \end{phonetics}
\end{entry}

\begin{entry}{体会}{7,6}{⼈、⼈}
  \begin{phonetics}{体会}{ti3hui4}[][HSK 3]
    \definition{s.}{conhecimento; compreensão; experiência pessoal}
    \definition{v.}{perceber; saber (ou aprender) com a experiência}
  \end{phonetics}
\end{entry}

\begin{entry}{体现}{7,8}{⼈、⾒}
  \begin{phonetics}{体现}{ti3xian4}[][HSK 3]
    \definition{v.}{refletir; incorporar; encarnar}
  \end{phonetics}
\end{entry}

\begin{entry}{体育}{7,8}{⼈、⾁}
  \begin{phonetics}{体育}{ti3yu4}[][HSK 2]
    \definition{s.}{treinamento físico | esportes | atividades esportivas}
  \end{phonetics}
\end{entry}

\begin{entry}{体育场}{7,8,6}{⼈、⾁、⼟}
  \begin{phonetics}{体育场}{ti3 yu4 chang3}[][HSK 2]
    \definition[个,座]{s.}{estádio | campo de esportes}
  \end{phonetics}
\end{entry}

\begin{entry}{体育馆}{7,8,11}{⼈、⾁、⾷}
  \begin{phonetics}{体育馆}{ti3 yu4 guan3}[][HSK 2]
    \definition[个]{s.}{ginásio | estádio}
  \end{phonetics}
\end{entry}

\begin{entry}{体重}{7,9}{⼈、⾥}
  \begin{phonetics}{体重}{ti3 zhong4}[][HSK 4]
    \definition{s.}{peso corporal}
  \end{phonetics}
\end{entry}

\begin{entry}{体验}{7,10}{⼈、⾺}
  \begin{phonetics}{体验}{ti3yan4}[][HSK 3]
    \definition[种]{s.}{experiência}
    \definition{v.}{aprender através da prática; aprender através da experiência pessoal}
  \end{phonetics}
\end{entry}

\begin{entry}{体检}{7,11}{⼈、⽊}
  \begin{phonetics}{体检}{ti3 jian3}[][HSK 4]
    \definition{s.}{exame clínico}
    \definition{v.}{fazer um exame médico}
  \end{phonetics}
\end{entry}

\begin{entry}{体操}{7,16}{⼈、⼿}
  \begin{phonetics}{体操}{ti3 cao1}[][HSK 4]
    \definition{s.}{ginástica; esportes, exercícios ou performances de vários movimentos, sem armas ou com o auxílio de determinados equipamentos}
  \end{phonetics}
\end{entry}

\begin{entry}{何}{7}{⼈}
  \begin{phonetics}{何}{he2}
    \definition*{s.}{sobrenome He}
    \definition{adv.}{expressa exclamação, equivalente a ``多么''}
    \definition{pron.}{O que?; Onde?; Por que? | expressa uma pergunta retórica, equivalente a ``岂'', ``怎''}
  \seealsoref{多么}{duo1me5}
  \seealsoref{岂}{qi3}
  \seealsoref{怎}{zen3}
  \end{phonetics}
\end{entry}

\begin{entry}{何况}{7,7}{⼈、⼎}
  \begin{phonetics}{何况}{he2kuang4}
    \definition{conj.}{além disso | muito menos}
  \end{phonetics}
\end{entry}

\begin{entry}{佛}{7}{⼈}
  \begin{phonetics}{佛}{fo2}
    \definition*{s.}{Buda, abreviação de 佛陀 | Budismo}
  \seealsoref{佛陀}{fo2tuo2}
  \end{phonetics}
  \begin{phonetics}{佛}{fu2}
    \definition{adv.}{aparentemente}
    \definition{s.}{ornamento da cabeça (feminino)}
  \end{phonetics}
\end{entry}

\begin{entry}{佛陀}{7,7}{⼈、⾩}
  \begin{phonetics}{佛陀}{fo2tuo2}
    \definition{s.}{Buda (uma pessoa que atingiu a Budeidade, ou especificamente Siddhartha Gautama)}
  \end{phonetics}
\end{entry}

\begin{entry}{作}{7}{⼈}
  \begin{phonetics}{作}{zuo1}
    \definition{adj.}{(gíria) incômodo}
    \definition{s.}{trabalhador | oficina | (pessoa) de alta manutenção}
  \end{phonetics}
  \begin{phonetics}{作}{zuo4}
    \definition{s.}{escritos ou obras}
    \definition{v.}{fazer | crescer | escrever ou compor | fingir | considerar como | sentir}
  \end{phonetics}
\end{entry}

\begin{entry}{作为}{7,4}{⼈、⼂}
  \begin{phonetics}{作为}{zuo4wei2}[][HSK 4]
    \definition{prep.}{como; na capacidade de; no caráter de; no papel de; em termos de uma certa identidade de uma pessoa ou de uma certa natureza de uma coisa}
    \definition{s.}{ato; ação; conduta; feito; comportamento | conquista; realização; especificamente, uma boa ação}
    \definition{v.}{considerar como; tomar por; olhar como; tratar como | realizar; fazer conquistas; deixar uma marca}
  \end{phonetics}
\end{entry}

\begin{entry}{作文}{7,4}{⼈、⽂}
  \begin{phonetics}{作文}{zuo4wen2}[][HSK 2]
    \definition[篇]{s.}{ensaio |  composição | redação}
    \definition{v.+compl.}{(de alunos) para escrever uma redação}
  \end{phonetics}
\end{entry}

\begin{entry}{作业}{7,5}{⼈、⼀}
  \begin{phonetics}{作业}{zuo4ye4}[][HSK 2]
    \definition[份,个]{s.}{tarefa escolar | trabalho | tarefa | operação}
  \end{phonetics}
\end{entry}

\begin{entry}{作出}{7,5}{⼈、⼐}
  \begin{phonetics}{作出}{zuo4 chu1}[][HSK 4]
    \definition{v.}{mostrar; tomar (decisões, conclusões, etc. por meio de consideração ou discussão); formar (uma conclusão, decisão, etc.) por meio de consideração ou discussão}
  \end{phonetics}
\end{entry}

\begin{entry}{作用}{7,5}{⼈、⽤}
  \begin{phonetics}{作用}{zuo4yong4}[][HSK 2]
    \definition{s.}{efeito | ação | função}
    \definition{v.}{afetar | agir em}
  \end{phonetics}
\end{entry}

\begin{entry}{作者}{7,8}{⼈、⽼}
  \begin{phonetics}{作者}{zuo4zhe3}[][HSK 3]
    \definition[位,名,个]{s.}{autor; escritor; uma pessoa que escreve artigos ou cria obras de arte}
  \end{phonetics}
\end{entry}

\begin{entry}{作品}{7,9}{⼈、⼝}
  \begin{phonetics}{作品}{zuo4pin3}[][HSK 3]
    \definition[个,部,篇,幅]{s.}{obra de arte; obras concluídas de literatura e arte}
  \end{phonetics}
\end{entry}

\begin{entry}{作家}{7,10}{⼈、⼧}
  \begin{phonetics}{作家}{zuo4jia1}[][HSK 2]
    \definition[位,个]{s.}{autor | escritor}
  \end{phonetics}
\end{entry}

\begin{entry}{你}{7}{⼈}
  \begin{phonetics}{你}{ni3}[][HSK 1]
    \definition{pron.}{você (informal) | tu | te | ti | contigo}
    \seeref{您}{nin2}
  \end{phonetics}
\end{entry}

\begin{entry}{你们}{7,5}{⼈、⼈}
  \begin{phonetics}{你们}{ni3men5}[][HSK 1]
    \definition{pron.}{vocês (informal) | vós | vos | convosco}
  \end{phonetics}
\end{entry}

\begin{entry}{你们的}{7,5,8}{⼈、⼈、⽩}
  \begin{phonetics}{你们的}{ni3men5 de5}
    \definition{pron.}{vossos}
  \end{phonetics}
\end{entry}

\begin{entry}{你好}{7,6}{⼈、⼥}
  \begin{phonetics}{你好}{ni3hao3}
    \definition{interj.}{Olá! | Oi!}
  \end{phonetics}
\end{entry}

\begin{entry}{你的}{7,8}{⼈、⽩}
  \begin{phonetics}{你的}{ni3 de5}
    \definition{pron.}{seu}
  \end{phonetics}
\end{entry}

\begin{entry}{佩服}{8,8}{⼈、⽉}
  \begin{phonetics}{佩服}{pei4fu2}
    \definition{v.}{admirar}
  \end{phonetics}
\end{entry}

\begin{entry}{使}{8}{⼈}
  \begin{phonetics}{使}{shi3}[][HSK 3]
    \definition*{s.}{sobrenome Shi}
    \definition{conj.}{se; supondo}
    \definition{s.}{enviado; mensageiro}
    \definition{v.}{enviar; dizer a alguém para fazer algo | usar; empregar; aplicar | fazer; causar; habilitar}
  \end{phonetics}
\end{entry}

\begin{entry}{使用}{8,5}{⼈、⽤}
  \begin{phonetics}{使用}{shi3yong4}[][HSK 2]
    \definition{v.}{usar | empregar | aplicar}
  \end{phonetics}
\end{entry}

\begin{entry}{使劲}{8,7}{⼈、⼒}
  \begin{phonetics}{使劲}{shi3 jin4}[][HSK 4]
    \definition{v.+compl.}{colocar energia; exercer toda a sua força | esforçar-se para ajudar; colocar energia para ajudar}
  \end{phonetics}
\end{entry}

\begin{entry}{例子}{8,3}{⼈、⼦}
  \begin{phonetics}{例子}{li4 zi5}[][HSK 2]
    \definition[个]{s.}{exemplo}
  \end{phonetics}
\end{entry}

\begin{entry}{例如}{8,6}{⼈、⼥}
  \begin{phonetics}{例如}{li4ru2}[][HSK 2]
    \definition{conj.}{por exemplo | como}
  \end{phonetics}
\end{entry}

\begin{entry}{供应}{8,7}{⼈、⼴}
  \begin{phonetics}{供应}{gong1 ying4}[][HSK 4]
    \definition{v.}{fornecer; acomodar}
  \end{phonetics}
\end{entry}

\begin{entry}{依偎}{8,11}{⼈、⼈}
  \begin{phonetics}{依偎}{yi1wei1}
    \definition{v.}{aninhar-se | aconchegar-se}
  \end{phonetics}
\end{entry}

\begin{entry}{依然}{8,12}{⼈、⽕}
  \begin{phonetics}{依然}{yi1ran2}[][HSK 4]
    \definition{adv.}{ainda; como antes;}
    \definition{v.}{estar quieto; estar como antes; estar como o original, sem alterações}
  \end{phonetics}
\end{entry}

\begin{entry}{依靠}{8,15}{⼈、⾮}
  \begin{phonetics}{依靠}{yi1kao4}[][HSK 4]
    \definition{s.}{apoio; suporte; algo em que se apoiar; alguém ou algo em quem você pode confiar}
    \definition{v.}{depender de; confiar em (alguém ou alguma coisa para atingir um determinado objetivo)}
  \end{phonetics}
\end{entry}

\begin{entry}{侵略}{9,11}{⼈、⽥}
  \begin{phonetics}{侵略}{qin1lve4}
    \definition{s.}{invasão}
    \definition{v.}{invadir}
  \end{phonetics}
\end{entry}

\begin{entry}{便于}{9,3}{⼈、⼆}
  \begin{phonetics}{便于}{bian4yu2}[][HSK 5]
    \definition{v.}{ser fácil para; ser conveniente para (algo ou fazer algo)}
  \end{phonetics}
\end{entry}

\begin{entry}{便利}{9,7}{⼈、⼑}
  \begin{phonetics}{便利}{bian4li4}[][HSK 5]
    \definition{adj.}{fácil; conveniente;}
    \definition{s.}{facilidade; conveniência; coisas ou condições convenientes}
    \definition{v.}{facilitar; fornecer ajuda para que os outros se sintam confortáveis}
  \end{phonetics}
\end{entry}

\begin{entry}{便条}{9,7}{⼈、⽊}
  \begin{phonetics}{便条}{bian4tiao2}[][HSK 5]
    \definition[张,个]{s.}{nota ou mensagem informal; geralmente uma mensagem ou notificação}
  \end{phonetics}
\end{entry}

\begin{entry}{便宜}{9,8}{⼈、⼧}
  \begin{phonetics}{便宜}{pian2yi5}[][HSK 2]
    \definition{adj.}{barato}
    \definition{v.}{deixar alguém levemente de lado}
  \end{phonetics}
\end{entry}

\begin{entry}{促进}{9,7}{⼈、⾡}
  \begin{phonetics}{促进}{cu4jin4}[][HSK 4]
    \definition{v.}{impulsionar; promover; avançar; incentivar o desenvolvimento}
  \end{phonetics}
\end{entry}

\begin{entry}{促使}{9,8}{⼈、⼈}
  \begin{phonetics}{促使}{cu4shi3}[][HSK 4]
    \definition{v.}{incitar; estimular; impelir; causar; provocar uma mudança em alguém ou em algo}
  \end{phonetics}
\end{entry}

\begin{entry}{促销}{9,12}{⼈、⾦}
  \begin{phonetics}{促销}{cu4 xiao1}[][HSK 4]
    \definition{v.}{promover vendas}
  \end{phonetics}
\end{entry}

\begin{entry}{俄}{9}{⼈}
  \begin{phonetics}{俄}{e2}
    \definition*{s.}{Rússia, abreviação de 俄罗斯}
  \seealsoref{俄罗斯}{e2luo2si1}
  \end{phonetics}
\end{entry}

\begin{entry}{俄罗斯}{9,8,12}{⼈、⽹、⽄}
  \begin{phonetics}{俄罗斯}{e2luo2si1}
    \definition*{s.}{Rússia}
  \end{phonetics}
\end{entry}

\begin{entry}{俄罗斯人}{9,8,12,2}{⼈、⽹、⽄、⼈}
  \begin{phonetics}{俄罗斯人}{e2luo2si1ren2}
    \definition{s.}{russo | pessoa ou povo da Rússia}
  \end{phonetics}
\end{entry}

\begin{entry}{保}{9}{⼈}
  \begin{phonetics}{保}{bao3}[][HSK 3]
    \definition*{s.}{sobrenome Bao}
    \definition{s.}{fiador
oficial responsável
sistema administrativo}
    \definition{v.}{defender | proteger |manter | preservar | conservar em boas condições | garantir | assegurar | ficar como fiador de alguém.}
  \end{phonetics}
\end{entry}

\begin{entry}{保卫}{9,3}{⼈、⼙}
  \begin{phonetics}{保卫}{bao3wei4}[][HSK 5]
    \definition{v.}{defender; proteger; salvaguardar}
  \end{phonetics}
\end{entry}

\begin{entry}{保存}{9,6}{⼈、⼦}
  \begin{phonetics}{保存}{bao3cun2}[][HSK 3]
    \definition{v.}{conservar | preservar | (computação) salvar (um arquivo, etc.)}
  \end{phonetics}
\end{entry}

\begin{entry}{保守}{9,6}{⼈、⼧}
  \begin{phonetics}{保守}{bao3shou3}[][HSK 4]
    \definition{adj.}{retrógrado; conservador; pensamentos e conceitos que são retrógrados e não conseguem acompanhar o desenvolvimento da situação}
    \definition{v.}{manter; guardar; evitar perder}
  \end{phonetics}
\end{entry}

\begin{entry}{保安}{9,6}{⼈、⼧}
  \begin{phonetics}{保安}{bao3 an1}[][HSK 3]
    \definition{s.}{guarda de segurança}
    \definition{v.}{manter seguro | garantir a segurança}
  \end{phonetics}
\end{entry}

\begin{entry}{保护}{9,7}{⼈、⼿}
  \begin{phonetics}{保护}{bao3hu4}[][HSK 3]
    \definition{s.}{proteção | salvaguarda}
    \definition{v.}{proteger | defender | salvaguardar}
  \end{phonetics}
\end{entry}

\begin{entry}{保护区}{9,7,4}{⼈、⼿、⼖}
  \begin{phonetics}{保护区}{bao3hu4qu1}
    \definition{s.}{área protegida | área de conservação}
  \end{phonetics}
\end{entry}

\begin{entry}{保护主义}{9,7,5,3}{⼈、⼿、⼂、⼂}
  \begin{phonetics}{保护主义}{bao3hu4zhu3yi4}
    \definition{s.}{protecionismo}
  \end{phonetics}
\end{entry}

\begin{entry}{保护色}{9,7,6}{⼈、⼿、⾊}
  \begin{phonetics}{保护色}{bao3hu4se4}
    \definition{s.}{camuflagem}
  \end{phonetics}
\end{entry}

\begin{entry}{保护剂}{9,7,8}{⼈、⼿、⼑}
  \begin{phonetics}{保护剂}{bao3hu4ji4}
    \definition{s.}{agente protetor}
  \end{phonetics}
\end{entry}

\begin{entry}{保护国}{9,7,8}{⼈、⼿、⼞}
  \begin{phonetics}{保护国}{bao3hu4guo2}
    \definition{s.}{protetorado}
  \end{phonetics}
\end{entry}

\begin{entry}{保护性}{9,7,8}{⼈、⼿、⼼}
  \begin{phonetics}{保护性}{bao3hu4xing4}
    \definition{s.}{proteção}
  \end{phonetics}
\end{entry}

\begin{entry}{保护物}{9,7,8}{⼈、⼿、⽜}
  \begin{phonetics}{保护物}{bao3hu4 wu4}
    \definition{s.}{protetor}
  \end{phonetics}
\end{entry}

\begin{entry}{保护者}{9,7,8}{⼈、⼿、⽼}
  \begin{phonetics}{保护者}{bao3hu4zhe3}
    \definition{s.}{protetor | segurador}
  \end{phonetics}
\end{entry}

\begin{entry}{保护神}{9,7,9}{⼈、⼿、⽰}
  \begin{phonetics}{保护神}{bao3hu4shen2}
    \definition{s.}{anjo da guarda | santo patrono}
  \end{phonetics}
\end{entry}

\begin{entry}{保证}{9,7}{⼈、⾔}
  \begin{phonetics}{保证}{bao3zheng4}[][HSK 3]
    \definition[个]{s.}{garantia}
    \definition{v.}{garantir}
  \end{phonetics}
\end{entry}

\begin{entry}{保养}{9,9}{⼈、⼋}
  \begin{phonetics}{保养}{bao3yang3}[][HSK 5]
    \definition{v.}{preservar; cuidar bem (ou conservar) da saúde |  fazer manutenção; conservar; manter; manter em bom estado de conservação}
  \end{phonetics}
\end{entry}

\begin{entry}{保持}{9,9}{⼈、⼿}
  \begin{phonetics}{保持}{bao3chi2}[][HSK 3]
    \definition{v.}{manter | segurar | reter | preservar}
  \end{phonetics}
\end{entry}

\begin{entry}{保险}{9,9}{⼈、⾩}
  \begin{phonetics}{保险}{bao3xian3}[][HSK 3]
    \definition[个]{adj./s.}{seguro}
    \definition{v.}{ter certeza | estar vinculado a}
  \end{phonetics}
\end{entry}

\begin{entry}{保留}{9,10}{⼈、⽥}
  \begin{phonetics}{保留}{bao3liu2}[][HSK 3]
    \definition{v.}{reter | continuar a ter | segurar | reservar}
  \end{phonetics}
\end{entry}

\begin{entry}{保密}{9,11}{⼈、⼧}
  \begin{phonetics}{保密}{bao3mi4}[][HSK 4]
    \definition{v.}{manter segredo; manter algo em segredo; manter a confidencialidade}
  \end{phonetics}
\end{entry}

\begin{entry}{信}{9}{⼈}
  \begin{phonetics}{信}{xin4}[][HSK 2,3]
    \definition*{s.}{sobrenome Xin}
    \definition{adj.}{verdade}
    \definition{adv.}{à vontade; ao acaso; sem plano}
    \definition[封,个,张]{s.}{carta; correio
mensagem; palavra; informação
sinal; evidência
confiança; fé
fusível
arsênico}
    \definition{v.}{acreditar; fazer um balanço; dar crédito | professar fé em; acreditar em}
  \end{phonetics}
\end{entry}

\begin{entry}{信心}{9,4}{⼈、⼼}
  \begin{phonetics}{信心}{xin4xin1}[][HSK 2]
    \definition[个]{s.}{confiança | fé (em alguém ou algo)}
  \end{phonetics}
\end{entry}

\begin{entry}{信号}{9,5}{⼈、⼝}
  \begin{phonetics}{信号}{xin4hao4}[][HSK 2]
    \definition[个]{s.}{sinal | ponte de sinalização}
  \end{phonetics}
\end{entry}

\begin{entry}{信用}{9,5}{⼈、⽤}
  \begin{phonetics}{信用}{xin4yong4}
    \definition{s.}{crédito (comércio)}
  \end{phonetics}
\end{entry}

\begin{entry}{信用卡}{9,5,5}{⼈、⽤、⼘}
  \begin{phonetics}{信用卡}{xin4yong4ka3}[][HSK 2]
    \definition[些]{s.}{cartão de crédito}
  \end{phonetics}
\end{entry}

\begin{entry}{信任}{9,6}{⼈、⼈}
  \begin{phonetics}{信任}{xin4ren4}[][HSK 3]
    \definition[个]{s.}{confiança; certeza; convicção}
    \definition{v.}{confiar; ter confiança em}
  \end{phonetics}
\end{entry}

\begin{entry}{信访}{9,6}{⼈、⾔}
  \begin{phonetics}{信访}{xin4fang3}
    \definition{s.}{carta de reclamação | carta de petição}
  \seealsoref{上访}{shang4fang3}
  \end{phonetics}
\end{entry}

\begin{entry}{信经}{9,8}{⼈、⽷}
  \begin{phonetics}{信经}{xin4jing1}
    \definition[个]{s.}{crença | credo (seção da missa católica)}
  \end{phonetics}
\end{entry}

\begin{entry}{信封}{9,9}{⼈、⼨}
  \begin{phonetics}{信封}{xin4feng1}[][HSK 3]
    \definition[个]{s.}{envelope de carta}
  \end{phonetics}
\end{entry}

\begin{entry}{信息}{9,10}{⼈、⼼}
  \begin{phonetics}{信息}{xin4xi1}[][HSK 2]
    \definition[个,条]{s.}{notícias | informação | mensagem}
  \end{phonetics}
\end{entry}

\begin{entry}{俩}{9}{⼈}
  \begin{phonetics}{俩}{lia3}[][HSK 4]
    \definition{num.}{ambos; dois; contração de ``两个'' | alguns; vários; refere-se a um pequeno número}
  \end{phonetics}
\end{entry}

\begin{entry}{俩钱}{9,10}{⼈、⾦}
  \begin{phonetics}{俩钱}{lia3qian2}
    \definition{s.}{uma pequena quantia de dinheiro}
  \end{phonetics}
\end{entry}

\begin{entry}{俭省}{9,9}{⼈、⽬}
  \begin{phonetics}{俭省}{jian3sheng3}
    \definition{adj.}{econômico}
  \end{phonetics}
\end{entry}

\begin{entry}{修}{9}{⼈}
  \begin{phonetics}{修}{xiu1}[][HSK 3]
    \definition*{s.}{sobrenome Xiu}
    \definition{adj.}{comprido; alto e esbelto}
    \definition{s.}{revisionismo}
    \definition{v.}{embelezar; decorar
consertar; reparar; reformar
escrever; compilar
estudar; cultivar
construir; edificar
aparar; podar}
  \end{phonetics}
\end{entry}

\begin{entry}{修改}{9,7}{⼈、⽁}
  \begin{phonetics}{修改}{xiu1gai3}[][HSK 3]
    \definition{v.}{revisar; alterar}
  \end{phonetics}
\end{entry}

\begin{entry}{修规}{9,8}{⼈、⾒}
  \begin{phonetics}{修规}{xiu1gui1}
    \definition{s.}{plano de construção}
  \end{phonetics}
\end{entry}

\begin{entry}{修理}{9,11}{⼈、⽟}
  \begin{phonetics}{修理}{xiu1li3}[][HSK 4]
    \definition{v.}{consertar; reparar; restaurar algo danificado à sua forma ou função original | aparar; podar; cortar com tesouras e outras ferramentas para deixar árvores, flores, cabelos, etc. arrumados | culpar; punir; criticar ou punir uma pessoa para mostrar que ela está errada}
  \end{phonetics}
\end{entry}

\begin{entry}{倂}{10}{⼈}
  \begin{phonetics}{倂}{bing4}
    \variantof{并}
  \end{phonetics}
\end{entry}

\begin{entry}{倍}{10}{⼈}
  \begin{phonetics}{倍}{bei4}[][HSK 4]
    \definition{adv.}{mais; especialmente}
    \definition{clas.}{vezes; para obter um número igual ao número original, você pode multiplicar o número por esse múltiplo}
    \definition{s.}{dobro; duas vezes mais}
  \end{phonetics}
\end{entry}

\begin{entry}{倒}{10}{⼈}
  \begin{phonetics}{倒}{dao3}[][HSK 2]
    \definition{v.}{cair no chão | deitar-se no chão | colapsar | ir à falência}
  \end{phonetics}
  \begin{phonetics}{倒}{dao4}[][HSK 2]
    \definition{adv.}{ao contrário da expectativa | ao contrário}
    \definition{v.}{inverter | colocar de cabeça para baixo ou de frente para trás | derramar | tombar}
  \end{phonetics}
\end{entry}

\begin{entry}{倒车}{10,4}{⼈、⾞}
  \begin{phonetics}{倒车}{dao3che1}[][HSK 4]
    \definition{v.}{mudar de trem ou ônibus; trocar de trem ou ônibus no meio do caminho}
  \end{phonetics}
  \begin{phonetics}{倒车}{dao4che1}[][HSK 4]
    \definition{v.}{dar marcha à ré (em um veículo)}
  \end{phonetics}
\end{entry}

\begin{entry}{倒地}{10,6}{⼈、⼟}
  \begin{phonetics}{倒地}{dao3di4}
    \definition{v.}{cair no chão}
  \end{phonetics}
\end{entry}

\begin{entry}{倒血霉}{10,6,15}{⼈、⾎、⾬}
  \begin{phonetics}{倒血霉}{dao3xue4mei2}
    \definition{v.}{ter muito azar (versão mais forte de 倒霉)}
  \seealsoref{倒霉}{dao3mei2}
  \end{phonetics}
\end{entry}

\begin{entry}{倒闭}{10,6}{⼈、⾨}
  \begin{phonetics}{倒闭}{dao3bi4}[][HSK 4]
    \definition{v.}{fechar; ir à falência; entrar em liquidação; sair do negócio}
  \end{phonetics}
\end{entry}

\begin{entry}{倒是}{10,9}{⼈、⽇}
  \begin{phonetics}{倒是}{dao4 shi4}[][HSK 5]
    \definition{adv.}{usado para indicar o oposto do que geralmente é verdade; ao contrário do senso comum; pelo contrário | usado para indicar o que é contrário aos fatos, com um toque de crítica; indica que as coisas não são assim (com um sentimento de culpa) | usado de algo inesperado; expressando surpresa | usado para indicar concessão | usado para indicar uma mudança de significado; indica um ponto de virada | usado para modificar ou suavizar uma declaração anterior; para suavizar o tom | usado para pressionar ou questionar alguém; para instar ou perguntar}
  \end{phonetics}
\end{entry}

\begin{entry}{倒楣}{10,13}{⼈、⽊}
  \begin{phonetics}{倒楣}{dao3mei2}
    \variantof{倒霉}
  \end{phonetics}
\end{entry}

\begin{entry}{倒霉}{10,15}{⼈、⾬}
  \begin{phonetics}{倒霉}{dao3mei2}
    \definition{adj.}{azarado}
    \definition{s.}{azar | má sorte}
    \definition{v.}{estar sem sorte | ter azar}
  \seealsoref{倒血霉}{dao3xue4mei2}
  \end{phonetics}
\end{entry}

\begin{entry}{倘使}{10,8}{⼈、⼈}
  \begin{phonetics}{倘使}{tang3shi3}
    \definition{conj.}{se | supondo que | no caso}
  \end{phonetics}
\end{entry}

\begin{entry}{倘或}{10,8}{⼈、⼽}
  \begin{phonetics}{倘或}{tang3huo4}
    \definition{conj.}{se | supondo que | no caso}
  \end{phonetics}
\end{entry}

\begin{entry}{倘若}{10,8}{⼈、⾋}
  \begin{phonetics}{倘若}{tang3ruo4}
    \definition{conj.}{se | supondo que | no caso}
  \end{phonetics}
\end{entry}

\begin{entry}{借}{10}{⼈}
  \begin{phonetics}{借}{jie4}[][HSK 2]
    \definition{adv.}{por meio de}
    \definition{v.}{pedir emprestado | emprestar | aproveitar (uma oportunidade)}
  \end{phonetics}
\end{entry}

\begin{entry}{借书证}{10,4,7}{⼈、⼄、⾔}
  \begin{phonetics}{借书证}{jie4shu1zheng4}
    \definition{s.}{cartão de biblioteca | (literalmente) cartão para pedir emprestado livros}
  \end{phonetics}
\end{entry}

\begin{entry}{倡导}{10,6}{⼈、⼨}
  \begin{phonetics}{倡导}{chang4dao3}[][HSK 5]
    \definition{v.}{iniciar; propor; promover; defender; advogar}
  \end{phonetics}
\end{entry}

\begin{entry}{值}{10}{⼈}
  \begin{phonetics}{值}{zhi2}[][HSK 3]
    \definition{s.}{preço; valor; valor numérico}
    \definition{v.}{valer a pena | acontecer com; ir de encontro | estar de serviço; ter sua vez em algo; assumir a posição de turno}
  \end{phonetics}
\end{entry}

\begin{entry}{值得}{10,11}{⼈、⼻}
  \begin{phonetics}{值得}{zhi2de5}[][HSK 3]
    \definition{adj.}{que vale a pena}
    \definition{v.}{merecer; valer a pena; ser digno; significa que fazer isso terá bons resultados; é valioso e significativo}
  \end{phonetics}
\end{entry}

\begin{entry}{倾城}{10,9}{⼈、⼟}
  \begin{phonetics}{倾城}{qing1cheng2}
    \definition{adj.}{sedutora (mulher)}
    \definition{adv.}{de todo o lugar | vindo de todos os lugares}
    \definition{v.}{arruinar e derrubar o estado}
  \end{phonetics}
\end{entry}

\begin{entry}{健身}{10,7}{⼈、⾝}
  \begin{phonetics}{健身}{jian4shen1}[][HSK 4]
    \definition{s.}{exercício físico | \emph{fitness}}
    \definition{v.+compl.}{exercitar-se; manter a forma; praticar um esporte, especialmente a ginástica, inclusive em aparelhos, para desenvolver força, flexibilidade, aumentar a resistência, melhorar a coordenação e o controle de todas as partes do corpo}
  \end{phonetics}
\end{entry}

\begin{entry}{健康}{10,11}{⼈、⼴}
  \begin{phonetics}{健康}{jian4kang1}[][HSK 2]
    \definition{adj.}{em forma | saudável | curado}
    \definition{s.}{saúde | físico}
  \end{phonetics}
\end{entry}

\begin{entry}{假}{11}{⼈}
  \begin{phonetics}{假}{jia3}[][HSK 2]
    \definition{adj.}{falso | artificial}
    \definition{v.}{emprestar}
  \end{phonetics}
  \begin{phonetics}{假}{jia4}
    \definition{s.}{férias}
  \end{phonetics}
\end{entry}

\begin{entry}{假如}{11,6}{⼈、⼥}
  \begin{phonetics}{假如}{jia3ru2}[][HSK 4]
    \definition{conj.}{se; supondo; no caso}
  \end{phonetics}
\end{entry}

\begin{entry}{假声}{11,7}{⼈、⼠}
  \begin{phonetics}{假声}{jia3sheng1}
    \definition{s.}{falsete}
  \seealsoref{真声}{zhen1sheng1}
  \end{phonetics}
\end{entry}

\begin{entry}{假证件}{11,7,6}{⼈、⾔、⼈}
  \begin{phonetics}{假证件}{jia3zheng4jian4}
    \definition{s.}{documentos falsos}
  \end{phonetics}
\end{entry}

\begin{entry}{假使}{11,8}{⼈、⼈}
  \begin{phonetics}{假使}{jia3shi3}
    \definition{conj.}{se | supondo | em caso}
  \end{phonetics}
\end{entry}

\begin{entry}{假的}{11,8}{⼈、⽩}
  \begin{phonetics}{假的}{jia3de5}
    \definition{adj.}{falso | substituto | simulado}
  \end{phonetics}
\end{entry}

\begin{entry}{假期}{11,12}{⼈、⽉}
  \begin{phonetics}{假期}{jia4 qi1}[][HSK 2]
    \definition[个]{s.}{férias | feriados | período de licença}
  \end{phonetics}
\end{entry}

\begin{entry}{偏偏}{11,11}{⼈、⼈}
  \begin{phonetics}{偏偏}{pian1pian1}
    \definition{adv.}{voluntariamente | insistentemente | persistentemente | ao contrário da expectativa | infelizmente (indicando que alguma coisa aconteceu ao contrário do que se esperava) | teimosamente (indicando que algo é o oposto ao que seria normal ou razoável) | precisamente (indicando que alguém ou um grupo é escolhido)}
  \end{phonetics}
\end{entry}

\begin{entry}{做}{11}{⼈}
  \begin{phonetics}{做}{zuo4}[][HSK 1]
    \definition{v.}{fazer}
  \end{phonetics}
\end{entry}

\begin{entry}{做生活}{11,5,9}{⼈、⽣、⽔}
  \begin{phonetics}{做生活}{zuo4sheng1huo2}
    \definition{v.}{fazer tabalhos manuais}
  \end{phonetics}
\end{entry}

\begin{entry}{做戏}{11,6}{⼈、⼽}
  \begin{phonetics}{做戏}{zuo4xi4}
    \definition{v.}{atuar em uma peça | fazer uma peça}
  \end{phonetics}
\end{entry}

\begin{entry}{做作}{11,7}{⼈、⼈}
  \begin{phonetics}{做作}{zuo4zuo5}
    \definition{adj.}{afetado | artificial}
  \end{phonetics}
\end{entry}

\begin{entry}{做饭}{11,7}{⼈、⾷}
  \begin{phonetics}{做饭}{zuo4 fan4}[][HSK 2]
    \definition{v.}{preparar uma refeição | cozinhar}
  \end{phonetics}
\end{entry}

\begin{entry}{做到}{11,8}{⼈、⼑}
  \begin{phonetics}{做到}{zuo4 dao4}[][HSK 2]
    \definition{v.}{realizar | alcançar}
  \end{phonetics}
\end{entry}

\begin{entry}{做法}{11,8}{⼈、⽔}
  \begin{phonetics}{做法}{zuo4fa3}[][HSK 2]
    \definition[个]{s.}{método para fazer | prática | receita | maneira de lidar com algo | método de trabalho}
  \end{phonetics}
\end{entry}

\begin{entry}{做客}{11,9}{⼈、⼧}
  \begin{phonetics}{做客}{zuo4 ke4}[][HSK 3]
    \definition{v.}{visitar; ser um convidado; visitar outras pessoas e ser você mesmo o convidado}
  \end{phonetics}
\end{entry}

\begin{entry}{做活}{11,9}{⼈、⽔}
  \begin{phonetics}{做活}{zuo4huo2}
    \definition{v.}{trabalhar para ganhar a vida (especialmente de mulher costureira)}
  \end{phonetics}
\end{entry}

\begin{entry}{做梦}{11,11}{⼈、⼣}
  \begin{phonetics}{做梦}{zuo4 meng4}[][HSK 4]
    \definition{s.}{sonho; ilusões e visões na consciência durante o sono}
    \definition{v.}{sonhar; ter um sonho | sonhar acordado, ter um sonho impossível (parábola de fantasias irrealistas)}[别​做​梦​了​,她​不​会​嫁​给​你​的​。(Pare de sonhar, ela não se casará com você.)]
  \end{phonetics}
\end{entry}

\begin{entry}{做眼}{11,11}{⼈、⽬}
  \begin{phonetics}{做眼}{zuo4yan3}
    \definition{v.}{agir como um guia | trabalhar como espião}
  \end{phonetics}
\end{entry}

\begin{entry}{停}{11}{⼈}
  \begin{phonetics}{停}{ting2}[][HSK 2]
    \definition{v.}{parar | estacionar (um carro)}
  \end{phonetics}
\end{entry}

\begin{entry}{停下}{11,3}{⼈、⼀}
  \begin{phonetics}{停下}{ting2 xia4}[][HSK 4]
    \definition{v.}{encerrar; desligar; parar}
  \end{phonetics}
\end{entry}

\begin{entry}{停工}{11,3}{⼈、⼯}
  \begin{phonetics}{停工}{ting2gong1}
    \definition{v.}{parar de trabalhar | parar a produção}
  \end{phonetics}
\end{entry}

\begin{entry}{停办}{11,4}{⼈、⼒}
  \begin{phonetics}{停办}{ting2ban4}
    \definition{v.}{cancelar | sair do negócio | desligar | terminar}
  \end{phonetics}
\end{entry}

\begin{entry}{停止}{11,4}{⼈、⽌}
  \begin{phonetics}{停止}{ting2 zhi3}[][HSK 3]
    \definition{v.}{parar; suspender; cessar; cancelar}
  \end{phonetics}
\end{entry}

\begin{entry}{停火}{11,4}{⼈、⽕}
  \begin{phonetics}{停火}{ting2huo3}
    \definition{s.}{cessar-fogo}
    \definition{v.+compl.}{cessar fogo}
  \end{phonetics}
\end{entry}

\begin{entry}{停车}{11,4}{⼈、⾞}
  \begin{phonetics}{停车}{ting2 che1}[][HSK 2]
    \definition{v.}{parar de trabalhar (uma máquina) | estacionar | parar (um veículo) | paralisar}
  \end{phonetics}
\end{entry}

\begin{entry}{停车场}{11,4,6}{⼈、⾞、⼟}
  \begin{phonetics}{停车场}{ting2 che1 chang3}[][HSK 2]
    \definition{s.}{parque de estacionamento}
  \end{phonetics}
\end{entry}

\begin{entry}{停业}{11,5}{⼈、⼀}
  \begin{phonetics}{停业}{ting2ye4}
    \definition{v.}{cessar a negociação (temporária ou permanentemente) | fechar}
  \end{phonetics}
\end{entry}

\begin{entry}{停用}{11,5}{⼈、⽤}
  \begin{phonetics}{停用}{ting2yong4}
    \definition{v.}{desabilitar | descontinuar | parar de usar | suspender}
  \end{phonetics}
\end{entry}

\begin{entry}{停电}{11,5}{⼈、⽥}
  \begin{phonetics}{停电}{ting2dian4}
    \definition{s.}{corte de energia}
    \definition{v.}{ter uma falha de energia}
  \end{phonetics}
\end{entry}

\begin{entry}{停当}{11,6}{⼈、⼹}
  \begin{phonetics}{停当}{ting2dang5}
    \definition{adj.}{realizado | preparado | assentado}
  \end{phonetics}
\end{entry}

\begin{entry}{停息}{11,10}{⼈、⼼}
  \begin{phonetics}{停息}{ting2xi1}
    \definition{v.}{cessar | parar}
  \end{phonetics}
\end{entry}

\begin{entry}{停留}{11,10}{⼈、⽥}
  \begin{phonetics}{停留}{ting2liu2}
    \definition{v.}{ficar em algum lugar temporariamente | demorar | permanecer}
  \end{phonetics}
\end{entry}

\begin{entry}{停课}{11,10}{⼈、⾔}
  \begin{phonetics}{停课}{ting2ke4}
    \definition{v.}{fechar (escola) | parar as aulas}
  \end{phonetics}
\end{entry}

\begin{entry}{停歇}{11,13}{⼈、⽋}
  \begin{phonetics}{停歇}{ting2xie1}
    \definition{v.}{parar para descansar}
  \end{phonetics}
\end{entry}

\begin{entry}{偶然}{11,12}{⼈、⽕}
  \begin{phonetics}{偶然}{ou3ran2}
    \definition{adv.}{por acaso | fortuitamente}
  \end{phonetics}
\end{entry}

\begin{entry}{偷}{11}{⼈}
  \begin{phonetics}{偷}{tou1}
    \definition{adv.}{furtivamente}
    \definition{v.}{furtar | roubar}
  \end{phonetics}
\end{entry}

\begin{entry}{偷安}{11,6}{⼈、⼧}
  \begin{phonetics}{偷安}{tou1'an1}
    \definition{v.}{buscar facilidade temporária}
  \end{phonetics}
\end{entry}

\begin{entry}{偷听}{11,7}{⼈、⼝}
  \begin{phonetics}{偷听}{tou1ting1}
    \definition{v.}{bisbilhotar; monitorar (secretamente)}
  \end{phonetics}
\end{entry}

\begin{entry}{偷窃}{11,9}{⼈、⽳}
  \begin{phonetics}{偷窃}{tou1qie4}
    \definition{v.}{furtar | roubar}
  \end{phonetics}
\end{entry}

\begin{entry}{偷情}{11,11}{⼈、⼼}
  \begin{phonetics}{偷情}{tou1qing2}
    \definition{v.}{manter um caso de amor clandestino}
  \end{phonetics}
\end{entry}

\begin{entry}{偷袭}{11,11}{⼈、⾐}
  \begin{phonetics}{偷袭}{tou1xi2}
    \definition{s.}{ataque surpresa}
    \definition{v.}{montar um ataque furtivo | invadir}
  \end{phonetics}
\end{entry}

\begin{entry}{偷渡}{11,12}{⼈、⽔}
  \begin{phonetics}{偷渡}{tou1du4}
    \definition{s.}{contrabando | imigração ilegal | clandestino (em um navio)}
    \definition{v.}{executar um bloqueio | roubar através da fronteira internacional}
  \end{phonetics}
\end{entry}

\begin{entry}{偷税}{11,12}{⼈、⽲}
  \begin{phonetics}{偷税}{tou1shui4}
    \definition{s.}{evasão fiscal}
  \end{phonetics}
\end{entry}

\begin{entry}{偸}{11}{⼈}
  \begin{phonetics}{偸}{tou1}
    \variantof{偷}
  \end{phonetics}
\end{entry}

\begin{entry}{傢具}{12,8}{⼈、⼋}
  \begin{phonetics}{傢具}{jia1ju4}
    \variantof{家具}
  \end{phonetics}
\end{entry}

\begin{entry}{傻瓜}{13,5}{⼈、⽠}
  \begin{phonetics}{傻瓜}{sha3gua1}
    \definition{adj.}{tolo | burro | simplório | idiota}
    \definition{v.}{enganar | iludir | lograr}
  \end{phonetics}
\end{entry}

\begin{entry}{傻眼}{13,11}{⼈、⽬}
  \begin{phonetics}{傻眼}{sha3yan3}
    \definition{adj.}{estupefato | atordoado}
  \end{phonetics}
\end{entry}

\begin{entry}{像}{13}{⼈}
  \begin{phonetics}{像}{xiang4}[][HSK 2]
    \definition{s.}{imagem | retrato | aparência}
    \definition{v.}{assemelhar-se | ser como}
  \end{phonetics}
\end{entry}

\begin{entry}{僧}{14}{⼈}
  \begin{phonetics}{僧}{seng1}
    \definition{s.}{monge Budista, abreviação de 僧伽}
    \seeref{僧伽}{seng1qie2}
  \end{phonetics}
\end{entry}

\begin{entry}{僧伽}{14,7}{⼈、⼈}
  \begin{phonetics}{僧伽}{seng1qie2}
    \definition{s.}{sangha ou sanga (Budismo) | a comunidade monástica | monge}
  \end{phonetics}
\end{entry}

\begin{entry}{儒教}{16,11}{⼈、⽁}
  \begin{phonetics}{儒教}{ru2jiao4}
    \definition*{s.}{Confucionismo}
  \end{phonetics}
\end{entry}

%%%%% EOF %%%%%

