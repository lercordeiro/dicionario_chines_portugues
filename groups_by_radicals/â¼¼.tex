%%%
%%% Radical "⼼"
%%%

\section*{Radical 61: ``⼼'' (忄、⺗)}\addcontentsline{toc}{section}{Radical 61: ⼼、忄、⺗}

\begin{entry}{心}{4}{⼼}[Kangxi 61]
  \begin{phonetics}{心}{xin1}[][HSK 3]
    \definition*{s.}{Xin, uma das mansões lunares; uma das vinte e oito constelações}
    \definition[颗,个]{s.}{o coraçã; órgão que impulsiona a circulação sanguínea no corpo humano e nos vertebrados| coração; mente; sentimento; intenção; refere-se aos órgãos do pensamento e ao pensamento, sentimentos, etc. | centro; núcleo; parte central}
  \end{phonetics}
\end{entry}

\begin{entry}{心中}{4,4}{⼼、⼁}
  \begin{phonetics}{心中}{xin1zhong1}[][HSK 2]
    \definition{s.}{no coração; na mente}
  \end{phonetics}
\end{entry}

\begin{entry}{心机}{4,6}{⼼、⽊}
  \begin{phonetics}{心机}{xin1ji1}
    \definition{s.}{pensamento | esquema}
  \end{phonetics}
\end{entry}

\begin{entry}{心声}{4,7}{⼼、⼠}
  \begin{phonetics}{心声}{xin1sheng1}
    \definition{s.}{desejo sincero | voz interior | aspiração}
  \end{phonetics}
\end{entry}

\begin{entry}{心里}{4,7}{⼼、⾥}
  \begin{phonetics}{心里}{xin1 li3}[][HSK 2]
    \definition[个]{s.}{no coração; no coração de alguém | no coração; na mente; na cabeça e no peito}
  \end{phonetics}
\end{entry}

\begin{entry}{心态}{4,8}{⼼、⼼}
  \begin{phonetics}{心态}{xin1tai4}[][HSK 5]
    \definition[种]{s.}{mentalidade; psicologia; estado mental}
  \end{phonetics}
\end{entry}

\begin{entry}{心疼}{4,10}{⼼、⽧}
  \begin{phonetics}{心疼}{xin1teng2}[][HSK 5]
    \definition{adj.}{angustiado}
    \definition{v.}{amar profundamente; sentir pena porque coisas valiosas foram destruídas ou perdidas; não querer se separar delas | sentir pena; ficar angustiado; preocupar-se e sofrer pelo sofrimento dos outros; estar disposto a cuidar mais por causa do carinho}
  \end{phonetics}
\end{entry}

\begin{entry}{心情}{4,11}{⼼、⼼}
  \begin{phonetics}{心情}{xin1qing2}[][HSK 2]
    \definition{s.}{humor; tom de sentimento; estado de espírito; estado emocional interior}
  \end{phonetics}
\end{entry}

\begin{entry}{心理}{4,11}{⼼、⽟}
  \begin{phonetics}{心理}{xin1li3}[][HSK 4]
    \definition{adj.}{psicológico}
    \definition{s.}{mentalidade; refere-se à reflexão da mente humana sobre coisas objetivas, incluindo sensação, percepção, memória, pensamento e emoções | psicologia}
  \end{phonetics}
\end{entry}

\begin{entry}{必}{5}{⼼}
  \begin{phonetics}{必}{bi4}[][HSK 5]
    \definition{adv.}{certamente; necessariamente; indica que algo é certo ou que alguém acredita que esteja correto | deve; tem que}
  \end{phonetics}
\end{entry}

\begin{entry}{必定}{5,8}{⼼、⼧}
  \begin{phonetics}{必定}{bi4ding4}
    \definition{adv.}{sem falta | certamente | com certeza | definitivamente | inevitavelmente | com determinação}
    \definition{v.}{estar vinculado a | ter certeza de}
  \end{phonetics}
\end{entry}

\begin{entry}{必修}{5,9}{⼼、⼈}
  \begin{phonetics}{必修}{bi4 xiu1}[][HSK 6]
    \definition{adj.}{(de um curso acadêmico) obrigatório; compulsório; mandatório; obrigatório estudar de acordo com os regulamentos (em oposição a 选修)}
  \seealsoref{选修}{xuan3 xiu1}
  \end{phonetics}
\end{entry}

\begin{entry}{必将}{5,9}{⼼、⼨}
  \begin{phonetics}{必将}{bi4 jiang1}[][HSK 6]
    \definition{adv.}{certamente; certamente irá; usado para expressar inevitabilidade (ou necessidade)}
  \end{phonetics}
\end{entry}

\begin{entry}{必要}{5,9}{⼼、⾑}
  \begin{phonetics}{必要}{bi4yao4}[][HSK 3]
    \definition{adj.}{necessário; essencial; indispensável}
    \definition[个,些]{s.}{necessidade; características indispensáveis}
  \end{phonetics}
\end{entry}

\begin{entry}{必须}{5,9}{⼼、⾴}
  \begin{phonetics}{必须}{bi4xu1}[][HSK 2]
    \definition{adv.}{necessariamente; obrigatoriamente; indica a necessidade lógica e emocional | deve; tem que; é obrigado a}
  \end{phonetics}
\end{entry}

\begin{entry}{必然}{5,12}{⼼、⽕}
  \begin{phonetics}{必然}{bi4ran2}[][HSK 3]
    \definition{adj.}{certo; inevitável; necessário; definido e inalterável; imutável}
    \definition{adv.}{inevitavelmente}
    \definition{s.}{necessidade; em filosofia, refere-se às leis objetivas do desenvolvimento que não são influenciadas pela vontade humana}
  \end{phonetics}
\end{entry}

\begin{entry}{必需}{5,14}{⼼、⾬}
  \begin{phonetics}{必需}{bi4 xu1}[][HSK 5]
    \definition{v.}{ser essencial; ser indispensável}
  \end{phonetics}
\end{entry}

\begin{entry}{忙}{6}{⼼}
  \begin{phonetics}{忙}{mang2}[][HSK 1]
    \definition*{s.}{Sobrenome Mang}
    \definition{adj.}{ocupado; movimentado; totalmente ocupado; muitas coisas para fazer, sem tempo livre (oposto de 闲) | imperativo; ansioso; urgente}
    \definition{v.}{apressar-se; agitar-se; fazer algo com urgência e constantemente | trabalhar; fazer}
  \seealsoref{闲}{xian2}
  \end{phonetics}
\end{entry}

\begin{entry}{忍}{7}{⼼}
  \begin{phonetics}{忍}{ren3}[][HSK 5]
    \definition{v.}{suportar; aguentar; tolerar; aturar | ter coragem para; ser insensível o suficiente para; ser capaz de endurecer o coração e fazer coisas que não se devem fazer por uma questão de razão}
  \end{phonetics}
\end{entry}

\begin{entry}{忍不住}{7,4,7}{⼼、⼀、⼈}
  \begin{phonetics}{忍不住}{ren3bu5zhu4}[][HSK 5]
    \definition{v.}{incapaz de suportar; não conseguir evitar fazer algo; não conseguir se controlar}
  \end{phonetics}
\end{entry}

\begin{entry}{忍受}{7,8}{⼼、⼜}
  \begin{phonetics}{忍受}{ren3shou4}[][HSK 5]
    \definition{v.}{suportar; sofrer; aguentar; tolerar; suportar com dificuldade o sofrimento, as dificuldades e as adversidades da vida}
  \end{phonetics}
\end{entry}

\begin{entry}{忍耐}{7,9}{⼼、⽽}
  \begin{phonetics}{忍耐}{ren3nai4}
    \definition{s.}{paciência | resistência}
    \definition{v.}{suportar | resistir | exercer paciência}
  \end{phonetics}
\end{entry}

\begin{entry}{志}{7}{⼼}
  \begin{phonetics}{志}{zhi4}
    \definition{s.}{vontade; ideal; aspiração; ambição | anais; registros; transcrição | marca; sinal}
    \definition{v.}{ter em mente; lembrar | pesar; medir comprimento e quantidade}
  \end{phonetics}
\end{entry}

\begin{entry}{志愿}{7,14}{⼼、⽕}
  \begin{phonetics}{志愿}{zhi4 yuan4}[][HSK 3]
    \definition{s.}{desejo; ideal; aspiração; os ideais, desejos ou objetivos que se deseja realizar no coração}
    \definition{v.}{ser voluntário; ser proativo e disposto a realizar trabalhos sem remuneração ou com remuneração baixa, mas que possam ajudar outras pessoas}
  \end{phonetics}
\end{entry}

\begin{entry}{志愿书}{7,14,4}{⼼、⽕、⼄}
  \begin{phonetics}{志愿书}{zhi4yuan4shu1}
    \definition{s.}{formulário de inscrição; formulário de adesão; carta de intenções}
  \end{phonetics}
\end{entry}

\begin{entry}{志愿者}{7,14,8}{⼼、⽕、⽼}
  \begin{phonetics}{志愿者}{zhi4yuan4zhe3}[][HSK 3]
    \definition[名,位,个]{s.}{voluntário; pessoas que se voluntariam para prestar serviços em atividades sociais, grandes eventos esportivos, conferências, etc.}
  \end{phonetics}
\end{entry}

\begin{entry}{忘}{7}{⼼}
  \begin{phonetics}{忘}{wang4}[][HSK 1]
    \definition{v.}{esquecer | ignorar; negligenciar}
  \end{phonetics}
\end{entry}

\begin{entry}{忘本}{7,5}{⼼、⽊}
  \begin{phonetics}{忘本}{wang4ben3}
    \definition{v.}{esquecer as próprias raízes}
  \end{phonetics}
\end{entry}

\begin{entry}{忘记}{7,5}{⼼、⾔}
  \begin{phonetics}{忘记}{wang4ji4}[][HSK 1]
    \definition{v.}{esquecer | ignorar; negligenciar | sair da memória de alguém; não ser lembrado | descartar da mente; ignorar}
  \end{phonetics}
\end{entry}

\begin{entry}{忘却}{7,7}{⼼、⼙}
  \begin{phonetics}{忘却}{wang4que4}
    \definition{v.}{esquecer}
  \end{phonetics}
\end{entry}

\begin{entry}{忘怀}{7,7}{⼼、⼼}
  \begin{phonetics}{忘怀}{wang4huai2}
    \definition{v.}{esquecer}
  \end{phonetics}
\end{entry}

\begin{entry}{忘恩}{7,10}{⼼、⼼}
  \begin{phonetics}{忘恩}{wang4'en1}
    \definition{v.}{ser ingrato}
  \end{phonetics}
\end{entry}

\begin{entry}{忘掉}{7,11}{⼼、⼿}
  \begin{phonetics}{忘掉}{wang4diao4}
    \definition{v.}{esquecer}
  \end{phonetics}
\end{entry}

\begin{entry}{忘餐}{7,16}{⼼、⾷}
  \begin{phonetics}{忘餐}{wang4can1}
    \definition{v.}{esquecer as refeições}
  \end{phonetics}
\end{entry}

\begin{entry}{忧}{7}{⼼}
  \begin{phonetics}{忧}{you1}
    \definition{s.}{tristeza; ansiedade; preocupação; cuidado; coisas que causam tristeza}
    \definition{v.}{preocupar-se; estar preocupado; estar ansioso; estar triste}
  \end{phonetics}
\end{entry}

\begin{entry}{忧郁}{7,8}{⼼、⾢}
  \begin{phonetics}{忧郁}{you1yu4}
    \definition{adj.}{deprimido | melancólico | desanimado}
    \definition{s.}{depressão | melancolia}
  \end{phonetics}
\end{entry}

\begin{entry}{快}{7}{⼼}
  \begin{phonetics}{快}{kuai4}[][HSK 1]
    \definition*{s.}{Sobrenome Kuai}
    \definition{adj.}{rápido; veloz (oposto a 慢) | apressado | perspicaz; ágil; inteligente; de ​​mente rápida | (de uma faca, espada, etc.) afiado (oposto a 钝) | direto; franco; sem rodeios | satisfeito; feliz; gratificado | rápido; veloz; alta velocidade; tempo de execução curto | satisfeito; feliz; contente | engenhoso; ágil | afiado; facas, tesouras, machados e outros objetos afiados | sincero}
    \definition{adv.}{em breve; antes de muito tempo; estar prestes a | rapidamente}
    \definition{s.}{policial; polícia | (antigo) oficial encarregado de efetuar prisões}
  \seealsoref{钝}{dun4}
  \seealsoref{慢}{man4}
  \end{phonetics}
\end{entry}

\begin{entry}{快车}{7,4}{⼼、⾞}
  \begin{phonetics}{快车}{kuai4 che1}[][HSK 6]
    \definition{s.}{trem ou ônibus expresso (em oposição a 慢车); um trem ou ônibus com menos paradas e tempos de viagem mais curtos (usado principalmente para transporte de passageiros)}
  \seealsoref{慢车}{man4 che1}
  \end{phonetics}
\end{entry}

\begin{entry}{快乐}{7,5}{⼼、⼃}
  \begin{phonetics}{快乐}{kuai4le4}[][HSK 2]
    \definition{adj.}{feliz; alegre; animado; prazeiroso}
    \definition{s.}{felicidade | alegria}
  \end{phonetics}
\end{entry}

\begin{entry}{快活}{7,9}{⼼、⽔}
  \begin{phonetics}{快活}{kuai4huo5}[][HSK 5]
    \definition{adj.}{feliz; alegre; contente; animado}
  \end{phonetics}
\end{entry}

\begin{entry}{快点儿}{7,9,2}{⼼、⽕、⼉}
  \begin{phonetics}{快点儿}{kuai4 dian3r5}[][HSK 2]
    \definition{v.}{apressar-se}
  \end{phonetics}
\end{entry}

\begin{entry}{快要}{7,9}{⼼、⾑}
  \begin{phonetics}{快要}{kuai4 yao4}[][HSK 2]
    \definition{adv.}{estar prestes a; estar indo para; estar à beira de; em breve; em pouco tempo; indica que a situação está prestes a ocorrer}
  \end{phonetics}
\end{entry}

\begin{entry}{快递}{7,10}{⼼、⾡}
  \begin{phonetics}{快递}{kuai4 di4}[][HSK 4]
    \definition[个]{s.}{correio rápido; entrega expressa; entrega rápida}
    \definition{v.}{entregar (serviço de entrega rápida por transportadoras especializadas)}
  \end{phonetics}
\end{entry}

\begin{entry}{快速}{7,10}{⼼、⾡}
  \begin{phonetics}{快速}{kuai4 su4}[][HSK 3]
    \definition{adj.}{rápido; veloz; de alta velocidade; descreve o tempo curto gasto para caminhar, fazer algo, etc.}
  \end{phonetics}
\end{entry}

\begin{entry}{快餐}{7,16}{⼼、⾷}
  \begin{phonetics}{快餐}{kuai4 can1}[][HSK 2]
    \definition[份,顿]{s.}{pedido (comida) rápido; \emph{fast food}; refere-se a refeições simples preparadas com antecedência e que podem ser servidas rapidamente}
  \end{phonetics}
\end{entry}

\begin{entry}{怀}{7}{⼼}
  \begin{phonetics}{怀}{huai2}
    \definition*{s.}{Sobrenome Huai}
    \definition{s.}{seio; peito | mente}
    \definition{v.}{manter em mente; estimar; abrigar | sentir falta; pensar em; ansiar por | conceber (uma criança)}
  \end{phonetics}
\end{entry}

\begin{entry}{怀旧}{7,5}{⼼、⽇}
  \begin{phonetics}{怀旧}{huai2jiu4}
    \definition{s.}{nostalgia}
    \definition{v.}{sentir-se nostálgico}
  \end{phonetics}
\end{entry}

\begin{entry}{怀念}{7,8}{⼼、⼼}
  \begin{phonetics}{怀念}{huai2nian4}[][HSK 4]
    \definition{v.}{pensar em; valorizar a memória de}
  \end{phonetics}
\end{entry}

\begin{entry}{怀疑}{7,14}{⼼、⽦}
  \begin{phonetics}{怀疑}{huai2yi2}[][HSK 4]
    \definition{v.}{duvidar; suspeitar | supor}
  \end{phonetics}
\end{entry}

\begin{entry}{念}{8}{⼼}
  \begin{phonetics}{念}{nian4}[][HSK 3]
    \definition*{s.}{Sobrenome Nian}
    \definition{num.}{vinte; 20; capitalização do número 廿}
    \definition{s.}{ideia; pensamento; pensamentos ou intenções internas}
    \definition{v.}{ler em voz alta | estudar; frequentar a escola | considerar; levar em conta | sentir falta; pensar em; pensar sobre; pensar frequentemente sobre}
  \seealsoref{廿}{nian4}
  \end{phonetics}
\end{entry}

\begin{entry}{忽}{8}{⼼}
  \begin{phonetics}{忽}{hu1}
    \definition*{s.}{Sobrenome Hu}
    \definition{adv.}{agora\dots, agora\dots | de repente; subitamente}[天气忽冷忽热。___O clima está frio em um minuto e quente no outro.]
    \definition{v.}{negligenciar; ignorar; não prestar atenção; não levar a sério}
  \end{phonetics}
\end{entry}

\begin{entry}{忽视}{8,8}{⼼、⾒}
  \begin{phonetics}{忽视}{hu1shi4}[][HSK 4]
    \definition{v.}{ignorar; negligenciar; menosprezar; desprezar; dar de ombros}
  \end{phonetics}
\end{entry}

\begin{entry}{忽然}{8,12}{⼼、⽕}
  \begin{phonetics}{忽然}{hu1ran2}[][HSK 2]
    \definition{adv.}{repentinamente; de repente; sem aviso prévio; significa que algo aconteceu de forma rápida e inesperada}
  \end{phonetics}
\end{entry}

\begin{entry}{态}{8}{⼼}
  \begin{phonetics}{态}{tai4}
    \definition{s.}{forma; aparência; condição | (física) estado | (linguística) voz}[气态___estado gasoso | 被动态___voz passiva]
  \end{phonetics}
\end{entry}

\begin{entry}{态度}{8,9}{⼼、⼴}
  \begin{phonetics}{态度}{tai4du5}[][HSK 2]
    \definition[种,个]{s.}{maneira; comportamento; atitude; comportamento e expressão facial das pessoas | atitude; abordagem; opinião sobre o assunto e medidas tomadas}
  \end{phonetics}
\end{entry}

\begin{entry}{怕}{8}{⼼}
  \begin{phonetics}{怕}{pa4}[][HSK 2]
    \definition{adv.}{(expressando suposição, julgamento, estimativa, etc.) talvez; suponho; receio (que)}
    \definition{adv.}{por medo; talvez; suponho}
    \definition{v.}{temer; ter medo; recear; sentir medo, ficar nervoso | estar preocupado com; estar preocupado por (ou sobre); ter medo de que algo possa acontecer | ser afetado por; não conseguir suportar; não aguentar mais}
  \end{phonetics}
\end{entry}

\begin{entry}{性}{8}{⼼}
  \begin{phonetics}{性}{xing4}[][HSK 3]
    \definition[个]{s.}{natureza; caráter; personalidade | propriedade; qualidade; natureza e características das coisas | sexo; gênero | sexualidade; relacionado com a reprodução e a sexualidade | caráter; temperamento}
    \definition{suf.}{indica uma determinada propriedade ou característica de algo; segue um substantivo, verbo ou adjetivo, formando um substantivo abstrato ou um adjetivo que expressa uma propriedade}
  \end{phonetics}
\end{entry}

\begin{entry}{性生活}{8,5,9}{⼼、⽣、⽔}
  \begin{phonetics}{性生活}{xing4sheng1huo2}
    \definition{s.}{vida sexual}
  \end{phonetics}
\end{entry}

\begin{entry}{性别}{8,7}{⼼、⼑}
  \begin{phonetics}{性别}{xing4bie2}[][HSK 3]
    \definition[种]{s.}{sexo; gênero}
  \end{phonetics}
\end{entry}

\begin{entry}{性质}{8,8}{⼼、⾙}
  \begin{phonetics}{性质}{xing4zhi4}[][HSK 4]
    \definition[个,种,类]{s.}{natureza; qualidade; caráter; propriedade; propriedade fundamental que distingue uma coisa de outra}
  \end{phonetics}
\end{entry}

\begin{entry}{性侵}{8,9}{⼼、⼈}
  \begin{phonetics}{性侵}{xing4qin1}
    \definition{s.}{agressão sexual}
    \definition{v.}{agredir sexualmente}
  \end{phonetics}
\end{entry}

\begin{entry}{性格}{8,10}{⼼、⽊}
  \begin{phonetics}{性格}{xing4ge2}[][HSK 3]
    \definition[种,个]{s.}{caráter; temperamento; as características psicológicas manifestadas na atitude e no comportamento em relação às pessoas e às coisas}
  \end{phonetics}
\end{entry}

\begin{entry}{性能}{8,10}{⼼、⾁}
  \begin{phonetics}{性能}{xing4neng2}[][HSK 5]
    \definition{s.}{natureza; propriedade; desempenho; função (de uma máquina, etc.); grau de conformidade dos produtos mecânicos ou outros produtos industriais com os requisitos de projeto}
  \end{phonetics}
\end{entry}

\begin{entry}{怪}{8}{⼼}
  \begin{phonetics}{怪}{guai4}[][HSK 4,5]
    \definition*{s.}{Sobrenome Guai}
    \definition{adj.}{estranho; esquisito; desconcertante | peculiar; excêntrico; pitoresco; monstruoso; anormal; incomum}
    \definition{adv.}{bastante; muito}
    \definition{s.}{monstro; demônio | diabo; ser maligno}
    \definition{v.}{culpar | achar algo estranho; maravilhar-se com; ficar surpreso | repreender; culpar; reclamar}
  \end{phonetics}
\end{entry}

\begin{entry}{怪兽}{8,11}{⼼、⼋}
  \begin{phonetics}{怪兽}{guai4shou4}
    \definition{s.}{animal raro | animal mítico | monstro}
  \end{phonetics}
\end{entry}

\begin{entry}{怪癖}{8,18}{⼼、⽧}
  \begin{phonetics}{怪癖}{guai4pi3}
    \definition{adj.}{peculiar}
    \definition{s.}{excentricidade | peculiaridade | hobby estranho}
  \end{phonetics}
\end{entry}

\begin{entry}{怎}{9}{⼼}
  \begin{phonetics}{怎}{zen3}
    \definition{adv.}{como}
  \end{phonetics}
\end{entry}

\begin{entry}{怎么}{9,3}{⼼、⼃}
  \begin{phonetics}{怎么}{zen3me5}[][HSK 1]
    \definition{pron.}{como?; o quê?; perguntas sobre natureza, situação, método, motivo, etc. | de qualquer maneira; não importa como; de uma certa maneira; referência geral à natureza, condição ou modo | que? (usado sozinho no início de uma frase para expressar surpresa) | usado após 不 e 没, indica um grau baixo e é uma forma mais educada de se expressar | usado em perguntas retóricas}
  \seealsoref{不}{bu4}
  \seealsoref{没}{mei2}
  \end{phonetics}
\end{entry}

\begin{entry}{怎么了}{9,3,2}{⼼、⼃、⼅}
  \begin{phonetics}{怎么了}{zen3me5le5}
    \definition{expr.}{O que aconteceu? | O que está acontecendo? | E aí?}
  \end{phonetics}
\end{entry}

\begin{entry}{怎么办}{9,3,4}{⼼、⼃、⼒}
  \begin{phonetics}{怎么办}{zen3 me5 ban4}[][HSK 2]
    \definition{adv.}{o que fazer?; o que deve ser feito?}
  \end{phonetics}
\end{entry}

\begin{entry}{怎么回事}{9,3,6,8}{⼼、⼃、⼞、⼅}
  \begin{phonetics}{怎么回事}{zen3me5hui2shi4}
    \definition{expr.}{O que aconteceu? | O que se passou?}
  \end{phonetics}
\end{entry}

\begin{entry}{怎么样}{9,3,10}{⼼、⼃、⽊}
  \begin{phonetics}{怎么样}{zen3me5yang4}[][HSK 2]
    \definition{adv.}{como?; o que?; como é?; como estão as coisas?; o que você acha?; pergunte sobre o método, natureza, situação, opinião, etc. | substitui uma ação ou situação não dita (usado apenas na forma negativa, mais eufemístico do que uma declaração direta); indaga sobre a natureza, condição, método, razão, etc.}
  \end{phonetics}
\end{entry}

\begin{entry}{怎么得了}{9,3,11,2}{⼼、⼃、⼻、⼅}
  \begin{phonetics}{怎么得了}{zen3me5de2liao3}
    \definition{expr.}{Como isso pode ser? | Que bagunça horrível! | O que deve ser feito?}
  \end{phonetics}
\end{entry}

\begin{entry}{怎么搞的}{9,3,13,8}{⼼、⼃、⼿、⽩}
  \begin{phonetics}{怎么搞的}{zen3me5gao3de5}
    \definition{expr.}{Como isso aconteceu? | O que deu errado? | E aí? | O que está errado?}
  \end{phonetics}
\end{entry}

\begin{entry}{怎样}{9,10}{⼼、⽊}
  \begin{phonetics}{怎样}{zen3 yang4}[][HSK 2]
    \definition{pron.}{como?; o que?; indagar sobre a natureza, condição ou método, etc. | como?; indica uma referência virtual | de uma certa maneira; de qualquer maneira; não importa como; indica qualquer | como?; usado como predicado, objeto ou complemento para indagar sobre uma situação}
  \end{phonetics}
\end{entry}

\begin{entry}{怒}{9}{⼼}
  \begin{phonetics}{怒}{nu4}
    \definition{adj.}{zangado; furioso | feroz; forte; descreve um forte impulso}
    \definition{adv.}{com força; vigorosamente; dinamicamente | com raiva}
    \definition{s.}{raiva; fúria}
    \definition{v.}{enfurecer-se; ficar com raiva}
  \end{phonetics}
\end{entry}

\begin{entry}{怒放}{9,8}{⼼、⽅}
  \begin{phonetics}{怒放}{nu4fang4}
    \definition{v.}{florescer em plena floração}
  \end{phonetics}
\end{entry}

\begin{entry}{怒骂}{9,9}{⼼、⾺}
  \begin{phonetics}{怒骂}{nu4ma4}
    \definition{v.}{praguejar de raiva}
  \end{phonetics}
\end{entry}

\begin{entry}{思}{9}{⼼}
  \begin{phonetics}{思}{si1}
    \definition*{s.}{Sobrenome Si}
    \definition{s.}{pensamento; ideias | pensamentos; emoções; humor}
    \definition{v.}{pensar; considerar; deliberar | pensar em; ansiar por}
  \end{phonetics}
\end{entry}

\begin{entry}{思考}{9,6}{⼼、⽼}
  \begin{phonetics}{思考}{si1kao3}[][HSK 4]
    \definition{v.}{pensar; ponderar; considerar; deliberar; envolver-se em atividades de pensamento, como análise, síntese, julgamento, raciocínio e generalização}
  \end{phonetics}
\end{entry}

\begin{entry}{思维}{9,11}{⼼、⽷}
  \begin{phonetics}{思维}{si1wei2}[][HSK 5]
    \definition[种]{s.}{pensamento; reflexão; organizar e transformar os materiais obtidos através do conhecimento sensorial para formar conceitos, julgamentos e raciocínios}
    \definition{v.}{pensar;}
  \end{phonetics}
\end{entry}

\begin{entry}{思想}{9,13}{⼼、⼼}
  \begin{phonetics}{思想}{si1xiang3}[][HSK 3]
    \definition[个,种]{s.}{reflexão; pensamento; ideologia; a existência objetiva é refletida na consciência das pessoas por meio de atividades de pensamento, que pertencem à cognição racional | ideia; pensamento}
  \end{phonetics}
\end{entry}

\begin{entry}{急}{9}{⼼}
  \begin{phonetics}{急}{ji2}[][HSK 2]
    \definition{adj.}{impaciente; ansioso | irritado; aborrecido; incomodado | rápido e intenso (em oposição a 缓); veloz | urgente; premente}
    \definition{s.}{urgência; emergência; assunto urgente e grave}
    \definition{v.}{preocupar; deixar ansioso | estar ansioso para ajudar; tratar os problemas dos outros como se fossem urgentes e ajudar a resolvê-los imediatamente}
  \seealsoref{缓}{huan3}
  \end{phonetics}
\end{entry}

\begin{entry}{急忙}{9,6}{⼼、⼼}
  \begin{phonetics}{急忙}{ji2mang2}[][HSK 4]
    \definition{adv.}{apressadamente; com pressa}
  \end{phonetics}
\end{entry}

\begin{entry}{急救}{9,11}{⼼、⽁}
  \begin{phonetics}{急救}{ji2jiu4}
    \definition{s.}{primeiros socorros}
    \definition{v.}{dar tratamento de emergência}
  \end{phonetics}
\end{entry}

\begin{entry}{怨}{9}{⼼}
  \begin{phonetics}{怨}{yuan4}[][HSK 5]
    \definition{s.}{ressentimento; inimizade; rancor}
    \definition{v.}{culpar; reclamar}
  \end{phonetics}
\end{entry}

\begin{entry}{怹}{9}{⼼}
  \begin{phonetics}{怹}{tan1}
    \definition{pron.}{ele, ela (cortês, em oposição a 他)}
  \seealsoref{他}{ta1}
  \end{phonetics}
\end{entry}

\begin{entry}{总}{9}{⼼}
  \begin{phonetics}{总}{zong3}[][HSK 3]
    \definition{adj.}{total; geral; global | responsável (liderança)}
    \definition{adv.}{sempre; invariavelmente | de qualquer forma; afinal; eventualmente; mais cedo ou mais tarde; no fim das contas | certamente; provavelmente; com certeza; expressa estimativa; suposição; equivalente a 大概}
    \definition{v.}{reunir; resumir; juntar; compilar}
  \seealsoref{大概}{da4gai4}
  \end{phonetics}
\end{entry}

\begin{entry}{总之}{9,3}{⼼、⼂}
  \begin{phonetics}{总之}{zong3zhi1}[][HSK 4]
    \definition{conj.}{em uma palavra; em suma; em resumo; indica que a declaração seguinte é uma declaração geral}
  \end{phonetics}
\end{entry}

\begin{entry}{总长}{9,4}{⼼、⾧}
  \begin{phonetics}{总长}{zong3chang2}
    \definition{s.}{comprimento total}
  \end{phonetics}
\end{entry}

\begin{entry}{总务}{9,5}{⼼、⼒}
  \begin{phonetics}{总务}{zong3wu4}
    \definition{s.}{divisão de assuntos gerais | assuntos gerais | pessoa responsável geral}
  \end{phonetics}
\end{entry}

\begin{entry}{总台}{9,5}{⼼、⼝}
  \begin{phonetics}{总台}{zong3tai2}
    \definition{s.}{recepção | balcão de recepção}
  \end{phonetics}
\end{entry}

\begin{entry}{总价}{9,6}{⼼、⼈}
  \begin{phonetics}{总价}{zong3jia4}
    \definition{s.}{preço total}
  \end{phonetics}
\end{entry}

\begin{entry}{总共}{9,6}{⼼、⼋}
  \begin{phonetics}{总共}{zong3gong4}[][HSK 4]
    \definition{adv.}{em tudo; em todos; no total; completamente; totalmente; em conjunto}
  \end{phonetics}
\end{entry}

\begin{entry}{总体}{9,7}{⼼、⼈}
  \begin{phonetics}{总体}{zong3 ti3}[][HSK 5]
    \definition{s.}{total; geral; conjunto; totalidade; massa; população; o todo formado pela união de vários indivíduos; a totalidade das coisas}
  \end{phonetics}
\end{entry}

\begin{entry}{总线}{9,8}{⼼、⽷}
  \begin{phonetics}{总线}{zong3xian4}
    \definition{s.}{barramento (computador) | \emph{computer bus}}
  \end{phonetics}
\end{entry}

\begin{entry}{总是}{9,9}{⼼、⽇}
  \begin{phonetics}{总是}{zong3shi4}[][HSK 3]
    \definition{adv.}{sempre; indica como tem sido durante um determinado período de tempo; um determinado estado permanece inalterado | afinal; significa que, independentemente do que acontecer, haverá ou será um resultado}
  \end{phonetics}
\end{entry}

\begin{entry}{总结}{9,9}{⼼、⽷}
  \begin{phonetics}{总结}{zong3jie2}[][HSK 3]
    \definition[个,篇]{s.}{resumo; síntese; conclusão resumida}
    \definition{v.}{resumir; sumariar; sintetizar; analisar e estudar as experiências para chegar a conclusões}
  \end{phonetics}
\end{entry}

\begin{entry}{总统}{9,9}{⼼、⽷}
  \begin{phonetics}{总统}{zong3tong3}[][HSK 4]
    \definition*[个,位,名]{s.}{Presidente (de um país); Título dos líderes de determinadas repúblicas}
  \end{phonetics}
\end{entry}

\begin{entry}{总值}{9,10}{⼼、⼈}
  \begin{phonetics}{总值}{zong3zhi2}
    \definition{s.}{valor total}
  \end{phonetics}
\end{entry}

\begin{entry}{总站}{9,10}{⼼、⽴}
  \begin{phonetics}{总站}{zong3zhan4}
    \definition{s.}{terminal}
  \end{phonetics}
\end{entry}

\begin{entry}{总得}{9,11}{⼼、⼻}
  \begin{phonetics}{总得}{zong3dei3}
    \definition{adv.}{prestes a}
    \definition{v.}{dever | precisar}
  \end{phonetics}
\end{entry}

\begin{entry}{总理}{9,11}{⼼、⽟}
  \begin{phonetics}{总理}{zong3li3}[][HSK 4]
    \definition*[个,位,名]{s.}{Primeiro-Ministro do Conselho de Estado; Título do líder do Conselho de Estado da China | Título do chefe de governo em determinados países | Primeiro-Ministro; Título de líderes de determinados partidos políticos | Título dos chefes de determinadas instituições e empresas nos velhos tempos}
    \definition{v.}{assumir a responsabilidade total;}
  \end{phonetics}
\end{entry}

\begin{entry}{总裁}{9,12}{⼼、⾐}
  \begin{phonetics}{总裁}{zong3cai2}[][HSK 5]
    \definition[位,名]{s.}{presidente (de uma empresa); nomes de certos líderes de partidos políticos ou grandes empresas}
  \end{phonetics}
\end{entry}

\begin{entry}{总数}{9,13}{⼼、⽁}
  \begin{phonetics}{总数}{zong3 shu4}[][HSK 5]
    \definition{s.}{soma; total; totalidade; inventário; número total; soma total}
  \end{phonetics}
\end{entry}

\begin{entry}{总督}{9,13}{⼼、⽬}
  \begin{phonetics}{总督}{zong3du1}
    \definition*{s.}{Governador-Geral | Governador | Vice-Rei}
  \end{phonetics}
\end{entry}

\begin{entry}{总算}{9,14}{⼼、⽵}
  \begin{phonetics}{总算}{zong3suan4}[][HSK 5]
    \definition{adv.}{finalmente; por fim; indica que, após um longo período de tempo, um desejo finalmente se tornou realidade | suficiente; considerando tudo; no geral; considerando todos os aspectos; significa que, em geral, está tudo bem}
  \end{phonetics}
\end{entry}

\begin{entry}{恒}{9}{⼼}
  \begin{phonetics}{恒}{heng2}
    \definition*{s.}{Sobrenome Heng}
    \definition{adj.}{permanente; duradouro | usual; comum; constante | usual; frequente; constante}
    \definition{s.}{perseverança; constância}
  \end{phonetics}
\end{entry}

\begin{entry}{恒星系}{9,9,7}{⼼、⽇、⽷}
  \begin{phonetics}{恒星系}{heng2xing1xi4}
    \definition{s.}{sistema estelar | galáxia}
  \end{phonetics}
\end{entry}

\begin{entry}{恢}{9}{⼼}
  \begin{phonetics}{恢}{hui1}
    \definition{adj.}{extenso; vasto | grande; ótimo}
    \definition{v.}{recuperar; restaurar; restabelecer}
  \end{phonetics}
\end{entry}

\begin{entry}{恢复}{9,9}{⼼、⼢}
  \begin{phonetics}{恢复}{hui1fu4}[][HSK 5]
    \definition{v.}{retomar; renovar; restaurar; voltar a | reviver; recuperar; reencontrar | restaurar; restabelecer; reabilitar; regenerar; ressurgir; restabelecer alguém em; recuperar o que foi perdido}
  \end{phonetics}
\end{entry}

\begin{entry}{恤}{9}{⼼}
  \begin{phonetics}{恤}{xu4}
    \definition{v.}{ter pena; simpatizar | dar alívio; compensar}
  \end{phonetics}
\end{entry}

\begin{entry}{恨}{9}{⼼}
  \begin{phonetics}{恨}{hen4}[][HSK 5]
    \definition{s.}{ódio; resentimento}
    \definition{v.}{odiar; ressentir-se}
  \end{phonetics}
\end{entry}

\begin{entry}{恰}{9}{⼼}
  \begin{phonetics}{恰}{qia4}
    \definition{adv.}{exatamente | apenas}
  \end{phonetics}
\end{entry}

\begin{entry}{恰好}{9,6}{⼼、⼥}
  \begin{phonetics}{恰好}{qia4hao3}
    \definition{adv.}{certo | por sorte | ao que parece | por sorte coincidência}
  \end{phonetics}
\end{entry}

\begin{entry}{恰到好处}{9,8,6,5}{⼼、⼑、⼥、⼡}
  \begin{phonetics}{恰到好处}{qia4dao4hao3chu4}
    \definition{expr.}{é simplesmente perfeito | é simplesmente correto}
  \end{phonetics}
\end{entry}

\begin{entry}{恋}{10}{⼼}
  \begin{phonetics}{恋}{lian4}
    \definition*{s.}{Sobrenome Lian}
    \definition{v.}{amor (romântico) | ansiar por; sentir-se apegado a | amar; apaixonar-se por | não querendo se separar de; sentir sua falta para sempre; não suportar ficar separado}
  \end{phonetics}
\end{entry}

\begin{entry}{恋爱}{10,10}{⼼、⽖}
  \begin{phonetics}{恋爱}{lian4'ai4}[][HSK 5]
    \definition[个,场,段]{s.}{namoro; afeto; amor romântico; ações que demonstram o amor mútuo}
    \definition{v.}{amar; estar apaixonado}
  \end{phonetics}
\end{entry}

\begin{entry}{恐}{10}{⼼}
  \begin{phonetics}{恐}{kong3}
    \definition{adv.}{talvez; provavelmente}
    \definition{v.}{temer; recear; ter medo de | ameaçar; aterrorizar; intimidar}
  \end{phonetics}
\end{entry}

\begin{entry}{恐龙}{10,5}{⼼、⿓}
  \begin{phonetics}{恐龙}{kong3long2}
    \definition[头,只]{s.}{dinossauro}
  \end{phonetics}
\end{entry}

\begin{entry}{恐怕}{10,8}{⼼、⼼}
  \begin{phonetics}{恐怕}{kong3pa4}[][HSK 3]
    \definition{adv.}{talvez; provavelmente; pode ser; expressa suposição; estimativa. | por medo de; expressar estimativa e preocupação}
    \definition{v.}{ter medo de; temer; recear}
  \end{phonetics}
\end{entry}

\begin{entry}{恐怖主义}{10,8,5,3}{⼼、⼼、⼂、⼂}
  \begin{phonetics}{恐怖主义}{kong3bu4zhu3yi4}
    \definition{adj.}{terrorista}
    \definition{s.}{terrorismo}
  \end{phonetics}
\end{entry}

\begin{entry}{恩}{10}{⼼}
  \begin{phonetics}{恩}{en1}
    \definition*{s.}{Sobrenome En}
    \definition{s.}{bondade; favor; graça; gentileza}
  \end{phonetics}
\end{entry}

\begin{entry}{恩赐}{10,12}{⼼、⾙}
  \begin{phonetics}{恩赐}{en1ci4}
    \definition{s.}{favor | caridade}
    \definition{v.}{conceder (favor, caridade)}
  \end{phonetics}
\end{entry}

\begin{entry}{恶}{10}{⼼}
  \begin{phonetics}{恶}{e3}
    \definition{part.}{elementos formadores de palavras}
  \end{phonetics}
  \begin{phonetics}{恶}{e4}
    \definition{adj.}{feroz | ruim; maligno; perverso | vicioso | feio | grosseiro}
    \definition{s.}{mal; vício; crime (oposto a 善) | maldade; comportamento muito ruim; coisas criminosas}
  \seealsoref{善}{shan4}
  \end{phonetics}
  \begin{phonetics}{恶}{wu4}
    \definition{v.}{não gostar; odiar; detestar; repugnar}
  \end{phonetics}
\end{entry}

\begin{entry}{恶心}{10,4}{⼼、⼼}
  \begin{phonetics}{恶心}{e3xin1}[][HSK 4]
    \definition{adj.}{nauseante; repugnante}
    \definition{s.}{enjoo; náusea; repugnância; sensação de enjoo; vontade de vomitar}
    \definition{v.}{repugnar; ser nauseante; vomitar}
  \end{phonetics}
  \begin{phonetics}{恶心}{e4xin1}
    \definition{s.}{mau hábito | hábito vicioso | vício}
  \end{phonetics}
\end{entry}

\begin{entry}{悄}{10}{⼼}
  \begin{phonetics}{悄}{qiao1}
    \definition{adj.}{quieto; silencioso}
  \end{phonetics}
  \begin{phonetics}{悄}{qiao3}
    \definition{adj.}{quieto; silencioso | triste; preocupado; aflito}
  \end{phonetics}
\end{entry}

\begin{entry}{悄悄}{10,10}{⼼、⼼}
  \begin{phonetics}{悄悄}{qiao1qiao1}[][HSK 5]
    \definition{adv.}{silenciosamente; em silêncio; aos sussuros; sem som ou em voz baixa; com o mínimo de ruído possível}
  \end{phonetics}
\end{entry}

\begin{entry}{悉}{11}{⼼}
  \begin{phonetics}{悉}{xi1}
    \definition*{s.}{Sobrenome Xi}
    \definition{adj.}{tudo; inteiro; total | detalhado}
    \definition{v.}{saber; aprender; ser informado de}
  \end{phonetics}
\end{entry}

\begin{entry}{悉心}{11,4}{⼼、⼼}
  \begin{phonetics}{悉心}{xi1xin1}
    \definition{adv.}{colocar o coração (e a alma) em algo | com muito cuidado}
  \end{phonetics}
\end{entry}

\begin{entry}{悉尼}{11,5}{⼼、⼫}
  \begin{phonetics}{悉尼}{xi1ni2}
    \definition*{s.}{Sidney}
  \end{phonetics}
\end{entry}

\begin{entry}{悉数}{11,13}{⼼、⽁}
  \begin{phonetics}{悉数}{xi1shu3}
    \definition{adv.}{enumerar em detalhes | explicar claramente}
  \end{phonetics}
  \begin{phonetics}{悉数}{xi1shu4}
    \definition{adv.}{todos | cada um | toda a soma}
  \end{phonetics}
\end{entry}

\begin{entry}{您}{11}{⼼}
  \begin{phonetics}{您}{nin2}[][HSK 1]
    \definition{pron.}{você; a forma de tratamento respeitosa da segunda pessoa do singular 你}
  \seealsoref{你}{ni3}
  \end{phonetics}
\end{entry}

\begin{entry}{悬}{11}{⼼}
  \begin{phonetics}{悬}{xuan2}[][HSK 6]
    \definition{adj.}{pendente; não resolvido; sem nenhum resultado | distante; a distância é grande; a diferença é grande | (dialeto) perigoso}
    \definition{v.}{pendurar; suspender | levantar; elevar | sentir-se ansioso; ser solícito | imaginar}
  \end{phonetics}
\end{entry}

\begin{entry}{悬挂}{11,9}{⼼、⼿}
  \begin{phonetics}{悬挂}{xuan2gua4}
    \definition{s.}{(veículo) suspensão}
    \definition{v.}{suspender}
  \end{phonetics}
\end{entry}

\begin{entry}{悬崖}{11,11}{⼼、⼭}
  \begin{phonetics}{悬崖}{xuan2ya2}
    \definition{s.}{precipício | penhasco}
  \end{phonetics}
\end{entry}

\begin{entry}{情}{11}{⼼}
  \begin{phonetics}{情}{qing2}
    \definition{s.}{sentimento; afeição | amor; paixão | paixão sexual; luxúria | favor; gentileza | situação; circunstâncias; condição | razão; sentido | sensibilidades; sentimentos}
  \end{phonetics}
\end{entry}

\begin{entry}{情节}{11,5}{⼼、⾋}
  \begin{phonetics}{情节}{qing2jie2}[][HSK 5]
    \definition{s.}{enredo; trama; desenrolar específico dos acontecimentos | circunstância; detalhes do crime ou erro | enredo; roteiro; refere-se especificamente ao processo de desenvolvimento e evolução dos conflitos e contradições em obras literárias narrativas}
  \end{phonetics}
\end{entry}

\begin{entry}{情况}{11,7}{⼼、⼎}
  \begin{phonetics}{情况}{qing2kuang4}[][HSK 3]
    \definition[种,个,些]{s.}{condição; situação; circunstâncias; estado das coisas | mudanças notáveis e impactantes}
  \end{phonetics}
\end{entry}

\begin{entry}{情形}{11,7}{⼼、⼺}
  \begin{phonetics}{情形}{qing2xing2}[][HSK 5]
    \definition[个]{s.}{situação; condição; circunstâncias; estado de coisas; a situação específica das coisas}
  \end{phonetics}
\end{entry}

\begin{entry}{情绪}{11,11}{⼼、⽷}
  \begin{phonetics}{情绪}{qing2xu4}
    \definition[种]{s.}{humor | estado da mente | mau humor}
  \end{phonetics}
\end{entry}

\begin{entry}{情景}{11,12}{⼼、⽇}
  \begin{phonetics}{情景}{qing2jing3}[][HSK 4]
    \definition[个]{s.}{cena; vista; circunstâncias}
  \end{phonetics}
\end{entry}

\begin{entry}{情感}{11,13}{⼼、⼼}
  \begin{phonetics}{情感}{qing2 gan3}[][HSK 3]
    \definition[份]{s.}{emoção; sentimento | afeição; apego; reações psicológicas positivas ou negativas a estímulos externos, como gosto, raiva, tristeza, medo, amor, nojo, etc.}
  \end{phonetics}
\end{entry}

\begin{entry}{惊}{11}{⼼}
  \begin{phonetics}{惊}{jing1}
    \definition{v.}{assustar; ficar assustado; ficar nervoso devido a estímulo repentino; ficar com medo | surpreender; chocar; alarmar}
  \end{phonetics}
\end{entry}

\begin{entry}{惊呆}{11,7}{⼼、⼝}
  \begin{phonetics}{惊呆}{jing1dai1}
    \definition{adj.}{estupefato | chocado}
  \end{phonetics}
\end{entry}

\begin{entry}{惊喜}{11,12}{⼼、⼝}
  \begin{phonetics}{惊喜}{jing1xi3}
    \definition{s.}{boa surpresa}
    \definition{v.}{ser agradavelmente surpreendido}
  \end{phonetics}
\end{entry}

\begin{entry}{惑}{12}{⼼}
  \begin{phonetics}{惑}{huo4}
    \definition{v.}{ficar confuso; ficar perplexo | iludir; enganar; confundir}
  \end{phonetics}
\end{entry}

\begin{entry}{惑星}{12,9}{⼼、⽇}
  \begin{phonetics}{惑星}{huo4xing1}
    \definition{s.}{planeta}
  \seealsoref{行星}{xing2xing1}
  \end{phonetics}
\end{entry}

\begin{entry}{惩}{12}{⼼}
  \begin{phonetics}{惩}{cheng2}
    \definition{v.}{receber ou dar aviso | punir; penalizar}
  \end{phonetics}
\end{entry}

\begin{entry}{惩处}{12,5}{⼼、⼡}
  \begin{phonetics}{惩处}{cheng2chu3}
    \definition{v.}{administrar justiça | punir}
  \end{phonetics}
\end{entry}

\begin{entry}{惩罚}{12,9}{⼼、⽹}
  \begin{phonetics}{惩罚}{cheng2fa2}
    \definition{v.}{punir | penalizar}
  \end{phonetics}
\end{entry}

\begin{entry}{愉}{12}{⼼}
  \begin{phonetics}{愉}{yu2}
    \definition{adj.}{satisfeito; feliz; alegre}
  \end{phonetics}
\end{entry}

\begin{entry}{愉快}{12,7}{⼼、⼼}
  \begin{phonetics}{愉快}{yu2kuai4}
    \definition{adj.}{alegre | delicioso | prazeroso | agradável | feliz | encantado}
    \definition{adv.}{alegremente | agradavelmente}
  \end{phonetics}
\end{entry}

\begin{entry}{愤}{12}{⼼}
  \begin{phonetics}{愤}{fen4}
    \definition{s.}{raiva; indignação; ressentimento; exasperação}
    \definition{v.}{ressentir-se; ficar indignado; ficar com raiva}
  \end{phonetics}
\end{entry}

\begin{entry}{愤世嫉俗}{12,5,13,9}{⼼、⼀、⼥、⼈}
  \begin{phonetics}{愤世嫉俗}{fen4shi4ji2su2}
    \definition{v.}{ser cínico | ser amargurado}
  \end{phonetics}
\end{entry}

\begin{entry}{愤怒}{12,9}{⼼、⼼}
  \begin{phonetics}{愤怒}{fen4nu4}
    \definition{adj.}{zangado | indignado}
    \definition{s.}{ira}
  \end{phonetics}
\end{entry}

\begin{entry}{慌}{12}{⼼}
  \begin{phonetics}{慌}{huang1}[][HSK 5]
    \definition{adj.}{agitado; confuso; que inspira terror}
    \definition{v.}{ficar com medo; ficar nervoso}
  \end{phonetics}
\end{entry}

\begin{entry}{慌忙}{12,6}{⼼、⼼}
  \begin{phonetics}{慌忙}{huang1 mang2}[][HSK 5]
    \definition{adj.}{apressado; afobado; com muita pressa}
    \definition{adv.}{apressadamente}
  \end{phonetics}
\end{entry}

\begin{entry}{想}{13}{⼼}
  \begin{phonetics}{想}{xiang3}[][HSK 1]
    \definition{v.}{pensar; ponderar; refletir | supor; contar; considerar; pensar; estimar | querer; gostaria de; sentir vontade (de fazer algo) | lembrar com saudade; sentir falta}
  \end{phonetics}
\end{entry}

\begin{entry}{想到}{13,8}{⼼、⼑}
  \begin{phonetics}{想到}{xiang3 dao4}[][HSK 2]
    \definition{v.}{pensar em; trazer à mente; ter no coração; ter uma ideia (na mente); ter uma ideia (no coração)}
  \end{phonetics}
\end{entry}

\begin{entry}{想念}{13,8}{⼼、⼼}
  \begin{phonetics}{想念}{xiang3nian4}[][HSK 4]
    \definition{v.}{sentir falta; pensar em; lembrar com carinho; ficar doente por; desejar ver novamente; lembrar com saudade}
  \end{phonetics}
\end{entry}

\begin{entry}{想法}{13,8}{⼼、⽔}
  \begin{phonetics}{想法}{xiang3 fa3}[][HSK 2]
    \definition[种]{s.}{ideia; opinião; pensamento; noção; o que alguém tem em mente; visões e opiniões sobre alguém ou algo obtidas através do pensamento}
    \definition{s.}{maneira de pensar | opinião | noção}
    \definition{v.}{tentar; pensar em uma maneira (de fazer algo); fazer o que puder; encontrar um jeito}
  \end{phonetics}
\end{entry}

\begin{entry}{想起}{13,10}{⼼、⾛}
  \begin{phonetics}{想起}{xiang3 qi3}[][HSK 2]
    \definition{v.}{recordar; lembrar; pensar em; trazer à mente; cruzar pelos pensamentos de alguém; passar pelo pensamento de alguém}
  \end{phonetics}
\end{entry}

\begin{entry}{想象}{13,11}{⼼、⾗}
  \begin{phonetics}{想象}{xiang3xiang4}[][HSK 4]
    \definition[个]{s.}{imaginação; refere-se ao processo mental de processamento e transformação de representações armazenadas na mente para formar novas imagens}
    \definition{v.}{imaginar; vislumbrar; visualizar; refere-se a ter uma imagem concreta de algo que não está na frente dos olhos}
  \end{phonetics}
\end{entry}

\begin{entry}{想想看}{13,13,9}{⼼、⼼、⽬}
  \begin{phonetics}{想想看}{xiang3xiang3kan4}
    \definition{v.}{pensar sobre isso}
  \end{phonetics}
\end{entry}

\begin{entry}{愁}{13}{⼼}
  \begin{phonetics}{愁}{chou2}[][HSK 5]
    \definition{adj.}{triste; pesaroso; angustiado; desconsolado; preocupado; deprimido}
    \definition{s.}{pesar; sofrimento; dor; tristeza}
    \definition{v.}{preocupar-se; estar preocupado; ficar ansioso; sentir ansiedade}
  \end{phonetics}
\end{entry}

\begin{entry}{愈}{13}{⼼}
  \begin{phonetics}{愈}{yu4}
    \definition{adv.}{mais e mais | ainda mais}
    \definition{v.}{recuperar | curar}
  \end{phonetics}
\end{entry}

\begin{entry}{意}{13}{⼼}
  \begin{phonetics}{意}{yi4}
    \definition*{s.}{Itália, abreviação de 意大利}
    \definition{s.}{ideia; significado; pensamento | desejo; vontade; intenção | significância}
  \seealsoref{意大利}{yi4da4li4}
  \end{phonetics}
\end{entry}

\begin{entry}{意义}{13,3}{⼼、⼂}
  \begin{phonetics}{意义}{yi4yi4}[][HSK 3]
    \definition[个,种,层,重,点]{s.}{sentido; significado; o significado expresso por meio de linguagem escrita ou outros sinais; o significado identificado por meio de ações ou obtenção | valor; efeito; significado; influência; impacto}
  \end{phonetics}
\end{entry}

\begin{entry}{意大利}{13,3,7}{⼼、⼤、⼑}
  \begin{phonetics}{意大利}{yi4da4li4}
    \definition*{s.}{Itália}
  \end{phonetics}
\end{entry}

\begin{entry}{意见}{13,4}{⼼、⾒}
  \begin{phonetics}{意见}{yi4jian4}[][HSK 2]
    \definition[种,点,条]{s.}{ideia; visão; opinião; sugestão; uma certa visão ou ideia sobre algo | objeção; reclamação; opinião divergente; (em relação a uma pessoa ou coisa) o sentimento de estar insatisfeito com algo porque está errado}
  \end{phonetics}
\end{entry}

\begin{entry}{意外}{13,5}{⼼、⼣}
  \begin{phonetics}{意外}{yi4wai4}[][HSK 3]
    \definition{adj.}{inesperado; imprevisto}
    \definition{adv.}{acidentalmente}
    \definition[个,种]{s.}{acidente; infortúnio; um infortúnio inesperado}
  \end{phonetics}
\end{entry}

\begin{entry}{意志}{13,7}{⼼、⼼}
  \begin{phonetics}{意志}{yi4zhi4}[][HSK 5]
    \definition[个,股]{s.}{vontade; determinação; desejo; força de vontade}
  \end{phonetics}
\end{entry}

\begin{entry}{意识}{13,7}{⼼、⾔}
  \begin{phonetics}{意识}{yi4shi2}[][HSK 5]
    \definition{s.}{consciência}
    \definition{s.}{consciência; percepção; grau de reconhecimento e importância atribuído a uma determinada questão}
    \definition{v.}{perceber; despertar para; estar ciente de; sentir, descobrir o que antes não se sentia ou não se descobria; geralmente é usado junto com 到}
  \seealsoref{到}{dao4}
  \end{phonetics}
\end{entry}

\begin{entry}{意译}{13,7}{⼼、⾔}
  \begin{phonetics}{意译}{yi4yi4}
    \definition{s.}{tradução livre | significado (de expressão estrangeira) | paráfrase | tradução do significado (em oposição à tradução literal)}
  \seealsoref{直译}{zhi2yi4}
  \end{phonetics}
\end{entry}

\begin{entry}{意味着}{13,8,11}{⼼、⼝、⽬}
  \begin{phonetics}{意味着}{yi4wei4zhe5}[][HSK 5]
    \definition{v.}{significar; subentender}
  \end{phonetics}
\end{entry}

\begin{entry}{意思}{13,9}{⼼、⼼}
  \begin{phonetics}{意思}{yi4si5}[][HSK 2]
    \definition[个]{s.}{ideia; significado; o significado da linguagem e das palavras; conteúdo ideológico | desejo; vontade; opiniões | um símbolo de afeto, apreciação, gratidão, etc. | dica; traço; sugestão; refere-se principalmente ao afeto entre homens e mulheres | diversão; interesse}
    \definition{v.}{dar uma dica; demonstrar sua gratidão com presentes ou outros meios}
  \end{phonetics}
\end{entry}

\begin{entry}{意指}{13,9}{⼼、⼿}
  \begin{phonetics}{意指}{yi4zhi3}
    \definition{v.}{implicar | significar}
  \end{phonetics}
\end{entry}

\begin{entry}{感}{13}{⼼}
  \begin{phonetics}{感}{gan3}
    \definition{s.}{sentido; sensação; sentimento; impressão | emoção; sentimento}
    \definition{v.}{sentir; perceber; estar ciente | mover; tocar; afetar | ser grato; ser agradecido | ser afetado (pelo frio); pegar um resfriado | (fotografia) sensibilizar | ser grato; apreciar | ser afetado}
  \end{phonetics}
\end{entry}

\begin{entry}{感兴趣}{13,6,15}{⼼、⼋、⾛}
  \begin{phonetics}{感兴趣}{gan3xing4qu4}[][HSK 4]
    \definition{v.}{estar interessado}
  \seealsoref{对……感兴趣}{dui4 gan3xing4qu4}
  \end{phonetics}
\end{entry}

\begin{entry}{感动}{13,6}{⼼、⼒}
  \begin{phonetics}{感动}{gan3dong4}[][HSK 2]
    \definition{v.}{mover (alguém) | tocar (alguém emocionalmente)}
  \end{phonetics}
\end{entry}

\begin{entry}{感到}{13,8}{⼼、⼑}
  \begin{phonetics}{感到}{gan3 dao4}[][HSK 2]
    \definition{v.}{sentir; achar; perceber}
  \end{phonetics}
\end{entry}

\begin{entry}{感受}{13,8}{⼼、⼜}
  \begin{phonetics}{感受}{gan3shou4}[][HSK 3]
    \definition{s.}{percepção ; compreenção; sentimento; experiência; influência do contato com o mundo exterior}
    \definition{v.}{sentir; sentir (através dos sentidos); experimentar; ser afetado}
  \end{phonetics}
\end{entry}

\begin{entry}{感冒}{13,9}{⼼、⽇}
  \begin{phonetics}{感冒}{gan3mao4}[][HSK 3]
    \definition{adj.}{interessado em}
    \definition[场,次]{s.}{resfriado; gripe comum; \emph{influenza}; doença infecciosa causada por um vírus, que tende a causar sintomas como garganta seca, congestão nasal, tosse, espirros, dor de cabeça e febre quando o corpo está excessivamente cansado, resfriado ou com a imunidade enfraquecida}
    \definition{v.}{pegar (ter) um resfriado}
  \end{phonetics}
\end{entry}

\begin{entry}{感染}{13,9}{⼼、⽊}
  \begin{phonetics}{感染}{gan3ran3}
    \definition{s.}{infecção}
    \definition{v.}{infectar | (figurativo) influenciar}
  \end{phonetics}
\end{entry}

\begin{entry}{感觉}{13,9}{⼼、⾒}
  \begin{phonetics}{感觉}{gan3jue2}[][HSK 2]
    \definition[个]{s.}{sentimento; sensação; percepção sensorial;}
    \definition{v.}{sentir; perceber; tomar consciência; sentir no coração, acreditar}
  \end{phonetics}
\end{entry}

\begin{entry}{感情}{13,11}{⼼、⼼}
  \begin{phonetics}{感情}{gan3qing2}[][HSK 3]
    \definition[份,个,种]{s.}{emoção; sentimento; reações psicológicas como amor, ódio, alegria, raiva, tristeza e felicidade, geradas por estímulos externos | amor; afeto; apego; preocupação e afeição por pessoas ou coisas}
  \end{phonetics}
\end{entry}

\begin{entry}{感谢}{13,12}{⼼、⾔}
  \begin{phonetics}{感谢}{gan3xie4}[][HSK 2]
    \definition{v.}{agradecer; ser grato; expressar gratidão com palavras ou ações}
  \end{phonetics}
\end{entry}

\begin{entry}{感想}{13,13}{⼼、⼼}
  \begin{phonetics}{感想}{gan3xiang3}[][HSK 5]
    \definition[个,条]{s.}{pensamentos; impressões; reflexões; resposta do pensamento decorrente da exposição ao mundo exterior}
  \end{phonetics}
\end{entry}

\begin{entry}{慢}{14}{⼼}
  \begin{phonetics}{慢}{man4}[][HSK 1]
    \definition*{s.}{Sobrenome Man}
    \definition{adj.}{lento; devagar; baixa velocidade; longa duração (em oposição a 快) | rude; arrogante; sem educação com as pessoas | frouxo; lento}
    \definition{adv.}{lentamente}
  \seealsoref{快}{kuai4}
  \end{phonetics}
\end{entry}

\begin{entry}{慢车}{14,4}{⼼、⾞}
  \begin{phonetics}{慢车}{man4 che1}[][HSK 6]
    \definition{s.}{trem lento com muitas paradas (oposto a 快车) | ônibus ou trem local; parada do trem}
  \seealsoref{快车}{kuai4 che1}
  \end{phonetics}
\end{entry}

\begin{entry}{慢动作}{14,6,7}{⼼、⼒、⼈}
  \begin{phonetics}{慢动作}{man4dong4zuo4}
    \definition{s.}{(cinema) câmera lenta}
  \end{phonetics}
\end{entry}

\begin{entry}{慢慢}{14,14}{⼼、⼼}
  \begin{phonetics}{慢慢}{man4 man4}[][HSK 3]
    \definition{adv.}{lentamente; vagarosamente; gradualmente | lentamente; vagarosamente; gradualmente; depois de um longo período de tempo}
  \end{phonetics}
\end{entry}

\begin{entry}{慰}{15}{⼼}
  \begin{phonetics}{慰}{wei4}
    \definition{adj.}{aliviado; em paz; confortável}
    \definition{v.}{consolar; confortar | ser (ficar) aliviado}
  \end{phonetics}
\end{entry}

\begin{entry}{慰问}{15,6}{⼼、⾨}
  \begin{phonetics}{慰问}{wei4wen4}[][HSK 5]
    \definition{v.}{visitar; consolar; expressar simpatia por; confortar e cumprimentar com palavras e presentes;  enfatizar o conforto e o cumprimento, frequentemente usado por superiores para subordinados}
  \end{phonetics}
\end{entry}

\begin{entry}{憧}{15}{⼼}
  \begin{phonetics}{憧}{chong1}
    \definition{adj.}{irresoluto; indeciso | estúpido; imbecil; confuso}
  \end{phonetics}
\end{entry}

\begin{entry}{憧憬}{15,15}{⼼、⼼}
  \begin{phonetics}{憧憬}{chong1jing3}
    \definition{v.}{ansiar por | esperar por}
  \end{phonetics}
\end{entry}

\begin{entry}{懂}{15}{⼼}
  \begin{phonetics}{懂}{dong3}[][HSK 2]
    \definition*{s.}{Sobrenome Dong}
    \definition{v.}{compreender; entender}
  \end{phonetics}
\end{entry}

\begin{entry}{懂得}{15,11}{⼼、⼻}
  \begin{phonetics}{懂得}{dong3 de5}[][HSK 2]
    \definition{v.}{saber (significado, prática, etc.); compreender; entender}
  \end{phonetics}
\end{entry}

\begin{entry}{懒}{16}{⼼}
  \begin{phonetics}{懒}{lan3}[][HSK 6]
    \definition{adj.}{indolente; preguiçoso (oposto de 勤) | lento; lânguido | ocioso; preguiçoso}
  \seealsoref{勤}{qin2}
  \end{phonetics}
\end{entry}

\begin{entry}{懒人}{16,2}{⼼、⼈}
  \begin{phonetics}{懒人}{lan3ren2}
    \definition{s.}{pessoa preguiçosa}
  \end{phonetics}
\end{entry}

\begin{entry}{懒汉}{16,5}{⼼、⽔}
  \begin{phonetics}{懒汉}{lan3han4}
    \definition{s.}{sujeito ocioso | vagabundo | preguiçosos}
  \end{phonetics}
\end{entry}

\begin{entry}{懒虫}{16,6}{⼼、⾍}
  \begin{phonetics}{懒虫}{lan3chong2}
    \definition{s.}{desleixado ocioso | (insulto) sujeito preguiçoso}
  \end{phonetics}
\end{entry}

\begin{entry}{懒怠}{16,9}{⼼、⼼}
  \begin{phonetics}{懒怠}{lan3dai4}
    \definition{s.}{preguiça}
  \end{phonetics}
\end{entry}

\begin{entry}{懒鬼}{16,9}{⼼、⿁}
  \begin{phonetics}{懒鬼}{lan3gui3}
    \definition{s.}{cara preguiçoso}
  \end{phonetics}
\end{entry}

\begin{entry}{懒得}{16,11}{⼼、⼻}
  \begin{phonetics}{懒得}{lan3de5}
    \definition{adv.}{demasiado preguiçoso}
    \definition{v.}{não sentir vontade (de fazer algo)}
  \end{phonetics}
\end{entry}

\begin{entry}{懒惰}{16,12}{⼼、⼼}
  \begin{phonetics}{懒惰}{lan3duo4}
    \definition{adj.}{preguiçoso}
  \end{phonetics}
\end{entry}

\begin{entry}{懒散}{16,12}{⼼、⽁}
  \begin{phonetics}{懒散}{lan3san3}
    \definition{adj.}{inativo | indolente | preguiçoso | negligente}
  \end{phonetics}
\end{entry}

\begin{entry}{懒腰}{16,13}{⼼、⾁}
  \begin{phonetics}{懒腰}{lan3yao1}
    \definition[个]{s.}{alongamento (do corpo)}
  \end{phonetics}
\end{entry}

\begin{entry}{懵}{18}{⼼}
  \begin{phonetics}{懵}{meng3}
    \definition{adj.}{confuso; ignorante; irracional | inconsciente; entorpecido}
  \end{phonetics}
\end{entry}

\begin{entry}{懵懂}{18,15}{⼼、⼼}
  \begin{phonetics}{懵懂}{meng3dong3}
    \definition{adj.}{confuso | ignorante}
  \end{phonetics}
\end{entry}

\begin{entry}{聼}{19}{⼼}
  \begin{phonetics}{聼}{ting1}
    \variantof{听}
  \end{phonetics}
\end{entry}

%%%%% EOF %%%%%

