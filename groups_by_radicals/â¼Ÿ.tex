%%%
%%% Radical "⼟"
%%%

\section*{Radical 32: ``⼟''}\addcontentsline{toc}{section}{Radical 32: ⼟}

\begin{Entry}{土}{3}{⼟}[Kangxi 32]
  \begin{Phonetics}{土}{tu3}[][HSK 3,6]
    \definition*{s.}{Sobrenome Tu}
    \definition{adj.}{local; nativo; local com características regionais| caseiro; indígena; o que é tradicional no país; popular | não refinado; não esclarecido; não está na moda; não é popular}
    \definition[堆,捧,层]{s.}{solo; terra | terra; território | ópio bruto | cidade natal; terra natal; pátria}
  \end{Phonetics}
\end{Entry}

\begin{Entry}{土地}{3,6}{⼟、⼟}
  \begin{Phonetics}{土地}{tu3di4}[][HSK 4]
    \definition[片,块,顷]{s.}{terra; solo; chão; superfície terrestre da Terra usada para cultivar, construir edifícios e viver | território; território de um país}
  \end{Phonetics}
  \begin{Phonetics}{土地}{tu3di5}
    \definition[片,块,顷]{s.}{deus da audeia; deus local; \emph{genius loci} deidade protetora de um local; (superstição) refere-se ao deus da terra que governa uma pequena área}
  \end{Phonetics}
\end{Entry}

\begin{Entry}{土豆}{3,7}{⼟、⾖}
  \begin{Phonetics}{土豆}{tu3dou4}[][HSK 5]
    \definition[颗,斤,个,棵]{s.}{batata; denominação comum da batata}
  \end{Phonetics}
\end{Entry}

\begin{Entry}{土豆泥}{3,7,8}{⼟、⾖、⽔}
  \begin{Phonetics}{土豆泥}{tu3dou4ni2}
    \definition{s.}{purê de batatas}
  \end{Phonetics}
\end{Entry}

\begin{Entry}{土鸡}{3,7}{⼟、⿃}
  \begin{Phonetics}{土鸡}{tu3ji1}
    \definition{s.}{galinha caipira}
  \end{Phonetics}
\end{Entry}

\begin{Entry}{圣}{5}{⼟}
  \begin{Phonetics}{圣}{sheng4}
    \definition*{s.}{Sobrenome Sheng}
    \definition{adj.}{santo; sagrado | imperial}
    \definition{s.}{santo; sábio | imperador | o maior mestre de uma determinada arte ou habilidade}
  \end{Phonetics}
\end{Entry}

\begin{Entry}{圣地}{5,6}{⼟、⼟}
  \begin{Phonetics}{圣地}{sheng4di4}
    \definition{s.}{terra santa (de uma religião) | lugar sagrado | santuário | cidade santa (como Jerusalém, Meca, etc.) | centro de interesse histórico}
  \end{Phonetics}
\end{Entry}

\begin{Entry}{圣诞节}{5,8,5}{⼟、⾔、⾋}
  \begin{Phonetics}{圣诞节}{sheng4 dan4 jie2}[][HSK 6]
    \definition*{s.}{Natal; Nascimento de Jesus Cristo em 25 de dezembro}
  \end{Phonetics}
\end{Entry}

\begin{Entry}{在}{6}{⼟}
  \begin{Phonetics}{在}{zai4}[][HSK 1]
    \definition{adv.}{em processo de; em curso de}
    \definition{prep.}{em; no (um lugar ou momento); indica tempo, local, âmbito, etc.}
    \definition{v.}{existir; estar vivo | estar em; estar no; estar em (um lugar); indica a localização de pessoas ou coisas | permanecer; ficar | depender de; residir em; repousar com | ingressar ou pertencer a uma organização; ser membro de uma organização}
  \end{Phonetics}
\end{Entry}

\begin{Entry}{在下}{6,3}{⼟、⼀}
  \begin{Phonetics}{在下}{zai4xia4}
    \definition{pron.}{eu mesmo (humildemente)}
  \end{Phonetics}
\end{Entry}

\begin{Entry}{在于}{6,3}{⼟、⼆}
  \begin{Phonetics}{在于}{zai4yu2}[][HSK 4]
    \definition{v.}{ser responsável por; caber a;  ser da competência de;  apontar a essência das coisas, ou do que elas se tratam | depender de; ser determinado por;  ser devido a (um determinado atributo)/(de um assunto a ser determinado)}
  \end{Phonetics}
\end{Entry}

\begin{Entry}{在内}{6,4}{⼟、⼌}
  \begin{Phonetics}{在内}{zai4 nei4}[][HSK 5]
    \definition{adj.}{incluido}
    \definition{adv.}{dentro; internamente; entre eles}
    \definition{v.}{ser incluído}
  \end{Phonetics}
\end{Entry}

\begin{Entry}{在乎}{6,5}{⼟、⼃}
  \begin{Phonetics}{在乎}{zai4hu5}[][HSK 4]
    \definition{v.}{preocupar-se; preocupar-se com; levar a sério | ser responsável por; caber ao; ser da competência de}
  \end{Phonetics}
\end{Entry}

\begin{Entry}{在地}{6,6}{⼟、⼟}
  \begin{Phonetics}{在地}{zai4di4}
    \definition{s.}{local}
  \end{Phonetics}
\end{Entry}

\begin{Entry}{在场}{6,6}{⼟、⼟}
  \begin{Phonetics}{在场}{zai4 chang3}[][HSK 5]
    \definition{v.}{estar presente; estar no local; estar em cena; estar presente onde as coisas estão acontecendo}
  \end{Phonetics}
\end{Entry}

\begin{Entry}{在此}{6,6}{⼟、⽌}
  \begin{Phonetics}{在此}{zai4 ci3}
    \definition{s.}{aqui}
  \end{Phonetics}
\end{Entry}

\begin{Entry}{在行}{6,6}{⼟、⾏}
  \begin{Phonetics}{在行}{zai4hang2}
    \definition{v.}{ser adepto de algo | ser um especialista em um comércio ou profissão}
  \end{Phonetics}
\end{Entry}

\begin{Entry}{在线}{6,8}{⼟、⽷}
  \begin{Phonetics}{在线}{zai4xian4}
    \definition{s.}{\emph{online}}
  \end{Phonetics}
\end{Entry}

\begin{Entry}{在家}{6,10}{⼟、⼧}
  \begin{Phonetics}{在家}{zai4 jia1}[][HSK 1]
    \definition{v.}{estar em; estar em casa; estar no local de trabalho ou alojamento; sem sair de casa | continuar sendo um leigo; permanecer leigo; para monges, freiras, taoístas e outros que 出家, as pessoas comuns são consideradas leigas}
  \seealsoref{出家}{chu1 jia1}
  \end{Phonetics}
\end{Entry}

\begin{Entry}{在教}{6,11}{⼟、⽁}
  \begin{Phonetics}{在教}{zai4 jiao4}
    \definition{v.}{ser um crente (em uma religião)}
  \end{Phonetics}
\end{Entry}

\begin{Entry}{在意}{6,13}{⼟、⼼}
  \begin{Phonetics}{在意}{zai4/yi4}
    \definition{v.+compl.}{preocupar-se | importar-se | levar a sério}
  \end{Phonetics}
\end{Entry}

\begin{Entry}{地}{6}{⼟}
  \begin{Phonetics}{地}{de5}[][HSK 1]
    \definition{part.}{(estrutural) utilizada antes de um verbo ou adjetivo, ligando-o ao adjunto adverbial modificador precedente}
  \end{Phonetics}
  \begin{Phonetics}{地}{di4}[][HSK 1]
    \definition*{s.}{A Terra | Sobrenome Di}
    \definition[块,片]{s.}{terra; solo | campos | chão; piso | posição; situação | contexto; base | distância percorrida (medida em 里 ou paradas 站) | indicando estado de espírito | território | lugar; local | parte do espaço | distância}
  \end{Phonetics}
\end{Entry}

\begin{Entry}{地上}{6,3}{⼟、⼀}
  \begin{Phonetics}{地上}{di4 shang5}[][HSK 1]
    \definition{adv.}{no chão; no solo; em terra}
  \end{Phonetics}
\end{Entry}

\begin{Entry}{地下}{6,3}{⼟、⼀}
  \begin{Phonetics}{地下}{di4 xia4}[][HSK 4]
    \definition{s.}{subterrâneo | secreta (atividade) | recursos ocultos}
  \end{Phonetics}
\end{Entry}

\begin{Entry}{地下水}{6,3,4}{⼟、⼀、⽔}
  \begin{Phonetics}{地下水}{di4xia4shui3}[][HSK 7-9]
    \definition{s.}{água subterrânea, principalmente água da chuva e outras águas superficiais que se infiltram no solo e se acumulam nas fendas do solo ou nas formações rochosas}
  \end{Phonetics}
\end{Entry}

\begin{Entry}{地下室}{6,3,9}{⼟、⼀、⼧}
  \begin{Phonetics}{地下室}{di4 xia4 shi4}[][HSK 6]
    \definition{s.}{subterrâneo; porão; adega | abóbadas; cripta}
  \end{Phonetics}
\end{Entry}

\begin{Entry}{地区}{6,4}{⼟、⼖}
  \begin{Phonetics}{地区}{di4qu1}[][HSK 3]
    \definition[个,片]{s.}{área; distrito; região; um lugar maior | prefeitura; unidade administrativa | latitudes; localidade; lado | em determinadas circunstâncias, algumas regiões administrativas locais da China, como Hong Kong e Macau, participam individualmente em algumas atividades internacionais}
    \definition{suf.}{como sufixo do nome da cidade, significa prefeitura ou condado}
  \end{Phonetics}
\end{Entry}

\begin{Entry}{地方}{6,4}{⼟、⽅}
  \begin{Phonetics}{地方}{di4fang1}
    \definition[个]{s.}{distrito; localidade;  em oposição a 中央, o número total de unidades administrativas em todos os níveis abaixo do centro | governo local e população; refere-se a outros setores que não o militar}
  \seealsoref{中央}{zhong1yang1}
  \end{Phonetics}
  \begin{Phonetics}{地方}{di4fang5}[][HSK 1,4]
    \definition[个,处,块]{s.}{lugar; cômodo; área; refere-se a um espaço específico | parte}
  \end{Phonetics}
\end{Entry}

\begin{Entry}{地名}{6,6}{⼟、⼝}
  \begin{Phonetics}{地名}{di4 ming2}[][HSK 6]
    \definition{s.}{nome de um lugar | nome de lugar | topônimo}
  \end{Phonetics}
\end{Entry}

\begin{Entry}{地位}{6,7}{⼟、⼈}
  \begin{Phonetics}{地位}{di4wei4}[][HSK 4]
    \definition[个]{s.}{lugar; status; posição; posição da pessoa ou do grupo nas relações sociais | lugar; posição (ocupada por uma pessoa ou coisa); espaço ocupado por uma pessoa ou coisa}
  \end{Phonetics}
\end{Entry}

\begin{Entry}{地址}{6,7}{⼟、⼟}
  \begin{Phonetics}{地址}{di4zhi3}[][HSK 4]
    \definition[个,条]{s.}{endereço; local de residência ou correspondência}
  \end{Phonetics}
\end{Entry}

\begin{Entry}{地形}{6,7}{⼟、⼺}
  \begin{Phonetics}{地形}{di4 xing2}[][HSK 5]
    \definition{s.}{topografia; forma do terreno; relevo; disposição do terreno; característica do relevo; característica da superfície; terreno}
  \end{Phonetics}
\end{Entry}

\begin{Entry}{地步}{6,7}{⼟、⽌}
  \begin{Phonetics}{地步}{di4bu4}[][HSK 7-9]
    \definition[个,种]{s.}{condição; situação; estado; geralmente ruim | grau; extensão; o grau de realização | margem de manobra; espaço}
  \end{Phonetics}
\end{Entry}

\begin{Entry}{地图}{6,8}{⼟、⼞}
  \begin{Phonetics}{地图}{di4tu2}[][HSK 1]
    \definition[张,本]{s.}{mapa; mapa que mostra a distribuição de coisas e fenômenos na superfície da Terra, com símbolos e textos, e às vezes também com cores}
  \end{Phonetics}
\end{Entry}

\begin{Entry}{地板}{6,8}{⼟、⽊}
  \begin{Phonetics}{地板}{di4 ban3}[][HSK 6]
    \definition[块]{s.}{piso de madeira; tábuas de madeira especiais para pavimentação do piso | piso; piso interno pavimentado com tábuas de madeira; geralmente se refere ao piso de um edifício}
  \end{Phonetics}
\end{Entry}

\begin{Entry}{地质}{6,8}{⼟、⾙}
  \begin{Phonetics}{地质}{di4zhi4}[][HSK 7-9]
    \definition[种]{s.}{geologia; composição e estrutura da crosta terrestre}
  \end{Phonetics}
\end{Entry}

\begin{Entry}{地带}{6,9}{⼟、⼱}
  \begin{Phonetics}{地带}{di4 dai4}[][HSK 5]
    \definition[个,条]{s.}{distrito; região; zona; área de uma determinada natureza ou extensão}
  \end{Phonetics}
\end{Entry}

\begin{Entry}{地段}{6,9}{⼟、⽎}
  \begin{Phonetics}{地段}{di4duan4}[][HSK 7-9]
    \definition[个]{s.}{setor (ou seção) de uma cidade, etc.; área | um setor de uma área; uma seção de uma área; alcance; extensão; lote}
  \end{Phonetics}
\end{Entry}

\begin{Entry}{地点}{6,9}{⼟、⽕}
  \begin{Phonetics}{地点}{di4dian3}[][HSK 1]
    \definition[个]{s.}{lugar; local; região; localização}
  \end{Phonetics}
\end{Entry}

\begin{Entry}{地狱}{6,9}{⼟、⽝}
  \begin{Phonetics}{地狱}{di4yu4}[][HSK 7-9]
    \definition[个,层,重,处]{s.}{inferno; submundo; algumas religiões se referem ao lugar onde a alma sofre após a morte | inferno na terra; lugar de tormento como o inferno; uma metáfora para um ambiente de vida sombrio e miserável}
  \end{Phonetics}
\end{Entry}

\begin{Entry}{地砖}{6,9}{⼟、⽯}
  \begin{Phonetics}{地砖}{di4zhuan1}
    \definition{s.}{ladrilho de piso}
  \end{Phonetics}
\end{Entry}

\begin{Entry}{地面}{6,9}{⼟、⾯}
  \begin{Phonetics}{地面}{di4 mian4}[][HSK 4]
    \definition{s.}{a superfície da Terra | térreo; piso; camada de material colocada no chão dentro e ao redor dos edifícios | localidade; chão | região; território; principalmente áreas administrativas}
  \end{Phonetics}
\end{Entry}

\begin{Entry}{地核}{6,10}{⼟、⽊}
  \begin{Phonetics}{地核}{di4he2}
    \definition{s.}{(geologia) núcleo da Terra}
  \end{Phonetics}
\end{Entry}

\begin{Entry}{地铁}{6,10}{⼟、⾦}
  \begin{Phonetics}{地铁}{di4tie3}[][HSK 2]
    \definition[条,班,列,趟]{s.}{metrô; trem subterrâneo; também se refere ao vagão do metrô}
  \end{Phonetics}
\end{Entry}

\begin{Entry}{地铁站}{6,10,10}{⼟、⾦、⽴}
  \begin{Phonetics}{地铁站}{di4 tie3 zhan4}[][HSK 2]
    \definition[个,座]{s.}{estação de metrô}
  \end{Phonetics}
\end{Entry}

\begin{Entry}{地域}{6,11}{⼟、⼟}
  \begin{Phonetics}{地域}{di4yu4}[][HSK 7-9]
    \definition{s.}{região; distrito}
  \end{Phonetics}
\end{Entry}

\begin{Entry}{地球}{6,11}{⼟、⽟}
  \begin{Phonetics}{地球}{di4qiu2}[][HSK 2]
    \definition[个]{s.}{o planeta Terra}
  \end{Phonetics}
\end{Entry}

\begin{Entry}{地理}{6,11}{⼟、⽟}
  \begin{Phonetics}{地理}{di4li3}[][HSK 7-9]
    \definition{s.}{características geográficas de um lugar; a situação geral do mundo ou das montanhas e rios de uma região; ambiente natural, como clima e produtos; transporte; assentamentos e outros fatores socioeconômicos | geografia}
  \end{Phonetics}
\end{Entry}

\begin{Entry}{地毯}{6,12}{⼟、⽑}
  \begin{Phonetics}{地毯}{di4tan3}[][HSK 7-9]
    \definition[块,张]{s.}{tapete; carpete}
  \end{Phonetics}
\end{Entry}

\begin{Entry}{地道}{6,12}{⼟、⾡}
  \begin{Phonetics}{地道}{di4dao4}[][HSK 7-9]
    \definition[条,个]{s.}{metrô; túnel; passagem subterrânea; galeria}
  \end{Phonetics}
  \begin{Phonetics}{地道}{di4dao5}[][HSK 7-9]
    \definition{adj.}{autêntico; puro; típico; alta qualidade, atendendo a certos padrões; como o produto real | excelente; honesto; de acordo com os padrões; de alta qualidade | (da moral e das qualidades de uma pessoa) muito bom; nobre, frequentemente usado em frases negativas}
  \end{Phonetics}
\end{Entry}

\begin{Entry}{地震}{6,15}{⼟、⾬}
  \begin{Phonetics}{地震}{di4zhen4}[][HSK 5]
    \definition[场,次,级]{s.}{sismo; terremoto; tremor de terra; vibrações na crosta terrestre}
    \definition{v.}{sacudir com vibrações sísmicas}
  \end{Phonetics}
\end{Entry}

\begin{Entry}{场}{6}{⼟}
  \begin{Phonetics}{场}{chang2}
    \definition{clas.}{usado para descrever o desenrolar dos acontecimentos}
    \definition{s.}{eira; espaço aberto e plano; um terreno plano, geralmente usado para secar grãos e moer cereais | mercado; feira rural}
  \end{Phonetics}
  \begin{Phonetics}{场}{chang3}[][HSK 2]
    \definition*{s.}{Sobrenome Chang}
    \definition{clas.}{usado para atividades culturais, recreativas e esportivas | usado para pequenos trechos de uma peça}
    \definition{s.}{um local amplo utilizado para um fim específico | palco; campo | cena | (física) campo (por exemplo: campo manético) | (para atividades recreativas, esportivas ou outras) | um lugar onde as pessoas se reúnem | fazenda; quinta | abertura; encerramento; refere-se ao processo completo de uma apresentação ou competição | local; ponto; o local onde ocorreu o incidente}
  \end{Phonetics}
\end{Entry}

\begin{Entry}{场合}{6,6}{⼟、⼝}
  \begin{Phonetics}{场合}{chang3he2}[][HSK 3]
    \definition[个,些,种,类]{s.}{ocasião; situação; um certo tempo, lugar ou situação}
  \end{Phonetics}
\end{Entry}

\begin{Entry}{场地}{6,6}{⼟、⼟}
  \begin{Phonetics}{场地}{chang3 di4}[][HSK 6]
    \definition[片,块,个]{s.}{área; pátio; espaço; lugar; quadra; campo; um lugar onde construções ou atividades são realizadas}
  \end{Phonetics}
\end{Entry}

\begin{Entry}{场所}{6,8}{⼟、⼾}
  \begin{Phonetics}{场所}{chang3suo3}[][HSK 3]
    \definition{s.}{lugar; sítio; arena; local da atividade}
  \end{Phonetics}
\end{Entry}

\begin{Entry}{场面}{6,9}{⼟、⾯}
  \begin{Phonetics}{场面}{chang3mian4}[][HSK 5]
    \definition[个,种,番]{s.}{espetáculo; cena (em teatro, ficção, etc.); uma cena em uma produção teatral, cinematográfica ou televisiva que consiste em um cenário, música e personagens | cena; ocasião; literatura narrativa que consiste em situações da vida em que os personagens se relacionam entre si em determinadas ocasiões | orquestra ou instrumentos de acompanhamento (em ópera); refere-se às pessoas e aos instrumentos musicais que acompanham a apresentação de uma ópera, divididos em dois tipos: música de sopro e cordas é uma cena cultural, e gongos e tambores são uma cena marcial | situação; referência geral a uma situação em um determinado contexto | frente; fachada; aparência; espetáculo superficial}
  \end{Phonetics}
\end{Entry}

\begin{Entry}{场馆}{6,11}{⼟、⾷}
  \begin{Phonetics}{场馆}{chang3 guan3}[][HSK 6]
    \definition{s.}{ginásios e estádios | arena | local esportivo}
  \end{Phonetics}
\end{Entry}

\begin{Entry}{场景}{6,12}{⼟、⽇}
  \begin{Phonetics}{场景}{chang3 jing3}[][HSK 6]
    \definition[个,幕,种]{s.}{espetáculo; cena (em drama, ficção, etc.); refere-se a cenas de drama, cinema, televisão e obras literárias | cena; visão; circunstâncias; cenas e situações}
  \end{Phonetics}
\end{Entry}

\begin{Entry}{赱}{6}{⼟}
  \begin{Phonetics}{赱}{zou3}
    \variantof{走}
  \end{Phonetics}
\end{Entry}

\begin{Entry}{坏}{7}{⼟}
  \begin{Phonetics}{坏}{huai4}[][HSK 1]
    \definition{adj.}{ruim; prejudicial; insatisfatório; péssimo | mal; extremamente; indica um grau profundo, geralmente usado após verbos ou adjetivos que expressam estado psicológico | podre; estragado; impróprio; prejudicial ao uso}
    \definition[种]{s.}{ideia maligna; truque sujo; péssima ideia}
    \definition{v.}{estragar; destruir; corromper}
  \end{Phonetics}
\end{Entry}

\begin{Entry}{坏人}{7,2}{⼟、⼈}
  \begin{Phonetics}{坏人}{huai4 ren2}[][HSK 2]
    \definition[个,种]{s.}{malfeitor; canalha; pessoa má; pessoa de má qualidade; pessoa que faz coisas ruins}
  \end{Phonetics}
\end{Entry}

\begin{Entry}{坏处}{7,5}{⼟、⼡}
  \begin{Phonetics}{坏处}{huai4 chu4}[][HSK 2]
    \definition[个]{s.}{dano; prejuízo; desvantagem; fatores prejudiciais a pessoas ou coisas}
  \end{Phonetics}
\end{Entry}

\begin{Entry}{坏蛋}{7,11}{⼟、⾍}
  \begin{Phonetics}{坏蛋}{huai4dan4}
    \definition{s.}{bastardo | canalha | pessoa má}
  \end{Phonetics}
\end{Entry}

\begin{Entry}{坐}{7}{⼟}
  \begin{Phonetics}{坐}{zuo4}[][HSK 1]
    \definition*{s.}{Sobrenome Zuo}
    \definition{adv.}{sem motivo algum; sem causa ou razão; sem motivo aparente}
    \definition{prep.}{porque; pelo fato de que; pela razão de que; pelo motivo de que}
    \definition{s.}{assento; lugar; posição}
    \definition{v.}{sentar; sentar-se; ocupar um lugar; colocar os glúteos sobre um objeto para apoiar o peso corporal | pegar; viajar de; pegar carona | ter as costas voltadas para | colocar (uma panela, chaleira, etc.) no fogo | recuo; coice (de rifles, armas, etc.)  | produzir frutos; formar sementes | ser punido; ser acusado de crime | contrair (ou ter) uma doença; sofrer de uma doença | (um edifício) afundar; ceder}
  \end{Phonetics}
\end{Entry}

\begin{Entry}{坐下}{7,3}{⼟、⼀}
  \begin{Phonetics}{坐下}{zuo4 xia5}[][HSK 1]
    \definition{v.}{sentar-se; tomar um assento; passar da posição em pé para a posição sentada}
  \end{Phonetics}
\end{Entry}

\begin{Entry}{坐车}{7,4}{⼟、⾞}
  \begin{Phonetics}{坐车}{zuo4che1}
    \definition{v.}{andar de carro, ônibus, trem, etc.}
  \end{Phonetics}
\end{Entry}

\begin{Entry}{坐好}{7,6}{⼟、⼥}
  \begin{Phonetics}{坐好}{zuo4hao3}
    \definition{v.}{sentar-se corretamente | sentar direito}
  \end{Phonetics}
\end{Entry}

\begin{Entry}{坐享}{7,8}{⼟、⼇}
  \begin{Phonetics}{坐享}{zuo4xiang3}
    \definition{v.}{curtir algo sem levantar um dedo}
  \end{Phonetics}
\end{Entry}

\begin{Entry}{坐垫}{7,9}{⼟、⼟}
  \begin{Phonetics}{坐垫}{zuo4dian4}
    \definition[块]{s.}{assento (motocicleta) | almofada}
  \end{Phonetics}
\end{Entry}

\begin{Entry}{坐标}{7,9}{⼟、⽊}
  \begin{Phonetics}{坐标}{zuo4biao1}
    \definition{s.}{Geometria: coordenada; um número ou conjunto de números que pode determinar a posição de um ponto no espaço}
  \end{Phonetics}
\end{Entry}

\begin{Entry}{坑}{7}{⼟}
  \begin{Phonetics}{坑}{keng1}
    \definition[个]{s.}{poço; buraco; cavidade | poço; túnel; caverna subterrânea}
    \definition{v.}{enredar; enganar; trapacear | nos tempos antigos, significava enterrar as pessoas vivas}
  \end{Phonetics}
\end{Entry}

\begin{Entry}{坑人}{7,2}{⼟、⼈}
  \begin{Phonetics}{坑人}{keng1/ren2}
    \definition{v.+compl.}{trapacear alguém}
  \end{Phonetics}
\end{Entry}

\begin{Entry}{块}{7}{⼟}
  \begin{Phonetics}{块}{kuai4}[][HSK 1]
    \definition{clas.}{usado para coisas em pedaços | usado para coisas em pedaços ou em algumas formas de folhas | usado para moedas de prata ou notas de papel equivalentes a 圆}
    \definition{s.}{pedaço; pedaço (de terra); peça; algo que forma um pedaço ou massa}
  \seealsoref{圆}{yuan2}
  \end{Phonetics}
\end{Entry}

\begin{Entry}{坚}{7}{⼟}
  \begin{Phonetics}{坚}{jian1}
    \definition*{s.}{Sobrenome Jian}
    \definition{adj.}{duro; firme; sólido; forte | firme; resoluto; constante}
    \definition{s.}{fortaleza; fortificação; um ponto fortemente fortificado; coisas sólidas, principalmente referindo-se a posições | armadura}
  \end{Phonetics}
\end{Entry}

\begin{Entry}{坚决}{7,6}{⼟、⼎}
  \begin{Phonetics}{坚决}{jian1jue2}[][HSK 3]
    \definition{adj.}{firme; resoluto; (atitude, opinião, ação, etc.) determinado e inabalável}
  \end{Phonetics}
\end{Entry}

\begin{Entry}{坚守}{7,6}{⼟、⼧}
  \begin{Phonetics}{坚守}{jian1shou3}
    \definition{v.}{agarrar-se}
  \end{Phonetics}
\end{Entry}

\begin{Entry}{坚固}{7,8}{⼟、⼞}
  \begin{Phonetics}{坚固}{jian1gu4}[][HSK 4]
    \definition{adj.}{firme; sólido; robusto; forte; durável; firmemente unidos e inquebráveis}
  \end{Phonetics}
\end{Entry}

\begin{Entry}{坚定}{7,8}{⼟、⼧}
  \begin{Phonetics}{坚定}{jian1ding4}[][HSK 5]
    \definition{adj.}{firme; inabalável; inamovível; (posição, opinião, vontade, etc.) firme e estável, inabalável}
    \definition{v.}{fortalecer}
  \end{Phonetics}
\end{Entry}

\begin{Entry}{坚持}{7,9}{⼟、⼿}
  \begin{Phonetics}{坚持}{jian1chi2}[][HSK 3]
    \definition{v.}{persistir em; perseverar em; defender; insistir em; manter-se fiel a; aderir a; persistir com determinação e não desistir quando se depara com dificuldades | aderir a; insistir em; não alterar (os princípios, opiniões, pontos de vista originais, etc.)}
  \end{Phonetics}
\end{Entry}

\begin{Entry}{坚强}{7,12}{⼟、⼸}
  \begin{Phonetics}{坚强}{jian1qiang2}[][HSK 3]
    \definition{adj.}{forte; firme; convicto; (qualidades humanas, personalidade, determinação, etc.) firme e forte, não vacila diante das dificuldades}
    \definition{v.}{fortalecer; tornar forte; é a qualidade, a determinação, etc., que não vacilam}
  \end{Phonetics}
\end{Entry}

\begin{Entry}{坝}{7}{⼟}
  \begin{Phonetics}{坝}{ba4}[][HSK 7-9]
    \definition[座,道,个,条]{s.}{barragem | dique; aterro | Dialeto: banco de areia | (usualmente em nomes de lugares) planície}
  \end{Phonetics}
\end{Entry}

\begin{Entry}{坟}{7}{⼟}
  \begin{Phonetics}{坟}{fen2}[][HSK 7-9]
    \definition[座,片,个]{s.}{sepultura; túmulo; cova}
  \end{Phonetics}
\end{Entry}

\begin{Entry}{坟墓}{7,13}{⼟、⼟}
  \begin{Phonetics}{坟墓}{fen2mu4}[][HSK 7-9]
    \definition[座,片,个]{s.}{sepultura; túmulo; a cova funerária e a sepultura acima dela}
  \end{Phonetics}
\end{Entry}

\begin{Entry}{坠}{7}{⼟}
  \begin{Phonetics}{坠}{zhui4}
    \definition{s.}{peso; objeto pendurado | um objeto pendurado; um pingente}
    \definition{v.}{cair; derrubar | curvar para baixo}
  \end{Phonetics}
\end{Entry}

\begin{Entry}{坠落}{7,12}{⼟、⾋}
  \begin{Phonetics}{坠落}{zhui4luo4}
    \definition{v.}{cair}
  \end{Phonetics}
\end{Entry}

\begin{Entry}{坡}{8}{⼟}
  \begin{Phonetics}{坡}{po1}[][HSK 6]
    \definition{adj.}{inclinado}
    \definition{s.}{declive | encosta}
  \end{Phonetics}
\end{Entry}

\begin{Entry}{坦}{8}{⼟}
  \begin{Phonetics}{坦}{tan3}
    \definition*{s.}{Sobrenome Tan}
    \definition{adj.}{nivelado; suave; plano | calmo; composto | aberto; sincero; franco}
  \end{Phonetics}
\end{Entry}

\begin{Entry}{坦克}{8,7}{⼟、⼗}
  \begin{Phonetics}{坦克}{tan3ke4}
    \definition{s.}{(empréstimo linguístico) tanque (veículo militar)}
  \end{Phonetics}
\end{Entry}

\begin{Entry}{垃}{8}{⼟}
  \begin{Phonetics}{垃}{la1}
    \definition[堆]{s.}{lixo}
  \end{Phonetics}
\end{Entry}

\begin{Entry}{垃圾}{8,6}{⼟、⼟}
  \begin{Phonetics}{垃圾}{la1 ji1}[][HSK 4]
    \definition{adj.}{lixo; inútil, ruim ou prejudicial}
    \definition[袋,桶,堆,车,片]{s.}{entulho; lixo; refugo; rejeito; resíduo; coisa inútil que é jogada fora; metáfora para alguém ou algo que perdeu seu valor ou serve a um propósito ruim}
  \end{Phonetics}
\end{Entry}

\begin{Entry}{垃圾工}{8,6,3}{⼟、⼟、⼯}
  \begin{Phonetics}{垃圾工}{la1ji1gong1}
    \definition{s.}{lixeiro | gari}
  \end{Phonetics}
\end{Entry}

\begin{Entry}{垃圾车}{8,6,4}{⼟、⼟、⾞}
  \begin{Phonetics}{垃圾车}{la1ji1che1}
    \definition{s.}{caminhão de lixo}
  \end{Phonetics}
\end{Entry}

\begin{Entry}{垃圾电邮}{8,6,5,7}{⼟、⼟、⽥、⾢}
  \begin{Phonetics}{垃圾电邮}{la1ji1 dian4you2}
    \definition{s.}{\emph{e-mail} de \emph{spam}}
  \seealsoref{垃圾邮件}{la1ji1 you2jian4}
  \end{Phonetics}
\end{Entry}

\begin{Entry}{垃圾邮件}{8,6,7,6}{⼟、⼟、⾢、⼈}
  \begin{Phonetics}{垃圾邮件}{la1ji1 you2jian4}
    \definition{s.}{\emph{spam}, \emph{e-mail} não solicitado}
  \seealsoref{垃圾电邮}{la1ji1 dian4you2}
  \end{Phonetics}
\end{Entry}

\begin{Entry}{垃圾食品}{8,6,9,9}{⼟、⼟、⾷、⼝}
  \begin{Phonetics}{垃圾食品}{la1ji1shi2pin3}
    \definition{s.}{\emph{junk food}}
  \end{Phonetics}
\end{Entry}

\begin{Entry}{垃圾堆}{8,6,11}{⼟、⼟、⼟}
  \begin{Phonetics}{垃圾堆}{la1ji1dui1}
    \definition{s.}{depósito de lixo}
  \end{Phonetics}
\end{Entry}

\begin{Entry}{垃圾筒}{8,6,12}{⼟、⼟、⽵}
  \begin{Phonetics}{垃圾筒}{la1ji1tong3}
    \definition{s.}{cesto de lixo}
  \end{Phonetics}
\end{Entry}

\begin{Entry}{垃圾箱}{8,6,15}{⼟、⼟、⾋}
  \begin{Phonetics}{垃圾箱}{la1ji1xiang1}
    \definition{s.}{cesto de lixo}
  \end{Phonetics}
\end{Entry}

\begin{Entry}{型}{9}{⼟}
  \begin{Phonetics}{型}{xing2}[][HSK 4]
    \definition{s.}{molde; modelo | modelo; tipo; padrão}
  \end{Phonetics}
\end{Entry}

\begin{Entry}{型号}{9,5}{⼟、⼝}
  \begin{Phonetics}{型号}{xing2 hao4}[][HSK 4]
    \definition[个,种]{s.}{modelo; tipo; refere-se ao desempenho, às especificações e ao tamanho de aeronaves, máquinas, implementos agrícolas, etc.}
  \end{Phonetics}
\end{Entry}

\begin{Entry}{垫}{9}{⼟}
  \begin{Phonetics}{垫}{dian4}[][HSK 7-9]
    \definition[个]{s.}{almofada}
    \definition{v.}{colocar algo sob; elevar ou nivelar; encher; preencher | pagar por alguém e esperar ser reembolsado mais tarde | colocar algo sob algo para elevá-lo ou nivelá-lo; usar algo para apoiar, espalhar ou forrar algo para torná-lo mais alto, mais grosso ou mais plano | preencher uma vaga; preencher uma lacuna}
  \end{Phonetics}
\end{Entry}

\begin{Entry}{垫子}{9,3}{⼟、⼦}
  \begin{Phonetics}{垫子}{dian4zi5}[][HSK 7-9]
    \definition{s.}{colchão; esteira; almofada; algo para colocar em uma cama, cadeira, banquinho ou em outro lugar}
  \end{Phonetics}
\end{Entry}

\begin{Entry}{垫底}{9,8}{⼟、⼴}
  \begin{Phonetics}{垫底}{dian4/di3}[][HSK 7-9]
    \definition{v.+compl.}{rebasear; forrar; cobrir}
  \end{Phonetics}
\end{Entry}

\begin{Entry}{城}{9}{⼟}
  \begin{Phonetics}{城}{cheng2}[][HSK 3]
    \definition*{s.}{Sobrenome Cheng}
    \definition[座,道,个]{s.}{muralha da cidade; muralha | cidade | centro de um determinado tipo (por exemplo, negócios, entretenimento, etc.)}
  \end{Phonetics}
\end{Entry}

\begin{Entry}{城乡}{9,3}{⼟、⼄}
  \begin{Phonetics}{城乡}{cheng2 xiang1}[][HSK 6]
    \definition{s.}{cidade e campo; áreas urbanas e rurais; a cidade e o campo | cidade e campo; urbano e rural}
  \end{Phonetics}
\end{Entry}

\begin{Entry}{城区}{9,4}{⼟、⼖}
  \begin{Phonetics}{城区}{cheng2 qu1}[][HSK 6]
    \definition{s.}{cidade propriamente dita (oposto a 郊区) | área metropolitana; área urbana; áreas urbanas e suburbanas (diferentes de 郊区)}
  \seealsoref{郊区}{jiao1 qu1}
  \end{Phonetics}
\end{Entry}

\begin{Entry}{城市}{9,5}{⼟、⼱}
  \begin{Phonetics}{城市}{cheng2shi4}[][HSK 3]
    \definition[个,座]{s.}{cidade; regiões com alta densidade populacional, comércio e indústria desenvolvidos e cuja população é predominantemente não agrícola são geralmente centros políticos, econômicos e culturais das regiões vizinhas}
  \end{Phonetics}
\end{Entry}

\begin{Entry}{城里}{9,7}{⼟、⾥}
  \begin{Phonetics}{城里}{cheng2 li3}[][HSK 5]
    \definition{s.}{na cidade; dentro da cidade; originalmente referia-se à área dentro das muralhas da cidade, agora refere-se principalmente à área urbana}
  \end{Phonetics}
\end{Entry}

\begin{Entry}{城度}{9,9}{⼟、⼴}
  \begin{Phonetics}{城度}{cheng2du4}[][HSK 3]
    \definition{s.}{na cidade; dentro da cidade; originalmente se referia à área dentro das muralhas da cidade, agora se refere principalmente à área urbana}
  \end{Phonetics}
\end{Entry}

\begin{Entry}{城堡}{9,12}{⼟、⼟}
  \begin{Phonetics}{城堡}{cheng2bao3}
    \definition[座,个]{s.}{forte; castelo; cidadela; uma pequena cidade com muralhas que facilitam a defesa}
  \end{Phonetics}
\end{Entry}

\begin{Entry}{城墙}{9,14}{⼟、⼟}
  \begin{Phonetics}{城墙}{cheng2qiang2}[][HSK 7-9]
    \definition{s.}{muralha (da cidade); as altas e grossas muralhas de proteção que cercam a cidade antiga}
  \end{Phonetics}
\end{Entry}

\begin{Entry}{城镇}{9,15}{⼟、⾦}
  \begin{Phonetics}{城镇}{cheng2 zhen4}[][HSK 6]
    \definition[个]{s.}{cidade; cidades e vilas}
  \end{Phonetics}
\end{Entry}

\begin{Entry}{埋}{10}{⼟}
  \begin{Phonetics}{埋}{mai2}[][HSK 6]
    \definition{v.}{cobrir (com terra, neve, etc.); enterrar | esconder | enterrar (uma pessoa morta)}
  \end{Phonetics}
  \begin{Phonetics}{埋}{man2}
    \definition{part.}{caracter formador de palavras}
  \end{Phonetics}
\end{Entry}

\begin{Entry}{埋伏}{10,6}{⼟、⼈}
  \begin{Phonetics}{埋伏}{mai2fu2}
    \definition{s.}{emboscada}
    \definition{v.}{emboscar}
  \end{Phonetics}
\end{Entry}

\begin{Entry}{埦}{11}{⼟}
  \begin{Phonetics}{埦}{wan3}
    \variantof{碗}
  \end{Phonetics}
\end{Entry}

\begin{Entry}{培}{11}{⼟}
  \begin{Phonetics}{培}{pei2}
    \definition{v.}{aterrar com terra; aterrar | fomentar; treinar | cultivar; crescer e desenvolver-se propositalmente}
  \end{Phonetics}
\end{Entry}

\begin{Entry}{培训}{11,5}{⼟、⾔}
  \begin{Phonetics}{培训}{pei2xun4}[][HSK 4]
    \definition{v.}{treinar (trabalhadores técnicos, quadros profissionais, etc.)}
  \end{Phonetics}
\end{Entry}

\begin{Entry}{培训班}{11,5,10}{⼟、⾔、⽟}
  \begin{Phonetics}{培训班}{pei2 xun4 ban1}[][HSK 4]
    \definition{s.}{aula de treinamento; curso de treinamento}
  \end{Phonetics}
\end{Entry}

\begin{Entry}{培育}{11,8}{⼟、⾁}
  \begin{Phonetics}{培育}{pei2yu4}[][HSK 4]
    \definition{v.}{criar; fomentar; educar; procriar; nutrir; cultivar}
  \end{Phonetics}
\end{Entry}

\begin{Entry}{培养}{11,9}{⼟、⼋}
  \begin{Phonetics}{培养}{pei2yang3}[][HSK 4]
    \definition{v.}{cultivar (plantas, microorganismos) | promover; treinar ou desenvolver; educar e treinar para um determinado propósito durante um longo período de tempo; fazer crescer | progredir gradualmente; desenvolver ou cultivar gradualmente (hábito, qualidade, caráter, emoção, estilo, interesse, habilidade, etc.)}
  \end{Phonetics}
\end{Entry}

\begin{Entry}{基}{11}{⼟}
  \begin{Phonetics}{基}{ji1}
    \definition{adj.}{chave; básico; primário; cardinal; fundamental}
    \definition{s.}{base; fundação | base; grupo; radical; (química) uma parte dos átomos contidos na molécula de um composto, quando considerada como uma unidade, é chamada de base}
  \end{Phonetics}
\end{Entry}

\begin{Entry}{基本}{11,5}{⼟、⽊}
  \begin{Phonetics}{基本}{ji1ben3}[][HSK 3]
    \definition{adj.}{básico; fundamental; elementar | principal}
    \definition{adv.}{basicamente; em geral; no geral; em termos gerais}
    \definition{s.}{fundação}
  \end{Phonetics}
\end{Entry}

\begin{Entry}{基本上}{11,5,3}{⼟、⽊、⼀}
  \begin{Phonetics}{基本上}{ji1 ben3 shang4}[][HSK 3]
    \definition{adv.}{basicamente; principalmente | em geral; de modo geral}
  \end{Phonetics}
\end{Entry}

\begin{Entry}{基本功}{11,5,5}{⼟、⽊、⼒}
  \begin{Phonetics}{基本功}{ji1ben3gong1}
    \definition{s.}{treinamento básico; habilidade básica; técnica essencial}
  \end{Phonetics}
\end{Entry}

\begin{Entry}{基本法}{11,5,8}{⼟、⽊、⽔}
  \begin{Phonetics}{基本法}{ji1ben3fa3}
    \definition{s.}{lei básica (constituição)}
  \end{Phonetics}
\end{Entry}

\begin{Entry}{基因}{11,6}{⼟、⼞}
  \begin{Phonetics}{基因}{ji1yin1}
    \definition{s.}{gene}
  \end{Phonetics}
\end{Entry}

\begin{Entry}{基地}{11,6}{⼟、⼟}
  \begin{Phonetics}{基地}{ji1di4}[][HSK 5]
    \definition{s.}{base; como base para alguns negócios | base; um local dedicado à realização de um negócio}
  \end{Phonetics}
\end{Entry}

\begin{Entry}{基金}{11,8}{⼟、⾦}
  \begin{Phonetics}{基金}{ji1jin1}[][HSK 5]
    \definition[只,笔]{s.}{fundo; fundos reservados ou destinados ao estabelecimento ou desenvolvimento de uma empresa}
  \end{Phonetics}
\end{Entry}

\begin{Entry}{基础}{11,10}{⼟、⽯}
  \begin{Phonetics}{基础}{ji1chu3}[][HSK 3]
    \definition[个,种,点,层]{s.}{base; fundamento; fundação; a essência ou o ponto de partida do desenvolvimento das coisas | básico; fundamental; refere-se às condições mínimas | fundação do edifício; base do edifício}
  \end{Phonetics}
\end{Entry}

\begin{Entry}{基督教}{11,13,11}{⼟、⽬、⽁}
  \begin{Phonetics}{基督教}{ji1 du1 jiao4}[][HSK 6]
    \definition*{s.}{Cristianismo; A Religião Cristã | Cristão}
  \end{Phonetics}
\end{Entry}

\begin{Entry}{堆}{11}{⼟}
  \begin{Phonetics}{堆}{dui1}[][HSK 5]
    \definition{clas.}{amontoado; pilha; multidão; usado para pilhas de coisas}
    \definition{s.}{amontoado; pilha; empilhamento | (em nomes de lugares)  colina; monte| multidão de pessoas ou coisas}
    \definition{v.}{empilhar; amontoar; acumular; juntar; reunir}
  \end{Phonetics}
\end{Entry}

\begin{Entry}{堆砌}{11,9}{⼟、⽯}
  \begin{Phonetics}{堆砌}{dui1qi4}[][HSK 7-9]
    \definition{v.}{Figurativo: preencher (escrever com frases elaboradas) | Literário: empilhar (tijolos) | embalar}
  \end{Phonetics}
\end{Entry}

\begin{Entry}{堕}{11}{⼟}
  \begin{Phonetics}{堕}{duo4}
    \definition{v.}{cair; afundar}
  \end{Phonetics}
\end{Entry}

\begin{Entry}{堕落}{11,12}{⼟、⾋}
  \begin{Phonetics}{堕落}{duo4luo4}[][HSK 7-9]
    \definition{adj.}{corrupto; decadente}
    \definition{v.}{cair; afundar; degenerar; deteriorar}
  \end{Phonetics}
\end{Entry}

\begin{Entry}{堵}{11}{⼟}
  \begin{Phonetics}{堵}{du3}[][HSK 4]
    \definition*{s.}{Sobrenome Du}
    \definition{adj.}{asfixiado; abafado; sufocado; oprimido}
    \definition{clas.}{usado para paredes}
    \definition{s.}{parede}
    \definition{v.}{impedir; bloquear}
  \end{Phonetics}
\end{Entry}

\begin{Entry}{堵车}{11,4}{⼟、⾞}
  \begin{Phonetics}{堵车}{du3/che1}[][HSK 4]
    \definition{s.}{congestionamento; tráfego intenso; ficar congestionado (no tráfego); bloqueio de vias devido ao excesso de tráfego, etc.}
    \definition{v.+compl.}{congestionar (trânsito)}
  \end{Phonetics}
\end{Entry}

\begin{Entry}{堵塞}{11,13}{⼟、⼟}
  \begin{Phonetics}{堵塞}{du3se4}[][HSK 7-9]
    \definition{v.}{parar; bloquear; tornar obstruído}
  \end{Phonetics}
\end{Entry}

\begin{Entry}{堡}{12}{⼟}
  \begin{Phonetics}{堡}{bao3}
    \definition{s.}{forte; fortaleza | uma terraplenagem | castelo | posição de defesa | usado em nomes de lugares}
  \end{Phonetics}
  \begin{Phonetics}{堡}{bu3}
    \definition{s.}{forte; vila; cidade; assentamento fortificado (uma vila ou cidade cercada por muros de terra, frequentemente usada em nomes de lugares) | cidade; frequentemente usado em nomes de lugares}
  \end{Phonetics}
  \begin{Phonetics}{堡}{pu4}
    \definition{s.}{cidade ou rua (frequentemente usado em nomes de lugares)}
  \end{Phonetics}
\end{Entry}

\begin{Entry}{堡垒}{12,9}{⼟、⼟}
  \begin{Phonetics}{堡垒}{bao3lei3}[][HSK 7-9]
    \definition[处,座,个]{s.}{forte; fortaleza; casamata | bastião | fortificação}
  \end{Phonetics}
\end{Entry}

\begin{Entry}{堤}{12}{⼟}
  \begin{Phonetics}{堤}{di1}[][HSK 7-9]
    \definition[道,条]{s.}{dique; aterro}
  \end{Phonetics}
\end{Entry}

\begin{Entry}{堤坝}{12,7}{⼟、⼟}
  \begin{Phonetics}{堤坝}{di1ba4}[][HSK 7-9]
    \definition[座,道,个]{s.}{diques; barragem; represa}
  \end{Phonetics}
\end{Entry}

\begin{Entry}{塔}{12}{⼟}
  \begin{Phonetics}{塔}{ta3}[][HSK 6]
    \definition*{s.}{Sobrenome Ta}
    \definition[个,座]{s.}{pagode budista; pagode | torre | (química) coluna; torre}[蒸馏塔===torre de destilação]
  \end{Phonetics}
\end{Entry}

\begin{Entry}{塑}{13}{⼟}
  \begin{Phonetics}{塑}{su4}
    \definition{s.}{plástico; material plástico}
    \definition{v.}{modelo; molde; forma}
  \end{Phonetics}
\end{Entry}

\begin{Entry}{塑料}{13,10}{⼟、⽃}
  \begin{Phonetics}{塑料}{su4 liao4}[][HSK 4]
    \definition[块,种]{s.}{plástico; compostos de polímeros feitos de resinas naturais ou sintéticas como componente principal}
  \end{Phonetics}
\end{Entry}

\begin{Entry}{塑料袋}{13,10,11}{⼟、⽃、⾐}
  \begin{Phonetics}{塑料袋}{su4liao4dai4}[][HSK 4]
    \definition[个,只]{s.}{saco plástico; sacola de plástico}
  \end{Phonetics}
\end{Entry}

\begin{Entry}{塞}{13}{⼟}
  \begin{Phonetics}{塞}{sai1}[][HSK 6]
    \definition{s.}{rolha; plugue}
    \definition{v.}{encher; conectar; preencher; espremer; bloquear | superar (para comparação)}
  \end{Phonetics}
\end{Entry}

\begin{Entry}{填}{13}{⼟}
  \begin{Phonetics}{填}{tian2}
    \definition{v.}{encher; rechear | reabastecer; suplementar; complementar | preencher; escrever dados em uma caixa (em um questionário ou formulário da \emph{Web})}
  \end{Phonetics}
\end{Entry}

\begin{Entry}{填空}{13,8}{⼟、⽳}
  \begin{Phonetics}{填空}{tian2kong4}[][HSK 4]
    \definition{v.}{preencher o espaço em branco (por exemplo, em um teste)}
  \end{Phonetics}
\end{Entry}

\begin{Entry}{墓}{13}{⼟}
  \begin{Phonetics}{墓}{mu4}
    \definition[座,个,号]{s.}{sepultura; túmulo; mausoléu}
  \end{Phonetics}
\end{Entry}

\begin{Entry}{墙}{14}{⼟}
  \begin{Phonetics}{墙}{qiang2}[][HSK 2]
    \definition[面,堵,道]{s.}{parede; barreira ou perímetro construído com tijolos, pedras, etc. | qualquer coisa com a forma ou função de uma parede; a parte de um objeto que funciona como parede ou divisória}
    \definition{v.}{(gíria) bloquear (um website) (usado geralmente na voz passiva: 被墙)}
  \end{Phonetics}
\end{Entry}

\begin{Entry}{墙纸}{14,7}{⼟、⽷}
  \begin{Phonetics}{墙纸}{qiang2zhi3}
    \definition{s.}{papel de parede}
  \end{Phonetics}
\end{Entry}

\begin{Entry}{墙壁}{14,16}{⼟、⼟}
  \begin{Phonetics}{墙壁}{qiang2 bi4}[][HSK 5]
    \definition[面,堵,道]{s.}{parede; barreira ou perímetro construído com tijolos, pedras ou terra}
  \end{Phonetics}
\end{Entry}

\begin{Entry}{墬}{14}{⼟}
  \begin{Phonetics}{墬}{di4}
    \variantof{地}
  \end{Phonetics}
\end{Entry}

\begin{Entry}{增}{15}{⼟}
  \begin{Phonetics}{增}{zeng1}[][HSK 5]
    \definition*{s.}{Sobrenome Zeng}
    \definition{v.}{aumentar; ganhar; adicionar}
  \end{Phonetics}
\end{Entry}

\begin{Entry}{增大}{15,3}{⼟、⼤}
  \begin{Phonetics}{增大}{zeng1 da4}[][HSK 5]
    \definition{v.}{ampliar; expandir; estender | amplificar}
  \end{Phonetics}
\end{Entry}

\begin{Entry}{增长}{15,4}{⼟、⾧}
  \begin{Phonetics}{增长}{zeng1 zhang3}[][HSK 3]
    \definition{v.}{subir; crescer; aumentar; melhorar a partir da base existente}
  \end{Phonetics}
\end{Entry}

\begin{Entry}{增加}{15,5}{⼟、⼒}
  \begin{Phonetics}{增加}{zeng1jia1}[][HSK 3]
    \definition{v.}{adicionar; aumentar; incrementar; adicionar mais ao que já existe}
  \end{Phonetics}
\end{Entry}

\begin{Entry}{增产}{15,6}{⼟、⼇}
  \begin{Phonetics}{增产}{zeng1/chan3}[][HSK 5]
    \definition{v.+compl.}{aumentar a produção}
  \end{Phonetics}
\end{Entry}

\begin{Entry}{增多}{15,6}{⼟、⼣}
  \begin{Phonetics}{增多}{zeng1 duo1}[][HSK 5]
    \definition{v.}{aumentar; crescer em número ou quantidade}
  \end{Phonetics}
\end{Entry}

\begin{Entry}{增进}{15,7}{⼟、⾡}
  \begin{Phonetics}{增进}{zeng1 jin4}[][HSK 6]
    \definition{v.}{melhorar; promover; aprofundar}
  \end{Phonetics}
\end{Entry}

\begin{Entry}{增值}{15,10}{⼟、⼈}
  \begin{Phonetics}{增值}{zeng1 zhi2}[][HSK 6]
    \definition{s.}{aumento de valor; apreciação; incremento | valor agregado}
  \end{Phonetics}
\end{Entry}

\begin{Entry}{增速}{15,10}{⼟、⾡}
  \begin{Phonetics}{增速}{zeng1su4}
    \definition{s.}{(economia) taxa de crescimento}
    \definition{v.}{acelerar;}
  \end{Phonetics}
\end{Entry}

\begin{Entry}{增强}{15,12}{⼟、⼸}
  \begin{Phonetics}{增强}{zeng1 qiang2}[][HSK 5]
    \definition{v.}{impulsionar; aprimorar; aumentar; fortalecer; tornar mais forte ou mais poderoso}
  \end{Phonetics}
\end{Entry}

\begin{Entry}{壁}{16}{⼟}
  \begin{Phonetics}{壁}{bi4}
    \definition*{s.}{Bi, a décima quarta das vinte e oito constelações em que a esfera celeste foi dividida, consistindo em duas estrelas em linha reta, uma em Pégaso e a outra em Andrômeda | A Estrela Bìxìu, uma das Vinte e Oito Mansões da astronomia tradicional chinesa}
    \definition[道]{s.}{parede | superfície plana como uma parede | penhasco | muralha; parapeito | barreira}
  \end{Phonetics}
\end{Entry}

\begin{Entry}{壁纸}{16,7}{⼟、⽷}
  \begin{Phonetics}{壁纸}{bi4zhi3}
    \definition{s.}{papel de parede; papel colado em paredes internas para decoração ou proteção, com diversos tipos e cores}
  \end{Phonetics}
\end{Entry}

\begin{Entry}{壁画}{16,8}{⼟、⽥}
  \begin{Phonetics}{壁画}{bi4hua4}[][HSK 7-9]
    \definition[幅]{s.}{mural; afresco; desenhos nas paredes ou tetos de edifícios}
  \end{Phonetics}
\end{Entry}

\begin{Entry}{壁虎}{16,8}{⼟、⾌}
  \begin{Phonetics}{壁虎}{bi4hu3}
    \definition{s.}{lagartixa}
  \end{Phonetics}
\end{Entry}

\begin{Entry}{壤}{20}{⼟}
  \begin{Phonetics}{壤}{rang3}
    \definition{s.}{solo | terra | (literário) a terra (em contraste com o céu 天)}
  \end{Phonetics}
\end{Entry}

%%%%% EOF %%%%%

