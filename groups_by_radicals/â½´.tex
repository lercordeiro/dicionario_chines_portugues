%%%
%%% Radical "⽴"
%%%

\section*{Radical 117: ``⽴''}\addcontentsline{toc}{section}{Radical 117: ⽴}

\begin{entry}{立}{5}{⽴}
  \begin{phonetics}{立}{li4}[][HSK 5]
    \definition{adj.}{ereto; vertical; na vertical}
    \definition{adv.}{imediatamente; instantaneamente}
    \definition{v.}{ficar em pé, com os pés no chão ou apoiados em algum objeto; o objeto deve estar na vertical | erguer; colocar (ou levantar) algo; colocar em pé | encontrar; criar; elaborar; formular; estabelecer | configurar; fundar; estabelecer | viver; existir | ascender ao trono; antigamente, referia-se à ascensão ao trono de um monarca | nomear; designar; antigamente, significava estabelecer uma determinada posição ou status}
  \end{phonetics}
\end{entry}

\begin{entry}{立场}{5,6}{⽴、⼟}
  \begin{phonetics}{立场}{li4chang3}[][HSK 5]
    \definition[个]{s.}{posição; postura; a posição e a atitude adotadas ao reconhecer e lidar com os problemas | ponto de vista; refere-se especificamente à atitude de reconhecer e lidar com questões a partir dos interesses de uma determinada classe, ou seja, a posição de classe}
  \end{phonetics}
\end{entry}

\begin{entry}{立即}{5,7}{⽴、⼙}
  \begin{phonetics}{立即}{li4ji2}[][HSK 4]
    \definition{adv.}{prontamente; imediatamente; de imediato}
  \end{phonetics}
\end{entry}

\begin{entry}{立刻}{5,8}{⽴、⼑}
  \begin{phonetics}{立刻}{li4ke4}[][HSK 3]
    \definition{adv.}{imediatamente; de ​​uma vez; indica que algo acontecerá imediatamente após um determinado momento}
  \end{phonetics}
\end{entry}

\begin{entry}{立法}{5,8}{⽴、⽔}
  \begin{phonetics}{立法}{li4fa3}
    \definition{s.}{legislação}
    \definition{v.}{promulgar leis | legislar}
  \end{phonetics}
\end{entry}

\begin{entry}{竖}{9}{⽴}
  \begin{phonetics}{竖}{shu4}
    \definition*{s.}{sobrenome Shu}
    \definition{adj.}{vertical; ereto; perpendicular ao solo}
    \definition{s.}{traço vertical (em caracteres chineses) | empregados domésticos; jovens criados}
    \definition{v.}{colocar em pé; erguer; ficar de pé; colocar o objeto perpendicular ao solo}
  \end{phonetics}
\end{entry}

\begin{entry}{站}{10}{⽴}
  \begin{phonetics}{站}{zhan4}[][HSK 1,2]
    \definition*{s.}{sobrenome Zhan}
    \definition{s.}{parada; estação; ponto de parada | central; estação; instituição criada para um determinado tipo de atividade | filial de uma empresa ou organização; local de trabalho criado para realizar uma determinada tarefa | \emph{website}; na rede de computadores, refere-se a um \emph{site}}
    \definition{v.}{ficar em pé; estar em pé | parar; interromper; fazer uma pausa}
  \end{phonetics}
\end{entry}

\begin{entry}{站长}{10,4}{⽴、⾧}
  \begin{phonetics}{站长}{zhan4zhang3}
    \definition{s.}{pessoa responsável pela estação de trem | chefe da estação | \emph{webmaster} | gerente de centro de voluntariado}
  \end{phonetics}
\end{entry}

\begin{entry}{站台}{10,5}{⽴、⼝}
  \begin{phonetics}{站台}{zhan4tai2}
    \definition{s.}{plataforma (em uma estação ferroviária)}
  \end{phonetics}
\end{entry}

\begin{entry}{站住}{10,7}{⽴、⼈}
  \begin{phonetics}{站住}{zhan4 zhu4}[][HSK 2]
    \definition{v.}{parar; deter; parar enquanto se move | ficar firme nos pés; manter os pés; permanecer firme | manter-se firme; consolidar a posição de alguém; estabelecer-se em uma determinada unidade ou lugar | sustentar a opinião}
  \end{phonetics}
\end{entry}

\begin{entry}{站姿}{10,9}{⽴、⼥}
  \begin{phonetics}{站姿}{zhan4zi1}
    \definition{s.}{postura}
  \end{phonetics}
\end{entry}

\begin{entry}{站点}{10,9}{⽴、⽕}
  \begin{phonetics}{站点}{zhan4dian3}
    \definition{s.}{\emph{website}}
  \end{phonetics}
\end{entry}

\begin{entry}{竞争}{10,6}{⽴、⼑}
  \begin{phonetics}{竞争}{jing4zheng1}[][HSK 5]
    \definition{v.}{competir; disputar; lutar; entre duas ou mais partes; em prol de seus próprios interesses; lutar pela vitória por meio de uma disputa de sua própria força contra outra}
  \end{phonetics}
\end{entry}

\begin{entry}{竞赛}{10,14}{⽴、⾙}
  \begin{phonetics}{竞赛}{jing4sai4}[][HSK 5]
    \definition[个]{s.}{concurso; competição; partida; corrida}
    \definition{v.}{correr; competir; competir uns com os outros por superioridade; em esportes, produção e outras atividades, para comparar competência, habilidade etc., usado principalmente na linguagem falada}
  \end{phonetics}
\end{entry}

\begin{entry}{童年}{12,6}{⽴、⼲}
  \begin{phonetics}{童年}{tong2 nian2}[][HSK 4]
    \definition{s.}{infância; primeiros anos de vida}
  \end{phonetics}
\end{entry}

\begin{entry}{童话}{12,8}{⽴、⾔}
  \begin{phonetics}{童话}{tong2hua4}[][HSK 4]
    \definition[个,部]{s.}{conto de fadas; gênero de literatura infantil no qual as histórias adequadas para a diversão das crianças são escritas com muita imaginação, fantasia e exagero}
  \end{phonetics}
\end{entry}

\begin{entry}{端}{14}{⽴}
  \begin{phonetics}{端}{duan1}
    \definition*{s.}{sobrenome Duan}
    \definition{adj.}{adequado; próprio | reto; correto}
    \definition{s.}{fim; extremidade | começo | item; ponto; pista, projeto ou aspecto | causa; razão | problema; incidente; coisas (geralmente se refere a coisas ruins, como acidentes, disputas, etc.)}
    \definition{v.}{carregar; segurar algo nivelado com ambas as mãos; segurar algo horizontalmente | erradicar; eliminar; acabar com; remover completamente; varrer | dar ares de superioridade | revelar}
  \end{phonetics}
\end{entry}

\begin{entry}{端午节}{14,4,5}{⽴、⼗、⾋}
  \begin{phonetics}{端午节}{duan1wu3jie2}
    \definition*{s.}{Festa do Duplo Cinco, Festival dos Barcos-Dragão (5º~dia do quinto mês lunar)}
  \end{phonetics}
\end{entry}

%%%%% EOF %%%%%

