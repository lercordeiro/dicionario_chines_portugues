%%%
%%% Radical "⼚"
%%%

\section*{Radical 27: ``⼚''}\addcontentsline{toc}{section}{Radical 27: ⼚}

\begin{entry}{厂}{2}{⼚}[Kangxi 27]
  \begin{phonetics}{厂}{chang3}[][HSK 3]
    \definition[家]{s.}{fábrica; moinho; planta; obra | pátio; depósito}
  \end{phonetics}
  \begin{phonetics}{厂}{han3}
    \definition{s.}{radical ``penhasco'' em caracteres chineses (radical Kangxi 27)}
  \end{phonetics}
\end{entry}

\begin{entry}{厂长}{2,4}{⼚、⾧}
  \begin{phonetics}{厂长}{chang3 zhang3}[][HSK 5]
    \definition[任,位,个]{s.}{diretor de fábrica; gerente de fábrica; líder responsável pela produção, pela vida e por todos os outros assuntos de toda a fábrica}
  \end{phonetics}
\end{entry}

\begin{entry}{历史}{4,5}{⼚、⼝}
  \begin{phonetics}{历史}{li4shi3}[][HSK 4]
    \definition[个,门,段]{s.}{histórico; registro do passado; processo de desenvolvimento da natureza e da sociedade humana; processo de desenvolvimento de uma coisa ou pessoa | história; eventos passados; experiência | história; refere-se ao tema da história}
  \end{phonetics}
\end{entry}

\begin{entry}{厉害}{5,10}{⼚、⼧}
  \begin{phonetics}{厉害}{li4hai5}
    \definition{adj.}{severo | rigoroso | exigente | radical | violento | feroz}
  \end{phonetics}
\end{entry}

\begin{entry}{压}{6}{⼚}
  \begin{phonetics}{压}{ya1}[][HSK 3]
    \definition{v.}{pressionar; empurrar para baixo; segurar; pesar | manter sob controle; controlar; manter sob; reprimir |exercer pressão sobre; suprimir; desencorajar; intimidar | aproximar-se; estar chegando perto}
  \end{phonetics}
  \begin{phonetics}{压}{ya4}
    \definition{adv.}{fundamentalmente; nunca (usado principalmente em frases negativas)}
  \seealsoref{压根儿}{ya4 gen1r5}
  \end{phonetics}
\end{entry}

\begin{entry}{压力}{6,2}{⼚、⼒}
  \begin{phonetics}{压力}{ya1li4}[][HSK 3]
    \definition[份,个]{s.}{pressão; força atuando perpendicularmente à superfície de um objeto | pressão; força esmagadora; metáfora para a força que coage e intimida as pessoas (principalmente nos aspectos espirituais e psicológicos) | tensão; fardo; fardos econômicos, psicológicos e espirituais que o mundo exterior traz às pessoas}
  \end{phonetics}
\end{entry}

\begin{entry}{压岁钱}{6,6,10}{⼚、⼭、⾦}
  \begin{phonetics}{压岁钱}{ya1sui4qian2}
    \definition{s.}{dinheiro da sorte | dinheiro dado às crianças como presente no Ano Novo Chinês}
  \end{phonetics}
\end{entry}

\begin{entry}{压根儿}{6,10,2}{⼚、⽊、⼉}
  \begin{phonetics}{压根儿}{ya4 gen1r5}
    \definition{adv.}{fundamentalmente; nunca (usado principalmente em frases negativas)}
  \end{phonetics}
\end{entry}

\begin{entry}{压碎}{6,13}{⼚、⽯}
  \begin{phonetics}{压碎}{ya1sui4}
    \definition{v.}{esmagar em pedaços}
  \end{phonetics}
\end{entry}

\begin{entry}{压韵}{6,13}{⼚、⾳}
  \begin{phonetics}{压韵}{ya1yun4}
    \variantof{押韵}
  \end{phonetics}
\end{entry}

\begin{entry}{厕纸}{8,7}{⼚、⽷}
  \begin{phonetics}{厕纸}{ce4zhi3}
    \definition{s.}{papel higiênico}
  \end{phonetics}
\end{entry}

\begin{entry}{厕所}{8,8}{⼚、⼾}
  \begin{phonetics}{厕所}{ce4suo3}
    \definition[间,处]{s.}{lavatório | \emph{toilette}}
  \end{phonetics}
\end{entry}

\begin{entry}{厘米}{9,6}{⼚、⽶}
  \begin{phonetics}{厘米}{li2mi3}[][HSK 4]
    \definition{clas.}{centímetro; unidade de comprimento, símbolo cm, 1 metro é igual a 100 centímetros}
  \end{phonetics}
\end{entry}

\begin{entry}{厚}{9}{⼚}
  \begin{phonetics}{厚}{hou4}[][HSK 4]
    \definition*{s.}{sobrenome Hou}
    \definition{adj.}{grosso; espesso | profundo | bondoso; gentil; magnânimo | grande; generoso | rico ou forte em sabor}
    \definition{s.}{espessura; profundidade}
    \definition{v.}{favorecer; enfatizar}
  \end{phonetics}
\end{entry}

\begin{entry}{原木}{10,4}{⼚、⽊}
  \begin{phonetics}{原木}{yuan2mu4}
    \definition{s.}{registro | \emph{logs}}
  \end{phonetics}
\end{entry}

\begin{entry}{原则}{10,6}{⼚、⼑}
  \begin{phonetics}{原则}{yuan2ze2}[][HSK 4]
    \definition{adv.}{em geral; em princípio; refere-se a um aspecto geral; geralmente}
    \definition[个]{s.}{princípios; leis ou padrões pelos quais alguém fala ou age}
  \end{phonetics}
\end{entry}

\begin{entry}{原因}{10,6}{⼚、⼞}
  \begin{phonetics}{原因}{yuan2yin1}[][HSK 2]
    \definition[个]{s.}{causa | razão | motivo}
  \end{phonetics}
\end{entry}

\begin{entry}{原色}{10,6}{⼚、⾊}
  \begin{phonetics}{原色}{yuan2 se4}
    \definition{s.}{cor primária}
  \end{phonetics}
\end{entry}

\begin{entry}{原来}{10,7}{⼚、⽊}
  \begin{phonetics}{原来}{yuan2lai2}[][HSK 2]
    \definition{adv.}{originalmente | como se vê | na verdade}
    \definition{v.}{vir a ser}
  \end{phonetics}
\end{entry}

\begin{entry}{原料}{10,10}{⼚、⽃}
  \begin{phonetics}{原料}{yuan2liao4}[][HSK 4]
    \definition[种,个]{s.}{matéria-prima; refere-se a materiais que não foram processados e fabricados, como minérios para metalurgia e algodão para têxteis}
  \end{phonetics}
\end{entry}

\begin{entry}{原理}{10,11}{⼚、⽟}
  \begin{phonetics}{原理}{yuan2li3}
    \definition{s.}{princípio | teoria}
  \end{phonetics}
\end{entry}

\begin{entry}{厨房}{12,8}{⼚、⼾}
  \begin{phonetics}{厨房}{chu2fang2}[][HSK 5]
    \definition[间,个]{s.}{cozinha}
  \end{phonetics}
\end{entry}

%%%%% EOF %%%%%

