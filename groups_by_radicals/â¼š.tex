%%%
%%% Radical "⼚"
%%%

\section*{Radical 27: ``⼚''}\addcontentsline{toc}{section}{Radical 27: ⼚}

\begin{Entry}{厂}{2}{⼚}[Kangxi 27]
  \begin{Phonetics}{厂}{an1}
    \definition{s.}{usado principalmente em nomes pessoais}[他名中有个厂字。===O nome dele contém a palavra `An'.]
  \end{Phonetics}
  \begin{Phonetics}{厂}{chang3}[][HSK 3]
    \definition[家,间]{s.}{fábrica; moinho; planta; obra | pátio; depósito; refere-se a um estabelecimento comercial com um amplo espaço para armazenamento de mercadorias e processamento}
  \end{Phonetics}
  \begin{Phonetics}{厂}{han3}
    \definition[家,间]{s.}{radical ``penhasco'' em caracteres chineses (radical Kangxi 27)}
  \end{Phonetics}
\end{Entry}

\begin{Entry}{厂长}{2,4}{⼚、⾧}
  \begin{Phonetics}{厂长}{chang3 zhang3}[][HSK 5]
    \definition[位,个,名]{s.}{diretor de fábrica; gerente de fábrica; líder responsável pela produção, pela vida e por todos os outros assuntos de toda a fábrica}
  \end{Phonetics}
\end{Entry}

\begin{Entry}{厂家}{2,10}{⼚、⼧}
  \begin{Phonetics}{厂家}{chang3jia1}[][HSK 7-9]
    \definition[个]{s.}{fábrica; refere-se aos aspectos de fábrica ou fábrica}
  \end{Phonetics}
\end{Entry}

\begin{Entry}{厂商}{2,11}{⼚、⼝}
  \begin{Phonetics}{厂商}{chang3 shang1}[][HSK 6]
    \definition[家,个]{s.}{empresa; fornecedor; fábrica; negócio; fabricante; uma unidade que produz e vende produtos; uma pessoa que administra uma fábrica}
  \end{Phonetics}
\end{Entry}

\begin{Entry}{厄}{4}{⼚}
  \begin{Phonetics}{厄}{e4}
    \definition[个]{s.}{ponto estratégico; lugares perigosos | adversidade; desastre; dificuldade}
    \definition{v.}{estar em perigo; estar abandonado; estar; encurralado}
  \end{Phonetics}
\end{Entry}

\begin{Entry}{厄运}{4,7}{⼚、⾡}
  \begin{Phonetics}{厄运}{e4yun4}[][HSK 7-9]
    \definition{s.}{adversidade; infortúnio; experiência infeliz}
  \end{Phonetics}
\end{Entry}

\begin{Entry}{厅}{4}{⼚}
  \begin{Phonetics}{厅}{ting1}[][HSK 5]
    \definition{s.}{salão; sala grande para reuniões ou receber convidados | escritório; nome de um departamento administrativo de uma grande organização | departamento governamental a nível provincial; nomes de alguns órgãos estaduais}
  \end{Phonetics}
\end{Entry}

\begin{Entry}{历}{4}{⼚}
  \begin{Phonetics}{历}{li4}
    \definition{adj.}{todas as anteriores (ocasiões, sessões, etc.)}
    \definition{adv.}{por toda parte; um por um}
    \definition{s.}{experiência; registro | almanaque; anuário; calendário}
    \definition{v.}{passar por; sofrer; experimentar | passar através; atravessar}
  \end{Phonetics}
\end{Entry}

\begin{Entry}{历史}{4,5}{⼚、⼝}
  \begin{Phonetics}{历史}{li4shi3}[][HSK 4]
    \definition[段]{s.}{história; registro do passado; processo de desenvolvimento da natureza e da sociedade humana; processo de desenvolvimento de uma coisa ou pessoa | história; eventos passados; experiência | história; refere-se ao tema da história}
  \end{Phonetics}
\end{Entry}

\begin{Entry}{厉}{5}{⼚}
  \begin{Phonetics}{厉}{li4}
    \definition*{s.}{Sobrenome Li}
    \definition{adj.}{rigoroso; estrito | severo; sombrio; sério}
  \end{Phonetics}
\end{Entry}

\begin{Entry}{厉害}{5,10}{⼚、⼧}
  \begin{Phonetics}{厉害}{li4hai5}[][HSK 5]
    \definition{adj.}{feroz; severo; descreve uma situação como sendo muito grave | severo; duro; descreve uma pessoa que é exigente com os outros, muito severa, muitas vezes deixando os outros um pouco assustados | incrível; talentoso; impressionante; usado para avaliar a capacidade de uma pessoa ou algo que ela fez que é notável | aterrorizante; assustador; descreve animais ferozes e assustadores}
  \end{Phonetics}
\end{Entry}

\begin{Entry}{压}{6}{⼚}
  \begin{Phonetics}{压}{ya1}[][HSK 3]
    \definition{v.}{pressionar; empurrar para baixo; segurar; pesar | acalmar emoções agitadas ou situações ruins; tranquilizar | intimidar; reprimir; exercer pressão sobre; usar poder, posição ou padrões morais para coagir ou restringir as pessoas, impedindo-as de se expressar, decidir ou se desenvolver livremente | aproximar-se; estar chegando perto | arquivar; deixar de lado | pressionar; metáfora para uma grande carga emocional e psicológica | superar; ultrapassar; voz, capacidade e presença mais fortes do que os outros | apostar em um determinado resultado ao jogar | pressionar; força na superfície de contato do objeto}
  \end{Phonetics}
  \begin{Phonetics}{压}{ya4}
    \definition{adv.}{fundamentalmente; nunca (usado principalmente em frases negativas)}
  \seealsoref{压根儿}{ya4gen1r5}
  \end{Phonetics}
\end{Entry}

\begin{Entry}{压力}{6,2}{⼚、⼒}
  \begin{Phonetics}{压力}{ya1li4}[][HSK 3]
    \definition[份,个]{s.}{pressão; força atuando perpendicularmente à superfície de um objeto | pressão; força esmagadora; metáfora para a força que coage e intimida as pessoas (principalmente nos aspectos espirituais e psicológicos) | tensão; fardo; os encargos econômicos, psicológicos e espirituais impostos pelo mundo exterior}
  \end{Phonetics}
\end{Entry}

\begin{Entry}{压岁钱}{6,6,10}{⼚、⼭、⾦}
  \begin{Phonetics}{压岁钱}{ya1sui4qian2}
    \definition{s.}{dinheiro da sorte | dinheiro dado às crianças como presente no Ano Novo Chinês}
  \end{Phonetics}
\end{Entry}

\begin{Entry}{压迫}{6,8}{⼚、⾡}
  \begin{Phonetics}{压迫}{ya1po4}[][HSK 6]
    \definition{v.}{oprimir; reprimir; confiar no poder para suprimir e forçar | contrair; uma força externa comprime uma parte de um organismo}
  \end{Phonetics}
\end{Entry}

\begin{Entry}{压根儿}{6,10,2}{⼚、⽊、⼉}
  \begin{Phonetics}{压根儿}{ya4gen1r5}
    \definition{adv.}{(geralmente no negativo) nunca; fundamentalmente}
  \end{Phonetics}
\end{Entry}

\begin{Entry}{压碎}{6,13}{⼚、⽯}
  \begin{Phonetics}{压碎}{ya1sui4}
    \definition{v.}{esmagar em pedaços}
  \end{Phonetics}
\end{Entry}

\begin{Entry}{压韵}{6,13}{⼚、⾳}
  \begin{Phonetics}{压韵}{ya1yun4}
    \variantof{押韵}
  \end{Phonetics}
\end{Entry}

\begin{Entry}{厕}{8}{⼚}
  \begin{Phonetics}{厕}{ce4}
    \definition[个,间]{s.}{latrina; fossa sanitária; (componente formador de palavras)}
  \seealsoref{茅厕}{mao2ce4}
  \end{Phonetics}
  \begin{Phonetics}{厕}{si5}
    \definition{s.}{componente formador de palavras | latrina; fossa sanitária}
  \seealsoref{茅厕}{mao2ce4}
  \end{Phonetics}
\end{Entry}

\begin{Entry}{厕纸}{8,7}{⼚、⽷}
  \begin{Phonetics}{厕纸}{ce4zhi3}
    \definition{s.}{papel higiênico}
  \end{Phonetics}
\end{Entry}

\begin{Entry}{厕所}{8,8}{⼚、⼾}
  \begin{Phonetics}{厕所}{ce4suo3}[][HSK 6]
    \definition[个,间]{s.}{banheiro; lavatório; sanitário; latrina; um lugar para as pessoas urinarem e defecarem}
  \end{Phonetics}
\end{Entry}

\begin{Entry}{厘}{9}{⼚}
  \begin{Phonetics}{厘}{li2}
    \definition*{s.}{Sobrenome Li}
    \definition{clas.}{li, uma unidade tradicional de comprimento, igual a 0,001 chi (市尺), e equivalente a 0,333 milímetro ou 0,013 polegada | li, uma unidade tradicional de peso, igual a 0,0001 jin (市斤), e equivalente a 5 centigramas ou 0,771 grãos | li, uma unidade tradicional de área, igual a 0,01 mu (市亩), e equivalente a 0,667 metro quadrado ou 0,797 jarda quadrada | li, unidade monetária chinesa, igual a 0,1 fen ou 0,001 yuan | li, unidade de taxa de juros, igual a 0,1\% de juros mensais ou 1\% de juros anuais | quantidade muito pequena; fração; o mínimo}
    \definition{v.}{regular; retificar | administrar}
  \seealsoref{市尺}{shi4 chi3}
  \seealsoref{市斤}{shi4jin1}
  \seealsoref{市亩}{shi4mu3}
  \end{Phonetics}
\end{Entry}

\begin{Entry}{厘米}{9,6}{⼚、⽶}
  \begin{Phonetics}{厘米}{li2mi3}[][HSK 4]
    \definition{clas.}{centímetro; unidade de comprimento, símbolo cm, 1 metro é igual a 100 centímetros}
  \end{Phonetics}
\end{Entry}

\begin{Entry}{厚}{9}{⼚}
  \begin{Phonetics}{厚}{hou4}[][HSK 4]
    \definition*{s.}{Sobrenome Hou}
    \definition{adj.}{espesso; grosso (oposto a 薄) | profundo | gentil; magnânimo | grande; generoso | rico ou forte em sabor}
    \definition[米,厘米]{s.}{espessura | profundidade}
    \definition{v.}{favorecer; enfatizar}
  \seealsoref{薄}{bao2}
  \end{Phonetics}
\end{Entry}

\begin{Entry}{原}{10}{⼚}
  \begin{Phonetics}{原}{yuan2}[][HSK 6]
    \definition*{s.}{Sobrenome Yuan}
    \definition{adj.}{inicial; básico; primitivo | cru; bruto; não processado | virgem; primário; original; antigo; inalterado}
    \definition{adv.}{originalmente}
    \definition[项,条,片]{s.}{planície; país aberto; terreno plano e amplo | início; fonte; origem; aparência original | origem; a raiz ou o começo das coisas}
    \definition{v.}{desculpar; perdoar; tolerar; compreender | rastrear; sondar; investigar (a origem das coisas)}
  \end{Phonetics}
\end{Entry}

\begin{Entry}{原木}{10,4}{⼚、⽊}
  \begin{Phonetics}{原木}{yuan2mu4}
    \definition{s.}{registro | \emph{logs}}
  \end{Phonetics}
\end{Entry}

\begin{Entry}{原先}{10,6}{⼚、⼉}
  \begin{Phonetics}{原先}{yuan2xian1}[][HSK 5]
    \definition{adj.}{antigo; original}
    \definition{s.}{antigamente; no início; no passado; no começo}
  \end{Phonetics}
\end{Entry}

\begin{Entry}{原则}{10,6}{⼚、⼑}
  \begin{Phonetics}{原则}{yuan2ze2}[][HSK 4]
    \definition{adv.}{em geral; em princípio; refere-se a um aspecto geral; geralmente}
    \definition[个,条,项,点]{s.}{princípios; leis ou padrões pelos quais alguém fala ou age}
  \end{Phonetics}
\end{Entry}

\begin{Entry}{原因}{10,6}{⼚、⼞}
  \begin{Phonetics}{原因}{yuan2yin1}[][HSK 2]
    \definition[个,条,种,些]{s.}{causa; razão; motivo; as condições que fazem com que algo aconteça ou produzam um certo resultado}
  \end{Phonetics}
\end{Entry}

\begin{Entry}{原有}{10,6}{⼚、⽉}
  \begin{Phonetics}{原有}{yuan2 you3}[][HSK 5]
    \definition{v.}{já estar pronto, não é necessário fazer ou procurar nada; ser o original}
  \end{Phonetics}
\end{Entry}

\begin{Entry}{原色}{10,6}{⼚、⾊}
  \begin{Phonetics}{原色}{yuan2 se4}
    \definition{s.}{cor primária}
  \end{Phonetics}
\end{Entry}

\begin{Entry}{原告}{10,7}{⼚、⼝}
  \begin{Phonetics}{原告}{yuan2gao4}[][HSK 6]
    \definition{s.}{(em casos civis) autor; solicitante | (em casos criminais) promotor; acusador; reclamante (oposto a 被告)}
  \seealsoref{被告}{bei4gao4}
  \end{Phonetics}
\end{Entry}

\begin{Entry}{原来}{10,7}{⼚、⽊}
  \begin{Phonetics}{原来}{yuan2lai2}[][HSK 2]
    \definition{adj.}{original; anterior; em primeiro lugar; inicialmente; inalterado}
    \definition{adv.}{na verdade; de fato; como se vê; expressar compreensão repentina}
    \definition{s.}{a princípio; no passado; antigamente}
  \end{Phonetics}
\end{Entry}

\begin{Entry}{原始}{10,8}{⼚、⼥}
  \begin{Phonetics}{原始}{yuan2shi3}[][HSK 5]
    \definition{s.}{original; de primeira mão | primitivo; mais antigo; não desenvolvido; não civilizado}
  \end{Phonetics}
\end{Entry}

\begin{Entry}{原料}{10,10}{⼚、⽃}
  \begin{Phonetics}{原料}{yuan2liao4}[][HSK 4]
    \definition[种,个]{s.}{matéria-prima; refere-se a materiais que não foram processados e fabricados, como minérios para metalurgia e algodão para têxteis}
  \end{Phonetics}
\end{Entry}

\begin{Entry}{原谅}{10,10}{⼚、⾔}
  \begin{Phonetics}{原谅}{yuan2liang4}[][HSK 6]
    \definition{v.}{perdoar; perdoar a negligência, os erros ou as falhas das pessoas sem culpá-las ou puni-las}
  \end{Phonetics}
\end{Entry}

\begin{Entry}{原理}{10,11}{⼚、⽟}
  \begin{Phonetics}{原理}{yuan2li3}[][HSK 5]
    \definition[个,条]{s.}{princípio; axioma; teoria; teoria básica ou princípio científico de significado universal}
  \end{Phonetics}
\end{Entry}

\begin{Entry}{厨}{12}{⼚}
  \begin{Phonetics}{厨}{chu2}
    \definition[个]{s.}{cozinha}
  \end{Phonetics}
\end{Entry}

\begin{Entry}{厨师}{12,6}{⼚、⼱}
  \begin{Phonetics}{厨师}{chu2 shi1}[][HSK 6]
    \definition[名,位,个]{s.}{chefe de cozinha; cozinheiro; alguém que é bom em cozinhar e faz disso uma profissão}
  \end{Phonetics}
\end{Entry}

\begin{Entry}{厨房}{12,8}{⼚、⼾}
  \begin{Phonetics}{厨房}{chu2fang2}[][HSK 5]
    \definition[间,个]{s.}{cozinha}
  \end{Phonetics}
\end{Entry}

%%%%% EOF %%%%%

