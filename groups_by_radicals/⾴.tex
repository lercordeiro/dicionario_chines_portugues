%%%
%%% Radical "⾴"
%%%
\section*{Radical 181: ``⾴'' (页)}\addcontentsline{toc}{section}{Radical 181: ⾴、页}

%%%%%%%%%% 页 %%%%%%%%%%
\subsection*{页}

\begin{Entry}{页}{6}{⾴}[Kangxi 181]
  \begin{Phonetics}{页}{ye4}[][HSK 1]
    \definition{clas.}{página; folha de papel; lâmina; antigamente, referia-se a uma folha de um livro encadernado; atualmente, refere-se a uma das faces de um livro impresso em ambos os lados}
    \definition{s.}{página; folha de papel; folhas soltas de um livro}
  \end{Phonetics}
\end{Entry}

%%%%%%%%%% 顶 %%%%%%%%%%
\subsection*{顶}

\begin{Entry}{顶}{8}{⾴}
  \begin{Phonetics}{顶}{ding3}[][HSK 4]
    \definition{adv.}{muito (linguagem falada); a maioria; extremamente; expressa o grau mais alto, equivalente a 最 e 极}
    \definition{clas.}{usado para coisas que têm um topo}
    \definition{prep.}{até}
    \definition{s.}{coroa da cabeça; parte mais alta do corpo ou objeto | topo; limite superior; ponto mais alto}
    \definition{v.}{carregar na cabeça; carregar em sua cabeça | empurrar (ou apoiar) para cima; empurrar por baixo (ou por trás) | dar cabeçadas; dar uma coronhada | sustentar; apoiar; suportar | resistir; ir contra; enfrentar | rebater; retorquir; responder de volta | cooperar; enfrentar; apoiar; dar suporte | igualar; ser equivalente a | substituir; tomar o lugar de | assumir o controle; transferir ou adquirir o direito de administrar um negócio ou alugar uma casa ou terreno}
  \seealsoref{极}{ji2}
  \seealsoref{最}{zui4}
  \end{Phonetics}
\end{Entry}

\begin{Entry}{顶多}{8,6}{⾴、⼣}
  \begin{Phonetics}{顶多}{ding3duo1}[][HSK 7-9]
    \definition{adv.}{na melhor das hipóteses; no máximo, na opinião do orador, o número real não será maior que o maior número estimado}
  \end{Phonetics}
\end{Entry}

\begin{Entry}{顶尖}{8,6}{⾴、⼩}
  \begin{Phonetics}{顶尖}{ding3jian1}[][HSK 7-9]
    \definition{adj.}{melhor; de primeira classe; de mais alto nível}
    \definition{s.}{centro; ápice | topo pontiagudo; ponta; pico; a parte mais alta e pontiaguda}
  \end{Phonetics}
\end{Entry}

\begin{Entry}{顶级}{8,6}{⾴、⽷}
  \begin{Phonetics}{顶级}{ding3ji2}[][HSK 7-9]
    \definition{adj.}{de primeira classe; de alta qualidade; de ponta}
  \end{Phonetics}
\end{Entry}

%%%%%%%%%% 项 %%%%%%%%%%
\subsection*{项}

\begin{Entry}{项}{9}{⾴}
  \begin{Phonetics}{项}{xiang4}[][HSK 4]
    \definition*{s.}{Sobrenome: Xiang}
    \definition{clas.}{usado para itens discriminados; taxonomia}
    \definition{s.}{nuca (do pescoço); a parte de trás do pescoço | soma (de dinheiro); fundos para fins especiais | termo; em álgebra, significa uma única fórmula que não é unida por um sinal de mais ou de menos | item}
  \end{Phonetics}
\end{Entry}

\begin{Entry}{项目}{9,5}{⾴、⽬}
  \begin{Phonetics}{项目}{xiang4mu4}[][HSK 4]
    \definition{s.}{evento; categorias em que as coisas são divididas | item; projeto; trabalhos de engenharia, acadêmicos, etc., de conteúdo específico}
  \end{Phonetics}
\end{Entry}

%%%%%%%%%% 顺 %%%%%%%%%%
\subsection*{顺}

\begin{Entry}{顺}{9}{⾴}
  \begin{Phonetics}{顺}{shun4}[][HSK 6]
    \definition{adj.}{(de escritos) legível; claro e bem escrito; organizado | favorável; harmonioso | favorável; bem-sucedido}
    \definition{prep.}{conforme a conveniência de alguém | ao longo; a introdução da rota, situação ou oportunidade que a ação segue pode ser seguida por 着 | com a corrente; na mesma direção |  com; na mesma direção que}
    \definition{v.}{organizar; colocar em ordem; tornar as coisas organizadas ou ordenadas | obedecer; ceder a; agir em submissão a | ser adequado; ser agradável}
  \seealsoref{着}{zhe5}
  \end{Phonetics}
\end{Entry}

\begin{Entry}{顺从}{9,4}{⾴、⼈}
  \begin{Phonetics}{顺从}{shun4cong2}
    \definition{v.}{obedecer | submeter-se}
  \end{Phonetics}
\end{Entry}

\begin{Entry}{顺心}{9,4}{⾴、⼼}
  \begin{Phonetics}{顺心}{shun4xin1}
    \definition{adj.}{satisfatório | satisfeito}
  \end{Phonetics}
\end{Entry}

\begin{Entry}{顺水}{9,4}{⾴、⽔}
  \begin{Phonetics}{顺水}{shun4shui3}
    \definition{v.}{ir com o fluxo}
  \end{Phonetics}
\end{Entry}

\begin{Entry}{顺延}{9,6}{⾴、⼵}
  \begin{Phonetics}{顺延}{shun4yan2}
    \definition{v.}{adiar | procrastinar}
  \end{Phonetics}
\end{Entry}

\begin{Entry}{顺当}{9,6}{⾴、⼹}
  \begin{Phonetics}{顺当}{shun4dang5}
    \definition{adv.}{suavemente}
  \end{Phonetics}
\end{Entry}

\begin{Entry}{顺耳}{9,6}{⾴、⽿}
  \begin{Phonetics}{顺耳}{shun4'er3}
    \definition{adj.}{agradável ao ouvido}
  \end{Phonetics}
\end{Entry}

\begin{Entry}{顺利}{9,7}{⾴、⼑}
  \begin{Phonetics}{顺利}{shun4li4}[][HSK 2]
    \definition{adj.}{sem problemas; com sucesso; sem dificuldades; sem contratempos; sem obstáculos; sem obstáculos ou dificuldades significativas no desempenho das tarefas}
  \end{Phonetics}
\end{Entry}

\begin{Entry}{顺序}{9,7}{⾴、⼴}
  \begin{Phonetics}{顺序}{shun4xu4}[][HSK 4]
    \definition{adv.}{por sua vez; na ordem correta; na devida ordem; na ordem adequada; na ordem apropriada}
    \definition[个]{s.}{ordem; sequência; sucessão; subsequência; sequência simples; ordem de prioridade}
  \end{Phonetics}
\end{Entry}

\begin{Entry}{顺畅}{9,8}{⾴、⽥}
  \begin{Phonetics}{顺畅}{shun4chang4}
    \definition{adj.}{liso e sem obstáculos | fluente}
  \end{Phonetics}
\end{Entry}

\begin{Entry}{顺便}{9,9}{⾴、⼈}
  \begin{Phonetics}{顺便}{shun4bian4}
    \definition{adv.}{convenientemente | de passagem | sem muito esforço extra}
  \end{Phonetics}
\end{Entry}

\begin{Entry}{顺叙}{9,9}{⾴、⼜}
  \begin{Phonetics}{顺叙}{shun4xu4}
    \definition{s.}{narrativa cronológica}
  \end{Phonetics}
\end{Entry}

\begin{Entry}{顺眼}{9,11}{⾴、⽬}
  \begin{Phonetics}{顺眼}{shun4yan3}
    \definition{adj.}{agradável aos olhos}
  \end{Phonetics}
\end{Entry}

\begin{Entry}{顺境}{9,14}{⾴、⼟}
  \begin{Phonetics}{顺境}{shun4jing4}
    \definition{s.}{circunstâncias fáceis (ou favoráveis) (oposto de 逆境)}
  \seealsoref{逆境}{ni4jing4}
  \end{Phonetics}
\end{Entry}

\begin{Entry}{顺嘴}{9,16}{⾴、⼝}
  \begin{Phonetics}{顺嘴}{shun4zui3}
    \definition{v.}{deixar escapar (sem pensar) | ler suavemente (texto) | adequar-se  ao gosto (comida)}
  \end{Phonetics}
\end{Entry}

%%%%%%%%%% 顽 %%%%%%%%%%
\subsection*{顽}

\begin{Entry}{顽}{10}{⾴}
  \begin{Phonetics}{顽}{wan2}
    \definition*{s.}{Sobrenome: Wan}
    \definition{adj.}{estúpido; denso; insensível | teimoso; obstinado; não é facilmente persuadido ou subjugado | travesso; pernicioso | cabeça dura; estúpido e ignorante}
    \definition{v.}{brincar; divertir-se; divertir-se | empregar; recorrer a | envolver-se em; tomar parte em}
  \end{Phonetics}
\end{Entry}

\begin{Entry}{顽皮}{10,5}{⾴、⽪}
  \begin{Phonetics}{顽皮}{wan2 pi2}[][HSK 6]
    \definition{adj.}{atrevido; travesso; arteiro; levado; (crianças, adolescentes, etc.) adoram brincar e causar problemas e não dão ouvidos a conselhos}
  \end{Phonetics}
\end{Entry}

\begin{Entry}{顽强}{10,12}{⾴、⼸}
  \begin{Phonetics}{顽强}{wan2qiang2}[][HSK 6]
    \definition{adj.}{firme; tenaz; indomável; forte; resistente}
  \end{Phonetics}
\end{Entry}

%%%%%%%%%% 顾 %%%%%%%%%%
\subsection*{顾}

\begin{Entry}{顾}{10}{⾴}
  \begin{Phonetics}{顾}{gu4}[][HSK 6]
    \definition*{s.}{Sobrenome: Gu}
    \definition{adv.}{em vez disso; pelo contrário; indica o oposto, equivalente a 却 ou 反而}
    \definition{conj.}{mas; no entanto}
    \definition{v.}{olhar para trás; olhar para; virar-se e olhar para | cuidar de; atender a; levar em conta ou consideração | visitar; chamar | sentir pena de}
  \seealsoref{反而}{fan3'er2}
  \seealsoref{却}{que4}
  \end{Phonetics}
\end{Entry}

\begin{Entry}{顾及}{10,3}{⾴、⼃}
  \begin{Phonetics}{顾及}{gu4ji2}[][HSK 7-9]
    \definition{v.}{atender a; levar em conta; dar consideração a; cuidar de; notar}
  \end{Phonetics}
\end{Entry}

\begin{Entry}{顾不上}{10,4,3}{⾴、⼀、⼀}
  \begin{Phonetics}{顾不上}{gu4bu5shang4}[][HSK 7-9]
    \definition{v.}{não conseguir; não conseguir atender; incapaz de cuidar de (fazer algo)}
  \end{Phonetics}
\end{Entry}

\begin{Entry}{顾不得}{10,4,11}{⾴、⼀、⼻}
  \begin{Phonetics}{顾不得}{gu4bu5de5}[][HSK 7-9]
    \definition{v.}{incapaz de mudar algo | incapaz de lidar com}
  \end{Phonetics}
\end{Entry}

\begin{Entry}{顾全大局}{10,6,3,7}{⾴、⼊、⼤、⼫}
  \begin{Phonetics}{顾全大局}{gu4quan2-da4ju2}[][HSK 7-9]
    \definition{expr.}{``Considere a situação geral.''; levar em conta os interesses do todo; considerar a situação como um todo; levar em consideração o panorama geral; trabalhar para o benefício de todos}
  \end{Phonetics}
\end{Entry}

\begin{Entry}{顾问}{10,6}{⾴、⾨}
  \begin{Phonetics}{顾问}{gu4wen4}[][HSK 5]
    \definition[个,位,名]{s.}{conselheiro; consultor; assessor; pessoas com conhecimento especializado ou experiência contratadas para prestar consultoria a organizações ou indivíduos}
  \end{Phonetics}
\end{Entry}

\begin{Entry}{顾客}{10,9}{⾴、⼧}
  \begin{Phonetics}{顾客}{gu4ke4}[][HSK 2]
    \definition[个,位,名,些]{s.}{cliente; comprador; consumidor; paciente}
  \end{Phonetics}
\end{Entry}

\begin{Entry}{顾虑}{10,10}{⾴、⾌}
  \begin{Phonetics}{顾虑}{gu4lv4}[][HSK 7-9]
    \definition[丝,点]{s.}{preocupação; escrúpulo; receio; apreensão}
    \definition{v.}{estar apreensivo (sobre as consequências da própria ação)}
  \end{Phonetics}
\end{Entry}

%%%%%%%%%% 顿 %%%%%%%%%%
\subsection*{顿}

\begin{Entry}{顿}{10}{⾴}
  \begin{Phonetics}{顿}{dun4}[][HSK 3]
    \definition*{s.}{Sobrenome: Dun}
    \definition{adj.}{cansado; fatigado}
    \definition{adv.}{de repente; imediatamente; indica que o tempo é curto, equivalente a 立刻}
    \definition{clas.}{usado para refeições | usado para surras, repreensões, castigos físicos, etc.}
    \definition{s.}{um lugar para ficar; acomodação e alimentação}
    \definition{v.}{pausar; parar; fazer uma pausa | pausar na escrita para reforçar o início ou o fim de um traço; ao escrever com pincel, pressione o pincel com força e pare um pouco sobre o papel | tocar o chão (com a cabeça) | bater o pé); chutar o chão ou bater no chão com um objeto | resolver; arranjar | montar acampamento; ficar temporariamente; parar para se hospedar; acampar}
  \seealsoref{立刻}{li4ke4}
  \end{Phonetics}
\end{Entry}

\begin{Entry}{顿时}{10,7}{⾴、⽇}
  \begin{Phonetics}{顿时}{dun4shi2}[][HSK 7-9]
    \definition{adv.}{de repente; imediatamente; repentinamente; indica que uma ação ou comportamento ocorre sob certas circunstâncias ou imediatamente após algo; usado principalmente na escrita; usado apenas para descrever eventos passados}
  \end{Phonetics}
\end{Entry}

%%%%%%%%%% 颁 %%%%%%%%%%
\subsection*{颁}

\begin{Entry}{颁}{10}{⾴}
  \begin{Phonetics}{颁}{ban1}
    \definition{v.}{promulgar; emitir; enviar | conceder ou conferir}
  \end{Phonetics}
\end{Entry}

\begin{Entry}{颁发}{10,5}{⾴、⼜}
  \begin{Phonetics}{颁发}{ban1fa1}[][HSK 7-9]
    \definition{v.}{promulgar; emitir (comandos, instruções, regulamentos, etc.) a um superior | premiar (prêmio, medalha, certificado, etc.)}
  \end{Phonetics}
\end{Entry}

\begin{Entry}{颁布}{10,5}{⾴、⼱}
  \begin{Phonetics}{颁布}{ban1bu4}[][HSK 7-9]
    \definition{v.}{promulgar; emitir; publicar; anunciar (leis, regulamentos, etc.), com um escopo de uso mais restrito do que 公布}
  \seealsoref{公布}{gong1bu4}
  \end{Phonetics}
\end{Entry}

\begin{Entry}{颁奖}{10,9}{⾴、⼤}
  \begin{Phonetics}{颁奖}{ban1/jiang3}[][HSK 7-9]
    \definition{v.+compl.}{conceder prêmios, bônus, certificados, etc.; distribuir prêmios, bônus, certificados, etc.}
  \end{Phonetics}
\end{Entry}

%%%%%%%%%% 预 %%%%%%%%%%
\subsection*{预}

\begin{Entry}{预}{10}{⾴}
  \begin{Phonetics}{预}{yu4}
    \definition{adv.}{antecipadamente}
    \definition{v.}{avançar | preparar}
  \end{Phonetics}
\end{Entry}

\begin{Entry}{预习}{10,3}{⾴、⼄}
  \begin{Phonetics}{预习}{yu4xi2}[][HSK 3]
    \definition{v.}{pré-visualizar; preparar uma lição; estudar antecipadamente as matérias que serão abordadas nas aulas}
  \end{Phonetics}
\end{Entry}

\begin{Entry}{预见}{10,4}{⾴、⾒}
  \begin{Phonetics}{预见}{yu4jian4}
    \definition{s.}{previsão; intuição; vislumbre}
    \definition{v.}{prever}
  \end{Phonetics}
\end{Entry}

\begin{Entry}{预计}{10,4}{⾴、⾔}
  \begin{Phonetics}{预计}{yu4 ji4}[][HSK 3]
    \definition{v.}{estimar; calcular com antecedência}
  \end{Phonetics}
\end{Entry}

\begin{Entry}{预订}{10,4}{⾴、⾔}
  \begin{Phonetics}{预订}{yu4ding4}[][HSK 4]
    \definition{v.}{reservar; fazer uma reserva}
  \end{Phonetics}
\end{Entry}

\begin{Entry}{预付}{10,5}{⾴、⼈}
  \begin{Phonetics}{预付}{yu4fu4}
    \definition{s.}{pré-pago}
    \definition{v.}{pagar antecipadamente}
  \end{Phonetics}
\end{Entry}

\begin{Entry}{预约}{10,6}{⾴、⽷}
  \begin{Phonetics}{预约}{yu4 yue1}[][HSK 6]
    \definition[个]{s.}{reserva}
    \definition{v.}{reservar; agendar; marcar compromisso; marcar uma consulta}
  \end{Phonetics}
\end{Entry}

\begin{Entry}{预防}{10,6}{⾴、⾩}
  \begin{Phonetics}{预防}{yu4fang2}[][HSK 3]
    \definition{v.}{prevenir; proteger-se contra; tomar precauções contra; preparar-se com antecedência para evitar que algo ruim aconteça}
  \end{Phonetics}
\end{Entry}

\begin{Entry}{预判}{10,7}{⾴、⼑}
  \begin{Phonetics}{预判}{yu4pan4}
    \definition{v.}{prever | antecipar}
  \end{Phonetics}
\end{Entry}

\begin{Entry}{预报}{10,7}{⾴、⼿}
  \begin{Phonetics}{预报}{yu4bao4}[][HSK 3]
    \definition[个,项]{s.}{boletim meteorológico; previsões meteorológicas antecipadas}
    \definition{v.}{prever (o tempo); relatar antes que algo aconteça, usado principalmente em relação ao clima, astronomia, desastres naturais, etc.}
  \end{Phonetics}
\end{Entry}

\begin{Entry}{预备}{10,8}{⾴、⼡}
  \begin{Phonetics}{预备}{yu4 bei4}[][HSK 5]
    \definition{v.}{preparar-se; ficar pronto}
  \end{Phonetics}
\end{Entry}

\begin{Entry}{预定}{10,8}{⾴、⼧}
  \begin{Phonetics}{预定}{yu4ding4}
    \definition{v.}{agendar com antecedência}
  \end{Phonetics}
\end{Entry}

\begin{Entry}{预购}{10,8}{⾴、⾙}
  \begin{Phonetics}{预购}{yu4gou4}
    \definition{s.}{compra antecipada}
    \definition{v.}{comprar antecipadamente}
  \end{Phonetics}
\end{Entry}

\begin{Entry}{预测}{10,9}{⾴、⽔}
  \begin{Phonetics}{预测}{yu4 ce4}[][HSK 4]
    \definition{v.}{prever; prognosticar; predizer}
  \end{Phonetics}
\end{Entry}

\begin{Entry}{预祝}{10,9}{⾴、⽰}
  \begin{Phonetics}{预祝}{yu4zhu4}
    \definition{v.}{parabenizar de antemão | oferecer os melhores votos para}
  \end{Phonetics}
\end{Entry}

\begin{Entry}{预览}{10,9}{⾴、⾒}
  \begin{Phonetics}{预览}{yu4lan3}
    \definition{s.}{visualização}
    \definition{v.}{visualizar}
  \end{Phonetics}
\end{Entry}

\begin{Entry}{预留}{10,10}{⾴、⽥}
  \begin{Phonetics}{预留}{yu4liu2}
    \definition{v.}{separar | reservar}
  \end{Phonetics}
\end{Entry}

\begin{Entry}{预配}{10,10}{⾴、⾣}
  \begin{Phonetics}{预配}{yu4pei4}
    \definition{s.}{pré-alocado | pré-cabeado}
    \definition{v.}{pré-alocar | pré-cabear}
  \end{Phonetics}
\end{Entry}

\begin{Entry}{预谋}{10,11}{⾴、⾔}
  \begin{Phonetics}{预谋}{yu4mou2}
    \definition{adj.}{premeditado}
    \definition{v.}{planejar algo com antecedência (especialmente um crime)}
  \end{Phonetics}
\end{Entry}

\begin{Entry}{预提}{10,12}{⾴、⼿}
  \begin{Phonetics}{预提}{yu4ti2}
    \definition{s.}{retenção}
    \definition{v.}{reter (imposto)}
  \end{Phonetics}
\end{Entry}

\begin{Entry}{预期}{10,12}{⾴、⽉}
  \begin{Phonetics}{预期}{yu4qi1}[][HSK 5]
    \definition{v.}{esperar; antecipar; imaginar; antecipar com expectativa}
  \end{Phonetics}
\end{Entry}

\begin{Entry}{预感}{10,13}{⾴、⼼}
  \begin{Phonetics}{预感}{yu4gan3}
    \definition{s.}{premonição}
    \definition{v.}{ter uma premonição}
  \end{Phonetics}
\end{Entry}

\begin{Entry}{预警}{10,19}{⾴、⾔}
  \begin{Phonetics}{预警}{yu4jing3}
    \definition{s.}{aviso | aviso antecipado}
  \end{Phonetics}
\end{Entry}

%%%%%%%%%% 领 %%%%%%%%%%
\subsection*{领}

\begin{Entry}{领}{11}{⾴}
  \begin{Phonetics}{领}{ling3}[][HSK 3]
    \definition{clas.}{usado para roupas, mantos, esteiras, tapetes, telas, etc.}
    \definition{s.}{pescoço; gargalo | gola; colarinho; faixa de pescoço | esboço; ponto principal; essência}
    \definition{v.}{conduzir; guiar; orientar | possuir; ser o possuidor de; ter jurisdição sobre | obter; conseguir; receber (o que foi distribuído) | aceitar; tomar |entender; compreender (o significado)}
  \end{Phonetics}
\end{Entry}

\begin{Entry}{领土}{11,3}{⾴、⼟}
  \begin{Phonetics}{领土}{ling3tu3}[][HSK 7-9]
    \definition[寸,片,块]{s.}{território | domínio}
  \end{Phonetics}
\end{Entry}

\begin{Entry}{领队}{11,4}{⾴、⾩}
  \begin{Phonetics}{领队}{ling3dui4}[][HSK 7-9]
    \definition{s.}{o líder de um grupo, equipe esportiva, etc.}
    \definition{v.}{liderar um grupo; liderar a equipe}
  \end{Phonetics}
\end{Entry}

\begin{Entry}{领会}{11,6}{⾴、⼈}
  \begin{Phonetics}{领会}{ling3hui4}[][HSK 7-9]
    \definition{v.}{compreender; entender; apreciar as coisas e obter compreensão}
  \end{Phonetics}
\end{Entry}

\begin{Entry}{领先}{11,6}{⾴、⼉}
  \begin{Phonetics}{领先}{ling3xian1}[][HSK 3]
    \definition{v.}{liderar; assumir a liderança; estar na liderança; (velocidade, desempenho, etc.) superar pessoas ou coisas semelhantes, estar na vanguarda}
  \end{Phonetics}
\end{Entry}

\begin{Entry}{领军}{11,6}{⾴、⼍}
  \begin{Phonetics}{领军}{ling3jun1}[][HSK 7-9]
    \definition{v.}{comandar um exército; liderar tropas | Figurativo: liderar; desempenhar um papel de liderança}
  \end{Phonetics}
\end{Entry}

\begin{Entry}{领导}{11,6}{⾴、⼨}
  \begin{Phonetics}{领导}{ling3dao3}[][HSK 3]
    \definition[个,位,名,些]{s.}{líder; liderança; pessoa que ocupa uma posição de liderança}
    \definition{v.}{liderar; exercer liderança; (elogio) liderar, gerenciar outras pessoas;  trabalhar com outras pessoas ou avançar em direção a um objetivo}
  \end{Phonetics}
\end{Entry}

\begin{Entry}{领事}{11,8}{⾴、⼅}
  \begin{Phonetics}{领事}{ling3shi4}[][HSK 7-9]
    \definition[位]{s.}{cônsul}
  \end{Phonetics}
\end{Entry}

\begin{Entry}{领事馆}{11,8,11}{⾴、⼅、⾷}
  \begin{Phonetics}{领事馆}{ling3shi4guan3}[][HSK 7-9]
    \definition[个]{s.}{consulado; um escritório de representação consular de um governo em uma cidade ou região de outro país}
  \end{Phonetics}
\end{Entry}

\begin{Entry}{领取}{11,8}{⾴、⼜}
  \begin{Phonetics}{领取}{ling3 qu3}[][HSK 6]
    \definition{v.}{sacar; receber; obter; receber o que lhe é enviado}
  \end{Phonetics}
\end{Entry}

\begin{Entry}{领养}{11,9}{⾴、⼋}
  \begin{Phonetics}{领养}{ling3yang3}[][HSK 7-9]
    \definition{s.}{adoção}
    \definition{v.}{adotar (uma criança) | assumir a responsabilidade por}
  \end{Phonetics}
\end{Entry}

\begin{Entry}{领带}{11,9}{⾴、⼱}
  \begin{Phonetics}{领带}{ling3 dai4}[][HSK 5]
    \definition[条]{s.}{colar; gargantilha; gravata}
  \end{Phonetics}
\end{Entry}

\begin{Entry}{领悟}{11,10}{⾴、⼼}
  \begin{Phonetics}{领悟}{ling3wu4}[][HSK 7-9]
    \definition{v.}{compreender; entender}
  \end{Phonetics}
\end{Entry}

\begin{Entry}{领袖}{11,10}{⾴、⾐}
  \begin{Phonetics}{领袖}{ling3xiu4}[][HSK 6]
    \definition[个,位,名]{s.}{líder de estados, grupos políticos, organizações de massa, etc.}
  \end{Phonetics}
\end{Entry}

\begin{Entry}{领域}{11,11}{⾴、⼟}
  \begin{Phonetics}{领域}{ling3yu4}[][HSK 7-9]
    \definition[块,片,个]{s.}{território; domínio; uma região na qual um país exerce sua soberania | campo; esfera; domínio; reino; o âmbito do pensamento acadêmico ou das atividades sociais}
  \end{Phonetics}
\end{Entry}

\begin{Entry}{领情}{11,11}{⾴、⼼}
  \begin{Phonetics}{领情}{ling3/qing2}
    \definition{v.+compl.}{sentir-se grato a alguém}
  \end{Phonetics}
\end{Entry}

\begin{Entry}{领略}{11,11}{⾴、⽥}
  \begin{Phonetics}{领略}{ling3lve4}[][HSK 7-9]
    \definition{v.}{perceber; apreciar; ter um gostinho de; entender as circunstâncias das coisas e então reconhecer seu significado ou discernir seu sabor}
  \end{Phonetics}
\end{Entry}

%%%%%%%%%% 颈 %%%%%%%%%%
\subsection*{颈}

\begin{Entry}{颈}{11}{⾴}
  \begin{Phonetics}{颈}{geng3}
    \definition{s.}{nuca}
  \end{Phonetics}
  \begin{Phonetics}{颈}{jing3}
    \definition{s.}{pescoço}
  \end{Phonetics}
\end{Entry}

\begin{Entry}{颈部}{11,10}{⾴、⾢}
  \begin{Phonetics}{颈部}{jing3bu4}[][HSK 7-9]
    \definition{s.}{pescoço}
  \end{Phonetics}
\end{Entry}

%%%%%%%%%% 颐 %%%%%%%%%%
\subsection*{颐}

\begin{Entry}{颐}{13}{⾴}
  \begin{Phonetics}{颐}{yi2}
    \definition{s.}{bochecha}
    \definition{v.}{manter-se em forma; cuidar de si mesmo}
  \end{Phonetics}
\end{Entry}

\begin{Entry}{颐和园}{13,8,7}{⾴、⼝、⼞}
  \begin{Phonetics}{颐和园}{yi2he2yuan2}
    \definition*{s.}{Palácio de Verão}
  \end{Phonetics}
\end{Entry}

%%%%%%%%%% 频 %%%%%%%%%%
\subsection*{频}

\begin{Entry}{频}{13}{⾴}
  \begin{Phonetics}{频}{pin2}
    \definition*{s.}{Sobrenome: Pin}
    \definition{adj.}{frequente}
    \definition{adv.}{frequentemente; repetidamente}
    \definition{s.}{Física: frequência; o número de vezes que um objeto vibra por segundo}
  \end{Phonetics}
\end{Entry}

\begin{Entry}{频道}{13,12}{⾴、⾡}
  \begin{Phonetics}{频道}{pin2dao4}[][HSK 5]
    \definition[个]{s.}{canal; canal de frequência; televisão e rádio, os sinais de som e imagem ocupam um determinado canal de frequência}
  \end{Phonetics}
\end{Entry}

\begin{Entry}{频繁}{13,17}{⾴、⽷}
  \begin{Phonetics}{频繁}{pin2fan2}[][HSK 5]
    \definition{adj.}{frequentemente}
    \definition{adj.}{frequente}
  \end{Phonetics}
\end{Entry}

%%%%%%%%%% 颗 %%%%%%%%%%
\subsection*{颗}

\begin{Entry}{颗}{14}{⾴}
  \begin{Phonetics}{颗}{ke1}[][HSK 5]
    \definition{clas.}{usado para grãos, pérolas, dentes, corações, satelites, pequenas esferas, etc.}
    \definition{s.}{grão; partícula; pequenas coisas redondas}
  \end{Phonetics}
\end{Entry}

%%%%%%%%%% 题 %%%%%%%%%%
\subsection*{题}

\begin{Entry}{题}{15}{⾴}
  \begin{Phonetics}{题}{ti2}[][HSK 2]
    \definition*{s.}{Sobrenome: Ti}
    \definition[个,道]{s.}{tópico; título; assunto; problema; frases que indicam o conteúdo de poemas ou discursos | questão; questões que devem ser respondidas durante os exercícios ou exames | antigamente, referia-se à testa}
    \definition{v.}{inscrever; escrever; assinar}
  \end{Phonetics}
\end{Entry}

\begin{Entry}{题目}{15,5}{⾴、⽬}
  \begin{Phonetics}{题目}{ti2mu4}[][HSK 3]
    \definition[个,道]{s.}{título; assunto; tópico; o título de um poema ou discurso | quebra-cabeça; problema de exercício; questões a serem respondidas em exercícios ou provas}
  \end{Phonetics}
\end{Entry}

\begin{Entry}{题材}{15,7}{⾴、⽊}
  \begin{Phonetics}{题材}{ti2cai2}[][HSK 5]
    \definition{s.}{tema; assunto; material que compõe as obras literárias e artísticas, ou seja, os eventos ou fenômenos da vida descritos concretamente nas obras}
  \end{Phonetics}
\end{Entry}

%%%%%%%%%% 颜 %%%%%%%%%%
\subsection*{颜}

\begin{Entry}{颜}{15}{⾴}
  \begin{Phonetics}{颜}{yan2}
    \definition*{s.}{Sobrenome: Yan}
    \definition{s.}{rosto; semblante; expressão facial | rosto; prestígio; dignidade | cor}
  \end{Phonetics}
\end{Entry}

\begin{Entry}{颜色}{15,6}{⾴、⾊}
  \begin{Phonetics}{颜色}{yan2 se4}[][HSK 2]
    \definition[个,种]{s.}{cor; a sensação visual de um objeto é uma impressão diferente produzida pelas diferentes quantidades de luz absorvidas e refletidas pelo objeto | tez; semblante; aparência; geralmente se refere à aparência de uma garota | olhar severo no rosto como um aviso; um olhar ou ação que faz os outros parecerem particularmente ferozes | a expressão mostrada no rosto}
  \end{Phonetics}
\end{Entry}

%%%%%%%%%% 额 %%%%%%%%%%
\subsection*{额}

\begin{Entry}{额}{15}{⾴}
  \begin{Phonetics}{额}{e2}
    \definition*{s.}{Sobrenome: E}
    \definition[块]{s.}{testa; a área abaixo do cabelo e acima das sobrancelhas em humanos; a área aproximadamente equivalente na cabeça de alguns animais | uma tábua horizontal; placa horizontal inscrita; uma placa pendurada no lintel de uma porta ou na parede | um número específico (ou quantidade); limite superior de número; número limitado | a parte superior de algo}
  \end{Phonetics}
\end{Entry}

\begin{Entry}{额外}{15,5}{⾴、⼣}
  \begin{Phonetics}{额外}{e2wai4}[][HSK 7-9]
    \definition{adj.}{extra; adicional; excede a quantidade ou intervalo prescrito}
  \end{Phonetics}
\end{Entry}

%%%%%%%%%% 颠 %%%%%%%%%%
\subsection*{颠}

\begin{Entry}{颠}{16}{⾴}
  \begin{Phonetics}{颠}{dian1}
    \definition*{s.}{Sobrenome: Dian}
    \definition{adj.}{mentalmente perturbado; insano; o mesmo que 癫}
    \definition{s.}{coroa (da cabeça) | topo; cume}
    \definition{v.}{sacudir; bater | cair; virar; tombar | Dialeto: correr; ir embora}
  \seealsoref{癫}{dian1}
  \end{Phonetics}
\end{Entry}

\begin{Entry}{颠倒}{16,10}{⾴、⼈}
  \begin{Phonetics}{颠倒}{dian1dao3}[][HSK 7-9]
    \definition{adj.}{confuso; desordenado}
    \definition{v.}{inverter; reverter; virar de cabeça para baixo}
  \end{Phonetics}
\end{Entry}

\begin{Entry}{颠覆}{16,18}{⾴、⾑}
  \begin{Phonetics}{颠覆}{dian1fu4}[][HSK 7-9]
    \definition{v.}{derrubar; subverter; virar; tombar | tombar; derrubar (um regime legítimo) por conspiração}
  \end{Phonetics}
\end{Entry}

%%%%%%%%%% 颤 %%%%%%%%%%
\subsection*{颤}

\begin{Entry}{颤}{19}{⾴}
  \begin{Phonetics}{颤}{chan4}
    \definition{v.}{tremer; estremecer | vibrar; tremer; sacudir}
  \end{Phonetics}
\end{Entry}

\begin{Entry}{颤抖}{19,7}{⾴、⼿}
  \begin{Phonetics}{颤抖}{chan4dou3}[][HSK 7-9]
    \definition{v.}{tremer; estremecer; tremular; tiritar}
  \end{Phonetics}
\end{Entry}

%%%%% EOF %%%%%

