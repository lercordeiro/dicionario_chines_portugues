%%%
%%% Radical "⼺"
%%%

\section*{Radical 59: ``⼺''}\addcontentsline{toc}{section}{Radical 59: ⼺}

\begin{entry}{形式}{7,6}{⼺、⼷}
  \begin{phonetics}{形式}{xing2shi4}[][HSK 3]
    \definition[种,个]{s.}{forma; formato; modalidade | aparência, estrutura ou estado de algo}
  \end{phonetics}
\end{entry}

\begin{entry}{形成}{7,6}{⼺、⼽}
  \begin{phonetics}{形成}{xing2cheng2}[][HSK 3]
    \definition{v.}{moldar; formar; tomar forma | tornar-se algo ou algo através do desenvolvimento e da mudança}
  \end{phonetics}
\end{entry}

\begin{entry}{形而上学}{7,6,3,8}{⼺、⽽、⼀、⼦}
  \begin{phonetics}{形而上学}{xing2'er2shang4xue2}
    \definition{s.}{metafísica}
  \end{phonetics}
\end{entry}

\begin{entry}{形状}{7,7}{⼺、⽝}
  \begin{phonetics}{形状}{xing2zhuang4}[][HSK 3]
    \definition[个]{s.}{forma; aparência | a aparência de um objeto ou figura formada pela combinação de superfícies ou linhas externas}
  \end{phonetics}
\end{entry}

\begin{entry}{形势}{7,8}{⼺、⼒}
  \begin{phonetics}{形势}{xing2shi4}[][HSK 4]
    \definition[个]{s.}{terreno; características topográficas; situação geográfica, principalmente de uma perspectiva militar | situação; circunstâncias; a situação geral, a tendência de como as coisas estão se desenvolvendo e mudando | geralmente não é usado em situações pessoais}
  \end{phonetics}
\end{entry}

\begin{entry}{形态}{7,8}{⼺、⼼}
  \begin{phonetics}{形态}{xing2tai4}[][HSK 5]
    \definition{s.}{forma; forma como as coisas se apresentam | forma; padrão; postura | morfologia; forma; (gramática) refere-se às formas internas de mudança das palavras, incluindo a formação de palavras e as mudanças morfológicas}
  \end{phonetics}
\end{entry}

\begin{entry}{形容}{7,10}{⼺、⼧}
  \begin{phonetics}{形容}{xing2rong2}[][HSK 4]
    \definition{s.}{aparência; semblante}
    \definition{v.}{descrever}
  \end{phonetics}
\end{entry}

\begin{entry}{形象}{7,11}{⼺、⾗}
  \begin{phonetics}{形象}{xing2xiang4}[][HSK 3]
    \definition{adj.}{vívido}
    \definition[个]{s.}{imagem; forma; figura | uma forma ou gesto específico que pode despertar os pensamentos ou emoções das pessoas | imagem literária; imagem artística | pessoas ou coisas com características diferentes criadas na literatura, no cinema e em outras artes}
  \end{phonetics}
\end{entry}

\begin{entry}{彩色}{11,6}{⼺、⾊}
  \begin{phonetics}{彩色}{cai3 se4}[][HSK 3]
    \definition[个,种]{s.}{multicolorido; cor; várias cores}
  \end{phonetics}
\end{entry}

\begin{entry}{彩虹}{11,9}{⼺、⾍}
  \begin{phonetics}{彩虹}{cai3hong2}
    \definition[道]{s.}{arco-íris}
  \end{phonetics}
\end{entry}

\begin{entry}{彩票}{11,11}{⼺、⽰}
  \begin{phonetics}{彩票}{cai3piao4}[][HSK 5]
    \definition[张]{s.}{bilhete de loteria}
  \end{phonetics}
\end{entry}

\begin{entry}{影子}{15,3}{⼺、⼦}
  \begin{phonetics}{影子}{ying3zi5}[][HSK 4]
    \definition[个,片]{s.}{sombra; imagem projetada por um objeto, etc., que bloqueia a luz | reflexão; reflexo; imagem de um objeto, etc., conforme aparece em um refletor, como um espelho, uma superfície de água, etc. | sinal; vestígio; vaga impressão}
  \end{phonetics}
\end{entry}

\begin{entry}{影片}{15,4}{⼺、⽚}
  \begin{phonetics}{影片}{ying3 pian4}[][HSK 2]
    \definition[部,盘,盒,卷]{s.}{filme; imagem | filme; película usada para reproduzir filmes}
  \end{phonetics}
\end{entry}

\begin{entry}{影视}{15,8}{⼺、⾒}
  \begin{phonetics}{影视}{ying3 shi4}[][HSK 3]
    \definition{s.}{cinema e televisão combinados}
  \end{phonetics}
\end{entry}

\begin{entry}{影响}{15,9}{⼺、⼝}
  \begin{phonetics}{影响}{ying3xiang3}[][HSK 2]
    \definition{s.}{efeito; influência; efeitos sobre pessoas ou coisas}
    \definition{v.}{afetar; influenciar; influência sobre os pensamentos ou ações dos outros}
  \end{phonetics}
\end{entry}

\begin{entry}{影像}{15,13}{⼺、⼈}
  \begin{phonetics}{影像}{ying3xiang4}
    \definition{s.}{imagem}
  \end{phonetics}
\end{entry}

%%%%% EOF %%%%%

