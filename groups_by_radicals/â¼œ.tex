%%%
%%% Radical "⼜"
%%%

\section*{Radical 29: ``⼜''}\addcontentsline{toc}{section}{Radical 29: ⼜}

\begin{Entry}{又}{2}{⼜}[Kangxi 29]
  \begin{Phonetics}{又}{you4}[][HSK 2]
    \definition{adv.}{indica repetição ou continuação | indica que várias situações ou propriedades existem simultaneamente | indica um nível mais profundo de significado | indica adicionar zero a números inteiros | indica duas coisas contraditórias | indica um ponto de virada, significando 可是 | usado em frases negativas ou perguntas retóricas para fortalecer o tom | além disso; indica informações adicionais ou suplementares}
  \seealsoref{可是}{ke3shi4}
  \end{Phonetics}
\end{Entry}

\begin{Entry}{又一次}{2,1,6}{⼜、⼀、⽋}
  \begin{Phonetics}{又一次}{you4yi2ci4}
    \definition{adv.}{outra vez | mais uma vez | de novo}
  \end{Phonetics}
\end{Entry}

\begin{Entry}{又……又……}{2,2}{⼜、⼜}
  \begin{Phonetics}{又……又……}{you4 you4}
    \definition{conj.}{\dots e\dots; tanto\dots como\dots; as duas palavras usadas depois de 又 não devem ter nenhuma conotação de contraste, ambas devem ser positivas ou negativas}[这件毛衣挺不错的,\underline{又}便宜\underline{又}漂亮。===Este suéter é muito bom, barato e bonito.]
  \end{Phonetics}
\end{Entry}

\begin{Entry}{又及}{2,3}{⼜、⼃}
  \begin{Phonetics}{又及}{you4ji2}
    \definition{s.}{P.S., \emph{postscript}}
  \end{Phonetics}
\end{Entry}

\begin{Entry}{又名}{2,6}{⼜、⼝}
  \begin{Phonetics}{又名}{you4ming2}
    \definition{s.}{também conhecido como | nome alternativo}
  \end{Phonetics}
\end{Entry}

\begin{Entry}{又称}{2,10}{⼜、⽲}
  \begin{Phonetics}{又称}{you4cheng1}
    \definition{s.}{também conhecido como}
  \end{Phonetics}
\end{Entry}

\begin{Entry}{叉}{3}{⼜}
  \begin{Phonetics}{叉}{cha1}[][HSK 5]
    \definition{s.}{garfo; forquilha | símbolo de cruz, ``×''}
    \definition{v.}{trabalhar com um garfo; garfar; pegar coisas com um garfo}
  \end{Phonetics}
  \begin{Phonetics}{叉}{cha2}
    \definition{v.}{bloquear; emperrar; congestionar}
  \end{Phonetics}
  \begin{Phonetics}{叉}{cha3}
    \definition{v.}{separar de modo a formar uma bifurcação; bifurcar}
  \end{Phonetics}
\end{Entry}

\begin{Entry}{叉子}{3,3}{⼜、⼦}
  \begin{Phonetics}{叉子}{cha1zi5}[][HSK 5]
    \definition[把,个]{s.}{garfo; ferramenta com mais de duas pontas em uma extremidade | tridente; forquilha; ferramentas de agricultura antigas}
  \end{Phonetics}
\end{Entry}

\begin{Entry}{友}{4}{⼜}
  \begin{Phonetics}{友}{you3}
    \definition{adj.}{amigável; bom relacionamento; próximo | de relações amigáveis}
    \definition{s.}{amigo; pessoas intimamente relacionadas}
  \end{Phonetics}
\end{Entry}

\begin{Entry}{友好}{4,6}{⼜、⼥}
  \begin{Phonetics}{友好}{you3hao3}[][HSK 2]
    \definition{adj.}{amigável; amistoso; muito próximo, relacionamento muito bom; como bons amigos}
    \definition{s.}{amigo próximo, íntimo; em ocasiões formais referem-se a bons amigos}
  \end{Phonetics}
\end{Entry}

\begin{Entry}{友谊}{4,10}{⼜、⾔}
  \begin{Phonetics}{友谊}{you3yi4}[][HSK 5]
    \definition[段,份]{s.}{amizade; amizade entre amigos}
  \end{Phonetics}
\end{Entry}

\begin{Entry}{双}{4}{⼜}
  \begin{Phonetics}{双}{shuang1}[][HSK 3]
    \definition*{s.}{Sobrenome Shuang}
    \definition{adj.}{dois; gêmeo; par; dual; em oposição a 单 | números pares | duplo; dobro}
    \definition{clas.}{usado para certos membros, órgãos ou coisas pareadas que são bilateralmente simétricas, por exemplo, sapatos, meias, pauzinhos, etc.}
  \seealsoref{单}{dan1}
  \end{Phonetics}
\end{Entry}

\begin{Entry}{双手}{4,4}{⼜、⼿}
  \begin{Phonetics}{双手}{shuang1 shou3}[][HSK 5]
    \definition{s.}{com as duas mãos; ambas as mãos; par de mãos}
  \end{Phonetics}
\end{Entry}

\begin{Entry}{双方}{4,4}{⼜、⽅}
  \begin{Phonetics}{双方}{shuang1fang1}[][HSK 3]
    \definition{s.}{ambos os lados; as duas partes; duas pessoas ou dois grupos frente a frente em um determinado relacionamento ou situação}
  \end{Phonetics}
\end{Entry}

\begin{Entry}{双方同意}{4,4,6,13}{⼜、⽅、⼝、⼼}
  \begin{Phonetics}{双方同意}{shuang1fang1tong2yi4}
    \definition{s.}{acordo bilateral}
  \end{Phonetics}
\end{Entry}

\begin{Entry}{双打}{4,5}{⼜、⼿}
  \begin{Phonetics}{双打}{shuang1 da3}[][HSK 6]
    \definition[场,局,次]{s.}{duplas (em esportes)}
  \end{Phonetics}
\end{Entry}

\begin{Entry}{双层床}{4,7,7}{⼜、⼫、⼴}
  \begin{Phonetics}{双层床}{shuang1ceng2chuang2}
    \definition{s.}{beliche}
  \end{Phonetics}
\end{Entry}

\begin{Entry}{反}{4}{⼜}
  \begin{Phonetics}{反}{fan3}[][HSK 4]
    \definition{adj.}{oposto; contrário; invertido}
    \definition{adv.}{pelo contrário; inversamente}
    \definition{v.}{inverter o lado; de cabeça para baixo; na direção oposta | virar; converter | retornar | opor-se; combater; voltar-se contra | rebelar-se; revoltar-se | inferir; deduzir; raciocinar por analogia}
  \end{Phonetics}
\end{Entry}

\begin{Entry}{反对}{4,5}{⼜、⼨}
  \begin{Phonetics}{反对}{fan3dui4}[][HSK 3]
    \definition{v.}{lutar; opor-se; objetar a; ser contra; discordar}
  \end{Phonetics}
\end{Entry}

\begin{Entry}{反对派}{4,5,9}{⼜、⼨、⽔}
  \begin{Phonetics}{反对派}{fan3dui4pai4}
    \definition{s.}{facção de oposição}
  \end{Phonetics}
\end{Entry}

\begin{Entry}{反对党}{4,5,10}{⼜、⼨、⼉}
  \begin{Phonetics}{反对党}{fan3dui4dang3}
    \definition{s.}{partido de oposição}
  \end{Phonetics}
\end{Entry}

\begin{Entry}{反对票}{4,5,11}{⼜、⼨、⽰}
  \begin{Phonetics}{反对票}{fan3dui4piao4}
    \definition{s.}{voto dissidente}
  \end{Phonetics}
\end{Entry}

\begin{Entry}{反正}{4,5}{⼜、⽌}
  \begin{Phonetics}{反正}{fan3zheng4}[][HSK 3]
    \definition{adv.}{de qualquer forma; de qualquer maneira; embora as circunstâncias sejam diferentes, o resultado é o mesmo | tudo igual; em qualquer caso; tom de voz que expressa afirmação categórica}
  \end{Phonetics}
\end{Entry}

\begin{Entry}{反而}{4,6}{⼜、⽽}
  \begin{Phonetics}{反而}{fan3'er2}[][HSK 4]
    \definition{adv.}{em vez disso; ao contrário de; contrário ao significado da frase anterior ou inesperado, desempenha o papel de uma reviravolta em uma frase}
  \end{Phonetics}
\end{Entry}

\begin{Entry}{反问}{4,6}{⼜、⾨}
  \begin{Phonetics}{反问}{fan3wen4}[][HSK 6]
    \definition{v.}{fazer uma pergunta em resposta; responder a uma pergunta com outra pergunta | fazer uma pergunta retórica (uma pergunta com significado negativo)}
  \end{Phonetics}
\end{Entry}

\begin{Entry}{反应}{4,7}{⼜、⼴}
  \begin{Phonetics}{反应}{fan3ying4}[][HSK 3]
    \definition[个]{s.}{reação; resposta; opiniões, atitudes ou ações causadas pelo acontecimento}
    \definition{v.}{reagir; responder; atividade correspondente causada pela estimulação do organismo}
  \end{Phonetics}
\end{Entry}

\begin{Entry}{反抗}{4,7}{⼜、⼿}
  \begin{Phonetics}{反抗}{fan3kang4}[][HSK 6]
    \definition{s.}{resistência}
    \definition{v.}{revoltar-se; resistir; opor-se com ação}
  \end{Phonetics}
\end{Entry}

\begin{Entry}{反响}{4,9}{⼜、⼝}
  \begin{Phonetics}{反响}{fan3 xiang3}[][HSK 6]
    \definition{s.}{eco; reverberação; repercusão}
  \end{Phonetics}
\end{Entry}

\begin{Entry}{反复}{4,9}{⼜、⼢}
  \begin{Phonetics}{反复}{fan3fu4}[][HSK 3]
    \definition{adv.}{repetidamente; de ​​novo e de novo; várias vezes}
    \definition{s.}{reversão; recaída; a situação anterior se repetiu}
    \definition{v.}{recuar; cortar e mudar; virar de cabeça para baixo; arrepender-se; aparecer várias vezes (usado principalmente em situações ruins)}
  \end{Phonetics}
\end{Entry}

\begin{Entry}{反映}{4,9}{⼜、⽇}
  \begin{Phonetics}{反映}{fan3ying4}[][HSK 4]
    \definition{s.}{reflexão; opiniões sobre pessoas ou situações}
    \definition{v.}{refletir; espelhar; figurativamente, trazer à tona a essência de uma questão objetiva (expressão idiomática); expressar a essência de algo objetivamente | relatar; tornar conhecido; informar às autoridades superiores | refletir; espelhar; a imagem de um objeto aparece invertida em outro objeto}
  \end{Phonetics}
\end{Entry}

\begin{Entry}{反省}{4,9}{⼜、⽬}
  \begin{Phonetics}{反省}{fan3xing3}
    \definition{v.}{examinar a consciência | questionar-se | refletir sobre si mesmo | sondar a alma}
  \end{Phonetics}
\end{Entry}

\begin{Entry}{反倒}{4,10}{⼜、⼈}
  \begin{Phonetics}{反倒}{fan3dao4}
    \definition{adv.}{em vez disso; pelo contrário}
    \definition{conj.}{em vez disso; pelo contrário; frequentemente acompanhadas por várias palavras que expressam negação}
  \end{Phonetics}
\end{Entry}

\begin{Entry}{发}{5}{⼜}
  \begin{Phonetics}{发}{fa1}[][HSK 2]
    \definition*{s.}{Sobrenome Fa}
    \definition{clas.}{bala, usada para munições e cartuchos}
    \definition{v.}{distribuir; enviar; entregar | emitir; disparar; lançar; descarregar | produzir; gerar; criar (dar origem a) | proferir; emitir; expressar | expandir; desenvolver | prosperar; prosperidade graças à aquisição de bens materiais | crescer ou expandir quando fermentado ou embebido | difundir; dispersar; espalhar | expor; descobrir; revelar | transformar-se; tornar-se; entrar em um determinado estado | demonstrar seus sentimentos; expressar (sentimentos) | sentir; ter um sentimento | começar; estabelecer | fazer com que se faça; iniciar um empreendimento; começar a agir; provocar uma ação}
  \end{Phonetics}
  \begin{Phonetics}{发}{fa4}
    \definition*{s.}{Sobrenome Fa}
    \definition[件]{s.}{cabelo}
  \end{Phonetics}
\end{Entry}

\begin{Entry}{发出}{5,5}{⼜、⼐}
  \begin{Phonetics}{发出}{fa1 chu1}[][HSK 3]
    \definition{v.}{fazer; produzir; deixar sair; ocorrer (som, dúvida, etc.) | emitir; anunciar; publicar; divulgar (ordens, instruções) | enviar (mercadorias, cartas, etc.); partir (veículos, etc.) | emitir; exalar (cheiro, calor, etc.)}
  \end{Phonetics}
\end{Entry}

\begin{Entry}{发布}{5,5}{⼜、⼱}
  \begin{Phonetics}{发布}{fa1bu4}[][HSK 5]
    \definition{v.}{emitir; publicar; liberar; anunciar; fazer ordens públicas, anúncios, notícias, etc.}
  \end{Phonetics}
\end{Entry}

\begin{Entry}{发生}{5,5}{⼜、⽣}
  \begin{Phonetics}{发生}{fa1sheng1}[][HSK 3]
    \definition{v.}{ocorrer; acontecer; tomar lugar; surgir algo que não existia antes}
  \end{Phonetics}
\end{Entry}

\begin{Entry}{发电}{5,5}{⼜、⽥}
  \begin{Phonetics}{发电}{fa1 dian4}[][HSK 6]
    \definition{s.}{geração de energia elétrica; produção de eletricidade; fornecimento de energia}
    \definition{v.}{gerar eletricidade (ou energia elétrica) | enviar um telegrama}
  \end{Phonetics}
\end{Entry}

\begin{Entry}{发动}{5,6}{⼜、⼒}
  \begin{Phonetics}{发动}{fa1dong4}[][HSK 3]
    \definition{v.}{iniciar; começar; lançar | chamar à ação; mobilizar; despertar | ligar o motor; dar a partida; dar o pontapé inicial (motor de combustão interna) | estimular; colocar em ação}
  \end{Phonetics}
\end{Entry}

\begin{Entry}{发动机}{5,6,6}{⼜、⼒、⽊}
  \begin{Phonetics}{发动机}{fa1dong4ji1}
    \definition[台]{s.}{motor}
  \end{Phonetics}
\end{Entry}

\begin{Entry}{发行}{5,6}{⼜、⾏}
  \begin{Phonetics}{发行}{fa1xing2}[][HSK 5]
    \definition{v.}{emitir; liberar; publicar; emitir ou vender de publicações recém-impressas, moeda, selos, etc.}
  \end{Phonetics}
\end{Entry}

\begin{Entry}{发达}{5,6}{⼜、⾡}
  \begin{Phonetics}{发达}{fa1da2}[][HSK 3]
    \definition{adj.}{desenvolvido; florescente; (coisas) Já estão bem desenvolvidas; (negócios) prosperam}
    \definition{v.}{desenvolver; promover; florescer; a pessoa tem um bom desempenho profissional e é muito bem-sucedida}
  \end{Phonetics}
\end{Entry}

\begin{Entry}{发抖}{5,7}{⼜、⼿}
  \begin{Phonetics}{发抖}{fa1dou3}
    \definition{v.}{tremer | sacudir | estremecer}
  \end{Phonetics}
\end{Entry}

\begin{Entry}{发言}{5,7}{⼜、⾔}
  \begin{Phonetics}{发言}{fa1yan2}[][HSK 3]
    \definition[个]{s.}{discurso; declaração; palestra; opiniões publicadas}
    \definition{v.+compl.}{falar; fazer uma declaração (discurso); expressar opinião (geralmente em reuniões)}
  \end{Phonetics}
\end{Entry}

\begin{Entry}{发言人}{5,7,2}{⼜、⾔、⼈}
  \begin{Phonetics}{发言人}{fa1 yan2 ren2}[][HSK 6]
    \definition{s.}{porta-voz}
  \end{Phonetics}
\end{Entry}

\begin{Entry}{发财}{5,7}{⼜、⾙}
  \begin{Phonetics}{发财}{fa1cai2}
    \definition{v.+compl.}{ficar rico | fazer fortuna}
  \end{Phonetics}
\end{Entry}

\begin{Entry}{发放}{5,8}{⼜、⽅}
  \begin{Phonetics}{发放}{fa1 fang4}[][HSK 6]
    \definition{v.}{conceder; estender; fornecer; (governo, organização) distribuir dinheiro ou suprimentos para os necessitados | emitir; enviar}
  \end{Phonetics}
\end{Entry}

\begin{Entry}{发明}{5,8}{⼜、⽇}
  \begin{Phonetics}{发明}{fa1ming2}[][HSK 3]
    \definition[个,项,种]{s.}{invenção; novos produtos ou métodos inventados}
    \definition{v.}{inventar; pesquisa que cria (novos produtos ou novos métodos) | expor; explicar; explicação criativa}
  \end{Phonetics}
\end{Entry}

\begin{Entry}{发明者}{5,8,8}{⼜、⽇、⽼}
  \begin{Phonetics}{发明者}{fa1ming2zhe3}
    \definition{s.}{inventor}
  \end{Phonetics}
\end{Entry}

\begin{Entry}{发泄}{5,8}{⼜、⽔}
  \begin{Phonetics}{发泄}{fa1xie4}
    \definition{v.}{soltar; abreviar; dar vazão a; desabafar emoções ou desejos}
  \end{Phonetics}
\end{Entry}

\begin{Entry}{发炎}{5,8}{⼜、⽕}
  \begin{Phonetics}{发炎}{fa1yan2}[][HSK 6]
    \definition{s.}{inflamação}
    \definition{v.}{irritar; inflamar; reação complexa de organismos a fatores patogênicos, como microrganismos, substâncias químicas e estímulos físicos; os sintomas sistêmicos incluem aumento da temperatura corporal, alterações na composição do sangue, vermelhidão local, inchaço, febre, dor, etc.}
  \end{Phonetics}
\end{Entry}

\begin{Entry}{发现}{5,8}{⼜、⾒}
  \begin{Phonetics}{发现}{fa1xian4}[][HSK 2]
    \definition[个,项]{s.}{descoberta; achado}
    \definition{v.}{encontrar; descobrir; detectar; identificar; através de pesquisa, exploração, etc., ver ou encontrar coisas ou leis que os antepassados não viram | descobrir; perceber; perceber; notar; estar ciente de}
  \end{Phonetics}
\end{Entry}

\begin{Entry}{发现者}{5,8,8}{⼜、⾒、⽼}
  \begin{Phonetics}{发现者}{fa1xian4 zhe3}
    \definition{s.}{descobridor}
  \end{Phonetics}
\end{Entry}

\begin{Entry}{发表}{5,8}{⼜、⾐}
  \begin{Phonetics}{发表}{fa1biao3}[][HSK 3]
    \definition{v.}{publicar; entregar; emitir; expressar; anunciar; expressar (opiniões) ou divulgar (assuntos) ao público, verbalmente ou por escrito | publicar em jornais (artigos, etc.)}
  \end{Phonetics}
\end{Entry}

\begin{Entry}{发型}{5,9}{⼜、⼟}
  \begin{Phonetics}{发型}{fa4xing2}
    \definition{s.}{penteado}
  \end{Phonetics}
\end{Entry}

\begin{Entry}{发怒}{5,9}{⼜、⼼}
  \begin{Phonetics}{发怒}{fa1 nu4}[][HSK 6]
    \definition{v.}{ficar com raiva; explodir; perder a paciência | entrar em fúria | entrar em fúria (paixão)}
  \end{Phonetics}
\end{Entry}

\begin{Entry}{发挥}{5,9}{⼜、⼿}
  \begin{Phonetics}{发挥}{fa1hui1}[][HSK 4]
    \definition{v.}{colocar em jogo; dar jogo a; dar espaço a; dar rédea solta a; revelar a natureza ou a capacidade interior | expressar; desenvolver (uma ideia, um tema, etc.); elaborar; fazer valer o ponto ou o motivo}
  \end{Phonetics}
\end{Entry}

\begin{Entry}{发觉}{5,9}{⼜、⾒}
  \begin{Phonetics}{发觉}{fa1jue2}[][HSK 5]
    \definition{v.}{vir a saber; estar ciente (de); perceber; tornar-se consciente | encontrar; detectar; perceber; descobrir}
  \end{Phonetics}
\end{Entry}

\begin{Entry}{发送}{5,9}{⼜、⾡}
  \begin{Phonetics}{发送}{fa1 song4}[][HSK 3]
    \definition{v.}{enviar; despachar | transmitir (rádio)}
  \end{Phonetics}
\end{Entry}

\begin{Entry}{发音}{5,9}{⼜、⾳}
  \begin{Phonetics}{发音}{fa1yin1}
    \definition{s.}{pronúncia}
    \definition{v.}{pronunciar}
  \end{Phonetics}
\end{Entry}

\begin{Entry}{发射}{5,10}{⼜、⼨}
  \begin{Phonetics}{发射}{fa1she4}[][HSK 5]
    \definition{v.}{subir; disparar; lançar; irradiar; projetar; descarregar; enviar algo (como uma bala, um projétil, um satélite, etc.) de um dispositivo em uma velocidade muito alta}
  \end{Phonetics}
\end{Entry}

\begin{Entry}{发展}{5,10}{⼜、⼫}
  \begin{Phonetics}{发展}{fa1zhan3}[][HSK 3]
    \definition{v.}{crescer; expandir; avançar; desenvolver; a mudança das coisas de pequeno para grande, de simples para complexo, de inferior para superior | recrutar; admitir expandir (organização, escala, etc.)}
  \end{Phonetics}
\end{Entry}

\begin{Entry}{发烧}{5,10}{⼜、⽕}
  \begin{Phonetics}{发烧}{fa1shao1}[][HSK 4]
    \definition{v.}{ter febre; a temperatura corporal normal de uma pessoa é de cerca de 37ºC; se exceder 37,5ºC, é febre}
  \end{Phonetics}
\end{Entry}

\begin{Entry}{发病}{5,10}{⼜、⽧}
  \begin{Phonetics}{发病}{fa1 bing4}[][HSK 6]
    \definition{v.}{(de uma doença) avanço | patogênese; morbidade | surto (de uma doença)}
  \end{Phonetics}
\end{Entry}

\begin{Entry}{发起}{5,10}{⼜、⾛}
  \begin{Phonetics}{发起}{fa1 qi3}[][HSK 6]
    \definition{s.}{iniciador; patrocinador}
    \definition{v.}{iniciar; patrocinar; começar; lançar}
  \end{Phonetics}
\end{Entry}

\begin{Entry}{发票}{5,11}{⼜、⽰}
  \begin{Phonetics}{发票}{fa1piao4}[][HSK 4]
    \definition[张,种]{s.}{conta; recibo; fatura; recibos emitidos por lojas ou outros escritórios de cobrança}
  \end{Phonetics}
\end{Entry}

\begin{Entry}{发愁}{5,13}{⼜、⼼}
  \begin{Phonetics}{发愁}{fa1chou2}
    \definition{v.+compl.}{preocupar-se | ficar ansioso | ficar triste}
  \end{Phonetics}
\end{Entry}

\begin{Entry}{发簪}{5,18}{⼜、⽵}
  \begin{Phonetics}{发簪}{fa4zan1}
    \definition{s.}{grampo de cabelo}
  \end{Phonetics}
\end{Entry}

\begin{Entry}{叔}{8}{⼜}
  \begin{Phonetics}{叔}{shu1}
    \definition*{s.}{Sobrenome Shu}
    \definition{s.}{irmão mais novo do pai; tio (por parte de pai)| irmão mais novo do marido | terceiro entre irmãos | tio | uma forma de tratamento para um homem um pouco mais jovem que o pai; tio | terceiro tio (de quatro irmãos) | primo mais novo da mãe}
  \end{Phonetics}
\end{Entry}

\begin{Entry}{叔叔}{8,8}{⼜、⼜}
  \begin{Phonetics}{叔叔}{shu1shu5}
    \definition[个,位,名]{s.}{tio; irmão mais novo do pai | tio, dirigindo-se a um homem da mesma geração que o pai e mais jovem em idade}
  \end{Phonetics}
\end{Entry}

\begin{Entry}{取}{8}{⼜}
  \begin{Phonetics}{取}{qu3}[][HSK 2]
    \definition{v.}{pegar; obter; buscar; pegar de um lugar; pegar nas mãos | visar; procurar; obter; provocar | adotar; assumir; escolher; selecionar}
  \end{Phonetics}
\end{Entry}

\begin{Entry}{取水}{8,4}{⼜、⽔}
  \begin{Phonetics}{取水}{qu3shui3}
    \definition{v.}{obter água (de um poço, etc.)}
  \end{Phonetics}
\end{Entry}

\begin{Entry}{取现}{8,8}{⼜、⾒}
  \begin{Phonetics}{取现}{qu3xian4}
    \definition{v.}{sacar dinheiro}
  \end{Phonetics}
\end{Entry}

\begin{Entry}{取胜}{8,9}{⼜、⾁}
  \begin{Phonetics}{取胜}{qu3sheng4}
    \definition{v.}{prevalecer sobre os oponentes | marcar uma vitória}
  \end{Phonetics}
\end{Entry}

\begin{Entry}{取悦}{8,10}{⼜、⼼}
  \begin{Phonetics}{取悦}{qu3yue4}
    \definition{v.}{tentar agradar}
  \end{Phonetics}
\end{Entry}

\begin{Entry}{取消}{8,10}{⼜、⽔}
  \begin{Phonetics}{取消}{qu3xiao1}[][HSK 3]
    \definition{v.}{cancelar; suspender; anular; abolir; revogar; rescindir; tornar o sistema original, regulamentos, qualificações, direitos, etc. inválidos}
  \end{Phonetics}
\end{Entry}

\begin{Entry}{取得}{8,11}{⼜、⼻}
  \begin{Phonetics}{取得}{qu3 de2}[][HSK 2]
    \definition{v.}{ganhar; adquirir; obter; ser o primeiro a conseguir}
  \end{Phonetics}
\end{Entry}

\begin{Entry}{取款}{8,12}{⼜、⽋}
  \begin{Phonetics}{取款}{qu3kuan3}[][HSK 6]
    \definition{v.}{sacar dinheiro (de um banco); retirar o dinheiro que você depositou (geralmente se refere a retirar dinheiro do banco)}
  \end{Phonetics}
\end{Entry}

\begin{Entry}{取款机}{8,12,6}{⼜、⽋、⽊}
  \begin{Phonetics}{取款机}{qu3 kuan3 ji1}[][HSK 6]
    \definition{s.}{ATM; caixa eletrônico; um caixa eletrônico é uma máquina que pode concluir automaticamente operações bancárias, como saques e consultas de saldo}
  \end{Phonetics}
\end{Entry}

\begin{Entry}{受}{8}{⼜}
  \begin{Phonetics}{受}{shou4}[][HSK 3]
    \definition{v.}{receber; aceitar | sofrer; ser submetido a | aguentar; suportar; tolerar | ser agradável}
  \end{Phonetics}
\end{Entry}

\begin{Entry}{受不了}{8,4,2}{⼜、⼀、⼅}
  \begin{Phonetics}{受不了}{shou4bu5liao3}[][HSK 4]
    \definition{v.}{ser insuportável; não poder suportar algo; não suportar algo}
  \end{Phonetics}
\end{Entry}

\begin{Entry}{受伤}{8,6}{⼜、⼈}
  \begin{Phonetics}{受伤}{shou4shang1}[][HSK 3]
    \definition{v.}{ser ferido; sofrer uma lesão}
  \end{Phonetics}
\end{Entry}

\begin{Entry}{受灾}{8,7}{⼜、⽕}
  \begin{Phonetics}{受灾}{shou4 zai1}[][HSK 5]
    \definition{v.}{ser atingido por um desastre natural (ou calamidade) | ser atingido por uma adversidade natural}
  \end{Phonetics}
\end{Entry}

\begin{Entry}{受到}{8,8}{⼜、⼑}
  \begin{Phonetics}{受到}{shou4dao4}[][HSK 2]
    \definition{v.}{receber; receber itens, mensagens, instruções, etc. fornecidos por outras pessoas}
  \end{Phonetics}
\end{Entry}

\begin{Entry}{受限}{8,8}{⼜、⾩}
  \begin{Phonetics}{受限}{shou4xian4}
    \definition{v.}{ser limitado | ser restrito | ser constrangido}
  \end{Phonetics}
\end{Entry}

\begin{Entry}{受得了}{8,11,2}{⼜、⼻、⼅}
  \begin{Phonetics}{受得了}{shou4de5liao3}
    \definition{v.}{suportar | aguentar}
  \end{Phonetics}
\end{Entry}

\begin{Entry}{变}{8}{⼜}
  \begin{Phonetics}{变}{bian4}[][HSK 2]
    \definition{adj.}{alterado; mutável; que pode mudar; que está mudando ou já mudou}
    \definition{s.}{uma reviravolta inesperada nos acontecimentos; mudanças significativas repentinas}
    \definition{v.}{mudar; tornar-se diferente; fazer mudanças | tornar-se; transformar-se; natureza, estado ou situação diferentes dos originais | alterar; mudar; transformar}
  \end{Phonetics}
\end{Entry}

\begin{Entry}{变为}{8,4}{⼜、⼂}
  \begin{Phonetics}{变为}{bian4 wei2}[][HSK 3]
    \definition{v.}{transformar-se em; tornar-se | mudar para}
  \end{Phonetics}
\end{Entry}

\begin{Entry}{变化}{8,4}{⼜、⼔}
  \begin{Phonetics}{变化}{bian4hua4}[][HSK 3]
    \definition[个]{s.}{mudança; variação; a nova situação após uma mudança em pessoas ou coisas}
    \definition{v.}{mudar;  variar}
  \end{Phonetics}
\end{Entry}

\begin{Entry}{变心}{8,4}{⼜、⼼}
  \begin{Phonetics}{变心}{bian4xin1}
    \definition{v.+compl.}{deixar de ser fiel}
  \end{Phonetics}
\end{Entry}

\begin{Entry}{变节}{8,5}{⼜、⾋}
  \begin{Phonetics}{变节}{bian4jie2}
    \definition{s.}{traição | deserção | vira-casaca}
    \definition{v.}{retratar-se e submeter-se; renunciar e render-se | mudar de lado politicamente}
  \end{Phonetics}
\end{Entry}

\begin{Entry}{变动}{8,6}{⼜、⼒}
  \begin{Phonetics}{变动}{bian4 dong4}[][HSK 5]
    \definition{s.}{mudança; alteração; oscilação; modificação; variação}
    \definition{v.}{mudar; alterar; oscilar; modificar; variar}
  \end{Phonetics}
\end{Entry}

\begin{Entry}{变异}{8,6}{⼜、⼶}
  \begin{Phonetics}{变异}{bian4yi4}
    \definition{s.}{variação; mutação; muta; diferenças nas características morfológicas e fisiológicas entre gerações da mesma espécie ou entre indivíduos da mesma geração}
    \definition{v.}{variar; mudar}
  \end{Phonetics}
\end{Entry}

\begin{Entry}{变成}{8,6}{⼜、⼽}
  \begin{Phonetics}{变成}{bian4 cheng2}[][HSK 2]
    \definition{v.}{crescer; tornar-se; fazer; desenvolver-se; revelar-se; resultar; acontecer; passar a ser; passar para; acumular-se; converter-se; transformar-se; transformar-se em; mudar-se em; transformação da situação ou condição anterior para a situação ou condição atual}
  \end{Phonetics}
\end{Entry}

\begin{Entry}{变迁}{8,6}{⼜、⾡}
  \begin{Phonetics}{变迁}{bian4qian1}
    \definition{s.}{mudanças; transição; vicissitudes; mudança em tendências ou condições; mudança de situação ou estágio}
  \end{Phonetics}
\end{Entry}

\begin{Entry}{变形}{8,7}{⼜、⼺}
  \begin{Phonetics}{变形}{bian4 xing2}[][HSK 6]
    \definition{v.+compl.}{deformar; ficar fora de forma | transformar; transformar-se em outras formas}
  \end{Phonetics}
\end{Entry}

\begin{Entry}{变更}{8,7}{⼜、⽈}
  \begin{Phonetics}{变更}{bian4 geng1}[][HSK 6]
    \definition{v.}{alterar; mudar; modificar}
  \end{Phonetics}
\end{Entry}

\begin{Entry}{变性}{8,8}{⼜、⼼}
  \begin{Phonetics}{变性}{bian4xing4}
    \definition{s.}{desnaturação | transexual}
    \definition{v.}{desnaturar | mudar de sexo}
  \end{Phonetics}
\end{Entry}

\begin{Entry}{变换}{8,10}{⼜、⼿}
  \begin{Phonetics}{变换}{bian4 huan4}[][HSK 6]
    \definition{v.}{variar; alternar; mudar a forma ou o conteúdo de algo de uma coisa para outra}
  \end{Phonetics}
\end{Entry}

\begin{Entry}{变装}{8,12}{⼜、⾐}
  \begin{Phonetics}{变装}{bian4zhuang1}
    \definition{v.}{trocar de roupa | vestir-se | vestir uma fantasia | disfarçar-se ou fantasiar-se de personagem real ou ficcional, \emph{cosplay} | travestir-se}
  \end{Phonetics}
\end{Entry}

\begin{Entry}{变数}{8,13}{⼜、⽁}
  \begin{Phonetics}{变数}{bian4shu4}
    \definition{s.}{(matemática) variável | fatores variáveis}
  \end{Phonetics}
\end{Entry}

%%%%% EOF %%%%%

