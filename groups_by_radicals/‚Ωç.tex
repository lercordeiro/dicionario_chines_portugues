%%%
%%% Radical "⽍"
%%%

\section*{Radical 78: ``⽍'' (歺)}\addcontentsline{toc}{section}{Radical 78: ⽍、歺}

\begin{Entry}{歹}{4}{⽍}[Kangxi 78]
  \begin{Phonetics}{歹}{dai3}
    \definition{adj.}{maligno; cruel; ruim, refere-se principalmente a pessoas e coisas}
  \end{Phonetics}
\end{Entry}

\begin{Entry}{歹徒}{4,10}{⽍、⼻}
  \begin{Phonetics}{歹徒}{dai3tu2}[][HSK 7-9]
    \definition[个,伙,群]{s.}{rufião; malfeitor; arruaceiro; canalha}
  \end{Phonetics}
\end{Entry}

\begin{Entry}{死}{6}{⽍}
  \begin{Phonetics}{死}{si3}[][HSK 3]
    \definition{adj.}{até a morte | implacável; mortal | fixo; rígido; inflexível | intransitável; fechado | (expressando raiva, reclamação, etc., às vezes jocosamente) maldito}
    \definition{adv.}{(frequentemente no negativo) teimosamente; inflexivelmente}
    \definition{v.}{morrer; estar morto (oposto a 生 e 活)}
  \seealsoref{活}{huo2}
  \seealsoref{生}{sheng1}
  \end{Phonetics}
\end{Entry}

\begin{Entry}{死亡}{6,3}{⽍、⼇}
  \begin{Phonetics}{死亡}{si3wang2}[][HSK 6]
    \definition{s.}{morte; condenação; dar o último suspiro; refere-se ao estado de vida desaparecendo |}
    \definition{v.}{morrer; estar morto; perder a vida (em oposição à 生存)}
  \seealsoref{生存}{sheng1cun2}
  \end{Phonetics}
\end{Entry}

\begin{Entry}{歼}{7}{⽍}
  \begin{Phonetics}{歼}{jian1}
    \definition{v.}{aniquilar; eliminar; destruir}
  \end{Phonetics}
\end{Entry}

\begin{Entry}{歼灭}{7,5}{⽍、⽕}
  \begin{Phonetics}{歼灭}{jian1mie4}[][HSK 7-9]
    \definition{v.}{aniquilar; eliminar; destruir}
  \end{Phonetics}
\end{Entry}

\begin{Entry}{残}{9}{⽍}
  \begin{Phonetics}{残}{can2}[][HSK 7-9]
    \definition{adj.}{incompleto; fragmentário; deficiente | remanescente; restante | cruel; feroz | opressivo; selvagem; bárbaro}
    \definition{v.}{ferir; danificar | estragar; prejudicar; destruir}
  \end{Phonetics}
\end{Entry}

\begin{Entry}{残忍}{9,7}{⽍、⼼}
  \begin{Phonetics}{残忍}{can2ren3}[][HSK 7-9]
    \definition{adj.}{cruel; implacável; impiedoso}
  \end{Phonetics}
\end{Entry}

\begin{Entry}{残留}{9,10}{⽍、⽥}
  \begin{Phonetics}{残留}{can2liu2}[][HSK 7-9]
    \definition{adj.}{residual; restante}
    \definition{s.}{vestígio; resto}
    \definition{v.}{sobrar}
  \end{Phonetics}
\end{Entry}

\begin{Entry}{残疾}{9,10}{⽍、⽧}
  \begin{Phonetics}{残疾}{can2ji2}[][HSK 6]
    \definition{s.}{deformidade; deficiência; deficiência física; defeitos de membros, órgãos ou funções fisiológicas}
  \end{Phonetics}
\end{Entry}

\begin{Entry}{残疾人}{9,10,2}{⽍、⽧、⼈}
  \begin{Phonetics}{残疾人}{can2 ji2 ren2}[][HSK 6]
    \definition[位,名]{s.}{pessoa com deficiência (ou incapacitada); o incapacitado (ou deficiente); pessoas com deficiência visual, auditiva, de linguagem, intelectual, física ou mental são os principais alvos da medicina de reabilitação}
  \end{Phonetics}
\end{Entry}

\begin{Entry}{残缺}{9,10}{⽍、⽸}
  \begin{Phonetics}{残缺}{can2que1}[][HSK 7-9]
    \definition{adj.}{incompleto; fragmentário; com partes faltando}
  \end{Phonetics}
\end{Entry}

\begin{Entry}{残酷}{9,14}{⽍、⾣}
  \begin{Phonetics}{残酷}{can2ku4}[][HSK 6]
    \definition{adj.}{cruel; brutal; implacável}
  \end{Phonetics}
\end{Entry}

\begin{Entry}{殖}{12}{⽍}
  \begin{Phonetics}{殖}{zhi2}
    \definition{v.}{crescer | reproduzir}
  \end{Phonetics}
\end{Entry}

%%%%% EOF %%%%%

