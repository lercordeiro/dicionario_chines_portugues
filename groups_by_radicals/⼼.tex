%%%
%%% Radical "⼼"
%%%

\section*{Radical 61: ``⼼'' (忄、⺗)}\addcontentsline{toc}{section}{Radical 61: ⼼、忄、⺗}

\begin{entry}{心}{4}{⼼}[Kangxi 61]
  \begin{phonetics}{心}{xin1}[][HSK 3]
    \definition*{s.}{Xin, uma das mansões lunares}
    \definition[颗]{s.}{o coração | coração; mente; sentimento; intenção | centro; núcleo}
  \end{phonetics}
\end{entry}

\begin{entry}{心中}{4,4}{⼼、⼁}
  \begin{phonetics}{心中}{xin1zhong1}[][HSK 2]
    \definition{adv.}{nos pensamentos | no coração}
    \definition{s.}{ponto central}
  \end{phonetics}
\end{entry}

\begin{entry}{心机}{4,6}{⼼、⽊}
  \begin{phonetics}{心机}{xin1ji1}
    \definition{s.}{pensamento | esquema}
  \end{phonetics}
\end{entry}

\begin{entry}{心声}{4,7}{⼼、⼠}
  \begin{phonetics}{心声}{xin1sheng1}
    \definition{s.}{desejo sincero | voz interior | aspiração}
  \end{phonetics}
\end{entry}

\begin{entry}{心里}{4,7}{⼼、⾥}
  \begin{phonetics}{心里}{xin1 li3}[][HSK 2]
    \definition[把]{s.}{no coração | no coração de alguém | na mente}
  \end{phonetics}
\end{entry}

\begin{entry}{心疼}{4,10}{⼼、⽧}
  \begin{phonetics}{心疼}{xin1teng2}
    \definition{adj.}{angustiado}
    \definition{v.}{sentir pena de alguém | arrepender-se | ressentir-se | ficar angustiado}
  \end{phonetics}
\end{entry}

\begin{entry}{心情}{4,11}{⼼、⼼}
  \begin{phonetics}{心情}{xin1qing2}[][HSK 2]
    \definition{s.}{humor | sentimento | estado de espírito}
  \end{phonetics}
\end{entry}

\begin{entry}{心理}{4,11}{⼼、⽟}
  \begin{phonetics}{心理}{xin1li3}[][HSK 4]
    \definition{adj.}{psicológico}
    \definition{s.}{mentalidade; refere-se à reflexão da mente humana sobre coisas objetivas, incluindo sensação, percepção, memória, pensamento e emoções | psicologia}
  \end{phonetics}
\end{entry}

\begin{entry}{必}{5}{⼼}
  \begin{phonetics}{必}{bi4}[][HSK 5]
    \definition{adv.}{certamente; necessariamente; indica que algo é certo ou que alguém acredita que esteja correto}
  \end{phonetics}
\end{entry}

\begin{entry}{必定}{5,8}{⼼、⼧}
  \begin{phonetics}{必定}{bi4ding4}
    \definition{adv.}{sem falta | certamente | com certeza | definitivamente | inevitavelmente | com determinação}
    \definition{v.}{estar vinculado a | ter certeza de}
  \end{phonetics}
\end{entry}

\begin{entry}{必要}{5,9}{⼼、⾑}
  \begin{phonetics}{必要}{bi4yao4}[][HSK 3]
    \definition{adj.}{necessário | essencial | indispensável}
    \definition[个,些]{s.}{necessidade}
  \end{phonetics}
\end{entry}

\begin{entry}{必须}{5,9}{⼼、⾴}
  \begin{phonetics}{必须}{bi4xu1}[][HSK 2]
    \definition{adv.}{necessariamente | obrigatoriamente}
  \end{phonetics}
\end{entry}

\begin{entry}{必然}{5,12}{⼼、⽕}
  \begin{phonetics}{必然}{bi4ran2}[][HSK 3]
    \definition{adj.}{certo | inevitável | necessário}
    \definition{adv.}{inevitavelmente}
    \definition{s.}{necessidade}
  \end{phonetics}
\end{entry}

\begin{entry}{必需}{5,14}{⼼、⾬}
  \begin{phonetics}{必需}{bi4 xu1}[][HSK 5]
    \definition{v.}{ser essencial; ser indispensável}
  \end{phonetics}
\end{entry}

\begin{entry}{忙}{6}{⼼}
  \begin{phonetics}{忙}{mang2}[][HSK 1]
    \definition{adj.}{ocupado}
    \definition{s.}{apressar}
  \end{phonetics}
\end{entry}

\begin{entry}{忍耐}{7,9}{⼼、⽽}
  \begin{phonetics}{忍耐}{ren3nai4}
    \definition{s.}{paciência | resistência}
    \definition{v.}{suportar | resistir | exercer paciência}
  \end{phonetics}
\end{entry}

\begin{entry}{志愿}{7,14}{⼼、⽕}
  \begin{phonetics}{志愿}{zhi4 yuan4}[][HSK 3]
    \definition{s.}{desejo; ideal; aspiração; meta que se espera alcançar}
    \definition{v.}{ser voluntário; tomar a iniciativa e esteja disposto a fazer um trabalho que não gere renda ou que tenha renda muito baixa, mas que possa ajudar outras pessoas}
  \end{phonetics}
\end{entry}

\begin{entry}{志愿者}{7,14,8}{⼼、⽕、⽼}
  \begin{phonetics}{志愿者}{zhi4yuan4zhe3}[][HSK 3]
    \definition{s.}{voluntário; pessoa que se voluntaria para servir em atividades de assistência social, eventos de grande porte, conferências, etc.}
  \end{phonetics}
\end{entry}

\begin{entry}{忘}{7}{⼼}
  \begin{phonetics}{忘}{wang4}[][HSK 1]
    \definition{v.}{esquecer | negligenciar | ignorar}
  \end{phonetics}
\end{entry}

\begin{entry}{忘本}{7,5}{⼼、⽊}
  \begin{phonetics}{忘本}{wang4ben3}
    \definition{v.}{esquecer as próprias raízes}
  \end{phonetics}
\end{entry}

\begin{entry}{忘记}{7,5}{⼼、⾔}
  \begin{phonetics}{忘记}{wang4ji4}[][HSK 1]
    \definition{v.}{esquecer}
  \end{phonetics}
\end{entry}

\begin{entry}{忘却}{7,7}{⼼、⼙}
  \begin{phonetics}{忘却}{wang4que4}
    \definition{v.}{esquecer}
  \end{phonetics}
\end{entry}

\begin{entry}{忘怀}{7,7}{⼼、⼼}
  \begin{phonetics}{忘怀}{wang4huai2}
    \definition{v.}{esquecer}
  \end{phonetics}
\end{entry}

\begin{entry}{忘恩}{7,10}{⼼、⼼}
  \begin{phonetics}{忘恩}{wang4'en1}
    \definition{v.}{ser ingrato}
  \end{phonetics}
\end{entry}

\begin{entry}{忘掉}{7,11}{⼼、⼿}
  \begin{phonetics}{忘掉}{wang4diao4}
    \definition{v.}{esquecer}
  \end{phonetics}
\end{entry}

\begin{entry}{忘餐}{7,16}{⼼、⾷}
  \begin{phonetics}{忘餐}{wang4can1}
    \definition{v.}{esquecer as refeições}
  \end{phonetics}
\end{entry}

\begin{entry}{忧郁}{7,8}{⼼、⾢}
  \begin{phonetics}{忧郁}{you1yu4}
    \definition{adj.}{deprimido | melancólico | desanimado}
    \definition{s.}{depressão | melancolia}
  \end{phonetics}
\end{entry}

\begin{entry}{快}{7}{⼼}
  \begin{phonetics}{快}{kuai4}[][HSK 1]
    \definition{adj.}{quase | rápido | depressa}
    \definition{v.}{apressar-se}
  \end{phonetics}
\end{entry}

\begin{entry}{快乐}{7,5}{⼼、⼃}
  \begin{phonetics}{快乐}{kuai4le4}[][HSK 2]
    \definition{adj.}{feliz | alegre}
    \definition{s.}{felicidade | alegria}
  \end{phonetics}
\end{entry}

\begin{entry}{快活}{7,9}{⼼、⽔}
  \begin{phonetics}{快活}{kuai4huo5}[][HSK 5]
    \definition{adj.}{feliz; alegre; contente; animado}
  \end{phonetics}
\end{entry}

\begin{entry}{快点儿}{7,9,2}{⼼、⽕、⼉}
  \begin{phonetics}{快点儿}{kuai4 dian3r5}[][HSK 2]
    \definition{v.}{apressar-se}
  \end{phonetics}
\end{entry}

\begin{entry}{快要}{7,9}{⼼、⾑}
  \begin{phonetics}{快要}{kuai4 yao4}[][HSK 2]
    \definition{adv.}{estar prestes a | estar indo para | estar à beira de | em breve | em nenhum momento}
  \end{phonetics}
\end{entry}

\begin{entry}{快递}{7,10}{⼼、⾡}
  \begin{phonetics}{快递}{kuai4 di4}[][HSK 4]
    \definition[个]{s.}{correio rápido; entrega expressa; entrega rápida}
    \definition{v.}{entregar (serviço de entrega rápida por transportadoras especializadas)}
  \end{phonetics}
\end{entry}

\begin{entry}{快速}{7,10}{⼼、⾡}
  \begin{phonetics}{快速}{kuai4 su4}[][HSK 3]
    \definition{adj.}{rápido; veloz; de alta velocidade}
  \end{phonetics}
\end{entry}

\begin{entry}{快餐}{7,16}{⼼、⾷}
  \begin{phonetics}{快餐}{kuai4 can1}[][HSK 2]
    \definition[份,顿]{s.}{comida rápida | \emph{fast food}}
  \end{phonetics}
\end{entry}

\begin{entry}{怀旧}{7,5}{⼼、⽇}
  \begin{phonetics}{怀旧}{huai2jiu4}
    \definition{s.}{nostalgia}
    \definition{v.}{sentir-se nostálgico}
  \end{phonetics}
\end{entry}

\begin{entry}{怀念}{7,8}{⼼、⼼}
  \begin{phonetics}{怀念}{huai2nian4}[][HSK 4]
    \definition{v.}{pensar em; valorizar a memória de}
  \end{phonetics}
\end{entry}

\begin{entry}{怀疑}{7,14}{⼼、⽦}
  \begin{phonetics}{怀疑}{huai2yi2}[][HSK 4]
    \definition{v.}{duvidar; suspeitar | supor}
  \end{phonetics}
\end{entry}

\begin{entry}{念}{8}{⼼}
  \begin{phonetics}{念}{nian4}[][HSK 3]
    \definition*{s.}{sobrenome Nian}
    \definition{num.}{vinte; 20}
    \definition{s.}{ideia; pensamento}
    \definition{v.}{ler em voz alta | estudar; frequentar a escola | considerar; levar em conta | sentir falta; pensar em}
  \end{phonetics}
\end{entry}

\begin{entry}{忽视}{8,8}{⼼、⾒}
  \begin{phonetics}{忽视}{hu1shi4}[][HSK 4]
    \definition{v.}{ignorar; negligenciar; menosprezar; desprezar; dar de ombros}
  \end{phonetics}
\end{entry}

\begin{entry}{忽然}{8,12}{⼼、⽕}
  \begin{phonetics}{忽然}{hu1ran2}[][HSK 2]
    \definition{adv.}{de repente}
  \end{phonetics}
\end{entry}

\begin{entry}{态度}{8,9}{⼼、⼴}
  \begin{phonetics}{态度}{tai4du5}[][HSK 2]
    \definition[个]{s.}{maneira | comportamento | atitude | atitude | abordagem}
  \end{phonetics}
\end{entry}

\begin{entry}{怕}{8}{⼼}
  \begin{phonetics}{怕}{pa4}[][HSK 2]
    \definition*{s.}{sobrenome Pa}
    \definition{adv.}{por medo; talvez; suponho}
    \definition{v.}{recear; temer; ter medo de | ser incapaz de suportar (ficar de pé, suportar) | ter medo de; ter medo de}
  \end{phonetics}
\end{entry}

\begin{entry}{性}{8}{⼼}
  \begin{phonetics}{性}{xing4}[][HSK 3]
    \definition*{s.}{sobrenome Xing}
    \definition[个]{s.}{natureza; caráter; disposição | propriedade; qualidade | sexo; gênero}
    \definition{suf.}{forma substantivo a partir de adjetivo | indica natureza, escopo ou maneira}
  \end{phonetics}
\end{entry}

\begin{entry}{性生活}{8,5,9}{⼼、⽣、⽔}
  \begin{phonetics}{性生活}{xing4sheng1huo2}
    \definition{s.}{vida sexual}
  \end{phonetics}
\end{entry}

\begin{entry}{性别}{8,7}{⼼、⼑}
  \begin{phonetics}{性别}{xing4bie2}[][HSK 3]
    \definition[种]{s.}{sexo; gênero}
  \end{phonetics}
\end{entry}

\begin{entry}{性质}{8,8}{⼼、⾙}
  \begin{phonetics}{性质}{xing4zhi4}[][HSK 4]
    \definition[个,种,类]{s.}{natureza; qualidade; caráter; propriedade; propriedade fundamental que distingue uma coisa de outra}
  \end{phonetics}
\end{entry}

\begin{entry}{性侵}{8,9}{⼼、⼈}
  \begin{phonetics}{性侵}{xing4qin1}
    \definition{s.}{agressão sexual}
    \definition{v.}{agredir sexualmente}
  \end{phonetics}
\end{entry}

\begin{entry}{性格}{8,10}{⼼、⽊}
  \begin{phonetics}{性格}{xing4ge2}[][HSK 3]
    \definition[种,个]{s.}{caráter; temperamento}
  \end{phonetics}
\end{entry}

\begin{entry}{怪}{8}{⼼}
  \begin{phonetics}{怪}{guai4}[][HSK 4,5]
    \definition*{s.}{sobrenome Guai}
    \definition{adj.}{estranho; esquisito; peculiar; excêntrico; pitoresco; monstruoso; desconcertante; anormal; incomum}
    \definition{adv.}{bastante; muito}
    \definition{s.}{monstro; demônio; diabo; ser maligno}
    \definition{v.}{achar algo estranho; admirar; ficar surpreso | culpar; repreender}
  \end{phonetics}
\end{entry}

\begin{entry}{怪兽}{8,11}{⼼、⼋}
  \begin{phonetics}{怪兽}{guai4shou4}
    \definition{s.}{animal raro | animal mítico | monstro}
  \end{phonetics}
\end{entry}

\begin{entry}{怪癖}{8,18}{⼼、⽧}
  \begin{phonetics}{怪癖}{guai4pi3}
    \definition{adj.}{peculiar}
    \definition{s.}{excentricidade | peculiaridade | hobby estranho}
  \end{phonetics}
\end{entry}

\begin{entry}{怎}{9}{⼼}
  \begin{phonetics}{怎}{zen3}
    \definition{adv.}{como}
  \end{phonetics}
\end{entry}

\begin{entry}{怎么}{9,3}{⼼、⼃}
  \begin{phonetics}{怎么}{zen3me5}[][HSK 1]
    \definition{pron.}{como? | o que?}
  \end{phonetics}
\end{entry}

\begin{entry}{怎么了}{9,3,2}{⼼、⼃、⼅}
  \begin{phonetics}{怎么了}{zen3me5le5}
    \definition{expr.}{O que aconteceu? | O que está acontecendo? | E aí?}
  \end{phonetics}
\end{entry}

\begin{entry}{怎么办}{9,3,4}{⼼、⼃、⼒}
  \begin{phonetics}{怎么办}{zen3 me5 ban4}[][HSK 2]
    \definition{adv.}{o que fazer?}
  \end{phonetics}
\end{entry}

\begin{entry}{怎么回事}{9,3,6,8}{⼼、⼃、⼞、⼅}
  \begin{phonetics}{怎么回事}{zen3me5hui2shi4}
    \definition{expr.}{O que aconteceu? | O que se passou?}
  \end{phonetics}
\end{entry}

\begin{entry}{怎么样}{9,3,10}{⼼、⼃、⽊}
  \begin{phonetics}{怎么样}{zen3me5yang4}[][HSK 2]
    \definition{adv.}{como? | que tal?}
  \end{phonetics}
\end{entry}

\begin{entry}{怎么得了}{9,3,11,2}{⼼、⼃、⼻、⼅}
  \begin{phonetics}{怎么得了}{zen3me5de2liao3}
    \definition{expr.}{Como isso pode ser? | Que bagunça horrível! | O que deve ser feito?}
  \end{phonetics}
\end{entry}

\begin{entry}{怎么搞的}{9,3,13,8}{⼼、⼃、⼿、⽩}
  \begin{phonetics}{怎么搞的}{zen3me5gao3de5}
    \definition{expr.}{Como isso aconteceu? | O que deu errado? | E aí? | O que está errado?}
  \end{phonetics}
\end{entry}

\begin{entry}{怎样}{9,10}{⼼、⽊}
  \begin{phonetics}{怎样}{zen3 yang4}[][HSK 2]
    \definition{pron.}{como | o que | de uma certa maneira | de qualquer maneira | não importa o quão}
  \end{phonetics}
\end{entry}

\begin{entry}{怒骂}{9,9}{⼼、⾺}
  \begin{phonetics}{怒骂}{nu4ma4}
    \definition{v.}{praguejar de raiva}
  \end{phonetics}
\end{entry}

\begin{entry}{思考}{9,6}{⼼、⽼}
  \begin{phonetics}{思考}{si1kao3}[][HSK 4]
    \definition{v.}{pensar; ponderar; considerar; deliberar; envolver-se em atividades de pensamento, como análise, síntese, julgamento, raciocínio e generalização}
  \end{phonetics}
\end{entry}

\begin{entry}{思想}{9,13}{⼼、⼼}
  \begin{phonetics}{思想}{si1xiang3}[][HSK 3]
    \definition[个]{s.}{reflexão; pensamento; ideologia | ideia}
  \end{phonetics}
\end{entry}

\begin{entry}{急}{9}{⼼}
  \begin{phonetics}{急}{ji2}[][HSK 2]
    \definition{adj.}{impaciente |ansioso | irritado | aborrecido |violento | urgente | premente}
    \definition{s.}{urgência | emergência}
    \definition{v.}{preocupar | estar ansioso para ajudar}
  \end{phonetics}
\end{entry}

\begin{entry}{急忙}{9,6}{⼼、⼼}
  \begin{phonetics}{急忙}{ji2mang2}[][HSK 4]
    \definition{adv.}{apressadamente; com pressa}
  \end{phonetics}
\end{entry}

\begin{entry}{急救}{9,11}{⼼、⽁}
  \begin{phonetics}{急救}{ji2jiu4}
    \definition{s.}{primeiros socorros}
    \definition{v.}{dar tratamento de emergência}
  \end{phonetics}
\end{entry}

\begin{entry}{怹}{9}{⼼}
  \begin{phonetics}{怹}{tan1}
    \definition{pron.}{ele, ela (cortês, em oposição a 他)}
    \seeref{他}{ta1}
  \end{phonetics}
\end{entry}

\begin{entry}{总}{9}{⼼}
  \begin{phonetics}{总}{zong3}[][HSK 3]
    \definition{adj.}{total; geral; global | responsável (liderança)}
    \definition{adv.}{sempre; invariavelmente | de qualquer forma; afinal; eventualmente; mais cedo ou mais tarde | seguramente; provavelmente; certamente}
    \definition{v.}{resumir; juntar; reunir}
  \end{phonetics}
\end{entry}

\begin{entry}{总之}{9,3}{⼼、⼂}
  \begin{phonetics}{总之}{zong3zhi1}[][HSK 4]
    \definition{conj.}{em uma palavra; em suma; em resumo; indica que a declaração seguinte é uma declaração geral}
  \end{phonetics}
\end{entry}

\begin{entry}{总长}{9,4}{⼼、⾧}
  \begin{phonetics}{总长}{zong3chang2}
    \definition{s.}{comprimento total}
  \end{phonetics}
\end{entry}

\begin{entry}{总务}{9,5}{⼼、⼒}
  \begin{phonetics}{总务}{zong3wu4}
    \definition{s.}{divisão de assuntos gerais | assuntos gerais | pessoa responsável geral}
  \end{phonetics}
\end{entry}

\begin{entry}{总台}{9,5}{⼼、⼝}
  \begin{phonetics}{总台}{zong3tai2}
    \definition{s.}{recepção | balcão de recepção}
  \end{phonetics}
\end{entry}

\begin{entry}{总价}{9,6}{⼼、⼈}
  \begin{phonetics}{总价}{zong3jia4}
    \definition{s.}{preço total}
  \end{phonetics}
\end{entry}

\begin{entry}{总共}{9,6}{⼼、⼋}
  \begin{phonetics}{总共}{zong3gong4}[][HSK 4]
    \definition{adv.}{em tudo; em todos; no total; completamente; totalmente; em conjunto}
  \end{phonetics}
\end{entry}

\begin{entry}{总线}{9,8}{⼼、⽷}
  \begin{phonetics}{总线}{zong3xian4}
    \definition{s.}{barramento (computador) | \emph{computer bus}}
  \end{phonetics}
\end{entry}

\begin{entry}{总是}{9,9}{⼼、⽇}
  \begin{phonetics}{总是}{zong3shi4}[][HSK 3]
    \definition{adv.}{sempre; indica que algo está acontecendo por um período de tempo; um certo estado permanece inalterado
 | afinal; significa que não importa o que aconteça, haverá um resultado.}
  \end{phonetics}
\end{entry}

\begin{entry}{总结}{9,9}{⼼、⽷}
  \begin{phonetics}{总结}{zong3jie2}[][HSK 3]
    \definition[个,份]{s.}{resumo; conclusão obtida}
    \definition{v.}{resumir; sumariar; analisar a experiência da pesquisa e tirar conclusões}
  \end{phonetics}
\end{entry}

\begin{entry}{总统}{9,9}{⼼、⽷}
  \begin{phonetics}{总统}{zong3tong3}[][HSK 4]
    \definition*[个,位,名]{s.}{Presidente (de um país); Título dos líderes de determinadas repúblicas}
  \end{phonetics}
\end{entry}

\begin{entry}{总值}{9,10}{⼼、⼈}
  \begin{phonetics}{总值}{zong3zhi2}
    \definition{s.}{valor total}
  \end{phonetics}
\end{entry}

\begin{entry}{总站}{9,10}{⼼、⽴}
  \begin{phonetics}{总站}{zong3zhan4}
    \definition{s.}{terminal}
  \end{phonetics}
\end{entry}

\begin{entry}{总得}{9,11}{⼼、⼻}
  \begin{phonetics}{总得}{zong3dei3}
    \definition{adv.}{prestes a}
    \definition{v.}{dever | precisar}
  \end{phonetics}
\end{entry}

\begin{entry}{总理}{9,11}{⼼、⽟}
  \begin{phonetics}{总理}{zong3li3}[][HSK 4]
    \definition*[个,位,名]{s.}{Primeiro-Ministro do Conselho de Estado; Título do líder do Conselho de Estado da China | Título do chefe de governo em determinados países | Primeiro-Ministro; Título de líderes de determinados partidos políticos | Título dos chefes de determinadas instituições e empresas nos velhos tempos}
    \definition{v.}{assumir a responsabilidade total;}
  \end{phonetics}
\end{entry}

\begin{entry}{总督}{9,13}{⼼、⽬}
  \begin{phonetics}{总督}{zong3du1}
    \definition*{s.}{Governador-Geral | Governador | Vice-Rei}
  \end{phonetics}
\end{entry}

\begin{entry}{恒星系}{9,9,7}{⼼、⽇、⽷}
  \begin{phonetics}{恒星系}{heng2xing1xi4}
    \definition{s.}{sistema estelar | galáxia}
  \end{phonetics}
\end{entry}

\begin{entry}{恢复}{9,9}{⼼、⼢}
  \begin{phonetics}{恢复}{hui1fu4}[][HSK 5]
    \definition{v.}{retomar; renovar; restaurar; voltar a | reviver; recuperar; reencontrar | restaurar; restabelecer; reabilitar; regenerar; ressurgir; restabelecer alguém em; recuperar o que foi perdido}
  \end{phonetics}
\end{entry}

\begin{entry}{T-恤}{9}{⼼}
  \begin{phonetics}{T-恤}{xu4}
    \definition{s.}{camiseta | pulôver | suéter}
  \end{phonetics}
\end{entry}

\begin{entry}{恨}{9}{⼼}
  \begin{phonetics}{恨}{hen4}[][HSK 5]
    \definition{s.}{ódio; resentimento}
    \definition{v.}{odiar}
  \end{phonetics}
\end{entry}

\begin{entry}{恰}{9}{⼼}
  \begin{phonetics}{恰}{qia4}
    \definition{adv.}{exatamente | apenas}
  \end{phonetics}
\end{entry}

\begin{entry}{恰好}{9,6}{⼼、⼥}
  \begin{phonetics}{恰好}{qia4hao3}
    \definition{adv.}{certo | por sorte | ao que parece | por sorte coincidência}
  \end{phonetics}
\end{entry}

\begin{entry}{恰到好处}{9,8,6,5}{⼼、⼑、⼥、⼡}
  \begin{phonetics}{恰到好处}{qia4dao4hao3chu4}
    \definition{expr.}{é simplesmente perfeito | é simplesmente correto}
  \end{phonetics}
\end{entry}

\begin{entry}{恋爱}{10,10}{⼼、⽖}
  \begin{phonetics}{恋爱}{lian4'ai4}[][HSK 5]
    \definition[个,场]{s.}{amor (romântico)}
    \definition[个,场]{s.}{namoro; afeto; amor romântico}
    \definition{v.}{amar; estar apaixonado}
  \end{phonetics}
\end{entry}

\begin{entry}{恐龙}{10,5}{⼼、⿓}
  \begin{phonetics}{恐龙}{kong3long2}
    \definition[头,只]{s.}{dinossauro}
  \end{phonetics}
\end{entry}

\begin{entry}{恐怕}{10,8}{⼼、⼼}
  \begin{phonetics}{恐怕}{kong3pa4}[][HSK 3]
    \definition{adv.}{talvez; provavelmente; pode ser | por medo de}
    \definition{v.}{ter medo de; temer; recear}
  \end{phonetics}
\end{entry}

\begin{entry}{恐怖主义}{10,8,5,3}{⼼、⼼、⼂、⼂}
  \begin{phonetics}{恐怖主义}{kong3bu4zhu3yi4}
    \definition{adj.}{terrorista}
    \definition{s.}{terrorismo}
  \end{phonetics}
\end{entry}

\begin{entry}{恩赐}{10,12}{⼼、⾙}
  \begin{phonetics}{恩赐}{en1ci4}
    \definition{s.}{favor | caridade}
    \definition{v.}{conceder (favor, caridade)}
  \end{phonetics}
\end{entry}

\begin{entry}{恶心}{10,4}{⼼、⼼}
  \begin{phonetics}{恶心}{e3xin1}[][HSK 4]
    \definition{adj.}{nauseante; repugnante}
    \definition{s.}{enjoo; náusea; repugnância; sensação de enjoo; vontade de vomitar}
    \definition{v.}{repugnar; ser nauseante; vomitar}
  \end{phonetics}
  \begin{phonetics}{恶心}{e4xin1}
    \definition{s.}{mau hábito | hábito vicioso | vício}
  \end{phonetics}
\end{entry}

\begin{entry}{悉心}{11,4}{⼼、⼼}
  \begin{phonetics}{悉心}{xi1xin1}
    \definition{adv.}{colocar o coração (e a alma) em algo | com muito cuidado}
  \end{phonetics}
\end{entry}

\begin{entry}{悉尼}{11,5}{⼼、⼫}
  \begin{phonetics}{悉尼}{xi1ni2}
    \definition*{s.}{Sidney}
  \end{phonetics}
\end{entry}

\begin{entry}{悉数}{11,13}{⼼、⽁}
  \begin{phonetics}{悉数}{xi1shu3}
    \definition{adv.}{enumerar em detalhes | explicar claramente}
  \end{phonetics}
  \begin{phonetics}{悉数}{xi1shu4}
    \definition{adv.}{todos | cada um | toda a soma}
  \end{phonetics}
\end{entry}

\begin{entry}{您}{11}{⼼}
  \begin{phonetics}{您}{nin2}[][HSK 1]
    \definition{pron.}{você (formal) | tu | te | ti | contigo}
    \seeref{你}{ni3}
  \end{phonetics}
\end{entry}

\begin{entry}{悬挂}{11,9}{⼼、⼿}
  \begin{phonetics}{悬挂}{xuan2gua4}
    \definition{s.}{(veículo) suspensão}
    \definition{v.}{suspender}
  \end{phonetics}
\end{entry}

\begin{entry}{悬崖}{11,11}{⼼、⼭}
  \begin{phonetics}{悬崖}{xuan2ya2}
    \definition{s.}{precipício | penhasco}
  \end{phonetics}
\end{entry}

\begin{entry}{情况}{11,7}{⼼、⼎}
  \begin{phonetics}{情况}{qing2kuang4}[][HSK 3]
    \definition[种,个,些]{s.}{condição; situação; circunstâncias; estado das coisas | mudança notável}
  \end{phonetics}
\end{entry}

\begin{entry}{情绪}{11,11}{⼼、⽷}
  \begin{phonetics}{情绪}{qing2xu4}
    \definition[种]{s.}{humor | estado da mente | mau humor}
  \end{phonetics}
\end{entry}

\begin{entry}{情景}{11,12}{⼼、⽇}
  \begin{phonetics}{情景}{qing2jing3}[][HSK 4]
    \definition[个]{s.}{cena; vista; circunstâncias}
  \end{phonetics}
\end{entry}

\begin{entry}{情感}{11,13}{⼼、⼼}
  \begin{phonetics}{情感}{qing2 gan3}[][HSK 3]
    \definition[份]{s.}{emoção; sentimento | afeição; apego}
    \definition{v.}{mover-se (emocionalmente)}
  \end{phonetics}
\end{entry}

\begin{entry}{惊呆}{11,7}{⼼、⼝}
  \begin{phonetics}{惊呆}{jing1dai1}
    \definition{adj.}{estupefato | chocado}
  \end{phonetics}
\end{entry}

\begin{entry}{惊喜}{11,12}{⼼、⼝}
  \begin{phonetics}{惊喜}{jing1xi3}
    \definition{s.}{boa surpresa}
    \definition{v.}{ser agradavelmente surpreendido}
  \end{phonetics}
\end{entry}

\begin{entry}{惑星}{12,9}{⼼、⽇}
  \begin{phonetics}{惑星}{huo4xing1}
    \definition{s.}{planeta}
  \seealsoref{行星}{xing2xing1}
  \end{phonetics}
\end{entry}

\begin{entry}{惩处}{12,5}{⼼、⼡}
  \begin{phonetics}{惩处}{cheng2chu3}
    \definition{v.}{administrar justiça | punir}
  \end{phonetics}
\end{entry}

\begin{entry}{惩罚}{12,9}{⼼、⽹}
  \begin{phonetics}{惩罚}{cheng2fa2}
    \definition{v.}{punir | penalizar}
  \end{phonetics}
\end{entry}

\begin{entry}{愉快}{12,7}{⼼、⼼}
  \begin{phonetics}{愉快}{yu2kuai4}
    \definition{adj.}{alegre | delicioso | prazeroso | agradável | feliz | encantado}
    \definition{adv.}{alegremente | agradavelmente}
  \end{phonetics}
\end{entry}

\begin{entry}{愤世嫉俗}{12,5,13,9}{⼼、⼀、⼥、⼈}
  \begin{phonetics}{愤世嫉俗}{fen4shi4ji2su2}
    \definition{v.}{ser cínico | ser amargurado}
  \end{phonetics}
\end{entry}

\begin{entry}{愤怒}{12,9}{⼼、⼼}
  \begin{phonetics}{愤怒}{fen4nu4}
    \definition{adj.}{zangado | indignado}
    \definition{s.}{ira}
  \end{phonetics}
\end{entry}

\begin{entry}{慌}{12}{⼼}
  \begin{phonetics}{慌}{huang1}[][HSK 5]
    \definition{adj.}{agitado; confuso; que inspira terror}
    \definition{v.}{ficar com medo; ficar nervoso}
  \end{phonetics}
\end{entry}

\begin{entry}{慌忙}{12,6}{⼼、⼼}
  \begin{phonetics}{慌忙}{huang1 mang2}[][HSK 5]
    \definition{adj.}{apressado; afobado; com muita pressa}
    \definition{adv.}{apressadamente}
  \end{phonetics}
\end{entry}

\begin{entry}{想}{13}{⼼}
  \begin{phonetics}{想}{xiang3}[][HSK 1]
    \definition{v.}{acreditar | sentir falta (sentir-se melancólico com a ausência de alguém ou algo) | supor | pensar | querer | desejar}
  \end{phonetics}
\end{entry}

\begin{entry}{想到}{13,8}{⼼、⼑}
  \begin{phonetics}{想到}{xiang3 dao4}[][HSK 2]
    \definition{v.}{pensar em | trazer à mente | ter no coração}
  \end{phonetics}
\end{entry}

\begin{entry}{想念}{13,8}{⼼、⼼}
  \begin{phonetics}{想念}{xiang3nian4}[][HSK 4]
    \definition{v.}{sentir falta; pensar em; lembrar com carinho; ficar doente por; desejar ver novamente; lembrar com saudade}
  \end{phonetics}
\end{entry}

\begin{entry}{想法}{13,8}{⼼、⽔}
  \begin{phonetics}{想法}{xiang3 fa3}[][HSK 2]
    \definition[个]{s.}{noção | opinião | jeito de pensar}
    \definition{s.}{maneira de pensar | opinião | noção}
    \definition{v.}{pensar em uma maneira (de fazer algo)}
  \end{phonetics}
\end{entry}

\begin{entry}{想起}{13,10}{⼼、⾛}
  \begin{phonetics}{想起}{xiang3 qi3}[][HSK 2]
    \definition{v.}{recordar | lembrar | pensar em | trazer à mente | cruzar pelos pensamentos de alguém | passar pelo pensamento de alguém}
  \end{phonetics}
\end{entry}

\begin{entry}{想象}{13,11}{⼼、⾗}
  \begin{phonetics}{想象}{xiang3xiang4}[][HSK 4]
    \definition[个]{s.}{imaginação; refere-se ao processo mental de processamento e transformação de representações armazenadas na mente para formar novas imagens}
    \definition{v.}{imaginar; vislumbrar; visualizar; refere-se a ter uma imagem concreta de algo que não está na frente dos olhos}
  \end{phonetics}
\end{entry}

\begin{entry}{想想看}{13,13,9}{⼼、⼼、⽬}
  \begin{phonetics}{想想看}{xiang3xiang3kan4}
    \definition{v.}{pensar sobre isso}
  \end{phonetics}
\end{entry}

\begin{entry}{愁}{13}{⼼}
  \begin{phonetics}{愁}{chou2}[][HSK 5]
    \definition{adj.}{triste; pesaroso; angustiado; desconsolado}
    \definition{s.}{pesar; sofrimento; dor; tristeza}
    \definition{v.}{preocupar-se; estar preocuoado; ficar ansioso; sentir ansiedade}
  \end{phonetics}
\end{entry}

\begin{entry}{愈}{13}{⼼}
  \begin{phonetics}{愈}{yu4}
    \definition{adv.}{mais e mais | ainda mais}
    \definition{v.}{recuperar | curar}
  \end{phonetics}
\end{entry}

\begin{entry}{意义}{13,3}{⼼、⼂}
  \begin{phonetics}{意义}{yi4yi4}[][HSK 3]
    \definition[个]{s.}{sentido; significado; significado expresso por palavras ou outros sinais; significado indicado por comportamento ou aquisição |valor; efeito; significância; impacto}
  \end{phonetics}
\end{entry}

\begin{entry}{意见}{13,4}{⼼、⾒}
  \begin{phonetics}{意见}{yi4jian4}[][HSK 2]
    \definition[点,条]{s.}{reclamação | ideia | objeção | opinião | sugestão}
  \end{phonetics}
\end{entry}

\begin{entry}{意外}{13,5}{⼼、⼣}
  \begin{phonetics}{意外}{yi4wai4}[][HSK 3]
    \definition{adj.}{inesperado; imprevisto}
    \definition{adv.}{acidentalmente}
    \definition[个]{s.}{acidente; infortúnio; infortúnio inesperado}
  \end{phonetics}
\end{entry}

\begin{entry}{意志}{13,7}{⼼、⼼}
  \begin{phonetics}{意志}{yi4zhi4}
    \definition[个]{s.}{determinação | desejo | força de vontade}
  \end{phonetics}
\end{entry}

\begin{entry}{意识}{13,7}{⼼、⾔}
  \begin{phonetics}{意识}{yi4shi2}
    \definition{s.}{consciência}
    \definition{v.}{(usualmente seguido de 到) estar ciente, constatar}
  \end{phonetics}
\end{entry}

\begin{entry}{意译}{13,7}{⼼、⾔}
  \begin{phonetics}{意译}{yi4yi4}
    \definition{s.}{tradução livre | significado (de expressão estrangeira) | paráfrase | tradução do significado (em oposição à tradução literal)}
  \seealsoref{直译}{zhi2yi4}
  \end{phonetics}
\end{entry}

\begin{entry}{意思}{13,9}{⼼、⼼}
  \begin{phonetics}{意思}{yi4si5}[][HSK 2]
    \definition[个]{s.}{interesse}
  \end{phonetics}
\end{entry}

\begin{entry}{意指}{13,9}{⼼、⼿}
  \begin{phonetics}{意指}{yi4zhi3}
    \definition{v.}{implicar | significar}
  \end{phonetics}
\end{entry}

\begin{entry}{感兴趣}{13,6,15}{⼼、⼋、⾛}
  \begin{phonetics}{感兴趣}{gan3xing4qu4}[][HSK 4]
    \definition{v.}{estar interessado}
    \seeref{对……感兴趣}{dui4 gan3xing4qu4}
  \end{phonetics}
\end{entry}

\begin{entry}{感动}{13,6}{⼼、⼒}
  \begin{phonetics}{感动}{gan3dong4}[][HSK 2]
    \definition{v.}{mover (alguém) | tocar (alguém emocionalmente)}
  \end{phonetics}
\end{entry}

\begin{entry}{感到}{13,8}{⼼、⼑}
  \begin{phonetics}{感到}{gan3 dao4}[][HSK 2]
    \definition{v.}{sentir | perceber}
  \end{phonetics}
\end{entry}

\begin{entry}{感受}{13,8}{⼼、⼜}
  \begin{phonetics}{感受}{gan3shou4}[][HSK 3]
    \definition{s.}{percepção ; sentimento; experiência}
    \definition{v.}{sentir; sentir (através dos sentidos); experimentar}
  \end{phonetics}
\end{entry}

\begin{entry}{感冒}{13,9}{⼼、⽇}
  \begin{phonetics}{感冒}{gan3mao4}[][HSK 3]
    \definition{adj.}{interessado}
    \definition[场,次]{s.}{resfriado; resfriado comum; gripe}
    \definition{v.}{pegar (ter) um resfriado}
  \end{phonetics}
\end{entry}

\begin{entry}{感染}{13,9}{⼼、⽊}
  \begin{phonetics}{感染}{gan3ran3}
    \definition{s.}{infecção}
    \definition{v.}{infectar | (figurativo) influenciar}
  \end{phonetics}
\end{entry}

\begin{entry}{感觉}{13,9}{⼼、⾒}
  \begin{phonetics}{感觉}{gan3jue2}[][HSK 2]
    \definition{s.}{sentimento | impressão | sensação}
    \definition{v.}{sentir | perceber}
  \end{phonetics}
\end{entry}

\begin{entry}{感情}{13,11}{⼼、⼼}
  \begin{phonetics}{感情}{gan3qing2}[][HSK 3]
    \definition[份,个,种]{s.}{emoção; sentimento | amor; afeição; apego}
  \end{phonetics}
\end{entry}

\begin{entry}{感谢}{13,12}{⼼、⾔}
  \begin{phonetics}{感谢}{gan3xie4}[][HSK 2]
    \definition{s.}{gratidão | agradecimento}
  \end{phonetics}
\end{entry}

\begin{entry}{感想}{13,13}{⼼、⼼}
  \begin{phonetics}{感想}{gan3xiang3}[][HSK 5]
    \definition[个,条]{s.}{pensamentos; impressões; reflexões; resposta do pensamento decorrente da exposição ao mundo exterior}
  \end{phonetics}
\end{entry}

\begin{entry}{慢}{14}{⼼}
  \begin{phonetics}{慢}{man4}[][HSK 1]
    \definition{adj.}{devagar}
  \end{phonetics}
\end{entry}

\begin{entry}{慢动作}{14,6,7}{⼼、⼒、⼈}
  \begin{phonetics}{慢动作}{man4dong4zuo4}
    \definition{s.}{(cinema) câmera lenta}
  \end{phonetics}
\end{entry}

\begin{entry}{慢慢}{14,14}{⼼、⼼}
  \begin{phonetics}{慢慢}{man4 man4}[][HSK 3]
    \definition{adv.}{lentamente; vagarosamente; gradualmente}
  \end{phonetics}
\end{entry}

\begin{entry}{憧憬}{15,15}{⼼、⼼}
  \begin{phonetics}{憧憬}{chong1jing3}
    \definition{v.}{ansiar por | esperar por}
  \end{phonetics}
\end{entry}

\begin{entry}{懂}{15}{⼼}
  \begin{phonetics}{懂}{dong3}[][HSK 2]
    \definition{v.}{compreender | entender}
  \end{phonetics}
\end{entry}

\begin{entry}{懂得}{15,11}{⼼、⼻}
  \begin{phonetics}{懂得}{dong3 de5}[][HSK 2]
    \definition{v.}{saber | entender | compreender}
  \end{phonetics}
\end{entry}

\begin{entry}{懒}{16}{⼼}
  \begin{phonetics}{懒}{lan3}
    \definition{adj.}{preguiçoso | indolente | vadio}
  \end{phonetics}
\end{entry}

\begin{entry}{懒人}{16,2}{⼼、⼈}
  \begin{phonetics}{懒人}{lan3ren2}
    \definition{s.}{pessoa preguiçosa}
  \end{phonetics}
\end{entry}

\begin{entry}{懒汉}{16,5}{⼼、⽔}
  \begin{phonetics}{懒汉}{lan3han4}
    \definition{s.}{sujeito ocioso | vagabundo | preguiçosos}
  \end{phonetics}
\end{entry}

\begin{entry}{懒虫}{16,6}{⼼、⾍}
  \begin{phonetics}{懒虫}{lan3chong2}
    \definition{s.}{desleixado ocioso | (insulto) sujeito preguiçoso}
  \end{phonetics}
\end{entry}

\begin{entry}{懒怠}{16,9}{⼼、⼼}
  \begin{phonetics}{懒怠}{lan3dai4}
    \definition{s.}{preguiça}
  \end{phonetics}
\end{entry}

\begin{entry}{懒鬼}{16,9}{⼼、⿁}
  \begin{phonetics}{懒鬼}{lan3gui3}
    \definition{s.}{cara preguiçoso}
  \end{phonetics}
\end{entry}

\begin{entry}{懒得}{16,11}{⼼、⼻}
  \begin{phonetics}{懒得}{lan3de5}
    \definition{adv.}{demasiado preguiçoso}
    \definition{v.}{não sentir vontade (de fazer algo)}
  \end{phonetics}
\end{entry}

\begin{entry}{懒惰}{16,12}{⼼、⼼}
  \begin{phonetics}{懒惰}{lan3duo4}
    \definition{adj.}{preguiçoso}
  \end{phonetics}
\end{entry}

\begin{entry}{懒散}{16,12}{⼼、⽁}
  \begin{phonetics}{懒散}{lan3san3}
    \definition{adj.}{inativo | indolente | preguiçoso | negligente}
  \end{phonetics}
\end{entry}

\begin{entry}{懒腰}{16,13}{⼼、⾁}
  \begin{phonetics}{懒腰}{lan3yao1}
    \definition[个]{s.}{alongamento (do corpo)}
  \end{phonetics}
\end{entry}

\begin{entry}{懵懂}{18,15}{⼼、⼼}
  \begin{phonetics}{懵懂}{meng3dong3}
    \definition{adj.}{confuso | ignorante}
  \end{phonetics}
\end{entry}

\begin{entry}{聼}{19}{⼼}
  \begin{phonetics}{聼}{ting1}
    \variantof{听}
  \end{phonetics}
\end{entry}

%%%%% EOF %%%%%

