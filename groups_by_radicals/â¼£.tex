%%%
%%% Radical "⼣"
%%%

\section*{Radical 36: ``⼣''}\addcontentsline{toc}{section}{Radical 36: ⼣}

\begin{entry}{夕}{3}{⼣}[Kangxi 36]
  \begin{phonetics}{夕}{xi1}
    \definition*{s.}{Sobrenome Xi}
    \definition{s.}{pôr do sol; crepúsculo | tarde; noite}
  \end{phonetics}
\end{entry}

\begin{entry}{夕阳}{3,6}{⼣、⾩}
  \begin{phonetics}{夕阳}{xi1yang2}
    \definition{s.}{pôr do sol}
  \seealsoref{日出}{ri4chu1}
  \end{phonetics}
\end{entry}

\begin{entry}{外}{5}{⼣}
  \begin{phonetics}{外}{wai4}[][HSK 1]
    \definition{adj.}{outro (que não o próprio) | não íntimo; não intimamente relacionado | não oficial | exterior; externo; do lado de fora | outros; referindo-se a um local fora da sua localização atual | do lado da mãe, da irmã ou da filha; referir-se a parentes do lado materno, irmãs ou filhas | informal; não oficial}
    \definition{adv.}{adicionalmente; além disso | para fora; para o exterior; fora | extra; além disso}
    \definition{s.}{fora; externo; exterior (oposto a 内, 里) | outro local; outro lugar | estrangeiro; país estrangeiro | lado externo | parentes de sua mãe, irmãs ou filhas}
  \seealsoref{里}{li3}
  \seealsoref{内}{nei4}
  \end{phonetics}
\end{entry}

\begin{entry}{外公}{5,4}{⼣、⼋}
  \begin{phonetics}{外公}{wai4gong1}
    \definition{s.}{avô materno}
  \end{phonetics}
\end{entry}

\begin{entry}{外币}{5,4}{⼣、⼱}
  \begin{phonetics}{外币}{wai4 bi4}[][HSK 6]
    \definition[种]{s.}{moeda estrangeira}
  \end{phonetics}
\end{entry}

\begin{entry}{外文}{5,4}{⼣、⽂}
  \begin{phonetics}{外文}{wai4 wen2}[][HSK 3]
    \definition[种,门]{s.}{língua ou escrita estrangeira}
  \end{phonetics}
\end{entry}

\begin{entry}{外水}{5,4}{⼣、⽔}
  \begin{phonetics}{外水}{wai4shui3}
    \definition{s.}{renda extra}
  \end{phonetics}
\end{entry}

\begin{entry}{外出}{5,5}{⼣、⼐}
  \begin{phonetics}{外出}{wai4 chu1}[][HSK 6]
    \definition{v.}{sair, especialmente para ir a outro lugar a negócios}
  \end{phonetics}
\end{entry}

\begin{entry}{外号}{5,5}{⼣、⼝}
  \begin{phonetics}{外号}{wai4hao4}
    \definition{s.}{apelido}
  \end{phonetics}
\end{entry}

\begin{entry}{外头}{5,5}{⼣、⼤}
  \begin{phonetics}{外头}{wai4 tou5}[][HSK 6]
    \definition{s.}{Coloquial: fora; ao ar livre (oposto a 里头)}
  \seealsoref{里头}{li3 tou5}
  \end{phonetics}
\end{entry}

\begin{entry}{外汇}{5,5}{⼣、⽔}
  \begin{phonetics}{外汇}{wai4 hui4}[][HSK 4]
    \definition{s.}{câmbio estrangeiro; moeda estrangeira; moedas estrangeiras e títulos, como cheques, letras de câmbio, notas promissórias, etc., conversíveis em moedas estrangeiras, usados na compensação do comércio internacional}
  \end{phonetics}
\end{entry}

\begin{entry}{外边}{5,5}{⼣、⾡}
  \begin{phonetics}{外边}{wai4 bian5}[][HSK 1]
    \definition{s.}{fora; exterior; externo; além de um determinado limite | local diferente de onde se vive ou trabalha; referindo-se a lugares distantes | exterior; externo; superfície}
  \end{phonetics}
\end{entry}

\begin{entry}{外交}{5,6}{⼣、⼇}
  \begin{phonetics}{外交}{wai4jiao1}[][HSK 3]
    \definition[个]{s.}{diplomacia; relações exteriores; atividades de um país nas relações internacionais, como participar de organizações e conferências internacionais, trocar enviados com outros países, conduzir negociações, assinar tratados e acordos, etc.}
  \end{phonetics}
\end{entry}

\begin{entry}{外交官}{5,6,8}{⼣、⼇、⼧}
  \begin{phonetics}{外交官}{wai4 jiao1 guan1}[][HSK 4]
    \definition{s.}{diplomata}
  \end{phonetics}
\end{entry}

\begin{entry}{外协}{5,6}{⼣、⼗}
  \begin{phonetics}{外协}{wai4xie2}
    \definition{s.}{terceirização | pessoas que julgam os outros pela aparência}
  \seealsoref{外貌协会}{wai4mao4xie2hui4}
  \end{phonetics}
\end{entry}

\begin{entry}{外地}{5,6}{⼣、⼟}
  \begin{phonetics}{外地}{wai4 di4}[][HSK 2]
    \definition{s.}{não local; outros lugares; locais fora da área local}
  \end{phonetics}
\end{entry}

\begin{entry}{外孙}{5,6}{⼣、⼦}
  \begin{phonetics}{外孙}{wai4sun1}
    \definition{s.}{filho da filha}
  \end{phonetics}
\end{entry}

\begin{entry}{外孙女}{5,6,3}{⼣、⼦、⼥}
  \begin{phonetics}{外孙女}{wai4sun1nv3}
    \definition{s.}{filha da filha}
  \end{phonetics}
\end{entry}

\begin{entry}{外衣}{5,6}{⼣、⾐}
  \begin{phonetics}{外衣}{wai4 yi1}[][HSK 6]
    \definition[件]{s.}{casaco; jaqueta; colete; sobreveste; envoltório; roupa externa (ou vestimenta); capa externa; vestido externo | semblante; aparência; feição}
  \end{phonetics}
\end{entry}

\begin{entry}{外观}{5,6}{⼣、⾒}
  \begin{phonetics}{外观}{wai4 guan1}[][HSK 6]
    \definition{s.}{aspecto; semblante; aparência; aparência exterior; a aparência de um objeto}
  \end{phonetics}
\end{entry}

\begin{entry}{外围}{5,7}{⼣、⼞}
  \begin{phonetics}{外围}{wai4wei2}
    \definition{adv.}{arredores}
  \end{phonetics}
\end{entry}

\begin{entry}{外来}{5,7}{⼣、⽊}
  \begin{phonetics}{外来}{wai4 lai2}[][HSK 6]
    \definition{adj.}{de fora; externo; estrangeiro}
  \end{phonetics}
\end{entry}

\begin{entry}{外事}{5,8}{⼣、⼅}
  \begin{phonetics}{外事}{wai4shi4}
    \definition{s.}{assuntos ou relações exteriores}
  \end{phonetics}
\end{entry}

\begin{entry}{外卖}{5,8}{⼣、⼗}
  \begin{phonetics}{外卖}{wai4 mai4}[][HSK 2]
    \definition[份,单,盒]{s.}{comida para viagem; levar para viagem}
    \definition{v.}{entregar; oferecer; refere-se à ação do comerciante entregar alimentos no local especificado pelo cliente}
  \end{phonetics}
\end{entry}

\begin{entry}{外国}{5,8}{⼣、⼞}
  \begin{phonetics}{外国}{wai4 guo2}[][HSK 1]
    \definition[个]{s.}{país estrangeiro}
  \end{phonetics}
\end{entry}

\begin{entry}{外国人}{5,8,2}{⼣、⼞、⼈}
  \begin{phonetics}{外国人}{wai4 guo2 ren2}
    \definition[个]{s.}{estrangeiro | alienígena}
  \end{phonetics}
\end{entry}

\begin{entry}{外界}{5,9}{⼣、⽥}
  \begin{phonetics}{外界}{wai4jie4}[][HSK 5]
    \definition{s.}{o exterior; o mundo externo; área fora de um determinado âmbito; sociedade externa}
  \end{phonetics}
\end{entry}

\begin{entry}{外科}{5,9}{⼣、⽲}
  \begin{phonetics}{外科}{wai4 ke1}[][HSK 6]
    \definition[名]{s.}{cirurgia; departamento cirúrgico; um departamento em uma instituição médica que usa principalmente cirurgia para tratar doenças internas e externas}
  \end{phonetics}
\end{entry}

\begin{entry}{外语}{5,9}{⼣、⾔}
  \begin{phonetics}{外语}{wai4 yu3}[][HSK 1]
    \definition[种,门]{s.}{língua estrangeira}
  \end{phonetics}
\end{entry}

\begin{entry}{外贸}{5,9}{⼣、⾙}
  \begin{phonetics}{外贸}{wai4mao4}
    \definition{s.}{comércio exterior}
  \end{phonetics}
\end{entry}

\begin{entry}{外面}{5,9}{⼣、⾯}
  \begin{phonetics}{外面}{wai4 mian4}[][HSK 3]
    \definition{s.}{o lado de fora; fora de um certo intervalo | exterior; aparência externa; a superfície de um objeto}
  \end{phonetics}
\end{entry}

\begin{entry}{外套}{5,10}{⼣、⼤}
  \begin{phonetics}{外套}{wai4 tao4}[][HSK 4]
    \definition[件,套]{s.}{casaco; jaqueta; paletó; sobretudo}
  \end{phonetics}
\end{entry}

\begin{entry}{外海}{5,10}{⼣、⽔}
  \begin{phonetics}{外海}{wai4hai3}
    \definition{s.}{mar aberto}
  \end{phonetics}
\end{entry}

\begin{entry}{外积}{5,10}{⼣、⽲}
  \begin{phonetics}{外积}{wai4ji1}
    \definition{s.}{produto exterior | (matemática) o produto vetorial de dois vetores}
  \end{phonetics}
\end{entry}

\begin{entry}{外资}{5,10}{⼣、⾙}
  \begin{phonetics}{外资}{wai4 zi1}[][HSK 6]
    \definition{s.}{capital estrangeiro (oposto a 内资); investimento estrangeiro; fundos estrangeiros; capital investido por países estrangeiros}
  \seealsoref{内资}{nei4 zi1}
  \end{phonetics}
\end{entry}

\begin{entry}{外部}{5,10}{⼣、⾢}
  \begin{phonetics}{外部}{wai4 bu4}[][HSK 6]
    \definition{s.}{fora; externo; fora de um certo intervalo | exterior; superfície}
  \end{phonetics}
\end{entry}

\begin{entry}{外婆}{5,11}{⼣、⼥}
  \begin{phonetics}{外婆}{wai4po2}
    \definition{s.}{avó materna}
  \end{phonetics}
\end{entry}

\begin{entry}{外插}{5,12}{⼣、⼿}
  \begin{phonetics}{外插}{wai4cha1}
    \definition{v.}{extrapolar | (computação) conectar (um dispositivo periférico, etc.)}
  \end{phonetics}
\end{entry}

\begin{entry}{外貌协会}{5,14,6,6}{⼣、⾘、⼗、⼈}
  \begin{phonetics}{外貌协会}{wai4mao4xie2hui4}
    \definition{s.}{``o clube da boa aparência'': pessoas que dão grande importância à aparência de uma pessoa}
  \seealsoref{外协}{wai4xie2}
  \end{phonetics}
\end{entry}

\begin{entry}{多}{6}{⼣}
  \begin{phonetics}{多}{duo1}[][HSK 1,2]
    \definition*{s.}{Sobrenome Duo}
    \definition{adj.}{grande quantidade (oposto de 少, 寡) | excessivo; desnecessário | excessivo; em demasia; indica um grande grau de diferença | mais do que o número correto ou necessário; em excesso}
    \definition{adv.}{acima de um valor especificado; e mais | em que medida; usado em frases interrogativas para indagar sobre grau ou quantidade, equivalente a 多么 | uma extensão não especificada; usado em frases exclamativas para expressar um alto grau, equivalente a 多么 | quase; significa que a maior parte do intervalo é assim | mais;  sobre; ímpar; usado depois de um quantificador para indicar uma fração}
    \definition{num.}{(após um número) ímpar}
    \definition{pref.}{multi- | poli-}
    \definition{v.}{ter (uma quantidade específica) a mais ou a mais (oposto a 少) | ter algo em abundância  | (em perguntas) até que ponto | (em exclamações) até que ponto | ter mais}
  \seealsoref{多么}{duo1me5}
  \seealsoref{寡}{gua3}
  \seealsoref{少}{shao3}
  \end{phonetics}
\end{entry}

\begin{entry}{多久}{6,3}{⼣、⼃}
  \begin{phonetics}{多久}{duo1 jiu3}[][HSK 2]
    \definition{pron.}{quanto tempo?; quanto tempo; perguntar quanto tempo leva}
  \end{phonetics}
\end{entry}

\begin{entry}{多么}{6,3}{⼣、⼃}
  \begin{phonetics}{多么}{duo1me5}[][HSK 2]
    \definition{adv.}{(em exclamações) como; o quê; em que medida; usado em frases exclamativas, indica um grau muito alto | em grau indeterminado; usado em frases declarativas, indica um grau mais profundo | como (usado em uma frase interrogativa para perguntar sobre grau ou número)}
  \end{phonetics}
\end{entry}

\begin{entry}{多大}{6,3}{⼣、⼤}
  \begin{phonetics}{多大}{duo1da4}
    \definition{adj.}{quantos anos? | que idade? | quão grande?}
  \end{phonetics}
\end{entry}

\begin{entry}{多云}{6,4}{⼣、⼆}
  \begin{phonetics}{多云}{duo1 yun2}[][HSK 2]
    \definition{adj.}{céu nublado; em meteorologia, refere-se a condições atmosféricas em que a cobertura de nuvens médias e baixas ocupa entre 40\% e 70\% da área do céu, ou a cobertura de nuvens altas ocupa entre 60\% e 100\% da área do céu}
  \end{phonetics}
\end{entry}

\begin{entry}{多少}{6,4}{⼣、⼩}
  \begin{phonetics}{多少}{duo1shao3}
    \definition{adv.}{um pouco; mais ou menos; até certo ponto}
    \definition{s.}{número; quantidade; volume}
  \end{phonetics}
  \begin{phonetics}{多少}{duo1shao5}[][HSK 1]
    \definition{adv.}{quantos?; quanto?; usado em perguntas para perguntar sobre quantidade | expressar uma quantidade ou número não especificado; quantidade indefinida}
  \end{phonetics}
\end{entry}

\begin{entry}{多方面}{6,4,9}{⼣、⽅、⾯}
  \begin{phonetics}{多方面}{duo1 fang1 mian4}[][HSK 6]
    \definition{adj.}{de muitas maneiras; todos os aspectos}
    \definition{s.}{multifacetado; multiaspecto}
  \end{phonetics}
\end{entry}

\begin{entry}{多半}{6,5}{⼣、⼗}
  \begin{phonetics}{多半}{duo1 ban4}[][HSK 6]
    \definition{adv.}{geralmente; mais frequentemente do que não}
    \definition{num.}{a maioria; a maior parte; mais da metade}
  \end{phonetics}
\end{entry}

\begin{entry}{多年}{6,6}{⼣、⼲}
  \begin{phonetics}{多年}{duo1 nian2}[][HSK 4]
    \definition{adv.}{por muitos anos; durante muitos anos}
  \end{phonetics}
\end{entry}

\begin{entry}{多次}{6,6}{⼣、⽋}
  \begin{phonetics}{多次}{duo1 ci4}[][HSK 4]
    \definition{adv.}{muitas vezes; de vez em quando; repetidamente; em muitas ocasiões}
  \end{phonetics}
\end{entry}

\begin{entry}{多咱}{6,9}{⼣、⼝}
  \begin{phonetics}{多咱}{duo1 zan5}
    \definition{adv.}{que horas?; quando?}
  \end{phonetics}
\end{entry}

\begin{entry}{多种}{6,9}{⼣、⽲}
  \begin{phonetics}{多种}{duo1 zhong3}[][HSK 4]
    \definition{adj.}{diverso; vários tipos de; múltiplo; diversificado}
  \end{phonetics}
\end{entry}

\begin{entry}{多重}{6,9}{⼣、⾥}
  \begin{phonetics}{多重}{duo1chong2}
    \definition{pref.}{multi (facetado, cultural, étnico, etc.)}
  \end{phonetics}
\end{entry}

\begin{entry}{多样}{6,10}{⼣、⽊}
  \begin{phonetics}{多样}{duo1 yang4}[][HSK 4]
    \definition{adj.}{diversos; variados; diversificado}
    \definition{s.}{diversidade}
  \end{phonetics}
\end{entry}

\begin{entry}{多媒体}{6,12,7}{⼣、⼥、⼈}
  \begin{phonetics}{多媒体}{duo1 mei2 ti3}[][HSK 6]
    \definition{s.}{multimídia; uma combinação de múltiplas mídias}
  \end{phonetics}
\end{entry}

\begin{entry}{多数}{6,13}{⼣、⽁}
  \begin{phonetics}{多数}{duo1 shu4}[][HSK 2]
    \definition{adj.}{maioria; a maioria; plural}
    \definition{pref.}{pluri-}
  \end{phonetics}
\end{entry}

\begin{entry}{夜}{8}{⼣}
  \begin{phonetics}{夜}{ye4}[][HSK 2]
    \definition{s.}{noite; tarde; noturno; o período do anoitecer ao amanhecer (em oposição a 日 ou 昼); em meteorologia, refere-se especificamente ao período das 20h do dia atual às 8h do dia seguinte}
  \seealsoref{日}{ri4}
  \seealsoref{昼}{zhou4}
  \end{phonetics}
\end{entry}

\begin{entry}{夜生活}{8,5,9}{⼣、⽣、⽔}
  \begin{phonetics}{夜生活}{ye4sheng1huo2}
    \definition{s.}{vida noturna}
  \end{phonetics}
\end{entry}

\begin{entry}{夜鸟}{8,5}{⼣、⿃}
  \begin{phonetics}{夜鸟}{ye4niao3}
    \definition{s.}{ave noturna}
  \end{phonetics}
\end{entry}

\begin{entry}{夜场}{8,6}{⼣、⼟}
  \begin{phonetics}{夜场}{ye4chang3}
    \definition{s.}{show noturno (em um teatro, etc.) | local de entretenimento noturno (bar, boate, discoteca, etc.)}
  \end{phonetics}
\end{entry}

\begin{entry}{夜里}{8,7}{⼣、⾥}
  \begin{phonetics}{夜里}{ye4li5}[][HSK 2]
    \definition{s.}{noturno; à noite; o período do anoitecer ao amanhecer}
  \end{phonetics}
\end{entry}

\begin{entry}{夜间}{8,7}{⼣、⾨}
  \begin{phonetics}{夜间}{ye4 jian1}[][HSK 5]
    \definition{s.}{noite; à noite; noturno; durante a noite}
  \end{phonetics}
\end{entry}

\begin{entry}{夜夜}{8,8}{⼣、⼣}
  \begin{phonetics}{夜夜}{ye4ye4}
    \definition{adv.}{toda noite}
  \end{phonetics}
\end{entry}

\begin{entry}{夜店}{8,8}{⼣、⼴}
  \begin{phonetics}{夜店}{ye4dian4}
    \definition{s.}{boate | \emph{nightclub}}
  \end{phonetics}
\end{entry}

\begin{entry}{夜晚}{8,11}{⼣、⽇}
  \begin{phonetics}{夜晚}{ye4wan3}
    \definition[个]{s.}{noite}
  \end{phonetics}
\end{entry}

\begin{entry}{夜深人静}{8,11,2,14}{⼣、⽔、⼈、⾭}
  \begin{phonetics}{夜深人静}{ye4shen1ren2jing4}
    \definition{expr.}{``Na calada da noite.''}
  \end{phonetics}
\end{entry}

\begin{entry}{夜幕}{8,13}{⼣、⼱}
  \begin{phonetics}{夜幕}{ye4mu4}
    \definition{s.}{cortina da noite}
  \end{phonetics}
\end{entry}

\begin{entry}{够}{11}{⼣}
  \begin{phonetics}{够}{gou4}[][HSK 2]
    \definition{adj.}{suficiente; adequado; apropriado; atingir e ultrapassar um determinado limite, difícil de suportar}
    \definition{adv.}{suficientemente; o suficiente (para atingir um determinado nível); indica que atingiu um determinado padrão ou nível elevado}
    \definition{v.}{alcançar (algo, esticando-se); (usando membros, etc.) esticar-se para alcançar ou tocar em locais de difícil acesso | atingir (um padrão ou nível); satisfazer ou atingir a quantidade, os padrões, etc. necessários}
  \end{phonetics}
\end{entry}

\begin{entry}{够不着}{11,4,11}{⼣、⼀、⽬}
  \begin{phonetics}{够不着}{gou4bu5zhao2}
    \definition{v.}{ser incapaz de alcançar}
  \end{phonetics}
\end{entry}

\begin{entry}{够本}{11,5}{⼣、⽊}
  \begin{phonetics}{够本}{gou4ben3}
    \definition{v.}{empatar | fazer valer o dinheiro}
  \end{phonetics}
\end{entry}

\begin{entry}{够呛}{11,7}{⼣、⼝}
  \begin{phonetics}{够呛}{gou4qiang4}
    \definition{adj.}{suficiente | terrível | insuportável | improvável}
  \end{phonetics}
\end{entry}

\begin{entry}{够味}{11,8}{⼣、⼝}
  \begin{phonetics}{够味}{gou4wei4}
    \definition{adj.}{excelente | na medida}
  \end{phonetics}
\end{entry}

\begin{entry}{够戗}{11,8}{⼣、⼽}
  \begin{phonetics}{够戗}{gou4qiang4}
    \variantof{够呛}
  \end{phonetics}
\end{entry}

\begin{entry}{够朋友}{11,8,4}{⼣、⽉、⼜}
  \begin{phonetics}{够朋友}{gou4peng2you5}
    \definition{v.}{ser um amigo verdadeiro}
  \end{phonetics}
\end{entry}

\begin{entry}{够格}{11,10}{⼣、⽊}
  \begin{phonetics}{够格}{gou4ge2}
    \definition{adj.}{apto | qualificado | apresentável}
  \end{phonetics}
\end{entry}

\begin{entry}{够得着}{11,11,11}{⼣、⼻、⽬}
  \begin{phonetics}{够得着}{gou4de5zhao2}
    \definition{v.}{estar à altura | alcançar}
  \end{phonetics}
\end{entry}

\begin{entry}{梦}{11}{⼣}
  \begin{phonetics}{梦}{meng4}[][HSK 4]
    \definition*{s.}{Sobrenome Meng}
    \definition[个,场]{s.}{sonho; atividade de representação no cérebro durante o sono}
    \definition{v.}{sonhar; ter um sonho}
  \end{phonetics}
\end{entry}

\begin{entry}{梦见}{11,4}{⼣、⾒}
  \begin{phonetics}{梦见}{meng4 jian4}[][HSK 4]
    \definition{v.}{sonhar; sonhar com; ver em um sonho}
  \end{phonetics}
\end{entry}

\begin{entry}{梦想}{11,13}{⼣、⼼}
  \begin{phonetics}{梦想}{meng4xiang3}[][HSK 4]
    \definition[个]{s.}{sonhar; esperança vã; sonho inalcançável}
    \definition{v.}{sonhar; sonhar com carinho; desejar ardentemente}
  \end{phonetics}
\end{entry}

%%%%% EOF %%%%%

