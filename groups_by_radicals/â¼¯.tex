%%%
%%% Radical "⼯"
%%%

\section*{Radical 48: ``⼯''}\addcontentsline{toc}{section}{Radical 48: ⼯}

\begin{Entry}{工}{3}{⼯}[Kangxi 48]
  \begin{Phonetics}{工}{gong1}
    \definition*{s.}{Sobrenome Gong}
    \definition{adj.}{fino; requintado; delicado}
    \definition{s.}{trabalhador; operário; artesão | trabalho; labor; trabalho produtivo | projeto; construção; refere-se à engenharia | indústria; refere-se à indústria | homem-dia; a quantidade de trabalho que um trabalhador faz em um dia | uma nota da escala em Gongchepu (工尺谱), correspondente a 3 na notação musical numerada | engenheiro; refere-se a engenheiros}
    \definition{v.}{ser versado em; ser bom em | trabalhar em; agora geralmente escrito como 功}
  \seealsoref{功}{gong1}
  \seealsoref{工尺谱}{gong1 che3 pu3}
  \end{Phonetics}
\end{Entry}

\begin{Entry}{工人}{3,2}{⼯、⼈}
  \begin{Phonetics}{工人}{gong1ren2}[][HSK 1]
    \definition[个,名]{s.}{trabalhador; operário; mão de obra; trabalhadores braçais que vivem do salário}
  \end{Phonetics}
\end{Entry}

\begin{Entry}{工厂}{3,2}{⼯、⼚}
  \begin{Phonetics}{工厂}{gong1chang3}[][HSK 3]
    \definition[个,家,座,间]{s.}{fábrica; moinho; planta; unidades que realizam atividades de produção industrial diretamente, geralmente incluindo diferentes oficinas}
  \end{Phonetics}
\end{Entry}

\begin{Entry}{工夫}{3,4}{⼯、⼤}
  \begin{Phonetics}{工夫}{gong1 fu1}
    \definition[个]{s.}{tempo | tempo livre; lazer}
  \end{Phonetics}
  \begin{Phonetics}{工夫}{gong1 fu5}[][HSK 3]
    \definition[个]{s.}{(um período de) tempo; o tempo ou energia gastos para realizar uma tarefa | tempo livre}
  \end{Phonetics}
\end{Entry}

\begin{Entry}{工尺谱}{3,4,14}{⼯、⼫、⾔}
  \begin{Phonetics}{工尺谱}{gong1 che3 pu3}
    \definition*{s.}{Gongchepu, notação musical tradicional chinesa}
    \definition{s.}{notação musical tradicional chinesa que usa caracteres chineses para representar notas musicais}
  \end{Phonetics}
\end{Entry}

\begin{Entry}{工艺}{3,4}{⼯、⾋}
  \begin{Phonetics}{工艺}{gong1 yi4}[][HSK 5]
    \definition{s.}{técnica; tecnologia; arte industrial; técnicas ou métodos de fabricação e processamento de produtos | artesanato; arte artesanal}
  \end{Phonetics}
\end{Entry}

\begin{Entry}{工艺品}{3,4,9}{⼯、⾋、⼝}
  \begin{Phonetics}{工艺品}{gong1 yi4 pin3}[][HSK 5]
    \definition[个,件]{s.}{trabalho manual; artesanato; habilidade manual; artigo artesanal; itens delicados produzidos com técnicas artesanais. Por exemplo, esculturas em jade, esmaltes Jingtailan, bordados, etc.}
  \end{Phonetics}
\end{Entry}

\begin{Entry}{工业}{3,5}{⼯、⼀}
  \begin{Phonetics}{工业}{gong1ye4}[][HSK 3]
    \definition{s.}{indústria; utilização de recursos naturais; fabricação de meios de produção; meios de subsistência; ou processamento de produtos agrícolas, produtos semiacabados, etc.}
  \end{Phonetics}
\end{Entry}

\begin{Entry}{工作}{3,7}{⼯、⼈}
  \begin{Phonetics}{工作}{gong1zuo4}[][HSK 1]
    \definition[份,个,分,项]{s.}{trabalho; emprego | dever; tarefa; negócio}
    \definition{v.}{trabalhar; operar (uma máquina); envolver-se em trabalho físico ou intelectual, também se refere de maneira geral a máquinas e ferramentas operadas por pessoas para realizar funções produtivas}
  \end{Phonetics}
\end{Entry}

\begin{Entry}{工作日}{3,7,4}{⼯、⼈、⽇}
  \begin{Phonetics}{工作日}{gong1 zuo4 ri4}[][HSK 5]
    \definition{s.}{dia de trabalho; dia útil; dias em que você deveria estar trabalhando de acordo com as regras | horas de trabalho por dia; horas do dia para fazer o trabalho necessário}
  \end{Phonetics}
\end{Entry}

\begin{Entry}{工具}{3,8}{⼯、⼋}
  \begin{Phonetics}{工具}{gong1ju4}[][HSK 3]
    \definition[个,件,套]{s.}{ferramenta; ferramentas utilizadas na produção| ferramenta; meio; instrumento; (metáfora) algo ou meio utilizado para atingir um determinado objetivo}
  \end{Phonetics}
\end{Entry}

\begin{Entry}{工资}{3,10}{⼯、⾙}
  \begin{Phonetics}{工资}{gong1zi1}[][HSK 3]
    \definition[份,笔,月,天]{s.}{pagamento; salário; remuneração; vencimentos; o pagamento em dinheiro ou em espécie feito ao trabalhador como remuneração pelo trabalho realizado}
  \end{Phonetics}
\end{Entry}

\begin{Entry}{工商}{3,11}{⼯、⼝}
  \begin{Phonetics}{工商}{gong1 shang1}[][HSK 6]
    \definition{s.}{indústria e comércio; um termo combinado para indústria e comércio}
  \end{Phonetics}
\end{Entry}

\begin{Entry}{工程}{3,12}{⼯、⽲}
  \begin{Phonetics}{工程}{gong1 cheng2}[][HSK 4]
    \definition[个,项]{s.}{projeto; programa; trabalhos que utilizam equipamentos grandes e complexos, como projetos de reconstrução urbana e projetos de cestas de alimentos, etc. | engenharia; departamentos de produção e manufatura usam equipamentos grandes e complexos para realizar seu trabalho}
  \end{Phonetics}
\end{Entry}

\begin{Entry}{工程师}{3,12,6}{⼯、⽲、⼱}
  \begin{Phonetics}{工程师}{gong1cheng2shi1}[][HSK 3]
    \definition[个,位,名,些]{s.}{engenheiro; um dos cargos técnicos é o de especialista capaz de realizar de forma independente o projeto e a execução de uma tarefa técnica específica}
  \end{Phonetics}
\end{Entry}

\begin{Entry}{工龄}{3,13}{⼯、⿒}
  \begin{Phonetics}{工龄}{gong1ling2}
    \definition{s.}{tempo de serviço | senioridade}
  \end{Phonetics}
\end{Entry}

\begin{Entry}{巨}{4}{⼯}
  \begin{Phonetics}{巨}{ju4}
    \definition*{s.}{Sobrenome Ju}
    \definition{adj.}{enorme; tremendo; gigantesco}
  \end{Phonetics}
\end{Entry}

\begin{Entry}{巨大}{4,3}{⼯、⼤}
  \begin{Phonetics}{巨大}{ju4da4}[][HSK 4]
    \definition{adj.}{enorme; tremendo; enorme; gigantesco; imenso}
  \end{Phonetics}
\end{Entry}

\begin{Entry}{左}{5}{⼯}
  \begin{Phonetics}{左}{zuo3}[][HSK 1]
    \definition*{s.}{Sobrenome Zuo}
    \definition{adj.}{estranho; herético; não ortodoxo | errado; incorreto | diferente; contrário; oposto | progressista; revolucionário; politicamente e ideologicamente progressista; radical}
    \definition{s.}{a esquerda; o lado esquerdo | leste; na antiguidade, referia-se especificamente à direção leste (com base na orientação para o sul) | a esquerda; ala esquerda; refere-se a uma posição inferior (na antiguidade, a direita era considerada superior e a esquerda, inferior)}
    \definition{v.}{assistir; auxiliar}
  \end{Phonetics}
\end{Entry}

\begin{Entry}{左右}{5,5}{⼯、⼝}
  \begin{Phonetics}{左右}{zuo3you4}[][HSK 3]
    \definition{s.}{os lados esquerdo e direito; esquerda e direita, também indicam os arredores | atendentes; acompanhantes; as pessoas que o acompanham | aproximadamente; mais ou menos; por aí; usado após números para indicar uma estimativa, com o mesmo significado de 上下}
    \definition{v.}{controlar; manipular; influenciar; dominar}
  \seealsoref{上下}{shang4 xia4}
  \end{Phonetics}
\end{Entry}

\begin{Entry}{左边}{5,5}{⼯、⾡}
  \begin{Phonetics}{左边}{zuo3bian5}[][HSK 1]
    \definition{s.}{esquerda; o lado esquerdo}
  \end{Phonetics}
\end{Entry}

\begin{Entry}{左派}{5,9}{⼯、⽔}
  \begin{Phonetics}{左派}{zuo3pai4}
    \definition{s.}{(política) esquerda | esquerdista}
  \end{Phonetics}
\end{Entry}

\begin{Entry}{左面}{5,9}{⼯、⾯}
  \begin{Phonetics}{左面}{zuo3mian4}
    \definition{s.}{esquerda | lado esquerdo}
  \end{Phonetics}
\end{Entry}

\begin{Entry}{左倾}{5,10}{⼯、⼈}
  \begin{Phonetics}{左倾}{zuo3qing1}
    \definition{s.}{esquerdista | progressivo}
  \end{Phonetics}
\end{Entry}

\begin{Entry}{左袒}{5,10}{⼯、⾐}
  \begin{Phonetics}{左袒}{zuo3tan3}
    \definition{v.}{ser tendencioso | ser parcial para | favorecer um lado | tomar partido com}
  \end{Phonetics}
\end{Entry}

\begin{Entry}{左舷}{5,11}{⼯、⾈}
  \begin{Phonetics}{左舷}{zuo3xian2}
    \definition{s.}{porto (lado de um navio)}
  \end{Phonetics}
\end{Entry}

\begin{Entry}{左翼}{5,17}{⼯、⽻}
  \begin{Phonetics}{左翼}{zuo3yi4}
    \definition{s.}{esquerda (política)}
  \end{Phonetics}
\end{Entry}

\begin{Entry}{巧}{5}{⼯}
  \begin{Phonetics}{巧}{qiao3}[][HSK 3]
    \definition{adj.}{habilidoso; engenhoso; esperto | oportuno; coincidente; fortuito | astuto; enganoso; enganador; traiçoeiro; ardiloso | (de mão, língua) hábil; loquaz}
    \definition{s.}{(tecnologia, artesanato) habilidade; destreza}
  \end{Phonetics}
\end{Entry}

\begin{Entry}{巧合}{5,6}{⼯、⼝}
  \begin{Phonetics}{巧合}{qiao3he2}
    \definition{s.}{coincidência; (coisas) coincidentes ou idênticas}
  \end{Phonetics}
\end{Entry}

\begin{Entry}{巧克力}{5,7,2}{⼯、⼗、⼒}
  \begin{Phonetics}{巧克力}{qiao3ke4li4}[][HSK 4]
    \definition[块,颗,盒,包]{s.}{Empréstimo linguístico: chocolate; alimentos feitos com cacau em pó como principal matéria-prima, açúcar e especiarias}
  \end{Phonetics}
\end{Entry}

\begin{Entry}{巧妙}{5,7}{⼯、⼥}
  \begin{Phonetics}{巧妙}{qiao3miao4}[][HSK 6]
    \definition{adj.}{inteligente; engenhoso; (método ou técnica, etc.) inteligente, além do comum}
  \end{Phonetics}
\end{Entry}

\begin{Entry}{巩}{6}{⼯}
  \begin{Phonetics}{巩}{gong3}
    \definition*{s.}{Sobrenome Gong}
    \definition{s.}{seguro | sólido}
    \definition{v.}{consolidar}
  \end{Phonetics}
\end{Entry}

\begin{Entry}{巩固}{6,8}{⼯、⼞}
  \begin{Phonetics}{巩固}{gong3gu4}[][HSK 6]
    \definition{adj.}{sólido; estável; consolidado; não facilmente abalado (usado principalmente para coisas abstratas)}
    \definition{v.}{consolidar}
  \end{Phonetics}
\end{Entry}

\begin{Entry}{差}{9}{⼯}
  \begin{Phonetics}{差}{cha1}
    \definition{adj.}{diferente; diferente ou inconsistente com um determinado padrão}
    \definition{adv.}{ligeiramente; comparativamente; um pouco}
    \definition{s.}{diferença; resto após a subtração de dois números | erro; engano}
  \end{Phonetics}
  \begin{Phonetics}{差}{cha4}[][HSK 1]
    \definition{adj.}{não está de acordo com o padrão; pobre; ruim; inferior | errado; incorreto | mesmo significado de 差 \dpy{cha1}}
    \definition{v.}{faltar}
  \end{Phonetics}
  \begin{Phonetics}{差}{chai1}
    \definition{s.}{tarefa; trabalho; ser enviado para fazer algo; deveres oficiais; posição | corvéia; mensageiro ou oficial de justiça em um yamen feudal; (velho) refere-se a pessoas que são enviadas para fazer coisas}
    \definition{v.}{enviar uma mensagem; despachar; fnviar (para fazer algo)}
  \end{Phonetics}
\end{Entry}

\begin{Entry}{差(一)点儿}{9,1,9,2}{⼯、⼀、⽕、⼉}
  \begin{Phonetics}{差(一)点儿}{cha1yi4dian3r5}[][HSK 5]
    \definition{adj.}{não é bom o suficiente; ligeiramente inferior a; não está à altura da marca;  (qualidade, tecnologia, desempenho, etc.) ligeiramente inferior}
    \definition{adv.}{quase; à beira de; praticamente; aproximadamente; significa que algo está perto de ser alcançado, mas não foi alcançado, ou algo foi alcançado, mas mal foi alcançado}
  \end{Phonetics}
\end{Entry}

\begin{Entry}{差不多}{9,4,6}{⼯、⼀、⼣}
  \begin{Phonetics}{差不多}{cha4bu5duo1}[][HSK 2]
    \definition{adj.}{semelhante; aproximadamente igual | não muito longe; quase certo (suficiente); basicamente, próximo dos padrões e requisitos; normal | prestes a (terminar; acabar); descreve que (algo) está quase acabando; (uma tarefa) está quase concluída}
    \definition{adv.}{quase; perto; indica proximidade}
  \end{Phonetics}
\end{Entry}

\begin{Entry}{差异}{9,6}{⼯、⼶}
  \begin{Phonetics}{差异}{cha1 yi4}[][HSK 6]
    \definition{s.}{diferença; divergência; discrepância}
  \end{Phonetics}
\end{Entry}

\begin{Entry}{差别}{9,7}{⼯、⼑}
  \begin{Phonetics}{差别}{cha1bie2}[][HSK 5]
    \definition{s.}{diferença; disparidade; dissimilaridade; distinção; não semelhança; diferenças na forma ou no conteúdo}
  \end{Phonetics}
\end{Entry}

\begin{Entry}{差点儿}{9,9,2}{⼯、⽕、⼉}
  \begin{Phonetics}{差点儿}{cha4dian3r5}
    \definition{adv.}{por pouco | por um triz | quase}
  \end{Phonetics}
\end{Entry}

\begin{Entry}{差距}{9,11}{⼯、⾜}
  \begin{Phonetics}{差距}{cha1ju4}[][HSK 5]
    \definition[个,些,段]{s.}{lacuna; disparidade; discrepância; diferença; grau de diferença entre as coisas, especialmente em termos de distância de algum padrão.}
  \end{Phonetics}
\end{Entry}

%%%%% EOF %%%%%

