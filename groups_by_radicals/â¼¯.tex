%%%
%%% Radical "⼯"
%%%

\section*{Radical 48: ``⼯''}\addcontentsline{toc}{section}{Radical 48: ⼯}

\begin{entry}{工}{3}{⼯}[Kangxi 48]
  \begin{phonetics}{工}{gong1}
    \definition{s.}{trabalho | trabalhador | habilidade | profissão | comércio | ofício}
  \end{phonetics}
\end{entry}

\begin{entry}{工人}{3,2}{⼯、⼈}
  \begin{phonetics}{工人}{gong1ren2}[][HSK 1]
    \definition{s.}{trabalhador | operário | mão de obra de fábrica}
  \end{phonetics}
\end{entry}

\begin{entry}{工厂}{3,2}{⼯、⼚}
  \begin{phonetics}{工厂}{gong1chang3}[][HSK 3]
    \definition[家,座,个]{s.}{fábrica; moinho; planta; obras}
  \end{phonetics}
\end{entry}

\begin{entry}{工夫}{3,4}{⼯、⼤}
  \begin{phonetics}{工夫}{gong1 fu1}
    \definition[个]{s.}{tempo | tempo livre; lazer}
  \end{phonetics}
  \begin{phonetics}{工夫}{gong1 fu5}[][HSK 3]
    \definition{s.}{(um período de) tempo | tempo livre}
  \end{phonetics}
\end{entry}

\begin{entry}{工尺谱}{3,4,14}{⼯、⼫、⾔}
  \begin{phonetics}{工尺谱}{gong1 che3 pu3}
    \definition{s.}{notação musical tradicional chinesa que usa caracteres chineses para representar notas musicais}
  \end{phonetics}
\end{entry}

\begin{entry}{工艺}{3,4}{⼯、⾋}
  \begin{phonetics}{工艺}{gong1yi4}
    \definition{s.}{artesanato}
  \end{phonetics}
\end{entry}

\begin{entry}{工艺品}{3,4,9}{⼯、⾋、⼝}
  \begin{phonetics}{工艺品}{gong1yi4pin3}
    \definition[个]{s.}{artigo de artesanato | trabalho manual}
  \end{phonetics}
\end{entry}

\begin{entry}{工业}{3,5}{⼯、⼀}
  \begin{phonetics}{工业}{gong1ye4}[][HSK 3]
    \definition{s.}{indústria}
  \end{phonetics}
\end{entry}

\begin{entry}{工作}{3,7}{⼯、⼈}
  \begin{phonetics}{工作}{gong1zuo4}[][HSK 1]
    \definition[个,份,项]{s.}{trabalho | tarefa}
    \definition{v.}{trabalhar | operar (uma máquina)}
  \end{phonetics}
\end{entry}

\begin{entry}{工具}{3,8}{⼯、⼋}
  \begin{phonetics}{工具}{gong1ju4}[][HSK 3]
    \definition[个]{s.}{ferramenta; implemento | ferramenta; meio; instrumento}
  \end{phonetics}
\end{entry}

\begin{entry}{工资}{3,10}{⼯、⾙}
  \begin{phonetics}{工资}{gong1zi1}[][HSK 3]
    \definition[份,个,年,月,天]{s.}{pagamento; salário}
  \end{phonetics}
\end{entry}

\begin{entry}{工程}{3,12}{⼯、⽲}
  \begin{phonetics}{工程}{gong1 cheng2}[][HSK 4]
    \definition[个,项]{s.}{projeto; programa; trabalhos que utilizam equipamentos grandes e complexos, como projetos de reconstrução urbana e projetos de cestas de alimentos, etc. | engenharia; departamentos de produção e manufatura usam equipamentos grandes e complexos para realizar seu trabalho}
  \end{phonetics}
\end{entry}

\begin{entry}{工程师}{3,12,6}{⼯、⽲、⼱}
  \begin{phonetics}{工程师}{gong1cheng2shi1}[][HSK 3]
    \definition[个,名]{s.}{engenheiro}
  \end{phonetics}
\end{entry}

\begin{entry}{工龄}{3,13}{⼯、⿒}
  \begin{phonetics}{工龄}{gong1ling2}
    \definition{s.}{tempo de serviço | senioridade}
  \end{phonetics}
\end{entry}

\begin{entry}{巨大}{4,3}{⼯、⼤}
  \begin{phonetics}{巨大}{ju4da4}[][HSK 4]
    \definition{adj.}{enorme; tremendo; enorme; gigantesco; imenso}
  \end{phonetics}
\end{entry}

\begin{entry}{左}{5}{⼯}
  \begin{phonetics}{左}{zuo3}[][HSK 1]
    \definition*{s.}{sobrenome Zuo}
    \definition{s.}{esquerda}
  \end{phonetics}
\end{entry}

\begin{entry}{左右}{5,5}{⼯、⼝}
  \begin{phonetics}{左右}{zuo3you4}[][HSK 3]
    \definition{adv.}{aproximadamente; ou mais ou menos; por aí; usado depois de um número para indicar um número aproximado, o mesmo que ``上下''}
    \definition{s.}{os lados esquerdo e direito; esquerda e direita, também significa circundar
atendentes; pessoas que te seguem}
    \definition{v.}{controlar; manipular; influenciar}
  \end{phonetics}
\end{entry}

\begin{entry}{左边}{5,5}{⼯、⾡}
  \begin{phonetics}{左边}{zuo3bian5}[][HSK 1]
    \definition{s.}{esquerda | lado esquerdo}
  \end{phonetics}
\end{entry}

\begin{entry}{左派}{5,9}{⼯、⽔}
  \begin{phonetics}{左派}{zuo3pai4}
    \definition{s.}{(política) esquerda | esquerdista}
  \end{phonetics}
\end{entry}

\begin{entry}{左面}{5,9}{⼯、⾯}
  \begin{phonetics}{左面}{zuo3mian4}
    \definition{s.}{esquerda | lado esquerdo}
  \end{phonetics}
\end{entry}

\begin{entry}{左倾}{5,10}{⼯、⼈}
  \begin{phonetics}{左倾}{zuo3qing1}
    \definition{s.}{esquerdista | progressivo}
  \end{phonetics}
\end{entry}

\begin{entry}{左袒}{5,10}{⼯、⾐}
  \begin{phonetics}{左袒}{zuo3tan3}
    \definition{v.}{ser tendencioso | ser parcial para | favorecer um lado | tomar partido com}
  \end{phonetics}
\end{entry}

\begin{entry}{左舷}{5,11}{⼯、⾈}
  \begin{phonetics}{左舷}{zuo3xian2}
    \definition{s.}{porto (lado de um navio)}
  \end{phonetics}
\end{entry}

\begin{entry}{左翼}{5,17}{⼯、⽻}
  \begin{phonetics}{左翼}{zuo3yi4}
    \definition{s.}{esquerda (política)}
  \end{phonetics}
\end{entry}

\begin{entry}{巧}{5}{⼯}
  \begin{phonetics}{巧}{qiao3}[][HSK 3]
    \definition{adj.}{habilidoso; engenhoso; esperto | oportuno; coincidente; fortuito | astuto; enganoso; enganador; traiçoeiro; ardiloso}
  \end{phonetics}
\end{entry}

\begin{entry}{巧合}{5,6}{⼯、⼝}
  \begin{phonetics}{巧合}{qiao3he2}
    \definition{s.}{coincidência}
    \definition{v.}{coincidir}
  \end{phonetics}
\end{entry}

\begin{entry}{巧克力}{5,7,2}{⼯、⼗、⼒}
  \begin{phonetics}{巧克力}{qiao3ke4li4}[][HSK 4]
    \definition[块]{s.}{(empréstimo linguístico) chocolate}
  \end{phonetics}
\end{entry}

\begin{entry}{差}{9}{⼯}
  \begin{phonetics}{差}{cha4}[][HSK 1]
    \definition{adv.}{ligeiramente | comparativamente | um pouco}
    \definition{s.}{differença | dissimilaridade | engano | equívoco}
  \end{phonetics}
\end{entry}

\begin{entry}{差(一)点儿}{9,1,9,2}{⼯、⼀、⽕、⼉}
  \begin{phonetics}{差(一)点儿}{cha1yi4dian3r5}[][HSK 5]
    \definition{adv.}{quase; à beira de; praticamente; aproximadamente; significa que algo está perto de ser alcançado, mas não foi alcançado, ou algo foi alcançado, mas mal foi alcançado}
  \end{phonetics}
\end{entry}

\begin{entry}{差不多}{9,4,6}{⼯、⼀、⼣}
  \begin{phonetics}{差不多}{cha4bu5duo1}[][HSK 2]
    \definition{adj.}{mais ou menos}
    \definition{adv.}{quase perto}
  \end{phonetics}
\end{entry}

\begin{entry}{差别}{9,7}{⼯、⼑}
  \begin{phonetics}{差别}{cha1bie2}[][HSK 5]
    \definition{s.}{diferença; disparidade; dissimilaridade; distinção; não semelhança; diferenças na forma ou no conteúdo}
  \end{phonetics}
\end{entry}

\begin{entry}{差点儿}{9,9,2}{⼯、⽕、⼉}
  \begin{phonetics}{差点儿}{cha4dian3r5}
    \definition{adv.}{por pouco | por um triz | quase}
  \end{phonetics}
\end{entry}

\begin{entry}{差距}{9,11}{⼯、⾜}
  \begin{phonetics}{差距}{cha1ju4}[][HSK 5]
    \definition[个,些,段]{s.}{lacuna; disparidade; discrepância; diferença; grau de diferença entre as coisas, especialmente em termos de distância de algum padrão.}
  \end{phonetics}
\end{entry}

%%%%% EOF %%%%%

