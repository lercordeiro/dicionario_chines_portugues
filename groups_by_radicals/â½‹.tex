%%%
%%% Radical "⽋"
%%%

\section*{Radical 76: ``⽋''}\addcontentsline{toc}{section}{Radical 76: ⽋}

\begin{entry}{欠}{4}{⽋}
  \begin{phonetics}{欠}{qian4}[][HSK 5]
    \definition{v.}{bocejar | levantar ligeiramente (uma parte do corpo) | estar em dívida; estar atrasado com; não devolver o que pediu emprestado a outra pessoa, ou não dar o que deveria ter dado a outra pessoa | faltar; não ser suficiente}
  \end{phonetics}
\end{entry}

\begin{entry}{次}{6}{⽋}
  \begin{phonetics}{次}{ci4}[][HSK 1,4]
    \definition*{s.}{sobrenome Ci}
    \definition{adj.}{de segunda categoria; de qualidade inferior}
    \definition{clas.}{usado para coisas ou ações que podem ser repetidas}
    \definition{num.}{segundo; próximo}
    \definition{pref.}{(química) hipo-, radical ácido ou composto contendo dois átomos de oxigênio a menos}
    \definition{s.}{ordem; sequência; classificação | local de parada em uma viagem; escala}
  \end{phonetics}
\end{entry}

\begin{entry}{欢}{6}{⽋}
  \begin{phonetics}{欢}{huan1}
    \definition*{s.}{Huan}
    \definition{adj.}{alegre; feliz; jubilante | vigoroso; energético; em pleno andamento; com grande impulso}
    \definition{s.}{amante; querida; um apelido usado por mulheres nos tempos antigos para se referir aos seus amantes; agora, geralmente se refere a alguém de quem você gosta ou com quem tem um relacionamento romântico}
  \end{phonetics}
\end{entry}

\begin{entry}{欢乐}{6,5}{⽋、⼃}
  \begin{phonetics}{欢乐}{huan1le4}[][HSK 3]
    \definition{adj.}{feliz; alegre; felicidade (geralmente coletiva)}
  \end{phonetics}
\end{entry}

\begin{entry}{欢快}{6,7}{⽋、⼼}
  \begin{phonetics}{欢快}{huan1kuai4}
    \definition{adj.}{feliz e sem ansiedade | vívido}
  \end{phonetics}
\end{entry}

\begin{entry}{欢迎}{6,7}{⽋、⾡}
  \begin{phonetics}{欢迎}{huan1ying2}[][HSK 2]
    \definition{adj.}{bem-vindo}
    \definition{v.}{dar as boas-vindas; cumprimentar; receber com alegria | dar as boas-vindas; receber favoravelmente (bem)}
  \end{phonetics}
\end{entry}

\begin{entry}{欣赏}{8,12}{⽋、⾙}
  \begin{phonetics}{欣赏}{xin1shang3}[][HSK 5]
    \definition{v.}{apreciar; admirar; valorizar; apreciar as coisas boas e descubrir o prazer que elas proporcionam | apreciar; gostar; considerar bom}
  \end{phonetics}
\end{entry}

\begin{entry}{欧}{8}{⽋}
  \begin{phonetics}{欧}{ou1}
    \definition*{s.}{sobrenome Ou | Europa, abreviação de 欧洲}
  \seealsoref{欧洲}{ou1zhou1}
  \end{phonetics}
\end{entry}

\begin{entry}{欧洲}{8,9}{⽋、⽔}
  \begin{phonetics}{欧洲}{ou1zhou1}
    \definition*{s.}{Europa}
  \end{phonetics}
\end{entry}

\begin{entry}{欧洲人}{8,9,2}{⽋、⽔、⼈}
  \begin{phonetics}{欧洲人}{ou1zhou1ren2}
    \definition{s.}{europeu | pessoa ou povo da Europa}
  \end{phonetics}
\end{entry}

\begin{entry}{欧洲共同体}{8,9,6,6,7}{⽋、⽔、⼋、⼝、⼈}
  \begin{phonetics}{欧洲共同体}{ou1zhou1 gong4tong2ti3}
    \definition*{s.}{Comunidade Europeia}
  \end{phonetics}
\end{entry}

\begin{entry}{欧盟}{8,13}{⽋、⽫}
  \begin{phonetics}{欧盟}{ou1meng2}
    \definition*{s.}{Uniáo Europeia}
  \end{phonetics}
\end{entry}

\begin{entry}{欱}{10}{⽋}
  \begin{phonetics}{欱}{he1}
    \definition{v.}{beber | beber bebida alcoólica}
    \variantof{喝}
  \end{phonetics}
\end{entry}

\begin{entry}{欲}{11}{⽋}
  \begin{phonetics}{欲}{yu4}
    \definition{adj.}{desejo | apetite | paixão | luxúria | ganância}
    \definition{v.}{desejar}
  \end{phonetics}
\end{entry}

\begin{entry}{款}{12}{⽋}
  \begin{phonetics}{款}{kuan3}
    \definition{clas.}{para versões ou modelos (de um produto)}
    \definition[笔,个]{s.}{montante de dinheiro | fundos | parágrafo | seção}
  \end{phonetics}
\end{entry}

\begin{entry}{歇}{13}{⽋}
  \begin{phonetics}{歇}{xie1}[][HSK 5]
    \definition*{s.}{sobrenome Xie}
    \definition{s.}{um pouco de tempo}
    \definition{v.}{descansar; fazer uma pausa | parar (o trabalho); encerrar o expediente | dormir; ir para a cama}
  \end{phonetics}
\end{entry}

\begin{entry}{歌}{14}{⽋}
  \begin{phonetics}{歌}{ge1}[][HSK 1]
    \definition[首,支,段]{s.}{canção; poesia cantável}
    \definition{v.}{cantar; entoar | louvar; exaltar; cantar louvores a}
  \end{phonetics}
\end{entry}

\begin{entry}{歌手}{14,4}{⽋、⼿}
  \begin{phonetics}{歌手}{ge1 shou3}[][HSK 3]
    \definition[个,位,名]{s.}{cantor; vocalista; pessoa com talento para cantar}
  \end{phonetics}
\end{entry}

\begin{entry}{歌曲}{14,6}{⽋、⽈}
  \begin{phonetics}{歌曲}{ge1 qu3}[][HSK 5]
    \definition{s.}{música; obra para as pessoas cantarem, uma combinação de poesia e música}
  \end{phonetics}
\end{entry}

\begin{entry}{歌声}{14,7}{⽋、⼠}
  \begin{phonetics}{歌声}{ge1 sheng1}[][HSK 3]
    \definition{s.}{canto; voz cantada; som do canto}
  \end{phonetics}
\end{entry}

\begin{entry}{歌迷}{14,9}{⽋、⾡}
  \begin{phonetics}{歌迷}{ge1 mi2}
    \definition{s.}{fã de um cantor; pessoas que gostam de ouvir música ou cantar e ficam fascinadas por isso}
  \end{phonetics}
\end{entry}

%%%%% EOF %%%%%

