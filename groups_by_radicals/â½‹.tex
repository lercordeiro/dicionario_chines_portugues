%%%
%%% Radical "⽋"
%%%

\section*{Radical 76: ``⽋''}\addcontentsline{toc}{section}{Radical 76: ⽋}

\begin{Entry}{欠}{4}{⽋}[Kangxi 76]
  \begin{Phonetics}{欠}{qian4}[][HSK 5]
    \definition{v.}{bocejar | levantar ligeiramente (uma parte do corpo) | estar em dívida; estar atrasado com; não devolver o que pediu emprestado a outra pessoa, ou não dar o que deveria ter dado a outra pessoa | faltar; não ser suficiente}
  \end{Phonetics}
\end{Entry}

\begin{Entry}{次}{6}{⽋}
  \begin{Phonetics}{次}{ci4}[][HSK 1,4]
    \definition*{s.}{Sobrenome Ci}
    \definition{adj.}{de segunda categoria; de qualidade inferior}
    \definition{clas.}{usado para coisas ou ações que podem ser repetidas}
    \definition{num.}{segundo; próximo}
    \definition{pref.}{(química) hipo-, radical ácido ou composto contendo dois átomos de oxigênio a menos}
    \definition{s.}{ordem; sequência; classificação | local de parada em uma viagem; escala}
  \end{Phonetics}
\end{Entry}

\begin{Entry}{次日}{6,4}{⽋、⽇}
  \begin{Phonetics}{次日}{ci4ri4}[][HSK 7-9]
    \definition{s.}{dia seguinte; amanhã}
  \end{Phonetics}
\end{Entry}

\begin{Entry}{次数}{6,13}{⽋、⽁}
  \begin{Phonetics}{次数}{ci4 shu4}[][HSK 6]
    \definition{s.}{frequência; número de vezes; o número de vezes que uma ação ou evento é repetido}
  \end{Phonetics}
\end{Entry}

\begin{Entry}{欢}{6}{⽋}
  \begin{Phonetics}{欢}{huan1}
    \definition*{s.}{Sobrenome Huan}
    \definition{adj.}{alegre; feliz; jubilante | vigoroso; energético; em pleno andamento; com grande impulso}
    \definition{s.}{amante; querida; um apelido usado por mulheres nos tempos antigos para se referir aos seus amantes; agora, geralmente se refere a alguém de quem você gosta ou com quem tem um relacionamento romântico}
  \end{Phonetics}
\end{Entry}

\begin{Entry}{欢乐}{6,5}{⽋、⼃}
  \begin{Phonetics}{欢乐}{huan1le4}[][HSK 3]
    \definition{adj.}{feliz; alegre; felicidade (geralmente coletiva)}
  \end{Phonetics}
\end{Entry}

\begin{Entry}{欢声笑语}{6,7,10,9}{⽋、⼠、⽵、⾔}
  \begin{Phonetics}{欢声笑语}{huan1sheng1-xiao4yu3}[][HSK 7-9]
    \definition{expr.}{risos felizes e vozes alegres}
  \end{Phonetics}
\end{Entry}

\begin{Entry}{欢快}{6,7}{⽋、⼼}
  \begin{Phonetics}{欢快}{huan1kuai4}[][HSK 7-9]
    \definition{adj.}{alegre; animado; alegre e despreocupado; feliz e alegre}
  \end{Phonetics}
\end{Entry}

\begin{Entry}{欢迎}{6,7}{⽋、⾡}
  \begin{Phonetics}{欢迎}{huan1ying2}[][HSK 2]
    \definition{adj.}{bem-vindo}
    \definition{v.}{dar as boas-vindas; cumprimentar; receber com alegria | dar as boas-vindas; receber favoravelmente (bem)}
  \end{Phonetics}
\end{Entry}

\begin{Entry}{欢呼}{6,8}{⽋、⼝}
  \begin{Phonetics}{欢呼}{huan1hu1}[][HSK 7-9]
    \definition{v.}{saudar; aplaudir; aclamar; dar vivas}
  \end{Phonetics}
\end{Entry}

\begin{Entry}{欢聚}{6,14}{⽋、⽿}
  \begin{Phonetics}{欢聚}{huan1ju4}[][HSK 7-9]
    \definition{s.}{celebração | festa}
    \definition{v.}{desfrutar de uma reunião feliz; reunir-se alegremente | celebrar | reunir-se socialmente}
  \end{Phonetics}
\end{Entry}

\begin{Entry}{欣}{8}{⽋}
  \begin{Phonetics}{欣}{xin1}
    \definition*{s.}{Sobrenome Xin}
    \definition{adj.}{alegre; feliz; contente}
  \end{Phonetics}
\end{Entry}

\begin{Entry}{欣赏}{8,12}{⽋、⾙}
  \begin{Phonetics}{欣赏}{xin1shang3}[][HSK 5]
    \definition{v.}{apreciar; admirar; valorizar; apreciar as coisas boas e descubrir o prazer que elas proporcionam | apreciar; gostar; considerar bom}
  \end{Phonetics}
\end{Entry}

\begin{Entry}{欧}{8}{⽋}
  \begin{Phonetics}{欧}{ou1}
    \definition*{s.}{Europa, abreviação de 欧洲 | Sobrenome Ou}
  \seealsoref{欧洲}{ou1zhou1}
  \end{Phonetics}
\end{Entry}

\begin{Entry}{欧阳询}{8,6,8}{⽋、⾩、⾔}
  \begin{Phonetics}{欧阳询}{ou1yang2 xun2}
    \definition*{s.}{Ouyang Xun (557-641), um dos quatro grandes calígrafos do início da dinastia Tang, 唐初四大家}
  \seealsoref{唐初四大家}{tang2 chu1 si4 da4jia1}
  \end{Phonetics}
\end{Entry}

\begin{Entry}{欧洲}{8,9}{⽋、⽔}
  \begin{Phonetics}{欧洲}{ou1zhou1}
    \definition*{s.}{Europa}
  \end{Phonetics}
\end{Entry}

\begin{Entry}{欧洲人}{8,9,2}{⽋、⽔、⼈}
  \begin{Phonetics}{欧洲人}{ou1zhou1ren2}
    \definition{s.}{europeu | pessoa ou povo da Europa}
  \end{Phonetics}
\end{Entry}

\begin{Entry}{欧洲共同体}{8,9,6,6,7}{⽋、⽔、⼋、⼝、⼈}
  \begin{Phonetics}{欧洲共同体}{ou1zhou1 gong4tong2ti3}
    \definition*{s.}{Comunidade Europeia}
  \end{Phonetics}
\end{Entry}

\begin{Entry}{欧盟}{8,13}{⽋、⽫}
  \begin{Phonetics}{欧盟}{ou1meng2}
    \definition*{s.}{União Europeia (EU)}
  \end{Phonetics}
\end{Entry}

\begin{Entry}{欱}{10}{⽋}
  \begin{Phonetics}{欱}{he1}
    \definition{v.}{beber | beber bebida alcoólica}
    \variantof{喝}
  \end{Phonetics}
\end{Entry}

\begin{Entry}{欲}{11}{⽋}
  \begin{Phonetics}{欲}{yu4}
    \definition{adj.}{desejo | apetite | paixão | luxúria | ganância}
    \definition{v.}{desejar}
  \end{Phonetics}
\end{Entry}

\begin{Entry}{欺}{12}{⽋}
  \begin{Phonetics}{欺}{qi1}
    \definition{v.}{enganar; trapacear | intimidar; tirar vantagem de alguém; tirar vantagem da fraqueza de (alguém, etc.)}
  \end{Phonetics}
\end{Entry}

\begin{Entry}{欺负}{12,6}{⽋、⾙}
  \begin{Phonetics}{欺负}{qi1fu5}[][HSK 6]
    \definition{v.}{violar, oprimir ou insultar com meios irracionais; \emph{bully}}
  \end{Phonetics}
\end{Entry}

\begin{Entry}{款}{12}{⽋}
  \begin{Phonetics}{款}{kuan3}
    \definition{clas.}{para versões ou modelos (de um produto)}
    \definition[笔,个]{s.}{montante de dinheiro | fundos | parágrafo | seção}
  \end{Phonetics}
\end{Entry}

\begin{Entry}{歇}{13}{⽋}
  \begin{Phonetics}{歇}{xie1}[][HSK 5]
    \definition*{s.}{Sobrenome Xie}
    \definition{s.}{um pouco de tempo}
    \definition{v.}{descansar; fazer uma pausa | parar (o trabalho); encerrar o expediente | dormir; ir para a cama}
  \end{Phonetics}
\end{Entry}

\begin{Entry}{歌}{14}{⽋}
  \begin{Phonetics}{歌}{ge1}[][HSK 1]
    \definition[首,支,段]{s.}{canção; poesia cantável}
    \definition{v.}{cantar; entoar | louvar; exaltar; cantar louvores a}
  \end{Phonetics}
\end{Entry}

\begin{Entry}{歌手}{14,4}{⽋、⼿}
  \begin{Phonetics}{歌手}{ge1 shou3}[][HSK 3]
    \definition[个,位,名]{s.}{cantor; vocalista; pessoa com talento para cantar}
  \end{Phonetics}
\end{Entry}

\begin{Entry}{歌曲}{14,6}{⽋、⽈}
  \begin{Phonetics}{歌曲}{ge1 qu3}[][HSK 5]
    \definition[首,支]{s.}{música; obra para as pessoas cantarem, uma combinação de poesia e música}
  \end{Phonetics}
\end{Entry}

\begin{Entry}{歌声}{14,7}{⽋、⼠}
  \begin{Phonetics}{歌声}{ge1 sheng1}[][HSK 3]
    \definition{s.}{canto; voz cantada; som do canto}
  \end{Phonetics}
\end{Entry}

\begin{Entry}{歌词}{14,7}{⽋、⾔}
  \begin{Phonetics}{歌词}{ge1 ci2}[][HSK 6]
    \definition{s.}{letra da música; libreto}
  \end{Phonetics}
\end{Entry}

\begin{Entry}{歌咏}{14,8}{⽋、⼝}
  \begin{Phonetics}{歌咏}{ge1yong3}[][HSK 7-9]
    \definition{v.}{cantar; cantar canções}
  \end{Phonetics}
\end{Entry}

\begin{Entry}{歌星}{14,9}{⽋、⽇}
  \begin{Phonetics}{歌星}{ge1 xing1}[][HSK 6]
    \definition[位,名]{s.}{cantor famoso; estrela da música}
  \end{Phonetics}
\end{Entry}

\begin{Entry}{歌迷}{14,9}{⽋、⾡}
  \begin{Phonetics}{歌迷}{ge1 mi2}
    \definition{s.}{fã de um cantor; pessoas que gostam de ouvir música ou cantar e ficam fascinadas por isso}
  \end{Phonetics}
\end{Entry}

\begin{Entry}{歌剧}{14,10}{⽋、⼑}
  \begin{Phonetics}{歌剧}{ge1ju4}[][HSK 7-9]
    \definition[场,出]{s.}{ópera | ópera ocidental; um drama que integra poesia, música, dança e outras artes, tendo o canto como principal característica}
  \end{Phonetics}
\end{Entry}

\begin{Entry}{歌颂}{14,10}{⽋、⾴}
  \begin{Phonetics}{歌颂}{ge1song4}[][HSK 7-9]
    \definition{v.}{cantar louvores de; exaltar; elogiar; elogio com poesia, geralmente se refere a elogiar com palavras, etc.}
  \end{Phonetics}
\end{Entry}

\begin{Entry}{歌唱}{14,11}{⽋、⼝}
  \begin{Phonetics}{歌唱}{ge1 chang4}[][HSK 6]
    \definition{v.}{cantar | cantar em louvor de; louvor através de cânticos, recitações, etc.}
  \end{Phonetics}
\end{Entry}

\begin{Entry}{歌舞}{14,14}{⽋、⾇}
  \begin{Phonetics}{歌舞}{ge1wu3}[][HSK 7-9]
    \definition{s.}{canto e dança}
  \end{Phonetics}
\end{Entry}

%%%%% EOF %%%%%

