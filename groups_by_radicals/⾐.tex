%%%
%%% Radical "⾐"
%%%

\section*{Radical 145: ``⾐'' (⻂)}\addcontentsline{toc}{section}{Radical 145: ⾐、⻂}

\begin{entry}{衣}{6}{⾐}[Kangxi 145]
  \begin{phonetics}{衣}{yi1}
    \definition[件]{s.}{roupa}
  \end{phonetics}
  \begin{phonetics}{衣}{yi4}
    \definition{v.}{vestir | vestir-se}
  \end{phonetics}
\end{entry}

\begin{entry}{衣甲}{6,5}{⾐、⽥}
  \begin{phonetics}{衣甲}{yi1jia3}
    \definition{s.}{armadura}
  \end{phonetics}
\end{entry}

\begin{entry}{衣服}{6,8}{⾐、⽉}
  \begin{phonetics}{衣服}{yi1fu5}[][HSK 1]
    \definition[套,件]{s.}{roupas; vestuário; algo que se veste para cobrir o corpo e se proteger do frio}
  \end{phonetics}
\end{entry}

\begin{entry}{衣柜}{6,8}{⾐、⽊}
  \begin{phonetics}{衣柜}{yi1gui4}
    \definition[个]{s.}{armário | guarda-roupa}
  \end{phonetics}
\end{entry}

\begin{entry}{衣架}{6,9}{⾐、⽊}
  \begin{phonetics}{衣架}{yi1 jia4}[][HSK 3]
    \definition[个]{s.}{cabideiro; móvel para pendurar roupas | estatura; figura; refere-se ao formato do corpo de uma pessoa; estrutura corporal}
  \end{phonetics}
\end{entry}

\begin{entry}{初}{7}{⾐}
  \begin{phonetics}{初}{chu1}[][HSK 3]
    \definition*{s.}{sobrenome Chu}
    \definition{adj.}{primeiro (em ordem) | elementar; rudimentar | original}
    \definition{adv.}{pela primeira vez; apenas começando; indica que a ação está ocorrendo pela primeira vez ou acabou de começar}
    \definition{pref.}{anexado aos numerais de um a dez para indicar ordem (primeiro ao décimo)}
    \definition{s.}{no início de; na primeira parte de | o estágio júnior (pleno; sênior)}
  \end{phonetics}
\end{entry}

\begin{entry}{初中}{7,4}{⾐、⼁}
  \begin{phonetics}{初中}{chu1 zhong1}[][HSK 3]
    \definition[所,个]{s.}{ensino médio; ensino fundamental}
  \end{phonetics}
\end{entry}

\begin{entry}{初心}{7,4}{⾐、⼼}
  \begin{phonetics}{初心}{chu1xin1}
    \definition{s.}{intenção original de alguém, aspiração, etc. | (budismo) ``mente do iniciante'' (ter a mente aberta quando estudando um assunto como um iniciante no assunto teria)}
  \end{phonetics}
\end{entry}

\begin{entry}{初级}{7,6}{⾐、⽷}
  \begin{phonetics}{初级}{chu1ji2}[][HSK 3]
    \definition{adj.}{elementar; primário; júnior; inicial; o nível mais baixo; de baixa qualidade}
  \end{phonetics}
\end{entry}

\begin{entry}{初步}{7,7}{⾐、⽌}
  \begin{phonetics}{初步}{chu1bu4}[][HSK 3]
    \definition{adj.}{inicial; preliminar; imaturo, incompleto}
  \end{phonetics}
\end{entry}

\begin{entry}{初期}{7,12}{⾐、⽉}
  \begin{phonetics}{初期}{chu1 qi1}[][HSK 5]
    \definition{s.}{primórdio; estágio inicial; primeiros dias; estágio preliminar; período inicial}
  \end{phonetics}
\end{entry}

\begin{entry}{补}{7}{⾐}
  \begin{phonetics}{补}{bu3}[][HSK 3]
    \definition*{s.}{sobrenome Bu}
    \definition{s.}{ajuda; uso; benefício; utilidade}
    \definition{v.}{reparar; consertar; remendar; adicionar materiais, consertar coisas quebradas | abastecer; encher; repor; adicionar suplemento; complementar; completar; preencher | nutrir}
  \end{phonetics}
\end{entry}

\begin{entry}{补充}{7,6}{⾐、⼉}
  \begin{phonetics}{补充}{bu3chong1}[][HSK 3]
    \definition{adj.}{adicional | suplementar}
    \definition{v.}{reabastecer; suplementar; complementar; aumentar uma parte quando houver insuficiência ou perda}
  \end{phonetics}
\end{entry}

\begin{entry}{补贴}{7,9}{⾐、⾙}
  \begin{phonetics}{补贴}{bu3tie1}[][HSK 5]
    \definition[笔,项,种,份]{s.}{subsídio; ajuda de custo; custos de indenização ou assistência concedida a empresas ou indivíduos pelo estado ou governo}
    \definition{v.}{subsidiar; compensar a falta de dinheiro ou coisas; refere-se principalmente à compensação financeira ou ajuda dada pelo estado ou governo a empresas ou indivíduos}
  \end{phonetics}
\end{entry}

\begin{entry}{补偿}{7,11}{⾐、⼈}
  \begin{phonetics}{补偿}{bu3chang2}[][HSK 5]
    \definition{v.}{compensar (perda, consumo); compensar (deficiências, diferenças)}
  \end{phonetics}
\end{entry}

\begin{entry}{表}{8}{⾐}
  \begin{phonetics}{表}{biao3}[][HSK 2]
    \definition*{s.}{sobrenome Biao}
    \definition{s.}{exterior; superfície; externo | a relação entre os filhos ou netos de um irmão e uma irmã ou de irmãs | modelo; exemplo; padrão | memorial a um imperador; um tipo de petição da antiguidade, frequentemente usado para expressar intenções; mais tarde, também usado para expressar opiniões sobre eventos importantes | formulário; lista; gráfico; tabela | medidor; instrumento para medir uma determinada quantidade | relógio; um dispositivo para medir o tempo, menor que um relógio, que geralmente pode ser carregado no bolso | medidor de luz solar; antiga vara de madeira para medir o tempo através da sombra do sol | coluna usada antigamente para marcação}
    \definition{v.}{mostrar; expressar; expressar ideias, pensamentos, sentimentos, etc. | administrar medicamentos para aliviar o resfriado; na medicina tradicional chinesa refere-se ao uso de medicamentos para dissipar o frio e o vento que afetam o corpo}
  \end{phonetics}
\end{entry}

\begin{entry}{表白}{8,5}{⾐、⽩}
  \begin{phonetics}{表白}{biao3bai2}
    \definition{s.}{declaração | confissão}
    \definition{v.}{confessar a si mesmo | expressar | revelar pensamentos ou sentimentos de alguém}
  \end{phonetics}
\end{entry}

\begin{entry}{表示}{8,5}{⾐、⽰}
  \begin{phonetics}{表示}{biao3shi4}[][HSK 2]
    \definition{s.}{expressão; indicação}
    \definition{v.}{mostrar; expressar; indicar | significar | expressar pensamentos e sentimentos através de palavras, ações ou expressões faciais}
  \end{phonetics}
\end{entry}

\begin{entry}{表扬}{8,6}{⾐、⼿}
  \begin{phonetics}{表扬}{biao3yang2}[][HSK 4]
    \definition[次,种,份]{s.}{elogios públicos por boas ações}
    \definition{v.}{elogiar; louvar}
  \end{phonetics}
\end{entry}

\begin{entry}{表扬信}{8,6,9}{⾐、⼿、⼈}
  \begin{phonetics}{表扬信}{biao3yang2 xin4}
    \definition{s.}{carta de elogio | depoimento}
  \end{phonetics}
\end{entry}

\begin{entry}{表达}{8,6}{⾐、⾡}
  \begin{phonetics}{表达}{biao3da2}[][HSK 3]
    \definition{v.}{entregar; expressar; mostrar; manifestar; transmitir; comunicar; refere-se ao processo de transmitir pensamentos, sentimentos ou opiniões pessoais a outras pessoas por meio de linguagem, texto, ações, etc.}
  \end{phonetics}
\end{entry}

\begin{entry}{表明}{8,8}{⾐、⽇}
  \begin{phonetics}{表明}{biao3ming2}[][HSK 3]
    \definition{v.}{indicar; demonstrar; expressar; marcar; expressar claramente; expressar de forma clara}
  \end{phonetics}
\end{entry}

\begin{entry}{表现}{8,8}{⾐、⾒}
  \begin{phonetics}{表现}{biao3xian4}[][HSK 3]
    \definition[个,种,份]{s.}{desempenho; expressão; manifestação; comportamento; as ideias, o estilo, as qualidades, o nível ou as capacidades demonstrados em ação.}
    \definition{v.}{mostrar; expressar; exibir; manifestar; descrever; demonstrar algum tipo de pensamento, espírito, qualidade, sentimento ou habilidade, etc. | exibir-se; demonstrar de forma inadequada e intencional alguma habilidade, ponto forte ou vantagem.}
  \end{phonetics}
\end{entry}

\begin{entry}{表面}{8,9}{⾐、⾯}
  \begin{phonetics}{表面}{biao3mian4}[][HSK 3]
    \definition{s.}{superfície; face; exterior; aparência | aparência; superficialidade | mostrador (placa); mostrador do relógio | aparência; a aparência externa das coisas ou a parte não essencial delas}
  \end{phonetics}
\end{entry}

\begin{entry}{表格}{8,10}{⾐、⽊}
  \begin{phonetics}{表格}{biao3ge2}[][HSK 3]
    \definition[张,份,个]{s.}{tabela; formulário}
  \end{phonetics}
\end{entry}

\begin{entry}{表情}{8,11}{⾐、⼼}
  \begin{phonetics}{表情}{biao3qing2}[][HSK 4]
    \definition[个,种,幅]{s.}{expressão; expressão facial; expressão de pensamentos e sentimentos internos por meio de mudanças faciais ou de gestos}
    \definition{v.}{expressar pensamentos e sentimentos internos por meio de mudanças faciais ou de gestos}
  \end{phonetics}
\end{entry}

\begin{entry}{表演}{8,14}{⾐、⽔}
  \begin{phonetics}{表演}{biao3yan3}[][HSK 3]
    \definition[场]{s.}{performance; exposição; refere-se às atividades expressas pelos atores por meio da linguagem, voz, expressões faciais, instrumentos musicais ou movimentos}
    \definition{v.}{atuar; representar; interpretar | demonstrar; fazer demonstrações | fingir; agir de forma afetada; metáfora para fingir deliberadamente uma determinada atitude para enganar alguém}
  \end{phonetics}
\end{entry}

\begin{entry}{表演艺术}{8,14,4,5}{⾐、⽔、⾋、⽊}
  \begin{phonetics}{表演艺术}{biao3yan3 yi4shu4}
    \definition{s.}{arte performática}
  \end{phonetics}
\end{entry}

\begin{entry}{表演者}{8,14,8}{⾐、⽔、⽼}
  \begin{phonetics}{表演者}{biao3yan3 zhe3}
    \definition{s.}{ator}
  \end{phonetics}
\end{entry}

\begin{entry}{表演特技}{8,14,10,7}{⾐、⽔、⽜、⼿}
  \begin{phonetics}{表演特技}{biao3yan3 te4ji4}
    \definition{s.}{acrobacia | pirueta | façanha}
  \end{phonetics}
\end{entry}

\begin{entry}{表演游戏}{8,14,12,6}{⾐、⽔、⽔、⼽}
  \begin{phonetics}{表演游戏}{biao3yan3 you2xi4}
    \definition{s.}{exibição dramática}
  \end{phonetics}
\end{entry}

\begin{entry}{表演赛}{8,14,14}{⾐、⽔、⾙}
  \begin{phonetics}{表演赛}{biao3yan3sai4}
    \definition{s.}{partida ou jogo de exibição}
  \end{phonetics}
\end{entry}

\begin{entry}{衬衣}{8,6}{⾐、⾐}
  \begin{phonetics}{衬衣}{chen4 yi1}[][HSK 3]
    \definition[件,个]{s.}{camisa; também se refere a uma peça de roupa usada por baixo do casaco}
  \end{phonetics}
\end{entry}

\begin{entry}{衬衫}{8,8}{⾐、⾐}
  \begin{phonetics}{衬衫}{chen4shan1}[][HSK 3]
    \definition[件,个]{s.}{camisa; blusa; camisa ocidental usada por baixo}
  \end{phonetics}
\end{entry}

\begin{entry}{袖}{10}{⾐}
  \begin{phonetics}{袖}{xiu4}
    \definition{s.}{manga (de camisa, de camiseta, etc.)}
  \end{phonetics}
\end{entry}

\begin{entry}{袜子}{10,3}{⾐、⼦}
  \begin{phonetics}{袜子}{wa4zi5}[][HSK 4]
    \definition[双,只,对]{s.}{meias; peúgas; meias-calças}
  \end{phonetics}
\end{entry}

\begin{entry}{被}{10}{⾐}
  \begin{phonetics}{被}{bei4}[][HSK 3]
    \definition{part.}{usada antes de verbos para formar frases verbais passivas}
    \definition{prep.}{usado em uma estrutura passiva para introduzir o executor da ação ou apenas a ação | usado em frases para expressar passividade, com o sujeito sendo o objeto}
    \definition{s.}{colcha}
    \definition{v.}{cobrir; espalhar | sofrer}
  \end{phonetics}
\end{entry}

\begin{entry}{被子}{10,3}{⾐、⼦}
  \begin{phonetics}{被子}{bei4zi5}[][HSK 3]
    \definition[床]{s.}{colcha}
  \end{phonetics}
\end{entry}

\begin{entry}{被动}{10,6}{⾐、⼒}
  \begin{phonetics}{被动}{bei4dong4}[][HSK 5]
    \definition{adj.}{passivo;  agir com base em um impulso externo (o oposto de 主动) | passivo; impossibilidade de prosseguir como pretendido devido a resistência ou interferência}
  \seealsoref{主动}{zhu3dong4}
  \end{phonetics}
\end{entry}

\begin{entry}{被告}{10,7}{⾐、⼝}
  \begin{phonetics}{被告}{bei4gao4}
    \definition{s.}{réu}
  \end{phonetics}
\end{entry}

\begin{entry}{被单}{10,8}{⾐、⼗}
  \begin{phonetics}{被单}{bei4dan1}
    \definition[床]{s.}{lençol}
  \end{phonetics}
\end{entry}

\begin{entry}{被迫}{10,8}{⾐、⾡}
  \begin{phonetics}{被迫}{bei4 po4}[][HSK 4]
    \definition{v.}{ser forçado; ser coagido; ser compelido; ser constrangido; ser forçado a fazer algo por força externa}
  \end{phonetics}
\end{entry}

\begin{entry}{被套}{10,10}{⾐、⼤}
  \begin{phonetics}{被套}{bei4tao4}
    \definition{s.}{capa de \emph{edredon}}
    \definition{v.}{ter dinheiro preso (em ações, imóveis, etc.)}
  \end{phonetics}
\end{entry}

\begin{entry}{被窝}{10,12}{⾐、⽳}
  \begin{phonetics}{被窝}{bei4wo1}
    \definition{s.}{colcha}
  \end{phonetics}
\end{entry}

\begin{entry}{袋}{11}{⾐}
  \begin{phonetics}{袋}{dai4}[][HSK 4]
    \definition{clas.}{para armazenamento em sacolas | para cachimbos, cigarros ou tabaco seco}
    \definition[口]{s.}{saco; sacola; bolso; bolsa}
  \end{phonetics}
\end{entry}

\begin{entry}{袭击}{11,5}{⾐、⼐}
  \begin{phonetics}{袭击}{xi2ji1}
    \definition{s.}{ataque (especialmente um ataque surpresa) | invasão}
    \definition{v.}{atacar}
  \end{phonetics}
\end{entry}

\begin{entry}{裁}{12}{⾐}
  \begin{phonetics}{裁}{cai2}
    \definition{s.}{decisão | julgamento}
    \definition{v.}{recortar (tecido de uma roupa) | cortar | aparar | reduzir | diminuir | cortar pessoal de uma equipe}
  \end{phonetics}
\end{entry}

\begin{entry}{裁判}{12,7}{⾐、⼑}
  \begin{phonetics}{裁判}{cai2pan4}[][HSK 5]
    \definition[个,位,名]{s.}{árbitro; juiz; pessoa que desempenha funções de arbitragem em esportes e outras competições}
    \definition{v.}{arbitrar; atuar como árbitro; em esportes e outras atividades competitivas, julgar o desempenho dos atletas, vitórias e derrotas, classificações e problemas que ocorrem durante a competição de acordo com as regras da competição | julgar; refere-se a um terceiro que faz um julgamento quando surge uma disputa entre duas partes}
  \end{phonetics}
\end{entry}

\begin{entry}{装}{12}{⾐}
  \begin{phonetics}{装}{zhuang1}[][HSK 2]
    \definition*{s.}{sobrenome Zhuang}
    \definition{s.}{vestido; traje; vestimenta; roupa | maquiagem e figurino de palco; maquiagem de ator}
    \definition{v.}{enfeitar; adornar; vestir; decorar; vestir-se; vestir-se bem | fingir; fazer de conta | segurar; embalar; carregar; colocar as coisas em recipientes; colocar as coisas no transporte | encaixar; instalar; equipar; aparelhar; montar | embalar; encaixotar; embrulhar produtos ou colocá-los em caixas, garrafas, etc.}
  \end{phonetics}
\end{entry}

\begin{entry}{装扮}{12,7}{⾐、⼿}
  \begin{phonetics}{装扮}{zhuang1ban4}
    \definition{v.}{enfeitar | decorar | disfarçar-me | vestir-se}
  \end{phonetics}
\end{entry}

\begin{entry}{装备}{12,8}{⾐、⼡}
  \begin{phonetics}{装备}{zhuang1bei4}
    \definition{s.}{equipamento}
    \definition{v.}{equipar}
  \end{phonetics}
\end{entry}

\begin{entry}{装饰}{12,8}{⾐、⾷}
  \begin{phonetics}{装饰}{zhuang1shi4}[][HSK 5]
    \definition[件,个]{s.}{decoração; acessórios decorativos}
    \definition{v.}{enfeitar; adornar; decorar; ornamentar; embelezar; destacar}
  \end{phonetics}
\end{entry}

\begin{entry}{装修}{12,9}{⾐、⼈}
  \begin{phonetics}{装修}{zhuang1 xiu1}[][HSK 4]
    \definition{v.}{equipar; renovar; decorar (equipar uma sala ou prédio com equipamentos ou decorações)}
  \end{phonetics}
\end{entry}

\begin{entry}{装配}{12,10}{⾐、⾣}
  \begin{phonetics}{装配}{zhuang1pei4}
    \definition{v.}{montar | encaixar}
  \end{phonetics}
\end{entry}

\begin{entry}{装置}{12,13}{⾐、⽹}
  \begin{phonetics}{装置}{zhuang1 zhi4}[][HSK 4]
    \definition{s.}{dispositivo; equipamento; máquinas, instrumentos ou outros equipamentos de construção mais complexa e com alguma função independente}
    \definition{v.}{instalar; ajustar; configurar; equipar; montar}
  \end{phonetics}
\end{entry}

\begin{entry}{裙子}{12,3}{⾐、⼦}
  \begin{phonetics}{裙子}{qun2zi5}[][HSK 3]
    \definition[条,件]{s.}{saia (peça de vestuário)}
  \end{phonetics}
\end{entry}

\begin{entry}{裤子}{12,3}{⾐、⼦}
  \begin{phonetics}{裤子}{ku4zi5}[][HSK 3]
    \definition[条]{s.}{calças; calções; roupas usadas abaixo da cintura, com cós, virilha e duas pernas}
  \end{phonetics}
\end{entry}

\begin{entry}{褐色}{14,6}{⾐、⾊}
  \begin{phonetics}{褐色}{he4 se4}
    \definition{s.}{cor marrom}
  \end{phonetics}
\end{entry}

%%%%% EOF %%%%%

