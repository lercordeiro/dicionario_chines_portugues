%%%
%%% Radical "⼆"
%%%

\section*{Radical 7: ``⼆''}\addcontentsline{toc}{section}{Radical 7: ⼆}

\begin{entry}{二}{2}{⼆}[Kangxi 7]
  \begin{phonetics}{二}{er4}[][HSK 1]
    \definition{num.}{dois; 2 | (dialeto de Pequim) estúpido}
  \end{phonetics}
\end{entry}

\begin{entry}{二手}{2,4}{⼆、⼿}
  \begin{phonetics}{二手}{er4 shou3}[][HSK 4]
    \definition{adj.}{usado; de segunda mão; refere-se especificamente a usados e revendidos}
  \end{phonetics}
\end{entry}

\begin{entry}{二战}{2,9}{⼆、⼽}
  \begin{phonetics}{二战}{er4zhan4}
    \definition*{s.}{Segunda Guerra Mundial}
  \end{phonetics}
\end{entry}

\begin{entry}{二维码}{2,11,8}{⼆、⽷、⽯}
  \begin{phonetics}{二维码}{er4 wei2 ma3}[][HSK 5]
    \definition{s.}{\emph{QR code}}
  \end{phonetics}
\end{entry}

\begin{entry}{于}{3}{⼆}
  \begin{phonetics}{于}{yu2}
    \definition*{s.}{sobrenome Yu}
    \definition{prep.}{indica tempo, local, extensão, etc. | indica a direção da ação | usada após um verbo para indicar doação, entrega, etc. | relacionamento do objeto ou da entidade introduzida | indica o ponto inicial ou o ponto de partida | indica comparação}
  \end{phonetics}
\end{entry}

\begin{entry}{于是}{3,9}{⼆、⽇}
  \begin{phonetics}{于是}{yu2shi4}[][HSK 4]
    \definition{conj.}{então; portanto; consequentemente; como resultado; indica que o último segue o primeiro e que o último é frequentemente causado pelo primeiro}
  \end{phonetics}
\end{entry}

\begin{entry}{云}{4}{⼆}
  \begin{phonetics}{云}{yun2}[][HSK 2]
    \definition*{s.}{sobrenome Yun}
    \definition[朵]{s.}{nuvem}
  \end{phonetics}
\end{entry}

\begin{entry}{云云}{4,4}{⼆、⼆}
  \begin{phonetics}{云云}{yun2yun2}
    \definition{adv.}{e assim por diante | assim e assim}
  \end{phonetics}
\end{entry}

\begin{entry}{云南}{4,9}{⼆、⼗}
  \begin{phonetics}{云南}{yun2nan2}
    \definition*{s.}{Yunnan}
  \end{phonetics}
\end{entry}

\begin{entry}{云端}{4,14}{⼆、⽴}
  \begin{phonetics}{云端}{yun2duan1}
    \definition{s.}{alto nas nuvens | (computação) a nuvem}
  \end{phonetics}
\end{entry}

\begin{entry}{互}{4}{⼆}
  \begin{phonetics}{互}{hu4}
    \definition{adj.}{mútuo | recíproco}
  \end{phonetics}
\end{entry}

\begin{entry}{互动}{4,6}{⼆、⼒}
  \begin{phonetics}{互动}{hu4dong4}
    \definition{s.}{interativo}
    \definition{v.}{interagir}
  \end{phonetics}
\end{entry}

\begin{entry}{互利}{4,7}{⼆、⼑}
  \begin{phonetics}{互利}{hu4li4}
    \definition{s.}{benefício mútuo}
  \end{phonetics}
\end{entry}

\begin{entry}{互相}{4,9}{⼆、⽬}
  \begin{phonetics}{互相}{hu4xiang1}[][HSK 3]
    \definition{adv.}{mutuamente; um ao outro}
  \end{phonetics}
\end{entry}

\begin{entry}{互联网}{4,12,6}{⼆、⽿、⽹}
  \begin{phonetics}{互联网}{hu4lian2wang3}[][HSK 3]
    \definition{s.}{\emph{Internet}}
  \seealsoref{网际网路}{wang3ji4wang3lu4}
  \seealsoref{网际网络}{wang3ji4wang3luo4}
  \seealsoref{网路}{wang3lu4}
  \end{phonetics}
\end{entry}

\begin{entry}{五}{4}{⼆}
  \begin{phonetics}{五}{wu3}[][HSK 1]
    \definition{num.}{cinco; 5}
  \end{phonetics}
\end{entry}

\begin{entry}{五五}{4,4}{⼆、⼆}
  \begin{phonetics}{五五}{wu3wu3}
    \definition{num.}{50-50}
    \definition{s.}{igual (partilha, parceria, etc.)}
  \end{phonetics}
\end{entry}

\begin{entry}{五体投地}{4,7,7,6}{⼆、⼈、⼿、⼟}
  \begin{phonetics}{五体投地}{wu3ti3tou2di4}
    \definition{expr.}{prostrar-se em admiração | adular alguém}
  \end{phonetics}
\end{entry}

\begin{entry}{五颜六色}{4,15,4,6}{⼆、⾴、⼋、⾊}
  \begin{phonetics}{五颜六色}{wu3 yan2 liu4 se4}[][HSK 4]
    \definition{adj.}{todas as cores sob o sol; multicolorido; colorido}
  \end{phonetics}
\end{entry}

\begin{entry}{井}{4}{⼆}
  \begin{phonetics}{井}{jing3}
    \definition{adj.}{puro | ordenado}
    \definition[口]{s.}{poço}
  \end{phonetics}
\end{entry}

\begin{entry}{亚运会}{6,7,6}{⼆、⾡、⼈}
  \begin{phonetics}{亚运会}{ya4 yun4 hui4}[][HSK 4]
    \definition*{s.}{Jogos Asiáticos}
  \end{phonetics}
\end{entry}

\begin{entry}{亚细亚洲}{6,8,6,9}{⼆、⽷、⼆、⽔}
  \begin{phonetics}{亚细亚洲}{ya4xi4ya4zhou1}
    \definition*{s.}{Ásia}
  \end{phonetics}
\end{entry}

\begin{entry}{亚洲}{6,9}{⼆、⽔}
  \begin{phonetics}{亚洲}{ya4zhou1}
    \definition*{s.}{Ásia, abreviação de~亚细亚洲}
    \seeref{亚细亚洲}{ya4xi4ya4zhou1}
  \end{phonetics}
\end{entry}

\begin{entry}{亚洲人}{6,9,2}{⼆、⽔、⼈}
  \begin{phonetics}{亚洲人}{ya4zhou1ren2}
    \definition{s.}{asiático | pessoa ou povo da Ásia}
  \end{phonetics}
\end{entry}

\begin{entry}{些}{8}{⼆}
  \begin{phonetics}{些}{xie1}[][HSK 4]
    \definition{adv.}{um pouco; um pouco mais; usado após um adjetivo ou parte de um verbo para indicar uma pequena quantidade, equivalente a ``一点儿''}
    \definition{clas.}{alguns; um pouco; denota uma quantidade indefinida}
  \seealsoref{一点儿}{yi4dian3r5}
  \end{phonetics}
\end{entry}

\begin{entry}{些许}{8,6}{⼆、⾔}
  \begin{phonetics}{些许}{xie1xu3}
    \definition{num.}{um pouco}
  \end{phonetics}
\end{entry}

%%%%% EOF %%%%%

