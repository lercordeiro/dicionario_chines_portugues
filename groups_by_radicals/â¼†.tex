%%%
%%% Radical "⼆"
%%%

\section*{Radical 7: ``⼆''}\addcontentsline{toc}{section}{Radical 7: ⼆}

\begin{Entry}{二}{2}{⼆}[Kangxi 7]
  \begin{Phonetics}{二}{er4}[][HSK 1]
    \definition{adj.}{diferente; refere-se a duas coisas ou coisas diferentes | bobo; pateta; tolo; sem inteligência | desleal; infiel; indiferente; sem determinação}
    \definition{num.}{dois; 2}
  \end{Phonetics}
\end{Entry}

\begin{Entry}{二手}{2,4}{⼆、⼿}
  \begin{Phonetics}{二手}{er4 shou3}[][HSK 4]
    \definition{adj.}{usado; de segunda mão; refere-se especificamente a usados e revendidos}
  \end{Phonetics}
\end{Entry}

\begin{Entry}{二战}{2,9}{⼆、⼽}
  \begin{Phonetics}{二战}{er4zhan4}
    \definition*{s.}{Segunda Guerra Mundial}
  \end{Phonetics}
\end{Entry}

\begin{Entry}{二胡}{2,9}{⼆、⾁}
  \begin{Phonetics}{二胡}{er4hu2}
    \definition{s.}{erhu; um instrumento de arco de duas cordas com um registro mais baixo que o 京胡; um tipo de 胡琴, a caixa de som é feita de bambu, madeira, etc., coberta com pele de cobra, etc., tem duas cordas e o tom é baixo e suave}
  \seealsoref{胡琴}{hu2qin2}
  \seealsoref{京胡}{jing1hu2}
  \end{Phonetics}
\end{Entry}

\begin{Entry}{二维码}{2,11,8}{⼆、⽷、⽯}
  \begin{Phonetics}{二维码}{er4 wei2 ma3}[][HSK 5]
    \definition[个]{s.}{\emph{QR code}; um identificador gráfico que distribui formas geométricas específicas em um plano ou direção bidimensional de acordo com certas regras para expressar um conjunto de informações}
  \end{Phonetics}
\end{Entry}

\begin{Entry}{于}{3}{⼆}
  \begin{Phonetics}{于}{yu2}[][HSK 6]
    \definition*{s.}{Sobrenome Yu}
    \definition{prep.}{indica hora, lugar, alcance, etc. | indica a direção da ação | usado depois de um verbo para indicar dar, entregar, etc. | apresentar a relação do objeto ou entidade introduzida | indica o ponto de início ou de partida | indica comparação | indica passividade}
  \end{Phonetics}
\end{Entry}

\begin{Entry}{于是}{3,9}{⼆、⽇}
  \begin{Phonetics}{于是}{yu2shi4}[][HSK 4]
    \definition{conj.}{então; portanto; consequentemente; como resultado; indica que o último segue o primeiro e que o último é frequentemente causado pelo primeiro}
  \end{Phonetics}
\end{Entry}

\begin{Entry}{亏}{3}{⼆}
  \begin{Phonetics}{亏}{kui1}[][HSK 5]
    \definition{adv.}{felizmente; por sorte; graças a | contrariamente, expressando sarcasmo}
    \definition{s.}{prejuízo; perda; déficit | perda; dano; ferida}
    \definition{v.}{perder dinheiro, etc.; ter um déficit; ter prejuízo | ter falta de; ser deficiente; carecer de | tratar injustamente; causar prejuízo; trair a confiança}
  \end{Phonetics}
\end{Entry}

\begin{Entry}{云}{4}{⼆}
  \begin{Phonetics}{云}{yun2}[][HSK 2]
    \definition*{s.}{Província de Yunnan, abreviação de 云南 | Sobrenome Yun}
    \definition[片,朵]{s.}{nuvem}
    \definition{v.}{dizer}
  \seealsoref{云南}{yun2nan2}
  \end{Phonetics}
\end{Entry}

\begin{Entry}{云云}{4,4}{⼆、⼆}
  \begin{Phonetics}{云云}{yun2yun2}
    \definition{adv.}{e assim por diante | assim e assim}
  \end{Phonetics}
\end{Entry}

\begin{Entry}{云南}{4,9}{⼆、⼗}
  \begin{Phonetics}{云南}{yun2nan2}
    \definition*{s.}{Província de Yunnan}
  \end{Phonetics}
\end{Entry}

\begin{Entry}{云端}{4,14}{⼆、⽴}
  \begin{Phonetics}{云端}{yun2duan1}
    \definition{s.}{alto nas nuvens | (computação) a nuvem}
  \end{Phonetics}
\end{Entry}

\begin{Entry}{互}{4}{⼆}
  \begin{Phonetics}{互}{hu4}
    \definition{adv.}{mutuamente; um ao outro}
    \definition{pron.}{um ao outro; mútuo}
  \end{Phonetics}
\end{Entry}

\begin{Entry}{互动}{4,6}{⼆、⼒}
  \begin{Phonetics}{互动}{hu4 dong4}[][HSK 6]
    \definition{v.}{interagir; participar juntos; promover uns aos outros}
  \end{Phonetics}
\end{Entry}

\begin{Entry}{互利}{4,7}{⼆、⼑}
  \begin{Phonetics}{互利}{hu4li4}
    \definition{s.}{benefício mútuo}
  \end{Phonetics}
\end{Entry}

\begin{Entry}{互相}{4,9}{⼆、⽬}
  \begin{Phonetics}{互相}{hu4xiang1}[][HSK 3]
    \definition{adv.}{mutuamente; um ao outro; expressa uma relação de igualdade entre as partes}
  \end{Phonetics}
\end{Entry}

\begin{Entry}{互联网}{4,12,6}{⼆、⽿、⽹}
  \begin{Phonetics}{互联网}{hu4 lian2 wang3}[][HSK 3]
    \definition{s.}{\emph{Internet}; uma enorme rede conectando computadores e redes de computadores}
  \seealsoref{网际网路}{wang3 ji4 wang3 lu4}
  \seealsoref{网际网络}{wang3 ji4 wang3 luo4}
  \seealsoref{网路}{wang3 lu4}
  \end{Phonetics}
\end{Entry}

\begin{Entry}{五}{4}{⼆}
  \begin{Phonetics}{五}{wu3}[][HSK 1]
    \definition*{s.}{Sobrenome Wu}
    \definition{num.}{cinco; 5}
    \definition{s.}{uma nota da escala em Gongchepu (工尺谱), correspondente a 6 na notação musical numerada}
  \seealsoref{工尺谱}{gong1 che3 pu3}
  \end{Phonetics}
\end{Entry}

\begin{Entry}{五五}{4,4}{⼆、⼆}
  \begin{Phonetics}{五五}{wu3wu3}
    \definition{num.}{50-50}
    \definition{s.}{igual (partilha, parceria, etc.)}
  \end{Phonetics}
\end{Entry}

\begin{Entry}{五体投地}{4,7,7,6}{⼆、⼈、⼿、⼟}
  \begin{Phonetics}{五体投地}{wu3ti3tou2di4}
    \definition{expr.}{prostrar-se em admiração | adular alguém}
  \end{Phonetics}
\end{Entry}

\begin{Entry}{五颜六色}{4,15,4,6}{⼆、⾴、⼋、⾊}
  \begin{Phonetics}{五颜六色}{wu3 yan2 liu4 se4}[][HSK 4]
    \definition{adj.}{de várias (ou todas) cores; multicolorido; colorido}
  \end{Phonetics}
\end{Entry}

\begin{Entry}{井}{4}{⼆}
  \begin{Phonetics}{井}{jing3}[][HSK 6]
    \definition*{s.}{Jing, uma das mansões lunares | Sobrenome Jing}
    \definition{adj.}{limpo; organizado}
    \definition[口]{s.}{poço; um buraco profundo cavado no chão para tirar água | algo em forma de poço | vila natal ou cidade natal}
  \end{Phonetics}
\end{Entry}

\begin{Entry}{亚}{6}{⼆}
  \begin{Phonetics}{亚}{ya4}
    \definition*{s.}{Ásia, abreviação de 亚洲 | Sobrenome Ya}
    \definition{adj.}{inferior | abaixo do padrão | (química) de menor valência atômica}
    \definition{pref.}{sub-}
  \seealsoref{亚洲}{ya4zhou1}
  \end{Phonetics}
\end{Entry}

\begin{Entry}{亚军}{6,6}{⼆、⼍}
  \begin{Phonetics}{亚军}{ya4jun1}[][HSK 5]
    \definition[个]{s.}{segundo lugar; vice-campeão; medalhista de prata}
  \end{Phonetics}
\end{Entry}

\begin{Entry}{亚运会}{6,7,6}{⼆、⾡、⼈}
  \begin{Phonetics}{亚运会}{ya4 yun4 hui4}[][HSK 4]
    \definition*{s.}{Jogos Asiáticos}
  \end{Phonetics}
\end{Entry}

\begin{Entry}{亚细亚洲}{6,8,6,9}{⼆、⽷、⼆、⽔}
  \begin{Phonetics}{亚细亚洲}{ya4xi4ya4zhou1}
    \definition*{s.}{Ásia}
  \end{Phonetics}
\end{Entry}

\begin{Entry}{亚洲}{6,9}{⼆、⽔}
  \begin{Phonetics}{亚洲}{ya4zhou1}
    \definition*{s.}{Ásia, abreviação de 亚细亚洲}
  \seealsoref{亚细亚洲}{ya4xi4ya4zhou1}
  \end{Phonetics}
\end{Entry}

\begin{Entry}{亚洲人}{6,9,2}{⼆、⽔、⼈}
  \begin{Phonetics}{亚洲人}{ya4zhou1ren2}
    \definition{s.}{asiático | pessoa ou povo da Ásia}
  \end{Phonetics}
\end{Entry}

\begin{Entry}{亚热带}{6,10,9}{⼆、⽕、⼱}
  \begin{Phonetics}{亚热带}{ya4re4dai4}
    \definition{s.}{zona ou clima subtropical; subtropical; semitropical}
  \end{Phonetics}
\end{Entry}

\begin{Entry}{些}{8}{⼆}
  \begin{Phonetics}{些}{xie1}[][HSK 4]
    \definition{adv.}{um pouco; um pouco mais; usado após um adjetivo ou parte de um verbo para indicar uma pequena quantidade, equivalente a 一点儿}
    \definition{clas.}{alguns; um pouco; denota uma quantidade indefinida}
  \seealsoref{一点儿}{yi4dian3r5}
  \end{Phonetics}
\end{Entry}

\begin{Entry}{些许}{8,6}{⼆、⾔}
  \begin{Phonetics}{些许}{xie1xu3}
    \definition{num.}{um pouco}
  \end{Phonetics}
\end{Entry}

%%%%% EOF %%%%%

