%%%
%%% Radical "⼂"
%%%

\section*{Radical 3: ``⼂'' (乀,乁)}\addcontentsline{toc}{section}{Radical 3: ⼂、乀,乁}

\begin{Entry}{义}{3}{⼂}
  \begin{Phonetics}{义}{yi4}
    \definition*{s.}{Sobrenome Yi}
    \definition{adj.}{justo; equitativo | adotado; adotivo | juramentado | artificial; falso}
    \definition[个]{s.}{justiça; retidão | laços humanos; relacionamento | significado; importância}
  \end{Phonetics}
\end{Entry}

\begin{Entry}{义务}{3,5}{⼂、⼒}
  \begin{Phonetics}{义务}{yi4wu4}[][HSK 4]
    \definition{adj.}{voluntário; fornecer serviços ou ajuda a outros gratuitamente}
    \definition[项]{s.}{dever; obrigação; responsabilidades perante a lei, em oposição a 权利 | obrigação moral; responsabilidade moral}
  \seealsoref{权利}{quan2li4}
  \end{Phonetics}
\end{Entry}

\begin{Entry}{之}{3}{⼂}
  \begin{Phonetics}{之}{zhi1}
    \definition*{s.}{Sobrenome Zhi}
    \definition{part.}{entre um atributivo e a palavra que ele modifica; equivalente a "的" | usado entre o sujeito e o predicado, em estruturas sujeito-predicado, de modo a torná-lo nominalizado}
    \definition{pron.}{substituto de uma pessoa ou coisa, limitado a ser usado como um objeto; substituir a pessoa ou coisa mencionada anteriormente | isto; isso; não substitui uma pessoa ou coisa específica, mas serve apenas para complementar sílabas}
    \definition{v.}{ir; deixar}
  \seealsoref{的}{de5}
  \end{Phonetics}
\end{Entry}

\begin{Entry}{之一}{3,1}{⼂、⼀}
  \begin{Phonetics}{之一}{zhi1 yi1}[][HSK 4]
    \definition[分]{s.}{um de (algo); pertence a um ou a todo um grupo de coisas com as mesmas características}
  \end{Phonetics}
\end{Entry}

\begin{Entry}{之下}{3,3}{⼂、⼀}
  \begin{Phonetics}{之下}{zhi1 xia4}[][HSK 5]
    \definition{s.}{usado para indicar algo abaixo de um determinado intervalo, posição, grau, etc.; indica um aspecto inferior em termos de alcance, posição, status, nível, Chengdu, etc. | usado para indicar as condições sob as quais algo acontece | usado para indicar o humor, estado em que alguém faz algo; expressa um determinado comportamento em um determinado estado de espírito ou situação}
  \end{Phonetics}
\end{Entry}

\begin{Entry}{之中}{3,4}{⼂、⼁}
  \begin{Phonetics}{之中}{zhi1 zhong1}[][HSK 5]
    \definition{prep.}{em; no meio de; entre}
  \end{Phonetics}
\end{Entry}

\begin{Entry}{之内}{3,4}{⼂、⼌}
  \begin{Phonetics}{之内}{zhi1 nei4}[][HSK 5]
    \definition{adv.}{em; dentro de; indica dentro de um determinado intervalo, limite ou período de tempo, etc.}
  \end{Phonetics}
\end{Entry}

\begin{Entry}{之外}{3,5}{⼂、⼣}
  \begin{Phonetics}{之外}{zhi1 wai4}[][HSK 5]
    \definition{adv.}{lado de fora; exceto; além de; além disso; refere-se a algo que excede um determinado limite}
  \end{Phonetics}
\end{Entry}

\begin{Entry}{之后}{3,6}{⼂、⼝}
  \begin{Phonetics}{之后}{zhi1 hou4}[][HSK 4]
    \definition{s.}{mais tarde; posteriormente; depois; desde então; para indicar que é depois de um determinado tempo ou de uma determinada coisa, 以后 é usado com frequência na linguagem falada; às vezes, também pode indicar que é depois de um determinado lugar ou local,  后面 é usado com frequência na linguagem falada}
  \seealsoref{后面}{hou4mian4}
  \seealsoref{以后}{yi3 hou4}
  \end{Phonetics}
\end{Entry}

\begin{Entry}{之间}{3,7}{⼂、⾨}
  \begin{Phonetics}{之间}{zhi1 jian1}[][HSK 4]
    \definition{s.}{(depois de um substantivo) entre; dentro de duas delimitações de tempo, local ou quantitativas | colocado após certos verbos ou advérbios de duas sílabas para indicar um curto período de tempo}
  \end{Phonetics}
\end{Entry}

\begin{Entry}{之前}{3,9}{⼂、⼑}
  \begin{Phonetics}{之前}{zhi1 qian2}[][HSK 4]
    \definition{adv.}{(referindo-se ao tempo) antes, antes de, atrás | (referindo-se ao local físico) na frente de | (usado independentemente) no passado, antes disso}
  \end{Phonetics}
\end{Entry}

\begin{Entry}{之类}{3,9}{⼂、⽶}
  \begin{Phonetics}{之类}{zhi1 lei4}[][HSK 6]
    \definition{s.}{usado para dar exemplos (coisas do tipo, desse tipo, assim); uma categoria de pessoas ou coisas que compartilham as mesmas características das pessoas ou coisas mencionadas anteriormente}[我喜欢香蕉、苹果之类的水果。===Eu gosto de frutas como bananas e maçãs.]
  \end{Phonetics}
\end{Entry}

\begin{Entry}{为}{4}{⼂}
  \begin{Phonetics}{为}{wei2}[][HSK 3]
    \definition*{s.}{Sobrenome Wei}
    \definition{part.}{frequentemente usado com 何 em uma pergunta retórica}
    \definition{prep.}{por; usado em frases passivas para introduzir o agente da ação, equivalente a 被 (frequentemente usado com 所)}
    \definition{suf.}{é anexado a alguns adjetivos ou advérbios monossilábicos para formar advérbios dissilábicos que expressam grau ou amplitude, geralmente modificando adjetivos ou verbos dissilábicos}
    \definition{v.}{fazer; agir | tornar-se; transformar-se em | ser; significar | servir como; agir como; desempenhar o papel de | fazer; trabalhar; indica certas ações e comportamentos, incluindo os significados de governança, engajamento, cenário e pesquisa}
  \seealsoref{被}{bei4}
  \seealsoref{何}{he2}
  \seealsoref{所}{suo3}
  \end{Phonetics}
  \begin{Phonetics}{为}{wei4}[][HSK 2,3]
    \definition*{s.}{Sobrenome Wei}
    \definition{part.}{com 何 em uma pergunta retórica para expressar dúvida}
    \definition{prep.}{por; usado em frases passivas para introduzir o agente da ação, equivalente a 被 (frequentemente usado com 所)}
    \definition{suf.}{é anexado a alguns adjetivos ou advérbios monossilábicos para formar advérbios dissilábicos que expressam grau ou amplitude, geralmente modificando adjetivos ou verbos dissilábicos}
    \definition{v.}{fazer; agir | tornar-se; transformar-se em | ser;  significar | servir como; agir como; desempenhar o papel de | fazer; trabalhar; indica certas ações e comportamentos, incluindo os significados de governança, engajamento, cenário e pesquisa}
  \seealsoref{被}{bei4}
  \seealsoref{何}{he2}
  \seealsoref{所}{suo3}
  \end{Phonetics}
\end{Entry}

\begin{Entry}{为了}{4,2}{⼂、⼅}
  \begin{Phonetics}{为了}{wei4le5}[][HSK 3]
    \definition{prep.}{para; por causa de; a fim de; o objetivo da introdução de ações comportamentais}
  \end{Phonetics}
\end{Entry}

\begin{Entry}{为什么}{4,4,3}{⼂、⼈、⼃}
  \begin{Phonetics}{为什么}{wei4shen2me5}[][HSK 2]
    \definition{adv.}{por que?; por que é que?; como é que?;  nota: 为什么不 geralmente tem o significado de conselho, o mesmo que 何不}
  \seealsoref{何不}{he2bu4}
  \end{Phonetics}
\end{Entry}

\begin{Entry}{为止}{4,4}{⼂、⽌}
  \begin{Phonetics}{为止}{wei2 zhi3}[][HSK 5]
    \definition{adv.}{até; até um determinado momento}
  \end{Phonetics}
\end{Entry}

\begin{Entry}{为主}{4,5}{⼂、⼂}
  \begin{Phonetics}{为主}{wei2 zhu3}[][HSK 5]
    \definition{v.}{dar prioridade a; dar preferência a; dar importância a}
  \end{Phonetics}
\end{Entry}

\begin{Entry}{为此}{4,6}{⼂、⽌}
  \begin{Phonetics}{为此}{wei4 ci3}[][HSK 6]
    \definition{conj.}{portanto; para este fim; por esta razão; para este propósito; nesta conexão; contexto de conexão, indicando que o comportamento descrito é devido aos motivos mencionados anteriormente}
  \end{Phonetics}
\end{Entry}

\begin{Entry}{为何}{4,7}{⼂、⼈}
  \begin{Phonetics}{为何}{wei4 he2}[][HSK 6]
    \definition{adv.}{por que?; por qual razão?}
  \seealsoref{为什么}{wei4shen2me5}
  \end{Phonetics}
\end{Entry}

\begin{Entry}{为难}{4,10}{⼂、⾫}
  \begin{Phonetics}{为难}{wei2nan2}[][HSK 5]
    \definition{adj.}{envergonhado; sentir-se constrangido; sentir-se sobrecarregado; sentir-se incapaz de lidar com algo}
    \definition{v.}{dificultar as coisas para; dificultar; contrariar}
  \end{Phonetics}
\end{Entry}

\begin{Entry}{为期}{4,12}{⼂、⽉}
  \begin{Phonetics}{为期}{wei2qi1}[][HSK 5]
    \definition{s.}{Literário: tempo restante}
    \definition{v.}{Literário: a ser concluído (até uma data definida, por um determinado período de tempo)}
  \end{Phonetics}
\end{Entry}

\begin{Entry}{主}{5}{⼂}
  \begin{Phonetics}{主}{zhu3}
    \definition*{s.}{Deus; Senhor; o nome do Deus em que se acredita o cristianismo, o judaísmo, etc.}
    \definition{adj.}{principal; primário; o mais básico; o mais importante | de si mesmo; por vontade própria; próprio; do próprio}
    \definition[位,名,个]{s.}{anfitrião; alguém que convida e recebe convidados (oposto de 宾 e 客) | mestre; dono; uma pessoa que possui poder ou propriedade; uma pessoa em posição dominante | pessoa ou parte interessada | decisão; opinião; visão definitiva | placa espiritual (ou memorial)}
    \definition{v.}{dirigir; administrar; assumir o comando de; presidir; assumir a responsabilidade primária | decidir; reivindicar | significar; indicar; prever um certo resultado}
  \seealsoref{宾}{bin1}
  \seealsoref{客}{ke4}
  \end{Phonetics}
\end{Entry}

\begin{Entry}{主人}{5,2}{⼂、⼈}
  \begin{Phonetics}{主人}{zhu3ren2}[][HSK 2]
    \definition[个,位]{s.}{mestre; uma pessoa que empregava tutores, contadores, etc. antigamente; uma pessoa que empregava empregados domésticos | anfitrião; Aaguém que entretém convidados (em oposição a 客人) | proprietário; uma pessoa que possui um certo tipo de bens ou poder}
  \seealsoref{客人}{ke4ren2}
  \end{Phonetics}
\end{Entry}

\begin{Entry}{主义}{5,3}{⼂、⼂}
  \begin{Phonetics}{主义}{zhu3yi4}
    \definition[种]{s.}{doutrina; um determinado sistema social ou sistema político e econômico | estilo de pensamento; um certo ponto de vista ou estilo | ideologia; teorias e doutrinas sistemáticas sobre a natureza, a sociedade humana, etc.}
    \definition{suf.}{-ismo}
  \end{Phonetics}
\end{Entry}

\begin{Entry}{主办}{5,4}{⼂、⼒}
  \begin{Phonetics}{主办}{zhu3ban4}[][HSK 5]
    \definition{v.}{manter; hospedar; dirigir; patrocinar}
  \end{Phonetics}
\end{Entry}

\begin{Entry}{主任}{5,6}{⼂、⼈}
  \begin{Phonetics}{主任}{zhu3ren4}[][HSK 3]
    \definition[个,位,名]{s.}{chefe; diretor; presidente; o principal responsável por um departamento ou instituição}
  \end{Phonetics}
\end{Entry}

\begin{Entry}{主动}{5,6}{⼂、⼒}
  \begin{Phonetics}{主动}{zhu3dong4}[][HSK 3]
    \definition{adj.}{ativo; positivo; agir sem esperar por um impulso externo (em oposição a 被动) | iniciativo; capaz de impulsionar as coisas por vontade própria; capaz de criar uma situação favorável e fazer as coisas acontecerem de acordo com suas próprias intenções (em oposição a 被动)}
  \seealsoref{被动}{bei4dong4}
  \end{Phonetics}
\end{Entry}

\begin{Entry}{主导}{5,6}{⼂、⼨}
  \begin{Phonetics}{主导}{zhu3dao3}[][HSK 5]
    \definition{adj.}{líder; dominante; guiado; principais e guias para que as coisas se desenvolvam em uma determinada direção}
    \definition{s.}{fator principal (ou orientador)}
  \end{Phonetics}
\end{Entry}

\begin{Entry}{主观}{5,6}{⼂、⾒}
  \begin{Phonetics}{主观}{zhu3guan1}[][HSK 5]
    \definition{adj.}{subjetivo; não com base nas condições reais, mas com base nos próprios desejos | subjetivo; filosoficamente, refere-se à consciência e aos aspectos espirituais dos seres humanos}
  \end{Phonetics}
\end{Entry}

\begin{Entry}{主体}{5,7}{⼂、⼈}
  \begin{Phonetics}{主体}{zhu3 ti3}[][HSK 5]
    \definition[个,些,种,群]{s.}{corpo principal; parte principal; parte principal; esteio; a parte principal das coisas | Filosofia: sujeito}
  \end{Phonetics}
\end{Entry}

\begin{Entry}{主张}{5,7}{⼂、⼸}
  \begin{Phonetics}{主张}{zhu3zhang1}[][HSK 3]
    \definition[个,项,些,种]{s.}{vista; posição; proposição}
    \definition{v.}{defender; apoiar; manter; representar; ter uma opinião sobre como agir, fazer uma sugestão}
  \end{Phonetics}
\end{Entry}

\begin{Entry}{主角}{5,7}{⼂、⾓}
  \begin{Phonetics}{主角}{zhu3 jue2}[][HSK 6]
    \definition[个,位,名]{s.}{liderança; papel principal; protagonista; um papel importante em uma peça, filme, etc.; um ator que desempenha um papel importante | (figurado) algo que tem grande influência em uma determinada área; refere-se ao personagem principal}
  \end{Phonetics}
\end{Entry}

\begin{Entry}{主持}{5,9}{⼂、⼿}
  \begin{Phonetics}{主持}{zhu3chi2}[][HSK 3]
    \definition[位,名]{s.}{anfitrião; a pessoa responsável por administrar e lidar com uma determinada atividade}
    \definition{v.}{dirigir; administrar; assumir o comando; encarregar-se de; ser responsável por gerenciar, organizar uma determinada atividade ou lidar com um determinado assunto | defender; apoiar; preservar; manter}
  \end{Phonetics}
\end{Entry}

\begin{Entry}{主持人}{5,9,2}{⼂、⼿、⼈}
  \begin{Phonetics}{主持人}{zhu3 chi2 ren2}[][HSK 6]
    \definition[个,位]{s.}{anfitrião; âncora; apresentador}
  \end{Phonetics}
\end{Entry}

\begin{Entry}{主要}{5,9}{⼂、⾑}
  \begin{Phonetics}{主要}{zhu3yao4}[][HSK 2]
    \definition{adj.}{principal; chefe; o mais importante na questão; o decisivo | principal; núcleo; a raiz ou parte mais importante de algo}
  \end{Phonetics}
\end{Entry}

\begin{Entry}{主席}{5,10}{⼂、⼱}
  \begin{Phonetics}{主席}{zhu3xi2}[][HSK 4]
    \definition*{s.}{Presidente (da China)}
    \definition[个,位,名]{s.}{presidente, \emph{chairman} (de uma reunião) | chefe; presidente (de uma organização ou estado)}
  \end{Phonetics}
\end{Entry}

\begin{Entry}{主席台}{5,10,5}{⼂、⼱、⼝}
  \begin{Phonetics}{主席台}{zhu3xi2tai2}
    \definition[个]{s.}{plataforma | tribuna}
  \end{Phonetics}
\end{Entry}

\begin{Entry}{主席团}{5,10,6}{⼂、⼱、⼞}
  \begin{Phonetics}{主席团}{zhu3xi2tuan2}
    \definition{s.}{presídio}
  \end{Phonetics}
\end{Entry}

\begin{Entry}{主流}{5,10}{⼂、⽔}
  \begin{Phonetics}{主流}{zhu3liu2}[][HSK 6]
    \definition{s.}{corrente principal; corrente mãe; convencional | tendência principal; aspecto essencial ou principal; falando metaforicamente, os principais aspectos do desenvolvimento das coisas}
  \end{Phonetics}
\end{Entry}

\begin{Entry}{主意}{5,13}{⼂、⼼}
  \begin{Phonetics}{主意}{zhu3yi5}[][HSK 3]
    \definition[个,种]{s.}{ideia; plano; decisão; método}
  \end{Phonetics}
\end{Entry}

\begin{Entry}{主管}{5,14}{⼂、⽵}
  \begin{Phonetics}{主管}{zhu3guan3}[][HSK 5]
    \definition[位,名,个,些]{s.}{pessoa responsável, como supervisor, gerente, diretor, etc.}
    \definition{v.}{estar encarregado de; ser responsável por; ser o principal responsável pela gestão de um trabalho; assumir a responsabilidade primária pela gestão (um certo aspecto)}
  \end{Phonetics}
\end{Entry}

\begin{Entry}{主题}{5,15}{⼂、⾴}
  \begin{Phonetics}{主题}{zhu3ti2}[][HSK 4]
    \definition[个]{s.}{tema; assunto; motivo; lema; ideias básicas expressas em toda a obra de literatura e arte por meio de imagens artísticas concretas | pontos/conteúdos principais; referência geral ao conteúdo principal de artigos, discursos, conferências, etc.}
  \end{Phonetics}
\end{Entry}

\begin{Entry}{举}{9}{⼂}
  \begin{Phonetics}{举}{ju3}[][HSK 2]
    \definition*{s.}{Sobrenome Ju}
    \definition{adj.}{inteiro; completo}
    \definition{s.}{ato; ação; movimento; comportamento | (nas dinastias Ming e Qing) candidato aprovado nos exames imperiais a nível provincial}
    \definition{v.}{levantar; erguer; sustentar | começar; iniciar; surgir | eleger; escolher; recomendar; selecionar | citar; enumerar; propor; revelar}
  \end{Phonetics}
\end{Entry}

\begin{Entry}{举办}{9,4}{⼂、⼒}
  \begin{Phonetics}{举办}{ju3ban4}[][HSK 3]
    \definition{v.}{conduzir; organizar; realizar}
  \end{Phonetics}
\end{Entry}

\begin{Entry}{举手}{9,4}{⼂、⼿}
  \begin{Phonetics}{举手}{ju3 shou3}[][HSK 2]
    \definition{v.}{levantar a mão ou as mãos; levantar a mão para sinalizar ou responder a uma pergunta}
  \end{Phonetics}
\end{Entry}

\begin{Entry}{举动}{9,6}{⼂、⼒}
  \begin{Phonetics}{举动}{ju3dong4}[][HSK 5]
    \definition{s.}{ato; atividade; movimento; ação}
  \end{Phonetics}
\end{Entry}

\begin{Entry}{举行}{9,6}{⼂、⾏}
  \begin{Phonetics}{举行}{ju3xing2}[][HSK 2]
    \definition{v.}{realizar (uma reunião, cerimônia, etc.); realizar (atividades formais ou solenes)}
  \end{Phonetics}
\end{Entry}

%%%%% EOF %%%%%

