%%%
%%% Radical "⼂"
%%%

\section*{Radical 3: ``⼂'' (乀,乁)}\addcontentsline{toc}{section}{Radical 3: ⼂、乀,乁}

\begin{entry}{义务}{3,5}{⼂、⼒}
  \begin{phonetics}{义务}{yi4wu4}[][HSK 4]
    \definition{s.}{dever; obrigação; responsabilidades perante a lei, em oposição a 权利}
  \seealsoref{权利}{quan2li4}
  \end{phonetics}
\end{entry}

\begin{entry}{之一}{3,1}{⼂、⼀}
  \begin{phonetics}{之一}{zhi1 yi1}[][HSK 4]
    \definition{s.}{um de (algo); pertence a um ou a todo um grupo de coisas com as mesmas características}
  \end{phonetics}
\end{entry}

\begin{entry}{之下}{3,3}{⼂、⼀}
  \begin{phonetics}{之下}{zhi1 xia4}[][HSK 5]
    \definition{s.}{usado para indicar algo abaixo de um determinado intervalo, posição, grau, etc.; indica um aspecto inferior em termos de alcance, posição, status, nível, Chengdu, etc. | usado para indicar as condições sob as quais algo acontece | usado para indicar o humor, estado em que alguém faz algo; expressa um determinado comportamento em um determinado estado de espírito ou situação}
  \end{phonetics}
\end{entry}

\begin{entry}{之中}{3,4}{⼂、⼁}
  \begin{phonetics}{之中}{zhi1 zhong1}[][HSK 5]
    \definition{prep.}{em; no meio de; entre}
  \end{phonetics}
\end{entry}

\begin{entry}{之内}{3,4}{⼂、⼌}
  \begin{phonetics}{之内}{zhi1 nei4}[][HSK 5]
    \definition{adv.}{em; dentro de; indica dentro de um determinado intervalo, limite ou período de tempo, etc.}
  \end{phonetics}
\end{entry}

\begin{entry}{之外}{3,5}{⼂、⼣}
  \begin{phonetics}{之外}{zhi1 wai4}[][HSK 5]
    \definition{adv.}{lado de fora; exceto; além de; além disso; refere-se a algo que excede um determinado limite}
  \end{phonetics}
\end{entry}

\begin{entry}{之后}{3,6}{⼂、⼝}
  \begin{phonetics}{之后}{zhi1 hou4}[][HSK 4]
    \definition{adv.}{mais tarde; posteriormente; depois; desde então; para indicar que é depois de um determinado tempo ou de uma determinada coisa, 以后 é usado com frequência na linguagem falada; às vezes, também pode indicar que é depois de um determinado lugar ou local,  后面 é usado com frequência na linguagem falada}
  \seealsoref{后面}{hou4mian4}
  \seealsoref{以后}{yi3 hou4}
  \end{phonetics}
\end{entry}

\begin{entry}{之间}{3,7}{⼂、⾨}
  \begin{phonetics}{之间}{zhi1 jian1}[][HSK 4]
    \definition{prep.}{(depois de um substantivo) entre; dentro de duas delimitações de tempo, local ou quantitativas | colocado após certos verbos ou advérbios de duas sílabas para indicar um curto período de tempo}
  \end{phonetics}
\end{entry}

\begin{entry}{之前}{3,9}{⼂、⼑}
  \begin{phonetics}{之前}{zhi1 qian2}[][HSK 4]
    \definition{adv.}{(referindo-se ao tempo) antes, antes de, atrás | (referindo-se ao local físico) na frente de | (usado independentemente) no passado, antes disso}
  \end{phonetics}
\end{entry}

\begin{entry}{为}{4}{⼂}
  \begin{phonetics}{为}{wei2}[][HSK 3]
    \definition*{s.}{sobrenome Wei}
    \definition{part.}{frequentemente usado com 何 para expressar dúvida}
    \definition{prep.}{como (na capacidade de) | por (na voz passiva)}
    \definition{suf.}{anexado a certos adjetivos monossilábicos, indicando grau ou alcançe | anexado a certos advérbios de grau para fortalecer o tom}
    \definition{v.}{fazer; agir | servir como; agir como; desempenhar o papel de | tornar-se; transformar-se em | ser; significar}
  \seealsoref{何}{he2}
  \end{phonetics}
  \begin{phonetics}{为}{wei4}[][HSK 2,3]
    \definition{prep.}{objeto da ação | indicando propósito | indicando razões | para; em direção a}
    \definition{v.}{apoiar; defender}
  \end{phonetics}
\end{entry}

\begin{entry}{为了}{4,2}{⼂、⼅}
  \begin{phonetics}{为了}{wei4le5}[][HSK 3]
    \definition{conj.}{para; por causa de; a fim de}
  \end{phonetics}
\end{entry}

\begin{entry}{为什么}{4,4,3}{⼂、⼈、⼃}
  \begin{phonetics}{为什么}{wei4shen2me5}[][HSK 2]
    \definition{adv.}{por que?}
  \end{phonetics}
\end{entry}

\begin{entry}{为止}{4,4}{⼂、⽌}
  \begin{phonetics}{为止}{wei2 zhi3}[][HSK 5]
    \definition{adv.}{até; até um determinado momento}
  \end{phonetics}
\end{entry}

\begin{entry}{为主}{4,5}{⼂、⼂}
  \begin{phonetics}{为主}{wei2 zhu3}[][HSK 5]
    \definition{v.}{dar prioridade a; dar preferência a; dar importância a}
  \end{phonetics}
\end{entry}

\begin{entry}{为难}{4,10}{⼂、⾫}
  \begin{phonetics}{为难}{wei2nan2}[][HSK 5]
    \definition{adj.}{envergonhado; sentir-se constrangido; sentir-se sobrecarregado; sentir-se incapaz de lidar com algo}
    \definition{v.}{dificultar as coisas para; dificultar; contrariar}
  \end{phonetics}
\end{entry}

\begin{entry}{为期}{4,12}{⼂、⽉}
  \begin{phonetics}{为期}{wei2qi1}[][HSK 5]
    \definition{s.}{tempo restante}
    \definition{v.}{a ser concluído (até uma data definida, por um determinado período de tempo)}
  \end{phonetics}
\end{entry}

\begin{entry}{主人}{5,2}{⼂、⼈}
  \begin{phonetics}{主人}{zhu3ren2}[][HSK 2]
    \definition[个,位]{s.}{mestre | anfitrião | proprietário | uma pessoa que tem um certo tipo de bens ou poder}
  \end{phonetics}
\end{entry}

\begin{entry}{主义}{5,3}{⼂、⼂}
  \begin{phonetics}{主义}{zhu3yi4}
    \definition{s.}{ideologia}
    \definition{suf.}{-ismo}
  \end{phonetics}
\end{entry}

\begin{entry}{主办}{5,4}{⼂、⼒}
  \begin{phonetics}{主办}{zhu3ban4}[][HSK 5]
    \definition{v.}{manter; hospedar; dirigir; patrocinar}
  \end{phonetics}
\end{entry}

\begin{entry}{主任}{5,6}{⼂、⼈}
  \begin{phonetics}{主任}{zhu3ren4}[][HSK 3]
    \definition[个,位]{s.}{chefe; diretor; presidente; o chefe de um departamento ou organização}
  \end{phonetics}
\end{entry}

\begin{entry}{主动}{5,6}{⼂、⼒}
  \begin{phonetics}{主动}{zhu3dong4}[][HSK 3]
    \definition{adj.}{ativo; positivo; agir sem ser pressionado por forças externas (em vez de ser 被动) | iniciativo; capaz de impulsionar as coisas por vontade própria; capaz de criar uma situação favorável e fazer as coisas acontecerem de acordo com suas próprias intenções (em oposição a 被动)}
  \seealsoref{被动}{bei4dong4}
  \end{phonetics}
\end{entry}

\begin{entry}{主导}{5,6}{⼂、⼨}
  \begin{phonetics}{主导}{zhu3dao3}[][HSK 5]
    \definition{adj.}{líder; dominante; guiado; principais e guias para que as coisas se desenvolvam em uma determinada direção}
    \definition{s.}{fator principal (ou orientador)}
  \end{phonetics}
\end{entry}

\begin{entry}{主观}{5,6}{⼂、⾒}
  \begin{phonetics}{主观}{zhu3guan1}[][HSK 5]
    \definition{adj.}{subjetivo; não com base nas condições reais, mas com base nos próprios desejos | subjetivo; filosoficamente, refere-se à consciência e aos aspectos espirituais dos seres humanos}
  \end{phonetics}
\end{entry}

\begin{entry}{主体}{5,7}{⼂、⼈}
  \begin{phonetics}{主体}{zhu3 ti3}[][HSK 5]
    \definition{s.}{corpo principal; parte principal; parte principal; esteio; a parte principal das coisas | (filosofia) sujeito}
  \end{phonetics}
\end{entry}

\begin{entry}{主张}{5,7}{⼂、⼸}
  \begin{phonetics}{主张}{zhu3zhang1}[][HSK 3]
    \definition[个]{s.}{vista; posição; proposição}
    \definition{v.}{segurar; advogar; manter; defender}
  \end{phonetics}
\end{entry}

\begin{entry}{主持}{5,9}{⼂、⼿}
  \begin{phonetics}{主持}{zhu3chi2}[][HSK 3]
    \definition{s.}{anfitrião; uma pessoa que é responsável ou lida com uma atividade}
    \definition{v.}{dirigir; gerir; tomar conta de | defender, manter}
  \end{phonetics}
\end{entry}

\begin{entry}{主要}{5,9}{⼂、⾑}
  \begin{phonetics}{主要}{zhu3yao4}[][HSK 2]
    \definition{adj.}{principal}
  \end{phonetics}
\end{entry}

\begin{entry}{主席}{5,10}{⼂、⼱}
  \begin{phonetics}{主席}{zhu3xi2}[][HSK 4]
    \definition*[个,位]{s.}{Presidente (da China)}
    \definition[个,位]{s.}{presidente, \emph{chairman}  (de uma reunião) |
chefe; presidente (de uma organização ou estado)}
  \end{phonetics}
\end{entry}

\begin{entry}{主席台}{5,10,5}{⼂、⼱、⼝}
  \begin{phonetics}{主席台}{zhu3xi2tai2}
    \definition[个]{s.}{plataforma | tribuna}
  \end{phonetics}
\end{entry}

\begin{entry}{主席团}{5,10,6}{⼂、⼱、⼞}
  \begin{phonetics}{主席团}{zhu3xi2tuan2}
    \definition{s.}{presídio}
  \end{phonetics}
\end{entry}

\begin{entry}{主意}{5,13}{⼂、⼼}
  \begin{phonetics}{主意}{zhu3yi5}[][HSK 3]
    \definition[个]{s.}{ideia; plano; decisão}
  \end{phonetics}
\end{entry}

\begin{entry}{主管}{5,14}{⼂、⽵}
  \begin{phonetics}{主管}{zhu3guan3}[][HSK 5]
    \definition[门]{s.}{pessoa responsável, como supervisor, gerente, diretor, etc.}
    \definition{v.}{estar encarregado de; ser responsável por; ser o principal responsável pela gestão de um trabalho; assumir a responsabilidade primária pela gestão (um certo aspecto)}
  \end{phonetics}
\end{entry}

\begin{entry}{主题}{5,15}{⼂、⾴}
  \begin{phonetics}{主题}{zhu3ti2}[][HSK 4]
    \definition[个]{s.}{tema; assunto; motivo; lema; ideias básicas expressas em toda a obra de literatura e arte por meio de imagens artísticas concretas | pontos/conteúdos principais; referência geral ao conteúdo principal de artigos, discursos, conferências, etc.}
  \end{phonetics}
\end{entry}

\begin{entry}{举}{9}{⼂}
  \begin{phonetics}{举}{ju3}[][HSK 2]
    \definition*{s.}{sobrenome Ju}
    \definition{adj.}{inteiro; completo}
    \definition{s.}{ato; ação; movimento; comportamento | (nas dinastias Ming e Qing) candidato aprovado nos exames imperiais a nível provincial}
    \definition{v.}{levantar; erguer; sustentar | começar; iniciar; surgir | eleger; escolher; recomendar; selecionar | citar; enumerar; propor; revelar}
  \end{phonetics}
\end{entry}

\begin{entry}{举办}{9,4}{⼂、⼒}
  \begin{phonetics}{举办}{ju3ban4}[][HSK 3]
    \definition{v.}{segurar; conduzir}
  \end{phonetics}
\end{entry}

\begin{entry}{举手}{9,4}{⼂、⼿}
  \begin{phonetics}{举手}{ju3 shou3}[][HSK 2]
    \definition{v.}{levantar a mão ou as mãos; levantar a mão para sinalizar ou responder a uma pergunta}
  \end{phonetics}
\end{entry}

\begin{entry}{举动}{9,6}{⼂、⼒}
  \begin{phonetics}{举动}{ju3dong4}[][HSK 5]
    \definition{s.}{ato; atividade; movimento; ação}
  \end{phonetics}
\end{entry}

\begin{entry}{举行}{9,6}{⼂、⾏}
  \begin{phonetics}{举行}{ju3xing2}[][HSK 2]
    \definition{v.}{realizar (uma reunião, cerimônia, etc.); realizar (atividades formais ou solenes)}
  \end{phonetics}
\end{entry}

%%%%% EOF %%%%%

