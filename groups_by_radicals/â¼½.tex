%%%
%%% Radical "⼽"
%%%

\section*{Radical 62: ``⼽''}\addcontentsline{toc}{section}{Radical 62: ⼽}

\begin{Entry}{戈}{4}{⼽}[Kangxi 62]
  \begin{Phonetics}{戈}{ge1}
    \definition*{s.}{Sobrenome Ge}
    \definition{s.}{Arcaico: machado-adaga (com cabo longo e lâmina horizontal) | punhal-machado; arma antiga, lâmina cruzada, feita de bronze ou ferro, com cabo longo}
  \end{Phonetics}
\end{Entry}

\begin{Entry}{戈壁}{4,16}{⼽、⼟}
  \begin{Phonetics}{戈壁}{ge1bi4}[][HSK 7-9]
    \definition*{s.}{Deserto de Gobi}
    \definition{s.}{deserto; refere-se a uma área desértica onde o solo é quase coberto por areia grossa e cascalho e há poucas plantas}
  \end{Phonetics}
\end{Entry}

\begin{Entry}{戏}{6}{⼽}
  \begin{Phonetics}{戏}{xi4}[][HSK 5]
    \definition*{s.}{Sobrenome Xi}
    \definition[场,部,出,台]{s.}{drama; peça; espetáculo; \emph{show}}
    \definition{v.}{brincar; praticar esportes; jogar | zombar; brincar; provocar}
  \end{Phonetics}
\end{Entry}

\begin{Entry}{戏曲}{6,6}{⼽、⽈}
  \begin{Phonetics}{戏曲}{xi4 qu3}[][HSK 6]
    \definition{s.}{drama; ópera chinesa; ópera tradicional; forma teatral tradicional | partes cantadas em 传奇 e zaju 杂剧}
  \seealsoref{传奇}{chuan2qi2}
  \seealsoref{杂剧}{za2ju4}
  \end{Phonetics}
\end{Entry}

\begin{Entry}{戏弄}{6,7}{⼽、⼶}
  \begin{Phonetics}{戏弄}{xi4nong4}
    \definition{v.}{zombar de | pregar peças | provocar}
  \end{Phonetics}
\end{Entry}

\begin{Entry}{戏法}{6,8}{⼽、⽔}
  \begin{Phonetics}{戏法}{xi4fa3}
    \definition{s.}{truque de mágica | prestidigitação}
  \end{Phonetics}
\end{Entry}

\begin{Entry}{戏耍}{6,9}{⼽、⽽}
  \begin{Phonetics}{戏耍}{xi4shua3}
    \definition{v.}{divertir-me | brincar com | provocar}
  \end{Phonetics}
\end{Entry}

\begin{Entry}{戏院}{6,9}{⼽、⾩}
  \begin{Phonetics}{戏院}{xi4yuan4}
    \definition{s.}{teatro}
  \end{Phonetics}
\end{Entry}

\begin{Entry}{戏剧}{6,10}{⼽、⼑}
  \begin{Phonetics}{戏剧}{xi4ju4}[][HSK 5]
    \definition[出,部]{s.}{drama; peça; teatro | roteiro; peça; cenário}
  \end{Phonetics}
\end{Entry}

\begin{Entry}{戏剧化地}{6,10,4,6}{⼽、⼑、⼔、⼟}
  \begin{Phonetics}{戏剧化地}{xi4ju4hua4di4}
    \definition{adv.}{dramaticamente | teatralmente}
  \end{Phonetics}
\end{Entry}

\begin{Entry}{戏剧性}{6,10,8}{⼽、⼑、⼼}
  \begin{Phonetics}{戏剧性}{xi4ju4xing4}
    \definition{adj.}{dramático}
  \end{Phonetics}
\end{Entry}

\begin{Entry}{戏剧家}{6,10,10}{⼽、⼑、⼧}
  \begin{Phonetics}{戏剧家}{xi4ju4jia1}
    \definition{s.}{dramaturgo}
  \end{Phonetics}
\end{Entry}

\begin{Entry}{戏剧效果}{6,10,10,8}{⼽、⼑、⽁、⽊}
  \begin{Phonetics}{戏剧效果}{xi4ju4xiao4guo3}
    \definition{s.}{efeito dramático}
  \end{Phonetics}
\end{Entry}

\begin{Entry}{戏剧般}{6,10,10}{⼽、⼑、⾈}
  \begin{Phonetics}{戏剧般}{xi4ju4ban1}
    \definition{adj.}{melodramático}
  \end{Phonetics}
\end{Entry}

\begin{Entry}{戏剧编剧}{6,10,12,10}{⼽、⼑、⽷、⼑}
  \begin{Phonetics}{戏剧编剧}{xi4ju4bian1ju4}
    \definition{s.}{dramaturgo}
  \end{Phonetics}
\end{Entry}

\begin{Entry}{戏剧演出}{6,10,14,5}{⼽、⼑、⽔、⼐}
  \begin{Phonetics}{戏剧演出}{xi4ju4yan3chu1}
    \definition{s.}{performance dramática}
  \end{Phonetics}
\end{Entry}

\begin{Entry}{戏谑}{6,11}{⼽、⾔}
  \begin{Phonetics}{戏谑}{xi4xue4}
    \definition{v.}{brincar | fazer piadas | ridicularizar}
  \end{Phonetics}
\end{Entry}

\begin{Entry}{成}{6}{⼽}
  \begin{Phonetics}{成}{cheng2}[][HSK 2,6]
    \definition*{s.}{Sobrenome Cheng}
    \definition{adj.}{capaz; competente | totalmente crescido; totalmente desenvolvido; maduro | estabelecido; Já definido; pronto para uso | em números ou quantidades consideráveis; inteiro; suficiente: enfatiza a quantidade ou a duração}
    \definition{clas.}{um décimo}
    \definition{interj.}{O.K.; tudo bem}
    \definition{s.}{resultado; conquista}
    \definition{v.}{ter sucesso; conseguir; ser bem-sucedido | tornar-se; transformar-se | ajudar a completar; realizar}
  \end{Phonetics}
\end{Entry}

\begin{Entry}{成人}{6,2}{⼽、⼈}
  \begin{Phonetics}{成人}{cheng2ren2}[][HSK 4]
    \definition[个,名,位]{s.}{adulto; crescido; pessoa adulta}
    \definition{v.}{crescer; tornar-se adulto}
  \end{Phonetics}
\end{Entry}

\begin{Entry}{成千上万}{6,3,3,3}{⼽、⼗、⼀、⼀}
  \begin{Phonetics}{成千上万}{cheng2qian1-shang4wan4}[][HSK 7-9]
    \definition{expr.}{aos milhares e dezenas de milhares; números incontáveis; descreve um grande número, também escrito como 成千成万 ou 成千累万 | milhares e dezenas de milhares; milhares e milhares}
  \seealsoref{成千成万}{cheng2qian1-cheng2wan4}
  \seealsoref{成千累万}{cheng2qian1-lei3wan4}
  \end{Phonetics}
\end{Entry}

\begin{Entry}{成千成万}{6,3,6,3}{⼽、⼗、⼽、⼀}
  \begin{Phonetics}{成千成万}{cheng2qian1-cheng2wan4}
    \definition{expr.}{milhares e dezenas de milhares; milhares e milhares | inumeráveis | Literário: aos milhares e dezenas de milhares; números incontáveis}
  \seealsoref{成千累万}{cheng2qian1-lei3wan4}
  \seealsoref{成千上万}{cheng2qian1-shang4wan4}
  \end{Phonetics}
\end{Entry}

\begin{Entry}{成千累万}{6,3,11,3}{⼽、⼗、⽷、⼀}
  \begin{Phonetics}{成千累万}{cheng2qian1-lei3wan4}
    \definition{expr.}{milhares e dezenas de milhares | milhares e milhares; inumeráveis | Literário: aos milhares e dezenas de milhares; números incontáveis}
  \seealsoref{成千成万}{cheng2qian1-cheng2wan4}
  \seealsoref{成千上万}{cheng2qian1-shang4wan4}
  \end{Phonetics}
\end{Entry}

\begin{Entry}{成才}{6,3}{⼽、⼿}
  \begin{Phonetics}{成才}{cheng2cai2}[][HSK 7-9]
    \definition{v.}{tornar-se uma pessoa útil | tornar-se uma pessoa digna de respeito | fazer algo de si mesmo}
  \end{Phonetics}
\end{Entry}

\begin{Entry}{成为}{6,4}{⼽、⼂}
  \begin{Phonetics}{成为}{cheng2wei2}[][HSK 2]
    \definition{v.}{tornar-se; transformar-se; revelar-se; passar de uma situação, identidade ou estado para outro}
  \end{Phonetics}
\end{Entry}

\begin{Entry}{成分}{6,4}{⼽、⼑}
  \begin{Phonetics}{成分}{cheng2fen4}[][HSK 6]
    \definition[个,些,种]{s.}{composição; ingrediente; elemento; parte componente; as várias substâncias ou fatores que compõem as coisas | a condição de classe de alguém; a profissão ou a condição econômica de alguém; refere-se à classe à qual uma família pertence; à principal experiência ou ocupação anterior de uma pessoa}
  \end{Phonetics}
\end{Entry}

\begin{Entry}{成天}{6,4}{⼽、⼤}
  \begin{Phonetics}{成天}{cheng2tian1}[][HSK 7-9]
    \definition{adv.}{o dia todo; o tempo todo}
  \end{Phonetics}
\end{Entry}

\begin{Entry}{成长}{6,4}{⼽、⾧}
  \begin{Phonetics}{成长}{cheng2zhang3}[][HSK 3]
    \definition{v.}{crescer; amadurecer; tornar-se adulto; o desenvolvimento de seres humanos, animais ou plantas desde a infância até a maturidade}
  \end{Phonetics}
\end{Entry}

\begin{Entry}{成功}{6,5}{⼽、⼒}
  \begin{Phonetics}{成功}{cheng2gong1}[][HSK 3]
    \definition{adj.}{bem-sucedido; frutífero}
    \definition[个,次]{s.}{sucesso}
    \definition{v.}{ter sucesso; obter os resultados esperados}
  \end{Phonetics}
\end{Entry}

\begin{Entry}{成本}{6,5}{⼽、⽊}
  \begin{Phonetics}{成本}{cheng2ben3}[][HSK 5]
    \definition{s.}{custo principal; custo; custo capitalizado; custo final; primeiro custo; custo próprio; custo de produção de um produto; inclui o custo dos materiais de produção consumidos durante o processo produtivo e a remuneração paga aos trabalhadores}
  \end{Phonetics}
\end{Entry}

\begin{Entry}{成立}{6,5}{⼽、⽴}
  \begin{Phonetics}{成立}{cheng2li4}[][HSK 3]
    \definition{v.}{fundar; estabelecer; criar; (organizações, instituições, etc.) começar a existir e a funcionar | ser válido; ser sustentável; fazer sentido; (teorias, pontos de vista, razões, etc.) fundamentados e válidos}
  \end{Phonetics}
\end{Entry}

\begin{Entry}{成交}{6,6}{⼽、⼇}
  \begin{Phonetics}{成交}{cheng2/jiao1}[][HSK 5]
    \definition{v.+compl.}{fechar um acordo; fazer uma barganha; concluir uma transação}
  \end{Phonetics}
\end{Entry}

\begin{Entry}{成吉思汗}{6,6,9,6}{⼽、⼝、⼼、⽔}
  \begin{Phonetics}{成吉思汗}{cheng2ji2si1han2}
    \definition*{s.}{Genghis Khan (1162-1227), fundador e governante do Império Mongol}
  \end{Phonetics}
\end{Entry}

\begin{Entry}{成年}{6,6}{⼽、⼲}
  \begin{Phonetics}{成年}{cheng2nian2}[][HSK 7-9]
    \definition{adv.}{o ano todo; durante todo o ano}
    \definition{v.}{atingir a maioridade (ser humano, animal, madeira); refere-se à idade em que uma pessoa atinge a maturidade, ou ao período em que animais superiores ou árvores atingem a maturidade}
  \end{Phonetics}
\end{Entry}

\begin{Entry}{成色}{6,6}{⼽、⾊}
  \begin{Phonetics}{成色}{cheng2se4}
    \definition{v.}{sair-se bem | ser bem sucedido}
  \end{Phonetics}
\end{Entry}

\begin{Entry}{成问题}{6,6,15}{⼽、⾨、⾴}
  \begin{Phonetics}{成问题}{cheng2wen4ti2}[][HSK 7-9]
    \definition{v.}{Coloquial: ser um problema; estar aberto a questionamentos (ou dúvidas, objeções)}
  \end{Phonetics}
\end{Entry}

\begin{Entry}{成员}{6,7}{⼽、⼝}
  \begin{Phonetics}{成员}{cheng2yuan2}[][HSK 3]
    \definition[个,些,名,位]{s.}{membro; membros de um grupo ou família}
  \end{Phonetics}
\end{Entry}

\begin{Entry}{成批}{6,7}{⼽、⼿}
  \begin{Phonetics}{成批}{cheng2pi1}
    \definition{s.}{em lotes | a granel}
  \end{Phonetics}
\end{Entry}

\begin{Entry}{成果}{6,8}{⼽、⽊}
  \begin{Phonetics}{成果}{cheng2guo3}[][HSK 3]
    \definition[个]{s.}{realização; resultado; conquista; recompensas no trabalho ou na carreira}
  \end{Phonetics}
\end{Entry}

\begin{Entry}{成品}{6,9}{⼽、⼝}
  \begin{Phonetics}{成品}{cheng2 pin3}[][HSK 6]
    \definition[批]{s.}{produto final; produto acabado; produto processado e pronto para ser fornecido}
  \end{Phonetics}
\end{Entry}

\begin{Entry}{成型}{6,9}{⼽、⼟}
  \begin{Phonetics}{成型}{cheng2xing2}[][HSK 7-9]
    \definition{v.}{(peças ou produtos) estar em forma acabada; assumir a forma necessária}
  \end{Phonetics}
\end{Entry}

\begin{Entry}{成活}{6,9}{⼽、⽔}
  \begin{Phonetics}{成活}{cheng2huo2}
    \definition{v.}{sobreviver}
  \end{Phonetics}
\end{Entry}

\begin{Entry}{成语}{6,9}{⼽、⾔}
  \begin{Phonetics}{成语}{cheng2yu3}[][HSK 5]
    \definition[条,则,句,个]{s.}{expressão idiomática; frase de conjunto (frases de quatro caracteres em chinês, geralmente com alusões literárias)}
  \end{Phonetics}
\end{Entry}

\begin{Entry}{成家}{6,10}{⼽、⼧}
  \begin{Phonetics}{成家}{cheng2/jia1}[][HSK 7-9]
    \definition{v.+compl.}{(um homem) casar; (um homem) estabelecer-se e casar-se | tornar-se um especialista (ou \emph{expert}); tornar-se um especialista reconhecido}
  \end{Phonetics}
\end{Entry}

\begin{Entry}{成效}{6,10}{⼽、⽁}
  \begin{Phonetics}{成效}{cheng2xiao4}[][HSK 5]
    \definition{s.}{efeito; resultado}
  \end{Phonetics}
\end{Entry}

\begin{Entry}{成都}{6,10}{⼽、⾢}
  \begin{Phonetics}{成都}{cheng2du1}
    \definition*{s.}{Chengdu}
  \end{Phonetics}
\end{Entry}

\begin{Entry}{成婚}{6,11}{⼽、⼥}
  \begin{Phonetics}{成婚}{cheng2hun1}
    \definition{v.}{casar-se}
  \end{Phonetics}
\end{Entry}

\begin{Entry}{成绩}{6,11}{⼽、⽷}
  \begin{Phonetics}{成绩}{cheng2ji4}[][HSK 2]
    \definition[项,个]{s.}{realização; sucesso; resultado (de trabalho ou estudo); refere-se à pontuação obtida em exames e competições; classificação, também se refere aos resultados alcançados no trabalho}
  \end{Phonetics}
\end{Entry}

\begin{Entry}{成就}{6,12}{⼽、⼪}
  \begin{Phonetics}{成就}{cheng2jiu4}[][HSK 3]
    \definition[个,项]{s.}{realização; conquista; sucesso; realizações profissionais}
    \definition{v.}{realizar; alcançar; completar; concluir (carreira)}
  \end{Phonetics}
\end{Entry}

\begin{Entry}{成群结对}{6,13,9,5}{⼽、⽺、⽷、⼨}
  \begin{Phonetics}{成群结对}{cheng2qun2-jie2dui4}[][HSK 7-9]
    \definition{expr.}{em grupos}
  \end{Phonetics}
\end{Entry}

\begin{Entry}{成熟}{6,15}{⼽、⽕}
  \begin{Phonetics}{成熟}{cheng2shu2}[][HSK 3]
    \definition{adj.}{maduro; amadurecido; totalmente desenvolvido; descreve que as oportunidades, condições, etc. estão perfeitas e que não haverá nenhum problema}
    \definition{v.}{amadurecer; atingir a maturidade; estar totalmente desenvolvido; frutas e outros frutos totalmente maduros, referindo-se ao desenvolvimento completo de organismos vivos}
  \end{Phonetics}
\end{Entry}

\begin{Entry}{成器}{6,16}{⼽、⼝}
  \begin{Phonetics}{成器}{cheng2qi4}
    \definition{v.}{tornar-se uma pessoa digna de respeito | fazer algo de si mesmo}
  \end{Phonetics}
\end{Entry}

\begin{Entry}{我}{7}{⼽}
  \begin{Phonetics}{我}{wo3}[][HSK 1]
    \definition{pron.}{eu; mim | um; qualquer um; usado para contrastar 他 e 我; refere-se a muitas pessoas em geral}
  \seealsoref{他}{ta1}
  \end{Phonetics}
\end{Entry}

\begin{Entry}{我们}{7,5}{⼽、⼈}
  \begin{Phonetics}{我们}{wo3men5}[][HSK 1]
    \definition{pron.}{nós; nos}
  \end{Phonetics}
\end{Entry}

\begin{Entry}{我们的}{7,5,8}{⼽、⼈、⽩}
  \begin{Phonetics}{我们的}{wo3men5 de5}
    \definition{pron.}{nosso, nossos}
  \end{Phonetics}
\end{Entry}

\begin{Entry}{我去}{7,5}{⼽、⼛}
  \begin{Phonetics}{我去}{wo3qu4}
    \definition{interj.}{(gíria) O que\dots!! | Oh meu Deus! | Isso é insano!}
  \end{Phonetics}
\end{Entry}

\begin{Entry}{我的}{7,8}{⼽、⽩}
  \begin{Phonetics}{我的}{wo3 de5}
    \definition{pron.}{meu, meus}
  \end{Phonetics}
\end{Entry}

\begin{Entry}{戒}{7}{⼽}
  \begin{Phonetics}{戒}{jie4}[][HSK 5]
    \definition[个,枚]{s.}{advertência; exortação | disciplina monástica budista; preceitos budistas | anel (dedo)}
    \definition{v.}{proteger-se contra; estar preparado; estar atento | advertir; exortar; admoestar | abandonar; parar; desistir; desistir (de um hábito ruim)}
  \end{Phonetics}
\end{Entry}

\begin{Entry}{或}{8}{⼽}
  \begin{Phonetics}{或}{huo4}[][HSK 2]
    \definition{adv.}{talvez; possivelmente; provavelmente | (geralmente na forma negativa) um pouco; ligeiramente}
    \definition{conj.}{ou (indicando escolha); ou\dots ou\dots}
    \definition{pron.}{alguém; algumas pessoas; refere-se a alguém ou algo, equivalente a 有人 ou 有的}
  \seealsoref{有的}{you3 de5}
  \seealsoref{有人}{you3 ren2}
  \end{Phonetics}
\end{Entry}

\begin{Entry}{或多或少}{8,6,8,4}{⼽、⼣、⼽、⼩}
  \begin{Phonetics}{或多或少}{huo4duo1-huo4shao3}[][HSK 7-9]
    \definition{expr.}{mais ou menos}
  \end{Phonetics}
\end{Entry}

\begin{Entry}{或许}{8,6}{⼽、⾔}
  \begin{Phonetics}{或许}{huo4xu3}[][HSK 4]
    \definition{adv.}{talvez; possivelmente; receio; não tenho certeza}
  \end{Phonetics}
\end{Entry}

\begin{Entry}{或者}{8,8}{⼽、⽼}
  \begin{Phonetics}{或者}{huo4zhe3}[][HSK 2]
    \definition{adv.}{talvez; possivelmente}
    \definition{conj.}{ou (usado em expressões afirmativas); ou\dots ou\dots; usado em frases narrativas para indicar uma relação de escolha | ou (usado para indicar equação); indica relação de equivalência, indicando que os objetos anterior e posterior são iguais}
  \end{Phonetics}
\end{Entry}

\begin{Entry}{或是}{8,9}{⼽、⽇}
  \begin{Phonetics}{或是}{huo4 shi4}[][HSK 5]
    \definition{adv.}{um ou outro; o outro}
    \definition{conj.}{ou; às vezes, é apenas uma de duas coisas}
  \end{Phonetics}
\end{Entry}

\begin{Entry}{战}{9}{⼽}
  \begin{Phonetics}{战}{zhan4}
    \definition{s.}{luta | guerra | batalha}
    \definition{v.}{lutar}
  \end{Phonetics}
\end{Entry}

\begin{Entry}{战士}{9,3}{⼽、⼠}
  \begin{Phonetics}{战士}{zhan4shi4}[][HSK 4]
    \definition[个,些,名,位]{s.}{soldado; membros mais jovens do exército | campeão; guerreiro; lutador; geralmente, uma pessoa que se engaja em alguma causa justa ou participa de alguma luta justa}
  \end{Phonetics}
\end{Entry}

\begin{Entry}{战友}{9,4}{⼽、⼜}
  \begin{Phonetics}{战友}{zhan4 you3}[][HSK 6]
    \definition{s.}{camarada de armas; companheiro de guerra | companheiro de batalha; pessoas lutando juntas}
  \end{Phonetics}
\end{Entry}

\begin{Entry}{战斗}{9,4}{⼽、⽃}
  \begin{Phonetics}{战斗}{zhan4dou4}[][HSK 4]
    \definition[场,次]{s.}{luta; batalha; combate; ação; conflito armado entre as partes oponentes}
    \definition{v.}{lutar; combater | lutar; metáfora para trabalho duro ou labor}
  \end{Phonetics}
\end{Entry}

\begin{Entry}{战术}{9,5}{⼽、⽊}
  \begin{Phonetics}{战术}{zhan4shu4}[][HSK 6]
    \definition[种,套]{s.}{táticas para resolver problemas; geralmente se refere ao método de resolução de problemas locais | táticas (militares); princípios e métodos de condução de combate}
  \end{Phonetics}
\end{Entry}

\begin{Entry}{战争}{9,6}{⼽、⼑}
  \begin{Phonetics}{战争}{zhan4zheng1}[][HSK 4]
    \definition[场,次]{s.}{guerra; conflito; luta armada entre povos, entre nações, entre classes ou entre grupos políticos}
  \end{Phonetics}
\end{Entry}

\begin{Entry}{战场}{9,6}{⼽、⼟}
  \begin{Phonetics}{战场}{zhan4 chang3}[][HSK 6]
    \definition[个,片,处]{s.}{campo de batalha; frente de batalha}
  \end{Phonetics}
\end{Entry}

\begin{Entry}{战胜}{9,9}{⼽、⾁}
  \begin{Phonetics}{战胜}{zhan4 sheng4}[][HSK 4]
    \definition{v.}{derrotar; vencer; superar; triunfar sobre; metáfora para superar dificuldades e alcançar o sucesso}
  \end{Phonetics}
\end{Entry}

\begin{Entry}{战略}{9,11}{⼽、⽥}
  \begin{Phonetics}{战略}{zhan4lve4}[][HSK 6]
    \definition[个,条]{s.}{estratégia; uma estratégia que orienta todo o processo de guerra (diferente de 战术) | estratégia; refere-se a uma diretriz geral}
  \seealsoref{战术}{zhan4shu4}
  \end{Phonetics}
\end{Entry}

\begin{Entry}{截}{14}{⼽}
  \begin{Phonetics}{截}{jie2}
    \definition{clas.}{seção; pedaço; comprimento}
    \definition{prep.}{por (um tempo especificado); até}
    \definition{v.}{cortar; romper | parar; verificar; interromper; interceptar}
  \end{Phonetics}
\end{Entry}

\begin{Entry}{截止}{14,4}{⼽、⽌}
  \begin{Phonetics}{截止}{jie2zhi3}[][HSK 6]
    \definition{adv.}{até (um certo limite de tempo); por (um tempo especificado)}
  \end{Phonetics}
\end{Entry}

\begin{Entry}{截至}{14,6}{⼽、⾄}
  \begin{Phonetics}{截至}{jie2zhi4}[][HSK 6]
    \definition{adv.}{a partir de; até (um certo limite de tempo); por (um tempo especificado)}
  \end{Phonetics}
\end{Entry}

\begin{Entry}{戴}{17}{⼽}
  \begin{Phonetics}{戴}{dai4}[][HSK 4]
    \definition*{s.}{Sobrenome Dai}
    \definition{v.}{usar/vestir (óculos, gravata, relógio de pulso, luvas); colocar objetos em sua cabeça, rosto, pescoço, peito, braços etc. | honrar; respeitar;}
  \end{Phonetics}
\end{Entry}

\begin{Entry}{戳}{18}{⼽}
  \begin{Phonetics}{戳}{chuo1}[][HSK 7-9]
    \definition{s.}{selo; carimbo, abreviação de 戳记}
    \definition{v.}{cutucar; esfaquear | Dialeto: torcer; embotar | Dialeto: ficar em pé}
  \seealsoref{戳记}{chuo1ji4}
  \end{Phonetics}
\end{Entry}

\begin{Entry}{戳记}{18,5}{⼽、⾔}
  \begin{Phonetics}{戳记}{chuo1ji4}
    \definition{s.}{carimbo; selo}
  \end{Phonetics}
\end{Entry}

%%%%% EOF %%%%%

