%%%
%%% Radical "⼎"
%%%

\section*{Radical 15: ``⼎''}\addcontentsline{toc}{section}{Radical 15: ⼎}

\begin{Entry}{冬}{5}{⼎}
  \begin{Phonetics}{冬}{dong1}
    \definition*{s.}{Sobrenome Dong}
    \definition{s.}{inverno}
    \definition{s.}{onomatopéia: som de um tambor batendo, batendo na porta, etc.}
  \end{Phonetics}
\end{Entry}

\begin{Entry}{冬天}{5,4}{⼎、⼤}
  \begin{Phonetics}{冬天}{dong1 tian1}[][HSK 2]
    \definition[个]{s.}{inverno; a quarta estação do ano, na China, geralmente se refere aos três meses entre outubro e dezembro do calendário lunar}
  \end{Phonetics}
\end{Entry}

\begin{Entry}{冬瓜}{5,5}{⼎、⽠}
  \begin{Phonetics}{冬瓜}{dong1gua1}
    \definition{s.}{melão de inverno}
  \end{Phonetics}
\end{Entry}

\begin{Entry}{冬季}{5,8}{⼎、⼦}
  \begin{Phonetics}{冬季}{dong1 ji4}[][HSK 4]
    \definition[个,次,种]{s.}{inverno; o quarto trimestre do ano, habitualmente referido na China como o período de três meses entre o início do inverno e o início da primavera, e também referido como ``décimo, décimo primeiro e décimo segundo'' meses do calendário lunar}
  \end{Phonetics}
\end{Entry}

\begin{Entry}{冰}{6}{⼎}
  \begin{Phonetics}{冰}{bing1}[][HSK 4]
    \definition[块,层,些]{s.}{gelo; água em estado sólido |  algo parecido com gelo | (gíria) metanfetamina}
    \definition{v.}{colocar gelo; colocar gelo ao redor; colocar no gelo; resfriar objetos com gelo ou água fria | sentir frio}
  \end{Phonetics}
\end{Entry}

\begin{Entry}{冰山}{6,3}{⼎、⼭}
  \begin{Phonetics}{冰山}{bing1shan1}[][HSK 7-9]
    \definition[座]{s.}{montanha gelada; montanha coberta de gelo | \emph{iceberg}; enormes blocos de gelo flutuando no mar | Figurativo: indivíduo ou grupo em que não se pode confiar por muito tempo; uma metáfora para um poder em que não se pode confiar por muito tempo}
  \end{Phonetics}
\end{Entry}

\begin{Entry}{冰天雪地}{6,4,11,6}{⼎、⼤、⾬、⼟}
  \begin{Phonetics}{冰天雪地}{bing1tian1-xue3di4}
    \definition{expr.}{um mundo de gelo e neve}
  \end{Phonetics}
\end{Entry}

\begin{Entry}{冰球}{6,11}{⼎、⽟}
  \begin{Phonetics}{冰球}{bing1qiu2}
    \definition[个]{s.}{hóquei no gelo | disco; a ``bola'' usada no hóquei no gelo}
  \end{Phonetics}
\end{Entry}

\begin{Entry}{冰雪}{6,11}{⼎、⾬}
  \begin{Phonetics}{冰雪}{bing1 xue3}[][HSK 4]
    \definition{adj.}{puro como gelo e neve; descreve uma pessoa como pura}
    \definition[片,场]{s.}{gelo e neve}
  \end{Phonetics}
\end{Entry}

\begin{Entry}{冰棍}{6,12}{⼎、⽊}
  \begin{Phonetics}{冰棍}{bing1gun4}
    \definition[根]{s.}{picolé}
  \end{Phonetics}
\end{Entry}

\begin{Entry}{冰棍儿}{6,12,2}{⼎、⽊、⼉}
  \begin{Phonetics}{冰棍儿}{bing1gun4r5}[][HSK 7-9]
    \definition[根,种,支]{s.}{picolé; pirulito congelado}
  \end{Phonetics}
\end{Entry}

\begin{Entry}{冰箱}{6,15}{⼎、⾋}
  \begin{Phonetics}{冰箱}{bing1xiang1}[][HSK 4]
    \definition[台,个]{s.}{geladeira; freezer; refrigerador; aparelhos para congelar alimentos ou medicamentos com gelo para mantê-los frios}
  \end{Phonetics}
\end{Entry}

\begin{Entry}{冰激凌}{6,16,10}{⼎、⽔、⼎}
  \begin{Phonetics}{冰激凌}{bing1ji1ling2}
    \definition{s.}{sorvete}
  \end{Phonetics}
\end{Entry}

\begin{Entry}{冰糕}{6,16}{⼎、⽶}
  \begin{Phonetics}{冰糕}{bing1gao1}
    \definition{s.}{sorvete | picolé}
  \end{Phonetics}
\end{Entry}

\begin{Entry}{冲}{6}{⼎}
  \begin{Phonetics}{冲}{chong1}[][HSK 4,6]
    \definition{s.}{via pública; local importante; via de passagem; via local importante | um trecho de planície em uma área montanhosa | (astronomia) oposição; os planetas externos orbitam até ficarem alinhados com a Terra e o Sol, e a Terra está no meio}
    \definition{v.}{atacar; apressar; correr; passar rapidamente; passar por um obstáculo | colidir; chocar; bater | despejar água fervente sobre | enxaguar; dar descarga; lavar | revelar (filme) | neutralizar a má sorte}
  \end{Phonetics}
  \begin{Phonetics}{冲}{chong4}
    \definition{adj.}{poderoso; com vigor; com muita força; vigoroso | forte; odor forte e pungente (olfato)}
    \definition{prep.}{de frente; em direção a | na força de; com base em; em virtude de}
    \definition{v.}{estampar (máquina de estamparia)}
  \end{Phonetics}
\end{Entry}

\begin{Entry}{冲击}{6,5}{⼎、⼐}
  \begin{Phonetics}{冲击}{chong1ji1}[][HSK 6]
    \definition{v.}{chicotear; bater | correr; voar; atacar; assaltar; atacar bravamente em direção a um alvo predeterminado | chocar; metáfora para interferência ou golpe sério}
  \end{Phonetics}
\end{Entry}

\begin{Entry}{冲动}{6,6}{⼎、⼒}
  \begin{Phonetics}{冲动}{chong1dong4}[][HSK 5]
    \definition{adj.}{impulsivo; impetuoso}
    \definition{s.}{impulso; impetuosidade; impulso de movimento; fenômeno psicológico no qual as emoções são particularmente fortes e o controle racional é fraco}
    \definition{v.}{ficar animado; ser impetuoso; agir por impulso}
  \end{Phonetics}
\end{Entry}

\begin{Entry}{冲刺}{6,8}{⼎、⼑}
  \begin{Phonetics}{冲刺}{chong1ci4}[][HSK 7-9]
    \definition{v.}{arrancar; correr; disparar | metáfora para fazer o maior esforço ao se aproximar de uma meta ou estar prestes a ter sucesso}
  \end{Phonetics}
\end{Entry}

\begin{Entry}{冲洗}{6,9}{⼎、⽔}
  \begin{Phonetics}{冲洗}{chong1xi3}[][HSK 7-9]
    \definition{v.}{lavar; enxaguar | revelar filme; revelar e fixar o material fotossensível exposto}
  \end{Phonetics}
\end{Entry}

\begin{Entry}{冲突}{6,9}{⼎、⽳}
  \begin{Phonetics}{冲突}{chong1tu1}[][HSK 5]
    \definition{v.}{chocar-se; entrar em conflito; conflitar | contradizer; duas coisas opostas que interferem uma na outra}
  \end{Phonetics}
\end{Entry}

\begin{Entry}{冲浪}{6,10}{⼎、⽔}
  \begin{Phonetics}{冲浪}{chong1lang4}[][HSK 7-9]
    \definition{v.}{surfar; um esporte aquático em que os atletas surfam em pranchas especialmente construídas e deslizam ao longo das ondas | metáfora para navegar na \emph{Internet}}
  \end{Phonetics}
\end{Entry}

\begin{Entry}{冲锋}{6,12}{⼎、⾦}
  \begin{Phonetics}{冲锋}{chong1feng1}
    \definition{v.}{cobrar | tomar de assalto}
  \end{Phonetics}
\end{Entry}

\begin{Entry}{冲撞}{6,15}{⼎、⼿}
  \begin{Phonetics}{冲撞}{chong1zhuang4}[][HSK 7-9]
    \definition{v.}{colidir; bater; sofrer impacto; voar; errar contra | ofender; ofender}
  \end{Phonetics}
\end{Entry}

\begin{Entry}{决}{6}{⼎}
  \begin{Phonetics}{决}{jue2}
    \definition{v.}{decidir; determinar | executar uma pessoa | (de um dique, etc.) romper; desabar}
  \end{Phonetics}
\end{Entry}

\begin{Entry}{决不}{6,4}{⼎、⼀}
  \begin{Phonetics}{决不}{jue2 bu4}[][HSK 5]
    \definition{adv.}{em hipótese alguma; nunca | definitivamente não; certamente não; sob nenhuma circunstância; de forma alguma}
  \end{Phonetics}
\end{Entry}

\begin{Entry}{决心}{6,4}{⼎、⼼}
  \begin{Phonetics}{决心}{jue2xin1}[][HSK 3]
    \definition{s.}{resolução; determinação; determinação inabalável}
    \definition{v.}{secidir-se; decidir fazer algo e não vacilar nem mudar de ideia}
  \end{Phonetics}
\end{Entry}

\begin{Entry}{决定}{6,8}{⼎、⼧}
  \begin{Phonetics}{决定}{jue2ding4}[][HSK 3]
    \definition{adj.}{decisivo; as leis objetivas levam as coisas a se desenvolverem e mudarem em determinada direção}
    \definition[项,个]{s.}{decisão; resolução; assuntos decididos}
    \definition{v.}{decidir; determinar; algo se torna a base ou o pré-requisito para outra coisa; desempenha um papel dominante | decidir; resolver; tomar uma decisão; propor uma forma de agir}
  \end{Phonetics}
\end{Entry}

\begin{Entry}{决策}{6,12}{⼎、⽵}
  \begin{Phonetics}{决策}{jue2ce4}[][HSK 6]
    \definition{s.}{decisão política; decisão de importância estratégica; estratégia ou método de decisão}
    \definition{v.}{formular políticas; tomar uma decisão estratégica; decidir sobre uma estratégia ou abordagem}
  \end{Phonetics}
\end{Entry}

\begin{Entry}{决赛}{6,14}{⼎、⾙}
  \begin{Phonetics}{决赛}{jue2sai4}[][HSK 3]
    \definition[场]{s.}{finais (de uma competição); em competições esportivas, a última partida ou rodada disputada para determinar a classificação}
  \end{Phonetics}
\end{Entry}

\begin{Entry}{况}{7}{⼎}
  \begin{Phonetics}{况}{kuang4}
    \definition*{s.}{Sobrenome Kuang}
    \definition{conj.}{além disso | mesmo; muito menos; sem mencionar}
    \definition{s.}{condição; situação}
    \definition{v.}{comparar}
  \end{Phonetics}
\end{Entry}

\begin{Entry}{况且}{7,5}{⼎、⼀}
  \begin{Phonetics}{况且}{kuang4qie3}
    \definition{conj.}{além disso | além do mais}
  \end{Phonetics}
\end{Entry}

\begin{Entry}{冷}{7}{⼎}
  \begin{Phonetics}{冷}{leng3}[][HSK 1]
    \definition*{s.}{Sobrenome Leng}
    \definition{adj.}{frio; baixa temperatura; sensação de frio | gelado; frio por natureza; sem entusiasmo; sem gentileza | desolado; pouco frequentado; quieto; sem agitação | negligenciado; indesejável; ignorado por todos | raro; estranho; incomum | feito em segredo; filmado de forma escondida; lançado secretamente}
    \definition{v.}{esfriar; resfriar | esfriar; congelar; tornar-se indiferente, apático | ignorar}
  \end{Phonetics}
\end{Entry}

\begin{Entry}{冷门}{7,3}{⼎、⾨}
  \begin{Phonetics}{冷门}{leng3men2}
    \definition{s.}{uma profissão, ofício ou ramo de aprendizagem que recebe pouca atenção | um vencedor inesperado; azarão}
  \end{Phonetics}
\end{Entry}

\begin{Entry}{冷气}{7,4}{⼎、⽓}
  \begin{Phonetics}{冷气}{leng3 qi4}[][HSK 6]
    \definition[股,阵]{s.}{ar frio (ou fresco); correntes de ar frio | ar condicionado; ar resfriado por equipamento de refrigeração | ar condicionado; equipamentos de ar condicionado}
  \end{Phonetics}
\end{Entry}

\begin{Entry}{冷水}{7,4}{⼎、⽔}
  \begin{Phonetics}{冷水}{leng3 shui3}[][HSK 6]
    \definition[杯,瓶]{s.}{água fria | água não fervida}
  \end{Phonetics}
\end{Entry}

\begin{Entry}{冷静}{7,14}{⼎、⾭}
  \begin{Phonetics}{冷静}{leng3jing4}[][HSK 4]
    \definition{adj.}{calmo; descreve uma pessoa que consegue ficar atenta em uma situação importante ou de emergência e não toma decisões aleatórias por causa de seus sentimentos no momento | (lugar) tranquilo; quieto; deserto}
  \end{Phonetics}
\end{Entry}

\begin{Entry}{冻}{7}{⼎}
  \begin{Phonetics}{冻}{dong4}[][HSK 5]
    \definition*{s.}{Sobrenome Dong}
    \definition{s.}{geleia; gelatina}
    \definition{v.}{congelar; ser congelado | ficar com frio ou sentir frio}
  \end{Phonetics}
\end{Entry}

\begin{Entry}{净}{8}{⼎}
  \begin{Phonetics}{净}{jing4}[][HSK 6]
    \definition{adj.}{limpo | (depois de um verbo) terminado; sem nada sobrando | líquido | vazio; oco; nu}
    \definition{adv.}{todo; o tempo todo | somente; meramente; nada além de | inteiramente; indica puro e nada mais}
    \definition{s.}{o “rosto pintado”, comumente conhecido como Hualian, 花脸, um tipo de personagem da ópera de Pequim, etc.}
    \definition{v.}{tornar limpo | limpar; lavar; esfregar para limpar}
  \seealsoref{花脸}{hua1lian3}
  \end{Phonetics}
\end{Entry}

\begin{Entry}{准}{10}{⼎}
  \begin{Phonetics}{准}{zhun3}[][HSK 3]
    \definition{adj.}{exato; preciso; algo determinado a ser imutável | preciso; exato; correto | perto; parcialmente; quase; próximo de algo em termos de padrão}
    \definition{adv.}{definitivamente; certamente}
    \definition{pref.}{quasi-; para-}
    \definition{prep.}{de acordo com; baseado em}
    \definition{s.}{norma; padrão; critério | confiança certa; uma ideia definida, certeza, etc. (geralmente usada depois de 有 ou 没有)}
    \definition{v.}{autorizar; conceder; consentir; permitir}
  \seealsoref{没有}{mei2 you3}
  \seealsoref{有}{you3}
  \end{Phonetics}
\end{Entry}

\begin{Entry}{准时}{10,7}{⼎、⽇}
  \begin{Phonetics}{准时}{zhun3shi2}[][HSK 4]
    \definition{adj.}{pontual}
    \definition{adv.}{na hora certa; dentro do prazo; no horário especificado}
  \end{Phonetics}
\end{Entry}

\begin{Entry}{准备}{10,8}{⼎、⼡}
  \begin{Phonetics}{准备}{zhun3bei4}[][HSK 1]
    \definition{v.}{preparar; ficar pronto; planejar ou organizar com antecedência | pretender; planejar}
  \end{Phonetics}
\end{Entry}

\begin{Entry}{准确}{10,12}{⼎、⽯}
  \begin{Phonetics}{准确}{zhun3que4}[][HSK 2]
    \definition{adj.}{exato; preciso; acurado; os resultados da ação são completamente consistentes com os resultados reais ou esperados}
  \end{Phonetics}
\end{Entry}

\begin{Entry}{凉}{10}{⼎}
  \begin{Phonetics}{凉}{liang2}[][HSK 2]
    \definition{adj.}{frio; gelado; ligeiramente fria (menos do que 冷) | sombrio; desolado; sem animação | desanimado; desapontado | usado para prevenir o calor e manter a temperatura amena; para proteção contra o calor}
    \definition{s.}{frio; refere-se a um ambiente fresco ou a uma brisa fresca}
  \seealsoref{冷}{leng3}
  \end{Phonetics}
  \begin{Phonetics}{凉}{liang4}
    \definition{v.}{deixar algo esfriar; deixar um objeto quente descansar por um tempo para que a temperatura diminua}
  \end{Phonetics}
\end{Entry}

\begin{Entry}{凉水}{10,4}{⼎、⽔}
  \begin{Phonetics}{凉水}{liang2 shui3}[][HSK 3]
    \definition{s.}{água fria; água não aquecida | água não fervida}
  \end{Phonetics}
\end{Entry}

\begin{Entry}{凉快}{10,7}{⼎、⼼}
  \begin{Phonetics}{凉快}{liang2kuai5}[][HSK 2]
    \definition{adj.}{fresco; refrescante}
    \definition{v.}{refrescar; refrescar-se; deixar o corpo fresco e revigorado}
  \end{Phonetics}
\end{Entry}

\begin{Entry}{凉鞋}{10,15}{⼎、⾰}
  \begin{Phonetics}{凉鞋}{liang2 xie2}[][HSK 6]
    \definition[双,只]{s.}{sandália; alpargata; alpercata; alparca ; sapatos de verão com cabedal ventilado}
  \end{Phonetics}
\end{Entry}

\begin{Entry}{减}{11}{⼎}
  \begin{Phonetics}{减}{jian3}[][HSK 4]
    \definition*{s.}{Sobrenome Jian}
    \definition{v.}{subtrair; remover uma parte da quantidade original | reduzir; diminuir; cortar}
  \end{Phonetics}
\end{Entry}

\begin{Entry}{减少}{11,4}{⼎、⼩}
  \begin{Phonetics}{减少}{jian3shao3}[][HSK 4]
    \definition{v.}{cair; reduzir; diminuir; subtrair uma parte}
  \end{Phonetics}
\end{Entry}

\begin{Entry}{减肥}{11,8}{⼎、⾁}
  \begin{Phonetics}{减肥}{jian3/fei2}[][HSK 4]
    \definition{v.+compl.}{perder peso; dieta, exercícios, medicamentos, massagem, cirurgia, etc., para reduzir o excesso de gordura corporal, de modo que o grau de obesidade seja reduzido}
  \end{Phonetics}
\end{Entry}

\begin{Entry}{减轻}{11,9}{⼎、⾞}
  \begin{Phonetics}{减轻}{jian3 qing1}[][HSK 5]
    \definition{v.}{aliviar; remeter; clarear; facilitar; mitigar}
  \end{Phonetics}
\end{Entry}

\begin{Entry}{凑}{11}{⼎}
  \begin{Phonetics}{凑}{cou4}[][HSK 7-9]
    \definition{v.}{reunir; coletar; ajuntar | acontecer por acaso; aproveitar; esbarrar em; alcançar; tirar vantagem de | aproximar; mover-se para perto de}
  \end{Phonetics}
\end{Entry}

\begin{Entry}{凑巧}{11,5}{⼎、⼯}
  \begin{Phonetics}{凑巧}{cou4qiao3}[][HSK 7-9]
    \definition{adj.}{afortunado; sortudo; coincidente; significa que é o momento certo ou que algo que você quer ou não quer está acontecendo}
  \end{Phonetics}
\end{Entry}

\begin{Entry}{凑合}{11,6}{⼎、⼝}
  \begin{Phonetics}{凑合}{cou4he5}[][HSK 7-9]
    \definition{v.}{contentar-se com algo; ser razoável; ser razoavelmente bom, mas não excelente; aceitar relutantemente coisas ou condições de um nível ou grau inferior | improvisar | reunir}
  \end{Phonetics}
\end{Entry}

%%%%% EOF %%%%%

