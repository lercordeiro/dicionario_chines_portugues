%%%
%%% Radical "⼎"
%%%

\section*{Radical 15: ``⼎''}\addcontentsline{toc}{section}{Radical 15: ⼎}

\begin{entry}{冬}{5}{⼎}
  \begin{phonetics}{冬}{dong1}
    \definition*{s.}{sobrenome Dong}
    \definition{interj.}{(onomatopéia) som de um tambor batendo, batendo na porta, etc.}
    \definition{s.}{inverno}
  \end{phonetics}
\end{entry}

\begin{entry}{冬天}{5,4}{⼎、⼤}
  \begin{phonetics}{冬天}{dong1 tian1}[][HSK 2]
    \definition[个]{s.}{inverno; a quarta estação do ano, na China, geralmente se refere aos três meses entre outubro e dezembro do calendário lunar}
  \end{phonetics}
\end{entry}

\begin{entry}{冬瓜}{5,5}{⼎、⽠}
  \begin{phonetics}{冬瓜}{dong1gua1}
    \definition{s.}{melão de inverno}
  \end{phonetics}
\end{entry}

\begin{entry}{冬季}{5,8}{⼎、⼦}
  \begin{phonetics}{冬季}{dong1 ji4}[][HSK 4]
    \definition[个]{s.}{inverno; o quarto trimestre do ano, habitualmente referido na China como o período de três meses entre o início do inverno e o início da primavera, e também referido como ``décimo, décimo primeiro e décimo segundo'' meses do calendário lunar}
  \end{phonetics}
\end{entry}

\begin{entry}{冰}{6}{⼎}
  \begin{phonetics}{冰}{bing1}[][HSK 4]
    \definition[块,层,些]{s.}{gelo; água em estado sólido |  algo parecido com gelo | (gíria) metanfetamina}
    \definition{v.}{colocar gelo; colocar gelo ao redor; colocar no gelo; resfriar objetos com gelo ou água fria | sentir frio}
  \end{phonetics}
\end{entry}

\begin{entry}{冰天雪地}{6,4,11,6}{⼎、⼤、⾬、⼟}
  \begin{phonetics}{冰天雪地}{bing1tian1-xue3di4}
    \definition{expr.}{um mundo de gelo e neve}
  \end{phonetics}
\end{entry}

\begin{entry}{冰球}{6,11}{⼎、⽟}
  \begin{phonetics}{冰球}{bing1qiu2}
    \definition{s.}{hóquei no gelo}
  \end{phonetics}
\end{entry}

\begin{entry}{冰雪}{6,11}{⼎、⾬}
  \begin{phonetics}{冰雪}{bing1 xue3}[][HSK 4]
    \definition{adj.}{puro como gelo e neve; descreve uma pessoa como pura}
    \definition{s.}{gelo e neve}
  \end{phonetics}
\end{entry}

\begin{entry}{冰棍}{6,12}{⼎、⽊}
  \begin{phonetics}{冰棍}{bing1gun4}
    \definition[根]{s.}{picolé}
  \end{phonetics}
\end{entry}

\begin{entry}{冰箱}{6,15}{⼎、⾋}
  \begin{phonetics}{冰箱}{bing1xiang1}[][HSK 4]
    \definition[台,个]{s.}{geladeira; freezer; refrigerador; aparelhos para congelar alimentos ou medicamentos com gelo para mantê-los frios}
  \end{phonetics}
\end{entry}

\begin{entry}{冰激凌}{6,16,10}{⼎、⽔、⼎}
  \begin{phonetics}{冰激凌}{bing1ji1ling2}
    \definition{s.}{sorvete}
  \end{phonetics}
\end{entry}

\begin{entry}{冰糕}{6,16}{⼎、⽶}
  \begin{phonetics}{冰糕}{bing1gao1}
    \definition{s.}{sorvete | picolé}
  \end{phonetics}
\end{entry}

\begin{entry}{冲}{6}{⼎}
  \begin{phonetics}{冲}{chong1}[][HSK 4]
    \definition{s.}{via pública; local importante; via de passagem; via local importante | um trecho de planície em uma área montanhosa | (astronomia) oposição; os planetas externos orbitam até ficarem alinhados com a Terra e o Sol, e a Terra está no meio}
    \definition{v.}{atacar; apressar; correr; passar rapidamente; passar por um obstáculo | colidir; chocar; bater | despejar água fervente sobre | enxaguar; dar descarga; lavar | revelar (filme) | neutralizar a má sorte}
  \end{phonetics}
  \begin{phonetics}{冲}{chong4}
    \definition{adj.}{poderoso; com vigor; com muita força; vigoroso | forte; odor forte e pungente (olfato)}
    \definition{prep.}{de frente; em direção a | na força de; com base em; em virtude de}
    \definition{v.}{estampar (máquina de estamparia)}
  \end{phonetics}
\end{entry}

\begin{entry}{冲动}{6,6}{⼎、⼒}
  \begin{phonetics}{冲动}{chong1dong4}[][HSK 5]
    \definition{adj.}{impulsivo; impetuoso}
    \definition{s.}{impulso; impetuosidade; impulso de movimento; fenômeno psicológico no qual as emoções são particularmente fortes e o controle racional é fraco}
    \definition{v.}{ficar animado; ser impetuoso; agir por impulso}
  \end{phonetics}
\end{entry}

\begin{entry}{冲突}{6,9}{⼎、⽳}
  \begin{phonetics}{冲突}{chong1tu1}[][HSK 5]
    \definition{v.}{chocar-se; entrar em conflito; conflitar | contradizer; duas coisas opostas que interferem uma na outra}
  \end{phonetics}
\end{entry}

\begin{entry}{冲浪}{6,10}{⼎、⽔}
  \begin{phonetics}{冲浪}{chong1lang4}
    \definition{s.}{surfe}
    \definition{v.}{surfar}
  \end{phonetics}
\end{entry}

\begin{entry}{冲锋}{6,12}{⼎、⾦}
  \begin{phonetics}{冲锋}{chong1feng1}
    \definition{v.}{cobrar | tomar de assalto}
  \end{phonetics}
\end{entry}

\begin{entry}{决不}{6,4}{⼎、⼀}
  \begin{phonetics}{决不}{jue2 bu4}[][HSK 5]
    \definition{adv.}{definitivamente não; certamente não; sob nenhuma circunstância; de forma alguma}
  \end{phonetics}
\end{entry}

\begin{entry}{决心}{6,4}{⼎、⼼}
  \begin{phonetics}{决心}{jue2xin1}[][HSK 3]
    \definition{s.}{resolução; determinação; determinação inabalável}
    \definition{v.}{secidir-se; decidir fazer algo e não vacilar nem mudar de ideia}
  \end{phonetics}
\end{entry}

\begin{entry}{决定}{6,8}{⼎、⼧}
  \begin{phonetics}{决定}{jue2ding4}[][HSK 3]
    \definition{adj.}{decisivo; as leis objetivas levam as coisas a se desenvolverem e mudarem em determinada direção}
    \definition[项,个]{s.}{decisão; resolução; assuntos decididos}
    \definition{v.}{decidir; determinar; algo se torna a base ou o pré-requisito para outra coisa; desempenha um papel dominante | decidir; resolver; tomar uma decisão; propor uma forma de agir}
  \end{phonetics}
\end{entry}

\begin{entry}{决赛}{6,14}{⼎、⾙}
  \begin{phonetics}{决赛}{jue2sai4}[][HSK 3]
    \definition[场]{s.}{finais (de uma competição); em competições esportivas, a última partida ou rodada disputada para determinar a classificação}
  \end{phonetics}
\end{entry}

\begin{entry}{况且}{7,5}{⼎、⼀}
  \begin{phonetics}{况且}{kuang4qie3}
    \definition{conj.}{além disso | além do mais}
  \end{phonetics}
\end{entry}

\begin{entry}{冷}{7}{⼎}
  \begin{phonetics}{冷}{leng3}[][HSK 1]
    \definition*{s.}{sobrenome Leng}
    \definition{adj.}{frio; baixa temperatura; sensação de frio | gelado; frio por natureza; sem entusiasmo; sem gentileza | desolado; pouco frequentado; quieto; sem agitação | negligenciado; indesejável; ignorado por todos | raro; estranho; incomum | feito em segredo; filmado de forma escondida; lançado secretamente}
    \definition{v.}{esfriar; resfriar | esfriar; congelar; tornar-se indiferente, apático | ignorar}
  \end{phonetics}
\end{entry}

\begin{entry}{冷门}{7,3}{⼎、⾨}
  \begin{phonetics}{冷门}{leng3men2}
    \definition{s.}{uma profissão, ofício ou ramo de aprendizagem que recebe pouca atenção | um vencedor inesperado; azarão}
  \end{phonetics}
\end{entry}

\begin{entry}{冷静}{7,14}{⼎、⾭}
  \begin{phonetics}{冷静}{leng3jing4}[][HSK 4]
    \definition{adj.}{calmo; descreve uma pessoa que consegue ficar atenta em uma situação importante ou de emergência e não toma decisões aleatórias por causa de seus sentimentos no momento | (lugar) tranquilo; quieto; deserto}
  \end{phonetics}
\end{entry}

\begin{entry}{冻}{7}{⼎}
  \begin{phonetics}{冻}{dong4}[][HSK 5]
    \definition*{s.}{sobrenome Dong}
    \definition{s.}{geleia; gelatina}
    \definition{v.}{congelar; ser congelado | ficar com frio ou sentir frio}
  \end{phonetics}
\end{entry}

\begin{entry}{准}{10}{⼎}
  \begin{phonetics}{准}{zhun3}[][HSK 3]
    \definition{adj.}{exato; preciso; algo determinado a ser imutável | preciso; exato; correto | perto; parcialmente; quase; próximo de algo em termos de padrão}
    \definition{adv.}{definitivamente; certamente}
    \definition{pref.}{quasi-; para-}
    \definition{prep.}{de acordo com; baseado em}
    \definition{s.}{norma; padrão; critério | confiança certa; uma ideia definida, certeza, etc. (geralmente usada depois de 有 ou 没有)}
    \definition{v.}{autorizar; conceder; consentir; permitir}
  \seealsoref{没有}{mei2 you3}
  \seealsoref{有}{you3}
  \end{phonetics}
\end{entry}

\begin{entry}{准时}{10,7}{⼎、⽇}
  \begin{phonetics}{准时}{zhun3shi2}[][HSK 4]
    \definition{adj.}{pontual}
    \definition{adv.}{na hora certa; dentro do prazo; no horário especificado}
  \end{phonetics}
\end{entry}

\begin{entry}{准备}{10,8}{⼎、⼡}
  \begin{phonetics}{准备}{zhun3bei4}[][HSK 1]
    \definition{v.}{preparar; ficar pronto; planejar ou organizar com antecedência | pretender; planejar}
  \end{phonetics}
\end{entry}

\begin{entry}{准确}{10,12}{⼎、⽯}
  \begin{phonetics}{准确}{zhun3que4}[][HSK 2]
    \definition{adj.}{exato; preciso; acurado; os resultados da ação são completamente consistentes com os resultados reais ou esperados}
  \end{phonetics}
\end{entry}

\begin{entry}{凉}{10}{⼎}
  \begin{phonetics}{凉}{liang2}[][HSK 2]
    \definition{adj.}{frio; gelado; ligeiramente fria (menos do que 冷) | sombrio; desolado; sem animação | desanimado; desapontado | usado para prevenir o calor e manter a temperatura amena; para proteção contra o calor}
    \definition{s.}{frio; refere-se a um ambiente fresco ou a uma brisa fresca}
  \seealsoref{冷}{leng3}
  \end{phonetics}
  \begin{phonetics}{凉}{liang4}
    \definition{v.}{deixar algo esfriar; deixar um objeto quente descansar por um tempo para que a temperatura diminua}
  \end{phonetics}
\end{entry}

\begin{entry}{凉水}{10,4}{⼎、⽔}
  \begin{phonetics}{凉水}{liang2 shui3}[][HSK 3]
    \definition{s.}{água fria; água não aquecida | água não fervida}
  \end{phonetics}
\end{entry}

\begin{entry}{凉快}{10,7}{⼎、⼼}
  \begin{phonetics}{凉快}{liang2kuai5}[][HSK 2]
    \definition{adj.}{fresco; refrescante}
    \definition{v.}{refrescar; refrescar-se; deixar o corpo fresco e revigorado}
  \end{phonetics}
\end{entry}

\begin{entry}{凉鞋}{10,15}{⼎、⾰}
  \begin{phonetics}{凉鞋}{liang2xie2}
    \definition{s.}{sandália | alpargata | alpercata | alparca}
  \end{phonetics}
\end{entry}

\begin{entry}{减}{11}{⼎}
  \begin{phonetics}{减}{jian3}[][HSK 4]
    \definition*{s.}{sobrenome Jian}
    \definition{v.}{subtrair; remover uma parte da quantidade original | reduzir; diminuir; cortar}
  \end{phonetics}
\end{entry}

\begin{entry}{减少}{11,4}{⼎、⼩}
  \begin{phonetics}{减少}{jian3shao3}[][HSK 4]
    \definition{v.}{cair; reduzir; diminuir; subtrair uma parte}
  \end{phonetics}
\end{entry}

\begin{entry}{减肥}{11,8}{⼎、⾁}
  \begin{phonetics}{减肥}{jian3fei2}[][HSK 4]
    \definition{v.+compl.}{perder peso; dieta, exercícios, medicamentos, massagem, cirurgia, etc., para reduzir o excesso de gordura corporal, de modo que o grau de obesidade seja reduzido}
  \end{phonetics}
\end{entry}

\begin{entry}{减轻}{11,9}{⼎、⾞}
  \begin{phonetics}{减轻}{jian3 qing1}[][HSK 5]
    \definition{v.}{aliviar; remeter; clarear; facilitar; mitigar}
  \end{phonetics}
\end{entry}

%%%%% EOF %%%%%

