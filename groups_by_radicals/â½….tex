%%%
%%% Radical "⽅"
%%%

\section*{Radical 70: ``⽅''}\addcontentsline{toc}{section}{Radical 70: ⽅}

\begin{entry}{方}{4}{⽅}[Kangxi 70]
  \begin{phonetics}{方}{fang1}[][HSK 4]
    \definition*{s.}{sobrenome Fang}
    \definition*{s.}{Alquimia 方术}
    \definition{adj.}{reto; honesto; imparcial}
    \definition{adv.}{exatamente quando; no momento em que}
    \definition{clas.}{usado para coisas quadradas | quadrado ou cúbico (geralmente metro quadrado ou cúbico)}
    \definition[个,张]{s.}{quadrado; um quadrado ou sólido com seis faces quadradas | (matemática) potência; o número de vezes que uma quantidade deve ser multiplicada por si mesma | direção | lado; festa | lugar; região; localidade | maneira; método; solução | prescrição | lei; regra}
  \seealsoref{方术}{fang1 shu4}
  \end{phonetics}
\end{entry}

\begin{entry}{方术}{4,5}{⽅、⽊}
  \begin{phonetics}{方术}{fang1 shu4}
    \definition{s.}{artes de cura, adivinhação, horóscopo etc. | artes sobrenaturais (antigo)}
  \end{phonetics}
\end{entry}

\begin{entry}{方向}{4,6}{⽅、⼝}
  \begin{phonetics}{方向}{fang1xiang4}[][HSK 2]
    \definition[个,种]{s.}{direção; orientação; referindo-se a leste, sul, oeste, norte, sudeste, sudoeste, nordeste, noroeste, etc. | objetivo; meta; finalidade}
  \end{phonetics}
\end{entry}

\begin{entry}{方式}{4,6}{⽅、⼷}
  \begin{phonetics}{方式}{fang1shi4}[][HSK 3]
    \definition[种,个]{s.}{maneira; método}
  \end{phonetics}
\end{entry}

\begin{entry}{方言}{4,7}{⽅、⾔}
  \begin{phonetics}{方言}{fang1yan2}
    \definition*{s.}{o primeiro dicionário de dialeto chinês, editado por Yang Xiong, 扬雄, no século I, contendo mais de 9.000 caracteres}
    \definition{s.}{dialeto}
  \seealsoref{扬雄}{yang2xiong2}
  \end{phonetics}
\end{entry}

\begin{entry}{方针}{4,7}{⽅、⾦}
  \begin{phonetics}{方针}{fang1zhen1}[][HSK 4]
    \definition[个]{s.}{política; diretriz; princípio orientador; orientação da direção e das metas de um empreendimento}
  \end{phonetics}
\end{entry}

\begin{entry}{方法}{4,8}{⽅、⽔}
  \begin{phonetics}{方法}{fang1fa3}[][HSK 2]
    \definition[种,个,套,类]{s.}{método; meio; maneira; sobre os meios e procedimentos para resolver questões relacionadas com o pensamento, a fala e as ações, etc.}
  \end{phonetics}
\end{entry}

\begin{entry}{方便}{4,9}{⽅、⼈}
  \begin{phonetics}{方便}{fang1bian4}[][HSK 2]
    \definition{adj.}{conveniente; sem complicações; sem dificuldades; muito fácil| adequado; condições ou circunstâncias adequadas}
    \definition{s.}{conveniência}
    \definition{v.}{ir ao banheiro; uma maneira delicada de dizer ``ir ao banheiro'' | facilitar; tornar algo conveniente para alguém; facilitar a realização de tarefas ou o alcance de objetivos | ter dinheiro sobrando}
  \end{phonetics}
\end{entry}

\begin{entry}{方便面}{4,9,9}{⽅、⼈、⾯}
  \begin{phonetics}{方便面}{fang1 bian4 mian4}[][HSK 2]
    \definition[袋,包,碗,桶]{s.}{macarrão instantâneo}
  \end{phonetics}
\end{entry}

\begin{entry}{方面}{4,9}{⽅、⾯}
  \begin{phonetics}{方面}{fang1mian4}[][HSK 2]
    \definition[个,种]{s.}{lado; campo; aspecto; respeito}
  \end{phonetics}
\end{entry}

\begin{entry}{方案}{4,10}{⽅、⽊}
  \begin{phonetics}{方案}{fang1'an4}[][HSK 4]
    \definition[个,套]{s.}{plano; esquema; programa; planos específicos para tratar de um determinado problema | o esquema criado pelo governo; medidas ou regulamentações formuladas e implementadas pelo governo ou autoridades relevantes}
  \end{phonetics}
\end{entry}

\begin{entry}{放}{8}{⽅}
  \begin{phonetics}{放}{fang4}[][HSK 1]
    \definition{v.}{deixar ir; libertar; soltar | ceder; deixar-se levar | levar para se alimentar; pastar | soltar; liberar (ou expelir) | exibir (um filme, etc.); reproduzir (um disco, etc.) | acender; inflamar | emprestar (dinheiro) com juros | tornar maior ou mais longo; soltar; abaixar | moderar (a atitude ou o comportamento de alguém) | (de flores) florescer; abrir | colocar; posicionar; deitar | fazer com que algo (ou alguém) caia no chão | deixar de lado; guardar (para uso futuro); conservar | (seguido por 着\dots 不\dots) permitir que algo permaneça (por fazer, por pegar, por usar, etc.) | adicionar; colocar | colocar em pastagem; soltar para caçar | deixar de lado; suspender; interromper | remover; aliviar; livrar-se; proteger; libertar | deixar-se levar; sem restrições; libertino | mandar embora; tirar o prisioneiro da prisão e deportá-lo para uma região remota | distribuir; emitir; lançar | atear fogo | expandir; ampliar; prolongar | reajustar-se até certo ponto; controlar suas ações, adotar uma determinada atitude, atingir um certo equilíbrio | derrubar}
  \end{phonetics}
\end{entry}

\begin{entry}{放下}{8,3}{⽅、⼀}
  \begin{phonetics}{放下}{fang4 xia4}[][HSK 2]
    \definition{v.}{deitar-se; colocar no chão| deixar ir; soltar; desistir; largar | colocar; acomodar; depositar}
  \end{phonetics}
\end{entry}

\begin{entry}{放大}{8,3}{⽅、⼤}
  \begin{phonetics}{放大}{fang4da4}[][HSK 5]
    \definition{v.}{amplificar; magnificar; aumentar; ampliar; aumentar o tamanho de imagens, textos, sons, etc.}
  \end{phonetics}
\end{entry}

\begin{entry}{放飞}{8,3}{⽅、⾶}
  \begin{phonetics}{放飞}{fang4fei1}
    \definition{s.}{deixar voar}
  \end{phonetics}
\end{entry}

\begin{entry}{放心}{8,4}{⽅、⼼}
  \begin{phonetics}{放心}{fang4xin1}[][HSK 2]
    \definition{adj.}{despreocupado}
    \definition{v.}{confiar; ter confiança em alguém; sentir-se aliviado; ficar tranquilo; ficar com a consciência tranquila}
  \end{phonetics}
\end{entry}

\begin{entry}{放出}{8,5}{⽅、⼐}
  \begin{phonetics}{放出}{fang4chu1}
    \definition{v.}{liberar | libertar}
  \end{phonetics}
\end{entry}

\begin{entry}{放电}{8,5}{⽅、⽥}
  \begin{phonetics}{放电}{fang4dian4}
    \definition{s.}{descarga elétrica}
  \end{phonetics}
\end{entry}

\begin{entry}{放任}{8,6}{⽅、⼈}
  \begin{phonetics}{放任}{fang4ren4}
    \definition{v.}{ignorar | saciar-se | deixar sozinho}
  \end{phonetics}
\end{entry}

\begin{entry}{放过}{8,6}{⽅、⾡}
  \begin{phonetics}{放过}{fang4guo4}
    \definition{v.}{deixar | deixar alguém escapar impune | passar despercebido}
  \end{phonetics}
\end{entry}

\begin{entry}{放弃}{8,7}{⽅、⼶}
  \begin{phonetics}{放弃}{fang4qi4}[][HSK 5]
    \definition{v.}{desistir, abandonar; descartar (direitos originais, reivindicações, opiniões, etc.)}
  \end{phonetics}
\end{entry}

\begin{entry}{放弃权利}{8,7,6,7}{⽅、⼶、⽊、⼑}
  \begin{phonetics}{放弃权利}{fang4qi4 quan2li4}
    \definition{s.}{renúncia}
  \end{phonetics}
\end{entry}

\begin{entry}{放弃者}{8,7,8}{⽅、⼶、⽼}
  \begin{phonetics}{放弃者}{fang4qi4zhe3}
    \definition{s.}{desistente}
  \end{phonetics}
\end{entry}

\begin{entry}{放走}{8,7}{⽅、⾛}
  \begin{phonetics}{放走}{fang4zou3}
    \definition{v.}{permitir (uma pessoa ou um animal) ir | liberar | libertar}
  \end{phonetics}
\end{entry}

\begin{entry}{放到}{8,8}{⽅、⼑}
  \begin{phonetics}{放到}{fang4 dao4}[][HSK 3]
    \definition{v.}{colocar em; meter}
  \end{phonetics}
\end{entry}

\begin{entry}{放学}{8,8}{⽅、⼦}
  \begin{phonetics}{放学}{fang4 xue2}[][HSK 1]
    \definition{v.+compl.}{encerrar; sair da escola; as aulas terminaram; a escola acabou (por hoje); voltar para casa depois de um dia ou meio dia de aula}
  \end{phonetics}
\end{entry}

\begin{entry}{放松}{8,8}{⽅、⽊}
  \begin{phonetics}{放松}{fang4song1}[][HSK 4]
    \definition{adj.}{relaxado; afrouxado; solto; desprendido}
    \definition{v.}{relaxar; afrouxar; soltar; desprender}
  \end{phonetics}
\end{entry}

\begin{entry}{放养}{8,9}{⽅、⼋}
  \begin{phonetics}{放养}{fang4yang3}
    \definition{v.}{criar (gado, peixes, culturas, etc.) | crescer | criar}
  \end{phonetics}
\end{entry}

\begin{entry}{放假}{8,11}{⽅、⼈}
  \begin{phonetics}{放假}{fang4 jia4}[][HSK 1]
    \definition{v.}{tirar férias (ou feriado); ter um dia de folga}
    \definition{v.+compl.}{tirar férias (ou feriado); começar as férias; ter um dia de folga; estar de férias (feriado)}
  \end{phonetics}
\end{entry}

\begin{entry}{放肆}{8,13}{⽅、⾀}
  \begin{phonetics}{放肆}{fang4si4}
    \definition{adj.}{atrevido | pesunçoso | devasso}
  \end{phonetics}
\end{entry}

\begin{entry}{放鞭炮}{8,18,9}{⽅、⾰、⽕}
  \begin{phonetics}{放鞭炮}{fang4bian1pao4}
    \definition{s.}{um conjunto de bombinhas ou traques}
  \end{phonetics}
\end{entry}

\begin{entry}{旁}{10}{⽅}
  \begin{phonetics}{旁}{pang2}[][HSK 5]
    \definition{adj.}{outro | abundante; abrangente}
    \definition{s.}{lado | radical lateral de um caractere chinês}
  \end{phonetics}
\end{entry}

\begin{entry}{旁边}{10,5}{⽅、⾡}
  \begin{phonetics}{旁边}{pang2bian1}[][HSK 1]
    \definition{s.}{junto a; próximo de; ao lado}
  \end{phonetics}
\end{entry}

\begin{entry}{旅行}{10,6}{⽅、⾏}
  \begin{phonetics}{旅行}{lv3xing2}[][HSK 2]
    \definition{v.}{viajar; passear; para tratar de assuntos ou passear, ir de um lugar para outro (geralmente se refere a distâncias longas)}
  \end{phonetics}
\end{entry}

\begin{entry}{旅行社}{10,6,7}{⽅、⾏、⽰}
  \begin{phonetics}{旅行社}{lv3 xing2 she4}[][HSK 3]
    \definition[家]{s.}{agência de viagens; agência especializada em serviços relacionados a viagens, que providencia hospedagem, transporte e outros serviços para viajantes}
  \end{phonetics}
\end{entry}

\begin{entry}{旅客}{10,9}{⽅、⼧}
  \begin{phonetics}{旅客}{lv3 ke4}[][HSK 2]
    \definition[名,位,个,些]{s.}{viajante; passageiro; as agências de transporte e turismo referem-se às pessoas que viajam}
  \end{phonetics}
\end{entry}

\begin{entry}{旅馆}{10,11}{⽅、⾷}
  \begin{phonetics}{旅馆}{lv3 guan3}[][HSK 3]
    \definition[家,个,所]{s.}{pousada; hotel; local comercial destinado ao alojamento de viajantes}
  \end{phonetics}
\end{entry}

\begin{entry}{旅游}{10,12}{⽅、⽔}
  \begin{phonetics}{旅游}{lv3you2}[][HSK 2]
    \definition{v.}{viajar para outros lugares para passear e fazer turismo}
  \end{phonetics}
\end{entry}

\begin{entry}{旅程}{10,12}{⽅、⽲}
  \begin{phonetics}{旅程}{lv3cheng2}
    \definition{s.}{jornada | viagem}
  \end{phonetics}
\end{entry}

\begin{entry}{旋转}{11,8}{⽅、⾞}
  \begin{phonetics}{旋转}{xuan2zhuan3}
    \definition{v.}{girar}
  \end{phonetics}
\end{entry}

\begin{entry}{族}{11}{⽅}
  \begin{phonetics}{族}{zu2}
    \definition{s.}{raça | nacionalidade | etnia | clã | por extensão, grupo social}
  \end{phonetics}
\end{entry}

\begin{entry}{旗}{14}{⽅}
  \begin{phonetics}{旗}{qi2}
    \definition[面]{s.}{bandeira}
  \end{phonetics}
\end{entry}

%%%%% EOF %%%%%

