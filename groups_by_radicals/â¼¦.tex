%%%
%%% Radical "⼦"
%%%

\section*{Radical 39: ``⼦''}\addcontentsline{toc}{section}{Radical 39: ⼦}

\begin{entry}{子}{3}{⼦}
  \begin{phonetics}{子}{zi3}
    \definition{adj.}{jovem | pequeno | tenro}
    \definition{clas.}{para objetos finos que podem ser pinçados com os dedos}
    \definition{pron.}{você}
    \definition{s.}{filho | pessoa | antigo título de respeito para um homem culto ou virtuoso | semente | ovo; ova | coisas pequenas e duras | moeda de cobre; cobre | o quarto título da classificação dos cinco títulos feudais de nobreza; visconde}
  \end{phonetics}
  \begin{phonetics}{子}{zi5}[][HSK 1]
    \definition{clas.}{sufixos de palavras de medida individuais}
    \definition{suf.}{sufixo para substantivos}
  \end{phonetics}
\end{entry}

\begin{entry}{子女}{3,3}{⼦、⼥}
  \begin{phonetics}{子女}{zi3 nv3}[][HSK 3]
    \definition{s.}{crianças; descendência; filhos e filhas}
  \end{phonetics}
\end{entry}

\begin{entry}{子弹}{3,11}{⼦、⼸}
  \begin{phonetics}{子弹}{zi3dan4}
    \definition[粒,颗,发]{s.}{bala (de revólver)}
  \end{phonetics}
\end{entry}

\begin{entry}{孔}{4}{⼦}
  \begin{phonetics}{孔}{kong3}
    \definition*{s.}{sobrenome Kong}
    \definition{clas.}{para habitações em cavernas}
    \definition[个]{s.}{buraco}
  \end{phonetics}
\end{entry}

\begin{entry}{孔子}{4,3}{⼦、⼦}
  \begin{phonetics}{孔子}{kong3zi3}
    \definition*{s.}{Confúcio (551-479 aC), pensador e filósofo social chinês}
  \seealsoref{孔夫子}{kong3fu1zi3}
  \end{phonetics}
\end{entry}

\begin{entry}{孔子学院}{4,3,8,9}{⼦、⼦、⼦、⾩}
  \begin{phonetics}{孔子学院}{kong3zi3 xue2yuan4}
    \definition*{s.}{Instituto Confúcio, organização estabelecida internacionalmente pela República Popular da China, que promove a língua e a cultura chinesas}
  \end{phonetics}
\end{entry}

\begin{entry}{孔夫子}{4,4,3}{⼦、⼤、⼦}
  \begin{phonetics}{孔夫子}{kong3fu1zi3}
    \definition*{s.}{Confúcio (551-479 aC), pensador e filósofo social chinês}
  \seealsoref{孔子}{kong3zi3}
  \end{phonetics}
\end{entry}

\begin{entry}{孔雀}{4,11}{⼦、⾫}
  \begin{phonetics}{孔雀}{kong3que4}
    \definition{s.}{pavão}
  \end{phonetics}
\end{entry}

\begin{entry}{字}{6}{⼦}
  \begin{phonetics}{字}{zi4}[][HSK 1]
    \definition[个]{s.}{carácter | letra | símbolo | palavra}
  \end{phonetics}
\end{entry}

\begin{entry}{字母}{6,5}{⼦、⽏}
  \begin{phonetics}{字母}{zi4mu3}[][HSK 4]
    \definition[个]{s.}{letra; letras de um alfabeto | caractere que representa uma consoante inicial (em fonologia)}
  \end{phonetics}
\end{entry}

\begin{entry}{字字珠玉}{6,6,10,5}{⼦、⼦、⽟、⽟}
  \begin{phonetics}{字字珠玉}{zi4zi4zhu1yu4}
    \definition{expr.}{cada palavra é uma jóia}
    \definition{s.}{escrita magnífica}
  \end{phonetics}
\end{entry}

\begin{entry}{字典}{6,8}{⼦、⼋}
  \begin{phonetics}{字典}{zi4 dian3}[][HSK 2]
    \definition[本]{s.}{dicionário de caracteres chineses (contendo verbetes de caracteres únicos, em contraste com 词典 que contém verbetes para palavras com um ou mais caracteres)}
    \seeref{词典}{ci2dian3}
  \end{phonetics}
\end{entry}

\begin{entry}{字眼}{6,11}{⼦、⽬}
  \begin{phonetics}{字眼}{zi4yan3}
    \definition[个]{s.}{palavras | redação}
  \end{phonetics}
\end{entry}

\begin{entry}{字脚}{6,11}{⼦、⾁}
  \begin{phonetics}{字脚}{zi4jiao3}
    \definition[典]{s.}{gancho no final da pincelada | serifa}
  \end{phonetics}
\end{entry}

\begin{entry}{存}{6}{⼦}
  \begin{phonetics}{存}{cun2}[][HSK 3]
    \definition{v.}{existir; viver; sobreviver | armazenar; manter | acumular; coletar | depositar | sair com; verificar |reservar; reter | permanecer em equilíbrio; estar em estoque | estimar; abrigar}
  \end{phonetics}
\end{entry}

\begin{entry}{存在}{6,6}{⼦、⼟}
  \begin{phonetics}{存在}{cun2zai4}[][HSK 3]
    \definition{s.}{existência; ser; ente}
    \definition{v.}{existir; ser}
  \end{phonetics}
\end{entry}

\begin{entry}{存款}{6,12}{⼦、⽋}
  \begin{phonetics}{存款}{cun2 kuan3}[][HSK 5]
    \definition[笔]{s.}{depósito; poupança bancária}
    \definition{v.}{depositar dinheiro; colocar dinheiro no banco}
  \end{phonetics}
\end{entry}

\begin{entry}{孙女}{6,3}{⼦、⼥}
  \begin{phonetics}{孙女}{sun1nv3}[][HSK 4]
    \definition{s.}{filha do filho; neta}
  \end{phonetics}
\end{entry}

\begin{entry}{孙子}{6,3}{⼦、⼦}
  \begin{phonetics}{孙子}{sun1zi3}
    \definition*{s.}{Sun Tzu, também conhecido por Sun Wu (孙武), general, estrategista e filósofo autor do ``Arte da Guerra'' (孙子兵法)}
    \seeref{孙武}{sun1wu3}
  \seealsoref{孙子兵法}{sun1zi3 bing1fa3}
  \end{phonetics}
  \begin{phonetics}{孙子}{sun1zi5}[][HSK 4]
    \definition{s.}{filho do filho; neto}
  \end{phonetics}
\end{entry}

\begin{entry}{孙子兵法}{6,3,7,8}{⼦、⼦、⼋、⽔}
  \begin{phonetics}{孙子兵法}{sun1zi3 bing1fa3}
    \definition*{s.}{``Arte da Guerra'', escrito por Sun Tzu (孫子)}
    \seeref{孙武}{sun1wu3}
    \seeref{孙子}{sun1zi3}
  \end{phonetics}
\end{entry}

\begin{entry}{孙武}{6,8}{⼦、⽌}
  \begin{phonetics}{孙武}{sun1wu3}
    \definition*{s.}{Sun Wu, também conhecido por Sun Tzu (孙子), general, estrategista e filósofo autor do ``Arte da Guerra'' (孙子兵法)}
    \seeref{孙子}{sun1zi3}
  \seealsoref{孙子兵法}{sun1zi3 bing1fa3}
  \end{phonetics}
\end{entry}

\begin{entry}{季}{8}{⼦}
  \begin{phonetics}{季}{ji4}[][HSK 4]
    \definition*{s.}{sobrenome Ji}
    \definition{s.}{estação; o ano é dividido em quatro estações, primavera, verão, outono e inverno, e uma estação dura três meses | temporada | o fim de uma era | o último mês de uma temporada | o quarto ou mais novo entre irmãos; último na ordem de precedência}
  \end{phonetics}
\end{entry}

\begin{entry}{季节}{8,5}{⼦、⾋}
  \begin{phonetics}{季节}{ji4jie2}[][HSK 4]
    \definition[个]{s.}{estação (clima); um período característico do ano}
  \end{phonetics}
\end{entry}

\begin{entry}{季度}{8,9}{⼦、⼴}
  \begin{phonetics}{季度}{ji4du4}[][HSK 4]
    \definition[个]{s.}{trimestre; período de tempo trimestral}
  \end{phonetics}
\end{entry}

\begin{entry}{孤独}{8,9}{⼦、⽝}
  \begin{phonetics}{孤独}{gu1du2}
    \definition{adj.}{solitário}
  \end{phonetics}
\end{entry}

\begin{entry}{学}{8}{⼦}
  \begin{phonetics}{学}{xue2}[][HSK 1]
    \definition{v.}{aprender | estudar}
  \end{phonetics}
\end{entry}

\begin{entry}{学习}{8,3}{⼦、⼄}
  \begin{phonetics}{学习}{xue2xi2}[][HSK 1]
    \definition{v.}{estudar | aprender}
  \end{phonetics}
\end{entry}

\begin{entry}{学分}{8,4}{⼦、⼑}
  \begin{phonetics}{学分}{xue2fen1}[][HSK 4]
    \definition{s.}{créditos de um curso; uma unidade de medida do peso e do tempo do curso no ensino superior; cada curso vale um crédito para uma aula por semana durante um semestre; alunos devem concluir o número necessário de créditos para se formar}
  \end{phonetics}
\end{entry}

\begin{entry}{学术}{8,5}{⼦、⽊}
  \begin{phonetics}{学术}{xue2shu4}[][HSK 4]
    \definition[个]{s.}{aprendizagem; aprendizado; ciências; aprendizado sistemático e especializado}
  \end{phonetics}
\end{entry}

\begin{entry}{学生}{8,5}{⼦、⽣}
  \begin{phonetics}{学生}{xue2sheng5}[][HSK 1]
    \definition{s.}{estudante | aluno}
  \end{phonetics}
\end{entry}

\begin{entry}{学生证}{8,5,7}{⼦、⽣、⾔}
  \begin{phonetics}{学生证}{xue2sheng5zheng4}
    \definition{s.}{cartão de identidade de estudante}
  \end{phonetics}
\end{entry}

\begin{entry}{学会}{8,6}{⼦、⼈}
  \begin{phonetics}{学会}{xue2hui4}
    \definition{s.}{instituto | associação (acadêmica) | sociedade científica, douta ou erudita}
    \definition{v.}{aprender | dominar (um assunto)}
  \end{phonetics}
\end{entry}

\begin{entry}{学好}{8,6}{⼦、⼥}
  \begin{phonetics}{学好}{xue2hao3}
    \definition{v.}{seguir bons exemplos | aprender bem}
  \end{phonetics}
\end{entry}

\begin{entry}{学年}{8,6}{⼦、⼲}
  \begin{phonetics}{学年}{xue2 nian2}[][HSK 4]
    \definition{s.}{ano letivo; ano acadêmico}
  \end{phonetics}
\end{entry}

\begin{entry}{学问}{8,6}{⼦、⾨}
  \begin{phonetics}{学问}{xue2wen4}[][HSK 4]
    \definition[个]{s.}{aprendizado, conhecimento, erudição; a compreensão correta do mundo objetivo que alguém tem | conhecimento; aprendizado sistemático; conhecimento sistemático sobre algo ou uma ciência que pode ser aprendido em um livro ou em uma experiência prática}
  \end{phonetics}
\end{entry}

\begin{entry}{学时}{8,7}{⼦、⽇}
  \begin{phonetics}{学时}{xue2 shi2}[][HSK 4]
    \definition{s.}{hora-aula; hora de aula | período}
  \end{phonetics}
\end{entry}

\begin{entry}{学费}{8,9}{⼦、⾙}
  \begin{phonetics}{学费}{xue2 fei4}[][HSK 3]
    \definition[个]{s.}{mensalidade (taxa); prêmio; taxas que os alunos devem pagar para estudar na escola, conforme estipulado pela escola | preço pelo que se aprendeu ao custo de cada um; uma metáfora para o preço pago para ganhar uma certa experiência | custo; preço; todas as despesas necessárias para os alunos estudarem}
  \end{phonetics}
\end{entry}

\begin{entry}{学院}{8,9}{⼦、⾩}
  \begin{phonetics}{学院}{xue2yuan4}[][HSK 1]
    \definition[所]{s.}{instituto}
  \end{phonetics}
\end{entry}

\begin{entry}{学校}{8,10}{⼦、⽊}
  \begin{phonetics}{学校}{xue2xiao4}[][HSK 1]
    \definition{s.}{escola | instituição de ensino}
  \end{phonetics}
\end{entry}

\begin{entry}{学期}{8,12}{⼦、⽉}
  \begin{phonetics}{学期}{xue2qi1}[][HSK 2]
    \definition[个]{s.}{semestre}
  \end{phonetics}
\end{entry}

\begin{entry}{孩子}{9,3}{⼦、⼦}
  \begin{phonetics}{孩子}{hai2zi5}[][HSK 1]
    \definition{s.}{criança | filho}
  \end{phonetics}
\end{entry}

%%%%% EOF %%%%%

