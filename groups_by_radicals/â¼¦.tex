%%%
%%% Radical "⼦"
%%%

\section*{Radical 39: ``⼦''}\addcontentsline{toc}{section}{Radical 39: ⼦}

\begin{Entry}{子}{3}{⼦}[Kangxi 39]
  \begin{Phonetics}{子}{zi3}
    \definition*{s.}{Sobrenome Zi}
    \definition{adj.}{pequeno; jovem; tenro | subsidiário; subordinado; derivado}
    \definition{clas.}{usado para objetos finos que podem ser pinçados com os dedos}
    \definition{pron.}{você;  antigamente, era uma forma de tratamento respeitosa para se referir a outras pessoas, equivalente a 您}
    \definition[个,位,名]{s.}{filho, criança; antigamente, referia-se aos filhos, mas atualmente refere-se especificamente aos filhos homens | pessoa | antigo título de respeito para um homem culto ou virtuoso; na antiguidade, referia-se especificamente a homens eruditos | visconde; o quarto posto na hierarquia dos cinco títulos feudais da nobreza | ovo | semente | coisas pequenas e duras; pequenos fragmentos ou grãos duros e sólidos | cobre; moeda de cobre | o primeiro dos doze ramos terrestres}
  \seealsoref{您}{nin2}
  \end{Phonetics}
  \begin{Phonetics}{子}{zi5}[][HSK 1]
    \definition{suf.}{sufixo para substantivos | sufixos de palavras de medida individuais; anexado a certas palavras classificadoras}
  \end{Phonetics}
\end{Entry}

\begin{Entry}{子女}{3,3}{⼦、⼥}
  \begin{Phonetics}{子女}{zi3 nv3}[][HSK 3]
    \definition[个]{s.}{crianças; descendentes; filhos e filhas}
  \end{Phonetics}
\end{Entry}

\begin{Entry}{子弹}{3,11}{⼦、⼸}
  \begin{Phonetics}{子弹}{zi3dan4}[][HSK 5]
    \definition[粒,颗,发]{s.}{bala; cartucho; munição}
  \end{Phonetics}
\end{Entry}

\begin{Entry}{孔}{4}{⼦}
  \begin{Phonetics}{孔}{kong3}
    \definition*{s.}{Abreviação de Confúcio, 孔子 | Sobrenome Kong}
    \definition{adj.}{Clássico: muito; bastante; razoavelmente}
    \definition{adj.}{aberto; desimpedido; claro; desobstruído}
    \definition{clas.}{usado para habitações em cavernas}
    \definition[个,排]{s.}{buraco; abertura; poro}
  \seealsoref{孔子}{kong3zi3}
  \end{Phonetics}
\end{Entry}

\begin{Entry}{孔子}{4,3}{⼦、⼦}
  \begin{Phonetics}{孔子}{kong3zi3}
    \definition*{s.}{Confúcio (551-479 aC), pensador e filósofo social chinês}
  \seealsoref{孔夫子}{kong3fu1zi3}
  \end{Phonetics}
\end{Entry}

\begin{Entry}{孔子学院}{4,3,8,9}{⼦、⼦、⼦、⾩}
  \begin{Phonetics}{孔子学院}{kong3zi3 xue2yuan4}
    \definition*{s.}{Instituto Confúcio, organização estabelecida internacionalmente pela República Popular da China, que promove a língua e a cultura chinesas}
  \end{Phonetics}
\end{Entry}

\begin{Entry}{孔夫子}{4,4,3}{⼦、⼤、⼦}
  \begin{Phonetics}{孔夫子}{kong3fu1zi3}
    \definition*{s.}{Confúcio (551-479 aC), pensador e filósofo social chinês}
  \seealsoref{孔子}{kong3zi3}
  \end{Phonetics}
\end{Entry}

\begin{Entry}{孔雀}{4,11}{⼦、⾫}
  \begin{Phonetics}{孔雀}{kong3que4}
    \definition{s.}{pavão}
  \end{Phonetics}
\end{Entry}

\begin{Entry}{字}{6}{⼦}
  \begin{Phonetics}{字}{zi4}[][HSK 1]
    \definition[个]{s.}{palavra; caractere; texto | pronúncia (de uma palavra ou caractere); som do caractere | tipo de impressão; estilo de caligrafia; forma de um caractere escrito ou impresso; refere-se às diferentes formas dos caracteres chineses; também se refere às diferentes escolas de caligrafia | escritas; obras de caligrafia | recibo; compromisso por escrito; documento | nome de estilo masculino adotado aos vinte anos de idade | sobrenome | um número indicado num contador elétrico, contador de água, etc.; registrar dos números dos medidores de consumo de água e eletricidade}
    \definition{v.}{ficar noiva (nos tempos antigos)}
  \end{Phonetics}
\end{Entry}

\begin{Entry}{字母}{6,5}{⼦、⽏}
  \begin{Phonetics}{字母}{zi4mu3}[][HSK 4]
    \definition[个,种]{s.}{letra; letras de um alfabeto | Fonologia: caractere que representa uma consoante inicial}
  \end{Phonetics}
\end{Entry}

\begin{Entry}{字字珠玉}{6,6,10,5}{⼦、⼦、⽟、⽟}
  \begin{Phonetics}{字字珠玉}{zi4zi4zhu1yu4}
    \definition{expr.}{cada palavra é uma jóia}
    \definition{s.}{escrita magnífica}
  \end{Phonetics}
\end{Entry}

\begin{Entry}{字典}{6,8}{⼦、⼋}
  \begin{Phonetics}{字典}{zi4 dian3}[][HSK 2]
    \definition[本,册,部]{s.}{dicionário de caracteres chineses (contendo verbetes de caracteres únicos, em contraste com 词典 que contém verbetes para palavras com um ou mais caracteres)}
  \seealsoref{词典}{ci2 dian3}
  \end{Phonetics}
\end{Entry}

\begin{Entry}{字眼}{6,11}{⼦、⽬}
  \begin{Phonetics}{字眼}{zi4yan3}
    \definition[个]{s.}{palavras | redação}
  \end{Phonetics}
\end{Entry}

\begin{Entry}{字脚}{6,11}{⼦、⾁}
  \begin{Phonetics}{字脚}{zi4jiao3}
    \definition[典]{s.}{gancho no final da pincelada | serifa}
  \end{Phonetics}
\end{Entry}

\begin{Entry}{存}{6}{⼦}
  \begin{Phonetics}{存}{cun2}[][HSK 3]
    \definition{v.}{existir; viver; sobreviver | armazenar; manter | acumular; coletar | depositar | sair com; verificar | reservar; reter | permanecer em equilíbrio; estar em estoque | estimar; abrigar}
  \end{Phonetics}
\end{Entry}

\begin{Entry}{存心}{6,4}{⼦、⼼}
  \begin{Phonetics}{存心}{cun2xin1}[][HSK 7-9]
    \definition{adv.}{intencionalmente; deliberadamente; de ​​propósito}
  \end{Phonetics}
\end{Entry}

\begin{Entry}{存在}{6,6}{⼦、⼟}
  \begin{Phonetics}{存在}{cun2zai4}[][HSK 3]
    \definition{s.}{existência; ser; ente; o mundo objetivo, que não depende da consciência humana para mudar, ou seja, a matéria}
    \definition{v.}{existir; ser; as coisas ocupam continuamente o tempo e o espaço; na verdade, ainda não desapareceram}
  \end{Phonetics}
\end{Entry}

\begin{Entry}{存折}{6,7}{⼦、⼿}
  \begin{Phonetics}{存折}{cun2zhe2}[][HSK 7-9]
    \definition{s.}{caderneta bancária; caderneta de poupança; livro de depósitos; livro de poupança bancária; um \emph{voucher} emitido por uma instituição financeira a um depositante como um certificado}
  \end{Phonetics}
\end{Entry}

\begin{Entry}{存放}{6,8}{⼦、⽅}
  \begin{Phonetics}{存放}{cun2fang4}[][HSK 7-9]
    \definition{v.}{armazenar; guardar; deixar com}
  \end{Phonetics}
\end{Entry}

\begin{Entry}{存款}{6,12}{⼦、⽋}
  \begin{Phonetics}{存款}{cun2 kuan3}[][HSK 5]
    \definition[些,笔]{s.}{depósito; poupança bancária}
    \definition{v.}{depositar dinheiro; colocar dinheiro no banco}
  \end{Phonetics}
\end{Entry}

\begin{Entry}{孙}{6}{⼦}
  \begin{Phonetics}{孙}{sun1}
    \definition*{s.}{Sobrenome Sun}
    \definition{s.}{neto; neta | gerações abaixo da do neto | parentes pertencentes à geração do neto | segundo crescimento das plantas}
  \end{Phonetics}
\end{Entry}

\begin{Entry}{孙女}{6,3}{⼦、⼥}
  \begin{Phonetics}{孙女}{sun1nv3}[][HSK 4]
    \definition[个]{s.}{filha do filho; neta}
  \end{Phonetics}
\end{Entry}

\begin{Entry}{孙子}{6,3}{⼦、⼦}
  \begin{Phonetics}{孙子}{sun1zi3}
    \definition*{s.}{Sun Tzu, também conhecido por Sun Wu, 孙武, general, estrategista e filósofo autor do ``Arte da Guerra'', 《孙子兵法》}
  \seealsoref{孙武}{sun1wu3}
  \seealsoref{孙子兵法}{sun1zi3 bing1fa3}
  \end{Phonetics}
  \begin{Phonetics}{孙子}{sun1zi5}[][HSK 4]
    \definition[个]{s.}{filho do filho; neto}
  \end{Phonetics}
\end{Entry}

\begin{Entry}{孙子兵法}{6,3,7,8}{⼦、⼦、⼋、⽔}
  \begin{Phonetics}{孙子兵法}{sun1zi3 bing1fa3}
    \definition*{s.}{``Arte da Guerra'', o antigo clássico chinês sobre estratégia militar, escrito por Sun Tzu, 孫子}
  \seealsoref{孙武}{sun1wu3}
  \seealsoref{孙子}{sun1zi3}
  \end{Phonetics}
\end{Entry}

\begin{Entry}{孙武}{6,8}{⼦、⽌}
  \begin{Phonetics}{孙武}{sun1wu3}
    \definition*{s.}{Sun Wu, também conhecido por Sun Tzu, 孙子, general, estrategista e filósofo autor do ``Arte da Guerra'', 《孙子兵法》}
  \seealsoref{孙子}{sun1zi3}
  \seealsoref{孙子兵法}{sun1zi3 bing1fa3}
  \end{Phonetics}
\end{Entry}

\begin{Entry}{季}{8}{⼦}
  \begin{Phonetics}{季}{ji4}[][HSK 4]
    \definition*{s.}{Sobrenome Ji}
    \definition{s.}{estação; o ano é dividido em quatro estações, primavera, verão, outono e inverno, e uma estação dura três meses | temporada | o fim de uma era | o último mês de uma temporada | o quarto ou mais novo entre irmãos; último na ordem de precedência}
  \end{Phonetics}
\end{Entry}

\begin{Entry}{季节}{8,5}{⼦、⾋}
  \begin{Phonetics}{季节}{ji4jie2}[][HSK 4]
    \definition[个]{s.}{estação (clima); um período característico do ano}
  \end{Phonetics}
\end{Entry}

\begin{Entry}{季度}{8,9}{⼦、⼴}
  \begin{Phonetics}{季度}{ji4du4}[][HSK 4]
    \definition[个]{s.}{trimestre; período de tempo trimestral}
  \end{Phonetics}
\end{Entry}

\begin{Entry}{孤}{8}{⼦}
  \begin{Phonetics}{孤}{gu1}
    \definition*{s.}{Sobrenome Gu}
    \definition{adj.}{sozinho; solitário; isolado}
    \definition{pron.}{eu; meu humilde eu (usado por príncipes feudais); título autoproclamado dos príncipes feudais}
    \definition[个,名,位]{s.}{órfão}
  \end{Phonetics}
\end{Entry}

\begin{Entry}{孤儿}{8,2}{⼦、⼉}
  \begin{Phonetics}{孤儿}{gu1 er2}[][HSK 6]
    \definition[个,名,位]{s.}{órfão; criança sem pais; crianças que perderam os pais}
  \end{Phonetics}
\end{Entry}

\begin{Entry}{孤立}{8,5}{⼦、⽴}
  \begin{Phonetics}{孤立}{gu1li4}[][HSK 7-9]
    \definition{adj.}{isolado; condenado ao ostracismo; descreve a falta de ajuda e simpatia}
    \definition{v.}{isolar; ostracizar; privar uma pessoa de ajuda, apoio e confiança}
  \end{Phonetics}
\end{Entry}

\begin{Entry}{孤单}{8,8}{⼦、⼗}
  \begin{Phonetics}{孤单}{gu1dan1}[][HSK 7-9]
    \definition{adj.}{sozinho; solitário | fraco; inadequado; descreve um pequeno número de pessoas e poder fraco}
  \end{Phonetics}
\end{Entry}

\begin{Entry}{孤陋寡闻}{8,8,14,9}{⼦、⾩、⼧、⾨}
  \begin{Phonetics}{孤陋寡闻}{gu1lou4-gua3wen2}[][HSK 7-9]
    \definition{expr.}{ignorante e mal informado | ignorante e inexperiente | mal informado e tacanho}
  \end{Phonetics}
\end{Entry}

\begin{Entry}{孤独}{8,9}{⼦、⽝}
  \begin{Phonetics}{孤独}{gu1du2}[][HSK 6]
    \definition{adj.}{sozinho; solitário}
  \end{Phonetics}
\end{Entry}

\begin{Entry}{孤零零}{8,13,13}{⼦、⾬、⾬}
  \begin{Phonetics}{孤零零}{gu1ling2ling2}[][HSK 7-9]
    \definition{adj.}{solitário; sozinho; completamente sozinho; sem apoio ou companhia}
  \end{Phonetics}
\end{Entry}

\begin{Entry}{学}{8}{⼦}
  \begin{Phonetics}{学}{xue2}[][HSK 1]
    \definition[所]{s.}{aprendizagem; conhecimento; sabedoria; erudição | objeto de estudo; ramo do conhecimento | escola; faculdade | teoria; doutrina}
    \definition{v.}{estudar; aprender | imitar; copiar}
  \end{Phonetics}
\end{Entry}

\begin{Entry}{学习}{8,3}{⼦、⼄}
  \begin{Phonetics}{学习}{xue2xi2}[][HSK 1]
    \definition{s.}{estudo}
    \definition{v.}{estudar; aprender; adquirir conhecimentos ou habilidades através da leitura, da audição, da pesquisa e da prática}
  \end{Phonetics}
\end{Entry}

\begin{Entry}{学分}{8,4}{⼦、⼑}
  \begin{Phonetics}{学分}{xue2fen1}[][HSK 4]
    \definition{s.}{créditos de um curso; uma unidade de medida do peso e do tempo do curso no ensino superior; cada curso vale um crédito para uma aula por semana durante um semestre; alunos devem concluir o número necessário de créditos para se formar}
  \end{Phonetics}
\end{Entry}

\begin{Entry}{学术}{8,5}{⼦、⽊}
  \begin{Phonetics}{学术}{xue2shu4}[][HSK 4]
    \definition[种]{s.}{aprendizagem; aprendizado; ciências; aprendizado sistemático e especializado}
  \end{Phonetics}
\end{Entry}

\begin{Entry}{学生}{8,5}{⼦、⽣}
  \begin{Phonetics}{学生}{xue2sheng5}[][HSK 1]
    \definition{s.}{aluno; estudante; pupilo}
  \end{Phonetics}
\end{Entry}

\begin{Entry}{学生证}{8,5,7}{⼦、⽣、⾔}
  \begin{Phonetics}{学生证}{xue2sheng5zheng4}
    \definition{s.}{cartão de identidade de estudante}
  \end{Phonetics}
\end{Entry}

\begin{Entry}{学会}{8,6}{⼦、⼈}
  \begin{Phonetics}{学会}{xue2 hui4}[][HSK 6]
    \definition[个]{s.}{sociedade; instituto; sociedade científica; um grupo acadêmico composto por pessoas que estudam um determinado assunto, como a Sociedade de Física, a Sociedade de Biologia, etc.}
    \definition{v.}{aprender; dominar; aprender e aplicar}
  \end{Phonetics}
\end{Entry}

\begin{Entry}{学好}{8,6}{⼦、⼥}
  \begin{Phonetics}{学好}{xue2hao3}
    \definition{v.}{seguir bons exemplos | aprender bem}
  \end{Phonetics}
\end{Entry}

\begin{Entry}{学年}{8,6}{⼦、⼲}
  \begin{Phonetics}{学年}{xue2 nian2}[][HSK 4]
    \definition{s.}{ano letivo; ano acadêmico}
  \end{Phonetics}
\end{Entry}

\begin{Entry}{学问}{8,6}{⼦、⾨}
  \begin{Phonetics}{学问}{xue2wen4}[][HSK 4]
    \definition[门,种,个,项]{s.}{aprendizado, conhecimento, erudição; a compreensão correta do mundo objetivo que alguém tem | conhecimento; aprendizado sistemático; conhecimento sistemático sobre algo ou uma ciência que pode ser aprendido em um livro ou em uma experiência prática}
  \end{Phonetics}
\end{Entry}

\begin{Entry}{学位}{8,7}{⼦、⼈}
  \begin{Phonetics}{学位}{xue2wei4}[][HSK 5]
    \definition[个]{s.}{grau; grau acadêmico; título concedido com base no nível acadêmico profissional, como doutorado, mestrado, etc.}
  \end{Phonetics}
\end{Entry}

\begin{Entry}{学员}{8,7}{⼦、⼝}
  \begin{Phonetics}{学员}{xue2 yuan2}[][HSK 6]
    \definition[位,名,批,个]{s.}{estudante; estagiário; geralmente se refere a pessoas que estudam em escolas ou cursos de treinamento diferentes de faculdades, escolas de ensino médio e escolas primárias}
  \end{Phonetics}
\end{Entry}

\begin{Entry}{学时}{8,7}{⼦、⽇}
  \begin{Phonetics}{学时}{xue2 shi2}[][HSK 4]
    \definition{s.}{hora-aula; hora de aula; período}
  \end{Phonetics}
\end{Entry}

\begin{Entry}{学者}{8,8}{⼦、⽼}
  \begin{Phonetics}{学者}{xue2 zhe3}[][HSK 5]
    \definition[位]{s.}{erudito; homem culto; pessoas que fazem pesquisas acadêmicas geralmente se referem àquelas que alcançaram certo sucesso acadêmico}
  \end{Phonetics}
\end{Entry}

\begin{Entry}{学科}{8,9}{⼦、⽲}
  \begin{Phonetics}{学科}{xue2 ke1}[][HSK 5]
    \definition[门,级]{s.}{ramo do aprendizado; disciplina | disciplina escolar; curso de estudo | cursos teóricos oferecidos em treinamento militar ou físico (oposto a 术科)  | disciplina acadêmica | curso | assunto; tema}
  \seealsoref{术科}{shu4ke1}
  \end{Phonetics}
\end{Entry}

\begin{Entry}{学费}{8,9}{⼦、⾙}
  \begin{Phonetics}{学费}{xue2 fei4}[][HSK 3]
    \definition[笔]{s.}{mensalidade (taxa); prêmio; taxas que os alunos devem pagar para estudar na escola, conforme estabelecido pela escola | preço pelo que se aprendeu ao custo do próprio bolso; a metáfora do preço a pagar para obter uma determinada experiência | custo; preço; todas as despesas necessárias durante o período de estudos do aluno}
  \end{Phonetics}
\end{Entry}

\begin{Entry}{学院}{8,9}{⼦、⾩}
  \begin{Phonetics}{学院}{xue2yuan4}[][HSK 1]
    \definition[个,所]{s.}{academia; instituto; um tipo de instituição de ensino superior que se concentra em uma determinada área de especialização, como faculdades de engenharia, faculdades de música, faculdades de educação, etc.}
  \end{Phonetics}
\end{Entry}

\begin{Entry}{学校}{8,10}{⼦、⽊}
  \begin{Phonetics}{学校}{xue2xiao4}[][HSK 1]
    \definition[所,个]{s.}{escola; instituição de ensino}
  \end{Phonetics}
\end{Entry}

\begin{Entry}{学期}{8,12}{⼦、⽉}
  \begin{Phonetics}{学期}{xue2qi1}[][HSK 2]
    \definition[个,段]{s.}{semestre; período escolar; um ano acadêmico é dividido em dois semestres, um semestre do início do outono até as férias de inverno e um semestre do início da primavera até as férias de verão}
  \end{Phonetics}
\end{Entry}

\begin{Entry}{孩}{9}{⼦}
  \begin{Phonetics}{孩}{hai2}
    \definition[个]{s.}{criança}
  \end{Phonetics}
\end{Entry}

\begin{Entry}{孩子}{9,3}{⼦、⼦}
  \begin{Phonetics}{孩子}{hai2 zi5}[][HSK 1]
    \definition[个]{s.}{criança; crianças; pessoas com idade entre alguns anos ou na adolescência, geralmente com menos de 14 anos | crianças; filho ou filha}
  \end{Phonetics}
\end{Entry}

\begin{Entry}{孵}{14}{⼦}
  \begin{Phonetics}{孵}{fu1}
    \definition{v.}{chocar; incubar; (pássaros) sentar em ovos}
  \end{Phonetics}
\end{Entry}

\begin{Entry}{孵化}{14,4}{⼦、⼔}
  \begin{Phonetics}{孵化}{fu1hua4}[][HSK 7-9]
    \definition{v.}{chocar; incubar | incubar; metaforicamente, cultivar e desenvolver coisas novas (agora se refere principalmente ao suporte a empresas de alta tecnologia recém-criadas)}
  \end{Phonetics}
\end{Entry}

%%%%% EOF %%%%%

