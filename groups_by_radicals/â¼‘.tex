%%%
%%% Radical "⼑"
%%%

\section*{Radical 18: ``⼑'' (刂、⺈)}\addcontentsline{toc}{section}{Radical 18: ⼑、刂、⺈}

\begin{entry}{刀}{2}{⼑}[Kangxi 18]
  \begin{phonetics}{刀}{dao1}[][HSK 3]
    \definition*{s.}{sobrenome Dao}
    \definition{clas.}{para cortes de faca ou facadas | para cem folhas (de papel)}
    \definition[把]{s.}{faca; espada | algo em forma de faca | moeda antiga em forma de faca}
  \end{phonetics}
\end{entry}

\begin{entry}{分}{4}{⼑}
  \begin{phonetics}{分}{fen1}[][HSK 1,2]
    \definition{adj.}{filial (de uma organização)}
    \definition{clas.}{parte ou subdivisão | fração | um décimo (de certas unidades) | unidade de comprimento equivalente a 0,33cm | unidade de área (=66,666 metros quadrados) | unidade de peso (=1/2 grama) | minuto (unidade de tempo) | minuto (unidade de medida angular) | 0,01 yuan (unidade de dinheiro) | taxa de juros | marca; ponto; unidade de contagem para avaliação de notas, etc.}
    \definition{s.}{fração}
    \definition{v.}{separar; dividir; partir; dividir algo inteiro em várias partes ou separar coisas que estão ligadas entre si | atribuir; designar; distribuir | distinguir; diferenciar; diferenciar um do outro}
  \end{phonetics}
  \begin{phonetics}{分}{fen4}[][HSK 2]
    \definition{s.}{componente | o que está dentro dos deveres ou direitos de alguém; limites das responsabilidades e direitos | afeto; sentimento de amizade}
    \definition{v.}{pensar; esperar; estimar}
  \end{phonetics}
\end{entry}

\begin{entry}{分之}{4,3}{⼑、⼂}
  \begin{phonetics}{分之}{fen1 zhi1}[][HSK 4]
    \definition{expr.}{indicando uma fração; formatação e leitura de frações, ou seja, partes de um total}[顾客减少了三分之一。 (O número de clientes caiu em um terço.)]
  \end{phonetics}
\end{entry}

\begin{entry}{分子}{4,3}{⼑、⼦}
  \begin{phonetics}{分子}{fen1zi3}
    \definition{s.}{molécula | (matemática) numerador de uma fração}
  \end{phonetics}
  \begin{phonetics}{分子}{fen4zi3}
    \definition{s.}{membros de uma classe ou grupo | elementos políticos (como intelectuais ou extremistas)}
  \end{phonetics}
\end{entry}

\begin{entry}{分为}{4,4}{⼑、⼂}
  \begin{phonetics}{分为}{fen1 wei2}[][HSK 4]
    \definition{v.}{subdividir; dividir algo em}
  \end{phonetics}
\end{entry}

\begin{entry}{分公司}{4,4,5}{⼑、⼋、⼝}
  \begin{phonetics}{分公司}{fen1gong1si1}
    \definition{s.}{sucursal | filial de companhia}
  \end{phonetics}
\end{entry}

\begin{entry}{分开}{4,4}{⼑、⼶}
  \begin{phonetics}{分开}{fen1 kai1}[][HSK 2]
    \definition{v.+compl.}{separar; dividir; desacoplar; desembalar; romper; desfolhar; decolar; romper; distribuir; separar de (em); dividir\dots de\dots | separar; fazer com que uma pessoa ou algo deixe de estar junto com outra pessoa ou coisa}
  \end{phonetics}
\end{entry}

\begin{entry}{分手}{4,4}{⼑、⼿}
  \begin{phonetics}{分手}{fen1shou3}[][HSK 4]
    \definition{v.+compl.}{separar; romper; terminar um relacionamento ou um casal | separar-se (de uma empresa); dizer adeus; despedir-se da família, dos amigos, etc.}
  \end{phonetics}
\end{entry}

\begin{entry}{分布}{4,5}{⼑、⼱}
  \begin{phonetics}{分布}{fen1bu4}[][HSK 4]
    \definition{v.}{espalhar; distribuir; dispersar (em uma determinada área)}
  \end{phonetics}
\end{entry}

\begin{entry}{分成}{4,6}{⼑、⼽}
  \begin{phonetics}{分成}{fen1 cheng2}[][HSK 5]
    \definition{v.}{dividir em; separar em; dividir dinheiro, bens, etc. de acordo com a porcentagem}
  \end{phonetics}
\end{entry}

\begin{entry}{分别}{4,7}{⼑、⼑}
  \begin{phonetics}{分别}{fen1bie2}[][HSK 3]
    \definition{adv.}{diferentemente; de maneiras diferentes}
    \definition{s.}{diferença}
    \definition{v.}{partir; deixar um ao outro | distinguir; diferenciar}
  \end{phonetics}
\end{entry}

\begin{entry}{分享}{4,8}{⼑、⼇}
  \begin{phonetics}{分享}{fen1 xiang3}[][HSK 5]
    \definition{v.}{compartilhar; partilhar}
  \end{phonetics}
\end{entry}

\begin{entry}{分析}{4,8}{⼑、⽊}
  \begin{phonetics}{分析}{fen1xi1}[][HSK 5]
    \definition{v.}{analisar; dividir uma coisa, um fenômeno, um conceito em componentes mais simples e descobrir as propriedades essenciais desses componentes e a relação entre eles (em oposição à 综合)}
  \seealsoref{综合}{zong1he2}
  \end{phonetics}
\end{entry}

\begin{entry}{分组}{4,8}{⼑、⽷}
  \begin{phonetics}{分组}{fen1 zu3}[][HSK 3]
    \definition{v.}{agrupar; dividir em grupos}
  \end{phonetics}
\end{entry}

\begin{entry}{分类}{4,9}{⼑、⽶}
  \begin{phonetics}{分类}{fen1 lei4}[][HSK 5]
    \definition{v.+compl.}{ordenar; classificar; categorizar; classificar as coisas de acordo com sua natureza e características}
  \end{phonetics}
\end{entry}

\begin{entry}{分钟}{4,9}{⼑、⾦}
  \begin{phonetics}{分钟}{fen1zhong1}[][HSK 2]
    \definition{clas.}{minuto (usado em intervalos de tempo); 60 segundos}
  \end{phonetics}
\end{entry}

\begin{entry}{分离}{4,10}{⼑、⼇}
  \begin{phonetics}{分离}{fen1 li2}[][HSK 5]
    \definition{v.}{cortar; separar (de coisas) | separar; sair; separar (de pessoas); partir (em uma longa viagem)}
  \end{phonetics}
\end{entry}

\begin{entry}{分配}{4,10}{⼑、⾣}
  \begin{phonetics}{分配}{fen1pei4}[][HSK 3]
    \definition{v.}{atribuir; dispor | atribuir; compartilhar; distribuir}
  \end{phonetics}
\end{entry}

\begin{entry}{分散}{4,12}{⼑、⽁}
  \begin{phonetics}{分散}{fen1san4}[][HSK 4]
    \definition{adj.}{espalhado; disperso; desviado; fragmentado; sem foco}
    \definition{v.}{dispersar; espalhar; descentralizar | separar-se; desunir-se}
  \end{phonetics}
\end{entry}

\begin{entry}{分量}{4,12}{⼑、⾥}
  \begin{phonetics}{分量}{fen1liang4}
    \definition{s.}{componente vetorial}
  \end{phonetics}
  \begin{phonetics}{分量}{fen4liang4}
    \definition{s.}{tamanho da porção (comida)}
  \end{phonetics}
  \begin{phonetics}{分量}{fen4liang5}
    \definition{s.}{quantidade | peso | medida}
  \end{phonetics}
\end{entry}

\begin{entry}{分数}{4,13}{⼑、⽁}
  \begin{phonetics}{分数}{fen1 shu4}[][HSK 2]
    \definition[个]{s.}{fração; número fracionário | nota; classificação; ponto; pontuação registrada ao avaliar o resultado ou a vitória/derrota}
  \end{phonetics}
\end{entry}

\begin{entry}{分解}{4,13}{⼑、⾓}
  \begin{phonetics}{分解}{fen1jie3}[][HSK 5]
    \definition{v.}{quebrar; separar em partes; dividir um todo em seus componentes | resolver; decompor; (química) transformar uma substância em duas ou mais substâncias por meio de uma reação química | desintegrar-se; dividir-se; desunir uma organização | mediar; fazer a paz; resolver conflitos e disputas | explicar; defender-se}
  \end{phonetics}
\end{entry}

\begin{entry}{切}{4}{⼑}
  \begin{phonetics}{切}{qie1}[][HSK 4]
    \definition{v.}{cortar; fatiar; separar itens com uma faca | cortar ou romper; truncar | (geometria) geometria, refere-se a quando uma linha, círculo ou superfície intercepta um círculo, arco ou esfera em apenas um ponto}
  \end{phonetics}
  \begin{phonetics}{切}{qie4}
    \definition{adj.}{ansioso; sério | duro; severo; rude; áspero}
    \definition{adv.}{com certeza; certamente}
    \definition{s.}{limiar; degrau}
    \definition{v.}{ser prático ou realista | ajustar-se ou corresponder | ser próximo ou íntimo | cortar algo em pedaços com uma faca | (medicina tradicional chinesa) tomar o pulso}
  \end{phonetics}
\end{entry}

\begin{entry}{切割}{4,12}{⼑、⼑}
  \begin{phonetics}{切割}{qie1ge1}
    \definition{v.}{cortar}
  \end{phonetics}
\end{entry}

\begin{entry}{争}{6}{⼑}
  \begin{phonetics}{争}{zheng1}[][HSK 3]
    \definition*{s.}{sobrenome Zheng}
    \definition{adv.}{como; por que}
    \definition{v.}{contender; competir; lutar por; esforçar-se | argumentar; disputar; debater | faltar; estar aquém de}
  \end{phonetics}
\end{entry}

\begin{entry}{争风吃醋}{6,4,6,15}{⼑、⾵、⼝、⾣}
  \begin{phonetics}{争风吃醋}{zheng1feng1chi1cu4}
    \definition{v.}{rivalizar com alguém pelo carinho de um homem ou mulher |estar com ciúmes de um rival em um caso de amor}
  \end{phonetics}
\end{entry}

\begin{entry}{争议}{6,5}{⼑、⾔}
  \begin{phonetics}{争议}{zheng1yi4}[][HSK 5]
    \definition{s.}{disputa; controvérsia; situações e questões em que há divergências de opinião}
    \definition{v.}{debater; discutir}
  \end{phonetics}
\end{entry}

\begin{entry}{争先}{6,6}{⼑、⼉}
  \begin{phonetics}{争先}{zheng1xian1}
    \definition{v.}{competir para ser o primeiro |contestar o primeiro lugar}
  \end{phonetics}
\end{entry}

\begin{entry}{争论}{6,6}{⼑、⾔}
  \begin{phonetics}{争论}{zheng1lun4}[][HSK 4]
    \definition{s.}{debate; discussão; argumentação; disputa}
    \definition{v.}{discutir; disputar; debater; argumentar; contestar}
  \end{phonetics}
\end{entry}

\begin{entry}{争取}{6,8}{⼑、⼜}
  \begin{phonetics}{争取}{zheng1qu3}[][HSK 3]
    \definition{v.}{esforçar-se por; lutar por; vencer}
  \end{phonetics}
\end{entry}

\begin{entry}{争霸}{6,21}{⼑、⾬}
  \begin{phonetics}{争霸}{zheng1ba4}
    \definition{s.}{hegemonia | uma luta de poder}
    \definition{v.}{disputar a hegemonia}
  \end{phonetics}
\end{entry}

\begin{entry}{划}{6}{⼑}
  \begin{phonetics}{划}{hua2}[][HSK 4]
    \definition{adj.}{rentável; vale (o esforço); compensa (fazer alguma coisa)}
    \definition{v.}{remar | ser vantajoso para alguém; ser uma pechincha | arranhar; cortar a superfície de; cortar em outra coisa com um objeto pontiagudo | arranhar; golpear;  esfregar uma coisa ou varrer sobre outra}
  \end{phonetics}
  \begin{phonetics}{划}{hua4}[][HSK 4]
    \definition{s.}{traço de um caracter chinês}
    \definition{v.}{delimitar; diferenciar; delinear | transferir; ceder | planejar; programar | desenhar; marcar; delinear; fazer linhas ou escrever como marcadores com uma caneta ou objeto semelhante a uma caneta}
  \end{phonetics}
\end{entry}

\begin{entry}{划分}{6,4}{⼑、⼑}
  \begin{phonetics}{划分}{hua4fen1}[][HSK 5]
    \definition{v.}{dividir; particionar; reparticionar | diferenciar; encontrar aspectos diferentes}
  \end{phonetics}
\end{entry}

\begin{entry}{划船}{6,11}{⼑、⾈}
  \begin{phonetics}{划船}{hua2 chuan2}[][HSK 3]
    \definition[次,回]{s.}{remo (ato de remar); passeios de barco}
    \definition{v.}{remar um barco}
  \end{phonetics}
\end{entry}

\begin{entry}{划艇}{6,12}{⼑、⾈}
  \begin{phonetics}{划艇}{hua2ting3}
    \definition{s.}{barco a remo}
  \end{phonetics}
\end{entry}

\begin{entry}{列}{6}{⼑}
  \begin{phonetics}{列}{lie4}[][HSK 4]
    \definition{v.}{organizar; formar uma linha; alinhar | listar; inserir em uma lista}
  \end{phonetics}
\end{entry}

\begin{entry}{列入}{6,2}{⼑、⼊}
  \begin{phonetics}{列入}{lie4 ru4}[][HSK 4]
    \definition{v.}{incluir em uma lista}
  \end{phonetics}
\end{entry}

\begin{entry}{列为}{6,4}{⼑、⼂}
  \begin{phonetics}{列为}{lie4 wei2}[][HSK 4]
    \definition{v.}{ser classificado como; ser listado como}
  \end{phonetics}
\end{entry}

\begin{entry}{列车}{6,4}{⼑、⾞}
  \begin{phonetics}{列车}{lie4che1}[][HSK 4]
    \definition{s.}{trem; trem em uma composição contínua, puxado por uma locomotiva e equipado com uma tripulação e marcações prescritas; geralmente um trem de passageiros}
  \end{phonetics}
\end{entry}

\begin{entry}{刘}{6}{⼑}
  \begin{phonetics}{刘}{liu2}
    \definition*{s.}{sobrenome Liu}
    \definition{s.}{(clássico) um tipo de machado de batalha}
    \definition{v.}{matar}
  \end{phonetics}
\end{entry}

\begin{entry}{刚}{6}{⼑}
  \begin{phonetics}{刚}{gang1}[][HSK 2]
    \definition*{s.}{sobrenome Gang}
    \definition{adj.}{duro; firme; rígido; forte; (personalidade, atitude) forte; (vontade) determinada}
    \definition{adv.}{apenas; exatamente; justamente | apenas; apenas por pouco; significa atingir um certo nível com dificuldade | apenas; há pouco tempo; indica que a ação ou situação ocorreu há pouco tempo | assim que; somente neste momento; aconteceu que; use a palavra 就 para indicar que duas coisas estão intimamente relacionadas.}
  \seealsoref{就}{jiu4}
  \end{phonetics}
\end{entry}

\begin{entry}{刚才}{6,3}{⼑、⼿}
  \begin{phonetics}{刚才}{gang1cai2}[][HSK 2]
    \definition{s.}{agora mesmo; há pouco; há pouco tempo; referindo-se ao período recente que acabou de passar}
  \end{phonetics}
\end{entry}

\begin{entry}{刚刚}{6,6}{⼑、⼑}
  \begin{phonetics}{刚刚}{gang1 gang5}[][HSK 2]
    \definition{adv.}{apenas; somente; exatamente; refere-se a algo que é adequado em termos de grau, quantidade, tempo, etc., nem mais nem menos, nem cedo nem tarde, atingindo um estado satisfatório ou que atende exatamente às necessidades | agora mesmo; há pouco; há um momento atrás; referindo-se a um período de tempo muito curto no passado}
  \end{phonetics}
\end{entry}

\begin{entry}{创业}{6,5}{⼑、⼀}
  \begin{phonetics}{创业}{chuang4ye4}[][HSK 3]
    \definition{s.}{empreendedorismo}
    \definition{v.}{começar um empreendimento; iniciar um negócio, uma empresa | esculpir}
  \end{phonetics}
\end{entry}

\begin{entry}{创立}{6,5}{⼑、⽴}
  \begin{phonetics}{创立}{chuang4li4}[][HSK 5]
    \definition{v.}{fundar; originar; estabelecer}
  \end{phonetics}
\end{entry}

\begin{entry}{创作}{6,7}{⼑、⼈}
  \begin{phonetics}{创作}{chuang4zuo4}[][HSK 3]
    \definition[个]{s.}{criação; trabalho criativo}
    \definition{v.}{escrever; criar; produzir; compor}
  \end{phonetics}
\end{entry}

\begin{entry}{创造}{6,10}{⼑、⾡}
  \begin{phonetics}{创造}{chuang4zao4}[][HSK 3]
    \definition{s.}{criação; inovação}
    \definition{v.}{criar; produzir; trazer à tona}
  \end{phonetics}
\end{entry}

\begin{entry}{创意}{6,13}{⼑、⼼}
  \begin{phonetics}{创意}{chuang4yi4}
    \definition{adj.}{criativo}
    \definition{s.}{criatividade}
  \end{phonetics}
\end{entry}

\begin{entry}{创新}{6,13}{⼑、⽄}
  \begin{phonetics}{创新}{chuang4xin1}[][HSK 3]
    \definition[个,种,次]{s.}{inovação}
    \definition{v.}{trazer novas ideias; inovar; abrir novos caminhos; criar algo novo}
  \end{phonetics}
\end{entry}

\begin{entry}{判断}{7,11}{⼑、⽄}
  \begin{phonetics}{判断}{pan4duan4}[][HSK 3]
    \definition[个]{s.}{julgamento}
    \definition{v.}{julgar; decidir}
  \end{phonetics}
\end{entry}

\begin{entry}{利}{7}{⼑}
  \begin{phonetics}{利}{li4}
    \definition*{s.}{sobrenome Li}
    \definition{adj.}{afiado; cortante | favorável; conveniente; sem dificuldades; sem ou com poucas dificuldades}
    \definition{s.}{benefício; vantagem | lucro; ganhos; juros}
    \definition{v.}{beneficiar; tornar vantajoso}
  \end{phonetics}
\end{entry}

\begin{entry}{利用}{7,5}{⼑、⽤}
  \begin{phonetics}{利用}{li4yong4}[][HSK 3]
    \definition{v.}{usar; utilizar; fazer uso de | explorar; tirar vantagem de}
  \end{phonetics}
\end{entry}

\begin{entry}{利息}{7,10}{⼑、⼼}
  \begin{phonetics}{利息}{li4xi1}[][HSK 4]
    \definition{s.}{acréscimo; juros; dinheiro recebido além do valor principal como resultado de depósitos ou empréstimos (diferenciado de 本金)}
  \seealsoref{本金}{ben3 jin1}
  \end{phonetics}
\end{entry}

\begin{entry}{利润}{7,10}{⼑、⽔}
  \begin{phonetics}{利润}{li4run4}[][HSK 5]
    \definition[笔]{s.}{lucro; o dinheiro ganho com atividades comerciais e industriais}
  \end{phonetics}
\end{entry}

\begin{entry}{利益}{7,10}{⼑、⽫}
  \begin{phonetics}{利益}{li4yi4}[][HSK 4]
    \definition[个,种]{s.}{ganho; lucro; juros; benefício}
  \end{phonetics}
\end{entry}

\begin{entry}{别}{7}{⼑}
  \begin{phonetics}{别}{bie2}[][HSK 1,4]
    \definition*{s.}{sobrenome Bie}
    \definition{adv.}{não; nada de (pedir a alguém para não fazer); é melhor não | talvez, usado em conjunto com a palavra 是 para indicar especulação.}
    \definition{pron.}{outro; algum outro}
    \definition{s.}{distinção; diferença | classificação}
    \definition{v.}{deixar; partir; separar | diferenciar; distinguir; encontrar aspectos diferentes | fixar objetos com pinos | girar; virar | aderir; colar; preder}
  \seealsoref{是}{shi4}
  \end{phonetics}
  \begin{phonetics}{别}{bie4}
    \definition{v.}{fazer com que alguém mude seus hábitos, opiniões, etc.}
  \end{phonetics}
\end{entry}

\begin{entry}{别人}{7,2}{⼑、⼈}
  \begin{phonetics}{别人}{bie2 ren2}[][HSK 1]
    \definition{pron.}{outros; outras pessoas}
    \definition{s.}{outros; pessoas; outras pessoas; refere-se a alguém diferente de si mesmo}
  \end{phonetics}
\end{entry}

\begin{entry}{别的}{7,8}{⼑、⽩}
  \begin{phonetics}{别的}{bie2 de5}[][HSK 1]
    \definition{pron.}{outro; o resto}
  \end{phonetics}
\end{entry}

\begin{entry}{别说}{7,9}{⼑、⾔}
  \begin{phonetics}{别说}{bie2shuo1}
    \definition{v.}{não falar de | não mencionar}
  \end{phonetics}
\end{entry}

\begin{entry}{刮}{8}{⼑}
  \begin{phonetics}{刮}{gua1}
    \definition{v.}{ventar | soprar (vento)}
  \end{phonetics}
\end{entry}

\begin{entry}{刮风}{8,4}{⼑、⾵}
  \begin{phonetics}{刮风}{gua1feng1}
    \definition{v.+compl.}{ventar | fazer vento}
  \end{phonetics}
\end{entry}

\begin{entry}{到}{8}{⼑}
  \begin{phonetics}{到}{dao4}[][HSK 1]
    \definition*{s.}{sobrenome Dao}
    \definition{adj.}{atencioso}
    \definition{prep.}{a; até; para; indica o tempo em que a ação ou comportamento foi alcançado}
    \definition{v.}{ir para; partir para | chegar; alcançar; chegar a | como complemento de um verbo para mostrar o resultado de uma ação}
  \end{phonetics}
\end{entry}

\begin{entry}{到处}{8,5}{⼑、⼡}
  \begin{phonetics}{到处}{dao4chu4}[][HSK 2]
    \definition{adv.}{em todos os lugares; em todos os locais; por toda parte}
  \end{phonetics}
\end{entry}

\begin{entry}{到达}{8,6}{⼑、⾡}
  \begin{phonetics}{到达}{dao4da2}[][HSK 3]
    \definition{v.}{chegar; alcançar}
  \end{phonetics}
\end{entry}

\begin{entry}{到来}{8,7}{⼑、⽊}
  \begin{phonetics}{到来}{dao4 lai2}[][HSK 5]
    \definition{v.}{chegar; chegar aqui de outro lugar}
  \end{phonetics}
\end{entry}

\begin{entry}{到底}{8,8}{⼑、⼴}
  \begin{phonetics}{到底}{dao4di3}[][HSK 3]
    \definition{adv.}{na terra (usado em frases interrogativas para expressar a determinação de alguém em encontrar uma resposta definitiva) | afinal | finalmente; por fim; no fim}
  \end{phonetics}
\end{entry}

\begin{entry}{制订}{8,4}{⼑、⾔}
  \begin{phonetics}{制订}{zhi4 ding4}[][HSK 4]
    \definition{v.}{esboçar; formular; elaborar; mapear}
  \end{phonetics}
\end{entry}

\begin{entry}{制成}{8,6}{⼑、⼽}
  \begin{phonetics}{制成}{zhi4 cheng2}[][HSK 5]
    \definition{v.}{fabricar; ser feito de; produzir}
  \end{phonetics}
\end{entry}

\begin{entry}{制约}{8,6}{⼑、⽷}
  \begin{phonetics}{制约}{zhi4yue1}[][HSK 5]
    \definition{v.}{limitar; verificar; restringir; a existência e a mudança de uma coisa determinam a existência e a mudança de outra coisa}
  \end{phonetics}
\end{entry}

\begin{entry}{制作}{8,7}{⼑、⼈}
  \begin{phonetics}{制作}{zhi4zuo4}[][HSK 3]
    \definition{v.}{fazer; fabricar; fazer a partir de matérias-primas, usado principalmente para itens menores e artesanais; criar designs com palavras, imagens, sons, etc. para criar gráficos, anúncios, filmes, jogos, etc.}
  \end{phonetics}
\end{entry}

\begin{entry}{制定}{8,8}{⼑、⼧}
  \begin{phonetics}{制定}{zhi4ding4}[][HSK 3]
    \definition{v.}{rascunhar; formular; elaborar; estabelecer}
  \end{phonetics}
\end{entry}

\begin{entry}{制度}{8,9}{⼑、⼴}
  \begin{phonetics}{制度}{zhi4du4}[][HSK 3]
    \definition[个]{s.}{regulamento; regulação; regulamentação | sistema; um sistema político, econômico, cultural e outro formado sob certas condições históricas}
  \end{phonetics}
\end{entry}

\begin{entry}{制造}{8,10}{⼑、⾡}
  \begin{phonetics}{制造}{zhi4zao4}[][HSK 3]
    \definition{v.}{fazer; produzir; manufaturar; transformar matérias-primas em produtos acabados | criar; agitar; criar artificialmente uma situação ou atmosfera ruim}
  \end{phonetics}
\end{entry}

\begin{entry}{制裁}{8,12}{⼑、⾐}
  \begin{phonetics}{制裁}{zhi4cai2}
    \definition{s.}{punição | sanção (inclusive econômica)}
    \definition{v.}{punir}
  \end{phonetics}
\end{entry}

\begin{entry}{刷}{8}{⼑}
  \begin{phonetics}{刷}{shua1}[][HSK 4]
    \definition{s.}{escova; pincel | (onomatopéia) farfalhar; descreve o som de uma passagem rápida}
    \definition{v.}{escovar; esfregar; remover com uma escova | borrar; colar; aplicar com um pincel | eliminar; remover; limpar}
  \end{phonetics}
  \begin{phonetics}{刷}{shua4}
    \definition{v.}{selecionar}
  \end{phonetics}
\end{entry}

\begin{entry}{刷子}{8,3}{⼑、⼦}
  \begin{phonetics}{刷子}{shua1zi5}[][HSK 4]
    \definition[把]{s.}{escova; escovão; utensílio feito de lã, fio de plástico, fio de metal, etc., para remover sujeira ou aplicar óleo de unção, etc., geralmente longo ou oval, alguns com alças}
  \end{phonetics}
\end{entry}

\begin{entry}{刷牙}{8,4}{⼑、⽛}
  \begin{phonetics}{刷牙}{shua1ya2}[][HSK 4]
    \definition{s.}{escovar os dentes}
  \end{phonetics}
\end{entry}

\begin{entry}{刹}{8}{⼑}
  \begin{phonetics}{刹}{cha4}
    \definition{s.}{mosteiro, templo ou santuário budista | abreviação de 刹多罗 | sânscrito ``ksetra''}
  \seealsoref{刹多罗}{cha4duo1luo2}
  \end{phonetics}
  \begin{phonetics}{刹}{sha1}
    \definition{v.}{frear}
  \end{phonetics}
\end{entry}

\begin{entry}{刹多罗}{8,6,8}{⼑、⼣、⽹}
  \begin{phonetics}{刹多罗}{cha4duo1luo2}
    \definition*{s.}{Kshatara, sânscrito ``ksetra''}
  \end{phonetics}
\end{entry}

\begin{entry}{刺}{8}{⼑}
  \begin{phonetics}{刺}{ci1}
    \definition{s.}{(onomatopéia) som de rasgo, fricção, etc.}
  \end{phonetics}
  \begin{phonetics}{刺}{ci4}[][HSK 4]
    \definition*{s.}{sobrenome Ci}
    \definition{s.}{espinho; farpa; algo afiado como uma agulha | cartão de visita | saliências; projeções pequenas e pontiagudas na superfície de um objeto ou na pele de uma pessoa}
    \definition{v.}{esfaquear; perfurar | irritar; estimular | assassinar | espionar; detectar | criticar}
  \end{phonetics}
\end{entry}

\begin{entry}{刺猬}{8,12}{⼑、⽝}
  \begin{phonetics}{刺猬}{ci4wei5}
    \definition{s.}{porco-espinho | ouriço}
  \end{phonetics}
\end{entry}

\begin{entry}{刺激}{8,16}{⼑、⽔}
  \begin{phonetics}{刺激}{ci4ji1}[][HSK 4]
    \definition{adj.}{animado; entusiasmado; sensação de empolgação e nervosismo}
    \definition[个]{s.}{estímulo; estimulação; fortes efeitos físicos ou psicológicos}
    \definition{v.}{irritar; provocar; estimular | incentivar; estimular; incitar; (por algum meio) para mudar as coisas para melhor, para fazer coisas positivas}
  \end{phonetics}
\end{entry}

\begin{entry}{刻}{8}{⼑}
  \begin{phonetics}{刻}{ke4}[][HSK 2,5]
    \definition{adj.}{cruel; severo; rude; indelicado | no mais alto grau}
    \definition{clas.}{um quarto (de uma hora, 15min)}
    \definition[件]{s.}{quarto (de hora); momento}
    \definition{v.}{esculpir; inscrever; gravar; talhar com uma faca (padrões, texto, etc.) | definir um limite de tempo | imprimir (CD)}
  \end{phonetics}
\end{entry}

\begin{entry}{刻画}{8,8}{⼑、⽥}
  \begin{phonetics}{刻画}{ke4hua4}
    \definition{v.}{retratar | tirar um retrato}
  \end{phonetics}
\end{entry}

\begin{entry}{刻钟}{8,9}{⼑、⾦}
  \begin{phonetics}{刻钟}{ke4 zhong1}
    \definition{s.}{um quarto de hora}
  \end{phonetics}
\end{entry}

\begin{entry}{前}{9}{⼑}
  \begin{phonetics}{前}{qian2}[][HSK 1]
    \definition*{s.}{sobrenome Qian}
    \definition{s.}{frente | futuro; perspectiva | atrás; antes; mais cedo do que uma coisa ou um momento | à frente; para a frente; na parte frontal (referindo-se ao espaço, em oposição a 后) | precedente; antes que algo aconteça | antigo; antigamente | topo; primeiro; primeiro na ordem | frente; campo de batalha | A.C. (Antes de~Cristo)}[前293年  (293 a.C.)]
    \definition{v.}{seguir em frente; ir em frente}
  \seealsoref{公元}{gong1yuan2}
  \seealsoref{后}{hou4}
  \end{phonetics}
\end{entry}

\begin{entry}{前天}{9,4}{⼑、⼤}
  \begin{phonetics}{前天}{qian2 tian1}[][HSK 1]
    \definition{adv.}{anteontem; dia anterior a ontem}
  \end{phonetics}
\end{entry}

\begin{entry}{前头}{9,5}{⼑、⼤}
  \begin{phonetics}{前头}{qian2 tou5}[][HSK 4]
    \definition{s.}{à frente; na frente; adiante}
  \end{phonetics}
\end{entry}

\begin{entry}{前边}{9,5}{⼑、⾡}
  \begin{phonetics}{前边}{qian2 bian5}[][HSK 1]
    \definition{adv.}{à frente; na frente}
  \end{phonetics}
\end{entry}

\begin{entry}{前后}{9,6}{⼑、⼝}
  \begin{phonetics}{前后}{qian2 hou4}[][HSK 3]
    \definition{s.}{em volta; sobre | do início ao fim | frente e verso}
  \end{phonetics}
\end{entry}

\begin{entry}{前年}{9,6}{⼑、⼲}
  \begin{phonetics}{前年}{qian2 nian2}[][HSK 2]
    \definition{adv.}{há dois anos; dois anos atrás}
  \end{phonetics}
\end{entry}

\begin{entry}{前进}{9,7}{⼑、⾡}
  \begin{phonetics}{前进}{qian2 jin4}[][HSK 3]
    \definition{v.}{marchar; avançar; para ir em frente; seguir em frente}
  \end{phonetics}
\end{entry}

\begin{entry}{前往}{9,8}{⼑、⼻}
  \begin{phonetics}{前往}{qian2 wang3}[][HSK 3]
    \definition{v.}{ir para; prosseguir para; partir para}
  \end{phonetics}
\end{entry}

\begin{entry}{前面}{9,9}{⼑、⾯}
  \begin{phonetics}{前面}{qian2mian4}[][HSK 3]
    \definition{s.}{frente | parte anterior; acima}
  \end{phonetics}
\end{entry}

\begin{entry}{前途}{9,10}{⼑、⾡}
  \begin{phonetics}{前途}{qian2tu2}[][HSK 4]
    \definition[片,段,种]{s.}{futuro; perspectiva; prospecto; originalmente, refere-se à jornada à frente, mas, metaforicamente, refere-se ao futuro.}
  \end{phonetics}
\end{entry}

\begin{entry}{前提}{9,12}{⼑、⼿}
  \begin{phonetics}{前提}{qian2ti2}[][HSK 5]
    \definition[个,项]{s.}{premissa; pressuposto | pré-requisito; pressuposição; condições prévias para que algo aconteça ou se desenvolva}
  \end{phonetics}
\end{entry}

\begin{entry}{前景}{9,12}{⼑、⽇}
  \begin{phonetics}{前景}{qian2jing3}[][HSK 5]
    \definition{s.}{primeiro plano (de uma vista, imagem, foto, etc.); as imagens que parecem mais próximas do espectador em pinturas, palcos e telas | vista; perspectiva; prospecto; ponto de vista; situações que podem ocorrer no trabalho, na carreira, etc.}
  \end{phonetics}
\end{entry}

\begin{entry}{剑}{9}{⼑}
  \begin{phonetics}{剑}{jian4}
    \definition{clas.}{para golpes de uma espada}
    \definition[口,把]{s.}{espada de dois gumes}
  \end{phonetics}
\end{entry}

\begin{entry}{剑客}{9,9}{⼑、⼧}
  \begin{phonetics}{剑客}{jian4ke4}
    \definition{s.}{espada | esgrimista, espadachim}
  \end{phonetics}
\end{entry}

\begin{entry}{剧本}{10,5}{⼑、⽊}
  \begin{phonetics}{剧本}{ju4ben3}[][HSK 5]
    \definition{s.}{cenário; roteiro (para drama, filme, etc.); gênero de obra literária que consiste em diálogos entre personagens (às vezes cantados) e indicações de palco}
  \end{phonetics}
\end{entry}

\begin{entry}{剧场}{10,6}{⼑、⼟}
  \begin{phonetics}{剧场}{ju4 chang3}[][HSK 3]
    \definition[个,坐]{s.}{teatro; um lugar para apresentações teatrais, canto e dança, etc.}
  \end{phonetics}
\end{entry}

\begin{entry}{剪}{11}{⼑}
  \begin{phonetics}{剪}{jian3}[][HSK 5]
    \definition[把]{s.}{tesouras; tesouras de poda; cortadores | pinças; tenazes}
    \definition{v.}{cortar; aparar; tosquiar; cortar (com uma tesoura) | exterminar; eliminar; acabar com}
  \end{phonetics}
\end{entry}

\begin{entry}{剪刀}{11,2}{⼑、⼑}
  \begin{phonetics}{剪刀}{jian3dao1}[][HSK 5]
    \definition[把]{s.}{tesoura; instrumento de ferro para cortar tecido, papel, barbante, etc., com duas lâminas interligadas que podem ser abertas e fechadas}
  \end{phonetics}
\end{entry}

\begin{entry}{剪子}{11,3}{⼑、⼦}
  \begin{phonetics}{剪子}{jian3 zi5}[][HSK 5]
    \definition[把]{s.}{cortador; tosquiadeira | tesoura}
  \end{phonetics}
\end{entry}

\begin{entry}{副}{11}{⼑}
  \begin{phonetics}{副}{fu4}
    \definition{clas.}{para pares, conjuntos de coisas e expressões faciais | para óculos, luvas, etc.}
  \end{phonetics}
\end{entry}

\begin{entry}{剩}{12}{⼑}
  \begin{phonetics}{剩}{sheng4}[][HSK 5]
    \definition*{s.}{sobrenome Sheng}
    \definition{v.}{permanecer; ser deixado (para trás);}
  \end{phonetics}
\end{entry}

\begin{entry}{剩下}{12,3}{⼑、⼀}
  \begin{phonetics}{剩下}{sheng4 xia4}[][HSK 5]
    \definition{v.}{permanecer; ser deixado (para trás); consumir e utilizar, restando apenas os resíduos}
  \end{phonetics}
\end{entry}

%%%%% EOF %%%%%

