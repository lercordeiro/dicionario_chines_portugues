%%%
%%% Radical "⼑"
%%%

\section*{Radical 18: ``⼑'' (刂、⺈)}\addcontentsline{toc}{section}{Radical 18: ⼑、刂、⺈}

\begin{Entry}{刀}{2}{⼑}[Kangxi 18]
  \begin{Phonetics}{刀}{dao1}[][HSK 3]
    \definition*{s.}{Sobrenome Dao}
    \definition{clas.}{unidade de medida para papel, geralmente cem folhas por pacote}
    \definition[把,口]{s.}{faca; espada; armas antigas, referindo-se a ferramentas para cortar, retalhar, raspar, golpear e fatiar, geralmente feitas de ferro e aço | ferramenta; ferramenta de corte; lâminas para tornos; fresas (ferramentas; ferramentas de ferro para máquinas) | algo com a forma de uma faca}
  \end{Phonetics}
\end{Entry}

\begin{Entry}{刁}{2}{⼑}
  \begin{Phonetics}{刁}{diao1}
    \definition*{s.}{Sobrenome Diao}
    \definition{adj.}{traiçoeiro; astuto | exigente; exigente com a comida; difícil}
    \definition{v.}{dificultar as coisas}
  \end{Phonetics}
\end{Entry}

\begin{Entry}{刁难}{2,10}{⼑、⾫}
  \begin{Phonetics}{刁难}{diao1nan4}[][HSK 7-9]
    \definition{v.}{criar dificuldades; tornar as coisas difíceis; dificultar deliberadamente as coisas para os outros}
  \end{Phonetics}
\end{Entry}

\begin{Entry}{分}{4}{⼑}
  \begin{Phonetics}{分}{fen1}[][HSK 1,2]
    \definition{adj.}{filial (de uma organização)}
    \definition{clas.}{parte ou subdivisão | fração | um décimo (de certas unidades) | unidade de comprimento equivalente a 0,33cm | unidade de área (=66,666 metros quadrados) | unidade de peso (=1/2 grama) | minuto (unidade de tempo) | minuto (unidade de medida angular) | 0,01 yuan (unidade de dinheiro) | taxa de juros | marca; ponto; unidade de contagem para avaliação de notas, etc.}
    \definition{s.}{fração}
    \definition{v.}{separar; dividir; partir; dividir algo inteiro em várias partes ou separar coisas que estão ligadas entre si | atribuir; designar; distribuir | distinguir; diferenciar; diferenciar um do outro}
  \end{Phonetics}
  \begin{Phonetics}{分}{fen4}[][HSK 2]
    \definition{s.}{componente | o que está dentro dos deveres ou direitos de alguém; limites das responsabilidades e direitos | afeto; sentimento de amizade}
    \definition{v.}{pensar; esperar; estimar}
  \end{Phonetics}
\end{Entry}

\begin{Entry}{分之}{4,3}{⼑、⼂}
  \begin{Phonetics}{分之}{fen1 zhi1}[][HSK 4]
    \definition{expr.}{indicando uma fração; formatação e leitura de frações, ou seja, partes de um total}[顾客减少了三分之一。===O número de clientes caiu em um terço.]
  \end{Phonetics}
\end{Entry}

\begin{Entry}{分子}{4,3}{⼑、⼦}
  \begin{Phonetics}{分子}{fen1zi3}
    \definition{s.}{molécula | (matemática) numerador de uma fração}
  \end{Phonetics}
  \begin{Phonetics}{分子}{fen4zi3}
    \definition{s.}{membros de uma classe ou grupo | elementos políticos (como intelectuais ou extremistas)}
  \end{Phonetics}
\end{Entry}

\begin{Entry}{分寸}{4,3}{⼑、⼨}
  \begin{Phonetics}{分寸}{fen1cun5}[][HSK 7-9]
    \definition{s.}{limites adequados para a fala ou ação; senso de propriedade (ou proporção)}
  \end{Phonetics}
\end{Entry}

\begin{Entry}{分工}{4,3}{⼑、⼯}
  \begin{Phonetics}{分工}{fen1 gong1}[][HSK 6]
    \definition[种]{s.}{divisão do trabalho}
    \definition{v.}{dividir o trabalho; envolver-se em várias tarefas diferentes, mas complementares}
  \end{Phonetics}
\end{Entry}

\begin{Entry}{分为}{4,4}{⼑、⼂}
  \begin{Phonetics}{分为}{fen1 wei2}[][HSK 4]
    \definition{v.}{subdividir; dividir algo em}
  \end{Phonetics}
\end{Entry}

\begin{Entry}{分公司}{4,4,5}{⼑、⼋、⼝}
  \begin{Phonetics}{分公司}{fen1gong1si1}
    \definition{s.}{sucursal | filial de companhia}
  \end{Phonetics}
\end{Entry}

\begin{Entry}{分化}{4,4}{⼑、⼔}
  \begin{Phonetics}{分化}{fen1hua4}[][HSK 7-9]
    \definition{s.}{diferenciação}
    \definition{v.}{dividir-se; fragmentar; romper; separar-se; separar | Biologia: (células ou tecidos) diferenciar}
  \end{Phonetics}
\end{Entry}

\begin{Entry}{分开}{4,4}{⼑、⼶}
  \begin{Phonetics}{分开}{fen1/kai1}[][HSK 2]
    \definition{v.+compl.}{separar; dividir; desacoplar; desembalar; romper; desfolhar; decolar; romper; distribuir; separar de (em); dividir\dots de\dots | separar; fazer com que uma pessoa ou algo deixe de estar junto com outra pessoa ou coisa}
  \end{Phonetics}
\end{Entry}

\begin{Entry}{分手}{4,4}{⼑、⼿}
  \begin{Phonetics}{分手}{fen1/shou3}[][HSK 4]
    \definition{v.+compl.}{separar; romper; terminar um relacionamento ou um casal | separar-se (de uma empresa); dizer adeus; despedir-se da família, dos amigos, etc.}
  \end{Phonetics}
\end{Entry}

\begin{Entry}{分支}{4,4}{⼑、⽀}
  \begin{Phonetics}{分支}{fen1zhi1}[][HSK 7-9]
    \definition{s.}{subdivisão; filial; afiliada; uma parte separada de um sistema ou corpo}
  \end{Phonetics}
\end{Entry}

\begin{Entry}{分外}{4,5}{⼑、⼣}
  \begin{Phonetics}{分外}{fen4wai4}[][HSK 7-9]
    \definition{adj.}{não é tarefa (dever) de alguém; além do dever de alguém; fora do escopo do dever de alguém}[他对分外的工作总是抢着干。===Ele está sempre disposto e ansioso para fazer trabalho extra.]
    \definition{adv.}{particularmente; especialmente; excepcionalmente}[雨后初晴的天空分外明朗。===O céu depois da chuva está excepcionalmente claro.]
  \end{Phonetics}
\end{Entry}

\begin{Entry}{分布}{4,5}{⼑、⼱}
  \begin{Phonetics}{分布}{fen1bu4}[][HSK 4]
    \definition{v.}{espalhar; distribuir; dispersar (em uma determinada área)}
  \end{Phonetics}
\end{Entry}

\begin{Entry}{分成}{4,6}{⼑、⼽}
  \begin{Phonetics}{分成}{fen1 cheng2}[][HSK 5]
    \definition{v.}{dividir em; separar em; dividir dinheiro, bens, etc. de acordo com a porcentagem}
  \end{Phonetics}
\end{Entry}

\begin{Entry}{分红}{4,6}{⼑、⽷}
  \begin{Phonetics}{分红}{fen1/hong2}[][HSK 7-9]
    \definition{v.+compl.}{receber dividendos; distribuir bônus; obter dividendos extras (lucros)}
  \end{Phonetics}
\end{Entry}

\begin{Entry}{分别}{4,7}{⼑、⼑}
  \begin{Phonetics}{分别}{fen1bie2}[][HSK 3]
    \definition{adv.}{diferentemente; de maneiras diferentes; expressar de maneiras diferentes | separadamente; individualmente; respectivamente}
    \definition{s.}{diferença; pontos diferentes}
    \definition{v.}{partir; deixar um ao outro; não estar mais junto | distinguir; diferenciar}
  \end{Phonetics}
\end{Entry}

\begin{Entry}{分享}{4,8}{⼑、⼇}
  \begin{Phonetics}{分享}{fen1 xiang3}[][HSK 5]
    \definition{v.}{compartilhar; partilhar}
  \end{Phonetics}
\end{Entry}

\begin{Entry}{分担}{4,8}{⼑、⼿}
  \begin{Phonetics}{分担}{fen1dan1}[][HSK 7-9]
    \definition{v.}{contribuir; compartilhar a responsabilidade por; participar}
  \end{Phonetics}
\end{Entry}

\begin{Entry}{分明}{4,8}{⼑、⽇}
  \begin{Phonetics}{分明}{fen1ming2}[][HSK 7-9]
    \definition{adj.}{claro; distinto; óbvio}
    \definition{adv.}{claramente; evidentemente; obviamente; os fatos são claros, óbvios e inquestionáveis}
  \end{Phonetics}
\end{Entry}

\begin{Entry}{分析}{4,8}{⼑、⽊}
  \begin{Phonetics}{分析}{fen1xi1}[][HSK 5]
    \definition{v.}{analisar; dividir uma coisa, um fenômeno, um conceito em componentes mais simples e descobrir as propriedades essenciais desses componentes e a relação entre eles (em oposição à 综合)}
  \seealsoref{综合}{zong1he2}
  \end{Phonetics}
\end{Entry}

\begin{Entry}{分歧}{4,8}{⼑、⽌}
  \begin{Phonetics}{分歧}{fen1qi2}[][HSK 7-9]
    \definition[点,个]{s.}{diferença; divergência; diferenças de pontos de vista, opiniões, etc.}
  \end{Phonetics}
\end{Entry}

\begin{Entry}{分泌}{4,8}{⼑、⽔}
  \begin{Phonetics}{分泌}{fen1mi4}[][HSK 7-9]
    \definition{v.}{secretar; excretar}
  \end{Phonetics}
\end{Entry}

\begin{Entry}{分组}{4,8}{⼑、⽷}
  \begin{Phonetics}{分组}{fen1 zu3}[][HSK 3]
    \definition{v.}{dividir em grupos}
  \end{Phonetics}
\end{Entry}

\begin{Entry}{分类}{4,9}{⼑、⽶}
  \begin{Phonetics}{分类}{fen1/lei4}[][HSK 5]
    \definition{v.+compl.}{ordenar; classificar; categorizar; classificar as coisas de acordo com sua natureza e características}
  \end{Phonetics}
\end{Entry}

\begin{Entry}{分钟}{4,9}{⼑、⾦}
  \begin{Phonetics}{分钟}{fen1zhong1}[][HSK 2]
    \definition{clas.}{minuto (usado em intervalos de tempo); 60 segundos}
  \end{Phonetics}
\end{Entry}

\begin{Entry}{分离}{4,10}{⼑、⼇}
  \begin{Phonetics}{分离}{fen1 li2}[][HSK 5]
    \definition{v.}{cortar; separar (de coisas) | separar; sair; separar (de pessoas); partir (em uma longa viagem)}
  \end{Phonetics}
\end{Entry}

\begin{Entry}{分赃}{4,10}{⼑、⾙}
  \begin{Phonetics}{分赃}{fen1/zang1}[][HSK 7-9]
    \definition{v.+compl.}{dividir ganhos ilícitos; compartilhar o saque}
  \end{Phonetics}
\end{Entry}

\begin{Entry}{分配}{4,10}{⼑、⾣}
  \begin{Phonetics}{分配}{fen1pei4}[][HSK 3]
    \definition{v.}{atribuir; dispor; organizar o trabalho, as tarefas, os recursos, o tempo, etc. | atribuir; compartilhar; distribuir dinheiro ou bens às pessoas envolvidas de acordo com um determinado plano, padrão ou regulamento}
  \end{Phonetics}
\end{Entry}

\begin{Entry}{分割}{4,12}{⼑、⼑}
  \begin{Phonetics}{分割}{fen1ge1}[][HSK 7-9]
    \definition{v.}{cortar; separar; quebrar; separar uma coisa inteira ou conectada}
  \end{Phonetics}
\end{Entry}

\begin{Entry}{分散}{4,12}{⼑、⽁}
  \begin{Phonetics}{分散}{fen1san4}[][HSK 4]
    \definition{adj.}{espalhado; disperso; desviado; fragmentado; sem foco}
    \definition{v.}{dispersar; espalhar; descentralizar | separar-se; desunir-se}
  \end{Phonetics}
\end{Entry}

\begin{Entry}{分裂}{4,12}{⼑、⾐}
  \begin{Phonetics}{分裂}{fen1lie4}[][HSK 6]
    \definition{s.}{fissão; divisão}
    \definition{v.}{dividir; separar; romper}
  \end{Phonetics}
\end{Entry}

\begin{Entry}{分量}{4,12}{⼑、⾥}
  \begin{Phonetics}{分量}{fen1liang4}
    \definition{s.}{componente vetorial}
  \end{Phonetics}
  \begin{Phonetics}{分量}{fen4liang4}[][HSK 7-9]
    \definition{s.}{tamanho da porção (comida)}[这道菜的分量太少了。===A porção deste prato era muito pequena.]
  \end{Phonetics}
  \begin{Phonetics}{分量}{fen4liang5}
    \definition{s.}{quantidade; peso; medida | Figurativo: peso (importância, prestígio, autoridade etc.); (de material escrito) densidade}
  \end{Phonetics}
\end{Entry}

\begin{Entry}{分数}{4,13}{⼑、⽁}
  \begin{Phonetics}{分数}{fen1 shu4}[][HSK 2]
    \definition[个]{s.}{fração; número fracionário | nota; classificação; ponto; pontuação registrada ao avaliar o resultado ou a vitória/derrota}
  \end{Phonetics}
\end{Entry}

\begin{Entry}{分解}{4,13}{⼑、⾓}
  \begin{Phonetics}{分解}{fen1jie3}[][HSK 5]
    \definition{v.}{quebrar; separar em partes; dividir um todo em seus componentes | resolver; decompor | Química: transformar uma substância em duas ou mais substâncias por meio de uma reação química | desintegrar-se; dividir-se; desunir uma organização | mediar; fazer a paz; resolver conflitos e disputas | explicar; defender-se}
  \end{Phonetics}
\end{Entry}

\begin{Entry}{分辨}{4,16}{⼑、⾟}
  \begin{Phonetics}{分辨}{fen1bian4}[][HSK 7-9]
    \definition{s.}{resolução; descreve o grau em que imagens, exibições, sensores, etc. podem distinguir claramente os detalhes}
    \definition{v.}{distinguir; diferenciar}
  \end{Phonetics}
\end{Entry}

\begin{Entry}{切}{4}{⼑}
  \begin{Phonetics}{切}{qie1}[][HSK 4]
    \definition{v.}{cortar; fatiar; separar itens com uma faca | cortar ou romper; truncar | Geometria: refere-se a quando uma linha, círculo ou superfície intercepta um círculo, arco ou esfera em apenas um ponto}
  \end{Phonetics}
  \begin{Phonetics}{切}{qie4}
    \definition{adj.}{ansioso; sério | duro; severo; rude; áspero}
    \definition{adv.}{com certeza; certamente}
    \definition{s.}{limiar; degrau}
    \definition{v.}{ser prático ou realista | ajustar-se ou corresponder | ser próximo ou íntimo | cortar algo em pedaços com uma faca | tomar o pulso (medicina tradicional chinesa)}
  \end{Phonetics}
\end{Entry}

\begin{Entry}{切实}{4,8}{⼑、⼧}
  \begin{Phonetics}{切实}{qie4shi2}[][HSK 6]
    \definition{adj.}{prático; viável; realista}
  \end{Phonetics}
\end{Entry}

\begin{Entry}{切割}{4,12}{⼑、⼑}
  \begin{Phonetics}{切割}{qie1ge1}
    \definition{v.}{esculpir; cortar algo com uma faca | cortar algo com máquina, fogo, arco voltaico; refere-se especificamente ao corte de materiais metálicos com máquinas-ferramentas ou à queima deles com arco elétrico, laser, etc.}
  \end{Phonetics}
\end{Entry}

\begin{Entry}{争}{6}{⼑}
  \begin{Phonetics}{争}{zheng1}[][HSK 3]
    \definition*{s.}{Sobrenome Zheng}
    \definition{adv.}{como; por que}
    \definition{v.}{competir; disputar; lutar; esforçar-se para obter ou alcançar | discutir; argumentar; contestar; debater | faltar; estar em falta}
  \end{Phonetics}
\end{Entry}

\begin{Entry}{争风吃醋}{6,4,6,15}{⼑、⾵、⼝、⾣}
  \begin{Phonetics}{争风吃醋}{zheng1feng1chi1cu4}
    \definition{v.}{rivalizar com alguém pelo carinho de um homem ou mulher |estar com ciúmes de um rival em um caso de amor}
  \end{Phonetics}
\end{Entry}

\begin{Entry}{争议}{6,5}{⼑、⾔}
  \begin{Phonetics}{争议}{zheng1yi4}[][HSK 5]
    \definition{s.}{disputa; controvérsia; situações e questões em que há divergências de opinião}
    \definition{v.}{debater; discutir}
  \end{Phonetics}
\end{Entry}

\begin{Entry}{争先}{6,6}{⼑、⼉}
  \begin{Phonetics}{争先}{zheng1xian1}
    \definition{v.}{competir para ser o primeiro |contestar o primeiro lugar}
  \end{Phonetics}
\end{Entry}

\begin{Entry}{争夺}{6,6}{⼑、⼤}
  \begin{Phonetics}{争夺}{zheng1duo2}[][HSK 6]
    \definition{v.}{disputar; competir}
  \end{Phonetics}
\end{Entry}

\begin{Entry}{争论}{6,6}{⼑、⾔}
  \begin{Phonetics}{争论}{zheng1lun4}[][HSK 4]
    \definition[番,场,次]{s.}{debate; discussão; argumentação; disputa}
    \definition{v.}{discutir; disputar; debater; argumentar; contestar}
  \end{Phonetics}
\end{Entry}

\begin{Entry}{争取}{6,8}{⼑、⼜}
  \begin{Phonetics}{争取}{zheng1qu3}[][HSK 3]
    \definition{v.}{lutar por; conquistar; vencer; se esforçar para conseguir}
  \end{Phonetics}
\end{Entry}

\begin{Entry}{争霸}{6,21}{⼑、⾬}
  \begin{Phonetics}{争霸}{zheng1ba4}
    \definition{s.}{hegemonia | uma luta pelo poder}
    \definition{v.}{disputar a hegemonia}
  \end{Phonetics}
\end{Entry}

\begin{Entry}{划}{6}{⼑}
  \begin{Phonetics}{划}{hua2}[][HSK 4]
    \definition{adj.}{rentável; vale (o esforço); compensa (fazer alguma coisa)}
    \definition{v.}{remar | ser vantajoso para alguém; ser uma pechincha | arranhar; cortar a superfície de; cortar em outra coisa com um objeto pontiagudo | arranhar; golpear;  esfregar uma coisa ou varrer sobre outra}
  \end{Phonetics}
  \begin{Phonetics}{划}{hua4}[][HSK 4]
    \definition*{s.}{Sobrenome Hua}
    \definition{s.}{traço de um caracter chinês}
    \definition{v.}{delimitar; diferenciar; delinear | transferir; ceder | planejar; programar | desenhar; marcar; delinear; fazer linhas ou escrever como marcadores com uma caneta ou objeto semelhante a uma caneta}
  \end{Phonetics}
\end{Entry}

\begin{Entry}{划分}{6,4}{⼑、⼑}
  \begin{Phonetics}{划分}{hua4fen1}[][HSK 5]
    \definition{v.}{dividir; particionar; reparticionar | diferenciar; encontrar aspectos diferentes}
  \end{Phonetics}
\end{Entry}

\begin{Entry}{划船}{6,11}{⼑、⾈}
  \begin{Phonetics}{划船}{hua2 chuan2}[][HSK 3]
    \definition[次,回]{s.}{remo (ato de remar); passeios de barco; a atividade ou esporte de “remar um barco com remos”}
    \definition{v.}{remar um barco; a ação ou comportamento de mover um barco na água usando remos}
  \end{Phonetics}
\end{Entry}

\begin{Entry}{划艇}{6,12}{⼑、⾈}
  \begin{Phonetics}{划艇}{hua2ting3}
    \definition{s.}{barco a remo}
  \end{Phonetics}
\end{Entry}

\begin{Entry}{列}{6}{⼑}
  \begin{Phonetics}{列}{lie4}[][HSK 4]
    \definition*{s.}{Sobrenome Lie}
    \definition{clas.}{usado para coisas em linhas e colunas}
    \definition{pron.}{cada um e todos; cada; muito}
    \definition{s.}{linha; arquivo; classificação (oposto a 行) | classificação; escopo | ranque | tipo}
    \definition{v.}{organizar; alinhar; colocar em ordem | listar; inserir em uma lista; classificar | formar uma linha}
  \seealsoref{行}{hang2}
  \end{Phonetics}
\end{Entry}

\begin{Entry}{列入}{6,2}{⼑、⼊}
  \begin{Phonetics}{列入}{lie4 ru4}[][HSK 4]
    \definition{v.}{listar; entrar em uma lista; ser incluído em | incluir em uma lista; juntar-se; registrar-se}
  \end{Phonetics}
\end{Entry}

\begin{Entry}{列为}{6,4}{⼑、⼂}
  \begin{Phonetics}{列为}{lie4 wei2}[][HSK 4]
    \definition{v.}{ser classificado como; ser listado como}
  \end{Phonetics}
\end{Entry}

\begin{Entry}{列车}{6,4}{⼑、⾞}
  \begin{Phonetics}{列车}{lie4che1}[][HSK 4]
    \definition[列,班,趟,辆,节]{s.}{trem; trem em uma composição contínua, puxado por uma locomotiva e equipado com uma tripulação e marcações prescritas; geralmente um trem de passageiros}
  \end{Phonetics}
\end{Entry}

\begin{Entry}{刘}{6}{⼑}
  \begin{Phonetics}{刘}{liu2}
    \definition*{s.}{Sobrenome Liu}
    \definition{s.}{Clássico: um tipo de machado de batalha}
    \definition{v.}{matar; massacrar}
  \end{Phonetics}
\end{Entry}

\begin{Entry}{刚}{6}{⼑}
  \begin{Phonetics}{刚}{gang1}[][HSK 2]
    \definition*{s.}{Sobrenome Gang}
    \definition{adj.}{duro; firme; rígido; forte; (personalidade, atitude) forte; (vontade) determinada}
    \definition{adv.}{apenas; exatamente; justamente | apenas; apenas por pouco; significa atingir um certo nível com dificuldade | apenas; há pouco tempo; indica que a ação ou situação ocorreu há pouco tempo | assim que; somente neste momento; aconteceu que; use a palavra 就 para indicar que duas coisas estão intimamente relacionadas}
  \seealsoref{就}{jiu4}
  \end{Phonetics}
\end{Entry}

\begin{Entry}{刚才}{6,3}{⼑、⼿}
  \begin{Phonetics}{刚才}{gang1cai2}[][HSK 2]
    \definition{s.}{agora mesmo; há pouco; há pouco tempo; referindo-se ao período recente que acabou de passar}
  \end{Phonetics}
\end{Entry}

\begin{Entry}{刚刚}{6,6}{⼑、⼑}
  \begin{Phonetics}{刚刚}{gang1 gang5}[][HSK 2]
    \definition{adv.}{apenas; somente; exatamente; refere-se a algo que é adequado em termos de grau, quantidade, tempo, etc., nem mais nem menos, nem cedo nem tarde, atingindo um estado satisfatório ou que atende exatamente às necessidades | agora mesmo; há pouco; há um momento atrás; referindo-se a um período de tempo muito curto no passado}
  \end{Phonetics}
\end{Entry}

\begin{Entry}{刚好}{6,6}{⼑、⼥}
  \begin{Phonetics}{刚好}{gang1 hao3}[][HSK 6]
    \definition{adj.}{apropriado; na medida certa}
    \definition{adv.}{apenas; acontece que; por acaso}
  \end{Phonetics}
\end{Entry}

\begin{Entry}{刚毅}{6,15}{⼑、⽎}
  \begin{Phonetics}{刚毅}{gang1yi4}[][HSK 7-9]
    \definition{adj.}{resoluto e firme | resoluto | robusto | firme}
  \end{Phonetics}
\end{Entry}

\begin{Entry}{创}{6}{⼑}
  \begin{Phonetics}{创}{chuang1}
    \definition{s.}{ferimento; trauma}
  \end{Phonetics}
  \begin{Phonetics}{创}{chuang4}[][HSK 7-9]
    \definition{v.}{começar (fazer algo); alcançar (algo pela primeira vez); estabelecer; fazer pela primeira vez | estabelecer; fundar; criar; perceber algo novo, como um começo | ferir; machucar}
  \end{Phonetics}
\end{Entry}

\begin{Entry}{创办}{6,4}{⼑、⼒}
  \begin{Phonetics}{创办}{chuang4 ban4}[][HSK 6]
    \definition{v.}{estabelecer; montar; fundar}
  \end{Phonetics}
\end{Entry}

\begin{Entry}{创业}{6,5}{⼑、⼀}
  \begin{Phonetics}{创业}{chuang4ye4}[][HSK 3]
    \definition{s.}{empreendedorismo}
    \definition{v.}{começar um empreendimento; iniciar/fundar um negócio, uma empresa;}
  \end{Phonetics}
\end{Entry}

\begin{Entry}{创立}{6,5}{⼑、⽴}
  \begin{Phonetics}{创立}{chuang4li4}[][HSK 5]
    \definition{v.}{fundar; originar; estabelecer}
  \end{Phonetics}
\end{Entry}

\begin{Entry}{创伤}{6,6}{⼑、⼈}
  \begin{Phonetics}{创伤}{chuang1shang1}[][HSK 7-9]
    \definition{s.}{ferida; parte do corpo lesionada, geralmente se refere a trauma | trauma; metáfora para dano emocional ou dano material}
  \end{Phonetics}
\end{Entry}

\begin{Entry}{创作}{6,7}{⼑、⼈}
  \begin{Phonetics}{创作}{chuang4zuo4}[][HSK 3]
    \definition[个]{s.}{criação; trabalho criativo; obras literárias e artísticas}
    \definition{v.}{escrever; criar; produzir; compor; criar obras artísticas}
  \end{Phonetics}
\end{Entry}

\begin{Entry}{创始人}{6,8,2}{⼑、⼥、⼈}
  \begin{Phonetics}{创始人}{chuang4shi3ren2}[][HSK 7-9]
    \definition{s.}{fundador; originador; iniciador | criador}
  \end{Phonetics}
\end{Entry}

\begin{Entry}{创建}{6,8}{⼑、⼵}
  \begin{Phonetics}{创建}{chuang4 jian4}[][HSK 6]
    \definition{v.}{fundar; estabelecer; montar}
  \end{Phonetics}
\end{Entry}

\begin{Entry}{创造}{6,10}{⼑、⾡}
  \begin{Phonetics}{创造}{chuang4zao4}[][HSK 3]
    \definition{s.}{criação; inovação; primeiro a concluir ou a alcançar resultados}
    \definition{v.}{criar; produzir; trazer à tona; fazer ou estabelecer pela primeira vez; referir-se de maneira geral a fazer ou estabelecer}
  \end{Phonetics}
\end{Entry}

\begin{Entry}{创意}{6,13}{⼑、⼼}
  \begin{Phonetics}{创意}{chuang4 yi4}[][HSK 6]
    \definition[个]{s.}{criatividade; originalidade; novidade; uma ideia original, conceito, etc.}
    \definition{v.}{inovar; criar um novo conceito, ideia, etc. | propor designs criativos, ideias, etc.}
  \end{Phonetics}
\end{Entry}

\begin{Entry}{创新}{6,13}{⼑、⽄}
  \begin{Phonetics}{创新}{chuang4xin1}[][HSK 3]
    \definition[个,种,次]{s.}{inovação; algo novo ou diferente, uma ideia}
    \definition{v.}{trazer novas ideias; inovar; abrir novos caminhos; criar ou fazer algo novo, diferente do que era antes}
  \end{Phonetics}
\end{Entry}

\begin{Entry}{判}{7}{⼑}
  \begin{Phonetics}{判}{pan4}[][HSK 6]
    \definition{adv.}{obviamente há uma diferença}
    \definition{v.}{distinguir; discriminar; separar | julgar; decidir; avaliar | sentenciar; condenar}
  \end{Phonetics}
\end{Entry}

\begin{Entry}{判断}{7,11}{⼑、⽄}
  \begin{Phonetics}{判断}{pan4duan4}[][HSK 3]
    \definition[个,项]{s.}{julgamento; conclusões tiradas após reflexão e análise}
    \definition{v.}{julgar; decidir}
  \end{Phonetics}
\end{Entry}

\begin{Entry}{利}{7}{⼑}
  \begin{Phonetics}{利}{li4}[][HSK 6]
    \definition*{s.}{Sobrenome Li}
    \definition{adj.}{afiado; cortante | favorável; conveniente; sem dificuldades; sem ou com poucas dificuldades}
    \definition{s.}{benefício; vantagem | lucro; ganhos; juros}
    \definition{v.}{beneficiar; tornar vantajoso}
  \end{Phonetics}
\end{Entry}

\begin{Entry}{利用}{7,5}{⼑、⽤}
  \begin{Phonetics}{利用}{li4yong4}[][HSK 3]
    \definition{v.}{usar; utilizar; fazer uso de; fazer com que algo ou alguém funcione bem| explorar; tirar vantagem de; usar meios para fazer com que pessoas ou coisas sirvam aos seus interesses}
  \end{Phonetics}
\end{Entry}

\begin{Entry}{利息}{7,10}{⼑、⼼}
  \begin{Phonetics}{利息}{li4xi1}[][HSK 4]
    \definition{s.}{acréscimo; juros; dinheiro recebido além do valor principal como resultado de depósitos ou empréstimos (diferenciado de 本金)}
  \seealsoref{本金}{ben3 jin1}
  \end{Phonetics}
\end{Entry}

\begin{Entry}{利润}{7,10}{⼑、⽔}
  \begin{Phonetics}{利润}{li4run4}[][HSK 5]
    \definition[笔,份]{s.}{lucro; o dinheiro ganho com atividades comerciais e industriais}
  \end{Phonetics}
\end{Entry}

\begin{Entry}{利益}{7,10}{⼑、⽫}
  \begin{Phonetics}{利益}{li4yi4}[][HSK 4]
    \definition[个,种]{s.}{ganho; lucro; juros; benefício}
  \end{Phonetics}
\end{Entry}

\begin{Entry}{别}{7}{⼑}
  \begin{Phonetics}{别}{bie2}[][HSK 1,4]
    \definition*{s.}{Sobrenome Bie}
    \definition{adv.}{não; nada de (pedir a alguém para não fazer); é melhor não | talvez, usado em conjunto com a palavra 是 para indicar especulação}
    \definition{pron.}{outro; algum outro}
    \definition{s.}{distinção; diferença | classificação}
    \definition{v.}{deixar; partir; separar | diferenciar; distinguir; encontrar aspectos diferentes | fixar objetos com pinos | girar; virar | aderir; colar; preder}
  \seealsoref{是}{shi4}
  \end{Phonetics}
  \begin{Phonetics}{别}{bie4}
    \definition{v.}{fazer com que alguém mude seus hábitos, opiniões, etc. | mudar a opinião de alguém (usado principalmente em 别不过)}
  \seealsoref{别不过}{bie2 bu2guo4}
  \end{Phonetics}
\end{Entry}

\begin{Entry}{别人}{7,2}{⼑、⼈}
  \begin{Phonetics}{别人}{bie2 ren2}[][HSK 1]
    \definition{pron.}{outros; outras pessoas}
    \definition{s.}{outros; pessoas; outras pessoas; refere-se a alguém diferente de si mesmo}
  \end{Phonetics}
\end{Entry}

\begin{Entry}{别不过}{7,4,6}{⼑、⼀、⾡}
  \begin{Phonetics}{别不过}{bie2 bu2guo4}
    \definition{expr.}{Não, mas}
  \end{Phonetics}
\end{Entry}

\begin{Entry}{别扭}{7,7}{⼑、⼿}
  \begin{Phonetics}{别扭}{bie4niu5}[][HSK 7-9]
    \definition{adj.}{estranho; desconfortável | desajeitado (ao falar ou escrever); não suave; não fluente | nervoso e desconfortável; muito nervoso; muito antinatural}
    \definition{v.}{ser difícil com alguém; relacionamentos ruins, opiniões diferentes, conflitos frequentes; falta de coordenação}
  \end{Phonetics}
\end{Entry}

\begin{Entry}{别具匠心}{7,8,6,4}{⼑、⼋、⼕、⼼}
  \begin{Phonetics}{别具匠心}{bie2ju4-jiang4xin1}[][HSK 7-9]
    \definition{expr.}{único e engenhoso | mostrar engenhosidade; ter originalidade}
  \end{Phonetics}
\end{Entry}

\begin{Entry}{别的}{7,8}{⼑、⽩}
  \begin{Phonetics}{别的}{bie2 de5}[][HSK 1]
    \definition{pron.}{outro; o resto}
  \end{Phonetics}
\end{Entry}

\begin{Entry}{别看}{7,9}{⼑、⽬}
  \begin{Phonetics}{别看}{bie2kan4}[][HSK 7-9]
    \definition{conj.}{apesar de}
  \end{Phonetics}
\end{Entry}

\begin{Entry}{别说}{7,9}{⼑、⾔}
  \begin{Phonetics}{别说}{bie2shuo1}[][HSK 7-9]
    \definition{v.}{Coloquial: não dizer nada de; não mencionar; deixar sozinho, usado no início de uma frase para reconhecer a declaração seguinte}[别说在下雨, 你现在出去也太晚了。===Não se preocupe com a chuva, já é tarde demais para você sair.]
  \end{Phonetics}
\end{Entry}

\begin{Entry}{别致}{7,10}{⼑、⾄}
  \begin{Phonetics}{别致}{bie2zhi4}[][HSK 7-9]
    \definition{adj.}{único; não convencional; novo, diferente do comum}
  \end{Phonetics}
\end{Entry}

\begin{Entry}{别提了}{7,12,2}{⼑、⼿、⼅}
  \begin{Phonetics}{别提了}{bie2ti2 le5}[][HSK 7-9]
    \definition{v.}{não mencionar mais isso, não falar mais sobre isso}
  \end{Phonetics}
\end{Entry}

\begin{Entry}{别墅}{7,14}{⼑、⼟}
  \begin{Phonetics}{别墅}{bie2shu4}[][HSK 7-9]
    \definition[栋,幢,座,套,个]{s.}{vila; mansão; residência de campo}
  \end{Phonetics}
\end{Entry}

\begin{Entry}{刮}{8}{⼑}
  \begin{Phonetics}{刮}{gua1}[][HSK 6]
    \definition{v.}{barbear; raspar; depilar | untar com (pasta, etc.)  | extorquir; pilhar; adquirir avidamente (propriedade) por vários meios | (do vento) soprar}
  \end{Phonetics}
\end{Entry}

\begin{Entry}{刮风}{8,4}{⼑、⾵}
  \begin{Phonetics}{刮风}{gua1/feng1}[][HSK 7-9]
    \definition{v.+compl.}{ventar; fazer vento; soprar (vento)}
  \end{Phonetics}
\end{Entry}

\begin{Entry}{到}{8}{⼑}
  \begin{Phonetics}{到}{dao4}[][HSK 1]
    \definition*{s.}{Sobrenome Dao}
    \definition{adj.}{atencioso}
    \definition{prep.}{a; até; para; indica o tempo em que a ação ou comportamento foi alcançado}
    \definition{v.}{ir para; partir para | chegar; alcançar; chegar a | como complemento de um verbo para mostrar o resultado de uma ação}
  \end{Phonetics}
\end{Entry}

\begin{Entry}{到处}{8,5}{⼑、⼡}
  \begin{Phonetics}{到处}{dao4chu4}[][HSK 2]
    \definition{adv.}{em todos os lugares; em todos os locais; por toda parte}
  \end{Phonetics}
\end{Entry}

\begin{Entry}{到头来}{8,5,7}{⼑、⼤、⽊}
  \begin{Phonetics}{到头来}{dao4tou2lai2}[][HSK 7-9]
    \definition{adv.}{finalmente; no fim; no final; resultado (usado principalmente em aspectos ruins)}
  \end{Phonetics}
\end{Entry}

\begin{Entry}{到达}{8,6}{⼑、⾡}
  \begin{Phonetics}{到达}{dao4da2}[][HSK 3]
    \definition{v.}{chegar (a um determinado local, a uma determinada fase); alcançar}
  \end{Phonetics}
\end{Entry}

\begin{Entry}{到位}{8,7}{⼑、⼈}
  \begin{Phonetics}{到位}{dao4/wei4}[][HSK 7-9]
    \definition{adj.}{bom; muito preciso; até um nível adequado/padrão/satisfatório}
    \definition{v.+compl.}{estar no lugar/posição; chegar ao local designado; alcançar o local especificado; atender aos requisitos especificados}
  \end{Phonetics}
\end{Entry}

\begin{Entry}{到来}{8,7}{⼑、⽊}
  \begin{Phonetics}{到来}{dao4 lai2}[][HSK 5]
    \definition{v.}{chegar; chegar aqui de outro lugar}
  \end{Phonetics}
\end{Entry}

\begin{Entry}{到底}{8,8}{⼑、⼴}
  \begin{Phonetics}{到底}{dao4di3}[][HSK 3]
    \definition{adv.}{na terra (usado em frases interrogativas para expressar a determinação de alguém em encontrar uma resposta definitiva) | afinal | finalmente; por fim; no fim; indica uma situação que finalmente se concretizou após várias mudanças ou reviravoltas}
  \end{Phonetics}
\end{Entry}

\begin{Entry}{到期}{8,12}{⼑、⽉}
  \begin{Phonetics}{到期}{dao4/qi1}[][HSK 6]
    \definition{v.+compl.}{expirar; amadurecer; tornar-se devido; tornar-se devido}
  \end{Phonetics}
\end{Entry}

\begin{Entry}{制}{8}{⼑}
  \begin{Phonetics}{制}{zhi4}
    \definition[套,项]{s.}{sistema | regras; regulamentos}
    \definition{v.}{formular; elaborar | fazer; fabricar | restringir; limitar; controlar; disciplinar}
  \end{Phonetics}
\end{Entry}

\begin{Entry}{制订}{8,4}{⼑、⾔}
  \begin{Phonetics}{制订}{zhi4 ding4}[][HSK 4]
    \definition{v.}{esboçar; formular; elaborar; mapear}
  \end{Phonetics}
\end{Entry}

\begin{Entry}{制成}{8,6}{⼑、⼽}
  \begin{Phonetics}{制成}{zhi4 cheng2}[][HSK 5]
    \definition{v.}{fabricar; ser feito de; produzir}
  \end{Phonetics}
\end{Entry}

\begin{Entry}{制约}{8,6}{⼑、⽷}
  \begin{Phonetics}{制约}{zhi4yue1}[][HSK 5]
    \definition{v.}{limitar; verificar; restringir; a existência e a mudança de uma coisa determinam a existência e a mudança de outra coisa}
  \end{Phonetics}
\end{Entry}

\begin{Entry}{制作}{8,7}{⼑、⼈}
  \begin{Phonetics}{制作}{zhi4zuo4}[][HSK 3]
    \definition{v.}{fazer; produzir; itens feitos com matérias-primas, geralmente pequenos e feitos à mão | fazer; produzir; criar gráficos, anúncios, filmes, jogos, etc., utilizando texto, imagens, sons, imagens, etc.}
  \end{Phonetics}
\end{Entry}

\begin{Entry}{制定}{8,8}{⼑、⼧}
  \begin{Phonetics}{制定}{zhi4ding4}[][HSK 3]
    \definition{v.}{rascunhar; formular; elaborar; estabelecer (leis, regulamentos, planos, etc.)}
  \end{Phonetics}
\end{Entry}

\begin{Entry}{制度}{8,9}{⼑、⼴}
  \begin{Phonetics}{制度}{zhi4du4}[][HSK 3]
    \definition[项,条,套,种]{s.}{regulamentação; regulamento; procedimentos operacionais ou diretrizes de conduta que todos devem seguir | sistema; o sistema político, econômico e cultural formado sob determinadas condições históricas}
  \end{Phonetics}
\end{Entry}

\begin{Entry}{制造}{8,10}{⼑、⾡}
  \begin{Phonetics}{制造}{zhi4zao4}[][HSK 3]
    \definition{v.}{fazer; produzir; manufaturar; transformar matérias-primas em produtos acabados | criar; agitar; criar artificialmente uma situação ou atmosfera desfavorável}
  \end{Phonetics}
\end{Entry}

\begin{Entry}{制裁}{8,12}{⼑、⾐}
  \begin{Phonetics}{制裁}{zhi4cai2}
    \definition{s.}{punição | sanção (inclusive econômica)}
    \definition{v.}{punir}
  \end{Phonetics}
\end{Entry}

\begin{Entry}{刷}{8}{⼑}
  \begin{Phonetics}{刷}{shua1}[][HSK 4]
    \definition{s.}{escova; pincel | (onomatopéia) farfalhar; descreve o som de uma passagem rápida}
    \definition{v.}{escovar; esfregar; remover com uma escova | borrar; colar; aplicar com um pincel | eliminar; remover; limpar | rolar; navegar; visualizar grandes quantidades de informações muito rapidamente em um curto período de tempo online ou em dispositivos móveis | deslizar (passar o cartão magnético)}
  \end{Phonetics}
  \begin{Phonetics}{刷}{shua4}
    \definition{adj.}{pálido ou branco-azulado}
    \definition{adv.}{bastante; completamente; extremamente; descreve movimentos ágeis}
  \end{Phonetics}
\end{Entry}

\begin{Entry}{刷子}{8,3}{⼑、⼦}
  \begin{Phonetics}{刷子}{shua1zi5}[][HSK 4]
    \definition[把,个]{s.}{escova; escovão; utensílio feito de lã, fio de plástico, fio de metal, etc., para remover sujeira ou aplicar óleo de unção, etc., geralmente longo ou oval, alguns com alças}
  \end{Phonetics}
\end{Entry}

\begin{Entry}{刷牙}{8,4}{⼑、⽛}
  \begin{Phonetics}{刷牙}{shua1ya2}[][HSK 4]
    \definition{s.}{escovar os dentes}
  \end{Phonetics}
\end{Entry}

\begin{Entry}{券}{8}{⼑}
  \begin{Phonetics}{券}{quan4}[][HSK 6]
    \definition[张]{s.}{certificado; bilhete; ingresso; uma conta ou pedaço de papel que serve como recibo}
  \end{Phonetics}
\end{Entry}

\begin{Entry}{刹}{8}{⼑}
  \begin{Phonetics}{刹}{cha4}
    \definition*{s.}{abreviação de Kshatara, 刹多罗, sânscrito ``ksetra''}
    \definition{s.}{mosteiro, templo ou santuário budista}
  \seealsoref{刹多罗}{sha1duo1luo2}
  \end{Phonetics}
  \begin{Phonetics}{刹}{sha1}
    \definition{v.}{acionar o(s) freio(s); frear; brecar}
  \end{Phonetics}
\end{Entry}

\begin{Entry}{刹多罗}{8,6,8}{⼑、⼣、⽹}
  \begin{Phonetics}{刹多罗}{sha1duo1luo2}
    \definition*{s.}{Kshatara, sânscrito ``ksetra''}
  \end{Phonetics}
\end{Entry}

\begin{Entry}{刺}{8}{⼑}
  \begin{Phonetics}{刺}{ci1}
    \definition{s.}{(onomatopéia) som de rasgo, fricção, etc.}
  \end{Phonetics}
  \begin{Phonetics}{刺}{ci4}[][HSK 4]
    \definition*{s.}{Sobrenome Ci}
    \definition{s.}{espinho; farpa; algo afiado como uma agulha | cartão de visita | saliências; projeções pequenas e pontiagudas na superfície de um objeto ou na pele de uma pessoa}
    \definition{v.}{esfaquear; perfurar | irritar; estimular | assassinar | espionar; detectar | criticar}
  \end{Phonetics}
\end{Entry}

\begin{Entry}{刺耳}{8,6}{⼑、⽿}
  \begin{Phonetics}{刺耳}{ci4'er3}[][HSK 7-9]
    \definition{adj.}{irritante (desagradável) para o ouvido; estridente; penetrante; áspero}
  \end{Phonetics}
\end{Entry}

\begin{Entry}{刺骨}{8,9}{⼑、⾻}
  \begin{Phonetics}{刺骨}{ci4gu3}[][HSK 7-9]
    \definition{adj.}{perfurante (até os ossos); cortante}
  \end{Phonetics}
\end{Entry}

\begin{Entry}{刺绣}{8,10}{⼑、⽷}
  \begin{Phonetics}{刺绣}{ci4xiu4}[][HSK 7-9]
    \definition{s.}{bordado; um artesanato popular tradicional que usa fios de seda coloridos para bordar padrões ou imagens em tecidos; produtos bordados}
    \definition{v.}{bordar; bordar padrões ou imagens em tecido usando fios de seda coloridos}
  \end{Phonetics}
\end{Entry}

\begin{Entry}{刺猬}{8,12}{⼑、⽝}
  \begin{Phonetics}{刺猬}{ci4wei5}
    \definition{s.}{porco-espinho | ouriço}
  \end{Phonetics}
\end{Entry}

\begin{Entry}{刺激}{8,16}{⼑、⽔}
  \begin{Phonetics}{刺激}{ci4ji1}[][HSK 4]
    \definition{adj.}{animado; entusiasmado; sensação de empolgação e nervosismo}
    \definition[个]{s.}{estímulo; estimulação; fortes efeitos físicos ou psicológicos}
    \definition{v.}{irritar; provocar; estimular | incentivar; estimular; incitar; (por algum meio) para mudar as coisas para melhor, para fazer coisas positivas}
  \end{Phonetics}
\end{Entry}

\begin{Entry}{刻}{8}{⼑}
  \begin{Phonetics}{刻}{ke4}[][HSK 2,5]
    \definition{adj.}{cruel; severo; rude; indelicado | no mais alto grau}
    \definition{clas.}{um quarto (de uma hora, 15min)}
    \definition[件]{s.}{quarto (de hora); momento}
    \definition{v.}{esculpir; inscrever; gravar; talhar com uma faca (padrões, texto, etc.) | definir um limite de tempo | imprimir (CD)}
  \end{Phonetics}
\end{Entry}

\begin{Entry}{刻画}{8,8}{⼑、⽥}
  \begin{Phonetics}{刻画}{ke4hua4}
    \definition{v.}{retratar | tirar um retrato}
  \end{Phonetics}
\end{Entry}

\begin{Entry}{刻钟}{8,9}{⼑、⾦}
  \begin{Phonetics}{刻钟}{ke4 zhong1}
    \definition{s.}{um quarto de hora}
  \end{Phonetics}
\end{Entry}

\begin{Entry}{前}{9}{⼑}
  \begin{Phonetics}{前}{qian2}[][HSK 1]
    \definition*{s.}{Sobrenome Qian}
    \definition{s.}{frente | futuro; perspectiva | atrás; antes; mais cedo do que uma coisa ou um momento | à frente; para a frente; na parte frontal (referindo-se ao espaço, em oposição a 后) | precedente; antes que algo aconteça | antigo; antigamente | topo; primeiro; primeiro na ordem | frente; campo de batalha | A.C. (Antes de~Cristo)}[前293年===293 a.C.]
    \definition{v.}{seguir em frente; ir em frente}
  \seealsoref{公元}{gong1yuan2}
  \seealsoref{后}{hou4}
  \end{Phonetics}
\end{Entry}

\begin{Entry}{前天}{9,4}{⼑、⼤}
  \begin{Phonetics}{前天}{qian2 tian1}[][HSK 1]
    \definition{adv.}{anteontem; dia anterior a ontem}
  \end{Phonetics}
\end{Entry}

\begin{Entry}{前方}{9,4}{⼑、⽅}
  \begin{Phonetics}{前方}{qian2 fang1}[][HSK 6]
    \definition{s.}{frente; o espaço à frente; a direção voltada para a frente; a frente (em oposição à 后方) | linha de frente; frente de batalha; áreas onde os exércitos de ambos os lados estão se aproximando ou lutando}
  \seealsoref{后方}{hou4 fang1}
  \end{Phonetics}
\end{Entry}

\begin{Entry}{前头}{9,5}{⼑、⼤}
  \begin{Phonetics}{前头}{qian2 tou5}[][HSK 4]
    \definition{s.}{à frente; na frente; adiante}
  \end{Phonetics}
\end{Entry}

\begin{Entry}{前边}{9,5}{⼑、⾡}
  \begin{Phonetics}{前边}{qian2 bian5}[][HSK 1]
    \definition{adv.}{à frente; na frente}
  \end{Phonetics}
\end{Entry}

\begin{Entry}{前后}{9,6}{⼑、⼝}
  \begin{Phonetics}{前后}{qian2 hou4}[][HSK 3]
    \definition{s.}{em volta; sobre; um período de tempo ligeiramente anterior ou posterior a um horário específico| do início ao fim; refere-se ao período de tempo do início ao fim de algo | frente e verso; na frente e atrás de algo}
  \end{Phonetics}
\end{Entry}

\begin{Entry}{前年}{9,6}{⼑、⼲}
  \begin{Phonetics}{前年}{qian2 nian2}[][HSK 2]
    \definition{adv.}{há dois anos; dois anos atrás}
  \end{Phonetics}
\end{Entry}

\begin{Entry}{前来}{9,7}{⼑、⽊}
  \begin{Phonetics}{前来}{qian2 lai2}[][HSK 6]
    \definition{v.}{vir; em direção à localização e direção do falante}
  \end{Phonetics}
\end{Entry}

\begin{Entry}{前进}{9,7}{⼑、⾡}
  \begin{Phonetics}{前进}{qian2 jin4}[][HSK 3]
    \definition{v.}{marchar; avançar; para ir em frente; seguir em frente; geralmente se refere ao desenvolvimento futuro}
  \end{Phonetics}
\end{Entry}

\begin{Entry}{前往}{9,8}{⼑、⼻}
  \begin{Phonetics}{前往}{qian2 wang3}[][HSK 3]
    \definition{v.}{ir para; prosseguir para; partir para; ir em frente}
  \end{Phonetics}
\end{Entry}

\begin{Entry}{前线}{9,8}{⼑、⽷}
  \begin{Phonetics}{前线}{qian2 xian4}
    \definition{s.}{linha de frente; frente (oposto à 后方) | frente de batalha; a área onde os dois exércitos se aproximam durante uma batalha (em oposição à 后方)}
  \seealsoref{后方}{hou4 fang1}
  \end{Phonetics}
\end{Entry}

\begin{Entry}{前面}{9,9}{⼑、⾯}
  \begin{Phonetics}{前面}{qian2mian4}[][HSK 3]
    \definition{s.}{frente; a parte frontal do espaço ou posição | parte anterior; acima; a parte que vem primeiro na ordem; a parte de um artigo ou discurso que precede a narração atual}
  \end{Phonetics}
\end{Entry}

\begin{Entry}{前途}{9,10}{⼑、⾡}
  \begin{Phonetics}{前途}{qian2tu2}[][HSK 4]
    \definition[片,段,种]{s.}{futuro; perspectiva; prospecto; originalmente, refere-se à jornada à frente, mas, metaforicamente, refere-se ao futuro.}
  \end{Phonetics}
\end{Entry}

\begin{Entry}{前提}{9,12}{⼑、⼿}
  \begin{Phonetics}{前提}{qian2ti2}[][HSK 5]
    \definition[个,项]{s.}{premissa; pressuposto | pré-requisito; pressuposição; condições prévias para que algo aconteça ou se desenvolva}
  \end{Phonetics}
\end{Entry}

\begin{Entry}{前景}{9,12}{⼑、⽇}
  \begin{Phonetics}{前景}{qian2jing3}[][HSK 5]
    \definition{s.}{primeiro plano (de uma vista, imagem, foto, etc.); as imagens que parecem mais próximas do espectador em pinturas, palcos e telas | vista; perspectiva; prospecto; ponto de vista; situações que podem ocorrer no trabalho, na carreira, etc.}
  \end{Phonetics}
\end{Entry}

\begin{Entry}{剑}{9}{⼑}
  \begin{Phonetics}{剑}{jian4}[][HSK 6]
    \definition[把,口]{s.}{espada; sabre; florete}
  \end{Phonetics}
\end{Entry}

\begin{Entry}{剑客}{9,9}{⼑、⼧}
  \begin{Phonetics}{剑客}{jian4ke4}
    \definition{s.}{espada | esgrimista, espadachim}
  \end{Phonetics}
\end{Entry}

\begin{Entry}{剥}{10}{⼑}
  \begin{Phonetics}{剥}{bao1}[][HSK 7-9]
    \definition{v.}{descascar; despelar; remover a casca ou pele externa}
  \end{Phonetics}
  \begin{Phonetics}{剥}{bo1}
    \definition{v.}{Dialeto: descascar; despelar; remover a casca ou pele externa | (pele, tinta, etc.) sair; descascar | privar; explorar}
  \end{Phonetics}
\end{Entry}

\begin{Entry}{剥夺}{10,6}{⼑、⼤}
  \begin{Phonetics}{剥夺}{bo1duo2}[][HSK 7-9]
    \definition{v.}{roubar algo de alguém; tirar algo de alguém à força | privar; privar por lei; cancelar de acordo com a lei}
  \end{Phonetics}
\end{Entry}

\begin{Entry}{剥削}{10,9}{⼑、⼑}
  \begin{Phonetics}{剥削}{bo1xue1}[][HSK 7-9]
    \definition{v.}{explorar; apropriar-se do trabalho ou dos frutos do trabalho de outrem sem remuneração}
  \end{Phonetics}
\end{Entry}

\begin{Entry}{剧}{10}{⼑}
  \begin{Phonetics}{剧}{ju4}[][HSK 6]
    \definition*{s.}{Sobrenome Ju}
    \definition{adj.}{agudo; grave; intenso; violento}
    \definition[部,个,种]{s.}{obra teatral; drama; peça; ópera}
  \end{Phonetics}
\end{Entry}

\begin{Entry}{剧本}{10,5}{⼑、⽊}
  \begin{Phonetics}{剧本}{ju4ben3}[][HSK 5]
    \definition[部,个]{s.}{cenário; roteiro (para drama, filme, etc.); gênero de obra literária que consiste em diálogos entre personagens (às vezes cantados) e indicações de palco}
  \end{Phonetics}
\end{Entry}

\begin{Entry}{剧场}{10,6}{⼑、⼟}
  \begin{Phonetics}{剧场}{ju4 chang3}[][HSK 3]
    \definition[个,坐]{s.}{teatro; local para apresentações teatrais, musicais, etc.}
  \end{Phonetics}
\end{Entry}

\begin{Entry}{剪}{11}{⼑}
  \begin{Phonetics}{剪}{jian3}[][HSK 5]
    \definition[把]{s.}{tesouras; tesouras de poda; cortadores | pinças; tenazes}
    \definition{v.}{cortar; aparar; tosquiar; cortar (com uma tesoura) | exterminar; eliminar; acabar com}
  \end{Phonetics}
\end{Entry}

\begin{Entry}{剪刀}{11,2}{⼑、⼑}
  \begin{Phonetics}{剪刀}{jian3dao1}[][HSK 5]
    \definition[把,个]{s.}{tesoura; tesoura de jardim; instrumento de ferro para cortar tecido, papel, barbante, etc., com duas lâminas interligadas que podem ser abertas e fechadas}
  \end{Phonetics}
\end{Entry}

\begin{Entry}{剪子}{11,3}{⼑、⼦}
  \begin{Phonetics}{剪子}{jian3 zi5}[][HSK 5]
    \definition[把]{s.}{tesouras; tesouras de podar; tosquiadeiras}
  \end{Phonetics}
\end{Entry}

\begin{Entry}{副}{11}{⼑}
  \begin{Phonetics}{副}{fu4}[][HSK 6]
    \definition{adj.}{segundo em exercício; deputado; auxiliar | subsidiário; incidental; secundário}
    \definition{clas.}{usado para conjuntos completos de itens; usado para \emph{kits} | usado para expressões faciais | usado para som ou voz}
    \definition{pref.}{vice-}
    \definition{s.}{assistente; ajudante; auxiliar; posição auxiliar; pessoa que ocupa uma posição auxiliar}
    \definition{v.}{ajustar; corresponder a; conformar-se a}
  \end{Phonetics}
\end{Entry}

\begin{Entry}{副作用}{11,7,5}{⼑、⼈、⽤}
  \begin{Phonetics}{副作用}{fu4zuo4yong4}[][HSK 7-9]
    \definition{s.}{efeito colateral; efeitos adversos além dos efeitos principais}
  \end{Phonetics}
\end{Entry}

\begin{Entry}{副研}{11,9}{⼑、⽯}
  \begin{Phonetics}{副研}{fu4yan2}
    \definition{s.}{pesquisador adjunto}
  \end{Phonetics}
\end{Entry}

\begin{Entry}{剩}{12}{⼑}
  \begin{Phonetics}{剩}{sheng4}[][HSK 5]
    \definition*{s.}{Sobrenome Sheng}
    \definition{v.}{permanecer; ser deixado (para trás)}
  \end{Phonetics}
\end{Entry}

\begin{Entry}{剩下}{12,3}{⼑、⼀}
  \begin{Phonetics}{剩下}{sheng4 xia4}[][HSK 5]
    \definition{v.}{permanecer; ser deixado (para trás); consumir e utilizar, restando apenas os resíduos}
  \end{Phonetics}
\end{Entry}

\begin{Entry}{割}{12}{⼑}
  \begin{Phonetics}{割}{ge1}[][HSK 7-9]
    \definition{v.}{cortar; ceifar | dividir; cortar}
  \end{Phonetics}
\end{Entry}

%%%%% EOF %%%%%

