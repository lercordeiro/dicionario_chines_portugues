%%%
%%% Radical "⾃"
%%%

\section*{Radical 132: ``⾃''}\addcontentsline{toc}{section}{Radical 132: ⾃}

\begin{Entry}{自}{6}{⾃}[Kangxi 132]
  \begin{Phonetics}{自}{zi4}[][HSK 4]
    \definition*{s.}{Sobrenome Zi}
    \definition{adv.}{certamente; com certeza; é claro; naturalmente}
    \definition{prep.}{de; desde; a partir de; apresenta o ponto de partida, a fonte ou o horário de início do comportamento da ação, equivalente a 从 e 由}
    \definition{pron.}{si mesmo; próprio | próprio; indica que a ação é iniciada por e direcionada a si mesmo | por si mesmo; indica que a ação é autoiniciada e não é causada por uma força externa}
    \definition{v.}{iniciar}
  \seealsoref{从}{cong2}
  \seealsoref{由}{you2}
  \end{Phonetics}
\end{Entry}

\begin{Entry}{自个儿}{6,3,2}{⾃、⼈、⼉}
  \begin{Phonetics}{自个儿}{zi4ge3r5}
    \definition{pron.}{(dialeto) a si mesmo, por si mesmo}
  \end{Phonetics}
\end{Entry}

\begin{Entry}{自己}{6,3}{⾃、⼰}
  \begin{Phonetics}{自己}{zi4ji3}[][HSK 2]
    \definition{pron.}{a si próprio; a si mesmo; refere-se ao substantivo ou pronome precedente (enfatiza principalmente que não é devido a forças externas)}
  \end{Phonetics}
\end{Entry}

\begin{Entry}{自己动手}{6,3,6,4}{⾃、⼰、⼒、⼿}
  \begin{Phonetics}{自己动手}{zi4ji3dong4shou3}
    \definition{v.}{fazer (algo) sozinho | ajudar-se a}
  \end{Phonetics}
\end{Entry}

\begin{Entry}{自从}{6,4}{⾃、⼈}
  \begin{Phonetics}{自从}{zi4cong2}[][HSK 3]
    \definition{prep.}{de; desde; a partir de; referir-se a um momento ou evento específico no passado}
  \end{Phonetics}
\end{Entry}

\begin{Entry}{自主}{6,5}{⾃、⼂}
  \begin{Phonetics}{自主}{zi4zhu3}[][HSK 3]
    \definition{v.}{agir por conta própria; decidir por si mesmo; manter a iniciativa em suas próprias mãos; tomar suas próprias decisões}
  \end{Phonetics}
\end{Entry}

\begin{Entry}{自由}{6,5}{⾃、⽥}
  \begin{Phonetics}{自由}{zi4you2}[][HSK 2]
    \definition{adj.}{livre; irrestrito}
    \definition[个]{s.}{liberdade; o direito de agir de acordo com a própria vontade dentro do âmbito da lei | liberdade; filosoficamente, liberdade é definida como o processo de as pessoas reconhecerem as leis que governam o desenvolvimento das coisas e aplicá-las conscientemente na prática}
  \end{Phonetics}
\end{Entry}

\begin{Entry}{自由泳}{6,5,8}{⾃、⽥、⽔}
  \begin{Phonetics}{自由泳}{zi4you2yong3}
    \definition{s.}{natação de estilo livre}
  \end{Phonetics}
\end{Entry}

\begin{Entry}{自动}{6,6}{⾃、⼒}
  \begin{Phonetics}{自动}{zi4dong4}[][HSK 3]
    \definition{adj.}{automático; auto-atuante; uso de dispositivos mecânicos, elétricos, etc, para funcionar automaticamente, sem necessidade de controle humano}
    \definition{adv.}{voluntariamente; por vontade própria; por iniciativa própria | automaticamente; espontaneamente; refere-se a movimentos, mudanças, etc., que não são causados pela ação humana, mas sim pelo próprio objeto}
  \end{Phonetics}
\end{Entry}

\begin{Entry}{自动化}{6,6,4}{⾃、⼒、⼔}
  \begin{Phonetics}{自动化}{zi4dong4hua4}
    \definition{s.}{automação}
  \end{Phonetics}
\end{Entry}

\begin{Entry}{自在}{6,6}{⾃、⼟}
  \begin{Phonetics}{自在}{zi4 zai4}[][HSK 6]
    \definition{adj.}{livre; irrestrito}
  \end{Phonetics}
\end{Entry}

\begin{Entry}{自杀}{6,6}{⾃、⽊}
  \begin{Phonetics}{自杀}{zi4 sha1}[][HSK 5]
    \definition{s.}{suicídio; auto-assassinato; auto-sacrifício}
    \definition{v.}{cometer suicídio; tentar suicídio; matar-se}
  \end{Phonetics}
\end{Entry}

\begin{Entry}{自行车}{6,6,4}{⾃、⾏、⾞}
  \begin{Phonetics}{自行车}{zi4xing2che1}[][HSK 2]
    \definition[辆]{s.}{bicicleta; um veículo de duas rodas que é impulsionado para a frente com os pedais}
  \end{Phonetics}
\end{Entry}

\begin{Entry}{自行车架}{6,6,4,9}{⾃、⾏、⾞、⽊}
  \begin{Phonetics}{自行车架}{zi4xing2che1jia4}
    \definition{s.}{suporte para bicicleta | bicicletário}
  \end{Phonetics}
\end{Entry}

\begin{Entry}{自行车馆}{6,6,4,11}{⾃、⾏、⾞、⾷}
  \begin{Phonetics}{自行车馆}{zi4xing2che1guan3}
    \definition{s.}{estádio de ciclismo | velódromo}
  \end{Phonetics}
\end{Entry}

\begin{Entry}{自行车赛}{6,6,4,14}{⾃、⾏、⾞、⾙}
  \begin{Phonetics}{自行车赛}{zi4xing2che1sai4}
    \definition{s.}{corrida de bicicleta}
  \end{Phonetics}
\end{Entry}

\begin{Entry}{自我}{6,7}{⾃、⼽}
  \begin{Phonetics}{自我}{zi4 wo3}[][HSK 6]
    \definition{pref.}{auto-}
    \definition{pron.}{a si mesmo; eu próprio; geralmente usado antes de verbos dissílabos para indicar que a ação é realizada por alguém e dirigida a si mesmo | indicar o próprio caráter diferente dos outros; refere-se às próprias características, personalidade, hobbies, etc.}
  \end{Phonetics}
\end{Entry}

\begin{Entry}{自我介绍}{6,7,4,8}{⾃、⼽、⼈、⽷}
  \begin{Phonetics}{自我介绍}{zi4wo3jie4shao4}
    \definition{s.}{defesa pessoal | auto-defesa}
  \end{Phonetics}
\end{Entry}

\begin{Entry}{自我安慰}{6,7,6,15}{⾃、⼽、⼧、⼼}
  \begin{Phonetics}{自我安慰}{zi4wo3'an1wei4}
    \definition{v.}{confortar-se | consolar-se | tranquilizar-se}
  \end{Phonetics}
\end{Entry}

\begin{Entry}{自我防卫}{6,7,6,3}{⾃、⼽、⾩、⼙}
  \begin{Phonetics}{自我防卫}{zi4wo3fang2wei4}
    \definition{s.}{defesa pessoal | auto-defesa}
  \end{Phonetics}
\end{Entry}

\begin{Entry}{自我吹嘘}{6,7,7,14}{⾃、⼽、⼝、⼝}
  \begin{Phonetics}{自我吹嘘}{zi4wo3 chui1xu1}
    \definition{expr.}{gabar-se}
    \definition{s.}{auto-ostentação; autoglorificação;}
  \end{Phonetics}
\end{Entry}

\begin{Entry}{自我批评}{6,7,7,7}{⾃、⼽、⼿、⾔}
  \begin{Phonetics}{自我批评}{zi4wo3 pi1ping2}
    \definition{s.}{autocrítica}
  \end{Phonetics}
\end{Entry}

\begin{Entry}{自我实现}{6,7,8,8}{⾃、⼽、⼧、⾒}
  \begin{Phonetics}{自我实现}{zi4wo3shi2xian4}
    \definition{s.}{(psicologia) auto-realização}
  \end{Phonetics}
\end{Entry}

\begin{Entry}{自我的人}{6,7,8,2}{⾃、⼽、⽩、⼈}
  \begin{Phonetics}{自我的人}{zi4wo3de5ren2}
    \definition{s.}{(minha, sua) própria pessoa | (afirmar) a própria personalidade}
  \end{Phonetics}
\end{Entry}

\begin{Entry}{自我保存}{6,7,9,6}{⾃、⼽、⼈、⼦}
  \begin{Phonetics}{自我保存}{zi4wo3 bao3cun2}
    \definition{v.}{autopreservação}
  \end{Phonetics}
\end{Entry}

\begin{Entry}{自我陶醉}{6,7,10,15}{⾃、⼽、⾩、⾣}
  \begin{Phonetics}{自我陶醉}{zi4wo3tao2zui4}
    \definition{s.}{narcisista | auto-imbuído | satisfeito consigo mesmo}
  \end{Phonetics}
\end{Entry}

\begin{Entry}{自我催眠}{6,7,13,10}{⾃、⼽、⼈、⽬}
  \begin{Phonetics}{自我催眠}{zi4wo3cui1mian2}
    \definition{v.}{consolar-me | tranquilizar-me}
  \end{Phonetics}
\end{Entry}

\begin{Entry}{自我意识}{6,7,13,7}{⾃、⼽、⼼、⾔}
  \begin{Phonetics}{自我意识}{zi4wo3 yi4shi2}
    \definition{s.}{autoconsciência; auto-consciente}
  \end{Phonetics}
\end{Entry}

\begin{Entry}{自我解嘲}{6,7,13,15}{⾃、⼽、⾓、⼝}
  \begin{Phonetics}{自我解嘲}{zi4wo3 jie3chao2}
    \definition{s.}{autodepreciação; referir-se às próprias fraquezas ou falhas com humor autodepreciativo}
    \definition{v.}{encontrar desculpas para; consolar-se}
  \end{Phonetics}
\end{Entry}

\begin{Entry}{自来水}{6,7,4}{⾃、⽊、⽔}
  \begin{Phonetics}{自来水}{zi4 lai2 shui3}[][HSK 6]
    \definition{s.}{água da torneira; água corrente; água purificada e desinfetada fornecida por sistema hidráulico através de tubulações | equipamentos para transporte de água natural tratada}
  \end{Phonetics}
\end{Entry}

\begin{Entry}{自言自语}{6,7,6,9}{⾃、⾔、⾃、⾔}
  \begin{Phonetics}{自言自语}{zi4 yan2 zi4 yu3}[][HSK 6]
    \definition{expr.}{falando sozinho; falar consigo mesmo; pensar em voz alta; solilóquio}
  \end{Phonetics}
\end{Entry}

\begin{Entry}{自身}{6,7}{⾃、⾝}
  \begin{Phonetics}{自身}{zi4 shen1}[][HSK 3]
    \definition{pron.}{eu mesmo (enfatizando que não é outra pessoa ou outra coisa)}
  \end{Phonetics}
\end{Entry}

\begin{Entry}{自学}{6,8}{⾃、⼦}
  \begin{Phonetics}{自学}{zi4 xue2}[][HSK 6]
    \definition{s.}{auto-estudo; autodidata; autoaprendizagem}
    \definition{v.}{estudar por conta própria; estudar de forma independente; ensinar a si mesmo}
  \end{Phonetics}
\end{Entry}

\begin{Entry}{自责}{6,8}{⾃、⾙}
  \begin{Phonetics}{自责}{zi4ze2}
    \definition{v.}{culpar-se}
  \end{Phonetics}
\end{Entry}

\begin{Entry}{自信}{6,9}{⾃、⼈}
  \begin{Phonetics}{自信}{zi4xin4}[][HSK 4]
    \definition{adj.}{confiante; descreve a crença em suas próprias habilidades, decisões, etc., tendo confiança em si mesmo}
    \definition[份,种]{s.}{autoconfiança; confiança em si mesmo}
    \definition{v.}{acreditar em si mesmo}
  \end{Phonetics}
\end{Entry}

\begin{Entry}{自觉}{6,9}{⾃、⾒}
  \begin{Phonetics}{自觉}{zi4jue2}[][HSK 3]
    \definition{adj.}{autoconsciente; de ​​livre e espontânea vontade; controlar o próprio comportamento e agir por iniciativa própria}
    \definition{v.}{estar ciente de}
  \end{Phonetics}
\end{Entry}

\begin{Entry}{自救}{6,11}{⾃、⽁}
  \begin{Phonetics}{自救}{zi4jiu4}
    \definition{s.}{autoajuda}
    \definition{v.}{salvar-se; prover-se e ajudar-se}
  \end{Phonetics}
\end{Entry}

\begin{Entry}{自然}{6,12}{⾃、⽕}
  \begin{Phonetics}{自然}{zi4ran2}[][HSK 3]
    \definition{adj.}{natural; no curso normal dos eventos; formado ou desenvolvido sem intervenção humana; algo que se desenvolve livremente}
    \definition{adv.}{naturalmente; definitivamente; certamente, isso significa que, de acordo com a lógica, deve ser assim}
    \definition{conj.}{usado para ligar duas frases, com a segunda introduzindo informações adicionais ou adversativas; indica explicação complementar ou uma mudança de significado}
    \definition{s.}{natureza; mundo natural; tudo o que não foi criado pelo ser humano}
  \end{Phonetics}
\end{Entry}

\begin{Entry}{自愿}{6,14}{⾃、⽕}
  \begin{Phonetics}{自愿}{zi4yuan4}[][HSK 5]
    \definition{adv.}{voluntariamente; por iniciativa própria; por vontade própria}
    \definition{s.}{voluntário}
  \end{Phonetics}
\end{Entry}

\begin{Entry}{自豪}{6,14}{⾃、⾗}
  \begin{Phonetics}{自豪}{zi4hao2}[][HSK 5]
    \definition{adj.}{orgulhar-se de; ter orgulho de; sentir-se honrado por possuir qualidades excelentes ou ter alcançado grandes conquistas, seja por si mesmo ou por um grupo ou indivíduo relacionado a si}
  \end{Phonetics}
\end{Entry}

\begin{Entry}{自燃}{6,16}{⾃、⽕}
  \begin{Phonetics}{自燃}{zi4ran2}
    \definition{s.}{combustão espontânea}
  \end{Phonetics}
\end{Entry}

\begin{Entry}{臭}{10}{⾃}
  \begin{Phonetics}{臭}{chou4}[][HSK 5]
    \definition{adj.}{sujo; malcheiroso; fedorento; contrário de 香 | repugnante; nojento; repulsivo | ruim; pobre; péssimo}
    \definition{adv.}{severamente; firmemente}
    \definition{v.}{falhar em detonar (bala)}
  \seealsoref{香}{xiang1}
  \end{Phonetics}
  \begin{Phonetics}{臭}{xiu4}
    \definition{s.}{odor; cheiro}
    \definition{v.}{cheirar; farejar; o mesmo que 嗅}
  \seealsoref{嗅}{xiu4}
  \end{Phonetics}
\end{Entry}

\begin{Entry}{臭气}{10,4}{⾃、⽓}
  \begin{Phonetics}{臭气}{chou4qi4}
    \definition{s.}{fedor}
  \end{Phonetics}
\end{Entry}

%%%%% EOF %%%%%

