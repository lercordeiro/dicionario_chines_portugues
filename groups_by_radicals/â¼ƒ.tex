%%%
%%% Radical "⼃"
%%%

\section*{Radical 4: ``⼃''}\addcontentsline{toc}{section}{Radical 4: ⼃}

\begin{Entry}{久}{3}{⼃}
  \begin{Phonetics}{久}{jiu3}[][HSK 3]
    \definition{adj.}{por muito tempo; longo período de tempo | duração de tempo especificada}
  \end{Phonetics}
\end{Entry}

\begin{Entry}{及}{3}{⼃}
  \begin{Phonetics}{及}{ji2}
    \definition*{s.}{Sobrenome Ji}
    \definition{conj.}{e; bem como; conectando substantivos paralelos ou frases nominais}
    \definition{v.}{alcançar; chegar até | ser comparável a; alcançar (geralmente usado em termos negativos) | chegar a tempo para | estender-se a; cuidar de; envolver | dar}
  \end{Phonetics}
\end{Entry}

\begin{Entry}{及时}{3,7}{⼃、⽇}
  \begin{Phonetics}{及时}{ji2shi2}[][HSK 3]
    \definition{adj.}{oportuno; na hora certa; adequado; na ocasião certa}
    \definition{adv.}{prontamente; sem demora; imediatamente}
  \end{Phonetics}
\end{Entry}

\begin{Entry}{及格}{3,10}{⼃、⽊}
  \begin{Phonetics}{及格}{ji2/ge2}[][HSK 4]
    \definition{v.+compl.}{passar; passar em um teste, exame, etc.}
  \end{Phonetics}
\end{Entry}

\begin{Entry}{乌}{4}{⼃}
  \begin{Phonetics}{乌}{wu1}
    \definition*{s.}{Sobrenome Wu}
    \definition{adj.}{preto; escuro}
    \definition{pron.}{como; o que é}
    \definition{s.}{corvo; gralha}
  \end{Phonetics}
\end{Entry}

\begin{Entry}{乌云}{4,4}{⼃、⼆}
  \begin{Phonetics}{乌云}{wu1 yun2}[][HSK 6]
    \definition[片]{s.}{nuvens negras; nuvens escuras | cabelo preto (de mulher); uma metáfora para o cabelo preto brilhante de uma mulher}
  \end{Phonetics}
\end{Entry}

\begin{Entry}{乌克兰}{4,7,5}{⼃、⼗、⼋}
  \begin{Phonetics}{乌克兰}{wu1ke4lan2}
    \definition*{s.}{Ucrânia}
  \end{Phonetics}
\end{Entry}

\begin{Entry}{乌龟}{4,7}{⼃、⿔}
  \begin{Phonetics}{乌龟}{wu1gui1}
    \definition[头,只]{s.}{tartaruga}
  \end{Phonetics}
\end{Entry}

\begin{Entry}{乐}{5}{⼃}
  \begin{Phonetics}{乐}{le4}[][HSK 3]
    \definition*{s.}{Sobrenome Le}
    \definition{adj.}{feliz; contente; rejubilante; animado; bem disposto}
    \definition{s.}{prazer; diversão; felicidade}
    \definition{v.}{desfrutar; ficar feliz em; amar; encontrar prazer em | rir; divertir-se}
  \end{Phonetics}
  \begin{Phonetics}{乐}{yue4}
    \definition*{s.}{Sobrenome Yue}
    \definition{s.}{música}
  \end{Phonetics}
\end{Entry}

\begin{Entry}{乐队}{5,4}{⼃、⾩}
  \begin{Phonetics}{乐队}{yue4 dui4}[][HSK 3]
    \definition[支,个]{s.}{orquestra; banda; um grupo composto por muitas pessoas que tocam diferentes instrumentos musicais}
  \end{Phonetics}
\end{Entry}

\begin{Entry}{乐曲}{5,6}{⼃、⽈}
  \begin{Phonetics}{乐曲}{yue4 qu3}[][HSK 6]
    \definition[支,首,段]{s.}{música; composição musical}
  \end{Phonetics}
\end{Entry}

\begin{Entry}{乐观}{5,6}{⼃、⾒}
  \begin{Phonetics}{乐观}{le4guan1}[][HSK 3]
    \definition{adj.}{esperançoso; otimista; confiante; espírito alegre, confiante no futuro (oposto a 悲观)}
  \seealsoref{悲观}{bei1guan1}
  \end{Phonetics}
\end{Entry}

\begin{Entry}{乐园}{5,7}{⼃、⼞}
  \begin{Phonetics}{乐园}{le4yuan2}
    \definition{s.}{paraíso}
  \end{Phonetics}
\end{Entry}

\begin{Entry}{乐高}{5,10}{⼃、⾼}
  \begin{Phonetics}{乐高}{le4gao1}
    \definition*{s.}{Lego (brinquedo)}
  \end{Phonetics}
\end{Entry}

\begin{Entry}{乐趣}{5,15}{⼃、⾛}
  \begin{Phonetics}{乐趣}{le4qu4}[][HSK 4]
    \definition[个,种,些,点]{s.}{alegria; deleite; prazer; implicação de fazer alguém se sentir feliz; um humor de preferência}
  \end{Phonetics}
\end{Entry}

\begin{Entry}{乒}{6}{⼃}
  \begin{Phonetics}{乒}{ping1}
    \definition{interj.}{(onomatopéia) estalo; estouro; estrondo | (onomatopéia)  ``ping''}
    \definition{s.}{(abreviação) tênis de mesa; pingue-pongue | (abreviação) bola de tênis de mesa; bola de pingue-pongue}
  \end{Phonetics}
\end{Entry}

\begin{Entry}{乒乓球}{6,6,11}{⼃、⼃、⽟}
  \begin{Phonetics}{乒乓球}{ping1pang1qiu2}
    \definition[个]{s.}{tênis de mesa |ping-pong}
  \end{Phonetics}
\end{Entry}

\begin{Entry}{乓}{6}{⼃}
  \begin{Phonetics}{乓}{pang1}
    \definition{interj.}{(onomatopéia) barulho repentino feito por tiros, uma porta batendo, coisas quebrando, etc.; estrondo; estouro; batida; colisão}
  \end{Phonetics}
\end{Entry}

\begin{Entry}{乖}{8}{⼃}
  \begin{Phonetics}{乖}{guai1}[][HSK 7-9]
    \definition{adj.}{(uma criança) bem comportado; bom; obediente | inteligente; astuto; esperto | (caráter, comportamento, etc.) estranho; anormal; irracional}
    \definition{v.}{perverter; ser contrário à razão; ir contra | (caráter, comportamento, etc.) ser anormal; ser estranho}
  \end{Phonetics}
\end{Entry}

\begin{Entry}{乖巧}{8,5}{⼃、⼯}
  \begin{Phonetics}{乖巧}{guai1qiao3}[][HSK 7-9]
    \definition{adj.}{fofo; adorável; agradável; descreve crianças, pequenos animais, etc. como sendo obedientes, fofos e simpáticos | inteligente; engenhoso; descreve uma pessoa que sempre fala ou faz coisas de acordo com os desejos de outras pessoas e é querida por elas}
  \end{Phonetics}
\end{Entry}

\begin{Entry*}{乖乖}{8,8}{⼃、⼃}
  \begin{Phonetics}{乖乖}{guai1guai1}
    \definition{adj.}{bem-comportado; obediente}
    \definition{s.}{bebezinho; pequenino; querido; docinho (usado apenas para crianças)}
  \end{Phonetics}
  \begin{Phonetics}{乖乖}{guai1guai5}
    \definition{expr.}{Uau!; Nossa!; Meu Deus!; Oh meu Deus!}
  \end{Phonetics}
\end{Entry*}

%%%%% EOF %%%%%

