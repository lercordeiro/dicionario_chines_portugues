%%%
%%% Radical "⽤"
%%%

\section*{Radical 101: ``⽤''}\addcontentsline{toc}{section}{Radical 101: ⽤}

\begin{entry}{用}{5}{⽤}[Kangxi 101]
  \begin{phonetics}{用}{yong4}[][HSK 1]
    \definition*{s.}{sobrenome Yong}
    \definition{conj.}{portanto; por isso; assim sendo; razões para a introdução, equivalentes a 因}
    \definition{prep.}{com; ação de introduzir ferramentas, meios, etc. utilizados ou empregados}
    \definition{s.}{despesas; gastos; custos | uso; utilidade; eficácia}
    \definition{v.}{usar; aplicar; empregar | necessitar (normalmente na forma negativa) | respeitosamente: comer; beber}
  \seealsoref{因}{yin1}
  \end{phonetics}
\end{entry}

\begin{entry}{用于}{5,3}{⽤、⼆}
  \begin{phonetics}{用于}{yong4 yu2}[][HSK 5]
    \definition{v.}{usar para; ser usado para; usar em}
  \end{phonetics}
\end{entry}

\begin{entry}{用不着}{5,4,11}{⽤、⼀、⽬}
  \begin{phonetics}{用不着}{yong4 bu4 zhao2}[][HSK 5]
    \definition{v.}{não precisar; não ter utilidade para; não haver necessidade de}
  \end{phonetics}
\end{entry}

\begin{entry}{用心}{5,4}{⽤、⼼}
  \begin{phonetics}{用心}{yong4xin1}
    \definition{s.}{motivo | intenção}
    \definition{v.+compl.}{ser diligente ou atencioso}
  \end{phonetics}
\end{entry}

\begin{entry}{用户}{5,4}{⽤、⼾}
  \begin{phonetics}{用户}{yong4hu4}[][HSK 5]
    \definition[个,位,名]{s.}{usuário; consumidor; entidades e indivíduos que utilizam determinados equipamentos públicos ou bens de consumo}
  \end{phonetics}
\end{entry}

\begin{entry}{用处}{5,5}{⽤、⼡}
  \begin{phonetics}{用处}{yong4chu5}
    \definition[个]{s.}{usabilidade | utilidade}
  \end{phonetics}
\end{entry}

\begin{entry}{用来}{5,7}{⽤、⽊}
  \begin{phonetics}{用来}{yong4 lai2}[][HSK 5]
    \definition{v.}{ser usado para; depender (dele) ou usar (ele) para atingir algum objetivo}
  \end{phonetics}
\end{entry}

\begin{entry}{用料}{5,10}{⽤、⽃}
  \begin{phonetics}{用料}{yong4liao4}
    \definition{s.}{ingredientes | materiais}
  \end{phonetics}
\end{entry}

\begin{entry}{用途}{5,10}{⽤、⾡}
  \begin{phonetics}{用途}{yong4tu2}[][HSK 4]
    \definition[个,种]{s.}{uso; aplicação; aspectos ou escopo da aplicação}
  \end{phonetics}
\end{entry}

\begin{entry}{甭}{9}{⽤}
  \begin{phonetics}{甭}{beng2}
    \definition{adv.}{não; não precisa; não tem que; contração de 不用}
  \seealsoref{不用}{bu2 yong4}
  \end{phonetics}
\end{entry}

%%%%% EOF %%%%%

