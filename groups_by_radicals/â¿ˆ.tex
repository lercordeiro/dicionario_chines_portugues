%%%
%%% Radical "⿈"
%%%

\section*{Radical 201: ``⿈'' (黄)}\addcontentsline{toc}{section}{Radical 201: ⿈、黄}

\begin{entry}{黄}{11}{⿈}[Kangxi 201]
  \begin{phonetics}{黄}{huang2}[][HSK 2]
    \definition*{s.}{sobrenome Huang ou Hwang}
    \definition{adj.}{amarelo | pornográfico}
  \end{phonetics}
\end{entry}

\begin{entry}{黄瓜}{11,5}{⿈、⽠}
  \begin{phonetics}{黄瓜}{huang2 gua1}[][HSK 4]
    \definition[根,棵,株]{s.}{pepino}
  \end{phonetics}
\end{entry}

\begin{entry}{黄色}{11,6}{⿈、⾊}
  \begin{phonetics}{黄色}{huang2 se4}[][HSK 2]
    \definition{s.}{cor amarela}
  \end{phonetics}
\end{entry}

\begin{entry}{黄昏}{11,8}{⿈、⽇}
  \begin{phonetics}{黄昏}{huang2hun1}
    \definition{s.}{anoitecer}
  \end{phonetics}
\end{entry}

\begin{entry}{黄河}{11,8}{⿈、⽔}
  \begin{phonetics}{黄河}{huang2he2}
    \definition*{s.}{Rio Amarelo | Rio Huang He}
  \end{phonetics}
\end{entry}

\begin{entry}{黄油}{11,8}{⿈、⽔}
  \begin{phonetics}{黄油}{huang2you2}
    \definition[盒]{s.}{manteiga}
  \end{phonetics}
\end{entry}

\begin{entry}{黄金}{11,8}{⿈、⾦}
  \begin{phonetics}{黄金}{huang2jin1}[][HSK 4]
    \definition{adj.}{de primeira qualidade; dourado;}
    \definition[块,克,两]{s.}{ouro; \emph{aurum}; um tipo de metal, de cor amarela, mais precioso, abreviação de ``金'', frequentemente falado como ``金子''.}
  \seealsoref{金}{jin1}
  \seealsoref{金子}{jin1zi5}
  \end{phonetics}
\end{entry}

%%%%% EOF %%%%%

