%%%
%%% Radical "⽄"
%%%
\section*{Radical 69: ``⽄''}\addcontentsline{toc}{section}{Radical 69: ⽄}\addcontentsline{loh}{figure}{\#\#\#\# 69: ⽄}

%%%%%%%%%% 斤 %%%%%%%%%%
\subsection*{斤}\addcontentsline{loh}{figure}{斤}

\begin{Entry}{斤}{4}{⽄}[Kangxi 69]
  \begin{Phonetics}{斤}{jin1}[][HSK 2]
    \definition{clas.}{uma unidade de peso (=500 gramas)}
    \definition{s.}{machado; cutelo; ferramentas antigas para cortar árvores}
  \end{Phonetics}
\end{Entry}

%%%%%%%%%% 斥 %%%%%%%%%%
\subsection*{斥}\addcontentsline{loh}{figure}{斥}

\begin{Entry}{斥}{5}{⽄}
  \begin{Phonetics}{斥}{chi4}
    \definition*{s.}{Sobrenome: Chi}
    \definition{adj.}{(solo) salino; alcalino}
    \definition{s.}{terra impregnada de sal, portanto estéril}
    \definition{v.}{repreender; censurar; denunciar; reprimir | repelir; excluir; expulsar | fornecer; prover | Literário: abrir; expandir | culpar; reprovar | estender; ampliar | Obsoleto: reconhecer; detectar}
  \end{Phonetics}
\end{Entry}

\begin{Entry}{斥骂}{5,9}{⽄、⾺}
  \begin{Phonetics}{斥骂}{chi4ma4}
    \definition{v.}{repreender}
  \end{Phonetics}
\end{Entry}

%%%%%%%%%% 斧 %%%%%%%%%%
\subsection*{斧}\addcontentsline{loh}{figure}{斧}

\begin{Entry}{斧}{8}{⽄}
  \begin{Phonetics}{斧}{fu3}
    \definition[把,只]{s.}{machado; machadinha | machado de batalha (um tipo de arma usada na China antiga)}
  \end{Phonetics}
\end{Entry}

\begin{Entry}{斧子}{8,3}{⽄、⼦}
  \begin{Phonetics}{斧子}{fu3zi5}[][HSK 7-9]
    \definition[把,个]{s.}{machado; machadinha}
  \end{Phonetics}
\end{Entry}

%%%%%%%%%% 斩 %%%%%%%%%%
\subsection*{斩}\addcontentsline{loh}{figure}{斩}

\begin{Entry}{斩}{8}{⽄}
  \begin{Phonetics}{斩}{zhan3}
    \definition*{s.}{Sobrenome: Zhan}
    \definition{v.}{matar; cortar; picar | (dialeto) tosquiar; chantagear | decapitar}
  \end{Phonetics}
\end{Entry}

\begin{Entry}{斩获}{8,10}{⽄、⾋}
  \begin{Phonetics}{斩获}{zhan3huo4}
    \definition{v.}{matar ou capturar (em batalha) | (figurativo) (esportes) marcar (um gol), ganhar (uma medalha) | (figurativo) colher recompensas, obter ganhos}
  \end{Phonetics}
\end{Entry}

%%%%%%%%%% 断 %%%%%%%%%%
\subsection*{断}\addcontentsline{loh}{figure}{断}

\begin{Entry}{断}{11}{⽄}
  \begin{Phonetics}{断}{duan4}[][HSK 3]
    \definition*{s.}{Sobrenome: Duan}
    \definition{adv.}{(geralmente na forma negativa) absolutamente; decididamente}
    \definition{v.}{quebrar; partir; (objetos longos) dividir em segmentos não conectados | parar; interromper; romper; isolar; fazer com que não se sucedam mais | desistir; abster-se de; parar de fumar, beber, etc. | julgar; decidir | interceptar}
  \end{Phonetics}
\end{Entry}

\begin{Entry}{断交}{11,6}{⽄、⼇}
  \begin{Phonetics}{断交}{duan4/jiao1}
    \definition{v.+compl.}{terminar uma amizade | romper relações diplomáticas}
  \end{Phonetics}
\end{Entry}

\begin{Entry}{断定}{11,8}{⽄、⼧}
  \begin{Phonetics}{断定}{duan4ding4}[][HSK 7-9]
    \definition{v.}{decidir; determinar; concluir; formar um julgamento; fazer um julgamento definitivo}
  \end{Phonetics}
\end{Entry}

\begin{Entry}{断断续续}{11,11,11,11}{⽄、⽄、⽷、⽷}
  \begin{Phonetics}{断断续续}{duan4duan4xu4xu4}[][HSK 7-9]
    \definition{adj.}{intermitente; esporádico; ocasional; aos trancos e barrancos}
  \end{Phonetics}
\end{Entry}

\begin{Entry}{断裂}{11,12}{⽄、⾐}
  \begin{Phonetics}{断裂}{duan4lie4}[][HSK 7-9]
    \definition{s.}{fraturar; quebrar; surtar | fender; rachar; romper}
  \end{Phonetics}
\end{Entry}

%%%%%%%%%% 斯 %%%%%%%%%%
\subsection*{斯}\addcontentsline{loh}{figure}{斯}

\begin{Entry}{斯}{12}{⽄}
  \begin{Phonetics}{斯}{si1}
    \definition*{s.}{Sobrenome: Si}
    \definition{adv.}{então; assim}
    \definition{pron.}{isto; aqui}
  \end{Phonetics}
\end{Entry}

\begin{Entry}{斯巴达}{12,4,6}{⽄、⼰、⾡}
  \begin{Phonetics}{斯巴达}{si1ba1da2}
    \definition*{s.}{Esparta}
  \end{Phonetics}
\end{Entry}

%%%%%%%%%% 新 %%%%%%%%%%
\subsection*{新}\addcontentsline{loh}{figure}{新}

\begin{Entry}{新}{13}{⽄}
  \begin{Phonetics}{新}{xin1}[][HSK 1]
    \definition*{s.}{Xinjiang, abreviação de 新疆 | Singapura, abreviação de 新加坡 | Sobrenome: Xin}
    \definition{adj.}{novo; fresco; inovador; atualizado; aparecer ou ser experimentado pela primeira vez | nunca utilizado; novo; não foi usado ou foi usado por pouco tempo | recém-casado}
    \definition{adv.}{recém; recentemente; há pouco tempo}
    \definition{pref.}{Química: meso-}
    \definition{v.}{atualizar; renovar}
  \seealsoref{新加坡}{xin1jia1po1}
  \seealsoref{新疆}{xin1jiang1}
  \end{Phonetics}
\end{Entry}

\begin{Entry}{新人}{13,2}{⽄、⼈}
  \begin{Phonetics}{新人}{xin1 ren2}[][HSK 6]
    \definition[位]{s.}{pessoas de um novo tipo; nova pessoa;  pessoa que virou uma nova página | nova personalidade; novo talento | recém-chegado; novo membro | noiva ou noivo; recém-casado | \emph{neoanthropus}; \emph{homo sapiens}}
  \end{Phonetics}
\end{Entry}

\begin{Entry}{新加坡}{13,5,8}{⽄、⼒、⼟}
  \begin{Phonetics}{新加坡}{xin1jia1po1}
    \definition*{s.}{Singapura}
  \end{Phonetics}
\end{Entry}

\begin{Entry}{新兴}{13,6}{⽄、⼋}
  \begin{Phonetics}{新兴}{xin1 xing1}[][HSK 6]
    \definition[个]{adj.}{recém-desenvolvido; crescente; florescente; emergente; descreve algo que está apenas começando a se tornar popular ou se desenvolver}
  \end{Phonetics}
\end{Entry}

\begin{Entry}{新年}{13,6}{⽄、⼲}
  \begin{Phonetics}{新年}{xin1 nian2}[][HSK 1]
    \definition*[个]{s.}{Ano Novo}
  \end{Phonetics}
\end{Entry}

\begin{Entry}{新郎}{13,8}{⽄、⾢}
  \begin{Phonetics}{新郎}{xin1lang2}[][HSK 4]
    \definition[位,名,个,些]{s.}{noivo; homens no momento do casamento}
  \end{Phonetics}
\end{Entry}

\begin{Entry}{新型}{13,9}{⽄、⼟}
  \begin{Phonetics}{新型}{xin1 xing2}[][HSK 4]
    \definition[种]{s.}{ultimo modelo; novo tipo; novo padrão; novo estilo}
  \end{Phonetics}
\end{Entry}

\begin{Entry}{新闻}{13,9}{⽄、⾨}
  \begin{Phonetics}{新闻}{xin1wen2}[][HSK 2]
    \definition[个,条,则,版]{s.}{notícias; notícias nacionais e internacionais reportadas em jornais, estações de rádio, etc. | notícias; refere-se a coisas importantes ou novas que aconteceram recentemente na sociedade}
  \end{Phonetics}
\end{Entry}

\begin{Entry}{新娘}{13,10}{⽄、⼥}
  \begin{Phonetics}{新娘}{xin1niang2}[][HSK 4]
    \definition[位,个]{s.}{noiva; a mulher no momento do casamento}
  \seealsoref{新娘子}{xin1niang2zi5}
  \end{Phonetics}
\end{Entry}

\begin{Entry}{新娘子}{13,10,3}{⽄、⼥、⼦}
  \begin{Phonetics}{新娘子}{xin1niang2zi5}
    \definition{s.}{noiva}
  \seealsoref{新娘}{xin1niang2}
  \end{Phonetics}
\end{Entry}

\begin{Entry}{新娘服装}{13,10,8,12}{⽄、⼥、⽉、⾐}
  \begin{Phonetics}{新娘服装}{xin1niang2 fu2zhuang1}
    \definition{s.}{vestido de noiva}
  \end{Phonetics}
\end{Entry}

\begin{Entry}{新鲜}{13,14}{⽄、⿂}
  \begin{Phonetics}{新鲜}{xin1xian1}
    \definition{adj.}{fresco (experiência, alimento, etc.)}
    \definition{s.}{frescor}
  \end{Phonetics}
\end{Entry}

\begin{Entry}{新疆}{13,19}{⽄、⼸}
  \begin{Phonetics}{新疆}{xin1jiang1}
    \definition*{s.}{Região Autônoma Uigur de Xinjiang}
  \end{Phonetics}
\end{Entry}

\begin{Entry}{新疆维吾尔自治区}{13,19,11,7,5,6,8,4}{⽄、⼸、⽷、⼝、⼩、⾃、⽔、⼖}
  \begin{Phonetics}{新疆维吾尔自治区}{xin1jiang1 wei2wu2'er3 zi4zhi4qu1}
    \definition*{s.}{Região Autônoma Uigur de Xinjiang}
  \end{Phonetics}
\end{Entry}

%%%%% EOF %%%%%

