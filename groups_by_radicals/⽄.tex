%%%
%%% Radical "⽄"
%%%

\section*{Radical 69: ``⽄''}\addcontentsline{toc}{section}{Radical 69: ⽄}

\begin{entry}{斤}{4}{⽄}[Kangxi 69]
  \begin{phonetics}{斤}{jin1}[][HSK 2]
    \definition{clas.}{uma unidade de peso (=500 gramas)}
    \definition{s.}{machado; cutelo; ferramentas antigas para cortar árvores}
  \end{phonetics}
\end{entry}

\begin{entry}{斥}{5}{⽄}
  \begin{phonetics}{斥}{chi4}
    \definition*{s.}{sobrenome Chi}
    \definition{adj.}{(do solo) salino; alcalino}
    \definition{s.}{terra impregnada de sal, portanto estéril}
    \definition{v.}{repreender; censurar; denunciar; reprimir | repelir; excluir; expulsar | fornecer; prover | (literário) abrir; expandir | culpar; reprovar | estender; ampliar | (datado) reconhecer; detectar}
  \end{phonetics}
\end{entry}

\begin{entry}{斥骂}{5,9}{⽄、⾺}
  \begin{phonetics}{斥骂}{chi4ma4}
    \definition{v.}{repreender}
  \end{phonetics}
\end{entry}

\begin{entry}{斩获}{8,10}{⽄、⾋}
  \begin{phonetics}{斩获}{zhan3huo4}
    \definition{v.}{matar ou capturar (em batalha) | (figurativo) (esportes) marcar (um gol), ganhar (uma medalha) | (figurativo) colher recompensas, obter ganhos}
  \end{phonetics}
\end{entry}

\begin{entry}{断}{11}{⽄}
  \begin{phonetics}{断}{duan4}[][HSK 3]
    \definition*{s.}{sobrenome Duan}
    \definition{adv.}{(geralmente na forma negativa) absolutamente; decididamente}
    \definition{v.}{quebrar; partir; (objetos longos) dividir em segmentos não conectados | parar; interromper; romper; isolar; fazer com que não se sucedam mais | desistir; abster-se de; parar de fumar, beber, etc. | julgar; decidir | interceptar}
  \end{phonetics}
\end{entry}

\begin{entry}{断交}{11,6}{⽄、⼇}
  \begin{phonetics}{断交}{duan4jiao1}
    \definition{v.+compl.}{terminar uma amizade | romper relações diplomáticas}
  \end{phonetics}
\end{entry}

\begin{entry}{斯巴达}{12,4,6}{⽄、⼰、⾡}
  \begin{phonetics}{斯巴达}{si1ba1da2}
    \definition*{s.}{Esparta}
  \end{phonetics}
\end{entry}

\begin{entry}{新}{13}{⽄}
  \begin{phonetics}{新}{xin1}[][HSK 1]
    \definition*{s.}{sobrenome Xin}
    \definition*{s.}{abreviação de Xinjiang (新疆)}
    \definition*{s.}{abreviação de Singapura (新加坡)}
    \definition{adj.}{novo; fresco; inovador; atualizado; aparecer ou ser experimentado pela primeira vez | nunca utilizado; novo; não foi usado ou foi usado por pouco tempo | recém-casado}
    \definition{adv.}{recém; recentemente; há pouco tempo}
    \definition{pref.}{(química) meso-}
    \definition{v.}{atualizar; renovar}
  \seealsoref{新加坡}{xin1jia1po1}
  \seealsoref{新疆}{xin1jiang1}
  \end{phonetics}
\end{entry}

\begin{entry}{新加坡}{13,5,8}{⽄、⼒、⼟}
  \begin{phonetics}{新加坡}{xin1jia1po1}
    \definition*{s.}{Singapura}
  \end{phonetics}
\end{entry}

\begin{entry}{新年}{13,6}{⽄、⼲}
  \begin{phonetics}{新年}{xin1 nian2}[][HSK 1]
    \definition*[个]{s.}{Ano Novo}
  \end{phonetics}
\end{entry}

\begin{entry}{新郎}{13,8}{⽄、⾢}
  \begin{phonetics}{新郎}{xin1lang2}[][HSK 4]
    \definition[位,个]{s.}{noivo; homens no momento do casamento}
  \end{phonetics}
\end{entry}

\begin{entry}{新型}{13,9}{⽄、⼟}
  \begin{phonetics}{新型}{xin1 xing2}[][HSK 4]
    \definition[种]{s.}{ultimo modelo; novo tipo; novo padrão; novo estilo}
  \end{phonetics}
\end{entry}

\begin{entry}{新闻}{13,9}{⽄、⾨}
  \begin{phonetics}{新闻}{xin1wen2}[][HSK 2]
    \definition[个,条,则,版]{s.}{notícias; notícias nacionais e internacionais reportadas em jornais, estações de rádio, etc. | notícias; refere-se a coisas importantes ou novas que aconteceram recentemente na sociedade}
  \end{phonetics}
\end{entry}

\begin{entry}{新娘}{13,10}{⽄、⼥}
  \begin{phonetics}{新娘}{xin1niang2}[][HSK 4]
    \definition[位,个]{s.}{noiva; a mulher no momento do casamento}
  \seealsoref{新娘子}{xin1niang2zi5}
  \end{phonetics}
\end{entry}

\begin{entry}{新娘子}{13,10,3}{⽄、⼥、⼦}
  \begin{phonetics}{新娘子}{xin1niang2zi5}
    \definition{s.}{noiva}
  \seealsoref{新娘}{xin1niang2}
  \end{phonetics}
\end{entry}

\begin{entry}{新娘服装}{13,10,8,12}{⽄、⼥、⽉、⾐}
  \begin{phonetics}{新娘服装}{xin1niang2 fu2zhuang1}
    \definition{s.}{roupas de noiva}
  \end{phonetics}
\end{entry}

\begin{entry}{新鲜}{13,14}{⽄、⿂}
  \begin{phonetics}{新鲜}{xin1xian1}
    \definition{adj.}{fresco (experiência, alimento, etc.)}
    \definition{s.}{frescor}
  \end{phonetics}
\end{entry}

\begin{entry}{新疆}{13,19}{⽄、⼸}
  \begin{phonetics}{新疆}{xin1jiang1}
    \definition*{s.}{Xinjiang}
  \end{phonetics}
\end{entry}

\begin{entry}{新疆维吾尔自治区}{13,19,11,7,5,6,8,4}{⽄、⼸、⽷、⼝、⼩、⾃、⽔、⼖}
  \begin{phonetics}{新疆维吾尔自治区}{xin1jiang1 wei2wu2'er3 zi4zhi4qu1}
    \definition*{s.}{Região Autônoma Uigur de Xinjiang}
  \end{phonetics}
\end{entry}

%%%%% EOF %%%%%

