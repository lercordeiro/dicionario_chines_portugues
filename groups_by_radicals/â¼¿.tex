%%%
%%% Radical "⼿"
%%%

\section*{Radical 64: ``⼿'' (扌、龵)}\addcontentsline{toc}{section}{Radical 64: ⼿、扌、龵}

\begin{entry}{才}{3}{⼿}
  \begin{phonetics}{才}{cai2}[][HSK 2,4]
    \definition*{s.}{sobrenome Cai}
    \definition{adv.}{indica que algo aconteceu há pouco tempo, agora mesmo | indica que algo acontece ou termina tarde | indica que algo só acontece sob certas condições, ou por um motivo ou propósito específico, seguido do que acontece depois, geralmente é precedida por palavras como “somente”, “deve”, “porque” ou “devido a” | em comparação, indica uma pequena quantidade, poucas ocorrências, pouca habilidade, etc.; meramente | indica ênfase no que está sendo dito, e o caractere “呢” é frequentemente usado no final da frase}
    \definition{conj.}{apenas quando}
    \definition{s.}{capacidade; talento; dom | pessoa capacitada}
  \seealsoref{呢}{ne5}
  \end{phonetics}
\end{entry}

\begin{entry}{才华}{3,6}{⼿、⼗}
  \begin{phonetics}{才华}{cai2hua2}
    \definition[份]{s.}{talento}
  \end{phonetics}
\end{entry}

\begin{entry}{才能}{3,10}{⼿、⾁}
  \begin{phonetics}{才能}{cai2 neng2}[][HSK 3]
    \definition[间]{s.}{talento | habilidade | dom | capacidade}
  \end{phonetics}
\end{entry}

\begin{entry}{才略}{3,11}{⼿、⽥}
  \begin{phonetics}{才略}{cai2lve4}
    \definition{s.}{habilidade e sagacidade}
  \end{phonetics}
\end{entry}

\begin{entry}{手}{4}{⼿}[Kangxi 64]
  \begin{phonetics}{手}{shou3}[][HSK 1]
    \definition{adj.}{conveniente}
    \definition{clas.}{de habilidade}
    \definition[双,只]{s.}{mão | pessoa envolvida em certos tipos de trabalho | pessoa qualificada para certos tipos de trabalho}
    \definition{v.}{segurar (formal)}
  \end{phonetics}
\end{entry}

\begin{entry}{手工}{4,3}{⼿、⼯}
  \begin{phonetics}{手工}{shou3gong1}[][HSK 4]
    \definition{s.}{trabalho manual; trabalho feito à mão | método de operação manual; método manual, sem máquina | remuneração por trabalho manual, braçal; custo de mão de obra braçal}
  \end{phonetics}
\end{entry}

\begin{entry}{手工艺人}{4,3,4,2}{⼿、⼯、⾋、⼈}
  \begin{phonetics}{手工艺人}{shou3gong1 yi4ren2}
    \definition{s.}{artesão}
  \end{phonetics}
\end{entry}

\begin{entry}{手术}{4,5}{⼿、⽊}
  \begin{phonetics}{手术}{shou3shu4}[][HSK 4]
    \definition[个]{s.}{cirurgia; operação (cirúrgica); método de tratamento no qual o médico usa uma faca, tesoura etc. para fazer uma incisão em uma parte do corpo do paciente}
    \definition{v.}{realizar uma cirurgia}
  \end{phonetics}
\end{entry}

\begin{entry}{手边}{4,5}{⼿、⾡}
  \begin{phonetics}{手边}{shou3bian1}
    \definition{adv.}{à mão | na mão}
  \end{phonetics}
\end{entry}

\begin{entry}{手机}{4,6}{⼿、⽊}
  \begin{phonetics}{手机}{shou3ji1}[][HSK 1]
    \definition[部,支]{s.}{telefone celular ou móvel}
  \end{phonetics}
\end{entry}

\begin{entry}{手里}{4,7}{⼿、⾥}
  \begin{phonetics}{手里}{shou3 li3}[][HSK 4]
    \definition[个]{s.}{(uma situação está) nas mãos de alguém | em mãos}
  \end{phonetics}
\end{entry}

\begin{entry}{手刹}{4,8}{⼿、⼑}
  \begin{phonetics}{手刹}{shou3sha1}
    \definition{s.}{freio de mão}
  \end{phonetics}
\end{entry}

\begin{entry}{手法}{4,8}{⼿、⽔}
  \begin{phonetics}{手法}{shou3fa3}[][HSK 5]
    \definition{s.}{habilidade; técnica; técnicas de criação (de obras literárias e artísticas) | truque; artifício; artimanha; refere-se a métodos inadequados usados para lidar com as pessoas}
  \end{phonetics}
\end{entry}

\begin{entry}{手表}{4,8}{⼿、⾐}
  \begin{phonetics}{手表}{shou3biao3}[][HSK 2]
    \definition[块,只,个]{s.}{relógio de pulso}
  \end{phonetics}
\end{entry}

\begin{entry}{手指}{4,9}{⼿、⼿}
  \begin{phonetics}{手指}{shou3zhi3}[][HSK 3]
    \definition[根,个]{s.}{dedo da mão}
  \end{phonetics}
\end{entry}

\begin{entry}{手段}{4,9}{⼿、⽎}
  \begin{phonetics}{手段}{shou3 duan4}[][HSK 5]
    \definition[种]{s.}{meios; meio; medida; método; métodos e técnicas utilizados para atingir um determinado objetivo | truque; artifício; métodos inadequados de lidar com as pessoas | habilidade; capacidade; delicadeza; sutileza; técnica}
  \end{phonetics}
\end{entry}

\begin{entry}{手套}{4,10}{⼿、⼤}
  \begin{phonetics}{手套}{shou3tao4}[][HSK 4]
    \definition[副,套,双,种]{s.}{luvas; itens usados ​​nas mãos, feitos de algodão, lã, couro, etc., para proteger as mãos ou manter o frio longe}
  \end{phonetics}
\end{entry}

\begin{entry}{手续}{4,11}{⼿、⽷}
  \begin{phonetics}{手续}{shou3xu4}[][HSK 3]
    \definition[个]{s.}{processo; formalidade; procedimento}
  \end{phonetics}
\end{entry}

\begin{entry}{手臂}{4,17}{⼿、⾁}
  \begin{phonetics}{手臂}{shou3bi4}
    \definition{s.}{braço}
  \end{phonetics}
\end{entry}

\begin{entry}{扑克}{5,7}{⼿、⼗}
  \begin{phonetics}{扑克}{pu1ke4}
    \definition{s.}{(empréstimo linguístico) (jogo) \emph{poker}  | baralho}
  \end{phonetics}
\end{entry}

\begin{entry}{扒犁}{5,11}{⼿、⽜}
  \begin{phonetics}{扒犁}{pa2li2}
    \definition{s.}{trenó}
    \seeref{爬犁}{pa2li2}
  \end{phonetics}
\end{entry}

\begin{entry}{打}{5}{⼿}
  \begin{phonetics}{打}{da2}
    \definition{clas./s.}{(empréstimo linguístico) dúzia}
  \end{phonetics}
  \begin{phonetics}{打}{da3}[][HSK 1,4,5]
    \definition{prep.}{de; desde; ponto de partida que indica lugar, tempo ou extensão | indica rotas e locais percorridos}
    \definition{v.}{golpear; acertar; bater | quebrar; esmagar | lutar; atacar; espancar | entrar com uma ação judicial; negociar; fazer representações | construir; edificar | fabricar (em uma ferraria); forjar | misturar; mexer; bater | amarrar; embalar | tricotar; tecer | desenhar; pintar; deixar uma marca; imprimir | abrir; perfurar; cavar | içar; levantar
enviar; despachar; projetar | emitir ou receber (um certificado, etc.) | remover; livrar-se de}
  \end{phonetics}
\end{entry}

\begin{entry}{打工}{5,3}{⼿、⼯}
  \begin{phonetics}{打工}{da3gong1}[][HSK 2]
    \definition{v.}{(para alunos) ter um emprego fora do horário de aula ou durante as férias | trabalhar em um emprego temporá rio ou casual}
  \end{phonetics}
\end{entry}

\begin{entry}{打工人}{5,3,2}{⼿、⼯、⼈}
  \begin{phonetics}{打工人}{da3gong1ren2}
    \definition{s.}{trabalhador}
  \end{phonetics}
\end{entry}

\begin{entry}{打开}{5,4}{⼿、⼶}
  \begin{phonetics}{打开}{da3 kai1}[][HSK 1]
    \definition{v.}{abrir | desdobrar | ligar | avançar | espalhar}
  \end{phonetics}
\end{entry}

\begin{entry}{打车}{5,4}{⼿、⾞}
  \begin{phonetics}{打车}{da3 che1}[][HSK 1]
    \definition{v.}{pegar um táxi | chamar um táxi}
  \end{phonetics}
\end{entry}

\begin{entry}{打击}{5,5}{⼿、⼐}
  \begin{phonetics}{打击}{da3ji1}[][HSK 5]
    \definition{v.}{golpear; atacar; reprimir; atacar para frustrar; machucar | bater; bater (em um tambor, etc.); golpear ou bater em algo}
  \end{phonetics}
\end{entry}

\begin{entry}{打包}{5,5}{⼿、⼓}
  \begin{phonetics}{打包}{da3bao1}[][HSK 5]
    \definition{v.}{levar a comida embora; levar para viagem; refere-se especificamente a comer em um restaurante e levar as sobras em uma caixa, sacola ou outro recipiente | embalar; empacotar | desembalar; desempacotar}
  \end{phonetics}
\end{entry}

\begin{entry}{打印}{5,5}{⼿、⼙}
  \begin{phonetics}{打印}{da3yin4}[][HSK 2]
    \definition{v.}{imprimir}
  \end{phonetics}
\end{entry}

\begin{entry}{打电话}{5,5,8}{⼿、⽥、⾔}
  \begin{phonetics}{打电话}{da3 dian4 hua4}[][HSK 1]
    \definition{v.}{telefonar | fazer uma chamada telefônica | dar um telefonema}
  \seealsoref{给……打电话}{gei3 da3 dian4 hua4}
  \end{phonetics}
\end{entry}

\begin{entry}{打压}{5,6}{⼿、⼚}
  \begin{phonetics}{打压}{da3ya1}
    \definition{v.}{reprimir | derrotar}
  \end{phonetics}
\end{entry}

\begin{entry}{打扫}{5,6}{⼿、⼿}
  \begin{phonetics}{打扫}{da3sao3}[][HSK 4]
    \definition{v.}{varrer; limpar; varrer para limpar}
  \end{phonetics}
\end{entry}

\begin{entry}{打听}{5,7}{⼿、⼝}
  \begin{phonetics}{打听}{da3ting5}[][HSK 3]
    \definition{v.}{perguntar sobre; indagar sobre; obter uma linha sobre}
  \end{phonetics}
\end{entry}

\begin{entry}{打屁股}{5,7,8}{⼿、⼫、⾁}
  \begin{phonetics}{打屁股}{da3pi4gu5}
    \definition{v.}{dar um tapa no bumbum de alguém}
  \end{phonetics}
\end{entry}

\begin{entry}{打扮}{5,7}{⼿、⼿}
  \begin{phonetics}{打扮}{da3ban5}[][HSK 5]
    \definition{s.}{estilo de se vestir; o modo de se vestir; as roupas que se usa}
    \definition{v.}{vestir-se bem; maquiar-se; dar uma boa aparência e vestir-se bem; adornar}
  \end{phonetics}
\end{entry}

\begin{entry}{打扰}{5,7}{⼿、⼿}
  \begin{phonetics}{打扰}{da3rao3}[][HSK 5]
    \definition{v.}{perturbar; incomodar; interferir no trabalho normal, na vida ou no que as outras pessoas estão fazendo, etc. | usado para expressar um pedido de desculpas por ajuda; gratidão por ajuda; hospitalidade recebida}
  \end{phonetics}
\end{entry}

\begin{entry}{打折}{5,7}{⼿、⼿}
  \begin{phonetics}{打折}{da3zhe2}[][HSK 4]
    \definition{v.+compl.}{dar desconto; dar um desconto; vender produtos a um preço reduzido em uma determinada porcentagem do preço original; metáfora para não cumprir 100\% do que foi originalmente padronizado ou prometido}
  \end{phonetics}
\end{entry}

\begin{entry}{打针}{5,7}{⼿、⾦}
  \begin{phonetics}{打针}{da3zhen1}[][HSK 4]
    \definition{v.+compl.}{dar ou receber uma injeção; injetar um medicamento líquido em um organismo com uma seringa}
  \end{phonetics}
\end{entry}

\begin{entry}{打的}{5,8}{⼿、⽩}
  \begin{phonetics}{打的}{da3di1}
    \definition{v.+compl.}{(coloquial) pegar um táxi | ir de táxi}
  \end{phonetics}
\end{entry}

\begin{entry}{打败}{5,8}{⼿、⾒}
  \begin{phonetics}{打败}{da3 bai4}[][HSK 4]
    \definition{v.}{derrotar; vencer; piorar | sofrer uma derrota; ser derrotado}
  \end{phonetics}
\end{entry}

\begin{entry}{打架}{5,9}{⼿、⽊}
  \begin{phonetics}{打架}{da3jia4}[][HSK 5]
    \definition{v.+compl.}{brigar; discutir; entrar em conflito | contradizer; conflitar; ser inconsistente}
  \end{phonetics}
\end{entry}

\begin{entry}{打结}{5,9}{⼿、⽷}
  \begin{phonetics}{打结}{da3jie2}
    \definition{v.}{dar um nó | amarrar}
  \end{phonetics}
\end{entry}

\begin{entry}{打骂}{5,9}{⼿、⾺}
  \begin{phonetics}{打骂}{da3ma4}
    \definition{v.}{bater e repreender}
  \end{phonetics}
\end{entry}

\begin{entry}{打破}{5,10}{⼿、⽯}
  \begin{phonetics}{打破}{da3 po4}[][HSK 3]
    \definition{v.}{quebrar; esmagar}
  \end{phonetics}
\end{entry}

\begin{entry}{打猎}{5,11}{⼿、⽝}
  \begin{phonetics}{打猎}{da3lie4}
    \definition{v.}{ir caçar}
  \end{phonetics}
\end{entry}

\begin{entry}{打球}{5,11}{⼿、⽟}
  \begin{phonetics}{打球}{da3 qiu2}[][HSK 1]
    \definition{v.}{jogar bola (com as mãos) | jogar (basquetebol, handbol, etc.)}
  \end{phonetics}
\end{entry}

\begin{entry}{打搅}{5,12}{⼿、⼿}
  \begin{phonetics}{打搅}{da3jiao3}
    \definition{v.}{perturbar | incomodar}
  \end{phonetics}
\end{entry}

\begin{entry}{打雷}{5,13}{⼿、⾬}
  \begin{phonetics}{打雷}{da3 lei2}[][HSK 4]
    \definition{v.}{trovejar; produzir ruídos altos quando as nuvens descarregam eletricidade}
  \end{phonetics}
\end{entry}

\begin{entry}{打算}{5,14}{⼿、⽵}
  \begin{phonetics}{打算}{da3suan4}[][HSK 2]
    \definition[个]{s.}{plano | intenção}
    \definition{v.}{pensar | planejar | pretender}
  \end{phonetics}
\end{entry}

\begin{entry}{打瞌睡}{5,15,13}{⼿、⽬、⽬}
  \begin{phonetics}{打瞌睡}{da3ke1shui4}
    \definition{v.}{cochilar}
  \end{phonetics}
\end{entry}

\begin{entry}{打磨}{5,16}{⼿、⽯}
  \begin{phonetics}{打磨}{da3mo2}
    \definition{v.}{polir | fazer brilhar}
  \end{phonetics}
\end{entry}

\begin{entry}{扔}{5}{⼿}
  \begin{phonetics}{扔}{reng1}[][HSK 5]
    \definition{v.}{arremessar; lançar; atirar; jogar | esquecer; jogar fora; descartar | colocar casualmente; deixar as pessoas ou as coisas de lado, não se importar}
  \end{phonetics}
\end{entry}

\begin{entry}{扔下}{5,3}{⼿、⼀}
  \begin{phonetics}{扔下}{reng1xia4}
    \definition{v.}{lançar (uma bomba) | derrubar}
  \end{phonetics}
\end{entry}

\begin{entry}{扔弃}{5,7}{⼿、⼶}
  \begin{phonetics}{扔弃}{reng1qi4}
    \definition{v.}{abandonar | descartar | jogar fora}
  \end{phonetics}
\end{entry}

\begin{entry}{扔掉}{5,11}{⼿、⼿}
  \begin{phonetics}{扔掉}{reng1diao4}
    \definition{v.}{jogar fora}
  \end{phonetics}
\end{entry}

\begin{entry}{扛}{6}{⼿}
  \begin{phonetics}{扛}{gang1}
    \definition{v.}{levantar com as duas mãos | carregar alguma coisa juntos (duas ou mais pessoas)}
  \end{phonetics}
  \begin{phonetics}{扛}{kang2}
    \definition{v.}{carregar no ombro de alguém |  (fig.) assumir (um fardo, dever, etc.)}
  \end{phonetics}
\end{entry}

\begin{entry}{执行}{6,6}{⼿、⾏}
  \begin{phonetics}{执行}{zhi2xing2}[][HSK 5]
    \definition{v.}{executar; implementar; realizar}
  \end{phonetics}
\end{entry}

\begin{entry}{执着}{6,11}{⼿、⽬}
  \begin{phonetics}{执着}{zhi2zhuo2}
    \definition{s.}{(budismo) apego}
    \definition{v.}{estar fortemente apegado a | ser dedicado | apegar-se a}
  \end{phonetics}
\end{entry}

\begin{entry}{扩大}{6,3}{⼿、⼤}
  \begin{phonetics}{扩大}{kuo4da4}[][HSK 4]
    \definition{v.}{ampliar; expandir; estender; alargar}
  \end{phonetics}
\end{entry}

\begin{entry}{扩展}{6,10}{⼿、⼫}
  \begin{phonetics}{扩展}{kuo4 zhan3}[][HSK 4]
    \definition{v.}{esticar; expandir; estender; espalhar}
  \end{phonetics}
\end{entry}

\begin{entry}{扫}{6}{⼿}
  \begin{phonetics}{扫}{sao3}[][HSK 4]
    \definition{v.}{varrer; limpar | passar rapidamente ao longo ou sobre; varrer | juntar tudo}
  \end{phonetics}
  \begin{phonetics}{扫}{sao4}
    \seeref{扫帚}{sao4zhou5}
  \end{phonetics}
\end{entry}

\begin{entry}{扫兴}{6,6}{⼿、⼋}
  \begin{phonetics}{扫兴}{sao3xing4}
    \definition{v.+compl.}{sentir-se decepcionado | entristecer alguém}
  \end{phonetics}
\end{entry}

\begin{entry}{扫帚}{6,8}{⼿、⼱}
  \begin{phonetics}{扫帚}{sao4zhou5}
    \definition[把]{s.}{vassoura; ferramenta de varredura feita de varas de bambu, etc., maior que uma vassora}
  \end{phonetics}
\end{entry}

\begin{entry}{扬雄}{6,12}{⼿、⾫}
  \begin{phonetics}{扬雄}{yang2xiong2}
    \definition*{s.}{Yang Xiong (53 AC-18 DC), estudioso, poeta e lexicógrafo, autor do primeiro dicionário de dialeto chinês 方言}
  \seealsoref{方言}{fang1yan2}
  \end{phonetics}
\end{entry}

\begin{entry}{扮演}{7,14}{⼿、⽔}
  \begin{phonetics}{扮演}{ban4yan3}[][HSK 5]
    \definition{v.}{desempenhar o papel de; ter um papel (em uma peça, etc.); atuar}
  \end{phonetics}
\end{entry}

\begin{entry}{扶}{7}{⼿}
  \begin{phonetics}{扶}{fu2}[][HSK 5]
    \definition*{s.}{sobrenome Fu}
    \definition{v.}{segurar; apoiar com a mão; segurar algo com o apoio das mãos para que ninguém, objeto ou pessoa caia | dar apoio a; ajudar uma pessoa deitada ou caída a se levantar com as mãos; endireitar um objeto caído com as mãos | ajudar; tirar de baixo}
  \end{phonetics}
\end{entry}

\begin{entry}{扶梯}{7,11}{⼿、⽊}
  \begin{phonetics}{扶梯}{fu2ti1}
    \definition{s.}{escada rolante}
  \end{phonetics}
\end{entry}

\begin{entry}{批}{7}{⼿}
  \begin{phonetics}{批}{pi1}[][HSK 4]
    \definition{adj.}{(compra ou venda) atacado; a granel; em grandes quantidades}
    \definition{clas.}{para mercadorias a granel, grande número de pessoas}
    \definition{s.}{fibras de algodão, linho, etc., prontas para serem estiradas e torcidas | anotação; comentário}
    \definition{v.}{escrever comentários ou críticas sobre documentos subordinados, textos de outras pessoas, tarefas etc. | refutar; criticar | dar um tapa}
  \end{phonetics}
\end{entry}

\begin{entry}{批评}{7,7}{⼿、⾔}
  \begin{phonetics}{批评}{pi1ping2}[][HSK 3]
    \definition{s.}{crítica}
    \definition{v.}{criticar; comentar sobre}
  \end{phonetics}
\end{entry}

\begin{entry}{批准}{7,10}{⼿、⼎}
  \begin{phonetics}{批准}{pi1zhun3}[][HSK 3]
    \definition{v.}{aprovar}
  \end{phonetics}
\end{entry}

\begin{entry}{找}{7}{⼿}
  \begin{phonetics}{找}{zhao3}[][HSK 1]
    \definition{v.}{andar à procura de | procurar | tentar procurar | dar troco | retornar algo}
  \end{phonetics}
\end{entry}

\begin{entry}{找见}{7,4}{⼿、⾒}
  \begin{phonetics}{找见}{zhao3jian4}
    \definition{v.}{encontrar (algo que está procurando)}
  \end{phonetics}
\end{entry}

\begin{entry}{找出}{7,5}{⼿、⼐}
  \begin{phonetics}{找出}{zhao3 chu1}[][HSK 2]
    \definition{v.}{encontrar | procurar}
  \end{phonetics}
\end{entry}

\begin{entry}{找回}{7,6}{⼿、⼞}
  \begin{phonetics}{找回}{zhao3hui2}
    \definition{v.}{recuperar algo}
  \end{phonetics}
\end{entry}

\begin{entry}{找寻}{7,6}{⼿、⼨}
  \begin{phonetics}{找寻}{zhao3xun2}
    \definition{v.}{encontrar falhas | procurar | buscar}
  \end{phonetics}
\end{entry}

\begin{entry}{找事}{7,8}{⼿、⼅}
  \begin{phonetics}{找事}{zhao3shi4}
    \definition{v.}{procurar emprego | começar uma briga}
  \end{phonetics}
\end{entry}

\begin{entry}{找到}{7,8}{⼿、⼑}
  \begin{phonetics}{找到}{zhao3dao4}[][HSK 1]
    \definition{v.}{encontrar}
  \end{phonetics}
\end{entry}

\begin{entry}{找钱}{7,10}{⼿、⾦}
  \begin{phonetics}{找钱}{zhao3qian2}
    \definition{v.}{dar troco}
  \end{phonetics}
\end{entry}

\begin{entry}{找着}{7,11}{⼿、⽬}
  \begin{phonetics}{找着}{zhao3zhao2}
    \definition{v.}{encontrar}
  \end{phonetics}
\end{entry}

\begin{entry}{找遍}{7,12}{⼿、⾡}
  \begin{phonetics}{找遍}{zhao3bian4}
    \definition{v.}{pentear | pesquisar em todos os lugares}
  \end{phonetics}
\end{entry}

\begin{entry}{找零}{7,13}{⼿、⾬}
  \begin{phonetics}{找零}{zhao3ling2}
    \definition{v.}{trocar dinheiro | dar troco}
  \end{phonetics}
\end{entry}

\begin{entry}{找辙}{7,16}{⼿、⾞}
  \begin{phonetics}{找辙}{zhao3zhe2}
    \definition{v.}{procurar um pretexto}
  \end{phonetics}
\end{entry}

\begin{entry}{技巧}{7,5}{⼿、⼯}
  \begin{phonetics}{技巧}{ji4qiao3}[][HSK 4]
    \definition{s.}{habilidade; técnica; habilidades engenhosas expressas em artes, artesanato, esportes, etc.}
  \end{phonetics}
\end{entry}

\begin{entry}{技术}{7,5}{⼿、⽊}
  \begin{phonetics}{技术}{ji4shu4}[][HSK 3]
    \definition[种,门,项]{s.}{habilidade; técnica; tecnologia}
  \end{phonetics}
\end{entry}

\begin{entry}{技俩}{7,9}{⼿、⼈}
  \begin{phonetics}{技俩}{ji4liang3}
    \definition{s.}{truque | estratagema | ardil | esquema | estratégia | tática}
  \end{phonetics}
\end{entry}

\begin{entry}{技能}{7,10}{⼿、⾁}
  \begin{phonetics}{技能}{ji4 neng2}[][HSK 5]
    \definition[种,项]{s.}{habilidade técnica; domínio de uma habilidade ou técnica; capacidade de adquirir e aplicar conhecimento}
  \end{phonetics}
\end{entry}

\begin{entry}{抄}{7}{⼿}
  \begin{phonetics}{抄}{chao1}[][HSK 4]
    \definition*{s.}{sobrenome Chao}
    \definition{v.}{copiar; transcrever | plagiar | revistar e confiscar; fazer uma batida | pegar um atalho | dobrar (os braços) | agarrar; pegar}
  \end{phonetics}
\end{entry}

\begin{entry}{抄写}{7,5}{⼿、⼍}
  \begin{phonetics}{抄写}{chao1 xie3}[][HSK 4]
    \definition{v.}{copiar; transcrever}
  \end{phonetics}
\end{entry}

\begin{entry}{把}{7}{⼿}
  \begin{phonetics}{把}{ba3}[][HSK 3]
    \definition{clas.}{para objetos com alça | para objetos pequenos:~punhado}
    \definition{part.}{partícula tornando o substantivo seguinte um objeto direto}
    \definition{v.}{conter | alcançar | segurar}
  \end{phonetics}
  \begin{phonetics}{把}{ba4}
    \definition{v.}{lidar}
  \end{phonetics}
\end{entry}

\begin{entry}{把风}{7,4}{⼿、⾵}
  \begin{phonetics}{把风}{ba3feng1}
    \definition{v.}{estar atento | vigiar (durante uma atividade clandestina)}
  \end{phonetics}
\end{entry}

\begin{entry}{把关}{7,6}{⼿、⼋}
  \begin{phonetics}{把关}{ba3guan1}
    \definition{v.}{verificar estritamente | examinar cuidadosamente para ver se algo é feito de acordo com um padrão fixo | fazer a verificação final | guardar uma passagem, fronteira}
  \end{phonetics}
\end{entry}

\begin{entry}{把守}{7,6}{⼿、⼧}
  \begin{phonetics}{把守}{ba3shou3}
    \definition{v.}{vigiar | guardar}
  \end{phonetics}
\end{entry}

\begin{entry}{把式}{7,6}{⼿、⼷}
  \begin{phonetics}{把式}{ba3shi4}
    \definition{s.}{pessoa qualificada em um comércio}
  \end{phonetics}
\end{entry}

\begin{entry}{把戏}{7,6}{⼿、⼽}
  \begin{phonetics}{把戏}{ba3xi4}
    \definition{s.}{acrobacia | malabarismo | truque barato}
  \end{phonetics}
\end{entry}

\begin{entry}{把玩}{7,8}{⼿、⽟}
  \begin{phonetics}{把玩}{ba3wan2}
    \definition{v.}{brincar com | mexer com}
  \end{phonetics}
\end{entry}

\begin{entry}{把持}{7,9}{⼿、⼿}
  \begin{phonetics}{把持}{ba3chi2}
    \definition{v.}{controlar | dominar | monopolizar}
  \end{phonetics}
\end{entry}

\begin{entry}{把柄}{7,9}{⼿、⽊}
  \begin{phonetics}{把柄}{ba3bing3}
    \definition{s.}{(figurativo) informações que podem ser usadas contra alguém}
  \end{phonetics}
\end{entry}

\begin{entry}{把脉}{7,9}{⼿、⾁}
  \begin{phonetics}{把脉}{ba3mai4}
    \definition{v.}{sentir ou tomar o pulso de alguém}
  \end{phonetics}
\end{entry}

\begin{entry}{把握}{7,12}{⼿、⼿}
  \begin{phonetics}{把握}{ba3wo4}[][HSK 3]
    \definition{s.}{seguro | garantia | certeza}
    \definition{v.}{agarrar | segurar | aproveitar}
  \end{phonetics}
\end{entry}

\begin{entry}{把稳}{7,14}{⼿、⽲}
  \begin{phonetics}{把稳}{ba3wen3}
    \definition{adj.}{confiável}
  \end{phonetics}
\end{entry}

\begin{entry}{抓}{7}{⼿}
  \begin{phonetics}{抓}{zhua1}[][HSK 3]
    \definition{v.}{agarrar | arranhar | capturar | compreender; conhecer a chave ou a chave das coisas ou problemas | focar em algo; fortalecer o poder de fazer (algo) ou administrar (algum aspecto) | atrair a atenção de alguém}
  \end{phonetics}
\end{entry}

\begin{entry}{抓住}{7,7}{⼿、⼈}
  \begin{phonetics}{抓住}{zhua1 zhu4}[][HSK 3]
    \definition{v.}{apanhar; prender; capturar (uma pessoa ou animal) e ter sucesso | segurar; agarrar; segurar algo e deixá-lo imóvel}
  \end{phonetics}
\end{entry}

\begin{entry}{抓紧}{7,10}{⼿、⽷}
  \begin{phonetics}{抓紧}{zhua1jin3}[][HSK 4]
    \definition{v.}{agarrar com firmeza; segurar firme e não soltar | prestar muita atenção a}
  \end{phonetics}
\end{entry}

\begin{entry}{投}{7}{⼿}
  \begin{phonetics}{投}{tou2}[][HSK 4]
    \definition*{s.}{sobrenome Tou}
    \definition{pron.}{para; indica tempo, equivalente a ``到'', ``临'' | para; em direção a; indica orientação, direção, equivalente a ``朝'' ou ``向''}
    \definition{s.}{um jogo durante uma festa em que o vencedor era decidido pelo número de flechas lançadas em um pote distante | jogo de dados}
    \definition{v.}{lançar; arremessar; atirar | deixar cair; colocar em; lançar | mergulhar em; lançar-se em; pular dentro | lançar; projetar; sombrear | entregar; postar; enviar | ir até; ir para; buscar; juntar-se | sentir-se atraído por; adaptar-se a; concordar com; atender a}
  \seealsoref{朝}{chao2}
  \seealsoref{到}{dao4}
  \seealsoref{临}{lin2}
  \seealsoref{向}{xiang4}
  \end{phonetics}
\end{entry}

\begin{entry}{投入}{7,2}{⼿、⼊}
  \begin{phonetics}{投入}{tou2ru4}[][HSK 4]
    \definition{adj.}{sisudo; dedicado; devotado; absorto}
    \definition{s.}{investimento; insumo; refere-se à aplicação de recursos}
    \definition{v.}{lançar em; colocar em; jogar em; por em | entrar em uma situação; participar de | aplicar; investir; colocar fundos em}
  \end{phonetics}
\end{entry}

\begin{entry}{投诉}{7,7}{⼿、⾔}
  \begin{phonetics}{投诉}{tou2su4}[][HSK 4]
    \definition{v.}{reclamar; queixar-se; reclamar às autoridades ou pessoas envolvidas}
  \end{phonetics}
\end{entry}

\begin{entry}{投资}{7,10}{⼿、⾙}
  \begin{phonetics}{投资}{tou2zi1}[][HSK 4]
    \definition[次]{s.}{investimento}
    \definition{v.}{investir; aplicar dinheiro; investir dinheiro em negócios}
  \end{phonetics}
\end{entry}

\begin{entry}{投资人}{7,10,2}{⼿、⾙、⼈}
  \begin{phonetics}{投资人}{tou2zi1ren2}
    \definition{s.}{investidor}
  \seealsoref{投资家}{tou2zi1jia1}
  \seealsoref{投资者}{tou2zi1zhe3}
  \end{phonetics}
\end{entry}

\begin{entry}{投资风险}{7,10,4,9}{⼿、⾙、⾵、⾩}
  \begin{phonetics}{投资风险}{tou2zi1feng1xian3}
    \definition{s.}{risco de investimento}
  \end{phonetics}
\end{entry}

\begin{entry}{投资回报率}{7,10,6,7,11}{⼿、⾙、⼞、⼿、⽞}
  \begin{phonetics}{投资回报率}{tou2zi1hui2bao4lv4}
    \definition{s.}{retorno sobre o investimento (ROI)}
  \end{phonetics}
\end{entry}

\begin{entry}{投资者}{7,10,8}{⼿、⾙、⽼}
  \begin{phonetics}{投资者}{tou2zi1zhe3}
    \definition{s.}{investidor}
  \seealsoref{投资家}{tou2zi1jia1}
  \seealsoref{投资人}{tou2zi1ren2}
  \end{phonetics}
\end{entry}

\begin{entry}{投资家}{7,10,10}{⼿、⾙、⼧}
  \begin{phonetics}{投资家}{tou2zi1jia1}
    \definition{s.}{investidor}
  \seealsoref{投资人}{tou2zi1ren2}
  \seealsoref{投资者}{tou2zi1zhe3}
  \end{phonetics}
\end{entry}

\begin{entry}{投递}{7,10}{⼿、⾡}
  \begin{phonetics}{投递}{tou2di4}
    \definition{v.}{despachar | enviar}
  \end{phonetics}
\end{entry}

\begin{entry}{投票}{7,11}{⼿、⽰}
  \begin{phonetics}{投票}{tou2piao4}
    \definition{v.+compl.}{votar | depositar um voto}
  \end{phonetics}
\end{entry}

\begin{entry}{折}{7}{⼿}
  \begin{phonetics}{折}{she2}
    \definition{v.}{estalar; quebrar | perder dinheiro em um negócio}
  \end{phonetics}
  \begin{phonetics}{折}{zhe1}
    \definition{v.}{rolar; virar | despejar algo de um recipiente em outro; ficar despejando algo entre dois recipientes}
  \end{phonetics}
  \begin{phonetics}{折}{zhe2}[][HSK 4]
    \definition*{s.}{sobrenome Zhe}
    \definition{clas.}{uma passagem em um roteiro de ópera miscelânea de Yuan, aproximadamente equivalente a uma cena ou ato em uma ópera moderna}
    \definition[张,个,些]{s.}{fratura; quebra | abatimento; desconto | traços dos caracteres chineses que têm o formato de "𠃍" e "乚", etc. | pasta; livreto; \emph{folder}}
    \definition{v.}{estalar; quebrar; fazer quebrar | perder; sofrer a perda de | voltar para trás; mudar de direção; retornar |ser convencido; estar cheio de admiração | equivaler a; converter em | dobrar}
  \end{phonetics}
\end{entry}

\begin{entry}{折转}{7,8}{⼿、⾞}
  \begin{phonetics}{折转}{zhe2zhuan3}
    \definition{s.}{reflexo (ângulo)}
    \definition{v.}{voltar atrás}
  \end{phonetics}
\end{entry}

\begin{entry}{抢}{7}{⼿}
  \begin{phonetics}{抢}{qiang1}
    \definition{prep.}{contra; direção relativa inversa}
    \definition{v.}{bater; tocar}
  \end{phonetics}
  \begin{phonetics}{抢}{qiang3}[][HSK 5]
    \definition{v.}{roubar; saquear | agarrar; apanhar; arrebatar | disputar; lutar por; ser o primeiro; competir para ser o primeiro | correr; apressar-se; fazer uma incursão | raspar; arranhar; raspar ou esfregar uma camada da superfície de um objeto}
  \end{phonetics}
\end{entry}

\begin{entry}{抢掠}{7,11}{⼿、⼿}
  \begin{phonetics}{抢掠}{qiang3lve4}
    \definition{s.}{saque | pilhagem}
    \definition{v.}{saquear | pilhar}
  \end{phonetics}
\end{entry}

\begin{entry}{抢救}{7,11}{⼿、⽁}
  \begin{phonetics}{抢救}{qiang3jiu4}[][HSK 5]
    \definition{v.}{salvar; resgatar; prestar de socorro ou assistência rápidos em situações de emergência | salvar; tomar medidas rápidas para evitar ou minimizar perdas iminentes.}
  \end{phonetics}
\end{entry}

\begin{entry}{护士}{7,3}{⼿、⼠}
  \begin{phonetics}{护士}{hu4shi5}[][HSK 4]
    \definition[名,位]{s.}{enfermeiro; pessoas especializadas em enfermagem em hospitais ou instituições epidemiológicas}
  \end{phonetics}
\end{entry}

\begin{entry}{护照}{7,13}{⼿、⽕}
  \begin{phonetics}{护照}{hu4zhao4}[][HSK 2]
    \definition[本,个]{s.}{passaporte}
  \end{phonetics}
\end{entry}

\begin{entry}{报}{7}{⼿}
  \begin{phonetics}{报}{bao4}[][HSK 3]
    \definition[份,张]{s.}{jornal | recompensa | relatório | vingança}
    \definition{v.}{anunciar | informar}
  \end{phonetics}
\end{entry}

\begin{entry}{报名}{7,6}{⼿、⼝}
  \begin{phonetics}{报名}{bao4ming2}[][HSK 2]
    \definition{v.+compl.}{matricular-se | alistar-se | inscrever-se | inserir o nome de alguém}
  \end{phonetics}
\end{entry}

\begin{entry}{报告}{7,7}{⼿、⼝}
  \begin{phonetics}{报告}{bao4gao4}[][HSK 3]
    \definition[份,篇,分,个,通]{s.}{relatório | discurso | palestra | aconselhamento}
    \definition{v.}{relatar | dar a conhecer | informar}
  \end{phonetics}
\end{entry}

\begin{entry}{报纸}{7,7}{⼿、⽷}
  \begin{phonetics}{报纸}{bao4zhi3}[][HSK 2]
    \definition[张]{s.}{jornal | diário}
  \end{phonetics}
\end{entry}

\begin{entry}{报到}{7,8}{⼿、⼑}
  \begin{phonetics}{报到}{bao4dao4}[][HSK 3]
    \definition{v.+compl.}{apresentar-se para o serviço | fazer check-in | registrar-se | assinar}
  \end{phonetics}
\end{entry}

\begin{entry}{报答}{7,12}{⼿、⽵}
  \begin{phonetics}{报答}{bao4da2}[][HSK 5]
    \definition{v.}{reembolsar; devolver; retribuir; pagar de volta; mostrar seu apreço de forma tangível}
  \end{phonetics}
\end{entry}

\begin{entry}{报道}{7,12}{⼿、⾡}
  \begin{phonetics}{报道}{bao4dao4}[][HSK 3]
    \definition[个,篇,分]{s.}{história | reportagem}
    \definition{v.}{cobrir | relatar (notícias)}
  \end{phonetics}
\end{entry}

\begin{entry}{报酬}{7,13}{⼿、⾣}
  \begin{phonetics}{报酬}{bao4chou5}
    \definition{s.}{recompensa | remuneração}
  \end{phonetics}
\end{entry}

\begin{entry}{报警}{7,19}{⼿、⾔}
  \begin{phonetics}{报警}{bao4jing3}[][HSK 5]
    \definition{v.}{relatar (um incidente) à polícia; relatar uma situação crítica ou sinalizar uma emergência às autoridades competentes}
  \end{phonetics}
\end{entry}

\begin{entry}{拒绝}{7,9}{⼿、⽷}
  \begin{phonetics}{拒绝}{ju4jue2}[][HSK 5]
    \definition{v.}{recusar; rejeitar; declinar; não aceitar (pedidos, sugestões ou presentes)}
  \end{phonetics}
\end{entry}

\begin{entry}{承办}{8,4}{⼿、⼒}
  \begin{phonetics}{承办}{cheng2ban4}[][HSK 5]
    \definition{v.}{empreender}
  \end{phonetics}
\end{entry}

\begin{entry}{承认}{8,4}{⼿、⾔}
  \begin{phonetics}{承认}{cheng2ren4}[][HSK 4]
    \definition{s.}{reconhecimento (diplomático, artístico, etc.)}
    \definition{v.}{admitir; reconhecer | dar reconhecimento diplomático; reconhecer}
  \end{phonetics}
\end{entry}

\begin{entry}{承受}{8,8}{⼿、⼜}
  \begin{phonetics}{承受}{cheng2shou4}[][HSK 4]
    \definition{v.}{suportar; resistir; realizar (tarefas, dificuldades, pressões, etc.); submeter-se a (testes, etc.) | herdar}
  \end{phonetics}
\end{entry}

\begin{entry}{承担}{8,8}{⼿、⼿}
  \begin{phonetics}{承担}{cheng2dan1}[][HSK 4]
    \definition{v.}{suportar; empreender; assumir; tomar conta de algo}
  \end{phonetics}
\end{entry}

\begin{entry}{披}{8}{⼿}
  \begin{phonetics}{披}{pi1}[][HSK 5]
    \definition{v.}{colocar sobre os ombros; enrolar em volta; cobrir ou colocar sobre os ombros | abrir; desenrolar; espalhar | abrir-se; rachar}
  \end{phonetics}
\end{entry}

\begin{entry}{抬}{8}{⼿}
  \begin{phonetics}{抬}{tai2}[][HSK 5]
    \definition{clas.}{para objetos que precisam ser carregados por pessoas quando transportados (por exemplo, móveis)}
    \definition{v.}{levantar; elevar; puxar para cima | (por duas ou mais pessoas) carregar; transportar; duas ou mais pessoas carregando algo com as mãos ou nos ombros | discutir, debater (geralmente sem sentido ou sem importância)}
  \end{phonetics}
\end{entry}

\begin{entry}{抬头}{8,5}{⼿、⼤}
  \begin{phonetics}{抬头}{tai2 tou2}[][HSK 5]
    \definition{s.}{(em recibos, contas, etc.) nome do comprador ou beneficiário, ou espaço para preencher esse nome | nome do comprador ou beneficiário; refere-se ao cabeçalho do documento ou da fatura}
    \definition{v.}{levantar a cabeça | ganhar terreno; olhar para cima; subir | começar uma nova linha, como sinal de respeito, ao mencionar o destinatário em cartas, correspondência oficial, etc.}
  \end{phonetics}
\end{entry}

\begin{entry}{抬杠}{8,7}{⼿、⽊}
  \begin{phonetics}{抬杠}{tai2gang4}
    \definition{v.+compl.}{discutir pelo prazer em discutir | discutir obstinadamente | brigar}
  \end{phonetics}
\end{entry}

\begin{entry}{抱}{8}{⼿}
  \begin{phonetics}{抱}{bao4}[][HSK 4]
    \definition*{s.}{sobrenome Bao}
    \definition{clas.}{braçada; medida dos dois braços}
    \definition{v.}{carregar no peito; segurar com ambos os braços; abraçar | ter o primeiro filho ou neto | adotar um bebê ou criança | ficar juntos, unidos | encaixar ou servir perfeitamente (roupas e sapatos do tamanho certo) | estimar; nutrir; abrigar; ter em mente | continuar; sobrecarregar com | chocar ovos}
  \end{phonetics}
\end{entry}

\begin{entry}{抱怨}{8,9}{⼿、⼼}
  \begin{phonetics}{抱怨}{bao4yuan4}[][HSK 5]
    \definition{v.}{reclamar ou expressar descontentamento ou insatisfação; falar com os outros sobre pessoas ou coisas com as quais você não está satisfeito}
  \end{phonetics}
\end{entry}

\begin{entry}{抵抗}{8,7}{⼿、⼿}
  \begin{phonetics}{抵抗}{di3kang4}
    \definition{s.}{resistência}
    \definition{v.}{resistir}
  \end{phonetics}
\end{entry}

\begin{entry}{抹泪}{8,8}{⼿、⽔}
  \begin{phonetics}{抹泪}{mo3lei4}
    \definition{v.}{limpar as lágrimas | (figurativo) derramar lágrimas}
  \end{phonetics}
\end{entry}

\begin{entry}{押}{8}{⼿}
  \begin{phonetics}{押}{ya1}
    \definition{v.}{deter sob custódia | escoltar e proteger | hipotecar | penhorar}
  \end{phonetics}
\end{entry}

\begin{entry}{押后}{8,6}{⼿、⼝}
  \begin{phonetics}{押后}{ya1hou4}
    \definition{v.}{encerrar | adiar}
  \end{phonetics}
\end{entry}

\begin{entry}{押运}{8,7}{⼿、⾡}
  \begin{phonetics}{押运}{ya1yun4}
    \definition{v.}{escoltar sob guarda | escoltar (bens ou fundos)}
  \end{phonetics}
\end{entry}

\begin{entry}{押注}{8,8}{⼿、⽔}
  \begin{phonetics}{押注}{ya1zhu4}
    \definition{v.}{apostar}
  \end{phonetics}
\end{entry}

\begin{entry}{押金}{8,8}{⼿、⾦}
  \begin{phonetics}{押金}{ya1jin1}[][HSK 5]
    \definition[笔,份,些]{s.}{caução; sinal; depósito; dinheiro como garantia}
  \end{phonetics}
\end{entry}

\begin{entry}{押送}{8,9}{⼿、⾡}
  \begin{phonetics}{押送}{ya1song4}
    \definition{v.}{enviar sob escolta | transportar um detido}
  \end{phonetics}
\end{entry}

\begin{entry}{押租}{8,10}{⼿、⽲}
  \begin{phonetics}{押租}{ya1zu1}
    \definition{s.}{depósito de aluguel}
  \end{phonetics}
\end{entry}

\begin{entry}{押韵}{8,13}{⼿、⾳}
  \begin{phonetics}{押韵}{ya1yun4}
    \definition{v.}{rimar}
  \end{phonetics}
\end{entry}

\begin{entry}{抽}{8}{⼿}
  \begin{phonetics}{抽}{chou1}[][HSK 4]
    \definition{v.}{retirar; tirar (do meio); retirar, puxar ou arrancar algo que está preso ou emaranhado em outra coisa | tirar, retirar (uma parte de um todo) | (certas plantas) começar a crescer, produzir | bombear | encolher; contrair |
chicotear; açoitar; surrar | dirigir; conduzir | encontrar tempo; libertar-se; sair de alguma coisa}
  \end{phonetics}
\end{entry}

\begin{entry}{抽奖}{8,9}{⼿、⼤}
  \begin{phonetics}{抽奖}{chou1 jiang3}[][HSK 4]
    \definition{s.}{loteria; sorteio de loteria}
  \end{phonetics}
\end{entry}

\begin{entry}{抽烟}{8,10}{⼿、⽕}
  \begin{phonetics}{抽烟}{chou1yan1}[][HSK 4]
    \definition{v.+compl.}{fumar (um cigarro ou um cachimbo)}
  \end{phonetics}
\end{entry}

\begin{entry}{担心}{8,4}{⼿、⼼}
  \begin{phonetics}{担心}{dan1xin1}[][HSK 4]
    \definition{v.}{preocupar-se; ficar ansioso; sentir-se desconfortável com algo}
  \end{phonetics}
\end{entry}

\begin{entry}{担任}{8,6}{⼿、⼈}
  \begin{phonetics}{担任}{dan1ren4}[][HSK 4]
    \definition{v.}{servir como; assumir o cargo de; ocupar o posto de; ocupar um determinado cargo ou emprego}
  \end{phonetics}
\end{entry}

\begin{entry}{担保}{8,9}{⼿、⼈}
  \begin{phonetics}{担保}{dan1bao3}[][HSK 4]
    \definition{v.}{garantir; atestar; expressar responsabilidade e garantir que não haverá problemas ou que eles serão resolvidos}
  \end{phonetics}
\end{entry}

\begin{entry}{拆}{8}{⼿}
  \begin{phonetics}{拆}{chai1}[][HSK 5]
    \definition{v.}{abrir; desmontar; separar o que está unido | derrubar; desmontar; demolir; referindo-se especificamente à demolição de edifícios}
  \end{phonetics}
\end{entry}

\begin{entry}{拆除}{8,9}{⼿、⾩}
  \begin{phonetics}{拆除}{chai1 chu2}[][HSK 5]
    \definition{v.}{desmontar; demolir; derrubar; remover}
  \end{phonetics}
\end{entry}

\begin{entry}{拉}{8}{⼿}
  \begin{phonetics}{拉}{la1}[][HSK 2]
    \definition{v.}{puxar | arrastar | desenhar | conversar | (coloquial) esvaziar as entranhas}
  \end{phonetics}
  \begin{phonetics}{拉}{la4}
    \definition{s.}{usado em 拉拉蛄 \dpy{la4la4gu3}}
    \seeref{拉拉蛄}{la4la4gu3}
  \end{phonetics}
\end{entry}

\begin{entry}{拉开}{8,4}{⼿、⼶}
  \begin{phonetics}{拉开}{la1 kai1}[][HSK 4]
    \definition{v.}{puxar para abrir; recuar| ampliar; espaçar; distanciar; afastar; separar}
  \end{phonetics}
\end{entry}

\begin{entry}{拉拉队}{8,8,4}{⼿、⼿、⾩}
  \begin{phonetics}{拉拉队}{la1la1dui4}
    \definition{s.}{claque | torcida}
  \end{phonetics}
\end{entry}

\begin{entry}{拉拉蛄}{8,8,11}{⼿、⼿、⾍}
  \begin{phonetics}{拉拉蛄}{la4la4gu3}
    \variantof{蝲蝲蛄}
  \end{phonetics}
\end{entry}

\begin{entry}{拍}{8}{⼿}
  \begin{phonetics}{拍}{pai1}[][HSK 3]
    \definition[只,把]{s.}{bastão; raquete | batida; tempo}
    \definition{v.}{bater palmas; bater; dar um tapa | chicotear; açoitar; bater | enviar (um telegrama, etc.) | tirar (uma foto); fotografar | bajular; lisonjear; adular}
  \end{phonetics}
\end{entry}

\begin{entry}{拍马}{8,3}{⼿、⾺}
  \begin{phonetics}{拍马}{pai1ma3}
    \definition{v.}{instigar um cavalo dando tapinhas em seu traseiro | lisonjear | bajular}
  \seealsoref{拍马屁}{pai1ma3pi4}
  \end{phonetics}
\end{entry}

\begin{entry}{拍马屁}{8,3,7}{⼿、⾺、⼫}
  \begin{phonetics}{拍马屁}{pai1ma3pi4}
    \definition{s.}{puxa-saco | bajulador}
    \definition{v.}{puxar o saco | bajular}
  \seealsoref{拍马}{pai1ma3}
  \end{phonetics}
\end{entry}

\begin{entry}{拍摄}{8,13}{⼿、⼿}
  \begin{phonetics}{拍摄}{pai1 she4}[][HSK 5]
    \definition{s.}{fotografar; tirar (uma foto); usar uma câmera fotográfica para capturar imagens de pessoas e objetos}
  \end{phonetics}
\end{entry}

\begin{entry}{拍照}{8,13}{⼿、⽕}
  \begin{phonetics}{拍照}{pai1 zhao4}[][HSK 4]
    \definition{v.+compl.}{fotografar; tirar uma foto}
  \end{phonetics}
\end{entry}

\begin{entry}{拐}{8}{⼿}
  \begin{phonetics}{拐}{guai3}
    \definition{s.}{bengala | muleta}
    \definition{v.}{virar (uma esquina, etc.) | cortar | sequestrar | fraudar | apropriar-se indevidamente}
  \end{phonetics}
\end{entry}

\begin{entry}{拔}{8}{⼿}
  \begin{phonetics}{拔}{ba2}[][HSK 5]
    \definition{v.aux.}{puxar para cima; puxar para fora; arrastar para fora |extrair; sugar | escolher; selecionar | superar; destacar-se entre | apreender; capturar | esfriar na água; mergulhar algo em água fria para que esfrie}
  \end{phonetics}
\end{entry}

\begin{entry}{拔尖}{8,6}{⼿、⼩}
  \begin{phonetics}{拔尖}{ba2jian1}
    \definition{adj.}{topo de linha | fora do comum | o melhor}
    \definition{v.+compl.}{empurrar-se para a frente | sentir que é superior aos outros}
  \end{phonetics}
\end{entry}

\begin{entry}{拖拉机}{8,8,6}{⼿、⼿、⽊}
  \begin{phonetics}{拖拉机}{tuo1la1ji1}
    \definition[台]{s.}{trator}
  \end{phonetics}
\end{entry}

\begin{entry}{拖鞋}{8,15}{⼿、⾰}
  \begin{phonetics}{拖鞋}{tuo1xie2}
    \definition[双,只]{s.}{chinelos | sandálias}
  \end{phonetics}
\end{entry}

\begin{entry}{招}{8}{⼿}
  \begin{phonetics}{招}{zhao1}
    \definition{adj.}{contagioso}
    \definition{s.}{um movimento (xadrez) | uma manobra | dispositivo | truque}
    \definition{v.}{recrutar | provocar | acenar | incorrer | infectar | confessar}
  \end{phonetics}
\end{entry}

\begin{entry}{招手}{8,4}{⼿、⼿}
  \begin{phonetics}{招手}{zhao1 shou3}[][HSK 5]
    \definition{v.+compl.}{acenar; chamar a atenção; levantar a mão e acenar com a palma, para indicar que a outra pessoa se aproxime ou para cumprimentá-la}
  \end{phonetics}
\end{entry}

\begin{entry}{招生}{8,5}{⼿、⽣}
  \begin{phonetics}{招生}{zhao1 sheng1}[][HSK 5]
    \definition{v.+compl.}{conseguir alunos; matricular novos alunos; recrutar novos alunos}
  \end{phonetics}
\end{entry}

\begin{entry}{招呼}{8,8}{⼿、⼝}
  \begin{phonetics}{招呼}{zhao1 hu5}[][HSK 4]
    \definition{v.}{chamar; chamar a atenção com palavras ou gestos | cumprimentar; saudar; cumprimentar ou despedir-se das pessoas com palavras ou gestos | pedir a alguém para fazer algo; fazer solicitações, pedir ajuda ou fazer coisas | receber e dar boas-vindas aos convidados}
  \end{phonetics}
\end{entry}

\begin{entry}{招数}{8,13}{⼿、⽁}
  \begin{phonetics}{招数}{zhao1shu4}
    \definition{s.}{estratégia | movimento (no xadrez, no palco, nas artes marciais) | esquema | truque}
  \end{phonetics}
\end{entry}

\begin{entry}{拥有}{8,6}{⼿、⽉}
  \begin{phonetics}{拥有}{yong1you3}[][HSK 5]
    \definition{v.}{possuir; deter; ter (grande quantidade de terras, população, bens, etc.)}
  \end{phonetics}
\end{entry}

\begin{entry}{拥抱}{8,8}{⼿、⼿}
  \begin{phonetics}{拥抱}{yong1bao4}[][HSK 5]
    \definition[个,次]{s.}{abraço;}
    \definition{v.}{abraçar; segurar em seus braços; abraçar para demonstrar afeto}
  \end{phonetics}
\end{entry}

\begin{entry}{拧开}{8,4}{⼿、⼶}
  \begin{phonetics}{拧开}{ning3kai1}
    \definition{v.}{desaparafusar | desatarrachar | torcer (uma tampa) | abrir (uma torneira) | ligar (girando um botão) | girar (maçaneta da porta)}
  \end{phonetics}
\end{entry}

\begin{entry}{拨转}{8,8}{⼿、⾞}
  \begin{phonetics}{拨转}{bo1zhuan3}
    \definition{v.}{transferir (fundos, etc.) | virar | dar a volta}
  \end{phonetics}
\end{entry}

\begin{entry}{拜访}{9,6}{⼿、⾔}
  \begin{phonetics}{拜访}{bai4fang3}[][HSK 5]
    \definition{v.}{visitar; fazer uma visita (respeitosamente)}
  \end{phonetics}
\end{entry}

\begin{entry}{括号}{9,5}{⼿、⼝}
  \begin{phonetics}{括号}{kuo4 hao4}[][HSK 4]
    \definition{s.}{chaves, colchetes e parênteses (em fórmulas aritméticas ou algébricas, os símbolos que indicam a combinação e a ordem de vários números ou termos) | colchetes e parênteses usados como um tipo de sinal de pontuação para mostrar a parte explicativa de uma passagem em um texto}
  \end{phonetics}
\end{entry}

\begin{entry}{拼}{9}{⼿}
  \begin{phonetics}{拼}{pin1}[][HSK 5]
    \definition{v.}{montar; juntar as peças | dar tudo de si no trabalho; estar disposto a arriscar a vida (em lutas, no trabalho, etc.); fazer tudo o que for preciso; arriscar tudo}
  \end{phonetics}
\end{entry}

\begin{entry}{拼命}{9,8}{⼿、⼝}
  \begin{phonetics}{拼命}{pin1ming4}
    \definition{adv.}{com toda a força | desesperadamente}
    \definition{v.+compl.}{arriscar a vida de alguém | desafiar a morte | colocar-se em uma luta desesperada | fazer algo desesperadamente | exercer a maior força}
  \end{phonetics}
\end{entry}

\begin{entry}{拼音}{9,9}{⼿、⾳}
  \begin{phonetics}{拼音}{pin1yin1}
    \definition{s.}{escrita fonética | pinyin (romanização chinesa)}
  \end{phonetics}
\end{entry}

\begin{entry}{拾}{9}{⼿}
  \begin{phonetics}{拾}{shi2}[][HSK 5]
    \definition{num.}{dez (usado no lugar do numeral 十 em cheques, notas bancárias, etc., para evitar erros ou alterações)}
    \definition{v.}{pegar (do chão); recolher}
  \end{phonetics}
\end{entry}

\begin{entry}{持续}{9,11}{⼿、⽷}
  \begin{phonetics}{持续}{chi2xu4}[][HSK 3]
    \definition{v.}{durar; continuar; sustentar}
  \end{phonetics}
\end{entry}

\begin{entry}{挂}{9}{⼿}
  \begin{phonetics}{挂}{gua4}[][HSK 3]
    \definition{clas.}{para conjuntos ou sequência de itens}
    \definition{v.}{pendurar; colocar; suspender | interromper chamada (telefônica) | colocar alguém em contato com; ligar; telefonar
pegar carona; ser pego | ter em mente; estar preocupado com | ser revestido com; ser coberto com | colocar em registro; registrar}
  \end{phonetics}
\end{entry}

\begin{entry}{挂号}{9,5}{⼿、⼝}
  \begin{phonetics}{挂号}{gua4hao4}
    \definition{v.+compl.}{registrar-se (em um hospital, etc.) | enviar através de carta registrada}
  \end{phonetics}
\end{entry}

\begin{entry}{挂号信}{9,5,9}{⼿、⼝、⼈}
  \begin{phonetics}{挂号信}{gua4hao4xin4}
    \definition{s.}{carta registrada}
  \end{phonetics}
\end{entry}

\begin{entry}{指}{9}{⼿}
  \begin{phonetics}{指}{zhi3}[][HSK 3]
    \definition*{s.}{sobrenome Zhi}
    \definition{clas.}{dígito; largura do dedo; a largura de um dedo é chamada de ``一指'', que é usado para medir profundidade, largura, etc.}
    \definition{s.}{dedo}
    \definition{v.}{apontar para | (pelo) eriçar | indicar; mostrar-se; apontar; demonstrar | referir-se a; dirigir-se a | confiar em; contar com; depender de | criticar; repreender}
  \end{phonetics}
\end{entry}

\begin{entry}{指出}{9,5}{⼿、⼐}
  \begin{phonetics}{指出}{zhi3 chu1}[][HSK 3]
    \definition{v.}{apontar; indicar}
  \end{phonetics}
\end{entry}

\begin{entry}{指甲}{9,5}{⼿、⽥}
  \begin{phonetics}{指甲}{zhi3jia5}[][HSK 5]
    \definition{s.}{unha; unha de agulha; unha de dedo; camada córnea na ponta dos dedos}
  \end{phonetics}
\end{entry}

\begin{entry}{指示}{9,5}{⼿、⽰}
  \begin{phonetics}{指示}{zhi3shi4}[][HSK 5]
    \definition{s.}{diretriz; instruções; para orientar o trabalho, os superiores emitem opiniões verbais ou escritas aos subordinados}
    \definition{v.}{indicar; apontar; apontar para alguém | instruir; superiores emitem opiniões verbais ou escritas para orientar o trabalho dos subordinados}
  \end{phonetics}
\end{entry}

\begin{entry}{指导}{9,6}{⼿、⼨}
  \begin{phonetics}{指导}{zhi3dao3}[][HSK 3]
    \definition{s.}{guia; pessoa que faz trabalho de orientação}
    \definition{v.}{guiar; dirigir; instruir}
  \end{phonetics}
\end{entry}

\begin{entry}{指责}{9,8}{⼿、⾙}
  \begin{phonetics}{指责}{zhi3ze2}[][HSK 5]
    \definition{v.}{censurar; criticar; encontrar falhas; repreender}
  \end{phonetics}
\end{entry}

\begin{entry}{指南针}{9,9,7}{⼿、⼗、⾦}
  \begin{phonetics}{指南针}{zhi3nan2zhen1}
    \definition{s.}{bússola}
  \end{phonetics}
\end{entry}

\begin{entry}{指挥}{9,9}{⼿、⼿}
  \begin{phonetics}{指挥}{zhi3hui1}[][HSK 4]
    \definition[个]{s.}{diretor; comandante; despachante; operador | maestro; condutor; pessoa na frente de uma orquestra ou coro que dá instruções sobre como tocar ou cantar}
    \definition{v.}{dirigir; conduzir; comandar; direcionar; emitir ordens de agendamento}
  \end{phonetics}
\end{entry}

\begin{entry}{指标}{9,9}{⼿、⽊}
  \begin{phonetics}{指标}{zhi3biao1}[][HSK 5]
    \definition{s.}{meta; cota; norma; índice; objetivos a serem alcançados | alvo; índice; refletir os requisitos de desenvolvimento em determinados aspectos através de números absolutos ou percentagens de aumento ou diminuição, inclui indicadores quantitativos e qualitativos}
  \end{phonetics}
\end{entry}

\begin{entry}{按}{9}{⼿}
  \begin{phonetics}{按}{an4}[][HSK 3]
    \definition{v.}{pressionar | empurrar para baixo | deixar de lado | arquivar | restringir | controlar}
  \end{phonetics}
\end{entry}

\begin{entry}{按时}{9,7}{⼿、⽇}
  \begin{phonetics}{按时}{an4shi2}[][HSK 4]
    \definition{adv.}{na hora; no horário; pontualmente; de acordo com o tempo estipulado}
  \end{phonetics}
\end{entry}

\begin{entry}{按照}{9,13}{⼿、⽕}
  \begin{phonetics}{按照}{an4zhao4}[][HSK 3]
    \definition{prep.}{de acordo com; em conformidade com; à luz de; com base em}
  \end{phonetics}
\end{entry}

\begin{entry}{按摩}{9,15}{⼿、⼿}
  \begin{phonetics}{按摩}{an4mo2}[][HSK 5]
    \definition{s.}{massagem; empurrar, pressionar, beliscar e amassar o corpo de uma pessoa com as mãos para promover a circulação sanguínea, aumentar a resistência da pele e regular a função dos nervos}
  \end{phonetics}
\end{entry}

\begin{entry}{挑}{9}{⼿}
  \begin{phonetics}{挑}{tiao1}[][HSK 4]
    \definition{clas.}{para coisas que são escolhidas ou selecionadas | para coisas que podem ser usadas como palhetas}
    \definition{s.}{vara comprida com algo pendurado nas pontas; haste de transporte}
    \definition{v.}{escolher; selecionar | fazer picuinhas; ser hipercrítico; ser meticuloso; ser excessivamente rigoroso nos detalhes | carregar com uma haste de transporte; carregar no ombro; pendurar coisas nas pontas de varas longas e carregá-las em seus ombros}
  \end{phonetics}
  \begin{phonetics}{挑}{tiao3}[][HSK 4]
    \definition{s.}{um dos traços dos caracteres chineses; inclinado para cima da esquerda para a direita}
    \definition{v.}{levantar; elevar; erguer | levantar ou apoiar com uma extremidade de uma vara ou objeto semelhante; segurar ou apoiar com a ponta de uma vara etc. | causar conflitos deliberadamente; provocar deliberadamente um conflito | (método de bordado) usar uma agulha para levantar os fios de urdidura ou trama, com a agulha e a linha passando por baixo para formar padrões e desenhos}
  \end{phonetics}
\end{entry}

\begin{entry}{挑战}{9,9}{⼿、⼽}
  \begin{phonetics}{挑战}{tiao3zhan4}[][HSK 4]
    \definition{v.}{desafiar; deixar um oponente deliberadamente irritado e sair para lutar ou lutar consigo mesmo; estimular um oponente a lutar consigo mesmo}
  \end{phonetics}
\end{entry}

\begin{entry}{挑选}{9,9}{⼿、⾡}
  \begin{phonetics}{挑选}{tiao1 xuan3}[][HSK 4]
    \definition{v.}{escolher; optar; selecionar; escolher a pessoa ou coisa certa para o trabalho}
  \end{phonetics}
\end{entry}

\begin{entry}{挑衅}{9,11}{⼿、⾎}
  \begin{phonetics}{挑衅}{tiao3xin4}
    \definition{s.}{provocação}
    \definition{v.}{provocar}
  \end{phonetics}
\end{entry}

\begin{entry}{挖}{9}{⼿}
  \begin{phonetics}{挖}{wa1}
    \definition{v.}{cavar | escavar}
  \end{phonetics}
\end{entry}

\begin{entry}{挖掘机}{9,11,6}{⼿、⼿、⽊}
  \begin{phonetics}{挖掘机}{wa1jue2ji1}
    \definition{s.}{escavadeira | escavador | escavadora | pá mecânica}
  \end{phonetics}
\end{entry}

\begin{entry}{挡}{9}{⼿}
  \begin{phonetics}{挡}{dang3}[][HSK 5]
    \definition{s.}{persiana; veneziana; paralama; coisas para cobrir ou bloquear | caixa de câmbio (automóvel)}
    \definition{v.}{bloquear; resistir; manter afastado; afastar | cobrir; bloquear; atrapalhar}
  \end{phonetics}
  \begin{phonetics}{挡}{dang4}
    \definition{v.}{organizar}
  \end{phonetics}
\end{entry}

\begin{entry}{挡风玻璃}{9,4,9,14}{⼿、⾵、⽟、⽟}
  \begin{phonetics}{挡风玻璃}{dang3feng1bo1li5}
    \definition{s.}{parabrisa}
  \end{phonetics}
\end{entry}

\begin{entry}{挣}{9}{⼿}
  \begin{phonetics}{挣}{zheng4}[][HSK 5]
    \definition{v.}{empurrar e puxar; tentar se livrar; lutar para se libertar; esforçar-se para se libertar das amarras | ganhar; fazer; trabalhar para; trocar trabalho por}
  \end{phonetics}
\end{entry}

\begin{entry}{挣扎}{9,4}{⼿、⼿}
  \begin{phonetics}{挣扎}{zheng1zha2}
    \definition{v.}{lutar}
  \end{phonetics}
\end{entry}

\begin{entry}{挣钱}{9,10}{⼿、⾦}
  \begin{phonetics}{挣钱}{zheng4qian2}[][HSK 5]
    \definition{v.+compl.}{ganhar dinheiro; fazer dinheiro; lucrar; trabalhar para ganhar dinheiro}
  \end{phonetics}
\end{entry}

\begin{entry}{挣得}{9,11}{⼿、⼻}
  \begin{phonetics}{挣得}{zheng4de2}
    \definition{v.}{ganhar renda ou dinheiro}
  \end{phonetics}
\end{entry}

\begin{entry}{挤}{9}{⼿}
  \begin{phonetics}{挤}{ji3}[][HSK 5]
    \definition{adj.}{lotado; congestionado; descreve um grande número de pessoas ou coisas e muito pouco espaço}
    \definition{v.}{empacotar; amontoar; aglomerar | sacudir; empurrar contra; empurrar alguém ou algo para longe com seu corpo com toda a força que puder| pressionar; apertar; expulsar por pressão}
  \end{phonetics}
\end{entry}

\begin{entry}{挥汗如雨}{9,6,6,8}{⼿、⽔、⼥、⾬}
  \begin{phonetics}{挥汗如雨}{hui1han4ru2yu3}
    \definition{s.}{suor derramado}
    \definition{v.}{pingar com suor}
  \end{phonetics}
\end{entry}

\begin{entry}{挺}{9}{⼿}
  \begin{phonetics}{挺}{ting3}[][HSK 2,4]
    \definition{adj.}{rígido; ereto; vertical; reto | distinto. que se destaca; que se sobressai; excepcional}
    \definition{adv.}{muito; bastante}
    \definition{clas.}{para metralhadoras}
    \definition{v.}{sobressair; endireitar-se; saliente ou protuberante | suportar; aguentar; ficar de pé; resistir}
  \end{phonetics}
\end{entry}

\begin{entry}{挺尸}{9,3}{⼿、⼫}
  \begin{phonetics}{挺尸}{ting3shi1}
    \definition{v.}{(coloquial) dormir | (literalmente) ficar deitado duro como um cadáver}
  \end{phonetics}
\end{entry}

\begin{entry}{挺立}{9,5}{⼿、⽴}
  \begin{phonetics}{挺立}{ting3li4}
    \definition{v.}{ficar ereto | ficar de pé}
  \end{phonetics}
\end{entry}

\begin{entry}{挺好}{9,6}{⼿、⼥}
  \begin{phonetics}{挺好}{ting3 hao3}[][HSK 2]
    \definition{adj.}{muito bom}
  \end{phonetics}
\end{entry}

\begin{entry}{挺过}{9,6}{⼿、⾡}
  \begin{phonetics}{挺过}{ting3guo4}
    \definition{s.}{sobreviver}
  \end{phonetics}
\end{entry}

\begin{entry}{挺住}{9,7}{⼿、⼈}
  \begin{phonetics}{挺住}{ting3zhu4}
    \definition{v.}{permanecer firme | manter-se firme (diante da adversidade ou da dor)}
  \end{phonetics}
\end{entry}

\begin{entry}{挺杆}{9,7}{⼿、⽊}
  \begin{phonetics}{挺杆}{ting3gan3}
    \definition{s.}{tucho (peça de máquina)}
  \end{phonetics}
\end{entry}

\begin{entry}{挺身}{9,7}{⼿、⾝}
  \begin{phonetics}{挺身}{ting3shen1}
    \definition{v.}{endireitar as costas}
  \end{phonetics}
\end{entry}

\begin{entry}{挺进}{9,7}{⼿、⾡}
  \begin{phonetics}{挺进}{ting3jin4}
    \definition{s.}{progresso | avanço}
    \definition{v.}{progredir | avançar}
  \end{phonetics}
\end{entry}

\begin{entry}{挺拔}{9,8}{⼿、⼿}
  \begin{phonetics}{挺拔}{ting3ba2}
    \definition{adj.}{alto e reto}
  \end{phonetics}
\end{entry}

\begin{entry}{挺腰}{9,13}{⼿、⾁}
  \begin{phonetics}{挺腰}{ting3yao1}
    \definition{v.}{arquear as costas | endireitar as costas}
  \end{phonetics}
\end{entry}

\begin{entry}{拳王}{10,4}{⼿、⽟}
  \begin{phonetics}{拳王}{quan2wang2}
    \definition{s.}{pugilista | boxeador}
  \end{phonetics}
\end{entry}

\begin{entry}{拳法}{10,8}{⼿、⽔}
  \begin{phonetics}{拳法}{quan2fa3}
    \definition{s.}{boxe | luta}
  \end{phonetics}
\end{entry}

\begin{entry}{拿}{10}{⼿}
  \begin{phonetics}{拿}{na2}[][HSK 1]
    \definition{part.}{usado da mesma forma que 把: para marcar o seguinte substantivo seguinte como objeto direto}
    \definition{v.}{segurar | tomar | pegar em}
  \end{phonetics}
\end{entry}

\begin{entry}{拿出}{10,5}{⼿、⼐}
  \begin{phonetics}{拿出}{na2 chu1}[][HSK 2]
    \definition{v.}{apresentar (evidências) | prover | apresentar (uma proposta) | colocar para fora | retirar}
  \end{phonetics}
\end{entry}

\begin{entry}{拿到}{10,8}{⼿、⼑}
  \begin{phonetics}{拿到}{na2 dao4}[][HSK 2]
    \definition{v.}{pegar | obter}
  \end{phonetics}
\end{entry}

\begin{entry}{挫折}{10,7}{⼿、⼿}
  \begin{phonetics}{挫折}{cuo4zhe2}
    \definition{s.}{revés | reverso | derrota | frustração | decepção}
    \definition{v.}{frustrar | desencorajar | subjugar}
  \end{phonetics}
\end{entry}

\begin{entry}{振动}{10,6}{⼿、⼒}
  \begin{phonetics}{振动}{zhen4dong4}[][HSK 5]
    \definition{s.}{vibração}
    \definition{v.}{sacudir; balançar; tremer; roncar; tagarelar; vibrar; oscilar; a física se refere ao movimento contínuo de um objeto em torno de um determinado ponto no espaço, como o movimento de um pêndulo, um diapasão ou uma corda de violão}
  \end{phonetics}
\end{entry}

\begin{entry}{捞}{10}{⼿}
  \begin{phonetics}{捞}{lao1}
    \definition{v.}{pescar | dragar}
  \end{phonetics}
\end{entry}

\begin{entry}{损失}{10,5}{⼿、⼤}
  \begin{phonetics}{损失}{sun3shi1}[][HSK 5]
    \definition{s.}{perda; desperdício; algo que se consome ou se perde sem custo algum}
    \definition{v.}{perder; consumir ou perder}
  \end{phonetics}
\end{entry}

\begin{entry}{损害}{10,10}{⼿、⼧}
  \begin{phonetics}{损害}{sun3 hai4}[][HSK 5]
    \definition{v.}{prejudicar; danificar; causar danos}
  \end{phonetics}
\end{entry}

\begin{entry}{捡}{10}{⼿}
  \begin{phonetics}{捡}{jian3}
    \definition{v.}{apanhar | recolher | coletar}
  \end{phonetics}
\end{entry}

\begin{entry}{换}{10}{⼿}
  \begin{phonetics}{换}{huan4}[][HSK 2]
    \definition{v.}{mudar | trocar | substituir | converter (moedas)}
  \end{phonetics}
\end{entry}

\begin{entry}{换钱}{10,10}{⼿、⾦}
  \begin{phonetics}{换钱}{huan4qian2}
    \definition{v.+compl.}{trocar dinheiro (em pequenas valores ou em outra moeda) | trocar (mercadorias) por dinheiro | vender}
  \end{phonetics}
\end{entry}

\begin{entry}{据说}{11,9}{⼿、⾔}
  \begin{phonetics}{据说}{ju4shuo1}[][HSK 3]
    \definition{v.}{ser dito; ser relatado}
  \end{phonetics}
\end{entry}

\begin{entry}{捷径}{11,8}{⼿、⼻}
  \begin{phonetics}{捷径}{jie2jing4}
    \definition{s.}{atalho}
  \end{phonetics}
\end{entry}

\begin{entry}{掉}{11}{⼿}
  \begin{phonetics}{掉}{diao4}[][HSK 2]
    \definition{v.}{cair | deixar cair}
  \end{phonetics}
\end{entry}

\begin{entry}{掉队}{11,4}{⼿、⾩}
  \begin{phonetics}{掉队}{diao4dui4}
    \definition{v.}{abandonar | ficar para trás}
  \end{phonetics}
\end{entry}

\begin{entry}{掉包}{11,5}{⼿、⼓}
  \begin{phonetics}{掉包}{diao4bao1}
    \definition{v.}{vender uma falsificação pelo artigo genuíno | roubar o item valioso de alguém e substituí-lo por um item de aparência semelhante, mas sem valor}
  \end{phonetics}
\end{entry}

\begin{entry}{掉线}{11,8}{⼿、⽷}
  \begin{phonetics}{掉线}{diao4xian4}
    \definition{v.}{desconectar-se (da \emph{Internet})}
  \end{phonetics}
\end{entry}

\begin{entry}{掉转}{11,8}{⼿、⾞}
  \begin{phonetics}{掉转}{diao4zhuan3}
    \definition{v.}{dar a volta}
  \end{phonetics}
\end{entry}

\begin{entry}{掉落}{11,12}{⼿、⾋}
  \begin{phonetics}{掉落}{diao4luo4}
    \definition{v.}{derrubar}
  \end{phonetics}
\end{entry}

\begin{entry}{掉膘}{11,15}{⼿、⾁}
  \begin{phonetics}{掉膘}{diao4biao1}
    \definition{v.}{perder peso (gado)}
  \end{phonetics}
\end{entry}

\begin{entry}{排}{11}{⼿}
  \begin{phonetics}{排}{pai2}[][HSK 2,3]
    \definition{clas.}{para linhas}
    \definition{s.}{linha | pelotão | jangada; balsa | torta}
    \definition{v.}{organizar; colocar em ordem | ensaiar | excluir; ejetar; descarregar | empurrar}
  \end{phonetics}
\end{entry}

\begin{entry}{排水}{11,4}{⼿、⽔}
  \begin{phonetics}{排水}{pai2shui3}
    \definition{v.}{drenar}
  \end{phonetics}
\end{entry}

\begin{entry}{排队}{11,4}{⼿、⾩}
  \begin{phonetics}{排队}{pai2dui4}[][HSK 2]
    \definition{v.+compl.}{formar uma fila | alinhar | listar | classificar}
  \end{phonetics}
\end{entry}

\begin{entry}{排列}{11,6}{⼿、⼑}
  \begin{phonetics}{排列}{pai2lie4}[][HSK 4]
    \definition{v.}{classificar; colocar; variar; organizar; pôr em ordem}
  \end{phonetics}
\end{entry}

\begin{entry}{排名}{11,6}{⼿、⼝}
  \begin{phonetics}{排名}{pai2 ming2}[][HSK 3]
    \definition{s.}{classificação; resultado}
  \end{phonetics}
\end{entry}

\begin{entry}{排除}{11,9}{⼿、⾩}
  \begin{phonetics}{排除}{pai2chu2}[][HSK 5]
    \definition{v.}{remover; superar; excluir; eliminar; livrar-se de}
  \end{phonetics}
\end{entry}

\begin{entry}{排球}{11,11}{⼿、⽟}
  \begin{phonetics}{排球}{pai2 qiu2}[][HSK 2]
    \definition[个]{s.}{voleibol}
  \end{phonetics}
\end{entry}

\begin{entry}{探亲}{11,9}{⼿、⼇}
  \begin{phonetics}{探亲}{tan4qin1}
    \definition{v.+compl.}{ir para casa para visitar a família}
  \end{phonetics}
\end{entry}

\begin{entry}{接}{11}{⼿}
  \begin{phonetics}{接}{jie1}[][HSK 2]
    \definition{v.}{ir buscar (alguém) |  ir ao encontro de (alguém) | receber}
  \end{phonetics}
\end{entry}

\begin{entry}{接下来}{11,3,7}{⼿、⼀、⽊}
  \begin{phonetics}{接下来}{jie1 xia4 lai2}[][HSK 2]
    \definition{expr.}{próximo | seguinte | aceitar}
  \end{phonetics}
\end{entry}

\begin{entry}{接(电话)}{11,5,8}{⼿、⽥、⾔}
  \begin{phonetics}{接(电话)}{jie1(dian4hua4)}
    \definition{v.}{atender (o telefone) | receber (uma ligação telefônica)}
  \end{phonetics}
\end{entry}

\begin{entry}{接近}{11,7}{⼿、⾡}
  \begin{phonetics}{接近}{jie1jin4}[][HSK 3]
    \definition{adj.}{perto; próximo}
    \definition{v.}{estar perto de; aproximar; aproximar-se}
  \end{phonetics}
\end{entry}

\begin{entry}{接连}{11,7}{⼿、⾡}
  \begin{phonetics}{接连}{jie1lian2}[][HSK 5]
    \definition{adv.}{no final; em sucessão; em uma fileira; um após o outro; seguindo o anterior; continuando}
  \end{phonetics}
\end{entry}

\begin{entry}{接到}{11,8}{⼿、⼑}
  \begin{phonetics}{接到}{jie1 dao4}[][HSK 2]
    \definition{v.}{receber (carta, etc.)}
  \end{phonetics}
\end{entry}

\begin{entry}{接受}{11,8}{⼿、⼜}
  \begin{phonetics}{接受}{jie1shou4}[][HSK 2]
    \definition{v.}{aceitar | concordar}
  \end{phonetics}
\end{entry}

\begin{entry}{接待}{11,9}{⼿、⼻}
  \begin{phonetics}{接待}{jie1dai4}[][HSK 3]
    \definition{v.}{receber (alguém); acolher; recepcionar}
  \end{phonetics}
\end{entry}

\begin{entry}{接班人}{11,10,2}{⼿、⽟、⼈}
  \begin{phonetics}{接班人}{jie1ban1ren2}
    \definition{s.}{sucessor}
  \end{phonetics}
\end{entry}

\begin{entry}{接着}{11,11}{⼿、⽬}
  \begin{phonetics}{接着}{jie1zhe5}[][HSK 2]
    \definition{adv.}{por sua vez | com um seguindo o outro}
    \definition{v.}{seguir | prosseguir | continuar | prosseguir | pegar}
  \end{phonetics}
\end{entry}

\begin{entry}{接触}{11,13}{⼿、⾓}
  \begin{phonetics}{接触}{jie1chu4}[][HSK 5]
    \definition{v.}{entrar em contato com | entrar em contato; tocar; interagir | engajar; o termo militar refere-se a fogo cruzado}
  \end{phonetics}
\end{entry}

\begin{entry}{控制}{11,8}{⼿、⼑}
  \begin{phonetics}{控制}{kong4zhi4}[][HSK 5]
    \definition{v.}{controlar; restringir; dominar; fazer com que não ultrapasse um determinado limite | controlar; dominar; comandar; ocupar, fazer com que não se perca}
  \end{phonetics}
\end{entry}

\begin{entry}{推}{11}{⼿}
  \begin{phonetics}{推}{tui1}[][HSK 2]
    \definition{v.}{empurrar | girar um moinho ou uma pedra de amolar | moer | impulsionar | promover | avançar | estender | deduzir | inferir | declinar | empurrar para longe | deslocar | adiar | diferir | eleger | selecionar | escolher | ter em alta estima | elogiar muito}
  \end{phonetics}
\end{entry}

\begin{entry}{推广}{11,3}{⼿、⼴}
  \begin{phonetics}{推广}{tui1guang3}[][HSK 3]
    \definition{s.}{extensão}
    \definition{v.}{espalhar; estender; promover; popularizar}
  \end{phonetics}
\end{entry}

\begin{entry}{推介}{11,4}{⼿、⼈}
  \begin{phonetics}{推介}{tui1jie4}
    \definition{s.}{promoção}
    \definition{v.}{promover | introduzir e recomendar}
  \end{phonetics}
\end{entry}

\begin{entry}{推开}{11,4}{⼿、⼶}
  \begin{phonetics}{推开}{tui1 kai1}[][HSK 3]
    \definition{v.}{declinar; rejeitar | empurrar para longe | empurrar para abrir (um portão, etc.) | estender; popularizar}
  \end{phonetics}
\end{entry}

\begin{entry}{推动}{11,6}{⼿、⼒}
  \begin{phonetics}{推动}{tui1 dong4}[][HSK 3]
    \definition{v.}{promover; atuar; impulsionar; empurrar para a frente; dar ímpeto a}
  \end{phonetics}
\end{entry}

\begin{entry}{推行}{11,6}{⼿、⾏}
  \begin{phonetics}{推行}{tui1 xing2}[][HSK 5]
    \definition{v.}{realizar; prosseguir; praticar | implementar; praticar; implementação generalizada; divulgar (experiências, métodos, etc.)}
  \end{phonetics}
\end{entry}

\begin{entry}{推进}{11,7}{⼿、⾡}
  \begin{phonetics}{推进}{tui1 jin4}[][HSK 3]
    \definition{v.}{avançar; empurrar; levar adiante; dar ímpeto a | empurrar; dirigir; avançar; seguir em frente; seguir em frente; pressionar para frente}
  \end{phonetics}
\end{entry}

\begin{entry}{推迟}{11,7}{⼿、⾡}
  \begin{phonetics}{推迟}{tui1chi2}[][HSK 4]
    \definition{v.}{adiar; postergar; tardar; deixar para mais tarde}
  \end{phonetics}
\end{entry}

\begin{entry}{推销}{11,12}{⼿、⾦}
  \begin{phonetics}{推销}{tui1xiao1}[][HSK 4]
    \definition{v.}{vender; comercializar; promover vendas; promover a comercialização de mercadorias}
  \end{phonetics}
\end{entry}

\begin{entry}{措施}{11,9}{⼿、⽅}
  \begin{phonetics}{措施}{cuo4shi1}[][HSK 4]
    \definition{s.}{medida; etapa; passo; abordagem adotada para lidar com as coisas}
  \end{phonetics}
\end{entry}

\begin{entry}{描写}{11,5}{⼿、⼍}
  \begin{phonetics}{描写}{miao2xie3}[][HSK 4]
    \definition{v.}{representar; retratar; descrever; usar a linguagem e as palavras para transmitir uma imagem concreta de uma pessoa, evento ou situação}
  \end{phonetics}
\end{entry}

\begin{entry}{描述}{11,8}{⼿、⾡}
  \begin{phonetics}{描述}{miao2 shu4}[][HSK 4]
    \definition[段,种]{s.}{descrição; trecho que descreve um evento ou uma cena}
    \definition{v.}{descrever; representar}
  \end{phonetics}
\end{entry}

\begin{entry}{掌}{12}{⼿}
  \begin{phonetics}{掌}{zhang3}
    \definition{s.}{palma da mão | sola do pé | pata | ferradura}
    \definition{v.}{dar um tapa | segurar na mão | empunhar}
  \end{phonetics}
\end{entry}

\begin{entry}{掌握}{12,12}{⼿、⼿}
  \begin{phonetics}{掌握}{zhang3wo4}[][HSK 5]
    \definition{v.}{compreender; dominar; conhecer bem; compreender as coisas; ser capaz de dominar ou utilizar plenamente | segurar; controlar; ter em mãos; tomar nas mãos}
  \end{phonetics}
\end{entry}

\begin{entry}{掱}{12}{⼿}
  \begin{phonetics}{掱}{shou3}
    \variantof{手}
  \end{phonetics}
\end{entry}

\begin{entry}{揉}{12}{⼿}
  \begin{phonetics}{揉}{rou2}
    \definition{v.}{amassar | massagear | esfregar}
  \end{phonetics}
\end{entry}

\begin{entry}{揉碎}{12,13}{⼿、⽯}
  \begin{phonetics}{揉碎}{rou2sui4}
    \definition{v.}{esmagar | desintegrar-se em pedaços}
  \end{phonetics}
\end{entry}

\begin{entry}{提}{12}{⼿}
  \begin{phonetics}{提}{ti2}[][HSK 2]
    \definition*{s.}{sobrenome Ti}
    \definition{s.}{concha | traço ascendente (em caracteres chineses)}
    \definition{v.}{carregar (na mão com o braço para baixo) | levantar | elevar | promover | avançar | mudar para um momento anterior | mover uma data para a frente | trazer à tona | apresentar | extrair | tirar | trazer | entregar | mencionar | referir-se a}
  \end{phonetics}
\end{entry}

\begin{entry}{提及}{12,3}{⼿、⼃}
  \begin{phonetics}{提及}{ti2ji2}
    \definition{v.}{mencionar | levantar (um assunto) | chamar a atenção de alguém}
  \end{phonetics}
\end{entry}

\begin{entry}{提升}{12,4}{⼿、⼗}
  \begin{phonetics}{提升}{ti2sheng1}
    \definition{v.}{promover (para uma posição de classificação mais alta) | levantar | içar | (figurativo) elevar, levantar, melhorar}
  \end{phonetics}
\end{entry}

\begin{entry}{提出}{12,5}{⼿、⼐}
  \begin{phonetics}{提出}{ti2 chu1}[][HSK 2]
    \definition{v.}{levantar | propor | expor | apresentar}
  \end{phonetics}
\end{entry}

\begin{entry}{提示}{12,5}{⼿、⽰}
  \begin{phonetics}{提示}{ti2shi4}[][HSK 5]
    \definition[个]{s.}{dica; lembrete; pistas ou informações fornecidas para chamar a atenção, fazer com que a outra pessoa pense ou compreenda}
    \definition{v.}{solicitar; lembrar; indicar; alertar; levantar questões que o outro não tenha pensado ou não tenha imaginado, para chamar a atenção dele}
  \end{phonetics}
\end{entry}

\begin{entry}{提问}{12,6}{⼿、⾨}
  \begin{phonetics}{提问}{ti2wen4}[][HSK 3]
    \definition{v.}{\emph{quiz}; fazer uma pergunta; colocar questões para}
  \end{phonetics}
\end{entry}

\begin{entry}{提供}{12,8}{⼿、⼈}
  \begin{phonetics}{提供}{ti2gong1}[][HSK 4]
    \definition{v.}{oferecer; fornecer; suprir; prover; proporcionar}
  \end{phonetics}
\end{entry}

\begin{entry}{提到}{12,8}{⼿、⼑}
  \begin{phonetics}{提到}{ti2 dao4}[][HSK 2]
    \definition{v.}{mencionar | referir-se a | levantar (assunto)}
  \end{phonetics}
\end{entry}

\begin{entry}{提前}{12,9}{⼿、⼑}
  \begin{phonetics}{提前}{ti2qian2}[][HSK 3]
    \definition{adv.}{antecipadamente}
    \definition{v.}{avançar; adiantar; mudar para uma data anterior; mover para a frente (uma data)}
  \end{phonetics}
\end{entry}

\begin{entry}{提倡}{12,10}{⼿、⼈}
  \begin{phonetics}{提倡}{ti2chang4}[][HSK 5]
    \definition{v.}{promover; incentivar; recomendar; apresentar as vantagens de algo para incentivar as pessoas a usá-lo ou implementá-lo}
  \end{phonetics}
\end{entry}

\begin{entry}{提起}{12,10}{⼿、⾛}
  \begin{phonetics}{提起}{ti2 qi3}[][HSK 5]
    \definition{v.}{mencionar; falar sobre; abordar | levantar; despertar; estimular; revigorar; alegrar/animar | iniciar; instituir; propor | levantar; pegar}
  \end{phonetics}
\end{entry}

\begin{entry}{提高}{12,10}{⼿、⾼}
  \begin{phonetics}{提高}{ti2gao1}[][HSK 2]
    \definition{v.}{melhorar | aumentar | elevar}
  \end{phonetics}
\end{entry}

\begin{entry}{提醒}{12,16}{⼿、⾣}
  \begin{phonetics}{提醒}{ti2xing3}[][HSK 4]
    \definition{v.+compl.}{alertar; avisar; advertir; lembrar; apontar para ou chamar a atenção para}
  \end{phonetics}
\end{entry}

\begin{entry}{插}{12}{⼿}
  \begin{phonetics}{插}{cha1}[][HSK 5]
    \definition{v.}{enfiar; inserir; colocar, apertar, empurrar ou perfurar uma coisa fina ou delgada; mergulhar |interpor; inserir; colocar no meio}
  \end{phonetics}
\end{entry}

\begin{entry}{插手}{12,4}{⼿、⼿}
  \begin{phonetics}{插手}{cha1shou3}
    \definition{v.+compl.}{envolver-se em | dar uma mão | ter (tomar) uma mão | cutucar o nariz de alguém | intrometer-se}
  \end{phonetics}
\end{entry}

\begin{entry}{插话}{12,8}{⼿、⾔}
  \begin{phonetics}{插话}{cha1hua4}
    \definition{s.}{interrupção | digressão}
    \definition{v.+compl.}{interromper (a fala de alguém)}
  \end{phonetics}
\end{entry}

\begin{entry}{握}{12}{⼿}
  \begin{phonetics}{握}{wo4}[][HSK 5]
    \definition{v.}{segurar; agarrar | agarrar; segurar; empunhar; controlar | pegar pela mão}
  \end{phonetics}
\end{entry}

\begin{entry}{握手}{12,4}{⼿、⼿}
  \begin{phonetics}{握手}{wo4shou3}[][HSK 3]
    \definition{v.+compl.}{apertar as mãos}
  \end{phonetics}
\end{entry}

\begin{entry}{援助}{12,7}{⼿、⼒}
  \begin{phonetics}{援助}{yuan2zhu4}
    \definition{s.}{assistência}
    \definition{v.}{ajudar | apoiar | assistir}
  \end{phonetics}
\end{entry}

\begin{entry}{搁浅}{12,8}{⼿、⽔}
  \begin{phonetics}{搁浅}{ge1qian3}
    \definition{v.}{ficar encalhado (navio) | encalhar | (figurativo) encontrar dificuldades e parar}
  \end{phonetics}
\end{entry}

\begin{entry}{搓}{12}{⼿}
  \begin{phonetics}{搓}{cuo1}
    \definition{s.}{torção}
    \definition{v.}{esfregar ou rolar entre as mãos ou dedos | torcer}
  \end{phonetics}
\end{entry}

\begin{entry}{搜}{12}{⼿}
  \begin{phonetics}{搜}{sou1}[][HSK 5]
    \definition{v.}{procurar | pesquisar | coletar; reunir | revistar}
  \end{phonetics}
\end{entry}

\begin{entry}{搜索}{12,10}{⼿、⽷}
  \begin{phonetics}{搜索}{sou1suo3}[][HSK 5]
    \definition{v.}{procurar; caçar; explorar; pesquisar cuidadosamente; refere-se especificamente à busca militar para identificar situações suspeitas em determinada região, área marítima ou aérea}
  \end{phonetics}
\end{entry}

\begin{entry}{搭讪}{12,5}{⼿、⾔}
  \begin{phonetics}{搭讪}{da1shan4}
    \definition{v.}{bater em alguém | incitar uma conversa | começar a conversar para acabar com um silêncio constrangedor ou uma situação embaraçosa}
  \end{phonetics}
\end{entry}

\begin{entry}{搭配}{12,10}{⼿、⾣}
  \begin{phonetics}{搭配}{da1pei4}
    \definition{v.}{emparelhar | combinar | organizar em pares | adicionar alguém em um grupo}
  \end{phonetics}
\end{entry}

\begin{entry}{搞}{13}{⼿}
  \begin{phonetics}{搞}{gao3}[][HSK 5]
    \definition{v.}{fazer; realizar; estar envolvido em; engajar-se em um estudo, fazer algo em relação a, etc. | fazer; produzir; gerar; trabalhar | iniciar; estabelecer; organizar; configurar | consertar (mudar) alguém; fazer alguém sofrer | obter; assegurar; agarrar |  (seguido de um complemento) fazer com que se torne; produzir um determinado efeito ou resultado}
  \end{phonetics}
\end{entry}

\begin{entry}{搞好}{13,6}{⼿、⼥}
  \begin{phonetics}{搞好}{gao3 hao3}[][HSK 5]
    \definition{v.}{fazer um bom trabalho; fazer bem; suar; tornar submisso, tornar útil, por meio de solicitações e presentes amigáveis; amolecer}
  \end{phonetics}
\end{entry}

\begin{entry}{搞乱}{13,7}{⼿、⼄}
  \begin{phonetics}{搞乱}{gao3luan4}
    \definition{v.}{estragar | confundir | bagunçar}
  \end{phonetics}
\end{entry}

\begin{entry}{搞定}{13,8}{⼿、⼧}
  \begin{phonetics}{搞定}{gao3ding4}
    \definition{v.}{consertar | resolver}
  \end{phonetics}
\end{entry}

\begin{entry}{搞鬼}{13,9}{⼿、⿁}
  \begin{phonetics}{搞鬼}{gao3gui3}
    \definition{v.}{fazer travessuras | fazer truques}
  \end{phonetics}
\end{entry}

\begin{entry}{搞笑}{13,10}{⼿、⽵}
  \begin{phonetics}{搞笑}{gao3xiao4}
    \definition{adj.}{engraçado | hilário}
    \definition{v.}{fazer as pessoas rirem}
  \end{phonetics}
\end{entry}

\begin{entry}{搞通}{13,10}{⼿、⾡}
  \begin{phonetics}{搞通}{gao3tong1}
    \definition{v.}{entender algo}
  \end{phonetics}
\end{entry}

\begin{entry}{搞钱}{13,10}{⼿、⾦}
  \begin{phonetics}{搞钱}{gao3qian2}
    \definition{v.}{fazer dinheiro | acumular dinheiro}
  \end{phonetics}
\end{entry}

\begin{entry}{搞混}{13,11}{⼿、⽔}
  \begin{phonetics}{搞混}{gao3hun4}
    \definition{v.}{confundir}
  \end{phonetics}
\end{entry}

\begin{entry}{搞错}{13,13}{⼿、⾦}
  \begin{phonetics}{搞错}{gao3cuo4}
    \definition{v.}{cometer um erro}
  \end{phonetics}
\end{entry}

\begin{entry}{搬}{13}{⼿}
  \begin{phonetics}{搬}{ban1}[][HSK 3]
    \definition{v.}{copiar indiscriminadamente | mover-se (ou seja, mudar-se) | mover-se (algo relativamente pesado ou volumoso) | mudar | mudar-se}
  \end{phonetics}
\end{entry}

\begin{entry}{搬口}{13,3}{⼿、⼝}
  \begin{phonetics}{搬口}{ban1kou3}
    \definition{v.}{tagarelar | (idioma) transmitir histórias | semear dissensão | contar histórias}
  \end{phonetics}
\end{entry}

\begin{entry}{搬动}{13,6}{⼿、⼒}
  \begin{phonetics}{搬动}{ban1dong4}
    \definition{v.}{mover-se (alguma coisa) | se mudar}
  \end{phonetics}
\end{entry}

\begin{entry}{搬弄}{13,7}{⼿、⼶}
  \begin{phonetics}{搬弄}{ban1nong4}
    \definition{v.}{causar problemas | mexer com alguém | mostrar (o que se pode fazer)}
  \end{phonetics}
\end{entry}

\begin{entry}{搬走}{13,7}{⼿、⾛}
  \begin{phonetics}{搬走}{ban1zou3}
    \definition{v.}{carregar}
  \end{phonetics}
\end{entry}

\begin{entry}{搬运}{13,7}{⼿、⾡}
  \begin{phonetics}{搬运}{ban1yun4}
    \definition{s.}{frete | transporte}
    \definition{v.}{carregar | transportar}
  \end{phonetics}
\end{entry}

\begin{entry}{搬家}{13,10}{⼿、⼧}
  \begin{phonetics}{搬家}{ban1jia1}[][HSK 3]
    \definition{s.}{mudança}
    \definition{v.+compl.}{mudar-se de casa}
  \end{phonetics}
\end{entry}

\begin{entry}{摄氏}{13,4}{⼿、⽒}
  \begin{phonetics}{摄氏}{she4shi4}
    \definition{s.}{graus Celsius (°C), centígrado}
  \end{phonetics}
\end{entry}

\begin{entry}{摄像}{13,13}{⼿、⼈}
  \begin{phonetics}{摄像}{she4 xiang4}[][HSK 5]
    \definition{v.}{gravar; filmar; filmar com câmera; fazer uma gravação de vídeo (com uma câmera de vídeo ou TV)}
  \end{phonetics}
\end{entry}

\begin{entry}{摄像机}{13,13,6}{⼿、⼈、⽊}
  \begin{phonetics}{摄像机}{she4 xiang4 ji1}[][HSK 5]
    \definition[个,部]{s.}{câmera de vídeo; dispositivo que pode ser usado para converter imagens captadas em sinais de imagem de televisão}
  \end{phonetics}
\end{entry}

\begin{entry}{摄影}{13,15}{⼿、⼺}
  \begin{phonetics}{摄影}{she4ying3}[][HSK 5]
    \definition{s.}{fotografia}
    \definition{v.}{fotografar; tirar uma foto; tirar fotos ou filmar}
  \end{phonetics}
\end{entry}

\begin{entry}{摄影师}{13,15,6}{⼿、⼺、⼱}
  \begin{phonetics}{摄影师}{she4 ying3 shi1}[][HSK 5]
    \definition[个]{s.}{fotógrafo; cinegrafista; operador de câmera; técnico de fotografia em estúdio fotográfico}
  \end{phonetics}
\end{entry}

\begin{entry}{摆}{13}{⼿}
  \begin{phonetics}{摆}{bai3}[][HSK 4]
    \definition*{s.}{sobrenome Bai | Festival de Ganbai; uma reunião realizada nas áreas Dai durante festivais religiosos, para celebrar uma boa colheita ou para trocar materiais; geralmente se refere a uma reunião em massa}
    \definition{s.}{pêndulo; um dispositivo mecânico que controla a frequência de vibração em relógios e instrumentos | a bainha inferior de um vestido, jaqueta ou saia}
    \definition{v.}{colocar; organizar | vestir; assumir | balançar; acenar; agitar para frente e para trás | expor; declarar claramente; listar | dizer; falar | libertar-se}
  \end{phonetics}
\end{entry}

\begin{entry}{摆手}{13,4}{⼿、⼿}
  \begin{phonetics}{摆手}{bai3shou3}
    \definition{v.+compl.}{gesticular com a mão (acenando, acenando adeus, etc.) | balançar os braços | acenar com as mãos}
  \end{phonetics}
\end{entry}

\begin{entry}{摆动}{13,6}{⼿、⼒}
  \begin{phonetics}{摆动}{bai3 dong4}[][HSK 4]
    \definition{v.}{balançar; balançar para frente e para trás; oscilar; vibrar}
  \end{phonetics}
\end{entry}

\begin{entry}{摆烂}{13,9}{⼿、⽕}
  \begin{phonetics}{摆烂}{bai3lan4}
    \definition{v.}{(neologismo, gíria) parar de lutar (especialmente quando se sabe que não pode ter sucesso) | deixar tudo ir para o inferno}
  \end{phonetics}
\end{entry}

\begin{entry}{摆脱}{13,11}{⼿、⾁}
  \begin{phonetics}{摆脱}{bai3tuo1}[][HSK 4]
    \definition{v.}{sacudir; rejeitar; romper com; libertar-se (ou desembaraçar-se) de; livrar-se de dificuldades, escravidão, controle, etc.}
  \end{phonetics}
\end{entry}

\begin{entry}{摇}{13}{⼿}
  \begin{phonetics}{摇}{yao2}[][HSK 4]
    \definition{v.}{chacoalhar; ondular; balançar; fazer com que um objeto se mova para frente e para trás | agitar algo | sacudir; chacoalhar; agitar algo para que se mova}
  \end{phonetics}
\end{entry}

\begin{entry}{摇头}{13,5}{⼿、⼤}
  \begin{phonetics}{摇头}{yao2tou2}[][HSK 5]
    \definition{v.+compl.}{sacudir; balançar a cabeça; balançar a cabeça para a esquerda e para a direita, indicando negação, desacordo ou impedimento}
  \end{phonetics}
\end{entry}

\begin{entry}{摇晃}{13,10}{⼿、⽇}
  \begin{phonetics}{摇晃}{yao2huang4}
    \definition{v.}{sacudir | agitar | balançar | chacoalhar}
  \end{phonetics}
\end{entry}

\begin{entry}{摸}{13}{⼿}
  \begin{phonetics}{摸}{mo1}[][HSK 4]
    \definition{v.}{sentir; acariciar; tocar; tocar (um objeto) levemente com a mão e depois removê-lo ou mover a mão suavemente sobre a superfície do objeto | sentir para; tatear para; sentir algo com as mãos | descobrir; sentir; sondar; explorar; tentar fazer ou entender | sentir o caminho; tatear no escuro; andar por estradas que você não consegue reconhecer | furtar; roubar}
  \end{phonetics}
\end{entry}

\begin{entry}{摔}{14}{⼿}
  \begin{phonetics}{摔}{shuai1}[][HSK 5]
    \definition{v.}{cair; tropeçar; perder o equilíbrio | mergulhar; precipitar-se; cair de uma altura elevada | quebrar; fazer cair e quebrar | lançar; atirar; arremessar; joguar coisas com força e para baixo | bater; golpear; bater com força para que o que está grudado cair}
  \end{phonetics}
\end{entry}

\begin{entry}{摔倒}{14,10}{⼿、⼈}
  \begin{phonetics}{摔倒}{shuai1dao3}[][HSK 5]
    \definition{v.}{cair; tropeçar; perder o equilíbrio e cair}
  \end{phonetics}
\end{entry}

\begin{entry}{摘}{14}{⼿}
  \begin{phonetics}{摘}{zhai1}[][HSK 5]
    \definition{v.}{pegar; arrancar; tirar; colher (flores, frutos, folhas de plantas); retirar (coisas que estão sendo usadas ou penduradas) | selecionar; fazer extrações de | pedir dinheiro emprestado em caso de necessidade urgente | vencer; ganhar; alcançar; obter}
  \end{phonetics}
\end{entry}

\begin{entry}{摩托}{15,6}{⼿、⼿}
  \begin{phonetics}{摩托}{mo2 tuo1}[][HSK 5]
    \definition[辆]{s.}{(empréstimo linguístico) motor; motor de combustão interna | (empréstimo linguístico) motocicleta, abreviação de 摩托车}
    \seeref{摩托车}{mo2tuo1che1}
  \end{phonetics}
\end{entry}

\begin{entry}{摩托车}{15,6,4}{⼿、⼿、⾞}
  \begin{phonetics}{摩托车}{mo2tuo1che1}
    \definition[辆,部]{s.}{(empréstimo linguístico) motocicleta}
  \end{phonetics}
\end{entry}

\begin{entry}{摩擦}{15,17}{⼿、⼿}
  \begin{phonetics}{摩擦}{mo2ca1}[][HSK 5]
    \definition{s.}{atrito; desacordo; conflito (entre duas partes); a ação de impedir o movimento relativo entre dois objetos em contato, produzida na superfície de contato | atrito; metáfora para o conflito entre as duas partes}
    \definition{v.}{esfregar}
  \end{phonetics}
\end{entry}

\begin{entry}{撒旦}{15,5}{⼿、⽇}
  \begin{phonetics}{撒旦}{sa1dan4}
    \definition*{s.}{Satã}
  \end{phonetics}
\end{entry}

\begin{entry}{撒旦主义}{15,5,5,3}{⼿、⽇、⼂、⼂}
  \begin{phonetics}{撒旦主义}{sa1dan4 zhu3yi4}
    \definition*{s.}{Satanismo}
  \end{phonetics}
\end{entry}

\begin{entry}{撒但}{15,7}{⼿、⼈}
  \begin{phonetics}{撒但}{sa1dan4}
    \variantof{撒旦}
  \end{phonetics}
\end{entry}

\begin{entry}{撞}{15}{⼿}
  \begin{phonetics}{撞}{zhuang4}[][HSK 5]
    \definition{v.}{chocar-se contra; chocar-se com; bater; colidir | encontrar-se por acaso; esbarrar em; deparar-se com | apressar; correr; empurrar | aproveitar a chance | esbarrar de repente em |  encontrar | confiar em; tentar | agir precipitadamente; invadir}
  \end{phonetics}
\end{entry}

\begin{entry}{撞车}{15,4}{⼿、⾞}
  \begin{phonetics}{撞车}{zhuang4che1}
    \definition{v.+compl.}{(figurativo) colidir (opiniões, cronogramas, etc.) | ser o mesmo (assunto) | colidir (com outro veículo)}
  \end{phonetics}
\end{entry}

\begin{entry}{撞运气}{15,7,4}{⼿、⾡、⽓}
  \begin{phonetics}{撞运气}{zhuang4yun4qi5}
    \definition{v.}{confiar no destino | tentar a sorte}
  \end{phonetics}
\end{entry}

\begin{entry}{撤}{15}{⼿}
  \begin{phonetics}{撤}{che4}
    \definition{v.}{remover, tirar}
  \end{phonetics}
\end{entry}

\begin{entry}{播出}{15,5}{⼿、⼐}
  \begin{phonetics}{播出}{bo1 chu1}[][HSK 3]
    \definition{v.}{transmitir | estar no ar}
  \end{phonetics}
\end{entry}

\begin{entry}{播放}{15,8}{⼿、⽅}
  \begin{phonetics}{播放}{bo1fang4}[][HSK 3]
    \definition{v.}{ir ao ar | transmitir por rádio | mostrar | transmitir (um programa de TV)}
  \end{phonetics}
\end{entry}

\begin{entry}{播音}{15,9}{⼿、⾳}
  \begin{phonetics}{播音}{bo1yin1}
    \definition{s.}{transmissão}
    \definition{v.+compl.}{transmitir}
  \end{phonetics}
\end{entry}

\begin{entry}{擒获}{15,10}{⼿、⾋}
  \begin{phonetics}{擒获}{qin2huo4}
    \definition{v.}{apreender | capturar}
  \end{phonetics}
\end{entry}

\begin{entry}{撼}{16}{⼿}
  \begin{phonetics}{撼}{han4}
    \definition{v.}{sacudir | vibrar}
  \end{phonetics}
\end{entry}

\begin{entry}{擅自}{16,6}{⼿、⾃}
  \begin{phonetics}{擅自}{shan4zi4}
    \definition{adv.}{sem permissão ou autorização | por iniciativa própria}
  \end{phonetics}
\end{entry}

\begin{entry}{操心}{16,4}{⼿、⼼}
  \begin{phonetics}{操心}{cao1xin1}
    \definition{v.+compl.}{preocupar-se com}
  \end{phonetics}
\end{entry}

\begin{entry}{操场}{16,6}{⼿、⼟}
  \begin{phonetics}{操场}{cao1chang3}[][HSK 4]
    \definition[个]{s.}{\emph{playground}; campo esportivo; locais para exercícios físicos ou exercícios militares}
  \end{phonetics}
\end{entry}

\begin{entry}{操作}{16,7}{⼿、⼈}
  \begin{phonetics}{操作}{cao1zuo4}[][HSK 4]
    \definition{s.}{operação}
    \definition{v.}{operar; seguir os requisitos e procedimentos prescritos| implementar; realizar; executar; refere-se à implementação concreta (planos, medidas, etc.)}
  \end{phonetics}
\end{entry}

\begin{entry}{擦}{17}{⼿}
  \begin{phonetics}{擦}{ca1}[][HSK 4]
    \definition{v.}{enxugar; esfregar; apagar; limpar; limpar esfregando com um pano, toalha de mão, etc. | espalhar sobre; colocar sobre | passar raspando | ralar (em pedaços); ralar frutas em um ralador para fazer fios finos}
  \end{phonetics}
\end{entry}

\begin{entry}{擦拭}{17,9}{⼿、⼿}
  \begin{phonetics}{擦拭}{ca1shi4}
    \definition{v.}{limpar com um pano}
  \end{phonetics}
\end{entry}

\begin{entry}{攀岩}{19,8}{⼿、⼭}
  \begin{phonetics}{攀岩}{pan1yan2}
    \definition{s.}{alpinista}
    \definition{v.}{escalar uma montanha}
  \end{phonetics}
\end{entry}

\begin{entry}{攀爬}{19,8}{⼿、⽖}
  \begin{phonetics}{攀爬}{pan1pa2}
    \definition{v.}{escalar}
  \end{phonetics}
\end{entry}

%%%%% EOF %%%%%

