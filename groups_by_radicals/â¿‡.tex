%%%
%%% Radical "⿇"
%%%

\section*{Radical 200: ``⿇''}\addcontentsline{toc}{section}{Radical 200: ⿇}

\begin{entry}{麻}{11}{⿇}
  \begin{phonetics}{麻}{ma2}
    \definition*{s.}{sobrenome Ma}
    \definition{adj.}{áspero; grosseiro | marcado; manchado | espinhas; manchas ásperas; cicatrizes deixadas após a varíola}
    \definition[棵,株]{s.}{nome geral para cânhamo, linho, etc. | fibra de cânhamo, linho, etc. para têxteis | sésamo; gergelim | marcas de varíola; um rosto com marcas de varíola}
    \definition{v.}{anestesiar | corromper (a mente de alguém); envenenar}
  \end{phonetics}
\end{entry}

\begin{entry}{麻将}{11,9}{⿇、⼨}
  \begin{phonetics}{麻将}{ma2jiang4}
    \definition*[副]{s.}{Mahjong}
  \end{phonetics}
\end{entry}

\begin{entry}{麻烦}{11,10}{⿇、⽕}
  \begin{phonetics}{麻烦}{ma2fan5}[][HSK 3]
    \definition{adj.}{incômodo; inconveniente; complicado; trabalhoso; burocrático | incômodo; inconveniente; (a situação) é confusa e complicada}
    \definition[个,些,点,堆]{s.}{problema; inconveniência; assuntos complicados e difíceis de resolver}
    \definition{v.}{incomodar; perturbar; incomodar alguém; irritar; aborrecer; causar incômodo ou sobrecarregar outras pessoas}
  \end{phonetics}
\end{entry}

\begin{entry}{麻辣豆腐}{11,14,7,14}{⿇、⾟、⾖、⾁}
  \begin{phonetics}{麻辣豆腐}{ma2la4 dou4fu5}
    \definition{s.}{tofú guisado em molho picante (prato)}
  \end{phonetics}
\end{entry}

%%%%% EOF %%%%%

