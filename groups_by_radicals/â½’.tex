%%%
%%% Radical "⽒"
%%%

\section*{Radical 83: ``⽒''}\addcontentsline{toc}{section}{Radical 83: ⽒}

\begin{entry}{民}{5}{⽒}
  \begin{phonetics}{民}{min2}
    \definition*{s.}{Sobrenome Min}
    \definition{adj.}{folclórico ; civil (não militar)}
    \definition{s.}{pessoa | membro de um grupo étnico | uma pessoa de uma determinada ocupação | do povo; folclore | civil; cidadão | o povo | um membro de uma nacionalidade}
  \end{phonetics}
\end{entry}

\begin{entry}{民工}{5,3}{⽒、⼯}
  \begin{phonetics}{民工}{min2 gong1}[][HSK 6]
    \definition{s.}{trabalhador trabalhando em um projeto público | trabalhador temporário alistado em um projeto público | agricultor que trabalha em empregos temporários na cidade | trabalhador migrante}
  \end{phonetics}
\end{entry}

\begin{entry}{民主}{5,5}{⽒、⼂}
  \begin{phonetics}{民主}{min2zhu3}[][HSK 6]
    \definition{adj.}{democrático; em consonância com os princípios democráticos}
    \definition[个]{s.}{democracia; direitos democráticos; refere-se ao direito do povo de participar da vida política e dos assuntos do Estado e de expressar livremente suas opiniões}
  \end{phonetics}
\end{entry}

\begin{entry}{民众}{5,6}{⽒、⼈}
  \begin{phonetics}{民众}{min2zhong4}
    \definition{s.}{a população | as massas | as pessoas comuns}
  \end{phonetics}
\end{entry}

\begin{entry}{民间}{5,7}{⽒、⾨}
  \begin{phonetics}{民间}{min2jian1}[][HSK 3]
    \definition{s.}{entre o povo | não governamental; de pessoa para pessoa}
  \end{phonetics}
\end{entry}

\begin{entry}{民族}{5,11}{⽒、⽅}
  \begin{phonetics}{民族}{min2zu2}[][HSK 3]
    \definition[个]{s.}{nação; uma comunidade estável formada ao longo da história pela humanidade, com uma língua comum, uma região comum, uma vida econômica comum e uma mentalidade comum expressa em uma cultura comum | grupo étnico; refere-se, de maneira geral, às comunidades formadas ao longo da história por pessoas em diferentes estágios de desenvolvimento social}
  \end{phonetics}
\end{entry}

\begin{entry}{民意}{5,13}{⽒、⼼}
  \begin{phonetics}{民意}{min2 yi4}[][HSK 6]
    \definition{s.}{vontade do povo; vontade popular | opinião pública}
  \end{phonetics}
\end{entry}

\begin{entry}{民歌}{5,14}{⽒、⽋}
  \begin{phonetics}{民歌}{min2 ge1}[][HSK 6]
    \definition[支,首]{s.}{canção folclórica; os nomes dos autores das canções transmitidas oralmente são muitas vezes desconhecidos}
  \end{phonetics}
\end{entry}

\begin{entry}{民警}{5,19}{⽒、⾔}
  \begin{phonetics}{民警}{min2 jing3}[][HSK 6]
    \definition{s.}{polícia; policial}
  \end{phonetics}
\end{entry}

%%%%% EOF %%%%%

