%%%
%%% Radical "⽒"
%%%

\section*{Radical 83: ``⽒''}\addcontentsline{toc}{section}{Radical 83: ⽒}

\begin{entry}{民}{5}{⽒}
  \begin{phonetics}{民}{min2}
    \definition*{s.}{sobrenome Min}
    \definition{adj.}{folclórico ; civil (não militar)}
    \definition{s.}{pessoa | membro de um grupo étnico | uma pessoa de uma determinada ocupação | do povo; folclore | civil; cidadão | o povo | um membro de uma nacionalidade}
  \end{phonetics}
\end{entry}

\begin{entry}{民主}{5,5}{⽒、⼂}
  \begin{phonetics}{民主}{min2zhu3}
    \definition{adj.}{democrático}
    \definition{s.}{democracia}
  \end{phonetics}
\end{entry}

\begin{entry}{民众}{5,6}{⽒、⼈}
  \begin{phonetics}{民众}{min2zhong4}
    \definition{s.}{a população | as massas | as pessoas comuns}
  \end{phonetics}
\end{entry}

\begin{entry}{民间}{5,7}{⽒、⾨}
  \begin{phonetics}{民间}{min2jian1}[][HSK 3]
    \definition{s.}{entre o povo | não governamental; de pessoa para pessoa}
  \end{phonetics}
\end{entry}

\begin{entry}{民族}{5,11}{⽒、⽅}
  \begin{phonetics}{民族}{min2zu2}[][HSK 3]
    \definition[个]{s.}{nação; uma comunidade estável formada ao longo da história pela humanidade, com uma língua comum, uma região comum, uma vida econômica comum e uma mentalidade comum expressa em uma cultura comum | grupo étnico; refere-se, de maneira geral, às comunidades formadas ao longo da história por pessoas em diferentes estágios de desenvolvimento social}
  \end{phonetics}
\end{entry}

%%%%% EOF %%%%%

