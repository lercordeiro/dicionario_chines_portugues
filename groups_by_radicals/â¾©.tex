%%%
%%% Radical "⾩"
%%%

\section*{Radical 170: ``⾩'' (⻖)}\addcontentsline{toc}{section}{Radical 170: ⾩、⻖}

\begin{Entry}{队}{4}{⾩}
  \begin{Phonetics}{队}{dui4}[][HSK 2]
    \definition*{s.}{Jovens Pioneiros, refere-se especificamente à Patrulha de Jovens Pioneiros}
    \definition[条,个,支]{s.}{fila de pessoas | equipe; grupo;}
  \end{Phonetics}
\end{Entry}

\begin{Entry}{队友}{4,4}{⾩、⼜}
  \begin{Phonetics}{队友}{dui4you3}
    \definition{s.}{companheiro de equipe}
  \end{Phonetics}
\end{Entry}

\begin{Entry}{队长}{4,4}{⾩、⾧}
  \begin{Phonetics}{队长}{dui4 zhang3}[][HSK 2]
    \definition[个,位,名]{s.}{capitão (de equipe); capitães | líder de equipe}
  \end{Phonetics}
\end{Entry}

\begin{Entry}{队伍}{4,6}{⾩、⼈}
  \begin{Phonetics}{队伍}{dui4wu3}[][HSK 6]
    \definition[条,行,列,支,个]{s.}{tropas; exército | fileiras; contingente | desfile}
  \end{Phonetics}
\end{Entry}

\begin{Entry}{队员}{4,7}{⾩、⼝}
  \begin{Phonetics}{队员}{dui4 yuan2}[][HSK 3]
    \definition[名,位,个]{s.}{membro da equipe; a composição de uma equipe}
  \end{Phonetics}
\end{Entry}

\begin{Entry}{防}{6}{⾩}
  \begin{Phonetics}{防}{fang2}[][HSK 3]
    \definition*{s.}{Sobrenome Fang}
    \definition{s.}{defesa | dique; aterro | barragem; represa; estrutura para conter a água}
    \definition{v.}{proteger contra; prevenir contra; tomar precauções contra | defender-se contra}
  \end{Phonetics}
\end{Entry}

\begin{Entry}{防止}{6,4}{⾩、⽌}
  \begin{Phonetics}{防止}{fang2zhi3}[][HSK 3]
    \definition{v.}{evitar; prevenir; prevenir; proteger contra; preparar-se com antecedência para evitar que coisas ruins aconteçam}
  \end{Phonetics}
\end{Entry}

\begin{Entry}{防守}{6,6}{⾩、⼧}
  \begin{Phonetics}{防守}{fang2shou3}[][HSK 6]
    \definition{v.}{defender; guardar}
  \end{Phonetics}
\end{Entry}

\begin{Entry}{防护}{6,7}{⾩、⼿}
  \begin{Phonetics}{防护}{fang2hu4}
    \definition{v.}{defender | proteger}
  \end{Phonetics}
\end{Entry}

\begin{Entry}{防治}{6,8}{⾩、⽔}
  \begin{Phonetics}{防治}{fang2zhi4}[][HSK 5]
    \definition{s.}{tratamento preventivo; prevenção e cura; profilaxia e tratamento}
  \end{Phonetics}
\end{Entry}

\begin{Entry}{防范}{6,9}{⾩、⾋}
  \begin{Phonetics}{防范}{fang2 fan4}[][HSK 6]
    \definition{v.}{vigiar; estar em guarda; ficar de olho}
  \end{Phonetics}
\end{Entry}

\begin{Entry}{防晒}{6,10}{⾩、⽇}
  \begin{Phonetics}{防晒}{fang2shai4}
    \definition{s.}{protetor solar}
  \end{Phonetics}
\end{Entry}

\begin{Entry}{阳}{6}{⾩}
  \begin{Phonetics}{阳}{yang2}
    \definition*{s.}{Sobrenome Yang}
    \definition{adj.}{em relevo | aberto; evidente; revelado | pertencente a este mundo; preocupado com os seres vivos; superstição se refere a coisas que pertencem aos vivos e ao mundo | positivo; carregado positivamente}
    \definition{s.}{o Sol; luz solar | ao sul de uma colina ou ao norte de um rio | yang (o princípio masculino ou positivo da natureza); na filosofia chinesa antiga, refere-se a um dos dois opostos que permeiam todas as coisas no universo (o outro lado é "yin") | masculino; refere-se aos órgãos genitais masculinos ou à função reprodutiva |
varanda; refere-se ao lugar onde o sol brilha (oposto a 阴)}
  \seealsoref{阴}{yin1}
  \seealsoref{阴阳}{yin1yang2}
  \end{Phonetics}
\end{Entry}

\begin{Entry}{阳台}{6,5}{⾩、⼝}
  \begin{Phonetics}{阳台}{yang2tai2}[][HSK 4]
    \definition[个,块,处]{s.}{varanda; terraço; sacada; pequeno terraço do edifício com grades para se refrescar, tomar sol ou olhar o horizonte}
  \end{Phonetics}
\end{Entry}

\begin{Entry}{阳光}{6,6}{⾩、⼉}
  \begin{Phonetics}{阳光}{yang2guang1}[][HSK 3]
    \definition{adj.}{alegre; otimista; personalidade positiva e alegre; cheio de vitalidade juvenil | aberto; transparente; público; conduzido sob supervisão pública}
    \definition[缕,束,道]{s.}{luz do sol; raio de sol}
  \end{Phonetics}
\end{Entry}

\begin{Entry}{阴}{6}{⾩}
  \begin{Phonetics}{阴}{yin1}[][HSK 2]
    \definition*{s.}{Yin, o princípio negativo de Yin e Yang | A Lua; refere-se a Taiyin | Sobrenome Yin}
    \definition{adj.}{nublado; opaco; sombrio | escondido; secreto; não exposto | sinistro | do mundo inferior; dos fantasmas | Física: negativo; cátodo | nublado; mais de 80\% do céu estão cobertos por nuvens | em talhe-doce; rebaixado | (matéria) carregada negativamente}
    \definition[片]{s.}{sombra; lugar sombrio | partes íntimas (especialmente da mulher) | ao norte de uma colina ou ao sul de um rio | verso | entalhe}
  \seealsoref{阳}{yang2}
  \seealsoref{阴阳}{yin1yang2}
  \end{Phonetics}
\end{Entry}

\begin{Entry}{阴天}{6,4}{⾩、⼤}
  \begin{Phonetics}{阴天}{yin1 tian1}[][HSK 2]
    \definition[个]{s.}{nublado; céu nublado; dia nublado; uma condição climática em que 80\% do céu está coberto por nuvens e apenas um pouco de sol pode ser visto}
  \end{Phonetics}
\end{Entry}

\begin{Entry}{阴阳}{6,6}{⾩、⾩}
  \begin{Phonetics}{阴阳}{yin1yang2}
    \definition*{s.}{Yin e Yang}
  \seealsoref{阳}{yang2}
  \seealsoref{阴}{yin1}
  \end{Phonetics}
\end{Entry}

\begin{Entry}{阴谋}{6,11}{⾩、⾔}
  \begin{Phonetics}{阴谋}{yin1mou2}[][HSK 6]
    \definition[个,场,起]{s.}{trama; conspiração; um esquema para fazer o mal em segredo}
    \definition{v.}{tramar; conspirar secretamente (fazer algo ruim)}
  \end{Phonetics}
\end{Entry}

\begin{Entry}{阴影}{6,15}{⾩、⼺}
  \begin{Phonetics}{阴影}{yin1 ying3}[][HSK 6]
    \definition{s.}{sombra; sombra escura | uma analogia de elementos negativos em negócios, relacionamentos, estado mental, etc.}
  \end{Phonetics}
\end{Entry}

\begin{Entry}{阵}{6}{⾩}
  \begin{Phonetics}{阵}{zhen4}[][HSK 4]
    \definition{clas.}{um parágrafo que mostra o processo de um evento ou ação}
    \definition{s.}{matriz de batalha (formação); termo tático antigo para as fileiras ou formações de uma equipe de combate | \emph{front}; frente de batalha; posição | um período de tempo}
  \end{Phonetics}
\end{Entry}

\begin{Entry}{阵地}{6,6}{⾩、⼟}
  \begin{Phonetics}{阵地}{zhen4di4}
    \definition{s.}{posição (militar) | frente de batalha | \emph{front}}
  \end{Phonetics}
\end{Entry}

\begin{Entry}{阶}{6}{⾩}
  \begin{Phonetics}{阶}{jie1}
    \definition{s.}{degrau; escada; escadaria | classificação | escala | ordem | estágio}
  \end{Phonetics}
\end{Entry}

\begin{Entry}{阶段}{6,9}{⾩、⽎}
  \begin{Phonetics}{阶段}{jie1duan4}[][HSK 4]
    \definition[个,段]{s.}{estágio; fase; período; bancada; gradação}
  \end{Phonetics}
\end{Entry}

\begin{Entry}{阻}{7}{⾩}
  \begin{Phonetics}{阻}{zu3}
    \definition{v.}{impedir; bloquear; obstruir; atrapalhar}
  \end{Phonetics}
\end{Entry}

\begin{Entry}{阻止}{7,4}{⾩、⽌}
  \begin{Phonetics}{阻止}{zu3zhi3}[][HSK 4]
    \definition{v.}{parar; reter; conter; interromper; impedir o avanço; impedir o movimento; obstruir}
  \end{Phonetics}
\end{Entry}

\begin{Entry}{阻击}{7,5}{⾩、⼐}
  \begin{Phonetics}{阻击}{zu3ji1}
    \definition{v.}{verificar | parar}
  \end{Phonetics}
\end{Entry}

\begin{Entry}{阻碍}{7,13}{⾩、⽯}
  \begin{Phonetics}{阻碍}{zu3'ai4}[][HSK 5]
    \definition{s.}{obstáculo; impedimento; barreira}
    \definition{v.}{bloquear; impedir; obstruir; impedir o bom andamento ou desenvolvimento}
  \end{Phonetics}
\end{Entry}

\begin{Entry}{阿}{7}{⾩}
  \begin{Phonetics}{阿}{a1}
    \definition{pref.}{em dialetos do sul para formar termos carinhosos, antes de nomes de animais de estimação, sobrenomes monossilábicos ou números que denotam ordem de antiguidade em uma; anexado a 大, 二, 三,\dots\ para indicar classificação (e, às vezes, intimidade) | antes dos termos de parentesco; na frente de um sobrenome, de um nome próprio ou de um determinado título, com uma conotação de intimidade | em alguns contextos, pode soar infantil ou muito informal (por exemplo, chamar um colega de trabalho por ``阿 + Nome'' sem intimidade)}[阿妈===mamãe | 阿明 ===forma carinhosa de chamar uma pessoa chamada Ming]
  \end{Phonetics}
  \begin{Phonetics}{阿}{e1}
    \definition*{s.}{Dong'e, um condado na província de Shandong | Sobrenome E}
    \definition{s.}{grande monte (ou colina) | um lugar sinuoso (montanha, água, etc.)}
    \definition{v.}{bajular; satisfazer}
  \end{Phonetics}
\end{Entry}

\begin{Entry}{阿拉伯语}{7,8,7,9}{⾩、⼿、⼈、⾔}
  \begin{Phonetics}{阿拉伯语}{a1la1bo2 yu3}[][HSK 7,8,9]
    \definition{s.}{árabe; idioma árabe}
  \end{Phonetics}
\end{Entry}
%%%%% EOF %%%%%

\begin{Entry}{阿姨}{7,9}{⾩、⼥}
  \begin{Phonetics}{阿姨}{a1yi2}[][HSK 4]
    \definition[个,位,名,些]{s.}{tia; uma forma de tratamento para uma mulher da geração dos pais; dirigir-se a uma mulher que tem aproximadamente a mesma idade da sua mãe, geralmente não é parente | babá em uma família; professora em um jardim de infância | tia; irmã da mãe (mais comum no sul da China)}[阿姨,生日快乐!===Tia, feliz aniversário! | 阿姨,这个苹果多少钱一斤?===Tia/Senhora, quanto custa o quilo dessas maçãs? | 阿姨,我想喝水。===Tia/Babá, eu quero beber água.]
  \end{Phonetics}
\end{Entry}

\begin{Entry}{阿哥}{7,10}{⾩、⼝}
  \begin{Phonetics}{阿哥}{a1ge1}
    \definition{s.}{irmão mais velho (afetivo) | forma afetuosa de tratamento entre homens da mesma idade}[阿哥,帮我拿一下书包!===Irmão, ajude-me com minha mochila escolar!]
  \end{Phonetics}
\end{Entry}

\begin{Entry}{附}{7}{⾩}
  \begin{Phonetics}{附}{fu4}
    \definition*{s.}{Sobrenome Fu}
    \definition{v.}{adicionar; anexar; incluir | chegar perto de; estar perto de | depender de; confiar em; cumprir com | concordar com; anexar a; aderir a; cumprir com; depender de}
  \end{Phonetics}
\end{Entry}

\begin{Entry}{附件}{7,6}{⾩、⼈}
  \begin{Phonetics}{附件}{fu4jian4}[][HSK 5]
    \definition*{s.}{\emph{Adnexa Uteri}, refere-se à genitália interna feminina que não seja o útero, as trompas de falópio e os ovários}
    \definition{s.}{apêndice; documentos que acompanham o documento principal | acessório; anexo; peças ou sobressalentes que não sejam peças principais de máquinas e equipamentos | anexo; documentos ou itens relevantes emitidos com o documento}
  \end{Phonetics}
\end{Entry}

\begin{Entry}{附近}{7,7}{⾩、⾡}
  \begin{Phonetics}{附近}{fu4jin4}[][HSK 4]
    \definition{adj.}{perto; vizinho}
    \definition{s.}{vizinhança; bairro}
  \end{Phonetics}
\end{Entry}

\begin{Entry}{陆}{7}{⾩}
  \begin{Phonetics}{陆}{liu4}
    \definition{num.}{seis, usado para o numeral 六 em cheques, etc. para evitar erros ou alterações}
  \seealsoref{六}{liu4}
  \end{Phonetics}
  \begin{Phonetics}{陆}{lu4}
    \definition*{s.}{Sobrenome Lu}
    \definition[个]{s.}{terra; terreno | rota terrestre; por terra}
  \end{Phonetics}
\end{Entry}

\begin{Entry}{陆军}{7,6}{⾩、⼍}
  \begin{Phonetics}{陆军}{lu4 jun1}[][HSK 6]
    \definition{s.}{força terrestre; exército}
  \end{Phonetics}
\end{Entry}

\begin{Entry}{陆地}{7,6}{⾩、⼟}
  \begin{Phonetics}{陆地}{lu4di4}[][HSK 4]
    \definition[块,片]{s.}{terra; terra seca (em oposição ao mar); superfície da Terra, excluindo os oceanos (e, às vezes, rios e lagos)}
  \end{Phonetics}
\end{Entry}

\begin{Entry}{陆续}{7,11}{⾩、⽷}
  \begin{Phonetics}{陆续}{lu4xu4}[][HSK 4]
    \definition{adv.}{sucessivamente; um após o outro; intermitentemente}
  \end{Phonetics}
\end{Entry}

\begin{Entry}{陆路}{7,13}{⾩、⾜}
  \begin{Phonetics}{陆路}{lu4lu4}
    \definition{s.}{rota terrestre}
  \end{Phonetics}
\end{Entry}

\begin{Entry}{降}{8}{⾩}
  \begin{Phonetics}{降}{jiang4}[][HSK 4]
    \definition*{s.}{Sobrenome Jiang}
    \definition{v.}{cair; descer; quedar-se (oposto de 升 ) | diminuir; reduzir; cair; abaixar | nascer}
  \seealsoref{升}{sheng1}
  \end{Phonetics}
\end{Entry}

\begin{Entry}{降价}{8,6}{⾩、⼈}
  \begin{Phonetics}{降价}{jiang4 jia4}[][HSK 4]
    \definition{v.}{ficar mais barato; cortar o preço; reduzir o preço}
  \end{Phonetics}
\end{Entry}

\begin{Entry}{降低}{8,7}{⾩、⼈}
  \begin{Phonetics}{降低}{jiang4di1}[][HSK 4]
    \definition{v.}{reduzir; cortar; diminuir; rebaixar; cair; abaixar}
  \end{Phonetics}
\end{Entry}

\begin{Entry}{降温}{8,12}{⾩、⽔}
  \begin{Phonetics}{降温}{jiang4 wen1}[][HSK 4]
    \definition{v.}{baixar a temperatura (como em uma oficina);  recusar | cair a temperatura | esfriar; resfriar; metáfora para um declínio no entusiasmo ou uma diminuição no ímpeto de algo}
  \end{Phonetics}
\end{Entry}

\begin{Entry}{降落}{8,12}{⾩、⾋}
  \begin{Phonetics}{降落}{jiang4luo4}[][HSK 4]
    \definition{v.}{aterrissar; descer; descer do céu}
  \end{Phonetics}
\end{Entry}

\begin{Entry}{限}{8}{⾩}
  \begin{Phonetics}{限}{xian4}
    \definition{s.}{limite | limiar}
    \definition{v.}{definir um limite; limitar; restringir}
  \end{Phonetics}
\end{Entry}

\begin{Entry}{限制}{8,8}{⾩、⼑}
  \begin{Phonetics}{限制}{xian4zhi4}[][HSK 4]
    \definition[些]{s.}{limite; restrição; confinamento}
    \definition{v.}{limitar; adstringir; restringir; confinar; fechar em (sobre)}
  \end{Phonetics}
\end{Entry}

\begin{Entry}{院}{9}{⾩}
  \begin{Phonetics}{院}{yuan4}[][HSK 2]
    \definition*{s.}{Sobrenome Yuan}
    \definition[个]{s.}{pátio; quintal; complexo | designação para certos escritórios governamentais e locais públicos | faculdade; academia; instituto de ensino superior | hospital}
  \end{Phonetics}
\end{Entry}

\begin{Entry}{院子}{9,3}{⾩、⼦}
  \begin{Phonetics}{院子}{yuan4zi5}[][HSK 2]
    \definition[个,座,处]{s.}{quintal; pátio; o espaço aberto na frente ou atrás de uma casa cercado por muros ou cercas}
  \end{Phonetics}
\end{Entry}

\begin{Entry}{院长}{9,4}{⾩、⾧}
  \begin{Phonetics}{院长}{yuan4zhang3}[][HSK 2]
    \definition[个,位,名]{s.}{reitor; diretor; o mais alto funcionário de qualquer instituição ou escola pública ou privada}
  \end{Phonetics}
\end{Entry}

\begin{Entry}{除}{9}{⾩}
  \begin{Phonetics}{除}{chu2}[][HSK 6]
    \definition*{s.}{Sobrenome Chu}
    \definition{prep.}{exceto; não incluído | além do mais}
    \definition{s.}{degraus de uma casa; degraus de uma porta; escadaria}
    \definition{v.}{remover; livrar-se de; eliminar; limpar | dividir; executar operação de divisão | nomear para o cargo}
  \end{Phonetics}
\end{Entry}

\begin{Entry}{除了}{9,2}{⾩、⼅}
  \begin{Phonetics}{除了}{chu2le5}[][HSK 3]
    \definition{prep.}{exceto; à parte; indica que o que foi dito não é levado em consideração | além disso; além de; usado em conjunto com 还, 也 e 只, indica que, além de algo, há ainda outra coisa | ou \dots ou \dots; usado em conjunto com 就是, significa "ou assim ou assado"}
  \seealsoref{还}{hai2}
  \seealsoref{就是}{jiu4 shi4}
  \seealsoref{也}{ye3}
  \seealsoref{只}{zhi3}
  \end{Phonetics}
\end{Entry}

\begin{Entry}{除夕}{9,3}{⾩、⼣}
  \begin{Phonetics}{除夕}{chu2xi1}[][HSK 5]
    \definition*{s.}{Véspera de Ano Novo Lunar; a noite do último dia do ano, também se refere ao último dia do ano}
  \end{Phonetics}
\end{Entry}

\begin{Entry}{除非}{9,8}{⾩、⾮}
  \begin{Phonetics}{除非}{chu2fei1}[][HSK 5]
    \definition{conj.}{a menos que; somente se; indica a única condição, equivalente a 只有, frequentemente combinada com 才, 否则, 不然, etc.}
  \seealsoref{不然}{bu4ran2}
  \seealsoref{才}{cai2}
  \seealsoref{否则}{fou3ze2}
  \seealsoref{只有}{zhi3 you3}
  \end{Phonetics}
\end{Entry}

\begin{Entry}{险}{9}{⾩}
  \begin{Phonetics}{险}{xian3}[][HSK 6]
    \definition{adj.}{perigoso; arriscado | sinistro; cruel; venenoso}
    \definition{adv.}{por um fio de cabelo; por centímetros; quase}
    \definition{s.}{lugar de difícil acesso; lugar perigoso e difícil de atravessar; passagem estreita; desfiladeiro | abreviação de seguro, 保险 | perigo; risco}
  \seealsoref{保险}{bao3xian3}
  \end{Phonetics}
\end{Entry}

\begin{Entry}{陪}{10}{⾩}
  \begin{Phonetics}{陪}{pei2}[][HSK 5]
    \definition{v.}{servir; acompanhar; cuidar; fazer companhia a alguém | auxiliar; ajudar}
  \end{Phonetics}
\end{Entry}

\begin{Entry}{陪同}{10,6}{⾩、⼝}
  \begin{Phonetics}{陪同}{pei2 tong2}[][HSK 6]
    \definition{v.}{acompanhar; acompanhar alguém para fazer uma atividade ou trabalhar junto}
  \end{Phonetics}
\end{Entry}

\begin{Entry}{陵}{10}{⾩}
  \begin{Phonetics}{陵}{ling2}
    \definition*{s.}{Sobrenome Ling}
    \definition{s.}{colina; monte | túmulo imperial; mausoléu}
    \definition{v.}{(literário) intimidar; violar}
  \end{Phonetics}
\end{Entry}

\begin{Entry}{陵园}{10,7}{⾩、⼞}
  \begin{Phonetics}{陵园}{ling2yuan2}
    \definition{s.}{cemitério}
  \end{Phonetics}
\end{Entry}

\begin{Entry}{陷}{10}{⾩}
  \begin{Phonetics}{陷}{xian4}
    \definition[个]{s.}{armadilha; cilada | defeito | deficiência; desvantagem}
    \definition{v.}{ficar preso (ou atolado); enredar | afundar; desabar | acusar falsamente; incriminar; armar | (de uma cidade, etc.) ser capturado; cair | ser enquadrado; ser capturado}
  \end{Phonetics}
\end{Entry}

\begin{Entry}{陷入}{10,2}{⾩、⼊}
  \begin{Phonetics}{陷入}{xian4ru4}[][HSK 6]
    \definition{v.}{afundar em; cair em; cair em uma situação desfavorável | estar perdido em; estar profundamente em; estar imerso em; metaforicamente, estar profundamente imerso em (uma situação ou pensamento) | estar atolado (lama fofa, areia, etc.)}
  \end{Phonetics}
\end{Entry}

\begin{Entry}{随}{11}{⾩}
  \begin{Phonetics}{随}{sui2}[][HSK 3]
    \definition*{s.}{Sobrenome Sui}
    \definition{adv.}{fazer algo imediatamente assim que ocorre, sem demora ou hesitação; usado antes de dois verbos ou frases verbais para indicar que a última ação segue a anterior}
    \definition{prep.}{junto com (alguma outra ação) | apresentando as condições das quais a ação depende}
    \definition{v.}{seguir; vir (ou ir) junto com | concordar com; adaptar-se a | deixar (alguém fazer o que quiser) | (dialeto) parecer-se com; assemelhar-se a | seguir ou agir de acordo com a condição ou circunstância da qual a ação depende}
  \end{Phonetics}
\end{Entry}

\begin{Entry}{随手}{11,4}{⾩、⼿}
  \begin{Phonetics}{随手}{sui2shou3}[][HSK 4]
    \definition{adv.}{convenientemente; sem problemas adicionais; casualmente}
  \end{Phonetics}
\end{Entry}

\begin{Entry}{随处}{11,5}{⾩、⼡}
  \begin{Phonetics}{随处}{sui2chu4}
    \definition{adv.}{em qualquer lugar}
  \end{Phonetics}
\end{Entry}

\begin{Entry}{随后}{11,6}{⾩、⼝}
  \begin{Phonetics}{随后}{sui2 hou4}[][HSK 5]
    \definition{adv.}{logo em seguida; logo depois; indica que segue imediatamente após a ação ou situação anterior (geralmente usado em conjunto com 就)}
  \seealsoref{就}{jiu4}
  \end{Phonetics}
\end{Entry}

\begin{Entry}{随地}{11,6}{⾩、⼟}
  \begin{Phonetics}{随地}{sui2di4}
    \definition{adv.}{qualquer lugar | todo lugar}
  \end{Phonetics}
\end{Entry}

\begin{Entry}{随机存取记忆体}{11,6,6,8,5,4,7}{⾩、⽊、⼦、⼜、⾔、⼼、⼈}
  \begin{Phonetics}{随机存取记忆体}{sui2ji1cun2qu3ji4yi4ti3}
    \definition{s.}{RAM (\emph{random access memory})}
  \seealsoref{内存}{nei4cun2}
  \seealsoref{随机存取存储器}{sui2ji1cun2qu3cun2chu3qi4}
  \end{Phonetics}
\end{Entry}

\begin{Entry}{随机存取存储器}{11,6,6,8,6,12,16}{⾩、⽊、⼦、⼜、⼦、⼈、⼝}
  \begin{Phonetics}{随机存取存储器}{sui2ji1cun2qu3cun2chu3qi4}
    \definition{s.}{RAM (\emph{random access memory})}
  \seealsoref{内存}{nei4cun2}
  \seealsoref{随机存取记忆体}{sui2ji1cun2qu3ji4yi4ti3}
  \end{Phonetics}
\end{Entry}

\begin{Entry}{随时}{11,7}{⾩、⽇}
  \begin{Phonetics}{随时}{sui2shi2}[][HSK 2]
    \definition{adv.}{a qualquer momento; em todos os momentos}
  \end{Phonetics}
\end{Entry}

\begin{Entry}{随便}{11,9}{⾩、⼈}
  \begin{Phonetics}{随便}{sui2bian4}[][HSK 2]
    \definition{adj.}{relaxado; descontraído; sem restrições; sem limitações | aleatório; casual; descuidado; indiferente; distraído, não pensa bem antes de falar ou agir | casual; informal; não dá importância aos detalhes}
    \definition{conj.}{qualquer; qualquer que seja; não importa}
    \definition{v.}{deixar alguém à vontade}
  \end{Phonetics}
\end{Entry}

\begin{Entry}{随着}{11,11}{⾩、⽬}
  \begin{Phonetics}{随着}{sui2zhe5}[][HSK 5]
    \definition{prep.}{junto com; na esteira de; em sintonia com; usado no início da frase ou antes do verbo, indica as condições necessárias para que uma ação, comportamento ou evento ocorra}
  \end{Phonetics}
\end{Entry}

\begin{Entry}{随意}{11,13}{⾩、⼼}
  \begin{Phonetics}{随意}{sui2yi4}[][HSK 5]
    \definition{adj.}{aleatório; casual; à vontade; como se deseja}
  \end{Phonetics}
\end{Entry}

\begin{Entry}{隐}{11}{⾩}
  \begin{Phonetics}{隐}{yin3}[][HSK 6]
    \definition*{s.}{Sobrenome Yin}
    \definition{adj.}{escondido; escondido profundamente | latente; adormecido; à espreita}
    \definition{pref.}{cripto-}
    \definition{s.}{segredo; assuntos ocultos}
    \definition{v.}{esconder; esconder da vista; ocultar}
  \end{Phonetics}
\end{Entry}

\begin{Entry}{隐私}{11,7}{⾩、⽲}
  \begin{Phonetics}{隐私}{yin3si1}[][HSK 6]
    \definition[点,些]{s.}{privacidade; segredos de alguém; assuntos pessoais que você não quer contar ou tornar públicos}
  \end{Phonetics}
\end{Entry}

\begin{Entry}{隐藏}{11,17}{⾩、⾋}
  \begin{Phonetics}{隐藏}{yin3 cang2}[][HSK 6]
    \definition{v.}{esconder; ocultar}
  \end{Phonetics}
\end{Entry}

\begin{Entry}{隔}{12}{⾩}
  \begin{Phonetics}{隔}{ge2}[][HSK 4]
    \definition{adj.}{seguinte; vizinho}
    \definition{v.}{separar; cortar; dividir; particionar | estar a uma distância de, após ou em um intervalo de | ficar de pé ou deitar entre}
  \end{Phonetics}
\end{Entry}

\begin{Entry}{隔开}{12,4}{⾩、⼶}
  \begin{Phonetics}{隔开}{ge2 kai1}[][HSK 4]
    \definition{v.}{separar; manter separado; barricar; separar completamente duas pessoas (ou coisas) ou duas partes de uma coisa que estão intimamente unidas}
  \end{Phonetics}
\end{Entry}

\begin{Entry}{隔壁}{12,16}{⾩、⼟}
  \begin{Phonetics}{隔壁}{ge2bi4}[][HSK 5]
    \definition{s.}{vizinho; casas ou pessoas vizinhas | septo; distante (socialmente distante) | anteparo; partição}
  \end{Phonetics}
\end{Entry}

\begin{Entry}{障}{13}{⾩}
  \begin{Phonetics}{障}{zhang4}
    \definition{s.}{barreira; bloco; obstáculo; obstruções}
    \definition{v.}{atrapalhar; obstruir; bloquear; cobrir}
  \end{Phonetics}
\end{Entry}

\begin{Entry}{障碍}{13,13}{⾩、⽯}
  \begin{Phonetics}{障碍}{zhang4'ai4}[][HSK 6]
    \definition[个,种]{s.}{barreira; obstáculo; bloqueio; obstrução; impedimento}
  \end{Phonetics}
\end{Entry}

\begin{Entry}{隧}{14}{⾩}
  \begin{Phonetics}{隧}{sui4}
    \definition{s.}{túnel; passagem subterrânea | estrada | subúrbios; áreas suburbanas}
    \definition{v.}{virar}
  \end{Phonetics}
\end{Entry}

\begin{Entry}{隧道}{14,12}{⾩、⾡}
  \begin{Phonetics}{隧道}{sui4dao4}
    \definition{s.}{túnel}
  \end{Phonetics}
\end{Entry}

%%%%% EOF %%%%%

