%%%
%%% Radical "⼭"
%%%

\section*{Radical 46: ``⼭''}\addcontentsline{toc}{section}{Radical 46: ⼭}

\begin{Entry}{山}{3}{⼭}[Kangxi 46]
  \begin{Phonetics}{山}{shan1}[][HSK 1]
    \definition*{s.}{Sobrenome Shan}
    \definition[座]{s.}{colina; maciço; montanha | qualquer coisa que se assemelhe a uma montanha | arbustos nos quais os bichos-da-seda tecem seus casulos; referindo-se a casulos de bicho-da-seda | eco; metáfora para um som muito alto}
  \end{Phonetics}
\end{Entry}

\begin{Entry}{山区}{3,4}{⼭、⼖}
  \begin{Phonetics}{山区}{shan1 qu1}[][HSK 5]
    \definition[片]{s.}{área montanhosa; região montanhosa | colina; serra; montanha | distrito montanhoso}
  \end{Phonetics}
\end{Entry}

\begin{Entry}{山东}{3,5}{⼭、⼀}
  \begin{Phonetics}{山东}{shan1dong1}
    \definition*{s.}{Província de Shandong (Shantung) no nordeste da China}
  \end{Phonetics}
\end{Entry}

\begin{Entry}{山羊}{3,6}{⼭、⽺}
  \begin{Phonetics}{山羊}{shan1yang2}
    \definition{s.}{cabra | (ginástica) cavalo de salto de pequeno porte}
  \end{Phonetics}
\end{Entry}

\begin{Entry}{山阴}{3,6}{⼭、⾩}
  \begin{Phonetics}{山阴}{shan1yin1}
    \definition*{s.}{Condado de Shanyin em Shuozhou, Shanxi}
    \definition{s.}{lado norte (ou sombreado) de uma montanha}
  \end{Phonetics}
\end{Entry}

\begin{Entry}{山体}{3,7}{⼭、⼈}
  \begin{Phonetics}{山体}{shan1ti3}
    \definition{s.}{forma de uma montanha}
  \end{Phonetics}
\end{Entry}

\begin{Entry}{山谷}{3,7}{⼭、⾕}
  \begin{Phonetics}{山谷}{shan1 gu3}[][HSK 6]
    \definition[条,个]{s.}{vale; desfiladeiro; ravina; a área baixa e estreita entre duas montanhas geralmente tem riachos no meio}
  \end{Phonetics}
\end{Entry}

\begin{Entry}{山坡}{3,8}{⼭、⼟}
  \begin{Phonetics}{山坡}{shan1 po1}[][HSK 6]
    \definition[个,座,片]{s.}{encosta; encosta da montanha; a inclinação entre o topo da montanha e o terreno plano}
  \end{Phonetics}
\end{Entry}

\begin{Entry}{山顶}{3,8}{⼭、⾴}
  \begin{Phonetics}{山顶}{shan1ding3}
    \definition{s.}{cume da montanha}
  \end{Phonetics}
\end{Entry}

\begin{Entry}{山峰}{3,10}{⼭、⼭}
  \begin{Phonetics}{山峰}{shan1 feng1}[][HSK 6]
    \definition[座,个]{s.}{pico (montanha); topo alto e pontudo da montanha}
  \end{Phonetics}
\end{Entry}

\begin{Entry}{山寨}{3,14}{⼭、⼧}
  \begin{Phonetics}{山寨}{shan1zhai4}
    \definition{s.}{fortaleza fortificada da vila | fortaleza da montanha (especialmente de bandidos) | falsificação | imitação | (fig.) pechincha}
  \end{Phonetics}
\end{Entry}

\begin{Entry}{岁}{6}{⼭}
  \begin{Phonetics}{岁}{sui4}[][HSK 1]
    \definition{clas.}{usado para anos (de idade)}
    \definition{s.}{ano (literário) | colheita do ano (literário) | idade | tempo (literário) | ano (de idade) | ano (para as colheitas)}
  \end{Phonetics}
\end{Entry}

\begin{Entry}{岁月}{6,4}{⼭、⽉}
  \begin{Phonetics}{岁月}{sui4yue4}[][HSK 5]
    \definition[段,番]{s.}{anos; ano e mês; refere-se a tempo em geral}
  \end{Phonetics}
\end{Entry}

\begin{Entry}{岁数}{6,13}{⼭、⽁}
  \begin{Phonetics}{岁数}{sui4 shu4}[][HSK 6]
    \definition{s.}{idade; anos; a idade de uma pessoa}
  \end{Phonetics}
\end{Entry}

\begin{Entry}{岂}{6}{⼭}
  \begin{Phonetics}{岂}{qi3}
    \definition*{s.}{Sobrenome Qi}
    \definition{adv.}{Litarário: expressa uma pergunta retórica, equivalente a 哪里, 怎么 e 难道}
  \seealsoref{哪里}{na3 li3}
  \seealsoref{难道}{nan2dao4}
  \seealsoref{怎么}{zen3me5}
  \end{Phonetics}
\end{Entry}

\begin{Entry}{岂有此理}{6,6,6,11}{⼭、⽉、⽌、⽟}
  \begin{Phonetics}{岂有此理}{qi3you3ci3li3}
    \definition{interj.}{Que exorbitante! | Absurdo! | Como isso pode ser assim? | Ridículo!}
  \end{Phonetics}
\end{Entry}

\begin{Entry}{岗}{7}{⼭}
  \begin{Phonetics}{岗}{gang3}
    \definition{s.}{outeiro; monte | crista; vergão (no rosto, pele, etc.) | sentinela; posto | trabalho | batida policial}
  \end{Phonetics}
\end{Entry}

\begin{Entry}{岗位}{7,7}{⼭、⼈}
  \begin{Phonetics}{岗位}{gang3 wei4}[][HSK 6]
    \definition[个,类]{s.}{posto; estação; originalmente se refere ao local guardado pelos militares e pela polícia, agora se refere a uma posição geral}
  \end{Phonetics}
\end{Entry}

\begin{Entry}{岛}{7}{⼭}
  \begin{Phonetics}{岛}{dao3}[][HSK 6]
    \definition[个,座]{s.}{ilha; uma massa de terra menor que um continente cercada por água}
  \end{Phonetics}
\end{Entry}

\begin{Entry}{岛屿}{7,6}{⼭、⼭}
  \begin{Phonetics}{岛屿}{dao3yu3}[][HSK 7-9]
    \definition[座,些,群]{s.}{ilha; ilhota}
  \end{Phonetics}
\end{Entry}

\begin{Entry}{岭}{8}{⼭}
  \begin{Phonetics}{岭}{ling3}
    \definition{s.}{cordilheira}
  \end{Phonetics}
\end{Entry}

\begin{Entry}{岸}{8}{⼭}
  \begin{Phonetics}{岸}{an4}[][HSK 5]
    \definition{adj.}{arrogante; orgulhoso; grandioso (de maneira sombria ou condescendente)}
    \definition[条,道,段,面]{s.}{margem; costa; litoral; terreno à beira da água}
  \end{Phonetics}
\end{Entry}

\begin{Entry}{岸上}{8,3}{⼭、⼀}
  \begin{Phonetics}{岸上}{an4 shang4}[][HSK 5]
    \definition{s.}{em terra; costa; margem | na margem do rio; na beira do rio}
  \end{Phonetics}
\end{Entry}

\begin{Entry}{峰}{10}{⼭}
  \begin{Phonetics}{峰}{feng1}
    \definition{clas.}{usado para camelos}
    \definition{s.}{pico; cume; o pico proeminente de uma montanha | coisa parecida com um pico; coisas em forma de montanhas}
  \end{Phonetics}
\end{Entry}

\begin{Entry}{峰会}{10,6}{⼭、⼈}
  \begin{Phonetics}{峰会}{feng1 hui4}[][HSK 6]
    \definition{s.}{cúpula; reunião de cúpula}
  \end{Phonetics}
\end{Entry}

\begin{Entry}{峰回路转}{10,6,13,8}{⼭、⼞、⾜、⾞}
  \begin{Phonetics}{峰回路转}{feng1hui2-lu4zhuan3}[][HSK 7-9]
    \definition{expr.}{cume em meio a elevações circundantes e estradas sinuosas;  (estrada de montanha) torcendo e virando; a estrada da montanha serpenteia em torno de cada novo pico | boa (ou nova) reviravolta nos acontecimentos; uma oportunidade surgiu inesperadamente; as coisas tomaram um novo rumo}
  \end{Phonetics}
\end{Entry}

\begin{Entry}{崇}{11}{⼭}
  \begin{Phonetics}{崇}{chong2}
    \definition*{s.}{Sobrenome Chong}
    \definition{adj.}{alto; elevado; sublime}
    \definition{v.}{adorar; reverenciar; venerar; estimar | respeitar}
  \end{Phonetics}
\end{Entry}

\begin{Entry}{崇尚}{11,8}{⼭、⼩}
  \begin{Phonetics}{崇尚}{chong2shang4}[][HSK 7-9]
    \definition{v.}{sustentar; defender; valorizar}[我们崇尚公平与正义。===Nós defendemos a justiça e a equidade.]
  \end{Phonetics}
\end{Entry}

\begin{Entry}{崇拜}{11,9}{⼭、⼿}
  \begin{Phonetics}{崇拜}{chong2bai4}[][HSK 6]
    \definition{v.}{adorar; idolatrar; venerar}
  \end{Phonetics}
\end{Entry}

\begin{Entry}{崇高}{11,10}{⼭、⾼}
  \begin{Phonetics}{崇高}{chong2gao1}[][HSK 7-9]
    \definition{adj.}{alto; elevado; sublime; nobre}
  \end{Phonetics}
\end{Entry}

\begin{Entry}{崖}{11}{⼭}
  \begin{Phonetics}{崖}{ya2}
    \definition{s.}{precipício | penhasco}
  \end{Phonetics}
\end{Entry}

\begin{Entry}{崩}{11}{⼭}
  \begin{Phonetics}{崩}{beng1}
    \definition{v.}{colapsar |  estourar; quebrar | atingir por explosão | matar atirando; atirar; executar | (de um imperador) morrer | rachar; romper | atingir | executar atirando}
  \end{Phonetics}
\end{Entry}

\begin{Entry}{崩溃}{11,12}{⼭、⽔}
  \begin{Phonetics}{崩溃}{beng1kui4}[][HSK 7-9]
    \definition{v.}{colapsar; desmoronar; cair aos pedaços; as coisas estão destruídas; as emoções das pessoas estão fora de controle}
  \end{Phonetics}
\end{Entry}

\begin{Entry}{巅}{19}{⼭}
  \begin{Phonetics}{巅}{dian1}
    \definition[个]{s.}{pico da montanha; cume; topo da montanha}
  \end{Phonetics}
\end{Entry}

\begin{Entry}{巅峰}{19,10}{⼭、⼭}
  \begin{Phonetics}{巅峰}{dian1feng1}[][HSK 7-9]
    \definition{s.}{um cume; um pico de montanha}
  \end{Phonetics}
\end{Entry}

%%%%% EOF %%%%%

