%%%
%%% Radical "⼢"
%%%

\section*{Radical 35: ``⼢''}\addcontentsline{toc}{section}{Radical 35: ⼢}

\begin{Entry}{复}{9}{⼢}
  \begin{Phonetics}{复}{fu4}
    \definition*{s.}{Sobrenome Fu}
    \definition{adj.}{composto; complexo; nem um único; dois ou mais}
    \definition{adv.}{de novo; novamente; indica o reaparecimento de uma situação, equivalente a 再}
    \definition{s.}{jaqueta; roupas forradas}
    \definition{v.}{virar; virar-se | responder; retornar | recuperar; retornar a; restaurar | vingar | duplicar; repetir}
  \seealsoref{再}{zai4}
  \end{Phonetics}
\end{Entry}

\begin{Entry}{复习}{9,3}{⼢、⼄}
  \begin{Phonetics}{复习}{fu4xi2}[][HSK 2]
    \definition{s.}{revisão}
    \definition{v.}{revisar; corrigir (lições, etc.); repetir o que já aprendeu para consolidar o conhecimento}
  \end{Phonetics}
\end{Entry}

\begin{Entry}{复元}{9,4}{⼢、⼉}
  \begin{Phonetics}{复元}{fu4/yuan2}
    \variantof{复原}
  \end{Phonetics}
\end{Entry}

\begin{Entry}{复印}{9,5}{⼢、⼙}
  \begin{Phonetics}{复印}{fu4yin4}[][HSK 3]
    \definition{v.}{fotografar; fotocopiar; duplicar; sem passar pelo processo de impressão, obter uma cópia diretamente do original (geralmente referindo-se à cópia feita com uma copiadora)}
  \end{Phonetics}
\end{Entry}

\begin{Entry}{复发}{9,5}{⼢、⼜}
  \begin{Phonetics}{复发}{fu4fa1}[][HSK 7-9]
    \definition{v.}{ter uma recaída; recorrer | reaparecer; recrudescer | recorrer (de uma doença) | recair (em um antigo estado ruim)}
  \end{Phonetics}
\end{Entry}

\begin{Entry}{复兴}{9,6}{⼢、⼋}
  \begin{Phonetics}{复兴}{fu4xing1}[][HSK 7-9]
    \definition{v.}{reviver; rejuvenescer | reviver; desenvolver-se e tornar-se mais forte}
  \end{Phonetics}
\end{Entry}

\begin{Entry}{复合}{9,6}{⼢、⼝}
  \begin{Phonetics}{复合}{fu4he2}[][HSK 7-9]
    \definition{v.}{compor; tornar complexo; combinar; juntar | recombinar; juntar; metáfora para estarem juntos novamente depois de estarem separados}
  \end{Phonetics}
\end{Entry}

\begin{Entry}{复杂}{9,6}{⼢、⽊}
  \begin{Phonetics}{复杂}{fu4za2}[][HSK 3]
    \definition{adj.}{complexo; complicado; em oposição a 单纯 e 简单}
  \seealsoref{单纯}{dan1chun2}
  \seealsoref{简单}{jian3dan1}
  \end{Phonetics}
\end{Entry}

\begin{Entry}{复苏}{9,7}{⼢、⾋}
  \begin{Phonetics}{复苏}{fu4 su1}[][HSK 6]
    \definition{s.}{recuperação}
    \definition{v.}{reviver; recuperar; ressuscitar; voltar à vida}
  \end{Phonetics}
\end{Entry}

\begin{Entry}{复制}{9,8}{⼢、⼑}
  \begin{Phonetics}{复制}{fu4zhi4}[][HSK 4]
    \definition{v.}{copiar; duplicar; reproduzir; fazer uma cópia de; fazer uma cópia do original ou reproduzi-lo, reimprimi-lo ou copiá-lo em sua forma original (geralmente referindo-se a relíquias culturais ou obras de arte)}
  \end{Phonetics}
\end{Entry}

\begin{Entry}{复刻}{9,8}{⼢、⼑}
  \begin{Phonetics}{复刻}{fu4ke4}
    \definition{v.}{reimprimir (um trabalho que esteve fora do catálogo) | reeditar (um disco de vinil, um CD, etc.) | replicar | recriar | (empréstimo linguístico) (computação) \emph{fork}}
  \end{Phonetics}
\end{Entry}

\begin{Entry}{复查}{9,9}{⼢、⽊}
  \begin{Phonetics}{复查}{fu4cha2}[][HSK 7-9]
    \definition{v.}{verificar novamente; reexaminar; revisar}
  \end{Phonetics}
\end{Entry}

\begin{Entry}{复活}{9,9}{⼢、⽔}
  \begin{Phonetics}{复活}{fu4huo2}[][HSK 7-9]
    \definition{s.}{ressurreição (cristianismo)}
    \definition{v.}{reviver; voltar à vida; morrer e voltar à vida, frequentemente usado como metáfora}
  \end{Phonetics}
\end{Entry}

\begin{Entry}{复活节}{9,9,5}{⼢、⽔、⾋}
  \begin{Phonetics}{复活节}{fu4huo2jie2}
    \definition*{s.}{Páscoa; festival cristão que comemora a ressurreição de Jesus ocorre no primeiro domingo após a primeira lua cheia após o equinócio da primavera}
  \end{Phonetics}
\end{Entry}

\begin{Entry}{复原}{9,10}{⼢、⼚}
  \begin{Phonetics}{复原}{fu4/yuan2}[][HSK 7-9]
    \definition{s.}{reconversão; recuperação; redefinição; reabilitação; restauração; recura; analepsia; analepse}
    \definition{v.+compl.}{recuperar-se de uma doença; ter a saúde restaurada | restaurar; reabilitar}
  \end{Phonetics}
\end{Entry}

\begin{Entry}{夏}{10}{⼢}
  \begin{Phonetics}{夏}{xia4}
    \definition*{s.}{Dinastia Xia (2070-1600 a.C.) | China; refere-se à China | Sobrenome Xia}
    \definition{s.}{verão}
  \end{Phonetics}
\end{Entry}

\begin{Entry}{夏天}{10,4}{⼢、⼤}
  \begin{Phonetics}{夏天}{xia4 tian1}[][HSK 2]
    \definition[个]{s.}{verão}
  \end{Phonetics}
\end{Entry}

\begin{Entry}{夏日}{10,4}{⼢、⽇}
  \begin{Phonetics}{夏日}{xia4ri4}
    \definition{s.}{horário de verão}
  \end{Phonetics}
\end{Entry}

\begin{Entry}{夏季}{10,8}{⼢、⼦}
  \begin{Phonetics}{夏季}{xia4 ji4}[][HSK 4]
    \definition[个]{s.}{verão; segundo trimestre do ano, habitualmente chamado na China de período de três meses, do início do verão ao início do outono, também chamado de ``quarto, quinto e sexto'' meses do calendário lunar}
  \end{Phonetics}
\end{Entry}

%%%%% EOF %%%%%

