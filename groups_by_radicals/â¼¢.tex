%%%
%%% Radical "⼢"
%%%

\section*{Radical 35: ``⼢''}\addcontentsline{toc}{section}{Radical 35: ⼢}

\begin{entry}{复}{9}{⼢}
  \begin{phonetics}{复}{fu4}
    \definition*{s.}{sobrenome Fu}
    \definition{adj.}{composto; complexo; nem um único; dois ou mais}
    \definition{adv.}{de novo; novamente; indica o reaparecimento de uma situação, equivalente a 再}
    \definition{s.}{jaqueta; roupas forradas}
    \definition{v.}{virar; virar-se | responder; retornar | recuperar; retornar a; restaurar | vingar | duplicar; repetir}
  \seealsoref{再}{zai4}
  \end{phonetics}
\end{entry}

\begin{entry}{复习}{9,3}{⼢、⼄}
  \begin{phonetics}{复习}{fu4xi2}[][HSK 2]
    \definition{s.}{revisão}
    \definition{v.}{revisar; corrigir (lições, etc.); repetir o que já aprendeu para consolidar o conhecimento}
  \end{phonetics}
\end{entry}

\begin{entry}{复印}{9,5}{⼢、⼙}
  \begin{phonetics}{复印}{fu4yin4}[][HSK 3]
    \definition{v.}{fotografar; fotocopiar; duplicar; sem passar pelo processo de impressão, obter uma cópia diretamente do original (geralmente referindo-se à cópia feita com uma copiadora)}
  \end{phonetics}
\end{entry}

\begin{entry}{复杂}{9,6}{⼢、⽊}
  \begin{phonetics}{复杂}{fu4za2}[][HSK 3]
    \definition{adj.}{complexo; complicado; em oposição a 单纯 e 简单}
  \seealsoref{单纯}{dan1chun2}
  \seealsoref{简单}{jian3dan1}
  \end{phonetics}
\end{entry}

\begin{entry}{复制}{9,8}{⼢、⼑}
  \begin{phonetics}{复制}{fu4zhi4}[][HSK 4]
    \definition{v.}{copiar; duplicar; reproduzir; fazer uma cópia de; fazer uma cópia do original ou reproduzi-lo, reimprimi-lo ou copiá-lo em sua forma original (geralmente referindo-se a relíquias culturais ou obras de arte)}
  \end{phonetics}
\end{entry}

\begin{entry}{复刻}{9,8}{⼢、⼑}
  \begin{phonetics}{复刻}{fu4ke4}
    \definition{v.}{reimprimir (um trabalho que esteve fora do catálogo) | reeditar (um disco de vinil, um CD, etc.) | replicar | recriar | (empréstimo linguístico) (computação) \emph{fork}}
  \end{phonetics}
\end{entry}

\begin{entry}{复活节}{9,9,5}{⼢、⽔、⾋}
  \begin{phonetics}{复活节}{fu4huo2jie2}
    \definition*{s.}{Páscoa}
  \end{phonetics}
\end{entry}

\begin{entry}{夏天}{10,4}{⼢、⼤}
  \begin{phonetics}{夏天}{xia4 tian1}[][HSK 2]
    \definition[个]{s.}{verão}
  \end{phonetics}
\end{entry}

\begin{entry}{夏日}{10,4}{⼢、⽇}
  \begin{phonetics}{夏日}{xia4ri4}
    \definition{s.}{horário de verão}
  \end{phonetics}
\end{entry}

\begin{entry}{夏季}{10,8}{⼢、⼦}
  \begin{phonetics}{夏季}{xia4 ji4}[][HSK 4]
    \definition{s.}{verão; segundo trimestre do ano, habitualmente chamado na China de período de três meses, do início do verão ao início do outono, também chamado de ``quarto, quinto e sexto'' meses do calendário lunar}
  \end{phonetics}
\end{entry}

%%%%% EOF %%%%%

