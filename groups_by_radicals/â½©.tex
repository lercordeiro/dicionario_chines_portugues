%%%
%%% Radical "⽩"
%%%

\section*{Radical 106: ``⽩''}\addcontentsline{toc}{section}{Radical 106: ⽩}

\begin{entry}{白}{5}{⽩}[Kangxi 106]
  \begin{phonetics}{白}{bai2}[][HSK 1,3]
    \definition*{s.}{sobrenome Bai}
    \definition{adj.}{branco | claro | puro; claro; simples; sem mistura; em branco | branco (como símbolo de reação) | escrito incorretamente ou pronunciado incorretamente | grátis; sem custos}
    \definition{adv.}{em vão; sem propósito; sem resultados}
    \definition{s.}{parte falada em ópera, etc.; frases de peças de teatro, etc. | dialeto local | funeral}
    \definition{v.}{explicar; apresentar; esclarecer; declarar | branquear | olhar para as pessoas com o branco dos olhos (olhar vazio, de desaprovação)}
  \end{phonetics}
\end{entry}

\begin{entry}{白天}{5,4}{⽩、⼤}
  \begin{phonetics}{白天}{bai2 tian1}[][HSK 1]
    \definition{adv.}{dia | de dia}
    \definition[个]{s.}{dia}
  \end{phonetics}
\end{entry}

\begin{entry}{白色}{5,6}{⽩、⾊}
  \begin{phonetics}{白色}{bai2 se4}[][HSK 2]
    \definition{s.}{cor branca}
  \end{phonetics}
\end{entry}

\begin{entry}{白苋}{5,7}{⽩、⾋}
  \begin{phonetics}{白苋}{bai2xian4}
    \definition{s.}{amaranto branco | brotos e folhas tenras de espinafre chinês usados como alimento}
  \end{phonetics}
\end{entry}

\begin{entry}{白拣}{5,8}{⽩、⼿}
  \begin{phonetics}{白拣}{bai2jian3}
    \definition{s.}{uma escolha barata}
    \definition{v.}{escolher algo que não custa nada}
  \end{phonetics}
\end{entry}

\begin{entry}{白菜}{5,11}{⽩、⾋}
  \begin{phonetics}{白菜}{bai2 cai4}[][HSK 3]
    \definition[棵,个]{s.}{acelga | repolho chinês}
  \end{phonetics}
\end{entry}

\begin{entry}{白萝卜}{5,11,2}{⽩、⾋、⼘}
  \begin{phonetics}{白萝卜}{bai2luo2bo5}
    \definition{s.}{rabanete branco}
  \end{phonetics}
\end{entry}

\begin{entry}{白蛋白}{5,11,5}{⽩、⾍、⽩}
  \begin{phonetics}{白蛋白}{bai2dan4bai2}
    \definition{s.}{albumina}
  \end{phonetics}
\end{entry}

\begin{entry}{白鹄}{5,12}{⽩、⿃}
  \begin{phonetics}{白鹄}{bai2hu2}
    \definition{s.}{cisne branco}
  \end{phonetics}
\end{entry}

\begin{entry}{白痴}{5,13}{⽩、⽧}
  \begin{phonetics}{白痴}{bai2chi1}
    \definition{adj./s.}{estúpido | imbecil}
  \end{phonetics}
\end{entry}

\begin{entry}{百}{6}{⽩}
  \begin{phonetics}{百}{bai3}[][HSK 1]
    \definition*{s.}{sobrenome Bai}
    \definition{num.}{cem; 100 | centena | cento}
  \end{phonetics}
\end{entry}

\begin{entry}{百分}{6,4}{⽩、⼑}
  \begin{phonetics}{百分}{bai3fen1}
    \definition{num.}{por cento}
    \definition{s.}{porcentagem}
  \end{phonetics}
\end{entry}

\begin{entry}{百货}{6,8}{⽩、⾙}
  \begin{phonetics}{百货}{bai3 huo4}[][HSK 4]
    \definition{s.}{mercadorias em geral; loja de departamentos; um termo geral para bens que incluem principalmente roupas, utensílios e necessidades diárias gerais}
  \end{phonetics}
\end{entry}

\begin{entry}{百般}{6,10}{⽩、⾈}
  \begin{phonetics}{百般}{bai3ban1}
    \definition{adv.}{de todas as maneiras possíveis | por todos os meios}
  \end{phonetics}
\end{entry}

\begin{entry}{的}{8}{⽩}
  \begin{phonetics}{的}{de5}
    \definition{part.}{de | partícula usada em possessivos | utilizada entre adjetivos e substantivos (opcional se o adjetivo possui apenas um caracter) | usado após um atributo | usado para formar uma expressão nominal | usado no final de uma frase declarativa para dar ênfase}
  \end{phonetics}
  \begin{phonetics}{的}{di1}
    \definition{s.}{abreviação de 的士: um táxi}
  \seealsoref{的士}{di1shi4}
  \end{phonetics}
  \begin{phonetics}{的}{di2}
    \definition{adv.}{realmente e verdadeiramente}
  \end{phonetics}
  \begin{phonetics}{的}{di4}
    \definition{adj.}{objetivo | claro}
  \end{phonetics}
\end{entry}

\begin{entry}{的士}{8,3}{⽩、⼠}
  \begin{phonetics}{的士}{di1shi4}
    \definition{s.}{(empréstimo linguístico) táxi}
  \end{phonetics}
\end{entry}

\begin{entry}{的话}{8,8}{⽩、⾔}
  \begin{phonetics}{的话}{de5 hua4}[][HSK 2]
    \definition{part.}{se | no caso | suponha que}
  \end{phonetics}
\end{entry}

\begin{entry}{的确}{8,12}{⽩、⽯}
  \begin{phonetics}{的确}{di2que4}[][HSK 4]
    \definition{adv.}{realmente; de fato, ao expressar certeza sobre a situação}
  \end{phonetics}
\end{entry}

\begin{entry}{皆}{9}{⽩}
  \begin{phonetics}{皆}{jie1}
    \definition{adv.}{todos | em todos os casos}
  \end{phonetics}
\end{entry}

\begin{entry}{皇帝}{9,9}{⽩、⼱}
  \begin{phonetics}{皇帝}{huang2di4}
    \definition[个]{s.}{imperador}
  \end{phonetics}
\end{entry}

%%%%% EOF %%%%%

