%%%
%%% Radical "⽩"
%%%

\section*{Radical 106: ``⽩''}\addcontentsline{toc}{section}{Radical 106: ⽩}

\begin{Entry}{白}{5}{⽩}[Kangxi 106]
  \begin{Phonetics}{白}{bai2}[][HSK 1,3]
    \definition*{s.}{Sobrenome Bai}
    \definition{adj.}{branco | claro; entendível; compreendível | puro; claro; simples; sem mistura; em branco | branco (como símbolo de reação) | escrito incorretamente ou pronunciado incorretamente}
    \definition{adv.}{em vão; sem propósito; sem resultados | gratuito; sem custos}
    \definition{s.}{parte falada em ópera, etc.; frases de peças de teatro, etc. | dialeto local | funeral}
    \definition{v.}{explicar; apresentar; esclarecer; declarar | branquear | olhar para as pessoas com o branco dos olhos (olhar vazio, de desaprovação); olhar para alguém com desdém}
  \end{Phonetics}
\end{Entry}

\begin{Entry}{白天}{5,4}{⽩、⼤}
  \begin{Phonetics}{白天}{bai2 tian1}[][HSK 1]
    \definition{adv.}{dia;  de dia}
    \definition[个]{s.}{dia; horário diurno; durante o dia}
  \end{Phonetics}
\end{Entry}

\begin{Entry}{白色}{5,6}{⽩、⾊}
  \begin{Phonetics}{白色}{bai2 se4}[][HSK 2]
    \definition{s.}{a cor branca}
  \end{Phonetics}
\end{Entry}

\begin{Entry}{白苋}{5,7}{⽩、⾋}
  \begin{Phonetics}{白苋}{bai2xian4}
    \definition{s.}{amaranto branco | brotos e folhas tenras de espinafre chinês usados como alimento}
  \end{Phonetics}
\end{Entry}

\begin{Entry}{白拣}{5,8}{⽩、⼿}
  \begin{Phonetics}{白拣}{bai2jian3}
    \definition{s.}{uma escolha barata}
    \definition{v.}{escolher algo que não custa nada}
  \end{Phonetics}
\end{Entry}

\begin{Entry}{白酒}{5,10}{⽩、⾣}
  \begin{Phonetics}{白酒}{bai2 jiu3}[][HSK 5]
    \definition[瓶,杯,壶]{s.}{aguardente branca; aguardente (geralmente destilada de sorgo ou milho); bebidas destiladas tradicionais chinesas, feitas de sorgo, milho, etc., transparentes e incolores, com alto teor alcoólico}
  \end{Phonetics}
\end{Entry}

\begin{Entry}{白菜}{5,11}{⽩、⾋}
  \begin{Phonetics}{白菜}{bai2 cai4}[][HSK 3]
    \definition[棵,种]{s.}{couve chinesa | \emph{pak choi}, um tipo de couve}
  \end{Phonetics}
\end{Entry}

\begin{Entry}{白萝卜}{5,11,2}{⽩、⾋、⼘}
  \begin{Phonetics}{白萝卜}{bai2luo2bo5}
    \definition{s.}{rabanete branco | \emph{daikon}}
  \end{Phonetics}
\end{Entry}

\begin{Entry}{白蛋白}{5,11,5}{⽩、⾍、⽩}
  \begin{Phonetics}{白蛋白}{bai2dan4bai2}
    \definition{s.}{albumina}
  \end{Phonetics}
\end{Entry}

\begin{Entry}{白领}{5,11}{⽩、⾴}
  \begin{Phonetics}{白领}{bai2 ling3}[][HSK 6]
    \definition[个,名,位,些]{s.}{colarinho branco; trabalhador de colarinho branco; refere-se a funcionários cujo trabalho principal envolve trabalho intelectual, são conhecidos por suas roupas elegantes, colarinhos e camisas brancas; atualmente é frequentemente usado para se referir àqueles que trabalham em cargos de gestão ou técnicos em empresas e ganham salários relativamente altos}
  \end{Phonetics}
\end{Entry}

\begin{Entry}{白鹄}{5,12}{⽩、⿃}
  \begin{Phonetics}{白鹄}{bai2hu2}
    \definition{s.}{cisne branco}
  \end{Phonetics}
\end{Entry}

\begin{Entry}{白痴}{5,13}{⽩、⽧}
  \begin{Phonetics}{白痴}{bai2chi1}
    \definition{adj.}{idiota; uma pessoa que sofre de idiotice; frequentemente usado para menosprezar alguém que é incompetente ou incapaz de fazer as coisas}
    \definition{s.}{idiotice; uma doença caracterizada por retardo mental, demência, fala arrastada, movimentos lentos e até mesmo incapacidade de cuidar de si mesmo}
  \end{Phonetics}
\end{Entry}

\begin{Entry}{百}{6}{⽩}
  \begin{Phonetics}{百}{bai3}[][HSK 1]
    \definition{adj.}{todos; todos os tipos de; multifacetados; numerosos}
    \definition{adv.}{muito; sempre}
    \definition{num.}{cem; 100}
  \end{Phonetics}
\end{Entry}

\begin{Entry}{百分}{6,4}{⽩、⼑}
  \begin{Phonetics}{百分}{bai3fen1}
    \definition{s.}{por cento | nota máxima; pontuação máxima; 100 pontos (em um sistema de classificação de cem pontos) | um jogo específico; um jogo de pôquer}
  \end{Phonetics}
\end{Entry}

\begin{Entry}{百分点}{6,4,9}{⽩、⼑、⽕}
  \begin{Phonetics}{百分点}{bai3 fen1 dian3}[][HSK 6]
    \definition[个]{s.}{ponto percentual; em estatística, um por cento é chamado de ponto percentual}
  \end{Phonetics}
\end{Entry}

\begin{Entry}{百货}{6,8}{⽩、⾙}
  \begin{Phonetics}{百货}{bai3 huo4}[][HSK 4]
    \definition{s.}{mercadorias em geral; loja de departamentos; um termo geral para bens que incluem principalmente roupas, utensílios e necessidades diárias gerais}
  \end{Phonetics}
\end{Entry}

\begin{Entry}{百般}{6,10}{⽩、⾈}
  \begin{Phonetics}{百般}{bai3ban1}
    \definition{adv.}{de todas as maneiras possíveis | por todos os meios}
  \end{Phonetics}
\end{Entry}

\begin{Entry}{的}{8}{⽩}
  \begin{Phonetics}{的}{de5}
    \definition{part.}{usado para indicar posse | formar uma frase nominal ou expressão nominal | substituir a pessoa ou coisa mencionada anteriormente | no final de uma frase declarativa, para dar ênfase; usado após o verbo predicativo, enfatiza o agente da ação, o tempo, o local, etc. | usado no final de uma frase declarativa, expressa afirmação, ênfase, certeza, etc. | indica que alguém obteve uma determinada posição ou status | usado com 是 para indicar predicado ou ênfase; indica que alguém é o objeto da ação | e assim por diante; e assim por diante; e similares; usado após palavras paralelas, significa 等等, 之类 | indica uma ação (o pronome é o objeto da ação); combinado com o verbo anterior, expressa uma ação, e o pronome é o objeto dessa ação}
  \seealsoref{等等}{deng3 deng3}
  \seealsoref{是}{shi4}
  \seealsoref{之类}{zhi1 lei4}
  \end{Phonetics}
  \begin{Phonetics}{的}{di1}
    \definition{s.}{abreviação de 的士: um táxi}
  \seealsoref{的士}{di1shi4}
  \end{Phonetics}
  \begin{Phonetics}{的}{di2}
    \definition{adv.}{verdadeiramente; exatamente; realmente}
  \end{Phonetics}
  \begin{Phonetics}{的}{di4}
    \definition{adj.}{alvo; centro do alvo}
  \end{Phonetics}
\end{Entry}

\begin{Entry}{的士}{8,3}{⽩、⼠}
  \begin{Phonetics}{的士}{di1shi4}
    \definition{s.}{(empréstimo linguístico) táxi}
  \end{Phonetics}
\end{Entry}

\begin{Entry}{的时候}{8,7,10}{⽩、⽇、⼈}
  \begin{Phonetics}{的时候}{de5 shi2hou4}
    \definition{part.}{naquele momento; quando; em; descreve o momento específico em que um evento ocorreu}
  \end{Phonetics}
\end{Entry}

\begin{Entry}{的话}{8,8}{⽩、⾔}
  \begin{Phonetics}{的话}{de5 hua4}[][HSK 2]
    \definition{part.}{se; caso; suponha que; partícula usada após uma frase hipotética para introduzir o texto seguinte}
  \end{Phonetics}
\end{Entry}

\begin{Entry}{的确}{8,12}{⽩、⽯}
  \begin{Phonetics}{的确}{di2que4}[][HSK 4]
    \definition{adv.}{realmente; de fato, ao expressar certeza sobre a situação}
  \end{Phonetics}
\end{Entry}

\begin{Entry}{皆}{9}{⽩}
  \begin{Phonetics}{皆}{jie1}
    \definition{adv.}{todos | em todos os casos}
  \end{Phonetics}
\end{Entry}

\begin{Entry}{皇}{9}{⽩}
  \begin{Phonetics}{皇}{huang2}
    \definition*{s.}{Sobrenome Huang}
    \definition{adj.}{grandioso; magnífico}
    \definition{s.}{imperador, o governante supremo de uma dinastia feudal após a Dinastia Qin; soberano}
  \end{Phonetics}
\end{Entry}

\begin{Entry}{皇帝}{9,9}{⽩、⼱}
  \begin{Phonetics}{皇帝}{huang2di4}[][HSK 6]
    \definition[个,位,任]{s.}{imperador; o título do mais alto governante feudal na China começou com o título de Imperador Qin Shi Huang}
  \end{Phonetics}
\end{Entry}

%%%%% EOF %%%%%

