%%%
%%% Radical "⾔"
%%%

\section*{Radical 149: ``⾔'' (讠、訁)}\addcontentsline{toc}{section}{Radical 149: ⾔、讠、訁}

\begin{Entry}{计}{4}{⾔}
  \begin{Phonetics}{计}{ji4}
    \definition*{s.}{Sobrenome Ji}
    \definition{s.}{medidor; aferidor; indicador; um instrumento para medir ou calcular graus, tempo, etc. | ideia; ardil; estratagema; plano}
    \definition{v.}{contar; calcular; numerar | planejar; traçar; imaginar}
  \end{Phonetics}
\end{Entry}

\begin{Entry}{计划}{4,6}{⾔、⼑}
  \begin{Phonetics}{计划}{ji4hua4}[][HSK 2]
    \definition[个,项]{s.}{plano; projeto; programa; trabalho, ações, conteúdo e etapas previamente definidos}
    \definition{v.}{planejar; traçar um plano}
  \end{Phonetics}
\end{Entry}

\begin{Entry}{计算}{4,14}{⾔、⽵}
  \begin{Phonetics}{计算}{ji4suan4}[][HSK 3]
    \definition{v.}{contar; calcular; computar; enumerar; encontrar a variável desconhecida | planejar; considerar | conspirar secretamente contra os outros; planejar secretamente prejudicar os outros}
  \end{Phonetics}
\end{Entry}

\begin{Entry}{计算机}{4,14,6}{⾔、⽵、⽊}
  \begin{Phonetics}{计算机}{ji4 suan4 ji1}[][HSK 2]
    \definition[部,台]{s.}{computador; calculadora; máquinas capazes de realizar cálculos matemáticos são feitas com dispositivos mecânicos, como calculadoras manuais, outras são feitas com componentes eletrônicos, como computadores eletrônicos}
  \end{Phonetics}
\end{Entry}

\begin{Entry}{计算机程序}{4,14,6,12,7}{⾔、⽵、⽊、⽲、⼴}
  \begin{Phonetics}{计算机程序}{ji4suan4ji1 cheng2xu4}
    \definition{s.}{programa de computador}
  \end{Phonetics}
\end{Entry}

\begin{Entry}{订}{4}{⾔}
  \begin{Phonetics}{订}{ding4}[][HSK 3]
    \definition{v.}{concluir; elaborar; concordar com |assinar (um jornal, etc.); reservar (assentos, ingressos, etc.); encomendar (mercadorias, etc.) | fazer correções; revisar | grampear junto; unir; usar linha ou arame para encadernar páginas soltas ou folhas de papel | julgar; determinar}
  \end{Phonetics}
\end{Entry}

\begin{Entry}{认}{4}{⾔}
  \begin{Phonetics}{认}{ren4}[][HSK 5]
    \definition{v.}{reconhecer; saber; distinguir; identificar | estabelecer uma determinada relação com; adotar | admitir; reconhecer; assumir | comprometer-se a fazer algo | (frequentemente seguido por 了) aceitar como inevitável; resignar-se}
  \seealsoref{了}{le5}
  \end{Phonetics}
\end{Entry}

\begin{Entry}{认为}{4,4}{⾔、⼂}
  \begin{Phonetics}{认为}{ren4wei2}[][HSK 2]
    \definition{v.}{pensar; considerar; manter; julgar; formar uma opinião sobre uma pessoa ou coisa, fazer um julgamento}
  \end{Phonetics}
\end{Entry}

\begin{Entry}{认出}{4,5}{⾔、⼐}
  \begin{Phonetics}{认出}{ren4 chu1}[][HSK 3]
    \definition{v.}{reconhecer; identificar; reconhecer alguém ou algo pela observação ou memória}
  \end{Phonetics}
\end{Entry}

\begin{Entry}{认可}{4,5}{⾔、⼝}
  \begin{Phonetics}{认可}{ren4ke3}[][HSK 3]
    \definition{v.}{aceitar; aprovar; confirmar; dar força legal a | permitir; concordar}
  \end{Phonetics}
\end{Entry}

\begin{Entry}{认同}{4,6}{⾔、⼝}
  \begin{Phonetics}{认同}{ren4 tong2}[][HSK 6]
    \definition{v.}{identificar; pensar que a outra pessoa tem algo em comum com você | aprovar; reconhecer}
  \end{Phonetics}
\end{Entry}

\begin{Entry}{认识}{4,7}{⾔、⾔}
  \begin{Phonetics}{认识}{ren4shi5}[][HSK 1]
    \definition[份]{s.}{cognição; conhecimento; compreensão; refere-se à reflexão da mente humana sobre o mundo objetivo}
    \definition{v.}{saber; compreender; reconhecer}
  \end{Phonetics}
\end{Entry}

\begin{Entry}{认定}{4,8}{⾔、⼧}
  \begin{Phonetics}{认定}{ren4ding4}[][HSK 5]
    \definition{v.}{afirmar; manter; acreditar firmemente; considerar com certeza | decidir-se por algo; confirmar; chegar a uma conclusão afirmativa}
  \end{Phonetics}
\end{Entry}

\begin{Entry}{认真}{4,10}{⾔、⼗}
  \begin{Phonetics}{认真}{ren4zhen1}[][HSK 1]
    \definition{adj.}{sério; sério e meticuloso}
    \definition{adv.}{seriamente}
    \definition{v.}{levar algo a sério; considerar como verdadeiro; levar a sério}
  \end{Phonetics}
\end{Entry}

\begin{Entry}{认得}{4,11}{⾔、⼻}
  \begin{Phonetics}{认得}{ren4 de5}[][HSK 3]
    \definition{v.}{saber; reconhecer; capacidade de confirmar a pessoa ou coisa que você vê}
  \end{Phonetics}
\end{Entry}

\begin{Entry}{讨}{5}{⾔}
  \begin{Phonetics}{讨}{tao3}
    \definition{v.}{enviar forças armadas para suprimir; enviar uma expedição punitiva contra; enviar exército ou despachar tropas para suprimir ou atacar | denunciar; condenar; censurar | exigir; pedir; implorar por | casar (com uma mulher) | incorrer; convidar | discutir; estudar | provocar; cortejar}
  \end{Phonetics}
\end{Entry}

\begin{Entry}{讨生活}{5,5,9}{⾔、⽣、⽔}
  \begin{Phonetics}{讨生活}{tao3sheng1huo2}
    \definition{v.}{ganhar a vida}
  \end{Phonetics}
\end{Entry}

\begin{Entry}{讨厌}{5,6}{⾔、⼚}
  \begin{Phonetics}{讨厌}{tao3yan4}[][HSK 5]
    \definition{adj.}{desagradável; repugnante; repulsivo; irritante; incômodo}
    \definition{v.}{odiar; não gostar; sentir repulsa por}
  \end{Phonetics}
\end{Entry}

\begin{Entry}{讨论}{5,6}{⾔、⾔}
  \begin{Phonetics}{讨论}{tao3lun4}[][HSK 2]
    \definition{v.}{discutir; conversar sobre; trocar opiniões ou debater as questões levantadas}
  \end{Phonetics}
\end{Entry}

\begin{Entry}{让}{5}{⾔}
  \begin{Phonetics}{让}{rang4}[][HSK 2]
    \definition*{s.}{Sobrenome Rang}
    \definition{prep.}{em uma frase passiva para introduzir o executor da ação | de acordo com; em conformidade com; à luz de; com base em; usado para expressar a opinião subjetiva de alguém}
    \definition{v.}{ceder; recuar; render-se; desistir; admitir | convidar; oferecer | deixar; permitir; fazer | deixar alguém ter algo por um preço justo | ser inferior a; não ser tão bom quanto | ceder; afastar-se | expressar desejos | esquivar-se; evitar; fugir | Usado antes de 我们, indica uma ordem ou sugestão para que todos façam algo juntos}
  \seealsoref{我们}{wo3men5}
  \end{Phonetics}
\end{Entry}

\begin{Entry}{让步}{5,7}{⾔、⽌}
  \begin{Phonetics}{让步}{rang4/bu4}
    \definition{v.+compl.}{fazer uma concessão | entregar | desistir | comprometer}
  \end{Phonetics}
\end{Entry}

\begin{Entry}{让座}{5,10}{⾔、⼴}
  \begin{Phonetics}{让座}{rang4 zuo4}[][HSK 6]
    \definition{v.}{oferecer seu lugar a alguém; ceder seu lugar a alguém | convidar os convidados para se sentarem}
  \end{Phonetics}
\end{Entry}

\begin{Entry}{训}{5}{⾔}
  \begin{Phonetics}{训}{xun4}
    \definition{s.}{instrução; ensinamento; ensino | padrão; modelo; exemplo; regra; diretriz | explicação ou interpretação crítica de um texto | treinamento; exercício}
    \definition{v.}{instruir; admoestar; dar uma palestra a alguém; ensinar | explicar; instruir; explicação do significado da palavra | treinar}
  \end{Phonetics}
\end{Entry}

\begin{Entry}{训诂}{5,7}{⾔、⾔}
  \begin{Phonetics}{训诂}{xun4gu3}
    \definition{s.}{estudos exegéticos (de textos antigos); exegese}
    \definition{v.}{explicação de palavras e frases em livros antigos | interpretar e elaborar glossários e comentários sobre textos clássicos}
  \end{Phonetics}
\end{Entry}

\begin{Entry}{训练}{5,8}{⾔、⽷}
  \begin{Phonetics}{训练}{xun4lian4}[][HSK 3]
    \definition{v.}{treinar; exercitar; planejar e executar de forma sistemática o desenvolvimento de habilidades ou competências específicas}
  \end{Phonetics}
\end{Entry}

\begin{Entry}{议}{5}{⾔}
  \begin{Phonetics}{议}{yi4}
    \definition[个,则,条]{s.}{opinião; visão}
    \definition{v.}{discutir; trocar pontos de vista sobre; conversar sobre | comentar; observar | fofocar; comentar}
  \end{Phonetics}
\end{Entry}

\begin{Entry}{议论}{5,6}{⾔、⾔}
  \begin{Phonetics}{议论}{yi4lun4}[][HSK 4]
    \definition[个]{s.}{comentário; discussão; opiniões ou pontos de vista sobre o que é bom ou ruim, certo ou errado em relação a pessoas ou coisas}
    \definition{v.}{discutir; comentar; falar sobre; expressar opiniões e trocar pontos de vista sobre o bom, o ruim, o certo e o errado de pessoas ou coisas}
  \end{Phonetics}
\end{Entry}

\begin{Entry}{议题}{5,15}{⾔、⾴}
  \begin{Phonetics}{议题}{yi4 ti2}[][HSK 6]
    \definition[项,个]{s.}{assunto; assunto em discussão; tópico para discussão}
  \end{Phonetics}
\end{Entry}

\begin{Entry}{记}{5}{⾔}
  \begin{Phonetics}{记}{ji4}[][HSK 1]
    \definition*{s.}{Sobrenome Ji}
    \definition{clas.}{tapas, palmadas, bofetadas, etc.; usado para indicar o número de vezes que uma determinada ação é realizada}
    \definition{s.}{assinatura; bloco de notas; livro ou artigo que registra fatos | insígnia; indicação; \& comercial; símbolo | marca de nascença; manchas escuras presentes na pele desde o nascimento}
    \definition{v.}{lembrar; ter em mente; guardar na memória; manter a imagem na mente | escrever (anotar); registrar; inscrever}
  \end{Phonetics}
\end{Entry}

\begin{Entry}{记忆}{5,4}{⾔、⼼}
  \begin{Phonetics}{记忆}{ji4yi4}[][HSK 5]
    \definition[段]{s.}{memória; manter em sua mente uma imagem do passado}
    \definition{v.}{recordar; lembrar; lembrar-se ou recordar alguém ou algo do passado}
  \end{Phonetics}
\end{Entry}

\begin{Entry}{记住}{5,7}{⾔、⼈}
  \begin{Phonetics}{记住}{ji4 zhu5}[][HSK 1]
    \definition{v.}{lembrar; aprender de cor; ter em mente; guardar na memória}
  \end{Phonetics}
\end{Entry}

\begin{Entry}{记录}{5,8}{⾔、⼹}
  \begin{Phonetics}{记录}{ji4lu4}[][HSK 3]
    \definition[份,名,位,个]{s.}{notas; registro | anotador; registrador; a pessoa que faz registros}
    \definition{v.}{tomar notas; registrar; escrever o que ouviu ou o que aconteceu; gravar o som ou a imagem com um gravador ou uma câmera de vídeo e transformar em algum tipo de obra}
  \end{Phonetics}
\end{Entry}

\begin{Entry}{记性}{5,8}{⾔、⼼}
  \begin{Phonetics}{记性}{ji4xing5}
    \definition{s.}{memória (habilidade em reter informações)}
  \end{Phonetics}
\end{Entry}

\begin{Entry}{记者}{5,8}{⾔、⽼}
  \begin{Phonetics}{记者}{ji4zhe3}[][HSK 3]
    \definition[群,名,位]{s.}{repórter; correspondente; jornalista; profissionais dedicados a entrevistar e reportar notícias para a mídia}
  \end{Phonetics}
\end{Entry}

\begin{Entry}{记载}{5,10}{⾔、⾞}
  \begin{Phonetics}{记载}{ji4zai3}[][HSK 4]
    \definition[段,种,条]{s.}{registro; conta; artigos e materiais que registram eventos}
    \definition{v.}{registrar; colocar por escrito}
  \end{Phonetics}
\end{Entry}

\begin{Entry}{记得}{5,11}{⾔、⼻}
  \begin{Phonetics}{记得}{ji4de5}[][HSK 1]
    \definition{v.}{lembrar; recordar; lembrar-se; não esquecer | manter algo em mente; (informal) não se esquecer de fazer algo, usado para lembrar}
  \end{Phonetics}
\end{Entry}

\begin{Entry}{讲}{6}{⾔}
  \begin{Phonetics}{讲}{jiang3}[][HSK 2]
    \definition[种]{s.}{palestra; discurso}
    \definition{v.}{contar; falar | explicar; transmitir oralmente; esclarecer | negociar; barganhar | ser exigente com; valorizar; dar importância}
  \end{Phonetics}
\end{Entry}

\begin{Entry}{讲究}{6,7}{⾔、⽳}
  \begin{Phonetics}{讲究}{jiang3jiu5}[][HSK 4]
    \definition{adj.}{requintado; elegante; de bom gosto; exigente com a vida e com outros aspectos, buscando alto nível, qualidade e detalhes}
    \definition{s.}{estudo cuidadoso; algo que merece atenção; elementos e aspectos que merecem atenção especial}
    \definition{v.}{dar ênfase a; ser específico sobre; prestar atenção a}
  \end{Phonetics}
\end{Entry}

\begin{Entry}{讲话}{6,8}{⾔、⾔}
  \begin{Phonetics}{讲话}{jiang3 hua4}[][HSK 2]
    \definition[个]{s.}{discurso; palestra | guia; introdução}
    \definition{v.}{falar; conversar; dirigir-se a alguém | criticar}
  \end{Phonetics}
\end{Entry}

\begin{Entry}{讲述}{6,8}{⾔、⾡}
  \begin{Phonetics}{讲述}{jiang3shu4}
    \definition{v.}{falar sobre | narrar | descrever}
  \end{Phonetics}
\end{Entry}

\begin{Entry}{讲座}{6,10}{⾔、⼴}
  \begin{Phonetics}{讲座}{jiang3zuo4}[][HSK 4]
    \definition[场,次]{s.}{palestra; um curso de palestras; a forma de instrução usada para ensinar um determinado assunto ou tópico, geralmente por meio de palestras ao vivo, seriados de rádio ou televisão ou seriados de jornal.}
  \end{Phonetics}
\end{Entry}

\begin{Entry}{讲课}{6,10}{⾔、⾔}
  \begin{Phonetics}{讲课}{jiang3 ke4}[][HSK 6]
    \definition{v.}{ensinar; dar palestras; proferir uma palestra | dar uma lição (palestra)}
  \end{Phonetics}
\end{Entry}

\begin{Entry}{许}{6}{⾔}
  \begin{Phonetics}{许}{xu3}
    \definition*{s.}{Xu, um estado da Dinastia Zhou | Sobrenome Xu}
    \definition{adv.}{um pouco;  talvez; expressa especulação ou estimativa, equivalente a 或者 ou 可能}
    \definition{part.}{cerca de; aproximadamente; usado depois de certos numerais, frases de quantidade ou 些 ou 少 para indicar um número próximo a um certo número}
    \definition{pron.}{muitos; um monte de}
    \definition{v.}{elogiar; aprovar | prometer; prometer dar antecipadamente; dedicar | permitir; concordar; aprovar | (uma menina) estar prometida a; refere-se especificamente ao noivado}
  \seealsoref{或者}{huo4zhe3}
  \seealsoref{可能}{ke3neng2}
  \seealsoref{少}{shao3}
  \seealsoref{些}{xie1}
  \end{Phonetics}
\end{Entry}

\begin{Entry}{许可}{6,5}{⾔、⼝}
  \begin{Phonetics}{许可}{xu3ke3}[][HSK 5]
    \definition{v.}{permitir; autorizar}
  \end{Phonetics}
\end{Entry}

\begin{Entry}{许多}{6,6}{⾔、⼣}
  \begin{Phonetics}{许多}{xu3duo1}[][HSK 2]
    \definition{num.}{muitos; muito; numerosos; uma grande quantidade de}
  \end{Phonetics}
\end{Entry}

\begin{Entry}{论}{6}{⾔}
  \begin{Phonetics}{论}{lun2}
    \definition*{s.}{Os Analectos de Confúcio, registro dos ditos e feitos de Confúcio e seus discípulos}
  \end{Phonetics}
  \begin{Phonetics}{论}{lun4}
    \definition*{s.}{Sobrenome Lun}
    \definition{prep.}{por (uma certa unidade de medida) | de acordo com (um certo sistema ou princípio)}
    \definition{s.}{visão; opinião; declaração | (frequentemente em títulos) dissertação; ensaio; tratado | teoria; doutrina | ideia; palavras ou artigos que analisam e explicam coisas}
    \definition{v.}{discutir; falar sobre; discursar sobre; comentar | mencionar; considerar; falar de | decidir sobre; determinar | decidir sobre a natureza da culpa; punir | argumentar; analisar e explicar coisas | considerar; ponderar; medir; avaliar}
  \end{Phonetics}
\end{Entry}

\begin{Entry}{论文}{6,4}{⾔、⽂}
  \begin{Phonetics}{论文}{lun4wen2}[][HSK 4]
    \definition[篇]{s.}{tese; redação; artigo; artigo que discute ou examina uma questão}
  \end{Phonetics}
\end{Entry}

\begin{Entry}{设}{6}{⾔}
  \begin{Phonetics}{设}{she4}
    \definition*{s.}{Sobrenome She}
    \definition{conj.}{se; no caso | (matemática) dado; suponha; se}
    \definition{v.}{configurar; estabelecer; encontrar; colocar em prática}
  \end{Phonetics}
\end{Entry}

\begin{Entry}{设计}{6,4}{⾔、⾔}
  \begin{Phonetics}{设计}{she4ji4}[][HSK 3]
    \definition[份]{s.}{plano; esquema; refere-se a um plano de design ou a um projeto para um plano, etc.}
    \definition{v.}{planejar; projetar; formular métodos, desenhos, etc. com antecedência, de acordo com determinados requisitos de finalidade, antes de iniciar oficialmente um trabalho | arquitetar; idear; tramar; fazer um plano}
  \end{Phonetics}
\end{Entry}

\begin{Entry}{设计师}{6,4,6}{⾔、⾔、⼱}
  \begin{Phonetics}{设计师}{she4 ji4 shi1}[][HSK 6]
    \definition[个,位,名,些]{s.}{planejador de projeto; designer | arquiteto}
  \end{Phonetics}
\end{Entry}

\begin{Entry}{设立}{6,5}{⾔、⽴}
  \begin{Phonetics}{设立}{she4li4}[][HSK 3]
    \definition{v.}{fundar; estabelecer; começar}
  \end{Phonetics}
\end{Entry}

\begin{Entry}{设备}{6,8}{⾔、⼡}
  \begin{Phonetics}{设备}{she4bei4}[][HSK 3]
    \definition[台,套]{s.}{instalação; equipamento; montagem; um conjunto de edifícios ou equipamentos necessários para executar uma determinada tarefa ou suprir uma determinada necessidade}
  \end{Phonetics}
\end{Entry}

\begin{Entry}{设施}{6,9}{⾔、⽅}
  \begin{Phonetics}{设施}{she4shi1}[][HSK 4]
    \definition{s.}{facilidade; instalação; instituições, sistemas, organizações, edifícios, etc., estabelecidos para realizar um trabalho ou atender a uma necessidade}
  \end{Phonetics}
\end{Entry}

\begin{Entry}{设想}{6,13}{⾔、⼼}
  \begin{Phonetics}{设想}{she4xiang3}[][HSK 5]
    \definition[个,种]{s.}{plano provisório (ou ideia); (item, tipo) refere-se a algo hipotético ou imaginário}
    \definition{v.}{imaginar; prever; conceber; supor | ter consideração por}
  \end{Phonetics}
\end{Entry}

\begin{Entry}{设置}{6,13}{⾔、⽹}
  \begin{Phonetics}{设置}{she4zhi4}[][HSK 4]
    \definition{v.}{estabelecer; colocar em prática; estabelecer ou criar instituições, empregos, profissões ou códigos, etc. | encaixar; ajustar; instalar; configurar; colocar}
  \end{Phonetics}
\end{Entry}

\begin{Entry}{访}{6}{⾔}
  \begin{Phonetics}{访}{fang3}
    \definition{v.}{visitar; fazer uma visita; ligar para | procurar por meio de investigação ou busca; tentar obter; obter uma entrevista | entrevistar | investigar; procurar por meio de investigação (pesquisar)}
  \end{Phonetics}
\end{Entry}

\begin{Entry}{访问}{6,6}{⾔、⾨}
  \begin{Phonetics}{访问}{fang3wen4}[][HSK 3]
    \definition{v.}{visitar; ligar; entrevistar; visitar e conversar com um objetivo específico | visitar um \emph{site}}
  \end{Phonetics}
\end{Entry}

\begin{Entry}{言}{7}{⾔}[Kangxi 149]
  \begin{Phonetics}{言}{yan2}
    \definition*{s.}{Sobrenome Yan}
    \definition{s.}{palavra; discurso; o que foi dito | palavra; caracter; uma frase ou palavra chinesa}
    \definition{v.}{dizer; falar}
  \end{Phonetics}
\end{Entry}

\begin{Entry}{言论}{7,6}{⾔、⾔}
  \begin{Phonetics}{言论}{yan2lun4}
    \definition{s.}{expressão de opinião |  visualizações | comentários | argumentos}
  \end{Phonetics}
\end{Entry}

\begin{Entry}{言语}{7,9}{⾔、⾔}
  \begin{Phonetics}{言语}{yan2 yu3}[][HSK 5]
    \definition{s.}{verbal; fala; linguagem falada; conversa; palavras}
  \end{Phonetics}
\end{Entry}

\begin{Entry}{证}{7}{⾔}
  \begin{Phonetics}{证}{zheng4}[][HSK 3]
    \definition{s.}{evidência; prova; testemunho | certificado; cartão | evidência; testemunha | doença; enfermidade}
    \definition{v.}{provar; demonstrar | verificar}
  \end{Phonetics}
\end{Entry}

\begin{Entry}{证书}{7,4}{⾔、⼄}
  \begin{Phonetics}{证书}{zheng4shu1}[][HSK 5]
    \definition[张,份,些]{s.}{certificado; documentos emitidos por instituições, grupos, etc., que comprovem experiência, nível, honras, poderes, etc.}
  \end{Phonetics}
\end{Entry}

\begin{Entry}{证件}{7,6}{⾔、⼈}
  \begin{Phonetics}{证件}{zheng4jian4}[][HSK 3]
    \definition[个,本,张,份]{s.}{documentos; credenciais; certificado; documentos que comprovem a identidade, experiência, etc., tais como carteira de estudante, carteira de trabalho, diploma de graduação, etc.}
  \end{Phonetics}
\end{Entry}

\begin{Entry}{证实}{7,8}{⾔、⼧}
  \begin{Phonetics}{证实}{zheng4shi2}[][HSK 5]
    \definition{v.}{verificar; afirmar; confirmar; corroborar; demonstrar; autenticar; provar que é verdadeiro}
  \end{Phonetics}
\end{Entry}

\begin{Entry}{证明}{7,8}{⾔、⽇}
  \begin{Phonetics}{证明}{zheng4ming2}[][HSK 3]
    \definition[个,份]{s.}{certificado; atestado; identificação; certificado ou carta de referência; documentos que comprovem identidade, experiência, etc., tais como carteira de estudante, carteira de trabalho, diploma de graduação, etc.}
    \definition{v.}{provar; testemunhar; sustentar; usar materiais confiáveis para demonstrar ou determinar a autenticidade de pessoas ou coisas}
  \end{Phonetics}
\end{Entry}

\begin{Entry}{证据}{7,11}{⾔、⼿}
  \begin{Phonetics}{证据}{zheng4ju4}[][HSK 3]
    \definition{s.}{prova; evidência; testemunho; fatos ou materiais que comprovam a veracidade de algo}
  \end{Phonetics}
\end{Entry}

\begin{Entry}{评}{7}{⾔}
  \begin{Phonetics}{评}{ping2}[][HSK 6]
    \definition*{s.}{Sobrenome Ping}
    \definition{v.}{comentar; criticar; revisar | julgar; avaliar}
  \end{Phonetics}
\end{Entry}

\begin{Entry}{评价}{7,6}{⾔、⼈}
  \begin{Phonetics}{评价}{ping2jia4}[][HSK 3]
    \definition[个,项,条,份]{s.}{avaliação; apreciação; comentários ou opiniões de pessoas sobre alguém ou algo}
    \definition{v.}{estimar valor; avaliar valor}
  \end{Phonetics}
\end{Entry}

\begin{Entry}{评论}{7,6}{⾔、⾔}
  \begin{Phonetics}{评论}{ping2lun4}[][HSK 5]
    \definition[篇,些]{s.}{revisão; comentário; artigos ou comentários críticos}
    \definition{v.}{discutir; comentar sobre algo ou alguém}
  \end{Phonetics}
\end{Entry}

\begin{Entry}{评估}{7,7}{⾔、⼈}
  \begin{Phonetics}{评估}{ping2gu1}[][HSK 5]
    \definition{v.}{estimar; avaliar; apreciar; avaliar e estimar (coisas abstratas)}
  \end{Phonetics}
\end{Entry}

\begin{Entry}{评选}{7,9}{⾔、⾡}
  \begin{Phonetics}{评选}{ping2 xuan3}[][HSK 6]
    \definition{v.}{escolher por meio de avaliação pública; avaliar e eleger}
  \end{Phonetics}
\end{Entry}

\begin{Entry}{诅}{7}{⾔}
  \begin{Phonetics}{诅}{zu3}
    \definition{v.}{amaldiçoar; xingar; imprecar; abusar; desejar algo maldoso | jurar; fazer um voto; prestar juramento}
  \end{Phonetics}
\end{Entry}

\begin{Entry}{诅咒}{7,8}{⾔、⼝}
  \begin{Phonetics}{诅咒}{zu3zhou4}
    \definition{v.}{amaldiçoar}
  \end{Phonetics}
\end{Entry}

\begin{Entry}{识}{7}{⾔}
  \begin{Phonetics}{识}{shi2}[][HSK 6]
    \definition{s.}{percepção; conhecimento}
    \definition{v.}{saber; reconhecer | saber; entender}
  \end{Phonetics}
  \begin{Phonetics}{识}{zhi4}
    \definition{s.}{marca; sinal; símbolo}
    \definition{v.}{lembrar; memorizar | anotar; registrar}
  \end{Phonetics}
\end{Entry}

\begin{Entry}{识字}{7,6}{⾔、⼦}
  \begin{Phonetics}{识字}{shi2 zi4}[][HSK 6]
    \definition{v.}{aprender a ler; tornar-se alfabetizado; reconhecer caracteres}
  \end{Phonetics}
\end{Entry}

\begin{Entry}{诊}{7}{⾔}
  \begin{Phonetics}{诊}{zhen3}
    \definition{v.}{examinar (um paciente)}
  \end{Phonetics}
\end{Entry}

\begin{Entry}{诊断}{7,11}{⾔、⽄}
  \begin{Phonetics}{诊断}{zhen3duan4}[][HSK 5]
    \definition{s.}{diagnóstico; diacrisis}
    \definition{v.}{diagnosticar; após examinar os sintomas do paciente, determinar a doença e seu desenvolvimento}
  \end{Phonetics}
\end{Entry}

\begin{Entry}{词}{7}{⾔}
  \begin{Phonetics}{词}{ci2}[][HSK 2]
    \definition[个,组,句,段,首]{s.}{palavra; termo; antigamente, referia-se a palavras vazias; atualmente, refere-se a palavras com forma fonética fixa e significado específico na língua; a menor unidade que pode ser usada de forma independente | discurso; declaração; linguagem; texto | ci (um tipo de poesia clássica chinesa, originária da dinastia Tang e plenamente desenvolvida na dinastia Song); gênero poético escrito de acordo com uma estrutura fixa, com versos de comprimentos variados | palavras; redação; refere-se genericamente ao teatro; a parte da letra cantada em harmonia com a melodia em canções e certas artes vocais}
  \end{Phonetics}
\end{Entry}

\begin{Entry}{词汇}{7,5}{⾔、⽔}
  \begin{Phonetics}{词汇}{ci2hui4}[][HSK 4]
    \definition[个,组,批,串,堆]{s.}{vocabulário; termo geral para palavras usadas em um idioma}
  \end{Phonetics}
\end{Entry}

\begin{Entry}{词典}{7,8}{⾔、⼋}
  \begin{Phonetics}{词典}{ci2 dian3}[][HSK 2]
    \definition[本,部]{s.}{dicionário, livro de referência que reúne palavras e explicações para consulta}
  \seealsoref{字典}{zi4 dian3}
  \end{Phonetics}
\end{Entry}

\begin{Entry}{词语}{7,9}{⾔、⾔}
  \begin{Phonetics}{词语}{ci2yu3}[][HSK 2]
    \definition[个,租]{s.}{termo; palavra; expressão; conjunto de palavras e frases}
  \end{Phonetics}
\end{Entry}

\begin{Entry}{试}{8}{⾔}
  \begin{Phonetics}{试}{shi4}[][HSK 1]
    \definition{s.}{teste; exame; avaliação de conhecimentos ou habilidades através de métodos específicos}
    \definition{v.}{tentar; investigar resultados ou verificar a natureza, não se envolver formalmente (em determinada atividade)}
  \end{Phonetics}
\end{Entry}

\begin{Entry}{试卷}{8,8}{⾔、⼙}
  \begin{Phonetics}{试卷}{shi4juan4}[][HSK 4]
    \definition[分,张]{s.}{folha de teste; folha de exame; papel usado para escrever as respostas nos exames}
  \end{Phonetics}
\end{Entry}

\begin{Entry}{试图}{8,8}{⾔、⼞}
  \begin{Phonetics}{试图}{shi4tu2}[][HSK 5]
    \definition{v.}{tentar; pretender, fazer o possível para realizar algo}
  \end{Phonetics}
\end{Entry}

\begin{Entry}{试点}{8,9}{⾔、⽕}
  \begin{Phonetics}{试点}{shi4 dian3}[][HSK 6]
    \definition[个]{s.}{local onde um experimento é conduzido; unidade experimental; local de teste; um lugar para pequenos experimentos}
    \definition{v.}{experimentar; fazer experimentos; realizar testes em pontos selecionados; lançar um projeto piloto}
  \end{Phonetics}
\end{Entry}

\begin{Entry}{试验}{8,10}{⾔、⾺}
  \begin{Phonetics}{试验}{shi4yan4}[][HSK 3]
    \definition{v.}{testar; fazer um teste; fazer um experimento; para examinar o efeito ou desempenho de algo, primeiro experimente em um laboratório ou em uma escala menor}
  \end{Phonetics}
\end{Entry}

\begin{Entry}{试题}{8,15}{⾔、⾴}
  \begin{Phonetics}{试题}{shi4 ti2}[][HSK 3]
    \definition[道]{s.}{questões de um exame}
  \end{Phonetics}
\end{Entry}

\begin{Entry}{诗}{8}{⾔}
  \begin{Phonetics}{诗}{shi1}[][HSK 4]
    \definition[首,句,行]{s.}{poesia; verso; poema; um gênero literário que reflete a vida e expressa emoções por meio de uma linguagem rítmica e rimada}
  \seealsoref{诗经}{shi1jing1}
  \end{Phonetics}
\end{Entry}

\begin{Entry}{诗人}{8,2}{⾔、⼈}
  \begin{Phonetics}{诗人}{shi1 ren2}[][HSK 4]
    \definition[个,位,名,些]{s.}{poeta; escritor de poesia}
  \end{Phonetics}
\end{Entry}

\begin{Entry}{诗句}{8,5}{⾔、⼝}
  \begin{Phonetics}{诗句}{shi1ju4}
    \definition[行]{s.}{verso | versículo}
  \end{Phonetics}
\end{Entry}

\begin{Entry}{诗词}{8,7}{⾔、⾔}
  \begin{Phonetics}{诗词}{shi1ci2}
    \definition{s.}{verso}
  \end{Phonetics}
\end{Entry}

\begin{Entry}{诗经}{8,8}{⾔、⽷}
  \begin{Phonetics}{诗经}{shi1jing1}
    \definition*{s.}{Shijing, o Livro das Canções, antiga coleção de poemas chineses e um dos Cinco Clássicos do Confucionismo}
  \end{Phonetics}
\end{Entry}

\begin{Entry}{诗意}{8,13}{⾔、⼼}
  \begin{Phonetics}{诗意}{shi1yi4}
    \definition{adj.}{poético}
    \definition{s.}{poesia}
  \end{Phonetics}
\end{Entry}

\begin{Entry}{诗歌}{8,14}{⾔、⽋}
  \begin{Phonetics}{诗歌}{shi1 ge1}[][HSK 5]
    \definition[本,首,段]{s.}{poesia; poemas e canções; refere-se a todos os tipos de poesia}
  \end{Phonetics}
\end{Entry}

\begin{Entry}{诚}{8}{⾔}
  \begin{Phonetics}{诚}{cheng2}
    \definition{adj.}{sincero; honesto; verdadeiro}
    \definition{adv.}{na verdade; realmente; de fato}
    \definition{s.}{sinceridade; genuinidade; seriedade}
  \end{Phonetics}
\end{Entry}

\begin{Entry}{诚心诚意}{8,4,8,13}{⾔、⼼、⾔、⼼}
  \begin{Phonetics}{诚心诚意}{cheng2xin1-cheng2yi4}[][HSK 7-9]
    \definition{expr.}{sincero e sério; com toda a sinceridade; é uma expressão idiomática chinesa que vem da Biografia de Ma Yuan no Livro da Dinastia Han Posterior | genuíno; sincero}
  \end{Phonetics}
\end{Entry}

\begin{Entry}{诚实}{8,8}{⾔、⼧}
  \begin{Phonetics}{诚实}{cheng2shi2}[][HSK 4]
    \definition{adj.}{honesto; sincero e honesto, não hipócrita}
  \end{Phonetics}
\end{Entry}

\begin{Entry}{诚实地}{8,8,6}{⾔、⼧、⼟}
  \begin{Phonetics}{诚实地}{cheng2shi2 di4}
    \definition{adv.}{honestamente}
  \end{Phonetics}
\end{Entry}

\begin{Entry}{诚信}{8,9}{⾔、⼈}
  \begin{Phonetics}{诚信}{cheng2 xin4}[][HSK 4]
    \definition{adj.}{honesto e confiável}
    \definition[种]{s.}{fé; honestidade; padrão e princípio de comportamento: não contar mentiras, prometer aos outros o que eles podem fazer e ter a confiança dos outros}
  \end{Phonetics}
\end{Entry}

\begin{Entry}{诚恳}{8,10}{⾔、⼼}
  \begin{Phonetics}{诚恳}{cheng2ken3}[][HSK 7-9]
    \definition{adj.}{sincero; sério; a atitude é muito real e pé no chão}
  \end{Phonetics}
\end{Entry}

\begin{Entry}{诚挚}{8,10}{⾔、⼿}
  \begin{Phonetics}{诚挚}{cheng2zhi4}[][HSK 7-9]
    \definition{adj.}{sincero; cordial; honesto}
  \end{Phonetics}
\end{Entry}

\begin{Entry}{诚意}{8,13}{⾔、⼼}
  \begin{Phonetics}{诚意}{cheng2yi4}[][HSK 7-9]
    \definition{s.}{boa fé; sinceridade; intenções sinceras}
  \end{Phonetics}
\end{Entry}

\begin{Entry}{话}{8}{⾔}
  \begin{Phonetics}{话}{hua4}[][HSK 1]
    \definition[句,段,番,种]{s.}{palavra; conversa; a voz que expressa os pensamentos quando falada, ou os caracteres que registram essa voz}
    \definition{v.}{falar sobre; falar a respeito}
  \end{Phonetics}
\end{Entry}

\begin{Entry}{话剧}{8,10}{⾔、⼑}
  \begin{Phonetics}{话剧}{hua4 ju4}[][HSK 3]
    \definition[场,幕,部,出,台]{s.}{drama moderno; peça de teatro; peça teatral representada através de diálogos e ações}
  \end{Phonetics}
\end{Entry}

\begin{Entry}{话题}{8,15}{⾔、⾴}
  \begin{Phonetics}{话题}{hua4ti2}[][HSK 3]
    \definition[个,种,项]{s.}{assunto de uma palestra; tópico de uma conversa; o foco da conversa}
  \end{Phonetics}
\end{Entry}

\begin{Entry}{诞}{8}{⾔}
  \begin{Phonetics}{诞}{dan4}
    \definition{adj.}{absurdo; fantástico; irreal; irracional}
    \definition{adv.}{absurdamente; fantasticamente}
    \definition{s.}{aniversário de nascimento | nascimento}
    \definition{v.}{nascer | dar à luz}
  \end{Phonetics}
\end{Entry}

\begin{Entry}{诞生}{8,5}{⾔、⽣}
  \begin{Phonetics}{诞生}{dan4sheng1}[][HSK 6]
    \definition{v.}{nascer; vir a existir; uma pessoa nasce; também significa que algo novo surgiu e tem um impacto positivo na sociedade}
  \end{Phonetics}
\end{Entry}

\begin{Entry}{诞辰}{8,7}{⾔、⾠}
  \begin{Phonetics}{诞辰}{dan4chen2}[][HSK 7-9]
    \definition[周年]{s.}{aniversário (usado principalmente para pessoas respeitadas)}[9月28日是孔子诞辰日。===28 de setembro é o aniversário de Confúcio.]
  \end{Phonetics}
\end{Entry}

\begin{Entry}{诟}{8}{⾔}
  \begin{Phonetics}{诟}{gou4}
    \definition*{s.}{Sobrenome Gou}
    \definition{s.}{vergonha; humilhação}
    \definition{v.}{insultar; xingar; falar de forma abusiva}
  \end{Phonetics}
\end{Entry}

\begin{Entry}{诟骂}{8,9}{⾔、⾺}
  \begin{Phonetics}{诟骂}{gou4ma4}
    \definition{v.}{abusar verbalmente | insultar | criticar}
  \end{Phonetics}
\end{Entry}

\begin{Entry}{询}{8}{⾔}
  \begin{Phonetics}{询}{xun2}
    \definition{v.}{perguntar; indagar; reunir informações | consultar; buscar conselho}
  \end{Phonetics}
\end{Entry}

\begin{Entry}{询问}{8,6}{⾔、⾨}
  \begin{Phonetics}{询问}{xun2wen4}[][HSK 5]
    \definition{v.}{indagar; perguntar sobre; pedir conselho}
  \end{Phonetics}
\end{Entry}

\begin{Entry}{该}{8}{⾔}
  \begin{Phonetics}{该}{gai1}[][HSK 2]
    \definition{adj.}{completo; integral; abrangente; inclusivo; o mesmo que 赅}
    \definition{pron.}{isto; aquilo; o referido; o acima mencionado; indica a pessoa ou coisa mencionada acima, equivalente a 此, 这个, etc.}
    \definition{v.}{deveria ser; deveria ser assim | caber a alguém; ser a vez (ou dever) de alguém fazer algo | merecer; servir a alguém de direito; indica que algo deve ser feito | dever | deve; provavelmente irá; muito provavelmente; pode ser razoavelmente ou naturalmente esperado que; expressa uma conclusão lógica ou provável com base na razão ou na experiência}
    \definition{v.aux.}{usado em frases exclamativas, tem a função de reforçar o tom}
  \seealsoref{此}{ci3}
  \seealsoref{赅}{gai1}
  \seealsoref{这个}{zhe4ge5}
  \end{Phonetics}
\end{Entry}

\begin{Entry}{详}{8}{⾔}
  \begin{Phonetics}{详}{xiang2}
    \definition{adj.}{conhecido; reconhecido; saber claramente | detalhado; minucioso; pormenorizado (oposto a 略)}
    \definition{s.}{detalhes; particularidades}
    \definition{v.}{contar; explicar; elaborar | saber claramente}
  \seealsoref{略}{lve4}
  \end{Phonetics}
\end{Entry}

\begin{Entry}{详细}{8,8}{⾔、⽷}
  \begin{Phonetics}{详细}{xiang2xi4}[][HSK 5]
    \definition{adj.}{explícito; detalhado; minucioso; circunstancial; meticuloso}
  \end{Phonetics}
\end{Entry}

\begin{Entry}{诧}{8}{⾔}
  \begin{Phonetics}{诧}{cha4}
    \definition{v.}{ficar surpreso}
  \end{Phonetics}
\end{Entry}

\begin{Entry}{诧异}{8,6}{⾔、⼶}
  \begin{Phonetics}{诧异}{cha4yi4}[][HSK 7-9]
    \definition{v.}{ficar surpreso; ficar espantado}
  \end{Phonetics}
\end{Entry}

\begin{Entry}{语}{9}{⾔}
  \begin{Phonetics}{语}{yu3}
    \definition{s.}{língua; linguagem | dito; provérbio; refere-se especialmente a coloquialismos, provérbios, expressões idiomáticas ou palavras de livros antigos | sinal; meio não linguístico de comunicar ideias ; ações ou sinais que substituem palavras para expressar significado | palavras; expressão; refere-se a uma palavra, frase ou sentença}
    \definition{v.}{dizer; falar | (pássaros, insetos, etc.) gorjear; pipilar}
  \end{Phonetics}
  \begin{Phonetics}{语}{yu4}
    \definition{v.}{contar; informar}
  \end{Phonetics}
\end{Entry}

\begin{Entry}{语气}{9,4}{⾔、⽓}
  \begin{Phonetics}{语气}{yu3qi4}
    \definition[个]{s.}{maneira de falar | tom}
  \end{Phonetics}
\end{Entry}

\begin{Entry}{语言}{9,7}{⾔、⾔}
  \begin{Phonetics}{语言}{yu3yan2}[][HSK 2]
    \definition[种,门]{s.}{linguagem; é uma ferramenta exclusiva dos humanos para expressar ideias e comunicar pensamentos; é um fenômeno social especial e consiste em um sistema específico de pronúncia, vocabulário e gramática | linguagem falada}
  \end{Phonetics}
\end{Entry}

\begin{Entry}{语言实验室}{9,7,8,10,9}{⾔、⾔、⼧、⾺、⼧}
  \begin{Phonetics}{语言实验室}{yu3yan2shi2yan4shi4}
    \definition{s.}{laboratório de línguas}
  \end{Phonetics}
\end{Entry}

\begin{Entry}{语法}{9,8}{⾔、⽔}
  \begin{Phonetics}{语法}{yu3fa3}[][HSK 4]
    \definition[个]{s.}{gramática; maneira como o idioma é estruturado, incluindo a formação e as variações de palavras, a organização de frases e sentenças | estudo da gramática; estudo das regras de estrutura linguística}
  \end{Phonetics}
\end{Entry}

\begin{Entry}{语法术语}{9,8,5,9}{⾔、⽔、⽊、⾔}
  \begin{Phonetics}{语法术语}{yu3fa3 shu4yu3}
    \definition{s.}{termo gramatical}
  \end{Phonetics}
\end{Entry}

\begin{Entry}{语音}{9,9}{⾔、⾳}
  \begin{Phonetics}{语音}{yu3 yin1}[][HSK 4]
    \definition{s.}{voz; pronúncia; sons da fala; som de alguém falando | pronúncia; som do idioma}
  \end{Phonetics}
\end{Entry}

\begin{Entry}{语调}{9,10}{⾔、⾔}
  \begin{Phonetics}{语调}{yu3diao4}
    \definition[个]{s.}{entonação}
  \end{Phonetics}
\end{Entry}

\begin{Entry}{误}{9}{⾔}
  \begin{Phonetics}{误}{wu4}[][HSK 6]
    \definition{adj.}{errado; falso; impreciso | acidental}
    \definition{adv.}{por engano; por acidente; não intencional}
    \definition{s.}{engano; erro}
    \definition{v.}{perder | dificultar; impedir; prejudicar | confundir; entender mal; cometer um erro | causar desvantagem a. causar dano}
  \end{Phonetics}
\end{Entry}

\begin{Entry}{误会}{9,6}{⾔、⼈}
  \begin{Phonetics}{误会}{wu4hui4}
    \definition[场]{s.}{mal-entendido; desentendimentos ou conflitos decorrentes de mal-entendidos}
    \definition{v.}{entender mal; entender errado; interpretar mal; não entender; não entender corretamente o significado}
  \end{Phonetics}
\end{Entry}

\begin{Entry}{误点}{9,9}{⾔、⽕}
  \begin{Phonetics}{误点}{wu4/dian3}
    \definition{v.+compl.}{atrasar | chegar tarde}
  \end{Phonetics}
\end{Entry}

\begin{Entry}{误解}{9,13}{⾔、⾓}
  \begin{Phonetics}{误解}{wu4jie3}[][HSK 5]
    \definition[个,种]{s.}{equívoco; mal-entendido; desentendimento}
    \definition{v.}{interpretar mal; interpretar erroneamente; não compreender corretamente}
  \end{Phonetics}
\end{Entry}

\begin{Entry}{诱}{9}{⾔}
  \begin{Phonetics}{诱}{you4}
    \definition{v.}{guiar; liderar; dirigir | atrair; seduzir; aliciar | induzir; causar; resultar de; levar a}
  \end{Phonetics}
\end{Entry}

\begin{Entry}{诱人}{9,2}{⾔、⼈}
  \begin{Phonetics}{诱人}{you4ren2}
    \definition{adj.}{atraente | cativante}
  \end{Phonetics}
\end{Entry}

\begin{Entry}{说}{9}{⾔}
  \begin{Phonetics}{说}{shui4}
    \definition{v.}{persuadir}
  \end{Phonetics}
  \begin{Phonetics}{说}{shuo1}[][HSK 1]
    \definition{s.}{uma teoria (normalmente o último caractere, como em 日心说, teoria heliocêntrica); ensinamentos; doutrina}
    \definition{v.}{falar; conversar; dizer | explicar | repreender | atuar como casamenteiro | referir-se a; indicar | criticar; aconselhar | fazer uma combinação; conciliar; mediar | discutir; falar sobre; conversar sobre | uma forma de expressão linguística da arte cênica}
  \seealsoref{日心说}{ri4 xin1 shuo1}
  \end{Phonetics}
\end{Entry}

\begin{Entry}{说不定}{9,4,8}{⾔、⼀、⼧}
  \begin{Phonetics}{说不定}{shuo1bu5ding4}[][HSK 4]
    \definition{adv.}{talvez; indica uma estimativa, possivelmente, provavelmente}
    \definition{v.}{não ter certeza; não estar certo; ser impreciso}
  \end{Phonetics}
\end{Entry}

\begin{Entry}{说好}{9,6}{⾔、⼥}
  \begin{Phonetics}{说好}{shuo1hao3}
    \definition{v.}{chegar a um acordo | concluir negociações}
  \end{Phonetics}
\end{Entry}

\begin{Entry}{说完}{9,7}{⾔、⼧}
  \begin{Phonetics}{说完}{shuo1-wan2}
    \definition{expr.}{acabar/terminar palavras}
  \end{Phonetics}
\end{Entry}

\begin{Entry}{说实话}{9,8,8}{⾔、⼧、⾔}
  \begin{Phonetics}{说实话}{shuo1 shi2 hua4}[][HSK 6]
    \definition{v.}{falar a verdade; dizer a verdade sobre (os próprios erros ou crimes)}
  \end{Phonetics}
\end{Entry}

\begin{Entry}{说明}{9,8}{⾔、⽇}
  \begin{Phonetics}{说明}{shuo1ming2}[][HSK 2]
    \definition[本,个]{s.}{legenda; instrução; explicação}
    \definition{v.}{mostrar; explicar; ilustrar | indicar; mostrar; provar; demonstrar; usar materiais confiáveis para demonstrar ou determinar a autenticidade de pessoas ou coisas}
  \end{Phonetics}
\end{Entry}

\begin{Entry}{说明书}{9,8,4}{⾔、⽇、⼄}
  \begin{Phonetics}{说明书}{shuo1 ming2 shu1}[][HSK 6]
    \definition[本]{s.}{manual; livro de instruções; descrições textuais da finalidade, especificações, desempenho e uso de itens, bem como enredos de peças e filmes, etc.}
  \end{Phonetics}
\end{Entry}

\begin{Entry}{说服}{9,8}{⾔、⽉}
  \begin{Phonetics}{说服}{shuo1fu2}[][HSK 4]
    \definition{v.}{persuadir; convencer; convencer a outra parte com palavras bem fundamentadas}
  \end{Phonetics}
\end{Entry}

\begin{Entry}{说法}{9,8}{⾔、⽔}
  \begin{Phonetics}{说法}{shuo1 fa3}[][HSK 5]
    \definition[种,个]{s.}{formulação; maneira de dizer uma coisa; formas de expressar opiniões | versão; argumento; declaração; opinião | explicação; acordo; palavras justas; razões ou fundamentos legítimos}
  \end{Phonetics}
\end{Entry}

\begin{Entry}{说话}{9,8}{⾔、⾔}
  \begin{Phonetics}{说话}{shuo1hua4}[][HSK 1]
    \definition{adv.}{imediatamente; em um minuto; refere-se ao tempo que leva para falar, indicando um período muito curto}
    \definition{v.}{falar; conversar; dizer; expressar o significado através da linguagem | conversar (conversa fiada); bater papo | fofocar; conversar; criticar; censurar}
  \end{Phonetics}
\end{Entry}

\begin{Entry}{说理}{9,11}{⾔、⽟}
  \begin{Phonetics}{说理}{shuo1li3}
    \definition{v.}{racionalizar | discutir logicamente}
  \end{Phonetics}
\end{Entry}

\begin{Entry}{说谎}{9,11}{⾔、⾔}
  \begin{Phonetics}{说谎}{shuo1/huang3}
    \definition{v.+compl.}{mentir | contar uma mentira}
  \end{Phonetics}
\end{Entry}

\begin{Entry}{请}{10}{⾔}
  \begin{Phonetics}{请}{qing3}[][HSK 1]
    \definition*{s.}{Sobrenome Qing}
    \definition{v.}{solicitar; perguntar | convidar; envolver | por favor; uma expressão educada usada quando você quer que alguém faça algo | comprar coisas sagradas para sacrifício, como incenso, velas, cavalos de papel e santuários de Buda; superstição se refere à compra de estátuas de Buda, santuários, etc. | entreter}
  \end{Phonetics}
\end{Entry}

\begin{Entry}{请问}{10,6}{⾔、⾨}
  \begin{Phonetics}{请问}{qing3 wen4}[][HSK 1]
    \definition{expr.}{Com licença, posso perguntar\dots? (para perguntar por qualquer coisa); uma maneira educada de pedir para alguém responder a uma pergunta}
  \end{Phonetics}
\end{Entry}

\begin{Entry}{请坐}{10,7}{⾔、⼟}
  \begin{Phonetics}{请坐}{qing3 zuo4}[][HSK 1]
    \definition{v.}{por favor, sente-se; convidar outras pessoas para sentar ou descansar}
  \end{Phonetics}
\end{Entry}

\begin{Entry}{请求}{10,7}{⾔、⽔}
  \begin{Phonetics}{请求}{qing3qiu2}[][HSK 2]
    \definition[个,次]{s.}{pedido; petição; solicitação; refere-se à exigência apresentada}
    \definition{v.}{pedir; solicitar; requerer; peticionar; fazer uma solicitação e pedir que a outra parte concorde com ela}
  \end{Phonetics}
\end{Entry}

\begin{Entry}{请进}{10,7}{⾔、⾡}
  \begin{Phonetics}{请进}{qing3 jin4}[][HSK 1]
    \definition{v.}{por favor entre; convidar alguém para um espaço ou lugar}
  \end{Phonetics}
\end{Entry}

\begin{Entry}{请客}{10,9}{⾔、⼧}
  \begin{Phonetics}{请客}{qing3/ke4}[][HSK 2]
    \definition{v.+compl.}{receber convidados; hospedar convidados | oferecer; convidar; pagar a conta; arcar com os custos; convidar alguém para comer, tomar chá, etc.}
  \end{Phonetics}
\end{Entry}

\begin{Entry}{请假}{10,11}{⾔、⼈}
  \begin{Phonetics}{请假}{qing3/jia4}[][HSK 1]
    \definition{v.+compl.}{pedir licença para sair; solicitar permissão para não trabalhar ou estudar por um determinado período de tempo devido a doença ou outros motivos}
  \end{Phonetics}
\end{Entry}

\begin{Entry}{请假条}{10,11,7}{⾔、⼈、⽊}
  \begin{Phonetics}{请假条}{qing3jia4tiao2}
    \definition{s.}{pedido de licença de ausência (do trabalho ou da escola)}
  \end{Phonetics}
\end{Entry}

\begin{Entry}{请教}{10,11}{⾔、⽁}
  \begin{Phonetics}{请教}{qing3jiao4}[][HSK 3]
    \definition{v.}{consultar; pedir conselho}
  \end{Phonetics}
\end{Entry}

\begin{Entry}{诸}{10}{⾔}
  \begin{Phonetics}{诸}{zhu1}
    \definition*{s.}{Sobrenome Zhu}
    \definition{adj.}{todos; cada; vários}
    \definition{prep.}{em; para; de}
  \end{Phonetics}
\end{Entry}

\begin{Entry}{诸位}{10,7}{⾔、⼈}
  \begin{Phonetics}{诸位}{zhu1wei4}[][HSK 6]
    \definition{pron.}{senhores; todos; todos vocês; senhoras e senhores; um termo educado que se refere a várias pessoas}
  \end{Phonetics}
\end{Entry}

\begin{Entry}{诺}{10}{⾔}
  \begin{Phonetics}{诺}{nuo4}
    \definition*{s.}{Sobrenome Nuo}
    \definition{interj.}{Sim!}
    \definition{v.}{prometer}
  \end{Phonetics}
\end{Entry}

\begin{Entry}{诺贝尔奖}{10,4,5,9}{⾔、⾙、⼩、⼤}
  \begin{Phonetics}{诺贝尔奖}{nuo4bei4'er3 jiang3}
    \definition*{s.}{Prêmio Nobel}
  \end{Phonetics}
\end{Entry}

\begin{Entry}{诺奖}{10,9}{⾔、⼤}
  \begin{Phonetics}{诺奖}{nuo4jiang3}
    \definition*{s.}{Prêmio Nobel, abreviação de 诺贝尔奖}
  \seealsoref{诺贝尔奖}{nuo4bei4'er3 jiang3}
  \end{Phonetics}
\end{Entry}

\begin{Entry}{读}{10}{⾔}
  \begin{Phonetics}{读}{dou4}
    \definition{s.}{vírgula; uma breve pausa na leitura}
  \end{Phonetics}
  \begin{Phonetics}{读}{du2}[][HSK 1]
    \definition*{s.}{Sobrenome Du}
    \definition{v.}{ler em voz alta | ler; ler o texto e compreendera seu significado | frequentar a escola; refere-se a ir à escola ou estudar | (computação) ler dados}
  \end{Phonetics}
\end{Entry}

\begin{Entry}{读书}{10,4}{⾔、⼄}
  \begin{Phonetics}{读书}{du2/shu1}[][HSK 1]
    \definition{v.+compl.}{ler; estudar | frequentar a escola}
  \end{Phonetics}
\end{Entry}

\begin{Entry}{读者}{10,8}{⾔、⽼}
  \begin{Phonetics}{读者}{du2 zhe3}[][HSK 3]
    \definition[个,位,名,些,群]{s.}{leitor; (para obras, autores, revistas, etc.) Pessoas que compram ou leem livros, revistas, artigos, jornais, etc.}
  \end{Phonetics}
\end{Entry}

\begin{Entry}{读音}{10,9}{⾔、⾳}
  \begin{Phonetics}{读音}{du2 yin1}[][HSK 2]
    \definition[种]{s.}{pronúncia}
  \end{Phonetics}
\end{Entry}

\begin{Entry}{课}{10}{⾔}
  \begin{Phonetics}{课}{ke4}[][HSK 1]
    \definition{clas.}{aula; lição; unidade de tempo de ensino; parágrafo do material didático}
    \definition[门,节]{s.}{classe; aula; ensino por etapas planejado | disciplina; curso | imposto; antiga referência a impostos | seção; departamentos de escritório criados no antigo governo}
    \definition{v.}{cobrar; impor; taxar}
  \end{Phonetics}
\end{Entry}

\begin{Entry}{课文}{10,4}{⾔、⽂}
  \begin{Phonetics}{课文}{ke4 wen2}[][HSK 1]
    \definition[篇,段]{s.}{texto (de uma lição); texto principal do livro didático (diferente das notas de rodapé, exercícios, etc.)}
  \end{Phonetics}
\end{Entry}

\begin{Entry}{课本}{10,5}{⾔、⽊}
  \begin{Phonetics}{课本}{ke4 ben3}[][HSK 1]
    \definition[本]{s.}{livro didático; livro-texto}
  \end{Phonetics}
\end{Entry}

\begin{Entry}{课堂}{10,11}{⾔、⼟}
  \begin{Phonetics}{课堂}{ke4 tang2}[][HSK 2]
    \definition[间,节,个]{s.}{sala de aula; local onde se realizam as aulas; local onde se realizam as atividades de ensino}
  \end{Phonetics}
\end{Entry}

\begin{Entry}{课程}{10,12}{⾔、⽲}
  \begin{Phonetics}{课程}{ke4cheng2}[][HSK 3]
    \definition[个,堂,节,门]{s.}{curso; currículo; as disciplinas e o programa letivo da escola}
  \end{Phonetics}
\end{Entry}

\begin{Entry}{课题}{10,15}{⾔、⾴}
  \begin{Phonetics}{课题}{ke4ti2}[][HSK 5]
    \definition[组]{s.}{uma questão para estudo ou discussão; principais questões a serem pesquisadas ou discutidas, ou assuntos importantes que precisam ser resolvidos com urgência | tarefa; problema; questões a serem resolvidas}
  \end{Phonetics}
\end{Entry}

\begin{Entry}{谁}{10}{⾔}
  \begin{Phonetics}{谁}{shei2}[][HSK 1]
    \definition{pron.}{quem? | (em pergunta retórica) quem?; usado em perguntas retóricas, para indicar que não há ninguém | refere-se a pessoas que não têm certeza, incluindo aquelas que não sabem | alguém; qualquer pessoa; indica qualquer pessoa ou qualquer um | repetido em uma frase para se referir a uma pessoa | (repetido em duas frases) quem quer que seja; fazer com que o sujeito e o objeto se refiram a duas pessoas diferentes}
  \end{Phonetics}
  \begin{Phonetics}{谁}{shui2}[][HSK 1]
  \end{Phonetics}
\end{Entry}

\begin{Entry}{调}{10}{⾔}
  \begin{Phonetics}{调}{diao4}[][HSK 3]
    \definition{s.}{sotaque; pronúncia | nota (musical) | melodia; música | tom; refere-se ao tom da fala, ou seja, a elevação e descida do tom das palavras | estilo; ambiente; estilo metafórico, talento, etc. | argumento; discurso}
    \definition{v.}{deslocar; mover; transferir; mover (pessoas, objetos, etc.) de um lugar para outro | examinar; investigar}
  \end{Phonetics}
  \begin{Phonetics}{调}{tiao2}[][HSK 3]
    \definition{adj.}{harmonioso; boa coordenação}
    \definition{v.}{misturar; ajustar; fazer o ajuste uniforme e apropriado | provocar; importunar; fazer pouco de | incitar; instigar; provocar; semear discórdia | mediar; trazer harmonia}
  \end{Phonetics}
\end{Entry}

\begin{Entry}{调皮}{10,5}{⾔、⽪}
  \begin{Phonetics}{调皮}{tiao2pi2}[][HSK 4]
    \definition{adj.}{travesso; malicioso; malandro | indisciplinado; desordeiro; indomável; astuto | inteligente e desonesto}
  \end{Phonetics}
\end{Entry}

\begin{Entry}{调节}{10,5}{⾔、⾋}
  \begin{Phonetics}{调节}{tiao2jie2}[][HSK 5]
    \definition{v.}{regular; ajustar; ajustar e controlar de várias maneiras para atender aos requisitos}
  \end{Phonetics}
\end{Entry}

\begin{Entry}{调动}{10,6}{⾔、⼒}
  \begin{Phonetics}{调动}{diao4dong4}[][HSK 5]
    \definition{v.}{mudar; transferir; pessoal, trabalho | mobilizar; despertar; pôr em jogo; melhorar (motivação, entusiasmo, etc.) por meio de alguns meios | reunir; manobrar; mover (tropas); mobilizar forças militares}
  \end{Phonetics}
\end{Entry}

\begin{Entry}{调律}{10,9}{⾔、⼻}
  \begin{Phonetics}{调律}{tiao2lv4}
    \definition{v.}{afinar (por exemplo, um piano)}
  \end{Phonetics}
\end{Entry}

\begin{Entry}{调查}{10,9}{⾔、⽊}
  \begin{Phonetics}{调查}{diao4cha2}[][HSK 3]
    \definition[项,个,份]{s.}{pesquisa; investigação; informações obtidas após perguntar a outras pessoas ou investigar}
    \definition{v.}{investigar; indagar; inquerir; examinar; realizar uma investigação (geralmente no local) para entender a situação}
  \end{Phonetics}
\end{Entry}

\begin{Entry}{调研}{10,9}{⾔、⽯}
  \begin{Phonetics}{调研}{diao4 yan2}[][HSK 6]
    \definition{v.}{pesquisar e estudar; investigar e pesquisar; pesquisar}
  \end{Phonetics}
\end{Entry}

\begin{Entry}{调解}{10,13}{⾔、⾓}
  \begin{Phonetics}{调解}{tiao2jie3}[][HSK 5]
    \definition{v.}{mediar; fazer as pazes; resolver conflitos através da persuasão}
  \end{Phonetics}
\end{Entry}

\begin{Entry}{调整}{10,16}{⾔、⽁}
  \begin{Phonetics}{调整}{tiao2zheng3}[][HSK 3]
    \definition{v.}{ajustar; revisar; regularizar; fazer as alterações apropriadas no estado original para se adaptar à nova situação}
  \end{Phonetics}
\end{Entry}

\begin{Entry}{谈}{10}{⾔}
  \begin{Phonetics}{谈}{tan2}[][HSK 3]
    \definition*{s.}{Sobrenome Tan}
    \definition{s.}{o que é dito ou falado; discurso}
    \definition{v.}{falar; bater papo; discutir}
  \end{Phonetics}
\end{Entry}

\begin{Entry}{谈判}{10,7}{⾔、⼑}
  \begin{Phonetics}{谈判}{tan2pan4}[][HSK 3]
    \definition{v.}{negociar; manter conversações; para resolver um grande problema, as partes relevantes trocaram opiniões entre si, na esperança de encontrar uma solução com a qual todos pudessem concordar}
  \end{Phonetics}
\end{Entry}

\begin{Entry}{谈话}{10,8}{⾔、⾔}
  \begin{Phonetics}{谈话}{tan2/hua4}[][HSK 3]
    \definition[次]{s.}{declaração; opiniões (principalmente políticas) expressas na forma de conversas}
    \definition{v.+compl.}{conversar; discutir | falar; refere-se especificamente ao uso da conversa para entender a situação, fazer trabalho ideológico, etc. (usado principalmente por superiores para subordinados)}
  \end{Phonetics}
\end{Entry}

\begin{Entry}{谈恋爱}{10,10,10}{⾔、⼼、⽖}
  \begin{Phonetics}{谈恋爱}{tan2lian4'ai4}
    \definition{v.}{namorar | apaixonar-se}
  \end{Phonetics}
\end{Entry}

\begin{Entry}{谎}{11}{⾔}
  \begin{Phonetics}{谎}{huang3}
    \definition[句]{s.}{mentira; falsidade}
    \definition{v.}{contar uma mentira; mentir}
  \end{Phonetics}
\end{Entry}

\begin{Entry}{谎话}{11,8}{⾔、⾔}
  \begin{Phonetics}{谎话}{huang3hua4}
    \definition{s.}{mentira}
  \end{Phonetics}
\end{Entry}

\begin{Entry}{谐}{11}{⾔}
  \begin{Phonetics}{谐}{xie2}
    \definition{adj.}{harmonioso | humorístico}
  \end{Phonetics}
\end{Entry}

\begin{Entry}{粤}{12}{⾔}
  \begin{Phonetics}{粤}{yue4}
    \definition*{s.}{Outro nome para a Província de Guangdong, 广东}
  \seealsoref{广东}{guang3dong1}
  \end{Phonetics}
\end{Entry}

\begin{Entry}{粤语}{12,9}{⾔、⾔}
  \begin{Phonetics}{粤语}{yue4yu3}
    \definition{s.}{cantonês | língua cantonesa}
  \end{Phonetics}
\end{Entry}

\begin{Entry}{詈}{12}{⾔}
  \begin{Phonetics}{詈}{li4}
    \definition{v.}{xingar; usar linguagem severa}
  \end{Phonetics}
\end{Entry}

\begin{Entry}{詈骂}{12,9}{⾔、⾺}
  \begin{Phonetics}{詈骂}{li4ma4}
    \definition{v.}{xingar | abusar}
  \end{Phonetics}
\end{Entry}

\begin{Entry}{谢}{12}{⾔}
  \begin{Phonetics}{谢}{xie4}
    \definition*{s.}{Sobrenome Xie}
    \definition{v.}{agradecer | desculpar-se; pedir desculpas; admitir a própria culpa | recusar; declinar; renunciar | murchar; perder de flores ou folhas}
  \end{Phonetics}
\end{Entry}

\begin{Entry}{谢天谢地}{12,4,12,6}{⾔、⼤、⾔、⼟}
  \begin{Phonetics}{谢天谢地}{xie4tian1xie4di4}
    \definition{expr.}{agradecer a Deus | agradecer aos céus}
  \end{Phonetics}
\end{Entry}

\begin{Entry}{谢世}{12,5}{⾔、⼀}
  \begin{Phonetics}{谢世}{xie4shi4}
    \definition{v.}{morrer | falecer}
  \end{Phonetics}
\end{Entry}

\begin{Entry}{谢恩}{12,10}{⾔、⼼}
  \begin{Phonetics}{谢恩}{xie4'en1}
    \definition{v.}{agradecer a alguém pelo favor (especialmente imperador ou oficial superior)}
  \end{Phonetics}
\end{Entry}

\begin{Entry}{谢病}{12,10}{⾔、⽧}
  \begin{Phonetics}{谢病}{xie4bing4}
    \definition{v.}{desculpar-se por causa de doença}
  \end{Phonetics}
\end{Entry}

\begin{Entry}{谢媒}{12,12}{⾔、⼥}
  \begin{Phonetics}{谢媒}{xie4mei2}
    \definition{v.}{agradecer ao casamenteiro}
  \end{Phonetics}
\end{Entry}

\begin{Entry}{谢谢}{12,12}{⾔、⾔}
  \begin{Phonetics}{谢谢}{xie4xie5}[][HSK 1]
    \definition{interj.}{Obrigado!}
    \definition{v.}{agradecer; agradecer a gentileza dos outros}
  \end{Phonetics}
\end{Entry}

\begin{Entry}{谢意}{12,13}{⾔、⼼}
  \begin{Phonetics}{谢意}{xie4yi4}
    \definition{s.}{gratidão}
  \end{Phonetics}
\end{Entry}

\begin{Entry}{谦}{12}{⾔}
  \begin{Phonetics}{谦}{qian1}
    \definition*{s.}{Sobrenome Qian}
    \definition{adj.}{modesto}
    \definition{s.}{modéstia}
  \end{Phonetics}
\end{Entry}

\begin{Entry}{谦虚}{12,11}{⾔、⾌}
  \begin{Phonetics}{谦虚}{qian1xu1}[][HSK 6]
    \definition{adj.}{modesto; não se orgulhe de suas próprias conquistas e esteja disposto a aceitar críticas e opiniões de outras pessoas}
    \definition{v.}{falar modestamente; quando recebo elogios e cumprimentos de outras pessoas, sinto que não sou tão bom}
  \end{Phonetics}
\end{Entry}

\begin{Entry}{谩}{13}{⾔}
  \begin{Phonetics}{谩}{man2}
    \definition{v.}{enganar; ludibriar; iludir}
  \end{Phonetics}
  \begin{Phonetics}{谩}{man4}
    \definition{v.}{ser desrespeitoso | caluniar | desconsiderar}
  \end{Phonetics}
\end{Entry}

\begin{Entry}{谩骂}{13,9}{⾔、⾺}
  \begin{Phonetics}{谩骂}{man4ma4}
    \definition{v.}{ridicularizar | abusar}
  \end{Phonetics}
\end{Entry}

\begin{Entry}{警}{19}{⾔}
  \begin{Phonetics}{警}{jing3}
    \definition{s.}{policial}
    \definition{v.}{alertar | avisar}
  \end{Phonetics}
\end{Entry}

\begin{Entry}{警告}{19,7}{⾔、⼝}
  \begin{Phonetics}{警告}{jing3gao4}[][HSK 5]
    \definition[个]{s.}{advertência (como medida disciplinar); uma forma de punição}
    \definition{v.}{avisar; advertir; admoestar}
  \end{Phonetics}
\end{Entry}

\begin{Entry}{警官}{19,8}{⾔、⼧}
  \begin{Phonetics}{警官}{jing3guan1}
    \definition{s.}{polícia | policial}
  \end{Phonetics}
\end{Entry}

\begin{Entry}{警察}{19,14}{⾔、⼧}
  \begin{Phonetics}{警察}{jing3cha2}[][HSK 3]
    \definition[个,位,名,群,队]{s.}{polícia; policial; oficial de polícia; as forças armadas que mantêm a segurança social do país são uma parte importante do aparato estatal; também se refere aos membros dessas forças armadas}
  \end{Phonetics}
\end{Entry}

\begin{Entry}{譬}{20}{⾔}
  \begin{Phonetics}{譬}{pi4}
    \definition{s.}{exemplo; analogia; metáfora}
    \definition{v.}{dar um exemplo; fazer uma analogia}
  \end{Phonetics}
\end{Entry}

\begin{Entry}{譬如}{20,6}{⾔、⼥}
  \begin{Phonetics}{譬如}{pi4ru2}
    \definition{conj.}{por exemplo | como}
  \end{Phonetics}
\end{Entry}

%%%%% EOF %%%%%

