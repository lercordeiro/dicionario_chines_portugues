%%%
%%% Radical "⾔"
%%%

\section*{Radical 149: ``⾔'' (讠、訁)}\addcontentsline{toc}{section}{Radical 149: ⾔、讠、訁}

\begin{entry}{计划}{4,6}{⾔、⼑}
  \begin{phonetics}{计划}{ji4hua4}[][HSK 2]
    \definition[个,项]{s.}{plano | projeto | programa}
    \definition{v.}{planejar | mapear}
  \end{phonetics}
\end{entry}

\begin{entry}{计算}{4,14}{⾔、⽵}
  \begin{phonetics}{计算}{ji4suan4}[][HSK 3]
    \definition{v.}{contar; calcular; computar; enumerar | planejar; considerar | conspirar secretamente contra os outros}
  \end{phonetics}
\end{entry}

\begin{entry}{计算机}{4,14,6}{⾔、⽵、⽊}
  \begin{phonetics}{计算机}{ji4 suan4 ji1}[][HSK 2]
    \definition[部,台]{s.}{computador | calculadora}
  \end{phonetics}
\end{entry}

\begin{entry}{订}{4}{⾔}
  \begin{phonetics}{订}{ding4}[][HSK 3]
    \definition{v.}{concluir; elaborar; concordar com |assinar (um jornal, etc.); reservar (assentos, ingressos, etc.); encomendar (mercadorias, etc.) |fazer correções; revisar | grampear junto; unir}
  \end{phonetics}
\end{entry}

\begin{entry}{认}{4}{⾔}
  \begin{phonetics}{认}{ren4}[][HSK 5]
    \definition{v.}{reconhecer; saber; distinguir; identificar | estabelecer uma determinada relação com; adotar | admitir; reconhecer; assumir | comprometer-se a fazer algo | (frequentemente seguido por ``了'') aceitar como inevitável; resignar-se | (frequentemente seguido por ``了''] resignar-se; aceitar como inevitável}
  \seealsoref{了}{le5}
  \end{phonetics}
\end{entry}

\begin{entry}{认为}{4,4}{⾔、⼂}
  \begin{phonetics}{认为}{ren4wei2}[][HSK 2]
    \definition{v.}{pensar | considerar | segurar | julgar}
  \end{phonetics}
\end{entry}

\begin{entry}{认出}{4,5}{⾔、⼐}
  \begin{phonetics}{认出}{ren4 chu1}[][HSK 3]
    \definition{v.}{reconhecer; decifrar; identificar}
  \end{phonetics}
\end{entry}

\begin{entry}{认可}{4,5}{⾔、⼝}
  \begin{phonetics}{认可}{ren4ke3}[][HSK 3]
    \definition{v.}{aceitar; aprovar; confirmar; dar força legal a | permitir; concordar}
  \end{phonetics}
\end{entry}

\begin{entry}{认识}{4,7}{⾔、⾔}
  \begin{phonetics}{认识}{ren4shi5}[][HSK 1]
    \definition{s.}{conhecimento | saber | entendimento}
    \definition{v.}{estar familiarizado com | conhecer alguém | saber | reconhecer}
  \end{phonetics}
\end{entry}

\begin{entry}{认定}{4,8}{⾔、⼧}
  \begin{phonetics}{认定}{ren4ding4}[][HSK 5]
    \definition{v.}{afirmar; manter; acreditar firmemente; considerar com certeza | decidir-se por algo; confirmar; chegar a uma conclusão afirmativa}
  \end{phonetics}
\end{entry}

\begin{entry}{认真}{4,10}{⾔、⼗}
  \begin{phonetics}{认真}{ren4zhen1}[][HSK 1]
    \definition{adj.}{sério | consciencioso}
    \definition{adv.}{seriamente}
    \definition{v.}{levar a sério}
  \end{phonetics}
\end{entry}

\begin{entry}{认得}{4,11}{⾔、⼻}
  \begin{phonetics}{认得}{ren4 de5}[][HSK 3]
    \definition{v.}{saber; reconhecer}
  \end{phonetics}
\end{entry}

\begin{entry}{讨生活}{5,5,9}{⾔、⽣、⽔}
  \begin{phonetics}{讨生活}{tao3sheng1huo2}
    \definition{v.}{ganhar a vida}
  \end{phonetics}
\end{entry}

\begin{entry}{讨厌}{5,6}{⾔、⼚}
  \begin{phonetics}{讨厌}{tao3yan4}[][HSK 5]
    \definition{adj.}{desagradável; repugnante; repulsivo; irritante; incômodo |}
    \definition{v.}{odiar; não gostar; sentir repulsa por}
  \end{phonetics}
\end{entry}

\begin{entry}{讨论}{5,6}{⾔、⾔}
  \begin{phonetics}{讨论}{tao3lun4}[][HSK 2]
    \definition{v.}{discutir | falar sobre}
  \end{phonetics}
\end{entry}

\begin{entry}{让}{5}{⾔}
  \begin{phonetics}{让}{rang4}[][HSK 2]
    \definition{v.}{deixar alguém fazer alguma coisa |fazer alguém (sentir-se triste, etc.) | permitir | conceder}
  \end{phonetics}
\end{entry}

\begin{entry}{让步}{5,7}{⾔、⽌}
  \begin{phonetics}{让步}{rang4bu4}
    \definition{v.+compl.}{fazer uma concessão | entregar | desistir | comprometer}
  \end{phonetics}
\end{entry}

\begin{entry}{训练}{5,8}{⾔、⽷}
  \begin{phonetics}{训练}{xun4lian4}[][HSK 3]
    \definition{v.}{treinar; exercitar; adquirir certas especialidades ou habilidades de forma planejada e passo a passo}
  \end{phonetics}
\end{entry}

\begin{entry}{议论}{5,6}{⾔、⾔}
  \begin{phonetics}{议论}{yi4lun4}[][HSK 4]
    \definition{s.}{comentário; discussão; opiniões ou pontos de vista sobre o que é bom ou ruim, certo ou errado em relação a pessoas ou coisas}
    \definition{v.}{discutir; comentar; falar sobre; expressar opiniões e trocar pontos de vista sobre o bom, o ruim, o certo e o errado de pessoas ou coisas}
  \end{phonetics}
\end{entry}

\begin{entry}{记}{5}{⾔}
  \begin{phonetics}{记}{ji4}[][HSK 1]
    \definition{clas.}{para tapas, palmadas, bofetadas, etc.}
    \definition{s.}{nota | registro | marca | sinal |marca de nascença}
    \definition{v.}{lembrar | ter em mente | memorizar | escrever (anotar) | registrar}
  \end{phonetics}
\end{entry}

\begin{entry}{记忆}{5,4}{⾔、⼼}
  \begin{phonetics}{记忆}{ji4yi4}[][HSK 5]
    \definition[段]{s.}{memória; manter em sua mente uma imagem do passado}
    \definition{v.}{recordar; lembrar; lembrar-se ou recordar alguém ou algo do passado}
  \end{phonetics}
\end{entry}

\begin{entry}{记住}{5,7}{⾔、⼈}
  \begin{phonetics}{记住}{ji4 zhu5}[][HSK 1]
    \definition{v.}{decorar | memorizar | ter em mente}
  \end{phonetics}
\end{entry}

\begin{entry}{记录}{5,8}{⾔、⼹}
  \begin{phonetics}{记录}{ji4lu4}[][HSK 3]
    \definition[个,位]{s.}{notas; registro | anotador; registrador}
    \definition{v.}{tomar notas; registrar}
  \end{phonetics}
\end{entry}

\begin{entry}{记性}{5,8}{⾔、⼼}
  \begin{phonetics}{记性}{ji4xing5}
    \definition{s.}{memória (habilidade em reter informações)}
  \end{phonetics}
\end{entry}

\begin{entry}{记者}{5,8}{⾔、⽼}
  \begin{phonetics}{记者}{ji4zhe3}[][HSK 3]
    \definition[群,名,位]{s.}{repórter; correspondente; jornalista}
  \end{phonetics}
\end{entry}

\begin{entry}{记载}{5,10}{⾔、⾞}
  \begin{phonetics}{记载}{ji4zai3}[][HSK 4]
    \definition[段,份]{s.}{registro; conta; artigos e materiais que registram eventos}
    \definition{v.}{registrar; colocar por escrito}
  \end{phonetics}
\end{entry}

\begin{entry}{记得}{5,11}{⾔、⼻}
  \begin{phonetics}{记得}{ji4de5}[][HSK 1]
    \definition{v.}{lembrar | lembrar-se}
  \end{phonetics}
\end{entry}

\begin{entry}{讲}{6}{⾔}
  \begin{phonetics}{讲}{jiang3}[][HSK 2]
    \definition{v.}{falar (de) | falar (sobre) | relacionar | dizer | contar | explicar | explicitar | elaborar (sobre) | tornar claro |interpretar | discutir | consultar | negociar}
  \end{phonetics}
\end{entry}

\begin{entry}{讲究}{6,7}{⾔、⽳}
  \begin{phonetics}{讲究}{jiang3jiu5}[][HSK 4]
    \definition{adj.}{requintado; elegante; de bom gosto; exigente com a vida e com outros aspectos, buscando alto nível, qualidade e detalhes}
    \definition{s.}{estudo cuidadoso; algo que merece atenção; elementos e aspectos que merecem atenção especial}
    \definition{v.}{dar ênfase a; ser específico sobre; prestar atenção a}
  \end{phonetics}
\end{entry}

\begin{entry}{讲话}{6,8}{⾔、⾔}
  \begin{phonetics}{讲话}{jiang3 hua4}[][HSK 2]
    \definition{s.}{discurso | guia | introdução}
    \definition{v.+compl.}{falar | conversar | abordar}
  \end{phonetics}
\end{entry}

\begin{entry}{讲述}{6,8}{⾔、⾡}
  \begin{phonetics}{讲述}{jiang3shu4}
    \definition{v.}{falar sobre | narrar | descrever}
  \end{phonetics}
\end{entry}

\begin{entry}{讲座}{6,10}{⾔、⼴}
  \begin{phonetics}{讲座}{jiang3zuo4}[][HSK 4]
    \definition[个]{s.}{palestra; um curso de palestras; a forma de instrução usada para ensinar um determinado assunto ou tópico, geralmente por meio de palestras ao vivo, seriados de rádio ou televisão ou seriados de jornal.}
  \end{phonetics}
\end{entry}

\begin{entry}{许}{6}{⾔}
  \begin{phonetics}{许}{xu3}
    \definition*{s.}{sobrenome Xu}
    \definition{adv.}{um pouco | talvez}
    \definition{v.}{permitir | prometer | elogiar}
  \end{phonetics}
\end{entry}

\begin{entry}{许可}{6,5}{⾔、⼝}
  \begin{phonetics}{许可}{xu3ke3}[][HSK 5]
    \definition{v.}{permitir; autorizar}
  \end{phonetics}
\end{entry}

\begin{entry}{许多}{6,6}{⾔、⼣}
  \begin{phonetics}{许多}{xu3duo1}[][HSK 2]
    \definition{num.}{muitos | muito | numerosos | uma grande quantidade de}
  \end{phonetics}
\end{entry}

\begin{entry}{论文}{6,4}{⾔、⽂}
  \begin{phonetics}{论文}{lun4wen2}[][HSK 4]
    \definition[篇]{s.}{tese; redação; artigo; artigo que discute ou examina uma questão}
  \end{phonetics}
\end{entry}

\begin{entry}{设计}{6,4}{⾔、⾔}
  \begin{phonetics}{设计}{she4ji4}[][HSK 3]
    \definition[份]{s.}{plano; esquema}
    \definition{v.}{planejar; projetar | inventar}
  \end{phonetics}
\end{entry}

\begin{entry}{设立}{6,5}{⾔、⽴}
  \begin{phonetics}{设立}{she4li4}[][HSK 3]
    \definition{v.}{encontrar; estabelecer; configurar}
  \end{phonetics}
\end{entry}

\begin{entry}{设备}{6,8}{⾔、⼡}
  \begin{phonetics}{设备}{she4bei4}[][HSK 3]
    \definition[个]{s.}{facilidade; equipamento; instalação}
  \end{phonetics}
\end{entry}

\begin{entry}{设施}{6,9}{⾔、⽅}
  \begin{phonetics}{设施}{she4shi1}[][HSK 4]
    \definition{s.}{facilidade; instalação; instituições, sistemas, organizações, edifícios, etc., estabelecidos para realizar um trabalho ou atender a uma necessidade}
  \end{phonetics}
\end{entry}

\begin{entry}{设想}{6,13}{⾔、⼼}
  \begin{phonetics}{设想}{she4xiang3}[][HSK 5]
    \definition[个,种]{s.}{plano provisório (ou ideia); (item, tipo) refere-se a algo hipotético ou imaginário}
    \definition{v.}{imaginar; prever; conceber; supor | ter consideração por}
  \end{phonetics}
\end{entry}

\begin{entry}{设置}{6,13}{⾔、⽹}
  \begin{phonetics}{设置}{she4zhi4}[][HSK 4]
    \definition{v.}{estabelecer; colocar em prática; estabelecer ou criar instituições, empregos, profissões ou códigos, etc. | encaixar; ajustar; instalar; configurar; colocar}
  \end{phonetics}
\end{entry}

\begin{entry}{访问}{6,6}{⾔、⾨}
  \begin{phonetics}{访问}{fang3wen4}[][HSK 3]
    \definition{v.}{visitar; ligar; entrevistar | visitar um \emph{site}}
  \end{phonetics}
\end{entry}

\begin{entry}{言论}{7,6}{⾔、⾔}
  \begin{phonetics}{言论}{yan2lun4}
    \definition{s.}{expressão de opinião |  visualizações | comentários | argumentos}
  \end{phonetics}
\end{entry}

\begin{entry}{言语}{7,9}{⾔、⾔}
  \begin{phonetics}{言语}{yan2 yu3}[][HSK 5]
    \definition{s.}{verbal; fala; linguagem falada; conversa; palavras}
  \end{phonetics}
\end{entry}

\begin{entry}{证}{7}{⾔}
  \begin{phonetics}{证}{zheng4}[][HSK 3]
    \definition{s.}{evidência; prova; testemunho; testemunha | certificado; cartão | doença; enfermidade}
    \definition{v.}{provar; verificar; demonstrar}
  \end{phonetics}
\end{entry}

\begin{entry}{证书}{7,4}{⾔、⼄}
  \begin{phonetics}{证书}{zheng4shu1}[][HSK 5]
    \definition[张,份,些]{s.}{certificado; documentos emitidos por instituições, grupos, etc., que comprovem experiência, nível, honras, poderes, etc.}
  \end{phonetics}
\end{entry}

\begin{entry}{证件}{7,6}{⾔、⼈}
  \begin{phonetics}{证件}{zheng4jian4}[][HSK 3]
    \definition[个,本,张]{s.}{documentos; credenciais; certificado}
  \end{phonetics}
\end{entry}

\begin{entry}{证实}{7,8}{⾔、⼧}
  \begin{phonetics}{证实}{zheng4shi2}[][HSK 5]
    \definition{v.}{verificar; afirmar; confirmar; corroborar; demonstrar; autenticar; provar que é verdadeiro}
  \end{phonetics}
\end{entry}

\begin{entry}{证明}{7,8}{⾔、⽇}
  \begin{phonetics}{证明}{zheng4ming2}[][HSK 3]
    \definition[个,份]{s.}{certificado; testemunho; identificação; certificado ou carta de certificação; documentos que comprovem identidade, experiência, etc., como carteira de estudante, carteira de trabalho, certificado de graduação, etc.}
    \definition{v.}{provar; testemunhar; sustentar; usar materiais confiáveis ​​para mostrar ou determinar a autenticidade de uma pessoa ou coisa}
  \end{phonetics}
\end{entry}

\begin{entry}{证据}{7,11}{⾔、⼿}
  \begin{phonetics}{证据}{zheng4ju4}[][HSK 3]
    \definition{s.}{prova; evidência; testemunho; fatos ou materiais relevantes que podem provar a autenticidade de algo}
  \end{phonetics}
\end{entry}

\begin{entry}{评价}{7,6}{⾔、⼈}
  \begin{phonetics}{评价}{ping2jia4}[][HSK 3]
    \definition[个,项,条,份]{s.}{avaliação; apreciação}
    \definition{v.}{estimar; avaliar}
  \end{phonetics}
\end{entry}

\begin{entry}{评论}{7,6}{⾔、⾔}
  \begin{phonetics}{评论}{ping2lun4}[][HSK 5]
    \definition[篇]{s.}{revisão; comentário; artigos ou comentários críticos}
    \definition{v.}{discutir; comentar sobre algo ou alguém}
  \end{phonetics}
\end{entry}

\begin{entry}{评估}{7,7}{⾔、⼈}
  \begin{phonetics}{评估}{ping2gu1}[][HSK 5]
    \definition{v.}{estimar; avaliar; apreciar; avaliar e estimar (coisas abstratas)}
  \end{phonetics}
\end{entry}

\begin{entry}{诅咒}{7,8}{⾔、⼝}
  \begin{phonetics}{诅咒}{zu3zhou4}
    \definition{v.}{amaldiçoar}
  \end{phonetics}
\end{entry}

\begin{entry}{诊断}{7,11}{⾔、⽄}
  \begin{phonetics}{诊断}{zhen3duan4}[][HSK 5]
    \definition{s.}{diagnóstico; diacrisis}
    \definition{v.}{diagnosticar; após examinar os sintomas do paciente, determinar a doença e seu desenvolvimento}
  \end{phonetics}
\end{entry}

\begin{entry}{词}{7}{⾔}
  \begin{phonetics}{词}{ci2}[][HSK 2]
    \definition[个,组]{s.}{discurso | declaração | linhas de jogo | um tipo de poesia clássica chinesa, originária da Dinastia Tang e totalmente desenvolvida na Dinastia Song | palavra  | termo}
  \end{phonetics}
\end{entry}

\begin{entry}{词汇}{7,5}{⾔、⽔}
  \begin{phonetics}{词汇}{ci2hui4}[][HSK 4]
    \definition[个,组,批,串,堆]{s.}{vocabulário; termo geral para palavras usadas em um idioma}
  \end{phonetics}
\end{entry}

\begin{entry}{词典}{7,8}{⾔、⼋}
  \begin{phonetics}{词典}{ci2dian3}[][HSK 2]
    \definition[部,本]{s.}{dicionário}
  \seealsoref{字典}{zi4 dian3}
  \end{phonetics}
\end{entry}

\begin{entry}{词语}{7,9}{⾔、⾔}
  \begin{phonetics}{词语}{ci2yu3}[][HSK 2]
    \definition{s.}{palavra (termo geral, incluindo desdemonossilábicas até frases curtas) | termo (por exemplo, termo técnico) | expressão}
  \end{phonetics}
\end{entry}

\begin{entry}{试}{8}{⾔}
  \begin{phonetics}{试}{shi4}[][HSK 1]
    \definition{s.}{exame | experimento | prova | teste}
    \definition{v.}{experimentar | provar | testar}
  \end{phonetics}
\end{entry}

\begin{entry}{试卷}{8,8}{⾔、⼙}
  \begin{phonetics}{试卷}{shi4juan4}[][HSK 4]
    \definition[分,张]{s.}{folha de teste; folha de exame; papel usado para escrever as respostas nos exames}
  \end{phonetics}
\end{entry}

\begin{entry}{试图}{8,8}{⾔、⼞}
  \begin{phonetics}{试图}{shi4tu2}[][HSK 5]
    \definition{v.}{tentar; pretender, fazer o possível para realizar algo}
  \end{phonetics}
\end{entry}

\begin{entry}{试验}{8,10}{⾔、⾺}
  \begin{phonetics}{试验}{shi4yan4}[][HSK 3]
    \definition{v.}{testar; fazer um teste; fazer um experimento}
  \end{phonetics}
\end{entry}

\begin{entry}{试题}{8,15}{⾔、⾴}
  \begin{phonetics}{试题}{shi4 ti2}[][HSK 3]
    \definition[道]{s.}{perguntas de teste; perguntas de exame}
  \end{phonetics}
\end{entry}

\begin{entry}{诗}{8}{⾔}
  \begin{phonetics}{诗}{shi1}[][HSK 4]
    \definition*{s.}{O Livro das Canções《诗经》| sobrenome Shi}
    \definition{s.}{poesia; verso; poema;}
  \seealsoref{诗经}{shi1jing1}
  \end{phonetics}
\end{entry}

\begin{entry}{诗人}{8,2}{⾔、⼈}
  \begin{phonetics}{诗人}{shi1 ren2}[][HSK 4]
    \definition{s.}{poeta; escritor de poesia}
  \end{phonetics}
\end{entry}

\begin{entry}{诗句}{8,5}{⾔、⼝}
  \begin{phonetics}{诗句}{shi1ju4}
    \definition[行]{s.}{verso | versículo}
  \end{phonetics}
\end{entry}

\begin{entry}{诗词}{8,7}{⾔、⾔}
  \begin{phonetics}{诗词}{shi1ci2}
    \definition{s.}{verso}
  \end{phonetics}
\end{entry}

\begin{entry}{诗经}{8,8}{⾔、⽷}
  \begin{phonetics}{诗经}{shi1jing1}
    \definition*{s.}{Shijing, o Livro das Canções, antiga coleção de poemas chineses e um dos Cinco Clássicos do Confucionismo}
  \end{phonetics}
\end{entry}

\begin{entry}{诗意}{8,13}{⾔、⼼}
  \begin{phonetics}{诗意}{shi1yi4}
    \definition{adj.}{poético}
    \definition{s.}{poesia}
  \end{phonetics}
\end{entry}

\begin{entry}{诗歌}{8,14}{⾔、⽋}
  \begin{phonetics}{诗歌}{shi1 ge1}[][HSK 5]
    \definition[本,首,段]{s.}{poesia; poemas e canções; refere-se a todos os tipos de poesia}
  \end{phonetics}
\end{entry}

\begin{entry}{诚实}{8,8}{⾔、⼧}
  \begin{phonetics}{诚实}{cheng2shi2}[][HSK 4]
    \definition{adj.}{honesto; sincero e honesto, não hipócrita}
  \end{phonetics}
\end{entry}

\begin{entry}{诚实地}{8,8,6}{⾔、⼧、⼟}
  \begin{phonetics}{诚实地}{cheng2shi2 di4}
    \definition{adv.}{honestamente}
  \end{phonetics}
\end{entry}

\begin{entry}{诚信}{8,9}{⾔、⼈}
  \begin{phonetics}{诚信}{cheng2 xin4}[][HSK 4]
    \definition{adj.}{honesto e confiável}
    \definition[种]{s.}{fé; honestidade; padrão e princípio de comportamento: não contar mentiras, prometer aos outros o que eles podem fazer e ter a confiança dos outros}
  \end{phonetics}
\end{entry}

\begin{entry}{话}{8}{⾔}
  \begin{phonetics}{话}{hua4}[][HSK 1]
    \definition[种,席,句,口,番]{s.}{fala | linguagem | dialeto}
  \end{phonetics}
\end{entry}

\begin{entry}{话剧}{8,10}{⾔、⼑}
  \begin{phonetics}{话剧}{hua4 ju4}[][HSK 3]
    \definition[台,部]{s.}{drama moderno; peça de teatro}
  \end{phonetics}
\end{entry}

\begin{entry}{话题}{8,15}{⾔、⾴}
  \begin{phonetics}{话题}{hua4ti2}[][HSK 3]
    \definition[个,种,项]{s.}{assunto de uma palestra; tópico de uma conversa}
  \end{phonetics}
\end{entry}

\begin{entry}{诟骂}{8,9}{⾔、⾺}
  \begin{phonetics}{诟骂}{gou4ma4}
    \definition{v.}{abusar verbalmente | insultar | criticar}
  \end{phonetics}
\end{entry}

\begin{entry}{询问}{8,6}{⾔、⾨}
  \begin{phonetics}{询问}{xun2wen4}[][HSK 5]
    \definition{v.}{indagar; perguntar sobre; pedir conselho}
  \end{phonetics}
\end{entry}

\begin{entry}{该}{8}{⾔}
  \begin{phonetics}{该}{gai1}[][HSK 2]
    \definition{v.}{deveria | é a vez de alguém fazer algo | merecer | dever}
  \end{phonetics}
\end{entry}

\begin{entry}{详细}{8,8}{⾔、⽷}
  \begin{phonetics}{详细}{xiang2xi4}[][HSK 5]
    \definition{adj.}{explícito; detalhado; minucioso; circunstancial; meticuloso}
  \end{phonetics}
\end{entry}

\begin{entry}{语}{9}{⾔}
  \begin{phonetics}{语}{yu3}
    \definition{s.}{língua; linguagem | dito; provérbio; refere-se especialmente a coloquialismos, provérbios, expressões idiomáticas ou palavras de livros antigos | sinal; meio não linguístico de comunicar ideias ; ações ou sinais que substituem palavras para expressar significado | palavras; expressão; refere-se a uma palavra, frase ou sentença}
    \definition{v.}{dizer; falar | (de pássaros, insetos, etc.) gorjear; pipilar}
  \end{phonetics}
  \begin{phonetics}{语}{yu4}
    \definition{v.}{contar; informar}
  \end{phonetics}
\end{entry}

\begin{entry}{语气}{9,4}{⾔、⽓}
  \begin{phonetics}{语气}{yu3qi4}
    \definition[个]{s.}{maneira de falar | tom}
  \end{phonetics}
\end{entry}

\begin{entry}{语言}{9,7}{⾔、⾔}
  \begin{phonetics}{语言}{yu3yan2}[][HSK 2]
    \definition[门,种]{s.}{linguagem | língua}
  \end{phonetics}
\end{entry}

\begin{entry}{语言实验室}{9,7,8,10,9}{⾔、⾔、⼧、⾺、⼧}
  \begin{phonetics}{语言实验室}{yu3yan2shi2yan4shi4}
    \definition{s.}{laboratório de línguas}
  \end{phonetics}
\end{entry}

\begin{entry}{语法}{9,8}{⾔、⽔}
  \begin{phonetics}{语法}{yu3fa3}[][HSK 4]
    \definition[个]{s.}{gramática; maneira como o idioma é estruturado, incluindo a formação e as variações de palavras, a organização de frases e sentenças | estudo da gramática; estudo das regras de estrutura linguística}
  \end{phonetics}
\end{entry}

\begin{entry}{语法术语}{9,8,5,9}{⾔、⽔、⽊、⾔}
  \begin{phonetics}{语法术语}{yu3fa3shu4yu3}
    \definition{s.}{termo gramatical}
  \end{phonetics}
\end{entry}

\begin{entry}{语音}{9,9}{⾔、⾳}
  \begin{phonetics}{语音}{yu3 yin1}[][HSK 4]
    \definition{s.}{voz; pronúncia; sons da fala; som de alguém falando | pronúncia; som do idioma}
  \end{phonetics}
\end{entry}

\begin{entry}{语调}{9,10}{⾔、⾔}
  \begin{phonetics}{语调}{yu3diao4}
    \definition[个]{s.}{entonação}
  \end{phonetics}
\end{entry}

\begin{entry}{误会}{9,6}{⾔、⼈}
  \begin{phonetics}{误会}{wu4hui4}
    \definition[场]{s.}{mal-entendido; desentendimentos ou conflitos decorrentes de mal-entendidos}
    \definition{v.}{entender mal; entender errado; interpretar mal; não entender; não entender corretamente o significado}
  \end{phonetics}
\end{entry}

\begin{entry}{误点}{9,9}{⾔、⽕}
  \begin{phonetics}{误点}{wu4dian3}
    \definition{v.+compl.}{atrasar | chegar tarde}
  \end{phonetics}
\end{entry}

\begin{entry}{误解}{9,13}{⾔、⾓}
  \begin{phonetics}{误解}{wu4jie3}[][HSK 5]
    \definition[种]{s.}{equívoco; mal-entendido; desentendimento}
    \definition{v.}{interpretar mal; interpretar erroneamente; não compreender corretamente}
  \end{phonetics}
\end{entry}

\begin{entry}{诱人}{9,2}{⾔、⼈}
  \begin{phonetics}{诱人}{you4ren2}
    \definition{adj.}{atraente | cativante}
  \end{phonetics}
\end{entry}

\begin{entry}{说}{9}{⾔}
  \begin{phonetics}{说}{shui4}
    \definition{v.}{persuadir}
  \end{phonetics}
  \begin{phonetics}{说}{shuo1}[][HSK 1]
    \definition{s.}{uma teoria (normalmente o último caractere, como em 日心说, teoria heliocêntrica)}
    \definition{v.}{falar | dizer | explicar | contar}
  \end{phonetics}
\end{entry}

\begin{entry}{说不定}{9,4,8}{⾔、⼀、⼧}
  \begin{phonetics}{说不定}{shuo1bu5ding4}[][HSK 4]
    \definition{adv.}{talvez; indica uma estimativa, possivelmente, provavelmente}
    \definition{v.}{não ter certeza; não estar certo; ser impreciso}
  \end{phonetics}
\end{entry}

\begin{entry}{说好}{9,6}{⾔、⼥}
  \begin{phonetics}{说好}{shuo1hao3}
    \definition{v.}{chegar a um acordo | concluir negociações}
  \end{phonetics}
\end{entry}

\begin{entry}{说完}{9,7}{⾔、⼧}
  \begin{phonetics}{说完}{shuo1-wan2}
    \definition{expr.}{acabar/terminar palavras}
  \end{phonetics}
\end{entry}

\begin{entry}{说明}{9,8}{⾔、⽇}
  \begin{phonetics}{说明}{shuo1ming2}[][HSK 2]
    \definition[本,个]{s.}{legenda | instrução | explicação}
    \definition{v.}{mostrar | explicar | ilustrar | indicar | provar | demonstrar}
  \end{phonetics}
\end{entry}

\begin{entry}{说服}{9,8}{⾔、⽉}
  \begin{phonetics}{说服}{shuo1fu2}[][HSK 4]
    \definition{v.}{persuadir; convencer; convencer a outra parte com palavras bem fundamentadas}
  \end{phonetics}
\end{entry}

\begin{entry}{说法}{9,8}{⾔、⽔}
  \begin{phonetics}{说法}{shuo1 fa3}[][HSK 5]
    \definition[种]{s.}{maneira de dizer uma coisa; palavras ou frases usadas para expressar significado | declaração; versão; argumento; opinião; ponto de vista | motivo; razão; motivos ou bases para a resolução do problema}
  \end{phonetics}
\end{entry}

\begin{entry}{说话}{9,8}{⾔、⾔}
  \begin{phonetics}{说话}{shuo1 hua4}[][HSK 1]
    \definition{adv.}{imediatamente | em um minuto}
    \definition{v.}{falar | dizer | bater-papo | conversar | fofocar}
  \end{phonetics}
\end{entry}

\begin{entry}{说理}{9,11}{⾔、⽟}
  \begin{phonetics}{说理}{shuo1li3}
    \definition{v.}{racionalizar | discutir logicamente}
  \end{phonetics}
\end{entry}

\begin{entry}{说谎}{9,11}{⾔、⾔}
  \begin{phonetics}{说谎}{shuo1huang3}
    \definition{v.+compl.}{mentir | contar uma mentira}
  \end{phonetics}
\end{entry}

\begin{entry}{请}{10}{⾔}
  \begin{phonetics}{请}{qing3}[][HSK 1]
    \definition{v.}{por favor (fazer alguma coisa) | perguntar | convidar | solicitar}
  \end{phonetics}
\end{entry}

\begin{entry}{请问}{10,6}{⾔、⾨}
  \begin{phonetics}{请问}{qing3wen4}[][HSK 1]
    \definition{expr.}{Com licença, posso perguntar\dots? (para perguntar por qualquer coisa)}
  \end{phonetics}
\end{entry}

\begin{entry}{请坐}{10,7}{⾔、⼟}
  \begin{phonetics}{请坐}{qing3 zuo4}[][HSK 1]
    \definition{v.}{por favor, sente-se}
  \end{phonetics}
\end{entry}

\begin{entry}{请求}{10,7}{⾔、⽔}
  \begin{phonetics}{请求}{qing3qiu2}[][HSK 2]
    \definition[个]{s.}{solicitação}
    \definition{v.}{solicitar | perguntar}
  \end{phonetics}
\end{entry}

\begin{entry}{请进}{10,7}{⾔、⾡}
  \begin{phonetics}{请进}{qing3 jin4}[][HSK 1]
    \definition{v.}{por favor entre}
  \end{phonetics}
\end{entry}

\begin{entry}{请客}{10,9}{⾔、⼧}
  \begin{phonetics}{请客}{qing3ke4}[][HSK 2]
    \definition{v.+compl.}{entreter os convidados | dar um jantar | convidar para jantar}
  \end{phonetics}
\end{entry}

\begin{entry}{请假}{10,11}{⾔、⼈}
  \begin{phonetics}{请假}{qing3 jia4}[][HSK 1]
    \definition{v.+compl.}{pedir licença para sair}
  \end{phonetics}
\end{entry}

\begin{entry}{请假条}{10,11,7}{⾔、⼈、⽊}
  \begin{phonetics}{请假条}{qing3jia4tiao2}
    \definition{s.}{pedido de licença de ausência (do trabalho ou da escola)}
  \end{phonetics}
\end{entry}

\begin{entry}{请教}{10,11}{⾔、⽁}
  \begin{phonetics}{请教}{qing3jiao4}[][HSK 3]
    \definition{v.}{consultar; pedir conselho}
  \end{phonetics}
\end{entry}

\begin{entry}{诺贝尔奖}{10,4,5,9}{⾔、⾙、⼩、⼤}
  \begin{phonetics}{诺贝尔奖}{nuo4bei4'er3 jiang3}
    \definition*{s.}{Prêmio Nobel}
  \end{phonetics}
\end{entry}

\begin{entry}{诺奖}{10,9}{⾔、⼤}
  \begin{phonetics}{诺奖}{nuo4jiang3}
    \definition*{s.}{Prêmio Nobel, abreviação de 诺贝尔奖}
    \seeref{诺贝尔奖}{nuo4bei4'er3 jiang3}
  \end{phonetics}
\end{entry}

\begin{entry}{读}{10}{⾔}
  \begin{phonetics}{读}{dou4}
    \definition{s.}{vírgula | frase marcada por pausa}
  \end{phonetics}
  \begin{phonetics}{读}{du2}[][HSK 1]
    \definition{v.}{ler em voz alta | ler | frequentar (escola) | estudar (uma matéria na escola) | pronunciar}
  \end{phonetics}
\end{entry}

\begin{entry}{读书}{10,4}{⾔、⼄}
  \begin{phonetics}{读书}{du2 shu1}[][HSK 1]
    \definition{v.+compl.}{ler | estudar | frequentar a escola}
  \end{phonetics}
\end{entry}

\begin{entry}{读者}{10,8}{⾔、⽼}
  \begin{phonetics}{读者}{du2 zhe3}[][HSK 3]
    \definition[个,位]{s.}{leitor}
  \end{phonetics}
\end{entry}

\begin{entry}{读音}{10,9}{⾔、⾳}
  \begin{phonetics}{读音}{du2 yin1}[][HSK 2]
    \definition{s.}{pronúncia}
  \end{phonetics}
\end{entry}

\begin{entry}{课}{10}{⾔}
  \begin{phonetics}{课}{ke4}[][HSK 1]
    \definition{s.}{aula | curso | lição | imposto | taxa |seção}
  \end{phonetics}
\end{entry}

\begin{entry}{课文}{10,4}{⾔、⽂}
  \begin{phonetics}{课文}{ke4 wen2}[][HSK 1]
    \definition{s.}{texto (de uma lição)}
  \end{phonetics}
\end{entry}

\begin{entry}{课本}{10,5}{⾔、⽊}
  \begin{phonetics}{课本}{ke4 ben3}[][HSK 1]
    \definition[本]{s.}{livro do aluno | manual}
  \end{phonetics}
\end{entry}

\begin{entry}{课堂}{10,11}{⾔、⼟}
  \begin{phonetics}{课堂}{ke4 tang2}[][HSK 2]
    \definition[间]{s.}{sala de aula}
  \end{phonetics}
\end{entry}

\begin{entry}{课程}{10,12}{⾔、⽲}
  \begin{phonetics}{课程}{ke4cheng2}[][HSK 3]
    \definition[个,堂,节,门]{s.}{curso; currículo}
  \end{phonetics}
\end{entry}

\begin{entry}{课题}{10,15}{⾔、⾴}
  \begin{phonetics}{课题}{ke4ti2}[][HSK 5]
    \definition{s.}{uma questão para estudo ou discussão; principais questões a serem pesquisadas ou discutidas, ou assuntos importantes que precisam ser resolvidos com urgência | tarefa; problema; questões a serem resolvidas}
  \end{phonetics}
\end{entry}

\begin{entry}{谁}{10}{⾔}
  \begin{phonetics}{谁}{shei2}[][HSK 1]
    \definition{pron.}{quem?}
  \end{phonetics}
  \begin{phonetics}{谁}{shui2}[][HSK 1]
    \definition{pron.}{quem?}
  \end{phonetics}
\end{entry}

\begin{entry}{调}{10}{⾔}
  \begin{phonetics}{调}{diao4}[][HSK 3]
    \definition{s.}{sotaque | nota (musical) | melodia; tom}
    \definition{v.}{transferir; deslocar; mover | distribuir; alocar | trocar; permutar; comutar}
  \end{phonetics}
  \begin{phonetics}{调}{tiao2}[][HSK 3]
    \definition*{s.}{sobrenome Tiao}
    \definition{v.}{harmonizar; adequar-se bem; encaixar-se perfeitamente | mesclar; ajustar; ajustar; misturar | mediar |gracejar; tirar sarro de | provocar; alienar}
  \end{phonetics}
\end{entry}

\begin{entry}{调皮}{10,5}{⾔、⽪}
  \begin{phonetics}{调皮}{tiao2pi2}[][HSK 4]
    \definition{adj.}{travesso; malicioso; malandro | indisciplinado; desordeiro; indomável; astuto | inteligente e desonesto}
  \end{phonetics}
\end{entry}

\begin{entry}{调节}{10,5}{⾔、⾋}
  \begin{phonetics}{调节}{tiao2jie2}[][HSK 5]
    \definition{v.}{regular; ajustar; ajustar e controlar de várias maneiras para atender aos requisitos}
  \end{phonetics}
\end{entry}

\begin{entry}{调动}{10,6}{⾔、⼒}
  \begin{phonetics}{调动}{diao4dong4}[][HSK 5]
    \definition{v.}{mudar; transferir; pessoal, trabalho | mobilizar; despertar; pôr em jogo; melhorar (motivação, entusiasmo, etc.) por meio de alguns meios | reunir; manobrar; mover (tropas); mobilizar forças militares}
  \end{phonetics}
\end{entry}

\begin{entry}{调律}{10,9}{⾔、⼻}
  \begin{phonetics}{调律}{tiao2lv4}
    \definition{v.}{afinar (por exemplo, um piano)}
  \end{phonetics}
\end{entry}

\begin{entry}{调查}{10,9}{⾔、⽊}
  \begin{phonetics}{调查}{diao4cha2}[][HSK 3]
    \definition[项,个]{s.}{pesquisa; investigação}
    \definition{v.}{investigar; indagar; inquerir}
  \end{phonetics}
\end{entry}

\begin{entry}{调解}{10,13}{⾔、⾓}
  \begin{phonetics}{调解}{tiao2jie3}[][HSK 5]
    \definition{v.}{mediar; fazer as pazes; resolver conflitos através da persuasão}
  \end{phonetics}
\end{entry}

\begin{entry}{调整}{10,16}{⾔、⽁}
  \begin{phonetics}{调整}{tiao2zheng3}[][HSK 3]
    \definition{v.}{ajustar; revisar; regular}
  \end{phonetics}
\end{entry}

\begin{entry}{谈}{10}{⾔}
  \begin{phonetics}{谈}{tan2}[][HSK 3]
    \definition*{s.}{sobrenome Tan}
    \definition{s.}{o que é dito ou falado}
    \definition{v.}{falar; bater papo; discutir}
  \end{phonetics}
\end{entry}

\begin{entry}{谈判}{10,7}{⾔、⼑}
  \begin{phonetics}{谈判}{tan2pan4}[][HSK 3]
    \definition{v.}{negociar; manter conversações}
  \end{phonetics}
\end{entry}

\begin{entry}{谈话}{10,8}{⾔、⾔}
  \begin{phonetics}{谈话}{tan2 hua4}[][HSK 3]
    \definition[次]{s.}{declaração}
    \definition{v.+compl.}{conversar; discutir | falar}
  \end{phonetics}
\end{entry}

\begin{entry}{谈恋爱}{10,10,10}{⾔、⼼、⽖}
  \begin{phonetics}{谈恋爱}{tan2lian4'ai4}
    \definition{v.}{namorar | apaixonar-se}
  \end{phonetics}
\end{entry}

\begin{entry}{谎话}{11,8}{⾔、⾔}
  \begin{phonetics}{谎话}{huang3hua4}
    \definition{s.}{mentira}
  \end{phonetics}
\end{entry}

\begin{entry}{谐}{11}{⾔}
  \begin{phonetics}{谐}{xie2}
    \definition{adj.}{harmonioso | humorístico}
  \end{phonetics}
\end{entry}

\begin{entry}{粤语}{12,9}{⾔、⾔}
  \begin{phonetics}{粤语}{yue4yu3}
    \definition{s.}{cantonês | língua cantonesa}
  \end{phonetics}
\end{entry}

\begin{entry}{詈骂}{12,9}{⾔、⾺}
  \begin{phonetics}{詈骂}{li4ma4}
    \definition{v.}{xingar | abusar}
  \end{phonetics}
\end{entry}

\begin{entry}{谢天谢地}{12,4,12,6}{⾔、⼤、⾔、⼟}
  \begin{phonetics}{谢天谢地}{xie4tian1xie4di4}
    \definition{expr.}{agradecer a Deus | agradecer aos céus}
  \end{phonetics}
\end{entry}

\begin{entry}{谢世}{12,5}{⾔、⼀}
  \begin{phonetics}{谢世}{xie4shi4}
    \definition{v.}{morrer | falecer}
  \end{phonetics}
\end{entry}

\begin{entry}{谢恩}{12,10}{⾔、⼼}
  \begin{phonetics}{谢恩}{xie4'en1}
    \definition{v.}{agradecer a alguém pelo favor (especialmente imperador ou oficial superior)}
  \end{phonetics}
\end{entry}

\begin{entry}{谢病}{12,10}{⾔、⽧}
  \begin{phonetics}{谢病}{xie4bing4}
    \definition{v.}{desculpar-se por causa de doença}
  \end{phonetics}
\end{entry}

\begin{entry}{谢媒}{12,12}{⾔、⼥}
  \begin{phonetics}{谢媒}{xie4mei2}
    \definition{v.}{agradecer ao casamenteiro}
  \end{phonetics}
\end{entry}

\begin{entry}{谢谢}{12,12}{⾔、⾔}
  \begin{phonetics}{谢谢}{xie4xie5}[][HSK 1]
    \definition{interj.}{Obrigado!}
    \definition{v.}{agradecer}
  \end{phonetics}
\end{entry}

\begin{entry}{谢意}{12,13}{⾔、⼼}
  \begin{phonetics}{谢意}{xie4yi4}
    \definition{s.}{gratidão}
  \end{phonetics}
\end{entry}

\begin{entry}{谩骂}{13,9}{⾔、⾺}
  \begin{phonetics}{谩骂}{man4ma4}
    \definition{v.}{ridicularizar | abusar}
  \end{phonetics}
\end{entry}

\begin{entry}{警}{19}{⾔}
  \begin{phonetics}{警}{jing3}
    \definition{s.}{policial}
    \definition{v.}{alertar | avisar}
  \end{phonetics}
\end{entry}

\begin{entry}{警告}{19,7}{⾔、⼝}
  \begin{phonetics}{警告}{jing3gao4}[][HSK 5]
    \definition[个]{s.}{advertência (como medida disciplinar); uma forma de punição}
    \definition{v.}{avisar; advertir; admoestar}
  \end{phonetics}
\end{entry}

\begin{entry}{警官}{19,8}{⾔、⼧}
  \begin{phonetics}{警官}{jing3guan1}
    \definition{s.}{polícia | policial}
  \end{phonetics}
\end{entry}

\begin{entry}{警察}{19,14}{⾔、⼧}
  \begin{phonetics}{警察}{jing3cha2}[][HSK 3]
    \definition[个,位,名,群,队]{s.}{polícia; policial; oficial de polícia}
  \end{phonetics}
\end{entry}

\begin{entry}{譬如}{20,6}{⾔、⼥}
  \begin{phonetics}{譬如}{pi4ru2}
    \definition{conj.}{por exemplo | como}
  \end{phonetics}
\end{entry}

%%%%% EOF %%%%%

