%%%
%%% Radical "⼗"
%%%

\section*{Radical 24: ``⼗''}\addcontentsline{toc}{section}{Radical 24: ⼗}

\begin{entry}{十}{2}{⼗}[Kangxi 24]
  \begin{phonetics}{十}{shi2}[][HSK 1]
    \definition{num.}{dez; 10 | dezena}
  \end{phonetics}
\end{entry}

\begin{entry}{十分}{2,4}{⼗、⼑}
  \begin{phonetics}{十分}{shi2fen1}[][HSK 2]
    \definition{adv.}{muito | extremamente | totalmente | absolutamente}
  \end{phonetics}
\end{entry}

\begin{entry}{十足}{2,7}{⼗、⾜}
  \begin{phonetics}{十足}{shi2zu2}
    \definition{adj.}{amplo | completo | cento por cento | tom puro (de alguma cor)}
  \end{phonetics}
\end{entry}

\begin{entry}{千}{3}{⼗}
  \begin{phonetics}{千}{qian1}[][HSK 2]
    \definition{num.}{mil; 1.000; 1000}
  \end{phonetics}
\end{entry}

\begin{entry}{千万}{3,3}{⼗、⼀}
  \begin{phonetics}{千万}{qian1wan4}[][HSK 3]
    \definition{adv.}{(usado para indicar desejos fortes) por todos os meios; sob quaisquer circunstâncias}
    \definition{num.}{dez milhões; milhões e milhões}
  \end{phonetics}
\end{entry}

\begin{entry}{千千万万}{3,3,3,3}{⼗、⼗、⼀、⼀}
  \begin{phonetics}{千千万万}{qian1qian1wan4wan4}
    \definition{num.}{inumerável | números incontáveis | milhares e milhares}
  \end{phonetics}
\end{entry}

\begin{entry}{千古}{3,5}{⼗、⼝}
  \begin{phonetics}{千古}{qian1gu3}
    \definition{adv.}{por toda a eternidade | em todas as idades}
    \definition{s.}{eternidade (usada em um dístico elegíaco, coroa de flores, etc., dedicada aos mortos)}
  \end{phonetics}
\end{entry}

\begin{entry}{千年}{3,6}{⼗、⼲}
  \begin{phonetics}{千年}{qian1nian2}
    \definition{s.}{milênio}
  \end{phonetics}
\end{entry}

\begin{entry}{千克}{3,7}{⼗、⼗}
  \begin{phonetics}{千克}{qian1 ke4}[][HSK 2]
    \definition{clas.}{kg | quilo | quilograma}
  \end{phonetics}
\end{entry}

\begin{entry}{升}{4}{⼗}
  \begin{phonetics}{升}{sheng1}[][HSK 3]
    \definition*{s.}{sobrenome Sheng}
    \definition{clas.}{litro (l)}
    \definition{s.}{sheng, uma unidade de medida seca para grãos (= 1 litro)}
    \definition{v.}{elevar; içar; subir; ascender | promover}
  \end{phonetics}
\end{entry}

\begin{entry}{升起}{4,10}{⼗、⾛}
  \begin{phonetics}{升起}{sheng1qi3}
    \definition{v.}{levantar | içar | subir}
  \end{phonetics}
\end{entry}

\begin{entry}{午}{4}{⼗}
  \begin{phonetics}{午}{wu3}
    \definition{s.}{período entre 11h00 e 13h00, meio-dia}
  \end{phonetics}
\end{entry}

\begin{entry}{午休}{4,6}{⼗、⼈}
  \begin{phonetics}{午休}{wu3xiu1}
    \definition{s.}{pausa para almoço | cochilo na hora do almoço | intervalo do meio-dia}
  \end{phonetics}
\end{entry}

\begin{entry}{午后}{4,6}{⼗、⼝}
  \begin{phonetics}{午后}{wu3hou4}
    \definition{s.}{tarde | período da tarde}
  \end{phonetics}
\end{entry}

\begin{entry}{午饭}{4,7}{⼗、⾷}
  \begin{phonetics}{午饭}{wu3fan4}[][HSK 1]
    \definition[份,顿,次,餐]{s.}{almoço}
  \seealsoref{午餐}{wu3 can1}
  \end{phonetics}
\end{entry}

\begin{entry}{午夜}{4,8}{⼗、⼣}
  \begin{phonetics}{午夜}{wu3ye4}
    \definition{s.}{meia-noite}
  \end{phonetics}
\end{entry}

\begin{entry}{午前}{4,9}{⼗、⼑}
  \begin{phonetics}{午前}{wu3qian2}
    \definition{s.}{\emph{A.M.} | manhã | período da manhã}
  \end{phonetics}
\end{entry}

\begin{entry}{午宴}{4,10}{⼗、⼧}
  \begin{phonetics}{午宴}{wu3yan4}
    \definition{s.}{banquete de almoço}
  \end{phonetics}
\end{entry}

\begin{entry}{午睡}{4,13}{⼗、⽬}
  \begin{phonetics}{午睡}{wu3 shui4}[][HSK 2]
    \definition{s.}{siesta}
    \definition{v.}{tirar uma soneca}
  \end{phonetics}
\end{entry}

\begin{entry}{午餐}{4,16}{⼗、⾷}
  \begin{phonetics}{午餐}{wu3 can1}[][HSK 2]
    \definition[份,顿,次]{s.}{almoço}
  \seealsoref{午饭}{wu3fan4}
  \end{phonetics}
\end{entry}

\begin{entry}{半}{5}{⼗}
  \begin{phonetics}{半}{ban4}[][HSK 1]
    \definition{adj.}{incompleto}
    \definition{num.}{(depois de um número) ``e meio''}
    \definition{pref.}{semi-}
    \definition{s.}{metade}
  \end{phonetics}
\end{entry}

\begin{entry}{半天}{5,4}{⼗、⼤}
  \begin{phonetics}{半天}{ban4 tian1}[][HSK 1]
    \definition{s.}{metade do dia | muito tempo | bastante tempo}
  \end{phonetics}
\end{entry}

\begin{entry}{半年}{5,6}{⼗、⼲}
  \begin{phonetics}{半年}{ban4 nian2}[][HSK 1]
    \definition{s.}{meio ano}
  \end{phonetics}
\end{entry}

\begin{entry}{半夜}{5,8}{⼗、⼣}
  \begin{phonetics}{半夜}{ban4 ye4}[][HSK 2]
    \definition{adv.}{no meio da noite | metade de uma noite}
    \definition{s.}{meia-noite}
  \end{phonetics}
\end{entry}

\begin{entry}{半音}{5,9}{⼗、⾳}
  \begin{phonetics}{半音}{ban4yin1}
    \definition{s.}{semitom}
  \end{phonetics}
\end{entry}

\begin{entry}{半球}{5,11}{⼗、⽟}
  \begin{phonetics}{半球}{ban4qiu2}
    \definition{s.}{hemisfério}
  \end{phonetics}
\end{entry}

\begin{entry}{华人}{6,2}{⼗、⼈}
  \begin{phonetics}{华人}{hua2 ren2}[][HSK 3]
    \definition{s.}{Chinês; chinês étnico | cidadãos estrangeiros de ascendência chinesa que adquiriram nacionalidade no seu país de residência}
  \end{phonetics}
\end{entry}

\begin{entry}{华氏}{6,4}{⼗、⽒}
  \begin{phonetics}{华氏}{hua2shi4}
    \definition{s.}{graus Fahrenheit (°F)}
  \end{phonetics}
\end{entry}

\begin{entry}{华语}{6,9}{⼗、⾔}
  \begin{phonetics}{华语}{hua2 yu3}[][HSK 5]
    \definition*{s.}{Chinês (idioma)}
  \end{phonetics}
\end{entry}

\begin{entry}{华夏}{6,10}{⼗、⼢}
  \begin{phonetics}{华夏}{hua2xia4}
    \definition*{s.}{Huaxia, nome antigo da China | Catai}
  \end{phonetics}
\end{entry}

\begin{entry}{华盛顿}{6,11,10}{⼗、⽫、⾴}
  \begin{phonetics}{华盛顿}{hua2sheng4dun4}
    \definition*{s.}{Washington}
  \end{phonetics}
\end{entry}

\begin{entry}{华裔}{6,13}{⼗、⾐}
  \begin{phonetics}{华裔}{hua2yi4}
    \definition{s.}{descendente de chinês}
  \end{phonetics}
\end{entry}

\begin{entry}{克}{7}{⼗}
  \begin{phonetics}{克}{ke4}[][HSK 2]
    \definition{clas.}{grama (g)}
    \definition{v.}{pode | ser capaz de | restringir | controlar | superar | subjugar | capturar (uma cidade, etc.) | digerir | cortar | reduzir | definir um limite de tempo}
  \end{phonetics}
\end{entry}

\begin{entry}{克服}{7,8}{⼗、⽉}
  \begin{phonetics}{克服}{ke4fu2}[][HSK 3]
    \definition{v.}{sobrepujar; superar; conquistar | suportar (dificuldades, inconveniências, etc.)}
  \end{phonetics}
\end{entry}

\begin{entry}{丧钟}{8,9}{⼗、⾦}
  \begin{phonetics}{丧钟}{sang1zhong1}
    \definition{s.}{sentença de morte}
  \end{phonetics}
\end{entry}

\begin{entry}{单}{8}{⼗}
  \begin{phonetics}{单}{chan2}
    \definition{s.}{usado em 单于 \dpy{chan2yu2}}
    \seeref{单于}{chan2yu2}
  \end{phonetics}
  \begin{phonetics}{单}{dan1}[][HSK 4]
    \definition{adj.}{sozinho; único | ímpar | sem forro (vestuário) | simples; poucos itens ou categorias; não é complexo | fino; fraco; frágil}
    \definition{adv.}{isoladamente; sozinho | somente; sozinho; unicamente | somente; apenas}
    \definition[个]{s.}{lençol; um pano grande para cobrir a cama | conta; lista; pedaços de papel que detalham coisas}
  \end{phonetics}
  \begin{phonetics}{单}{shan4}
    \definition*{s.}{sobrenome Shan}
  \end{phonetics}
\end{entry}

\begin{entry}{单一}{8,1}{⼗、⼀}
  \begin{phonetics}{单一}{dan1 yi1}[][HSK 5]
    \definition{adj.}{único; unitário; exclusivo}
  \end{phonetics}
\end{entry}

\begin{entry}{单于}{8,3}{⼗、⼆}
  \begin{phonetics}{单于}{chan2yu2}
    \definition{s.}{rei de Xiongnu (匈奴)}
  \seealsoref{匈奴}{xiong1nu2}
  \end{phonetics}
\end{entry}

\begin{entry}{单元}{8,4}{⼗、⼉}
  \begin{phonetics}{单元}{dan1yuan2}[][HSK 3]
    \definition[个,组,套]{s.}{unidade (de algo)}
  \end{phonetics}
\end{entry}

\begin{entry}{单位}{8,7}{⼗、⼈}
  \begin{phonetics}{单位}{dan1wei4}[][HSK 2]
    \definition[个]{s.}{unidade (como padrão de medida) | unidade (como uma organização, departamento, divisão, seção, etc.)}
  \end{phonetics}
\end{entry}

\begin{entry}{单纯}{8,7}{⼗、⽷}
  \begin{phonetics}{单纯}{dan1chun2}[][HSK 4]
    \definition{adj.}{puro; simples; descomplicado}
    \definition{adv.}{sozinho; puramente; meramente}
  \end{phonetics}
\end{entry}

\begin{entry}{单质}{8,8}{⼗、⾙}
  \begin{phonetics}{单质}{dan1zhi4}
    \definition{s.}{substância simples (consistindo puramente de um elemento, como diamante, grafite, etc.)}
  \end{phonetics}
\end{entry}

\begin{entry}{单独}{8,9}{⼗、⽝}
  \begin{phonetics}{单独}{dan1du2}[][HSK 4]
    \definition{adv.}{solo; sozinho; por si mesmo; por conta própria}
  \end{phonetics}
\end{entry}

\begin{entry}{单调}{8,10}{⼗、⾔}
  \begin{phonetics}{单调}{dan1diao4}[][HSK 4]
    \definition{adj.}{maçante; monótono}
  \end{phonetics}
\end{entry}

\begin{entry}{单脚滑行车}{8,11,12,6,4}{⼗、⾁、⽔、⾏、⾞}
  \begin{phonetics}{单脚滑行车}{dan1jiao3hua2xing2che1}
    \definition{s.}{\emph{scooter}}
  \end{phonetics}
\end{entry}

\begin{entry}{卖}{8}{⼗}
  \begin{phonetics}{卖}{mai4}[][HSK 2]
    \definition{v.}{vender}
  \end{phonetics}
\end{entry}

\begin{entry}{南}{9}{⼗}
  \begin{phonetics}{南}{nan2}[][HSK 1]
    \definition*{s.}{sobrenome Nan}
    \definition{s.}{sul}
  \end{phonetics}
\end{entry}

\begin{entry}{南方}{9,4}{⼗、⽅}
  \begin{phonetics}{南方}{nan2 fang1}[][HSK 2]
    \definition{s.}{sul | o Sul | a parte sul do país}
  \end{phonetics}
\end{entry}

\begin{entry}{南北}{9,5}{⼗、⼔}
  \begin{phonetics}{南北}{nan2 bei3}[][HSK 5]
    \definition{s.}{norte e sul | de norte a sul}
  \end{phonetics}
\end{entry}

\begin{entry}{南边}{9,5}{⼗、⾡}
  \begin{phonetics}{南边}{nan2bian5}[][HSK 1]
    \definition{adv.}{sul | lado sul | parte sul | ao sul de}
  \end{phonetics}
\end{entry}

\begin{entry}{南极}{9,7}{⼗、⽊}
  \begin{phonetics}{南极}{nan2ji2}[][HSK 5]
    \definition*{s.}{Polo Sul; Polo Antártico | Polo sul magnético}
    \definition{s.}{pólo sul magnético}
  \end{phonetics}
\end{entry}

\begin{entry}{南面}{9,9}{⼗、⾯}
  \begin{phonetics}{南面}{nan2mian4}
    \definition{s.}{sul | lado sul}
  \end{phonetics}
\end{entry}

\begin{entry}{南部}{9,10}{⼗、⾢}
  \begin{phonetics}{南部}{nan2 bu4}[][HSK 3]
    \definition{s.}{parte sul; sul | a parte sul}
  \end{phonetics}
\end{entry}

\begin{entry}{真}{10}{⼗}
  \begin{phonetics}{真}{zhen1}[][HSK 1]
    \definition{adj.}{genuíno}
    \definition{adv.}{que\dots tão\dots! | realmente}
  \end{phonetics}
\end{entry}

\begin{entry}{真切}{10,4}{⼗、⼑}
  \begin{phonetics}{真切}{zhen1qie4}
    \definition{adj.}{claro | distinto | honesto | sincero | vívido}
  \end{phonetics}
\end{entry}

\begin{entry}{真心}{10,4}{⼗、⼼}
  \begin{phonetics}{真心}{zhen1xin1}
    \definition{adj.}{sincero}
    \definition[片]{s.}{sinceridade}
  \end{phonetics}
\end{entry}

\begin{entry}{真牛}{10,4}{⼗、⽜}
  \begin{phonetics}{真牛}{zhen1niu2}
    \definition{adj.}{(gíria) muito legal, incrível}
  \end{phonetics}
\end{entry}

\begin{entry}{真正}{10,5}{⼗、⽌}
  \begin{phonetics}{真正}{zhen1zheng4}[][HSK 2]
    \definition{adj.}{verdadeiro | real | genuíno}
    \definition{adv.}{realmente | de ​​fato}
  \end{phonetics}
\end{entry}

\begin{entry}{真声}{10,7}{⼗、⼠}
  \begin{phonetics}{真声}{zhen1sheng1}
    \definition{s.}{voz natural | voz verdadeira}
    \seeref{假声}{jia3sheng1}
  \end{phonetics}
\end{entry}

\begin{entry}{真实}{10,8}{⼗、⼧}
  \begin{phonetics}{真实}{zhen1shi2}[][HSK 3]
    \definition{adj.}{verdadeiro; real; autêntico}
  \end{phonetics}
\end{entry}

\begin{entry}{真的}{10,8}{⼗、⽩}
  \begin{phonetics}{真的}{zhen1 de5}[][HSK 1]
    \definition{adv.}{realmente | verdadeiramente}
  \end{phonetics}
\end{entry}

\begin{entry}{真珠}{10,10}{⼗、⽟}
  \begin{phonetics}{真珠}{zhen1zhu1}
    \variantof{珍珠}
  \end{phonetics}
\end{entry}

\begin{entry}{真真}{10,10}{⼗、⼗}
  \begin{phonetics}{真真}{zhen1zhen1}
    \definition{adv.}{genuinamente | realmente | escrupulosamente}
  \end{phonetics}
\end{entry}

\begin{entry}{真理}{10,11}{⼗、⽟}
  \begin{phonetics}{真理}{zhen1li3}
    \definition[个]{s.}{verdade}
  \end{phonetics}
\end{entry}

\begin{entry}{真释}{10,12}{⼗、⾤}
  \begin{phonetics}{真释}{zhen1shi4}
    \definition{s.}{razão genuína | explicação verdadeira}
  \end{phonetics}
\end{entry}

\begin{entry}{博士}{12,3}{⼗、⼠}
  \begin{phonetics}{博士}{bo2shi4}[][HSK 5]
    \definition{s.}{doutorado; grau de doutor; nível mais alto de um diploma; também, uma pessoa que obteve esse diploma | doutor; antigo título honorífico para uma pessoa que é habilidosa em um determinado ofício ou especializada em uma determinada ocupação | doutor; autoridades que ensinavam as escrituras na China nos tempos antigos}
  \end{phonetics}
\end{entry}

\begin{entry}{博文}{12,4}{⼗、⽂}
  \begin{phonetics}{博文}{bo2wen2}
    \definition{s.}{artigo em um blog}
    \definition{v.}{escrever um artigo em um blog}
  \end{phonetics}
\end{entry}

\begin{entry}{博主}{12,5}{⼗、⼂}
  \begin{phonetics}{博主}{bo2zhu3}
    \definition{s.}{blogueiro}
  \end{phonetics}
\end{entry}

\begin{entry}{博物馆}{12,8,11}{⼗、⽜、⾷}
  \begin{phonetics}{博物馆}{bo2wu4guan3}[][HSK 5]
    \definition[个]{s.}{museu; locais para coleta, armazenamento, pesquisa, exibição e exposição de relíquias culturais ou espécimes relacionados à história, cultura, arte, ciências naturais, ciência e tecnologia, etc.}
  \end{phonetics}
\end{entry}

\begin{entry}{博客}{12,9}{⼗、⼧}
  \begin{phonetics}{博客}{bo2 ke4}[][HSK 5]
    \definition{s.}{\emph{blog}; página da Web ou site gerenciado por um indivíduo, geralmente composto por postagens organizadas da mais recente para a mais antiga | blogueiro; \emph{blogger}; pessoas que possuem ou escrevem \emph{blogs}}
  \end{phonetics}
\end{entry}

\begin{entry}{博览会}{12,9,6}{⼗、⾒、⼈}
  \begin{phonetics}{博览会}{bo2lan3hui4}[][HSK 5]
    \definition[次]{s.}{exposição; feira internacional; exposições de produtos em grande escala}
  \end{phonetics}
\end{entry}

%%%%% EOF %%%%%

