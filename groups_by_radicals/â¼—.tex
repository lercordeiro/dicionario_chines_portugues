%%%
%%% Radical "⼗"
%%%

\section*{Radical 24: ``⼗''}\addcontentsline{toc}{section}{Radical 24: ⼗}

\begin{Entry}{十}{2}{⼗}[Kangxi 24]
  \begin{Phonetics}{十}{shi2}[][HSK 1]
    \definition*{s.}{Sobrenome Shi}
    \definition{num.}{dez; 10 | dezena | completo; no topo; máximo; referindo-se a algo que atingiu o ápice da perfeição ou plenitude | um monte de; indica que há muitos}
  \end{Phonetics}
\end{Entry}

\begin{Entry}{十分}{2,4}{⼗、⼑}
  \begin{Phonetics}{十分}{shi2fen1}[][HSK 2]
    \definition{adv.}{muito; totalmente; completamente; extremamente; indica um nível muito alto}
  \end{Phonetics}
\end{Entry}

\begin{Entry}{十足}{2,7}{⼗、⾜}
  \begin{Phonetics}{十足}{shi2zu2}[][HSK 5]
    \definition{adj.}{puro e simples; apenas este componente ou esta característica é muito evidente | 100\%; completo; total; muito satisfatório; muito adequado}
  \end{Phonetics}
\end{Entry}

\begin{Entry}{千}{3}{⼗}
  \begin{Phonetics}{千}{qian1}[][HSK 2]
    \definition*{s.}{Sobrenome Qian}
    \definition{num.}{mil; 1.000; 1000 | a grande quantidade de; um grande número de}
  \end{Phonetics}
\end{Entry}

\begin{Entry}{千万}{3,3}{⼗、⼀}
  \begin{Phonetics}{千万}{qian1wan4}[][HSK 3]
    \definition{adv.}{(usado para indicar desejos fortes) por todos os meios; sob quaisquer circunstâncias; expressa uma exortação sincera, equivalente a 务必}
    \definition{num.}{dez milhões; 10.000.000; 1000.0000; milhões e milhões; um número aproximado, indicando um grande número}
  \seealsoref{务必}{wu4bi4}
  \end{Phonetics}
\end{Entry}

\begin{Entry}{千千万万}{3,3,3,3}{⼗、⼗、⼀、⼀}
  \begin{Phonetics}{千千万万}{qian1qian1wan4wan4}
    \definition{num.}{inumerável | números incontáveis | milhares e milhares}
  \end{Phonetics}
\end{Entry}

\begin{Entry}{千古}{3,5}{⼗、⼝}
  \begin{Phonetics}{千古}{qian1gu3}
    \definition{adv.}{por toda a eternidade | em todas as idades}
    \definition{s.}{eternidade (usada em um dístico elegíaco, coroa de flores, etc., dedicada aos mortos)}
  \end{Phonetics}
\end{Entry}

\begin{Entry}{千年}{3,6}{⼗、⼲}
  \begin{Phonetics}{千年}{qian1nian2}
    \definition{s.}{milênio}
  \end{Phonetics}
\end{Entry}

\begin{Entry}{千克}{3,7}{⼗、⼗}
  \begin{Phonetics}{千克}{qian1 ke4}[][HSK 2]
    \definition{clas.}{kg; quilo; quilograma; 1 quilograma equivale a 1.000 gramas, ou 2 jin (斤)}
  \seealsoref{斤}{jin1}
  \end{Phonetics}
\end{Entry}

\begin{Entry}{升}{4}{⼗}
  \begin{Phonetics}{升}{sheng1}[][HSK 3]
    \definition*{s.}{Sobrenome Sheng}
    \definition{clas.}{litro (l)}
    \definition{s.}{sheng, uma unidade de medida seca para grãos (= 1 litro), um décimo de 斗}
    \definition{v.}{elevar; içar; subir; ascender; subir ou subir mais alto (oposto de 降) | promover; melhorar (nível)}
  \seealsoref{斗}{dou4}
  \seealsoref{降}{jiang4}
  \end{Phonetics}
\end{Entry}

\begin{Entry}{升级}{4,6}{⼗、⽷}
  \begin{Phonetics}{升级}{sheng1 ji2}[][HSK 6]
    \definition{v.+compl.}{atualizar (software) | (guerra) escalar; (tensão) aprofundar | subir um ou mais níveis; passar de uma série ou classe inferior para uma série ou classe superior}
  \end{Phonetics}
\end{Entry}

\begin{Entry}{升学}{4,8}{⼗、⼦}
  \begin{Phonetics}{升学}{sheng1 xue2}[][HSK 6]
    \definition{v.}{ir para uma universidade, faculdade; entrar em uma universidade, faculdade}
  \end{Phonetics}
\end{Entry}

\begin{Entry}{升值}{4,10}{⼗、⼈}
  \begin{Phonetics}{升值}{sheng1 zhi2}[][HSK 6]
    \definition{v.}{Economia: reavaliar; apreciar | Figurativo: aumento de valor | valorização; apreciação; aumentar o valor; aumentar os preços}
  \end{Phonetics}
\end{Entry}

\begin{Entry}{升起}{4,10}{⼗、⾛}
  \begin{Phonetics}{升起}{sheng1qi3}
    \definition{v.}{levantar | içar | subir}
  \end{Phonetics}
\end{Entry}

\begin{Entry}{升高}{4,10}{⼗、⾼}
  \begin{Phonetics}{升高}{sheng1 gao1}[][HSK 5]
    \definition{v.}{subir; ascender | promover; elevar; intensificar; potencializar; melhorar}
  \end{Phonetics}
\end{Entry}

\begin{Entry}{午}{4}{⼗}
  \begin{Phonetics}{午}{wu3}
    \definition{s.}{meio-dia; período entre 11h00 e 13h00 | wu (sétimo dos doze Ramos Terrestres)}
  \end{Phonetics}
\end{Entry}

\begin{Entry}{午休}{4,6}{⼗、⼈}
  \begin{Phonetics}{午休}{wu3xiu1}
    \definition{s.}{pausa para almoço | cochilo na hora do almoço | intervalo do meio-dia}
  \end{Phonetics}
\end{Entry}

\begin{Entry}{午后}{4,6}{⼗、⼝}
  \begin{Phonetics}{午后}{wu3hou4}
    \definition{s.}{tarde | período da tarde}
  \end{Phonetics}
\end{Entry}

\begin{Entry}{午饭}{4,7}{⼗、⾷}
  \begin{Phonetics}{午饭}{wu3 fan4}[][HSK 1]
    \definition[顿]{s.}{almoço}
  \seealsoref{午餐}{wu3 can1}
  \end{Phonetics}
\end{Entry}

\begin{Entry}{午夜}{4,8}{⼗、⼣}
  \begin{Phonetics}{午夜}{wu3ye4}
    \definition{s.}{meia-noite}
  \end{Phonetics}
\end{Entry}

\begin{Entry}{午前}{4,9}{⼗、⼑}
  \begin{Phonetics}{午前}{wu3qian2}
    \definition{s.}{\emph{A.M.} | manhã | período da manhã}
  \end{Phonetics}
\end{Entry}

\begin{Entry}{午宴}{4,10}{⼗、⼧}
  \begin{Phonetics}{午宴}{wu3yan4}
    \definition{s.}{banquete de almoço}
  \end{Phonetics}
\end{Entry}

\begin{Entry}{午睡}{4,13}{⼗、⽬}
  \begin{Phonetics}{午睡}{wu3 shui4}[][HSK 2]
    \definition{s.}{\emph{siesta}; cochilo da tarde; soneca do meio-dia}
    \definition{v.}{tirar uma soneca depois do almoço}
  \end{Phonetics}
\end{Entry}

\begin{Entry}{午餐}{4,16}{⼗、⾷}
  \begin{Phonetics}{午餐}{wu3 can1}[][HSK 2]
    \definition[份,顿,次]{s.}{almoço}
  \seealsoref{午饭}{wu3 fan4}
  \end{Phonetics}
\end{Entry}

\begin{Entry}{半}{5}{⼗}
  \begin{Phonetics}{半}{ban4}[][HSK 1]
    \definition{adv.}{parcialmente; usado antes de verbos ou adjetivos para indicar incompletude}
    \definition{num.}{(depois de um número) ``e meio'' | meio; metade | na metade; no meio | muito pouco; o mínimo}
  \end{Phonetics}
\end{Entry}

\begin{Entry}{半天}{5,4}{⼗、⼤}
  \begin{Phonetics}{半天}{ban4 tian1}[][HSK 1]
    \definition{s.}{metade do dia; metade do dia dividida pelo meio-dia | um longo tempo; bastante tempo; refere-se a um período de tempo relativamente longo (com um tom exagerado)}
  \end{Phonetics}
\end{Entry}

\begin{Entry}{半决赛}{5,6,14}{⼗、⼎、⾙}
  \begin{Phonetics}{半决赛}{ban4 jue2 sai4}[][HSK 6]
    \definition{s.}{semifinais}
  \end{Phonetics}
\end{Entry}

\begin{Entry}{半年}{5,6}{⼗、⼲}
  \begin{Phonetics}{半年}{ban4 nian2}[][HSK 1]
    \definition{s.}{meio ano}
  \end{Phonetics}
\end{Entry}

\begin{Entry}{半夜}{5,8}{⼗、⼣}
  \begin{Phonetics}{半夜}{ban4 ye4}[][HSK 2]
    \definition{s.}{no meio da noite; metade da noite | por volta da meia-noite, também se refere à madrugada}
  \end{Phonetics}
\end{Entry}

\begin{Entry}{半音}{5,9}{⼗、⾳}
  \begin{Phonetics}{半音}{ban4yin1}
    \definition{s.}{semitom; na música, uma oitava é dividida em doze notas e o intervalo entre duas notas adjacentes é chamado de semitom}
  \end{Phonetics}
\end{Entry}

\begin{Entry}{半球}{5,11}{⼗、⽟}
  \begin{Phonetics}{半球}{ban4qiu2}
    \definition{s.}{hemisfério}
  \end{Phonetics}
\end{Entry}

\begin{Entry}{华}{6}{⼗}
  \begin{Phonetics}{华}{hua2}
    \definition*{s.}{China; refere-se à China (anteriormente conhecida como Huaxia, 华夏, mais tarde chamada de Zhonghua, 中华, ou simplesmente Hua, 华)}
    \definition{adj.}{esplêndido; magnífico | próspero; florescente | chamativo; extravagante; vaidoso | grisalho}
    \definition{s.}{corona; um halo colorido ao redor do sol ou da lua causado pela difração da luz através das nuvens | creme; melhor parte; a melhor parte das coisas | chinês; refere-se à nacionalidade Han (língua e escrita) | vezes; anos; refere-se a (bons) momentos | elixir; essência líquida; substâncias formadas pela sedimentação de minerais na água de nascente | Seu, palavra honorífica, usada para se referir a coisas relacionadas à outra pessoa}
  \seealsoref{华夏}{hua2xia4}
  \seealsoref{中华}{zhong1 hua2}
  \end{Phonetics}
  \begin{Phonetics}{华}{hua4}
    \definition*{s.}{Huashan Mountain (na província de Shaanxi) | Sobrenome Hua}
  \end{Phonetics}
\end{Entry}

\begin{Entry}{华人}{6,2}{⼗、⼈}
  \begin{Phonetics}{华人}{hua2 ren2}[][HSK 3]
    \definition[名,位,个]{s.}{Chinês; chinês étnico | chineses no exterior; refere-se a cidadãos estrangeiros de ascendência chinesa que obtiveram a nacionalidade do país em que residem}
  \end{Phonetics}
\end{Entry}

\begin{Entry}{华氏}{6,4}{⼗、⽒}
  \begin{Phonetics}{华氏}{hua2shi4}
    \definition{s.}{graus Fahrenheit (°F)}
  \end{Phonetics}
\end{Entry}

\begin{Entry}{华语}{6,9}{⼗、⾔}
  \begin{Phonetics}{华语}{hua2 yu3}[][HSK 5]
    \definition*{s.}{Chinês (idioma)}
  \end{Phonetics}
\end{Entry}

\begin{Entry}{华夏}{6,10}{⼗、⼢}
  \begin{Phonetics}{华夏}{hua2xia4}
    \definition*{s.}{Huaxia, nome antigo da China | Catai}
  \end{Phonetics}
\end{Entry}

\begin{Entry}{华盛顿}{6,11,10}{⼗、⽫、⾴}
  \begin{Phonetics}{华盛顿}{hua2sheng4dun4}
    \definition*{s.}{Washington}
  \end{Phonetics}
\end{Entry}

\begin{Entry}{华裔}{6,13}{⼗、⾐}
  \begin{Phonetics}{华裔}{hua2yi4}
    \definition{s.}{descendente de chinês}
  \end{Phonetics}
\end{Entry}

\begin{Entry}{协}{6}{⼗}
  \begin{Phonetics}{协}{xie2}
    \definition*{s.}{Sobrenome Xie}
    \definition{adv.}{conjuntamente; coordenadamente; juntos}
    \definition{s.}{harmonioso}
    \definition{v.}{auxiliar; assistir; ajudar}
  \end{Phonetics}
\end{Entry}

\begin{Entry}{协议}{6,5}{⼗、⾔}
  \begin{Phonetics}{协议}{xie2yi4}[][HSK 5]
    \definition[份,项]{s.}{acordo; tratado; decisão conjunta alcançada através de negociação e consulta}
    \definition{v.}{concordar em}
  \end{Phonetics}
\end{Entry}

\begin{Entry}{协议书}{6,5,4}{⼗、⾔、⼄}
  \begin{Phonetics}{协议书}{xie2 yi4 shu1}[][HSK 5]
    \definition{s.}{contrato | protocolo}
  \end{Phonetics}
\end{Entry}

\begin{Entry}{协会}{6,6}{⼗、⼈}
  \begin{Phonetics}{协会}{xie2hui4}[][HSK 6]
    \definition[个]{s.}{sociedade; instituto; associação; uma organização de massa formada para promover uma causa comum}
  \end{Phonetics}
\end{Entry}

\begin{Entry}{协助}{6,7}{⼗、⼒}
  \begin{Phonetics}{协助}{xie2zhu4}[][HSK 6]
    \definition{v.}{ajudar; auxiliar; dar assistência; fornecer ajuda}
  \end{Phonetics}
\end{Entry}

\begin{Entry}{协调}{6,10}{⼗、⾔}
  \begin{Phonetics}{协调}{xie2tiao2}[][HSK 6]
    \definition{adj.}{coordenado; harmonioso; em sintonia}
    \definition{v.}{coordenar; concertar; integrar; harmonizar; fazer a harmonia apropriada}
  \end{Phonetics}
\end{Entry}

\begin{Entry}{协商}{6,11}{⼗、⼝}
  \begin{Phonetics}{协商}{xie2shang1}[][HSK 6]
    \definition{v.}{discutir; consultar; negociar; várias partes discutiram e decidiram em conjunto para chegar à mesma visão}
  \end{Phonetics}
\end{Entry}

\begin{Entry}{克}{7}{⼗}
  \begin{Phonetics}{克}{ke4}[][HSK 2]
    \definition*{s.}{Sobrenome Ke}
    \definition{clas.}{g, grama, unidade de peso | unidade tibetana de volume ou medida seca (com capacidade para cerca de 25 斤, de cevada) | unidade tibetana de área de terra equivalente a cerca de 1 亩}
    \definition{v.}{poder; ser capaz de | tolerar; conter; restringir; suprimir| subjugar; capturar; conquistar (uma cidade, etc.) | digerir (alimentos) | reduzir; diminuir | definir um limite de tempo}
  \seealsoref{斤}{jin1}
  \seealsoref{亩}{mu3}
  \end{Phonetics}
\end{Entry}

\begin{Entry}{克服}{7,8}{⼗、⽉}
  \begin{Phonetics}{克服}{ke4fu2}[][HSK 3]
    \definition{v.}{sobrepujar; superar; conquistar; vencer com força de vontade e determinação (deficiências, erros, fenômenos negativos, condições desfavoráveis, etc.) | aguentar; suportar (dificuldades, inconveniências, etc.)}
  \end{Phonetics}
\end{Entry}

\begin{Entry}{丧}{8}{⼗}
  \begin{Phonetics}{丧}{sang1}
    \definition{adj.}{decepcionado; deprimido; desanimado}
    \definition{v.}{perder | desanimar; frustrar}
  \end{Phonetics}
  \begin{Phonetics}{丧}{sang4}
    \definition{adj.}{decepcionado | desanimado}
    \definition{v.}{estar enlutado (do cônjuge etc.) | morrer}
  \end{Phonetics}
\end{Entry}

\begin{Entry}{丧失}{8,5}{⼗、⼤}
  \begin{Phonetics}{丧失}{sang4shi1}[][HSK 6]
    \definition{v.}{perder (algo que se tem)}
  \end{Phonetics}
\end{Entry}

\begin{Entry}{丧钟}{8,9}{⼗、⾦}
  \begin{Phonetics}{丧钟}{sang1zhong1}
    \definition{s.}{sentença de morte}
  \end{Phonetics}
\end{Entry}

\begin{Entry}{单}{8}{⼗}
  \begin{Phonetics}{单}{chan2}
    \definition{s.}{usado em 单于 \dpy{chan2yu2}}
  \seealsoref{单于}{chan2yu2}
  \end{Phonetics}
  \begin{Phonetics}{单}{dan1}[][HSK 4]
    \definition*{s.}{Sobrenome Dan}
    \definition{adj.}{sozinho; único | ímpar; número ímpar (oposto a 双) | simples; poucos projetos e tipos; estrutura e ideias simples | fino; fraco; frágil}
    \definition{adv.}{isoladamente; sozinho; indica que uma ação ou coisa está dentro de um escopo limitado e não é combinada com outras; equivale a 只 ou 仅}
    \definition[个]{s.}{lençol; um único pedaço grande de pano usado para cobrir | conta; lista; pedaços de papel para anotações detalhadas (geralmente folhas soltas)}
  \seealsoref{仅}{jin3}
  \seealsoref{双}{shuang1}
  \seealsoref{只}{zhi3}
  \end{Phonetics}
  \begin{Phonetics}{单}{shan4}
    \definition*{s.}{Sobrenome Shan}
    \definition{s.}{material de tecido de largura simples (dupla) | número singular (plural)}
  \end{Phonetics}
\end{Entry}

\begin{Entry}{单一}{8,1}{⼗、⼀}
  \begin{Phonetics}{单一}{dan1 yi1}[][HSK 5]
    \definition{adj.}{único; unitário; exclusivo}
  \end{Phonetics}
\end{Entry}

\begin{Entry}{单于}{8,3}{⼗、⼆}
  \begin{Phonetics}{单于}{chan2yu2}
    \definition{s.}{rei de Xiongnu (匈奴)}
  \seealsoref{匈奴}{xiong1nu2}
  \end{Phonetics}
\end{Entry}

\begin{Entry}{单元}{8,4}{⼗、⼉}
  \begin{Phonetics}{单元}{dan1yuan2}[][HSK 3]
    \definition[个,组,套]{s.}{unidade (de algo); um conjunto completo, com parágrafos e sistemas próprios, que forma uma unidade independente}
  \end{Phonetics}
\end{Entry}

\begin{Entry}{单打}{8,5}{⼗、⼿}
  \begin{Phonetics}{单打}{dan1 da3}[][HSK 6]
    \definition[场,局,次]{s.}{Esporte: simples; competição um contra um}
  \end{Phonetics}
\end{Entry}

\begin{Entry}{单位}{8,7}{⼗、⼈}
  \begin{Phonetics}{单位}{dan1wei4}[][HSK 2]
    \definition[个,家]{s.}{unidade (como padrão de medida) | unidade (como uma organização, departamento, divisão, seção, etc.) | unidade (grupo de pessoas como um todo) | unidade de trabalho (local de trabalho, especialmente na República Popular da China antes da reforma econômica)}
  \end{Phonetics}
\end{Entry}

\begin{Entry}{单纯}{8,7}{⼗、⽷}
  \begin{Phonetics}{单纯}{dan1chun2}[][HSK 4]
    \definition{adj.}{puro; simples; descomplicado}
    \definition{adv.}{sozinho; puramente; meramente}
  \end{Phonetics}
\end{Entry}

\begin{Entry}{单质}{8,8}{⼗、⾙}
  \begin{Phonetics}{单质}{dan1zhi4}
    \definition{s.}{substância simples (consistindo puramente de um elemento, como diamante, grafite, etc.)}
  \end{Phonetics}
\end{Entry}

\begin{Entry}{单独}{8,9}{⼗、⽝}
  \begin{Phonetics}{单独}{dan1du2}[][HSK 4]
    \definition{adv.}{solo; sozinho; por si mesmo; por conta própria}
  \end{Phonetics}
\end{Entry}

\begin{Entry}{单调}{8,10}{⼗、⾔}
  \begin{Phonetics}{单调}{dan1diao4}[][HSK 4]
    \definition{adj.}{maçante; monótono}
  \end{Phonetics}
\end{Entry}

\begin{Entry}{单脚滑行车}{8,11,12,6,4}{⼗、⾁、⽔、⾏、⾞}
  \begin{Phonetics}{单脚滑行车}{dan1jiao3hua2xing2che1}
    \definition{s.}{\emph{scooter}}
  \end{Phonetics}
\end{Entry}

\begin{Entry}{卖}{8}{⼗}
  \begin{Phonetics}{卖}{mai4}[][HSK 2]
    \definition*{s.}{Sobrenome Mai}
    \definition{clas.}{um prato (nos tempos antigos); antigamente, os restaurantes chamavam cada prato vendido de 一卖 (uma porção)}
    \definition{v.}{vender (oposto de 买) | trair (o próprio país ou amigos); alcançar objetivos pessoais à custa dos interesses do país, da nação e dos outros | não poupar esforços; esforçar-se ao máximo; tentar fazer o máximo possível | mostrar-se intencionalmente; exibir-se | vender o próprio trabalho; trabalhar em troca de dinheiro}
  \seealsoref{买}{mai3}
  \end{Phonetics}
\end{Entry}

\begin{Entry}{南}{9}{⼗}
  \begin{Phonetics}{南}{nan2}[][HSK 1]
    \definition*{s.}{Sobrenome Nan}
    \definition{s.}{sul; uma das quatro direções básicas, o lado direito quando se está de frente para o sol pela manhã (oposto ao 北) | especificamente no sul da China}
  \seealsoref{北}{bei3}
  \end{Phonetics}
\end{Entry}

\begin{Entry}{南方}{9,4}{⼗、⽅}
  \begin{Phonetics}{南方}{nan2 fang1}[][HSK 2]
    \definition{s.}{sul; indica a direção sul | o sul; a região sul}
  \end{Phonetics}
\end{Entry}

\begin{Entry}{南北}{9,5}{⼗、⼔}
  \begin{Phonetics}{南北}{nan2 bei3}[][HSK 5]
    \definition{s.}{(território) norte e sul | (distância) de norte a sul}
  \end{Phonetics}
\end{Entry}

\begin{Entry}{南边}{9,5}{⼗、⾡}
  \begin{Phonetics}{南边}{nan2 bian5}[][HSK 1]
    \definition{s.}{sul; lado sul}
  \end{Phonetics}
\end{Entry}

\begin{Entry}{南极}{9,7}{⼗、⽊}
  \begin{Phonetics}{南极}{nan2ji2}[][HSK 5]
    \definition*{s.}{Polo Sul; Polo Antártico | Polo sul magnético}
    \definition{s.}{polo sul magnético}
  \end{Phonetics}
\end{Entry}

\begin{Entry}{南京}{9,8}{⼗、⼇}
  \begin{Phonetics}{南京}{nan2jing1}
    \definition*{s.}{Nanquim, capital da província de Jiangsu, 江苏}
  \seealsoref{江苏}{jiang1su1}
  \end{Phonetics}
\end{Entry}

\begin{Entry}{南面}{9,9}{⼗、⾯}
  \begin{Phonetics}{南面}{nan2mian4}
    \definition{s.}{sul | lado sul}
  \end{Phonetics}
\end{Entry}

\begin{Entry}{南部}{9,10}{⼗、⾢}
  \begin{Phonetics}{南部}{nan2 bu4}[][HSK 3]
    \definition{s.}{parte sul; sul | a parte sul}
  \end{Phonetics}
\end{Entry}

\begin{Entry}{真}{10}{⼗}
  \begin{Phonetics}{真}{zhen1}[][HSK 1]
    \definition*{s.}{Sobrenome Zhen}
    \definition{adj.}{verdadeiro; real; genuíno (oposto de 假, 伪) | claro; inequívoco | genuíno; conforme os fatos objetivos (em oposição a 假 e 伪) | sincero}
    \definition{adv.}{realmente; verdadeiramente; de fato}
    \definition{s.}{escrita regular | retrato; imagem; cópia exata de algo | instintos naturais (ou caráter, disposição); natureza; qualidade inerente; origem | estado original; refere-se à forma original das coisas}
  \seealsoref{假}{jia4}
  \seealsoref{伪}{wei3}
  \end{Phonetics}
\end{Entry}

\begin{Entry}{真切}{10,4}{⼗、⼑}
  \begin{Phonetics}{真切}{zhen1qie4}
    \definition{adj.}{claro | distinto | honesto | sincero | vívido}
  \end{Phonetics}
\end{Entry}

\begin{Entry}{真心}{10,4}{⼗、⼼}
  \begin{Phonetics}{真心}{zhen1xin1}
    \definition{adj.}{sincero}
    \definition[片]{s.}{sinceridade}
  \end{Phonetics}
\end{Entry}

\begin{Entry}{真牛}{10,4}{⼗、⽜}
  \begin{Phonetics}{真牛}{zhen1niu2}
    \definition{adj.}{(gíria) muito legal, incrível}
  \end{Phonetics}
\end{Entry}

\begin{Entry}{真正}{10,5}{⼗、⽌}
  \begin{Phonetics}{真正}{zhen1zheng4}[][HSK 2]
    \definition{adj.}{verdadeiro; real; genuíno}
    \definition{adv.}{realmente; de fato; expressa afirmação de uma ação ou situação, equivalente a 确实}
  \seealsoref{确实}{que4shi2}
  \end{Phonetics}
\end{Entry}

\begin{Entry}{真声}{10,7}{⼗、⼠}
  \begin{Phonetics}{真声}{zhen1sheng1}
    \definition{s.}{voz modal; voz natural; voz verdadeira (oposto a 假声)}
  \seealsoref{假声}{jia3sheng1}
  \end{Phonetics}
\end{Entry}

\begin{Entry}{真实}{10,8}{⼗、⼧}
  \begin{Phonetics}{真实}{zhen1shi2}[][HSK 3]
    \definition{adj.}{verdadeiro; real; autêntico; de acordo com fatos objetivos}
  \end{Phonetics}
\end{Entry}

\begin{Entry}{真的}{10,8}{⼗、⽩}
  \begin{Phonetics}{真的}{zhen1 de5}[][HSK 1]
    \definition{adv.}{realmente; salientar que a situação existe realmente | verdadeiramente; realmente; existente na realidade; consistente com os fatos objetivos}
  \end{Phonetics}
\end{Entry}

\begin{Entry}{真诚}{10,8}{⼗、⾔}
  \begin{Phonetics}{真诚}{zhen1 cheng2}[][HSK 5]
    \definition{adj.}{verdadeiro; honesto; sério; sincero; genuíno; descreve uma pessoa que fala e age com sinceridade, de coração, fazendo com que os outros acreditem nela}
  \end{Phonetics}
\end{Entry}

\begin{Entry}{真相}{10,9}{⼗、⽬}
  \begin{Phonetics}{真相}{zhen1xiang4}[][HSK 5]
    \definition[个]{s.}{face; verdade; verdade nua e crua; a situação real; o estado real das coisas; a verdadeira situação}
  \end{Phonetics}
\end{Entry}

\begin{Entry}{真珠}{10,10}{⼗、⽟}
  \begin{Phonetics}{真珠}{zhen1zhu1}
    \variantof{珍珠}
  \end{Phonetics}
\end{Entry}

\begin{Entry}{真真}{10,10}{⼗、⼗}
  \begin{Phonetics}{真真}{zhen1zhen1}
    \definition{adv.}{genuinamente | realmente | escrupulosamente}
  \end{Phonetics}
\end{Entry}

\begin{Entry}{真理}{10,11}{⼗、⽟}
  \begin{Phonetics}{真理}{zhen1li3}[][HSK 5]
    \definition[条,个]{s.}{verdade; o reflexo correto das coisas objetivas e suas leis no cérebro humano}
  \end{Phonetics}
\end{Entry}

\begin{Entry}{真释}{10,12}{⼗、⾤}
  \begin{Phonetics}{真释}{zhen1shi4}
    \definition{s.}{razão genuína | explicação verdadeira}
  \end{Phonetics}
\end{Entry}

\begin{Entry}{博}{12}{⼗}
  \begin{Phonetics}{博}{bo2}
    \definition*{s.}{Sobrenome Bo}
    \definition{adj.}{rico; abundante | erudito; bem informado | solto; grande | grande}
    \definition{s.}{doutor em filosofia; doutorado}
    \definition{v.}{ter um amplo conhecimento de; ser bem lido | ganhar; vencer | jogar}
  \end{Phonetics}
\end{Entry}

\begin{Entry}{博士}{12,3}{⼗、⼠}
  \begin{Phonetics}{博士}{bo2shi4}[][HSK 5]
    \definition[位,名,个,些]{s.}{doutorado; grau de doutor; nível mais alto de um diploma; também, uma pessoa que obteve esse diploma | doutor; antigo título honorífico para uma pessoa que é habilidosa em um determinado ofício ou especializada em uma determinada ocupação | doutor; autoridades que ensinavam as escrituras na China nos tempos antigos}
  \end{Phonetics}
\end{Entry}

\begin{Entry}{博文}{12,4}{⼗、⽂}
  \begin{Phonetics}{博文}{bo2wen2}
    \definition{s.}{artigo em um blog}
    \definition{v.}{escrever um artigo em um blog}
  \end{Phonetics}
\end{Entry}

\begin{Entry}{博主}{12,5}{⼗、⼂}
  \begin{Phonetics}{博主}{bo2zhu3}
    \definition{s.}{blogueiro}
  \end{Phonetics}
\end{Entry}

\begin{Entry}{博物馆}{12,8,11}{⼗、⽜、⾷}
  \begin{Phonetics}{博物馆}{bo2wu4guan3}[][HSK 5]
    \definition[座,个]{s.}{museu; locais para coleta, armazenamento, pesquisa, exibição e exposição de relíquias culturais ou espécimes relacionados à história, cultura, arte, ciências naturais, ciência e tecnologia, etc.}
  \end{Phonetics}
\end{Entry}

\begin{Entry}{博客}{12,9}{⼗、⼧}
  \begin{Phonetics}{博客}{bo2 ke4}[][HSK 5]
    \definition[个]{s.}{\emph{blog}; página da Web ou site gerenciado por um indivíduo, geralmente composto por postagens organizadas da mais recente para a mais antiga | blogueiro; \emph{blogger}; pessoas que possuem ou escrevem \emph{blogs}}
  \end{Phonetics}
\end{Entry}

\begin{Entry}{博览会}{12,9,6}{⼗、⾒、⼈}
  \begin{Phonetics}{博览会}{bo2lan3hui4}[][HSK 5]
    \definition[次,届]{s.}{exposição; feira internacional; exposições de produtos em grande escala}
  \end{Phonetics}
\end{Entry}

%%%%% EOF %%%%%

