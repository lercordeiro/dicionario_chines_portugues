%%%
%%% Radical "⼌"
%%%

\section*{Radical 13: ``⼌''}\addcontentsline{toc}{section}{Radical 13: ⼌}

\begin{entry}{内}{4}{⼌}
  \begin{phonetics}{内}{nei4}[][HSK 3]
    \definition*{s.}{sobrenome Nei}
    \definition{adj.}{interno; interior}
    \definition{prep.}{dentro}
    \definition{s.}{interior; lado de dentro; parte de dentro | a esposa ou parentes dela}
  \end{phonetics}
\end{entry}

\begin{entry}{内心}{4,4}{⼌、⼼}
  \begin{phonetics}{内心}{nei4 xin1}[][HSK 3]
    \definition{s.}{coração; interior; íntimo do ser}
  \end{phonetics}
\end{entry}

\begin{entry}{内存}{4,6}{⼌、⼦}
  \begin{phonetics}{内存}{nei4cun2}
    \definition{s.}{armazenamento interno | memória do computador | RAM (\emph{random access memory})}
  \seealsoref{随机存取存储器}{sui2ji1cun2qu3cun2chu3qi4}
  \seealsoref{随机存取记忆体}{sui2ji1cun2qu3ji4yi4ti3}
  \end{phonetics}
\end{entry}

\begin{entry}{内省}{4,9}{⼌、⽬}
  \begin{phonetics}{内省}{nei4xing3}
    \definition{s.}{introspecção}
    \definition{v.}{refletir sobre si mesmo}
  \end{phonetics}
\end{entry}

\begin{entry}{内科}{4,9}{⼌、⽲}
  \begin{phonetics}{内科}{nei4ke1}[][HSK 4]
    \definition{s.}{medicina geral; clínica geral; clínica médica}
  \end{phonetics}
\end{entry}

\begin{entry}{内容}{4,10}{⼌、⼧}
  \begin{phonetics}{内容}{nei4rong2}[][HSK 3]
    \definition[个]{s.}{conteúdo; substância}
  \end{phonetics}
\end{entry}

\begin{entry}{内部}{4,10}{⼌、⾢}
  \begin{phonetics}{内部}{nei4bu4}[][HSK 4]
    \definition{s.}{interior; dentro; interno; dentro de um determinado intervalo}
  \end{phonetics}
\end{entry}

\begin{entry}{内燃机}{4,16,6}{⼌、⽕、⽊}
  \begin{phonetics}{内燃机}{nei4ran2ji1}
    \definition{s.}{motor de combustão interna}
  \end{phonetics}
\end{entry}

\begin{entry}{再}{6}{⼌}
  \begin{phonetics}{再}{zai4}[][HSK 1]
    \definition{adv.}{de novo | outra vez | uma segunda vez | não importa como\dots (seguido por um adjetivo ou verbo, e então (normalmente) 也 ou 都 para dar ênfase)}
  \end{phonetics}
\end{entry}

\begin{entry}{再三}{6,3}{⼌、⼀}
  \begin{phonetics}{再三}{zai4san1}[][HSK 4]
    \definition{adv.}{repetidamente; repetidas vezes; de novo e de novo}
  \end{phonetics}
\end{entry}

\begin{entry}{再不}{6,4}{⼌、⼀}
  \begin{phonetics}{再不}{zai4bu4}
    \definition{adv.}{nunca mais}
  \end{phonetics}
\end{entry}

\begin{entry}{再见}{6,4}{⼌、⾒}
  \begin{phonetics}{再见}{zai4jian4}[][HSK 1]
    \definition{v.}{adeus | até à vista | até à próxima | até logo}
  \end{phonetics}
\end{entry}

\begin{entry}{再发}{6,5}{⼌、⼜}
  \begin{phonetics}{再发}{zai4fa1}
    \definition{v.}{reenviar}
  \end{phonetics}
\end{entry}

\begin{entry}{再生}{6,5}{⼌、⽣}
  \begin{phonetics}{再生}{zai4sheng1}
    \definition{s.}{reciclagem | regeneração}
    \definition{v.}{reciclar | renascer | regenerar}
  \end{phonetics}
\end{entry}

\begin{entry}{再审}{6,8}{⼌、⼧}
  \begin{phonetics}{再审}{zai4shen3}
    \definition{s.}{novo julgamento | revisão}
    \definition{v.}{ouvir um caso novamente}
  \end{phonetics}
\end{entry}

\begin{entry}{再者}{6,8}{⼌、⽼}
  \begin{phonetics}{再者}{zai4zhe3}
    \definition{conj.}{além do mais | além disso}
  \end{phonetics}
\end{entry}

\begin{entry}{再育}{6,8}{⼌、⾁}
  \begin{phonetics}{再育}{zai4yu4}
    \definition{v.}{aumentar | multiplicar | proliferar}
  \end{phonetics}
\end{entry}

\begin{entry}{再临}{6,9}{⼌、⼁}
  \begin{phonetics}{再临}{zai4lin2}
    \definition{v.}{vir de novo}
  \end{phonetics}
\end{entry}

\begin{entry}{再度}{6,9}{⼌、⼴}
  \begin{phonetics}{再度}{zai4du4}
    \definition{adv.}{outra vez | mais uma vez}
  \end{phonetics}
\end{entry}

\begin{entry}{再说}{6,9}{⼌、⾔}
  \begin{phonetics}{再说}{zai4shuo1}
    \definition{conj.}{além do mais | além disso | o que mais}
    \definition{v.}{adiar uma discussão para mais tarde | dizer novamente}
  \end{phonetics}
\end{entry}

\begin{entry}{再读}{6,10}{⼌、⾔}
  \begin{phonetics}{再读}{zai4du2}
    \definition{v.}{ler novamente | rever (uma lição, etc.)}
  \end{phonetics}
\end{entry}

\begin{entry}{罔}{8}{⼌}
  \begin{phonetics}{罔}{wang3}
    \definition{v.}{enganar}
  \end{phonetics}
\end{entry}

%%%%% EOF %%%%%

