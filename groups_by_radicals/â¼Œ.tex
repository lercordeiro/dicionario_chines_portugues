%%%
%%% Radical "⼌"
%%%

\section*{Radical 13: ``⼌''}\addcontentsline{toc}{section}{Radical 13: ⼌}

\begin{Entry}{内}{4}{⼌}
  \begin{Phonetics}{内}{nei4}[][HSK 3]
    \definition*{s.}{Sobrenome Nei}
    \definition{s.}{dentro; interior; parte interna ou lateral (oposto de 外) |  (antes de um substantivo ou verbo na formação de uma palavra composta) interno | (depois de um substantivo para indicar lugar, tempo, escopo ou limites) dentro; em | coração; mente | esposa ou parentes da esposa}
  \seealsoref{外}{wai4}
  \end{Phonetics}
\end{Entry}

\begin{Entry}{内心}{4,4}{⼌、⼼}
  \begin{Phonetics}{内心}{nei4 xin1}[][HSK 3]
    \definition{s.}{coração; interior; íntimo do ser}
  \end{Phonetics}
\end{Entry}

\begin{Entry}{内外}{4,5}{⼌、⼣}
  \begin{Phonetics}{内外}{nei4 wai4}[][HSK 6]
    \definition{s.}{dentro e fora; nacional e estrangeiro; interno e externo | ao redor; aproximadamente; número aproximado de exibição}
  \end{Phonetics}
\end{Entry}

\begin{Entry}{内在}{4,6}{⼌、⼟}
  \begin{Phonetics}{内在}{nei4zai4}[][HSK 5]
    \definition{adj.}{intrínseco; algo que existe em si mesmo, mas que não pode ser descoberto através da observação direta | interno; imanente; difícil de perceber}
  \end{Phonetics}
\end{Entry}

\begin{Entry}{内地}{4,6}{⼌、⼟}
  \begin{Phonetics}{内地}{nei4 di4}[][HSK 6]
    \definition{s.}{interior; sertão | China continental (RPC excluindo Hong Kong e Macau, mas incluindo ilhas como Hainan) | Japão (usado em Taiwan durante a colonização japonesa)}
  \end{Phonetics}
\end{Entry}

\begin{Entry}{内存}{4,6}{⼌、⼦}
  \begin{Phonetics}{内存}{nei4cun2}
    \definition{s.}{armazenamento interno | memória do computador | RAM (\emph{random access memory})}
  \seealsoref{随机存取存储器}{sui2ji1cun2qu3cun2chu3qi4}
  \seealsoref{随机存取记忆体}{sui2ji1cun2qu3ji4yi4ti3}
  \end{Phonetics}
\end{Entry}

\begin{Entry}{内衣}{4,6}{⼌、⾐}
  \begin{Phonetics}{内衣}{nei4 yi1}[][HSK 6]
    \definition[件,个]{s.}{roupa íntima}
  \end{Phonetics}
\end{Entry}

\begin{Entry}{内省}{4,9}{⼌、⽬}
  \begin{Phonetics}{内省}{nei4xing3}
    \definition{s.}{introspecção}
    \definition{v.}{refletir sobre si mesmo}
  \end{Phonetics}
\end{Entry}

\begin{Entry}{内科}{4,9}{⼌、⽲}
  \begin{Phonetics}{内科}{nei4ke1}[][HSK 4]
    \definition{s.}{medicina geral; clínica geral; clínica médica}
  \end{Phonetics}
\end{Entry}

\begin{Entry}{内容}{4,10}{⼌、⼧}
  \begin{Phonetics}{内容}{nei4rong2}[][HSK 3]
    \definition[份,个,项]{s.}{conteúdo; substância; a substância ou significado contido em algo}
  \end{Phonetics}
\end{Entry}

\begin{Entry}{内资}{4,10}{⼌、⾙}
  \begin{Phonetics}{内资}{nei4 zi1}
    \definition{s.}{capital nacional; financiamento interno; investimento de fontes nacionais (em oposição a 外资)}
  \seealsoref{外资}{wai4 zi1}
  \end{Phonetics}
\end{Entry}

\begin{Entry}{内部}{4,10}{⼌、⾢}
  \begin{Phonetics}{内部}{nei4bu4}[][HSK 4]
    \definition{s.}{interior; dentro; interno; dentro de um determinado intervalo}
  \end{Phonetics}
\end{Entry}

\begin{Entry}{内燃机}{4,16,6}{⼌、⽕、⽊}
  \begin{Phonetics}{内燃机}{nei4ran2ji1}
    \definition{s.}{motor de combustão interna}
  \end{Phonetics}
\end{Entry}

\begin{Entry}{册}{5}{⼌}
  \begin{Phonetics}{册}{ce4}[][HSK 5]
    \definition{clas.}{usado para cópias de livros}
    \definition{s.}{volume; livro | cópia; volume | ordem imperial para conferir um título}
    \definition{v.}{conferir um título}
  \end{Phonetics}
\end{Entry}

\begin{Entry}{再}{6}{⼌}
  \begin{Phonetics}{再}{zai4}[][HSK 1]
    \definition{adv.}{mais uma vez; além disso; ainda mais; indica a repetição ou continuação de uma mesma ação ou comportamento; refere-se principalmente a ações ou comportamentos não realizados ou contínuos | usado antes do adjetivo, indica intensificação, equivalente a 更 ou 更加 | (para uma ação adiada, precedida por uma expressão de tempo ou condição) então; somente então; depois de algo; indica que a ação ocorrerá após a conclusão de outra ação | além disso; indica um complemento, equivalente a 另外 ou 又 | próxima vez; indica que a ação ocorrerá após um determinado período de tempo | novamente; de novo}
  \seealsoref{更}{geng4}
  \seealsoref{更加}{geng4 jia1}
  \seealsoref{另外}{ling4wai4}
  \seealsoref{又}{you4}
  \end{Phonetics}
\end{Entry}

\begin{Entry}{再三}{6,3}{⼌、⼀}
  \begin{Phonetics}{再三}{zai4san1}[][HSK 4]
    \definition{adv.}{repetidamente; repetidas vezes; de novo e de novo}
  \end{Phonetics}
\end{Entry}

\begin{Entry}{再也}{6,3}{⼌、⼄}
  \begin{Phonetics}{再也}{zai4 ye3}[][HSK 5]
    \definition{adv.}{não mais; nunca mais; uma determinada situação ou ação nunca mais ocorrerá}
  \end{Phonetics}
\end{Entry}

\begin{Entry}{再不}{6,4}{⼌、⼀}
  \begin{Phonetics}{再不}{zai4bu4}
    \definition{adv.}{nunca mais}
  \end{Phonetics}
\end{Entry}

\begin{Entry}{再见}{6,4}{⼌、⾒}
  \begin{Phonetics}{再见}{zai4jian4}[][HSK 1]
    \definition{v.}{adeus; tchau; até logo; até mais; até mais tarde}
  \end{Phonetics}
\end{Entry}

\begin{Entry}{再发}{6,5}{⼌、⼜}
  \begin{Phonetics}{再发}{zai4 fa1}
    \definition{v.}{reenviar; reeditar}
  \end{Phonetics}
\end{Entry}

\begin{Entry}{再生}{6,5}{⼌、⽣}
  \begin{Phonetics}{再生}{zai4 sheng1}[][HSK 6]
    \definition{v.}{reviver; ressuscitar; ressuscitar dos mortos | reproduzir; regenerar | reprocessar; reciclar; regenerar; processar um determinado produto residual para restaurar seu desempenho original e transformá-lo em um novo produto}
  \end{Phonetics}
\end{Entry}

\begin{Entry}{再次}{6,6}{⼌、⽋}
  \begin{Phonetics}{再次}{zai4 ci4}[][HSK 5]
    \definition{adv.}{mais uma vez; uma segunda vez; outra vez}
  \end{Phonetics}
\end{Entry}

\begin{Entry}{再审}{6,8}{⼌、⼧}
  \begin{Phonetics}{再审}{zai4shen3}
    \definition{s.}{novo julgamento | revisão}
    \definition{v.}{ouvir um caso novamente}
  \end{Phonetics}
\end{Entry}

\begin{Entry}{再者}{6,8}{⼌、⽼}
  \begin{Phonetics}{再者}{zai4zhe3}
    \definition{conj.}{além do mais | além disso}
  \end{Phonetics}
\end{Entry}

\begin{Entry}{再育}{6,8}{⼌、⾁}
  \begin{Phonetics}{再育}{zai4yu4}
    \definition{v.}{aumentar | multiplicar | proliferar}
  \end{Phonetics}
\end{Entry}

\begin{Entry}{再临}{6,9}{⼌、⼁}
  \begin{Phonetics}{再临}{zai4lin2}
    \definition{v.}{vir de novo}
  \end{Phonetics}
\end{Entry}

\begin{Entry}{再度}{6,9}{⼌、⼴}
  \begin{Phonetics}{再度}{zai4du4}
    \definition{adv.}{outra vez | mais uma vez}
  \end{Phonetics}
\end{Entry}

\begin{Entry}{再说}{6,9}{⼌、⾔}
  \begin{Phonetics}{再说}{zai4 shuo1}[][HSK 6]
    \definition{conj.}{além disso; o que é mais; indicando uma razão adicional, equivalente a 况且}
    \definition{v.}{adiar para mais tarde; deixar para processamento ou consideração posterior}
  \seealsoref{况且}{kuang4qie3}
  \end{Phonetics}
\end{Entry}

\begin{Entry}{再读}{6,10}{⼌、⾔}
  \begin{Phonetics}{再读}{zai4du2}
    \definition{v.}{ler novamente | rever (uma lição, etc.)}
  \end{Phonetics}
\end{Entry}

\begin{Entry}{罔}{8}{⼌}
  \begin{Phonetics}{罔}{wang3}
    \definition{v.}{enganar}
  \end{Phonetics}
\end{Entry}

%%%%% EOF %%%%%

