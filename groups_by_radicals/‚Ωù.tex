%%%
%%% Radical "⽝"
%%%

\section*{Radical 94: ``⽝'' (犭)}\addcontentsline{toc}{section}{Radical 94: ⽝、犭}

\begin{Entry}{犬}{4}{⽝}[Kangxi 94]
  \begin{Phonetics}{犬}{quan3}
    \definition{s.}{cachorro}
  \end{Phonetics}
\end{Entry}

\begin{Entry}{犯}{5}{⽝}
  \begin{Phonetics}{犯}{fan4}[][HSK 6]
    \definition{s.}{criminoso}
    \definition{v.}{ofender; violar; ir contra | atacar; violar; trabalhar contra | fazer; ocorrer | voltar a; ter uma recorrência de; recair; retornar a (velhos hábitos)}
  \end{Phonetics}
\end{Entry}

\begin{Entry}{犯法}{5,8}{⽝、⽔}
  \begin{Phonetics}{犯法}{fan4fa3}
    \definition{v.}{violar (quebrar) a lei}
  \end{Phonetics}
\end{Entry}

\begin{Entry}{犯规}{5,8}{⽝、⾒}
  \begin{Phonetics}{犯规}{fan4 gui1}[][HSK 6]
    \definition{v.}{quebrar as regras; violar regras | Esporte: cometer uma falta contra}
  \end{Phonetics}
\end{Entry}

\begin{Entry}{犯罪}{5,13}{⽝、⽹}
  \begin{Phonetics}{犯罪}{fan4/zui4}[][HSK 6]
    \definition{v.+compl.}{cometer  um crime}
  \end{Phonetics}
\end{Entry}

\begin{Entry}{状}{7}{⽝}
  \begin{Phonetics}{状}{zhuang4}
    \definition{s.}{forma | estado; condição | conta; registro | reclamação escrita; queixa; reclamação legal | certificado | situação; circunstâncias | documento oficial; documentos de elogio, nomeação, etc.}
    \definition{v.}{descrever | narrar}
  \end{Phonetics}
\end{Entry}

\begin{Entry}{状况}{7,7}{⽝、⼎}
  \begin{Phonetics}{状况}{zhuang4kuang4}[][HSK 3]
    \definition[个,种]{s.}{estado; \emph{status}; situação; condição; estado de coisas; a aparência ou o estado em que as coisas se apresentam}
  \end{Phonetics}
\end{Entry}

\begin{Entry}{状态}{7,8}{⽝、⼼}
  \begin{Phonetics}{状态}{zhuang4tai4}[][HSK 3]
    \definition[种,个]{s.}{\emph{status}; estado; condição; situação; estado de coisas; a forma manifestada por pessoas ou coisas}
  \end{Phonetics}
\end{Entry}

\begin{Entry}{犹}{7}{⽝}
  \begin{Phonetics}{犹}{you2}
    \definition*{s.}{Sobrenome You}
    \definition{adv.}{ainda | assim como; exatamente como; como se}
    \definition{v.}{ser exatamente como; ser como}
  \end{Phonetics}
\end{Entry}

\begin{Entry}{犹豫}{7,15}{⽝、⾗}
  \begin{Phonetics}{犹豫}{you2yu4}[][HSK 5]
    \definition{adj.}{hesitante; indeciso, incapaz de decidir ou agir}
    \definition{v.}{hesitar; ser indeciso}
  \end{Phonetics}
\end{Entry}

\begin{Entry}{狂}{7}{⽝}
  \begin{Phonetics}{狂}{kuang2}[][HSK 5]
    \definition*{s.}{Sobrenome Kuang}
    \definition{adj.}{louco; maluco | violento; selvagem | selvagem; delirante; furioso; desenfreado; desinibido; sem restrições | arrogante; autoritário}
  \end{Phonetics}
\end{Entry}

\begin{Entry}{狂欢节}{7,6,5}{⽝、⽋、⾋}
  \begin{Phonetics}{狂欢节}{kuang2huan1 jie2}
    \definition*{s.}{Carnaval}
  \end{Phonetics}
\end{Entry}

\begin{Entry}{狒}{8}{⽝}
  \begin{Phonetics}{狒}{fei4}
    \definition{s.}{babuíno (uma espécie de macaco)}
  \end{Phonetics}
\end{Entry}

\begin{Entry}{狒狒}{8,8}{⽝、⽝}
  \begin{Phonetics}{狒狒}{fei4fei4}
    \definition{s.}{babuíno}
  \end{Phonetics}
\end{Entry}

\begin{Entry}{狗}{8}{⽝}
  \begin{Phonetics}{狗}{gou3}[][HSK 2]
    \definition[条,只,群]{s.}{cão; cachorro | palavrão usado para se referir a pessoas más ou seus capangas}
  \end{Phonetics}
\end{Entry}

\begin{Entry}{狠}{9}{⽝}
  \begin{Phonetics}{狠}{hen3}[][HSK 6]
    \definition{adj.}{impiedoso; implacável; feroz | firme; resoluto; severo; determinado}
    \definition{adv.}{muito; bastante; bastante | também, frequentemente usado antes de um adjetivo sem intensificar seu significado, ou seja, como um elemento sintático sem sentido}
    \definition{v.}{endurecer (o coração); suprimir (os próprios sentimentos)}
    \variantof{很}
  \seealsoref{很}{hen3}
  \end{Phonetics}
\end{Entry}

\begin{Entry}{独}{9}{⽝}
  \begin{Phonetics}{独}{du2}
    \definition*{s.}{Sobrenome Du}
    \definition{adj.}{só; solteiro | (coloquial) distante | único; só}
    \definition{adv.}{unicamente; somente | sozinho; por si mesmo; em solidão}
    \definition{s.}{idosos sem descendência; os sem filhos}
  \end{Phonetics}
\end{Entry}

\begin{Entry}{独立}{9,5}{⽝、⽴}
  \begin{Phonetics}{独立}{du2li4}[][HSK 4]
    \definition{adj.}{independente; por conta própria | separado; respectivo; descreve algo que é separado e não está em contato com outra coisa}
    \definition{v.}{ficar sozinho | alcançar a independência; tornar-se um país independente; liberdade de um Estado, regime ou organização contra interferência, controle e dominação por forças externas}
  \end{Phonetics}
\end{Entry}

\begin{Entry}{独自}{9,6}{⽝、⾃}
  \begin{Phonetics}{独自}{du2 zi4}[][HSK 4]
    \definition{adv.}{sozinho; por si mesmo; por conta própria}
  \end{Phonetics}
\end{Entry}

\begin{Entry}{独特}{9,10}{⽝、⽜}
  \begin{Phonetics}{独特}{du2te4}[][HSK 4]
    \definition{adj.}{único; distinto; original; especial}
  \end{Phonetics}
\end{Entry}

\begin{Entry}{狭}{9}{⽝}
  \begin{Phonetics}{狭}{xia2}
    \definition{adj.}{estreito (oposto a 广)}
  \seealsoref{广}{guang3}
  \end{Phonetics}
\end{Entry}

\begin{Entry}{猃}{10}{⽝}
  \begin{Phonetics}{猃}{xian3}
    \definition{s.}{(arcaico) um tipo de cão com focinho longo}
  \end{Phonetics}
\end{Entry}

\begin{Entry}{猃狁}{10,7}{⽝、⽝}
  \begin{Phonetics}{猃狁}{xian3yun3}
    \definition*{s.}{Termo da dinastia Zhou para uma tribo nômade do norte mais tarde chamou o Xiongnu (匈奴) nas dinastias Qin e Han}
  \seealsoref{匈奴}{xiong1nu2}
  \end{Phonetics}
\end{Entry}

\begin{Entry}{猎}{11}{⽝}
  \begin{Phonetics}{猎}{lie4}
    \definition[个]{s.}{traje de caça}
    \definition{v.}{caçar | procurar; perseguir}
  \end{Phonetics}
\end{Entry}

\begin{Entry}{猎物}{11,8}{⽝、⽜}
  \begin{Phonetics}{猎物}{lie4wu4}
    \definition{s.}{presa (vítima de um predador)}
  \end{Phonetics}
\end{Entry}

\begin{Entry}{猖}{11}{⽝}
  \begin{Phonetics}{猖}{chang1}
    \definition{adj.}{louco; indisciplinado; dissoluto; licencioso; precipitado; imprudente | Literário: feroz}
  \end{Phonetics}
\end{Entry}

\begin{Entry}{猖狂}{11,7}{⽝、⽝}
  \begin{Phonetics}{猖狂}{chang1kuang2}[][HSK 7-9]
    \definition{adj.}{selvagem; desenfreado; furioso; imprudente; arrogante e presunçoso}
  \end{Phonetics}
\end{Entry}

\begin{Entry}{猛}{11}{⽝}
  \begin{Phonetics}{猛}{meng3}[][HSK 6]
    \definition*{s.}{Sobrenome Meng}
    \definition{adj.}{feroz; violento | enérgico; vigoroso | valente}
    \definition{adv.}{de repente; abruptamente | vigorosamente; com força repentina | (coloquial) ao contentamento do coração; de todo o coração | ferozmente; violentamente}
  \end{Phonetics}
\end{Entry}

\begin{Entry}{猛然}{11,12}{⽝、⽕}
  \begin{Phonetics}{猛然}{meng3ran2}
    \definition{adv.}{de repente; abruptamente; indica ação repentina e rápida}
  \end{Phonetics}
\end{Entry}

\begin{Entry}{猜}{11}{⽝}
  \begin{Phonetics}{猜}{cai1}[][HSK 5]
    \definition{v.}{adivinhar; conjecturar; especular | suspeitar; ser cauteloso com os outros; desconfiar dos outros}
  \end{Phonetics}
\end{Entry}

\begin{Entry}{猜忌}{11,7}{⽝、⼼}
  \begin{Phonetics}{猜忌}{cai1 ji4}
    \definition{v.}{ser desconfiado e invejoso | ser desconfiado e ciumento de}
  \end{Phonetics}
\end{Entry}

\begin{Entry}{猜测}{11,9}{⽝、⽔}
  \begin{Phonetics}{猜测}{cai1 ce4}[][HSK 5]
    \definition[个,种]{s.}{advinhação; conjectura; suposição; especulação}
    \definition{v.}{adivinhar; conjecturar; especular; estimar a partir da imaginação}
  \end{Phonetics}
\end{Entry}

\begin{Entry}{猜谜}{11,11}{⽝、⾔}
  \begin{Phonetics}{猜谜}{cai1 mi2}[][HSK 7-9]
    \definition{v.}{adivinhar um enigma}
  \end{Phonetics}
\end{Entry}

\begin{Entry}{猜想}{11,13}{⽝、⼼}
  \begin{Phonetics}{猜想}{cai1xiang3}[][HSK 7-9]
    \definition{s.}{suposição; conjectura; palpite; especulação}
    \definition{v.}{supor; adivinhar; suspeitar}
  \end{Phonetics}
\end{Entry}

\begin{Entry}{猜疑}{11,14}{⽝、⽦}
  \begin{Phonetics}{猜疑}{cai1 yi2}
    \definition{v.}{abrigar suspeitas; ser desconfiado; ter receios; levantar suspeitas do nada}
  \end{Phonetics}
\end{Entry}

\begin{Entry}{猪}{11}{⽝}
  \begin{Phonetics}{猪}{zhu1}[][HSK 3]
    \definition[头,只,口]{s.}{porco; suíno}
  \end{Phonetics}
\end{Entry}

\begin{Entry}{猪头}{11,5}{⽝、⼤}
  \begin{Phonetics}{猪头}{zhu1tou2}
    \definition{s.}{tolo | idiota}
  \end{Phonetics}
\end{Entry}

\begin{Entry}{猪柳}{11,9}{⽝、⽊}
  \begin{Phonetics}{猪柳}{zhu1liu3}
    \definition{s.}{filé de porco}
  \end{Phonetics}
\end{Entry}

\begin{Entry}{猪笼}{11,11}{⽝、⽵}
  \begin{Phonetics}{猪笼}{zhu1long2}
    \definition{s.}{estrutura cilíndrica de bambu ou arame usada para restringir um porco durante o transporte}
  \end{Phonetics}
\end{Entry}

\begin{Entry}{猪窠}{11,13}{⽝、⽳}
  \begin{Phonetics}{猪窠}{zhu1ke1}
    \definition{s.}{chiqueiro}
  \end{Phonetics}
\end{Entry}

\begin{Entry}{猫}{11}{⽝}
  \begin{Phonetics}{猫}{mao1}[][HSK 2]
    \definition*[只,种,群,窝,个]{s.}{gato |  Empréstimo linguístico: MODEM}
    \definition{v.}{esconder-se; entrar em esconderijo | inclinar-se para a frente; curvar-se}
  \end{Phonetics}
  \begin{Phonetics}{猫}{mao2}
    \definition{v.}{utilizado em 猫腰 \dpy{mao2yao1}}
  \seealsoref{猫腰}{mao2yao1}
  \end{Phonetics}
\end{Entry}

\begin{Entry}{猫腰}{11,13}{⽝、⾁}
  \begin{Phonetics}{猫腰}{mao2yao1}
    \definition{v.}{curvar-se}
  \end{Phonetics}
\end{Entry}

\begin{Entry}{猫熊}{11,14}{⽝、⽕}
  \begin{Phonetics}{猫熊}{mao1xiong2}
    \definition[把,只]{s.}{panda gigante}
  \seealsoref{熊猫}{xiong2mao1}
  \end{Phonetics}
\end{Entry}

\begin{Entry}{猩}{12}{⽝}
  \begin{Phonetics}{猩}{xing1}
    \definition[只]{s.}{orangotango}
  \end{Phonetics}
\end{Entry}

\begin{Entry}{猩猩}{12,12}{⽝、⽝}
  \begin{Phonetics}{猩猩}{xing1xing5}
    \definition{s.}{orangotango}
  \end{Phonetics}
\end{Entry}

\begin{Entry}{猴}{12}{⽝}
  \begin{Phonetics}{猴}{hou2}[][HSK 5]
    \definition{adj.}{esperto; inteligente; perspicaz | travesso (menino)}
    \definition[只,群]{s.}{macaco}
  \end{Phonetics}
\end{Entry}

\begin{Entry}{猴子}{12,3}{⽝、⼦}
  \begin{Phonetics}{猴子}{hou2zi5}
    \definition[只]{s.}{macaco}
  \end{Phonetics}
\end{Entry}

\begin{Entry}{献}{13}{⽝}
  \begin{Phonetics}{献}{xian4}[][HSK 5]
    \definition{v.}{oferecer; apresentar; dedicar; doar | mostrar; apresentar; exibir | exibir-se; mostrar-se para que os outros vejam}
  \end{Phonetics}
\end{Entry}

\begin{Entry}{獞}{15}{⽝}
  \begin{Phonetics}{獞}{tong2}
    \definition{s.}{nome de uma variedade de cão | tribos selvagens no sul da China}
  \end{Phonetics}
  \begin{Phonetics}{獞}{zhuang4}
    \variantof{壮}
  \end{Phonetics}
\end{Entry}

%%%%% EOF %%%%%

