%%%
%%% Radical "⽝"
%%%

\section*{Radical 94: ``⽝'' (犭)}\addcontentsline{toc}{section}{Radical 94: ⽝、犭}

\begin{entry}{犬}{4}{⽝}[Kangxi 94]
  \begin{phonetics}{犬}{quan3}
    \definition{s.}{cachorro}
  \end{phonetics}
\end{entry}

\begin{entry}{犯}{5}{⽝}
  \begin{phonetics}{犯}{fan4}[][HSK 6]
    \definition{s.}{criminoso}
    \definition{v.}{ofender; violar; ir contra | atacar; violar; trabalhar contra | fazer; ocorrer | voltar a; ter uma recorrência de; recair; retornar a (velhos hábitos)}
  \end{phonetics}
\end{entry}

\begin{entry}{犯法}{5,8}{⽝、⽔}
  \begin{phonetics}{犯法}{fan4fa3}
    \definition{v.}{violar (quebrar) a lei}
  \end{phonetics}
\end{entry}

\begin{entry}{犯罪}{5,13}{⽝、⽹}
  \begin{phonetics}{犯罪}{fan4zui4}
    \definition{v.+compl.}{cometer  um crime (uma ofensa)}
  \end{phonetics}
\end{entry}

\begin{entry}{状}{7}{⽝}
  \begin{phonetics}{状}{zhuang4}
    \definition{s.}{forma | estado; condição | conta; registro | reclamação escrita; queixa; reclamação legal | certificado | situação; circunstâncias | documento oficial; documentos de elogio, nomeação, etc.}
    \definition{v.}{descrever | narrar}
  \end{phonetics}
\end{entry}

\begin{entry}{状况}{7,7}{⽝、⼎}
  \begin{phonetics}{状况}{zhuang4kuang4}[][HSK 3]
    \definition[个,种]{s.}{estado; \emph{status}; situação; condição; estado de coisas; a aparência ou o estado em que as coisas se apresentam}
  \end{phonetics}
\end{entry}

\begin{entry}{状态}{7,8}{⽝、⼼}
  \begin{phonetics}{状态}{zhuang4tai4}[][HSK 3]
    \definition[种,个]{s.}{\emph{status}; estado; condição; situação; estado de coisas; a forma manifestada por pessoas ou coisas}
  \end{phonetics}
\end{entry}

\begin{entry}{犹}{7}{⽝}
  \begin{phonetics}{犹}{you2}
    \definition*{s.}{Sobrenome You}
    \definition{adv.}{ainda | assim como; exatamente como; como se}
    \definition{v.}{ser exatamente como; ser como}
  \end{phonetics}
\end{entry}

\begin{entry}{犹豫}{7,15}{⽝、⾗}
  \begin{phonetics}{犹豫}{you2yu4}[][HSK 5]
    \definition{adj.}{hesitante; indeciso, incapaz de decidir ou agir}
    \definition{v.}{hesitar; ser indeciso}
  \end{phonetics}
\end{entry}

\begin{entry}{狂}{7}{⽝}
  \begin{phonetics}{狂}{kuang2}[][HSK 5]
    \definition*{s.}{Sobrenome Kuang}
    \definition{adj.}{louco; maluco | violento; selvagem | selvagem; delirante; furioso; desenfreado; desinibido; sem restrições | arrogante; autoritário}
  \end{phonetics}
\end{entry}

\begin{entry}{狂欢节}{7,6,5}{⽝、⽋、⾋}
  \begin{phonetics}{狂欢节}{kuang2huan1jie2}
    \definition*{s.}{Carnaval}
  \end{phonetics}
\end{entry}

\begin{entry}{狒}{8}{⽝}
  \begin{phonetics}{狒}{fei4}
    \definition{s.}{babuíno (uma espécie de macaco)}
  \end{phonetics}
\end{entry}

\begin{entry}{狒狒}{8,8}{⽝、⽝}
  \begin{phonetics}{狒狒}{fei4fei4}
    \definition{s.}{babuíno}
  \end{phonetics}
\end{entry}

\begin{entry}{狗}{8}{⽝}
  \begin{phonetics}{狗}{gou3}[][HSK 2]
    \definition[条,只,群]{s.}{cão; cachorro | palavrão usado para se referir a pessoas más ou seus capangas}
  \end{phonetics}
\end{entry}

\begin{entry}{狠}{9}{⽝}
  \begin{phonetics}{狠}{hen3}[][HSK 6]
    \definition{adj.}{impiedoso; implacável; feroz | firme; resoluto; severo; determinado}
    \definition{adv.}{muito; bastante; bastante | também, frequentemente usado antes de um adjetivo sem intensificar seu significado, ou seja, como um elemento sintático sem sentido}
    \definition{v.}{endurecer (o coração); suprimir (os próprios sentimentos)}
    \variantof{很}
  \seealsoref{很}{hen3}
  \end{phonetics}
\end{entry}

\begin{entry}{独}{9}{⽝}
  \begin{phonetics}{独}{du2}
    \definition*{s.}{Sobrenome Du}
    \definition{adj.}{só; solteiro | (coloquial) distante | único; só}
    \definition{adv.}{unicamente; somente | sozinho; por si mesmo; em solidão}
    \definition{s.}{idosos sem descendência; os sem filhos}
  \end{phonetics}
\end{entry}

\begin{entry}{独立}{9,5}{⽝、⽴}
  \begin{phonetics}{独立}{du2li4}[][HSK 4]
    \definition{adj.}{independente; por conta própria | separado; respectivo; descreve algo que é separado e não está em contato com outra coisa}
    \definition{prep.}{independente de; separado de; não mais anexado à unidade original, mas uma unidade separada}
    \definition{v.}{ficar sozinho | alcançar a independência; tornar-se um país independente; liberdade de um Estado, regime ou organização contra interferência, controle e dominação por forças externas}
  \end{phonetics}
\end{entry}

\begin{entry}{独自}{9,6}{⽝、⾃}
  \begin{phonetics}{独自}{du2 zi4}[][HSK 4]
    \definition{adj.}{sozinho; por si mesmo; por conta própria}
  \end{phonetics}
\end{entry}

\begin{entry}{独特}{9,10}{⽝、⽜}
  \begin{phonetics}{独特}{du2te4}[][HSK 4]
    \definition{adj.}{único; distinto; original; especial}
  \end{phonetics}
\end{entry}

\begin{entry}{狭}{9}{⽝}
  \begin{phonetics}{狭}{xia2}
    \definition{adj.}{estreito}
  \end{phonetics}
\end{entry}

\begin{entry}{猃}{10}{⽝}
  \begin{phonetics}{猃}{xian3}
    \definition{s.}{(arcaico) um tipo de cão com focinho longo}
  \end{phonetics}
\end{entry}

\begin{entry}{猃狁}{10,7}{⽝、⽝}
  \begin{phonetics}{猃狁}{xian3yun3}
    \definition*{s.}{Termo da dinastia Zhou para uma tribo nômade do norte mais tarde chamou o Xiongnu (匈奴) nas dinastias Qin e Han}
  \seealsoref{匈奴}{xiong1nu2}
  \end{phonetics}
\end{entry}

\begin{entry}{猎}{11}{⽝}
  \begin{phonetics}{猎}{lie4}
    \definition[个]{s.}{traje de caça}
    \definition{v.}{caçar | procurar; perseguir}
  \end{phonetics}
\end{entry}

\begin{entry}{猎物}{11,8}{⽝、⽜}
  \begin{phonetics}{猎物}{lie4wu4}
    \definition{s.}{presa (vítima de um predador)}
  \end{phonetics}
\end{entry}

\begin{entry}{猛}{11}{⽝}
  \begin{phonetics}{猛}{meng3}[][HSK 6]
    \definition*{s.}{Sobrenome Meng}
    \definition{adj.}{feroz; violento | enérgico; vigoroso | valente}
    \definition{adv.}{de repente; abruptamente | vigorosamente; com força repentina | (coloquial) ao contentamento do coração; de todo o coração | ferozmente; violentamente}
  \end{phonetics}
\end{entry}

\begin{entry}{猛然}{11,12}{⽝、⽕}
  \begin{phonetics}{猛然}{meng3ran2}
    \definition{adv.}{de repente; abruptamente; indica ação repentina e rápida}
  \end{phonetics}
\end{entry}

\begin{entry}{猜}{11}{⽝}
  \begin{phonetics}{猜}{cai1}[][HSK 5]
    \definition{v.}{adivinhar; conjecturar; especular | suspeitar; ser cauteloso com os outros; desconfiar dos outros}
  \end{phonetics}
\end{entry}

\begin{entry}{猜忌}{11,7}{⽝、⼼}
  \begin{phonetics}{猜忌}{cai1 ji4}
    \definition{v.}{ser desconfiado e invejoso | ser desconfiado e ciumento de}
  \end{phonetics}
\end{entry}

\begin{entry}{猜测}{11,9}{⽝、⽔}
  \begin{phonetics}{猜测}{cai1 ce4}[][HSK 5]
    \definition[个,种]{s.}{advinhação; conjectura; suposição; especulação}
    \definition{v.}{adivinhar; conjecturar; especular; estimar a partir da imaginação}
  \end{phonetics}
\end{entry}

\begin{entry}{猜想}{11,13}{⽝、⼼}
  \begin{phonetics}{猜想}{cai1 xiang3}
    \definition{s.}{suposição; conjectura; palpite; especulação}
    \definition{v.}{supor; adivinhar; suspeitar}
  \end{phonetics}
\end{entry}

\begin{entry}{猜疑}{11,14}{⽝、⽦}
  \begin{phonetics}{猜疑}{cai1 yi2}
    \definition{v.}{abrigar suspeitas; ser desconfiado; ter receios; levantar suspeitas do nada}
  \end{phonetics}
\end{entry}

\begin{entry}{猪}{11}{⽝}
  \begin{phonetics}{猪}{zhu1}[][HSK 3]
    \definition[头,只,口]{s.}{porco; suíno}
  \end{phonetics}
\end{entry}

\begin{entry}{猪头}{11,5}{⽝、⼤}
  \begin{phonetics}{猪头}{zhu1tou2}
    \definition{s.}{tolo | idiota}
  \end{phonetics}
\end{entry}

\begin{entry}{猪柳}{11,9}{⽝、⽊}
  \begin{phonetics}{猪柳}{zhu1liu3}
    \definition{s.}{filé de porco}
  \end{phonetics}
\end{entry}

\begin{entry}{猪笼}{11,11}{⽝、⽵}
  \begin{phonetics}{猪笼}{zhu1long2}
    \definition{s.}{estrutura cilíndrica de bambu ou arame usada para restringir um porco durante o transporte}
  \end{phonetics}
\end{entry}

\begin{entry}{猪窠}{11,13}{⽝、⽳}
  \begin{phonetics}{猪窠}{zhu1ke1}
    \definition{s.}{chiqueiro}
  \end{phonetics}
\end{entry}

\begin{entry}{猫}{11}{⽝}
  \begin{phonetics}{猫}{mao1}[][HSK 2]
    \definition[只,种,群,窝,个]{s.}{gato |  (empréstimo linguístico) MODEM}
    \definition{v.}{esconder-se; entrar em esconderijo | inclinar-se para a frente; curvar-se}
  \end{phonetics}
  \begin{phonetics}{猫}{mao2}
    \definition{v.}{utilizado em 猫腰 \dpy{mao2yao1}}
  \seealsoref{猫腰}{mao2yao1}
  \end{phonetics}
\end{entry}

\begin{entry}{猫腰}{11,13}{⽝、⾁}
  \begin{phonetics}{猫腰}{mao2yao1}
    \definition{v.}{curvar-se}
  \end{phonetics}
\end{entry}

\begin{entry}{猫熊}{11,14}{⽝、⽕}
  \begin{phonetics}{猫熊}{mao1xiong2}
    \definition[把,只]{s.}{panda gigante}
  \seealsoref{熊猫}{xiong2mao1}
  \end{phonetics}
\end{entry}

\begin{entry}{猩}{12}{⽝}
  \begin{phonetics}{猩}{xing1}
    \definition[只]{s.}{orangotango}
  \end{phonetics}
\end{entry}

\begin{entry}{猩猩}{12,12}{⽝、⽝}
  \begin{phonetics}{猩猩}{xing1xing5}
    \definition{s.}{orangotango}
  \end{phonetics}
\end{entry}

\begin{entry}{猴}{12}{⽝}
  \begin{phonetics}{猴}{hou2}[][HSK 5]
    \definition{adj.}{esperto; inteligente; perspicaz | travesso (menino)}
    \definition[只,群]{s.}{macaco}
  \end{phonetics}
\end{entry}

\begin{entry}{猴子}{12,3}{⽝、⼦}
  \begin{phonetics}{猴子}{hou2zi5}
    \definition[只]{s.}{macaco}
  \end{phonetics}
\end{entry}

\begin{entry}{献}{13}{⽝}
  \begin{phonetics}{献}{xian4}[][HSK 5]
    \definition{v.}{oferecer; apresentar; dedicar; doar | mostrar; apresentar; exibir | exibir-se; mostrar-se para que os outros vejam}
  \end{phonetics}
\end{entry}

%%%%% EOF %%%%%

