%%%
%%% Radical "⽝"
%%%

\section*{Radical 94: ``⽝'' (犭)}\addcontentsline{toc}{section}{Radical 94: ⽝、犭}

\begin{entry}{犬}{4}{⽝}[Kangxi 94]
  \begin{phonetics}{犬}{quan3}
    \definition{s.}{cachorro}
  \end{phonetics}
\end{entry}

\begin{entry}{犯法}{5,8}{⽝、⽔}
  \begin{phonetics}{犯法}{fan4fa3}
    \definition{v.}{violar (quebrar) a lei}
  \end{phonetics}
\end{entry}

\begin{entry}{犯罪}{5,13}{⽝、⽹}
  \begin{phonetics}{犯罪}{fan4zui4}
    \definition{v.+compl.}{cometer  um crime (uma ofensa)}
  \end{phonetics}
\end{entry}

\begin{entry}{状况}{7,7}{⽝、⼎}
  \begin{phonetics}{状况}{zhuang4kuang4}[][HSK 3]
    \definition[个,种]{s.}{estado; \emph{status}; condição; estado de coisas}
  \end{phonetics}
\end{entry}

\begin{entry}{状态}{7,8}{⽝、⼼}
  \begin{phonetics}{状态}{zhuang4tai4}[][HSK 3]
    \definition[种,个]{s.}{\emph{status}; estado; condição; estado de coisas; a forma em que uma pessoa ou coisa aparece}
  \end{phonetics}
\end{entry}

\begin{entry}{狂}{7}{⽝}
  \begin{phonetics}{狂}{kuang2}[][HSK 5]
    \definition*{s.}{sobrenome Kuang}
    \definition{adj.}{louco; maluco | violento; selvagem | selvagem; delirante; furioso; desenfreado; desinibido; sem restrições | arrogante; autoritário}
  \end{phonetics}
\end{entry}

\begin{entry}{狂欢节}{7,6,5}{⽝、⽋、⾋}
  \begin{phonetics}{狂欢节}{kuang2huan1jie2}
    \definition*{s.}{Carnaval}
  \end{phonetics}
\end{entry}

\begin{entry}{狒狒}{8,8}{⽝、⽝}
  \begin{phonetics}{狒狒}{fei4fei4}
    \definition{s.}{babuíno}
  \end{phonetics}
\end{entry}

\begin{entry}{狗}{8}{⽝}
  \begin{phonetics}{狗}{gou3}[][HSK 2]
    \definition[条,只]{s.}{cão | cachorro}
  \end{phonetics}
\end{entry}

\begin{entry}{独}{9}{⽝}
  \begin{phonetics}{独}{du2}
    \definition{adj.}{sozinho | solitário | solteiro}
    \definition{adv.}{apenas}
  \end{phonetics}
\end{entry}

\begin{entry}{独立}{9,5}{⽝、⽴}
  \begin{phonetics}{独立}{du2li4}[][HSK 4]
    \definition{adj.}{independente; por conta própria | separado; respectivo; descreve algo que é separado e não está em contato com outra coisa}
    \definition{prep.}{independente de; separado de; não mais anexado à unidade original, mas uma unidade separada}
    \definition{v.}{ficar sozinho | alcançar a independência; tornar-se um país independente; liberdade de um Estado, regime ou organização contra interferência, controle e dominação por forças externas}
  \end{phonetics}
\end{entry}

\begin{entry}{独自}{9,6}{⽝、⾃}
  \begin{phonetics}{独自}{du2 zi4}[][HSK 4]
    \definition{adj.}{sozinho; por si mesmo; por conta própria}
  \end{phonetics}
\end{entry}

\begin{entry}{独特}{9,10}{⽝、⽜}
  \begin{phonetics}{独特}{du2te4}[][HSK 4]
    \definition{adj.}{único; distinto; original; especial}
  \end{phonetics}
\end{entry}

\begin{entry}{狭}{9}{⽝}
  \begin{phonetics}{狭}{xia2}
    \definition{adj.}{estreito}
  \end{phonetics}
\end{entry}

\begin{entry}{猃狁}{10,7}{⽝、⽝}
  \begin{phonetics}{猃狁}{xian3yun3}
    \definition*{s.}{Termo da dinastia Zhou para uma tribo nômade do norte mais tarde chamou o Xiongnu (匈奴) nas dinastias Qin e Han}
  \seealsoref{匈奴}{xiong1nu2}
  \end{phonetics}
\end{entry}

\begin{entry}{猎物}{11,8}{⽝、⽜}
  \begin{phonetics}{猎物}{lie4wu4}
    \definition{s.}{presa (vítima de um predador)}
  \end{phonetics}
\end{entry}

\begin{entry}{猛}{11}{⽝}
  \begin{phonetics}{猛}{meng3}
    \definition{adj.}{feroz | violento | corajoso | abrupto | (gíria) incrível}
    \definition{adv.}{de repente}
  \end{phonetics}
\end{entry}

\begin{entry}{猛然}{11,12}{⽝、⽕}
  \begin{phonetics}{猛然}{meng3ran2}
    \definition{adv.}{de repente | abruptamente}
  \end{phonetics}
\end{entry}

\begin{entry}{猜}{11}{⽝}
  \begin{phonetics}{猜}{cai1}[][HSK 5]
    \definition{v.}{adivinhar; conjecturar; especular | suspeitar; ser cauteloso com os outros; desconfiar dos outros}
  \end{phonetics}
\end{entry}

\begin{entry}{猜测}{11,9}{⽝、⽔}
  \begin{phonetics}{猜测}{cai1 ce4}[][HSK 5]
    \definition[个,种]{s.}{advinhação; conjectura; suposição; especulação}
    \definition{v.}{adivinhar; conjecturar; especular; estimar a partir da imaginação}
  \end{phonetics}
\end{entry}

\begin{entry}{猪}{11}{⽝}
  \begin{phonetics}{猪}{zhu1}[][HSK 3]
    \definition[口,头]{s.}{porco; suíno}
  \end{phonetics}
\end{entry}

\begin{entry}{猪头}{11,5}{⽝、⼤}
  \begin{phonetics}{猪头}{zhu1tou2}
    \definition{s.}{tolo | idiota}
  \end{phonetics}
\end{entry}

\begin{entry}{猪柳}{11,9}{⽝、⽊}
  \begin{phonetics}{猪柳}{zhu1liu3}
    \definition{s.}{filé de porco}
  \end{phonetics}
\end{entry}

\begin{entry}{猪笼}{11,11}{⽝、⽵}
  \begin{phonetics}{猪笼}{zhu1long2}
    \definition{s.}{estrutura cilíndrica de bambu ou arame usada para restringir um porco durante o transporte}
  \end{phonetics}
\end{entry}

\begin{entry}{猪窠}{11,13}{⽝、⽳}
  \begin{phonetics}{猪窠}{zhu1ke1}
    \definition{s.}{chiqueiro}
  \end{phonetics}
\end{entry}

\begin{entry}{猫}{11}{⽝}
  \begin{phonetics}{猫}{mao1}[][HSK 2]
    \definition[只]{s.}{gato |  (empréstimo linguístico) (coloquial) MODEM}
    \definition{v.}{(dialeto) esconder-se}
  \end{phonetics}
  \begin{phonetics}{猫}{mao2}
    \definition{v.}{utilizado em 猫腰 \dpy{mao2yao1}}
    \seeref{猫腰}{mao2yao1}
  \end{phonetics}
\end{entry}

\begin{entry}{猫腰}{11,13}{⽝、⾁}
  \begin{phonetics}{猫腰}{mao2yao1}
    \definition{v.}{curvar-se}
  \end{phonetics}
\end{entry}

\begin{entry}{猫熊}{11,14}{⽝、⽕}
  \begin{phonetics}{猫熊}{mao1xiong2}
    \definition[把,只]{s.}{panda gigante}
  \seealsoref{熊猫}{xiong2mao1}
  \end{phonetics}
\end{entry}

\begin{entry}{猩猩}{12,12}{⽝、⽝}
  \begin{phonetics}{猩猩}{xing1xing5}
    \definition{s.}{orangotango}
  \end{phonetics}
\end{entry}

\begin{entry}{猴}{12}{⽝}
  \begin{phonetics}{猴}{hou2}[][HSK 5]
    \definition{adj.}{esperto; inteligente; perspicaz}
    \definition[只,群]{s.}{macaco}
  \end{phonetics}
\end{entry}

\begin{entry}{猴子}{12,3}{⽝、⼦}
  \begin{phonetics}{猴子}{hou2zi5}
    \definition[只]{s.}{macaco}
  \end{phonetics}
\end{entry}

%%%%% EOF %%%%%

