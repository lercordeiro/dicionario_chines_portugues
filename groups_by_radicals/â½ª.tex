%%%
%%% Radical "⽪"
%%%

\section*{Radical 107: ``⽪''}\addcontentsline{toc}{section}{Radical 107: ⽪}

\begin{entry}{皮}{5}{⽪}[Kangxi 107]
  \begin{phonetics}{皮}{pi2}[][HSK 3]
    \definition*{s.}{Sobrenome Pi}
    \definition{adj.}{macios e encharcados; não mais crocantes | malandro; travesso | apático; endurecido; indiferente devido a repetidas repreensões | pegajoso; tenaz; resiliente}
    \definition{pref.}{pico- (um trilhonésimo)}
    \definition[层,块,张,个]{s.}{pele; casca; uma camada de tecido na superfície dos organismos animais e vegetais | pele; couro; couro processado | capa; embalagem; a camada externa que envolve algo | superfície do objeto | folha; peça larga e plana (de algum material fino) | borracha}
  \end{phonetics}
\end{entry}

\begin{entry}{皮下}{5,3}{⽪、⼀}
  \begin{phonetics}{皮下}{pi2xia4}
    \definition{adj.}{(injeção) subcutâneo | sob a pele}
  \end{phonetics}
\end{entry}

\begin{entry}{皮包}{5,5}{⽪、⼓}
  \begin{phonetics}{皮包}{pi2 bao1}[][HSK 3]
    \definition[个,只,款]{s.}{bolsa; pasta; portfólio; bolsas de couro}
  \end{phonetics}
\end{entry}

\begin{entry}{皮卡}{5,5}{⽪、⼘}
  \begin{phonetics}{皮卡}{pi2ka3}
    \definition{s.}{(empréstimo linguístico) \emph{pick-up} | caminhonete}
  \end{phonetics}
\end{entry}

\begin{entry}{皮卡丘}{5,5,5}{⽪、⼘、⼀}
  \begin{phonetics}{皮卡丘}{pi2ka3qiu1}
    \definition*{s.}{Pikachu (Pokémon, 口袋妖怪)}
  \seealsoref{口袋妖怪}{kou3dai4 yao1guai4}
  \end{phonetics}
\end{entry}

\begin{entry}{皮肤}{5,8}{⽪、⾁}
  \begin{phonetics}{皮肤}{pi2fu1}[][HSK 5]
    \definition{adj.}{superficial}
    \definition[层,块]{s.}{pele; couro; derme}
  \end{phonetics}
\end{entry}

\begin{entry}{皮球}{5,11}{⽪、⽟}
  \begin{phonetics}{皮球}{pi2 qiu2}[][HSK 6]
    \definition{s.}{bola (feita de borracha, couro etc.)}
  \end{phonetics}
\end{entry}

\begin{entry}{皮鞋}{5,15}{⽪、⾰}
  \begin{phonetics}{皮鞋}{pi2xie2}[][HSK 5]
    \definition[双,只,款]{s.}{sapatos feitos de couro}
  \end{phonetics}
\end{entry}

\begin{entry}{颇}{11}{⽪}
  \begin{phonetics}{颇}{po1}
    \definition*{s.}{Sobrenome Po}
    \definition{adv.}{muito, bastante (linguagem escrita)}
  \end{phonetics}
\end{entry}

%%%%% EOF %%%%%

