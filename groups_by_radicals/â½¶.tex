%%%
%%% Radical "⽶"
%%%

\section*{Radical 119: ``⽶''}\addcontentsline{toc}{section}{Radical 119: ⽶}

\begin{entry}{米}{6}{⽶}[Kangxi 119]
  \begin{phonetics}{米}{mi3}[][HSK 2,3]
    \definition*{s.}{sobrenome Mi}
    \definition{clas.}{metro (m); unidade principal de comprimento do sistema métrico}
    \definition[粒,斤]{s.}{arroz | sementes descascadas; refere-se a sementes comestíveis descascadas ou sem casca | qualquer coisa que se assemelhe a um grão de arroz}
  \end{phonetics}
\end{entry}

\begin{entry}{米饭}{6,7}{⽶、⾷}
  \begin{phonetics}{米饭}{mi3fan4}[][HSK 1]
    \definition{s.}{arroz (cozido)}
  \end{phonetics}
\end{entry}

\begin{entry}{类}{9}{⽶}
  \begin{phonetics}{类}{lei4}[][HSK 3]
    \definition*{s.}{sobrenome Lei}
    \definition{clas.}{tipo; espécie; categoria usada para pessoas ou coisas}
    \definition{s.}{classe; categoria; tipo; variedade; a combinação de muitas coisas semelhantes ou iguais}
    \definition{v.}{assemelhar-se a; ser semelhante a}
  \end{phonetics}
\end{entry}

\begin{entry}{类似}{9,6}{⽶、⼈}
  \begin{phonetics}{类似}{lei4si4}[][HSK 3]
    \definition{adj.}{semelhante; análogo}
  \end{phonetics}
\end{entry}

\begin{entry}{类型}{9,9}{⽶、⼟}
  \begin{phonetics}{类型}{lei4xing2}[][HSK 4]
    \definition[种,个]{s.}{tipo; espécie; categoria; tipos formados por coisas com características comuns}
  \end{phonetics}
\end{entry}

\begin{entry}{粉}{10}{⽶}
  \begin{phonetics}{粉}{fen3}
    \definition{adj.}{branco | rosa}
    \definition{s.}{pó | cosméticos em pó | farinha de trigo | macarrão ou outro alimento feito de feijão, arroz, batata, amido de batata-doce, etc. | macarrão de arroz}
    \definition{v.}{virar pó | caiar}
  \end{phonetics}
\end{entry}

\begin{entry}{粉丝}{10,5}{⽶、⼀}
  \begin{phonetics}{粉丝}{fen3si1}
    \definition{s.}{(empréstimo linguístico) fã | entusiasta de alguém ou alguma coisa}
    \definition[把]{s.}{aletria de amido de feijão | aletria chinesa | macarrão de celofane ou macarrão de vidro (transparente)}
  \end{phonetics}
\end{entry}

\begin{entry}{粉色}{10,6}{⽶、⾊}
  \begin{phonetics}{粉色}{fen3 se4}
    \definition{s.}{cor-de-rosa}
  \end{phonetics}
\end{entry}

\begin{entry}{粗}{11}{⽶}
  \begin{phonetics}{粗}{cu1}[][HSK 4]
    \definition{adj.}{largo (em diâmetro); grosso | grosseiro; rude; áspero | áspero; rouco | descuidado; negligente | rude; sem refinamento; vulgar}
    \definition{adv.}{grosseiramente; vagamente}
  \end{phonetics}
\end{entry}

\begin{entry}{粗心}{11,4}{⽶、⼼}
  \begin{phonetics}{粗心}{cu1xin1}[][HSK 4]
    \definition{adj.}{descuidado; irrefletido; (fazer as coisas) de forma desleixada, sem cuidado}
  \end{phonetics}
\end{entry}

\begin{entry}{粗心地做}{11,4,6,11}{⽶、⼼、⼟、⼈}
  \begin{phonetics}{粗心地做}{cu1xin1 di4 zuo4}
    \definition{adj.}{feito descuidadamente}
  \end{phonetics}
\end{entry}

\begin{entry}{粗糙}{11,16}{⽶、⽶}
  \begin{phonetics}{粗糙}{cu1cao1}
    \definition{adj.}{áspero | grosseiro}
  \end{phonetics}
\end{entry}

\begin{entry}{缓}{12}{⽶}
  \begin{phonetics}{缓}{huan3}
    \definition{adj.}{lento; sem pressa | sem tensão; relaxado}
    \definition{v.}{atrasar; adiar; protelar | recuperar; reviver; voltar a si}
  \end{phonetics}
\end{entry}

\begin{entry}{缓解}{12,13}{⽶、⾓}
  \begin{phonetics}{缓解}{huan3jie3}[][HSK 4]
    \definition{v.}{facilitar; aliviar; atenuar; amenizar; reduzir}
  \end{phonetics}
\end{entry}

\begin{entry}{粮食}{13,9}{⽶、⾷}
  \begin{phonetics}{粮食}{liang2shi5}[][HSK 4]
    \definition[种,斤,吨,袋]{s.}{alimentos; grãos; termo geral para os vários tipos de arroz, feijão, etc. que podem ser consumidos}
  \end{phonetics}
\end{entry}

\begin{entry}{精力}{14,2}{⽶、⼒}
  \begin{phonetics}{精力}{jing1li4}[][HSK 4]
    \definition[些]{s.}{energia; vigor; força mental e física}
  \end{phonetics}
\end{entry}

\begin{entry}{精灵}{14,7}{⽶、⽕}
  \begin{phonetics}{精灵}{jing1ling2}
    \definition{s.}{espírito | fada | elfo | duende | gênio}
  \end{phonetics}
\end{entry}

\begin{entry}{精品}{14,9}{⽶、⼝}
  \begin{phonetics}{精品}{jing1pin3}
    \definition{s.}{produtos de qualidade | produto premium | bom trabalho (de arte)}
  \end{phonetics}
\end{entry}

\begin{entry}{精神}{14,9}{⽶、⽰}
  \begin{phonetics}{精神}{jing1shen2}[][HSK 3]
    \definition[种,个,类,股]{s.}{espírito; mente; estado mental; refere-se à consciência, às atividades mentais e ao estado psicológico geral de uma pessoa | substância; espírito; essência; propósito; significado principal}
  \end{phonetics}
  \begin{phonetics}{精神}{jing1shen5}[][HSK 3]
    \definition{adj.}{animado; espirituoso; vigoroso; descreve uma pessoa como cheia de energia | muito bonito; boa aparência, bom físico}
    \definition[种,个,类,股]{s.}{impulso; vigor; vitalidade}
  \end{phonetics}
\end{entry}

\begin{entry}{精致}{14,10}{⽶、⾄}
  \begin{phonetics}{精致}{jing1zhi4}
    \definition{adj.}{delicado | exótico | refinado}
  \end{phonetics}
\end{entry}

\begin{entry}{精彩}{14,11}{⽶、⼺}
  \begin{phonetics}{精彩}{jing1cai3}[][HSK 3]
    \definition{adj.}{brilhante; esplêndido; maravilhoso}
  \end{phonetics}
\end{entry}

\begin{entry}{糆}{15}{⽶}
  \begin{phonetics}{糆}{mian4}
    \variantof{面}
  \end{phonetics}
\end{entry}

\begin{entry}{糊里糊涂}{15,7,15,10}{⽶、⾥、⽶、⽔}
  \begin{phonetics}{糊里糊涂}{hu2li5hu2tu5}
    \definition{adj.}{desnorteado | perturbado}
  \end{phonetics}
\end{entry}

\begin{entry}{糕}{16}{⽶}
  \begin{phonetics}{糕}{gao1}
    \definition{s.}{bolo; alimentos feitos de farinha de arroz, farinha de trigo, etc.}
  \end{phonetics}
\end{entry}

\begin{entry}{糕点}{16,9}{⽶、⽕}
  \begin{phonetics}{糕点}{gao1dian3}
    \definition{s.}{bolos | pastéis}
  \end{phonetics}
\end{entry}

\begin{entry}{糕点师}{16,9,6}{⽶、⽕、⼱}
  \begin{phonetics}{糕点师}{gao1dian3 shi1}
    \definition{s.}{confeiteiro}
  \end{phonetics}
\end{entry}

\begin{entry}{糕点店}{16,9,8}{⽶、⽕、⼴}
  \begin{phonetics}{糕点店}{gao1dian3 dian4}
    \definition{s.}{confeitaria}
  \end{phonetics}
\end{entry}

\begin{entry}{糖}{16}{⽶}
  \begin{phonetics}{糖}{tang2}[][HSK 3]
    \definition[包,斤,勺,袋,块]{s.}{açúcar; um tipo de açúcar; um tipo de composto orgânico, que pode ser dividido em três tipos: monossacarídeos, dissacarídeos e polissacarídeos; é a principal substância que produz energia térmica no corpo humano, como glicose, sacarose, lactose, amido, etc. | açúcar; açúcar comestível; termo geral para açúcar | doces; balas | carboidrato; algo doce e calórico}
  \end{phonetics}
\end{entry}

\begin{entry}{糖醋鱼}{16,15,8}{⽶、⾣、⿂}
  \begin{phonetics}{糖醋鱼}{tang2cu4yu2}
    \definition{s.}{peixe guisado em molho agridoce (prato)}
  \end{phonetics}
\end{entry}

\begin{entry}{糟}{17}{⽶}
  \begin{phonetics}{糟}{zao1}[][HSK 5]
    \definition{adj.}{pobre; apodrecido; deteriorado | estragado; em uma bagunça; em um estado miserável (terrível) | (situação ou circunstância) ruim; desfavorável}
    \definition{s.}{resíduos de destilação de bebidas alcoólicas; resíduos do processo de fermentação do vinho}
    \definition{v.}{marinar alimentos em vinho ou mosto}
  \end{phonetics}
\end{entry}

\begin{entry}{糟糕}{17,16}{⽶、⽶}
  \begin{phonetics}{糟糕}{zao1gao1}[][HSK 5]
    \definition{adj.}{(corpo, situação, etc.) muito ruim, péssimo}
    \definition{interj.}{que terrível; que má sorte; muito ruim}
  \end{phonetics}
\end{entry}

%%%%% EOF %%%%%

