%%%
%%% Radical "⽶"
%%%

\section*{Radical 119: ``⽶''}\addcontentsline{toc}{section}{Radical 119: ⽶}

\begin{Entry}{米}{6}{⽶}[Kangxi 119]
  \begin{Phonetics}{米}{mi3}[][HSK 2,3]
    \definition*{s.}{Sobrenome Mi}
    \definition{clas.}{m, metro; unidade principal de comprimento do sistema métrico}
    \definition[粒,斤]{s.}{arroz | sementes descascadas; refere-se a sementes comestíveis descascadas ou sem casca | qualquer coisa que se assemelhe a um grão de arroz}
  \end{Phonetics}
\end{Entry}

\begin{Entry}{米饭}{6,7}{⽶、⾷}
  \begin{Phonetics}{米饭}{mi3fan4}[][HSK 1]
    \definition{s.}{arroz (cozido)}
  \end{Phonetics}
\end{Entry}

\begin{Entry}{类}{9}{⽶}
  \begin{Phonetics}{类}{lei4}[][HSK 3]
    \definition*{s.}{Sobrenome Lei}
    \definition{clas.}{tipo; espécie; categoria usada para pessoas ou coisas}
    \definition{s.}{classe; categoria; tipo; variedade; a combinação de muitas coisas semelhantes ou iguais}
    \definition{v.}{assemelhar-se a; ser semelhante a}
  \end{Phonetics}
\end{Entry}

\begin{Entry}{类似}{9,6}{⽶、⼈}
  \begin{Phonetics}{类似}{lei4si4}[][HSK 3]
    \definition{adj.}{semelhante; análogo}
  \end{Phonetics}
\end{Entry}

\begin{Entry}{类型}{9,9}{⽶、⼟}
  \begin{Phonetics}{类型}{lei4xing2}[][HSK 4]
    \definition[种,个]{s.}{tipo; espécie; categoria; tipos formados por coisas com características comuns}
  \end{Phonetics}
\end{Entry}

\begin{Entry}{粉}{10}{⽶}
  \begin{Phonetics}{粉}{fen3}[][HSK 7-9]
    \definition{adj.}{branco | rosa}
    \definition{s.}{pó | cosméticos em pó | farinha de trigo | macarrão ou outro alimento feito de feijão, arroz, batata, amido de batata-doce, etc. | macarrão de arroz}
    \definition{v.}{virar pó | Dialeto: caiar}
  \end{Phonetics}
\end{Entry}

\begin{Entry}{粉丝}{10,5}{⽶、⼀}
  \begin{Phonetics}{粉丝}{fen3si1}[][HSK 7-9]
    \definition{s.}{(empréstimo linguístico) fã | entusiasta de alguém ou alguma coisa}
    \definition[个,群,位,名,些,批]{s.}{aletria de amido de feijão ou batata; aletria chinesa; macarrão de celofane ou macarrão de vidro (transparente) | Empréstimo linguístico: fã; refere-se a uma pessoa que é obcecada ou adora uma celebridade}
  \end{Phonetics}
\end{Entry}

\begin{Entry}{粉色}{10,6}{⽶、⾊}
  \begin{Phonetics}{粉色}{fen3 se4}
    \definition{s.}{cor-de-rosa}
  \end{Phonetics}
\end{Entry}

\begin{Entry}{粉碎}{10,13}{⽶、⽯}
  \begin{Phonetics}{粉碎}{fen3sui4}[][HSK 7-9]
    \definition{adj.}{pulverizado; quebrado em pedaços; descreve algo que está muito quebrado, quebrado em partículas muito pequenas}
    \definition{v.}{esmagar; transformar as coisas em partículas muito pequenas | esmagar; quebrar; estilhaçar; fazer com que a outra parte falhe ou seja completamente destruída}
  \end{Phonetics}
\end{Entry}

\begin{Entry}{粗}{11}{⽶}
  \begin{Phonetics}{粗}{cu1}[][HSK 4]
    \definition{adj.}{largo (em diâmetro); grosso | grosseiro; rude; áspero | áspero; rouco | descuidado; negligente | rude; sem refinamento; vulgar}
    \definition{adv.}{grosseiramente; vagamente}
  \end{Phonetics}
\end{Entry}

\begin{Entry}{粗心}{11,4}{⽶、⼼}
  \begin{Phonetics}{粗心}{cu1xin1}[][HSK 4]
    \definition{adj.}{descuidado; irrefletido; (fazer as coisas) de forma desleixada, sem cuidado}
  \end{Phonetics}
\end{Entry}

\begin{Entry}{粗心大意}{11,4,3,13}{⽶、⼼、⼤、⼼}
  \begin{Phonetics}{粗心大意}{cu1xin1-da4yi4}[][HSK 7-9]
    \definition{expr.}{ser negligente; descuidado; inadvertido; desmiolado; descuidado e negligente; negligente; remisso; refere-se a fazer as coisas de forma descuidada}
  \end{Phonetics}
\end{Entry}

\begin{Entry}{粗心地做}{11,4,6,11}{⽶、⼼、⼟、⼈}
  \begin{Phonetics}{粗心地做}{cu1xin1 di4 zuo4}
    \definition{adj.}{feito descuidadamente}
  \end{Phonetics}
\end{Entry}

\begin{Entry}{粗略}{11,11}{⽶、⽥}
  \begin{Phonetics}{粗略}{cu1lve4}[][HSK 7-9]
    \definition{adj.}{grosseiro; rudimentar; superficial}
  \end{Phonetics}
\end{Entry}

\begin{Entry}{粗鲁}{11,12}{⽶、⿂}
  \begin{Phonetics}{粗鲁}{cu1lu3}[][HSK 7-9]
    \definition{adj.}{rude; grosseiro; incivilizado}
  \end{Phonetics}
\end{Entry}

\begin{Entry}{粗暴}{11,15}{⽶、⽇}
  \begin{Phonetics}{粗暴}{cu1bao4}[][HSK 7-9]
    \definition{adj.}{rude; áspero; bruto; brutal; violento}
  \end{Phonetics}
\end{Entry}

\begin{Entry}{粗糙}{11,16}{⽶、⽶}
  \begin{Phonetics}{粗糙}{cu1cao1}[][HSK 7-9]
    \definition{adj.}{áspero; grosseiro; não é liso; não é redondo; não é fino | desleixado; descuidado; não meticuloso}
  \end{Phonetics}
\end{Entry}

\begin{Entry}{粥}{12}{⽶}
  \begin{Phonetics}{粥}{yu4}
    \definition{v.}{dar a luz; ter filhos}
  \end{Phonetics}
  \begin{Phonetics}{粥}{zhou1}[][HSK 6]
    \definition[碗,锅,口]{s.}{mingau; mingau de aveia; alimentos semilíquidos feitos de grãos ou grãos misturados com outras coisas}
  \end{Phonetics}
\end{Entry}

\begin{Entry}{粪}{12}{⽶}
  \begin{Phonetics}{粪}{fen4}[][HSK 7-9]
    \definition[缸,桶]{s.}{excremento; fezes; esterco}
    \definition{v.}{Literário: aplicar esterco; fertilizar | Literário: limpar; remover; eliminar; acabar com}
  \end{Phonetics}
\end{Entry}

\begin{Entry}{粪便}{12,9}{⽶、⼈}
  \begin{Phonetics}{粪便}{fen4bian4}[][HSK 7-9]
    \definition{s.}{excremento; fezes; esterco; excrementos e urina}
  \end{Phonetics}
\end{Entry}

\begin{Entry}{缓}{12}{⽶}
  \begin{Phonetics}{缓}{huan3}
    \definition{adj.}{lento; sem pressa | sem tensão; relaxado}
    \definition{v.}{atrasar; adiar; protelar | recuperar; reviver; voltar a si}
  \end{Phonetics}
\end{Entry}

\begin{Entry}{缓解}{12,13}{⽶、⾓}
  \begin{Phonetics}{缓解}{huan3jie3}[][HSK 4]
    \definition{v.}{facilitar; aliviar; atenuar; amenizar; reduzir}
  \end{Phonetics}
\end{Entry}

\begin{Entry}{粮}{13}{⽶}
  \begin{Phonetics}{粮}{liang2}
    \definition[斤,粒]{s.}{grãos; alimentos; provisões | imposto sobre grãos | nutrição | imposto agrícola; grãos como imposto agrícola}
  \end{Phonetics}
\end{Entry}

\begin{Entry}{粮食}{13,9}{⽶、⾷}
  \begin{Phonetics}{粮食}{liang2shi5}[][HSK 4]
    \definition[种,吨,袋,颗,粒]{s.}{alimentos; grãos; termo geral para os vários tipos de arroz, feijão, etc. que podem ser consumidos}
  \end{Phonetics}
\end{Entry}

\begin{Entry}{精}{14}{⽶}
  \begin{Phonetics}{精}{jing1}[][HSK 6]
    \definition{adv.}{muito; extremamente; antes de certos adjetivos, significa 十分 ou 非常}
    \definition{s.}{refinado; escolhido; escolha; purificado ou selecionado | perfeito; excelente; melhor | fino (em oposição a 粗); preciso; meticuloso | inteligente; astuto; esperto | habilidoso; versado; proficiente | extrato; essência; essência refinada ou selecionada; extraída | energia; espírito | semente; esperma; sêmen | \emph{goblin}; espírito; elfo; demônio}
  \seealsoref{粗}{cu1}
  \seealsoref{非常}{fei1chang2}
  \seealsoref{十分}{shi2fen1}
  \end{Phonetics}
\end{Entry}

\begin{Entry}{精力}{14,2}{⽶、⼒}
  \begin{Phonetics}{精力}{jing1li4}[][HSK 4]
    \definition[些]{s.}{energia; vigor; força mental e física}
  \end{Phonetics}
\end{Entry}

\begin{Entry}{精灵}{14,7}{⽶、⽕}
  \begin{Phonetics}{精灵}{jing1ling2}
    \definition{s.}{espírito | fada | elfo | duende | gênio}
  \end{Phonetics}
\end{Entry}

\begin{Entry}{精品}{14,9}{⽶、⼝}
  \begin{Phonetics}{精品}{jing1pin3}[][HSK 6]
    \definition[个]{s.}{belas obras (de arte); objetos de arte | produtos de qualidade; artigos de excelente qualidade; produto \emph{premium}}
  \end{Phonetics}
\end{Entry}

\begin{Entry}{精神}{14,9}{⽶、⽰}
  \begin{Phonetics}{精神}{jing1shen2}[][HSK 3]
    \definition[种,个,类,股]{s.}{espírito; mente; estado mental; refere-se à consciência, às atividades mentais e ao estado psicológico geral de uma pessoa | substância; espírito; essência; propósito; significado principal}
  \end{Phonetics}
  \begin{Phonetics}{精神}{jing1shen5}[][HSK 3]
    \definition{adj.}{animado; espirituoso; vigoroso; descreve uma pessoa como cheia de energia | muito bonito; boa aparência, bom físico}
    \definition[种,个,类,股]{s.}{impulso; vigor; vitalidade}
  \end{Phonetics}
\end{Entry}

\begin{Entry}{精美}{14,9}{⽶、⽺}
  \begin{Phonetics}{精美}{jing1 mei3}[][HSK 6]
    \definition{adj.}{elegante; requintado}
  \end{Phonetics}
\end{Entry}

\begin{Entry}{精致}{14,10}{⽶、⾄}
  \begin{Phonetics}{精致}{jing1zhi4}
    \definition{adj.}{delicado | exótico | refinado}
  \end{Phonetics}
\end{Entry}

\begin{Entry}{精彩}{14,11}{⽶、⼺}
  \begin{Phonetics}{精彩}{jing1cai3}[][HSK 3]
    \definition{adj.}{brilhante; esplêndido; maravilhoso}
  \end{Phonetics}
\end{Entry}

\begin{Entry}{糆}{15}{⽶}
  \begin{Phonetics}{糆}{mian4}
    \variantof{面}
  \end{Phonetics}
\end{Entry}

\begin{Entry}{糊}{15}{⽶}
  \begin{Phonetics}{糊}{hu1}
    \definition{v.}{colar; untar; usar uma pasta mais espessa para revestir costuras, furos ou superfícies planas}
  \end{Phonetics}
  \begin{Phonetics}{糊}{hu2}[][HSK 7-9]
    \definition{adj.}{queimado}
    \definition{s.}{mingau; pasta; papa}
    \definition{v.}{colar com pasta; colar | (comida) ser queimado}
  \end{Phonetics}
  \begin{Phonetics}{糊}{hu4}
    \definition{s.}{pasta; comida que parece mingau}
  \end{Phonetics}
\end{Entry}

\begin{Entry}{糊里糊涂}{15,7,15,10}{⽶、⾥、⽶、⽔}
  \begin{Phonetics}{糊里糊涂}{hu2 li5 hu2tu5}
    \definition{adj.}{desnorteado | perturbado}
  \end{Phonetics}
\end{Entry}

\begin{Entry}{糊涂}{15,10}{⽶、⽔}
  \begin{Phonetics}{糊涂}{hu2tu5}[][HSK 7-9]
    \definition{adj.}{confuso; perplexo; desnorteado; com compreensão pouco clara ou confusa das coisas | confuso; com conteúdo confuso}
  \end{Phonetics}
\end{Entry}

\begin{Entry}{糕}{16}{⽶}
  \begin{Phonetics}{糕}{gao1}
    \definition{s.}{bolo; alimentos feitos de farinha de arroz, farinha de trigo, etc.}
  \end{Phonetics}
\end{Entry}

\begin{Entry}{糕点}{16,9}{⽶、⽕}
  \begin{Phonetics}{糕点}{gao1dian3}
    \definition{s.}{bolos | pastéis}
  \end{Phonetics}
\end{Entry}

\begin{Entry}{糕点师}{16,9,6}{⽶、⽕、⼱}
  \begin{Phonetics}{糕点师}{gao1dian3 shi1}
    \definition{s.}{confeiteiro}
  \end{Phonetics}
\end{Entry}

\begin{Entry}{糕点店}{16,9,8}{⽶、⽕、⼴}
  \begin{Phonetics}{糕点店}{gao1dian3 dian4}
    \definition{s.}{confeitaria}
  \end{Phonetics}
\end{Entry}

\begin{Entry}{糖}{16}{⽶}
  \begin{Phonetics}{糖}{tang2}[][HSK 3]
    \definition[包,斤,勺,袋,块]{s.}{açúcar; um tipo de açúcar; um tipo de composto orgânico, que pode ser dividido em três tipos: monossacarídeos, dissacarídeos e polissacarídeos; é a principal substância que produz energia térmica no corpo humano, como glicose, sacarose, lactose, amido, etc. | açúcar; açúcar comestível; termo geral para açúcar | doces; balas | carboidrato; algo doce e calórico}
  \end{Phonetics}
\end{Entry}

\begin{Entry}{糖醋鱼}{16,15,8}{⽶、⾣、⿂}
  \begin{Phonetics}{糖醋鱼}{tang2cu4yu2}
    \definition{s.}{peixe guisado em molho agridoce (prato)}
  \end{Phonetics}
\end{Entry}

\begin{Entry}{糟}{17}{⽶}
  \begin{Phonetics}{糟}{zao1}[][HSK 5]
    \definition{adj.}{pobre; apodrecido; deteriorado | estragado; em uma bagunça; em um estado miserável (terrível) | (situação ou circunstância) ruim; desfavorável}
    \definition{s.}{resíduos de destilação de bebidas alcoólicas; resíduos do processo de fermentação do vinho}
    \definition{v.}{marinar alimentos em vinho ou mosto | desperdiçar; estragar; destruir}
  \end{Phonetics}
\end{Entry}

\begin{Entry}{糟糕}{17,16}{⽶、⽶}
  \begin{Phonetics}{糟糕}{zao1gao1}[][HSK 5]
    \definition{adj.}{(corpo, situação, etc.) muito ruim, péssimo}
    \definition{interj.}{que terrível; que má sorte; muito ruim}
  \end{Phonetics}
\end{Entry}

%%%%% EOF %%%%%

