%%%
%%% Radical "⼏"
%%%

\section*{Radical 16: ``⼏''}\addcontentsline{toc}{section}{Radical 16: ⼏}

\begin{entry}{几}{2}{⼏}[Kangxi 16]
  \begin{phonetics}{几}{ji1}
    \definition{adv.}{quase; praticamente}
    \definition{s.}{uma mesa pequena}
  \end{phonetics}
  \begin{phonetics}{几}{ji3}[][HSK 1]
    \definition{adv.}{quanto?, usado para perguntar sobre quantidade e tempo}
    \definition{num.}{alguns; vários; poucos; indica um número indeterminado maior que um e menor que dez}
  \end{phonetics}
\end{entry}

\begin{entry}{几乎}{2,5}{⼏、⼃}
  \begin{phonetics}{几乎}{ji1hu1}[][HSK 4]
    \definition{adv.}{quase; praticamente; próximo a | perto de; quase; à beira de}
  \end{phonetics}
\end{entry}

\begin{entry}{几何}{2,7}{⼏、⼈}
  \begin{phonetics}{几何}{ji3he2}
    \definition{s.}{geometria}
  \end{phonetics}
\end{entry}

\begin{entry}{几率}{2,11}{⼏、⽞}
  \begin{phonetics}{几率}{ji1lv4}
    \definition{s.}{probabilidade; um evento pode ou não ocorrer sob as mesmas condições, a grandeza que indica a possibilidade de ocorrência é chamada de probabilidade}
  \end{phonetics}
\end{entry}

\begin{entry}{凡}{3}{⼏}
  \begin{phonetics}{凡}{fan2}
    \definition*{s.}{Sobrenome Fan}
    \definition{adj.}{comum; ordinário}
    \definition{adv.}{qualquer; todos; todo | em tudo; completamente}
    \definition{s.}{este mundo mortal; a terra | o mundo secular; refere-se ao mundo humano | uma nota da escala em Gongchepu (工尺谱), correspondente a 4 na notação musical numerada | ideia geral; esboço}
  \seealsoref{工尺谱}{gong1 che3 pu3}
  \end{phonetics}
\end{entry}

\begin{entry}{凡是}{3,9}{⼏、⽇}
  \begin{phonetics}{凡是}{fan2shi4}[][HSK 6]
    \definition{adv.}{todos; qualquer; cada; resumir tudo dentro de um determinado âmbito}
  \end{phonetics}
\end{entry}

\begin{entry}{凤}{4}{⼏}
  \begin{phonetics}{凤}{feng4}
    \definition[只]{s.}{fênix}
    \definition{s.}{Sobrenome Feng}
  \end{phonetics}
\end{entry}

\begin{entry}{凤凰}{4,11}{⼏、⼏}
  \begin{phonetics}{凤凰}{feng4huang2}
    \definition{s.}{fênix}
  \end{phonetics}
\end{entry}

\begin{entry}{凭}{8}{⼏}
  \begin{phonetics}{凭}{ping2}[][HSK 5]
    \definition{prep.}{introduzir a ação ou o comportamento com base em algo; quando a frase nominal após 凭 é longa, pode-se adicionar 着 após 凭}
    \definition[张]{s.}{prova; evidência}
    \definition{v.}{apoiar-se; encostar-se | confiar em; depender de | basear-se em; tomar como base}
  \seealsoref{着}{zhe5}
  \end{phonetics}
\end{entry}

%%%%% EOF %%%%%

