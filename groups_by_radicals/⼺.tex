%%%
%%% Radical "⼺"
%%%
\section*{Radical 59: ``⼺''}\addcontentsline{toc}{section}{Radical 59: ⼺}

%%%%%%%%%% 形 %%%%%%%%%%
\subsection*{形}

\begin{Entry}{形}{7}{⼺}
  \begin{Phonetics}{形}{xing2}[][HSK 6]
    \definition{s.}{forma; formato | corpo; entidade}
    \definition{v.}{aparecer; revelar; mostrar | comparar; contrastar}
  \end{Phonetics}
\end{Entry}

\begin{Entry}{形式}{7,6}{⼺、⼷}
  \begin{Phonetics}{形式}{xing2shi4}[][HSK 3]
    \definition[种,个]{s.}{forma; formato; modalidade; a aparência, estrutura ou estado das coisas, etc.}
  \end{Phonetics}
\end{Entry}

\begin{Entry}{形成}{7,6}{⼺、⼽}
  \begin{Phonetics}{形成}{xing2cheng2}[][HSK 3]
    \definition{v.}{moldar; formar; tomar forma; tornar-se algo ou surgir uma situação após mudanças e desenvolvimentos}
  \end{Phonetics}
\end{Entry}

\begin{Entry}{形而上学}{7,6,3,8}{⼺、⽽、⼀、⼦}
  \begin{Phonetics}{形而上学}{xing2'er2shang4xue2}
    \definition{s.}{metafísica}
  \end{Phonetics}
\end{Entry}

\begin{Entry}{形状}{7,7}{⼺、⽝}
  \begin{Phonetics}{形状}{xing2zhuang4}[][HSK 3]
    \definition[个,种]{s.}{forma; aparência ; aspecto; a aparência de um objeto ou figura, representada pela combinação de superfícies ou linhas externas}
  \end{Phonetics}
\end{Entry}

\begin{Entry}{形势}{7,8}{⼺、⼒}
  \begin{Phonetics}{形势}{xing2shi4}[][HSK 4]
    \definition[个,种]{s.}{terreno; características topográficas; situação geográfica, principalmente de uma perspectiva militar | situação; circunstâncias; a situação geral, a tendência de como as coisas estão se desenvolvendo e mudando | geralmente não é usado em situações pessoais}
  \end{Phonetics}
\end{Entry}

\begin{Entry}{形态}{7,8}{⼺、⼼}
  \begin{Phonetics}{形态}{xing2tai4}[][HSK 5]
    \definition[种]{s.}{forma; forma como as coisas se apresentam | forma; padrão; postura | morfologia; forma; Gramática: refere-se às formas internas de mudança das palavras, incluindo a formação de palavras e as mudanças morfológicas}
  \end{Phonetics}
\end{Entry}

\begin{Entry}{形容}{7,10}{⼺、⼧}
  \begin{Phonetics}{形容}{xing2rong2}[][HSK 4]
    \definition{s.}{aparência; semblante}
    \definition{v.}{descrever}
  \end{Phonetics}
\end{Entry}

\begin{Entry}{形象}{7,11}{⼺、⾗}
  \begin{Phonetics}{形象}{xing2xiang4}[][HSK 3]
    \definition{adj.}{vívido; expressão concreta e vívida}
    \definition[个,种]{s.}{imagem; forma; figura; formas ou posturas específicas que podem despertar pensamentos ou emoções nas pessoas | imagem literária; imagem artística; pessoas ou coisas com características diferentes criadas na literatura, no cinema e em outras artes}
  \end{Phonetics}
\end{Entry}

%%%%%%%%%% 彩 %%%%%%%%%%
\subsection*{彩}

\begin{Entry}{彩}{11}{⼺}
  \begin{Phonetics}{彩}{cai3}
    \definition{s.}{cor | aplausos; vivas | variedade; brilho; esplendor | prêmio; loteria | sangue de uma ferida | habilidades especiais empregadas em mágica ou ópera para alcançar um efeito desejado | seda colorida | cores variadas | graça na arte; graciosidade | prêmio de loteria; ganhos | efeitos especiais no teatro chinês (simbolizando sangue, fogo, etc.)}
  \end{Phonetics}
\end{Entry}

\begin{Entry}{彩电}{11,5}{⼺、⽥}
  \begin{Phonetics}{彩电}{cai3dian4}[][HSK 7-9]
    \definition[台,个]{s.}{TV à cores}
  \end{Phonetics}
\end{Entry}

\begin{Entry}{彩色}{11,6}{⼺、⾊}
  \begin{Phonetics}{彩色}{cai3 se4}[][HSK 3]
    \definition[个,种]{s.}{multicolorido; cor; várias cores}
  \end{Phonetics}
\end{Entry}

\begin{Entry}{彩虹}{11,9}{⼺、⾍}
  \begin{Phonetics}{彩虹}{cai3hong2}[][HSK 7-9]
    \definition[道,条]{s.}{arco-íris}
  \end{Phonetics}
\end{Entry}

\begin{Entry}{彩票}{11,11}{⼺、⽰}
  \begin{Phonetics}{彩票}{cai3piao4}[][HSK 5]
    \definition[张,注]{s.}{bilhete de loteria; um título com números, vendido pelo valor de face; após o sorteio, o portador do bilhete premiado pode reivindicar o prêmio de acordo com o regulamento}
  \end{Phonetics}
\end{Entry}

\begin{Entry}{彩霞}{11,17}{⼺、⾬}
  \begin{Phonetics}{彩霞}{cai3xia2}[][HSK 7-9]
    \definition[片]{s.}{nuvens rosadas (ou cor-de-rosa) | nuvens tingidas com tons de pôr do sol; nuvens coloridas}
  \end{Phonetics}
\end{Entry}

%%%%%%%%%% 彬 %%%%%%%%%%
\subsection*{彬}

\begin{Entry}{彬}{11}{⼺}
  \begin{Phonetics}{彬}{bin1}
    \definition*{s.}{Sobrenome: Bin}
    \definition{adj.}{Literário: fino; elegante}
  \end{Phonetics}
\end{Entry}

\begin{Entry}{彬彬有礼}{11,11,6,5}{⼺、⼺、⽉、⽰}
  \begin{Phonetics}{彬彬有礼}{bin1bin1-you3li3}[][HSK 7-9]
    \definition{expr.}{refinado e cortês; urbano}
  \end{Phonetics}
\end{Entry}

%%%%%%%%%% 影 %%%%%%%%%%
\subsection*{影}

\begin{Entry}{影}{15}{⼺}
  \begin{Phonetics}{影}{ying3}
    \definition*{s.}{Sobrenome: Ying}
    \definition{s.}{sombra | reflexão; imagem | traço; sinal; impressão vaga | fotografia; imagem | filme | jogo de sombras; pantomima de sombra}
    \definition{v.}{(dialeto) esconder; ocultar | copiar; rastrear | fotocopiar}
  \end{Phonetics}
\end{Entry}

\begin{Entry}{影子}{15,3}{⼺、⼦}
  \begin{Phonetics}{影子}{ying3zi5}[][HSK 4]
    \definition[个,片]{s.}{sombra; imagem projetada por um objeto, etc., que bloqueia a luz | reflexão; reflexo; imagem de um objeto, etc., conforme aparece em um refletor, como um espelho, uma superfície de água, etc. | sinal; vestígio; vaga impressão}
  \end{Phonetics}
\end{Entry}

\begin{Entry}{影片}{15,4}{⼺、⽚}
  \begin{Phonetics}{影片}{ying3 pian4}[][HSK 2]
    \definition[部,盘,盒,卷]{s.}{filme; imagem | filme; película usada para reproduzir filmes}
  \end{Phonetics}
\end{Entry}

\begin{Entry}{影视}{15,8}{⼺、⾒}
  \begin{Phonetics}{影视}{ying3 shi4}[][HSK 3]
    \definition{s.}{cinema e televisão combinados; denominação conjunta para cinema e TV}
  \end{Phonetics}
\end{Entry}

\begin{Entry}{影响}{15,9}{⼺、⼝}
  \begin{Phonetics}{影响}{ying3xiang3}[][HSK 2]
    \definition{s.}{efeito; influência; efeitos sobre pessoas ou coisas}
    \definition{v.}{afetar; influenciar; influência sobre os pensamentos ou ações dos outros}
  \end{Phonetics}
\end{Entry}

\begin{Entry}{影响力}{15,9,2}{⼺、⼝、⼒}
  \begin{Phonetics}{影响力}{ying3 xiang3 li4}[][HSK 6]
    \definition{s.}{impacto | influência}
  \end{Phonetics}
\end{Entry}

\begin{Entry}{影星}{15,9}{⼺、⽇}
  \begin{Phonetics}{影星}{ying3 xing1}[][HSK 6]
    \definition{s.}{estrela de cinema}
  \end{Phonetics}
\end{Entry}

\begin{Entry}{影迷}{15,9}{⼺、⾡}
  \begin{Phonetics}{影迷}{ying3 mi2}[][HSK 6]
    \definition[个,名,位]{s.}{fã de cinema; entusiasta de cinema; pessoas viciadas em assistir filmes}
  \end{Phonetics}
\end{Entry}

\begin{Entry}{影像}{15,13}{⼺、⼈}
  \begin{Phonetics}{影像}{ying3xiang4}
    \definition{s.}{imagem}
  \end{Phonetics}
\end{Entry}

%%%%% EOF %%%%%

