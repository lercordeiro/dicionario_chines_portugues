%%%
%%% Radical "⾥"
%%%

\section*{Radical 166: ``⾥''}\addcontentsline{toc}{section}{Radical 166: ⾥}

\begin{Entry}{里}{7}{⾥}[Kangxi 166]
  \begin{Phonetics}{里}{li3}[][HSK 1]
    \definition*{s.}{Sobrenome Li}
    \definition{clas.}{li, uma unidade chinesa de comprimento (= 1/2 quilômetro)}
    \definition{s.}{forro; revestimento; interior; parte de trás do tecido | interno; dentro; no interior | vizinhança; vizinhos | cidade natal; local de origem}
  \end{Phonetics}
\end{Entry}

\begin{Entry}{里头}{7,5}{⾥、⼤}
  \begin{Phonetics}{里头}{li3 tou5}[][HSK 2]
    \definition{s.}{dentro}
  \end{Phonetics}
\end{Entry}

\begin{Entry}{里边}{7,5}{⾥、⾡}
  \begin{Phonetics}{里边}{li3 bian5}[][HSK 1]
    \definition{s.}{em; dentro; no interior}
  \end{Phonetics}
\end{Entry}

\begin{Entry}{里面}{7,9}{⾥、⾯}
  \begin{Phonetics}{里面}{li3 mian4}[][HSK 3]
    \definition{s.}{dentro; interior}
  \end{Phonetics}
\end{Entry}

\begin{Entry}{里斯本}{7,12,5}{⾥、⽄、⽊}
  \begin{Phonetics}{里斯本}{li3si1ben3}
    \definition*{s.}{Lisboa}
  \end{Phonetics}
\end{Entry}

\begin{Entry}{里斯本大学}{7,12,5,3,8}{⾥、⽄、⽊、⼤、⼦}
  \begin{Phonetics}{里斯本大学}{li3si1ben3 da4xue2}
    \definition*{s.}{Universidade de Lisboa}
  \end{Phonetics}
\end{Entry}

\begin{Entry}{重}{9}{⾥}
  \begin{Phonetics}{重}{chong2}
    \definition*{s.}{Sobrenome Chong}
    \definition{adv.}{novamente; mais uma vez}
    \definition{clas.}{usado para camadas}
    \definition{v.}{repetir; duplicar}
  \end{Phonetics}
  \begin{Phonetics}{重}{zhong4}[][HSK 1,3]
    \definition{adj.}{pesado; densidade elevada | profundo; sério; grau profundo | importante; significativo | discreto; prudente | considerável em quantidade ou valor}
    \definition[斤,公,斤,吨]{s.}{peso}
    \definition{v.}{colocar (colocar, pôr) ênfase em; dar valor a; atribuir importância a}
  \end{Phonetics}
\end{Entry}

\begin{Entry}{重大}{9,3}{⾥、⼤}
  \begin{Phonetics}{重大}{zhong4da4}[][HSK 3]
    \definition{adj.}{excelente; importante; significativo; de grande importância}
  \end{Phonetics}
\end{Entry}

\begin{Entry}{重申}{9,5}{⾥、⽥}
  \begin{Phonetics}{重申}{chong2shen1}[][HSK 7-9]
    \definition{v.}{reafirmar; reiterar}[他重申了自己对古巴的承诺。===Ele reiterou seu compromisso com Cuba.]
  \end{Phonetics}
\end{Entry}

\begin{Entry}{重合}{9,6}{⾥、⼝}
  \begin{Phonetics}{重合}{chong2he2}[][HSK 7-9]
    \definition{s.}{coincidir; sobrepor-se}[两个假期的时间重合了。===Os dois feriados se sobrepõem.]
  \end{Phonetics}
\end{Entry}

\begin{Entry}{重阳节}{9,6,5}{⾥、⾩、⾋}
  \begin{Phonetics}{重阳节}{chong2yang2jie2}
    \definition*{s.}{Festa do Duplo Nove, Festival Yang, dia de subir aos lugares mais altos para evitar calamidades e dia do culto aos antepassados (9º dia do nono mês lunar)}
  \end{Phonetics}
\end{Entry}

\begin{Entry}{重返}{9,7}{⾥、⾡}
  \begin{Phonetics}{重返}{chong2fan3}[][HSK 7-9]
    \definition{v.}{retornar; voltar; de volta ao lugar original}
  \end{Phonetics}
\end{Entry}

\begin{Entry}{重建}{9,8}{⾥、⼵}
  \begin{Phonetics}{重建}{chong2 jian4}[][HSK 6]
    \definition{s.}{restabelecimento; reconstrução}
    \definition{v.}{reconstruir; reconstruir; restabelecer; reabilitar}
  \end{Phonetics}
\end{Entry}

\begin{Entry}{重现}{9,8}{⾥、⾒}
  \begin{Phonetics}{重现}{chong2xian4}[][HSK 7-9]
    \definition{v.}{reaparecer | reproduzir}
  \end{Phonetics}
\end{Entry}

\begin{Entry}{重组}{9,8}{⾥、⽷}
  \begin{Phonetics}{重组}{chong2 zu3}[][HSK 6]
    \definition{v.}{reestruturar; reorganizar; remanejar}
  \end{Phonetics}
\end{Entry}

\begin{Entry}{重视}{9,8}{⾥、⾒}
  \begin{Phonetics}{重视}{zhong4shi4}[][HSK 2]
    \definition{v.}{valorizar; dar peso a; atribuir importância a; prestar atenção a; considerar a virtude ou o talento de uma pessoa ou o papel de algo como importante e levá-lo a sério}
  \end{Phonetics}
\end{Entry}

\begin{Entry}{重复}{9,9}{⾥、⼢}
  \begin{Phonetics}{重复}{chong2fu4}[][HSK 2]
    \definition{v.}{repetir; iterar; duplicar; reduplicar | fazer algo novamente; repetir as mesmas palavras, fazer as mesmas coisas}
  \end{Phonetics}
\end{Entry}

\begin{Entry}{重点}{9,9}{⾥、⽕}
  \begin{Phonetics}{重点}{chong2dian3}
    \definition[个]{adj./adv./s.}{nota principal; ponto-chave; ponto focal; ênfase}
  \end{Phonetics}
  \begin{Phonetics}{重点}{zhong4dian3}[][HSK 2]
    \definition[个]{s.}{nota principal; ponto-chave; ponto}
  \end{Phonetics}
\end{Entry}

\begin{Entry}{重要}{9,9}{⾥、⾑}
  \begin{Phonetics}{重要}{zhong4yao4}[][HSK 1]
    \definition{adj.}{importante; significativo; relevante; de grande importância, função e impacto}
  \end{Phonetics}
\end{Entry}

\begin{Entry}{重重}{9,9}{⾥、⾥}
  \begin{Phonetics}{重重}{chong2chong2}
    \definition{adv.}{camada após camada | um após o outro}
  \end{Phonetics}
  \begin{Phonetics}{重重}{zhong4zhong4}
    \definition{adv.}{fortemente | severamente}
  \end{Phonetics}
\end{Entry}

\begin{Entry}{重逢}{9,10}{⾥、⾡}
  \begin{Phonetics}{重逢}{chong2feng2}
    \definition{s.}{reunião}
    \definition{v.}{encontrar-se novamente | reunir-se}
  \end{Phonetics}
\end{Entry}

\begin{Entry}{重量}{9,12}{⾥、⾥}
  \begin{Phonetics}{重量}{zhong4liang4}[][HSK 4]
    \definition[个]{s.}{peso; a magnitude da força da gravidade em um objeto}
  \end{Phonetics}
\end{Entry}

\begin{Entry}{重叠}{9,13}{⾥、⼜}
  \begin{Phonetics}{重叠}{chong2die2}[][HSK 7-9]
    \definition{s.}{reduplicação; gramaticalmente, refere-se ao uso da mesma palavra ou frase duas vezes}
    \definition{v.}{sobrepor; por um em cima do outro; colocar as mesmas coisas juntas camada por camada}
  \end{Phonetics}
\end{Entry}

\begin{Entry}{重新}{9,13}{⾥、⽄}
  \begin{Phonetics}{重新}{chong2xin1}[][HSK 2]
    \definition{adv.}{novamente; de novo; significa repetir uma ação ou comportamento já realizado | indica que se deve começar do início (mudança de método ou conteúdo)}
  \end{Phonetics}
\end{Entry}

\begin{Entry}{重播}{9,15}{⾥、⼿}
  \begin{Phonetics}{重播}{chong2bo1}[][HSK 7-9]
    \definition{v.}{retransmitir (um programa de rádio ou TV) | Agricultura: ressemear (o mesmo campo)}
  \end{Phonetics}
\end{Entry}

\begin{Entry}{野}{11}{⾥}
  \begin{Phonetics}{野}{ye3}[][HSK 6]
    \definition*{s.}{Sobrenome Ye}
    \definition{adj.}{(de plantas ou animais) selvagem; incultivado; não domesticado; indomável (opp. 家) | rude; áspero | desenfreado; abandonado; indisciplinado | ilícito; sem licença}
    \definition{s.}{espaço aberto; o aberto | limite; fronteira | não está no poder; fora do cargo}
  \seealsoref{家}{jia1}
  \end{Phonetics}
\end{Entry}

\begin{Entry}{野生}{11,5}{⾥、⽣}
  \begin{Phonetics}{野生}{ye3 sheng1}[][HSK 6]
    \definition{adj.}{selvagem; não cultivado; não domesticado}
  \end{Phonetics}
\end{Entry}

\begin{Entry}{量}{12}{⾥}
  \begin{Phonetics}{量}{liang2}[][HSK 4]
    \definition{v.}{medir | estimar; dimensionar}
  \end{Phonetics}
  \begin{Phonetics}{量}{liang4}
    \definition{s.}{instrumento de medida; antigamente, o termo se referia a objetos como baldes e litros, que medem o volume | capacidade (para tolerância ou ingestão de alimentos ou bebidas); refere-se ao limite do que pode ser acomodado | quantidade; valor; volume; número}
    \definition{v.}{estimar; medir; pesar}
  \end{Phonetics}
\end{Entry}

%%%%% EOF %%%%%

