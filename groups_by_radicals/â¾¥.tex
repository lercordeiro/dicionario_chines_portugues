%%%
%%% Radical "⾥"
%%%

\section*{Radical 166: ``⾥''}\addcontentsline{toc}{section}{Radical 166: ⾥}

\begin{entry}{里}{7}{⾥}[Kangxi 166]
  \begin{phonetics}{里}{li3}[][HSK 1]
    \definition*{s.}{sobrenome Li}
    \definition{clas.}{li, uma unidade chinesa de comprimento (= 1/2 quilômetro)}
    \definition{s.}{forro; revestimento; interior; parte de trás do tecido | interno; dentro; no interior | vizinhança; vizinhos | cidade natal; local de origem}
  \end{phonetics}
\end{entry}

\begin{entry}{里头}{7,5}{⾥、⼤}
  \begin{phonetics}{里头}{li3 tou5}[][HSK 2]
    \definition{s.}{dentro}
  \end{phonetics}
\end{entry}

\begin{entry}{里边}{7,5}{⾥、⾡}
  \begin{phonetics}{里边}{li3 bian5}[][HSK 1]
    \definition{s.}{em; dentro; no interior}
  \end{phonetics}
\end{entry}

\begin{entry}{里面}{7,9}{⾥、⾯}
  \begin{phonetics}{里面}{li3 mian4}[][HSK 3]
    \definition{s.}{dentro; interior}
  \end{phonetics}
\end{entry}

\begin{entry}{里斯本}{7,12,5}{⾥、⽄、⽊}
  \begin{phonetics}{里斯本}{li3si1ben3}
    \definition*{s.}{Lisboa}
  \end{phonetics}
\end{entry}

\begin{entry}{里斯本大学}{7,12,5,3,8}{⾥、⽄、⽊、⼤、⼦}
  \begin{phonetics}{里斯本大学}{li3si1ben3 da4xue2}
    \definition*{s.}{Universidade de Lisboa}
  \end{phonetics}
\end{entry}

\begin{entry}{重}{9}{⾥}
  \begin{phonetics}{重}{chong2}
    \definition*{s.}{sobrenome Chong}
    \definition{adv.}{novamente; mais uma vez}
    \definition{clas.}{usado para camadas}
    \definition{v.}{repetir; duplicar}
  \end{phonetics}
  \begin{phonetics}{重}{zhong4}[][HSK 1,3]
    \definition{adj.}{pesado; densidade elevada | profundo; sério; grau profundo | importante; significativo | discreto; prudente | considerável em quantidade ou valor}
    \definition[斤,公,斤,吨]{s.}{peso}
    \definition{v.}{enfatizar; valorizar; dar importância a}
  \end{phonetics}
\end{entry}

\begin{entry}{重大}{9,3}{⾥、⼤}
  \begin{phonetics}{重大}{zhong4da4}[][HSK 3]
    \definition{adj.}{grande; importante; significativo; de grande importância}
  \end{phonetics}
\end{entry}

\begin{entry}{重阳节}{9,6,5}{⾥、⾩、⾋}
  \begin{phonetics}{重阳节}{chong2yang2jie2}
    \definition*{s.}{Festa do Duplo Nove, Festival Yang, dia de subir aos lugares mais altos para evitar calamidades e dia do culto aos antepassados (9º dia do nono mês lunar)}
  \end{phonetics}
\end{entry}

\begin{entry}{重视}{9,8}{⾥、⾒}
  \begin{phonetics}{重视}{zhong4shi4}[][HSK 2]
    \definition{v.}{valorizar; dar peso a; atribuir importância a; prestar atenção a; considerar a virtude ou o talento de uma pessoa ou o papel de algo como importante e levá-lo a sério}
  \end{phonetics}
\end{entry}

\begin{entry}{重复}{9,9}{⾥、⼢}
  \begin{phonetics}{重复}{chong2fu4}[][HSK 2]
    \definition{v.}{repetir; iterar; duplicar; reduplicar | fazer algo novamente; repetir as mesmas palavras, fazer as mesmas coisas}
  \end{phonetics}
\end{entry}

\begin{entry}{重点}{9,9}{⾥、⽕}
  \begin{phonetics}{重点}{chong2dian3}
    \definition[个]{adj./adv./s.}{nota principal; ponto-chave; ponto focal; ênfase}
  \end{phonetics}
  \begin{phonetics}{重点}{zhong4dian3}[][HSK 2]
    \definition[个]{s.}{nota principal; ponto-chave; ponto}
  \end{phonetics}
\end{entry}

\begin{entry}{重要}{9,9}{⾥、⾑}
  \begin{phonetics}{重要}{zhong4yao4}[][HSK 1]
    \definition{adj.}{importante; significativo; relevante; de grande importância, função e impacto}
  \end{phonetics}
\end{entry}

\begin{entry}{重重}{9,9}{⾥、⾥}
  \begin{phonetics}{重重}{chong2chong2}
    \definition{adv.}{camada após camada | um após o outro}
  \end{phonetics}
  \begin{phonetics}{重重}{zhong4zhong4}
    \definition{adv.}{fortemente | severamente}
  \end{phonetics}
\end{entry}

\begin{entry}{重逢}{9,10}{⾥、⾡}
  \begin{phonetics}{重逢}{chong2feng2}
    \definition{s.}{reunião}
    \definition{v.}{encontrar-se novamente | reunir-se}
  \end{phonetics}
\end{entry}

\begin{entry}{重量}{9,12}{⾥、⾥}
  \begin{phonetics}{重量}{zhong4liang4}[][HSK 4]
    \definition[个]{s.}{peso; a magnitude da força da gravidade em um objeto}
  \end{phonetics}
\end{entry}

\begin{entry}{重新}{9,13}{⾥、⽄}
  \begin{phonetics}{重新}{chong2xin1}[][HSK 2]
    \definition{adv.}{novamente; de novo; significa repetir uma ação ou comportamento já realizado | indica que se deve começar do início (mudança de método ou conteúdo)}
  \end{phonetics}
\end{entry}

\begin{entry}{野}{11}{⾥}
  \begin{phonetics}{野}{ye3}
    \definition{adj.}{selvagem | rude}
    \definition{s.}{campo | espaço aberto | limite}
  \end{phonetics}
\end{entry}

\begin{entry}{野生}{11,5}{⾥、⽣}
  \begin{phonetics}{野生}{ye3sheng1}
    \definition{adj.}{selvagem | não domesticado}
  \end{phonetics}
\end{entry}

\begin{entry}{量}{12}{⾥}
  \begin{phonetics}{量}{liang2}[][HSK 4]
    \definition{v.}{medir | estimar; dimensionar}
  \end{phonetics}
  \begin{phonetics}{量}{liang4}
    \definition{s.}{instrumento de medida; antigamente, o termo se referia a objetos como baldes e litros, que medem o volume | capacidade (para tolerância ou ingestão de alimentos ou bebidas); refere-se ao limite do que pode ser acomodado | quantidade; valor; volume; número}
    \definition{v.}{estimar; medir; pesar}
  \end{phonetics}
\end{entry}

%%%%% EOF %%%%%

