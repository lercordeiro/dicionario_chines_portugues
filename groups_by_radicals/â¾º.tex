%%%
%%% Radical "⾺"
%%%

\section*{Radical 187: ``⾺'' (马)}\addcontentsline{toc}{section}{Radical 187: ⾺、马}

\begin{entry}{马}{3}{⾺}[Kangxi 187]
  \begin{phonetics}{马}{ma3}[][HSK 3]
    \definition*{s.}{Sobrenome Ma}
    \definition{adj.}{grande; extenso; amplo}
    \definition[匹,头,只,群]{s.}{cavalo | a peça do cavalo no xadrez chinês}
  \end{phonetics}
\end{entry}

\begin{entry}{马上}{3,3}{⾺、⼀}
  \begin{phonetics}{马上}{ma3shang4}[][HSK 1]
    \definition{adv.}{imediatamente; de uma só vez; em um piscar de olhos | em breve; em um futuro próximo; em um curto espaço de tempo}
  \end{phonetics}
\end{entry}

\begin{entry}{马马虎虎}{3,3,8,8}{⾺、⾺、⾌、⾌}
  \begin{phonetics}{马马虎虎}{ma3ma3hu3hu3}
    \definition{adj.}{descuidado | casual | tolerável | vago | mais ou menos}
  \end{phonetics}
\end{entry}

\begin{entry}{马车}{3,4}{⾺、⾞}
  \begin{phonetics}{马车}{ma3 che1}[][HSK 6]
    \definition[辆]{s.}{carruagem (puxada por cavalo); carroça; charrete}
  \end{phonetics}
\end{entry}

\begin{entry}{马耳他}{3,6,5}{⾺、⽿、⼈}
  \begin{phonetics}{马耳他}{ma3'er3ta1}
    \definition*{s.}{Malta}
  \end{phonetics}
\end{entry}

\begin{entry}{马克思列宁主义}{3,7,9,6,5,5,3}{⾺、⼗、⼼、⼑、⼧、⼂、⼂}
  \begin{phonetics}{马克思列宁主义}{ma3ke4si1 lie4ning2 zhu3yi4}
    \definition*{s.}{Marxismo-Leninismo}
  \end{phonetics}
\end{entry}

\begin{entry}{马尾}{3,7}{⾺、⼫}
  \begin{phonetics}{马尾}{ma3wei3}
    \definition{s.}{(penteado) rabo de cavalo | cauda de cavalo}
  \end{phonetics}
\end{entry}

\begin{entry}{马路}{3,13}{⾺、⾜}
  \begin{phonetics}{马路}{ma3lu4}[][HSK 1]
    \definition[条]{s.}{estrada; rua; avenida; estradas largas e planas para o tráfego de carros e cavalos nas cidades ou nos subúrbios}
  \end{phonetics}
\end{entry}

\begin{entry}{驱}{7}{⾺}
  \begin{phonetics}{驱}{qu1}
    \definition{v.}{expulsar | repelir}
  \end{phonetics}
\end{entry}

\begin{entry}{驴}{7}{⾺}
  \begin{phonetics}{驴}{lv2}
    \definition[头]{s.}{burro | asno | jumento | jegue}
  \end{phonetics}
\end{entry}

\begin{entry}{驻}{8}{⾺}
  \begin{phonetics}{驻}{zhu4}[][HSK 6]
    \definition{v.}{parar; ficar | estar estacionado; acampar; (tropas ou pessoal) viver no local onde desempenham suas funções; (organização) estar localizada em um determinado lugar}
  \end{phonetics}
\end{entry}

\begin{entry}{驻军}{8,6}{⾺、⼍}
  \begin{phonetics}{驻军}{zhu4jun1}
    \definition{s.}{guarnição}
    \definition{v.}{guarcener ou prover uma tropa}
  \end{phonetics}
\end{entry}

\begin{entry}{驾}{8}{⾺}
  \begin{phonetics}{驾}{jia4}
    \definition*{s.}{Sobrenome Jia}
    \definition{s.}{carruagem do imperador; refere-se especificamente ao carro do imperador, referindo-se ao imperador | referindo-se a um veículo, usado como um termo respeitoso para uma pessoa}
    \definition{v.}{atrelar; puxar (uma carroça, etc.) | dirigir (um veículo); pilotar (um avião); velejar (um barco) | montar; cavalgar}
  \end{phonetics}
\end{entry}

\begin{entry}{驾驶}{8,8}{⾺、⾺}
  \begin{phonetics}{驾驶}{jia4shi3}[][HSK 5]
    \definition{v.}{dirigir; pilotar; conduzir; guiar; operar (um carro, navio, avião, trator, etc.) para fazê-lo mover}
  \end{phonetics}
\end{entry}

\begin{entry}{驾照}{8,13}{⾺、⽕}
  \begin{phonetics}{驾照}{jia4 zhao4}[][HSK 5]
    \definition[本,张]{s.}{carteira de motorista}
  \end{phonetics}
\end{entry}

\begin{entry}{骂}{9}{⾺}
  \begin{phonetics}{骂}{ma4}[][HSK 5]
    \definition{v.}{abusar; xingar; insultar; insultar alguém com palavras grosseiras ou maliciosas | repreender; censurar; condenar}
  \end{phonetics}
\end{entry}

\begin{entry}{骂名}{9,6}{⾺、⼝}
  \begin{phonetics}{骂名}{ma4ming2}
    \definition{s.}{infâmia}
  \end{phonetics}
\end{entry}

\begin{entry}{骂街}{9,12}{⾺、⾏}
  \begin{phonetics}{骂街}{ma4jie1}
    \definition{v.}{gritar abusos na rua}
  \end{phonetics}
\end{entry}

\begin{entry}{骄}{9}{⾺}
  \begin{phonetics}{骄}{jiao1}
    \definition{adj.}{orgulhoso; arrogante; vaidoso | Literário: feroz; intenso; forte; violento}
  \end{phonetics}
\end{entry}

\begin{entry}{骄傲}{9,12}{⾺、⼈}
  \begin{phonetics}{骄傲}{jiao1'ao4}[][HSK 6]
    \definition{adj.}{arrogante; vaidoso; orgulhoso}
    \definition{s.}{orgulho; pessoas ou coisas das quais se orgulhar}
  \end{phonetics}
\end{entry}

\begin{entry}{骆}{9}{⾺}
  \begin{phonetics}{骆}{luo4}
    \definition*{s.}{Sobrenome Luo}
    \definition[只]{s.}{Arcaico: um cavalo branco com crina preta, mencionado em antigos livros chineses}
  \end{phonetics}
\end{entry}

\begin{entry}{骆驼}{9,8}{⾺、⾺}
  \begin{phonetics}{骆驼}{luo4tuo5}
    \definition[峰,匹,头]{s.}{camelo | coloquial: cabeça-dura, idiota}
  \end{phonetics}
\end{entry}

\begin{entry}{骑}{11}{⾺}
  \begin{phonetics}{骑}{qi2}[][HSK 2]
    \definition{s.}{cavalos ou outros animais para montaria | cavalaria; cavaleiro, também se refere genericamente a qualquer pessoa que monta a cavalo}
    \definition{v.}{montar (um animal ou bicicleta); sentar-se na parte de trás de | montar; abranger ambos os lados}
  \end{phonetics}
\end{entry}

\begin{entry}{骑车}{11,4}{⾺、⾞}
  \begin{phonetics}{骑车}{qi2 che1}[][HSK 2]
    \definition{v.}{andar de bicicleta; pedalar}
  \end{phonetics}
\end{entry}

\begin{entry}{骗}{12}{⾺}
  \begin{phonetics}{骗}{pian4}[][HSK 5]
    \definition{v.}{enganar; trapacear; iludir; ludibriar; usar mentiras ou meios fraudulentos para fazer alguém acreditar ou ser enganado | enganar; fraudar | montar (um cavalo); balançar (ou saltar) para a sela}
  \end{phonetics}
\end{entry}

\begin{entry}{骗子}{12,3}{⾺、⼦}
  \begin{phonetics}{骗子}{pian4 zi5}[][HSK 5]
    \definition[个]{s.}{trapaceiro; vigarista; fraudador; impostor; golpista; pessoa que obtém bens de forma fraudulenta}
  \end{phonetics}
\end{entry}

\begin{entry}{骚}{12}{⾺}
  \begin{phonetics}{骚}{sao1}
    \definition*{s.}{Abreviação de Li Sao (Encontrando a Tristeza), um poema do poeta e estadista do século IV a.C. Qu Yuan (屈原)}
    \definition{adj.}{coquete; (de uma mulher) lasciva | masculino (de alguns animais domésticos)}
    \definition{s.}{escritos literários; geralmente se refere à poesia | o cheiro de urina; mau cheiro}
    \definition{v.}{perturbar}
  \seealsoref{屈原}{qu1yuan2}
  \end{phonetics}
\end{entry}

\begin{entry}{骚乱}{12,7}{⾺、⼄}
  \begin{phonetics}{骚乱}{sao1luan4}
    \definition{s.}{rebelião | perturbação | tumulto}
    \definition{v.}{criar um distúrbio}
  \end{phonetics}
\end{entry}

%%%%% EOF %%%%%

