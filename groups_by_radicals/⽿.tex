%%%
%%% Radical "⽿"
%%%

\section*{Radical 128: ``⽿''}\addcontentsline{toc}{section}{Radical 128: ⽿}

\begin{entry}{耳}{6}{⽿}
  \begin{phonetics}{耳}{er3}
    \definition*{s.}{sobrenome Er}
    \definition{part.}{(clássico) somente; apenas}
    \definition{s.}{orelha | coisa parecida com uma orelha | em ambos os lados; lado | orelha de um utensílio}
  \end{phonetics}
\end{entry}

\begin{entry}{耳朵}{6,6}{⽿、⽊}
  \begin{phonetics}{耳朵}{er3duo5}[][HSK 5]
    \definition[双,只,个,对]{s.}{orelha; ouvido; órgão da audição e do equilíbrio}
  \end{phonetics}
\end{entry}

\begin{entry}{耳机}{6,6}{⽿、⽊}
  \begin{phonetics}{耳机}{er3 ji1}[][HSK 4]
    \definition[副,个,对]{s.}{fone de ouvido; receptor (de telefone); dispositivos que permitem que uma pessoa ouça sons sozinha, como ouvir música, histórias, chamadas telefônicas etc., usados na cabeça ou inseridos nos ouvidos}
  \end{phonetics}
\end{entry}

\begin{entry}{耽}{10}{⽿}
  \begin{phonetics}{耽}{dan1}
    \definition*{s.}{sobrenome Dan}
    \definition{v.}{atrasar | (literário) abandonar-se a; entregar-se a}
  \end{phonetics}
\end{entry}

\begin{entry}{耽心}{10,4}{⽿、⼼}
  \begin{phonetics}{耽心}{dan1xin1}
    \variantof{担心}
  \end{phonetics}
\end{entry}

\begin{entry}{聊天}{11,4}{⽿、⼤}
  \begin{phonetics}{聊天}{liao2tian1}
    \definition{v.+compl.}{papear | bater papo}
  \end{phonetics}
\end{entry}

\begin{entry}{职工}{11,3}{⽿、⼯}
  \begin{phonetics}{职工}{zhi2 gong1}[][HSK 3]
    \definition[个,位,名,些]{s.}{pessoal; trabalhadores e funcionários administrativos}
  \end{phonetics}
\end{entry}

\begin{entry}{职业}{11,5}{⽿、⼀}
  \begin{phonetics}{职业}{zhi2ye4}[][HSK 3]
    \definition{adj.}{profissional; não amador}
    \definition[种,份,个]{s.}{ocupação; profissão; vocação; o trabalho que um indivíduo realiza na sociedade como sua principal fonte de subsistência}
  \end{phonetics}
\end{entry}

\begin{entry}{职务}{11,5}{⽿、⼒}
  \begin{phonetics}{职务}{zhi2wu4}[][HSK 5]
    \definition{s.}{cargo; posto; deveres; função; funções que devem ser desempenhadas de acordo com as especificações do cargo}
  \end{phonetics}
\end{entry}

\begin{entry}{职位}{11,7}{⽿、⼈}
  \begin{phonetics}{职位}{zhi2wei4}[][HSK 5]
    \definition[个]{s.}{posto; posição; cargo que exerce determinadas funções em órgãos ou entidades}
  \end{phonetics}
\end{entry}

\begin{entry}{职员}{11,7}{⽿、⼝}
  \begin{phonetics}{职员}{zhi2yuan2}
    \definition[个,位]{s.}{empregado | trabalhador de escritório | membro da equipe}
  \end{phonetics}
\end{entry}

\begin{entry}{职能}{11,10}{⽿、⾁}
  \begin{phonetics}{职能}{zhi2neng2}[][HSK 5]
    \definition{s.}{função; funções ou papéis que as organizações, instituições, etc. devem desempenhar}
  \end{phonetics}
\end{entry}

\begin{entry}{联合}{12,6}{⽿、⼝}
  \begin{phonetics}{联合}{lian2he2}[][HSK 3]
    \definition{adj.}{conjunto; unido; federal; combinado}
    \definition{s.}{aliado; união; aliança; conectar-se ou unir-se para agir em conjunto}
  \end{phonetics}
\end{entry}

\begin{entry}{联合会}{12,6,6}{⽿、⼝、⼈}
  \begin{phonetics}{联合会}{lian2he2hui4}
    \definition{s.}{federação}
  \end{phonetics}
\end{entry}

\begin{entry}{联合国}{12,6,8}{⽿、⼝、⼞}
  \begin{phonetics}{联合国}{lian2 he2 guo2}[][HSK 3]
    \definition*{s.}{Nações Unidas; Organização internacional fundada em 1945, após o fim da Segunda Guerra Mundial, com sede em Nova Iorque, Estados Unidos ; as suas principais instituições são a Assembleia Geral, o Conselho de Segurança, o Conselho Econômico e Social e o Secretariado; de acordo com a Carta das Nações Unidas, os seus principais objetivos são manter a paz e a segurança internacionais, desenvolver relações amigáveis entre os países e promover a cooperação internacional nas áreas econômica e cultural}
  \end{phonetics}
\end{entry}

\begin{entry}{联系}{12,7}{⽿、⽷}
  \begin{phonetics}{联系}{lian2xi4}[][HSK 3]
    \definition[个,种,层]{s.}{relacionamento; relacionamento entre duas coisas}
    \definition{v.}{entrar em contato; contatar; comunicar-se com alguém por telefone, e-mail ou carta | agendar; entrar em contato com; estabelecer algum tipo de relação com a outra parte | relacionar; combinar; integrar}
  \end{phonetics}
\end{entry}

\begin{entry}{联络}{12,9}{⽿、⽷}
  \begin{phonetics}{联络}{lian2luo4}[][HSK 5]
    \definition{v.}{entrar em contato; comunicar-se; entrar em contato com}
  \end{phonetics}
\end{entry}

\begin{entry}{联想}{12,13}{⽿、⼼}
  \begin{phonetics}{联想}{lian2xiang3}[][HSK 5]
    \definition*{s.}{Lenovo (empresa)}
    \definition{v.}{associar-se a; estabelecer uma conexão mental; lembrar-se de algo; lembrar-se de outras pessoas ou coisas relacionadas devido a alguém ou algo; evocar outros conceitos relacionados devido a um determinado conceito |}
  \end{phonetics}
\end{entry}

\begin{entry}{聚}{14}{⽿}
  \begin{phonetics}{聚}{ju4}[][HSK 4]
    \definition{v.}{reunir-se; juntar-se}
  \end{phonetics}
\end{entry}

\begin{entry}{聚会}{14,6}{⽿、⼈}
  \begin{phonetics}{聚会}{ju4hui4}[][HSK 4]
    \definition[个,次]{s.}{reunião; encontro; confraternização; festa}
    \definition{v.}{encontrar-se; reunir-se}
  \end{phonetics}
\end{entry}

\begin{entry}{聚散}{14,12}{⽿、⽁}
  \begin{phonetics}{聚散}{ju4san4}
    \definition{s.}{juntos e separados | agregação e dissipação}
  \end{phonetics}
\end{entry}

\begin{entry}{聪}{15}{⽿}
  \begin{phonetics}{聪}{cong1}
    \definition{adj.}{audição aguçada | brilhante; inteligente; esperto | perspicaz}
    \definition{s.}{(literário) faculdades auditivas}
  \end{phonetics}
\end{entry}

\begin{entry}{聪明}{15,8}{⽿、⽇}
  \begin{phonetics}{聪明}{cong1ming5}[][HSK 5]
    \definition{adj.}{brilhante; esperto; inteligente; intelecto bem desenvolvido com boa memória e capacidade de compreensão}
  \end{phonetics}
\end{entry}

\begin{entry}{聪慧}{15,15}{⽿、⼼}
  \begin{phonetics}{聪慧}{cong1hui4}
    \definition{adj.}{inteligente | brilhante}
  \end{phonetics}
\end{entry}

%%%%% EOF %%%%%

