%%%
%%% Radical "⽜"
%%%

\section*{Radical 93: ``⽜'' (牜、⺧)}\addcontentsline{toc}{section}{Radical 93: ⽜、牜、⺧}

\begin{Entry}{牛}{4}{⽜}[Kangxi 93]
  \begin{Phonetics}{牛}{niu2}[][HSK 3,5]
    \definition*{s.}{Sobrenome Niu}
    \definition{adj.}{muito capaz ou bom; descreve pessoas ou coisas como sendo muito capazes, muito competentes | teimoso; arrogante; descreve uma pessoa que é muito orgulhosa ou muito insistente em suas opiniões, difícil de mudar}
    \definition{clas.}{Newton (medida física de força)}
    \definition[头]{s.}{gado; boi | niu (nona das vinte e oito constelações em que a esfera celeste foi dividida, consistindo de seis estrelas, três em Áries e três em Sagitário)}
  \end{Phonetics}
\end{Entry}

\begin{Entry}{牛人}{4,2}{⽜、⼈}
  \begin{Phonetics}{牛人}{niu2ren2}
    \definition{s.}{(coloquial) o cara | verdadeiro especialista | \emph{badass}}
  \end{Phonetics}
\end{Entry}

\begin{Entry}{牛仔裤}{4,5,12}{⽜、⼈、⾐}
  \begin{Phonetics}{牛仔裤}{niu2zai3ku4}[][HSK 5]
    \definition[条]{s.}{calças jeans; calças geralmente feitas de tecido jeans azul grosso}
  \end{Phonetics}
\end{Entry}

\begin{Entry}{牛奶}{4,5}{⽜、⼥}
  \begin{Phonetics}{牛奶}{niu2nai3}[][HSK 1]
    \definition[杯,袋,瓶,盒,箱,桶]{s.}{leite}
  \end{Phonetics}
\end{Entry}

\begin{Entry}{牛肉}{4,6}{⽜、⾁}
  \begin{Phonetics}{牛肉}{niu2rou4}
    \definition{s.}{carne de vaca | bife}
  \end{Phonetics}
\end{Entry}

\begin{Entry}{牛郎织女}{4,8,8,3}{⽜、⾢、⽷、⼥}
  \begin{Phonetics}{牛郎织女}{niu2 lang2 zhi1nv3}
    \definition*{s.}{Vaqueiro e Tecelã (personagens de contos folclóricos) | Altair e Vega (estrelas)}[我们这些牛郎织女都恨透了那条无情的“天河”。===Nós, o Vaqueiro e a Tecelã, odiamos a implacável ``Via Láctea''.]
    \definition{s.}{marido e mulher que vivem longe um do outro}
  \end{Phonetics}
\end{Entry}

\begin{Entry}{牛顿}{4,10}{⽜、⾴}
  \begin{Phonetics}{牛顿}{niu2dun4}
    \definition*{s.}{Newton (nome) | N; Newton, unidade de força do SI}
  \end{Phonetics}
\end{Entry}

\begin{Entry}{牦}{8}{⽜}
  \begin{Phonetics}{牦}{mao2}
    \definition[头]{s.}{iaque; boi da Tartária}
  \end{Phonetics}
\end{Entry}

\begin{Entry}{牦牛}{8,4}{⽜、⽜}
  \begin{Phonetics}{牦牛}{mao2niu2}
    \definition{s.}{iaque}
  \end{Phonetics}
\end{Entry}

\begin{Entry}{物}{8}{⽜}
  \begin{Phonetics}{物}{wu4}
    \definition{s.}{coisa; matéria; objeto | mundo exterior distinto de si mesmo; outras pessoas; refere-se a outras pessoas além de si mesmo ou ao ambiente em relação a si mesmo | essência; conteúdo; substância | criatura; criação}
  \end{Phonetics}
\end{Entry}

\begin{Entry}{物业}{8,5}{⽜、⼀}
  \begin{Phonetics}{物业}{wu4ye4}[][HSK 5]
    \definition[处]{s.}{propriedade; gestão de propriedades; gestão patrimonial; administração de imóveis | empresa de administração de imóveis; empresa de gestão imobiliária; empresa de administração de bens imóveis}
  \end{Phonetics}
\end{Entry}

\begin{Entry}{物价}{8,6}{⽜、⼈}
  \begin{Phonetics}{物价}{wu4 jia4}[][HSK 5]
    \definition[个]{s.}{preços das commodities; preços das matérias-primas; preço das mercadorias}
  \end{Phonetics}
\end{Entry}

\begin{Entry}{物质}{8,8}{⽜、⾙}
  \begin{Phonetics}{物质}{wu4zhi4}[][HSK 5]
    \definition[种,类,个]{s.}{matéria; substância; algo que existe além do espírito, que pode ser visto, tocado, cheirado ou detectado por instrumentos científicos | material; meios de subsistência; coisas que permitem às pessoas viver ou viver melhor, como comida, roupas, casas, dinheiro, etc.}
  \end{Phonetics}
\end{Entry}

\begin{Entry}{物品}{8,9}{⽜、⼝}
  \begin{Phonetics}{物品}{wu4 pin3}[][HSK 6]
    \definition[件,个]{s.}{artigos; itens; bens}
  \end{Phonetics}
\end{Entry}

\begin{Entry}{物理}{8,11}{⽜、⽟}
  \begin{Phonetics}{物理}{wu4li3}
    \definition{s.}{física (disciplina)}
  \end{Phonetics}
\end{Entry}

\begin{Entry}{牵}{9}{⽜}
  \begin{Phonetics}{牵}{qian1}[][HSK 6]
    \definition{v.}{conduzir (segurando a mão, o cabresto, etc.); puxar | envolver-se | sentir falta; preocupar-se com | controlar; restringir; ser retido; ser constrangido}
  \end{Phonetics}
\end{Entry}

\begin{Entry}{特}{10}{⽜}
  \begin{Phonetics}{特}{te4}[][HSK 6]
    \definition{adj.}{especial; incomum; particular; excepcional; diferente do geral | especial; solteiro; solitário}
    \definition{adv.}{muito; extremamente | especialmente; para um propósito especial |mas; somente}
    \definition{clas.}{TEX; abreviação para unidades de medida como TEX; a unidade de medida TEX indica a espessura de um fio têxtil através do seu peso}
    \definition{s.}{espião; agente secreto}
  \end{Phonetics}
\end{Entry}

\begin{Entry}{特大}{10,3}{⽜、⼤}
  \begin{Phonetics}{特大}{te4 da4}[][HSK 6]
    \definition{adj.}{especialmente (excepcionalmente) grande; o mais}
  \end{Phonetics}
\end{Entry}

\begin{Entry}{特价}{10,6}{⽜、⼈}
  \begin{Phonetics}{特价}{te4 jia4}[][HSK 4]
    \definition{s.}{oferta especial; preço de barganha; preço especial reduzido}
  \end{Phonetics}
\end{Entry}

\begin{Entry}{特地}{10,6}{⽜、⼟}
  \begin{Phonetics}{特地}{te4 di4}[][HSK 6]
    \definition{adv.}{especialmente; propositalmente; para um propósito especial}
  \end{Phonetics}
\end{Entry}

\begin{Entry}{特有}{10,6}{⽜、⽉}
  \begin{Phonetics}{特有}{te4 you3}[][HSK 5]
    \definition{adj.}{específico; peculiar; característico; único; exclusivo; especial}
  \end{Phonetics}
\end{Entry}

\begin{Entry}{特色}{10,6}{⽜、⾊}
  \begin{Phonetics}{特色}{te4se4}[][HSK 3]
    \definition{s.}{característica; característica distintiva; a cor única, estilo, etc. de um objeto}
  \end{Phonetics}
\end{Entry}

\begin{Entry}{特别}{10,7}{⽜、⼑}
  \begin{Phonetics}{特别}{te4bie2}[][HSK 2]
    \definition{adj.}{especial; particular; fora do comum; diferente dos outros, com características próprias}
    \definition{adv.}{especialmente; particularmente | ainda mais; em particular; frequentemente usado com 是 | especialmente; deliberadamente; para um propósito específico}
  \seealsoref{是}{shi4}
  \end{Phonetics}
\end{Entry}

\begin{Entry}{特别快车}{10,7,7,4}{⽜、⼑、⼼、⾞}
  \begin{Phonetics}{特别快车}{te4bie2 kuai4che1}
    \definition{s.}{trem expresso; expresso; expresso especial; refere-se a trens de passageiros que param em menos estações e têm menor tempo de viagem do que trens expressos diretos}
  \end{Phonetics}
\end{Entry}

\begin{Entry}{特快}{10,7}{⽜、⼼}
  \begin{Phonetics}{特快}{te4 kuai4}[][HSK 6]
    \definition{adj.}{expresso (trem, entrega etc.)}
    \definition{s.}{trem expresso (opp. 普快); abreviação de 特别快车}
  \seealsoref{特别快车}{te4bie2 kuai4che1}
  \end{Phonetics}
\end{Entry}

\begin{Entry}{特技}{10,7}{⽜、⼿}
  \begin{Phonetics}{特技}{te4ji4}
    \definition{s.}{efeito especial | dublê}
  \end{Phonetics}
\end{Entry}

\begin{Entry}{特定}{10,8}{⽜、⼧}
  \begin{Phonetics}{特定}{te4ding4}[][HSK 5]
    \definition{adj.}{específico; especialmente designado | dado; especificado; específico (pessoa, hora, lugar, local, ambiente, etc.)}
  \end{Phonetics}
\end{Entry}

\begin{Entry}{特征}{10,8}{⽜、⼻}
  \begin{Phonetics}{特征}{te4zheng1}[][HSK 4]
    \definition[个,种]{s.}{característica; aparência ou o fenômeno característico de uma pessoa ou coisa que pode ser visto de fora}
  \end{Phonetics}
\end{Entry}

\begin{Entry}{特性}{10,8}{⽜、⼼}
  \begin{Phonetics}{特性}{te4 xing4}[][HSK 5]
    \definition[种,个]{s.}{propriedade específica (ou característica) | característica; sabores | propriedade}
  \end{Phonetics}
\end{Entry}

\begin{Entry}{特点}{10,9}{⽜、⽕}
  \begin{Phonetics}{特点}{te4dian3}[][HSK 2]
    \definition[个,大]{s.}{característica; peculiaridade; traço distintivo; a singularidade de uma pessoa ou coisa}
  \end{Phonetics}
\end{Entry}

\begin{Entry}{特殊}{10,10}{⽜、⽍}
  \begin{Phonetics}{特殊}{te4shu1}[][HSK 4]
    \definition{adj.}{especial; particular; peculiar; excepcional; incomum}
  \end{Phonetics}
\end{Entry}

\begin{Entry}{特意}{10,13}{⽜、⼼}
  \begin{Phonetics}{特意}{te4yi4}[][HSK 6]
    \definition{adv.}{especialmente; para um propósito especial}
  \end{Phonetics}
\end{Entry}

\begin{Entry}{牺}{10}{⽜}
  \begin{Phonetics}{牺}{xi1}
    \definition{s.}{um animal de cor uniforme para sacrifício; sacrifício; gado com pelagem pura usado para sacrifício}
  \end{Phonetics}
\end{Entry}

\begin{Entry}{牺牲}{10,9}{⽜、⽜}
  \begin{Phonetics}{牺牲}{xi1sheng1}[][HSK 6]
    \definition[份]{s.}{sacrifício; um animal abatido para sacrifício; refere-se ao sacrifício da própria vida ou dos próprios interesses por um propósito justo, ou refere-se ao preço pago por um determinado propósito}
    \definition{v.}{sacrificar-se; morrer como mártir; dar a própria vida; sacrificar sua vida pela justiça | sacrificar; desistir; fazer algo às custas de; geralmente se refere a pagar um preço ou sofrer danos por alguém ou algo}
  \end{Phonetics}
\end{Entry}

\begin{Entry}{犟}{16}{⽜}
  \begin{Phonetics}{犟}{jiang4}
    \variantof{强}
  \end{Phonetics}
\end{Entry}

%%%%% EOF %%%%%

