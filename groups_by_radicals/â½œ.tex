%%%
%%% Radical "⽜"
%%%

\section*{Radical 93: ``⽜'' (牜、⺧)}\addcontentsline{toc}{section}{Radical 93: ⽜、牜、⺧}

\begin{entry}{牛}{4}{⽜}[Kangxi 93]
  \begin{phonetics}{牛}{niu2}[][HSK 3,5]
    \definition*{s.}{sobrenome Niu}
    \definition{adj.}{muito capaz ou bom | teimoso; arrogante}
    \definition{clas.}{Newton (medida física de força)}
    \definition[头]{s.}{gado; boi | niu (nona das vinte e oito constelações em que a esfera celeste foi dividida, consistindo de seis estrelas, três em Áries e três em Sagitário)}
  \end{phonetics}
\end{entry}

\begin{entry}{牛人}{4,2}{⽜、⼈}
  \begin{phonetics}{牛人}{niu2ren2}
    \definition{s.}{(coloquial) o cara | verdadeiro especialista | \emph{badass}}
  \end{phonetics}
\end{entry}

\begin{entry}{牛仔裤}{4,5,12}{⽜、⼈、⾐}
  \begin{phonetics}{牛仔裤}{niu2zai3ku4}[][HSK 5]
    \definition[条]{s.}{calças jeans}
  \end{phonetics}
\end{entry}

\begin{entry}{牛奶}{4,5}{⽜、⼥}
  \begin{phonetics}{牛奶}{niu2nai3}[][HSK 1]
    \definition[杯,袋,瓶,盒,箱,桶]{s.}{leite}
  \end{phonetics}
\end{entry}

\begin{entry}{牛肉}{4,6}{⽜、⾁}
  \begin{phonetics}{牛肉}{niu2rou4}
    \definition{s.}{carne de vaca | bife}
  \end{phonetics}
\end{entry}

\begin{entry}{牛郎织女}{4,8,8,3}{⽜、⾢、⽷、⼥}
  \begin{phonetics}{牛郎织女}{niu2lang2zhi1nv3}
    \definition*{s.}{Vaqueiro e Tecelã (personagens de contos folclóricos) | amantes separados | Altair e Vega (estrelas)}
  \end{phonetics}
\end{entry}

\begin{entry}{牛顿}{4,10}{⽜、⾴}
  \begin{phonetics}{牛顿}{niu2dun4}
    \definition*{s.}{Newton (nome) | newton (N, unidade de força do SI)}
  \end{phonetics}
\end{entry}

\begin{entry}{牦牛}{8,4}{⽜、⽜}
  \begin{phonetics}{牦牛}{mao2niu2}
    \definition{s.}{iaque}
  \end{phonetics}
\end{entry}

\begin{entry}{物业}{8,5}{⽜、⼀}
  \begin{phonetics}{物业}{wu4ye4}[][HSK 5]
    \definition[处]{s.}{propriedade; gestão de propriedades; gestão patrimonial; administração de imóveis | empresa de administração de imóveis; empresa de gestão imobiliária; empresa de administração de bens imóveis}
  \end{phonetics}
\end{entry}

\begin{entry}{物价}{8,6}{⽜、⼈}
  \begin{phonetics}{物价}{wu4 jia4}[][HSK 5]
    \definition[个]{s.}{preços das commodities; preços das matérias-primas; preço das mercadorias}
  \end{phonetics}
\end{entry}

\begin{entry}{物质}{8,8}{⽜、⾙}
  \begin{phonetics}{物质}{wu4zhi4}[][HSK 5]
    \definition[个]{s.}{matéria; substância; algo que existe além do espírito, que pode ser visto, tocado, cheirado ou detectado por instrumentos científicos | material; meios de subsistência; coisas que permitem às pessoas viver ou viver melhor, como comida, roupas, casas, dinheiro, etc.}
  \end{phonetics}
\end{entry}

\begin{entry}{物理}{8,11}{⽜、⽟}
  \begin{phonetics}{物理}{wu4li3}
    \definition{s.}{física (disciplina)}
  \end{phonetics}
\end{entry}

\begin{entry}{特价}{10,6}{⽜、⼈}
  \begin{phonetics}{特价}{te4 jia4}[][HSK 4]
    \definition{s.}{oferta especial; preço de barganha; preço especial reduzido}
  \end{phonetics}
\end{entry}

\begin{entry}{特地}{10,6}{⽜、⼟}
  \begin{phonetics}{特地}{te4di4}
    \definition{adv.}{especialmente | propositalmente}
  \end{phonetics}
\end{entry}

\begin{entry}{特有}{10,6}{⽜、⽉}
  \begin{phonetics}{特有}{te4 you3}[][HSK 5]
    \definition{adj.}{específico; peculiar; característico; único; exclusivo; especial}
  \end{phonetics}
\end{entry}

\begin{entry}{特色}{10,6}{⽜、⾊}
  \begin{phonetics}{特色}{te4se4}[][HSK 3]
    \definition{s.}{característica; característica distintiva | a cor única, estilo, etc. de um objeto}
  \end{phonetics}
\end{entry}

\begin{entry}{特别}{10,7}{⽜、⼑}
  \begin{phonetics}{特别}{te4bie2}[][HSK 2]
    \definition{adj.}{especial; particular; fora do comum; diferente dos outros, com características próprias}
    \definition{adv.}{especialmente; particularmente | ainda mais; em particular; frequentemente usado com 是 | especialmente; deliberadamente; para um propósito específico}
  \seealsoref{是}{shi4}
  \end{phonetics}
\end{entry}

\begin{entry}{特技}{10,7}{⽜、⼿}
  \begin{phonetics}{特技}{te4ji4}
    \definition{s.}{efeito especial | dublê}
  \end{phonetics}
\end{entry}

\begin{entry}{特定}{10,8}{⽜、⼧}
  \begin{phonetics}{特定}{te4ding4}[][HSK 5]
    \definition{adj.}{específico; especialmente designado | dado; especificado; específico (pessoa, hora, lugar, local, ambiente, etc.)}
  \end{phonetics}
\end{entry}

\begin{entry}{特征}{10,8}{⽜、⼻}
  \begin{phonetics}{特征}{te4zheng1}[][HSK 4]
    \definition[个,种]{s.}{característica; aparência ou o fenômeno característico de uma pessoa ou coisa que pode ser visto de fora}
  \end{phonetics}
\end{entry}

\begin{entry}{特性}{10,8}{⽜、⼼}
  \begin{phonetics}{特性}{te4 xing4}[][HSK 5]
    \definition[个]{s.}{propriedade específica (ou característica) | característica; sabores | propriedade}
  \end{phonetics}
\end{entry}

\begin{entry}{特点}{10,9}{⽜、⽕}
  \begin{phonetics}{特点}{te4dian3}[][HSK 2]
    \definition[个,大]{s.}{característica; peculiaridade; traço distintivo; a singularidade de uma pessoa ou coisa}
  \end{phonetics}
\end{entry}

\begin{entry}{特殊}{10,10}{⽜、⽍}
  \begin{phonetics}{特殊}{te4shu1}[][HSK 4]
    \definition{adj.}{especial; particular; peculiar; excepcional; incomum}
  \end{phonetics}
\end{entry}

\begin{entry}{牺牲}{10,9}{⽜、⽜}
  \begin{phonetics}{牺牲}{xi1sheng1}
    \definition{s.}{abate de um animal como sacrifício}
    \definition{v.}{sacrificar a vida de alguém | sacrificar (algo de valor)}
  \end{phonetics}
\end{entry}

\begin{entry}{犟}{16}{⽜}
  \begin{phonetics}{犟}{jiang4}
    \variantof{强}
  \end{phonetics}
\end{entry}

%%%%% EOF %%%%%

