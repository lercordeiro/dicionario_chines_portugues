%%%
%%% Radical "⼩"
%%%

\section*{Radical 42: ``⼩'' (⺌、⺍)}\addcontentsline{toc}{section}{Radical 42: ⼩、⺌、⺍}

\begin{Entry}{小}{3}{⼩}[Kangxi 42]
  \begin{Phonetics}{小}{xiao3}[][HSK 1,2]
    \definition*{s.}{Sobrenome Xiao}
    \definition{adj.}{menor; pequeno; insignificante; pouco; volume, área, quantidade, intensidade, etc. não são grandes | jovem | expressões humildes, referindo-se a si mesmo ou a pessoas ou coisas relacionadas a si mesmo | por um tempo; por um curto período; por um curto período de tempo | o mais novo; o último na ordem de antiguidade; em último lugar na classificação}
    \definition{pref.}{usado antes do sobrenome, nome, posição na família, etc.}
    \definition{s.}{os jovens; pessoas mais jovens | concubina}
  \end{Phonetics}
\end{Entry}

\begin{Entry}{小于}{3,3}{⼩、⼆}
  \begin{Phonetics}{小于}{xiao3 yu2}[][HSK 6]
    \definition{prep.}{menor que; menos que; indica que um número ou quantidade é menor que outro}
  \end{Phonetics}
\end{Entry}

\begin{Entry}{小小}{3,3}{⼩、⼩}
  \begin{Phonetics}{小小}{xiao3xiao3}
    \definition{adj.}{muito pequeno}
  \end{Phonetics}
\end{Entry}

\begin{Entry}{小区}{3,4}{⼩、⼖}
  \begin{Phonetics}{小区}{xiao3qu1}
    \definition{s.}{conjunto habitacional, comunidade, bairro | célula (telecomunicações)}
  \end{Phonetics}
\end{Entry}

\begin{Entry}{小心}{3,4}{⼩、⼼}
  \begin{Phonetics}{小心}{xiao3xin1}[][HSK 2]
    \definition{adj.}{cuidadoso; atento; com cautela}
    \definition{v.}{ter cuidado; ser cauteloso; estar atento; tomar cuidado; prestar atenção}
  \end{Phonetics}
\end{Entry}

\begin{Entry}{小气鬼}{3,4,9}{⼩、⽓、⿁}
  \begin{Phonetics}{小气鬼}{xiao3qi4gui3}
    \definition{adj.}{avarento | mão-de-vaca | miserável | pão-duro}
  \end{Phonetics}
\end{Entry}

\begin{Entry}{小白菜}{3,5,11}{⼩、⽩、⾋}
  \begin{Phonetics}{小白菜}{xiao3bai2cai4}
    \definition[棵]{s.}{\emph{bok choy} | couve chinesa}
  \end{Phonetics}
\end{Entry}

\begin{Entry}{小众}{3,6}{⼩、⼈}
  \begin{Phonetics}{小众}{xiao3zhong4}
    \definition{s.}{minoria da população | nicho (mercado, etc.)}
  \end{Phonetics}
\end{Entry}

\begin{Entry}{小伙子}{3,6,3}{⼩、⼈、⼦}
  \begin{Phonetics}{小伙子}{xiao3huo3zi5}[][HSK 4]
    \definition[位]{s.}{rapaz jovem; jovem colega}
  \end{Phonetics}
\end{Entry}

\begin{Entry}{小吃}{3,6}{⼩、⼝}
  \begin{Phonetics}{小吃}{xiao3chi1}[][HSK 4]
    \definition[家]{s.}{lanche; petiscos; comida com especialidades locais, não muito para uma porção | prato frio; prato feito; cortes de frios na culinária ocidental | pratos pequenos e baratos; pratos simples em restaurantes com porções pequenas e preços baixos}
  \end{Phonetics}
\end{Entry}

\begin{Entry}{小声}{3,7}{⼩、⼠}
  \begin{Phonetics}{小声}{xiao3 sheng1}[][HSK 2]
    \definition{v.}{falar em voz baixa; falar baixinho; sussurar}
  \end{Phonetics}
\end{Entry}

\begin{Entry}{小时}{3,7}{⼩、⽇}
  \begin{Phonetics}{小时}{xiao3shi2}[][HSK 1]
    \definition{clas.}{hora; unidade de medida legal do tempo, 1 hora equivale a 60 minutos, é 1/24 de um dia}
    \definition[个]{s.}{hora; refere-se a um período de uma hora}
  \end{Phonetics}
\end{Entry}

\begin{Entry}{小时候}{3,7,10}{⼩、⽇、⼈}
  \begin{Phonetics}{小时候}{xiao3 shi2 hou5}[][HSK 2]
    \definition{s.}{na infância; quando alguém era jovem; refere-se à infância}
  \end{Phonetics}
\end{Entry}

\begin{Entry}{小麦}{3,7}{⼩、⿆}
  \begin{Phonetics}{小麦}{xiao3mai4}[][HSK 6]
    \definition[粒,公斤,吨,棵]{s.}{trigo}
  \end{Phonetics}
\end{Entry}

\begin{Entry}{小姐}{3,8}{⼩、⼥}
  \begin{Phonetics}{小姐}{xiao3jie5}[][HSK 1]
    \definition[个,位]{s.}{jovem senhora; anteriormente, era assim que se referiam às filhas de famílias ricas. | senhorita; título honorífico para mulheres jovens | (gíria) prostituta}
  \end{Phonetics}
\end{Entry}

\begin{Entry}{小学}{3,8}{⼩、⼦}
  \begin{Phonetics}{小学}{xiao3 xue2}[][HSK 1]
    \definition[个]{s.}{escola primária (ou fundamental); escolas que oferecem ensino fundamental básico | estudos filológicos; antigamente, referia-se ao estudo da escrita, da fonética e da exegese}
  \end{Phonetics}
\end{Entry}

\begin{Entry}{小学生}{3,8,5}{⼩、⼦、⽣}
  \begin{Phonetics}{小学生}{xiao3 xue2 sheng1}[][HSK 1]
    \definition{s.}{aluno; estudante; estudante do sexo masculino (男); estudante do sexo feminino (女) | um aluno mais novo (do que os outros da sua turma) | (dialeto) um menino pequeno}
  \seealsoref{男}{nan2}
  \seealsoref{女}{nv3}
  \end{Phonetics}
\end{Entry}

\begin{Entry}{小朋友}{3,8,4}{⼩、⽉、⼜}
  \begin{Phonetics}{小朋友}{xiao3 peng2 you3}[][HSK 1]
    \definition[个]{s.}{criança; crianças; refere-se a crianças e adolescentes | (termo usado por um adulto para se dirigir a uma criança) amiguinho; menino (ou menina); termo carinhoso para se referir a crianças e adolescentes}
  \end{Phonetics}
\end{Entry}

\begin{Entry}{小狗}{3,8}{⼩、⽝}
  \begin{Phonetics}{小狗}{xiao3 gou3}
    \definition{s.}{filhote de cachorro}
  \end{Phonetics}
\end{Entry}

\begin{Entry}{小组}{3,8}{⼩、⽷}
  \begin{Phonetics}{小组}{xiao3 zu3}[][HSK 2]
    \definition[个,名,位]{s.}{grupo; um pequeno grupo de pessoas}
  \end{Phonetics}
\end{Entry}

\begin{Entry}{小型}{3,9}{⼩、⼟}
  \begin{Phonetics}{小型}{xiao3 xing2}[][HSK 4]
    \definition{adj.}{de tamanho pequeno; em pequena escala; miniatura; tipo pequeno; tamanho de bolso; tipo compacto}
    \definition{s.}{Mediterrâneo: escunas, pequenos veleiros de pesca ou turismo | pequeno \emph{rover} lunar (duas pessoas)}
  \end{Phonetics}
\end{Entry}

\begin{Entry}{小孩儿}{3,9,2}{⼩、⼦、⼉}
  \begin{Phonetics}{小孩儿}{xiao3hai2r5}[][HSK 1]
    \definition[个]{s.}{criança; bebê}
  \end{Phonetics}
\end{Entry}

\begin{Entry}{小屋}{3,9}{⼩、⼫}
  \begin{Phonetics}{小屋}{xiao3wu1}
    \definition{s.}{cabana | chalé | cabine}
  \end{Phonetics}
\end{Entry}

\begin{Entry}{小树}{3,9}{⼩、⽊}
  \begin{Phonetics}{小树}{xiao3shu4}
    \definition[棵]{s.}{muda | arbusto | árvore pequena}
  \end{Phonetics}
\end{Entry}

\begin{Entry}{小洋白菜}{3,9,5,11}{⼩、⽔、⽩、⾋}
  \begin{Phonetics}{小洋白菜}{xiao3 yang2bai2cai4}
    \definition{s.}{couve de bruxelas}
  \end{Phonetics}
\end{Entry}

\begin{Entry}{小说}{3,9}{⼩、⾔}
  \begin{Phonetics}{小说}{xiao3shuo1}[][HSK 2]
    \definition[本,部,篇,章]{s.}{história; romance; ficção; uma forma literária que reflete a vida social por meio da descrição de personagens, ambiente e enredo}
  \end{Phonetics}
\end{Entry}

\begin{Entry}{小费}{3,9}{⼩、⾙}
  \begin{Phonetics}{小费}{xiao3 fei4}[][HSK 6]
    \definition[笔]{s.}{gorjeta; gratificação; dinheiro extra pago por clientes e viajantes a funcionários de serviços em setores de serviços, como hotéis e pousadas}
  \end{Phonetics}
\end{Entry}

\begin{Entry}{小偷儿}{3,11,2}{⼩、⼈、⼉}
  \begin{Phonetics}{小偷儿}{xiao3 tou1er5}[][HSK 5]
    \definition{s.}{ladrão insignificante (ou furtivo); ladrãozinho | ladrão}
  \end{Phonetics}
\end{Entry}

\begin{Entry}{小腿}{3,13}{⼩、⾁}
  \begin{Phonetics}{小腿}{xiao3tui3}
    \definition{s.}{perna (do joelho ao calcanhar) | haste}
  \end{Phonetics}
\end{Entry}

\begin{Entry}{少}{4}{⼩}
  \begin{Phonetics}{少}{shao3}[][HSK 1]
    \definition{adj.}{menos; pouco (oposto a 多); escasso; não atingir a quantidade original ou esperada}
    \definition{adv.}{um momento; um instante; provisoriamente; ligeiramente}
    \definition{v.}{faltar; ser insuficiente | dever | perder; desaparecer; extraviar | parar; desistir}
  \seealsoref{多}{duo1}
  \end{Phonetics}
  \begin{Phonetics}{少}{shao4}
    \definition*{s.}{Sobrenome Shao}
    \definition{s.}{jovem (em oposição a 老)}
    \definition{s.}{jovem mestre; filho de uma família rica}
  \seealsoref{老}{lao3}
  \end{Phonetics}
\end{Entry}

\begin{Entry}{少儿}{4,2}{⼩、⼉}
  \begin{Phonetics}{少儿}{shao4 er2}[][HSK 6]
    \definition{s.}{criança}
  \end{Phonetics}
\end{Entry}

\begin{Entry}{少年}{4,6}{⼩、⼲}
  \begin{Phonetics}{少年}{shao4 nian2}[][HSK 2]
    \definition[个,名,位]{s.}{adolescente; juventude; atualmente, a faixa etária geralmente referida é de 10 anos ou mais a 18 anos ou mais | menor; jovem; juvenil; refere-se a menores na faixa etária anterior | jovem; adolescente; rapaz}
  \end{Phonetics}
\end{Entry}

\begin{Entry}{少数}{4,13}{⼩、⽁}
  \begin{Phonetics}{少数}{shao3 shu4}[][HSK 2]
    \definition{s.}{número pequeno; poucos; minoria}
  \end{Phonetics}
\end{Entry}

\begin{Entry}{尖}{6}{⼩}
  \begin{Phonetics}{尖}{jian1}[][HSK 6]
    \definition{adj.}{pontiagudo; afilado; agudo | agudo; estridente; penetrante | mesquinho; pão-duro | mordaz; cáustico}
    \definition{s.}{ponto; ponta; topo | o melhor do seu tipo; a melhor escolha; a nata da safra; uma pessoa ou coisa notável}
    \definition{v.}{tornar (a voz, etc.) aguda; estridente}
  \end{Phonetics}
\end{Entry}

\begin{Entry}{尚}{8}{⼩}
  \begin{Phonetics}{尚}{shang4}
    \definition*{s.}{Sobrenome Shang}
    \definition{adv.}{ainda}
    \definition{s.}{costume predominante; refere-se à tendência predominante na sociedade; coisas que geralmente são admiradas pelas pessoas}
    \definition{v.}{valorizar; estimar; dar grande importância a}
  \end{Phonetics}
\end{Entry}

\begin{Entry}{尚且}{8,5}{⼩、⼀}
  \begin{Phonetics}{尚且}{shang4 qie3}
    \definition{conj.}{nem\dots; muito menos\dots; é usado antes do verbo da primeira oração de uma frase complexa para apresentar alguns exemplos óbvios para comparação, a segunda oração frequentemente usa 何况 ou 更 para ecoar e tirar conclusões inevitáveis ​​sobre exemplos semelhantes com diferentes graus de gravidade}
  \seealsoref{更}{geng4}
  \seealsoref{何况}{he2kuang4}
  \end{Phonetics}
\end{Entry}

\begin{Entry}{尚且……何况……}{8,5,7,7}{⼩、⼀、⼈、⼎}
  \begin{Phonetics}{尚且……何况……}{shang4qie3 he2kuang4}
    \definition{conj.}{ainda que\dots, \dots; além do mais\dots e muito menos\dots}
  \end{Phonetics}
\end{Entry}

\begin{Entry}{尝}{9}{⼩}
  \begin{Phonetics}{尝}{chang2}[][HSK 5]
    \definition{adv.}{alguma vez; uma vez}
    \definition{v.}{provar; experimentar o sabor de | provar; experimentar; conhecer | tentar; testar}
  \end{Phonetics}
\end{Entry}

\begin{Entry}{尝试}{9,8}{⼩、⾔}
  \begin{Phonetics}{尝试}{chang2shi4}[][HSK 5]
    \definition{v.}{tentar; provar; experimentar}
  \end{Phonetics}
\end{Entry}

%%%%% EOF %%%%%

