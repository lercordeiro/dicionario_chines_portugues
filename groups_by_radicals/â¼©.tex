%%%
%%% Radical "⼩"
%%%

\section*{Radical 42: ``⼩'' (⺌、⺍)}\addcontentsline{toc}{section}{Radical 42: ⼩、⺌、⺍}

\begin{entry}{小}{3}{⼩}[Kangxi 42]
  \begin{phonetics}{小}{xiao3}[][HSK 1,2]
    \definition*{s.}{Sobrenome Xiao}
    \definition{adj.}{menor; pequeno; insignificante; pouco; volume, área, quantidade, intensidade, etc. não são grandes | jovem | expressões humildes, referindo-se a si mesmo ou a pessoas ou coisas relacionadas a si mesmo | por um tempo; por um curto período; por um curto período de tempo | o mais novo; o último na ordem de antiguidade; em último lugar na classificação}
    \definition{pref.}{usado antes do sobrenome, nome, posição na família, etc.}
    \definition{s.}{os jovens; pessoas mais jovens | concubina}
  \end{phonetics}
\end{entry}

\begin{entry}{小小}{3,3}{⼩、⼩}
  \begin{phonetics}{小小}{xiao3xiao3}
    \definition{adj.}{muito pequeno}
  \end{phonetics}
\end{entry}

\begin{entry}{小区}{3,4}{⼩、⼖}
  \begin{phonetics}{小区}{xiao3qu1}
    \definition{s.}{conjunto habitacional, comunidade, bairro | célula (telecomunicações)}
  \end{phonetics}
\end{entry}

\begin{entry}{小心}{3,4}{⼩、⼼}
  \begin{phonetics}{小心}{xiao3xin1}[][HSK 2]
    \definition{adj.}{cuidadoso; atento; com cautela}
    \definition{v.}{ter cuidado; ser cauteloso; estar atento; tomar cuidado; prestar atenção}
  \end{phonetics}
\end{entry}

\begin{entry}{小气鬼}{3,4,9}{⼩、⽓、⿁}
  \begin{phonetics}{小气鬼}{xiao3qi4gui3}
    \definition{adj.}{avarento | mão-de-vaca | miserável | pão-duro}
  \end{phonetics}
\end{entry}

\begin{entry}{小白菜}{3,5,11}{⼩、⽩、⾋}
  \begin{phonetics}{小白菜}{xiao3bai2cai4}
    \definition[棵]{s.}{\emph{bok choy} | couve chinesa}
  \end{phonetics}
\end{entry}

\begin{entry}{小众}{3,6}{⼩、⼈}
  \begin{phonetics}{小众}{xiao3zhong4}
    \definition{s.}{minoria da população | nicho (mercado, etc.)}
  \end{phonetics}
\end{entry}

\begin{entry}{小伙子}{3,6,3}{⼩、⼈、⼦}
  \begin{phonetics}{小伙子}{xiao3huo3zi5}[][HSK 4]
    \definition[个]{s.}{rapaz jovem; jovem colega}
  \end{phonetics}
\end{entry}

\begin{entry}{小吃}{3,6}{⼩、⼝}
  \begin{phonetics}{小吃}{xiao3chi1}[][HSK 4]
    \definition{s.}{lanche; petiscos; comida com especialidades locais, não muito para uma porção | prato frio; prato feito; cortes de frios na culinária ocidental | pratos pequenos e baratos; pratos simples em restaurantes com porções pequenas e preços baixos}
  \end{phonetics}
\end{entry}

\begin{entry}{小声}{3,7}{⼩、⼠}
  \begin{phonetics}{小声}{xiao3 sheng1}[][HSK 2]
    \definition{v.}{falar em voz baixa; falar baixinho; sussurar}
  \end{phonetics}
\end{entry}

\begin{entry}{小时}{3,7}{⼩、⽇}
  \begin{phonetics}{小时}{xiao3shi2}[][HSK 1]
    \definition{clas.}{hora; unidade de medida legal do tempo, 1 hora equivale a 60 minutos, é 1/24 de um dia}
    \definition[个]{s.}{hora; refere-se a um período de uma hora}
  \end{phonetics}
\end{entry}

\begin{entry}{小时候}{3,7,10}{⼩、⽇、⼈}
  \begin{phonetics}{小时候}{xiao3 shi2 hou5}[][HSK 2]
    \definition{s.}{na infância; quando alguém era jovem; refere-se à infância}
  \end{phonetics}
\end{entry}

\begin{entry}{小姐}{3,8}{⼩、⼥}
  \begin{phonetics}{小姐}{xiao3jie5}[][HSK 1]
    \definition[个,位]{s.}{jovem senhora; anteriormente, era assim que se referiam às filhas de famílias ricas. | senhorita; título honorífico para mulheres jovens | (gíria) prostituta}
  \end{phonetics}
\end{entry}

\begin{entry}{小学}{3,8}{⼩、⼦}
  \begin{phonetics}{小学}{xiao3 xue2}[][HSK 1]
    \definition[个]{s.}{escola primária (ou fundamental); escolas que oferecem ensino fundamental básico | estudos filológicos; antigamente, referia-se ao estudo da escrita, da fonética e da exegese}
  \end{phonetics}
\end{entry}

\begin{entry}{小学生}{3,8,5}{⼩、⼦、⽣}
  \begin{phonetics}{小学生}{xiao3 xue2 sheng1}[][HSK 1]
    \definition{s.}{aluno; estudante; estudante do sexo masculino (男); estudante do sexo feminino (女) | um aluno mais novo (do que os outros da sua turma) | (dialeto) um menino pequeno}
  \seealsoref{男}{nan2}
  \seealsoref{女}{nv3}
  \end{phonetics}
\end{entry}

\begin{entry}{小朋友}{3,8,4}{⼩、⽉、⼜}
  \begin{phonetics}{小朋友}{xiao3 peng2 you3}[][HSK 1]
    \definition[个]{s.}{criança; crianças; refere-se a crianças e adolescentes | (termo usado por um adulto para se dirigir a uma criança) amiguinho; menino (ou menina); termo carinhoso para se referir a crianças e adolescentes}
  \end{phonetics}
\end{entry}

\begin{entry}{小狗}{3,8}{⼩、⽝}
  \begin{phonetics}{小狗}{xiao3 gou3}
    \definition{s.}{filhote de cachorro}
  \end{phonetics}
\end{entry}

\begin{entry}{小组}{3,8}{⼩、⽷}
  \begin{phonetics}{小组}{xiao3 zu3}[][HSK 2]
    \definition[个,名,位]{s.}{grupo; um pequeno grupo de pessoas}
  \end{phonetics}
\end{entry}

\begin{entry}{小型}{3,9}{⼩、⼟}
  \begin{phonetics}{小型}{xiao3 xing2}[][HSK 4]
    \definition{adj.}{de tamanho pequeno; em pequena escala; miniatura; tipo pequeno; tamanho de bolso; tipo compacto}
    \definition{s.}{(Mediterrâneo) escunas, pequenos veleiros de pesca ou turismo | pequeno \emph{rover} lunar (duas pessoas)}
  \end{phonetics}
\end{entry}

\begin{entry}{小孩儿}{3,9,2}{⼩、⼦、⼉}
  \begin{phonetics}{小孩儿}{xiao3hai2r5}[][HSK 1]
    \definition[个]{s.}{criança; bebê}
  \end{phonetics}
\end{entry}

\begin{entry}{小屋}{3,9}{⼩、⼫}
  \begin{phonetics}{小屋}{xiao3wu1}
    \definition{s.}{cabana | chalé | cabine}
  \end{phonetics}
\end{entry}

\begin{entry}{小树}{3,9}{⼩、⽊}
  \begin{phonetics}{小树}{xiao3shu4}
    \definition[棵]{s.}{muda | arbusto | árvore pequena}
  \end{phonetics}
\end{entry}

\begin{entry}{小洋白菜}{3,9,5,11}{⼩、⽔、⽩、⾋}
  \begin{phonetics}{小洋白菜}{xiao3 yang2bai2cai4}
    \definition{s.}{couve de bruxelas}
  \end{phonetics}
\end{entry}

\begin{entry}{小说}{3,9}{⼩、⾔}
  \begin{phonetics}{小说}{xiao3shuo1}[][HSK 2]
    \definition[本,部,篇,章]{s.}{história; romance; ficção; uma forma literária que reflete a vida social por meio da descrição de personagens, ambiente e enredo}
  \end{phonetics}
\end{entry}

\begin{entry}{小偷儿}{3,11,2}{⼩、⼈、⼉}
  \begin{phonetics}{小偷儿}{xiao3 tou1er5}[][HSK 5]
    \definition{s.}{ladrão insignificante (ou furtivo); ladrãozinho | ladrão}
  \end{phonetics}
\end{entry}

\begin{entry}{小腿}{3,13}{⼩、⾁}
  \begin{phonetics}{小腿}{xiao3tui3}
    \definition{s.}{perna (do joelho ao calcanhar) | haste}
  \end{phonetics}
\end{entry}

\begin{entry}{少}{4}{⼩}
  \begin{phonetics}{少}{shao3}[][HSK 1]
    \definition{adj.}{menos; pouco (oposto a 多); escasso; não atingir a quantidade original ou esperada}
    \definition{adv.}{um momento; um instante; provisoriamente; ligeiramente}
    \definition{v.}{faltar; ser insuficiente | dever | perder; desaparecer; extraviar | parar; desistir}
  \seealsoref{多}{duo1}
  \end{phonetics}
  \begin{phonetics}{少}{shao4}
    \definition*{s.}{Sobrenome Shao}
    \definition{s.}{jovem (em oposição a 老)}
    \definition{s.}{jovem mestre; filho de uma família rica}
  \seealsoref{老}{lao3}
  \end{phonetics}
\end{entry}

\begin{entry}{少儿}{4,2}{⼩、⼉}
  \begin{phonetics}{少儿}{shao4 er2}[][HSK 6]
    \definition{s.}{criança}
  \end{phonetics}
\end{entry}

\begin{entry}{少年}{4,6}{⼩、⼲}
  \begin{phonetics}{少年}{shao4 nian2}[][HSK 2]
    \definition[个,名,位]{s.}{adolescente; juventude; atualmente, a faixa etária geralmente referida é de 10 anos ou mais a 18 anos ou mais | menor; jovem; juvenil; refere-se a menores na faixa etária anterior | jovem; adolescente; rapaz}
  \end{phonetics}
\end{entry}

\begin{entry}{少数}{4,13}{⼩、⽁}
  \begin{phonetics}{少数}{shao3 shu4}[][HSK 2]
    \definition{s.}{número pequeno; poucos; minoria}
  \end{phonetics}
\end{entry}

\begin{entry}{尖}{6}{⼩}
  \begin{phonetics}{尖}{jian1}[][HSK 6]
    \definition{adj.}{pontiagudo; afilado; agudo | agudo; estridente; penetrante | mesquinho; pão-duro | mordaz; cáustico}
    \definition{s.}{ponto; ponta; topo | o melhor do seu tipo; a melhor escolha; a nata da safra; uma pessoa ou coisa notável}
    \definition{v.}{tornar (a voz, etc.) aguda; estridente}
  \end{phonetics}
\end{entry}

\begin{entry}{尚}{8}{⼩}
  \begin{phonetics}{尚}{shang4}
    \definition*{s.}{Sobrenome Shang}
    \definition{adv.}{ainda}
    \definition{s.}{costume predominante; refere-se à tendência predominante na sociedade; coisas que geralmente são admiradas pelas pessoas}
    \definition{v.}{valorizar; estimar; dar grande importância a}
  \end{phonetics}
\end{entry}

\begin{entry}{尚且}{8,5}{⼩、⼀}
  \begin{phonetics}{尚且}{shang4qie3}
    \definition{conj.}{até | ainda}
  \end{phonetics}
\end{entry}

\begin{entry}{尚且……何况……}{8,5,7,7}{⼩、⼀、⼈、⼎}
  \begin{phonetics}{尚且……何况……}{shang4qie3 he2kuang4}
    \definition{conj.}{ainda que\dots, \dots}
  \end{phonetics}
\end{entry}

\begin{entry}{尝}{9}{⼩}
  \begin{phonetics}{尝}{chang2}[][HSK 5]
    \definition{adv.}{alguma vez; uma vez}
    \definition{v.}{provar; experimentar o sabor de | provar; experimentar; conhecer | tentar; testar}
  \end{phonetics}
\end{entry}

\begin{entry}{尝试}{9,8}{⼩、⾔}
  \begin{phonetics}{尝试}{chang2shi4}[][HSK 5]
    \definition{v.}{tentar; provar; experimentar}
  \end{phonetics}
\end{entry}

%%%%% EOF %%%%%

