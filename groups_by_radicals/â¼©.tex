%%%
%%% Radical "⼩"
%%%

\section*{Radical 42: ``⼩'' (⺌、⺍)}\addcontentsline{toc}{section}{Radical 42: ⼩、⺌、⺍}

\begin{entry}{小}{3}{⼩}[Kangxi 42]
  \begin{phonetics}{小}{xiao3}[][HSK 1,2]
    \definition{adj.}{pequeno | jovem}
  \end{phonetics}
\end{entry}

\begin{entry}{小小}{3,3}{⼩、⼩}
  \begin{phonetics}{小小}{xiao3xiao3}
    \definition{adj.}{muito pequeno}
  \end{phonetics}
\end{entry}

\begin{entry}{小区}{3,4}{⼩、⼖}
  \begin{phonetics}{小区}{xiao3qu1}
    \definition{s.}{conjunto habitacional, comunidade, bairro | célula (telecomunicações)}
  \end{phonetics}
\end{entry}

\begin{entry}{小心}{3,4}{⼩、⼼}
  \begin{phonetics}{小心}{xiao3xin1}[][HSK 2]
    \definition{adj.}{cuidado}
  \end{phonetics}
\end{entry}

\begin{entry}{小气鬼}{3,4,9}{⼩、⽓、⿁}
  \begin{phonetics}{小气鬼}{xiao3qi4gui3}
    \definition{adj.}{avarento | mão-de-vaca | miserável | pão-duro}
  \end{phonetics}
\end{entry}

\begin{entry}{小白菜}{3,5,11}{⼩、⽩、⾋}
  \begin{phonetics}{小白菜}{xiao3bai2cai4}
    \definition[棵]{s.}{\emph{bok choy} | couve chinesa}
  \end{phonetics}
\end{entry}

\begin{entry}{小众}{3,6}{⼩、⼈}
  \begin{phonetics}{小众}{xiao3zhong4}
    \definition{s.}{minoria da população | nicho (mercado, etc.)}
  \end{phonetics}
\end{entry}

\begin{entry}{小伙子}{3,6,3}{⼩、⼈、⼦}
  \begin{phonetics}{小伙子}{xiao3huo3zi5}[][HSK 4]
    \definition[个]{s.}{rapaz jovem; jovem colega}
  \end{phonetics}
\end{entry}

\begin{entry}{小吃}{3,6}{⼩、⼝}
  \begin{phonetics}{小吃}{xiao3chi1}[][HSK 4]
    \definition{s.}{lanche; petiscos; comida com especialidades locais, não muito para uma porção | prato frio; prato feito; cortes de frios na culinária ocidental | pratos pequenos e baratos; pratos simples em restaurantes com porções pequenas e preços baixos}
  \end{phonetics}
\end{entry}

\begin{entry}{小声}{3,7}{⼩、⼠}
  \begin{phonetics}{小声}{xiao3 sheng1}[][HSK 2]
    \definition{v.}{falar em voz baixa | sussurar}
  \end{phonetics}
\end{entry}

\begin{entry}{小时}{3,7}{⼩、⽇}
  \begin{phonetics}{小时}{xiao3shi2}[][HSK 1]
    \definition{adv.}{hora | para horas}
    \definition[个]{s.}{hora}
  \end{phonetics}
\end{entry}

\begin{entry}{小时候}{3,7,10}{⼩、⽇、⼈}
  \begin{phonetics}{小时候}{xiao3 shi2 hou5}[][HSK 2]
    \definition{s.}{na infância | quando alguém era jovem}
  \end{phonetics}
\end{entry}

\begin{entry}{小姐}{3,8}{⼩、⼥}
  \begin{phonetics}{小姐}{xiao3jie5}[][HSK 1]
    \definition[个,位]{s.}{senhorita | jovem senhora | (gíria) prostituta}
  \end{phonetics}
\end{entry}

\begin{entry}{小学}{3,8}{⼩、⼦}
  \begin{phonetics}{小学}{xiao3xue2}[][HSK 1]
    \definition{s.}{escola ensino fundamental}
  \end{phonetics}
\end{entry}

\begin{entry}{小学生}{3,8,5}{⼩、⼦、⽣}
  \begin{phonetics}{小学生}{xiao3xue2sheng1}[][HSK 1]
    \definition{s.}{aluno, estudante de escola primária}
  \end{phonetics}
\end{entry}

\begin{entry}{小朋友}{3,8,4}{⼩、⽉、⼜}
  \begin{phonetics}{小朋友}{xiao3peng2you3}[][HSK 1]
    \definition{s.}{criança | [termo de tratamento usado por um adulto para uma criança] amiguinho}
  \end{phonetics}
\end{entry}

\begin{entry}{小狗}{3,8}{⼩、⽝}
  \begin{phonetics}{小狗}{xiao3 gou3}
    \definition{s.}{filhote de cachorro}
  \end{phonetics}
\end{entry}

\begin{entry}{小组}{3,8}{⼩、⽷}
  \begin{phonetics}{小组}{xiao3 zu3}[][HSK 2]
    \definition[个]{s.}{grupo}
  \end{phonetics}
\end{entry}

\begin{entry}{小型}{3,9}{⼩、⼟}
  \begin{phonetics}{小型}{xiao3 xing2}[][HSK 4]
    \definition{adj.}{de tamanho pequeno; em pequena escala; miniatura; tipo pequeno; tamanho de bolso; tipo compacto}
    \definition{s.}{(Mediterrâneo) escunas, pequenos veleiros de pesca ou turismo | pequeno \emph{rover} lunar (duas pessoas)}
  \end{phonetics}
\end{entry}

\begin{entry}{小孩儿}{3,9,2}{⼩、⼦、⼉}
  \begin{phonetics}{小孩儿}{xiao3hai2r5}[][HSK 1]
    \definition[个]{s.}{criança | bebê}
  \end{phonetics}
\end{entry}

\begin{entry}{小屋}{3,9}{⼩、⼫}
  \begin{phonetics}{小屋}{xiao3wu1}
    \definition{s.}{cabana | chalé | cabine}
  \end{phonetics}
\end{entry}

\begin{entry}{小树}{3,9}{⼩、⽊}
  \begin{phonetics}{小树}{xiao3shu4}
    \definition[棵]{s.}{muda | arbusto | árvore pequena}
  \end{phonetics}
\end{entry}

\begin{entry}{小洋白菜}{3,9,5,11}{⼩、⽔、⽩、⾋}
  \begin{phonetics}{小洋白菜}{xiao3 yang2bai2cai4}
    \definition{s.}{couve de bruxelas}
  \end{phonetics}
\end{entry}

\begin{entry}{小说}{3,9}{⼩、⾔}
  \begin{phonetics}{小说}{xiao3shuo1}[][HSK 2]
    \definition[本,部]{s.}{romance | ficção}
  \end{phonetics}
\end{entry}

\begin{entry}{小腿}{3,13}{⼩、⾁}
  \begin{phonetics}{小腿}{xiao3tui3}
    \definition{s.}{perna (do joelho ao calcanhar) | haste}
  \end{phonetics}
\end{entry}

\begin{entry}{少}{4}{⼩}
  \begin{phonetics}{少}{shao3}[][HSK 1]
    \definition{adj.}{pouco, poucos}
    \definition{v.}{sentir falta | faltar | parar (de fazer algo)}
  \end{phonetics}
  \begin{phonetics}{少}{shao4}
    \definition{s.}{jovem}
  \end{phonetics}
\end{entry}

\begin{entry}{少年}{4,6}{⼩、⼲}
  \begin{phonetics}{少年}{shao4 nian2}[][HSK 2]
    \definition[个]{s.}{adolescente; juventude; atualmente, a faixa etária geralmente referida é de 10 anos ou mais a 18 anos ou mais | menor; jovem; juvenil; refere-se a menores na faixa etária anterior | jovem; adolescente; rapaz}
  \end{phonetics}
\end{entry}

\begin{entry}{少数}{4,13}{⼩、⽁}
  \begin{phonetics}{少数}{shao3 shu4}[][HSK 2]
    \definition{s.}{pequeno número | poucos | minoria}
  \end{phonetics}
\end{entry}

\begin{entry}{尚且}{8,5}{⼩、⼀}
  \begin{phonetics}{尚且}{shang4qie3}
    \definition{conj.}{até | ainda}
  \end{phonetics}
\end{entry}

\begin{entry}{尚且……何况……}{8,5,7,7}{⼩、⼀、⼈、⼎}
  \begin{phonetics}{尚且……何况……}{shang4qie3 he2kuang4}
    \definition{conj.}{ainda que\dots, \dots}
  \end{phonetics}
\end{entry}

%%%%% EOF %%%%%

