%%%
%%% Radical "⼛"
%%%

\section*{Radical 28: ``⼛''}\addcontentsline{toc}{section}{Radical 28: ⼛}

\begin{entry}{去}{5}{⼛}
  \begin{phonetics}{去}{qu4}[][HSK 1]
    \definition*{v.aux.}{usado entre uma frase verbal (ou frase preposicional) e um verbo para indicar que o primeiro é um método ou atitude e o último é um propósito | usado depois de um verbo para indicar que a ação está longe da localização do falante}
    \definition{adj.}{passado; último; refere-se ao tempo passado (um ano)}
    \definition{adv.}{muito; extremamente; usado depois de adjetivos como 大, 多 e 远, significa 极 ou 非常}
    \definition{s.}{tom descendente, um dos quatro tons do chinês clássico e o quarto tom na pronúncia padrão do chinês moderno}
    \definition{v.}{ir; partir; sair | estar separado de | perder | remover; livrar-se de | ir (a algum lugar) para fazer algo; sair do local onde o interlocutor se encontra para outro lugar (oposto a 来) | ir para; estar indo para (fazer algo lá); usado antes de outro verbo para indicar fazer algo | desempenhar o papel de; representar o papel de; interpretar papéis em óperas | enviar; fazer ir; despachar}
  \seealsoref{大}{da4}
  \seealsoref{多}{duo1}
  \seealsoref{非常}{fei1chang2}
  \seealsoref{极}{ji2}
  \seealsoref{来}{lai2}
  \seealsoref{远}{yuan3}
  \end{phonetics}
\end{entry}

\begin{entry}{去世}{5,5}{⼛、⼀}
  \begin{phonetics}{去世}{qu4shi4}[][HSK 3]
    \definition{v.}{morrer; falecer (um adulto)}
  \end{phonetics}
\end{entry}

\begin{entry}{去年}{5,6}{⼛、⼲}
  \begin{phonetics}{去年}{qu4nian2}[][HSK 1]
    \definition{s.}{ano passado}
  \end{phonetics}
\end{entry}

\begin{entry}{去死}{5,6}{⼛、⽍}
  \begin{phonetics}{去死}{qu4si3}
    \definition{interj.}{Caia morto! | Vá para o Inferno!}
  \end{phonetics}
\end{entry}

\begin{entry}{丢}{6}{⼛}
  \begin{phonetics}{丢}{diu1}[][HSK 5]
    \definition{v.}{perder; extraviar; estar ausente | lançar; atirar | colocar (deixar) de lado | deixar (para trás)}
  \end{phonetics}
\end{entry}

\begin{entry}{丢下}{6,3}{⼛、⼀}
  \begin{phonetics}{丢下}{diu1xia4}
    \definition{v.}{abandonar}
  \end{phonetics}
\end{entry}

\begin{entry}{丢开}{6,4}{⼛、⼶}
  \begin{phonetics}{丢开}{diu1kai1}
    \definition{v.}{jogar fora ou deixar de lado | esquecer por um tempo}
  \end{phonetics}
\end{entry}

\begin{entry}{丢失}{6,5}{⼛、⼤}
  \begin{phonetics}{丢失}{diu1shi1}
    \definition{v.}{perder}
  \end{phonetics}
\end{entry}

\begin{entry}{丢弃}{6,7}{⼛、⼶}
  \begin{phonetics}{丢弃}{diu1qi4}
    \definition{v.}{jogar fora | descartar}
  \end{phonetics}
\end{entry}

\begin{entry}{丢官}{6,8}{⼛、⼧}
  \begin{phonetics}{丢官}{diu1guan1}
    \definition{v.}{perder um cargo oficial}
  \end{phonetics}
\end{entry}

\begin{entry}{丢掉}{6,11}{⼛、⼿}
  \begin{phonetics}{丢掉}{diu1diao4}
    \definition{v.}{jogar fora | descartar |perder}
  \end{phonetics}
\end{entry}

\begin{entry}{丢脸}{6,11}{⼛、⾁}
  \begin{phonetics}{丢脸}{diu1lian3}
    \definition{adj.}{vergonhoso}
  \end{phonetics}
\end{entry}

\begin{entry}{县}{7}{⼛}
  \begin{phonetics}{县}{xian4}[][HSK 4]
    \definition[个]{s.}{condado; unidade de divisão administrativa}
  \end{phonetics}
\end{entry}

\begin{entry}{参与}{8,3}{⼛、⼀}
  \begin{phonetics}{参与}{can1yu4}[][HSK 4]
    \definition{v.}{participar de; tomar parte em; ter uma mão em; envolver-se em; participar (no planejamento, discussão e condução dos assuntos)}
  \end{phonetics}
\end{entry}

\begin{entry}{参加}{8,5}{⼛、⼒}
  \begin{phonetics}{参加}{can1jia1}[][HSK 2]
    \definition{v.}{aderir (a organizações); participar; participar (de atividades); participar de alguma organização ou atividade | dar (conselho, sugestão, etc.)}
  \end{phonetics}
\end{entry}

\begin{entry}{参考}{8,6}{⼛、⽼}
  \begin{phonetics}{参考}{can1kao3}[][HSK 4]
    \definition{v.}{consultar; referir-se a; acessar informações relevantes para estudo ou pesquisa | consultar; referir-se a; lidar com coisas, observar, ler, aprender e usar materiais relevantes}
  \end{phonetics}
\end{entry}

\begin{entry}{参观}{8,6}{⼛、⾒}
  \begin{phonetics}{参观}{can1guan1}[][HSK 2]
    \definition{v.}{visitar; dar uma olhada; observação no local (resultados do trabalho, carreira, instalações, locais históricos e pontos turísticos, etc.)}
  \end{phonetics}
\end{entry}

%%%%% EOF %%%%%

