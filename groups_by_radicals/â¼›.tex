%%%
%%% Radical "⼛"
%%%

\section*{Radical 28: ``⼛''}\addcontentsline{toc}{section}{Radical 28: ⼛}

\begin{Entry}{去}{5}{⼛}
  \begin{Phonetics}{去}{qu4}[][HSK 1]
    \definition{adj.}{passado; último; refere-se ao tempo passado (um ano)}
    \definition{adv.}{muito; extremamente; usado depois de adjetivos como 大, 多 e 远, significa 极 ou 非常}
    \definition{s.}{tom descendente, um dos quatro tons do chinês clássico e o quarto tom na pronúncia padrão do chinês moderno}
    \definition{v.}{ir; partir; sair | estar separado de | perder | remover; livrar-se de | ir (a algum lugar) para fazer algo; sair do local onde o interlocutor se encontra para outro lugar (oposto a 来) | ir para; estar indo para (fazer algo lá); usado antes de outro verbo para indicar fazer algo | desempenhar o papel de; representar o papel de; interpretar papéis em óperas | enviar; fazer ir; despachar}
    \definition{v.aux.}{usado entre uma frase verbal (ou frase preposicional) e um verbo para indicar que o primeiro é um método ou atitude e o último é um propósito | usado depois de um verbo para indicar que a ação está longe da localização do falante}
  \seealsoref{大}{da4}
  \seealsoref{多}{duo1}
  \seealsoref{非常}{fei1chang2}
  \seealsoref{极}{ji2}
  \seealsoref{来}{lai2}
  \seealsoref{远}{yuan3}
  \end{Phonetics}
\end{Entry}

\begin{Entry}{去世}{5,5}{⼛、⼀}
  \begin{Phonetics}{去世}{qu4shi4}[][HSK 3]
    \definition{v.}{(usado apenas para adultos, com conotações solenes) morrer; falecer; deixar este mundo}
  \end{Phonetics}
\end{Entry}

\begin{Entry}{去年}{5,6}{⼛、⼲}
  \begin{Phonetics}{去年}{qu4nian2}[][HSK 1]
    \definition{s.}{ano passado}
  \end{Phonetics}
\end{Entry}

\begin{Entry}{去死}{5,6}{⼛、⽍}
  \begin{Phonetics}{去死}{qu4si3}
    \definition{interj.}{Caia morto! | Vá para o Inferno!}
  \end{Phonetics}
\end{Entry}

\begin{Entry}{去掉}{5,11}{⼛、⼿}
  \begin{Phonetics}{去掉}{qu4 diao4}[][HSK 6]
    \definition{v.}{livrar-se de; tirar; acabar com; abandonar; erradicar}
  \end{Phonetics}
\end{Entry}

\begin{Entry}{丢}{6}{⼛}
  \begin{Phonetics}{丢}{diu1}[][HSK 5]
    \definition{v.}{perder; extraviar; estar ausente | lançar; atirar | colocar (deixar) de lado | deixar (para trás)}
  \end{Phonetics}
\end{Entry}

\begin{Entry}{丢下}{6,3}{⼛、⼀}
  \begin{Phonetics}{丢下}{diu1xia4}
    \definition{v.}{abandonar}
  \end{Phonetics}
\end{Entry}

\begin{Entry}{丢开}{6,4}{⼛、⼶}
  \begin{Phonetics}{丢开}{diu1kai1}
    \definition{v.}{jogar fora ou deixar de lado | esquecer por um tempo}
  \end{Phonetics}
\end{Entry}

\begin{Entry}{丢失}{6,5}{⼛、⼤}
  \begin{Phonetics}{丢失}{diu1shi1}
    \definition{v.}{perder}
  \end{Phonetics}
\end{Entry}

\begin{Entry}{丢弃}{6,7}{⼛、⼶}
  \begin{Phonetics}{丢弃}{diu1qi4}
    \definition{v.}{jogar fora | descartar}
  \end{Phonetics}
\end{Entry}

\begin{Entry}{丢官}{6,8}{⼛、⼧}
  \begin{Phonetics}{丢官}{diu1guan1}
    \definition{v.}{perder um cargo oficial}
  \end{Phonetics}
\end{Entry}

\begin{Entry}{丢掉}{6,11}{⼛、⼿}
  \begin{Phonetics}{丢掉}{diu1diao4}
    \definition{v.}{jogar fora | descartar |perder}
  \end{Phonetics}
\end{Entry}

\begin{Entry}{丢脸}{6,11}{⼛、⾁}
  \begin{Phonetics}{丢脸}{diu1lian3}
    \definition{adj.}{vergonhoso}
  \end{Phonetics}
\end{Entry}

\begin{Entry}{县}{7}{⼛}
  \begin{Phonetics}{县}{xian4}[][HSK 4]
    \definition[个]{s.}{condado; unidade de divisão administrativa}
  \end{Phonetics}
\end{Entry}

\begin{Entry}{参}{8}{⼛}
  \begin{Phonetics}{参}{can1}
    \definition{v.}{juntar-se; entrar; tomar parte em; participar | referir; consultar; comparar com outros materiais | ligar para prestar homenagem a; fazer uma visita |  (significado antigo) acusar um funcionário perante o imperador; relatar ou expor ao imperador | explorar e compreender (verdade, significado, etc.)}
  \end{Phonetics}
\end{Entry}

\begin{Entry}{参与}{8,3}{⼛、⼀}
  \begin{Phonetics}{参与}{can1yu4}[][HSK 4]
    \definition{v.}{participar de; tomar parte em; ter uma mão em; envolver-se em; participar (no planejamento, discussão e condução dos assuntos)}
  \end{Phonetics}
\end{Entry}

\begin{Entry}{参见}{8,4}{⼛、⾒}
  \begin{Phonetics}{参见}{can1jian4}[][HSK 7-9]
    \definition{v.}{(em referências) ver; ver também | consultar; visualizar | ler algo para referência | prestar homenagem a (um superior, etc.)}
  \end{Phonetics}
\end{Entry}

\begin{Entry}{参加}{8,5}{⼛、⼒}
  \begin{Phonetics}{参加}{can1jia1}[][HSK 2]
    \definition{v.}{aderir (a organizações); participar; participar (de atividades); participar de alguma organização ou atividade | dar (conselho, sugestão, etc.)}
  \end{Phonetics}
\end{Entry}

\begin{Entry}{参军}{8,6}{⼛、⼍}
  \begin{Phonetics}{参军}{can1/jun1}[][HSK 7-9]
    \definition{s.}{oficial do estado-maior militar; título oficial antigo}
    \definition{v.+compl.}{juntar-se ao exército; alistar-se}
  \end{Phonetics}
\end{Entry}

\begin{Entry}{参考}{8,6}{⼛、⽼}
  \begin{Phonetics}{参考}{can1kao3}[][HSK 4]
    \definition{v.}{consultar; referir-se a; acessar informações relevantes para estudo ou pesquisa | consultar; referir-se a; lidar com coisas, observar, ler, aprender e usar materiais relevantes}
  \end{Phonetics}
\end{Entry}

\begin{Entry}{参观}{8,6}{⼛、⾒}
  \begin{Phonetics}{参观}{can1guan1}[][HSK 2]
    \definition{v.}{visitar; dar uma olhada; observação no local (resultados do trabalho, carreira, instalações, locais históricos e pontos turísticos, etc.)}
  \end{Phonetics}
\end{Entry}

\begin{Entry}{参展}{8,10}{⼛、⼫}
  \begin{Phonetics}{参展}{can1 zhan3}[][HSK 6]
    \definition{v.}{expor ou participar de uma feira comercial, etc.}
  \end{Phonetics}
\end{Entry}

\begin{Entry}{参谋}{8,11}{⼛、⾔}
  \begin{Phonetics}{参谋}{can1mou2}[][HSK 7-9]
    \definition{s.}{oficial de estado-maior; pessoal militar envolvido em planejamento militar e outros assuntos |conselheiro; consultor}
    \definition{v.}{aconselhar; dar conselhos}
  \end{Phonetics}
\end{Entry}

\begin{Entry}{参照}{8,13}{⼛、⽕}
  \begin{Phonetics}{参照}{can1zhao4}[][HSK 7-9]
    \definition{v.}{consultar; referir-se a; referir-se e imitar (métodos, experiências, etc.)}
  \end{Phonetics}
\end{Entry}

\begin{Entry}{参赛}{8,14}{⼛、⾙}
  \begin{Phonetics}{参赛}{can1 sai4}[][HSK 6]
    \definition{v.}{participar de uma partida (ou competição); competir}
  \end{Phonetics}
\end{Entry}

%%%%% EOF %%%%%

