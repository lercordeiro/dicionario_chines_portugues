%%%
%%% Radical "⾵"
%%%

\section*{Radical 182: ``⾵'' (风)}\addcontentsline{toc}{section}{Radical 182: ⾵、风}

\begin{entry}{风}{4}{⾵}[Kangxi 182]
  \begin{phonetics}{风}{feng1}[][HSK 1]
    \definition*{s.}{Sobrenome Feng}
    \definition{adj.}{lendário; sem fundamento concreto | rápido; veloz | promíscuo; libertino; sedutor}
    \definition[阵,丝]{s.}{vento; fluxo de ar | prática; ambiente; costume | cena; vista | notícias; fofocas; rumores | comportamento; maneira; estilo | canção folclórica | certas doenças}
    \definition{v.}{colocar para secar ou arejar; secar ao vento}
  \end{phonetics}
\end{entry}

\begin{entry}{风光}{4,6}{⾵、⼉}
  \begin{phonetics}{风光}{feng1guang1}[][HSK 5]
    \definition{s.}{cena; vista; paisagens naturais e humanas}
  \end{phonetics}
\end{entry}

\begin{entry}{风俗}{4,9}{⾵、⼈}
  \begin{phonetics}{风俗}{feng1su2}[][HSK 4]
    \definition[种,个,些]{s.}{costumes; a soma de costumes sociais, maneiras, hábitos, etc., desenvolvidos ao longo do tempo.}
  \end{phonetics}
\end{entry}

\begin{entry}{风度}{4,9}{⾵、⼴}
  \begin{phonetics}{风度}{feng1du4}[][HSK 5]
    \definition{s.}{postura; comportamento; porte; conduta; atitude}
  \end{phonetics}
\end{entry}

\begin{entry}{风险}{4,9}{⾵、⾩}
  \begin{phonetics}{风险}{feng1xian3}[][HSK 3]
    \definition[个,种]{s.}{risco; perigo; ameaça; riscos possíveis}
  \end{phonetics}
\end{entry}

\begin{entry}{风扇}{4,10}{⾵、⼾}
  \begin{phonetics}{风扇}{feng1shan4}
    \definition{s.}{ventilador elétrico}
  \end{phonetics}
\end{entry}

\begin{entry}{风格}{4,10}{⾵、⽊}
  \begin{phonetics}{风格}{feng1ge2}[][HSK 4]
    \definition{s.}{modo; estilo; maneira; caráter | características das criações literárias de diferentes épocas, povos, escolas ou indivíduos em termos de conteúdo ideológico e técnicas artísticas}
  \end{phonetics}
\end{entry}

\begin{entry}{风景}{4,12}{⾵、⽇}
  \begin{phonetics}{风景}{feng1jing3}[][HSK 4]
    \definition[种,处,道]{s.}{cenário; paisagem; cenários e vistas que podem ser apreciados, inclui paisagens, flores, árvores, edifícios e determinados fenômenos naturais}
  \end{phonetics}
\end{entry}

\begin{entry}{风筝}{4,12}{⾵、⽵}
  \begin{phonetics}{风筝}{feng1zheng5}
    \definition{s.}{pipa | papagaio | pandorga}
  \end{phonetics}
\end{entry}

\begin{entry}{风暴}{4,15}{⾵、⽇}
  \begin{phonetics}{风暴}{feng1bao4}[][HSK 6]
    \definition{s.}{tempestade; vendaval; um termo geral para perturbações violentas na atmosfera e mudanças drásticas no clima, como tempestades de areia, tornados, ciclones tropicais, etc. | tempestade; comoção violenta; uma metáfora para um evento tão poderoso que abala toda a sociedade}
  \end{phonetics}
\end{entry}

\begin{entry}{飒}{9}{⾵}
  \begin{phonetics}{飒}{sa4}
    \definition{adj.}{(das mulheres) natural e desenfreada; elegante; valente}
    \definition{interj.}{(onomatopéia) farfalhar; sussurrar | (onomatopéia) som do vento}
    \definition{v.}{murchar}
  \end{phonetics}
\end{entry}

\begin{entry}{飒飒}{9,9}{⾵、⾵}
  \begin{phonetics}{飒飒}{sa4sa4}
    \definition{s.}{o farfalhar | sussurro | murmúrio (do vento nas árvores, o mar, etc.)}
  \end{phonetics}
\end{entry}

\begin{entry}{飘}{15}{⾵}
  \begin{phonetics}{飘}{piao1}
    \definition{adj.}{complacente | frívolo | fraco | instável | bambo | cambaleante}
    \definition{v.}{flutuar (no ar) | esvoaçar | tremular}
  \end{phonetics}
\end{entry}

%%%%% EOF %%%%%

