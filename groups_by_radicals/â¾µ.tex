%%%
%%% Radical "⾵"
%%%

\section*{Radical 182: ``⾵'' (风)}\addcontentsline{toc}{section}{Radical 182: ⾵、风}

\begin{Entry}{风}{4}{⾵}[Kangxi 182]
  \begin{Phonetics}{风}{feng1}[][HSK 1]
    \definition*{s.}{Sobrenome Feng}
    \definition{adj.}{lendário; sem fundamento concreto | rápido; veloz | promíscuo; libertino; sedutor}
    \definition[阵,丝]{s.}{vento; fluxo de ar | prática; ambiente; costume | cena; vista | notícias; fofocas; rumores | comportamento; maneira; estilo | canção folclórica | certas doenças}
    \definition{v.}{colocar para secar ou arejar; secar ao vento}
  \end{Phonetics}
\end{Entry}

\begin{Entry}{风力}{4,2}{⾵、⼒}
  \begin{Phonetics}{风力}{feng1li4}[][HSK 7-9]
    \definition{s.}{força do vento; o poder do vento | energia eólica; alimentado pelo vento}
  \end{Phonetics}
\end{Entry}

\begin{Entry}{风云}{4,4}{⾵、⼆}
  \begin{Phonetics}{风云}{feng1yun2}[][HSK 7-9]
    \definition{s.}{vento e nuvem; uma situação tempestuosa ou instável}
  \end{Phonetics}
\end{Entry}

\begin{Entry}{风气}{4,4}{⾵、⽓}
  \begin{Phonetics}{风气}{feng1qi4}[][HSK 7-9]
    \definition[种]{s.}{ethos; atmosfera; humor geral; prática comum; um \emph{hobby} ou hábito popular na sociedade ou em um grupo}
  \seealsoref{风尚}{feng1shang4}
  \end{Phonetics}
\end{Entry}

\begin{Entry}{风水}{4,4}{⾵、⽔}
  \begin{Phonetics}{风水}{feng1shui3}[][HSK 7-9]
    \definition*{s.}{Feng Shui}
    \definition{s.}{geomancia; refere-se à localização geográfica de locais residenciais, cemitérios, etc., como a direção dos veios de terra, montanhas e rios}
  \end{Phonetics}
\end{Entry}

\begin{Entry}{风风雨雨}{4,4,8,8}{⾵、⾵、⾬、⾬}
  \begin{Phonetics}{风风雨雨}{feng1feng1yu3yu3}[][HSK 7-9]
    \definition{expr.}{altos e baixos | dificuldades e sofrimentos | fofocas infundadas}
  \end{Phonetics}
\end{Entry}

\begin{Entry}{风光}{4,6}{⾵、⼉}
  \begin{Phonetics}{风光}{feng1guang1}[][HSK 5]
    \definition{s.}{cena; vista; paisagens naturais e humanas}
  \end{Phonetics}
\end{Entry}

\begin{Entry}{风沙}{4,7}{⾵、⽔}
  \begin{Phonetics}{风沙}{feng1sha1}[][HSK 7-9]
    \definition{s.}{areia soprada pelo vento; areia levada pelo vento}
  \end{Phonetics}
\end{Entry}

\begin{Entry}{风味}{4,8}{⾵、⼝}
  \begin{Phonetics}{风味}{feng1wei4}[][HSK 7-9]
    \definition{s.}{sabor especial; cor (ou sabor) local; características das coisas (referindo-se principalmente às características locais)}
  \end{Phonetics}
\end{Entry}

\begin{Entry}{风和日丽}{4,8,4,7}{⾵、⼝、⽇、⼀}
  \begin{Phonetics}{风和日丽}{feng1he2-ri4li4}[][HSK 7-9]
    \definition{expr.}{vento moderado, sol bonito; tempo bom e ensolarado; sol brilhante e uma brisa suave; clima quente e ensolarado; descreve o clima ensolarado e ameno (usado principalmente na primavera)}
  \end{Phonetics}
\end{Entry}

\begin{Entry}{风尚}{4,8}{⾵、⼩}
  \begin{Phonetics}{风尚}{feng1shang4}[][HSK 7-9]
    \definition{s.}{costume (ou prática, hábito) predominante; as tendências e costumes sociais predominantes durante um determinado período}
  \seealsoref{风气}{feng1qi4}
  \end{Phonetics}
\end{Entry}

\begin{Entry}{风波}{4,8}{⾵、⽔}
  \begin{Phonetics}{风波}{feng1bo1}[][HSK 7-9]
    \definition[场]{s.}{Meteorologia: vento e ondas --- perturbação | Literário: tempestade | perturbação; onda de vento; uma tempestade em copo d'água | tempestade; crise; perturbação}
  \end{Phonetics}
\end{Entry}

\begin{Entry}{风采}{4,8}{⾵、⾤}
  \begin{Phonetics}{风采}{feng1cai3}[][HSK 7-9]
    \definition{s.}{graça; elegância; comportamento; carisma | estilo; graça; talento literário}
  \end{Phonetics}
\end{Entry}

\begin{Entry}{风雨}{4,8}{⾵、⾬}
  \begin{Phonetics}{风雨}{feng1yu3}[][HSK 7-9]
    \definition{s.}{o clima; vento e chuva | tribulações; estresse e tempestade; provações e dificuldades; uma metáfora para uma situação difícil e dura}
  \end{Phonetics}
\end{Entry}

\begin{Entry}{风俗}{4,9}{⾵、⼈}
  \begin{Phonetics}{风俗}{feng1su2}[][HSK 4]
    \definition[种,个,些]{s.}{costumes; a soma de costumes sociais, maneiras, hábitos, etc., desenvolvidos ao longo do tempo}
  \end{Phonetics}
\end{Entry}

\begin{Entry}{风度}{4,9}{⾵、⼴}
  \begin{Phonetics}{风度}{feng1du4}[][HSK 5]
    \definition{s.}{postura; comportamento; porte; conduta; atitude}
  \end{Phonetics}
\end{Entry}

\begin{Entry}{风范}{4,9}{⾵、⾋}
  \begin{Phonetics}{风范}{feng1fan4}[][HSK 7-9]
    \definition{s.}{comportamento; porte; postura | estilo; maneira; ar | modelo; protótipo}
  \end{Phonetics}
\end{Entry}

\begin{Entry}{风险}{4,9}{⾵、⾩}
  \begin{Phonetics}{风险}{feng1xian3}[][HSK 3]
    \definition[个,种]{s.}{risco; perigo; ameaça; riscos possíveis}
  \end{Phonetics}
\end{Entry}

\begin{Entry}{风扇}{4,10}{⾵、⼾}
  \begin{Phonetics}{风扇}{feng1shan4}
    \definition{s.}{ventilador elétrico}
  \end{Phonetics}
\end{Entry}

\begin{Entry}{风格}{4,10}{⾵、⽊}
  \begin{Phonetics}{风格}{feng1ge2}[][HSK 4]
    \definition{s.}{modo; estilo; maneira; caráter | características das criações literárias de diferentes épocas, povos, escolas ou indivíduos em termos de conteúdo ideológico e técnicas artísticas}
  \end{Phonetics}
\end{Entry}

\begin{Entry}{风流}{4,10}{⾵、⽔}
  \begin{Phonetics}{风流}{feng1liu2}[][HSK 7-9]
    \definition{adj.}{refinado e saboroso | descontrolado em espírito e comportamento | romântico; amoroso; licencioso | distinto e admirável | meritório e talentoso | talentoso e romântico; talentoso em letras e não convencional no estilo de vida | dissoluto; solto}
  \end{Phonetics}
\end{Entry}

\begin{Entry}{风浪}{4,10}{⾵、⽔}
  \begin{Phonetics}{风浪}{feng1lang4}[][HSK 7-9]
    \definition{s.}{tempestade; ondas tempestuosas; vento e ondas na água | dificuldades; sofrimento; uma metáfora para uma experiência difícil ou perigosa}
  \end{Phonetics}
\end{Entry}

\begin{Entry}{风情}{4,11}{⾵、⼼}
  \begin{Phonetics}{风情}{feng1qing2}[][HSK 7-9]
    \definition{s.}{charme; bom temperamento; graça; expressão | charme; panorama; gosto elegante | charme; sentimentos amorosos; pornografia | costumes e práticas locais | a condição do vento; informações sobre direção e velocidade do vento | sensação; a sensação do ambiente circundante em um determinado ambiente}
  \end{Phonetics}
\end{Entry}

\begin{Entry}{风景}{4,12}{⾵、⽇}
  \begin{Phonetics}{风景}{feng1jing3}[][HSK 4]
    \definition[种,处,道]{s.}{cenário; paisagem; cenários e vistas que podem ser apreciados, inclui paisagens, flores, árvores, edifícios e determinados fenômenos naturais}
  \end{Phonetics}
\end{Entry}

\begin{Entry}{风筝}{4,12}{⾵、⽵}
  \begin{Phonetics}{风筝}{feng1zheng5}[][HSK 7-9]
    \definition[个,只]{s.}{pipa; papagaio; pandorga; é feito de tiras de bambu amarradas em um esqueleto em forma de pássaros, insetos, peixes, dragões, etc., coberto com papel ou seda, e flutua no ar com a ajuda do vento, as pessoas puxam a longa linha amarrada a ele para controlá-lo}
  \end{Phonetics}
\end{Entry}

\begin{Entry}{风貌}{4,14}{⾵、⾘}
  \begin{Phonetics}{风貌}{feng1mao4}[][HSK 7-9]
    \definition{s.}{estilo e características; o estilo e a aparência das coisas | vista; cena; cenário | aparência e porte elegantes; o comportamento e a aparência de uma pessoa}
  \end{Phonetics}
\end{Entry}

\begin{Entry}{风暴}{4,15}{⾵、⽇}
  \begin{Phonetics}{风暴}{feng1bao4}[][HSK 6]
    \definition{s.}{tempestade; vendaval; um termo geral para perturbações violentas na atmosfera e mudanças drásticas no clima, como tempestades de areia, tornados, ciclones tropicais, etc. | tempestade; comoção violenta; uma metáfora para um evento tão poderoso que abala toda a sociedade}
  \end{Phonetics}
\end{Entry}

\begin{Entry}{风趣}{4,15}{⾵、⾛}
  \begin{Phonetics}{风趣}{feng1qu4}[][HSK 7-9]
    \definition{adj.}{espirituoso; humorístico; (fala, texto, etc.) humorístico e interessante}
    \definition{s.}{sagacidade; humor; refere-se ao humor e ao gosto humorísticos e interessantes}
  \end{Phonetics}
\end{Entry}

\begin{Entry}{风餐露宿}{4,16,21,11}{⾵、⾷、⾬、⼧}
  \begin{Phonetics}{风餐露宿}{feng1can1-lu4su4}[][HSK 7-9]
    \definition{expr.}{comer ao vento e dormir no orvalho --- suportar as dificuldades de uma jornada árdua | o sol brilha no chão; para descrever as dificuldades de viajar ou viver ao ar livre, também é dito ``dormir ao ar livre''}
  \end{Phonetics}
\end{Entry}

\begin{Entry}{飒}{9}{⾵}
  \begin{Phonetics}{飒}{sa4}
    \definition{adj.}{(das mulheres) natural e desenfreada; elegante; valente}
    \definition{interj.}{(onomatopéia) farfalhar; sussurrar | (onomatopéia) som do vento}
    \definition{v.}{murchar}
  \end{Phonetics}
\end{Entry}

\begin{Entry}{飒飒}{9,9}{⾵、⾵}
  \begin{Phonetics}{飒飒}{sa4sa4}
    \definition{s.}{o farfalhar | sussurro | murmúrio (do vento nas árvores, o mar, etc.)}
  \end{Phonetics}
\end{Entry}

\begin{Entry}{飘}{15}{⾵}
  \begin{Phonetics}{飘}{piao1}
    \definition{adj.}{complacente | frívolo | fraco | instável | bambo | cambaleante}
    \definition{v.}{flutuar (no ar) | esvoaçar | tremular}
  \end{Phonetics}
\end{Entry}

\begin{Entry}{飙}{16}{⾵}
  \begin{Phonetics}{飙}{biao1}
    \definition{s.}{tempestade; furacão; redemoinho | Literário: vento violento; redemoinho}
  \end{Phonetics}
\end{Entry}

\begin{Entry}{飙升}{16,4}{⾵、⼗}
  \begin{Phonetics}{飙升}{biao1sheng1}[][HSK 7-9]
    \definition{v.}{disparar; subir rapidamente; (preço, quantidade, etc.) aumentam rapidamente}
  \end{Phonetics}
\end{Entry}

%%%%% EOF %%%%%

