%%%
%%% Radical "⽕"
%%%

\section*{Radical 86: ``⽕'' (灬)}\addcontentsline{toc}{section}{Radical 86: ⽕、灬}

\begin{entry}{火}{4}{⽕}[Kangxi 86]
  \begin{phonetics}{火}{huo3}[][HSK 3,4]
    \definition*{s.}{sobrenome Huo}
    \definition{adj.}{ardente; flamejante; vermelho como fogo | efervescente; próspero}
    \definition{adv.}{urgentemente}
    \definition{clas.}{para unidades militares (antigo)}
    \definition{s.}{fogo | armas de fogo; munições | calor interno (uma das seis causas de doenças) | a ação de lutar}
    \definition{v.}{ficar com raiva; perder a paciência}
  \end{phonetics}
\end{entry}

\begin{entry}{火车}{4,4}{⽕、⾞}
  \begin{phonetics}{火车}{huo3 che1}[][HSK 1]
    \definition[列,节,班,趟]{s.}{trem | comboio}
  \end{phonetics}
\end{entry}

\begin{entry}{火车司机}{4,4,5,6}{⽕、⾞、⼝、⽊}
  \begin{phonetics}{火车司机}{huo3che1 si1ji1}
    \definition{s.}{maquinista de trem}
  \end{phonetics}
\end{entry}

\begin{entry}{火灾}{4,7}{⽕、⽕}
  \begin{phonetics}{火灾}{huo3 zai1}[][HSK 5]
    \definition[场]{s.}{fogo (como um desastre); conflagração; desastres causados por incêndios}
  \end{phonetics}
\end{entry}

\begin{entry}{火柴}{4,10}{⽕、⽊}
  \begin{phonetics}{火柴}{huo3chai2}[][HSK 5]
    \definition[根,盒]{s.}{fósforo (palito de fósforo); fósforo de segurança; iniciador de fogo feito de uma tira fina de madeira mergulhada em um composto de fósforo ou enxofre}
  \end{phonetics}
\end{entry}

\begin{entry}{火海}{4,10}{⽕、⽔}
  \begin{phonetics}{火海}{huo3hai3}
    \definition{s.}{um mar de chamas}
  \end{phonetics}
\end{entry}

\begin{entry}{火腿}{4,13}{⽕、⾁}
  \begin{phonetics}{火腿}{huo3 tui3}[][HSK 5]
    \definition[道,个]{s.}{presunto; as pernas de porco marinadas mais famosas são produzidas em Jinhua, na província de Zhejiang, e em Xuanwei, na província de Yunnan.}
  \end{phonetics}
\end{entry}

\begin{entry}{灭火}{5,4}{⽕、⽕}
  \begin{phonetics}{灭火}{mie4huo3}
    \definition{s.}{combate a incêndios}
    \definition{v.}{extinguir um incêndio}
  \end{phonetics}
\end{entry}

\begin{entry}{灯}{6}{⽕}
  \begin{phonetics}{灯}{deng1}[][HSK 2]
    \definition[盏]{s.}{lâmpada | lanterna | luz}
  \end{phonetics}
\end{entry}

\begin{entry}{灯丝}{6,5}{⽕、⼀}
  \begin{phonetics}{灯丝}{deng1si1}
    \definition{s.}{filamento (de uma lâmpada)}
  \end{phonetics}
\end{entry}

\begin{entry}{灯号}{6,5}{⽕、⼝}
  \begin{phonetics}{灯号}{deng1hao4}
    \definition{s.}{sinal luminoso | luz indicadora}
  \end{phonetics}
\end{entry}

\begin{entry}{灯光}{6,6}{⽕、⼉}
  \begin{phonetics}{灯光}{deng1 guang1}[][HSK 4]
    \definition{s.}{iluminação; luminosidade da lâmpada | luminação (palco); equipamento de iluminação para palco ou estúdio}
  \end{phonetics}
\end{entry}

\begin{entry}{灯泡}{6,8}{⽕、⽔}
  \begin{phonetics}{灯泡}{deng1pao4}
    \definition[个]{s.}{lâmpada | (gíria) terceiro indesejado estragando encontro de casal}
  \seealsoref{电灯泡}{dian4deng1pao4}
  \end{phonetics}
\end{entry}

\begin{entry}{灯标}{6,9}{⽕、⽊}
  \begin{phonetics}{灯标}{deng1biao1}
    \definition{s.}{farol | luz de farol}
  \end{phonetics}
\end{entry}

\begin{entry}{灰色}{6,6}{⽕、⾊}
  \begin{phonetics}{灰色}{hui1 se4}[][HSK 5]
    \definition{adj.}{obscuro; ambíguo | sombrio; pessimista}
    \definition[个]{s.}{cor cinza; acinzentado}
  \end{phonetics}
\end{entry}

\begin{entry}{灵感}{7,13}{⽕、⼼}
  \begin{phonetics}{灵感}{ling2gan3}
    \definition{s.}{inspiração | explosão de criatividade em empreendimento científico ou artístico}
  \end{phonetics}
\end{entry}

\begin{entry}{灵魂}{7,13}{⽕、⿁}
  \begin{phonetics}{灵魂}{ling2hun2}
    \definition{s.}{alma | espírito}
  \end{phonetics}
\end{entry}

\begin{entry}{灶台}{7,5}{⽕、⼝}
  \begin{phonetics}{灶台}{zao4tai2}
    \definition{s.}{fogão}
  \end{phonetics}
\end{entry}

\begin{entry}{炎热}{8,10}{⽕、⽕}
  \begin{phonetics}{炎热}{yan2re4}
    \definition{adj.}{extremamente quente | escaldante (clima)}
  \end{phonetics}
\end{entry}

\begin{entry}{炒}{8}{⽕}
  \begin{phonetics}{炒}{chao3}
    \definition{v.}{saltear | demitir (alguém)}
  \end{phonetics}
\end{entry}

\begin{entry}{点}{9}{⽕}
  \begin{phonetics}{点}{dian3}[][HSK 1]
    \definition{clas.}{para itens | hora cheia}
    \definition{s.}{ponto | gota | mancha | horas | ponto (no espaço ou no tempo) | traço de ponto em caracteres chineses}
    \definition{v.}{desenhar um ponto | verificar uma lista | escolher | pedir (comida em um restaurante) | tocar brevemente | sugerir | acender | derramar um líquido gota a gota}
  \end{phonetics}
\end{entry}

\begin{entry}{点火}{9,4}{⽕、⽕}
  \begin{phonetics}{点火}{dian3huo3}
    \definition{s.}{ignição}
    \definition{v.}{inflamar | acender um fogo | agitar | dar partida em um motor | (figurativo) provocar problemas}
  \end{phonetics}
\end{entry}

\begin{entry}{点头}{9,5}{⽕、⼤}
  \begin{phonetics}{点头}{dian3 tou2}[][HSK 2]
    \definition{v.}{acenar com a cabeça}
  \end{phonetics}
\end{entry}

\begin{entry}{点名}{9,6}{⽕、⼝}
  \begin{phonetics}{点名}{dian3 ming2}[][HSK 4]
    \definition{v.}{fazer a lista de chamada; manter o controle da presença de alguém; chamar nomes para controle de presença | mencionar alguém pelo nome}
  \end{phonetics}
\end{entry}

\begin{entry}{点燃}{9,16}{⽕、⽕}
  \begin{phonetics}{点燃}{dian3 ran2}[][HSK 5]
    \definition{v.}{acender; inflamar; acender uma fogueira, para iluminar}
  \end{phonetics}
\end{entry}

\begin{entry}{烂}{9}{⽕}
  \begin{phonetics}{烂}{lan4}[][HSK 5]
    \definition{adj.}{macio; pastoso; amassado | podre; deteriorado | quebrado; esfarrapado; gasto | desorganizado; indigno}
    \definition{adv.}{totalmente; extremamente; completamente; expressa um grau muito profundo}
    \definition{v.}{apodrecer; infeccionar; decompor-se}
  \end{phonetics}
\end{entry}

\begin{entry}{烈士}{10,3}{⽕、⼠}
  \begin{phonetics}{烈士}{lie4shi4}
    \definition{s.}{mártir}
  \end{phonetics}
\end{entry}

\begin{entry}{烟}{10}{⽕}
  \begin{phonetics}{烟}{yan1}[][HSK 3]
    \definition*[竜]{s.}{sobrenome Yan}
    \definition[根]{s.}{fumaça; gases produzidos quando substâncias queimam | névoa; vapor; algo como fumaça | tabaco; planta de tabaco | fumo; cigarro; termo geral para cigarros, charutos, etc. | ópio}
    \definition{v.}{ficar irritado com a fumaça (olhos)}
  \end{phonetics}
\end{entry}

\begin{entry}{烟火}{10,4}{⽕、⽕}
  \begin{phonetics}{烟火}{yan1huo3}
    \definition{s.}{fogo de artifício}
  \end{phonetics}
\end{entry}

\begin{entry}{烟叶}{10,5}{⽕、⼝}
  \begin{phonetics}{烟叶}{yan1ye4}
    \definition{s.}{folha de tabaco}
  \end{phonetics}
\end{entry}

\begin{entry}{烟头}{10,5}{⽕、⼤}
  \begin{phonetics}{烟头}{yan1tou2}
    \definition[根]{s.}{bituca de cigarro}
  \end{phonetics}
\end{entry}

\begin{entry}{烟囱}{10,7}{⽕、⼞}
  \begin{phonetics}{烟囱}{yan1cong1}
    \definition{s.}{chaminé}
  \end{phonetics}
\end{entry}

\begin{entry}{烟花}{10,7}{⽕、⾋}
  \begin{phonetics}{烟花}{yan1hua1}
    \definition{s.}{fogos de artifício}
  \end{phonetics}
\end{entry}

\begin{entry}{烟雨}{10,8}{⽕、⾬}
  \begin{phonetics}{烟雨}{yan1yu3}
    \definition{s.}{chuvisco | garoa}
  \end{phonetics}
\end{entry}

\begin{entry}{烟草}{10,9}{⽕、⾋}
  \begin{phonetics}{烟草}{yan1cao3}
    \definition{s.}{tabaco}
  \end{phonetics}
\end{entry}

\begin{entry}{烤}{10}{⽕}
  \begin{phonetics}{烤}{kao3}
    \definition{v.}{assar | grelhar}
  \end{phonetics}
\end{entry}

\begin{entry}{烤肉}{10,6}{⽕、⾁}
  \begin{phonetics}{烤肉}{kao3 rou4}[][HSK 5]
    \definition[块,串,片,盘]{s.}{churrasco (literalmente carne assada)}
  \end{phonetics}
\end{entry}

\begin{entry}{烤鸭}{10,10}{⽕、⿃}
  \begin{phonetics}{烤鸭}{kao3ya1}[][HSK 5]
    \definition{s.}{pato assado; pato recheado e assado em um forno especial após ser abatido}
  \end{phonetics}
\end{entry}

\begin{entry}{烦}{10}{⽕}
  \begin{phonetics}{烦}{fan2}[][HSK 4]
    \definition{adj.}{(estar) cansado; (estar) aborrecido; irritado; incomodado | supérfluo e confuso, bagunçado}
    \definition{s.}{problema; incômodo}
    \definition{v.}{incomodar; solicitar; colocar alguém em dificuldade (para fazer algo)}
  \end{phonetics}
\end{entry}

\begin{entry}{烧}{10}{⽕}
  \begin{phonetics}{烧}{shao1}[][HSK 4]
    \definition[次]{s.}{febre; temperatura corporal mais alta do que o normal}
    \definition{v.}{queimar; pegar fogo | cozinhar; aquecer; assar | guisar depois de fritar ou fritar depois de guisar | assar; grelhar os ingredientes dos alimentos diretamente sobre o fogo | ter febre; estar com febre | danificar (matar ou murchar) as plantas pelo uso excessivo (ou inadequado) de fertilizantes | tornar-se arrogante ou presunçoso; metáfora de estar em uma boa posição e se deixar levar}
  \end{phonetics}
\end{entry}

\begin{entry}{烧烤}{10,10}{⽕、⽕}
  \begin{phonetics}{烧烤}{shao1kao3}
    \definition{s.}{churrasco}
    \definition{v.}{assar}
  \end{phonetics}
\end{entry}

\begin{entry}{热}{10}{⽕}
  \begin{phonetics}{热}{re4}[][HSK 1]
    \definition{adj.}{quente (clima) | fervente | ardente | fervoroso}
    \definition{v.}{aquecer | ferver}
  \end{phonetics}
\end{entry}

\begin{entry}{热门}{10,3}{⽕、⾨}
  \begin{phonetics}{热门}{re4men2}[][HSK 5]
    \definition{adj.}{popular}
    \definition{s.}{algo que desperta o interesse popular; metáfora para algo que está na moda e recebe a atenção de todos (em contraste com ``冷门'').}
  \seealsoref{冷门}{leng3men2}
  \end{phonetics}
\end{entry}

\begin{entry}{热心}{10,4}{⽕、⼼}
  \begin{phonetics}{热心}{re4xin1}[][HSK 4]
    \definition{adj.}{ardente; sincero; entusiasmado; afetuoso; apaixonado; interessado}
    \definition{v.}{ser entusiasmado com alguma coisa}
  \end{phonetics}
\end{entry}

\begin{entry}{热血沸腾}{10,6,8,13}{⽕、⾎、⽔、⾁}
  \begin{phonetics}{热血沸腾}{re4xue4fei4teng2}
    \definition{expr.}{ferver o sangue | apaixonar-se}
  \end{phonetics}
\end{entry}

\begin{entry}{热泪盈眶}{10,8,9,11}{⽕、⽔、⽫、⽬}
  \begin{phonetics}{热泪盈眶}{re4lei4ying2kuang4}
    \definition{expr.}{olhos cheios de lágrimas de emoção | extremamente emocionado}
  \end{phonetics}
\end{entry}

\begin{entry}{热闹}{10,8}{⽕、⾾}
  \begin{phonetics}{热闹}{re4nao5}[][HSK 4]
    \definition{adj.}{animado; agitado; movimentado com barulho e excitação; descreve uma cena animada com uma atmosfera calorosa}
    \definition{s.}{uma vista emocionante; uma cena de agitação e excitação; atmosfera acolhedora}
    \definition{v.}{animar; divertir-se}
  \end{phonetics}
\end{entry}

\begin{entry}{热烈}{10,10}{⽕、⽕}
  \begin{phonetics}{热烈}{re4lie4}[][HSK 3]
    \definition{adj.}{caloroso; calorosamente; fervoroso; ardente; entusiasmado}
  \end{phonetics}
\end{entry}

\begin{entry}{热爱}{10,10}{⽕、⽖}
  \begin{phonetics}{热爱}{re4'ai4}[][HSK 3]
    \definition{v.}{amar ardentemente; amar de coração; ter amor profundo por}
  \end{phonetics}
\end{entry}

\begin{entry}{热情}{10,11}{⽕、⼼}
  \begin{phonetics}{热情}{re4qing2}[][HSK 2]
    \definition{adj.}{caloroso | fervoroso | entusiasmado}
    \definition{s.}{entusiasmo | ardor | devoção | calor | zelo}
  \end{phonetics}
\end{entry}

\begin{entry}{热量}{10,12}{⽕、⾥}
  \begin{phonetics}{热量}{re4 liang4}[][HSK 5]
    \definition{s.}{calor; quantidade de calor; calorias; em física, refere-se à energia transferida entre objetos com temperaturas diferentes, do objeto com temperatura mais alta para o objeto com temperatura mais baixa}
  \end{phonetics}
\end{entry}

\begin{entry}{惨}{11}{⽕}
  \begin{phonetics}{惨}{can3}
    \definition{adj.}{miserável | cruel | desumano | desastroso | trágico | sombrio}
  \end{phonetics}
\end{entry}

\begin{entry}{焊}{11}{⽕}
  \begin{phonetics}{焊}{han4}
    \definition{v.}{soldar}
  \end{phonetics}
\end{entry}

\begin{entry}{悲伤}{12,6}{⽕、⼈}
  \begin{phonetics}{悲伤}{bei1 shang1}[][HSK 5]
    \definition{adj.}{triste; pesaroso}
  \end{phonetics}
\end{entry}

\begin{entry}{悲剧}{12,10}{⽕、⼑}
  \begin{phonetics}{悲剧}{bei1 ju4}[][HSK 5]
    \definition[部,出]{s.}{tragédia; drama trágico; uma das principais categorias de teatro, caracterizada basicamente pela representação do conflito irreconciliável entre o protagonista e a realidade e seu final trágico | tragédia; evento triste; metáfora para encontro infeliz}
  \end{phonetics}
\end{entry}

\begin{entry}{焚香}{12,9}{⽕、⾹}
  \begin{phonetics}{焚香}{fen2xiang1}
    \definition{v.}{queimar incenso}
  \end{phonetics}
\end{entry}

\begin{entry}{焦虑}{12,10}{⽕、⾌}
  \begin{phonetics}{焦虑}{jiao1lv4}
    \definition{adj.}{ansioso | preocupado | apreensivo}
  \end{phonetics}
\end{entry}

\begin{entry}{然}{12}{⽕}
  \begin{phonetics}{然}{ran2}
    \definition{conj.}{mas | no entanto}
  \end{phonetics}
\end{entry}

\begin{entry}{然后}{12,6}{⽕、⼝}
  \begin{phonetics}{然后}{ran2hou4}[][HSK 2]
    \definition{conj.}{depois | logo | portanto}
  \end{phonetics}
\end{entry}

\begin{entry}{然而}{12,6}{⽕、⽽}
  \begin{phonetics}{然而}{ran2'er2}[][HSK 4]
    \definition{conj.}{ainda; mas; contudo; todavia; usado no início de uma frase para indicar uma transição; para indicar uma transição, geralmente é precedido por uma conjunção como ``虽然'' para indicar concessão}
  \seealsoref{虽然}{sui1 ran2}
  \end{phonetics}
\end{entry}

\begin{entry}{煎}{13}{⽕}
  \begin{phonetics}{煎}{jian1}
    \definition{v.}{fritar | refogar}
  \end{phonetics}
\end{entry}

\begin{entry}{煎饼}{13,9}{⽕、⾷}
  \begin{phonetics}{煎饼}{jian1bing3}
    \definition[张]{s.}{jianbing, crepe chinês | panqueca}
  \end{phonetics}
\end{entry}

\begin{entry}{煎蛋}{13,11}{⽕、⾍}
  \begin{phonetics}{煎蛋}{jian1dan4}
    \definition{s.}{ovos fritos}
  \end{phonetics}
\end{entry}

\begin{entry}{煤}{13}{⽕}
  \begin{phonetics}{煤}{mei2}[][HSK 5]
    \definition[吨,堆,块]{s.}{carvão; carvão vegetal; minério sólido preto}
  \end{phonetics}
\end{entry}

\begin{entry}{煤气}{13,4}{⽕、⽓}
  \begin{phonetics}{煤气}{mei2 qi4}[][HSK 5]
    \definition[把]{s.}{gás; gás de carvão; gás obtido a partir do processamento do carvão não tem cor nem odor, é tóxico e pode ser queimado ou utilizado como matéria-prima na indústria química | envenenamento por monóxido de carbono}
  \end{phonetics}
\end{entry}

\begin{entry}{照}{13}{⽕}
  \begin{phonetics}{照}{zhao4}[][HSK 3]
    \definition{adv.}{de acordo com; indica agir de acordo com o original ou um certo padrão}
    \definition{prep.}{em direção a; na direção de | de acordo com; conforme}
    \definition{s.}{imagem; fotografia | permissão; licença | brilho; iluminação}
    \definition{v.}{brilhar; acender; iluminar | refletir; espelhar; olhar para sua própria imagem em um espelho, etc. | filmar; fotografar; tirar uma foto (fotografia) | cuidar de; tomar conta de | notificar | contrastar | entender}
  \end{phonetics}
\end{entry}

\begin{entry}{照片}{13,4}{⽕、⽚}
  \begin{phonetics}{照片}{zhao4pian4}[][HSK 2]
    \definition[张,套,幅]{s.}{fotografia | foto}
  \end{phonetics}
\end{entry}

\begin{entry}{照片子}{13,4,3}{⽕、⽚、⼦}
  \begin{phonetics}{照片子}{zhao4pian4zi5}
    \definition{v.}{tirar um raio X}
  \end{phonetics}
\end{entry}

\begin{entry}{照片底版}{13,4,8,8}{⽕、⽚、⼴、⽚}
  \begin{phonetics}{照片底版}{zhao4pian4di3ban3}
    \definition{s.}{placa fotográfica}
  \end{phonetics}
\end{entry}

\begin{entry}{照亮}{13,9}{⽕、⼇}
  \begin{phonetics}{照亮}{zhao4liang4}
    \definition{s.}{iluminação}
    \definition{v.}{iluminar}
  \end{phonetics}
\end{entry}

\begin{entry}{照相}{13,9}{⽕、⽬}
  \begin{phonetics}{照相}{zhao4 xiang4}[][HSK 2]
    \definition{v.+compl.}{tirar fotografia}
  \end{phonetics}
\end{entry}

\begin{entry}{照相机}{13,9,6}{⽕、⽬、⽊}
  \begin{phonetics}{照相机}{zhao4xiang4ji1}
    \definition[个,架,部,台,只]{s.}{câmera/máquina fotográfica}
  \end{phonetics}
\end{entry}

\begin{entry}{照准}{13,10}{⽕、⼎}
  \begin{phonetics}{照准}{zhao4zhun3}
    \definition{s.}{solicitação concedida (uso formal em documento antigo)}
    \definition{v.}{mirar (arma)}
  \end{phonetics}
\end{entry}

\begin{entry}{照顾}{13,10}{⽕、⾴}
  \begin{phonetics}{照顾}{zhao4gu4}[][HSK 2]
    \definition{v.}{cuidar de | atender a | oferecer tratamento preferencial | (de um cliente) patrocinar | fazer compras em | dar consideração a | mostrar consideração por | levar em conta | fazer concessões para}
  \end{phonetics}
\end{entry}

\begin{entry}{照骗}{13,12}{⽕、⾺}
  \begin{phonetics}{照骗}{zhao4pian4}
    \definition{s.}{imagem ``photoshopada''}
  \end{phonetics}
\end{entry}

\begin{entry}{照像}{13,13}{⽕、⼈}
  \begin{phonetics}{照像}{zhao4 xiang4}
    \variantof{照相}
  \end{phonetics}
\end{entry}

\begin{entry}{照像机}{13,13,6}{⽕、⼈、⽊}
  \begin{phonetics}{照像机}{zhao4xiang4ji1}
    \variantof{照相机}
  \end{phonetics}
\end{entry}

\begin{entry}{愿}{14}{⽕}
  \begin{phonetics}{愿}{yuan4}
    \definition{adj.}{honesto e prudente}
  \end{phonetics}
\end{entry}

\begin{entry}{愿望}{14,11}{⽕、⽉}
  \begin{phonetics}{愿望}{yuan4wang4}[][HSK 3]
    \definition[个]{s.}{desejo; aspiração; a ideia de esperar atingir um determinado objetivo no futuro}
  \end{phonetics}
\end{entry}

\begin{entry}{愿意}{14,13}{⽕、⼼}
  \begin{phonetics}{愿意}{yuan4yi4}[][HSK 2]
    \definition{s.}{desejo | esperança}
    \definition{v.}{estar disposto | estar pronto}
  \end{phonetics}
\end{entry}

\begin{entry}{熊}{14}{⽕}
  \begin{phonetics}{熊}{xiong2}
    \definition*{s.}{sobrenome Xiong}
    \definition{adj.}{incapaz}
    \definition[把]{s.}{urso}
    \definition{v.}{repreender}
  \end{phonetics}
\end{entry}

\begin{entry}{熊猫}{14,11}{⽕、⽝}
  \begin{phonetics}{熊猫}{xiong2mao1}
    \definition[把,只]{s.}{panda gigante}
  \seealsoref{猫熊}{mao1xiong2}
  \end{phonetics}
\end{entry}

\begin{entry}{熏香}{14,9}{⽕、⾹}
  \begin{phonetics}{熏香}{xun1xiang1}
    \definition{s.}{incenso}
  \end{phonetics}
\end{entry}

\begin{entry}{熟}{15}{⽕}
  \begin{phonetics}{熟}{shu2}[][HSK 2]
    \definition{adj.}{maduro | cozido | feito | processado | familiar | qualificado | experiente | profundo}
  \end{phonetics}
\end{entry}

\begin{entry}{熟人}{15,2}{⽕、⼈}
  \begin{phonetics}{熟人}{shu2 ren2}[][HSK 3]
    \definition[位]{s.}{amigo; conhecido}
  \end{phonetics}
\end{entry}

\begin{entry}{熟练}{15,8}{⽕、⽷}
  \begin{phonetics}{熟练}{shu2lian4}[][HSK 4]
    \definition{adj.}{especializado; proficiente; qualificado; habilidoso}
  \end{phonetics}
\end{entry}

\begin{entry}{熟悉}{15,11}{⽕、⼼}
  \begin{phonetics}{熟悉}{shu2xi1}[][HSK 5]
    \definition{adj.}{familiarizado com; não ser estranho}
    \definition{v.}{estar familiarizado com; saber claramente que | conhecer bem algo ou alguém; compreender e dominar (a situação) através da observação ou da experiência}
  \end{phonetics}
\end{entry}

\begin{entry}{燃料}{16,10}{⽕、⽃}
  \begin{phonetics}{燃料}{ran2 liao4}[][HSK 4]
    \definition{s.}{combustível; carburante; substâncias que podem gerar calor e energia luminosa quando queimadas podem ser divididas em três tipos de acordo com sua forma: combustível sólido (como carvão, carvão vegetal, madeira), combustível líquido (como gasolina, querosene) e combustível gasoso (como gás de carvão, biogás); também se refere a substâncias que podem gerar energia nuclear, como urânio, plutônio, etc.}
  \end{phonetics}
\end{entry}

\begin{entry}{燃烧}{16,10}{⽕、⽕}
  \begin{phonetics}{燃烧}{ran2shao1}[][HSK 4]
    \definition{s.}{combustão | flama}
    \definition{v.}{queimar; acender | arder; inflamar; ferver; metáfora para as emoções de uma pessoa serem muito fortes, como um fogo ardente}
  \end{phonetics}
\end{entry}

\begin{entry}{爆米花}{19,6,7}{⽕、⽶、⾋}
  \begin{phonetics}{爆米花}{bao4mi3hua1}
    \definition{s.}{pipoca (de milho) | pipoca de arroz}
  \end{phonetics}
\end{entry}

\begin{entry}{爆炸}{19,9}{⽕、⽕}
  \begin{phonetics}{爆炸}{bao4zha4}
    \definition{s.}{explosão}
    \definition{v.}{explodir | detonar}
  \end{phonetics}
\end{entry}

%%%%% EOF %%%%%

