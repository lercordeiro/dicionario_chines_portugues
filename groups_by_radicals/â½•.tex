%%%
%%% Radical "⽕"
%%%

\section*{Radical 86: ``⽕'' (灬)}\addcontentsline{toc}{section}{Radical 86: ⽕、灬}

\begin{Entry}{火}{4}{⽕}[Kangxi 86]
  \begin{Phonetics}{火}{huo3}[][HSK 3,4]
    \definition*{s.}{Sobrenome Huo}
    \definition{adj.}{ardente; flamejante; vermelho como fogo | efervescente; próspero}
    \definition{adv.}{urgentemente}
    \definition{clas.}{usado para unidades militares (antigo)}
    \definition[场,把,团,堆]{s.}{fogo; a luz e as chamas emitidas pela combustão de um objeto | fúria; metáfora para emoções agitadas, irritadas ou raivosas | calor interno (uma das seis causas de doenças) | armas de fogo e munições | a ação de lutar}
    \definition{v.}{ficar com raiva; perder a paciência}
  \end{Phonetics}
\end{Entry}

\begin{Entry}{火车}{4,4}{⽕、⾞}
  \begin{Phonetics}{火车}{huo3 che1}[][HSK 1]
    \definition[个,列,节,班,趟]{s.}{trem; comboio}
  \end{Phonetics}
\end{Entry}

\begin{Entry}{火车司机}{4,4,5,6}{⽕、⾞、⼝、⽊}
  \begin{Phonetics}{火车司机}{huo3che1 si1ji1}
    \definition{s.}{maquinista de trem}
  \end{Phonetics}
\end{Entry}

\begin{Entry}{火灾}{4,7}{⽕、⽕}
  \begin{Phonetics}{火灾}{huo3 zai1}[][HSK 5]
    \definition[起,场]{s.}{fogo (como um desastre); conflagração; desastres causados por incêndios}
  \end{Phonetics}
\end{Entry}

\begin{Entry}{火柴}{4,10}{⽕、⽊}
  \begin{Phonetics}{火柴}{huo3chai2}[][HSK 5]
    \definition[根,盒,包]{s.}{fósforo (palito de fósforo); fósforo de segurança; iniciador de fogo feito de uma tira fina de madeira mergulhada em um composto de fósforo ou enxofre}
  \end{Phonetics}
\end{Entry}

\begin{Entry}{火海}{4,10}{⽕、⽔}
  \begin{Phonetics}{火海}{huo3hai3}
    \definition{s.}{um mar de chamas}
  \end{Phonetics}
\end{Entry}

\begin{Entry}{火腿}{4,13}{⽕、⾁}
  \begin{Phonetics}{火腿}{huo3 tui3}[][HSK 5]
    \definition[道,个]{s.}{presunto; as pernas de porco marinadas mais famosas são produzidas em Jinhua, na província de Zhejiang, e em Xuanwei, na província de Yunnan.}
  \end{Phonetics}
\end{Entry}

\begin{Entry}{火箭}{4,15}{⽕、⽵}
  \begin{Phonetics}{火箭}{huo3jian4}[][HSK 6]
    \definition[个,艘,发,枚]{s.}{foguete; uma aeronave de alta velocidade que utiliza força de reação para se impulsionar para a frente; é usado para lançar satélites, naves espaciais, etc.; também pode ser equipado com uma ogiva para fabricar um míssil}
  \end{Phonetics}
\end{Entry}

\begin{Entry}{灭}{5}{⽕}
  \begin{Phonetics}{灭}{mie4}[][HSK 6]
    \definition{v.}{extinguir-se | extinguir; apagar; desligar | afogar; inundar; submergir | perecer; destruir | exterminar; apagar; acabar com; tornar inexistente}
  \end{Phonetics}
\end{Entry}

\begin{Entry}{灭火}{5,4}{⽕、⽕}
  \begin{Phonetics}{灭火}{mie4huo3}
    \definition{s.}{combate a incêndios}
    \definition{v.}{extinguir um incêndio}
  \end{Phonetics}
\end{Entry}

\begin{Entry}{灯}{6}{⽕}
  \begin{Phonetics}{灯}{deng1}[][HSK 2]
    \definition*{s.}{Sobrenome Deng}
    \definition[盏,个]{s.}{lâmpada; luz; lanterna; dispositivo luminoso, usado principalmente para iluminação | queimador; um aparelho que brilha e aquece como uma lâmpada e pode ser usado para aquecer | tubo; válvula; o nome popular dado aos tubos eletrônicos com formato semelhante a lâmpadas encontrados em aparelhos antigos, como rádios}
  \end{Phonetics}
\end{Entry}

\begin{Entry}{灯丝}{6,5}{⽕、⼀}
  \begin{Phonetics}{灯丝}{deng1si1}
    \definition{s.}{filamento (de uma lâmpada)}
  \end{Phonetics}
\end{Entry}

\begin{Entry}{灯号}{6,5}{⽕、⼝}
  \begin{Phonetics}{灯号}{deng1hao4}
    \definition{s.}{sinal luminoso | luz indicadora}
  \end{Phonetics}
\end{Entry}

\begin{Entry}{灯光}{6,6}{⽕、⼉}
  \begin{Phonetics}{灯光}{deng1 guang1}[][HSK 4]
    \definition[束,盏,点,打]{s.}{iluminação; luminosidade da lâmpada | luminação (palco); equipamento de iluminação para palco ou estúdio}
  \end{Phonetics}
\end{Entry}

\begin{Entry}{灯泡}{6,8}{⽕、⽔}
  \begin{Phonetics}{灯泡}{deng1pao4}
    \definition[个]{s.}{lâmpada | (gíria) terceiro indesejado estragando encontro de casal}
  \seealsoref{电灯泡}{dian4deng1pao4}
  \end{Phonetics}
\end{Entry}

\begin{Entry}{灯标}{6,9}{⽕、⽊}
  \begin{Phonetics}{灯标}{deng1biao1}
    \definition{s.}{farol | luz de farol}
  \end{Phonetics}
\end{Entry}

\begin{Entry}{灰}{6}{⽕}
  \begin{Phonetics}{灰}{hui1}
    \definition{adj.}{cinza (cor) | desanimado; desencorajado; deprimido}
    \definition[把,堆]{s.}{cinzas; pó que sobra após a queima de um objeto | pó; poeira; substância em pó | cal; argamassa (de cal)}
  \end{Phonetics}
\end{Entry}

\begin{Entry}{灰色}{6,6}{⽕、⾊}
  \begin{Phonetics}{灰色}{hui1 se4}[][HSK 5]
    \definition{adj.}{obscuro; ambíguo | sombrio; pessimista}
    \definition[种]{s.}{cor cinza; acinzentado}
  \end{Phonetics}
\end{Entry}

\begin{Entry}{灵}{7}{⽕}
  \begin{Phonetics}{灵}{ling2}
    \definition*{s.}{Sobrenome Ling}
    \definition{adj.}{rápido; inteligente; afiado | eficaz; efetivo | flexível; hábil}
    \definition{s.}{espírito; alma | inteligência; mente | fada; duende; elfo | restos mortais do falecido; esquife | carro funerário; caixão ou algo relacionado aos mortos}
  \end{Phonetics}
\end{Entry}

\begin{Entry}{灵活}{7,9}{⽕、⽔}
  \begin{Phonetics}{灵活}{ling2huo2}[][HSK 6]
    \definition[种,点,些]{adj.}{ágil; rápido; ligeiro; descreve a capacidade de fazer rapidamente mudanças apropriadas com base na situação ao lidar com as coisas | flexível; elástico; descreve reações rápidas, como movimentos e funções cerebrais}
  \end{Phonetics}
\end{Entry}

\begin{Entry}{灵感}{7,13}{⽕、⼼}
  \begin{Phonetics}{灵感}{ling2gan3}
    \definition{s.}{inspiração | explosão de criatividade em empreendimento científico ou artístico}
  \end{Phonetics}
\end{Entry}

\begin{Entry}{灵魂}{7,13}{⽕、⿁}
  \begin{Phonetics}{灵魂}{ling2hun2}
    \definition{s.}{alma | espírito}
  \end{Phonetics}
\end{Entry}

\begin{Entry}{灶}{7}{⽕}
  \begin{Phonetics}{灶}{zao4}
    \definition[口,个]{s.}{fogão de cozinha; fogão de cozinha | cozinha; bagunça; cantina}
  \end{Phonetics}
\end{Entry}

\begin{Entry}{灶台}{7,5}{⽕、⼝}
  \begin{Phonetics}{灶台}{zao4tai2}
    \definition{s.}{fogão}
  \end{Phonetics}
\end{Entry}

\begin{Entry}{灾}{7}{⽕}
  \begin{Phonetics}{灾}{zai1}[][HSK 5]
    \definition[个,场]{s.}{calamidade; desastre | infortúnio pessoal; adversidade | azar}
  \end{Phonetics}
\end{Entry}

\begin{Entry}{灾区}{7,4}{⽕、⼖}
  \begin{Phonetics}{灾区}{zai1 qu1}[][HSK 5]
    \definition{s.}{área de desastre; área afetada por catástrofes}
  \end{Phonetics}
\end{Entry}

\begin{Entry}{灾害}{7,10}{⽕、⼧}
  \begin{Phonetics}{灾害}{zai1hai4}[][HSK 5]
    \definition[场,次,个]{s.}{desastre; calamidade; danos causados pela seca, inundações, pragas, granizo, guerras, etc.}
  \end{Phonetics}
\end{Entry}

\begin{Entry}{灾难}{7,10}{⽕、⾫}
  \begin{Phonetics}{灾难}{zai1nan4}[][HSK 5]
    \definition[场,次,个,种]{s.}{desastre; sofrimento; calamidade; catástrofe; danos e sofrimentos causados por desastres naturais ou guerras}
  \end{Phonetics}
\end{Entry}

\begin{Entry}{炎}{8}{⽕}
  \begin{Phonetics}{炎}{yan2}
    \definition{adj.}{escaldante; ardente}
    \definition{s.}{inflamação | poder; influência}
  \end{Phonetics}
\end{Entry}

\begin{Entry}{炎热}{8,10}{⽕、⽕}
  \begin{Phonetics}{炎热}{yan2re4}
    \definition{adj.}{extremamente quente | escaldante (clima)}
  \end{Phonetics}
\end{Entry}

\begin{Entry}{炒}{8}{⽕}
  \begin{Phonetics}{炒}{chao3}[][HSK 6]
    \definition{v.}{saltear; refogar; aquecer os alimentos em uma panela e mexer repetidamente para cozinhá-los ou secá-los | especular (na bolsa de valores, etc.) | exagerar; dar publicidade exagerada; a fim de ampliar a influência, por meio de publicidade repetida e exagerada na mídia | demitir; despedir}
  \end{Phonetics}
\end{Entry}

\begin{Entry}{炒作}{8,7}{⽕、⼈}
  \begin{Phonetics}{炒作}{chao3 zuo4}[][HSK 6]
    \definition{v.}{promover (na mídia); exagerar artificialmente e promover ou desvalorizar de forma inadequada | especular; comprar e vender frequentemente no mercado de negociação para obter lucros}
  \end{Phonetics}
\end{Entry}

\begin{Entry}{炒股}{8,8}{⽕、⾁}
  \begin{Phonetics}{炒股}{chao3 gu3}[][HSK 6]
    \definition{v.+compl.}{especular em ações; comprar e vender ações; jogar no mercado}
  \end{Phonetics}
\end{Entry}

\begin{Entry}{炮}{9}{⽕}
  \begin{Phonetics}{炮}{bao1}
    \definition{v.}{processar; o método de preparação da medicina chinesa é colocar as ervas cruas em uma panela de ferro em alta temperatura e fritá-las até que fiquem marrons e estourem | secar alimentos pelo calor; refogar}
  \end{Phonetics}
  \begin{Phonetics}{炮}{pao2}
    \definition{v.}{(medicina tradicional chinesa) preparar a medicina chinesa assando-a em uma panela de ferro quente até dourar e estalar}
  \end{Phonetics}
  \begin{Phonetics}{炮}{pao4}[][HSK 6]
    \definition{s.}{arma grande; canhão; peça de artilharia | fogo de artifício | buraco de explosão cheio de dinamite | canhão, uma das peças do xadrez chinês}
  \end{Phonetics}
\end{Entry}

\begin{Entry}{炸}{9}{⽕}
  \begin{Phonetics}{炸}{zha2}
    \definition{v.}{explodir; estourar; romper | dinamitar; bombardear; explodir; detonar com explosivos | encolerizar-se; explodir em fúria | correr; fugir em pânico}
  \end{Phonetics}
  \begin{Phonetics}{炸}{zha4}[][HSK 6]
    \definition{v.}{fritar em gordura ou óleo | escaldar (como forma de cozinhar)}
  \end{Phonetics}
\end{Entry}

\begin{Entry}{炸药}{9,9}{⽕、⾋}
  \begin{Phonetics}{炸药}{zha4 yao4}[][HSK 6]
    \definition[包,种]{s.}{explosivo; cargas explosivas; dinamite; substâncias que explodem quando aquecidas ou impactadas, produzindo grandes quantidades de energia e gases de alta temperatura, como dinamite e pólvora negra}
  \end{Phonetics}
\end{Entry}

\begin{Entry}{炸弹}{9,11}{⽕、⼸}
  \begin{Phonetics}{炸弹}{zha4 dan4}[][HSK 6]
    \definition{s.}{bomba; uma arma com invólucro de ferro e explosivos dentro que explodem quando um fusível é acionado, geralmente lançada de um avião}
  \end{Phonetics}
\end{Entry}

\begin{Entry}{点}{9}{⽕}
  \begin{Phonetics}{点}{dian3}[][HSK 1]
    \definition{clas.}{hora cheia | ponto, uma unidade de medida para tipos; antigamente, a contagem do tempo durante a noite era feita por turnos, sendo cada turno dividido em cinco pontos | quantidade ínfima; um pouco; um pouquinho; alguma coisa; indica uma pequena quantidade | usado para itens}
    \definition{s.}{gota (de líquido); (ponto) pequena gota de líquido | mancha; ponto; salpico; (um pouco) Um pequeno vestígio | (ponto) Traço de um caractere chinês, cuja forma é ``、''  | ponto; (matemática) refere-se a uma figura geométrica que não tem comprimento, largura ou altura, mas apenas uma posição | gongo, instrumento musical de metal | ponto decimal; refere-se ao ponto decimal, símbolo matemático que representa os números decimais | lugar específico | lanche leve; petisco | lugar; grau; sinalização de um determinado local ou grau | hora marcada; hora regulamentar | aspecto; característica; partes ou aspectos específicos de algo | ritmo; batida}
    \definition{v.}{andar na ponta dos pés | dar uma dica, sugestão | tocar levemente com o dedo, pincel ou vara; tocar muito brevemente; passar rapidamente | acenar; baixar ligeiramente a cabeça e levantar rapidamente | gotejar; fazer cair líquido | semear em buracos; plantar com um plantador | verificar um por um | colocar um ponto; usar caneta e outras ferramentas para adicionar ideias | sugerir; indicar; dar uma dica | decorar; realçar | selecionar; escolher; especificar o que é exigido | acender; queimar; inflamar | (pedido) comer uma pequena quantidade de comida para saciar a fome}
  \end{Phonetics}
\end{Entry}

\begin{Entry}{点火}{9,4}{⽕、⽕}
  \begin{Phonetics}{点火}{dian3huo3}
    \definition{s.}{ignição}
    \definition{v.}{inflamar | acender um fogo | agitar | dar partida em um motor | (figurativo) provocar problemas}
  \end{Phonetics}
\end{Entry}

\begin{Entry}{点头}{9,5}{⽕、⼤}
  \begin{Phonetics}{点头}{dian3 tou2}[][HSK 2]
    \definition{v.}{acenar com a cabeça; balançar a cabeça; mover ligeiramente a cabeça para baixo; indicar permissão, aprovação, compreensão ou saudação}
  \end{Phonetics}
\end{Entry}

\begin{Entry}{点名}{9,6}{⽕、⼝}
  \begin{Phonetics}{点名}{dian3 ming2}[][HSK 4]
    \definition{v.}{fazer a lista de chamada; manter o controle da presença de alguém; chamar nomes para controle de presença | mencionar alguém pelo nome}
  \end{Phonetics}
\end{Entry}

\begin{Entry}{点燃}{9,16}{⽕、⽕}
  \begin{Phonetics}{点燃}{dian3 ran2}[][HSK 5]
    \definition{v.}{acender; inflamar; acender uma fogueira, para iluminar}
  \end{Phonetics}
\end{Entry}

\begin{Entry}{炼}{9}{⽕}
  \begin{Phonetics}{炼}{lian4}
    \definition{v.}{fundir; refinar | temperar (um metal) com fogo | pesar a palavra; procurar a frase certa; polir | trabalhar; tornar uma substância pura ou resistente por aquecimento, etc. | polir; fazer as palavras bonitas e concisas}
  \end{Phonetics}
\end{Entry}

\begin{Entry}{烂}{9}{⽕}
  \begin{Phonetics}{烂}{lan4}[][HSK 5]
    \definition{adj.}{macio; pastoso; amassado | podre; deteriorado | quebrado; esfarrapado; gasto | desorganizado; indigno}
    \definition{adv.}{totalmente; extremamente; completamente; expressa um grau muito profundo}
    \definition{v.}{apodrecer; infeccionar; decompor-se}
  \end{Phonetics}
\end{Entry}

\begin{Entry}{烈}{10}{⽕}
  \begin{Phonetics}{烈}{lie4}
    \definition*{s.}{Sobrenome Lie}
    \definition{adj.}{forte; violento; intenso; feroz | justo; severo | firme; convicto; rigoroso}
    \definition{s.}{pessoa que morreu por uma causa justa | conquistas; façanhas | mártir sacrificando-se por uma causa justa}
  \end{Phonetics}
\end{Entry}

\begin{Entry}{烈士}{10,3}{⽕、⼠}
  \begin{Phonetics}{烈士}{lie4shi4}
    \definition{s.}{mártir}
  \end{Phonetics}
\end{Entry}

\begin{Entry}{烟}{10}{⽕}
  \begin{Phonetics}{烟}{yan1}[][HSK 3]
    \definition[股,支,根,盒,包]{s.}{fumaça; gás produzido pela combustão de materiais, misturado com pequenas partículas não completamente queimadas | névoa; neblina | tabaco; planta de tabaco | fumo; cigarro; termo geral para cigarros, charutos, etc. | ópio | fuligem; fumaça de carvão}
    \definition{v.}{ficar irritado com a fumaça (os olhos lacrimejam ou não conseguem abrir)}
  \end{Phonetics}
\end{Entry}

\begin{Entry}{烟火}{10,4}{⽕、⽕}
  \begin{Phonetics}{烟火}{yan1huo3}
    \definition{s.}{fogo de artifício}
  \end{Phonetics}
\end{Entry}

\begin{Entry}{烟叶}{10,5}{⽕、⼝}
  \begin{Phonetics}{烟叶}{yan1ye4}
    \definition{s.}{folha de tabaco}
  \end{Phonetics}
\end{Entry}

\begin{Entry}{烟头}{10,5}{⽕、⼤}
  \begin{Phonetics}{烟头}{yan1tou2}
    \definition[根]{s.}{bituca de cigarro}
  \end{Phonetics}
\end{Entry}

\begin{Entry}{烟囱}{10,7}{⽕、⼞}
  \begin{Phonetics}{烟囱}{yan1cong1}
    \definition{s.}{chaminé}
  \end{Phonetics}
\end{Entry}

\begin{Entry}{烟花}{10,7}{⽕、⾋}
  \begin{Phonetics}{烟花}{yan1 hua1}[][HSK 6]
    \definition[场,朵]{s.}{fogos de artifício; uma coisa que emite faíscas de várias cores quando exposta à observação | prostituta; antigamente, referia-se a algo relacionado à prostituição}
  \end{Phonetics}
\end{Entry}

\begin{Entry}{烟雨}{10,8}{⽕、⾬}
  \begin{Phonetics}{烟雨}{yan1yu3}
    \definition{s.}{chuvisco | garoa}
  \end{Phonetics}
\end{Entry}

\begin{Entry}{烟草}{10,9}{⽕、⾋}
  \begin{Phonetics}{烟草}{yan1cao3}
    \definition{s.}{tabaco}
  \end{Phonetics}
\end{Entry}

\begin{Entry}{烤}{10}{⽕}
  \begin{Phonetics}{烤}{kao3}
    \definition{v.}{assar | grelhar}
  \end{Phonetics}
\end{Entry}

\begin{Entry}{烤肉}{10,6}{⽕、⾁}
  \begin{Phonetics}{烤肉}{kao3 rou4}[][HSK 5]
    \definition[块,串,片,盘]{s.}{churrasco (literalmente carne assada)}
  \end{Phonetics}
\end{Entry}

\begin{Entry}{烤鸭}{10,10}{⽕、⿃}
  \begin{Phonetics}{烤鸭}{kao3ya1}[][HSK 5]
    \definition[只,盘]{s.}{pato assado; pato recheado e assado em um forno especial após ser abatido}
  \end{Phonetics}
\end{Entry}

\begin{Entry}{烦}{10}{⽕}
  \begin{Phonetics}{烦}{fan2}[][HSK 4]
    \definition{adj.}{redundante e confuso | supérfluo e confuso; muito bagunçado}
    \definition{v.}{aborrecer | irritar; incomodar; estar cansado de; ficar irritado | incomodar; solicitar}
  \end{Phonetics}
\end{Entry}

\begin{Entry}{烧}{10}{⽕}
  \begin{Phonetics}{烧}{shao1}[][HSK 4]
    \definition[次]{s.}{febre; temperatura corporal mais alta do que o normal}
    \definition{v.}{queimar; pegar fogo | cozinhar; aquecer; assar | guisar depois de fritar ou fritar depois de guisar | assar; grelhar os ingredientes dos alimentos diretamente sobre o fogo | ter febre; estar com febre | danificar (matar ou murchar) as plantas pelo uso excessivo (ou inadequado) de fertilizantes | tornar-se arrogante ou presunçoso; metáfora de estar em uma boa posição e se deixar levar}
  \end{Phonetics}
\end{Entry}

\begin{Entry}{烧烤}{10,10}{⽕、⽕}
  \begin{Phonetics}{烧烤}{shao1kao3}
    \definition{s.}{churrasco}
    \definition{v.}{assar}
  \end{Phonetics}
\end{Entry}

\begin{Entry}{热}{10}{⽕}
  \begin{Phonetics}{热}{re4}[][HSK 1]
    \definition{adj.}{quente; temperatura elevada | ardente; caloroso; profundamente afetuoso | ansioso; invejoso; descreve inveja e desejo de possuir algo | térmico; altamente radioativo | popular; muito procurado; muito apreciado por muitas pessoas}
    \definition{s.}{calor; energia liberada pelo movimento irregular das moléculas dentro de um objeto | febre; febre alta causada por doença | moda passageira; mania; febre}
    \definition{v.}{aquecer (geralmente se refere a alimentos)}
  \end{Phonetics}
\end{Entry}

\begin{Entry}{热门}{10,3}{⽕、⾨}
  \begin{Phonetics}{热门}{re4men2}[][HSK 5]
    \definition{adj.}{popular}
    \definition{s.}{algo que desperta o interesse popular; metáfora para algo que está na moda e recebe a atenção de todos (em contraste com 冷门)}
  \seealsoref{冷门}{leng3men2}
  \end{Phonetics}
\end{Entry}

\begin{Entry}{热心}{10,4}{⽕、⼼}
  \begin{Phonetics}{热心}{re4xin1}[][HSK 4]
    \definition{adj.}{ardente; sincero; entusiasmado; afetuoso; apaixonado; interessado}
    \definition{v.}{ser entusiasmado com alguma coisa}
  \end{Phonetics}
\end{Entry}

\begin{Entry}{热水}{10,4}{⽕、⽔}
  \begin{Phonetics}{热水}{re4 shui3}[][HSK 6]
    \definition{s.}{água quente; água em temperatura mais alta}
  \end{Phonetics}
\end{Entry}

\begin{Entry}{热水器}{10,4,16}{⽕、⽔、⼝}
  \begin{Phonetics}{热水器}{re4 shui3 qi4}[][HSK 6]
    \definition[台]{s.}{aquecedor de água; aparelhos que aquecem água usando eletricidade, gás natural, gás liquefeito de petróleo ou energia solar}
  \end{Phonetics}
\end{Entry}

\begin{Entry}{热血沸腾}{10,6,8,13}{⽕、⾎、⽔、⾁}
  \begin{Phonetics}{热血沸腾}{re4xue4fei4teng2}
    \definition{expr.}{ferver o sangue | apaixonar-se}
  \end{Phonetics}
\end{Entry}

\begin{Entry}{热泪盈眶}{10,8,9,11}{⽕、⽔、⽫、⽬}
  \begin{Phonetics}{热泪盈眶}{re4lei4ying2kuang4}
    \definition{expr.}{olhos cheios de lágrimas de emoção | extremamente emocionado}
  \end{Phonetics}
\end{Entry}

\begin{Entry}{热线}{10,8}{⽕、⽷}
  \begin{Phonetics}{热线}{re4 xian4}[][HSK 6]
    \definition[条]{s.}{raio infravermelho | linha direta; \emph{hot line}; uma linha telefônica ou telegráfica direta; uma linha para um ponto de acesso | rota quente (ou movimentada, popular) | raio de calor}
  \end{Phonetics}
\end{Entry}

\begin{Entry}{热闹}{10,8}{⽕、⾾}
  \begin{Phonetics}{热闹}{re4nao5}[][HSK 4]
    \definition{adj.}{animado; agitado; movimentado com barulho e excitação; descreve uma cena animada com uma atmosfera calorosa}
    \definition{s.}{uma vista emocionante; uma cena de agitação e excitação; atmosfera acolhedora}
    \definition{v.}{animar; divertir-se}
  \end{Phonetics}
\end{Entry}

\begin{Entry}{热点}{10,9}{⽕、⽕}
  \begin{Phonetics}{热点}{re4 dian3}[][HSK 6]
    \definition{s.}{ponto de acesso; \emph{hotspot}}
  \end{Phonetics}
\end{Entry}

\begin{Entry}{热烈}{10,10}{⽕、⽕}
  \begin{Phonetics}{热烈}{re4lie4}[][HSK 3]
    \definition{adj.}{caloroso; fervoroso; ardente; entusiasmado; excitado}
  \end{Phonetics}
\end{Entry}

\begin{Entry}{热爱}{10,10}{⽕、⽖}
  \begin{Phonetics}{热爱}{re4'ai4}[][HSK 3]
    \definition{v.}{amar ardentemente; amar de coração; ter amor profundo por; amar apaixonadamente}
  \end{Phonetics}
\end{Entry}

\begin{Entry}{热情}{10,11}{⽕、⼼}
  \begin{Phonetics}{热情}{re4qing2}[][HSK 2]
    \definition{adj.}{caloroso; fervoroso; entusiasmado; cordial; descreve sentimentos calorosos por alguém}
    \definition{s.}{entusiasmo; ardor; devoção; calor humano; zelo; sentimentos calorosos}
  \end{Phonetics}
\end{Entry}

\begin{Entry}{热量}{10,12}{⽕、⾥}
  \begin{Phonetics}{热量}{re4 liang4}[][HSK 5]
    \definition{s.}{calor; quantidade de calor; calorias; em física, refere-se à energia transferida entre objetos com temperaturas diferentes, do objeto com temperatura mais alta para o objeto com temperatura mais baixa}
  \end{Phonetics}
\end{Entry}

\begin{Entry}{惨}{11}{⽕}
  \begin{Phonetics}{惨}{can3}[][HSK 6]
    \definition{adj.}{miserável; trágico | cruel; brutal; implacável | desastroso; terrível; esmagador | lamentável; desaventurado | em um grau sério; grau grave; dano grave | selvagem; desumano; vicioso; cruel}
  \end{Phonetics}
\end{Entry}

\begin{Entry}{焊}{11}{⽕}
  \begin{Phonetics}{焊}{han4}
    \definition{v.}{soldar; usar metal fundido para reparar objetos de metal ou conectar peças de metal}
  \end{Phonetics}
\end{Entry}

\begin{Entry}{悲}{12}{⽕}
  \begin{Phonetics}{悲}{bei1}
    \definition{adj.}{triste; pesaroso; melancólico | compassivo; misericordioso}
  \end{Phonetics}
\end{Entry}

\begin{Entry}{悲伤}{12,6}{⽕、⼈}
  \begin{Phonetics}{悲伤}{bei1 shang1}[][HSK 5]
    \definition{adj.}{triste; pesaroso}
  \end{Phonetics}
\end{Entry}

\begin{Entry}{悲观}{12,6}{⽕、⾒}
  \begin{Phonetics}{悲观}{bei1guan1}
    \definition{adj.}{pessimista; negativismo, falta de confiança no futuro (oposto a 乐观)}
  \seealsoref{乐观}{le4guan1}
  \end{Phonetics}
\end{Entry}

\begin{Entry}{悲剧}{12,10}{⽕、⼑}
  \begin{Phonetics}{悲剧}{bei1 ju4}[][HSK 5]
    \definition[出,部]{s.}{tragédia; drama trágico; uma das principais categorias de teatro, caracterizada basicamente pela representação do conflito irreconciliável entre o protagonista e a realidade e seu final trágico | tragédia; evento triste; metáfora para encontro infeliz}
  \end{Phonetics}
\end{Entry}

\begin{Entry}{悲惨}{12,11}{⽕、⽕}
  \begin{Phonetics}{悲惨}{bei1can3}[][HSK 6]
    \definition{adj.}{trágico; miserável; extremamente doloroso e triste}
  \end{Phonetics}
\end{Entry}

\begin{Entry}{焚}{12}{⽕}
  \begin{Phonetics}{焚}{fen2}
    \definition{v.}{queimar}
  \end{Phonetics}
\end{Entry}

\begin{Entry}{焚香}{12,9}{⽕、⾹}
  \begin{Phonetics}{焚香}{fen2xiang1}
    \definition{v.}{queimar incenso}
  \end{Phonetics}
\end{Entry}

\begin{Entry}{焦}{12}{⽕}
  \begin{Phonetics}{焦}{jiao1}
    \definition*{s.}{Sobrenome Jiao}
    \definition{adj.}{queimado; chamuscado; carbonizado | preocupado; ansioso}
    \definition{clas.}{J; Joule, abreviação}
    \definition{pref.}{(química) piro-}
    \definition{s.}{Metalurgia: coque}
  \end{Phonetics}
\end{Entry}

\begin{Entry}{焦点}{12,9}{⽕、⽕}
  \begin{Phonetics}{焦点}{jiao1dian3}[][HSK 6]
    \definition{s.}{foco; ponto focal; Matemática: refere-se a um ponto que tem uma relação especial com uma elipse, hipérbole, parábola, etc. | foco; ponto focal; Óptica: refere-se à intersecção de feixes de luz paralelos após serem refratados por uma lente ou refletidos por um espelho curvo | foco; questão central; metaforicamente, uma coisa ou princípio que chama a atenção para o foco}
  \end{Phonetics}
\end{Entry}

\begin{Entry}{焦虑}{12,10}{⽕、⾌}
  \begin{Phonetics}{焦虑}{jiao1lv4}
    \definition{adj.}{ansioso | preocupado | apreensivo}
  \end{Phonetics}
\end{Entry}

\begin{Entry}{然}{12}{⽕}
  \begin{Phonetics}{然}{ran2}
    \definition{conj.}{mas | no entanto}
  \end{Phonetics}
\end{Entry}

\begin{Entry}{然后}{12,6}{⽕、⼝}
  \begin{Phonetics}{然后}{ran2hou4}[][HSK 2]
    \definition{conj.}{então; depois disso; posteriormente; indica que algo segue após uma ação ou situação}
  \end{Phonetics}
\end{Entry}

\begin{Entry}{然而}{12,6}{⽕、⽽}
  \begin{Phonetics}{然而}{ran2'er2}[][HSK 4]
    \definition{conj.}{ainda; mas; contudo; todavia; usado no início de uma frase para indicar uma transição; para indicar uma transição, geralmente é precedido por uma conjunção como 虽然 para indicar concessão}
  \seealsoref{虽然}{sui1 ran2}
  \end{Phonetics}
\end{Entry}

\begin{Entry}{煮}{12}{⽕}
  \begin{Phonetics}{煮}{zhu3}[][HSK 6]
    \definition*{s.}{Sobrenome Zhu}
    \definition{v.}{ferver; cozinhar; aquecer alimentos ou outros itens em água}
  \end{Phonetics}
\end{Entry}

\begin{Entry}{煎}{13}{⽕}
  \begin{Phonetics}{煎}{jian1}
    \definition{v.}{fritar | refogar}
  \end{Phonetics}
\end{Entry}

\begin{Entry}{煎饼}{13,9}{⽕、⾷}
  \begin{Phonetics}{煎饼}{jian1bing3}
    \definition[张]{s.}{jianbing, crepe chinês | panqueca}
  \end{Phonetics}
\end{Entry}

\begin{Entry}{煎蛋}{13,11}{⽕、⾍}
  \begin{Phonetics}{煎蛋}{jian1dan4}
    \definition{s.}{ovos fritos}
  \end{Phonetics}
\end{Entry}

\begin{Entry}{煤}{13}{⽕}
  \begin{Phonetics}{煤}{mei2}[][HSK 5]
    \definition[块,吨,斤,堆]{s.}{carvão; carvão vegetal; minério sólido preto}
  \end{Phonetics}
\end{Entry}

\begin{Entry}{煤气}{13,4}{⽕、⽓}
  \begin{Phonetics}{煤气}{mei2 qi4}[][HSK 5]
    \definition[罐,瓶]{s.}{gás; gás de carvão; gás obtido a partir do processamento do carvão não tem cor nem odor, é tóxico e pode ser queimado ou utilizado como matéria-prima na indústria química | envenenamento por monóxido de carbono}
  \end{Phonetics}
\end{Entry}

\begin{Entry}{照}{13}{⽕}
  \begin{Phonetics}{照}{zhao4}[][HSK 3]
    \definition{adv.}{de acordo com; significa agir de acordo com o original ou com determinados padrões}
    \definition{prep.}{em direção a; na direção de | de acordo com; em conformidade com}
    \definition{s.}{imagem; fotografia | permissão; licença; autorização | brilho; iluminação}
    \definition{v.}{brilhar; iluminar | refletir; espelhar; olhar para sua própria imagem em um espelho, etc. | filmar; fotografar; tirar uma foto (fotografia) | cuidar de; tomar conta de; zelar por | notificar; informar | contrastar; comparar; verificar | entender; compreender}
  \end{Phonetics}
\end{Entry}

\begin{Entry}{照片}{13,4}{⽕、⽚}
  \begin{Phonetics}{照片}{zhao4pian4}[][HSK 2]
    \definition[张,套,幅]{s.}{fotografia, foto, imagem}
  \end{Phonetics}
\end{Entry}

\begin{Entry}{照片子}{13,4,3}{⽕、⽚、⼦}
  \begin{Phonetics}{照片子}{zhao4pian4zi5}
    \definition{v.}{tirar um raio X}
  \end{Phonetics}
\end{Entry}

\begin{Entry}{照片底版}{13,4,8,8}{⽕、⽚、⼴、⽚}
  \begin{Phonetics}{照片底版}{zhao4pian4 di3ban3}
    \definition{s.}{placa fotográfica; negativo fotográfico}
  \end{Phonetics}
\end{Entry}

\begin{Entry}{照亮}{13,9}{⽕、⼇}
  \begin{Phonetics}{照亮}{zhao4liang4}
    \definition{s.}{iluminação}
    \definition{v.}{iluminar}
  \end{Phonetics}
\end{Entry}

\begin{Entry}{照相}{13,9}{⽕、⽬}
  \begin{Phonetics}{照相}{zhao4 xiang4}[][HSK 2]
    \definition{v.+compl.}{fotografar; tirar fotos; tirar uma foto; tirar uma fotografia}
  \end{Phonetics}
\end{Entry}

\begin{Entry}{照相机}{13,9,6}{⽕、⽬、⽊}
  \begin{Phonetics}{照相机}{zhao4xiang4ji1}
    \definition[个,架,部,台,只]{s.}{câmera/máquina fotográfica}
  \end{Phonetics}
\end{Entry}

\begin{Entry}{照准}{13,10}{⽕、⼎}
  \begin{Phonetics}{照准}{zhao4zhun3}
    \definition{s.}{solicitação concedida (uso formal em documento antigo)}
    \definition{v.}{mirar (arma)}
  \end{Phonetics}
\end{Entry}

\begin{Entry}{照样}{13,10}{⽕、⽊}
  \begin{Phonetics}{照样}{zhao4yang4}[][HSK 6]
    \definition{adv.}{como antes; da mesma maneira; isso significa que, embora as condições externas tenham mudado ou sido afetadas, uma determinada situação ou estado permanece inalterado | após um padrão; de um modelo}
    \definition{v.}{fazer algo de uma certa maneira}
  \end{Phonetics}
\end{Entry}

\begin{Entry}{照顾}{13,10}{⽕、⾴}
  \begin{Phonetics}{照顾}{zhao4gu4}[][HSK 2]
    \definition{v.}{cuidar; cuidar de; atender | oferecer tratamento preferencial; prestar atenção especial e dar tratamento preferencial | (de um cliente) patrocinar; comprar em; cuidar de clientes que vêm comprar coisas ou solicitar serviços em lojas ou indústrias de serviços | dar consideração a; mostrar consideração por; levar em conta; fazer concessões a}
  \end{Phonetics}
\end{Entry}

\begin{Entry}{照骗}{13,12}{⽕、⾺}
  \begin{Phonetics}{照骗}{zhao4pian4}
    \definition{s.}{Gíria da \emph{Internet}:  foto lisonjeira (trocadilho com 照片) | imagem alterada digitalmente; ``photoshopada''}
  \seealsoref{照片}{zhao4pian4}
  \end{Phonetics}
\end{Entry}

\begin{Entry}{照像}{13,13}{⽕、⼈}
  \begin{Phonetics}{照像}{zhao4 xiang4}
    \variantof{照相}
  \end{Phonetics}
\end{Entry}

\begin{Entry}{照像机}{13,13,6}{⽕、⼈、⽊}
  \begin{Phonetics}{照像机}{zhao4xiang4ji1}
    \variantof{照相机}
  \end{Phonetics}
\end{Entry}

\begin{Entry}{照耀}{13,20}{⽕、⽻}
  \begin{Phonetics}{照耀}{zhao4yao4}[][HSK 6]
    \definition{v.}{brilhar; iluminar | esclarecer}
  \end{Phonetics}
\end{Entry}

\begin{Entry}{愿}{14}{⽕}
  \begin{Phonetics}{愿}{yuan4}[][HSK 5]
    \definition{adj.}{honesto e prudente}
    \definition{s.}{esperança; desejo; vontade; a ideia de alcançar algum objetivo no futuro | voto (feito perante o Buda ou um deus); o desejo de retribuição feito ao rezar para os deuses e Buda}
    \definition{v.}{estar disposto; estar pronto; de bom grado, concordar porque está de acordo com seus desejos | ter esperança; desejar; qerer alcançar algum desejo}
  \end{Phonetics}
\end{Entry}

\begin{Entry}{愿望}{14,11}{⽕、⽉}
  \begin{Phonetics}{愿望}{yuan4wang4}[][HSK 3]
    \definition[个,种]{s.}{desejo; aspiração; a ideia de alcançar algum objetivo no futuro.}
  \end{Phonetics}
\end{Entry}

\begin{Entry}{愿意}{14,13}{⽕、⼼}
  \begin{Phonetics}{愿意}{yuan4yi4}[][HSK 2]
    \definition{v.}{estar disposto; estar pronto | desejar; ter esperança}
  \end{Phonetics}
\end{Entry}

\begin{Entry}{熊}{14}{⽕}
  \begin{Phonetics}{熊}{xiong2}[][HSK 5]
    \definition*{s.}{Sobrenome Xiong}
    \definition[头,只]{s.}{urso}
    \definition{v.}{repreender; censurar}
  \end{Phonetics}
\end{Entry}

\begin{Entry}{熊猫}{14,11}{⽕、⽝}
  \begin{Phonetics}{熊猫}{xiong2mao1}
    \definition[把,只]{s.}{panda gigante}
  \seealsoref{猫熊}{mao1xiong2}
  \end{Phonetics}
\end{Entry}

\begin{Entry}{熏}{14}{⽕}
  \begin{Phonetics}{熏}{xun1}
    \definition{v.}{expor à fumaça ou vapores; fumigar | tratar (carne, peixe, etc.) com fumaça; defumar | tornar perfumado com incenso, etc. | sufocar (asfixia e envenenamento por gás)}
  \end{Phonetics}
\end{Entry}

\begin{Entry}{熏香}{14,9}{⽕、⾹}
  \begin{Phonetics}{熏香}{xun1xiang1}
    \definition{s.}{incenso}
  \end{Phonetics}
\end{Entry}

\begin{Entry}{熟}{15}{⽕}
  \begin{Phonetics}{熟}{shu2}[][HSK 2]
    \definition{adj.}{maduro (frutos) | pronto; cozido | processado, fabricado ou exercitado | familiar, bem conhecido; conhecido por ser comum ou frequentemente utilizado | habilidoso;  (trabalho, tecnologia) experiente; não é novato | profundo; sólido}
  \end{Phonetics}
\end{Entry}

\begin{Entry}{熟人}{15,2}{⽕、⼈}
  \begin{Phonetics}{熟人}{shu2 ren2}[][HSK 3]
    \definition[位,名,个,些]{s.}{amigo; conhecido; pessoas que se conhecem há muito tempo; pessoas que são muito familiares}
  \end{Phonetics}
\end{Entry}

\begin{Entry}{熟练}{15,8}{⽕、⽷}
  \begin{Phonetics}{熟练}{shu2lian4}[][HSK 4]
    \definition{adj.}{especializado; proficiente; qualificado; habilidoso}
  \end{Phonetics}
\end{Entry}

\begin{Entry}{熟悉}{15,11}{⽕、⼼}
  \begin{Phonetics}{熟悉}{shu2xi1}[][HSK 5]
    \definition{adj.}{familiarizado com; não ser estranho}
    \definition{v.}{estar familiarizado com; saber claramente que | conhecer bem algo ou alguém; compreender e dominar (a situação) através da observação ou da experiência}
  \end{Phonetics}
\end{Entry}

\begin{Entry}{燃}{16}{⽕}
  \begin{Phonetics}{燃}{ran2}
    \definition{v.}{queimar | acender; inflamar}
  \end{Phonetics}
\end{Entry}

\begin{Entry}{燃料}{16,10}{⽕、⽃}
  \begin{Phonetics}{燃料}{ran2 liao4}[][HSK 4]
    \definition[种]{s.}{combustível; carburante; substâncias que podem gerar calor e energia luminosa quando queimadas podem ser divididas em três tipos de acordo com sua forma: combustível sólido (como carvão, carvão vegetal, madeira), combustível líquido (como gasolina, querosene) e combustível gasoso (como gás de carvão, biogás); também se refere a substâncias que podem gerar energia nuclear, como urânio, plutônio, etc.}
  \end{Phonetics}
\end{Entry}

\begin{Entry}{燃烧}{16,10}{⽕、⽕}
  \begin{Phonetics}{燃烧}{ran2shao1}[][HSK 4]
    \definition{v.}{queimar; acender | arder; inflamar; ferver; metáfora para as emoções de uma pessoa serem muito fortes, como um fogo ardente}
  \end{Phonetics}
\end{Entry}

\begin{Entry}{爆}{19}{⽕}
  \begin{Phonetics}{爆}{bao4}[][HSK 6]
    \definition{v.}{explodir; estourar | fritar rapidamente; ferver rapidamente | aparecer (ou ocorrer) inesperadamente}
  \end{Phonetics}
\end{Entry}

\begin{Entry}{爆发}{19,5}{⽕、⼜}
  \begin{Phonetics}{爆发}{bao4fa1}[][HSK 6]
    \definition{v.}{entrar em erupção; explodir | estourar; irromper; ocorrer de forma repentina e violenta}
  \end{Phonetics}
\end{Entry}

\begin{Entry}{爆米花}{19,6,7}{⽕、⽶、⾋}
  \begin{Phonetics}{爆米花}{bao4mi3hua1}
    \definition{s.}{pipoca (de milho) | pipoca de arroz}
  \end{Phonetics}
\end{Entry}

\begin{Entry}{爆炸}{19,9}{⽕、⽕}
  \begin{Phonetics}{爆炸}{bao4zha4}[][HSK 6]
    \definition{s.}{explosão}
    \definition{v.}{explodir; explodir; detonar | aumentar bruscamente em um curto espaço de tempo (de quantidade)}
  \end{Phonetics}
\end{Entry}

%%%%% EOF %%%%%

