%%%
%%% Radical "⽈"
%%%

\section*{Radical 73: ``⽈''}\addcontentsline{toc}{section}{Radical 73: ⽈}

\begin{entry}{曲棍球}{6,12,11}{⽈、⽊、⽟}
  \begin{phonetics}{曲棍球}{qu1gun4qiu2}
    \definition{s.}{hóquei em campo}
  \end{phonetics}
\end{entry}

\begin{entry}{更}{7}{⽈}
  \begin{phonetics}{更}{geng1}
    \definition*{s.}{sobrenome Geng}
    \definition{clas.}{um dos cinco períodos de duas horas em que a noite era anteriormente dividida; vigília; antigamente, a noite era dividida em cinco turnos, cada um com aproximadamente duas horas de duração}
    \definition{v.}{alterar; substituir | experimentar}
  \end{phonetics}
  \begin{phonetics}{更}{geng4}[][HSK 2]
    \definition{adv.}{mais; ainda mais | além disso; além do mais; ainda mais}
  \end{phonetics}
\end{entry}

\begin{entry}{更加}{7,5}{⽈、⼒}
  \begin{phonetics}{更加}{geng4 jia1}[][HSK 3]
    \definition{adv.}{mais; ainda mais; em maior grau; indica um nível mais profundo ou um aumento ou diminuição quantitativa adicional}
  \end{phonetics}
\end{entry}

\begin{entry}{更换}{7,10}{⽈、⼿}
  \begin{phonetics}{更换}{geng1 huan4}[][HSK 5]
    \definition{v.}{alterar; mudar; substituir; comutar}
  \end{phonetics}
\end{entry}

\begin{entry}{更新}{7,13}{⽈、⽄}
  \begin{phonetics}{更新}{geng1xin1}[][HSK 5]
    \definition{v.}{renovar; atualizar; substituir; remover o antigo e substituir pelo novo}
  \end{phonetics}
\end{entry}

\begin{entry}{曾}{12}{⽈}
  \begin{phonetics}{曾}{ceng2}[][HSK 4]
    \definition{adv.}{uma vez; antigamente; há algum tempo; usado para indicar ação ou estado passado}
  \end{phonetics}
  \begin{phonetics}{曾}{zeng1}
    \definition*{s.}{sobrenome Zeng}
    \definition{s.}{relacionamento entre bisnetos e bisavós}
  \end{phonetics}
\end{entry}

\begin{entry}{曾经}{12,8}{⽈、⽷}
  \begin{phonetics}{曾经}{ceng2jing1}[][HSK 3]
    \definition{adv.}{uma vez; indica que houve algum comportamento ou situação}
  \end{phonetics}
\end{entry}

\begin{entry}{替}{12}{⽈}
  \begin{phonetics}{替}{ti4}[][HSK 4]
    \definition{prep.}{para; em nome de}
    \definition{s.}{decadência; declínio; enfraquecimento}
    \definition{v.}{substituir; substituir por; tomar o lugar de}
  \end{phonetics}
\end{entry}

\begin{entry}{替代}{12,5}{⽈、⼈}
  \begin{phonetics}{替代}{ti4 dai4}[][HSK 4]
    \definition{v.}{substituir; suplantar}
  \end{phonetics}
\end{entry}

\begin{entry}{最}{12}{⽈}
  \begin{phonetics}{最}{zui4}[][HSK 1]
    \definition{adv.}{(diante de um adjetivo ou verbo) o mais | (colocado antes de um substantivo de localidade ou de uma palavra que indica um lugar)  mais distante ou mais próximo de (um lugar) | mais; melhor; pior; primeiro; muito; menos; acima de tudo; indica que uma determinada característica excede todas as outras pessoas ou coisas do mesmo tipo}
    \definition{s.}{o máximo; o melhor (ou o mais alto, o maior, etc.)}
  \end{phonetics}
\end{entry}

\begin{entry}{最少}{12,4}{⽈、⼩}
  \begin{phonetics}{最少}{zui4shao3}
    \definition{adv.}{finalmente}
  \end{phonetics}
\end{entry}

\begin{entry}{最优}{12,6}{⽈、⼈}
  \begin{phonetics}{最优}{zui4you1}
    \definition{adj.}{ótimo}
  \end{phonetics}
\end{entry}

\begin{entry}{最先}{12,6}{⽈、⼉}
  \begin{phonetics}{最先}{zui4xian1}
    \definition{adv.}{o primeiro}
  \end{phonetics}
\end{entry}

\begin{entry}{最后}{12,6}{⽈、⼝}
  \begin{phonetics}{最后}{zui4hou4}[][HSK 1]
    \definition{s.}{último; final; definitivo; refere-se ao tempo, local, etc. que vem depois de outros tempos, locais, etc. na ordem sequencial}
  \end{phonetics}
\end{entry}

\begin{entry}{最多}{12,6}{⽈、⼣}
  \begin{phonetics}{最多}{zui4duo1}
    \definition{adv.}{no máximo | máximo}
  \end{phonetics}
\end{entry}

\begin{entry}{最好}{12,6}{⽈、⼥}
  \begin{phonetics}{最好}{zui4hao3}[][HSK 1]
    \definition{adj.}{melhor; de primeira qualidade; excelente}
    \definition{adv.}{seria melhor; seria o ideal; indica a escolha mais adequada entre várias possibilidades}
  \end{phonetics}
\end{entry}

\begin{entry}{最初}{12,7}{⽈、⾐}
  \begin{phonetics}{最初}{zui4chu1}[][HSK 4]
    \definition{adj.}{primordial; inicial; primeiro}
    \definition{adv.}{inicialmente; originalmente}
    \definition{s.}{o período mais antigo; início; começo}
  \end{phonetics}
\end{entry}

\begin{entry}{最近}{12,7}{⽈、⾡}
  \begin{phonetics}{最近}{zui4jin4}[][HSK 2]
    \definition{adj.}{mais próximo}
    \definition{s.}{recentemente; ultimamente; de tarde; refere-se aos dias antes ou logo depois de um discurso | em breve; no futuro próximo; o futuro próximo}
  \end{phonetics}
\end{entry}

\begin{entry}{最远}{12,7}{⽈、⾡}
  \begin{phonetics}{最远}{zui4yuan3}
    \definition{adv.}{mais distante | mais longe}
  \end{phonetics}
\end{entry}

\begin{entry}{最佳}{12,8}{⽈、⼈}
  \begin{phonetics}{最佳}{zui4jia1}
    \definition{adj.}{melhor (atleta, filme etc) | ótimo}
  \end{phonetics}
\end{entry}

\begin{entry}{最终}{12,8}{⽈、⽷}
  \begin{phonetics}{最终}{zui4zhong1}
    \definition{adv.}{pelo menos | finalmente}
    \definition{s.}{final | ultimato}
  \end{phonetics}
\end{entry}

\begin{entry}{最高}{12,10}{⽈、⾼}
  \begin{phonetics}{最高}{zui4gao1}
    \definition{adj.}{altíssimo | supremo | mais alto}
  \end{phonetics}
\end{entry}

\begin{entry}{最善}{12,12}{⽈、⼝}
  \begin{phonetics}{最善}{zui4shan4}
    \definition{adj.}{ótimo | o melhor}
  \end{phonetics}
\end{entry}

\begin{entry}{最新}{12,13}{⽈、⽄}
  \begin{phonetics}{最新}{zui4xin1}
    \definition{adv.}{mais recente | mais novo}
  \end{phonetics}
\end{entry}

%%%%% EOF %%%%%

