%%%
%%% Radical "⽯"
%%%

\section*{Radical 112: ``⽯''}\addcontentsline{toc}{section}{Radical 112: ⽯}

\begin{entry}{石头}{5,5}{⽯、⼤}
  \begin{phonetics}{石头}{shi2tou5}[][HSK 3]
    \definition[块,堆,些]{s.}{rocha; pedra}
  \end{phonetics}
\end{entry}

\begin{entry}{石油}{5,8}{⽯、⽔}
  \begin{phonetics}{石油}{shi2you2}[][HSK 3]
    \definition[桶,吨,升]{s.}{óleo; óleo fóssil; petróleo}
  \end{phonetics}
\end{entry}

\begin{entry}{矿泉水}{8,9,4}{⽯、⽔、⽔}
  \begin{phonetics}{矿泉水}{kuang4quan2shui3}[][HSK 4]
    \definition[瓶,杯]{s.}{água mineral de nascente}
  \end{phonetics}
\end{entry}

\begin{entry}{码头}{8,5}{⽯、⼤}
  \begin{phonetics}{码头}{ma3tou2}[][HSK 5]
    \definition[个]{s.}{doca; cais; píer; molhe; edifícios à beira-mar ou à beira do rio destinados exclusivamente à atracação de embarcações, embarque e desembarque de passageiros e carga e descarga de mercadorias | cidade portuária; centro comercial e de transportes; refere-se a uma cidade comercial com transporte terrestre e marítimo bem desenvolvido.}
  \end{phonetics}
\end{entry}

\begin{entry}{砂}{9}{⽯}
  \begin{phonetics}{砂}{sha1}
    \variantof{沙}
  \end{phonetics}
\end{entry}

\begin{entry}{砍}{9}{⽯}
  \begin{phonetics}{砍}{kan3}
    \definition{v.}{cortar}
  \end{phonetics}
\end{entry}

\begin{entry}{砍刀}{9,2}{⽯、⼑}
  \begin{phonetics}{砍刀}{kan3dao1}
    \definition{s.}{facão | machete}
  \end{phonetics}
\end{entry}

\begin{entry}{砍头}{9,5}{⽯、⼤}
  \begin{phonetics}{砍头}{kan3tou2}
    \definition{v.}{decapitar}
  \end{phonetics}
\end{entry}

\begin{entry}{砍价}{9,6}{⽯、⼈}
  \begin{phonetics}{砍价}{kan3jia4}
    \definition{v.}{barganhar | cortar ou derrubar um preço}
  \end{phonetics}
\end{entry}

\begin{entry}{砍伤}{9,6}{⽯、⼈}
  \begin{phonetics}{砍伤}{kan3shang1}
    \definition{v.}{ferir com lâmina ou machado}
  \end{phonetics}
\end{entry}

\begin{entry}{砍杀}{9,6}{⽯、⽊}
  \begin{phonetics}{砍杀}{kan3sha1}
    \definition{v.}{atacar com arma branca}
  \end{phonetics}
\end{entry}

\begin{entry}{砍死}{9,6}{⽯、⽍}
  \begin{phonetics}{砍死}{kan3si3}
    \definition{v.}{matar com um machado}
  \end{phonetics}
\end{entry}

\begin{entry}{砍树}{9,9}{⽯、⽊}
  \begin{phonetics}{砍树}{kan3shu4}
    \definition{v.}{derrubar árvores}
  \end{phonetics}
\end{entry}

\begin{entry}{砍掉}{9,11}{⽯、⼿}
  \begin{phonetics}{砍掉}{kan3diao4}
    \definition{v.}{amputar}
  \end{phonetics}
\end{entry}

\begin{entry}{砍断}{9,11}{⽯、⽄}
  \begin{phonetics}{砍断}{kan3duan4}
    \definition{v.}{cortar}
  \end{phonetics}
\end{entry}

\begin{entry}{研究}{9,7}{⽯、⽳}
  \begin{phonetics}{研究}{yan2jiu1}[][HSK 4]
    \definition{v.}{estudar; pesquisar | discutir; considerar}
  \end{phonetics}
\end{entry}

\begin{entry}{研究生}{9,7,5}{⽯、⽳、⽣}
  \begin{phonetics}{研究生}{yan2 jiu1 sheng1}[][HSK 4]
    \definition[位,名]{s.}{pós-graduado; estudante de pós-graduação}
  \end{phonetics}
\end{entry}

\begin{entry}{研究所}{9,7,8}{⽯、⽳、⼾}
  \begin{phonetics}{研究所}{yan2 jiu1 suo3}[][HSK 5]
    \definition[个]{s.}{instituto de pesquisa; instituição de pesquisa científica envolvida em pesquisas em um determinado campo}
  \end{phonetics}
\end{entry}

\begin{entry}{研制}{9,8}{⽯、⼑}
  \begin{phonetics}{研制}{yan2 zhi4}[][HSK 4]
    \definition{v.}{desenvolver; fabricar; produzir | triturar; (medicina chinesa) moer}
  \end{phonetics}
\end{entry}

\begin{entry}{砖}{9}{⽯}
  \begin{phonetics}{砖}{zhuan1}
    \definition[块]{s.}{tijolo}
  \end{phonetics}
\end{entry}

\begin{entry}{破}{10}{⽯}
  \begin{phonetics}{破}{po4}[][HSK 3]
    \definition{adj.}{quebrado; danificado; rasgado; desgastado | pobre; ruim; insignificante; péssimo; miserável}
    \definition{v.}{estar quebrado; estar danificado | quebrar; avariar; danificar | quebrar; dividir; cortar; cinzelar | trocar (dinheiro) | romper; quebrar (avanço) | livrar-se de; destruir; romper com
derrotar; capturar (uma cidade, etc.) | despender; gastar (dinheiro) | expor a verdade de; desnudar}
  \end{phonetics}
\end{entry}

\begin{entry}{破产}{10,6}{⽯、⼇}
  \begin{phonetics}{破产}{po4chan3}[][HSK 4]
    \definition{v.+compl.}{falir; ir à falência; tornar-se insolvente; entrar em liquidação; perder todo o patrimônio | falhar; fracassar; não dar em nada; figura de linguagem (geralmente com uma conotação depreciativa)}
  \end{phonetics}
\end{entry}

\begin{entry}{破坏}{10,7}{⽯、⼟}
  \begin{phonetics}{破坏}{po4huai4}[][HSK 3]
    \definition{s.}{destruição | dano}
    \definition{v.}{demolir; naufragar; soçobrar; destruir; obliterar | quebrar; violar (um acordo, regulamento, etc.) | prejudicar; perturbar; sabotar; causar grande dano | reverter; mudar (um sistema social, costume, etc.) completamente ou violentamente | destruir; decompor}
  \end{phonetics}
\end{entry}

\begin{entry}{破坏性}{10,7,8}{⽯、⼟、⼼}
  \begin{phonetics}{破坏性}{po4huai4xing4}
    \definition{adj.}{destrutivo}
    \definition{s.}{poder destrutivo}
  \end{phonetics}
\end{entry}

\begin{entry}{砸}{10}{⽯}
  \begin{phonetics}{砸}{za2}
    \definition{v.}{esmagar | bater | falhar | estragar}
  \end{phonetics}
\end{entry}

\begin{entry}{硕士}{11,3}{⽯、⼠}
  \begin{phonetics}{硕士}{shuo4shi4}[][HSK 5]
    \definition[个,位,名]{s.}{mestrado}
  \end{phonetics}
\end{entry}

\begin{entry}{硬}{12}{⽯}
  \begin{phonetics}{硬}{ying4}[][HSK 4,5]
    \definition{adj.}{duro; rígido; resistente;  objeto resistente e não se deforma facilmente quando submetido a forças externas (em oposição a 软) | firme; forte; resistente; obstinado; (vontade, atitude, etc.) inabalável, forte e poderoso | capaz (pessoa); boa (qualidade) | rígido; severo; sem flexibilidade | duro; rígido; rigoroso; imutável}
    \definition{adv.}{conseguir fazer algo com dificuldade; indica fazer algo à força, independentemente das circunstâncias}
  \seealsoref{软}{ruan3}
  \end{phonetics}
\end{entry}

\begin{entry}{硬件}{12,6}{⽯、⼈}
  \begin{phonetics}{硬件}{ying4jian4}[][HSK 5]
    \definition{s.}{\emph{hardware}; nome genérico dado aos vários elementos, componentes e dispositivos que constituem um computador | máquina, materiais; equipamento; referência a máquinas, equipamentos, materiais físicos, etc., utilizados nos processos de produção, pesquisa científica, gestão, etc.}
  \end{phonetics}
\end{entry}

\begin{entry}{确}{12}{⽯}
  \begin{phonetics}{确}{que4}
    \definition{adj.}{autenticado | sólido | firme | real | verdadeiro}
  \end{phonetics}
\end{entry}

\begin{entry}{确认}{12,4}{⽯、⾔}
  \begin{phonetics}{确认}{que4ren4}[][HSK 4]
    \definition{v.}{afirmar; confirmar; reconhecer; confirmar explicitamente (fatos, princípios, etc.)}
  \end{phonetics}
\end{entry}

\begin{entry}{确立}{12,5}{⽯、⽴}
  \begin{phonetics}{确立}{que4li4}[][HSK 5]
    \definition{v.}{estabelecer; criar; construir; estabelecer ou consolidar firmemente}
  \end{phonetics}
\end{entry}

\begin{entry}{确定}{12,8}{⽯、⼧}
  \begin{phonetics}{确定}{que4ding4}[][HSK 3]
    \definition{adj.}{definido; certo}
    \definition{v.}{consertar; definir; determinar}
  \end{phonetics}
\end{entry}

\begin{entry}{确实}{12,8}{⽯、⼧}
  \begin{phonetics}{确实}{que4shi2}[][HSK 3]
    \definition{adj.}{verdadeiro; confiável}
    \definition{adv.}{verdadeiramente; realmente; de ​​fato}
  \end{phonetics}
\end{entry}

\begin{entry}{确保}{12,9}{⽯、⼈}
  \begin{phonetics}{确保}{que4bao3}[][HSK 3]
    \definition{v.}{assegurar; garantir}
  \end{phonetics}
\end{entry}

\begin{entry}{碍事}{13,8}{⽯、⼅}
  \begin{phonetics}{碍事}{ai4shi4}
    \definition{s.}{(usualmente em frases negativas) sem consequência, não importa}
    \definition{v.+compl.}{estar no caminho | ser um obstáculo}
  \end{phonetics}
\end{entry}

\begin{entry}{碎}{13}{⽯}
  \begin{phonetics}{碎}{sui4}[][HSK 5]
    \definition*{s.}{sobrenome Sui}
    \definition{adj.}{quebrado; fragmentado | tagarela; falante}
    \definition{v.}{(transitivo ou intransitivo) quebrar em pedaços; esmagar}
  \end{phonetics}
\end{entry}

\begin{entry}{碗}{13}{⽯}
  \begin{phonetics}{碗}{wan3}[][HSK 2]
    \definition*{s.}{sobrenome Wan}
    \definition{clas.}{usado para medição de alimentos e bebidas}
    \definition[只,个]{s.}{tigela | objeto em forma de tigela |}
  \end{phonetics}
\end{entry}

\begin{entry}{碗子}{13,3}{⽯、⼦}
  \begin{phonetics}{碗子}{wan3zi5}
    \definition{s.}{tigela}
  \end{phonetics}
\end{entry}

\begin{entry}{碗柜}{13,8}{⽯、⽊}
  \begin{phonetics}{碗柜}{wan3gui4}
    \definition{s.}{armário}
  \end{phonetics}
\end{entry}

\begin{entry}{碰}{13}{⽯}
  \begin{phonetics}{碰}{peng4}[][HSK 2]
    \definition{v.}{tocar; bater; esbarrar | encontrar; esbarrar | arriscar; tentar | tentar a sorte | reunir-se para discutir; ter uma reunião curta}
  \end{phonetics}
\end{entry}

\begin{entry}{碰见}{13,4}{⽯、⾒}
  \begin{phonetics}{碰见}{peng4 jian4}[][HSK 2]
    \definition{v.}{encontrar; encontrar-se; sem combinar, encontrar-se por acaso}
  \end{phonetics}
\end{entry}

\begin{entry}{碰头}{13,5}{⽯、⼤}
  \begin{phonetics}{碰头}{peng4tou2}
    \definition{s.}{colisão | conflito}
    \definition{v.}{colidir}
    \definition{v.+compl.}{conhecer e discutir | juntar ideias | ver-se}
  \end{phonetics}
\end{entry}

\begin{entry}{碰运气}{13,7,4}{⽯、⾡、⽓}
  \begin{phonetics}{碰运气}{peng4yun4qi5}
    \definition{v.}{deixar algo ao acaso | tentar a sorte}
  \end{phonetics}
\end{entry}

\begin{entry}{碰到}{13,8}{⽯、⼑}
  \begin{phonetics}{碰到}{peng4 dao4}[][HSK 2]
    \definition{v.}{encontrar (com); esbarrar; cruzar}
  \end{phonetics}
\end{entry}

\begin{entry}{碳}{14}{⽯}
  \begin{phonetics}{碳}{tan4}
    \definition{s.}{carbono (elemento químico)}
  \end{phonetics}
\end{entry}

\begin{entry}{磁带}{14,9}{⽯、⼱}
  \begin{phonetics}{磁带}{ci2dai4}
    \definition[盘,盒]{s.}{cassete | fita magnética}
  \end{phonetics}
\end{entry}

\begin{entry}{磁铁}{14,10}{⽯、⾦}
  \begin{phonetics}{磁铁}{ci2tie3}
    \definition{s.}{imã | magneto}
  \seealsoref{吸铁石}{xi1tie3shi2}
  \end{phonetics}
\end{entry}

\begin{entry}{磁盘}{14,11}{⽯、⽫}
  \begin{phonetics}{磁盘}{ci2pan2}
    \definition{s.}{disquete}
  \end{phonetics}
\end{entry}

\begin{entry}{碾碎}{15,13}{⽯、⽯}
  \begin{phonetics}{碾碎}{nian3sui4}
    \definition{v.}{pulverizar | esmagar}
  \end{phonetics}
\end{entry}

\begin{entry}{磨}{16}{⽯}
  \begin{phonetics}{磨}{mo2}
    \definition{v.}{moer | polir | afiar | desgastar | esfregar}
  \end{phonetics}
  \begin{phonetics}{磨}{mo4}
    \definition{s.}{mó (pedra pesada e redonda para moinho)}
    \definition{v.}{moer}
  \end{phonetics}
\end{entry}

\begin{entry}{磨菇}{16,11}{⽯、⾋}
  \begin{phonetics}{磨菇}{mo2gu5}
    \variantof{蘑菇}
  \end{phonetics}
\end{entry}

%%%%% EOF %%%%%

