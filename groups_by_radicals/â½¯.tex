%%%
%%% Radical "⽯"
%%%

\section*{Radical 112: ``⽯''}\addcontentsline{toc}{section}{Radical 112: ⽯}

\begin{Entry}{石}{5}{⽯}[Kangxi 112]
  \begin{Phonetics}{石}{dan4}
    \definition{clas.}{dan, uma unidade de medida seca para grãos; unidade de capacidade, 10 斗 é igual a 1 石}
  \seealsoref{斗}{dou4}
  \end{Phonetics}
  \begin{Phonetics}{石}{shi2}
    \definition*{s.}{Sobrenome Shi}
    \definition{s.}{pedra; rocha; o material duro que constitui a crosta terrestre é composto por uma coleção de minerais | inscrição em pedra; esculturas em pedra}
  \end{Phonetics}
\end{Entry}

\begin{Entry}{石头}{5,5}{⽯、⼤}
  \begin{Phonetics}{石头}{shi2tou5}[][HSK 3]
    \definition[块,堆,些]{s.}{rocha; pedra; uma substância muito dura que é o principal material da superfície da Terra}
  \end{Phonetics}
\end{Entry}

\begin{Entry}{石油}{5,8}{⽯、⽔}
  \begin{Phonetics}{石油}{shi2you2}[][HSK 3]
    \definition[桶,吨,升]{s.}{óleo; óleo fóssil; petróleo; um líquido inflamável extraído do solo, geralmente marrom escuro, preto ou verde escuro, do qual gasolina e outras substâncias podem ser obtidas}
  \end{Phonetics}
\end{Entry}

\begin{Entry}{矿}{8}{⽯}
  \begin{Phonetics}{矿}{kuang4}[][HSK 6]
    \definition[个,座]{s.}{depósito de minério | minério | mina}
  \end{Phonetics}
\end{Entry}

\begin{Entry}{矿泉水}{8,9,4}{⽯、⽔、⽔}
  \begin{Phonetics}{矿泉水}{kuang4quan2shui3}[][HSK 4]
    \definition[瓶,杯,口]{s.}{água mineral de nascente}
  \end{Phonetics}
\end{Entry}

\begin{Entry}{码}{8}{⽯}
  \begin{Phonetics}{码}{ma3}
    \definition{clas.}{refere-se a um assunto específico ou a uma categoria de assuntos; refere-se a uma coisa ou a uma classe de coisas | jarda; unidade de comprimento britânica e americana}
    \definition{s.}{um sinal ou objeto que indica número; código; símbolos ou ferramentas que indicam números}
    \definition{v.}{empilhar; acumular}
  \end{Phonetics}
\end{Entry}

\begin{Entry}{码头}{8,5}{⽯、⼤}
  \begin{Phonetics}{码头}{ma3tou2}[][HSK 5]
    \definition[个,座]{s.}{doca; cais; píer; molhe; edifícios à beira-mar ou à beira do rio destinados exclusivamente à atracação de embarcações, embarque e desembarque de passageiros e carga e descarga de mercadorias | cidade portuária; centro comercial e de transportes; refere-se a uma cidade comercial com transporte terrestre e marítimo bem desenvolvido.}
  \end{Phonetics}
\end{Entry}

\begin{Entry}{砂}{9}{⽯}
  \begin{Phonetics}{砂}{sha1}
    \variantof{沙}
  \end{Phonetics}
\end{Entry}

\begin{Entry}{砍}{9}{⽯}
  \begin{Phonetics}{砍}{kan3}
    \definition{v.}{cortar}
  \end{Phonetics}
\end{Entry}

\begin{Entry}{砍刀}{9,2}{⽯、⼑}
  \begin{Phonetics}{砍刀}{kan3dao1}
    \definition{s.}{facão | machete}
  \end{Phonetics}
\end{Entry}

\begin{Entry}{砍头}{9,5}{⽯、⼤}
  \begin{Phonetics}{砍头}{kan3tou2}
    \definition{v.}{decapitar}
  \end{Phonetics}
\end{Entry}

\begin{Entry}{砍价}{9,6}{⽯、⼈}
  \begin{Phonetics}{砍价}{kan3jia4}
    \definition{v.}{barganhar | cortar ou derrubar um preço}
  \end{Phonetics}
\end{Entry}

\begin{Entry}{砍伤}{9,6}{⽯、⼈}
  \begin{Phonetics}{砍伤}{kan3shang1}
    \definition{v.}{ferir com lâmina ou machado}
  \end{Phonetics}
\end{Entry}

\begin{Entry}{砍杀}{9,6}{⽯、⽊}
  \begin{Phonetics}{砍杀}{kan3sha1}
    \definition{v.}{atacar com arma branca}
  \end{Phonetics}
\end{Entry}

\begin{Entry}{砍死}{9,6}{⽯、⽍}
  \begin{Phonetics}{砍死}{kan3si3}
    \definition{v.}{matar com um machado}
  \end{Phonetics}
\end{Entry}

\begin{Entry}{砍树}{9,9}{⽯、⽊}
  \begin{Phonetics}{砍树}{kan3shu4}
    \definition{v.}{derrubar árvores}
  \end{Phonetics}
\end{Entry}

\begin{Entry}{砍掉}{9,11}{⽯、⼿}
  \begin{Phonetics}{砍掉}{kan3diao4}
    \definition{v.}{amputar}
  \end{Phonetics}
\end{Entry}

\begin{Entry}{砍断}{9,11}{⽯、⽄}
  \begin{Phonetics}{砍断}{kan3duan4}
    \definition{v.}{cortar}
  \end{Phonetics}
\end{Entry}

\begin{Entry}{研}{9}{⽯}
  \begin{Phonetics}{研}{yan2}
    \definition{s.}{(abreviação)  pesquisador adjunto, 副研}
    \definition{v.}{moer; esmerilhar; triturar; pulverizar | estudar; pesquisar}
  \seealsoref{副研}{fu4yan2}
  \end{Phonetics}
\end{Entry}

\begin{Entry}{研发}{9,5}{⽯、⼜}
  \begin{Phonetics}{研发}{yan2 fa1}[][HSK 6]
    \definition{s.}{pesquisa e desenvolvimento; P\&D}
    \definition{v.}{pesquisar e/ou desenvolver}
  \end{Phonetics}
\end{Entry}

\begin{Entry}{研究}{9,7}{⽯、⽳}
  \begin{Phonetics}{研究}{yan2jiu1}[][HSK 4]
    \definition{v.}{estudar; pesquisar | discutir; considerar}
  \end{Phonetics}
\end{Entry}

\begin{Entry}{研究生}{9,7,5}{⽯、⽳、⽣}
  \begin{Phonetics}{研究生}{yan2 jiu1 sheng1}[][HSK 4]
    \definition[位,名,个,些]{s.}{pós-graduado; estudante de pós-graduação}
  \end{Phonetics}
\end{Entry}

\begin{Entry}{研究所}{9,7,8}{⽯、⽳、⼾}
  \begin{Phonetics}{研究所}{yan2 jiu1 suo3}[][HSK 5]
    \definition[家,个]{s.}{instituto de pesquisa; instituição de pesquisa científica envolvida em pesquisas em um determinado campo}
  \end{Phonetics}
\end{Entry}

\begin{Entry}{研制}{9,8}{⽯、⼑}
  \begin{Phonetics}{研制}{yan2 zhi4}[][HSK 4]
    \definition{v.}{desenvolver; fabricar; produzir | triturar; (medicina chinesa) moer}
  \end{Phonetics}
\end{Entry}

\begin{Entry}{砖}{9}{⽯}
  \begin{Phonetics}{砖}{zhuan1}
    \definition[块]{s.}{tijolo}
  \end{Phonetics}
\end{Entry}

\begin{Entry}{破}{10}{⽯}
  \begin{Phonetics}{破}{po4}[][HSK 3]
    \definition{adj.}{quebrado; danificado; rasgado; desgastado | insignificante; péssimo; medíocre}
    \definition{v.}{quebrar; danificar | dividir; cortar; separar | trocar (dinheiro) | livrar-se de; destruir; romper com | derrotar; capturar (uma cidade, etc.) | gastar dinheiro | revelar a verdade sobre; expor | mudar; romper; quebrar (regras, hábitos, ideias, etc.)}
  \end{Phonetics}
\end{Entry}

\begin{Entry}{破产}{10,6}{⽯、⼇}
  \begin{Phonetics}{破产}{po4chan3}[][HSK 4]
    \definition{v.+compl.}{falir; ir à falência; tornar-se insolvente; entrar em liquidação; perder todo o patrimônio | falhar; fracassar; não dar em nada; figura de linguagem (geralmente com uma conotação depreciativa)}
  \end{Phonetics}
\end{Entry}

\begin{Entry}{破坏}{10,7}{⽯、⼟}
  \begin{Phonetics}{破坏}{po4huai4}[][HSK 3]
    \definition{v.}{demolir; naufragar; soçobrar; destruir; obliterar | quebrar; violar (um acordo, regulamento, etc.); não cumprir (disposições legais, regras, acordos, princípios, etc.) | prejudicar; perturbar; sabotar; causar grande dano; causar danos às coisas | reverter; mudar (um sistema social, costume, etc.) completamente ou violentamente | destruir; decompor; danificar o tecido ou a estrutura de um objeto}
  \end{Phonetics}
\end{Entry}

\begin{Entry}{破坏性}{10,7,8}{⽯、⼟、⼼}
  \begin{Phonetics}{破坏性}{po4huai4xing4}
    \definition{adj.}{destrutivo}
    \definition{s.}{poder destrutivo}
  \end{Phonetics}
\end{Entry}

\begin{Entry}{砸}{10}{⽯}
  \begin{Phonetics}{砸}{za2}
    \definition{v.}{esmagar | bater | falhar | estragar}
  \end{Phonetics}
\end{Entry}

\begin{Entry}{硕}{11}{⽯}
  \begin{Phonetics}{硕}{shuo4}
    \definition*{s.}{Sobrenome Shuo}
    \definition{adj.}{grande; enorme}
    \definition{s.}{mestrado (MBA)}
  \end{Phonetics}
\end{Entry}

\begin{Entry}{硕士}{11,3}{⽯、⼠}
  \begin{Phonetics}{硕士}{shuo4shi4}[][HSK 5]
    \definition[个,位,名]{s.}{mestrado; um diploma concedido por uma universidade ou faculdade a um aluno após um ou dois anos de estudo adicional após o bacharelado}
  \end{Phonetics}
\end{Entry}

\begin{Entry}{硬}{12}{⽯}
  \begin{Phonetics}{硬}{ying4}[][HSK 4,5]
    \definition{adj.}{duro; rígido; resistente;  objeto resistente e não se deforma facilmente quando submetido a forças externas (em oposição a 软) | firme; forte; resistente; obstinado; (vontade, atitude, etc.) inabalável, forte e poderoso | capaz (pessoa); boa (qualidade) | rígido; severo; sem flexibilidade | duro; rígido; rigoroso; imutável}
    \definition{adv.}{conseguir fazer algo com dificuldade; indica fazer algo à força, independentemente das circunstâncias}
  \seealsoref{软}{ruan3}
  \end{Phonetics}
\end{Entry}

\begin{Entry}{硬件}{12,6}{⽯、⼈}
  \begin{Phonetics}{硬件}{ying4jian4}[][HSK 5]
    \definition[种]{s.}{\emph{hardware}; nome genérico dado aos vários elementos, componentes e dispositivos que constituem um computador | máquina, materiais; equipamento; referência a máquinas, equipamentos, materiais físicos, etc., utilizados nos processos de produção, pesquisa científica, gestão, etc.}
  \end{Phonetics}
\end{Entry}

\begin{Entry}{确}{12}{⽯}
  \begin{Phonetics}{确}{que4}
    \definition{adj.}{autenticado | sólido | firme | real | verdadeiro}
  \end{Phonetics}
\end{Entry}

\begin{Entry}{确认}{12,4}{⽯、⾔}
  \begin{Phonetics}{确认}{que4ren4}[][HSK 4]
    \definition{v.}{afirmar; confirmar; reconhecer; confirmar explicitamente (fatos, princípios, etc.)}
  \end{Phonetics}
\end{Entry}

\begin{Entry}{确立}{12,5}{⽯、⽴}
  \begin{Phonetics}{确立}{que4li4}[][HSK 5]
    \definition{v.}{estabelecer; criar; construir; estabelecer ou consolidar firmemente}
  \end{Phonetics}
\end{Entry}

\begin{Entry}{确定}{12,8}{⽯、⼧}
  \begin{Phonetics}{确定}{que4ding4}[][HSK 3]
    \definition{adj.}{definido; certo; claro}
    \definition{v.}{firmar; definir; determinar; tomar uma decisão clara e não mudar}
  \end{Phonetics}
\end{Entry}

\begin{Entry}{确实}{12,8}{⽯、⼧}
  \begin{Phonetics}{确实}{que4shi2}[][HSK 3]
    \definition{adj.}{verdadeiro; confiável; autêntico}
    \definition{adv.}{verdadeiramente; realmente; de ​​fato; afirmar a autenticidade de fatos objetivos}
  \end{Phonetics}
\end{Entry}

\begin{Entry}{确保}{12,9}{⽯、⼈}
  \begin{Phonetics}{确保}{que4bao3}[][HSK 3]
    \definition{v.}{assegurar; garantir; manter ou garantir com certeza}
  \end{Phonetics}
\end{Entry}

\begin{Entry}{碍}{13}{⽯}
  \begin{Phonetics}{碍}{ai4}
    \definition{v.}{atrapalhar; dificultar; obstruir; estar no caminho de | levar em consideração}
  \end{Phonetics}
\end{Entry}

\begin{Entry}{碍事}{13,8}{⽯、⼅}
  \begin{Phonetics}{碍事}{ai4shi4}
    \definition{s.}{(usualmente em frases negativas) sem consequência, não importa}
    \definition{v.+compl.}{estar no caminho | ser um obstáculo}
  \end{Phonetics}
\end{Entry}

\begin{Entry}{碎}{13}{⽯}
  \begin{Phonetics}{碎}{sui4}[][HSK 5]
    \definition*{s.}{Sobrenome Sui}
    \definition{adj.}{quebrado; fragmentado | tagarela; falante}
    \definition{v.}{quebrar em pedaços; esmagar}
  \end{Phonetics}
\end{Entry}

\begin{Entry}{碗}{13}{⽯}
  \begin{Phonetics}{碗}{wan3}[][HSK 2]
    \definition*{s.}{Sobrenome Wan}
    \definition{clas.}{usado para medição de alimentos e bebidas}
    \definition[只,个]{s.}{tigela | objeto em forma de tigela |}
  \end{Phonetics}
\end{Entry}

\begin{Entry}{碗子}{13,3}{⽯、⼦}
  \begin{Phonetics}{碗子}{wan3zi5}
    \definition{s.}{tigela}
  \end{Phonetics}
\end{Entry}

\begin{Entry}{碗柜}{13,8}{⽯、⽊}
  \begin{Phonetics}{碗柜}{wan3gui4}
    \definition{s.}{armário}
  \end{Phonetics}
\end{Entry}

\begin{Entry}{碰}{13}{⽯}
  \begin{Phonetics}{碰}{peng4}[][HSK 2]
    \definition{v.}{tocar; bater; esbarrar | encontrar; esbarrar | arriscar; tentar | tentar a sorte | reunir-se para discutir; ter uma reunião curta}
  \end{Phonetics}
\end{Entry}

\begin{Entry}{碰见}{13,4}{⽯、⾒}
  \begin{Phonetics}{碰见}{peng4 jian4}[][HSK 2]
    \definition{v.}{encontrar; encontrar-se; sem combinar, encontrar-se por acaso}
  \end{Phonetics}
\end{Entry}

\begin{Entry}{碰头}{13,5}{⽯、⼤}
  \begin{Phonetics}{碰头}{peng4tou2}
    \definition{s.}{colisão | conflito}
    \definition{v.}{colidir}
    \definition{v.+compl.}{conhecer e discutir | juntar ideias | ver-se}
  \end{Phonetics}
\end{Entry}

\begin{Entry}{碰运气}{13,7,4}{⽯、⾡、⽓}
  \begin{Phonetics}{碰运气}{peng4yun4qi5}
    \definition{v.}{deixar algo ao acaso | tentar a sorte}
  \end{Phonetics}
\end{Entry}

\begin{Entry}{碰到}{13,8}{⽯、⼑}
  \begin{Phonetics}{碰到}{peng4 dao4}[][HSK 2]
    \definition{v.}{encontrar (com); esbarrar; cruzar}
  \end{Phonetics}
\end{Entry}

\begin{Entry}{碳}{14}{⽯}
  \begin{Phonetics}{碳}{tan4}
    \definition{s.}{carbono (elemento químico)}
  \end{Phonetics}
\end{Entry}

\begin{Entry}{磁}{14}{⽯}
  \begin{Phonetics}{磁}{ci2}
    \definition[块]{s.}{porcelana | (física) magnetismo; propriedade de atrair ferro, níquel, etc. | (dialeto)  (de relação) próximo; íntimo}
  \end{Phonetics}
\end{Entry}

\begin{Entry}{磁带}{14,9}{⽯、⼱}
  \begin{Phonetics}{磁带}{ci2dai4}
    \definition[盘,盒]{s.}{cassete | fita magnética}
  \end{Phonetics}
\end{Entry}

\begin{Entry}{磁铁}{14,10}{⽯、⾦}
  \begin{Phonetics}{磁铁}{ci2tie3}
    \definition{s.}{imã | magneto}
  \seealsoref{吸铁石}{xi1tie3shi2}
  \end{Phonetics}
\end{Entry}

\begin{Entry}{磁盘}{14,11}{⽯、⽫}
  \begin{Phonetics}{磁盘}{ci2pan2}
    \definition{s.}{disquete}
  \end{Phonetics}
\end{Entry}

\begin{Entry}{碾}{15}{⽯}
  \begin{Phonetics}{碾}{nian3}
    \definition[台,个]{s.}{rolo e mó; rolo de pedra | rolo compressor}
    \definition{v.}{moer ou descascar com um rolo; esmagar | (literário) cortar e polir (jade, vidro, etc.) | achatar | pisar; pisotear, 轧}
  \seealsoref{辗}{zhan3}
  \end{Phonetics}
\end{Entry}

\begin{Entry}{碾碎}{15,13}{⽯、⽯}
  \begin{Phonetics}{碾碎}{nian3sui4}
    \definition{v.}{pulverizar | esmagar}
  \end{Phonetics}
\end{Entry}

\begin{Entry}{磨}{16}{⽯}
  \begin{Phonetics}{磨}{mo2}[][HSK 6]
    \definition{v.}{esfregar; desgastar | moer; refletir; polir | desgastar; esgotar; cansar; exaurir | incomodar; causar problemas | destruir; obliterar; extinguir-se | ficar ocioso; perder tempo; perder tempo; procrastinar}
  \end{Phonetics}
  \begin{Phonetics}{磨}{mo4}
    \definition[盘]{s.}{mó (pedra pesada e redonda para moinho)}
    \definition{v.}{moer; esfarelar; triturar | virar; inverter a marcha}
  \end{Phonetics}
\end{Entry}

\begin{Entry}{磨菇}{16,11}{⽯、⾋}
  \begin{Phonetics}{磨菇}{mo2gu5}
    \variantof{蘑菇}
  \end{Phonetics}
\end{Entry}

%%%%% EOF %%%%%

