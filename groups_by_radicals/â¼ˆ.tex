%%%
%%% Radical "⼈"
%%%

\section*{Radical 9: ``⼈'' (亻、𠆢)}\addcontentsline{toc}{section}{Radical 9: ⼈、亻、𠆢}

\begin{Entry}{人}{2}{⼈}[Kangxi 9]
  \begin{Phonetics}{人}{ren2}[][HSK 1]
    \definition*{s.}{Sobrenome Ren}
    \definition[个,名,位]{s.}{homem; pessoa; pessoas; ser humano | todos; cada um; todo mundo | adulto; crescido | uma pessoa envolvida em uma atividade específica | pessoas; outras pessoas | caráter; personalidade; qualidade, caráter ou reputação de uma pessoa | como alguém se sente; estado de saúde de alguém | mão de obra; força de trabalho}
  \end{Phonetics}
\end{Entry}

\begin{Entry}{人力}{2,2}{⼈、⼒}
  \begin{Phonetics}{人力}{ren2 li4}[][HSK 5]
    \definition{s.}{mão de obra; trabalho manual; força de trabalho}
  \end{Phonetics}
\end{Entry}

\begin{Entry}{人力车}{2,2,4}{⼈、⼒、⾞}
  \begin{Phonetics}{人力车}{ren2 li4 che1}
    \definition{s.}{veículo de duas rodas puxado ou empurrado por um homem (oposto a 兽力车 e 机动车) | Datado: riquixá | uma carroça puxada ou empurrada por humanos}
  \seealsoref{机动车}{ji1 dong4 che1}
  \seealsoref{兽力车}{shou4 li4 che1}
  \end{Phonetics}
\end{Entry}

\begin{Entry}{人口}{2,3}{⼈、⼝}
  \begin{Phonetics}{人口}{ren2kou3}[][HSK 2]
    \definition[个,群]{s.}{população; o número total de pessoas que vivem em uma determinada região durante um determinado período de tempo | número de membros da família; o número total de pessoas em uma família | pessoas; público; população; referência geral a pessoas | rumores do povo; referindo-se à opinião pública}
  \end{Phonetics}
\end{Entry}

\begin{Entry}{人士}{2,3}{⼈、⼠}
  \begin{Phonetics}{人士}{ren2shi4}[][HSK 5]
    \definition{s.}{pessoa; figura; personalidade; figura pública; pessoas com certa influência social}
  \end{Phonetics}
\end{Entry}

\begin{Entry}{人工}{2,3}{⼈、⼯}
  \begin{Phonetics}{人工}{ren2gong1}[][HSK 3]
    \definition{adj.}{feito pelo homem; artificial (oposto a 天然)}
    \definition[个]{s.}{trabalho manual; trabalho feito à mão | mão de obra; homem-dia; uma unidade de cálculo da quantidade de trabalho realizado}
  \seealsoref{天然}{tian1 ran2}
  \end{Phonetics}
\end{Entry}

\begin{Entry}{人工智能}{2,3,12,10}{⼈、⼯、⽇、⾁}
  \begin{Phonetics}{人工智能}{ren2 gong1 zhi4 neng2}
    \definition*{s.}{Inteligência Artificial (IA)}
  \end{Phonetics}
\end{Entry}

\begin{Entry}{人才}{2,3}{⼈、⼿}
  \begin{Phonetics}{人才}{ren2cai2}[][HSK 3]
    \definition{adj.}{aparência bonita, elegante}
    \definition[个,些,位]{s.}{talento; pessoal qualificado; pessoa com capacidade; uma pessoa com capacidade e integridade política; uma pessoa com talentos especiais | aparência bonita; refere-se à aparência; especialmente à aparência bonita}
  \end{Phonetics}
\end{Entry}

\begin{Entry}{人们}{2,5}{⼈、⼈}
  \begin{Phonetics}{人们}{ren2 men5}[][HSK 2]
    \definition{s.}{homens; pessoas; o público; referindo-se a muitas pessoas; todos}
  \end{Phonetics}
\end{Entry}

\begin{Entry}{人民}{2,5}{⼈、⽒}
  \begin{Phonetics}{人民}{ren2 min2}[][HSK 3]
    \definition[群,批,个,国]{s.}{o povo; refere-se a um certo tipo de pessoas; membros básicos da sociedade com as massas trabalhadoras como o corpo principal}
  \end{Phonetics}
\end{Entry}

\begin{Entry}{人民币}{2,5,4}{⼈、⽒、⼱}
  \begin{Phonetics}{人民币}{ren2min2bi4}[][HSK 3]
    \definition*[块,张,元]{s.}{Renminbi (RMB); Yuan Chinês (CYN); nome da moeda chinesa}
  \end{Phonetics}
\end{Entry}

\begin{Entry}{人生}{2,5}{⼈、⽣}
  \begin{Phonetics}{人生}{ren2sheng1}[][HSK 3]
    \definition{s.}{vida; sobrevivência e vida humana}
  \end{Phonetics}
\end{Entry}

\begin{Entry}{人权}{2,6}{⼈、⽊}
  \begin{Phonetics}{人权}{ren2quan2}[][HSK 6]
    \definition{s.}{direitos humanos}[最基本的人权是生存权。===O direito humano mais básico é o direito à vida.]
  \seealsoref{人权法}{ren2quan2fa3}
  \end{Phonetics}
\end{Entry}

\begin{Entry}{人权法}{2,6,8}{⼈、⽊、⽔}
  \begin{Phonetics}{人权法}{ren2quan2fa3}
    \definition*{s.}{Direitos Humanos}
  \seealsoref{人权}{ren2quan2}
  \end{Phonetics}
\end{Entry}

\begin{Entry}{人行道}{2,6,12}{⼈、⾏、⾡}
  \begin{Phonetics}{人行道}{ren2xing2dao4}
    \definition{s.}{calçada}
  \end{Phonetics}
\end{Entry}

\begin{Entry}{人员}{2,7}{⼈、⼝}
  \begin{Phonetics}{人员}{ren2yuan2}[][HSK 3]
    \definition[个,位,名]{s.}{funcionários ; uma pessoa que ocupa uma determinada posição| pessoal; membros de um grupo}
  \end{Phonetics}
\end{Entry}

\begin{Entry}{人材}{2,7}{⼈、⽊}
  \begin{Phonetics}{人材}{ren2cai2}
    \variantof{人才}
  \end{Phonetics}
\end{Entry}

\begin{Entry}{人间}{2,7}{⼈、⾨}
  \begin{Phonetics}{人间}{ren2jian1}[][HSK 5]
    \definition{s.}{o mundo humano; o Mundo; a Terra}
  \end{Phonetics}
\end{Entry}

\begin{Entry}{人物}{2,8}{⼈、⽜}
  \begin{Phonetics}{人物}{ren2wu4}[][HSK 5]
    \definition[个,位,名]{s.}{personagem; personagens criados em obras literárias e artísticas | figura; personalidade; homem influente; refere-se a pessoas com grande talento e status; também se refere a pessoas com certas características ou que são representativas em algum aspecto | pintura figurativa; um tipo de pintura tradicional chinesa com personagens como tema}
  \end{Phonetics}
\end{Entry}

\begin{Entry}{人鱼}{2,8}{⼈、⿂}
  \begin{Phonetics}{人鱼}{ren2yu2}
    \definition{s.}{sereia | peixe-boi | salamandra gigante}
  \end{Phonetics}
\end{Entry}

\begin{Entry}{人类}{2,9}{⼈、⽶}
  \begin{Phonetics}{人类}{ren2lei4}[][HSK 3]
    \definition[种]{s.}{humano; humanidade; raça humana; um termo geral para pessoas}
  \end{Phonetics}
\end{Entry}

\begin{Entry}{人家}{2,10}{⼈、⼧}
  \begin{Phonetics}{人家}{ren2jia1}[][HSK 4]
    \definition[户,个]{s.}{lar; família; família do noivo; casa do futuro marido}
  \end{Phonetics}
  \begin{Phonetics}{人家}{ren2jia5}
    \definition{pron.}{outros; uma pessoa ou pessoas diferentes do falante ou ouvinte; refere-se a alguém diferente de si mesmo ou de outra pessoa | certa pessoa ou pessoas (a pessoa ou pessoas mencionadas em um contexto próximo, aproximadamente equivalente ao pronome de terceira pessoa);  refere-se a uma pessoa ou algumas pessoas, com significado semelhante a 他 | eu; mim (usado retoricamente no lugar do primeiro pronome pessoal, muitas vezes expressando descontentamento de forma jocosa; geralmente usado quando se fala com pessoas próximas, para significar 自己, usado principamente por meninas)}
  \seealsoref{他}{ta1}
  \seealsoref{自己}{zi4ji3}
  \end{Phonetics}
\end{Entry}

\begin{Entry}{人海}{2,10}{⼈、⽔}
  \begin{Phonetics}{人海}{ren2hai3}
    \definition{s.}{uma multidão | um mar de pessoas}
  \end{Phonetics}
\end{Entry}

\begin{Entry}{人道}{2,12}{⼈、⾡}
  \begin{Phonetics}{人道}{ren2dao4}
    \definition{s.}{solidariedade humana | humanitarismo | humano | a ``maneira humana'', um dos estágios do ciclo de reencarnação (budismo) | relação sexual}
  \end{Phonetics}
\end{Entry}

\begin{Entry}{人像}{2,13}{⼈、⼈}
  \begin{Phonetics}{人像}{ren2xiang4}
    \definition{s.}{``retrato'' de uma pessoa (esboço, foto, escultura, etc.)}
  \end{Phonetics}
\end{Entry}

\begin{Entry}{人数}{2,13}{⼈、⽁}
  \begin{Phonetics}{人数}{ren2 shu4}[][HSK 2]
    \definition{s.}{número de pessoas; significa o número total de pessoas, uma quantidade de pessoas; normalmente, usa-se números para fazer estatísticas específicas, mas às vezes também se usa um intervalo aproximado para fazer estimativas}
  \end{Phonetics}
\end{Entry}

\begin{Entry}{人群}{2,13}{⼈、⽺}
  \begin{Phonetics}{人群}{ren2 qun2}[][HSK 3]
    \definition[个,类]{s.}{multidão; ajuntamento; torpel; aglomeração; um grupo de pessoas}
  \end{Phonetics}
\end{Entry}

\begin{Entry}{个}{3}{⼈}
  \begin{Phonetics}{个}{ge3}
    \definition{pron.}{usado em 自个儿}
  \seealsoref{自个儿}{zi4ge3r5}
  \end{Phonetics}
  \begin{Phonetics}{个}{ge4}[][HSK 1]
    \definition{adj.}{individual}
    \definition{clas.}{usado antes de substantivos que não têm palavras de medida específicas | usado na frente do divisor; usado na frente do número aproximado | usado após verbos com objeto direto |  usado entre verbos e complementos}
    \definition{part.}{usado após pronomes demonstrativos | adicionado após certas palavras de tempo}
  \end{Phonetics}
\end{Entry}

\begin{Entry}{个人}{3,2}{⼈、⼈}
  \begin{Phonetics}{个人}{ge4ren2}[][HSK 3]
    \definition{pron.}{pessoal; si mesmo}
    \definition[个]{s.}{indivíduo; pessoa}
  \end{Phonetics}
\end{Entry}

\begin{Entry}{个儿}{3,2}{⼈、⼉}
  \begin{Phonetics}{个儿}{ge4r5}[][HSK 5]
    \definition{s.}{tamanho; altura; estatura; tamanho do corpo ou do objeto | pessoas ou coisas consideradas isoladamente; referir-se a uma pessoa ou coisa individualmente}
  \end{Phonetics}
\end{Entry}

\begin{Entry}{个子}{3,3}{⼈、⼦}
  \begin{Phonetics}{个子}{ge4zi5}[][HSK 2]
    \definition[个,种,些]{s.}{altura; estatura; refere-se ao tamanho do corpo humano e também ao tamanho do corpo dos animais}
  \end{Phonetics}
\end{Entry}

\begin{Entry}{个体}{3,7}{⼈、⼈}
  \begin{Phonetics}{个体}{ge4ti3}[][HSK 4]
    \definition[个,位]{s.}{uma única pessoa ou organismo}
  \end{Phonetics}
\end{Entry}

\begin{Entry}{个别}{3,7}{⼈、⼑}
  \begin{Phonetics}{个别}{ge4bie2}[][HSK 4]
    \definition{adj.}{muito poucos; excepcionais}
    \definition{adv.}{separadamente; individualmente; isoladamente}
  \end{Phonetics}
\end{Entry}

\begin{Entry}{个性}{3,8}{⼈、⼼}
  \begin{Phonetics}{个性}{ge4xing4}[][HSK 3]
    \definition[种,点儿]{s.}{individualidade; personalidade; caráter individual; as características relativamente fixas de uma pessoa, formadas sob determinadas condições sociais e influências educacionais | propriedade específica; caráter específico; a propriedade ou característica especial que distingue uma coisa de outras coisas}
  \end{Phonetics}
\end{Entry}

\begin{Entry}{亿}{3}{⼈}
  \begin{Phonetics}{亿}{yi4}[][HSK 2]
    \definition*{s.}{Sobrenome Yi}
    \definition{num.}{cem milhões; 100.000.000; 1.0000.0000}
  \end{Phonetics}
\end{Entry}

\begin{Entry}{什}{4}{⼈}
  \begin{Phonetics}{什}{shen2}
    \definition{pron.}{o que; qualquer coisa}
  \seealsoref{什么}{shen2me5}
  \end{Phonetics}
  \begin{Phonetics}{什}{shi2}
    \definition*{s.}{Sobrenome Shi}
    \definition{adj.}{variado; sortido; diverso; vários; misturados}
    \definition{num.}{(em frações ou múltiplos) dez}
    \definition{s.}{várias coisas; artigos diversos}
  \end{Phonetics}
\end{Entry}

\begin{Entry}{什么}{4,3}{⼈、⼃}
  \begin{Phonetics}{什么}{shen2me5}[][HSK 1]
    \definition{pron.}{o que?; expressar dúvida, perguntar sobre o mundo, locais, pessoas ou coisas | usado para se referir a algo indefinido; expressar incerteza | qualquer; todos; refere-se a todas as pessoas ou coisas | dois 什么 são usados juntos, indicando que o primeiro determina o segundo | usado para expressar surpresa ou insatisfação | usado para expressar discordância com o que foi dito; expressar negação | usado antes de elementos paralelos para indicar que a lista é infinita}
  \end{Phonetics}
\end{Entry}

\begin{Entry}{什么时候}{4,3,7,10}{⼈、⼃、⽇、⼈}
  \begin{Phonetics}{什么时候}{shen2me5shi2hou5}
    \definition{adv.}{quando? | a que horas?}
  \end{Phonetics}
\end{Entry}

\begin{Entry}{什么样}{4,3,10}{⼈、⼃、⽊}
  \begin{Phonetics}{什么样}{shen2 me5 yang4}[][HSK 2]
    \definition{pron.}{que tipo?; usado para perguntar sobre a natureza, características ou aparência de algo |  o quê?; de que tipo?; usado para perguntar sobre a situação ou o estado de alguém ou algo}
  \end{Phonetics}
\end{Entry}

\begin{Entry}{仅}{4}{⼈}
  \begin{Phonetics}{仅}{jin3}[][HSK 3]
    \definition{adv.}{somente; meramente; por muito pouco}
  \end{Phonetics}
\end{Entry}

\begin{Entry}{仅仅}{4,4}{⼈、⼈}
  \begin{Phonetics}{仅仅}{jin3 jin3}[][HSK 3]
    \definition{adv.}{somente; meramente; por muito pouco; indica que está limitado a um determinado âmbito}
  \end{Phonetics}
\end{Entry}

\begin{Entry}{仅此而已}{4,6,6,3}{⼈、⽌、⽽、⼰}
  \begin{Phonetics}{仅此而已}{jin3ci3'er2yi3}
    \definition{adv.}{apenas isso e nada mais | isso é tudo}
  \end{Phonetics}
\end{Entry}

\begin{Entry}{今}{4}{⼈}
  \begin{Phonetics}{今}{jin1}
    \definition*{s.}{Sobrenome Jin}
    \definition{s.}{agora; o presente | moderno (em oposição a 古) | de hoje; deste ano | isso; isto}
  \seealsoref{古}{gu3}
  \end{Phonetics}
\end{Entry}

\begin{Entry}{今天}{4,4}{⼈、⼤}
  \begin{Phonetics}{今天}{jin1tian1}[][HSK 1]
    \definition{adv.}{hoje; neste dia | agora; o momento ou a época atual}
  \end{Phonetics}
\end{Entry}

\begin{Entry}{今日}{4,4}{⼈、⽇}
  \begin{Phonetics}{今日}{jin1 ri4}[][HSK 5]
    \definition{s.}{hoje}
  \end{Phonetics}
\end{Entry}

\begin{Entry}{今后}{4,6}{⼈、⼝}
  \begin{Phonetics}{今后}{jin1 hou4}[][HSK 2]
    \definition{s.}{a partir de agora; doravante; no futuro; desde o momento em que falamos}
  \end{Phonetics}
\end{Entry}

\begin{Entry}{今年}{4,6}{⼈、⼲}
  \begin{Phonetics}{今年}{jin1 nian2}[][HSK 1]
    \definition{adv.}{este ano}
  \end{Phonetics}
\end{Entry}

\begin{Entry}{介}{4}{⼈}
  \begin{Phonetics}{介}{jie4}
    \definition*{s.}{Sobrenome Jie}
    \definition{adj.}{direto; honesto e franco; correto}
    \definition{s.}{armadura | concha (crustáceos e criaturas aquáticas) | preposição}
    \definition{v.}{estar situado entre; interpor | levar a sério; levar em conta; ter em mente}
  \end{Phonetics}
\end{Entry}

\begin{Entry}{介绍}{4,8}{⼈、⽷}
  \begin{Phonetics}{介绍}{jie4shao4}[][HSK 1]
    \definition{s.}{introdução; apresentação}
    \definition{v.}{introduzir; apresentar | recomendar; sugerir | dar a conhecer; informar}
  \end{Phonetics}
\end{Entry}

\begin{Entry}{仍}{4}{⼈}
  \begin{Phonetics}{仍}{reng2}[][HSK 3]
    \definition{adv.}{ainda; repetidamente; frequentemente; continuamente}
    \definition{v.}{permanecer}
  \end{Phonetics}
\end{Entry}

\begin{Entry}{仍旧}{4,5}{⼈、⽇}
  \begin{Phonetics}{仍旧}{reng2jiu4}[][HSK 5]
    \definition{adv.}{ainda; ainda assim; contudo}
    \definition{v.}{permanecer igual; continuar sendo}
  \end{Phonetics}
\end{Entry}

\begin{Entry}{仍然}{4,12}{⼈、⽕}
  \begin{Phonetics}{仍然}{reng2ran2}[][HSK 3]
    \definition{adv.}{ainda; contudo; como antes; indica que a situação continua inalterada ou retorna ao seu estado original}
  \end{Phonetics}
\end{Entry}

\begin{Entry}{从}{4}{⼈}
  \begin{Phonetics}{从}{cong2}[][HSK 1]
    \definition*{s.}{Sobrenome Cong}
    \definition{adj.}{secundário; acessório | relacionamento entre primos, etc., do mesmo avô paterno, bisavô ou de um ancestral comum ainda mais antigo; do mesmo clã}
    \definition{adv.}{(seguido de uma negativa) jamais | jamais; usado antes de palavras negativas, indica que algo nunca aconteceu desde o passado, equivalente a 从来}
    \definition{prep.}{de (um tempo, um lugar ou um ponto de vista) | via, através ou após (um local) | de; via; através de; passado (um lugar); introdução das ações, trajetórias e locais| (de comportamento) de; introdução de ações e comportamentos com base em referências e fundamentos}
    \definition{s.}{seguidor; acompanhante}
    \definition{v.}{seguir; cumprir; obedecer | participar; estar envolvido em | seguir o princípio de; empregar o método de | estar envolvido em | seguir; adotar (um determinado princípio ou atitude)}
  \seealsoref{从来}{cong2lai2}
  \end{Phonetics}
\end{Entry}

\begin{Entry}{从小}{4,3}{⼈、⼩}
  \begin{Phonetics}{从小}{cong2 xiao3}[][HSK 2]
    \definition{adv.}{desde a infância; desde muito jovem; quando criança}
  \end{Phonetics}
\end{Entry}

\begin{Entry}{从不}{4,4}{⼈、⼀}
  \begin{Phonetics}{从不}{cong2bu4}[][HSK 6]
    \definition{adv.}{nunca; impossível (para expressar surpresa ou choque)}
  \end{Phonetics}
\end{Entry}

\begin{Entry}{从中}{4,4}{⼈、⼁}
  \begin{Phonetics}{从中}{cong2 zhong1}[][HSK 5]
    \definition{adv.}{de; dentre; daí}
  \end{Phonetics}
\end{Entry}

\begin{Entry}{从未}{4,5}{⼈、⽊}
  \begin{Phonetics}{从未}{cong2wei4}
    \definition{adv.}{nunca}
  \end{Phonetics}
\end{Entry}

\begin{Entry}{从此}{4,6}{⼈、⽌}
  \begin{Phonetics}{从此}{cong2ci3}[][HSK 4]
    \definition{conj.}{doravante; portanto; a partir deste momento; de agora em diante; a partir de então}
  \end{Phonetics}
\end{Entry}

\begin{Entry}{从而}{4,6}{⼈、⽽}
  \begin{Phonetics}{从而}{cong2'er2}[][HSK 5]
    \definition{conj.}{assim; por isso; portanto; desse modo; por esse motivo; conjunção usada no início do texto seguinte para expressar o resultado, propósito ou ação posterior, o que é equivalente a 因此就}
  \seealsoref{因此就}{yin1ci3 jiu4}
  \end{Phonetics}
\end{Entry}

\begin{Entry}{从来}{4,7}{⼈、⽊}
  \begin{Phonetics}{从来}{cong2lai2}[][HSK 3]
    \definition{adv.}{sempre; o tempo todo; em todos os momentos; do passado até o presente}
  \end{Phonetics}
\end{Entry}

\begin{Entry}{从没}{4,7}{⼈、⽔}
  \begin{Phonetics}{从没}{cong2 mei2}[][HSK 6]
    \definition{adv.}{nunca (no passado)}
  \end{Phonetics}
\end{Entry}

\begin{Entry}{从事}{4,8}{⼈、⼅}
  \begin{Phonetics}{从事}{cong2shi4}[][HSK 3]
    \definition{v.}{trabalhar; empreender; empenhar-se em; envolver-se em; dedicar-se ou comprometer-se (a uma causa); participar (de algo) | lidar com; deve ter um advérbio antes e não pode ter um objeto depois}
  \end{Phonetics}
\end{Entry}

\begin{Entry}{从前}{4,9}{⼈、⼑}
  \begin{Phonetics}{从前}{cong2qian2}[][HSK 3]
    \definition{s.}{antes; antigamente; no passado | era uma vez; há muito tempo atrás}
  \end{Phonetics}
\end{Entry}

\begin{Entry}{仓}{4}{⼈}
  \begin{Phonetics}{仓}{cang1}
    \definition*{s.}{Sobrenome Cang}
    \definition{s.}{armazém; depósito}
  \end{Phonetics}
\end{Entry}

\begin{Entry}{仓库}{4,7}{⼈、⼴}
  \begin{Phonetics}{仓库}{cang1ku4}[][HSK 6]
    \definition[间,座,个]{s.}{armazém; depósito; um edifício usado para armazenar grandes quantidades de alimentos ou outros suprimentos}
  \end{Phonetics}
\end{Entry}

\begin{Entry}{以}{4}{⼈}
  \begin{Phonetics}{以}{yi3}
    \definition*{s.}{Sobrenome Yi}
    \definition{conj.}{e; bem como; o mesmo que 而 | de modo a; a fim de; usado no início da frase seguinte para expressar o propósito de atingir um determinado objetivo}
    \definition{prep.}{por; com | de acordo com | por causa de; porque | em (uma data fixa); em (um certo momento) | colocado antes de palavras posicionais simples, indica os limites de tempo, posição e quantidade}
  \seealsoref{而}{er2}
  \end{Phonetics}
\end{Entry}

\begin{Entry}{以上}{4,3}{⼈、⼀}
  \begin{Phonetics}{以上}{yi3 shang4}[][HSK 2]
    \definition[本]{s.}{mais do que; sobre; acima; indica posição, ordem ou número acima de um determinado ponto | o acima; o precedente; o acima mencionado; refere-se às palavras anteriores}
  \end{Phonetics}
\end{Entry}

\begin{Entry}{以下}{4,3}{⼈、⼀}
  \begin{Phonetics}{以下}{yi3 xia4}[][HSK 2]
    \definition[所]{s.}{abaixo; sob; indica posição, ordem ou número abaixo de um certo ponto | seguinte; refere-se às seguintes palavras}
  \end{Phonetics}
\end{Entry}

\begin{Entry}{以及}{4,3}{⼈、⼃}
  \begin{Phonetics}{以及}{yi3ji2}[][HSK 4]
    \definition{conj.}{assim como; juntamente como; bem como; também}
  \end{Phonetics}
\end{Entry}

\begin{Entry}{以为}{4,4}{⼈、⼂}
  \begin{Phonetics}{以为}{yi3wei2}[][HSK 2]
    \definition{v.}{pensar; acreditar; considerar (geralmente erroneamente); expressa opiniões e atitudes em relação às coisas, geralmente erradas}
  \end{Phonetics}
\end{Entry}

\begin{Entry}{以内}{4,4}{⼈、⼌}
  \begin{Phonetics}{以内}{yi3 nei4}[][HSK 4]
    \definition{adv.}{dentro de; menos que; não mais que; dentro de certos limites de tempo, premissas, quantidade e escopo}
  \end{Phonetics}
\end{Entry}

\begin{Entry}{以外}{4,5}{⼈、⼣}
  \begin{Phonetics}{以外}{yi3 wai4}[][HSK 2]
    \definition{s.}{além; exceto; fora; diferente de; fora dos limites de um determinado tempo, quantidade ou lugar}
  \end{Phonetics}
\end{Entry}

\begin{Entry}{以后}{4,6}{⼈、⼝}
  \begin{Phonetics}{以后}{yi3 hou4}[][HSK 2]
    \definition{s.}{depois; mais tarde; após; daqui em diante}
  \end{Phonetics}
\end{Entry}

\begin{Entry}{以此}{4,6}{⼈、⽌}
  \begin{Phonetics}{以此}{yi3ci3}
    \definition{adv.}{devido a esta | deste modo | por isso | com isso}
  \end{Phonetics}
\end{Entry}

\begin{Entry}{以至}{4,6}{⼈、⾄}
  \begin{Phonetics}{以至}{yi3zhi4}
    \definition{adv.}{até}
    \definition{conj.}{a tal ponto que\dots}
  \seealsoref{以至于}{yi3zhi4yu2}
  \end{Phonetics}
\end{Entry}

\begin{Entry}{以至于}{4,6,3}{⼈、⾄、⼆}
  \begin{Phonetics}{以至于}{yi3zhi4yu2}
    \definition{adv.}{até}
    \definition{conj.}{na medida em que\dots}
  \seealsoref{以至}{yi3zhi4}
  \end{Phonetics}
\end{Entry}

\begin{Entry}{以色列}{4,6,6}{⼈、⾊、⼑}
  \begin{Phonetics}{以色列}{yi3se4lie4}
    \definition*{s.}{Israel}
  \end{Phonetics}
\end{Entry}

\begin{Entry}{以免}{4,7}{⼈、⼉}
  \begin{Phonetics}{以免}{yi3mian3}
    \definition{conj.}{para evitar isso}
  \end{Phonetics}
\end{Entry}

\begin{Entry}{以来}{4,7}{⼈、⽊}
  \begin{Phonetics}{以来}{yi3lai2}[][HSK 3]
    \definition{s.}{desde (em termos de tempo); indica um período de tempo desde um determinado momento no passado até o presente}
  \end{Phonetics}
\end{Entry}

\begin{Entry}{以求}{4,7}{⼈、⽔}
  \begin{Phonetics}{以求}{yi3qiu2}
    \definition{conj.}{a fim de}
  \end{Phonetics}
\end{Entry}

\begin{Entry}{以往}{4,8}{⼈、⼻}
  \begin{Phonetics}{以往}{yi3wang3}[][HSK 5]
    \definition{s.}{antes; anterior; no passado}
  \end{Phonetics}
\end{Entry}

\begin{Entry}{以便}{4,9}{⼈、⼈}
  \begin{Phonetics}{以便}{yi3bian4}[][HSK 5]
    \definition{conj.}{para que; de modo que; a fim de; com o objetivo de; para o propósito de; usado no início da segunda parte da frase, indica que o objetivo mencionado na segunda parte será facilmente alcançado}
  \end{Phonetics}
\end{Entry}

\begin{Entry}{以前}{4,9}{⼈、⼑}
  \begin{Phonetics}{以前}{yi3qian2}[][HSK 2]
    \definition{s.}{antes; antigamente; anteriormente (no tempo); agora ou o período anterior ao tempo indicado}
  \end{Phonetics}
\end{Entry}

\begin{Entry}{以期}{4,12}{⼈、⽉}
  \begin{Phonetics}{以期}{yi3qi1}
    \definition{v.}{tentando | esperando | esperando por}
  \end{Phonetics}
\end{Entry}

\begin{Entry}{仔}{5}{⼈}
  \begin{Phonetics}{仔}{zi1}
    \definition{s./v.}{usado em 仔肩}
  \seealsoref{仔肩}{zi1jian1}
  \end{Phonetics}
  \begin{Phonetics}{仔}{zi3}
    \definition{adj.}{jovem | cuidadoso; pequeno; fino}
  \end{Phonetics}
\end{Entry}

\begin{Entry}{仔细}{5,8}{⼈、⽷}
  \begin{Phonetics}{仔细}{zi3xi4}[][HSK 5]
    \definition{adj.}{cuidadoso; atencioso; descreve alguém que é cuidadoso e meticuloso ao fazer as coisas; não é descuidado | frugal; econômico; descreve o uso moderado de dinheiro ou bens, sem desperdício}
    \definition{v.}{ter cuidado; prestar atenção; ter muito cuidado e evitar que aconteçam coisas ruins}
  \end{Phonetics}
\end{Entry}

\begin{Entry}{仔肩}{5,8}{⼈、⾁}
  \begin{Phonetics}{仔肩}{zi1jian1}
    \definition{s.}{encargos oficiais (ou responsabilidades)}
    \definition{v.}{assumir a responsabilidade por algo}
  \end{Phonetics}
\end{Entry}

\begin{Entry}{他}{5}{⼈}
  \begin{Phonetics}{他}{ta1}[][HSK 1]
    \definition{pron.}{ele | outro; referindo-se a outro; diferente | usado após o verbo, indica referência vaga | alguém; todos; usado em conjunto com 你, significa qualquer pessoa ou muitas pessoas | em outro lugar; outro lugar}
  \seealsoref{你}{ni3}
  \seealsoref{怹}{tan1}
  \end{Phonetics}
\end{Entry}

\begin{Entry}{他们}{5,5}{⼈、⼈}
  \begin{Phonetics}{他们}{ta1men5}[][HSK 1]
    \definition{pron.}{eles}
  \end{Phonetics}
\end{Entry}

\begin{Entry}{他们的}{5,5,8}{⼈、⼈、⽩}
  \begin{Phonetics}{他们的}{ta1men5 de5}
    \definition{pron.}{deles}
  \end{Phonetics}
\end{Entry}

\begin{Entry}{他妈的}{5,6,8}{⼈、⼥、⽩}
  \begin{Phonetics}{他妈的}{ta1ma1de5}
    \definition{interj.}{Dane-se! | Foda-se!}
  \end{Phonetics}
\end{Entry}

\begin{Entry}{他的}{5,8}{⼈、⽩}
  \begin{Phonetics}{他的}{ta1 de5}
    \definition{pron.}{dele}
  \end{Phonetics}
\end{Entry}

\begin{Entry}{付}{5}{⼈}
  \begin{Phonetics}{付}{fu4}[][HSK 3]
    \definition*{s.}{Sobrenome Fu}
    \definition{clas.}{usado para pares ou conjuntos de coisas | usado para expressões faciais}
    \definition{v.}{comprometer-se com; entregar (ou transferir) para | pagar; refere-se especificamente a dar dinheiro}
  \end{Phonetics}
\end{Entry}

\begin{Entry}{付出}{5,5}{⼈、⼐}
  \begin{Phonetics}{付出}{fu4 chu1}[][HSK 4]
    \definition{v.}{pagar; gastar; entregar (dinheiro, consideração, etc.)}
  \end{Phonetics}
\end{Entry}

\begin{Entry}{付款}{5,12}{⼈、⽋}
  \begin{Phonetics}{付款}{fu4kuan3}
    \definition{s.}{pagamento}
    \definition{v.+compl.}{pagar uma quantia em dinheiro}
  \end{Phonetics}
\end{Entry}

\begin{Entry}{仙}{5}{⼈}
  \begin{Phonetics}{仙}{xian1}
    \definition{s.}{imortal}
  \end{Phonetics}
\end{Entry}

\begin{Entry}{代}{5}{⼈}
  \begin{Phonetics}{代}{dai4}[][HSK 3]
    \definition*{s.}{Sobrenome Dai}
    \definition{s.}{dinastia | geração; hierarquia familiar | era; o segundo nível da divisão geológica é o período, acima do qual está a era e abaixo do qual está o período, por exemplo, o Paleozóico, o Mesozóico e o Cenozóico pertencem à era Phanerozoico | período histórico; época}
    \definition{v.}{tomar o lugar de; estar no lugar de | agir em nome de; exercer}
  \end{Phonetics}
\end{Entry}

\begin{Entry}{代价}{5,6}{⼈、⼈}
  \begin{Phonetics}{代价}{dai4jia4}[][HSK 5]
    \definition[种,个]{s.}{preço; material, energia gasta ou sacrifícios feitos para atingir um objetivo | custo; preço; dinheiro pago para obter algo}
  \end{Phonetics}
\end{Entry}

\begin{Entry}{代言}{5,7}{⼈、⾔}
  \begin{Phonetics}{代言}{dai4yan2}
    \definition{v.}{ser um porta-voz | ser um embaixador (para uma marca) | endossar}
  \end{Phonetics}
\end{Entry}

\begin{Entry}{代表}{5,8}{⼈、⾐}
  \begin{Phonetics}{代表}{dai4biao3}[][HSK 3]
    \definition[位,名,个,些]{s.}{deputado; delegado; representante; pessoas eleitas para representar eleitores ou expressar opiniões, ou pessoas encarregadas ou designadas para representar indivíduos, grupos ou governos ou expressar opiniões | representante oficial; pessoas ou coisas que refletem as características comuns de um grupo específico}
    \definition{v.}{representar; defender | usar pessoas ou coisas para expressar um significado ou conceito específico}
  \end{Phonetics}
\end{Entry}

\begin{Entry}{代表团}{5,8,6}{⼈、⾐、⼞}
  \begin{Phonetics}{代表团}{dai4 biao3 tuan2}[][HSK 3]
    \definition[个]{s.}{delegação; contingente; um grupo temporário de grande dimensão formado para participar de uma determinada atividade em nome de um país, governo ou outra organização social}
  \end{Phonetics}
\end{Entry}

\begin{Entry}{代称}{5,10}{⼈、⽲}
  \begin{Phonetics}{代称}{dai4cheng1}
    \definition{s.}{nome alternativo | antonomásia}
    \definition{v.}{referir-se a algo ou alguém por outro nome}
  \end{Phonetics}
\end{Entry}

\begin{Entry}{代理}{5,11}{⼈、⽟}
  \begin{Phonetics}{代理}{dai4li3}[][HSK 5]
    \definition{v.}{agir em nome de alguém em uma posição de responsabilidade; substituir alguém | agir como procurador; agir como agente; ser encarregado pelas partes de realizar atividades e conduzir assuntos em seu nome dentro do escopo de sua autorização}
  \end{Phonetics}
\end{Entry}

\begin{Entry}{代替}{5,12}{⼈、⽈}
  \begin{Phonetics}{代替}{dai4ti4}[][HSK 4]
    \definition{v.}{substituir; substituir por; tomar o lugar de}
  \end{Phonetics}
\end{Entry}

\begin{Entry}{令}{5}{⼈}
  \begin{Phonetics}{令}{ling2}
    \definition*{s.}{Antigo nome geográfico, na região atual de Linyi, província de Shanxi | Sobrenome Ling}
  \end{Phonetics}
  \begin{Phonetics}{令}{ling3}
    \definition{clas.}{resma (de papel); unidade de medida de papel: 500 folhas inteiras de papel original produzidas mecanicamente equivalem a 1 resma}
  \end{Phonetics}
  \begin{Phonetics}{令}{ling4}[][HSK 5]
    \definition{adj.}{bom; excelente | termos de cortesia usados para se referir aos familiares e parentes da outra pessoa}
    \definition{s.}{ordem; decreto; comando; ordem emitida pela autoridade superior | um título oficial; administradores de certos departamentos governamentais na antiguidade | temporada; estação; clima e fenologia de uma determinada estação | poema-canção; letra curta}
    \definition{v.}{ordenar; comandar | fazer com que alguém; fazer com que; permitir que}
  \end{Phonetics}
\end{Entry}

\begin{Entry}{令人}{5,2}{⼈、⼈}
  \begin{Phonetics}{令人}{ling4ren2}
    \definition{v.}{causar alguém (a fazer alguma coisa) | fazer alguém ficar zangado, encantado, etc.}
  \end{Phonetics}
\end{Entry}

\begin{Entry}{仪}{5}{⼈}
  \begin{Phonetics}{仪}{yi2}
    \definition*{s.}{Sobrenome Yi}
    \definition{s.}{aparência; porte | cerimônia; rito | presente; dádiva | aparelho; instrumento}
    \definition{v.}{(literário) admirar; ansiar por}
  \end{Phonetics}
\end{Entry}

\begin{Entry}{仪式}{5,6}{⼈、⼷}
  \begin{Phonetics}{仪式}{yi2shi4}[][HSK 6]
    \definition{s.}{rito; cerimônia; procedimento e formato da cerimônia}
  \end{Phonetics}
\end{Entry}

\begin{Entry}{仪器}{5,16}{⼈、⼝}
  \begin{Phonetics}{仪器}{yi2qi4}[][HSK 6]
    \definition[台]{s.}{aparelho; instrumento; ferramentas ou equipamentos utilizados para observação, medição, inspeção, etc. em pesquisas ou experimentos científicos; geralmente, são relativamente precisos e padronizados}
  \end{Phonetics}
\end{Entry}

\begin{Entry}{们}{5}{⼈}
  \begin{Phonetics}{们}{men5}[][HSK 1]
    \definition{suf.}{usado após pronomes ou substantivos que se referem a pessoas para indicar pluralidade}
  \end{Phonetics}
\end{Entry}

\begin{Entry}{仰}{6}{⼈}
  \begin{Phonetics}{仰}{yang3}[][HSK 6]
    \definition*{s.}{Sobrenome Yang}
    \definition{v.}{levantar (oposto a 俯) | virar para cima | admirar; respeitar | confiar em; depender de}
  \seealsoref{俯}{fu3}
  \end{Phonetics}
\end{Entry}

\begin{Entry}{件}{6}{⼈}
  \begin{Phonetics}{件}{jian4}[][HSK 2]
    \definition*{s.}{Sobrenome Jian}
    \definition{clas.}{item; peça; artigo; usado para coisas individuais}
    \definition{s.}{refere-se a coisas que podem ser contadas uma a uma | papel; carta; documento; correspondência}
  \end{Phonetics}
\end{Entry}

\begin{Entry}{价}{6}{⼈}
  \begin{Phonetics}{价}{jia4}[][HSK 5]
    \definition{s.}{preço | valor; (figurativo) valores (éticos, culturais etc.) | Química: valência}
  \end{Phonetics}
\end{Entry}

\begin{Entry}{价值}{6,10}{⼈、⼈}
  \begin{Phonetics}{价值}{jia4zhi2}[][HSK 3]
    \definition{s.}{valor; o trabalho social necessário condensado nos produtos | valor; importância; efeitos positivos}
  \end{Phonetics}
\end{Entry}

\begin{Entry}{价格}{6,10}{⼈、⽊}
  \begin{Phonetics}{价格}{jia4ge2}[][HSK 3]
    \definition[个,种]{s.}{preço; tarifa; o valor monetário da mercadoria}
  \end{Phonetics}
\end{Entry}

\begin{Entry}{价钱}{6,10}{⼈、⾦}
  \begin{Phonetics}{价钱}{jia4 qian2}[][HSK 3]
    \definition[个,种,笔]{s.}{preço}
  \end{Phonetics}
\end{Entry}

\begin{Entry}{任}{6}{⼈}
  \begin{Phonetics}{任}{ren4}[][HSK 3]
    \definition{clas.}{usado para o número de mandatos cumpridos em um cargo oficial}
    \definition{conj.}{não importa (como, o que, etc.); orações de conexão, ou usadas antes de pronomes interrogativos, para expressar incondicionalidade, equivalente a 不管 ou 无论}
    \definition{s.}{escritório; posto oficial; cargo | dever; fardo; responsabilidade}
    \definition{v.}{nomear; designar alguém para um cargo | assumir um emprego; assumir um posto; assumir uma posição | deixar; permitir; dar rédea solta a | suportar; empreender | ceder; permitir sem restrições; deixar (alguém) fazer o que quiser}
  \seealsoref{不管}{bu4guan3}
  \seealsoref{无论}{wu2lun4}
  \end{Phonetics}
\end{Entry}

\begin{Entry}{任务}{6,5}{⼈、⼒}
  \begin{Phonetics}{任务}{ren4wu5}[][HSK 3]
    \definition[项,个,种,些]{s.}{tarefa; dever; missão; designação; trabalho designado; responsabilidades designadas}
  \end{Phonetics}
\end{Entry}

\begin{Entry}{任何}{6,7}{⼈、⼈}
  \begin{Phonetics}{任何}{ren4he2}[][HSK 3]
    \definition{pron.}{qualquer; qualquer que seja; o que for; não importa o que}
  \end{Phonetics}
\end{Entry}

\begin{Entry}{任凭}{6,8}{⼈、⼏}
  \begin{Phonetics}{任凭}{ren4 ping2}
    \definition{conj.}{não importa (como, o quê, etc.) | mesmo que; embora}
    \definition{v.}{permitir; deixar (algo como: fazer o que lhe agrada); conforme a conveniência de alguém}
  \end{Phonetics}
\end{Entry}

\begin{Entry}{份}{6}{⼈}
  \begin{Phonetics}{份}{fen4}
    \definition{clas.}{usado para emparelhar itens em grupos | usado para jornais, documentos, etc. | usado para partes de um todo | usado para aparência, estado, etc.}
    \definition{s.}{porção; parte | a unidade de divisão; usado após 省, 县, 年, 月,  indica a unidade de divisão | grau; extensão de algo}
  \seealsoref{年}{nian2}
  \seealsoref{省}{sheng3}
  \seealsoref{县}{xian4}
  \seealsoref{月}{yue4}
  \end{Phonetics}
\end{Entry}

\begin{Entry}{仿}{6}{⼈}
  \begin{Phonetics}{仿}{fang3}
    \definition{adv.}{semelhante; como}
    \definition{s.}{caracteres escritos segundo um modelo de caligrafia | cartas modeladas a partir de uma cópia; palavras escritas de acordo com o modelo}
    \definition{v.}{imitar; copiar | assemelhar-se; ser como}
  \end{Phonetics}
\end{Entry}

\begin{Entry}{仿佛}{6,7}{⼈、⼈}
  \begin{Phonetics}{仿佛}{fang3fu2}[][HSK 6]
    \definition{adv.}{parece que; como se}
    \definition{v.}{ser como; parecer}
  \end{Phonetics}
\end{Entry}

\begin{Entry}{企}{6}{⼈}
  \begin{Phonetics}{企}{qi3}
    \definition{v.}{ficar na ponta dos pés | esperar ansiosamente por algo; ansiar por | planejar um projeto}
  \end{Phonetics}
\end{Entry}

\begin{Entry}{企业}{6,5}{⼈、⼀}
  \begin{Phonetics}{企业}{qi3ye4}[][HSK 4]
    \definition[家,个]{s.}{empresa; estabelecimento; empreendimento; negócio; setores envolvidos em atividades econômicas como produção, transporte, comércio, etc., como fábricas, minas, ferrovias, empresas comerciais, etc.}
  \end{Phonetics}
\end{Entry}

\begin{Entry}{企图}{6,8}{⼈、⼞}
  \begin{Phonetics}{企图}{qi3tu2}[][HSK 6]
    \definition[种]{s.}{plano; tentativa; intenção (principalmente negativa)}
    \definition{v.}{procurar; tentar; pretender}
  \end{Phonetics}
\end{Entry}

\begin{Entry}{伊}{6}{⼈}
  \begin{Phonetics}{伊}{yi1}
    \definition*{s.}{Iraque, abreviação de 伊拉克 | Irã,abreviação de  伊朗 | Sobrenome Yi}
    \definition{part.}{(chinês clássico) partícula introdutória sem significado específico}
    \definition{pron.}{(antigo) pronome de terceira pessoa do singular ("ele" ou "ela") | pronome de segunda pessoa do singular ("você") | que (precedendo um substantivo)}
  \seealsoref{伊拉克}{yi1la1ke4}
  \seealsoref{伊朗}{yi1lang3}
  \end{Phonetics}
\end{Entry}

\begin{Entry}{伊马姆}{6,3,8}{⼈、⾺、⼥}
  \begin{Phonetics}{伊马姆}{yi1ma3mu3}
    \definition*{s.}{Islã}
  \seealsoref{伊玛目}{yi1ma3mu4}
  \seealsoref{伊曼}{yi1man4}
  \seealsoref{伊斯兰}{yi1si1lan2}
  \end{Phonetics}
\end{Entry}

\begin{Entry}{伊玛目}{6,7,5}{⼈、⽟、⽬}
  \begin{Phonetics}{伊玛目}{yi1ma3mu4}
    \definition*{s.}{Islã}
  \seealsoref{伊马姆}{yi1ma3mu3}
  \seealsoref{伊曼}{yi1man4}
  \seealsoref{伊斯兰}{yi1si1lan2}
  \end{Phonetics}
\end{Entry}

\begin{Entry}{伊拉克}{6,8,7}{⼈、⼿、⼗}
  \begin{Phonetics}{伊拉克}{yi1la1ke4}
    \definition*{s.}{Iraque}
  \end{Phonetics}
\end{Entry}

\begin{Entry}{伊朗}{6,10}{⼈、⽉}
  \begin{Phonetics}{伊朗}{yi1lang3}
    \definition*{s.}{Irã}
  \end{Phonetics}
\end{Entry}

\begin{Entry}{伊曼}{6,11}{⼈、⽈}
  \begin{Phonetics}{伊曼}{yi1man4}
    \definition*{s.}{Islã}
  \seealsoref{伊马姆}{yi1ma3mu3}
  \seealsoref{伊玛目}{yi1ma3mu4}
  \seealsoref{伊斯兰}{yi1si1lan2}
  \end{Phonetics}
\end{Entry}

\begin{Entry}{伊斯兰}{6,12,5}{⼈、⽄、⼋}
  \begin{Phonetics}{伊斯兰}{yi1si1lan2}
    \definition*{s.}{Islã}
  \seealsoref{伊马姆}{yi1ma3mu3}
  \seealsoref{伊玛目}{yi1ma3mu4}
  \seealsoref{伊曼}{yi1man4}
  \end{Phonetics}
\end{Entry}

\begin{Entry}{休}{6}{⼈}
  \begin{Phonetics}{休}{xiu1}
    \definition{adj.}{feliz; alegre; festivo}
    \definition{adv.}{não; indica proibição ou dissuasão, equivalente a 别 ou 不要}
    \definition{s.}{fortuna e infortúnio; bom e mau}
    \definition{v.}{parar; cessar | descansar | abandonar a esposa e mandá-la para casa; antigamente, o marido mandava a esposa de volta para a casa dos pais e rompia o relacionamento conjugal}
  \seealsoref{别}{bie2}
  \seealsoref{不要}{bu2 yao4}
  \end{Phonetics}
\end{Entry}

\begin{Entry}{休兵}{6,7}{⼈、⼋}
  \begin{Phonetics}{休兵}{xiu1bing1}
    \definition{s.}{armistício; cessar fogo}
    \definition{v.}{cessar fogo}
  \end{Phonetics}
\end{Entry}

\begin{Entry}{休闲}{6,7}{⼈、⾨}
  \begin{Phonetics}{休闲}{xiu1xian2}[][HSK 5]
    \definition{s.}{ócio; lazer; tempo livre}
    \definition{v.}{desfrutar do lazer; sair de férias; aproveitar o tempo livre; parar de trabalhar ou estudar, estar em um estado de lazer e descontração | ficar ocioso}
  \end{Phonetics}
\end{Entry}

\begin{Entry}{休息}{6,10}{⼈、⼼}
  \begin{Phonetics}{休息}{xiu1xi5}[][HSK 1]
    \definition{s.}{descanço}
    \definition{v.}{descansar; descansar um pouco; fazer uma pausa; interromper o trabalho, os estudos ou as atividades para recuperar as energias | dormir}
  \end{Phonetics}
\end{Entry}

\begin{Entry}{休息室}{6,10,9}{⼈、⼼、⼧}
  \begin{Phonetics}{休息室}{xiu1xi1shi4}
    \definition{s.}{saguão | salão}
  \end{Phonetics}
\end{Entry}

\begin{Entry}{休假}{6,11}{⼈、⼈}
  \begin{Phonetics}{休假}{xiu1 jia4}[][HSK 2]
    \definition{v.+compl.}{ter um feriado; tirar férias; sair de férias}
  \end{Phonetics}
\end{Entry}

\begin{Entry}{休憩}{6,16}{⼈、⼼}
  \begin{Phonetics}{休憩}{xiu1qi4}
    \definition{v.}{relaxar | descansar | dar um tempo}
  \end{Phonetics}
\end{Entry}

\begin{Entry}{休整}{6,16}{⼈、⽁}
  \begin{Phonetics}{休整}{xiu1zheng3}
    \definition{v.}{(militar) descansar e reorganizar}
  \end{Phonetics}
\end{Entry}

\begin{Entry}{众}{6}{⼈}
  \begin{Phonetics}{众}{zhong4}
    \definition*{s.}{Câmara dos Deputados, abreviação de 众议院}
    \definition{adj.}{numerosos}
    \definition{s.}{multidão; as massas}
  \seealsoref{众议院}{zhong4yi4yuan4}
  \end{Phonetics}
\end{Entry}

\begin{Entry}{众议院}{6,5,9}{⼈、⾔、⾩}
  \begin{Phonetics}{众议院}{zhong4yi4yuan4}
    \definition*{s.}{Casa baixa da Assembléia Bicameral | Câmara dos Deputados}
  \end{Phonetics}
\end{Entry}

\begin{Entry}{众多}{6,6}{⼈、⼣}
  \begin{Phonetics}{众多}{zhong4 duo1}[][HSK 5]
    \definition{adj.}{muitos; numerosos; multitudinários}
  \end{Phonetics}
\end{Entry}

\begin{Entry}{优}{6}{⼈}
  \begin{Phonetics}{优}{you1}
    \definition{adj.}{excelente; bom; excepcional | amplo; abundante}
    \definition{s.}{Arcaico: ator ou atriz}
    \definition{s.}{Sobrenome You}
    \definition{v.}{dar tratamento preferencial}
  \end{Phonetics}
\end{Entry}

\begin{Entry}{优于}{6,3}{⼈、⼆}
  \begin{Phonetics}{优于}{you1yu2}
    \definition{v.}{superar}
  \end{Phonetics}
\end{Entry}

\begin{Entry}{优先}{6,6}{⼈、⼉}
  \begin{Phonetics}{优先}{you1xian1}[][HSK 5]
    \definition{adj.}{anterior; sênior; subjacente}
    \definition{v.}{ter prioridade; ter precedência; colocar-se à frente de outras pessoas ou assuntos}
  \end{Phonetics}
\end{Entry}

\begin{Entry}{优伶}{6,7}{⼈、⼈}
  \begin{Phonetics}{优伶}{you1ling2}
    \definition{s.}{ator}
  \end{Phonetics}
\end{Entry}

\begin{Entry}{优秀}{6,7}{⼈、⽲}
  \begin{Phonetics}{优秀}{you1xiu4}[][HSK 4]
    \definition{adj.}{esplêndido; excelente; extraordinário; excepcional; notável; descreve moral, qualidades, realizações, aprendizado, etc. muito bons.}
  \end{Phonetics}
\end{Entry}

\begin{Entry}{优良}{6,7}{⼈、⾉}
  \begin{Phonetics}{优良}{you1 liang2}[][HSK 4]
    \definition{adj.}{ótimo; bom; excelente; (variedade, qualidade, desempenho, estilo, etc.) muito bom}
  \end{Phonetics}
\end{Entry}

\begin{Entry}{优势}{6,8}{⼈、⼒}
  \begin{Phonetics}{优势}{you1shi4}[][HSK 3]
    \definition[种,个]{s.}{vantagem; superioridade; preponderância; posição dominante; uma situação favorável que permite superar o adversário}
  \end{Phonetics}
\end{Entry}

\begin{Entry}{优质}{6,8}{⼈、⾙}
  \begin{Phonetics}{优质}{you1 zhi4}[][HSK 6]
    \definition{adj.}{excelente qualidade; alta qualidade; qualidade superior; alto grau}
  \end{Phonetics}
\end{Entry}

\begin{Entry}{优厚}{6,9}{⼈、⼚}
  \begin{Phonetics}{优厚}{you1hou4}
    \definition{adj.}{generoso}
  \end{Phonetics}
\end{Entry}

\begin{Entry}{优点}{6,9}{⼈、⽕}
  \begin{Phonetics}{优点}{you1dian3}[][HSK 3]
    \definition[个,项,种,些]{s.}{mérito; virtude; ponto forte; vantagem (em oposição a 缺点)}
  \seealsoref{缺点}{que1dian3}
  \end{Phonetics}
\end{Entry}

\begin{Entry}{优美}{6,9}{⼈、⽺}
  \begin{Phonetics}{优美}{you1mei3}[][HSK 4]
    \definition{adj.}{fino; elegante; gracioso; bonito}
  \end{Phonetics}
\end{Entry}

\begin{Entry}{优选}{6,9}{⼈、⾡}
  \begin{Phonetics}{优选}{you1xuan3}
    \definition{v.}{otimizar}
  \end{Phonetics}
\end{Entry}

\begin{Entry}{优格}{6,10}{⼈、⽊}
  \begin{Phonetics}{优格}{you1ge2}
    \definition{s.}{iogurte}
  \end{Phonetics}
\end{Entry}

\begin{Entry}{优盘}{6,11}{⼈、⽫}
  \begin{Phonetics}{优盘}{you1pan2}
    \definition{s.}{unidade de memória USB}
  \seealsoref{闪存盘}{shan3cun2pan2}
  \end{Phonetics}
\end{Entry}

\begin{Entry}{优惠}{6,12}{⼈、⼼}
  \begin{Phonetics}{优惠}{you1hui4}[][HSK 5]
    \definition{adj.}{especial; pechincha; reduzido; com desconto | favorável; preferencial; melhores condições ou tratamento do que o normal, permitindo que as pessoas obtenham mais benefícios}
  \end{Phonetics}
\end{Entry}

\begin{Entry}{优等}{6,12}{⼈、⽵}
  \begin{Phonetics}{优等}{you1deng3}
    \definition{adj.}{excelente | de primeira linha | alta classe | da mais alta ordem, superior}
  \end{Phonetics}
\end{Entry}

\begin{Entry}{优裕}{6,12}{⼈、⾐}
  \begin{Phonetics}{优裕}{you1yu4}
    \definition{adj.}{abundante | bastante}
    \definition{s.}{abundância}
  \end{Phonetics}
\end{Entry}

\begin{Entry}{伙}{6}{⼈}
  \begin{Phonetics}{伙}{huo3}[][HSK 4]
    \definition{clas.}{grupo; multidão; banda}
    \definition{s.}{iguaria; alimentação; refeições | parceiro; companheiro | coletivo de colegas}
    \definition{v.}{combinar; unir}
  \end{Phonetics}
\end{Entry}

\begin{Entry}{伙伴}{6,7}{⼈、⼈}
  \begin{Phonetics}{伙伴}{huo3ban4}[][HSK 4]
    \definition[个,位,群]{s.}{parceiro; companheiro; antigo sistema militar de dez pessoas para uma fogueira, o chefe da fogueira, uma pessoa encarregada de cozinhar, com a fogueira é chamado de parceiro da fogueira, agora se refere à participação comum em uma determinada organização ou engajada em certas atividades}
  \end{Phonetics}
\end{Entry}

\begin{Entry}{会}{6}{⼈}
  \begin{Phonetics}{会}{hui4}[][HSK 1,2]
    \definition{adv.}{um momento}
    \definition{clas.}{momento; um curto período de tempo}
    \definition{s.}{reunião; festa; conferência; reunião com um objetivo específico | reunião; reunião no trabalho | feira do templo; festival religioso | associação; sociedade; sindicato; certas organizações públicas | oportunidade; ocasião; momento oportuno | cidade principal; capital; cidade central}
    \definition{suf.}{união; grupo; associação}
    \definition{v.}{ser provável que; ter certeza de; indica que é possível realizar (é possível responder à pergunta separadamente) |  poder; ser capaz de; significa saber como fazer ou ter a capacidade de fazer (geralmente se refere a coisas que precisam ser aprendidas) | saber; compreender; entender | encontrar; ver | reunir-se; reunir; agregar; juntar | destacar-se em; ser bom em; ser hábil em; indica proficiência | pagar (ou custear) contas}
  \end{Phonetics}
  \begin{Phonetics}{会}{kuai4}
    \definition[个,场,次]{s.}{contabilidade}
    \definition{v.}{computar; calcular; equilibrar uma conta}
  \end{Phonetics}
\end{Entry}

\begin{Entry}{会见}{6,4}{⼈、⾒}
  \begin{Phonetics}{会见}{hui4 jian4}[][HSK 6]
    \definition{v.}{entrevistar; encontrar-se com (especialmente um visitante estrangeiro)}
  \end{Phonetics}
\end{Entry}

\begin{Entry}{会计}{6,4}{⼈、⾔}
  \begin{Phonetics}{会计}{kuai4ji4}[][HSK 4]
    \definition[个,位,名]{s.}{contabilidade | contador; contabilista; guarda-livros; pessoal que trabalha como contador}
  \end{Phonetics}
\end{Entry}

\begin{Entry}{会长}{6,4}{⼈、⾧}
  \begin{Phonetics}{会长}{hui4 zhang3}[][HSK 6]
    \definition[位,名,个,些]{s.}{presidente de uma associação ou sociedade | presidente de um clube, comitê etc.}
  \end{Phonetics}
\end{Entry}

\begin{Entry}{会议}{6,5}{⼈、⾔}
  \begin{Phonetics}{会议}{hui4yi4}[][HSK 3]
    \definition[次,届,个,场]{s.}{reunião; conferência; reunião organizada pela organização relevante para ouvir opiniões, discutir questões e distribuir tarefas | conselho; congresso; um órgão ou organização permanente que discute e trata frequentemente assuntos importantes}
  \end{Phonetics}
\end{Entry}

\begin{Entry}{会员}{6,7}{⼈、⼝}
  \begin{Phonetics}{会员}{hui4 yuan2}[][HSK 3]
    \definition[位,名,个,些]{s.}{membro; associado; membros de certos grupos ou organizações}
  \end{Phonetics}
\end{Entry}

\begin{Entry}{会首}{6,9}{⼈、⾸}
  \begin{Phonetics}{会首}{hui4shou3}
    \definition{s.}{chefe de uma sociedade | patrocinador de uma organização}
  \end{Phonetics}
\end{Entry}

\begin{Entry}{会谈}{6,10}{⼈、⾔}
  \begin{Phonetics}{会谈}{hui4 tan2}[][HSK 5]
    \definition{v.}{manter conversações; comumente usado em assuntos internacionais ou atividades diplomáticas}
  \end{Phonetics}
\end{Entry}

\begin{Entry}{伞}{6}{⼈}
  \begin{Phonetics}{伞}{san3}[][HSK 4]
    \definition*{s.}{Sobrenome San}
    \definition[把]{s.}{guarda-chuva; proteção contra chuva ou sol | algo que tem o formato de um guarda-chuva}
  \end{Phonetics}
\end{Entry}

\begin{Entry}{伟}{6}{⼈}
  \begin{Phonetics}{伟}{wei3}
    \definition*{s.}{Sobrenome Wei}
    \definition{adj.}{grande; ótimo; poderoso | Literário: grande}
  \end{Phonetics}
\end{Entry}

\begin{Entry}{伟大}{6,3}{⼈、⼤}
  \begin{Phonetics}{伟大}{wei3da4}[][HSK 3]
    \definition{adj.}{ótimo; excelente; extrovertido; descreve uma pessoa com moral e qualidades excelentes, habilidades e realizações excepcionais, que inspira grande respeito | ótimo; magnífico; descreve algo de grande importância, com impacto significativo, acima do normal, algo notável}
  \end{Phonetics}
\end{Entry}

\begin{Entry}{传}{6}{⼈}
  \begin{Phonetics}{传}{chuan2}[][HSK 3]
    \definition{v.}{passar; passar adiante | passar adiante; legar; passar de\dots para\dots; passar da geração anterior para a seguinte | transmitir (conhecimento, habilidade, etc.); comunicar; ensinar | espalhar; propagar | transmitir; conduzir; transferir | transmitir; expressar | convocar; dar ordem para chamar alguém | infectar; ser contagioso | enviar documentos por e-mail ou fax}
  \end{Phonetics}
  \begin{Phonetics}{传}{zhuan4}
    \definition{s.}{comentários sobre clássicos; obras que explicam as escrituras| biografia | romances sobre eventos históricos; obras que narram histórias históricas}
  \end{Phonetics}
\end{Entry}

\begin{Entry}{传出}{6,5}{⼈、⼐}
  \begin{Phonetics}{传出}{chuan2 chu1}[][HSK 6]
    \definition{adj.}{eferente (nervo)}
    \definition{v.}{disseminar | transmitir para fora}
  \end{Phonetics}
\end{Entry}

\begin{Entry}{传达}{6,6}{⼈、⾡}
  \begin{Phonetics}{传达}{chuan2da2}[][HSK 5]
    \definition{s.}{recepção e registro de chamadas em um estabelecimento público | zelador | recepcionista}
    \definition{v.}{passar adiante (informações, etc.); transmitir; retransmitir; comunicar; expressar}
  \end{Phonetics}
\end{Entry}

\begin{Entry}{传来}{6,7}{⼈、⽊}
  \begin{Phonetics}{传来}{chuan2 lai2}[][HSK 3]
    \definition{v.}{(um som) passar; transmitir de algum lugar para o local onde o locutor se encontra | (notícias) chegar; transmitir mensagens ou informações}
  \end{Phonetics}
\end{Entry}

\begin{Entry}{传言}{6,7}{⼈、⾔}
  \begin{Phonetics}{传言}{chuan2 yan2}[][HSK 6]
    \definition[个,种,些]{s.}{boato; rumor}
    \definition{v.}{passar uma mensagem | datado: fazer um discurso | falar; fazer uma declaração}
  \end{Phonetics}
\end{Entry}

\begin{Entry}{传奇}{6,8}{⼈、⼤}
  \begin{Phonetics}{传奇}{chuan2qi2}
    \definition[个,项,段]{s.}{contos das dinastias Tang e Song (618-1279); contos de maravilhas | dramas poéticos das dinastias Ming e Qing (1368-1911); dramas em verso | lenda; romance; histórias lendárias}
  \end{Phonetics}
\end{Entry}

\begin{Entry}{传承}{6,8}{⼈、⼿}
  \begin{Phonetics}{传承}{chuan2cheng2}
    \definition{s.}{herança | tradição continuada}
    \definition{v.}{transmitir (para as gerações futuras) | passar adiante (desde os tempos antigos)}
  \end{Phonetics}
\end{Entry}

\begin{Entry}{传给}{6,9}{⼈、⽷}
  \begin{Phonetics}{传给}{chuan2gei3}
    \definition{v.}{passar para | transferir para | entregar a}
  \end{Phonetics}
\end{Entry}

\begin{Entry}{传统}{6,9}{⼈、⽷}
  \begin{Phonetics}{传统}{chuan2tong3}[][HSK 4]
    \definition{adj.}{tradicional; histórico; transmitido de geração em geração | antiquado, conservador e fora de sintonia com os tempos}
    \definition[个,种,项]{s.}{tradição; costume; fatores sociais, como costumes, moral, ideias, estilos, artes, instituições etc., que são transmitidos de uma geração para outra e que são característicos da sociedade}
  \end{Phonetics}
\end{Entry}

\begin{Entry}{传说}{6,9}{⼈、⾔}
  \begin{Phonetics}{传说}{chuan2shuo1}[][HSK 3]
    \definition[个,种,段]{s.}{lenda; conto popular; folclore; coisas lendárias; especificamente, lendas populares}
    \definition{v.}{dizer que; ser dito; passar de boca em boca; transmitir oralmente, segundo a tradição}
  \end{Phonetics}
\end{Entry}

\begin{Entry}{传真}{6,10}{⼈、⼗}
  \begin{Phonetics}{传真}{chuan2zhen1}[][HSK 5]
    \definition[部,台,份,个]{s.}{\emph{FAX}, facsímile; texto, diagramas, fotografias, etc., transmitidos por aparelho de fax}
    \definition{v.}{enviar um fax | retratar; reproduzir}[她的画传真了古代建筑。===Suas pinturas são reproduções fiéis da arquitetura antiga.]
  \end{Phonetics}
\end{Entry}

\begin{Entry}{传递}{6,10}{⼈、⾡}
  \begin{Phonetics}{传递}{chuan2 di4}[][HSK 5]
    \definition{v.}{transmitir; entregar; transferir; passar adiante}
  \end{Phonetics}
\end{Entry}

\begin{Entry}{传媒}{6,12}{⼈、⼥}
  \begin{Phonetics}{传媒}{chuan2 mei2}[][HSK 6]
    \definition{s.}{meios de comunicação de massa; mídia; jornais, rádio, televisão, \emph{Internet} e outras ferramentas de notícias | meio; veículo; vetor; o meio ou via de transmissão da doença}
  \end{Phonetics}
\end{Entry}

\begin{Entry}{传输}{6,13}{⼈、⾞}
  \begin{Phonetics}{传输}{chuan2 shu1}[][HSK 6]
    \definition{v.}{transmitir, transportar (energia, informação, etc.)}
  \end{Phonetics}
\end{Entry}

\begin{Entry}{传播}{6,15}{⼈、⼿}
  \begin{Phonetics}{传播}{chuan2bo1}[][HSK 3]
    \definition{v.}{espalhar; difundir; propagar; disseminar}
  \end{Phonetics}
\end{Entry}

\begin{Entry}{伤}{6}{⼈}
  \begin{Phonetics}{伤}{shang1}[][HSK 3]
    \definition*{s.}{Sobrenome Shang}
    \definition[处]{s.}{ferida; ferimento}
    \definition{v.}{ferir; machucar | ter os sentimentos feridos | estar angustiado | enjoar de algo; desenvolver aversão a algo | ser prejudicial a; entravar}
  \end{Phonetics}
\end{Entry}

\begin{Entry}{伤亡}{6,3}{⼈、⼇}
  \begin{Phonetics}{伤亡}{shang1 wang2}[][HSK 6]
    \definition{s.}{ferimentos e mortes; feridos e mortos; pessoas feridas e mortas; baixas}
    \definition{v.}{ser ferido e morto}
  \end{Phonetics}
\end{Entry}

\begin{Entry}{伤口}{6,3}{⼈、⼝}
  \begin{Phonetics}{伤口}{shang1 kou3}[][HSK 6]
    \definition[处]{s.}{corte; ferida; onde a pele, os músculos, etc. são feridos, rompidos ou onde são realizadas aberturas cirúrgicas}
  \end{Phonetics}
\end{Entry}

\begin{Entry}{伤心}{6,4}{⼈、⼼}
  \begin{Phonetics}{伤心}{shang1xin1}[][HSK 3]
    \definition{v.+compl.}{estar triste; lamentar; estar com o coração partido; sentir-se triste por causa de infortúnio ou decepção}
  \end{Phonetics}
\end{Entry}

\begin{Entry}{伤员}{6,7}{⼈、⼝}
  \begin{Phonetics}{伤员}{shang1 yuan2}[][HSK 6]
    \definition[名,位,个]{s.}{Exército: pessoal ferido; os feridos}
  \end{Phonetics}
\end{Entry}

\begin{Entry}{伤害}{6,10}{⼈、⼧}
  \begin{Phonetics}{伤害}{shang1hai4}[][HSK 4]
    \definition[种]{v.}{ferir; prejudicar; machucar; magoar; causar danos físicos ou mentais}
  \end{Phonetics}
\end{Entry}

\begin{Entry}{伦}{6}{⼈}
  \begin{Phonetics}{伦}{lun2}
    \definition*{s.}{Sobrenome Lun}
    \definition{s.}{relações humanas (especialmente como concebidas pela ética feudal) | lógica; ordem | par; correspondência; (mesma) classe | ética; relações humanas | sequência lógica; ordem | o mesmo tipo; semelhante; igual}
  \end{Phonetics}
\end{Entry}

\begin{Entry}{伦敦}{6,12}{⼈、⽁}
  \begin{Phonetics}{伦敦}{lun2dun1}
    \definition*{s.}{Londres}
  \end{Phonetics}
\end{Entry}

\begin{Entry}{伪}{6}{⼈}
  \begin{Phonetics}{伪}{wei3}
    \definition{adj.}{falso; falsificado | fantoche; colaboracionista; ilegal | forjado; falso}
    \definition{pref.}{pseudo-; quasi-; quase-}
  \end{Phonetics}
\end{Entry}

\begin{Entry}{似}{6}{⼈}
  \begin{Phonetics}{似}{shi4}
    \definition{v.}{ver; parecer}
  \end{Phonetics}
  \begin{Phonetics}{似}{si4}
    \definition*{s.}{Sobrenome Si}
    \definition{adv.}{parece; como se}
    \definition{v.}{ser semelhante; parecer-se com | parecer; aparecer | exceder}
  \end{Phonetics}
\end{Entry}

\begin{Entry}{似乎}{6,5}{⼈、⼃}
  \begin{Phonetics}{似乎}{si4hu1}[][HSK 4]
    \definition{adv.}{como se; aparentemente; se parece como}
  \end{Phonetics}
\end{Entry}

\begin{Entry}{似的}{6,8}{⼈、⽩}
  \begin{Phonetics}{似的}{shi4de5}[][HSK 4]
    \definition{part.}{como; como\dots como; como se (embora); usada após uma palavra ou frase para indicar uma semelhança com algo ou uma situação | usada para indicar alto grau}
  \end{Phonetics}
\end{Entry}

\begin{Entry}{似曾相识}{6,12,9,7}{⼈、⽈、⽬、⾔}
  \begin{Phonetics}{似曾相识}{si4ceng2xiang1shi2}
    \definition{s.}{\emph{déjà vu} (a experiência de ver exatamente a mesma situação pela segunda vez) | situação aparentemente familiar}
  \end{Phonetics}
\end{Entry}

\begin{Entry}{估}{7}{⼈}
  \begin{Phonetics}{估}{gu1}
    \definition{v.}{estimar; avaliar; aferir}
  \end{Phonetics}
  \begin{Phonetics}{估}{gu4}
    \definition{adj.}{velho | roupas de segunda mão}
  \end{Phonetics}
\end{Entry}

\begin{Entry}{估计}{7,4}{⼈、⾔}
  \begin{Phonetics}{估计}{gu1ji4}[][HSK 5]
    \definition{v.}{fazer contas; estimar; calcular; julgar a natureza, quantidade, mudança, etc. de uma coisa em uma determinada situação | parecer; parecer como se; aparentar; fazer inferências aproximadas sobre a natureza, a quantidade e a mudança das coisas com base em determinadas circunstâncias}
  \end{Phonetics}
\end{Entry}

\begin{Entry}{伲}{7}{⼈}
  \begin{Phonetics}{伲}{ni4}
    \definition{pron.}{Dialeto: eu; nós; meu; nosso}
  \seealsoref{你}{ni3}
  \end{Phonetics}
\end{Entry}

\begin{Entry}{伴}{7}{⼈}
  \begin{Phonetics}{伴}{ban4}
    \definition[个,位]{s.}{companheiro; parceiro}
    \definition{v.}{acompanhar; estar perto}[伴君如伴虎。===Acompanhar o rei é como acompanhar um tigre.]
  \end{Phonetics}
\end{Entry}

\begin{Entry}{伴侣}{7,8}{⼈、⼈}
  \begin{Phonetics}{伴侣}{ban4lv3}
    \definition{s.}{companheiro | parceiro}
  \end{Phonetics}
\end{Entry}

\begin{Entry}{伸}{7}{⼈}
  \begin{Phonetics}{伸}{shen1}[][HSK 5]
    \definition{v.}{alongar; esticar; estender}
  \end{Phonetics}
\end{Entry}

\begin{Entry}{但}{7}{⼈}
  \begin{Phonetics}{但}{dan4}[][HSK 2]
    \definition*{s.}{Sobrenome Dan}
    \definition{adv.}{apenas; meramente; indica uma restrição ao âmbito da ação, equivalente a 只 ou 仅}
    \definition{conj.}{mas; ainda assim; mesmo assim; no entanto; contudo; usado na última oração, conecta duas orações, expressando uma relação de transição, equivalente a 可是 ou 不过}
  \seealsoref{不过}{bu2guo4}
  \seealsoref{仅}{jin3}
  \seealsoref{可是}{ke3shi4}
  \seealsoref{只}{zhi3}
  \end{Phonetics}
\end{Entry}

\begin{Entry}{但是}{7,9}{⼈、⽇}
  \begin{Phonetics}{但是}{dan4 shi4}[][HSK 2]
    \definition{conj.}{mas; contudo; no entanto; mesmo assim; usado na segunda parte da frase para indicar uma mudança, geralmente acompanhada de expressões como 虽然 ou 尽管}
  \seealsoref{尽管}{jin3guan3}
  \seealsoref{虽然}{sui1 ran2}
  \end{Phonetics}
\end{Entry}

\begin{Entry}{位}{7}{⼈}
  \begin{Phonetics}{位}{wei4}[][HSK 2]
    \definition*{s.}{Sobrenome Wei}
    \definition{clas.}{usado para pessoas (com cortesia, respeito) | usado para bits binários}[十六位===16 bits]
    \definition{s.}{lugar; localização; o lugar onde ou onde alguém está localizado | posto; \emph{status}; posição; a posição de uma pessoa em uma determinada área da vida social | trono; refere-se especificamente ao status do imperador | lugar; dígito; a posição de cada dígito em um número}
  \end{Phonetics}
\end{Entry}

\begin{Entry}{位于}{7,3}{⼈、⼆}
  \begin{Phonetics}{位于}{wei4yu2}[][HSK 4]
    \definition{v.}{estar localizado; estar situado}
  \end{Phonetics}
\end{Entry}

\begin{Entry}{位子}{7,3}{⼈、⼦}
  \begin{Phonetics}{位子}{wei4zi5}
    \definition{s.}{lugar | assento}
  \end{Phonetics}
\end{Entry}

\begin{Entry}{位居}{7,8}{⼈、⼫}
  \begin{Phonetics}{位居}{wei4ju1}
    \definition{v.}{estar localizado em}
  \end{Phonetics}
\end{Entry}

\begin{Entry}{位置}{7,13}{⼈、⽹}
  \begin{Phonetics}{位置}{wei4zhi4}[][HSK 4]
    \definition[个,处]{s.}{assento; lugar; localização | lugar; posição; \emph{status} | posição (por exemplo: cargo no escritório)}
  \end{Phonetics}
\end{Entry}

\begin{Entry}{低}{7}{⼈}
  \begin{Phonetics}{低}{di1}[][HSK 2]
    \definition*{s.}{Sobrenome Di}
    \definition{adj.}{baixo; distância pequena de baixo para cima; próximo ao solo | abaixo da média; abaixo do padrão geral | inferior (em grau); de nível inferior}
    \definition{v.}{deixar cair; pendurar; abaixar (a cabeça)}
  \end{Phonetics}
\end{Entry}

\begin{Entry}{低于}{7,3}{⼈、⼆}
  \begin{Phonetics}{低于}{di1 yu2}[][HSK 5]
    \definition{v.}{ser inferior a; algo ou fenômeno é, de alguma forma, inferior ou pior do que outra coisa}
  \end{Phonetics}
\end{Entry}

\begin{Entry}{低头}{7,5}{⼈、⼤}
  \begin{Phonetics}{低头}{di1 tou2}[][HSK 6]
    \definition{v.}{abaixar a cabeça; curvar a cabeça; pendurar a cabeça | ceder; submeter-se; refere-se à rendição e à admissão da derrota}
  \end{Phonetics}
\end{Entry}

\begin{Entry}{低温}{7,12}{⼈、⽔}
  \begin{Phonetics}{低温}{di1 wen1}[][HSK 6]
    \definition{s.}{baixa temperatura | Meteorologia: microtermia  | Medicina: hipotermia}
  \end{Phonetics}
\end{Entry}

\begin{Entry}{低等}{7,12}{⼈、⽵}
  \begin{Phonetics}{低等}{di1 deng3}
    \definition{adj.}{inferior; classe baixa (oposto a 高等) | inferior}
  \seealsoref{高等}{gao1 deng3}
  \end{Phonetics}
\end{Entry}

\begin{Entry}{低潮}{7,15}{⼈、⽔}
  \begin{Phonetics}{低潮}{di1chao2}
    \definition{s.}{maré baixa/vazante; o nível mais baixo da maré durante um ciclo de maré (distinto da 高潮) | vazante baixa; o ponto mais baixo; uma metáfora para o baixo estágio de desenvolvimento das coisas}
  \seealsoref{高潮}{gao1chao2}
  \end{Phonetics}
\end{Entry}

\begin{Entry}{住}{7}{⼈}
  \begin{Phonetics}{住}{zhu4}[][HSK 1]
    \definition{adv.}{firmemente; indica estabilidade ou firmeza}
    \definition{v.}{viver; residir; morar; ficar | parar; cessar | (após um verbo) com firmeza; até parar | hospedar; acomodar | parar; interromper | ser competente; ser qualificado; estar à altura; usado com 得 ou 不, indica que a força é suficiente (ou insuficiente)}
  \seealsoref{不}{bu4}
  \seealsoref{得}{de5}
  \end{Phonetics}
\end{Entry}

\begin{Entry}{住处}{7,5}{⼈、⼡}
  \begin{Phonetics}{住处}{zhu4chu4}
    \definition{s.}{morada | habitação | residência}
  \end{Phonetics}
\end{Entry}

\begin{Entry}{住宅}{7,6}{⼈、⼧}
  \begin{Phonetics}{住宅}{zhu4zhai2}[][HSK 6]
    \definition[套,处,栋,座]{s.}{habitação; residência}
  \end{Phonetics}
\end{Entry}

\begin{Entry}{住房}{7,8}{⼈、⼾}
  \begin{Phonetics}{住房}{zhu4fang2}[][HSK 2]
    \definition[套,处]{s.}{habitação; alojamento; casas para as pessoas morarem}
  \end{Phonetics}
\end{Entry}

\begin{Entry}{住所}{7,8}{⼈、⼾}
  \begin{Phonetics}{住所}{zhu4suo3}
    \definition[处]{s.}{morada | habitação | residência}
  \end{Phonetics}
\end{Entry}

\begin{Entry}{住院}{7,9}{⼈、⾩}
  \begin{Phonetics}{住院}{zhu4 yuan4}[][HSK 2]
    \definition{v.}{estar hospitalizado; estar no hospital; ser internado no hospital para tratamento}
  \end{Phonetics}
\end{Entry}

\begin{Entry}{住嘴}{7,16}{⼈、⼝}
  \begin{Phonetics}{住嘴}{zhu4zui3}
    \definition{interj.}{Cale-se!}
    \definition{v.}{calar | calar-se}
  \end{Phonetics}
\end{Entry}

\begin{Entry}{体}{7}{⼈}
  \begin{Phonetics}{体}{ti3}
    \definition{s.}{corpo; parte do corpo | substância; objeto; estado de uma substância | estilo; forma | sistema | estilo de caligrafia | tipo de letra; fonte | (linguística) aspecto (de um verbo) | estrutura; a forma escrita do texto; o gênero da obra}
    \definition{v.}{fazer ou vivenciar algo pessoalmente | colocar-se na posição de outro; colocar-se mentalmente na posição do outro; colocar-se no lugar do outro}
  \end{Phonetics}
\end{Entry}

\begin{Entry}{体力}{7,2}{⼈、⼒}
  \begin{Phonetics}{体力}{ti3 li4}[][HSK 5]
    \definition{s.}{força física; vigor físico (ou corporal); a força do corpo humano para sustentar suas próprias atividades}
  \end{Phonetics}
\end{Entry}

\begin{Entry}{体内}{7,4}{⼈、⼌}
  \begin{Phonetics}{体内}{ti3nei4}
    \definition{adj.}{dentro do corpo | \emph{in vivo} (versus \emph{in vitro} | interno a}
  \end{Phonetics}
\end{Entry}

\begin{Entry}{体会}{7,6}{⼈、⼈}
  \begin{Phonetics}{体会}{ti3hui4}[][HSK 3]
    \definition[个,些,种]{s.}{conhecimento; compreensão; experiência pessoal}
    \definition{v.}{perceber; saber (ou aprender) com a experiência}
  \end{Phonetics}
\end{Entry}

\begin{Entry}{体现}{7,8}{⼈、⾒}
  \begin{Phonetics}{体现}{ti3xian4}[][HSK 3]
    \definition{v.}{refletir; incorporar; encarnar; uma certa qualidade ou fenômeno se manifesta especificamente em uma determinada coisa}
  \end{Phonetics}
\end{Entry}

\begin{Entry}{体育}{7,8}{⼈、⾁}
  \begin{Phonetics}{体育}{ti3yu4}[][HSK 2]
    \definition{s.}{cultura física; treinamento físico; educação cuja principal tarefa é desenvolver a capacidade física e fortalecer a constituição física, alcançada através da participação em várias atividades esportivas | esportes; atividades esportivas; refere-se a esportes}
  \end{Phonetics}
\end{Entry}

\begin{Entry}{体育场}{7,8,6}{⼈、⾁、⼟}
  \begin{Phonetics}{体育场}{ti3 yu4 chang3}[][HSK 2]
    \definition[个,座]{s.}{estádio; campo esportivo; espaço ao ar livre para a prática de exercícios físicos ou competições esportivas}
  \end{Phonetics}
\end{Entry}

\begin{Entry}{体育馆}{7,8,11}{⼈、⾁、⾷}
  \begin{Phonetics}{体育馆}{ti3 yu4 guan3}[][HSK 2]
    \definition[个,座,家]{s.}{ginásio; locais esportivos ou competições em ambientes fechados geralmente têm arquibancadas fixas}
  \end{Phonetics}
\end{Entry}

\begin{Entry}{体重}{7,9}{⼈、⾥}
  \begin{Phonetics}{体重}{ti3 zhong4}[][HSK 4]
    \definition{s.}{peso corporal}
  \end{Phonetics}
\end{Entry}

\begin{Entry}{体积}{7,10}{⼈、⽲}
  \begin{Phonetics}{体积}{ti3ji1}[][HSK 5]
    \definition[个]{s.}{volume; quantidade; o tamanho do espaço ocupado pelo objeto}
  \end{Phonetics}
\end{Entry}

\begin{Entry}{体验}{7,10}{⼈、⾺}
  \begin{Phonetics}{体验}{ti3yan4}[][HSK 3]
    \definition[种]{s.}{experiência; a sensação adquirida pela experiência pessoal}
    \definition{v.}{aprender através da prática; aprender através da experiência pessoal; entender as coisas através da experiência pessoal}
  \end{Phonetics}
\end{Entry}

\begin{Entry}{体检}{7,11}{⼈、⽊}
  \begin{Phonetics}{体检}{ti3 jian3}[][HSK 4]
    \definition{v.}{fazer um exame médico}
  \end{Phonetics}
\end{Entry}

\begin{Entry}{体操}{7,16}{⼈、⼿}
  \begin{Phonetics}{体操}{ti3 cao1}[][HSK 4]
    \definition{s.}{ginástica; esportes, exercícios ou performances de vários movimentos, sem armas ou com o auxílio de determinados equipamentos}
  \end{Phonetics}
\end{Entry}

\begin{Entry}{何}{7}{⼈}
  \begin{Phonetics}{何}{he2}
    \definition*{s.}{Sobrenome He}
    \definition{adv.}{enfatiza um alto grau de intensidade, equivalente a 多么}
    \definition{pron.}{O que?; Qual?; em nome de pessoas ou coisas, equivalente a 什么 | Onde?; em nome do lugar, equivalente a 哪里 | Por que?; Como?; a razão, é equivalente a 为什么 e 怎么}
  \seealsoref{多么}{duo1me5}
  \seealsoref{哪里}{na3 li3}
  \seealsoref{岂}{qi3}
  \seealsoref{什么}{shen2me5}
  \seealsoref{为什么}{wei4shen2me5}
  \seealsoref{怎}{zen3}
  \seealsoref{怎么}{zen3me5}
  \end{Phonetics}
\end{Entry}

\begin{Entry}{何不}{7,4}{⼈、⼀}
  \begin{Phonetics}{何不}{he2bu4}
    \definition{adv.}{por que não?; use o tom interrogativo para expressar "deveria" ou "pode"}
  \end{Phonetics}
\end{Entry}

\begin{Entry}{何况}{7,7}{⼈、⼎}
  \begin{Phonetics}{何况}{he2kuang4}
    \definition{conj.}{além disso | muito menos}
  \end{Phonetics}
\end{Entry}

\begin{Entry}{何故}{7,9}{⼈、⽁}
  \begin{Phonetics}{何故}{he2gu4}
    \definition{adv.}{qual razão?; por que? | para quê? | qual é o motivo?}
  \end{Phonetics}
\end{Entry}

\begin{Entry}{佛}{7}{⼈}
  \begin{Phonetics}{佛}{fo2}[][HSK 6]
    \definition*{s.}{Buda, abreviação de 佛陀 | Budismo}
    \definition{s.}{imagem de Buda | budista | nome de Buda; escritura budista | uma pessoa que alcançou a perfeição na prática espiritual; budista real | estátua do Buda}
  \seealsoref{佛陀}{fo2tuo2}
  \end{Phonetics}
  \begin{Phonetics}{佛}{fu2}
    \definition{adv.}{aparentemente}
    \definition{s.}{ornamento da cabeça (feminino)}
  \end{Phonetics}
\end{Entry}

\begin{Entry}{佛陀}{7,7}{⼈、⾩}
  \begin{Phonetics}{佛陀}{fo2tuo2}
    \definition{s.}{Buda, um título para Sakyamuni ou uma pessoa que atingiu a iluminação | Buda, uma pessoa que atingiu a Budeidade, ou especificamente Siddhartha Gautama}
  \end{Phonetics}
\end{Entry}

\begin{Entry}{佛教}{7,11}{⼈、⽁}
  \begin{Phonetics}{佛教}{fo2 jiao4}[][HSK 6]
    \definition*{s.}{Budismo; uma das principais religiões do mundo, diz-se que foi fundada por Sakyamuni, um príncipe do antigo reino indiano de Kapilavastu (no atual Nepal), no século VI ou V a.C.; foi amplamente difundida em muitos países asiáticos e introduzida na China no final da Dinastia Han Ocidental}
  \end{Phonetics}
\end{Entry}

\begin{Entry}{作}{7}{⼈}
  \begin{Phonetics}{作}{zuo1}
    \definition{adj.}{(gíria) incômodo}
    \definition{s.}{trabalhador | oficina | (pessoa) de alta manutenção}
  \end{Phonetics}
  \begin{Phonetics}{作}{zuo4}[][HSK 6]
    \definition*{s.}{Sobrenome Zuo}
    \definition{s.}{trabalho; escritos; obras}
    \definition{v.}{subir; aumentar; crescer; aparecer | escrever; compor; criar | afetar; fingir; assumir; fingir deliberadamente ser uma determinada pessoa | considerar como; tomar algo ou alguém por | ter; sentir | ser; agir como; tornar-se; servir como | envolver-se em uma atividade; realizar alguma atividade | fazer; criar}
  \end{Phonetics}
\end{Entry}

\begin{Entry}{作为}{7,4}{⼈、⼂}
  \begin{Phonetics}{作为}{zuo4wei2}[][HSK 4]
    \definition{prep.}{como; na capacidade de; no caráter de; no papel de; em termos de uma certa identidade de uma pessoa ou de uma certa natureza de uma coisa}
    \definition{s.}{ato; ação; conduta; feito; comportamento | conquista; realização; especificamente, uma boa ação}
    \definition{v.}{considerar como; tomar por; olhar como; tratar como | realizar; fazer conquistas; deixar uma marca}
  \end{Phonetics}
\end{Entry}

\begin{Entry}{作文}{7,4}{⼈、⽂}
  \begin{Phonetics}{作文}{zuo4wen2}[][HSK 2]
    \definition[篇]{s.}{ensaio; composição; redação}
    \definition{v.+compl.}{(de alunos) escrever uma redação, artigo ou ensaio}
  \end{Phonetics}
\end{Entry}

\begin{Entry}{作业}{7,5}{⼈、⼀}
  \begin{Phonetics}{作业}{zuo4ye4}[][HSK 2]
    \definition[份,个]{s.}{tarefa escolar; tarefa de casa atribuída pelos professores aos alunos}
    \definition{v.}{trabalhar; executar tarefa}
  \end{Phonetics}
\end{Entry}

\begin{Entry}{作出}{7,5}{⼈、⼐}
  \begin{Phonetics}{作出}{zuo4 chu1}[][HSK 4]
    \definition{v.}{mostrar; tomar (decisões, conclusões, etc. por meio de consideração ou discussão); formar (uma conclusão, decisão, etc.) por meio de consideração ou discussão}
  \end{Phonetics}
\end{Entry}

\begin{Entry}{作用}{7,5}{⼈、⽤}
  \begin{Phonetics}{作用}{zuo4yong4}[][HSK 2]
    \definition[副]{s.}{efeito; ação; função; a influência sobre as coisas; o efeito; a utilidade}
    \definition{v.}{afetar; agir sobre; realizar atividades que têm algum impacto nas coisas}
  \end{Phonetics}
\end{Entry}

\begin{Entry}{作废}{7,8}{⼈、⼴}
  \begin{Phonetics}{作废}{zuo4fei4}[][HSK 6]
    \definition{v.}{anular; tornar inválido; abandonar devido a falha}
  \end{Phonetics}
\end{Entry}

\begin{Entry}{作者}{7,8}{⼈、⽼}
  \begin{Phonetics}{作者}{zuo4zhe3}[][HSK 3]
    \definition[位,名,个]{s.}{autor; escritor; pessoas que escrevem artigos ou criam obras de arte}
  \end{Phonetics}
\end{Entry}

\begin{Entry}{作品}{7,9}{⼈、⼝}
  \begin{Phonetics}{作品}{zuo4pin3}[][HSK 3]
    \definition[个,部,篇,幅]{s.}{obra de arte; obras literárias e artísticas}
  \end{Phonetics}
\end{Entry}

\begin{Entry}{作战}{7,9}{⼈、⼽}
  \begin{Phonetics}{作战}{zuo4 zhan4}[][HSK 6]
    \definition{s.}{lutar; combater; batalhar}
  \end{Phonetics}
\end{Entry}

\begin{Entry}{作家}{7,10}{⼈、⼧}
  \begin{Phonetics}{作家}{zuo4jia1}[][HSK 2]
    \definition[位,名,个,些]{s.}{escritor; autor; pessoas que alcançaram sucesso na criação literária}
  \end{Phonetics}
\end{Entry}

\begin{Entry}{你}{7}{⼈}
  \begin{Phonetics}{你}{ni3}[][HSK 1]
    \definition{pron.}{você (segunda pessoa do singular); refere-se à pessoa com quem se está conversando | (referindo-se a qualquer pessoa) você; um; qualquer um | com 我 ou 你 em estruturas paralelas para indicar várias ou muitas pessoas se comportando da mesma maneira}
  \seealsoref{您}{nin2}
  \seealsoref{我}{wo3}
  \end{Phonetics}
\end{Entry}

\begin{Entry}{你们}{7,5}{⼈、⼈}
  \begin{Phonetics}{你们}{ni3men5}[][HSK 1]
    \definition{pron.}{você (segunda pessoa do plural); refere-se a mais de uma pessoa ou a várias pessoas, incluindo a outra parte}
  \end{Phonetics}
\end{Entry}

\begin{Entry}{你们的}{7,5,8}{⼈、⼈、⽩}
  \begin{Phonetics}{你们的}{ni3men5 de5}
    \definition{pron.}{vossos}
  \end{Phonetics}
\end{Entry}

\begin{Entry}{你好}{7,6}{⼈、⼥}
  \begin{Phonetics}{你好}{ni3hao3}
    \definition{interj.}{Olá! | Oi!}
  \end{Phonetics}
\end{Entry}

\begin{Entry}{你的}{7,8}{⼈、⽩}
  \begin{Phonetics}{你的}{ni3 de5}
    \definition{pron.}{seu}
  \end{Phonetics}
\end{Entry}

\begin{Entry}{佩}{8}{⼈}
  \begin{Phonetics}{佩}{pei4}
    \definition{s.}{um ornamento usado como pingente amarrados em cintos nos tempos antigos}
    \definition{v.}{vestir (na cintura, etc.) | (arcaico) admirar | (arcaico) usar, especialmente uma pistola ou espada, na cintura}
  \end{Phonetics}
\end{Entry}

\begin{Entry}{佩服}{8,8}{⼈、⽉}
  \begin{Phonetics}{佩服}{pei4fu2}
    \definition{v.}{admirar}
  \end{Phonetics}
\end{Entry}

\begin{Entry}{使}{8}{⼈}
  \begin{Phonetics}{使}{shi3}[][HSK 3]
    \definition{conj.}{se; supondo; usado como a primeira cláusula de uma frase complexa; indica uma relação hipotética; equivalente a 假如}
    \definition{s.}{enviado; mensageiro; pessoas em uma missão}
    \definition{v.}{enviar; despachar; dizer a alguém para fazer algo | usar; empregar; aplicar | deixar; chamar; habilitar}
  \seealsoref{假如}{jia3ru2}
  \end{Phonetics}
\end{Entry}

\begin{Entry}{使用}{8,5}{⼈、⽤}
  \begin{Phonetics}{使用}{shi3yong4}[][HSK 2]
    \definition{v.}{usar; empregar; aplicar; fazer com que pessoas, equipamentos, fundos, etc. sirvam a um determinado propósito}
  \end{Phonetics}
\end{Entry}

\begin{Entry}{使劲}{8,7}{⼈、⼒}
  \begin{Phonetics}{使劲}{shi3 jin4}[][HSK 4]
    \definition{v.+compl.}{colocar energia; exercer toda a sua força | esforçar-se para ajudar; colocar energia para ajudar}
  \end{Phonetics}
\end{Entry}

\begin{Entry}{使得}{8,11}{⼈、⼻}
  \begin{Phonetics}{使得}{shi3 de5}[][HSK 5]
    \definition{v.}{ser utilizável; poder ser usado | ser viável; ser exequível; ser possível;  poder fazer | fazer; tornar; causar um determinado resultado (intenção, plano, coisa)}
  \end{Phonetics}
\end{Entry}

\begin{Entry}{例}{8}{⼈}
  \begin{Phonetics}{例}{li4}
    \definition{adj.}{regular; rotineiro}
    \definition{s.}{exemplo; instância | precedente | caso; instância | regras; estatutos; regulamentos}
    \definition{v.}{analogizar}
  \end{Phonetics}
\end{Entry}

\begin{Entry}{例子}{8,3}{⼈、⼦}
  \begin{Phonetics}{例子}{li4 zi5}[][HSK 2]
    \definition[个]{s.}{exemplo; algo usado para ajudar a explicar ou provar uma determinada situação ou afirmação}
  \end{Phonetics}
\end{Entry}

\begin{Entry}{例外}{8,5}{⼈、⼣}
  \begin{Phonetics}{例外}{li4wai4}[][HSK 5]
    \definition[个,种]{s.}{exceção; situações que não se enquadram nas regras gerais ou nas leis comuns}
    \definition{v.}{ser excepcional; ser uma exceção}
  \end{Phonetics}
\end{Entry}

\begin{Entry}{例如}{8,6}{⼈、⼥}
  \begin{Phonetics}{例如}{li4ru2}[][HSK 2]
    \definition{conj.}{por exemplo; tal como; como por exemplo; colocado antes do exemplo, indica que o exemplo vem a seguir}
  \end{Phonetics}
\end{Entry}

\begin{Entry}{供}{8}{⼈}
  \begin{Phonetics}{供}{gong1}
    \definition*{s.}{Sobrenome Gong}
    \definition{v.}{fornecer; alimentar |  fornecer algo (para uso ou conveniência de); fornecer algumas condições de exploração à outra parte}
  \end{Phonetics}
  \begin{Phonetics}{供}{gong4}
    \definition{s.}{oferendas | confissão}
    \definition{v.}{depositar (oferendas) | confessar}
  \end{Phonetics}
\end{Entry}

\begin{Entry}{供应}{8,7}{⼈、⼴}
  \begin{Phonetics}{供应}{gong1 ying4}[][HSK 4]
    \definition{v.}{fornecer; prover de}
  \end{Phonetics}
\end{Entry}

\begin{Entry}{供给}{8,9}{⼈、⽷}
  \begin{Phonetics}{供给}{gong1ji3}[][HSK 6]
    \definition{s.}{fornecer; prover; fornecer produção e necessidades de vida, dinheiro, etc. para aqueles que precisam}
  \end{Phonetics}
\end{Entry}

\begin{Entry}{依}{8}{⼈}
  \begin{Phonetics}{依}{yi1}
    \definition*{s.}{Sobrenome Yi}
    \definition{prep.}{de acordo com; à luz de; julgando por}
    \definition{v.}{depender de; ser dependente de; confiar em | cumprir; ouvir; ceder a | inclinar-se; descansar sobre (ou contra)}
  \end{Phonetics}
\end{Entry}

\begin{Entry}{依旧}{8,5}{⼈、⽇}
  \begin{Phonetics}{依旧}{yi1jiu4}[][HSK 5]
    \definition{adv.}{ainda; como antes; como sempre}
  \end{Phonetics}
\end{Entry}

\begin{Entry}{依次}{8,6}{⼈、⽋}
  \begin{Phonetics}{依次}{yi1 ci4}[][HSK 6]
    \definition{adv.}{sucessivamente; na ordem correta; em ordem}
  \end{Phonetics}
\end{Entry}

\begin{Entry}{依法}{8,8}{⼈、⽔}
  \begin{Phonetics}{依法}{yi1 fa3}[][HSK 5]
    \definition{adv.}{e acordo com regras (ou métodos) fixas | de acordo com a lei; por força da lei; em conformidade com as disposições legais}
  \end{Phonetics}
\end{Entry}

\begin{Entry}{依偎}{8,11}{⼈、⼈}
  \begin{Phonetics}{依偎}{yi1wei1}
    \definition{v.}{aninhar-se | aconchegar-se}
  \end{Phonetics}
\end{Entry}

\begin{Entry}{依据}{8,11}{⼈、⼿}
  \begin{Phonetics}{依据}{yi1ju4}[][HSK 5]
    \definition{prep.}{julgando por; de acordo com; à luz de; com base em; de acordo com; introduzir algo que possa servir como premissa ou base}
    \definition[个]{s.}{base; evidência; fundamento; base para tomar uma decisão ou realizar uma ação}
    \definition{v.}{basear-se em; confiar em; depdender de; usar algo como premissa ou base}
  \end{Phonetics}
\end{Entry}

\begin{Entry}{依然}{8,12}{⼈、⽕}
  \begin{Phonetics}{依然}{yi1ran2}[][HSK 4]
    \definition{adv.}{ainda; como antes}
    \definition{v.}{estar quieto; estar como antes; estar como o original, sem alterações}
  \end{Phonetics}
\end{Entry}

\begin{Entry}{依照}{8,13}{⼈、⽕}
  \begin{Phonetics}{依照}{yi1 zhao4}[][HSK 5]
    \definition{prep.}{de acordo com; à luz de; introduzir certos padrões para os eventos, o que equivale a 按照}
    \definition{v.}{seguir (com base em algo)}
  \seealsoref{按照}{an4zhao4}
  \end{Phonetics}
\end{Entry}

\begin{Entry}{依赖}{8,13}{⼈、⾙}
  \begin{Phonetics}{依赖}{yi1lai4}[][HSK 6]
    \definition{v.}{confiar em; ser dependente de; ser completamente dependente e inseparável | depender de; ser mutuamente dependentes e inseparáveis}
  \end{Phonetics}
\end{Entry}

\begin{Entry}{依靠}{8,15}{⼈、⾮}
  \begin{Phonetics}{依靠}{yi1kao4}[][HSK 4]
    \definition{s.}{apoio; suporte; algo em que se apoiar; alguém ou algo em quem você pode confiar}
    \definition{v.}{depender de; confiar em (alguém ou alguma coisa para atingir um determinado objetivo)}
  \end{Phonetics}
\end{Entry}

\begin{Entry}{侧}{8}{⼈}
  \begin{Phonetics}{侧}{ce4}[][HSK 6]
    \definition*{s.}{Sobrenome Ce}
    \definition{s.}{lado | inclinação}
    \definition{v.}{inclinar; inclinar-se}
  \end{Phonetics}
  \begin{Phonetics}{侧}{zhai1}
    \definition{adj.}{inclinado; torto}
  \end{Phonetics}
\end{Entry}

\begin{Entry}{侵}{9}{⼈}
  \begin{Phonetics}{侵}{qin1}
    \definition*{s.}{Sobrenome Qin}
    \definition{prep.}{aproximando-se; aproximar}
    \definition{v.}{invadir; intrometer-se em; infringir | aproximar-se (amanhecer)}
  \end{Phonetics}
\end{Entry}

\begin{Entry}{侵犯}{9,5}{⼈、⽝}
  \begin{Phonetics}{侵犯}{qin1fan4}[][HSK 6]
    \definition{v.}{violar; invadir; infringir; interferência ilegal com terceiros e violação de seus direitos | violar; fazer incursões; invadir o território de outro país}
  \end{Phonetics}
\end{Entry}

\begin{Entry}{侵略}{9,11}{⼈、⽥}
  \begin{Phonetics}{侵略}{qin1lve4}
    \definition{s.}{invasão}
    \definition{v.}{invadir}
  \end{Phonetics}
\end{Entry}

\begin{Entry}{便}{9}{⼈}
  \begin{Phonetics}{便}{bian4}[][HSK 6]
    \definition{adj.}{prático; conveniente | simples; comum; informal}
    \definition{adv.}{então; apenas no caso de; mesmo significado e uso de 就}
    \definition{conj.}{mesmo que; expressa uma concessão hipotética}
    \definition{s.}{facilidade; conveniência; o momento certo; a oportunidade | fezes ou urina}
    \definition{v.}{aliviar-se; excretar fezes e urina}
  \seealsoref{就}{jiu4}
  \end{Phonetics}
  \begin{Phonetics}{便}{pian2}
    \definition*{s.}{Sobrenome Pian}
    \definition{adj.}{silencioso e confortável}
  \end{Phonetics}
\end{Entry}

\begin{Entry}{便于}{9,3}{⼈、⼆}
  \begin{Phonetics}{便于}{bian4yu2}[][HSK 5]
    \definition{v.}{ser fácil para; ser conveniente para (algo ou fazer algo)}
  \end{Phonetics}
\end{Entry}

\begin{Entry}{便利}{9,7}{⼈、⼑}
  \begin{Phonetics}{便利}{bian4li4}[][HSK 5]
    \definition{adj.}{fácil; conveniente}
    \definition{s.}{facilidade; conveniência; coisas ou condições convenientes}
    \definition{v.}{facilitar; fornecer ajuda para que os outros se sintam confortáveis}
  \end{Phonetics}
\end{Entry}

\begin{Entry}{便条}{9,7}{⼈、⽊}
  \begin{Phonetics}{便条}{bian4tiao2}[][HSK 5]
    \definition[张,个]{s.}{nota ou mensagem informal; geralmente uma mensagem ou notificação}
  \end{Phonetics}
\end{Entry}

\begin{Entry}{便宜}{9,8}{⼈、⼧}
  \begin{Phonetics}{便宜}{bian4yi2}
    \definition{adj.}{prático; conveniente; adequado}
  \end{Phonetics}
  \begin{Phonetics}{便宜}{pian2yi5}[][HSK 2]
    \definition{adj.}{barato; acessível}
    \definition[个,份,件]{s.}{vantagem em algum aspecto | ganho; lucro; vantagem; benefício indevido}
    \definition{v.}{deixar alguém escapar impune; obter algum benefício}
  \end{Phonetics}
\end{Entry}

\begin{Entry}{便是}{9,9}{⼈、⽇}
  \begin{Phonetics}{便是}{bian4 shi4}[][HSK 6]
    \definition{adv.}{exatamente; precisamente; para expressar afirmação ou ênfase}
    \definition{conj.}{mesmo; mesmo que; usado para introduzir um caso extremo hipotético, enfatizando que o mesmo resultado ocorreria em circunstâncias tão extremas, sem mencionar circunstâncias normais; você também pode usar 即便是}
    \definition{part.}{usada no final de uma frase para expressar afirmação}
  \seealsoref{即便是}{ji2bian4 shi4}
  \end{Phonetics}
\end{Entry}

\begin{Entry}{促}{9}{⼈}
  \begin{Phonetics}{促}{cu4}
    \definition{adj.}{curto; apressado; urgente}
    \definition{v.}{urgir; promover | estar perto de; estar perto}
  \end{Phonetics}
\end{Entry}

\begin{Entry}{促进}{9,7}{⼈、⾡}
  \begin{Phonetics}{促进}{cu4jin4}[][HSK 4]
    \definition{v.}{impulsionar; promover; avançar; incentivar o desenvolvimento}
  \end{Phonetics}
\end{Entry}

\begin{Entry}{促使}{9,8}{⼈、⼈}
  \begin{Phonetics}{促使}{cu4shi3}[][HSK 4]
    \definition{v.}{incitar; estimular; impelir; causar; provocar uma mudança em alguém ou em algo}
  \end{Phonetics}
\end{Entry}

\begin{Entry}{促销}{9,12}{⼈、⾦}
  \begin{Phonetics}{促销}{cu4 xiao1}[][HSK 4]
    \definition{v.}{promover vendas}
  \end{Phonetics}
\end{Entry}

\begin{Entry}{俄}{9}{⼈}
  \begin{Phonetics}{俄}{e2}
    \definition*{s.}{Rússia, abreviação de 俄罗斯}
    \definition{adv.}{muito em breve; em breve; de repente}
  \seealsoref{俄罗斯}{e2luo2si1}
  \end{Phonetics}
\end{Entry}

\begin{Entry}{俄罗斯}{9,8,12}{⼈、⽹、⽄}
  \begin{Phonetics}{俄罗斯}{e2luo2si1}
    \definition*{s.}{Rússia}
  \end{Phonetics}
\end{Entry}

\begin{Entry}{俄罗斯人}{9,8,12,2}{⼈、⽹、⽄、⼈}
  \begin{Phonetics}{俄罗斯人}{e2luo2si1ren2}
    \definition{s.}{russo | pessoa ou povo da Rússia}
  \end{Phonetics}
\end{Entry}

\begin{Entry}{保}{9}{⼈}
  \begin{Phonetics}{保}{bao3}[][HSK 3]
    \definition*{s.}{Sobrenome Bao}
    \definition{s.}{fiador; babá ou responsável pela guarda de crianças | oficial responsável; sistema administrativo; unidade administrativa do antigo registro civil}
    \definition{v.}{defender; proteger | manter; preservar; conservar em boas condições | assegurar; garantir | ser fiador de alguém}
  \end{Phonetics}
\end{Entry}

\begin{Entry}{保卫}{9,3}{⼈、⼙}
  \begin{Phonetics}{保卫}{bao3wei4}[][HSK 5]
    \definition{v.}{defender; proteger; salvaguardar; proteger-se de ser violado}
  \end{Phonetics}
\end{Entry}

\begin{Entry}{保存}{9,6}{⼈、⼦}
  \begin{Phonetics}{保存}{bao3cun2}[][HSK 3]
    \definition{v.}{salvar; preservar; conservar; manter a existência com ênfase em que as coisas, as propriedades, os significados, os estilos, etc. não sofram perdas ou mudanças | (computação) salvar (um arquivo, etc.)}
  \end{Phonetics}
\end{Entry}

\begin{Entry}{保守}{9,6}{⼈、⼧}
  \begin{Phonetics}{保守}{bao3shou3}[][HSK 4]
    \definition{adj.}{retrógrado; conservador; pensamentos e conceitos que são retrógrados e não conseguem acompanhar o desenvolvimento da situação}
    \definition{v.}{manter; guardar; evitar perder}
  \end{Phonetics}
\end{Entry}

\begin{Entry}{保安}{9,6}{⼈、⼧}
  \begin{Phonetics}{保安}{bao3 an1}[][HSK 3]
    \definition[个,位,名]{s.}{guarda de segurança; segurança}
    \definition{v.}{proteger; manter em segurança; defender a segurança social | garantir a segurança; proteger a segurança dos trabalhadores e prevenir acidentes durante o processo de produção}
  \end{Phonetics}
\end{Entry}

\begin{Entry}{保护}{9,7}{⼈、⼿}
  \begin{Phonetics}{保护}{bao3hu4}[][HSK 3]
    \definition{v.}{proteger, guardar, cuidar; salvaguardar; cuidar ao máximo, para que não seja danificado, referindo-se principalmente a coisas concretas}
  \end{Phonetics}
\end{Entry}

\begin{Entry}{保护区}{9,7,4}{⼈、⼿、⼖}
  \begin{Phonetics}{保护区}{bao3hu4qu1}
    \definition[个,片]{s.}{zona de proteção | área de preservação; reserva natural}
  \end{Phonetics}
\end{Entry}

\begin{Entry}{保护主义}{9,7,5,3}{⼈、⼿、⼂、⼂}
  \begin{Phonetics}{保护主义}{bao3hu4zhu3yi4}
    \definition{s.}{protecionismo}
  \end{Phonetics}
\end{Entry}

\begin{Entry}{保护色}{9,7,6}{⼈、⼿、⾊}
  \begin{Phonetics}{保护色}{bao3hu4se4}
    \definition{s.}{camuflagem | coloração protetora}
  \end{Phonetics}
\end{Entry}

\begin{Entry}{保护剂}{9,7,8}{⼈、⼿、⼑}
  \begin{Phonetics}{保护剂}{bao3hu4ji4}
    \definition{s.}{agente protetor; protetor}
  \end{Phonetics}
\end{Entry}

\begin{Entry}{保护国}{9,7,8}{⼈、⼿、⼞}
  \begin{Phonetics}{保护国}{bao3hu4guo2}
    \definition{s.}{protetorado}
  \end{Phonetics}
\end{Entry}

\begin{Entry}{保护性}{9,7,8}{⼈、⼿、⼼}
  \begin{Phonetics}{保护性}{bao3hu4xing4}
    \definition{s.}{proteção; protetor}
  \end{Phonetics}
\end{Entry}

\begin{Entry}{保护物}{9,7,8}{⼈、⼿、⽜}
  \begin{Phonetics}{保护物}{bao3hu4 wu4}
    \definition{s.}{protetor}
  \end{Phonetics}
\end{Entry}

\begin{Entry}{保护者}{9,7,8}{⼈、⼿、⽼}
  \begin{Phonetics}{保护者}{bao3hu4zhe3}
    \definition{s.}{protetor | segurador}
  \end{Phonetics}
\end{Entry}

\begin{Entry}{保护神}{9,7,9}{⼈、⼿、⽰}
  \begin{Phonetics}{保护神}{bao3hu4shen2}
    \definition{s.}{anjo da guarda | santo patrono}
  \end{Phonetics}
\end{Entry}

\begin{Entry}{保证}{9,7}{⼈、⾔}
  \begin{Phonetics}{保证}{bao3zheng4}[][HSK 3]
    \definition[种,份]{s.}{compromisso; garantia; caução; aval; condições ou coisas que garantem a realização de algo}
    \definition{v.}{prometer; garantir; assegurar; certamente concluir algo; garantir que determinados padrões e requisitos sejam alcançados}
  \end{Phonetics}
\end{Entry}

\begin{Entry}{保养}{9,9}{⼈、⼋}
  \begin{Phonetics}{保养}{bao3yang3}[][HSK 5]
    \definition{v.}{preservar; cuidar bem (ou conservar) da saúde |  fazer manutenção; conservar; manter; manter em bom estado de conservação}
  \end{Phonetics}
\end{Entry}

\begin{Entry}{保持}{9,9}{⼈、⼿}
  \begin{Phonetics}{保持}{bao3chi2}[][HSK 3]
    \definition{v.}{manter; conservar; reter; preservar; manter um determinado estado, para que não desapareça ou não se altere}
  \end{Phonetics}
\end{Entry}

\begin{Entry}{保险}{9,9}{⼈、⾩}
  \begin{Phonetics}{保险}{bao3xian3}[][HSK 3]
    \definition{adj.}{seguro; pode ficar tranquilo}
    \definition[个,份,种]{s.}{seguro; um tipo de seguro comercial que garante que o segurado receba uma indenização em caso de prejuízo}
    \definition{v.}{ter certeza; estar obrigado a; garantir que algo aconteça (o que as pessoas desejam)}
  \end{Phonetics}
\end{Entry}

\begin{Entry}{保健}{9,10}{⼈、⼈}
  \begin{Phonetics}{保健}{bao3 jian4}[][HSK 6]
    \definition{s.}{cuidados de saúde; proteção da saúde}
    \definition{v.}{cuidar da sua saúde; proteger sua saúde}
  \end{Phonetics}
\end{Entry}

\begin{Entry}{保留}{9,10}{⼈、⽥}
  \begin{Phonetics}{保留}{bao3liu2}[][HSK 3]
    \definition{v.}{manter; continuar a ter; manter o estado original inalterado | conter; reter; deixar ficar; não tirar | reservar; colocar os direitos, opiniões, etc. de lado, não exercê-los ou expressá-los por enquanto}
  \end{Phonetics}
\end{Entry}

\begin{Entry}{保密}{9,11}{⼈、⼧}
  \begin{Phonetics}{保密}{bao3mi4}[][HSK 4]
    \definition{v.}{manter segredo; manter algo em segredo; manter a confidencialidade}
  \end{Phonetics}
\end{Entry}

\begin{Entry}{信}{9}{⼈}
  \begin{Phonetics}{信}{xin4}[][HSK 2,3]
    \definition*{s.}{Sobrenome Xin}
    \definition{adj.}{verdade}
    \definition[封,个,张]{s.}{carta; correio | mensagem; notícia; informação | sinal; evidência | confiança; fé; crédito | detonador (de bombas, etc.) | arsênico}
    \definition{v.}{acreditar; fazer um balanço; dar crédito | deixar à vontade; deixar à mercê; deixar ao acaso | professar fé em; acreditar em}
  \end{Phonetics}
\end{Entry}

\begin{Entry}{信心}{9,4}{⼈、⼼}
  \begin{Phonetics}{信心}{xin4xin1}[][HSK 2]
    \definition[个]{s.}{confiança; fé (em alguém ou algo) ; a crença de que os desejos se tornarão realidade}
  \end{Phonetics}
\end{Entry}

\begin{Entry}{信号}{9,5}{⼈、⼝}
  \begin{Phonetics}{信号}{xin4hao4}[][HSK 2]
    \definition[个,道]{s.}{sinal; luz, ondas de rádio, som, movimento, etc. usados para transmitir mensagens ou comandos | ponte de sinalização; marcação para chamar a atenção, ajudar na identificação e na memória}
  \end{Phonetics}
\end{Entry}

\begin{Entry}{信用}{9,5}{⼈、⽤}
  \begin{Phonetics}{信用}{xin4 yong4}[][HSK 6]
    \definition{adj.}{crédito; não é necessária nenhuma garantia material e o dinheiro pode ser reembolsado no prazo}
    \definition[些]{s.}{crédito; confiabilidade; a confiança que você ganha ao fazer o que prometeu | crédito; uma relação de empréstimo-pagamento ou situação em que o empréstimo é condicionado ao pagamento; uma situação em que um banco empresta dinheiro temporariamente a um cliente e este posteriormente devolve o dinheiro ao banco}
  \end{Phonetics}
\end{Entry}

\begin{Entry}{信用卡}{9,5,5}{⼈、⽤、⼘}
  \begin{Phonetics}{信用卡}{xin4yong4ka3}[][HSK 2]
    \definition[张]{s.}{cartão de crédito; moeda eletrônica emitida por um banco ou outra instituição especializada para consumidores; os titulares do cartão podem usá-lo para sacar dinheiro ou fazer compras de acordo com os regulamentos}
  \end{Phonetics}
\end{Entry}

\begin{Entry}{信仰}{9,6}{⼈、⼈}
  \begin{Phonetics}{信仰}{xin4yang3}[][HSK 6]
    \definition[种]{s.}{crença; religião; refere-se à ideia de acreditar, adorar e tomar algo como padrão e guia para palavras e ações}
    \definition{v.}{acreditar; crer em; acreditar e adorar uma determinada religião ou doutrina e tomá-la como guia para palavras e ações}
  \end{Phonetics}
\end{Entry}

\begin{Entry}{信任}{9,6}{⼈、⼈}
  \begin{Phonetics}{信任}{xin4ren4}[][HSK 3]
    \definition{s.}{confiança; um estado mental positivo e conexão emocional}
    \definition{v.}{confiar; ter confiança em; acreditar e ousar confiar}
  \end{Phonetics}
\end{Entry}

\begin{Entry}{信访}{9,6}{⼈、⾔}
  \begin{Phonetics}{信访}{xin4fang3}
    \definition{s.}{carta de reclamação | carta de petição}
  \seealsoref{上访}{shang4fang3}
  \end{Phonetics}
\end{Entry}

\begin{Entry}{信念}{9,8}{⼈、⼼}
  \begin{Phonetics}{信念}{xin4nian4}[][HSK 5]
    \definition[个,种]{s.}{fé; crença; convicção; concepções consideradas corretas e acreditadas com convicção}
  \end{Phonetics}
\end{Entry}

\begin{Entry}{信经}{9,8}{⼈、⽷}
  \begin{Phonetics}{信经}{xin4jing1}
    \definition[个]{s.}{crença | credo (seção da missa católica)}
  \end{Phonetics}
\end{Entry}

\begin{Entry}{信封}{9,9}{⼈、⼨}
  \begin{Phonetics}{信封}{xin4feng1}[][HSK 3]
    \definition[个,封]{s.}{envelope para cartas}
  \end{Phonetics}
\end{Entry}

\begin{Entry}{信息}{9,10}{⼈、⼼}
  \begin{Phonetics}{信息}{xin4xi1}[][HSK 2]
    \definition[个,条,段,些]{s.}{notícias; informações; as últimas notícias sobre alguém ou alguma coisa | mensagem; informação; na teoria da informação, uma mensagem transmitida usando símbolos, cujo conteúdo é desconhecido pelo receptor}
  \end{Phonetics}
\end{Entry}

\begin{Entry}{信箱}{9,15}{⼈、⾋}
  \begin{Phonetics}{信箱}{xin4 xiang1}[][HSK 5]
    \definition{s.}{caixa de correio; caixa postal instalada pelos correios para que as pessoas possam depositar cartas | caixa postal; caixas com números, localizadas nos correios, que podem ser alugadas para receber correspondência; chamadas de caixas postais exclusivas}
  \end{Phonetics}
\end{Entry}

\begin{Entry}{俩}{9}{⼈}
  \begin{Phonetics}{俩}{lia3}[][HSK 4]
    \definition{num.}{ambos; dois; contração de 两个 | alguns; vários; refere-se a um pequeno número}
  \end{Phonetics}
\end{Entry}

\begin{Entry}{俩钱}{9,10}{⼈、⾦}
  \begin{Phonetics}{俩钱}{lia3qian2}
    \definition{s.}{uma pequena quantia de dinheiro}
  \end{Phonetics}
\end{Entry}

\begin{Entry}{俭}{9}{⼈}
  \begin{Phonetics}{俭}{jian3}
    \definition*{s.}{Sobrenome Jian}
    \definition{adj.}{econômico; frugal | querendo; faltando; curto}
  \end{Phonetics}
\end{Entry}

\begin{Entry}{俭省}{9,9}{⼈、⽬}
  \begin{Phonetics}{俭省}{jian3sheng3}
    \definition{adj.}{econômico}
  \end{Phonetics}
\end{Entry}

\begin{Entry}{修}{9}{⼈}
  \begin{Phonetics}{修}{xiu1}[][HSK 3]
    \definition*{s.}{Sobrenome Xiu}
    \definition{adj.}{comprido; alto e esbelto}
    \definition{s.}{revisionismo}
    \definition{v.}{embelezar; decorar | consertar; reparar; reformar | escrever; redigir; compilar | estudar; cultivar; aprender e praticar para aperfeiçoar ou melhorar (o caráter e o conhecimento) | construir; edificar | cortar ou aparar, para deixar bonito e arrumado | dedicar-se à prática da religião}
  \end{Phonetics}
\end{Entry}

\begin{Entry}{修车}{9,4}{⼈、⾞}
  \begin{Phonetics}{修车}{xiu1 che1}[][HSK 6]
    \definition{v.}{consertar uma bicicleta (carro etc.)}[我打算明天去修车。===Pretendo consertar meu carro amanhã.]
  \end{Phonetics}
\end{Entry}

\begin{Entry}{修改}{9,7}{⼈、⽁}
  \begin{Phonetics}{修改}{xiu1gai3}[][HSK 3]
    \definition{v.}{revisar; retocar; corrigir erros e falhas em artigos, planos, etc.}
  \end{Phonetics}
\end{Entry}

\begin{Entry}{修建}{9,8}{⼈、⼵}
  \begin{Phonetics}{修建}{xiu1jian4}[][HSK 5]
    \definition{v.}{construir; erguer; animar; edificar; construir com tijolos, telhas, madeira, cimento, areia, etc.}
  \end{Phonetics}
\end{Entry}

\begin{Entry}{修规}{9,8}{⼈、⾒}
  \begin{Phonetics}{修规}{xiu1gui1}
    \definition{s.}{plano de construção}
  \end{Phonetics}
\end{Entry}

\begin{Entry}{修养}{9,9}{⼈、⼋}
  \begin{Phonetics}{修养}{xiu1yang3}[][HSK 5]
    \definition[种]{s.}{treinamento; domínio; realização; refere-se a um determinado nível em termos de teoria, conhecimento, arte, pensamento, etc. | auto-cultivo; refere-se à atitude e ao comportamento cultivados ao longo do tempo, em conformidade com as exigências sociais}
  \end{Phonetics}
\end{Entry}

\begin{Entry}{修复}{9,9}{⼈、⼢}
  \begin{Phonetics}{修复}{xiu1fu4}[][HSK 5]
    \definition{v.}{reparar; restaurar; renovar | reparar; melhorar e restaurar (o relacionamento)}
  \end{Phonetics}
\end{Entry}

\begin{Entry}{修理}{9,11}{⼈、⽟}
  \begin{Phonetics}{修理}{xiu1li3}[][HSK 4]
    \definition{v.}{consertar; reparar; restaurar algo danificado à sua forma ou função original | aparar; podar; cortar com tesouras e outras ferramentas para deixar árvores, flores, cabelos, etc. arrumados | culpar; punir; criticar ou punir uma pessoa para mostrar que ela está errada}
  \end{Phonetics}
\end{Entry}

\begin{Entry}{俯}{10}{⼈}
  \begin{Phonetics}{俯}{fu3}
    \definition{v.}{curvar (a cabeça) (oposto a 仰) | inclinar-se | (datado, em documentos ou cartas oficiais) condescender com | curvar-se; fazer uma reverência}
  \seealsoref{仰}{yang3}
  \end{Phonetics}
\end{Entry}

\begin{Entry}{俱}{10}{⼈}
  \begin{Phonetics}{俱}{ju4}
    \definition{adv.}{(literário) tudo; completamente; inteiramente}
  \end{Phonetics}
\end{Entry}

\begin{Entry}{俱乐部}{10,5,10}{⼈、⼃、⾢}
  \begin{Phonetics}{俱乐部}{ju4le4bu4}[][HSK 5]
    \definition[个,家,间]{s.}{clube; grupos e locais para atividades sociais, políticas, literárias, recreativas e outras}
  \end{Phonetics}
\end{Entry}

\begin{Entry}{倂}{10}{⼈}
  \begin{Phonetics}{倂}{bing4}
    \variantof{并}
  \end{Phonetics}
\end{Entry}

\begin{Entry}{倍}{10}{⼈}
  \begin{Phonetics}{倍}{bei4}[][HSK 4]
    \definition{adv.}{ainda mais; especialmente | (antes de certos adjetivos) muito; particularmente; é pronunciado como um som erhua e é usado antes de certos adjetivos para expressar um alto grau de profundidade, equivalente a 非常 ou 特别}
    \definition{clas.}{vezes; usado após um numeral, significa que o valor anterior é multiplicado por este número}[增长了五倍。===Aumentou cinco vezes. | 二的三倍是六。===Três vezes dois é seis.]
  \seealsoref{非常}{fei1chang2}
  \seealsoref{特别}{te4bie2}
  \end{Phonetics}
\end{Entry}

\begin{Entry}{倒}{10}{⼈}
  \begin{Phonetics}{倒}{dao3}[][HSK 2]
    \definition{v.}{cair; tombar | falhar; entrar em colapso | ficar rouco | mudar; trocar; transferir; converter | movimentar-se; manobrar | oferecer (casa, loja) para venda; vender mercadorias ou lojas a terceiros a um preço fixo | derrubar; derrubar com}
  \end{Phonetics}
  \begin{Phonetics}{倒}{dao4}[][HSK 2]
    \definition{adj.}{inverso; invertido; de cabeça para baixo}
    \definition{adv.}{mas; pelo contrário; expressa o contrário do esperado, equivalente a 反倒 | indicando que algo não é o que se pensa; indica que as coisas não são assim | usado para indicar uma transição ou concessão | transmitindo a sensação de ``urgência''; expressa pressa ou insistência, com um tom impaciente}
    \definition{v.}{ser inverso; estar invertido; estar de cabeça para baixo; inverter a posição original para cima e para baixo ou para a frente e para trás | recuar; virar de cabeça para baixo; fazer mover na direção oposta ou inverter | inclinar ou virar o recipiente para retirar o conteúdo; inclinar; derramar}
  \seealsoref{反倒}{fan3dao4}
  \end{Phonetics}
\end{Entry}

\begin{Entry}{倒车}{10,4}{⼈、⾞}
  \begin{Phonetics}{倒车}{dao3che1}[][HSK 4]
    \definition{v.}{trocar de trem ou ônibus (no meio do caminho)}
  \end{Phonetics}
  \begin{Phonetics}{倒车}{dao4che1}[][HSK 4]
    \definition{v.}{dar marcha à ré (em um veículo)}
  \end{Phonetics}
\end{Entry}

\begin{Entry}{倒地}{10,6}{⼈、⼟}
  \begin{Phonetics}{倒地}{dao3di4}
    \definition{v.}{cair no chão}
  \end{Phonetics}
\end{Entry}

\begin{Entry}{倒血霉}{10,6,15}{⼈、⾎、⾬}
  \begin{Phonetics}{倒血霉}{dao3xue4mei2}
    \definition{v.}{ter muito azar (versão mais forte de 倒霉)}
  \seealsoref{倒霉}{dao3mei2}
  \end{Phonetics}
\end{Entry}

\begin{Entry}{倒闭}{10,6}{⼈、⾨}
  \begin{Phonetics}{倒闭}{dao3bi4}[][HSK 4]
    \definition{v.}{fechar; ir à falência; entrar em liquidação; sair do negócio; (empresa, loja ou banco) deixar de operar devido ao baixo desempenho}
  \end{Phonetics}
\end{Entry}

\begin{Entry}{倒是}{10,9}{⼈、⽇}
  \begin{Phonetics}{倒是}{dao4 shi4}[][HSK 5]
    \definition{adv.}{usado para indicar o oposto do que geralmente é verdade; ao contrário do senso comum; pelo contrário | usado para indicar o que é contrário aos fatos, com um toque de crítica; indica que as coisas não são assim (com um sentimento de culpa) | usado de algo inesperado; expressando surpresa | usado para indicar concessão | usado para indicar uma mudança de significado; indica um ponto de virada | usado para modificar ou suavizar uma declaração anterior; para suavizar o tom | usado para pressionar ou questionar alguém; para instar ou perguntar}
  \end{Phonetics}
\end{Entry}

\begin{Entry}{倒楣}{10,13}{⼈、⽊}
  \begin{Phonetics}{倒楣}{dao3mei2}
    \variantof{倒霉}
  \end{Phonetics}
\end{Entry}

\begin{Entry}{倒霉}{10,15}{⼈、⾬}
  \begin{Phonetics}{倒霉}{dao3mei2}
    \definition{adj.}{azarado}
    \definition{s.}{azar | má sorte}
    \definition{v.}{estar sem sorte | ter azar}
  \seealsoref{倒血霉}{dao3xue4mei2}
  \end{Phonetics}
\end{Entry}

\begin{Entry}{倘}{10}{⼈}
  \begin{Phonetics}{倘}{chang2}
  \end{Phonetics}
  \begin{Phonetics}{倘}{tang3}
    \definition{conj.}{se; supondo; no caso}
  \end{Phonetics}
\end{Entry}

\begin{Entry}{倘使}{10,8}{⼈、⼈}
  \begin{Phonetics}{倘使}{tang3shi3}
    \definition{conj.}{se | supondo que | no caso}
  \end{Phonetics}
\end{Entry}

\begin{Entry}{倘或}{10,8}{⼈、⼽}
  \begin{Phonetics}{倘或}{tang3huo4}
    \definition{conj.}{se | supondo que | no caso}
  \end{Phonetics}
\end{Entry}

\begin{Entry}{倘若}{10,8}{⼈、⾋}
  \begin{Phonetics}{倘若}{tang3ruo4}
    \definition{conj.}{se | supondo que | no caso}
  \end{Phonetics}
\end{Entry}

\begin{Entry}{借}{10}{⼈}
  \begin{Phonetics}{借}{jie4}[][HSK 2]
    \definition{adv.}{por meio de}
    \definition{v.}{emprestar | pedir emprestado | usar como pretexto | aproveitar; tirar proveito (de uma oportunidade, etc.)}
  \end{Phonetics}
\end{Entry}

\begin{Entry}{借书证}{10,4,7}{⼈、⼄、⾔}
  \begin{Phonetics}{借书证}{jie4shu1zheng4}
    \definition{s.}{cartão de biblioteca | (literalmente) cartão para pedir emprestado livros}
  \end{Phonetics}
\end{Entry}

\begin{Entry}{借鉴}{10,13}{⼈、⾦}
  \begin{Phonetics}{借鉴}{jie4jian4}[][HSK 6]
    \definition{s.}{tirar lições de; aproveitar a experiência de; ganhar experiência e lições com o passado ou com as experiências de outras pessoas}
  \end{Phonetics}
\end{Entry}

\begin{Entry}{倡}{10}{⼈}
  \begin{Phonetics}{倡}{chang4}
    \definition{v.}{iniciar; propor; defender | promover; assumir a liderança}
  \end{Phonetics}
\end{Entry}

\begin{Entry}{倡导}{10,6}{⼈、⼨}
  \begin{Phonetics}{倡导}{chang4dao3}[][HSK 5]
    \definition{v.}{iniciar; propor; promover; defender; advogar}
  \end{Phonetics}
\end{Entry}

\begin{Entry}{债}{10}{⼈}
  \begin{Phonetics}{债}{zhai4}[][HSK 6]
    \definition[笔]{s.}{dívida | empréstimo}
  \end{Phonetics}
\end{Entry}

\begin{Entry}{值}{10}{⼈}
  \begin{Phonetics}{值}{zhi2}[][HSK 3]
    \definition{adj.}{significativo; valioso; digno de nota}
    \definition{prep.}{quando; introduz o momento em que algo acontece ou existe, equivalente a 当 ou 在}
    \definition{s.}{preço; valor | valor de um número, de uma variável}
    \definition{v.}{valer; custar; a mercadoria é adequada ao preço | ir de encontro; encontrar; cruzar | estar de serviço; ter sua vez em algo; assumir o cargo que lhe cabe | é a vez de (executar uma determinada função pública)}
  \seealsoref{当}{dang1}
  \seealsoref{在}{zai4}
  \end{Phonetics}
\end{Entry}

\begin{Entry}{值班}{10,10}{⼈、⽟}
  \begin{Phonetics}{值班}{zhi2ban1}[][HSK 5]
    \definition{v.}{estar em serviço ou plantão; trabalhar em um turno; (em rodízio) desempenhar funções durante um período de tempo determinado}
  \end{Phonetics}
\end{Entry}

\begin{Entry}{值得}{10,11}{⼈、⼻}
  \begin{Phonetics}{值得}{zhi2de5}[][HSK 3]
    \definition{adj.}{que tem valor; (fazer algo) é vantajoso, sem prejuízos}
    \definition{v.}{merecer; ter valor; significa que fazer isso terá bons resultados; que é valioso e significativo}
  \end{Phonetics}
\end{Entry}

\begin{Entry}{倾}{10}{⼈}
  \begin{Phonetics}{倾}{qing1}
    \definition{s.}{desvio; tendência}
    \definition{v.}{inclinar; inclinar-se; dobrar-se | colapsar | virar e despejar; esvaziar | fazer tudo o que puder; usar todos os recursos | sobrecarregar; dominar; dominar | admirar | superar}
  \end{Phonetics}
\end{Entry}

\begin{Entry}{倾向}{10,6}{⼈、⼝}
  \begin{Phonetics}{倾向}{qing1xiang4}[][HSK 6]
    \definition{s.}{tendência; desvio; inclinação; direção do desenvolvimento}
    \definition{v.}{preferir; estar inclinado a; concordar com uma determinada opinião}
  \end{Phonetics}
\end{Entry}

\begin{Entry}{倾城}{10,9}{⼈、⼟}
  \begin{Phonetics}{倾城}{qing1cheng2}
    \definition{adj.}{sedutora (mulher)}
    \definition{adv.}{de todo o lugar | vindo de todos os lugares}
    \definition{v.}{arruinar e derrubar o estado}
  \end{Phonetics}
\end{Entry}

\begin{Entry}{健}{10}{⼈}
  \begin{Phonetics}{健}{jian4}
    \definition{adj.}{forte; saudável; bem definido | ser forte em; ser bom em; apresentar um grau superior à média em determinado aspecto}
    \definition{v.}{fortalecer; endurecer; revigorar}
  \end{Phonetics}
\end{Entry}

\begin{Entry}{健全}{10,6}{⼈、⼊}
  \begin{Phonetics}{健全}{jian4quan2}[][HSK 5]
    \definition{adj.}{saudável; íntegro; capaz; apto; robusto e sem mácula | sólido; completo; perfeito}
    \definition{v.}{aperfeiçoar; melhorar; fortalecer; reforçar}
  \end{Phonetics}
\end{Entry}

\begin{Entry}{健身}{10,7}{⼈、⾝}
  \begin{Phonetics}{健身}{jian4shen1}[][HSK 4]
    \definition{s.}{exercício físico | \emph{fitness}}
    \definition{v.+compl.}{exercitar-se; manter a forma; praticar um esporte, especialmente a ginástica, inclusive em aparelhos, para desenvolver força, flexibilidade, aumentar a resistência, melhorar a coordenação e o controle de todas as partes do corpo}
  \end{Phonetics}
\end{Entry}

\begin{Entry}{健康}{10,11}{⼈、⼴}
  \begin{Phonetics}{健康}{jian4kang1}[][HSK 2]
    \definition{adj.}{em forma; saudável; descreve que a pessoa está em ótimo estado físico ou mental, sem nenhum problema | sudável; tudo está normal, sem problemas | saudável; livre de doenças; bom para a saúde}
    \definition{s.}{saúde; físico; estado de saúde}
  \end{Phonetics}
\end{Entry}

\begin{Entry}{假}{11}{⼈}
  \begin{Phonetics}{假}{jia3}[][HSK 2]
    \definition{adj.}{falso; artificial}
    \definition{conj.}{se; caso; no caso de; conecta frases, expressa relação hipotética, geralmente usada com 如, 若 e 使, equivalente a 如果}
    \definition[个,天]{s.}{falsificação; coisas falsas, irreais ou forjadas}
    \definition{v.}{emprestar | valer-se de; aproveitar; utilizar | supor; presumir; pressupor}
  \seealsoref{如}{ru2}
  \seealsoref{如果}{ru2guo3}
  \seealsoref{若}{ruo4}
  \seealsoref{使}{shi3}
  \end{Phonetics}
  \begin{Phonetics}{假}{jia4}
    \definition[个,天]{s.}{feriado; férias; período de suspensão temporária do trabalho ou dos estudos, legal ou aprovado | licença; afastamento temporário; período de licença temporária do trabalho ou dos estudos, após aprovação}
  \end{Phonetics}
\end{Entry}

\begin{Entry}{假日}{11,4}{⼈、⽇}
  \begin{Phonetics}{假日}{jia4 ri4}[][HSK 6]
    \definition[节]{s.}{feriado; dia de folga}
  \end{Phonetics}
\end{Entry}

\begin{Entry}{假如}{11,6}{⼈、⼥}
  \begin{Phonetics}{假如}{jia3ru2}[][HSK 4]
    \definition{conj.}{se; supondo; no caso}
  \end{Phonetics}
\end{Entry}

\begin{Entry}{假声}{11,7}{⼈、⼠}
  \begin{Phonetics}{假声}{jia3sheng1}
    \definition{s.}{falsete}
  \seealsoref{真声}{zhen1sheng1}
  \end{Phonetics}
\end{Entry}

\begin{Entry}{假证件}{11,7,6}{⼈、⾔、⼈}
  \begin{Phonetics}{假证件}{jia3zheng4jian4}
    \definition{s.}{documentos falsos}
  \end{Phonetics}
\end{Entry}

\begin{Entry}{假使}{11,8}{⼈、⼈}
  \begin{Phonetics}{假使}{jia3shi3}
    \definition{conj.}{se | supondo | em caso}
  \end{Phonetics}
\end{Entry}

\begin{Entry}{假的}{11,8}{⼈、⽩}
  \begin{Phonetics}{假的}{jia3de5}
    \definition{adj.}{falso | substituto | simulado}
  \end{Phonetics}
\end{Entry}

\begin{Entry}{假期}{11,12}{⼈、⽉}
  \begin{Phonetics}{假期}{jia4 qi1}[][HSK 2]
    \definition[个,段,次,种]{s.}{férias; feriados; período de licença}
  \end{Phonetics}
\end{Entry}

\begin{Entry}{偏}{11}{⼈}
  \begin{Phonetics}{偏}{pian1}[][HSK 6]
    \definition{adj.}{parcial; preconceituoso; injusto; focando apenas em um lado | torto; inclinado (oposto de 正) | não dominante; auxiliar | remoto; periférico; longe do centro; incomum}
    \definition{adv.}{intencionalmente; insistentemente; persistentemente; indica ir intencionalmente contra o senso comum ou a solicitação de outra pessoa}
    \definition{expr.}{uma expressão educada para indicar que alguém já tomou chá ou comeu}
    \definition{v.}{divergir; não ser igual a; ser diferente de; exceder ou ficar aquém dos padrões normais | desviar-se; afastar-se; sair na direção certa}
  \seealsoref{正}{zheng4}
  \end{Phonetics}
\end{Entry}

\begin{Entry}{偏偏}{11,11}{⼈、⼈}
  \begin{Phonetics}{偏偏}{pian1pian1}
    \definition{adv.}{voluntariamente | insistentemente | persistentemente | ao contrário da expectativa | infelizmente (indicando que alguma coisa aconteceu ao contrário do que se esperava) | teimosamente (indicando que algo é o oposto ao que seria normal ou razoável) | precisamente (indicando que alguém ou um grupo é escolhido)}
  \end{Phonetics}
\end{Entry}

\begin{Entry}{做}{11}{⼈}
  \begin{Phonetics}{做}{zuo4}[][HSK 1]
    \definition{v.}{fabricar; produzir; criar | escrever; compor | fazer; trabalhar em; dedicar-se a; exercer uma determinada profissão ou atividade | realizar uma festa em família; comemorar | ser; tornar-se; agir como; atuar como | ser usado como | formar ou estabelecer um relacionamento; conectar-se (em algum tipo de relação) | fingir (alguma coisa) | cozinhar; preparar}
  \end{Phonetics}
\end{Entry}

\begin{Entry}{做生活}{11,5,9}{⼈、⽣、⽔}
  \begin{Phonetics}{做生活}{zuo4sheng1huo2}
    \definition{v.}{fazer tabalhos manuais}
  \end{Phonetics}
\end{Entry}

\begin{Entry}{做戏}{11,6}{⼈、⼽}
  \begin{Phonetics}{做戏}{zuo4xi4}
    \definition{v.}{atuar em uma peça | fazer uma peça}
  \end{Phonetics}
\end{Entry}

\begin{Entry}{做作}{11,7}{⼈、⼈}
  \begin{Phonetics}{做作}{zuo4zuo5}
    \definition{adj.}{afetado | artificial}
  \end{Phonetics}
\end{Entry}

\begin{Entry}{做饭}{11,7}{⼈、⾷}
  \begin{Phonetics}{做饭}{zuo4 fan4}[][HSK 2]
    \definition{v.}{cozinhar; preparar uma refeição; cozinhar refeições e transformar alimentos crus em alimentos cozidos}
  \end{Phonetics}
\end{Entry}

\begin{Entry}{做到}{11,8}{⼈、⼑}
  \begin{Phonetics}{做到}{zuo4 dao4}[][HSK 2]
    \definition{v.}{alcançar; realizar; atingir um determinado objetivo; atingir um determinado padrão}
  \end{Phonetics}
\end{Entry}

\begin{Entry}{做法}{11,8}{⼈、⽔}
  \begin{Phonetics}{做法}{zuo4fa3}[][HSK 2]
    \definition[种,个]{s.}{método; maneira de fazer algo; métodos de lidar com coisas ou fazer coisas}
  \end{Phonetics}
\end{Entry}

\begin{Entry}{做客}{11,9}{⼈、⼧}
  \begin{Phonetics}{做客}{zuo4 ke4}[][HSK 3]
    \definition{v.}{visitar; ser um convidado; ser hóspede}
  \end{Phonetics}
\end{Entry}

\begin{Entry}{做活}{11,9}{⼈、⽔}
  \begin{Phonetics}{做活}{zuo4huo2}
    \definition{v.}{trabalhar para ganhar a vida (especialmente de mulher costureira)}
  \end{Phonetics}
\end{Entry}

\begin{Entry}{做梦}{11,11}{⼈、⼣}
  \begin{Phonetics}{做梦}{zuo4 meng4}[][HSK 4]
    \definition{s.}{sonho; ilusões e visões na consciência durante o sono}
    \definition{v.}{sonhar; ter um sonho | sonhar acordado, ter um sonho impossível; metáfora parafantasiasa irrealista}[别​做​梦​了​,她​不​会​嫁​给​你​的​。===Pare de sonhar, ela não se casará com você.]
  \end{Phonetics}
\end{Entry}

\begin{Entry}{做眼}{11,11}{⼈、⽬}
  \begin{Phonetics}{做眼}{zuo4yan3}
    \definition{v.}{agir como um guia | trabalhar como espião}
  \end{Phonetics}
\end{Entry}

\begin{Entry}{停}{11}{⼈}
  \begin{Phonetics}{停}{ting2}[][HSK 2]
    \definition{adj.}{pronto; resolvido; bem organizado}
    \definition{clas.}{usado para partes (de um total); porções}
    \definition{v.}{parar; interromper; cessar; fazer uma pausa | permanecer; ficar; fazer uma parada (para descansar) | estacionar; ancorar; atracar}
  \end{Phonetics}
\end{Entry}

\begin{Entry}{停下}{11,3}{⼈、⼀}
  \begin{Phonetics}{停下}{ting2 xia4}[][HSK 4]
    \definition{v.}{encerrar; desligar; parar}
  \end{Phonetics}
\end{Entry}

\begin{Entry}{停工}{11,3}{⼈、⼯}
  \begin{Phonetics}{停工}{ting2gong1}
    \definition{v.}{parar de trabalhar | parar a produção}
  \end{Phonetics}
\end{Entry}

\begin{Entry}{停办}{11,4}{⼈、⼒}
  \begin{Phonetics}{停办}{ting2ban4}
    \definition{v.}{cancelar | sair do negócio | desligar | terminar}
  \end{Phonetics}
\end{Entry}

\begin{Entry}{停止}{11,4}{⼈、⽌}
  \begin{Phonetics}{停止}{ting2 zhi3}[][HSK 3]
    \definition{v.}{parar; suspender; cessar; cancelar}
  \end{Phonetics}
\end{Entry}

\begin{Entry}{停火}{11,4}{⼈、⽕}
  \begin{Phonetics}{停火}{ting2huo3}
    \definition{s.}{cessar-fogo}
    \definition{v.+compl.}{cessar fogo}
  \end{Phonetics}
\end{Entry}

\begin{Entry}{停车}{11,4}{⼈、⾞}
  \begin{Phonetics}{停车}{ting2 che1}[][HSK 2]
    \definition{v.}{(veículo) parar; frear | estacionar o veículo | parar; deixar de funcionar}
  \end{Phonetics}
\end{Entry}

\begin{Entry}{停车场}{11,4,6}{⼈、⾞、⼟}
  \begin{Phonetics}{停车场}{ting2 che1 chang3}[][HSK 2]
    \definition[个]{s.}{estacionamento; área de estacionamento; local para estacionamento de veículos}
  \end{Phonetics}
\end{Entry}

\begin{Entry}{停业}{11,5}{⼈、⼀}
  \begin{Phonetics}{停业}{ting2ye4}
    \definition{v.}{cessar a negociação (temporária ou permanentemente) | fechar}
  \end{Phonetics}
\end{Entry}

\begin{Entry}{停用}{11,5}{⼈、⽤}
  \begin{Phonetics}{停用}{ting2yong4}
    \definition{v.}{desabilitar | descontinuar | parar de usar | suspender}
  \end{Phonetics}
\end{Entry}

\begin{Entry}{停电}{11,5}{⼈、⽥}
  \begin{Phonetics}{停电}{ting2dian4}
    \definition{s.}{corte de energia}
    \definition{v.}{ter uma falha de energia}
  \end{Phonetics}
\end{Entry}

\begin{Entry}{停当}{11,6}{⼈、⼹}
  \begin{Phonetics}{停当}{ting2dang5}
    \definition{adj.}{realizado | preparado | assentado}
  \end{Phonetics}
\end{Entry}

\begin{Entry}{停息}{11,10}{⼈、⼼}
  \begin{Phonetics}{停息}{ting2xi1}
    \definition{v.}{cessar | parar}
  \end{Phonetics}
\end{Entry}

\begin{Entry}{停留}{11,10}{⼈、⽥}
  \begin{Phonetics}{停留}{ting2 liu2}[][HSK 5]
    \definition{v.}{permanecer; ficar por muito tempo; parar temporariamente em algum lugar, sem continuar avançando | permanecer; parar por um longo tempo; parar em um determinado estágio ou nível, sem evoluir}
  \end{Phonetics}
\end{Entry}

\begin{Entry}{停课}{11,10}{⼈、⾔}
  \begin{Phonetics}{停课}{ting2ke4}
    \definition{v.}{fechar (escola) | parar as aulas}
  \end{Phonetics}
\end{Entry}

\begin{Entry}{停歇}{11,13}{⼈、⽋}
  \begin{Phonetics}{停歇}{ting2xie1}
    \definition{v.}{parar para descansar}
  \end{Phonetics}
\end{Entry}

\begin{Entry}{偶}{11}{⼈}
  \begin{Phonetics}{偶}{ou3}
    \definition{adv.}{por acaso; por acidente; de vez em quando; ocasionalmente | par; número par; pareado (em oposição a 奇)}
    \definition{s.}{imagem; ídolo; figuras feitas de madeira, barro, etc. | companheiro; cônjuge; parceiro; refere-se a um casal ou a um dos casais}
  \seealsoref{奇}{qi2}
  \end{Phonetics}
\end{Entry}

\begin{Entry}{偶尔}{11,5}{⼈、⼩}
  \begin{Phonetics}{偶尔}{ou3'er3}[][HSK 5]
    \definition{adj.}{ocasional}
    \definition{adv.}{ocasionalmente; de vez em quando; às vezes}
  \end{Phonetics}
\end{Entry}

\begin{Entry}{偶然}{11,12}{⼈、⽕}
  \begin{Phonetics}{偶然}{ou3ran2}[][HSK 5]
    \definition{adj.}{acidental; ocasional}
    \definition{adv.}{por acaso; acidentalmente; sem querer; inesperadamente | ocasionalmente; de vez em quando; às vezes}
  \end{Phonetics}
\end{Entry}

\begin{Entry}{偶像}{11,13}{⼈、⼈}
  \begin{Phonetics}{偶像}{ou3xiang4}[][HSK 5]
    \definition[位,个,名]{s.}{ídolo; pessoa amada pelas pessoas; refere-se a uma pessoa que é apreciada por todos e que, em certos aspectos, é digna de admiração e respeito}
  \end{Phonetics}
\end{Entry}

\begin{Entry}{偷}{11}{⼈}
  \begin{Phonetics}{偷}{tou1}[][HSK 5]
    \definition{adv.}{furtivamente; secretamente; às escondidas}
    \definition{s.}{ladrão; furtador}
    \definition{v.}{roubar; furtar; levar sem pagar; roubar os bens alheios às escondidas | encontrar (tempo) | deixar-se levar; viver apenas para o presente, sem se preocupar com o futuro}
  \end{Phonetics}
\end{Entry}

\begin{Entry}{偷安}{11,6}{⼈、⼧}
  \begin{Phonetics}{偷安}{tou1'an1}
    \definition{v.}{buscar facilidade temporária}
  \end{Phonetics}
\end{Entry}

\begin{Entry}{偷听}{11,7}{⼈、⼝}
  \begin{Phonetics}{偷听}{tou1ting1}
    \definition{v.}{bisbilhotar; monitorar (secretamente)}
  \end{Phonetics}
\end{Entry}

\begin{Entry}{偷窃}{11,9}{⼈、⽳}
  \begin{Phonetics}{偷窃}{tou1qie4}
    \definition{v.}{furtar | roubar}
  \end{Phonetics}
\end{Entry}

\begin{Entry}{偷偷}{11,11}{⼈、⼈}
  \begin{Phonetics}{偷偷}{tou1 tou1}[][HSK 5]
    \definition{adv.}{secretamente; dissimuladamente; furtivamente; às escondidas; descreve uma ação que não é notada pelos outros; em segredo ou em privado, não revelada}
  \end{Phonetics}
\end{Entry}

\begin{Entry}{偷情}{11,11}{⼈、⼼}
  \begin{Phonetics}{偷情}{tou1qing2}
    \definition{v.}{manter um caso de amor clandestino}
  \end{Phonetics}
\end{Entry}

\begin{Entry}{偷袭}{11,11}{⼈、⾐}
  \begin{Phonetics}{偷袭}{tou1xi2}
    \definition{s.}{ataque surpresa}
    \definition{v.}{montar um ataque furtivo | invadir}
  \end{Phonetics}
\end{Entry}

\begin{Entry}{偷渡}{11,12}{⼈、⽔}
  \begin{Phonetics}{偷渡}{tou1du4}
    \definition{s.}{contrabando | imigração ilegal | clandestino (em um navio)}
    \definition{v.}{executar um bloqueio | roubar através da fronteira internacional}
  \end{Phonetics}
\end{Entry}

\begin{Entry}{偷税}{11,12}{⼈、⽲}
  \begin{Phonetics}{偷税}{tou1shui4}
    \definition{s.}{evasão fiscal}
  \end{Phonetics}
\end{Entry}

\begin{Entry}{偸}{11}{⼈}
  \begin{Phonetics}{偸}{tou1}
    \variantof{偷}
  \end{Phonetics}
\end{Entry}

\begin{Entry}{傍}{12}{⼈}
  \begin{Phonetics}{傍}{bang4}
    \definition*{s.}{Sobrenome Bang}
    \definition{v.}{estar perto de (à distância); aproximar-se | estar perto de (no tempo) | depender de; confiar em}
  \end{Phonetics}
\end{Entry}

\begin{Entry}{傍晚}{12,11}{⼈、⽇}
  \begin{Phonetics}{傍晚}{bang4wan3}[][HSK 6]
    \definition[个]{s.}{ao entardecer; ao cair da noite; (tarde) refere-se ao momento em que se aproxima o anoitecer, frequentemente usado na linguagem escrita}
  \end{Phonetics}
\end{Entry}

\begin{Entry}{傢}{12}{⼈}
  \begin{Phonetics}{傢}{jia1}
    \definition{s.}{usado em 家伙  e 家俱}
    \variantof{家}
  \seealsoref{傢伙}{jia1huo5}
  \seealsoref{家俱}{jia1ju4}
  \end{Phonetics}
\end{Entry}

\begin{Entry}{傢伙}{12,6}{⼈、⼈}
  \begin{Phonetics}{傢伙}{jia1huo5}
    \variantof{家伙}
  \end{Phonetics}
\end{Entry}

\begin{Entry}{傢俱}{12,10}{⼈、⼈}
  \begin{Phonetics}{傢俱}{jia1ju4}
    \variantof{家俱}
  \end{Phonetics}
\end{Entry}

\begin{Entry}{储}{12}{⼈}
  \begin{Phonetics}{储}{chu3}
    \definition*{s.}{Sobrenome Chu}
    \definition{s.}{herdeiro de um trono | herdeiro}
    \definition{v.}{armazenar | guardar; manter (ter) em reserva}
  \end{Phonetics}
\end{Entry}

\begin{Entry}{储存}{12,6}{⼈、⼦}
  \begin{Phonetics}{储存}{chu3cun2}[][HSK 6]
    \definition{v.}{armazenar; depositar; colocar em; economizar dinheiro ou coisas que você não precisará em um futuro próximo}
  \end{Phonetics}
\end{Entry}

\begin{Entry}{傻}{13}{⼈}
  \begin{Phonetics}{傻}{sha3}[][HSK 5]
    \definition{adj.}{estúpido; confuso; burro; idiota; inflexível; (ação ou pensamento) mecânico}
  \end{Phonetics}
\end{Entry}

\begin{Entry}{傻瓜}{13,5}{⼈、⽠}
  \begin{Phonetics}{傻瓜}{sha3gua1}
    \definition{adj.}{tolo | burro | simplório | idiota}
    \definition{v.}{enganar | iludir | lograr}
  \end{Phonetics}
\end{Entry}

\begin{Entry}{傻眼}{13,11}{⼈、⽬}
  \begin{Phonetics}{傻眼}{sha3yan3}
    \definition{adj.}{estupefato | atordoado}
  \end{Phonetics}
\end{Entry}

\begin{Entry}{像}{13}{⼈}
  \begin{Phonetics}{像}{xiang4}[][HSK 2]
    \definition{adv.}{parecer; parecer como se}
    \definition{s.}{imagem; retrato; semelhança a alguém | imagem}
    \definition{v.}{assemelhar-se; ser como; parecer-se com | ser como; ser tal como}
  \end{Phonetics}
\end{Entry}

\begin{Entry}{僧}{14}{⼈}
  \begin{Phonetics}{僧}{seng1}
    \definition*{s.}{Sobrenome Seng}
    \definition[位,名,个]{s.}{monge Budista, abreviação de 僧伽}
  \seealsoref{僧伽}{seng1qie2}
  \end{Phonetics}
\end{Entry}

\begin{Entry}{僧伽}{14,7}{⼈、⼈}
  \begin{Phonetics}{僧伽}{seng1qie2}
    \definition{s.}{sangha ou sanga (Budismo) | a comunidade monástica | monge}
  \end{Phonetics}
\end{Entry}

\begin{Entry}{僮}{14}{⼈}
  \begin{Phonetics}{僮}{tong2}
    \definition*{s.}{Sobrenome Tong}
  \end{Phonetics}
  \begin{Phonetics}{僮}{zhuang4}
    \variantof{壮}
  \end{Phonetics}
\end{Entry}

\begin{Entry}{儒}{16}{⼈}
  \begin{Phonetics}{儒}{ru2}
    \definition*{s.}{Confucionismo; Confucionista | Sobrenome Ru}
    \definition{s.}{(antigo) erudito; homem culto}
  \end{Phonetics}
\end{Entry}

\begin{Entry}{儒教}{16,11}{⼈、⽁}
  \begin{Phonetics}{儒教}{ru2jiao4}
    \definition*{s.}{Confucionismo}
  \end{Phonetics}
\end{Entry}

%%%%% EOF %%%%%

