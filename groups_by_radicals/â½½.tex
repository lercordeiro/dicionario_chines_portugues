%%%
%%% Radical "⽽"
%%%

\section*{Radical 126: ``⽽''}\addcontentsline{toc}{section}{Radical 126: ⽽}

\begin{Entry}{而}{6}{⽽}[Kangxi 126]
  \begin{Phonetics}{而}{er2}[][HSK 4]
    \definition{conj.}{e (coordenação) | e ainda (restrição) | conexão de componentes com continuidade semântica | conecxão de componentes afirmativos e negativos que se complementam | conexão de componentes com significados opostos para indicar um contraste |  conexão de componentes de causa e efeito no raciocínio | significa “chegar” ou “alcançar” | conexão de componentes que indicam tempo ou modo ao verbo | inserido entre o sujeito e o predicado, significa 如果}
  \seealsoref{如果}{ru2guo3}
  \end{Phonetics}
\end{Entry}

\begin{Entry}{而且}{6,5}{⽽、⼀}
  \begin{Phonetics}{而且}{er2 qie3}[][HSK 2]
    \definition{conj.}{e também; indica igualdade | e isso; não só\dots mas (também); indica um passo adiante}
  \end{Phonetics}
\end{Entry}

\begin{Entry}{而况}{6,7}{⽽、⼎}
  \begin{Phonetics}{而况}{er2kuang4}
    \definition{conj.}{além disso | além do mais}
  \end{Phonetics}
\end{Entry}

\begin{Entry}{而是}{6,9}{⽽、⽇}
  \begin{Phonetics}{而是}{er2 shi4}[][HSK 4]
    \definition{conj.}{mas; em vez disso; geralmente usada em conjunto com 不是 para formar o correlativo 不是……而是, indicando uma relação paralela}
  \seealsoref{不是……而是}{bu4shi4 er2 shi4}
  \end{Phonetics}
\end{Entry}

\begin{Entry}{耍}{9}{⽽}
  \begin{Phonetics}{耍}{shua3}
    \definition{v.}{brincar com | empunhar | agir (legal, calmo, tranquilo, descolado, etc.) | exibir (uma habilidade, o temperamento de alguém, etc.)}
  \end{Phonetics}
\end{Entry}

\begin{Entry}{耍赖}{9,13}{⽽、⾙}
  \begin{Phonetics}{耍赖}{shua3lai4}
    \definition{v.}{agir descaradamente | recusar -se a reconhecer que alguém perdeu o jogo ou fez uma promessa, etc. | agir como um idiota | agir como se algo nunca tivesse acontecido}
  \end{Phonetics}
\end{Entry}

\begin{Entry}{耐}{9}{⽽}
  \begin{Phonetics}{耐}{nai4}
    \definition{v.}{ser capaz de suportar; aguentar}
  \end{Phonetics}
\end{Entry}

\begin{Entry}{耐心}{9,4}{⽽、⼼}
  \begin{Phonetics}{耐心}{nai4xin1}[][HSK 5]
    \definition{adj.}{paciente}
    \definition[些]{s.}{paciência; uma pessoa que não se importa com problemas e é paciente}
  \end{Phonetics}
\end{Entry}

%%%%% EOF %%%%%

