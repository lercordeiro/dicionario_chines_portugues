%%%
%%% Radical "⽂"
%%%

\section*{Radical 67: ``⽂''}\addcontentsline{toc}{section}{Radical 67: ⽂}

\begin{entry}{文}{4}{⽂}
  \begin{phonetics}{文}{wen2}
    \definition*{s.}{Sobrenome Wen}
    \definition{adj.}{civil; não militar, não violento, oposto a 武 | suave; refinado | refinado; literário ; descreve que o conteúdo de um artigo ou discurso é difícil de entender}
    \definition{clas.}{usado para moedas de cobre antigas (moedas de cobre com palavras gravadas em um lado)}
    \definition{s.}{roteiro; escrita; personagem | escrita; composição literária; artigo | linguagem literária; redação | cultura; refere-se ao estado manifestado quando a sociedade atinge um estágio superior de desenvolvimento | ritual; refere-se ao antigo sistema ritual e musical | certos fenômenos naturais; refere-se a certos fenômenos na natureza ou na sociedade humana | literatos; coisas não militares (ao contrário de "武") | arte liberal; refere-se às ciências humanas e sociais | documento; refere-se a documentos oficiais | padrão; textura}
    \definition{v.}{tatuar padrões ou palavras no corpo ou no rosto | cobrir; pintar por cima}
  \seealsoref{武}{wu3}
  \end{phonetics}
\end{entry}

\begin{entry}{文化}{4,4}{⽂、⼔}
  \begin{phonetics}{文化}{wen2hua4}[][HSK 3]
    \definition[个,种]{s.}{cultura; civilização; tudo o que os seres humanos criaram em termos materiais e espirituais ao longo da história social | cultura; alfabetização; escolaridade; educação; o nível de conhecimento das pessoas e a capacidade de usar a linguagem escrita}
  \end{phonetics}
\end{entry}

\begin{entry}{文化水平}{4,4,4,5}{⽂、⼔、⽔、⼲}
  \begin{phonetics}{文化水平}{wen2hua4 shui3ping2}
    \definition{s.}{nível educacional}
  \end{phonetics}
\end{entry}

\begin{entry}{文化史}{4,4,5}{⽂、⼔、⼝}
  \begin{phonetics}{文化史}{wen2hua4shi3}
    \definition*{s.}{História Cultural}
  \end{phonetics}
\end{entry}

\begin{entry}{文化层}{4,4,7}{⽂、⼔、⼫}
  \begin{phonetics}{文化层}{wen2hua4ceng2}
    \definition{s.}{nível de cultura (em sítio arqueológico)}
  \end{phonetics}
\end{entry}

\begin{entry}{文化宫}{4,4,9}{⽂、⼔、⼧}
  \begin{phonetics}{文化宫}{wen2hua4gong1}
    \definition{s.}{palácio cultural}
  \end{phonetics}
\end{entry}

\begin{entry}{文化热}{4,4,10}{⽂、⼔、⽕}
  \begin{phonetics}{文化热}{wen2hua4re4}
    \definition{s.}{mania cultural | febre cultural}
  \end{phonetics}
\end{entry}

\begin{entry}{文化圈}{4,4,11}{⽂、⼔、⼞}
  \begin{phonetics}{文化圈}{wen2hua4quan1}
    \definition{s.}{esfera de influência cultural}
  \end{phonetics}
\end{entry}

\begin{entry}{文化障碍}{4,4,13,13}{⽂、⼔、⾩、⽯}
  \begin{phonetics}{文化障碍}{wen2hua4zhang4'ai4}
    \definition{s.}{barreira cultural}
  \end{phonetics}
\end{entry}

\begin{entry}{文艺}{4,4}{⽂、⾋}
  \begin{phonetics}{文艺}{wen2yi4}[][HSK 5]
    \definition{s.}{termo genérico para literatura e arte | performance (arte); refere-se especificamente às artes performativas, como música e dança}
  \end{phonetics}
\end{entry}

\begin{entry}{文件}{4,6}{⽂、⼈}
  \begin{phonetics}{文件}{wen2jian4}[][HSK 3]
    \definition[份,堆,叠]{s.}{documentos oficiais; papéis; instrumentos; termo genérico para documentos oficiais, cartas, etc. | os arquivos no computador; informações registradas no celular ou computador | artigos ou trabalhos sobre teorias políticas, atualidades, pesquisas acadêmicas, etc.; textos ou artigos sobre teoria política, políticas, etc.}
  \end{phonetics}
\end{entry}

\begin{entry}{文字}{4,6}{⽂、⼦}
  \begin{phonetics}{文字}{wen2zi4}[][HSK 3]
    \definition[种,类,段,行,篇]{s.}{caracteres; caligrafia; escrita; símbolos escritos para registrar a linguagem| linguagem escrita; a forma escrita da língua}
  \end{phonetics}
\end{entry}

\begin{entry}{文学}{4,8}{⽂、⼦}
  \begin{phonetics}{文学}{wen2xue2}[][HSK 3]
    \definition[个,种]{s.}{literatura; a arte de moldar imagens e refletir a vida social através da linguagem e da escrita, incluindo romances, poesia, prosa, teatro, etc.}
  \end{phonetics}
\end{entry}

\begin{entry}{文学系}{4,8,7}{⽂、⼦、⽷}
  \begin{phonetics}{文学系}{wen2xue2 xi4}
    \definition*{s.}{Faculdade de Letras}
  \end{phonetics}
\end{entry}

\begin{entry}{文明}{4,8}{⽂、⽇}
  \begin{phonetics}{文明}{wen2ming2}[][HSK 3]
    \definition{adj.}{civilizado; sociedade desenvolvida e com alto nível cultural}
    \definition[个,种]{s.}{cultura; civilização}
  \end{phonetics}
\end{entry}

\begin{entry}{文章}{4,11}{⽂、⾳}
  \begin{phonetics}{文章}{wen2zhang1}[][HSK 3]
    \definition[篇,段,页,系列]{s.}{ensaio; artigo; texto independente; também se refere a obras literárias em geral | significado oculto; significado implícito | trabalho; coisas que podem ser feitas}
  \end{phonetics}
\end{entry}

%%%%% EOF %%%%%

