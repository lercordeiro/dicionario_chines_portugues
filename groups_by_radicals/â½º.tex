%%%
%%% Radical "⽺"
%%%

\section*{Radical 123: ``⽺'' (⺶、⺷)}\addcontentsline{toc}{section}{Radical 123: ⽺、⺶、⺷}

\begin{entry}{羊}{6}{⽺}[Kangxi 123]
  \begin{phonetics}{羊}{yang2}[][HSK 3]
    \definition*{s.}{sobrenome Yang}
    \definition{s.}{carneiro; ovelha; bode; cabra}
  \end{phonetics}
\end{entry}

\begin{entry}{美}{9}{⽺}
  \begin{phonetics}{美}{mei3}[][HSK 3]
    \definition*{s.}{Abreviatura de América (美洲) | Abreviatura de Estados Unidos da América (美国)}
    \definition{adj.}{lindo; bonito; belo; atraente | satisfatório; bom; agradável}
    \definition{v.}{embelezar; enfeitar | orgulhar-se de; estar satisfeito consigo mesmo}
  \seealsoref{美国}{mei3guo2}
  \seealsoref{美洲}{mei3zhou1}
  \end{phonetics}
\end{entry}

\begin{entry}{美女}{9,3}{⽺、⼥}
  \begin{phonetics}{美女}{mei3 nv3}[][HSK 4]
    \definition[个,位]{s.}{beldade; mulher bonita; uma jovem linda}
  \end{phonetics}
\end{entry}

\begin{entry}{美元}{9,4}{⽺、⼉}
  \begin{phonetics}{美元}{mei3yuan2}[][HSK 3]
    \definition*[元,笔,沓]{s.}{Dólar Americano}
  \end{phonetics}
\end{entry}

\begin{entry}{美术}{9,5}{⽺、⽊}
  \begin{phonetics}{美术}{mei3shu4}[][HSK 3]
    \definition[种]{s.}{arte; belas artes | pintura}
  \end{phonetics}
\end{entry}

\begin{entry}{美甲}{9,5}{⽺、⽥}
  \begin{phonetics}{美甲}{mei3jia3}
    \definition{s.}{manicure e/ou pedicure}
  \end{phonetics}
\end{entry}

\begin{entry}{美好}{9,6}{⽺、⼥}
  \begin{phonetics}{美好}{mei3 hao3}[][HSK 3]
    \definition{adj.}{bem; feliz; glorioso}
  \end{phonetics}
\end{entry}

\begin{entry}{美丽}{9,7}{⽺、⼀}
  \begin{phonetics}{美丽}{mei3li4}[][HSK 3]
    \definition{adj.}{bonito; lindo}
  \end{phonetics}
\end{entry}

\begin{entry}{美味}{9,8}{⽺、⼝}
  \begin{phonetics}{美味}{mei3wei4}
    \definition{adj.}{delicioso}
    \definition{s.}{comida deliciosa | delicadeza (\emph{delicacy})}
  \end{phonetics}
\end{entry}

\begin{entry}{美国}{9,8}{⽺、⼞}
  \begin{phonetics}{美国}{mei3guo2}
    \definition*{s.}{Estados Unidos da América}
  \end{phonetics}
\end{entry}

\begin{entry}{美国人}{9,8,2}{⽺、⼞、⼈}
  \begin{phonetics}{美国人}{mei3guo2ren2}
    \definition{s.}{americano | pessoa ou povo dos Estados Unidos da América}
  \end{phonetics}
\end{entry}

\begin{entry}{美学}{9,8}{⽺、⼦}
  \begin{phonetics}{美学}{mei3xue2}
    \definition{s.}{estética}
  \end{phonetics}
\end{entry}

\begin{entry}{美金}{9,8}{⽺、⾦}
  \begin{phonetics}{美金}{mei3 jin1}[][HSK 4]
    \definition{s.}{USD; dólar americano: a moeda local dos Estados Unidos}
  \end{phonetics}
\end{entry}

\begin{entry}{美洲}{9,9}{⽺、⽔}
  \begin{phonetics}{美洲}{mei3zhou1}
    \definition*{s.}{América (incluindo Norte, Central e Sul)}
  \end{phonetics}
\end{entry}

\begin{entry}{美洲人}{9,9,2}{⽺、⽔、⼈}
  \begin{phonetics}{美洲人}{mei3zhou1ren2}
    \definition{s.}{americano | pessoa ou povo do continente Americano}
  \end{phonetics}
\end{entry}

\begin{entry}{美食}{9,9}{⽺、⾷}
  \begin{phonetics}{美食}{mei3 shi2}[][HSK 3]
    \definition[种,道,桌]{s.}{iguaria; comida deliciosa}
  \end{phonetics}
\end{entry}

\begin{entry}{羡慕}{12,14}{⽺、⼼}
  \begin{phonetics}{羡慕}{xian4mu4}
    \definition{v.}{invejar | admirar}
  \end{phonetics}
\end{entry}

\begin{entry}{群}{13}{⽺}
  \begin{phonetics}{群}{qun2}[][HSK 3]
    \definition{clas.}{grupo; rebanho; manada}
    \definition{s.}{multidão; grupo}
  \end{phonetics}
\end{entry}

\begin{entry}{群山}{13,3}{⽺、⼭}
  \begin{phonetics}{群山}{qun2shan1}
    \definition{s.}{montanhas | uma cadeia de colinas}
  \end{phonetics}
\end{entry}

\begin{entry}{群众}{13,6}{⽺、⼈}
  \begin{phonetics}{群众}{qun2zhong4}[][HSK 5]
    \definition[个,名,位]{s.}{as massas; refere-se ao povo em geral | não filiado; apartidário; refere-se a pessoas que não são membros do Partido Comunista Chinês nem da Liga da Juventude Comunista |
alguém que não ocupa uma posição de liderança}
  \end{phonetics}
\end{entry}

\begin{entry}{群体}{13,7}{⽺、⼈}
  \begin{phonetics}{群体}{qun2 ti3}[][HSK 5]
    \definition{s.}{colônia; um conjunto composto por muitos indivíduos da mesma espécie que estão fisicamente conectados, exemplos incluem corais entre os animais e certas algas entre as plantas | grupos; refere-se, de maneira geral, ao conjunto formado por muitos indivíduos interligados que compartilham características essenciais em comum}
  \end{phonetics}
\end{entry}

%%%%% EOF %%%%%

