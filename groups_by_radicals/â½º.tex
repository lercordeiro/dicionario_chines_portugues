%%%
%%% Radical "⽺"
%%%

\section*{Radical 123: ``⽺'' (⺶、⺷)}\addcontentsline{toc}{section}{Radical 123: ⽺、⺶、⺷}

\begin{Entry}{羊}{6}{⽺}[Kangxi 123]
  \begin{Phonetics}{羊}{yang2}[][HSK 3]
    \definition*{s.}{Sobrenome Yang}
    \definition[只,头,群]{s.}{carneiro; ovelha; bode; cabra; antílope}
  \end{Phonetics}
\end{Entry}

\begin{Entry}{美}{9}{⽺}
  \begin{Phonetics}{美}{mei3}[][HSK 3]
    \definition*{s.}{Abreviatura de América, 美洲 | Abreviatura de Estados Unidos da América, 美国 | As Américas, 美洲}
    \definition{adj.}{belo; bonito (oposto de 丑) | muito satisfatório; bom; agradável}
    \definition{s.}{beleza (oposto de 丑)}
    \definition{v.}{embelezar; tornar mais bonito | estar satisfeito consigo mesmo; orgulhar-se; sentir-se presunçoso}
  \seealsoref{丑}{chou3}
  \seealsoref{美国}{mei3guo2}
  \seealsoref{美洲}{mei3zhou1}
  \end{Phonetics}
\end{Entry}

\begin{Entry}{美女}{9,3}{⽺、⼥}
  \begin{Phonetics}{美女}{mei3 nv3}[][HSK 4]
    \definition[个,位,名,些]{s.}{beldade; mulher bonita; uma jovem linda}
  \end{Phonetics}
\end{Entry}

\begin{Entry}{美元}{9,4}{⽺、⼉}
  \begin{Phonetics}{美元}{mei3yuan2}[][HSK 3]
    \definition*[元,笔,沓]{s.}{Dólar Americano; a moeda dos Estados Unidos}
  \end{Phonetics}
\end{Entry}

\begin{Entry}{美术}{9,5}{⽺、⽊}
  \begin{Phonetics}{美术}{mei3shu4}[][HSK 3]
    \definition[种]{s.}{arte; artes plásticas: arte que ocupa um determinado espaço, compõe imagens estéticas e permite que as pessoas apreciem visualmente, incluindo pintura, escultura, arquitetura, etc. | pintura; pintura tradicional chinesa}
  \end{Phonetics}
\end{Entry}

\begin{Entry}{美甲}{9,5}{⽺、⽥}
  \begin{Phonetics}{美甲}{mei3jia3}
    \definition{s.}{manicure e/ou pedicure}
  \end{Phonetics}
\end{Entry}

\begin{Entry}{美好}{9,6}{⽺、⼥}
  \begin{Phonetics}{美好}{mei3 hao3}[][HSK 3]
    \definition{adj.}{bem; feliz; glorioso; descreve a vida, os desejos, etc. como sendo muito bons e satisfatórios}
  \end{Phonetics}
\end{Entry}

\begin{Entry}{美丽}{9,7}{⽺、⼀}
  \begin{Phonetics}{美丽}{mei3li4}[][HSK 3]
    \definition{adj.}{bonito; lindo; capaz de proporcionar uma sensação de beleza}
  \end{Phonetics}
\end{Entry}

\begin{Entry}{美味}{9,8}{⽺、⼝}
  \begin{Phonetics}{美味}{mei3wei4}
    \definition{adj.}{delicioso}
    \definition{s.}{comida deliciosa | delicadeza (\emph{delicacy})}
  \end{Phonetics}
\end{Entry}

\begin{Entry}{美国}{9,8}{⽺、⼞}
  \begin{Phonetics}{美国}{mei3guo2}
    \definition*{s.}{Estados Unidos da América}
  \end{Phonetics}
\end{Entry}

\begin{Entry}{美国人}{9,8,2}{⽺、⼞、⼈}
  \begin{Phonetics}{美国人}{mei3guo2ren2}
    \definition{s.}{americano | pessoa ou povo dos Estados Unidos da América}
  \end{Phonetics}
\end{Entry}

\begin{Entry}{美学}{9,8}{⽺、⼦}
  \begin{Phonetics}{美学}{mei3xue2}
    \definition{s.}{estética; a ciência que estuda as leis e os princípios gerais da beleza na natureza, na sociedade e na arte explora principalmente a natureza da beleza, a relação entre arte e realidade e as leis gerais da criação artística}
  \end{Phonetics}
\end{Entry}

\begin{Entry}{美金}{9,8}{⽺、⾦}
  \begin{Phonetics}{美金}{mei3 jin1}[][HSK 4]
    \definition{s.}{USD; dólar americano: a moeda local dos Estados Unidos}
  \end{Phonetics}
\end{Entry}

\begin{Entry}{美洲}{9,9}{⽺、⽔}
  \begin{Phonetics}{美洲}{mei3zhou1}
    \definition*{s.}{América (incluindo Norte, Central e Sul)}
  \end{Phonetics}
\end{Entry}

\begin{Entry}{美洲人}{9,9,2}{⽺、⽔、⼈}
  \begin{Phonetics}{美洲人}{mei3zhou1ren2}
    \definition{s.}{americano | pessoa ou povo do continente Americano}
  \end{Phonetics}
\end{Entry}

\begin{Entry}{美食}{9,9}{⽺、⾷}
  \begin{Phonetics}{美食}{mei3 shi2}[][HSK 3]
    \definition[种,道,桌]{s.}{iguaria; (gastronomia) comida saborosa}
  \end{Phonetics}
\end{Entry}

\begin{Entry}{美容}{9,10}{⽺、⼧}
  \begin{Phonetics}{美容}{mei3 rong2}[][HSK 6]
    \definition{v.}{embelezar; melhorar a aparência de alguém; deixar seu rosto bonito retocando, cuidando, etc.}
  \end{Phonetics}
\end{Entry}

\begin{Entry}{羡}{12}{⽺}
  \begin{Phonetics}{羡}{xian4}
    \definition{v.}{admirar; invejar}
  \end{Phonetics}
\end{Entry}

\begin{Entry}{羡慕}{12,14}{⽺、⼼}
  \begin{Phonetics}{羡慕}{xian4mu4}
    \definition{v.}{invejar | admirar}
  \end{Phonetics}
\end{Entry}

\begin{Entry}{群}{13}{⽺}
  \begin{Phonetics}{群}{qun2}[][HSK 3]
    \definition*{s.}{Sobrenome Qun}
    \definition{adj.}{em grupos; numerosos}
    \definition{clas.}{usado para grupos de pessoas ou coisas; grupo; rebanho; manada}
    \definition{s.}{multidão; grupo; muitas pessoas ou coisas reunidas | as massas; um grupo de pessoas; refere-se a um grande número de pessoas}
  \end{Phonetics}
\end{Entry}

\begin{Entry}{群山}{13,3}{⽺、⼭}
  \begin{Phonetics}{群山}{qun2shan1}
    \definition{s.}{montanhas | uma cadeia de colinas}
  \end{Phonetics}
\end{Entry}

\begin{Entry}{群众}{13,6}{⽺、⼈}
  \begin{Phonetics}{群众}{qun2zhong4}[][HSK 5]
    \definition[个,名,位]{s.}{as massas; refere-se ao povo em geral | não filiado; apartidário; refere-se a pessoas que não são membros do Partido Comunista Chinês nem da Liga da Juventude Comunista | alguém que não ocupa uma posição de liderança}
  \end{Phonetics}
\end{Entry}

\begin{Entry}{群体}{13,7}{⽺、⼈}
  \begin{Phonetics}{群体}{qun2 ti3}[][HSK 5]
    \definition[个]{s.}{colônia; um conjunto composto por muitos indivíduos da mesma espécie que estão fisicamente conectados, exemplos incluem corais entre os animais e certas algas entre as plantas | grupos; refere-se, de maneira geral, ao conjunto formado por muitos indivíduos interligados que compartilham características essenciais em comum}
  \end{Phonetics}
\end{Entry}

%%%%% EOF %%%%%

