%%%
%%% Radical "⽹"
%%%

\section*{Radical 122: ``⽹'' (⺲、罓、⺳)}\addcontentsline{toc}{section}{Radical 122: ⽹、⺲、罓、⺳}

\begin{entry}{网}{6}{⽹}[Kangxi 122]
  \begin{phonetics}{网}{wang3}[][HSK 2]
    \definition[张]{s.}{rede; um dispositivo feito de corda ou barbante para capturar peixes e pássaros | algo que parece uma rede | rede; uma rede de organizações; um sistema}
    \definition{v.}{pegar com uma rede | cobrir como com uma rede}
  \end{phonetics}
\end{entry}

\begin{entry}{网上}{6,3}{⽹、⼀}
  \begin{phonetics}{网上}{wang3 shang4}[][HSK 1]
    \definition{s.}{\emph{online}; refere-se a acessar a Internet através de um computador ou celular para pesquisar e consultar informações na rede}
  \end{phonetics}
\end{entry}

\begin{entry}{网上银行}{6,3,11,6}{⽹、⼀、⾦、⾏}
  \begin{phonetics}{网上银行}{wang3shang4yin2hang2}
    \definition[个]{s.}{banco \emph{online} | acesso a operações bancárias via \emph{Internet}}
  \seealsoref{网银}{wang3yin2}
  \end{phonetics}
\end{entry}

\begin{entry}{网友}{6,4}{⽹、⼜}
  \begin{phonetics}{网友}{wang3 you3}[][HSK 1]
    \definition{s.}{internauta; usuário da \emph{Internet}; amigos que se conhecem pela Internet; também usado como forma de tratamento entre internautas}
  \end{phonetics}
\end{entry}

\begin{entry}{网址}{6,7}{⽹、⼟}
  \begin{phonetics}{网址}{wang3 zhi3}[][HSK 4]
    \definition{s.}{\emph{website}; endereço da \emph{web}; endereço de um \emph{site} na \emph{Internet}, que os usuários podem acessar, consultar e obter recursos de informações nesse \emph{site} clicando nele}
  \end{phonetics}
\end{entry}

\begin{entry}{网际网络}{6,7,6,9}{⽹、⾩、⽹、⽷}
  \begin{phonetics}{网际网络}{wang3ji4wang3luo4}
    \definition*{s.}{Internet}
  \seealsoref{互联网}{hu4lian2wang3}
  \seealsoref{网际网路}{wang3ji4wang3lu4}
  \seealsoref{网路}{wang3lu4}
  \end{phonetics}
\end{entry}

\begin{entry}{网际网路}{6,7,6,13}{⽹、⾩、⽹、⾜}
  \begin{phonetics}{网际网路}{wang3ji4wang3lu4}
    \definition*{s.}{Internet}
  \seealsoref{互联网}{hu4lian2wang3}
  \seealsoref{网际网络}{wang3ji4wang3luo4}
  \seealsoref{网路}{wang3lu4}
  \end{phonetics}
\end{entry}

\begin{entry}{网络}{6,9}{⽹、⽷}
  \begin{phonetics}{网络}{wang3luo4}[][HSK 4]
    \definition{s.}{rede; um sistema que consiste em ramificações interconectadas; em um sistema elétrico, um circuito ou parte de um circuito que consiste em vários elementos que permitem a transmissão de sinais elétricos de acordo com determinados requisitos | rede; rede de computadores}
  \end{phonetics}
\end{entry}

\begin{entry}{网站}{6,10}{⽹、⽴}
  \begin{phonetics}{网站}{wang3zhan4}[][HSK 2]
    \definition[个,家]{s.}{\emph{web}; \emph{website}; um site virtual na Internet para uma organização ou indivíduo, geralmente consistindo em uma página inicial e muitas páginas da web}
  \end{phonetics}
\end{entry}

\begin{entry}{网罟}{6,10}{⽹、⽹}
  \begin{phonetics}{网罟}{wang3gu3}
    \definition{s.}{(fig.) a rede da justiça | rede usada para capturar peixes (ou outros animais, como pássaros)}
  \end{phonetics}
\end{entry}

\begin{entry}{网球}{6,11}{⽹、⽟}
  \begin{phonetics}{网球}{wang3qiu2}[][HSK 2]
    \definition[个,颗,些]{s.}{tênis (esporte) | bola de tênis}
  \end{phonetics}
\end{entry}

\begin{entry}{网银}{6,11}{⽹、⾦}
  \begin{phonetics}{网银}{wang3yin2}
    \definition{s.}{banco \emph{online} | acesso a operações bancárias via \emph{Internet}}
  \seealsoref{网上银行}{wang3shang4yin2hang2}
  \end{phonetics}
\end{entry}

\begin{entry}{网路}{6,13}{⽹、⾜}
  \begin{phonetics}{网路}{wang3lu4}
    \definition{s.}{\emph{Internet}}
  \seealsoref{互联网}{hu4lian2wang3}
  \seealsoref{网际网路}{wang3ji4wang3lu4}
  \seealsoref{网际网络}{wang3ji4wang3luo4}
  \end{phonetics}
\end{entry}

\begin{entry}{罗}{8}{⽹}
  \begin{phonetics}{罗}{luo2}
    \definition*{s.}{Sobrenome Luo}
    \definition{v.}{coletar | juntar | pegar | peneirar}
  \end{phonetics}
\end{entry}

\begin{entry}{罚}{9}{⽹}
  \begin{phonetics}{罚}{fa2}[][HSK 5]
    \definition{s.}{punição; penalidade}
    \definition{v.}{punir; penalizar; multar; confiscar}
  \end{phonetics}
\end{entry}

\begin{entry}{罚款}{9,12}{⽹、⽋}
  \begin{phonetics}{罚款}{fa2kuan3}[][HSK 5]
    \definition[笔,次,宗]{s.}{multa; penalidade; refere-se ao dinheiro pago por uma pessoa ou entidade de acordo com as disposições de um delito ou violação de contrato ou contrato}
    \definition{v.+compl.}{multar; penalizar; exigir, de acordo com os regulamentos, uma determinada quantia de dinheiro de uma pessoa ou entidade que tenha violado a lei ou descumprido um regulamento ou contrato}
  \end{phonetics}
\end{entry}

\begin{entry}{罢}{10}{⽹}
  \begin{phonetics}{罢}{ba4}
    \definition{v.}{parar; cessar | revogar; destituir; encerrar | terminar | abandonar uma ideia; esqueçer sobre algo; deixar estar (passar)}
  \end{phonetics}
  \begin{phonetics}{罢}{ba5}
    \definition{part.}{partícula final, a mesma que 吧}
  \seealsoref{吧}{ba5}
  \end{phonetics}
\end{entry}

\begin{entry}{罢了}{10,2}{⽹、⼅}
  \begin{phonetics}{罢了}{ba4 le5}[][HSK 6]
    \definition{part.}{usado no final de uma frase, significa 仅此而已, geralmente seguido de 无非, 不过, 只是}
  \seealsoref{不过}{bu2guo4}
  \seealsoref{仅此而已}{jin3ci3'er2yi3}
  \seealsoref{无非}{wu2fei1}
  \seealsoref{只是}{zhi3 shi4}
  \end{phonetics}
  \begin{phonetics}{罢了}{ba4 liao3}
    \definition{part.}{uma partícula modal indicando (não se preocupe, ok)}
  \end{phonetics}
\end{entry}

\begin{entry}{罢工}{10,3}{⽹、⼯}
  \begin{phonetics}{罢工}{ba4gong1}[][HSK 6]
    \definition{v.}{parar de trabalhar; entrar em greve; abandonar o emprego}
  \end{phonetics}
\end{entry}

\begin{entry}{罪}{13}{⽹}
  \begin{phonetics}{罪}{zui4}[][HSK 6]
    \definition{s.}{crime; culpa | falha; culpa | sofrimento; dor; dificuldade | má conduta; transgressão; negligência | agonia; dor; sofrimento}
    \definition{v.}{colocar a culpa em alguém; culpar}
  \end{phonetics}
\end{entry}

\begin{entry}{罪犯}{13,5}{⽹、⽝}
  \begin{phonetics}{罪犯}{zui4fan4}
    \definition{s.}{criminoso}
  \end{phonetics}
\end{entry}

\begin{entry}{罪行}{13,6}{⽹、⾏}
  \begin{phonetics}{罪行}{zui4xing2}
    \definition{s.}{crime | ofensa}
  \end{phonetics}
\end{entry}

\begin{entry}{置}{13}{⽹}
  \begin{phonetics}{置}{zhi4}
    \definition{v.}{colocar | configurar; estabelecer; instalar | comprar | organizar; consertar}
  \end{phonetics}
\end{entry}

\begin{entry}{置疑}{13,14}{⽹、⽦}
  \begin{phonetics}{置疑}{zhi4yi2}
    \definition{v.}{duvidar}
  \end{phonetics}
\end{entry}

%%%%% EOF %%%%%

