%%%
%%% Radical "⾍"
%%%

\section*{Radical 142: ``⾍''}\addcontentsline{toc}{section}{Radical 142: ⾍}

\begin{Entry}{虫}{6}{⾍}[Kangxi 142]
  \begin{Phonetics}{虫}{chong2}
    \definition[只,条]{s.}{inseto; verme | (pejorativo) pessoas que se comportam de forma desprezível | fã; viciado | forma inferior de vida animal, incluindo insetos, larvas de insetos, vermes e criaturas semelhantes | pessoa com uma característica indesejável específica}
  \end{Phonetics}
\end{Entry}

\begin{Entry}{虫子}{6,3}{⾍、⼦}
  \begin{Phonetics}{虫子}{chong2 zi5}[][HSK 4]
    \definition[条,只,种]{s.}{percevejo; besouro; inseto; verme; criaturas semelhantes a insetos}
  \end{Phonetics}
\end{Entry}

\begin{Entry}{虽}{9}{⾍}
  \begin{Phonetics}{虽}{sui1}[][HSK 6]
    \definition{conj.}{no entanto; embora | mesmo se}
  \end{Phonetics}
\end{Entry}

\begin{Entry}{虽然}{9,12}{⾍、⽕}
  \begin{Phonetics}{虽然}{sui1 ran2}[][HSK 2]
    \definition{conj.}{apesar de; embora (frequentemente usado correlativamente com 可是, 但是, etc); geralmente é usado no início de uma frase para indicar que o fato anterior foi reconhecido, mas não mudará o que acontecerá em seguida}
  \seealsoref{但是}{dan4 shi4}
  \seealsoref{可是}{ke3shi4}
  \end{Phonetics}
\end{Entry}

\begin{Entry}{虾}{9}{⾍}
  \begin{Phonetics}{虾}{xia1}
    \definition{s.}{camarão}
  \end{Phonetics}
\end{Entry}

\begin{Entry}{蚂}{9}{⾍}
  \begin{Phonetics}{蚂}{ma1}
    \definition{part.}{caracter formador de palavras}
    \definition[只]{s.}{libélula}
  \end{Phonetics}
  \begin{Phonetics}{蚂}{ma3}
    \definition{part.}{caracter formador de palavras}
  \end{Phonetics}
  \begin{Phonetics}{蚂}{ma4}
    \definition{part.}{caracter formador de palavras}
  \end{Phonetics}
\end{Entry}

\begin{Entry}{蚂蚁}{9,9}{⾍、⾍}
  \begin{Phonetics}{蚂蚁}{ma3yi3}
    \definition{s.}{formiga}
  \end{Phonetics}
\end{Entry}

\begin{Entry}{蚊}{10}{⾍}
  \begin{Phonetics}{蚊}{wen2}
    \definition{s.}{mosquito; pernilongo}
  \end{Phonetics}
\end{Entry}

\begin{Entry}{蚊子}{10,3}{⾍、⼦}
  \begin{Phonetics}{蚊子}{wen2zi5}
    \definition{s.}{pernilongo}
  \end{Phonetics}
\end{Entry}

\begin{Entry}{蚊香}{10,9}{⾍、⾹}
  \begin{Phonetics}{蚊香}{wen2xiang1}
    \definition{s.}{incenso ou espiral repelente de mosquitos}
  \end{Phonetics}
\end{Entry}

\begin{Entry}{蚕}{10}{⾍}
  \begin{Phonetics}{蚕}{can2}
    \definition[只,条]{s.}{bicho-da-seda; um inseto que pode fiar seda e fazer casulos}
  \end{Phonetics}
\end{Entry}

\begin{Entry}{蚕纸}{10,7}{⾍、⽷}
  \begin{Phonetics}{蚕纸}{can2zhi3}
    \definition{s.}{papel onde o bicho-da-seda põe seus ovos}
  \end{Phonetics}
\end{Entry}

\begin{Entry}{蚝}{10}{⾍}
  \begin{Phonetics}{蚝}{hao2}
    \definition[只]{s.}{ostra}
  \end{Phonetics}
\end{Entry}

\begin{Entry}{蛇}{11}{⾍}
  \begin{Phonetics}{蛇}{she2}[][HSK 5]
    \definition[条]{s.}{cobra; serpente; répteis}
  \end{Phonetics}
\end{Entry}

\begin{Entry}{蛋}{11}{⾍}
  \begin{Phonetics}{蛋}{dan4}[][HSK 2]
    \definition[个,只]{s.}{ovo; ovos produzidos por aves, tartarugas, cobras, etc. | algo em forma de ovo | tolo; idiota; metáfora para pessoas com determinadas características (com conotação pejorativa) | se perder; colocado após certos verbos, forma um verbo com conotação pejorativa | testículos; em algumas regiões, refere-se aos testículos de certos animais ou pessoas}
  \end{Phonetics}
\end{Entry}

\begin{Entry}{蛋糕}{11,16}{⾍、⽶}
  \begin{Phonetics}{蛋糕}{dan4gao1}[][HSK 5]
    \definition[个,块,盒]{s.}{bolo; bolo fofo feito de ovos e farinha com açúcar e óleo}
  \end{Phonetics}
\end{Entry}

\begin{Entry}{蜘}{14}{⾍}
  \begin{Phonetics}{蜘}{zhi1}
    \definition[只]{s.}{aranha}
  \seealsoref{蜘蛛}{zhi1zhu1}
  \end{Phonetics}
\end{Entry}

\begin{Entry}{蜘蛛}{14,12}{⾍、⾍}
  \begin{Phonetics}{蜘蛛}{zhi1zhu1}
    \definition{s.}{aranha}
  \end{Phonetics}
\end{Entry}

\begin{Entry}{蜘蛛网}{14,12,6}{⾍、⾍、⽹}
  \begin{Phonetics}{蜘蛛网}{zhi1zhu1 wang3}
    \definition{s.}{teia de aranha}
  \end{Phonetics}
\end{Entry}

\begin{Entry}{蜜}{14}{⾍}
  \begin{Phonetics}{蜜}{mi4}
    \definition{adj.}{melado; doce}
    \definition{s.}{mel | semelhante ao mel | coisas parecidas com mel; melaço}
  \end{Phonetics}
\end{Entry}

\begin{Entry}{蜜桃}{14,10}{⾍、⽊}
  \begin{Phonetics}{蜜桃}{mi4tao2}
    \definition{s.}{pêssego suculento}
  \end{Phonetics}
\end{Entry}

\begin{Entry}{蜡}{14}{⾍}
  \begin{Phonetics}{蜡}{la4}
    \definition{s.}{cera; óleos produzidos por animais, minerais ou plantas | vela}
  \end{Phonetics}
\end{Entry}

\begin{Entry}{蜡烛}{14,10}{⾍、⽕}
  \begin{Phonetics}{蜡烛}{la4zhu2}
    \definition[根,支]{s.}{vela | círio | peça, geralmente de cera, que possui um pavio e se utiliza para iluminar}
  \end{Phonetics}
\end{Entry}

\begin{Entry}{蜥}{14}{⾍}
  \begin{Phonetics}{蜥}{xi1}
    \definition{s.}{lagarto}
  \end{Phonetics}
\end{Entry}

\begin{Entry}{蜥易}{14,8}{⾍、⽇}
  \begin{Phonetics}{蜥易}{xi1yi4}
    \variantof{蜥蜴}
  \end{Phonetics}
\end{Entry}

\begin{Entry}{蜥蜴}{14,14}{⾍、⾍}
  \begin{Phonetics}{蜥蜴}{xi1yi4}
    \definition{s.}{lagarto}
  \end{Phonetics}
\end{Entry}

\begin{Entry}{蜻}{14}{⾍}
  \begin{Phonetics}{蜻}{qing1}
    \definition[只]{s.}{libélula, 蜻蜓}
  \seealsoref{蜻蜓}{qing1ting2}
  \end{Phonetics}
\end{Entry}

\begin{Entry}{蜻蜓}{14,12}{⾍、⾍}
  \begin{Phonetics}{蜻蜓}{qing1ting2}
    \definition{s.}{libélula}
  \end{Phonetics}
\end{Entry}

\begin{Entry}{蜻蝏}{14,15}{⾍、⾍}
  \begin{Phonetics}{蜻蝏}{qing1ting2}
    \variantof{蜻蜓}
  \end{Phonetics}
\end{Entry}

\begin{Entry}{蝉}{14}{⾍}
  \begin{Phonetics}{蝉}{chan2}
    \definition[只,个]{s.}{cigarra}
  \seealsoref{知了}{zhi1liao3}
  \end{Phonetics}
\end{Entry}

\begin{Entry}{蝌}{15}{⾍}
  \begin{Phonetics}{蝌}{ke1}
    \definition[只]{s.}{girino}
  \end{Phonetics}
\end{Entry}

\begin{Entry}{蝌蚪}{15,10}{⾍、⾍}
  \begin{Phonetics}{蝌蚪}{ke1dou3}
    \definition{s.}{girino}
  \end{Phonetics}
\end{Entry}

\begin{Entry}{蝲}{15}{⾍}
  \begin{Phonetics}{蝲}{la4}
    \definition{s.}{lagostim de água doce}
  \seealsoref{蝲蛄}{la4gu3}
  \end{Phonetics}
\end{Entry}

\begin{Entry}{蝲蛄}{15,11}{⾍、⾍}
  \begin{Phonetics}{蝲蛄}{la4gu3}
    \definition{s.}{lagostim; lagostim de água doce}
  \end{Phonetics}
\end{Entry}

\begin{Entry}{蝲蝲蛄}{15,15,11}{⾍、⾍、⾍}
  \begin{Phonetics}{蝲蝲蛄}{la4la4gu3}
    \definition{s.}{grilo toupeira}
  \end{Phonetics}
\end{Entry}

\begin{Entry}{蝴}{15}{⾍}
  \begin{Phonetics}{蝴}{hu2}
    \definition[对]{s.}{borboleta}
  \end{Phonetics}
\end{Entry}

\begin{Entry}{蝴蝶}{15,15}{⾍、⾍}
  \begin{Phonetics}{蝴蝶}{hu2die2}
    \definition[只]{s.}{borboleta}
  \end{Phonetics}
\end{Entry}

\begin{Entry}{螺}{17}{⾍}
  \begin{Phonetics}{螺}{luo2}
    \definition{s.}{concha em espiral | caracol | búzio}
  \end{Phonetics}
\end{Entry}

\begin{Entry}{螺丝}{17,5}{⾍、⼀}
  \begin{Phonetics}{螺丝}{luo2si1}
    \definition{s.}{parafuso}
  \end{Phonetics}
\end{Entry}

%%%%% EOF %%%%%

