%%%
%%% Radical "⾦"
%%%

\section*{Radical 167: ``⾦'' (钅、釒)}\addcontentsline{toc}{section}{Radical 167: ⾦、钅、釒}

\begin{Entry}{针}{7}{⾦}
  \begin{Phonetics}{针}{zhen1}[][HSK 4]
    \definition*{s.}{Sobrenome Zhen}
    \definition[根,枚,套,支]{s.}{agulha; ferramentas para costura de roupas | objetos semelhantes a agulhas; algo longo e fino como uma agulha | injeção | ponto de costura | pontos de acupuntura na medicina chinesa}
  \end{Phonetics}
\end{Entry}

\begin{Entry}{针对}{7,5}{⾦、⼨}
  \begin{Phonetics}{针对}{zhen1dui4}[][HSK 4]
    \definition{prep.}{em conexão com; de acordo com; à luz de; introdução de objetos de comportamento com uma finalidade clara}
    \definition{v.}{contrariar; apontar para; ter como objetivo; ser direcionado contra; fazer algo especificamente sobre um problema ou uma pessoa}
  \end{Phonetics}
\end{Entry}

\begin{Entry}{钉}{7}{⾦}
  \begin{Phonetics}{钉}{ding1}
    \definition[根,颗,枚]{s.}{prego; tacha}
    \definition{v.}{seguir de perto | instar; pressionar | olhar fixamente; fixar os olhos em; o mesmo que 盯}
  \seealsoref{盯}{ding1}
  \end{Phonetics}
  \begin{Phonetics}{钉}{ding4}[][HSK 7-9]
    \definition{v.}{pregar; prender com pregos | costurar; costurar com agulha e linha}
  \end{Phonetics}
\end{Entry}

\begin{Entry}{钉子}{7,3}{⾦、⼦}
  \begin{Phonetics}{钉子}{ding1zi5}[][HSK 7-9]
    \definition[颗,根,枚]{s.}{prego; prego de ferro | obstáculo; dificuldade; inconveniente | sabotador}
    \definition{v.}{bater em um obstáculo; encontrar um obstáculo; encontrar uma rejeição (de); correr contra um obstáculo}
  \end{Phonetics}
\end{Entry}

\begin{Entry}{金}{8}{⾦}[Kangxi 167]
  \begin{Phonetics}{金}{jin1}[][HSK 3]
    \definition*{s.}{Dinastia Jin (1115-1234) | Sobrenome Jin}
    \definition{adj.}{dourado | altamente respeitado; precioso. metáfora de nobreza}
    \definition[锭,块]{s.}{ouro | metal | dinheiro | instrumento antigo de percussão de metal}
  \end{Phonetics}
\end{Entry}

\begin{Entry}{金子}{8,3}{⾦、⼦}
  \begin{Phonetics}{金子}{jin1zi5}
    \definition{s.}{ouro; elemento metálico, símbolo Au (aurum) amarelo-avermelhado, macio, dúctil, quimicamente estável é um metal precioso, usado para fabricar dinheiro, ornamentos etc.}
  \end{Phonetics}
\end{Entry}

\begin{Entry}{金刚石}{8,6,5}{⾦、⼑、⽯}
  \begin{Phonetics}{金刚石}{jin1gang1shi2}
    \definition{s.}{diamante, também chamado de 钻石}[金刚石比什么金属都硬。===O diamante é mais duro que qualquer metal.]
  \seealsoref{钻石}{zuan4shi2}
  \end{Phonetics}
\end{Entry}

\begin{Entry}{金色}{8,6}{⾦、⾊}
  \begin{Phonetics}{金色}{jin1 se4}
    \definition{s.}{cor ouro; dourado}
  \end{Phonetics}
\end{Entry}

\begin{Entry}{金钱}{8,10}{⾦、⾦}
  \begin{Phonetics}{金钱}{jin1 qian2}[][HSK 6]
    \definition[沓,笔,堆]{s.}{dinheiro; moeda}
  \end{Phonetics}
\end{Entry}

\begin{Entry}{金牌}{8,12}{⾦、⽚}
  \begin{Phonetics}{金牌}{jin1 pai2}[][HSK 3]
    \definition[枚]{s.}{medalha de ouro; refere-se à medalha conquistada pelo campeão em uma competição esportiva | ficha de ouro; placa de ouro usada como símbolo}
  \end{Phonetics}
\end{Entry}

\begin{Entry}{金额}{8,15}{⾦、⾴}
  \begin{Phonetics}{金额}{jin1 e2}[][HSK 6]
    \definition[份,笔]{s.}{quantidade de dinheiro; soma de dinheiro}
  \end{Phonetics}
\end{Entry}

\begin{Entry}{金融}{8,16}{⾦、⿀}
  \begin{Phonetics}{金融}{jin1rong2}[][HSK 6]
    \definition{s.}{finanças; serviços bancários; refere-se a atividades econômicas como a emissão, circulação e retirada de moeda, a concessão e retirada de empréstimos, o depósito e retirada de depósitos e transações de câmbio}
  \end{Phonetics}
\end{Entry}

\begin{Entry}{钓}{8}{⾦}
  \begin{Phonetics}{钓}{diao4}
    \definition{v.}{pescar com anzol e linha | buscar (fama e ganho pessoal) | fisgar; defraudar por meios desleais}
  \end{Phonetics}
\end{Entry}

\begin{Entry}{钓鱼}{8,8}{⾦、⿂}
  \begin{Phonetics}{钓鱼}{diao4yu2}[][HSK 7-9]
    \definition{v.}{pescar; ir pescar; atividade de captura de peixes com equipamentos de pesca na beira da água, que é uma forma de lazer e entretenimento | Figurativo: aprisionar; Internet: 钓鱼 significa que alguém publica deliberadamente algo que pode causar controvérsia, raiva ou outras emoções fortes, a fim de atrair pessoas para responder e discutir}
  \end{Phonetics}
\end{Entry}

\begin{Entry}{钙}{9}{⾦}
  \begin{Phonetics}{钙}{gai4}[][HSK 7-9]
    \definition[克,毫克]{s.}{Ca, cálcio}
  \end{Phonetics}
\end{Entry}

\begin{Entry}{钝}{9}{⾦}
  \begin{Phonetics}{钝}{dun4}
    \definition{adj.}{sem corte; opaco (oposto a 快, 利, 锐) | estúpido; sem noção | maçante}
  \seealsoref{快}{kuai4}
  \seealsoref{利}{li4}
  \seealsoref{锐}{rui4}
  \end{Phonetics}
\end{Entry}

\begin{Entry}{钞}{9}{⾦}
  \begin{Phonetics}{钞}{chao1}
    \definition*{s.}{Sobrenome Chao}
    \definition[张,把,叠,摞]{s.}{cédula; nota de banco; papel-moeda; nota bancária | escritos coletados; texto transcrito}
    \definition{v.}{copiar; transcrever; o mesmo que 抄}
  \seealsoref{抄}{chao1}
  \end{Phonetics}
\end{Entry}

\begin{Entry}{钞票}{9,11}{⾦、⽰}
  \begin{Phonetics}{钞票}{chao1piao4}[][HSK 7-9]
    \definition[张,扎]{s.}{cédula; nota de banco; papel-moeda}
  \end{Phonetics}
\end{Entry}

\begin{Entry}{钟}{9}{⾦}
  \begin{Phonetics}{钟}{zhong1}[][HSK 3]
    \definition*{s.}{Sobrenome Zhong}
    \definition[顶,个,口]{s.}{sino; campainha; um instrumento de percussão antigo, oco, feito de cobre ou ferro | relógio; um aparelho para medir o tempo que não se leva consigo | tempo medido em horas e minutos; referindo-se ao tempo ou momento| um recipiente antigo para guardar vinho, com barriga grande e gargalo pequeno | sino; refere-se especificamente aos sinos pendurados em templos ou outros locais, cujo som é usado para marcar as horas, alertar ou convocar pessoas}
    \definition{v.}{focar; concentrar (as afeições de alguém, etc.)}
  \end{Phonetics}
\end{Entry}

\begin{Entry}{钟头}{9,5}{⾦、⼤}
  \begin{Phonetics}{钟头}{zhong1 tou2}[][HSK 6]
    \definition[个]{s.}{hora; 60 minutos}[三四个钟头过去了。===Três ou quatro horas se passaram.]
  \end{Phonetics}
\end{Entry}

\begin{Entry}{钟室}{9,9}{⾦、⼧}
  \begin{Phonetics}{钟室}{zhong1shi4}
    \definition{s.}{campanário | sala do relógio}
  \end{Phonetics}
\end{Entry}

\begin{Entry}{钟罩}{9,13}{⾦、⽹}
  \begin{Phonetics}{钟罩}{zhong1zhao4}
    \definition{s.}{redoma | dossel de sino}
  \end{Phonetics}
\end{Entry}

\begin{Entry}{钢}{9}{⾦}
  \begin{Phonetics}{钢}{gang1}[][HSK 7-9]
    \definition[吨,块,根]{s.}{aço; liga de ferro e carbono}
  \end{Phonetics}
\end{Entry}

\begin{Entry}{钢丝}{9,5}{⾦、⼀}
  \begin{Phonetics}{钢丝}{gang1si1}
    \definition{s.}{cabo de aço | corda bamba}
  \end{Phonetics}
\end{Entry}

\begin{Entry}{钢笔}{9,10}{⾦、⽵}
  \begin{Phonetics}{钢笔}{gang1 bi3}[][HSK 5]
    \definition[支,杆]{s.}{caneta-tinteiro; canetas com ponta metálica}
  \end{Phonetics}
\end{Entry}

\begin{Entry}{钢琴}{9,12}{⾦、⽟}
  \begin{Phonetics}{钢琴}{gang1qin2}[][HSK 5]
    \definition[架,台]{s.}{piano}
  \end{Phonetics}
\end{Entry}

\begin{Entry}{钥}{9}{⾦}
  \begin{Phonetics}{钥}{yao4}
    \definition{s.}{chave}
  \end{Phonetics}
\end{Entry}

\begin{Entry}{钥匙}{9,11}{⾦、⼔}
  \begin{Phonetics}{钥匙}{yao4shi5}
    \definition[把]{s.}{chave}
  \end{Phonetics}
\end{Entry}

\begin{Entry}{钥匙孔}{9,11,4}{⾦、⼔、⼦}
  \begin{Phonetics}{钥匙孔}{yao4shi5kong3}
    \definition{s.}{buraco da fechadura}
  \end{Phonetics}
\end{Entry}

\begin{Entry}{钥匙卡}{9,11,5}{⾦、⼔、⼘}
  \begin{Phonetics}{钥匙卡}{yao4shi5ka3}
    \definition{s.}{cartão de acesso}
  \end{Phonetics}
\end{Entry}

\begin{Entry}{钥匙洞孔}{9,11,9,4}{⾦、⼔、⽔、⼦}
  \begin{Phonetics}{钥匙洞孔}{yao4shi5dong4kong3}
    \definition{s.}{buraco da fechadura}
  \end{Phonetics}
\end{Entry}

\begin{Entry}{钥匙圈}{9,11,11}{⾦、⼔、⼞}
  \begin{Phonetics}{钥匙圈}{yao4shi5quan1}
    \definition{s.}{chaveiro}
  \end{Phonetics}
\end{Entry}

\begin{Entry}{钩}{9}{⾦}
  \begin{Phonetics}{钩}{gou1}[][HSK 7-9]
    \definition*{s.}{Sobrenome Gou}
    \definition[只,个]{s.}{gancho | traço de gancho em caracteres chineses | marca de verificação; visto; \emph{tick}; \emph{check mark} | marca em forma de gancho | uma espada em forma de gancho | forma falada do numeral 9 em certas ocasiões}
    \definition{v.}{prender com um gancho; enganchar | fazer crochê | costurar com pontos grandes | costurar com pontos longos}
  \end{Phonetics}
\end{Entry}

\begin{Entry}{钩子}{9,3}{⾦、⼦}
  \begin{Phonetics}{钩子}{gou1zi5}[][HSK 7-9]
    \definition[个]{s.}{gancho | coisa parecida com um gancho}
  \end{Phonetics}
\end{Entry}

\begin{Entry}{钱}{10}{⾦}
  \begin{Phonetics}{钱}{qian2}[][HSK 1]
    \definition*{s.}{Sobrenome Qian}
    \definition{clas.}{qian, uma unidade de peso (=5 gramas) | qian, uma unidade de peso (um décimo de um tael 两)}
    \definition[笔]{s.}{dinheiro; riqueza; bens | moeda de cobre; dinheiro | objeto em forma de moeda de cobre | fundo; montante | dinheiro guardado ou gasto para algum fim específico (geralmente se refere a quantias significativas de dinheiro que entram e saem de órgãos públicos, organizações, etc.)}
  \seealsoref{两}{liang3}
  \end{Phonetics}
\end{Entry}

\begin{Entry}{钱包}{10,5}{⾦、⼓}
  \begin{Phonetics}{钱包}{qian2 bao1}[][HSK 1]
    \definition[个]{s.}{carteira; bolsa; bolsa de dinheiro}
  \end{Phonetics}
\end{Entry}

\begin{Entry}{钻}{10}{⾦}
  \begin{Phonetics}{钻}{zuan1}
    \definition{v.}{furar; perfurar; girar um objeto pontiagudo para perfurar outro objeto | perfurar; entrar; penetrar; passar por | aprofundar-se; estudar intensivamente; fazer um estudo penetrante de | buscar ganho pessoal; tramar; refere-se a esquemas}
  \end{Phonetics}
  \begin{Phonetics}{钻}{zuan4}[][HSK 6]
    \definition[把]{s.}{broca; pua; sonda; existem muitos tipos de ferramentas para perfuração, incluindo manivela, elétrica e pneumática | joia; diamante}
    \definition{v.}{furar; perfurar;  girar um objeto pontiagudo para perfurar outro objeto}
  \end{Phonetics}
\end{Entry}

\begin{Entry}{钻石}{10,5}{⾦、⽯}
  \begin{Phonetics}{钻石}{zuan4shi2}
    \definition[颗]{s.}{diamante}
  \end{Phonetics}
\end{Entry}

\begin{Entry}{钻戒}{10,7}{⾦、⼽}
  \begin{Phonetics}{钻戒}{zuan4jie4}
    \definition[只]{s.}{anel de diamante}
  \end{Phonetics}
\end{Entry}

\begin{Entry}{钿}{10}{⾦}
  \begin{Phonetics}{钿}{dian4}
    \definition{s.}{ornamento incrustado antigo em forma de flor | enfeite de cabelo feminino com flores douradas | incrustação de madrepérola; um padrão incrustado com conchas de caracóis em madeira e laca}
    \definition{v.}{incrustar com ouro, prata, etc.}
  \end{Phonetics}
  \begin{Phonetics}{钿}{tian2}
    \definition{s.}{(dialeto) moeda | dinheiro; moeda | uma quantia de dinheiro}
  \end{Phonetics}
\end{Entry}

\begin{Entry}{铁}{10}{⾦}
  \begin{Phonetics}{铁}{tie3}[][HSK 3]
    \definition*{s.}{Sobrenome Tie}
    \definition{adj.}{duro; forte; sólido como ferro; metáfora para natureza dura; vontade forte | violento | inabalável; inalterável; determinado; metáfora para violência ou crueldade}
    \definition{s.}{ferro (Fe) | arma; armamento; refere-se a facas, armas de fogo, etc.}
    \definition{v.}{resolver; determinar}
  \end{Phonetics}
\end{Entry}

\begin{Entry}{铁轨}{10,6}{⾦、⾞}
  \begin{Phonetics}{铁轨}{tie3gui3}
    \definition[根]{s.}{trilho | trilho ferroviário}
  \end{Phonetics}
\end{Entry}

\begin{Entry}{铁路}{10,13}{⾦、⾜}
  \begin{Phonetics}{铁路}{tie3 lu4}[][HSK 3]
    \definition[条,公里]{s.}{ferrovia; estrada de ferro; uma estrada com trilhos de aço dispostos no leito da estrada para a circulação de trens}
  \end{Phonetics}
\end{Entry}

\begin{Entry}{铃}{10}{⾦}
  \begin{Phonetics}{铃}{ling2}[][HSK 5]
    \definition[串,个]{s.}{sino; instrumento musical feito de metal | objetos em forma de sino | cápsula; botão; broto}
  \end{Phonetics}
\end{Entry}

\begin{Entry}{铃声}{10,7}{⾦、⼠}
  \begin{Phonetics}{铃声}{ling2 sheng1}[][HSK 5]
    \definition{s.}{o tilintar de sinos; o som de um sino tocando}
  \end{Phonetics}
\end{Entry}

\begin{Entry}{铅}{10}{⾦}
  \begin{Phonetics}{铅}{qian1}
    \definition[根,盒]{s.}{chumbo (Pb) | grafite (em um lápis); grafite preta |}
  \end{Phonetics}
\end{Entry}

\begin{Entry}{铅笔}{10,10}{⾦、⽵}
  \begin{Phonetics}{铅笔}{qian1bi3}[][HSK 6]
    \definition[支,盒,种,枝,杆]{s.}{lápis; canetas com pontas de grafite ou argila pigmentada}
  \end{Phonetics}
\end{Entry}

\begin{Entry}{铜}{11}{⾦}
  \begin{Phonetics}{铜}{tong2}
    \definition[块]{s.}{cobre (Cu)}
  \end{Phonetics}
\end{Entry}

\begin{Entry}{铜牌}{11,12}{⾦、⽚}
  \begin{Phonetics}{铜牌}{tong2 pai2}[][HSK 6]
    \definition[枚]{s.}{medalha de bronze; o bronze | placa de bronze com nome ou logotipo comercial, etc.}
  \end{Phonetics}
\end{Entry}

\begin{Entry}{铲}{11}{⾦}
  \begin{Phonetics}{铲}{chan3}[][HSK 7-9]
    \definition[个,把]{s.}{pá}
    \definition{v.}{trabalhar com uma pá (ou enxada) | levantar (mover) com uma pá}
  \end{Phonetics}
\end{Entry}

\begin{Entry}{铲子}{11,3}{⾦、⼦}
  \begin{Phonetics}{铲子}{chan3zi5}[][HSK 7-9]
    \definition[把]{s.}{pá; uma ferramenta com uma chapa de ferro grossa, quase quadrada, em uma extremidade e um cabo longo na outra}
  \end{Phonetics}
\end{Entry}

\begin{Entry}{铲车}{11,4}{⾦、⾞}
  \begin{Phonetics}{铲车}{chan3che1}
    \definition[台]{s.}{empilhadeira}
  \end{Phonetics}
\end{Entry}

\begin{Entry}{银}{11}{⾦}
  \begin{Phonetics}{银}{yin2}[][HSK 3]
    \definition*{s.}{Sobrenome Yin}
    \definition{adj.}{prateado; como a cor da prata}
    \definition[锭]{s.}{Ag, prata | refere-se a moeda ou a coisas relacionadas com moeda}
  \end{Phonetics}
\end{Entry}

\begin{Entry}{银色}{11,6}{⾦、⾊}
  \begin{Phonetics}{银色}{yin2 se4}
    \definition{s.}{cor prata; prateado}
  \end{Phonetics}
\end{Entry}

\begin{Entry}{银行}{11,6}{⾦、⾏}
  \begin{Phonetics}{银行}{yin2hang2}[][HSK 2]
    \definition[个,家,所]{s.}{banco; instituições financeiras que operam depósitos, empréstimos, câmbio, poupança e outros negócios}
  \end{Phonetics}
\end{Entry}

\begin{Entry}{银行卡}{11,6,5}{⾦、⾏、⼘}
  \begin{Phonetics}{银行卡}{yin2 hang2 ka3}[][HSK 2]
    \definition{s.}{cartão bancário; cartão ATM}
  \end{Phonetics}
\end{Entry}

\begin{Entry}{银河}{11,8}{⾦、⽔}
  \begin{Phonetics}{银河}{yin2he2}
    \definition*{s.}{Via Láctea}
  \seealsoref{银河系}{yin2he2xi4}
  \end{Phonetics}
\end{Entry}

\begin{Entry}{银河系}{11,8,7}{⾦、⽔、⽷}
  \begin{Phonetics}{银河系}{yin2he2xi4}
    \definition*{s.}{Galáxia Via Láctea}
  \seealsoref{银河}{yin2he2}
  \end{Phonetics}
\end{Entry}

\begin{Entry}{银牌}{11,12}{⾦、⽚}
  \begin{Phonetics}{银牌}{yin2 pai2}[][HSK 3]
    \definition[枚]{s.}{medalha de prata; um tipo de medalha, concedida ao segundo colocado}
  \end{Phonetics}
\end{Entry}

\begin{Entry}{铺}{12}{⾦}
  \begin{Phonetics}{铺}{pu1}[][HSK 6]
    \definition{clas.}{usado para kang, etc.; kang, uma plataforma de alvenaria ou de barro em uma extremidade de um cômodo, aquecida no inverno por fogueiras embaixo e coberta com esteiras para dormir}
    \definition{v.}{espalhar; estender; desdobrar | colocar; pavimentar}
  \end{Phonetics}
  \begin{Phonetics}{铺}{pu4}
    \definition{s.}{pequena loja; depósito | uma cama feita de tábuas de madeira; geralmente se refere a uma cama | estação de correios; antiga estação de correios (usada principalmente em nomes de lugares)}
  \end{Phonetics}
\end{Entry}

\begin{Entry}{铺垫}{12,9}{⾦、⼟}
  \begin{Phonetics}{铺垫}{pu1dian4}
    \definition{s.}{cobre leito | colcha | roupa de cama}
    \definition{v.}{pavimentar}
  \end{Phonetics}
\end{Entry}

\begin{Entry}{销}{12}{⾦}
  \begin{Phonetics}{销}{xiao1}
    \definition*{s.}{Sobrenome Xiao}
    \definition{s.}{gasto; despesa | pino}
    \definition{v.}{derreter (metal) | cancelar; anular | vender; comercializar | aferrolhar; fixar; prender; pregar | fixar com um parafuso; parafusar | gastar (consumo) | inserir um pino}
  \end{Phonetics}
\end{Entry}

\begin{Entry}{销售}{12,11}{⾦、⼝}
  \begin{Phonetics}{销售}{xiao1shou4}[][HSK 4]
    \definition{v.}{vender; comercializar}
  \end{Phonetics}
\end{Entry}

\begin{Entry}{锁}{12}{⾦}
  \begin{Phonetics}{锁}{suo3}[][HSK 5]
    \definition[把]{s.}{fechadura; dispositivo que impede que as pessoas abram facilmente a parte que se abre e fecha | correntes; cadeado e correntes | qualquer coisa com a forma de um cadeado antigo}
    \definition{v.}{trancar; trancar com chave | costurar com ponto fixo | tricotar}
  \end{Phonetics}
\end{Entry}

\begin{Entry}{锅}{12}{⾦}
  \begin{Phonetics}{锅}{guo1}[][HSK 5]
    \definition[口,个,只]{s.}{panela; frigideira; utensílios de cozinha, redondos e côncavos, feitos principalmente de ferro, alumínio, etc. | parte que se parece com um pote em alguns objetos}
  \end{Phonetics}
\end{Entry}

\begin{Entry}{锐}{12}{⾦}
  \begin{Phonetics}{锐}{rui4}
    \definition*{s.}{Sobrenome Rui}
    \definition{adj.}{afiado; aguçado (oposto a 钝) | agudo; perspicaz | rápido; ágil; veloz}
    \definition{adv.}{rapidamente; de ​​repente}
    \definition{s.}{vigor; espírito de luta | armas afiadas}
  \seealsoref{钝}{dun4}
  \end{Phonetics}
\end{Entry}

\begin{Entry}{鉴}{13}{⾦}
  \begin{Phonetics}{鉴}{jian4}[][HSK 6]
    \definition*{s.}{Sobrenome Jian}
    \definition{expr.}{uma expressão idiomática antiga usada para escrever cartas, depois da saudação inicial para pedir que alguém leia a carta}
    \definition{s.}{espelho (feito de bronze ou latão); espelho de bronze antigo | advertência; lição objetiva}
    \definition{v.}{Literário: refletir; espelhar | inspecionar; examinar; escrutinar; olhar cuidadosamente}
  \end{Phonetics}
\end{Entry}

\begin{Entry}{鉴定}{13,8}{⾦、⼧}
  \begin{Phonetics}{鉴定}{jian4ding4}[][HSK 6]
    \definition{s.}{avaliação dos pontos fortes e fracos de uma pessoa; avaliação de pessoas ou coisas}
    \definition{v.}{avaliar; identificar; autenticar; determinar; identificar e determinar (a autenticidade e a qualidade das coisas) | conduzir uma avaliação; avaliar o desempenho de uma pessoa ao longo de um determinado período de tempo}
  \end{Phonetics}
\end{Entry}

\begin{Entry}{错}{13}{⾦}
  \begin{Phonetics}{错}{cuo4}[][HSK 1]
    \definition{adj.}{errado; equivocado; errôneo | (na negativa) nada ruim; muito bom | entrelaçado e recortado; intrincado; complexo | ruim; pobre; péssimo (usado apenas em negativas)}
    \definition{s.}{falha; demérito | erro; engano | (arcaico) pedra de amolar para polir jade}
    \definition{v.}{estar entrelaçado e serrilhado; ser intrincado | moer; esfregar | abrir caminho; sair do caminho | alternar; escalonar | estar fora de alinhamento | deslocar | evitar; fazer com que não se encontre ou não entre em conflito | polir; polir pedras preciosas | (literário) incrustar ou revestir com ouro, prata, etc. | interseccionar; cruzar; entrecruzar}
  \end{Phonetics}
\end{Entry}

\begin{Entry}{错过}{13,6}{⾦、⾡}
  \begin{Phonetics}{错过}{cuo4 guo4}[][HSK 6]
    \definition{v.}{perder (oportunidade); deixar escapar}
  \end{Phonetics}
\end{Entry}

\begin{Entry}{错位}{13,7}{⾦、⼈}
  \begin{Phonetics}{错位}{cuo4/wei4}[][HSK 7-9]
    \definition{s.}{deslocamento (por exemplo, de ossos quebrados) | julgamento errôneo | contato defeituoso | inversão (médica, por exemplo, parto prematuro) | fora de alinhamento}
    \definition{v.+compl.}{Medicina: deslocar | deslocar; inverter | extraviar}
  \end{Phonetics}
\end{Entry}

\begin{Entry}{错别字}{13,7,6}{⾦、⼑、⼦}
  \begin{Phonetics}{错别字}{cuo4bie2zi4}[][HSK 7-9]
    \definition{s.}{caracteres escritos incorretamente ou mal pronunciados; erros de digitação e ortografia}[文章里有一些错别字。===Há alguns erros de digitação no artigo.]
  \end{Phonetics}
\end{Entry}

\begin{Entry}{错觉}{13,9}{⾦、⾒}
  \begin{Phonetics}{错觉}{cuo4jue2}[][HSK 7-9]
    \definition{s.}{ilusão; concepção errônea; impressão errada; percepção incorreta de coisas objetivas devido a alguns motivos}
  \end{Phonetics}
\end{Entry}

\begin{Entry}{错误}{13,9}{⾦、⾔}
  \begin{Phonetics}{错误}{cuo4wu4}[][HSK 3]
    \definition{adj.}{equivocado; errado; errôneo; incorreto; não condizente com a realidade objetiva}
    \definition[个,次]{s.}{engano; erro; erro grosseiro; falha; coisas, comportamentos, etc. incorretos}
  \end{Phonetics}
\end{Entry}

\begin{Entry}{错综复杂}{13,11,9,6}{⾦、⽷、⼢、⽊}
  \begin{Phonetics}{错综复杂}{cuo4zong1-fu4za2}[][HSK 7-9]
    \definition{expr.}{desconcertante; complicado e confuso; complexo e misturado; muito complicado; intrincado; complexo}
  \end{Phonetics}
\end{Entry}

\begin{Entry}{锤}{13}{⾦}
  \begin{Phonetics}{锤}{chui2}
    \definition[把,个]{s.}{uma bola de metal com uma alça ou corrente, usada como arma; maça | algo como um martelo | martelo}
    \definition{v.}{martelar para dar forma; bater com um martelo}
  \end{Phonetics}
\end{Entry}

\begin{Entry}{锦}{13}{⾦}
  \begin{Phonetics}{锦}{jin3}
    \definition*{s.}{Sobrenome Jin}
    \definition{adj.}{brilhante e bonito (cores brilhantes e lindas)}
    \definition[块]{s.}{brocado; tecidos de seda com padrões coloridos}
  \end{Phonetics}
\end{Entry}

\begin{Entry}{锦上添花}{13,3,11,7}{⾦、⼀、⽔、⾋}
  \begin{Phonetics}{锦上添花}{jin3 shang4 tian1 hua1}
    \definition{expr.}{adicionar flores ao brocado --- tornar o que é bom ainda melhor; melhorar | dourando o lírio}
  \end{Phonetics}
\end{Entry}

\begin{Entry}{键}{13}{⾦}
  \begin{Phonetics}{键}{jian4}[][HSK 5]
    \definition[个]{s.}{chave | tecla (de uma máquina de escrever, piano, etc.) | Química: ligação | Literário: ferrolho (de uma porta) | pino (para máquinas)  | etapa crucial}
  \end{Phonetics}
\end{Entry}

\begin{Entry}{键盘}{13,11}{⾦、⽫}
  \begin{Phonetics}{键盘}{jian4pan2}[][HSK 5]
    \definition[台,个]{s.}{teclado; cravo; painel de teclas}
  \end{Phonetics}
\end{Entry}

\begin{Entry}{锺}{14}{⾦}
  \begin{Phonetics}{锺}{zhong1}
    \variantof{钟}
  \end{Phonetics}
\end{Entry}

\begin{Entry}{锻}{14}{⾦}
  \begin{Phonetics}{锻}{duan4}
    \definition{v.}{forjar; moldar}
  \end{Phonetics}
\end{Entry}

\begin{Entry}{锻炼}{14,9}{⾦、⽕}
  \begin{Phonetics}{锻炼}{duan4lian4}[][HSK 4]
    \definition{v.}{exercitar-se; fazer (ou fazer) exercícios; submeter-se a treinamento físico; fortalecer o corpo por meio do esporte | fortalecer; endurecer; aprimorar as habilidades de trabalho e de vida por meio de trabalho e outras atividades | forjar ou moldar metal para torná-lo mais refinado; refere-se à transformação de materiais metálicos em objetos de determinada forma e tamanho por meio de aquecimento, batimento, prensagem etc.}
  \end{Phonetics}
\end{Entry}

\begin{Entry}{镀}{14}{⾦}
  \begin{Phonetics}{镀}{du4}
    \definition{v.}{cobrir ou revestir (com um metal)}
  \end{Phonetics}
\end{Entry}

\begin{Entry}{镀金}{14,8}{⾦、⾦}
  \begin{Phonetics}{镀金}{du4jin1}
    \definition{v.}{banhar a ouro | dourar | (figurativo) fazer algo muito comum parecer especial}
  \end{Phonetics}
\end{Entry}

\begin{Entry}{镇}{15}{⾦}
  \begin{Phonetics}{镇}{zhen4}[][HSK 6]
    \definition{adj.}{inteiro; indica um período inteiro de tempo}
    \definition{adv.}{frequentemente; muitas vezes}
    \definition{s.}{posto de guarnição | cidade; divisão administrativa | centro comercial}
    \definition{v.}{suprimir; segurar; manter pressionado |  acalmar-se; recompor-se; estabilizar | guardar; guarnecer; fortalecer; usar a força para manter a estabilidade | resfriar com gelo; esfriar em água fria | acalmar; suprimir; dissuadir | suprimir pela força; sancionar}
  \end{Phonetics}
\end{Entry}

\begin{Entry}{镖}{16}{⾦}
  \begin{Phonetics}{镖}{biao1}
    \definition{s.}{dardo | arma de arremesso | mercadorias enviadas sob a proteção de uma escolta armada}
  \end{Phonetics}
\end{Entry}

\begin{Entry}{镜}{16}{⾦}
  \begin{Phonetics}{镜}{jing4}
    \definition*{s.}{Sobrenome Jing}
    \definition[面,副]{s.}{espelho | lente; vidro; dispositivos para auxiliar a visão ou conduzir experimentos ópticos}
    \definition{v.}{espelhar | perceber | usar como referência}
  \end{Phonetics}
\end{Entry}

\begin{Entry}{镜子}{16,3}{⾦、⼦}
  \begin{Phonetics}{镜子}{jing4zi5}[][HSK 4]
    \definition[面,个]{s.}{espelho; instrumento de reflexão de imagem liso e plano, antigamente esmerilhado a partir de um disco grosso de cobre fundido, atualmente feito de vidro plano revestido de prata ou alumínio | óculos; óculos de grau}
  \end{Phonetics}
\end{Entry}

\begin{Entry}{镜头}{16,5}{⾦、⼤}
  \begin{Phonetics}{镜头}{jing4tou2}[][HSK 4]
    \definition[个,组]{s.}{lente de câmera; objetiva; combinação de várias lentes, usada para formar uma imagem | foto; cena}
  \end{Phonetics}
\end{Entry}

%%%%% EOF %%%%%

