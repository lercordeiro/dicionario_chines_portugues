%%%
%%% Radical "⾈"
%%%

\section*{Radical 137: ``⾈''}\addcontentsline{toc}{section}{Radical 137: ⾈}

\begin{Entry}{航}{10}{⾈}
  \begin{Phonetics}{航}{hang2}
    \definition*{s.}{Sobrenome Hang}
    \definition[趟]{s.}{barco; navio}
    \definition{v.}{navegar (por água ou ar) | velejar}
  \end{Phonetics}
\end{Entry}

\begin{Entry}{航天员}{10,4,7}{⾈、⼤、⼝}
  \begin{Phonetics}{航天员}{hang2tian1yuan2}
    \definition{s.}{astronauta}
  \end{Phonetics}
\end{Entry}

\begin{Entry}{航空}{10,8}{⾈、⽳}
  \begin{Phonetics}{航空}{hang2kong1}[][HSK 4]
    \definition{s.}{viagem; aviação; refere-se ao voo de uma aeronave no ar}
  \end{Phonetics}
\end{Entry}

\begin{Entry}{航班}{10,10}{⾈、⽟}
  \begin{Phonetics}{航班}{hang2ban1}[][HSK 4]
    \definition[个,次]{s.}{número do voo; voo programado; o horário de um navio ou avião de passageiros}
  \end{Phonetics}
\end{Entry}

\begin{Entry}{般}{10}{⾈}
  \begin{Phonetics}{般}{ban1}
    \definition{clas.}{tipo; classe; gênero; amostra}
    \definition{part.}{(o mesmo) que; como; semelhante}
  \end{Phonetics}
  \begin{Phonetics}{般}{bo1}
    \definition{s.}{utilizado em 般若}
  \seealsoref{般若}{bo1re3}
  \end{Phonetics}
  \begin{Phonetics}{般}{pan2}
    \definition{adj.}{feliz; bem-aventurado}
  \end{Phonetics}
\end{Entry}

\begin{Entry}{般乐}{10,5}{⾈、⼃}
  \begin{Phonetics}{般乐}{pan2le4}
    \definition{v.}{jogar | divertir-se}
  \end{Phonetics}
\end{Entry}

\begin{Entry}{般若}{10,8}{⾈、⾋}
  \begin{Phonetics}{般若}{bo1re3}
    \definition*{s.}{Prajña (sânscrito), \emph{insight} sobre a verdadeira natureza da realidade}
    \definition{s.}{budismo: sabedoria}
  \end{Phonetics}
\end{Entry}

\begin{Entry}{舱}{10}{⾈}
  \begin{Phonetics}{舱}{cang1}[][HSK 7-9]
    \definition{s.}{cabine (de um avião ou navio) | módulo (de uma nave espacial) | espaço em um navio ou aeronave para transportar pessoas, carga ou máquinas}
  \end{Phonetics}
\end{Entry}

\begin{Entry}{船}{11}{⾈}
  \begin{Phonetics}{船}{chuan2}[][HSK 2]
    \definition*{s.}{Sobrenome Chuan}
    \definition[条,艘,叶,只]{s.}{barco; navio | embarcação; meio de transporte aquático, nome genérico para embarcações}
  \end{Phonetics}
\end{Entry}

\begin{Entry}{船长}{11,4}{⾈、⾧}
  \begin{Phonetics}{船长}{chuan2 zhang3}[][HSK 6]
    \definition{s.}{capitão do navio; mestre; marinheiro; comandante; o oficial chefe a bordo}
  \end{Phonetics}
\end{Entry}

\begin{Entry}{船只}{11,5}{⾈、⼝}
  \begin{Phonetics}{船只}{chuan2 zhi1}[][HSK 6]
    \definition[艘,条]{s.}{transporte marítimo; embarcação | navio; veleiro}
  \end{Phonetics}
\end{Entry}

\begin{Entry}{船员}{11,7}{⾈、⼝}
  \begin{Phonetics}{船员}{chuan2 yuan2}[][HSK 6]
    \definition[名,位,个]{s.}{tripulação (do navio) | membro da tripulação (do navio); marinheiro; marujo; barqueiro; velejador}
  \end{Phonetics}
\end{Entry}

\begin{Entry}{船桨}{11,10}{⾈、⽊}
  \begin{Phonetics}{船桨}{chuan2jiang3}[][HSK 7-9]
    \definition{s.}{remo}
  \end{Phonetics}
\end{Entry}

\begin{Entry}{船舶}{11,11}{⾈、⾈}
  \begin{Phonetics}{船舶}{chuan2bo2}[][HSK 7-9]
    \definition[艘,条,只]{s.}{transporte marítimo; barcos e navios; refere-se a vários navios}
  \end{Phonetics}
\end{Entry}

\begin{Entry}{艁}{13}{⾈}
  \begin{Phonetics}{艁}{zao4}
    \variantof{造}
  \end{Phonetics}
\end{Entry}

%%%%% EOF %%%%%

