%%%
%%% Radical "⾢"
%%%

\section*{Radical 163: ``⾢'' (⻏)}\addcontentsline{toc}{section}{Radical 163: ⾢、⻏}

\begin{entry}{那}{6}{⾢}
  \begin{phonetics}{那}{na1}
    \definition*{s.}{sobrenome Na}
  \end{phonetics}
  \begin{phonetics}{那}{na3}
    \variantof{哪}
  \end{phonetics}
  \begin{phonetics}{那}{na4}[][HSK 1,2]
    \definition{conj.}{nessa situação | nesse caso}
    \definition{pron.}{aquele | aquilo}
  \end{phonetics}
  \begin{phonetics}{那}{nuo2}
    \definition*{s.}{sobrenome Nuo}
  \end{phonetics}
\end{entry}

\begin{entry}{那儿}{6,2}{⾢、⼉}
  \begin{phonetics}{那儿}{na4r5}[][HSK 1]
    \definition{pron.}{lá | ali}
  \end{phonetics}
\end{entry}

\begin{entry}{那么}{6,3}{⾢、⼃}
  \begin{phonetics}{那么}{na4 me5}[][HSK 2]
    \definition{adv.}{então | como aquele | dessa maneira}
  \end{phonetics}
\end{entry}

\begin{entry}{那边}{6,5}{⾢、⾡}
  \begin{phonetics}{那边}{na4bian5}[][HSK 1]
    \definition{pron.}{ali | acolá}
  \end{phonetics}
\end{entry}

\begin{entry}{那会儿}{6,6,2}{⾢、⼈、⼉}
  \begin{phonetics}{那会儿}{na4 hui4r5}[][HSK 2]
    \definition{pron.}{então | naquela época}
  \end{phonetics}
\end{entry}

\begin{entry}{那时}{6,7}{⾢、⽇}
  \begin{phonetics}{那时}{na4 shi2}[][HSK 2]
    \definition{pron.}{então | naquela época | naqueles dias}
  \end{phonetics}
\end{entry}

\begin{entry}{那时候}{6,7,10}{⾢、⽇、⼈}
  \begin{phonetics}{那时候}{na4 shi2 hou5}[][HSK 2]
    \definition{adv.}{naquela hora}
  \end{phonetics}
\end{entry}

\begin{entry}{那里}{6,7}{⾢、⾥}
  \begin{phonetics}{那里}{na4 li3}[][HSK 1]
    \definition{pron.}{lá | ali}
  \end{phonetics}
\end{entry}

\begin{entry}{那些}{6,8}{⾢、⼆}
  \begin{phonetics}{那些}{na4xie1}[][HSK 1]
    \definition{pron.}{aqueles}
  \end{phonetics}
\end{entry}

\begin{entry}{那样}{6,10}{⾢、⽊}
  \begin{phonetics}{那样}{na4 yang4}[][HSK 2]
    \definition{pron.}{assim | tal | como esse | desse tipo}
  \end{phonetics}
\end{entry}

\begin{entry}{那麽}{6,14}{⾢、⿇}
  \begin{phonetics}{那麽}{na4 me5}
    \variantof{那么}
  \end{phonetics}
\end{entry}

\begin{entry}{邮包}{7,5}{⾢、⼓}
  \begin{phonetics}{邮包}{you2bao1}
    \definition{s.}{encomenda postal}
  \end{phonetics}
\end{entry}

\begin{entry}{邮市}{7,5}{⾢、⼱}
  \begin{phonetics}{邮市}{you2shi4}
    \definition{s.}{mercado postal}
  \end{phonetics}
\end{entry}

\begin{entry}{邮电}{7,5}{⾢、⽥}
  \begin{phonetics}{邮电}{you2dian4}
    \definition*{s.}{Correios e Telecomunicações}
  \end{phonetics}
\end{entry}

\begin{entry}{邮件}{7,6}{⾢、⼈}
  \begin{phonetics}{邮件}{you2 jian4}[][HSK 3]
    \definition[封,个]{s.}{correspondência; correio; assunto postal; um termo geral para cartas, encomendas, etc. recebidas, transportadas e entregues pelos correios | \emph{e-mail}; mensagens enviadas e recebidas por meio eletrônico}
  \end{phonetics}
\end{entry}

\begin{entry}{邮局}{7,7}{⾢、⼫}
  \begin{phonetics}{邮局}{you2ju2}[][HSK 4]
    \definition[家]{s.}{correio; agência dos correios; organizações que lidam com serviços postais}
  \end{phonetics}
\end{entry}

\begin{entry}{邮费}{7,9}{⾢、⾙}
  \begin{phonetics}{邮费}{you2fei4}
    \definition{s.}{postagem}
    \definition{v.}{postar}
  \end{phonetics}
\end{entry}

\begin{entry}{邮迷}{7,9}{⾢、⾡}
  \begin{phonetics}{邮迷}{you2mi2}
    \definition{s.}{filatelista | colecionador de selos}
  \end{phonetics}
\end{entry}

\begin{entry}{邮资}{7,10}{⾢、⾙}
  \begin{phonetics}{邮资}{you2zi1}
    \definition{s.}{postagem}
  \end{phonetics}
\end{entry}

\begin{entry}{邮递}{7,10}{⾢、⾡}
  \begin{phonetics}{邮递}{you2di4}
    \definition{v.}{enviar por correio}
  \end{phonetics}
\end{entry}

\begin{entry}{邮票}{7,11}{⾢、⽰}
  \begin{phonetics}{邮票}{you2 piao4}[][HSK 3]
    \definition[枚,张,套,版]{s.}{selo; selo postal; um \emph{voucher} vendido pelos correios e afixado na correspondência para indicar que a postagem foi paga}
  \end{phonetics}
\end{entry}

\begin{entry}{邮箱}{7,15}{⾢、⾋}
  \begin{phonetics}{邮箱}{you2 xiang1}[][HSK 3]
    \definition{s.}{caixa de correio | \emph{mailbox}; refere-se ao endereço de \emph{e-mail}}
  \end{phonetics}
\end{entry}

\begin{entry}{邻居}{7,8}{⾢、⼫}
  \begin{phonetics}{邻居}{lin2ju1}[][HSK 5]
    \definition[个,位,家]{s.}{vizinho; pessoas ou famílias que moram muito perto}
  \end{phonetics}
\end{entry}

\begin{entry}{郁郁葱葱}{8,8,12,12}{⾢、⾢、⾋、⾋}
  \begin{phonetics}{郁郁葱葱}{yu4yu4cong1cong1}
    \definition{expr.}{verdejante e exuberante}
  \end{phonetics}
\end{entry}

\begin{entry}{郊区}{8,4}{⾢、⼖}
  \begin{phonetics}{郊区}{jiao1qu1}[][HSK 5]
    \definition[个,片,块]{s.}{subúrbios; arredores; periferia; área ao redor da cidade que está administrativamente sob a jurisdição da cidade}
  \end{phonetics}
\end{entry}

\begin{entry}{部}{10}{⾢}
  \begin{phonetics}{部}{bu4}[][HSK 3]
    \definition{clas.}{para obras de literatura, filmes, máquinas etc.}
    \definition[根]{s.}{departamento | divisão | ministério | seção | parte | tropas}
  \end{phonetics}
\end{entry}

\begin{entry}{部下}{10,3}{⾢、⼀}
  \begin{phonetics}{部下}{bu4xia4}
    \definition{s.}{subordinado | tropas sob comando de alguém}
  \end{phonetics}
\end{entry}

\begin{entry}{部门}{10,3}{⾢、⾨}
  \begin{phonetics}{部门}{bu4men2}[][HSK 3]
    \definition[个]{s.}{filial | departamento | divisão | seção}
  \end{phonetics}
\end{entry}

\begin{entry}{部分}{10,4}{⾢、⼑}
  \begin{phonetics}{部分}{bu4fen5}[][HSK 2]
    \definition[个]{s.}{parte | parte de | uma parte de | pedaço | secção}
  \end{phonetics}
\end{entry}

\begin{entry}{部长}{10,4}{⾢、⾧}
  \begin{phonetics}{部长}{bu4 zhang3}[][HSK 3]
    \definition[个,位,名]{s.}{ministro | chefe de departamento | chefe de seção}
  \end{phonetics}
\end{entry}

\begin{entry}{部队}{10,4}{⾢、⾩}
  \begin{phonetics}{部队}{bu4dui4}
    \definition[个]{s.}{exército | forças armadas | tropas | unidades}
  \end{phonetics}
\end{entry}

\begin{entry}{部位}{10,7}{⾢、⼈}
  \begin{phonetics}{部位}{bu4wei4}[][HSK 5]
    \definition{s.}{lugar; posição (usado principalmente para o corpo humano)}
  \end{phonetics}
\end{entry}

\begin{entry}{部族}{10,11}{⾢、⽅}
  \begin{phonetics}{部族}{bu4zu2}
    \definition{adj.}{tribal}
    \definition{s.}{tribo}
  \end{phonetics}
\end{entry}

\begin{entry}{部属}{10,12}{⾢、⼫}
  \begin{phonetics}{部属}{bu4shu3}
    \definition{s.}{afiliado a um ministério | subordinado | tropas sob comando de alguém}
  \end{phonetics}
\end{entry}

\begin{entry}{部署}{10,13}{⾢、⽹}
  \begin{phonetics}{部署}{bu4shu3}
    \definition{s.}{implantação}
    \definition{v.}{implantar}
  \end{phonetics}
\end{entry}

\begin{entry}{都}{10}{⾢}
  \begin{phonetics}{都}{dou1}[][HSK 1]
    \definition{adv.}{todos | ambos | inteiramente | até | já (usado para dar ênfase) | (não) em tudo}
  \end{phonetics}
  \begin{phonetics}{都}{du1}
    \definition*{s.}{sobrenome Du}
    \definition{s.}{capital | metrópole}
  \end{phonetics}
\end{entry}

%%%%% EOF %%%%%

