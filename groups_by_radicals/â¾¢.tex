%%%
%%% Radical "⾢"
%%%

\section*{Radical 163: ``⾢'' (⻏)}\addcontentsline{toc}{section}{Radical 163: ⾢、⻏}

\begin{entry}{那}{6}{⾢}
  \begin{phonetics}{那}{na1}
    \definition*{s.}{Sobrenome Na}
  \end{phonetics}
  \begin{phonetics}{那}{na3}
    \definition{adv.}{expressa negação em perguntas retóricas}
    \definition{pron.}{qual? | qualquer que seja; qualquer que; para expressar incerteza em uma declaração | variante de 哪}
  \seealsoref{哪}{na3}
  \end{phonetics}
  \begin{phonetics}{那}{na4}[][HSK 1,2]
    \definition{conj.}{então; nessa situação; nesse caso; o mesmo que 那么}
    \definition{pron.}{aquele; aquilo; indica pessoas ou coisas distantes | aquele; aquilo; expressa muitas coisas, sem se referir especificamente a uma pessoa ou coisa, e é frequentemente usado em conjunto com 这}
  \seealsoref{那么}{na4 me5}
  \seealsoref{这}{zhe4}
  \end{phonetics}
  \begin{phonetics}{那}{ne4}
    \definition{conj.}{então; nesse caso; o mesmo que 那么}
    \definition{pron.}{aquele; aquilo; pronúncia coloquial de 那 (\dpy{na4})}
  \seealsoref{那么}{na4 me5}
  \end{phonetics}
  \begin{phonetics}{那}{nei4}
    \definition{conj.}{então; o mesmo que 那么}
    \definition{pron.}{aquele; aquilo; A pronúncia coloquial de 那 (\dpy{na4})}
  \seealsoref{那么}{na4 me5}
  \end{phonetics}
  \begin{phonetics}{那}{nuo2}
    \definition*{s.}{Sobrenome Nuo}
  \end{phonetics}
\end{entry}

\begin{entry}{那儿}{6,2}{⾢、⼉}
  \begin{phonetics}{那儿}{na4r5}[][HSK 1]
    \definition{pron.}{lá; ali; naquele lugar | então; naquela época (usado após 打, 从 e 由)}
  \seealsoref{从}{cong2}
  \seealsoref{打}{da3}
  \seealsoref{由}{you2}
  \end{phonetics}
\end{entry}

\begin{entry}{那个}{6,3}{⾢、⼈}
  \begin{phonetics}{那个}{na4ge5}
    \definition{pron.}{aquele | usado antes de verbos e adjetivos para indicar exagero | para substituir o discurso direto inconveniente}
  \end{phonetics}
\end{entry}

\begin{entry}{那么}{6,3}{⾢、⼃}
  \begin{phonetics}{那么}{na4 me5}[][HSK 2]
    \definition{conj.}{então; nesse caso; afirmar o resultado esperado ou fazer um julgamento}
    \definition{pron.}{assim; dessa maneira; indica a natureza, o estado, a forma, o grau, etc. | assim; sobre; colocado antes do numeral, indica uma estimativa}
  \end{phonetics}
\end{entry}

\begin{entry}{那边}{6,5}{⾢、⾡}
  \begin{phonetics}{那边}{na4 bian5}[][HSK 1]
    \definition{pron.}{ali; acolá; aquele lado}
  \end{phonetics}
\end{entry}

\begin{entry}{那会儿}{6,6,2}{⾢、⼈、⼉}
  \begin{phonetics}{那会儿}{na4 hui4r5}[][HSK 2]
    \definition{pron.}{então; naquela época; refere-se ao passado ou ao futuro}
  \end{phonetics}
\end{entry}

\begin{entry}{那时}{6,7}{⾢、⽇}
  \begin{phonetics}{那时}{na4 shi2}[][HSK 2]
    \definition{pron.}{então; naquela época; naqueles dias; geralmente se refere a um período de tempo distante do presente}
  \seealsoref{那时候}{na4 shi2 hou5}
  \end{phonetics}
\end{entry}

\begin{entry}{那时候}{6,7,10}{⾢、⽇、⼈}
  \begin{phonetics}{那时候}{na4 shi2 hou5}[][HSK 2]
    \definition{adv.}{naquela hora; em algum momento no passado}
  \seealsoref{那时}{na4 shi2}
  \end{phonetics}
\end{entry}

\begin{entry}{那里}{6,7}{⾢、⾥}
  \begin{phonetics}{那里}{na4 li3}[][HSK 1]
    \definition{pron./s.}{lá; ali; aquele lugar; indica um lugar distante}
  \end{phonetics}
\end{entry}

\begin{entry}{那些}{6,8}{⾢、⼆}
  \begin{phonetics}{那些}{na4 xie1}[][HSK 1]
    \definition{pron.}{aqueles; indica duas ou mais pessoas ou coisas}
  \end{phonetics}
\end{entry}

\begin{entry}{那咱}{6,9}{⾢、⼝}
  \begin{phonetics}{那咱}{na4 zan5}
    \definition{s.}{(informal) naquela época; então | (antigo) naquela época}
  \end{phonetics}
\end{entry}

\begin{entry}{那样}{6,10}{⾢、⽊}
  \begin{phonetics}{那样}{na4 yang4}[][HSK 2]
    \definition{pron.}{assim; tal; desse tipo; desse gênero; dessa natureza; desse tipo; indica a natureza, o estado, a maneira, o grau ou refere-se a uma ação ou situação específica}
  \end{phonetics}
\end{entry}

\begin{entry}{那麽}{6,14}{⾢、⿇}
  \begin{phonetics}{那麽}{na4 me5}
    \variantof{那么}
  \end{phonetics}
\end{entry}

\begin{entry}{邮}{7}{⾢}
  \begin{phonetics}{邮}{you2}
    \definition*{s.}{Sobrenome You}
    \definition{s.}{postal; correio; refere-se a serviços postais | agência dos correios}
    \definition{v.}{postar; enviar pelo correio}
  \end{phonetics}
\end{entry}

\begin{entry}{邮包}{7,5}{⾢、⼓}
  \begin{phonetics}{邮包}{you2bao1}
    \definition{s.}{encomenda postal}
  \end{phonetics}
\end{entry}

\begin{entry}{邮市}{7,5}{⾢、⼱}
  \begin{phonetics}{邮市}{you2shi4}
    \definition{s.}{mercado postal}
  \end{phonetics}
\end{entry}

\begin{entry}{邮电}{7,5}{⾢、⽥}
  \begin{phonetics}{邮电}{you2dian4}
    \definition*{s.}{Correios e Telecomunicações}
  \end{phonetics}
\end{entry}

\begin{entry}{邮件}{7,6}{⾢、⼈}
  \begin{phonetics}{邮件}{you2 jian4}[][HSK 3]
    \definition[封,份,个,条]{s.}{correspondência; correio; assunto postal; termo que se refere a cartas, encomendas, etc., recebidos, transportados e entregues pelos correios | \emph{e-mail}; refere-se a e-mails, informações recebidas e enviadas através de caixas de correio eletrônico na \emph{Internet}, etc.}
  \end{phonetics}
\end{entry}

\begin{entry}{邮局}{7,7}{⾢、⼫}
  \begin{phonetics}{邮局}{you2ju2}[][HSK 4]
    \definition[家]{s.}{correio; agência dos correios; organizações que lidam com serviços postais}
  \end{phonetics}
\end{entry}

\begin{entry}{邮费}{7,9}{⾢、⾙}
  \begin{phonetics}{邮费}{you2fei4}
    \definition{s.}{postagem}
    \definition{v.}{postar}
  \end{phonetics}
\end{entry}

\begin{entry}{邮迷}{7,9}{⾢、⾡}
  \begin{phonetics}{邮迷}{you2mi2}
    \definition{s.}{filatelista | colecionador de selos}
  \end{phonetics}
\end{entry}

\begin{entry}{邮资}{7,10}{⾢、⾙}
  \begin{phonetics}{邮资}{you2zi1}
    \definition{s.}{postagem}
  \end{phonetics}
\end{entry}

\begin{entry}{邮递}{7,10}{⾢、⾡}
  \begin{phonetics}{邮递}{you2di4}
    \definition{v.}{enviar por correio}
  \end{phonetics}
\end{entry}

\begin{entry}{邮票}{7,11}{⾢、⽰}
  \begin{phonetics}{邮票}{you2 piao4}[][HSK 3]
    \definition[枚,张,套,版]{s.}{selo; selo postal; comprovante vendido pelos correios, usado para colar nas correspondências para indicar que o porte foi pago}
  \end{phonetics}
\end{entry}

\begin{entry}{邮箱}{7,15}{⾢、⾋}
  \begin{phonetics}{邮箱}{you2 xiang1}[][HSK 3]
    \definition{s.}{caixa de correio | \emph{mailbox}; refere-se ao endereço de \emph{e-mail}}
  \end{phonetics}
\end{entry}

\begin{entry}{邻}{7}{⾢}
  \begin{phonetics}{邻}{lin2}
    \definition{adj.}{vizinho; perto; adjacente; perto; próximo}
    \definition{s.}{vizinho | bairro; vizinhança}
  \end{phonetics}
\end{entry}

\begin{entry}{邻居}{7,8}{⾢、⼫}
  \begin{phonetics}{邻居}{lin2ju1}[][HSK 5]
    \definition[个,位,家]{s.}{vizinho; pessoas ou famílias que moram muito perto}
  \end{phonetics}
\end{entry}

\begin{entry}{郁}{8}{⾢}
  \begin{phonetics}{郁}{yu4}
    \definition*{s.}{Sobrenome Yu}
    \definition{adj.}{fortemente perfumado | luxuriante; exuberante | sombrio; deprimido}
  \end{phonetics}
\end{entry}

\begin{entry}{郁郁葱葱}{8,8,12,12}{⾢、⾢、⾋、⾋}
  \begin{phonetics}{郁郁葱葱}{yu4yu4cong1cong1}
    \definition{expr.}{verdejante e exuberante}
  \end{phonetics}
\end{entry}

\begin{entry}{郊}{8}{⾢}
  \begin{phonetics}{郊}{jiao1}
    \definition*{s.}{Sobrenome Jiao}
    \definition{s.}{subúrbios; periferias; áreas ao redor da cidade}
  \end{phonetics}
\end{entry}

\begin{entry}{郊区}{8,4}{⾢、⼖}
  \begin{phonetics}{郊区}{jiao1 qu1}[][HSK 5]
    \definition[个,片,块]{s.}{subúrbios; arredores; periferia; área ao redor da cidade que está administrativamente sob a jurisdição da cidade}
  \end{phonetics}
\end{entry}

\begin{entry}{部}{10}{⾢}
  \begin{phonetics}{部}{bu4}[][HSK 3]
    \definition*{s.}{Sobrenome Bu}
    \definition{clas.}{usado para obras de literatura, livros, filmes, etc.}
    \definition[根]{s.}{parte; seção | unidade; ministério; departamento; conselho | sede; matriz; quartel general | tropas; forças | divisão; região}
    \definition{v.}{comandar; liderar}
  \end{phonetics}
\end{entry}

\begin{entry}{部下}{10,3}{⾢、⼀}
  \begin{phonetics}{部下}{bu4xia4}
    \definition{s.}{subordinado | tropas sob comando de alguém}
  \end{phonetics}
\end{entry}

\begin{entry}{部门}{10,3}{⾢、⾨}
  \begin{phonetics}{部门}{bu4men2}[][HSK 3]
    \definition[个]{s.}{departamento; ramo; classe; seção; partes ou unidades que compõem um todo}
  \end{phonetics}
\end{entry}

\begin{entry}{部分}{10,4}{⾢、⼑}
  \begin{phonetics}{部分}{bu4fen5}[][HSK 2]
    \definition[个,些,快,份]{s.}{parte; seção; porção; parte do todo; alguns indivíduos dentro do todo | ramo; parte separada de um sistema ou entidade}
  \end{phonetics}
\end{entry}

\begin{entry}{部长}{10,4}{⾢、⾧}
  \begin{phonetics}{部长}{bu4 zhang3}[][HSK 3]
    \definition[个,位,名]{s.}{ministro; chefe de departamento; um alto funcionário do estado encarregado pelo chefe de estado ou chefe executivo do governo da gestão das atividades governamentais de um departamento | chefe de seção; líder tribal}
  \end{phonetics}
\end{entry}

\begin{entry}{部队}{10,4}{⾢、⾩}
  \begin{phonetics}{部队}{bu4 dui4}[][HSK 6]
    \definition[支,个]{s.}{militar; exército; forças armadas | tropas; refere-se a uma parte do exército}
  \end{phonetics}
\end{entry}

\begin{entry}{部位}{10,7}{⾢、⼈}
  \begin{phonetics}{部位}{bu4wei4}[][HSK 5]
    \definition{s.}{lugar; posição (usado principalmente para o corpo humano)}
  \end{phonetics}
\end{entry}

\begin{entry}{部族}{10,11}{⾢、⽅}
  \begin{phonetics}{部族}{bu4zu2}
    \definition{adj.}{tribal}
    \definition{s.}{tribo}
  \end{phonetics}
\end{entry}

\begin{entry}{部属}{10,12}{⾢、⼫}
  \begin{phonetics}{部属}{bu4shu3}
    \definition{s.}{afiliado a um ministério | subordinado | tropas sob comando de alguém}
  \end{phonetics}
\end{entry}

\begin{entry}{部署}{10,13}{⾢、⽹}
  \begin{phonetics}{部署}{bu4shu3}
    \definition{s.}{implantação}
    \definition{v.}{implantar}
  \end{phonetics}
\end{entry}

\begin{entry}{都}{10}{⾢}
  \begin{phonetics}{都}{dou1}[][HSK 1]
    \definition{adv.}{todos; representa a soma total | apenas por causa de; usado em conjunto com a palavra 是, explica o motivo | mesmo; até; indicativo de ênfase | já; significa 已经}
  \seealsoref{是}{shi4}
  \seealsoref{已经}{yi3jing1}
  \end{phonetics}
  \begin{phonetics}{都}{du1}
    \definition*{s.}{Sobrenome Du}
    \definition[座]{s.}{capital | cidade grande; metrópole}
  \end{phonetics}
\end{entry}

\begin{entry}{都市}{10,5}{⾢、⼱}
  \begin{phonetics}{都市}{du1 shi4}[][HSK 6]
    \definition[个]{s.}{cidade grande; grandes cidades}
  \end{phonetics}
\end{entry}

%%%%% EOF %%%%%

