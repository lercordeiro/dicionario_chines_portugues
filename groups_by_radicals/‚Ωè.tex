%%%
%%% Radical "⽏"
%%%

\section*{Radical 80: ``⽏'' (母)}\addcontentsline{toc}{section}{Radical 80: ⽏、母}

\begin{Entry}{母}{5}{⽏}[Kangxi 80]
  \begin{Phonetics}{母}{mu3}[][HSK 6]
    \definition*{s.}{Sobrenome Mu}
    \definition{adj.}{fêmea}
    \definition[位,名,个,些]{s.}{mãe | fêmea (animal) (oposto a 公) | origem; pais | parentes idosas; geralmente se refere a mulheres idosas | côncavo | fonte; algo que tem a capacidade ou função de produzir outras coisas}
  \seealsoref{公}{gong1}
  \end{Phonetics}
\end{Entry}

\begin{Entry}{母女}{5,3}{⽏、⼥}
  \begin{Phonetics}{母女}{mu3 nv3}[][HSK 6]
    \definition{s.}{mãe e filha}
  \end{Phonetics}
\end{Entry}

\begin{Entry}{母子}{5,3}{⽏、⼦}
  \begin{Phonetics}{母子}{mu3 zi3}[][HSK 6]
    \definition{s.}{mãe e filho}
  \end{Phonetics}
\end{Entry}

\begin{Entry}{母鸡}{5,7}{⽏、⿃}
  \begin{Phonetics}{母鸡}{mu3ji1}[][HSK 6]
    \definition{s.}{galinha}
  \end{Phonetics}
\end{Entry}

\begin{Entry}{母亲}{5,9}{⽏、⼇}
  \begin{Phonetics}{母亲}{mu3qin1}[][HSK 3]
    \definition[位,名,个,些]{s.}{mãe}
  \end{Phonetics}
\end{Entry}

\begin{Entry}{母语}{5,9}{⽏、⾔}
  \begin{Phonetics}{母语}{mu3yu3}
    \definition{s.}{língua materna | língua nativa}
  \end{Phonetics}
\end{Entry}

\begin{Entry}{每}{7}{⽏}
  \begin{Phonetics}{每}{mei3}[][HSK 3]
    \definition{adv.}{cada um; cada qual; indica qualquer uma das repetições ou um conjunto de repetições de um movimento}
    \definition{pron.}{cada; cada um; cada qual; refere-se a qualquer indivíduo do grupo, enfatizando as semelhanças entre os indivíduos}
  \end{Phonetics}
\end{Entry}

\begin{Entry}{每个}{7,3}{⽏、⼈}
  \begin{Phonetics}{每个}{mei3ge4}
    \definition{pron.}{cada; cada um}
  \end{Phonetics}
\end{Entry}

\begin{Entry}{每个人}{7,3,2}{⽏、⼈、⼈}
  \begin{Phonetics}{每个人}{mei3ge5ren2}
    \definition{pron.}{todo mundo | todos}
  \end{Phonetics}
\end{Entry}

\begin{Entry}{每天}{7,4}{⽏、⼤}
  \begin{Phonetics}{每天}{mei3tian1}
    \definition{adv.}{todo dia | cada dia}
  \end{Phonetics}
\end{Entry}

\begin{Entry}{每次}{7,6}{⽏、⽋}
  \begin{Phonetics}{每次}{mei3ci4}
    \definition{adv.}{toda vez | cada vez}
  \end{Phonetics}
\end{Entry}

\begin{Entry}{毒}{9}{⽏}
  \begin{Phonetics}{毒}{du2}[][HSK 5]
    \definition*{s.}{Sobrenome Du}
    \definition{adj.}{veneno; toxina; propriedade ou substância prejudicial aos organismos vivos | droga; narcóticos | vírus; vírus de computador | influência venenosa}
    \definition{adj.}{venenoso; tóxico; envenenado | malicioso; cruel; feroz}
    \definition{v.}{matar com veneno; envenenar | envenenar (a mente de alguém)}
  \end{Phonetics}
\end{Entry}

\begin{Entry}{毒杀}{9,6}{⽏、⽊}
  \begin{Phonetics}{毒杀}{du2sha1}
    \definition{v.}{matar por envenenamento}
  \end{Phonetics}
\end{Entry}

\begin{Entry}{毒物}{9,8}{⽏、⽜}
  \begin{Phonetics}{毒物}{du2wu4}
    \definition{s.}{substância venenosa | toxina}
  \end{Phonetics}
\end{Entry}

\begin{Entry}{毒品}{9,9}{⽏、⼝}
  \begin{Phonetics}{毒品}{du2pin3}[][HSK 6]
    \definition[种,点]{s.}{drogas; veneno; narcóticos; refere-se ao ópio, morfina, heroína, etc. usados ​​como vício}
  \end{Phonetics}
\end{Entry}

\begin{Entry}{毒害}{9,10}{⽏、⼧}
  \begin{Phonetics}{毒害}{du2hai4}
    \definition{s.}{envenenamento}
    \definition{v.}{envenenar (prejudicar com uma substância tóxica) | envenenar (as mentes das pessoas)}
  \end{Phonetics}
\end{Entry}

\begin{Entry}{毒蛇}{9,11}{⽏、⾍}
  \begin{Phonetics}{毒蛇}{du2she2}
    \definition{s.}{víbora | cobra venenosa}
  \end{Phonetics}
\end{Entry}

%%%%% EOF %%%%%

