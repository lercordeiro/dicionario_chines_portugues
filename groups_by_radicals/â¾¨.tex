%%%
%%% Radical "⾨"
%%%

\section*{Radical 169: ``⾨'' (门)}\addcontentsline{toc}{section}{Radical 169: ⾨、门}

\begin{entry}{门}{3}{⾨}[Kangxi 169]
  \begin{phonetics}{门}{men2}[][HSK 1]
    \definition*{s.}{sobrenome Men}
    \definition{clas.}{para equipamentos de artilharia (por exemplo: canhões) | para trabalhos escolares, ciência e tecnologia, etc. | para idiomas | para casamentos | para parentes}
    \definition[个,把,道,扇]{s.}{entradas e saídas de edifícios, veículos, navios, aviões, etc. | válvula; interruptor; algo que funciona como um interruptor ou como uma porta | habilidade; método; acesso; maneira de fazer algo | família; ramo de uma família ou clã | seita (religiosa); escola (de pensamento); faculdades acadêmicas, ideológicas ou religiosas | classe; categoria; ramo de estudo; refere-se à categoria geral de coisas | filo; segundo nível da classificação biológica | (computador) \emph{gate}; porta (lógica) | porta; portão; entrada; refere-se a uma porta que pode ser aberta e fechada, instalada na entrada e saída | qualquer abertura; partes de objetos que podem ser abertas e fechadas | orifício no corpo humano; refere-se especificamente aos orifícios do corpo humano | estudar com o mesmo professor; refere-se especificamente ao professor ou mestre | posição em um jogo de apostas (em relação ao local onde se senta ou onde se faz uma aposta)}
  \end{phonetics}
\end{entry}

\begin{entry}{门口}{3,3}{⾨、⼝}
  \begin{phonetics}{门口}{men2 kou3}[][HSK 1]
    \definition[个]{s.}{porta; portão; entrada; porta de entrada}
  \end{phonetics}
\end{entry}

\begin{entry}{门诊}{3,7}{⾨、⾔}
  \begin{phonetics}{门诊}{men2 zhen3}[][HSK 5]
    \definition{s.}{(no hospital) clínica ambulatorial; seção para pacientes ambulatoriais; local onde os médicos atendem pacientes que não estão internados no hospital}
  \end{phonetics}
\end{entry}

\begin{entry}{门票}{3,11}{⾨、⽰}
  \begin{phonetics}{门票}{men2 piao4}[][HSK 1]
    \definition{s.}{bilhete de entrada; bilhete de admissão; ingressos para locais de turismo, entretenimento, etc.}
  \end{phonetics}
\end{entry}

\begin{entry}{闪}{5}{⾨}
  \begin{phonetics}{闪}{shan3}[][HSK 4]
    \definition*{s.}{sobrenome Shan}
    \definition{s.}{relâmpago}
    \definition{v.}{esquivar-se; desviar; sair do caminho | torcer; distender | surgir de repente | cintilar; brilhar | deixar para trás; abandonar | (corpo) oscilar dramaticamente}
  \end{phonetics}
\end{entry}

\begin{entry}{闪电}{5,5}{⾨、⽥}
  \begin{phonetics}{闪电}{shan3dian4}[][HSK 4]
    \definition[道]{s.}{relâmpago; descargas elétricas entre nuvens ou entre nuvens e o solo}
  \seealsoref{雷电}{lei2dian4}
  \end{phonetics}
\end{entry}

\begin{entry}{闪存盘}{5,6,11}{⾨、⼦、⽫}
  \begin{phonetics}{闪存盘}{shan3cun2pan2}
    \definition{s.}{unidade de memória \emph{USB} | cartão de memória}
  \seealsoref{优盘}{you1pan2}
  \end{phonetics}
\end{entry}

\begin{entry}{闭幕}{6,13}{⾨、⼱}
  \begin{phonetics}{闭幕}{bi4 mu4}[][HSK 5]
    \definition{v.+compl.}{fechar; concluir; (conferência, exposição, etc.) terminar | cair a cortina; abaixar a cortina; terminar a apresentação e a cortina se fechar em frente ao palco}
  \end{phonetics}
\end{entry}

\begin{entry}{闭幕式}{6,13,6}{⾨、⼱、⼷}
  \begin{phonetics}{闭幕式}{bi4 mu4 shi4}[][HSK 5]
    \definition{s.}{cerimônia de encerramento; cerimônia formal realizada no final de uma conferência ou exposição}
  \end{phonetics}
\end{entry}

\begin{entry}{闭嘴}{6,16}{⾨、⼝}
  \begin{phonetics}{闭嘴}{bi4zui3}
    \definition{expr.}{Cale-se!}
  \end{phonetics}
\end{entry}

\begin{entry}{问}{6}{⾨}
  \begin{phonetics}{问}{wen4}[][HSK 1]
    \definition*{s.}{sobrenome Wen}
    \definition{prep.}{de; introduzir o objeto da ação, equivalente a 向 e 跟}
    \definition{v.}{perguntar; indagar; fazer com que as pessoas respondam ou esclareçam coisas que não sabem ou não têm certeza | perguntar (ou indagar) sobre | examinar; interrogar | intervir; responsabilizar; investigar | cuidar; preocupar-se; gerenciar; interferir}
  \seealsoref{跟}{gen1}
  \seealsoref{向}{xiang4}
  \end{phonetics}
\end{entry}

\begin{entry}{问市}{6,5}{⾨、⼱}
  \begin{phonetics}{问市}{wen4shi4}
    \definition{v.}{chegar ao mercado | bater o mercado | atingir o mercado}
  \end{phonetics}
\end{entry}

\begin{entry}{问安}{6,6}{⾨、⼧}
  \begin{phonetics}{问安}{wen4'an1}
    \definition{s.}{saudações}
    \definition{v.}{dar cumprimentos a | prestar homenagem}
  \end{phonetics}
\end{entry}

\begin{entry}{问卷}{6,8}{⾨、⼙}
  \begin{phonetics}{问卷}{wen4juan4}
    \definition[份]{s.}{questionário}
  \end{phonetics}
\end{entry}

\begin{entry}{问候}{6,10}{⾨、⼈}
  \begin{phonetics}{问候}{wen4hou4}[][HSK 4]
    \definition{s.}{homenagem | saudação}
    \definition{v.}{prestar homenagem; enviar uma saudação;  dar os respeitos (cumprimentos) a alguém | (fig.) (coloquial) fazer referência ofensiva a (alguém querido pela pessoa com quem se está falando)}
  \end{phonetics}
\end{entry}

\begin{entry}{问鼎}{6,12}{⾨、⿍}
  \begin{phonetics}{问鼎}{wen4ding3}
    \definition{v.}{visar (o primeiro lugar, etc.) | aspirar ao trono}
  \end{phonetics}
\end{entry}

\begin{entry}{问路}{6,13}{⾨、⾜}
  \begin{phonetics}{问路}{wen4 lu4}[][HSK 2]
    \definition{v.}{perguntar o caminho; pedir direções}
  \end{phonetics}
\end{entry}

\begin{entry}{问题}{6,15}{⾨、⾴}
  \begin{phonetics}{问题}{wen4ti2}[][HSK 2]
    \definition{adj.}{desqualificado; indesejável; anormal, não atende aos requisitos}
    \definition[个,种,类,串]{s.}{pergunta; problema; perguntas a serem respondidas | problema; questão; contradições que precisam ser estudadas e resolvidas | problema; acidente; incidente | chave; ponto crucial; pontos importantes}
  \end{phonetics}
\end{entry}

\begin{entry}{闯}{6}{⾨}
  \begin{phonetics}{闯}{chuang3}[][HSK 5]
    \definition{v.}{apressar; correr; carregar | moderar a si mesmo (lutando contra dificuldades e perigos)}
  \end{phonetics}
\end{entry}

\begin{entry}{闲}{7}{⾨}
  \begin{phonetics}{闲}{xian2}[][HSK 5]
    \definition{adj.}{ocioso; não ocupado; desocupado; sem coisas para fazer; sem atividades; tempo livre | desocupado; (casa, objeto, etc.) não em uso; ocioso | não oficial; não sério; não relacionado ao negócio}
    \definition{s.}{lazer; tempo livre}
  \end{phonetics}
\end{entry}

\begin{entry}{间}{7}{⾨}
  \begin{phonetics}{间}{jian1}[][HSK 1]
    \definition{clas.}{a menor unidade de uma casa; a menor unidade habitacional; cômodo}
    \definition{s.}{espaço entre duas partes  | (em um) tempo ou espaço definido | sala; quarto | uma seção de uma sala ou o espaço lateral entre dois pares de pilares | com um tempo ou espaço definido}
  \end{phonetics}
  \begin{phonetics}{间}{jian4}
    \definition{s.}{espaço entre as duas partes; abertura; lacuna}
    \definition{v.}{separar | semear a discórdia | desbastar (mudas); podar; remover ou arrancar as mudas em excesso}
  \end{phonetics}
\end{entry}

\begin{entry}{间或}{7,8}{⾨、⼽}
  \begin{phonetics}{间或}{jian4huo4}
    \definition{adv.}{às vezes | ocasionalmente | de vez em quando}
  \end{phonetics}
\end{entry}

\begin{entry}{间接}{7,11}{⾨、⼿}
  \begin{phonetics}{间接}{jian4jie1}[][HSK 5]
    \definition{adj.}{indireto; de segunda mão; em oposição a 直接}
  \seealsoref{直接}{zhi2jie1}
  \end{phonetics}
\end{entry}

\begin{entry}{闷热}{7,10}{⾨、⽕}
  \begin{phonetics}{闷热}{men1re4}
    \definition{adj.}{abafado | quente e abafado | sufocantemente quente | quente e sensual}
  \end{phonetics}
\end{entry}

\begin{entry}{闸门}{8,3}{⾨、⾨}
  \begin{phonetics}{闸门}{zha2men2}
    \definition{s.}{eclusa | comporta}
  \end{phonetics}
\end{entry}

\begin{entry}{闻}{9}{⾨}
  \begin{phonetics}{闻}{wen2}[][HSK 2]
    \definition*{s.}{sobrenome Wen}
    \definition{adj.}{bem conhecido; famoso}
    \definition{s.}{notícia; história | reputação | boato; rumor}
    \definition{v.}{cheirar | ouvir}
  \end{phonetics}
\end{entry}

\begin{entry}{阁下}{9,3}{⾨、⼀}
  \begin{phonetics}{阁下}{ge2xia4}
    \definition{pron.}{Sua Excelência | Sua Majestade | \emph{Sire}}
  \end{phonetics}
\end{entry}

\begin{entry}{阅兵式}{10,7,6}{⾨、⼋、⼷}
  \begin{phonetics}{阅兵式}{yue4bing1shi4}
    \definition{s.}{parada militar}
  \end{phonetics}
\end{entry}

\begin{entry}{阅览室}{10,9,9}{⾨、⾒、⼧}
  \begin{phonetics}{阅览室}{yue4 lan3 shi4}[][HSK 5]
    \definition[间]{s.}{sala de leitura; a biblioteca dispõe de salas para leitura e pesquisa, equipadas com mesas e cadeiras adequadas, livros, jornais, revistas, etc.}
  \end{phonetics}
\end{entry}

\begin{entry}{阅读}{10,10}{⾨、⾔}
  \begin{phonetics}{阅读}{yue4du2}[][HSK 4]
    \definition{s.}{leitura}
    \definition{v.}{ler; examinar; olhar (livros, jornais, etc.) e entender seu conteúdo}
  \end{phonetics}
\end{entry}

\begin{entry}{阅读广度}{10,10,3,9}{⾨、⾔、⼴、⼴}
  \begin{phonetics}{阅读广度}{yue4du2guang3du4}
    \definition{s.}{intervalo de leitura}
  \end{phonetics}
\end{entry}

\begin{entry}{阅读时间}{10,10,7,7}{⾨、⾔、⽇、⾨}
  \begin{phonetics}{阅读时间}{yue4du2shi2jian1}
    \definition{s.}{tempo de leitura}
  \end{phonetics}
\end{entry}

\begin{entry}{阅读理解}{10,10,11,13}{⾨、⾔、⽟、⾓}
  \begin{phonetics}{阅读理解}{yue4du2li3jie3}
    \definition{s.}{compreensão de leitura}
  \end{phonetics}
\end{entry}

\begin{entry}{阅读装置}{10,10,12,13}{⾨、⾔、⾐、⽹}
  \begin{phonetics}{阅读装置}{yue4du2zhuang1zhi4}
    \definition{s.}{dispositivo de leitura (por exemplo, para códigos de barras, etiquetas RFID, etc.)}
  \end{phonetics}
\end{entry}

\begin{entry}{阅读障碍}{10,10,13,13}{⾨、⾔、⾩、⽯}
  \begin{phonetics}{阅读障碍}{yue4du2zhang4ai4}
    \definition{s.}{dislexia}
  \end{phonetics}
\end{entry}

\begin{entry}{阅读器}{10,10,16}{⾨、⾔、⼝}
  \begin{phonetics}{阅读器}{yue4du2qi4}
    \definition{s.}{leitor (\emph{software})}
  \end{phonetics}
\end{entry}

%%%%% EOF %%%%%

