%%%
%%% Radical "⾨"
%%%

\section*{Radical 169: ``⾨'' (门)}\addcontentsline{toc}{section}{Radical 169: ⾨、门}

\begin{entry}{门}{3}{⾨}[Kangxi 169]
  \begin{phonetics}{门}{men2}[][HSK 1]
    \definition*{s.}{sobrenome Men}
    \definition{clas.}{para canhão | para lição de casa, tecnologia, etc.}
    \definition{s.}{porta | portão | entrada; saída | interruptor | válvula |maneira | método | acesso | família | casa | escola (de pensamento) | seita (religiosa) | ramo de estudo | categoria; classe | filo}
  \end{phonetics}
\end{entry}

\begin{entry}{门口}{3,3}{⾨、⼝}
  \begin{phonetics}{门口}{men2kou3}[][HSK 1]
    \definition[个]{s.}{porta | portão}
  \end{phonetics}
\end{entry}

\begin{entry}{门票}{3,11}{⾨、⽰}
  \begin{phonetics}{门票}{men2piao4}[][HSK 1]
    \definition{s.}{bilhete de entrada | bilhete de admissão}
  \end{phonetics}
\end{entry}

\begin{entry}{闪}{5}{⾨}
  \begin{phonetics}{闪}{shan3}[][HSK 4]
    \definition*{s.}{sobrenome Shan}
    \definition{s.}{relâmpago}
    \definition{v.}{esquivar-se; desviar; sair do caminho | torcer; distender | surgir de repente | cintilar; brilhar | deixar para trás; abandonar | (corpo) oscilar dramaticamente}
  \end{phonetics}
\end{entry}

\begin{entry}{闪电}{5,5}{⾨、⽥}
  \begin{phonetics}{闪电}{shan3dian4}[][HSK 4]
    \definition[道]{s.}{relâmpago; descargas elétricas entre nuvens ou entre nuvens e o solo}
  \seealsoref{雷电}{lei2dian4}
  \end{phonetics}
\end{entry}

\begin{entry}{闪存盘}{5,6,11}{⾨、⼦、⽫}
  \begin{phonetics}{闪存盘}{shan3cun2pan2}
    \definition{s.}{unidade de memória \emph{USB} | cartão de memória}
  \seealsoref{优盘}{you1pan2}
  \end{phonetics}
\end{entry}

\begin{entry}{闭嘴}{6,16}{⾨、⼝}
  \begin{phonetics}{闭嘴}{bi4zui3}
    \definition{expr.}{Cale-se!}
  \end{phonetics}
\end{entry}

\begin{entry}{问}{6}{⾨}
  \begin{phonetics}{问}{wen4}[][HSK 1]
    \definition{v.}{perguntar}
  \end{phonetics}
\end{entry}

\begin{entry}{问市}{6,5}{⾨、⼱}
  \begin{phonetics}{问市}{wen4shi4}
    \definition{v.}{chegar ao mercado | bater o mercado | atingir o mercado}
  \end{phonetics}
\end{entry}

\begin{entry}{问安}{6,6}{⾨、⼧}
  \begin{phonetics}{问安}{wen4'an1}
    \definition{s.}{saudações}
    \definition{v.}{dar cumprimentos a | prestar homenagem}
  \end{phonetics}
\end{entry}

\begin{entry}{问卷}{6,8}{⾨、⼙}
  \begin{phonetics}{问卷}{wen4juan4}
    \definition[份]{s.}{questionário}
  \end{phonetics}
\end{entry}

\begin{entry}{问候}{6,10}{⾨、⼈}
  \begin{phonetics}{问候}{wen4hou4}[][HSK 4]
    \definition{s.}{homenagem | saudação}
    \definition{v.}{prestar homenagem; enviar uma saudação;  dar os respeitos (cumprimentos) a alguém | (fig.) (coloquial) fazer referência ofensiva a (alguém querido pela pessoa com quem se está falando)}
  \end{phonetics}
\end{entry}

\begin{entry}{问鼎}{6,12}{⾨、⿍}
  \begin{phonetics}{问鼎}{wen4ding3}
    \definition{v.}{visar (o primeiro lugar, etc.) | aspirar ao trono}
  \end{phonetics}
\end{entry}

\begin{entry}{问路}{6,13}{⾨、⾜}
  \begin{phonetics}{问路}{wen4 lu4}[][HSK 2]
    \definition{v.}{perguntar sobre o caminho | pedir por direções}
  \end{phonetics}
\end{entry}

\begin{entry}{问题}{6,15}{⾨、⾴}
  \begin{phonetics}{问题}{wen4ti2}[][HSK 2]
    \definition[个]{s.}{pergunta | questão | problema}
  \end{phonetics}
\end{entry}

\begin{entry}{间}{7}{⾨}
  \begin{phonetics}{间}{jian1}
    \definition{adv.}{entre | dentro de um tempo ou espaço definidos}
    \definition{clas.}{para salas}
    \definition{s.}{sala | seção de uma sala ou espaço lateral entre dois pares de pilares}
  \end{phonetics}
  \begin{phonetics}{间}{jian4}[][HSK 1]
    \definition{s.}{lacuna}
    \definition{v.}{separar | podar (mudas) | semear descontentamento}
  \end{phonetics}
\end{entry}

\begin{entry}{间或}{7,8}{⾨、⼽}
  \begin{phonetics}{间或}{jian4huo4}
    \definition{adv.}{às vezes | ocasionalmente | de vez em quando}
  \end{phonetics}
\end{entry}

\begin{entry}{间接}{7,11}{⾨、⼿}
  \begin{phonetics}{间接}{jian4jie1}
    \definition{adj.}{indireto}
  \seealsoref{直接}{zhi2jie1}
  \end{phonetics}
\end{entry}

\begin{entry}{闷热}{7,10}{⾨、⽕}
  \begin{phonetics}{闷热}{men1re4}
    \definition{adj.}{abafado | quente e abafado | sufocantemente quente | quente e sensual}
  \end{phonetics}
\end{entry}

\begin{entry}{闸门}{8,3}{⾨、⾨}
  \begin{phonetics}{闸门}{zha2men2}
    \definition{s.}{eclusa | comporta}
  \end{phonetics}
\end{entry}

\begin{entry}{闻}{9}{⾨}
  \begin{phonetics}{闻}{wen2}[][HSK 2]
    \definition*{s.}{sobrenome Wen}
    \definition{s.}{notícias | reputação | fama}
    \definition{v.}{ouvir | cheirar | farejar}
  \end{phonetics}
\end{entry}

\begin{entry}{阁下}{9,3}{⾨、⼀}
  \begin{phonetics}{阁下}{ge2xia4}
    \definition{pron.}{Sua Excelência | Sua Majestade | \emph{Sire}}
  \end{phonetics}
\end{entry}

\begin{entry}{阅兵式}{10,7,6}{⾨、⼋、⼷}
  \begin{phonetics}{阅兵式}{yue4bing1shi4}
    \definition{s.}{parada militar}
  \end{phonetics}
\end{entry}

\begin{entry}{阅览室}{10,9,9}{⾨、⾒、⼧}
  \begin{phonetics}{阅览室}{yue4lan3shi4}
    \definition[间]{s.}{sala de leitura}
  \end{phonetics}
\end{entry}

\begin{entry}{阅读}{10,10}{⾨、⾔}
  \begin{phonetics}{阅读}{yue4du2}[][HSK 4]
    \definition{s.}{leitura}
    \definition{v.}{ler; examinar; olhar (livros, jornais, etc.) e entender seu conteúdo}
  \end{phonetics}
\end{entry}

\begin{entry}{阅读广度}{10,10,3,9}{⾨、⾔、⼴、⼴}
  \begin{phonetics}{阅读广度}{yue4du2guang3du4}
    \definition{s.}{intervalo de leitura}
  \end{phonetics}
\end{entry}

\begin{entry}{阅读时间}{10,10,7,7}{⾨、⾔、⽇、⾨}
  \begin{phonetics}{阅读时间}{yue4du2shi2jian1}
    \definition{s.}{tempo de leitura}
  \end{phonetics}
\end{entry}

\begin{entry}{阅读理解}{10,10,11,13}{⾨、⾔、⽟、⾓}
  \begin{phonetics}{阅读理解}{yue4du2li3jie3}
    \definition{s.}{compreensão de leitura}
  \end{phonetics}
\end{entry}

\begin{entry}{阅读装置}{10,10,12,13}{⾨、⾔、⾐、⽹}
  \begin{phonetics}{阅读装置}{yue4du2zhuang1zhi4}
    \definition{s.}{dispositivo de leitura (por exemplo, para códigos de barras, etiquetas RFID, etc.)}
  \end{phonetics}
\end{entry}

\begin{entry}{阅读障碍}{10,10,13,13}{⾨、⾔、⾩、⽯}
  \begin{phonetics}{阅读障碍}{yue4du2zhang4ai4}
    \definition{s.}{dislexia}
  \end{phonetics}
\end{entry}

\begin{entry}{阅读器}{10,10,16}{⾨、⾔、⼝}
  \begin{phonetics}{阅读器}{yue4du2qi4}
    \definition{s.}{leitor (\emph{software})}
  \end{phonetics}
\end{entry}

%%%%% EOF %%%%%

