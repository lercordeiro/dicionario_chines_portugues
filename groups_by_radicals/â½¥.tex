%%%
%%% Radical "⽥"
%%%

\section*{Radical 102: ``⽥''}\addcontentsline{toc}{section}{Radical 102: ⽥}

\begin{Entry}{田}{5}{⽥}[Kangxi 102]
  \begin{Phonetics}{田}{tian2}[][HSK 6]
    \definition*{s.}{Sobrenome Tian}
    \definition[亩,块,片]{s.}{campo; terra; terra de cultivo | área aberta rica em algum produto natural; campo}
    \definition{v.}{(arcaico) caçar}
  \end{Phonetics}
\end{Entry}

\begin{Entry}{田园}{5,7}{⽥、⼞}
  \begin{Phonetics}{田园}{tian2yuan2}
    \definition{adj.}{bucólico}
    \definition{s.}{campo | interior | rural}
  \end{Phonetics}
\end{Entry}

\begin{Entry}{田径}{5,8}{⽥、⼻}
  \begin{Phonetics}{田径}{tian2jing4}[][HSK 6]
    \definition{s.}{Esporte: atletismo}[他参加了这次的田径赛。===Ele participou da competição de atletismo.]
  \end{Phonetics}
\end{Entry}

\begin{Entry}{由}{5}{⽥}
  \begin{Phonetics}{由}{you2}[][HSK 3]
    \definition*{s.}{Sobrenome You}
    \definition{prep.}{por causa de; devido a | por; indica que algo deve ser feito por alguém | indica confiança em; indica dependência em | de; indica o ponto de partida | por; através de}
    \definition[个]{s.}{causa; razão; motivo}
    \definition{v.}{atravessar; passar por; seguir o caminho de | obedecer; seguir}
  \end{Phonetics}
\end{Entry}

\begin{Entry}{由于}{5,3}{⽥、⼆}
  \begin{Phonetics}{由于}{you2yu2}[][HSK 3]
    \definition{conj.}{porque; uma vez que; visto que;  usado no início da frase anterior, indica a razão, e a frase seguinte indica o resultado}
    \definition{prep.}{devido a; graças a; por causa de; em virtude de; como resultado de; introduzir a causa da ocorrência de eventos, ações, etc.}
  \end{Phonetics}
\end{Entry}

\begin{Entry}{由此}{5,6}{⽥、⽌}
  \begin{Phonetics}{由此}{you2 ci3}[][HSK 5]
    \definition{adv.}{assim; por meio disto; disto; daí; por causa disto; portanto; daqui; de agora em diante}
  \end{Phonetics}
\end{Entry}

\begin{Entry}{甲}{5}{⽥}
  \begin{Phonetics}{甲}{jia3}[][HSK 5]
    \definition*{s.}{Sobrenome Jia}
    \definition{s.}{alfa; primeiro lugar; o primeiro dos caules celestiais, geralmente usado para indicar o primeiro em ordem ou classificação | concha; carapaça; crustáceos | unha; crostas queratinosas nos dedos das mãos e dos pés | armadura; equipamento de proteção feito de metal | Datado: unidade de administração civil composta por 10 residências | uma palavra substituta para uma pessoa ou coisa indefinida; usado como pronome}
    \definition{v.}{ocupar o primeiro lugar; ser melhor do que}
  \end{Phonetics}
\end{Entry}

\begin{Entry}{甲骨文}{5,9,4}{⽥、⾻、⽂}
  \begin{Phonetics}{甲骨文}{jia3gu3wen2}
    \definition{s.}{escrituras de oráculos | inscrições em ossos de oráculos (forma original de escritura chinesa)}
  \end{Phonetics}
\end{Entry}

\begin{Entry}{申}{5}{⽥}
  \begin{Phonetics}{申}{shen1}
    \definition*{s.}{O nono dos doze Ramos Terrestres | Outro nome para Xangai, 上海 | Sobrenome Shen}
    \definition{v.}{declarar; explicar; expressar}
  \seealsoref{上海}{shang4hai3}
  \end{Phonetics}
\end{Entry}

\begin{Entry}{申请}{5,10}{⽥、⾔}
  \begin{Phonetics}{申请}{shen1qing3}[][HSK 4]
    \definition[份,批,项]{s.}{a solicitação para; o requerimento para; um pedido para ser visto pelos superiores ou departamentos relevantes}
    \definition{v.}{solicitar; apresentar uma solicitação; apresentar os motivos e fazer o pedido aos superiores ou aos departamentos competentes}
  \end{Phonetics}
\end{Entry}

\begin{Entry}{电}{5}{⽥}
  \begin{Phonetics}{电}{dian4}[][HSK 1]
    \definition*{s.}{Sobrenome Dian}
    \definition{s.}{eletricidade; energia elétrica | telegrama | relâmpago}
    \definition{v.}{dar ou receber um choque elétrico | enviar telegrama, telefonar ou enviar fax}
  \end{Phonetics}
\end{Entry}

\begin{Entry}{电力}{5,2}{⽥、⼒}
  \begin{Phonetics}{电力}{dian4 li4}[][HSK 6]
    \definition{s.}{energia elétrica; fornecimento de energia elétrica | energia elétrica | eletricidade}
  \end{Phonetics}
\end{Entry}

\begin{Entry}{电子}{5,3}{⽥、⼦}
  \begin{Phonetics}{电子}{dian4zi3}
    \definition{s.}{eletrônico | elétron}
  \end{Phonetics}
\end{Entry}

\begin{Entry}{电子名片}{5,3,6,4}{⽥、⼦、⼝、⽚}
  \begin{Phonetics}{电子名片}{dian4zi3 ming2pian4}
    \definition{s.}{cartão de visita eletrônico}
  \end{Phonetics}
\end{Entry}

\begin{Entry}{电子邮件}{5,3,7,6}{⽥、⼦、⾢、⼈}
  \begin{Phonetics}{电子邮件}{dian4zi3you2jian4}[][HSK 3]
    \definition[封,份,个,条]{s.}{correio eletrônico; \emph{e-mail}}
  \seealsoref{电邮}{dian4you2}
  \end{Phonetics}
\end{Entry}

\begin{Entry}{电子版}{5,3,8}{⽥、⼦、⽚}
  \begin{Phonetics}{电子版}{dian4 zi3 ban3}[][HSK 5]
    \definition[个]{s.}{edição eletrônica}
  \end{Phonetics}
\end{Entry}

\begin{Entry}{电车}{5,4}{⽥、⾞}
  \begin{Phonetics}{电车}{dian4 che1}[][HSK 6]
    \definition[辆,班,趟,路]{s.}{bonde; veículos de transporte público urbano movidos por linhas aéreas e acionados por motores de tração}
  \end{Phonetics}
\end{Entry}

\begin{Entry}{电车司机}{5,4,5,6}{⽥、⾞、⼝、⽊}
  \begin{Phonetics}{电车司机}{dian4che1 si1ji1}
    \definition{s.}{motorista de bonde}
  \end{Phonetics}
\end{Entry}

\begin{Entry}{电台}{5,5}{⽥、⼝}
  \begin{Phonetics}{电台}{dian4 tai2}[][HSK 3]
    \definition[个,家]{s.}{transceptor; transmissor-receptor | aparelho de rádio; estação de rádio; estação de transmissão}
  \end{Phonetics}
\end{Entry}

\begin{Entry}{电讯}{5,5}{⽥、⾔}
  \begin{Phonetics}{电讯}{dian4xun4}[][HSK 7-9]
    \definition{s.}{despacho (telegráfico) | telecomunicações; uma mensagem enviada por telefone, telégrafo ou rádio (telegráfico) | sinais de comunicação de rádio; sinais de rádio}
  \end{Phonetics}
\end{Entry}

\begin{Entry}{电冰箱}{5,6,15}{⽥、⼎、⾋}
  \begin{Phonetics}{电冰箱}{dian4bing1xiang1}
    \definition[台]{s.}{frigorífico | refrigerador}
  \end{Phonetics}
\end{Entry}

\begin{Entry}{电动}{5,6}{⽥、⼒}
  \begin{Phonetics}{电动}{dian4 dong4}[][HSK 6]
    \definition{adj.}{motorizado; acionado por energia elétrica; operado por energia elétrica elétrico}
  \end{Phonetics}
\end{Entry}

\begin{Entry}{电动车}{5,6,4}{⽥、⼒、⾞}
  \begin{Phonetics}{电动车}{dian4 dong4 che1}[][HSK 4]
    \definition{s.}{veículo elétrico (\emph{scooter}, bicicleta, carro, etc.)}
  \end{Phonetics}
\end{Entry}

\begin{Entry}{电池}{5,6}{⽥、⽔}
  \begin{Phonetics}{电池}{dian4chi2}[][HSK 5]
    \definition[节,块,组,个]{s.}{célula; bateria}
  \end{Phonetics}
\end{Entry}

\begin{Entry}{电灯}{5,6}{⽥、⽕}
  \begin{Phonetics}{电灯}{dian4 deng1}[][HSK 4]
    \definition[盏,个]{s.}{luz elétrica; lâmpada elétrica; lâmpadas que usam eletricidade como fonte de energia}
  \end{Phonetics}
\end{Entry}

\begin{Entry}{电灯泡}{5,6,8}{⽥、⽕、⽔}
  \begin{Phonetics}{电灯泡}{dian4deng1pao4}
    \definition{s.}{lâmpada elétrica | (gíria) terceiro convidado indesejado}
  \end{Phonetics}
\end{Entry}

\begin{Entry}{电网}{5,6}{⽥、⽹}
  \begin{Phonetics}{电网}{dian4wang3}[][HSK 7-9]
    \definition[张,个]{s.}{rede de arame eletrificada; emaranhamento de fios energizados | rede elétrica (ou grade)}
  \end{Phonetics}
\end{Entry}

\begin{Entry}{电报}{5,7}{⽥、⼿}
  \begin{Phonetics}{电报}{dian4bao4}[][HSK 7-9]
    \definition[封,份,个]{s.}{telegrama; cabo; telégrafo; mensagem}
  \end{Phonetics}
\end{Entry}

\begin{Entry}{电邮}{5,7}{⽥、⾢}
  \begin{Phonetics}{电邮}{dian4you2}
    \definition{s.}{correio eletrônico, \emph{e-mail} | abreviação de~电子邮件}
  \seealsoref{电子邮件}{dian4zi3you2jian4}
  \end{Phonetics}
\end{Entry}

\begin{Entry}{电饭锅}{5,7,12}{⽥、⾷、⾦}
  \begin{Phonetics}{电饭锅}{dian4 fan4 guo1}[][HSK 5]
    \definition[台,个]{s.}{panela elétrica de arroz}
  \end{Phonetics}
\end{Entry}

\begin{Entry}{电线}{5,8}{⽥、⽷}
  \begin{Phonetics}{电线}{dian4xian4}[][HSK 7-9]
    \definition[根,个]{s.}{fio; fio elétrico; condutor de corrente; condutor que transmite energia elétrica}
  \end{Phonetics}
\end{Entry}

\begin{Entry}{电视}{5,8}{⽥、⾒}
  \begin{Phonetics}{电视}{dian4shi4}[][HSK 1]
    \definition[部,台,个]{s.}{televisão; TV; televisor}
  \end{Phonetics}
\end{Entry}

\begin{Entry}{电视台}{5,8,5}{⽥、⾒、⼝}
  \begin{Phonetics}{电视台}{dian4 shi4 tai2}[][HSK 3]
    \definition[家,座,个]{s.}{canal de TV; estação de televisão; locais e instituições que transmitem programas de televisão}
  \end{Phonetics}
\end{Entry}

\begin{Entry}{电视机}{5,8,6}{⽥、⾒、⽊}
  \begin{Phonetics}{电视机}{dian4 shi4 ji1}[][HSK 1]
    \definition[个,台]{s.}{aparelho de TV; receptor de televisão; receptor de imagem; televisor; aparelho de televisão}
  \end{Phonetics}
\end{Entry}

\begin{Entry}{电视剧}{5,8,10}{⽥、⾒、⼑}
  \begin{Phonetics}{电视剧}{dian4 shi4 ju4}[][HSK 3]
    \definition[部,集,个]{s.}{série de TV; drama de TV; novela; drama escrito e gravado para transmissão pela televisão}
  \end{Phonetics}
\end{Entry}

\begin{Entry}{电话}{5,8}{⽥、⾔}
  \begin{Phonetics}{电话}{dian4 hua4}[][HSK 1]
    \definition[部]{s.}{telefone; aparelho telefônico; telefonia}
    \definition[通]{s.}{chamada telefônica; telefonema}
  \end{Phonetics}
\end{Entry}

\begin{Entry}{电信}{5,9}{⽥、⼈}
  \begin{Phonetics}{电信}{dian4xin4}[][HSK 7-9]
    \definition{s.}{telecomunicações; métodos de transmissão de informações usando tecnologias de comunicação com fio, rádio e óptica, incluindo telégrafo, telefone, etc.}
  \end{Phonetics}
\end{Entry}

\begin{Entry}{电脑}{5,10}{⽥、⾁}
  \begin{Phonetics}{电脑}{dian4nao3}[][HSK 1]
    \definition[个,台]{s.}{computador eletrônico}
  \end{Phonetics}
\end{Entry}

\begin{Entry}{电脑语言}{5,10,9,7}{⽥、⾁、⾔、⾔}
  \begin{Phonetics}{电脑语言}{dian4nao3yu3yan2}
    \definition{s.}{linguagem de programação | linguagem de computador}
  \end{Phonetics}
\end{Entry}

\begin{Entry}{电铃}{5,10}{⽥、⾦}
  \begin{Phonetics}{电铃}{dian4ling2}[][HSK 7-9]
    \definition[个]{s.}{campainha elétrica}
  \end{Phonetics}
\end{Entry}

\begin{Entry}{电梯}{5,11}{⽥、⽊}
  \begin{Phonetics}{电梯}{dian4ti1}[][HSK 4]
    \definition[部,台,架]{s.}{elevador}
  \end{Phonetics}
\end{Entry}

\begin{Entry}{电梯司机}{5,11,5,6}{⽥、⽊、⼝、⽊}
  \begin{Phonetics}{电梯司机}{dian4ti1 si1ji1}
    \definition{s.}{ascensorista}
  \end{Phonetics}
\end{Entry}

\begin{Entry}{电源}{5,13}{⽥、⽔}
  \begin{Phonetics}{电源}{dian4yuan2}[][HSK 4]
    \definition[台,个,套]{s.}{fonte de alimentação; fonte de energia; fonte de energia elétrica; dispositivo que fornece energia elétrica a um aparelho, como uma bateria, um gerador, etc.}
  \end{Phonetics}
\end{Entry}

\begin{Entry}{电影}{5,15}{⽥、⼺}
  \begin{Phonetics}{电影}{dian4ying3}[][HSK 1]
    \definition[部,片,幕,场]{s.}{filme; longa-metragem; cinema}
  \end{Phonetics}
\end{Entry}

\begin{Entry}{电影艺术}{5,15,4,5}{⽥、⼺、⾋、⽊}
  \begin{Phonetics}{电影艺术}{dian4ying3 yi4shu4}
    \definition{s.}{arte cinematográfica}
  \end{Phonetics}
\end{Entry}

\begin{Entry}{电影术}{5,15,5}{⽥、⼺、⽊}
  \begin{Phonetics}{电影术}{dian4ying3 shu4}
    \definition{s.}{cinematografia}
  \end{Phonetics}
\end{Entry}

\begin{Entry}{电影节}{5,15,5}{⽥、⼺、⾋}
  \begin{Phonetics}{电影节}{dian4ying3jie2}
    \definition{s.}{festival de cinema}
  \end{Phonetics}
\end{Entry}

\begin{Entry}{电影奖}{5,15,9}{⽥、⼺、⼤}
  \begin{Phonetics}{电影奖}{dian4ying3jiang3}
    \definition{s.}{premiações de cinema}
  \end{Phonetics}
\end{Entry}

\begin{Entry}{电影界}{5,15,9}{⽥、⼺、⽥}
  \begin{Phonetics}{电影界}{dian4ying3jie4}
    \definition{s.}{indústria cinematográfica}
  \end{Phonetics}
\end{Entry}

\begin{Entry}{电影院}{5,15,9}{⽥、⼺、⾩}
  \begin{Phonetics}{电影院}{dian4 ying3 yuan4}[][HSK 1]
    \definition[家,座,个]{s.}{cinema; sala de cinema; teatro; salão de cinema; local comercial dedicado à exibição de filmes}
  \end{Phonetics}
\end{Entry}

\begin{Entry}{电影音乐}{5,15,9,5}{⽥、⼺、⾳、⼃}
  \begin{Phonetics}{电影音乐}{dian4ying3 yin1yue4}
    \definition{s.}{música cinematográfica}
  \end{Phonetics}
\end{Entry}

\begin{Entry}{电影票}{5,15,11}{⽥、⼺、⽰}
  \begin{Phonetics}{电影票}{dian4ying3piao4}
    \definition{s.}{ingresso de filme}
  \end{Phonetics}
\end{Entry}

\begin{Entry}{电器}{5,16}{⽥、⼝}
  \begin{Phonetics}{电器}{dian4 qi4}[][HSK 6]
    \definition[件,种]{s.}{dispositivo elétrico; cargas em circuitos e dispositivos usados ​​para controlar, regular ou proteger circuitos, motores, etc.; como alto-falantes; interruptores; resistores; fusíveis, etc. | eletrodomésticos ou aparelhos elétricos domésticos; refere-se a eletrodomésticos, como televisores, gravadores, geladeiras, máquinas de lavar, etc.}
  \end{Phonetics}
\end{Entry}

\begin{Entry}{男}{7}{⽥}
  \begin{Phonetics}{男}{nan2}[][HSK 1]
    \definition{adj.}{homem; macho; masculino (em oposição a 女)}
    \definition[个,位]{s.}{filho; menino | homem | barão (o mais baixo de cinco ordens de nobreza)}
  \seealsoref{女}{nv3}
  \end{Phonetics}
\end{Entry}

\begin{Entry}{男人}{7,2}{⽥、⼈}
  \begin{Phonetics}{男人}{nan2 ren2}[][HSK 1]
    \definition[个]{s.}{homem adulto; macho; cavalheiro | marido}
  \end{Phonetics}
\end{Entry}

\begin{Entry}{男士}{7,3}{⽥、⼠}
  \begin{Phonetics}{男士}{nan2 shi4}[][HSK 4]
    \definition{s.}{cavalheiro; \emph{gentleman}}
  \end{Phonetics}
\end{Entry}

\begin{Entry}{男女}{7,3}{⽥、⼥}
  \begin{Phonetics}{男女}{nan2 nv3}[][HSK 4]
    \definition{s.}{homens e mulheres; masculino e feminino}
  \end{Phonetics}
\end{Entry}

\begin{Entry}{男子}{7,3}{⽥、⼦}
  \begin{Phonetics}{男子}{nan2zi3}[][HSK 3]
    \definition[个,位]{s.}{uma pessoa do sexo masculino; um homem}
  \end{Phonetics}
\end{Entry}

\begin{Entry}{男生}{7,5}{⽥、⽣}
  \begin{Phonetics}{男生}{nan2 sheng1}[][HSK 1]
    \definition[个]{s.}{menino; estudante; estudante do sexo masculino; aluno do sexo masculino}
  \end{Phonetics}
\end{Entry}

\begin{Entry}{男性}{7,8}{⽥、⼼}
  \begin{Phonetics}{男性}{nan2 xing4}[][HSK 5]
    \definition{s.}{masculino; homem; masculinidade; em oposição a 女性}
  \seealsoref{女性}{nv3 xing4}
  \end{Phonetics}
\end{Entry}

\begin{Entry}{男朋友}{7,8,4}{⽥、⽉、⼜}
  \begin{Phonetics}{男朋友}{nan2 peng2 you5}[][HSK 1]
    \definition{s.}{namorado}
  \end{Phonetics}
\end{Entry}

\begin{Entry}{男孩儿}{7,9,2}{⽥、⼦、⼉}
  \begin{Phonetics}{男孩儿}{nan2hai2r5}[][HSK 1]
    \definition{s.}{menino; rapaz}
  \end{Phonetics}
\end{Entry}

\begin{Entry}{画}{8}{⽥}
  \begin{Phonetics}{画}{hua4}[][HSK 2]
    \definition*{s.}{Sobrenome Hua}
    \definition{clas.}{traços (de um caractere chinês)}
    \definition[张,幅]{s.}{desenho; pintura; imagem; figura desenhada | traço horizontal (em caracteres chineses)}
    \definition{v.}{desenhar; pintar | desenhar; marcar; assinar}
  \seealsoref{划}{hua4}
  \end{Phonetics}
\end{Entry}

\begin{Entry}{画儿}{8,2}{⽥、⼉}
  \begin{Phonetics}{画儿}{hua4r5}[][HSK 2]
    \definition[幅,张]{s.}{imagem; desenho; pintura; obra de arte pintada}
  \end{Phonetics}
\end{Entry}

\begin{Entry}{画册}{8,5}{⽥、⼌}
  \begin{Phonetics}{画册}{hua4ce4}[][HSK 7-9]
    \definition[部,本]{s.}{álbum de imagens; álbum de pinturas; pinturas ou imagens encadernadas}
  \end{Phonetics}
\end{Entry}

\begin{Entry}{画龙点睛}{8,5,9,13}{⽥、⿓、⽕、⽬}
  \begin{Phonetics}{画龙点睛}{hua4long2-dian3jing1}[][HSK 7-9]
    \definition{expr.}{``Toque final.''; dar vida a um dragão pintado colocando as pupilas dos seus olhos -- adicionar o toque que dá vida a uma obra de arte; adicionar o toque final; adicionar uma palavra apropriada para concluir o ponto; adicionar uma ou duas palavras para finalizar o ponto; um toque crucial que reforça um ponto que de outra forma seria difícil de explicar; dar os retoques finais no trabalho de alguém; completar uma imagem}
  \end{Phonetics}
\end{Entry}

\begin{Entry}{画地为牢}{8,6,4,7}{⽥、⼟、⼂、⼧}
  \begin{Phonetics}{画地为牢}{hua4di4wei2lao2}
    \definition{expr.}{desenhar um círculo no chão para servir como uma prisão; restringir as atividades de alguém a uma área ou esfera designada; limitar; restringir | (literário) ser confinado dentro de um círculo desenhado no chão | (figurativo) limitar-se a uma gama restrita de atividades}
  \end{Phonetics}
\end{Entry}

\begin{Entry}{画面}{8,9}{⽥、⾯}
  \begin{Phonetics}{画面}{hua4 mian4}[][HSK 5]
    \definition[个,幅,帧]{s.}{quadro; aparência geral de uma imagem; imagem apresentada no quadro, na tela, etc.}
  \end{Phonetics}
\end{Entry}

\begin{Entry}{画家}{8,10}{⽥、⼧}
  \begin{Phonetics}{画家}{hua4 jia1}[][HSK 2]
    \definition[个,位,名,些]{s.}{pintor; pessoa com talento para pintura}
  \end{Phonetics}
\end{Entry}

\begin{Entry}{画展}{8,10}{⽥、⼫}
  \begin{Phonetics}{画展}{hua4zhan3}[][HSK 7-9]
    \definition{s.}{exposição de pinturas; exposição de arte}
  \end{Phonetics}
\end{Entry}

\begin{Entry}{画蛇添足}{8,11,11,7}{⽥、⾍、⽔、⾜}
  \begin{Phonetics}{画蛇添足}{hua4she2-tian1zu2}[][HSK 7-9]
    \definition{expr.}{arruinar o efeito adicionando algo supérfluo; dourar; fazer coisas desnecessárias pode sair pela culatra e piorar as coisas}
  \end{Phonetics}
\end{Entry}

\begin{Entry}{畅}{8}{⽥}
  \begin{Phonetics}{畅}{chang4}
    \definition*{s.}{Sobrenome Chang}
    \definition{adj.}{suave; desimpedido; sem obstáculos; desobstruído | livre; desinibido}
  \end{Phonetics}
\end{Entry}

\begin{Entry}{畅谈}{8,10}{⽥、⾔}
  \begin{Phonetics}{畅谈}{chang4tan2}[][HSK 7-9]
    \definition{v.}{falar livremente e com entusiasmo; falar livremente e com satisfação; falar com entusiasmo sobre}
  \end{Phonetics}
\end{Entry}

\begin{Entry}{畅销}{8,12}{⽥、⾦}
  \begin{Phonetics}{畅销}{chang4xiao1}[][HSK 7-9]
    \definition{adj.}{mais vendido; \emph{best-seller}}
    \definition{v.}{vender bem; ter grande procura; ter um mercado pronto; ser um \emph{best-seller}}
  \end{Phonetics}
\end{Entry}

\begin{Entry}{界}{9}{⽥}
  \begin{Phonetics}{界}{jie4}[][HSK 6]
    \definition{s.}{fronteira; limite | escopo; extensão | círculos | divisão primária; reino | era geológica | (matemática) limite | mundo; faixa dividida por ocupação, emprego ou gênero, etc. | grupo}
  \end{Phonetics}
\end{Entry}

\begin{Entry}{界碑}{9,13}{⽥、⽯}
  \begin{Phonetics}{界碑}{jie4bei1}
    \definition{s.}{marco de fronteira}
  \end{Phonetics}
\end{Entry}

\begin{Entry}{留}{10}{⽥}
  \begin{Phonetics}{留}{liu2}[][HSK 2]
    \definition*{s.}{Sobrenome Liu}
    \definition{v.}{ficar; permanecer; parar em um determinado local ou posição; não se afastar | estudar no exterior (geralmente seguido pelo nome de um país com uma sílaba) | pedir a alguém para ficar; manter alguém onde está | concentrar-se em; concentrar a atenção em algo | manter; guardar; reservar; não joger fora | acumular; deixar crescer | aceitar; receber | transmitir (legado); deixar para trás}
  \end{Phonetics}
\end{Entry}

\begin{Entry}{留下}{10,3}{⽥、⼀}
  \begin{Phonetics}{留下}{liu2 xia4}[][HSK 2]
    \definition{v.}{deixar; parar em algum lugar}
  \end{Phonetics}
\end{Entry}

\begin{Entry}{留言}{10,7}{⽥、⾔}
  \begin{Phonetics}{留言}{liu2 yan2}[][HSK 6]
    \definition[条]{s.}{mensagem; recado}
    \definition{v.}{deixar uma mensagem; deixar seus comentários}
  \end{Phonetics}
\end{Entry}

\begin{Entry}{留学}{10,8}{⽥、⼦}
  \begin{Phonetics}{留学}{liu2xue2}[][HSK 3]
    \definition{v.}{estudar no exterior; permanecer no estrangeiro para estudar ou pesquisar}
  \end{Phonetics}
\end{Entry}

\begin{Entry}{留学生}{10,8,5}{⽥、⼦、⽣}
  \begin{Phonetics}{留学生}{liu2 xue2 sheng1}[][HSK 2]
    \definition[个,位,名,批,帮]{s.}{estudante estrangeiro; estudante que retornou; estudante que estuda no exterior}
  \end{Phonetics}
\end{Entry}

\begin{Entry}{留神}{10,9}{⽥、⽰}
  \begin{Phonetics}{留神}{liu2/shen2}
    \definition{v.+compl.}{tomar cuidado | prestar atenção | manter os olhos abertos}
  \end{Phonetics}
\end{Entry}

\begin{Entry}{畜}{10}{⽥}
  \begin{Phonetics}{畜}{chu4}
    \definition*{s.}{Sobrenome Chu}
    \definition{s.}{animal doméstico; gado; bestas, principalmente referindo-se ao gado}
  \end{Phonetics}
  \begin{Phonetics}{畜}{xu4}
    \definition{v.}{criar (animais domésticos)}
  \end{Phonetics}
\end{Entry}

\begin{Entry}{略}{11}{⽥}
  \begin{Phonetics}{略}{lve4}
    \definition{adv.}{ligeiramente | marginalmente | aproximadamente}
  \end{Phonetics}
\end{Entry}

\begin{Entry}{略微}{11,13}{⽥、⼻}
  \begin{Phonetics}{略微}{lve4wei1}
    \definition{adv.}{ligeiramente | marginalmente | aproximadamente}
  \end{Phonetics}
\end{Entry}

\begin{Entry}{番}{12}{⽥}
  \begin{Phonetics}{番}{fan1}[][HSK 6]
    \definition{adj.}{estrangeiro; de tribos estrangeiras; estrangeiro ou alienígena}
    \definition{clas.}{usado para o número de vezes que uma ação é executada, equivalente a 回 ou 次 | usado para o tipo de coisas, equivalente a 种}
    \definition{s.}{estrangeiro; de tribos estrangeiras; (velho) refere-se a países estrangeiros ou raças estrangeiras | tomate; batata-doce | aborígenes; nativos; povos indígenas}
    \definition{v.}{revezar; rotacionar; substituir}
  \seealsoref{次}{ci4}
  \seealsoref{回}{hui2}
  \seealsoref{种}{zhong3}
  \end{Phonetics}
\end{Entry}

\begin{Entry}{番茄}{12,8}{⽥、⾋}
  \begin{Phonetics}{番茄}{fan1 qie2}[][HSK 6]
    \definition[个,斤,磅,公斤]{s.}{tomate | tomateiro}
  \end{Phonetics}
\end{Entry}

\begin{Entry}{畸}{13}{⽥}
  \begin{Phonetics}{畸}{ji1}
    \definition{adj.}{assimétrico; desequilibrado | irregular; anormal}
    \definition{s.}{Literário: uma quantidade fracionária (acima daquela mencionada em um número redondo); lotes ímpares | deformidade}
  \end{Phonetics}
\end{Entry}

\begin{Entry}{畸形}{13,7}{⽥、⼺}
  \begin{Phonetics}{畸形}{ji1xing2}[][HSK 7-9]
    \definition{adj.}{Medicina: deformado; malformado | desequilibrado; torto; desigual; anormal | deformidade; dismorfia; dismorfose; malformação; disgenesia; monstruosidade; aberração; desenvolvimento anormal de uma parte}
  \end{Phonetics}
\end{Entry}

%%%%% EOF %%%%%

