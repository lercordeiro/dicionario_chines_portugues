%%%
%%% Radical "⽼"
%%%

\section*{Radical 125: ``⽼'' (耂)}\addcontentsline{toc}{section}{Radical 125: ⽼、耂}

\begin{entry}{老}{6}{⽼}[Kangxi 125]
  \begin{phonetics}{老}{lao3}[][HSK 1,2]
    \definition*{s.}{sobrenome Lao}
    \definition{adj.}{velho; envelhecido; idade avançada | antigo; de longa data; existe há muito tempo | antigo; desatualizado; obsoleto; ultrapassado  | antigo; tradicional; original | coberto de vegetação; os vegetais cresceram além do período ideal para serem consumidos | resistente; endurecido; alimentos muito cozidos | escuro; profundo; (sobre cores) | último nascido; o mais novo | veterano; experiente; sofisticado}
    \definition{adv.}{longo; por muito tempo | sempre (fazendo algo) | muito}
    \definition{pref.}{usado para designar pessoas, ordem de classificação, certos nomes de animais e plantas}
    \definition{s.}{idosos; pessoas mais velhas | ancião; sênior; um título respeitoso para pessoas mais velhas}
    \definition{v.}{envelhecer | morrer; referindo-se à morte de um idoso}
  \end{phonetics}
\end{entry}

\begin{entry}{老人}{6,2}{⽼、⼈}
  \begin{phonetics}{老人}{lao3 ren2}[][HSK 1]
    \definition[位]{s.}{homem ou mulher de idade avançada; o idoso; o velho}
  \end{phonetics}
\end{entry}

\begin{entry}{老人家}{6,2,10}{⽼、⼈、⼧}
  \begin{phonetics}{老人家}{lao3 ren2 jia1}
    \definition{s.}{senhor ancião | madame | senhora | termo educado para mulher ou homem velho}
  \end{phonetics}
\end{entry}

\begin{entry}{老公}{6,4}{⽼、⼋}
  \begin{phonetics}{老公}{lao3 gong1}[][HSK 4]
    \definition[个]{s.}{marido; esposo}
  \end{phonetics}
\end{entry}

\begin{entry}{老太太}{6,4,4}{⽼、⼤、⼤}
  \begin{phonetics}{老太太}{lao3 tai4 tai5}[][HSK 3]
    \definition[位]{s.}{velha senhora; (em tratamento direto)Venerável Senhora; uma maneira respeitosa de chamar uma senhora idosa | (forma de tratamento) sua velha mãe; minha velha mãe, avó ou sogra}
  \end{phonetics}
\end{entry}

\begin{entry}{老头儿}{6,5,2}{⽼、⼤、⼉}
  \begin{phonetics}{老头儿}{lao3 tou2r5}[][HSK 3]
    \definition{s.}{(coloquial) velho; velho rabugento}
  \seealsoref{老头子}{lao3 tou2zi5}
  \end{phonetics}
\end{entry}

\begin{entry}{老头子}{6,5,3}{⽼、⼤、⼦}
  \begin{phonetics}{老头子}{lao3 tou2zi5}
    \definition{s.}{(coloquial) velho; velho rabugento}
  \seealsoref{老头儿}{lao3 tou2r5}
  \end{phonetics}
\end{entry}

\begin{entry}{老师}{6,6}{⽼、⼱}
  \begin{phonetics}{老师}{lao3shi1}[][HSK 1]
    \definition[个,位]{s.}{professor; título honorífico para professores; refere-se, de maneira geral, a pessoas que transmitem cultura e tecnologia ou que são dignas de admiração em termos de ideias, moralidade e conhecimentos profissionais}
  \end{phonetics}
\end{entry}

\begin{entry}{老年}{6,6}{⽼、⼲}
  \begin{phonetics}{老年}{lao3 nian2}[][HSK 2]
    \definition[个]{s.}{idoso; velhice; idade acima de 60 ou 70 anos}
  \end{phonetics}
\end{entry}

\begin{entry}{老百姓}{6,6,8}{⽼、⽩、⼥}
  \begin{phonetics}{老百姓}{lao3bai3xing4}[][HSK 3]
    \definition[个]{s.}{povo(s); civis; pessoas comuns; pessoas ordinárias}
  \end{phonetics}
\end{entry}

\begin{entry}{老兵}{6,7}{⽼、⼋}
  \begin{phonetics}{老兵}{lao3bing1}
    \definition{s.}{velho soldado | veterano de guerra | veterano (alguém que tem muita experiência em algum domínio)}
  \end{phonetics}
\end{entry}

\begin{entry}{老实}{6,8}{⽼、⼧}
  \begin{phonetics}{老实}{lao3shi5}[][HSK 4]
    \definition{adj.}{franco; sincero; honesto | bom; bem-comportado | ingênuo; simplório; meio bobo; facilmente enganado; eufemismo para pouco inteligente}
  \end{phonetics}
\end{entry}

\begin{entry}{老朋友}{6,8,4}{⽼、⽉、⼜}
  \begin{phonetics}{老朋友}{lao3 peng2 you3}[][HSK 2]
    \definition[个,位,名]{s.}{velho amigo; refere-se a amigos que conhecemos há muito tempo e com quem temos uma relação íntima}
  \end{phonetics}
\end{entry}

\begin{entry}{老板}{6,8}{⽼、⽊}
  \begin{phonetics}{老板}{lao3ban3}[][HSK 3]
    \definition[个,位]{s.}{chefe; dono | tratamento respeitoso a uma estrela de ópera ou a um líder de trupe}
  \end{phonetics}
\end{entry}

\begin{entry}{老虎}{6,8}{⽼、⾌}
  \begin{phonetics}{老虎}{lao3hu3}
    \definition[只]{s.}{tigre}
  \seealsoref{虎}{hu3}
  \end{phonetics}
\end{entry}

\begin{entry}{老是}{6,9}{⽼、⽇}
  \begin{phonetics}{老是}{lao3 shi4}[][HSK 2]
    \definition{adv.}{sempre; indica que a ação continua ou que o estado permanece inalterado, equivalente a 一直}
  \seealsoref{一直}{yi4zhi2}
  \end{phonetics}
\end{entry}

\begin{entry}{老家}{6,10}{⽼、⼧}
  \begin{phonetics}{老家}{lao3 jia1}[][HSK 4]
    \definition{s.}{cidade natal; local de origem | lugar nativo; refere-se às gerações anteriores da família ou ao local onde a pessoa nasceu ou viveu}
  \end{phonetics}
\end{entry}

\begin{entry}{老婆}{6,11}{⽼、⼥}
  \begin{phonetics}{老婆}{lao3po2}[][HSK 4]
    \definition[个]{s.}{esposa}
  \end{phonetics}
\end{entry}

\begin{entry}{考}{6}{⽼}
  \begin{phonetics}{考}{kao3}[][HSK 1]
    \definition*{s.}{sobrenome Kao}
    \definition{adj.}{antigo; velho; com idade avançada}
    \definition{s.}{o pai falecido de alguém}
    \definition{v.}{examinar; dar (fazer) um exame, teste ou questionário | verificar; inspecionar | estudar; verificar; investigar | perguntar; testar; fazer perguntas para que o outro responda, a fim de testar suas habilidades em determinada área}
  \end{phonetics}
\end{entry}

\begin{entry}{考生}{6,5}{⽼、⽣}
  \begin{phonetics}{考生}{kao3 sheng1}[][HSK 2]
    \definition{s.}{candidato a exame; alunos inscritos para o exame de admissão}
  \end{phonetics}
\end{entry}

\begin{entry}{考试}{6,8}{⽼、⾔}
  \begin{phonetics}{考试}{kao3shi4}[][HSK 1]
    \definition[次]{s.}{teste; exame; prova; atividades realizadas para verificar conhecimentos ou habilidades}
    \definition{v.+compl.}{testar; avaliar; avaliar conhecimentos e habilidades por meio de perguntas escritas ou orais.}
  \end{phonetics}
\end{entry}

\begin{entry}{考核}{6,10}{⽼、⽊}
  \begin{phonetics}{考核}{kao3he2}[][HSK 5]
    \definition{v.}{examinar; checar; avaliar; avaliar (a proficiência de alguém)}
  \end{phonetics}
\end{entry}

\begin{entry}{考虑}{6,10}{⽼、⾌}
  \begin{phonetics}{考虑}{kao3lv4}[][HSK 4]
    \definition{v.}{considerar; refletir sobre; levar em conta}
  \end{phonetics}
\end{entry}

\begin{entry}{考验}{6,10}{⽼、⾺}
  \begin{phonetics}{考验}{kao3yan4}[][HSK 3]
    \definition[场,个,种]{s.}{teste; julgamento}
    \definition{v.}{testar}
  \end{phonetics}
\end{entry}

\begin{entry}{考察}{6,14}{⽼、⼧}
  \begin{phonetics}{考察}{kao3cha2}[][HSK 4]
    \definition{v.}{inspecionar; investigar; observar e estudar}
  \end{phonetics}
\end{entry}

\begin{entry}{者}{8}{⽼}
  \begin{phonetics}{者}{zhe3}[][HSK 3]
    \definition{part.}{usado depois de um adjetivo ou verbo, ou depois de uma frase com um adjetivo ou verbo, para indicar uma pessoa ou coisa que tem esse atributo ou realiza essa ação | usado depois de ``trabalho'' e ``-ismo'' para se referir a pessoas que fazem um determinado trabalho ou acreditam em uma determinada ideologia | usado depois de numerais como dois, três,etc., referindo-se a vários itens mencionados no contexto | usado depois de palavras, frases ou cláusulas para indicar uma pausa}
    \definition{pron.}{usado principalmente no vernáculo antigo, significando o mesmo que 这}
    \definition{suf.}{voluntário}
  \seealsoref{这}{zhe4}
  \end{phonetics}
\end{entry}

%%%%% EOF %%%%%

