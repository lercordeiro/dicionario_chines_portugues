%%%
%%% Radical "⼱"
%%%

\section*{Radical 50: ``⼱''}\addcontentsline{toc}{section}{Radical 50: ⼱}

\begin{Entry}{市}{5}{⼱}
  \begin{Phonetics}{市}{shi4}[][HSK 2]
    \definition{s.}{mercado; lugar onde se concentra o comércio | cidade; município; áreas densamente povoadas, com indústrias, comércio e cultura desenvolvidos | relativo ao sistema tradicional chinês de pesos e medidas; unidades administrativas, incluindo cidades sob jurisdição direta e cidades sob jurisdição provincial (ou autônoma) | unidade padrão de mercado; pertencente ao sistema municipal (unidades de medida) | preço de transação no mercado}
    \definition{v.}{comprar ou vender; fazer transações}
  \end{Phonetics}
\end{Entry}

\begin{Entry}{市中心}{5,4,4}{⼱、⼁、⼼}
  \begin{Phonetics}{市中心}{shi4zhong1xin1}
    \definition{s.}{centro da cidade}
  \end{Phonetics}
\end{Entry}

\begin{Entry}{市区}{5,4}{⼱、⼖}
  \begin{Phonetics}{市区}{shi4 qu1}[][HSK 4]
    \definition[个]{s.}{\emph{downtown}; centro da cidade; distrito urbano; áreas que ficam dentro dos limites da cidade e geralmente têm uma alta concentração de população e estoque de moradias}
  \end{Phonetics}
\end{Entry}

\begin{Entry}{市升}{5,4}{⼱、⼗}
  \begin{Phonetics}{市升}{shi4sheng1}
    \definition{clas.}{sheng; uma unidade tradicional de volume, equivalente a 1 litro ou 1,76 \emph{pints} ou 0,22 galão}
  \end{Phonetics}
\end{Entry}

\begin{Entry}{市尺}{5,4}{⼱、⼫}
  \begin{Phonetics}{市尺}{shi4 chi3}
    \definition{clas.}{chi, uma unidade tradicional de comprimento, equivalente a 0,333 metros ou 1,094 pés}
  \end{Phonetics}
\end{Entry}

\begin{Entry}{市斤}{5,4}{⼱、⽄}
  \begin{Phonetics}{市斤}{shi4jin1}
    \definition{clas.}{jin, uma unidade tradicional de peso, cada uma contendo 10 liang (市两)  e equivalente a 0,5 quilogramas ou 1,102 libras}
  \seealsoref{市两}{shi4liang3}
  \end{Phonetics}
\end{Entry}

\begin{Entry}{市长}{5,4}{⼱、⾧}
  \begin{Phonetics}{市长}{shi4 zhang3}[][HSK 2]
    \definition[个,位,名]{s.}{prefeito; chefe administrativo responsável pela administração de uma cidade}
  \end{Phonetics}
\end{Entry}

\begin{Entry}{市民}{5,5}{⼱、⽒}
  \begin{Phonetics}{市民}{shi4 min2}[][HSK 6]
    \definition[位,名]{s.}{habitantes da cidade; residente da cidade; moradores da cidade | cidadão; refere-se especificamente aos artesãos e comerciantes de pequeno e médio porte nas cidades da sociedade feudal tardia}
  \end{Phonetics}
\end{Entry}

\begin{Entry}{市场}{5,6}{⼱、⼟}
  \begin{Phonetics}{市场}{shi4chang3}[][HSK 3]
    \definition[家]{s.}{mercado (também no abstrato); um lugar fixo onde as pessoas compram e vendem coisas juntas | área de \emph{marketing}; região onde o produto é vendido | âmbito de influência (figurado); uma metáfora para o escopo e o grau em que uma determinada ideia ou comportamento é aceito por outros}
  \end{Phonetics}
\end{Entry}

\begin{Entry}{市两}{5,7}{⼱、⼀}
  \begin{Phonetics}{市两}{shi4liang3}
    \definition{clas.}{liang, uma unidade tradicional de peso, igual a 0,1 jin (市斤), e equivalente a 50 gramas ou 1,763 onças}
  \seealsoref{市斤}{shi4jin1}
  \end{Phonetics}
\end{Entry}

\begin{Entry}{市亩}{5,7}{⼱、⼇}
  \begin{Phonetics}{市亩}{shi4mu3}
    \definition{clas.}{mu, uma unidade tradicional de área, igual a 60 zhang quadrados (平方市丈) e equivalente a 6,667 ares ou 0,165 acre}
  \seealsoref{平方市丈}{ping2fang1 shi4 zhang4}
  \end{Phonetics}
\end{Entry}

\begin{Entry}{布}{5}{⼱}
  \begin{Phonetics}{布}{bu4}[][HSK 3]
    \definition*{s.}{Sobrenome Bu}
    \definition[块,幅,匹]{s.}{tecido; tecido de algodão; algodão, linho ou fibras sintéticas tecidas, que podem ser utilizadas como material para confecção de roupas ou outros objetos | uma moeda antiga | algo parecido com um pano}
    \definition{v.}{declarar; anunciar; publicar; proclamar | divulgar; espalhar por toda parte; difundir amplamente | implantar; dispor; organizar}
  \end{Phonetics}
\end{Entry}

\begin{Entry}{布局}{5,7}{⼱、⼫}
  \begin{Phonetics}{布局}{bu4ju2}[][HSK 7-9]
    \definition{s.}{\emph{layout}; distribuição; arranjo geral; arranjo abrangente: planejamento e arranjo da estrutura geral das coisas; especialmente o arranjo de materiais e tramas na criação artística}
    \definition{v.}{planejar; compor uma imagem, ensaio, etc. (geralmente se refere a escrever, pintar, jogar xadrez, etc.) | posicionar as peças em um tabuleiro de xadrez}
  \end{Phonetics}
\end{Entry}

\begin{Entry}{布谷鸟}{5,7,5}{⼱、⾕、⿃}
  \begin{Phonetics}{布谷鸟}{bu4gu3niao3}
    \definition{s.}{cuco (pássaro)}
  \seealsoref{杜鹃}{du4juan1}
  \seealsoref{杜鹃鸟}{du4juan1niao3}
  \seealsoref{杜宇}{du4yu3}
  \end{Phonetics}
\end{Entry}

\begin{Entry}{布满}{5,13}{⼱、⽔}
  \begin{Phonetics}{布满}{bu4 man3}[][HSK 6]
    \definition{v.}{abundar em; estar cheio de; espalhar-se e preencher um certo espaço}
  \end{Phonetics}
\end{Entry}

\begin{Entry}{布置}{5,13}{⼱、⽹}
  \begin{Phonetics}{布置}{bu4zhi4}[][HSK 4]
    \definition{v.}{arrumar; organizar; decorar; colocar adequadamente objetos ou paisagismo, conforme necessário | designar; tomar providências para; dar instruções sobre; organizar trabalho, atividades, etc.}
  \end{Phonetics}
\end{Entry}

\begin{Entry}{布署}{5,13}{⼱、⽹}
  \begin{Phonetics}{布署}{bu4shu3}
    \variantof{部署}
  \end{Phonetics}
\end{Entry}

\begin{Entry}{帅}{5}{⼱}
  \begin{Phonetics}{帅}{shuai4}[][HSK 4]
    \definition*{s.}{Sobrenome Shuai}
    \definition{adj.}{bonito; arrojado; elegante; inteligente}
    \definition{interj.}{Legal!}
    \definition[位,名,个,些]{s.}{comandante em chefe; o mais alto comandante do exército | comandante em chefe, a peça principal no xadrez chinês}
  \end{Phonetics}
\end{Entry}

\begin{Entry}{帅哥}{5,10}{⼱、⼝}
  \begin{Phonetics}{帅哥}{shuai4 ge1}[][HSK 4]
    \definition[个,位,名,些]{s.}{rapaz bonito; um garoto que é bonito e atraente na aparência}
  \end{Phonetics}
\end{Entry}

\begin{Entry}{师}{6}{⼱}
  \begin{Phonetics}{师}{shi1}
    \definition*{s.}{Sobrenome Shi}
    \definition[位,名,个]{s.}{professor; tutor; mestre | exemplo; modelo a seguir | título honorífico para um monge budista; (termo de respeito para um monge ou freira) mestre; mãe | do seu mestre ou professor | divisão; tropas; exército}
    \definition{suf.}{pessoa qualificada em determinada profissão}
    \definition{v.}{Literário: imitar; aprender}
  \end{Phonetics}
\end{Entry}

\begin{Entry}{师父}{6,4}{⼱、⽗}
  \begin{Phonetics}{师父}{shi1 fu5}[][HSK 6]
    \definition[个,位,名,些]{s.}{mestre; mestre trabalhador; um título respeitoso dado por um aprendiz ao seu mestre | um título respeitoso para monges, freiras e sacerdotes taoístas}
  \end{Phonetics}
\end{Entry}

\begin{Entry}{师生}{6,5}{⼱、⽣}
  \begin{Phonetics}{师生}{shi1 sheng1}[][HSK 6]
    \definition{s.}{mestre e discípulo; professores e alunos; um nome combinado para professores e alunos}
  \end{Phonetics}
\end{Entry}

\begin{Entry}{师傅}{6,12}{⼱、⼈}
  \begin{Phonetics}{师傅}{shi1fu5}[][HSK 5]
    \definition[个,位,名]{s.}{mestre; um trabalhador qualificado; título honorífico para pessoas habilidosas | mestre; professor (em certos ofícios); pessoas que ensinam técnicas em áreas como engenharia, comércio e teatro}
  \end{Phonetics}
\end{Entry}

\begin{Entry}{希}{7}{⼱}
  \begin{Phonetics}{希}{xi1}
    \definition*{s.}{Sobrenome Xi}
    \definition{v.}{ter esperança}
  \end{Phonetics}
\end{Entry}

\begin{Entry}{希望}{7,11}{⼱、⽉}
  \begin{Phonetics}{希望}{xi1wang4}[][HSK 3]
    \definition[个,丝,点]{s.}{esperança; desejo; expectativa; a possibilidade de alcançar um determinado objetivo ou de ocorrer uma determinada situação ideal no futuro | aquilo em que a esperança é depositada; o objeto da esperança}
    \definition{v.}{ter esperança; desejar; esperar; pensar em alcançar algum objetivo ou que alguma situação ocorra}
  \end{Phonetics}
\end{Entry}

\begin{Entry}{帘}{8}{⼱}
  \begin{Phonetics}{帘}{lian2}
    \definition[块,个]{s.}{bandeira em mastro sobre adega; bandeira como placa de loja | cortina; tela de bambu ou tecido; objetos para cobrir portas e janelas}
  \end{Phonetics}
\end{Entry}

\begin{Entry}{帝}{9}{⼱}
  \begin{Phonetics}{帝}{di4}
    \definition*{s.}{Ser Supremo; Deus}
    \definition[位,名,个]{s.}{imperador | (abreviação) imperialismo}
  \end{Phonetics}
\end{Entry}

\begin{Entry}{帝国}{9,8}{⼱、⼞}
  \begin{Phonetics}{帝国}{di4guo2}
    \definition{adj.}{imperial}
    \definition{s.}{império}
  \end{Phonetics}
\end{Entry}

\begin{Entry}{带}{9}{⼱}
  \begin{Phonetics}{带}{dai4}[][HSK 2]
    \definition*{s.}{Sobrenome Dai}
    \definition[根]{s.}{cinto; faixa; banda; fita; fita adesiva; algo parecido com uma fita | pneu | zona; área; faixa; cinturão; região; uma determinada área geográfica com determinadas características | leucorreia; corrimento branco; corrimento vaginal}
    \definition{v.}{levar; trazer; transportar | liderar; dirigir; conduzir; assumir | cuidar de crianças; criar filhos; educar | fazer uma coisa e, ao mesmo tempo, fazer outra coisa |suportar; conter | ter algo anexado, simultâneo | trazer consigo | carregar consigo | demonstrar; parecer | incluir; acrescentar}
  \end{Phonetics}
\end{Entry}

\begin{Entry}{带队}{9,4}{⼱、⾩}
  \begin{Phonetics}{带队}{dai4dui4}[][HSK 7-9]
    \definition{v.}{liderar um grupo (para fazer algo)}
  \end{Phonetics}
\end{Entry}

\begin{Entry}{带头}{9,5}{⼱、⼤}
  \begin{Phonetics}{带头}{dai4/tou2}[][HSK 7-9]
    \definition{v.+compl.}{assumir a liderança; ser o primeiro; tomar a iniciativa; dar o exemplo}
  \end{Phonetics}
\end{Entry}

\begin{Entry}{带头人}{9,5,2}{⼱、⼤、⼈}
  \begin{Phonetics}{带头人}{dai4tou2 ren2}[][HSK 7-9]
    \definition{s.}{líder; pioneiro}
  \end{Phonetics}
\end{Entry}

\begin{Entry}{带动}{9,6}{⼱、⼒}
  \begin{Phonetics}{带动}{dai4 dong4}[][HSK 3]
    \definition{v.}{dirigir; ativar; fazer algo funcionar; acionar | liderar; trazer; estimular; motivar; atrair; liderar o avanço; dar o exemplo e fazer com que os outros sigam o exemplo}
  \end{Phonetics}
\end{Entry}

\begin{Entry}{带有}{9,6}{⼱、⽉}
  \begin{Phonetics}{带有}{dai4 you3}[][HSK 5]
    \definition{v.}{ter; envolver; carregar; implicar}
  \end{Phonetics}
\end{Entry}

\begin{Entry}{带来}{9,7}{⼱、⽊}
  \begin{Phonetics}{带来}{dai4 lai2}[][HSK 2]
    \definition{v.}{provocar; produzir; causar}
  \end{Phonetics}
\end{Entry}

\begin{Entry}{带领}{9,11}{⼱、⾴}
  \begin{Phonetics}{带领}{dai4ling3}[][HSK 3]
    \definition{v.}{guiar, na frente, liderando | liderar e comandar}
  \end{Phonetics}
\end{Entry}

\begin{Entry}{带路}{9,13}{⼱、⾜}
  \begin{Phonetics}{带路}{dai4/lu4}[][HSK 7-9]
    \definition{v.+compl.}{mostrar o caminho; agir como um guia; guiar}
  \end{Phonetics}
\end{Entry}

\begin{Entry}{帮}{9}{⼱}
  \begin{Phonetics}{帮}{bang1}[][HSK 1]
    \definition*{s.}{Sobrenome Bang}
    \definition{clas.}{um grupo de; um bando de; uma gangue de; um grupo de pessoas}
    \definition{s.}{lateral; superior; partes ao lado ou ao redor do objeto | folha externa; parte mais grossa das folhas externas dos vegetais | gangue; banda; grupo; conglomerado}
    \definition{v.}{ajudar; assistir; auxiliar | trabalho; refere-se ao envolvimento em trabalho assalariado}
  \end{Phonetics}
\end{Entry}

\begin{Entry}{帮忙}{9,6}{⼱、⼼}
  \begin{Phonetics}{帮忙}{bang1/mang2}[][HSK 1]
    \definition{v.+compl.}{ajudar; dar uma mão; dar uma mãozinha; fazer um favor; fazer uma boa ação; ajudar os outros a fazer algo, referindo-se, de maneira geral, a oferecer ajuda quando alguém está com dificuldades}
  \end{Phonetics}
\end{Entry}

\begin{Entry}{帮佣}{9,7}{⼱、⼈}
  \begin{Phonetics}{帮佣}{bang1yong1}
    \definition{s.}{trabalhador doméstico; empregada doméstica; servo; servente}
    \definition{v.}{trabalhar ou ser contratado como trabalhador doméstico, servo, etc.}
  \end{Phonetics}
\end{Entry}

\begin{Entry}{帮助}{9,7}{⼱、⼒}
  \begin{Phonetics}{帮助}{bang1zhu4}[][HSK 2]
    \definition[个,次,回,份,种]{s.}{ajuda; auxílio; socorro; função de promoção ou auxílio}
    \definition{v.}{ajudar; assistir; apoiar; quando alguém está passando por dificuldades, oferecer apoio financeiro ou material, ou ainda apoio moral, dar conselhos, pensar em soluções, fazer coisas por essa pessoa, etc.}
  \end{Phonetics}
\end{Entry}

\begin{Entry}{帮教}{9,11}{⼱、⽁}
  \begin{Phonetics}{帮教}{bang1jiao4}
    \definition{v.}{orientar}
  \end{Phonetics}
\end{Entry}

\begin{Entry}{席}{10}{⼱}
  \begin{Phonetics}{席}{xi2}
    \definition*{s.}{Sobrenome Xi}
    \definition[卷,张]{s.}{esteira | assento; lugar; caixa | assento (em uma assembleia legislativa) | festim; banquete; jantar}
  \end{Phonetics}
\end{Entry}

\begin{Entry}{席卷}{10,8}{⼱、⼙}
  \begin{Phonetics}{席卷}{xi2juan3}
    \definition{v.}{engolfar | varrer | levar tudo para fora}
  \end{Phonetics}
\end{Entry}

\begin{Entry}{常}{11}{⼱}
  \begin{Phonetics}{常}{chang2}[][HSK 1]
    \definition*{s.}{Sobrenome Chang}
    \definition{adj.}{normal; comum; ordinário; indica frequência, normalidade, universalidade | constante; invariável; imutável; permanente}
    \definition{adv.}{frequentemente; geralmente; com frequência;}
    \definition{s.}{normas; disciplina, ordem social e lei e ordem do Estado}
  \end{Phonetics}
\end{Entry}

\begin{Entry}{常人}{11,2}{⼱、⼈}
  \begin{Phonetics}{常人}{chang2ren2}[][HSK 7-9]
    \definition{s.}{pessoa comum; homem da rua}
  \end{Phonetics}
\end{Entry}

\begin{Entry}{常见}{11,4}{⼱、⾒}
  \begin{Phonetics}{常见}{chang2 jian4}[][HSK 2]
    \definition{adj.}{comum; frequentemente visto}
  \end{Phonetics}
\end{Entry}

\begin{Entry}{常用}{11,5}{⼱、⽤}
  \begin{Phonetics}{常用}{chang2 yong4}[][HSK 2]
    \definition{adj.}{em uso comum; frequentemente utilizado}
  \end{Phonetics}
\end{Entry}

\begin{Entry}{常年}{11,6}{⼱、⼲}
  \begin{Phonetics}{常年}{chang2 nian2}[][HSK 6]
    \definition{adj.}{perene; anual}
    \definition{adv.}{ano após ano; ao longo do ano; durante todo o ano; longo prazo}
  \end{Phonetics}
\end{Entry}

\begin{Entry}{常问问题}{11,6,6,15}{⼱、⾨、⾨、⾴}
  \begin{Phonetics}{常问问题}{chang2wen4wen4ti2}
    \definition{s.}{FAQ; perguntas frequentes}
  \end{Phonetics}
\end{Entry}

\begin{Entry}{常识}{11,7}{⼱、⾔}
  \begin{Phonetics}{常识}{chang2shi2}[][HSK 4]
    \definition[门]{s.}{senso comum; conhecimento geral; conhecimento que uma pessoa comum deve ter}
  \end{Phonetics}
\end{Entry}

\begin{Entry}{常态}{11,8}{⼱、⼼}
  \begin{Phonetics}{常态}{chang2tai4}[][HSK 7-9]
    \definition{s.}{normalidade; \emph{habitus}; comportamento normal; condições normais; estado normal ou usual}
  \end{Phonetics}
\end{Entry}

\begin{Entry}{常规}{11,8}{⼱、⾒}
  \begin{Phonetics}{常规}{chang2 gui1}[][HSK 6]
    \definition[个,种]{s.}{convenção; prática comum; rotina | (medicina) rotina | regra; sulco}
  \end{Phonetics}
\end{Entry}

\begin{Entry}{常常}{11,11}{⼱、⼱}
  \begin{Phonetics}{常常}{chang2 chang2}[][HSK 1]
    \definition{adv.}{frequentemente; muitas vezes; geralmente; indica que a ação ocorreu várias vezes}
  \end{Phonetics}
\end{Entry}

\begin{Entry}{常理}{11,11}{⼱、⽟}
  \begin{Phonetics}{常理}{chang2li3}[][HSK 7-9]
    \definition{s.}{regra geral; o que é normal | senso comum; pensamento lógico | raciocínio convencional e moral}
  \end{Phonetics}
\end{Entry}

\begin{Entry}{常温}{11,12}{⼱、⽔}
  \begin{Phonetics}{常温}{chang2wen1}[][HSK 7-9]
    \definition{s.}{temperatura atmosférica normal; temperatura ordinária | homeotermia}
  \end{Phonetics}
\end{Entry}

\begin{Entry}{帽}{12}{⼱}
  \begin{Phonetics}{帽}{mao4}
    \definition[个,顶]{s.}{chapéu; boné | capa; uma coisa que cobre um objeto e tem a função ou formato de um chapéu | elmo; capacete}
  \end{Phonetics}
\end{Entry}

\begin{Entry}{帽子}{12,3}{⼱、⼦}
  \begin{Phonetics}{帽子}{mao4zi5}[][HSK 4]
    \definition[顶,个,种]{s.}{boné; chapéu; capacete | etiqueta; rótulo; marca}
  \end{Phonetics}
\end{Entry}

\begin{Entry}{幅}{12}{⼱}
  \begin{Phonetics}{幅}{fu2}[][HSK 5]
    \definition{clas.}{usado para tecidos, telas de lã, pinturas, etc.}
    \definition{s.}{largura do tecido, seda, tweed, etc. | tamanho; largura; geralmente se refere à largura}
  \end{Phonetics}
\end{Entry}

\begin{Entry}{幅度}{12,9}{⼱、⼴}
  \begin{Phonetics}{幅度}{fu2du4}[][HSK 5]
    \definition{s.}{alcance; escopo; extensão; largura; largura da propagação de um objeto que vibra ou balança, uma metáfora para a magnitude de uma mudança em algo}
  \end{Phonetics}
\end{Entry}

\begin{Entry}{幕}{13}{⼱}
  \begin{Phonetics}{幕}{mu4}
    \definition{s.}{cortina ou tela | dossel ou tenda | quartel de um general | ato (de uma peça)}
  \end{Phonetics}
\end{Entry}

%%%%% EOF %%%%%

