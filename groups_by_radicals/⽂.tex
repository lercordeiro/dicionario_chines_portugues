%%%
%%% Radical "⽂"
%%%

\section*{Radical 67: ``⽂''}\addcontentsline{toc}{section}{Radical 67: ⽂}

\begin{entry}{文化}{4,4}{⽂、⼔}
  \begin{phonetics}{文化}{wen2hua4}[][HSK 3]
    \definition[个,种]{s.}{cultura; civilização | cultura; alfabetização; escolaridade; educação}
  \end{phonetics}
\end{entry}

\begin{entry}{文化水平}{4,4,4,5}{⽂、⼔、⽔、⼲}
  \begin{phonetics}{文化水平}{wen2hua4 shui3ping2}
    \definition{s.}{nível educacional}
  \end{phonetics}
\end{entry}

\begin{entry}{文化史}{4,4,5}{⽂、⼔、⼝}
  \begin{phonetics}{文化史}{wen2hua4shi3}
    \definition*{s.}{História Cultural}
  \end{phonetics}
\end{entry}

\begin{entry}{文化层}{4,4,7}{⽂、⼔、⼫}
  \begin{phonetics}{文化层}{wen2hua4ceng2}
    \definition{s.}{nível de cultura (em sítio arqueológico)}
  \end{phonetics}
\end{entry}

\begin{entry}{文化宫}{4,4,9}{⽂、⼔、⼧}
  \begin{phonetics}{文化宫}{wen2hua4gong1}
    \definition{s.}{palácio cultural}
  \end{phonetics}
\end{entry}

\begin{entry}{文化热}{4,4,10}{⽂、⼔、⽕}
  \begin{phonetics}{文化热}{wen2hua4re4}
    \definition{s.}{mania cultural | febre cultural}
  \end{phonetics}
\end{entry}

\begin{entry}{文化圈}{4,4,11}{⽂、⼔、⼞}
  \begin{phonetics}{文化圈}{wen2hua4quan1}
    \definition{s.}{esfera de influência cultural}
  \end{phonetics}
\end{entry}

\begin{entry}{文化障碍}{4,4,13,13}{⽂、⼔、⾩、⽯}
  \begin{phonetics}{文化障碍}{wen2hua4zhang4'ai4}
    \definition{s.}{barreira cultural}
  \end{phonetics}
\end{entry}

\begin{entry}{文艺}{4,4}{⽂、⾋}
  \begin{phonetics}{文艺}{wen2yi4}[][HSK 5]
    \definition{s.}{termo genérico para literatura e arte | performance (arte); refere-se especificamente às artes performativas, como música e dança}
  \end{phonetics}
\end{entry}

\begin{entry}{文件}{4,6}{⽂、⼈}
  \begin{phonetics}{文件}{wen2jian4}[][HSK 3]
    \definition[份,分]{s.}{documentos oficiais; papéis; instrumentos | os arquivos no computador | artigos ou trabalhos sobre teorias políticas, atualidades, pesquisas acadêmicas, etc.}
  \end{phonetics}
\end{entry}

\begin{entry}{文字}{4,6}{⽂、⼦}
  \begin{phonetics}{文字}{wen2zi4}[][HSK 3]
    \definition[种,类,段,行,篇]{s.}{personagens; roteiro; escrita | linguagem escrita}
  \end{phonetics}
\end{entry}

\begin{entry}{文学}{4,8}{⽂、⼦}
  \begin{phonetics}{文学}{wen2xue2}[][HSK 3]
    \definition[个,种]{s.}{literatura}
  \end{phonetics}
\end{entry}

\begin{entry}{文学系}{4,8,7}{⽂、⼦、⽷}
  \begin{phonetics}{文学系}{wen2xue2 xi4}
    \definition*{s.}{Faculdade de Letras}
  \end{phonetics}
\end{entry}

\begin{entry}{文明}{4,8}{⽂、⽇}
  \begin{phonetics}{文明}{wen2ming2}[][HSK 3]
    \definition{adj.}{civilizado}
    \definition[个]{s.}{cultura; civilização}
  \end{phonetics}
\end{entry}

\begin{entry}{文章}{4,11}{⽂、⾳}
  \begin{phonetics}{文章}{wen2zhang1}[][HSK 3]
    \definition[篇,段,页]{s.}{ensaio; dissertação; artigo | significado oculto; significado implícito | trabalho (coisas para fazer)}
  \end{phonetics}
\end{entry}

%%%%% EOF %%%%%

