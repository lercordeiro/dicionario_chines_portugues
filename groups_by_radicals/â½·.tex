%%%
%%% Radical "⽷"
%%%

\section*{Radical 120: ``⽷'' (纟、糹)}\addcontentsline{toc}{section}{Radical 120: ⽷、纟、糹}

\begin{entry}{纠}{5}{⽷}
  \begin{phonetics}{纠}{jiu1}
    \definition*{s.}{Sobrenome Jiu}
    \definition{v.}{emaranhar | reunir-se | corrigir; retificar | supervisionar; superintender}
  \end{phonetics}
\end{entry}

\begin{entry}{纠正}{5,5}{⽷、⽌}
  \begin{phonetics}{纠正}{jiu1zheng4}[][HSK 6]
    \definition{v.}{fazer certo; corrigir (deficiências ou erros em pensamentos, ações, métodos, etc.)}
  \end{phonetics}
\end{entry}

\begin{entry}{纠纷}{5,7}{⽷、⽷}
  \begin{phonetics}{纠纷}{jiu1fen1}[][HSK 6]
    \definition[个,次]{s.}{questão; disputa; existem contradições ou conflitos de interesse entre as duas partes que precisam ser resolvidos}
  \end{phonetics}
\end{entry}

\begin{entry}{纠葛}{5,12}{⽷、⾋}
  \begin{phonetics}{纠葛}{jiu1ge2}
    \definition{s.}{emaranhado | disputa}
  \end{phonetics}
\end{entry}

\begin{entry}{红}{6}{⽷}
  \begin{phonetics}{红}{hong2}[][HSK 2]
    \definition*{s.}{Sobrenome Hong}
    \definition{adj.}{vermelho | popular; bem-sucedido; símbolo de sucesso ou valorização | vermelho; revolucionário; símbolo da revolução | festivo; símbolo de alegria}
    \definition{s.}{tecido vermelho, bandeirinhas, etc. usados em ocasiões festivas | bônus; dividendo}
  \end{phonetics}
\end{entry}

\begin{entry}{红心}{6,4}{⽷、⼼}
  \begin{phonetics}{红心}{hong2 xin1}
    \definition{s.}{coração vermelho, um coração leal à causa da revolução proletária | alvo | coração ♥ (em jogos de cartas) | red, heart-shaped symbol}
  \seealsoref{方片}{fang1 pian4}
  \seealsoref{黑桃}{hei1 tao2}
  \seealsoref{梅花}{mei2 hua1}
  \end{phonetics}
\end{entry}

\begin{entry}{红包}{6,5}{⽷、⼓}
  \begin{phonetics}{红包}{hong2 bao1}[][HSK 4]
    \definition[个]{s.}{saco de papel vermelho ou envelope contendo dinheiro como presente, gorjeta ou bônus | suborno; propina}
  \end{phonetics}
\end{entry}

\begin{entry}{红色}{6,6}{⽷、⾊}
  \begin{phonetics}{红色}{hong2 se4}[][HSK 2]
    \definition{adj.}{vermelho; revolucionário; símbolo da revolução ou da consciência política elevada}
    \definition{s.}{cor vermelha}
  \end{phonetics}
\end{entry}

\begin{entry}{红宝石}{6,8,5}{⽷、⼧、⽯}
  \begin{phonetics}{红宝石}{hong2bao3shi2}
    \definition{s.}{rubi}
  \end{phonetics}
\end{entry}

\begin{entry}{红线}{6,8}{⽷、⽷}
  \begin{phonetics}{红线}{hong2xian4}
    \definition{s.}{linha vermelha}
  \end{phonetics}
\end{entry}

\begin{entry}{红茶}{6,9}{⽷、⾋}
  \begin{phonetics}{红茶}{hong2 cha2}[][HSK 3]
    \definition[杯,壶,斤,种]{s.}{chá preto; chá acabado produzido através de fermentação completa}
  \end{phonetics}
\end{entry}

\begin{entry}{红烧}{6,10}{⽷、⽕}
  \begin{phonetics}{红烧}{hong2shao1}
    \definition{s.}{guisado em molho de soja (prato)}
  \end{phonetics}
\end{entry}

\begin{entry}{红酒}{6,10}{⽷、⾣}
  \begin{phonetics}{红酒}{hong2 jiu3}[][HSK 3]
    \definition[瓶,杯,壶,斤,箱]{s.}{vinho tinto}
  \end{phonetics}
\end{entry}

\begin{entry}{红绿灯}{6,11,6}{⽷、⽷、⽕}
  \begin{phonetics}{红绿灯}{hong2lv4deng1}
    \definition[个]{s.}{semáforo | sinal de trânsito}
  \end{phonetics}
\end{entry}

\begin{entry}{红薯}{6,16}{⽷、⾋}
  \begin{phonetics}{红薯}{hong2shu3}
    \definition{s.}{batata doce}
  \end{phonetics}
\end{entry}

\begin{entry}{约}{6}{⽷}
  \begin{phonetics}{约}{yao1}
    \definition{adj.}{econômico; frugal | simples; breve | indistinto}
    \definition{adv.}{cerca de; ao redor; aproximadamente}
    \definition{s.}{pacto; acordo; nomeação; coisa prometida}
    \definition{v.}{marcar uma consulta; organizar | perguntar ou convidar com antecedência | restringir; conter | reduzir (fração aproximada)}
  \end{phonetics}
  \begin{phonetics}{约}{yue1}[][HSK 3]
    \definition*{s.}{Sobrenome Yue}
    \definition{adj.}{econômico; frugal | simples; breve; resumido | indistinto; confuso}
    \definition{adv.}{cerca de; ao redor; aproximadamente}
    \definition{s.}{pacto; acordo; nomeação; o que foi combinado}
    \definition{v.}{combinar; propor ou discutir antecipadamente (o que deve ser respeitado por todos) | convidar com antecedência | restringir; conter | reduzir (fração aproximada)}
  \end{phonetics}
\end{entry}

\begin{entry}{约会}{6,6}{⽷、⼈}
  \begin{phonetics}{约会}{yue1hui4}[][HSK 4]
    \definition[个,次]{s.}{data; compromisso; engajamento; reunião pré-agendada}
    \definition{v.}{marcar uma reunião; marcar um encontro;}
  \end{phonetics}
\end{entry}

\begin{entry}{约束}{6,7}{⽷、⽊}
  \begin{phonetics}{约束}{yue1shu4}[][HSK 5]
    \definition{adj.}{amarrado}
    \definition{s.}{restrição; constrangimento; engajamento}
    \definition{v.}{amarrar; prender; reprimir; restringir; manter dentro de si}
  \end{phonetics}
\end{entry}

\begin{entry}{级}{6}{⽷}
  \begin{phonetics}{级}{ji2}[][HSK 2]
    \definition{clas.}{usado para degraus, escadas, pisos de torres, etc.}
    \definition[个,种]{s.}{nível; classificação; grau; classe | série; turma; qualquer uma das divisões anuais de um curso escolar | degrau}
  \end{phonetics}
\end{entry}

\begin{entry}{纪}{6}{⽷}
  \begin{phonetics}{纪}{ji3}
    \definition*{s.}{Sobrenome Ji}
    \definition{s.}{disciplina | um período de doze anos (na China antiga); um período de anos | (geologia) subdivisão de uma era geológica; período}
    \definition{v.}{colocar por escrito; registrar; mesmo significado de 记, usado principalmente em 记录, 纪年, 纪元, 纪传, etc. | classificar (fios de seda)}
  \seealsoref{记}{ji4}
  \seealsoref{纪传}{ji4 zhuan4}
  \seealsoref{记录}{ji4lu4}
  \seealsoref{纪年}{ji4nian2}
  \seealsoref{纪元}{ji4yuan2}
  \end{phonetics}
  \begin{phonetics}{纪}{ji4}
    \definition*{s.}{Sobrenome Ji}
    \definition{s.}{disciplina | idade; época | (geologia) período | um período de doze anos (na China antiga); um período de anos | (geologia) subdivisão de uma era geológica}
    \definition{v.}{colocar por escrito; registrar | registrar, mesmo significado de 记, usado principalmente em 记录, 纪年, 纪元, 纪传, etc. | classificar (fios de seda)}
  \seealsoref{记}{ji4}
  \seealsoref{纪传}{ji4 zhuan4}
  \seealsoref{记录}{ji4lu4}
  \seealsoref{纪年}{ji4nian2}
  \seealsoref{纪元}{ji4yuan2}
  \end{phonetics}
\end{entry}

\begin{entry}{纪元}{6,4}{⽷、⼉}
  \begin{phonetics}{纪元}{ji4yuan2}
    \definition{s.}{o início de uma era (por exemplo, o reinado de um imperador) | época; era}
  \end{phonetics}
\end{entry}

\begin{entry}{纪传}{6,6}{⽷、⼈}
  \begin{phonetics}{纪传}{ji4 zhuan4}
    \definition{s.}{crônica; biografia}
  \end{phonetics}
\end{entry}

\begin{entry}{纪传体}{6,6,7}{⽷、⼈、⼈}
  \begin{phonetics}{纪传体}{ji4 zhuan4 ti3}
    \definition{s.}{história apresentada em uma série de biografias | gênero histórico baseado em biografia}
  \end{phonetics}
\end{entry}

\begin{entry}{纪年}{6,6}{⽷、⼲}
  \begin{phonetics}{纪年}{ji4nian2}
    \definition{s.}{cronologia; uma maneira de numerar os anos | registro cronológico de eventos; anais; um dos gêneros de livros históricos é organizar fatos históricos em ordem cronológica}
  \end{phonetics}
\end{entry}

\begin{entry}{纪录}{6,8}{⽷、⼹}
  \begin{phonetics}{纪录}{ji4lu4}[][HSK 3]
    \definition[项,个]{s.}{recorde (esportes); o número mais alto ou mais baixo registrado em um determinado período de tempo}
  \end{phonetics}
\end{entry}

\begin{entry}{纪念}{6,8}{⽷、⼼}
  \begin{phonetics}{纪念}{ji4nian4}[][HSK 3]
    \definition[个,次]{s.}{lembrança; recordação; usado para representar uma lembrança (objeto)}
    \definition{v.}{comemorar; expressar saudade por pessoas ou coisas através de objetos ou ações}
  \end{phonetics}
\end{entry}

\begin{entry}{纪律}{6,9}{⽷、⼻}
  \begin{phonetics}{纪律}{ji4lv4}[][HSK 4]
    \definition{s.}{disciplina; código de conduta que cada membro da vida coletiva deve observar}
  \end{phonetics}
\end{entry}

\begin{entry}{系}{7}{⽷}
  \begin{phonetics}{系}{ji4}
    \definition{v.}{amarrar; prender; abotoar; dar um nó}
  \end{phonetics}
  \begin{phonetics}{系}{xi4}[][HSK 3,4]
    \definition*{s.}{Sobrenome Xi}
    \definition{s.}{sistema; série | departamento; faculdade; unidades administrativas de ensino divididas por disciplina nas instituições de ensino superior}
    \definition{v.}{relacionar-se com; suportar; depender de | sentir-se ansioso; estar preocupado | amarrar; prender | ser; expressa julgamento, equivalente a 是}
  \seealsoref{是}{shi4}
  \end{phonetics}
\end{entry}

\begin{entry}{系囚}{7,5}{⽷、⼞}
  \begin{phonetics}{系囚}{xi4qiu2}
    \definition{s.}{prisioneiro}
  \end{phonetics}
\end{entry}

\begin{entry}{系列}{7,6}{⽷、⼑}
  \begin{phonetics}{系列}{xi4lie4}[][HSK 4]
    \definition{s.}{série; conjunto; conjunto de coisas relacionadas (matemática)}
  \end{phonetics}
\end{entry}

\begin{entry}{系统}{7,9}{⽷、⽷}
  \begin{phonetics}{系统}{xi4tong3}[][HSK 4]
    \definition{adj.}{sistemático; organizado}
    \definition[个]{s.}{sistema; relação de tipos semelhantes (ou seja, grupo de coisas semelhantes)}
  \end{phonetics}
\end{entry}

\begin{entry}{纯}{7}{⽷}
  \begin{phonetics}{纯}{chun2}[][HSK 4]
    \definition{adj.}{puro; não misturado; livre de impurezas | simples; puro e simples | habilidoso; proficiente; bem versado}
    \definition{adv.}{puramente; completamente; totalmente | genuinamente}
  \end{phonetics}
\end{entry}

\begin{entry}{纯净水}{7,8,4}{⽷、⼎、⽔}
  \begin{phonetics}{纯净水}{chun2 jing4 shui3}[][HSK 4]
    \definition{s.}{água purificada}
  \end{phonetics}
\end{entry}

\begin{entry}{纯真}{7,10}{⽷、⼗}
  \begin{phonetics}{纯真}{chun2zhen1}
    \definition{adj.}{inocente e não afetado | puro e não adulterado}
    \definition{s.}{inocência}
  \end{phonetics}
\end{entry}

\begin{entry}{纵}{7}{⽷}
  \begin{phonetics}{纵}{zong4}
    \definition{adj.}{de norte a sul; geograficamente norte-sul | longitudinal | vertical; horizontal; paralelo ao lado longo do objeto | amassado; com rugas}
    \definition{conj.}{embora; mesmo que}
    \definition{v.}{libertar; deixar ir | entregar-se a; deixar-se levar | pular; saltar}
  \end{phonetics}
\end{entry}

\begin{entry}{纷}{7}{⽷}
  \begin{phonetics}{纷}{fen1}
    \definition[场]{adj.}{confuso; emaranhado; desordenado | muitos e variados; profusos; numerosos}
  \end{phonetics}
\end{entry}

\begin{entry}{纷纷}{7,7}{⽷、⽷}
  \begin{phonetics}{纷纷}{fen1fen1}[][HSK 4]
    \definition{adj.}{numeroso e confuso; muitos e desordenados}
    \definition{adv.}{um após o outro; em sucessão; em rápida sucessão}
  \end{phonetics}
\end{entry}

\begin{entry}{纸}{7}{⽷}
  \begin{phonetics}{纸}{zhi3}[][HSK 2]
    \definition{clas.}{usado para documentos, cartas, etc.}
    \definition[张,沓]{s.}{papel; uma folha fina de material usada para escrever, pintar, imprimir, embalar, etc., feita principalmente de fibras vegetais | papel joss; papel de incenso; refere-se especificamente a itens supersticiosos, como papel-moeda}
  \end{phonetics}
\end{entry}

\begin{entry}{纸巾}{7,3}{⽷、⼱}
  \begin{phonetics}{纸巾}{zhi3jin1}
    \definition[张,包]{s.}{lenço | guardanapo | papel toalha}
  \end{phonetics}
\end{entry}

\begin{entry}{纸币}{7,4}{⽷、⼱}
  \begin{phonetics}{纸币}{zhi3bi4}
    \definition[张]{s.}{nota (dinheiro) | cédula}
  \end{phonetics}
\end{entry}

\begin{entry}{纸尿裤}{7,7,12}{⽷、⼫、⾐}
  \begin{phonetics}{纸尿裤}{zhi3niao4ku4}
    \definition{s.}{fralda descartável}
  \end{phonetics}
\end{entry}

\begin{entry}{纸张}{7,7}{⽷、⼸}
  \begin{phonetics}{纸张}{zhi3zhang1}
    \definition{s.}{papel}
  \end{phonetics}
\end{entry}

\begin{entry}{纸烟}{7,10}{⽷、⽕}
  \begin{phonetics}{纸烟}{zhi3yan1}
    \definition{s.}{cigarro}
  \end{phonetics}
\end{entry}

\begin{entry}{纹}{7}{⽷}
  \begin{phonetics}{纹}{wen2}
    \definition[个]{s.}{linhas; veios; grãos; rugas na pele | padrão; desenho em tecido de seda; listras ou padrões em tecidos de seda; geralmente se refere a padrões lineares na superfície de um objeto}
  \end{phonetics}
\end{entry}

\begin{entry}{纹路}{7,13}{⽷、⾜}
  \begin{phonetics}{纹路}{wen2lu4}
    \definition{s.}{padrão de linhas | rugas | veias | veias (em mármore ou impressão digital) | grãos (em madeira, etc.)}
  \end{phonetics}
\end{entry}

\begin{entry}{线}{8}{⽷}
  \begin{phonetics}{线}{xian4}[][HSK 3]
    \definition{clas.}{usado para coisas abstratas, o número é limitado a ``一''}
    \definition[根,个]{s.}{fio; corda; arame; objetos finos e longos feitos de seda, algodão, metal, etc. | linha; figura formada pelo movimento arbitrário de um ponto| feito de fio de algodão | algo em forma de linha, fio, etc. | rota de transporte; linha | linha de demarcação; limite; zona de fronteira; zona de transição | beira; borda | linha ideológica e política | pista; fio}
  \end{phonetics}
\end{entry}

\begin{entry}{线香}{8,9}{⽷、⾹}
  \begin{phonetics}{线香}{xian4xiang1}
    \definition{s.}{bastão ou vareta de incenso}
  \end{phonetics}
\end{entry}

\begin{entry}{线索}{8,10}{⽷、⽷}
  \begin{phonetics}{线索}{xian4suo3}[][HSK 5]
    \definition[条,个]{s.}{pista; fio; metáfora para o desenvolvimento das coisas ou a maneira de explorar um problema | fio; linha; refere-se ao contexto de desenvolvimento do enredo em obras literárias}
  \end{phonetics}
\end{entry}

\begin{entry}{练}{8}{⽷}
  \begin{phonetics}{练}{lian4}[][HSK 2]
    \definition*{s.}{Sobrenome Lian}
    \definition{adj.}{habilidoso; experiente; bem treinado}
    \definition{s.}{seda branca}
    \definition{v.}{tratar, amaciar e branquear a seda por meio de fervura; cozinhar seda crua ou tecidos de seda crua | treinar; praticar; exercitar}
  \end{phonetics}
\end{entry}

\begin{entry}{练习}{8,3}{⽷、⼄}
  \begin{phonetics}{练习}{lian4xi2}[][HSK 2]
    \definition[项,次]{s.}{exercício (em livros); tarefas ou exercícios organizados para consolidar os resultados da aprendizagem}
    \definition{v.}{praticar; exercitar; repitir várias vezes até ficar bem treinado}
  \end{phonetics}
\end{entry}

\begin{entry}{组}{8}{⽷}
  \begin{phonetics}{组}{zu3}[][HSK 2]
    \definition{clas.}{usado para conjuntos, séries, suítes, baterias}
    \definition[个]{s.}{grupo; uma unidade composta por um pequeno número de pessoas}
    \definition{v.}{formar; organizar; combinar pessoas ou coisas dispersas em um todo ou sistema}
  \end{phonetics}
\end{entry}

\begin{entry}{组长}{8,4}{⽷、⾧}
  \begin{phonetics}{组长}{zu3 zhang3}[][HSK 2]
    \definition[名,位,个]{s.}{líder de grupo; um supervisor de grupo}
  \end{phonetics}
\end{entry}

\begin{entry}{组合}{8,6}{⽷、⼝}
  \begin{phonetics}{组合}{zu3he2}[][HSK 3]
    \definition{s.}{associação; combinação; o todo organizado | combinação; retirar n elementos diferentes de m elementos e agrupá-los, independentemente da ordem, em que cada grupo contenha pelo menos um elemento diferente, o resultado obtido é chamado de combinação de n elementos de m}
    \definition{v.}{compor; constituir; formar}
  \end{phonetics}
\end{entry}

\begin{entry}{组成}{8,6}{⽷、⼽}
  \begin{phonetics}{组成}{zu3cheng2}[][HSK 2]
    \definition{v.}{formar; compor; inventar}
  \end{phonetics}
\end{entry}

\begin{entry}{组织}{8,8}{⽷、⽷}
  \begin{phonetics}{组织}{zu3zhi1}[][HSK 5]
    \definition{s.}{organização; um coletivo ou grupo estabelecido de acordo com determinados objetivos e princípios | sistema organizado; vários fatores interligados de determinada maneira, formando um sistema | tecer; a combinação de linhas horizontais e verticais nos têxteis | tecido; os seres humanos, os animais, as plantas e outros seres vivos são compostos por uma combinação de células com formas e funções semelhantes, que formam os tecidos; os tecidos são as unidades que compõem os diversos órgãos}
  \end{phonetics}
\end{entry}

\begin{entry}{细}{8}{⽷}
  \begin{phonetics}{细}{xi4}[][HSK 4]
    \definition{adj.}{fino; delgado; esguio; esbelto; em oposição a 粗 | fino; em partículas pequenas; grãos pequenos | fino e macio;  um sussuro | fino; requintado; delicado | cuidadoso; detalhado; meticuloso | ínfimo; minúsculo; insignificante; diminuto | jovem; pequeno}
  \seealsoref{粗}{cu1}
  \end{phonetics}
\end{entry}

\begin{entry}{细节}{8,5}{⽷、⾋}
  \begin{phonetics}{细节}{xi4jie2}[][HSK 4]
    \definition{s.}{detalhe; particularidade; aspectos secundários ou partes sutis de um enredo ou episódios secundários usados em uma obra literária para expressar o caráter de uma pessoa ou as características essenciais de uma coisa}
  \end{phonetics}
\end{entry}

\begin{entry}{细致}{8,10}{⽷、⾄}
  \begin{phonetics}{细致}{xi4zhi4}[][HSK 4]
    \definition{adj.}{meticuloso; cuidadoso; minucioso | intrincado; delicado}
  \end{phonetics}
\end{entry}

\begin{entry}{细菌战}{8,11,9}{⽷、⾋、⼽}
  \begin{phonetics}{细菌战}{xi4jun1zhan4}
    \definition{s.}{guerra biológica}
  \end{phonetics}
\end{entry}

\begin{entry}{织}{8}{⽷}
  \begin{phonetics}{织}{zhi1}[][HSK 6]
    \definition{v.}{tecer; fazer fios ou linhas cruzarem para fazer seda, tecido, lã, etc. | tricotar; usar agulhas para fazer fios ou linhas entrelaçados para confeccionar suéteres, meias, rendas, redes, etc. | sobrepor-se; entrelaçar-se; cruzar; entrelaçar}
  \end{phonetics}
\end{entry}

\begin{entry}{终}{8}{⽷}
  \begin{phonetics}{终}{zhong1}
    \definition*{s.}{Sobrenome Zhong}
    \definition{adj.}{tudo; todo; inteiro; o tempo todo do começo ao fim}
    \definition{adv.}{afinal; no final; eventualmente; finalmente}
    \definition{s.}{fim; término | tempo todo; período inteiro; o tempo todo | final | morte; refere-se à morte}
  \end{phonetics}
\end{entry}

\begin{entry}{终于}{8,3}{⽷、⼆}
  \begin{phonetics}{终于}{zhong1yu2}[][HSK 3]
    \definition{adv.}{finalmente; por fim; eventualmente; no final; indica uma situação que surge após várias mudanças ou espera}
  \end{phonetics}
\end{entry}

\begin{entry}{终止}{8,4}{⽷、⽌}
  \begin{phonetics}{终止}{zhong1zhi3}[][HSK 5]
    \definition{v.}{parar; terminar | anular; encerrar; expirar; revogar}
  \end{phonetics}
\end{entry}

\begin{entry}{终究}{8,7}{⽷、⽳}
  \begin{phonetics}{终究}{zhong1jiu1}
    \definition{adv.}{afinal de contas; enfatiza que, não importa o que aconteça, a natureza das pessoas e das coisas não mudará e que as características básicas devem ser reconhecidas (tem o efeito de fortalecer o tom) |  no final; indica que um determinado resultado ocorrerá ou não, frequentemente usado em especulações, julgamentos etc. | afinal de contas; indica que, apesar do grande esforço ou da grande esperança, o resultado objetivo ainda é insatisfatório, geralmente com o significado de pesar ou pena | afinal de contas; indica que um resultado desejado finalmente aparece}
  \end{phonetics}
\end{entry}

\begin{entry}{终身}{8,7}{⽷、⾝}
  \begin{phonetics}{终身}{zhong1shen1}[][HSK 5]
    \definition{s.}{vida inteira; por toda a vida; por toda a vida}
  \end{phonetics}
\end{entry}

\begin{entry}{终点}{8,9}{⽷、⽕}
  \begin{phonetics}{终点}{zhong1dian3}[][HSK 5]
    \definition[个]{s.}{destino; ponto terminal; ponto de chegada; lugar onde uma jornada termina | final; refere-se especificamente ao local onde a corrida é interrompida}
  \end{phonetics}
\end{entry}

\begin{entry}{经}{8}{⽷}
  \begin{phonetics}{经}{jing1}
    \definition*{s.}{Sobrenome Jing}
    \definition{s.}{livro sagrado | escritura | clássicos | longitude | menstruação | canal}
    \definition{v.}{passar | sofrer | suportar | deformar (têxtil)}
  \end{phonetics}
\end{entry}

\begin{entry}{经历}{8,4}{⽷、⼚}
  \begin{phonetics}{经历}{jing1li4}[][HSK 3]
    \definition[个,次,段,种]{s.}{experiência; coisas que você viu, fez ou sofreu pessoalmente}
    \definition{v.}{passar por; atravessar; ter visto, feito ou sofrido pessoalmente}
  \end{phonetics}
\end{entry}

\begin{entry}{经过}{8,6}{⽷、⾡}
  \begin{phonetics}{经过}{jing1guo4}[][HSK 2]
    \definition{prep.}{depois; através; como resultado de; passar por uma atividade ou evento que traz novas mudanças para pessoas ou coisas}
    \definition[个,段,番]{s.}{processo; curso; experiência}
    \definition{v.}{passar; atravessar; passar por; através de (local, tempo, ação, etc.)}
  \end{phonetics}
\end{entry}

\begin{entry}{经典}{8,8}{⽷、⼋}
  \begin{phonetics}{经典}{jing1dian3}[][HSK 4]
    \definition{adj.}{clássico; (escritos ou obras, etc.) que são típicos, autorizados}
    \definition{s.}{clássicos; escritos tradicionais e valiosos; os livros mais importantes e fundamentais da religião | escrituras; escritos de doutrinas religiosas}
  \end{phonetics}
\end{entry}

\begin{entry}{经济}{8,9}{⽷、⽔}
  \begin{phonetics}{经济}{jing1ji4}[][HSK 3]
    \definition{adj.}{econômico;  parcimonioso; descreve algo que custa pouco e rende muito; preço acessível}
    \definition{s.}{economia; a soma das relações de produção social em um determinado período histórico|econômico; de valor industrial ou econômico; refere-se à economia nacional; também se refere a um determinado setor da economia nacional | economia; refere-se às atividades econômicas, incluindo produção, circulação, distribuição e consumo, bem como atividades ou processos financeiros, de seguros, etc. | renda; situação financeira; refere-se à situação financeira de uma pessoa}
    \definition{v.}{governar o país e beneficiar o povo}
  \end{phonetics}
\end{entry}

\begin{entry}{经费}{8,9}{⽷、⾙}
  \begin{phonetics}{经费}{jing1fei4}[][HSK 5]
    \definition[笔]{s.}{fundos; desembolso; gastos | despesas; gastos}
  \end{phonetics}
\end{entry}

\begin{entry}{经验}{8,10}{⽷、⾺}
  \begin{phonetics}{经验}{jing1yan4}[][HSK 3]
    \definition[个,次,种]{s.}{experiência; conhecimento ou habilidades adquiridos através da prática}
    \definition{v.}{experimentar; passar por; ter visto, feito ou sofrido pessoalmente}
  \end{phonetics}
\end{entry}

\begin{entry}{经常}{8,11}{⽷、⼱}
  \begin{phonetics}{经常}{jing1chang2}[][HSK 2]
    \definition{adj.}{habitual; cotidiano; diário; do dia a dia}
    \definition{adv.}{frequentemente; regularmente; constantemente; com frequência; indica que a ação ocorre repetidamente}
  \end{phonetics}
\end{entry}

\begin{entry}{经理}{8,11}{⽷、⽟}
  \begin{phonetics}{经理}{jing1li3}[][HSK 2]
    \definition[个,位,名]{s.}{gerente; diretor; pessoas responsáveis pela gestão e administração de empresas ou corporações}
  \end{phonetics}
\end{entry}

\begin{entry}{经营}{8,11}{⽷、⾋}
  \begin{phonetics}{经营}{jing1ying2}[][HSK 3]
    \definition{v.}{executar; gerenciar; operar; envolver-se em; planejar e gerenciar (empresas, etc.) | gerenciar; refere-se a planos e organizações em geral}
  \end{phonetics}
\end{entry}

\begin{entry}{结}{9}{⽷}
  \begin{phonetics}{结}{jie1}
    \definition{v.}{dar (frutos); formar (sementes); produzir frutos ou sementes (uma planta)}
  \end{phonetics}
  \begin{phonetics}{结}{jie2}[][HSK 4]
    \definition*{s.}{Sobrenome Jie}
    \definition{s.}{nó | declaração juramentada; garantia por escrito; documento que, antigamente, significava um reconhecimento de encerramento ou uma garantia de responsabilidade}
    \definition{v.}{amarrar; tricotar; dar nó; tecer | formar; forjar; cimentar; solidificar | resolver; concluir | combinar; formar um relacionamento}
  \end{phonetics}
\end{entry}

\begin{entry}{结合}{9,6}{⽷、⼝}
  \begin{phonetics}{结合}{jie2he2}[][HSK 3]
    \definition{v.}{ligar; unir; combinar; integrar; formar uma relação estreita entre pessoas ou coisas | casar-se; unir-se em matrimônio; referir-se especificamente a casais}
  \end{phonetics}
\end{entry}

\begin{entry}{结论}{9,6}{⽷、⾔}
  \begin{phonetics}{结论}{jie2lun4}[][HSK 4]
    \definition[个]{s.}{conclusão; palavra final sobre uma pessoa ou coisa após investigação e pesquisa | veredito; julgamento deduzido de premissas também é chamado de conclusão}
  \end{phonetics}
\end{entry}

\begin{entry}{结局}{9,7}{⽷、⼫}
  \begin{phonetics}{结局}{jie2ju2}
    \definition{s.}{conclusão | fim | final}
  \end{phonetics}
\end{entry}

\begin{entry}{结束}{9,7}{⽷、⽊}
  \begin{phonetics}{结束}{jie2shu4}[][HSK 3]
    \definition{v.}{finalizar; fechar; terminar; concluir; encerrar; desenvolver ou avançar até a fase final, sem continuidade}
  \end{phonetics}
\end{entry}

\begin{entry}{结束工作}{9,7,3,7}{⽷、⽊、⼯、⼈}
  \begin{phonetics}{结束工作}{jie2shu4gong1zuo4}
    \definition{s.}{trabalho final}
    \definition{v.}{terminar o trabalho}
  \end{phonetics}
\end{entry}

\begin{entry}{结束区}{9,7,4}{⽷、⽊、⼖}
  \begin{phonetics}{结束区}{jie2shu4 qu1}
    \definition{s.}{zona final}
  \end{phonetics}
\end{entry}

\begin{entry}{结束文本}{9,7,4,5}{⽷、⽊、⽂、⽊}
  \begin{phonetics}{结束文本}{jie2shu4 wen2ben3}
    \definition{s.}{texto final}
  \end{phonetics}
\end{entry}

\begin{entry}{结束剂}{9,7,8}{⽷、⽊、⼑}
  \begin{phonetics}{结束剂}{jie2shu4 ji4}
    \definition{s.}{finalizador}
  \end{phonetics}
\end{entry}

\begin{entry}{结束语}{9,7,9}{⽷、⽊、⾔}
  \begin{phonetics}{结束语}{jie2shu4yu3}
    \definition{s.}{conclusões finais | considerações finais}
  \end{phonetics}
\end{entry}

\begin{entry}{结束辩论}{9,7,16,6}{⽷、⽊、⾟、⾔}
  \begin{phonetics}{结束辩论}{jie2shu4 bian4 lun4}
    \definition{s.}{debate de encerramento}
  \end{phonetics}
\end{entry}

\begin{entry}{结社自由}{9,7,6,5}{⽷、⽰、⾃、⽥}
  \begin{phonetics}{结社自由}{jie2she4zi4you2}
    \definition{s.}{(constitucional) liberdade de associação}
  \end{phonetics}
\end{entry}

\begin{entry}{结实}{9,8}{⽷、⼧}
  \begin{phonetics}{结实}{jie1shi5}[][HSK 3]
    \definition{adj.}{sólido; resistente; durável | forte; resistente; robusto}
  \end{phonetics}
\end{entry}

\begin{entry}{结构}{9,8}{⽷、⽊}
  \begin{phonetics}{结构}{jie2gou4}[][HSK 4]
    \definition[个,座]{s.}{estrutura; composição; construção; formação; constituição; tecido; forma; sistematização; mecânica; organização | arquitetura; estrutura; construção; construção de partes de edifícios com suporte de carga ou com carga externa | textura (geológico)}
  \end{phonetics}
\end{entry}

\begin{entry}{结果}{9,8}{⽷、⽊}
  \begin{phonetics}{结果}{jie1guo3}
    \definition{v.}{dar frutos}
  \end{phonetics}
  \begin{phonetics}{结果}{jie2guo3}[][HSK 2]
    \definition{conj.}{como resultado | no final}
    \definition{s.}{resultado | conclusão | consequência}
    \definition{v.}{despachar | matar}
  \end{phonetics}
\end{entry}

\begin{entry}{结婚}{9,11}{⽷、⼥}
  \begin{phonetics}{结婚}{jie2hun1}[][HSK 3]
    \definition{v.+compl.}{casar; casar-se; casar-se bem;}
  \end{phonetics}
\end{entry}

\begin{entry}{结婚礼服}{9,11,5,8}{⽷、⼥、⽰、⽉}
  \begin{phonetics}{结婚礼服}{jie2hun1 li3 fu2}
    \definition{s.}{vestido de casamento}
  \end{phonetics}
\end{entry}

\begin{entry}{绕}{9}{⽷}
  \begin{phonetics}{绕}{rao4}[][HSK 5]
    \definition*{s.}{Sobrenome Rao}
    \definition{v.}{enrolar; bobinar; rebobinar | mover-se em círculo; girar; revolver | fazer um desvio; contornar; dar a volta | confundir; desorientar}
  \end{phonetics}
\end{entry}

\begin{entry}{绘}{9}{⽷}
  \begin{phonetics}{绘}{hui4}
    \definition{v.}{pintar; desenhar}
  \end{phonetics}
\end{entry}

\begin{entry}{绘画}{9,8}{⽷、⽥}
  \begin{phonetics}{绘画}{hui4 hua4}[][HSK 6]
    \definition{s.}{desenho; pintura}
    \definition{v.}{desenhar; pintar}
  \end{phonetics}
\end{entry}

\begin{entry}{给}{9}{⽷}
  \begin{phonetics}{给}{gei3}[][HSK 1]
    \definition{prep.}{por; expressa significado passivo; tem o mesmo significado que 被, 叫; pode ser seguido pelo agente da ação; o agente da ação também pode não aparecer na frase | para; a; seguido por quem se beneficia da ação; igual a 为 | em direção a; seguido pelo destinatário da ação; o mesmo que 向 | indica transmissão}
    \definition{v.}{dar; conceder; fazer com que a outra parte obtenha algo | passar; pagar; indicar que a outra pessoa faça algo | deixar; permitir que alguém faça algo; autorizar alguém a fazer algo}
    \definition{v.aux.}{usado antes de verbos predicativos que expressam passividade, disposição, etc., para reforçar o tom}
  \seealsoref{被}{bei4}
  \seealsoref{叫}{jiao4}
  \seealsoref{为}{wei4}
  \seealsoref{向}{xiang4}
  \end{phonetics}
  \begin{phonetics}{给}{ji3}
    \definition{adj.}{abundante; próspero; bem provido para}
    \definition{v.}{fornecer; prover}
  \end{phonetics}
\end{entry}

\begin{entry}{给予}{9,4}{⽷、⼅}
  \begin{phonetics}{给予}{ji3yu3}[][HSK 6]
    \definition{v.}{dar; conceder; dar em troca}
  \end{phonetics}
\end{entry}

\begin{entry}{给……打电话}{9,5,5,8}{⽷、⼿、⽥、⾔}
  \begin{phonetics}{给……打电话}{gei3 da3 dian4 hua4}
    \definition{expr.}{telefonar para alguém}
  \seealsoref{打电话}{da3 dian4 hua4}
  \end{phonetics}
\end{entry}

\begin{entry}{绝}{9}{⽷}
  \begin{phonetics}{绝}{jue2}[][HSK 6]
    \definition{adj.}{exausto; esgotado; acabado | desesperado; sem esperança | único; soberbo; incomparável | não deixar margem de manobra; não fazer concessões; intransigente}
    \definition{adv.}{extremamente; mais | (antes de uma negativa) absolutamente; no mínimo; por qualquer meio; em qualquer conta}
    \definition{s.}{(literário) jueju, um poema de quatro linhas}
    \definition{v.}{cortar; romper | parar de respirar; morrer}
  \end{phonetics}
\end{entry}

\begin{entry}{绝大多数}{9,3,6,13}{⽷、⼤、⼣、⽁}
  \begin{phonetics}{绝大多数}{jue2 da4 duo1 shu4}[][HSK 6]
    \definition{expr.}{maioria absoluta | uma maioria esmagadora}
  \end{phonetics}
\end{entry}

\begin{entry}{绝不}{9,4}{⽷、⼀}
  \begin{phonetics}{绝不}{jue2bu4}
    \definition{adv.}{definitivamente não | de forma alguma | sob nenhuma circunstância}
  \end{phonetics}
\end{entry}

\begin{entry}{绝对}{9,5}{⽷、⼨}
  \begin{phonetics}{绝对}{jue2dui4}[][HSK 3]
    \definition{adj.}{absoluto; sem condições; sem restrições | absoluto; extremo; incompleto; sem margem para negociação ou alteração}
    \definition{adv.}{absolutamente; completamente; com certeza}
  \end{phonetics}
\end{entry}

\begin{entry}{绝招}{9,8}{⽷、⼿}
  \begin{phonetics}{绝招}{jue2zhao1}
    \definition{s.}{habilidade única | movimento delicado inesperado (como último recurso) | golpe de mestre | golpe final}
  \end{phonetics}
\end{entry}

\begin{entry}{绝版}{9,8}{⽷、⽚}
  \begin{phonetics}{绝版}{jue2ban3}
    \definition{adj.}{esgotado | fora de catálogo}
  \end{phonetics}
\end{entry}

\begin{entry}{绝望}{9,11}{⽷、⽉}
  \begin{phonetics}{绝望}{jue2 wang4}[][HSK 5]
    \definition{v.+compl.}{desesperar; desistir de toda esperança; perder toda esperança de}
  \end{phonetics}
\end{entry}

\begin{entry}{统}{9}{⽷}
  \begin{phonetics}{统}{tong3}
    \definition{adv.}{todos; juntos; de forma unificada | inteiramente; totalmente}
    \definition{s.}{interligado; inter-relacionado | sistema interconectado | qualquer parte em forma de tubo de uma peça de roupa, etc.; o mesmo que 筒}
    \definition{v.}{reunir em um; unir | unir; liderar; comandar}
  \seealsoref{筒}{tong3}
  \end{phonetics}
\end{entry}

\begin{entry}{统一}{9,1}{⽷、⼀}
  \begin{phonetics}{统一}{tong3yi1}[][HSK 4]
    \definition{adj.}{unificado; unitário; centralizado; consistente}
    \definition{v.}{unificar; unir; integrar; padronizar}
  \end{phonetics}
\end{entry}

\begin{entry}{统计}{9,4}{⽷、⾔}
  \begin{phonetics}{统计}{tong3ji4}[][HSK 4]
    \definition{v.}{compilar estatísticas; refere-se à realização de trabalho estatístico, ou seja, coletar, reunir, analisar e extrapolar dados sobre um fenômeno | somar; adicionar; contar}
  \end{phonetics}
\end{entry}

\begin{entry}{素}{10}{⽷}
  \begin{phonetics}{素}{su4}
    \definition{adj.}{branco; de cor natural | simples; natural; singelo; de cor simples | nativo; original | normal; usual; geral}
    \definition{adv.}{geralmente; sempre; habitualmente}
    \definition{s.}{vegetais, frutas e outros alimentos (em oposição à 荤) | matéria-prima; matéria-prima básico; tecidos de seda naturais e não processados | elemento; os componentes básicos de algo}
  \seealsoref{荤}{hun1}
  \end{phonetics}
\end{entry}

\begin{entry}{素质}{10,8}{⽷、⾙}
  \begin{phonetics}{素质}{su4zhi4}[][HSK 6]
    \definition[个,种]{s.}{qualidade; características; caráter; o nível físico, moral, mental, intelectual e cultural de uma pessoa}
  \end{phonetics}
\end{entry}

\begin{entry}{索}{10}{⽷}
  \begin{phonetics}{索}{suo3}
    \definition*{s.}{Sobrenome Suo}
    \definition{adj.}{completamente sozinho; sozinho | maçante; insípido; sem significado}
    \definition[根]{s.}{corda; cabo; cordão; corrente | uma corda grande}
    \definition{v.}{(literário) pesquisar | exigir; pedir}
  \end{phonetics}
\end{entry}

\begin{entry}{索性}{10,8}{⽷、⼼}
  \begin{phonetics}{索性}{suo3xing4}
    \definition{adv.}{poderia muito bem | simplesmente | apenas}
  \end{phonetics}
\end{entry}

\begin{entry}{紧}{10}{⽷}
  \begin{phonetics}{紧}{jin3}[][HSK 3]
    \definition{adj.}{tenso; apertado; o estado em que um objeto se encontra após ser submetido a uma grande força de tração ou pressão.| seguro; firme | cerrado; apertado | urgente; premente; tenso | rigoroso; rígido; severo | difícil; sem dinheiro}
    \definition{v.}{apertar; tornar mais apertado}
  \end{phonetics}
\end{entry}

\begin{entry}{紧张}{10,7}{⽷、⼸}
  \begin{phonetics}{紧张}{jin3zhang1}[][HSK 3]
    \definition{adj.}{nervoso; tenso; mentalmente em estado de alerta, excitado e inquieto | apertado; em falta; o que está disponível não satisfaz os requisitos| tenso; intenso; intenso ou urgente, causando tensão mental}
  \end{phonetics}
\end{entry}

\begin{entry}{紧急}{10,9}{⽷、⼼}
  \begin{phonetics}{紧急}{jin3ji2}[][HSK 3]
    \definition{adj./adj.}{urgente; premente; crítico}
  \end{phonetics}
\end{entry}

\begin{entry}{紧紧}{10,10}{⽷、⽷}
  \begin{phonetics}{紧紧}{jin3 jin3}[][HSK 5]
    \definition{adv.}{firmemente; estreitamente; apertadamente; prestar muita atenção (em algo)}
  \end{phonetics}
\end{entry}

\begin{entry}{紧密}{10,11}{⽷、⼧}
  \begin{phonetics}{紧密}{jin3 mi4}[][HSK 4]
    \definition{adj.}{próximos; inseparáveis | incessante; rápido e intenso}
  \end{phonetics}
\end{entry}

\begin{entry}{绣}{10}{⽷}
  \begin{phonetics}{绣}{xiu4}
    \definition{s.}{bordado}
    \definition{v.}{bordar}
  \end{phonetics}
\end{entry}

\begin{entry}{继}{10}{⽷}
  \begin{phonetics}{继}{ji4}
    \definition{adv.}{então; depois}
    \definition{s.}{filhos; prole}
    \definition{v.}{continuar; ter sucesso; seguir}
  \end{phonetics}
\end{entry}

\begin{entry}{继承}{10,8}{⽷、⼿}
  \begin{phonetics}{继承}{ji4cheng2}[][HSK 5]
    \definition{v.}{herdar (o patrimônio de uma pessoa falecida, etc.) de acordo com a lei | continuar; geralmente se refere à aceitação do estilo, da cultura, do conhecimento, etc., daqueles que nos precederam | continuar; os descendentes continuam o trabalho deixado por seus antecessores.}
  \end{phonetics}
\end{entry}

\begin{entry}{继续}{10,11}{⽷、⽷}
  \begin{phonetics}{继续}{ji4xu4}[][HSK 3]
    \definition{s.}{continuação}
    \definition{v.}{continuar; prosseguir | prosseguir; continuar; seguir em frente (com); (atividades, eventos, etc.) continuar após uma pausa ou um determinado período de tempo}
  \end{phonetics}
\end{entry}

\begin{entry}{累}{11}{⽷}
  \begin{phonetics}{累}{lei2}
    \definition*{s.}{Sobrenome Lei}
    \definition{adj.}{incômodo; complicado}
    \definition{s.}{corda; cordão | touro na época de acasalamento}
    \definition{v.}{amarrar; prender; atar | copular}
  \end{phonetics}
  \begin{phonetics}{累}{lei3}
    \definition*{s.}{Sobrenome Lei}
    \definition{adj.}{em andamento; repetido; contínuo}
    \definition{v.}{acumular; empilhar; colocar em cima de outro | envolver; implicar | construir empilhando tijolos, pedras, terra, etc.}
  \end{phonetics}
  \begin{phonetics}{累}{lei4}[][HSK 1]
    \definition{adj.}{cansado; exausto; fatigado}
    \definition{v.}{cansar; desgastar; fatigar; esgotar | labutar; trabalhar duro}
  \end{phonetics}
\end{entry}

\begin{entry}{绰}{11}{⽷}
  \begin{phonetics}{绰}{chuo4}
    \definition{adj.}{amplo; espaçoso | (do porte de uma menina) graciosa; flexível}
  \end{phonetics}
\end{entry}

\begin{entry}{绰号}{11,5}{⽷、⼝}
  \begin{phonetics}{绰号}{chuo4hao4}
    \definition{s.}{apelido}
  \end{phonetics}
\end{entry}

\begin{entry}{绳}{11}{⽷}
  \begin{phonetics}{绳}{sheng2}
    \definition*{s.}{Sobrenome Sheng}
    \definition[根]{s.}{corda; cordão; barbante | a linha no marcador de tinta de carpinteiro}
    \definition{v.}{restringir; corrigir; sancionar | medir | continuar}
  \end{phonetics}
\end{entry}

\begin{entry}{绳子}{11,3}{⽷、⼦}
  \begin{phonetics}{绳子}{sheng2zi5}
    \definition[条]{s.}{corda | cordão}
  \end{phonetics}
\end{entry}

\begin{entry}{维}{11}{⽷}
  \begin{phonetics}{维}{wei2}
    \definition*{s.}{Sobrenome Wei}
    \definition{s.}{pensamento | dimensão; conceitos básicos de geometria e teoria do espaço}
    \definition{v.}{ligar; amarrar; manter unido; conectar | manter; manter; salvaguardar; preservar}
  \end{phonetics}
\end{entry}

\begin{entry}{维吾尔}{11,7,5}{⽷、⼝、⼩}
  \begin{phonetics}{维吾尔}{wei2wu2'er3}
    \definition*{s.}{Etnia Uigur de Xinjiang}
  \end{phonetics}
\end{entry}

\begin{entry}{维护}{11,7}{⽷、⼿}
  \begin{phonetics}{维护}{wei2hu4}[][HSK 4]
    \definition{v.}{defender; proteger; manter; preservar}
  \end{phonetics}
\end{entry}

\begin{entry}{维修}{11,9}{⽷、⼈}
  \begin{phonetics}{维修}{wei2xiu1}[][HSK 4]
    \definition{v.}{prestar serviços; manter; reparar; manter em (bom) estado de conservação}
  \end{phonetics}
\end{entry}

\begin{entry}{维持}{11,9}{⽷、⼿}
  \begin{phonetics}{维持}{wei2chi2}[][HSK 4]
    \definition{v.}{manter; conservar; guardar; manter vivo}
  \end{phonetics}
\end{entry}

\begin{entry}{绷}{11}{⽷}
  \begin{phonetics}{绷}{beng1}
    \definition{s.}{estrutura de cama amarrada com cordas, tiras de vime, etc.}
    \definition{v.}{esticar (ou puxar) com força | saltar; quicar | alinhavar; fixar | (dialeto) conseguir fazer algo com dificuldade | (roupas) apertar | costurar ou alfinetar com parcimônia | (dialeto) fraudar; roubar dinheiro}
  \end{phonetics}
  \begin{phonetics}{绷}{beng3}
    \definition{v.}{mostrar uma cara sombria, tensa; parecer descontente | conter o próprio temperamento}
  \end{phonetics}
\end{entry}

\begin{entry}{绷带}{11,9}{⽷、⼱}
  \begin{phonetics}{绷带}{beng1dai4}
    \definition{s.}{curativo | (empréstimo linguístico) \emph{bandage}}
  \end{phonetics}
\end{entry}

\begin{entry}{综}{11}{⽷}
  \begin{phonetics}{综}{zeng4}
    \definition{s.}{liço; fuso; um dispositivo em um tear que separa os fios da urdidura em um padrão alternado para permitir a passagem da lançadeira}
  \end{phonetics}
  \begin{phonetics}{综}{zong1}
    \definition*{s.}{Sobrenome Zong}
    \definition{v.}{reunir; resumir | combinar; reunir}
  \end{phonetics}
\end{entry}

\begin{entry}{综合}{11,6}{⽷、⼝}
  \begin{phonetics}{综合}{zong1he2}[][HSK 4]
    \definition{s.}{síntese}
    \definition{v.}{sintetizar; resumir as partes de uma coisa em um todo unificado após análise (em oposição a 分析); reunir coisas de um tipo ou natureza diferente}
  \seealsoref{分析}{fen1xi1}
  \end{phonetics}
\end{entry}

\begin{entry}{绿}{11}{⽷}
  \begin{phonetics}{绿}{lv4}[][HSK 2]
    \definition*{s.}{Sobrenome Lü}
    \definition{adj.}{verde}
    \definition{v.}{tornar-se verde; ficar verde}
  \end{phonetics}
\end{entry}

\begin{entry}{绿化}{11,4}{⽷、⼔}
  \begin{phonetics}{绿化}{lv4 hua4}[][HSK 6]
    \definition{v.}{tornar verde plantando árvores, flores, etc.; arborizar; reflorestar; plantar árvores, flores e plantas para embelezar o ambiente ou prevenir a erosão do solo}
  \end{phonetics}
\end{entry}

\begin{entry}{绿色}{11,6}{⽷、⾊}
  \begin{phonetics}{绿色}{lv4 se4}[][HSK 2]
    \definition{adj.}{verde; ecológico; sem poluição; em conformidade com os requisitos ambientais}
    \definition{s.}{cor verde}
  \end{phonetics}
\end{entry}

\begin{entry}{绿豆}{11,7}{⽷、⾖}
  \begin{phonetics}{绿豆}{lv4dou4}
    \definition{s.}{vagens}
  \end{phonetics}
\end{entry}

\begin{entry}{绿豆芽}{11,7,7}{⽷、⾖、⾋}
  \begin{phonetics}{绿豆芽}{lv4dou4 ya2}
    \definition{s.}{broto de feijão verde}
  \end{phonetics}
\end{entry}

\begin{entry}{绿茶}{11,9}{⽷、⾋}
  \begin{phonetics}{绿茶}{lv4 cha2}[][HSK 3]
    \definition{s.}{chá verde; chá produzido apenas através dos processos de maturação, enrolamento (ou sem enrolamento) e secagem, sem passar por fermentação, com cor verde-claro}
  \end{phonetics}
\end{entry}

\begin{entry}{紫}{12}{⽷}
  \begin{phonetics}{紫}{zi3}[][HSK 5]
    \definition*{s.}{Sobrenome Zi}
    \definition{adj.}{roxo; púrpura; violeta; cor resultante da combinação do vermelho e do azul}
  \end{phonetics}
\end{entry}

\begin{entry}{紫色}{12,6}{⽷、⾊}
  \begin{phonetics}{紫色}{zi3 se4}
    \definition{s.}{cor púrpura | cor violeta}
  \end{phonetics}
\end{entry}

\begin{entry}{絫}{12}{⽷}
  \begin{phonetics}{絫}{lei3}
    \variantof{累}
  \end{phonetics}
\end{entry}

\begin{entry}{编}{12}{⽷}
  \begin{phonetics}{编}{bian1}[][HSK 4]
    \definition*{s.}{Sobrenome Bian}
    \definition{s.}{livro; volume; parte de um livro | organização e pessoal; estabelecimento}
    \definition{v.}{tecer; trançar; entrançar | fazer uma lista; organizar em uma lista; organizar; agrupar | editar; compilar | compor; escrever | fabricar; inventar; fazer; preparar}
  \end{phonetics}
\end{entry}

\begin{entry}{编制}{12,8}{⽷、⼑}
  \begin{phonetics}{编制}{bian1 zhi4}[][HSK 6]
    \definition{s.}{estabelecimento; organização e pessoal; refere-se à estrutura organizacional de uma unidade, cotas de pessoal, alocação de tarefas, etc.}
    \definition{v.}{tecer; trançar; entrelaçar tiras de vime, salgueiro, bambu, etc. para fazer objetos | resolver; realizar; elaborar; fazer de acordo com os dados (procedimentos, planos, etc.)}
  \end{phonetics}
\end{entry}

\begin{entry}{编程}{12,12}{⽷、⽲}
  \begin{phonetics}{编程}{bian1cheng2}
    \definition{v.}{programar computador}
  \end{phonetics}
\end{entry}

\begin{entry}{编辑}{12,13}{⽷、⾞}
  \begin{phonetics}{编辑}{bian1ji2}[][HSK 5]
    \definition{v.}{editar; compilar; organizar e processar dados ou trabalhos existentes}
  \end{phonetics}
  \begin{phonetics}{编辑}{bian1ji5}[][HSK 5]
    \definition{s.}{editor; compilador; pessoa que organiza e processa dados ou trabalhos existentes}
  \end{phonetics}
\end{entry}

\begin{entry}{缘}{12}{⽷}
  \begin{phonetics}{缘}{yuan2}
    \definition{s.}{causa | razão | karma | destino | predestinação}
  \end{phonetics}
\end{entry}

\begin{entry}{缘分}{12,4}{⽷、⼑}
  \begin{phonetics}{缘分}{yuan2fen4}
    \definition{s.}{destino ou acaso que une as pessoas | afinidade ou relacionamento predestinado | destino (Budismo)}
  \end{phonetics}
\end{entry}

\begin{entry}{缝}{13}{⽷}
  \begin{phonetics}{缝}{feng2}
    \definition{v.}{costurar}
  \end{phonetics}
  \begin{phonetics}{缝}{feng4}
    \definition[道]{s.}{costura | fenda; rachadura; fissura; brecha}
  \end{phonetics}
\end{entry}

\begin{entry}{缝纫}{13,6}{⽷、⽷}
  \begin{phonetics}{缝纫}{feng2ren4}
    \definition{v.}{costurar}
  \end{phonetics}
\end{entry}

\begin{entry}{缝纫机}{13,6,6}{⽷、⽷、⽊}
  \begin{phonetics}{缝纫机}{feng2ren4ji1}
    \definition[架]{s.}{máquina de costura}
  \end{phonetics}
\end{entry}

\begin{entry}{缩}{14}{⽷}
  \begin{phonetics}{缩}{suo1}
    \definition*{s.}{Sobrenome Suo}
    \definition{v.}{contrair; encolher | recuar; retirar-se | economizar}
  \end{phonetics}
\end{entry}

\begin{entry}{缩小}{14,3}{⽷、⼩}
  \begin{phonetics}{缩小}{suo1 xiao3}[][HSK 4]
    \definition{v.}{reduzir, estreitar, encolher;  tornar menor (em oposição a 放大)}
  \seealsoref{放大}{fang4da4}
  \end{phonetics}
\end{entry}

\begin{entry}{缩短}{14,12}{⽷、⽮}
  \begin{phonetics}{缩短}{suo1duan3}[][HSK 4]
    \definition{v.}{encurtar; reduzir; diminuir}
  \end{phonetics}
\end{entry}

\begin{entry}{缩影卡片}{14,15,5,4}{⽷、⼺、⼘、⽚}
  \begin{phonetics}{缩影卡片}{suo1ying3 ka3pian4}
    \definition{s.}{cartão em miniatura}
  \end{phonetics}
\end{entry}

\begin{entry}{繁}{17}{⽷}
  \begin{phonetics}{繁}{fan2}
    \definition{adj.}{em grande número; numerosos; múltiplos (oposto a 简) | em grande número; numerosos; complexos; complicado}
    \definition{v.}{propagar; multiplicar}
  \seealsoref{简}{jian3}
  \end{phonetics}
\end{entry}

\begin{entry}{繁荣}{17,9}{⽷、⾋}
  \begin{phonetics}{繁荣}{fan2rong2}[][HSK 5]
    \definition{adj.}{florescente; próspero}
    \definition{v.}{promover; prosperar}
  \end{phonetics}
\end{entry}

\begin{entry}{繁殖}{17,12}{⽷、⽍}
  \begin{phonetics}{繁殖}{fan2zhi2}[][HSK 6]
    \definition{v.}{criar; reproduzir; propagar; multiplicar; os organismos produzem novos indivíduos}
  \end{phonetics}
\end{entry}

%%%%% EOF %%%%%

