%%%
%%% Radical "⽷"
%%%

\section*{Radical 120: ``⽷'' (纟、糹)}\addcontentsline{toc}{section}{Radical 120: ⽷、纟、糹}

\begin{entry}{纠葛}{5,12}{⽷、⾋}
  \begin{phonetics}{纠葛}{jiu1ge2}
    \definition{s.}{emaranhado | disputa}
  \end{phonetics}
\end{entry}

\begin{entry}{红}{6}{⽷}
  \begin{phonetics}{红}{hong2}[][HSK 2]
    \definition*{s.}{sobrenome Hong}
    \definition{adj.}{vermelho | popular | revolucionário}
    \definition{s.}{bônus}
  \end{phonetics}
\end{entry}

\begin{entry}{红包}{6,5}{⽷、⼓}
  \begin{phonetics}{红包}{hong2 bao1}[][HSK 4]
    \definition[个]{s.}{saco de papel vermelho ou envelope contendo dinheiro como presente, gorjeta ou bônus | suborno; propina}
  \end{phonetics}
\end{entry}

\begin{entry}{红色}{6,6}{⽷、⾊}
  \begin{phonetics}{红色}{hong2 se4}[][HSK 2]
    \definition{s.}{cor vermelha}
  \end{phonetics}
\end{entry}

\begin{entry}{红宝石}{6,8,5}{⽷、⼧、⽯}
  \begin{phonetics}{红宝石}{hong2bao3shi2}
    \definition{s.}{rubi}
  \end{phonetics}
\end{entry}

\begin{entry}{红线}{6,8}{⽷、⽷}
  \begin{phonetics}{红线}{hong2xian4}
    \definition{s.}{linha vermelha}
  \end{phonetics}
\end{entry}

\begin{entry}{红茶}{6,9}{⽷、⾋}
  \begin{phonetics}{红茶}{hong2 cha2}[][HSK 3]
    \definition[杯,壶,斤,种]{s.}{chá preto}
  \end{phonetics}
\end{entry}

\begin{entry}{红烧}{6,10}{⽷、⽕}
  \begin{phonetics}{红烧}{hong2shao1}
    \definition{s.}{guisado em molho de soja (prato)}
  \end{phonetics}
\end{entry}

\begin{entry}{红酒}{6,10}{⽷、⾣}
  \begin{phonetics}{红酒}{hong2 jiu3}[][HSK 3]
    \definition{s.}{vinho tinto}
  \end{phonetics}
\end{entry}

\begin{entry}{红绿灯}{6,11,6}{⽷、⽷、⽕}
  \begin{phonetics}{红绿灯}{hong2lv4deng1}
    \definition[个]{s.}{semáforo | sinal de trânsito}
  \end{phonetics}
\end{entry}

\begin{entry}{红薯}{6,16}{⽷、⾋}
  \begin{phonetics}{红薯}{hong2shu3}
    \definition{s.}{batata doce}
  \end{phonetics}
\end{entry}

\begin{entry}{约}{6}{⽷}
  \begin{phonetics}{约}{yao1}
    \definition{adj.}{econômico; frugal | simples; breve | indistinto}
    \definition{adv.}{cerca de; ao redor; aproximadamente}
    \definition{s.}{pacto; acordo; nomeação; coisa prometida}
    \definition{v.}{marcar uma consulta; organizar | perguntar ou convidar com antecedência | restringir; conter | reduzir (fração aproximada)}
  \end{phonetics}
  \begin{phonetics}{约}{yue1}[][HSK 3]
    \definition*{s.}{sobrenome Yue}
    \definition{adj.}{econômico; frugal | simples; breve | indistinto}
    \definition{adv.}{cerca de; ao redor; aproximadamente}
    \definition{s.}{pacto; acordo; nomeação; coisa prometida}
    \definition{v.}{marcar uma consulta; organizar | perguntar ou convidar com antecedência | restringir; conter | reduzir (fração aproximada)}
  \end{phonetics}
\end{entry}

\begin{entry}{约会}{6,6}{⽷、⼈}
  \begin{phonetics}{约会}{yue1hui4}[][HSK 4]
    \definition[个,次]{s.}{data; compromisso; engajamento; reunião pré-agendada}
    \definition{v.}{marcar uma reunião; marcar um encontro;}
  \end{phonetics}
\end{entry}

\begin{entry}{级}{6}{⽷}
  \begin{phonetics}{级}{ji2}[][HSK 2]
    \definition{clas.}{para passo, estágio}
    \definition{s.}{nível | classificação | grau | qualquer uma das divisões anuais de um curso escolar: série, classe, etc. | etapa}
  \end{phonetics}
\end{entry}

\begin{entry}{纪录}{6,8}{⽷、⼹}
  \begin{phonetics}{纪录}{ji4lu4}[][HSK 3]
    \definition{s.}{recorde (esportes)}
  \end{phonetics}
\end{entry}

\begin{entry}{纪念}{6,8}{⽷、⼼}
  \begin{phonetics}{纪念}{ji4nian4}[][HSK 3]
    \definition[个]{s.}{lembrança | aniversário (comemoração)}
    \definition{v.}{comemorar}
  \end{phonetics}
\end{entry}

\begin{entry}{纪律}{6,9}{⽷、⼻}
  \begin{phonetics}{纪律}{ji4lv4}[][HSK 4]
    \definition{s.}{disciplina; código de conduta que cada membro da vida coletiva deve observar}
  \end{phonetics}
\end{entry}

\begin{entry}{系}{7}{⽷}
  \begin{phonetics}{系}{ji4}
    \definition{v.}{amarrar; prender; abotoar}
  \end{phonetics}
  \begin{phonetics}{系}{xi4}[][HSK 3,4]
    \definition*{s.}{sobrenome Xi}
    \definition{s.}{faculdade (da universidade) | departamento}
    \definition{v.}{sistema; série | departamento; faculdade}
    \definition{v.}{relacionar-se com; suportar; depender de | sentir-se ansioso; estar preocupado | amarrar; prender | ser}
  \end{phonetics}
\end{entry}

\begin{entry}{系囚}{7,5}{⽷、⼞}
  \begin{phonetics}{系囚}{xi4qiu2}
    \definition{s.}{prisioneiro}
  \end{phonetics}
\end{entry}

\begin{entry}{系列}{7,6}{⽷、⼑}
  \begin{phonetics}{系列}{xi4lie4}[][HSK 4]
    \definition{s.}{série; conjunto; conjunto de coisas relacionadas (matemática)}
  \end{phonetics}
\end{entry}

\begin{entry}{系统}{7,9}{⽷、⽷}
  \begin{phonetics}{系统}{xi4tong3}[][HSK 4]
    \definition{adj.}{sistemático; organizado}
    \definition[个]{s.}{sistema; relação de tipos semelhantes (ou seja, grupo de coisas semelhantes)}
  \end{phonetics}
\end{entry}

\begin{entry}{纯}{7}{⽷}
  \begin{phonetics}{纯}{chun2}[][HSK 4]
    \definition{adj.}{puro; não misturado; livre de impurezas | simples; puro e simples | habilidoso; proficiente; bem versado}
  \end{phonetics}
\end{entry}

\begin{entry}{纯净水}{7,8,4}{⽷、⼎、⽔}
  \begin{phonetics}{纯净水}{chun2 jing4 shui3}[][HSK 4]
    \definition{s.}{água purificada}
  \end{phonetics}
\end{entry}

\begin{entry}{纯真}{7,10}{⽷、⼗}
  \begin{phonetics}{纯真}{chun2zhen1}
    \definition{adj.}{inocente e não afetado | puro e não adulterado}
    \definition{s.}{inocência}
  \end{phonetics}
\end{entry}

\begin{entry}{纷纷}{7,7}{⽷、⽷}
  \begin{phonetics}{纷纷}{fen1fen1}[][HSK 4]
    \definition{adj.}{numeroso e confuso; muitos e desordenados}
    \definition{adv.}{um após o outro; em sucessão; em rápida sucessão}
  \end{phonetics}
\end{entry}

\begin{entry}{纸}{7}{⽷}
  \begin{phonetics}{纸}{zhi3}[][HSK 2]
    \definition{clas.}{para documentos, cartas, etc.}
    \definition[张,沓]{s.}{papel}
  \end{phonetics}
\end{entry}

\begin{entry}{纸巾}{7,3}{⽷、⼱}
  \begin{phonetics}{纸巾}{zhi3jin1}
    \definition[张,包]{s.}{lenço | guardanapo | papel toalha}
  \end{phonetics}
\end{entry}

\begin{entry}{纸币}{7,4}{⽷、⼱}
  \begin{phonetics}{纸币}{zhi3bi4}
    \definition[张]{s.}{nota (dinheiro) | cédula}
  \end{phonetics}
\end{entry}

\begin{entry}{纸尿裤}{7,7,12}{⽷、⼫、⾐}
  \begin{phonetics}{纸尿裤}{zhi3niao4ku4}
    \definition{s.}{fralda descartável}
  \end{phonetics}
\end{entry}

\begin{entry}{纸张}{7,7}{⽷、⼸}
  \begin{phonetics}{纸张}{zhi3zhang1}
    \definition{s.}{papel}
  \end{phonetics}
\end{entry}

\begin{entry}{纸烟}{7,10}{⽷、⽕}
  \begin{phonetics}{纸烟}{zhi3yan1}
    \definition{s.}{cigarro}
  \end{phonetics}
\end{entry}

\begin{entry}{纹路}{7,13}{⽷、⾜}
  \begin{phonetics}{纹路}{wen2lu4}
    \definition{s.}{padrão de linhas | rugas | veias | veias (em mármore ou impressão digital) | grãos (em madeira, etc.)}
  \end{phonetics}
\end{entry}

\begin{entry}{线}{8}{⽷}
  \begin{phonetics}{线}{xian4}[][HSK 3]
    \definition{clas.}{para coisas abstratas, o número é limitado a ``一''}
    \definition{s.}{fio; corda; arame | linha | feito de fio de algodão | algo em forma de linha, fio, etc. | rota; linha | linha de demarcação; limite | beira; borda | linha ideológica e política | pista; fio}
  \end{phonetics}
\end{entry}

\begin{entry}{线香}{8,9}{⽷、⾹}
  \begin{phonetics}{线香}{xian4xiang1}
    \definition{s.}{bastão ou vareta de incenso}
  \end{phonetics}
\end{entry}

\begin{entry}{练}{8}{⽷}
  \begin{phonetics}{练}{lian4}[][HSK 2]
    \definition{s.}{exercício | (literário) seda branca}
    \definition{v.}{praticar | treinar | aperfeiçoar (habilidade) | ferver e esfregar seda crua}
  \end{phonetics}
\end{entry}

\begin{entry}{练习}{8,3}{⽷、⼄}
  \begin{phonetics}{练习}{lian4xi2}[][HSK 2]
    \definition[个]{s.}{prática | exercício}
    \definition{v.}{praticar | exercitar}
  \end{phonetics}
\end{entry}

\begin{entry}{组}{8}{⽷}
  \begin{phonetics}{组}{zu3}[][HSK 2]
    \definition*{s.}{sobrenome Zu}
    \definition{clas.}{para conjuntos, séries, suítes, baterias}
    \definition{s.}{grupo}
    \definition{v.}{organizar | formar}
  \end{phonetics}
\end{entry}

\begin{entry}{组长}{8,4}{⽷、⾧}
  \begin{phonetics}{组长}{zu3 zhang3}[][HSK 2]
    \definition[名,位,个]{s.}{líder de grupo}
  \end{phonetics}
\end{entry}

\begin{entry}{组合}{8,6}{⽷、⼝}
  \begin{phonetics}{组合}{zu3he2}[][HSK 3]
    \definition{s.}{associação; combinação
combinação; pegue n elementos de m elementos diferentes e agrupe-os em grupos, independentemente da ordem, onde cada grupo contém pelo menos um componente diferente, o resultado é chamado de combinação de n de m.}
    \definition{v.}{criar; compor; constituir}
  \end{phonetics}
\end{entry}

\begin{entry}{组成}{8,6}{⽷、⼽}
  \begin{phonetics}{组成}{zu3cheng2}[][HSK 2]
    \definition{v.}{formar | compor | inventar}
  \end{phonetics}
\end{entry}

\begin{entry}{细}{8}{⽷}
  \begin{phonetics}{细}{xi4}[][HSK 4]
    \definition{adj.}{fino; delgado; esguio; esbelto; em oposição a ``粗'' | fino; em partículas pequenas; grãos pequenos | fino e macio;  um sussuro | fino; requintado; delicado | cuidadoso; detalhado; meticuloso | ínfimo; minúsculo; insignificante; diminuto | jovem; pequeno}
  \seealsoref{粗}{cu1}
  \end{phonetics}
\end{entry}

\begin{entry}{细节}{8,5}{⽷、⾋}
  \begin{phonetics}{细节}{xi4jie2}[][HSK 4]
    \definition{s.}{detalhe; particularidade; aspectos secundários ou partes sutis de um enredo ou episódios secundários usados em uma obra literária para expressar o caráter de uma pessoa ou as características essenciais de uma coisa}
  \end{phonetics}
\end{entry}

\begin{entry}{细致}{8,10}{⽷、⾄}
  \begin{phonetics}{细致}{xi4zhi4}[][HSK 4]
    \definition{adj.}{meticuloso; cuidadoso; minucioso | intrincado; delicado}
  \end{phonetics}
\end{entry}

\begin{entry}{细菌战}{8,11,9}{⽷、⾋、⼽}
  \begin{phonetics}{细菌战}{xi4jun1zhan4}
    \definition{s.}{guerra biológica}
  \end{phonetics}
\end{entry}

\begin{entry}{织}{8}{⽷}
  \begin{phonetics}{织}{zhi1}
    \definition{v.}{tecer | tricotar}
  \end{phonetics}
\end{entry}

\begin{entry}{终于}{8,3}{⽷、⼆}
  \begin{phonetics}{终于}{zhong1yu2}[][HSK 3]
    \definition{adv.}{finalmente; eventualmente; no final; indica uma situação que ocorre após várias mudanças ou esperas}
  \end{phonetics}
\end{entry}

\begin{entry}{终究}{8,7}{⽷、⽳}
  \begin{phonetics}{终究}{zhong1jiu1}
    \definition{adv.}{afinal de contas; enfatiza que, não importa o que aconteça, a natureza das pessoas e das coisas não mudará e que as características básicas devem ser reconhecidas (tem o efeito de fortalecer o tom) |  no final; indica que um determinado resultado ocorrerá ou não, frequentemente usado em especulações, julgamentos etc. | afinal de contas; indica que, apesar do grande esforço ou da grande esperança, o resultado objetivo ainda é insatisfatório, geralmente com o significado de pesar ou pena | afinal de contas; indica que um resultado desejado finalmente aparece}
  \end{phonetics}
\end{entry}

\begin{entry}{经}{8}{⽷}
  \begin{phonetics}{经}{jing1}
    \definition*{s.}{sobrenome Jing}
    \definition{s.}{livro sagrado | escritura | clássicos | longitude | menstruação | canal}
    \definition{v.}{passar | sofrer | suportar | deformar (têxtil)}
  \end{phonetics}
\end{entry}

\begin{entry}{经历}{8,4}{⽷、⼚}
  \begin{phonetics}{经历}{jing1li4}[][HSK 3]
    \definition[个,次,段,种]{s.}{experiência}
    \definition{v.}{passar por}
  \end{phonetics}
\end{entry}

\begin{entry}{经过}{8,6}{⽷、⾡}
  \begin{phonetics}{经过}{jing1guo4}[][HSK 2]
    \definition[个]{s.}{processo | curso}
    \definition{v.}{passar | passar por}
  \end{phonetics}
\end{entry}

\begin{entry}{经典}{8,8}{⽷、⼋}
  \begin{phonetics}{经典}{jing1dian3}[][HSK 4]
    \definition{adj.}{clássico; (escritos ou obras, etc.) que são típicos, autorizados}
    \definition{s.}{clássicos; escritos tradicionais e valiosos; os livros mais importantes e fundamentais da religião | escrituras; escritos de doutrinas religiosas}
  \end{phonetics}
\end{entry}

\begin{entry}{经济}{8,9}{⽷、⽔}
  \begin{phonetics}{经济}{jing1ji4}[][HSK 3]
    \definition{adj.}{econômico;  parcimonioso}
    \definition{s.}{economia |economia nacional; setor da economia nacional | renda; condição financeira}
    \definition{v.}{governar o país}
  \end{phonetics}
\end{entry}

\begin{entry}{经费}{8,9}{⽷、⾙}
  \begin{phonetics}{经费}{jing1fei4}[][HSK 5]
    \definition[笔]{s.}{fundos; desembolso; gastos | despesas; gastos}
  \end{phonetics}
\end{entry}

\begin{entry}{经验}{8,10}{⽷、⾺}
  \begin{phonetics}{经验}{jing1yan4}[][HSK 3]
    \definition[个,次,种]{s.}{experiência}
    \definition{v.}{experimentar; passar por}
  \end{phonetics}
\end{entry}

\begin{entry}{经常}{8,11}{⽷、⼱}
  \begin{phonetics}{经常}{jing1chang2}[][HSK 2]
    \definition{adv.}{constantemente | diariamente | dia-a-dia | todo dia | frequentemente | sempre | regularmente}
  \end{phonetics}
\end{entry}

\begin{entry}{经理}{8,11}{⽷、⽟}
  \begin{phonetics}{经理}{jing1li3}[][HSK 2]
    \definition[个,位,名]{s.}{diretor | gerente}
  \end{phonetics}
\end{entry}

\begin{entry}{经营}{8,11}{⽷、⾋}
  \begin{phonetics}{经营}{jing1ying2}[][HSK 3]
    \definition{v.}{executar; gerenciar; operar; envolver-se em | gerenciar}
  \end{phonetics}
\end{entry}

\begin{entry}{结}{9}{⽷}
  \begin{phonetics}{结}{jie1}
    \definition{v.}{dar (frutos); formar (sementes); produzir frutos ou sementes (uma planta)}
  \end{phonetics}
  \begin{phonetics}{结}{jie2}[][HSK 4]
    \definition*{s.}{sobrenome Jie}
    \definition{s.}{nó | declaração juramentada; garantia por escrito; documento que, antigamente, significava um reconhecimento de encerramento ou uma garantia de responsabilidade}
    \definition{v.}{amarrar; tricotar; dar nó; tecer | formar; forjar; cimentar; solidificar | resolver; concluir | combinar; formar um relacionamento}
  \end{phonetics}
\end{entry}

\begin{entry}{结合}{9,6}{⽷、⼝}
  \begin{phonetics}{结合}{jie2he2}[][HSK 3]
    \definition{v.}{ligar; unir; combinar; integrar | casar-se; unir-se em matrimônio}
  \end{phonetics}
\end{entry}

\begin{entry}{结论}{9,6}{⽷、⾔}
  \begin{phonetics}{结论}{jie2lun4}[][HSK 4]
    \definition[个]{s.}{conclusão; palavra final sobre uma pessoa ou coisa após investigação e pesquisa | veredito; julgamento deduzido de premissas também é chamado de conclusão}
  \end{phonetics}
\end{entry}

\begin{entry}{结局}{9,7}{⽷、⼫}
  \begin{phonetics}{结局}{jie2ju2}
    \definition{s.}{conclusão | fim | final}
  \end{phonetics}
\end{entry}

\begin{entry}{结束}{9,7}{⽷、⽊}
  \begin{phonetics}{结束}{jie2shu4}[][HSK 3]
    \definition{v.}{finalizar; fechar; terminar; concluir; encerrar}
  \end{phonetics}
\end{entry}

\begin{entry}{结束工作}{9,7,3,7}{⽷、⽊、⼯、⼈}
  \begin{phonetics}{结束工作}{jie2shu4gong1zuo4}
    \definition{s.}{trabalho final}
    \definition{v.}{terminar o trabalho}
  \end{phonetics}
\end{entry}

\begin{entry}{结束区}{9,7,4}{⽷、⽊、⼖}
  \begin{phonetics}{结束区}{jie2shu4 qu1}
    \definition{s.}{zona final}
  \end{phonetics}
\end{entry}

\begin{entry}{结束文本}{9,7,4,5}{⽷、⽊、⽂、⽊}
  \begin{phonetics}{结束文本}{jie2shu4 wen2ben3}
    \definition{s.}{texto final}
  \end{phonetics}
\end{entry}

\begin{entry}{结束剂}{9,7,8}{⽷、⽊、⼑}
  \begin{phonetics}{结束剂}{jie2shu4 ji4}
    \definition{s.}{finalizador}
  \end{phonetics}
\end{entry}

\begin{entry}{结束语}{9,7,9}{⽷、⽊、⾔}
  \begin{phonetics}{结束语}{jie2shu4yu3}
    \definition{s.}{conclusões finais | considerações finais}
  \end{phonetics}
\end{entry}

\begin{entry}{结束辩论}{9,7,16,6}{⽷、⽊、⾟、⾔}
  \begin{phonetics}{结束辩论}{jie2shu4 bian4 lun4}
    \definition{s.}{debate de encerramento}
  \end{phonetics}
\end{entry}

\begin{entry}{结社自由}{9,7,6,5}{⽷、⽰、⾃、⽥}
  \begin{phonetics}{结社自由}{jie2she4zi4you2}
    \definition{s.}{(constitucional) liberdade de associação}
  \end{phonetics}
\end{entry}

\begin{entry}{结实}{9,8}{⽷、⼧}
  \begin{phonetics}{结实}{jie1shi5}[][HSK 3]
    \definition{adj.}{sólido; resistente; durável | forte; resistente; robusto}
  \end{phonetics}
\end{entry}

\begin{entry}{结构}{9,8}{⽷、⽊}
  \begin{phonetics}{结构}{jie2gou4}[][HSK 4]
    \definition[个,座]{s.}{estrutura; composição; construção; formação; constituição; tecido; forma; sistematização; mecânica; organização | arquitetura; estrutura; construção; construção de partes de edifícios com suporte de carga ou com carga externa | textura (geológico)}
  \end{phonetics}
\end{entry}

\begin{entry}{结果}{9,8}{⽷、⽊}
  \begin{phonetics}{结果}{jie1guo3}
    \definition{v.}{dar frutos}
  \end{phonetics}
  \begin{phonetics}{结果}{jie2guo3}[][HSK 2]
    \definition{s.}{resultado | conclusão}
    \definition{v.}{despachar | matar}
  \end{phonetics}
\end{entry}

\begin{entry}{结婚}{9,11}{⽷、⼥}
  \begin{phonetics}{结婚}{jie2hun1}[][HSK 3]
    \definition{v.+compl.}{casar; casar-se}
  \end{phonetics}
\end{entry}

\begin{entry}{结婚礼服}{9,11,5,8}{⽷、⼥、⽰、⽉}
  \begin{phonetics}{结婚礼服}{jie2hun1 li3 fu2}
    \definition{s.}{vestido de casamento}
  \end{phonetics}
\end{entry}

\begin{entry}{给}{9}{⽷}
  \begin{phonetics}{给}{gei3}[][HSK 1]
    \definition{prep.}{a | para}
    \definition{v.}{dar | permitir | fazer alguma coisa (para alguém)}
  \end{phonetics}
  \begin{phonetics}{给}{ji3}
    \definition{v.}{fornecer | prover}
  \end{phonetics}
\end{entry}

\begin{entry}{给……打电话}{9,5,5,8}{⽷、⼿、⽥、⾔}
  \begin{phonetics}{给……打电话}{gei3 da3 dian4 hua4}
    \definition{expr.}{telefonar para alguém}
    \seeref{打电话}{da3 dian4 hua4}
  \end{phonetics}
\end{entry}

\begin{entry}{绝不}{9,4}{⽷、⼀}
  \begin{phonetics}{绝不}{jue2bu4}
    \definition{adv.}{definitivamente não | de forma alguma | sob nenhuma circunstância}
  \end{phonetics}
\end{entry}

\begin{entry}{绝对}{9,5}{⽷、⼨}
  \begin{phonetics}{绝对}{jue2dui4}[][HSK 3]
    \definition{adj.}{absoluto; extremo}
    \definition{adv.}{absolutamente}
  \end{phonetics}
\end{entry}

\begin{entry}{绝招}{9,8}{⽷、⼿}
  \begin{phonetics}{绝招}{jue2zhao1}
    \definition{s.}{habilidade única | movimento delicado inesperado (como último recurso) | golpe de mestre | golpe final}
  \end{phonetics}
\end{entry}

\begin{entry}{绝版}{9,8}{⽷、⽚}
  \begin{phonetics}{绝版}{jue2ban3}
    \definition{adj.}{esgotado | fora de catálogo}
  \end{phonetics}
\end{entry}

\begin{entry}{绝望}{9,11}{⽷、⽉}
  \begin{phonetics}{绝望}{jue2 wang4}[][HSK 5]
    \definition{v.+compl.}{desesperar; desistir de toda esperança; perder toda esperança de}
  \end{phonetics}
\end{entry}

\begin{entry}{统一}{9,1}{⽷、⼀}
  \begin{phonetics}{统一}{tong3yi1}[][HSK 4]
    \definition{adj.}{unificado; unitário; centralizado; consistente}
    \definition{v.}{unificar; unir; integrar; padronizar}
  \end{phonetics}
\end{entry}

\begin{entry}{统计}{9,4}{⽷、⾔}
  \begin{phonetics}{统计}{tong3ji4}[][HSK 4]
    \definition{v.}{compilar estatísticas; refere-se à realização de trabalho estatístico, ou seja, coletar, reunir, analisar e extrapolar dados sobre um fenômeno | somar; adicionar; contar}
  \end{phonetics}
\end{entry}

\begin{entry}{索性}{10,8}{⽷、⼼}
  \begin{phonetics}{索性}{suo3xing4}
    \definition{adv.}{poderia muito bem | simplesmente | apenas}
  \end{phonetics}
\end{entry}

\begin{entry}{紧}{10}{⽷}
  \begin{phonetics}{紧}{jin3}[][HSK 3]
    \definition{adj.}{tenso; apertado | seguro; firme | cerrado; apertado | urgente; premente; tenso | rigoroso; rígido; severo | difícil; sem dinheiro}
    \definition{v.}{apertar}
  \end{phonetics}
\end{entry}

\begin{entry}{紧张}{10,7}{⽷、⼸}
  \begin{phonetics}{紧张}{jin3zhang1}[][HSK 3]
    \definition{adj.}{nervoso; tenso | apertado; em falta | tenso; intenso; coado}
  \end{phonetics}
\end{entry}

\begin{entry}{紧急}{10,9}{⽷、⼼}
  \begin{phonetics}{紧急}{jin3ji2}[][HSK 3]
    \definition{adj.}{urgente}
    \definition{adj.}{urgente; premente; crítico}
    \definition{s.}{emergência}
  \end{phonetics}
\end{entry}

\begin{entry}{紧紧}{10,10}{⽷、⽷}
  \begin{phonetics}{紧紧}{jin3 jin3}[][HSK 5]
    \definition{adv.}{firmemente; estreitamente; apertadamente; prestar muita atenção (em algo)}
  \end{phonetics}
\end{entry}

\begin{entry}{紧密}{10,11}{⽷、⼧}
  \begin{phonetics}{紧密}{jin3 mi4}[][HSK 4]
    \definition{adj.}{próximos; inseparáveis | incessante; rápido e intenso}
  \end{phonetics}
\end{entry}

\begin{entry}{绣}{10}{⽷}
  \begin{phonetics}{绣}{xiu4}
    \definition{s.}{bordado}
    \definition{v.}{bordar}
  \end{phonetics}
\end{entry}

\begin{entry}{继承}{10,8}{⽷、⼿}
  \begin{phonetics}{继承}{ji4cheng2}[][HSK 5]
    \definition{v.}{herdar (o patrimônio de uma pessoa falecida, etc.) de acordo com a lei | continuar; geralmente se refere à aceitação do estilo, da cultura, do conhecimento, etc., daqueles que nos precederam | continuar; os descendentes continuam o trabalho deixado por seus antecessores.}
  \end{phonetics}
\end{entry}

\begin{entry}{继续}{10,11}{⽷、⽷}
  \begin{phonetics}{继续}{ji4xu4}[][HSK 3]
    \definition{v.}{continuar; prosseguir}
  \end{phonetics}
\end{entry}

\begin{entry}{累}{11}{⽷}
  \begin{phonetics}{累}{lei2}
    \definition*{s.}{sobrenome Lei}
    \definition{s.}{corda}
    \definition{v.}{amarrar | torcer}
  \end{phonetics}
  \begin{phonetics}{累}{lei3}
    \definition{adj.}{contínuo | repetido}
    \definition{v.}{acumular | envolver ou implicar}
  \end{phonetics}
  \begin{phonetics}{累}{lei4}[][HSK 1]
    \definition{adj.}{cansado | fatigado}
    \definition{v.}{forçar | desgastar | trabalhar duro}
  \end{phonetics}
\end{entry}

\begin{entry}{绰号}{11,5}{⽷、⼝}
  \begin{phonetics}{绰号}{chuo4hao4}
    \definition{s.}{apelido}
  \end{phonetics}
\end{entry}

\begin{entry}{绳子}{11,3}{⽷、⼦}
  \begin{phonetics}{绳子}{sheng2zi5}
    \definition[条]{s.}{corda | cordão}
  \end{phonetics}
\end{entry}

\begin{entry}{维吾尔}{11,7,5}{⽷、⼝、⼩}
  \begin{phonetics}{维吾尔}{wei2wu2'er3}
    \definition*{s.}{Grupo étnico Uigur de Xinjiang}
  \end{phonetics}
\end{entry}

\begin{entry}{维护}{11,7}{⽷、⼿}
  \begin{phonetics}{维护}{wei2hu4}[][HSK 4]
    \definition{v.}{defender; proteger; manter; preservar}
  \end{phonetics}
\end{entry}

\begin{entry}{维修}{11,9}{⽷、⼈}
  \begin{phonetics}{维修}{wei2xiu1}[][HSK 4]
    \definition{v.}{prestar serviços; manter; reparar; manter em (bom) estado de conservação}
  \end{phonetics}
\end{entry}

\begin{entry}{维持}{11,9}{⽷、⼿}
  \begin{phonetics}{维持}{wei2chi2}[][HSK 4]
    \definition{v.}{manter; conservar; guardar; manter vivo}
  \end{phonetics}
\end{entry}

\begin{entry}{绷带}{11,9}{⽷、⼱}
  \begin{phonetics}{绷带}{beng1dai4}
    \definition{s.}{curativo | (empréstimo linguístico) \emph{bandage}}
  \end{phonetics}
\end{entry}

\begin{entry}{综合}{11,6}{⽷、⼝}
  \begin{phonetics}{综合}{zong1he2}[][HSK 4]
    \definition{s.}{síntese}
    \definition{v.}{sintetizar; resumir as partes de uma coisa em um todo unificado após análise (em oposição a ``分析''); reunir coisas de um tipo ou natureza diferente}
  \seealsoref{分析}{fen1xi1}
  \end{phonetics}
\end{entry}

\begin{entry}{绿}{11}{⽷}
  \begin{phonetics}{绿}{lv4}[][HSK 2]
    \definition{adj.}{verde}
  \end{phonetics}
\end{entry}

\begin{entry}{绿色}{11,6}{⽷、⾊}
  \begin{phonetics}{绿色}{lv4 se4}[][HSK 2]
    \definition{s.}{cor verde}
  \end{phonetics}
\end{entry}

\begin{entry}{绿豆}{11,7}{⽷、⾖}
  \begin{phonetics}{绿豆}{lv4dou4}
    \definition{s.}{vagens}
  \end{phonetics}
\end{entry}

\begin{entry}{绿豆芽}{11,7,7}{⽷、⾖、⾋}
  \begin{phonetics}{绿豆芽}{lv4dou4 ya2}
    \definition{s.}{broto de feijão verde}
  \end{phonetics}
\end{entry}

\begin{entry}{绿茶}{11,9}{⽷、⾋}
  \begin{phonetics}{绿茶}{lv4 cha2}[][HSK 3]
    \definition{s.}{chá verde}
  \end{phonetics}
\end{entry}

\begin{entry}{紫}{12}{⽷}
  \begin{phonetics}{紫}{zi3}
    \definition{adj.}{púrpura | violeta}
  \end{phonetics}
\end{entry}

\begin{entry}{紫色}{12,6}{⽷、⾊}
  \begin{phonetics}{紫色}{zi3 se4}
    \definition{s.}{cor púrpura | cor violeta}
  \end{phonetics}
\end{entry}

\begin{entry}{絫}{12}{⽷}
  \begin{phonetics}{絫}{lei3}
    \variantof{累}
  \end{phonetics}
\end{entry}

\begin{entry}{编}{12}{⽷}
  \begin{phonetics}{编}{bian1}[][HSK 4]
    \definition*{s.}{sobrenome Bian}
    \definition{s.}{livro; volume; parte de um livro}
    \definition{v.}{tecer; trançar; entrançar | fazer uma lista; organizar em uma lista; organizar; agrupar | editar; compilar | compor; escrever | fabricar; inventar; fazer; preparar}
  \end{phonetics}
\end{entry}

\begin{entry}{编程}{12,12}{⽷、⽲}
  \begin{phonetics}{编程}{bian1cheng2}
    \definition{s.}{programa de computador}
    \definition{v.}{programar computador}
  \end{phonetics}
\end{entry}

\begin{entry}{编辑}{12,13}{⽷、⾞}
  \begin{phonetics}{编辑}{bian1ji2}[][HSK 5]
    \definition{v.}{editar; compilar; organizar e processar dados ou trabalhos existentes}
  \end{phonetics}
  \begin{phonetics}{编辑}{bian1ji5}[][HSK 5]
    \definition{s.}{editor; compilador; pessoa que organiza e processa dados ou trabalhos existentes}
  \end{phonetics}
\end{entry}

\begin{entry}{缘}{12}{⽷}
  \begin{phonetics}{缘}{yuan2}
    \definition{s.}{causa | razão | karma | destino | predestinação}
  \end{phonetics}
\end{entry}

\begin{entry}{缘分}{12,4}{⽷、⼑}
  \begin{phonetics}{缘分}{yuan2fen4}
    \definition{s.}{destino ou acaso que une as pessoas | afinidade ou relacionamento predestinado | destino (Budismo)}
  \end{phonetics}
\end{entry}

\begin{entry}{缝纫}{13,6}{⽷、⽷}
  \begin{phonetics}{缝纫}{feng2ren4}
    \definition{v.}{costurar}
  \end{phonetics}
\end{entry}

\begin{entry}{缝纫机}{13,6,6}{⽷、⽷、⽊}
  \begin{phonetics}{缝纫机}{feng2ren4ji1}
    \definition[架]{s.}{máquina de costura}
  \end{phonetics}
\end{entry}

\begin{entry}{缩小}{14,3}{⽷、⼩}
  \begin{phonetics}{缩小}{suo1 xiao3}[][HSK 4]
    \definition{v.}{reduzir, estreitar, encolher;  tornar menor (em oposição a ``放大'')}
  \seealsoref{放大}{fang4da4}
  \end{phonetics}
\end{entry}

\begin{entry}{缩短}{14,12}{⽷、⽮}
  \begin{phonetics}{缩短}{suo1duan3}[][HSK 4]
    \definition{v.}{encurtar; reduzir; diminuir}
  \end{phonetics}
\end{entry}

\begin{entry}{缩影卡片}{14,15,5,4}{⽷、⼺、⼘、⽚}
  \begin{phonetics}{缩影卡片}{suo1ying3 ka3pian4}
    \definition{s.}{cartão em miniatura}
  \end{phonetics}
\end{entry}

\begin{entry}{繁荣}{17,9}{⽷、⾋}
  \begin{phonetics}{繁荣}{fan2rong2}[][HSK 5]
    \definition{adj.}{florescente; próspero}
    \definition{v.}{promover; prosperar}
  \end{phonetics}
\end{entry}

%%%%% EOF %%%%%

