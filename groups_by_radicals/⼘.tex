%%%
%%% Radical "⼘"
%%%

\section*{Radical 25: ``⼘''}\addcontentsline{toc}{section}{Radical 25: ⼘}

\begin{entry}{占}{5}{⼘}
  \begin{phonetics}{占}{zhan1}
    \definition*{s.}{sobrenome Zhan}
    \definition{v.}{praticar adivinhação | advinhar}
  \end{phonetics}
  \begin{phonetics}{占}{zhan4}[][HSK 2]
    \definition{v.}{ocupar | apreender | tomar | constituir | manter | compor | dar conta de}
  \end{phonetics}
\end{entry}

\begin{entry}{占有}{5,6}{⼘、⽉}
  \begin{phonetics}{占有}{zhan4 you3}[][HSK 5]
    \definition{v.}{possuir; ter; ocupar e possuir | manter; ocupar; estar em (uma determinada posição) | possuir; deter; ter; dominar}
  \end{phonetics}
\end{entry}

\begin{entry}{占领}{5,11}{⼘、⾴}
  \begin{phonetics}{占领}{zhan4ling3}[][HSK 5]
    \definition{v.}{manter; tomar; ocupar; capturar; conquistar (posições ou territórios) com forças armadas | ocupar; capturar; possuir}
  \end{phonetics}
\end{entry}

\begin{entry}{卡}{5}{⼘}
  \begin{phonetics}{卡}{ka3}[][HSK 2]
    \definition{clas.}{calorias (cal)}
    \definition[张,片]{s.}{cartão; documento semelhante a um cartão | cassete; dispositivo tipo compartimento para colocar fitas cassete no gravador | caminhão}
  \end{phonetics}
  \begin{phonetics}{卡}{qia3}
    \definition*{s.}{sobrenome Qia}
    \definition[张,片]{s.}{clipe; prendedor; pinça; utensílio para prender objetos | posto de controle; posto de guarda ou posto de controle localizado em vias de comunicação importantes ou em locais com terreno acidentado}
    \definition{v.}{encravar; ficar preso; impedir de se mover | parar; controlar; impedir | pressionar firmemente com a palma da mão}
  \end{phonetics}
\end{entry}

\begin{entry}{卡片}{5,4}{⼘、⽚}
  \begin{phonetics}{卡片}{ka3pian4}
    \definition{s.}{cartão}
  \end{phonetics}
\end{entry}

\begin{entry}{卡片游戏}{5,4,12,6}{⼘、⽚、⽔、⼽}
  \begin{phonetics}{卡片游戏}{ka3pian4 you2xi4}
    \definition{s.}{carta de baralho}
  \end{phonetics}
\end{entry}

\begin{entry}{卡车司机}{5,4,5,6}{⼘、⾞、⼝、⽊}
  \begin{phonetics}{卡车司机}{ka3che1 si1ji1}
    \definition{s.}{motorista de caminhão}
  \end{phonetics}
\end{entry}

\begin{entry}{卡通}{5,10}{⼘、⾡}
  \begin{phonetics}{卡通}{ka3tong1}
    \definition{s.}{(empréstimo linguístico) \emph{cartoon}}
  \end{phonetics}
\end{entry}

\begin{entry}{卢旺达}{5,8,6}{⼘、⽇、⾡}
  \begin{phonetics}{卢旺达}{lu2wang4da2}
    \definition*{s.}{Ruanda}
  \end{phonetics}
\end{entry}

%%%%% EOF %%%%%

