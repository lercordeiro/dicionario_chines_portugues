%%%
%%% Radical "⽴"
%%%

\section*{Radical 117: ``⽴''}\addcontentsline{toc}{section}{Radical 117: ⽴}

\begin{entry}{立即}{5,7}{⽴、⼙}
  \begin{phonetics}{立即}{li4ji2}[][HSK 4]
    \definition{adv.}{prontamente; imediatamente; de imediato}
  \end{phonetics}
\end{entry}

\begin{entry}{立刻}{5,8}{⽴、⼑}
  \begin{phonetics}{立刻}{li4ke4}[][HSK 3]
    \definition{adv.}{imediatamente; de ​​uma vez}
  \end{phonetics}
\end{entry}

\begin{entry}{立法}{5,8}{⽴、⽔}
  \begin{phonetics}{立法}{li4fa3}
    \definition{s.}{legislação}
    \definition{v.}{promulgar leis | legislar}
  \end{phonetics}
\end{entry}

\begin{entry}{站}{10}{⽴}
  \begin{phonetics}{站}{zhan4}[][HSK 1]
    \definition{s.}{estação | ponto | parada}
  \end{phonetics}
\end{entry}

\begin{entry}{站长}{10,4}{⽴、⾧}
  \begin{phonetics}{站长}{zhan4zhang3}
    \definition{s.}{pessoa responsável pela estação de trem | chefe da estação | \emph{webmaster} | gerente de centro de voluntariado}
  \end{phonetics}
\end{entry}

\begin{entry}{站台}{10,5}{⽴、⼝}
  \begin{phonetics}{站台}{zhan4tai2}
    \definition{s.}{plataforma (em uma estação ferroviária)}
  \end{phonetics}
\end{entry}

\begin{entry}{站住}{10,7}{⽴、⼈}
  \begin{phonetics}{站住}{zhan4 zhu4}[][HSK 2]
    \definition{v.}{parar | deter | ficar firme em pé | manter os pés firmes | manter a própria posição | consolidar a própria posição | reter água | ser sustentável}
  \end{phonetics}
\end{entry}

\begin{entry}{站姿}{10,9}{⽴、⼥}
  \begin{phonetics}{站姿}{zhan4zi1}
    \definition{s.}{postura}
  \end{phonetics}
\end{entry}

\begin{entry}{站点}{10,9}{⽴、⽕}
  \begin{phonetics}{站点}{zhan4dian3}
    \definition{s.}{\emph{website}}
  \end{phonetics}
\end{entry}

\begin{entry}{竞赛}{10,14}{⽴、⾙}
  \begin{phonetics}{竞赛}{jing4sai4}
    \definition{s.}{concurso | competição | partida | corrida}
    \definition{v.}{competir | correr}
  \end{phonetics}
\end{entry}

\begin{entry}{童年}{12,6}{⽴、⼲}
  \begin{phonetics}{童年}{tong2 nian2}[][HSK 4]
    \definition{s.}{infância; primeiros anos de vida}
  \end{phonetics}
\end{entry}

\begin{entry}{童话}{12,8}{⽴、⾔}
  \begin{phonetics}{童话}{tong2hua4}[][HSK 4]
    \definition[个,部]{s.}{conto de fadas; gênero de literatura infantil no qual as histórias adequadas para a diversão das crianças são escritas com muita imaginação, fantasia e exagero}
  \end{phonetics}
\end{entry}

\begin{entry}{端午节}{14,4,5}{⽴、⼗、⾋}
  \begin{phonetics}{端午节}{duan1wu3jie2}
    \definition*{s.}{Festa do Duplo Cinco, Festival dos Barcos-Dragão (5º~dia do quinto mês lunar)}
  \end{phonetics}
\end{entry}

%%%%% EOF %%%%%

