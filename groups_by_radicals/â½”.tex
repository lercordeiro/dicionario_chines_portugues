%%%
%%% Radical "⽔"
%%%

\section*{Radical 85: ``⽔'' (氵、氺)}\addcontentsline{toc}{section}{Radical 85: ⽔、氵、氺}

\begin{entry}{水}{4}{⽔}[Kangxi 85]
  \begin{phonetics}{水}{shui3}[][HSK 1]
    \definition*{s.}{Etnia Shui, que vive principalmente em Guizhou | Sobrenome Shui}
    \definition{adj.}{de má qualidade; mal feito; de baixa qualidade e conteúdo}
    \definition{clas.}{usado para número de lavagens}
    \definition[条,杯]{s.}{água | rio | termo geral para rios, lagos, mares, etc.; água | corrente; fluxo de água | um líquido; suco ralo | teor de prata nas moedas | encargos adicionais ou receitas | água, um dos cinco elementos}
  \end{phonetics}
\end{entry}

\begin{entry}{水分}{4,4}{⽔、⼑}
  \begin{phonetics}{水分}{shui3 fen4}[][HSK 5]
    \definition{s.}{teor de umidade; água contida em um objeto | exagero; metáfora de algo falso}
  \end{phonetics}
\end{entry}

\begin{entry}{水平}{4,5}{⽔、⼲}
  \begin{phonetics}{水平}{shui3ping2}[][HSK 2]
    \definition{adj.}{horizontal; nivelado; paralelo à superfície da água}
    \definition{s.}{padrão; nível; o nível alcançado em determinado aspecto}
  \end{phonetics}
\end{entry}

\begin{entry}{水平以下}{4,5,4,3}{⽔、⼲、⼈、⼀}
  \begin{phonetics}{水平以下}{shui3ping2 yi3xia4}
    \definition{s.}{sub-nível}
  \end{phonetics}
\end{entry}

\begin{entry}{水平尺}{4,5,4}{⽔、⼲、⼫}
  \begin{phonetics}{水平尺}{shui3ping2chi3}
    \definition{s.}{nível espiritual}
  \end{phonetics}
\end{entry}

\begin{entry}{水平仪}{4,5,5}{⽔、⼲、⼈}
  \begin{phonetics}{水平仪}{shui3ping2yi2}
    \definition{s.}{nível (dispositivo para determinar horizontal) | nível espiritual | nível de topógrafo}
  \end{phonetics}
\end{entry}

\begin{entry}{水平视差}{4,5,8,9}{⽔、⼲、⾒、⼯}
  \begin{phonetics}{水平视差}{shui3ping2 shi4cha1}
    \definition{s.}{paralaxe horizontal}
  \end{phonetics}
\end{entry}

\begin{entry}{水平度}{4,5,9}{⽔、⼲、⼴}
  \begin{phonetics}{水平度}{shui3ping2 du4}
    \definition{s.}{nivelamento}
  \end{phonetics}
\end{entry}

\begin{entry}{水平轴}{4,5,9}{⽔、⼲、⾞}
  \begin{phonetics}{水平轴}{shui3ping2zhou2}
    \definition{s.}{eixo horizontal}
  \end{phonetics}
\end{entry}

\begin{entry}{水平面}{4,5,9}{⽔、⼲、⾯}
  \begin{phonetics}{水平面}{shui3ping2mian4}
    \definition{s.}{plano horizontal | nível-da-água | superfície horizontal}
  \end{phonetics}
\end{entry}

\begin{entry}{水边}{4,5}{⽔、⾡}
  \begin{phonetics}{水边}{shui3bian1}
    \definition{s.}{beira d'água | beira-mar | costa (de mar, lago ou rio)}
  \end{phonetics}
\end{entry}

\begin{entry}{水产品}{4,6,9}{⽔、⼇、⼝}
  \begin{phonetics}{水产品}{shui3 chan3 pin3}[][HSK 5]
    \definition{s.}{produto aquático (peixes, camarões, etc.)}
  \end{phonetics}
\end{entry}

\begin{entry}{水污染}{4,6,9}{⽔、⽔、⽊}
  \begin{phonetics}{水污染}{shui3wu1ran3}
    \definition{s.}{poluição da água}
  \end{phonetics}
\end{entry}

\begin{entry}{水库}{4,7}{⽔、⼴}
  \begin{phonetics}{水库}{shui3 ku4}[][HSK 5]
    \definition[座]{s.}{reservatório; lago artificial construído pelo homem, que utiliza barragens e outras estruturas para represar a água e regular o fluxo, podendo ser utilizado para armazenamento de água, geração de energia e piscicultura, entre outros fins}
  \end{phonetics}
\end{entry}

\begin{entry}{水灵}{4,7}{⽔、⽕}
  \begin{phonetics}{水灵}{shui3ling2}
    \definition{adj.}{cheio de vida (sobre uma pessoa, etc.) | úmido e brilhante (sobre os olhos) | fresco (sobre frutas, etc.) | brilhante | aparência saudável}
  \end{phonetics}
\end{entry}

\begin{entry}{水灾}{4,7}{⽔、⽕}
  \begin{phonetics}{水灾}{shui3 zai1}[][HSK 5]
    \definition{s.}{inundação; desastres causados por excesso de chuvas, entre outros motivos}
  \end{phonetics}
\end{entry}

\begin{entry}{水果}{4,8}{⽔、⽊}
  \begin{phonetics}{水果}{shui3guo3}[][HSK 1]
    \definition[个]{s.}{fruta; um nome genérico para frutas com alto teor de água que podem ser consumidas, como peras, pêssegos, maçãs, etc.}
  \end{phonetics}
\end{entry}

\begin{entry}{水波}{4,8}{⽔、⽔}
  \begin{phonetics}{水波}{shui3bo1}
    \definition{s.}{ondulação (na água) | onda}
  \end{phonetics}
\end{entry}

\begin{entry}{水泥}{4,8}{⽔、⽔}
  \begin{phonetics}{水泥}{shui3ni2}[][HSK 6]
    \definition[袋,层]{s.}{cimento; um tipo de material mineral em pó que pode endurecer gradualmente no ar e na água após a mistura com água}
  \end{phonetics}
\end{entry}

\begin{entry}{水饺}{4,9}{⽔、⾷}
  \begin{phonetics}{水饺}{shui3jiao3}
    \definition{s.}{\emph{dumplings} | pastéis chineses cozidos}
  \end{phonetics}
\end{entry}

\begin{entry}{水瓶}{4,10}{⽔、⽡}
  \begin{phonetics}{水瓶}{shui3 ping2}
    \definition{s.}{garrada de água}
  \end{phonetics}
\end{entry}

\begin{entry}{水培}{4,11}{⽔、⼟}
  \begin{phonetics}{水培}{shui3pei2}
    \definition{v.}{cultivar plantas hidroponicamente}
  \end{phonetics}
\end{entry}

\begin{entry}{水豚}{4,11}{⽔、⾗}
  \begin{phonetics}{水豚}{shui3tun2}
    \definition{s.}{capivara}
  \end{phonetics}
\end{entry}

\begin{entry}{水路}{4,13}{⽔、⾜}
  \begin{phonetics}{水路}{shui3lu4}
    \definition{s.}{hidrovia}
  \end{phonetics}
\end{entry}

\begin{entry}{水槽}{4,15}{⽔、⽊}
  \begin{phonetics}{水槽}{shui3cao2}
    \definition{s.}{pia (de cozinha)}
  \end{phonetics}
\end{entry}

\begin{entry}{永}{5}{⽔}
  \begin{phonetics}{永}{yong3}
    \definition*{s.}{Sobrenome Yong}
    \definition{adj.}{sempre; para sempre; perpetuamente}
    \definition{adv.}{para sempre; significa um tempo muito longo sem fim, o que equivale a 永远}
  \seealsoref{永远}{yong3yuan3}
  \end{phonetics}
\end{entry}

\begin{entry}{永不}{5,4}{⽔、⼀}
  \begin{phonetics}{永不}{yong3bu4}
    \definition{adv.}{nunca}
  \end{phonetics}
\end{entry}

\begin{entry}{永远}{5,7}{⽔、⾡}
  \begin{phonetics}{永远}{yong3yuan3}[][HSK 2]
    \definition{adv.}{sempre; para sempre; Indica um longo período de tempo sem fim}
    \definition{s.}{eternidade; um futuro que nunca acaba}
  \end{phonetics}
\end{entry}

\begin{entry}{汇}{5}{⽔}
  \begin{phonetics}{汇}{hui4}[][HSK 4]
    \definition{s.}{montagem; coleção; coisas coletadas}
    \definition{v.}{convergir | reunir; coletar | remeter | trocar (câmbio de moedas)}
  \end{phonetics}
\end{entry}

\begin{entry}{汇报}{5,7}{⽔、⼿}
  \begin{phonetics}{汇报}{hui4bao4}[][HSK 4]
    \definition[份,次]{s.}{relatório; referindo-se ao conteúdo de declarações escritas ou orais feitas a um superior ou pessoa relevante para apresentar uma situação ou refletir um problema}
    \definition{v.}{relatar; fazer um relato de}
  \end{phonetics}
\end{entry}

\begin{entry}{汇率}{5,11}{⽔、⽞}
  \begin{phonetics}{汇率}{hui4lv4}[][HSK 4]
    \definition[个]{s.}{taxa de câmbio; relação entre a moeda de um país e a de outro}
  \end{phonetics}
\end{entry}

\begin{entry}{汇款}{5,12}{⽔、⽋}
  \begin{phonetics}{汇款}{hui4 kuan3}[][HSK 5]
    \definition[笔,个]{s.}{remessa; dinheiro enviado ou recebido}
    \definition{v.+compl.}{remeter dinheiro; fazer uma remessa; enviar dinheiro}
  \end{phonetics}
\end{entry}

\begin{entry}{汉}{5}{⽔}
  \begin{phonetics}{汉}{han4}
    \definition*{s.}{Dinastia Han (206 a.C.-220d.C.)  | Astronomia: A Via Láctea | Sobrenome Han}
    \definition{s.}{grupo étnico Han | chinês (língua) | homem}
  \end{phonetics}
\end{entry}

\begin{entry}{汉字}{5,6}{⽔、⼦}
  \begin{phonetics}{汉字}{han4 zi4}[][HSK 1]
    \definition[个]{s.}{caractere chinês; ideograma chinês; sinograma; com pouquíssimas exceções, os caracteres chineses representam uma sílaba cada um}
  \end{phonetics}
\end{entry}

\begin{entry}{汉服}{5,8}{⽔、⽉}
  \begin{phonetics}{汉服}{han4fu2}
    \definition{s.}{vestido chinês tradicional Han}
  \end{phonetics}
\end{entry}

\begin{entry}{汉语}{5,9}{⽔、⾔}
  \begin{phonetics}{汉语}{han4yu3}[][HSK 1]
    \definition[门]{s.}{língua chinesa, mandarim}
  \end{phonetics}
\end{entry}

\begin{entry}{汉堡王}{5,12,4}{⽔、⼟、⽟}
  \begin{phonetics}{汉堡王}{han4bao3wang2}
    \definition*{s.}{Burguer King, restaurante \emph{fast-food}}
  \end{phonetics}
\end{entry}

\begin{entry}{汉堡包}{5,12,5}{⽔、⼟、⼓}
  \begin{phonetics}{汉堡包}{han4bao3bao1}
    \definition[个]{s.}{hambúrguer}
  \end{phonetics}
\end{entry}

\begin{entry}{汉葡词典}{5,12,7,8}{⽔、⾋、⾔、⼋}
  \begin{phonetics}{汉葡词典}{han4-pu2 ci2dian3}
    \definition[部,本]{s.}{dicionário chinês-português}
  \seealsoref{葡汉词典}{pu2-han4 ci2dian3}
  \end{phonetics}
\end{entry}

\begin{entry}{汗}{6}{⽔}
  \begin{phonetics}{汗}{han2}
    \definition*{s.}{Abreviação de Khan}[他是成吉思汗。___Ele é Genghis Khan.]
  \end{phonetics}
  \begin{phonetics}{汗}{han4}[][HSK 5]
    \definition{s.}{suor; transpiração; perspiração}
  \end{phonetics}
\end{entry}

\begin{entry}{汗水}{6,4}{⽔、⽔}
  \begin{phonetics}{汗水}{han4shui3}
    \definition*{s.}{Rio Han (Hanshui)}
    \definition{s.}{suor | transpiração}
  \end{phonetics}
\end{entry}

\begin{entry}{汗液}{6,11}{⽔、⽔}
  \begin{phonetics}{汗液}{han4ye4}
    \definition{s.}{suor}
  \end{phonetics}
\end{entry}

\begin{entry}{汗腺}{6,13}{⽔、⾁}
  \begin{phonetics}{汗腺}{han4xian4}
    \definition{s.}{glândula sudorípara}
  \end{phonetics}
\end{entry}

\begin{entry}{江}{6}{⽔}
  \begin{phonetics}{江}{jiang1}[][HSK 4]
    \definition*{s.}{Rio Changjiang | Sobrenome Jiang}
    \definition[条,道]{s.}{rio grande}
  \end{phonetics}
\end{entry}

\begin{entry}{江水}{6,4}{⽔、⽔}
  \begin{phonetics}{江水}{jiang1shui3}
    \definition{s.}{água do rio}
  \end{phonetics}
\end{entry}

\begin{entry}{江西}{6,6}{⽔、⾑}
  \begin{phonetics}{江西}{jiang1xi1}
    \definition*{s.}{Jiangxi}
  \end{phonetics}
\end{entry}

\begin{entry}{江苏}{6,7}{⽔、⾋}
  \begin{phonetics}{江苏}{jiang1su1}
    \definition*{s.}{Província de Jiangsu}
  \end{phonetics}
\end{entry}

\begin{entry}{江南水乡}{6,9,4,3}{⽔、⼗、⽔、⼄}
  \begin{phonetics}{江南水乡}{jiang1nan2shui3xiang1}
    \definition*{s.}{Vila Aquática de Jiangnan | Cidades Aquáticas}
  \end{phonetics}
\end{entry}

\begin{entry}{池}{6}{⽔}
  \begin{phonetics}{池}{chi2}
    \definition*{s.}{Sobrenome Chi}
    \definition[个,片]{s.}{piscina; lagoa | qualquer espaço fechado com laterais elevadas | baias (em um teatro); a parte frontal do salão principal do teatro | fosso}
  \end{phonetics}
\end{entry}

\begin{entry}{池子}{6,3}{⽔、⼦}
  \begin{phonetics}{池子}{chi2 zi5}[][HSK 5]
    \definition{s.}{lago; lagoa; viveiro | piscina; piscina do balneário | (antigo) arquibancada (primeiras fileiras em um teatro) | pista de dança de um salão de baile}
  \end{phonetics}
\end{entry}

\begin{entry}{污}{6}{⽔}
  \begin{phonetics}{污}{wu1}
    \definition{adj.}{sujo; imundo; imundo | corrupto}
    \definition{s.}{sujeira; imundície | esgoto; água suja; coisas sujas}
    \definition{v.}{contaminar; sujar | manchar}
  \end{phonetics}
\end{entry}

\begin{entry}{污水}{6,4}{⽔、⽔}
  \begin{phonetics}{污水}{wu1shui3}[][HSK 5]
    \definition{s.}{água suja (ou poluída, residual); esgoto; lodo | efluente; drenagem; água suja; água poluída; água residual}
  \end{phonetics}
\end{entry}

\begin{entry}{污染}{6,9}{⽔、⽊}
  \begin{phonetics}{污染}{wu1ran3}[][HSK 5]
    \definition{s.}{poluição}
    \definition{v.}{poluir; contaminar com substâncias nocivas e prejudiciais; refere-se especificamente à destruição do ambiente natural causada por substâncias nocivas, tais como gases, líquidos e resíduos emitidos por indústrias, minas, veículos, etc. | contaminar; metáfora de que pensamentos prejudiciais causam efeitos negativos nas pessoas}
  \end{phonetics}
\end{entry}

\begin{entry}{污染区}{6,9,4}{⽔、⽊、⼖}
  \begin{phonetics}{污染区}{wu1ran3qu1}
    \definition{s.}{área contaminada}
  \end{phonetics}
\end{entry}

\begin{entry}{污染物}{6,9,8}{⽔、⽊、⽜}
  \begin{phonetics}{污染物}{wu1ran3wu4}
    \definition{s.}{poluente}
  \seealsoref{污染物质}{wu1ran3 wu4zhi4}
  \end{phonetics}
\end{entry}

\begin{entry}{污染物质}{6,9,8,8}{⽔、⽊、⽜、⾙}
  \begin{phonetics}{污染物质}{wu1ran3 wu4zhi4}
    \definition{s.}{poluente}
  \seealsoref{污染物}{wu1ran3wu4}
  \end{phonetics}
\end{entry}

\begin{entry}{汤}{6}{⽔}
  \begin{phonetics}{汤}{shang1}
    \definition{s.}{correnteza forte}
  \end{phonetics}
  \begin{phonetics}{汤}{tang1}[][HSK 3]
    \definition*{s.}{Sobrenome Tang}
    \definition[勺,碗,杯,锅]{s.}{água quente; água fervente | fontes termais | água utilizada para ferver algo| sopa; caldo | uma preparação líquida de ervas medicinais; decocção}
  \end{phonetics}
\end{entry}

\begin{entry}{求}{7}{⽔}
  \begin{phonetics}{求}{qiu2}[][HSK 2]
    \definition*{s.}{Sobrenome Qiu}
    \definition{v.}{implorar; solicitar; suplicar; rogar | lutar por; buscar; investigar | tentar; procurar; tentar obter | demandar}
  \end{phonetics}
\end{entry}

\begin{entry}{求职}{7,11}{⽔、⽿}
  \begin{phonetics}{求职}{qiu2 zhi2}[][HSK 6]
    \definition{v.}{procurar emprego; candidatar-se a um emprego; encontrar um emprego}
  \end{phonetics}
\end{entry}

\begin{entry}{汹}{7}{⽔}
  \begin{phonetics}{汹}{xiong1}
    \definition{adj.}{turbulento; tempestuoso | rugindo; estrondoso | tumultuado}
  \end{phonetics}
\end{entry}

\begin{entry}{汹涌}{7,10}{⽔、⽔}
  \begin{phonetics}{汹涌}{xiong1yong3}
    \definition{adj.}{turbulento}
    \definition{v.}{aumentar ou emergir violentamente (oceano, rio, lago, etc.)}
  \end{phonetics}
\end{entry}

\begin{entry}{汽}{7}{⽔}
  \begin{phonetics}{汽}{qi4}
    \definition{s.}{vapor | vaporizador}
  \end{phonetics}
\end{entry}

\begin{entry}{汽水}{7,4}{⽔、⽔}
  \begin{phonetics}{汽水}{qi4 shui3}[][HSK 4]
    \definition[罐,瓶]{s.}{refrigerante; refrigerante gaseificado; bebida refrescante, feita com a pressão de dióxido de carbono para dissolver na água e adicionar açúcar, suco de frutas, especiarias etc.}
  \end{phonetics}
\end{entry}

\begin{entry}{汽车}{7,4}{⽔、⾞}
  \begin{phonetics}{汽车}{qi4 che1}[][HSK 1]
    \definition[辆,种,款]{s.}{automóvel; carro; veículo motorizado; veículo movido a motor de combustão interna, que circula principalmente em rodovias ou ruas, geralmente com quatro ou mais pneus de borracha, usado para transportar pessoas ou mercadorias}
  \end{phonetics}
\end{entry}

\begin{entry}{汽油}{7,8}{⽔、⽔}
  \begin{phonetics}{汽油}{qi4you2}[][HSK 4]
    \definition{s.}{gasolina; mistura líquida de hidrocarbonetos com volatilidade e combustibilidade, que é usada como combustível a partir do fracionamento ou craqueamento do petróleo}
  \end{phonetics}
\end{entry}

\begin{entry}{沉}{7}{⽔}
  \begin{phonetics}{沉}{chen2}[][HSK 4]
    \definition{adj.}{profundo | pesado | pesado (sentir-se pesado)}
    \definition{v.}{afundar; submergir; imergir | manter baixo; abaixar | descansar; parar}
  \end{phonetics}
\end{entry}

\begin{entry}{沉重}{7,9}{⽔、⾥}
  \begin{phonetics}{沉重}{chen2zhong4}[][HSK 4]
    \definition{adj.}{(pressão, fardo, etc.) muito pesado; profundo | sério; pesado; humor pouco animador; fardo pesado de pensamentos}
  \end{phonetics}
\end{entry}

\begin{entry}{沉默}{7,16}{⽔、⿊}
  \begin{phonetics}{沉默}{chen2mo4}[][HSK 4]
    \definition{adj.}{silencioso; reticente; taciturno; não comunicativo}
    \definition{v.}{silenciar; não falar por causa de alguma coisa}
  \end{phonetics}
\end{entry}

\begin{entry}{沙}{7}{⽔}
  \begin{phonetics}{沙}{sha1}
    \definition*{s.}{Sobrenome Sha}
    \definition[粒]{s.}{areia | cascalho | grânulo | pó}
  \end{phonetics}
\end{entry}

\begin{entry}{沙子}{7,3}{⽔、⼦}
  \begin{phonetics}{沙子}{sha1 zi5}[][HSK 3]
    \definition[粒,把,堆,袋,车]{s.}{areia; grão; pequenas pedras | \emph{pellets}; grãos pequenos; coisas parecidas com areia}
  \end{phonetics}
\end{entry}

\begin{entry}{沙发}{7,5}{⽔、⼜}
  \begin{phonetics}{沙发}{sha1fa1}[][HSK 3]
    \definition[套,组,个,张]{s.}{sofá; assentos com molas ou espuma plástica espessa, etc., com apoios de braços em ambos os lados}
  \end{phonetics}
\end{entry}

\begin{entry}{沙鱼}{7,8}{⽔、⿂}
  \begin{phonetics}{沙鱼}{sha1yu2}
    \variantof{鲨鱼}
  \end{phonetics}
\end{entry}

\begin{entry}{沙特}{7,10}{⽔、⽜}
  \begin{phonetics}{沙特}{sha1te4}
    \definition*{s.}{Saudita | Arábia Saudita, abreviação de 沙特阿拉伯}
  \seealsoref{沙特阿拉伯}{sha1te4 a1la1bo2}
  \end{phonetics}
\end{entry}

\begin{entry}{沙特阿拉伯}{7,10,7,8,7}{⽔、⽜、⾩、⼿、⼈}
  \begin{phonetics}{沙特阿拉伯}{sha1te4 a1la1bo2}
    \definition*{s.}{Arábia Saudita}
  \end{phonetics}
\end{entry}

\begin{entry}{沙漠}{7,13}{⽔、⽔}
  \begin{phonetics}{沙漠}{sha1mo4}[][HSK 5]
    \definition[个]{s.}{deserto; superfície totalmente coberta por areia, sem água corrente, clima seco e vegetação escassa}
  \end{phonetics}
\end{entry}

\begin{entry}{沟}{7}{⽔}
  \begin{phonetics}{沟}{gou1}[][HSK 5]
    \definition[条]{s.}{canal; vala; sarjeta; trincheira; cursos d'água ou fortificações escavados | ranhura; sulco raso; uma depressão que se assemelha a uma vala | ravina; barranco; cursos d'água}
  \end{phonetics}
\end{entry}

\begin{entry}{沟通}{7,10}{⽔、⾡}
  \begin{phonetics}{沟通}{gou1tong1}[][HSK 5]
    \definition{v.}{comunicar; comunicar-se para entender as ideias, opiniões, etc. | conectar; ligar; estabelecer um paralelo entre os dois}
  \end{phonetics}
\end{entry}

\begin{entry}{没}{7}{⽔}
  \begin{phonetics}{没}{mei2}[][HSK 1]
    \definition{adv.}{não; nunca; negar que uma ação ou situação tenha ocorrido, com o significado de 不曾}
    \definition{pref.}{não (prefixo negativo para verbos, traduzido para outras línguas com verbos no pretérito)}
    \definition{v.}{não possuir; não ter | não existe; não há | ninguém; usado antes de 谁, 什么, 哪个, significa 全都不 | não ser tão bom quanto; ser inferior a; não chega a; não é tão bom quanto | menor que; insuficiente}
  \seealsoref{不曾}{bu4 ceng2}
  \seealsoref{哪个}{na3ge5}
  \seealsoref{全都不}{quan2dou1 bu4}
  \seealsoref{谁}{shei2}
  \seealsoref{什么}{shen2me5}
  \end{phonetics}
  \begin{phonetics}{没}{mo4}
    \definition{adj.}{último; final}
    \definition{v.}{afundar na água; submergir | transbordar; subir além; exceder ou ultrapassar | esconder-se; desaparecer; sumir; ocultar-se | confiscar; expropriar | morrer}
    \variantof{没}
  \end{phonetics}
\end{entry}

\begin{entry}{没了}{7,2}{⽔、⼅}
  \begin{phonetics}{没了}{mei2le5}
    \definition{v.}{estar morto | deixar de existir}
  \end{phonetics}
\end{entry}

\begin{entry}{没什么}{7,4,3}{⽔、⼈、⼃}
  \begin{phonetics}{没什么}{mei2 shen2 me5}[][HSK 1]
    \definition{expr.}{não é nada; está tudo bem; não importa}
  \end{phonetics}
\end{entry}

\begin{entry}{没用}{7,5}{⽔、⽤}
  \begin{phonetics}{没用}{mei2 yong4}[][HSK 3]
    \definition{adj.}{inútil; imprestável; sem valor; sem préstimo; vão; que não serve para nada}
  \end{phonetics}
\end{entry}

\begin{entry}{没关系}{7,6,7}{⽔、⼋、⽷}
  \begin{phonetics}{没关系}{mei2guan1xi5}[][HSK 1]
    \definition{v.}{está tudo bem; não é nada; não importa; não se preocupe}
  \seealsoref{没有关系}{mei2you3guan1xi5}
  \end{phonetics}
\end{entry}

\begin{entry}{没收}{7,6}{⽔、⽁}
  \begin{phonetics}{没收}{mo4 shou1}[][HSK 6]
    \definition{v.}{confiscar; expropriar; os bens e pertences de pessoas ou grupos que violem leis ou proibições serão tornados propriedade pública, de acordo com a lei}
  \end{phonetics}
\end{entry}

\begin{entry}{没有}{7,6}{⽔、⽉}
  \begin{phonetics}{没有}{mei2 you3}[][HSK 1]
    \definition{adv.}{ainda não; (usado com o pretérito) não; ação ou estado negativo ocorreu}
    \definition{v.}{não há; não tem; não existe}
  \end{phonetics}
\end{entry}

\begin{entry}{没有关系}{7,6,6,7}{⽔、⽉、⼋、⽷}
  \begin{phonetics}{没有关系}{mei2you3guan1xi5}
    \definition{v.}{não ter problema | não ter importância | não fazer mal}
  \seealsoref{没关系}{mei2guan1xi5}
  \end{phonetics}
\end{entry}

\begin{entry}{没有次序}{7,6,6,7}{⽔、⽉、⽋、⼴}
  \begin{phonetics}{没有次序}{mei2you3 ci4xu4}
    \definition{adj.}{sem ordem; nenhuma ordem}
  \end{phonetics}
\end{entry}

\begin{entry}{没有哪一种东西}{7,6,9,1,9,5,6}{⽔、⽉、⼝、⼀、⽲、⼀、⾑}
  \begin{phonetics}{没有哪一种东西}{mei2you3 na3 yi4 zhong3 dong1xi1}
    \definition{pron.}{nada; não existe tal coisa}
  \end{phonetics}
\end{entry}

\begin{entry}{没有谁}{7,6,10}{⽔、⽉、⾔}
  \begin{phonetics}{没有谁}{mei2you3 shei2}
    \definition{pron.}{ninguém}
  \end{phonetics}
\end{entry}

\begin{entry}{没有意思}{7,6,13,9}{⽔、⽉、⼼、⼼}
  \begin{phonetics}{没有意思}{mei2you3yi4si5}
    \definition{adj.}{tedioso | chato | sem interesse}
  \end{phonetics}
\end{entry}

\begin{entry}{没事儿}{7,8,2}{⽔、⼅、⼉}
  \begin{phonetics}{没事儿}{mei2 shi4r5}[][HSK 1]
    \definition{expr.}{fora de perigo; nada sério | não importa; não é nada; está tudo bem; não importa | está tudo bem; sem problemas; não se preocupe com isso; não é grande coisa; não há nada errado}
    \definition{v.}{não ter nada para fazer; ser livre; estar perdido | estar desempregado; estar sem trabalho | não ter responsabilidade}
  \end{phonetics}
\end{entry}

\begin{entry}{没法儿}{7,8,2}{⽔、⽔、⼉}
  \begin{phonetics}{没法儿}{mei2 fa3r5}[][HSK 4]
    \definition{adv.}{não pode; sem chance}
  \end{phonetics}
\end{entry}

\begin{entry}{没想到}{7,13,8}{⽔、⼼、⼑}
  \begin{phonetics}{没想到}{mei2 xiang3 dao4}[][HSK 4]
    \definition{expr.}{não esperava; inesperado}
  \end{phonetics}
\end{entry}

\begin{entry}{没错}{7,13}{⽔、⾦}
  \begin{phonetics}{没错}{mei2 cuo4}[][HSK 4]
    \definition{adv.}{está certo; é isso mesmo; não há como errar}
  \end{phonetics}
\end{entry}

\begin{entry}{河}{8}{⽔}
  \begin{phonetics}{河}{he2}[][HSK 2]
    \definition*{s.}{Astronomia: o sistema da Via Láctea | O Rio Amarelo; O Rio Huanghe | Sobrenome He}
    \definition[条,道]{s.}{rio; refere-se a grandes cursos de água}
  \end{phonetics}
\end{entry}

\begin{entry}{河蚌}{8,10}{⽔、⾍}
  \begin{phonetics}{河蚌}{he2bang4}
    \definition{s.}{mexilhões | bivalves cultivados em rios e lagos}
  \end{phonetics}
\end{entry}

\begin{entry}{油}{8}{⽔}
  \begin{phonetics}{油}{you2}[][HSK 2]
    \definition*{s.}{Sobrenome You}
    \definition{adj.}{oleoso; gorduroso}
    \definition[瓶,滴,层]{s.}{óleo; gordura; graxa; petróleo}
    \definition{v.}{aplicar óleo de tungue, verniz ou tinta | estar manchado ou sujo com óleo ou graxa | aplicar óleo de tungue ou tinta}
  \end{phonetics}
\end{entry}

\begin{entry}{治}{8}{⽔}
  \begin{phonetics}{治}{zhi4}[][HSK 4]
    \definition*{s.}{Sobrenome Zhi}
    \definition{adj.}{calmo e pacífico}
    \definition{s.}{sede de um antigo governo local}
    \definition{v.}{reger; administrar; governar; gerenciar; gerir | tratar (uma doença); curar; sarar | eliminar; controlar pragas | controlar (um rio); restaurar um curso d'água por meio de dragagem | punir; castigar | estudar; pesquisar; explorar}
  \end{phonetics}
\end{entry}

\begin{entry}{治安}{8,6}{⽔、⼧}
  \begin{phonetics}{治安}{zhi4'an1}[][HSK 5]
    \definition{s.}{ordem pública; segurança pública; ordem social estável}
  \end{phonetics}
\end{entry}

\begin{entry}{治疗}{8,7}{⽔、⽧}
  \begin{phonetics}{治疗}{zhi4liao2}[][HSK 4]
    \definition{s.}{diagnóstico; tratamento}
    \definition{v.}{tratar; curar; remediar; eliminar doenças por meio de medicamentos, cirurgia, etc.}
  \end{phonetics}
\end{entry}

\begin{entry}{治理}{8,11}{⽔、⽟}
  \begin{phonetics}{治理}{zhi4li3}[][HSK 5]
    \definition{s.}{governo | governança}
    \definition{v.}{dirigir; gerenciar; governar; administrar | tratar; aproveitar; colocar sob controle; colocar em ordem}
  \end{phonetics}
\end{entry}

\begin{entry}{治愈}{8,13}{⽔、⼼}
  \begin{phonetics}{治愈}{zhi4yu4}
    \definition{v.}{curar | restaurar a saúde}
  \end{phonetics}
\end{entry}

\begin{entry}{沿}{8}{⽔}
  \begin{phonetics}{沿}{yan2}[][HSK 6]
    \definition{prep.}{ao longo}
    \definition{s.}{beira; borda; acabamento}
    \definition{v.}{seguir (uma tradição, padrão, etc.) | enfeitar (com fita, faixa, etc.)}
  \end{phonetics}
\end{entry}

\begin{entry}{泄}{8}{⽔}
  \begin{phonetics}{泄}{xie4}
    \definition*{s.}{Sobrenome Xie}
    \definition{v.}{deixar sair (um fluido ou gás); descarregar; liberar | revelar (um segredo); vazar (notícias, segredos, etc.) | dar vazão a; desabafar}
  \end{phonetics}
\end{entry}

\begin{entry}{泄气}{8,4}{⽔、⽓}
  \begin{phonetics}{泄气}{xie4qi4}
    \definition{adj.}{decepcionante | frustrante | patético}
    \definition{v.+compl.}{perder o coração | sentir-se desencorajado | ficar desanimado}
  \end{phonetics}
\end{entry}

\begin{entry}{泄底}{8,8}{⽔、⼴}
  \begin{phonetics}{泄底}{xie4di3}
    \definition{v.}{revelar ou expor o que está no fundo de algo | divulgar a história interna; vazar segredos}
  \end{phonetics}
\end{entry}

\begin{entry}{泄洪}{8,9}{⽔、⽔}
  \begin{phonetics}{泄洪}{xie4hong2}
    \definition{v.}{liberar água da enchente (descarga de inundação)}
  \end{phonetics}
\end{entry}

\begin{entry}{泄愤}{8,12}{⽔、⼼}
  \begin{phonetics}{泄愤}{xie4fen4}
    \definition{v.}{dar vazão à raiva}
  \end{phonetics}
\end{entry}

\begin{entry}{泄露}{8,21}{⽔、⾬}
  \begin{phonetics}{泄露}{xie4lou4}
    \definition{v.}{vazar; deixar escapar; divulgar; revelar (um segredo, etc.) | vazar; escapar; descarregar (um fluido ou gás)}
  \end{phonetics}
\end{entry}

\begin{entry}{法}{8}{⽔}
  \begin{phonetics}{法}{fa3}[][HSK 4]
    \definition*{s.}{Doutrina budista; o dharma | França, abreviação de 法国 | Sobrenome Fa}
    \definition{adj.}{(usado após advérbios negativos) legal; cumpridor da lei}
    \definition{clas.}{F; Farad, medida de capacitância}
    \definition{s.}{lei; termo geral para regras de comportamento estabelecidas ou endossadas pelo Estado | maneira; método; modo; meios | padrão; modelo | artes mágicas; feitiço}
    \definition{v.}{seguir; imitar; aprender (os pontos fortes dos outros) |}
  \seealsoref{法国}{fa3guo2}
  \end{phonetics}
\end{entry}

\begin{entry}{法文}{8,4}{⽔、⽂}
  \begin{phonetics}{法文}{fa3wen2}
    \definition[份]{s.}{françês, língua francesa}
  \end{phonetics}
\end{entry}

\begin{entry}{法网}{8,6}{⽔、⽹}
  \begin{phonetics}{法网}{fa3wang3}
    \definition*{s.}{Torneio de Roland Garros (French Open), torneio de tênis}
  \end{phonetics}
\end{entry}

\begin{entry}{法制}{8,8}{⽔、⼑}
  \begin{phonetics}{法制}{fa3 zhi4}[][HSK 5]
    \definition{s.}{legalidade; instituições jurídicas; sistema jurídico}
  \end{phonetics}
\end{entry}

\begin{entry}{法国}{8,8}{⽔、⼞}
  \begin{phonetics}{法国}{fa3guo2}
    \definition*{s.}{França}
  \end{phonetics}
\end{entry}

\begin{entry}{法国人}{8,8,2}{⽔、⼞、⼈}
  \begin{phonetics}{法国人}{fa3guo2ren2}
    \definition{s.}{francês | pessoa ou povo da França}
  \end{phonetics}
\end{entry}

\begin{entry}{法官}{8,8}{⽔、⼧}
  \begin{phonetics}{法官}{fa3 guan1}[][HSK 4]
    \definition[位]{s.}{juiz; justiça; termo genérico para um membro do judiciário em um tribunal de justiça}
  \end{phonetics}
\end{entry}

\begin{entry}{法规}{8,8}{⽔、⾒}
  \begin{phonetics}{法规}{fa3 gui1}[][HSK 5]
    \definition{s.}{lei e regulamento; estatuto; termo geral para leis, decretos, regulamentos, regras, estatutos, etc.}
  \end{phonetics}
\end{entry}

\begin{entry}{法庭}{8,9}{⽔、⼴}
  \begin{phonetics}{法庭}{fa3 ting2}[][HSK 6]
    \definition{s.}{corte; tribunal | tribunal; um órgão estatal que exerce o poder judicial de forma independente}
  \end{phonetics}
\end{entry}

\begin{entry}{法律}{8,9}{⽔、⼻}
  \begin{phonetics}{法律}{fa3lv4}[][HSK 4]
    \definition[项,条,套,个]{s.}{lei; estatuto; regras de conduta formuladas pelo legislativo e cuja aplicação é garantida pelo poder estatal}
  \end{phonetics}
\end{entry}

\begin{entry}{法语}{8,9}{⽔、⾔}
  \begin{phonetics}{法语}{fa3 yu3}[][HSK 6]
    \definition[种,门,句,段]{s.}{françês, língua francesa}
  \end{phonetics}
\end{entry}

\begin{entry}{法院}{8,9}{⽔、⾩}
  \begin{phonetics}{法院}{fa3yuan4}[][HSK 4]
    \definition[所,座]{s.}{tribunal; corte; órgãos estatais que exercem poder judicial independente}
  \end{phonetics}
\end{entry}

\begin{entry}{泡}{8}{⽔}
  \begin{phonetics}{泡}{pao1}
    \definition{adj.}{esponjoso; oco e macio; não duro}
    \definition{clas.}{usado para fezes e urina}
    \definition[串,个]{s.}{algo fofo e macio | pequeno lago}
  \end{phonetics}
  \begin{phonetics}{泡}{pao4}[][HSK 6]
    \definition[串,个]{s.}{bolha | algo em forma de bolha}
    \definition{v.}{mergulhar; encharcar | despejar água fervente em (chá, sopa instantânea, etc.) | enrolar; demorar-se; ficar por aí | (coloquial) (de um homem) brincar no campo; brincar com uma mulher | perder tempo; matar o tempo deliberadamente}
  \end{phonetics}
\end{entry}

\begin{entry}{波}{8}{⽔}
  \begin{phonetics}{波}{bo1}
    \definition*{s.}{Polônia, abreviação de 波兰 | Sobrenome Bo}
    \definition{s.}{ondas, a superfície irregular da água em rios, lagos e oceanos | onda, o processo de propagação da vibração | mudanças inesperadas; uma reviravolta inesperada nos acontecimentos; metáfora para mudanças inesperadas nas coisas | olhos; metáfora do olhar errante}
  \seealsoref{波兰}{bo1lan2}
  \end{phonetics}
\end{entry}

\begin{entry}{波兰}{8,5}{⽔、⼋}
  \begin{phonetics}{波兰}{bo1lan2}
    \definition*{s.}{Polônia}
  \end{phonetics}
\end{entry}

\begin{entry}{波动}{8,6}{⽔、⼒}
  \begin{phonetics}{波动}{bo1 dong4}[][HSK 6]
    \definition{s.}{ondulação; flutuação; movimento de onda}
    \definition{v.}{ondular; flutuar}
  \end{phonetics}
\end{entry}

\begin{entry}{波音}{8,9}{⽔、⾳}
  \begin{phonetics}{波音}{bo1yin1}
    \definition*{s.}{Boeing (empresa aeroespacial)}
    \definition{s.}{mordente (música)}
  \end{phonetics}
\end{entry}

\begin{entry}{波浪}{8,10}{⽔、⽔}
  \begin{phonetics}{波浪}{bo1lang4}[][HSK 6]
    \definition{s.}{onda; a superfície irregular da água nos rios, lagos e oceanos, geralmente se refere a águas menores e mais bonitas, frequentemente usada na linguagem falada}
  \end{phonetics}
\end{entry}

\begin{entry}{泥}{8}{⽔}
  \begin{phonetics}{泥}{ni2}[][HSK 6]
    \definition*{s.}{Sobrenome Ni}
    \definition{s.}{lama; atoleiro | pasta ou polpa; amassado | qualquer matéria pastosa; purê de vegetais ou frutas}
  \end{phonetics}
  \begin{phonetics}{泥}{ni4}
    \definition{adj.}{fanático; teimoso; obstinado; cabeçudo}
    \definition{v.}{cobrir ou rebocar com gesso, massa de vidraceiro, etc.}
  \end{phonetics}
\end{entry}

\begin{entry}{泥潭}{8,15}{⽔、⽔}
  \begin{phonetics}{泥潭}{ni2tan2}
    \definition{s.}{atoleiro | lamaçal | charco | pântano}
  \end{phonetics}
\end{entry}

\begin{entry}{注}{8}{⽔}
  \begin{phonetics}{注}{zhu4}
    \definition{s.}{apostas (em jogos de azar) | notas (em um texto)}
    \definition{v.}{derramar; encher | concentrar-se em; fixar-se em; focar em  | anotar; explicar com notas | registrar; gravar | irrigar | dar exegese ou explicação}
  \end{phonetics}
\end{entry}

\begin{entry}{注册}{8,5}{⽔、⼌}
  \begin{phonetics}{注册}{zhu4ce4}[][HSK 5]
    \definition{v.}{inscrever-se; matricular-se; registrar-se; registrar-se junto à autoridade ou escola competente para obter status legal; refere-se especificamente ao usuário de uma determinada rede de computadores que insere o nome de usuário, senha, etc. na rede para obter permissão para usar a rede}
  \end{phonetics}
\end{entry}

\begin{entry}{注册人}{8,5,2}{⽔、⼌、⼈}
  \begin{phonetics}{注册人}{zhu4ce4ren2}
    \definition{s.}{registrante}
  \end{phonetics}
\end{entry}

\begin{entry}{注册表}{8,5,8}{⽔、⼌、⾐}
  \begin{phonetics}{注册表}{zhu4ce4biao3}
    \definition*{s.}{Registro do Windows}
  \end{phonetics}
\end{entry}

\begin{entry}{注册商标}{8,5,11,9}{⽔、⼌、⼝、⽊}
  \begin{phonetics}{注册商标}{zhu4ce4shang1biao1}
    \definition{s.}{marca registrada}
  \end{phonetics}
\end{entry}

\begin{entry}{注视}{8,8}{⽔、⾒}
  \begin{phonetics}{注视}{zhu4shi4}[][HSK 5]
    \definition{v.}{olhar atentamente para; observar atentamente}
  \end{phonetics}
\end{entry}

\begin{entry}{注重}{8,9}{⽔、⾥}
  \begin{phonetics}{注重}{zhu4zhong4}[][HSK 5]
    \definition{v.}{enfatizar; dar ênfase a; dar ênfase a; prestar atenção a; dar importância a}
  \end{phonetics}
\end{entry}

\begin{entry}{注射}{8,10}{⽔、⼨}
  \begin{phonetics}{注射}{zhu4she4}[][HSK 5]
    \definition{v.}{injetar; usar uma seringa para administrar medicamento líquido em um organismo}
  \end{phonetics}
\end{entry}

\begin{entry}{注意}{8,13}{⽔、⼼}
  \begin{phonetics}{注意}{zhu4yi4}[][HSK 3]
    \definition{v.}{prestar atenção; notar; ficar de olho; concentrar os pensamentos em um aspecto específico}
  \end{phonetics}
\end{entry}

\begin{entry}{注意力}{8,13,2}{⽔、⼼、⼒}
  \begin{phonetics}{注意力}{zhu4yi4li4}
    \definition{s.}{atenção}
  \end{phonetics}
\end{entry}

\begin{entry}{注意力缺失症}{8,13,2,10,5,10}{⽔、⼼、⼒、⽸、⼤、⽧}
  \begin{phonetics}{注意力缺失症}{zhu4yi4li4que1shi1zheng4}
    \definition{s.}{transtorno de déficit de atenção}
  \end{phonetics}
\end{entry}

\begin{entry}{注意地}{8,13,6}{⽔、⼼、⼟}
  \begin{phonetics}{注意地}{zhu4yi4di4}
    \definition{s.}{área de cuidado, de observação}
  \end{phonetics}
\end{entry}

\begin{entry}{泪}{8}{⽔}
  \begin{phonetics}{泪}{lei4}[][HSK 4]
    \definition[滴]{s.}{lágrima}
  \end{phonetics}
\end{entry}

\begin{entry}{泪水}{8,4}{⽔、⽔}
  \begin{phonetics}{泪水}{lei4 shui3}[][HSK 4]
    \definition{s.}{lágrima}
  \end{phonetics}
\end{entry}

\begin{entry}{泳}{8}{⽔}
  \begin{phonetics}{泳}{yong3}
    \definition{v.}{nadar}
  \end{phonetics}
\end{entry}

\begin{entry}{泳池}{8,6}{⽔、⽔}
  \begin{phonetics}{泳池}{yong3chi2}
    \definition{s.}{piscina}
  \seealsoref{游泳池}{you2 yong3 chi2}
  \seealsoref{游泳馆}{you2yong3guan3}
  \end{phonetics}
\end{entry}

\begin{entry}{泳衣}{8,6}{⽔、⾐}
  \begin{phonetics}{泳衣}{yong3yi1}
    \definition{s.}{roupa de banho | maiô}
  \seealsoref{游泳衣}{you2yong3yi1}
  \end{phonetics}
\end{entry}

\begin{entry}{泼}{8}{⽔}
  \begin{phonetics}{泼}{po1}[][HSK 5]
    \definition{adj.}{rude e irracional; mal-humorado}
    \definition{v.}{espalhar; salpicar; derramar; derramar ou espalhar o líquido com força para fora |}
  \end{phonetics}
\end{entry}

\begin{entry}{浅}{8}{⽔}
  \begin{phonetics}{浅}{jian1}
    \definition{adj.}{murmurando, fluindo suavemente, gorgolejando suavemente}
    \definition{s.}{(onomatopéia) som de água em movimento |}
  \end{phonetics}
  \begin{phonetics}{浅}{qian3}[][HSK 4]
    \definition{adj.}{raso; superficial;  (em oposição a 深) | fácil; simples; redação, conteúdo, etc. simples e fáceis de entender | superficial; não é profundo em aprendizado, percepção e sabedoria | não próximo; não íntimo; sentimentos não profundos | (cor) claro; pálido;  cor pouco intensa; leve |experiência breve; duração de tempo breve | baixo grau; peso leve; nível baixo}
  \seealsoref{深}{shen1}
  \end{phonetics}
\end{entry}

\begin{entry}{泉}{9}{⽔}
  \begin{phonetics}{泉}{quan2}[][HSK 5]
    \definition*{s.}{Sobrenome Quan}
    \definition[股,眼,汪]{s.}{fonte (de água mineral) | a nascente de um rio | termo antigo para moeda}
  \end{phonetics}
\end{entry}

\begin{entry}{洋}{9}{⽔}
  \begin{phonetics}{洋}{yang2}[][HSK 6]
    \definition*{s.}{Sobrenome Yang}
    \definition{adj.}{vasto; rico; transbordante | estrangeiro (especialmente ocidental) | moderno (oposto a 土)}
    \definition[个,片]{s.}{oceano | moeda de prata}
  \seealsoref{土}{tu3}
  \end{phonetics}
\end{entry}

\begin{entry}{洋葱}{9,12}{⽔、⾋}
  \begin{phonetics}{洋葱}{yang2cong1}
    \definition{s.}{cebola}
  \end{phonetics}
\end{entry}

\begin{entry}{洒}{9}{⽔}
  \begin{phonetics}{洒}{sa3}[][HSK 5]
    \definition{adj.}{natural e sem restrições; confortável (sem restrições)}
    \definition{v.}{derramar; espalhar; borrifar; salpicar; fazer com que (água ou outra coisa) caia de forma dispersa | derramar; cair de forma dispersa}
  \end{phonetics}
\end{entry}

\begin{entry}{洒水}{9,4}{⽔、⽔}
  \begin{phonetics}{洒水}{sa3shui3}
    \definition{v.}{borrifar}
  \end{phonetics}
\end{entry}

\begin{entry}{洗}{9}{⽔}
  \begin{phonetics}{洗}{xi3}[][HSK 1]
    \definition[个]{s.}{pequeno recipiente contendo água para enxaguar os pincéis de escrever | batismo}
    \definition{v.}{lavar; tomar banho; remover a sujeira do objeto com água ou outro solvente | batizar | eliminar; corrigir; reparar | saquear; matar e pilhar; matar ou roubar tudo, como se tivesse sido lavado | revelar filmes; imprimir fotos | apagar; limpar (uma gravação, etc.) | embaralhar (cartas, etc.)}
  \end{phonetics}
\end{entry}

\begin{entry}{洗手}{9,4}{⽔、⼿}
  \begin{phonetics}{洗手}{xi3shou3}
    \definition{v.}{ir ao banheiro | lavar as mãos}
  \end{phonetics}
\end{entry}

\begin{entry}{洗手不干}{9,4,4,3}{⽔、⼿、⼀、⼲}
  \begin{phonetics}{洗手不干}{xi3shou3bu2gan4}
    \definition{v.}{parar totalmente de fazer algo}
  \end{phonetics}
\end{entry}

\begin{entry}{洗手池}{9,4,6}{⽔、⼿、⽔}
  \begin{phonetics}{洗手池}{xi3shou3chi2}
    \definition{s.}{pia de banheiro | lavatório}
  \seealsoref{洗手盆}{xi3shou3pen2}
  \end{phonetics}
\end{entry}

\begin{entry}{洗手间}{9,4,7}{⽔、⼿、⾨}
  \begin{phonetics}{洗手间}{xi3shou3jian1}[][HSK 1]
    \definition[个]{s.}{banheiro; lavatório; lavabo}
  \end{phonetics}
\end{entry}

\begin{entry}{洗手乳}{9,4,8}{⽔、⼿、⼄}
  \begin{phonetics}{洗手乳}{xi3shou3ru3}
    \definition{s.}{sabonete líquido para lavar as mãos}
  \seealsoref{洗手液}{xi3shou3ye4}
  \end{phonetics}
\end{entry}

\begin{entry}{洗手盆}{9,4,9}{⽔、⼿、⽫}
  \begin{phonetics}{洗手盆}{xi3shou3pen2}
    \definition{s.}{pia de banheiro | lavatório}
  \seealsoref{洗手池}{xi3shou3chi2}
  \end{phonetics}
\end{entry}

\begin{entry}{洗手液}{9,4,11}{⽔、⼿、⽔}
  \begin{phonetics}{洗手液}{xi3shou3ye4}
    \definition{s.}{sabonete líquido para lavar as mãos}
  \seealsoref{洗手乳}{xi3shou3ru3}
  \end{phonetics}
\end{entry}

\begin{entry}{洗礼}{9,5}{⽔、⽰}
  \begin{phonetics}{洗礼}{xi3li3}
    \definition{s.}{batismo}
    \definition{v.}{batizar}
  \end{phonetics}
\end{entry}

\begin{entry}{洗衣机}{9,6,6}{⽔、⾐、⽊}
  \begin{phonetics}{洗衣机}{xi3 yi1 ji1}[][HSK 2]
    \definition[台]{s.}{máquina de lavar roupa; eletrodomésticos para lavagem automática ou semiautomática de roupas}
  \end{phonetics}
\end{entry}

\begin{entry}{洗劫}{9,7}{⽔、⼒}
  \begin{phonetics}{洗劫}{xi3jie2}
    \definition{v.}{saquear | pilhar | roubar}
  \end{phonetics}
\end{entry}

\begin{entry}{洗净}{9,8}{⽔、⼎}
  \begin{phonetics}{洗净}{xi3jing4}
    \definition{v.}{lavar (limpeza)}
  \end{phonetics}
\end{entry}

\begin{entry}{洗胃}{9,9}{⽔、⾁}
  \begin{phonetics}{洗胃}{xi3wei4}
    \definition{s.}{(medicina) lavagem gástrica}
    \definition{v.}{ter o estômago lavado}
  \end{phonetics}
\end{entry}

\begin{entry}{洗涤}{9,10}{⽔、⽔}
  \begin{phonetics}{洗涤}{xi3di2}
    \definition{s.}{enxágue | lava}
    \definition{v.}{enxaguar | lavar}
  \end{phonetics}
\end{entry}

\begin{entry}{洗涤间}{9,10,7}{⽔、⽔、⾨}
  \begin{phonetics}{洗涤间}{xi3di2jian1}
    \definition{s.}{lavanderia}
  \end{phonetics}
\end{entry}

\begin{entry}{洗脱}{9,11}{⽔、⾁}
  \begin{phonetics}{洗脱}{xi3tuo1}
    \definition{v.}{limpar | purgar | lavar}
  \end{phonetics}
\end{entry}

\begin{entry}{洗碗}{9,13}{⽔、⽯}
  \begin{phonetics}{洗碗}{xi3wan3}
    \definition{v.}{lavar pratos}
  \end{phonetics}
\end{entry}

\begin{entry}{洗澡}{9,16}{⽔、⽔}
  \begin{phonetics}{洗澡}{xi3zao3}[][HSK 2]
    \definition{v.+compl.}{tomar banho; tomar banho de chuveiro; lavar-se}
  \end{phonetics}
\end{entry}

\begin{entry}{洗澡间}{9,16,7}{⽔、⽔、⾨}
  \begin{phonetics}{洗澡间}{xi3zao3jian1}
    \definition[间]{s.}{banheiro}
  \end{phonetics}
\end{entry}

\begin{entry}{洞}{9}{⽔}
  \begin{phonetics}{洞}{dong4}[][HSK 5]
    \definition{adj.}{profundo; minucioso; claro; completo; abrangente}
    \definition{s.}{buraco; cavidade; orifício; furo; parte penetrante ou profundamente recuada de um objeto; uma caverna}
  \end{phonetics}
\end{entry}

\begin{entry}{洞穴}{9,5}{⽔、⽳}
  \begin{phonetics}{洞穴}{dong4xue2}
    \definition{s.}{caverna}
  \end{phonetics}
\end{entry}

\begin{entry}{洪}{9}{⽔}
  \begin{phonetics}{洪}{hong2}
    \definition*{s.}{Sobrenome Hong}
    \definition{adj.}{alto; vasto | grande; grandioso}
    \definition[场]{s.}{enchente; inundação}
  \end{phonetics}
\end{entry}

\begin{entry}{洪水}{9,4}{⽔、⽔}
  \begin{phonetics}{洪水}{hong2shui3}[][HSK 6]
    \definition[场]{s.}{dilúvio; inundação; enchente; um aumento repentino em um rio causado por chuva forte ou derretimento de neve}
  \end{phonetics}
\end{entry}

\begin{entry}{洲}{9}{⽔}
  \begin{phonetics}{洲}{zhou1}
    \definition{s.}{continente | ilha em um rio}
  \end{phonetics}
\end{entry}

\begin{entry}{活}{9}{⽔}
  \begin{phonetics}{活}{huo2}[][HSK 3]
    \definition{adj.}{vivo; vivendo; indica que (alguma ação) foi realizada enquanto a pessoa ainda estava viva | vívido; animado; ativo | móvel; em movimento; ativo}
    \definition{adv.}{exatamente; simplesmente; expressa um grau elevado, equivalente a 真正 ou 简直}
    \definition{s.}{emprego; meios de subsistência; trabalho (geralmente refere-se a trabalho físico) | produto; algo fabricado}
    \definition{v.}{viver; ter vida; sobreviver (em oposição a 死) | salvar (a vida de uma pessoa); fazer sobreviver; manter a vida}
  \seealsoref{简直}{jian3zhi2}
  \seealsoref{死}{si3}
  \seealsoref{真正}{zhen1zheng4}
  \end{phonetics}
\end{entry}

\begin{entry}{活力}{9,2}{⽔、⼒}
  \begin{phonetics}{活力}{huo2li4}[][HSK 5]
    \definition{s.}{vigor; vitalidade; energia; muito forte, geralmente usado para descrever pessoas, cidades, empresas, economias, etc.}
  \end{phonetics}
\end{entry}

\begin{entry}{活动}{9,6}{⽔、⼒}
  \begin{phonetics}{活动}{huo2dong4}[][HSK 2]
    \definition{adj.}{móvel; flexível para alterações ou mudanças}
    \definition[些,个,种,类,次]{s.}{atividade; ação tomada com o objetivo de alcançar um determinado objetivo}
    \definition{v.}{fazer exercício; movimentar-se | usar influência pessoal; usar meios irregulares | mover-se}
  \end{phonetics}
\end{entry}

\begin{entry}{活泼}{9,8}{⽔、⽔}
  \begin{phonetics}{活泼}{huo2po1}[][HSK 5]
    \definition{adj.}{vívido; ativo; animado; brilhante; vivaz; cheio de vida | reativo; (química) significa que a substância é ativa e reage facilmente com outras substâncias}
  \end{phonetics}
\end{entry}

\begin{entry}{活着}{9,11}{⽔、⽬}
  \begin{phonetics}{活着}{huo2zhe5}
    \definition{adj.}{vivo}
  \end{phonetics}
\end{entry}

\begin{entry}{活跃}{9,11}{⽔、⾜}
  \begin{phonetics}{活跃}{huo2yue4}[][HSK 6]
    \definition{adj.}{ativo; dinâmico; pensamentos, ações ou atividades positivas; ocorrências frequentes | rápido; ativo; dinâmico}
    \definition{v.}{animar; tornar ativo | ser ativo}
  \end{phonetics}
\end{entry}

\begin{entry}{活路}{9,13}{⽔、⾜}
  \begin{phonetics}{活路}{huo2lu4}
    \definition{s.}{maneira de sobreviver | meio de subsistência}
  \end{phonetics}
  \begin{phonetics}{活路}{huo2lu5}
    \definition{s.}{labor | trabalho físico}
  \end{phonetics}
\end{entry}

\begin{entry}{派}{9}{⽔}
  \begin{phonetics}{派}{pai4}[][HSK 3]
    \definition{adj.}{elegante; bonito; imponente}
    \definition{clas.}{usado para grupos, escolas de pensamento ou arte, etc. | usado para um discursos, situações, cenas, etc.}
    \definition[个,块,种]{s.}{panelinha; facção; pessoas com ideias, visões e estilos semelhantes | torta; um alimento recheado comumente consumido pelos ocidentais, geralmente doce | maneira e ar; estilo ou comportamento | afluente; braço de rio}
    \definition{v.}{enviar; despachar; arranjar ou ordenar que uma pessoa faça algo; providenciar transporte | alocar; repartir; distribuir}
  \end{phonetics}
\end{entry}

\begin{entry}{派出}{9,5}{⽔、⼐}
  \begin{phonetics}{派出}{pai4 chu1}[][HSK 6]
    \definition{v.}{despachar; expedi | enviar}
  \end{phonetics}
\end{entry}

\begin{entry}{浊}{9}{⽔}
  \begin{phonetics}{浊}{zhuo2}
    \definition*{s.}{Sobrenome Zhuo}
    \definition{adj.}{turvo; lamacento; imundo (oposto a 清) | profundo e espesso | caótico; confuso; corrompido}
  \seealsoref{清}{qing1}
  \end{phonetics}
\end{entry}

\begin{entry}{测}{9}{⽔}
  \begin{phonetics}{测}{ce4}[][HSK 4]
    \definition{v.}{pesquisar; sondar; medir | conjecturar; advinhar}
  \end{phonetics}
\end{entry}

\begin{entry}{测定}{9,8}{⽔、⼧}
  \begin{phonetics}{测定}{ce4 ding4}[][HSK 6]
    \definition{v.}{verificar por medição (ou levantamento); determinar; medir; avaliar}
  \end{phonetics}
\end{entry}

\begin{entry}{测试}{9,8}{⽔、⾔}
  \begin{phonetics}{测试}{ce4 shi4}[][HSK 4]
    \definition[个]{s.}{exame; teste; medição do conhecimento humano, das habilidades ou do funcionamento de máquinas, ferramentas ou instrumentos}
    \definition{v.}{examinar | testar, medição do desempenho e da precisão de máquinas, instrumentos, aparelhos, etc.}
  \end{phonetics}
\end{entry}

\begin{entry}{测量}{9,12}{⽔、⾥}
  \begin{phonetics}{测量}{ce4liang2}[][HSK 4]
    \definition{v.}{aferir; pesquisar; medir; determinar valores relevantes para espaço, tempo, temperatura, velocidade, função, etc.}
  \end{phonetics}
\end{entry}

\begin{entry}{浓}{9}{⽔}
  \begin{phonetics}{浓}{nong2}[][HSK 4]
    \definition{adj.}{denso; espesso; concentrado; um líquido ou gás que contém mais de um determinado ingrediente | grande; forte; profundo (de grau ou extensão) | profundo; (algumas cores) escuro}
  \end{phonetics}
\end{entry}

\begin{entry}{流}{10}{⽔}
  \begin{phonetics}{流}{liu2}[][HSK 2]
    \definition*{s.}{Sobrenome Liu}
    \definition{adj.}{fluente; tão suave quanto a água corrente}
    \definition{clas.}{lúmen; abreviação de lumens, 流明}
    \definition[名,个]{s.}{corrente de água | corrente; algo que se assemelha a um fluxo de água | razão; taxa; classe; grau; ramificação; facção; hierarquia}
    \definition{v.}{(de líquido) fluir | vaguear; vagar; mover-se de um lugar para outro; movimentar-se sem direção fixa | espalhar; circular; transmitir; divulgar | degenerar; mudar para pior; tender (aspecto negativo) | banir; enviar para o exílio | correr (ou fluir) como líquido; refere-se à parte do rio após deixar sua nascente (em contraste com a 源)}
  \seealsoref{流明}{liu2ming2}
  \seealsoref{源}{yuan2}
  \end{phonetics}
\end{entry}

\begin{entry}{流水}{10,4}{⽔、⽔}
  \begin{phonetics}{流水}{liu2shui3}
    \definition{s.}{água corrente | (negócio) rotatividade}
  \end{phonetics}
\end{entry}

\begin{entry}{流传}{10,6}{⽔、⼈}
  \begin{phonetics}{流传}{liu2chuan2}[][HSK 4]
    \definition{v.}{espalhar; circular; passar adiante}
  \end{phonetics}
\end{entry}

\begin{entry}{流动}{10,6}{⽔、⼒}
  \begin{phonetics}{流动}{liu2 dong4}[][HSK 5]
    \definition{v.}{fluir; correr; circular | ir de um lugar para outro; mover-se; mudar frequentemente de posição}
  \end{phonetics}
\end{entry}

\begin{entry}{流行}{10,6}{⽔、⾏}
  \begin{phonetics}{流行}{liu2xing2}[][HSK 2]
    \definition{adj.}{popular; na moda; muito popular}
    \definition{v.}{ser popular; prevalecer; espalhar-se amplamente; divulgar amplamente}
  \end{phonetics}
\end{entry}

\begin{entry}{流行性感冒}{10,6,8,13,9}{⽔、⾏、⼼、⼼、⽇}
  \begin{phonetics}{流行性感冒}{liu2xing2 xing4 gan3mao4}
    \definition{s.}{gripe muito forte; influenza}
  \end{phonetics}
\end{entry}

\begin{entry}{流利}{10,7}{⽔、⼑}
  \begin{phonetics}{流利}{liu2li4}[][HSK 2]
    \definition{adj.}{fluente; suave; lúcido; falar e escrever com fluência e clareza | com fluência; sem dificuldades}
  \end{phonetics}
\end{entry}

\begin{entry}{流明}{10,8}{⽔、⽇}
  \begin{phonetics}{流明}{liu2ming2}
    \definition{s.}{(empréstimo linguístico) lúmen (unidade de fluxo luminoso)}
  \end{phonetics}
\end{entry}

\begin{entry}{流星}{10,9}{⽔、⽇}
  \begin{phonetics}{流星}{liu2xing1}
    \definition{s.}{meteoro | estrela cadente}
  \end{phonetics}
\end{entry}

\begin{entry}{流通}{10,10}{⽔、⾡}
  \begin{phonetics}{流通}{liu2tong1}[][HSK 5]
    \definition{v.}{(de ar, dinheiro, mercadorias, etc.) fluir; circular}
  \end{phonetics}
\end{entry}

\begin{entry}{流感}{10,13}{⽔、⼼}
  \begin{phonetics}{流感}{liu2 gan3}[][HSK 6]
    \definition{s.}{gripe; influenza; abreviação de 流行性感冒}
  \seealsoref{流行性感冒}{liu2xing2 xing4 gan3mao4}
  \end{phonetics}
\end{entry}

\begin{entry}{浙}{10}{⽔}
  \begin{phonetics}{浙}{zhe4}
    \definition{s.}{abreviação de província de Zhejiang,  浙江, no leste da China}
  \seealsoref{浙江}{zhe4jiang1}
  \end{phonetics}
\end{entry}

\begin{entry}{浙江}{10,6}{⽔、⽔}
  \begin{phonetics}{浙江}{zhe4jiang1}
    \definition*{s.}{Província de Zhejiang}
  \end{phonetics}
\end{entry}

\begin{entry}{浪}{10}{⽔}
  \begin{phonetics}{浪}{lang4}
    \definition*{s.}{Sobrenome Lang}
    \definition{adj.}{desenfreado; perdulário}
    \definition{adv.}{livremente}
    \definition[朵,阵,波]{s.}{onda; vagalhão; rebentação | algo ondulatório | coisas ondulando como ondas}
    \definition{v.}{passear; divagar}
  \end{phonetics}
\end{entry}

\begin{entry}{浪花}{10,7}{⽔、⾋}
  \begin{phonetics}{浪花}{lang4hua1}
    \definition[朵]{s.}{\emph{spray} | \emph{spray} do oceano | (figurativo) acontecimentos de sua vida}
  \end{phonetics}
\end{entry}

\begin{entry}{浪费}{10,9}{⽔、⾙}
  \begin{phonetics}{浪费}{lang4fei4}[][HSK 3]
    \definition{adj.}{desperdiçado; extravagante; não econômico}
    \definition{v.}{desperdiçar; dissipar; esbanjar; ser extravagante; uso excessivo ou inadequado de bens, recursos humanos, tempo, etc.}
  \end{phonetics}
\end{entry}

\begin{entry}{浪漫}{10,14}{⽔、⽔}
  \begin{phonetics}{浪漫}{lang4man4}[][HSK 5]
    \definition{adj.}{romântico; poético | não convencional; boêmio; abandonado; libertino; devasso; comportar-se de maneira descuidada e descuidada (geralmente se referindo a relacionamentos entre homens e mulheres) | irrealista; impraticável}
  \end{phonetics}
\end{entry}

\begin{entry}{浮}{10}{⽔}
  \begin{phonetics}{浮}{fu2}[][HSK 6]
    \definition*{s.}{Sobrenome Fu}
    \definition{adj.}{superficial; na superfície | móvel; removível | temporário; provisório | superficial e frívolo; volátil; impetuoso | oco; vazio; inflado | excessivo; excedente}
    \definition{v.}{flutuar (oposto a 沉) | (dialeto) nadar | flutuar; derivar; flutuar na superfície do líquido}
  \seealsoref{沉}{chen2}
  \end{phonetics}
\end{entry}

\begin{entry}{浮力}{10,2}{⽔、⼒}
  \begin{phonetics}{浮力}{fu2li4}
    \definition{s.}{flutuabilidade}
  \end{phonetics}
\end{entry}

\begin{entry}{浮图}{10,8}{⽔、⼞}
  \begin{phonetics}{浮图}{fu2tu2}
    \definition*{s.}{Termo alternativo para 佛陀}
    \variantof{浮屠}
  \seealsoref{佛陀}{fo2tuo2}
  \end{phonetics}
\end{entry}

\begin{entry}{浮屠}{10,11}{⽔、⼫}
  \begin{phonetics}{浮屠}{fu2tu2}
    \definition*{s.}{Buda | Templo (Stupa) Budista (transliteração de Pali Thuo)}
  \end{phonetics}
\end{entry}

\begin{entry}{海}{10}{⽔}
  \begin{phonetics}{海}{hai3}[][HSK 2]
    \definition*{s.}{Sobrenome Hai}
    \definition{adj.}{extragrande; de grande capacidade; descreve capacidade, tom de voz, etc.}
    \definition{adv.}{aleatoriamente; sem rumo; sem limites; sem restrições}
    \definition[片]{s.}{mar; grande lago; a parte do oceano próxima à costa, alguns grandes lagos também são chamados de mar | grande número de pessoas ou coisas reunidas; metáfora para muitas coisas semelhantes que formam um grande conjunto}
  \end{phonetics}
\end{entry}

\begin{entry}{海水}{10,4}{⽔、⽔}
  \begin{phonetics}{海水}{hai3 shui3}[][HSK 4]
    \definition[把]{s.}{água do mar; salmoura}
  \end{phonetics}
\end{entry}

\begin{entry}{海风}{10,4}{⽔、⾵}
  \begin{phonetics}{海风}{hai3feng1}
    \definition{s.}{brisa do mar | vento que vem do mar}
  \end{phonetics}
\end{entry}

\begin{entry}{海外}{10,5}{⽔、⼣}
  \begin{phonetics}{海外}{hai3 wai4}[][HSK 6]
    \definition[次]{s.}{fora das fronteiras nacionais; no exterior}
  \end{phonetics}
\end{entry}

\begin{entry}{海边}{10,5}{⽔、⾡}
  \begin{phonetics}{海边}{hai3 bian1}[][HSK 2]
    \definition{s.}{praia; costa; litoral; orla marítima; a parte marginal do oceano e as grandes áreas de água salgada cercadas por terra firme, onde a terra e a água se encontram, formam a costa}
  \end{phonetics}
\end{entry}

\begin{entry}{海关}{10,6}{⽔、⼋}
  \begin{phonetics}{海关}{hai3guan1}[][HSK 3]
    \definition[个]{s.}{alfândega; órgão administrativo nacional, sua principal função é supervisionar e inspecionar os bens e meios de transporte que entram e saem do país, cobrar impostos alfandegários e reprimir o contrabando}
  \end{phonetics}
\end{entry}

\begin{entry}{海军}{10,6}{⽔、⼍}
  \begin{phonetics}{海军}{hai3 jun1}[][HSK 6]
    \definition[支,名,位,个]{s.}{marinha; o exército que luta no mar geralmente é composto por navios de superfície, submarinos, aviação naval, fuzileiros navais e outros ramos, além de diversas forças profissionais}
  \end{phonetics}
\end{entry}

\begin{entry}{海报}{10,7}{⽔、⼿}
  \begin{phonetics}{海报}{hai3 bao4}[][HSK 6]
    \definition[张,份,幅]{s.}{pôster; cartaz; cartazes anunciando apresentações culturais, exibições de filmes ou competições esportivas, etc.}
  \end{phonetics}
\end{entry}

\begin{entry}{海里}{10,7}{⽔、⾥}
  \begin{phonetics}{海里}{hai3li3}
    \definition{s.}{milha náutica}
  \end{phonetics}
\end{entry}

\begin{entry}{海底}{10,8}{⽔、⼴}
  \begin{phonetics}{海底}{hai3 di3}[][HSK 6]
    \definition{s.}{fundo do mar; fundo do oceano; solo oceânico}
  \end{phonetics}
\end{entry}

\begin{entry}{海洋}{10,9}{⽔、⽔}
  \begin{phonetics}{海洋}{hai3yang2}[][HSK 6]
    \definition[片,个]{s.}{mar; oceano; um termo geral para os mares e oceanos que formam uma entidade contínua na superfície da Terra; também pode ser usado para descrever um grande número de coisas semelhantes}
  \end{phonetics}
\end{entry}

\begin{entry}{海鸥}{10,9}{⽔、⿃}
  \begin{phonetics}{海鸥}{hai3'ou1}
    \definition{s.}{gaivota}
  \end{phonetics}
\end{entry}

\begin{entry}{海浪}{10,10}{⽔、⽔}
  \begin{phonetics}{海浪}{hai3 lang4}[][HSK 6]
    \definition{s.}{ondas do mar}
  \end{phonetics}
\end{entry}

\begin{entry}{海绵}{10,11}{⽔、⽷}
  \begin{phonetics}{海绵}{hai3mian2}
    \definition{s.}{(zoologia) esponja do mar | esponja (feita de poliéster ou celulose, etc.) | espuma de borracha}
  \end{phonetics}
\end{entry}

\begin{entry}{海棠}{10,12}{⽔、⽊}
  \begin{phonetics}{海棠}{hai3tang2}
    \definition{s.}{begônia}
  \end{phonetics}
\end{entry}

\begin{entry}{海湾}{10,12}{⽔、⽔}
  \begin{phonetics}{海湾}{hai3 wan1}[][HSK 6]
    \definition{s.}{baía; golfo | lago}
  \end{phonetics}
\end{entry}

\begin{entry}{海鲜}{10,14}{⽔、⿂}
  \begin{phonetics}{海鲜}{hai3xian1}[][HSK 4]
    \definition[种,份,桌,批,些]{s.}{frutos do mar; mariscos; peixes marinhos frescos, camarões, etc., para consumo |}
  \end{phonetics}
\end{entry}

\begin{entry}{消}{10}{⽔}
  \begin{phonetics}{消}{xiao1}
    \definition{v.}{desaparecer | dissipar; remover; eliminar; fazer desaparecer | passar o tempo de forma descontraída (recreação) | precisar; tomar (necessidade, geralmente precedido por 不, 几, 何)}
  \seealsoref{不}{bu4}
  \seealsoref{何}{he2}
  \seealsoref{几}{ji3}
  \end{phonetics}
\end{entry}

\begin{entry}{消化}{10,4}{⽔、⼔}
  \begin{phonetics}{消化}{xiao1hua4}[][HSK 4]
    \definition{v.}{digerir (alimentos) | digerir (conhecimento); pensar e absorver; uma metáfora para a compreensão total de novos conhecimentos ou informações e a capacidade de transformá-los em algo que possa ser usado}
  \end{phonetics}
\end{entry}

\begin{entry}{消失}{10,5}{⽔、⼤}
  \begin{phonetics}{消失}{xiao1shi1}[][HSK 3]
    \definition{v.}{desaparecer; desvanecer; dissolver; dissipar; evaporar; sumir}
  \end{phonetics}
\end{entry}

\begin{entry}{消防}{10,6}{⽔、⾩}
  \begin{phonetics}{消防}{xiao1fang2}[][HSK 5]
    \definition{s.}{combate a incêncios; controle de incêndios}
  \end{phonetics}
\end{entry}

\begin{entry}{消防员}{10,6,7}{⽔、⾩、⼝}
  \begin{phonetics}{消防员}{xiao1fang2yuan2}
    \definition{s.}{bombeiro}
  \end{phonetics}
\end{entry}

\begin{entry}{消极}{10,7}{⽔、⽊}
  \begin{phonetics}{消极}{xiao1ji2}[][HSK 5]
    \definition{adj.}{negativo; oposto; adverso | passivo; inativo; sem ambição; sem iniciativa; desanimado; apático}
  \end{phonetics}
\end{entry}

\begin{entry}{消毒}{10,9}{⽔、⽏}
  \begin{phonetics}{消毒}{xiao1du2}[][HSK 5]
    \definition{v.}{desinfetar; esterilizar; matar os microrganismos causadores de doenças por meios físicos ou químicos}
  \end{phonetics}
\end{entry}

\begin{entry}{消费}{10,9}{⽔、⾙}
  \begin{phonetics}{消费}{xiao1fei4}[][HSK 3]
    \definition{v.}{gastar; consumir; consumir materiais para satisfazer as necessidades de produção ou de vida (geralmente refere-se ao consumo doméstico) | consumir (recursos naturais)}
  \end{phonetics}
\end{entry}

\begin{entry}{消费者}{10,9,8}{⽔、⾙、⽼}
  \begin{phonetics}{消费者}{xiao1 fei4 zhe3}[][HSK 5]
    \definition{s.}{consumidor; cliente; consumo; indivíduos membros da sociedade que compram e utilizam bens e serviços para consumo pessoal}
  \end{phonetics}
\end{entry}

\begin{entry}{消除}{10,9}{⽔、⾩}
  \begin{phonetics}{消除}{xiao1chu2}[][HSK 5]
    \definition{v.}{dissipar; remover; eliminar; limpar; enxugar}
  \end{phonetics}
\end{entry}

\begin{entry}{消息}{10,10}{⽔、⼼}
  \begin{phonetics}{消息}{xiao1xi5}[][HSK 3]
    \definition[个,条,篇,些]{s.}{notícias; informação; reportagem sobre pessoas ou situações | notícias; novidades;}
  \end{phonetics}
\end{entry}

\begin{entry}{涉}{10}{⽔}
  \begin{phonetics}{涉}{she4}[][HSK 6]
    \definition*{s.}{Sobrenome She}
    \definition{v.}{vadear; atravessar ou passar um rio ou um obstáculo | passar por; experimentar | envolver; implicar}
  \end{phonetics}
\end{entry}

\begin{entry}{涉及}{10,3}{⽔、⼃}
  \begin{phonetics}{涉及}{she4ji2}[][HSK 6]
    \definition{v.}{envolver; relacionar-se com; referir-se a; tocar em}
  \end{phonetics}
\end{entry}

\begin{entry}{涨}{10}{⽔}
  \begin{phonetics}{涨}{zhang3}[][HSK 5,6]
    \definition{v.}{subir; inchar; aumentar; elevar; melhorar}
  \end{phonetics}
  \begin{phonetics}{涨}{zhang4}
    \definition{v.}{inchar; ter o volume aumentado | ser inundado por uma torrente de sangue; ter uma dor de cabeça; ficar com o rosto vermelho de raiva | ser mais, maior, etc. do que o esperado}
  \end{phonetics}
\end{entry}

\begin{entry}{涨价}{10,6}{⽔、⼈}
  \begin{phonetics}{涨价}{zhang3 jia4}[][HSK 5]
    \definition{s.}{aumento de preços}
    \definition{v.+compl.}{(de preços) subir; aumentar o preço}
  \end{phonetics}
\end{entry}

\begin{entry}{液}{11}{⽔}
  \begin{phonetics}{液}{ye4}
    \definition{s.}{líquido; fluido; suco}
  \end{phonetics}
\end{entry}

\begin{entry}{液体}{11,7}{⽔、⼈}
  \begin{phonetics}{液体}{ye4ti3}
    \definition{adj./s.}{líquido}
  \end{phonetics}
\end{entry}

\begin{entry}{涵}{11}{⽔}
  \begin{phonetics}{涵}{han2}
    \definition{s.}{bueiro; galeria}
    \definition{v.}{conter; incorporar}
  \end{phonetics}
\end{entry}

\begin{entry}{淀}{11}{⽔}
  \begin{phonetics}{淀}{dian4}
    \definition{s.}{lago raso, frequentemente usado em nomes de lugares}
    \definition{v.}{formar sedimentos | sedimentar; precipitar}
  \end{phonetics}
\end{entry}

\begin{entry}{淋}{11}{⽔}
  \begin{phonetics}{淋}{lin2}
    \definition{v.}{borrifar | pingar | derramar | encharcar}
  \end{phonetics}
  \begin{phonetics}{淋}{lin4}
    \definition{s.}{gonorréia}
    \definition{v.}{filtrar | coar | drenar}
  \end{phonetics}
\end{entry}

\begin{entry}{淡}{11}{⽔}
  \begin{phonetics}{淡}{dan4}[][HSK 4]
    \definition*{s.}{Sobrenome Dan}
    \definition{adj.}{sem gosto; fraco; não tem sabor forte; não é salgado | leve; fraco; pálido | indiferente; frio; sem entusiasmo | frouxo; sem brilho | sem sentido; trivial | fino; leve}
  \end{phonetics}
\end{entry}

\begin{entry}{淤}{11}{⽔}
  \begin{phonetics}{淤}{yu1}
    \definition{adj.}{assoreado}
    \definition{s.}{lodo}
    \definition[出]{s.}{(medicina chinesa) estase de sangue}
    \definition{v.}{ficar assoreado; ficar sufocado com lodo | derramar; transbordar}
  \end{phonetics}
\end{entry}

\begin{entry}{淤泥}{11,8}{⽔、⽔}
  \begin{phonetics}{淤泥}{yu1ni2}
    \definition{s.}{lodo}
  \end{phonetics}
\end{entry}

\begin{entry}{深}{11}{⽔}
  \begin{phonetics}{深}{shen1}[][HSK 3]
    \definition*{s.}{Sobrenome Shen}
    \definition{adj.}{profundo | difícil; intenso; profundo | completo; penetrante; intenso; profundo | próximo; íntimo; afeição profunda; relacionamento próximo | escuro; profundo | tardio}
    \definition{adv.}{muito; grandemente; profundamente}
    \definition{s.}{profundidade}
  \seealsoref{浅}{qian3}
  \end{phonetics}
\end{entry}

\begin{entry}{深入}{11,2}{⽔、⼊}
  \begin{phonetics}{深入}{shen1 ru4}[][HSK 3]
    \definition{adj.}{profundo; completo}
    \definition{v.}{ir fundo em; penetrar em; penetrar o exterior; alcançar o interior ou o centro de algo}
  \end{phonetics}
\end{entry}

\begin{entry}{深化}{11,4}{⽔、⼔}
  \begin{phonetics}{深化}{shen1 hua4}[][HSK 6]
    \definition{v.}{aprofundar; avançar; intensificar; tornar-se mais profundo; tornar mais profundo}
  \end{phonetics}
\end{entry}

\begin{entry}{深处}{11,5}{⽔、⼡}
  \begin{phonetics}{深处}{shen1 chu4}[][HSK 5]
    \definition{s.}{profundidades; recantos; recessos | profundezas}
  \end{phonetics}
\end{entry}

\begin{entry}{深刻}{11,8}{⽔、⼑}
  \begin{phonetics}{深刻}{shen1ke4}[][HSK 3]
    \definition{adj.}{profundo; instenso; chegar à essência de um assunto ou problema}
  \end{phonetics}
\end{entry}

\begin{entry}{深夜}{11,8}{⽔、⼣}
  \begin{phonetics}{深夜}{shen1ye4}
    \definition{adv.}{tarde da noite}
  \end{phonetics}
\end{entry}

\begin{entry}{深厚}{11,9}{⽔、⼚}
  \begin{phonetics}{深厚}{shen1hou4}[][HSK 4]
    \definition{adj.}{profundo; sentimentos fortes | sólido; profundamente enraizado; fundação sólida}
  \end{phonetics}
\end{entry}

\begin{entry}{深度}{11,9}{⽔、⼴}
  \begin{phonetics}{深度}{shen1 du4}[][HSK 5]
    \definition{adj.}{(em grau ou extensão) profundo; sério; grave}
    \definition{s.}{profundidade; grau de profundidade; | profundidade; rigor; meticulosidade; grau de contato com a essência das coisas | estágio avançado (ou em deterioração) de desenvolvimento; grau de crescimento e desenvolvimento das coisas}
  \end{phonetics}
\end{entry}

\begin{entry}{深深}{11,11}{⽔、⽔}
  \begin{phonetics}{深深}{shen1 shen1}[][HSK 6]
    \definition{adj.}{profundo; intenso}
    \definition{adv.}{profundamente; intensamente; descreve um grau profundo ou forte}
  \end{phonetics}
\end{entry}

\begin{entry}{混}{11}{⽔}
  \begin{phonetics}{混}{hun4}[][HSK 6]
    \definition{adj.}{confuso; imundo; turvo; lamacento; impuro}
    \definition{adv.}{de forma imprudente; irresponsável; irrefletidamente}
    \definition{v.}{misturar; confundir; misturar verdadeiro e falso | passar por; esgueirar-se | vagar à deriva; arrastar-se; sobreviver de maneira superficial; contentar-se com | se dar bem com alguém}
  \end{phonetics}
\end{entry}

\begin{entry}{混合}{11,6}{⽔、⼝}
  \begin{phonetics}{混合}{hun4he2}[][HSK 6]
    \definition{s.}{híbrido; composto; refere-se a duas ou mais substâncias misturadas sem reação química, mas ainda mantendo suas respectivas propriedades (diferente de 化合)}
    \definition{v.}{misturar; mixar; misturar-se}
  \seealsoref{化合}{hua4he2}
  \end{phonetics}
\end{entry}

\begin{entry}{混乱}{11,7}{⽔、⼄}
  \begin{phonetics}{混乱}{hun4luan4}[][HSK 6]
    \definition{adj.}{caótico; confuso; desordenado; desorganizado; fora de ordem}
    \definition[片]{s.}{caos; confusão}
  \end{phonetics}
\end{entry}

\begin{entry}{混饭}{11,7}{⽔、⾷}
  \begin{phonetics}{混饭}{hun4fan4}
    \definition{v.+compl.}{trabalhar para viver}
  \end{phonetics}
\end{entry}

\begin{entry}{添}{11}{⽔}
  \begin{phonetics}{添}{tian1}[][HSK 6]
    \definition{v.}{adicionar; aumentar | dar à luz}
  \end{phonetics}
\end{entry}

\begin{entry}{清}{11}{⽔}
  \begin{phonetics}{清}{qing1}[][HSK 6]
    \definition*{s.}{Dinastia Qing (1644-1911) | Sobrenome Qing}
    \definition{adj.}{claro; não misturado; (líquido ou gasoso) puro e sem mistura (em oposição a 浊) | silencioso; quieto | justo e honesto | distinto; claro; esclarecido | simples; puro, sem qualquer adulteração ou combinação | limpo; puro}
    \definition{v.}{limpar; tornar limpo | resolver; esclarecer; pagar; liquidar | contar; inspecionar}
  \seealsoref{浊}{zhuo2}
  \end{phonetics}
\end{entry}

\begin{entry}{清彻}{11,7}{⽔、⼻}
  \begin{phonetics}{清彻}{qing1che4}
    \variantof{清澈}
  \end{phonetics}
\end{entry}

\begin{entry}{清明节}{11,8,5}{⽔、⽇、⾋}
  \begin{phonetics}{清明节}{qing1 ming2 jie2}[][HSK 6]
    \definition*{s.}{Qingming ou Festival do Brilho Puro ou Dia da Varredura de Túmulos, Dia dos Finados (uma das 24~divisões do ano solar no calendário lunar chinês:~dia~4 ou 5~de~abril solar)}
  \end{phonetics}
\end{entry}

\begin{entry}{清洁}{11,9}{⽔、⽔}
  \begin{phonetics}{清洁}{qing1jie2}[][HSK 6]
    \definition{adj.}{limpo; sem poeira, gordura, etc.}
    \definition{v.}{limpar}
  \end{phonetics}
\end{entry}

\begin{entry}{清洁工}{11,9,3}{⽔、⽔、⼯}
  \begin{phonetics}{清洁工}{qing1 jie2 gong1}[][HSK 6]
    \definition{s.}{coletor de lixo; trabalhador de saneamento; limpador de rua; trabalhadores envolvidos na limpeza do ambiente, remoção de lixo e fezes, etc.}
  \end{phonetics}
\end{entry}

\begin{entry}{清洗}{11,9}{⽔、⽔}
  \begin{phonetics}{清洗}{qing1 xi3}[][HSK 6]
    \definition{v.}{enxaguar; lavar; limpar | purgar; limpar | eliminar}
  \end{phonetics}
\end{entry}

\begin{entry}{清凉}{11,10}{⽔、⼎}
  \begin{phonetics}{清凉}{qing1liang2}
    \definition{adj.}{fresco | refrescante | (roupa) ousada, reveladora}
  \end{phonetics}
\end{entry}

\begin{entry}{清唱}{11,11}{⽔、⼝}
  \begin{phonetics}{清唱}{qing1chang4}
    \definition{v.}{cantar à capela}
  \end{phonetics}
\end{entry}

\begin{entry}{清晨}{11,11}{⽔、⽇}
  \begin{phonetics}{清晨}{qing1chen2}[][HSK 5]
    \definition{s.}{matinal; manhã cedo; geralmente se refere ao período do amanhecer até logo após o nascer do sol}
  \end{phonetics}
\end{entry}

\begin{entry}{清爽}{11,11}{⽔、⽘}
  \begin{phonetics}{清爽}{qing1shuang3}
    \definition{adj.}{refrescante | relaxado}
  \end{phonetics}
\end{entry}

\begin{entry}{清理}{11,11}{⽔、⽟}
  \begin{phonetics}{清理}{qing1li3}[][HSK 5]
    \definition{v.}{esclarecer; resolver; verificar; colocar em ordem; organizar tudo e jogar fora o que não for útil}
  \end{phonetics}
\end{entry}

\begin{entry}{清晰}{11,12}{⽔、⽇}
  \begin{phonetics}{清晰}{qing1xi1}
    \definition{adj.}{claro | distinto}
  \end{phonetics}
\end{entry}

\begin{entry}{清楚}{11,13}{⽔、⽊}
  \begin{phonetics}{清楚}{qing1chu5}[][HSK 2]
    \definition{adj.}{claro; distinto; compreensível; organizado; fácil de identificar e entender | plenamente consciente de; claro sobre}
    \definition{v.}{ter clareza sobre; compreender; ação que expressa compreensão e conhecimento}
  \end{phonetics}
\end{entry}

\begin{entry}{清澈}{11,15}{⽔、⽔}
  \begin{phonetics}{清澈}{qing1che4}
    \definition{adj.}{claro | límpido}
  \end{phonetics}
\end{entry}

\begin{entry}{清醒}{11,16}{⽔、⾣}
  \begin{phonetics}{清醒}{qing1xing3}[][HSK 4]
    \definition{adj.}{sóbrio; lúcido; totalmente acordado}
  \end{phonetics}
\end{entry}

\begin{entry}{渐}{11}{⽔}
  \begin{phonetics}{渐}{jian1}
    \definition{v.}{encharcar; ficar saturado com | fluir para}
  \end{phonetics}
  \begin{phonetics}{渐}{jian4}
    \definition{adv.}{gradualmente; por graus}
  \end{phonetics}
\end{entry}

\begin{entry}{渐渐}{11,11}{⽔、⽔}
  \begin{phonetics}{渐渐}{jian4 jian4}[][HSK 4]
    \definition{adv.}{gradualmente; pouco a pouco; passo a passo; indica um aumento ou diminuição gradual em grau ou quantidade}
  \end{phonetics}
\end{entry}

\begin{entry}{渔}{11}{⽔}
  \begin{phonetics}{渔}{yu2}
    \definition[条]{s.}{pescador}
    \definition{v.}{pescar}
  \end{phonetics}
\end{entry}

\begin{entry}{渔夫}{11,4}{⽔、⼤}
  \begin{phonetics}{渔夫}{yu2fu1}
    \definition{s.}{pescador}
  \end{phonetics}
\end{entry}

\begin{entry}{渔民}{11,5}{⽔、⽒}
  \begin{phonetics}{渔民}{yu2min2}
    \definition{s.}{pescadores | povo pescador}
  \end{phonetics}
\end{entry}

\begin{entry}{渔场}{11,6}{⽔、⼟}
  \begin{phonetics}{渔场}{yu2chang3}
    \definition{s.}{área de pesca}
  \end{phonetics}
\end{entry}

\begin{entry}{渔汛}{11,6}{⽔、⽔}
  \begin{phonetics}{渔汛}{yu2xun4}
    \definition{s.}{temporada de pesca}
  \end{phonetics}
\end{entry}

\begin{entry}{渔网}{11,6}{⽔、⽹}
  \begin{phonetics}{渔网}{yu2wang3}
    \definition{s.}{rede de pesca}
  \end{phonetics}
\end{entry}

\begin{entry}{渔具}{11,8}{⽔、⼋}
  \begin{phonetics}{渔具}{yu2ju4}
    \definition{s.}{equipamento de pesca}
  \end{phonetics}
\end{entry}

\begin{entry}{渔轮}{11,8}{⽔、⾞}
  \begin{phonetics}{渔轮}{yu2lun2}
    \definition{s.}{navio de pesca}
  \end{phonetics}
\end{entry}

\begin{entry}{渔捞}{11,10}{⽔、⼿}
  \begin{phonetics}{渔捞}{yu2lao1}
    \definition{s.}{pesca (como atividade comercial)}
  \end{phonetics}
\end{entry}

\begin{entry}{渔猎}{11,11}{⽔、⽝}
  \begin{phonetics}{渔猎}{yu2lie4}
    \definition{s.}{pesca e caça}
    \definition{v.}{saquear | pilhar}
  \end{phonetics}
\end{entry}

\begin{entry}{渔笼}{11,11}{⽔、⽵}
  \begin{phonetics}{渔笼}{yu2long2}
    \definition{s.}{gaiola de pesca | armadilha de pesca}
  \end{phonetics}
\end{entry}

\begin{entry}{渔船}{11,11}{⽔、⾈}
  \begin{phonetics}{渔船}{yu2chuan2}
    \definition[条]{s.}{barco de pesca}
  \seealsoref{鱼船}{yu2chuan2}
  \end{phonetics}
\end{entry}

\begin{entry}{渔船队}{11,11,4}{⽔、⾈、⾩}
  \begin{phonetics}{渔船队}{yu2chuan2dui4}
    \definition{s.}{frota pesqueira}
  \end{phonetics}
\end{entry}

\begin{entry}{渡}{12}{⽔}
  \begin{phonetics}{渡}{du4}[][HSK 6]
    \definition{s.}{(usualmente em nomes de lugares) travessia de balsa}
    \definition{v.}{atravessar (um rio, o mar, etc.) | superar; sobreviver | transportar (pessoas, mercadorias, etc.) através}
  \end{phonetics}
\end{entry}

\begin{entry}{渡过}{12,6}{⽔、⾡}
  \begin{phonetics}{渡过}{du4guo4}
    \definition{v.}{atravessar | passar por}
  \end{phonetics}
\end{entry}

\begin{entry}{温}{12}{⽔}
  \begin{phonetics}{温}{wen1}
    \definition{adj.}{morno; quente; suave}
    \definition{s.}{temperatura | doenças transmissíveis agudas; praga}
    \definition{v.}{aquecer; reaquecer; aquecer ligeiramente | revisar; repassar}
  \end{phonetics}
\end{entry}

\begin{entry}{温和}{12,8}{⽔、⼝}
  \begin{phonetics}{温和}{wen1he2}[][HSK 5]
    \definition{adj.}{gentil; suave; moderado}
  \end{phonetics}
\end{entry}

\begin{entry}{温度}{12,9}{⽔、⼴}
  \begin{phonetics}{温度}{wen1du4}[][HSK 2]
    \definition[度,级,档,个]{s.}{temperatura}
  \end{phonetics}
\end{entry}

\begin{entry}{温度计}{12,9,4}{⽔、⼴、⾔}
  \begin{phonetics}{温度计}{wen1du4ji4}
    \definition{s.}{termógrafo | termômetro}
  \end{phonetics}
\end{entry}

\begin{entry}{温度表}{12,9,8}{⽔、⼴、⾐}
  \begin{phonetics}{温度表}{wen1du4biao3}
    \definition{s.}{termômetro}
  \end{phonetics}
\end{entry}

\begin{entry}{温度梯度}{12,9,11,9}{⽔、⼴、⽊、⼴}
  \begin{phonetics}{温度梯度}{wen1du4ti1du4}
    \definition{s.}{gradiente de temperatura}
  \end{phonetics}
\end{entry}

\begin{entry}{温柔}{12,9}{⽔、⽊}
  \begin{phonetics}{温柔}{wen1rou2}
    \definition{adj.}{gentil e suave | terno | doce (comumente usado para descrever uma menina ou mulher)}
  \end{phonetics}
\end{entry}

\begin{entry}{温暖}{12,13}{⽔、⽇}
  \begin{phonetics}{温暖}{wen1nuan3}[][HSK 3]
    \definition{adj.}{caloroso; gentil; amigável | caloroso; quente}
    \definition{v.}{aquecer; fazer com que se sinta calor}
  \end{phonetics}
\end{entry}

\begin{entry}{港}{12}{⽔}
  \begin{phonetics}{港}{gang3}
    \definition*{s.}{Sobrenome Gang}
    \definition{s.}{Hong Kong, abreviação de 香港 | porto}
  \seealsoref{香港}{xiang1gang3}
  \end{phonetics}
\end{entry}

\begin{entry}{港口}{12,3}{⽔、⼝}
  \begin{phonetics}{港口}{gang3kou3}[][HSK 6]
    \definition[个,座]{s.}{porto; locais com certas condições naturais e instalações portuárias para atracação de navios, embarque e desembarque de passageiros e coleta e distribuição de cargas}
  \end{phonetics}
\end{entry}

\begin{entry}{渴}{12}{⽔}
  \begin{phonetics}{渴}{ke3}[][HSK 1]
    \definition{adj.}{sedento}
    \definition{adv.}{ansiosamente; metáfora de urgência}
    \definition{v.}{desejar; ansiar por}
  \end{phonetics}
\end{entry}

\begin{entry}{渴望}{12,11}{⽔、⽉}
  \begin{phonetics}{渴望}{ke3wang4}[][HSK 5]
    \definition{v.}{aspirar; (ter sede, ansiar, desejar) por}
  \end{phonetics}
\end{entry}

\begin{entry}{游}{12}{⽔}
  \begin{phonetics}{游}{you2}[][HSK 3]
    \definition*{s.}{Sobrenome You}
    \definition{adj.}{itinerante; não fixo; que se move frequentemente}
    \definition{s.}{parte de um rio; uma seção do rio; trecho; bacia; curso}
    \definition{v.}{nadar | vagar por aí; caminhar; viajar; fazer turismo | associar com (comunicação) | vagar; passear; andar tranquilamente por todos os lugares}
  \end{phonetics}
\end{entry}

\begin{entry}{游戏}{12,6}{⽔、⼽}
  \begin{phonetics}{游戏}{you2xi4}[][HSK 3]
    \definition[场]{s.}{jogo; recreação; atividades recreativas, como esconde-esconde, adivinhar charadas, etc.; certas atividades esportivas não competitivas; jogos recreativos}
    \definition{v.}{jogar; fazer atividades divertidas e agradáveis, sozinho ou com outras pessoas}
  \end{phonetics}
\end{entry}

\begin{entry}{游泳}{12,8}{⽔、⽔}
  \begin{phonetics}{游泳}{you2yong3}[][HSK 3]
    \definition[次]{s.}{natação; refere-se ao esporte ou atividade de natação}
    \definition{v.+compl.}{nadar; pessoas ou animais nadando na água}
  \end{phonetics}
\end{entry}

\begin{entry}{游泳池}{12,8,6}{⽔、⽔、⽔}
  \begin{phonetics}{游泳池}{you2 yong3 chi2}[][HSK 5]
    \definition[场,个]{s.}{piscina; piscinas artificiais para natação, divididas em duas categorias: internas e externas}
  \seealsoref{泳池}{yong3chi2}
  \seealsoref{游泳馆}{you2yong3guan3}
  \end{phonetics}
\end{entry}

\begin{entry}{游泳衣}{12,8,6}{⽔、⽔、⾐}
  \begin{phonetics}{游泳衣}{you2yong3yi1}
    \definition{s.}{roupa de banho}
  \seealsoref{泳衣}{yong3yi1}
  \end{phonetics}
\end{entry}

\begin{entry}{游泳馆}{12,8,11}{⽔、⽔、⾷}
  \begin{phonetics}{游泳馆}{you2yong3guan3}
    \definition[场]{s.}{piscina}
  \seealsoref{泳池}{yong3chi2}
  \seealsoref{游泳池}{you2 yong3 chi2}
  \end{phonetics}
\end{entry}

\begin{entry}{游泳镜}{12,8,16}{⽔、⽔、⾦}
  \begin{phonetics}{游泳镜}{you2yong3jing4}
    \definition{s.}{óculos de natação}
  \end{phonetics}
\end{entry}

\begin{entry}{游客}{12,9}{⽔、⼧}
  \begin{phonetics}{游客}{you2 ke4}[][HSK 2]
    \definition[个,位,名,群]{s.}{visitante; turista | (jogo online) jogador convidado}
  \end{phonetics}
\end{entry}

\begin{entry}{游艇}{12,12}{⽔、⾈}
  \begin{phonetics}{游艇}{you2ting3}
    \definition[只]{s.}{barcaça | iate}
  \end{phonetics}
\end{entry}

\begin{entry}{湖}{12}{⽔}
  \begin{phonetics}{湖}{hu2}[][HSK 2]
    \definition*{s.}{Huzhou, abreviação de 湖州 | Um nome que se refere às províncias de Hunan, 湖南,  e Hubei, 湖北}
    \definition[个,片]{s.}{lago}
  \seealsoref{湖北}{hu2bei3}
  \seealsoref{湖南}{hu2nan2}
  \seealsoref{湖州}{hu2zhou1}
  \end{phonetics}
\end{entry}

\begin{entry}{湖北}{12,5}{⽔、⼔}
  \begin{phonetics}{湖北}{hu2bei3}
    \definition*{s.}{Província de Hubei (Hupeh), na China central}
  \end{phonetics}
\end{entry}

\begin{entry}{湖州}{12,6}{⽔、⼮}
  \begin{phonetics}{湖州}{hu2zhou1}
    \definition*{s.}{Cidade de Huzhou, em Zhejiang}
  \end{phonetics}
\end{entry}

\begin{entry}{湖南}{12,9}{⽔、⼗}
  \begin{phonetics}{湖南}{hu2nan2}
    \definition*{s.}{Província de Hunan}
  \end{phonetics}
\end{entry}

\begin{entry}{湿}{12}{⽔}
  \begin{phonetics}{湿}{shi1}[][HSK 4]
    \definition{adj.}{molhado; úmido; algo com água ou com muita água dentro}
  \end{phonetics}
\end{entry}

\begin{entry}{滑}{12}{⽔}
  \begin{phonetics}{滑}{hua2}[][HSK 5]
    \definition*{s.}{Sobrenome Hua}
    \definition{adj.}{escorregadio; liso; objetos com superfícies lisas e baixo atrito | astuto; ardiloso; escorregadio}
    \definition{v.}{escorregar; deslizar | se atrapalhar; se safar de algo}
  \end{phonetics}
\end{entry}

\begin{entry}{滑雪}{12,11}{⽔、⾬}
  \begin{phonetics}{滑雪}{hua2xue3}
    \definition{v.+compl.}{esquiar | praticar esqui}
  \end{phonetics}
\end{entry}

\begin{entry}{源}{13}{⽔}
  \begin{phonetics}{源}{yuan2}
    \definition*{s.}{Sobrenome Yuan}
    \definition{s.}{nascente (de um rio); fonte | fonte; origem; causa}
    \definition{v.}{originar-se; provir de}
  \end{phonetics}
\end{entry}

\begin{entry}{滔}{13}{⽔}
  \begin{phonetics}{滔}{tao1}
    \definition{adj.}{(de água) transbordando | arrogante | turbulento | largo e longo; grande}
    \definition{v.}{inundar; alagar}
  \end{phonetics}
\end{entry}

\begin{entry}{滔天}{13,4}{⽔、⼤}
  \begin{phonetics}{滔天}{tao1tian1}
    \definition{adj.}{(ondas, raiva, desastres, crimes, etc.) imponente, avassalador, imenso}
  \end{phonetics}
\end{entry}

\begin{entry}{滚}{13}{⽔}
  \begin{phonetics}{滚}{gun3}[][HSK 5]
    \definition*{s.}{Sobrenome Gun}
    \definition{adj.}{rolante | fervente | precipitado; torrencial}
    \definition{adv.}{muito; em um grau elevado}
    \definition{v.}{rolar; girar; virar | escapar; fugir; ir embora | ferver | amarrar; aparar; fazer bainha}
  \end{phonetics}
\end{entry}

\begin{entry}{滚轮}{13,8}{⽔、⾞}
  \begin{phonetics}{滚轮}{gun3lun2}
    \definition{s.}{pneu | dial rotativo | roda de rolagem (\emph{scroll})  (mouse de computador)}
  \end{phonetics}
\end{entry}

\begin{entry}{滚滚}{13,13}{⽔、⽔}
  \begin{phonetics}{滚滚}{gun3gun3}
    \definition*{s.}{Apelido para um panda}
    \definition{v.}{mover-se | rolar | fluir continuamente}
  \end{phonetics}
\end{entry}

\begin{entry}{满}{13}{⽔}
  \begin{phonetics}{满}{man3}[][HSK 2]
    \definition*{s.}{Etnia Manchu | Sobrenome Man}
    \definition{adj.}{cheio; repleto; lotado; totalmente cheio; atingindo o limite da capacidade | tudo; inteiro; completo | presunçoso; complacente; orgulhoso}
    \definition{adv.}{muito; um tanto; bastante | completamente; inteiramente; perfeitamente}
    \definition{v.}{encher | sentir-se satisfeito; sentir que já é o suficiente | expirar; atingir o limite; atingir um determinado prazo ou limite}
  \end{phonetics}
\end{entry}

\begin{entry}{满分}{13,4}{⽔、⼑}
  \begin{phonetics}{满分}{man3fen1}
    \definition{s.}{pontuação completa}
  \end{phonetics}
\end{entry}

\begin{entry}{满足}{13,7}{⽔、⾜}
  \begin{phonetics}{满足}{man3zu2}[][HSK 3]
    \definition{v.}{estar satisfeito; contentar-se; sentir-se satisfeito | satisfazer}
  \end{phonetics}
\end{entry}

\begin{entry}{满意}{13,13}{⽔、⼼}
  \begin{phonetics}{满意}{man3yi4}[][HSK 2]
    \definition{adj.}{satisfeito; contente; gratificado}
    \definition{v.}{estar satisfeito; sentir-se contente; satisfazer os seus desejos; estar de acordo com os seus desejos}
  \end{phonetics}
\end{entry}

\begin{entry}{满满}{13,13}{⽔、⽔}
  \begin{phonetics}{满满}{man3man3}
    \definition{adj.}{completo | densamente empacotado}
  \end{phonetics}
\end{entry}

\begin{entry}{滴}{14}{⽔}
  \begin{phonetics}{滴}{di1}[][HSK 6]
    \definition{clas.}{gota; quantificador para "gotejamento"}
    \definition{s.}{uma gota}
    \definition{v.}{pingar}
  \end{phonetics}
\end{entry}

\begin{entry}{漂}{14}{⽔}
  \begin{phonetics}{漂}{piao1}
    \definition{v.}{flutuar | estar a deriva}
  \end{phonetics}
  \begin{phonetics}{漂}{piao3}
    \definition{v.}{alvejar | branquear}
  \end{phonetics}
  \begin{phonetics}{漂}{piao4}
    \definition{adj.}{usado em 漂亮}
  \seealsoref{漂亮}{piao4liang5}
  \end{phonetics}
\end{entry}

\begin{entry}{漂亮}{14,9}{⽔、⼇}
  \begin{phonetics}{漂亮}{piao4liang5}[][HSK 2]
    \definition{adj.}{bonito; lindo; atraente; de boa aparência; esteticamente agradável | excelente; notável | não pode ser utilizado para descrever homens}
  \end{phonetics}
\end{entry}

\begin{entry}{漂流}{14,10}{⽔、⽔}
  \begin{phonetics}{漂流}{piao1liu2}
    \definition{s.}{\emph{rafting}}
    \definition{v.}{ser levado pela correnteza | flutuar ao longo ou sobre}
  \end{phonetics}
\end{entry}

\begin{entry}{漏}{14}{⽔}
  \begin{phonetics}{漏}{lou4}[][HSK 5]
    \definition{s.}{relógio de água; ampulheta | falha; ponto fraco | gonorreia; a medicina tradicional chinesa refere-se a certas doenças que causam secreção de pus, sangue e muco | unidade de tempo medida por um relógio de água durante a noite}
    \definition{v.}{(líquido, gás, etc.) pingar; vazar; escorrer; cair (de um buraco ou fenda) | vazar; deixar escapar; divulgar | perder; deixar de fora por engano | vazar; o objeto tem poros e pode vazar coisas | há uma fuga de ar}
  \end{phonetics}
\end{entry}

\begin{entry}{漏电}{14,5}{⽔、⽥}
  \begin{phonetics}{漏电}{lou4dian4}
    \definition{v.}{vazar eletricidade}
  \end{phonetics}
\end{entry}

\begin{entry}{漏洞}{14,9}{⽔、⽔}
  \begin{phonetics}{漏洞}{lou4 dong4}[][HSK 5]
    \definition[个]{s.}{vazamento; rachadura; lacunas ou buracos desnecessários que permitem que coisas vazem | falha; defeito; lacuna; (fala, ação, método, etc.) imperfeições}
  \end{phonetics}
\end{entry}

\begin{entry}{演}{14}{⽔}
  \begin{phonetics}{演}{yan3}[][HSK 3]
    \definition{v.}{desenvolver; evoluir | deduzir; elaborar | exercitar; praticar | representar; atuar; encenar | desempenhar}
  \end{phonetics}
\end{entry}

\begin{entry}{演出}{14,5}{⽔、⼐}
  \begin{phonetics}{演出}{yan3chu1}[][HSK 3]
    \definition[场,次]{s.}{show; concerto; performance}
    \definition{v.}{apresentar; representar; fazer um show; apresentar peças teatrais, danças, artes cênicas, acrobacias, etc. para o público apreciar}
  \end{phonetics}
\end{entry}

\begin{entry}{演讲}{14,6}{⽔、⾔}
  \begin{phonetics}{演讲}{yan3jiang3}[][HSK 4]
    \definition[场,次]{s.}{palestra; discurso; ato ou a atividade de apresentar ou expressar ideias, opiniões ou informações oralmente em público ou diante de um público}
    \definition{v.}{dar uma palestra; fazer um discurso; informar o público sobre uma determinada área de conhecimento ou opinião sobre um determinado assunto}
  \end{phonetics}
\end{entry}

\begin{entry}{演员}{14,7}{⽔、⼝}
  \begin{phonetics}{演员}{yan3yuan2}[][HSK 3]
    \definition[个,位,名]{s.}{ator; artista; pessoas que participam de apresentações teatrais, cinematográficas, de dança, de artes cênicas, de acrobacias, etc.}
  \end{phonetics}
\end{entry}

\begin{entry}{演唱}{14,11}{⽔、⼝}
  \begin{phonetics}{演唱}{yan3 chang4}[][HSK 3]
    \definition{v.}{cantar em uma performance; apresentar canções, óperas, peças teatrais, etc.}
  \end{phonetics}
\end{entry}

\begin{entry}{演唱会}{14,11,6}{⽔、⼝、⼈}
  \begin{phonetics}{演唱会}{yan3 chang4 hui4}[][HSK 3]
    \definition[个,场]{s.}{recital vocal; concerto vocal; uma forma de apresentação centrada no canto, acompanhada por movimentos de dança simples}
  \end{phonetics}
\end{entry}

\begin{entry}{漫}{14}{⽔}
  \begin{phonetics}{漫}{man4}
    \definition{adj.}{livre; irrestrito; casual | longo; extenso | em todos os lugares; por toda parte | aleatório; irrestrito; livre}
    \definition{adv.}{não}
    \definition{v.}{transbordar; inundar | estar em todo lugar}
  \end{phonetics}
\end{entry}

\begin{entry}{漫长}{14,4}{⽔、⾧}
  \begin{phonetics}{漫长}{man4chang2}[][HSK 5]
    \definition{adj.}{muito longo; interminável; (tempo, espaço) dura muito tempo}
  \end{phonetics}
\end{entry}

\begin{entry}{漫画}{14,8}{⽔、⽥}
  \begin{phonetics}{漫画}{man4hua4}[][HSK 5]
    \definition[幅,本,张,套]{s.}{desenho animado; caricatura}
  \end{phonetics}
\end{entry}

\begin{entry}{漫骂}{14,9}{⽔、⾺}
  \begin{phonetics}{漫骂}{man4ma4}
    \variantof{谩骂}
  \end{phonetics}
\end{entry}

\begin{entry}{潜}{15}{⽔}
  \begin{phonetics}{潜}{qian2}
    \definition*{s.}{Sobrenome Qian}
    \definition{adj.}{latente; oculto}
    \definition{adv.}{furtivamente; secretamente; às escondidas}
    \definition{v.}{ir para debaixo d'água; esconder-se debaixo d'água; mergulhar | esconder | vadear (atravessar) na água | enterrar | fugir de casa}
  \end{phonetics}
\end{entry}

\begin{entry}{潜力}{15,2}{⽔、⼒}
  \begin{phonetics}{潜力}{qian2li4}[][HSK 6]
    \definition{s.}{potencial; potencialidade; capacidade latente; as habilidades e possibilidades de desenvolvimento que as pessoas e as coisas ainda não demonstraram}
  \end{phonetics}
\end{entry}

\begin{entry}{潜在}{15,6}{⽔、⼟}
  \begin{phonetics}{潜在}{qian2zai4}
    \definition{adj.}{oculto | latente}
    \definition{s.}{potencial}
  \end{phonetics}
\end{entry}

\begin{entry}{潮}{15}{⽔}
  \begin{phonetics}{潮}{chao2}[][HSK 4]
    \definition{adj.}{úmido; molhado | inferior; de qualidade ruim | inferior; não muito habilidoso}
    \definition{s.}{maré; água da maré | surto; corrente; maré; uma metáfora para mudanças sociais em grande escala ou para os altos e baixos de um movimento (social)}
    \definition{s.}{Chaozhou, uma cidade na província de Guangdong}
  \end{phonetics}
\end{entry}

\begin{entry}{潮流}{15,10}{⽔、⽔}
  \begin{phonetics}{潮流}{chao2liu2}[][HSK 4]
    \definition{s.}{maré; corrente de maré; movimento da água devido às marés | tendência; analogia com mudanças sociais ou tendências de desenvolvimento}
  \end{phonetics}
\end{entry}

\begin{entry}{潮绣}{15,10}{⽔、⽷}
  \begin{phonetics}{潮绣}{chao2xiu4}
    \definition*{s.}{Bordado Chaozhou}
  \end{phonetics}
\end{entry}

\begin{entry}{潮湿}{15,12}{⽔、⽔}
  \begin{phonetics}{潮湿}{chao2shi1}[][HSK 4]
    \definition{adj.}{molhado; úmido; umedecido; que contém mais água do que o normal}
  \end{phonetics}
\end{entry}

\begin{entry}{澳}{15}{⽔}
  \begin{phonetics}{澳}{ao4}
    \definition*{s.}{Abreviação de Austrália, 澳大利亚 | Sobrenome Ao}
    \definition{s.}{baía; uma entrada do mar; um lugar curvo na costa onde os barcos podem ser atracados, frequentemente usado em nomes de lugares}
  \seealsoref{澳大利亚}{ao4da4li4ya4}
  \end{phonetics}
\end{entry}

\begin{entry}{澳大利亚}{15,3,7,6}{⽔、⼤、⼑、⼆}
  \begin{phonetics}{澳大利亚}{ao4da4li4ya4}
    \definition*{s.}{Austrália}
  \end{phonetics}
\end{entry}

\begin{entry}{激}{16}{⽔}
  \begin{phonetics}{激}{ji1}
    \definition*{s.}{Sobrenome Ji}
    \definition{adj.}{afiado; feroz; violento | vívido}
    \definition{adv.}{bruscamente; ferozmente; violentamente}
    \definition{s.}{o impacto de ondas fortes contra a costa}
    \definition{v.}{bater; avançar; correr | despertar; estimular; incitar; excitar | ficar doente por se molhar | esfriar (colocando água gelada, etc.)}
  \end{phonetics}
\end{entry}

\begin{entry}{激动}{16,6}{⽔、⼒}
  \begin{phonetics}{激动}{ji1dong4}[][HSK 4]
    \definition{adj.}{animado; entusiasmado; empolgado}
    \definition{v.}{agitar; excitar; tornar fortes os sentimentos de alguém}
  \end{phonetics}
\end{entry}

\begin{entry}{激烈}{16,10}{⽔、⽕}
  \begin{phonetics}{激烈}{ji1lie4}[][HSK 4]
    \definition{adj.}{agudo; afiado; feroz; violento; intenso}
  \end{phonetics}
\end{entry}

\begin{entry}{激情}{16,11}{⽔、⼼}
  \begin{phonetics}{激情}{ji1qing2}[][HSK 6]
    \definition{s.}{paixão; emoções fortes e explosivas, como êxtase, raiva, etc.}
  \end{phonetics}
\end{entry}

\begin{entry}{瀑}{18}{⽔}
  \begin{phonetics}{瀑}{bao4}
    \definition{s.}{chuva torrencial; tempestade}
  \end{phonetics}
  \begin{phonetics}{瀑}{pu4}
    \definition{s.}{cachoeira; catarata}
  \end{phonetics}
\end{entry}

\begin{entry}{瀑布}{18,5}{⽔、⼱}
  \begin{phonetics}{瀑布}{pu4bu4}
    \definition{s.}{queda de água | cachoeira | cascata | catarata}
  \end{phonetics}
\end{entry}

%%%%% EOF %%%%%

