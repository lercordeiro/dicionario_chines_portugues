%%%
%%% Radical "⽔"
%%%

\section*{Radical 85: ``⽔'' (氵、氺)}\addcontentsline{toc}{section}{Radical 85: ⽔、氵、氺}

\begin{entry}{水}{4}{⽔}[Kangxi 85]
  \begin{phonetics}{水}{shui3}[][HSK 1]
    \definition*{s.}{sobrenome Shui}
    \definition{clas.}{para número de lavagens}
    \definition{s.}{água | líquido | encargos ou receitas adicionais}
  \end{phonetics}
\end{entry}

\begin{entry}{水平}{4,5}{⽔、⼲}
  \begin{phonetics}{水平}{shui3ping2}[][HSK 2]
    \definition{s.}{nível (de realização, etc.) | padrão | nível horizontal}
  \end{phonetics}
\end{entry}

\begin{entry}{水平以下}{4,5,4,3}{⽔、⼲、⼈、⼀}
  \begin{phonetics}{水平以下}{shui3ping2 yi3xia4}
    \definition{s.}{sub-nível}
  \end{phonetics}
\end{entry}

\begin{entry}{水平尺}{4,5,4}{⽔、⼲、⼫}
  \begin{phonetics}{水平尺}{shui3ping2chi3}
    \definition{s.}{nível espiritual}
  \end{phonetics}
\end{entry}

\begin{entry}{水平仪}{4,5,5}{⽔、⼲、⼈}
  \begin{phonetics}{水平仪}{shui3ping2yi2}
    \definition{s.}{nível (dispositivo para determinar horizontal) | nível espiritual | nível de topógrafo}
  \end{phonetics}
\end{entry}

\begin{entry}{水平视差}{4,5,8,9}{⽔、⼲、⾒、⼯}
  \begin{phonetics}{水平视差}{shui3ping2 shi4cha1}
    \definition{s.}{paralaxe horizontal}
  \end{phonetics}
\end{entry}

\begin{entry}{水平度}{4,5,9}{⽔、⼲、⼴}
  \begin{phonetics}{水平度}{shui3ping2 du4}
    \definition{s.}{nivelamento}
  \end{phonetics}
\end{entry}

\begin{entry}{水平轴}{4,5,9}{⽔、⼲、⾞}
  \begin{phonetics}{水平轴}{shui3ping2zhou2}
    \definition{s.}{eixo horizontal}
  \end{phonetics}
\end{entry}

\begin{entry}{水平面}{4,5,9}{⽔、⼲、⾯}
  \begin{phonetics}{水平面}{shui3ping2mian4}
    \definition{s.}{plano horizontal | nível-da-água | superfície horizontal}
  \end{phonetics}
\end{entry}

\begin{entry}{水边}{4,5}{⽔、⾡}
  \begin{phonetics}{水边}{shui3bian1}
    \definition{s.}{beira d'água | beira-mar | costa (de mar, lago ou rio)}
  \end{phonetics}
\end{entry}

\begin{entry}{水污染}{4,6,9}{⽔、⽔、⽊}
  \begin{phonetics}{水污染}{shui3wu1ran3}
    \definition{s.}{poluição da água}
  \end{phonetics}
\end{entry}

\begin{entry}{水灵}{4,7}{⽔、⽕}
  \begin{phonetics}{水灵}{shui3ling2}
    \definition{adj.}{cheio de vida (sobre uma pessoa, etc.) | úmido e brilhante (sobre os olhos) | fresco (sobre frutas, etc.) | brilhante | aparência saudável}
  \end{phonetics}
\end{entry}

\begin{entry}{水果}{4,8}{⽔、⽊}
  \begin{phonetics}{水果}{shui3guo3}[][HSK 1]
    \definition[个]{s.}{fruta}
  \end{phonetics}
\end{entry}

\begin{entry}{水波}{4,8}{⽔、⽔}
  \begin{phonetics}{水波}{shui3bo1}
    \definition{s.}{ondulação (na água) | onda}
  \end{phonetics}
\end{entry}

\begin{entry}{水饺}{4,9}{⽔、⾷}
  \begin{phonetics}{水饺}{shui3jiao3}
    \definition{s.}{\emph{dumplings} | pastéis chineses cozidos}
  \end{phonetics}
\end{entry}

\begin{entry}{水瓶}{4,10}{⽔、⽡}
  \begin{phonetics}{水瓶}{shui3 ping2}
    \definition{s.}{garrada de água}
  \end{phonetics}
\end{entry}

\begin{entry}{水培}{4,11}{⽔、⼟}
  \begin{phonetics}{水培}{shui3pei2}
    \definition{v.}{cultivar plantas hidroponicamente}
  \end{phonetics}
\end{entry}

\begin{entry}{水豚}{4,11}{⽔、⾗}
  \begin{phonetics}{水豚}{shui3tun2}
    \definition{s.}{capivara}
  \end{phonetics}
\end{entry}

\begin{entry}{水路}{4,13}{⽔、⾜}
  \begin{phonetics}{水路}{shui3lu4}
    \definition{s.}{hidrovia}
  \end{phonetics}
\end{entry}

\begin{entry}{水槽}{4,15}{⽔、⽊}
  \begin{phonetics}{水槽}{shui3cao2}
    \definition{s.}{pia (de cozinha)}
  \end{phonetics}
\end{entry}

\begin{entry}{永不}{5,4}{⽔、⼀}
  \begin{phonetics}{永不}{yong3bu4}
    \definition{adv.}{nunca}
  \end{phonetics}
\end{entry}

\begin{entry}{永远}{5,7}{⽔、⾡}
  \begin{phonetics}{永远}{yong3yuan3}[][HSK 2]
    \definition{adv.}{para sempre, sempre | permanentemente}
  \end{phonetics}
\end{entry}

\begin{entry}{汇}{5}{⽔}
  \begin{phonetics}{汇}{hui4}[][HSK 4]
    \definition{s.}{coisas coletadas; conjunto; coleção}
    \definition{v.}{convergir | reunir-se | remeter; transferir por meio de agências postais e telegráficas, bancos}
  \end{phonetics}
\end{entry}

\begin{entry}{汇报}{5,7}{⽔、⼿}
  \begin{phonetics}{汇报}{hui4bao4}[][HSK 4]
    \definition[份,次]{s.}{relatório; referindo-se ao conteúdo de declarações escritas ou orais feitas a um superior ou pessoa relevante para apresentar uma situação ou refletir um problema}
    \definition{v.}{relatar; fazer um relato de}
  \end{phonetics}
\end{entry}

\begin{entry}{汇率}{5,11}{⽔、⽞}
  \begin{phonetics}{汇率}{hui4lv4}[][HSK 4]
    \definition[个]{s.}{taxa de câmbio; relação entre a moeda de um país e a de outro}
  \end{phonetics}
\end{entry}

\begin{entry}{汉}{5}{⽔}
  \begin{phonetics}{汉}{han4}
    \definition{s.}{grupo étnico Han | chinês (língua) | dinastia Han (206 a.C.-220d.C.) | homem}
  \end{phonetics}
\end{entry}

\begin{entry}{汉字}{5,6}{⽔、⼦}
  \begin{phonetics}{汉字}{han4 zi4}[][HSK 1]
    \definition[个]{s.}{caracter chinês}
  \end{phonetics}
\end{entry}

\begin{entry}{汉服}{5,8}{⽔、⽉}
  \begin{phonetics}{汉服}{han4fu2}
    \definition{s.}{vestido chinês tradicional Han}
  \end{phonetics}
\end{entry}

\begin{entry}{汉语}{5,9}{⽔、⾔}
  \begin{phonetics}{汉语}{han4yu3}[][HSK 1]
    \definition[门]{s.}{língua chinesa, mandarim}
  \end{phonetics}
\end{entry}

\begin{entry}{汉堡王}{5,12,4}{⽔、⼟、⽟}
  \begin{phonetics}{汉堡王}{han4bao3wang2}
    \definition*{s.}{Burguer King (restaurante \emph{fast-food})}
  \end{phonetics}
\end{entry}

\begin{entry}{汉堡包}{5,12,5}{⽔、⼟、⼓}
  \begin{phonetics}{汉堡包}{han4bao3bao1}
    \definition[个]{s.}{hambúrguer}
  \end{phonetics}
\end{entry}

\begin{entry}{汉葡词典}{5,12,7,8}{⽔、⾋、⾔、⼋}
  \begin{phonetics}{汉葡词典}{han4-pu2 ci2dian3}
    \definition[部,本]{s.}{dicionário chinês-português}
  \seealsoref{葡汉词典}{pu2-han4 ci2dian3}
  \end{phonetics}
\end{entry}

\begin{entry}{汗水}{6,4}{⽔、⽔}
  \begin{phonetics}{汗水}{han4shui3}
    \definition*{s.}{Rio Han (Hanshui)}
    \definition{s.}{suor | transpiração}
  \end{phonetics}
\end{entry}

\begin{entry}{汗液}{6,11}{⽔、⽔}
  \begin{phonetics}{汗液}{han4ye4}
    \definition{s.}{suor}
  \end{phonetics}
\end{entry}

\begin{entry}{汗腺}{6,13}{⽔、⾁}
  \begin{phonetics}{汗腺}{han4xian4}
    \definition{s.}{glândula sudorípara}
  \end{phonetics}
\end{entry}

\begin{entry}{江}{6}{⽔}
  \begin{phonetics}{江}{jiang1}[][HSK 4]
    \definition*{s.}{Rio Changjiang | sobrenome Jiang}
    \definition[条,道]{s.}{rio grande}
  \end{phonetics}
\end{entry}

\begin{entry}{江水}{6,4}{⽔、⽔}
  \begin{phonetics}{江水}{jiang1shui3}
    \definition{s.}{água do rio}
  \end{phonetics}
\end{entry}

\begin{entry}{江西}{6,6}{⽔、⾑}
  \begin{phonetics}{江西}{jiang1xi1}
    \definition*{s.}{Jiangxi}
  \end{phonetics}
\end{entry}

\begin{entry}{江南水乡}{6,9,4,3}{⽔、⼗、⽔、⼄}
  \begin{phonetics}{江南水乡}{jiang1nan2shui3xiang1}
    \definition*{s.}{Vila Aquática de Jiangnan | Cidades Aquáticas}
  \end{phonetics}
\end{entry}

\begin{entry}{池}{6}{⽔}
  \begin{phonetics}{池}{chi2}
    \definition*{s.}{sobrenome Chi}
    \definition{s.}{lagoa | reservatório | fosso}
  \end{phonetics}
\end{entry}

\begin{entry}{污水}{6,4}{⽔、⽔}
  \begin{phonetics}{污水}{wu1shui3}
    \definition{s.}{esgoto}
  \end{phonetics}
\end{entry}

\begin{entry}{污染}{6,9}{⽔、⽊}
  \begin{phonetics}{污染}{wu1ran3}
    \definition{s.}{poluição}
    \definition{v.}{poluir}
  \end{phonetics}
\end{entry}

\begin{entry}{污染区}{6,9,4}{⽔、⽊、⼖}
  \begin{phonetics}{污染区}{wu1ran3qu1}
    \definition{s.}{área contaminada}
  \end{phonetics}
\end{entry}

\begin{entry}{污染物}{6,9,8}{⽔、⽊、⽜}
  \begin{phonetics}{污染物}{wu1ran3wu4}
    \definition{s.}{poluente}
  \seealsoref{污染物质}{wu1ran3 wu4zhi4}
  \end{phonetics}
\end{entry}

\begin{entry}{污染物质}{6,9,8,8}{⽔、⽊、⽜、⾙}
  \begin{phonetics}{污染物质}{wu1ran3 wu4zhi4}
    \definition{s.}{poluente}
  \seealsoref{污染物}{wu1ran3wu4}
  \end{phonetics}
\end{entry}

\begin{entry}{汤}{6}{⽔}
  \begin{phonetics}{汤}{shang1}
    \definition{s.}{correnteza forte}
  \end{phonetics}
  \begin{phonetics}{汤}{tang1}[][HSK 3]
    \definition*{s.}{sobrenome Tang}
    \definition{s.}{água quente; água fervente
fontes termais
sopa; caldo
uma preparação líquida de ervas medicinais; decocção}
  \end{phonetics}
\end{entry}

\begin{entry}{求}{7}{⽔}
  \begin{phonetics}{求}{qiu2}[][HSK 2]
    \definition*{s.}{sobrenome Qiu}
    \definition{s.}{demanda}
    \definition{v.}{pedir | implorar | solicitar | suplicar | esforçar-se por | procurar | tentar}
  \end{phonetics}
\end{entry}

\begin{entry}{汹涌}{7,10}{⽔、⽔}
  \begin{phonetics}{汹涌}{xiong1yong3}
    \definition{adj.}{turbulento}
    \definition{v.}{aumentar ou emergir violentamente (oceano, rio, lago, etc.)}
  \end{phonetics}
\end{entry}

\begin{entry}{汽水}{7,4}{⽔、⽔}
  \begin{phonetics}{汽水}{qi4 shui3}[][HSK 4]
    \definition[罐,瓶]{s.}{refrigerante; refrigerante gaseificado; bebida refrescante, feita com a pressão de dióxido de carbono para dissolver na água e adicionar açúcar, suco de frutas, especiarias etc.}
  \end{phonetics}
\end{entry}

\begin{entry}{汽车}{7,4}{⽔、⾞}
  \begin{phonetics}{汽车}{qi4che1}[][HSK 1]
    \definition[辆]{s.}{automóvel | carro | veículo motorizado}
  \end{phonetics}
\end{entry}

\begin{entry}{汽油}{7,8}{⽔、⽔}
  \begin{phonetics}{汽油}{qi4you2}[][HSK 4]
    \definition{s.}{gasolina; mistura líquida de hidrocarbonetos com volatilidade e combustibilidade, que é usada como combustível a partir do fracionamento ou craqueamento do petróleo}
  \end{phonetics}
\end{entry}

\begin{entry}{沉}{7}{⽔}
  \begin{phonetics}{沉}{chen2}[][HSK 4]
    \definition{adj.}{profundo | pesado | pesado (sentir-se pesado)}
    \definition{v.}{afundar; submergir; imergir | manter baixo; abaixar | descansar; parar}
  \end{phonetics}
\end{entry}

\begin{entry}{沉重}{7,9}{⽔、⾥}
  \begin{phonetics}{沉重}{chen2zhong4}[][HSK 4]
    \definition{adj.}{(pressão, fardo, etc.) muito pesado; profundo | sério; pesado; humor pouco animador; fardo pesado de pensamentos}
  \end{phonetics}
\end{entry}

\begin{entry}{沉默}{7,16}{⽔、⿊}
  \begin{phonetics}{沉默}{chen2mo4}[][HSK 4]
    \definition{adj.}{silencioso; reticente; taciturno; não comunicativo}
    \definition{v.}{silenciar; não falar por causa de alguma coisa}
  \end{phonetics}
\end{entry}

\begin{entry}{沙}{7}{⽔}
  \begin{phonetics}{沙}{sha1}
    \definition*{s.}{sobrenome Sha}
    \definition[粒]{s.}{areia | cascalho | grânulo | pó}
  \end{phonetics}
\end{entry}

\begin{entry}{沙子}{7,3}{⽔、⼦}
  \begin{phonetics}{沙子}{sha1 zi5}[][HSK 3]
    \definition[粒,把]{s.}{areia; grão | \emph{pellets}; grãos pequenos}
  \end{phonetics}
\end{entry}

\begin{entry}{沙发}{7,5}{⽔、⼜}
  \begin{phonetics}{沙发}{sha1fa1}[][HSK 3]
    \definition[套,组,个,张]{s.}{sofá; divã}
  \end{phonetics}
\end{entry}

\begin{entry}{沙鱼}{7,8}{⽔、⿂}
  \begin{phonetics}{沙鱼}{sha1yu2}
    \variantof{鲨鱼}
  \end{phonetics}
\end{entry}

\begin{entry}{沙特}{7,10}{⽔、⽜}
  \begin{phonetics}{沙特}{sha1te4}
    \definition*{s.}{Saudita | abreviação de 沙特阿拉伯}
    \seeref{沙特阿拉伯}{sha1te4 a1la1bo2}
  \end{phonetics}
\end{entry}

\begin{entry}{沙特阿拉伯}{7,10,7,8,7}{⽔、⽜、⾩、⼿、⼈}
  \begin{phonetics}{沙特阿拉伯}{sha1te4 a1la1bo2}
    \definition*{s.}{Arábia Saudita}
  \end{phonetics}
\end{entry}

\begin{entry}{沙漠}{7,13}{⽔、⽔}
  \begin{phonetics}{沙漠}{sha1mo4}
    \definition[个]{s.}{deserto}
  \end{phonetics}
\end{entry}

\begin{entry}{没}{7}{⽔}
  \begin{phonetics}{没}{mei2}[][HSK 1]
    \definition{adv.}{não ter | não há | ficar sem}
    \definition{pref.}{não (prefixo negativo para verbos, traduzido para outras línguas com verbos no pretérito)}
  \end{phonetics}
  \begin{phonetics}{没}{mo4}
    \definition{adj.}{afogado}
    \definition{v.}{acabar | morrer | inundar}
    \variantof{没}
  \end{phonetics}
\end{entry}

\begin{entry}{没了}{7,2}{⽔、⼅}
  \begin{phonetics}{没了}{mei2le5}
    \definition{v.}{estar morto | deixar de existir}
  \end{phonetics}
\end{entry}

\begin{entry}{没什么}{7,4,3}{⽔、⼈、⼃}
  \begin{phonetics}{没什么}{mei2 shen2me5}[][HSK 1]
    \definition{expr.}{não é nada | está tudo bem | não importa | não importa}
  \end{phonetics}
\end{entry}

\begin{entry}{没用}{7,5}{⽔、⽤}
  \begin{phonetics}{没用}{mei2 yong4}[][HSK 3]
    \definition{adj.}{inútil; imprestável; sem valor; sem préstimo; vão; que não serve para nada}
  \end{phonetics}
\end{entry}

\begin{entry}{没关系}{7,6,7}{⽔、⼋、⽷}
  \begin{phonetics}{没关系}{mei2 guan1xi5}[][HSK 1]
    \definition{v.}{não ter problema | não ter importância | não fazer mal}
    \seeref{没有关系}{mei2you3guan1xi5}
  \end{phonetics}
\end{entry}

\begin{entry}{没有}{7,6}{⽔、⽉}
  \begin{phonetics}{没有}{mei2you3}[][HSK 1]
    \definition{v.}{não há | não tem | não existe}
  \end{phonetics}
\end{entry}

\begin{entry}{没有关系}{7,6,6,7}{⽔、⽉、⼋、⽷}
  \begin{phonetics}{没有关系}{mei2you3guan1xi5}
    \definition{v.}{não ter problema | não ter importância | não fazer mal}
    \seeref{没关系}{mei2 guan1xi5}
  \end{phonetics}
\end{entry}

\begin{entry}{没有意思}{7,6,13,9}{⽔、⽉、⼼、⼼}
  \begin{phonetics}{没有意思}{mei2you3yi4si5}
    \definition{adj.}{tedioso | chato | sem interesse}
  \end{phonetics}
\end{entry}

\begin{entry}{没事儿}{7,8,2}{⽔、⼅、⼉}
  \begin{phonetics}{没事儿}{mei2 shi4r5}[][HSK 1]
    \definition{expr.}{livre de trabalho | sem problemas | não é importante |não é nada |deixa para lá}
    \definition{v.}{ter tempo livre}
  \end{phonetics}
\end{entry}

\begin{entry}{没法儿}{7,8,2}{⽔、⽔、⼉}
  \begin{phonetics}{没法儿}{mei2 fa3r5}[][HSK 4]
    \definition{adv.}{não pode; sem chance}
  \end{phonetics}
\end{entry}

\begin{entry}{没想到}{7,13,8}{⽔、⼼、⼑}
  \begin{phonetics}{没想到}{mei2 xiang3 dao4}[][HSK 4]
    \definition{expr.}{não esperava; inesperado}
  \end{phonetics}
\end{entry}

\begin{entry}{没错}{7,13}{⽔、⾦}
  \begin{phonetics}{没错}{mei2 cuo4}[][HSK 4]
    \definition{adv.}{está certo; é isso mesmo; não há como errar}
  \end{phonetics}
\end{entry}

\begin{entry}{河}{8}{⽔}
  \begin{phonetics}{河}{he2}[][HSK 2]
    \definition[条,道]{s.}{rio}
  \end{phonetics}
\end{entry}

\begin{entry}{河蚌}{8,10}{⽔、⾍}
  \begin{phonetics}{河蚌}{he2bang4}
    \definition{s.}{mexilhões | bivalves cultivados em rios e lagos}
  \end{phonetics}
\end{entry}

\begin{entry}{油}{8}{⽔}
  \begin{phonetics}{油}{you2}[][HSK 2]
    \definition{adj.}{oleoso | gorduroso | superficial | astuto}
    \definition{s.}{óleo | gordura | graxa | petróleo}
    \definition{v.}{aplicar óleo de tungue, tinta ou verniz}
  \end{phonetics}
\end{entry}

\begin{entry}{治}{8}{⽔}
  \begin{phonetics}{治}{zhi4}[][HSK 4]
    \definition*{s.}{sobrenome Zhi}
    \definition{adj.}{calmo e pacífico}
    \definition{s.}{sede de um antigo governo local}
    \definition{v.}{reger; administrar; governar; gerenciar; gerir | tratar (uma doença); curar; sarar | eliminar; controlar pragas | controlar (um rio); restaurar um curso d'água por meio de dragagem | punir; castigar | estudar; pesquisar; explorar}
  \end{phonetics}
\end{entry}

\begin{entry}{治疗}{8,7}{⽔、⽧}
  \begin{phonetics}{治疗}{zhi4liao2}[][HSK 4]
    \definition{s.}{diagnóstico; tratamento}
    \definition{v.}{tratar; curar; remediar; eliminar doenças por meio de medicamentos, cirurgia, etc.}
  \end{phonetics}
\end{entry}

\begin{entry}{治理}{8,11}{⽔、⽟}
  \begin{phonetics}{治理}{zhi4li3}
    \definition{s.}{governança | governo}
    \definition{v.}{gerir para melhor | administrar | por em ordem}
  \end{phonetics}
\end{entry}

\begin{entry}{治愈}{8,13}{⽔、⼼}
  \begin{phonetics}{治愈}{zhi4yu4}
    \definition{v.}{curar | restaurar a saúde}
  \end{phonetics}
\end{entry}

\begin{entry}{泄气}{8,4}{⽔、⽓}
  \begin{phonetics}{泄气}{xie4qi4}
    \definition{adj.}{decepcionante | frustrante | patético}
    \definition{v.+compl.}{perder o coração | sentir-se desencorajado | ficar desanimado}
  \end{phonetics}
\end{entry}

\begin{entry}{法}{8}{⽔}
  \begin{phonetics}{法}{fa3}[][HSK 4]
    \definition*{s.}{França, abreviação de~法国 | Doutrina budista; o dharma | sobrenome Fa}
    \definition{s.}{lei; termo geral para regras de comportamento estabelecidas ou endossadas pelo Estado | maneira; método; modo; meios | padrão; modelo | artes mágicas; feitiço}
    \definition{v.}{fazer com que algo pareça, funcione, etc. como outra coisa; imitar; seguir; acompanhar}
  \seealsoref{法国}{fa3guo2}
  \end{phonetics}
\end{entry}

\begin{entry}{法文}{8,4}{⽔、⽂}
  \begin{phonetics}{法文}{fa3wen2}
    \definition*{s.}{françês, língua francesa}
  \end{phonetics}
\end{entry}

\begin{entry}{法网}{8,6}{⽔、⽹}
  \begin{phonetics}{法网}{fa3wang3}
    \definition*{s.}{Torneio de Roland Garros (French Open), torneio de tênis}
  \end{phonetics}
\end{entry}

\begin{entry}{法国}{8,8}{⽔、⼞}
  \begin{phonetics}{法国}{fa3guo2}
    \definition*{s.}{França}
  \end{phonetics}
\end{entry}

\begin{entry}{法国人}{8,8,2}{⽔、⼞、⼈}
  \begin{phonetics}{法国人}{fa3guo2ren2}
    \definition{s.}{francês | pessoa ou povo da França}
  \end{phonetics}
\end{entry}

\begin{entry}{法官}{8,8}{⽔、⼧}
  \begin{phonetics}{法官}{fa3 guan1}[][HSK 4]
    \definition[位]{s.}{juiz; justiça; termo genérico para um membro do judiciário em um tribunal de justiça}
  \end{phonetics}
\end{entry}

\begin{entry}{法律}{8,9}{⽔、⼻}
  \begin{phonetics}{法律}{fa3lv4}[][HSK 4]
    \definition[项,条,套,个]{s.}{lei; estatuto; regras de conduta formuladas pelo legislativo e cuja aplicação é garantida pelo poder estatal}
  \end{phonetics}
\end{entry}

\begin{entry}{法语}{8,9}{⽔、⾔}
  \begin{phonetics}{法语}{fa3yu3}
    \definition{s.}{françês, língua francesa}
  \end{phonetics}
\end{entry}

\begin{entry}{法院}{8,9}{⽔、⾩}
  \begin{phonetics}{法院}{fa3yuan4}[][HSK 4]
    \definition[所,座]{s.}{tribunal; corte; órgãos estatais que exercem poder judicial independente}
  \end{phonetics}
\end{entry}

\begin{entry}{泡}{8}{⽔}
  \begin{phonetics}{泡}{pao1}
    \definition{adj.}{estufado | inchado | esponjoso}
    \definition{clas.}{para urina ou fezes}
    \definition{s.}{pequeno lago (especialmente em nomes de lugares)}
  \end{phonetics}
  \begin{phonetics}{泡}{pao4}
    \definition{clas.}{para ocorrências de uma ação | para número de infusões}
    \definition{s.}{bolha | espuma}
    \definition{v.}{encharcar | infundir | pegar (uma garota) | sair com (um parceiro sexual)}
  \end{phonetics}
\end{entry}

\begin{entry}{波}{8}{⽔}
  \begin{phonetics}{波}{bo1}
    \definition*{s.}{Polônia, abreviação de 波兰}
    \definition{s.}{onda | ondulação | tempestade | surto}
    \seeref{波兰}{bo1lan2}
  \end{phonetics}
\end{entry}

\begin{entry}{波兰}{8,5}{⽔、⼋}
  \begin{phonetics}{波兰}{bo1lan2}
    \definition*{s.}{Polônia}
  \end{phonetics}
\end{entry}

\begin{entry}{波音}{8,9}{⽔、⾳}
  \begin{phonetics}{波音}{bo1yin1}
    \definition*{s.}{Boeing (empresa aeroespacial)}
    \definition{s.}{mordente (música)}
  \end{phonetics}
\end{entry}

\begin{entry}{泥}{8}{⽔}
  \begin{phonetics}{泥}{ni2}
    \definition{s.}{lama | argila | pasta | polpa}
  \end{phonetics}
  \begin{phonetics}{泥}{ni4}
    \definition{adj.}{contido}
  \end{phonetics}
\end{entry}

\begin{entry}{泥潭}{8,15}{⽔、⽔}
  \begin{phonetics}{泥潭}{ni2tan2}
    \definition{s.}{atoleiro | lamaçal | charco | pântano}
  \end{phonetics}
\end{entry}

\begin{entry}{注册}{8,5}{⽔、⼌}
  \begin{phonetics}{注册}{zhu4ce4}
    \definition{v.}{inscrever-se | matricular-se | registrar-se}
  \end{phonetics}
\end{entry}

\begin{entry}{注册人}{8,5,2}{⽔、⼌、⼈}
  \begin{phonetics}{注册人}{zhu4ce4ren2}
    \definition{s.}{registrante}
  \end{phonetics}
\end{entry}

\begin{entry}{注册表}{8,5,8}{⽔、⼌、⾐}
  \begin{phonetics}{注册表}{zhu4ce4biao3}
    \definition*{s.}{Registro do Windows}
  \end{phonetics}
\end{entry}

\begin{entry}{注册商标}{8,5,11,9}{⽔、⼌、⼝、⽊}
  \begin{phonetics}{注册商标}{zhu4ce4shang1biao1}
    \definition{s.}{marca registrada}
  \end{phonetics}
\end{entry}

\begin{entry}{注意}{8,13}{⽔、⼼}
  \begin{phonetics}{注意}{zhu4yi4}[][HSK 3]
    \definition{v.}{prestar atenção em; tomar nota de; ficar de olho em; concentrar seus pensamentos em uma coisa}
  \end{phonetics}
\end{entry}

\begin{entry}{注意力}{8,13,2}{⽔、⼼、⼒}
  \begin{phonetics}{注意力}{zhu4yi4li4}
    \definition{s.}{atenção}
  \end{phonetics}
\end{entry}

\begin{entry}{注意力缺失症}{8,13,2,10,5,10}{⽔、⼼、⼒、⽸、⼤、⽧}
  \begin{phonetics}{注意力缺失症}{zhu4yi4li4que1shi1zheng4}
    \definition{s.}{transtorno de déficit de atenção}
  \end{phonetics}
\end{entry}

\begin{entry}{注意地}{8,13,6}{⽔、⼼、⼟}
  \begin{phonetics}{注意地}{zhu4yi4di4}
    \definition{s.}{área de cuidado, de observação}
  \end{phonetics}
\end{entry}

\begin{entry}{泪}{8}{⽔}
  \begin{phonetics}{泪}{lei4}[][HSK 4]
    \definition[滴]{s.}{lágrima}
  \end{phonetics}
\end{entry}

\begin{entry}{泪水}{8,4}{⽔、⽔}
  \begin{phonetics}{泪水}{lei4 shui3}[][HSK 4]
    \definition{s.}{lágrima}
  \end{phonetics}
\end{entry}

\begin{entry}{泳池}{8,6}{⽔、⽔}
  \begin{phonetics}{泳池}{yong3chi2}
    \definition{s.}{piscina}
  \seealsoref{游泳池}{you2yong3chi2}
  \seealsoref{游泳馆}{you2yong3guan3}
  \end{phonetics}
\end{entry}

\begin{entry}{泳衣}{8,6}{⽔、⾐}
  \begin{phonetics}{泳衣}{yong3yi1}
    \definition{s.}{roupa de banho | maiô}
  \seealsoref{游泳衣}{you2yong3yi1}
  \end{phonetics}
\end{entry}

\begin{entry}{浅}{8}{⽔}
  \begin{phonetics}{浅}{jian1}
    \definition{adj.}{murmurando, fluindo suavemente, gorgolejando suavemente}
    \definition{s.}{(onomatopéia) som de água em movimento |}
  \end{phonetics}
  \begin{phonetics}{浅}{qian3}[][HSK 4]
    \definition{adj.}{raso; superficial;  (em oposição a ``深'') | fácil; simples; redação, conteúdo, etc. simples e fáceis de entender | superficial; não é profundo em aprendizado, percepção e sabedoria | não próximo; não íntimo; sentimentos não profundos | (cor) claro; pálido;  cor pouco intensa; leve |experiência breve; duração de tempo breve | baixo grau; peso leve; nível baixo}
  \seealsoref{深}{shen1}
  \end{phonetics}
\end{entry}

\begin{entry}{洋葱}{9,12}{⽔、⾋}
  \begin{phonetics}{洋葱}{yang2cong1}
    \definition{s.}{cebola}
  \end{phonetics}
\end{entry}

\begin{entry}{洒水}{9,4}{⽔、⽔}
  \begin{phonetics}{洒水}{sa3shui3}
    \definition{v.}{borrifar}
  \end{phonetics}
\end{entry}

\begin{entry}{洗}{9}{⽔}
  \begin{phonetics}{洗}{xi3}[][HSK 1]
    \definition{v.}{lavar | revelar (fotos) | tomar banho}
  \end{phonetics}
\end{entry}

\begin{entry}{洗手}{9,4}{⽔、⼿}
  \begin{phonetics}{洗手}{xi3shou3}
    \definition{v.}{ir ao banheiro | lavar as mãos}
  \end{phonetics}
\end{entry}

\begin{entry}{洗手不干}{9,4,4,3}{⽔、⼿、⼀、⼲}
  \begin{phonetics}{洗手不干}{xi3shou3bu2gan4}
    \definition{v.}{parar totalmente de fazer algo}
  \end{phonetics}
\end{entry}

\begin{entry}{洗手池}{9,4,6}{⽔、⼿、⽔}
  \begin{phonetics}{洗手池}{xi3shou3chi2}
    \definition{s.}{pia de banheiro | lavatório}
  \seealsoref{洗手盆}{xi3shou3pen2}
  \end{phonetics}
\end{entry}

\begin{entry}{洗手间}{9,4,7}{⽔、⼿、⾨}
  \begin{phonetics}{洗手间}{xi3shou3jian1}[][HSK 1]
    \definition{s.}{sanitário | toilette | banheiro}
  \end{phonetics}
\end{entry}

\begin{entry}{洗手乳}{9,4,8}{⽔、⼿、⼄}
  \begin{phonetics}{洗手乳}{xi3shou3ru3}
    \definition{s.}{sabonete líquido para lavar as mãos}
  \seealsoref{洗手液}{xi3shou3ye4}
  \end{phonetics}
\end{entry}

\begin{entry}{洗手盆}{9,4,9}{⽔、⼿、⽫}
  \begin{phonetics}{洗手盆}{xi3shou3pen2}
    \definition{s.}{pia de banheiro | lavatório}
  \seealsoref{洗手池}{xi3shou3chi2}
  \end{phonetics}
\end{entry}

\begin{entry}{洗手液}{9,4,11}{⽔、⼿、⽔}
  \begin{phonetics}{洗手液}{xi3shou3ye4}
    \definition{s.}{sabonete líquido para lavar as mãos}
  \seealsoref{洗手乳}{xi3shou3ru3}
  \end{phonetics}
\end{entry}

\begin{entry}{洗礼}{9,5}{⽔、⽰}
  \begin{phonetics}{洗礼}{xi3li3}
    \definition{s.}{batismo}
    \definition{v.}{batizar}
  \end{phonetics}
\end{entry}

\begin{entry}{洗衣机}{9,6,6}{⽔、⾐、⽊}
  \begin{phonetics}{洗衣机}{xi3 yi1 ji1}[][HSK 2]
    \definition[台]{s.}{máquina de lavar roupa}
  \end{phonetics}
\end{entry}

\begin{entry}{洗劫}{9,7}{⽔、⼒}
  \begin{phonetics}{洗劫}{xi3jie2}
    \definition{v.}{saquear | pilhar | roubar}
  \end{phonetics}
\end{entry}

\begin{entry}{洗净}{9,8}{⽔、⼎}
  \begin{phonetics}{洗净}{xi3jing4}
    \definition{v.}{lavar (limpeza)}
  \end{phonetics}
\end{entry}

\begin{entry}{洗胃}{9,9}{⽔、⾁}
  \begin{phonetics}{洗胃}{xi3wei4}
    \definition{s.}{(medicina) lavagem gástrica}
    \definition{v.}{ter o estômago lavado}
  \end{phonetics}
\end{entry}

\begin{entry}{洗涤}{9,10}{⽔、⽔}
  \begin{phonetics}{洗涤}{xi3di2}
    \definition{s.}{enxágue | lava}
    \definition{v.}{enxaguar | lavar}
  \end{phonetics}
\end{entry}

\begin{entry}{洗涤间}{9,10,7}{⽔、⽔、⾨}
  \begin{phonetics}{洗涤间}{xi3di2jian1}
    \definition{s.}{lavanderia}
  \end{phonetics}
\end{entry}

\begin{entry}{洗脱}{9,11}{⽔、⾁}
  \begin{phonetics}{洗脱}{xi3tuo1}
    \definition{v.}{limpar | purgar | lavar}
  \end{phonetics}
\end{entry}

\begin{entry}{洗碗}{9,13}{⽔、⽯}
  \begin{phonetics}{洗碗}{xi3wan3}
    \definition{v.}{lavar pratos}
  \end{phonetics}
\end{entry}

\begin{entry}{洗澡}{9,16}{⽔、⽔}
  \begin{phonetics}{洗澡}{xi3zao3}[][HSK 2]
    \definition{v.+compl.}{tomar banho | duchar-se | lavar-se}
  \end{phonetics}
\end{entry}

\begin{entry}{洗澡间}{9,16,7}{⽔、⽔、⾨}
  \begin{phonetics}{洗澡间}{xi3zao3jian1}
    \definition[间]{s.}{banheiro}
  \end{phonetics}
\end{entry}

\begin{entry}{洞穴}{9,5}{⽔、⽳}
  \begin{phonetics}{洞穴}{dong4xue2}
    \definition{s.}{caverna}
  \end{phonetics}
\end{entry}

\begin{entry}{洪水}{9,4}{⽔、⽔}
  \begin{phonetics}{洪水}{hong2shui3}
    \definition{s.}{enchente | inundação | dilúvio}
  \end{phonetics}
\end{entry}

\begin{entry}{洲}{9}{⽔}
  \begin{phonetics}{洲}{zhou1}
    \definition{s.}{continente | ilha em um rio}
  \end{phonetics}
\end{entry}

\begin{entry}{活}{9}{⽔}
  \begin{phonetics}{活}{huo2}[][HSK 3]
    \definition{adj.}{vivo; vivendo | vívido; animado; ativo | móvel; em movimento}
    \definition{adv.}{exatamente; simplesmente}
    \definition{s.}{trabalho | produto}
    \definition{v.}{viver | salvar (a vida de uma pessoa)}
  \end{phonetics}
\end{entry}

\begin{entry}{活力}{9,2}{⽔、⼒}
  \begin{phonetics}{活力}{huo2li4}
    \definition{s.}{energia | vitalidade | vigor | força vital}
  \end{phonetics}
\end{entry}

\begin{entry}{活动}{9,6}{⽔、⼒}
  \begin{phonetics}{活动}{huo2dong4}[][HSK 2]
    \definition[项,个]{s.}{atividade | evento | campanha}
    \definition{v.}{exercer | operar}
  \end{phonetics}
\end{entry}

\begin{entry}{活着}{9,11}{⽔、⽬}
  \begin{phonetics}{活着}{huo2zhe5}
    \definition{adj.}{vivo}
  \end{phonetics}
\end{entry}

\begin{entry}{活路}{9,13}{⽔、⾜}
  \begin{phonetics}{活路}{huo2lu4}
    \definition{s.}{maneira de sobreviver | meio de subsistência}
  \end{phonetics}
  \begin{phonetics}{活路}{huo2lu5}
    \definition{s.}{labor | trabalho físico}
  \end{phonetics}
\end{entry}

\begin{entry}{派}{9}{⽔}
  \begin{phonetics}{派}{pai4}[][HSK 3]
    \definition{adj.}{elegante; bonito}
    \definition{clas.}{para grupos, escolas de pensamento ou arte, etc. | para um discursos, atmosferas, cenas, etc.}
    \definition{s.}{panelinha; grupo exclusivo; facção | torta | estilo | afluente; braço de rio}
    \definition{v.}{enviar; despachar | alocar; repartir; distribuir}
  \end{phonetics}
\end{entry}

\begin{entry}{测}{9}{⽔}
  \begin{phonetics}{测}{ce4}[][HSK 4]
    \definition{v.}{pesquisar; sondar; medir | conjecturar; inferir}
  \end{phonetics}
\end{entry}

\begin{entry}{测试}{9,8}{⽔、⾔}
  \begin{phonetics}{测试}{ce4 shi4}[][HSK 4]
    \definition[个]{s.}{exame; teste; medição do conhecimento humano, das habilidades ou do funcionamento de máquinas, ferramentas ou instrumentos}
    \definition{v.}{examinar | testar, medição do desempenho e da precisão de máquinas, instrumentos, aparelhos, etc.}
  \end{phonetics}
\end{entry}

\begin{entry}{测量}{9,12}{⽔、⾥}
  \begin{phonetics}{测量}{ce4liang2}[][HSK 4]
    \definition{v.}{aferir; pesquisar; medir; determinar valores relevantes para espaço, tempo, temperatura, velocidade, função, etc.}
  \end{phonetics}
\end{entry}

\begin{entry}{浓}{9}{⽔}
  \begin{phonetics}{浓}{nong2}[][HSK 4]
    \definition{adj.}{denso; espesso; concentrado; um líquido ou gás que contém mais de um determinado ingrediente | grande; forte; profundo (de grau ou extensão) | profundo; (algumas cores) escuro}
  \end{phonetics}
\end{entry}

\begin{entry}{流}{10}{⽔}
  \begin{phonetics}{流}{liu2}[][HSK 2]
    \definition[名,个]{s.}{fluxo de água | correnteza | córrego | algo que se assemelha a um fluxo de água | corrente | fluxo | classe | grau | taxa (de variação)}
    \definition{v.}{fluir
deriva; mover; vagar
espalhar
degenerar; mudar para pior
enviar para o exílio; banir}
  \end{phonetics}
\end{entry}

\begin{entry}{流水}{10,4}{⽔、⽔}
  \begin{phonetics}{流水}{liu2shui3}
    \definition{s.}{água corrente | (negócio) rotatividade}
  \end{phonetics}
\end{entry}

\begin{entry}{流传}{10,6}{⽔、⼈}
  \begin{phonetics}{流传}{liu2chuan2}[][HSK 4]
    \definition{v.}{espalhar; circular; passar adiante}
  \end{phonetics}
\end{entry}

\begin{entry}{流行}{10,6}{⽔、⾏}
  \begin{phonetics}{流行}{liu2xing2}[][HSK 2]
    \definition{adj.}{(estilo de roupa, música, etc.) popular, na moda}
    \definition{v.}{(doença contagiosa, etc.) espalhar | propagar}
  \end{phonetics}
\end{entry}

\begin{entry}{流利}{10,7}{⽔、⼑}
  \begin{phonetics}{流利}{liu2li4}[][HSK 2]
    \definition{adj.}{fluente (em uma língua)}
  \end{phonetics}
\end{entry}

\begin{entry}{流星}{10,9}{⽔、⽇}
  \begin{phonetics}{流星}{liu2xing1}
    \definition{s.}{meteoro | estrela cadente}
  \end{phonetics}
\end{entry}

\begin{entry}{浙江}{10,6}{⽔、⽔}
  \begin{phonetics}{浙江}{zhe4jiang1}
    \definition*{s.}{Zhejiang}
  \end{phonetics}
\end{entry}

\begin{entry}{浪花}{10,7}{⽔、⾋}
  \begin{phonetics}{浪花}{lang4hua1}
    \definition[朵]{s.}{\emph{spray} | \emph{spray} do oceano | (figurativo) acontecimentos de sua vida}
  \end{phonetics}
\end{entry}

\begin{entry}{浪费}{10,9}{⽔、⾙}
  \begin{phonetics}{浪费}{lang4fei4}[][HSK 3]
    \definition{adj.}{desperdiçado}
    \definition{adv.}{extravagantemente}
    \definition{v.}{desperdiçar; dissipar; esbanjar; ser extravagante}
  \end{phonetics}
\end{entry}

\begin{entry}{浪漫}{10,14}{⽔、⽔}
  \begin{phonetics}{浪漫}{lang4man4}
    \definition{adj.}{romântico}
  \end{phonetics}
\end{entry}

\begin{entry}{浮力}{10,2}{⽔、⼒}
  \begin{phonetics}{浮力}{fu2li4}
    \definition{s.}{flutuabilidade}
  \end{phonetics}
\end{entry}

\begin{entry}{浮图}{10,8}{⽔、⼞}
  \begin{phonetics}{浮图}{fu2tu2}
    \definition*{s.}{Termo alternativo para 佛陀}
    \variantof{浮屠}
  \seealsoref{佛陀}{fo2tuo2}
  \end{phonetics}
\end{entry}

\begin{entry}{浮屠}{10,11}{⽔、⼫}
  \begin{phonetics}{浮屠}{fu2tu2}
    \definition*{s.}{Buda | Templo (Stupa) Budista (transliteração de Pali Thuo)}
  \end{phonetics}
\end{entry}

\begin{entry}{海}{10}{⽔}
  \begin{phonetics}{海}{hai3}[][HSK 2]
    \definition*{s.}{sobrenome Hai}
    \definition[个,片]{s.}{mar | oceano}
  \end{phonetics}
\end{entry}

\begin{entry}{海水}{10,4}{⽔、⽔}
  \begin{phonetics}{海水}{hai3 shui3}[][HSK 4]
    \definition[把]{s.}{água do mar; salmoura}
  \end{phonetics}
\end{entry}

\begin{entry}{海风}{10,4}{⽔、⾵}
  \begin{phonetics}{海风}{hai3feng1}
    \definition{s.}{brisa do mar | vento que vem do mar}
  \end{phonetics}
\end{entry}

\begin{entry}{海边}{10,5}{⽔、⾡}
  \begin{phonetics}{海边}{hai3 bian1}[][HSK 2]
    \definition{s.}{costa marítima | litoral | beira-mar | praia}
  \end{phonetics}
\end{entry}

\begin{entry}{海关}{10,6}{⽔、⼋}
  \begin{phonetics}{海关}{hai3guan1}[][HSK 3]
    \definition{s.}{alfândega}
  \end{phonetics}
\end{entry}

\begin{entry}{海里}{10,7}{⽔、⾥}
  \begin{phonetics}{海里}{hai3li3}
    \definition{s.}{milha náutica}
  \end{phonetics}
\end{entry}

\begin{entry}{海底}{10,8}{⽔、⼴}
  \begin{phonetics}{海底}{hai3di3}
    \definition{adj.}{submarino}
    \definition{s.}{fundo do mar | solo oceânico | fundo do oceano}
  \end{phonetics}
\end{entry}

\begin{entry}{海鸥}{10,9}{⽔、⿃}
  \begin{phonetics}{海鸥}{hai3'ou1}
    \definition{s.}{gaivota}
  \end{phonetics}
\end{entry}

\begin{entry}{海浪}{10,10}{⽔、⽔}
  \begin{phonetics}{海浪}{hai3lang4}
    \definition{s.}{ondas do mar}
  \end{phonetics}
\end{entry}

\begin{entry}{海绵}{10,11}{⽔、⽷}
  \begin{phonetics}{海绵}{hai3mian2}
    \definition{s.}{(zoologia) esponja do mar | esponja (feita de poliéster ou celulose, etc.) | espuma de borracha}
  \end{phonetics}
\end{entry}

\begin{entry}{海棠}{10,12}{⽔、⽊}
  \begin{phonetics}{海棠}{hai3tang2}
    \definition{s.}{begônia}
  \end{phonetics}
\end{entry}

\begin{entry}{海鲜}{10,14}{⽔、⿂}
  \begin{phonetics}{海鲜}{hai3xian1}[][HSK 4]
    \definition[种,份,桌,批,些]{s.}{frutos do mar; mariscos; peixes marinhos frescos, camarões, etc., para consumo |}
  \end{phonetics}
\end{entry}

\begin{entry}{消化}{10,4}{⽔、⼔}
  \begin{phonetics}{消化}{xiao1hua4}[][HSK 4]
    \definition{v.}{digerir (alimentos) | digerir (conhecimento); pensar e absorver; uma metáfora para a compreensão total de novos conhecimentos ou informações e a capacidade de transformá-los em algo que possa ser usado}
  \end{phonetics}
\end{entry}

\begin{entry}{消失}{10,5}{⽔、⼤}
  \begin{phonetics}{消失}{xiao1shi1}[][HSK 3]
    \definition{v.}{desaparecer; desvanecer; dissolver; dissipar; evaporar; sumir}
  \end{phonetics}
\end{entry}

\begin{entry}{消防}{10,6}{⽔、⾩}
  \begin{phonetics}{消防}{xiao1fang2}
    \definition{s.}{combate a incêncios | controle de incêndios}
  \end{phonetics}
\end{entry}

\begin{entry}{消防员}{10,6,7}{⽔、⾩、⼝}
  \begin{phonetics}{消防员}{xiao1fang2yuan2}
    \definition{s.}{bombeiro}
  \end{phonetics}
\end{entry}

\begin{entry}{消费}{10,9}{⽔、⾙}
  \begin{phonetics}{消费}{xiao1fei4}[][HSK 3]
    \definition{v.}{gastar; consumir | consumir (recursos naturais)}
  \end{phonetics}
\end{entry}

\begin{entry}{消息}{10,10}{⽔、⼼}
  \begin{phonetics}{消息}{xiao1xi5}[][HSK 3]
    \definition[个,条,篇]{s.}{notícias; informação}
  \end{phonetics}
\end{entry}

\begin{entry}{涨价}{10,6}{⽔、⼈}
  \begin{phonetics}{涨价}{zhang3jia4}
    \definition{s.}{aumento de preços}
    \definition{v.+compl.}{avaliar (em valor) | dar preço | aumentar o preço}
  \end{phonetics}
\end{entry}

\begin{entry}{液体}{11,7}{⽔、⼈}
  \begin{phonetics}{液体}{ye4ti3}
    \definition{adj./s.}{líquido}
  \end{phonetics}
\end{entry}

\begin{entry}{涵}{11}{⽔}
  \begin{phonetics}{涵}{han2}
    \definition{s.}{bueiro | galeria}
    \definition{v.}{conter | incluir | entupir}
  \end{phonetics}
\end{entry}

\begin{entry}{淀}{11}{⽔}
  \begin{phonetics}{淀}{dian4}
    \definition{adj.}{pantanoso}
    \definition{s.}{lago raso | pântano}
    \definition{v.}{formar sedimentos | precipitar}
  \end{phonetics}
\end{entry}

\begin{entry}{淋}{11}{⽔}
  \begin{phonetics}{淋}{lin2}
    \definition{v.}{borrifar | pingar | derramar | encharcar}
  \end{phonetics}
  \begin{phonetics}{淋}{lin4}
    \definition{s.}{gonorréia}
    \definition{v.}{filtrar | coar | drenar}
  \end{phonetics}
\end{entry}

\begin{entry}{淡}{11}{⽔}
  \begin{phonetics}{淡}{dan4}[][HSK 4]
    \definition*{s.}{sobrenome Dan}
    \definition{adj.}{fino; leve | sem gosto; fraco; não tem sabor forte; não é salgado | leve; fraco; pálido | indiferente; frio; sem entusiasmo | frouxo; sem brilho | sem sentido; trivial}
  \end{phonetics}
\end{entry}

\begin{entry}{淤泥}{11,8}{⽔、⽔}
  \begin{phonetics}{淤泥}{yu1ni2}
    \definition{s.}{lodo}
  \end{phonetics}
\end{entry}

\begin{entry}{深}{11}{⽔}
  \begin{phonetics}{深}{shen1}[][HSK 3]
    \definition*{s.}{sobrenome Shen}
    \definition{adj.}{profundo | difícil; intenso; profundo | completo; penetrante; intenso; profundo | próximo; íntimo | escuro; profundo | tardio}
    \definition{adv.}{muito; grandemente; profundamente}
    \definition{s.}{profundidade}
  \seealsoref{浅}{qian3}
  \end{phonetics}
\end{entry}

\begin{entry}{深入}{11,2}{⽔、⼊}
  \begin{phonetics}{深入}{shen1 ru4}[][HSK 3]
    \definition{adj.}{minucioso; meticuloso; profundo}
    \definition{v.}{ir fundo em; penetrar em}
  \end{phonetics}
\end{entry}

\begin{entry}{深刻}{11,8}{⽔、⼑}
  \begin{phonetics}{深刻}{shen1ke4}[][HSK 3]
    \definition{adj.}{profundo; instenso}
  \end{phonetics}
\end{entry}

\begin{entry}{深夜}{11,8}{⽔、⼣}
  \begin{phonetics}{深夜}{shen1ye4}
    \definition{adv.}{tarde da noite}
  \end{phonetics}
\end{entry}

\begin{entry}{深厚}{11,9}{⽔、⼚}
  \begin{phonetics}{深厚}{shen1hou4}[][HSK 4]
    \definition{adj.}{profundo; sentimentos fortes | sólido; profundamente enraizado; fundação sólida}
  \end{phonetics}
\end{entry}

\begin{entry}{深深}{11,11}{⽔、⽔}
  \begin{phonetics}{深深}{shen1shen1}
    \definition{adj.}{profundo}
    \definition{adv.}{profundamente}
  \end{phonetics}
\end{entry}

\begin{entry}{混乱}{11,7}{⽔、⼄}
  \begin{phonetics}{混乱}{hun4luan4}
    \definition{adj.}{confuso | caótico | desordenado}
    \definition{s.}{caos}
  \end{phonetics}
\end{entry}

\begin{entry}{混饭}{11,7}{⽔、⾷}
  \begin{phonetics}{混饭}{hun4fan4}
    \definition{v.+compl.}{trabalhar para viver}
  \end{phonetics}
\end{entry}

\begin{entry}{清}{11}{⽔}
  \begin{phonetics}{清}{qing1}
    \definition*{s.}{sobrenome Qing}
    \definition{adj.}{claro | limpo (água, etc.) | tranquilo | quieto | puro | não corrompido | distinto}
    \definition{v.}{limpar | resolver (contas)}
  \end{phonetics}
\end{entry}

\begin{entry}{清彻}{11,7}{⽔、⼻}
  \begin{phonetics}{清彻}{qing1che4}
    \variantof{清澈}
  \end{phonetics}
\end{entry}

\begin{entry}{清明节}{11,8,5}{⽔、⽇、⾋}
  \begin{phonetics}{清明节}{qing1ming2jie2}
    \definition*{s.}{Dia Qingming, Dia dos Finados (uma das 24~divisões do ano solar no calendário lunar chinês:~dia~4 ou 5~de~abril solar)}
  \end{phonetics}
\end{entry}

\begin{entry}{清凉}{11,10}{⽔、⼎}
  \begin{phonetics}{清凉}{qing1liang2}
    \definition{adj.}{fresco | refrescante | (roupa) ousada, reveladora}
  \end{phonetics}
\end{entry}

\begin{entry}{清唱}{11,11}{⽔、⼝}
  \begin{phonetics}{清唱}{qing1chang4}
    \definition{v.}{cantar à capela}
  \end{phonetics}
\end{entry}

\begin{entry}{清爽}{11,11}{⽔、⽘}
  \begin{phonetics}{清爽}{qing1shuang3}
    \definition{adj.}{refrescante | relaxado}
  \end{phonetics}
\end{entry}

\begin{entry}{清理}{11,11}{⽔、⽟}
  \begin{phonetics}{清理}{qing1li3}
    \definition{v.}{limpar | arrumar | descartar}
  \end{phonetics}
\end{entry}

\begin{entry}{清晰}{11,12}{⽔、⽇}
  \begin{phonetics}{清晰}{qing1xi1}
    \definition{adj.}{claro | distinto}
  \end{phonetics}
\end{entry}

\begin{entry}{清楚}{11,13}{⽔、⽊}
  \begin{phonetics}{清楚}{qing1chu5}[][HSK 2]
    \definition{adj.}{claro | límpido}
    \definition{v.}{ser claro sobre | entender completamente}
  \end{phonetics}
\end{entry}

\begin{entry}{清澈}{11,15}{⽔、⽔}
  \begin{phonetics}{清澈}{qing1che4}
    \definition{adj.}{claro | límpido}
  \end{phonetics}
\end{entry}

\begin{entry}{清醒}{11,16}{⽔、⾣}
  \begin{phonetics}{清醒}{qing1xing3}[][HSK 4]
    \definition{adj.}{sóbrio; lúcido; totalmente acordado}
  \end{phonetics}
\end{entry}

\begin{entry}{渐渐}{11,11}{⽔、⽔}
  \begin{phonetics}{渐渐}{jian4 jian4}[][HSK 4]
    \definition{adv.}{gradualmente; pouco a pouco; passo a passo; indica um aumento ou diminuição gradual em grau ou quantidade}
  \end{phonetics}
\end{entry}

\begin{entry}{渔}{11}{⽔}
  \begin{phonetics}{渔}{yu2}
    \definition[条]{s.}{pescador}
    \definition{v.}{pescar}
  \end{phonetics}
\end{entry}

\begin{entry}{渔夫}{11,4}{⽔、⼤}
  \begin{phonetics}{渔夫}{yu2fu1}
    \definition{s.}{pescador}
  \end{phonetics}
\end{entry}

\begin{entry}{渔民}{11,5}{⽔、⽒}
  \begin{phonetics}{渔民}{yu2min2}
    \definition{s.}{pescadores | povo pescador}
  \end{phonetics}
\end{entry}

\begin{entry}{渔场}{11,6}{⽔、⼟}
  \begin{phonetics}{渔场}{yu2chang3}
    \definition{s.}{área de pesca}
  \end{phonetics}
\end{entry}

\begin{entry}{渔汛}{11,6}{⽔、⽔}
  \begin{phonetics}{渔汛}{yu2xun4}
    \definition{s.}{temporada de pesca}
  \end{phonetics}
\end{entry}

\begin{entry}{渔网}{11,6}{⽔、⽹}
  \begin{phonetics}{渔网}{yu2wang3}
    \definition{s.}{rede de pesca}
  \end{phonetics}
\end{entry}

\begin{entry}{渔具}{11,8}{⽔、⼋}
  \begin{phonetics}{渔具}{yu2ju4}
    \definition{s.}{equipamento de pesca}
  \end{phonetics}
\end{entry}

\begin{entry}{渔轮}{11,8}{⽔、⾞}
  \begin{phonetics}{渔轮}{yu2lun2}
    \definition{s.}{navio de pesca}
  \end{phonetics}
\end{entry}

\begin{entry}{渔捞}{11,10}{⽔、⼿}
  \begin{phonetics}{渔捞}{yu2lao1}
    \definition{s.}{pesca (como atividade comercial)}
  \end{phonetics}
\end{entry}

\begin{entry}{渔猎}{11,11}{⽔、⽝}
  \begin{phonetics}{渔猎}{yu2lie4}
    \definition{s.}{pesca e caça}
    \definition{v.}{saquear | pilhar}
  \end{phonetics}
\end{entry}

\begin{entry}{渔笼}{11,11}{⽔、⽵}
  \begin{phonetics}{渔笼}{yu2long2}
    \definition{s.}{gaiola de pesca | armadilha de pesca}
  \end{phonetics}
\end{entry}

\begin{entry}{渔船}{11,11}{⽔、⾈}
  \begin{phonetics}{渔船}{yu2chuan2}
    \definition[条]{s.}{barco de pesca}
  \seealsoref{鱼船}{yu2chuan2}
  \end{phonetics}
\end{entry}

\begin{entry}{渔船队}{11,11,4}{⽔、⾈、⾩}
  \begin{phonetics}{渔船队}{yu2chuan2dui4}
    \definition{s.}{frota pesqueira}
  \end{phonetics}
\end{entry}

\begin{entry}{渡过}{12,6}{⽔、⾡}
  \begin{phonetics}{渡过}{du4guo4}
    \definition{v.}{atravessar | passar por}
  \end{phonetics}
\end{entry}

\begin{entry}{温度}{12,9}{⽔、⼴}
  \begin{phonetics}{温度}{wen1du4}[][HSK 2]
    \definition[个]{s.}{temperatura}
  \end{phonetics}
\end{entry}

\begin{entry}{温度计}{12,9,4}{⽔、⼴、⾔}
  \begin{phonetics}{温度计}{wen1du4ji4}
    \definition{s.}{termógrafo | termômetro}
  \end{phonetics}
\end{entry}

\begin{entry}{温度表}{12,9,8}{⽔、⼴、⾐}
  \begin{phonetics}{温度表}{wen1du4biao3}
    \definition{s.}{termômetro}
  \end{phonetics}
\end{entry}

\begin{entry}{温度梯度}{12,9,11,9}{⽔、⼴、⽊、⼴}
  \begin{phonetics}{温度梯度}{wen1du4ti1du4}
    \definition{s.}{gradiente de temperatura}
  \end{phonetics}
\end{entry}

\begin{entry}{温柔}{12,9}{⽔、⽊}
  \begin{phonetics}{温柔}{wen1rou2}
    \definition{adj.}{gentil e suave | terno | doce (comumente usado para descrever uma menina ou mulher)}
  \end{phonetics}
\end{entry}

\begin{entry}{温暖}{12,13}{⽔、⽇}
  \begin{phonetics}{温暖}{wen1nuan3}[][HSK 3]
    \definition{adj.}{caloroso; gentil}
    \definition{v.}{aquecer (fazer você se sentir aquecido)}
  \end{phonetics}
\end{entry}

\begin{entry}{渴}{12}{⽔}
  \begin{phonetics}{渴}{ke3}[][HSK 1]
    \definition{adj.}{sedento}
  \end{phonetics}
\end{entry}

\begin{entry}{游}{12}{⽔}
  \begin{phonetics}{游}{you2}[][HSK 3]
    \definition*{s.}{sobrenome You}
    \definition{adj.}{itinerante; errante; não fixo; frequentemente em movimento}
    \definition{s.}{parte de um rio; uma seção do rio}
    \definition{v.}{nadar; pessoas ou animais se movendo na água | vagar por aí; vagar; viajar; passear | associar com (comunicação)}
  \end{phonetics}
\end{entry}

\begin{entry}{游戏}{12,6}{⽔、⼽}
  \begin{phonetics}{游戏}{you2xi4}[][HSK 3]
    \definition[场]{s.}{jogo; recreação; atividades recreativas, como esconde-esconde, adivinhação de enigmas de lanternas e algumas atividades esportivas informais, como bola recreativa, também são chamadas de jogos}
    \definition{v.}{jogar; fazer atividades relaxantes e prazerosas sozinho ou com outras pessoas}
  \end{phonetics}
\end{entry}

\begin{entry}{游泳}{12,8}{⽔、⽔}
  \begin{phonetics}{游泳}{you2yong3}[][HSK 3]
    \definition[次]{s.}{natação; refere-se ao esporte ou atividade de natação}
    \definition{v.+compl.}{nadar; pessoas ou animais nadando na água}
  \end{phonetics}
\end{entry}

\begin{entry}{游泳池}{12,8,6}{⽔、⽔、⽔}
  \begin{phonetics}{游泳池}{you2yong3chi2}
    \definition[场]{s.}{piscina}
  \seealsoref{泳池}{yong3chi2}
  \seealsoref{游泳馆}{you2yong3guan3}
  \end{phonetics}
\end{entry}

\begin{entry}{游泳衣}{12,8,6}{⽔、⽔、⾐}
  \begin{phonetics}{游泳衣}{you2yong3yi1}
    \definition{s.}{roupa de banho}
  \seealsoref{泳衣}{yong3yi1}
  \end{phonetics}
\end{entry}

\begin{entry}{游泳馆}{12,8,11}{⽔、⽔、⾷}
  \begin{phonetics}{游泳馆}{you2yong3guan3}
    \definition[场]{s.}{piscina}
  \seealsoref{泳池}{yong3chi2}
  \seealsoref{游泳池}{you2yong3chi2}
  \end{phonetics}
\end{entry}

\begin{entry}{游泳镜}{12,8,16}{⽔、⽔、⾦}
  \begin{phonetics}{游泳镜}{you2yong3jing4}
    \definition{s.}{óculos de natação}
  \end{phonetics}
\end{entry}

\begin{entry}{游客}{12,9}{⽔、⼧}
  \begin{phonetics}{游客}{you2 ke4}[][HSK 2]
    \definition{s.}{viajante | turista | (jogo online) jogador convidado}
  \end{phonetics}
\end{entry}

\begin{entry}{游艇}{12,12}{⽔、⾈}
  \begin{phonetics}{游艇}{you2ting3}
    \definition[只]{s.}{barcaça | iate}
  \end{phonetics}
\end{entry}

\begin{entry}{湖}{12}{⽔}
  \begin{phonetics}{湖}{hu2}[][HSK 2]
    \definition[个,片]{s.}{lago}
  \end{phonetics}
\end{entry}

\begin{entry}{湖南}{12,9}{⽔、⼗}
  \begin{phonetics}{湖南}{hu2nan2}
    \definition*{s.}{Hunan}
  \end{phonetics}
\end{entry}

\begin{entry}{湿}{12}{⽔}
  \begin{phonetics}{湿}{shi1}[][HSK 4]
    \definition{adj.}{molhado; úmido; algo com água ou com muita água dentro}
  \end{phonetics}
\end{entry}

\begin{entry}{滑}{12}{⽔}
  \begin{phonetics}{滑}{hua2}
    \definition*{s.}{sobrenome Hua}
    \definition{adj.}{deslizado}
    \definition{v.}{deslizar}
  \end{phonetics}
\end{entry}

\begin{entry}{滑雪}{12,11}{⽔、⾬}
  \begin{phonetics}{滑雪}{hua2xue3}
    \definition{v.+compl.}{esquiar | praticar esqui}
  \end{phonetics}
\end{entry}

\begin{entry}{滔天}{13,4}{⽔、⼤}
  \begin{phonetics}{滔天}{tao1tian1}
    \definition{adj.}{(ondas, raiva, desastres, crimes, etc.) imponente, avassalador, imenso}
  \end{phonetics}
\end{entry}

\begin{entry}{滚轮}{13,8}{⽔、⾞}
  \begin{phonetics}{滚轮}{gun3lun2}
    \definition{s.}{pneu | dial rotativo | roda de rolagem (\emph{scroll})  (mouse de computador)}
  \end{phonetics}
\end{entry}

\begin{entry}{滚滚}{13,13}{⽔、⽔}
  \begin{phonetics}{滚滚}{gun3gun3}
    \definition*{s.}{Apelido para um panda}
    \definition{v.}{mover-se | rolar | fluir continuamente}
  \end{phonetics}
\end{entry}

\begin{entry}{满}{13}{⽔}
  \begin{phonetics}{满}{man3}[][HSK 2]
    \definition{adj.}{completo | preenchido | embalado | satisfeito | contente}
    \definition{adv.}{completamente | bastante}
    \definition{v.}{preencher | atingir o limite | satisfazer}
  \end{phonetics}
\end{entry}

\begin{entry}{满分}{13,4}{⽔、⼑}
  \begin{phonetics}{满分}{man3fen1}
    \definition{s.}{pontuação completa}
  \end{phonetics}
\end{entry}

\begin{entry}{满足}{13,7}{⽔、⾜}
  \begin{phonetics}{满足}{man3zu2}[][HSK 3]
    \definition{v.}{estar satisfeito; contentar-se | satisfazer; causar satisfação; contentar}
  \end{phonetics}
\end{entry}

\begin{entry}{满意}{13,13}{⽔、⼼}
  \begin{phonetics}{满意}{man3yi4}[][HSK 2]
    \definition{adj.}{satisfatório}
  \end{phonetics}
\end{entry}

\begin{entry}{满满}{13,13}{⽔、⽔}
  \begin{phonetics}{满满}{man3man3}
    \definition{adj.}{completo | densamente empacotado}
  \end{phonetics}
\end{entry}

\begin{entry}{滴}{14}{⽔}
  \begin{phonetics}{滴}{di1}
    \definition{s.}{uma gota}
    \definition{v.}{pingar}
  \end{phonetics}
\end{entry}

\begin{entry}{漂}{14}{⽔}
  \begin{phonetics}{漂}{piao1}
    \definition{v.}{flutuar | estar a deriva}
  \end{phonetics}
  \begin{phonetics}{漂}{piao3}
    \definition{v.}{alvejar | branquear}
  \end{phonetics}
  \begin{phonetics}{漂}{piao4}
    \definition{adj.}{usado em 漂亮}
    \seeref{漂亮}{piao4liang5}
  \end{phonetics}
\end{entry}

\begin{entry}{漂亮}{14,9}{⽔、⼇}
  \begin{phonetics}{漂亮}{piao4liang5}[][HSK 2]
    \definition{adj.}{bonita, linda | bonito, lindo (para objetos inanimados)}
  \end{phonetics}
\end{entry}

\begin{entry}{漂流}{14,10}{⽔、⽔}
  \begin{phonetics}{漂流}{piao1liu2}
    \definition{s.}{\emph{rafting}}
    \definition{v.}{ser levado pela correnteza | flutuar ao longo ou sobre}
  \end{phonetics}
\end{entry}

\begin{entry}{漏}{14}{⽔}
  \begin{phonetics}{漏}{lou4}
    \definition{s.}{relógio d'água ou ampulheta}
    \definition{v.}{vazar | divulgar | deixar de fora por engano}
  \end{phonetics}
\end{entry}

\begin{entry}{漏电}{14,5}{⽔、⽥}
  \begin{phonetics}{漏电}{lou4dian4}
    \definition{v.}{vazar eletricidade}
  \end{phonetics}
\end{entry}

\begin{entry}{演}{14}{⽔}
  \begin{phonetics}{演}{yan3}[][HSK 3]
    \definition{v.}{desenvolver; evoluir | deduzir; elaborar | exercitar; praticar | representar; atuar}
  \end{phonetics}
\end{entry}

\begin{entry}{演出}{14,5}{⽔、⼐}
  \begin{phonetics}{演出}{yan3chu1}[][HSK 3]
    \definition[场,次]{s.}{show; concerto; performance}
    \definition{v.}{apresentar; representar; fazer um show; apresentar peças de teatro, dança, arte popular, acrobacias, etc. para o público aproveitar}
  \end{phonetics}
\end{entry}

\begin{entry}{演讲}{14,6}{⽔、⾔}
  \begin{phonetics}{演讲}{yan3jiang3}[][HSK 4]
    \definition[场,次]{s.}{palestra; discurso; ato ou a atividade de apresentar ou expressar ideias, opiniões ou informações oralmente em público ou diante de um público}
    \definition{v.}{dar uma palestra; fazer um discurso; informar o público sobre uma determinada área de conhecimento ou opinião sobre um determinado assunto}
  \end{phonetics}
\end{entry}

\begin{entry}{演员}{14,7}{⽔、⼝}
  \begin{phonetics}{演员}{yan3yuan2}[][HSK 3]
    \definition[个,位,名]{s.}{ator; performer; participantes de teatro, cinema, dança, arte popular, acrobacia e outras apresentações}
  \end{phonetics}
\end{entry}

\begin{entry}{演唱}{14,11}{⽔、⼝}
  \begin{phonetics}{演唱}{yan3 chang4}[][HSK 3]
    \definition{v.}{cantar em uma performance; apresentar canções, óperas, dramas, etc.}
  \end{phonetics}
\end{entry}

\begin{entry}{演唱会}{14,11,6}{⽔、⼝、⼈}
  \begin{phonetics}{演唱会}{yan3 chang4 hui4}[][HSK 3]
    \definition[个,场]{s.}{concerto; recital vocal; concerto vocal}
  \end{phonetics}
\end{entry}

\begin{entry}{漫骂}{14,9}{⽔、⾺}
  \begin{phonetics}{漫骂}{man4ma4}
    \variantof{谩骂}
  \end{phonetics}
\end{entry}

\begin{entry}{潜在}{15,6}{⽔、⼟}
  \begin{phonetics}{潜在}{qian2zai4}
    \definition{adj.}{oculto | latente}
    \definition{s.}{potencial}
  \end{phonetics}
\end{entry}

\begin{entry}{潮}{15}{⽔}
  \begin{phonetics}{潮}{chao2}[][HSK 4]
    \definition{adj.}{úmido; molhado | inferior; de qualidade ruim | inferior; não muito habilidoso}
    \definition{s.}{maré; água da maré | surto; corrente; maré; uma metáfora para mudanças sociais em grande escala ou para os altos e baixos de um movimento (social)}
  \end{phonetics}
\end{entry}

\begin{entry}{潮流}{15,10}{⽔、⽔}
  \begin{phonetics}{潮流}{chao2liu2}[][HSK 4]
    \definition{s.}{maré; corrente de maré; movimento da água devido às marés | tendência; analogia com mudanças sociais ou tendências de desenvolvimento}
  \end{phonetics}
\end{entry}

\begin{entry}{潮湿}{15,12}{⽔、⽔}
  \begin{phonetics}{潮湿}{chao2shi1}[][HSK 4]
    \definition{adj.}{molhado; úmido; umedecido; que contém mais água do que o normal}
  \end{phonetics}
\end{entry}

\begin{entry}{澳}{15}{⽔}
  \begin{phonetics}{澳}{ao4}
    \definition*{s.}{Austrália, abreviação de 澳大利亚}
  \seealsoref{澳大利亚}{ao4da4li4ya4}
  \end{phonetics}
\end{entry}

\begin{entry}{澳大利亚}{15,3,7,6}{⽔、⼤、⼑、⼆}
  \begin{phonetics}{澳大利亚}{ao4da4li4ya4}
    \definition*{s.}{Austrália}
  \end{phonetics}
\end{entry}

\begin{entry}{激动}{16,6}{⽔、⼒}
  \begin{phonetics}{激动}{ji1dong4}[][HSK 4]
    \definition{adj.}{animado; entusiasmado; empolgado}
    \definition{v.}{agitar; excitar; tornar fortes os sentimentos de alguém}
  \end{phonetics}
\end{entry}

\begin{entry}{激烈}{16,10}{⽔、⽕}
  \begin{phonetics}{激烈}{ji1lie4}[][HSK 4]
    \definition{adj.}{agudo; afiado; feroz; violento; intenso}
  \end{phonetics}
\end{entry}

\begin{entry}{瀑布}{18,5}{⽔、⼱}
  \begin{phonetics}{瀑布}{pu4bu4}
    \definition{s.}{queda de água | cachoeira | cascata | catarata}
  \end{phonetics}
\end{entry}

%%%%% EOF %%%%%

