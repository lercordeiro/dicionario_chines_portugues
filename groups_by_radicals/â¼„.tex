%%%
%%% Radical "⼄"
%%%

\section*{Radical 5: ``⼄'' (乚、乛、⺄)}\addcontentsline{toc}{section}{Radical 5: ⼄、乚、乛、⺄}

\begin{Entry}{乙}{1}{⼄}[Kangxi 5]
  \begin{Phonetics}{乙}{yi3}[][HSK 5]
    \definition*{s.}{Sobrenome Yi}
    \definition{num.}{segundo}
    \definition{s.}{o segundo lugar do Tian Gan | uma nota da escala em Gongchepu (工尺谱); nível superior na música tradicional chinesa}
  \seealsoref{工尺谱}{gong1 che3 pu3}
  \end{Phonetics}
\end{Entry}

\begin{Entry}{九}{2}{⼄}
  \begin{Phonetics}{九}{jiu3}[][HSK 1]
    \definition*{s.}{Sobrenome Jiu}
    \definition{adj.}{muitos; numerosos; indica várias vezes ou a maioria das vezes}
    \definition{num.}{nove; 9}
    \definition{s.}{cada um dos nove períodos de nove dias começando no dia seguinte ao solstício de inverno}
  \end{Phonetics}
\end{Entry}

\begin{Entry}{也}{3}{⼄}
  \begin{Phonetics}{也}{ye3}[][HSK 1]
    \definition*{s.}{Sobrenome Ye}
    \definition{adv.}{também; igualmente; assim como; da mesma forma; usado em frases simples, implica que é igual a outra coisa | assim como (expressar ênfase) | (expressar que as consequências são as mesmas) | também (expressar ufemismo; expressar um tom diplomático) | usado em frases compostas paralelas, indica que duas ou mais coisas têm algo em comum (pode ser usado em todas as frases ou apenas na última frase)}
    \definition{part.}{usado no meio de uma frase, destacando um elemento da frase sobre o qual deve ser feita uma afirmação | usado no final de uma frase, indicando uma explicação ou um julgamento; usado no final da frase, indica tom afirmativo e também pode reforçar o tom interrogativo, exclamativo ou imperativo}
  \end{Phonetics}
\end{Entry}

\begin{Entry}{也好}{3,6}{⼄、⼥}
  \begin{Phonetics}{也好}{ye3 hao3}[][HSK 5]
    \definition{part.}{pode não ser uma má ideia; também pode ser | (reduplicado) se\dots ou\dots; não importa se | pode não ser uma má ideia | se\dots ou\dots; usado em conjunto, significa que não está condicionado a uma determinada situação}
  \end{Phonetics}
\end{Entry}

\begin{Entry}{也有今天}{3,6,4,4}{⼄、⽉、⼈、⼤}
  \begin{Phonetics}{也有今天}{ye3you3jin1tian1}
    \definition{expr.}{obter apenas o que merece | todo cachorro tem seu dia | obter a sua parte (coisas boas ou ruins) | servir alguém bem}
  \end{Phonetics}
\end{Entry}

\begin{Entry}{也许}{3,6}{⼄、⾔}
  \begin{Phonetics}{也许}{ye3xu3}[][HSK 2]
    \definition{adv.}{talvez; provavelmente; estou com medo; para expressar incerteza; para expressar uma alta probabilidade}
  \end{Phonetics}
\end{Entry}

\begin{Entry}{也就是}{3,12,9}{⼄、⼪、⽇}
  \begin{Phonetics}{也就是}{ye3jiu4shi4}
    \definition{adv.}{i.e., isso é | ou seja}
  \end{Phonetics}
\end{Entry}

\begin{Entry}{也就是说}{3,12,9,9}{⼄、⼪、⽇、⾔}
  \begin{Phonetics}{也就是说}{ye3jiu4shi4shuo1}
    \definition{adv.}{em outras palavras | então | isto é | por isso}
  \end{Phonetics}
\end{Entry}

\begin{Entry}{习}{3}{⼄}
  \begin{Phonetics}{习}{xi2}
    \definition*{s.}{Sobrenome Xi}
    \definition{s.}{hábito; costume; prática usual; um comportamento que se desenvolve inconscientemente por meio de ações repetidas ao longo de um longo período de tempo}
    \definition{v.}{revisar; praticar; exercitar | acostumado a; familiarizado com; familiarizado com algo por meio de contato frequente | estudar; aprender (pássaro)}
  \end{Phonetics}
\end{Entry}

\begin{Entry}{习惯}{3,11}{⼄、⼼}
  \begin{Phonetics}{习惯}{xi2guan4}[][HSK 2]
    \definition[个,种]{s.}{hábito; costume; prática usual; comportamentos, tendências ou tendências sociais que se desenvolvem gradualmente ao longo de um longo período de tempo e são difíceis de mudar}
    \definition{v.}{estar acostumado a; ter o hábito de}
  \end{Phonetics}
\end{Entry}

\begin{Entry}{乡}{3}{⼄}
  \begin{Phonetics}{乡}{xiang1}[][HSK 5]
    \definition[个,座,片]{s.}{país; campo; vilarejo; área rural | local de origem; vila ou cidade natal | município (uma unidade administrativa rural subordinada ao condado) | vila natal; cidade natal | terra ou local famoso por produzir algo}
  \end{Phonetics}
\end{Entry}

\begin{Entry}{乡巴佬}{3,4,8}{⼄、⼰、⼈}
  \begin{Phonetics}{乡巴佬}{xiang1ba1lao3}
    \definition{s.}{aldeão | caipira}
  \end{Phonetics}
\end{Entry}

\begin{Entry}{乡村}{3,7}{⼄、⽊}
  \begin{Phonetics}{乡村}{xiang1 cun1}[][HSK 5]
    \definition{adj.}{rural | rústico}
    \definition[个]{s.}{vila; campo; área rural; principalmente envolvido na agricultura; áreas com distribuição populacional mais dispersa em relação às cidades}
  \end{Phonetics}
\end{Entry}

\begin{Entry}{书}{4}{⼄}
  \begin{Phonetics}{书}{shu1}[][HSK 1]
    \definition*{s.}{Sobrenome Shu}
    \definition[本,册,部,套,卷]{s.}{livro; obras encadernadas | carta; carta especial | documento | estilo de caligrafia; escrita}
    \definition{v.}{escrever; registrar}
  \end{Phonetics}
\end{Entry}

\begin{Entry}{书包}{4,5}{⼄、⼓}
  \begin{Phonetics}{书包}{shu1 bao1}[][HSK 1]
    \definition[个,款]{s.}{mochila para guardar livros e materiais escolares}
  \end{Phonetics}
\end{Entry}

\begin{Entry}{书记}{4,5}{⼄、⾔}
  \begin{Phonetics}{书记}{shu1ji5}
    \definition{s.}{secretário (chefe de um ramo de um partido socialista ou comunista) | atendente | balconista | escriturário}
  \end{Phonetics}
\end{Entry}

\begin{Entry}{书店}{4,8}{⼄、⼴}
  \begin{Phonetics}{书店}{shu1 dian4}[][HSK 1]
    \definition[个,家]{s.}{livraria; lojas que vendem livros}
  \end{Phonetics}
\end{Entry}

\begin{Entry}{书房}{4,8}{⼄、⼾}
  \begin{Phonetics}{书房}{shu1 fang2}[][HSK 6]
    \definition[间]{s.}{uma biblioteca (em uma residência privada); espaço para leitura e escrita}
  \end{Phonetics}
\end{Entry}

\begin{Entry}{书柜}{4,8}{⼄、⽊}
  \begin{Phonetics}{书柜}{shu1 gui4}[][HSK 5]
    \definition{s.}{estante; armário de livros}
  \end{Phonetics}
\end{Entry}

\begin{Entry}{书法}{4,8}{⼄、⽔}
  \begin{Phonetics}{书法}{shu1fa3}[][HSK 5]
    \definition[幅,卷,种,派]{s.}{caligrafia; arte de escrever caracteres, especialmente arte de escrever caracteres chineses com um pincel}
  \end{Phonetics}
\end{Entry}

\begin{Entry}{书架}{4,9}{⼄、⽊}
  \begin{Phonetics}{书架}{shu1jia4}[][HSK 3]
    \definition[个,种,套]{s.}{estante de livros}
  \end{Phonetics}
\end{Entry}

\begin{Entry}{书桌}{4,10}{⼄、⽊}
  \begin{Phonetics}{书桌}{shu1 zhuo1}[][HSK 5]
    \definition[个,张]{s.}{escrivaninha; mesa para ler e escrever}
  \end{Phonetics}
\end{Entry}

\begin{Entry}{买}{6}{⼄}
  \begin{Phonetics}{买}{mai3}[][HSK 1]
    \definition*{s.}{Sobrenome Mai}
    \definition{v.}{comprar; adquirir | comprar; subornar; usar dinheiro ou outros meios para angariar apoio| pedir; obter; trocar dinheiro por coisas}
  \end{Phonetics}
\end{Entry}

\begin{Entry}{买东西}{6,5,6}{⼄、⼀、⾑}
  \begin{Phonetics}{买东西}{mai3 dong1xi5}
    \definition{v.}{fazer compras; comprar bens ou serviços}
  \end{Phonetics}
\end{Entry}

\begin{Entry}{买卖}{6,8}{⼄、⼗}
  \begin{Phonetics}{买卖}{mai3 mai4}[][HSK 5]
    \definition[笔,桩,宗,家]{s.}{negócio; compra e venda; transação | Privado: loja; armazém}
  \end{Phonetics}
\end{Entry}

\begin{Entry}{乱}{7}{⼄}
  \begin{Phonetics}{乱}{luan4}[][HSK 3]
    \definition{adj.}{em desordem; em confusão; em desarrumação; sem ordem nem organização | em um estado mental confuso | (de uma sociedade) turbulento; agitado | (de relações sexuais) impróprio; promíscuo}
    \definition{adv.}{aleatoriamente; arbitrariamente; indiscriminadamente; sem restrições; à vontade}
    \definition{s.}{motim; agitação; tumulto; revolta; guerra; calamidade}
    \definition{v.}{confundir; embaralhar; misturar; causar desordem}
  \end{Phonetics}
\end{Entry}

\begin{Entry}{乳}{8}{⼄}
  \begin{Phonetics}{乳}{ru3}
    \definition{adj.}{recém-nascido (animal); lactente}
    \definition{s.}{mama; peito | leite (em geral) | qualquer líquido semelhante ao leite}
    \definition{v.}{dar à luz}
  \end{Phonetics}
\end{Entry}

\begin{Entry}{乳制品}{8,8,9}{⼄、⼑、⼝}
  \begin{Phonetics}{乳制品}{ru3 zhi4 pin3}[][HSK 6]
    \definition{s.}{produtos lácteos}
  \end{Phonetics}
\end{Entry}

\begin{Entry}{乳房}{8,8}{⼄、⼾}
  \begin{Phonetics}{乳房}{ru3fang2}
    \definition{s.}{seio | mama | úbere}
  \end{Phonetics}
\end{Entry}

%%%%% EOF %%%%%

