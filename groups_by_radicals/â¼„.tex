%%%
%%% Radical "⼄"
%%%

\section*{Radical 5: ``⼄'' (乚、乛、⺄)}\addcontentsline{toc}{section}{Radical 5: ⼄、乚、乛、⺄}

\begin{entry}{九}{2}{⼄}
  \begin{phonetics}{九}{jiu3}[][HSK 1]
    \definition{num.}{nove; 9}
  \end{phonetics}
\end{entry}

\begin{entry}{也}{3}{⼄}
  \begin{phonetics}{也}{ye3}[][HSK 1]
    \definition*{s.}{sobrenome Ye}
    \definition{adv.}{também | (em frases negativas) nem, tampouco}
  \end{phonetics}
\end{entry}

\begin{entry}{也有今天}{3,6,4,4}{⼄、⽉、⼈、⼤}
  \begin{phonetics}{也有今天}{ye3you3jin1tian1}
    \definition{expr.}{obter apenas o que merece | todo cachorro tem seu dia | obter a sua parte (coisas boas ou ruins) | servir alguém bem}
  \end{phonetics}
\end{entry}

\begin{entry}{也许}{3,6}{⼄、⾔}
  \begin{phonetics}{也许}{ye3xu3}[][HSK 2]
    \definition{adv.}{possivelmente | talvez}
  \end{phonetics}
\end{entry}

\begin{entry}{也就是}{3,12,9}{⼄、⼪、⽇}
  \begin{phonetics}{也就是}{ye3jiu4shi4}
    \definition{adv.}{i.e., isso é | ou seja}
  \end{phonetics}
\end{entry}

\begin{entry}{也就是说}{3,12,9,9}{⼄、⼪、⽇、⾔}
  \begin{phonetics}{也就是说}{ye3jiu4shi4shuo1}
    \definition{adv.}{em outras palavras | então | isto é | por isso}
  \end{phonetics}
\end{entry}

\begin{entry}{习惯}{3,11}{⼄、⼼}
  \begin{phonetics}{习惯}{xi2guan4}[][HSK 2]
    \definition[个]{s.}{hábito | costume | prática usual}
    \definition{v.}{ser acostumado a | ter o hábito de}
  \end{phonetics}
\end{entry}

\begin{entry}{乡巴佬}{3,4,8}{⼄、⼰、⼈}
  \begin{phonetics}{乡巴佬}{xiang1ba1lao3}
    \definition{s.}{aldeão | caipira}
  \end{phonetics}
\end{entry}

\begin{entry}{乡村}{3,7}{⼄、⽊}
  \begin{phonetics}{乡村}{xiang1cun1}
    \definition{adj.}{rural | rústico}
    \definition{s.}{vila | campo}
  \end{phonetics}
\end{entry}

\begin{entry}{书}{4}{⼄}
  \begin{phonetics}{书}{shu1}[][HSK 1]
    \definition[本,册,部]{s.}{livro | carta | documento}
  \end{phonetics}
\end{entry}

\begin{entry}{书包}{4,5}{⼄、⼓}
  \begin{phonetics}{书包}{shu1bao1}[][HSK 1]
    \definition[个,款]{s.}{mochila escolar}
  \end{phonetics}
\end{entry}

\begin{entry}{书记}{4,5}{⼄、⾔}
  \begin{phonetics}{书记}{shu1ji5}
    \definition{s.}{secretário (chefe de um ramo de um partido socialista ou comunista) | atendente | balconista | escriturário}
  \end{phonetics}
\end{entry}

\begin{entry}{书店}{4,8}{⼄、⼴}
  \begin{phonetics}{书店}{shu1dian4}[][HSK 1]
    \definition[家]{s.}{livraria}
  \end{phonetics}
\end{entry}

\begin{entry}{书架}{4,9}{⼄、⽊}
  \begin{phonetics}{书架}{shu1jia4}[][HSK 3]
    \definition[个]{s.}{estante de livros}
  \end{phonetics}
\end{entry}

\begin{entry}{买}{6}{⼄}
  \begin{phonetics}{买}{mai3}[][HSK 1]
    \definition{v.}{comprar}
  \end{phonetics}
\end{entry}

\begin{entry}{买东西}{6,5,6}{⼄、⼀、⾑}
  \begin{phonetics}{买东西}{mai3dong1xi5}
    \definition{v.}{fazer compras}
  \end{phonetics}
\end{entry}

\begin{entry}{乱}{7}{⼄}
  \begin{phonetics}{乱}{luan4}[][HSK 3]
    \definition{adj.}{bagunçado; confuso; desordenado | turbulento; perturbado (estado de espírito) | arbitrário; aleatório}
    \definition{adv.}{em confusão ou desordem; em um estado de espírito confuso}
    \definition{s.}{caos; tumulto; agitação; turbilhão | comportamento sexual promíscuo; promiscuidade}
    \definition{v.}{confundir; embaralhar; misturar}
  \end{phonetics}
\end{entry}

\begin{entry}{乳房}{8,8}{⼄、⼾}
  \begin{phonetics}{乳房}{ru3fang2}
    \definition{s.}{seio | mama | úbere}
  \end{phonetics}
\end{entry}

%%%%% EOF %%%%%

