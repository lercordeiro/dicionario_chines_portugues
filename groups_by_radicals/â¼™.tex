%%%
%%% Radical "⼙"
%%%

\section*{Radical 26: ``⼙'' (㔾)}\addcontentsline{toc}{section}{Radical 26: ⼙、㔾}

\begin{entry}{卫生}{3,5}{⼙、⽣}
  \begin{phonetics}{卫生}{wei4 sheng1}[][HSK 3]
    \definition{adj.}{bom para a saúde; higiênico}
    \definition{s.}{higiene; saneamento}
  \end{phonetics}
\end{entry}

\begin{entry}{卫生巾}{3,5,3}{⼙、⽣、⼱}
  \begin{phonetics}{卫生巾}{wei4sheng1jin1}
    \definition{s.}{absorvente higiênico}
  \end{phonetics}
\end{entry}

\begin{entry}{卫生厅}{3,5,4}{⼙、⽣、⼚}
  \begin{phonetics}{卫生厅}{wei4sheng1ting1}
    \definition*{s.}{Departamento de Saúde (da província)}
  \end{phonetics}
\end{entry}

\begin{entry}{卫生防疫}{3,5,6,9}{⼙、⽣、⾩、⽧}
  \begin{phonetics}{卫生防疫}{wei4sheng1 fang2yi4}
    \definition{s.}{prevenção contra a epidemia}
  \end{phonetics}
\end{entry}

\begin{entry}{卫生局}{3,5,7}{⼙、⽣、⼫}
  \begin{phonetics}{卫生局}{wei4sheng1ju2}
    \definition*{s.}{Departamento de Saúde | Escritório de Saúde}
  \end{phonetics}
\end{entry}

\begin{entry}{卫生纸}{3,5,7}{⼙、⽣、⽷}
  \begin{phonetics}{卫生纸}{wei4sheng1zhi3}
    \definition{s.}{papel higiênico}
  \end{phonetics}
\end{entry}

\begin{entry}{卫生间}{3,5,7}{⼙、⽣、⾨}
  \begin{phonetics}{卫生间}{wei4sheng1jian1}[][HSK 3]
    \definition[间,个]{s.}{banheiro; sanitário; \emph{toilette}}
  \end{phonetics}
\end{entry}

\begin{entry}{卫生套}{3,5,10}{⼙、⽣、⼤}
  \begin{phonetics}{卫生套}{wei4sheng1tao4}
    \definition[只]{s.}{preservativo | camisinha}
  \end{phonetics}
\end{entry}

\begin{entry}{卫生部}{3,5,10}{⼙、⽣、⾢}
  \begin{phonetics}{卫生部}{wei4sheng1bu4}
    \definition*{s.}{Ministério da Saúde}
  \end{phonetics}
\end{entry}

\begin{entry}{卫生球}{3,5,11}{⼙、⽣、⽟}
  \begin{phonetics}{卫生球}{wei4sheng1qiu2}
    \definition{s.}{naftalina}
  \end{phonetics}
\end{entry}

\begin{entry}{卫生棉}{3,5,12}{⼙、⽣、⽊}
  \begin{phonetics}{卫生棉}{wei4sheng1mian2}
    \definition{s.}{absorvente | algodão absorvente esterilizado (usado para curativos ou limpeza de feridas) | absorvente tampão}
  \end{phonetics}
\end{entry}

\begin{entry}{卫生署}{3,5,13}{⼙、⽣、⽹}
  \begin{phonetics}{卫生署}{wei4sheng1shu3}
    \definition*{s.}{Agência de Saúde (ou Escritório, ou Departamento)}
  \end{phonetics}
\end{entry}

\begin{entry}{卫星}{3,9}{⼙、⽇}
  \begin{phonetics}{卫星}{wei4xing1}[][HSK 5]
    \definition[个,颗]{s.}{satélite; lua; corpos celestes orbitando planetas | satélite artificial | algo que gira em torno de um centro}
  \end{phonetics}
\end{entry}

\begin{entry}{印刷}{5,8}{⼙、⼑}
  \begin{phonetics}{印刷}{yin4shua1}[][HSK 5]
    \definition{v.}{imprimir; imprimir textos, imagens, etc. em papel}
  \end{phonetics}
\end{entry}

\begin{entry}{印象}{5,11}{⼙、⾗}
  \begin{phonetics}{印象}{yin4xiang4}[][HSK 3]
    \definition[种]{s.}{impressão; marca; ideia; os vestígios deixados por coisas objetivas na mente das pessoas}
  \end{phonetics}
\end{entry}

\begin{entry}{危急}{6,9}{⼙、⼼}
  \begin{phonetics}{危急}{wei1ji2}
    \definition{adj.}{crítico | desesperadora (situação)}
  \end{phonetics}
\end{entry}

\begin{entry}{危险}{6,9}{⼙、⾩}
  \begin{phonetics}{危险}{wei1xian3}[][HSK 3]
    \definition{adj.}{arriscado; perigoso}
  \end{phonetics}
\end{entry}

\begin{entry}{危害}{6,10}{⼙、⼧}
  \begin{phonetics}{危害}{wei1hai4}[][HSK 3]
    \definition{s.}{prejuízo; perigo; dano}
    \definition{v.}{prejudicar; pôr em perigo; pôr em risco}
  \end{phonetics}
\end{entry}

\begin{entry}{危难}{6,10}{⼙、⾫}
  \begin{phonetics}{危难}{wei1nan4}
    \definition{s.}{calamidade}
  \end{phonetics}
\end{entry}

\begin{entry}{即}{7}{⼙}
  \begin{phonetics}{即}{ji2}
    \definition{conj.}{e | até | mesmo se/embora}
  \end{phonetics}
\end{entry}

\begin{entry}{即使}{7,8}{⼙、⼈}
  \begin{phonetics}{即使}{ji2shi3}[][HSK 5]
    \definition{conj.}{mesmo; mesmo que; mesmo se; apesar de; expressando uma concessão hipotética}
  \end{phonetics}
\end{entry}

\begin{entry}{即或}{7,8}{⼙、⼽}
  \begin{phonetics}{即或}{ji2huo4}
    \definition{conj.}{mesmo se/embora}
  \end{phonetics}
\end{entry}

\begin{entry}{即若}{7,8}{⼙、⾋}
  \begin{phonetics}{即若}{ji2ruo4}
    \definition{conj.}{mesmo se/embora}
  \end{phonetics}
\end{entry}

\begin{entry}{即便}{7,9}{⼙、⼈}
  \begin{phonetics}{即便}{ji2bian4}
    \definition{conj.}{mesmo se/embora}
  \end{phonetics}
\end{entry}

\begin{entry}{即将}{7,9}{⼙、⼨}
  \begin{phonetics}{即将}{ji2jiang1}[][HSK 4]
    \definition{adv.}{em breve; estar prestes a; estar a ponto de}
  \end{phonetics}
\end{entry}

\begin{entry}{即是}{7,9}{⼙、⽇}
  \begin{phonetics}{即是}{ji2shi4}
    \definition{conj.}{aquilo é}
  \end{phonetics}
\end{entry}

\begin{entry}{却}{7}{⼙}
  \begin{phonetics}{却}{que4}[][HSK 4]
    \definition{adv.}{mas; contudo; no entanto; enquanto; indica um ponto de virada}
    \definition{v.}{recuar; retroceder | afastar; repelir; desencorajar | declinar; recusar; rejeitar | usado depois de certos verbos para indicar a conclusão de uma ação}
  \end{phonetics}
\end{entry}

\begin{entry}{却是}{7,9}{⼙、⽇}
  \begin{phonetics}{却是}{que4shi4}
    \definition{conj.}{no entanto | realmente | o fato é\dots | mas isso é\dots}
  \end{phonetics}
\end{entry}

\begin{entry}{卷}{8}{⼙}
  \begin{phonetics}{卷}{juan3}[][HSK 4]
    \definition{clas.}{para pequenas coisas enroladas (maço de papel dinheiro, carretel de filme, etc.) | para rolos, carretéis, bobinas, etc.}
    \definition[张]{s.}{rolo; carretel; bobina}
    \definition{v.}{enrolar; dobrar algo em um cilindro ou semicírculo | varrer; carregar; levar junto | envolver-se; participar}
  \end{phonetics}
  \begin{phonetics}{卷}{juan4}[][HSK 4]
    \definition{clas.}{para capítulos, seções ou volumes; fascículos}
    \definition{s.}{livro; livros e pinturas que são enrolados para coleção; geralmente se refere a pinturas e caligrafia | papel de exame | arquivo; dossiê}
  \end{phonetics}
\end{entry}

%%%%% EOF %%%%%

