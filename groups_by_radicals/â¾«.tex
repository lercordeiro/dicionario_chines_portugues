%%%
%%% Radical "⾫"
%%%

\section*{Radical 172: ``⾫''}\addcontentsline{toc}{section}{Radical 172: ⾫}

\begin{entry}{难}{10}{⾫}
  \begin{phonetics}{难}{nan2}[][HSK 1]
    \definition{adj.}{difícil; duro; problemático (oposto a 易) | dificilmente possível; inevitável | ruim; desagradável | problemático; improvável}
    \definition{s.}{dificuldade}
    \definition{v.}{colocar alguém em uma situação difícil}
  \seealsoref{易}{yi4}
  \end{phonetics}
  \begin{phonetics}{难}{nan4}
    \definition{s.}{catástrofe; calamidade; desastre; adversidade; grande infortúnio}
    \definition{v.}{acusar; culpar}
  \end{phonetics}
\end{entry}

\begin{entry}{难以}{10,4}{⾫、⼈}
  \begin{phonetics}{难以}{nan2 yi3}[][HSK 5]
    \definition{adj.}{difícil; complicado}
  \end{phonetics}
\end{entry}

\begin{entry}{难过}{10,6}{⾫、⾡}
  \begin{phonetics}{难过}{nan2guo4}[][HSK 2]
    \definition{adj.}{triste; ruim; psicologicamente desconfortável | difícil; árduo}
  \end{phonetics}
\end{entry}

\begin{entry}{难免}{10,7}{⾫、⼉}
  \begin{phonetics}{难免}{nan4mian3}[][HSK 4]
    \definition{adj.}{inevitável; difícil de evitar}
  \end{phonetics}
\end{entry}

\begin{entry}{难听}{10,7}{⾫、⼝}
  \begin{phonetics}{难听}{nan2 ting1}[][HSK 2]
    \definition{adj.}{desagradável de ouvir | ofensivo; grosseiro; vulgar e desagradável | escandaloso; indigno}
  \end{phonetics}
\end{entry}

\begin{entry}{难受}{10,8}{⾫、⼜}
  \begin{phonetics}{难受}{nan2shou4}[][HSK 2]
    \definition{adj.}{sentir dor; sentir-se mal; sentir-se desconfortável | sentir-se mal; sentir-se infeliz; de mau humor; triste}
  \end{phonetics}
\end{entry}

\begin{entry}{难度}{10,9}{⾫、⼴}
  \begin{phonetics}{难度}{nan2 du4}[][HSK 3]
    \definition{s.}{dificuldade; grau de dificuldade}
  \end{phonetics}
\end{entry}

\begin{entry}{难看}{10,9}{⾫、⽬}
  \begin{phonetics}{难看}{nan2 kan4}[][HSK 2]
    \definition{adj.}{feio; desagradável à vista | vergonhoso; embaraçoso; desonroso; sem glória; sem dignidade}
  \end{phonetics}
\end{entry}

\begin{entry}{难得}{10,11}{⾫、⼻}
  \begin{phonetics}{难得}{nan2de2}[][HSK 5]
    \definition{adj.}{raro; difícil de encontrar; difícil de obter ou realizar, indicando que é valioso}
    \definition{adv.}{raramente; com pouca frequência}
  \end{phonetics}
\end{entry}

\begin{entry}{难道}{10,12}{⾫、⾡}
  \begin{phonetics}{难道}{nan2dao4}[][HSK 3]
    \definition{adv.}{certamente não significa que\dots?; é possível que\dots?; não me diga\dots; poderia ser que\dots?; usado em frases interrogativas para reforçar o tom interrogativo; frequentemente usado com palavras como "吗" e "不成".}
  \seealsoref{不成}{bu4 cheng2}
  \seealsoref{吗}{ma5}
  \end{phonetics}
\end{entry}

\begin{entry}{难题}{10,15}{⾫、⾴}
  \begin{phonetics}{难题}{nan2 ti2}[][HSK 2]
    \definition[个,道]{s.}{desafio; problema difícil; questão difícil; questões difíceis de responder ou resolver}
  \end{phonetics}
\end{entry}

\begin{entry}{雄伟}{12,6}{⾫、⼈}
  \begin{phonetics}{雄伟}{xiong2wei3}[][HSK 5]
    \definition{adj.}{magnífico; magnificente | imponente; magnífico}
  \end{phonetics}
\end{entry}

\begin{entry}{集中}{12,4}{⾫、⼁}
  \begin{phonetics}{集中}{ji2zhong1}[][HSK 3]
    \definition{adj.}{centralizado; concentrado}
    \definition{v.}{concentrar; centralizar; focar; acumular; reunir (oposto de 分散) | reunir pessoas, coisas, forças, etc. dispersas; resumir opiniões, experiências, etc.}
  \seealsoref{分散}{fen1san4}
  \end{phonetics}
\end{entry}

\begin{entry}{集合}{12,6}{⾫、⼝}
  \begin{phonetics}{集合}{ji2he2}[][HSK 4]
    \definition{v.}{reunir-se; juntar-se | reunir, juntar, convocar}
  \end{phonetics}
\end{entry}

\begin{entry}{集团}{12,6}{⾫、⼞}
  \begin{phonetics}{集团}{ji2tuan2}[][HSK 5]
    \definition[个]{s.}{anel; bloco; grupo; panelinha; círculo; grupo organizado para agir em conjunto com um determinado objetivo | grupo; entidade econômica com uma direção de negócios especializada, liderada por uma grande empresa com forte poder econômico e alta visibilidade, e formada pela combinação ou fusão de empresas relacionadas}
  \end{phonetics}
\end{entry}

\begin{entry}{集体}{12,7}{⾫、⼈}
  \begin{phonetics}{集体}{ji2ti3}[][HSK 3]
    \definition{s.}{coletivo; comunidade; grupo; equipe; organizações ou grupos em que muitas pessoas trabalham, estudam e vivem juntas}
  \end{phonetics}
\end{entry}

\begin{entry}{雕}{16}{⾫}
  \begin{phonetics}{雕}{diao1}
    \definition*{s.}{sobrenome Diao}
    \definition{s.}{abutre; águia | escultura ou obras esculpidas}
    \definition{v.}{esculpir; gravar}
  \end{phonetics}
\end{entry}

\begin{entry}{雕刻}{16,8}{⾫、⼑}
  \begin{phonetics}{雕刻}{diao1ke4}
    \definition{s.}{escultura}
    \definition{v.}{esculpir | gravar}
  \end{phonetics}
\end{entry}

%%%%% EOF %%%%%

