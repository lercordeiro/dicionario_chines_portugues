%%%
%%% Radical "⼸"
%%%

\section*{Radical 57: ``⼸''}\addcontentsline{toc}{section}{Radical 57: ⼸}

\begin{Entry}{引}{4}{⼸}
  \begin{Phonetics}{引}{yin3}[][HSK 4]
    \definition*{s.}{Sobrenome Yin}
    \definition{clas.}{uma unidade de comprimento (=33⅓ metros)}
    \definition{v.}{puxar; esticar | liderar; conduzir; guiar | sair; deixar | sobressair | atrair; provocar; trazer à existência | causar; provocar | citar; ser usado como evidência ou justificativa}
  \end{Phonetics}
\end{Entry}

\begin{Entry}{引导}{4,6}{⼸、⼨}
  \begin{Phonetics}{引导}{yin3dao3}[][HSK 4]
    \definition{v.}{conduzir; guiar; liderar; andar na frente e deixar que os outros sigam atrás para ver ou andar; usar imagens ou sinais para mostrar às pessoas para onde ir | esclarecer; fornecer orientação em termos de ideias, métodos, conceitos, etc.}
  \end{Phonetics}
\end{Entry}

\begin{Entry}{引进}{4,7}{⼸、⾡}
  \begin{Phonetics}{引进}{yin3 jin4}[][HSK 4]
    \definition{v.}{importar; trazer de fora | recomendar; dar uma indicação}
  \end{Phonetics}
\end{Entry}

\begin{Entry}{引起}{4,10}{⼸、⾛}
  \begin{Phonetics}{引起}{yin3qi3}[][HSK 4]
    \definition{v.}{causar; despertar; levar a; desencadear; dar origem a}
  \end{Phonetics}
\end{Entry}

\begin{Entry}{引擎}{4,16}{⼸、⼿}
  \begin{Phonetics}{引擎}{yin3qing2}
    \definition[台]{s.}{motor | (empréstimo linguístico) \emph{engine}}
  \end{Phonetics}
\end{Entry}

\begin{Entry}{弘}{5}{⼸}
  \begin{Phonetics}{弘}{hong2}
    \definition{adj.}{grande; grandioso; magnífico}
    \definition{s.}{Sobrenome Hong}
    \definition{v.}{ampliar; expandir | promover}
  \end{Phonetics}
\end{Entry}

\begin{Entry}{弟}{7}{⼸}
  \begin{Phonetics}{弟}{di4}[][HSK 1]
    \definition*{s.}{Sobrenome Di}
    \definition[个]{s.}{irmão mais novo | (entre amigos homens) eu | geralmente se refere a colegas do sexo masculino mais jovens na família ou entre parentes | forma humilde que os amigos usam para se referir uns aos outros, usada principalmente em correspondência}
  \end{Phonetics}
\end{Entry}

\begin{Entry}{弟弟}{7,7}{⼸、⼸}
  \begin{Phonetics}{弟弟}{di4 di5}[][HSK 1]
    \definition[个,位]{s.}{irmão mais novo | primo}
  \end{Phonetics}
\end{Entry}

\begin{Entry}{弟妹}{7,8}{⼸、⼥}
  \begin{Phonetics}{弟妹}{di4mei4}
    \definition{s.}{esposa do irmão mais novo}
  \end{Phonetics}
\end{Entry}

\begin{Entry}{张}{7}{⼸}
  \begin{Phonetics}{张}{zhang1}[][HSK 3]
    \definition*{s.}{Zhang, uma das vinte e oito constelações | Zhang, uma das mansões lunares | Sobrenome Zhang}
    \definition{adj.}{indulgente; desenfreado; devasso; libertino}
    \definition{clas.}{usado para papel, couro, etc. | usado para cama, mesa, etc. | usado para o rosto, boca, etc. | usado para arco}
    \definition{v.}{abrir; espalhar; esticar | expor; exibir | (de uma loja) iniciar atividades comerciais; abrir | olhar; contemplar | expandir; estender; ampliar; exagerar | fixar (uma corda de arco); encordoar (um instrumento musical); puxar a corda do arco}
  \end{Phonetics}
\end{Entry}

\begin{Entry}{张三}{7,3}{⼸、⼀}
  \begin{Phonetics}{张三}{zhang1san1}
    \definition*{s.}{Zhang San | Zé Ninguém | Nome para uma pessoa não especificada, 1 de 3}
  \seealsoref{李四}{li3si4}
  \seealsoref{王五}{wang2wu3}
  \end{Phonetics}
\end{Entry}

\begin{Entry}{张狂}{7,7}{⼸、⽝}
  \begin{Phonetics}{张狂}{zhang1kuang2}
    \definition{adj.}{impetuoso | frenético | insolente}
  \end{Phonetics}
\end{Entry}

\begin{Entry}{弯}{9}{⼸}
  \begin{Phonetics}{弯}{wan1}[][HSK 4]
    \definition{adj.}{curvo; tortuoso; torto | para algo curvo, como a lua, etc. | dobrado; flexível}
    \definition[个,道]{s.}{curva; dobra; volta}
    \definition{v.}{dobrar; flexionar; curvar | Literário: desenhar}
  \end{Phonetics}
\end{Entry}

\begin{Entry}{弯曲}{9,6}{⼸、⽈}
  \begin{Phonetics}{弯曲}{wan1 qu1}[][HSK 6]
    \definition{s.}{torto; curvo; sinuoso; tortuoso; não reto}
    \definition{v.}{dobrar; curvar; flexionar}
  \end{Phonetics}
\end{Entry}

\begin{Entry}{弱}{10}{⼸}
  \begin{Phonetics}{弱}{ruo4}[][HSK 4]
    \definition{adj.}{fraco; debilitado | jovem | inferior; pior | colocado depois de uma fração ou decimal para indicar que é um pouco menor que esse número (oposto a 强)}
    \definition{v.}{perder (através da morte)}
  \seealsoref{强}{qiang2}
  \end{Phonetics}
\end{Entry}

\begin{Entry}{弹}{11}{⼸}
  \begin{Phonetics}{弹}{dan4}
    \definition{s.}{bola; pelota; pequenas bolas disparadas com um estilingue | bomba; bala; explosivos que podem ser lançados ou arremessados, com poder destrutivo e letal}
  \end{Phonetics}
  \begin{Phonetics}{弹}{tan2}[][HSK 5]
    \definition{v.}{enviar; atirar (como com uma catapulta, etc.); usar a elasticidade de um objeto para lançar outro objeto | afofar; preparar fibras; usar um dispositivo elástico para amolecer as fibras | virar; sacudir | dedilhar; tocar (um instrumento musical de cordas) | acusar; atacar; criticar; relatar | saltar; quicar}
  \end{Phonetics}
\end{Entry}

\begin{Entry}{强}{12}{⼸}
  \begin{Phonetics}{强}{jiang4}
    \definition{adj.}{teimoso; inflexível}
  \end{Phonetics}
  \begin{Phonetics}{强}{qiang2}[][HSK 3]
    \definition*{s.}{Sobrenome Qiang}
    \definition{adj.}{forte; poderoso  (em oposição a 弱) | melhor; superior | mais; extra; adicional; um pouco mais que; usado após uma fração ou decimal para indicar que é um pouco maior que o número | resoluto; firme | violento | alto padrão}
    \definition{v.}{fortalecer; tornar forte; tornar poderoso}
  \seealsoref{弱}{ruo4}
  \end{Phonetics}
  \begin{Phonetics}{强}{qiang3}
    \definition{v.}{fazer um esforço; esforçar-se}
  \end{Phonetics}
\end{Entry}

\begin{Entry}{强大}{12,3}{⼸、⼤}
  \begin{Phonetics}{强大}{qiang2 da4}[][HSK 3]
    \definition{adj.}{forte; poderoso; potente; possante; descreve força forte e grande poder}
  \end{Phonetics}
\end{Entry}

\begin{Entry}{强化}{12,4}{⼸、⼔}
  \begin{Phonetics}{强化}{qiang2 hua4}[][HSK 6]
    \definition{v.}{intensificar; fortalecer; consolidar; tornar mais forte, melhorar sua habilidade e nível}
  \end{Phonetics}
\end{Entry}

\begin{Entry}{强壮}{12,6}{⼸、⼠}
  \begin{Phonetics}{强壮}{qiang2 zhuang4}[][HSK 6]
    \definition{s.}{(corpo) forte, poderoso, robusto, resistente}
    \definition{v.}{fortalecer; construir}
  \end{Phonetics}
\end{Entry}

\begin{Entry}{强势}{12,8}{⼸、⼒}
  \begin{Phonetics}{强势}{qiang2 shi4}[][HSK 6]
    \definition*{adj.}{forte; poderoso; dominante}
    \definition{s.}{momento; ímpeto; grande impulso; forte impulso | força; influência dominante; forças poderosas}
  \end{Phonetics}
\end{Entry}

\begin{Entry}{强迫}{12,8}{⼸、⾡}
  \begin{Phonetics}{强迫}{qiang3po4}[][HSK 5]
    \definition{v.}{impelir; forçar; impor; compelir; aplicar pessão para obedecer}
  \end{Phonetics}
\end{Entry}

\begin{Entry}{强度}{12,9}{⼸、⼴}
  \begin{Phonetics}{强度}{qiang2 du4}[][HSK 5]
    \definition[个,种]{s.}{intensidade; força | magnitude; rigor; avidez}
  \end{Phonetics}
\end{Entry}

\begin{Entry}{强烈}{12,10}{⼸、⽕}
  \begin{Phonetics}{强烈}{qiang2lie4}[][HSK 3]
    \definition{adj.}{muito forte; intenso; poderoso | violento; impetuoso; nível muito alto; atitude muito firme, sem espaço para mudanças | afiado; marcante; mostrado em contraste; muito claro}
  \end{Phonetics}
\end{Entry}

\begin{Entry}{强调}{12,10}{⼸、⾔}
  \begin{Phonetics}{强调}{qiang2diao4}[][HSK 3]
    \definition{v.}{salientar; sublinhar; enfatizar; dar ênfase a; vincar}
  \end{Phonetics}
\end{Entry}

\begin{Entry}{强盗}{12,11}{⼸、⽫}
  \begin{Phonetics}{强盗}{qiang2 dao4}[][HSK 6]
    \definition[个,群,伙,帮]{s.}{ladrão; bandido; uma pessoa que usa violência para confiscar a propriedade de outros; também se refere a uma pessoa ou força que se envolve em comportamento semelhante}
  \end{Phonetics}
\end{Entry}

\begin{Entry}{彀}{13}{⼸}
  \begin{Phonetics}{彀}{gou4}
    \definition{adj.}{suficiente; adequado}
    \definition{v.}{puxar um arco ao máximo}
  \end{Phonetics}
\end{Entry}

%%%%% EOF %%%%%

