%%%
%%% Radical "⾭"
%%%

\section*{Radical 174: ``⾭'' (青)}\addcontentsline{toc}{section}{Radical 174: ⾭、青}

\begin{entry}{青}{8}{⾭}
  \begin{phonetics}{青}{qing1}[][HSK 5]
    \definition*{s.}{sobrenome Qing}
    \definition*{s.}{abreviação de Província de Qinghai}
    \definition{adj.}{azul ou verde | preto | jovens (pessoas)}
    \definition{s.}{grama verde | colheitas jovens (não maduras) | tiras de bambu verde}
  \end{phonetics}
\end{entry}

\begin{entry}{青天}{8,4}{⾭、⼤}
  \begin{phonetics}{青天}{qing1tian1}
    \definition{s.}{céu claro, limpo ou azul}
  \end{phonetics}
\end{entry}

\begin{entry}{青少年}{8,4,6}{⾭、⼩、⼲}
  \begin{phonetics}{青少年}{qing1shao4nian2}[][HSK 2]
    \definition[位,名,个,些]{s.}{adolescentes}
  \end{phonetics}
\end{entry}

\begin{entry}{青玉米}{8,5,6}{⾭、⽟、⽶}
  \begin{phonetics}{青玉米}{qing1yu4mi3}
    \definition{s.}{milho verde}
  \end{phonetics}
\end{entry}

\begin{entry}{青年}{8,6}{⾭、⼲}
  \begin{phonetics}{青年}{qing1 nian2}[][HSK 2]
    \definition[个,位,名,些]{s.}{juventude; jovem; refere-se ao período entre os 15 e os 30 anos de idade.}
  \end{phonetics}
\end{entry}

\begin{entry}{青年节}{8,6,5}{⾭、⼲、⾋}
  \begin{phonetics}{青年节}{qing1nian2jie2}
    \definition*{s.}{Dia da Juventude (4 de maio)}
  \end{phonetics}
\end{entry}

\begin{entry}{青春}{8,9}{⾭、⽇}
  \begin{phonetics}{青春}{qing1chun1}[][HSK 4]
    \definition[个]{s.}{juventude; jovialidade}
  \end{phonetics}
\end{entry}

\begin{entry}{青菜}{8,11}{⾭、⾋}
  \begin{phonetics}{青菜}{qing1cai4}
    \definition{s.}{verduras}
  \end{phonetics}
\end{entry}

\begin{entry}{青铜}{8,11}{⾭、⾦}
  \begin{phonetics}{青铜}{qing1tong2}
    \definition{s.}{bronze (liga de cobre, 銅, e estanho, 锡)}
  \end{phonetics}
\end{entry}

\begin{entry}{青椒}{8,12}{⾭、⽊}
  \begin{phonetics}{青椒}{qing1jiao1}
    \definition{s.}{pimenta verde}
  \end{phonetics}
\end{entry}

\begin{entry}{青蛙}{8,12}{⾭、⾍}
  \begin{phonetics}{青蛙}{qing1wa1}
    \definition{adj.}{(gíria velha) cara feio}
    \definition[只]{s.}{sapo}
  \end{phonetics}
\end{entry}

\begin{entry}{静}{14}{⾭}
  \begin{phonetics}{静}{jing4}[][HSK 3]
    \definition*{s.}{sobrenome Jing}
    \definition{adj.}{tranquilo;  sossegado; calmo; imóvel | silencioso; quieto; sem emitir nenhum som | calmo, sereno; serenidade; (interior) paz}
    \definition{v.}{acalmar-se; aquietar-se; tranquilizar (o coração)}
  \end{phonetics}
\end{entry}

%%%%% EOF %%%%%

