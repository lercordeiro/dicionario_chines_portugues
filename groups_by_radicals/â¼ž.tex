%%%
%%% Radical "⼞"
%%%

\section*{Radical 31: ``⼞''}\addcontentsline{toc}{section}{Radical 31: ⼞}

\begin{entry}{囘}{5}{⼞}
  \begin{phonetics}{囘}{hui2}
    \variantof{回}
  \end{phonetics}
\end{entry}

\begin{entry}{四}{5}{⼞}
  \begin{phonetics}{四}{si4}[][HSK 1]
    \definition{num.}{quatro; 4}
  \end{phonetics}
\end{entry}

\begin{entry}{四川}{5,3}{⼞、⼮}
  \begin{phonetics}{四川}{si4chuan1}
    \definition*{s.}{Sichuan}
  \end{phonetics}
\end{entry}

\begin{entry}{四季分明}{5,8,4,8}{⼞、⼦、⼑、⽇}
  \begin{phonetics}{四季分明}{si4ji4-fen1ming2}
    \definition{expr.}{as quatro estações são muito distintas}
  \end{phonetics}
\end{entry}

\begin{entry}{四季如春}{5,8,6,9}{⼞、⼦、⼥、⽇}
  \begin{phonetics}{四季如春}{si4ji4-ru2chun1}
    \definition{expr.}{é primavera todo o ano | clima favorável durante todo o ano | quatro estações como a primavera}
  \end{phonetics}
\end{entry}

\begin{entry}{回}{6}{⼞}
  \begin{phonetics}{回}{hui2}[][HSK 1,2]
    \definition{clas.}{de atos de uma peça de teatro}
    \definition{s.}{seção ou capítulo (de um livro clássico) | grupo étnico Hui (mulçumanos chineses)}
    \definition{v.}{regressar | voltar | dar a volta | responder | resolver | circular | curvar}
  \end{phonetics}
\end{entry}

\begin{entry}{回去}{6,5}{⼞、⼛}
  \begin{phonetics}{回去}{hui2 qu4}[][HSK 1]
    \definition{v.}{regressar | voltar | estar de volta | (a partir da minha localização)}
  \end{phonetics}
\end{entry}

\begin{entry}{回来}{6,7}{⼞、⽊}
  \begin{phonetics}{回来}{hui2 lai5}[][HSK 1]
    \definition{v.}{regressar | voltar | estar de volta | (para a minha localização)}
  \end{phonetics}
\end{entry}

\begin{entry}{回到}{6,8}{⼞、⼑}
  \begin{phonetics}{回到}{hui2 dao4}[][HSK 1]
    \definition{v.}{retornar a}
  \end{phonetics}
\end{entry}

\begin{entry}{回国}{6,8}{⼞、⼞}
  \begin{phonetics}{回国}{hui2 guo2}[][HSK 2]
    \definition{v.}{retornar ao seu país (terra natal)}
  \end{phonetics}
\end{entry}

\begin{entry}{回信}{6,9}{⼞、⼈}
  \begin{phonetics}{回信}{hui2xin4}
    \definition{s.}{uma carta em resposta | uma mensagem verbal em resposta}
    \definition{v.+compl.}{escrever em resposta | escrever de volta | responder uma carta | responder verbalmente uma mensagem}
  \end{phonetics}
\end{entry}

\begin{entry}{回复}{6,9}{⼞、⼢}
  \begin{phonetics}{回复}{hui2 fu4}[][HSK 4]
    \definition{v.}{responder (a uma carta) | retornar ao estado normal; restaurar algo ao seu estado original}
  \end{phonetics}
\end{entry}

\begin{entry}{回家}{6,10}{⼞、⼧}
  \begin{phonetics}{回家}{hui2 jia1}[][HSK 1]
    \definition{v.}{ir (voltar) para casa | estar em casa | retornar para casa}
  \end{phonetics}
\end{entry}

\begin{entry}{回旋}{6,11}{⼞、⽅}
  \begin{phonetics}{回旋}{hui2xuan2}
    \definition{v.}{circular | rodar | dar a volta}
  \end{phonetics}
\end{entry}

\begin{entry}{回答}{6,12}{⼞、⽵}
  \begin{phonetics}{回答}{hui2da2}[][HSK 1]
    \definition{v.}{responder}
  \end{phonetics}
\end{entry}

\begin{entry}{因为}{6,4}{⼞、⼂}
  \begin{phonetics}{因为}{yin1wei4}[][HSK 2]
    \definition{conj.}{porque | devido a | por conta de}
  \end{phonetics}
\end{entry}

\begin{entry}{因为……所以……}{6,4,8,4}{⼞、⼂、⼾、⼈}
  \begin{phonetics}{因为……所以……}{yin1wei4 suo3yi3}[][HSK 2]
    \definition{conj.}{porque\dots portanto\dots}
  \end{phonetics}
\end{entry}

\begin{entry}{因此}{6,6}{⼞、⽌}
  \begin{phonetics}{因此}{yin1ci3}[][HSK 3]
    \definition{conj.}{assim; por isso; portanto; consequentemente}
  \end{phonetics}
\end{entry}

\begin{entry}{因此就}{6,6,12}{⼞、⽌、⼪}
  \begin{phonetics}{因此就}{yin1ci3 jiu4}
    \definition{conj.}{portanto}
  \end{phonetics}
\end{entry}

\begin{entry}{因而}{6,6}{⼞、⽽}
  \begin{phonetics}{因而}{yin1'er2}
    \definition{conj.}{então | portanto | por esta razão | consequentemente}
  \end{phonetics}
\end{entry}

\begin{entry}{团}{6}{⼞}
  \begin{phonetics}{团}{tuan2}[][HSK 3]
    \definition*{s.}{Liga da Juventude Comunista da China; Liga}
    \definition{adj.}{redondo; circular | coletivo}
    \definition{clas.}{coisas usadas para formar um grupo}
    \definition[个]{s.}{bolinho de massa | algo em forma de bola | grupo; corpo; sociedade; organização | regimento}
    \definition{v.}{enrolar algo para formar uma bola; rolar | reunir; unir; conglomerar}
  \end{phonetics}
\end{entry}

\begin{entry}{团队}{6,4}{⼞、⾩}
  \begin{phonetics}{团队}{tuan2dui4}
    \definition{s.}{equipe}
  \end{phonetics}
\end{entry}

\begin{entry}{团体}{6,7}{⼞、⼈}
  \begin{phonetics}{团体}{tuan2ti3}[][HSK 3]
    \definition[个]{s.}{equipe; grupo; organização}
  \end{phonetics}
\end{entry}

\begin{entry}{团结}{6,9}{⼞、⽷}
  \begin{phonetics}{团结}{tuan2jie2}[][HSK 3]
    \definition{adj.}{unido; amigável; harmonioso}
    \definition{v.}{unir; reunir}
  \end{phonetics}
\end{entry}

\begin{entry}{囯}{7}{⼞}
  \begin{phonetics}{囯}{guo2}
    \variantof{国}
  \end{phonetics}
\end{entry}

\begin{entry}{困}{7}{⼞}
  \begin{phonetics}{困}{kun4}[][HSK 3]
    \definition{adj.}{cansado | sonolento}
    \definition{v.}{estar encalhado; estar em grande pressão | cercar; prender; sitiar; cercar; rodear}
  \end{phonetics}
\end{entry}

\begin{entry}{困难}{7,10}{⼞、⾫}
  \begin{phonetics}{困难}{kun4nan5}[][HSK 3]
    \definition{adj.}{dificuldades financeiras; circunstâncias difíceis | complicado; nodoso; difícil; duro;}
    \definition[种]{s.}{dificuldade; situação difícil}
  \end{phonetics}
\end{entry}

\begin{entry}{围}{7}{⼞}
  \begin{phonetics}{围}{wei2}[][HSK 3]
    \definition*{s.}{sobrenome Wei}
    \definition{clas.}{o comprimento dos dois polegares e indicadores ou o comprimento de ambos os braços quando unidos}
    \definition{s.}{em volta de tudo; ao redor}
    \definition{v.}{cercar; rodear; circundar; encurralar | enrolar; envolver}
  \end{phonetics}
\end{entry}

\begin{entry}{围巾}{7,3}{⼞、⼱}
  \begin{phonetics}{围巾}{wei2jin1}[][HSK 4]
    \definition[条]{s.}{lenço; cachecol; echarpe; gravata; tiras longas de malha ou tecido usadas ao redor do pescoço para aquecimento, proteção do colarinho ou decoração}
  \end{phonetics}
\end{entry}

\begin{entry}{固定}{8,8}{⼞、⼧}
  \begin{phonetics}{固定}{gu4ding4}[][HSK 4]
    \definition{adj.}{fixo; regular; inalterado ou imóvel}
    \definition{v.}{consertar; tornar fixo, não mover novamente; colocar as coisas em ordem, não mudá-las novamente}
  \end{phonetics}
\end{entry}

\begin{entry}{国}{8}{⼞}
  \begin{phonetics}{国}{guo2}[][HSK 1]
    \definition*{s.}{sobrenome Guo}
    \definition[个]{s.}{país | nação}
  \end{phonetics}
\end{entry}

\begin{entry}{国人}{8,2}{⼞、⼈}
  \begin{phonetics}{国人}{guo2ren2}
    \definition{s.}{compatriota}
  \end{phonetics}
\end{entry}

\begin{entry}{国内}{8,4}{⼞、⼌}
  \begin{phonetics}{国内}{guo2 nei4}[][HSK 3]
    \definition{s.}{interno (a um país); doméstico; lar}
  \end{phonetics}
\end{entry}

\begin{entry}{国外}{8,5}{⼞、⼣}
  \begin{phonetics}{国外}{guo2 wai4}[][HSK 1]
    \definition{adj.}{no exterior | externo (assuntos) | estrangeiro}
  \end{phonetics}
\end{entry}

\begin{entry}{国庆}{8,6}{⼞、⼴}
  \begin{phonetics}{国庆}{guo2 qing4}[][HSK 3]
    \definition*{s.}{Dia Nacional}
  \end{phonetics}
\end{entry}

\begin{entry}{国庆节}{8,6,5}{⼞、⼴、⾋}
  \begin{phonetics}{国庆节}{guo2qing4jie2}
    \definition*{s.}{Dia Nacional (1~de~outubro)}
  \end{phonetics}
\end{entry}

\begin{entry}{国际}{8,7}{⼞、⾩}
  \begin{phonetics}{国际}{guo2ji4}[][HSK 2]
    \definition{adj.}{internacional}
  \end{phonetics}
\end{entry}

\begin{entry}{国际儿童节}{8,7,2,12,5}{⼞、⾩、⼉、⽴、⾋}
  \begin{phonetics}{国际儿童节}{guo2ji4 er2tong2jie2}
    \definition*{s.}{Dia Internacional das Crianças (1~de~junho)}
  \end{phonetics}
\end{entry}

\begin{entry}{国际妇女节}{8,7,6,3,5}{⼞、⾩、⼥、⼥、⾋}
  \begin{phonetics}{国际妇女节}{guo2ji4 fu4nv3jie2}
    \definition*{s.}{Dia Internacional das Mulheres (8~de~março)}
  \end{phonetics}
\end{entry}

\begin{entry}{国际劳动节}{8,7,7,6,5}{⼞、⾩、⼒、⼒、⾋}
  \begin{phonetics}{国际劳动节}{guo2ji4 lao2dong4 jie2}
    \definition*{s.}{Dia Internacional dos Trabalhadores (1~de~maio)}
  \end{phonetics}
\end{entry}

\begin{entry}{国语}{8,9}{⼞、⾔}
  \begin{phonetics}{国语}{guo2yu3}
    \definition*{s.}{Língua Chinesa (Mandarim), enfatizando sua natureza nacional}
  \end{phonetics}
\end{entry}

\begin{entry}{国家}{8,10}{⼞、⼧}
  \begin{phonetics}{国家}{guo2jia1}[][HSK 1]
    \definition[个]{s.}{país | nação | estado}
  \end{phonetics}
\end{entry}

\begin{entry}{国宾馆}{8,10,11}{⼞、⼧、⾷}
  \begin{phonetics}{国宾馆}{guo2bin1guan3}
    \definition{s.}{pousada estadual}
  \end{phonetics}
\end{entry}

\begin{entry}{国旗}{8,14}{⼞、⽅}
  \begin{phonetics}{国旗}{guo2qi2}
    \definition[面]{s.}{bandeira (de um país)}
  \end{phonetics}
\end{entry}

\begin{entry}{国歌}{8,14}{⼞、⽋}
  \begin{phonetics}{国歌}{guo2ge1}
    \definition{s.}{hino nacional}
  \end{phonetics}
\end{entry}

\begin{entry}{图}{8}{⼞}
  \begin{phonetics}{图}{tu2}[][HSK 3]
    \definition*{s.}{sobrenome Tu}
    \definition[张]{s.}{mapa; gráfico; imagem; desenho | plano; esquema; tentativa}
    \definition{v.}{procurar; perseguir | desenhar; pintar | inventar; planejar; tentar}
  \end{phonetics}
\end{entry}

\begin{entry}{图书馆}{8,4,11}{⼞、⼄、⾷}
  \begin{phonetics}{图书馆}{tu2shu1guan3}[][HSK 1]
    \definition[家,个]{s.}{biblioteca}
  \end{phonetics}
\end{entry}

\begin{entry}{图片}{8,4}{⼞、⽚}
  \begin{phonetics}{图片}{tu2 pian4}[][HSK 2]
    \definition[张,幅]{s.}{imagem | fotografia}
  \end{phonetics}
\end{entry}

\begin{entry}{图画}{8,8}{⼞、⽥}
  \begin{phonetics}{图画}{tu2 hua4}[][HSK 3]
    \definition{s.}{desenho; imagem; pintura}
  \end{phonetics}
\end{entry}

\begin{entry}{图案}{8,10}{⼞、⽊}
  \begin{phonetics}{图案}{tu2'an4}[][HSK 4]
    \definition{s.}{padrão; desenho; padrões e gráficos usados para decoração de edifícios, tecidos, artes e artesanato, etc.}
  \end{phonetics}
\end{entry}

\begin{entry}{圆}{10}{⼞}
  \begin{phonetics}{圆}{yuan2}[][HSK 4]
    \definition*{s.}{sobrenome Yuan}
    \definition{adj.}{redondo; circular; esférico; arredondado | diplomático; satisfatório}
    \definition[个]{s.}{círculo; circunferência | uma moeda de valor e peso fixos}
    \definition{v.}{tornar plausível; justificar; tornar completo; completar}
  \end{phonetics}
\end{entry}

\begin{entry}{圆满}{10,13}{⼞、⽔}
  \begin{phonetics}{圆满}{yuan2man3}[][HSK 4]
    \definition{adj.}{perfeito; satisfatório; sem defeitos}
  \end{phonetics}
\end{entry}

\begin{entry}{圈}{11}{⼞}
  \begin{phonetics}{圈}{juan1}
    \definition{v.}{prender aves e animais de criação | prender; colocar na cadeia, prisão | confinar}
  \end{phonetics}
  \begin{phonetics}{圈}{juan4}
    \definition*{s.}{sobrenome Juan}
    \definition{s.}{curral; local onde o gado ou as aves são mantidos, geralmente cercado ou murado, alguns com galpões}
  \end{phonetics}
  \begin{phonetics}{圈}{quan1}[][HSK 4]
    \definition[个]{s.}{anel; círculo; refere-se a algo em forma de anel | domínio; grupo; escopo; círculo(s)}
    \definition{v.}{cercar; rodear; circundar | marcar com um círculo}
  \end{phonetics}
\end{entry}

\begin{entry}{圈粉}{11,10}{⼞、⽶}
  \begin{phonetics}{圈粉}{quan1fen3}
    \definition{s.}{(neologismo, coloquial) ganhar alguém como fã, obter novos fãs}
  \end{phonetics}
\end{entry}

%%%%% EOF %%%%%

