%%%
%%% Radical "⼞"
%%%

\section*{Radical 31: ``⼞''}\addcontentsline{toc}{section}{Radical 31: ⼞}

\begin{Entry}{囘}{5}{⼞}
  \begin{Phonetics}{囘}{hui2}
    \variantof{回}
  \end{Phonetics}
\end{Entry}

\begin{Entry}{囚}{5}{⼞}
  \begin{Phonetics}{囚}{qiu2}
    \definition[个,群,位,名,些,批]{s.}{prisioneiro; condenado}
    \definition{v.}{aprisionar}
  \end{Phonetics}
\end{Entry}

\begin{Entry}{囚犯}{5,5}{⼞、⽝}
  \begin{Phonetics}{囚犯}{qiu2fan4}
    \definition[名]{s.}{prisioneiro; condenado}
  \end{Phonetics}
\end{Entry}

\begin{Entry}{四}{5}{⼞}
  \begin{Phonetics}{四}{si4}[][HSK 1]
    \definition*{s.}{Sobrenome Si}
    \definition{num.}{quatro; 4}
    \definition{s.}{uma nota da escala em Gongchepu (工尺谱), correspondente ao 6 na notação musical numerada}
  \seealsoref{工尺谱}{gong1 che3 pu3}
  \end{Phonetics}
\end{Entry}

\begin{Entry}{四川}{5,3}{⼞、⼮}
  \begin{Phonetics}{四川}{si4chuan1}
    \definition*{s.}{Província de Sichuan}
  \end{Phonetics}
\end{Entry}

\begin{Entry}{四处}{5,5}{⼞、⼡}
  \begin{Phonetics}{四处}{si4 chu4}[][HSK 6]
    \definition{adv.}{em volta; ao redor; em todos os lugares; em todas as direções}
  \end{Phonetics}
\end{Entry}

\begin{Entry}{四周}{5,8}{⼞、⼝}
  \begin{Phonetics}{四周}{si4 zhou1}[][HSK 5]
    \definition{s.}{ao redor; por todos os lados; a parte que circunda o centro}
  \end{Phonetics}
\end{Entry}

\begin{Entry}{四季分明}{5,8,4,8}{⼞、⼦、⼑、⽇}
  \begin{Phonetics}{四季分明}{si4ji4-fen1ming2}
    \definition{expr.}{as quatro estações são muito distintas}
  \end{Phonetics}
\end{Entry}

\begin{Entry}{四季如春}{5,8,6,9}{⼞、⼦、⼥、⽇}
  \begin{Phonetics}{四季如春}{si4ji4-ru2chun1}
    \definition{expr.}{é primavera todo o ano | clima favorável durante todo o ano | quatro estações como a primavera}
  \end{Phonetics}
\end{Entry}

\begin{Entry}{回}{6}{⼞}
  \begin{Phonetics}{回}{hui2}[][HSK 1,2]
    \definition*{s.}{Sobrenome Hui}
    \definition*{s.}{Etnia Hui (mulçumanos chineses)}
    \definition{clas.}{usado para coisas, ações, número de vezes |  um trecho de um conto; um capítulo de um romance em capítulos | seção ou capítulo (de um livro clássico)}
    \definition{v.}{circular; enrolar | retornar; voltar; voltar ao lugar de origem | dar meia-volta | responder; contestar | relatar; reportar; responder}
  \end{Phonetics}
\end{Entry}

\begin{Entry}{回忆}{6,4}{⼞、⼼}
  \begin{Phonetics}{回忆}{hui2yi4}[][HSK 5]
    \definition[个,段]{s.}{memória; lembrança de eventos ou experiências passadas}
    \definition{v.}{lembrar; recordar}
  \end{Phonetics}
\end{Entry}

\begin{Entry}{回去}{6,5}{⼞、⼛}
  \begin{Phonetics}{回去}{hui2 qu4}[][HSK 1]
    \definition{v.}{retornar; voltar; estar de volta ; (a partir da minha localização)}
  \end{Phonetics}
\end{Entry}

\begin{Entry}{回头}{6,5}{⼞、⼤}
  \begin{Phonetics}{回头}{hui2 tou2}[][HSK 5]
    \definition{adv.}{mais tarde; depois de um tempo}
    \definition{conj.}{ou então; usado no início da segunda metade de uma frase para indicar o que acontecerá se você não fizer o que fez na primeira metade da frase}
    \definition{v.}{dar a meia-volta; virar a cabeça; virar a cabeça para trás | retornar; voltar | arrepender-se; corrigir seu caminho; reconhecer e corrigir erros}
  \end{Phonetics}
\end{Entry}

\begin{Entry}{回收}{6,6}{⼞、⽁}
  \begin{Phonetics}{回收}{hui2shou1}[][HSK 5]
    \definition{v.}{reciclar; reciclar itens (geralmente resíduos ou produtos antigos) para reutilização | recuperar; recolher; recuperar o que foi emitido ou demitido}
  \end{Phonetics}
\end{Entry}

\begin{Entry}{回应}{6,7}{⼞、⼴}
  \begin{Phonetics}{回应}{hui2 ying4}[][HSK 6]
    \definition{v.}{responder}
  \end{Phonetics}
\end{Entry}

\begin{Entry}{回报}{6,7}{⼞、⼿}
  \begin{Phonetics}{回报}{hui2bao4}[][HSK 5]
    \definition{s.}{recompensa; pagamento; benefícios recebidos como resultado de assistência, esforço ou afeto | retornos; benefícios recebidos por meio de investimentos}
    \definition{v.}{pagar de volta; beneficiar pessoas ou organizações que os ajudaram ou cuidaram deles de alguma forma}
  \end{Phonetics}
\end{Entry}

\begin{Entry}{回来}{6,7}{⼞、⽊}
  \begin{Phonetics}{回来}{hui2 lai5}[][HSK 1]
    \definition{v.}{voltar; regressar (para a minha localização) | retornar; usado após um verbo, significa ``vir ao lugar original''}
  \end{Phonetics}
\end{Entry}

\begin{Entry}{回到}{6,8}{⼞、⼑}
  \begin{Phonetics}{回到}{hui2 dao4}[][HSK 1]
    \definition{v.}{retornar para; voltar e chegar (ao lugar onde estava originalmente); (após uma mudança nas circunstâncias) retornar ao estado original}
  \end{Phonetics}
\end{Entry}

\begin{Entry}{回国}{6,8}{⼞、⼞}
  \begin{Phonetics}{回国}{hui2 guo2}[][HSK 2]
    \definition{v.}{regressar ao seu país (terra natal); referindo-se a voltar do exterior}
  \end{Phonetics}
\end{Entry}

\begin{Entry}{回信}{6,9}{⼞、⼈}
  \begin{Phonetics}{回信}{hui2/xin4}[][HSK 5]
    \definition[封]{s.}{uma carta em resposta; uma mensagem verbal em resposta}
    \definition{v.+compl.}{escrever em resposta; escrever de volta; responder uma carta; responder verbalmente uma mensagem}
  \end{Phonetics}
\end{Entry}

\begin{Entry}{回复}{6,9}{⼞、⼢}
  \begin{Phonetics}{回复}{hui2 fu4}[][HSK 4]
    \definition{v.}{responder (a uma carta) | retornar ao estado normal; restaurar algo ao seu estado original}
  \end{Phonetics}
\end{Entry}

\begin{Entry}{回家}{6,10}{⼞、⼧}
  \begin{Phonetics}{回家}{hui2 jia1}[][HSK 1]
    \definition{v.}{ir (voltar) para casa; estar em casa; voltar para casa}
  \end{Phonetics}
\end{Entry}

\begin{Entry}{回顾}{6,10}{⼞、⾴}
  \begin{Phonetics}{回顾}{hui2gu4}[][HSK 5]
    \definition{v.}{olhar para trás | revisar; fazer uma retrospectiva; olhar para trás, pensar no passado}
  \end{Phonetics}
\end{Entry}

\begin{Entry}{回旋}{6,11}{⼞、⽅}
  \begin{Phonetics}{回旋}{hui2xuan2}
    \definition{v.}{circular | rodar | dar a volta}
  \end{Phonetics}
\end{Entry}

\begin{Entry}{回答}{6,12}{⼞、⽵}
  \begin{Phonetics}{回答}{hui2da2}[][HSK 1]
    \definition[个]{s.}{resposta}
    \definition{v.}{responder; explicar a questão; expressar opinião sobre a solicitação}
  \end{Phonetics}
\end{Entry}

\begin{Entry}{回避}{6,16}{⼞、⾌}
  \begin{Phonetics}{回避}{hui2bi4}
    \definition{v.}{fugir (de um problema); em direito, refere-se especificamente à não participação nos procedimentos de um caso de um oficial de justiça, etc., que tenha interesse no caso ou nas partes do caso | esquivar-se; evadir-se; evitar (encontrar alguém)}
  \end{Phonetics}
\end{Entry}

\begin{Entry}{因}{6}{⼞}
  \begin{Phonetics}{因}{yin1}[][HSK 6]
    \definition*{s.}{Sobrenome Yin}
    \definition{conj.}{porque; orações de conexão, indicando relações de causa e efeito}
    \definition{prep.}{com base em; à luz de; de acordo com; a introdução da ação comportamental equivale a 按照 ou 根据}
    \definition{s.}{causa; motivo; condições em que algo ocorre ou causa um determinado resultado (em oposição a 果)}
    \definition{v.}{seguir; continuar; fazer como sempre fez | estar em conformidade com; estar de acordo com; depender; contar com}
  \seealsoref{按照}{an4zhao4}
  \seealsoref{根据}{gen1ju4}
  \seealsoref{果}{guo3}
  \end{Phonetics}
\end{Entry}

\begin{Entry}{因为}{6,4}{⼞、⼂}
  \begin{Phonetics}{因为}{yin1wei4}[][HSK 2]
    \definition{conj.}{porque; indica o motivo e a frase seguinte indica o resultado}
    \definition{prep.}{por causa de; por conta de; indica razão ou justificativa}
  \end{Phonetics}
\end{Entry}

\begin{Entry}{因为……所以……}{6,4,8,4}{⼞、⼂、⼾、⼈}
  \begin{Phonetics}{因为……所以……}{yin1wei4 suo3yi3}[][HSK 2]
    \definition{conj.}{porque\dots portanto\dots}
  \end{Phonetics}
\end{Entry}

\begin{Entry}{因此}{6,6}{⼞、⽌}
  \begin{Phonetics}{因此}{yin1ci3}[][HSK 3]
    \definition{conj.}{assim; portanto; consequentemente}
  \end{Phonetics}
\end{Entry}

\begin{Entry}{因此就}{6,6,12}{⼞、⽌、⼪}
  \begin{Phonetics}{因此就}{yin1ci3 jiu4}
    \definition{conj.}{portanto}
  \end{Phonetics}
\end{Entry}

\begin{Entry}{因而}{6,6}{⼞、⽽}
  \begin{Phonetics}{因而}{yin1'er2}[][HSK 5]
    \definition{conj.}{assim; como resultado; com o resultado que; conecta frases, indicando relação de causa e efeito}
  \end{Phonetics}
\end{Entry}

\begin{Entry}{因素}{6,10}{⼞、⽷}
  \begin{Phonetics}{因素}{yin1su4}[][HSK 6]
    \definition[个,种]{s.}{fator; elemento; os componentes que constituem a essência das coisas | fator; as razões ou condições que determinam o sucesso ou o fracasso de algo}
  \end{Phonetics}
\end{Entry}

\begin{Entry}{团}{6}{⼞}
  \begin{Phonetics}{团}{tuan2}[][HSK 3]
    \definition*{s.}{Liga da Juventude Comunista da China; Liga}
    \definition{adj.}{redondo; circular}
    \definition{clas.}{usado para algo em forma de bola}
    \definition[个]{s.}{bolinho de massa; comida em forma de bola feita de arroz ou farinha | algo em forma de bola | grupo; corpo; sociedade; organização; um grupo envolvido em um determinado trabalho ou atividade | regimento; unidade organizacional militar, geralmente abaixo do nível de divisão e acima do nível de batalhão}
    \definition{v.}{enrolar algo para formar uma bola; rolar | reunir; unir; conglomerar}
  \end{Phonetics}
\end{Entry}

\begin{Entry}{团长}{6,4}{⼞、⾧}
  \begin{Phonetics}{团长}{tuan2 zhang3}[][HSK 5]
    \definition[位,名]{s.}{comandante do regimento | chefe (ou presidente) de uma delegação, trupe, etc. | líder de uma delegação}
  \end{Phonetics}
\end{Entry}

\begin{Entry}{团队}{6,4}{⼞、⾩}
  \begin{Phonetics}{团队}{tuan2 dui4}[][HSK 6]
    \definition[个,支,种]{s.}{equipe; time; grupo; um grupo de alguma natureza}
  \end{Phonetics}
\end{Entry}

\begin{Entry}{团体}{6,7}{⼞、⼈}
  \begin{Phonetics}{团体}{tuan2ti3}[][HSK 3]
    \definition[种,个]{s.}{equipe; grupo; organização; um grupo de pessoas com objetivos e interesses comuns}
  \end{Phonetics}
\end{Entry}

\begin{Entry}{团结}{6,9}{⼞、⽷}
  \begin{Phonetics}{团结}{tuan2jie2}[][HSK 3]
    \definition{adj.}{unido; amigável; harmonioso; relação harmoniosa e coexistência harmoniosa}
    \definition{v.}{unir; reunir}
  \end{Phonetics}
\end{Entry}

\begin{Entry}{园}{7}{⼞}
  \begin{Phonetics}{园}{yuan2}[][HSK 6]
    \definition*{s.}{Sobrenome Yuan}
    \definition[个]{s.}{jardim; terreno; plantação; terra para cultivar plantas | local para recreação pública; locais para passeios turísticos e entretenimento | área para fins especiais | uma área de terra para o cultivo de plantas; um lugar onde vegetais, flores, frutas e árvores são cultivados}
  \end{Phonetics}
\end{Entry}

\begin{Entry}{园地}{7,6}{⼞、⼟}
  \begin{Phonetics}{园地}{yuan2 di4}[][HSK 6]
    \definition{s.}{jardim; campo | Figurativo: campo; escopo}
  \end{Phonetics}
\end{Entry}

\begin{Entry}{园林}{7,8}{⼞、⽊}
  \begin{Phonetics}{园林}{yuan2lin2}[][HSK 5]
    \definition[处,座,个]{s.}{parque; jardim; área paisagística com plantas e árvores para as pessoas apreciarem e descansarem.}
  \end{Phonetics}
\end{Entry}

\begin{Entry}{囯}{7}{⼞}
  \begin{Phonetics}{囯}{guo2}
    \definition*{s.}{Sobrenome Guo}
    \definition{adj.}{do estado; nacional | do nosso país; Chinês | do país}
    \definition{s.}{país; nação; estado | o melhor da nação | o melhor; o mais bonito do país}
    \variantof{国}
  \end{Phonetics}
\end{Entry}

\begin{Entry}{困}{7}{⼞}
  \begin{Phonetics}{困}{kun4}[][HSK 3]
    \definition{adj.}{cansado; exausto; fatigado | difícil; complicado; difícil e penoso; pobre e miserável | sonolento; com sono; cansado, com vontade de dormir}
    \definition{v.}{ficar encalhado; estar em apuros; preso em dificuldades e sofrimentos ou limitado por circunstâncias e condições que não pode escapar | cercar; envolver; imobilizar; controlar dentro de um determinado limite | dormir}
  \end{Phonetics}
\end{Entry}

\begin{Entry}{困扰}{7,7}{⼞、⼿}
  \begin{Phonetics}{困扰}{kun4 rao3}[][HSK 5]
    \definition{v.}{perturbar; deixar perplexo; perseguir}
  \end{Phonetics}
\end{Entry}

\begin{Entry}{困难}{7,10}{⼞、⾫}
  \begin{Phonetics}{困难}{kun4nan5}[][HSK 3]
    \definition{adj.}{dificuldades financeiras; circunstâncias difíceis | complicado; complexo; difícil; árduo; a situação é complexa e há muitos obstáculos}
    \definition[种]{s.}{dificuldade; situação difícil; problemas ou situações difíceis de resolver no trabalho e na vida}
  \end{Phonetics}
\end{Entry}

\begin{Entry}{围}{7}{⼞}
  \begin{Phonetics}{围}{wei2}[][HSK 3]
    \definition*{s.}{Sobrenome Wei}
    \definition{clas.}{o comprimento das duas mãos com os polegares e os dedos indicadores juntos ou dos dois braços juntos}
    \definition{s.}{em volta de tudo; ao redor}
    \definition{v.}{cercar; rodear; circundar; encurralar; cercar tudo, impedindo a passagem entre o interior e o exterior | envolver; contornar}
  \end{Phonetics}
\end{Entry}

\begin{Entry}{围巾}{7,3}{⼞、⼱}
  \begin{Phonetics}{围巾}{wei2jin1}[][HSK 4]
    \definition[条]{s.}{lenço; cachecol; echarpe; gravata; tiras longas de malha ou tecido usadas ao redor do pescoço para aquecimento, proteção do colarinho ou decoração}
  \end{Phonetics}
\end{Entry}

\begin{Entry}{围绕}{7,9}{⼞、⽷}
  \begin{Phonetics}{围绕}{wei2rao4}[][HSK 5]
    \definition{v.}{girar; circundar; dar voltas; girar em torno de algo; cercar | concentrar-se em; centrar-se em; centrar-se em uma questão ou evento (para realizar atividades)}
  \end{Phonetics}
\end{Entry}

\begin{Entry}{固}{8}{⼞}
  \begin{Phonetics}{固}{gu4}
    \definition*{s.}{Sobrenome Gu}
    \definition{adj.}{sólido; firme; forte | duro; sólido | mal informado; superficial; ignorante}
    \definition{adv.}{firmemente; resolutamente | originalmente; em primeiro lugar | certamente; reconhecidamente; seguramente}
    \definition{conj.}{usado da mesma forma que 固然}
    \definition{v.}{solidificar; consolidar; fortalecer | defender; proteger}
  \seealsoref{固然}{gu4ran2}
  \end{Phonetics}
\end{Entry}

\begin{Entry}{固执}{8,6}{⼞、⼿}
  \begin{Phonetics}{固执}{gu4zhi5}[][HSK 7-9]
    \definition{adj.}{obstinado; teimoso; mantém suas próprias opiniões e não quer mudá-las, mesmo que estejam erradas}
  \end{Phonetics}
\end{Entry}

\begin{Entry}{固定}{8,8}{⼞、⼧}
  \begin{Phonetics}{固定}{gu4ding4}[][HSK 4]
    \definition{adj.}{fixo; regular; inalterado ou imóvel}
    \definition{v.}{consertar; tornar fixo, não mover novamente; colocar as coisas em ordem, não mudá-las novamente}
  \end{Phonetics}
\end{Entry}

\begin{Entry}{固然}{8,12}{⼞、⽕}
  \begin{Phonetics}{固然}{gu4ran2}[][HSK 7-9]
    \definition{conj.}{usado para introduzir uma cláusula adversativa admitindo primeiro um certo fato; quando usado na primeira metade de uma frase, a segunda metade geralmente tem 可是 ou 但是 para ecoá-lo, indicando que o fato A é reconhecido, mas o fato B não se torna inválido por causa do fato A | admitir um fato sem negar outro; indica o reconhecimento de um fato, levando a uma transição no texto seguinte; indica o reconhecimento do fato A e não nega o fato B}
  \seealsoref{但是}{dan4 shi4}
  \seealsoref{可是}{ke3shi4}
  \end{Phonetics}
\end{Entry}

\begin{Entry}{国}{8}{⼞}
  \begin{Phonetics}{国}{guo2}[][HSK 1]
    \definition*{s.}{Sobrenome Guo}
    \definition{adj.}{nacional; do estado; representante do país | o melhor de um país}
    \definition[个]{s.}{estado; nação; país}
  \end{Phonetics}
\end{Entry}

\begin{Entry}{国人}{8,2}{⼞、⼈}
  \begin{Phonetics}{国人}{guo2ren2}
    \definition{s.}{compatriota}
  \end{Phonetics}
\end{Entry}

\begin{Entry}{国土}{8,3}{⼞、⼟}
  \begin{Phonetics}{国土}{guo2tu3}[][HSK 7-9]
    \definition{s.}{terra; território; território nacional}
  \end{Phonetics}
\end{Entry}

\begin{Entry}{国内}{8,4}{⼞、⼌}
  \begin{Phonetics}{国内}{guo2 nei4}[][HSK 3]
    \definition{s.}{interno (a um país); doméstico; lar; dentro de um determinado país}
  \end{Phonetics}
\end{Entry}

\begin{Entry}{国王}{8,4}{⼞、⽟}
  \begin{Phonetics}{国王}{guo2wang2}[][HSK 6]
    \definition[位,名,个,些]{s.}{rei; soberanos; o governante supremo de algumas monarquias antigas; nos tempos modernos, refere-se ao chefe de estado de algumas monarquias}
  \end{Phonetics}
\end{Entry}

\begin{Entry}{国外}{8,5}{⼞、⼣}
  \begin{Phonetics}{国外}{guo2 wai4}[][HSK 1]
    \definition{adj.}{externo; no exterior; fora do país; outros lugares fora do país; geralmente chamados de exterior;  exterior não é o mesmo que estrangeiro}
  \end{Phonetics}
\end{Entry}

\begin{Entry}{国民}{8,5}{⼞、⽒}
  \begin{Phonetics}{国民}{guo2 min2}[][HSK 5]
    \definition{adj.}{nacional}
    \definition[个]{s.}{membro de uma nação; povo de uma nação}
  \end{Phonetics}
\end{Entry}

\begin{Entry}{国产}{8,6}{⼞、⼇}
  \begin{Phonetics}{国产}{guo2 chan3}[][HSK 6]
    \definition{adj.}{doméstico; feito na China; produzido internamente, especificamente na China}
  \end{Phonetics}
\end{Entry}

\begin{Entry}{国会}{8,6}{⼞、⼈}
  \begin{Phonetics}{国会}{guo2 hui4}[][HSK 6]
    \definition{s.}{parlamento; congresso}
  \end{Phonetics}
\end{Entry}

\begin{Entry}{国庆}{8,6}{⼞、⼴}
  \begin{Phonetics}{国庆}{guo2 qing4}[][HSK 3]
    \definition*{s.}{Dia Nacional, o dia em que um país comemora sua independência ou fundação}
  \end{Phonetics}
\end{Entry}

\begin{Entry}{国庆节}{8,6,5}{⼞、⼴、⾋}
  \begin{Phonetics}{国庆节}{guo2qing4jie2}
    \definition*{s.}{Dia Nacional (1~de~outubro)}
  \end{Phonetics}
\end{Entry}

\begin{Entry}{国有}{8,6}{⼞、⽉}
  \begin{Phonetics}{国有}{guo2you3}[][HSK 7-9]
    \definition{v.}{pertencer ao estado; ser nacionalizado}
  \end{Phonetics}
\end{Entry}

\begin{Entry}{国防}{8,6}{⼞、⾩}
  \begin{Phonetics}{国防}{guo2fang2}[][HSK 7-9]
    \definition{s.}{defesa nacional; as instalações humanas, materiais e militares que um país possui para defender sua soberania territorial e impedir invasões estrangeiras}
  \end{Phonetics}
\end{Entry}

\begin{Entry}{国际}{8,7}{⼞、⾩}
  \begin{Phonetics}{国际}{guo2ji4}[][HSK 2]
    \definition{adj.}{internacional; entre países; entre nações}
    \definition{s.}{internacional; o mundo; entre nações; entre países de todo o mundo}
  \end{Phonetics}
\end{Entry}

\begin{Entry}{国际儿童节}{8,7,2,12,5}{⼞、⾩、⼉、⽴、⾋}
  \begin{Phonetics}{国际儿童节}{guo2ji4 er2tong2jie2}
    \definition*{s.}{Dia Internacional das Crianças (1~de~junho)}
  \end{Phonetics}
\end{Entry}

\begin{Entry}{国际妇女节}{8,7,6,3,5}{⼞、⾩、⼥、⼥、⾋}
  \begin{Phonetics}{国际妇女节}{guo2ji4 fu4nv3jie2}
    \definition*{s.}{Dia Internacional das Mulheres (8~de~março)}
  \end{Phonetics}
\end{Entry}

\begin{Entry}{国际劳动节}{8,7,7,6,5}{⼞、⾩、⼒、⼒、⾋}
  \begin{Phonetics}{国际劳动节}{guo2ji4 lao2dong4 jie2}
    \definition*{s.}{Dia Internacional dos Trabalhadores (1~de~maio)}
  \end{Phonetics}
\end{Entry}

\begin{Entry}{国学}{8,8}{⼞、⼦}
  \begin{Phonetics}{国学}{guo2xue2}[][HSK 7-9]
    \definition*{s.}{Arcaico: O Colégio Imperial}
    \definition{s.}{estudos da cultura clássica chinesa (história, filosofia, literatura, língua, etc.) | cultura nacional chinesa | estudos da antiga civilização chinesa}
  \end{Phonetics}
\end{Entry}

\begin{Entry}{国宝}{8,8}{⼞、⼧}
  \begin{Phonetics}{国宝}{guo2bao3}[][HSK 7-9]
    \definition[件]{s.}{tesouro nacional}
  \end{Phonetics}
\end{Entry}

\begin{Entry}{国画}{8,8}{⼞、⽥}
  \begin{Phonetics}{国画}{guo2hua4}[][HSK 7-9]
    \definition[幅,张,卷]{s.}{pintura tradicional chinesa | arte chinesa | pintura nacional}
  \end{Phonetics}
\end{Entry}

\begin{Entry}{国语}{8,9}{⼞、⾔}
  \begin{Phonetics}{国语}{guo2yu3}
    \definition*{s.}{Língua Chinesa (Mandarim), enfatizando sua natureza nacional}
  \end{Phonetics}
\end{Entry}

\begin{Entry}{国家}{8,10}{⼞、⼧}
  \begin{Phonetics}{国家}{guo2jia1}[][HSK 1]
    \definition[个]{s.}{país; estado; nação; um lugar reconhecido internacionalmente e com soberania independente, incluindo as pessoas e as instituições administrativas desse lugar}
  \end{Phonetics}
\end{Entry}

\begin{Entry}{国宾馆}{8,10,11}{⼞、⼧、⾷}
  \begin{Phonetics}{国宾馆}{guo2bin1guan3}
    \definition{s.}{pousada estadual}
  \end{Phonetics}
\end{Entry}

\begin{Entry}{国情}{8,11}{⼞、⼼}
  \begin{Phonetics}{国情}{guo2qing2}[][HSK 7-9]
    \definition{s.}{condição (ou estado) do país; condições nacionais; as condições e características básicas da natureza social, política, economia, cultura etc. de um país também se referem especificamente às condições e características básicas de um país em um determinado período de tempo}
  \end{Phonetics}
\end{Entry}

\begin{Entry}{国旗}{8,14}{⼞、⽅}
  \begin{Phonetics}{国旗}{guo2qi2}
    \definition[面]{s.}{bandeira (de um país)}
  \end{Phonetics}
\end{Entry}

\begin{Entry}{国歌}{8,14}{⼞、⽋}
  \begin{Phonetics}{国歌}{guo2 ge1}[][HSK 6]
    \definition[首,支]{s.}{hino nacional; o hino nacional da China, oficialmente designado pelo estado como a música que representa o país, é "Marcha dos Voluntários"}
  \end{Phonetics}
\end{Entry}

\begin{Entry}{国徽}{8,17}{⼞、⼻}
  \begin{Phonetics}{国徽}{guo2hui1}[][HSK 7-9]
    \definition{s.}{emblema nacional; o emblema nacional da China, oficialmente designado pelo estado para representar o país, apresenta a Praça da Paz Celestial sob o céu brilhante de cinco estrelas, cercada por espigas de grãos e engrenagens}
  \end{Phonetics}
\end{Entry}

\begin{Entry}{国籍}{8,20}{⼞、⽵}
  \begin{Phonetics}{国籍}{guo2ji2}[][HSK 5]
    \definition[个]{s.}{nacionalidade; cidadania; refere-se à identidade de um indivíduo como pertencente a um Estado | identidade nacional (de um avião, navio, etc.)}
  \end{Phonetics}
\end{Entry}

\begin{Entry}{图}{8}{⼞}
  \begin{Phonetics}{图}{tu2}[][HSK 3]
    \definition*{s.}{Sobrenome Tu}
    \definition[张]{s.}{mapa; gráfico; imagem; desenho | plano; esquema; tentativa}
    \definition{v.}{procurar; perseguir; esperar obter| desenhar; retratar; pintar | imaginar; planejar; pensar; maquinar}
  \end{Phonetics}
\end{Entry}

\begin{Entry}{图书}{8,4}{⼞、⼄}
  \begin{Phonetics}{图书}{tu2 shu1}[][HSK 6]
    \definition{s.}{livros; um termo geral para publicações como livros e álbuns de imagens}[这些图书都可以借阅。===Esses livros estão disponíveis para empréstimo.]
  \end{Phonetics}
\end{Entry}

\begin{Entry}{图书馆}{8,4,11}{⼞、⼄、⾷}
  \begin{Phonetics}{图书馆}{tu2shu1guan3}[][HSK 1]
    \definition[个,家]{s.}{biblioteca; instituição que coleta, organiza e armazena livros e materiais para leitura e consulta}
  \end{Phonetics}
\end{Entry}

\begin{Entry}{图片}{8,4}{⼞、⽚}
  \begin{Phonetics}{图片}{tu2 pian4}[][HSK 2]
    \definition[张,幅]{s.}{imagem; fotografia; um termo geral para imagens, fotografias, decalques, etc. usados para ilustrar algo}
  \end{Phonetics}
\end{Entry}

\begin{Entry}{图画}{8,8}{⼞、⽥}
  \begin{Phonetics}{图画}{tu2 hua4}[][HSK 3]
    \definition[幅,张,套]{s.}{desenho; imagem; pintura}
  \end{Phonetics}
\end{Entry}

\begin{Entry}{图案}{8,10}{⼞、⽊}
  \begin{Phonetics}{图案}{tu2'an4}[][HSK 4]
    \definition[种,个]{s.}{padrão; desenho; padrões e gráficos usados para decoração de edifícios, tecidos, artes e artesanato, etc.}
  \end{Phonetics}
\end{Entry}

\begin{Entry}{圆}{10}{⼞}
  \begin{Phonetics}{圆}{yuan2}[][HSK 4]
    \definition*{s.}{Sobrenome Yuan}
    \definition{adj.}{redondo; circular; esférico; arredondado | diplomático; satisfatório}
    \definition[个,轮]{s.}{círculo; circunferência | uma moeda de valor e peso fixos}
    \definition{v.}{tornar plausível; justificar; tornar completo; completar}
  \end{Phonetics}
\end{Entry}

\begin{Entry}{圆珠笔}{10,10,10}{⼞、⽟、⽵}
  \begin{Phonetics}{圆珠笔}{yuan2 zhu1 bi3}[][HSK 6]
    \definition[支,枝]{s.}{caneta esferográfica}
  \end{Phonetics}
\end{Entry}

\begin{Entry}{圆满}{10,13}{⼞、⽔}
  \begin{Phonetics}{圆满}{yuan2man3}[][HSK 4]
    \definition{adj.}{perfeito; satisfatório; sem defeitos}
  \end{Phonetics}
\end{Entry}

\begin{Entry}{圈}{11}{⼞}
  \begin{Phonetics}{圈}{juan1}
    \definition{v.}{prender aves e animais de criação | prender; colocar na cadeia, prisão | confinar}
  \end{Phonetics}
  \begin{Phonetics}{圈}{juan4}
    \definition*{s.}{Sobrenome Juan}
    \definition{s.}{curral; local onde o gado ou as aves são mantidos, geralmente cercado ou murado, alguns com galpões}
  \end{Phonetics}
  \begin{Phonetics}{圈}{quan1}[][HSK 4]
    \definition[个]{s.}{anel; círculo; refere-se a algo em forma de anel | domínio; grupo; escopo; círculo(s)}
    \definition{v.}{cercar; rodear; circundar | marcar com um círculo}
  \end{Phonetics}
\end{Entry}

\begin{Entry}{圈粉}{11,10}{⼞、⽶}
  \begin{Phonetics}{圈粉}{quan1fen3}
    \definition{s.}{(neologismo, coloquial) ganhar alguém como fã, obter novos fãs}
  \end{Phonetics}
\end{Entry}

%%%%% EOF %%%%%

