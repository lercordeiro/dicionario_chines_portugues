%%%
%%% Radical "⼨"
%%%

\section*{Radical 41: ``⼨''}\addcontentsline{toc}{section}{Radical 41: ⼨}

\begin{Entry}{寸}{3}{⼨}[Kangxi 41]
  \begin{Phonetics}{寸}{cun4}[][HSK 5]
    \definition*{s.}{Sobrenome Cun}
    \definition{adj.}{muito pouco; muito curto; pequeno | Ddialeto: coincidência}
    \definition{clas.}{cun, uma unidade tradicional de comprimento, igual a 0,1 市尺 e equivalente a 3,333 centímetros ou 1,312 polegadas | cun, uma unidade de comprimento (=13 decímetros)}
  \seealsoref{市尺}{shi4 chi3}
  \end{Phonetics}
\end{Entry}

\begin{Entry}{对}{5}{⼨}
  \begin{Phonetics}{对}{dui4}[][HSK 1,2]
    \definition{adj.}{certo; correto; em conformidade com determinados padrões | oposto; contrário}
    \definition{adv.}{mutuamente; cara a cara}
    \definition{clas.}{usado para pessoas ou coisas que formam pares; casais}
    \definition{prep.}{o que diz respeito a; relativo a; com relação a; introduz o objeto da ação}
    \definition[幅]{s.}{dístico; refere-se a um par de versos | par; parceiro; pessoas ou coisas que se complementam}
    \definition{v.}{responder; dar uma resposta | tratar; lidar com; combater | ser treinado para; ser direcionado para; enfrentar | colocar (duas coisas) em contato; encaixar uma na outra; combinar ou cooperar entre si | comparar; verificar; identificar; comparar e verificar se estão de acordo | definir; ajustar; ajustar para atender a determinados requisitos | misturar (refere-se principalmente a líquidos); adicionar | dividir ao meio; dividir em duas partes iguais | combinar; concordar; dar-se bem; harmonizar-se}
  \end{Phonetics}
\end{Entry}

\begin{Entry}{对于}{5,3}{⼨、⼆}
  \begin{Phonetics}{对于}{dui4yu2}[][HSK 4]
    \definition{prep.}{para; relativo a; no que diz respeito a; a respeito de}
  \end{Phonetics}
\end{Entry}

\begin{Entry}{对不起}{5,4,10}{⼨、⼀、⾛}
  \begin{Phonetics}{对不起}{dui4bu5qi3}[][HSK 1]
    \definition{interj.}{Desculpe! | Desculpe-me! | Perdoe-me! | Com licença?}
    \definition{v.}{desculpar; pedir desculpas; perdoar}
  \end{Phonetics}
\end{Entry}

\begin{Entry}{对手}{5,4}{⼨、⼿}
  \begin{Phonetics}{对手}{dui4shou3}[][HSK 3]
    \definition[个,名,位,对]{s.}{oponente; adversário na competição | igual; correspondente; refere-se especificamente ao adversário em uma competição em que as habilidades e o nível são praticamente iguais}
  \end{Phonetics}
\end{Entry}

\begin{Entry}{对方}{5,4}{⼨、⽅}
  \begin{Phonetics}{对方}{dui4fang1}[][HSK 3]
    \definition{s.}{outro lado; lado oposto; outra parte; a parte contrária ao sujeito da ação ou outras pessoas envolvidas em um determinado evento ou situação}
  \end{Phonetics}
\end{Entry}

\begin{Entry}{对比}{5,4}{⼨、⽐}
  \begin{Phonetics}{对比}{dui4bi3}[][HSK 4]
    \definition{s.}{razão; proporção | contraste; comparação; diferenças ou lacunas encontradas após comparação}
    \definition{v.}{contrastar; comparar}
  \end{Phonetics}
\end{Entry}

\begin{Entry}{对付}{5,5}{⼨、⼈}
  \begin{Phonetics}{对付}{dui4fu5}[][HSK 4]
    \definition{adj.}{em bons termos; estar em termos agradáveis ​​(frequentemente usado em negativas); dialeto usado para descrever duas pessoas que têm um bom relacionamento e se dão bem, frequentemente usado para negar}
    \definition{v.}{enfrentar; tratar; lidar com | fazer acontecer; (informal) fazer algo que você não quer fazer; aceitar algo que você não gosta}
  \end{Phonetics}
\end{Entry}

\begin{Entry}{对外}{5,5}{⼨、⼣}
  \begin{Phonetics}{对外}{dui4 wai4}[][HSK 6]
    \definition{adj.}{externo; para fora | estrangeiro; no exterior}
  \end{Phonetics}
\end{Entry}

\begin{Entry}{对立}{5,5}{⼨、⽴}
  \begin{Phonetics}{对立}{dui4li4}[][HSK 5]
    \definition{v.}{opor-se; contrastar; filosoficamente, refere-se a duas coisas ou dois aspectos da mesma coisa que se contradizem, se excluem ou entram em conflito entre si | opor-se; ser antagônico a}
  \end{Phonetics}
\end{Entry}

\begin{Entry}{对……有兴趣}{5,6,6,15}{⼨、⽉、⼋、⾛}
  \begin{Phonetics}{对……有兴趣}{dui4 you3xing4qu4}
    \definition{expr.}{estar interessado em\dots; ter interesse em\dots; interessar-se por\dots}
  \seealsoref{对……感兴趣}{dui4 gan3xing4qu4}
  \end{Phonetics}
\end{Entry}

\begin{Entry}{对应}{5,7}{⼨、⼴}
  \begin{Phonetics}{对应}{dui4ying4}[][HSK 5]
    \definition{adj.}{homólogo; correspondente}
    \definition{v.}{corresponder; ser equivalente a}
  \end{Phonetics}
\end{Entry}

\begin{Entry}{对抗}{5,7}{⼨、⼿}
  \begin{Phonetics}{对抗}{dui4kang4}[][HSK 6]
    \definition{v.}{antagonizar; confrontar | resistir; opor-se; contra-atacar}
  \end{Phonetics}
\end{Entry}

\begin{Entry}{对话}{5,8}{⼨、⾔}
  \begin{Phonetics}{对话}{dui4hua4}[][HSK 2]
    \definition[段,番,个]{s.}{diálogo; conversa; refere-se especificamente a diálogos entre personagens em obras literárias, como peças de teatro e romances}
    \definition{v.}{conversar com; comunicar-se com | manter um diálogo; conversar uns com os outros}
  \end{Phonetics}
\end{Entry}

\begin{Entry}{对待}{5,9}{⼨、⼻}
  \begin{Phonetics}{对待}{dui4dai4}[][HSK 3]
    \definition{v.}{tratar; abordar; manusear; estar em uma posição relacionada ou comparada a outra; expressar uma certa atitude ou agir de determinada maneira em relação a pessoas ou coisas}
  \end{Phonetics}
\end{Entry}

\begin{Entry}{对……说}{5,9}{⼨、⾔}
  \begin{Phonetics}{对……说}{dui4 shuo5}
    \definition{v.}{dizer a alguém}
  \end{Phonetics}
\end{Entry}

\begin{Entry}{对面}{5,9}{⼨、⾯}
  \begin{Phonetics}{对面}{dui4mian4}[][HSK 2]
    \definition{adv.}{cara a cara}
    \definition[面]{s.}{lado oposto; o outro lado; os nomes dados às duas margens opostas de ruas, rios, etc. | bem na frente; diretamente à frente}
  \end{Phonetics}
\end{Entry}

\begin{Entry}{对得起}{5,11,10}{⼨、⼻、⾛}
  \begin{Phonetics}{对得起}{dui4de5qi3}
    \definition{v.}{não decepcionar alguém | tratar alguém de maneira justa | ser digno de}
  \end{Phonetics}
\end{Entry}

\begin{Entry}{对象}{5,11}{⼨、⾗}
  \begin{Phonetics}{对象}{dui4xiang4}[][HSK 3]
    \definition[个,位]{s.}{alvo; objeto; a pessoa ou coisa que serve como objetivo ao agir ou pensar | parceiro; namorado; namorada; refere-se especificamente à pessoa amada}
  \end{Phonetics}
\end{Entry}

\begin{Entry}{对……感兴趣}{5,13,6,15}{⼨、⼼、⼋、⾛}
  \begin{Phonetics}{对……感兴趣}{dui4 gan3xing4qu4}
    \definition{expr.}{estar interessado em\dots; ter interesse em\dots; interessar-se por\dots}
  \seealsoref{对……有兴趣}{dui4 you3xing4qu4}
  \end{Phonetics}
\end{Entry}

\begin{Entry}{对……熟悉}{5,15,11}{⼨、⽕、⼼}
  \begin{Phonetics}{对……熟悉}{dui4 shu2xi1}
    \definition{expr.}{estar familiarizado com\dots}
  \end{Phonetics}
\end{Entry}

\begin{Entry}{寺}{6}{⼨}
  \begin{Phonetics}{寺}{si4}[][HSK 6]
    \definition*{s.}{Sobrenome Si}
    \definition[座]{s.}{templo | (Islã) mesquita | (datado) ministério; agência governamental na China antiga}
  \end{Phonetics}
\end{Entry}

\begin{Entry}{寺庙}{6,8}{⼨、⼴}
  \begin{Phonetics}{寺庙}{si4miao4}
    \definition{s.}{templo | mosteiro | santuário}
  \end{Phonetics}
\end{Entry}

\begin{Entry}{寻}{6}{⼨}
  \begin{Phonetics}{寻}{xun2}
    \definition*{s.}{Sobrenome Xun}
    \definition{clas.}{uma unidade antiga de comprimento, igual a 8尺}
    \definition{v.}{procurar; pesquisar; buscar}
  \seealsoref{尺}{chi3}
  \end{Phonetics}
\end{Entry}

\begin{Entry}{寻找}{6,7}{⼨、⼿}
  \begin{Phonetics}{寻找}{xun2zhao3}[][HSK 4]
    \definition{v.}{buscar; procurar; pesquisar; encontrar, que pode ser usado tanto para coisas concretas quanto para coisas abstratas}
  \end{Phonetics}
\end{Entry}

\begin{Entry}{寻求}{6,7}{⼨、⽔}
  \begin{Phonetics}{寻求}{xun2 qiu2}[][HSK 5]
    \definition{v.}{procurar; perseguir; explorar; ir em busca de}
  \end{Phonetics}
\end{Entry}

\begin{Entry}{导}{6}{⼨}
  \begin{Phonetics}{导}{dao3}
    \definition[个,位,名,些]{s.}{guia turístico | diretor}
    \definition{v.}{liderar; guiar | conduzir; transmitir | ensinar; instruir; dar orientação a}
  \end{Phonetics}
\end{Entry}

\begin{Entry}{导致}{6,10}{⼨、⾄}
  \begin{Phonetics}{导致}{dao3zhi4}[][HSK 4]
    \definition{v.}{causar; levar a; dar origem a (um resultado ruim)}
  \end{Phonetics}
\end{Entry}

\begin{Entry}{导弹}{6,11}{⼨、⼸}
  \begin{Phonetics}{导弹}{dao3dan4}
    \definition[枚]{s.}{míssil (guiado)}
  \end{Phonetics}
\end{Entry}

\begin{Entry}{导游}{6,12}{⼨、⽔}
  \begin{Phonetics}{导游}{dao3you2}[][HSK 4]
    \definition[个,位,名]{s.}{guia turístico; pessoas que trabalham como guias turísticos}
    \definition{v.}{guiar; conduzir um passeio turístico}
  \end{Phonetics}
\end{Entry}

\begin{Entry}{导演}{6,14}{⼨、⽔}
  \begin{Phonetics}{导演}{dao3yan3}[][HSK 3]
    \definition[位,名,个]{s.}{diretor; pessoa que exerce a função de diretor}
    \definition{v.}{dirigir (um filme, peça, etc.); ensaio de peças teatrais ou filmagem de filmes e séries de TV; organização e orientação do trabalho de produção}
  \end{Phonetics}
\end{Entry}

\begin{Entry}{寿}{7}{⼨}
  \begin{Phonetics}{寿}{shou4}
    \definition[个,份]{s.}{vida longa; velhice | vida; idade | aniversário | (eufenismo) funerário; preparado antes da morte | longevidade}
  \end{Phonetics}
\end{Entry}

\begin{Entry}{寿司}{7,5}{⼨、⼝}
  \begin{Phonetics}{寿司}{shou4 si1}[][HSK 5]
    \definition[份]{s.}{\emph{sushi}; iguaria tradicional japonesa}
  \end{Phonetics}
\end{Entry}

\begin{Entry}{封}{9}{⼨}
  \begin{Phonetics}{封}{feng1}[][HSK 2,5]
    \definition*{s.}{Sobrenome Feng}
    \definition{clas.}{usado para objetos selados, especialmente cartas}
    \definition{s.}{feudalismo | embalagem; envelope | pacote}
    \definition{v.}{conferir (um título, território, etc.) a | selar | acender uma fogueira | fechar}
  \end{Phonetics}
\end{Entry}

\begin{Entry}{封口}{9,3}{⼨、⼝}
  \begin{Phonetics}{封口}{feng1kou3}
    \definition{v.}{selar | fechar | curar (uma ferida) | manter os lábios selados}
  \end{Phonetics}
\end{Entry}

\begin{Entry}{封印}{9,5}{⼨、⼙}
  \begin{Phonetics}{封印}{feng1yin4}
    \definition{s.}{selo (em envelopes)}
  \end{Phonetics}
\end{Entry}

\begin{Entry}{封闭}{9,6}{⼨、⾨}
  \begin{Phonetics}{封闭}{feng1bi4}[][HSK 4]
    \definition{adj.}{fechado; aqueles que não têm contato com o mundo exterior; aqueles que são muito conservadores (em seu pensamento) e não se comunicam com os outros}
    \definition{v.}{selar; fechar; lacrar; vedar; de modo a impedir a passagem, o uso ou a abertura}
  \end{Phonetics}
\end{Entry}

\begin{Entry}{封冻}{9,7}{⼨、⼎}
  \begin{Phonetics}{封冻}{feng1dong4}
    \definition{v.}{congelar (água ou terra)}
  \end{Phonetics}
\end{Entry}

\begin{Entry}{封底}{9,8}{⼨、⼴}
  \begin{Phonetics}{封底}{feng1di3}
    \definition{s.}{contracapa de um livro}
  \end{Phonetics}
\end{Entry}

\begin{Entry}{封建}{9,8}{⼨、⼵}
  \begin{Phonetics}{封建}{feng1jian4}
    \definition{adj.}{feudal}
    \definition{s.}{feudalismo}
  \end{Phonetics}
\end{Entry}

\begin{Entry}{封面}{9,9}{⼨、⾯}
  \begin{Phonetics}{封面}{feng1mian4}
    \definition{s.}{capa (de uma publicação) | sobrecapa}
  \end{Phonetics}
\end{Entry}

\begin{Entry}{封斋}{9,10}{⼨、⽂}
  \begin{Phonetics}{封斋}{feng1zhai1}
    \definition*{s.}{Ramadã (Islã)}
  \end{Phonetics}
\end{Entry}

\begin{Entry}{封盖}{9,11}{⼨、⽫}
  \begin{Phonetics}{封盖}{feng1gai4}
    \definition{s.}{boné | capa | selo}
    \definition{v.}{cobrir}
  \end{Phonetics}
\end{Entry}

\begin{Entry}{将}{9}{⼨}
  \begin{Phonetics}{将}{jiang1}[][HSK 5]
    \definition*{s.}{Sobrenome Jiang}
    \definition{adv.}{estar indo para; parcialmente\dots parcialmente\dots}
    \definition{part.}{expressar uma direção, como 进来, 出去; usado no meio de verbos e complementos que indicam tendência, como 进来, 出去, etc.}
    \definition{prep.}{com; por meio de; por | usado da mesma forma que 把}
    \definition{v.}{fazer algo; lidar com (um assunto) | dar um cheque-mate | cuidar (da saúde) | incitar alguém a agir; desafiar; estimular | segurar; pegar | colocar; tirar | levar; trazer | dar suporte; dar apoio}
  \seealsoref{把}{ba3}
  \seealsoref{出去}{chu1 qu4}
  \seealsoref{进来}{jin4 lai2}
  \end{Phonetics}
  \begin{Phonetics}{将}{jiang4}
    \definition{s.}{general; nome do posto; abaixo de marechal de campo; acima de coronel}
    \definition{v.}{comandar; liderar}
  \end{Phonetics}
  \begin{Phonetics}{将}{qiang1}
    \definition{v.}{pedir; apelar para}
  \end{Phonetics}
\end{Entry}

\begin{Entry}{将军}{9,6}{⼨、⼍}
  \begin{Phonetics}{将军}{jiang1/jun1}[][HSK 6]
    \definition[位,名]{s.}{general; geralmente se refere a generais seniores}
    \definition{v.+compl.}{dar xeque-mate; atacar o general ou rei do oponente no xadrez; colocar alguém em grandes apuros; metáfora para dar a alguém um problema difícil ou dificultar a tarefa para essa pessoa}
  \end{Phonetics}
\end{Entry}

\begin{Entry}{将来}{9,7}{⼨、⽊}
  \begin{Phonetics}{将来}{jiang1lai2}[][HSK 3]
    \definition[个]{s.}{no futuro (geralmente se refere a um período mais longo)}
  \end{Phonetics}
\end{Entry}

\begin{Entry}{将近}{9,7}{⼨、⾡}
  \begin{Phonetics}{将近}{jiang1jin4}[][HSK 3]
    \definition{adv.}{quase}
  \end{Phonetics}
\end{Entry}

\begin{Entry}{将要}{9,9}{⼨、⾑}
  \begin{Phonetics}{将要}{jiang1 yao4}[][HSK 5]
    \definition{adv.}{irá; deverá; estará prestes a; irá a; indica que um ato ou situação ocorre logo em seguida}
  \end{Phonetics}
\end{Entry}

\begin{Entry}{射}{10}{⼨}
  \begin{Phonetics}{射}{she4}[][HSK 5]
    \definition*{s.}{Sobrenome She}
    \definition{v.}{atirar; disparar | descarregar em jato; jorrar | emitir (luz, calor, etc.) | irradiar | aludir a algo ou alguém; insinuar}
  \end{Phonetics}
\end{Entry}

\begin{Entry}{射击}{10,5}{⼨、⼐}
  \begin{Phonetics}{射击}{she4ji1}[][HSK 5]
    \definition{s.}{tiro; tiro ao alvo}
    \definition{v.}{disparar; atirar}
  \end{Phonetics}
\end{Entry}

\begin{Entry}{尊}{12}{⼨}
  \begin{Phonetics}{尊}{zun1}
    \definition*{s.}{Sobrenome Zun}
    \definition{adj.}{sênior; de uma geração sênior; alto status ou antiguidade}
    \definition{clas.}{usado para estátuas, canhões, etc.}
    \definition{pron.}{seu; vossa; antigamente, referia-se a pessoas ou coisas relacionadas entre si}
    \definition{s.}{um tipo de recipiente para vinho usado nos tempos antigos}
    \definition{v.}{respeitar; reverenciar; venerar; honrar}
  \end{Phonetics}
\end{Entry}

\begin{Entry}{尊重}{12,9}{⼨、⾥}
  \begin{Phonetics}{尊重}{zun1zhong4}[][HSK 5]
    \definition{adj.}{sério; adequado; correto; (linguagem, comportamento) não ser descuidado; não ser leviano}
    \definition{v.}{respeitar; valorizar; estimar; tratar com educação; valorizar | tratar com seriedade; levar a sério e tratar com seriedade}
  \end{Phonetics}
\end{Entry}

\begin{Entry}{尊敬}{12,12}{⼨、⽁}
  \begin{Phonetics}{尊敬}{zun1jing4}[][HSK 5]
    \definition{adj.}{respeitoso; respeitável}
    \definition{v.}{respeitar; honrar; estimar}
  \end{Phonetics}
\end{Entry}

%%%%% EOF %%%%%

