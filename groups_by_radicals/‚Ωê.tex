%%%
%%% Radical "⽐"
%%%

\section*{Radical 81: ``⽐''}\addcontentsline{toc}{section}{Radical 81: ⽐}

\begin{Entry}{比}{4}{⽐}[Kangxi 81]
  \begin{Phonetics}{比}{bi3}[][HSK 1]
    \definition*{s.}{Abreviação de Bélgica, 比利时}
    \definition{adj.}{específico}
    \definition{adv.}{recentemente}
    \definition{part.}{partícula usada para comparação (superioridade)}
    \definition{prep.}{que; do que | (seguido por um substantivo e adjetivo) mais \{adj.\} do que \{s.\}}
    \definition{s.}{razão; proporção | contraste; comparação | metáfora em poesia; técnica de composição poética}
    \definition{v.}{estar ao lado de; estar próximo a | igualar; comparar; competir; contrastar; emular; comparar superioridade, inferioridade, comprimento, distância, qualidade, etc. | assemelhar-se a; comparar com; fazer uma analogia | gesticular; fazer gestos | ser treinado em; ser direcionado a | copiar; imitar | poder ser comparado | apegar-se a; depender}
  \seealsoref{比利时}{bi3li4shi2}
  \end{Phonetics}
\end{Entry}

\begin{Entry}{比不上}{4,4,3}{⽐、⼀、⼀}
  \begin{Phonetics}{比不上}{bi3 bu5 shang4}[][HSK 7-9]
    \definition{expr.}{não se compara a; uma coisa não pode ser comparada com outra, não pode atingir o mesmo nível}
  \end{Phonetics}
\end{Entry}

\begin{Entry}{比分}{4,4}{⽐、⼑}
  \begin{Phonetics}{比分}{bi3 fen1}[][HSK 4]
    \definition{s.}{pontuação; comparação de pontuações entre as duas equipes em uma partida}
  \end{Phonetics}
\end{Entry}

\begin{Entry}{比方}{4,4}{⽐、⽅}
  \begin{Phonetics}{比方}{bi3fang1}[][HSK 5]
    \definition{conj.}{se; suponha que; expressa uma hipótese, equivalente a 如果 (com eufemismos)}
    \definition{s.}{analogia; exemplo; instância; expressão que usa uma coisa para descrever outra (expressão idiomática); (figurativo) usar uma coisa para descrever outra}
    \definition{v.}{ilustrar; exemplificar; fazer uma analogia; usar uma coisa para descrever outra (expressão idiomática)}
  \seealsoref{如果}{ru2guo3}
  \end{Phonetics}
\end{Entry}

\begin{Entry}{比比皆是}{4,4,9,9}{⽐、⽐、⽩、⽇}
  \begin{Phonetics}{比比皆是}{bi3bi3-jie1shi4}[][HSK 7-9]
    \definition{suf.}{pode ser encontrado em todos os lugares; ao redor | encontrar o olhar em todos os lugares; ser grande em número; ser visto em todos os lugares -- em abundância; um após o outro}
  \end{Phonetics}
\end{Entry}

\begin{Entry}{比亚迪}{4,6,8}{⽐、⼆、⾡}
  \begin{Phonetics}{比亚迪}{bi3ya4di2}
    \definition*{s.}{Montadora BYD}
  \end{Phonetics}
\end{Entry}

\begin{Entry}{比如}{4,6}{⽐、⼥}
  \begin{Phonetics}{比如}{bi3ru2}[][HSK 2]
    \definition{conj.}{por exemplo; tal como; suponha; digamos; a seguir, apresentamos alguns exemplos; na linguagem coloquial, também se pode dizer 比如说}
  \seealsoref{比如说}{bi3 ru2 shuo1}
  \end{Phonetics}
\end{Entry}

\begin{Entry}{比如说}{4,6,9}{⽐、⼥、⾔}
  \begin{Phonetics}{比如说}{bi3 ru2 shuo1}[][HSK 2]
    \definition{adv.}{por exemplo}
  \seealsoref{比如}{bi3ru2}
  \end{Phonetics}
\end{Entry}

\begin{Entry}{比利时}{4,7,7}{⽐、⼑、⽇}
  \begin{Phonetics}{比利时}{bi3li4shi2}
    \definition*{s.}{Bélgica}
  \end{Phonetics}
\end{Entry}

\begin{Entry}{比例}{4,8}{⽐、⼈}
  \begin{Phonetics}{比例}{bi3li4}[][HSK 3]
    \definition{s.}{escala; razão; relação de múltiplos entre dois números | porporção; a quantidade que uma parte representa no todo | proporção; na descrição do grau de coordenação das coisas}
  \end{Phonetics}
\end{Entry}

\begin{Entry}{比试}{4,8}{⽐、⾔}
  \begin{Phonetics}{比试}{bi3shi4}[][HSK 7-9]
    \definition{v.}{competir; ter uma competição; contestar | medir com a mão ou o braço; fazer um gesto de medição}
  \end{Phonetics}
\end{Entry}

\begin{Entry}{比拼}{4,9}{⽐、⼿}
  \begin{Phonetics}{比拼}{bi3pin1}
    \definition{s.}{concurso}
    \definition{v.}{competir ferozmente}
  \end{Phonetics}
\end{Entry}

\begin{Entry}{比重}{4,9}{⽐、⾥}
  \begin{Phonetics}{比重}{bi3zhong4}[][HSK 5]
    \definition{s.}{proporção; o peso da parte em relação ao todo | Física: densidade específica; a relação entre o peso de um objeto e seu volume}
  \end{Phonetics}
\end{Entry}

\begin{Entry}{比起}{4,10}{⽐、⾛}
  \begin{Phonetics}{比起}{bi3qi3}[][HSK 7-9]
    \definition{adv./prep.}{comparado com}
  \end{Phonetics}
\end{Entry}

\begin{Entry}{比较}{4,10}{⽐、⾞}
  \begin{Phonetics}{比较}{bi3jiao4}[][HSK 3]
    \definition{adv.}{razoavelmente; relativamente; bastante; um pouco; comparativamente; indica um certo grau, com o significado de 相当}
    \definition{prep.}{usado para comparar uma diferença de grau. para distinguir as diferenças ou superioridades entre duas ou mais coisas semelhantes}
    \definition{v.}{comparar; contrastar; usado para comparar diferenças em propriedades e graus; para distinguir semelhanças, diferenças ou superioridade entre duas ou mais coisas semelhantes}
  \seealsoref{相当}{xiang1dang1}
  \end{Phonetics}
\end{Entry}

\begin{Entry}{比萨饼}{4,11,9}{⽐、⾋、⾷}
  \begin{Phonetics}{比萨饼}{bi3sa4bing3}
    \definition[张]{s.}{pizza}
  \end{Phonetics}
\end{Entry}

\begin{Entry}{比喻}{4,12}{⽐、⼝}
  \begin{Phonetics}{比喻}{bi3yu4}[][HSK 7-9]
    \definition[个,种]{s.}{analogia; metáfora; um método ou exemplo de uso de uma imagem ou coisa concreta para ilustrar outra coisa}
    \definition{v.}{fazer uma analogia; usar uma metáfora; usar algo semelhante para comparar com algo que você quer dizer}
  \end{Phonetics}
\end{Entry}

\begin{Entry}{比赛}{4,14}{⽐、⾙}
  \begin{Phonetics}{比赛}{bi3sai4}[][HSK 3]
    \definition[场,次,轮,站,个]{s.}{competição; atividades da competição}
    \definition{v.}{competir; disputar; comparar o nível e a qualidade das habilidades e competências}
  \end{Phonetics}
\end{Entry}

\begin{Entry}{毕}{6}{⽐}
  \begin{Phonetics}{毕}{bi4}
    \definition*{s.}{Bi, uma das mansões lunares; a décima nona das vinte e oito constelações em que a esfera celeste foi dividida, consistindo de oito estrelas, seis em Híades e duas em Touro | Sobrenome Bi}
    \definition{adv.}{tudo; completamente; totalmente}
    \definition{v.}{terminar; realizar; concluir  | completar; terminar}
  \end{Phonetics}
\end{Entry}

\begin{Entry}{毕业}{6,5}{⽐、⼀}
  \begin{Phonetics}{毕业}{bi4/ye4}[][HSK 4]
    \definition{v.+compl.}{formar-se}
  \end{Phonetics}
\end{Entry}

\begin{Entry}{毕业生}{6,5,5}{⽐、⼀、⽣}
  \begin{Phonetics}{毕业生}{bi4 ye4 sheng1}[][HSK 4]
    \definition[个,名,位,些]{s.}{diplomado; graduado; bacharel; pessoa que recebeu um diploma, grau ou certificado}
  \end{Phonetics}
\end{Entry}

\begin{Entry}{毕竟}{6,11}{⽐、⾳}
  \begin{Phonetics}{毕竟}{bi4jing4}[][HSK 5]
    \definition{adv.}{afinal de contas; quando tudo estiver dito e feito; em última análise; indica um resultado que não pode ser alterado, enfatizando que se trata de uma causa ou fato que precisa ser enfocado para referência | significa 到底, 究竟, 终究, indicando a conclusão final alcançada}
  \seealsoref{到底}{dao4di3}
  \seealsoref{究竟}{jiu1jing4}
  \seealsoref{终究}{zhong1jiu1}
  \end{Phonetics}
\end{Entry}

%%%%% EOF %%%%%

