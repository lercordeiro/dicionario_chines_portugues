%%%
%%% Radical "⾓"
%%%

\section*{Radical 148: ``⾓'' (⻆、⻇)}\addcontentsline{toc}{section}{Radical 148: ⾓、⻆、⻇}

\begin{Entry}{角}{7}{⾓}[Kangxi 148]
  \begin{Phonetics}{角}{jiao3}[][HSK 2]
    \definition*{s.}{Jiao, uma das mansões lunares}
    \definition{clas.}{uma unidade monetária fracionária na China (=1/10 de um yuan ou 10 fen)}
    \definition[个,只,对]{s.}{chifre; o objeto duro que cresce na cabeça de bovinos, ovinos, veados, etc. | buzina; corneta; instrumentos musicais tocados no exército antigo | algo com a forma de um chifre | cabo; promontório; península | esquina; canto; a junção entre duas arestas de um objeto | ângulo}
  \end{Phonetics}
  \begin{Phonetics}{角}{jue2}
    \definition*{s.}{Sobrenome Jue}
    \definition[个,只,对]{s.}{papel; parte; personagem | tipo de papel (no drama tradicional chinês); categorias de divisão profissional do trabalho entre atores de ópera | ator ou atriz | uma antiga taça de vinho com três pernas | uma nota da antiga escala chinesa de cinco tons, correspondente a 3 na notação musical numerada}
    \definition{v.}{competir; contender; lutar}
  \end{Phonetics}
\end{Entry}

\begin{Entry}{角色}{7,6}{⾓、⾊}
  \begin{Phonetics}{角色}{jue2se4}[][HSK 4]
    \definition{s.}{papel; personagem em uma peça; personagem representado por um ator | papel; função; parte; uma metáfora para um certo tipo de pessoas na vida social}
  \end{Phonetics}
\end{Entry}

\begin{Entry}{角度}{7,9}{⾓、⼴}
  \begin{Phonetics}{角度}{jiao3du4}[][HSK 2]
    \definition[个,种]{s.}{perspectiva; ponto de vista; o ponto de partida para ver as coisas | ângulo; o tamanho do ângulo; normalmente expresso em graus ou radianos}
  \end{Phonetics}
\end{Entry}

\begin{Entry}{角落}{7,12}{⾓、⾋}
  \begin{Phonetics}{角落}{jiao3luo4}[][HSK 7-9]
    \definition[个,处]{s.}{canto; recanto; o ângulo côncavo na junção de duas paredes ou estruturas semelhantes | um lugar remoto}
  \end{Phonetics}
\end{Entry}

\begin{Entry}{解}{13}{⾓}
  \begin{Phonetics}{解}{jie3}[][HSK 6]
    \definition{s.}{solução; o valor de uma variável desconhecida em uma equação algébrica}
    \definition{v.}{dividir; separar | desfazer; desatar; abrir algo que esteja amarrado ou encadernado | acalmar; dissipar; dispensar; eliminar | resolver; explicar; interpretar | entender; compreender | aliviar-se (excreção de urina e fezes) | dissolver; desintegrar | (cálculo analítico) resolver; solucionar}
  \end{Phonetics}
\end{Entry}

\begin{Entry}{解开}{13,4}{⾓、⼶}
  \begin{Phonetics}{解开}{jie3 kai1}[][HSK 3]
    \definition{v.}{desatar; desamarrar; desabotoar; desamarrar ou desfazer nós}
  \end{Phonetics}
\end{Entry}

\begin{Entry}{解决}{13,6}{⾓、⼎}
  \begin{Phonetics}{解决}{jie3jue2}[][HSK 3]
    \definition{v.}{solucionar; resolver; liquidar; resolver problemas com resultados | acabar com; descartar; eliminar (o inimigo)}
  \end{Phonetics}
\end{Entry}

\begin{Entry}{解压}{13,6}{⾓、⼚}
  \begin{Phonetics}{解压}{jie3ya1}
    \definition{v.}{aliviar o estresse | (computação) descomprimir}
  \end{Phonetics}
\end{Entry}

\begin{Entry}{解体}{13,7}{⾓、⼈}
  \begin{Phonetics}{解体}{jie3ti3}[][HSK 7-9]
    \definition{v.}{(corpo orgânico) decompor-se | (sistema social, organização, etc.) desintegrar-se | desmontar; desintegrar; quebrar; desmantelar}
  \end{Phonetics}
\end{Entry}

\begin{Entry}{解围}{13,7}{⾓、⼞}
  \begin{Phonetics}{解围}{jie3/wei2}[][HSK 7-9]
    \definition{v.+compl.}{forçar um inimigo a levantar um cerco; resgatar de um cerco; vir em socorro dos sitiados | ajudar a sair de uma situação difícil; evitar constrangimento; livrar alguém de um constrangimento; livrar alguém de uma enrascada; amenizar o constrangimento de alguém}
  \end{Phonetics}
\end{Entry}

\begin{Entry}{解放}{13,8}{⾓、⽅}
  \begin{Phonetics}{解放}{jie3fang4}[][HSK 5]
    \definition*{s.}{Libertação (que significou o fim do domínio do regime reacionário Kuomintang em 1949 e ao estabelecimento da República Popular da China)}
    \definition{v.}{libertar; emancipar; eliminar as restrições para permitir o desenvolvimento da liberdade}
  \end{Phonetics}
\end{Entry}

\begin{Entry}{解析}{13,8}{⾓、⽊}
  \begin{Phonetics}{解析}{jie3xi1}[][HSK 7-9]
    \definition{v.}{analisar; examinar minuciosamente; resolver}
  \end{Phonetics}
\end{Entry}

\begin{Entry}{解说}{13,9}{⾓、⾔}
  \begin{Phonetics}{解说}{jie3 shuo1}[][HSK 6]
    \definition{v.}{narrar; comentar; fazer um comentário; explicar oralmente}
  \end{Phonetics}
\end{Entry}

\begin{Entry}{解除}{13,9}{⾓、⾩}
  \begin{Phonetics}{解除}{jie3chu2}[][HSK 5]
    \definition{v.}{remover; aliviar; livrar-se de; eliminar}
  \end{Phonetics}
\end{Entry}

\begin{Entry}{解剖}{13,10}{⾓、⼑}
  \begin{Phonetics}{解剖}{jie3pou1}[][HSK 7-9]
    \definition{v.}{dissecar | analisar; metaforicamente, refere-se à observação e análise aprofundada das coisas.}
  \end{Phonetics}
\end{Entry}

\begin{Entry}{解读}{13,10}{⾓、⾔}
  \begin{Phonetics}{解读}{jie3du2}[][HSK 7-9]
    \definition{v.}{decodificar; interpretar; explicar; compreender por meio da análise}
  \end{Phonetics}
\end{Entry}

\begin{Entry}{解救}{13,11}{⾓、⽁}
  \begin{Phonetics}{解救}{jie3jiu4}[][HSK 7-9]
    \definition{v.}{salvar; resgatar; sair do perigo ou da dificuldade}
  \end{Phonetics}
\end{Entry}

\begin{Entry}{解脱}{13,11}{⾓、⾁}
  \begin{Phonetics}{解脱}{jie3tuo1}[][HSK 7-9]
    \definition{v.}{libertar-se das preocupações mundanas; no budismo,  se refere à libertação do sofrimento e à conquista da liberdade | libertar-se (ou desvencilhar-se); livrar-se de | absolver; exonerar}
  \end{Phonetics}
\end{Entry}

\begin{Entry}{解散}{13,12}{⾓、⽁}
  \begin{Phonetics}{解散}{jie3san4}[][HSK 7-9]
    \definition{v.}{dispensar | dissolver; desmantelar; cancelar}
  \end{Phonetics}
\end{Entry}

\begin{Entry}{解答}{13,12}{⾓、⽵}
  \begin{Phonetics}{解答}{jie3da2}[][HSK 7-9]
    \definition{v.}{responder; explicar}
  \end{Phonetics}
\end{Entry}

\begin{Entry}{解释}{13,12}{⾓、⾤}
  \begin{Phonetics}{解释}{jie3shi4}[][HSK 4]
    \definition{v.}{explicar; expor; interpretar | analisar; explicaro significado, razões, justificativas, etc.}
  \end{Phonetics}
\end{Entry}

\begin{Entry}{解雇}{13,12}{⾓、⾫}
  \begin{Phonetics}{解雇}{jie3gu4}[][HSK 7-9]
    \definition{v.}{demitir; dispensar; exonerar}
  \end{Phonetics}
\end{Entry}

\begin{Entry}{触}{13}{⾓}
  \begin{Phonetics}{触}{chu4}
    \definition{v.}{tocar; contatar | atacar; dar um toque | tocar/mover alguém emocionalmente; agitar os sentimentos de alguém | levar um choque elétrico; ser eletrocutado}
  \end{Phonetics}
\end{Entry}

\begin{Entry}{触犯}{13,5}{⾓、⽝}
  \begin{Phonetics}{触犯}{chu4fan4}[][HSK 7-9]
    \definition{v.}{ofender; violar; ir contra; infringir}
  \end{Phonetics}
\end{Entry}

\begin{Entry}{触目惊心}{13,5,11,4}{⾓、⽬、⼼、⼼}
  \begin{Phonetics}{触目惊心}{chu4mu4-jing1xin1}[][HSK 7-9]
    \definition{expr.}{ver a cena que é terrível para a mente; atingir os olhos e despertar a mente; uma visão medonha; horripilante; chocante (para as pessoas); assustador | (uma cena) surpreendente; chocante}
  \end{Phonetics}
\end{Entry}

\begin{Entry}{触动}{13,6}{⾓、⼒}
  \begin{Phonetics}{触动}{chu4dong4}[][HSK 7-9]
    \definition{v.}{tocar em algo; tocar (um interruptor, uma tela, etc.) | comover alguém; despertar os sentimentos de alguém; ter uma mudança emocional causada por algum estímulo; ser movido | afetar; um comportamento que afeta, perturba ou prejudica outras pessoas}
  \end{Phonetics}
\end{Entry}

\begin{Entry}{触觉}{13,9}{⾓、⾒}
  \begin{Phonetics}{触觉}{chu4jue2}[][HSK 7-9]
    \definition{s.}{sensação tátil; sentido do tato | tato; recepção do toque; tigmestesia; pselaphesia; pselaphesis; tactilidade}
  \end{Phonetics}
\end{Entry}

\begin{Entry}{触摸}{13,13}{⾓、⼿}
  \begin{Phonetics}{触摸}{chu4mo1}[][HSK 7-9]
    \definition{s.}{tocar; acariciar brevemente uma parte do corpo}
  \end{Phonetics}
\end{Entry}

%%%%% EOF %%%%%

