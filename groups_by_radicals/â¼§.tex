%%%
%%% Radical "⼧"
%%%

\section*{Radical 40: ``⼧''}\addcontentsline{toc}{section}{Radical 40: ⼧}

\begin{Entry}{宁}{5}{⼧}
  \begin{Phonetics}{宁}{ning2}
    \definition*{s.}{Região Autônoma de Ningxia Hui, abreviação de 宁夏回族自治区 | outro nome para Nanquim, 南京 | Sobrenome Ning}
    \definition{adj.}{calmo, pacífico, sereno | saudável}
    \definition{v.}{Literário: fazer uma visita (aos pais ou aos mais velhos); | Literário: pacificar; apaziguar}
  \seealsoref{南京}{nan2jing1}
  \seealsoref{宁夏回族自治区}{ning2xia4 hui2zu2 zi4zhi4qu1}
  \end{Phonetics}
  \begin{Phonetics}{宁}{ning4}
    \definition{conj.}{mais\dots do que\dots, melhor\dots do que\dots}
  \end{Phonetics}
\end{Entry}

\begin{Entry}{宁可}{5,5}{⼧、⼝}
  \begin{Phonetics}{宁可}{ning4ke3}
    \definition{conj.}{mais\dots do que\dots | melhor\dots do que\dots}
  \end{Phonetics}
\end{Entry}

\begin{Entry}{宁可……也不……}{5,5,3,4}{⼧、⼝、⼄、⼀}
  \begin{Phonetics}{宁可……也不……}{ning4ke3 ye3bu4}
    \definition{conj.}{preferiria\dots do que\dots}
  \end{Phonetics}
\end{Entry}

\begin{Entry}{宁可……也要……}{5,5,3,9}{⼧、⼝、⼄、⾑}
  \begin{Phonetics}{宁可……也要……}{ning4ke3 ye3yao4}
    \definition{conj.}{mesmo que tenhamos que\dots nós iremos\dots}
  \end{Phonetics}
\end{Entry}

\begin{Entry}{宁肯}{5,8}{⼧、⾁}
  \begin{Phonetics}{宁肯}{ning4ken3}
    \definition{conj.}{mais\dots do que\dots, melhor\dots do que\dots}
  \end{Phonetics}
\end{Entry}

\begin{Entry}{宁夏回族自治区}{5,10,6,11,6,8,4}{⼧、⼢、⼞、⽅、⾃、⽔、⼖}
  \begin{Phonetics}{宁夏回族自治区}{ning2xia4 hui2zu2 zi4zhi4qu1}
    \definition*{s.}{Região Autônoma de Ningxia Hui}
  \end{Phonetics}
\end{Entry}

\begin{Entry}{宁愿}{5,14}{⼧、⽕}
  \begin{Phonetics}{宁愿}{ning4yuan4}
    \definition{conj.}{mais\dots do que\dots, melhor\dots do que\dots}
  \end{Phonetics}
\end{Entry}

\begin{Entry}{宁静}{5,14}{⼧、⾭}
  \begin{Phonetics}{宁静}{ning2 jing4}[][HSK 4]
    \definition{adj.}{calmo; tranquilo; pacífico}
  \end{Phonetics}
\end{Entry}

\begin{Entry}{它}{5}{⼧}
  \begin{Phonetics}{它}{ta1}[][HSK 2]
    \definition*{s.}{Sobrenome Ta}
    \definition{pron.}{ele; referência a algo além da pessoa (para objetos inanimados) | ele; usado após o verbo, indica referência vaga}
  \end{Phonetics}
\end{Entry}

\begin{Entry}{它们}{5,5}{⼧、⼈}
  \begin{Phonetics}{它们}{ta1 men5}[][HSK 2]
    \definition{pron.}{eles; usado para se referir a mais de uma coisa não humana; geralmente se refere a animais, objetos ou conceitos abstratos}
  \end{Phonetics}
\end{Entry}

\begin{Entry}{宇}{6}{⼧}
  \begin{Phonetics}{宇}{yu3}
    \definition*{s.}{Sobrenome Yu}
    \definition[座,栋]{s.}{beirais; calha; casa | espaço; universo; mundo | postura; porte}
  \end{Phonetics}
\end{Entry}

\begin{Entry}{宇宙}{6,8}{⼧、⼧}
  \begin{Phonetics}{宇宙}{yu3zhou4}
    \definition{s.}{universo | cosmos}
  \end{Phonetics}
\end{Entry}

\begin{Entry}{宇航员}{6,10,7}{⼧、⾈、⼝}
  \begin{Phonetics}{宇航员}{yu3 hang2 yuan2}[][HSK 6]
    \definition[位,名,个,些]{s.}{astronauta; cosmonauta;}
  \end{Phonetics}
\end{Entry}

\begin{Entry}{守}{6}{⼧}
  \begin{Phonetics}{守}{shou3}[][HSK 4]
    \definition*{s.}{Sobrenome Shou}
    \definition{adv.}{próximo; perto de; perto de algum lugar em posição, perto de algum lugar}
    \definition{v.}{guardar; defender; estar presente para cuidar; não ir embora | manter vigilância; defender do ataque do oponente em uma luta ou confronto | observar; cumprir; respeitar; fazer as coisas como elas devem ser feitas | manter, observar a integridade; honrar a palavra de alguém; manter a palavra de alguém}
  \end{Phonetics}
\end{Entry}

\begin{Entry}{守门员}{6,3,7}{⼧、⾨、⼝}
  \begin{Phonetics}{守门员}{shou3men2yuan2}
    \definition{s.}{goleiro}
  \end{Phonetics}
\end{Entry}

\begin{Entry}{安}{6}{⼧}
  \begin{Phonetics}{安}{an1}[][HSK 4]
    \definition*{s.}{Sobrenome An}
    \definition{adj.}{pacífico; quieto; tranquilo; calmo | seguro; protegido (oposto a 危) | com boa saúde | em paz; bem}
    \definition{adv.}{pacificamente; silenciosamente | com segurança; em segurança | em perguntas retóricas: como?}
    \definition{pron.}{usado como pronome interrogativo, como em 哪里,怎么; 谁,何,如何}
    \definition{s.}{segurança; proteção; paz | ampère; abreviação de ampère, 安培}
    \definition{v.}{tranquilizar (a mente de alguém); acalmar | contentar-se; ficar satisfeito | colocar em uma posição adequada; encontrar um lugar para | instalar; consertar; encaixar; configurar | trazer (uma acusação contra alguém); dar (a alguém um apelido); reivindicar (crédito por algo) | abrigar (uma intenção) | acalmar; estabilizar | sentir-se satisfeito e à vontade}
  \seealsoref{安培}{an1pei2}
  \seealsoref{何}{he2}
  \seealsoref{哪里}{na3 li3}
  \seealsoref{如何}{ru2he2}
  \seealsoref{谁}{shei2}
  \seealsoref{危}{wei1}
  \seealsoref{怎么}{zen3me5}
  \end{Phonetics}
\end{Entry}

\begin{Entry}{安心}{6,4}{⼧、⼼}
  \begin{Phonetics}{安心}{an1xin1}[][HSK 7-9]
    \definition{adj.}{aliviado; à vontade; tranquilo; seguro}
    \definition{v.}{abrigar (más) intenções; acalentar certas intenções; ter (pensamentos ruins) em mente}
  \end{Phonetics}
\end{Entry}

\begin{Entry}{安宁}{6,5}{⼧、⼧}
  \begin{Phonetics}{安宁}{an1ning2}[][HSK 7-9]
    \definition{adj.}{pacífico; tranquilo; descreve um estado de ordem normal sem fatores externos que causem desordem ou inquietação | calmo; composto; livre de preocupações; sem preocupações, ansiedades ou inquietações}
  \end{Phonetics}
\end{Entry}

\begin{Entry}{安全}{6,6}{⼧、⼊}
  \begin{Phonetics}{安全}{an1quan2}[][HSK 2]
    \definition{adj.}{seguro; protegido; sem perigo; sem ameaças; sem acidentes}
    \definition{s.}{segurança; proteção; refere-se a um estado ou conceito, geralmente indicando ausência de ameaças ou perigo}
  \end{Phonetics}
\end{Entry}

\begin{Entry}{安抚}{6,7}{⼧、⼿}
  \begin{Phonetics}{安抚}{an1fu3}[][HSK 7-9]
    \definition{v.}{pacificar; consolar; apaziguar; tranquilizar e acalmar}
  \end{Phonetics}
\end{Entry}

\begin{Entry}{安定}{6,8}{⼧、⼧}
  \begin{Phonetics}{安定}{an1ding4}[][HSK 7-9]
    \definition{adj.}{estável; tranquilo; estabelecido; pacífico e em ordem; estável e normal, sem flutuações}
    \definition{s.}{tranquilizante; medicina ocidental comumente usada, com efeitos sedativos e anticonvulsivantes}
    \definition{v.}{acalmar; estabilizar; manter}
  \end{Phonetics}
\end{Entry}

\begin{Entry}{安神}{6,9}{⼧、⽰}
  \begin{Phonetics}{安神}{an1/shen2}
    \definition{v.+compl.}{acalmar os nervos | aliviar a inquietação pela tranquilização da mente e do corpo}
  \end{Phonetics}
\end{Entry}

\begin{Entry}{安家}{6,10}{⼧、⼧}
  \begin{Phonetics}{安家}{an1/jia1}
    \definition{v.+compl.}{montar uma casa | estabelecer-se}
  \end{Phonetics}
\end{Entry}

\begin{Entry}{安眠药}{6,10,9}{⼧、⽬、⾋}
  \begin{Phonetics}{安眠药}{an1mian2yao4}[][HSK 7-9]
    \definition[片,颗,粒,瓶]{s.}{comprimido para dormir; soporífero; pílula para dormir; medicamentos que podem suprimir o córtex cerebral e induzir o sono}
  \end{Phonetics}
\end{Entry}

\begin{Entry}{安培}{6,11}{⼧、⼟}
  \begin{Phonetics}{安培}{an1pei2}
    \definition{clas.}{A; empréstimo linguístico: ampere; física: unidade de corrente elétrica}
  \end{Phonetics}
\end{Entry}

\begin{Entry}{安排}{6,11}{⼧、⼿}
  \begin{Phonetics}{安排}{an1pai2}[][HSK 3]
    \definition{s.}{plano; programação; organização; tabela do plano de atividades ou horários}
    \definition{v.}{organizar (assuntos) de acordo com a sequência ou regras; tratar as coisas de acordo com uma determinada ordem ou regras | atribuir tarefas a alguém; colocar as pessoas nos cargos de trabalho determinados, conforme planejado}
  \end{Phonetics}
\end{Entry}

\begin{Entry}{安检}{6,11}{⼧、⽊}
  \begin{Phonetics}{安检}{an1 jian3}[][HSK 6]
    \definition{s.}{verificação de segurança}
    \definition{v.}{realizar verificação de segurança}
  \end{Phonetics}
\end{Entry}

\begin{Entry}{安逸}{6,11}{⼧、⾡}
  \begin{Phonetics}{安逸}{an1yi4}[][HSK 7-9]
    \definition{adj.}{fácil; fácil e confortável; relaxado e confortável}
    \definition{s.}{conforto e facilidade; conforto}
  \end{Phonetics}
\end{Entry}

\begin{Entry}{安装}{6,12}{⼧、⾐}
  \begin{Phonetics}{安装}{an1zhuang1}[][HSK 3]
    \definition{v.}{instalar; consertar; configurar; fixar máquinas ou equipamentos (geralmente conjuntos) em um determinado local, de acordo com métodos e especificações específicos}
  \end{Phonetics}
\end{Entry}

\begin{Entry}{安置}{6,13}{⼧、⽹}
  \begin{Phonetics}{安置}{an1zhi4}[][HSK 4]
    \definition{v.}{providenciar; encontrar um lugar para; ajudar a estabelecer-se; colocar pessoas ou coisas em uma determinada posição ou organizá-las adequadamente}
  \end{Phonetics}
\end{Entry}

\begin{Entry}{安稳}{6,14}{⼧、⽲}
  \begin{Phonetics}{安稳}{an1wen3}[][HSK 7-9]
    \definition{adj.}{seguro; suave e estável; estável | composto; calmo e equilibrado; (comportamento) estável; calmo}
  \end{Phonetics}
\end{Entry}

\begin{Entry}{安静}{6,14}{⼧、⾭}
  \begin{Phonetics}{安静}{an1jing4}[][HSK 2]
    \definition{adj.}{silencioso; tranquilo; sem som; sem barulho e sem algazarra}
  \end{Phonetics}
\end{Entry}

\begin{Entry}{安慰}{6,15}{⼧、⼼}
  \begin{Phonetics}{安慰}{an1wei4}[][HSK 5]
    \definition{adj.}{confortar; tranquilizar; consolar; apaziguar;}
    \definition[句,通,番,声,个]{s.}{conforto; consolo; comportamento que alivia a dor de alguém e o acalma com palavras ou gestos}
    \definition{v.}{confortar; consolar; acalmar e confortar; deixar a mente tranquila}
  \end{Phonetics}
\end{Entry}

\begin{Entry}{宋}{7}{⼧}
  \begin{Phonetics}{宋}{song4}
    \definition*{s.}{Dinastia Song (960-1279) | Song das dinastias do sul (420-479) | Sobrenome Song}
    \definition{clas.}{sone; unidade de intensidade sonora}
  \end{Phonetics}
\end{Entry}

\begin{Entry}{完}{7}{⼧}
  \begin{Phonetics}{完}{wan2}[][HSK 2]
    \definition*{s.}{Sobrenome Wan}
    \definition{adj.}{inteiro; intacto; completo}
    \definition{v.}{acabar; terminar; completar | pagar | estar terminado; estar pronto para | esgotar; ser usado}
  \end{Phonetics}
\end{Entry}

\begin{Entry}{完了}{7,2}{⼧、⼅}
  \begin{Phonetics}{完了}{wan2 le5}[][HSK 5]
    \definition{v.}{acabar; terminar; concluir; chegar ao fim}
  \end{Phonetics}
\end{Entry}

\begin{Entry}{完人}{7,2}{⼧、⼈}
  \begin{Phonetics}{完人}{wan2ren2}
    \definition{s.}{pessoa perfeita}
  \end{Phonetics}
\end{Entry}

\begin{Entry}{完全}{7,6}{⼧、⼊}
  \begin{Phonetics}{完全}{wan2quan2}[][HSK 2]
    \definition{adj.}{inteiro; completo; não falta nada, está tudo completo}
    \definition{adv.}{completamente; representa tudo}
  \end{Phonetics}
\end{Entry}

\begin{Entry}{完成}{7,6}{⼧、⼽}
  \begin{Phonetics}{完成}{wan2cheng2}[][HSK 2]
    \definition{v.}{realizar; completar; terminar; cumprir; levar ao sucesso}
  \end{Phonetics}
\end{Entry}

\begin{Entry}{完毕}{7,6}{⼧、⽐}
  \begin{Phonetics}{完毕}{wan2bi4}
    \definition{v.}{completar | terminar | acabar}
  \end{Phonetics}
\end{Entry}

\begin{Entry}{完完全全}{7,7,6,6}{⼧、⼧、⼊、⼊}
  \begin{Phonetics}{完完全全}{wan2wan2quan2quan2}
    \definition{adv.}{completamente}
  \end{Phonetics}
\end{Entry}

\begin{Entry}{完备}{7,8}{⼧、⼡}
  \begin{Phonetics}{完备}{wan2bei4}
    \definition{adj.}{completo | impecável | perfeito}
    \definition{v.}{não deixar nada a desejar}
  \end{Phonetics}
\end{Entry}

\begin{Entry}{完美}{7,9}{⼧、⽺}
  \begin{Phonetics}{完美}{wan2mei3}[][HSK 3]
    \definition{adj.}{perfeito; impecável; consumado}
  \end{Phonetics}
\end{Entry}

\begin{Entry}{完善}{7,12}{⼧、⼝}
  \begin{Phonetics}{完善}{wan2shan4}[][HSK 3]
    \definition{adj.}{perfeito; consumado}
    \definition{v.}{refinar; melhorar; tornar perfeito}
  \end{Phonetics}
\end{Entry}

\begin{Entry}{完税}{7,12}{⼧、⽲}
  \begin{Phonetics}{完税}{wan2shui4}
    \definition{v.}{pagar imposto}
  \end{Phonetics}
\end{Entry}

\begin{Entry}{完满}{7,13}{⼧、⽔}
  \begin{Phonetics}{完满}{wan2man3}
    \definition{adj.}{satisfatório | bem-sucedido}
  \end{Phonetics}
\end{Entry}

\begin{Entry}{完整}{7,16}{⼧、⽁}
  \begin{Phonetics}{完整}{wan2zheng3}[][HSK 3]
    \definition{adj.}{intacto; inteiro; completo; integrado; nenhum dano ou mutilação}
  \end{Phonetics}
\end{Entry}

\begin{Entry}{宏}{7}{⼧}
  \begin{Phonetics}{宏}{hong2}
    \definition*{s.}{Sobrenome Hong}
    \definition{adj.}{grande; grandioso; magnífico}
    \definition{v.}{divulgar algo; promover algo; atualmente, geralmente é escrito como 弘}
  \seealsoref{弘}{hong2}
  \end{Phonetics}
\end{Entry}

\begin{Entry}{宏大}{7,3}{⼧、⼤}
  \begin{Phonetics}{宏大}{hong2 da4}[][HSK 6]
    \definition{adj.}{grande; ótimo | imenso; vasto}
  \end{Phonetics}
\end{Entry}

\begin{Entry}{宏伟}{7,6}{⼧、⼈}
  \begin{Phonetics}{宏伟}{hong2wei3}[][HSK 7-9]
    \definition{adj.}{grandioso; magnífico; (escala, planta, etc.) magnífico e grandioso}
  \end{Phonetics}
\end{Entry}

\begin{Entry}{宏观}{7,6}{⼧、⾒}
  \begin{Phonetics}{宏观}{hong2guan1}[][HSK 7-9]
    \definition{adj.}{macroscópico; grande norma, relacionada ao todo}
    \definition{pref.}{macro-}
  \end{Phonetics}
\end{Entry}

\begin{Entry}{牢}{7}{⼧}
  \begin{Phonetics}{牢}{lao2}[][HSK 6]
    \definition*{s.}{Sobrenome Lao}
    \definition{adj.}{firme; durável}
    \definition{s.}{prisão; cadeia | (cercado para animais) curral; baia; galinheiro; estábulo; estrebaria; cocheira | (arcaico) animal de sacrifício}
  \end{Phonetics}
\end{Entry}

\begin{Entry}{宗}{8}{⼧}
  \begin{Phonetics}{宗}{zong1}
    \definition*{s.}{Sobrenome Zong}
    \definition{adj.}{do mesmo clã; da mesma família}
    \definition{clas.}{usado para matérias, cargas, etc.}
    \definition{s.}{ancestral; antepassado | clã; família | seita; facção; escola | objetivo principal; propósito | modelo; grande mestre | Datado: unidade administrativa no Tibete, equivalente a um condado | templo ancestral}
    \definition{v.}{(no trabalho acadêmico ou artístico) tomar como modelo; modelar-se em}
  \end{Phonetics}
\end{Entry}

\begin{Entry}{宗教}{8,11}{⼧、⽁}
  \begin{Phonetics}{宗教}{zong1jiao4}[][HSK 6]
    \definition[种]{s.}{religião; uma ideologia social é um reflexo ilusório do mundo objetivo, exigindo que as pessoas acreditem em Deus, no Xintoísmo, em espíritos, no carma, etc., e que depositem suas esperanças no chamado céu ou vida após a morte}
  \end{Phonetics}
\end{Entry}

\begin{Entry}{官}{8}{⼧}
  \begin{Phonetics}{官}{guan1}[][HSK 4]
    \definition*{s.}{Sobrenome Guan}
    \definition{adj.}{propriedade do governo; pertencente ao governo ou ao público | público}
    \definition[个,位,名,些]{s.}{funcionário do governo; oficial; servidor público; titular de cargo; funcionário público nomeado acima de um determinado nível | órgão (parte do tecido do corpo)}
  \end{Phonetics}
\end{Entry}

\begin{Entry}{官方}{8,4}{⼧、⽅}
  \begin{Phonetics}{官方}{guan1fang1}[][HSK 4]
    \definition{s.}{autoridade; (do ou pelo) governo | oficial (de uma organização ou instituição)}
  \end{Phonetics}
\end{Entry}

\begin{Entry}{官司}{8,5}{⼧、⼝}
  \begin{Phonetics}{官司}{guan1 si5}[][HSK 6]
    \definition[场,个]{s.}{ação judicial}
  \end{Phonetics}
\end{Entry}

\begin{Entry}{官吏}{8,6}{⼧、⼝}
  \begin{Phonetics}{官吏}{guan1li4}[][HSK 7-9]
    \definition{s.}{funcionários do governo | burocrata | oficial}
  \end{Phonetics}
\end{Entry}

\begin{Entry}{官兵}{8,7}{⼧、⼋}
  \begin{Phonetics}{官兵}{guan1bing1}[][HSK 7-9]
    \definition{s.}{oficiais e soldados | Datado: tropas governamentais}
  \end{Phonetics}
\end{Entry}

\begin{Entry}{官员}{8,7}{⼧、⼝}
  \begin{Phonetics}{官员}{guan1yuan2}[][HSK 7-9]
    \definition[名,位]{s.}{oficial; funcionários do governo de um determinado nível}
  \end{Phonetics}
\end{Entry}

\begin{Entry}{官桂}{8,10}{⼧、⽊}
  \begin{Phonetics}{官桂}{guan1gui4}
    \definition{s.}{canela; também escrito como 肉桂}
  \seealsoref{肉桂}{rou4gui4}
  \end{Phonetics}
\end{Entry}

\begin{Entry}{官僚}{8,14}{⼧、⼈}
  \begin{Phonetics}{官僚}{guan1liao2}[][HSK 7-9]
    \definition{s.}{burocrata | burocracia | oficial}
  \end{Phonetics}
\end{Entry}

\begin{Entry}{官僚主义}{8,14,5,3}{⼧、⼈、⼂、⼂}
  \begin{Phonetics}{官僚主义}{guan1liao2 zhu3yi4}[][HSK 7-9]
    \definition{s.}{burocracia; burocratismo}
  \end{Phonetics}
\end{Entry}

\begin{Entry}{定}{8}{⼧}
  \begin{Phonetics}{定}{ding4}[][HSK 4]
    \definition{adj.}{calmo; estável}
    \definition{adv.}{certamente; com certeza; definitivamente; espressa certeza ou necessidade}
    \definition{v.}{decidir; fixar; definir; determinar; ter certeza | acalmar; estabilizar; tornar estável | assinar (um jornal, etc.); reservar (assentos, ingressos, etc.); encomendar (mercadorias, etc.)}
  \end{Phonetics}
\end{Entry}

\begin{Entry}{定义}{8,3}{⼧、⼂}
  \begin{Phonetics}{定义}{ding4yi4}[][HSK 7-9]
    \definition[个,种]{s.}{definição; delimitação; uma descrição precisa e concisa das características essenciais de uma coisa ou da conotação e extensão de um conceito}
    \definition{v.}{definir}
  \end{Phonetics}
\end{Entry}

\begin{Entry}{定为}{8,4}{⼧、⼂}
  \begin{Phonetics}{定为}{ding4wei2}[][HSK 7-9]
    \definition{v.}{prescrever como; estar marcado para}
  \end{Phonetics}
\end{Entry}

\begin{Entry}{定心丸}{8,4,3}{⼧、⼼、⼂}
  \begin{Phonetics}{定心丸}{ding4xin1wan2}[][HSK 7-9]
    \definition{s.}{algo que tranquiliza a mente; alívio | algo capaz de tranquilizar a mente de alguém; algo que acalma os nervos; tranquiliza a mente (paz); palavras ou ações que podem acalmar pensamentos e emoções}
  \end{Phonetics}
\end{Entry}

\begin{Entry}{定价}{8,6}{⼧、⼈}
  \begin{Phonetics}{定价}{ding4 jia4}[][HSK 6]
    \definition{s.}{fixação de preços; preço especificado}
    \definition{v.}{fixar um preço | fazer um preço; definir um preço}
  \end{Phonetics}
\end{Entry}

\begin{Entry}{定向}{8,6}{⼧、⼝}
  \begin{Phonetics}{定向}{ding4xiang4}[][HSK 7-9]
    \definition{adj.}{direcional; orientado}
    \definition{s.}{orientação; sentido de orientação; orientação predeterminada}
    \definition{v.}{orientar}
  \end{Phonetics}
\end{Entry}

\begin{Entry}{定论}{8,6}{⼧、⾔}
  \begin{Phonetics}{定论}{ding4lun4}[][HSK 7-9]
    \definition{s.}{conclusão final | veredito final}
    \definition{v.}{concluir; chegar a uma conclusão (ou julgamento)}
  \end{Phonetics}
\end{Entry}

\begin{Entry}{定位}{8,7}{⼧、⼈}
  \begin{Phonetics}{定位}{ding4 wei4}[][HSK 6]
    \definition{s.}{posição; localização; posição medida ou definida}
    \definition{v.}{localizar; posicionar; orientar; avaliar algo; usar instrumentos para determinar a localização de objetos; definir o \emph{status} das coisas}
  \end{Phonetics}
\end{Entry}

\begin{Entry}{定时}{8,7}{⼧、⽇}
  \begin{Phonetics}{定时}{ding4 shi2}[][HSK 6]
    \definition{s.}{em um horário fixo; em intervalos regulares}
    \definition{v.}{cronometrar; fixar um tempo (para fazer algo)}
  \end{Phonetics}
\end{Entry}

\begin{Entry}{定居}{8,8}{⼧、⼫}
  \begin{Phonetics}{定居}{ding4/ju1}[][HSK 7-9]
    \definition{v.+compl.}{estabelecer-se; fixar residência; viver permanentemente em um determinado lugar}
  \end{Phonetics}
\end{Entry}

\begin{Entry}{定金}{8,8}{⼧、⾦}
  \begin{Phonetics}{定金}{ding4jin1}[][HSK 7-9]
    \definition{s.}{sinal; depósito; o mesmo que 订金}
  \seealsoref{订金}{ding4jin1}
  \end{Phonetics}
\end{Entry}

\begin{Entry}{定做}{8,11}{⼧、⼈}
  \begin{Phonetics}{定做}{ding4zuo4}[][HSK 7-9]
    \definition{v.}{ter algo feito sob encomenda (medida); feito sob medida; personalizar}
  \end{Phonetics}
\end{Entry}

\begin{Entry}{定期}{8,12}{⼧、⽉}
  \begin{Phonetics}{定期}{ding4qi1}[][HSK 3]
    \definition{adj.}{regular; periódico; em intervalos regulares; com prazo determinado; por tempo limitado}
    \definition{v.}{fixar (definir) uma data; determinar a data; confirmar a data}
  \end{Phonetics}
\end{Entry}

\begin{Entry}{宝}{8}{⼧}
  \begin{Phonetics}{宝}{bao3}[][HSK 4]
    \definition*{s.}{Sobrenome Bao}
    \definition{adj.}{antigo; precioso; estimado}
    \definition{pron.}{estimado; um termo educado usado para se referir à família, loja, etc. de alguém}
    \definition[个,件]{s.}{tesouro; objeto estimado; coisa preciosa | dinheiro; moeda; moeda antiga com furo quadrado no centro; moeda de prata}
  \end{Phonetics}
\end{Entry}

\begin{Entry}{宝贝}{8,4}{⼧、⾙}
  \begin{Phonetics}{宝贝}{bao3bei4}[][HSK 4]
    \definition{adj.}{excêntrico; estranho; imprestável; um termo depreciativo para uma pessoa incompetente ou ridícula}
    \definition[个,件]{s.}{tesouro; objeto estimado; coisa preciosa | querida; \emph{darling}; \emph{baby}; apelido para crianças}
  \end{Phonetics}
\end{Entry}

\begin{Entry}{宝石}{8,5}{⼧、⽯}
  \begin{Phonetics}{宝石}{bao3 shi2}[][HSK 4]
    \definition[颗,枚,块,粒]{s.}{gema; jóia; pedra preciosa; mineral precioso que tem um brilho lindo e uma dureza de mais de sete graus, não é afetado pela atmosfera ou por produtos químicos e pode ser usado como decoração, suporte de instrumentos ou abrasivos}
  \end{Phonetics}
\end{Entry}

\begin{Entry}{宝库}{8,7}{⼧、⼴}
  \begin{Phonetics}{宝库}{bao3ku4}[][HSK 7-9]
    \definition[座,个]{s.}{tesouro; casa de tesouro; um lugar onde coisas preciosas são armazenadas (frequentemente usado metaforicamente)}[图书馆是知识的宝库。===A biblioteca é um tesouro de conhecimento.]
  \end{Phonetics}
\end{Entry}

\begin{Entry}{宝宝}{8,8}{⼧、⼧}
  \begin{Phonetics}{宝宝}{bao3 bao5}[][HSK 4]
    \definition[个,位]{s.}{querida; \emph{darling}; \emph{baby}; apelido para crianças}
  \end{Phonetics}
\end{Entry}

\begin{Entry}{宝贵}{8,9}{⼧、⾙}
  \begin{Phonetics}{宝贵}{bao3gui4}[][HSK 4]
    \definition{adj.}{precioso; extremamente valioso, muito raro, pode ser usado para descrever coisas específicas, também pode ser usado para descrever coisas abstratas | valioso; como um tesouro}
  \end{Phonetics}
\end{Entry}

\begin{Entry}{宝藏}{8,17}{⼧、⾋}
  \begin{Phonetics}{宝藏}{bao3zang4}[][HSK 7-9]
    \definition[座,个]{s.}{depósitos preciosos (minerais); tesouros ou riquezas armazenadas, principalmente minerais}
  \end{Phonetics}
\end{Entry}

\begin{Entry}{实}{8}{⼧}
  \begin{Phonetics}{实}{shi2}
    \definition{adj.}{sólido; cheio por dentro; sem espaços vazios (oposto de 虚) | verdadeiro; real; atual; sincero | forte; eficaz; concreto; real}
    \definition{adv.}{verdadeiramente; realmente; de fato; originalmente}
    \definition{s.}{fato; realidade | semente; fruto}
    \definition{v.}{preencher}
  \seealsoref{虚}{xu1}
  \end{Phonetics}
\end{Entry}

\begin{Entry}{实力}{8,2}{⼧、⼒}
  \begin{Phonetics}{实力}{shi2li4}[][HSK 3]
    \definition{s.}{força real; geralmente se refere à força militar e econômica de um país, grupo ou indivíduo, e também se refere à capacidade de um indivíduo ou grupo em uma competição}
  \end{Phonetics}
\end{Entry}

\begin{Entry}{实习}{8,3}{⼧、⼄}
  \begin{Phonetics}{实习}{shi2xi2}[][HSK 2]
    \definition{s.}{estagiário; prática; estágio}
    \definition{v.}{aplicar e testar os conhecimentos teóricos aprendidos no trabalho prático, a fim de exercitar a capacidade profissional}
  \end{Phonetics}
\end{Entry}

\begin{Entry}{实用}{8,5}{⼧、⽤}
  \begin{Phonetics}{实用}{shi2yong4}[][HSK 4]
    \definition{adj.}{prático; pragmático; funcional; atende aos requisitos reais da aplicação}
    \definition{v.}{colocar em uso prático}
  \end{Phonetics}
\end{Entry}

\begin{Entry}{实在}{8,6}{⼧、⼟}
  \begin{Phonetics}{实在}{shi2zai4}[][HSK 2]
    \definition{adj.}{honesto; sincero | verdadeiro; honesto; realista; não é falso, não é enganador}
    \definition{adv.}{verdadeiramente; de fato; na verdade; usado para reforçar o tom afirmativo, enfatizando que a situação é realmente assim}
  \end{Phonetics}
\end{Entry}

\begin{Entry}{实行}{8,6}{⼧、⾏}
  \begin{Phonetics}{实行}{shi2xing2}[][HSK 3]
    \definition{v.}{praticar; implementar; executar; colocar em prática; realizar (programa, política, plano, etc.) por meio de ação}
  \end{Phonetics}
\end{Entry}

\begin{Entry}{实际}{8,7}{⼧、⾩}
  \begin{Phonetics}{实际}{shi2ji4}[][HSK 2]
    \definition{adj.}{real; efetivo; concreto; prático | factual; prático; realista; de acordo com os fatos}
    \definition{s.}{realidade; prática; coisas e situações que existem objetivamente}
  \end{Phonetics}
\end{Entry}

\begin{Entry}{实际上}{8,7,3}{⼧、⾩、⼀}
  \begin{Phonetics}{实际上}{shi2 ji4 shang4}[][HSK 3]
    \definition{adv.}{de fato; na verdade}
  \end{Phonetics}
\end{Entry}

\begin{Entry}{实现}{8,8}{⼧、⾒}
  \begin{Phonetics}{实现}{shi2xian4}[][HSK 2]
    \definition{v.}{alcançar; atingir; realizar; concretizar; tornar (ideais, planos, etc.) realidade}
  \end{Phonetics}
\end{Entry}

\begin{Entry}{实施}{8,9}{⼧、⽅}
  \begin{Phonetics}{实施}{shi2shi1}[][HSK 4]
    \definition{v.}{colocar em vigor; implementar (leis, políticas, etc.); executar; trazer (colocar) algo em vigor; fazer cumprir; colocar algo em (prática)}
  \end{Phonetics}
\end{Entry}

\begin{Entry}{实验}{8,10}{⼧、⾺}
  \begin{Phonetics}{实验}{shi2yan4}[][HSK 3]
    \definition[个,次]{s.}{teste; experimento; trabalho de laboratório}
    \definition{v.}{testar; experimentar; realizar uma operação ou se envolver em uma atividade para testar uma teoria ou hipótese científica}
  \end{Phonetics}
\end{Entry}

\begin{Entry}{实验室}{8,10,9}{⼧、⾺、⼧}
  \begin{Phonetics}{实验室}{shi2 yan4 shi4}[][HSK 3]
    \definition[个,间]{s.}{laboratório; salas especiais para experimentos científicos}
  \end{Phonetics}
\end{Entry}

\begin{Entry}{实惠}{8,12}{⼧、⼼}
  \begin{Phonetics}{实惠}{shi2hui4}[][HSK 5]
    \definition{adj.}{sólido; substancial; benefícios práticos}
    \definition{s.}{benefício material; benefícios tangíveis; benefícios reais}
  \end{Phonetics}
\end{Entry}

\begin{Entry}{实践}{8,12}{⼧、⾜}
  \begin{Phonetics}{实践}{shi2jian4}[][HSK 6]
    \definition{s.}{prática; filosoficamente, refere-se às ações conscientes das pessoas para transformar a natureza e a sociedade; as atividades de produção são as atividades práticas mais básicas e também incluem atividades políticas, experimentos científicos, educação cultural, etc.}
    \definition{v.}{praticar; realizar; implementar planos e intenções em ações específicas}
  \end{Phonetics}
\end{Entry}

\begin{Entry}{宠}{8}{⼧}
  \begin{Phonetics}{宠}{chong3}[][HSK 7-9]
    \definition*{s.}{Sobrenome Chong}
    \definition{v.}{mimar; estragar; conceder favor a | regalar; encontrar favor com alguém; estar nas boas graças de alguém}
  \end{Phonetics}
\end{Entry}

\begin{Entry}{宠物}{8,8}{⼧、⽜}
  \begin{Phonetics}{宠物}{chong3wu4}[][HSK 6]
    \definition[只]{s.}{animal de estimação; refere-se a pequenos animais criados na família}
  \end{Phonetics}
\end{Entry}

\begin{Entry}{宠爱}{8,10}{⼧、⽖}
  \begin{Phonetics}{宠爱}{chong3'ai4}[][HSK 7-9]
    \definition{v.}{mimar; fazer de alguém um animal de estimação}[爷爷总是宠爱他的孙子。===O avô sempre mima o neto.]
  \end{Phonetics}
\end{Entry}

\begin{Entry}{审}{8}{⼧}
  \begin{Phonetics}{审}{shen3}[][HSK 6]
    \definition*{s.}{Sobrenome Shen}
    \definition{adj.}{cuidadoso; detalhado; completo}
    \definition{adv.}{Literpario: realmente; de ​​fato; como esperado}
    \definition{v.}{examinar; analizar | julgar; interrogar | Literário: saber}
  \end{Phonetics}
\end{Entry}

\begin{Entry}{审查}{8,9}{⼧、⽊}
  \begin{Phonetics}{审查}{shen3cha2}[][HSK 6]
    \definition{v.}{examinar; investigar; verificar se algo está correto e apropriado (geralmente referindo-se a planos, propostas, escritos, qualificações pessoais, etc.); ler e avaliar (provas ou trabalhos de exame)}
  \end{Phonetics}
\end{Entry}

\begin{Entry}{客}{9}{⼧}
  \begin{Phonetics}{客}{ke4}
    \definition*{s.}{Sobrenome Ke}
    \definition{adj.}{objetivo; independente da consciência humana | estrangeiro; não desta região, unidade ou indústria}
    \definition{clas.}{porção (de comida, bebida, etc.); em algumas áreas, é usado para vender alimentos e bebidas em porções}
    \definition[个,位,名,些]{s.}{convidado; visitante; aquele que é convidado; aquele que vem visitar (em oposição a 主) | viajante; passageiro | comerciante viajante; refere-se especificamente a comerciantes que transportam mercadorias de um lugar para o outro | cliente; patrono; consumidor | uma pessoa envolvida em alguma atividade específica; pessoas que viajam fazendo algum tipo de atividade}
    \definition{v.}{ser um estranho; estabelecer-se (ou viver) em um lugar estranho; estar longe de casa ou morar no exterior}
  \seealsoref{主}{zhu3}
  \end{Phonetics}
\end{Entry}

\begin{Entry}{客人}{9,2}{⼧、⼈}
  \begin{Phonetics}{客人}{ke4ren2}[][HSK 2]
    \definition[位,个,桌,拨,批]{s.}{visitante; convidado | cliente; passageiro; hóspede; viajante}
  \end{Phonetics}
\end{Entry}

\begin{Entry}{客厅}{9,4}{⼧、⼚}
  \begin{Phonetics}{客厅}{ke4ting1}[][HSK 5]
    \definition[间,个]{s.}{sala de estar; sala de visitas; sala para receber convidados}
  \end{Phonetics}
\end{Entry}

\begin{Entry}{客户}{9,4}{⼧、⼾}
  \begin{Phonetics}{客户}{ke4hu4}[][HSK 5]
    \definition[位,个,家,批]{s.}{cliente; consumidor}
  \end{Phonetics}
\end{Entry}

\begin{Entry}{客气}{9,4}{⼧、⽓}
  \begin{Phonetics}{客气}{ke4qi5}[][HSK 5]
    \definition{adj.}{educado; modesto; cortês}
    \definition{v.}{ser educado; ser cortês; fazer comentários educados ou agir educadamente}
  \end{Phonetics}
\end{Entry}

\begin{Entry}{客车}{9,4}{⼧、⾞}
  \begin{Phonetics}{客车}{ke4 che1}[][HSK 6]
    \definition[辆,列,次,趟]{s.}{ônibus; veículo de passageiros; veículos que transportam passageiros em ferrovias e estradas}
  \end{Phonetics}
\end{Entry}

\begin{Entry}{客观}{9,6}{⼧、⾒}
  \begin{Phonetics}{客观}{ke4guan1}[][HSK 3]
    \definition{adj.}{objetivo; justo e razoável; imparcial; com base na situação real, sem preconceitos pessoais}
    \definition{s.}{objetivo; existe fora da consciência, sem depender da consciência subjetiva}
  \end{Phonetics}
\end{Entry}

\begin{Entry}{客运}{9,7}{⼧、⾡}
  \begin{Phonetics}{客运}{ke4yun4}
    \definition{s.}{transporte de passageiros; (setor de transporte) o negócio de transporte de passageiros}
  \end{Phonetics}
\end{Entry}

\begin{Entry}{宣}{9}{⼧}
  \begin{Phonetics}{宣}{xuan1}
    \definition*{s.}{Sobrenome Xuan}
    \definition{v.}{declarar; proclamar; anunciar; falar publicamente | drenar (líquidos)}
  \end{Phonetics}
\end{Entry}

\begin{Entry}{宣布}{9,5}{⼧、⼱}
  \begin{Phonetics}{宣布}{xuan1bu4}[][HSK 3]
    \definition{v.}{declarar; proclamar; pronunciar; anunciar; informar oficialmente a todos sobre as últimas decisões e situações}
  \end{Phonetics}
\end{Entry}

\begin{Entry}{宣传}{9,6}{⼧、⼈}
  \begin{Phonetics}{宣传}{xuan1chuan2}[][HSK 3]
    \definition[个]{v.}{propagar; divulgar; fazer propaganda; explicar e esclarecer às pessoas, para que elas acreditem e sigam as ações}
  \end{Phonetics}
\end{Entry}

\begin{Entry}{宣扬}{9,6}{⼧、⼿}
  \begin{Phonetics}{宣扬}{xuan1yang2}
    \definition{v.}{divulgar | anunciar | espalhar por toda parte}
  \end{Phonetics}
\end{Entry}

\begin{Entry}{室}{9}{⼧}
  \begin{Phonetics}{室}{shi4}[][HSK 3]
    \definition*{s.}{Shi, a décima terceira das vinte e oito constelações da esfera celeste, composta por duas estrelas em linha reta na constelação de Pégaso | Sobrenome Shi}
    \definition{s.}{sala; quarto; casa | departamento; sala como unidade administrativa ou de trabalho; órgãos públicos, fábricas, escolas e outras unidades de trabalho internas | esposa; familiares ou esposa | família; clã | cavidade; órgão com forma semelhante a uma câmara}
  \end{Phonetics}
\end{Entry}

\begin{Entry}{宪}{9}{⼧}
  \begin{Phonetics}{宪}{xian4}
    \definition*{s.}{Sobrenome Xian}
    \definition{s.}{estatuto; decreto | constituição}
  \end{Phonetics}
\end{Entry}

\begin{Entry}{宪制}{9,8}{⼧、⼑}
  \begin{Phonetics}{宪制}{xian4zhi4}
    \definition{adj.}{constitucional}
    \definition{s.}{sistema de governo constitucional}
  \end{Phonetics}
\end{Entry}

\begin{Entry}{宪法法院}{9,8,8,9}{⼧、⽔、⽔、⾩}
  \begin{Phonetics}{宪法法院}{xian4fa3fa3yuan4}
    \definition{s.}{tribunal constitucional}
  \end{Phonetics}
\end{Entry}

\begin{Entry}{宪政}{9,9}{⼧、⽁}
  \begin{Phonetics}{宪政}{xian4zheng4}
    \definition{s.}{governo constitucional}
  \end{Phonetics}
\end{Entry}

\begin{Entry}{宫}{9}{⼧}
  \begin{Phonetics}{宫}{gong1}[][HSK 6]
    \definition*{s.}{Sobrenome Gong}
    \definition[座]{s.}{palácio imperial; palácio; casas onde o imperador, a imperatriz, o príncipe, etc. vivem | morada de seres sobrenaturais; palácio; paraíso; casas onde vivem os deuses na mitologia | templo (usado em um nome de templo) | local para atividades culturais e recreativas; um edifício para atividades culturais e recreativas; casas para fins culturais e de entretenimento | útero | uma nota da antiga escala chinesa de cinco tons, correspondente a 1 na notação musical numerada}
  \end{Phonetics}
\end{Entry}

\begin{Entry}{宫廷}{9,6}{⼧、⼵}
  \begin{Phonetics}{宫廷}{gong1ting2}[][HSK 7-9]
    \definition{s.}{palácio imperial; residência do imperador; palácio | corte real ou imperial; o monarca e seus funcionários; corte}
  \end{Phonetics}
\end{Entry}

\begin{Entry}{宫殿}{9,13}{⼧、⽎}
  \begin{Phonetics}{宫殿}{gong1dian4}[][HSK 7-9]
    \definition[座]{s.}{palácio; geralmente se refere às casas magníficas onde os imperadores vivem}
  \end{Phonetics}
\end{Entry}

\begin{Entry}{害}{10}{⼧}
  \begin{Phonetics}{害}{hai4}[][HSK 5]
    \definition{adj.}{prejudicial; destrutivo; injurioso; nocivo}
    \definition{s.}{mal; maldade; dano; calamidade}
    \definition{v.}{prejudicar; fazer mal a; causar problemas a | matar; assassinar | sofrer de; contrair (uma doença) | sentir-se (envergonhado, com medo, etc.); despertar (um sentimento ou uma emoção)}
  \end{Phonetics}
\end{Entry}

\begin{Entry}{害虫}{10,6}{⼧、⾍}
  \begin{Phonetics}{害虫}{hai4chong2}[][HSK 7-9]
    \definition[种,只,个]{s.}{verme; bicho; inseto nocivo (ou destrutivo); praga (oposto a 益虫)}
  \seealsoref{益虫}{yi4chong2}
  \end{Phonetics}
\end{Entry}

\begin{Entry}{害怕}{10,8}{⼧、⼼}
  \begin{Phonetics}{害怕}{hai4pa4}[][HSK 3]
    \definition{v.}{estar assustado; ter medo; encontrar dificuldades, perigos, etc., e sentir-se inquieto ou nervoso}
  \end{Phonetics}
\end{Entry}

\begin{Entry}{害羞}{10,10}{⼧、⽺}
  \begin{Phonetics}{害羞}{hai4/xiu1}[][HSK 7-9]
    \definition{v.+compl.}{ser tímido; parecer tímido; tornar-se tímido}
  \end{Phonetics}
\end{Entry}

\begin{Entry}{害臊}{10,17}{⼧、⾁}
  \begin{Phonetics}{害臊}{hai4/sao4}[][HSK 7-9]
    \definition{v.+compl.}{sentir vergonha; ser tímido}
  \end{Phonetics}
\end{Entry}

\begin{Entry}{宴}{10}{⼧}
  \begin{Phonetics}{宴}{yan4}
    \definition{adj.}{tranquilo e confortável}
    \definition[个,场]{s.}{festa; banquete | facilidade e conforto; felicidade; lazer}
    \definition{v.}{entreter em um banquete; festejar}
  \end{Phonetics}
\end{Entry}

\begin{Entry}{宴会}{10,6}{⼧、⼈}
  \begin{Phonetics}{宴会}{yan4hui4}[][HSK 6]
    \definition[个,场,次]{s.}{festa; banquete; jantar; uma reunião onde convidados e anfitriões bebem e comem juntos (referindo-se a uma ocasião mais solene)}
  \end{Phonetics}
\end{Entry}

\begin{Entry}{家}{10}{⼧}
  \begin{Phonetics}{家}{jia1}[][HSK 1,2]
    \definition*{s.}{Sobrenome Jia}
    \definition{adj.}{domado; domesticado; criado; alimentado | interno}
    \definition{clas.}{usado para famílias ou estabelecimentos comerciais; para uso doméstico; lojas; fábricas, etc.}
    \definition{pron.}{Educado: meu (irmã, tio, etc.)}
    \definition[个]{s.}{família; domicílio; clã | lar; casa; residência da família | pessoa ou família envolvida em um determinado comércio; pessoas que trabalham em determinada profissão ou que possuem determinada identidade | especialista em um determinado campo; pessoa que possui conhecimentos especializados ou se dedica a atividades específicas | escola de pensamento; rscola acadêmica | (em cartas de baralho, mah-jong etc.) festa; lado; refere-se a jogar xadrez ou cartas, em que uma das partes joga contra a outra | nacionalidade; referindo-se à etnia | membros da família; parentes; pessoas ou famílias com quem você tem algum tipo de relação | membro do mesmo clã; pessoas com o mesmo sobrenome}
    \definition{suf.}{sufixo substantivo para designar um especialista em alguma atividade, como um músico ou revolucionário, para designar uma profissão como em -eiro, -ista, por exemplo 科学家}
  \seealsoref{科学家}{ke1xue2jia1}
  \end{Phonetics}
\end{Entry}

\begin{Entry}{家人}{10,2}{⼧、⼈}
  \begin{Phonetics}{家人}{jia1 ren2}[][HSK 1]
    \definition{s.}{família (de alguém); membro da família; os membros de uma família}
  \end{Phonetics}
\end{Entry}

\begin{Entry}{家乡}{10,3}{⼧、⼄}
  \begin{Phonetics}{家乡}{jia1xiang1}[][HSK 3]
    \definition[片,座]{s.}{cidade natal; o lugar onde sua família vive há gerações}
  \end{Phonetics}
\end{Entry}

\begin{Entry}{家长}{10,4}{⼧、⾧}
  \begin{Phonetics}{家长}{jia1 zhang3}[][HSK 2]
    \definition[位,名,个]{s.}{pais; patriarca; tutor; guardião; refere-se aos pais ou outros responsáveis legais}
  \end{Phonetics}
\end{Entry}

\begin{Entry}{家务}{10,5}{⼧、⼒}
  \begin{Phonetics}{家务}{jia1wu4}[][HSK 4]
    \definition[堆,次,件]{s.}{trabalho doméstico; tarefas domésticas}
  \end{Phonetics}
\end{Entry}

\begin{Entry}{家用}{10,5}{⼧、⽤}
  \begin{Phonetics}{家用}{jia1yong4}[][HSK 7-9]
    \definition{adj.}{doméstico; para uso doméstico}
    \definition[本]{s.}{despesas familiares; dinheiro para manutenção da casa}
    \definition{v.}{ser usado em casa; para uso doméstico}
  \end{Phonetics}
\end{Entry}

\begin{Entry}{家用电器}{10,5,5,16}{⼧、⽤、⽥、⼝}
  \begin{Phonetics}{家用电器}{jia1yong4 dian4qi4}
    \definition{s.}{eletrodoméstico; refere-se a diversos aparelhos elétricos utilizados na vida doméstica e coletiva}
  \end{Phonetics}
\end{Entry}

\begin{Entry}{家电}{10,5}{⼧、⽥}
  \begin{Phonetics}{家电}{jia1 dian4}[][HSK 6]
    \definition[件,台]{s.}{eletrodomésticos, abreviação de 家用电器}
  \seealsoref{家用电器}{jia1yong4 dian4qi4}
  \end{Phonetics}
\end{Entry}

\begin{Entry}{家伙}{10,6}{⼧、⼈}
  \begin{Phonetics}{家伙}{jia1huo5}[][HSK 7-9]
    \definition[些,个,群,帮]{s.}{ferramenta; utensílio; arma; refere-se a ferramentas ou armas | cara; companheiro; refere-se a pessoas (com desprezo ou humor)  | gado; animal doméstico}
  \end{Phonetics}
\end{Entry}

\begin{Entry}{家园}{10,7}{⼧、⼞}
  \begin{Phonetics}{家园}{jia1 yuan2}[][HSK 6]
    \definition{s.}{casa; terra natal; um jardim em casa, geralmente referindo-se à cidade natal ou à família}
  \end{Phonetics}
\end{Entry}

\begin{Entry}{家里}{10,7}{⼧、⾥}
  \begin{Phonetics}{家里}{jia1 li3}[][HSK 1]
    \definition{s.}{(em) casa; (em sua) família | esposa}
  \end{Phonetics}
\end{Entry}

\begin{Entry}{家具}{10,8}{⼧、⼋}
  \begin{Phonetics}{家具}{jia1ju4}[][HSK 3]
    \definition[件,套,些,个]{s.}{móveis; mobiliário de casa; utensílios domésticos, incluem principalmente camas, mesas, cadeiras, armários, etc.}
  \end{Phonetics}
\end{Entry}

\begin{Entry}{家庭}{10,9}{⼧、⼴}
  \begin{Phonetics}{家庭}{jia1ting2}[][HSK 2]
    \definition[个,户]{s.}{família}
  \end{Phonetics}
\end{Entry}

\begin{Entry}{家政}{10,9}{⼧、⽁}
  \begin{Phonetics}{家政}{jia1zheng4}[][HSK 7-9]
    \definition{s.}{tarefas domésticas; trabalho de gestão doméstica}
  \end{Phonetics}
\end{Entry}

\begin{Entry}{家俱}{10,10}{⼧、⼈}
  \begin{Phonetics}{家俱}{jia1ju4}
    \definition{s.}{mobília}
  \end{Phonetics}
\end{Entry}

\begin{Entry}{家家户户}{10,10,4,4}{⼧、⼧、⼾、⼾}
  \begin{Phonetics}{家家户户}{jia1jia1hu4hu4}[][HSK 7-9]
    \definition{expr.}{cada família; cada lar}
  \end{Phonetics}
\end{Entry}

\begin{Entry}{家教}{10,11}{⼧、⽁}
  \begin{Phonetics}{家教}{jia1jiao4}[][HSK 7-9]
    \definition[个,名,位]{s.}{educação; educação familiar; disciplina doméstica; a educação de moral e etiqueta que os pais dão aos seus filhos |  tutor}
  \end{Phonetics}
\end{Entry}

\begin{Entry}{家族}{10,11}{⼧、⽅}
  \begin{Phonetics}{家族}{jia1zu2}[][HSK 7-9]
    \definition[个]{s.}{clã; família; uma organização social baseada em relações de sangue, incluindo várias gerações do mesmo sangue}
  \end{Phonetics}
\end{Entry}

\begin{Entry}{家喻户晓}{10,12,4,10}{⼧、⼝、⼾、⽇}
  \begin{Phonetics}{家喻户晓}{jia1yu4-hu4xiao3}[][HSK 7-9]
    \definition{expr.}{nome familiar | conhecido por todas as famílias; amplamente conhecido; conhecido por todos}[他是家喻户晓的演员。===Ele é um ator famoso.]
  \end{Phonetics}
\end{Entry}

\begin{Entry}{家属}{10,12}{⼧、⼫}
  \begin{Phonetics}{家属}{jia1shu3}[][HSK 3]
    \definition{s.}{membros da família; dependentes (familiares); os membros da família que não sejam o próprio chefe da família, ou seja, os membros da família que não sejam o próprio trabalhador}
  \end{Phonetics}
\end{Entry}

\begin{Entry}{家禽}{10,12}{⼧、⽱}
  \begin{Phonetics}{家禽}{jia1qin2}[][HSK 7-9]
    \definition{s.}{aves domésticas | ave; pássaro doméstico}
  \end{Phonetics}
\end{Entry}

\begin{Entry}{家境}{10,14}{⼧、⼟}
  \begin{Phonetics}{家境}{jia1jing4}[][HSK 7-9]
    \definition{s.}{situação financeira familiar; circunstâncias familiares}
  \end{Phonetics}
\end{Entry}

\begin{Entry}{容}{10}{⼧}
  \begin{Phonetics}{容}{rong2}
    \definition*{s.}{Sobrenome Rong}
    \definition{adv.}{talvez; provavelmente; possivelmente}
    \definition{s.}{expressão facial e tez | aparência; o estado ou condição das coisas}
    \definition{v.}{permitir; quando os outros querem fazer algo, deixe-os fazer | tolerar; ser capaz de aceitar pessoas ou coisas com as quais você não está satisfeito | conter (número de pessoas ou coisas que podem ser colocadas em um determinado espaço)}
  \end{Phonetics}
\end{Entry}

\begin{Entry}{容易}{10,8}{⼧、⽇}
  \begin{Phonetics}{容易}{rong2yi4}[][HSK 3]
    \definition{adj.}{fácil; simples; sem complicações | provável; passível; inclinado; indica uma alta probabilidade de algo acontecer}
  \end{Phonetics}
\end{Entry}

\begin{Entry}{容貌}{10,14}{⼧、⾘}
  \begin{Phonetics}{容貌}{rong2mao4}
    \definition{s.}{aparência | aspecto | características}
  \end{Phonetics}
\end{Entry}

\begin{Entry}{宽}{10}{⼧}
  \begin{Phonetics}{宽}{kuan1}[][HSK 4]
    \definition*{s.}{Sobrenome Kuan}
    \definition{adj.}{largo; amplo; espaçoso; grandes distâncias horizontais (em oposição a 窄) | leniente; generoso; indulgente | bem de vida; rico; confortável}
    \definition[米]{s.}{largura; amplitude}[桌子有一米宽。===A mesa tem um metro de largura.]
    \definition{v.}{relaxar; aliviar}
  \seealsoref{窄}{zhai3}
  \end{Phonetics}
\end{Entry}

\begin{Entry}{宽广}{10,3}{⼧、⼴}
  \begin{Phonetics}{宽广}{kuan1 guang3}[][HSK 4]
    \definition{adj.}{vasto; amplo; espaçoso; extenso}
  \end{Phonetics}
\end{Entry}

\begin{Entry}{宽度}{10,9}{⼧、⼴}
  \begin{Phonetics}{宽度}{kuan1 du4}[][HSK 5]
    \definition{s.}{largura; amplitude; duração; o grau de largura e estreiteza; a distância horizontal (no caso de um retângulo, a distância entre os dois lados mais longos)}
  \end{Phonetics}
\end{Entry}

\begin{Entry}{宽阔}{10,12}{⼧、⾨}
  \begin{Phonetics}{宽阔}{kuan1 kuo4}[][HSK 6]
    \definition{adj.}{amplo; largo; espaçoso | tolerante; mente aberta; descreve uma mente alegre e ampla}
  \end{Phonetics}
\end{Entry}

\begin{Entry}{宽影片}{10,15,4}{⼧、⼺、⽚}
  \begin{Phonetics}{宽影片}{kuan1ying3pian4}
    \definition{s.}{filme \emph{widescreen}}
  \end{Phonetics}
\end{Entry}

\begin{Entry}{宾}{10}{⼧}
  \begin{Phonetics}{宾}{bin1}
    \definition*{s.}{Sobrenome Bin}
    \definition[个,位,名,些]{s.}{convidado}
  \end{Phonetics}
\end{Entry}

\begin{Entry}{宾馆}{10,11}{⼧、⾷}
  \begin{Phonetics}{宾馆}{bin1guan3}[][HSK 5]
    \definition[家,个,座]{s.}{hotel; acomodações públicas para hóspedes}
  \end{Phonetics}
\end{Entry}

\begin{Entry}{宿}{11}{⼧}
  \begin{Phonetics}{宿}{su4}
    \definition*{s.}{Sobrenome Su}
    \definition{adj.}{de longa data; antigo; velho | veterano; velho; experiente}
    \definition{v.}{hospedar-se para passar a noite; passar a noite}
  \end{Phonetics}
  \begin{Phonetics}{宿}{xiu3}
    \definition{s.}{usado para calcular a noite}[谈了半宿。===Conversamos por metade da noite.]
  \end{Phonetics}
  \begin{Phonetics}{宿}{xiu4}
    \definition{s.}{(astronomia) um termo antigo para constelação}
  \end{Phonetics}
\end{Entry}

\begin{Entry}{宿舍}{11,8}{⼧、⾆}
  \begin{Phonetics}{宿舍}{su4she4}[][HSK 5]
    \definition[间,幢]{s.}{alojamento; dormitório; república; albergue; casas onde escolas, empresas, etc. acomodam seus alunos ou funcionários}
  \end{Phonetics}
\end{Entry}

\begin{Entry}{寂}{11}{⼧}
  \begin{Phonetics}{寂}{ji4}
    \definition{adj.}{quieto; parado; silencioso | solitário}
  \end{Phonetics}
\end{Entry}

\begin{Entry}{寂寞}{11,13}{⼧、⼧}
  \begin{Phonetics}{寂寞}{ji4mo4}[][HSK 7-9]
    \definition{adj.}{solitário; só; isolado; deserto | quieto; parado; silencioso}
  \end{Phonetics}
\end{Entry}

\begin{Entry}{寂寥}{11,14}{⼧、⼧}
  \begin{Phonetics}{寂寥}{ji4liao2}
    \definition{s.}{solidão | vasto e vazio | quieto e desolado (literário)}
  \end{Phonetics}
\end{Entry}

\begin{Entry}{寂静}{11,14}{⼧、⾭}
  \begin{Phonetics}{寂静}{ji4jing4}[][HSK 7-9]
    \definition{adj.}{quieto; parado; silencioso; sem som; muito silencioso}
  \end{Phonetics}
\end{Entry}

\begin{Entry}{寄}{11}{⼧}
  \begin{Phonetics}{寄}{ji4}[][HSK 4]
    \definition*{s.}{Sobrenome Ji}
    \definition{adj.}{adotado; fomentado; promovido}
    \definition{v.}{enviar; postar; remeter | confiar; depositar; colocar | depender de; apegar-se a}
  \end{Phonetics}
\end{Entry}

\begin{Entry}{寄予}{11,4}{⼧、⼅}
  \begin{Phonetics}{寄予}{ji4yu3}
    \definition{v.}{expressar | colocar (esperança, importância, etc.) em | mostrar}
  \end{Phonetics}
\end{Entry}

\begin{Entry}{寄生}{11,5}{⼧、⽣}
  \begin{Phonetics}{寄生}{ji4sheng1}
    \definition{s.}{parasita | parasitismo}
    \definition{v.}{viver tirando vantagem dos outros | viver dentro ou sobre outro organismo como um parasita}
  \end{Phonetics}
\end{Entry}

\begin{Entry}{寄生生活}{11,5,5,9}{⼧、⽣、⽣、⽔}
  \begin{Phonetics}{寄生生活}{ji4sheng1sheng1huo2}
    \definition{s.}{parasitismo | vida parasitária}
  \end{Phonetics}
\end{Entry}

\begin{Entry}{寄存}{11,6}{⼧、⼦}
  \begin{Phonetics}{寄存}{ji4cun2}
    \definition{v.}{depositar | deixar algo com alguém | armazenar}
  \end{Phonetics}
\end{Entry}

\begin{Entry}{寄托}{11,6}{⼧、⼿}
  \begin{Phonetics}{寄托}{ji4tuo1}[][HSK 7-9]
    \definition{v.}{deixar com alguém; confiar aos cuidados de alguém; confiar | repousar; colocar (esperança, etc.) em; encontrar sustento em; depositar ideais, esperanças, sentimentos, etc. em (alguém ou alguma coisa)}
  \end{Phonetics}
\end{Entry}

\begin{Entry}{寄卖}{11,8}{⼧、⼗}
  \begin{Phonetics}{寄卖}{ji4mai4}
    \definition{v.}{consignar para venda}
  \end{Phonetics}
\end{Entry}

\begin{Entry}{寄居}{11,8}{⼧、⼫}
  \begin{Phonetics}{寄居}{ji4ju1}
    \definition{s.}{morar longe de casa}
  \end{Phonetics}
\end{Entry}

\begin{Entry}{寄放}{11,8}{⼧、⽅}
  \begin{Phonetics}{寄放}{ji4fang4}
    \definition{v.}{deixar algo com alguém}
  \end{Phonetics}
\end{Entry}

\begin{Entry}{寄养}{11,9}{⼧、⼋}
  \begin{Phonetics}{寄养}{ji4yang3}
    \definition{v.}{embarcar | promover | colocar sob os cuidados de alguém (uma criança, animal de estimação, etc.)}
  \end{Phonetics}
\end{Entry}

\begin{Entry}{寄送}{11,9}{⼧、⾡}
  \begin{Phonetics}{寄送}{ji4song4}
    \definition{v.}{enviar | transmitir}
  \end{Phonetics}
\end{Entry}

\begin{Entry}{寄递}{11,10}{⼧、⾡}
  \begin{Phonetics}{寄递}{ji4di4}
    \definition{s.}{entrega de correspondência}
  \end{Phonetics}
\end{Entry}

\begin{Entry}{寄售}{11,11}{⼧、⼝}
  \begin{Phonetics}{寄售}{ji4shou4}
    \definition{v.}{venda em consignação}
  \end{Phonetics}
\end{Entry}

\begin{Entry}{寄宿}{11,11}{⼧、⼧}
  \begin{Phonetics}{寄宿}{ji4su4}
    \definition{s.}{embarque}
    \definition{v.}{embarcar}
  \end{Phonetics}
\end{Entry}

\begin{Entry}{寄望}{11,11}{⼧、⽉}
  \begin{Phonetics}{寄望}{ji4wang4}
    \definition{v.}{depositar esperanças em}
  \end{Phonetics}
\end{Entry}

\begin{Entry}{密}{11}{⼧}
  \begin{Phonetics}{密}{mi4}[][HSK 4]
    \definition*{s.}{Sobrenome Mi}
    \definition{adj.}{fechado; denso; espesso | íntimo; próximo; afetuoso | delicado; fino; cuidadoso; meticuloso}
    \definition{adv.}{secretamente}
    \definition{s.}{segredo | densidade | senha; \emph{password}}
  \end{Phonetics}
\end{Entry}

\begin{Entry}{密切}{11,4}{⼧、⼑}
  \begin{Phonetics}{密切}{mi4qie4}[][HSK 4]
    \definition{adj.}{próximo; íntimo; relacionamento próximo}
    \definition{adv.}{cuidadosamente; atentamente; intimamente}
    \definition{v.}{tornar-se próximo; tornar-se íntimo; conectar-se}
  \end{Phonetics}
\end{Entry}

\begin{Entry}{密码}{11,8}{⼧、⽯}
  \begin{Phonetics}{密码}{mi4ma3}[][HSK 4]
    \definition[个,种]{s.}{código; senha; um código secreto especialmente formulado usado entre as partes acordadas (diferente do 明码)}
  \seealsoref{明码}{ming2ma3}
  \end{Phonetics}
\end{Entry}

\begin{Entry}{富}{12}{⼧}
  \begin{Phonetics}{富}{fu4}[][HSK 3]
    \definition*{s.}{Sobrenome Fu}
    \definition{adj.}{rico; abastado; abundante; refere-se a ter muito dinheiro (oposto de 贫) | rico; abundante}
    \definition{v.}{tornar-se rico; enriquecer}
  \seealsoref{贫}{pin2}
  \end{Phonetics}
\end{Entry}

\begin{Entry}{富人}{12,2}{⼧、⼈}
  \begin{Phonetics}{富人}{fu4 ren2}[][HSK 6]
    \definition{s.}{os ricos; os abastados}
  \end{Phonetics}
\end{Entry}

\begin{Entry}{富有}{12,6}{⼧、⽉}
  \begin{Phonetics}{富有}{fu4 you3}[][HSK 6]
    \definition{adj.}{rico; abastado; possuir uma grande quantidade de propriedades | rico em espírito; metáfora para uma vida espiritual rica}
    \definition{v.}{ser rico ou abundante em; principalmente referindo-se a coisas abstratas com significados positivos que são suficientes}
  \end{Phonetics}
\end{Entry}

\begin{Entry}{富含}{12,7}{⼧、⼝}
  \begin{Phonetics}{富含}{fu4han2}[][HSK 7-9]
    \definition{v.}{ser rico em}[橘子富含维生素。===As laranjas são ricas em vitaminas.]
  \end{Phonetics}
\end{Entry}

\begin{Entry}{富足}{12,7}{⼧、⾜}
  \begin{Phonetics}{富足}{fu4zu2}[][HSK 7-9]
    \definition{adj.}{rico; pleno; abundante}
  \end{Phonetics}
\end{Entry}

\begin{Entry}{富翁}{12,10}{⼧、⽺}
  \begin{Phonetics}{富翁}{fu4weng1}[][HSK 7-9]
    \definition[个,名,位]{s.}{homem rico; homem de riqueza; pessoas que possuem muitas propriedades}
  \end{Phonetics}
\end{Entry}

\begin{Entry}{富强}{12,12}{⼧、⼸}
  \begin{Phonetics}{富强}{fu4qiang2}[][HSK 7-9]
    \definition{adj.}{próspero e forte; próspero e poderoso; rico e poderoso; (país) rico e poderoso}
  \end{Phonetics}
\end{Entry}

\begin{Entry}{富裕}{12,12}{⼧、⾐}
  \begin{Phonetics}{富裕}{fu4yu4}[][HSK 7-9]
    \definition{adj.}{rico; próspero; abastado; em boas condições econômicas e com dinheiro suficiente}
  \end{Phonetics}
\end{Entry}

\begin{Entry}{富豪}{12,14}{⼧、⾗}
  \begin{Phonetics}{富豪}{fu4hao2}[][HSK 7-9]
    \definition{s.}{rico; pessoa com muito dinheiro e grande poder}
  \end{Phonetics}
\end{Entry}

\begin{Entry}{寒}{12}{⼧}
  \begin{Phonetics}{寒}{han2}
    \definition*{s.}{Sobrenome Han}
    \definition{adj.}{frio | pobre; necessitado | (autodepreciativo) meu/minha humilde\dots | assustado; medroso | com medo; tremendo (de medo) | humilde}
    \definition{s.}{estação fria; inverno (oposto a 暑) | (medicina chinesa) sintomas causados por fatores frios}
  \seealsoref{暑}{shu3}
  \end{Phonetics}
\end{Entry}

\begin{Entry}{寒冷}{12,7}{⼧、⼎}
  \begin{Phonetics}{寒冷}{han2 leng3}[][HSK 4]
    \definition[度,阵,股]{adj.}{frio; frígido; gélido; gelado}
  \end{Phonetics}
\end{Entry}

\begin{Entry}{寒假}{12,11}{⼧、⼈}
  \begin{Phonetics}{寒假}{han2jia4}[][HSK 4]
    \definition[个,段]{s.}{férias de inverno (feriados); férias escolares no meio do inverno, em janeiro e fevereiro (na China)}
  \end{Phonetics}
\end{Entry}

\begin{Entry}{寓}{12}{⼧}
  \begin{Phonetics}{寓}{yu4}
    \definition[座,间,栋]{s.}{residência; morada}
    \definition{v.}{(literário) residir; viver | implicar; conter}
  \end{Phonetics}
\end{Entry}

\begin{Entry}{寓意}{12,13}{⼧、⼼}
  \begin{Phonetics}{寓意}{yu4yi4}
    \definition{s.}{moral (de uma história),  lição a ser aprendida, implicação, mensagem, significado metafórico}
  \end{Phonetics}
\end{Entry}

\begin{Entry}{察}{14}{⼧}
  \begin{Phonetics}{察}{cha2}
    \definition*{s.}{Sobrenome Cha}
    \definition{v.}{examinar; investigar; escrutinar | observar; olhar atentamente; investigar}
  \end{Phonetics}
\end{Entry}

\begin{Entry}{察看}{14,9}{⼧、⽬}
  \begin{Phonetics}{察看}{cha2kan4}[][HSK 7-9]
    \definition{v.}{observar; olhar atentamente; inspecionar}
  \end{Phonetics}
\end{Entry}

\begin{Entry}{察觉}{14,9}{⼧、⾒}
  \begin{Phonetics}{察觉}{cha2jue2}[][HSK 7-9]
    \definition{v.}{detectar; perceber; estar ciente de; estar consciente de; descobrir; ver}
  \end{Phonetics}
\end{Entry}

\begin{Entry}{寡}{14}{⼧}
  \begin{Phonetics}{寡}{gua3}
    \definition{adj.}{poucos; escassos (oposto a 众, 多)  | insípido; sem sabor | pouco; escasso | insípido; sem graça}
    \definition{pron.}{eu; título autoproclamado de um antigo monarca}
    \definition{s.}{viúva | viuvez; a natureza ou estado de uma mulher viúva que vive sozinha}
  \seealsoref{多}{duo1}
  \seealsoref{众}{zhong4}
  \end{Phonetics}
\end{Entry}

\begin{Entry}{寡妇}{14,6}{⼧、⼥}
  \begin{Phonetics}{寡妇}{gua3fu5}[][HSK 7-9]
    \definition[个]{s.}{viúva; uma mulher cujo marido morreu}
  \end{Phonetics}
\end{Entry}

\begin{Entry}{寨}{14}{⼧}
  \begin{Phonetics}{寨}{zhai4}
    \definition{s.}{fortaleza | paliçada | acampamento | vila (paliçada)}
  \end{Phonetics}
\end{Entry}

%%%%% EOF %%%%%

