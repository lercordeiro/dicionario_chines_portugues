%%%
%%% Radical "⼧"
%%%

\section*{Radical 40: ``⼧''}\addcontentsline{toc}{section}{Radical 40: ⼧}

\begin{entry}{宁}{5}{⼧}
  \begin{phonetics}{宁}{ning2}
    \definition*{s.}{sobrenome Ning}
    \definition{adj.}{calmo, pacífico, sereno | saudável}
  \end{phonetics}
  \begin{phonetics}{宁}{ning4}
    \definition{conj.}{mais\dots do que\dots, melhor\dots do que\dots}
  \end{phonetics}
\end{entry}

\begin{entry}{宁可}{5,5}{⼧、⼝}
  \begin{phonetics}{宁可}{ning4ke3}
    \definition{conj.}{mais\dots do que\dots | melhor\dots do que\dots}
  \end{phonetics}
\end{entry}

\begin{entry}{宁可……也不……}{5,5,3,4}{⼧、⼝、⼄、⼀}
  \begin{phonetics}{宁可……也不……}{ning4ke3 ye3bu4}
    \definition{conj.}{em vez de\dots}
  \end{phonetics}
\end{entry}

\begin{entry}{宁可……也要……}{5,5,3,9}{⼧、⼝、⼄、⾑}
  \begin{phonetics}{宁可……也要……}{ning4ke3 ye3yao4}
    \definition{conj.}{mesmo que tenhamos que\dots nós iremos\dots}
  \end{phonetics}
\end{entry}

\begin{entry}{宁肯}{5,8}{⼧、⾁}
  \begin{phonetics}{宁肯}{ning4ken3}
    \definition{conj.}{mais\dots do que\dots, melhor\dots do que\dots}
  \end{phonetics}
\end{entry}

\begin{entry}{宁愿}{5,14}{⼧、⽕}
  \begin{phonetics}{宁愿}{ning4yuan4}
    \definition{conj.}{mais\dots do que\dots, melhor\dots do que\dots}
  \end{phonetics}
\end{entry}

\begin{entry}{宁静}{5,14}{⼧、⾭}
  \begin{phonetics}{宁静}{ning2 jing4}[][HSK 4]
    \definition{adj.}{calmo; tranquilo; pacífico}
  \end{phonetics}
\end{entry}

\begin{entry}{它}{5}{⼧}
  \begin{phonetics}{它}{ta1}[][HSK 2]
    \definition*{s.}{sobrenome Ta}
    \definition{pron.}{ele; referência a algo além da pessoa (para objetos inanimados) | ele; usado após o verbo, indica referência vaga}
  \end{phonetics}
\end{entry}

\begin{entry}{它们}{5,5}{⼧、⼈}
  \begin{phonetics}{它们}{ta1 men5}[][HSK 2]
    \definition{pron.}{eles; usado para se referir a mais de uma coisa não humana; geralmente se refere a animais, objetos ou conceitos abstratos}
  \end{phonetics}
\end{entry}

\begin{entry}{宇宙}{6,8}{⼧、⼧}
  \begin{phonetics}{宇宙}{yu3zhou4}
    \definition{s.}{universo | cosmos}
  \end{phonetics}
\end{entry}

\begin{entry}{宇航员}{6,10,7}{⼧、⾈、⼝}
  \begin{phonetics}{宇航员}{yu3hang2yuan2}
    \definition{s.}{astronauta}
  \end{phonetics}
\end{entry}

\begin{entry}{守}{6}{⼧}
  \begin{phonetics}{守}{shou3}[][HSK 4]
    \definition*{s.}{sobrenome Shou}
    \definition{adv.}{próximo; perto de; perto de algum lugar em posição, perto de algum lugar}
    \definition{v.}{guardar; defender; estar presente para cuidar; não ir embora | manter vigilância; defender do ataque do oponente em uma luta ou confronto | observar; cumprir; respeitar; fazer as coisas como elas devem ser feitas | manter, observar a integridade; honrar a palavra de alguém; manter a palavra de alguém}
  \end{phonetics}
\end{entry}

\begin{entry}{守门员}{6,3,7}{⼧、⾨、⼝}
  \begin{phonetics}{守门员}{shou3men2yuan2}
    \definition{s.}{goleiro}
  \end{phonetics}
\end{entry}

\begin{entry}{安}{6}{⼧}
  \begin{phonetics}{安}{an1}[][HSK 4]
    \definition{adj.}{pacífico; quieto; tranquilo; calmo; estáve; sem perturbação | seguro; protegido; com boa saúde; em paz; bem}
    \definition{pron.}{onde; como}
    \definition{s.}{segurança; proteção; paz; conforto | ampère; (eletricidade) abreviação de ampère}
    \definition{v.}{deixar (a mente de alguém) à vontade; acalmar; estabilizar | satisfazer; estar satisfeito; sentir-se satisfeito e à vontade | colocar em uma posição adequada; encontrar um lugar para | instalar; consertar; encaixar; configurar | trazer (uma acusação contra alguém); dar (a alguém um apelido) | abrigar (uma intenção); manter; segurar}
  \end{phonetics}
\end{entry}

\begin{entry}{安全}{6,6}{⼧、⼊}
  \begin{phonetics}{安全}{an1quan2}[][HSK 2]
    \definition{adj.}{seguro; protegido; sem perigo; sem ameaças; sem acidentes}
    \definition{s.}{segurança; proteção; refere-se a um estado ou conceito, geralmente indicando ausência de ameaças ou perigo}
  \end{phonetics}
\end{entry}

\begin{entry}{安神}{6,9}{⼧、⽰}
  \begin{phonetics}{安神}{an1shen2}
    \definition{v.+compl.}{acalmar os nervos | aliviar a inquietação pela tranquilização da mente e do corpo}
  \end{phonetics}
\end{entry}

\begin{entry}{安家}{6,10}{⼧、⼧}
  \begin{phonetics}{安家}{an1jia1}
    \definition{v.+compl.}{montar uma casa | estabelecer-se}
  \end{phonetics}
\end{entry}

\begin{entry}{安排}{6,11}{⼧、⼿}
  \begin{phonetics}{安排}{an1pai2}[][HSK 3]
    \definition{s.}{arranjos | planos}
    \definition{v.}{organizar | programar | fazer planos}
  \end{phonetics}
\end{entry}

\begin{entry}{安装}{6,12}{⼧、⾐}
  \begin{phonetics}{安装}{an1zhuang1}[][HSK 3]
    \definition{v.}{instalar | consertar | configurar}
  \end{phonetics}
\end{entry}

\begin{entry}{安置}{6,13}{⼧、⽹}
  \begin{phonetics}{安置}{an1zhi4}[][HSK 4]
    \definition{v.}{providenciar; encontrar um lugar para; ajudar a estabelecer-se; colocar pessoas ou coisas em uma determinada posição ou organizá-las adequadamente}
  \end{phonetics}
\end{entry}

\begin{entry}{安静}{6,14}{⼧、⾭}
  \begin{phonetics}{安静}{an1jing4}[][HSK 2]
    \definition{adj.}{silencioso; tranquilo; sem som; sem barulho e sem algazarra}
  \end{phonetics}
\end{entry}

\begin{entry}{安慰}{6,15}{⼧、⼼}
  \begin{phonetics}{安慰}{an1wei4}[][HSK 5]
    \definition{adj.}{confortar; tranquilizar; consolar; apaziguar;}
    \definition[个]{s.}{conforto; consolo; comportamento que alivia a dor de alguém e o acalma com palavras ou gestos}
    \definition{v.}{confortar; consolar; acalmar e confortar; deixar a mente tranquila}
  \end{phonetics}
\end{entry}

\begin{entry}{宋}{7}{⼧}
  \begin{phonetics}{宋}{song4}
    \definition*{s.}{sobrenome Song}
    \definition{s.}{Dinastia Song (960-1279) | Song das dinastias do sul (420-479)}
  \end{phonetics}
\end{entry}

\begin{entry}{完}{7}{⼧}
  \begin{phonetics}{完}{wan2}[][HSK 2]
    \definition*{s.}{sobrenome Wan}
    \definition{adj.}{inteiro; intacto; completo}
    \definition{v.}{acabar; terminar; completar | pagar | estar terminado; estar pronto para | esgotar; ser usado}
  \end{phonetics}
\end{entry}

\begin{entry}{完了}{7,2}{⼧、⼅}
  \begin{phonetics}{完了}{wan2 le5}[][HSK 5]
    \definition{v.}{acabar; terminar; concluir; chegar ao fim}
  \end{phonetics}
\end{entry}

\begin{entry}{完人}{7,2}{⼧、⼈}
  \begin{phonetics}{完人}{wan2ren2}
    \definition{s.}{pessoa perfeita}
  \end{phonetics}
\end{entry}

\begin{entry}{完全}{7,6}{⼧、⼊}
  \begin{phonetics}{完全}{wan2quan2}[][HSK 2]
    \definition{adj.}{inteiro; completo; não falta nada, está tudo completo}
    \definition{adv.}{completamente; representa tudo}
  \end{phonetics}
\end{entry}

\begin{entry}{完成}{7,6}{⼧、⼽}
  \begin{phonetics}{完成}{wan2cheng2}[][HSK 2]
    \definition{v.}{realizar; completar; terminar; cumprir; levar ao sucesso}
  \end{phonetics}
\end{entry}

\begin{entry}{完毕}{7,6}{⼧、⽐}
  \begin{phonetics}{完毕}{wan2bi4}
    \definition{v.}{completar | terminar | acabar}
  \end{phonetics}
\end{entry}

\begin{entry}{完完全全}{7,7,6,6}{⼧、⼧、⼊、⼊}
  \begin{phonetics}{完完全全}{wan2wan2quan2quan2}
    \definition{adv.}{completamente}
  \end{phonetics}
\end{entry}

\begin{entry}{完备}{7,8}{⼧、⼡}
  \begin{phonetics}{完备}{wan2bei4}
    \definition{adj.}{completo | impecável | perfeito}
    \definition{v.}{não deixar nada a desejar}
  \end{phonetics}
\end{entry}

\begin{entry}{完美}{7,9}{⼧、⽺}
  \begin{phonetics}{完美}{wan2mei3}[][HSK 3]
    \definition{adj.}{perfeito; impecável; consumado}
    \definition{adv.}{perfeitamente}
    \definition{s.}{perfeição}
  \end{phonetics}
\end{entry}

\begin{entry}{完善}{7,12}{⼧、⼝}
  \begin{phonetics}{完善}{wan2shan4}[][HSK 3]
    \definition{adj.}{perfeito; consumado}
    \definition{v.}{refinar; melhorar; tornar perfeito}
  \end{phonetics}
\end{entry}

\begin{entry}{完税}{7,12}{⼧、⽲}
  \begin{phonetics}{完税}{wan2shui4}
    \definition{v.}{pagar imposto}
  \end{phonetics}
\end{entry}

\begin{entry}{完满}{7,13}{⼧、⽔}
  \begin{phonetics}{完满}{wan2man3}
    \definition{adj.}{satisfatório | bem-sucedido}
  \end{phonetics}
\end{entry}

\begin{entry}{完整}{7,16}{⼧、⽁}
  \begin{phonetics}{完整}{wan2zheng3}[][HSK 3]
    \definition{adj.}{intacto; inteiro; completo; integrado}
  \end{phonetics}
\end{entry}

\begin{entry}{官}{8}{⼧}
  \begin{phonetics}{官}{guan1}[][HSK 4]
    \definition*{s.}{sobrenome Guan}
    \definition{adj.}{propriedade do governo; pertencente ao governo ou ao público | público}
    \definition[个,位]{s.}{funcionário do governo; oficial; servidor público; titular de cargo; funcionário público nomeado acima de um determinado nível | órgão (parte do tecido do corpo)}
  \end{phonetics}
\end{entry}

\begin{entry}{官方}{8,4}{⼧、⽅}
  \begin{phonetics}{官方}{guan1fang1}[][HSK 4]
    \definition{s.}{autoridade; (do ou pelo) governo | oficial (de uma organização ou instituição)}
  \end{phonetics}
\end{entry}

\begin{entry}{官桂}{8,10}{⼧、⽊}
  \begin{phonetics}{官桂}{guan1gui4}
    \definition{s.}{canela}
  \seealsoref{肉桂}{rou4gui4}
  \end{phonetics}
\end{entry}

\begin{entry}{定}{8}{⼧}
  \begin{phonetics}{定}{ding4}[][HSK 4]
    \definition*{s.}{sobrenome Ding}
    \definition{adj.}{calmo; estável | fixo; estabelecido; fixado; inalterado}
    \definition{adv.}{certamente; com certeza; definitivamente}
    \definition{v.}{decidir; fixar; definir; determinar; ter certeza | consertar; fazer com que seja consertado | acalmar; estabilizar; tornar estável | assinar (um jornal, etc.); reservar (assentos, ingressos, etc.); encomendar (mercadorias, etc.)}
  \end{phonetics}
\end{entry}

\begin{entry}{定期}{8,12}{⼧、⽉}
  \begin{phonetics}{定期}{ding4qi1}[][HSK 3]
    \definition{adj.}{regular; periódico; em intervalos regulares}
    \definition{v.}{fixar (definir) uma data}
  \end{phonetics}
\end{entry}

\begin{entry}{宝}{8}{⼧}
  \begin{phonetics}{宝}{bao3}[][HSK 4]
    \definition*{s.}{sobrenome Bao}
    \definition{adj.}{antigo; precioso; estimado}
    \definition[个,件]{s.}{tesouro; objeto estimado; coisa preciosa | dispositivo de jogo; ferramenta de jogo | dinheiro; moeda; moeda antiga com furo quadrado no centro; moeda de prata}
  \end{phonetics}
\end{entry}

\begin{entry}{宝贝}{8,4}{⼧、⾙}
  \begin{phonetics}{宝贝}{bao3bei4}[][HSK 4]
    \definition{adj.}{excêntrico; estranho; imprestável; um termo depreciativo para uma pessoa incompetente ou ridícula}
    \definition[个,件]{s.}{tesouro; objeto estimado; coisa preciosa | querida; \emph{darling}; \emph{baby}; apelido para crianças}
  \end{phonetics}
\end{entry}

\begin{entry}{宝石}{8,5}{⼧、⽯}
  \begin{phonetics}{宝石}{bao3 shi2}[][HSK 4]
    \definition[颗,枚,块]{s.}{gema; jóia; pedra preciosa; mineral precioso que tem um brilho lindo e uma dureza de mais de sete graus, não é afetado pela atmosfera ou por produtos químicos e pode ser usado como decoração, suporte de instrumentos ou abrasivos}
  \end{phonetics}
\end{entry}

\begin{entry}{宝宝}{8,8}{⼧、⼧}
  \begin{phonetics}{宝宝}{bao3 bao5}[][HSK 4]
    \definition[个]{s.}{querida; \emph{darling}; \emph{baby}; apelido para crianças}
  \end{phonetics}
\end{entry}

\begin{entry}{宝贵}{8,9}{⼧、⾙}
  \begin{phonetics}{宝贵}{bao3gui4}[][HSK 4]
    \definition{adj.}{precioso; extremamente valioso, muito raro, pode ser usado para descrever coisas específicas, também pode ser usado para descrever coisas abstratas | valioso; como um tesouro}
  \end{phonetics}
\end{entry}

\begin{entry}{实力}{8,2}{⼧、⼒}
  \begin{phonetics}{实力}{shi2li4}[][HSK 3]
    \definition{s.}{força | geralmente se refere à força militar e econômica de um país, grupo ou indivíduo, e também se refere à habilidade de um indivíduo ou grupo em um jogo.}
  \end{phonetics}
\end{entry}

\begin{entry}{实习}{8,3}{⼧、⼄}
  \begin{phonetics}{实习}{shi2xi2}[][HSK 2]
    \definition{s.}{estagiário; prática; estágio}
    \definition{v.}{aplicar e testar os conhecimentos teóricos aprendidos no trabalho prático, a fim de exercitar a capacidade profissional}
  \end{phonetics}
\end{entry}

\begin{entry}{实用}{8,5}{⼧、⽤}
  \begin{phonetics}{实用}{shi2yong4}[][HSK 4]
    \definition{adj.}{prático; pragmático; funcional; atende aos requisitos reais da aplicação}
    \definition{v.}{colocar em uso prático}
  \end{phonetics}
\end{entry}

\begin{entry}{实在}{8,6}{⼧、⼟}
  \begin{phonetics}{实在}{shi2zai4}[][HSK 2]
    \definition{adj.}{honesto; sincero | verdadeiro; honesto; realista; não é falso, não é enganador}
    \definition{adv.}{verdadeiramente; de fato; na verdade; usado para reforçar o tom afirmativo, enfatizando que a situação é realmente assim}
  \end{phonetics}
\end{entry}

\begin{entry}{实行}{8,6}{⼧、⾏}
  \begin{phonetics}{实行}{shi2xing2}[][HSK 3]
    \definition{v.}{praticar; implementar; executar; pôr em prática}
  \end{phonetics}
\end{entry}

\begin{entry}{实际}{8,7}{⼧、⾩}
  \begin{phonetics}{实际}{shi2ji4}[][HSK 2]
    \definition{adj.}{real; efetivo; concreto; prático | factual; prático; realista; de acordo com os fatos}
    \definition{s.}{realidade; prática; coisas e situações que existem objetivamente}
  \end{phonetics}
\end{entry}

\begin{entry}{实际上}{8,7,3}{⼧、⾩、⼀}
  \begin{phonetics}{实际上}{shi2 ji4 shang4}[][HSK 3]
    \definition{adv.}{de fato; na verdade; como aliás}
  \end{phonetics}
\end{entry}

\begin{entry}{实现}{8,8}{⼧、⾒}
  \begin{phonetics}{实现}{shi2xian4}[][HSK 2]
    \definition{v.}{alcançar; atingir; realizar; concretizar; tornar (ideais, planos, etc.) realidade}
  \end{phonetics}
\end{entry}

\begin{entry}{实施}{8,9}{⼧、⽅}
  \begin{phonetics}{实施}{shi2shi1}[][HSK 4]
    \definition{v.}{colocar em vigor; implementar (leis, políticas, etc.); executar; trazer (colocar) algo em vigor; fazer cumprir; colocar algo em (prática)}
  \end{phonetics}
\end{entry}

\begin{entry}{实验}{8,10}{⼧、⾺}
  \begin{phonetics}{实验}{shi2yan4}[][HSK 3]
    \definition[个,次]{s.}{teste; experimento; trabalho de laboratório}
    \definition{v.}{testar; experimentar}
  \end{phonetics}
\end{entry}

\begin{entry}{实验室}{8,10,9}{⼧、⾺、⼧}
  \begin{phonetics}{实验室}{shi2 yan4 shi4}[][HSK 3]
    \definition[个,间]{s.}{laboratório}
  \end{phonetics}
\end{entry}

\begin{entry}{实惠}{8,12}{⼧、⼼}
  \begin{phonetics}{实惠}{shi2hui4}[][HSK 5]
    \definition{adj.}{sólido; substancial; benefícios práticos}
    \definition{s.}{benefício material; benefícios tangíveis; benefícios reais}
  \end{phonetics}
\end{entry}

\begin{entry}{宠物}{8,8}{⼧、⽜}
  \begin{phonetics}{宠物}{chong3wu4}
    \definition{s.}{animal de estimação}
  \end{phonetics}
\end{entry}

\begin{entry}{客人}{9,2}{⼧、⼈}
  \begin{phonetics}{客人}{ke4ren2}[][HSK 2]
    \definition[位,个,桌,拨,批]{s.}{visitante; convidado | cliente; passageiro; hóspede; viajante}
  \end{phonetics}
\end{entry}

\begin{entry}{客厅}{9,4}{⼧、⼚}
  \begin{phonetics}{客厅}{ke4ting1}[][HSK 5]
    \definition[间,个]{s.}{sala de estar; sala de visitas; sala para receber convidados}
  \end{phonetics}
\end{entry}

\begin{entry}{客户}{9,4}{⼧、⼾}
  \begin{phonetics}{客户}{ke4hu4}[][HSK 5]
    \definition{s.}{cliente; consumidor}
  \end{phonetics}
\end{entry}

\begin{entry}{客气}{9,4}{⼧、⽓}
  \begin{phonetics}{客气}{ke4qi5}[][HSK 5]
    \definition{adj.}{educado; modesto; cortês}
    \definition{v.}{ser educado; ser cortês; fazer comentários educados ou agir educadamente}
  \end{phonetics}
\end{entry}

\begin{entry}{客观}{9,6}{⼧、⾒}
  \begin{phonetics}{客观}{ke4guan1}[][HSK 3]
    \definition{adj.}{objetivo; justo e razoável; imparcial}
    \definition{s.}{objetivo}
  \end{phonetics}
\end{entry}

\begin{entry}{宣布}{9,5}{⼧、⼱}
  \begin{phonetics}{宣布}{xuan1bu4}[][HSK 3]
    \definition{v.}{declarar; proclamar; pronunciar; anunciar | anunciar oficialmente e publicamente as últimas decisões e situações a todos}
  \end{phonetics}
\end{entry}

\begin{entry}{宣传}{9,6}{⼧、⼈}
  \begin{phonetics}{宣传}{xuan1chuan2}[][HSK 3]
    \definition{v.}{propagar; disseminar; conduzir propaganda | explicar às massas para que elas possam acreditar e agir de acordo}
  \end{phonetics}
\end{entry}

\begin{entry}{宣扬}{9,6}{⼧、⼿}
  \begin{phonetics}{宣扬}{xuan1yang2}
    \definition{v.}{divulgar | anunciar | espalhar por toda parte}
  \end{phonetics}
\end{entry}

\begin{entry}{室}{9}{⼧}
  \begin{phonetics}{室}{shi4}[][HSK 3]
    \definition*{s.}{sobrenome Shi}
    \definition*{s.}{Shi, uma das mansões lunares}
    \definition{s.}{sala; aposento; cômodo |  seção; escritório | esposa}
  \end{phonetics}
\end{entry}

\begin{entry}{宪制}{9,8}{⼧、⼑}
  \begin{phonetics}{宪制}{xian4zhi4}
    \definition{adj.}{constitucional}
    \definition{s.}{sistema de governo constitucional}
  \end{phonetics}
\end{entry}

\begin{entry}{宪法法院}{9,8,8,9}{⼧、⽔、⽔、⾩}
  \begin{phonetics}{宪法法院}{xian4fa3fa3yuan4}
    \definition{s.}{tribunal constitucional}
  \end{phonetics}
\end{entry}

\begin{entry}{宪政}{9,9}{⼧、⽁}
  \begin{phonetics}{宪政}{xian4zheng4}
    \definition{s.}{governo constitucional}
  \end{phonetics}
\end{entry}

\begin{entry}{害}{10}{⼧}
  \begin{phonetics}{害}{hai4}[][HSK 5]
    \definition{adj.}{prejudicial; destrutivo; injurioso; nocivo}
    \definition{s.}{mal; maldade; dano; calamidade}
    \definition{v.}{prejudicar; fazer mal a; causar problemas a | matar; assassinar | sofrer de; contrair (uma doença) | sentir-se (envergonhado, com medo, etc.); despertar (um sentimento ou uma emoção)}
  \end{phonetics}
\end{entry}

\begin{entry}{害怕}{10,8}{⼧、⼼}
  \begin{phonetics}{害怕}{hai4pa4}[][HSK 3]
    \definition{v.}{estar assustado; ter medo}
  \end{phonetics}
\end{entry}

\begin{entry}{害羞}{10,10}{⼧、⽺}
  \begin{phonetics}{害羞}{hai4xiu1}
    \definition{adj.}{tímido | envergonhado}
  \end{phonetics}
\end{entry}

\begin{entry}{家}{10}{⼧}
  \begin{phonetics}{家}{jia1}[][HSK 1,2]
    \definition*{s.}{sobrenome Jia}
    \definition{adj.}{domado; domesticado; criado; alimentado | interno}
    \definition{clas.}{usado para famílias ou estabelecimentos comerciais; para uso doméstico; lojas; fábricas, etc.}
    \definition{pron.}{(educado) meu (irmã, tio, etc.)}
    \definition[个]{s.}{família; domicílio; clã | lar; casa; residência da família | pessoa ou família envolvida em um determinado comércio; pessoas que trabalham em determinada profissão ou que possuem determinada identidade | especialista em um determinado campo; pessoa que possui conhecimentos especializados ou se dedica a atividades específicas | escola de pensamento; rscola acadêmica | (em cartas de baralho, mah-jong etc.) festa; lado; refere-se a jogar xadrez ou cartas, em que uma das partes joga contra a outra | nacionalidade; referindo-se à etnia | membros da família; parentes; pessoas ou famílias com quem você tem algum tipo de relação | membro do mesmo clã; pessoas com o mesmo sobrenome}
    \definition{suf.}{sufixo substantivo para designar um especialista em alguma atividade, como um músico ou revolucionário, para designar uma profissão como em -eiro, -ista, por exemplo 科学家}
  \seealsoref{科学家}{ke1xue2jia1}
  \end{phonetics}
\end{entry}

\begin{entry}{家人}{10,2}{⼧、⼈}
  \begin{phonetics}{家人}{jia1 ren2}[][HSK 1]
    \definition{s.}{família (de alguém); membro da família; os membros de uma família}
  \end{phonetics}
\end{entry}

\begin{entry}{家乡}{10,3}{⼧、⼄}
  \begin{phonetics}{家乡}{jia1xiang1}[][HSK 3]
    \definition[个]{s.}{cidade natal}
  \end{phonetics}
\end{entry}

\begin{entry}{家长}{10,4}{⼧、⾧}
  \begin{phonetics}{家长}{jia1 zhang3}[][HSK 2]
    \definition[位,名,个]{s.}{pais; patriarca; tutor; guardião; refere-se aos pais ou outros responsáveis legais}
  \end{phonetics}
\end{entry}

\begin{entry}{家务}{10,5}{⼧、⼒}
  \begin{phonetics}{家务}{jia1wu4}[][HSK 4]
    \definition[堆,次,件]{s.}{trabalho doméstico; tarefas domésticas}
  \end{phonetics}
\end{entry}

\begin{entry}{家伙}{10,6}{⼧、⼈}
  \begin{phonetics}{家伙}{jia1huo5}
    \definition{s.}{prato, implemento ou móvel doméstico | animal doméstico | (coloquial) o cara | indivíduo | arma}
  \end{phonetics}
\end{entry}

\begin{entry}{家里}{10,7}{⼧、⾥}
  \begin{phonetics}{家里}{jia1 li3}[][HSK 1]
    \definition{s.}{(em) casa; (em sua) família | esposa}
  \end{phonetics}
\end{entry}

\begin{entry}{家具}{10,8}{⼧、⼋}
  \begin{phonetics}{家具}{jia1ju4}[][HSK 3]
    \definition[件,套]{s.}{móveis; mobiliário de casa}
  \end{phonetics}
\end{entry}

\begin{entry}{家庭}{10,9}{⼧、⼴}
  \begin{phonetics}{家庭}{jia1ting2}[][HSK 2]
    \definition[个,户]{s.}{família}
  \end{phonetics}
\end{entry}

\begin{entry}{家俱}{10,10}{⼧、⼈}
  \begin{phonetics}{家俱}{jia1ju4}
    \variantof{家具}
  \end{phonetics}
\end{entry}

\begin{entry}{家属}{10,12}{⼧、⼫}
  \begin{phonetics}{家属}{jia1shu3}[][HSK 3]
    \definition{s.}{membros da família; dependentes (familiares)}
  \end{phonetics}
\end{entry}

\begin{entry}{容易}{10,8}{⼧、⽇}
  \begin{phonetics}{容易}{rong2yi4}[][HSK 3]
    \definition{adj.}{fácil; simples | provável; responsável; apto}
  \end{phonetics}
\end{entry}

\begin{entry}{容貌}{10,14}{⼧、⾘}
  \begin{phonetics}{容貌}{rong2mao4}
    \definition{s.}{aparência | aspecto | características}
  \end{phonetics}
\end{entry}

\begin{entry}{宽}{10}{⼧}
  \begin{phonetics}{宽}{kuan1}[][HSK 4]
    \definition*{s.}{sobrenome Kuan}
    \definition{adj.}{largo; amplo; grandes distâncias horizontais | leniente; generoso; indulgente | bem de vida; confortável | espaçoso}
    \definition{s.}{largura; amplitude}
    \definition{v.}{relaxar; aliviar}
  \end{phonetics}
\end{entry}

\begin{entry}{宽广}{10,3}{⼧、⼴}
  \begin{phonetics}{宽广}{kuan1 guang3}[][HSK 4]
    \definition{adj.}{vasto; amplo; espaçoso; extenso}
  \end{phonetics}
\end{entry}

\begin{entry}{宽度}{10,9}{⼧、⼴}
  \begin{phonetics}{宽度}{kuan1 du4}[][HSK 5]
    \definition{s.}{largura; amplitude; duração; o grau de largura e estreiteza; a distância horizontal (no caso de um retângulo, a distância entre os dois lados mais longos)}
  \end{phonetics}
\end{entry}

\begin{entry}{宽影片}{10,15,4}{⼧、⼺、⽚}
  \begin{phonetics}{宽影片}{kuan1ying3pian4}
    \definition{s.}{filme \emph{widescreen}}
  \end{phonetics}
\end{entry}

\begin{entry}{宾馆}{10,11}{⼧、⾷}
  \begin{phonetics}{宾馆}{bin1guan3}[][HSK 5]
    \definition[家,个,座]{s.}{hotel; acomodações públicas para hóspedes}
  \end{phonetics}
\end{entry}

\begin{entry}{宿舍}{11,8}{⼧、⾆}
  \begin{phonetics}{宿舍}{su4she4}[][HSK 5]
    \definition[间,幢]{s.}{alojamento; dormitório; república; albergue; casas onde escolas, empresas, etc. acomodam seus alunos ou funcionários}
  \end{phonetics}
\end{entry}

\begin{entry}{寂寞}{11,13}{⼧、⼧}
  \begin{phonetics}{寂寞}{ji4mo4}
    \definition{adj.}{sozinho | solitário | (de um lugar) silencioso}
  \end{phonetics}
\end{entry}

\begin{entry}{寂寥}{11,14}{⼧、⼧}
  \begin{phonetics}{寂寥}{ji4liao2}
    \definition{s.}{solidão | vasto e vazio | quieto e desolado (literário)}
  \end{phonetics}
\end{entry}

\begin{entry}{寄}{11}{⼧}
  \begin{phonetics}{寄}{ji4}[][HSK 4]
    \definition{adj.}{adotado; fomentado; promovido}
    \definition{v.}{enviar; postar; remeter | confiar; depositar; colocar | depender de; apegar-se a}
  \end{phonetics}
\end{entry}

\begin{entry}{寄予}{11,4}{⼧、⼅}
  \begin{phonetics}{寄予}{ji4yu3}
    \definition{v.}{expressar | colocar (esperança, importância, etc.) em | mostrar}
  \end{phonetics}
\end{entry}

\begin{entry}{寄生}{11,5}{⼧、⽣}
  \begin{phonetics}{寄生}{ji4sheng1}
    \definition{s.}{parasita | parasitismo}
    \definition{v.}{viver tirando vantagem dos outros | viver dentro ou sobre outro organismo como um parasita}
  \end{phonetics}
\end{entry}

\begin{entry}{寄生生活}{11,5,5,9}{⼧、⽣、⽣、⽔}
  \begin{phonetics}{寄生生活}{ji4sheng1sheng1huo2}
    \definition{s.}{parasitismo | vida parasitária}
  \end{phonetics}
\end{entry}

\begin{entry}{寄存}{11,6}{⼧、⼦}
  \begin{phonetics}{寄存}{ji4cun2}
    \definition{v.}{depositar | deixar algo com alguém | armazenar}
  \end{phonetics}
\end{entry}

\begin{entry}{寄托}{11,6}{⼧、⼿}
  \begin{phonetics}{寄托}{ji4tuo1}
    \definition{v.}{investir (sua esperança, energia, etc.) em algo | confiar (a alguém) | colocar (a esperança, a energia, etc.) em}
  \end{phonetics}
\end{entry}

\begin{entry}{寄卖}{11,8}{⼧、⼗}
  \begin{phonetics}{寄卖}{ji4mai4}
    \definition{v.}{consignar para venda}
  \end{phonetics}
\end{entry}

\begin{entry}{寄居}{11,8}{⼧、⼫}
  \begin{phonetics}{寄居}{ji4ju1}
    \definition{s.}{morar longe de casa}
  \end{phonetics}
\end{entry}

\begin{entry}{寄放}{11,8}{⼧、⽅}
  \begin{phonetics}{寄放}{ji4fang4}
    \definition{v.}{deixar algo com alguém}
  \end{phonetics}
\end{entry}

\begin{entry}{寄养}{11,9}{⼧、⼋}
  \begin{phonetics}{寄养}{ji4yang3}
    \definition{v.}{embarcar | promover | colocar sob os cuidados de alguém (uma criança, animal de estimação, etc.)}
  \end{phonetics}
\end{entry}

\begin{entry}{寄送}{11,9}{⼧、⾡}
  \begin{phonetics}{寄送}{ji4song4}
    \definition{v.}{enviar | transmitir}
  \end{phonetics}
\end{entry}

\begin{entry}{寄递}{11,10}{⼧、⾡}
  \begin{phonetics}{寄递}{ji4di4}
    \definition{s.}{entrega de correspondência}
  \end{phonetics}
\end{entry}

\begin{entry}{寄售}{11,11}{⼧、⼝}
  \begin{phonetics}{寄售}{ji4shou4}
    \definition{v.}{venda em consignação}
  \end{phonetics}
\end{entry}

\begin{entry}{寄宿}{11,11}{⼧、⼧}
  \begin{phonetics}{寄宿}{ji4su4}
    \definition{s.}{embarque}
    \definition{v.}{embarcar}
  \end{phonetics}
\end{entry}

\begin{entry}{寄望}{11,11}{⼧、⽉}
  \begin{phonetics}{寄望}{ji4wang4}
    \definition{v.}{depositar esperanças em}
  \end{phonetics}
\end{entry}

\begin{entry}{密}{11}{⼧}
  \begin{phonetics}{密}{mi4}[][HSK 4]
    \definition*{s.}{sobrenome Mi}
    \definition{adj.}{fechado; denso; espesso | íntimo; próximo; afetuoso | delicado; fino; cuidadoso; meticuloso}
    \definition{adv.}{secretamente}
    \definition{s.}{segredo | densidade}
  \end{phonetics}
\end{entry}

\begin{entry}{密切}{11,4}{⼧、⼑}
  \begin{phonetics}{密切}{mi4qie4}[][HSK 4]
    \definition{adj.}{próximo; íntimo; relacionamento próximo}
    \definition{adv.}{cuidadosamente; atentamente; intimamente}
    \definition{v.}{tornar-se próximo; tornar-se íntimo; conectar-se}
  \end{phonetics}
\end{entry}

\begin{entry}{密码}{11,8}{⼧、⽯}
  \begin{phonetics}{密码}{mi4ma3}[][HSK 4]
    \definition[个]{s.}{código; senha;}
  \end{phonetics}
\end{entry}

\begin{entry}{富}{12}{⼧}
  \begin{phonetics}{富}{fu4}[][HSK 3]
    \definition*{s.}{sobrenome Fu}
    \definition{adj.}{rico; póspero | rico; abundante}
    \definition{s.}{fortuna; riqueza}
  \end{phonetics}
\end{entry}

\begin{entry}{寒冷}{12,7}{⼧、⼎}
  \begin{phonetics}{寒冷}{han2 leng3}[][HSK 4]
    \definition{adj.}{frio; frígido; gélido; gelado}
  \end{phonetics}
\end{entry}

\begin{entry}{寒假}{12,11}{⼧、⼈}
  \begin{phonetics}{寒假}{han2jia4}[][HSK 4]
    \definition[个]{s.}{férias de inverno (feriados); férias escolares no meio do inverno, em janeiro e fevereiro (na China)}
  \end{phonetics}
\end{entry}

\begin{entry}{寓意}{12,13}{⼧、⼼}
  \begin{phonetics}{寓意}{yu4yi4}
    \definition{s.}{moral (de uma história),  lição a ser aprendida, implicação, mensagem, significado metafórico}
  \end{phonetics}
\end{entry}

\begin{entry}{寡}{14}{⼧}
  \begin{phonetics}{寡}{gua3}
    \definition{adj.}{poucos; escassos (oposto a 众, 多)  | insípido; sem sabor | pouco; escasso | insípido; sem graça}
  \seealsoref{多}{duo1}
  \seealsoref{众}{zhong4}
  \end{phonetics}
\end{entry}

\begin{entry}{寨}{14}{⼧}
  \begin{phonetics}{寨}{zhai4}
    \definition{s.}{fortaleza | paliçada | acampamento | vila (paliçada)}
  \end{phonetics}
\end{entry}

%%%%% EOF %%%%%

