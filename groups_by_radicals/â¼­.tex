%%%
%%% Radical "⼭"
%%%

\section*{Radical 46: ``⼭''}\addcontentsline{toc}{section}{Radical 46: ⼭}

\begin{entry}{山}{3}{⼭}[Kangxi 46]
  \begin{phonetics}{山}{shan1}[][HSK 1]
    \definition*{s.}{sobrenome Shan}
    \definition[座]{s.}{colina; maciço; montanha | qualquer coisa que se assemelhe a uma montanha | arbustos nos quais os bichos-da-seda tecem seus casulos; referindo-se a casulos de bicho-da-seda | eco; metáfora para um som muito alto}
  \end{phonetics}
\end{entry}

\begin{entry}{山区}{3,4}{⼭、⼖}
  \begin{phonetics}{山区}{shan1 qu1}[][HSK 5]
    \definition[个]{s.}{área montanhosa; região montanhosa | colina; serra; montanha | distrito montanhoso}
  \end{phonetics}
\end{entry}

\begin{entry}{山东}{3,5}{⼭、⼀}
  \begin{phonetics}{山东}{shan1dong1}
    \definition*{s.}{Shandong}
  \end{phonetics}
\end{entry}

\begin{entry}{山羊}{3,6}{⼭、⽺}
  \begin{phonetics}{山羊}{shan1yang2}
    \definition{s.}{cabra | (ginástica) cavalo de salto de pequeno porte}
  \end{phonetics}
\end{entry}

\begin{entry}{山体}{3,7}{⼭、⼈}
  \begin{phonetics}{山体}{shan1ti3}
    \definition{s.}{forma de uma montanha}
  \end{phonetics}
\end{entry}

\begin{entry}{山谷}{3,7}{⼭、⾕}
  \begin{phonetics}{山谷}{shan1gu3}
    \definition{s.}{vale | ravina}
  \end{phonetics}
\end{entry}

\begin{entry}{山顶}{3,8}{⼭、⾴}
  \begin{phonetics}{山顶}{shan1ding3}
    \definition{s.}{cume da montanha}
  \end{phonetics}
\end{entry}

\begin{entry}{山寨}{3,14}{⼭、⼧}
  \begin{phonetics}{山寨}{shan1zhai4}
    \definition{s.}{fortaleza fortificada da vila | fortaleza da montanha (especialmente de bandidos) | falsificação | imitação | (fig.) pechincha}
  \end{phonetics}
\end{entry}

\begin{entry}{岁}{6}{⼭}
  \begin{phonetics}{岁}{sui4}[][HSK 1]
    \definition{clas.}{usado para anos (de idade)}
    \definition{s.}{ano (literário) | colheita do ano (literário) | idade | tempo (literário) | ano (de idade) | ano (para as colheitas)}
  \end{phonetics}
\end{entry}

\begin{entry}{岁月}{6,4}{⼭、⽉}
  \begin{phonetics}{岁月}{sui4yue4}[][HSK 5]
    \definition{s.}{anos; ano e mês; refere-se a tempo em geral}
  \end{phonetics}
\end{entry}

\begin{entry}{岂}{6}{⼭}
  \begin{phonetics}{岂}{qi3}
    \definition*{s.}{sobrenome Qi}
    \definition{adv.}{expressa uma pergunta retórica, equivalente a 哪里, 怎么 e 难道}
  \seealsoref{哪里}{na3 li3}
  \seealsoref{难道}{nan2dao4}
  \seealsoref{怎么}{zen3me5}
  \end{phonetics}
\end{entry}

\begin{entry}{岂有此理}{6,6,6,11}{⼭、⽉、⽌、⽟}
  \begin{phonetics}{岂有此理}{qi3you3ci3li3}
    \definition{interj.}{Que exorbitante! | Absurdo! | Como isso pode ser assim? | Ridículo!}
  \end{phonetics}
\end{entry}

\begin{entry}{岭}{8}{⼭}
  \begin{phonetics}{岭}{ling3}
    \definition{s.}{cordilheira}
  \end{phonetics}
\end{entry}

\begin{entry}{岸}{8}{⼭}
  \begin{phonetics}{岸}{an4}[][HSK 5]
    \definition{adj.}{elevado; grandioso (de maneira sombria ou condescendente)}
    \definition[个]{s.}{margem; costa; litoral; terreno à beira da água}
  \end{phonetics}
\end{entry}

\begin{entry}{岸上}{8,3}{⼭、⼀}
  \begin{phonetics}{岸上}{an4 shang4}[][HSK 5]
    \definition{s.}{em terra; costa; margem | na margem do rio; na beira do rio}
  \end{phonetics}
\end{entry}

\begin{entry}{崇}{11}{⼭}
  \begin{phonetics}{崇}{chong2}
    \definition*{s.}{sobrenome Chong}
    \definition{adj.}{alto | sublime | elevado}
    \definition{v.}{estimar | adorar}
  \end{phonetics}
\end{entry}

\begin{entry}{崖}{11}{⼭}
  \begin{phonetics}{崖}{ya2}
    \definition{s.}{precipício | penhasco}
  \end{phonetics}
\end{entry}

\begin{entry}{崩}{11}{⼭}
  \begin{phonetics}{崩}{beng1}
    \definition{s.}{morte de rei ou imperador | desaparecimento}
    \definition{v.}{entrar em colapso | cair em ruínas}
  \end{phonetics}
\end{entry}

%%%%% EOF %%%%%

