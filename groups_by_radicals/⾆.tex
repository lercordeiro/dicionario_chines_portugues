%%%
%%% Radical "⾆"
%%%

\section*{Radical 135: ``⾆''}\addcontentsline{toc}{section}{Radical 135: ⾆}

\begin{Entry}{舌}{6}{⾆}[Kangxi 135]
  \begin{Phonetics}{舌}{she2}
    \definition*{s.}{Sobrenome She}
    \definition[片,条]{s.}{língua (de um ser humano ou animal); glossa | algo em forma de língua | língua de sino; badalo}
  \end{Phonetics}
\end{Entry}

\begin{Entry}{舌头}{6,5}{⾆、⼤}
  \begin{Phonetics}{舌头}{she2tou5}[][HSK 6]
    \definition[个]{s.}{língua; órgão que auxilia no paladar, na mastigação e na pronúncia | espião}
  \end{Phonetics}
\end{Entry}

\begin{Entry}{舍}{8}{⾆}
  \begin{Phonetics}{舍}{she3}
    \definition{v.}{abandonar; desistir; descartar; jogar fora | dar esmola; dispensar caridade}
  \end{Phonetics}
  \begin{Phonetics}{舍}{she4}
    \definition*{s.}{Sobrenome She}
    \definition{clas.}{uma unidade antiga de distância igual a 30 li, 里}
    \definition{pron.}{meu, uma palavra humilde usada para se referir aos parentes mais jovens ou de geração inferior}
    \definition{s.}{cabana; casa | minha casa; minha humilde morada | chiqueiro; galpão; curral de gado}
  \seealsoref{里}{li3}
  \end{Phonetics}
\end{Entry}

\begin{Entry}{舍不得}{8,4,11}{⾆、⼀、⼻}
  \begin{Phonetics}{舍不得}{she3bu5de5}[][HSK 5]
    \definition{v.}{não se pode abandonar ou deixar, não se quer usar ou descartar; detestar separar-me ou usar}
  \end{Phonetics}
\end{Entry}

\begin{Entry}{舍得}{8,11}{⾆、⼻}
  \begin{Phonetics}{舍得}{she3 de5}[][HSK 5]
    \definition{v.}{não guardar rancor; estar disposto a abrir mão de algo; estar disposto a gastar dinheiro, tempo, etc.; estar disposto a abrir mão de pessoas, oportunidades, coisas, etc. que são importantes para você}
  \end{Phonetics}
\end{Entry}

\begin{Entry*}{敌}{10}{⾆}
  \begin{Phonetics}{敌}{di2}
    \definition[个,名,位,种]{s.}{inimigo; adversário}
    \definition{v.}{opor-se; lutar; resistir; suportar | combinar; igualar}
  \end{Phonetics}
\end{Entry*}

\begin{Entry}{敌人}{10,2}{⾆、⼈}
  \begin{Phonetics}{敌人}{di2ren2}[][HSK 4]
    \definition[群,伙,帮,个,队]{s.}{inimigo; pessoa hostil; parte hostil}
  \end{Phonetics}
\end{Entry}

\begin{Entry}{舒}{12}{⾆}
  \begin{Phonetics}{舒}{shu1}
    \definition*{s.}{Sobrenome Shu}
    \definition{adj.}{lento; vagaroso; sem pressa | confortável; relaxado e feliz}
    \definition{v.}{esticar; desdobrar | alongar; relaxar}
  \end{Phonetics}
\end{Entry}

\begin{Entry}{舒服}{12,8}{⾆、⽉}
  \begin{Phonetics}{舒服}{shu1fu5}[][HSK 2]
    \definition{adj.}{confortável; sentir-se relaxado e feliz, tanto física quanto mentalmente}
  \end{Phonetics}
\end{Entry}

\begin{Entry}{舒适}{12,9}{⾆、⾡}
  \begin{Phonetics}{舒适}{shu1shi4}[][HSK 4]
    \definition{adj.}{aconchegante; confortável; acolhedor; cômodo}
  \end{Phonetics}
\end{Entry}

%%%%% EOF %%%%%

