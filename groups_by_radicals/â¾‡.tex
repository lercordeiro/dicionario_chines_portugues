%%%
%%% Radical "⾇"
%%%

\section*{Radical 136: ``⾇''}\addcontentsline{toc}{section}{Radical 136: ⾇}

\begin{entry}{舞}{14}{⾇}
  \begin{phonetics}{舞}{wu3}[][HSK 5]
    \definition{s.}{dança | palco; metáfora do domínio das atividades sociais}
    \definition{v.}{mover-se como numa dança | dançar com algo nas mãos; brincar com | florescer; empunhar; brandir | esvoaçar | fazer malabarismos; brincar com}
  \end{phonetics}
\end{entry}

\begin{entry}{舞厅}{14,4}{⾇、⼚}
  \begin{phonetics}{舞厅}{wu3ting1}
    \definition[间]{s.}{salão de dança | salão de baile}
  \end{phonetics}
\end{entry}

\begin{entry}{舞厅舞}{14,4,14}{⾇、⼚、⾇}
  \begin{phonetics}{舞厅舞}{wu3ting1wu3}
    \definition{s.}{dança de salão}
  \end{phonetics}
\end{entry}

\begin{entry}{舞台}{14,5}{⾇、⼝}
  \begin{phonetics}{舞台}{wu3 tai2}[][HSK 3]
    \definition[个]{s.}{palco; arena}
  \end{phonetics}
\end{entry}

\begin{entry}{舞会}{14,6}{⾇、⼈}
  \begin{phonetics}{舞会}{wu3hui4}
    \definition{s.}{baile}
  \end{phonetics}
\end{entry}

\begin{entry}{舞会舞}{14,6,14}{⾇、⼈、⾇}
  \begin{phonetics}{舞会舞}{wu3hui4wu3}
    \definition{s.}{baile}
  \end{phonetics}
\end{entry}

\begin{entry}{舞抃}{14,7}{⾇、⼿}
  \begin{phonetics}{舞抃}{wu3bian4}
    \definition{s.}{dançar por prazer}
  \end{phonetics}
\end{entry}

\begin{entry}{舞蹈}{14,17}{⾇、⾜}
  \begin{phonetics}{舞蹈}{wu3dao3}
    \definition{s.}{dança (ato performático)}
  \end{phonetics}
\end{entry}

%%%%% EOF %%%%%

