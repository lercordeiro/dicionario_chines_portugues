%%%
%%% Radical "⾇"
%%%

\section*{Radical 136: ``⾇''}\addcontentsline{toc}{section}{Radical 136: ⾇}

\begin{Entry}{舞}{14}{⾇}
  \begin{Phonetics}{舞}{wu3}[][HSK 5]
    \definition[支,段,个]{s.}{dança | palco; metáfora do domínio das atividades sociais}
    \definition{v.}{mover-se como numa dança | dançar com algo nas mãos; brincar com | florescer; empunhar; brandir | esvoaçar | fazer malabarismos; brincar com}
  \end{Phonetics}
\end{Entry}

\begin{Entry}{舞厅}{14,4}{⾇、⼚}
  \begin{Phonetics}{舞厅}{wu3ting1}
    \definition[间]{s.}{salão de dança | salão de baile}
  \end{Phonetics}
\end{Entry}

\begin{Entry}{舞厅舞}{14,4,14}{⾇、⼚、⾇}
  \begin{Phonetics}{舞厅舞}{wu3ting1wu3}
    \definition{s.}{dança de salão}
  \end{Phonetics}
\end{Entry}

\begin{Entry}{舞台}{14,5}{⾇、⼝}
  \begin{Phonetics}{舞台}{wu3 tai2}[][HSK 3]
    \definition[个]{s.}{palco; plataforma elevada usada exclusivamente para apresentações artísticas, geralmente localizada na parte frontal de teatros e auditórios | palco; metáfora do campo das atividades sociais}
  \end{Phonetics}
\end{Entry}

\begin{Entry}{舞会}{14,6}{⾇、⼈}
  \begin{Phonetics}{舞会}{wu3hui4}
    \definition{s.}{baile}
  \end{Phonetics}
\end{Entry}

\begin{Entry}{舞会舞}{14,6,14}{⾇、⼈、⾇}
  \begin{Phonetics}{舞会舞}{wu3hui4wu3}
    \definition{s.}{baile}
  \end{Phonetics}
\end{Entry}

\begin{Entry}{舞抃}{14,7}{⾇、⼿}
  \begin{Phonetics}{舞抃}{wu3bian4}
    \definition{s.}{dançar por prazer}
  \end{Phonetics}
\end{Entry}

\begin{Entry}{舞蹈}{14,17}{⾇、⾜}
  \begin{Phonetics}{舞蹈}{wu3dao3}[][HSK 6]
    \definition[段,支,场,个]{s.}{dança; uma forma de arte que usa movimentos rítmicos como principal meio de expressão, podendo expressar a vida, os pensamentos e os sentimentos das pessoas, geralmente acompanhada de música}
    \definition{v.}{dançar}
  \end{Phonetics}
\end{Entry}

%%%%% EOF %%%%%

