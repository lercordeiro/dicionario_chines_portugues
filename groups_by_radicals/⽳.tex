%%%
%%% Radical "⽳"
%%%

\section*{Radical 116: ``⽳''}\addcontentsline{toc}{section}{Radical 116: ⽳}

\begin{entry}{究}{7}{⽳}
  \begin{phonetics}{究}{jiu1}
    \definition{adv.}{na verdade; realmente; afinal}
    \definition{v.}{estudar cuidadosamente; aprofundar; investigar; rastrear}
  \end{phonetics}
\end{entry}

\begin{entry}{究竟}{7,11}{⽳、⾳}
  \begin{phonetics}{究竟}{jiu1jing4}[][HSK 4]
    \definition{adv.}{de fato; exatamente; usado em frases interrogativas para buscar | afinal de contas, no final; ênfase em fatos ou motivos}
    \definition{s.}{resultado; desfecho; a causa, o efeito ou a história completa do que aconteceu}
  \end{phonetics}
\end{entry}

\begin{entry}{穷}{7}{⽳}
  \begin{phonetics}{穷}{qiong2}[][HSK 4]
    \definition{adj.}{remoto; isolado; de difícil acesso | pobre; atingido pela pobreza | situação difícil, sem saída}
    \definition{adv.}{completamente | extremamente}
    \definition{v.}{exaurir; esgotar; consmir | ir até o fim; perseguir completamente perseguido; sondar profundamente | gastar}
  \end{phonetics}
\end{entry}

\begin{entry}{穷人}{7,2}{⽳、⼈}
  \begin{phonetics}{穷人}{qiong2 ren2}[][HSK 4]
    \definition{s.}{os pobres; pessoas pobres}
  \end{phonetics}
\end{entry}

\begin{entry}{空}{8}{⽳}
  \begin{phonetics}{空}{kong1}[][HSK 3]
    \definition*{s.}{Sobrenome Kong}
    \definition{adj.}{vazio; oco; nulo; não inclui nada; não contém nada ou não tem conteúdo; irrealista}
    \definition{adv.}{por nada; em vão; sem efeito}
    \definition{s.}{céu; ar | vazio; vazio do mundo dos sentidos}
  \end{phonetics}
  \begin{phonetics}{空}{kong4}[][HSK 4]
    \definition{adj.}{vazio; oco; nulo; que não contém nada; que não tem nada ou nenhum conteúdo; impraticável}
    \definition{adv.}{para nada; em vão; sem efeito}
    \definition{s.}{céu; ar | vazio; ausência do mundo dos sentidos}
  \end{phonetics}
\end{entry}

\begin{entry}{空儿}{8,2}{⽳、⼉}
  \begin{phonetics}{空儿}{kong4r5}[][HSK 3]
    \definition[个]{s.}{tempo livre; sem horário específico | sala; espaço (não utilizado); área ainda não utilizada}
    \definition{v.}{ter tempo livre}
  \end{phonetics}
\end{entry}

\begin{entry}{空中}{8,4}{⽳、⼁}
  \begin{phonetics}{空中}{kong1 zhong1}[][HSK 5]
    \definition{adj.}{aéreo; aerotransportado; refere-se à transmissão de sinais de rádio}
    \definition{s.}{no céu; no ar}
  \end{phonetics}
\end{entry}

\begin{entry}{空中小姐}{8,4,3,8}{⽳、⼁、⼩、⼥}
  \begin{phonetics}{空中小姐}{kong1zhong1xiao3jie3}
    \definition{s.}{aeromoça}
  \end{phonetics}
\end{entry}

\begin{entry}{空心菜}{8,4,11}{⽳、⼼、⾋}
  \begin{phonetics}{空心菜}{kong1xin1cai4}
    \definition{s.}{espinafre aquático | \emph{ong choy} | repolho do pântano | convolvulus aquático | glória-da-manhã aquática}
  \seealsoref{蕹菜}{weng4cai4}
  \end{phonetics}
\end{entry}

\begin{entry}{空气}{8,4}{⽳、⽓}
  \begin{phonetics}{空气}{kong1qi4}[][HSK 2]
    \definition[缕,股,份,阵]{s.}{ar; gases que compõe a atmosfera terrestre | atmosfera}
  \end{phonetics}
\end{entry}

\begin{entry}{空军}{8,6}{⽳、⼍}
  \begin{phonetics}{空军}{kong1 jun1}[][HSK 6]
    \definition[名,位,个,支]{s.}{força aérea; um exército que luta no ar, geralmente composto por várias unidades de aviação e unidades terrestres da força aérea}
  \end{phonetics}
\end{entry}

\begin{entry}{空间}{8,7}{⽳、⾨}
  \begin{phonetics}{空间}{kong1jian1}[][HSK 4]
    \definition[个]{s.}{espaço; recinto; cômodo; espaço em branco; interespaço}
  \end{phonetics}
\end{entry}

\begin{entry}{空间站}{8,7,10}{⽳、⾨、⽴}
  \begin{phonetics}{空间站}{kong1jian1zhan4}
    \definition{s.}{estação espacial}
  \end{phonetics}
\end{entry}

\begin{entry}{空姐}{8,8}{⽳、⼥}
  \begin{phonetics}{空姐}{kong1jie3}
    \definition{s.}{aeromoça | comissária de bordo | abreviação de 空中小姐}
  \seealsoref{空中小姐}{kong1zhong1xiao3jie3}
  \end{phonetics}
\end{entry}

\begin{entry}{空调}{8,10}{⽳、⾔}
  \begin{phonetics}{空调}{kong1tiao2}[][HSK 3]
    \definition[台,个]{s.}{ar-condicionado;  condicionador de ar}
  \end{phonetics}
\end{entry}

\begin{entry}{穿}{9}{⽳}
  \begin{phonetics}{穿}{chuan1}[][HSK 1]
    \definition{adj.}{direto; através; usado após certos verbos, indica um estado de revelação completa}
    \definition{s.}{vestuário; roupas; refere-se a roupas, sapatos, meias, etc.}
    \definition{v.}{usar; vestir; estar vestido; ter \dots vestido;  vestir roupas, sapatos, meias, etc. | perfurar através de; penetrar; formar orifícios por meio de cinzéis, brocas ou pontas afiadas | enfiar; amarrar; usar cordas e fios para ligar coisas | passar por; atravessar; passar por; através de (buracos, fendas, espaços vazios, etc.)}
  \end{phonetics}
\end{entry}

\begin{entry}{穿上}{9,3}{⽳、⼀}
  \begin{phonetics}{穿上}{chuan1 shang4}[][HSK 4]
    \definition{v.}{vestir (roupas, etc.); colocar roupas}
  \end{phonetics}
\end{entry}

\begin{entry}{突}{9}{⽳}
  \begin{phonetics}{突}{tu1}
    \definition{adv.}{de repente; abruptamente; inesperadamente}
    \definition{s.}{chaminé}
    \definition{v.}{avançar rapidamente; atacar | projetar; destacar-se | romper | projetar-se; inchar; fazer bojo}
  \end{phonetics}
\end{entry}

\begin{entry}{突出}{9,5}{⽳、⼐}
  \begin{phonetics}{突出}{tu1chu1}[][HSK 3]
    \definition{adj.}{proeminente; excelente; mais que a média}
    \definition{v.}{romper | enfatizar; destacar; dar destaque a | sobressair; projetar-se; destacar-se}
  \end{phonetics}
\end{entry}

\begin{entry}{突破}{9,10}{⽳、⽯}
  \begin{phonetics}{突破}{tu1po4}[][HSK 5]
    \definition{v.}{romper; fazer uma descoberta revolucionária; concentrar esforços em um único ponto de ataque, reunir o sucesso | quebrar (limite); superar (dificuldade); superar dificuldades; ultrapassar os números ou limites anteriores, superar recordes anteriores, etc.; romper com as limitações e restrições anteriores}
  \end{phonetics}
\end{entry}

\begin{entry}{突然}{9,12}{⽳、⽕}
  \begin{phonetics}{突然}{tu1ran2}[][HSK 3]
    \definition{adj.}{repentino; abrupto; inesperado}
    \definition{adv.}{de repente; abruptamente; inesperadamente; subitamente}
  \end{phonetics}
\end{entry}

\begin{entry}{窗}{12}{⽳}
  \begin{phonetics}{窗}{chuang1}
    \definition[扇,个]{s.}{janela}
  \end{phonetics}
\end{entry}

\begin{entry}{窗口}{12,3}{⽳、⼝}
  \begin{phonetics}{窗口}{chuang1 kou3}[][HSK 6]
    \definition[个,号]{s.}{janela; em frente à janela; perto da janela | janela; postigo; refere-se a uma abertura especial em forma de janela | janela; meio; intermediário; peça de exibição; campo de testes; uma metáfora para um lugar com muitas interações com o mundo exterior e através do qual o entendimento mútuo é alcançado |  janela; uma metáfora para um lugar que pode refletir ou exibir a totalidade ou parte de algo |  caixa de diálogo; uma caixa de operação quadrada para aplicativos ou arquivos que aparece na tela do computador}
  \end{phonetics}
\end{entry}

\begin{entry}{窗子}{12,3}{⽳、⼦}
  \begin{phonetics}{窗子}{chuang1 zi5}[][HSK 4]
    \definition{s.}{janela}
  \end{phonetics}
\end{entry}

\begin{entry}{窗户}{12,4}{⽳、⼾}
  \begin{phonetics}{窗户}{chuang1hu5}[][HSK 4]
    \definition[个,扇,面,排]{s.}{janela; dispositivo de ventilação e transmissão de luz nas paredes}
  \end{phonetics}
\end{entry}

\begin{entry}{窗台}{12,5}{⽳、⼝}
  \begin{phonetics}{窗台}{chuang1 tai2}[][HSK 4]
    \definition{s.}{parapeito da janela; peitoril; parte plana de uma janela que segura a moldura}
  \end{phonetics}
\end{entry}

\begin{entry}{窗帘}{12,8}{⽳、⼱}
  \begin{phonetics}{窗帘}{chuang1lian2}[][HSK 5]
    \definition[副,幅,个,套,片,对]{s.}{cortinas para janelas}
  \end{phonetics}
\end{entry}

\begin{entry}{窾}{17}{⽳}
  \begin{phonetics}{窾}{cuan4}
    \definition{adj.}{vazio | seco | destituído; pobre}
    \definition{s.}{buraco | lei}
    \definition{v.}{esconder}
  \end{phonetics}
  \begin{phonetics}{窾}{kuan3}
    \definition{adj.}{oco}
    \definition{s.}{rachadura; cavidade | (onomatopéia) água batendo na rocha}
    \definition{v.}{escavar um buraco}
  \end{phonetics}
\end{entry}

%%%%% EOF %%%%%

