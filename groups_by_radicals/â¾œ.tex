%%%
%%% Radical "⾜"
%%%

\section*{Radical 157: ``⾜''}\addcontentsline{toc}{section}{Radical 157: ⾜}

\begin{Entry}{足}{7}{⾜}[Kangxi 157]
  \begin{Phonetics}{足}{ju4}
    \definition{adj.}{excessivo}
  \end{Phonetics}
  \begin{Phonetics}{足}{zu2}[][HSK 6]
    \definition{adj.}{amplo; bastante; suficiente}
    \definition{adv.}{totalmente; tanto quanto; indica suficiente até certo ponto ou grau | (geralmente no negativo) o suficiente; suficientemente; é totalmente possível; vale a pena}
    \definition{s.}{pé; um termo geral para os membros inferiores do corpo humano; especificamente a parte abaixo do tornozelo | futebol; refere-se ao futebol ou a um time de futebol | pé (de utensílios); a parte inferior do objeto tem o formato de uma pé e serve de suporte}
  \end{Phonetics}
\end{Entry}

\begin{Entry}{足以}{7,4}{⾜、⼈}
  \begin{Phonetics}{足以}{zu2yi3}[][HSK 6]
    \definition{adj.}{suficiente; totalmente capaz de; o suficiente para fazer algo}
    \definition{v.}{bastar}
  \end{Phonetics}
\end{Entry}

\begin{Entry}{足月}{7,4}{⾜、⽉}
  \begin{Phonetics}{足月}{zu2yue4}
    \definition{s.}{gestação completa}
  \end{Phonetics}
\end{Entry}

\begin{Entry}{足足}{7,7}{⾜、⾜}
  \begin{Phonetics}{足足}{zu2zu2}
    \definition{adv.}{tanto quanto | extremamente | completamente | não menos que}
  \end{Phonetics}
\end{Entry}

\begin{Entry}{足够}{7,11}{⾜、⼣}
  \begin{Phonetics}{足够}{zu2 gou4}[][HSK 3]
    \definition{adj.}{bastante; amplo; suficiente; atingir o nível adequado ou capaz de satisfazer as necessidades}
    \definition{v.}{satisfazer; ser suficiente; estar a contento}
  \end{Phonetics}
\end{Entry}

\begin{Entry}{足球}{7,11}{⾜、⽟}
  \begin{Phonetics}{足球}{zu2qiu2}[][HSK 3]
    \definition[个,只,颗,袋]{s.}{futebol | bola de futebol}
  \end{Phonetics}
\end{Entry}

\begin{Entry}{足球队}{7,11,4}{⾜、⽟、⾩}
  \begin{Phonetics}{足球队}{zu2qiu2 dui4}
    \definition[个,支]{s.}{time de futebol}
  \end{Phonetics}
\end{Entry}

\begin{Entry}{足球协会}{7,11,6,6}{⾜、⽟、⼗、⼈}
  \begin{Phonetics}{足球协会}{zu2qiu2xie2hui4}
    \definition*{s.}{Associação de Futebol}
  \end{Phonetics}
\end{Entry}

\begin{Entry}{足球场}{7,11,6}{⾜、⽟、⼟}
  \begin{Phonetics}{足球场}{zu2qiu2chang3}
    \definition{s.}{campo de futebol}
  \end{Phonetics}
\end{Entry}

\begin{Entry}{足球迷}{7,11,9}{⾜、⽟、⾡}
  \begin{Phonetics}{足球迷}{zu2qiu2mi2}
    \definition{s.}{fã (ou entusiasta) de futebol}
  \end{Phonetics}
\end{Entry}

\begin{Entry}{足球赛}{7,11,14}{⾜、⽟、⾙}
  \begin{Phonetics}{足球赛}{zu2qiu2sai4}
    \definition{s.}{competição de futebol | partida de futebol}
  \end{Phonetics}
\end{Entry}

\begin{Entry}{距}{11}{⾜}
  \begin{Phonetics}{距}{ju4}
    \definition{s.}{distância | espora (de um galo, etc.)}
    \definition{v.}{estar separado (longe) de; estar distante de}
  \end{Phonetics}
\end{Entry}

\begin{Entry}{距离}{11,10}{⾜、⼇}
  \begin{Phonetics}{距离}{ju4li2}[][HSK 4]
    \definition[个,段]{s.}{distância}
    \definition{v.}{estar distante de}
  \end{Phonetics}
\end{Entry}

\begin{Entry}{跌}{12}{⾜}
  \begin{Phonetics}{跌}{die1}[][HSK 6]
    \definition{s.}{(de um objeto, etc.) queda; tombo | (de preços, etc.) queda}
    \definition{v.}{cair; tombar; perder o equilíbrio e cair | cair (objetos caindo); descer | cair (queda de preços)}
  \end{Phonetics}
\end{Entry}

\begin{Entry}{跑}{12}{⾜}
  \begin{Phonetics}{跑}{pao2}
    \definition{v.}{(de animais) bater com a pata (no chão); (de animais) escavar o solo com suas garras ou cascos}
  \end{Phonetics}
  \begin{Phonetics}{跑}{pao3}[][HSK 1]
    \definition{v.}{correr; pessoas ou animais que se movem rapidamente para a frente com as pernas e os pés | caminhar; passear | fugir; escapar | correr de um lado para outro; fazer rondas; correr atrás de algo | de um líquido ou gás) vazar; evaporar | (como complemento de um verbo) fora; longe | participar de uma corrida}
  \end{Phonetics}
\end{Entry}

\begin{Entry}{跑马}{12,3}{⾜、⾺}
  \begin{Phonetics}{跑马}{pao3ma3}
    \definition{s.}{corrida de cavalos}
    \definition{v.}{andar a cavalo em ritmo acelerado}
  \end{Phonetics}
\end{Entry}

\begin{Entry}{跑步}{12,7}{⾜、⽌}
  \begin{Phonetics}{跑步}{pao3/bu4}[][HSK 3]
    \definition{v.+compl.}{correr; trotar}
  \end{Phonetics}
\end{Entry}

\begin{Entry}{跑肚}{12,7}{⾜、⾁}
  \begin{Phonetics}{跑肚}{pao3du4}
    \definition{v.}{(coloquial) ter diarréia}
  \end{Phonetics}
\end{Entry}

\begin{Entry}{跑调}{12,10}{⾜、⾔}
  \begin{Phonetics}{跑调}{pao3diao4}
    \definition{v.}{(coloquial) estar fora do tom ou desafinado (enquanto canta)}
  \end{Phonetics}
\end{Entry}

\begin{Entry}{跑掉}{12,11}{⾜、⼿}
  \begin{Phonetics}{跑掉}{pao3diao4}
    \definition{v.}{fugir}
  \end{Phonetics}
\end{Entry}

\begin{Entry}{跑腿}{12,13}{⾜、⾁}
  \begin{Phonetics}{跑腿}{pao3tui3}
    \definition{v.}{realizar tarefas}
  \end{Phonetics}
\end{Entry}

\begin{Entry}{跑酷}{12,14}{⾜、⾣}
  \begin{Phonetics}{跑酷}{pao3ku4}
    \definition*{s.}{Eempréstimo linguístico: Parkour}
  \end{Phonetics}
\end{Entry}

\begin{Entry}{跑题}{12,15}{⾜、⾴}
  \begin{Phonetics}{跑题}{pao3ti2}
    \definition{v.}{divagar | fugir do assunto | tergiversar}
  \end{Phonetics}
\end{Entry}

\begin{Entry}{跟}{13}{⾜}
  \begin{Phonetics}{跟}{gen1}[][HSK 1]
    \definition{conj.}{e; expressa uma relação de união; 和}
    \definition{prep.}{com; Introduzir objetos relacionados à mesma ação, equivalente a 同 | para; em direção a | de; introduzir o objeto de comparação; equivalente a 从, 由 | como; objetos que causam comparações e semelhanças}
    \definition[个]{s.}{calcanhar; parte posterior do pé ou parte posterior do sapato ou meia | base (de um objeto)}
    \definition{v.}{seguir; acompanhar; seguir imediatamente na mesma direção | (de uma mulher) estar casada com; casar-se com alguém}
  \seealsoref{从}{cong2}
  \seealsoref{和}{he2}
  \seealsoref{同}{tong2}
  \seealsoref{由}{you2}
  \end{Phonetics}
\end{Entry}

\begin{Entry}{跟上}{13,3}{⾜、⼀}
  \begin{Phonetics}{跟上}{gen1shang5}[][HSK 7-9]
    \definition{v.}{acompanhar; alcançar; manter-se a par de}
  \end{Phonetics}
\end{Entry}

\begin{Entry}{跟不上}{13,4,3}{⾜、⼀、⼀}
  \begin{Phonetics}{跟不上}{gen1 bu5 shang4}[][HSK 7-9]
    \definition{v.}{não é capaz de acompanhar; não conseguir alcançar}
  \end{Phonetics}
\end{Entry}

\begin{Entry}{跟前}{13,9}{⾜、⼑}
  \begin{Phonetics}{跟前}{gen1qian2}[][HSK 5]
    \definition{s.}{próximo; perto de; na frente de; (na ou para) a presença de alguém | o tempo imediatamente anterior a algum evento; tempo que se aproxima}
  \end{Phonetics}
  \begin{Phonetics}{跟前}{gen1qian5}
    \definition{v.}{(filhos de alguém) viver com alguém (exclusivamente com relação à presença ou ausência de crianças)}
  \end{Phonetics}
\end{Entry}

\begin{Entry}{跟随}{13,11}{⾜、⾩}
  \begin{Phonetics}{跟随}{gen1sui2}[][HSK 5]
    \definition{s.}{seguidor; usado para se referir a alguém que seguiu}
    \definition{v.}{seguir; ir atrás; acompanhar}
  \end{Phonetics}
\end{Entry}

\begin{Entry}{跟踪}{13,15}{⾜、⾜}
  \begin{Phonetics}{跟踪}{gen1zong1}[][HSK 7-9]
    \definition{v.}{rastrear; alcançar; seguir; seguir atrás; seguir alguém; seguir os rastros de; seguir de perto}
  \end{Phonetics}
\end{Entry}

\begin{Entry}{跨}{13}{⾜}
  \begin{Phonetics}{跨}{kua4}[][HSK 6]
    \definition{adj.}{localizado ao lado de; anexo a}
    \definition{v.}{dar um passo; andar a passos largos | disputar; ficar de pernas abertas | atravessar; ir além (dos limites de uma certa quantidade, tempo, região, etc.)}
  \end{Phonetics}
\end{Entry}

\begin{Entry}{跪}{13}{⾜}
  \begin{Phonetics}{跪}{gui4}[][HSK 6]
    \definition{v.}{ajoelhar-se; dobrar os joelhos de modo que um ou ambos os joelhos toquem o chão}
  \end{Phonetics}
\end{Entry}

\begin{Entry}{跪拜}{13,9}{⾜、⼿}
  \begin{Phonetics}{跪拜}{gui4bai4}
    \definition{v.}{prostrar-se | ajoelhar-se e adorar}
  \end{Phonetics}
\end{Entry}

\begin{Entry}{路}{13}{⾜}
  \begin{Phonetics}{路}{lu4}[][HSK 1]
    \definition*{s.}{Sobrenome Lu}
    \definition{clas.}{tipo; classe | linha; coluna; usado para um grupo de pessoas ou uma equipe; para organizar em ordem}
    \definition[条]{s.}{estrada; caminho; via | viagem; jornada; distância | maneira; meios | sequência; linha; lógica | região; distrito | rota | classe; classificação; grau | linha; fileira}
  \end{Phonetics}
\end{Entry}

\begin{Entry}{路上}{13,3}{⾜、⼀}
  \begin{Phonetics}{路上}{lu4 shang5}[][HSK 1]
    \definition{s.}{na estrada | a caminho; na rota; em processo de mudança de um lugar para outro}
  \end{Phonetics}
\end{Entry}

\begin{Entry}{路口}{13,3}{⾜、⼝}
  \begin{Phonetics}{路口}{lu4 kou3}[][HSK 1]
    \definition[个]{s.}{cruzamento; intersecção; onde as estradas se encontram}
  \end{Phonetics}
\end{Entry}

\begin{Entry}{路边}{13,5}{⾜、⾡}
  \begin{Phonetics}{路边}{lu4 bian1}[][HSK 2]
    \definition{s.}{calçada; beira da estrada; margem da rua}
  \end{Phonetics}
\end{Entry}

\begin{Entry}{路过}{13,6}{⾜、⾡}
  \begin{Phonetics}{路过}{lu4 guo4}[][HSK 6]
    \definition{v.}{passar por (algum lugar); atravessar}
  \end{Phonetics}
\end{Entry}

\begin{Entry}{路线}{13,8}{⾜、⽷}
  \begin{Phonetics}{路线}{lu4 xian4}[][HSK 3]
    \definition[条]{s.}{rota; caminho; linha; a estrada percorrida de um lugar a outro | linha; diretriz (de política, ideologia, campo de trabalho); a via fundamental a seguir em termos ideológicos, políticos ou profissionais}
  \end{Phonetics}
\end{Entry}

\begin{Entry}{跳}{13}{⾜}
  \begin{Phonetics}{跳}{tiao4}[][HSK 3]
    \definition{v.}{pular; saltar | mover para cima e para baixo | pular (por cima); fazer omissões | quicar; a força elástica faz com que o objeto se mova repentinamente para cima | pulsar; palpitar; contrair-se | pular sobre;  saltar sobre; cruzar}
  \end{Phonetics}
\end{Entry}

\begin{Entry}{跳水}{13,4}{⾜、⽔}
  \begin{Phonetics}{跳水}{tiao4 shui3}[][HSK 6]
    \definition{s.}{Esporte: mergulho}
    \definition{v.}{mergulhar | Figurativo: (preços, lucros, etc.) cair drasticamente; cair repentinamente; mergulhar; despencar | cometer suicídio pulando na água | mergulhar (na água)}
  \end{Phonetics}
\end{Entry}

\begin{Entry}{跳电}{13,5}{⾜、⽥}
  \begin{Phonetics}{跳电}{tiao4dian4}
    \definition{v.}{desarmar (um disjuntor ou interruptor)}
  \end{Phonetics}
\end{Entry}

\begin{Entry}{跳伞}{13,6}{⾜、⼈}
  \begin{Phonetics}{跳伞}{tiao4san3}
    \definition{s.}{paraquedas}
    \definition{v.}{saltar de paraquedas}
  \end{Phonetics}
\end{Entry}

\begin{Entry}{跳远}{13,7}{⾜、⾡}
  \begin{Phonetics}{跳远}{tiao4 yuan3}[][HSK 3]
    \definition{s.}{salto em distância (atletismo)}
  \end{Phonetics}
\end{Entry}

\begin{Entry}{跳挡}{13,9}{⾜、⼿}
  \begin{Phonetics}{跳挡}{tiao4dang3}
    \definition{v.}{pular marcha (de um carro) | perder a marcha}
  \end{Phonetics}
\end{Entry}

\begin{Entry}{跳蚤}{13,9}{⾜、⾍}
  \begin{Phonetics}{跳蚤}{tiao4zao5}
    \definition{s.}{pulga}
  \end{Phonetics}
\end{Entry}

\begin{Entry}{跳高}{13,10}{⾜、⾼}
  \begin{Phonetics}{跳高}{tiao4 gao1}[][HSK 3]
    \definition{s.}{salto em altura (atletismo)}
    \definition{v.}{saltar em altura}
  \end{Phonetics}
\end{Entry}

\begin{Entry}{跳绳}{13,11}{⾜、⽷}
  \begin{Phonetics}{跳绳}{tiao4sheng2}
    \definition{v.}{pular corda}
  \end{Phonetics}
\end{Entry}

\begin{Entry}{跳跳糖}{13,13,16}{⾜、⾜、⽶}
  \begin{Phonetics}{跳跳糖}{tiao4tiao4tang2}
    \definition{s.}{\emph{Pop Rocks}, \emph{popping candy}}
  \end{Phonetics}
\end{Entry}

\begin{Entry}{跳频}{13,13}{⾜、⾴}
  \begin{Phonetics}{跳频}{tiao4pin2}
    \definition{s.}{FHSS, \emph{Frequency-Hopping Spread Spectrum}, método de transmissão de sinais de rádio}
  \end{Phonetics}
\end{Entry}

\begin{Entry}{跳舞}{13,14}{⾜、⾇}
  \begin{Phonetics}{跳舞}{tiao4/wu3}[][HSK 3]
    \definition{v.+compl.}{dançar (como performance); executar dança, especialmente dança de salão}
  \end{Phonetics}
\end{Entry}

\begin{Entry}{踏}{15}{⾜}
  \begin{Phonetics}{踏}{ta1}
    \definition{part.}{Caracter formador de palavras}
  \end{Phonetics}
  \begin{Phonetics}{踏}{ta4}[][HSK 6]
    \definition{v.}{por os pés em; pisar em; esmagar com o pé | fazer uma investigação ou levantamento no local}
  \end{Phonetics}
\end{Entry}

\begin{Entry}{踏实}{15,8}{⾜、⼧}
  \begin{Phonetics}{踏实}{ta1shi5}[][HSK 6]
    \definition{adj.}{confiável; sério; estável e seguro; descreve uma atitude séria em relação ao trabalho ou estudo | à vontade; livre de ansiedade; descreve uma mente ou sentimento estável, sem qualquer preocupação ou ansiedade}
  \end{Phonetics}
\end{Entry}

\begin{Entry}{踏板}{15,8}{⾜、⽊}
  \begin{Phonetics}{踏板}{ta4ban3}
    \definition{s.}{pedal (em um carro, em um piano, etc.) |  apoio para os pés | estribo}
  \end{Phonetics}
\end{Entry}

\begin{Entry}{踢}{15}{⾜}
  \begin{Phonetics}{踢}{ti1}[][HSK 6]
    \definition{v.}{chutar | jogar (por exemplo, futebol)}
  \end{Phonetics}
\end{Entry}

\begin{Entry}{踢蹋舞}{15,17,14}{⾜、⾜、⾇}
  \begin{Phonetics}{踢蹋舞}{ti1ta4wu3}
    \definition{s.}{sapateado | passo de dança}
  \end{Phonetics}
\end{Entry}

\begin{Entry}{踢爆}{15,19}{⾜、⽕}
  \begin{Phonetics}{踢爆}{ti1bao4}
    \definition{v.}{expor | revelar}
  \end{Phonetics}
\end{Entry}

\begin{Entry}{踩}{15}{⾜}
  \begin{Phonetics}{踩}{cai3}[][HSK 6]
    \definition{v.}{pisar; pisotear | pisar; metáfora: depreciar ou estragar | rastrear; antigamente significava rastrear (bandidos) ou investigar (casos)}
  \end{Phonetics}
\end{Entry}

\begin{Entry}{踹}{16}{⾜}
  \begin{Phonetics}{踹}{chuai4}[][HSK 7-9]
    \definition{v.}{chutar (com a sola do pé) | pisar; pisotear; pisar em}
  \end{Phonetics}
\end{Entry}

\begin{Entry}{蹦}{18}{⾜}
  \begin{Phonetics}{蹦}{beng4}[][HSK 7-9]
    \definition{v.}{pular; saltar; quicar}
  \end{Phonetics}
\end{Entry}

\begin{Entry}{蹦极}{18,7}{⾜、⽊}
  \begin{Phonetics}{蹦极}{beng4ji2}
    \definition{s.}{\emph{bungee jumping}}
  \end{Phonetics}
\end{Entry}

\begin{Entry}{蹬}{19}{⾜}
  \begin{Phonetics}{蹬}{deng1}[][HSK 7-9]
    \definition{v.}{pressionar com o pé; pisar; pisar em | Dialeto: calçar (sapatos ou calças); usar (sapatos) | Gíria: despejar (algo)}
  \end{Phonetics}
  \begin{Phonetics}{蹬}{deng4}
    \definition{s.}{lutar; ter dificuldade}
  \seealsoref{蹭蹬}{ceng4deng4}
  \end{Phonetics}
\end{Entry}

\begin{Entry}{蹭}{19}{⾜}
  \begin{Phonetics}{蹭}{ceng4}[][HSK 7-9]
    \definition{v.}{esfregar; raspar; arranhar | esfregar em algo e ficar manchado; ser manchado com; manchar por fricção | mover-se lentamente; demorar-se; arrastar-se | Dialeto: roubar}
  \end{Phonetics}
\end{Entry}

\begin{Entry}{蹭蹬}{19,19}{⾜、⾜}
  \begin{Phonetics}{蹭蹬}{ceng4deng4}
    \definition{interj.}{Droga!}
    \definition{v.}{enfrentar contratempos; estar sem sorte; ter má sorte}
  \end{Phonetics}
\end{Entry}

\begin{Entry}{蹲}{19}{⾜}
  \begin{Phonetics}{蹲}{dun1}[][HSK 6]
    \definition{v.}{agachamento sobre os calcanhares; dobrar as pernas o máximo possível, como se estivesse sentado, mas não deixar as nádegas tocarem o chão | ficar; metáfora para ficar ocioso em casa}
  \end{Phonetics}
\end{Entry}

\begin{Entry}{蹲下}{19,3}{⾜、⼀}
  \begin{Phonetics}{蹲下}{dun1xia4}
    \definition{v.}{agachar | agachar-se}
  \end{Phonetics}
\end{Entry}

%%%%% EOF %%%%%

