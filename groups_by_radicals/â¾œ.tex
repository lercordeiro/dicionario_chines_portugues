%%%
%%% Radical "⾜"
%%%

\section*{Radical 157: ``⾜''}\addcontentsline{toc}{section}{Radical 157: ⾜}

\begin{entry}{足}{7}{⾜}[Kangxi 157]
  \begin{phonetics}{足}{ju4}
    \definition{adj.}{excessivo}
  \end{phonetics}
  \begin{phonetics}{足}{zu2}
    \definition{adj.}{amplo}
    \definition{s.}{pé}
    \definition{v.}{ser suficiente}
  \end{phonetics}
\end{entry}

\begin{entry}{足月}{7,4}{⾜、⽉}
  \begin{phonetics}{足月}{zu2yue4}
    \definition{s.}{gestação completa}
  \end{phonetics}
\end{entry}

\begin{entry}{足足}{7,7}{⾜、⾜}
  \begin{phonetics}{足足}{zu2zu2}
    \definition{adv.}{tanto quanto | extremamente | completamente | não menos que}
  \end{phonetics}
\end{entry}

\begin{entry}{足够}{7,11}{⾜、⼣}
  \begin{phonetics}{足够}{zu2 gou4}[][HSK 3]
    \definition{adj.}{bastante; amplo; suficiente; na medida em que deve ser ou pode atender às necessidades}
    \definition{v.}{satisfazer; ser suficiente; estar a contento}
  \end{phonetics}
\end{entry}

\begin{entry}{足球}{7,11}{⾜、⽟}
  \begin{phonetics}{足球}{zu2qiu2}[][HSK 3]
    \definition[个,只,颗,袋]{s.}{futebol | bola de futebol}
  \end{phonetics}
\end{entry}

\begin{entry}{足球队}{7,11,4}{⾜、⽟、⾩}
  \begin{phonetics}{足球队}{zu2qiu2dui4}
    \definition{s.}{time de futebol}
  \end{phonetics}
\end{entry}

\begin{entry}{足球协会}{7,11,6,6}{⾜、⽟、⼗、⼈}
  \begin{phonetics}{足球协会}{zu2qiu2xie2hui4}
    \definition*{s.}{Associação de Futebol}
  \end{phonetics}
\end{entry}

\begin{entry}{足球场}{7,11,6}{⾜、⽟、⼟}
  \begin{phonetics}{足球场}{zu2qiu2chang3}
    \definition{s.}{campo de futebol}
  \end{phonetics}
\end{entry}

\begin{entry}{足球迷}{7,11,9}{⾜、⽟、⾡}
  \begin{phonetics}{足球迷}{zu2qiu2mi2}
    \definition{s.}{fã de futebol}
  \end{phonetics}
\end{entry}

\begin{entry}{足球赛}{7,11,14}{⾜、⽟、⾙}
  \begin{phonetics}{足球赛}{zu2qiu2sai4}
    \definition{s.}{competição de futebol | partida de futebol}
  \end{phonetics}
\end{entry}

\begin{entry}{距离}{11,10}{⾜、⼇}
  \begin{phonetics}{距离}{ju4li2}[][HSK 4]
    \definition[个]{s.}{distância}
    \definition{v.}{estar distante de}
  \end{phonetics}
\end{entry}

\begin{entry}{跑}{12}{⾜}
  \begin{phonetics}{跑}{pao2}
    \definition{v.}{(de um animal) dar patadas (no chão)}
  \end{phonetics}
  \begin{phonetics}{跑}{pao3}[][HSK 1]
    \definition{v.}{vazar ou evaporar (sobre um gás ou líquido) | escapar | correr | correr (em tarefas, etc.) | fugir}
  \end{phonetics}
\end{entry}

\begin{entry}{跑马}{12,3}{⾜、⾺}
  \begin{phonetics}{跑马}{pao3ma3}
    \definition{s.}{corrida de cavalos}
    \definition{v.}{andar a cavalo em ritmo acelerado}
  \end{phonetics}
\end{entry}

\begin{entry}{跑步}{12,7}{⾜、⽌}
  \begin{phonetics}{跑步}{pao3bu4}[][HSK 3]
    \definition{s.}{corrida}
    \definition{v.+compl.}{correr; trotar}
  \end{phonetics}
\end{entry}

\begin{entry}{跑肚}{12,7}{⾜、⾁}
  \begin{phonetics}{跑肚}{pao3du4}
    \definition{v.}{(coloquial) ter diarréia}
  \end{phonetics}
\end{entry}

\begin{entry}{跑调}{12,10}{⾜、⾔}
  \begin{phonetics}{跑调}{pao3diao4}
    \definition{v.}{(coloquial) estar fora do tom ou desafinado (enquanto canta)}
  \end{phonetics}
\end{entry}

\begin{entry}{跑掉}{12,11}{⾜、⼿}
  \begin{phonetics}{跑掉}{pao3diao4}
    \definition{v.}{fugir}
  \end{phonetics}
\end{entry}

\begin{entry}{跑腿}{12,13}{⾜、⾁}
  \begin{phonetics}{跑腿}{pao3tui3}
    \definition{v.}{realizar tarefas}
  \end{phonetics}
\end{entry}

\begin{entry}{跑酷}{12,14}{⾜、⾣}
  \begin{phonetics}{跑酷}{pao3ku4}
    \definition*{s.}{(empréstimo linguístico) \emph{Parkour}}
  \end{phonetics}
\end{entry}

\begin{entry}{跑题}{12,15}{⾜、⾴}
  \begin{phonetics}{跑题}{pao3ti2}
    \definition{v.}{divagar | fugir do assunto | tergiversar}
  \end{phonetics}
\end{entry}

\begin{entry}{跟}{13}{⾜}
  \begin{phonetics}{跟}{gen1}[][HSK 1]
    \definition{conj.}{e; com}
    \definition{prep.}{com}
    \definition{v.}{acompanhar junto | seguir de perto | ir com}
  \end{phonetics}
\end{entry}

\begin{entry}{跪拜}{13,9}{⾜、⼿}
  \begin{phonetics}{跪拜}{gui4bai4}
    \definition{v.}{prostrar-se | ajoelhar-se e adorar}
  \end{phonetics}
\end{entry}

\begin{entry}{路}{13}{⾜}
  \begin{phonetics}{路}{lu4}[][HSK 1]
    \definition*{s.}{sobrenome Lu}
    \definition[条]{s.}{caminho | estrada | via | jornada | linha (ônibus, etc.) | rota}
  \end{phonetics}
\end{entry}

\begin{entry}{路上}{13,3}{⾜、⼀}
  \begin{phonetics}{路上}{lu4shang5}[][HSK 1]
    \definition{adv.}{na estrada | no caminho | a caminho}
  \end{phonetics}
\end{entry}

\begin{entry}{路口}{13,3}{⾜、⼝}
  \begin{phonetics}{路口}{lu4kou3}[][HSK 1]
    \definition{s.}{cruzamento | interseção (de estradas)}
  \end{phonetics}
\end{entry}

\begin{entry}{路边}{13,5}{⾜、⾡}
  \begin{phonetics}{路边}{lu4 bian1}[][HSK 2]
    \definition{s.}{meio-fio | acostamento}
  \end{phonetics}
\end{entry}

\begin{entry}{路线}{13,8}{⾜、⽷}
  \begin{phonetics}{路线}{lu4 xian4}[][HSK 3]
    \definition[条]{s.}{rota; caminho; linha | linha; diretriz (de política, ideologia, campo de trabalho)}
  \end{phonetics}
\end{entry}

\begin{entry}{跳}{13}{⾜}
  \begin{phonetics}{跳}{tiao4}[][HSK 3]
    \definition{v.}{pular; saltar; quicar | mover para cima e para baixo; pulsar; palpitar; contrair-se | pular; saltar por cima}
  \end{phonetics}
\end{entry}

\begin{entry}{跳水}{13,4}{⾜、⽔}
  \begin{phonetics}{跳水}{tiao4shui3}
    \definition{s.}{mergulho esportivo}
    \definition{v.}{mergulhar (na água) | cometer suicídio pulando na água | (figurativo, preços das ações, etc.) cair dramaticamente}
  \end{phonetics}
\end{entry}

\begin{entry}{跳电}{13,5}{⾜、⽥}
  \begin{phonetics}{跳电}{tiao4dian4}
    \definition{v.}{desarmar (um disjuntor ou interruptor)}
  \end{phonetics}
\end{entry}

\begin{entry}{跳伞}{13,6}{⾜、⼈}
  \begin{phonetics}{跳伞}{tiao4san3}
    \definition{s.}{paraquedas}
    \definition{v.}{saltar de paraquedas}
  \end{phonetics}
\end{entry}

\begin{entry}{跳远}{13,7}{⾜、⾡}
  \begin{phonetics}{跳远}{tiao4 yuan3}[][HSK 3]
    \definition{s.}{salto em distância (atletismo)}
  \end{phonetics}
\end{entry}

\begin{entry}{跳挡}{13,9}{⾜、⼿}
  \begin{phonetics}{跳挡}{tiao4dang3}
    \definition{v.}{pular marcha (de um carro) | perder a marcha}
  \end{phonetics}
\end{entry}

\begin{entry}{跳蚤}{13,9}{⾜、⾍}
  \begin{phonetics}{跳蚤}{tiao4zao5}
    \definition{s.}{pulga}
  \end{phonetics}
\end{entry}

\begin{entry}{跳高}{13,10}{⾜、⾼}
  \begin{phonetics}{跳高}{tiao4 gao1}[][HSK 3]
    \definition{s.}{salto em altura (atletismo)}
  \end{phonetics}
\end{entry}

\begin{entry}{跳绳}{13,11}{⾜、⽷}
  \begin{phonetics}{跳绳}{tiao4sheng2}
    \definition{v.}{pular corda}
  \end{phonetics}
\end{entry}

\begin{entry}{跳跳糖}{13,13,16}{⾜、⾜、⽶}
  \begin{phonetics}{跳跳糖}{tiao4tiao4tang2}
    \definition{s.}{\emph{Pop Rocks}, \emph{popping candy}}
  \end{phonetics}
\end{entry}

\begin{entry}{跳频}{13,13}{⾜、⾴}
  \begin{phonetics}{跳频}{tiao4pin2}
    \definition{s.}{FHSS, \emph{Frequency-Hopping Spread Spectrum}, método de transmissão de sinais de rádio}
  \end{phonetics}
\end{entry}

\begin{entry}{跳舞}{13,14}{⾜、⾇}
  \begin{phonetics}{跳舞}{tiao4wu3}[][HSK 3]
    \definition{v.+compl.}{dançar (como performance)}
  \end{phonetics}
\end{entry}

\begin{entry}{踏板}{15,8}{⾜、⽊}
  \begin{phonetics}{踏板}{ta4ban3}
    \definition{s.}{pedal (em um carro, em um piano, etc.) |  apoio para os pés | estribo}
  \end{phonetics}
\end{entry}

\begin{entry}{踢}{15}{⾜}
  \begin{phonetics}{踢}{ti1}
    \definition{v.}{chutar | jogar (por exemplo, futebol) | dar pontapés em}
  \end{phonetics}
\end{entry}

\begin{entry}{踢蹋舞}{15,17,14}{⾜、⾜、⾇}
  \begin{phonetics}{踢蹋舞}{ti1ta4wu3}
    \definition{s.}{sapateado | passo de dança}
  \end{phonetics}
\end{entry}

\begin{entry}{踢爆}{15,19}{⾜、⽕}
  \begin{phonetics}{踢爆}{ti1bao4}
    \definition{v.}{expor | revelar}
  \end{phonetics}
\end{entry}

\begin{entry}{蹦极}{18,7}{⾜、⽊}
  \begin{phonetics}{蹦极}{beng4ji2}
    \definition{s.}{\emph{bungee jumping}}
  \end{phonetics}
\end{entry}

\begin{entry}{蹲下}{19,3}{⾜、⼀}
  \begin{phonetics}{蹲下}{dun1xia4}
    \definition{v.}{agachar | agachar-se}
  \end{phonetics}
\end{entry}

%%%%% EOF %%%%%

