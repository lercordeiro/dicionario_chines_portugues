%%%
%%% Radical "⽼"
%%%

\section*{Radical 125: ``⽼'' (耂)}\addcontentsline{toc}{section}{Radical 125: ⽼、耂}

\begin{Entry}{老}{6}{⽼}[Kangxi 125]
  \begin{Phonetics}{老}{lao3}[][HSK 1,2]
    \definition*{s.}{Sobrenome Lao}
    \definition{adj.}{velho; envelhecido; idade avançada | antigo; de longa data; existe há muito tempo | antigo; desatualizado; obsoleto; ultrapassado  | antigo; tradicional; original | coberto de vegetação; os vegetais cresceram além do período ideal para serem consumidos | resistente; endurecido; alimentos muito cozidos | escuro; profundo; (sobre cores) | último nascido; o mais novo | veterano; experiente; sofisticado}
    \definition{adv.}{longo; por muito tempo | sempre (fazendo algo) | muito}
    \definition{pref.}{usado para designar pessoas, ordem de classificação, certos nomes de animais e plantas}
    \definition{s.}{idosos; pessoas mais velhas | ancião; sênior; um título respeitoso para pessoas mais velhas}
    \definition{v.}{envelhecer | morrer; referindo-se à morte de um idoso}
  \end{Phonetics}
\end{Entry}

\begin{Entry}{老人}{6,2}{⽼、⼈}
  \begin{Phonetics}{老人}{lao3 ren2}[][HSK 1]
    \definition[位]{s.}{homem ou mulher de idade avançada; o idoso; o velho}
  \end{Phonetics}
\end{Entry}

\begin{Entry}{老人家}{6,2,10}{⽼、⼈、⼧}
  \begin{Phonetics}{老人家}{lao3 ren2 jia1}
    \definition[位,名,个]{s.}{avô; avó; pessoa idosa venerável; um título respeitoso para os idosos | maneira de chamar o pai ou a mãe idosos na frente dos outros; referir-se aos próprios pais ou aos pais, professores, etc. de outras pessoas}
  \end{Phonetics}
\end{Entry}

\begin{Entry}{老乡}{6,3}{⽼、⼄}
  \begin{Phonetics}{老乡}{lao3 xiang1}[][HSK 6]
    \definition[个,位]{s.}{conterrâneo; conterrâneo | uma maneira de chamar um fazendeiro cujo nome você não conhece}
  \end{Phonetics}
\end{Entry}

\begin{Entry}{老公}{6,4}{⽼、⼋}
  \begin{Phonetics}{老公}{lao3 gong1}[][HSK 4]
    \definition[个,位,名]{s.}{marido; esposo}
  \end{Phonetics}
\end{Entry}

\begin{Entry}{老太太}{6,4,4}{⽼、⼤、⼤}
  \begin{Phonetics}{老太太}{lao3 tai4 tai5}[][HSK 3]
    \definition[位,名,个]{s.}{velha senhora; (em tratamento direto)Venerável Senhora; uma maneira respeitosa de chamar uma senhora idosa; título honorífico para mulheres idosas | (forma de tratamento) sua velha mãe; minha velha mãe, avó ou sogra; referindo-se à própria mãe, à mãe do outro ou à mãe de outra pessoa, à sogra ou à sogra política}
  \end{Phonetics}
\end{Entry}

\begin{Entry}{老头儿}{6,5,2}{⽼、⼤、⼉}
  \begin{Phonetics}{老头儿}{lao3 tou2r5}[][HSK 3]
    \definition{s.}{(coloquial) (com um tom de intimidade) velho; velho amigo}
  \seealsoref{老头子}{lao3 tou2zi5}
  \end{Phonetics}
\end{Entry}

\begin{Entry}{老头子}{6,5,3}{⽼、⼤、⼦}
  \begin{Phonetics}{老头子}{lao3 tou2zi5}
    \definition{s.}{velho antiquado (ou velho rabugento) | (referindo-se ao marido idoso) meu velho | chefe de uma sociedade secreta | (coloquial) velho; velho rabugento}
  \seealsoref{老头儿}{lao3 tou2r5}
  \end{Phonetics}
\end{Entry}

\begin{Entry}{老师}{6,6}{⽼、⼱}
  \begin{Phonetics}{老师}{lao3shi1}[][HSK 1]
    \definition[个,位]{s.}{professor; título honorífico para professores; refere-se, de maneira geral, a pessoas que transmitem cultura e tecnologia ou que são dignas de admiração em termos de ideias, moralidade e conhecimentos profissionais}
  \end{Phonetics}
\end{Entry}

\begin{Entry}{老年}{6,6}{⽼、⼲}
  \begin{Phonetics}{老年}{lao3 nian2}[][HSK 2]
    \definition[个]{s.}{idoso; velhice; idade acima de 60 ou 70 anos}
  \end{Phonetics}
\end{Entry}

\begin{Entry}{老百姓}{6,6,8}{⽼、⽩、⼥}
  \begin{Phonetics}{老百姓}{lao3bai3xing4}[][HSK 3]
    \definition[些]{s.}{povo; civis; pessoas comuns; residentes (em contraste com militares e funcionários públicos)}
  \end{Phonetics}
\end{Entry}

\begin{Entry}{老兵}{6,7}{⽼、⼋}
  \begin{Phonetics}{老兵}{lao3bing1}
    \definition{s.}{velho soldado | veterano de guerra | veterano (alguém que tem muita experiência em algum domínio)}
  \end{Phonetics}
\end{Entry}

\begin{Entry}{老实}{6,8}{⽼、⼧}
  \begin{Phonetics}{老实}{lao3shi5}[][HSK 4]
    \definition{adj.}{franco; sincero; honesto | bom; bem-comportado | ingênuo; simplório; meio bobo; facilmente enganado; eufemismo para pouco inteligente}
  \end{Phonetics}
\end{Entry}

\begin{Entry}{老朋友}{6,8,4}{⽼、⽉、⼜}
  \begin{Phonetics}{老朋友}{lao3 peng2 you3}[][HSK 2]
    \definition[个,位,名]{s.}{velho amigo; refere-se a amigos que conhecemos há muito tempo e com quem temos uma relação íntima}
  \end{Phonetics}
\end{Entry}

\begin{Entry}{老板}{6,8}{⽼、⽊}
  \begin{Phonetics}{老板}{lao3ban3}[][HSK 3]
    \definition[个,位]{s.}{chefe; dono; líder; refere-se ao gerente de uma empresa comercial ou industrial | antigo título honorífico dado a atores famosos de ópera ou atores que também eram diretores de companhias de ópera}
  \end{Phonetics}
\end{Entry}

\begin{Entry}{老虎}{6,8}{⽼、⾌}
  \begin{Phonetics}{老虎}{lao3hu3}
    \definition[只]{s.}{tigre}
  \seealsoref{虎}{hu3}
  \end{Phonetics}
\end{Entry}

\begin{Entry}{老是}{6,9}{⽼、⽇}
  \begin{Phonetics}{老是}{lao3 shi4}[][HSK 2]
    \definition{adv.}{sempre; indica que a ação continua ou que o estado permanece inalterado, equivalente a 一直}
  \seealsoref{一直}{yi4zhi2}
  \end{Phonetics}
\end{Entry}

\begin{Entry}{老家}{6,10}{⽼、⼧}
  \begin{Phonetics}{老家}{lao3 jia1}[][HSK 4]
    \definition{s.}{cidade natal; local de origem | lugar nativo; refere-se às gerações anteriores da família ou ao local onde a pessoa nasceu ou viveu}
  \end{Phonetics}
\end{Entry}

\begin{Entry}{老婆}{6,11}{⽼、⼥}
  \begin{Phonetics}{老婆}{lao3po2}[][HSK 4]
    \definition[个,位,名]{s.}{esposa}
  \end{Phonetics}
\end{Entry}

\begin{Entry}{考}{6}{⽼}
  \begin{Phonetics}{考}{kao3}[][HSK 1]
    \definition*{s.}{Sobrenome Kao}
    \definition{adj.}{antigo; velho; com idade avançada}
    \definition{s.}{o pai falecido de alguém}
    \definition{v.}{examinar; dar (fazer) um exame, teste ou questionário | verificar; inspecionar | estudar; verificar; investigar | perguntar; testar; fazer perguntas para que o outro responda, a fim de testar suas habilidades em determinada área}
  \end{Phonetics}
\end{Entry}

\begin{Entry}{考生}{6,5}{⽼、⽣}
  \begin{Phonetics}{考生}{kao3 sheng1}[][HSK 2]
    \definition{s.}{candidato a exame; alunos inscritos para o exame de admissão}
  \end{Phonetics}
\end{Entry}

\begin{Entry}{考场}{6,6}{⽼、⼟}
  \begin{Phonetics}{考场}{kao3 chang3}[][HSK 6]
    \definition{s.}{sala de exames}
  \end{Phonetics}
\end{Entry}

\begin{Entry}{考试}{6,8}{⽼、⾔}
  \begin{Phonetics}{考试}{kao3/shi4}[][HSK 1]
    \definition[次]{s.}{teste; exame; prova; atividades realizadas para verificar conhecimentos ou habilidades}
    \definition{v.+compl.}{testar; avaliar; avaliar conhecimentos e habilidades por meio de perguntas escritas ou orais.}
  \end{Phonetics}
\end{Entry}

\begin{Entry}{考核}{6,10}{⽼、⽊}
  \begin{Phonetics}{考核}{kao3he2}[][HSK 5]
    \definition{v.}{examinar; checar; avaliar; avaliar (a proficiência de alguém)}
  \end{Phonetics}
\end{Entry}

\begin{Entry}{考虑}{6,10}{⽼、⾌}
  \begin{Phonetics}{考虑}{kao3lv4}[][HSK 4]
    \definition{v.}{considerar; refletir sobre; levar em conta}
  \end{Phonetics}
\end{Entry}

\begin{Entry}{考验}{6,10}{⽼、⾺}
  \begin{Phonetics}{考验}{kao3yan4}[][HSK 3]
    \definition[场,个,种]{s.}{teste; julgamento; atividade realizada para verificar se as habilidades, ideias, moral e qualidades de uma pessoa atendem aos requisitos}
    \definition{v.}{testar; testar as capacidades, ideias, moral e qualidades de uma pessoa através de situações, ações ou ambientes difíceis, para verificar se elas atendem aos requisitos}
  \end{Phonetics}
\end{Entry}

\begin{Entry}{考察}{6,14}{⽼、⼧}
  \begin{Phonetics}{考察}{kao3cha2}[][HSK 4]
    \definition{v.}{inspecionar; investigar; observar e estudar}
  \end{Phonetics}
\end{Entry}

\begin{Entry}{考题}{6,15}{⽼、⾴}
  \begin{Phonetics}{考题}{kao3 ti2}[][HSK 6]
    \definition{s.}{questões de exame; prova de exame; tópicos de exame}
  \end{Phonetics}
\end{Entry}

\begin{Entry}{者}{8}{⽼}
  \begin{Phonetics}{者}{zhe3}[][HSK 3]
    \definition*{s.}{Sobrenome Zhe}
    \definition{part.}{significa 是 e é usado após palavras, frases e orações para indicar uma pausa}
    \definition{pron.}{usado para se referir à pessoa, coisa ou assunto que realiza uma ação ou possui um determinado atributo | pessoas; caras (Usado para se referir a alguém envolvido em uma determinada profissão, que acredita em uma determinada ideologia ou que tem uma forte tendência para algo) | usado após certos números ou palavras direcionais para se referir a coisas mencionadas anteriormente | significado semelhante a 这 (mais comum na linguagem coloquial antiga)}
  \seealsoref{是}{shi4}
  \seealsoref{这}{zhe4}
  \end{Phonetics}
\end{Entry}

%%%%% EOF %%%%%

