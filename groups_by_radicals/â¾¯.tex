%%%
%%% Radical "⾯"
%%%

\section*{Radical 176: ``⾯'' (靣)}\addcontentsline{toc}{section}{Radical 176: ⾯、靣}

\begin{entry}{面}{9}{⾯}[Kangxi 176]
  \begin{phonetics}{面}{mian4}[][HSK 2]
    \definition*{s.}{Sobrenome Mian}
    \definition{adj.}{macio e farinhento; descreve algo que é muito macio ao comer | superficial}
    \definition{adv.}{diretamente; pessoalmente; na frente de alguém; cara a cara}
    \definition{clas.}{usado para objetos planos | usado para indicar o número de vezes que as pessoas se encontram}
    \definition[斤,两,碗]{s.}{face; parte frontal da cabeça; rosto | topo; superfície | capa; exterior; a parte externa de um objeto ou a face frontal de um tecido (em oposição à 里) | (matemática) superfície | cara; sentimento; emoção | geral; área total; abrangente; toda a região | lado; aspecto | escopo; escala; extensão; alcance; âmbito | farinha; farinha de trigo | pó; algo em pó | macarrão; \emph{noodle}}
    \definition{suf.}{sufixo para localização ou direção; anexado ao final de palavras que indicam localização, equivalente a 边}
    \definition{v.}{encarar algo | encontrar; revelar-se}
  \seealsoref{边}{bian1}
  \seealsoref{里}{li3}
  \end{phonetics}
\end{entry}

\begin{entry}{面子}{9,3}{⾯、⼦}
  \begin{phonetics}{面子}{mian4zi5}[][HSK 5]
    \definition{s.}{face; exterior; parte externa; superfície do objeto | imagem; reputação; prestígio; decência; vaidade superficial | sentimentos; sensibilidades | pó}
  \end{phonetics}
\end{entry}

\begin{entry}{面包}{9,5}{⾯、⼓}
  \begin{phonetics}{面包}{mian4bao1}[][HSK 1]
    \definition[个,片,袋,块]{s.}{pão}[我买八个面包了。___Comprei oito pães. | 他在吃两片面包。___Ele está comendo duas fatias de pão. | 我在家里带了一袋面包。___Trouxe um saco de pão para casa. | 我拿了一块面包。___Peguei um pedaço de pão.]
  \end{phonetics}
\end{entry}

\begin{entry}{面对}{9,5}{⾯、⼨}
  \begin{phonetics}{面对}{mian4dui4}[][HSK 3]
    \definition{v.}{enfrentar; defrontar; olhar para (uma pessoa ou um objeto específico) | confrontar (problema); problemas, dificuldades e outras questões que precisam ser resolvidas e que merecem atenção}
  \end{phonetics}
\end{entry}

\begin{entry}{面对面}{9,5,9}{⾯、⼨、⾯}
  \begin{phonetics}{面对面}{mian4 dui4 mian4}[][HSK 6]
    \definition*{expr.}{frente a frente; cara a cara; vis-à-vis}
  \end{phonetics}
\end{entry}

\begin{entry}{面对面吃面}{9,5,9,6,9}{⾯、⼨、⾯、⼝、⾯}
  \begin{phonetics}{面对面吃面}{mian4dui4mian4 chi1 mian4}
    \definition{expr.}{Comer macarrão cara a cara; indica que o seu estado atual, ou algumas das posições em que você está, ou algumas das coisas que você fez são muito claras}
  \end{phonetics}
\end{entry}

\begin{entry}{面向}{9,6}{⾯、⼝}
  \begin{phonetics}{面向}{mian4 xiang4}[][HSK 6]
    \definition{v.}{virar o rosto para; virar na direção de; defrontar; voltado para algum lugar | estar orientado para as necessidades de; atender a; principalmente para um certo tipo de pessoas}
  \end{phonetics}
\end{entry}

\begin{entry}{面团}{9,6}{⾯、⼞}
  \begin{phonetics}{面团}{mian4tuan2}
    \definition{s.}{massa | pasta}
  \end{phonetics}
\end{entry}

\begin{entry}{面条}{9,7}{⾯、⽊}
  \begin{phonetics}{面条}{mian4tiao2}
    \definition{s.}{macarrão | espaguete}
  \end{phonetics}
\end{entry}

\begin{entry}{面条儿}{9,7,2}{⾯、⽊、⼉}
  \begin{phonetics}{面条儿}{mian4 tiao2r5}[][HSK 1]
    \definition{s.}{macarrão; \emph{noodles}}
  \end{phonetics}
\end{entry}

\begin{entry}{面试}{9,8}{⾯、⾔}
  \begin{phonetics}{面试}{mian4 shi4}[][HSK 4]
    \definition[次]{s.}{entrevista; audição}
  \end{phonetics}
\end{entry}

\begin{entry}{面临}{9,9}{⾯、⼁}
  \begin{phonetics}{面临}{mian4lin2}[][HSK 4]
    \definition{v.}{ser confrontado com; encontrar (uma situação) na frente de}
  \end{phonetics}
\end{entry}

\begin{entry}{面前}{9,9}{⾯、⼑}
  \begin{phonetics}{面前}{mian4 qian2}[][HSK 2]
    \definition{s.}{antes; na frente de; diante de}
  \end{phonetics}
\end{entry}

\begin{entry}{面积}{9,10}{⾯、⽲}
  \begin{phonetics}{面积}{mian4ji1}[][HSK 3]
    \definition{s.}{área (de um andar, pedaço de terreno, etc.); área de uma superfície; o tamanho de uma superfície plana ou da superfície de um objeto}
  \end{phonetics}
\end{entry}

\begin{entry}{面貌}{9,14}{⾯、⾘}
  \begin{phonetics}{面貌}{mian4mao4}[][HSK 5]
    \definition[种,个]{s.}{rosto; traços faciais; formato do rosto; aparência | aparência; aspecto; aparência (das coisas)}
  \end{phonetics}
\end{entry}

%%%%% EOF %%%%%

