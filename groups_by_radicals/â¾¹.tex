%%%
%%% Radical "⾹"
%%%

\section*{Radical 186: ``⾹''}\addcontentsline{toc}{section}{Radical 186: ⾹}

\begin{Entry}{香}{9}{⾹}[Kangxi 186]
  \begin{Phonetics}{香}{xiang1}[][HSK 3]
    \definition*{s.}{Sobrenome Xiang}
    \definition{adj.}{aromático; perfumado; fragrante; cheiroso; oposto a 臭 | saboroso; saboroso; delicioso; apetitoso | com gosto; com bom apetite | (sono) profundo; dormir confortavelmente e tranquilamente | popular; valorizado; apreciado}
    \definition[根,炷]{s.}{especiaria; perfume; fragrância; aromatizante; substância com aroma intenso | incenso; bastão de incenso; tiras finas feitas de serragem e especiarias, queimadas em rituais para honrar os antepassados ou deuses e budas, e também usadas para afastar odores desagradáveis ou mosquitos| antigamente, referia-se a coisas relacionadas com mulheres ou mulheres}
  \seealsoref{臭}{chou4}
  \end{Phonetics}
\end{Entry}

\begin{Entry}{香气}{9,4}{⾹、⽓}
  \begin{Phonetics}{香气}{xiang1qi4}
    \definition{s.}{fragrância | aroma | incenso}
  \end{Phonetics}
\end{Entry}

\begin{Entry}{香皂}{9,7}{⾹、⽩}
  \begin{Phonetics}{香皂}{xiang1zao4}
    \definition{s.}{sabonete | sabonete perfumado}
  \end{Phonetics}
\end{Entry}

\begin{Entry}{香肠}{9,7}{⾹、⾁}
  \begin{Phonetics}{香肠}{xiang1chang2}[][HSK 5]
    \definition[根]{s.}{salsicha; linguiça; alimento feito com intestino de porco, recheado com carne picada e temperos}
  \end{Phonetics}
\end{Entry}

\begin{Entry}{香味}{9,8}{⾹、⼝}
  \begin{Phonetics}{香味}{xiang1wei4}
    \definition[股]{s.}{fragrância | cheiro doce}
  \end{Phonetics}
\end{Entry}

\begin{Entry}{香波}{9,8}{⾹、⽔}
  \begin{Phonetics}{香波}{xiang1bo1}
    \definition{s.}{xampu}
  \end{Phonetics}
\end{Entry}

\begin{Entry}{香炉}{9,8}{⾹、⽕}
  \begin{Phonetics}{香炉}{xiang1lu2}
    \definition{s.}{incensário (para queimar incenso) | queimador de incenso | insensório, turíbulo}
  \end{Phonetics}
\end{Entry}

\begin{Entry}{香烟}{9,10}{⾹、⽕}
  \begin{Phonetics}{香烟}{xiang1yan1}
    \definition[支,条]{s.}{cigarro | fumaça de incenso queimado}
  \end{Phonetics}
\end{Entry}

\begin{Entry}{香艳}{9,10}{⾹、⾊}
  \begin{Phonetics}{香艳}{xiang1yan4}
    \definition{adj.}{atraente | erótico | romântico}
  \end{Phonetics}
\end{Entry}

\begin{Entry}{香港}{9,12}{⾹、⽔}
  \begin{Phonetics}{香港}{xiang1gang3}
    \definition*{s.}{Hong Kong}
  \seealsoref{香港岛}{xiang1gang3 dao3}
  \end{Phonetics}
\end{Entry}

\begin{Entry}{香港岛}{9,12,7}{⾹、⽔、⼭}
  \begin{Phonetics}{香港岛}{xiang1gang3 dao3}
    \definition*{s.}{Ilha de Hong Kong}
  \seealsoref{香港}{xiang1gang3}
  \end{Phonetics}
\end{Entry}

\begin{Entry}{香槟酒}{9,14,10}{⾹、⽊、⾣}
  \begin{Phonetics}{香槟酒}{xiang1bin1jiu3}
    \definition[杯]{s.}{(empréstimo linguístico) \emph{champagne}}
  \end{Phonetics}
\end{Entry}

\begin{Entry}{香蕈}{9,15}{⾹、⾋}
  \begin{Phonetics}{香蕈}{xiang1xun4}
    \definition{s.}{\emph{shiitake}, cogumelo comestível}
  \end{Phonetics}
\end{Entry}

\begin{Entry}{香蕉}{9,15}{⾹、⾋}
  \begin{Phonetics}{香蕉}{xiang1jiao1}[][HSK 3]
    \definition[枝,根,个,把,串,束,弓]{s.}{banana}
  \end{Phonetics}
\end{Entry}

%%%%% EOF %%%%%

