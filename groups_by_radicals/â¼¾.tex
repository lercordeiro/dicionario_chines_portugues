%%%
%%% Radical "⼾"
%%%

\section*{Radical 63: ``⼾'' (户、戸)}\addcontentsline{toc}{section}{Radical 63: ⼾、户、戸}

\begin{entry}{户}{4}{⼾}[Kangxi 63]
  \begin{phonetics}{户}{hu4}[][HSK 4]
    \definition*{s.}{sobrenome Hu}
    \definition[个]{s.}{porta com um painel; porta | domicílio; residência; família | status familiar | conta (banco)}
  \end{phonetics}
\end{entry}

\begin{entry}{房子}{8,3}{⼾、⼦}
  \begin{phonetics}{房子}{fang2 zi5}[][HSK 1]
    \definition[栋,幢,座,套,间,个]{s.}{casa; edifício; prédio}
  \end{phonetics}
\end{entry}

\begin{entry}{房东}{8,5}{⼾、⼀}
  \begin{phonetics}{房东}{fang2dong1}[][HSK 3]
    \definition[个,位]{s.}{dono;  proprietário; senhorio}
  \end{phonetics}
\end{entry}

\begin{entry}{房主}{8,5}{⼾、⼂}
  \begin{phonetics}{房主}{fang2zhu3}
    \definition{s.}{proprietário | dono de um imóvel}
  \end{phonetics}
\end{entry}

\begin{entry}{房间}{8,7}{⼾、⾨}
  \begin{phonetics}{房间}{fang2jian1}[][HSK 1]
    \definition[个,间,套]{s.}{sala; câmara; escritório; apartamento; divisões internas da casa}
  \end{phonetics}
\end{entry}

\begin{entry}{房屋}{8,9}{⼾、⼫}
  \begin{phonetics}{房屋}{fang2 wu1}[][HSK 3]
    \definition[间,所,套]{s.}{casas; habitação; edifícios}
  \end{phonetics}
\end{entry}

\begin{entry}{房租}{8,10}{⼾、⽲}
  \begin{phonetics}{房租}{fang2 zu1}[][HSK 3]
    \definition[笔]{s.}{aluguel}
  \end{phonetics}
\end{entry}

\begin{entry}{所}{8}{⼾}
  \begin{phonetics}{所}{suo3}[][HSK 3]
    \definition*{s.}{sobrenome Suo}
    \definition{clas.}{para casas, etc.}
    \definition{part.}{usado com 为 ou 被 para indicar voz passiva | usado com verbos, representa a entidade que recebe a ação | usado em conjunto com verbos, seguido de um substantivo que recebe a ação | usado com verbos, 者 ou 的 depois do verbo representa a entidade que recebe a ação}
    \definition{s.}{lugar | usado como nome de órgãos governamentais ou outros locais de trabalho}
  \end{phonetics}
\end{entry}

\begin{entry}{所以}{8,4}{⼾、⼈}
  \begin{phonetics}{所以}{suo3 yi3}[][HSK 2]
    \definition{adv.}{portanto | então | como resultado}
    \definition{conj.}{por isso | como resultado | a razão porque}
  \end{phonetics}
\end{entry}

\begin{entry}{所长}{8,4}{⼾、⾧}
  \begin{phonetics}{所长}{suo3 chang2}
    \definition{s.}{aquilo em que alguém é bom; o ponto forte de alguém; o forte de alguém}
  \end{phonetics}
  \begin{phonetics}{所长}{suo3 zhang3}[][HSK 3]
    \definition{s.}{chefe de um instituto, etc. | superintendente}
  \end{phonetics}
\end{entry}

\begin{entry}{所在}{8,6}{⼾、⼟}
  \begin{phonetics}{所在}{suo3 zai4}[][HSK 5]
    \definition[个]{s.}{lugar; local; localização | o lugar onde alguém ou algo está}
  \end{phonetics}
\end{entry}

\begin{entry}{所有}{8,6}{⼾、⽉}
  \begin{phonetics}{所有}{suo3you3}[][HSK 2]
    \definition{adj.}{todo | tudo}
    \definition{s.}{posses}
    \definition{v.}{possuir | ser dono de}
  \end{phonetics}
\end{entry}

\begin{entry}{扇}{10}{⼾}
  \begin{phonetics}{扇}{shan1}[][HSK 5]
    \definition{s.}{ventilar; agitar um leque para fazer o ar circular | dar um tapa; bater com a palma da mão | bater asas; esvoaçar | incitar; instigar; estimular; agitar}
  \end{phonetics}
  \begin{phonetics}{扇}{shan4}[][HSK 5]
    \definition{clas.}{para portas, janelas, etc.}
    \definition[把]{s.}{leque | folha; algo em forma de placa ou folha}
  \end{phonetics}
\end{entry}

\begin{entry}{扇子}{10,3}{⼾、⼦}
  \begin{phonetics}{扇子}{shan4zi5}[][HSK 5]
    \definition[把]{s.}{leque; abano; abanador; utensílios que produzem vento ao serem agitados}
  \end{phonetics}
\end{entry}

%%%%% EOF %%%%%

