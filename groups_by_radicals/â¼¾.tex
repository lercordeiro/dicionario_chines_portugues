%%%
%%% Radical "⼾"
%%%

\section*{Radical 63: ``⼾'' (户、戸)}\addcontentsline{toc}{section}{Radical 63: ⼾、户、戸}

\begin{Entry}{户}{4}{⼾}[Kangxi 63]
  \begin{Phonetics}{户}{hu4}[][HSK 4]
    \definition*{s.}{Sobrenome Hu}
    \definition[个]{s.}{porta com um painel; porta | domicílio; residência; família | status familiar | conta (banco)}
  \end{Phonetics}
\end{Entry}

\begin{Entry}{户外}{4,5}{⼾、⼣}
  \begin{Phonetics}{户外}{hu4 wai4}[][HSK 6]
    \definition{s.}{ao ar livre; espaço aberto ao ar livre}
  \end{Phonetics}
\end{Entry}

\begin{Entry}{房}{8}{⼾}
  \begin{Phonetics}{房}{fang2}
    \definition*{s.}{Fang, a quarta das vinte e oito constelações nas quais a esfera celeste foi dividida, consistindo de quatro estrelas quase em linha reta em Escorpião | Sobrenome Fang}
    \definition[幢,个,间]{s.}{casa; edifício | sala; quarto; câmara | estrutura semelhante a uma casa | um ramo de uma família extensa | loja; estoque | local de trabalho do artesão; oficina; moinho}
  \end{Phonetics}
\end{Entry}

\begin{Entry}{房子}{8,3}{⼾、⼦}
  \begin{Phonetics}{房子}{fang2 zi5}[][HSK 1]
    \definition[栋,幢,座,套,间]{s.}{casa; edifício; prédio}
  \end{Phonetics}
\end{Entry}

\begin{Entry}{房东}{8,5}{⼾、⼀}
  \begin{Phonetics}{房东}{fang2dong1}[][HSK 3]
    \definition[个,位,名]{s.}{dono;  proprietário; senhorio; pessoas que alugam ou emprestam imóveis (para os 房客 )}
  \seealsoref{房客}{fang2ke4}
  \end{Phonetics}
\end{Entry}

\begin{Entry}{房主}{8,5}{⼾、⼂}
  \begin{Phonetics}{房主}{fang2zhu3}
    \definition{s.}{proprietário | dono de um imóvel}
  \end{Phonetics}
\end{Entry}

\begin{Entry}{房价}{8,6}{⼾、⼈}
  \begin{Phonetics}{房价}{fang2 jia4}[][HSK 6]
    \definition{s.}{custo de moradia; tarifa de quarto | preço da casa}
  \end{Phonetics}
\end{Entry}

\begin{Entry}{房间}{8,7}{⼾、⾨}
  \begin{Phonetics}{房间}{fang2jian1}[][HSK 1]
    \definition[个,间,套]{s.}{sala; câmara; escritório; apartamento; divisões internas da casa}
  \end{Phonetics}
\end{Entry}

\begin{Entry}{房客}{8,9}{⼾、⼧}
  \begin{Phonetics}{房客}{fang2ke4}[][HSK 3]
    \definition{s.}{inquilino (de um quarto ou casa); hóspede (oposto a 房东) | inquilino; hóspede; pessoas que alugam ou emprestam imóveis para moradia (para o 房东)}
  \seealsoref{房东}{fang2dong1}
  \end{Phonetics}
\end{Entry}

\begin{Entry}{房屋}{8,9}{⼾、⼫}
  \begin{Phonetics}{房屋}{fang2 wu1}[][HSK 3]
    \definition[间,所,套]{s.}{casas; habitação; edifícios}
  \end{Phonetics}
\end{Entry}

\begin{Entry}{房租}{8,10}{⼾、⽲}
  \begin{Phonetics}{房租}{fang2 zu1}[][HSK 3]
    \definition[笔]{s.}{aluguel}
  \end{Phonetics}
\end{Entry}

\begin{Entry}{所}{8}{⼾}
  \begin{Phonetics}{所}{suo3}[][HSK 3,6]
    \definition*{s.}{Sobrenome Suo}
    \definition{clas.}{usado para casas, etc.}
    \definition{part.}{usado com 为 ou 被 para indicar voz passiva | usado antes do verbo para formar um substantivo ou para qualificar um substantivo | usado antes do verbo na estrutura sujeito-predicado usada como complemento, indica que o termo central é o objeto}
    \definition{s.}{lugar | usado como nome de órgãos governamentais ou outros locais de trabalho}
  \seealsoref{被}{bei4}
  \seealsoref{为}{wei4}
  \end{Phonetics}
\end{Entry}

\begin{Entry}{所以}{8,4}{⼾、⼈}
  \begin{Phonetics}{所以}{suo3 yi3}[][HSK 2]
    \definition{conj.}{assim; portanto; como resultado; conecta frases, expressa resultados e costuma corresponder a expressões como 因为 e 由于}
    \definition[个]{s.}{motivo real; causa real; comportamento adequado}
  \seealsoref{因为}{yin1wei4}
  \seealsoref{由于}{you2yu2}
  \end{Phonetics}
\end{Entry}

\begin{Entry}{所长}{8,4}{⼾、⾧}
  \begin{Phonetics}{所长}{suo3 chang2}
    \definition{s.}{aquilo em que alguém é bom; o ponto forte de alguém; o forte de alguém}
  \end{Phonetics}
  \begin{Phonetics}{所长}{suo3 zhang3}[][HSK 3]
    \definition{s.}{chefe de um instituto, etc. | superintendente}
  \end{Phonetics}
\end{Entry}

\begin{Entry}{所在}{8,6}{⼾、⼟}
  \begin{Phonetics}{所在}{suo3 zai4}[][HSK 5]
    \definition[个]{s.}{lugar; local; localização | o lugar onde alguém ou algo está}
  \end{Phonetics}
\end{Entry}

\begin{Entry}{所有}{8,6}{⼾、⽉}
  \begin{Phonetics}{所有}{suo3you3}[][HSK 2]
    \definition{adj.}{todo | tudo}
    \definition{adj.}{tudo}
    \definition{s.}{bens; posses;}
    \definition{v.}{possuir; ter}
  \end{Phonetics}
\end{Entry}

\begin{Entry}{扁}{9}{⼾}
  \begin{Phonetics}{扁}{bian3}[][HSK 6]
    \definition{adj.}{plano}
    \definition{v.}{(coloquial)  bater em alguém}
  \end{Phonetics}
  \begin{Phonetics}{扁}{pian1}
    \definition{adj.}{pequeno | fora do caminho; remoto}
  \end{Phonetics}
\end{Entry}

\begin{Entry}{扁舟}{9,6}{⼾、⾈}
  \begin{Phonetics}{扁舟}{pian1 zhou1}
    \definition[叶,艘]{s.}{pequeno barco; esquife}
  \end{Phonetics}
\end{Entry}

\begin{Entry}{扇}{10}{⼾}
  \begin{Phonetics}{扇}{shan1}[][HSK 5]
    \definition{s.}{ventilar; agitar um leque para fazer o ar circular | dar um tapa; bater com a palma da mão | bater asas; esvoaçar | incitar; instigar; estimular; agitar}
  \end{Phonetics}
  \begin{Phonetics}{扇}{shan4}[][HSK 5]
    \definition{clas.}{usado para portas, janelas, etc.}
    \definition[把]{s.}{leque | folha; algo em forma de placa ou folha}
  \end{Phonetics}
\end{Entry}

\begin{Entry}{扇子}{10,3}{⼾、⼦}
  \begin{Phonetics}{扇子}{shan4zi5}[][HSK 5]
    \definition[把,个]{s.}{leque; abano; abanador; utensílios que produzem vento ao serem agitados}
  \end{Phonetics}
\end{Entry}

%%%%% EOF %%%%%

