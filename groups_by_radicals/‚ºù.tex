%%%
%%% Radical "⼝"
%%%

\section*{Radical 30: ``⼝''}\addcontentsline{toc}{section}{Radical 30: ⼝}

\begin{entry}{口}{3}{⼝}[Kangxi 30]
  \begin{phonetics}{口}{kou3}[][HSK 1]
    \definition{clas.}{para coisas com bocas (pessoas, animais domésticos, canhões, etc.) | para mordidas ou bocados}
    \definition{s.}{boca}
  \end{phonetics}
\end{entry}

\begin{entry}{口号}{3,5}{⼝、⼝}
  \begin{phonetics}{口号}{kou3 hao4}[][HSK 5]
    \definition[个]{s.}{\emph{slogan}; palavra de ordem; lema}
  \end{phonetics}
\end{entry}

\begin{entry}{口语}{3,9}{⼝、⾔}
  \begin{phonetics}{口语}{kou3 yu3}[][HSK 4]
    \definition[门]{s.}{linguagem oral; linguagem falada; linguagem coloquial; linguagem usada em conversas}
  \end{phonetics}
\end{entry}

\begin{entry}{口音}{3,9}{⼝、⾳}
  \begin{phonetics}{口音}{kou3yin1}
    \definition{s.}{sons da fala oral (linguística)}
  \end{phonetics}
  \begin{phonetics}{口音}{kou3yin5}
    \definition{s.}{sotaque | voz}
  \end{phonetics}
\end{entry}

\begin{entry}{口香糖}{3,9,16}{⼝、⾹、⽶}
  \begin{phonetics}{口香糖}{kou3xiang1tang2}
    \definition{s.}{goma de mascar | chiclete}
  \end{phonetics}
\end{entry}

\begin{entry}{口袋}{3,11}{⼝、⾐}
  \begin{phonetics}{口袋}{kou3dai4}[][HSK 4]
    \definition[个]{s.}{bolso | saco; sacola; artigos de tecido ou couro}
  \end{phonetics}
\end{entry}

\begin{entry}{口袋妖怪}{3,11,7,8}{⼝、⾐、⼥、⼼}
  \begin{phonetics}{口袋妖怪}{kou3dai4 yao1guai4}
    \definition*{s.}{\emph{Pokémon}}
  \end{phonetics}
\end{entry}

\begin{entry}{古}{5}{⼝}
  \begin{phonetics}{古}{gu3}[][HSK 3]
    \definition*{s.}{sobrenome Gu}
    \definition{adj.}{arcaico; antigo; antiquíssimo}
    \definition{pref.}{``paleo''; ``arqueo''}
    \definition{s.}{antiguidade; arcaísmo | livros ou ortodoxias de antigos sábios | uma forma de poesia pré-Tang}
  \end{phonetics}
\end{entry}

\begin{entry}{古人}{5,2}{⼝、⼈}
  \begin{phonetics}{古人}{gu3ren2}
    \definition{s.}{pessoas dos tempos antigos | os antigos | espécies humanas extintas, como \emph{Homo erectus} ou \emph{Homo neanderthalensis} | (literário) pessoa falecida}
  \end{phonetics}
\end{entry}

\begin{entry}{古代}{5,5}{⼝、⼈}
  \begin{phonetics}{古代}{gu3dai4}[][HSK 3]
    \definition{s.}{tempos antigos | sociedade antiga; sociedade primitiva | antigamente}
  \end{phonetics}
\end{entry}

\begin{entry}{古老}{5,6}{⼝、⽼}
  \begin{phonetics}{古老}{gu3 lao3}[][HSK 5]
    \definition{adj.}{antigo; antiquado; histórico}
  \end{phonetics}
\end{entry}

\begin{entry}{古城}{5,9}{⼝、⼟}
  \begin{phonetics}{古城}{gu3cheng2}
    \definition{s.}{cidade antiga}
  \end{phonetics}
\end{entry}

\begin{entry}{古铜色}{5,11,6}{⼝、⾦、⾊}
  \begin{phonetics}{古铜色}{gu3tong2 se4}
    \definition{s.}{cor bronze}
  \end{phonetics}
\end{entry}

\begin{entry}{句}{5}{⼝}
  \begin{phonetics}{句}{gou4}
    \variantof{勾}
  \end{phonetics}
  \begin{phonetics}{句}{ju4}[][HSK 2]
    \definition{clas.}{para orações, frases ou linhas de versos}
    \definition{s.}{sentença | cláusula | frase}
  \end{phonetics}
\end{entry}

\begin{entry}{句子}{5,3}{⼝、⼦}
  \begin{phonetics}{句子}{ju4zi5}[][HSK 2]
    \definition[个]{s.}{sentença | frase | oração}
  \end{phonetics}
\end{entry}

\begin{entry}{另一方面}{5,1,4,9}{⼝、⼀、⽅、⾯}
  \begin{phonetics}{另一方面}{ling4 yi4 fang1 mian4}[][HSK 3]
    \definition{adv./conj.}{outro aspecto | por outro lado; por sua vez; em contrapartida}
  \end{phonetics}
\end{entry}

\begin{entry}{另外}{5,5}{⼝、⼣}
  \begin{phonetics}{另外}{ling4wai4}[][HSK 3]
    \definition{adv.}{além disso; em adição; ademais; além do mais; além de que}
    \definition{pron.}{além disso}
  \end{phonetics}
\end{entry}

\begin{entry}{只}{5}{⼝}
  \begin{phonetics}{只}{zhi1}[][HSK 3]
    \definition*{s.}{sobrenome Zhi}
    \definition{adj.}{solteiro; solitário}
    \definition{clas.}{para um de um par | para animais pequenos (pássaros, gatos, cães, etc.) | para certos utensílios, aparelhos | para navios}
  \end{phonetics}
  \begin{phonetics}{只}{zhi3}[][HSK 2]
    \definition{adv.}{só; somente; apenas; simplesmente; meramente}
  \end{phonetics}
\end{entry}

\begin{entry}{只好}{5,6}{⼝、⼥}
  \begin{phonetics}{只好}{zhi3hao3}[][HSK 3]
    \definition{v.}{ter que; ser forçado a; não ter escolha a não ser}
  \end{phonetics}
\end{entry}

\begin{entry}{只有}{5,6}{⼝、⽉}
  \begin{phonetics}{只有}{zhi3 you3}[][HSK 3]
    \definition{adv.}{somente; tem que; forçado a}
    \definition{conj.}{somente se; conecta cláusulas, expressa condições necessárias, geralmente corresponde a ``才'' (apenas) e ``方'' (que significa apenas)}
  \seealsoref{才}{cai2}
  \seealsoref{方}{fang1}
  \end{phonetics}
\end{entry}

\begin{entry}{只有……才……}{5,6,3}{⼝、⽉、⼿}
  \begin{phonetics}{只有……才……}{zhi3you3 cai2}
    \definition{conj.}{só se\dots então\dots}
  \end{phonetics}
\end{entry}

\begin{entry}{只身}{5,7}{⼝、⾝}
  \begin{phonetics}{只身}{zhi1shen1}
    \definition{adv.}{sozinho | por si só}
  \end{phonetics}
\end{entry}

\begin{entry}{只怕}{5,8}{⼝、⼼}
  \begin{phonetics}{只怕}{zhi3pa4}
    \definition{adv.}{receio que\dots | talvez | muito provavelmente}
  \end{phonetics}
\end{entry}

\begin{entry}{只是}{5,9}{⼝、⽇}
  \begin{phonetics}{只是}{zhi3 shi4}[][HSK 3]
    \definition{adv.}{somente; meramente; apenas; mas; para enfatizar que é limitado a uma determinada situação ou escopo}
    \definition{conj.}{somente; mas; exceto que; orações de conexão, indicando uma ligeira transição, equivalente a ``不过''}
  \seealsoref{不过}{bu2guo4}
  \end{phonetics}
\end{entry}

\begin{entry}{只要}{5,9}{⼝、⾑}
  \begin{phonetics}{只要}{zhi3yao4}[][HSK 2]
    \definition{conj.}{se apenas | contanto que}
  \end{phonetics}
\end{entry}

\begin{entry}{只要……就……}{5,9,12}{⼝、⾑、⼪}
  \begin{phonetics}{只要……就……}{zhi3yao4 jiu4}
    \definition{conj.}{contanto que/desde que/se somente\dots, então\dots}
  \end{phonetics}
\end{entry}

\begin{entry}{只消}{5,10}{⼝、⽔}
  \begin{phonetics}{只消}{zhi3xiao1}
    \definition{conj.}{desde que}
  \end{phonetics}
\end{entry}

\begin{entry}{只能}{5,10}{⼝、⾁}
  \begin{phonetics}{只能}{zhi3 neng2}[][HSK 2]
    \definition{adv.}{só pode | obrigado a fazer algo}
  \end{phonetics}
\end{entry}

\begin{entry}{只读}{5,10}{⼝、⾔}
  \begin{phonetics}{只读}{zhi3du2}
    \definition{s.}{somente leitura (computação) | \emph{read-only}}
  \end{phonetics}
\end{entry}

\begin{entry}{只顾}{5,10}{⼝、⾴}
  \begin{phonetics}{只顾}{zhi3gu4}
    \definition{adv.}{exclusivamente preocupado (com uma coisa)}
    \definition{v.}{cuidar de apenas um aspecto}
  \end{phonetics}
\end{entry}

\begin{entry}{只得}{5,11}{⼝、⼻}
  \begin{phonetics}{只得}{zhi3de5}
    \definition{v.}{ser obrigado a | não ter outra alternativa senão}
  \end{phonetics}
\end{entry}

\begin{entry}{叫}{5}{⼝}
  \begin{phonetics}{叫}{jiao4}[][HSK 1,3]
    \definition{adj.}{macho (animal)}
    \definition{prep.}{apresenta a voz ativa na construção passiva.}
    \definition{v.}{chorar; gritar | chamar; cumprimentar | contratar; encomendar | nomear; chamar | pedir; licitar | anunciar-se}
  \end{phonetics}
\end{entry}

\begin{entry}{叫作}{5,7}{⼝、⼈}
  \begin{phonetics}{叫作}{jiao4 zuo4}[][HSK 2]
    \definition{v.}{ser chamado de | ser conhecido como}
  \end{phonetics}
\end{entry}

\begin{entry}{召开}{5,4}{⼝、⼶}
  \begin{phonetics}{召开}{zhao4kai1}[][HSK 4]
    \definition{v.}{convocar; chamar pessoas para uma reunião; realizar (uma reunião)}
  \end{phonetics}
\end{entry}

\begin{entry}{叮嘱}{5,15}{⼝、⼝}
  \begin{phonetics}{叮嘱}{ding1zhu3}
    \definition{v.}{exortar | avisar | insistir de novo e de novo}
  \end{phonetics}
\end{entry}

\begin{entry}{可}{5}{⼝}
  \begin{phonetics}{可}{ke3}[][HSK 5]
    \definition*{s.}{sobrenome Ke}
    \definition{adv.}{indica ênfase |
indica o fortalecimento de perguntas retóricas |
indica um tom de questionamento mais forte |
sobre; a respeito de;}
    \definition{conj.}{mas; ainda}
    \definition{v.}{aprovar; concordar com | poder; permitir; ser capaz de | precisar (fazer); valer a pena (fazer); merecer | ajustar; adequar | estar pronto para; estar disposto a; pretender}
  \end{phonetics}
  \begin{phonetics}{可}{ke4}
    \definition{s.}{governante supremo de uma tribo nômade do norte; Khan (可汗), título do governante supremo dos antigos grupos étnicos xianbei, turco, uigur e mongol}
  \seealsoref{可汗}{ke4han2}
  \end{phonetics}
\end{entry}

\begin{entry}{可口可乐}{5,3,5,5}{⼝、⼝、⼝、⼃}
  \begin{phonetics}{可口可乐}{ke3kou3ke3le4}
    \definition*{s.}{(empréstimo linguístico) Coca-Cola}
  \end{phonetics}
\end{entry}

\begin{entry}{可以}{5,4}{⼝、⼈}
  \begin{phonetics}{可以}{ke3yi3}[][HSK 2]
    \definition{v.}{ser capaz de | poder}
  \end{phonetics}
\end{entry}

\begin{entry}{可见}{5,4}{⼝、⾒}
  \begin{phonetics}{可见}{ke3jian4}[][HSK 4]
    \definition{adj.}{visível; concebível; algo que é óbvio ou evidente}
    \definition{conj.}{isso mostra; isto prova; é, portanto, claro (ou evidente, óbvio) que}
    \definition{v.}{ser ou estar visível ; ser ou estar claro}
  \end{phonetics}
\end{entry}

\begin{entry}{可乐}{5,5}{⼝、⼃}
  \begin{phonetics}{可乐}{ke3 le4}[][HSK 3]
    \definition*{s.}{\emph{coke}; coca; coca-cola}
  \end{phonetics}
\end{entry}

\begin{entry}{可卡因}{5,5,6}{⼝、⼘、⼞}
  \begin{phonetics}{可卡因}{ke3ka3yin1}
    \definition{s.}{(empréstimo linguístico) cocaína}
  \end{phonetics}
\end{entry}

\begin{entry}{可汗}{5,6}{⼝、⽔}
  \begin{phonetics}{可汗}{ke4han2}
    \definition{s.}{khan (empréstimo linguístico); cham}
  \end{phonetics}
\end{entry}

\begin{entry}{可怕}{5,8}{⼝、⼼}
  \begin{phonetics}{可怕}{ke3pa4}[][HSK 2]
    \definition{adj.}{horrível | terrível | formidável | assustador | hediondo}
    \definition{adv.}{terrivelmente}
  \end{phonetics}
\end{entry}

\begin{entry}{可怜}{5,8}{⼝、⼼}
  \begin{phonetics}{可怜}{ke3lian2}[][HSK 5]
    \definition{adj.}{pobre; lamentável; lastimável | miserável (de quantidade ou qualidade); descreve um número pequeno ou um lugar tão pequeno que não vale a pena falar sobre ele}
    \definition{v.}{ter pena; ter piedade de; ter simpatia por pessoas que tiveram coisas muito ruins acontecendo com elas}
  \end{phonetics}
\end{entry}

\begin{entry}{可是}{5,9}{⼝、⽇}
  \begin{phonetics}{可是}{ke3shi4}[][HSK 2]
    \definition{adv.}{(usado para dar ênfase) de fato}
    \definition{conj.}{porém | contudo | mas}
  \end{phonetics}
\end{entry}

\begin{entry}{可爱}{5,10}{⼝、⽖}
  \begin{phonetics}{可爱}{ke3'ai4}[][HSK 2]
    \definition{adj.}{adorável | querido | fofo}
  \end{phonetics}
\end{entry}

\begin{entry}{可能}{5,10}{⼝、⾁}
  \begin{phonetics}{可能}{ke3neng2}[][HSK 2]
    \definition{adj.}{possível | provável}
    \definition{adv.}{possivelmente | provavelmente}
    \definition[个]{s.}{possibilidade | probabilidade}
  \end{phonetics}
\end{entry}

\begin{entry}{可惜}{5,11}{⼝、⼼}
  \begin{phonetics}{可惜}{ke3xi1}[][HSK 5]
    \definition{adj.}{é uma pena; é muito ruim; é lamentável}
    \definition{adv.}{infelizmente}
  \end{phonetics}
\end{entry}

\begin{entry}{可编程}{5,12,12}{⼝、⽷、⽲}
  \begin{phonetics}{可编程}{ke3bian1cheng2}
    \definition{adj.}{programável}
  \end{phonetics}
\end{entry}

\begin{entry}{可靠}{5,15}{⼝、⾮}
  \begin{phonetics}{可靠}{ke3kao4}[][HSK 3]
    \definition{adj.}{confiável | verdadeiro; autêntico}
  \end{phonetics}
\end{entry}

\begin{entry*}{可擦写可编程只读存储器}{5,17,5,5,12,12,5,10,6,12,16}{⼝、⼿、⼍、⼝、⽷、⽲、⼝、⾔、⼦、⼈、⼝}
  \begin{phonetics}{可擦写可编程只读存储器}{ke3ca1xie3ke3bian1cheng2zhi1du2cun2chu3qi4}
    \definition{s.}{EPROM (\emph{erasable programmable read-only memory})}
  \end{phonetics}
\end{entry*}

\begin{entry}{台}{5}{⼝}
  \begin{phonetics}{台}{tai2}[][HSK 3]
    \definition*{s.}{sobrenome Tai}
    \definition{clas.}{para aparelhos e máquinas}
    \definition{s.}{torre | plataforma; palco | suporte; pedestal | qualquer coisa em forma de plataforma ou palco | mesa; escrivaninha | estação de transmissão | um serviço telefônico especial |termo de tratamento respeitoso (nos tempos antigos)}
  \end{phonetics}
\end{entry}

\begin{entry}{台上}{5,3}{⼝、⼀}
  \begin{phonetics}{台上}{tai2 shang4}[][HSK 4]
    \definition{s.}{no palco}
  \end{phonetics}
\end{entry}

\begin{entry}{台下}{5,3}{⼝、⼀}
  \begin{phonetics}{台下}{tai2xia4}
    \definition{s.}{platéia | fora do palco}
  \end{phonetics}
\end{entry}

\begin{entry}{台风}{5,4}{⼝、⾵}
  \begin{phonetics}{台风}{tai2feng1}[][HSK 5]
    \definition[场,阵,级]{s.}{tufão; classificação de um ciclone tropical ocorrido no oeste do Pacífico Norte | postura; presença de palco; comportamento ou estilo que os atores demonstram no palco}
  \end{phonetics}
\end{entry}

\begin{entry}{台阶}{5,6}{⼝、⾩}
  \begin{phonetics}{台阶}{tai2jie1}[][HSK 4]
    \definition[个,级]{s.}{escada; escadaria | passos; metáfora para uma maneira ou oportunidade de evitar constrangimentos causados ​​por um impasse | nova fase; novo nível; novo patamar; metáfora para novas conquistas ou novos patamares alcançados no estudo ou no trabalho}
  \end{phonetics}
\end{entry}

\begin{entry}{右}{5}{⼝}
  \begin{phonetics}{右}{you4}[][HSK 1]
    \definition{s.}{(política) a Direita}
    \definition{s.}{direita}
  \end{phonetics}
\end{entry}

\begin{entry}{右手}{5,4}{⼝、⼿}
  \begin{phonetics}{右手}{you4shou3}
    \definition{s.}{mão direita | lado direito}
  \end{phonetics}
\end{entry}

\begin{entry}{右边}{5,5}{⼝、⾡}
  \begin{phonetics}{右边}{you4bian5}[][HSK 1]
    \definition{adv.}{à direita | ao lado direito}
  \end{phonetics}
\end{entry}

\begin{entry}{右侧}{5,8}{⼝、⼈}
  \begin{phonetics}{右侧}{you4ce4}
    \definition{s.}{lateral direita | lado direito}
  \end{phonetics}
\end{entry}

\begin{entry}{右转}{5,8}{⼝、⾞}
  \begin{phonetics}{右转}{you4zhuan3}
    \definition{v.}{virar à direita}
  \end{phonetics}
\end{entry}

\begin{entry}{右面}{5,9}{⼝、⾯}
  \begin{phonetics}{右面}{you4mian4}
    \definition{s.}{lado direito}
  \end{phonetics}
\end{entry}

\begin{entry}{右倾}{5,10}{⼝、⼈}
  \begin{phonetics}{右倾}{you4qing1}
    \definition{adj.}{conservador | reacionário}
  \end{phonetics}
\end{entry}

\begin{entry}{右袒}{5,10}{⼝、⾐}
  \begin{phonetics}{右袒}{you4tan3}
    \definition{v.}{ser tendencioso | ser parcial | favorecer um lado | tomar partido}
  \end{phonetics}
\end{entry}

\begin{entry}{叶子}{5,3}{⼝、⼦}
  \begin{phonetics}{叶子}{ye4zi5}[][HSK 4]
    \definition[片]{s.}{folha; termo genérico para as folhas de uma planta}
  \end{phonetics}
\end{entry}

\begin{entry}{号}{5}{⼝}
  \begin{phonetics}{号}{hao2}
    \definition[个]{s.}{rugido | choro}
  \end{phonetics}
  \begin{phonetics}{号}{hao4}[][HSK 1]
    \definition{clas.}{para indicar o número de pessoas}
    \definition{num.}{dia do mês | usado para indicar o número de pessoas}
    \definition[个]{s.}{número ordinal | dia de um mês | marca | sinal | estabelecimento comercial | tamanho | buzina (instrumento de sopro) | toque de corneta | nome suposto}
    \definition{suf.}{sufixo de navio}
    \definition{v.}{tomar um pulso}
  \end{phonetics}
\end{entry}

\begin{entry}{号召}{5,5}{⼝、⼝}
  \begin{phonetics}{号召}{hao4zhao4}[][HSK 5]
    \definition{s.}{chamado; apelo; desejo ou pedido solene (de um governo, partido político, organização etc.) para que as massas façam algo}
    \definition{v.}{chamar;  (governo, partido político, organização, etc.) fazer um pedido solene às massas para que façam algo, na esperança de que todos se esforcem para alcançá-lo}
  \end{phonetics}
\end{entry}

\begin{entry}{号角}{5,7}{⼝、⾓}
  \begin{phonetics}{号角}{hao4jiao3}
    \definition{s.}{corneta | trombeta}
  \end{phonetics}
\end{entry}

\begin{entry}{号码}{5,8}{⼝、⽯}
  \begin{phonetics}{号码}{hao4ma3}[][HSK 4]
    \definition[个,组,串]{s.}{número}
  \end{phonetics}
\end{entry}

\begin{entry}{司机}{5,6}{⼝、⽊}
  \begin{phonetics}{司机}{si1ji1}[][HSK 2]
    \definition{s.}{condutor | motorista | chofer}
  \end{phonetics}
\end{entry}

\begin{entry}{吃}{6}{⼝}
  \begin{phonetics}{吃}{chi1}[][HSK 1]
    \definition{v.}{comer | consumir | comer em (uma cafeteria, etc.) | erradicar | destruir | absorver}
  \end{phonetics}
\end{entry}

\begin{entry}{吃力}{6,2}{⼝、⼒}
  \begin{phonetics}{吃力}{chi1li4}[][HSK 5]
    \definition{adj.}{suado; extenuante; trabalhoso; laborioso | cansado; fatigado}
  \end{phonetics}
\end{entry}

\begin{entry}{吃饭}{6,7}{⼝、⾷}
  \begin{phonetics}{吃饭}{chi1 fan4}[][HSK 1]
    \definition{v.+compl.}{comer | ter (comer) uma refeição | manter vivo | ganhar a vida}
  \end{phonetics}
\end{entry}

\begin{entry}{吃屎}{6,9}{⼝、⼫}
  \begin{phonetics}{吃屎}{chi1 shi3}
    \definition{expr.}{Coma merda!}
  \end{phonetics}
\end{entry}

\begin{entry}{吃惊}{6,11}{⼝、⼼}
  \begin{phonetics}{吃惊}{chi1jing1}[][HSK 4]
    \definition{v.+compl.}{ficar assustado; ficar chocado; ficar espantado; pegar de surpresa; ficar assustado inesperadamente}
  \end{phonetics}
\end{entry}

\begin{entry}{各}{6}{⼝}
  \begin{phonetics}{各}{ge4}[][HSK 3]
    \definition{adv.}{indica que mais de uma pessoa ou coisa está fazendo algo ou tem um determinado atributo}
    \definition{pron.}{todo; todos; cada | diferentes entre si; vários}
  \end{phonetics}
\end{entry}

\begin{entry}{各个}{6,3}{⼝、⼈}
  \begin{phonetics}{各个}{ge4 ge4}[][HSK 4]
    \definition{adv./pron.}{cada | um a um; um após o outro}
  \end{phonetics}
\end{entry}

\begin{entry}{各地}{6,6}{⼝、⼟}
  \begin{phonetics}{各地}{ge4 di4}[][HSK 3]
    \definition{s.}{todos os lugares; vários lugares}
  \end{phonetics}
\end{entry}

\begin{entry}{各自}{6,6}{⼝、⾃}
  \begin{phonetics}{各自}{ge4zi4}[][HSK 3]
    \definition{pron.}{cada; respectivo; por si mesmo}
  \end{phonetics}
\end{entry}

\begin{entry}{各位}{6,7}{⼝、⼈}
  \begin{phonetics}{各位}{ge4 wei4}[][HSK 3]
    \definition{pron.}{todos | cada}
  \end{phonetics}
\end{entry}

\begin{entry}{各种}{6,9}{⼝、⽲}
  \begin{phonetics}{各种}{ge4 zhong3}[][HSK 3]
    \definition{adv.}{todos os tipos; vários; cada tipo}
  \end{phonetics}
\end{entry}

\begin{entry}{合}{6}{⼝}
  \begin{phonetics}{合}{he2}[][HSK 3]
    \definition*{s.}{sobrenome He}
    \definition{adj.}{todo; completo; inteiro}
    \definition{clas.}{para rodadas}
    \definition{s.}{conjunção}
    \definition{v.}{fechar | juntar; combinar | adequar-se; concordar; conformar-se a | ser igual a; somar}
  \end{phonetics}
\end{entry}

\begin{entry}{合同}{6,6}{⼝、⼝}
  \begin{phonetics}{合同}{he2tong5}[][HSK 4]
    \definition[个,份]{s.}{contrato; acordo; uma disposição para observância mútua por duas ou mais partes na condução de um assunto com o objetivo de determinar seus respectivos direitos e obrigações.}
  \end{phonetics}
\end{entry}

\begin{entry}{合并}{6,6}{⼝、⼲}
  \begin{phonetics}{合并}{he2bing4}[][HSK 5]
    \definition{v.}{fundir; amalgamar; combinar várias coisas em uma coisa só | (doença) ser complicada por outra doença; uma doença levar a outra, ataques simultâneos (de várias doenças)}
  \end{phonetics}
\end{entry}

\begin{entry}{合成}{6,6}{⼝、⼽}
  \begin{phonetics}{合成}{he2cheng2}[][HSK 5]
    \definition{s.}{compor; integrar; combinar; misturar | sintetizar, reação química para transformar uma substância com uma composição simples em uma substância com uma composição complexa}
  \end{phonetics}
\end{entry}

\begin{entry}{合作}{6,7}{⼝、⼈}
  \begin{phonetics}{合作}{he2zuo4}[][HSK 3]
    \definition[个]{s.}{cooperação; colaboração}
    \definition{v.}{cooperar; colaborar; trabalhar em conjunto}
  \end{phonetics}
\end{entry}

\begin{entry}{合法}{6,8}{⼝、⽔}
  \begin{phonetics}{合法}{he2fa3}[][HSK 3]
    \definition{adj.}{legal; legítimo; correto}
  \end{phonetics}
\end{entry}

\begin{entry}{合宪性}{6,9,8}{⼝、⼧、⼼}
  \begin{phonetics}{合宪性}{he2xian4xing4}
    \definition{s.}{constitucionalismo}
  \end{phonetics}
\end{entry}

\begin{entry}{合适}{6,9}{⼝、⾡}
  \begin{phonetics}{合适}{he2shi4}[][HSK 2]
    \definition{adj.}{certo | adequado | apropriado}
  \end{phonetics}
\end{entry}

\begin{entry}{合格}{6,10}{⼝、⽊}
  \begin{phonetics}{合格}{he2ge2}[][HSK 3]
    \definition{adj.}{qualificado; de acordo com o padrão}
  \end{phonetics}
\end{entry}

\begin{entry}{合资}{6,10}{⼝、⾙}
  \begin{phonetics}{合资}{he2zi1}
    \definition{s.}{\emph{joint-venture} com capitais mistos}
  \end{phonetics}
\end{entry}

\begin{entry}{合理}{6,11}{⼝、⽟}
  \begin{phonetics}{合理}{he2li3}[][HSK 3]
    \definition{adj.}{racional; razoável; equitativo}
  \end{phonetics}
\end{entry}

\begin{entry}{吉他}{6,5}{⼝、⼈}
  \begin{phonetics}{吉他}{ji2ta1}
    \definition[把]{s.}{(empréstimo linguístico) guitarra}
  \end{phonetics}
\end{entry}

\begin{entry}{同}{6}{⼝}
  \begin{phonetics}{同}{tong2}
    \definition{adj.}{junto}
    \definition{adv.}{junto com}
  \end{phonetics}
\end{entry}

\begin{entry}{同伙}{6,6}{⼝、⼈}
  \begin{phonetics}{同伙}{tong2huo3}
    \definition[个]{s.}{cúmplice | colega}
  \end{phonetics}
\end{entry}

\begin{entry}{同时}{6,7}{⼝、⽇}
  \begin{phonetics}{同时}{tong2shi2}[][HSK 2]
    \definition{conj.}{além disso}
    \definition{s.}{enquanto isso | ao mesmo tempo}
  \end{phonetics}
\end{entry}

\begin{entry}{同事}{6,8}{⼝、⼅}
  \begin{phonetics}{同事}{tong2shi4}[][HSK 2]
    \definition{s.}{colega | colega de trabalho | companheiro}
  \end{phonetics}
\end{entry}

\begin{entry}{同学}{6,8}{⼝、⼦}
  \begin{phonetics}{同学}{tong2xue2}[][HSK 1]
    \definition[位,个]{s.}{colega de classe | colega estudante}
  \end{phonetics}
\end{entry}

\begin{entry}{同性恋}{6,8,10}{⼝、⼼、⼼}
  \begin{phonetics}{同性恋}{tong2xing4lian4}
    \definition{s.}{homossexualidade | pessoa gay | amor gay}
  \end{phonetics}
\end{entry}

\begin{entry}{同屋}{6,9}{⼝、⼫}
  \begin{phonetics}{同屋}{tong2wu1}
    \definition[个]{s.}{companheiro de quarto | colega de quarto}
  \end{phonetics}
\end{entry}

\begin{entry}{同砚}{6,9}{⼝、⽯}
  \begin{phonetics}{同砚}{tong2yan4}
    \definition[位,个]{s.}{colega de classe | colega estudante}
  \end{phonetics}
\end{entry}

\begin{entry}{同样}{6,10}{⼝、⽊}
  \begin{phonetics}{同样}{tong2 yang4}[][HSK 2]
    \definition{adj.}{igual | similar}
  \end{phonetics}
\end{entry}

\begin{entry}{同流合污}{6,10,6,6}{⼝、⽔、⼝、⽔}
  \begin{phonetics}{同流合污}{tong2liu2he2wu1}
    \definition{expr.}{chafurdar na lama com alguém | seguir o mau exemplo dos outros}
  \end{phonetics}
\end{entry}

\begin{entry}{同情}{6,11}{⼝、⼼}
  \begin{phonetics}{同情}{tong2qing2}[][HSK 4]
    \definition{s.}{simpatia}
    \definition{v.}{simpatizar com; solidarizar-se; compadecer-se; ter empatia emocional pelo que os outros estão passando}
  \end{phonetics}
\end{entry}

\begin{entry}{同意}{6,13}{⼝、⼼}
  \begin{phonetics}{同意}{tong2yi4}[][HSK 3]
    \definition{v.}{concordar; consentir; aprovar; concordar com; dizer sim}
  \end{phonetics}
\end{entry}

\begin{entry}{名}{6}{⼝}
  \begin{phonetics}{名}{ming2}[][HSK 2]
    \definition*{s.}{sobrenome Ming}
    \definition{s.}{nome | denominação | fama | reputação}
  \end{phonetics}
\end{entry}

\begin{entry}{名人}{6,2}{⼝、⼈}
  \begin{phonetics}{名人}{ming2 ren2}[][HSK 4]
    \definition{s.}{celebridade; pessoa famosa}
  \end{phonetics}
\end{entry}

\begin{entry}{名片}{6,4}{⼝、⽚}
  \begin{phonetics}{名片}{ming2pian4}[][HSK 4]
    \definition[张,盒,叠]{s.}{cartão de visita; um pedaço de papel retangular com o nome, o cargo, o endereço etc. impressos}
  \end{phonetics}
\end{entry}

\begin{entry}{名字}{6,6}{⼝、⼦}
  \begin{phonetics}{名字}{ming2zi5}[][HSK 1]
    \definition[个]{s.}{nome (de uma pessoa ou coisa)}
  \end{phonetics}
\end{entry}

\begin{entry}{名单}{6,8}{⼝、⼗}
  \begin{phonetics}{名单}{ming2 dan1}[][HSK 2]
    \definition[个]{s.}{lista | lista de nomes}
  \end{phonetics}
\end{entry}

\begin{entry}{名称}{6,10}{⼝、⽲}
  \begin{phonetics}{名称}{ming2 cheng1}[][HSK 2]
    \definition[个,种]{s.}{nome | designação}
  \end{phonetics}
\end{entry}

\begin{entry}{名牌儿}{6,12,2}{⼝、⽚、⼉}
  \begin{phonetics}{名牌儿}{ming2 pai2r5}[][HSK 4]
    \definition*{s.}{Marca famosa}
  \end{phonetics}
\end{entry}

\begin{entry}{后}{6}{⼝}
  \begin{phonetics}{后}{hou4}[][HSK 1]
    \definition*{s.}{sobrenome Hou}
    \definition{adv.}{atrás | depois | mais tarde}
    \definition{s.}{traseiro | descendência | posteridade |imperatriz | rainha | soberano | governante}
  \end{phonetics}
\end{entry}

\begin{entry}{后天}{6,4}{⼝、⼤}
  \begin{phonetics}{后天}{hou4 tian1}[][HSK 1]
    \definition{adv.}{depois de amanhã}
  \end{phonetics}
\end{entry}

\begin{entry}{后头}{6,5}{⼝、⼤}
  \begin{phonetics}{后头}{hou4 tou5}[][HSK 4]
    \definition{adv.}{posteriormente | atrás | mais tarde}
    \definition{s.}{a parte de trás | a parte traseira}
  \end{phonetics}
\end{entry}

\begin{entry}{后边}{6,5}{⼝、⾡}
  \begin{phonetics}{后边}{hou4 bian5}[][HSK 1]
    \definition{adv.}{atrás | detrás}
  \end{phonetics}
\end{entry}

\begin{entry}{后年}{6,6}{⼝、⼲}
  \begin{phonetics}{后年}{hou4nian2}[][HSK 3]
    \definition{s.}{o ano que vem; daqui a dois anos}
  \end{phonetics}
\end{entry}

\begin{entry}{后来}{6,7}{⼝、⽊}
  \begin{phonetics}{后来}{hou4lai2}[][HSK 2]
    \definition{adv.}{mais tarde}
  \end{phonetics}
\end{entry}

\begin{entry}{后果}{6,8}{⼝、⽊}
  \begin{phonetics}{后果}{hou4guo3}[][HSK 3]
    \definition{s.}{consequência; resultado}
  \end{phonetics}
\end{entry}

\begin{entry}{后面}{6,9}{⼝、⾯}
  \begin{phonetics}{后面}{hou4mian4}[][HSK 3]
    \definition{adv.}{parte de trás; retaguarda; atrás | atrás; perto do fim; na parte de trás | mais tarde; depois}
  \end{phonetics}
  \begin{phonetics}{后面}{hou4mian5}[][HSK 3]
    \definition{adv.}{parte de trás; retaguarda; atrás | atrás; perto do fim; na parte de trás | mais tarde; depois}
  \end{phonetics}
\end{entry}

\begin{entry}{后悔}{6,10}{⼝、⼼}
  \begin{phonetics}{后悔}{hou4hui3}[][HSK 5]
    \definition{v.}{lamentar; ter remorso; arrepender-se}
  \end{phonetics}
\end{entry}

\begin{entry}{吐}{6}{⼝}
  \begin{phonetics}{吐}{tu3}[][HSK 5]
    \definition{v.}{cuspir; sair pela boca | surgir ou aparecer pela boca ou por uma fenda | dizer; contar; falar abertamente}
  \end{phonetics}
  \begin{phonetics}{吐}{tu4}[][HSK 5]
    \definition{v.}{vomitar; sair pela boca | vomitar; expelir; metáfora para ser forçado a devolver bens usurpados}
  \end{phonetics}
\end{entry}

\begin{entry}{向}{6}{⼝}
  \begin{phonetics}{向}{xiang4}[][HSK 2]
    \definition*{s.}{sobrenome Xiang}
    \definition{prep.}{para}
    \definition{v.}{enfrentar | virar para | apoiar}
  \end{phonetics}
\end{entry}

\begin{entry}{向汪}{6,7}{⼝、⽔}
  \begin{phonetics}{向汪}{xiang4wang1}
    \definition{v.}{esperar que}
  \end{phonetics}
\end{entry}

\begin{entry}{向往}{6,8}{⼝、⼻}
  \begin{phonetics}{向往}{xiang4wang3}
    \definition{v.}{ansiar por | esperar ansiosamente por}
  \end{phonetics}
\end{entry}

\begin{entry}{吓人}{6,2}{⼝、⼈}
  \begin{phonetics}{吓人}{xia4ren2}
    \definition{adj.}{apavorante | assustador}
    \definition{v.+compl.}{assustar-se | tomar um susto}
  \end{phonetics}
\end{entry}

\begin{entry}{吗}{6}{⼝}
  \begin{phonetics}{吗}{ma2}
    \definition{adv.}{(coloquial) que?}
  \end{phonetics}
  \begin{phonetics}{吗}{ma3}
    \definition{s.}{usada em 吗啡}
    \seeref{吗啡}{ma3fei1}
  \end{phonetics}
  \begin{phonetics}{吗}{ma5}[][HSK 1]
    \definition{part.}{partícula interrogativa, usada em perguntas ``sim-não''}
  \end{phonetics}
\end{entry}

\begin{entry}{吗啡}{6,11}{⼝、⼝}
  \begin{phonetics}{吗啡}{ma3fei1}
    \definition{s.}{morfina (empréstimo linguístico)}
  \end{phonetics}
\end{entry}

\begin{entry}{吸}{6}{⼝}
  \begin{phonetics}{吸}{xi1}[][HSK 4]
    \definition{v.}{inalar; inspirar; aspirar; itroduzir líquidos, gases, etc. no corpo | absorver; sugar | atrair; atrair para si mesmo; atrair (interesse, investimento etc.)}
  \end{phonetics}
\end{entry}

\begin{entry}{吸引}{6,4}{⼝、⼸}
  \begin{phonetics}{吸引}{xi1yin3}[][HSK 4]
    \definition{v.}{atrair; apelar para; chamar a atenção de outros objetos, forças ou pessoas para si mesmo}
  \end{phonetics}
\end{entry}

\begin{entry}{吸收}{6,6}{⼝、⽁}
  \begin{phonetics}{吸收}{xi1shou1}[][HSK 4]
    \definition{v.}{imbuir; absorver; assimilar; sugar;  chupar; (animais, plantas, etc.) extrair material de fora dos tecidos para o interior dos tecidos | absorver; chupar;  sugar alguma substância de fora para dentro | recrutar; alistar; inscrever-se; matricular-se; admitir; (organizações ou coletivos) aceitar novos membros | absorver; aproveitar e usar a experiência, o conhecimento, o dinheiro e outras coisas valiosas de outras pessoas | absorver; diminuir, atenuar ou eliminar determinados efeitos ou fenômenos}
  \end{phonetics}
\end{entry}

\begin{entry}{吸烟}{6,10}{⼝、⽕}
  \begin{phonetics}{吸烟}{xi1yan1}[][HSK 4]
    \definition{v.+compl.}{fumar}
  \end{phonetics}
\end{entry}

\begin{entry}{吸铁石}{6,10,5}{⼝、⾦、⽯}
  \begin{phonetics}{吸铁石}{xi1tie3shi2}
    \definition{s.}{imã | magneto}
  \seealsoref{磁铁}{ci2tie3}
  \end{phonetics}
\end{entry}

\begin{entry}{吸管}{6,14}{⼝、⽵}
  \begin{phonetics}{吸管}{xi1 guan3}[][HSK 4]
    \definition[根,个]{s.}{tubo de sucção; sugador; canudo (para beber); refere-se ao tubo fino usado para sugar bebidas | conta-gotas; pipeta; cateter para bombeamento de líquidos usando pressão de ar}
  \end{phonetics}
\end{entry}

\begin{entry}{君主立宪制}{7,5,5,9,8}{⼝、⼂、⽴、⼧、⼑}
  \begin{phonetics}{君主立宪制}{jun1zhu3li4xian4zhi4}
    \definition{s.}{monarquia constitucional}
  \end{phonetics}
\end{entry}

\begin{entry}{吟诗}{7,8}{⼝、⾔}
  \begin{phonetics}{吟诗}{yin2shi1}
    \definition{v.}{recitar poesia}
  \end{phonetics}
\end{entry}

\begin{entry}{否认}{7,4}{⼝、⾔}
  \begin{phonetics}{否认}{fou3ren4}[][HSK 3]
    \definition{v.}{negar; repudiar}
  \end{phonetics}
\end{entry}

\begin{entry}{否则}{7,6}{⼝、⼑}
  \begin{phonetics}{否则}{fou3ze2}[][HSK 4]
    \definition{conj.}{senão; se não; ou então; se não for isso}
  \end{phonetics}
\end{entry}

\begin{entry}{否定}{7,8}{⼝、⼧}
  \begin{phonetics}{否定}{fou3ding4}[][HSK 3]
    \definition{adj.}{negativo}
    \definition{s.}{negativo (resposta); negação}
    \definition{v.}{rejeitar; negar}
  \end{phonetics}
\end{entry}

\begin{entry}{吧}{7}{⼝}
  \begin{phonetics}{吧}{ba1}
    \definition{s.}{som de estalo, som crepitante}
    \definition{v.}{puxar o cachimbo; fumar | abreviação de ``bar''}
  \end{phonetics}
  \begin{phonetics}{吧}{ba5}[][HSK 1]
    \definition{part.}{indica discussão, sugestão, solicitação ou comando no final de uma frase | indica concordância ou aprovação no final de uma frase | indica uma pergunta ou especulação no final de uma frase | indica incerteza no final de uma frase | em uma frase, indica uma pausa, carrega um tom hipotético, frequentemente apresenta um contraste e implica um dilema}
  \end{phonetics}
\end{entry}

\begin{entry}{吨}{7}{⼝}
  \begin{phonetics}{吨}{dun1}[][HSK 5]
    \definition{clas.}{tonelada}
  \end{phonetics}
\end{entry}

\begin{entry}{含}{7}{⼝}
  \begin{phonetics}{含}{han2}[][HSK 4]
    \definition{v.}{manter na boca (sem engolir ou cuspir) | conter; incluir | cuidar; acalentar; abrigar}
  \end{phonetics}
\end{entry}

\begin{entry}{含义}{7,3}{⼝、⼂}
  \begin{phonetics}{含义}{han2yi4}[][HSK 4]
    \definition[个,种,层]{s.}{sentido; mensagem; significado; implicação}
  \end{phonetics}
\end{entry}

\begin{entry}{含有}{7,6}{⼝、⽉}
  \begin{phonetics}{含有}{han2 you3}[][HSK 4]
    \definition{v.}{conter; ter; incluir}
  \end{phonetics}
\end{entry}

\begin{entry}{含金量}{7,8,12}{⼝、⾦、⾥}
  \begin{phonetics}{含金量}{han2jin1liang4}
    \definition{adj.}{conteúdo de ouro | (fig.) valioso}
  \end{phonetics}
\end{entry}

\begin{entry}{含量}{7,12}{⼝、⾥}
  \begin{phonetics}{含量}{han2 liang4}[][HSK 4]
    \definition{s.}{conteúdo; a quantidade de um componente contido em uma substância}
  \end{phonetics}
\end{entry}

\begin{entry}{听}{7}{⼝}
  \begin{phonetics}{听}{ting1}[][HSK 1]
    \definition{clas.}{para bebidas enlatadas}
    \definition{s.}{lata de bebida (empréstimo linguístico, do inglês ``\emph{tin}'')}
    \definition{v.}{ouvir | escutar | obedecer}
  \end{phonetics}
\end{entry}

\begin{entry}{听力}{7,2}{⼝、⼒}
  \begin{phonetics}{听力}{ting1li4}[][HSK 3]
    \definition{s.}{audição; capacidade auditiva | compreensão auditiva (na aprendizagem de línguas)}
  \end{phonetics}
\end{entry}

\begin{entry}{听力理解}{7,2,11,13}{⼝、⼒、⽟、⾓}
  \begin{phonetics}{听力理解}{ting1li4li3jie3}
    \definition{s.}{compreensão auditiva}
  \end{phonetics}
\end{entry}

\begin{entry}{听小骨}{7,3,9}{⼝、⼩、⾻}
  \begin{phonetics}{听小骨}{ting1xiao3gu3}
    \definition{s.}{ossículos (do ouvido médio)}
  \seealsoref{听骨}{ting1gu3}
  \end{phonetics}
\end{entry}

\begin{entry}{听见}{7,4}{⼝、⾒}
  \begin{phonetics}{听见}{ting1 jian4}[][HSK 1]
    \definition{v.}{ouvir}
  \end{phonetics}
\end{entry}

\begin{entry}{听写}{7,5}{⼝、⼍}
  \begin{phonetics}{听写}{ting1xie3}[][HSK 1]
    \definition{s.}{ditado}
    \definition{v.}{transcrever música de ouvido | escrever (em um exercício de ditado)}
  \end{phonetics}
\end{entry}

\begin{entry}{听众}{7,6}{⼝、⼈}
  \begin{phonetics}{听众}{ting1 zhong4}[][HSK 3]
    \definition{s.}{audiência; ouvintes}
  \end{phonetics}
\end{entry}

\begin{entry}{听会}{7,6}{⼝、⼈}
  \begin{phonetics}{听会}{ting1hui4}
    \definition{v.}{participar de uma reunião (e ouvir o que é discutido)}
  \end{phonetics}
\end{entry}

\begin{entry}{听戏}{7,6}{⼝、⼽}
  \begin{phonetics}{听戏}{ting1xi4}
    \definition{v.}{assistir a uma ópera | ver uma ópera}
  \end{phonetics}
\end{entry}

\begin{entry}{听讲}{7,6}{⼝、⾔}
  \begin{phonetics}{听讲}{ting1 jiang3}[][HSK 2]
    \definition{v.+compl.}{assistir a uma palestra; ouvir uma conversa}
  \end{phonetics}
\end{entry}

\begin{entry}{听来}{7,7}{⼝、⽊}
  \begin{phonetics}{听来}{ting1lai2}
    \definition{v.}{ouvir de algum lugar | soar (antigo, estrangeiro, excitante, certo, etc.) | soar como se (ou seja, dar uma impressão ao ouvinte)}
  \end{phonetics}
\end{entry}

\begin{entry}{听凭}{7,8}{⼝、⼏}
  \begin{phonetics}{听凭}{ting1ping2}
    \definition{v.}{permitir (alguém a fazer o que desejar)}
  \end{phonetics}
\end{entry}

\begin{entry}{听到}{7,8}{⼝、⼑}
  \begin{phonetics}{听到}{ting1dao4}[][HSK 1]
    \definition{v.}{ouvir | notar}
  \end{phonetics}
\end{entry}

\begin{entry}{听命}{7,8}{⼝、⼝}
  \begin{phonetics}{听命}{ting1ming4}
    \definition{v.}{obedecer ordens | receber ordens}
  \end{phonetics}
\end{entry}

\begin{entry}{听说}{7,9}{⼝、⾔}
  \begin{phonetics}{听说}{ting1 shuo1}[][HSK 2]
    \definition{v.}{ouvir dizer}
  \end{phonetics}
\end{entry}

\begin{entry}{听骨}{7,9}{⼝、⾻}
  \begin{phonetics}{听骨}{ting1gu3}
    \definition{s.}{ossículos (do ouvido médio)}
  \seealsoref{听小骨}{ting1xiao3gu3}
  \end{phonetics}
\end{entry}

\begin{entry}{听断}{7,11}{⼝、⽄}
  \begin{phonetics}{听断}{ting1duan4}
    \definition{v.}{ouvir e decidir | julgar (ou seja, ouvir e julgar em um tribunal)}
  \end{phonetics}
\end{entry}

\begin{entry}{听随}{7,11}{⼝、⾩}
  \begin{phonetics}{听随}{ting1sui2}
    \definition{v.}{permitir | obedecer}
  \end{phonetics}
\end{entry}

\begin{entry}{启发}{7,5}{⼝、⼜}
  \begin{phonetics}{启发}{qi3fa1}[][HSK 5]
    \definition{s.}{iluminação; esclarecimento; fenômenos e princípios que levam as pessoas a refletir e a abrir suas mentes}
    \definition{v.}{despertar; inspirar; esclarecer; orientar, fazer com que compreendam}
  \end{phonetics}
\end{entry}

\begin{entry}{启动}{7,6}{⼝、⼒}
  \begin{phonetics}{启动}{qi3 dong4}[][HSK 5]
    \definition{v.}{ligar (uma máquina); acionar; ligar máquinas, equipamentos elétricos, etc., para começar a trabalhar | entrar em vigor; começar a vigorar e a ser implementados planos, projetos, documentos jurídicos, etc.}
  \end{phonetics}
\end{entry}

\begin{entry}{启事}{7,8}{⼝、⼅}
  \begin{phonetics}{启事}{qi3shi4}[][HSK 5]
    \definition{s.}{aviso; anúncio; texto publicado em jornais ou afixado em paredes com o objetivo de divulgar publicamente algo}
  \end{phonetics}
\end{entry}

\begin{entry}{吵}{7}{⼝}
  \begin{phonetics}{吵}{chao3}[][HSK 3]
    \definition{adj.}{barulhento; ruidoso}
    \definition{v.}{perturbar fazendo barulho; fazer barulho | discutir; brigar; disputar}
  \end{phonetics}
\end{entry}

\begin{entry}{吵架}{7,9}{⼝、⽊}
  \begin{phonetics}{吵架}{chao3jia4}[][HSK 3]
    \definition{v.+compl.}{brigar; discutir; ter uma briga}
  \end{phonetics}
\end{entry}

\begin{entry}{吹}{7}{⼝}
  \begin{phonetics}{吹}{chui1}[][HSK 2]
    \definition{v.}{soprar | tocar (instrumentos de sopro) | bajular |  louvar aos céus | separar (casal)  | fracassar}
  \end{phonetics}
\end{entry}

\begin{entry}{吹牛}{7,4}{⼝、⽜}
  \begin{phonetics}{吹牛}{chui1niu2}
    \definition{v.+compl.}{ogulhar-se | gabar-se | destacar-se}
  \end{phonetics}
\end{entry}

\begin{entry}{吾}{7}{⼝}
  \begin{phonetics}{吾}{wu2}
    \definition*{s.}{sobrenome Wu}
    \definition{pron.}{eu | (antigo) meu}
  \end{phonetics}
\end{entry}

\begin{entry}{呀}{7}{⼝}
  \begin{phonetics}{呀}{ya5}[][HSK 4]
    \definition{part.}{usado no lugar de 啊 quando a palavra anterior termina com o som a, e, i, o ou ü}
  \seealsoref{啊}{a5}
  \end{phonetics}
\end{entry}

\begin{entry}{呆}{7}{⼝}
  \begin{phonetics}{呆}{dai1}[][HSK 5]
    \definition*{s.}{sobrenome Dai}
    \definition{adj.}{maçante; de raciocínio lento | em branco; de madeira; rígido; inflexível}
    \definition{v.}{ficar; permanecer;}
  \end{phonetics}
\end{entry}

\begin{entry}{告别}{7,7}{⼝、⼑}
  \begin{phonetics}{告别}{gao4bie2}[][HSK 3]
    \definition{v.+compl.}{dizer adeus a | deixar; partir de | prestar as últimas homenagens ao falecido}
  \end{phonetics}
\end{entry}

\begin{entry}{告诉}{7,7}{⼝、⾔}
  \begin{phonetics}{告诉}{gao4su4}
    \definition{v.}{apresentar queixa | registar uma reclamação}
  \end{phonetics}
  \begin{phonetics}{告诉}{gao4su5}[][HSK 1]
    \definition{v.}{contar | dar a conhecer | informar}
  \end{phonetics}
\end{entry}

\begin{entry}{告急}{7,9}{⼝、⼼}
  \begin{phonetics}{告急}{gao4ji2}
    \definition{v.}{estar em estado de emergência | relatar uma emergência | solicitar assistência de emergência}
  \end{phonetics}
\end{entry}

\begin{entry}{员}{7}{⼝}
  \begin{phonetics}{员}{yuan2}[][HSK 3]
    \definition{clas.}{para comandantes militares}
    \definition{s.}{uma pessoa envolvida em algum campo de atividade; refere-se a pessoas que trabalham ou estudam | membro; refere-se aos membros de um grupo ou organização}
  \end{phonetics}
\end{entry}

\begin{entry}{员工}{7,3}{⼝、⼯}
  \begin{phonetics}{员工}{yuan2gong1}[][HSK 3]
    \definition[位,名,个]{s.}{funcionário; atendente; balconista; empregado; trabalhador; pessoal}
  \end{phonetics}
\end{entry}

\begin{entry}{呢}{8}{⼝}
  \begin{phonetics}{呢}{ne5}[][HSK 1]
    \definition{part.}{(no final de uma frase declarativa) partícula que indica a continuação de um estado ou ação |  partícula para perguntar sobre a localização (``Onde está\dots?'') | partícula indicando  afirmação forte | partícula indicando que uma pergunta feita anteriormente deve ser aplicada à palavra anterior (``E quanto a\dots?'', ``E\dots?'') | partícula sinalizando uma pausa, para enfatizar as palavras anteriores e permitir que o ouvinte tenha tempo para compreendê-las (``ok?'', ``você está comigo ?'')}
  \end{phonetics}
  \begin{phonetics}{呢}{ni2}
    \definition{s.}{material de lã}
  \end{phonetics}
\end{entry}

\begin{entry}{周}{8}{⼝}
  \begin{phonetics}{周}{zhou1}[][HSK 2]
    \definition*{s.}{sobrenome Zhou | Dinastia Zhou (1046-256 BC)}
    \definition{adv.}{semanalmente}
    \definition{s.}{círculo | circunferência | ciclo | uma volta (em um circuito) | semana}
    \definition{v.}{fazer um circuito |circular | ajudar financeiramente}
  \end{phonetics}
\end{entry}

\begin{entry}{周末}{8,5}{⼝、⽊}
  \begin{phonetics}{周末}{zhou1mo4}[][HSK 2]
    \definition{s.}{final-de-semana}
  \end{phonetics}
\end{entry}

\begin{entry}{周年}{8,6}{⼝、⼲}
  \begin{phonetics}{周年}{zhou1nian2}[][HSK 2]
    \definition{s.}{aniversário}
  \end{phonetics}
\end{entry}

\begin{entry}{周围}{8,7}{⼝、⼞}
  \begin{phonetics}{周围}{zhou1wei2}[][HSK 3]
    \definition{s.}{ao redor; em torno; vizinhança; a parte que circunda o centro}
  \end{phonetics}
\end{entry}

\begin{entry}{味}{8}{⼝}
  \begin{phonetics}{味}{wei4}
    \definition{clas.}{para medicamentos}
    \definition{s.}{cheiro | gosto}
  \end{phonetics}
\end{entry}

\begin{entry}{味儿}{8,2}{⼝、⼉}
  \begin{phonetics}{味儿}{wei4r5}[][HSK 4]
    \definition{s.}{gosto; sabor; propriedade de uma substância que dá à língua uma determinada sensação de sabor | cheiro; odor; propriedade de uma substância que dá ao nariz um determinado sentido de cheiro | interesse; significado; deleite}
  \end{phonetics}
\end{entry}

\begin{entry}{味道}{8,12}{⼝、⾡}
  \begin{phonetics}{味道}{wei4dao5}[][HSK 2]
    \definition{s.}{sabor | (dialeto) odor, cheiro | (figurativo) sentimento (de…), dica (de…) | (figurativo) interesse, prazer}
  \end{phonetics}
\end{entry}

\begin{entry}{呵}{8}{⼝}
  \begin{phonetics}{呵}{a1}
    \variantof{啊}
  \end{phonetics}
  \begin{phonetics}{呵}{he1}
    \definition{expr.}{Meu Deus! | expelir a respiração}
  \end{phonetics}
\end{entry}

\begin{entry}{呼吸}{8,6}{⼝、⼝}
  \begin{phonetics}{呼吸}{hu1xi1}[][HSK 4]
    \definition{s.}{um suspiro; metáfora para um período de tempo muito curto}
    \definition{v.}{respirar}
  \end{phonetics}
\end{entry}

\begin{entry}{呼啸}{8,11}{⼝、⼝}
  \begin{phonetics}{呼啸}{hu1xiao4}
    \definition{v.}{assobiar}
  \end{phonetics}
\end{entry}

\begin{entry}{命令}{8,5}{⼝、⼈}
  \begin{phonetics}{命令}{ming4ling4}[][HSK 5]
    \definition[道,个]{s.}{ordem; comando; instruções emitidas pelos superiores aos subordinados}
    \definition{v.}{ordenar; comandar}
  \end{phonetics}
\end{entry}

\begin{entry}{命运}{8,7}{⼝、⾡}
  \begin{phonetics}{命运}{ming4yun4}[][HSK 3]
    \definition[个]{s.}{tendência de desenvolvimento; tendência de futuro | destino; sina; sorte}
  \end{phonetics}
\end{entry}

\begin{entry}{和}{8}{⼝}
  \begin{phonetics}{和}{he2}[][HSK 1]
    \definition*{s.}{sobrenome He}
    \definition{conj.}{e (somente para palavras) | junto com}
    \definition{s.}{união | paz | japonês (comida, roupa, etc.) | harmonia}
  \end{phonetics}
  \begin{phonetics}{和}{he4}
    \definition{v.}{compor um poema em resposta (ao poema de alguém) usando a mesma sequência de rimas | juntar-se à cantoria | cantar junto com outros}
  \end{phonetics}
  \begin{phonetics}{和}{hu2}
    \definition{v.}{completar um conjunto de Mahjong ou cartas de baralho}
  \end{phonetics}
  \begin{phonetics}{和}{huo2}
    \definition{v.}{combinar uma substância em pó (farinha, gesso, etc.) com água}
  \end{phonetics}
  \begin{phonetics}{和}{huo4}
    \definition{clas.}{para enxágues de roupas | para fervuras de ervas medicinais}
    \definition{v.}{misturar (ingredientes) | misturar}
  \end{phonetics}
\end{entry}

\begin{entry}{和平}{8,5}{⼝、⼲}
  \begin{phonetics}{和平}{he2ping2}[][HSK 3]
    \definition{adj.}{pacífico; não violento}
    \definition{s.}{paz}
  \end{phonetics}
\end{entry}

\begin{entry}{和平共处}{8,5,6,5}{⼝、⼲、⼋、⼡}
  \begin{phonetics}{和平共处}{he2ping2gong4chu3}
    \definition{s.}{coexistência pacífica de nações, sociedades, etc.}
  \end{phonetics}
\end{entry}

\begin{entry}{和谐}{8,11}{⼝、⾔}
  \begin{phonetics}{和谐}{he2xie2}
    \definition{adj.}{harmonioso}
    \definition{s.}{harmonia}
    \definition{v.}{(eufemismo) censurar}
  \end{phonetics}
\end{entry}

\begin{entry}{咒骂}{8,9}{⼝、⾺}
  \begin{phonetics}{咒骂}{zhou4ma4}
    \definition{v.}{xingar | amaldiçoar | execrar}
  \end{phonetics}
\end{entry}

\begin{entry}{咖啡}{8,11}{⼝、⼝}
  \begin{phonetics}{咖啡}{ka1fei1}[][HSK 3]
    \definition[杯]{s.}{(empréstimo linguístico) café}
  \end{phonetics}
\end{entry}

\begin{entry}{咖啡色}{8,11,6}{⼝、⼝、⾊}
  \begin{phonetics}{咖啡色}{ka1fei1 se4}
    \definition{s.}{cor café}
  \end{phonetics}
\end{entry}

\begin{entry}{咖啡馆}{8,11,11}{⼝、⼝、⾷}
  \begin{phonetics}{咖啡馆}{ka1fei1guan3}
    \definition[家]{s.}{cafeteria}
  \end{phonetics}
\end{entry}

\begin{entry}{咱}{9}{⼝}
  \begin{phonetics}{咱}{zan2}[][HSK 2]
    \definition{pron.}{eu}
  \end{phonetics}
\end{entry}

\begin{entry}{咱们}{9,5}{⼝、⼈}
  \begin{phonetics}{咱们}{zan2men5}[][HSK 2]
    \definition{pron.}{nós (incluindo o orador e a(s) pessoa(s) com quem se fala)}
  \end{phonetics}
\end{entry}

\begin{entry}{咱俩}{9,9}{⼝、⼈}
  \begin{phonetics}{咱俩}{zan2lia3}
    \definition{pron.}{nós dois}
  \end{phonetics}
\end{entry}

\begin{entry}{咱家}{9,10}{⼝、⼧}
  \begin{phonetics}{咱家}{za2jia1}
    \definition{pron.}{eu (frequentemente usado na literatura vernácula antiga) | me | mim | comigo}
  \end{phonetics}
\end{entry}

\begin{entry}{咳}{9}{⼝}
  \begin{phonetics}{咳}{hai1}
    \definition{interj.}{expressa tristeza, arrependimento ou espanto}
  \end{phonetics}
  \begin{phonetics}{咳}{ke2}[][HSK 5]
    \definition{v.}{tossir}
  \end{phonetics}
\end{entry}

\begin{entry}{咳嗽}{9,14}{⼝、⼝}
  \begin{phonetics}{咳嗽}{ke2sou5}
    \definition{v.}{ter tosse | tossir}
  \end{phonetics}
\end{entry}

\begin{entry}{咸}{9}{⼝}
  \begin{phonetics}{咸}{xian2}[][HSK 4]
    \definition*{s.}{sobrenome Xian}
    \definition{adj.}{salgado; em conserva; sabor salgado}
    \definition{adv.}{todos; indica a totalidade de um intervalo, equivalente a ``全'' e ``都''}
  \seealsoref{都}{dou1}
  \seealsoref{全}{quan2}
  \end{phonetics}
\end{entry}

\begin{entry}{咸水}{9,4}{⼝、⽔}
  \begin{phonetics}{咸水}{xian2shui3}
    \definition{s.}{salmora | água salgada}
  \end{phonetics}
\end{entry}

\begin{entry}{咸肉}{9,6}{⼝、⾁}
  \begin{phonetics}{咸肉}{xian2rou4}
    \definition{s.}{\emph{bacon} | carne curada com sal}
  \end{phonetics}
\end{entry}

\begin{entry}{咸鱼}{9,8}{⼝、⿂}
  \begin{phonetics}{咸鱼}{xian2yu2}
    \definition{s.}{peixe salgado}
  \end{phonetics}
\end{entry}

\begin{entry}{咸涩}{9,10}{⼝、⽔}
  \begin{phonetics}{咸涩}{xian2se4}
    \definition{s.}{ácido | salgado e amargo}
  \end{phonetics}
\end{entry}

\begin{entry}{咸盐}{9,10}{⼝、⽫}
  \begin{phonetics}{咸盐}{xian2yan2}
    \definition{s.}{(coloquial) sal | sal de mesa}
  \end{phonetics}
\end{entry}

\begin{entry}{咸淡}{9,11}{⼝、⽔}
  \begin{phonetics}{咸淡}{xian2dan4}
    \definition{s.}{água salobra | grau de salinidade | salgado e sem sal (sabores)}
  \end{phonetics}
\end{entry}

\begin{entry}{咸菜}{9,11}{⼝、⾋}
  \begin{phonetics}{咸菜}{xian2cai4}
    \definition{s.}{legumes salgados | \emph{pickles}}
  \end{phonetics}
\end{entry}

\begin{entry}{品}{9}{⼝}
  \begin{phonetics}{品}{pin3}[][HSK 5]
    \definition*{s.}{sobrenome Pin}
    \definition{s.}{artigo; produto | grau; classe; classificação; nível | caráter; qualidade | classificação; os graus dos funcionários públicos antigos, num total de nove graus}
    \definition{v.}{provar; saborear; degustar algo com discernimento | soprar; tocar (instrumentos de sopro) | avaliar; distinguir o bom do ruim}
  \end{phonetics}
\end{entry}

\begin{entry}{品质}{9,8}{⼝、⾙}
  \begin{phonetics}{品质}{pin3zhi4}[][HSK 4]
    \definition[个,种]{s.}{qualidade; caráter; natureza do pensamento, da compreensão, do caráter, etc., conforme expresso no comportamento, no estilo, etc. | qualidade (de produtos, mercadorias, etc.)}
  \end{phonetics}
\end{entry}

\begin{entry}{品种}{9,9}{⼝、⽲}
  \begin{phonetics}{品种}{pin3zhong3}[][HSK 5]
    \definition[个]{s.}{raça; linhagem; variedade; refere-se a um grupo de organismos com características genéticas comuns, formados por meio da seleção e cultivo artificiais de culturas, gado, aves, etc. | variedade; sortimento; referência geral ao tipo de item}
  \end{phonetics}
\end{entry}

\begin{entry}{品德}{9,15}{⼝、⼻}
  \begin{phonetics}{品德}{pin3de2}
    \definition{s.}{caráter moral | moralidade}
  \end{phonetics}
\end{entry}

\begin{entry}{哄}{9}{⼝}
  \begin{phonetics}{哄}{hong1}
    \definition{s.}{gargalhadas | risadas ruidosas | algazarra | rugido | clamor}
  \end{phonetics}
  \begin{phonetics}{哄}{hong3}
    \definition{v.}{enganar | persuadir | divertir (uma criança)}
  \end{phonetics}
  \begin{phonetics}{哄}{hong4}
    \definition{s.}{tumulto | agitação | perturbação}
  \end{phonetics}
\end{entry}

\begin{entry}{哇塞}{9,13}{⼝、⼟}
  \begin{phonetics}{哇塞}{wa1sai1}
    \definition{interj.}{(gíria) Uau!}
  \end{phonetics}
\end{entry}

\begin{entry}{哇噻}{9,16}{⼝、⼝}
  \begin{phonetics}{哇噻}{wa1sai1}
    \variantof{哇塞}
  \end{phonetics}
\end{entry}

\begin{entry}{哈马斯}{9,3,12}{⼝、⾺、⽄}
  \begin{phonetics}{哈马斯}{ha1ma3si1}
    \definition*{s.}{Hamas (Grupo Palestino)}
  \end{phonetics}
\end{entry}

\begin{entry}{哈哈}{9,9}{⼝、⼝}
  \begin{phonetics}{哈哈}{ha1 ha1}[][HSK 3]
    \definition{expr.}{(onomatopéia)  ha ha; o som de uma risada alta}
  \end{phonetics}
\end{entry}

\begin{entry}{响}{9}{⼝}
  \begin{phonetics}{响}{xiang3}[][HSK 2]
    \definition{adj.}{barulhento}
    \definition[声,阵]{s.}{som | barulho | eco}
    \definition{v.}{fazer um som | soar | tocar}
  \end{phonetics}
\end{entry}

\begin{entry}{哪}{9}{⼝}
  \begin{phonetics}{哪}{na3}[][HSK 1,4]
    \definition{adv.}{para expressar uma pergunta retórica}
    \definition{pron.}{qual?; o que? | qualquer; ser usado em um sentido geral}
  \end{phonetics}
  \begin{phonetics}{哪}{na5}
    \definition{part.}{usado depois de uma palavra com a terminação -n, é equivalente a ``啊''}
  \seealsoref{啊}{a5}
  \end{phonetics}
  \begin{phonetics}{哪}{nei3}
    \definition{part.}{qual? (interrogativo, seguido de classificador ou numeral-classificador)}
  \end{phonetics}
\end{entry}

\begin{entry}{哪儿}{9,2}{⼝、⼉}
  \begin{phonetics}{哪儿}{na3r5}[][HSK 1]
    \definition{adv.}{onde?}
  \end{phonetics}
\end{entry}

\begin{entry}{哪里}{9,7}{⼝、⾥}
  \begin{phonetics}{哪里}{na3 li3}[][HSK 1]
    \definition{adv.}{onde?}
  \end{phonetics}
\end{entry}

\begin{entry}{哪些}{9,8}{⼝、⼆}
  \begin{phonetics}{哪些}{na3xie1}[][HSK 1]
    \definition{pron.}{quais?}
  \end{phonetics}
\end{entry}

\begin{entry}{哪国人}{9,8,2}{⼝、⼞、⼈}
  \begin{phonetics}{哪国人}{na3 guo2ren2}
    \definition{expr.}{de qual país?}
  \end{phonetics}
\end{entry}

\begin{entry}{哪怕}{9,8}{⼝、⼼}
  \begin{phonetics}{哪怕}{na3pa4}[][HSK 4]
    \definition{conj.}{mesmo; mesmo se; mesmo que; não importa o quão}
  \end{phonetics}
\end{entry}

\begin{entry}{哥}{10}{⼝}
  \begin{phonetics}{哥}{ge1}[][HSK 1]
    \definition{s.}{irmão mais velho}
    \seeref{哥哥}{ge1 ge5}
  \end{phonetics}
\end{entry}

\begin{entry}{哥们}{10,5}{⼝、⼈}
  \begin{phonetics}{哥们}{ge1men5}
    \definition{expr.}{\emph{Brothers!}}
    \definition{s.}{(coloquial) cara | irmão (forma diminuta de tratamento entre homens)}
  \end{phonetics}
\end{entry}

\begin{entry}{哥哥}{10,10}{⼝、⼝}
  \begin{phonetics}{哥哥}{ge1 ge5}[][HSK 1]
    \definition[个,位]{s.}{irmão mais velho}
  \end{phonetics}
\end{entry}

\begin{entry}{哥斯拉}{10,12,8}{⼝、⽄、⼿}
  \begin{phonetics}{哥斯拉}{ge1si1la1}
    \definition*{s.}{Godzilla}
  \seealsoref{酷斯拉}{ku4si1la1}
  \end{phonetics}
\end{entry}

\begin{entry}{哦}{10}{⼝}
  \begin{phonetics}{哦}{e2}
    \definition{v.}{entoar cântico}
  \end{phonetics}
  \begin{phonetics}{哦}{o2}
    \definition{interj.}{Oh! (indicando dúvida ou surpresa)}
  \end{phonetics}
  \begin{phonetics}{哦}{o4}
    \definition{interj.}{Oh! (indicando que acabou de aprender algo)}
  \end{phonetics}
  \begin{phonetics}{哦}{o5}
    \definition{part.}{final da frase que transmite informalidade, calor, simpatia ou intimidade; também pode indicar que alguém está declarando um fato de que a outra pessoa não está ciente}
  \end{phonetics}
\end{entry}

\begin{entry}{哭}{10}{⼝}
  \begin{phonetics}{哭}{ku1}[][HSK 2]
    \definition{v.}{chorar}
  \end{phonetics}
\end{entry}

\begin{entry}{哭墙}{10,14}{⼝、⼟}
  \begin{phonetics}{哭墙}{ku1qiang2}
    \definition*{s.}{Muro das Lamentações (Jerusalém)}
  \end{phonetics}
\end{entry}

\begin{entry}{哮喘}{10,12}{⼝、⼝}
  \begin{phonetics}{哮喘}{xiao4chuan3}
    \definition{s.}{asma}
  \end{phonetics}
\end{entry}

\begin{entry}{哲理}{10,11}{⼝、⽟}
  \begin{phonetics}{哲理}{zhe2li3}
    \definition{s.}{filosofia | teoria filosófica}
  \end{phonetics}
\end{entry}

\begin{entry}{唇}{10}{⼝}
  \begin{phonetics}{唇}{chun2}
    \definition{s.}{lábios}
  \end{phonetics}
\end{entry}

\begin{entry}{唐人街}{10,2,12}{⼝、⼈、⾏}
  \begin{phonetics}{唐人街}{tang2ren2 jie1}
    \definition*{s.}{Bairro Chinês | \emph{Chinatown}}
  \seealsoref{中国城}{zhong1guo2cheng2}
  \end{phonetics}
\end{entry}

\begin{entry}{啊}{10}{⼝}
  \begin{phonetics}{啊}{a1}[][HSK 2]
    \definition{interj.}{Ah! | Oh! | interjeição de surpresa}
  \end{phonetics}
  \begin{phonetics}{啊}{a2}[][HSK 2]
    \definition{interj.}{Eh? | Que? | interjeição expressando dúvida ou exigindo resposta}
  \end{phonetics}
  \begin{phonetics}{啊}{a3}[][HSK 2]
    \definition{interj.}{Eh? | Meu! | E aí? | Que? | interjeição de surpresa ou dúvida}
  \end{phonetics}
  \begin{phonetics}{啊}{a4}[][HSK 2]
    \definition{interj.}{Ah! | OK! | Oh, é você! | Hum! | expressão de reconhecimento | interjeição de acordo}
  \end{phonetics}
  \begin{phonetics}{啊}{a5}[][HSK 2,4]
    \definition{adv.}{assim por diante}
    \definition{part.}{no final de sentença para expressar admiração | no final de sentença mostrando afirmação, aprovação, urgência, aconselhamento, etc. | no final de sentença para indicar uma pergunta | para pausar ligeiramente uma frase, chamando a atenção para as palavras seguintes | após cada um dos itens listados}
  \end{phonetics}
\end{entry}

\begin{entry}{啊呀}{10,7}{⼝、⼝}
  \begin{phonetics}{啊呀}{a1ya1}
    \definition{interj.}{Oh meu Deus! | interjeição de surpresa}
  \end{phonetics}
\end{entry}

\begin{entry}{啊哟}{10,9}{⼝、⼝}
  \begin{phonetics}{啊哟}{a1yo5}
    \definition{interj.}{Meu Deus! | Oh! | Ai! | interjeição de surpresa ou dor}
  \end{phonetics}
\end{entry}

\begin{entry}{唬}{11}{⼝}
  \begin{phonetics}{唬}{hu3}
    \definition{v.}{blefar, exagerar para assustar ou confundir}
  \end{phonetics}
\end{entry}

\begin{entry}{售货员}{11,8,7}{⼝、⾙、⼝}
  \begin{phonetics}{售货员}{shou4huo4yuan2}[][HSK 4]
    \definition[个]{s.}{vendedor; balconista; assistente de loja; equipe que vende produtos em lojas}
  \end{phonetics}
\end{entry}

\begin{entry}{唯一}{11,1}{⼝、⼀}
  \begin{phonetics}{唯一}{wei2yi1}[][HSK 5]
    \definition{adj.}{único; exclusivo; singular}
  \end{phonetics}
\end{entry}

\begin{entry}{唱}{11}{⼝}
  \begin{phonetics}{唱}{chang4}[][HSK 1]
    \definition{v.}{cantar}
  \end{phonetics}
\end{entry}

\begin{entry}{唱片}{11,4}{⼝、⽚}
  \begin{phonetics}{唱片}{chang4 pian4}[][HSK 4]
    \definition[枚,张]{s.}{disco; disco feito de goma-laca, plástico, etc. com ranhuras em espiral na superfície para registrar alterações no som que podem reproduzir o som gravado em um fonógrafo}
  \end{phonetics}
\end{entry}

\begin{entry}{唱歌}{11,14}{⼝、⽋}
  \begin{phonetics}{唱歌}{chang4 ge1}[][HSK 1]
    \definition{v.+compl.}{cantar}
  \end{phonetics}
\end{entry}

\begin{entry}{唾骂}{11,9}{⼝、⾺}
  \begin{phonetics}{唾骂}{tuo4ma4}
    \definition{v.}{insultar | amaldiçoar}
  \end{phonetics}
\end{entry}

\begin{entry}{商人}{11,2}{⼝、⼈}
  \begin{phonetics}{商人}{shang1 ren2}[][HSK 2]
    \definition[位,名]{s.}{comerciante | mercador | homem de negócios}
  \end{phonetics}
\end{entry}

\begin{entry}{商业}{11,5}{⼝、⼀}
  \begin{phonetics}{商业}{shang1ye4}[][HSK 3]
    \definition[个]{s.}{barganha; negócio; comércio}
  \end{phonetics}
\end{entry}

\begin{entry}{商务}{11,5}{⼝、⼒}
  \begin{phonetics}{商务}{shang1wu4}[][HSK 4]
    \definition[种,类,项]{s.}{negócios; assuntos de negócios; assuntos comerciais}
  \end{phonetics}
\end{entry}

\begin{entry}{商场}{11,6}{⼝、⼟}
  \begin{phonetics}{商场}{shang1chang3}[][HSK 1]
    \definition[家]{s.}{mercado | shopping | loja de departamentos | o mundo dos negócios}
  \end{phonetics}
\end{entry}

\begin{entry}{商店}{11,8}{⼝、⼴}
  \begin{phonetics}{商店}{shang1dian4}[][HSK 1]
    \definition[家,个]{s.}{loja}
  \end{phonetics}
\end{entry}

\begin{entry}{商品}{11,9}{⼝、⼝}
  \begin{phonetics}{商品}{shang1pin3}[][HSK 3]
    \definition[种,个,件,批]{s.}{bens; mercadoria; \emph{merchandising}}
  \end{phonetics}
\end{entry}

\begin{entry}{商标}{11,9}{⼝、⽊}
  \begin{phonetics}{商标}{shang1biao1}[][HSK 5]
    \definition[个]{s.}{marca; marca registrada; \emph{trademark}; marca ou símbolo (desenho, padrão, texto, etc.) gravado ou impresso na superfície ou embalagem de um produto, para diferenciá-lo de outros produtos semelhantes}
  \end{phonetics}
\end{entry}

\begin{entry}{商贸}{11,9}{⼝、⾙}
  \begin{phonetics}{商贸}{shang1mao4}
    \definition{s.}{comércio}
  \end{phonetics}
\end{entry}

\begin{entry}{商量}{11,12}{⼝、⾥}
  \begin{phonetics}{商量}{shang1liang5}[][HSK 2]
    \definition{v.}{consultar | discutir | falar sobre}
  \end{phonetics}
\end{entry}

\begin{entry}{啤酒}{11,10}{⼝、⾣}
  \begin{phonetics}{啤酒}{pi2jiu3}[][HSK 3]
    \definition[杯,瓶,罐,桶,缸]{s.}{(empréstimo linguístico) cerveja}
  \end{phonetics}
\end{entry}

\begin{entry}{啤酒馆}{11,10,11}{⼝、⾣、⾷}
  \begin{phonetics}{啤酒馆}{pi2jiu3guan3}
    \definition{s.}{cervejaria}
  \end{phonetics}
\end{entry}

\begin{entry}{啥}{11}{⼝}
  \begin{phonetics}{啥}{sha2}
    \definition{adv.}{Equivalente a 什么 (dialeto)}
  \end{phonetics}
\end{entry}

\begin{entry}{啵}{11}{⼝}
  \begin{phonetics}{啵}{bo1}
    \definition{s.}{(onomatopéia) borbulhar}
  \end{phonetics}
  \begin{phonetics}{啵}{bo5}
    \definition{part.}{partícula gramaticalmente equivalente a 吧}
  \seealsoref{吧}{ba5}
  \end{phonetics}
\end{entry}

\begin{entry}{喂}{12}{⼝}
  \begin{phonetics}{喂}{wei4}[][HSK 2,4]
    \definition{interj.}{Ei!, Olá!, para chamar atenção | Alô? (quando respondendo uma chamada telefônica, pronuncia-se como \dpy{wei2})}
    \definition{v.}{criar; alimentar (animais); dar comida a um animal |
alimentar (pessoas); colocar alimentos, medicamentos, etc. na boca de alguém}
  \end{phonetics}
\end{entry}

\begin{entry}{喂奶}{12,5}{⼝、⼥}
  \begin{phonetics}{喂奶}{wei4nai3}
    \definition{v.}{amamentar}
  \end{phonetics}
\end{entry}

\begin{entry}{喂母乳}{12,5,8}{⼝、⽏、⼄}
  \begin{phonetics}{喂母乳}{wei4mu3ru3}
    \definition{s.}{amamentação}
  \end{phonetics}
\end{entry}

\begin{entry}{喂养}{12,9}{⼝、⼋}
  \begin{phonetics}{喂养}{wei4yang3}
    \definition{v.}{alimentar (uma criança, animal doméstico, etc.) | manter | criar (um animal)}
  \end{phonetics}
\end{entry}

\begin{entry}{喂食}{12,9}{⼝、⾷}
  \begin{phonetics}{喂食}{wei4shi2}
    \definition{v.}{alimentar}
  \end{phonetics}
\end{entry}

\begin{entry}{喂哺}{12,10}{⼝、⼝}
  \begin{phonetics}{喂哺}{wei4bu3}
    \definition{v.}{alimentar (um bebê)}
  \end{phonetics}
\end{entry}

\begin{entry}{喂料}{12,10}{⼝、⽃}
  \begin{phonetics}{喂料}{wei4liao4}
    \definition{v.}{alimentar (também no sentido figurativo)}
  \end{phonetics}
\end{entry}

\begin{entry}{善于}{12,3}{⼝、⼆}
  \begin{phonetics}{善于}{shan4yu2}[][HSK 4]
    \definition{adv./v.}{ser bom em; ser hábil em}
  \end{phonetics}
\end{entry}

\begin{entry}{善良}{12,7}{⼝、⾉}
  \begin{phonetics}{善良}{shan4liang2}[][HSK 4]
    \definition{adj.}{de bom coração; bom e honesto; de bom coração e cheio de boa vontade}
  \end{phonetics}
\end{entry}

\begin{entry}{善意}{12,13}{⼝、⼼}
  \begin{phonetics}{善意}{shan4yi4}
    \definition{s.}{boa vontade | benevolência | bondade}
  \end{phonetics}
\end{entry}

\begin{entry}{喊}{12}{⼝}
  \begin{phonetics}{喊}{han3}[][HSK 2]
    \definition{clas.}{gritar | berrar | chamar (uma pessoa)}
  \end{phonetics}
\end{entry}

\begin{entry}{喔}{12}{⼝}
  \begin{phonetics}{喔}{o1}
    \definition{interj.}{Oh!, Entendi!, usado para indicar realização, compreensão}
  \end{phonetics}
\end{entry}

\begin{entry}{喜欢}{12,6}{⼝、⽋}
  \begin{phonetics}{喜欢}{xi3huan5}[][HSK 1]
    \definition{v.}{gostar}
  \end{phonetics}
\end{entry}

\begin{entry}{喜剧}{12,10}{⼝、⼑}
  \begin{phonetics}{喜剧}{xi3 ju4}[][HSK 5]
    \definition[部,出]{s.}{comédia (oposto de ``悲剧'') | comédia; uma das principais categorias do teatro; usa o exagero para satirizar e ridicularizar o feio; fenômenos retrógrados; destaca as contradições inerentes a esses fenômenos e seu conflito com coisas saudáveis; costuma provocar risadas; o final geralmente é feliz}
  \seealsoref{悲剧}{bei1 ju4}
  \end{phonetics}
\end{entry}

\begin{entry}{喜爱}{12,10}{⼝、⽖}
  \begin{phonetics}{喜爱}{xi3 ai4}[][HSK 4]
    \definition{v.}{gostar; amar; ter afeição por; estar interessado em; ter uma queda ou sentir interesse por pessoas ou coisas}
  \end{phonetics}
\end{entry}

\begin{entry}{喝}{12}{⼝}
  \begin{phonetics}{喝}{he1}[][HSK 1]
    \definition{interj.}{Meu Deus!}
    \definition{v.}{beber}
  \end{phonetics}
  \begin{phonetics}{喝}{he4}
    \definition{v.}{gritar bem alto}
  \end{phonetics}
\end{entry}

\begin{entry}{喝彩}{12,11}{⼝、⼺}
  \begin{phonetics}{喝彩}{he4cai3}
    \definition{s.}{aclamar | torcer}
  \end{phonetics}
\end{entry}

\begin{entry}{喝醉}{12,15}{⼝、⾣}
  \begin{phonetics}{喝醉}{he1zui4}
    \definition{v.}{ficar bêbado}
  \end{phonetics}
\end{entry}

\begin{entry}{喷}{12}{⼝}
  \begin{phonetics}{喷}{pen1}[][HSK 5]
    \definition{v.}{jorrar; esguichar; expelir sob pressão |
borrifar; espalhar; pulverizar}
  \end{phonetics}
  \begin{phonetics}{喷}{pen4}
    \definition{s.}{na época; tempo no mercado; época em que frutas, peixes e camarões são comercializados em grande quantidade | colheita; número de vezes que floresceu e frutificou; número de vezes que foi colhido na maturação}
  \end{phonetics}
\end{entry}

\begin{entry}{喻}{12}{⼝}
  \begin{phonetics}{喻}{yu4}
    \definition{s.}{analogia | símile | metáfora | alegoria}
    \definition{v.}{descrever algo como}
  \end{phonetics}
\end{entry}

\begin{entry}{嗄}{13}{⼝}
  \begin{phonetics}{嗄}{a2}
    \definition{adj.}{rouco}
    \variantof{啊}
  \end{phonetics}
\end{entry}

\begin{entry}{嗅}{13}{⼝}
  \begin{phonetics}{嗅}{xiu4}
    \definition{v.}{cheirar; farejar; identificar odores pelo nariz}
  \end{phonetics}
\end{entry}

\begin{entry}{嗡嗡}{13,13}{⼝、⼝}
  \begin{phonetics}{嗡嗡}{weng1weng1}
    \definition{s.}{zumbido}
    \definition{v.}{zumbir}
  \end{phonetics}
\end{entry}

\begin{entry}{嘟}{13}{⼝}
  \begin{phonetics}{嘟}{du1}
    \definition{s.}{buzina | bip}
    \definition{v.}{fazer beicinho}
  \end{phonetics}
\end{entry}

\begin{entry}{嘉年华}{14,6,6}{⼝、⼲、⼗}
  \begin{phonetics}{嘉年华}{jia1nian2hua2}
    \definition{s.}{(empréstimo linguístico) carnaval}
  \end{phonetics}
\end{entry}

\begin{entry}{嘱}{15}{⼝}
  \begin{phonetics}{嘱}{zhu3}
    \definition{v.}{juntar-se | implorar | incitar}
  \end{phonetics}
\end{entry}

\begin{entry}{嘱托}{15,6}{⼝、⼿}
  \begin{phonetics}{嘱托}{zhu3tuo1}
    \definition{v.}{confiar uma tarefa a alguém}
  \end{phonetics}
\end{entry}

\begin{entry}{嘱咐}{15,8}{⼝、⼝}
  \begin{phonetics}{嘱咐}{zhu3fu5}
    \definition{v.}{ordenar | dizer | exortar}
  \end{phonetics}
\end{entry}

\begin{entry}{噎}{15}{⼝}
  \begin{phonetics}{噎}{ye1}
    \definition{v.}{engasgar | sufocar}
  \end{phonetics}
\end{entry}

\begin{entry}{嘴}{16}{⼝}
  \begin{phonetics}{嘴}{zui3}[][HSK 2]
    \definition[张]{s.}{boca | qualquer coisa com formato ou função semelhante a uma boca}
    \definition{v.}{falar}
  \end{phonetics}
\end{entry}

\begin{entry}{嘴巴}{16,4}{⼝、⼰}
  \begin{phonetics}{嘴巴}{zui3 ba5}[][HSK 4]
    \definition[张]{s.}{boca}
  \end{phonetics}
\end{entry}

\begin{entry}{嘴巴子}{16,4,3}{⼝、⼰、⼦}
  \begin{phonetics}{嘴巴子}{zui3ba5zi5}
    \definition{s.}{tapa | bofetada}
  \end{phonetics}
\end{entry}

\begin{entry}{器}{16}{⼝}
  \begin{phonetics}{器}{qi4}
    \definition[台]{s.}{dispositivo | ferramenta | utensílio}
  \end{phonetics}
\end{entry}

\begin{entry}{器官}{16,8}{⼝、⼧}
  \begin{phonetics}{器官}{qi4guan1}[][HSK 4]
    \definition[个]{s.}{órgão; aparelho; parte de um organismo que consiste em vários tipos de tecidos celulares que podem desempenhar uma função fisiológica separada}
  \end{phonetics}
\end{entry}

\begin{entry}{嚣张}{18,7}{⼝、⼸}
  \begin{phonetics}{嚣张}{xiao1zhang1}
    \definition{adj.}{desenfreado | arrogante | agressivo}
  \end{phonetics}
\end{entry}

%%%%% EOF %%%%%

