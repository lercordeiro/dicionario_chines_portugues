%%%
%%% Radical "⼝"
%%%

\section*{Radical 30: ``⼝''}\addcontentsline{toc}{section}{Radical 30: ⼝}

\begin{Entry}{口}{3}{⼝}[Kangxi 30]
  \begin{Phonetics}{口}{kou3}[][HSK 1]
    \definition*{s.}{Sobrenome Kou}
    \definition{clas.}{usado para coisas com bocas (pessoas, animais domésticos, canhões, etc.) | usado para mordidas ou bocados | usado para idiomas}
    \definition{s.}{boca | borda; boca; o espaço externo ao recipiente | saída; entrada; local de entrada e saída | o gosto de alguém | corte; buraco; ferida |  a borda de uma faca; lâminas de facas, espadas, tesouras, etc. | a idade de um animal de tração | seção; departamento; sistema integrado de departamentos relacionados | conversa, discurso; pronunciamento; referência à fala | um portão da Grande Muralha (frequentemente usado em nomes de lugares)}
  \end{Phonetics}
\end{Entry}

\begin{Entry}{口号}{3,5}{⼝、⼝}
  \begin{Phonetics}{口号}{kou3 hao4}[][HSK 5]
    \definition[个,条,些]{s.}{\emph{slogan}; palavra de ordem; lema}
  \end{Phonetics}
\end{Entry}

\begin{Entry}{口吃}{3,6}{⼝、⼝}
  \begin{Phonetics}{口吃}{kou3chi1}
    \definition{s.}{gagueira; espasmofemia; balbucinato; mogilalia; battarismo; battarismo; iscnofonia; pselismo; o fenômeno de repetir palavras ou interromper frases ao falar é um defeito habitual de linguagem comumente conhecido como gagueira}
  \end{Phonetics}
\end{Entry}

\begin{Entry}{口吃病}{3,6,10}{⼝、⼝、⽧}
  \begin{Phonetics}{口吃病}{kou3chi1 bing4}
    \definition{s.}{doença da gagueira}
  \end{Phonetics}
\end{Entry}

\begin{Entry}{口试}{3,8}{⼝、⾔}
  \begin{Phonetics}{口试}{kou3 shi4}[][HSK 6]
    \definition{s.}{exame oral (ou teste); um tipo de exame que exige que os candidatos respondam a perguntas oralmente (em oposição a 笔试)}
    \definition{v.}{examinar oralmente}
  \seealsoref{笔试}{bi3 shi4}
  \end{Phonetics}
\end{Entry}

\begin{Entry}{口语}{3,9}{⼝、⾔}
  \begin{Phonetics}{口语}{kou3 yu3}[][HSK 4]
    \definition[门]{s.}{linguagem oral; linguagem falada; linguagem coloquial; linguagem usada em conversas}
  \end{Phonetics}
\end{Entry}

\begin{Entry}{口音}{3,9}{⼝、⾳}
  \begin{Phonetics}{口音}{kou3yin1}
    \definition{s.}{sons da fala oral (linguística)}
  \end{Phonetics}
  \begin{Phonetics}{口音}{kou3yin5}
    \definition{s.}{sotaque | voz}
  \end{Phonetics}
\end{Entry}

\begin{Entry}{口香糖}{3,9,16}{⼝、⾹、⽶}
  \begin{Phonetics}{口香糖}{kou3xiang1tang2}
    \definition{s.}{goma de mascar | chiclete}
  \end{Phonetics}
\end{Entry}

\begin{Entry}{口袋}{3,11}{⼝、⾐}
  \begin{Phonetics}{口袋}{kou3dai4}[][HSK 4]
    \definition[个,只]{s.}{bolso | saco; sacola; artigos de tecido ou couro}
  \end{Phonetics}
\end{Entry}

\begin{Entry}{口袋妖怪}{3,11,7,8}{⼝、⾐、⼥、⼼}
  \begin{Phonetics}{口袋妖怪}{kou3dai4 yao1guai4}
    \definition*{s.}{Pokémon (franquia de mídia japonesa)}
  \end{Phonetics}
\end{Entry}

\begin{Entry}{古}{5}{⼝}
  \begin{Phonetics}{古}{gu3}[][HSK 3]
    \definition*{s.}{Cuba, abreviação de 古巴 | Sobrenome Gu}
    \definition{adj.}{antigo; milenar; ancestral; secular | simples e sincero | velho | arcaico}
    \definition{pref.}{(distante no tempo; antigo; primitivo) paleo-; arqueo-}
    \definition{s.}{tempos antigos (oposto a 今) | antiguidade; ancestralidade | livros ou ortodoxias dos sábios antigos, a tradição do Tao | uma forma de poesia pré-Tang}
  \seealsoref{古巴}{gu3ba1}
  \seealsoref{今}{jin1}
  \end{Phonetics}
\end{Entry}

\begin{Entry}{古人}{5,2}{⼝、⼈}
  \begin{Phonetics}{古人}{gu3ren2}[][HSK 7-9]
    \definition{s.}{os antigos; antepassados ​​(em oposição a 今人) | pessoas dos tempos antigos | espécies humanas extintas, como \emph{Homo erectus} ou \emph{Homo neanderthalensis} | Lliterário: pessoa falecida}
  \seealsoref{今人}{jin1ren2}
  \end{Phonetics}
\end{Entry}

\begin{Entry}{古今中外}{5,4,4,5}{⼝、⼈、⼁、⼣}
  \begin{Phonetics}{古今中外}{gu3jin1-zhong1wai4}[][HSK 7-9]
    \definition{expr.}{``Antigo e moderno, chinês e estrangeiro.''; em todos os tempos e em todas as terras}
  \end{Phonetics}
\end{Entry}

\begin{Entry}{古巴}{5,4}{⼝、⼰}
  \begin{Phonetics}{古巴}{gu3ba1}
    \definition*{s.}{Cuba}
  \end{Phonetics}
\end{Entry}

\begin{Entry}{古代}{5,5}{⼝、⼈}
  \begin{Phonetics}{古代}{gu3dai4}[][HSK 3]
    \definition{s.}{tempos antigos; o passado é um período muito distante do presente (diferentemente de 近代 e 现代); na periodização histórica chinesa, geralmente se refere ao período anterior a meados do século XIX | sociedade antiga; sociedade primitiva; refere-se especificamente à era da sociedade escravista (às vezes também inclui a era comunal primitiva) |antigamente; tempos antigos; no passado}
  \seealsoref{近代}{jin4dai4}
  \seealsoref{现代}{xian4dai4}
  \end{Phonetics}
\end{Entry}

\begin{Entry}{古朴}{5,6}{⼝、⽊}
  \begin{Phonetics}{古朴}{gu3pu3}[][HSK 7-9]
    \definition{adj.}{simples e pouco sofisticado (arte, arquitetura, etc.); descreve a aparência sem muita decoração ou modificação, dando às pessoas uma sensação antiga, e também descreve o comportamento das pessoas como simples e sincero}
  \end{Phonetics}
\end{Entry}

\begin{Entry}{古老}{5,6}{⼝、⽼}
  \begin{Phonetics}{古老}{gu3 lao3}[][HSK 5]
    \definition{adj.}{antigo; antiquado; histórico}
  \end{Phonetics}
\end{Entry}

\begin{Entry}{古典}{5,8}{⼝、⼋}
  \begin{Phonetics}{古典}{gu3dian3}[][HSK 6]
    \definition{adj.}{clássico; descreve uma obra ou coisa como tendo características tradicionais ou exemplares}
    \definition{s.}{os clássicos}
  \end{Phonetics}
\end{Entry}

\begin{Entry}{古怪}{5,8}{⼝、⼼}
  \begin{Phonetics}{古怪}{gu3guai4}[][HSK 7-9]
    \definition{adj.}{pitoresco; excêntrico; esquisito; estranho; muito diferente do habitual, surpreendente; desconhecido e raro; raro e inovador}
  \end{Phonetics}
\end{Entry}

\begin{Entry}{古城}{5,9}{⼝、⼟}
  \begin{Phonetics}{古城}{gu3cheng2}
    \definition{s.}{cidade antiga}
  \end{Phonetics}
\end{Entry}

\begin{Entry}{古迹}{5,9}{⼝、⾡}
  \begin{Phonetics}{古迹}{gu3ji4}[][HSK 7-9]
    \definition[处]{s.}{sítio histórico; local de interesse histórico; construções antigas ou outras relíquias de grande importância}
  \end{Phonetics}
\end{Entry}

\begin{Entry}{古铜色}{5,11,6}{⼝、⾦、⾊}
  \begin{Phonetics}{古铜色}{gu3tong2 se4}
    \definition{s.}{cor bronze}
  \end{Phonetics}
\end{Entry}

\begin{Entry}{古董}{5,12}{⼝、⾋}
  \begin{Phonetics}{古董}{gu3dong3}[][HSK 7-9]
    \definition{adj.}{antiquado; uma metáfora para coisas ultrapassadas ou pessoas teimosas}
    \definition[件,个]{s.}{objeto de arte; raridade; antiguidade; artefatos transmitidos desde os tempos antigos podem ser usados ​​como referência para a compreensão da cultura antiga}
  \end{Phonetics}
\end{Entry}

\begin{Entry}{古装}{5,12}{⼝、⾐}
  \begin{Phonetics}{古装}{gu3 zhuang1}
    \definition[套]{s.}{traje antigo; roupas tradicionais; roupas de estilo antigo}
  \end{Phonetics}
\end{Entry}

\begin{Entry}{句}{5}{⼝}
  \begin{Phonetics}{句}{gou4}
    \variantof{勾}
  \end{Phonetics}
  \begin{Phonetics}{句}{ju4}[][HSK 2]
    \definition{clas.}{para sentenças, frases ou linhas de versos}
    \definition{s.}{frase; sentença}
  \end{Phonetics}
\end{Entry}

\begin{Entry}{句子}{5,3}{⼝、⼦}
  \begin{Phonetics}{句子}{ju4zi5}[][HSK 2]
    \definition[个,句]{s.}{sentença; uma unidade linguística composta por palavras ou frases que expressa um significado completo}
  \end{Phonetics}
\end{Entry}

\begin{Entry}{另}{5}{⼝}
  \begin{Phonetics}{另}{ling4}[][HSK 6]
    \definition*{s.}{Sobrenome Ling}
    \definition{adv.}{além disso; indica que está fora do escopo da declaração | no lugar de; em vez de}
    \definition{pron.}{(com substantivos) outro; diferente; refere-se a pessoas ou coisas fora do escopo do que é dito}
  \end{Phonetics}
\end{Entry}

\begin{Entry}{另一方面}{5,1,4,9}{⼝、⼀、⽅、⾯}
  \begin{Phonetics}{另一方面}{ling4 yi4 fang1 mian4}[][HSK 3]
    \definition{adv./conj.}{outro aspecto | por outro lado; por sua vez; em contrapartida}
  \end{Phonetics}
\end{Entry}

\begin{Entry}{另外}{5,5}{⼝、⼣}
  \begin{Phonetics}{另外}{ling4wai4}[][HSK 3]
    \definition{adv.}{além disso; em adição; ademais; além do mais; além de que; além do que já foi dito}
    \definition{conj.}{além disso; usada entre duas ou mais frases, indica algo além do que foi mencionado anteriormente}
    \definition{pron.}{outro; além das pessoas ou coisas mencionadas anteriormente}
  \end{Phonetics}
\end{Entry}

\begin{Entry}{只}{5}{⼝}
  \begin{Phonetics}{只}{zhi1}[][HSK 3]
    \definition{adj.}{solteiro; solitário; único; muito raro}
    \definition{clas.}{usado para um de um par | usado para animais pequenos (pássaros, gatos, cães, etc.) | usado para certos utensílios, aparelhos | usado para navios}
  \end{Phonetics}
  \begin{Phonetics}{只}{zhi3}[][HSK 2]
    \definition{adv.}{somente; apenas; meramente | simplesmente; usado para limitar o escopo, indicando que não há nada além disso, equivalente a 仅仅}
  \seealsoref{仅仅}{jin3 jin3}
  \end{Phonetics}
\end{Entry}

\begin{Entry}{只不过}{5,4,6}{⼝、⼀、⾡}
  \begin{Phonetics}{只不过}{zhi3 bu2 guo4}[][HSK 5]
    \definition{adv.}{somente; apenas; meramente; não mais do que}
  \end{Phonetics}
\end{Entry}

\begin{Entry}{只见}{5,4}{⼝、⾒}
  \begin{Phonetics}{只见}{zhi3 jian4}[][HSK 5]
    \definition{v.}{somente ver; ver; só vi, e de repente percebi uma certa situação}
  \end{Phonetics}
\end{Entry}

\begin{Entry}{只好}{5,6}{⼝、⼥}
  \begin{Phonetics}{只好}{zhi3hao3}[][HSK 3]
    \definition{v.}{ter que; ser forçado a; não ter escolha a não ser; significa que só pode ser assim, não há outra opção}
  \end{Phonetics}
\end{Entry}

\begin{Entry}{只有}{5,6}{⼝、⽉}
  \begin{Phonetics}{只有}{zhi3 you3}[][HSK 3]
    \definition{adv.}{somente; tem que; forçado a}
    \definition{conj.}{somente se; conecta frases, expressa condições necessárias, geralmente corresponde a 才 e 方}
  \seealsoref{才}{cai2}
  \seealsoref{方}{fang1}
  \end{Phonetics}
\end{Entry}

\begin{Entry}{只有……才……}{5,6,3}{⼝、⽉、⼿}
  \begin{Phonetics}{只有……才……}{zhi3you3 cai2}
    \definition{conj.}{só se\dots então\dots}
  \end{Phonetics}
\end{Entry}

\begin{Entry}{只身}{5,7}{⼝、⾝}
  \begin{Phonetics}{只身}{zhi1shen1}
    \definition{adv.}{sozinho | por si só}
  \end{Phonetics}
\end{Entry}

\begin{Entry}{只怕}{5,8}{⼝、⼼}
  \begin{Phonetics}{只怕}{zhi3pa4}
    \definition{adv.}{receio que\dots | talvez | muito provavelmente}
  \end{Phonetics}
\end{Entry}

\begin{Entry}{只是}{5,9}{⼝、⽇}
  \begin{Phonetics}{只是}{zhi3 shi4}[][HSK 3]
    \definition{adv.}{somente; meramente; apenas; expressa ênfase limitada a uma determinada situação ou âmbito}
    \definition{conj.}{somente; mas; exceto que; conecta frases, indicando uma ligeira transição, equivalente a 不过}
  \seealsoref{不过}{bu2guo4}
  \end{Phonetics}
\end{Entry}

\begin{Entry}{只要}{5,9}{⼝、⾑}
  \begin{Phonetics}{只要}{zhi3yao4}[][HSK 2]
    \definition{conj.}{desde que; se apenas; contanto que; indica condições necessárias (就 ou 可 são frequentemente usados depois)}
  \seealsoref{便}{bian4}
  \seealsoref{就}{jiu4}
  \end{Phonetics}
\end{Entry}

\begin{Entry}{只要……就……}{5,9,12}{⼝、⾑、⼪}
  \begin{Phonetics}{只要……就……}{zhi3yao4 jiu4}
    \definition{conj.}{contanto que/desde que/se somente\dots, então\dots}
  \end{Phonetics}
\end{Entry}

\begin{Entry}{只消}{5,10}{⼝、⽔}
  \begin{Phonetics}{只消}{zhi3xiao1}
    \definition{conj.}{desde que}
  \end{Phonetics}
\end{Entry}

\begin{Entry}{只能}{5,10}{⼝、⾁}
  \begin{Phonetics}{只能}{zhi3 neng2}[][HSK 2]
    \definition{adv.}{só pode; obrigado a fazer algo; isso significa que devido à limitação da capacidade pessoal ou às condições objetivas, não há outra escolha senão esta}
  \end{Phonetics}
\end{Entry}

\begin{Entry}{只读}{5,10}{⼝、⾔}
  \begin{Phonetics}{只读}{zhi3du2}
    \definition{s.}{somente leitura (computação) | \emph{read-only}}
  \end{Phonetics}
\end{Entry}

\begin{Entry}{只顾}{5,10}{⼝、⾴}
  \begin{Phonetics}{只顾}{zhi3 gu4}[][HSK 6]
    \definition{adv.}{meramente; simplesmente; apenas se importa com; indica que a atenção está focada em apenas um aspecto}
    \definition{v.}{considerar apenas uma coisa}
  \end{Phonetics}
\end{Entry}

\begin{Entry}{只得}{5,11}{⼝、⼻}
  \begin{Phonetics}{只得}{zhi3 de5}[][HSK 6]
    \definition{v.}{não ter alternativa senão; obrigado a; ter que; ser obrigado a}
  \end{Phonetics}
\end{Entry}

\begin{Entry}{只管}{5,14}{⼝、⽵}
  \begin{Phonetics}{只管}{zhi3 guan3}[][HSK 6]
    \definition{adv.}{por todos os meios; expressa incentivo para que os outros façam algo com confiança, sem se preocuparem com outras coisas | apenas; simplesmente; significa fazer uma coisa com seriedade, sem se preocupar com outras coisas}
  \end{Phonetics}
\end{Entry}

\begin{Entry}{叫}{5}{⼝}
  \begin{Phonetics}{叫}{jiao4}[][HSK 1,3]
    \definition{adj.}{macho (animal)}
    \definition{prep.}{usado em frases passivas; introduz o agente da ação; equivalente a 被 | combinado com 看, 说; usado para expressar suas ideias e pontos de vista}
    \definition{v.}{chorar; gritar; berrar | nomear; chamar | chamar; chamar a atenção | cumprimentar; saudar; dizer olá | pedir; ordenar; licitar | permitir; concordar com algo; concordar em fazer algo | contratar; encomendar; comprar o que você precisa}
  \seealsoref{被}{bei4}
  \seealsoref{看}{kan4}
  \seealsoref{说}{shuo1}
  \end{Phonetics}
\end{Entry}

\begin{Entry}{叫作}{5,7}{⼝、⼈}
  \begin{Phonetics}{叫作}{jiao4 zuo4}[][HSK 2]
    \definition{v.}{ser chamado de; ser conhecido como}
  \end{Phonetics}
\end{Entry}

\begin{Entry}{召}{5}{⼝}
  \begin{Phonetics}{召}{shao4}
    \definition*{s.}{Sobrenome Shao}
    \definition{s.}{(frequentemente em nomes de lugares mongóis) templo; mosteiro}
    \definition{v.}{convocar; intimar; invocar}
  \end{Phonetics}
  \begin{Phonetics}{召}{zhao4}
    \definition*{s.}{Sobrenome Zhao}
    \definition{s.}{templo}
    \definition{v.}{chamar; intimar; convocar; invocar}
  \end{Phonetics}
\end{Entry}

\begin{Entry}{召开}{5,4}{⼝、⼶}
  \begin{Phonetics}{召开}{zhao4kai1}[][HSK 4]
    \definition{v.}{convocar; chamar pessoas para uma reunião; realizar (uma reunião)}
  \end{Phonetics}
\end{Entry}

\begin{Entry}{叮}{5}{⼝}
  \begin{Phonetics}{叮}{ding1}
    \definition{v.}{picar; ferroar | dizer ou perguntar novamente para ter certeza; verificar; insistir; certificar-se | sondar; perseguir}
  \end{Phonetics}
\end{Entry}

\begin{Entry}{叮嘱}{5,15}{⼝、⼝}
  \begin{Phonetics}{叮嘱}{ding1zhu3}[][HSK 7-9]
    \definition{v.}{advertir; exortar; insistir repetidamente; instruir repetidamente; dizer à outra pessoa para lembrar o que deve e o que não deve ser feito}
  \end{Phonetics}
\end{Entry}

\begin{Entry}{可}{5}{⼝}
  \begin{Phonetics}{可}{ke3}[][HSK 5]
    \definition*{s.}{Sobrenome Ke}
    \definition{adv.}{indica ênfase | indica o fortalecimento de perguntas retóricas | indica um tom de questionamento mais forte | sobre; a respeito de}
    \definition{conj.}{mas; ainda}
    \definition{v.}{aprovar; concordar com | poder; permitir; ser capaz de | precisar (fazer); valer a pena (fazer); merecer | ajustar; adequar | estar pronto para; estar disposto a; pretender}
  \end{Phonetics}
  \begin{Phonetics}{可}{ke4}
    \definition{s.}{governante supremo de uma tribo nômade do norte; Khan (可汗), título do governante supremo dos antigos grupos étnicos xianbei, turco, uigur e mongol}
  \seealsoref{可汗}{ke4han2}
  \end{Phonetics}
\end{Entry}

\begin{Entry}{可口可乐}{5,3,5,5}{⼝、⼝、⼝、⼃}
  \begin{Phonetics}{可口可乐}{ke3kou3ke3le4}
    \definition*{s.}{Empréstimo linguístico: Coca-Cola}
  \end{Phonetics}
\end{Entry}

\begin{Entry}{可以}{5,4}{⼝、⼈}
  \begin{Phonetics}{可以}{ke3yi3}[][HSK 2]
    \definition{adj.}{aceitável; nada mal; muito bom | impressionante; espantoso; tremendo}
    \definition{v.}{poder; ter condições, capacidade e tempo para fazer algo ou ter alguma utilidade | permitir; poder | valer a pena fazer; considerar que vale a pena, recomendar fazer algo}
  \end{Phonetics}
\end{Entry}

\begin{Entry}{可见}{5,4}{⼝、⾒}
  \begin{Phonetics}{可见}{ke3jian4}[][HSK 4]
    \definition{adj.}{visível; concebível; algo que é óbvio ou evidente}
    \definition{conj.}{isso mostra; isto prova; é, portanto, claro (ou evidente, óbvio) que}
    \definition{v.}{ser ou estar visível ; ser ou estar claro}
  \end{Phonetics}
\end{Entry}

\begin{Entry}{可乐}{5,5}{⼝、⼃}
  \begin{Phonetics}{可乐}{ke3 le4}[][HSK 3]
    \definition*[罐,杯,瓶,听,口]{s.}{\emph{coke}; coca; coca-cola}
    \definition{adj.}{engraçado; divertido; risível}
  \end{Phonetics}
\end{Entry}

\begin{Entry}{可卡因}{5,5,6}{⼝、⼘、⼞}
  \begin{Phonetics}{可卡因}{ke3ka3yin1}
    \definition{s.}{(empréstimo linguístico) cocaína}
  \end{Phonetics}
\end{Entry}

\begin{Entry}{可汗}{5,6}{⼝、⽔}
  \begin{Phonetics}{可汗}{ke4han2}
    \definition{s.}{khan (empréstimo linguístico); cham}
  \end{Phonetics}
\end{Entry}

\begin{Entry}{可怕}{5,8}{⼝、⼼}
  \begin{Phonetics}{可怕}{ke3pa4}[][HSK 2]
    \definition{adj.}{assustador; terrível; hediondo; medonho; horrível; aterrorizante}
    \definition{adv.}{terrivelmente}
  \end{Phonetics}
\end{Entry}

\begin{Entry}{可怜}{5,8}{⼝、⼼}
  \begin{Phonetics}{可怜}{ke3lian2}[][HSK 5]
    \definition{adj.}{pobre; lamentável; lastimável | miserável (de quantidade ou qualidade); descreve um número pequeno ou um lugar tão pequeno que não vale a pena falar sobre ele}
    \definition{v.}{ter pena; ter piedade de; ter simpatia por pessoas que tiveram coisas muito ruins acontecendo com elas}
  \end{Phonetics}
\end{Entry}

\begin{Entry}{可是}{5,9}{⼝、⽇}
  \begin{Phonetics}{可是}{ke3shi4}[][HSK 2]
    \definition{adv.}{de fato (usado para dar ênfase), equivalente a 的确}
    \definition{conj.}{mas; no entanto; contudo; conecta frases, expressa uma relação de transição, equivalente a 但是}
  \seealsoref{但是}{dan4 shi4}
  \seealsoref{的确}{di2que4}
  \end{Phonetics}
\end{Entry}

\begin{Entry}{可爱}{5,10}{⼝、⽖}
  \begin{Phonetics}{可爱}{ke3'ai4}[][HSK 2]
    \definition{adj.}{adorável; simpático; encantador | bonitinho; adorável | amado; querido; encantador; cativante; relacionamento próximo, sentimentos profundos | fofo; bonito}
  \end{Phonetics}
\end{Entry}

\begin{Entry}{可能}{5,10}{⼝、⾁}
  \begin{Phonetics}{可能}{ke3neng2}[][HSK 2]
    \definition{adj.}{possível}
    \definition{adv.}{possivelmente}
    \definition[种]{s.}{possibilidade; tendências ou oportunidades que podem se tornar realidade}
  \end{Phonetics}
\end{Entry}

\begin{Entry}{可惜}{5,11}{⼝、⼼}
  \begin{Phonetics}{可惜}{ke3xi1}[][HSK 5]
    \definition{adj.}{é uma pena; é muito ruim; é lamentável}
    \definition{adv.}{infelizmente}
  \end{Phonetics}
\end{Entry}

\begin{Entry}{可编程}{5,12,12}{⼝、⽷、⽲}
  \begin{Phonetics}{可编程}{ke3bian1cheng2}
    \definition{adj.}{programável}
  \end{Phonetics}
\end{Entry}

\begin{Entry}{可靠}{5,15}{⼝、⾮}
  \begin{Phonetics}{可靠}{ke3kao4}[][HSK 3]
    \definition{adj.}{confiável; digno de confiança | verdadeiro; autêntico; descrever notícias e outras informações como verdadeiras, de modo que as pessoas possam acreditar nelas}
  \end{Phonetics}
\end{Entry}

\begin{Entry*}{可擦写可编程只读存储器}{5,17,5,5,12,12,5,10,6,12,16}{⼝、⼿、⼍、⼝、⽷、⽲、⼝、⾔、⼦、⼈、⼝}
  \begin{Phonetics}{可擦写可编程只读存储器}{ke3 ca1 xie3 ke3 bian1cheng2 zhi1 du2 cun2chu3qi4}
    \definition{s.}{EPROM (\emph{erasable programmable read-only memory})}
  \end{Phonetics}
\end{Entry*}

\begin{Entry}{台}{5}{⼝}
  \begin{Phonetics}{台}{tai2}[][HSK 3]
    \definition*{s.}{Sobrenome Tai}
    \definition{clas.}{usado para certas máquinas, aparelhos, instrumentos, etc | usado para uma performance completa, como drama, música e dança}
    \definition{s.}{torre | plataforma; palco | suporte; pedestal | qualquer coisa em forma de plataforma ou palco | mesa; escrivaninha | estação de transmissão; refere-se a estações de rádio | um serviço telefônico especial; refere-se à estação telefônica | ``seu'' (um termo respeitoso usado antigamente para se dirigir a alguém) | tufão}
  \end{Phonetics}
\end{Entry}

\begin{Entry}{台上}{5,3}{⼝、⼀}
  \begin{Phonetics}{台上}{tai2 shang4}[][HSK 4]
    \definition{s.}{no palco}
  \end{Phonetics}
\end{Entry}

\begin{Entry}{台下}{5,3}{⼝、⼀}
  \begin{Phonetics}{台下}{tai2xia4}
    \definition{s.}{platéia | fora do palco}
  \end{Phonetics}
\end{Entry}

\begin{Entry}{台风}{5,4}{⼝、⾵}
  \begin{Phonetics}{台风}{tai2feng1}[][HSK 5]
    \definition[场,阵,级]{s.}{tufão; classificação de um ciclone tropical ocorrido no oeste do Pacífico Norte | postura; presença de palco; comportamento ou estilo que os atores demonstram no palco}
  \end{Phonetics}
\end{Entry}

\begin{Entry}{台灯}{5,6}{⼝、⽕}
  \begin{Phonetics}{台灯}{tai2 deng1}[][HSK 6]
    \definition[个,盏]{s.}{luminária de mesa; luminária de leitura; uma luminária com base para uso sobre uma mesa}
  \end{Phonetics}
\end{Entry}

\begin{Entry}{台阶}{5,6}{⼝、⾩}
  \begin{Phonetics}{台阶}{tai2jie1}[][HSK 4]
    \definition[个,级]{s.}{escada; escadaria | passos; metáfora para uma maneira ou oportunidade de evitar constrangimentos causados ​​por um impasse | nova fase; novo nível; novo patamar; metáfora para novas conquistas ou novos patamares alcançados no estudo ou no trabalho}
  \end{Phonetics}
\end{Entry}

\begin{Entry}{右}{5}{⼝}
  \begin{Phonetics}{右}{you4}[][HSK 1]
    \definition*{s.}{Sobrenome You}
    \definition{adj.}{conservador; reacionário}
    \definition{s.}{a direita; o lado direito | oeste; na antiguidade, referia-se especificamente à direção oeste (com base na orientação para o sul) | o lado direito como o lado de precedência; posição ou nível mais elevado (os antigos costumavam considerar a direita como mais respeitável)}
    \definition{v.}{favorecer; apoiar; reverenciar}
  \end{Phonetics}
\end{Entry}

\begin{Entry}{右手}{5,4}{⼝、⼿}
  \begin{Phonetics}{右手}{you4shou3}
    \definition{s.}{mão direita | lado direito}
  \end{Phonetics}
\end{Entry}

\begin{Entry}{右边}{5,5}{⼝、⾡}
  \begin{Phonetics}{右边}{you4bian5}[][HSK 1]
    \definition{s.}{a direita; o lado direito; do lado direito}
  \end{Phonetics}
\end{Entry}

\begin{Entry}{右侧}{5,8}{⼝、⼈}
  \begin{Phonetics}{右侧}{you4ce4}
    \definition{s.}{lateral direita | lado direito}
  \end{Phonetics}
\end{Entry}

\begin{Entry}{右转}{5,8}{⼝、⾞}
  \begin{Phonetics}{右转}{you4zhuan3}
    \definition{v.}{virar à direita}
  \end{Phonetics}
\end{Entry}

\begin{Entry}{右面}{5,9}{⼝、⾯}
  \begin{Phonetics}{右面}{you4mian4}
    \definition{s.}{lado direito}
  \end{Phonetics}
\end{Entry}

\begin{Entry}{右倾}{5,10}{⼝、⼈}
  \begin{Phonetics}{右倾}{you4qing1}
    \definition{adj.}{conservador | reacionário}
  \end{Phonetics}
\end{Entry}

\begin{Entry}{右袒}{5,10}{⼝、⾐}
  \begin{Phonetics}{右袒}{you4tan3}
    \definition{v.}{ser tendencioso | ser parcial | favorecer um lado | tomar partido}
  \end{Phonetics}
\end{Entry}

\begin{Entry}{叶}{5}{⼝}
  \begin{Phonetics}{叶}{ye4}
    \definition*{s.}{Sobrenome Ye}
    \definition[枝]{s.}{folha; folhagem | coisa parecida com uma folha | página; folha | parte de um período histórico; segmentos de período mais longos | lóbulo; lóbulos do cérebro, pulmões e fígado}
  \end{Phonetics}
\end{Entry}

\begin{Entry}{叶子}{5,3}{⼝、⼦}
  \begin{Phonetics}{叶子}{ye4zi5}[][HSK 4]
    \definition[片]{s.}{folha; termo genérico para as folhas de uma planta}
  \end{Phonetics}
\end{Entry}

\begin{Entry}{号}{5}{⼝}
  \begin{Phonetics}{号}{hao2}
    \definition{v.}{uivar; gritar; gritar em voz alta e prolongada | lamentar; chorar alto | uivar; (vento) assobiar, assoviar}
  \end{Phonetics}
  \begin{Phonetics}{号}{hao4}[][HSK 1]
    \definition{clas.}{usado para o número de pessoas |  tipo; espécie; classificação | usado para pessoas ou negócios; número de vezes utilizado para transações}
    \definition[把]{s.}{nome | nome presumido; nome alternativo; pseudônimo; apelido | casa de negócios; loja | marca; sinal; sinalização | número | data | ordem; no exército, as ordens são transmitidas verbalmente ou por meio de clarins | qualquer instrumento de sopro e latão; trombeta usada no exército ou em bandas | qualquer coisa usada como buzina | chamada de corneta; qualquer chamada feita em uma corneta; usar um apito para emitir um som com um significado específico | pessoa em uma condição especial; pessoas que se encontram em uma situação especial}
    \definition{suf.}{sufixo de navio}
    \definition{v.}{marcar; fazer uma marca | sentir; colocar a mão no pulso do paciente e avaliar a situação através do fluxo sanguíneo}
  \end{Phonetics}
\end{Entry}

\begin{Entry}{号召}{5,5}{⼝、⼝}
  \begin{Phonetics}{号召}{hao4zhao4}[][HSK 5]
    \definition{s.}{chamado; apelo; desejo ou pedido solene (de um governo, partido político, organização etc.) para que as massas façam algo}
    \definition{v.}{chamar;  (governo, partido político, organização, etc.) fazer um pedido solene às massas para que façam algo, na esperança de que todos se esforcem para alcançá-lo}
  \end{Phonetics}
\end{Entry}

\begin{Entry}{号角}{5,7}{⼝、⾓}
  \begin{Phonetics}{号角}{hao4jiao3}
    \definition{s.}{corneta | trombeta}
  \end{Phonetics}
\end{Entry}

\begin{Entry}{号码}{5,8}{⼝、⽯}
  \begin{Phonetics}{号码}{hao4ma3}[][HSK 4]
    \definition[个,组,串]{s.}{número}
  \end{Phonetics}
\end{Entry}

\begin{Entry}{号称}{5,10}{⼝、⽲}
  \begin{Phonetics}{号称}{hao4cheng1}[][HSK 7-9]
    \definition{v.}{ser conhecido como; ser conhecido por um certo nome | afirmar ser; alegar}
  \end{Phonetics}
\end{Entry}

\begin{Entry}{司}{5}{⼝}
  \begin{Phonetics}{司}{si1}
    \definition*{s.}{Sobrenome Si}
    \definition{s.}{departamento (sob um ministério); um departamento dentro de uma agência de nível ministerial}
    \definition{v.}{assumir o comando de; atender; administrar; operar; gerenciar}
  \end{Phonetics}
\end{Entry}

\begin{Entry}{司长}{5,4}{⼝、⾧}
  \begin{Phonetics}{司长}{si1 zhang3}[][HSK 6]
    \definition[位,名]{s.}{diretor-geral | chefe de gabinete}
  \end{Phonetics}
\end{Entry}

\begin{Entry}{司机}{5,6}{⼝、⽊}
  \begin{Phonetics}{司机}{si1ji1}[][HSK 2]
    \definition[个,名,位]{s.}{motorista; motorista particular; chofer; motoristas de veículos de transporte público, como trens, ônibus e bondes}
  \end{Phonetics}
\end{Entry}

\begin{Entry}{叹}{5}{⼝}
  \begin{Phonetics}{叹}{tan4}
    \definition{v.}{suspirar | exclamar com admiração; aclamar; louvar |recitar com cadência; entoar cântico; entoar}
  \end{Phonetics}
\end{Entry}

\begin{Entry}{叹气}{5,4}{⼝、⽓}
  \begin{Phonetics}{叹气}{tan4qi4}[][HSK 6]
    \definition{v.}{suspirar; soltar um suspiro; soltar um longo suspiro e fazer um som devido à insatisfação ou desamparo}
  \end{Phonetics}
\end{Entry}

\begin{Entry}{叼}{5}{⼝}
  \begin{Phonetics}{叼}{diao1}[][HSK 7-9]
    \definition{v.}{segurar na boca; segurar com a boca}
  \end{Phonetics}
\end{Entry}

\begin{Entry}{吃}{6}{⼝}
  \begin{Phonetics}{吃}{chi1}[][HSK 1]
    \definition{s.}{alimentos; necessidades básicas}
    \definition{v.}{comer; pegar; fazer; colocar alimentos na boca, mastigar e engolir (incluindo sugar e beber) | viver; depender de algo para viver | aniquilar; eliminar (usado principalmente em jogos de guerra e jogos de tabuleiro) | esgotar; exaurir; ser um fardo; ser um esforço | absorver | sofrer; incorrer | entender; compreender | entrar um objeto em outro | expressar aceitação psicológica | fazer suas refeições; comer}
  \end{Phonetics}
\end{Entry}

\begin{Entry}{吃力}{6,2}{⼝、⼒}
  \begin{Phonetics}{吃力}{chi1li4}[][HSK 5]
    \definition{adj.}{suado; extenuante; trabalhoso; laborioso | cansado; fatigado}
  \end{Phonetics}
\end{Entry}

\begin{Entry}{吃亏}{6,3}{⼝、⼆}
  \begin{Phonetics}{吃亏}{chi1/kui1}[][HSK 7-9]
    \definition{adv.}{em desvantagem; em situação desfavorável}
    \definition{v.+compl.}{sofrer perdas; sofrer aflição; levar a pior; levar uma surra}
  \end{Phonetics}
\end{Entry}

\begin{Entry}{吃不上}{6,4,3}{⼝、⼀、⼀}
  \begin{Phonetics}{吃不上}{chi1bu5shang4}[][HSK 7-9]
    \definition{v.}{incapaz de comer alguma coisa | pular uma refeição; perder a chance de comer | não conseguir comer alguma coisa; não ter comida para comer}
  \end{Phonetics}
\end{Entry}

\begin{Entry}{吃饭}{6,7}{⼝、⾷}
  \begin{Phonetics}{吃饭}{chi1/fan4}[][HSK 1]
    \definition{v.+compl.}{comer; ter (comer) uma refeição | manter-se vivo;  ganhar a vida; refere-se à vida ou à sobrevivência em geral}
  \end{Phonetics}
\end{Entry}

\begin{Entry}{吃苦}{6,8}{⼝、⾋}
  \begin{Phonetics}{吃苦}{chi1/ku3}[][HSK 7-9]
    \definition{v.+compl.}{suportar dificuldades; sofrer}[他在工作中吃了很多苦。===Ele sofreu muito em seu trabalho.]
  \end{Phonetics}
\end{Entry}

\begin{Entry}{吃屎}{6,9}{⼝、⼫}
  \begin{Phonetics}{吃屎}{chi1 shi3}
    \definition{expr.}{Coma merda!}
  \end{Phonetics}
\end{Entry}

\begin{Entry}{吃惊}{6,11}{⼝、⼼}
  \begin{Phonetics}{吃惊}{chi1/jing1}[][HSK 4]
    \definition{v.+compl.}{ficar assustado; ficar chocado; ficar espantado; pegar de surpresa; ficar assustado inesperadamente}
  \end{Phonetics}
\end{Entry}

\begin{Entry}{吃喝玩乐}{6,12,8,5}{⼝、⼝、⽟、⼃}
  \begin{Phonetics}{吃喝玩乐}{chi1-he1-wan2-le4}[][HSK 7-9]
    \definition{expr.}{comer, beber e se divertir --- passar o tempo com prazer | abandonar-se a uma vida de prazer}
  \end{Phonetics}
\end{Entry}

\begin{Entry}{各}{6}{⼝}
  \begin{Phonetics}{各}{ge4}[][HSK 3]
    \definition{adv.}{de várias maneiras; de diversas formas; respectivamente; indica que algo é feito separadamente ou que possui uma determinada característica separadamente}
    \definition{pron.}{todo; todos; cada; refere-se a todos os indivíduos dentro de um determinado intervalo, equivalente a 每个}
  \seealsoref{每个}{mei3ge4}
  \end{Phonetics}
\end{Entry}

\begin{Entry}{各个}{6,3}{⼝、⼈}
  \begin{Phonetics}{各个}{ge4 ge4}[][HSK 4]
    \definition{adv./pron.}{cada | um a um; um após o outro}
  \end{Phonetics}
\end{Entry}

\begin{Entry}{各地}{6,6}{⼝、⼟}
  \begin{Phonetics}{各地}{ge4 di4}[][HSK 3]
    \definition{s.}{em todos os lugares; em vários locais}
  \end{Phonetics}
\end{Entry}

\begin{Entry}{各式各样}{6,6,6,10}{⼝、⼷、⼝、⽊}
  \begin{Phonetics}{各式各样}{ge4shi4-ge4yang4}[][HSK 7-9]
    \definition{expr.}{todo tipo de\dots; todos os tipos de\dots; todos os tipos de; de ​​várias maneiras; de todas as descrições; uma variedade de; uma variedade de variedades com cores diferentes}
  \end{Phonetics}
\end{Entry}

\begin{Entry}{各自}{6,6}{⼝、⾃}
  \begin{Phonetics}{各自}{ge4zi4}[][HSK 3]
    \definition{pron.}{por si mesmo; por conta própria; cada um por si | cada um; indica cada uma das partes envolvidas}
  \end{Phonetics}
\end{Entry}

\begin{Entry}{各位}{6,7}{⼝、⼈}
  \begin{Phonetics}{各位}{ge4 wei4}[][HSK 3]
    \definition{pron.}{todos; toda a gente; todo mundo | cada um}
  \end{Phonetics}
\end{Entry}

\begin{Entry}{各奔前程}{6,8,9,12}{⼝、⼤、⼑、⽲}
  \begin{Phonetics}{各奔前程}{ge4ben4qian2cheng2}[][HSK 7-9]
    \definition{expr.}{``Cada um segue seu próprio caminho.''; cada pessoa tem sua própria vida para viver; cada um deles desenvolve sua própria carreira ambiciosa; cada um segue seu próprio curso}
  \end{Phonetics}
\end{Entry}

\begin{Entry}{各种}{6,9}{⼝、⽲}
  \begin{Phonetics}{各种}{ge4 zhong3}[][HSK 3]
    \definition{adv.}{todos os tipos; vários tipos}
  \end{Phonetics}
\end{Entry}

\begin{Entry}{合}{6}{⼝}
  \begin{Phonetics}{合}{he2}[][HSK 3]
    \definition{adj.}{todo; completo; inteiro}
    \definition{clas.}{usado para rodadas | 100ml | medida para grãos secos igual a um décimo de 升, ou um centésimo de 斗}
    \definition{s.}{casamento; união matrimonial | (astronomia) conjunção | nota da escala em Gongchepu (工尺谱), correspondente ao 5 na notação musical numerada}
    \definition{v.}{fechar | juntar; combinar (oposto de 分) | adequar-se; concordar; conformar-se a | ser igual a; somar | ser adequado}
  \seealsoref{斗}{dou4}
  \seealsoref{分}{fen1}
  \seealsoref{工尺谱}{gong1 che3 pu3}
  \seealsoref{升}{sheng1}
  \end{Phonetics}
\end{Entry}

\begin{Entry}{合计}{6,4}{⼝、⾔}
  \begin{Phonetics}{合计}{he2ji4}[][HSK 7-9]
    \definition{v.}{pensar sobre; descobrir | consultar | somar; totalizar}
  \end{Phonetics}
  \begin{Phonetics}{合计}{he2ji5}
    \definition{v.}{totalizar; somar; estimar | discutir; negociar; deliberar}
  \end{Phonetics}
\end{Entry}

\begin{Entry}{合乎}{6,5}{⼝、⼃}
  \begin{Phonetics}{合乎}{he2hu1}[][HSK 7-9]
    \definition{v.}{conformar-se com (ou a); corresponder a; concordar com; coincidir com; ser consistente com}
  \end{Phonetics}
\end{Entry}

\begin{Entry}{合伙}{6,6}{⼝、⼈}
  \begin{Phonetics}{合伙}{he2huo3}[][HSK 7-9]
    \definition{v.}{formar uma parceria; formar uma parceria relativamente fixa (para se envolver em atividades comerciais ou fazer coisas ruins)}
  \end{Phonetics}
\end{Entry}

\begin{Entry}{合同}{6,6}{⼝、⼝}
  \begin{Phonetics}{合同}{he2tong5}[][HSK 4]
    \definition[个,份]{s.}{contrato; acordo; uma disposição para observância mútua por duas ou mais partes na condução de um assunto com o objetivo de determinar seus respectivos direitos e obrigações.}
  \end{Phonetics}
\end{Entry}

\begin{Entry}{合并}{6,6}{⼝、⼲}
  \begin{Phonetics}{合并}{he2bing4}[][HSK 5]
    \definition{v.}{fundir; amalgamar; combinar várias coisas em uma coisa só | (doença) ser complicada por outra doença; uma doença levar a outra, ataques simultâneos (de várias doenças)}
  \end{Phonetics}
\end{Entry}

\begin{Entry}{合成}{6,6}{⼝、⼽}
  \begin{Phonetics}{合成}{he2cheng2}[][HSK 5]
    \definition{s.}{compor; integrar; combinar; misturar | Química: sintetizar, reação química para transformar uma substância com uma composição simples em uma substância com uma composição complexa}
  \end{Phonetics}
\end{Entry}

\begin{Entry}{合约}{6,6}{⼝、⽷}
  \begin{Phonetics}{合约}{he2 yue1}[][HSK 6]
    \definition[份]{s.}{contrato; geralmente se refere a contratos com cláusulas mais simples}
  \end{Phonetics}
\end{Entry}

\begin{Entry}{合作}{6,7}{⼝、⼈}
  \begin{Phonetics}{合作}{he2zuo4}[][HSK 3]
    \definition{v.}{cooperar; colaborar; trabalhar em conjunto; trabalhar em conjunto para realizar algo ou concluir uma tarefa}
  \end{Phonetics}
\end{Entry}

\begin{Entry}{合作社}{6,7,7}{⼝、⼈、⽰}
  \begin{Phonetics}{合作社}{he2zuo4she4}[][HSK 7-9]
    \definition{s.}{cooperativa | cooperativa de trabalhadores ou produtores agrícolas, etc.}
  \end{Phonetics}
\end{Entry}

\begin{Entry}{合法}{6,8}{⼝、⽔}
  \begin{Phonetics}{合法}{he2fa3}[][HSK 3]
    \definition{adj.}{legal; legítimo; lícito;  justo; válido; em conformidade com as disposições legais}
  \end{Phonetics}
\end{Entry}

\begin{Entry}{合宪性}{6,9,8}{⼝、⼧、⼼}
  \begin{Phonetics}{合宪性}{he2xian4xing4}
    \definition{s.}{constitucionalismo}
  \end{Phonetics}
\end{Entry}

\begin{Entry}{合适}{6,9}{⼝、⾡}
  \begin{Phonetics}{合适}{he2shi4}[][HSK 2]
    \definition{adj.}{correto; adequado; apropriado; conveniente; em conformidade com a realidade ou com os requisitos objetivos}
  \end{Phonetics}
\end{Entry}

\begin{Entry}{合格}{6,10}{⼝、⽊}
  \begin{Phonetics}{合格}{he2ge2}[][HSK 3]
    \definition{adj.}{qualificado; dentro dos padrões; em conformidade com os requisitos ou normas}
  \end{Phonetics}
\end{Entry}

\begin{Entry}{合资}{6,10}{⼝、⾙}
  \begin{Phonetics}{合资}{he2zi1}[][HSK 7-9]
    \definition{s.}{consórcio; \emph{joint-venture} com capitais mistos; investimento conjunto por duas ou mais partes (diferente de 独资)}
    \definition{v.}{investir conjuntamente em}
  \seealsoref{独资}{du2zi1}
  \end{Phonetics}
\end{Entry}

\begin{Entry}{合唱}{6,11}{⼝、⼝}
  \begin{Phonetics}{合唱}{he2chang4}[][HSK 7-9]
    \definition{v.}{cantar em coro; cantar ou apresentar-se junto}[他们一起合唱一台戏。===Eles cantam uma ópera juntos.]
  \end{Phonetics}
\end{Entry}

\begin{Entry}{合情合理}{6,11,6,11}{⼝、⼼、⼝、⽟}
  \begin{Phonetics}{合情合理}{he2qing2-he2li3}[][HSK 7-9]
    \definition{expr.}{razoável; razoável e lógico; justo e racional; justificável e sensato; justo e razoável; justo e sensato}
  \end{Phonetics}
\end{Entry}

\begin{Entry}{合理}{6,11}{⼝、⽟}
  \begin{Phonetics}{合理}{he2li3}[][HSK 3]
    \definition{adj.}{racional; razoável; equitativo; razoável ou lógico}
  \end{Phonetics}
\end{Entry}

\begin{Entry}{合影}{6,15}{⼝、⼺}
  \begin{Phonetics}{合影}{he2/ying3}[][HSK 7-9]
    \definition[张,个]{s.}{foto de grupo; imagem de grupo}
    \definition{v.+compl.}{tirar uma foto em grupo; tirar uma foto}
  \end{Phonetics}
\end{Entry}

\begin{Entry}{吉}{6}{⼝}
  \begin{Phonetics}{吉}{ji2}
    \definition*{s.}{Província de Jilin, abreviação de 吉林 | Sobrenome Ji}
    \definition{adj.}{sortudo; propício; auspicioso (oposto de 凶)}
  \seealsoref{吉林}{ji2lin2}
  \seealsoref{凶}{xiong1}
  \end{Phonetics}
\end{Entry}

\begin{Entry}{吉他}{6,5}{⼝、⼈}
  \begin{Phonetics}{吉他}{ji2ta1}[][HSK 7-9]
    \definition[把]{s.}{Empréstimo linguístico: violão}
  \end{Phonetics}
\end{Entry}

\begin{Entry}{吉利}{6,7}{⼝、⼑}
  \begin{Phonetics}{吉利}{ji2 li4}[][HSK 6]
    \definition{adj.}{sortudo; auspicioso; propício}
  \end{Phonetics}
\end{Entry}

\begin{Entry}{吉林}{6,8}{⼝、⽊}
  \begin{Phonetics}{吉林}{ji2lin2}
    \definition*{s.}{Província de Jilin}
  \end{Phonetics}
\end{Entry}

\begin{Entry}{吉祥}{6,10}{⼝、⽰}
  \begin{Phonetics}{吉祥}{ji2xiang2}[][HSK 6]
    \definition{adj.}{sortudo; auspicioso; propício}
    \definition[个,种]{s.}{sorte; auspiciosidade; propiciação; um sinal ou símbolo de boa sorte ou fortuna}
  \end{Phonetics}
\end{Entry}

\begin{Entry}{吉祥物}{6,10,8}{⼝、⽰、⽜}
  \begin{Phonetics}{吉祥物}{ji2xiang2wu4}[][HSK 7-9]
    \definition{s.}{mascote}
  \end{Phonetics}
\end{Entry}

\begin{Entry}{吉普}{6,12}{⼝、⽇}
  \begin{Phonetics}{吉普}{ji2pu3}[][HSK 7-9]
    \definition*{s.}{1. Jeep (marca de carro)}
    \definition[辆]{s.}{Empéstimo linguístico: jipe}[他开着吉普车去沙漠旅行。===Ele fez uma viagem pelo deserto em seu jipe.]
  \seealsoref{吉普车}{ji2pu3che1}
  \end{Phonetics}
\end{Entry}

\begin{Entry}{吉普车}{6,12,4}{⼝、⽇、⾞}
  \begin{Phonetics}{吉普车}{ji2pu3che1}
    \definition[辆]{s.}{Empréstimo linguístico: jipe ​​(veículo militar)}
  \seealsoref{吉普}{ji2pu3}
  \end{Phonetics}
\end{Entry}

\begin{Entry}{吊}{6}{⼝}
  \begin{Phonetics}{吊}{diao4}[][HSK 6]
    \definition{clas.}{uma sequência de 1.000 em dinheiro; antigamente, uma unidade monetária geralmente era composta por mil pequenas moedas de cobre}
    \definition{s.}{guindaste}
    \definition{v.}{pendurar; suspender | levantar ou abaixar com uma corda, etc. | colocar um forro de pele; adicionar revestimentos ou forros aos barris de couro para fazer roupas | revogar; retirar; recuperar documentos emitidos | lamentar; prestar homenagem os mortos ou oferecer condolências às famílias ou grupos que sofreram uma perda}
  \end{Phonetics}
\end{Entry}

\begin{Entry}{吊销}{6,12}{⼝、⾦}
  \begin{Phonetics}{吊销}{diao4xiao1}[][HSK 7-9]
    \definition{v.}{revogar; retirar; desativar; cancelar; reclamar e cancelar (certificados emitidos)}
  \end{Phonetics}
\end{Entry}

\begin{Entry}{同}{6}{⼝}
  \begin{Phonetics}{同}{tong2}[][HSK 6]
    \definition{adj.}{como; igual; parecido; similar; o mesmo; sem diferença}
    \definition{adv.}{juntos; em comum; indica que diferentes atores realizam uma determinada ação juntos ou estão na mesma situação, o que equivale a 一同 ou 一起}
    \definition{v.}{ser o mesmo que}
  \seealsoref{一起}{yi4qi3}
  \seealsoref{一同}{yi4tong2}
  \end{Phonetics}
  \begin{Phonetics}{同}{tong4}
    \definition[条,处]{s.}{beco; rua estreita}
  \seealsoref{胡同}{hu2tong5}
  \end{Phonetics}
\end{Entry}

\begin{Entry}{同一}{6,1}{⼝、⼀}
  \begin{Phonetics}{同一}{tong2 yi1}[][HSK 6]
    \definition{adj.}{mesmo; idêntico}
    \definition[讲]{s.}{identidade; unidade}
  \end{Phonetics}
\end{Entry}

\begin{Entry}{同伙}{6,6}{⼝、⼈}
  \begin{Phonetics}{同伙}{tong2huo3}
    \definition[个]{s.}{cúmplice | colega}
  \end{Phonetics}
\end{Entry}

\begin{Entry}{同行}{6,6}{⼝、⾏}
  \begin{Phonetics}{同行}{tong2 hang2}[][HSK 6]
    \definition{s.}{do mesmo ofício ou ocupação; pessoas no mesmo setor}
    \definition{v.}{ser do mesmo ofício ou ocupação; trabalhar no mesmo setor}
  \end{Phonetics}
\end{Entry}

\begin{Entry}{同时}{6,7}{⼝、⽇}
  \begin{Phonetics}{同时}{tong2shi2}[][HSK 2]
    \definition{conj.}{além disso; além do mais; ainda mais; indica uma relação de equivalência, geralmente com um significado mais profundo}
    \definition{s.}{enquanto isso; ao mesmo tempo}
  \end{Phonetics}
\end{Entry}

\begin{Entry}{同事}{6,8}{⼝、⼅}
  \begin{Phonetics}{同事}{tong2shi4}[][HSK 2]
    \definition[个,位,名]{s.}{companheiro; colega; colega de trabalho; pessoas que trabalham juntas}
    \definition{v.}{trabalhar no mesmo lugar; trabalhar juntos; trabalhar na mesma unidade}
  \end{Phonetics}
\end{Entry}

\begin{Entry}{同学}{6,8}{⼝、⼦}
  \begin{Phonetics}{同学}{tong2xue2}[][HSK 1]
    \definition[位,个,些]{s.}{colega de escola; colega de turma; colega de estudos; pessoas que estudam na mesma escola}
  \end{Phonetics}
\end{Entry}

\begin{Entry}{同性恋}{6,8,10}{⼝、⼼、⼼}
  \begin{Phonetics}{同性恋}{tong2xing4lian4}
    \definition{s.}{homossexualidade | pessoa gay | amor gay}
  \end{Phonetics}
\end{Entry}

\begin{Entry}{同屋}{6,9}{⼝、⼫}
  \begin{Phonetics}{同屋}{tong2wu1}
    \definition[个]{s.}{companheiro de quarto | colega de quarto}
  \end{Phonetics}
\end{Entry}

\begin{Entry}{同砚}{6,9}{⼝、⽯}
  \begin{Phonetics}{同砚}{tong2yan4}
    \definition[位,个]{s.}{colega de classe | colega estudante}
  \end{Phonetics}
\end{Entry}

\begin{Entry}{同胞}{6,9}{⼝、⾁}
  \begin{Phonetics}{同胞}{tong2bao1}[][HSK 6]
    \definition{s.}{nascidos dos mesmos pais | compatriota; conterrâneo; pessoas do mesmo país ou etnia}
  \end{Phonetics}
\end{Entry}

\begin{Entry}{同样}{6,10}{⼝、⽊}
  \begin{Phonetics}{同样}{tong2 yang4}[][HSK 2]
    \definition{adj.}{igual; semelhante; similar; idêntico; sem diferença}
  \end{Phonetics}
\end{Entry}

\begin{Entry}{同流合污}{6,10,6,6}{⼝、⽔、⼝、⽔}
  \begin{Phonetics}{同流合污}{tong2liu2he2wu1}
    \definition{expr.}{chafurdar na lama com alguém | seguir o mau exemplo dos outros}
  \end{Phonetics}
\end{Entry}

\begin{Entry}{同情}{6,11}{⼝、⼼}
  \begin{Phonetics}{同情}{tong2qing2}[][HSK 4]
    \definition{s.}{simpatia}
    \definition{v.}{simpatizar com; solidarizar-se; compadecer-se; ter empatia emocional pelo que os outros estão passando}
  \end{Phonetics}
\end{Entry}

\begin{Entry}{同期}{6,12}{⼝、⽉}
  \begin{Phonetics}{同期}{tong2 qi1}[][HSK 6]
    \definition{s.}{o período correspondente; o mesmo período; no mesmo tempo}
  \end{Phonetics}
\end{Entry}

\begin{Entry}{同意}{6,13}{⼝、⼼}
  \begin{Phonetics}{同意}{tong2yi4}[][HSK 3]
    \definition{v.}{concordar; consentir; aprovar; concordar com; dizer sim}
  \end{Phonetics}
\end{Entry}

\begin{Entry}{名}{6}{⼝}
  \begin{Phonetics}{名}{ming2}[][HSK 2]
    \definition*{s.}{Sobrenome Ming}
    \definition{adj.}{notável; famoso; conhecido; renomado}
    \definition{clas.}{usado para pessoas | usado para classificação por ordem}
    \definition{s.}{nome; denominação | desculpa; pretexto | fama; reputação}
    \definition{v.}{nome próprio (é) | expressar; descrever | possuir; tomar; ter}
  \end{Phonetics}
\end{Entry}

\begin{Entry}{名人}{6,2}{⼝、⼈}
  \begin{Phonetics}{名人}{ming2 ren2}[][HSK 4]
    \definition[位,个]{s.}{celebridade; pessoa famosa}
  \end{Phonetics}
\end{Entry}

\begin{Entry}{名义}{6,3}{⼝、⼂}
  \begin{Phonetics}{名义}{ming2 yi4}[][HSK 6]
    \definition{s.}{nominal; em nome (geralmente seguido por 上); um nome ou título usado como base para fazer algo}[有人盗用我名义申请贷款。===Alguém solicitou um empréstimo em meu nome. | 他们只是名义上的夫妻。===Eles são marido e mulher apenas no nome.]
  \seealsoref{上}{shang4}
  \end{Phonetics}
\end{Entry}

\begin{Entry}{名片}{6,4}{⼝、⽚}
  \begin{Phonetics}{名片}{ming2pian4}[][HSK 4]
    \definition[张,盒,叠]{s.}{cartão de visita; um pedaço de papel retangular com o nome, o cargo, o endereço etc. impressos}
  \end{Phonetics}
\end{Entry}

\begin{Entry}{名字}{6,6}{⼝、⼦}
  \begin{Phonetics}{名字}{ming2zi5}[][HSK 1]
    \definition[个]{s.}{nome; nome próprio | nome (de uma coisa)}
  \end{Phonetics}
\end{Entry}

\begin{Entry}{名单}{6,8}{⼝、⼗}
  \begin{Phonetics}{名单}{ming2 dan1}[][HSK 2]
    \definition[个,份]{s.}{lista com nomes de pessoas ou nomes de organizações}
  \end{Phonetics}
\end{Entry}

\begin{Entry}{名胜}{6,9}{⼝、⾁}
  \begin{Phonetics}{名胜}{ming2 sheng4}[][HSK 6]
    \definition[处,个]{s.}{pontos turísticos; atrações famosas; lugares famosos com locais históricos ou belas paisagens}
  \end{Phonetics}
\end{Entry}

\begin{Entry}{名称}{6,10}{⼝、⽲}
  \begin{Phonetics}{名称}{ming2 cheng1}[][HSK 2]
    \definition[个,种]{s.}{nomes, apelidos e formas de se referir a pessoas ou coisas}
  \end{Phonetics}
\end{Entry}

\begin{Entry}{名牌儿}{6,12,2}{⼝、⽚、⼉}
  \begin{Phonetics}{名牌儿}{ming2 pai2r5}[][HSK 4]
    \definition{s.}{marca famosa}
  \end{Phonetics}
\end{Entry}

\begin{Entry}{名誉}{6,13}{⼝、⾔}
  \begin{Phonetics}{名誉}{ming2yu4}[][HSK 6]
    \definition{adj.}{honorário; nominal (geralmente se refere ao nome de um presente, com um sentido de respeito)}[他是学校的名誉教授。===Ele é professor honorário da escola.]
    \definition{s.}{fama; reputação; honra}[名誉才是最神圣的。===Reputação é a coisa mais sagrada. | 我用自己的名誉发誓。===Juro pela minha honra.]
  \end{Phonetics}
\end{Entry}

\begin{Entry}{名额}{6,15}{⼝、⾴}
  \begin{Phonetics}{名额}{ming2'e2}[][HSK 6]
    \definition[个]{s.}{cota de pessoas; número de pessoas designadas ou permitidas; número necessário de pessoal}
  \end{Phonetics}
\end{Entry}

\begin{Entry}{后}{6}{⼝}
  \begin{Phonetics}{后}{hou4}[][HSK 1]
    \definition*{s.}{Sobrenome Hou}
    \definition{s.}{atrás; traseiro; a direção oposta àquela para a qual a pessoa está voltada; a direção oposta àquela para a qual a parte de trás de uma casa está voltada (o oposto de 前)  | depois; mais tarde no tempo; futuro (em oposição a 先 ou 前) | último | posteridade; descendência | rainha; imperatriz | governante; soberano; monarca antigo}
  \seealsoref{前}{qian2}
  \seealsoref{先}{xian1}
  \end{Phonetics}
\end{Entry}

\begin{Entry}{后人}{6,2}{⼝、⼈}
  \begin{Phonetics}{后人}{hou4ren2}[][HSK 7-9]
    \definition{s.}{gerações posteriores; gerações futuras | posteridade; descendentes; futuridade}
  \end{Phonetics}
\end{Entry}

\begin{Entry}{后天}{6,4}{⼝、⼤}
  \begin{Phonetics}{后天}{hou4 tian1}[][HSK 1]
    \definition{s.}{depois de amanhã; período em que uma pessoa ou animal vive e cresce sozinho após deixar o útero materno (em oposição a 先天)}
  \seealsoref{先天}{xian1tian1}
  \end{Phonetics}
\end{Entry}

\begin{Entry}{后方}{6,4}{⼝、⽅}
  \begin{Phonetics}{后方}{hou4 fang1}
    \definition{s.}{traseira; retaguarda (oposto à 前线 e 前方) | na parte de trás; na parte traseira}
  \seealsoref{前方}{qian2 fang1}
  \seealsoref{前线}{qian2 xian4}
  \end{Phonetics}
\end{Entry}

\begin{Entry}{后代}{6,5}{⼝、⼈}
  \begin{Phonetics}{后代}{hou4dai4}[][HSK 7-9]
    \definition{s.}{períodos posteriores (na história); eras posteriores; a era após uma certa era | gerações posteriores; posteridade; descendentes; gerações futuras}
  \end{Phonetics}
\end{Entry}

\begin{Entry}{后台}{6,5}{⼝、⼝}
  \begin{Phonetics}{后台}{hou4tai2}[][HSK 7-9]
    \definition{s.}{bastidores; plano de fundo | apoiador de bastidores; apoiador dos bastidores; uma metáfora para uma pessoa ou grupo que manipula ou apoia algo nos bastidores}
  \end{Phonetics}
\end{Entry}

\begin{Entry}{后头}{6,5}{⼝、⼤}
  \begin{Phonetics}{后头}{hou4 tou5}[][HSK 4]
    \definition{adv.}{posteriormente; atrás; mais tarde}
    \definition{s.}{a parte de trás; a parte traseira}
  \end{Phonetics}
\end{Entry}

\begin{Entry}{后边}{6,5}{⼝、⾡}
  \begin{Phonetics}{后边}{hou4 bian5}[][HSK 1]
    \definition{adv.}{costas; traseira; atrás}
  \end{Phonetics}
\end{Entry}

\begin{Entry}{后年}{6,6}{⼝、⼲}
  \begin{Phonetics}{后年}{hou4nian2}[][HSK 3]
    \definition{s.}{daqui a dois anos; no ano seguinte ao próximo ano}
  \end{Phonetics}
\end{Entry}

\begin{Entry}{后来}{6,7}{⼝、⽊}
  \begin{Phonetics}{后来}{hou4lai2}[][HSK 2]
    \definition{adv.}{mais tarde; depois; refere-se a um período posterior a um determinado momento no passado}
  \end{Phonetics}
\end{Entry}

\begin{Entry}{后备}{6,8}{⼝、⼡}
  \begin{Phonetics}{后备}{hou4bei4}[][HSK 7-9]
    \definition{s.}{reserva; preparado para reabastecimento (pessoal, suprimentos, etc.)}
  \end{Phonetics}
\end{Entry}

\begin{Entry}{后备箱}{6,8,15}{⼝、⼡、⾋}
  \begin{Phonetics}{后备箱}{hou4bei4xiang1}[][HSK 7-9]
    \definition{s.}{porta-malas (de um carro)}
  \end{Phonetics}
\end{Entry}

\begin{Entry}{后果}{6,8}{⼝、⽊}
  \begin{Phonetics}{后果}{hou4guo3}[][HSK 3]
    \definition{s.}{consequência; resultado (geralmente negativo)}
  \end{Phonetics}
\end{Entry}

\begin{Entry}{后者}{6,8}{⼝、⽼}
  \begin{Phonetics}{后者}{hou4zhe3}[][HSK 7-9]
    \definition{pron.}{o último; a última de duas ou mais pessoas ou coisas mencionadas ou autoevidentes}
    \definition{s.}{o último (oposto a 前者)}
  \seealsoref{前者}{qian2zhe3}
  \end{Phonetics}
\end{Entry}

\begin{Entry}{后盾}{6,9}{⼝、⽬}
  \begin{Phonetics}{后盾}{hou4dun4}[][HSK 7-9]
    \definition{s.}{apoio; força de apoio | suporte; assistência; suporte; apoiador}
  \end{Phonetics}
\end{Entry}

\begin{Entry}{后退}{6,9}{⼝、⾡}
  \begin{Phonetics}{后退}{hou4tui4}[][HSK 7-9]
    \definition{v.}{recuar; retrocerder; retornar (para um lugar posterior ou para um estágio anterior de desenvolvimento)}
  \end{Phonetics}
\end{Entry}

\begin{Entry}{后面}{6,9}{⼝、⾯}
  \begin{Phonetics}{后面}{hou4mian4}
    \definition{adv.}{parte de trás; retaguarda; atrás; a parte posterior do espaço ou localização | mais tarde; depois; no futuro; a parte posterior de um artigo ou discurso em relação ao que está sendo narrado no momento}
  \end{Phonetics}
\end{Entry}

\begin{Entry}{后悔}{6,10}{⼝、⼼}
  \begin{Phonetics}{后悔}{hou4hui3}[][HSK 5]
    \definition{v.}{lamentar; ter remorso; arrepender-se}
  \end{Phonetics}
\end{Entry}

\begin{Entry}{后顾之忧}{6,10,3,7}{⼝、⾴、⼂、⼼}
  \begin{Phonetics}{后顾之忧}{hou4gu4zhi1you1}[][HSK 7-9]
    \definition{expr.}{preocupações com o que ficou para trás; preocupações com problemas futuros; preocupações persistentes; preocupações não resolvidas; preocupações ou problemas potenciais; ``Ansiedade que exige olhar para trás.''; refere-se a preocupações com o lar, a família ou o futuro que surgem ao seguir em frente ou sair}
  \end{Phonetics}
\end{Entry}

\begin{Entry}{后续}{6,11}{⼝、⽷}
  \begin{Phonetics}{后续}{hou4xu4}[][HSK 7-9]
    \definition{adj.}{subsequente; de acompanhamento; decorrente; seguimento}
    \definition{v.}{casar novamente após a morte da esposa}
  \end{Phonetics}
\end{Entry}

\begin{Entry}{后期}{6,12}{⼝、⽉}
  \begin{Phonetics}{后期}{hou4qi1}[][HSK 7-9]
    \definition{s.}{estágio posterior; período posterior; a última fase de um período}
  \end{Phonetics}
\end{Entry}

\begin{Entry}{后遗症}{6,12,10}{⼝、⾡、⽧}
  \begin{Phonetics}{后遗症}{hou4yi2zheng4}[][HSK 7-9]
    \definition{s.}{sequelas; sintomas como defeitos ou disfunções orgânicas que permanecem após a recuperação de certas doenças | ressaca; efeito colateral; consequência; efeito residual}
  \end{Phonetics}
\end{Entry}

\begin{Entry}{后勤}{6,13}{⼝、⼒}
  \begin{Phonetics}{后勤}{hou4qin2}[][HSK 7-9]
    \definition{s.}{logística; serviços de retaguarda; todo o trabalho de fornecimento de áreas distantes da linha de frente para as áreas de linha de frente; trabalho administrativo em agências governamentais, empresas, etc., incluindo finanças, reparos, etc.}
  \end{Phonetics}
\end{Entry}

\begin{Entry}{后裔}{6,13}{⼝、⾐}
  \begin{Phonetics}{后裔}{hou4yi4}[][HSK 7-9]
    \definition{s.}{descendente (de uma pessoa morta); prole | descendente; posteridade; progênie}
  \end{Phonetics}
\end{Entry}

\begin{Entry}{吐}{6}{⼝}
  \begin{Phonetics}{吐}{tu3}[][HSK 5]
    \definition{v.}{cuspir; sair pela boca | surgir ou aparecer pela boca ou por uma fenda | dizer; contar; falar abertamente}
  \end{Phonetics}
  \begin{Phonetics}{吐}{tu4}[][HSK 5]
    \definition{v.}{vomitar; sair pela boca | vomitar; expelir; metáfora para ser forçado a devolver bens usurpados}
  \end{Phonetics}
\end{Entry}

\begin{Entry}{向}{6}{⼝}
  \begin{Phonetics}{向}{xiang4}[][HSK 2]
    \definition*{s.}{Sobrenome Xiang}
    \definition{adv.}{sempre; o tempo todo}
    \definition{prep.}{em direção a; para}
    \definition{s.}{direção | a janela voltada para o norte}
    \definition{v.}{encarar; virar-se para | estar do lado de; ser parcial com; tomar o partido de alguém}
  \end{Phonetics}
\end{Entry}

\begin{Entry}{向上}{6,3}{⼝、⼀}
  \begin{Phonetics}{向上}{xiang4 shang4}[][HSK 5]
    \definition{adv.}{o superior; acima}
    \definition{v.}{mover-se; subir; ir para um lugar mais alto; ir para um lugar mais alto em relação a um determinado ponto; ir para um desenvolvimento mais alto que o atual | avançar; continuar se aperfeiçoar; subir na vida; desenvolver-se em direção ao progresso}
  \end{Phonetics}
\end{Entry}

\begin{Entry}{向导}{6,6}{⼝、⼨}
  \begin{Phonetics}{向导}{xiang4dao3}[][HSK 5]
    \definition[位]{s.}{guia; a pessoa que lidera todos e lhes indica a direção ao caminhar}
    \definition{v.}{agir como um guia; mostrar a alguém o caminho; levar alguém a algum lugar}
  \end{Phonetics}
\end{Entry}

\begin{Entry}{向汪}{6,7}{⼝、⽔}
  \begin{Phonetics}{向汪}{xiang4wang1}
    \definition{v.}{esperar que}
  \end{Phonetics}
\end{Entry}

\begin{Entry}{向往}{6,8}{⼝、⼻}
  \begin{Phonetics}{向往}{xiang4wang3}
    \definition{v.}{ansiar por | esperar ansiosamente por}
  \end{Phonetics}
\end{Entry}

\begin{Entry}{向前}{6,9}{⼝、⼑}
  \begin{Phonetics}{向前}{xiang4 qian2}[][HSK 5]
    \definition{adv.}{para frente; adiante}
    \definition{v.}{avançar; ir em direção à frente; mover-se para frente; avançar um pouco mais}
  \end{Phonetics}
\end{Entry}

\begin{Entry}{吓}{6}{⼝}
  \begin{Phonetics}{吓}{xia4}[][HSK 5]
    \definition{interj.}{interjeição que demonstra espanto; Interjeição que expressa insatisfação}
    \definition{v.}{ameaçar; intimidar; usar ameaças ou meios coercitivos para intimidar ou assustar}
  \end{Phonetics}
\end{Entry}

\begin{Entry}{吓人}{6,2}{⼝、⼈}
  \begin{Phonetics}{吓人}{xia4/ren2}
    \definition{adj.}{apavorante | assustador}
    \definition{v.+compl.}{assustar-se | tomar um susto}
  \end{Phonetics}
\end{Entry}

\begin{Entry}{吗}{6}{⼝}
  \begin{Phonetics}{吗}{ma2}
    \definition{adv.}{(coloquial) que?}
  \end{Phonetics}
  \begin{Phonetics}{吗}{ma3}
    \definition{s.}{usada em 吗啡, morfina}
  \seealsoref{吗啡}{ma3fei1}
  \end{Phonetics}
  \begin{Phonetics}{吗}{ma5}[][HSK 1]
    \definition{part.}{usado no final de uma pergunta | como uma pausa em uma frase antes de introduzir o ponto principal | usado no final de uma pergunta retórica}
  \end{Phonetics}
\end{Entry}

\begin{Entry}{吗啡}{6,11}{⼝、⼝}
  \begin{Phonetics}{吗啡}{ma3fei1}
    \definition{s.}{morfina (empréstimo linguístico)}
  \end{Phonetics}
\end{Entry}

\begin{Entry}{吸}{6}{⼝}
  \begin{Phonetics}{吸}{xi1}[][HSK 4]
    \definition{v.}{inalar; inspirar; aspirar (oposto a 呼) | sugar (líquidos) | absorver; sugar | atrair; atrair para si mesmo | aspirar; introdução de líquidos, gases, etc. no corpo}
  \seealsoref{呼}{hu1}
  \end{Phonetics}
\end{Entry}

\begin{Entry}{吸引}{6,4}{⼝、⼸}
  \begin{Phonetics}{吸引}{xi1yin3}[][HSK 4]
    \definition{v.}{atrair; apelar para; chamar a atenção de outros objetos, forças ou pessoas para si mesmo}
  \end{Phonetics}
\end{Entry}

\begin{Entry}{吸收}{6,6}{⼝、⽁}
  \begin{Phonetics}{吸收}{xi1shou1}[][HSK 4]
    \definition{v.}{imbuir; absorver; assimilar; sugar;  chupar; (animais, plantas, etc.) extrair material de fora dos tecidos para o interior dos tecidos | absorver; chupar;  sugar alguma substância de fora para dentro | recrutar; alistar; inscrever-se; matricular-se; admitir; (organizações ou coletivos) aceitar novos membros | absorver; aproveitar e usar a experiência, o conhecimento, o dinheiro e outras coisas valiosas de outras pessoas | absorver; diminuir, atenuar ou eliminar determinados efeitos ou fenômenos}
  \end{Phonetics}
\end{Entry}

\begin{Entry}{吸毒}{6,9}{⼝、⽏}
  \begin{Phonetics}{吸毒}{xi1 du2}[][HSK 6]
    \definition{s.}{droga}
    \definition{v.}{usar drogas viciantes; ser viciado em um narcótico; consumir drogas}
  \end{Phonetics}
\end{Entry}

\begin{Entry}{吸烟}{6,10}{⼝、⽕}
  \begin{Phonetics}{吸烟}{xi1/yan1}[][HSK 4]
    \definition{v.+compl.}{fumar}
  \end{Phonetics}
\end{Entry}

\begin{Entry}{吸铁石}{6,10,5}{⼝、⾦、⽯}
  \begin{Phonetics}{吸铁石}{xi1tie3shi2}
    \definition{s.}{imã | magneto}
  \seealsoref{磁铁}{ci2tie3}
  \end{Phonetics}
\end{Entry}

\begin{Entry}{吸管}{6,14}{⼝、⽵}
  \begin{Phonetics}{吸管}{xi1 guan3}[][HSK 4]
    \definition[根,个,支]{s.}{tubo de sucção; sugador; canudo (para beber); refere-se ao tubo fino usado para sugar bebidas | conta-gotas; pipeta; cateter para bombeamento de líquidos usando pressão de ar}
  \end{Phonetics}
\end{Entry}

\begin{Entry}{君}{7}{⼝}
  \begin{Phonetics}{君}{jun1}
    \definition*{s.}{Sobrenome Jun}
    \definition[个,位,名,些]{s.}{monarca; soberano; governante supremo | (como título) Senhor; Sr. | (literário) (em trato direto) você; senhor | cavalheiro | governante}
  \end{Phonetics}
\end{Entry}

\begin{Entry}{君主立宪制}{7,5,5,9,8}{⼝、⼂、⽴、⼧、⼑}
  \begin{Phonetics}{君主立宪制}{jun1zhu3li4xian4zhi4}
    \definition{s.}{monarquia constitucional}
  \end{Phonetics}
\end{Entry}

\begin{Entry}{吞}{7}{⼝}
  \begin{Phonetics}{吞}{tun1}[][HSK 6]
    \definition*{s.}{Sobrenome Tun}
    \definition{v.}{engolir; engolir em seco | tomar posse de; anexar | engolir; tragar; devorar; engolir inteiro ou em pedaços | absorver; engolir; engolfar}
  \end{Phonetics}
\end{Entry}

\begin{Entry}{吟}{7}{⼝}
  \begin{Phonetics}{吟}{yin2}
    \definition*{s.}{Sobrenome Yin}
    \definition{s.}{canção (como um tipo de poesia clássica) | grito de certos animais ou insetos}
    \definition{v.}{cantar; recitar | gemer; lamentar}
  \end{Phonetics}
\end{Entry}

\begin{Entry}{吟诗}{7,8}{⼝、⾔}
  \begin{Phonetics}{吟诗}{yin2shi1}
    \definition{v.}{recitar poesia}
  \end{Phonetics}
\end{Entry}

\begin{Entry}{否}{7}{⼝}
  \begin{Phonetics}{否}{fou3}
    \definition{adv.}{não; expressa discordância, equivalente à palavra falada 不 | usado no final de uma pergunta para indicar investigação | 是否, 能否 e 可否 que significa respectivamente 是不是, 能不能 e 可不可}
    \definition{v.}{negar}
  \seealsoref{不}{bu4}
  \seealsoref{可}{ke3}
  \seealsoref{能}{neng2}
  \seealsoref{是}{shi4}
  \end{Phonetics}
  \begin{Phonetics}{否}{pi3}
    \definition{adj.}{ruim; maligno; perverso}
    \definition{v.}{censurar}
  \end{Phonetics}
\end{Entry}

\begin{Entry}{否认}{7,4}{⼝、⾔}
  \begin{Phonetics}{否认}{fou3ren4}[][HSK 3]
    \definition{v.}{negar; repudiar; não reconhecer}
  \end{Phonetics}
\end{Entry}

\begin{Entry}{否决}{7,6}{⼝、⼎}
  \begin{Phonetics}{否决}{fou3jue2}[][HSK 7-9]
    \definition{v.}{rejeitar; votar contra; vetar; anular}
  \end{Phonetics}
\end{Entry}

\begin{Entry}{否则}{7,6}{⼝、⼑}
  \begin{Phonetics}{否则}{fou3ze2}[][HSK 4]
    \definition{conj.}{senão; se não; ou então; se não for isso}
  \end{Phonetics}
\end{Entry}

\begin{Entry}{否定}{7,8}{⼝、⼧}
  \begin{Phonetics}{否定}{fou3ding4}[][HSK 3]
    \definition{adj.}{negativo; contrário}
    \definition{v.}{rejeitar; negar a existência ou a autenticidade de algo}
  \end{Phonetics}
\end{Entry}

\begin{Entry}{吧}{7}{⼝}
  \begin{Phonetics}{吧}{ba1}
    \definition{s.}{som de estalo, som crepitante |  abreviação de bar, 酒吧 | cibercafé; um local público que fornece computadores e serviços de \emph{Internet} onde as pessoas podem navegar, jogar, etc.}
    \definition{v.}{fumar; dar uma tragada (puxar) no cachimbo}
  \seealsoref{酒吧}{jiu3ba1}
  \end{Phonetics}
  \begin{Phonetics}{吧}{ba5}[][HSK 1]
    \definition{part.}{indica discussão, sugestão, solicitação ou comando no final de uma frase | indica concordância ou aprovação no final de uma frase | indica uma pergunta ou especulação no final de uma frase | indica incerteza no final de uma frase | em uma frase, indica uma pausa, carrega um tom hipotético, frequentemente apresenta um contraste e implica um dilema}
  \end{Phonetics}
\end{Entry}

\begin{Entry}{吨}{7}{⼝}
  \begin{Phonetics}{吨}{dun1}[][HSK 5]
    \definition{clas.}{tonelada}
  \end{Phonetics}
\end{Entry}

\begin{Entry}{吩}{7}{⼝}
  \begin{Phonetics}{吩}{fen1}
    \definition{v.}{deixar instruções; instruir | ordenar; mandar}
  \end{Phonetics}
\end{Entry}

\begin{Entry}{吩咐}{7,8}{⼝、⼝}
  \begin{Phonetics}{吩咐}{fen1fu4}[][HSK 7-9]
    \definition{v.}{dizer; instruir; comandar; dizer a alguém para lembrar o que deve ou não ser feito}
  \end{Phonetics}
\end{Entry}

\begin{Entry}{含}{7}{⼝}
  \begin{Phonetics}{含}{han2}[][HSK 4]
    \definition{v.}{manter na boca (sem engolir ou cuspir) | conter; incluir | cuidar; acalentar; abrigar}
  \end{Phonetics}
\end{Entry}

\begin{Entry}{含义}{7,3}{⼝、⼂}
  \begin{Phonetics}{含义}{han2yi4}[][HSK 4]
    \definition[个,种,层]{s.}{sentido; mensagem; significado; implicação; o significado contido em (palavras, frases, sentenças e discursos)}
  \end{Phonetics}
\end{Entry}

\begin{Entry}{含有}{7,6}{⼝、⽉}
  \begin{Phonetics}{含有}{han2 you3}[][HSK 4]
    \definition{v.}{conter; ter; incluir}
  \end{Phonetics}
\end{Entry}

\begin{Entry}{含金量}{7,8,12}{⼝、⾦、⾥}
  \begin{Phonetics}{含金量}{han2jin1liang4}
    \definition{adj.}{conteúdo de ouro | (fig.) valioso}
  \end{Phonetics}
\end{Entry}

\begin{Entry}{含量}{7,12}{⼝、⾥}
  \begin{Phonetics}{含量}{han2 liang4}[][HSK 4]
    \definition{s.}{conteúdo; a quantidade de um componente contido em uma substância}
  \end{Phonetics}
\end{Entry}

\begin{Entry}{含蓄}{7,13}{⼝、⾋}
  \begin{Phonetics}{含蓄}{han2xu4}[][HSK 7-9]
    \definition{v.}{conter; incorporar}
  \end{Phonetics}
\end{Entry}

\begin{Entry}{含糊}{7,15}{⼝、⽶}
  \begin{Phonetics}{含糊}{han2hu5}[][HSK 7-9]
    \definition{adj.}{(atitude, palavras, etc.) vago; ambíguo; pouco claro | (falar, fazer coisas, etc.) descuidado; desleixado (usado principalmente em termos negativos) | covarde; demonstrando fraqueza (usado principalmente em sentido negativo)}
  \end{Phonetics}
\end{Entry}

\begin{Entry}{听}{7}{⼝}
  \begin{Phonetics}{听}{ting1}[][HSK 1]
    \definition{clas.}{latas; usado para bebidas e alimentos para levar consigo}
    \definition{s.}{lata; embalagem metálica; recipiente cilíndrico utilizado para armazenar bebidas, alimentos, etc.}
    \definition{v.}{ouvir; escutar | obedecer; dar ouvidos; aceitar | supervisionar; administrar; tratar (assuntos políticos); julgar (casos) | permitir; deixar ser; deixar fazer}
  \end{Phonetics}
  \begin{Phonetics}{听}{yin3}
    \definition[个]{s.}{lata; embalagem metálica}
  \end{Phonetics}
\end{Entry}

\begin{Entry}{听力}{7,2}{⼝、⼒}
  \begin{Phonetics}{听力}{ting1li4}[][HSK 3]
    \definition{s.}{audição; capacidade auditiva | compreensão auditiva (na aprendizagem de línguas)}
  \end{Phonetics}
\end{Entry}

\begin{Entry}{听力理解}{7,2,11,13}{⼝、⼒、⽟、⾓}
  \begin{Phonetics}{听力理解}{ting1li4li3jie3}
    \definition{s.}{compreensão auditiva}
  \end{Phonetics}
\end{Entry}

\begin{Entry}{听小骨}{7,3,9}{⼝、⼩、⾻}
  \begin{Phonetics}{听小骨}{ting1xiao3gu3}
    \definition{s.}{ossículos (do ouvido médio)}
  \seealsoref{听骨}{ting1gu3}
  \end{Phonetics}
\end{Entry}

\begin{Entry}{听见}{7,4}{⼝、⾒}
  \begin{Phonetics}{听见}{ting1 jian4}[][HSK 1]
    \definition{v.}{ouvir; conseguir ouvir}
  \end{Phonetics}
\end{Entry}

\begin{Entry}{听写}{7,5}{⼝、⼍}
  \begin{Phonetics}{听写}{ting1 xie3}[][HSK 1]
    \definition{s.}{ditado}
    \definition{v.}{ouvir e escrever}
  \end{Phonetics}
\end{Entry}

\begin{Entry}{听众}{7,6}{⼝、⼈}
  \begin{Phonetics}{听众}{ting1 zhong4}[][HSK 3]
    \definition[位,名,个]{s.}{audiência; ouvintes; pessoas que ouvem palestras, música ou transmissões}
  \end{Phonetics}
\end{Entry}

\begin{Entry}{听会}{7,6}{⼝、⼈}
  \begin{Phonetics}{听会}{ting1hui4}
    \definition{v.}{participar de uma reunião (e ouvir o que é discutido)}
  \end{Phonetics}
\end{Entry}

\begin{Entry}{听戏}{7,6}{⼝、⼽}
  \begin{Phonetics}{听戏}{ting1xi4}
    \definition{v.}{assistir a uma ópera | ver uma ópera}
  \end{Phonetics}
\end{Entry}

\begin{Entry}{听讲}{7,6}{⼝、⾔}
  \begin{Phonetics}{听讲}{ting1/jiang3}[][HSK 2]
    \definition{v.+compl.}{assistir a uma palestra; ouvir palestras ou discursos}
  \end{Phonetics}
\end{Entry}

\begin{Entry}{听来}{7,7}{⼝、⽊}
  \begin{Phonetics}{听来}{ting1lai2}
    \definition{v.}{ouvir de algum lugar | soar (antigo, estrangeiro, excitante, certo, etc.) | soar como se (ou seja, dar uma impressão ao ouvinte)}
  \end{Phonetics}
\end{Entry}

\begin{Entry}{听凭}{7,8}{⼝、⼏}
  \begin{Phonetics}{听凭}{ting1ping2}
    \definition{v.}{permitir (alguém a fazer o que desejar)}
  \end{Phonetics}
\end{Entry}

\begin{Entry}{听到}{7,8}{⼝、⼑}
  \begin{Phonetics}{听到}{ting1 dao4}[][HSK 1]
    \definition{v.}{ouvir, escutar; ouvir atentamente, escutar atentamente}
  \end{Phonetics}
\end{Entry}

\begin{Entry}{听取}{7,8}{⼝、⼜}
  \begin{Phonetics}{听取}{ting1 qu3}[][HSK 6]
    \definition{v.}{ouvir (opiniões, reflexões, relatórios, etc.)}
  \end{Phonetics}
\end{Entry}

\begin{Entry}{听命}{7,8}{⼝、⼝}
  \begin{Phonetics}{听命}{ting1ming4}
    \definition{v.}{obedecer ordens | receber ordens}
  \end{Phonetics}
\end{Entry}

\begin{Entry}{听说}{7,9}{⼝、⾔}
  \begin{Phonetics}{听说}{ting1 shuo1}[][HSK 2]
    \definition{v.}{ser informado; ouvir falar de; ouvir dizer | ouvir e falar}
  \end{Phonetics}
\end{Entry}

\begin{Entry}{听骨}{7,9}{⼝、⾻}
  \begin{Phonetics}{听骨}{ting1gu3}
    \definition{s.}{ossículos (do ouvido médio)}
  \seealsoref{听小骨}{ting1xiao3gu3}
  \end{Phonetics}
\end{Entry}

\begin{Entry}{听断}{7,11}{⼝、⽄}
  \begin{Phonetics}{听断}{ting1duan4}
    \definition{v.}{ouvir e decidir | julgar (ou seja, ouvir e julgar em um tribunal)}
  \end{Phonetics}
\end{Entry}

\begin{Entry}{听随}{7,11}{⼝、⾩}
  \begin{Phonetics}{听随}{ting1sui2}
    \definition{v.}{permitir | obedecer}
  \end{Phonetics}
\end{Entry}

\begin{Entry}{启}{7}{⼝}
  \begin{Phonetics}{启}{qi3}
    \definition*{s.}{Sobrenome Qi}
    \definition{s.}{nota; carta; um dos antigos estilos literários, uma carta relativamente curta}
    \definition{v.}{abrir | despertar; iluminar | começar; iniciar | declarar; informar}
  \end{Phonetics}
\end{Entry}

\begin{Entry}{启发}{7,5}{⼝、⼜}
  \begin{Phonetics}{启发}{qi3fa1}[][HSK 5]
    \definition{s.}{iluminação; esclarecimento; fenômenos e princípios que levam as pessoas a refletir e a abrir suas mentes}
    \definition{v.}{despertar; inspirar; esclarecer; orientar, fazer com que compreendam}
  \end{Phonetics}
\end{Entry}

\begin{Entry}{启动}{7,6}{⼝、⼒}
  \begin{Phonetics}{启动}{qi3 dong4}[][HSK 5]
    \definition{v.}{ligar (uma máquina); acionar; ligar máquinas, equipamentos elétricos, etc., para começar a trabalhar | entrar em vigor; começar a vigorar e a ser implementados planos, projetos, documentos jurídicos, etc.}
  \end{Phonetics}
\end{Entry}

\begin{Entry}{启事}{7,8}{⼝、⼅}
  \begin{Phonetics}{启事}{qi3shi4}[][HSK 5]
    \definition[个,则,份,张,条]{s.}{aviso; anúncio; texto publicado em jornais ou afixado em paredes com o objetivo de divulgar publicamente algo}
  \end{Phonetics}
\end{Entry}

\begin{Entry}{吵}{7}{⼝}
  \begin{Phonetics}{吵}{chao3}[][HSK 3]
    \definition{adj.}{barulhento; ruidoso e perturbador}
    \definition{v.}{brigar; discutir; disputar}
  \end{Phonetics}
\end{Entry}

\begin{Entry}{吵架}{7,9}{⼝、⽊}
  \begin{Phonetics}{吵架}{chao3/jia4}[][HSK 3]
    \definition{v.+compl.}{brigar; discutir; ter uma discussão acalorada}
  \end{Phonetics}
\end{Entry}

\begin{Entry}{吵嘴}{7,16}{⼝、⼝}
  \begin{Phonetics}{吵嘴}{chao3/zui3}[][HSK 7-9]
    \definition{v.}{brigar; discutir}[我们之间从不吵嘴。===Nós nunca brigamos.]
  \end{Phonetics}
\end{Entry}

\begin{Entry}{吹}{7}{⼝}
  \begin{Phonetics}{吹}{chui1}[][HSK 2]
    \definition{v.}{soprar; baforar | tocar (instrumentos de sopro) | (do vento) soprar | gabar-se; vangloriar-se | elogiar; louvar aos céus; adular | (relacionamento) romper; separar-se; (assunto) fracassar}
  \end{Phonetics}
\end{Entry}

\begin{Entry}{吹了}{7,2}{⼝、⼅}
  \begin{Phonetics}{吹了}{chui1 le5}[][HSK 7-9]
    \definition{v.}{quebrar; falhar | ter esfriado (um relacionamento) | ter morrido | não ter tido sucesso | ter se separado}
  \end{Phonetics}
\end{Entry}

\begin{Entry}{吹牛}{7,4}{⼝、⽜}
  \begin{Phonetics}{吹牛}{chui1/niu2}[][HSK 7-9]
    \definition{v.+compl.}{gabar-se; vangloriar-se; falar alto; falar com arrogância; falar sem parar}
  \end{Phonetics}
\end{Entry}

\begin{Entry}{吹捧}{7,11}{⼝、⼿}
  \begin{Phonetics}{吹捧}{chui1peng3}[][HSK 7-9]
    \definition{v.}{bajular; elogiar até os céus; elogiar abundantemente}
  \end{Phonetics}
\end{Entry}

\begin{Entry}{吼}{7}{⼝}
  \begin{Phonetics}{吼}{hou3}[][HSK 7-9]
    \definition{v.}{rugido; uivo | rugido; gritar alto quando estiver com raiva}
  \end{Phonetics}
\end{Entry}

\begin{Entry}{吾}{7}{⼝}
  \begin{Phonetics}{吾}{wu2}
    \definition*{s.}{Sobrenome Wu}
    \definition{pron.}{eu; nós}
  \end{Phonetics}
\end{Entry}

\begin{Entry}{呀}{7}{⼝}
  \begin{Phonetics}{呀}{ya5}[][HSK 4]
    \definition{part.}{usado no lugar de 啊 quando a palavra anterior termina com o som a, e, i, o ou ü}
  \seealsoref{啊}{a5}
  \end{Phonetics}
\end{Entry}

\begin{Entry}{呆}{7}{⼝}
  \begin{Phonetics}{呆}{dai1}[][HSK 5]
    \definition*{s.}{Sobrenome Dai}
    \definition{adj.}{maçante; de raciocínio lento | em branco; de madeira; rígido; inflexível}
    \definition{v.}{ficar; permanecer}
  \end{Phonetics}
\end{Entry}

\begin{Entry}{呈}{7}{⼝}
  \begin{Phonetics}{呈}{cheng2}
    \definition*{s.}{Sobrenome Cheng}
    \definition{s.}{documento submetido a um superior; petição; memorial}
    \definition{v.}{apresentar; assumir (forma, cor, etc.) | submeter; apresentar; enviar respeitosamente}
  \end{Phonetics}
\end{Entry}

\begin{Entry}{呈现}{7,8}{⼝、⾒}
  \begin{Phonetics}{呈现}{cheng2xian4}[][HSK 7-9]
    \definition{v.}{apresentar (uma certa aparência); aparecer. mostrar uma determinada forma, cor ou tendência, etc., para que as pessoas possam ver}[大海呈现出碧蓝的颜色。===O mar apresenta uma cor azul vibrante.]
  \end{Phonetics}
\end{Entry}

\begin{Entry}{告}{7}{⼝}
  \begin{Phonetics}{告}{gao4}[][HSK 7-9]
    \definition*{s.}{Sobrenome Gao}
    \definition{s.}{anúncio; declaração; notificação}
    \definition{v.}{informar; contar; notificar; explicar aos outros | acusar; processar; relatar | pedir; requisitar; solicitar | dar a conhecer; mostrar | anunciar; declarar; proclamar}
  \end{Phonetics}
\end{Entry}

\begin{Entry}{告示}{7,5}{⼝、⽰}
  \begin{Phonetics}{告示}{gao4shi5}[][HSK 7-9]
    \definition[张,条,篇]{s.}{nota oficial; boletim; cartaz | Datado: slogan; cartaz}
    \definition{v.}{notificar; anunciar}
  \end{Phonetics}
\end{Entry}

\begin{Entry}{告别}{7,7}{⼝、⼑}
  \begin{Phonetics}{告别}{gao4/bie2}[][HSK 3]
    \definition{v.+compl.}{dizer adeus a; expressar a outros, por meio de palavras, que está prestes a partir | deixar; sair; partir de | prestar as últimas homenagens ao falecido}
  \end{Phonetics}
\end{Entry}

\begin{Entry}{告状}{7,7}{⼝、⽝}
  \begin{Phonetics}{告状}{gao4/zhuang4}[][HSK 7-9]
    \definition{v.+compl.}{ir à justiça contra alguém; apresentar uma queixa ao tribunal e solicitar que o caso seja aberto para julgamento | apresentar uma acusação ou queixa; informar aos pais ou superiores que você ou outras pessoas foram vítimas de \emph{bullying} ou injustiças}
  \end{Phonetics}
\end{Entry}

\begin{Entry}{告诉}{7,7}{⼝、⾔}
  \begin{Phonetics}{告诉}{gao4su4}
    \definition{v.}{dizer; informar (dar a conhecer); dizer aos outros, para que todos saibam}
  \end{Phonetics}
  \begin{Phonetics}{告诉}{gao4su5}[][HSK 1]
    \definition{v.}{dizer; informar (dar a conhecer)}
  \end{Phonetics}
\end{Entry}

\begin{Entry}{告知}{7,8}{⼝、⽮}
  \begin{Phonetics}{告知}{gao4zhi1}[][HSK 7-9]
    \definition{v.}{contar; informar; transmitir; familiarizar}
  \end{Phonetics}
\end{Entry}

\begin{Entry}{告急}{7,9}{⼝、⼼}
  \begin{Phonetics}{告急}{gao4ji2}
    \definition{v.}{estar em estado de emergência; ser crítico | relatar uma emergência; pedir ajuda de emergência | estar em uma emergência; fazer uma solicitação urgente de ajuda em uma emergência}
  \end{Phonetics}
\end{Entry}

\begin{Entry}{告诫}{7,9}{⼝、⾔}
  \begin{Phonetics}{告诫}{gao4jie4}[][HSK 7-9]
    \definition{v.}{avisar; advertir; exortar; admoestar}
  \end{Phonetics}
\end{Entry}

\begin{Entry}{告辞}{7,13}{⼝、⾟}
  \begin{Phonetics}{告辞}{gao4ci2}[][HSK 7-9]
    \definition{v.}{despedir-se (do anfitrião) | despedir-se}
  \end{Phonetics}
\end{Entry}

\begin{Entry}{员}{7}{⼝}
  \begin{Phonetics}{员}{yuan2}[][HSK 3]
    \definition{clas.}{para comandantes militares}
    \definition{s.}{uma pessoa envolvida em algum campo de atividade; refere-se a pessoas que trabalham ou estudam | membro; refere-se aos membros de um grupo ou organização | vizinhança}
  \end{Phonetics}
\end{Entry}

\begin{Entry}{员工}{7,3}{⼝、⼯}
  \begin{Phonetics}{员工}{yuan2gong1}[][HSK 3]
    \definition[位,名,个]{s.}{equipe; funcionário; trabalhador; pessoal}
  \end{Phonetics}
\end{Entry}

\begin{Entry}{呢}{8}{⼝}
  \begin{Phonetics}{呢}{ne5}[][HSK 1]
    \definition{part.}{usada no final de frases interrogativas (especificamente perguntas, perguntas de escolha e perguntas retóricas) para indicar um tom interrogativo | usada no final de uma frase declarativa, indica que uma ação ou situação está em andamento | usada em frases para indicar uma pausa (muitas vezes em pares) | usada no final de uma frase declarativa para confirmar um fato e convencer o interlocutor (com um tom de indicação e exagero)}
  \end{Phonetics}
  \begin{Phonetics}{呢}{ni2}
    \definition{s.}{(tecido feito de) lã; tecido de lã (para roupas pesadas); tecido de lã pesada; revestimento ou roupa de lã}
  \end{Phonetics}
\end{Entry}

\begin{Entry}{周}{8}{⼝}
  \begin{Phonetics}{周}{zhou1}[][HSK 2]
    \definition*{s.}{Dinastia Zhou (1046-256 BC) | Dinastia Zhou do Norte (557-581), uma das Dinastias do Norte | Dinastia Zhou Posterior (951-960), uma das Cinco Dinastias | Sobrenome Zhou}
    \definition{adj.}{universal; inteiro; por toda parte | atencioso; pensativo; completo; minucioso}
    \definition{adv.}{semanalmente}
    \definition{clas.}{usado para rodadas, voltas}
    \definition{s.}{periferia; arredores; círculo | semana | ciclo}
    \definition{v.}{fazer um circuito; mover-se em um curso circular | ajudar alguém}
  \end{Phonetics}
\end{Entry}

\begin{Entry}{周末}{8,5}{⼝、⽊}
  \begin{Phonetics}{周末}{zhou1mo4}[][HSK 2]
    \definition[个]{s.}{final-de-semana}
  \end{Phonetics}
\end{Entry}

\begin{Entry}{周年}{8,6}{⼝、⼲}
  \begin{Phonetics}{周年}{zhou1nian2}[][HSK 2]
    \definition{s.}{aniversário}
  \end{Phonetics}
\end{Entry}

\begin{Entry}{周围}{8,7}{⼝、⼞}
  \begin{Phonetics}{周围}{zhou1wei2}[][HSK 3]
    \definition{s.}{ao redor; redondeza; vizinhança; a parte ao redor do centro}
  \end{Phonetics}
\end{Entry}

\begin{Entry}{周期}{8,12}{⼝、⽉}
  \begin{Phonetics}{周期}{zhou1qi1}[][HSK 5]
    \definition[个]{s.}{período; ciclo; no processo de mudança e movimento das coisas, certas características se repetem várias vezes, com um intervalo de tempo entre cada repetição | período; ciclo; refere-se a um processo em que certas características se repetem várias vezes, e o tempo decorrido entre duas ocorrências consecutivas | classificação dos elementos na tabela periódica}
  \end{Phonetics}
\end{Entry}

\begin{Entry}{味}{8}{⼝}
  \begin{Phonetics}{味}{wei4}
    \definition{clas.}{usado para ingredientes (de uma receita de medicina chinesa)}
    \definition{s.}{gosto; sabor | cheiro; odor | interesse; deleite | acepipe; \emph{delicacy} | significância; significado}
    \definition{v.}{distinguir (provar) o sabor de; saborear}
  \end{Phonetics}
\end{Entry}

\begin{Entry}{味儿}{8,2}{⼝、⼉}
  \begin{Phonetics}{味儿}{wei4r5}[][HSK 4]
    \definition{s.}{gosto; sabor; propriedade de uma substância que dá à língua uma determinada sensação de sabor | cheiro; odor; propriedade de uma substância que dá ao nariz um determinado sentido de cheiro | interesse; significado; deleite}
  \end{Phonetics}
\end{Entry}

\begin{Entry}{味道}{8,12}{⼝、⾡}
  \begin{Phonetics}{味道}{wei4dao5}[][HSK 2]
    \definition[个,股,种]{s.}{gosto; sabor | sensação; gosto; experiência | interesse; deleite | cheiro; odor}
  \end{Phonetics}
\end{Entry}

\begin{Entry}{呵}{8}{⼝}
  \begin{Phonetics}{呵}{a1}
    \variantof{啊}
  \end{Phonetics}
  \begin{Phonetics}{呵}{he1}
    \definition{interj.}{Meu Deus!| Ah!; Oh!}
    \definition{v.}{expirar (com a boca aberta) | repreender}
  \end{Phonetics}
\end{Entry}

\begin{Entry}{呵护}{8,7}{⼝、⼿}
  \begin{Phonetics}{呵护}{he1hu4}[][HSK 7-9]
    \definition{v.}{proteger; cuidar bem de}
  \end{Phonetics}
\end{Entry}

\begin{Entry}{呶}{8}{⼝}
  \begin{Phonetics}{呶}{nao2}
    \definition{interj.}{(onomatopéia) ruído alto e contínuo}
    \definition{v.}{(literário) gritar; clamar; falar ruidosamente}
  \seealsoref{努}{nu3}
  \end{Phonetics}
\end{Entry}

\begin{Entry}{呼}{8}{⼝}
  \begin{Phonetics}{呼}{hu1}
    \definition*{s.}{Sobrenome Hu}
    \definition{s.}{Onomatopéia: descreve o som do vento}
    \definition{v.}{expirar | gritar; clamar | chamar; ligar; ligar para alguém}
  \end{Phonetics}
\end{Entry}

\begin{Entry}{呼风唤雨}{8,4,10,8}{⼝、⾵、⼝、⾬}
  \begin{Phonetics}{呼风唤雨}{hu1feng1-huan4yu3}[][HSK 7-9]
    \definition{expr.}{``Fazer vento e chover.''; refere-se originalmente ao poder mágico de imortais e taoístas; atualmente, é usado como metáfora para a capacidade de controlar a natureza e, às vezes, como metáfora para a realização de atividades inflamatórias; invocar vento e chuva --- exercer poderes mágicos; causar problemas}
  \end{Phonetics}
\end{Entry}

\begin{Entry}{呼吁}{8,6}{⼝、⼝}
  \begin{Phonetics}{呼吁}{hu1yu4}[][HSK 7-9]
    \definition{v.}{apelar; chamar; apelar a um indivíduo ou sociedade, solicitar assistência ou hospedar um apelo a um indivíduo ou sociedade, na esperança de ganhar simpatia e apoio}
  \end{Phonetics}
\end{Entry}

\begin{Entry}{呼吸}{8,6}{⼝、⼝}
  \begin{Phonetics}{呼吸}{hu1xi1}[][HSK 4]
    \definition{s.}{um suspiro; metáfora para um período de tempo muito curto}
    \definition{v.}{respirar}
  \end{Phonetics}
\end{Entry}

\begin{Entry}{呼声}{8,7}{⼝、⼠}
  \begin{Phonetics}{呼声}{hu1sheng1}[][HSK 7-9]
    \definition[片]{s.}{choro; voz}[良心的呼声。===A voz da consciência.]
  \end{Phonetics}
\end{Entry}

\begin{Entry}{呼应}{8,7}{⼝、⼴}
  \begin{Phonetics}{呼应}{hu1ying4}[][HSK 7-9]
    \definition{v.}{ecoar; trabalhar em conjunto (com alguém); entrar em contato ou cuidar um do outro um dia de cada vez}
  \end{Phonetics}
\end{Entry}

\begin{Entry}{呼唤}{8,10}{⼝、⼝}
  \begin{Phonetics}{呼唤}{hu1huan4}[][HSK 7-9]
    \definition{v.}{chamar; gritar para}
  \end{Phonetics}
\end{Entry}

\begin{Entry}{呼啦啦}{8,11,11}{⼝、⼝、⼝}
  \begin{Phonetics}{呼啦啦}{hu1 la1 la1}
    \definition{s.}{Onomatopéia: som de bater asas}
  \end{Phonetics}
\end{Entry}

\begin{Entry}{呼啸}{8,11}{⼝、⼝}
  \begin{Phonetics}{呼啸}{hu1xiao4}
    \definition{v.}{assobiar}
  \end{Phonetics}
\end{Entry}

\begin{Entry}{呼救}{8,11}{⼝、⽁}
  \begin{Phonetics}{呼救}{hu1jiu4}[][HSK 7-9]
    \definition{v.}{pedir ajuda; enviar sinais de SOS}
  \end{Phonetics}
\end{Entry}

\begin{Entry}{命}{8}{⼝}
  \begin{Phonetics}{命}{ming4}[][HSK 6]
    \definition[条]{s.}{vida | sorte; destino; fado | ordem; comando; instrução | atribuição de um nome, título etc.}
    \definition{v.}{ordenar; nomear | atribuir (um nome etc.)}
  \end{Phonetics}
\end{Entry}

\begin{Entry}{命令}{8,5}{⼝、⼈}
  \begin{Phonetics}{命令}{ming4ling4}[][HSK 5]
    \definition[条,项,道,个]{s.}{ordem; comando; instruções emitidas pelos superiores aos subordinados}
    \definition{v.}{ordenar; comandar}
  \end{Phonetics}
\end{Entry}

\begin{Entry}{命运}{8,7}{⼝、⾡}
  \begin{Phonetics}{命运}{ming4yun4}[][HSK 3]
    \definition[个]{s.}{tendência de desenvolvimento; tendência de futuro; metáfora para a direção e tendência do desenvolvimento e das mudanças | destino; sina; sorte; refere-se à vida e à morte, à riqueza e à pobreza e a todas as experiências da vida}
  \end{Phonetics}
\end{Entry}

\begin{Entry}{和}{8}{⼝}
  \begin{Phonetics}{和}{he2}[][HSK 1]
    \definition*{s.}{Sobrenome He}
    \definition{adj.}{gentil; suave; amável | harmonioso; em boas condições}
    \definition{conj.}{e (somente para palavras); unidos com}
    \definition{prep.}{relacionado com | para; com; indica correlação; comparação, etc.}
    \definition{s.}{soma; soma total | japonês; refere-se ao Japão}
    \definition{v.}{disputar; reconciliar; acabar com a guerra ou a disputa | empatar; (próxima edição ou torneio) sem vencedor}
  \end{Phonetics}
  \begin{Phonetics}{和}{he4}
    \definition{v.}{compor um poema em resposta (ao poema de alguém) usando a mesma sequência de rimas | juntar-se à cantoria; cantar junto com outros em harmonia}
  \end{Phonetics}
  \begin{Phonetics}{和}{hu2}
    \definition{v.}{completar um conjunto de Mahjong, 麻将, ou cartas de baralho}
  \seealsoref{麻将}{ma2jiang4}
  \end{Phonetics}
  \begin{Phonetics}{和}{huo2}
    \definition{v.}{combinar uma substância em pó (farinha, gesso, etc.) com água; adicionar líquido ao pó e mexer ou amassar até ficar pegajoso ou espesso}
  \end{Phonetics}
  \begin{Phonetics}{和}{huo4}
    \definition{clas.}{usado para enxágues de roupas | usado para fervuras de ervas medicinais}
    \definition{v.}{misturar (ingredientes); misturar pós ou grãos; misturar com água para obter uma consistência mais líquida}
  \end{Phonetics}
\end{Entry}

\begin{Entry}{和气}{8,4}{⼝、⽓}
  \begin{Phonetics}{和气}{he2qi5}[][HSK 7-9]
    \definition{adj.}{gentil; bondoso; educado | amigável; amável; harmonioso}
    \definition{s.}{amizade; relações harmoniosas; atmosfera harmoniosa; sentimentos harmoniosos}
  \end{Phonetics}
\end{Entry}

\begin{Entry}{和平}{8,5}{⼝、⼲}
  \begin{Phonetics}{和平}{he2ping2}[][HSK 3]
    \definition{adj.}{pacífico; moderado; não violento | pacífico; tranquilo; sereno}
    \definition{s.}{paz ;ausência de guerra}
  \end{Phonetics}
\end{Entry}

\begin{Entry}{和平共处}{8,5,6,5}{⼝、⼲、⼋、⼡}
  \begin{Phonetics}{和平共处}{he2ping2 gong4chu3}[][HSK 7-9]
    \definition{expr.}{coexistência pacífica de nações, sociedades etc.; refere-se a países com diferentes sistemas sociais que resolvem disputas pacificamente e desenvolvem laços econômicos e culturais com base na igualdade e no benefício mútuo}
  \end{Phonetics}
\end{Entry}

\begin{Entry}{和尚}{8,8}{⼝、⼩}
  \begin{Phonetics}{和尚}{he2shang5}[][HSK 7-9]
    \definition[个,名,位]{s.}{monge budista; refere-se aos monges budistas do sexo masculino que praticam o budismo}
  \end{Phonetics}
\end{Entry}

\begin{Entry}{和谐}{8,11}{⼝、⾔}
  \begin{Phonetics}{和谐}{he2xie2}[][HSK 6]
    \definition{adj.}{harmonioso; sem conflitos | em perfeita harmonia; ajuste adequado e simétrico}
    \definition{v.}{(eufemismo) censurar}
  \end{Phonetics}
\end{Entry}

\begin{Entry}{和睦}{8,13}{⼝、⽬}
  \begin{Phonetics}{和睦}{he2mu4}[][HSK 7-9]
    \definition{adj.}{harmonioso; amigável; amistoso}
    \definition{s.}{harmonia; concórdia}
  \end{Phonetics}
\end{Entry}

\begin{Entry}{和解}{8,13}{⼝、⾓}
  \begin{Phonetics}{和解}{he2jie3}[][HSK 7-9]
    \definition{v.}{reconciliar}
  \end{Phonetics}
\end{Entry}

\begin{Entry}{和蔼}{8,14}{⼝、⾋}
  \begin{Phonetics}{和蔼}{he2'ai3}[][HSK 7-9]
    \definition{adj.}{gentil; afável; amável}
  \end{Phonetics}
\end{Entry}

\begin{Entry}{咒}{8}{⼝}
  \begin{Phonetics}{咒}{zhou4}
    \definition[个,句]{s.}{encantamento; feitiço}
    \definition{v.}{amaldiçoar; condenar | maltratar; dizer que você espera que as pessoas não tenham sucesso}
  \end{Phonetics}
\end{Entry}

\begin{Entry}{咒骂}{8,9}{⼝、⾺}
  \begin{Phonetics}{咒骂}{zhou4ma4}
    \definition{v.}{xingar | amaldiçoar | execrar}
  \end{Phonetics}
\end{Entry}

\begin{Entry}{咖}{8}{⼝}
  \begin{Phonetics}{咖}{ka1}
    \definition[杯]{s.}{classe | café | graduação}
  \end{Phonetics}
\end{Entry}

\begin{Entry}{咖啡}{8,11}{⼝、⼝}
  \begin{Phonetics}{咖啡}{ka1fei1}[][HSK 3]
    \definition[杯,瓶,罐,壶,包,袋,盒]{s.}{(empréstimo linguístico) café}
  \end{Phonetics}
\end{Entry}

\begin{Entry}{咖啡色}{8,11,6}{⼝、⼝、⾊}
  \begin{Phonetics}{咖啡色}{ka1fei1 se4}
    \definition{s.}{cor café}
  \end{Phonetics}
\end{Entry}

\begin{Entry}{咖啡馆}{8,11,11}{⼝、⼝、⾷}
  \begin{Phonetics}{咖啡馆}{ka1fei1guan3}
    \definition[家]{s.}{cafeteria}
  \end{Phonetics}
\end{Entry}

\begin{Entry}{哎}{8}{⼝}
  \begin{Phonetics}{哎}{ai1}[][HSK 7-9]
    \definition{interj.}{Por que?; Ei!; Ai!; expressar surpresa ou insatisfação | Ei!; Cuidado!}
  \end{Phonetics}
\end{Entry}

\begin{Entry}{哎呀}{8,7}{⼝、⼝}
  \begin{Phonetics}{哎呀}{ai1ya1}[][HSK 7-9]
    \definition{interj.}{expressar surpresa ou espanto | expressar reclamação, impaciência, etc.}
  \end{Phonetics}
\end{Entry}

\begin{Entry}{咨}{9}{⼝}
  \begin{Phonetics}{咨}{zi1}
    \definition[行]{s.}{comunicação oficial; relatório entregue pelo chefe de um governo sobre assuntos de Estado}
    \definition{v.}{consultar; discutir com}
  \end{Phonetics}
\end{Entry}

\begin{Entry}{咨询}{9,8}{⼝、⾔}
  \begin{Phonetics}{咨询}{zi1xun2}[][HSK 6]
    \definition{v.}{consultar; aconselhar-se com; buscar conselho de; pedir conselhos}
  \end{Phonetics}
\end{Entry}

\begin{Entry}{咬}{9}{⼝}
  \begin{Phonetics}{咬}{yao3}[][HSK 5]
    \definition{v.}{morder; estalar; pressionar os dentes superiores e inferiores com força | latir | agarrar; morder | incriminar outra pessoa (geralmente inocente) quando culpada ou interrogada | pronunciar; articular; pronunciar corretamente | corroer (metais); irritar (a pele) | ser minucioso (com relação ao uso de palavras) | aproximar-se de; pressionar em direção a; avançar sobre}
  \end{Phonetics}
\end{Entry}

\begin{Entry}{咱}{9}{⼝}
  \begin{Phonetics}{咱}{za2}
  \end{Phonetics}
  \begin{Phonetics}{咱}{zan2}[][HSK 2]
    \definition{pron.}{nós; nos (incluindo tanto o falante quanto a pessoa ou pessoas às quais se dirige) | eu; mim |}
  \end{Phonetics}
  \begin{Phonetics}{咱}{zan5}
    \definition{adv.}{quando; agora; então; naquele momento; usado em 这咱, 那咱, 多咱, uma combinação das duas palavras 早晚}
  \seealsoref{多咱}{duo1 zan5}
  \seealsoref{那咱}{na4 zan5}
  \seealsoref{早晚}{zao3 wan3}
  \seealsoref{这咱}{zhe4 zan5}
  \end{Phonetics}
\end{Entry}

\begin{Entry}{咱们}{9,5}{⼝、⼈}
  \begin{Phonetics}{咱们}{zan2men5}[][HSK 2]
    \definition{pron.}{dirige-se tanto ao falante (eu, nós) quanto ao ouvinte (você, vocês) | eu; mim; refere-se ao próprio orador, eu}
  \end{Phonetics}
\end{Entry}

\begin{Entry}{咱俩}{9,9}{⼝、⼈}
  \begin{Phonetics}{咱俩}{zan2lia3}
    \definition{pron.}{nós dois}
  \end{Phonetics}
\end{Entry}

\begin{Entry}{咱家}{9,10}{⼝、⼧}
  \begin{Phonetics}{咱家}{za2jia1}
    \definition{pron.}{eu (frequentemente usado na literatura vernácula antiga) | me | mim | comigo}
  \end{Phonetics}
\end{Entry}

\begin{Entry}{咳}{9}{⼝}
  \begin{Phonetics}{咳}{hai1}
    \definition{interj.}{expressa tristeza, arrependimento ou espanto}
  \end{Phonetics}
  \begin{Phonetics}{咳}{ke2}[][HSK 5]
    \definition{v.}{tossir}
  \end{Phonetics}
\end{Entry}

\begin{Entry}{咳嗽}{9,14}{⼝、⼝}
  \begin{Phonetics}{咳嗽}{ke2sou5}
    \definition{v.}{ter tosse | tossir}
  \end{Phonetics}
\end{Entry}

\begin{Entry}{咸}{9}{⼝}
  \begin{Phonetics}{咸}{xian2}[][HSK 4]
    \definition*{s.}{Sobrenome Xian}
    \definition{adj.}{salgado; em conserva; sabor salgado}
    \definition{adv.}{todos; indica a totalidade de um intervalo, equivalente a 全 e 都}
  \seealsoref{都}{dou1}
  \seealsoref{全}{quan2}
  \end{Phonetics}
\end{Entry}

\begin{Entry}{咸水}{9,4}{⼝、⽔}
  \begin{Phonetics}{咸水}{xian2shui3}
    \definition{s.}{salmora | água salgada}
  \end{Phonetics}
\end{Entry}

\begin{Entry}{咸肉}{9,6}{⼝、⾁}
  \begin{Phonetics}{咸肉}{xian2rou4}
    \definition{s.}{\emph{bacon} | carne curada com sal}
  \end{Phonetics}
\end{Entry}

\begin{Entry}{咸鱼}{9,8}{⼝、⿂}
  \begin{Phonetics}{咸鱼}{xian2yu2}
    \definition{s.}{peixe salgado}
  \end{Phonetics}
\end{Entry}

\begin{Entry}{咸涩}{9,10}{⼝、⽔}
  \begin{Phonetics}{咸涩}{xian2se4}
    \definition{s.}{ácido | salgado e amargo}
  \end{Phonetics}
\end{Entry}

\begin{Entry}{咸盐}{9,10}{⼝、⽫}
  \begin{Phonetics}{咸盐}{xian2yan2}
    \definition{s.}{(coloquial) sal | sal de mesa}
  \end{Phonetics}
\end{Entry}

\begin{Entry}{咸淡}{9,11}{⼝、⽔}
  \begin{Phonetics}{咸淡}{xian2dan4}
    \definition{s.}{água salobra | grau de salinidade | salgado e sem sal (sabores)}
  \end{Phonetics}
\end{Entry}

\begin{Entry}{咸菜}{9,11}{⼝、⾋}
  \begin{Phonetics}{咸菜}{xian2cai4}
    \definition{s.}{legumes salgados | \emph{pickles}}
  \end{Phonetics}
\end{Entry}

\begin{Entry}{哀}{9}{⼝}
  \begin{Phonetics}{哀}{ai1}
    \definition*{s.}{Sobrenome Ai}
    \definition{adj.}{triste; pesaroso}
    \definition{adv.}{tristemente; lamentavelmente}
    \definition{s.}{luto | tristeza; pesar | pena; misericórdia}
    \definition{v.}{lamentar; lamentar-se por | Literário: estar triste}
  \end{Phonetics}
\end{Entry}

\begin{Entry}{哀求}{9,7}{⼝、⽔}
  \begin{Phonetics}{哀求}{ai1qiu2}[][HSK 7-9]
    \definition{v.}{suplicar; implorar | suplicar; implorar piedosamente}
  \end{Phonetics}
\end{Entry}

\begin{Entry}{品}{9}{⼝}
  \begin{Phonetics}{品}{pin3}[][HSK 5]
    \definition*{s.}{Sobrenome Pin}
    \definition{s.}{artigo; produto | grau; classe; classificação; nível | caráter; qualidade | classificação; os graus dos funcionários públicos antigos, num total de nove graus}
    \definition{v.}{provar; saborear; degustar algo com discernimento | soprar; tocar (instrumentos de sopro) | avaliar; distinguir o bom do ruim}
  \end{Phonetics}
\end{Entry}

\begin{Entry}{品质}{9,8}{⼝、⾙}
  \begin{Phonetics}{品质}{pin3zhi4}[][HSK 4]
    \definition[个,种]{s.}{qualidade; caráter; natureza do pensamento, da compreensão, do caráter, etc., conforme expresso no comportamento, no estilo, etc. | qualidade (de produtos, mercadorias, etc.)}
  \end{Phonetics}
\end{Entry}

\begin{Entry}{品种}{9,9}{⼝、⽲}
  \begin{Phonetics}{品种}{pin3zhong3}[][HSK 5]
    \definition[个,些]{s.}{raça; linhagem; variedade; refere-se a um grupo de organismos com características genéticas comuns, formados por meio da seleção e cultivo artificiais de culturas, gado, aves, etc. | variedade; sortimento; referência geral ao tipo de item}
  \end{Phonetics}
\end{Entry}

\begin{Entry}{品牌}{9,12}{⼝、⽚}
  \begin{Phonetics}{品牌}{pin3 pai2}[][HSK 6]
    \definition[个,种]{s.}{marca registrada; nome de marca}
  \end{Phonetics}
\end{Entry}

\begin{Entry}{品德}{9,15}{⼝、⼻}
  \begin{Phonetics}{品德}{pin3de2}
    \definition{s.}{caráter moral | moralidade}
  \end{Phonetics}
\end{Entry}

\begin{Entry}{哄}{9}{⼝}
  \begin{Phonetics}{哄}{hong1}[][HSK 7-9]
    \definition{interj.}{Onomatopéia: gargalhadas ou alvoroço}
    \definition{s.}{rugido; clamor; comoção}
  \end{Phonetics}
  \begin{Phonetics}{哄}{hong3}[][HSK 7-9]
    \definition{v.}{brincar; enganar; tapear | persuadir; agradar os outros com palavras ou ações, especialmente observando ou cuidando de crianças}
  \end{Phonetics}
  \begin{Phonetics}{哄}{hong4}[][HSK 7-9]
    \definition{s.}{comoção; tumulto}
  \end{Phonetics}
\end{Entry}

\begin{Entry}{哄堂大笑}{9,11,3,10}{⼝、⼟、⼤、⽵}
  \begin{Phonetics}{哄堂大笑}{hong1tang2-da4xiao4}[][HSK 7-9]
    \definition{expr.}{fazer a sala inteira rugir (em alvoroço); (causar) uma explosão geral de risos; ``Todos caíram na gargalhada.''; uma explosão de risos na plateia; ``A plateia caiu na gargalhada.''; ``As pessoas de toda a casa caíram na gargalhada.''; ``A sala inteira riu (balançando).''}
  \end{Phonetics}
\end{Entry}

\begin{Entry}{哆}{9}{⼝}
  \begin{Phonetics}{哆}{duo1}
    \definition{part.}{usado em 哆嗦}
  \seealsoref{哆嗦}{duo1suo5}
  \end{Phonetics}
\end{Entry}

\begin{Entry}{哆嗦}{9,13}{⼝、⼝}
  \begin{Phonetics}{哆嗦}{duo1suo5}[][HSK 7-9]
    \definition{v.}{tremer; estremecer (tremores corporais involuntários devido a estímulos externos)}
  \end{Phonetics}
\end{Entry}

\begin{Entry}{哇}{9}{⼝}
  \begin{Phonetics}{哇}{wa1}
    \definition{interj.}{(onomatopéia) som de choro ou vômito | Uau!; expressa surpresa}
  \end{Phonetics}
  \begin{Phonetics}{哇}{wa5}[][HSK 6]
    \definition{part.}{a mudança do som de 啊 devido à influência do som final da palavra anterior, ``u'' ou ``ao''}
  \seealsoref{啊}{a5}
  \end{Phonetics}
\end{Entry}

\begin{Entry}{哇塞}{9,13}{⼝、⼟}
  \begin{Phonetics}{哇塞}{wa1sai1}
    \definition{interj.}{Uau!; uma exclamação de espanto, admiração, etc.}
  \end{Phonetics}
\end{Entry}

\begin{Entry}{哇噻}{9,16}{⼝、⼝}
  \begin{Phonetics}{哇噻}{wa1sai1}
    \variantof{哇塞}
  \end{Phonetics}
\end{Entry}

\begin{Entry}{哈}{9}{⼝}
  \begin{Phonetics}{哈}{ha1}
    \definition{interj.}{Onomatopéia: ha; descreve o riso, usado principalmente em duplicata | indica orgulho ou satisfação, frequentemente usado de forma duplicada}
    \definition{v.}{soprar; expirar (com a boca aberta) | dobrar}
  \seealsoref{哈哈}{ha1 ha1}
  \end{Phonetics}
  \begin{Phonetics}{哈}{ha3}
    \definition*{s.}{Sobrenome Ha}
    \definition{v.}{repreender}
  \end{Phonetics}
\end{Entry}

\begin{Entry}{哈马斯}{9,3,12}{⼝、⾺、⽄}
  \begin{Phonetics}{哈马斯}{ha1ma3si1}
    \definition*{s.}{Hamas (Grupo Palestino)}
  \end{Phonetics}
\end{Entry}

\begin{Entry}{哈哈}{9,9}{⼝、⼝}
  \begin{Phonetics}{哈哈}{ha1 ha1}[][HSK 3]
    \definition{expr.}{(onomatopéia)  ha ha; o som de uma gargalhada}
  \end{Phonetics}
\end{Entry}

\begin{Entry}{响}{9}{⼝}
  \begin{Phonetics}{响}{xiang3}[][HSK 2]
    \definition{adj.}{barulhento; ressonante}
    \definition[声,阵]{s.}{som; ruído; barulho | eco}
    \definition{v.}{tocar; soar; ressoar; fazer um som | soar; fazer algo emitir um som}
  \end{Phonetics}
\end{Entry}

\begin{Entry}{响声}{9,7}{⼝、⼠}
  \begin{Phonetics}{响声}{xiang3 sheng1}[][HSK 6]
    \definition{s.}{som; ruído}
  \end{Phonetics}
\end{Entry}

\begin{Entry}{响亮}{9,9}{⼝、⼇}
  \begin{Phonetics}{响亮}{xiang3liang4}
    \definition{adj.}{vibrante; ressonante; sonoro; ressonante; alto e claro}
  \end{Phonetics}
\end{Entry}

\begin{Entry}{哗}{9}{⼝}
  \begin{Phonetics}{哗}{hua1}
    \definition{s.}{(onomatopéia) sons de impacto, batida, fluxo de água, etc.}
  \end{Phonetics}
  \begin{Phonetics}{哗}{hua2}
    \definition{v.}{ser barulhento; fazer alvoroço}
  \end{Phonetics}
\end{Entry}

\begin{Entry}{哗变}{9,8}{⼝、⼜}
  \begin{Phonetics}{哗变}{hua2bian4}[][HSK 7-9]
    \definition{s.}{(um exército) motim | rebelião}
  \end{Phonetics}
\end{Entry}

\begin{Entry}{哗啦啦}{9,11,11}{⼝、⼝、⼝}
  \begin{Phonetics}{哗啦啦}{hua1la1 la5}
    \definition{s.}{(onomatopéia) som de colisão, batida}
  \end{Phonetics}
\end{Entry}

\begin{Entry}{哗然}{9,12}{⼝、⽕}
  \begin{Phonetics}{哗然}{hua2ran2}[][HSK 7-9]
    \definition{adj.}{Literário: barulhento; em alvoroço; em comoção}[举座哗然。===Todo o público ficou em alvoroço.]
  \end{Phonetics}
\end{Entry}

\begin{Entry}{哪}{9}{⼝}
  \begin{Phonetics}{哪}{na3}[][HSK 1,4]
    \definition{adv.}{para expressar uma pergunta retórica, indicando que é impossível}
    \definition{pron.}{qual?; o que?; expressa a necessidade de determinar um entre várias pessoas ou coisas | qualquer; ser usado em um sentido geral | qual?; o que?; (usado sozinho, o mesmo que 什么, frequentemente usado de forma intercambiável com 什么) | qualquer; qualquer que seja; refere-se a qualquer um, geralmente seguido por 都 ou 也, ou usando dois 哪 antes e depois | qual (indica algo incerto)}
  \seealsoref{都}{dou1}
  \seealsoref{什么}{shen2me5}
  \seealsoref{也}{ye3}
  \end{Phonetics}
  \begin{Phonetics}{哪}{na5}
    \definition{part.}{usado depois de uma palavra com a terminação -n, é equivalente a 啊}
  \seealsoref{啊}{a5}
  \end{Phonetics}
  \begin{Phonetics}{哪}{nei3}
    \definition{part.}{qual? (interrogativo, seguido de classificador ou numeral-classificador)}
  \end{Phonetics}
\end{Entry}

\begin{Entry}{哪儿}{9,2}{⼝、⼉}
  \begin{Phonetics}{哪儿}{na3r5}[][HSK 1]
    \definition{adv.}{usado para perguntas retóricas, indicando negação}
    \definition{pron.}{onde? | onde quer que seja; em qualquer lugar | usado como uma resposta educada a um elogio}
  \end{Phonetics}
\end{Entry}

\begin{Entry}{哪个}{9,3}{⼝、⼈}
  \begin{Phonetics}{哪个}{na3ge5}
    \definition{pron.}{qual deles (pergunta sobre o objeto) | quem (perguntar a alguém ou indicar qualquer pessoa)}
  \end{Phonetics}
\end{Entry}

\begin{Entry}{哪里}{9,7}{⼝、⾥}
  \begin{Phonetics}{哪里}{na3 li3}[][HSK 1]
    \definition{adv.}{usado em perguntas retóricas para expressar um significado negativo}
    \definition{pron.}{onde?; em que lugar? | onde quer que seja; em qualquer lugar | usado como uma resposta educada a um elogio}
  \end{Phonetics}
\end{Entry}

\begin{Entry}{哪些}{9,8}{⼝、⼆}
  \begin{Phonetics}{哪些}{na3xie1}[][HSK 1]
    \definition{pron.}{quais?}
  \end{Phonetics}
\end{Entry}

\begin{Entry}{哪国人}{9,8,2}{⼝、⼞、⼈}
  \begin{Phonetics}{哪国人}{na3 guo2ren2}
    \definition{expr.}{de qual país?}
  \end{Phonetics}
\end{Entry}

\begin{Entry}{哪怕}{9,8}{⼝、⼼}
  \begin{Phonetics}{哪怕}{na3pa4}[][HSK 4]
    \definition{conj.}{mesmo; mesmo se; mesmo que; não importa o quão}
  \end{Phonetics}
\end{Entry}

\begin{Entry}{哥}{10}{⼝}
  \begin{Phonetics}{哥}{ge1}[][HSK 1]
    \definition[个,位,名,些]{s.}{irmão mais velho | forma de se dirigir a um parente masculino mais velho de sua geração | irmão; termo amigável para se dirigir a conhecidos mais velhos do sexo masculino}
  \seealsoref{哥哥}{ge1 ge5}
  \end{Phonetics}
\end{Entry}

\begin{Entry}{哥们}{10,5}{⼝、⼈}
  \begin{Phonetics}{哥们}{ge1men5}
    \definition{expr.}{\emph{Brothers!}}
    \definition{s.}{(coloquial) cara | irmão (forma diminuta de tratamento entre homens)}
  \end{Phonetics}
\end{Entry}

\begin{Entry}{哥哥}{10,10}{⼝、⼝}
  \begin{Phonetics}{哥哥}{ge1 ge5}[][HSK 1]
    \definition[个,位]{s.}{irmão mais velho | primo}
  \end{Phonetics}
\end{Entry}

\begin{Entry}{哥斯拉}{10,12,8}{⼝、⽄、⼿}
  \begin{Phonetics}{哥斯拉}{ge1si1la1}
    \definition*{s.}{Godzilla}
  \seealsoref{酷斯拉}{ku4si1la1}
  \end{Phonetics}
\end{Entry}

\begin{Entry}{哦}{10}{⼝}
  \begin{Phonetics}{哦}{e2}
    \definition{v.}{cantar suavemente (um poema)}
  \end{Phonetics}
  \begin{Phonetics}{哦}{o2}
    \definition{interj.}{Oh! (indicando dúvida ou surpresa)}
  \end{Phonetics}
  \begin{Phonetics}{哦}{o4}
    \definition{interj.}{Oh! (indicando que acabou de aprender algo)}
  \end{Phonetics}
  \begin{Phonetics}{哦}{o5}
    \definition{part.}{usada no final da frase para indicar que uma pessoa está afirmando um fato que a outra pessoa não sabe | usada no final da frase para transmitir informalidade, calor, simpatia ou intimidade}
  \end{Phonetics}
\end{Entry}

\begin{Entry}{哩}{10}{⼝}
  \begin{Phonetics}{哩}{li3}
    \definition{clas.}{milha (unidade de comprimento igual a 1.609,344 m)}
  \end{Phonetics}
  \begin{Phonetics}{哩}{li5}
    \definition{part.}{(dialeto) final modal semelhante a 呢 ou 啦, usado em um tom definido, mas um tanto exagerado}
  \seealsoref{啦}{la5}
  \seealsoref{呢}{ne5}
  \end{Phonetics}
\end{Entry}

\begin{Entry}{哩哩啦啦}{10,10,11,11}{⼝、⼝、⼝、⼝}
  \begin{Phonetics}{哩哩啦啦}{li1 li1 la1 la1}
    \definition{adj.}{espalhado; disperso; disseminado; difuso; esporádico; aqui e ali}
  \end{Phonetics}
\end{Entry}

\begin{Entry}{哭}{10}{⼝}
  \begin{Phonetics}{哭}{ku1}[][HSK 2]
    \definition{v.}{chorar; soluçar; lamentar-se; chorar de dor ou emoção}
  \end{Phonetics}
\end{Entry}

\begin{Entry}{哭墙}{10,14}{⼝、⼟}
  \begin{Phonetics}{哭墙}{ku1qiang2}
    \definition*{s.}{Muro das Lamentações (Jerusalém)}
  \end{Phonetics}
\end{Entry}

\begin{Entry}{哮}{10}{⼝}
  \begin{Phonetics}{哮}{xiao4}
    \definition{s.}{respiração pesada; chiado}
    \definition{v.}{rugir; uivar}
  \end{Phonetics}
\end{Entry}

\begin{Entry}{哮喘}{10,12}{⼝、⼝}
  \begin{Phonetics}{哮喘}{xiao4chuan3}
    \definition{s.}{asma; sintomas de dispneia: os pacientes sentem que a respiração está muito difícil; pneumonia, insuficiência cardíaca, bronquite crônica e outras doenças causadas por espasmo da musculatura lisa respiratória frequentemente apresentam esse sintoma}
    \definition{v.}{sofrer de asma}
  \end{Phonetics}
\end{Entry}

\begin{Entry}{哲}{10}{⼝}
  \begin{Phonetics}{哲}{zhe2}
    \definition{adj.}{sábio; sagaz}
    \definition[位,名,个]{s.}{pessoas sábias; sábio |sabedoria}
  \end{Phonetics}
\end{Entry}

\begin{Entry}{哲学}{10,8}{⼝、⼦}
  \begin{Phonetics}{哲学}{zhe2xue2}[][HSK 6]
    \definition{s.}{filosofia; é uma disciplina que explora questões fundamentais e conceitos básicos}
  \end{Phonetics}
\end{Entry}

\begin{Entry}{哲理}{10,11}{⼝、⽟}
  \begin{Phonetics}{哲理}{zhe2li3}
    \definition{s.}{filosofia | teoria filosófica}
  \end{Phonetics}
\end{Entry}

\begin{Entry}{哺}{10}{⼝}
  \begin{Phonetics}{哺}{bu3}
    \definition{s.}{comida na boca; chorando por comida | alimentos de mastigação; mastigando comida}
    \definition{v.}{alimentar (um bebê); amamentar}
  \end{Phonetics}
\end{Entry}

\begin{Entry}{哺育}{10,8}{⼝、⾁}
  \begin{Phonetics}{哺育}{bu3yu4}[][HSK 7-9]
    \definition{v.}{alimentar | Figurativo: nutrir; fomentar | desenvolver}
  \end{Phonetics}
\end{Entry}

\begin{Entry}{哼}{10}{⼝}
  \begin{Phonetics}{哼}{heng1}[][HSK 7-9]
    \definition{v.}{gemer; bufar | cantarolar}
  \end{Phonetics}
  \begin{Phonetics}{哼}{hng5}
    \definition{interj.}{Hmm; Humph; expressa insatisfação, desprezo, desdém ou indignação}
  \end{Phonetics}
\end{Entry}

\begin{Entry}{唇}{10}{⼝}
  \begin{Phonetics}{唇}{chun2}
    \definition[片]{s.}{lábios}
  \end{Phonetics}
\end{Entry}

\begin{Entry}{唉}{10}{⼝}
  \begin{Phonetics}{唉}{ai1}
    \definition{interj.}{Sim!; Certo!; Bem! | Ai de mim!; o som dos suspiros}
  \end{Phonetics}
  \begin{Phonetics}{唉}{ai4}[][HSK 7-9]
    \definition{interj.}{Oh!; Ah!; Bem!; interjeição que expressa tristeza ou arrependimento | Bem!; Argh!; usado para responder ou reconhecer}
  \end{Phonetics}
\end{Entry}

\begin{Entry}{唐}{10}{⼝}
  \begin{Phonetics}{唐}{tang2}
    \definition*{s.}{Dinastia estabelecida pelo Imperador Yao, 尧, no período lendário da história chinesa | Dinastia Tang (618-907) | Dinastia Tang posterior (923-936), uma das cinco dinastias | Sobrenome Tang}
    \definition{adj.}{exagerado; bombástico; orgulhoso | em vão; por nada}
  \seealsoref{尧}{yao2}
  \end{Phonetics}
\end{Entry}

\begin{Entry}{唐人街}{10,2,12}{⼝、⼈、⾏}
  \begin{Phonetics}{唐人街}{tang2ren2 jie1}
    \definition*[条,座]{s.}{Bairro Chinês; Chinatown; refere-se ao mercado de rua onde os chineses do exterior vivem e abrem muitas lojas com características chinesas}
  \seealsoref{中国城}{zhong1guo2cheng2}
  \end{Phonetics}
\end{Entry}

\begin{Entry}{唐初四大家}{10,7,5,3,10}{⼝、⾐、⼞、⼤、⼧}
  \begin{Phonetics}{唐初四大家}{tang2 chu1 si4 da4jia1}
    \definition*{s.}{Quatro grandes calígrafos do início da dinastia Tang; refere-se a Yu Shi'nan 虞世南, Ouyang Xun 欧阳询, Chu Suiliang 褚遂良 e Xue Ji 薛稷}
  \seealsoref{褚遂良}{chu3 sui4liang2}
  \seealsoref{欧阳询}{ou1yang2 xun2}
  \seealsoref{薛稷}{xue1 ji4}
  \seealsoref{虞世南}{yu2 shi4'nan2}
  \end{Phonetics}
\end{Entry}

\begin{Entry}{唤}{10}{⼝}
  \begin{Phonetics}{唤}{huan4}
    \definition{v.}{chamar; fazer um barulho alto para fazer a outra parte acordar, prestar atenção ou vir até você}
  \end{Phonetics}
\end{Entry}

\begin{Entry}{唤起}{10,10}{⼝、⾛}
  \begin{Phonetics}{唤起}{huan4qi3}[][HSK 7-9]
    \definition{v.}{despertar | chamar; evocar}
  \end{Phonetics}
\end{Entry}

\begin{Entry}{啊}{10}{⼝}
  \begin{Phonetics}{啊}{a1}[][HSK 2]
    \definition{interj.}{Ah!; Oh!; expressar surpresa ou admiração}
  \end{Phonetics}
  \begin{Phonetics}{啊}{a2}[][HSK 2]
    \definition{interj.}{Eh?; Ei?; Que?; Por que?; expressar questionamento, dúvida ou solicitar opinião}
  \end{Phonetics}
  \begin{Phonetics}{啊}{a3}[][HSK 2]
    \definition{interj.}{Eh?; Meu!; E aí?; Que?; expressar surpresa e dúvida}
  \end{Phonetics}
  \begin{Phonetics}{啊}{a4}[][HSK 2]
    \definition{interj.}{Bem!; Sim!; expressa concordância, pronúncia mais curta | Oh!; Ah!; indica que compreendeu, com pronúncia mais longa | Oh!; expressa surpresa ou admiração, com pronúncia mais longa | Desgraça!; expressa tristeza ou pesar}
  \end{Phonetics}
  \begin{Phonetics}{啊}{a5}[][HSK 2,4]
    \definition{part.}{usado no final da frase para expressar admiração | usado no final da frase para expressar afirmação, justificativa, insistência, recomendação, etc. | usado no final da frase para indicar dúvida | usado para fazer uma pequena pausa na frase, chamando a atenção para o que vem a seguir | usado após os itens enumerados | usado após verbos repetitivos, indica um processo longo}
  \end{Phonetics}
\end{Entry}

\begin{Entry}{啊呀}{10,7}{⼝、⼝}
  \begin{Phonetics}{啊呀}{a1ya1}
    \definition{interj.}{Oh meu Deus! | interjeição de surpresa}
  \end{Phonetics}
\end{Entry}

\begin{Entry}{啊哟}{10,9}{⼝、⼝}
  \begin{Phonetics}{啊哟}{a1yo5}
    \definition{interj.}{Meu Deus! | Oh! | Ai! | interjeição de surpresa ou dor}
  \end{Phonetics}
\end{Entry}

\begin{Entry}{唬}{11}{⼝}
  \begin{Phonetics}{唬}{hu3}
    \definition{v.}{blefar, exagerar para assustar ou confundir}
  \end{Phonetics}
\end{Entry}

\begin{Entry}{售}{11}{⼝}
  \begin{Phonetics}{售}{shou4}
    \definition{v.}{vender | fazer (o plano, truque, etc.) funcionar; continuar (as intrigas) | realizar (intrigas)}
  \end{Phonetics}
\end{Entry}

\begin{Entry}{售货员}{11,8,7}{⼝、⾙、⼝}
  \begin{Phonetics}{售货员}{shou4huo4yuan2}[][HSK 4]
    \definition[名,位]{s.}{vendedor; balconista; assistente de loja; equipe que vende produtos em lojas}
  \end{Phonetics}
\end{Entry}

\begin{Entry}{唯}{11}{⼝}
  \begin{Phonetics}{唯}{wei2}
    \definition{adv.}{somente; sozinho | ainda; somente; exceto que}
  \end{Phonetics}
  \begin{Phonetics}{唯}{wei3}
    \definition{interj.}{Sim!; Yea!; significa uma palavra que indica acordo}
  \end{Phonetics}
\end{Entry}

\begin{Entry}{唯一}{11,1}{⼝、⼀}
  \begin{Phonetics}{唯一}{wei2yi1}[][HSK 5]
    \definition{adj.}{único; exclusivo; singular; apenas um; nenhum outro}
  \end{Phonetics}
\end{Entry}

\begin{Entry}{唱}{11}{⼝}
  \begin{Phonetics}{唱}{chang4}[][HSK 1]
    \definition*{s.}{Sobrenome Chang}
    \definition{s.}{uma música ou uma parte cantada de uma ópera chinesa; canções; letras de óperas tradicionais}
    \definition{v.}{cantar; seguir o ritmo da música | chorar; chamar; gritar, falar ou recitar em voz alta}
  \end{Phonetics}
\end{Entry}

\begin{Entry}{唱片}{11,4}{⼝、⽚}
  \begin{Phonetics}{唱片}{chang4 pian4}[][HSK 4]
    \definition[枚,张]{s.}{disco; disco feito de goma-laca, plástico, etc. com ranhuras em espiral na superfície para registrar alterações no som que podem reproduzir o som gravado em um fonógrafo}
  \end{Phonetics}
\end{Entry}

\begin{Entry}{唱歌}{11,14}{⼝、⽋}
  \begin{Phonetics}{唱歌}{chang4/ge1}[][HSK 1]
    \definition{v.+compl.}{cantar (uma música); emitir sons com entonação ritmada e melodiosa; emitir sons (musicais) com a boca; emitir sons de acordo com a melodia}
  \end{Phonetics}
\end{Entry}

\begin{Entry}{唾}{11}{⼝}
  \begin{Phonetics}{唾}{tuo4}
    \definition[口]{s.}{saliva; cuspe}
    \definition{v.}{cuspir (mostrar desprezo) | rejeitar}
  \end{Phonetics}
\end{Entry}

\begin{Entry}{唾骂}{11,9}{⼝、⾺}
  \begin{Phonetics}{唾骂}{tuo4ma4}
    \definition{v.}{insultar | amaldiçoar}
  \end{Phonetics}
\end{Entry}

\begin{Entry}{商}{11}{⼝}
  \begin{Phonetics}{商}{shang1}
    \definition*{s.}{Dinastia Shang (1600-1046 a.C.) | Shang, nome da estrela da constelação do coração entre as vinte e oito constelações | Sobrenome Shang}
    \definition{s.}{comércio; negócio; a atividade econômica de compra e venda de mercadorias | comerciante; negociante; comerciante; empresário; pessoas que compram e vendem mercadorias | (matemática) quociente;  o resultado de uma operação de divisão em aritmética | uma nota da antiga escala chinesa de cinco tons, correspondente a 2 na notação musical numerada}
    \definition{v.}{discutir; consultar; trocar ideias}
  \end{Phonetics}
\end{Entry}

\begin{Entry}{商人}{11,2}{⼝、⼈}
  \begin{Phonetics}{商人}{shang1 ren2}[][HSK 2]
    \definition[位,名]{s.}{comerciante; mercador; empresário; homem de negócios; pessoas que trabalham com a distribuição de mercadorias}
  \end{Phonetics}
\end{Entry}

\begin{Entry}{商业}{11,5}{⼝、⼀}
  \begin{Phonetics}{商业}{shang1ye4}[][HSK 3]
    \definition[个,种]{s.}{barganha; negócio; comércio; atividade econômica que circula mercadorias por meio de compra e venda}
  \end{Phonetics}
\end{Entry}

\begin{Entry}{商务}{11,5}{⼝、⼒}
  \begin{Phonetics}{商务}{shang1wu4}[][HSK 4]
    \definition[种,类,项]{s.}{negócios; assuntos de negócios; assuntos comerciais}
  \end{Phonetics}
\end{Entry}

\begin{Entry}{商场}{11,6}{⼝、⼟}
  \begin{Phonetics}{商场}{shang1 chang3}[][HSK 1]
    \definition[家]{s.}{mercado; shopping center; loja de departamentos; loja de grande área com uma variedade completa de produtos | o mundo dos negócios; referindo-se ao mundo dos negócios | mercado; mercado composto por várias lojas reunidas em um ou vários edifícios interligados}
  \end{Phonetics}
\end{Entry}

\begin{Entry}{商店}{11,8}{⼝、⼴}
  \begin{Phonetics}{商店}{shang1dian4}[][HSK 1]
    \definition[间,家,个]{s.}{loja; armazém; local de venda de mercadorias em recinto fechado}
  \end{Phonetics}
\end{Entry}

\begin{Entry}{商品}{11,9}{⼝、⼝}
  \begin{Phonetics}{商品}{shang1pin3}[][HSK 3]
    \definition[种,个,件,批]{s.}{bens; mercadoria; \emph{merchande}; os produtos do trabalho produzidos para troca têm a dupla natureza de valor de uso e valor; as mercadorias incorporam diferentes relações de produção em diferentes sistemas sociais}
  \end{Phonetics}
\end{Entry}

\begin{Entry}{商城}{11,9}{⼝、⼟}
  \begin{Phonetics}{商城}{shang1 cheng2}[][HSK 6]
    \definition{s.}{um mercado; um centro comercial; um \emph{shopping center}; refere-se a um complexo comercial contíguo com um grande espaço de construção}
  \end{Phonetics}
\end{Entry}

\begin{Entry}{商标}{11,9}{⼝、⽊}
  \begin{Phonetics}{商标}{shang1biao1}[][HSK 5]
    \definition[个]{s.}{marca; marca registrada; \emph{trademark}; marca ou símbolo (desenho, padrão, texto, etc.) gravado ou impresso na superfície ou embalagem de um produto, para diferenciá-lo de outros produtos semelhantes}
  \end{Phonetics}
\end{Entry}

\begin{Entry}{商贸}{11,9}{⼝、⾙}
  \begin{Phonetics}{商贸}{shang1mao4}
    \definition{s.}{comércio}
  \end{Phonetics}
\end{Entry}

\begin{Entry}{商量}{11,12}{⼝、⾥}
  \begin{Phonetics}{商量}{shang1liang5}[][HSK 2]
    \definition{v.}{consultar; discutir; conversar sobre; discutir e trocar opiniões}
  \end{Phonetics}
\end{Entry}

\begin{Entry}{啤}{11}{⼝}
  \begin{Phonetics}{啤}{pi2}
    \definition{s.}{cerveja}
  \end{Phonetics}
\end{Entry}

\begin{Entry}{啤酒}{11,10}{⼝、⾣}
  \begin{Phonetics}{啤酒}{pi2jiu3}[][HSK 3]
    \definition[杯,瓶,罐,桶,缸]{s.}{(empréstimo linguístico) cerveja; uma bebida de baixo teor alcoólico feita de malte de cevada e lúpulo, com espuma e aroma especial}
  \end{Phonetics}
\end{Entry}

\begin{Entry}{啤酒馆}{11,10,11}{⼝、⾣、⾷}
  \begin{Phonetics}{啤酒馆}{pi2jiu3guan3}
    \definition{s.}{cervejaria}
  \end{Phonetics}
\end{Entry}

\begin{Entry}{啥}{11}{⼝}
  \begin{Phonetics}{啥}{sha2}
    \definition{pron.}{Dialeto: O que?; equivalente a 什么}
  \end{Phonetics}
\end{Entry}

\begin{Entry}{啦}{11}{⼝}
  \begin{Phonetics}{啦}{la1}
    \definition{s.}{(onomatoméia) som de canto, aplausos etc.; usado para palavras como 呼啦啦, 哗啦啦, 哩哩啦啦, etc.}
  \seealsoref{呼啦啦}{hu1 la1 la1}
  \seealsoref{哗啦啦}{hua1la1 la5}
  \seealsoref{哩哩啦啦}{li1 li1 la1 la1}
  \end{Phonetics}
  \begin{Phonetics}{啦}{la5}[][HSK 6]
    \definition{part.}{uma palavra composta de 了 e 啊, que tem o significado de ambos}
  \seealsoref{啊}{a5}
  \seealsoref{了}{le5}
  \end{Phonetics}
\end{Entry}

\begin{Entry}{啵}{11}{⼝}
  \begin{Phonetics}{啵}{bo1}
    \definition{part.}{denotando pedido, comando, etc.; o uso é semelhante ao de 吧, que é mais comum no vernáculo antigo}
    \definition{v.aux.}{indicando uma sugestão, pedido ou comando suave | indicando consentimento ou aprovação | em uma pergunta tendenciosa que pede a confirmação de uma suposição | indicando alguma dúvida na mente do falante | marcando uma pausa após suposições como alternativas}
  \seealsoref{吧}{ba5}
  \end{Phonetics}
  \begin{Phonetics}{啵}{bo5}
    \definition{part.}{partícula gramaticalmente equivalente a 吧}
  \seealsoref{吧}{ba5}
  \end{Phonetics}
\end{Entry}

\begin{Entry}{喂}{12}{⼝}
  \begin{Phonetics}{喂}{wei4}[][HSK 2,4]
    \definition{interj.}{Ei!, Olá!, para chamar atenção | Alô? (quando respondendo uma chamada telefônica, pronuncia-se como \dpy{wei2})}
    \definition{v.}{criar; alimentar (animais); dar comida a um animal | alimentar (pessoas); colocar alimentos, medicamentos, etc. na boca de alguém}
  \end{Phonetics}
\end{Entry}

\begin{Entry}{喂奶}{12,5}{⼝、⼥}
  \begin{Phonetics}{喂奶}{wei4nai3}
    \definition{v.}{amamentar}
  \end{Phonetics}
\end{Entry}

\begin{Entry}{喂母乳}{12,5,8}{⼝、⽏、⼄}
  \begin{Phonetics}{喂母乳}{wei4mu3ru3}
    \definition{s.}{amamentação}
  \end{Phonetics}
\end{Entry}

\begin{Entry}{喂养}{12,9}{⼝、⼋}
  \begin{Phonetics}{喂养}{wei4yang3}
    \definition{v.}{alimentar (uma criança, animal doméstico, etc.) | manter | criar (um animal)}
  \end{Phonetics}
\end{Entry}

\begin{Entry}{喂食}{12,9}{⼝、⾷}
  \begin{Phonetics}{喂食}{wei4shi2}
    \definition{v.}{alimentar}
  \end{Phonetics}
\end{Entry}

\begin{Entry}{喂哺}{12,10}{⼝、⼝}
  \begin{Phonetics}{喂哺}{wei4bu3}
    \definition{v.}{alimentar (um bebê)}
  \end{Phonetics}
\end{Entry}

\begin{Entry}{喂料}{12,10}{⼝、⽃}
  \begin{Phonetics}{喂料}{wei4liao4}
    \definition{v.}{alimentar (também no sentido figurativo)}
  \end{Phonetics}
\end{Entry}

\begin{Entry}{善}{12}{⼝}
  \begin{Phonetics}{善}{shan4}
    \definition*{s.}{Sobrenome Shan}
    \definition{adj.}{bom; bem | bom; satisfatório | gentil; amigável | familiar}
    \definition{adv.}{bom; bem}
    \definition{s.}{boa ação; ato benevolente; coisas boas (em oposição a 恶)}
    \definition{v.}{fazer sucesso; fazer bem; fazer acontecer | ser bom em; ser especialista (versado) em | ser apto a}
  \seealsoref{恶}{e4}
  \end{Phonetics}
\end{Entry}

\begin{Entry}{善于}{12,3}{⼝、⼆}
  \begin{Phonetics}{善于}{shan4yu2}[][HSK 4]
    \definition{adv./v.}{ser bom em; ser hábil em}
  \end{Phonetics}
\end{Entry}

\begin{Entry}{善良}{12,7}{⼝、⾉}
  \begin{Phonetics}{善良}{shan4liang2}[][HSK 4]
    \definition{adj.}{de bom coração; bom e honesto; de bom coração e cheio de boa vontade}
  \end{Phonetics}
\end{Entry}

\begin{Entry}{善意}{12,13}{⼝、⼼}
  \begin{Phonetics}{善意}{shan4yi4}
    \definition{s.}{boa vontade | benevolência | bondade}
  \end{Phonetics}
\end{Entry}

\begin{Entry}{喉}{12}{⼝}
  \begin{Phonetics}{喉}{hou2}
    \definition{s.}{laringe; garganta; a parte do órgão respiratório de humanos e vertebrados terrestres, localizada entre a faringe e a traqueia, tem as funções de ventilação e pronúncia; a faringe e a laringe são geralmente misturadas e chamadas de garganta ou caixa vocal}
  \end{Phonetics}
\end{Entry}

\begin{Entry}{喉咙}{12,8}{⼝、⼝}
  \begin{Phonetics}{喉咙}{hou2long2}[][HSK 7-9]
    \definition{s.}{garganta; laringe}
  \end{Phonetics}
\end{Entry}

\begin{Entry}{喊}{12}{⼝}
  \begin{Phonetics}{喊}{han3}[][HSK 2]
    \definition{v.}{gritar; clamar; berrar | chamar (uma pessoa) | chamar; dirigir-se a}
  \end{Phonetics}
\end{Entry}

\begin{Entry}{喔}{12}{⼝}
  \begin{Phonetics}{喔}{o1}
    \definition{interj.}{Oh!, Entendi!, usado para indicar realização, compreensão}
  \end{Phonetics}
\end{Entry}

\begin{Entry}{喘}{12}{⼝}
  \begin{Phonetics}{喘}{chuan3}[][HSK 7-9]
    \definition{s.}{Medicina: asma}
    \definition{v.}{respirar pesadamente; ofegar por ar; ofegar | sopro; respiração rápida}
  \end{Phonetics}
\end{Entry}

\begin{Entry}{喘息}{12,10}{⼝、⼼}
  \begin{Phonetics}{喘息}{chuan3xi1}[][HSK 7-9]
    \definition{s.}{sopro; respiração rápida |respirador; pausas curtas durante atividades intensas | síndrome caracterizada por dispneia | (ponto de acupuntura) chuanxi}
    \definition{v.}{ofegar; ofegar por ar | fazer uma pausa para respirar; fazer uma pausa}
  \end{Phonetics}
\end{Entry}

\begin{Entry}{喜}{12}{⼝}
  \begin{Phonetics}{喜}{xi3}
    \definition{adj.}{feliz; satisfeito; encantado}
    \definition[桩,件]{s.}{evento feliz (especialmente casamento); ocasião para celebração; algo para comemorar | gravidez | casamento ou coisas relacionadas a ele}
    \definition{v.}{gostar; fonte de; ter inclinação para | precisa; requer; combina melhor com; (um certo organismo) precisa ou é adequado para (um certo ambiente ou algo)}
  \end{Phonetics}
\end{Entry}

\begin{Entry}{喜欢}{12,6}{⼝、⽋}
  \begin{Phonetics}{喜欢}{xi3huan5}[][HSK 1]
    \definition{adj.}{feliz; encantado; exultante; cheio de alegria}
    \definition{v.}{gostar; amar; ter afeição por; estar interessado em; ter uma boa impressão ou interesse por alguém ou algo}
  \end{Phonetics}
\end{Entry}

\begin{Entry}{喜剧}{12,10}{⼝、⼑}
  \begin{Phonetics}{喜剧}{xi3 ju4}[][HSK 5]
    \definition[部,出]{s.}{comédia (oposto de 悲剧) | comédia; uma das principais categorias do teatro; usa o exagero para satirizar e ridicularizar o feio; fenômenos retrógrados; destaca as contradições inerentes a esses fenômenos e seu conflito com coisas saudáveis; costuma provocar risadas; o final geralmente é feliz}
  \seealsoref{悲剧}{bei1 ju4}
  \end{Phonetics}
\end{Entry}

\begin{Entry}{喜爱}{12,10}{⼝、⽖}
  \begin{Phonetics}{喜爱}{xi3 ai4}[][HSK 4]
    \definition{v.}{gostar; amar; ter afeição por; estar interessado em; ter uma queda ou sentir interesse por pessoas ou coisas}
  \end{Phonetics}
\end{Entry}

\begin{Entry}{喝}{12}{⼝}
  \begin{Phonetics}{喝}{he1}[][HSK 1]
    \definition{interj.}{Meu Deus!; Oh!; Ah!; Uau!}
    \definition{s.}{bebida; especificamente, vinho}
    \definition{v.}{beber; engolir líquidos ou alimentos líquidos | beber bebida alcoólica; referência específica ao consumo de álcool}
  \end{Phonetics}
  \begin{Phonetics}{喝}{he4}
    \definition{v.}{gritar bem alto}
  \end{Phonetics}
\end{Entry}

\begin{Entry}{喝采}{12,8}{⼝、⾤}
  \begin{Phonetics}{喝采}{he4/cai3}[][HSK 7-9]
    \definition{v.+compl.}{aclamar; aplaudir}
  \end{Phonetics}
\end{Entry}

\begin{Entry}{喝彩}{12,11}{⼝、⼺}
  \begin{Phonetics}{喝彩}{he4cai3}
    \definition{s.}{aclamar | torcer}
  \end{Phonetics}
\end{Entry}

\begin{Entry}{喝醉}{12,15}{⼝、⾣}
  \begin{Phonetics}{喝醉}{he1zui4}
    \definition{v.}{ficar bêbado}
  \end{Phonetics}
\end{Entry}

\begin{Entry}{喷}{12}{⼝}
  \begin{Phonetics}{喷}{pen1}[][HSK 5]
    \definition{v.}{jorrar; esguichar; expelir sob pressão | borrifar; espalhar; pulverizar}
  \end{Phonetics}
  \begin{Phonetics}{喷}{pen4}
    \definition{s.}{na época; tempo no mercado; época em que frutas, peixes e camarões são comercializados em grande quantidade | colheita; número de vezes que floresceu e frutificou; número de vezes que foi colhido na maturação}
  \end{Phonetics}
\end{Entry}

\begin{Entry}{喻}{12}{⼝}
  \begin{Phonetics}{喻}{yu4}
    \definition{s.}{analogia | símile | metáfora | alegoria}
    \definition{v.}{descrever algo como}
  \end{Phonetics}
\end{Entry}

\begin{Entry}{嗄}{13}{⼝}
  \begin{Phonetics}{嗄}{a2}
    \variantof{啊}
  \end{Phonetics}
  \begin{Phonetics}{嗄}{sha4}
    \definition{adj.}{rouco}
  \end{Phonetics}
\end{Entry}

\begin{Entry}{嗅}{13}{⼝}
  \begin{Phonetics}{嗅}{xiu4}
    \definition{v.}{cheirar; farejar; identificar odores pelo nariz}
  \end{Phonetics}
\end{Entry}

\begin{Entry}{嗡}{13}{⼝}
  \begin{Phonetics}{嗡}{weng1}
    \definition[出]{part.}{(onomatopéia) zumbido; zunido; zunzum; descreve o som do bater de asas de um inseto}
  \end{Phonetics}
\end{Entry}

\begin{Entry}{嗡嗡}{13,13}{⼝、⼝}
  \begin{Phonetics}{嗡嗡}{weng1weng1}
    \definition{s.}{zumbido}
    \definition{v.}{zumbir}
  \end{Phonetics}
\end{Entry}

\begin{Entry}{嘟}{13}{⼝}
  \begin{Phonetics}{嘟}{du1}
    \definition{part.}{(onomatopéia) buzina}
    \definition{v.}{fazer beicinho}
  \end{Phonetics}
\end{Entry}

\begin{Entry}{嘉}{14}{⼝}
  \begin{Phonetics}{嘉}{jia1}
    \definition*{s.}{Sobrenome Jia}
    \definition{adj.}{bom; ótimo | auspicioso | excelente}
    \definition{v.}{elogiar; recomendar}
    \definition{v.}{elogiar}
  \end{Phonetics}
\end{Entry}

\begin{Entry}{嘉年华}{14,6,6}{⼝、⼲、⼗}
  \begin{Phonetics}{嘉年华}{jia1nian2hua2}[][HSK 7-9]
    \definition{s.}{Empréstimo linguístico: carnaval}
  \end{Phonetics}
\end{Entry}

\begin{Entry}{嘉宾}{14,10}{⼝、⼧}
  \begin{Phonetics}{嘉宾}{jia1bin1}[][HSK 6]
    \definition[个,位,名,些]{s.}{convidado}
  \end{Phonetics}
\end{Entry}

\begin{Entry}{嘛}{14}{⼝}
  \begin{Phonetics}{嘛}{ma5}[][HSK 6]
    \definition{part.}{usado no final de uma declaração para expressar que é claro que é verdade que é óbvio | usado no final de uma frase imperativa para expressar expectativa ou dissuasão | usado em uma frase para indicar uma pausa e chamar a atenção da outra pessoa}
  \end{Phonetics}
\end{Entry}

\begin{Entry}{嘱}{15}{⼝}
  \begin{Phonetics}{嘱}{zhu3}
    \definition{v.}{juntar-se | implorar | incitar}
  \end{Phonetics}
\end{Entry}

\begin{Entry}{嘱托}{15,6}{⼝、⼿}
  \begin{Phonetics}{嘱托}{zhu3tuo1}
    \definition{v.}{confiar uma tarefa a alguém}
  \end{Phonetics}
\end{Entry}

\begin{Entry}{嘱咐}{15,8}{⼝、⼝}
  \begin{Phonetics}{嘱咐}{zhu3fu5}
    \definition{v.}{ordenar | dizer | exortar}
  \end{Phonetics}
\end{Entry}

\begin{Entry}{嘲}{15}{⼝}
  \begin{Phonetics}{嘲}{chao2}
    \definition{v.}{ridicularizar; zombar; fazer piada de}
  \end{Phonetics}
  \begin{Phonetics}{嘲}{zhao1}
    \definition{s.}{Onomatopéia: barulho clamoroso feito por várias pessoas falando ou cantando, ou por instrumentos musicais, ou pássaros cantando; descreve um som caótico e fragmentado}
  \end{Phonetics}
\end{Entry}

\begin{Entry}{嘲弄}{15,7}{⼝、⼶}
  \begin{Phonetics}{嘲弄}{chao2nong4}[][HSK 7-9]
    \definition{v.}{zombar; zombar de}
  \end{Phonetics}
\end{Entry}

\begin{Entry}{嘲笑}{15,10}{⼝、⽵}
  \begin{Phonetics}{嘲笑}{chao2xiao4}[][HSK 7-9]
    \definition{v.}{ridicularizar; zombar; rir de; zombar de; fazer graça de; usar palavras para zombar de alguém}
  \end{Phonetics}
\end{Entry}

\begin{Entry}{嘹}{15}{⼝}
  \begin{Phonetics}{嘹}{liao2}
    \definition{adj.}{(som) alto e claro | som claro | grito (de guindastes, etc.)}
  \end{Phonetics}
\end{Entry}

\begin{Entry}{嘹亮}{15,9}{⼝、⼇}
  \begin{Phonetics}{嘹亮}{liao2liang4}
    \definition{adj.}{ressonante; alto e claro}
  \end{Phonetics}
\end{Entry}

\begin{Entry}{嘿}{15}{⼝}
  \begin{Phonetics}{嘿}{hei1}[][HSK 7-9]
    \definition{interj.}{Ei!; indicando uma saudação ou chamar a atenção | expressando orgulho ou satisfação | expressando espanto, surpresa}
  \end{Phonetics}
  \begin{Phonetics}{嘿}{mo4}
    \definition{adj.}{quieto; silencioso; tácito}
  \end{Phonetics}
\end{Entry}

\begin{Entry}{噎}{15}{⼝}
  \begin{Phonetics}{噎}{ye1}
    \definition{v.}{engasgar | sufocar}
  \end{Phonetics}
\end{Entry}

\begin{Entry}{嘴}{16}{⼝}
  \begin{Phonetics}{嘴}{zui3}[][HSK 2]
    \definition[张]{s.}{boca; boca humana ou animal | qualquer coisa com formato ou função semelhante a uma boca | fala | comida}
  \end{Phonetics}
\end{Entry}

\begin{Entry}{嘴巴}{16,4}{⼝、⼰}
  \begin{Phonetics}{嘴巴}{zui3 ba5}[][HSK 4]
    \definition[张]{s.}{boca}
  \end{Phonetics}
\end{Entry}

\begin{Entry}{嘴巴子}{16,4,3}{⼝、⼰、⼦}
  \begin{Phonetics}{嘴巴子}{zui3ba5zi5}
    \definition{s.}{tapa | bofetada}
  \end{Phonetics}
\end{Entry}

\begin{Entry}{器}{16}{⼝}
  \begin{Phonetics}{器}{qi4}
    \definition[台]{s.}{dispositivo | ferramenta | utensílio}
  \end{Phonetics}
\end{Entry}

\begin{Entry}{器官}{16,8}{⼝、⼧}
  \begin{Phonetics}{器官}{qi4guan1}[][HSK 4]
    \definition[个,种]{s.}{órgão; aparelho; parte de um organismo que consiste em vários tipos de tecidos celulares que podem desempenhar uma função fisiológica separada}
  \end{Phonetics}
\end{Entry}

\begin{Entry}{嚣}{18}{⼝}
  \begin{Phonetics}{嚣}{xiao1}
    \definition*{s.}{Sobrenome Xiao}
    \definition{adj.}{lazer}
    \definition{v.}{clamar; fazer barulho}
  \end{Phonetics}
\end{Entry}

\begin{Entry}{嚣张}{18,7}{⼝、⼸}
  \begin{Phonetics}{嚣张}{xiao1zhang1}
    \definition{adj.}{desenfreado | arrogante | agressivo}
  \end{Phonetics}
\end{Entry}

%%%%% EOF %%%%%

