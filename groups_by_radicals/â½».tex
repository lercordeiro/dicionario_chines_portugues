%%%
%%% Radical "⽻"
%%%

\section*{Radical 124: ``⽻''}\addcontentsline{toc}{section}{Radical 124: ⽻}

\begin{Entry}{羽}{6}{⽻}[Kangxi 124]
  \begin{Phonetics}{羽}{yu3}
    \definition*{s.}{Sobrenome Yu}
    \definition{s.}{pena; pluma | asas (de pássaros ou insetos) | uma nota da antiga escala chinesa de cinco tons, correspondente a 6 na notação musical numerada}
  \end{Phonetics}
\end{Entry}

\begin{Entry}{羽毛}{6,4}{⽻、⽑}
  \begin{Phonetics}{羽毛}{yu3mao2}
    \definition{s.}{pena | plumagem | pluma}
  \end{Phonetics}
\end{Entry}

\begin{Entry}{羽毛笔}{6,4,10}{⽻、⽑、⽵}
  \begin{Phonetics}{羽毛笔}{yu3mao2bi3}
    \definition{s.}{caneta de pena}
  \end{Phonetics}
\end{Entry}

\begin{Entry}{羽毛球}{6,4,11}{⽻、⽑、⽟}
  \begin{Phonetics}{羽毛球}{yu3mao2qiu2}[][HSK 5]
    \definition[只,个]{s.}{\emph{badminton}; esporte com bola, as regras e equipamentos são bastante semelhantes ao tênis | peteca}
  \end{Phonetics}
\end{Entry}

\begin{Entry}{羽林}{6,8}{⽻、⽊}
  \begin{Phonetics}{羽林}{yu3lin2}
    \definition{s.}{escolta armada}
  \end{Phonetics}
\end{Entry}

\begin{Entry}{羽冠}{6,9}{⽻、⼍}
  \begin{Phonetics}{羽冠}{yu3guan1}
    \definition{s.}{crista emplumada (de pássaro)}
  \end{Phonetics}
\end{Entry}

\begin{Entry}{羽绒服}{6,9,8}{⽻、⽷、⽉}
  \begin{Phonetics}{羽绒服}{yu3rong2fu2}[][HSK 5]
    \definition[件,个]{s.}{jaqueta de plumas; peça de vestuário com enchimento de plumas; casaco cujo interior é preenchido com penas de pato ou ganso}
  \end{Phonetics}
\end{Entry}

\begin{Entry}{羽流}{6,10}{⽻、⽔}
  \begin{Phonetics}{羽流}{yu3liu2}
    \definition{s.}{pluma}
  \end{Phonetics}
\end{Entry}

\begin{Entry}{翅}{10}{⽻}
  \begin{Phonetics}{翅}{chi4}
    \definition[只]{s.}{asa | barbatana de tubarão | coisa parecida com uma asa}
  \end{Phonetics}
\end{Entry}

\begin{Entry}{翅膀}{10,14}{⽻、⾁}
  \begin{Phonetics}{翅膀}{chi4bang3}[][HSK 7-9]
    \definition[只,个,对]{s.}{asa; os órgãos de voo de animais como pássaros e insetos geralmente aparecem em pares | barbatana; aba; lâmina; a parte de algo que tem o formato ou age como uma asa}
  \end{Phonetics}
\end{Entry}

\begin{Entry}{翠}{14}{⽻}
  \begin{Phonetics}{翠}{cui4}
    \definition{adj.}{verde; verde esmeralda}
    \definition{s.}{martim-pescador | jadeíte; jade}
  \end{Phonetics}
\end{Entry}

\begin{Entry}{翠绿}{14,11}{⽻、⽷}
  \begin{Phonetics}{翠绿}{cui4lv4}[][HSK 7-9]
    \definition{adj.}{verde esmeralda; verde jade}
  \end{Phonetics}
\end{Entry}

\begin{Entry}{翻}{18}{⽻}
  \begin{Phonetics}{翻}{fan1}[][HSK 4]
    \definition{v.}{virar; dar a volta; inverter; mudar de posição; torcer; reverter | vasculhar; procurar; pesquisar; mover objetos para localizar algo | reverter; retrair; retirar | passar por cima; ultrapassar; cruzar | multiplicar | traduzir; decodificar | romper-se; cair; desentender-se com alguém}
  \end{Phonetics}
\end{Entry}

\begin{Entry}{翻天覆地}{18,4,18,6}{⽻、⼤、⾑、⼟}
  \begin{Phonetics}{翻天覆地}{fan1tian1-fu4di4}[][HSK 7-9]
    \definition{expr.}{virar o mundo de cabeça para baixo; uma mudança tremenda; abalar a terra; marcar época; virar o céu e a terra; sacudir o próprio chão (mundo); virar o mundo de cabeça para baixo; mudanças titânicas; ``Céu e terra virados de cabeça para baixo.''}
  \end{Phonetics}
\end{Entry}

\begin{Entry}{翻过}{18,6}{⽻、⾡}
  \begin{Phonetics}{翻过}{fan1guo4}
    \definition{v.}{virar |  transformar}
  \end{Phonetics}
\end{Entry}

\begin{Entry}{翻来覆去}{18,7,18,5}{⽻、⽊、⾑、⼛}
  \begin{Phonetics}{翻来覆去}{fan1lai2-fu4qu4}[][HSK 7-9]
    \definition{expr.}{jogar de um lado para o outro; tocar a mesma corda; repetir várias vezes; virar e se virar; dizer repetidamente; jogar inquieto de um lado para o outro; virar de um lado para o outro}
  \end{Phonetics}
\end{Entry}

\begin{Entry}{翻译}{18,7}{⽻、⾔}
  \begin{Phonetics}{翻译}{fan1yi4}[][HSK 4]
    \definition[个,位,名]{s.}{tradutor; intérprete; pessoas que fazem trabalhos de tradução}
    \definition{v.}{traduzir; interpretar; colocar o significado de palavras de um idioma em palavras de outro idioma (expressão idiomática); expressar um significado em outro idioma}
  \end{Phonetics}
\end{Entry}

\begin{Entry}{翻脸}{18,11}{⽻、⾁}
  \begin{Phonetics}{翻脸}{fan1/lian3}
    \definition{v.+compl.}{brigar com alguém | tornar-se hostil}
  \end{Phonetics}
\end{Entry}

\begin{Entry}{翻番}{18,12}{⽻、⽥}
  \begin{Phonetics}{翻番}{fan1/fan1}[][HSK 7-9]
    \definition{v.+compl.}{aumentar em um número especificado de vezes; dobrar}
  \end{Phonetics}
\end{Entry}

%%%%% EOF %%%%%

