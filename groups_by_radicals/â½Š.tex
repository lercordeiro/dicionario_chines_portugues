%%%
%%% Radical "⽊"
%%%

\section*{Radical 75: ``⽊''}\addcontentsline{toc}{section}{Radical 75: ⽊}

\begin{Entry}{木}{4}{⽊}[Kangxi 75]
  \begin{Phonetics}{木}{mu4}
    \definition{adj.}{de madeira; feito de madeira | estúpido; de raciocínio lento; atordoado; lento para reagir | simplório; chato | entorpecido; de madeira; dormência localizada ou perda de sensibilidade}
    \definition{s.}{árvore | madeira; madeiramento | caixão}
  \end{Phonetics}
\end{Entry}

\begin{Entry}{木头}{4,5}{⽊、⼤}
  \begin{Phonetics}{木头}{mu4tou5}[][HSK 3]
    \definition[根,块,堆,截]{s.}{tronco; madeira; lenha; denominação genérica para madeira e materiais de madeira}
  \end{Phonetics}
\end{Entry}

\begin{Entry}{木偶}{4,11}{⽊、⼈}
  \begin{Phonetics}{木偶}{mu4'ou3}
    \definition{s.}{fantoche, marionete}
  \end{Phonetics}
\end{Entry}

\begin{Entry}{未}{5}{⽊}
  \begin{Phonetics}{未}{wei4}
    \definition*{s.}{Sobrenome Wei}
    \definition{adv.}{Literário: não tem; não fez; (oposto a 已) | Literário: não}
    \definition{part.}{ou não; no final das perguntas, indicando dúvida}[今可以言未?===Posso falar agora?]
    \definition{s.}{wei (oitavo dos doze Ramos Terrestres)}
  \seealsoref{已}{yi3}
  \end{Phonetics}
\end{Entry}

\begin{Entry}{未必}{5,5}{⽊、⼼}
  \begin{Phonetics}{未必}{wei4bi4}[][HSK 4]
    \definition{adv.}{não tenho certeza; talvez não; não necessariamente}
  \end{Phonetics}
\end{Entry}

\begin{Entry}{未来}{5,7}{⽊、⽊}
  \begin{Phonetics}{未来}{wei4lai2}[][HSK 4]
    \definition{adj.}{próximo (refere-se ao tempo)}
    \definition[个,段,种]{s.}{futuro; o amanhã}
  \end{Phonetics}
\end{Entry}

\begin{Entry}{末}{5}{⽊}
  \begin{Phonetics}{末}{mo4}[][HSK 4]
    \definition{adj.}{último; final}
    \definition{s.}{ponta; terminal; extremidade; o final de algo | não essenciais; detalhes secundários | fim; final | pó; poeira | um papel na ópera tradicional}
  \end{Phonetics}
\end{Entry}

\begin{Entry}{本}{5}{⽊}
  \begin{Phonetics}{本}{ben3}[][HSK 1,6]
    \definition*{s.}{Sobrenome Ben}
    \definition{adj.}{original; inerente | principal; central}
    \definition{adv.}{originalmente}
    \definition{clas.}{para livros, dicionários, periódicos, arquivos, etc. | para vídeos de uma determinada duração | para peças de teatro, ópera}
    \definition{prep.}{de acordo com; em consonância com; em conformidade com; equivalentes a 依照 e 按照}
    \definition{pron.}{nativo; próprio; refere-se ao próprio interlocutor ou ao grupo, instituição, empresa, local, etc. ao qual o interlocutor pertence | isto; atual; presente}
    \definition[个]{s.}{caule ou raiz de plantas | base; origem; fundamento; fundação;  alicerce | capital; capital social | livro; caderno; livreto | edição; versão | cópia; roteiro; manuscrito | memorial do trono; na era feudal, referia-se a um documento oficial}
    \definition{v.}{seguir; basear-se em; estar de acordo com}
  \seealsoref{按照}{an4zhao4}
  \seealsoref{依照}{yi1 zhao4}
  \end{Phonetics}
\end{Entry}

\begin{Entry}{本人}{5,2}{⽊、⼈}
  \begin{Phonetics}{本人}{ben3ren2}[][HSK 5]
    \definition{pron.}{eu (mim, mim mesmo); o orador refere-se a si mesmo | a si mesmo; em pessoa; refere-se à própria pessoa ou à pessoa mencionada anteriormente}
  \end{Phonetics}
\end{Entry}

\begin{Entry}{本土}{5,3}{⽊、⼟}
  \begin{Phonetics}{本土}{ben3 tu3}[][HSK 6]
    \definition{s.}{território metropolitano; pátria-mãe; refere-se ao território do país | país (ou terra) natal de alguém; nativo; cidade natal; local original de crescimento}
  \end{Phonetics}
\end{Entry}

\begin{Entry}{本子}{5,3}{⽊、⼦}
  \begin{Phonetics}{本子}{ben3 zi5}[][HSK 1]
    \definition[个,本]{s.}{livro; caderno | edição | impressão | licença; certificado de competência emitido por uma instituição especializada, obtido após aprovação no exame | \emph{script}; roteiro}
  \end{Phonetics}
\end{Entry}

\begin{Entry}{本分}{5,4}{⽊、⼑}
  \begin{Phonetics}{本分}{ben3fen4}[][HSK 7-9]
    \definition{adj.}{honesto; decente}
    \definition{s.}{o trabalho de alguém; o dever de alguém; as próprias responsabilidades e obrigações | estar satisfeito com sua posição e ambiente atuais}
  \end{Phonetics}
\end{Entry}

\begin{Entry}{本地}{5,6}{⽊、⼟}
  \begin{Phonetics}{本地}{ben3 di4}[][HSK 6]
    \definition{s.}{local; nativo; localidade; a área onde as pessoas e as coisas estão localizadas; uma área específica referida em uma narrativa}
  \end{Phonetics}
\end{Entry}

\begin{Entry}{本色}{5,6}{⽊、⾊}
  \begin{Phonetics}{本色}{ben3se4}[][HSK 7-9]
    \definition{s.}{(algo, geralmente tecidos não tingidos) cor natural; a cor original de algo (geralmente se refere a tecidos não tingidos)}
  \end{Phonetics}
\end{Entry}

\begin{Entry}{本来}{5,7}{⽊、⽊}
  \begin{Phonetics}{本来}{ben3lai2}[][HSK 3]
    \definition{adj.}{original}
    \definition{adv.}{anteriormente; originalmente; indica que antes disso | claro; em primeiro lugar; como deveria ser; indica que algo é natural ou óbvio}
  \end{Phonetics}
\end{Entry}

\begin{Entry}{本身}{5,7}{⽊、⾝}
  \begin{Phonetics}{本身}{ben3shen1}[][HSK 6]
    \definition{pron.}{próprio; em si mesmo; refere-se à pessoa, unidade ou coisa em si}
  \end{Phonetics}
\end{Entry}

\begin{Entry}{本事}{5,8}{⽊、⼅}
  \begin{Phonetics}{本事}{ben3shi4}
    \definition{s.}{habilidade; aptidão; capacidade; competência; refere-se às habilidades, capacidades ou talentos que uma pessoa possui em determinada área | habilidade; aptidão; capacidade; competência; a capacidade e os meios necessários para atingir um determinado objetivo ou concluir uma determinada tarefa | status; poder; posição; autoridade; refere-se à identidade, posição ou poder de uma pessoa.}
  \end{Phonetics}
  \begin{Phonetics}{本事}{ben3shi5}[][HSK 3]
    \definition{s.}{habilidade; capacidade; talento; aptidão}
  \end{Phonetics}
\end{Entry}

\begin{Entry}{本性}{5,8}{⽊、⼼}
  \begin{Phonetics}{本性}{ben3xing4}[][HSK 7-9]
    \definition{s.}{qualidade inerente | instintos naturais | natureza}
  \end{Phonetics}
\end{Entry}

\begin{Entry}{本质}{5,8}{⽊、⾙}
  \begin{Phonetics}{本质}{ben3zhi4}[][HSK 6]
    \definition{s.}{essência; natureza; caráter inato; qualidade intrínseca; refere-se aos atributos fundamentais inerentes às próprias coisas, que desempenham um papel decisivo na natureza, condição e desenvolvimento das coisas (distinguido de 现象)}
  \seealsoref{现象}{xian4xiang4}
  \end{Phonetics}
\end{Entry}

\begin{Entry}{本金}{5,8}{⽊、⾦}
  \begin{Phonetics}{本金}{ben3 jin1}
    \definition{s.}{capital; capital para a operação do comércio e da indústria; capital para a operação de negócios | valor principal; dinheiro retirado ao depositar ou tomar emprestado (diferente de 利息)}
  \seealsoref{利息}{li4xi1}
  \end{Phonetics}
\end{Entry}

\begin{Entry}{本科}{5,9}{⽊、⽲}
  \begin{Phonetics}{本科}{ben3ke1}[][HSK 4]
    \definition{s.}{graduação; bacharelado; o curso básico de uma universidade ou faculdade}
  \end{Phonetics}
\end{Entry}

\begin{Entry}{本能}{5,10}{⽊、⾁}
  \begin{Phonetics}{本能}{ben3neng2}[][HSK 7-9]
    \definition{adv.}{instintivamente; pela luz da natureza; inconscientemente, subconscientemente}
    \definition{s.}{instinto; as habilidades que humanos e animais desenvolvem durante o processo de evolução são fixadas pela hereditariedade e não precisam ser ensinadas}
  \end{Phonetics}
\end{Entry}

\begin{Entry}{本钱}{5,10}{⽊、⾦}
  \begin{Phonetics}{本钱}{ben3qian2}[][HSK 7-9]
    \definition{s.}{capital; dinheiro usado para gerar lucros, juros ou para participar de atividades como jogos de azar | habilidades, qualificações ou condições que podem ser usadas para fazer algo; metaforicamente falando, qualificações, habilidades e outras condições em que se pode confiar}
  \end{Phonetics}
\end{Entry}

\begin{Entry}{本着}{5,11}{⽊、⽬}
  \begin{Phonetics}{本着}{ben3zhe5}[][HSK 7-9]
    \definition{prep.}{com base em; de acordo com; em conformidade com; à luz de}
  \end{Phonetics}
\end{Entry}

\begin{Entry}{本领}{5,11}{⽊、⾴}
  \begin{Phonetics}{本领}{ben3 ling3}[][HSK 3]
    \definition[项,个,种]{s.}{habilidade; capacidade; faculdade; poder; destreza; talento}
  \end{Phonetics}
\end{Entry}

\begin{Entry}{本期}{5,12}{⽊、⽉}
  \begin{Phonetics}{本期}{ben3 qi1}[][HSK 6]
    \definition{adv.}{o período atual | este prazo (geralmente em finanças)}
  \end{Phonetics}
\end{Entry}

\begin{Entry}{本意}{5,13}{⽊、⼼}
  \begin{Phonetics}{本意}{ben3yi4}[][HSK 7-9]
    \definition{s.}{ideia original; intenção real (original)}
  \end{Phonetics}
\end{Entry}

\begin{Entry}{术}{5}{⽊}
  \begin{Phonetics}{术}{shu4}
    \definition*{s.}{Sobrenome Shu}
    \definition{s.}{arte; habilidade; técnica; tecnologia; acadêmico | método; tática; estratégia}
  \end{Phonetics}
  \begin{Phonetics}{术}{zhu2}
    \definition{s.}{vários gêneros de flores da família Asteraceae (margaridas e crisântemos)}
  \end{Phonetics}
\end{Entry}

\begin{Entry}{术科}{5,9}{⽊、⽲}
  \begin{Phonetics}{术科}{shu4ke1}
    \definition{s.}{cursos técnicos oferecidos em treinamento militar ou físico (oposto a 学科)}
  \seealsoref{学科}{xue2 ke1}
  \end{Phonetics}
\end{Entry}

\begin{Entry}{朵}{6}{⽊}
  \begin{Phonetics}{朵}{duo3}[][HSK 5]
    \definition*{s.}{Sobrenome Duo}
    \definition{clas.}{usado para flores, nuvens ou coisas que se assemelham a flores e nuvens}
  \end{Phonetics}
\end{Entry}

\begin{Entry}{机}{6}{⽊}
  \begin{Phonetics}{机}{ji1}
    \definition*{s.}{Sobrenome Ji}
    \definition{adj.}{flexível; perspicaz; destreza; agilidade}
    \definition[台]{s.}{máquina; motor | avião; aeronave; aeroplano; refere-se especificamente a aeronaves | ponto crucial; os fatores-chave para a ocorrência e mudança das coisas | chance; ocasião; oportunidade; um momento crítico ou oportuno para o desenvolvimento e mudança das coisas | organismo; funções vitais dos organismos | besta; mecanismo de disparo de flechas de madeira em uma besta antiga | assuntos importantes; assuntos extremamente importantes e confidenciais | ideia; intenção}
  \end{Phonetics}
\end{Entry}

\begin{Entry}{机甲}{6,5}{⽊、⽥}
  \begin{Phonetics}{机甲}{ji1jia3}
    \definition{s.}{\emph{mecha} (robôs operados pelo homem em mangá japonês)}
  \end{Phonetics}
\end{Entry}

\begin{Entry}{机会}{6,6}{⽊、⼈}
  \begin{Phonetics}{机会}{ji1hui4}[][HSK 2]
    \definition[个,次,种,些]{s.}{chance; oportunidade; momento favorável raro}
  \end{Phonetics}
\end{Entry}

\begin{Entry}{机关}{6,6}{⽊、⼋}
  \begin{Phonetics}{机关}{ji1 guan1}[][HSK 6]
    \definition{adj.}{operado por máquina | controlado mecanicamente}
    \definition[个]{s.}{engrenagem; mecanismo; Antigo: refere-se a certos dispositivos controlados mecanicamente; também se refere às peças de frenagem de dispositivos mecânicos | escritório; órgão; corpo; instituição | esquema; maquinação; estratagema; um plano cuidadoso e inteligente}
  \end{Phonetics}
\end{Entry}

\begin{Entry}{机动}{6,6}{⽊、⼒}
  \begin{Phonetics}{机动}{ji1dong4}[][HSK 7-9]
    \definition{adj.}{motorizado; movido a energia | flexível; manobrável; conveniente; móvel | em reserva; para uso emergencial}
  \end{Phonetics}
\end{Entry}

\begin{Entry}{机动车}{6,6,4}{⽊、⼒、⾞}
  \begin{Phonetics}{机动车}{ji1 dong4 che1}[][HSK 6]
    \definition{s.}{veículo motorizado (oposto a 人力车) | veículo automotor; automóvel de passageiros: veículo comercial concebido e tecnicamente adequado para o transporte de passageiros e respetiva bagagem, incluindo o banco do condutor}
  \seealsoref{人力车}{ren2 li4 che1}
  \end{Phonetics}
\end{Entry}

\begin{Entry}{机场}{6,6}{⽊、⼟}
  \begin{Phonetics}{机场}{ji1chang3}[][HSK 1]
    \definition[个,家,处,座]{s.}{aeródromo; campo de aviação; aeroporto; campo de voo}
  \end{Phonetics}
\end{Entry}

\begin{Entry}{机灵}{6,7}{⽊、⽕}
  \begin{Phonetics}{机灵}{ji1ling5}[][HSK 7-9]
    \definition{adj.}{inteligente; esperto; astuto; espirituoso}
  \end{Phonetics}
\end{Entry}

\begin{Entry}{机制}{6,8}{⽊、⼑}
  \begin{Phonetics}{机制}{ji1 zhi4}[][HSK 5]
    \definition{s.}{mecanismo; processado por máquina; feito por máquina}
  \end{Phonetics}
\end{Entry}

\begin{Entry}{机构}{6,8}{⽊、⽊}
  \begin{Phonetics}{机构}{ji1gou4}[][HSK 4]
    \definition[所]{s.}{órgão; organização; instituição; instalações; aparelhamento; configuração | mecanismo; funcionamento interno de uma máquina ou unidade | estrutura interna de uma organização}
  \end{Phonetics}
\end{Entry}

\begin{Entry}{机舱}{6,10}{⽊、⾈}
  \begin{Phonetics}{机舱}{ji1cang1}[][HSK 7-9]
    \definition{s.}{sala de máquinas (de um navio) | compartimento de passageiros (de uma aeronave); cabine | espaço de máquinas; cabine de aeronave}
  \end{Phonetics}
\end{Entry}

\begin{Entry}{机密}{6,11}{⽊、⼧}
  \begin{Phonetics}{机密}{ji1mi4}[][HSK 7-9]
    \definition{adj.}{secreto; classificado; privado; confidencial}
    \definition{s.}{segredo; assuntos confidenciais}
  \end{Phonetics}
\end{Entry}

\begin{Entry}{机械}{6,11}{⽊、⽊}
  \begin{Phonetics}{机械}{ji1xie4}[][HSK 6]
    \definition{adj.}{rígido; mecânico; inflexível; uma metáfora para uma abordagem rígida e imutável}
    \definition[台,部,个]{s.}{máquina; maquinário; mecanismo; vários dispositivos compostos por princípios mecânicos}
  \end{Phonetics}
\end{Entry}

\begin{Entry}{机票}{6,11}{⽊、⽰}
  \begin{Phonetics}{机票}{ji1 piao4}[][HSK 1]
    \definition[张]{s.}{passagem aérea; passagem de avião}
  \seealsoref{飞机票}{fei1ji1 piao4}
  \end{Phonetics}
\end{Entry}

\begin{Entry}{机智}{6,12}{⽊、⽇}
  \begin{Phonetics}{机智}{ji1zhi4}[][HSK 7-9]
    \definition{adj.}{engenhoso; perspicaz; inteligente e adaptável}
  \end{Phonetics}
\end{Entry}

\begin{Entry}{机遇}{6,12}{⽊、⾡}
  \begin{Phonetics}{机遇}{ji1yu4}[][HSK 4]
    \definition[个]{s.}{chance; oportunidade; circunstâncias favoráveis}
  \end{Phonetics}
\end{Entry}

\begin{Entry}{机器}{6,16}{⽊、⼝}
  \begin{Phonetics}{机器}{ji1qi4}[][HSK 3]
    \definition[台,部,个]{s.}{máquina; maquinário; motor; dispositivos e máquinas que são montados a partir de peças, podem funcionar, transformar energia ou produzir trabalho útil podem ser usados como ferramentas de produção, reduzindo a intensidade do trabalho humano e aumentando a produtividade | aparato; sistema político e econômico}
  \end{Phonetics}
\end{Entry}

\begin{Entry}{机器人}{6,16,2}{⽊、⼝、⼈}
  \begin{Phonetics}{机器人}{ji1 qi4 ren2}[][HSK 5]
    \definition[个,些]{s.}{androide; golem | pessoa mecânica | robô}
  \end{Phonetics}
\end{Entry}

\begin{Entry}{杀}{6}{⽊}
  \begin{Phonetics}{杀}{sha1}[][HSK 5]
    \definition{adv.}{em extremo; excessivamente; usado após um verbo, indica grau intenso}
    \definition{v.}{matar; abater; esquartejar | lutar; entrar em batalha | enfraquecer; reduzir; diminuir | decolar; neutralizar}
  \end{Phonetics}
\end{Entry}

\begin{Entry}{杀气}{6,4}{⽊、⽓}
  \begin{Phonetics}{杀气}{sha1qi4}
    \definition{s.}{espírito assassino | aura de morte}
    \definition{v.}{desabafar a raiva de alguém}
  \end{Phonetics}
\end{Entry}

\begin{Entry}{杀毒}{6,9}{⽊、⽏}
  \begin{Phonetics}{杀毒}{sha1 du2}[][HSK 5]
    \definition{s.}{Computação: antivírus}
    \definition{v.}{esterilizar; desinfetar | Computação: eliminar um vírus}
  \end{Phonetics}
\end{Entry}

\begin{Entry}{杂}{6}{⽊}
  \begin{Phonetics}{杂}{za2}[][HSK 6]
    \definition{adj.}{diversos; misto; misturados | extra; irregular | variado}
    \definition{v.}{misturar}
  \end{Phonetics}
\end{Entry}

\begin{Entry}{杂志}{6,7}{⽊、⼼}
  \begin{Phonetics}{杂志}{za2zhi4}[][HSK 3]
    \definition[本,期,份,种]{s.}{jornal; revista; publicação}
  \end{Phonetics}
\end{Entry}

\begin{Entry}{杂志社}{6,7,7}{⽊、⼼、⽰}
  \begin{Phonetics}{杂志社}{za2zhi4 she4}
    \definition{s.}{editora de revista; a organização responsável pela edição, publicação e distribuição de revistas}
  \end{Phonetics}
\end{Entry}

\begin{Entry}{杂技}{6,7}{⽊、⼿}
  \begin{Phonetics}{杂技}{za2ji4}
    \definition[场,个]{s.}{acrobacia; um termo geral para várias performances (como habilidades com carros, ventriloquia, equilíbrio de tigelas, andar na corda bamba, dança do leão, mágica, etc.)}
  \end{Phonetics}
\end{Entry}

\begin{Entry}{杂剧}{6,10}{⽊、⼑}
  \begin{Phonetics}{杂剧}{za2ju4}
    \definition{s.}{(na Dinastia Song) peça de variedades que consiste em um prelúdio, a peça principal em uma ou duas cenas e um epílogo musical (nenhuma das peças de variedades Song existe até hoje) |  (na Dinastia Yuan) drama poético que consiste em quatro atos ou sequências de canções (折), ocasionalmente incluindo uma ``cunha'' (楔子) na forma de um prólogo (colocado antes do primeiro ato) ou um interlúdio (colocado entre os atos), todas as partes cantadas nos quatro atos atribuídas ao protagonista, seja homem ou mulher | uma forma de comédia musical da dinastia Yuan; uma forma de performance caracterizada pelo humor e pela brincadeira na Dinastia Song, desenvolveu-se como uma forma de ópera na Dinastia Yuan, cada obra consiste em quatro atos, às vezes com um prólogo no início ou entre os atos, cada ato é composto por um conjunto de melodias nórdicas na mesma melodia e rima palaciana, além de versos de convidados}
  \seealsoref{楔子}{xie1zi5}
  \seealsoref{折}{zhe2}
  \end{Phonetics}
\end{Entry}

\begin{Entry}{权}{6}{⽊}
  \begin{Phonetics}{权}{quan2}[][HSK 6]
    \definition*{s.}{Sobrenome Quan}
    \definition{adv.}{provisoriamente; por enquanto}
    \definition{s.}{Lliterário: contrapeso; peso deslizante de uma balança romana | poder; autoridade | direito | posição vantajosa | conveniência}
    \definition{v.}{pesar; medir o peso}
  \end{Phonetics}
\end{Entry}

\begin{Entry}{权力}{6,2}{⽊、⼒}
  \begin{Phonetics}{权力}{quan2li4}[][HSK 6]
    \definition[种]{s.}{poder; autoridade; o poder de liderança no âmbito da responsabilidade | poder; coerção política; o poder coercitivo do status social e político}
  \end{Phonetics}
\end{Entry}

\begin{Entry}{权利}{6,7}{⽊、⼑}
  \begin{Phonetics}{权利}{quan2li4}[][HSK 4]
    \definition[项,种,个,条,份]{s.}{direito; interesse; os poderes e benefícios (em oposição a 义务) exercidos por um cidadão ou pessoa jurídica de acordo com a lei}
  \seealsoref{义务}{yi4wu4}
  \end{Phonetics}
\end{Entry}

\begin{Entry}{杆}{7}{⽊}
  \begin{Phonetics}{杆}{gan1}[][HSK 6]
    \definition{s.}{poste; pólo; mastro}
  \end{Phonetics}
  \begin{Phonetics}{杆}{gan3}
    \definition{clas.}{usado para objetos semelhantes a hastes}
    \definition{s.}{eixo; braço | haste; barra; poste; a parte longa e fina de um objeto, semelhante a um bastão}
  \end{Phonetics}
\end{Entry}

\begin{Entry}{李}{7}{⽊}
  \begin{Phonetics}{李}{li3}
    \definition*{s.}{Sobrenome Li}
    \definition[棵]{s.}{ameixa | ameixeira}
  \end{Phonetics}
\end{Entry}

\begin{Entry}{李子}{7,3}{⽊、⼦}
  \begin{Phonetics}{李子}{li3zi5}
    \definition[个]{s.}{ameixa}
  \end{Phonetics}
\end{Entry}

\begin{Entry}{李四}{7,5}{⽊、⼞}
  \begin{Phonetics}{李四}{li3si4}
    \definition*{s.}{Li Si | Zé Ninguém | Nome para uma pessoa não especificada, 2 de 3}
  \seealsoref{王五}{wang2wu3}
  \seealsoref{张三}{zhang1san1}
  \end{Phonetics}
\end{Entry}

\begin{Entry}{材}{7}{⽊}
  \begin{Phonetics}{材}{cai2}
    \definition[份]{s.}{madeira | material; geralmente se refere a coisas que podem ser transformadas diretamente em produtos acabados | material; materiais para escrita ou referência | pessoa capaz; pessoas talentosas | habilidade; talento; aptidão | caixão}
  \end{Phonetics}
\end{Entry}

\begin{Entry}{材料}{7,10}{⽊、⽃}
  \begin{Phonetics}{材料}{cai2liao4}[][HSK 4]
    \definition[份,个,种]{s.}{material; algo para fazer um produto acabado | material (figura de linguagem) | dados; material para estudo, pesquisa, etc.; conteúdo de uma obra}
  \end{Phonetics}
\end{Entry}

\begin{Entry}{村}{7}{⽊}
  \begin{Phonetics}{村}{cun1}[][HSK 3]
    \definition{adj.}{rústico; grosseiro}
    \definition[个,座]{s.}{aldeia; vila | área povoada de certo tipo}
  \end{Phonetics}
\end{Entry}

\begin{Entry}{村儿}{7,2}{⽊、⼉}
  \begin{Phonetics}{村儿}{cun1r5}
    \definition{s.}{vila; aldeia}
  \end{Phonetics}
\end{Entry}

\begin{Entry}{村庄}{7,6}{⽊、⼴}
  \begin{Phonetics}{村庄}{cun1 zhuang1}[][HSK 6]
    \definition[个,座,片]{s.}{aldeia; vila; onde vivem os agricultores}
  \end{Phonetics}
\end{Entry}

\begin{Entry}{杜}{7}{⽊}
  \begin{Phonetics}{杜}{du4}
    \definition*{s.}{Sobrenome Du}
    \definition{s.}{pêra de folha de bétula}
    \definition{v.}{excluir; parar; impedir; bloquear}
  \end{Phonetics}
\end{Entry}

\begin{Entry}{杜宇}{7,6}{⽊、⼧}
  \begin{Phonetics}{杜宇}{du4yu3}
    \definition{s.}{cuco (pássaro)}
  \seealsoref{布谷鸟}{bu4gu3niao3}
  \seealsoref{杜鹃}{du4juan1}
  \seealsoref{杜鹃鸟}{du4juan1niao3}
  \end{Phonetics}
\end{Entry}

\begin{Entry}{杜绝}{7,9}{⽊、⽷}
  \begin{Phonetics}{杜绝}{du4jue2}[][HSK 7-9]
    \definition{v.}{parar; pôr fim a; bloquear a fonte e parar (coisas ruins)}
  \end{Phonetics}
\end{Entry}

\begin{Entry}{杜鹃}{7,12}{⽊、⿃}
  \begin{Phonetics}{杜鹃}{du4juan1}
    \definition{s.}{cuco (pássaro)}
  \seealsoref{布谷鸟}{bu4gu3niao3}
  \seealsoref{杜鹃鸟}{du4juan1niao3}
  \seealsoref{杜宇}{du4yu3}
  \end{Phonetics}
\end{Entry}

\begin{Entry}{杜鹃鸟}{7,12,5}{⽊、⿃、⿃}
  \begin{Phonetics}{杜鹃鸟}{du4juan1niao3}
    \definition{s.}{cuco (pássaro)}
  \seealsoref{布谷鸟}{bu4gu3niao3}
  \seealsoref{杜鹃}{du4juan1}
  \seealsoref{杜宇}{du4yu3}
  \end{Phonetics}
\end{Entry}

\begin{Entry}{束}{7}{⽊}
  \begin{Phonetics}{束}{shu4}[][HSK 3]
    \definition*{s.}{Sobrenome Shu}
    \definition{clas.}{usado para cachos, molhos, feixes, feixes de luz, etc.}
    \definition{s.}{monte; pacote; maço; feixe; cacho; coisas agrupadas ou reunidas em tiras}
    \definition{v.}{atar; amarrar; vincular | controlar; restringir}
  \end{Phonetics}
\end{Entry}

\begin{Entry}{束腰}{7,13}{⽊、⾁}
  \begin{Phonetics}{束腰}{shu4yao1}
    \definition{s.}{cinto | cinta | cinturão}
  \end{Phonetics}
\end{Entry}

\begin{Entry}{杠}{7}{⽊}
  \begin{Phonetics}{杠}{gang1}
    \definition{s.}{pequena ponte | mastro de bandeira}
  \end{Phonetics}
  \begin{Phonetics}{杠}{gang4}
    \definition{s.}{vara grossa | (esportes) barra | peça sobressalente em forma de haste; peça sobressalente em forma de haste usada para máquinas-ferramentas | varas robustas usadas para carregar um caixão | (em um texto) linha grossa desenhada ao lado ou abaixo das palavras como uma marca | (coloquial) padrão; critério}
    \definition{v.}{marcar com uma linha grossa | afiar (faca, navalha, etc.)}
  \end{Phonetics}
\end{Entry}

\begin{Entry}{杠铃}{7,10}{⽊、⾦}
  \begin{Phonetics}{杠铃}{gang4ling2}[][HSK 7-9]
    \definition{s.}{barra; levantamento de peso; equipamento de levantamento de peso, com placas metálicas em forma de disco instaladas em ambas as extremidades da barra horizontal}
  \end{Phonetics}
\end{Entry}

\begin{Entry}{条}{7}{⽊}
  \begin{Phonetics}{条}{tiao2}[][HSK 2]
    \definition*{s.}{Sobrenome Tiao}
    \definition{clas.}{usado para objetos longos e finos; usado para sintetizar certas coisas longas e retangulares em quantidades fixas | usado para itemização | aplicado ao corpo humano}
    \definition{s.}{galho; galhos finos e longos | tira; faixa | item; artigo | ordem; método | nota; anotação em papel}
  \end{Phonetics}
\end{Entry}

\begin{Entry}{条目}{7,5}{⽊、⽬}
  \begin{Phonetics}{条目}{tiao2mu4}
    \definition{s.}{cláusulas e subcláusulas (em documento formal) | verbete (em um dicionário, enciclopédia, etc.)}
  \end{Phonetics}
\end{Entry}

\begin{Entry}{条件}{7,6}{⽊、⼈}
  \begin{Phonetics}{条件}{tiao2jian4}[][HSK 2]
    \definition[个,项,些]{s.}{condição; termo; fator; fatores que restringem a ocorrência, existência ou desenvolvimento das coisas | requisito; pré-requisito; qualificação; requisitos ou padrões estabelecidos para determinadas coisas | situação; estado; condição}
  \end{Phonetics}
\end{Entry}

\begin{Entry}{条例}{7,8}{⽊、⼈}
  \begin{Phonetics}{条例}{tiao2li4}
    \definition{s.}{código de conduta | ordenanças | regulamentos | regras | estatutos}
  \end{Phonetics}
\end{Entry}

\begin{Entry}{条贯}{7,8}{⽊、⾙}
  \begin{Phonetics}{条贯}{tiao2guan4}
    \definition{s.}{ordem | procedimentos | sequência | sistema}
  \end{Phonetics}
\end{Entry}

\begin{Entry}{条幅}{7,12}{⽊、⼱}
  \begin{Phonetics}{条幅}{tiao2fu2}
    \definition{s.}{faixa | banner | pergaminho de parede (para pintura ou caligrafia)}
  \end{Phonetics}
\end{Entry}

\begin{Entry}{来}{7}{⽊}
  \begin{Phonetics}{来}{lai2}[][HSK 1]
    \definition*{s.}{Sobrenome Lai}
    \definition{part.}{usado após uma palavra numérica ou de quantidade; indica uma quantidade aproximada | usado depois de numerais como 一, 二, 三; para listar razões ou fatos, etc.}
    \definition{s.}{usado após uma expressão de tempo para indicar uma duração que vai do passado ao presente}
    \definition{v.}{vir; chegar; de outro lugar para o lugar onde o interlocutor se encontra | aparecer; acontecer; vir; (problemas, coisas, etc.) ocorrerem; surgirem | substitui um verbo com significado específico, indicando a realização de uma ação específica | estar indo para; usado antes de outro verbo, indica que algo será feito | vir para fazer algo; usado após outro verbo, indica que se vai fazer algo | usado para indicar um propósito; expressar o objetivo, fazer algo usando o método, a atitude ou a direção anteriores | usado com 得 ou 不 para indicar possibilidade, capacidade ou hábito}
  \seealsoref{不}{bu4}
  \seealsoref{得}{de5}
  \end{Phonetics}
\end{Entry}

\begin{Entry}{来不及}{7,4,3}{⽊、⼀、⼃}
  \begin{Phonetics}{来不及}{lai2bu5ji2}[][HSK 4]
    \definition{v.}{ser tarde demais; não ter tempo; não ter tempo suficiente (para fazer algo); não ser possível participar ou se atualizar devido a restrições de tempo}
  \end{Phonetics}
\end{Entry}

\begin{Entry}{来自}{7,6}{⽊、⾃}
  \begin{Phonetics}{来自}{lai2zi4}[][HSK 2]
    \definition{v.}{vir de (um local) | \emph{From:} (cabeçalho de \emph{e -mail})}
  \end{Phonetics}
\end{Entry}

\begin{Entry}{来到}{7,8}{⽊、⼑}
  \begin{Phonetics}{来到}{lai2 dao4}[][HSK 1]
    \definition{v.}{chegar; vir}
  \end{Phonetics}
\end{Entry}

\begin{Entry}{来往}{7,8}{⽊、⼻}
  \begin{Phonetics}{来往}{lai2 wang3}[][HSK 6]
    \definition{s.}{negociação; contato com alguém; interações sociais}
    \definition{v.}{ir e vir | ter negócios com alguém}
  \end{Phonetics}
\end{Entry}

\begin{Entry}{来信}{7,9}{⽊、⼈}
  \begin{Phonetics}{来信}{lai2 xin4}[][HSK 5]
    \definition[封]{s.}{sua carta; carta recebida; carta ao interlocutor}
    \definition{v.}{enviar uma carta para aqui; enviar uma carta para o remetente}
  \end{Phonetics}
\end{Entry}

\begin{Entry}{来得及}{7,11,3}{⽊、⼻、⼃}
  \begin{Phonetics}{来得及}{lai2de5ji2}[][HSK 4]
    \definition{v.}{ainda ter tempo; ser capaz de fazer isso; ser capaz de fazer algo a tempo; ainda ter tempo de cuidar ou de colocar em dia}
  \end{Phonetics}
\end{Entry}

\begin{Entry}{来源}{7,13}{⽊、⽔}
  \begin{Phonetics}{来源}{lai2yuan2}[][HSK 4]
    \definition{s.}{origem; causa; fonte; tabula rasa (ou seja, o lugar de onde as coisas vêm)}
    \definition{v.}{originar-se; surgir; ter origem; (algo) originar (seguido de 于)}
  \seealsoref{于}{yu2}
  \end{Phonetics}
\end{Entry}

\begin{Entry}{极}{7}{⽊}
  \begin{Phonetics}{极}{ji2}[][HSK 4]
    \definition*{s.}{Sobrenome Ji}
    \definition{adj.}{máximo; extremo; final; supremo}
    \definition{adv.}{extremamente; excessivamente}
    \definition{s.}{o ponto máximo, mais alto; extremo; ápice; ponto culminante | pólo; as extremidades norte e sul da Terra; as extremidades de um ímã; a extremidade de uma fonte de alimentação ou de um aparelho elétrico onde a corrente entra ou sai do aparelho}
    \definition{v.}{chegar ao fim de; levar a extremos | Literário: fazer o máximo possível}
  \end{Phonetics}
\end{Entry}

\begin{Entry}{……极了}{7,2}{⽊、⼅}
  \begin{Phonetics}{……极了}{ji2le5}[][HSK 3]
    \definition{expr.}{extremamente; alto grau de expressão}
  \end{Phonetics}
\end{Entry}

\begin{Entry}{极力}{7,2}{⽊、⼒}
  \begin{Phonetics}{极力}{ji2li4}[][HSK 7-9]
    \definition{v.}{fazer o máximo; não poupar esforços; tentar todos os meios possíveis}
  \end{Phonetics}
\end{Entry}

\begin{Entry}{极为}{7,4}{⽊、⼂}
  \begin{Phonetics}{极为}{ji2wei2}[][HSK 7-9]
    \definition{adv.}{extremamente; excessivamente}
  \end{Phonetics}
\end{Entry}

\begin{Entry}{极少数}{7,4,13}{⽊、⼩、⽁}
  \begin{Phonetics}{极少数}{ji2 shao3shu4}[][HSK 7-9]
    \definition{num.}{pequena minoria; apenas alguns; um punhado; extremamente poucos}
  \end{Phonetics}
\end{Entry}

\begin{Entry}{极其}{7,8}{⽊、⼋}
  \begin{Phonetics}{极其}{ji2qi2}[][HSK 4]
    \definition{adv.}{mais; extremamente; excessivamente}
  \end{Phonetics}
\end{Entry}

\begin{Entry}{极限}{7,8}{⽊、⾩}
  \begin{Phonetics}{极限}{ji2xian4}[][HSK 7-9]
    \definition[个]{s.}{limite; extremidade; limite máximo}
  \end{Phonetics}
\end{Entry}

\begin{Entry}{极度}{7,9}{⽊、⼴}
  \begin{Phonetics}{极度}{ji2du4}[][HSK 7-9]
    \definition{adv.}{extremamente; profundamente}
    \definition{s.}{extremo; excedente; o máximo; pólo}
  \end{Phonetics}
\end{Entry}

\begin{Entry}{极端}{7,14}{⽊、⽴}
  \begin{Phonetics}{极端}{ji2duan1}[][HSK 6]
    \definition{adj.}{extremo; absoluto; sem quaisquer restrições}
    \definition{adv.}{excessivamente; extremamente; alto grau de expressão}
    \definition{s.}{extremo; extremidade; o auge do desenvolvimento}
  \end{Phonetics}
\end{Entry}

\begin{Entry}{杯}{8}{⽊}
  \begin{Phonetics}{杯}{bei1}[][HSK 1]
    \definition{clas.}{para certos recipientes de líquidos: copo, xícara, etc.}
    \definition[只,个]{s.}{copo; caneca; xícara | taça; troféu; prêmio em forma de taça}
  \end{Phonetics}
\end{Entry}

\begin{Entry}{杯子}{8,3}{⽊、⼦}
  \begin{Phonetics}{杯子}{bei1 zi5}[][HSK 1]
    \definition[个,只,种]{s.}{xícara; copo; recipiente para bebidas ou outros líquidos, geralmente cilíndrico ou com a parte inferior ligeiramente mais estreita, com capacidade geralmente pequena}
  \end{Phonetics}
\end{Entry}

\begin{Entry}{杯具}{8,8}{⽊、⼋}
  \begin{Phonetics}{杯具}{bei1ju4}
    \definition{s.}{parachoque | fiasco | (gíria) tragédia}
  \end{Phonetics}
\end{Entry}

\begin{Entry}{杰}{8}{⽊}
  \begin{Phonetics}{杰}{jie2}
    \definition{adj.}{notável; proeminente; fora do comum}
    \definition[位,名,个,些]{s.}{pessoa excepcional; herói; uma pessoa com talentos excepcionais}
  \end{Phonetics}
\end{Entry}

\begin{Entry}{杰出}{8,5}{⽊、⼐}
  \begin{Phonetics}{杰出}{jie2chu1}[][HSK 6]
    \definition{adj.}{notável; proeminente; (talento, realização) excepcional}
  \end{Phonetics}
\end{Entry}

\begin{Entry}{松}{8}{⽊}
  \begin{Phonetics}{松}{song1}[][HSK 4]
    \definition*{s.}{Sobrenome Song}
    \definition{adj.}{solto; frouxo; folgado | leve e crocante; macio | relaxado; confortável}
    \definition[棵]{s.}{pinheiro | fio de carne seca; carne moída seca; alimentos macios ou quebradiços}
    \definition{v.}{afrouxar; relaxar; abrandar | desamarrar; desatar; liberar}
  \end{Phonetics}
\end{Entry}

\begin{Entry}{松木}{8,4}{⽊、⽊}
  \begin{Phonetics}{松木}{song1mu4}
    \definition{s.}{pinheiro}
  \end{Phonetics}
\end{Entry}

\begin{Entry}{松树}{8,9}{⽊、⽊}
  \begin{Phonetics}{松树}{song1 shu4}[][HSK 4]
    \definition[棵]{s.}{pinheiro; conífera comum, geralmente com folhas longas e pontiagudas e cones lenhosos}
  \end{Phonetics}
\end{Entry}

\begin{Entry}{板}{8}{⽊}
  \begin{Phonetics}{板}{ban3}[][HSK 3]
    \definition{adj.}{rígido; não natural; inflexível}
    \definition[块,个]{s.}{tábua; placa; prato; objeto rígido em forma de placa | veneziana; persiana; refere-se especificamente aos painéis de portas de lojas | badalos (instrumento musical que marca o ritmo) | uma batida acentuada (ritmo) na música e na ópera tradicional | chefe}
    \definition{v.}{parecer sério | corrigir maus hábitos ou defeitos | ser rígido como uma tábua}
  \end{Phonetics}
\end{Entry}

\begin{Entry}{板块}{8,7}{⽊、⼟}
  \begin{Phonetics}{板块}{ban3kuai4}[][HSK 7-9]
    \definition[个]{s.}{placa tectônica; segmentos móveis da crosta terrestre | seção; uma metáfora para uma combinação de partes que têm algo em comum ou conectado}
  \end{Phonetics}
\end{Entry}

\begin{Entry}{构}{8}{⽊}
  \begin{Phonetics}{构}{gou4}
    \definition{s.}{composição literária}
    \definition{v.}{construir; formar; compor | fabricar; inventar | construir; erguer uma casa}
    \variantof{够}
  \end{Phonetics}
\end{Entry}

\begin{Entry}{构成}{8,6}{⽊、⼽}
  \begin{Phonetics}{构成}{gou4cheng2}[][HSK 4]
    \definition{s.}{parte; componente; composição; estrutura}
    \definition{v.}{formar; compor; constituir; compor; encaixar muitas partes para formar um todo | consistir; causar; formar (principalmente em termos jurídicos)}
  \end{Phonetics}
\end{Entry}

\begin{Entry}{构建}{8,8}{⽊、⼵}
  \begin{Phonetics}{构建}{gou4 jian4}[][HSK 6]
    \definition{v.}{estabelecer (usado principalmente para coisas abstratas); montar; instalar}
  \end{Phonetics}
\end{Entry}

\begin{Entry}{构思}{8,9}{⽊、⼼}
  \begin{Phonetics}{构思}{gou4si1}[][HSK 7-9]
    \definition{s.}{concepção (ideia); o resultado da concepção}
    \definition{v.}{elaborar o enredo de uma obra literária ou a composição de uma pintura; pensar bem antes de escrever artigos ou criar obras literárias}
  \end{Phonetics}
\end{Entry}

\begin{Entry}{构造}{8,10}{⽊、⾡}
  \begin{Phonetics}{构造}{gou4 zao4}[][HSK 4]
    \definition[种]{s.}{estrutura; construção; disposição, organização e inter-relação dos componentes}
    \definition{v.}{formar; construir}
  \end{Phonetics}
\end{Entry}

\begin{Entry}{构想}{8,13}{⽊、⼼}
  \begin{Phonetics}{构想}{gou4xiang3}[][HSK 7-9]
    \definition{s.}{ideia; concepção; ideias formadas}
    \definition[种,个]{v.}{pensar (em um plano, projeto, etc.); conceber; usar a mente ao escrever ou criar arte}
  \end{Phonetics}
\end{Entry}

\begin{Entry}{枕}{8}{⽊}
  \begin{Phonetics}{枕}{zhen3}
    \definition*{s.}{Sobrenome Zhen}
    \definition[个]{s.}{travesseiro; almofada | Mecânica: bloco}
    \definition{v.}{descansar a cabeça no travesseiro, almofada}
  \end{Phonetics}
\end{Entry}

\begin{Entry}{果}{8}{⽊}
  \begin{Phonetics}{果}{guo3}
    \definition*{s.}{Sobrenome Guo}
    \definition{adj.}{resoluto; determinado; sem exitação}
    \definition{adv.}{realmente; como esperado; com certeza; isso significa que as coisas são consistentes com as expectativas, equivalente a 果然}
    \definition{conj.}{se realmente; se de fato}
    \definition[个,些,种]{s.}{fruta; fruto da planta | resultado; consequência; o resultado final de um assunto (em oposição à 因)}
  \seealsoref{果然}{guo3ran2}
  \seealsoref{因}{yin1}
  \end{Phonetics}
\end{Entry}

\begin{Entry}{果子}{8,3}{⽊、⼦}
  \begin{Phonetics}{果子}{guo3zi5}
    \definition{s.}{fruta}
  \end{Phonetics}
\end{Entry}

\begin{Entry}{果汁}{8,5}{⽊、⽔}
  \begin{Phonetics}{果汁}{guo3zhi1}[][HSK 3]
    \definition[杯,瓶,种]{s.}{suco; suco de frutas frescas; também se refere a bebidas feitas com suco de frutas frescas}
  \end{Phonetics}
\end{Entry}

\begin{Entry}{果园}{8,7}{⽊、⼞}
  \begin{Phonetics}{果园}{guo3yuan2}[][HSK 7-9]
    \definition[个,座]{s.}{pomar; um jardim onde são plantadas árvores frutíferas}
  \end{Phonetics}
\end{Entry}

\begin{Entry}{果实}{8,8}{⽊、⼧}
  \begin{Phonetics}{果实}{guo3shi2}[][HSK 4]
    \definition[种]{s.}{fruta; o órgão que se desenvolve a partir do ovário ou com outras partes da flor após a fertilização da flor | ganhos; frutos;  uma metáfora para conquista ou recompensa por trabalho árduo}
  \end{Phonetics}
\end{Entry}

\begin{Entry}{果树}{8,9}{⽊、⽊}
  \begin{Phonetics}{果树}{guo3 shu4}[][HSK 6]
    \definition[棵,个,片]{s.}{árvore frutífera; árvores cujos frutos são principalmente comestíveis, como pessegueiros e macieiras}
  \end{Phonetics}
\end{Entry}

\begin{Entry}{果真}{8,10}{⽊、⼗}
  \begin{Phonetics}{果真}{guo3zhen1}[][HSK 7-9]
    \definition{adv.}{realmente; como esperado; com certeza}
    \definition{conj.}{se de fato; se realmente; se for o caso}[果真如此, 我就放心了。===Se for esse o caso, então ficarei aliviado.]
  \end{Phonetics}
\end{Entry}

\begin{Entry}{果断}{8,11}{⽊、⽄}
  \begin{Phonetics}{果断}{guo3duan4}[][HSK 7-9]
    \definition{adj.}{resoluto; decisivo; agir decisivamente sem hesitação}
  \end{Phonetics}
\end{Entry}

\begin{Entry}{果然}{8,12}{⽊、⽕}
  \begin{Phonetics}{果然}{guo3ran2}[][HSK 3]
    \definition{adv.}{realmente; como esperado; com certeza; indica que os fatos correspondem ao que foi dito ou esperado}
    \definition{conj.}{se realmente; se de fato; suponha que os fatos correspondam ao que foi dito ou esperado}
  \end{Phonetics}
\end{Entry}

\begin{Entry}{果酱}{8,13}{⽊、⾣}
  \begin{Phonetics}{果酱}{guo3 jiang4}[][HSK 6]
    \definition{s.}{geléia | compota ou doce (de frutas); fruta em conserva}
  \end{Phonetics}
\end{Entry}

\begin{Entry}{枝}{8}{⽊}
  \begin{Phonetics}{枝}{zhi1}[][HSK 6]
    \definition*{s.}{Sobrenome Zhi}
    \definition{clas.}{usado para flores com galhos, ramos | usado para objetos em forma de haste}
    \definition{s.}{ramo; galho}
  \end{Phonetics}
\end{Entry}

\begin{Entry}{枪}{8}{⽊}
  \begin{Phonetics}{枪}{qiang1}[][HSK 5]
    \definition*{s.}{Sobrenome Qiang}
    \definition[把,杆,支,挺]{s.}{lança | arma; rifle; arma de fogo | uma coisa em forma de arma | enxada; ferramenta para cavar a terra}
    \definition{v.}{escrever artigos ou responder perguntas para outras pessoas}
  \end{Phonetics}
\end{Entry}

\begin{Entry}{枫}{8}{⽊}
  \begin{Phonetics}{枫}{feng1}
    \definition[棵]{s.}{goma doce chinesa | bordo; \emph{maple}}
  \end{Phonetics}
\end{Entry}

\begin{Entry}{枫叶}{8,5}{⽊、⼝}
  \begin{Phonetics}{枫叶}{feng1ye4}
    \definition{s.}{folha de bordo (maple, tipo de árvore)}
  \end{Phonetics}
\end{Entry}

\begin{Entry}{柜}{8}{⽊}
  \begin{Phonetics}{柜}{gui4}
    \definition{s.}{baú; armário; gabinete | loja; balcão}
  \end{Phonetics}
  \begin{Phonetics}{柜}{ju3}
    \definition{s.}{faia; salgueiro}
  \end{Phonetics}
\end{Entry}

\begin{Entry}{柜子}{8,3}{⽊、⼦}
  \begin{Phonetics}{柜子}{gui4 zi5}[][HSK 5]
    \definition[个]{s.}{gabinete; armário; dispositivo para guardar roupas, documentos, livros, etc.}
  \end{Phonetics}
\end{Entry}

\begin{Entry}{柜台}{8,5}{⽊、⼝}
  \begin{Phonetics}{柜台}{gui4tai2}[][HSK 7-9]
    \definition[个,排,组]{s.}{bar; balcão; uma longa área semelhante a uma mesa em uma loja ou banco usada para vender mercadorias ou conduzir negócios}
  \end{Phonetics}
\end{Entry}

\begin{Entry}{枯}{9}{⽊}
  \begin{Phonetics}{枯}{ku1}
    \definition{adj.}{murcho | (de um poço, rio, etc.) seco | chato; desinteressante | magro e abatido; emaciado}
    \definition[片]{s.}{borra; resíduo}
  \end{Phonetics}
\end{Entry}

\begin{Entry}{枯木}{9,4}{⽊、⽊}
  \begin{Phonetics}{枯木}{ku1mu4}
    \definition{s.}{árvore morta | madeira morta}
  \end{Phonetics}
\end{Entry}

\begin{Entry}{架}{9}{⽊}
  \begin{Phonetics}{架}{jia4}[][HSK 3]
    \definition{clas.}{usado para coisas com pilares ou componentes mecânicos | quadrado (usado para montanhas)}
    \definition{s.}{estrutura; organização do corpo humano ou das coisas | prateleira; estante; suporte; componentes que sustentam objetos ou utensílios para colocar objetos, etc.}
    \definition{v.}{colocar para cima; erigir | brigar; discutir | resistir; repelir; afastar | sequestrar; levar alguém à força}
  \end{Phonetics}
\end{Entry}

\begin{Entry}{架子}{9,3}{⽊、⼦}
  \begin{Phonetics}{架子}{jia4zi5}[][HSK 7-9]
    \definition[个,种,套]{s.}{estrutura; suporte; um objeto feito de madeira, metal ou outros materiais que pode ser usado para armazenar ou pendurar coisas | esboço; estrutura; a organização e estrutura das coisas | ares; arrogância; maneiras altivas; pensar que você é melhor que os outros e fingir ser de uma certa maneira | postura; posição; pose}
  \end{Phonetics}
\end{Entry}

\begin{Entry}{架式}{9,6}{⽊、⼷}
  \begin{Phonetics}{架式}{jia4shi5}
    \variantof{架势}
  \end{Phonetics}
\end{Entry}

\begin{Entry}{架势}{9,8}{⽊、⼒}
  \begin{Phonetics}{架势}{jia4shi5}[][HSK 7-9]
    \definition{s.}{postura; atitude; posição (sobre um assunto, etc.)}
  \end{Phonetics}
\end{Entry}

\begin{Entry}{柁}{9}{⽊}
  \begin{Phonetics}{柁}{tuo2}
    \definition{s.}{leme; leme | viga; uma grande viga horizontal em uma treliça de telhado de madeira}
  \seealsoref{舵}{duo4}
  \end{Phonetics}
\end{Entry}

\begin{Entry}{柏}{9}{⽊}
  \begin{Phonetics}{柏}{bai3}
  \seealsoref{柏树}{bai3shu4}
  \end{Phonetics}
  \begin{Phonetics}{柏}{bo2}
    \definition{s.}{cipreste | usado para transcrever nomes}[柏林,德国城市名。===Berlim, uma cidade alemã.]
  \end{Phonetics}
  \begin{Phonetics}{柏}{bo4}
    \definition{s.}{cedro; cipreste amarelo}
  \end{Phonetics}
\end{Entry}

\begin{Entry}{柏林}{9,8}{⽊、⽊}
  \begin{Phonetics}{柏林}{bo2lin2}
    \definition*{s.}{Berlim, capital da Alemanha}
  \end{Phonetics}
\end{Entry}

\begin{Entry}{柏树}{9,9}{⽊、⽊}
  \begin{Phonetics}{柏树}{bai3shu4}[][HSK 7-9]
    \definition[棵,株]{s.}{cipreste}
  \end{Phonetics}
\end{Entry}

\begin{Entry}{某}{9}{⽊}
  \begin{Phonetics}{某}{mou3}[][HSK 3]
    \definition{pron.}{alguém ou algo indefinido; refere-se a pessoas ou coisas incertas | referindo-se a si mesmo; em vez do seu próprio nome | alguns; certos; refere-se a uma pessoa ou coisa específica cujo nome não se sabe ou não se pode revelar | tal e tal; substituir o nome de outra pessoa (geralmente com um tom rude)}
  \end{Phonetics}
\end{Entry}

\begin{Entry}{染}{9}{⽊}
  \begin{Phonetics}{染}{ran3}[][HSK 5]
    \definition*{s.}{Sobrenome Ran}
    \definition{s.}{soja fermentada e temperada em forma de pasta}
    \definition{v.}{tingir; pintar | pegar (uma doença); cair em (um mau hábito, etc.) | sujar; contaminar | pegar (contrair) (uma doença) | adquirir (um mau hábito, etc.); contaminar}
  \end{Phonetics}
\end{Entry}

\begin{Entry}{柔}{9}{⽊}
  \begin{Phonetics}{柔}{rou2}
    \definition*{s.}{Sobrenome Rou}
    \definition{adj.}{macio; flexível; maleável | gentil; flexível; brando}
    \definition{v.}{tornar macio; amolecer | apaziguar}
  \end{Phonetics}
\end{Entry}

\begin{Entry}{柔软}{9,8}{⽊、⾞}
  \begin{Phonetics}{柔软}{rou2ruan3}
    \definition{adj.}{macio | suave}
  \end{Phonetics}
\end{Entry}

\begin{Entry}{柠}{9}{⽊}
  \begin{Phonetics}{柠}{ning2}
    \definition{s.}{limão}
  \end{Phonetics}
\end{Entry}

\begin{Entry}{柠檬}{9,17}{⽊、⽊}
  \begin{Phonetics}{柠檬}{ning2meng2}
    \definition[个,片,只]{s.}{limão}
  \end{Phonetics}
\end{Entry}

\begin{Entry}{查}{9}{⽊}
  \begin{Phonetics}{查}{cha2}[][HSK 2]
    \definition{v.}{examinar; verificar cuidadosamente | examinar; investigar; entender bem a situação | procurar; consultar; revisar (documentos bibliográficos)}
  \end{Phonetics}
  \begin{Phonetics}{查}{zha1}
    \definition*{s.}{Sobrenome Zha}
    \definition{s.}{espinheiro-chinês}
  \end{Phonetics}
\end{Entry}

\begin{Entry}{查出}{9,5}{⽊、⼐}
  \begin{Phonetics}{查出}{cha2 chu1}[][HSK 6]
    \definition{v.}{rastrear; desentocar}
  \end{Phonetics}
\end{Entry}

\begin{Entry}{查处}{9,5}{⽊、⼡}
  \begin{Phonetics}{查处}{cha2chu3}[][HSK 7-9]
    \definition{v.}{investigar e lidar (com um caso criminal)}
  \end{Phonetics}
\end{Entry}

\begin{Entry}{查找}{9,7}{⽊、⼿}
  \begin{Phonetics}{查找}{cha2zhao3}[][HSK 7-9]
    \definition{v.}{procurar; pesquisar; tentar encontrar as informações que você precisa}
  \end{Phonetics}
\end{Entry}

\begin{Entry}{查明}{9,8}{⽊、⽇}
  \begin{Phonetics}{查明}{cha2ming2}[][HSK 7-9]
    \definition{v.}{provar por meio de investigação; descobrir; apurar}
  \end{Phonetics}
\end{Entry}

\begin{Entry}{查询}{9,8}{⽊、⾔}
  \begin{Phonetics}{查询}{cha2 xun2}[][HSK 5]
    \definition{v.}{indagar; inquirir; perguntar sobre}
  \end{Phonetics}
\end{Entry}

\begin{Entry}{查看}{9,9}{⽊、⽬}
  \begin{Phonetics}{查看}{cha2 kan4}[][HSK 6]
    \definition{v.}{verificar; examinar; checar; investigar; verificar e observar a existência das coisas}
  \end{Phonetics}
\end{Entry}

\begin{Entry}{柬}{9}{⽊}
  \begin{Phonetics}{柬}{jian3}
    \definition*{s.}{Sobrenome Jian}
    \definition[张,封]{s.}{cartão; nota; carta; um termo geral para cartas, cartões de visita, postagens, etc.}
  \end{Phonetics}
\end{Entry}

\begin{Entry}{柬埔寨}{9,10,14}{⽊、⼟、⼧}
  \begin{Phonetics}{柬埔寨}{jian3pu3zhai4}
    \definition*{s.}{Camboja}
  \end{Phonetics}
\end{Entry}

\begin{Entry}{柱}{9}{⽊}
  \begin{Phonetics}{柱}{zhu4}
    \definition*{s.}{Sobrenome Zhu}
    \definition[根]{s.}{poste; pilar; coluna | algo em forma de coluna | Matemática: cilindro}
  \end{Phonetics}
\end{Entry}

\begin{Entry}{柱子}{9,3}{⽊、⼦}
  \begin{Phonetics}{柱子}{zhu4 zi5}[][HSK 6]
    \definition{s.}{poste; pilar; coluna; estrutura de suporte vertical de um edifício, feita de madeira, pedra, aço, concreto armado, etc.}
  \end{Phonetics}
\end{Entry}

\begin{Entry}{柳}{9}{⽊}
  \begin{Phonetics}{柳}{liu3}
    \definition*{s.}{Liu, a vigésima quarta das vinte e oito constelações, consistindo de oito estrelas em Hydra | Liu, uma das mansões lunares | Sobrenome Liu}
    \definition[棵]{s.}{salgueiro}
  \end{Phonetics}
\end{Entry}

\begin{Entry}{柳橙汁}{9,16,5}{⽊、⽊、⽔}
  \begin{Phonetics}{柳橙汁}{liu3cheng2zhi1}
    \definition[瓶,杯,罐,盒]{s.}{suco de laranja}
  \seealsoref{橙汁}{cheng2zhi1}
  \seealsoref{橘子汁}{ju2zi5zhi1}
  \end{Phonetics}
\end{Entry}

\begin{Entry}{标}{9}{⽊}
  \begin{Phonetics}{标}{biao1}[][HSK 7-9]
    \definition{clas.}{usado para equipes (o numeral é limitado a um, 一, o que é comum no chinês moderno)}
    \definition[个]{s.}{copa da árvore (significado original) | marca; sinal | padrão; cota | sinal externo; sintoma | prêmio; troféu | oferta; licitação comercial pública | a ponta de uma árvore | aparência externa; ramos ou superfícies | partes aéreas das plantas | rótulo; etiqueta; identificação; sinal | regimento na Dinastia Qing; uma das organizações militares no final da Dinastia Qing}
    \definition{v.}{colocar uma marca, etiqueta ou rótulo em; rotular | agrupar; formar equipe | marcar; expressar com palavras ou outras coisas |}
  \end{Phonetics}
\end{Entry}

\begin{Entry}{标本}{9,5}{⽊、⽊}
  \begin{Phonetics}{标本}{biao1ben3}[][HSK 7-9]
    \definition{s.}{espécime; amostra | Medicina chinesa: causa raiz e sintomas de uma doença}
  \end{Phonetics}
\end{Entry}

\begin{Entry}{标示}{9,5}{⽊、⽰}
  \begin{Phonetics}{标示}{biao1shi4}[][HSK 7-9]
    \definition{v.}{marcar; indicar | indicar; exibir como texto ou gráfico}
  \end{Phonetics}
\end{Entry}

\begin{Entry}{标志}{9,7}{⽊、⼼}
  \begin{Phonetics}{标志}{biao1zhi4}[][HSK 4]
    \definition[个,种]{s.}{sinal; marca; logotipo; símbolo; emblema; marcações que caracterizam um objeto para facilitar a identificação}
    \definition{v.}{marcar; indicar; simbolizar; identificar}
  \end{Phonetics}
\end{Entry}

\begin{Entry}{标语}{9,9}{⽊、⾔}
  \begin{Phonetics}{标语}{biao1yu3}[][HSK 7-9]
    \definition[幅,张,条,个]{s.}{\emph{slogan}; cartaz; \emph{slogans} curtos de propaganda afixados ou pendurados em locais públicos}
  \end{Phonetics}
\end{Entry}

\begin{Entry}{标准}{9,10}{⽊、⼎}
  \begin{Phonetics}{标准}{biao1zhun3}[][HSK 3]
    \definition{adj.}{padrão (que serve como ou está em conformidade com um padrão); em conformidade com os documentos e princípios regulamentares}
    \definition[个,条,项,种]{s.}{padrão; critério; critérios de avaliação das coisas}
  \end{Phonetics}
\end{Entry}

\begin{Entry}{标致}{9,10}{⽊、⾄}
  \begin{Phonetics}{标致}{biao1zhi4}
    \definition*{s.}{Peugeot, montadora de automóveis}
    \definition{adj.}{bela aparência e postura (principalmente para mulheres)}
  \end{Phonetics}
  \begin{Phonetics}{标致}{biao1zhi5}[][HSK 7-9]
    \definition{adj.}{bonita (mulher)}
  \end{Phonetics}
\end{Entry}

\begin{Entry}{标签}{9,13}{⽊、⽵}
  \begin{Phonetics}{标签}{biao1qian1}[][HSK 7-9]
    \definition[个,张,枚,套]{s.}{rótulo; etiqueta; um pedaço de papel anexado ou amarrado a um item para indicar o nome do produto, finalidade, preço, etc.}
  \end{Phonetics}
\end{Entry}

\begin{Entry}{标榜}{9,14}{⽊、⽊}
  \begin{Phonetics}{标榜}{biao1bang3}[][HSK 7-9]
    \definition{v.}{ostentar; anunciar; desfilar | elogiar; elogiar excessivamente | dar publicidade favorável a; fazer uma exibição de; gabar-se | impulsionar; elogiar excessivamente; gabar-se de}
  \end{Phonetics}
\end{Entry}

\begin{Entry}{标题}{9,15}{⽊、⾴}
  \begin{Phonetics}{标题}{biao1ti2}[][HSK 3]
    \definition[个,条,篇]{s.}{título; manchete; cabeçalho; resumo conciso do conteúdo da obra}
  \end{Phonetics}
\end{Entry}

\begin{Entry}{栋}{9}{⽊}
  \begin{Phonetics}{栋}{dong4}[][HSK 7-9]
    \definition*{s.}{Sobrenome Dong}
    \definition{clas.}{edifício; prédio}
    \definition{s.}{Literário: cumeeira; viga principal}
  \end{Phonetics}
\end{Entry}

\begin{Entry}{栋梁}{9,11}{⽊、⽊}
  \begin{Phonetics}{栋梁}{dong4liang2}[][HSK 7-9]
    \definition{s.}{cumeeira e vigas; esteio (de organização); viga de cumeeira; cumeeira; placa de cumeeira; tábua de cumeeira | pessoa capaz de suportar grandes responsabilidades | pilar (do estado)}
  \end{Phonetics}
\end{Entry}

\begin{Entry}{栏}{9}{⽊}
  \begin{Phonetics}{栏}{lan2}
    \definition{s.}{cerca; corrimão; balaustrada | curral; galpão; celeiro; chiqueiro | coluna (de uma página ou tabela, ou de um jornal) | quadro (de avisos); prancha; tabuleiro | Esporte: obstáculo}
  \end{Phonetics}
\end{Entry}

\begin{Entry}{栏目}{9,5}{⽊、⽬}
  \begin{Phonetics}{栏目}{lan2mu4}[][HSK 6]
    \definition[个,档]{s.}{coluna; programa; seções nomeadas de jornais, revistas, etc. divididas de acordo com a natureza de seu conteúdo}
  \end{Phonetics}
\end{Entry}

\begin{Entry}{树}{9}{⽊}
  \begin{Phonetics}{树}{shu4}[][HSK 1]
    \definition*{s.}{Sobrenome Shu}
    \definition[棵,株]{s.}{árvore; nome comum das plantas lenhosas}
    \definition{v.}{plantar; cultivar | configurar; manter; estabelecer}
  \end{Phonetics}
\end{Entry}

\begin{Entry}{树木}{9,4}{⽊、⽊}
  \begin{Phonetics}{树木}{shu4mu4}
    \definition{s.}{árvore}
  \end{Phonetics}
\end{Entry}

\begin{Entry}{树叶}{9,5}{⽊、⼝}
  \begin{Phonetics}{树叶}{shu4ye4}[][HSK 4]
    \definition[片,枚,堆]{s.}{folha; folhagem}
  \end{Phonetics}
\end{Entry}

\begin{Entry}{树林}{9,8}{⽊、⽊}
  \begin{Phonetics}{树林}{shu4 lin2}[][HSK 4]
    \definition[片,座]{s.}{bosque; muitas árvores que crescem em fragmentos, menores que as florestas}
  \end{Phonetics}
\end{Entry}

\begin{Entry}{树莓}{9,10}{⽊、⾋}
  \begin{Phonetics}{树莓}{shu4mei2}
    \definition{s.}{framboesa}
  \end{Phonetics}
\end{Entry}

\begin{Entry}{校}{10}{⽊}
  \begin{Phonetics}{校}{jiao4}
    \definition{v.}{verificar | comparar | revisar}
  \end{Phonetics}
  \begin{Phonetics}{校}{xiao4}
    \definition[所]{s.}{oficial militar | escola}
  \end{Phonetics}
\end{Entry}

\begin{Entry}{校长}{10,4}{⽊、⾧}
  \begin{Phonetics}{校长}{xiao4zhang3}[][HSK 2]
    \definition[个,位,名]{s.}{diretor; presidente; reitor; o mais alto líder administrativo e empresarial de uma escola}
  \end{Phonetics}
\end{Entry}

\begin{Entry}{校园}{10,7}{⽊、⼞}
  \begin{Phonetics}{校园}{xiao4 yuan2}[][HSK 2]
    \definition[个]{s.}{campus; pátio da escola; refere-se a todos os terrenos e edifícios dentro da área escolar}
  \end{Phonetics}
\end{Entry}

\begin{Entry}{校服}{10,8}{⽊、⽉}
  \begin{Phonetics}{校服}{xiao4fu2}
    \definition{s.}{uniforme escolar}
  \end{Phonetics}
\end{Entry}

\begin{Entry}{校规}{10,8}{⽊、⾒}
  \begin{Phonetics}{校规}{xiao4gui1}
    \definition{s.}{regras e regulamentos escolares}
  \end{Phonetics}
\end{Entry}

\begin{Entry}{校监}{10,10}{⽊、⽫}
  \begin{Phonetics}{校监}{xiao4jian1}
    \definition{s.}{diretor | supervisor (de escola)}
  \end{Phonetics}
\end{Entry}

\begin{Entry}{样}{10}{⽊}
  \begin{Phonetics}{样}{yang4}[][HSK 6]
    \definition{clas.}{usado para tipos de coisas}[这里有四样东西。===Há quatro coisas aqui.]
    \definition[个]{s.}{aparência; aspecto;  forma; aparência; a forma do objeto | modelo; amostra; padrão; coisas usadas como padrões | ar; maneira; aparência; a aparência ou expressão de uma pessoa | tendência; probabilidade; a situação ou tendência das coisas}
  \end{Phonetics}
\end{Entry}

\begin{Entry}{样儿}{10,2}{⽊、⼉}
  \begin{Phonetics}{样儿}{yang4r5}
    \definition{s.}{aparência | forma | modelo}
  \seealsoref{样子}{yang4zi5}
  \end{Phonetics}
\end{Entry}

\begin{Entry}{样子}{10,3}{⽊、⼦}
  \begin{Phonetics}{样子}{yang4zi5}[][HSK 2]
    \definition[个,种,副]{s.}{forma; aparência; estilo | ar; maneira; modalidade; estado | tendência; probabilidade; usado com 看 e 照 para expressar uma estimativa de uma tendência | modelo; amostra; padrão; uma pessoa ou coisa que pode ser usada como um padrão para as pessoas verificarem, seguirem ou aprenderem com ela}
  \seealsoref{看}{kan4}
  \seealsoref{样儿}{yang4r5}
  \seealsoref{照}{zhao4}
  \end{Phonetics}
\end{Entry}

\begin{Entry}{样品}{10,9}{⽊、⼝}
  \begin{Phonetics}{样品}{yang4pin3}
    \definition{s.}{amostra | espécime}
  \end{Phonetics}
\end{Entry}

\begin{Entry}{样样}{10,10}{⽊、⽊}
  \begin{Phonetics}{样样}{yang4yang4}
    \definition{adv.}{todos os tipos}
  \end{Phonetics}
\end{Entry}

\begin{Entry}{样章}{10,11}{⽊、⾳}
  \begin{Phonetics}{样章}{yang4zhang1}
    \definition{s.}{capítulo de amostra}
  \end{Phonetics}
\end{Entry}

\begin{Entry}{核}{10}{⽊}
  \begin{Phonetics}{核}{he2}[][HSK 7-9]
    \definition{adj.}{Literário: verdadeiro; fiel}
    \definition{s.}{poço; pedra; caroço | núcleo | núcleo atômico}
    \definition{v.}{examinar; verificar}
  \end{Phonetics}
  \begin{Phonetics}{核}{hu2}
    \definition{s.}{semente; o mesmo que 核}
  \end{Phonetics}
\end{Entry}

\begin{Entry}{核心}{10,4}{⽊、⼼}
  \begin{Phonetics}{核心}{he2xin1}[][HSK 6]
    \definition[个]{s.}{núcleo; elite; coração; centro; parte principal (em termos de relacionamento entre as coisas)}
  \end{Phonetics}
\end{Entry}

\begin{Entry}{核对}{10,5}{⽊、⼨}
  \begin{Phonetics}{核对}{he2dui4}[][HSK 7-9]
    \definition{v.}{verificar; checar; verificar cuidadosamente (para ver se corresponde)}
  \end{Phonetics}
\end{Entry}

\begin{Entry}{核电站}{10,5,10}{⽊、⽥、⽴}
  \begin{Phonetics}{核电站}{he2dian4zhan4}[][HSK 7-9]
    \definition{s.}{usina nuclear; usina que utiliza energia nuclear para gerar eletricidade}
  \end{Phonetics}
\end{Entry}

\begin{Entry}{核实}{10,8}{⽊、⼧}
  \begin{Phonetics}{核实}{he2shi2}[][HSK 7-9]
    \definition{v.}{verificar; checar; verificar se é verdade}
  \end{Phonetics}
\end{Entry}

\begin{Entry}{核武器}{10,8,16}{⽊、⽌、⼝}
  \begin{Phonetics}{核武器}{he2wu3qi4}[][HSK 7-9]
    \definition[个]{s.}{arma nuclear}
  \end{Phonetics}
\end{Entry}

\begin{Entry}{核桃}{10,10}{⽊、⽊}
  \begin{Phonetics}{核桃}{he2tao5}[][HSK 7-9]
    \definition[颗,个,棵,顆]{s.}{noz | nogueira}
  \end{Phonetics}
\end{Entry}

\begin{Entry}{核能}{10,10}{⽊、⾁}
  \begin{Phonetics}{核能}{he2neng2}[][HSK 7-9]
    \definition{s.}{energia nuclear}
  \end{Phonetics}
\end{Entry}

\begin{Entry}{根}{10}{⽊}
  \begin{Phonetics}{根}{gen1}[][HSK 4]
    \definition*{s.}{Sobrenome Gen}
    \definition{adv.}{completamente; minuciosamente; radicalmente}
    \definition{clas.}{usado para objetos finos, alongados}
    \definition{s.}{raiz (de uma planta) | descendentes; posteridade; analogia com as gerações futuras | raiz (abreviação de raiz quadrada) | radical (química, refere-se a radicais carregados) | base; pé; raiz; parte inferior, base ou parte de um objeto que está presa a outra coisa | a parte de baixo das coisas; fonte; a origem  das coisas | base; fundamento}
  \end{Phonetics}
\end{Entry}

\begin{Entry}{根本}{10,5}{⽊、⽊}
  \begin{Phonetics}{根本}{gen1ben3}[][HSK 3]
    \definition{adj.}{básico; essencial; fundamental; importante; decisivo}
    \definition{adv.}{nunca; simplesmente; de forma alguma | radicalmente; completamente; nunca (mais usado em negativas)}
    \definition[个]{s.}{base; raiz; fundação; a origem, a base ou a parte mais importante das coisas}
  \end{Phonetics}
\end{Entry}

\begin{Entry}{根治}{10,8}{⽊、⽔}
  \begin{Phonetics}{根治}{gen1zhi4}[][HSK 7-9]
    \definition{v.}{efetuar uma cura radical; curar de uma vez por todas; colocar sob controle permanente; curar completamente (referindo-se à erradicação de pragas ou doenças)}
  \end{Phonetics}
\end{Entry}

\begin{Entry}{根基}{10,11}{⽊、⼟}
  \begin{Phonetics}{根基}{gen1ji1}[][HSK 7-9]
    \definition{s.}{base; fundação; alicerce; parte subterrânea de um edifício | recursos; propriedade acumulada ao longo de um longo período}
  \end{Phonetics}
\end{Entry}

\begin{Entry}{根据}{10,11}{⽊、⼿}
  \begin{Phonetics}{根据}{gen1ju4}[][HSK 4]
    \definition{prep.}{com base em; de acordo com; à luz de}
    \definition[个]{s.}{base; fundamentos; razão; fundo; alicerce}
    \definition{v.}{basear; usar algo como premissa para uma conclusão ou como base para uma ação verbal}
  \end{Phonetics}
\end{Entry}

\begin{Entry}{根深蒂固}{10,11,12,8}{⽊、⽔、⾋、⼞}
  \begin{Phonetics}{根深蒂固}{gen1shen1-di4gu4}[][HSK 7-9]
    \definition{expr.}{arraigado; inveterado; tornar-se profundamente enraizado em; profundamente enraizado; profundamente enraizado; profundamente enraizado e firmemente plantado -- bem fundado; ter uma base firme; ter raízes profundas e uma base firme; bem estabelecido; significa que a fundação é sólida e não se abala facilmente}
  \end{Phonetics}
\end{Entry}

\begin{Entry}{根源}{10,13}{⽊、⽔}
  \begin{Phonetics}{根源}{gen1yuan2}[][HSK 7-9]
    \definition{s.}{fonte; origem; raiz | raízes da grama; fonte; nascente; raiz; fundo}
    \definition{v.}{originar-se; provir de}
  \end{Phonetics}
\end{Entry}

\begin{Entry}{格}{10}{⽊}
  \begin{Phonetics}{格}{ge1}
    \definition{s.}{Onomatopéia: estalo (som); riso zombeteiro}
  \end{Phonetics}
  \begin{Phonetics}{格}{ge2}[][HSK 7-9]
    \definition*{s.}{Sobrenome Ge}
    \definition{s.}{quadrados formados por linhas cruzadas; quadriculado; grade | divisão (horizontal ou não); treliça | padrão; forma; formato; estilo | caso; as categorias morfológicas de substantivos, pronomes e adjetivos em algumas línguas}
    \definition{v.}{resistir; dificultar; obstruir; impedir | estudar cuidadosamente; investigar | lutar; bater}
  \end{Phonetics}
\end{Entry}

\begin{Entry}{格兰菜}{10,5,11}{⽊、⼋、⾋}
  \begin{Phonetics}{格兰菜}{ge2lan2cai4}
    \definition{s.}{brócolis chinês | couve chinesa | mostarda}
  \seealsoref{芥蓝}{gai4lan2}
  \end{Phonetics}
\end{Entry}

\begin{Entry}{格外}{10,5}{⽊、⼣}
  \begin{Phonetics}{格外}{ge2wai4}[][HSK 4]
    \definition{adv.}{especialmente; particularmente; ainda mais; indica mais do que a média | adicionalmente; indica adicional ou extra}
  \end{Phonetics}
\end{Entry}

\begin{Entry}{格式}{10,6}{⽊、⼷}
  \begin{Phonetics}{格式}{ge2shi5}[][HSK 7-9]
    \definition[种]{s.}{forma; estilo; \emph{layout}; padrão; formato; modo}
  \end{Phonetics}
\end{Entry}

\begin{Entry}{格局}{10,7}{⽊、⼫}
  \begin{Phonetics}{格局}{ge2ju2}[][HSK 7-9]
    \definition{s.}{padrão; configuração; estrutura; estilo; maneira; arranjo | a visão ou percepção de uma situação geral; a visão de uma pessoa, a altura e a profundidade da consideração do problema}
  \end{Phonetics}
\end{Entry}

\begin{Entry}{格格不入}{10,10,4,2}{⽊、⽊、⼀、⼊}
  \begin{Phonetics}{格格不入}{ge2ge2-bu2ru4}[][HSK 7-9]
    \definition{expr.}{incompatível com; fora de sintonia com; estranho; fora do seu elemento; como uma estaca quadrada em um buraco redondo; desarmônico}
  \end{Phonetics}
\end{Entry}

\begin{Entry}{栽}{10}{⽊}
  \begin{Phonetics}{栽}{zai1}
    \definition{v.}{cultivar | plantar}
  \end{Phonetics}
\end{Entry}

\begin{Entry}{栽种}{10,9}{⽊、⽲}
  \begin{Phonetics}{栽种}{zai1zhong4}
    \definition{v.}{plantar}
  \end{Phonetics}
\end{Entry}

\begin{Entry}{栽倒}{10,10}{⽊、⼈}
  \begin{Phonetics}{栽倒}{zai1dao3}
    \definition{v.}{cair | sofrer uma queda}
  \end{Phonetics}
\end{Entry}

\begin{Entry}{栽赃}{10,10}{⽊、⾙}
  \begin{Phonetics}{栽赃}{zai1zang1}
    \definition{v.}{enquadrar alguém (plantar provas nele)}
  \end{Phonetics}
\end{Entry}

\begin{Entry}{栽培}{10,11}{⽊、⼟}
  \begin{Phonetics}{栽培}{zai1pei2}
    \definition{v.}{cultivar | educar | patrocinar | treinar}
  \end{Phonetics}
\end{Entry}

\begin{Entry}{栽培种}{10,11,9}{⽊、⼟、⽲}
  \begin{Phonetics}{栽培种}{zai1pei2 zhong3}
    \definition{s.}{espécies cultivadas}
  \end{Phonetics}
\end{Entry}

\begin{Entry}{栽植}{10,12}{⽊、⽊}
  \begin{Phonetics}{栽植}{zai1zhi2}
    \definition{v.}{plantar | transplantar}
  \end{Phonetics}
\end{Entry}

\begin{Entry}{桂}{10}{⽊}
  \begin{Phonetics}{桂}{gui4}
    \definition*{s.}{outro nome para o rio Guijiang 桂江 (em Guangxi 广西) | outro nome para Guangxi 广西 (Região Autônoma de Zhuang) | Sobrenome Gui}
    \definition[棵]{s.}{louro; loureiro | osmanthus de aroma doce | árvore de casca de cássia | canela; osmanthus}
  \seealsoref{广西}{guang3xi1}
  \seealsoref{桂江}{gui4jiang1}
  \end{Phonetics}
\end{Entry}

\begin{Entry}{桂江}{10,6}{⽊、⽔}
  \begin{Phonetics}{桂江}{gui4jiang1}
    \definition*{s.}{Rio Guijiang}
  \end{Phonetics}
\end{Entry}

\begin{Entry}{桂花}{10,7}{⽊、⾋}
  \begin{Phonetics}{桂花}{gui4hua1}[][HSK 7-9]
    \definition{s.}{jasmim do imperador; um arbusto perene ou pequena árvore, cujas flores também são chamadas de osmanthus, são muito perfumadas e podem ser usadas para extrair óleos aromáticos ou fazer especiarias. Variedades comuns incluem Jingui 金桂 (flores amarelo-alaranjadas), Dangui 丹桂 (flores vermelho-alaranjadas), Yingui 银桂 (flores branco-amareladas) e Sijigui 四季桂 (flores branco-amareladas).}
  \end{Phonetics}
\end{Entry}

\begin{Entry}{桃}{10}{⽊}
  \begin{Phonetics}{桃}{tao2}[][HSK 5]
    \definition*{s.}{Sobrenome Tao}
    \definition[个,箱,袋,斤,棵,种]{s.}{pêssego | em forma de pêssego | pessegueiro}
  \end{Phonetics}
\end{Entry}

\begin{Entry}{桃花}{10,7}{⽊、⾋}
  \begin{Phonetics}{桃花}{tao2 hua1}[][HSK 5]
    \definition[朵,枝,株]{s.}{Figurativo: caso amoroso | flor de pessegueiro}
  \end{Phonetics}
\end{Entry}

\begin{Entry}{桃树}{10,9}{⽊、⽊}
  \begin{Phonetics}{桃树}{tao2 shu4}[][HSK 5]
    \definition[棵,株]{s.}{pêssego (árvore) | pessegueiro; pêssegos}
  \end{Phonetics}
\end{Entry}

\begin{Entry}{案}{10}{⽊}
  \begin{Phonetics}{案}{an4}
    \definition{s.}{mesa; escrivaninha; mesa longa | caso; caso de direito (legal) | registro; arquivo; arquivo de caso | um plano submetido para consideração; proposta; um documento que propõe planos, sugestões, métodos, etc.}
  \end{Phonetics}
\end{Entry}

\begin{Entry}{案件}{10,6}{⽊、⼈}
  \begin{Phonetics}{案件}{an4jian4}[][HSK 7-9]
    \definition[个,起,件,类]{s.}{caso; caso de direito; caso legal; contencioso e eventos ilegais}
  \end{Phonetics}
\end{Entry}

\begin{Entry}{桌}{10}{⽊}
  \begin{Phonetics}{桌}{zhuo1}
    \definition{clas.}{usado para mesas de convidados em um banquete etc.}
    \definition{s.}{mesa}
  \end{Phonetics}
\end{Entry}

\begin{Entry}{桌子}{10,3}{⽊、⼦}
  \begin{Phonetics}{桌子}{zhuo1zi5}[][HSK 1]
    \definition[张,套]{s.}{mesa; escrivaninha; móveis, com uma superfície plana na parte superior e uma estrutura de suporte na parte inferior, para colocar objetos ou realizar atividades}
  \end{Phonetics}
\end{Entry}

\begin{Entry}{桌布}{10,5}{⽊、⼱}
  \begin{Phonetics}{桌布}{zhuo1bu4}
    \definition[条,块,张]{s.}{(computação) plano de fundo da área de trabalho | toalha de mesa | papel de parede}
  \end{Phonetics}
\end{Entry}

\begin{Entry}{桌机}{10,6}{⽊、⽊}
  \begin{Phonetics}{桌机}{zhuo1ji1}
    \definition{s.}{computador \emph{desktop}}
  \end{Phonetics}
\end{Entry}

\begin{Entry}{桌灯}{10,6}{⽊、⽕}
  \begin{Phonetics}{桌灯}{zhuo1deng1}
    \definition{s.}{luminária | lâmpada de mesa}
  \end{Phonetics}
\end{Entry}

\begin{Entry}{桌面}{10,9}{⽊、⾯}
  \begin{Phonetics}{桌面}{zhuo1mian4}
    \definition{s.}{área de trabalho | mesa}
  \end{Phonetics}
\end{Entry}

\begin{Entry}{桌球}{10,11}{⽊、⽟}
  \begin{Phonetics}{桌球}{zhuo1qiu2}
    \definition{s.}{bilhar | sinuca | mesa de ping-pong}
  \end{Phonetics}
\end{Entry}

\begin{Entry}{桌游}{10,12}{⽊、⽔}
  \begin{Phonetics}{桌游}{zhuo1you2}
    \definition{s.}{jogo de tabuleiro}
  \end{Phonetics}
\end{Entry}

\begin{Entry}{桑}{10}{⽊}
  \begin{Phonetics}{桑}{sang1}
    \definition*{s.}{Sobrenome Sang}
    \definition[棵]{s.}{amoreira}
  \end{Phonetics}
\end{Entry}

\begin{Entry}{桑巴舞}{10,4,14}{⽊、⼰、⾇}
  \begin{Phonetics}{桑巴舞}{sang1ba1wu3}
    \definition{s.}{samba}
  \end{Phonetics}
\end{Entry}

\begin{Entry}{桑树}{10,9}{⽊、⽊}
  \begin{Phonetics}{桑树}{sang1shu4}
    \definition{s.}{amoreira, suas folhas são utilizadas para alimentar bichos-da-seda}
  \end{Phonetics}
\end{Entry}

\begin{Entry}{档}{10}{⽊}
  \begin{Phonetics}{档}{dang4}[][HSK 6]
    \definition{clas.}{festa; usado para eventos, shows}
    \definition{s.}{prateleiras (para arquivos); compartimentos para documentos | arquivos; arquivos | travessa (de uma mesa, etc.) | qualidade; nota}
  \end{Phonetics}
\end{Entry}

\begin{Entry}{档次}{10,6}{⽊、⽋}
  \begin{Phonetics}{档次}{dang4ci4}[][HSK 7-9]
    \definition{s.}{classe; grau; qualidade; nível; diferentes níveis divididos de acordo com certos padrões}
  \end{Phonetics}
\end{Entry}

\begin{Entry}{档案}{10,10}{⽊、⽊}
  \begin{Phonetics}{档案}{dang4'an4}[][HSK 6]
    \definition[份,个]{s.}{arquivos; registro; dossiê; arquivos e materiais armazenados de forma classificada para referência futura}
  \end{Phonetics}
\end{Entry}

\begin{Entry}{桥}{10}{⽊}
  \begin{Phonetics}{桥}{qiao2}[][HSK 3]
    \definition*{s.}{Sobrenome Qiao}
    \definition[座]{s.}{ponte; construção que atravessa a água conectando as duas margens}
  \end{Phonetics}
\end{Entry}

\begin{Entry}{桥梁}{10,11}{⽊、⽊}
  \begin{Phonetics}{桥梁}{qiao2liang2}[][HSK 6]
    \definition[座]{s.}{ponte; acesso; uma obra construída na superfície do rio, conectando as duas margens | ponte; metáfora para pessoas ou coisas que podem se comunicar}
  \end{Phonetics}
\end{Entry}

\begin{Entry}{桩}{10}{⽊}
  \begin{Phonetics}{桩}{zhuang1}
    \definition{clas.}{para eventos, casos, transações, assuntos, etc.}
    \definition{s.}{toco | estaca | pilha}
  \end{Phonetics}
\end{Entry}

\begin{Entry}{梅}{11}{⽊}
  \begin{Phonetics}{梅}{mei2}
    \definition*{s.}{Sobrenome Mei}
    \definition{s.}{ameixa | flor de ameixa | ameixeira | estação chuvosa}
  \end{Phonetics}
\end{Entry}

\begin{Entry}{梅花}{11,7}{⽊、⾋}
  \begin{Phonetics}{梅花}{mei2 hua1}[][HSK 6]
    \definition[朵,枝,片,瓣,束,株]{s.}{paus ♣ (um naipe em jogos de cartas) | flor de ameixa | doçura-de-inverno; refere-se especificamente à flor-de-inverno ; também se refere a algo que se parece com esta flor}
  \seealsoref{方片}{fang1 pian4}
  \seealsoref{黑桃}{hei1 tao2}
  \seealsoref{红心}{hong2 xin1}
  \end{Phonetics}
\end{Entry}

\begin{Entry}{梅赛德斯-奔驰}{11,14,15,12,8,6}{⽊、⾙、⼻、⽄、⼤、⾺}
  \begin{Phonetics}{梅赛德斯-奔驰}{mei2sai4de2si1-ben1chi2}
    \definition*{s.}{Mercedes-Benz}
  \end{Phonetics}
\end{Entry}

\begin{Entry}{梨}{11}{⽊}
  \begin{Phonetics}{梨}{li2}[][HSK 5]
    \definition*{s.}{Sobrenome Li}
    \definition[个,只,斤,棵,种]{s.}{perira; árvore de pera | pera}
  \end{Phonetics}
\end{Entry}

\begin{Entry}{梯}{11}{⽊}
  \begin{Phonetics}{梯}{ti1}
    \definition*{s.}{Sobrenome Ti}
    \definition{adj.}{em forma de escada; em socalcos}
    \definition[个]{s.}{escada; degrau; socalco (são plataformas niveladas, semelhantes a degraus, cortadas em encostas de morros para permitir o cultivo agrícola e evitar a erosão do solo)}
  \end{Phonetics}
\end{Entry}

\begin{Entry}{梯恩梯}{11,10,11}{⽊、⼼、⽊}
  \begin{Phonetics}{梯恩梯}{ti1'en1ti1}
    \definition{s.}{(empréstimo linguístico) TNT, trinitrotolueno}
  \end{Phonetics}
\end{Entry}

\begin{Entry}{检}{11}{⽊}
  \begin{Phonetics}{检}{jian3}
    \definition*{s.}{Sobrenome Jian}
    \definition{v.}{verificar; inspecionar; examinar | conter-se; ter cuidado na conduta}
  \end{Phonetics}
\end{Entry}

\begin{Entry}{检查}{11,9}{⽊、⽊}
  \begin{Phonetics}{检查}{jian3cha2}[][HSK 2]
    \definition[份,个,次]{s.}{autocrítica; reconhecer e criticar os próprios erros verbais ou escritos}
    \definition{v.}{verificar; inspecionar; examinar; verificar cuidadosamente para descobrir o problema | criticar a si mesmo; identificar seus pontos fracos e erros, e criticar seu próprio comportamento}
  \end{Phonetics}
\end{Entry}

\begin{Entry}{检测}{11,9}{⽊、⽔}
  \begin{Phonetics}{检测}{jian3 ce4}[][HSK 4]
    \definition{v.}{testar; detectar; verificar}
  \end{Phonetics}
\end{Entry}

\begin{Entry}{检验}{11,10}{⽊、⾺}
  \begin{Phonetics}{检验}{jian3yan4}[][HSK 5]
    \definition{v.}{testar; examinar; inspecionar}
  \end{Phonetics}
\end{Entry}

\begin{Entry}{毫}{11}{⽊}
  \begin{Phonetics}{毫}{hao2}
    \definition{adv.}{nem um pouco; absolutamente nenhum; completamente sem}
    \definition{clas.}{hao, uma unidade de comprimento igual a um milésimo de polegada ou 1/30 de milímetro | hao, uma unidade de peso igual a um milésimo de um centavo ou 0,005 grama |
uma fração minúscula; uma parte muito pequena}
    \definition{pref.}{mili-, usado com a unidade de uma quantidade física para representar um milésimo dessa quantidade}
    \definition{s.}{cabelo longo e fino | pincel de escrita | uma das duas ou três alças de uma balança para pendurar na mão do usuário | cerda; uma corda de mão em uma balança ou equilíbrio | fio de cabelo}
  \end{Phonetics}
\end{Entry}

\begin{Entry}{毫不}{11,4}{⽊、⼀}
  \begin{Phonetics}{毫不}{hao2 bu4}[][HSK 7-9]
    \definition{adv.}{dificilmente; de ​​jeito nenhum; nem um pouco}
  \end{Phonetics}
\end{Entry}

\begin{Entry}{毫不犹豫}{11,4,7,15}{⽊、⼀、⽝、⾗}
  \begin{Phonetics}{毫不犹豫}{hao2 bu4 you2yu4}[][HSK 7-9]
    \definition{expr.}{sem hesitação; sem a menor hesitação}
  \end{Phonetics}
\end{Entry}

\begin{Entry}{毫不费力}{11,4,9,2}{⽊、⼀、⾙、⼒}
  \begin{Phonetics}{毫不费力}{hao2bu2fei4li4}
    \definition{expr.}{sem esforço; não despender o menor esforço}
  \end{Phonetics}
\end{Entry}

\begin{Entry}{毫升}{11,4}{⽊、⼗}
  \begin{Phonetics}{毫升}{hao2 sheng1}[][HSK 4]
    \definition{clas.}{mililitro; unidade de volume, milésimo de um litro (ml)}
  \end{Phonetics}
\end{Entry}

\begin{Entry}{毫无}{11,4}{⽊、⽆}
  \begin{Phonetics}{毫无}{hao2wu2}[][HSK 7-9]
    \definition{adv.}{não; nada; de jeito nenhum}
  \end{Phonetics}
\end{Entry}

\begin{Entry}{毫米}{11,6}{⽊、⽶}
  \begin{Phonetics}{毫米}{hao2mi3}[][HSK 4]
    \definition{clas.}{milímetro; unidade legal de medida de comprimento, 1 mm equivale a 0,1 cm}
  \end{Phonetics}
\end{Entry}

\begin{Entry}{渠}{11}{⽊}
  \begin{Phonetics}{渠}{qu2}
    \definition*{s.}{Sobrenome Qu}
    \definition{adj.}{Literário: grande}
    \definition{pron.}{Dialeto: ele; ela}
    \definition[条]{s.}{canal; vala; fosso; trincheira | borda externa da roda | escudo}
  \end{Phonetics}
\end{Entry}

\begin{Entry}{渠道}{11,12}{⽊、⾡}
  \begin{Phonetics}{渠道}{qu2dao4}[][HSK 6]
    \definition[条,个,种]{s.}{vala de irrigação; os cursos de água escavados pelos trabalhadores para drenagem e irrigação | maneira; meio; caminho}
  \end{Phonetics}
\end{Entry}

\begin{Entry}{棉}{12}{⽊}
  \begin{Phonetics}{棉}{mian2}
    \definition{adj.}{almofadado com algodão; acolchoado}
    \definition[些,种,类]{s.}{termo genérico para algodão ou paina | algodão | material semelhante ao algodão | acolchoado ou estofado de algodão}
  \end{Phonetics}
\end{Entry}

\begin{Entry}{棍}{12}{⽊}
  \begin{Phonetics}{棍}{gun4}[][HSK 7-9]
    \definition[根]{s.}{vara; bastão; porrete | canalha; patife; ladino; bandido}
  \end{Phonetics}
\end{Entry}

\begin{Entry}{棍子}{12,3}{⽊、⼦}
  \begin{Phonetics}{棍子}{gun4zi5}[][HSK 7-9]
    \definition[根]{s.}{vara; bastão; um objeto longo e redondo feito de madeira, bambu ou metal}
  \end{Phonetics}
\end{Entry}

\begin{Entry}{棒}{12}{⽊}
  \begin{Phonetics}{棒}{bang4}[][HSK 5]
    \definition{adj.}{bom; forte; excelente}
    \definition[根]{s.}{porrete; bastão; cajado; clava}
  \end{Phonetics}
\end{Entry}

\begin{Entry}{棒冰}{12,6}{⽊、⼎}
  \begin{Phonetics}{棒冰}{bang4bing1}
    \definition{s.}{picolé}
  \end{Phonetics}
\end{Entry}

\begin{Entry}{棒球}{12,11}{⽊、⽟}
  \begin{Phonetics}{棒球}{bang4qiu2}[][HSK 7-9]
    \definition[个,只]{s.}{beisebol}
  \end{Phonetics}
\end{Entry}

\begin{Entry}{棒棒糖}{12,12,16}{⽊、⽊、⽶}
  \begin{Phonetics}{棒棒糖}{bang4bang4tang2}
    \definition[根]{s.}{pirulito}
  \end{Phonetics}
\end{Entry}

\begin{Entry}{棕}{12}{⽊}
  \begin{Phonetics}{棕}{zong1}
    \definition{adj.}{marrom}
    \definition[个]{s.}{palmeira | fibra de palmeira; fibra de coco}
  \end{Phonetics}
\end{Entry}

\begin{Entry}{棕褐色}{12,14,6}{⽊、⾐、⾊}
  \begin{Phonetics}{棕褐色}{zong1he4 se4}
    \definition{s.}{cor sépia | bronzeado}
  \end{Phonetics}
\end{Entry}

\begin{Entry}{棘}{12}{⽊}
  \begin{Phonetics}{棘}{ji2}
    \definition*{s.}{Sobrenome Ji}
    \definition{s.}{árvore de jujuba | arbustos espinhosos; silvas | espinho}
  \end{Phonetics}
\end{Entry}

\begin{Entry}{棘手}{12,4}{⽊、⼿}
  \begin{Phonetics}{棘手}{ji2shou3}[][HSK 7-9]
    \definition{adj.}{complicado; difícil; espinhoso; difícil de manusear}
  \end{Phonetics}
\end{Entry}

\begin{Entry}{森}{12}{⽊}
  \begin{Phonetics}{森}{sen1}
    \definition{adj.}{cheio de árvores | multitudinário; em multidões | escuro; sombrio}
  \end{Phonetics}
\end{Entry}

\begin{Entry}{森林}{12,8}{⽊、⽊}
  \begin{Phonetics}{森林}{sen1lin2}[][HSK 4]
    \definition[片,座,处]{s.}{floresta; bosque; normalmente, refere-se a uma grande área de árvores em crescimento; na silvicultura, refere-se a um grande número de árvores que crescem em uma área razoavelmente grande de terra, juntamente com os animais e outras plantas}
  \end{Phonetics}
\end{Entry}

\begin{Entry}{棵}{12}{⽊}
  \begin{Phonetics}{棵}{ke1}[][HSK 4]
    \definition{clas.}{usado para plantas, árvores}
  \end{Phonetics}
\end{Entry}

\begin{Entry}{棹}{12}{⽊}
  \begin{Phonetics}{棹}{zhuo1}
    \variantof{桌}
  \end{Phonetics}
\end{Entry}

\begin{Entry}{棺}{12}{⽊}
  \begin{Phonetics}{棺}{guan1}
    \definition[副]{s.}{caixão; esquife; ataúde}
  \end{Phonetics}
\end{Entry}

\begin{Entry}{棺材}{12,7}{⽊、⽊}
  \begin{Phonetics}{棺材}{guan1cai5}[][HSK 7-9]
    \definition[具,口]{s.}{caixão; esquife; ataúde; féretro; urna funerária usada para enterrar os mortos, geralmente feito de madeira}
  \end{Phonetics}
\end{Entry}

\begin{Entry}{椅}{12}{⽊}
  \begin{Phonetics}{椅}{yi3}
    \definition*{s.}{Sobrenome Yi}
    \definition{s.}{cadeira}
  \end{Phonetics}
\end{Entry}

\begin{Entry}{椅子}{12,3}{⽊、⼦}
  \begin{Phonetics}{椅子}{yi3zi5}[][HSK 2]
    \definition[把,套,排]{s.}{cadeira; assentos com encosto, feitos principalmente de madeira, bambu, rattan, etc.; móveis com pernas, mas sem encosto para as pessoas se sentarem}
  \end{Phonetics}
\end{Entry}

\begin{Entry}{植}{12}{⽊}
  \begin{Phonetics}{植}{zhi2}
    \definition*{s.}{Sobrenome Zhi}
    \definition{s.}{flora; planta; vegetação}
    \definition{v.}{plantar; crescer; cultivar | configurar; estabelecer}
  \end{Phonetics}
\end{Entry}

\begin{Entry}{植物}{12,8}{⽊、⽜}
  \begin{Phonetics}{植物}{zhi2wu4}[][HSK 4]
    \definition[种,株,棵,盆]{s.}{planta; vegetação; flora}
  \end{Phonetics}
\end{Entry}

\begin{Entry}{椰}{12}{⽊}
  \begin{Phonetics}{椰}{ye1}
    \definition[只,棵]{s.}{coqueiro; coco}
  \end{Phonetics}
\end{Entry}

\begin{Entry}{椰汁}{12,5}{⽊、⽔}
  \begin{Phonetics}{椰汁}{ye1zhi1}
    \definition{s.}{água de coco}
  \end{Phonetics}
\end{Entry}

\begin{Entry}{楔}{13}{⽊}
  \begin{Phonetics}{楔}{xie1}
    \definition[个]{s.}{cunha | pino; pregos de madeira; pregos de bambu}
    \definition{v.}{cunhar}
  \end{Phonetics}
\end{Entry}

\begin{Entry}{楔子}{13,3}{⽊、⼦}
  \begin{Phonetics}{楔子}{xie1zi5}
    \definition{s.}{cunha | pino | prólogo ou interlúdio no drama da Dinastia Yuan | prólogo em alguns romances modernos; introduções a óperas e romances | calço; chuteira; lascas de madeira inseridas nas juntas de encaixe e espiga, etc. | estaca de madeira; estaca de bambu; pregos de madeira; pregos de bambu}
  \end{Phonetics}
\end{Entry}

\begin{Entry}{楼}{13}{⽊}
  \begin{Phonetics}{楼}{lou2}[][HSK 1]
    \definition*{s.}{Sobrenome Lou}
    \definition{clas.}{andar, piso}
    \definition[层,座,栋]{s.}{um prédio com muitos andares | piso; andar | superestrutura; uma estrutura com um convés superior; um andar adicional construído sobre uma casa ou outro edifício | nome usado para certas lojas ou locais de entretenimento | arco ornamental; certas construções decorativas altas com passagens por baixo}
  \end{Phonetics}
\end{Entry}

\begin{Entry}{楼上}{13,3}{⽊、⼀}
  \begin{Phonetics}{楼上}{lou2 shang4}[][HSK 1]
    \definition{s.}{no andar de cima | autor anterior em um tópico do fórum; em plataformas como fóruns na internet, refere-se à pessoa que se manifesta antes de você.}
  \end{Phonetics}
\end{Entry}

\begin{Entry}{楼下}{13,3}{⽊、⼀}
  \begin{Phonetics}{楼下}{lou2 xia4}[][HSK 1]
    \definition{s.}{no andar de baixo}
  \end{Phonetics}
\end{Entry}

\begin{Entry}{楼房}{13,8}{⽊、⼾}
  \begin{Phonetics}{楼房}{lou2 fang2}[][HSK 6]
    \definition[栋,幢,座,套,层]{s.}{um edifício de dois ou mais andares}
  \end{Phonetics}
\end{Entry}

\begin{Entry}{楼梯}{13,11}{⽊、⽊}
  \begin{Phonetics}{楼梯}{lou2 ti1}[][HSK 4]
    \definition[个,层,段,阶]{s.}{escada; escadaria; degraus no meio de dois andares para permitir que as pessoas subam ou desçam as escadas}
  \end{Phonetics}
\end{Entry}

\begin{Entry}{楼道}{13,12}{⽊、⾡}
  \begin{Phonetics}{楼道}{lou2 dao4}[][HSK 6]
    \definition[个]{s.}{corredor; passagem | passagem (em edifício de vários andares)}
  \end{Phonetics}
\end{Entry}

\begin{Entry}{概}{13}{⽊}
  \begin{Phonetics}{概}{gai4}
    \definition{adj.}{geral; aproximado}
    \definition{adv.}{sem exceção; categoricamente}
    \definition{s.}{ideia principal; esboço geral | maneira de se portar e conduzir; comportamento}
    \definition{v.}{generalizar; exemplificar; tipificar}
  \end{Phonetics}
\end{Entry}

\begin{Entry}{概论}{13,6}{⽊、⾔}
  \begin{Phonetics}{概论}{gai4lun4}[][HSK 7-9]
    \definition{s.}{esboço; introdução; enquete; frequentemente usado em títulos de livros: um resumo da discussão}[«艺术历史概论»===«Introdução à História da Arte»]
  \end{Phonetics}
\end{Entry}

\begin{Entry}{概况}{13,7}{⽊、⼎}
  \begin{Phonetics}{概况}{gai4kuang4}[][HSK 7-9]
    \definition{s.}{situação geral; levantamento; breve relato (de algo); fatos básicos}[个人概况==Perfil Pessoal]
  \end{Phonetics}
\end{Entry}

\begin{Entry}{概念}{13,8}{⽊、⼼}
  \begin{Phonetics}{概念}{gai4nian4}[][HSK 3]
    \definition[个,种,项]{s.}{ideia; noção; conceito; concepção; uma forma de pensamento que resume as características comuns de algo em uma palavra}
  \end{Phonetics}
\end{Entry}

\begin{Entry}{概括}{13,9}{⽊、⼿}
  \begin{Phonetics}{概括}{gai4kuo4}[][HSK 4]
    \definition{adj.}{genérico; simples e claro, captando o conteúdo principal}
    \definition{s.}{generalização}
    \definition{v.}{generalizar; resumir}
  \end{Phonetics}
\end{Entry}

\begin{Entry}{概率}{13,11}{⽊、⽞}
  \begin{Phonetics}{概率}{gai4lv4}[][HSK 7-9]
    \definition{s.}{acaso; probabilidade; a probabilidade de um certo tipo de evento ocorrer nas mesmas condições}[成功的概率只有8\%。===A probabilidade de sucesso é de apenas 8\%.]
  \end{Phonetics}
\end{Entry}

\begin{Entry}{槐}{13}{⽊}
  \begin{Phonetics}{槐}{huai2}
    \definition*{s.}{Sobrenome Huai}
    \definition{s.}{sophora japonica; alfarrobeira; acácia}
  \end{Phonetics}
\end{Entry}

\begin{Entry}{槐树}{13,9}{⽊、⽊}
  \begin{Phonetics}{槐树}{huai2shu4}[][HSK 7-9]
    \definition[棵,株]{s.}{acácia; árvore de alfarroba; árvore de pagode}
  \end{Phonetics}
\end{Entry}

\begin{Entry}{㮸}{14}{⽊}
  \begin{Phonetics}{㮸}{song4}
    \variantof{送}
  \end{Phonetics}
\end{Entry}

\begin{Entry}{榜}{14}{⽊}
  \begin{Phonetics}{榜}{bang3}
    \definition[块]{s.}{lista publicada de nomes | Literário: placa horizontal inscrita | aviso; anúncio; proclamação antiga}
  \end{Phonetics}
\end{Entry}

\begin{Entry}{榜首}{14,9}{⽊、⾸}
  \begin{Phonetics}{榜首}{bang3shou3}[][HSK 7-9]
    \definition{s.}{cabeça da lista de candidatos aprovados; primeiro lugar em um concurso, etc. | topo da lista}
  \end{Phonetics}
\end{Entry}

\begin{Entry}{榜样}{14,10}{⽊、⽊}
  \begin{Phonetics}{榜样}{bang3yang4}[][HSK 7-9]
    \definition[个,位]{s.}{exemplo; modelo; padrão; pessoas ou coisas boas que valem a pena aprender, usado principalmente na linguagem falada}
  \end{Phonetics}
\end{Entry}

\begin{Entry}{槃}{14}{⽊}
  \begin{Phonetics}{槃}{pan2}
    \variantof{盘}
  \end{Phonetics}
\end{Entry}

\begin{Entry}{模}{14}{⽊}
  \begin{Phonetics}{模}{mo2}
    \definition{s.}{padrão | modelo; exemplo | modelo (pessoa) | exame simulado | módulo}
    \definition{v.}{imitar | copiar; emular}
  \end{Phonetics}
  \begin{Phonetics}{模}{mu2}
    \definition*{s.}{Sobrenome Mu}
    \definition{s.}{molde; padrão; matriz}
  \end{Phonetics}
\end{Entry}

\begin{Entry}{模仿}{14,6}{⽊、⼈}
  \begin{Phonetics}{模仿}{mo2fang3}[][HSK 5]
    \definition{v.}{copiar; imitar; aprender a fazer algo seguindo um modelo pronto}
  \end{Phonetics}
\end{Entry}

\begin{Entry}{模式}{14,6}{⽊、⼷}
  \begin{Phonetics}{模式}{mo2shi4}[][HSK 5]
    \definition{s.}{modelo; modo; padrão; a forma padrão de algo ou o modelo padrão que as pessoas podem seguir}
  \end{Phonetics}
\end{Entry}

\begin{Entry}{模具}{14,8}{⽊、⼋}
  \begin{Phonetics}{模具}{mu2ju4}
    \definition{s.}{molde | matriz | padrão}
  \end{Phonetics}
\end{Entry}

\begin{Entry}{模型}{14,9}{⽊、⼟}
  \begin{Phonetics}{模型}{mo2xing2}[][HSK 4]
    \definition[个]{s.}{modelo; padrão; itens feitos em escala com base em objetos ou desenhos | molde; padrão; molde para fundir máquinas, objetos, etc.}
  \end{Phonetics}
\end{Entry}

\begin{Entry}{模范}{14,9}{⽊、⾋}
  \begin{Phonetics}{模范}{mo2fan4}[][HSK 5]
    \definition{adj.}{exemplar}
    \definition{s.}{modelo; exemplo excelente; pessoa exemplar; coisa exemplar; pessoas ou coisas exemplares que servem de modelo}
  \end{Phonetics}
\end{Entry}

\begin{Entry}{模样}{14,10}{⽊、⽊}
  \begin{Phonetics}{模样}{mu2yang4}[][HSK 5]
    \definition[副,种]{s.}{aparência; a aparência ou o estilo de vestir de uma pessoa | indicando uma estimativa aproximada de tempo ou idade; expressão de estimativas relativas a tempo, idade, etc. | tendência; situação; inclinação}
  \end{Phonetics}
\end{Entry}

\begin{Entry}{模特儿}{14,10,2}{⽊、⽜、⼉}
  \begin{Phonetics}{模特儿}{mo2 te4r5}[][HSK 4]
    \definition[个,名,位]{s.}{modelo (pessoa que posa para um fotógrafo ou pintor ou escultor); objeto de representação ou referência usado por artistas para esboços e esculturas, como o corpo humano, objetos, modelos etc.; também se refere aos arquétipos que os estudiosos da literatura usam para retratar seus personagens | modelo (uma pessoa que usa roupas para exibir modas); pessoa ou manequim usado para exibir estilos de roupas}
  \end{Phonetics}
\end{Entry}

\begin{Entry}{模糊}{14,15}{⽊、⽶}
  \begin{Phonetics}{模糊}{mo2hu5}[][HSK 5]
    \definition{adj.}{vago; confuso; indistinto}
    \definition{v.}{confundir; desorientar}
  \end{Phonetics}
\end{Entry}

\begin{Entry}{槽}{15}{⽊}
  \begin{Phonetics}{槽}{cao2}[][HSK 7-9]
    \definition{clas.}{usado para portas | usado para porcos}
    \definition[个,道]{s.}{cocho | sulco; entalhe | canal | manjedoura (para água, ração animal, vinho, cuba); um recipiente para alimentar o gado, geralmente é retangular, alto em todos os lados e côncavo no meio, como uma caixa sem tampa | tanque de fermentação; cuba de vinho; geralmente se refere a certos utensílios com lados altos e côncavos no meio | leito do rio; fossa; refere-se a certos cursos d'água ou valas com lados altos e um meio côncavo | ranhura; fenda; uma depressão semelhante a um sulco em um objeto}
  \end{Phonetics}
\end{Entry}

\begin{Entry}{横}{15}{⽊}
  \begin{Phonetics}{横}{heng2}[][HSK 6]
    \definition{adj.}{horizontal; transversal; paralelo ao plano horizontal (oposto de 竖 e 直) | em ângulo reto com; direção esquerda-direita (em oposição à 竖, 直 ou 纵) | e leste a oeste ou de oeste a leste; direção leste-oeste (oposta a 纵) | desenfreado; turbulento | violento; feroz; irracional}
    \definition{adv.}{de qualquer forma; em qualquer caso | provavelmente; muito provavelmente}
    \definition{s.}{traço horizontal (em caracteres chineses)}
    \definition{v.}{deitar-se transversalmente; estar de lado | colocar algo transversalmente (ou horizontalmente)}
  \seealsoref{竖}{shu4}
  \seealsoref{直}{zhi2}
  \seealsoref{纵}{zong4}
  \end{Phonetics}
  \begin{Phonetics}{横}{heng4}[][HSK 7-9]
    \definition{adj.}{chocante e irracional; inesperado}
  \end{Phonetics}
\end{Entry}

\begin{Entry}{横七竖八}{15,2,9,2}{⽊、⼀、⽴、⼋}
  \begin{Phonetics}{横七竖八}{heng2qi1-shu4ba1}[][HSK 7-9]
    \definition{expr.}{em desordem; em seis e sete; desorganizado}
  \end{Phonetics}
\end{Entry}

\begin{Entry}{横向}{15,6}{⽊、⼝}
  \begin{Phonetics}{横向}{heng2xiang4}[][HSK 7-9]
    \definition{adj.}{horizontal; transversal (oposto a 竖向,纵向) | lateral | ortogonal | perpendicular}
  \seealsoref{竖向}{shu4xiang4}
  \seealsoref{纵向}{zong4xiang4}
  \end{Phonetics}
\end{Entry}

\begin{Entry}{横竖}{15,9}{⽊、⽴}
  \begin{Phonetics}{横竖}{heng2shu5}
    \definition{adv.}{de qualquer forma; em qualquer maneira; isso significa que não importa o que aconteça, o resultado ou a conclusão não mudará; equivale a 反正}
  \seealsoref{反正}{fan3zheng4}
  \end{Phonetics}
\end{Entry}

\begin{Entry}{樱}{15}{⽊}
  \begin{Phonetics}{樱}{ying1}
    \definition[个,棵,朵]{s.}{cereja | cerejeira oriental; flores de cerejeira}
  \end{Phonetics}
\end{Entry}

\begin{Entry}{樱桃}{15,10}{⽊、⽊}
  \begin{Phonetics}{樱桃}{ying1tao2}
    \definition{s.}{cereja}
  \end{Phonetics}
\end{Entry}

\begin{Entry}{橄}{15}{⽊}
  \begin{Phonetics}{橄}{gan3}
    \definition*{s.}{Sobrenome Gan}
  \end{Phonetics}
\end{Entry}

\begin{Entry}{橄榄球}{15,13,11}{⽊、⽊、⽟}
  \begin{Phonetics}{橄榄球}{gan3lan3qiu2}
    \definition{s.}{futebol jogado com bola oval (rúgbi, futebol americano, regras australianas, etc.)}
  \end{Phonetics}
\end{Entry}

\begin{Entry}{橘}{16}{⽊}
  \begin{Phonetics}{橘}{ju2}
    \definition[只,棵]{s.}{tangerina}
  \end{Phonetics}
\end{Entry}

\begin{Entry}{橘子汁}{16,3,5}{⽊、⼦、⽔}
  \begin{Phonetics}{橘子汁}{ju2zi5zhi1}
    \definition[瓶,杯,罐,盒]{s.}{suco de laranja}
  \seealsoref{橙汁}{cheng2zhi1}
  \seealsoref{柳橙汁}{liu3cheng2zhi1}
  \end{Phonetics}
\end{Entry}

\begin{Entry}{橙}{16}{⽊}
  \begin{Phonetics}{橙}{cheng2}
    \definition{s.}{laranja; fruta da laranjeira | laranjeira; pé de laranja | cor laranja}
  \end{Phonetics}
\end{Entry}

\begin{Entry}{橙汁}{16,5}{⽊、⽔}
  \begin{Phonetics}{橙汁}{cheng2zhi1}[][HSK 7-9]
    \definition[瓶,杯,罐,盒]{s.}{laranjada; suco de laranja}
  \seealsoref{橘子汁}{ju2zi5zhi1}
  \seealsoref{柳橙汁}{liu3cheng2zhi1}
  \end{Phonetics}
\end{Entry}

\begin{Entry}{橙色}{16,6}{⽊、⾊}
  \begin{Phonetics}{橙色}{cheng2 se4}
    \definition{s.}{cor de laranja}
  \end{Phonetics}
\end{Entry}

%%%%% EOF %%%%%

