%%%
%%% Radical "⾟"
%%%

\section*{Radical 160: ``⾟''}\addcontentsline{toc}{section}{Radical 160: ⾟}

\begin{entry}{辛苦}{7,8}{⾟、⾋}
  \begin{phonetics}{辛苦}{xin1ku3}
    \definition{adj.}{exaustivo | duro | árduo}
    \definition{s.}{dificuldades}
    \definition{v.}{trabalhar duro | ter muitos problemas}
  \end{phonetics}
\end{entry}

\begin{entry}{辞典}{13,8}{⾟、⼋}
  \begin{phonetics}{辞典}{ci2 dian3}[][HSK 5]
    \definition[本,部]{s.}{dicionário; coleção de termos especializados ou enciclopédicos, organizados em uma determinada ordem e explicados, para fins de referência}
    \variantof{词典}
  \end{phonetics}
\end{entry}

\begin{entry}{辞职}{13,11}{⾟、⽿}
  \begin{phonetics}{辞职}{ci2zhi2}[][HSK 5]
    \definition{v.+compl.}{renunciar; deixar o cargo; entregar a renúncia; pedir para ser dispensado de suas funções}
  \end{phonetics}
\end{entry}

\begin{entry}{辣}{14}{⾟}
  \begin{phonetics}{辣}{la4}[][HSK 4]
    \definition{adj.}{apimentado; picante; pungente; quente | cruel; implacável; venenoso; vicioso}
    \definition{v.}{queimar; picar; formigar; ter uma irritação picante (boca, nariz ou olhos)}
  \end{phonetics}
\end{entry}

\begin{entry}{辩论}{16,6}{⾟、⾔}
  \begin{phonetics}{辩论}{bian4lun4}[][HSK 4]
    \definition[场,次]{s.}{debate; argumento; a atividade comportamental em si de argumentar ou refutar diferentes pontos de vista ou afirmações, ou uma ocasião ou situação em que tal argumentação ou refutação é feita}
    \definition{v.}{debater; obter um entendimento unificado ou correto, ambos os lados usam linguagem, palavras etc. para explicar seus pontos de vista, apontar os erros ou as contradições do outro lado}
  \end{phonetics}
\end{entry}

\begin{entry}{辫子}{17,3}{⾟、⼦}
  \begin{phonetics}{辫子}{bian4zi5}
    \definition[根,条]{s.}{trança | um erro ou falha que pode ser explorado por um oponente | alça}
  \end{phonetics}
\end{entry}

%%%%% EOF %%%%%

