%%%
%%% Radical "⽳"
%%%

\section*{Radical 116: ``⽳''}\addcontentsline{toc}{section}{Radical 116: ⽳}

\begin{entry}{究竟}{7,11}{⽳、⾳}
  \begin{phonetics}{究竟}{jiu1jing4}[][HSK 4]
    \definition{adv.}{de fato; exatamente; usado em frases interrogativas para buscar | afinal de contas, no final; ênfase em fatos ou motivos}
    \definition{s.}{resultado; desfecho; a causa, o efeito ou a história completa do que aconteceu}
  \end{phonetics}
\end{entry}

\begin{entry}{穷}{7}{⽳}
  \begin{phonetics}{穷}{qiong2}[][HSK 4]
    \definition{adj.}{remoto; isolado; de difícil acesso | pobre; atingido pela pobreza | situação difícil, sem saída}
    \definition{adv.}{completamente | extremamente}
    \definition{v.}{exaurir; esgotar; consmir | ir até o fim; perseguir completamente perseguido; sondar profundamente | gastar}
  \end{phonetics}
\end{entry}

\begin{entry}{穷人}{7,2}{⽳、⼈}
  \begin{phonetics}{穷人}{qiong2 ren2}[][HSK 4]
    \definition{s.}{os pobres; pessoas pobres}
  \end{phonetics}
\end{entry}

\begin{entry}{空}{8}{⽳}
  \begin{phonetics}{空}{kong1}[][HSK 3]
    \definition*{s.}{sobrenome Kong}
    \definition{adj.}{vazio; oco; nulo}
    \definition{adv.}{por nada; em vão}
    \definition{s.}{céu; ar | vazio; vazio do mundo dos sentidos}
  \end{phonetics}
  \begin{phonetics}{空}{kong4}[][HSK 4]
    \definition*{s.}{sobrenome Quan}
    \definition{adj.}{vazio; oco; nulo; que não contém nada; que não tem nada ou nenhum conteúdo; impraticável}
    \definition{adv.}{para nada; em vão; sem efeito}
    \definition{s.}{céu; ar | vazio; ausência do mundo dos sentidos}
  \end{phonetics}
\end{entry}

\begin{entry}{空儿}{8,2}{⽳、⼉}
  \begin{phonetics}{空儿}{kong4r5}[][HSK 3]
    \definition{s.}{tempo livre | espaço (não utilizado)}
    \definition{v.}{ter tempo livre}
  \end{phonetics}
\end{entry}

\begin{entry}{空中小姐}{8,4,3,8}{⽳、⼁、⼩、⼥}
  \begin{phonetics}{空中小姐}{kong1zhong1xiao3jie3}
    \definition{s.}{aeromoça}
  \end{phonetics}
\end{entry}

\begin{entry}{空心菜}{8,4,11}{⽳、⼼、⾋}
  \begin{phonetics}{空心菜}{kong1xin1cai4}
    \definition{s.}{espinafre aquático | \emph{ong choy} | repolho do pântano | convolvulus aquático | glória-da-manhã aquática}
  \seealsoref{蕹菜}{weng4cai4}
  \end{phonetics}
\end{entry}

\begin{entry}{空气}{8,4}{⽳、⽓}
  \begin{phonetics}{空气}{kong1qi4}[][HSK 2]
    \definition{s.}{ar | atmosfera}
  \end{phonetics}
\end{entry}

\begin{entry}{空间}{8,7}{⽳、⾨}
  \begin{phonetics}{空间}{kong1jian1}[][HSK 4]
    \definition[个]{s.}{espaço; recinto; cômodo; espaço em branco; interespaço}
  \end{phonetics}
\end{entry}

\begin{entry}{空间站}{8,7,10}{⽳、⾨、⽴}
  \begin{phonetics}{空间站}{kong1jian1zhan4}
    \definition{s.}{estação espacial}
  \end{phonetics}
\end{entry}

\begin{entry}{空姐}{8,8}{⽳、⼥}
  \begin{phonetics}{空姐}{kong1jie3}
    \definition{s.}{aeromoça | comissária de bordo | abreviação de 空中小姐}
    \seeref{空中小姐}{kong1zhong1xiao3jie3}
  \end{phonetics}
\end{entry}

\begin{entry}{空调}{8,10}{⽳、⾔}
  \begin{phonetics}{空调}{kong1tiao2}[][HSK 3]
    \definition[台]{s.}{ar-condicionado;  condicionador de ar}
  \end{phonetics}
\end{entry}

\begin{entry}{穿}{9}{⽳}
  \begin{phonetics}{穿}{chuan1}[][HSK 1]
    \definition{v.}{vestir}
  \end{phonetics}
\end{entry}

\begin{entry}{穿上}{9,3}{⽳、⼀}
  \begin{phonetics}{穿上}{chuan1 shang4}[][HSK 4]
    \definition{v.}{vestir (roupas, etc.); colocar roupas}
  \end{phonetics}
\end{entry}

\begin{entry}{突出}{9,5}{⽳、⼐}
  \begin{phonetics}{突出}{tu1chu1}[][HSK 3]
    \definition{adj.}{proeminente; excelente}
    \definition{v.}{romper | enfatizar; destacar; dar destaque a | sobressair; projetar-se; destacar-se}
  \end{phonetics}
\end{entry}

\begin{entry}{突然}{9,12}{⽳、⽕}
  \begin{phonetics}{突然}{tu1ran2}[][HSK 3]
    \definition{adj.}{repentino; abrupto; inesperado}
    \definition{adv.}{de repente; abruptamente; inesperadamente}
  \end{phonetics}
\end{entry}

\begin{entry}{窗子}{12,3}{⽳、⼦}
  \begin{phonetics}{窗子}{chuang1 zi5}[][HSK 4]
    \definition{s.}{janela}
  \end{phonetics}
\end{entry}

\begin{entry}{窗户}{12,4}{⽳、⼾}
  \begin{phonetics}{窗户}{chuang1hu5}[][HSK 4]
    \definition[个,扇,面,排]{s.}{janela; dispositivo de ventilação e transmissão de luz nas paredes}
  \end{phonetics}
\end{entry}

\begin{entry}{窗台}{12,5}{⽳、⼝}
  \begin{phonetics}{窗台}{chuang1 tai2}[][HSK 4]
    \definition{s.}{parapeito da janela; peitoril; parte plana de uma janela que segura a moldura}
  \end{phonetics}
\end{entry}

\begin{entry}{窗帘}{12,8}{⽳、⼱}
  \begin{phonetics}{窗帘}{chuang1lian2}
    \definition{s.}{cortina}
  \end{phonetics}
\end{entry}

\begin{entry}{窾}{17}{⽳}
  \begin{phonetics}{窾}{cuan4}
    \definition{v.}{esconder}
  \end{phonetics}
  \begin{phonetics}{窾}{kuan3}
    \definition{adj.}{oco}
    \definition{s.}{rachadura | cavidade | (onomatopéia) água atingindo a rocha}
    \definition{v.}{escavar um buraco}
  \end{phonetics}
\end{entry}

%%%%% EOF %%%%%

