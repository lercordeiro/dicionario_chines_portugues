%%%
%%% Radical "⾳"
%%%

\section*{Radical 180: ``⾳''}\addcontentsline{toc}{section}{Radical 180: ⾳}

\begin{Entry}{音}{9}{⾳}[Kangxi 180]
  \begin{Phonetics}{音}{yin1}
    \definition[个,种]{s.}{som; som musical | notícias; novidades; informação | tom; refere-se especificamente a uma sílaba ou fonética | sílaba; refere-se a sílabas (um caractere chinês é uma sílaba)}
    \definition{v.}{vocalizar}
  \end{Phonetics}
\end{Entry}

\begin{Entry}{音乐}{9,5}{⾳、⼃}
  \begin{Phonetics}{音乐}{yin1yue4}[][HSK 2]
    \definition[种,段,张,曲]{s.}{música; ramo da arte que cria imagens artísticas, expressa pensamentos e sentimentos e reflete a vida real por meio da melodia e do ritmo da música; geralmente é dividido em duas categorias: música vocal e música instrumental}
  \end{Phonetics}
\end{Entry}

\begin{Entry}{音乐厅}{9,5,4}{⾳、⼃、⼚}
  \begin{Phonetics}{音乐厅}{yin1yue4ting1}
    \definition{s.}{auditório | teatro | \emph{concert hall}}
  \end{Phonetics}
\end{Entry}

\begin{Entry}{音乐节}{9,5,5}{⾳、⼃、⾋}
  \begin{Phonetics}{音乐节}{yin1yue4jie2}
    \definition{s.}{festival de música}
  \end{Phonetics}
\end{Entry}

\begin{Entry}{音乐会}{9,5,6}{⾳、⼃、⼈}
  \begin{Phonetics}{音乐会}{yin1 yue4 hui4}[][HSK 2]
    \definition[场]{s.}{concerto; atividades de execução de obras musicais}
  \end{Phonetics}
\end{Entry}

\begin{Entry}{音乐光碟}{9,5,6,14}{⾳、⼃、⼉、⽯}
  \begin{Phonetics}{音乐光碟}{yin1yue4guang1die2}
    \definition{s.}{CD de música}
  \end{Phonetics}
\end{Entry}

\begin{Entry}{音乐学}{9,5,8}{⾳、⼃、⼦}
  \begin{Phonetics}{音乐学}{yin1yue4xue2}
    \definition{s.}{musicologia}
  \end{Phonetics}
\end{Entry}

\begin{Entry}{音乐学院}{9,5,8,9}{⾳、⼃、⼦、⾩}
  \begin{Phonetics}{音乐学院}{yin1yue4xue2yuan4}
    \definition{s.}{conservatório | academia de música}
  \end{Phonetics}
\end{Entry}

\begin{Entry}{音乐院}{9,5,9}{⾳、⼃、⾩}
  \begin{Phonetics}{音乐院}{yin1yue4yuan4}
    \definition{s.}{conservatório | instituto de música}
  \end{Phonetics}
\end{Entry}

\begin{Entry}{音乐家}{9,5,10}{⾳、⼃、⼧}
  \begin{Phonetics}{音乐家}{yin1yue4jia1}
    \definition{s.}{músico}
  \end{Phonetics}
\end{Entry}

\begin{Entry}{音节}{9,5}{⾳、⾋}
  \begin{Phonetics}{音节}{yin1 jie2}[][HSK 2]
    \definition{s.}{sílaba}
  \end{Phonetics}
\end{Entry}

\begin{Entry}{音量}{9,12}{⾳、⾥}
  \begin{Phonetics}{音量}{yin1 liang4}[][HSK 6]
    \definition[把]{s.}{volume; volume do som; a força de um som}
  \end{Phonetics}
\end{Entry}

\begin{Entry}{音像}{9,13}{⾳、⼈}
  \begin{Phonetics}{音像}{yin1 xiang4}[][HSK 6]
    \definition{s.}{audiovisual; produtos audiovisuais; o nome coletivo para gravações de áudio e vídeo}
  \end{Phonetics}
\end{Entry}

\begin{Entry}{竟}{11}{⾳}
  \begin{Phonetics}{竟}{jing4}
    \definition{adj.}{todo; por toda parte; do começo ao fim}
    \definition{adv.}{no final; eventualmente | na verdade; inesperadamente; significa algo inesperado, equivalente a 居然}
    \definition{v.}{terminar; completar | investigar}
  \seealsoref{居然}{ju1ran2}
  \end{Phonetics}
\end{Entry}

\begin{Entry}{竟然}{11,12}{⾳、⽕}
  \begin{Phonetics}{竟然}{jing4ran2}[][HSK 4]
    \definition{adv.}{de fato; inesperadamente; para surpresa de alguém; chegar ao ponto de; indica que algo é um pouco inesperado}
  \end{Phonetics}
\end{Entry}

\begin{Entry}{章}{11}{⾳}
  \begin{Phonetics}{章}{zhang1}[][HSK 6]
    \definition*{s.}{Sobrenome Zhang}
    \definition[枚,个,方]{s.}{(para um livro, carta, etc.) capítulo; seção | ordem | regras; regulamentos; constituição | item; cláusula | (arcaico) memorial ao imperador; memorial ao trono | (arcaico) figura; padrão decorativo | selo; carimbo | distintivo; insígnia; medalha | verso; trecho do poema | escrita literária}
  \end{Phonetics}
\end{Entry}

\begin{Entry}{章鱼}{11,8}{⾳、⿂}
  \begin{Phonetics}{章鱼}{zhang1yu2}
    \definition{s.}{polvo | octópode}
  \end{Phonetics}
\end{Entry}

%%%%% EOF %%%%%

