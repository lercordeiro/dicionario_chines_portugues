%%%
%%% Radical "⾒"
%%%

\section*{Radical 147: ``⾒'' (见)}\addcontentsline{toc}{section}{Radical 147: ⾒、见}

\begin{entry}{见}{4}{⾒}
  \begin{phonetics}{见}{jian4}[][HSK 1]
    \definition*{s.}{sobrenome Jian}
    \definition{part.}{usado antes de um verbo para indicar voz passiva ou para expressar como isso me afeta}
    \definition{s.}{visão; ideia; opinião sobre algo; ponto de vista}
    \definition{v.}{ver; avistar | encontrar-se com; ser exposto a | parecer ser; mostrar evidência de | ver; referir-se a; indicar a fonte ou o local onde deve ser consultado | ver; encontrar; convocar}
  \end{phonetics}
  \begin{phonetics}{见}{xian4}
    \definition{v.}{aparecer | também escrito como 现}
  \seealsoref{现}{xian4}
  \end{phonetics}
\end{entry}

\begin{entry}{见过}{4,6}{⾒、⾡}
  \begin{phonetics}{见过}{jian4 guo4}[][HSK 2]
    \definition{s.}{visto (ver); já viu alguém ou algo; indica um momento no passado; alguém já viu ou encontrou um determinado objeto}
  \end{phonetics}
\end{entry}

\begin{entry}{见到}{4,8}{⾒、⼑}
  \begin{phonetics}{见到}{jian4 dao4}[][HSK 2]
    \definition{v.}{ver | encontrar; esbarrar; deparar-se com}
  \end{phonetics}
\end{entry}

\begin{entry}{见面}{4,9}{⾒、⾯}
  \begin{phonetics}{见面}{jian4mian4}[][HSK 1]
    \definition{v.+compl.}{encontrar-se com alguém;  ver um ao outro; ver alguém face-a-face}
  \end{phonetics}
\end{entry}

\begin{entry}{观}{6}{⾒}
  \begin{phonetics}{观}{guan1}
    \definition*{s.}{Templo taoísta; `Koon'}
    \definition{s.}{visão; vista | perspectiva; visão; conceito | aparência; perspectiva | alcance de visão | noção; ideia; conhecimento ou visão das coisas | ponto de vista; postura; uma visão de uma coisa}
    \definition{v.}{olhar para; assistir; observar | contemplar}
  \end{phonetics}
  \begin{phonetics}{观}{guan4}
    \definition*{s.}{sobrenome Guan}
    \definition{s.}{mosteiro taoísta | torre de vigia do portão do palácio | plataforma}
  \end{phonetics}
\end{entry}

\begin{entry}{观众}{6,6}{⾒、⼈}
  \begin{phonetics}{观众}{guan1zhong4}[][HSK 3]
    \definition[位,名,批,个]{s.}{espectador; público; audiência; pessoas que assistem a espetáculos ou competições}
  \end{phonetics}
\end{entry}

\begin{entry}{观念}{6,8}{⾒、⼼}
  \begin{phonetics}{观念}{guan1nian4}[][HSK 3]
    \definition[种,个]{s.}{ideia; conceito; consciência ideológica}
  \end{phonetics}
\end{entry}

\begin{entry}{观点}{6,9}{⾒、⽕}
  \begin{phonetics}{观点}{guan1dian3}[][HSK 2]
    \definition[个,种]{s.}{ponto de vista; perspectiva; a visão ou atitude que se tem sobre algo a partir de uma determinada posição ou perspectiva | ponto de vista; perspectiva; a posição ou perspectiva adotada ao analisar uma questão}
  \end{phonetics}
\end{entry}

\begin{entry}{观看}{6,9}{⾒、⽬}
  \begin{phonetics}{观看}{guan1 kan4}[][HSK 3]
    \definition{v.}{assistir; ver propositadamente; observar}
  \end{phonetics}
\end{entry}

\begin{entry}{观察}{6,14}{⾒、⼧}
  \begin{phonetics}{观察}{guan1cha2}[][HSK 3]
    \definition{v.}{assistir; pesquisar; observar; examinar cuidadosamente coisas ou fenômenos}
  \end{phonetics}
\end{entry}

\begin{entry}{现}{8}{⾒}
  \begin{phonetics}{现}{xian4}
    \definition{adj.}{presente | atual}
    \definition{v.}{aparecer}
  \seealsoref{见}{xian4}
  \end{phonetics}
\end{entry}

\begin{entry}{现代}{8,5}{⾒、⼈}
  \begin{phonetics}{现代}{xian4dai4}[][HSK 3]
    \definition*{s.}{Hyundai, empresa sul-coreana}
    \definition{adj.}{moderno; contemporâneo; com características, estilo e conceitos modernos, refletindo a vanguarda, a moda e a inovação da atualidade}
    \definition{s.}{tempos modernos; era contemporânea; atualmente, na divisão cronológica da história da China, refere-se principalmente ao período desde o Movimento 4 de Maio até os dias atuais}
  \end{phonetics}
\end{entry}

\begin{entry}{现在}{8,6}{⾒、⼟}
  \begin{phonetics}{现在}{xian4zai4}[][HSK 1]
    \definition{adv.}{agora; no momento; atualmente; neste momento, quando se fala, às vezes inclui um período de tempo mais ou menos longo antes ou depois da fala (diferente de 过去 ou 将来)}
  \seealsoref{过去}{guo4 qu4}
  \seealsoref{将来}{jiang1lai2}
  \end{phonetics}
\end{entry}

\begin{entry}{现场}{8,6}{⾒、⼟}
  \begin{phonetics}{现场}{xian4chang3}[][HSK 3]
    \definition[个,处]{s.}{local onde ocorreu o acidente, incidente ou desastre| local; ponto; local onde se realizam diretamente atividades como produção, apresentações e competições}
  \end{phonetics}
\end{entry}

\begin{entry}{现有}{8,6}{⾒、⽉}
  \begin{phonetics}{现有}{xian4 you3}[][HSK 5]
    \definition{adj.}{agora disponível; existente}
    \definition{v.}{estar disponível agora; existir | (literário) ter em mãos; ter em posse}
  \end{phonetics}
\end{entry}

\begin{entry}{现抓}{8,7}{⾒、⼿}
  \begin{phonetics}{现抓}{xian4zhua1}
    \definition{v.}{improvisar}
  \end{phonetics}
\end{entry}

\begin{entry}{现状}{8,7}{⾒、⽝}
  \begin{phonetics}{现状}{xian4zhuang4}[][HSK 5]
    \definition{s.}{situação atual; situação atual}
  \end{phonetics}
\end{entry}

\begin{entry}{现实}{8,8}{⾒、⼧}
  \begin{phonetics}{现实}{xian4shi2}[][HSK 3]
    \definition{adj.}{real; efetivo; verdadeiro; de acordo com circunstâncias objetivas}
    \definition[个]{s.}{realidade; factualidade; coisas que existem objetivamente}
  \end{phonetics}
\end{entry}

\begin{entry}{现货}{8,8}{⾒、⾙}
  \begin{phonetics}{现货}{xian4huo4}
    \definition{s.}{produtos à vista}
  \end{phonetics}
\end{entry}

\begin{entry}{现货的}{8,8,8}{⾒、⾙、⽩}
  \begin{phonetics}{现货的}{xian4huo4 de5}
    \definition{s.}{produtos em estoque}
  \end{phonetics}
\end{entry}

\begin{entry}{现金}{8,8}{⾒、⾦}
  \begin{phonetics}{现金}{xian4jin1}[][HSK 3]
    \definition[笔]{s.}{dinheiro; dinheiro vivo; moeda que pode ser usada diretamente | reserva de dinheiro em um banco; o dinheiro guardado no cofre do banco}
  \end{phonetics}
\end{entry}

\begin{entry}{现做}{8,11}{⾒、⼈}
  \begin{phonetics}{现做}{xian4zuo4}
    \definition{adj.}{fresco}
    \definition{v.}{fazer (comida) no local}
  \end{phonetics}
\end{entry}

\begin{entry}{现象}{8,11}{⾒、⾗}
  \begin{phonetics}{现象}{xian4xiang4}[][HSK 3]
    \definition[个,种]{s.}{aparência (das coisas); fenômeno; a forma externa e as relações manifestadas pelas coisas em seu desenvolvimento e mudança}
  \end{phonetics}
\end{entry}

\begin{entry}{规}{8}{⾒}
  \begin{phonetics}{规}{gui1}
    \definition*{s.}{sobrenome Gui}
    \definition[个,种]{s.}{bússola | regulamentação; regra | (mecânica) medidor | compasso; ferramenta para desenhar círculos}
    \definition{v.}{admoestar; aconselhar; advertir | planejar; fazer planos}
  \end{phonetics}
\end{entry}

\begin{entry}{规划}{8,6}{⾒、⼑}
  \begin{phonetics}{规划}{gui1hua4}[][HSK 5]
    \definition{s.}{plano; projeto; planejamento; programa; programação; esquematização; plano de desenvolvimento de longo prazo mais abrangente |}
    \definition{v.}{planejar; programar;}
  \end{phonetics}
\end{entry}

\begin{entry}{规则}{8,6}{⾒、⼑}
  \begin{phonetics}{规则}{gui1ze2}[][HSK 4]
    \definition{adj.}{ordenado; regular; descreve a forma, estrutura, arranjo, etc., que se conformam a uma determinada maneira organizada}
    \definition{s.}{regra; regulamento; sistema ou código de conduta prescrito para observância comum | lei; norma}
  \end{phonetics}
\end{entry}

\begin{entry}{规定}{8,8}{⾒、⼧}
  \begin{phonetics}{规定}{gui1ding4}[][HSK 3]
    \definition[个,条,项,款]{s.}{regra; regulamento; estipulação; tomar decisões sobre a forma, o método, a quantidade ou a qualidade de algo}
    \definition{v.}{estipular; prover; prescrever; estabelecer requisitos ou restrições em termos de métodos, qualidade, quantidade, tempo, etc.}
  \end{phonetics}
\end{entry}

\begin{entry}{规律}{8,9}{⾒、⼻}
  \begin{phonetics}{规律}{gui1lv4}[][HSK 4]
    \definition{adj.}{estável; regular; coisas, comportamentos, fenômenos, etc. que ocorrem em um determinado momento}
    \definition{s.}{lei; padrão regular; conexão essencial e recorrente entre as coisas}
  \end{phonetics}
\end{entry}

\begin{entry}{规范}{8,9}{⾒、⾋}
  \begin{phonetics}{规范}{gui1fan4}[][HSK 3]
    \definition{adj.}{regular; normal; padrão; que atende às especificações; em conformidade com as normas}
    \definition{s.}{norma; padrão; diretriz}
    \definition{v.}{regular; padronizar; tornar conforme as normas}
  \end{phonetics}
\end{entry}

\begin{entry}{规模}{8,14}{⾒、⽊}
  \begin{phonetics}{规模}{gui1mo2}[][HSK 4]
    \definition[个]{s.}{escala; escopo; dimensões; padrão, forma ou escopo (de um empreendimento, instituição, projeto, movimento, etc.)}
  \end{phonetics}
\end{entry}

\begin{entry}{视为}{8,4}{⾒、⼂}
  \begin{phonetics}{视为}{shi4 wei2}[][HSK 5]
    \definition{v.}{considerar; ver como; considerar como; considerar ser; achar que é}
  \end{phonetics}
\end{entry}

\begin{entry}{视角}{8,7}{⾒、⾓}
  \begin{phonetics}{视角}{shi4jiao3}
    \definition{s.}{ângulo do qual se observa um objeto | (figurativo) perspectiva, ponto de vista, quadro de referência | (cinematografia) ângulo da câmera | (percepção visual) ângulo visual (o ângulo que um objeto visto subtende no olho) | (fotografia) ângulo de visão}
  \end{phonetics}
\end{entry}

\begin{entry}{视频}{8,13}{⾒、⾴}
  \begin{phonetics}{视频}{shi4pin2}[][HSK 5]
    \definition[个,段]{s.}{vídeo; videoclipe}
  \end{phonetics}
\end{entry}

\begin{entry}{败}{8}{⾒}
  \begin{phonetics}{败}{bai4}[][HSK 4]
    \definition{adj.}{ruim; deteriorado; murcho; dilapidado; decadente}
    \definition{v.}{ser derrotado; perder (oposto a 胜) | derrotar; bater | falha (oposto a 成) | estragar; arruinar | decair; murchar | quebrar; neutralizar; dissipar}
  \seealsoref{成}{cheng2}
  \seealsoref{胜}{sheng4}
  \end{phonetics}
\end{entry}

\begin{entry}{觉得}{9,11}{⾒、⼻}
  \begin{phonetics}{觉得}{jue2de5}[][HSK 1]
    \definition{v.}{sentir; estar ciente; pressentir; causar uma sensação | pensar; sentir; encontrar; considerar (tom menos assertivo)}
  \end{phonetics}
\end{entry}

%%%%% EOF %%%%%

