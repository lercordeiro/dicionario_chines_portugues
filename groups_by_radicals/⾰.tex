%%%
%%% Radical "⾰"
%%%

\section*{Radical 177: ``⾰''}\addcontentsline{toc}{section}{Radical 177: ⾰}

\begin{Entry}{革}{9}{⾰}[Kangxi 177]
  \begin{Phonetics}{革}{ge2}
    \definition*{s.}{Sobrenome Ge}
    \definition{s.}{couro; pele; peles de animais depiladas e processadas}
    \definition{v.}{mudar; transformar; reformar | demitir; remover do cargo; expulsar}
  \end{Phonetics}
\end{Entry}

\begin{Entry}{革新}{9,13}{⾰、⽄}
  \begin{Phonetics}{革新}{ge2 xin1}[][HSK 6]
    \definition{v.}{inovar; renovar; livrar-se do velho e criar o novo}
  \end{Phonetics}
\end{Entry}

\begin{Entry}{靶}{13}{⾰}
  \begin{Phonetics}{靶}{ba3}
    \definition{s.}{alvo; um alvo para prática de tiro}
  \end{Phonetics}
\end{Entry}

\begin{Entry}{靶子}{13,3}{⾰、⼦}
  \begin{Phonetics}{靶子}{ba3zi5}[][HSK 7-9]
    \definition[个]{s.}{alvo; alvos para prática de tiro ou arco e flecha}
  \end{Phonetics}
\end{Entry}

\begin{Entry}{鞋}{15}{⾰}
  \begin{Phonetics}{鞋}{xie2}[][HSK 2]
    \definition[双,只]{s.}{sapatos; usado nos pés; algo que toca o chão ao caminhar; sem cano alto}
  \end{Phonetics}
\end{Entry}

\begin{Entry}{鞭}{18}{⾰}
  \begin{Phonetics}{鞭}{bian1}
    \definition[条]{s.}{chicote; oçoite; chibata | um bastão de ferro usado como arma na China antiga | algo parecido com um chicote | uma série de pequenos fogos de artifício | pênis de animal; refere-se ao pênis de certos mamíferos usado para fins medicinais ou comestíveis}
    \definition{v.}{açoitar; chicotear; flagelar}
  \end{Phonetics}
\end{Entry}

\begin{Entry}{鞭炮}{18,9}{⾰、⽕}
  \begin{Phonetics}{鞭炮}{bian1pao4}[][HSK 7-9]
    \definition[串,挂,盒,捆,箱,个]{s.}{\emph{maroon}, um tipo de foguete usado como alarme ou aviso; fogos de artifício; um termo geral para fogos de artifício grandes e pequenos}
  \end{Phonetics}
\end{Entry}

\begin{Entry}{鞭策}{18,12}{⾰、⽵}
  \begin{Phonetics}{鞭策}{bian1ce4}[][HSK 7-9]
    \definition{v.}{estimular; incitar; incentivar}
  \end{Phonetics}
\end{Entry}

%%%%% EOF %%%%%

