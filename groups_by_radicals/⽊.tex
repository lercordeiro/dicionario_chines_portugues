%%%
%%% Radical "⽊"
%%%

\section*{Radical 75: ``⽊''}\addcontentsline{toc}{section}{Radical 75: ⽊}

\begin{entry}{木头}{4,5}{⽊、⼤}
  \begin{phonetics}{木头}{mu4tou5}[][HSK 3]
    \definition{adj.}{estúpido; cabeça-dura}
    \definition[块,根]{s.}{tronco; madeira; viga; prancha}
  \end{phonetics}
\end{entry}

\begin{entry}{木偶}{4,11}{⽊、⼈}
  \begin{phonetics}{木偶}{mu4'ou3}
    \definition{s.}{fantoche, marionete}
  \end{phonetics}
\end{entry}

\begin{entry}{未}{5}{⽊}
  \begin{phonetics}{未}{wei4}
    \definition{adv.}{não ter | ainda não}
  \end{phonetics}
\end{entry}

\begin{entry}{未必}{5,5}{⽊、⼼}
  \begin{phonetics}{未必}{wei4bi4}[][HSK 4]
    \definition{adv.}{não tenho certeza; talvez não; não necessariamente}
  \end{phonetics}
\end{entry}

\begin{entry}{未来}{5,7}{⽊、⽊}
  \begin{phonetics}{未来}{wei4lai2}[][HSK 4]
    \definition{adj.}{próximo (refere-se ao tempo)}
    \definition[个]{s.}{futuro; o amanhã}
  \end{phonetics}
\end{entry}

\begin{entry}{末}{5}{⽊}
  \begin{phonetics}{末}{mo4}[][HSK 4]
    \definition{adj.}{último; final}
    \definition{s.}{ponta; terminal; extremidade; o final de algo | não essenciais; detalhes secundários | fim; final | pó; poeira | um papel na ópera tradicional}
  \end{phonetics}
\end{entry}

\begin{entry}{本}{5}{⽊}
  \begin{phonetics}{本}{ben3}[][HSK 1]
    \definition*{s.}{sobrenome Ben}
    \definition{adj.}{original; inerente | principal; central}
    \definition{adv.}{originalmente}
    \definition{clas.}{para livros, dicionários, periódicos, arquivos, etc. | para filmes com uma certa duração | para peças de teatro}
    \definition{prep.}{de acordo com; em consonância com; em conformidade com; equivalentes a 依照 e 按照}
    \definition{pron.}{nativo; próprio; refere-se ao próprio interlocutor ou ao grupo, instituição, empresa, local, etc. ao qual o interlocutor pertence | isto; atual; presente}
    \definition[个]{s.}{caule ou raiz de plantas | base; origem; fundamento; fundação;  alicerce | capital; capital social | livro; caderno; livreto | edição; versão | cópia; roteiro; manuscrito | memorial do trono; na era feudal, referia-se a um documento oficial}
    \definition{v.}{seguir; basear-se em; estar de acordo com}
  \seealsoref{按照}{an4zhao4}
  \seealsoref{依照}{yi1 zhao4}
  \end{phonetics}
\end{entry}

\begin{entry}{本人}{5,2}{⽊、⼈}
  \begin{phonetics}{本人}{ben3ren2}[][HSK 5]
    \definition{pron.}{eu (mim, mim mesmo); o orador refere-se a si mesmo | a si mesmo; em pessoa; refere-se à própria pessoa ou à pessoa mencionada anteriormente}
  \end{phonetics}
\end{entry}

\begin{entry}{本子}{5,3}{⽊、⼦}
  \begin{phonetics}{本子}{ben3 zi5}[][HSK 1]
    \definition[个,本]{s.}{livro; caderno | edição | impressão | licença; certificado de competência emitido por uma instituição especializada, obtido após aprovação no exame | \emph{script}; roteiro}
  \end{phonetics}
\end{entry}

\begin{entry}{本来}{5,7}{⽊、⽊}
  \begin{phonetics}{本来}{ben3lai2}[][HSK 3]
    \definition{adv.}{originalmente | apropriadamente | legalmente}
  \end{phonetics}
\end{entry}

\begin{entry}{本事}{5,8}{⽊、⼅}
  \begin{phonetics}{本事}{ben3shi4}
    \definition{s.}{habilidade | capacidade | \emph{status} | poder | posição | autoridade}
  \end{phonetics}
  \begin{phonetics}{本事}{ben3shi5}[][HSK 3]
    \definition{s.}{habilidade | capacidade |\emph{status} | poder | posição | autoridade}
  \end{phonetics}
\end{entry}

\begin{entry}{本金}{5,8}{⽊、⾦}
  \begin{phonetics}{本金}{ben3 jin1}
    \definition{s.}{capital; capital para a operação do comércio e da indústria; capital para a operação de negócios |
valor principal; dinheiro retirado ao depositar ou tomar emprestado (diferente de 利息)}
  \seealsoref{利息}{li4xi1}
  \end{phonetics}
\end{entry}

\begin{entry}{本科}{5,9}{⽊、⽲}
  \begin{phonetics}{本科}{ben3ke1}[][HSK 4]
    \definition{s.}{graduação; bacharelado; o curso básico de uma universidade ou faculdade}
  \end{phonetics}
\end{entry}

\begin{entry}{本领}{5,11}{⽊、⾴}
  \begin{phonetics}{本领}{ben3 ling3}[][HSK 3]
    \definition[项,个]{s.}{capacidade | faculdade | poder | habilidade | talento}
  \end{phonetics}
\end{entry}

\begin{entry}{术科}{5,9}{⽊、⽲}
  \begin{phonetics}{术科}{shu4ke1}
    \definition{s.}{cursos técnicos oferecidos em treinamento militar ou físico (oposto a 学科)}
  \seealsoref{学科}{xue2 ke1}
  \end{phonetics}
\end{entry}

\begin{entry}{朵}{6}{⽊}
  \begin{phonetics}{朵}{duo3}[][HSK 5]
    \definition*{s.}{sobrenome Duo}
    \definition{clas.}{para flores, nuvens ou coisas que se assemelham a flores e nuvens}
  \end{phonetics}
\end{entry}

\begin{entry}{机甲}{6,5}{⽊、⽥}
  \begin{phonetics}{机甲}{ji1jia3}
    \definition{s.}{\emph{mecha} (robôs operados pelo homem em mangá japonês)}
  \end{phonetics}
\end{entry}

\begin{entry}{机会}{6,6}{⽊、⼈}
  \begin{phonetics}{机会}{ji1hui4}[][HSK 2]
    \definition[个,次,种,些]{s.}{chance; oportunidade; momento favorável raro}
  \end{phonetics}
\end{entry}

\begin{entry}{机场}{6,6}{⽊、⼟}
  \begin{phonetics}{机场}{ji1chang3}[][HSK 1]
    \definition[个,家,处,座]{s.}{aeródromo; campo de aviação; aeroporto; campo de voo}
  \end{phonetics}
\end{entry}

\begin{entry}{机制}{6,8}{⽊、⼑}
  \begin{phonetics}{机制}{ji1 zhi4}[][HSK 5]
    \definition{s.}{mecanismo; processado por máquina; feito por máquina}
  \end{phonetics}
\end{entry}

\begin{entry}{机构}{6,8}{⽊、⽊}
  \begin{phonetics}{机构}{ji1gou4}[][HSK 4]
    \definition[所]{s.}{órgão; organização; instituição; instalações; aparelhamento; configuração | mecanismo; funcionamento interno de uma máquina ou unidade | estrutura interna de uma organização}
  \end{phonetics}
\end{entry}

\begin{entry}{机械}{6,11}{⽊、⽊}
  \begin{phonetics}{机械}{ji1xie4}
    \definition{s.}{máquina | maquinaria | mecânica}
  \end{phonetics}
\end{entry}

\begin{entry}{机票}{6,11}{⽊、⽰}
  \begin{phonetics}{机票}{ji1 piao4}[][HSK 1]
    \definition[张]{s.}{passagem aérea; passagem de avião}
  \seealsoref{飞机票}{fei1ji1 piao4}
  \end{phonetics}
\end{entry}

\begin{entry}{机遇}{6,12}{⽊、⾡}
  \begin{phonetics}{机遇}{ji1yu4}[][HSK 4]
    \definition[个]{s.}{chance; oportunidade; circunstâncias favoráveis}
  \end{phonetics}
\end{entry}

\begin{entry}{机器}{6,16}{⽊、⼝}
  \begin{phonetics}{机器}{ji1qi4}[][HSK 3]
    \definition[台,部,个]{s.}{máquina; maquinário; motor | aparelho; dispositivo}
  \end{phonetics}
\end{entry}

\begin{entry}{机器人}{6,16,2}{⽊、⼝、⼈}
  \begin{phonetics}{机器人}{ji1 qi4 ren2}[][HSK 5]
    \definition[个]{s.}{androide; golem | pessoa mecânica | robô}
  \end{phonetics}
\end{entry}

\begin{entry}{杀}{6}{⽊}
  \begin{phonetics}{杀}{sha1}[][HSK 5]
    \definition{adv.}{em extremo; excessivamente; usado após um verbo, indica grau intenso}
    \definition{v.}{matar; abater; esquartejar | lutar; entrar em batalha | enfraquecer; reduzir; diminuir | decolar; neutralizar}
  \end{phonetics}
\end{entry}

\begin{entry}{杀气}{6,4}{⽊、⽓}
  \begin{phonetics}{杀气}{sha1qi4}
    \definition{s.}{espírito assassino | aura de morte}
    \definition{v.}{desabafar a raiva de alguém}
  \end{phonetics}
\end{entry}

\begin{entry}{杀毒}{6,9}{⽊、⽏}
  \begin{phonetics}{杀毒}{sha1 du2}[][HSK 5]
    \definition{s.}{(computação) antivírus}
    \definition{v.}{esterilizar; desinfetar | (computação) eliminar um vírus}
  \end{phonetics}
\end{entry}

\begin{entry}{杂志}{6,7}{⽊、⼼}
  \begin{phonetics}{杂志}{za2zhi4}[][HSK 3]
    \definition[本,份,期]{s.}{diário; revista}
  \end{phonetics}
\end{entry}

\begin{entry}{杂志社}{6,7,7}{⽊、⼼、⽰}
  \begin{phonetics}{杂志社}{za2zhi4she4}
    \definition{s.}{editora de revista}
  \end{phonetics}
\end{entry}

\begin{entry}{杂技}{6,7}{⽊、⼿}
  \begin{phonetics}{杂技}{za2ji4}
    \definition[场]{s.}{acrobacia}
  \end{phonetics}
\end{entry}

\begin{entry}{权利}{6,7}{⽊、⼑}
  \begin{phonetics}{权利}{quan2li4}[][HSK 4]
    \definition[项,种,个,条,份]{s.}{direito; interesse; os poderes e benefícios (em oposição a 义务) exercidos por um cidadão ou pessoa jurídica de acordo com a lei}
  \seealsoref{义务}{yi4wu4}
  \end{phonetics}
\end{entry}

\begin{entry}{李}{7}{⽊}
  \begin{phonetics}{李}{li3}
    \definition*{s.}{sobrenome Li}
    \definition{s.}{ameixa}
  \end{phonetics}
\end{entry}

\begin{entry}{李子}{7,3}{⽊、⼦}
  \begin{phonetics}{李子}{li3zi5}
    \definition[个]{s.}{ameixa}
  \end{phonetics}
\end{entry}

\begin{entry}{李四}{7,5}{⽊、⼞}
  \begin{phonetics}{李四}{li3si4}
    \definition*{s.}{Li Si | Zé Ninguém | nome para uma pessoa não especificada, 2 de 3}
  \seealsoref{王五}{wang2wu3}
  \seealsoref{张三}{zhang1san1}
  \end{phonetics}
\end{entry}

\begin{entry}{材料}{7,10}{⽊、⽃}
  \begin{phonetics}{材料}{cai2liao4}[][HSK 4]
    \definition[份,个,种]{s.}{material; algo para fazer um produto acabado | material (figura de linguagem) | dados; material para estudo, pesquisa, etc.; conteúdo de uma obra}
  \end{phonetics}
\end{entry}

\begin{entry}{村}{7}{⽊}
  \begin{phonetics}{村}{cun1}[][HSK 3]
    \definition{adj.}{rústico; grosseiro}
    \definition{s.}{aldeia; vila}
  \end{phonetics}
\end{entry}

\begin{entry}{杜宇}{7,6}{⽊、⼧}
  \begin{phonetics}{杜宇}{du4yu3}
    \definition{s.}{cuco (pássaro)}
  \seealsoref{布谷鸟}{bu4gu3niao3}
  \seealsoref{杜鹃}{du4juan1}
  \seealsoref{杜鹃鸟}{du4juan1niao3}
  \end{phonetics}
\end{entry}

\begin{entry}{杜鹃}{7,12}{⽊、⿃}
  \begin{phonetics}{杜鹃}{du4juan1}
    \definition{s.}{cuco (pássaro)}
  \seealsoref{布谷鸟}{bu4gu3niao3}
  \seealsoref{杜鹃鸟}{du4juan1niao3}
  \seealsoref{杜宇}{du4yu3}
  \end{phonetics}
\end{entry}

\begin{entry}{杜鹃鸟}{7,12,5}{⽊、⿃、⿃}
  \begin{phonetics}{杜鹃鸟}{du4juan1niao3}
    \definition{s.}{cuco (pássaro)}
  \seealsoref{布谷鸟}{bu4gu3niao3}
  \seealsoref{杜鹃}{du4juan1}
  \seealsoref{杜宇}{du4yu3}
  \end{phonetics}
\end{entry}

\begin{entry}{束}{7}{⽊}
  \begin{phonetics}{束}{shu4}[][HSK 3]
    \definition*{s.}{sobrenome Shu}
    \definition{clas.}{para cachos, molhos, feixes, feixes de luz, etc.}
    \definition{s.}{monte; pacote; maço; feixe; cacho}
    \definition{v.}{atar; amarrar; vincular | controlar; restringir}
  \end{phonetics}
\end{entry}

\begin{entry}{束腰}{7,13}{⽊、⾁}
  \begin{phonetics}{束腰}{shu4yao1}
    \definition{s.}{cinto | cinta | cinturão}
  \end{phonetics}
\end{entry}

\begin{entry}{杠}{7}{⽊}
  \begin{phonetics}{杠}{gang1}
    \definition{s.}{mastro de bandeira | poste | passarela}
  \end{phonetics}
  \begin{phonetics}{杠}{gang4}
    \definition{s.}{vara grossa | barra | linha grossa | padrão, critério | hífen, traço}
    \definition{v.}{marcar com uma linha grossa | afiar (faca, navalha, etc.)}
  \end{phonetics}
\end{entry}

\begin{entry}{条}{7}{⽊}
  \begin{phonetics}{条}{tiao2}[][HSK 2]
    \definition{clas.}{para coisas longas e finas (fita, rio, estrada, calças, etc.)}
    \definition{s.}{artigo | cláusula (de lei ou tratado) | item | faixa}
  \end{phonetics}
\end{entry}

\begin{entry}{条目}{7,5}{⽊、⽬}
  \begin{phonetics}{条目}{tiao2mu4}
    \definition{s.}{cláusulas e subcláusulas (em documento formal) | verbete (em um dicionário, enciclopédia, etc.)}
  \end{phonetics}
\end{entry}

\begin{entry}{条件}{7,6}{⽊、⼈}
  \begin{phonetics}{条件}{tiao2jian4}[][HSK 2]
    \definition[个]{s.}{circunstâncias | condição | fator | pré-requisito | qualificação | requisito}
  \end{phonetics}
\end{entry}

\begin{entry}{条例}{7,8}{⽊、⼈}
  \begin{phonetics}{条例}{tiao2li4}
    \definition{s.}{código de conduta | ordenanças | regulamentos | regras | estatutos}
  \end{phonetics}
\end{entry}

\begin{entry}{条贯}{7,8}{⽊、⾙}
  \begin{phonetics}{条贯}{tiao2guan4}
    \definition{s.}{ordem | procedimentos | sequência | sistema}
  \end{phonetics}
\end{entry}

\begin{entry}{条幅}{7,12}{⽊、⼱}
  \begin{phonetics}{条幅}{tiao2fu2}
    \definition{s.}{faixa | banner | pergaminho de parede (para pintura ou caligrafia)}
  \end{phonetics}
\end{entry}

\begin{entry}{来}{7}{⽊}
  \begin{phonetics}{来}{lai2}[][HSK 1]
    \definition*{s.}{sobrenome Lai}
    \definition{part.}{usado após uma palavra numérica ou de quantidade; indica uma quantidade aproximada | usado depois de numerais como 一, 二, 三; para listar razões ou fatos, etc.}
    \definition{s.}{usado após uma expressão de tempo para indicar uma duração que vai do passado ao presente}
    \definition{v.}{vir; chegar; de outro lugar para o lugar onde o interlocutor se encontra | aparecer; acontecer; vir; (problemas, coisas, etc.) ocorrerem; surgirem | substitui um verbo com significado específico, indicando a realização de uma ação específica | estar indo para; usado antes de outro verbo, indica que algo será feito | vir para fazer algo; usado após outro verbo, indica que se vai fazer algo | usado para indicar um propósito; expressar o objetivo, fazer algo usando o método, a atitude ou a direção anteriores | usado com 得 ou 不 para indicar possibilidade, capacidade ou hábito}
  \seealsoref{不}{bu4}
  \seealsoref{得}{de5}
  \end{phonetics}
\end{entry}

\begin{entry}{来不及}{7,4,3}{⽊、⼀、⼃}
  \begin{phonetics}{来不及}{lai2bu5ji2}[][HSK 4]
    \definition{v.}{ser tarde demais; não ter tempo; não ter tempo suficiente (para fazer algo); não ser possível participar ou se atualizar devido a restrições de tempo}
  \end{phonetics}
\end{entry}

\begin{entry}{来自}{7,6}{⽊、⾃}
  \begin{phonetics}{来自}{lai2zi4}[][HSK 2]
    \definition{v.}{vir de (um local) | \emph{From:} (cabeçalho de \emph{e -mail})}
  \end{phonetics}
\end{entry}

\begin{entry}{来到}{7,8}{⽊、⼑}
  \begin{phonetics}{来到}{lai2 dao4}[][HSK 1]
    \definition{v.}{chegar; vir}
  \end{phonetics}
\end{entry}

\begin{entry}{来信}{7,9}{⽊、⼈}
  \begin{phonetics}{来信}{lai2 xin4}[][HSK 5]
    \definition{s.}{sua carta; carta recebida; carta ao interlocutor}
    \definition{v.}{enviar uma carta para aqui; enviar uma carta para o remetente}
  \end{phonetics}
\end{entry}

\begin{entry}{来得及}{7,11,3}{⽊、⼻、⼃}
  \begin{phonetics}{来得及}{lai2de5ji2}[][HSK 4]
    \definition{v.}{ainda ter tempo; ser capaz de fazê-lo; ser capaz de fazer algo a tempo; ainda ter tempo de chegar lá ou de se atualizar}
  \end{phonetics}
\end{entry}

\begin{entry}{来源}{7,13}{⽊、⽔}
  \begin{phonetics}{来源}{lai2yuan2}[][HSK 4]
    \definition{s.}{origem; causa; fonte; tabula rasa (ou seja, o lugar de onde as coisas vêm)}
    \definition{v.}{originar-se; surgir; ter origem; (algo) originar (seguido de 于)}
  \seealsoref{于}{yu2}
  \end{phonetics}
\end{entry}

\begin{entry}{极}{7}{⽊}
  \begin{phonetics}{极}{ji2}[][HSK 4]
    \definition*{s.}{sobrenome Ji}
    \definition{adj.}{máximo; extremo; final; supremo}
    \definition{adv.}{extremamente; excessivamente}
    \definition{s.}{o ponto máximo, mais alto; extremo; ápice; ponto culminante |
pólo; as extremidades norte e sul da Terra; as extremidades de um ímã; a extremidade de uma fonte de alimentação ou de um aparelho elétrico onde a corrente entra ou sai do aparelho}
    \definition{v.}{chegar ao fim de; levar a extremos}
  \end{phonetics}
\end{entry}

\begin{entry}{……极了}{7,2}{⽊、⼅}
  \begin{phonetics}{……极了}{ji2le5}[][HSK 3]
    \definition{expr.}{extremamente}
  \end{phonetics}
\end{entry}

\begin{entry}{极其}{7,8}{⽊、⼋}
  \begin{phonetics}{极其}{ji2qi2}[][HSK 4]
    \definition{adv.}{mais; extremamente; excessivamente}
  \end{phonetics}
\end{entry}

\begin{entry}{杯}{8}{⽊}
  \begin{phonetics}{杯}{bei1}[][HSK 1]
    \definition{clas.}{para certos recipientes de líquidos: copo, xícara, etc.}
    \definition[只,个]{s.}{copo; caneca; xícara | taça; troféu; prêmio em forma de taça}
  \end{phonetics}
\end{entry}

\begin{entry}{杯子}{8,3}{⽊、⼦}
  \begin{phonetics}{杯子}{bei1 zi5}[][HSK 1]
    \definition[个,只,种]{s.}{xícara; copo; recipiente para bebidas ou outros líquidos, geralmente cilíndrico ou com a parte inferior ligeiramente mais estreita, com capacidade geralmente pequena}
  \end{phonetics}
\end{entry}

\begin{entry}{杯具}{8,8}{⽊、⼋}
  \begin{phonetics}{杯具}{bei1ju4}
    \definition{s.}{parachoque | fiasco | (gíria) tragédia}
  \end{phonetics}
\end{entry}

\begin{entry}{松}{8}{⽊}
  \begin{phonetics}{松}{song1}[][HSK 4]
    \definition*{s.}{sobrenome Song}
    \definition{adj.}{solto; frouxo; folgado | abastado; rico; próspero | leve e crocante; macio}
    \definition[棵]{s.}{pinheiro | fio de carne seca; carne moída seca; alimentos macios ou quebradiços |}
    \definition{v.}{afrouxar; relaxar; soltar}
  \end{phonetics}
\end{entry}

\begin{entry}{松木}{8,4}{⽊、⽊}
  \begin{phonetics}{松木}{song1mu4}
    \definition{s.}{pinheiro}
  \end{phonetics}
\end{entry}

\begin{entry}{松树}{8,9}{⽊、⽊}
  \begin{phonetics}{松树}{song1 shu4}[][HSK 4]
    \definition[棵]{s.}{pinheiro; conífera comum, geralmente com folhas longas e pontiagudas e cones lenhosos}
  \end{phonetics}
\end{entry}

\begin{entry}{板}{8}{⽊}
  \begin{phonetics}{板}{ban3}[][HSK 3]
    \definition{adj.}{rígido; não natural | duro}
    \definition{clas.}{para cartões, papéis}
    \definition{s.}{tábua; placa; prato | veneziana; persiana; refere-se especificamente aos painéis de portas de lojas | badalos (instrumento musical que marca o ritmo) | uma batida acentuada (ritmo) na música e na ópera tradicional | chefe}
    \definition{v.}{parecer sério | livrar-se de maus hábitos ou falhas}
  \end{phonetics}
\end{entry}

\begin{entry}{构}{8}{⽊}
  \begin{phonetics}{构}{gou4}
    \definition{s.}{composição literária}
    \definition{v.}{construir | formar | compor}
    \variantof{够}
  \end{phonetics}
\end{entry}

\begin{entry}{构成}{8,6}{⽊、⼽}
  \begin{phonetics}{构成}{gou4cheng2}[][HSK 4]
    \definition{s.}{parte; componente; composição; estrutura}
    \definition{v.}{formar; compor; constituir; compor; encaixar muitas partes para formar um todo | consistir; causar; formar (principalmente em termos jurídicos)}
  \end{phonetics}
\end{entry}

\begin{entry}{构造}{8,10}{⽊、⾡}
  \begin{phonetics}{构造}{gou4 zao4}[][HSK 4]
    \definition[种]{s.}{estrutura; construção; disposição, organização e inter-relação dos componentes}
    \definition{v.}{formar; construir}
  \end{phonetics}
\end{entry}

\begin{entry}{枕}{8}{⽊}
  \begin{phonetics}{枕}{zhen3}
    \definition{s.}{travesseiro | almofada}
  \end{phonetics}
\end{entry}

\begin{entry}{果}{8}{⽊}
  \begin{phonetics}{果}{guo3}
    \definition*{s.}{sobrenome Guo}
    \definition{adj.}{resoluto; determinado; sem exitação}
    \definition{adv.}{realmente; como esperado; com certeza; isso significa que as coisas são consistentes com as expectativas, equivalente a 果然}
    \definition{conj.}{se realmente; se de fato;}
    \definition[个,些,种]{s.}{fruta; fruto da planta | resultado; consequência; o resultado final de um assunto (em oposição à 因)}
  \seealsoref{果然}{guo3ran2}
  \seealsoref{因}{yin1}
  \end{phonetics}
\end{entry}

\begin{entry}{果子}{8,3}{⽊、⼦}
  \begin{phonetics}{果子}{guo3zi5}
    \definition{s.}{fruta}
  \end{phonetics}
\end{entry}

\begin{entry}{果汁}{8,5}{⽊、⽔}
  \begin{phonetics}{果汁}{guo3zhi1}[][HSK 3]
    \definition[杯,瓶,种]{s.}{suco; suco de fruta}
  \end{phonetics}
\end{entry}

\begin{entry}{果实}{8,8}{⽊、⼧}
  \begin{phonetics}{果实}{guo3shi2}[][HSK 4]
    \definition[种]{s.}{fruta; o órgão que se desenvolve a partir do ovário ou com outras partes da flor após a fertilização da flor | ganhos; frutos;  uma metáfora para conquista ou recompensa por trabalho árduo}
  \end{phonetics}
\end{entry}

\begin{entry}{果然}{8,12}{⽊、⽕}
  \begin{phonetics}{果然}{guo3ran2}[][HSK 3]
    \definition{adv.}{realmente; como esperado; com certeza}
    \definition{conj.}{se realmente; se de fato}
  \end{phonetics}
\end{entry}

\begin{entry}{果酱}{8,13}{⽊、⾣}
  \begin{phonetics}{果酱}{guo3jiang4}
    \definition{s.}{geléia | compota ou doce (de frutas)}
  \end{phonetics}
\end{entry}

\begin{entry}{枪}{8}{⽊}
  \begin{phonetics}{枪}{qiang1}[][HSK 5]
    \definition*{s.}{sobrenome Qiang}
    \definition{s.}{lança | arma; rifle; arma de fogo | uma coisa em forma de arma | enxada; ferramenta para cavar a terra}
    \definition{v.}{escrever artigos ou responder perguntas para outras pessoas}
  \end{phonetics}
\end{entry}

\begin{entry}{枫叶}{8,5}{⽊、⼝}
  \begin{phonetics}{枫叶}{feng1ye4}
    \definition{s.}{folha de bordo (maple, tipo de árvore)}
  \end{phonetics}
\end{entry}

\begin{entry}{柜子}{8,3}{⽊、⼦}
  \begin{phonetics}{柜子}{gui4 zi5}[][HSK 5]
    \definition[个]{s.}{gabinete; armário; dispositivo para guardar roupas, documentos, livros, etc.}
  \end{phonetics}
\end{entry}

\begin{entry}{枯木}{9,4}{⽊、⽊}
  \begin{phonetics}{枯木}{ku1mu4}
    \definition{s.}{árvore morta | madeira morta}
  \end{phonetics}
\end{entry}

\begin{entry}{架}{9}{⽊}
  \begin{phonetics}{架}{jia4}[][HSK 3]
    \definition{clas.}{para coisas com pilares ou componentes mecânicos | quadrado (usado para montanhas)}
    \definition{s.}{quadro; prateleira; suporte | briga; discussão}
    \definition{v.}{colocar para cima; erigir | afastar; resistir | suportar; ajudar | sequestrar; levar alguém embora à força}
  \end{phonetics}
\end{entry}

\begin{entry}{架式}{9,6}{⽊、⼷}
  \begin{phonetics}{架式}{jia4shi5}
    \variantof{架势}
  \end{phonetics}
\end{entry}

\begin{entry}{架势}{9,8}{⽊、⼒}
  \begin{phonetics}{架势}{jia4shi5}
    \definition{s.}{postura | atitude | posição (sobre um assunto, etc.)}
  \end{phonetics}
\end{entry}

\begin{entry}{柏树}{9,9}{⽊、⽊}
  \begin{phonetics}{柏树}{bai3shu4}
    \definition{s.}{cipreste}
  \end{phonetics}
\end{entry}

\begin{entry}{某}{9}{⽊}
  \begin{phonetics}{某}{mou3}[][HSK 3]
    \definition{pron.}{um certo alguém ou coisa; algum | usado para substituir seu próprio nome}
  \end{phonetics}
\end{entry}

\begin{entry}{染}{9}{⽊}
  \begin{phonetics}{染}{ran3}[][HSK 5]
    \definition*{s.}{sobrenome Ran}
    \definition{s.}{soja fermentada e temperada em forma de pasta}
    \definition{v.}{tingir; pintar | pegar (uma doença); cair em (um mau hábito, etc.) | sujar; contaminar | pegar (contrair) (uma doença) | adquirir (um mau hábito, etc.); contaminar}
  \end{phonetics}
\end{entry}

\begin{entry}{柔软}{9,8}{⽊、⾞}
  \begin{phonetics}{柔软}{rou2ruan3}
    \definition{adj.}{macio | suave}
  \end{phonetics}
\end{entry}

\begin{entry}{柠檬}{9,17}{⽊、⽊}
  \begin{phonetics}{柠檬}{ning2meng2}
    \definition{s.}{limão}
  \end{phonetics}
\end{entry}

\begin{entry}{查}{9}{⽊}
  \begin{phonetics}{查}{cha2}[][HSK 2]
    \definition{v.}{examinar; verificar cuidadosamente | examinar; investigar; entender bem a situação | procurar; consultar; revisar (documentos bibliográficos)}
  \end{phonetics}
  \begin{phonetics}{查}{zha1}
    \definition*{s.}{sobrenome Zha}
    \definition{s.}{espinheiro-chinês}
  \end{phonetics}
\end{entry}

\begin{entry}{查询}{9,8}{⽊、⾔}
  \begin{phonetics}{查询}{cha2 xun2}[][HSK 5]
    \definition{v.}{indagar; inquirir; perguntar sobre}
  \end{phonetics}
\end{entry}

\begin{entry}{柬埔寨}{9,10,14}{⽊、⼟、⼧}
  \begin{phonetics}{柬埔寨}{jian3pu3zhai4}
    \definition*{s.}{Camboja}
  \end{phonetics}
\end{entry}

\begin{entry}{柳}{9}{⽊}
  \begin{phonetics}{柳}{liu3}
    \definition*{s.}{sobrenome Liu}
    \definition{s.}{salgueiro}
  \end{phonetics}
\end{entry}

\begin{entry}{柳橙汁}{9,16,5}{⽊、⽊、⽔}
  \begin{phonetics}{柳橙汁}{liu3cheng2zhi1}
    \definition[瓶,杯,罐,盒]{s.}{suco de laranja}
  \seealsoref{橙汁}{cheng2zhi1}
  \seealsoref{橘子汁}{ju2zi5zhi1}
  \end{phonetics}
\end{entry}

\begin{entry}{标志}{9,7}{⽊、⼼}
  \begin{phonetics}{标志}{biao1zhi4}[][HSK 4]
    \definition[个,种]{s.}{sinal; marca; logotipo; símbolo; emblema; marcações que caracterizam um objeto para facilitar a identificação}
    \definition{v.}{marcar; indicar; simbolizar; identificar}
  \end{phonetics}
\end{entry}

\begin{entry}{标准}{9,10}{⽊、⼎}
  \begin{phonetics}{标准}{biao1zhun3}[][HSK 3]
    \definition{adj.}{criterioso | padronizado | normatizado}
    \definition[个]{s.}{critério | padrão (oficial) | norma}
  \end{phonetics}
\end{entry}

\begin{entry}{标题}{9,15}{⽊、⾴}
  \begin{phonetics}{标题}{biao1ti2}[][HSK 3]
    \definition[个,条,篇]{s.}{título | manchete | cabeçalho}
  \end{phonetics}
\end{entry}

\begin{entry}{树}{9}{⽊}
  \begin{phonetics}{树}{shu4}[][HSK 1]
    \definition*{s.}{sobrenome Shu}
    \definition[棵,株]{s.}{árvore; nome comum das plantas lenhosas}
    \definition{v.}{plantar; cultivar | configurar; manter; estabelecer}
  \end{phonetics}
\end{entry}

\begin{entry}{树木}{9,4}{⽊、⽊}
  \begin{phonetics}{树木}{shu4mu4}
    \definition{s.}{árvore}
  \end{phonetics}
\end{entry}

\begin{entry}{树叶}{9,5}{⽊、⼝}
  \begin{phonetics}{树叶}{shu4ye4}[][HSK 4]
    \definition[片,枚,堆]{s.}{folha; folhagem;}
  \end{phonetics}
\end{entry}

\begin{entry}{树林}{9,8}{⽊、⽊}
  \begin{phonetics}{树林}{shu4 lin2}[][HSK 4]
    \definition{s.}{bosque; muitas árvores que crescem em fragmentos, menores que as florestas}
  \end{phonetics}
\end{entry}

\begin{entry}{树莓}{9,10}{⽊、⾋}
  \begin{phonetics}{树莓}{shu4mei2}
    \definition{s.}{framboesa}
  \end{phonetics}
\end{entry}

\begin{entry}{校}{10}{⽊}
  \begin{phonetics}{校}{jiao4}
    \definition{v.}{verificar | comparar | revisar}
  \end{phonetics}
  \begin{phonetics}{校}{xiao4}
    \definition[所]{s.}{oficial militar | escola}
  \end{phonetics}
\end{entry}

\begin{entry}{校长}{10,4}{⽊、⾧}
  \begin{phonetics}{校长}{xiao4zhang3}[][HSK 2]
    \definition[个,位,名]{s.}{diretor de escola | reitor (universidade)}
  \end{phonetics}
\end{entry}

\begin{entry}{校园}{10,7}{⽊、⼞}
  \begin{phonetics}{校园}{xiao4 yuan2}[][HSK 2]
    \definition{s.}{campus}
  \end{phonetics}
\end{entry}

\begin{entry}{校服}{10,8}{⽊、⽉}
  \begin{phonetics}{校服}{xiao4fu2}
    \definition{s.}{uniforme escolar}
  \end{phonetics}
\end{entry}

\begin{entry}{校规}{10,8}{⽊、⾒}
  \begin{phonetics}{校规}{xiao4gui1}
    \definition{s.}{regras e regulamentos escolares}
  \end{phonetics}
\end{entry}

\begin{entry}{校监}{10,10}{⽊、⽫}
  \begin{phonetics}{校监}{xiao4jian1}
    \definition{s.}{diretor | supervisor (de escola)}
  \end{phonetics}
\end{entry}

\begin{entry}{样}{10}{⽊}
  \begin{phonetics}{样}{yang4}
    \definition{s.}{aparência | forma | modelo}
  \end{phonetics}
\end{entry}

\begin{entry}{样儿}{10,2}{⽊、⼉}
  \begin{phonetics}{样儿}{yang4r5}
    \definition{s.}{aparência | forma | modelo}
  \seealsoref{样子}{yang4zi5}
  \end{phonetics}
\end{entry}

\begin{entry}{样子}{10,3}{⽊、⼦}
  \begin{phonetics}{样子}{yang4zi5}[][HSK 2]
    \definition{s.}{aparência | forma | modelo}
  \seealsoref{样儿}{yang4r5}
  \end{phonetics}
\end{entry}

\begin{entry}{样品}{10,9}{⽊、⼝}
  \begin{phonetics}{样品}{yang4pin3}
    \definition{s.}{amostra | espécime}
  \end{phonetics}
\end{entry}

\begin{entry}{样样}{10,10}{⽊、⽊}
  \begin{phonetics}{样样}{yang4yang4}
    \definition{adv.}{todos os tipos}
  \end{phonetics}
\end{entry}

\begin{entry}{样章}{10,11}{⽊、⾳}
  \begin{phonetics}{样章}{yang4zhang1}
    \definition{s.}{capítulo de amostra}
  \end{phonetics}
\end{entry}

\begin{entry}{核}{10}{⽊}
  \begin{phonetics}{核}{he2}
    \definition{adj.}{nuclear}
    \definition{s.}{poço | pedra | núcleo}
    \definition{v.}{examinar | checar | verificar}
  \end{phonetics}
\end{entry}

\begin{entry}{根}{10}{⽊}
  \begin{phonetics}{根}{gen1}[][HSK 4]
    \definition*{s.}{sobrenome Gen}
    \definition{adv.}{completamente; minuciosamente; radicalmente}
    \definition{clas.}{para objetos finos, alongados}
    \definition{s.}{raiz (de uma planta) | descendentes; posteridade; analogia com as gerações futuras | raiz (abreviação de raiz quadrada) | radical (química, refere-se a radicais carregados) | base; pé; raiz; parte inferior, base ou parte de um objeto que está presa a outra coisa | a parte de baixo das coisas; fonte; a origem  das coisas | base; fundamento}
  \end{phonetics}
\end{entry}

\begin{entry}{根本}{10,5}{⽊、⽊}
  \begin{phonetics}{根本}{gen1ben3}[][HSK 3]
    \definition{adj.}{básico; essencial; fundamental}
    \definition{adv.}{sempre; simplesmente; absolutamente; de qualquer modo | radically; thoroughly}
    \definition[个]{s.}{base; raiz; fundação}
  \end{phonetics}
\end{entry}

\begin{entry}{根据}{10,11}{⽊、⼿}
  \begin{phonetics}{根据}{gen1ju4}[][HSK 4]
    \definition[个]{prep.}{com base em; de acordo com; à luz de}
    \definition{s.}{base; fundamentos; razão; fundo; alicerce}
    \definition{v.}{basear; usar algo como premissa para uma conclusão ou como base para uma ação verbal}
  \end{phonetics}
\end{entry}

\begin{entry}{格兰菜}{10,5,11}{⽊、⼋、⾋}
  \begin{phonetics}{格兰菜}{ge2lan2cai4}
    \definition{s.}{brócolis chinês | couve chinesa | mostarda}
  \seealsoref{芥蓝}{gai4lan2}
  \end{phonetics}
\end{entry}

\begin{entry}{格外}{10,5}{⽊、⼣}
  \begin{phonetics}{格外}{ge2wai4}[][HSK 4]
    \definition{adv.}{especialmente; particularmente; ainda mais; indica mais do que a média | adicionalmente; indica adicional ou extra}
  \end{phonetics}
\end{entry}

\begin{entry}{栽}{10}{⽊}
  \begin{phonetics}{栽}{zai1}
    \definition{v.}{cultivar | plantar}
  \end{phonetics}
\end{entry}

\begin{entry}{栽种}{10,9}{⽊、⽲}
  \begin{phonetics}{栽种}{zai1zhong4}
    \definition{v.}{plantar}
  \end{phonetics}
\end{entry}

\begin{entry}{栽倒}{10,10}{⽊、⼈}
  \begin{phonetics}{栽倒}{zai1dao3}
    \definition{v.}{cair | sofrer uma queda}
  \end{phonetics}
\end{entry}

\begin{entry}{栽赃}{10,10}{⽊、⾙}
  \begin{phonetics}{栽赃}{zai1zang1}
    \definition{v.}{enquadrar alguém (plantar provas nele)}
  \end{phonetics}
\end{entry}

\begin{entry}{栽培}{10,11}{⽊、⼟}
  \begin{phonetics}{栽培}{zai1pei2}
    \definition{v.}{cultivar | educar | patrocinar | treinar}
  \end{phonetics}
\end{entry}

\begin{entry}{栽培种}{10,11,9}{⽊、⼟、⽲}
  \begin{phonetics}{栽培种}{zai1pei2 zhong3}
    \definition{s.}{espécies cultivadas}
  \end{phonetics}
\end{entry}

\begin{entry}{栽植}{10,12}{⽊、⽊}
  \begin{phonetics}{栽植}{zai1zhi2}
    \definition{v.}{plantar | transplantar}
  \end{phonetics}
\end{entry}

\begin{entry}{桃}{10}{⽊}
  \begin{phonetics}{桃}{tao2}[][HSK 5]
    \definition*{s.}{sobrenome Tao}
    \definition{s.}{pêssego | em forma de pêssego | pessegueiro}
  \end{phonetics}
\end{entry}

\begin{entry}{桃花}{10,7}{⽊、⾋}
  \begin{phonetics}{桃花}{tao2 hua1}[][HSK 5]
    \definition{s.}{(figurativo) caso amoroso | flor de pessegueiro}
  \end{phonetics}
\end{entry}

\begin{entry}{桃树}{10,9}{⽊、⽊}
  \begin{phonetics}{桃树}{tao2 shu4}[][HSK 5]
    \definition[株]{s.}{pêssego (árvore) | pessegueiro; pêssegos}
  \end{phonetics}
\end{entry}

\begin{entry}{桌}{10}{⽊}
  \begin{phonetics}{桌}{zhuo1}
    \definition{clas.}{para mesas de convidados em um banquete etc.}
    \definition{s.}{mesa}
  \end{phonetics}
\end{entry}

\begin{entry}{桌子}{10,3}{⽊、⼦}
  \begin{phonetics}{桌子}{zhuo1zi5}[][HSK 1]
    \definition[张,套]{s.}{mesa; escrivaninha; móveis, com uma superfície plana na parte superior e uma estrutura de suporte na parte inferior, para colocar objetos ou realizar atividades}
  \end{phonetics}
\end{entry}

\begin{entry}{桌布}{10,5}{⽊、⼱}
  \begin{phonetics}{桌布}{zhuo1bu4}
    \definition[条,块,张]{s.}{(computação) plano de fundo da área de trabalho | toalha de mesa | papel de parede}
  \end{phonetics}
\end{entry}

\begin{entry}{桌机}{10,6}{⽊、⽊}
  \begin{phonetics}{桌机}{zhuo1ji1}
    \definition{s.}{computador \emph{desktop}}
  \end{phonetics}
\end{entry}

\begin{entry}{桌灯}{10,6}{⽊、⽕}
  \begin{phonetics}{桌灯}{zhuo1deng1}
    \definition{s.}{luminária | lâmpada de mesa}
  \end{phonetics}
\end{entry}

\begin{entry}{桌面}{10,9}{⽊、⾯}
  \begin{phonetics}{桌面}{zhuo1mian4}
    \definition{s.}{área de trabalho | mesa}
  \end{phonetics}
\end{entry}

\begin{entry}{桌球}{10,11}{⽊、⽟}
  \begin{phonetics}{桌球}{zhuo1qiu2}
    \definition{s.}{bilhar | sinuca | mesa de ping-pong}
  \end{phonetics}
\end{entry}

\begin{entry}{桌游}{10,12}{⽊、⽔}
  \begin{phonetics}{桌游}{zhuo1you2}
    \definition{s.}{jogo de tabuleiro}
  \end{phonetics}
\end{entry}

\begin{entry}{桑}{10}{⽊}
  \begin{phonetics}{桑}{sang1}
    \definition*{s.}{sobrenome Sang}
    \definition{s.}{amoreira}
  \end{phonetics}
\end{entry}

\begin{entry}{桑巴舞}{10,4,14}{⽊、⼰、⾇}
  \begin{phonetics}{桑巴舞}{sang1ba1wu3}
    \definition{s.}{samba}
  \end{phonetics}
\end{entry}

\begin{entry}{桑树}{10,9}{⽊、⽊}
  \begin{phonetics}{桑树}{sang1shu4}
    \definition{s.}{amoreira, suas folhas são utilizadas para alimentar bichos-da-seda}
  \end{phonetics}
\end{entry}

\begin{entry}{桥}{10}{⽊}
  \begin{phonetics}{桥}{qiao2}[][HSK 3]
    \definition*{s.}{sobrenome Qiao}
    \definition[座]{s.}{ponte}
  \end{phonetics}
\end{entry}

\begin{entry}{桩}{10}{⽊}
  \begin{phonetics}{桩}{zhuang1}
    \definition{clas.}{para eventos, casos, transações, assuntos, etc.}
    \definition{s.}{toco | estaca | pilha}
  \end{phonetics}
\end{entry}

\begin{entry}{梨}{11}{⽊}
  \begin{phonetics}{梨}{li2}[][HSK 5]
    \definition*{s.}{sobrenome Li}
    \definition[个,颗]{s.}{perira; árvore de pera | pera}
  \end{phonetics}
\end{entry}

\begin{entry}{梯恩梯}{11,10,11}{⽊、⼼、⽊}
  \begin{phonetics}{梯恩梯}{ti1'en1ti1}
    \definition{s.}{(empréstimo linguístico) TNT, trinitrotolueno}
  \end{phonetics}
\end{entry}

\begin{entry}{检查}{11,9}{⽊、⽊}
  \begin{phonetics}{检查}{jian3cha2}[][HSK 2]
    \definition[份,个,次]{s.}{autocrítica; reconhecer e criticar os próprios erros verbais ou escritos}
    \definition{v.}{verificar; inspecionar; examinar; verificar cuidadosamente para descobrir o problema | criticar a si mesmo; identificar seus pontos fracos e erros, e criticar seu próprio comportamento}
  \end{phonetics}
\end{entry}

\begin{entry}{检测}{11,9}{⽊、⽔}
  \begin{phonetics}{检测}{jian3 ce4}[][HSK 4]
    \definition{v.}{testar; detectar; verificar}
  \end{phonetics}
\end{entry}

\begin{entry}{检验}{11,10}{⽊、⾺}
  \begin{phonetics}{检验}{jian3yan4}[][HSK 5]
    \definition{v.}{testar; examinar; inspecionar}
  \end{phonetics}
\end{entry}

\begin{entry}{毫不费力}{11,4,9,2}{⽊、⼀、⾙、⼒}
  \begin{phonetics}{毫不费力}{hao2bu2fei4li4}
    \definition{adj.}{sem esforço | não gastando o menor esforço}
  \end{phonetics}
\end{entry}

\begin{entry}{毫升}{11,4}{⽊、⼗}
  \begin{phonetics}{毫升}{hao2 sheng1}[][HSK 4]
    \definition{clas.}{mililitro; unidade de volume, milésimo de um litro (ml)}
  \end{phonetics}
\end{entry}

\begin{entry}{毫米}{11,6}{⽊、⽶}
  \begin{phonetics}{毫米}{hao2mi3}[][HSK 4]
    \definition{clas.}{milímetro; unidade legal de medida de comprimento, 1 mm equivale a 0,1 cm}
  \end{phonetics}
\end{entry}

\begin{entry}{棉}{12}{⽊}
  \begin{phonetics}{棉}{mian2}
    \definition{s.}{termo genérico para algodão ou paina | algodão | acolchoado ou estofado com algodão}
  \end{phonetics}
\end{entry}

\begin{entry}{棒}{12}{⽊}
  \begin{phonetics}{棒}{bang4}[][HSK 5]
    \definition{adj.}{bom; forte; excelente}
    \definition[根]{s.}{porrete; vara; bastão; cacete; haste}
  \end{phonetics}
\end{entry}

\begin{entry}{棒冰}{12,6}{⽊、⼎}
  \begin{phonetics}{棒冰}{bang4bing1}
    \definition{s.}{picolé}
  \end{phonetics}
\end{entry}

\begin{entry}{棒棒糖}{12,12,16}{⽊、⽊、⽶}
  \begin{phonetics}{棒棒糖}{bang4bang4tang2}
    \definition[根]{s.}{pirulito}
  \end{phonetics}
\end{entry}

\begin{entry}{棕褐色}{12,14,6}{⽊、⾐、⾊}
  \begin{phonetics}{棕褐色}{zong1he4 se4}
    \definition{s.}{cor sépia | bronzeado}
  \end{phonetics}
\end{entry}

\begin{entry}{森林}{12,8}{⽊、⽊}
  \begin{phonetics}{森林}{sen1lin2}[][HSK 4]
    \definition[片,座,处]{s.}{floresta; bosque; normalmente, refere-se a uma grande área de árvores em crescimento; na silvicultura, refere-se a um grande número de árvores que crescem em uma área razoavelmente grande de terra, juntamente com os animais e outras plantas}
  \end{phonetics}
\end{entry}

\begin{entry}{棵}{12}{⽊}
  \begin{phonetics}{棵}{ke1}[][HSK 4]
    \definition{clas.}{para plantas, árvores}
  \end{phonetics}
\end{entry}

\begin{entry}{棹}{12}{⽊}
  \begin{phonetics}{棹}{zhuo1}
    \variantof{桌}
  \end{phonetics}
\end{entry}

\begin{entry}{棺}{12}{⽊}
  \begin{phonetics}{棺}{guan1}
    \definition{s.}{caixão | esquife | ataúde}
  \end{phonetics}
\end{entry}

\begin{entry}{椅子}{12,3}{⽊、⼦}
  \begin{phonetics}{椅子}{yi3zi5}[][HSK 2]
    \definition[把,套]{s.}{cadeira}
  \end{phonetics}
\end{entry}

\begin{entry}{植物}{12,8}{⽊、⽜}
  \begin{phonetics}{植物}{zhi2wu4}[][HSK 4]
    \definition[种,株,盆,棵]{s.}{planta; vegetação; flora}
  \end{phonetics}
\end{entry}

\begin{entry}{椰汁}{12,5}{⽊、⽔}
  \begin{phonetics}{椰汁}{ye1zhi1}
    \definition{s.}{água de coco}
  \end{phonetics}
\end{entry}

\begin{entry}{楼}{13}{⽊}
  \begin{phonetics}{楼}{lou2}[][HSK 1]
    \definition*{s.}{sobrenome Lou}
    \definition{clas.}{andar, piso}
    \definition[层,座,栋]{s.}{um prédio com muitos andares | piso; andar | superestrutura; uma estrutura com um convés superior; um andar adicional construído sobre uma casa ou outro edifício | nome usado para certas lojas ou locais de entretenimento | arco ornamental; certas construções decorativas altas com passagens por baixo}
  \end{phonetics}
\end{entry}

\begin{entry}{楼上}{13,3}{⽊、⼀}
  \begin{phonetics}{楼上}{lou2 shang4}[][HSK 1]
    \definition{s.}{no andar de cima | autor anterior em um tópico do fórum; em plataformas como fóruns na internet, refere-se à pessoa que se manifesta antes de você.}
  \end{phonetics}
\end{entry}

\begin{entry}{楼下}{13,3}{⽊、⼀}
  \begin{phonetics}{楼下}{lou2 xia4}[][HSK 1]
    \definition{s.}{no andar de baixo}
  \end{phonetics}
\end{entry}

\begin{entry}{楼梯}{13,11}{⽊、⽊}
  \begin{phonetics}{楼梯}{lou2 ti1}[][HSK 4]
    \definition[个]{s.}{escada; escadaria; degraus no meio de dois andares para permitir que as pessoas subam ou desçam as escadas}
  \end{phonetics}
\end{entry}

\begin{entry}{概念}{13,8}{⽊、⼼}
  \begin{phonetics}{概念}{gai4nian4}[][HSK 3]
    \definition[个]{s.}{ideia; noção; conceito; concepção}
  \end{phonetics}
\end{entry}

\begin{entry}{概括}{13,9}{⽊、⼿}
  \begin{phonetics}{概括}{gai4kuo4}[][HSK 4]
    \definition{adj.}{genérico; simples e claro, captando o conteúdo principal}
    \definition{s.}{generalização}
    \definition{v.}{generalizar; resumir}
  \end{phonetics}
\end{entry}

\begin{entry}{㮸}{14}{⽊}
  \begin{phonetics}{㮸}{song4}
    \variantof{送}
  \end{phonetics}
\end{entry}

\begin{entry}{槃}{14}{⽊}
  \begin{phonetics}{槃}{pan2}
    \variantof{盘}
  \end{phonetics}
\end{entry}

\begin{entry}{模仿}{14,6}{⽊、⼈}
  \begin{phonetics}{模仿}{mo2fang3}[][HSK 5]
    \definition{v.}{copiar; imitar; aprender a fazer algo seguindo um modelo pronto}
  \end{phonetics}
\end{entry}

\begin{entry}{模式}{14,6}{⽊、⼷}
  \begin{phonetics}{模式}{mo2shi4}[][HSK 5]
    \definition{s.}{modelo; modo; padrão; a forma padrão de algo ou o modelo padrão que as pessoas podem seguir}
  \end{phonetics}
\end{entry}

\begin{entry}{模具}{14,8}{⽊、⼋}
  \begin{phonetics}{模具}{mu2ju4}
    \definition{s.}{molde | matriz | padrão}
  \end{phonetics}
\end{entry}

\begin{entry}{模型}{14,9}{⽊、⼟}
  \begin{phonetics}{模型}{mo2xing2}[][HSK 4]
    \definition[个]{s.}{modelo; padrão; itens feitos em escala com base em objetos ou desenhos | molde; padrão; molde para fundir máquinas, objetos, etc.}
  \end{phonetics}
\end{entry}

\begin{entry}{模范}{14,9}{⽊、⾋}
  \begin{phonetics}{模范}{mo2fan4}[][HSK 5]
    \definition{adj.}{exemplar}
    \definition{s.}{modelo; exemplo excelente; pessoa exemplar; coisa exemplar; pessoas ou coisas exemplares que servem de modelo}
  \end{phonetics}
\end{entry}

\begin{entry}{模样}{14,10}{⽊、⽊}
  \begin{phonetics}{模样}{mu2yang4}[][HSK 5]
    \definition[副,种]{s.}{aparência; a aparência ou o estilo de vestir de uma pessoa |
indicando uma estimativa aproximada de tempo ou idade; expressão de estimativas relativas a tempo, idade, etc. | tendência; situação; inclinação}
  \end{phonetics}
\end{entry}

\begin{entry}{模特儿}{14,10,2}{⽊、⽜、⼉}
  \begin{phonetics}{模特儿}{mo2 te4r5}[][HSK 4]
    \definition[个]{s.}{modelo (pessoa que posa para um fotógrafo ou pintor ou escultor); objeto de representação ou referência usado por artistas para esboços e esculturas, como o corpo humano, objetos, modelos etc.; também se refere aos arquétipos que os estudiosos da literatura usam para retratar seus personagens | modelo (uma pessoa que usa roupas para exibir modas); pessoa ou manequim usado para exibir estilos de roupas}
  \end{phonetics}
\end{entry}

\begin{entry}{模糊}{14,15}{⽊、⽶}
  \begin{phonetics}{模糊}{mo2hu5}[][HSK 5]
    \definition{adj.}{vago; confuso; indistinto}
    \definition{v.}{confundir; desorientar}
  \end{phonetics}
\end{entry}

\begin{entry}{槽}{15}{⽊}
  \begin{phonetics}{槽}{cao2}
    \definition{s.}{calha | canal | sulco | manjedoura}
  \end{phonetics}
\end{entry}

\begin{entry}{横竖}{15,9}{⽊、⽴}
  \begin{phonetics}{横竖}{heng2shu5}
    \definition{adv.}{de qualquer maneira | independentemente (linguagem falada)}
  \end{phonetics}
\end{entry}

\begin{entry}{樱桃}{15,10}{⽊、⽊}
  \begin{phonetics}{樱桃}{ying1tao2}
    \definition{s.}{cereja}
  \end{phonetics}
\end{entry}

\begin{entry}{橄榄球}{15,13,11}{⽊、⽊、⽟}
  \begin{phonetics}{橄榄球}{gan3lan3qiu2}
    \definition{s.}{futebol jogado com bola oval (rúgbi, futebol americano, regras australianas, etc.)}
  \end{phonetics}
\end{entry}

\begin{entry}{橘子汁}{16,3,5}{⽊、⼦、⽔}
  \begin{phonetics}{橘子汁}{ju2zi5zhi1}
    \definition[瓶,杯,罐,盒]{s.}{suco de laranja}
  \seealsoref{橙汁}{cheng2zhi1}
  \seealsoref{柳橙汁}{liu3cheng2zhi1}
  \end{phonetics}
\end{entry}

\begin{entry}{橙汁}{16,5}{⽊、⽔}
  \begin{phonetics}{橙汁}{cheng2zhi1}
    \definition[瓶,杯,罐,盒]{s.}{suco de laranja}
  \seealsoref{橘子汁}{ju2zi5zhi1}
  \seealsoref{柳橙汁}{liu3cheng2zhi1}
  \end{phonetics}
\end{entry}

\begin{entry}{橙色}{16,6}{⽊、⾊}
  \begin{phonetics}{橙色}{cheng2 se4}
    \definition{s.}{cor de laranja}
  \end{phonetics}
\end{entry}

%%%%% EOF %%%%%

