%%%
%%% Radical "⽇"
%%%

\section*{Radical 72: ``⽇''}\addcontentsline{toc}{section}{Radical 72: ⽇}

\begin{entry}{日}{4}{⽇}[Kangxi 72]
  \begin{phonetics}{日}{ri4}[][HSK 1]
    \definition*{s.}{Japão, abreviação de 日本}
    \definition{clas.}{usado para contar o número de dias}
    \definition{s.}{sol | dia (em oposição a 夜); período diurno | diariamente; todos os dias; a cada dia que passa | um dia específico; dia especial | tempo; refere-se a um período de tempo | dia; uma rotação da Terra}
  \seealsoref{日本}{ri4ben3}
  \seealsoref{夜}{ye4}
  \end{phonetics}
\end{entry}

\begin{entry}{日子}{4,3}{⽇、⼦}
  \begin{phonetics}{日子}{ri4zi5}[][HSK 2]
    \definition[个,段,些,番]{s.}{dia; data; referência a uma data específica | dias; tempo; referência ao número de dias e horas | vida; subsistência; refere-se à vida ou ao sustento}
  \end{phonetics}
\end{entry}

\begin{entry}{日历}{4,4}{⽇、⼚}
  \begin{phonetics}{日历}{ri4li4}[][HSK 4]
    \definition[张,本]{s.}{caledário; livro com o ano, mês, dia, semana, termo solar, aniversário, etc. registrados, um livro por ano, uma página por dia, aberto diariamente}
  \end{phonetics}
\end{entry}

\begin{entry}{日心说}{4,4,9}{⽇、⼼、⾔}
  \begin{phonetics}{日心说}{ri4 xin1 shuo1}
    \definition{s.}{teoria heliocêntrica | a teoria de que o sol está no centro do universo}
  \end{phonetics}
\end{entry}

\begin{entry}{日出}{4,5}{⽇、⼐}
  \begin{phonetics}{日出}{ri4chu1}
    \definition{s.}{nascer do sol}
  \seealsoref{夕阳}{xi1yang2}
  \end{phonetics}
\end{entry}

\begin{entry}{日本}{4,5}{⽇、⽊}
  \begin{phonetics}{日本}{ri4ben3}
    \definition*{s.}{Japão}
  \end{phonetics}
\end{entry}

\begin{entry}{日本人}{4,5,2}{⽇、⽊、⼈}
  \begin{phonetics}{日本人}{ri4ben3ren2}
    \definition{s.}{japonês | pessoa ou povo do Japão}
  \end{phonetics}
\end{entry}

\begin{entry}{日记}{4,5}{⽇、⾔}
  \begin{phonetics}{日记}{ri4ji4}[][HSK 4]
    \definition[本,篇,册]{s.}{diário; artigo que registra eventos e pensamentos diários}
  \end{phonetics}
\end{entry}

\begin{entry}{日光灯}{4,6,6}{⽇、⼉、⽕}
  \begin{phonetics}{日光灯}{ri4guang1deng1}
    \definition{s.}{lâmpada fluorescente}
  \end{phonetics}
\end{entry}

\begin{entry}{日报}{4,7}{⽇、⼿}
  \begin{phonetics}{日报}{ri4 bao4}[][HSK 2]
    \definition[份,种]{s.}{diário; jornais diários; jornal publicado todas as manhãs}
  \end{phonetics}
\end{entry}

\begin{entry}{日夜}{4,8}{⽇、⼣}
  \begin{phonetics}{日夜}{ri4 ye4}[][HSK 6]
    \definition{s.}{dia e noite; noite e dia; 24 horas por dia}
  \end{phonetics}
\end{entry}

\begin{entry}{日语}{4,9}{⽇、⾔}
  \begin{phonetics}{日语}{ri4 yu3}[][HSK 6]
    \definition{s.}{japonês; língua japonesa}
  \end{phonetics}
\end{entry}

\begin{entry}{日常}{4,11}{⽇、⼱}
  \begin{phonetics}{日常}{ri4chang2}[][HSK 3]
    \definition{adj.}{usual; diário; cotidiano; dia a dia; pertencem ao habitual}
  \end{phonetics}
\end{entry}

\begin{entry}{日期}{4,12}{⽇、⽉}
  \begin{phonetics}{日期}{ri4qi1}[][HSK 1]
    \definition[个,段]{s.}{data; a data ou período específico em que algo aconteceu}
  \end{phonetics}
\end{entry}

\begin{entry}{旧}{5}{⽇}
  \begin{phonetics}{旧}{jiu4}[][HSK 3]
    \definition{adj.}{passado; antigo; velho; ultrapassado (em oposição a 新)| usado; desgastado; velho; descolorido ou deformado devido ao uso prolongado ou ao tempo | antigo; único; que já existiu; anterior}
    \definition{s.}{velha amizade; velho amigo}
  \seealsoref{新}{xin1}
  \end{phonetics}
\end{entry}

\begin{entry}{早}{6}{⽇}
  \begin{phonetics}{早}{zao3}[][HSK 1]
    \definition{adj.}{precoce; antes do previsto ou planejado; antes do tempo; antes de um determinado momento |}
    \definition{adv.}{há muito tempo; desde cedo; por muito tempo; há muito tempo atrás}
    \definition{interj.}{bom dia; saudações, usadas para cumprimentar uns aos outros ao se encontrarem pela manhã}
    \definition[个]{s.}{manhã}
  \end{phonetics}
\end{entry}

\begin{entry}{早上}{6,3}{⽇、⼀}
  \begin{phonetics}{早上}{zao3shang5}[][HSK 1]
    \definition[个]{s.}{de manhã cedo; madrugada; o período antes e depois do nascer do sol; geralmente, desde o amanhecer até às 8h ou 9h da manhã; às vezes também se refere ao período entre o amanhecer e o meio-dia}
  \end{phonetics}
\end{entry}

\begin{entry}{早亡}{6,3}{⽇、⼇}
  \begin{phonetics}{早亡}{zao3wang2}
    \definition[个]{s.}{morte prematura}
    \definition{v.}{morrer prematuramente}
  \end{phonetics}
\end{entry}

\begin{entry}{早已}{6,3}{⽇、⼰}
  \begin{phonetics}{早已}{zao3 yi3}[][HSK 3]
    \definition{adv.}{há muito tempo; por muito tempo | (dialeto) no passado}
  \end{phonetics}
\end{entry}

\begin{entry}{早车}{6,4}{⽇、⾞}
  \begin{phonetics}{早车}{zao3che1}
    \definition{s.}{trem matutino | ônibus matutino}
  \end{phonetics}
\end{entry}

\begin{entry}{早安}{6,6}{⽇、⼧}
  \begin{phonetics}{早安}{zao3'an1}
    \definition{interj.}{Bom dia!}
  \end{phonetics}
\end{entry}

\begin{entry}{早早儿}{6,6,2}{⽇、⽇、⼉}
  \begin{phonetics}{早早儿}{zao3zao3r5}
    \definition{adv.}{o mais cedo possível | o mais breve possível}
  \end{phonetics}
\end{entry}

\begin{entry}{早饭}{6,7}{⽇、⾷}
  \begin{phonetics}{早饭}{zao3 fan4}[][HSK 1]
    \definition[份,顿]{s.}{o café da manhã}
  \end{phonetics}
\end{entry}

\begin{entry}{早知}{6,8}{⽇、⽮}
  \begin{phonetics}{早知}{zao3zhi1}
    \definition{v.}{prever | se alguém soubesse antes, \dots}
  \end{phonetics}
\end{entry}

\begin{entry}{早前}{6,9}{⽇、⼑}
  \begin{phonetics}{早前}{zao3qian2}
    \definition{adv.}{previamente}
  \end{phonetics}
\end{entry}

\begin{entry}{早晚}{6,11}{⽇、⽇}
  \begin{phonetics}{早晚}{zao3 wan3}
    \definition{adv./s.}{manhã e noite | mais cedo ou mais tarde; cedo ou tarde | algum tempo no futuro; algum dia; em algum momento no futuro}
  \end{phonetics}
\end{entry}

\begin{entry}{早晨}{6,11}{⽇、⽇}
  \begin{phonetics}{早晨}{zao3 chen2}[][HSK 2]
    \definition[个,段,番]{s.}{manhã cedo; manhãzinha; o período do amanhecer às oito ou nove horas; às vezes, o período da meia-noite ao meio-dia}
  \end{phonetics}
\end{entry}

\begin{entry}{早就}{6,12}{⽇、⼪}
  \begin{phonetics}{早就}{zao3 jiu4}[][HSK 2]
    \definition{adv.}{já; há muito tempo; há muito tempo atrás}
  \end{phonetics}
\end{entry}

\begin{entry}{早期}{6,12}{⽇、⽉}
  \begin{phonetics}{早期}{zao3 qi1}[][HSK 5]
    \definition{s.}{prófase; estágio inicial; fase inicial; a fase inicial de uma determinada época, processo ou vida de uma pessoa}
  \end{phonetics}
\end{entry}

\begin{entry}{早餐}{6,16}{⽇、⾷}
  \begin{phonetics}{早餐}{zao3 can1}[][HSK 2]
    \definition[份,桌,顿]{s.}{café da manhã; desejum}
  \end{phonetics}
\end{entry}

\begin{entry}{时}{7}{⽇}
  \begin{phonetics}{时}{shi2}[][HSK 3]
    \definition*{s.}{Sobrenome Shi}
    \definition{adj.}{atual; presente | temporário; oportuno}
    \definition{adv.}{de vez em quando; ocasionalmente; de ​​tempos em tempos; equivalente a 常常 ou 经常 | às vezes\dots às vezes\dots; dois caracteres 时 usados juntos são equivalentes a ``有时……有时……'' e ``一会儿……一会儿……''}
    \definition{clas.}{hora, cada uma das 24 partes iguais de um dia e uma noite; também usada como unidade legal de tempo}
    \definition{s.}{dias; tempos; longo período de tempo; refere-se a um período de tempo | tempo; tempo fixo; refere-se ao tempo especificado | hora; hora do dia | temporada | chance; oportunidade; momento oportuno | atual; presente | tempo verbal; uma categoria gramatical que utiliza certas formas gramaticais para indicar o momento em que uma ação ocorre; geralmente é dividida em presente, pretérito e futuro}
  \seealsoref{常常}{chang2 chang2}
  \seealsoref{经常}{jing1chang2}
  \seealsoref{一会儿……一会儿……}{yi1hui4r5 yi1hui4r5}
  \seealsoref{有时……有时……}{you3shi2 you3shi2}
  \end{phonetics}
\end{entry}

\begin{entry}{时代}{7,5}{⽇、⼈}
  \begin{phonetics}{时代}{shi2dai4}[][HSK 3]
    \definition[个]{s.}{idade; era; tempos; época; períodos e fases históricas divididas de acordo com condições econômicas, políticas, culturais e outras | um período na vida de alguém; uma fase na vida de uma pessoa}
  \end{phonetics}
\end{entry}

\begin{entry}{时节}{7,5}{⽇、⾋}
  \begin{phonetics}{时节}{shi2 jie2}[][HSK 6]
    \definition{s.}{temporada; um período de tempo em um ano com certas características, geralmente relacionadas à estação ou ao termo solar | época; tempo}
  \end{phonetics}
\end{entry}

\begin{entry}{时光}{7,6}{⽇、⼉}
  \begin{phonetics}{时光}{shi2guang1}[][HSK 5]
    \definition[台]{s.}{tempo; passagem do tempo | dias; horas; anos; épocas; períodos}
  \end{phonetics}
\end{entry}

\begin{entry}{时机}{7,6}{⽇、⽊}
  \begin{phonetics}{时机}{shi2ji1}[][HSK 5]
    \definition{s.}{oportunidade; momento oportuno}
  \end{phonetics}
\end{entry}

\begin{entry}{时而}{7,6}{⽇、⽽}
  \begin{phonetics}{时而}{shi2'er2}[][HSK 6]
    \definition{adv.}{às vezes; de tempos em tempos; indica que algo acontece repetidamente em intervalos irregulares}
  \end{phonetics}
\end{entry}

\begin{entry}{时而……,时而……}{7,6,7,6}{⽇、⽽、⽇、⽽}
  \begin{phonetics}{时而……,时而……}{shi2'er2 shi2'er2}[][HSK 6]
    \definition{adv.}{agora\dots, agora\dots; às vezes\dots, às vezes\dots; usado antes e depois; indica que diferentes fenômenos ou coisas ocorrem alternadamente ou mudam continuamente dentro de um determinado período de tempo}[时而下雨,时而晴天。___Às vezes chove, às vezes faz sol. | 这个地方时而热,时而冷。___Este lugar às vezes é quente e às vezes frio.]
  \end{phonetics}
\end{entry}

\begin{entry}{时时}{7,7}{⽇、⽇}
  \begin{phonetics}{时时}{shi2 shi2}[][HSK 6]
    \definition{adv.}{frequentemente; sempre; constantemente; indica que algo acontece várias vezes dentro de um determinado período de tempo}
  \end{phonetics}
\end{entry}

\begin{entry}{时间}{7,7}{⽇、⾨}
  \begin{phonetics}{时间}{shi2jian1}[][HSK 1]
    \definition[段]{s.}{tempo; refere-se à forma de existência do movimento da matéria, um sistema contínuo composto pelo passado, presente e futuro | tempo; período (duração); um período de tempo com início e fim | tempo (um ponto); em algum momento do tempo}
  \end{phonetics}
\end{entry}

\begin{entry}{时事}{7,8}{⽇、⼅}
  \begin{phonetics}{时事}{shi2shi4}[][HSK 5]
    \definition{s.}{acontecimentos atuais; assuntos atuais; eventos atuais | tendências atuais | como as coisas estão indo | a situação atual}
  \end{phonetics}
\end{entry}

\begin{entry}{时刻}{7,8}{⽇、⼑}
  \begin{phonetics}{时刻}{shi2ke4}[][HSK 3]
    \definition{adv.}{constantemente; sempre; a cada momento; frequentemente}
    \definition[个,段]{s.}{tempo; hora; momento; conjuntura; um ponto no tempo}
  \end{phonetics}
\end{entry}

\begin{entry}{时差}{7,9}{⽇、⼯}
  \begin{phonetics}{时差}{shi2cha1}
    \definition{s.}{diferença de tempo | \emph{jet lag}}
  \end{phonetics}
\end{entry}

\begin{entry}{时候}{7,10}{⽇、⼈}
  \begin{phonetics}{时候}{shi2hou5}[][HSK 1]
    \definition[个]{s.}{(um ponto no) tempo; momento; um determinado momento no tempo | (a duração do) tempo; um período de tempo com início e fim}
  \end{phonetics}
\end{entry}

\begin{entry}{时常}{7,11}{⽇、⼱}
  \begin{phonetics}{时常}{shi2chang2}[][HSK 5]
    \definition{adv.}{frequentemente; com frequência}
  \end{phonetics}
\end{entry}

\begin{entry}{时期}{7,12}{⽇、⽉}
  \begin{phonetics}{时期}{shi2qi1}[][HSK 6]
    \definition[个,段]{s.}{um período específico; um período de tempo com uma certa característica}
  \end{phonetics}
\end{entry}

\begin{entry}{时装}{7,12}{⽇、⾐}
  \begin{phonetics}{时装}{shi2 zhuang1}[][HSK 6]
    \definition{s.}{vestido da moda; a última moda; os últimos estilos de roupas | roupas contemporâneas (em oposição ao 古装)}
  \seealsoref{古装}{gu3 zhuang1}
  \end{phonetics}
\end{entry}

\begin{entry}{旷}{7}{⽇}
  \begin{phonetics}{旷}{kuang4}
    \definition*{s.}{Sobrenome Kuang}
    \definition{adj.}{vasto; espaçoso | livre de preocupações e ideias mesquinhas | folgado}
    \definition{v.}{negligenciar ou desperdiçar | estar ausente de | desperdiçar; abandonar; negligenciar}
  \end{phonetics}
\end{entry}

\begin{entry}{旷野}{7,11}{⽇、⾥}
  \begin{phonetics}{旷野}{kuang4ye3}
    \definition{s.}{região selvagem}
  \end{phonetics}
\end{entry}

\begin{entry}{昌}{8}{⽇}
  \begin{phonetics}{昌}{chang1}
    \definition*{s.}{Sobrenome Chang}
    \definition{adj.}{próspero; florescente | adequado; bom}
  \end{phonetics}
\end{entry}

\begin{entry}{昌盛}{8,11}{⽇、⽫}
  \begin{phonetics}{昌盛}{chang1 sheng4}[][HSK 6]
    \definition{adj.}{(país, nação, etc.) próspero; florescente}
  \end{phonetics}
\end{entry}

\begin{entry}{明}{8}{⽇}
  \begin{phonetics}{明}{ming2}
    \definition*{s.}{Dinastia Ming (1368-1644) | Sobrenome Ming}
    \definition{adj.}{claro; brilhante; brilhante | claro; distinto; de fácil entendimento | aberto; evidente; explícito; exposto | de ​​olhos aguçados; boa visão; visão nítida | honesto}
    \definition{adv.}{claramente; definitivamente; aparentemente; de fato}
    \definition{s.}{imediatamente a seguir no tempo; ao lado deste ano e hoje; visão}
    \definition{v.}{mostrar; revelar; tornar conhecido; deixar claro | entender; compreender}
  \end{phonetics}
\end{entry}

\begin{entry}{明天}{8,4}{⽇、⼤}
  \begin{phonetics}{明天}{ming2tian1}[][HSK 1]
    \definition{s.}{amanhã | futuro próximo}
  \end{phonetics}
\end{entry}

\begin{entry}{明日}{8,4}{⽇、⽇}
  \begin{phonetics}{明日}{ming2 ri4}[][HSK 6]
    \definition{s.}{amanhã}
  \seealsoref{明天}{ming2tian1}
  \end{phonetics}
\end{entry}

\begin{entry}{明白}{8,5}{⽇、⽩}
  \begin{phonetics}{明白}{ming2bai5}[][HSK 1]
    \definition{adj.}{claro; óbvio; evidente; inequívoco | sensato; razoável | aberto; franco; inequívoco; explícito}
    \definition{v.}{entender; compreender; saber}
  \end{phonetics}
\end{entry}

\begin{entry}{明年}{8,6}{⽇、⼲}
  \begin{phonetics}{明年}{ming2 nian2}[][HSK 1]
    \definition{s.}{próximo ano}
  \end{phonetics}
\end{entry}

\begin{entry}{明明}{8,8}{⽇、⽇}
  \begin{phonetics}{明明}{ming2ming2}[][HSK 5]
    \definition{adv.}{obviamente; claramente; sem dúvida; indica que o fenômeno ou princípio é evidente}
  \end{phonetics}
\end{entry}

\begin{entry}{明亮}{8,9}{⽇、⼇}
  \begin{phonetics}{明亮}{ming2 liang4}[][HSK 5]
    \definition{adj.}{claro; bem iluminado | brilhante; resplandecente | claro; simples; compreensível}
  \end{phonetics}
\end{entry}

\begin{entry}{明星}{8,9}{⽇、⽇}
  \begin{phonetics}{明星}{ming2xing1}[][HSK 2]
    \definition[个,位,颗,名]{s.}{estrela; ator, atleta, cantor famosos, etc. | talento de ponta; profissional de destaque; também é usado como metáfora para pessoas ou grupos que se destacam pelo seu bom desempenho ou excelência | estrela brilhante; estrela resplandecente; referindo-se a estrelas muito brilhantes}
  \end{phonetics}
\end{entry}

\begin{entry}{明显}{8,9}{⽇、⽇}
  \begin{phonetics}{明显}{ming2xian3}[][HSK 3]
    \definition{adj.}{claro; óbvio; distinto; claramente visível}
  \end{phonetics}
\end{entry}

\begin{entry}{明珠}{8,10}{⽇、⽟}
  \begin{phonetics}{明珠}{ming2zhu1}
    \definition{s.}{pérola | jóia (de grande valor)}
  \end{phonetics}
\end{entry}

\begin{entry}{明确}{8,12}{⽇、⽯}
  \begin{phonetics}{明确}{ming2que4}[][HSK 3]
    \definition{adj.}{claro; definido; específico}
    \definition{v.}{deixar claro; tornar definitivo; tornar um ponto de vista, uma tarefa, etc. claro, compreensível e definitivo}
  \end{phonetics}
\end{entry}

\begin{entry}{昏}{8}{⽇}
  \begin{phonetics}{昏}{hun1}
    \definition*{s.}{Sobrenome Hun}
    \definition{adj.}{escuro; fraco; embaçado | confuso; embaraçado; inconsciente}
    \definition{s.}{crepúsculo; tarde}
    \definition{v.}{perder a consciência; desmaiar}
  \end{phonetics}
\end{entry}

\begin{entry}{易}{8}{⽇}
  \begin{phonetics}{易}{yi4}
    \definition*{s.}{Sobrenome Yi}
    \definition{adj.}{fácil | amigável; pacífico}
    \definition{v.}{modificar; transformar | trocar | subestimar; desprezar}
  \end{phonetics}
\end{entry}

\begin{entry}{昔}{8}{⽇}
  \begin{phonetics}{昔}{xi1}
    \definition{s.}{tempos antigos; o passado; era uma vez}
  \end{phonetics}
\end{entry}

\begin{entry}{昔日}{8,4}{⽇、⽇}
  \begin{phonetics}{昔日}{xi1ri4}
    \definition{adj.}{passado}
  \end{phonetics}
\end{entry}

\begin{entry}{冒}{9}{⽇}
  \begin{phonetics}{冒}{mao4}[][HSK 5]
    \definition*{s.}{Sobrenome Mao}
    \definition{adv.}{com ousadia; precipitadamente | fingidamente; falsamente; fraudulentamente}
    \definition{v.}{emitir; liberar; enviar (para cima) | arriscar; ser corajoso}
  \end{phonetics}
\end{entry}

\begin{entry}{冒险}{9,9}{⽇、⾩}
  \begin{phonetics}{冒险}{mao4xian3}
    \definition{adj.}{corajoso}
    \definition{s.}{risco | aventura}
    \definition{v.+compl.}{correr risco | arriscar-se | aventurar-se em}
  \end{phonetics}
\end{entry}

\begin{entry}{星}{9}{⽇}
  \begin{phonetics}{星}{xing1}
    \definition*{s.}{Xing, a vigésima quinta das vinte e oito constelações em que a esfera celeste era dividida na antiga astronomia chinesa, consistindo em sete estrelas em Hydra}
    \definition[颗]{s.}{estrela | (astronomia) corpo celeste | partícula | pequenas marcas no braço de uma balança romana indicando jin e suas frações | artista famoso (estrela de cinema, estrela de jogos de bola, etc.) | satélite (artificial) | pequena quantidade}
  \end{phonetics}
\end{entry}

\begin{entry}{星火}{9,4}{⽇、⽕}
  \begin{phonetics}{星火}{xing1huo3}
    \definition{s.}{trilha de meteoro (usada principalmente em expressões como 急如星火) | faísca}
  \end{phonetics}
\end{entry}

\begin{entry}{星辰}{9,7}{⽇、⾠}
  \begin{phonetics}{星辰}{xing1chen2}
    \definition{s.}{estrelas}
  \end{phonetics}
\end{entry}

\begin{entry}{星表}{9,8}{⽇、⾐}
  \begin{phonetics}{星表}{xing1biao3}
    \definition{s.}{catálogo de estrelas}
  \end{phonetics}
\end{entry}

\begin{entry}{星星}{9,9}{⽇、⽇}
  \begin{phonetics}{星星}{xing1 xing5}[][HSK 2]
    \definition[颗,群,片]{s.}{estrela; em astronomia, refere-se aos corpos celestes luminosos no universo, como as estrelas que brilham no céu noturno | estrela; uma metáfora para alguém ou algo que se destaca em um determinado campo e atrai atenção | objetos em forma de estrela}
  \end{phonetics}
\end{entry}

\begin{entry}{星座}{9,10}{⽇、⼴}
  \begin{phonetics}{星座}{xing1zuo4}
    \definition[张]{s.}{signo astrológico | constelação}
  \end{phonetics}
\end{entry}

\begin{entry}{星期}{9,12}{⽇、⽉}
  \begin{phonetics}{星期}{xing1qi1}[][HSK 1]
    \definition[个]{s.}{semana | dias da semana; usado em conjunto com 日, 一, 二, 三, 四, 五, 六, 天, indica um determinado dia da semana | abreviação de domingo}
  \seealsoref{星期二}{xing1 qi1 er4}
  \seealsoref{星期六}{xing1 qi1 liu4}
  \seealsoref{星期日}{xing1 qi1 ri4}
  \seealsoref{星期三}{xing1 qi1 san1}
  \seealsoref{星期四}{xing1 qi1 si4}
  \seealsoref{星期天}{xing1 qi1 tian1}
  \seealsoref{星期五}{xing1 qi1 wu3}
  \seealsoref{星期一}{xing1 qi1 yi1}
  \end{phonetics}
\end{entry}

\begin{entry}{星期一}{9,12,1}{⽇、⽉、⼀}
  \begin{phonetics}{星期一}{xing1 qi1 yi1}[][HSK 1]
    \definition{s.}{segunda-feira}
  \end{phonetics}
\end{entry}

\begin{entry}{星期二}{9,12,2}{⽇、⽉、⼆}
  \begin{phonetics}{星期二}{xing1 qi1 er4}[][HSK 1]
    \definition{s.}{terça-feira}
  \end{phonetics}
\end{entry}

\begin{entry}{星期三}{9,12,3}{⽇、⽉、⼀}
  \begin{phonetics}{星期三}{xing1 qi1 san1}[][HSK 1]
    \definition{s.}{quarta-feira}
  \end{phonetics}
\end{entry}

\begin{entry}{星期五}{9,12,4}{⽇、⽉、⼆}
  \begin{phonetics}{星期五}{xing1 qi1 wu3}[][HSK 1]
    \definition{s.}{sexta-feira}
  \end{phonetics}
\end{entry}

\begin{entry}{星期六}{9,12,4}{⽇、⽉、⼋}
  \begin{phonetics}{星期六}{xing1 qi1 liu4}[][HSK 1]
    \definition{s.}{sábado}
  \end{phonetics}
\end{entry}

\begin{entry}{星期天}{9,12,4}{⽇、⽉、⼤}
  \begin{phonetics}{星期天}{xing1 qi1 tian1}[][HSK 1]
    \definition{s.}{domingo}
  \seealsoref{星期日}{xing1 qi1 ri4}
  \end{phonetics}
\end{entry}

\begin{entry}{星期日}{9,12,4}{⽇、⽉、⽇}
  \begin{phonetics}{星期日}{xing1 qi1 ri4}[][HSK 1]
    \definition{s.}{domingo}
  \seealsoref{星期天}{xing1 qi1 tian1}
  \end{phonetics}
\end{entry}

\begin{entry}{星期四}{9,12,5}{⽇、⽉、⼞}
  \begin{phonetics}{星期四}{xing1 qi1 si4}[][HSK 1]
    \definition{s.}{quinta-feira}
  \end{phonetics}
\end{entry}

\begin{entry}{春}{9}{⽇}
  \begin{phonetics}{春}{chun1}
    \definition*{s.}{Sobrenome Chun}
    \definition{s.}{primavera | amor; luxúria | vida; vitalidade}
  \end{phonetics}
\end{entry}

\begin{entry}{春天}{9,4}{⽇、⼤}
  \begin{phonetics}{春天}{chun1 tian1}
    \definition[个,段,季,番]{s.}{primavera; época da primavera | primavera; renascimento; uma atmosfera cheia de energia e esperança}
  \end{phonetics}
\end{entry}

\begin{entry}{春节}{9,5}{⽇、⾋}
  \begin{phonetics}{春节}{chun1 jie2}[][HSK 2]
    \definition*[个]{s.}{Festival da Primavera (Ano Novo Chinês); o primeiro dia do primeiro mês do calendário lunar, também se refere aos dias seguintes ao primeiro dia do primeiro mês}
  \end{phonetics}
\end{entry}

\begin{entry}{春季}{9,8}{⽇、⼦}
  \begin{phonetics}{春季}{chun1 ji4}[][HSK 4]
    \definition{s.}{primavera; primeiro trimestre do ano, que na China se refere ao período de três meses entre o início da primavera e o início do verão, e também se refere aos três meses do calendário lunar, a saber, o primeiro, o segundo e o terceiro meses}
  \end{phonetics}
\end{entry}

\begin{entry}{昨}{9}{⽇}
  \begin{phonetics}{昨}{zuo2}
    \definition{s.}{ontem}
  \end{phonetics}
\end{entry}

\begin{entry}{昨天}{9,4}{⽇、⼤}
  \begin{phonetics}{昨天}{zuo2tian1}[][HSK 1]
    \definition{s.}{ontem}
  \end{phonetics}
\end{entry}

\begin{entry}{昨日}{9,4}{⽇、⽇}
  \begin{phonetics}{昨日}{zuo2ri4}
    \definition{adv.}{ontem}
  \end{phonetics}
\end{entry}

\begin{entry}{昨夜}{9,8}{⽇、⼣}
  \begin{phonetics}{昨夜}{zuo2ye4}
    \definition{adv.}{noite passada}
  \end{phonetics}
\end{entry}

\begin{entry}{昨晚}{9,11}{⽇、⽇}
  \begin{phonetics}{昨晚}{zuo2wan3}
    \definition{adv.}{noite passada | ontem à noite}
  \end{phonetics}
\end{entry}

\begin{entry}{是}{9}{⽇}
  \begin{phonetics}{是}{shi4}[][HSK 1]
    \definition*{s.}{Sobrenome Shi}
    \definition{adj.}{correto; certo | verdadeiro}
    \definition{adv.}{(expressar afirmação firme) de fato; realmente}
    \definition{pron.}{isso; isto |  todos; qualquer um; usado antes de substantivos, tem o significado de 凡是}
    \definition{s.}{assuntos (importantes); grandes planos}
    \definition{v.}{usado como “ser” antes de substantivos ou pronomes para identificar, descrever ou ampliar o sujeito; indica que duas coisas são iguais, ou que a segunda explica a primeira | usado entre duas palavras idênticas; relacionar duas palavras semelhantes |  (usado antes de substantivos) ser exatamente; ser corretamente; usado antes de substantivos, tem o significado de 适合 | elogiar; justificar | expressar afirmação ou concordância (frequentemente usado sozinho) | usado para escolher perguntas, perguntas sim/não ou perguntas retóricas | (usado no início de uma frase) enfatizar uma determinada parte de uma frase | usado em perguntas sim-não}
  \seealsoref{凡是}{fan2shi4}
  \seealsoref{适合}{shi4he2}
  \end{phonetics}
\end{entry}

\begin{entry}{是不是}{9,4,9}{⽇、⼀、⽇}
  \begin{phonetics}{是不是}{shi4 bu2 shi4}[][HSK 1]
    \definition{expr.}{sim ou não; é ou não é; se ou não; questões levantadas sobre a confirmação e a negação dos fatos}
  \end{phonetics}
\end{entry}

\begin{entry}{是否}{9,7}{⽇、⼝}
  \begin{phonetics}{是否}{shi4fou3}[][HSK 4]
    \definition{adv.}{se; se ou não}
  \end{phonetics}
\end{entry}

\begin{entry}{是的}{9,8}{⽇、⽩}
  \begin{phonetics}{是的}{shi4de5}
    \definition{adv.}{sim | está certo}
  \end{phonetics}
\end{entry}

\begin{entry}{昼}{9}{⽇}
  \begin{phonetics}{昼}{zhou4}
    \definition*{s.}{Sobrenome Zhou}
    \definition{s.}{diurno; luz do dia; dia (oposição à 夜) | dia; o período do amanhecer ao anoitecer; diurno}
  \seealsoref{夜}{ye4}
  \end{phonetics}
\end{entry}

\begin{entry}{显}{9}{⽇}
  \begin{phonetics}{显}{xian3}[][HSK 5]
    \definition*{s.}{Sobrenome Xian}
    \definition{adj.}{aparente; óbvio; perceptível | ilustre e influente | evidente; óbvio}
    \definition{v.}{mostrar; exibir; manifestar | aparecer; mostrar; revelar}
  \end{phonetics}
\end{entry}

\begin{entry}{显示}{9,5}{⽇、⽰}
  \begin{phonetics}{显示}{xian3shi4}[][HSK 3]
    \definition{v.}{mostrar; manifestar-se claramente| exibir; ostentar}
  \end{phonetics}
\end{entry}

\begin{entry}{显得}{9,11}{⽇、⼻}
  \begin{phonetics}{显得}{xian3de5}[][HSK 3]
    \definition{v.}{parecer; aparecer; manifestar (alguma situação)}
  \end{phonetics}
\end{entry}

\begin{entry}{显著}{9,11}{⽇、⽬}
  \begin{phonetics}{显著}{xian3zhu4}[][HSK 4]
    \definition{adj.}{notável; significativo; notável; extraordinário; muito óbvio; muito claramente demonstrado; muito facilmente visto ou sentido}
  \end{phonetics}
\end{entry}

\begin{entry}{显然}{9,12}{⽇、⽕}
  \begin{phonetics}{显然}{xian3ran2}[][HSK 3]
    \definition{adj.}{claro; evidente; óbvio; fatos, verdades e outras coisas que são fáceis de descobrir, perceber ou sentir claramente}
  \end{phonetics}
\end{entry}

\begin{entry}{晒}{10}{⽇}
  \begin{phonetics}{晒}{shai4}[][HSK 4]
    \definition{v.}{(sol) brilhar sobre | aquecer-se; secar ao sol; colocar algo sob a luz do sol para secar | ignorar (alguém) | mostrar; divulgar o conteúdo de sua vida privada na Internet}
  \end{phonetics}
\end{entry}

\begin{entry}{晒干}{10,3}{⽇、⼲}
  \begin{phonetics}{晒干}{shai4gan1}
    \definition{v.}{secar ao sol}
  \end{phonetics}
\end{entry}

\begin{entry}{晕}{10}{⽇}
  \begin{phonetics}{晕}{yun1}[][HSK 6]
    \definition{adj.}{tonto; vertiginoso; confuso; sensação de que as coisas estão girando ao seu redor e, às vezes, sensação de que você vai cair}
    \definition{v.}{desmaiar; desfalecer}
  \end{phonetics}
  \begin{phonetics}{晕}{yun4}
    \definition{s.}{auréola; o círculo de luz formado pela refração da luz solar ou do luar através dos cristais de gelo nas nuvens | halo em torno de alguma cor ou luz; áreas desfocadas em torno de luz, sombra e cor}
    \definition{v.}{ficar tonto; desmaiar; desfalecer; sensação de tontura, como se os objetos ao seu redor estivessem girando e como se você estivesse prestes a cair}
  \end{phonetics}
\end{entry}

\begin{entry}{晚}{11}{⽇}
  \begin{phonetics}{晚}{wan3}[][HSK 1]
    \definition*{s.}{Sobrenome Wan}
    \definition{adj.}{tarde; tardio; passado o prazo acordado | júnior; mais jovem | mais tarde no tempo}
    \definition{s.}{noite; à noite; após o pôr do sol | últimos anos; última vida; um período posterior; refere-se especificamente à velhice de uma pessoa | pôr do sol; ao pôr do sol}
  \end{phonetics}
\end{entry}

\begin{entry}{晚上}{11,3}{⽇、⼀}
  \begin{phonetics}{晚上}{wan3shang5}[][HSK 1]
    \definition[个]{s.}{noite; o período entre o pôr do sol e a madrugada}
  \end{phonetics}
\end{entry}

\begin{entry}{晚会}{11,6}{⽇、⼈}
  \begin{phonetics}{晚会}{wan3hui4}[][HSK 2]
    \definition[场,个,次]{s.}{festa noturna; entretenimento noturno}
  \end{phonetics}
\end{entry}

\begin{entry}{晚安}{11,6}{⽇、⼧}
  \begin{phonetics}{晚安}{wan3'an1}[][HSK 2]
    \definition{expr.}{Tenha uma boa noite; uma frase educada usada para se despedir ou cumprimentar as pessoas à noite}
  \end{phonetics}
\end{entry}

\begin{entry}{晚报}{11,7}{⽇、⼿}
  \begin{phonetics}{晚报}{wan3 bao4}[][HSK 2]
    \definition[份,张]{s.}{jornal vespertino; um jornal publicado todas as tardes}
  \end{phonetics}
\end{entry}

\begin{entry}{晚近}{11,7}{⽇、⾡}
  \begin{phonetics}{晚近}{wan3jin4}
    \definition{adj.}{recente | mais recente no passado}
    \definition{adv.}{ultimamente | recentemente}
  \end{phonetics}
\end{entry}

\begin{entry}{晚饭}{11,7}{⽇、⾷}
  \begin{phonetics}{晚饭}{wan3 fan4}[][HSK 1]
    \definition[顿]{s.}{jantar}
  \end{phonetics}
\end{entry}

\begin{entry}{晚育}{11,8}{⽇、⾁}
  \begin{phonetics}{晚育}{wan3yu4}
    \definition{s.}{parto tardio}
    \definition{v.}{ter um filho mais tarde}
  \end{phonetics}
\end{entry}

\begin{entry}{晚点}{11,9}{⽇、⽕}
  \begin{phonetics}{晚点}{wan3 dian3}[][HSK 4]
    \definition{adj.}{atrasado}
    \definition{s.}{jantar leve}
    \definition{v.}{atrasar; retardar; adiar; (carro, navio, avião) partir, correr ou chegar mais tarde do que o horário especificado}
  \end{phonetics}
\end{entry}

\begin{entry}{晚景}{11,12}{⽇、⽇}
  \begin{phonetics}{晚景}{wan3jing3}
    \definition{s.}{circunstâncias dos anos de declínio de alguém | cena noturna}
  \end{phonetics}
\end{entry}

\begin{entry}{晚餐}{11,16}{⽇、⾷}
  \begin{phonetics}{晚餐}{wan3 can1}[][HSK 2]
    \definition[份,顿,次]{s.}{ceia; jantar}
  \end{phonetics}
\end{entry}

\begin{entry}{普}{12}{⽇}
  \begin{phonetics}{普}{pu3}
    \definition*{s.}{Sobrenome Pu}
    \definition{adj.}{geral; universal}
  \end{phonetics}
\end{entry}

\begin{entry}{普及}{12,3}{⽇、⼃}
  \begin{phonetics}{普及}{pu3ji2}[][HSK 3]
    \definition{adj.}{popular; universal; onipresente; amplamente compreendido, aceito ou utilizado}
    \definition[种]{v.}{popularizar; disseminar; espalhar entre as pessoas; promover amplamente o conhecimento, a educação, a tecnologia, etc. para popularizá-los}
  \end{phonetics}
\end{entry}

\begin{entry}{普通}{12,10}{⽇、⾡}
  \begin{phonetics}{普通}{pu3 tong1}[][HSK 2]
    \definition{adj.}{comum; normal; geral; médio; em geral, nada de especial, como a maioria das pessoas ou coisas}
  \end{phonetics}
\end{entry}

\begin{entry}{普通话}{12,10,8}{⽇、⾡、⾔}
  \begin{phonetics}{普通话}{pu3tong1hua4}[][HSK 2]
    \definition*{s.}{Mandarim (literalmente "linguagem comum") | Putonghua (fala comum da língua chinesa) | Língua oficial da China}
  \end{phonetics}
\end{entry}

\begin{entry}{普遍}{12,12}{⽇、⾡}
  \begin{phonetics}{普遍}{pu3bian4}[][HSK 3]
    \definition{adj.}{geral; comum; universal; difundido; a existência é muito ampla; tem semelhança}
  \end{phonetics}
\end{entry}

\begin{entry}{景}{12}{⽇}
  \begin{phonetics}{景}{jing3}[][HSK 6]
    \definition*{s.}{Sobrenome Jing}
    \definition{adj.}{grandioso; elevado; grande}
    \definition{s.}{vista; cenário; cena | situação; condição | cenário (de uma peça ou filme) | cena (de uma peça)}
    \definition{v.}{admirar; reverenciar; respeitar}
  \end{phonetics}
\end{entry}

\begin{entry}{景色}{12,6}{⽇、⾊}
  \begin{phonetics}{景色}{jing3se4}[][HSK 3]
    \definition[片,幅,道,处]{s.}{vista; cena; cenário; paisagem}
  \end{phonetics}
\end{entry}

\begin{entry}{景点}{12,9}{⽇、⽕}
  \begin{phonetics}{景点}{jing3 dian3}[][HSK 6]
    \definition[个,处]{s.}{local cênico; atração turística; um lugar onde se concentram as atrações turísticas, incluindo atrações naturais e culturais}
  \end{phonetics}
\end{entry}

\begin{entry}{景象}{12,11}{⽇、⾗}
  \begin{phonetics}{景象}{jing3 xiang4}[][HSK 5]
    \definition[个]{s.}{cena; visão; vista; quadro}
  \end{phonetics}
\end{entry}

\begin{entry}{晴}{12}{⽇}
  \begin{phonetics}{晴}{qing2}[][HSK 2]
    \definition{adj.}{ensolarado; bom; claro; não há nuvens no céu ou há poucas nuvens}
  \end{phonetics}
\end{entry}

\begin{entry}{晴天}{12,4}{⽇、⼤}
  \begin{phonetics}{晴天}{qing2 tian1}[][HSK 2]
    \definition[个]{s.}{dia ensolarado; tempo sem nuvens ou com poucas nuvens; em meteorologia, refere-se a um tempo em que a cobertura de nuvens no céu é inferior a 10\%}
  \end{phonetics}
\end{entry}

\begin{entry}{晴朗}{12,10}{⽇、⽉}
  \begin{phonetics}{晴朗}{qing2lang3}[][HSK 5]
    \definition{adj.}{bom; claro; ensolarado; céu limpo e sem nuvens}
  \end{phonetics}
\end{entry}

\begin{entry}{智}{12}{⽇}
  \begin{phonetics}{智}{zhi4}
    \definition*{s.}{Sobrenome Zhi}
    \definition{adj.}{engenhoso; sábio; inteligente; astuto}
    \definition{s.}{discernimento; engenhosidade; sagacidade | inteligência; conhecimento; sabedoria; percepção}
  \end{phonetics}
\end{entry}

\begin{entry}{智力}{12,2}{⽇、⼒}
  \begin{phonetics}{智力}{zhi4li4}[][HSK 4]
    \definition{s.}{inteligência; refere-se à capacidade de uma pessoa de conhecer e entender coisas objetivas e aplicar o conhecimento e a experiência para resolver problemas, incluindo memória, observação, imaginação, pensamento e julgamento}
  \end{phonetics}
\end{entry}

\begin{entry}{智能}{12,10}{⽇、⾁}
  \begin{phonetics}{智能}{zhi4neng2}[][HSK 4]
    \definition{adj.}{inteligente (telefone, sistema, etc.); descreve máquinas, equipamentos, tecnologia, etc. que foram processados com alta tecnologia e têm a capacidade de falar, pensar, calcular, resolver problemas, etc., como um ser humano}
    \definition{s.}{intelecto; a capacidade de aprender, agir, pensar, inventar, criar, resolver problemas, etc.}
  \end{phonetics}
\end{entry}

\begin{entry}{智商}{12,11}{⽇、⼝}
  \begin{phonetics}{智商}{zhi4shang1}
    \definition{s.}{quociente de inteligência, QI}
  \end{phonetics}
\end{entry}

\begin{entry}{智障}{12,13}{⽇、⾩}
  \begin{phonetics}{智障}{zhi4zhang4}
    \definition{adj./s.}{retardado}
  \end{phonetics}
\end{entry}

\begin{entry}{智慧}{12,15}{⽇、⼼}
  \begin{phonetics}{智慧}{zhi4hui4}
    \definition{s.}{sabedoria | inteligência}
  \end{phonetics}
\end{entry}

\begin{entry}{暂}{12}{⽇}
  \begin{phonetics}{暂}{zan4}
    \definition{adj.}{de curta duração (oposto a 久) | curto; momentâneo; pouco tempo}
    \definition{adv.}{temporariamente; por enquanto}
  \seealsoref{久}{jiu3}
  \end{phonetics}
\end{entry}

\begin{entry}{暂时}{12,7}{⽇、⽇}
  \begin{phonetics}{暂时}{zan4shi2}[][HSK 5]
    \definition{adj.}{transitório; temporário}
    \definition{adv.}{por enquanto; em pouco tempo}
  \end{phonetics}
\end{entry}

\begin{entry}{暂停}{12,11}{⽇、⼈}
  \begin{phonetics}{暂停}{zan4 ting2}[][HSK 5]
    \definition{s.}{suspensão temporária; refere-se especificamente à suspensão temporária de certas competições desportivas de acordo com as regras}
    \definition{v.}{pausar; suspender; esgotar o tempo}
  \end{phonetics}
\end{entry}

\begin{entry}{暑}{12}{⽇}
  \begin{phonetics}{暑}{shu3}
    \definition{adj.}{calor; clima quente; quente (em oposição a 寒)}
    \definition{s.}{verão}
  \seealsoref{寒}{han2}
  \end{phonetics}
\end{entry}

\begin{entry}{暑假}{12,11}{⽇、⼈}
  \begin{phonetics}{暑假}{shu3 jia4}[][HSK 4]
    \definition[个]{s.}{férias de verão; feriado de verão; férias escolares de verão, na China, durante o sétimo e o oitavo meses do calendário gregoriano}
  \end{phonetics}
\end{entry}

\begin{entry}{暖}{13}{⽇}
  \begin{phonetics}{暖}{nuan3}[][HSK 5]
    \definition{adj.}{caloroso; cordial}
    \definition{v.}{aquecer; esquentar; aquecer algo ou aquecer o corpo}
  \end{phonetics}
\end{entry}

\begin{entry}{暖气}{13,4}{⽇、⽓}
  \begin{phonetics}{暖气}{nuan3qi4}[][HSK 4]
    \definition[个]{s.}{aquecedor; aquecimento; aquecimento central}
  \end{phonetics}
\end{entry}

\begin{entry}{暖和}{13,8}{⽇、⼝}
  \begin{phonetics}{暖和}{nuan3huo5}[][HSK 3]
    \definition{adj.}{morno; nem frio nem quente}
    \definition{v.}{aquecer; esquentar}
  \end{phonetics}
\end{entry}

\begin{entry}{暗}{13}{⽇}
  \begin{phonetics}{暗}{an4}[][HSK 4]
    \definition{adj.}{escuro; opaco; sem graça; pouca luz | escondido; secreto; não revelado | pouco claro; nebuloso; vago; confuso | subterrâneo}
    \definition{adv.}{secretamente | no escuro}
  \end{phonetics}
\end{entry}

\begin{entry}{暗示}{13,5}{⽇、⽰}
  \begin{phonetics}{暗示}{an4shi4}[][HSK 4]
    \definition[个]{s.}{sugestão; insinuação; intimação; (psicologia) refere-se ao uso de palavras, gestos, expressões, etc. para fazer as pessoas aceitarem involuntariamente uma determinada opinião ou fazerem algo}
    \definition{v.}{dar uma dica; sugerir secretamente; indicar algo a alguém usando outras palavras, expressões faciais ou gestos sem dizer em voz alta}
  \end{phonetics}
\end{entry}

\begin{entry}{暗香}{13,9}{⽇、⾹}
  \begin{phonetics}{暗香}{an4xiang1}
    \definition{s.}{fragrância sutil}
  \end{phonetics}
\end{entry}

\begin{entry}{暗恋}{13,10}{⽇、⼼}
  \begin{phonetics}{暗恋}{an4lian4}
    \definition{s.}{amor secreto}
    \definition{v.}{estar secretamente apaixonado por}
  \end{phonetics}
\end{entry}

\begin{entry}{暴}{15}{⽇}
  \begin{phonetics}{暴}{bao4}
    \definition*{s.}{Sobrenome Bao}
    \definition{adj.}{repentino e violento | cruel; selvagem; feroz | temperamental | severo e tirânico; brutal | irritável; irascível; impaciente}
    \definition{adv.}{de repente e ferozmente}
    \definition{s.}{violência; ferocidade}
    \definition{v.}{sobressair; destacar-se; inchar | expor; transmitir | desperdiçar; arruinar; estragar}
  \end{phonetics}
\end{entry}

\begin{entry}{暴力}{15,2}{⽇、⼒}
  \begin{phonetics}{暴力}{bao4li4}[][HSK 6]
    \definition{s.}{violência; força (usada em tempos de conflito); poder de coerção}
  \end{phonetics}
\end{entry}

\begin{entry}{暴风雨}{15,4,8}{⽇、⾵、⾬}
  \begin{phonetics}{暴风雨}{bao4 feng1 yu3}[][HSK 6]
    \definition{s.}{tempestade; tormenta; temporal; borrasca; vento e chuva fortes e violentos}
  \end{phonetics}
\end{entry}

\begin{entry}{暴行}{15,6}{⽇、⾏}
  \begin{phonetics}{暴行}{bao4xing2}
    \definition{s.}{ato selvagem | atrocidade | indignação}
  \end{phonetics}
\end{entry}

\begin{entry}{暴乱}{15,7}{⽇、⼄}
  \begin{phonetics}{暴乱}{bao4luan4}
    \definition{s.}{rebelião | revolta | tumulto}
  \end{phonetics}
\end{entry}

\begin{entry}{暴雨}{15,8}{⽇、⾬}
  \begin{phonetics}{暴雨}{bao4yu3}[][HSK 6]
    \definition[场,次,阵]{s.}{tempestade; chuva torrencial; chuva forte com precipitação intensa; em meteorologia, refere-se a chuvas de 16 mm ou mais em uma hora ou 50 mm ou mais em 24 horas}
  \end{phonetics}
\end{entry}

\begin{entry}{暴躁}{15,20}{⽇、⾜}
  \begin{phonetics}{暴躁}{bao4zao4}
    \definition{adj.}{irascível | irritável}
  \end{phonetics}
\end{entry}

\begin{entry}{暴露}{15,21}{⽇、⾬}
  \begin{phonetics}{暴露}{bao4lu4}[][HSK 6]
    \definition{adj.}{reveladoras (roupas inadequadas que expõem muito o corpo)}
    \definition{v.}{expor; desnudar; revelar; tornar público algo oculto}
  \end{phonetics}
\end{entry}

%%%%% EOF %%%%%

