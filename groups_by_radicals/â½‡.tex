%%%
%%% Radical "⽇"
%%%

\section*{Radical 72: ``⽇''}\addcontentsline{toc}{section}{Radical 72: ⽇}

\begin{entry}{日}{4}{⽇}[Kangxi 72]
  \begin{phonetics}{日}{ri4}[][HSK 1]
    \definition*{s.}{Japão, abreviação de~日本}
    \definition{clas.}{dia (mais usado em escrita) | data, dia do mês}
    \seeref{日本}{ri4ben3}
  \end{phonetics}
\end{entry}

\begin{entry}{日子}{4,3}{⽇、⼦}
  \begin{phonetics}{日子}{ri4zi5}[][HSK 2]
    \definition{s.}{dia | uma data (calendário) | dias de vida de alguém}
  \end{phonetics}
\end{entry}

\begin{entry}{日历}{4,4}{⽇、⼚}
  \begin{phonetics}{日历}{ri4li4}[][HSK 4]
    \definition[张,本]{s.}{caledário; livro com o ano, mês, dia, semana, termo solar, aniversário, etc. registrados, um livro por ano, uma página por dia, aberto diariamente}
  \end{phonetics}
\end{entry}

\begin{entry}{日出}{4,5}{⽇、⼐}
  \begin{phonetics}{日出}{ri4chu1}
    \definition{s.}{nascer do sol}
  \seealsoref{夕阳}{xi1yang2}
  \end{phonetics}
\end{entry}

\begin{entry}{日本}{4,5}{⽇、⽊}
  \begin{phonetics}{日本}{ri4ben3}
    \definition*{s.}{Japão}
  \end{phonetics}
\end{entry}

\begin{entry}{日本人}{4,5,2}{⽇、⽊、⼈}
  \begin{phonetics}{日本人}{ri4ben3ren2}
    \definition{s.}{japonês | pessoa ou povo do Japão}
  \end{phonetics}
\end{entry}

\begin{entry}{日记}{4,5}{⽇、⾔}
  \begin{phonetics}{日记}{ri4ji4}[][HSK 4]
    \definition[本,篇,册]{s.}{diário; artigo que registra eventos e pensamentos diários}
  \end{phonetics}
\end{entry}

\begin{entry}{日光灯}{4,6,6}{⽇、⼉、⽕}
  \begin{phonetics}{日光灯}{ri4guang1deng1}
    \definition{s.}{lâmpada fluorescente}
  \end{phonetics}
\end{entry}

\begin{entry}{日报}{4,7}{⽇、⼿}
  \begin{phonetics}{日报}{ri4 bao4}[][HSK 2]
    \definition[张]{s.}{diário | jornal diários}
  \end{phonetics}
\end{entry}

\begin{entry}{日常}{4,11}{⽇、⼱}
  \begin{phonetics}{日常}{ri4chang2}[][HSK 3]
    \definition{adv.}{usual; diário; cotidiano; dia a dia}
  \end{phonetics}
\end{entry}

\begin{entry}{日期}{4,12}{⽇、⽉}
  \begin{phonetics}{日期}{ri4qi1}[][HSK 1]
    \definition{s.}{data}
  \end{phonetics}
\end{entry}

\begin{entry}{旧}{5}{⽇}
  \begin{phonetics}{旧}{jiu4}[][HSK 3]
    \definition*{s.}{sobrenome Jiu}
    \definition{adj.}{passado; antigo; velho | usado; desgastado; velho}
    \definition{s.}{velha amizade; velho amigo}
  \end{phonetics}
\end{entry}

\begin{entry}{早}{6}{⽇}
  \begin{phonetics}{早}{zao3}[][HSK 1]
    \definition{adj.}{prematuramente}
    \definition{adv.}{cedo | antecipadamente | breve}
    \definition{s.}{manhã}
  \end{phonetics}
\end{entry}

\begin{entry}{早上}{6,3}{⽇、⼀}
  \begin{phonetics}{早上}{zao3shang5}[][HSK 1]
    \definition{adv.}{manhã cedo | manhãzinha}
    \definition[个]{s.}{manhã}
  \end{phonetics}
\end{entry}

\begin{entry}{早亡}{6,3}{⽇、⼇}
  \begin{phonetics}{早亡}{zao3wang2}
    \definition[个]{s.}{morte prematura}
    \definition{v.}{morrer prematuramente}
  \end{phonetics}
\end{entry}

\begin{entry}{早已}{6,3}{⽇、⼰}
  \begin{phonetics}{早已}{zao3 yi3}[][HSK 3]
    \definition{adv.}{há muito tempo; por muito tempo}
  \end{phonetics}
\end{entry}

\begin{entry}{早车}{6,4}{⽇、⾞}
  \begin{phonetics}{早车}{zao3che1}
    \definition{s.}{trem matutino | ônibus matutino}
  \end{phonetics}
\end{entry}

\begin{entry}{早安}{6,6}{⽇、⼧}
  \begin{phonetics}{早安}{zao3'an1}
    \definition{interj.}{Bom dia!}
  \end{phonetics}
\end{entry}

\begin{entry}{早早儿}{6,6,2}{⽇、⽇、⼉}
  \begin{phonetics}{早早儿}{zao3zao3r5}
    \definition{adv.}{o mais cedo possível | o mais breve possível}
  \end{phonetics}
\end{entry}

\begin{entry}{早饭}{6,7}{⽇、⾷}
  \begin{phonetics}{早饭}{zao3fan4}[][HSK 1]
    \definition[份,顿,次,餐]{s.}{café da manhã}
  \end{phonetics}
\end{entry}

\begin{entry}{早知}{6,8}{⽇、⽮}
  \begin{phonetics}{早知}{zao3zhi1}
    \definition{v.}{prever | se alguém soubesse antes, \dots}
  \end{phonetics}
\end{entry}

\begin{entry}{早前}{6,9}{⽇、⼑}
  \begin{phonetics}{早前}{zao3qian2}
    \definition{adv.}{previamente}
  \end{phonetics}
\end{entry}

\begin{entry}{早晨}{6,11}{⽇、⽇}
  \begin{phonetics}{早晨}{zao3 chen2}[][HSK 2]
    \definition{adv.}{manhã cedo | manhãzinha}
    \definition[个]{s.}{manhã}
  \end{phonetics}
\end{entry}

\begin{entry}{早就}{6,12}{⽇、⼪}
  \begin{phonetics}{早就}{zao3 jiu4}[][HSK 2]
    \definition{adv.}{já em um momento anterior}
  \end{phonetics}
\end{entry}

\begin{entry}{早餐}{6,16}{⽇、⾷}
  \begin{phonetics}{早餐}{zao3 can1}[][HSK 2]
    \definition[份,顿,次]{s.}{café da manhã}
  \end{phonetics}
\end{entry}

\begin{entry}{时}{7}{⽇}
  \begin{phonetics}{时}{shi2}[][HSK 3]
    \definition*{s.}{sobrenome Shi}
    \definition{adj.}{atual; presente | a tempo; feito a tempo}
    \definition{adv.}{de vez em quando; ocasionalmente; de ​​tempos em tempos | às vezes\dots às vezes\dots}
    \definition{clas.}{hora; horas}
    \definition{s.}{dias; tempos; longo período de tempo | tempo; tempo fixo | hora; hora do dia | temporada | chance; oportunidade | atualidade; presente | tempo verbal}
  \end{phonetics}
\end{entry}

\begin{entry}{时代}{7,5}{⽇、⼈}
  \begin{phonetics}{时代}{shi2dai4}[][HSK 3]
    \definition[个]{s.}{idade; era; tempos; época | um período na vida de alguém}
  \end{phonetics}
\end{entry}

\begin{entry}{时光}{7,6}{⽇、⼉}
  \begin{phonetics}{时光}{shi2guang1}
    \definition{s.}{tempo | época | período de tempo}
  \end{phonetics}
\end{entry}

\begin{entry}{时时}{7,7}{⽇、⽇}
  \begin{phonetics}{时时}{shi2shi2}
    \definition{adv.}{muitas vezes | constantemente}
  \end{phonetics}
\end{entry}

\begin{entry}{时间}{7,7}{⽇、⾨}
  \begin{phonetics}{时间}{shi2jian1}[][HSK 1]
    \definition{s.}{(conceito de, duração de, um ponto no) tempo}
  \end{phonetics}
\end{entry}

\begin{entry}{时刻}{7,8}{⽇、⼑}
  \begin{phonetics}{时刻}{shi2ke4}[][HSK 3]
    \definition{adv.}{constantemente; sempre}
    \definition[个,段]{s.}{tempo; hora; momento; conjuntura}
  \end{phonetics}
\end{entry}

\begin{entry}{时差}{7,9}{⽇、⼯}
  \begin{phonetics}{时差}{shi2cha1}
    \definition{s.}{diferença de tempo | \emph{jet lag}}
  \end{phonetics}
\end{entry}

\begin{entry}{时候}{7,10}{⽇、⼈}
  \begin{phonetics}{时候}{shi2hou5}[][HSK 1]
    \definition{adv.}{quando?}
    \definition{s.}{duração de tempo | momento | período | tempo}
  \end{phonetics}
\end{entry}

\begin{entry}{旷野}{7,11}{⽇、⾥}
  \begin{phonetics}{旷野}{kuang4ye3}
    \definition{s.}{região selvagem}
  \end{phonetics}
\end{entry}

\begin{entry}{明天}{8,4}{⽇、⼤}
  \begin{phonetics}{明天}{ming2tian1}[][HSK 1]
    \definition{adv.}{amanhã}
  \end{phonetics}
\end{entry}

\begin{entry}{明白}{8,5}{⽇、⽩}
  \begin{phonetics}{明白}{ming2bai5}[][HSK 1]
    \definition{adj.}{compreendido | percebido | óbvio | inequívoco}
    \definition{v.}{compreender | perceber}
  \end{phonetics}
\end{entry}

\begin{entry}{明年}{8,6}{⽇、⼲}
  \begin{phonetics}{明年}{ming2nian2}[][HSK 1]
    \definition{adv.}{próximo ano}
  \end{phonetics}
\end{entry}

\begin{entry}{明明}{8,8}{⽇、⽇}
  \begin{phonetics}{明明}{ming2ming2}[][HSK 5]
    \definition{adv.}{obviamente; claramente; sem dúvida; indica que o fenômeno ou princípio é evidente}
  \end{phonetics}
\end{entry}

\begin{entry}{明亮}{8,9}{⽇、⼇}
  \begin{phonetics}{明亮}{ming2 liang4}[][HSK 5]
    \definition{adj.}{claro; bem iluminado | brilhante; resplandecente | claro; simples; compreensível}
  \end{phonetics}
\end{entry}

\begin{entry}{明星}{8,9}{⽇、⽇}
  \begin{phonetics}{明星}{ming2xing1}[][HSK 2]
    \definition[个,位,颗]{s.}{estrela | talento de ponta | estrela (artista) | estrela brilhante | estrela brilhante}
  \end{phonetics}
\end{entry}

\begin{entry}{明显}{8,9}{⽇、⽇}
  \begin{phonetics}{明显}{ming2xian3}[][HSK 3]
    \definition{adj.}{claro; óbvio; distinto}
  \end{phonetics}
\end{entry}

\begin{entry}{明珠}{8,10}{⽇、⽟}
  \begin{phonetics}{明珠}{ming2zhu1}
    \definition{s.}{pérola | jóia (de grande valor)}
  \end{phonetics}
\end{entry}

\begin{entry}{明确}{8,12}{⽇、⽯}
  \begin{phonetics}{明确}{ming2que4}[][HSK 3]
    \definition{adj.}{claro; definido; específico}
    \definition{v.}{deixar claro; tornar definitivo}
  \end{phonetics}
\end{entry}

\begin{entry}{昔日}{8,4}{⽇、⽇}
  \begin{phonetics}{昔日}{xi1ri4}
    \definition{adj.}{passado}
  \end{phonetics}
\end{entry}

\begin{entry}{冒}{9}{⽇}
  \begin{phonetics}{冒}{mao4}[][HSK 5]
    \definition*{s.}{sobrenome Mao}
    \definition{adv.}{com ousadia; precipitadamente | fingidamente; falsamente; fraudulentamente}
    \definition{v.}{emitir; liberar; enviar (para cima) | arriscar; ser corajoso}
  \end{phonetics}
\end{entry}

\begin{entry}{冒险}{9,9}{⽇、⾩}
  \begin{phonetics}{冒险}{mao4xian3}
    \definition{adj.}{corajoso}
    \definition{s.}{risco | aventura}
    \definition{v.+compl.}{correr risco | arriscar-se | aventurar-se em}
  \end{phonetics}
\end{entry}

\begin{entry}{星火}{9,4}{⽇、⽕}
  \begin{phonetics}{星火}{xing1huo3}
    \definition{s.}{trilha de meteoro (usada principalmente em expressões como 急如星火) | faísca}
  \end{phonetics}
\end{entry}

\begin{entry}{星辰}{9,7}{⽇、⾠}
  \begin{phonetics}{星辰}{xing1chen2}
    \definition{s.}{estrelas}
  \end{phonetics}
\end{entry}

\begin{entry}{星表}{9,8}{⽇、⾐}
  \begin{phonetics}{星表}{xing1biao3}
    \definition{s.}{catálogo de estrelas}
  \end{phonetics}
\end{entry}

\begin{entry}{星星}{9,9}{⽇、⽇}
  \begin{phonetics}{星星}{xing1 xing5}[][HSK 2]
    \definition{s.}{estrela}
  \end{phonetics}
\end{entry}

\begin{entry}{星座}{9,10}{⽇、⼴}
  \begin{phonetics}{星座}{xing1zuo4}
    \definition[张]{s.}{signo astrológico | constelação}
  \end{phonetics}
\end{entry}

\begin{entry}{星期}{9,12}{⽇、⽉}
  \begin{phonetics}{星期}{xing1qi1}[][HSK 1]
    \definition[个]{s.}{semana}
  \end{phonetics}
\end{entry}

\begin{entry}{星期一}{9,12,1}{⽇、⽉、⼀}
  \begin{phonetics}{星期一}{xing1qi1yi1}[][HSK 1]
    \definition{s.}{segunda-feira}
  \end{phonetics}
\end{entry}

\begin{entry}{星期二}{9,12,2}{⽇、⽉、⼆}
  \begin{phonetics}{星期二}{xing1qi1'er4}[][HSK 1]
    \definition{s.}{terça-feira}
  \end{phonetics}
\end{entry}

\begin{entry}{星期三}{9,12,3}{⽇、⽉、⼀}
  \begin{phonetics}{星期三}{xing1qi1san1}[][HSK 1]
    \definition{s.}{quarta-feira}
  \end{phonetics}
\end{entry}

\begin{entry}{星期五}{9,12,4}{⽇、⽉、⼆}
  \begin{phonetics}{星期五}{xing1qi1wu3}[][HSK 1]
    \definition{s.}{sexta-feira}
  \end{phonetics}
\end{entry}

\begin{entry}{星期六}{9,12,4}{⽇、⽉、⼋}
  \begin{phonetics}{星期六}{xing1qi1liu4}[][HSK 1]
    \definition{s.}{sábado}
  \end{phonetics}
\end{entry}

\begin{entry}{星期天}{9,12,4}{⽇、⽉、⼤}
  \begin{phonetics}{星期天}{xing1qi1tian1}[][HSK 1]
    \definition{s.}{domingo}
  \seealsoref{星期日}{xing1qi1ri4}
  \end{phonetics}
\end{entry}

\begin{entry}{星期日}{9,12,4}{⽇、⽉、⽇}
  \begin{phonetics}{星期日}{xing1qi1ri4}[][HSK 1]
    \definition{s.}{domingo}
  \seealsoref{星期天}{xing1qi1tian1}
  \end{phonetics}
\end{entry}

\begin{entry}{星期四}{9,12,5}{⽇、⽉、⼞}
  \begin{phonetics}{星期四}{xing1qi1si4}[][HSK 1]
    \definition{s.}{quinta-feira}
  \end{phonetics}
\end{entry}

\begin{entry}{春}{9}{⽇}
  \begin{phonetics}{春}{chun1}
    \definition*{s.}{sobrenome Chun}
    \definition{s.}{primavera | amor | luxúria | vida | vitalidade}
  \end{phonetics}
\end{entry}

\begin{entry}{春天}{9,4}{⽇、⼤}
  \begin{phonetics}{春天}{chun1 tian1}
    \definition[个]{s.}{primavera}
  \end{phonetics}
\end{entry}

\begin{entry}{春节}{9,5}{⽇、⾋}
  \begin{phonetics}{春节}{chun1 jie2}[][HSK 2]
    \definition*{s.}{Festival da Primavera (Ano Novo Chinês)}
  \end{phonetics}
\end{entry}

\begin{entry}{春季}{9,8}{⽇、⼦}
  \begin{phonetics}{春季}{chun1 ji4}[][HSK 4]
    \definition{s.}{primavera; primeiro trimestre do ano, que na China se refere ao período de três meses entre o início da primavera e o início do verão, e também se refere aos três meses do calendário lunar, a saber, o primeiro, o segundo e o terceiro meses}
  \end{phonetics}
\end{entry}

\begin{entry}{昨}{9}{⽇}
  \begin{phonetics}{昨}{zuo2}
    \definition{s.}{ontem}
  \end{phonetics}
\end{entry}

\begin{entry}{昨天}{9,4}{⽇、⼤}
  \begin{phonetics}{昨天}{zuo2tian1}[][HSK 1]
    \definition{adv.}{ontem}
  \end{phonetics}
\end{entry}

\begin{entry}{昨日}{9,4}{⽇、⽇}
  \begin{phonetics}{昨日}{zuo2ri4}
    \definition{adv.}{ontem}
  \end{phonetics}
\end{entry}

\begin{entry}{昨夜}{9,8}{⽇、⼣}
  \begin{phonetics}{昨夜}{zuo2ye4}
    \definition{adv.}{noite passada}
  \end{phonetics}
\end{entry}

\begin{entry}{昨晚}{9,11}{⽇、⽇}
  \begin{phonetics}{昨晚}{zuo2wan3}
    \definition{adv.}{noite passada | ontem à noite}
  \end{phonetics}
\end{entry}

\begin{entry}{是}{9}{⽇}
  \begin{phonetics}{是}{shi4}[][HSK 1]
    \definition{adj.}{correto | certo | verdadeiro | (reconhecimento respeitoso de um comando) muito bem}
    \definition{adv.}{(advérbio para afirmação enfática)}
    \definition{v.}{ser (somente seguido por substantivos)}
  \end{phonetics}
\end{entry}

\begin{entry}{是否}{9,7}{⽇、⼝}
  \begin{phonetics}{是否}{shi4fou3}[][HSK 4]
    \definition{adv.}{se; se ou não}
  \end{phonetics}
\end{entry}

\begin{entry}{是的}{9,8}{⽇、⽩}
  \begin{phonetics}{是的}{shi4de5}
    \definition{adv.}{sim | está certo}
  \end{phonetics}
\end{entry}

\begin{entry}{显示}{9,5}{⽇、⽰}
  \begin{phonetics}{显示}{xian3shi4}[][HSK 3]
    \definition{v.}{mostrar | exibir}
  \end{phonetics}
\end{entry}

\begin{entry}{显得}{9,11}{⽇、⼻}
  \begin{phonetics}{显得}{xian3de5}[][HSK 3]
    \definition{v.}{parecer; aparecer}
  \end{phonetics}
\end{entry}

\begin{entry}{显著}{9,11}{⽇、⽬}
  \begin{phonetics}{显著}{xian3zhu4}[][HSK 4]
    \definition{adj.}{notável; significativo; notável; extraordinário; muito óbvio; muito claramente demonstrado; muito facilmente visto ou sentido}
  \end{phonetics}
\end{entry}

\begin{entry}{显然}{9,12}{⽇、⽕}
  \begin{phonetics}{显然}{xian3ran2}[][HSK 3]
    \definition{adj.}{claro; evidente; óbvio}
    \definition{adv.}{claramente; evidentemente; obviamente}
  \end{phonetics}
\end{entry}

\begin{entry}{晒}{10}{⽇}
  \begin{phonetics}{晒}{shai4}[][HSK 4]
    \definition{v.}{(sol) brilhar sobre | aquecer-se; secar ao sol; colocar algo sob a luz do sol para secar | ignorar (alguém) | mostrar; divulgar o conteúdo de sua vida privada na Internet}
  \end{phonetics}
\end{entry}

\begin{entry}{晒干}{10,3}{⽇、⼲}
  \begin{phonetics}{晒干}{shai4gan1}
    \definition{v.}{secar ao sol}
  \end{phonetics}
\end{entry}

\begin{entry}{晚}{11}{⽇}
  \begin{phonetics}{晚}{wan3}[][HSK 1]
    \definition{adj.}{tarde | noite}
  \end{phonetics}
\end{entry}

\begin{entry}{晚上}{11,3}{⽇、⼀}
  \begin{phonetics}{晚上}{wan3shang5}[][HSK 1]
    \definition{adv.}{noite | à noite}
  \end{phonetics}
\end{entry}

\begin{entry}{晚会}{11,6}{⽇、⼈}
  \begin{phonetics}{晚会}{wan3hui4}[][HSK 2]
    \definition[个]{s.}{festa noturna}
  \end{phonetics}
\end{entry}

\begin{entry}{晚安}{11,6}{⽇、⼧}
  \begin{phonetics}{晚安}{wan3'an1}[][HSK 2]
    \definition{v.}{boa noite}
  \end{phonetics}
\end{entry}

\begin{entry}{晚报}{11,7}{⽇、⼿}
  \begin{phonetics}{晚报}{wan3 bao4}[][HSK 2]
    \definition{s.}{jornal da noite}
  \end{phonetics}
\end{entry}

\begin{entry}{晚近}{11,7}{⽇、⾡}
  \begin{phonetics}{晚近}{wan3jin4}
    \definition{adj.}{recente | mais recente no passado}
    \definition{adv.}{ultimamente | recentemente}
  \end{phonetics}
\end{entry}

\begin{entry}{晚饭}{11,7}{⽇、⾷}
  \begin{phonetics}{晚饭}{wan3fan4}[][HSK 1]
    \definition[份,顿,次,餐]{s.}{jantar}
  \end{phonetics}
\end{entry}

\begin{entry}{晚育}{11,8}{⽇、⾁}
  \begin{phonetics}{晚育}{wan3yu4}
    \definition{s.}{parto tardio}
    \definition{v.}{ter um filho mais tarde}
  \end{phonetics}
\end{entry}

\begin{entry}{晚点}{11,9}{⽇、⽕}
  \begin{phonetics}{晚点}{wan3 dian3}[][HSK 4]
    \definition{adj.}{atrasado}
    \definition{s.}{jantar leve}
    \definition{v.}{atrasar; retardar; adiar; (carro, navio, avião) partir, correr ou chegar mais tarde do que o horário especificado}
  \end{phonetics}
\end{entry}

\begin{entry}{晚景}{11,12}{⽇、⽇}
  \begin{phonetics}{晚景}{wan3jing3}
    \definition{s.}{circunstâncias dos anos de declínio de alguém | cena noturna}
  \end{phonetics}
\end{entry}

\begin{entry}{晚餐}{11,16}{⽇、⾷}
  \begin{phonetics}{晚餐}{wan3can1}[][HSK 2]
    \definition[份,顿,次]{s.}{jantar | refeição noturna}
  \end{phonetics}
\end{entry}

\begin{entry}{普及}{12,3}{⽇、⼃}
  \begin{phonetics}{普及}{pu3ji2}[][HSK 3]
    \definition{adj.}{popular; universal; onipresente}
    \definition{v.}{popularizar; disseminar; espalhar entre o povo}
  \end{phonetics}
\end{entry}

\begin{entry}{普通}{12,10}{⽇、⾡}
  \begin{phonetics}{普通}{pu3 tong1}[][HSK 2]
    \definition{adj.}{ordinário | comum | geral | médio}
  \end{phonetics}
\end{entry}

\begin{entry}{普通话}{12,10,8}{⽇、⾡、⾔}
  \begin{phonetics}{普通话}{pu3tong1hua4}[][HSK 2]
    \definition*{s.}{Mandarim (literalmente ``linguagem comum'') | Putonghua (fala comum da língua chinesa) | discurso comum}
  \end{phonetics}
\end{entry}

\begin{entry}{普遍}{12,12}{⽇、⾡}
  \begin{phonetics}{普遍}{pu3bian4}[][HSK 3]
    \definition{adj.}{geral; comum; universal; difundido}
  \end{phonetics}
\end{entry}

\begin{entry}{景色}{12,6}{⽇、⾊}
  \begin{phonetics}{景色}{jing3se4}[][HSK 3]
    \definition[片,幅,道,处]{s.}{vista; cena; cenário; paisagem}
  \end{phonetics}
\end{entry}

\begin{entry}{景象}{12,11}{⽇、⾗}
  \begin{phonetics}{景象}{jing3 xiang4}[][HSK 5]
    \definition[个]{s.}{cena; visão; vista; quadro}
  \end{phonetics}
\end{entry}

\begin{entry}{晴}{12}{⽇}
  \begin{phonetics}{晴}{qing2}[][HSK 2]
    \definition{adj.}{ensolarado | claro}
  \end{phonetics}
\end{entry}

\begin{entry}{晴天}{12,4}{⽇、⼤}
  \begin{phonetics}{晴天}{qing2 tian1}[][HSK 2]
    \definition[个]{s.}{dia ensolarado}
  \end{phonetics}
\end{entry}

\begin{entry}{智力}{12,2}{⽇、⼒}
  \begin{phonetics}{智力}{zhi4li4}[][HSK 4]
    \definition{s.}{inteligência; refere-se à capacidade de uma pessoa de conhecer e entender coisas objetivas e aplicar o conhecimento e a experiência para resolver problemas, incluindo memória, observação, imaginação, pensamento e julgamento}
  \end{phonetics}
\end{entry}

\begin{entry}{智能}{12,10}{⽇、⾁}
  \begin{phonetics}{智能}{zhi4neng2}[][HSK 4]
    \definition{adj.}{inteligente (telefone, sistema, etc.); descreve máquinas, equipamentos, tecnologia, etc. que foram processados com alta tecnologia e têm a capacidade de falar, pensar, calcular, resolver problemas, etc., como um ser humano}
    \definition{s.}{intelecto; a capacidade de aprender, agir, pensar, inventar, criar, resolver problemas, etc.}
  \end{phonetics}
\end{entry}

\begin{entry}{智商}{12,11}{⽇、⼝}
  \begin{phonetics}{智商}{zhi4shang1}
    \definition{s.}{quociente de inteligência, QI}
  \end{phonetics}
\end{entry}

\begin{entry}{智障}{12,13}{⽇、⾩}
  \begin{phonetics}{智障}{zhi4zhang4}
    \definition{adj./s.}{retardado}
  \end{phonetics}
\end{entry}

\begin{entry}{智慧}{12,15}{⽇、⼼}
  \begin{phonetics}{智慧}{zhi4hui4}
    \definition{s.}{sabedoria | inteligência}
  \end{phonetics}
\end{entry}

\begin{entry}{暑假}{12,11}{⽇、⼈}
  \begin{phonetics}{暑假}{shu3 jia4}[][HSK 4]
    \definition[个]{s.}{férias de verão; feriado de verão; férias escolares de verão, na China, durante o sétimo e o oitavo meses do calendário gregoriano}
  \end{phonetics}
\end{entry}

\begin{entry}{暖}{13}{⽇}
  \begin{phonetics}{暖}{nuan3}[][HSK 5]
    \definition{adj.}{caloroso; cordial}
    \definition{v.}{aquecer; esquentar; aquecer algo ou aquecer o corpo}
  \end{phonetics}
\end{entry}

\begin{entry}{暖气}{13,4}{⽇、⽓}
  \begin{phonetics}{暖气}{nuan3qi4}[][HSK 4]
    \definition[个]{s.}{aquecedor; aquecimento; aquecimento central}
  \end{phonetics}
\end{entry}

\begin{entry}{暖和}{13,8}{⽇、⼝}
  \begin{phonetics}{暖和}{nuan3huo5}[][HSK 3]
    \definition{adj.}{morno; agradável e quente}
    \definition{v.}{aquecer}
  \end{phonetics}
\end{entry}

\begin{entry}{暗}{13}{⽇}
  \begin{phonetics}{暗}{an4}[][HSK 4]
    \definition{adj.}{escuro; opaco; sem graça; pouca luz | escondido; secreto; não revelado | pouco claro; nebuloso; vago; confuso | subterrâneo}
    \definition{adv.}{secretamente | no escuro}
  \end{phonetics}
\end{entry}

\begin{entry}{暗示}{13,5}{⽇、⽰}
  \begin{phonetics}{暗示}{an4shi4}[][HSK 4]
    \definition[个]{s.}{sugestão; insinuação; intimação; (psicologia) refere-se ao uso de palavras, gestos, expressões, etc. para fazer as pessoas aceitarem involuntariamente uma determinada opinião ou fazerem algo}
    \definition{v.}{dar uma dica; sugerir secretamente; indicar algo a alguém usando outras palavras, expressões faciais ou gestos sem dizer em voz alta}
  \end{phonetics}
\end{entry}

\begin{entry}{暗香}{13,9}{⽇、⾹}
  \begin{phonetics}{暗香}{an4xiang1}
    \definition{s.}{fragrância sutil}
  \end{phonetics}
\end{entry}

\begin{entry}{暗恋}{13,10}{⽇、⼼}
  \begin{phonetics}{暗恋}{an4lian4}
    \definition{s.}{amor secreto}
    \definition{v.}{estar secretamente apaixonado por}
  \end{phonetics}
\end{entry}

\begin{entry}{暴力}{15,2}{⽇、⼒}
  \begin{phonetics}{暴力}{bao4li4}
    \definition{adj.}{violento}
    \definition{s.}{violência}
  \end{phonetics}
\end{entry}

\begin{entry}{暴行}{15,6}{⽇、⾏}
  \begin{phonetics}{暴行}{bao4xing2}
    \definition{s.}{ato selvagem | atrocidade | indignação}
  \end{phonetics}
\end{entry}

\begin{entry}{暴乱}{15,7}{⽇、⼄}
  \begin{phonetics}{暴乱}{bao4luan4}
    \definition{s.}{rebelião | revolta | tumulto}
  \end{phonetics}
\end{entry}

\begin{entry}{暴雨}{15,8}{⽇、⾬}
  \begin{phonetics}{暴雨}{bao4yu3}
    \definition[场,阵]{s.}{tempestade | chuva torrencial}
  \end{phonetics}
\end{entry}

\begin{entry}{暴躁}{15,20}{⽇、⾜}
  \begin{phonetics}{暴躁}{bao4zao4}
    \definition{adj.}{irascível | irritável}
  \end{phonetics}
\end{entry}

%%%%% EOF %%%%%

