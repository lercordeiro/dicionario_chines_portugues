%%%
%%% Radical "⾒"
%%%

\section*{Radical 147: ``⾒'' (见)}\addcontentsline{toc}{section}{Radical 147: ⾒、见}

\begin{Entry}{见}{4}{⾒}[Kangxi 147]
  \begin{Phonetics}{见}{jian4}[][HSK 1]
    \definition*{s.}{Sobrenome Jian}
    \definition{part.}{usado antes de um verbo para indicar voz passiva ou para expressar como isso me afeta}
    \definition{s.}{visão; ideia; opinião sobre algo; ponto de vista}
    \definition{v.}{ver; avistar | encontrar-se com; ser exposto a | parecer ser; mostrar evidência de | ver; referir-se a; indicar a fonte ou o local onde deve ser consultado | ver; encontrar; convocar}
  \end{Phonetics}
  \begin{Phonetics}{见}{xian4}
    \definition{v.}{aparecer; também escrito como 现}
  \seealsoref{现}{xian4}
  \end{Phonetics}
\end{Entry}

\begin{Entry}{见过}{4,6}{⾒、⾡}
  \begin{Phonetics}{见过}{jian4 guo4}[][HSK 2]
    \definition{s.}{visto (ver); já viu alguém ou algo; indica um momento no passado; alguém já viu ou encontrou um determinado objeto}
  \end{Phonetics}
\end{Entry}

\begin{Entry}{见到}{4,8}{⾒、⼑}
  \begin{Phonetics}{见到}{jian4 dao4}[][HSK 2]
    \definition{v.}{ver | encontrar; esbarrar; deparar-se com}
  \end{Phonetics}
\end{Entry}

\begin{Entry}{见面}{4,9}{⾒、⾯}
  \begin{Phonetics}{见面}{jian4/mian4}[][HSK 1]
    \definition{v.+compl.}{encontrar-se com alguém;  ver um ao outro; ver alguém face-a-face}
  \end{Phonetics}
\end{Entry}

\begin{Entry}{观}{6}{⾒}
  \begin{Phonetics}{观}{guan1}
    \definition*{s.}{Templo taoísta; ``Koon''}
    \definition{s.}{visão; vista | perspectiva; visão; conceito | aparência; perspectiva | alcance de visão | noção; ideia; conhecimento ou visão das coisas | ponto de vista; postura; uma visão de uma coisa}
    \definition{v.}{olhar para; assistir; observar | contemplar}
  \end{Phonetics}
  \begin{Phonetics}{观}{guan4}
    \definition*{s.}{Sobrenome Guan}
    \definition{s.}{mosteiro taoísta | torre de vigia do portão do palácio | plataforma}
  \end{Phonetics}
\end{Entry}

\begin{Entry}{观众}{6,6}{⾒、⼈}
  \begin{Phonetics}{观众}{guan1zhong4}[][HSK 3]
    \definition[位,名,批,个]{s.}{espectador; público; audiência; pessoas que assistem a espetáculos ou competições}
  \end{Phonetics}
\end{Entry}

\begin{Entry}{观光}{6,6}{⾒、⼉}
  \begin{Phonetics}{观光}{guan1guang1}[][HSK 6]
    \definition{v.}{visitar; passear; fazer turismo; fazer um passeio em um país ou lugar estrangeiro}
  \end{Phonetics}
\end{Entry}

\begin{Entry}{观念}{6,8}{⾒、⼼}
  \begin{Phonetics}{观念}{guan1nian4}[][HSK 3]
    \definition[种,个]{s.}{ideia; conceito; consciência ideológica}
  \end{Phonetics}
\end{Entry}

\begin{Entry}{观测}{6,9}{⾒、⽔}
  \begin{Phonetics}{观测}{guan1ce4}[][HSK 7-9]
    \definition{v.}{pesquisar; observar e medir; observar e medir (astronomia, geografia, clima, direção, etc.) | observar; assistir e analisar; observar e medir (situação)}
  \end{Phonetics}
\end{Entry}

\begin{Entry}{观点}{6,9}{⾒、⽕}
  \begin{Phonetics}{观点}{guan1dian3}[][HSK 2]
    \definition[个,种]{s.}{ponto de vista; perspectiva; a visão ou atitude que se tem sobre algo a partir de uma determinada posição ou perspectiva | ponto de vista; perspectiva; a posição ou perspectiva adotada ao analisar uma questão}
  \end{Phonetics}
\end{Entry}

\begin{Entry}{观看}{6,9}{⾒、⽬}
  \begin{Phonetics}{观看}{guan1 kan4}[][HSK 3]
    \definition{v.}{assistir; ver propositadamente; observar}
  \end{Phonetics}
\end{Entry}

\begin{Entry}{观望}{6,11}{⾒、⽉}
  \begin{Phonetics}{观望}{guan1wang4}[][HSK 7-9]
    \definition{v.}{esperar para ver; observar (de lado) | olhar ao redor}
  \end{Phonetics}
\end{Entry}

\begin{Entry}{观赏}{6,12}{⾒、⾙}
  \begin{Phonetics}{观赏}{guan1shang3}[][HSK 7-9]
    \definition{v.}{ver e admirar; apreciar a vista de; assistir e aproveitar}
  \end{Phonetics}
\end{Entry}

\begin{Entry}{观感}{6,13}{⾒、⼼}
  \begin{Phonetics}{观感}{guan1gan3}[][HSK 7-9]
    \definition{s.}{impressões; observações | impressões de alguém}
  \end{Phonetics}
\end{Entry}

\begin{Entry}{观察}{6,14}{⾒、⼧}
  \begin{Phonetics}{观察}{guan1cha2}[][HSK 3]
    \definition{v.}{assistir; pesquisar; observar; examinar cuidadosamente coisas ou fenômenos}
  \end{Phonetics}
\end{Entry}

\begin{Entry}{观摩}{6,15}{⾒、⼿}
  \begin{Phonetics}{观摩}{guan1mo2}[][HSK 7-9]
    \definition{v.}{inspecionar e aprender com o trabalho uns dos outros; visualizar e emular; observar, refere-se principalmente a observar as conquistas uns dos outros, trocar experiências e aprender uns com os outros}
  \end{Phonetics}
\end{Entry}

\begin{Entry}{现}{8}{⾒}
  \begin{Phonetics}{现}{xian4}
    \definition{adj.}{(dinheiro, etc.) em mãos}
    \definition{adv.}{recente; de improviso; naquela época; temporariamente}
    \definition{s.}{presente; atual; existente | dinheiro; dinheiro de pronto}
    \definition{v.}{mostrar; revelar; aparecer; tornar-se visível}
  \seealsoref{见}{xian4}
  \end{Phonetics}
\end{Entry}

\begin{Entry}{现代}{8,5}{⾒、⼈}
  \begin{Phonetics}{现代}{xian4dai4}[][HSK 3]
    \definition*{s.}{Hyundai, empresa sul-coreana}
    \definition{adj.}{moderno; contemporâneo; com características, estilo e conceitos modernos, refletindo a vanguarda, a moda e a inovação da atualidade}
    \definition{s.}{tempos modernos; era contemporânea; atualmente, na divisão cronológica da história da China, refere-se principalmente ao período desde o Movimento 4 de Maio até os dias atuais}
  \end{Phonetics}
\end{Entry}

\begin{Entry}{现在}{8,6}{⾒、⼟}
  \begin{Phonetics}{现在}{xian4zai4}[][HSK 1]
    \definition{adv.}{agora; no momento; atualmente; neste momento, quando se fala, às vezes inclui um período de tempo mais ou menos longo antes ou depois da fala (diferente de 过去 ou 将来)}
  \seealsoref{过去}{guo4 qu4}
  \seealsoref{将来}{jiang1lai2}
  \end{Phonetics}
\end{Entry}

\begin{Entry}{现场}{8,6}{⾒、⼟}
  \begin{Phonetics}{现场}{xian4chang3}[][HSK 3]
    \definition[个,处]{s.}{local onde ocorreu o acidente, incidente ou desastre| local; ponto; local onde se realizam diretamente atividades como produção, apresentações e competições}
  \end{Phonetics}
\end{Entry}

\begin{Entry}{现有}{8,6}{⾒、⽉}
  \begin{Phonetics}{现有}{xian4 you3}[][HSK 5]
    \definition{adj.}{agora disponível; existente}
    \definition{v.}{estar disponível agora; existir | Literário: ter em mãos; ter em posse}
  \end{Phonetics}
\end{Entry}

\begin{Entry}{现抓}{8,7}{⾒、⼿}
  \begin{Phonetics}{现抓}{xian4zhua1}
    \definition{v.}{improvisar}
  \end{Phonetics}
\end{Entry}

\begin{Entry}{现状}{8,7}{⾒、⽝}
  \begin{Phonetics}{现状}{xian4zhuang4}[][HSK 5]
    \definition{s.}{situação atual}
  \end{Phonetics}
\end{Entry}

\begin{Entry}{现实}{8,8}{⾒、⼧}
  \begin{Phonetics}{现实}{xian4shi2}[][HSK 3]
    \definition{adj.}{real; efetivo; verdadeiro; de acordo com circunstâncias objetivas}
    \definition[个]{s.}{realidade; factualidade; coisas que existem objetivamente}
  \end{Phonetics}
\end{Entry}

\begin{Entry}{现货}{8,8}{⾒、⾙}
  \begin{Phonetics}{现货}{xian4huo4}
    \definition{s.}{produtos à vista}
  \end{Phonetics}
\end{Entry}

\begin{Entry}{现货的}{8,8,8}{⾒、⾙、⽩}
  \begin{Phonetics}{现货的}{xian4huo4 de5}
    \definition{s.}{produtos em estoque}
  \end{Phonetics}
\end{Entry}

\begin{Entry}{现金}{8,8}{⾒、⾦}
  \begin{Phonetics}{现金}{xian4jin1}[][HSK 3]
    \definition[笔]{s.}{dinheiro; dinheiro vivo; moeda que pode ser usada diretamente | reserva de dinheiro em um banco; o dinheiro guardado no cofre do banco}
  \end{Phonetics}
\end{Entry}

\begin{Entry}{现做}{8,11}{⾒、⼈}
  \begin{Phonetics}{现做}{xian4zuo4}
    \definition{adj.}{fresco}
    \definition{v.}{fazer (comida) no local}
  \end{Phonetics}
\end{Entry}

\begin{Entry}{现象}{8,11}{⾒、⾗}
  \begin{Phonetics}{现象}{xian4xiang4}[][HSK 3]
    \definition[个,种]{s.}{aparência (das coisas); fenômeno; a forma externa e as relações manifestadas pelas coisas em seu desenvolvimento e mudança}
  \end{Phonetics}
\end{Entry}

\begin{Entry}{规}{8}{⾒}
  \begin{Phonetics}{规}{gui1}
    \definition*{s.}{Sobrenome Gui}
    \definition[个,种]{s.}{bússola | regulamentação; regra | (mecânica) medidor | compasso; ferramenta para desenhar círculos}
    \definition{v.}{admoestar; aconselhar; advertir | planejar; fazer planos}
  \end{Phonetics}
\end{Entry}

\begin{Entry}{规划}{8,6}{⾒、⼑}
  \begin{Phonetics}{规划}{gui1hua4}[][HSK 5]
    \definition[个,项]{s.}{plano; projeto; planejamento; programa; programação; esquematização; plano de desenvolvimento de longo prazo mais abrangente}
    \definition{v.}{planejar; programar}
  \end{Phonetics}
\end{Entry}

\begin{Entry}{规则}{8,6}{⾒、⼑}
  \begin{Phonetics}{规则}{gui1ze2}[][HSK 4]
    \definition{adj.}{ordenado; regular; descreve a forma, estrutura, arranjo, etc., que se conformam a uma determinada maneira organizada}
    \definition{s.}{regra; regulamento; sistema ou código de conduta prescrito para observância comum | lei; norma}
  \end{Phonetics}
\end{Entry}

\begin{Entry}{规定}{8,8}{⾒、⼧}
  \begin{Phonetics}{规定}{gui1ding4}[][HSK 3]
    \definition[个,条,项,款]{s.}{regra; regulamento; estipulação; tomar decisões sobre a forma, o método, a quantidade ou a qualidade de algo}
    \definition{v.}{estipular; prover; prescrever; estabelecer requisitos ou restrições em termos de métodos, qualidade, quantidade, tempo, etc.}
  \end{Phonetics}
\end{Entry}

\begin{Entry}{规律}{8,9}{⾒、⼻}
  \begin{Phonetics}{规律}{gui1lv4}[][HSK 4]
    \definition{adj.}{estável; regular; coisas, comportamentos, fenômenos, etc. que ocorrem em um determinado momento}
    \definition{s.}{lei; padrão regular; conexão essencial e recorrente entre as coisas}
  \end{Phonetics}
\end{Entry}

\begin{Entry}{规矩}{8,9}{⾒、⽮}
  \begin{Phonetics}{规矩}{gui1ju5}[][HSK 7-9]
    \definition{adj.}{adequado; bem comportado; bem disciplinado; honesto e correto; de acordo com os padrões ou o senso comum}
    \definition[条,个,项]{s.}{regra; costume; prática estabelecida; certos padrões, regras ou costumes}
  \end{Phonetics}
\end{Entry}

\begin{Entry}{规范}{8,9}{⾒、⾋}
  \begin{Phonetics}{规范}{gui1fan4}[][HSK 3]
    \definition{adj.}{regular; normal; padrão; que atende às especificações; em conformidade com as normas}
    \definition{s.}{norma; padrão; diretriz}
    \definition{v.}{regular; padronizar; tornar conforme as normas}
  \end{Phonetics}
\end{Entry}

\begin{Entry}{规格}{8,10}{⾒、⽊}
  \begin{Phonetics}{规格}{gui1ge2}[][HSK 7-9]
    \definition[种]{s.}{normas; padrões; especificações; padrões de qualidade do produto, como determinados tamanho, peso, precisão, desempenho, etc. | formato; padrão; requisito; geralmente se refere a requisitos ou condições especificados}
  \end{Phonetics}
\end{Entry}

\begin{Entry}{规模}{8,14}{⾒、⽊}
  \begin{Phonetics}{规模}{gui1mo2}[][HSK 4]
    \definition[个,种]{s.}{escala; escopo; dimensões; padrão, forma ou escopo (de um empreendimento, instituição, projeto, movimento, etc.)}
  \end{Phonetics}
\end{Entry}

\begin{Entry}{视}{8}{⾒}
  \begin{Phonetics}{视}{shi4}
    \definition{v.}{olhar para | considerar; olhar para | inspecionar; observar}
  \end{Phonetics}
\end{Entry}

\begin{Entry}{视为}{8,4}{⾒、⼂}
  \begin{Phonetics}{视为}{shi4 wei2}[][HSK 5]
    \definition{v.}{considerar; ver como; considerar como; considerar ser; achar que é}
  \end{Phonetics}
\end{Entry}

\begin{Entry}{视角}{8,7}{⾒、⾓}
  \begin{Phonetics}{视角}{shi4jiao3}
    \definition{s.}{ângulo do qual se observa um objeto | (figurativo) perspectiva, ponto de vista, quadro de referência | (cinematografia) ângulo da câmera | (percepção visual) ângulo visual (o ângulo que um objeto visto subtende no olho) | (fotografia) ângulo de visão}
  \end{Phonetics}
\end{Entry}

\begin{Entry}{视频}{8,13}{⾒、⾴}
  \begin{Phonetics}{视频}{shi4pin2}[][HSK 5]
    \definition[个,段,条]{s.}{vídeo; videoclipe}
  \end{Phonetics}
\end{Entry}

\begin{Entry}{败}{8}{⾒}
  \begin{Phonetics}{败}{bai4}[][HSK 4]
    \definition{adj.}{ruim; deteriorado; murcho; dilapidado; decadente}
    \definition{v.}{ser derrotado; perder (oposto a 胜) | derrotar; bater | falha (oposto a 成) | estragar; arruinar | decair; murchar | quebrar; neutralizar; dissipar}
  \seealsoref{成}{cheng2}
  \seealsoref{胜}{sheng4}
  \end{Phonetics}
\end{Entry}

\begin{Entry}{觉}{9}{⾒}
  \begin{Phonetics}{觉}{jiao4}[][HSK 6]
    \definition[个]{s.}{sono; o processo desde adormecer até acordar}
  \end{Phonetics}
  \begin{Phonetics}{觉}{jue2}
    \definition{s.}{sentimento; senso; percepção e discriminação de estímulos externos}
    \definition{v.}{sentir; perceber | acordar | tornar-se consciente; tornar-se desperto; despertar; entender}
  \end{Phonetics}
\end{Entry}

\begin{Entry}{觉悟}{9,10}{⾒、⼼}
  \begin{Phonetics}{觉悟}{jue2wu4}[][HSK 6]
    \definition{s.}{consciência; percepção; compreensão; nível de consciência}
    \definition{v.}{vir a compreender; tornar-se consciente de; tornar-se politicamente desperto; despertar}
  \end{Phonetics}
\end{Entry}

\begin{Entry}{觉得}{9,11}{⾒、⼻}
  \begin{Phonetics}{觉得}{jue2de5}[][HSK 1]
    \definition{v.}{sentir; estar ciente; pressentir; causar uma sensação | pensar; sentir; encontrar; considerar (tom menos assertivo)}
  \end{Phonetics}
\end{Entry}

%%%%% EOF %%%%%

