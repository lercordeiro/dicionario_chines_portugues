%%%
%%% Radical "⼒"
%%%

\section*{Radical 19: ``⼒''}\addcontentsline{toc}{section}{Radical 19: ⼒}

\begin{entry}{力}{2}{⼒}[Kangxi 19]
  \begin{phonetics}{力}{li4}[][HSK 3]
    \definition*{s.}{sobrenome Li}
    \definition{adv.}{energicamente; arduamente; vigorosamente}
    \definition{s.}{poder; força; habilidade; capacidade | força; energia; poder | força física}
    \definition{v.}{fazer tudo o que puder; fazer todo o esforço}
  \end{phonetics}
\end{entry}

\begin{entry}{力气}{2,4}{⼒、⽓}
  \begin{phonetics}{力气}{li4qi5}[][HSK 4]
    \definition[点,把]{s.}{força física | esforço}
  \end{phonetics}
\end{entry}

\begin{entry}{力量}{2,12}{⼒、⾥}
  \begin{phonetics}{力量}{li4liang5}[][HSK 3]
    \definition[出]{s.}{força física; força espiritual | habilidade; capacidade | eficácia; efeito | força (pessoa ou grupo que tem muito poder ou influência)}
  \end{phonetics}
\end{entry}

\begin{entry}{劝}{4}{⼒}
  \begin{phonetics}{劝}{quan4}[][HSK 5]
    \definition*{s.}{sobrenome Quan}
    \definition{v.}{insistir; aconselhar; tentar persuadir; persuadir, argumentar para que as pessoas obedeçam | incentivar; encorajar}
  \end{phonetics}
\end{entry}

\begin{entry}{办}{4}{⼒}
  \begin{phonetics}{办}{ban4}[][HSK 2]
    \definition{v.}{lidar com | lidar | gerenciar | configurar}
  \end{phonetics}
\end{entry}

\begin{entry}{办公}{4,4}{⼒、⼋}
  \begin{phonetics}{办公}{ban4gong1}
    \definition{v.+compl.}{lidar com negócios oficiais | trabalhar (especialmente em um escritório)}
  \end{phonetics}
\end{entry}

\begin{entry}{办公室}{4,4,9}{⼒、⼋、⼧}
  \begin{phonetics}{办公室}{ban4gong1shi4}[][HSK 2]
    \definition[间]{s.}{gabinete | escritório}
  \end{phonetics}
\end{entry}

\begin{entry}{办事}{4,8}{⼒、⼅}
  \begin{phonetics}{办事}{ban4 shi4}[][HSK 4]
    \definition{v.}{trabalhar | lidar com assuntos; manipular transações}
  \end{phonetics}
\end{entry}

\begin{entry}{办法}{4,8}{⼒、⽔}
  \begin{phonetics}{办法}{ban4fa3}[][HSK 2]
    \definition[条,个]{s.}{meio (de se fazer alguma coisa) | método | medida}
  \end{phonetics}
\end{entry}

\begin{entry}{办理}{4,11}{⼒、⽟}
  \begin{phonetics}{办理}{ban4li3}[][HSK 3]
    \definition{v.}{conduzir | manusear | transacionar}
  \end{phonetics}
\end{entry}

\begin{entry}{功夫}{5,4}{⼒、⼤}
  \begin{phonetics}{功夫}{gong1fu5}[][HSK 3]
    \definition*{s.}{Gongfu (Kung Fu), arte marcial}
    \definition[番]{s.}{habilidade; feitura | luta acrobática; habilidade em artes marciais | esforço; tempo e energia}
  \end{phonetics}
\end{entry}

\begin{entry}{功臣}{5,6}{⼒、⾂}
  \begin{phonetics}{功臣}{gong1chen2}
    \definition{s.}{oficial meritório | pessoa que presta serviço excepcional, herói | (fig.) alguém que desempenha um papel vital}
  \end{phonetics}
\end{entry}

\begin{entry}{功能}{5,10}{⼒、⾁}
  \begin{phonetics}{功能}{gong1neng2}[][HSK 3]
    \definition[种,项]{s.}{função}
  \end{phonetics}
\end{entry}

\begin{entry}{功课}{5,10}{⼒、⾔}
  \begin{phonetics}{功课}{gong1 ke4}[][HSK 3]
    \definition[份,门]{s.}{trabalho escolar; dever de casa | tarefa; lições; lição escolar}
  \end{phonetics}
\end{entry}

\begin{entry}{加}{5}{⼒}
  \begin{phonetics}{加}{jia1}[][HSK 2]
    \definition*{s.}{Canadá, abreviação de~加拿大 | sobrenome Jia}
    \seeref{加拿大}{jia1na2da4}
  \end{phonetics}
\end{entry}

\begin{entry}{加入}{5,2}{⼒、⼊}
  \begin{phonetics}{加入}{jia1ru4}[][HSK 4]
    \definition{v.}{juntar-se; unir-se; aderir a; tornar-se um membro de uma organização, grupo | adicionar; colocar em}
  \end{phonetics}
\end{entry}

\begin{entry}{加上}{5,3}{⼒、⼀}
  \begin{phonetics}{加上}{jia1 shang4}[][HSK 5]
    \definition{conj.}{além disso; em adição}
    \definition{v.}{adicionar; acrescentar; dar; aumentar}
  \end{phonetics}
\end{entry}

\begin{entry}{加工}{5,3}{⼒、⼯}
  \begin{phonetics}{加工}{jia1gong1}[][HSK 3]
    \definition{s.}{processo | trabalho (de uma máquina)}
    \definition{v.}{processar | melhorar; polir}
  \end{phonetics}
\end{entry}

\begin{entry}{加以}{5,4}{⼒、⼈}
  \begin{phonetics}{加以}{jia1 yi3}[][HSK 5]
    \definition{conj.}{além disso; em adição; indica outras razões ou condições}
    \definition{v.}{usado na frente de palavras dissilábicas para indicar como um objeto mencionado deve ser tratado ou descartado | usado antes de um verbo polifônico ou de um substantivo formado a partir de um verbo para indicar como tratar ou lidar com o que foi mencionado anteriormente}
  \end{phonetics}
\end{entry}

\begin{entry}{加快}{5,7}{⼒、⼼}
  \begin{phonetics}{加快}{jia1 kuai4}[][HSK 3]
    \definition{v.}{acelerar; aumentar a velocidade}
  \end{phonetics}
\end{entry}

\begin{entry}{加油}{5,8}{⼒、⽔}
  \begin{phonetics}{加油}{jia1you2}[][HSK 2]
    \definition{v.+compl.}{lubrificar | encher o tanque de combustível | fazer um esforço maior | fazer um esforço extra}
  \end{phonetics}
\end{entry}

\begin{entry}{加油站}{5,8,10}{⼒、⽔、⽴}
  \begin{phonetics}{加油站}{jia1you2zhan4}[][HSK 4]
    \definition[个,家]{s.}{posto de gasolina; posto de combustível; postos de abastecimento para venda a varejo de gasolina e óleo para carros e outros veículos motorizados}
  \end{phonetics}
\end{entry}

\begin{entry}{加拿大}{5,10,3}{⼒、⼿、⼤}
  \begin{phonetics}{加拿大}{jia1na2da4}
    \definition{s.}{Canadá}
  \end{phonetics}
\end{entry}

\begin{entry}{加拿大人}{5,10,3,2}{⼒、⼿、⼤、⼈}
  \begin{phonetics}{加拿大人}{jia1na2da4ren2}
    \definition{s.}{canadense | pessoa ou povo do Canadá}
  \end{phonetics}
\end{entry}

\begin{entry}{加热}{5,10}{⼒、⽕}
  \begin{phonetics}{加热}{jia1 re4}[][HSK 5]
    \definition{v.}{aquecer; esquentar; aumentar a temperatura de um objeto}
  \end{phonetics}
\end{entry}

\begin{entry}{加班}{5,10}{⼒、⽟}
  \begin{phonetics}{加班}{jia1ban1}[][HSK 4]
    \definition{v.+compl.}{fazer horas extras; trabalhar horas extras}
  \end{phonetics}
\end{entry}

\begin{entry}{加速}{5,10}{⼒、⾡}
  \begin{phonetics}{加速}{jia1 su4}[][HSK 5]
    \definition{v.}{acelerar; agilizar}
  \end{phonetics}
\end{entry}

\begin{entry}{加速度}{5,10,9}{⼒、⾡、⼴}
  \begin{phonetics}{加速度}{jia1su4du4}
    \definition{s.}{aceleração}
  \end{phonetics}
\end{entry}

\begin{entry}{加强}{5,12}{⼒、⼸}
  \begin{phonetics}{加强}{jia1 qiang2}[][HSK 3]
    \definition{v.}{fortalecer; engrandecer; reforçar}
  \end{phonetics}
\end{entry}

\begin{entry}{务实}{5,8}{⼒、⼧}
  \begin{phonetics}{务实}{wu4shi2}
    \definition{adj.}{pragmático}
    \definition{v.}{lidar com assuntos concretos}
  \end{phonetics}
\end{entry}

\begin{entry}{动}{6}{⼒}
  \begin{phonetics}{动}{dong4}[][HSK 1]
    \definition{v.}{mover | movimentar}
  \end{phonetics}
\end{entry}

\begin{entry}{动人}{6,2}{⼒、⼈}
  \begin{phonetics}{动人}{dong4 ren2}[][HSK 3]
    \definition{adj.}{em movimento; tocando}
  \end{phonetics}
\end{entry}

\begin{entry}{动力}{6,2}{⼒、⼒}
  \begin{phonetics}{动力}{dong4li4}
    \definition{s.}{poder; força motriz | ímpeto; força motriz (ou propulsora)}
  \end{phonetics}
\end{entry}

\begin{entry}{动手}{6,4}{⼒、⼿}
  \begin{phonetics}{动手}{dong4shou3}[][HSK 5]
    \definition{v.+compl.}{iniciar o trabalho; começar a trabalhar | tocar; manusear; manipular | bater; levantar a mão (para bater); espancar}
  \end{phonetics}
\end{entry}

\begin{entry}{动机}{6,6}{⼒、⽊}
  \begin{phonetics}{动机}{dong4ji1}[][HSK 5]
    \definition[部]{s.}{motivo; razão; intenção; ideias que motivam as pessoas a se envolverem em determinados comportamentos}
  \end{phonetics}
\end{entry}

\begin{entry}{动作}{6,7}{⼒、⼈}
  \begin{phonetics}{动作}{dong4zuo4}[][HSK 1]
    \definition[个]{s.}{movimento | ação}
    \definition{v.}{mover | agir}
  \end{phonetics}
\end{entry}

\begin{entry}{动员}{6,7}{⼒、⼝}
  \begin{phonetics}{动员}{dong4yuan2}[][HSK 5]
    \definition{v.}{despertar; mobilizar; iniciar (para fazer algo ou participar de uma atividade) | mobilizar toda a nação; transferir dos setores militar, político e econômico para uma situação de guerra}
  \end{phonetics}
\end{entry}

\begin{entry}{动身}{6,7}{⼒、⾝}
  \begin{phonetics}{动身}{dong4shen1}
    \definition{v.+compl.}{fazer uma jornada | começar uma jornada | partir | partir em uma jornada | sair (para um lugar distante)}
  \end{phonetics}
\end{entry}

\begin{entry}{动态}{6,8}{⼒、⼼}
  \begin{phonetics}{动态}{dong4tai4}[][HSK 5]
    \definition{s.}{tendências; desenvolvimentos; tendência geral dos assuntos; causa provável de ação; curso dos acontecimentos | expressão; comportamento ativo | estado dinâmico; condição dinâmica; de ou em relação a um estado de movimento}
  \end{phonetics}
\end{entry}

\begin{entry}{动物}{6,8}{⼒、⽜}
  \begin{phonetics}{动物}{dong4wu4}[][HSK 2]
    \definition[只,群,个]{s.}{animal}
  \end{phonetics}
\end{entry}

\begin{entry}{动物园}{6,8,7}{⼒、⽜、⼞}
  \begin{phonetics}{动物园}{dong4 wu4 yuan2}[][HSK 2]
    \definition[个]{s.}{jardim zoológico | zoo}
  \end{phonetics}
\end{entry}

\begin{entry}{动画片}{6,8,4}{⼒、⽥、⽚}
  \begin{phonetics}{动画片}{dong4hua4pian4}[][HSK 4]
    \definition[部]{s.}{desenho animado; animações; filme de animação}
  \end{phonetics}
\end{entry}

\begin{entry}{动感}{6,13}{⼒、⼼}
  \begin{phonetics}{动感}{dong4gan3}
    \definition{adj.}{dinâmica | vívida}
    \definition{adv.}{dinamicamente}
    \definition{s.}{senso de movimento (geralmente em uma obra de arte estática)}
  \end{phonetics}
\end{entry}

\begin{entry}{动摇}{6,13}{⼒、⼿}
  \begin{phonetics}{动摇}{dong4 yao2}[][HSK 4]
    \definition{adj.}{instável}
    \definition{v.}{ondular; pairar; agitar; balançar; sacudir | hesitar; vacilar; esmorecer; abalar}
  \end{phonetics}
\end{entry}

\begin{entry}{动漫}{6,14}{⼒、⽔}
  \begin{phonetics}{动漫}{dong4man4}
    \definition{s.}{desenhos animados | quadrinhos | anime | mangá}
  \end{phonetics}
\end{entry}

\begin{entry}{助兴}{7,6}{⼒、⼋}
  \begin{phonetics}{助兴}{zhu4xing4}
    \definition{v.+compl.}{animar as coisas | juntar-se à diversão}
  \end{phonetics}
\end{entry}

\begin{entry}{努力}{7,2}{⼒、⼒}
  \begin{phonetics}{努力}{nu3li4}[][HSK 2]
    \definition{adj.}{diligente | aplicado}
    \definition{s.}{esforçar-se | se esforçar}
  \end{phonetics}
\end{entry}

\begin{entry}{劳工同事}{7,3,6,8}{⼒、⼯、⼝、⼅}
  \begin{phonetics}{劳工同事}{lao2gong1 tong2shi4}
    \definition{s.}{colaborador | colega de trabalho}
  \end{phonetics}
\end{entry}

\begin{entry}{劳动}{7,6}{⼒、⼒}
  \begin{phonetics}{劳动}{lao2dong4}[][HSK 5]
    \definition[次]{s.}{trabalho; mão de obra; atividades intelectuais ou físicas que podem criar valor | trabalho físico; trabalho manual; referindo-se especificamente ao trabalho físico}
    \definition{v.}{realizar trabalho físico}
  \end{phonetics}
\end{entry}

\begin{entry}{势力}{8,2}{⼒、⼒}
  \begin{phonetics}{势力}{shi4li4}[][HSK 5]
    \definition{s.}{força; poder; influência; forças políticas, econômicas, militares, etc.}
  \end{phonetics}
\end{entry}

\begin{entry}{勇士}{9,3}{⼒、⼠}
  \begin{phonetics}{勇士}{yong3shi4}
    \definition{s.}{um guerreiro | uma pessoa corajosa}
  \end{phonetics}
\end{entry}

\begin{entry}{勇气}{9,4}{⼒、⽓}
  \begin{phonetics}{勇气}{yong3qi4}[][HSK 4]
    \definition[种,股]{s.}{coragem; arrojo; nervos; coragem para agir sem medo}
  \end{phonetics}
\end{entry}

\begin{entry}{勇敢}{9,11}{⼒、⽁}
  \begin{phonetics}{勇敢}{yong3gan3}[][HSK 4]
    \definition{adj.}{bravo; valente; galante; corajoso}
  \end{phonetics}
\end{entry}

\begin{entry}{勤奋}{13,8}{⼒、⼤}
  \begin{phonetics}{勤奋}{qin2fen4}[][HSK 5]
    \definition{adj.}{diligente; assíduo; trabalhador; descreve alguém que se esforça continuamente nos estudos ou no trabalho}
  \end{phonetics}
\end{entry}

%%%%% EOF %%%%%

