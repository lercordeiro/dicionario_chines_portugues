%%%
%%% Radical "⽢"
%%%

\section*{Radical 99: ``⽢''}\addcontentsline{toc}{section}{Radical 99: ⽢}

\begin{Entry}{甘}{5}{⽢}[Kangxi 99]
  \begin{Phonetics}{甘}{gan1}
    \definition*{s.}{Província de Gansu, abreviação de 甘肃 | Sobrenome Gan}
    \definition{adj.}{doce; agradável; satisfatório}
    \definition{v.}{estar disposto a; estar contente ou satisfeito com}
  \seealsoref{甘肃}{gan1su4}
  \end{Phonetics}
\end{Entry}

\begin{Entry}{甘心}{5,4}{⽢、⼼}
  \begin{Phonetics}{甘心}{gan1xin1}[][HSK 7-9]
    \definition{v.}{estar contente com; estar disposto a | reconciliar-se com; resignar-se com; contentar-se com}
  \end{Phonetics}
\end{Entry}

\begin{Entry}{甘肃}{5,8}{⽢、⾀}
  \begin{Phonetics}{甘肃}{gan1su4}
    \definition*{s.}{Província de Gansu}
  \end{Phonetics}
\end{Entry}

\begin{Entry}{甘薯}{5,16}{⽢、⾋}
  \begin{Phonetics}{甘薯}{gan1shu3}
    \definition{s.}{batata doce}
  \end{Phonetics}
\end{Entry}

\begin{Entry}{甚}{9}{⽢}
  \begin{Phonetics}{甚}{shen4}
    \definition{adv.}{muito; extremamente}
    \definition{pron.}{o que}
    \definition{v.}{exceder; superar}
  \seealsoref{什么}{shen2me5}
  \end{Phonetics}
\end{Entry}

\begin{Entry}{甚而}{9,6}{⽢、⽽}
  \begin{Phonetics}{甚而}{shen4'er2}
    \definition{conj.}{(ir) tão longe quanto | tanto que}
  \end{Phonetics}
\end{Entry}

\begin{Entry}{甚至}{9,6}{⽢、⾄}
  \begin{Phonetics}{甚至}{shen4zhi4}[][HSK 4]
    \definition{conj.}{e até mesmo; nem mesmo; para apresentar uma situação típica e especial, para enfatizar a profundidade e a seriedade de uma situação}
  \end{Phonetics}
\end{Entry}

\begin{Entry}{甚或}{9,8}{⽢、⼽}
  \begin{Phonetics}{甚或}{shen4huo4}
    \definition{conj.}{(ir) tão longe quanto | tanto que}
  \end{Phonetics}
\end{Entry}

\begin{Entry}{甜}{11}{⽢}
  \begin{Phonetics}{甜}{tian2}[][HSK 3]
    \definition{adj.}{doce; melado | agradável; confortável; fazer as pessoas se sentirem confortáveis e felizes | (sono) profundo | feliz; descreve o sentimento de felicidade}
  \end{Phonetics}
\end{Entry}

\begin{Entry}{甜心}{11,4}{⽢、⼼}
  \begin{Phonetics}{甜心}{tian2xin1}
    \definition{s.}{querido}
  \end{Phonetics}
\end{Entry}

\begin{Entry}{甜头}{11,5}{⽢、⼤}
  \begin{Phonetics}{甜头}{tian2tou5}
    \definition{s.}{benefício | sabor doce (de poder, sucesso, etc.)}
  \end{Phonetics}
\end{Entry}

\begin{Entry}{甜玉米}{11,5,6}{⽢、⽟、⽶}
  \begin{Phonetics}{甜玉米}{tian2 yu4mi3}
    \definition{s.}{milho doce}
  \end{Phonetics}
\end{Entry}

\begin{Entry}{甜言}{11,7}{⽢、⾔}
  \begin{Phonetics}{甜言}{tian2yan2}
    \definition{s.}{boa conversa | palavras amáveis}
  \end{Phonetics}
\end{Entry}

\begin{Entry}{甜品}{11,9}{⽢、⼝}
  \begin{Phonetics}{甜品}{tian2pin3}
    \definition{s.}{sobremesa}
  \end{Phonetics}
\end{Entry}

\begin{Entry}{甜食}{11,9}{⽢、⾷}
  \begin{Phonetics}{甜食}{tian2shi2}
    \definition{s.}{doces | sobremesa}
  \end{Phonetics}
\end{Entry}

\begin{Entry}{甜酒}{11,10}{⽢、⾣}
  \begin{Phonetics}{甜酒}{tian2jiu3}
    \definition{s.}{licor doce}
  \end{Phonetics}
\end{Entry}

\begin{Entry}{甜甜圈}{11,11,11}{⽢、⽢、⼞}
  \begin{Phonetics}{甜甜圈}{tian2tian2quan1}
    \definition{s.}{rosquinha | \emph{doughnut}}
  \end{Phonetics}
\end{Entry}

\begin{Entry}{甜菊}{11,11}{⽢、⾋}
  \begin{Phonetics}{甜菊}{tian2ju2}
    \definition{s.}{estévia, arbusto cujas folhas produzem um substituto para o açúcar}
  \end{Phonetics}
\end{Entry}

\begin{Entry}{甜筒}{11,12}{⽢、⽵}
  \begin{Phonetics}{甜筒}{tian2tong3}
    \definition{s.}{sorvete de casquinha}
  \end{Phonetics}
\end{Entry}

\begin{Entry}{甜稚}{11,13}{⽢、⽲}
  \begin{Phonetics}{甜稚}{tian2zhi4}
    \definition{s.}{doce e inocente}
  \end{Phonetics}
\end{Entry}

\begin{Entry}{甜酸}{11,14}{⽢、⾣}
  \begin{Phonetics}{甜酸}{tian2suan1}
    \definition{adj.}{agridoce}
  \end{Phonetics}
\end{Entry}

%%%%% EOF %%%%%

