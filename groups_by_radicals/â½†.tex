%%%
%%% Radical "⽆"
%%%

\section*{Radical 71: ``⽆'' (旡)}\addcontentsline{toc}{section}{Radical 71: ⽆、旡}

\begin{entry}{无}{4}{⽆}[Kangxi 71]
  \begin{phonetics}{无}{wu2}[][HSK 4]
    \definition{adv.}{não; não ter algo; não há\dots}
    \definition{conj.}{independentemente de; não importa se, o que, etc.}
    \definition{v.}{não ter; estar sem; não existir;}
  \end{phonetics}
\end{entry}

\begin{entry}{无人}{4,2}{⽆、⼈}
  \begin{phonetics}{无人}{wu2ren2}
    \definition{adj.}{não tripulado | desabitado}
  \end{phonetics}
\end{entry}

\begin{entry}{无人机}{4,2,6}{⽆、⼈、⽊}
  \begin{phonetics}{无人机}{wu2ren2ji1}
    \definition{s.}{\emph{drone} | veículo aéreo não tripulado}
  \end{phonetics}
\end{entry}

\begin{entry}{无论}{4,6}{⽆、⾔}
  \begin{phonetics}{无论}{wu2lun4}[][HSK 4]
    \definition{conj.}{não importa o quê; não importa como; independentemente de; indica que as condições são diferentes, mas resultado é o mesmo |}
  \seealsoref{无论……也……}{wu2lun4 ye3}
  \end{phonetics}
\end{entry}

\begin{entry}{无论……也……}{4,6,3}{⽆、⾔、⼄}
  \begin{phonetics}{无论……也……}{wu2lun4 ye3}
    \definition{conj.}{não apenas\dots, (o que, quem, como, etc.), \dots}
  \end{phonetics}
\end{entry}

\begin{entry}{无所谓}{4,8,11}{⽆、⼾、⾔}
  \begin{phonetics}{无所谓}{wu2suo3wei4}[][HSK 4]
    \definition{v.}{não pode ser designado como; não merece o nome de; ser incapaz de dizer ou contar | não ter importância; ser indiferente;}
  \end{phonetics}
\end{entry}

\begin{entry}{无法}{4,8}{⽆、⽔}
  \begin{phonetics}{无法}{wu2 fa3}[][HSK 4]
    \definition{adj.}{incapaz; incapacitado}
    \definition{v.}{não há nada a ser feito}
  \end{phonetics}
\end{entry}

\begin{entry}{无视}{4,8}{⽆、⾒}
  \begin{phonetics}{无视}{wu2shi4}
    \definition{v.}{ignorar | desconsiderar}
  \end{phonetics}
\end{entry}

\begin{entry}{无限}{4,8}{⽆、⾩}
  \begin{phonetics}{无限}{wu2 xian4}[][HSK 4]
    \definition{adj.}{infinito; ilimitado; sem limites; sem fim à vista}
  \end{phonetics}
\end{entry}

\begin{entry}{无故}{4,9}{⽆、⽁}
  \begin{phonetics}{无故}{wu2gu4}
    \definition{adv.}{sem causa ou razão | sem motivo}
  \end{phonetics}
\end{entry}

\begin{entry}{无骨}{4,9}{⽆、⾻}
  \begin{phonetics}{无骨}{wu2 gu3}
    \definition{adj.}{desossado}
  \end{phonetics}
\end{entry}

\begin{entry}{无敌}{4,10}{⽆、⾆}
  \begin{phonetics}{无敌}{wu2di2}
    \definition{adj.}{invencível | inigualável}
  \end{phonetics}
\end{entry}

\begin{entry}{无氧}{4,10}{⽆、⽓}
  \begin{phonetics}{无氧}{wu2yang3}
    \definition{adj.}{anaeróbico}
  \end{phonetics}
\end{entry}

\begin{entry}{无聊}{4,11}{⽆、⽿}
  \begin{phonetics}{无聊}{wu2liao2}[][HSK 4]
    \definition{adj.}{entediado; aborrecido; sentir-se desinteressado porque não há nada para fazer | tolo; bobo; sem sentido; descreve palavras ou coisas ditas ou feitas como sem sentido e irritantes; descreve pessoas ou coisas como sem sentido e pouco atraentes}
  \end{phonetics}
\end{entry}

\begin{entry}{无数}{4,13}{⽆、⽁}
  \begin{phonetics}{无数}{wu2shu4}[][HSK 4]
    \definition{adj.}{incontável; inumerável | inseguro; incerto; não conhecer a história ou os detalhes internos; não ter certeza}
  \end{phonetics}
\end{entry}

\begin{entry}{既}{9}{⽆}
  \begin{phonetics}{既}{ji4}[][HSK 4]
    \definition*{s.}{sobrenome Ji}
    \definition{adv.}{já}
    \definition{conj.}{desde; como; agora que | assim como; e também; ambos\dots e\dots; usado em conjunto com advérbios como ``且、又、也'' para indicar uma combinação de ambas as situações}
  \seealsoref{且}{qie3}
  \seealsoref{也}{ye3}
  \seealsoref{又}{you4}
  \end{phonetics}
\end{entry}

\begin{entry}{既又}{9,2}{⽆、⼜}
  \begin{phonetics}{既又}{ji4you4}
    \definition{conj.}{desde | como | agora isso | os dois e | assim como}
  \end{phonetics}
\end{entry}

\begin{entry}{既不……又不……}{9,4,2,4}{⽆、⼀、⼜、⼀}
  \begin{phonetics}{既不……又不……}{ji4bu4 you4bu4}
    \definition{conj.}{nem mesmo\dots}
  \end{phonetics}
\end{entry}

\begin{entry}{既然}{9,12}{⽆、⽕}
  \begin{phonetics}{既然}{ji4ran2}[][HSK 4]
    \definition{conj.}{como; desde; agora que; usado na primeira metade de uma frase, muitas vezes repetido na segunda metade pelos advérbios ``就、也、还'' para indicar que a premissa é primeiro declarada e depois inferida}
  \seealsoref{还}{hai2}
  \seealsoref{就}{jiu4}
  \seealsoref{也}{ye3}
  \end{phonetics}
\end{entry}

\begin{entry}{旣}{11}{⽆}
  \begin{phonetics}{旣}{ji4}
    \variantof{既}
  \end{phonetics}
\end{entry}

%%%%% EOF %%%%%

