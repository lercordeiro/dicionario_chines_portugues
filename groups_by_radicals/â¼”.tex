%%%
%%% Radical "⼔"
%%%

\section*{Radical 21: ``⼔''}\addcontentsline{toc}{section}{Radical 21: ⼔}

\begin{entry}{化}{4}{⼔}
  \begin{phonetics}{化}{hua1}
    \variantof{花}
  \end{phonetics}
  \begin{phonetics}{化}{hua4}[][HSK 3]
    \definition*{s.}{Sobrenome Hua}
    \definition{s.}{química | cultura; costumes e tradições}
    \definition{suf.}{modernizar; modernização; anexado a componentes nominais ou adjetivos para formar verbos, indicando a transformação em algum estado ou qualidade}
    \definition{v.}{mudar; converter; transformar; causasr mudanças | converter; influenciar; influenciar e induzir as pessoas com palavras e ações, levando-as a mudar | derreter; dissolver; fundir | digerir | queimar; reduzir a cinzas | (monge, taoísta) morrer | (de monges budistas ou sacerdotes taoístas) pedir esmolas; arrecadar bens, alimentos, etc.}
    \variantof{花}
  \end{phonetics}
\end{entry}

\begin{entry}{化石}{4,5}{⼔、⽯}
  \begin{phonetics}{化石}{hua4shi2}[][HSK 5]
    \definition{s.}{fóssil; restos, relíquias ou vestígios de organismos antigos enterrados no solo e transformados em objetos semelhantes a pedras}
  \end{phonetics}
\end{entry}

\begin{entry}{化学}{4,8}{⼔、⼦}
  \begin{phonetics}{化学}{hua4xue2}
    \definition{s.}{química (disciplina)}
  \end{phonetics}
\end{entry}

\begin{entry}{北}{5}{⼔}
  \begin{phonetics}{北}{bei3}[][HSK 1]
    \definition*{s.}{Norte (os países desenvolvidos) | Sobrenome Bei}
    \definition{s.}{norte; uma das quatro direções básicas, a esquerda quando se está de frente para o sol pela manhã (oposta ao 南)}
    \definition{v.}{ser derrotado}
  \seealsoref{南}{nan2}
  \end{phonetics}
\end{entry}

\begin{entry}{北大西洋公约组织}{5,3,6,9,4,6,8,8}{⼔、⼤、⾑、⽔、⼋、⽷、⽷、⽷}
  \begin{phonetics}{北大西洋公约组织}{bei3 da4xi1 yang2 gong1 yue1 zu3zhi1}
    \definition*{s.}{Organização do Tratado do Atlântico Norte, OTAN}
  \end{phonetics}
\end{entry}

\begin{entry}{北方}{5,4}{⼔、⽅}
  \begin{phonetics}{北方}{bei3fang1}[][HSK 2]
    \definition{s.}{norte; indicando a direção norte | o Norte; a parte norte da China, especialmente a área ao norte do rio Huang He}
  \end{phonetics}
\end{entry}

\begin{entry}{北边}{5,5}{⼔、⾡}
  \begin{phonetics}{北边}{bei3 bian1}[][HSK 1]
    \definition{s.}{norte; o lado norte}
  \end{phonetics}
\end{entry}

\begin{entry}{北约}{5,6}{⼔、⽷}
  \begin{phonetics}{北约}{bei3yue1}
    \definition*{s.}{OTAN, Organização do Tratado do Atlântico Norte; Abreviação de 北大西洋公约组织}
  \seealsoref{北大西洋公约组织}{bei3 da4xi1 yang2 gong1 yue1 zu3zhi1}
  \end{phonetics}
\end{entry}

\begin{entry}{北极}{5,7}{⼔、⽊}
  \begin{phonetics}{北极}{bei3ji2}[][HSK 5]
    \definition*{s.}{Polo Norte; Polo Ártico}
    \definition{s.}{polo norte magnético; o ponto mais setentrional da Terra, também se refere à região mais setentrional da Terra}
  \end{phonetics}
\end{entry}

\begin{entry}{北京}{5,8}{⼔、⼇}
  \begin{phonetics}{北京}{bei3 jing1}[][HSK 1]
    \definition*{s.}{Pequim (Beijing), Capital da República Popular da China | Capital da China, localizada no nordeste do país, fundada em 700 a.C., a cidade é um importante centro comercial, industrial e cultural}
  \end{phonetics}
\end{entry}

\begin{entry}{北面}{5,9}{⼔、⾯}
  \begin{phonetics}{北面}{bei3mian4}
    \definition{s.}{norte; o lado norte}
  \end{phonetics}
\end{entry}

\begin{entry}{北部}{5,10}{⼔、⾢}
  \begin{phonetics}{北部}{bei3 bu4}[][HSK 3]
    \definition{s.}{parte norte de uma região ou país}
  \end{phonetics}
\end{entry}

%%%%% EOF %%%%%

