%%%
%%% Radical "⾋"
%%%

\section*{Radical 140: ``⾋'' (⺿)}\addcontentsline{toc}{section}{Radical 140: ⾋、⺿}

\begin{entry}{艺}{4}{⾋}
  \begin{phonetics}{艺}{yi4}
    \definition*{s.}{Sobrenome Yi}
    \definition[个,种]{s.}{habilidade | arte | regra; norma | padrão; critério; diretrizes | limite}
    \definition{v.}{plantar; crescer}
  \end{phonetics}
\end{entry}

\begin{entry}{艺人}{4,2}{⾋、⼈}
  \begin{phonetics}{艺人}{yi4ren2}
    \definition{s.}{artista | ator}
  \end{phonetics}
\end{entry}

\begin{entry}{艺术}{4,5}{⾋、⽊}
  \begin{phonetics}{艺术}{yi4shu4}[][HSK 3]
    \definition{adj.}{artístico,único; elegante}
    \definition[个,种,门,项,类]{s.}{arte; literatura e arte | habilidade; arte; ofício; métodos criativos}
  \end{phonetics}
\end{entry}

\begin{entry}{节}{5}{⾋}
  \begin{phonetics}{节}{jie1}
    \definition{adj.}{momento crucial; momento crítico; momento decisivo; metáfora para algo importante, decisivo ou oportuno}
  \end{phonetics}
  \begin{phonetics}{节}{jie2}[][HSK 2]
    \definition*{s.}{Sobrenome Jie}
    \definition{clas.}{nó (kn), velocidade de um barco | para seções, comprimentos}
    \definition[个]{s.}{junta; botão; nó; geralmente se refere à parte da grama ou caule da grama onde as folhas crescem ou à parte onde os galhos e troncos das plantas são conectados | parte; divisão; um trecho de algo interligado; uma parte do todo | festival; feriado; dia memorável; um período de tempo ou um dia com características específicas | item; assunto | castidade; integridade ética e moral | articulação; o local onde os ossos humanos ou animais se conectam | etiqueta; cerimonial | batida; ritmo | registro; documento utilizado na antiguidade para comprovar a identidade | estação do ano | sílaba}
    \definition{v.}{economizar; conservar; poupar | resumir; extrair; retirar uma parte do todo | controlar; restringir; moderar}
  \end{phonetics}
\end{entry}

\begin{entry}{节日}{5,4}{⾋、⽇}
  \begin{phonetics}{节日}{jie2ri4}[][HSK 2]
    \definition[个,种,类]{s.}{festival; feriado; dia de comemoração tradicional; dia comemorativo estabelecido por lei}
  \end{phonetics}
\end{entry}

\begin{entry}{节目}{5,5}{⾋、⽬}
  \begin{phonetics}{节目}{jie2mu4}[][HSK 2]
    \definition[个,场,项,台]{s.}{programa; item (em um programa); programas artísticos ou projetos transmitidos por rádios e televisões}
  \end{phonetics}
\end{entry}

\begin{entry}{节约}{5,6}{⾋、⽷}
  \begin{phonetics}{节约}{jie2yue1}[][HSK 3]
    \definition{adj.}{econômico; sem luxo}
    \definition{v.}{guardar; economizar; usar com moderação; economizar gastos desnecessários}
  \end{phonetics}
\end{entry}

\begin{entry}{节奏}{5,9}{⾋、⼤}
  \begin{phonetics}{节奏}{jie2zou4}[][HSK 6]
    \definition[个,种]{s.}{ritmo; o fenômeno da alternância regular de comprimento, força e fraqueza das notas na música | padrão regular; uma metáfora para um processo de ajuste adequado com tensão e relaxamento}
  \end{phonetics}
\end{entry}

\begin{entry}{节省}{5,9}{⾋、⽬}
  \begin{phonetics}{节省}{jie2sheng3}[][HSK 4]
    \definition{adj.}{econômico; parcimonioso}
    \definition{v.}{economizar; conservar; usar com moderação; reduzir; eliminar ou minimizar o esgotamento de itens potencialmente esgotáveis}
  \end{phonetics}
\end{entry}

\begin{entry}{节能}{5,10}{⾋、⾁}
  \begin{phonetics}{节能}{jie2 neng2}[][HSK 6]
    \definition{v.}{economizar no consumo de energia; conservar energia}
  \end{phonetics}
\end{entry}

\begin{entry}{节假日}{5,11,4}{⾋、⼈、⽇}
  \begin{phonetics}{节假日}{jie2 jia4 ri4}[][HSK 6]
    \definition[个]{s.}{feriados; festivais e feriados}
  \end{phonetics}
\end{entry}

\begin{entry}{芋}{6}{⾋}
  \begin{phonetics}{芋}{yu4}
    \definition*{s.}{Sobrenome Yu}
    \definition{s.}{taro; erva perene | tubérculos; geralmente se refere a batatas, etc.}
  \end{phonetics}
\end{entry}

\begin{entry}{芋头}{6,5}{⾋、⼤}
  \begin{phonetics}{芋头}{yu4tou5}
    \definition{s.}{taro, similar ao inhame e batata doce}
  \end{phonetics}
\end{entry}

\begin{entry}{芋头色}{6,5,6}{⾋、⼤、⾊}
  \begin{phonetics}{芋头色}{yu4tou5se4}
    \definition{s.}{cor lilás}
  \end{phonetics}
\end{entry}

\begin{entry}{芝}{6}{⾋}
  \begin{phonetics}{芝}{zhi1}
    \definition*{s.}{Sobrenome Zhi}
    \definition{s.}{Arcaico: fungo mágico, ganoderma brilhante | Arcaico: raiz de angélica dahuriana}
  \end{phonetics}
\end{entry}

\begin{entry}{芝麻}{6,11}{⾋、⿇}
  \begin{phonetics}{芝麻}{zhi1ma5}
    \definition{s.}{semente de gergelim}
  \end{phonetics}
\end{entry}

\begin{entry}{芥}{7}{⾋}
  \begin{phonetics}{芥}{gai4}
    \definition{s.}{mostarda}
  \seealsoref{芥蓝}{gai4lan2}
  \end{phonetics}
  \begin{phonetics}{芥}{jie4}
    \definition{s.}{mostarda}
  \end{phonetics}
\end{entry}

\begin{entry}{芥兰}{7,5}{⾋、⼋}
  \begin{phonetics}{芥兰}{gai4lan2}
    \variantof{芥蓝}
  \end{phonetics}
  \begin{phonetics}{芥兰}{jie4lan2}
    \definition{s.}{couve}
  \end{phonetics}
\end{entry}

\begin{entry}{芥蓝}{7,13}{⾋、⾋}
  \begin{phonetics}{芥蓝}{gai4lan2}
    \definition{s.}{brócolis chinês | couve chinesa | mostarda}
  \seealsoref{格兰菜}{ge2lan2cai4}
  \end{phonetics}
\end{entry}

\begin{entry}{芦}{7}{⾋}
  \begin{phonetics}{芦}{lu2}
    \definition*{s.}{Sobrenome Lu}
    \definition{s.}{junco}
  \end{phonetics}
\end{entry}

\begin{entry}{芦笋}{7,10}{⾋、⽵}
  \begin{phonetics}{芦笋}{lu2sun3}
    \definition{s.}{aspargos}
  \end{phonetics}
\end{entry}

\begin{entry}{芯}{7}{⾋}
  \begin{phonetics}{芯}{xin1}
    \definition{s.}{medula de junco | pavio}
  \end{phonetics}
  \begin{phonetics}{芯}{xin4}
    \definition{s.}{núcleo; a parte central de um objeto | língua de cobra}
  \end{phonetics}
\end{entry}

\begin{entry}{芯片}{7,4}{⾋、⽚}
  \begin{phonetics}{芯片}{xin1pian4}
    \definition{s.}{\emph{chip} de computador | \emph{microchip}}
  \end{phonetics}
\end{entry}

\begin{entry}{花}{7}{⾋}
  \begin{phonetics}{花}{hua1}[][HSK 1,2,4]
    \definition*{s.}{Sobrenome Hua}
    \definition{adj.}{multicolorido; colorido | embaçado; obscuro; deslumbrado e confuso | extravagante; florido; vistoso}
    \definition[朵,支,束,把,盆,簇]{s.}{flor; órgãos de reprodução sexual de plantas com sementes | flor; planta ornamental |  qualquer coisa que se assemelhe a uma flor | fogos de artifício | padrão; design; design decorativo | flor; metáfora para a essência de uma causa | prostituta; cortesã; referindo-se a prostitutas ou a assuntos relacionados a prostitutas | algodão | varíola | ferimento; ferida; lesões traumáticas sofridas em combate}
    \definition{v.}{gastar; despender; consumir}
  \end{phonetics}
\end{entry}

\begin{entry}{花儿}{7,2}{⾋、⼉}
  \begin{phonetics}{花儿}{hua1r5}
    \definition[朵,支,束,把,盆,簇]{s.}{flor}
  \end{phonetics}
\end{entry}

\begin{entry}{花生}{7,5}{⾋、⽣}
  \begin{phonetics}{花生}{hua1sheng1}[][HSK 6]
    \definition[把,颗,粒,袋]{s.}{amendoim}
  \seealsoref{落花生}{luo4 hua1 sheng1}
  \end{phonetics}
\end{entry}

\begin{entry}{花园}{7,7}{⾋、⼞}
  \begin{phonetics}{花园}{hua1 yuan2}[][HSK 2]
    \definition[个,座]{s.}{jardim; um local onde se plantam flores e árvores para passear e descansar}
  \end{phonetics}
\end{entry}

\begin{entry}{花店}{7,8}{⾋、⼴}
  \begin{phonetics}{花店}{hua1dian4}
    \definition{s.}{floricultura}
  \end{phonetics}
\end{entry}

\begin{entry}{花茶}{7,9}{⾋、⾋}
  \begin{phonetics}{花茶}{hua1cha2}
    \definition[杯,壶]{s.}{chá perfumado}
  \end{phonetics}
\end{entry}

\begin{entry}{花费}{7,9}{⾋、⾙}
  \begin{phonetics}{花费}{hua1 fei4}[][HSK 6]
    \definition[笔]{s.}{dinheiro gasto; despesas | custo; gastos; desembolso | despesa}
    \definition{v.}{gastar (tempo ou dinheiro)}
  \end{phonetics}
\end{entry}

\begin{entry}{花样游泳}{7,10,12,8}{⾋、⽊、⽔、⽔}
  \begin{phonetics}{花样游泳}{hua1yang4you2yong3}
    \definition{s.}{nado sincronizado}
  \end{phonetics}
\end{entry}

\begin{entry}{花瓶}{7,10}{⾋、⽡}
  \begin{phonetics}{花瓶}{hua1 ping2}[][HSK 6]
    \definition[个,对]{s.}{vaso de flores; vaso usado para arranjos florais colocado em ambientes internos como decoração | Figurativo: um ornamento; mulher empregada não por sua habilidade, mas por sua aparência; uma metáfora para uma pessoa ou coisa que é usada apenas para exibição e não tem uso prático}
  \end{phonetics}
\end{entry}

\begin{entry}{花脸}{7,11}{⾋、⾁}
  \begin{phonetics}{花脸}{hua1lian3}
    \definition*{s.}{Hualian, personagem do rosto florido, um nome popular para 净 (assim chamado devido à elaborada pintura facial)}
  \seealsoref{净}{jing4}
  \end{phonetics}
\end{entry}

\begin{entry}{花椰菜}{7,12,11}{⾋、⽊、⾋}
  \begin{phonetics}{花椰菜}{hua1ye1cai4}
    \definition{s.}{couve-flor}
  \end{phonetics}
\end{entry}

\begin{entry}{芹}{7}{⾋}
  \begin{phonetics}{芹}{qin2}
    \definition[把,棵]{s.}{aipo | aipo chinês}
  \end{phonetics}
\end{entry}

\begin{entry}{芹菜}{7,11}{⾋、⾋}
  \begin{phonetics}{芹菜}{qin2cai4}
    \definition{s.}{salsão}
  \end{phonetics}
\end{entry}

\begin{entry}{苏}{7}{⾋}
  \begin{phonetics}{苏}{su1}
    \definition*{s.}{Suzhou, abreviação de 苏州 | Província de Jiangsu, abreviação de 江苏 | União Soviética, abreviação de 苏联 | Sobrenome Su}
    \definition{s.}{perilla planta da família das mentas}
    \definition{v.}{reviver; vir a; acordar}
  \seealsoref{江苏}{jiang1su1}
  \seealsoref{苏联}{su1lian2}
  \seealsoref{苏州}{su1zhou1}
  \end{phonetics}
\end{entry}

\begin{entry}{苏州}{7,6}{⾋、⼮}
  \begin{phonetics}{苏州}{su1zhou1}
    \definition*{s.}{Suzhou, cidade na Província de Jiangsu}
  \end{phonetics}
\end{entry}

\begin{entry}{苏格兰}{7,10,5}{⾋、⽊、⼋}
  \begin{phonetics}{苏格兰}{su1ge2lan2}
    \definition*{s.}{Escócia}
  \end{phonetics}
\end{entry}

\begin{entry}{苏联}{7,12}{⾋、⽿}
  \begin{phonetics}{苏联}{su1lian2}
    \definition*{s.}{União das Repúblicas Socialistas Soviéticas (1922-1991)}
  \end{phonetics}
\end{entry}

\begin{entry}{若}{8}{⾋}
  \begin{phonetics}{若}{ruo4}[][HSK 6]
    \definition*{s.}{Sobrenome Ruo}
    \definition{adv.}{como se; como se fosse; usado antes do verbo para indicar que o que foi dito é mais ou menos assim, equivalente a 好像}
    \definition{conj.}{se; usado na primeira parte de uma frase composta, expressa uma relação hipotética, equivalente a 如果}
    \definition{pron.}{você; referir-se ao interlocutor como 你 ou 你的}
    \definition{v.}{parecer}
  \seealsoref{好像}{hao3xiang4}
  \seealsoref{你}{ni3}
  \seealsoref{你的}{ni3 de5}
  \seealsoref{如果}{ru2guo3}
  \end{phonetics}
\end{entry}

\begin{entry}{苦}{8}{⾋}
  \begin{phonetics}{苦}{ku3}[][HSK 4]
    \definition{adj.}{amargo | difícil; doloroso; sofrido | desgastado; gasto demais}
    \definition{adv.}{meticulosamente; fazendo o máximo possível; de forma árdua; pacientemente}
    \definition{v.}{causar sofrimento a alguém; causar dificuldades a alguém | sofrer com; ser incomodado por; sentir-se angustiado com uma situação}
  \end{phonetics}
\end{entry}

\begin{entry}{苦瓜}{8,5}{⾋、⽠}
  \begin{phonetics}{苦瓜}{ku3gua1}
    \definition{s.}{melão amargo (cabaça amarga, pêra bálsamo, maçã bálsamo, pepino amargo)}
  \end{phonetics}
\end{entry}

\begin{entry}{英}{8}{⾋}
  \begin{phonetics}{英}{ying1}
    \definition*{s.}{Reino Unido, abreviação de 英国 | Sobrenome Ying}
    \definition{s.}{flor | herói; pessoa excepcional | uma pessoa de talento ou sabedoria extraordinários}
  \seealsoref{英国}{ying1guo2}
  \end{phonetics}
\end{entry}

\begin{entry}{英文}{8,4}{⾋、⽂}
  \begin{phonetics}{英文}{ying1 wen2}[][HSK 2]
    \definition{s.}{inglês, língua inglesa; a forma escrita do inglês}
  \end{phonetics}
\end{entry}

\begin{entry}{英国}{8,8}{⾋、⼞}
  \begin{phonetics}{英国}{ying1guo2}
    \definition*{s.}{Reino Unido; Grã-Bretanha; Inglaterra}
  \end{phonetics}
\end{entry}

\begin{entry}{英国人}{8,8,2}{⾋、⼞、⼈}
  \begin{phonetics}{英国人}{ying1guo2ren2}
    \definition{s.}{inglês | pessoa ou povo do Reino Unido}
  \end{phonetics}
\end{entry}

\begin{entry}{英勇}{8,9}{⾋、⼒}
  \begin{phonetics}{英勇}{ying1yong3}[][HSK 4]
    \definition{adj.}{heroico; valente; bravo; corajoso; extraordinariamente corajoso}
  \end{phonetics}
\end{entry}

\begin{entry}{英语}{8,9}{⾋、⾔}
  \begin{phonetics}{英语}{ying1 yu3}[][HSK 2]
    \definition{s.}{inglês, língua inglesa}
  \end{phonetics}
\end{entry}

\begin{entry}{英雄}{8,12}{⾋、⾫}
  \begin{phonetics}{英雄}{ying1xiong2}
    \definition[个]{s.}{herói}
  \end{phonetics}
\end{entry}

\begin{entry}{苹}{8}{⾋}
  \begin{phonetics}{苹}{ping2}
    \definition[个]{s.}{uma espécie de artemísia | maçã | lentilha-d'água}
  \end{phonetics}
\end{entry}

\begin{entry}{苹果}{8,8}{⾋、⽊}
  \begin{phonetics}{苹果}{ping2guo3}[][HSK 3]
    \definition[个,斤,筐,箱,棵,种]{s.}{maçã}
  \end{phonetics}
\end{entry}

\begin{entry}{茄}{8}{⾋}
  \begin{phonetics}{茄}{jia1}
    \definition{s.}{caracter fonético usado em empréstimos linguísticos para o som "jia", embora 夹 seja mais comum}
  \seealsoref{夹}{jia1}
  \end{phonetics}
  \begin{phonetics}{茄}{qie2}
    \definition[只]{s.}{berinjela}
  \end{phonetics}
\end{entry}

\begin{entry}{茄子}{8,3}{⾋、⼦}
  \begin{phonetics}{茄子}{qie2 zi5}[][HSK 6]
    \definition{interj.}{Onomatopéia: ``xis'' fonético (ao ser fotografado), equivale ao ``diga xis''}
    \definition[个,根]{s.}{berinjela (fruto e planta)}
  \end{phonetics}
\end{entry}

\begin{entry}{茅}{8}{⾋}
  \begin{phonetics}{茅}{mao2}
    \definition*{s.}{Sobrenome Mao}
    \definition[座]{s.}{capim-cogon | planta semelhante ao capim-cogon (como palha)}
  \end{phonetics}
\end{entry}

\begin{entry}{茅厕}{8,8}{⾋、⼚}
  \begin{phonetics}{茅厕}{mao2ce4}
    \definition{s.}{(dialeto) latrina}
  \end{phonetics}
\end{entry}

\begin{entry}{范}{9}{⾋}
  \begin{phonetics}{范}{fan4}
    \definition*{s.}{Sobrenome Fan}
    \definition{s.}{padrão; molde; matriz | modelo; exemplo; modelo a seguir | limites; escopo | restrição; limite}
  \end{phonetics}
\end{entry}

\begin{entry}{范围}{9,7}{⾋、⼞}
  \begin{phonetics}{范围}{fan4wei2}[][HSK 3]
    \definition[个]{s.}{escopo; limite; alcance}
    \definition{v.}{estabelecer limites para; limitar o escopo de}
  \end{phonetics}
\end{entry}

\begin{entry}{茶}{9}{⾋}
  \begin{phonetics}{茶}{cha2}[][HSK 1]
    \definition{adj.}{moreno; fulvo; amarelo-acastanhado}
    \definition[杯,壶]{s.}{chá (a bebida); bebida feita com folhas de chá | chá (a planta) | certos tipos de bebidas ou alimentos líquidos | árvore de chá-de-óleo | camélia}
  \end{phonetics}
\end{entry}

\begin{entry}{茶叶}{9,5}{⾋、⼝}
  \begin{phonetics}{茶叶}{cha2 ye4}[][HSK 4]
    \definition[盒,罐,包,片]{s.}{chá; folhas de chá; as folhas jovens da planta do chá que são processadas para produzir bebidas}
  \end{phonetics}
\end{entry}

\begin{entry}{草}{9}{⾋}
  \begin{phonetics}{草}{cao3}[][HSK 2]
    \definition*{s.}{Sobrenome Cao}
    \definition{adj.}{descuidado; rude | rascunho; inicial | femea; na linguagem coloquial, refere-se a animais domésticos e aves fêmeas | precipitado; pouco cuidadoso | rascunho; não definitivo; preliminar; informal}
    \definition[种,棵,撮,株,根]{s.}{grama; gramado | palha | campo; zona rural; área selvagem | letra cursiva | letra cursiva (ou caligráfica) de um alfabeto fonético | rascunho | caligrafia cursiva; um tipo de escrita chinesa}
    \definition{v.}{esboçar; redigir}
  \end{phonetics}
\end{entry}

\begin{entry}{草地}{9,6}{⾋、⼟}
  \begin{phonetics}{草地}{cao3 di4}[][HSK 2]
    \definition[片,块]{s.}{prado; gramado; campo; pastagem ou grande área de terra plantada com pastagem | gramado; relvado; local com grama alta ou gramado}
  \end{phonetics}
\end{entry}

\begin{entry}{草纸}{9,7}{⾋、⽷}
  \begin{phonetics}{草纸}{cao3zhi3}
    \definition{s.}{papel pardo | pergaminho | papel de palha áspero | papel higiênico}
  \end{phonetics}
\end{entry}

\begin{entry}{草原}{9,10}{⾋、⼚}
  \begin{phonetics}{草原}{cao3 yuan2}[][HSK 5]
    \definition[片,个]{s.}{estepe; pradaria; grandes áreas de terra coberta de vegetação em áreas semiáridas, intercaladas com árvores tolerantes à seca}
  \end{phonetics}
\end{entry}

\begin{entry}{草莓}{9,10}{⾋、⾋}
  \begin{phonetics}{草莓}{cao3mei2}
    \definition[颗]{s.}{morango}
  \end{phonetics}
\end{entry}

\begin{entry}{荒}{9}{⾋}
  \begin{phonetics}{荒}{huang1}
    \definition*{s.}{Sobrenome Huang}
    \definition{adj.}{(terra) não utilizada; não cultivada | desolado; estéril | irracional; delirante; fantástico; absurdo | incerto; duvidoso | dissoluto; autoindulgente | grosseiramente processado; bruto}
    \definition[片,块]{s.}{terra devastada; terra inculta; deserto | fome; quebra de safra | escassez | lixo; restos | terra selvagem (floresta)}
    \definition{v.}{(coloquial) negligenciar; estar fora de prática}
  \end{phonetics}
\end{entry}

\begin{entry}{荒芜}{9,7}{⾋、⾋}
  \begin{phonetics}{荒芜}{huang1wu2}
    \definition{adj.}{estéril}
  \end{phonetics}
\end{entry}

\begin{entry}{荔}{9}{⾋}
  \begin{phonetics}{荔}{li4}
    \definition[颗]{s.}{lichia | (arcaico) uma espécie de grama semelhante à taboa}
  \end{phonetics}
\end{entry}

\begin{entry}{荔枝}{9,8}{⾋、⽊}
  \begin{phonetics}{荔枝}{li4zhi1}
    \definition{s.}{lichia}
  \end{phonetics}
\end{entry}

\begin{entry}{荣}{9}{⾋}
  \begin{phonetics}{荣}{rong2}
    \definition*{s.}{Sobrenome Rong}
    \definition{adj.}{próspero; florescente | exuberante | glorioso}
    \definition{s.}{honra; glória (oposto a 辱) | guarda-sol chinês | flor; flor de planta herbácea | beirais virados para cima}
    \definition{v.}{glorificar; luxuriar; crescer abundantemente; florescer | florescer | lançar}
  \seealsoref{辱}{ru3}
  \end{phonetics}
\end{entry}

\begin{entry}{荤}{9}{⾋}[Kangxi 9]
  \begin{phonetics}{荤}{hun1}
    \definition{adj.}{obsceno; lascivo; vulgar}
    \definition{s.}{carne ou peixe (oposto a 素) | Budismo: vegetais picantes proibidos aos vegetarianos budistas, como cebola, alho-poró, alho, etc. | alimentos não vegetarianos (carne, peixe etc.) | vegetais com cheiro forte (alho etc.)}
  \seealsoref{素}{su4}
  \end{phonetics}
\end{entry}

\begin{entry}{药}{9}{⾋}
  \begin{phonetics}{药}{yao4}[][HSK 2]
    \definition*{s.}{Sobrenome Yao}
    \definition[片,粒,颗,瓶,服]{s.}{droga; loção; remédio; medicamento; substâncias que podem prevenir e tratar doenças, pragas ou melhorar funções corporais | certos produtos químicos com efeitos específicos}
    \definition{v.}{curar com remédios; tomar remédios para tratar doenças | matar com veneno; envenenar}
  \end{phonetics}
\end{entry}

\begin{entry}{药丸}{9,3}{⾋、⼂}
  \begin{phonetics}{药丸}{yao4wan2}
    \definition[粒]{s.}{pílula}
  \end{phonetics}
\end{entry}

\begin{entry}{药水}{9,4}{⾋、⽔}
  \begin{phonetics}{药水}{yao4 shui3}[][HSK 2]
    \definition*{s.}{Yaksu na Coreia do Norte, perto da fronteira com Liaoning e a província de Jilin}
    \definition{s.}{medicamento líquido; líquido medicinal | loção | remédio engarrafado | medicamento em forma líquida}
  \end{phonetics}
\end{entry}

\begin{entry}{药片}{9,4}{⾋、⽚}
  \begin{phonetics}{药片}{yao4 pian4}[][HSK 2]
    \definition[颗,片]{s.}{pílula; comprimido; preparações em comprimidos}
  \end{phonetics}
\end{entry}

\begin{entry}{药补}{9,7}{⾋、⾐}
  \begin{phonetics}{药补}{yao4bu3}
    \definition{s.}{suplemento dietético medicinal que ajuda a melhorar a saúde}
  \end{phonetics}
\end{entry}

\begin{entry}{药典}{9,8}{⾋、⼋}
  \begin{phonetics}{药典}{yao4dian3}
    \definition{s.}{farmacopéia}
  \end{phonetics}
\end{entry}

\begin{entry}{药店}{9,8}{⾋、⼴}
  \begin{phonetics}{药店}{yao4 dian4}[][HSK 2]
    \definition[家]{s.}{farmácia; drogaria; lojas que vendem medicamentos}
  \end{phonetics}
\end{entry}

\begin{entry}{药房}{9,8}{⾋、⼾}
  \begin{phonetics}{药房}{yao4fang2}
    \definition{s.}{farmácia | drogaria}
  \end{phonetics}
\end{entry}

\begin{entry}{药物}{9,8}{⾋、⽜}
  \begin{phonetics}{药物}{yao4 wu4}[][HSK 4]
    \definition{s.}{droga; medicamento; remédio; substâncias que controlam doenças, pragas, etc.}
  \end{phonetics}
\end{entry}

\begin{entry}{药品}{9,9}{⾋、⼝}
  \begin{phonetics}{药品}{yao4pin3}
    \definition{s.}{medicamento | remédio | droga}
  \end{phonetics}
\end{entry}

\begin{entry}{药签}{9,13}{⾋、⽵}
  \begin{phonetics}{药签}{yao4qian1}
    \definition{s.}{cotonete médico}
  \end{phonetics}
\end{entry}

\begin{entry}{药膳}{9,16}{⾋、⾁}
  \begin{phonetics}{药膳}{yao4shan4}
    \definition{s.}{dieta medicinal}
  \end{phonetics}
\end{entry}

\begin{entry}{药罐}{9,23}{⾋、⽸}
  \begin{phonetics}{药罐}{yao4guan4}
    \definition{s.}{frasco de remédio}
  \end{phonetics}
\end{entry}

\begin{entry}{荷}{10}{⾋}
  \begin{phonetics}{荷}{he2}
    \definition*{s.}{Países Baixos; Holanda, abreviação de 荷兰 | Sobrenome He}
    \definition{s.}{lótus}
  \seealsoref{荷兰}{he2lan2}
  \end{phonetics}
  \begin{phonetics}{荷}{he4}
    \definition{s.}{fardo; responsabilidade}
    \definition{v.}{carregar no ombro ou nas costas | aceitar um favor, frequentemente usado em cartas para expressar cortesia}
  \end{phonetics}
\end{entry}

\begin{entry}{荷兰}{10,5}{⾋、⼋}
  \begin{phonetics}{荷兰}{he2lan2}
    \definition*{s.}{Países Baixos; Holanda}
  \end{phonetics}
\end{entry}

\begin{entry}{荷花}{10,7}{⾋、⾋}
  \begin{phonetics}{荷花}{he2hua1}
    \definition{s.}{lótus}
  \end{phonetics}
\end{entry}

\begin{entry}{莎}{10}{⾋}
  \begin{phonetics}{莎}{sha1}
    \definition{s.}{em nomes pessoais e de lugares | cigarra | fonético "sha" usado na transliteração}
  \end{phonetics}
  \begin{phonetics}{莎}{suo1}
  \end{phonetics}
\end{entry}

\begin{entry}{莎莎舞}{10,10,14}{⾋、⾋、⾇}
  \begin{phonetics}{莎莎舞}{sha1sha1wu3}
    \definition{s.}{salsa (dança)}
  \end{phonetics}
\end{entry}

\begin{entry}{莫}{10}{⾋}
  \begin{phonetics}{莫}{mo4}
    \definition*{s.}{Sobrenome Mo}
    \definition{adv.}{não, frequentemente usado em frases imperativas | não; não pode | pode ser que; não pode ser que; é possível que}
    \definition{pron.}{nenhum; nada; ninguém; significa 没有谁 ou 没有哪一种东西}
  \seealsoref{没有哪一种东西}{mei2you3 na3 yi4 zhong3 dong1xi1}
  \seealsoref{没有谁}{mei2you3 shei2}
  \end{phonetics}
\end{entry}

\begin{entry}{莫名其妙}{10,6,8,7}{⾋、⼝、⼋、⼥}
  \begin{phonetics}{莫名其妙}{mo4ming2qi2miao4}
    \definition{adj.}{desconcertante | bizzaro | inexplicável | perplexo}
  \end{phonetics}
\end{entry}

\begin{entry}{莫非}{10,8}{⾋、⾮}
  \begin{phonetics}{莫非}{mo4fei1}
    \definition{expr.}{Não é mesmo?; é frequentemente usado com 不成}
    \definition{v.}{pode ser que; é possível que}
  \seealsoref{不成}{bu4 cheng2}
  \end{phonetics}
\end{entry}

\begin{entry}{莲}{10}{⾋}
  \begin{phonetics}{莲}{lian2}
    \definition*{s.}{Sobrenome Lian}
    \definition[粒]{s.}{lótus}
  \end{phonetics}
\end{entry}

\begin{entry}{莲花}{10,7}{⾋、⾋}
  \begin{phonetics}{莲花}{lian2hua1}
    \definition{s.}{flor de lótus | lírio aquático}
  \end{phonetics}
\end{entry}

\begin{entry}{莲藕}{10,18}{⾋、⾋}
  \begin{phonetics}{莲藕}{lian2'ou3}
    \definition{s.}{raiz de Lotus}
  \end{phonetics}
\end{entry}

\begin{entry}{获}{10}{⾋}
  \begin{phonetics}{获}{huo4}[][HSK 4]
    \definition*{s.}{Sobrenome Huo}
    \definition{v.}{capturar; pegar | obter; ganhar; colher | colher; ceifar}
  \end{phonetics}
\end{entry}

\begin{entry}{获取}{10,8}{⾋、⼜}
  \begin{phonetics}{获取}{huo4 qu3}[][HSK 4]
    \definition{v.}{adquirir; obter; ganhar; colher}
  \end{phonetics}
\end{entry}

\begin{entry}{获奖}{10,9}{⾋、⼤}
  \begin{phonetics}{获奖}{huo4 jiang3}[][HSK 4]
    \definition{v.}{ganhar prêmio; ser recompensado; ganhar um prêmio; receber um prêmio}
  \end{phonetics}
\end{entry}

\begin{entry}{获得}{10,11}{⾋、⼻}
  \begin{phonetics}{获得}{huo4de2}[][HSK 4]
    \definition{v.}{adquirir; ganhar; obter; alcançar}
  \end{phonetics}
\end{entry}

\begin{entry}{菜}{11}{⾋}
  \begin{phonetics}{菜}{cai4}[][HSK 1]
    \definition*{s.}{Sobrenome Cai}
    \definition{adj.}{pouca habilidade; baixo nível; baixa capacidade}
    \definition[棵,个,道]{s.}{legumes; verduras; plantas que podem ser usadas como alimentos complementares | óleo de canola | prato; item ou prato do cardápio (seja de carne ou de vegetais)}
  \end{phonetics}
\end{entry}

\begin{entry}{菜刀}{11,2}{⾋、⼑}
  \begin{phonetics}{菜刀}{cai4dao1}
    \definition[把]{s.}{faca de vegetais | faca de cozinha | cutelo}
  \end{phonetics}
\end{entry}

\begin{entry}{菜单}{11,8}{⾋、⼗}
  \begin{phonetics}{菜单}{cai4dan1}[][HSK 2]
    \definition[个,分,张]{s.}{menu; lista de pratos | menu (para computadores); lista utilizada para selecionar várias operações diferentes}
  \end{phonetics}
\end{entry}

\begin{entry}{菠}{11}{⾋}
  \begin{phonetics}{菠}{bo1}
    \definition{s.}{espinafre}
  \end{phonetics}
\end{entry}

\begin{entry}{菠菜}{11,11}{⾋、⾋}
  \begin{phonetics}{菠菜}{bo1cai4}
    \definition[棵]{s.}{espinafre}
  \end{phonetics}
\end{entry}

\begin{entry}{菱}{11}{⾋}
  \begin{phonetics}{菱}{ling2}
    \definition{s.}{maruca; caltrop aquático; castanha d'água}
  \end{phonetics}
\end{entry}

\begin{entry}{菱角}{11,7}{⾋、⾓}
  \begin{phonetics}{菱角}{ling2jiao5}
    \definition{s.}{castanha d'água}
  \end{phonetics}
\end{entry}

\begin{entry}{营}{11}{⾋}
  \begin{phonetics}{营}{ying2}
    \definition*{s.}{Sobrenome Ying}
    \definition{s.}{acampamento; quartel; onde o exército está estacionado | batalhão; unidades militares}
    \definition{v.}{procurar | operar; executar; gerenciar}
  \end{phonetics}
\end{entry}

\begin{entry}{营业}{11,5}{⾋、⼀}
  \begin{phonetics}{营业}{ying2ye4}[][HSK 4]
    \definition{v.}{fazer negócios; estar aberto para negócios}
  \end{phonetics}
\end{entry}

\begin{entry}{营养}{11,9}{⾋、⼋}
  \begin{phonetics}{营养}{ying2yang3}[][HSK 3]
    \definition[种]{s.}{nutrição; alimentação; a função do organismo de absorver as substâncias necessárias do meio externo para manter atividades vitais, como crescimento e desenvolvimento | nutrição; alimentação; ato ou processo de fornecer nutrição}
  \end{phonetics}
\end{entry}

\begin{entry}{落}{12}{⾋}
  \begin{phonetics}{落}{la4}[][HSK 5]
    \definition{v.}{deixar de fora; estar ausente | deixar para trás; esquecer de trazer; deixar algo em algum lugar e esquecer de levar| ficar para trás (ou cair); não conseguir acompanhar}
  \end{phonetics}
  \begin{phonetics}{落}{lao4}
    \definition{v.}{cair; cair de uma altura elevada | se abaixar; descer; ir para baixo | permanecer; fazer uma parada; deixar para trás | obter; ter; receber}
  \end{phonetics}
  \begin{phonetics}{落}{luo4}[][HSK 4]
    \definition*{s.}{Sobrenome Luo}
    \definition{s.}{paradeiro; lugar para ficar; local de descanso | assentamento; local de reunião | parte curta; área pequena; refere-se a um pequeno lugar ou área}
    \definition{v.}{cair; cair de uma altura elevada | se abaixar; descer; ir para baixo | abaixar; deixar cair (ou descer); fazer descer | afundar; declinar; descer | ficar para trás; ficar para trás ou ficar de fora | permanecer; fazer uma parada; deixar para trás | cair sobre; repousar com | obter; ter; receber | anotar; escrever no papel | cair em; entrar em; ficar preso}
  \end{phonetics}
\end{entry}

\begin{entry}{落日}{12,4}{⾋、⽇}
  \begin{phonetics}{落日}{luo4ri4}
    \definition{s.}{pôr do sol}
  \end{phonetics}
\end{entry}

\begin{entry}{落后}{12,6}{⾋、⼝}
  \begin{phonetics}{落后}{luo4hou4}[][HSK 3]
    \definition{adj.}{atrasado; trabalho em atraso, nível de desenvolvimento ou grau de reconhecimento (em oposição a 进步)}
    \definition{v.}{ficar para trás; ficar atrasado; ficar para trás em relação aos outros durante o avanço ou o progresso do trabalho}
  \seealsoref{进步}{jin4bu4}
  \end{phonetics}
\end{entry}

\begin{entry}{落汤鸡}{12,6,7}{⾋、⽔、⿃}
  \begin{phonetics}{落汤鸡}{luo4tang1ji1}
    \definition{s.}{uma pessoa que parece encharcada e acamada| sofrimento profundo}
  \end{phonetics}
\end{entry}

\begin{entry}{落花生}{12,7,5}{⾋、⾋、⽣}
  \begin{phonetics}{落花生}{luo4 hua1 sheng1}
    \definition{s.}{amendoim | noz de macaco}
  \end{phonetics}
\end{entry}

\begin{entry}{落实}{12,8}{⾋、⼧}
  \begin{phonetics}{落实}{luo4shi2}[][HSK 5]
    \definition{adj.}{sentimento de tranquilidade; (humor) estável; seguro}
    \definition{v.}{implementar; ser praticável; tornar os planos, políticas, medidas, etc. específicos e compreensíveis, de modo a que possam ser realizados | implementar; colocar em prática; pôr em prática significa que os planos, políticas e medidas são específicos e claros, e podem ser realizados}
  \end{phonetics}
\end{entry}

\begin{entry}{葡}{12}{⾋}
  \begin{phonetics}{葡}{pu2}
    \definition*{s.}{Portugal, abreviação de 葡萄牙}
  \seealsoref{葡萄牙}{pu2tao2ya2}
  \end{phonetics}
\end{entry}

\begin{entry}{葡文}{12,4}{⾋、⽂}
  \begin{phonetics}{葡文}{pu2wen2}
    \definition{s.}{português, língua portuguesa}
  \seealsoref{葡萄牙文}{pu2tao2ya2wen2}
  \end{phonetics}
\end{entry}

\begin{entry}{葡汉词典}{12,5,7,8}{⾋、⽔、⾔、⼋}
  \begin{phonetics}{葡汉词典}{pu2-han4 ci2dian3}
    \definition{s.}{dicionário português-chinês}
  \seealsoref{汉葡词典}{han4-pu2 ci2dian3}
  \end{phonetics}
\end{entry}

\begin{entry}{葡语}{12,9}{⾋、⾔}
  \begin{phonetics}{葡语}{pu2yu3}
    \definition{s.}{português, língua portuguesa}
  \seealsoref{葡萄牙语}{pu2tao2ya2yu3}
  \end{phonetics}
\end{entry}

\begin{entry}{葡萄}{12,11}{⾋、⾋}
  \begin{phonetics}{葡萄}{pu2tao5}[][HSK 5]
    \definition[棵,串]{s.}{parreira | uva}
  \end{phonetics}
\end{entry}

\begin{entry}{葡萄牙}{12,11,4}{⾋、⾋、⽛}
  \begin{phonetics}{葡萄牙}{pu2tao2ya2}
    \definition{s.}{Portugal}
  \end{phonetics}
\end{entry}

\begin{entry}{葡萄牙文}{12,11,4,4}{⾋、⾋、⽛、⽂}
  \begin{phonetics}{葡萄牙文}{pu2tao2ya2wen2}
    \definition{s.}{português, língua portuguesa}
  \seealsoref{葡文}{pu2wen2}
  \end{phonetics}
\end{entry}

\begin{entry}{葡萄牙语}{12,11,4,9}{⾋、⾋、⽛、⾔}
  \begin{phonetics}{葡萄牙语}{pu2tao2ya2yu3}
    \definition{s.}{português, língua portuguesa}
  \seealsoref{葡语}{pu2yu3}
  \end{phonetics}
\end{entry}

\begin{entry}{葡萄酒}{12,11,10}{⾋、⾋、⾣}
  \begin{phonetics}{葡萄酒}{pu2 tao2 jiu3}[][HSK 5]
    \definition[瓶]{s.}{vinho (de uva)}
  \end{phonetics}
\end{entry}

\begin{entry}{葫}{12}{⾋}
  \begin{phonetics}{葫}{hu2}
    \definition{s.}{cabaça}
  \end{phonetics}
\end{entry}

\begin{entry}{葫芦}{12,7}{⾋、⾋}
  \begin{phonetics}{葫芦}{hu2lu5}
    \definition{adj.}{confuso}
    \definition{s.}{cabaça | termo genérico para bloco e equipamento (ou partes dele)}
  \end{phonetics}
\end{entry}

\begin{entry}{葬}{12}{⾋}
  \begin{phonetics}{葬}{zang4}
    \definition{v.}{enterrar (os mortos) | sepultar}
  \end{phonetics}
\end{entry}

\begin{entry}{葱}{12}{⾋}
  \begin{phonetics}{葱}{cong1}
    \definition{adj.}{verde; turquesa}
    \definition[根,把,捆]{s.}{cebola; cebolinha}
  \end{phonetics}
\end{entry}

\begin{entry}{葵}{12}{⾋}
  \begin{phonetics}{葵}{kui2}
    \definition*{s.}{Sobrenome Kui}
    \definition[朵]{s.}{certas ervas de flores grandes}
  \end{phonetics}
\end{entry}

\begin{entry}{葵花}{12,7}{⾋、⾋}
  \begin{phonetics}{葵花}{kui2hua1}
    \definition{s.}{girassol (flor)}
  \end{phonetics}
\end{entry}

\begin{entry}{蒙}{13}{⾋}
  \begin{phonetics}{蒙}{meng1}[][HSK 6]
    \definition{adj.}{inconsciente; sem sentido;  em coma | confuso; perplexo}
    \definition{v.}{enganar; enganar; trapacear; iludir; trair | fazer um palpite ousado; dar um palpite ousado; arriscar-se}
  \end{phonetics}
  \begin{phonetics}{蒙}{meng2}[][HSK 6]
    \definition*{s.}{Sobrenome Meng}
    \definition{adj.}{ignorância; analfabetismo; falta de instrução | nebuloso; aparência pequena e pouco clara, como chuva ou neblina}
    \definition{s.}{aberto; inicial}
    \definition{v.}{cobrir; espalhar | receber apoio | receber; encontrar-se com; encontrar-se; palavras respeitosas; expressam os benefícios recebidos de outros | sofrer; incorrer}
  \end{phonetics}
  \begin{phonetics}{蒙}{meng3}
    \definition{s.}{grupo étnico mongol; mongol}
  \end{phonetics}
\end{entry}

\begin{entry}{蒙面}{13,9}{⾋、⾯}
  \begin{phonetics}{蒙面}{meng2mian4}
    \definition{adj.}{descarado | desavergonhado | mascarado}
    \definition{v.}{cobrir o rosto | usar uma máscara}
  \end{phonetics}
\end{entry}

\begin{entry}{蓝}{13}{⾋}
  \begin{phonetics}{蓝}{lan2}[][HSK 2]
    \definition*{s.}{Sobrenome Lan}
    \definition{adj.}{azul}
    \definition{s.}{planta índigo; anil | plantas azuis; refere-se a certas plantas que podem ser usadas como corante azul ou certas plantas cujas folhas são azul-esverdeadas}
  \end{phonetics}
\end{entry}

\begin{entry}{蓝色}{13,6}{⾋、⾊}
  \begin{phonetics}{蓝色}{lan2 se4}[][HSK 2]
    \definition[抹,片,缕,团,块]{s.}{cor azul}
  \end{phonetics}
\end{entry}

\begin{entry}{蓝领}{13,11}{⾋、⾴}
  \begin{phonetics}{蓝领}{lan2 ling3}[][HSK 6]
    \definition[名,位,个]{s.}{trabalhador braçal}
  \end{phonetics}
\end{entry}

\begin{entry}{蔓}{14}{⾋}
  \begin{phonetics}{蔓}{man2}
    \definition{s.}{couve-chinesa | nabo}
  \end{phonetics}
  \begin{phonetics}{蔓}{man4}
    \definition{s.}{uma videira com gavinhas; caule fino que não consegue ficar em pé}
    \definition{v.}{rastejar; espalhar; estender}
  \end{phonetics}
  \begin{phonetics}{蔓}{wan4}
    \definition*{s.}{Sobrenome Wan}
    \definition{s.}{uma videira com gavinhas; caule fino que não consegue ficar em pé}
  \end{phonetics}
\end{entry}

\begin{entry}{蔓草}{14,9}{⾋、⾋}
  \begin{phonetics}{蔓草}{man4cao3}
    \definition{s.}{videira | trepadeira}
  \end{phonetics}
\end{entry}

\begin{entry}{箱}{15}{⾋}
  \begin{phonetics}{箱}{xiang1}[][HSK 4]
    \definition{s.}{caixa; estojo; baú | qualquer coisa no formato de caixa}
  \end{phonetics}
\end{entry}

\begin{entry}{箱子}{15,3}{⾋、⼦}
  \begin{phonetics}{箱子}{xiang1 zi5}[][HSK 4]
    \definition[个,只]{s.}{baú; caixa; estojo; maleta; pasta executiva}
  \end{phonetics}
\end{entry}

\begin{entry}{蔬}{15}{⾋}
  \begin{phonetics}{蔬}{shu1}
    \definition{s.}{vegetais}
  \end{phonetics}
\end{entry}

\begin{entry}{蔬菜}{15,11}{⾋、⾋}
  \begin{phonetics}{蔬菜}{shu1cai4}[][HSK 5]
    \definition[样,种]{s.}{verduras; legumes; vegetais; ervas que podem ser usadas na culinária}
  \end{phonetics}
\end{entry}

\begin{entry}{蕃}{15}{⾋}
  \begin{phonetics}{蕃}{bo1}
    \definition[种]{s.}{estrangeiros}
  \end{phonetics}
  \begin{phonetics}{蕃}{fan1}
    \definition[种]{s.}{estrangeiros; aborígenes}
  \end{phonetics}
  \begin{phonetics}{蕃}{fan2}
    \definition{adj.}{exuberante; próspero}
    \definition{v.}{multiplicar; proliferar}
  \end{phonetics}
\end{entry}

\begin{entry}{蕃茄}{15,8}{⾋、⾋}
  \begin{phonetics}{蕃茄}{fan1 qie2}
    \variantof{番茄}
  \end{phonetics}
\end{entry}

\begin{entry}{蕹}{16}{⾋}
  \begin{phonetics}{蕹}{weng4}
    \definition{s.}{espinafre-d’água ou \emph{ong choy}, usado como vegetal no sul da China e no sudeste da Ásia}
  \end{phonetics}
\end{entry}

\begin{entry}{蕹菜}{16,11}{⾋、⾋}
  \begin{phonetics}{蕹菜}{weng4cai4}
    \definition{s.}{espinafre aquático | \emph{ong choy} | repolho do pântano | convolvulus aquático | glória-da-manhã aquática}
  \seealsoref{空心菜}{kong1xin1cai4}
  \end{phonetics}
\end{entry}

\begin{entry}{薄}{16}{⾋}
  \begin{phonetics}{薄}{bao2}[][HSK 4]
    \definition{adj.}{fino; frágil | frio; indiferente; carente de calor | leve; fraco | pobre; infértil}
  \end{phonetics}
  \begin{phonetics}{薄}{bo2}
    \definition*{s.}{Sobrenome Bo}
    \definition{adj.}{pequeno; leve; magro | mau; cruel; mesquinho | frívolo; fútil; não solene | fraco; frágil}
    \definition{v.}{desprezar; tratar com desprezo; menosprezar | aproximar-se}
  \end{phonetics}
  \begin{phonetics}{薄}{bo4}
    \definition{s.}{menta; uma erva perene com aroma refrescante nos caules e folhas}
  \end{phonetics}
\end{entry}

\begin{entry}{薄弱}{16,10}{⾋、⼸}
  \begin{phonetics}{薄弱}{bo2ruo4}[][HSK 5]
    \definition{adj.}{fraco; frágil}
  \end{phonetics}
\end{entry}

\begin{entry}{薯}{16}{⾋}
  \begin{phonetics}{薯}{shu3}
    \definition{s.}{batata | inhame}
  \end{phonetics}
\end{entry}

\begin{entry}{薯片}{16,4}{⾋、⽚}
  \begin{phonetics}{薯片}{shu3 pian4}[][HSK 6]
    \definition{s.}{batatas fritas (\emph{chips}); batatas fritas crocantes ; flocos finos feitos de batatas}
  \end{phonetics}
\end{entry}

\begin{entry}{薯条}{16,7}{⾋、⽊}
  \begin{phonetics}{薯条}{shu3 tiao2}[][HSK 6]
    \definition{s.}{batatas fritas (palito)}
  \end{phonetics}
\end{entry}

\begin{entry}{藏}{17}{⾋}
  \begin{phonetics}{藏}{cang2}[][HSK 6]
    \definition*{s.}{Sobrenome Cang}
    \definition{v.}{esconder; ocultar; esconder da vista | armazenar; coletar; colocar de lado}
  \end{phonetics}
  \begin{phonetics}{藏}{zang4}
    \definition*{s.}{Escrituras budistas ou taoístas; um termo geral para clássicos budistas ou taoístas | Região Autônoma do Tibete, 西藏}
    \definition{s.}{depósito; local de armazenamento; armazém; local onde grandes quantidades de coisas são armazenadas | os tibetanos, 藏族; grupo étnico Zang (ou tibetano)}
  \seealsoref{西藏}{xi1zang4}
  \seealsoref{藏族}{zang4zu2}
  \end{phonetics}
\end{entry}

\begin{entry}{藏族}{17,11}{⾋、⽅}
  \begin{phonetics}{藏族}{zang4zu2}
    \definition*{s.}{Etnia Zang (ou tibetana); Os Zangs (ou tibetanos) , distribuídos pela Região Autônoma do Tibete e pelas províncias de Qinghai, Sichuan, Gansu e Yunnan}
  \end{phonetics}
\end{entry}

\begin{entry}{蘑}{19}{⾋}
  \begin{phonetics}{蘑}{mo2}
    \definition{s.}{cogumelo}
  \end{phonetics}
\end{entry}

\begin{entry}{蘑菇}{19,11}{⾋、⾋}
  \begin{phonetics}{蘑菇}{mo2gu5}
    \definition{s.}{cogumelo}
    \definition{v.}{mandriar | embromar | amofinar | incomodar alguém com solicitações ou interrupções frequentes ou persistentes}
  \end{phonetics}
\end{entry}

%%%%% EOF %%%%%

