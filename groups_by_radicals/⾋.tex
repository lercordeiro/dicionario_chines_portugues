%%%
%%% Radical "⾋"
%%%
\section*{Radical 140: ``⾋'' (⺿)}\addcontentsline{toc}{section}{Radical 140: ⾋、⺿}\addcontentsline{loh}{figure}{\#\#\#\# 140: ⾋}

%%%%%%%%%% 艺 %%%%%%%%%%
\subsection*{艺}\addcontentsline{loh}{figure}{艺}

\begin{Entry}{艺}{4}{⾋}
  \begin{Phonetics}{艺}{yi4}
    \definition*{s.}{Sobrenome: Yi}
    \definition[个,种]{s.}{habilidade | arte | regra; norma | padrão; critério; diretrizes | limite}
    \definition{v.}{plantar; crescer}
  \end{Phonetics}
\end{Entry}

\begin{Entry}{艺人}{4,2}{⾋、⼈}
  \begin{Phonetics}{艺人}{yi4 ren2}[][HSK 6]
    \definition[位]{s.}{artista performático; ator profissional; ator ou artista (em teatro local, narradores, acrobacia ou outro show business); atores de ópera, arte popular, acrobacia, cinema e televisão, etc. | artesão; artífice}
  \end{Phonetics}
\end{Entry}

\begin{Entry}{艺术}{4,5}{⾋、⽊}
  \begin{Phonetics}{艺术}{yi4shu4}[][HSK 3]
    \definition{adj.}{artístico,único; elegante}
    \definition[个,种,门,项,类]{s.}{arte; literatura e arte | habilidade; arte; ofício; métodos criativos}
  \end{Phonetics}
\end{Entry}

%%%%%%%%%% 艾 %%%%%%%%%%
\subsection*{艾}\addcontentsline{loh}{figure}{艾}

\begin{Entry}{艾}{5}{⾋}
  \begin{Phonetics}{艾}{ai4}
    \definition*{s.}{Botânica: Artemísia chinesa (Artemisia argyi)}
    \definition*{s.}{Sobrenome: Ai}
    \definition{adj.}{Literário: gracioso; bonito; lindo}
    \definition{s.}{artemísia; absinto; artemísia chinesa}
    \definition{v.}{Literário: parar; terminar}
  \end{Phonetics}
  \begin{Phonetics}{艾}{yi4}
    \definition{adj.}{estável}
    \definition{v.}{ser corrigido; estar corrigido}
  \end{Phonetics}
\end{Entry}

\begin{Entry}{艾滋病}{5,12,10}{⾋、⽔、⽧}
  \begin{Phonetics}{艾滋病}{ai4zi1bing4}[][HSK 7-9]
    \definition*{s.}{Síndrome da Imunodeficiência Adquirida (AIDS)}
  \end{Phonetics}
\end{Entry}

%%%%%%%%%% 节 %%%%%%%%%%
\subsection*{节}\addcontentsline{loh}{figure}{节}

\begin{Entry}{节}{5}{⾋}
  \begin{Phonetics}{节}{jie1}
    \definition{adj.}{momento crucial; momento crítico; momento decisivo; metáfora para algo importante, decisivo ou oportuno}
  \end{Phonetics}
  \begin{Phonetics}{节}{jie2}[][HSK 2]
    \definition*{s.}{Sobrenome: Jie}
    \definition{clas.}{nó (kn), velocidade de um barco | para seções, comprimentos}
    \definition[个]{s.}{junta; botão; nó; geralmente se refere à parte da grama ou caule da grama onde as folhas crescem ou à parte onde os galhos e troncos das plantas são conectados | parte; divisão; um trecho de algo interligado; uma parte do todo | festival; feriado; dia memorável; um período de tempo ou um dia com características específicas | item; assunto | castidade; integridade ética e moral | articulação; o local onde os ossos humanos ou animais se conectam | etiqueta; cerimonial | batida; ritmo | registro; documento utilizado na antiguidade para comprovar a identidade | estação do ano | sílaba}
    \definition{v.}{economizar; conservar; poupar | resumir; extrair; retirar uma parte do todo | controlar; restringir; moderar}
  \end{Phonetics}
\end{Entry}

\begin{Entry}{节日}{5,4}{⾋、⽇}
  \begin{Phonetics}{节日}{jie2ri4}[][HSK 2]
    \definition[个,种,类]{s.}{festival; feriado; dia de comemoração tradicional; dia comemorativo estabelecido por lei}
  \end{Phonetics}
\end{Entry}

\begin{Entry}{节气}{5,4}{⾋、⽓}
  \begin{Phonetics}{节气}{jie2qi4}[][HSK 7-9]
    \definition{s.}{termo solar (é qualquer um dos 24 períodos nos calendários lunisolares tradicionais chineses); com base na duração do dia e da noite, na altura da sombra do meio-dia e em outros fatores, vários pontos são designados ao longo do ano, cada ponto é chamado de termo solar; os termos solares indicam a posição da Terra em sua órbita, ou seja, a posição do Sol na eclíptica; geralmente, também se referem ao dia da semana em que cada ponto ocorre}
  \end{Phonetics}
\end{Entry}

\begin{Entry}{节水}{5,4}{⾋、⽔}
  \begin{Phonetics}{节水}{jie2shui3}[][HSK 7-9]
    \definition{v.}{economizar água}
  \end{Phonetics}
\end{Entry}

\begin{Entry}{节目}{5,5}{⾋、⽬}
  \begin{Phonetics}{节目}{jie2mu4}[][HSK 2]
    \definition[个,场,项,台]{s.}{programa; item (em um programa); programas artísticos ou projetos transmitidos por rádios e televisões}
  \end{Phonetics}
\end{Entry}

\begin{Entry}{节约}{5,6}{⾋、⽷}
  \begin{Phonetics}{节约}{jie2yue1}[][HSK 3]
    \definition{adj.}{econômico; sem luxo}
    \definition{v.}{guardar; economizar; usar com moderação; economizar gastos desnecessários}
  \end{Phonetics}
\end{Entry}

\begin{Entry}{节衣缩食}{5,6,14,9}{⾋、⾐、⽷、⾷}
  \begin{Phonetics}{节衣缩食}{jie2yi1-suo1shi2}[][HSK 7-9]
    \definition{expr.}{``Reduza os gastos com comida e roupas.''; economizar em comida e roupas; ser mais econômico; viver frugalmente; praticar austeridade; praticar uma economia rigorosa}
  \end{Phonetics}
\end{Entry}

\begin{Entry}{节俭}{5,9}{⾋、⼈}
  \begin{Phonetics}{节俭}{jie2jian3}[][HSK 7-9]
    \definition{adj.}{econômico; frugal}
  \end{Phonetics}
\end{Entry}

\begin{Entry}{节奏}{5,9}{⾋、⼤}
  \begin{Phonetics}{节奏}{jie2zou4}[][HSK 6]
    \definition[个,种]{s.}{ritmo; o fenômeno da alternância regular de comprimento, força e fraqueza das notas na música | padrão regular; uma metáfora para um processo de ajuste adequado com tensão e relaxamento}
  \end{Phonetics}
\end{Entry}

\begin{Entry}{节省}{5,9}{⾋、⽬}
  \begin{Phonetics}{节省}{jie2sheng3}[][HSK 4]
    \definition{adj.}{econômico; parcimonioso}
    \definition{v.}{economizar; conservar; usar com moderação; reduzir; eliminar ou minimizar o esgotamento de itens potencialmente esgotáveis}
  \end{Phonetics}
\end{Entry}

\begin{Entry}{节能}{5,10}{⾋、⾁}
  \begin{Phonetics}{节能}{jie2 neng2}[][HSK 6]
    \definition{v.}{economizar no consumo de energia; conservar energia}
  \end{Phonetics}
\end{Entry}

\begin{Entry}{节假日}{5,11,4}{⾋、⼈、⽇}
  \begin{Phonetics}{节假日}{jie2 jia4 ri4}[][HSK 6]
    \definition[个]{s.}{feriados; festivais e feriados}
  \end{Phonetics}
\end{Entry}

%%%%%%%%%% 芋 %%%%%%%%%%
\subsection*{芋}\addcontentsline{loh}{figure}{芋}

\begin{Entry}{芋}{6}{⾋}
  \begin{Phonetics}{芋}{yu4}
    \definition*{s.}{Sobrenome: Yu}
    \definition{s.}{taro; erva perene | tubérculos; geralmente se refere a batatas, etc.}
  \end{Phonetics}
\end{Entry}

\begin{Entry}{芋头}{6,5}{⾋、⼤}
  \begin{Phonetics}{芋头}{yu4tou5}
    \definition{s.}{taro, similar ao inhame e batata doce}
  \end{Phonetics}
\end{Entry}

\begin{Entry}{芋头色}{6,5,6}{⾋、⼤、⾊}
  \begin{Phonetics}{芋头色}{yu4tou5se4}
    \definition{s.}{cor lilás}
  \end{Phonetics}
\end{Entry}

%%%%%%%%%% 芝 %%%%%%%%%%
\subsection*{芝}\addcontentsline{loh}{figure}{芝}

\begin{Entry}{芝}{6}{⾋}
  \begin{Phonetics}{芝}{zhi1}
    \definition*{s.}{Sobrenome: Zhi}
    \definition{s.}{Arcaico: fungo mágico, ganoderma brilhante | Arcaico: raiz de angélica dahuriana}
  \end{Phonetics}
\end{Entry}

\begin{Entry}{芝麻}{6,11}{⾋、⿇}
  \begin{Phonetics}{芝麻}{zhi1ma5}
    \definition{s.}{semente de gergelim}
  \end{Phonetics}
\end{Entry}

%%%%%%%%%% 芥 %%%%%%%%%%
\subsection*{芥}\addcontentsline{loh}{figure}{芥}

\begin{Entry}{芥}{7}{⾋}
  \begin{Phonetics}{芥}{gai4}
    \definition{s.}{mostarda}
  \seealsoref{芥蓝}{gai4lan2}
  \end{Phonetics}
  \begin{Phonetics}{芥}{jie4}
    \definition{s.}{mostarda}
  \end{Phonetics}
\end{Entry}

\begin{Entry}{芥兰}{7,5}{⾋、⼋}
  \begin{Phonetics}{芥兰}{gai4lan2}
    \variantof{芥蓝}
  \end{Phonetics}
  \begin{Phonetics}{芥兰}{jie4lan2}
    \definition{s.}{couve}
  \end{Phonetics}
\end{Entry}

\begin{Entry}{芥蓝}{7,13}{⾋、⾋}
  \begin{Phonetics}{芥蓝}{gai4lan2}
    \definition{s.}{brócolis chinês; couve chinesa; mostarda}
  \seealsoref{格兰菜}{ge2lan2cai4}
  \end{Phonetics}
\end{Entry}

%%%%%%%%%% 芦 %%%%%%%%%%
\subsection*{芦}\addcontentsline{loh}{figure}{芦}

\begin{Entry}{芦}{7}{⾋}
  \begin{Phonetics}{芦}{lu2}
    \definition*{s.}{Sobrenome: Lu}
    \definition{s.}{junco}
  \end{Phonetics}
\end{Entry}

\begin{Entry}{芦花}{7,7}{⾋、⾋}
  \begin{Phonetics}{芦花}{lu2hua1}[][HSK 7-9]
    \definition{s.}{amentos de junco | flores de junco; os densos pelos brancos no caule da flor do junco}
  \end{Phonetics}
\end{Entry}

\begin{Entry}{芦笋}{7,10}{⾋、⽵}
  \begin{Phonetics}{芦笋}{lu2sun3}
    \definition{s.}{aspargos}
  \end{Phonetics}
\end{Entry}

%%%%%%%%%% 芬 %%%%%%%%%%
\subsection*{芬}\addcontentsline{loh}{figure}{芬}

\begin{Entry}{芬}{7}{⾋}
  \begin{Phonetics}{芬}{fen1}
    \definition*{s.}{Sobrenome: Fen}
    \definition{adj.}{perfumado}
    \definition[阵,股]{s.}{cheiro doce; fragrância; aroma}
  \end{Phonetics}
\end{Entry}

\begin{Entry}{芬芳}{7,7}{⾋、⾋}
  \begin{Phonetics}{芬芳}{fen1fang1}[][HSK 7-9]
    \definition{adj.}{perfumado; com cheiro doce}
    \definition{s.}{fragrância; aroma; cheiro doce}
  \end{Phonetics}
\end{Entry}

%%%%%%%%%% 芭 %%%%%%%%%%
\subsection*{芭}\addcontentsline{loh}{figure}{芭}

\begin{Entry}{芭}{7}{⾋}
  \begin{Phonetics}{芭}{ba1}
    \definition{s.}{Arcaico: uma planta perfumada | junco; um tipo de erva | flor}
  \end{Phonetics}
\end{Entry}

\begin{Entry}{芭蕾}{7,16}{⾋、⾋}
  \begin{Phonetics}{芭蕾}{ba1lei3}[][HSK 7-9]
    \definition[场]{s.}{Empréstimo linguístico: balé; um tipo de dança popular na Europa, em que as dançarinas costumam tocar o chão com os dedos dos pés}
  \seealsoref{芭蕾舞}{ba1lei3wu3}
  \end{Phonetics}
\end{Entry}

\begin{Entry}{芭蕾舞}{7,16,14}{⾋、⾋、⾇}
  \begin{Phonetics}{芭蕾舞}{ba1lei3wu3}
    \definition{s.}{Empréstimo linguístico: balé; um tipo de dança clássica europeia; também conhecido como balé clássico europeu}
  \end{Phonetics}
\end{Entry}

%%%%%%%%%% 芯 %%%%%%%%%%
\subsection*{芯}\addcontentsline{loh}{figure}{芯}

\begin{Entry}{芯}{7}{⾋}
  \begin{Phonetics}{芯}{xin1}
    \definition{s.}{medula de junco | pavio}
  \end{Phonetics}
  \begin{Phonetics}{芯}{xin4}
    \definition{s.}{núcleo; a parte central de um objeto | língua de cobra}
  \end{Phonetics}
\end{Entry}

\begin{Entry}{芯片}{7,4}{⾋、⽚}
  \begin{Phonetics}{芯片}{xin1pian4}
    \definition{s.}{\emph{chip} de computador; \emph{microchip}; um substrato (geralmente uma pastilha de silício) que contém um circuito integrado completo}
  \end{Phonetics}
\end{Entry}

%%%%%%%%%% 花 %%%%%%%%%%
\subsection*{花}\addcontentsline{loh}{figure}{花}

\begin{Entry}{花}{7}{⾋}
  \begin{Phonetics}{花}{hua1}[][HSK 1,2,4]
    \definition*{s.}{Sobrenome: Hua}
    \definition{adj.}{multicolorido; colorido | embaçado; obscuro; deslumbrado e confuso | extravagante; florido; vistoso}
    \definition[朵,支,束,把,盆,簇]{s.}{flor; órgãos de reprodução sexual de plantas com sementes | flor; planta ornamental |  qualquer coisa que se assemelhe a uma flor | fogos de artifício | padrão; design; design decorativo | flor; metáfora para a essência de uma causa | prostituta; cortesã; referindo-se a prostitutas ou a assuntos relacionados a prostitutas | algodão | varíola | ferimento; ferida; lesões traumáticas sofridas em combate}
    \definition{v.}{gastar; despender; consumir}
  \end{Phonetics}
\end{Entry}

\begin{Entry}{花儿}{7,2}{⾋、⼉}
  \begin{Phonetics}{花儿}{hua1r5}
    \definition[朵,支,束,把,盆,簇]{s.}{flor}
  \end{Phonetics}
\end{Entry}

\begin{Entry}{花卉}{7,5}{⾋、⼗}
  \begin{Phonetics}{花卉}{hua1hui4}[][HSK 7-9]
    \definition{s.}{flores e plantas | pintura de flores e plantas no estilo tradicional chinês}
  \end{Phonetics}
\end{Entry}

\begin{Entry}{花生}{7,5}{⾋、⽣}
  \begin{Phonetics}{花生}{hua1sheng1}[][HSK 6]
    \definition[把,颗,粒,袋]{s.}{amendoim}
  \seealsoref{落花生}{luo4 hua1 sheng1}
  \end{Phonetics}
\end{Entry}

\begin{Entry}{花园}{7,7}{⾋、⼞}
  \begin{Phonetics}{花园}{hua1 yuan2}[][HSK 2]
    \definition[个,座]{s.}{jardim; um local onde se plantam flores e árvores para passear e descansar}
  \end{Phonetics}
\end{Entry}

\begin{Entry}{花纹}{7,7}{⾋、⽷}
  \begin{Phonetics}{花纹}{hua1wen2}[][HSK 7-9]
    \definition[条,道,个]{s.}{figura; padrão decorativo; (padrões) várias listras e gráficos}
  \end{Phonetics}
\end{Entry}

\begin{Entry}{花店}{7,8}{⾋、⼴}
  \begin{Phonetics}{花店}{hua1dian4}
    \definition{s.}{floricultura}
  \end{Phonetics}
\end{Entry}

\begin{Entry}{花茶}{7,9}{⾋、⾋}
  \begin{Phonetics}{花茶}{hua1cha2}
    \definition[杯,壶]{s.}{chá perfumado}
  \end{Phonetics}
\end{Entry}

\begin{Entry}{花费}{7,9}{⾋、⾙}
  \begin{Phonetics}{花费}{hua1 fei4}[][HSK 6]
    \definition[笔]{s.}{dinheiro gasto; despesas | custo; gastos; desembolso | despesa}
    \definition{v.}{gastar (tempo ou dinheiro)}
  \end{Phonetics}
\end{Entry}

\begin{Entry}{花样}{7,10}{⾋、⽊}
  \begin{Phonetics}{花样}{hua1yang4}[][HSK 7-9]
    \definition[种]{s.}{padrão, geralmente referindo-se a vários padrões ou tipos | truque | variedade; padrão decorativo; padrão de bordado, geralmente cortado ou esculpido em papel}
  \end{Phonetics}
\end{Entry}

\begin{Entry}{花样游泳}{7,10,12,8}{⾋、⽊、⽔、⽔}
  \begin{Phonetics}{花样游泳}{hua1yang4 you2yong3}
    \definition{s.}{nado sincronizado}
  \end{Phonetics}
\end{Entry}

\begin{Entry}{花瓶}{7,10}{⾋、⽡}
  \begin{Phonetics}{花瓶}{hua1 ping2}[][HSK 6]
    \definition[个,对]{s.}{vaso de flores; vaso usado para arranjos florais colocado em ambientes internos como decoração | Figurativo: um ornamento; mulher empregada não por sua habilidade, mas por sua aparência; uma metáfora para uma pessoa ou coisa que é usada apenas para exibição e não tem uso prático}
  \end{Phonetics}
\end{Entry}

\begin{Entry}{花脸}{7,11}{⾋、⾁}
  \begin{Phonetics}{花脸}{hua1lian3}
    \definition*{s.}{Hualian, personagem do rosto florido, um nome popular para 净 (assim chamado devido à elaborada pintura facial)}
  \seealsoref{净}{jing4}
  \end{Phonetics}
\end{Entry}

\begin{Entry}{花椰菜}{7,12,11}{⾋、⽊、⾋}
  \begin{Phonetics}{花椰菜}{hua1ye1cai4}
    \definition{s.}{couve-flor}
  \end{Phonetics}
\end{Entry}

\begin{Entry}{花瓣}{7,19}{⾋、⽠}
  \begin{Phonetics}{花瓣}{hua1ban4}[][HSK 7-9]
    \definition[片,个]{s.}{pétala; um componente da corola, semelhante em estrutura a uma folha, mas com células contendo vários pigmentos, resultando em uma variedade de cores}
  \end{Phonetics}
\end{Entry}

%%%%%%%%%% 芹 %%%%%%%%%%
\subsection*{芹}\addcontentsline{loh}{figure}{芹}

\begin{Entry}{芹}{7}{⾋}
  \begin{Phonetics}{芹}{qin2}
    \definition[把,棵]{s.}{aipo | aipo chinês}
  \end{Phonetics}
\end{Entry}

\begin{Entry}{芹菜}{7,11}{⾋、⾋}
  \begin{Phonetics}{芹菜}{qin2cai4}
    \definition{s.}{salsão}
  \end{Phonetics}
\end{Entry}

%%%%%%%%%% 苍 %%%%%%%%%%
\subsection*{苍}\addcontentsline{loh}{figure}{苍}

\begin{Entry}{苍}{7}{⾋}
  \begin{Phonetics}{苍}{cang1}
    \definition*{s.}{Sobrenome: Cang}
    \definition{adj.}{verde escuro (ou azul); ciano (inclui azul e verde) | cinza; acinzentado}
    \definition{s.}{o céu azul; o céu acima}
  \end{Phonetics}
\end{Entry}

\begin{Entry}{苍蝇}{7,14}{⾋、⾍}
  \begin{Phonetics}{苍蝇}{cang1ying5}[][HSK 7-9]
    \definition[只,个,群]{s.}{mosca;mosca doméstica}
  \end{Phonetics}
\end{Entry}

%%%%%%%%%% 苏 %%%%%%%%%%
\subsection*{苏}\addcontentsline{loh}{figure}{苏}

\begin{Entry}{苏}{7}{⾋}
  \begin{Phonetics}{苏}{su1}
    \definition*{s.}{Suzhou, abreviação de 苏州 | Província de Jiangsu, abreviação de 江苏 | União Soviética, abreviação de 苏联 | Sobrenome: Su}
    \definition{s.}{perilla planta da família das mentas}
    \definition{v.}{reviver; vir a; acordar}
  \seealsoref{江苏}{jiang1su1}
  \seealsoref{苏联}{su1lian2}
  \seealsoref{苏州}{su1zhou1}
  \end{Phonetics}
\end{Entry}

\begin{Entry}{苏州}{7,6}{⾋、⼮}
  \begin{Phonetics}{苏州}{su1zhou1}
    \definition*{s.}{Suzhou, cidade na Província de Jiangsu}
  \end{Phonetics}
\end{Entry}

\begin{Entry}{苏格兰}{7,10,5}{⾋、⽊、⼋}
  \begin{Phonetics}{苏格兰}{su1ge2lan2}
    \definition*{s.}{Escócia}
  \end{Phonetics}
\end{Entry}

\begin{Entry}{苏联}{7,12}{⾋、⽿}
  \begin{Phonetics}{苏联}{su1lian2}
    \definition*{s.}{União das Repúblicas Socialistas Soviéticas (1922-1991)}
  \end{Phonetics}
\end{Entry}

%%%%%%%%%% 苗 %%%%%%%%%%
\subsection*{苗}\addcontentsline{loh}{figure}{苗}

\begin{Entry}{苗}{8}{⾋}
  \begin{Phonetics}{苗}{miao2}[][HSK 7-9]
    \definition*{s.}{Miao, grupo étnico, abreviação de 苗族 | Sobrenome: Miao}
    \definition[棵,株,尾,头,些]{s.}{planta jovem; muda; broto; plantas recém-germinadas | descendente; filho | filhotes de alguns animais; animais domésticos recém-nascidos | vacina | algo que se assemelha a uma planta jovem}
  \seealsoref{苗族}{miao2zu2}
  \end{Phonetics}
\end{Entry}

\begin{Entry}{苗头}{8,5}{⾋、⼤}
  \begin{Phonetics}{苗头}{miao2tou5}[][HSK 7-9]
    \definition{s.}{sintoma de uma tendência; indício de um novo desenvolvimento; uma tendência ou situação de desenvolvimento ligeiramente emergente}
  \end{Phonetics}
\end{Entry}

\begin{Entry}{苗条}{8,7}{⾋、⽊}
  \begin{Phonetics}{苗条}{miao2tiao5}[][HSK 7-9]
    \definition{adj.}{(figura feminina) magra; esbelta; esbelta e graciosa}
  \end{Phonetics}
\end{Entry}

\begin{Entry}{苗族}{8,11}{⾋、⽅}
  \begin{Phonetics}{苗族}{miao2zu2}
    \definition*{s.}{Grupo étnico Hmong ou Miao do sudoeste da China; uma das minorias étnicas da China, distribuída por Guizhou 贵州, Hunan 湖南, Yunnan 云南, Guangxi 广西, Sichuan 四川, Guangdong 广东 e Hubei 湖北}
  \seealsoref{广东}{guang3dong1}
  \seealsoref{广西}{guang3xi1}
  \seealsoref{贵州}{gui4zhou1}
  \seealsoref{湖北}{hu2bei3}
  \seealsoref{湖南}{hu2nan2}
  \seealsoref{四川}{si4chuan1}
  \seealsoref{云南}{yun2nan2}
  \end{Phonetics}
\end{Entry}

%%%%%%%%%% 苛 %%%%%%%%%%
\subsection*{苛}\addcontentsline{loh}{figure}{苛}

\begin{Entry}{苛}{8}{⾋}
  \begin{Phonetics}{苛}{ke1}
    \definition{adj.}{duro; severo; exigente; opressivo | excessivamente elaborado; exorbitante; diverso; variado}
  \end{Phonetics}
\end{Entry}

\begin{Entry}{苛刻}{8,8}{⾋、⼑}
  \begin{Phonetics}{苛刻}{ke1ke4}[][HSK 7-9]
    \definition{adj.}{severo; rigoroso; os requisitos são muito rigorosos ou as condições são muito elevadas}
  \end{Phonetics}
\end{Entry}

%%%%%%%%%% 若 %%%%%%%%%%
\subsection*{若}\addcontentsline{loh}{figure}{若}

\begin{Entry}{若}{8}{⾋}
  \begin{Phonetics}{若}{ruo4}[][HSK 6]
    \definition*{s.}{Sobrenome: Ruo}
    \definition{adv.}{como se; como se fosse; usado antes do verbo para indicar que o que foi dito é mais ou menos assim, equivalente a 好像}
    \definition{conj.}{se; usado na primeira parte de uma frase composta, expressa uma relação hipotética, equivalente a 如果}
    \definition{pron.}{você; referir-se ao interlocutor como 你 ou 你的}
    \definition{v.}{parecer}
  \seealsoref{好像}{hao3xiang4}
  \seealsoref{你}{ni3}
  \seealsoref{你的}{ni3 de5}
  \seealsoref{如果}{ru2guo3}
  \end{Phonetics}
\end{Entry}

%%%%%%%%%% 苦 %%%%%%%%%%
\subsection*{苦}\addcontentsline{loh}{figure}{苦}

\begin{Entry}{苦}{8}{⾋}
  \begin{Phonetics}{苦}{ku3}[][HSK 4]
    \definition{adj.}{amargo; descreve um sabor parecido com o de melão amargo ou raiz de coptis (em oposição a 甘 ou 甜) | difícil; doloroso; sofrido}
    \definition{adv.}{meticulosamente; diligentemente; pacientemente}
    \definition{v.}{causar sofrimento a alguém; dificultar a vida de alguém; causar dor; tornar desconfortável | sofrer de; ser incomodado por; sentir-se angustiado com uma situação | estar desgastado; cortar demais; descrever a superação de um certo nível em algum aspecto}
  \seealsoref{甘}{gan1}
  \seealsoref{甜}{tian2}
  \end{Phonetics}
\end{Entry}

\begin{Entry}{苦力}{8,2}{⾋、⼒}
  \begin{Phonetics}{苦力}{ku3li4}[][HSK 7-9]
    \definition{s.}{trabalhador hindu ou chinês; carregador | Empréstimo linguístico: \emph{coolie}, trabalhador chinês não qualificado nos tempos coloniais | trabalho amargo | trabalho árduo}
  \end{Phonetics}
\end{Entry}

\begin{Entry}{苦心}{8,4}{⾋、⼼}
  \begin{Phonetics}{苦心}{ku3xin1}[][HSK 7-9]
    \definition{adv.}{com esmero; assiduamente; meticulosamente; com esforço ou dedicação; isso demonstra que você dedicou muito esforço e energia}
    \definition{s.}{dores; dificuldade tomada; o esforço e a energia investidos em trabalhar arduamente por algo}
  \end{Phonetics}
\end{Entry}

\begin{Entry}{苦瓜}{8,5}{⾋、⽠}
  \begin{Phonetics}{苦瓜}{ku3gua1}
    \definition{s.}{melão amargo (cabaça amarga, pêra bálsamo, maçã bálsamo, pepino amargo)}
  \end{Phonetics}
\end{Entry}

\begin{Entry}{苦练}{8,8}{⾋、⽷}
  \begin{Phonetics}{苦练}{ku3 lian4}[][HSK 7-9]
    \definition{v.}{treinar diligentemente; praticar bastante}
  \end{Phonetics}
\end{Entry}

\begin{Entry}{苦恼}{8,9}{⾋、⼼}
  \begin{Phonetics}{苦恼}{ku3nao3}[][HSK 7-9]
    \definition{adj.}{aflito; preocupado; atormentado}
    \definition{v.}{atormentar; causar dor e sofrimento}
  \end{Phonetics}
\end{Entry}

\begin{Entry}{苦笑}{8,10}{⾋、⽵}
  \begin{Phonetics}{苦笑}{ku3xiao4}[][HSK 7-9]
    \definition{s.}{sorriso forçado; sorriso irônico; sorriso amargo}
    \definition{v.}{forçar um sorriso; esboçar um sorriso irônico; dar uma risada amarga; produzir um sorriso forçado; forçar um sorriso quando você não está de bom humor}
  \end{Phonetics}
\end{Entry}

\begin{Entry}{苦难}{8,10}{⾋、⾫}
  \begin{Phonetics}{苦难}{ku3nan4}[][HSK 7-9]
    \definition{adj.}{sofrimento; dificuldade}
    \definition{s.}{sofrimento; miséria; angústia; tribulação; dor e desastre}
  \end{Phonetics}
\end{Entry}

%%%%%%%%%% 英 %%%%%%%%%%
\subsection*{英}\addcontentsline{loh}{figure}{英}

\begin{Entry}{英}{8}{⾋}
  \begin{Phonetics}{英}{ying1}
    \definition*{s.}{Reino Unido, abreviação de 英国 | Sobrenome: Ying}
    \definition{s.}{flor | herói; pessoa excepcional | uma pessoa de talento ou sabedoria extraordinários}
  \seealsoref{英国}{ying1guo2}
  \end{Phonetics}
\end{Entry}

\begin{Entry}{英文}{8,4}{⾋、⽂}
  \begin{Phonetics}{英文}{ying1 wen2}[][HSK 2]
    \definition{s.}{inglês, língua inglesa; a forma escrita do inglês}
  \end{Phonetics}
\end{Entry}

\begin{Entry}{英国}{8,8}{⾋、⼞}
  \begin{Phonetics}{英国}{ying1guo2}
    \definition*{s.}{Reino Unido; Grã-Bretanha; Inglaterra}
  \end{Phonetics}
\end{Entry}

\begin{Entry}{英国人}{8,8,2}{⾋、⼞、⼈}
  \begin{Phonetics}{英国人}{ying1guo2ren2}
    \definition{s.}{inglês | pessoa ou povo do Reino Unido}
  \end{Phonetics}
\end{Entry}

\begin{Entry}{英明}{8,8}{⾋、⽇}
  \begin{Phonetics}{英明}{ying1ming2}
    \definition{adj.}{sábio; brilhante; excelente e sábio}
  \end{Phonetics}
\end{Entry}

\begin{Entry}{英勇}{8,9}{⾋、⼒}
  \begin{Phonetics}{英勇}{ying1yong3}[][HSK 4]
    \definition{adj.}{heroico; valente; bravo; corajoso; extraordinariamente corajoso}
  \end{Phonetics}
\end{Entry}

\begin{Entry}{英语}{8,9}{⾋、⾔}
  \begin{Phonetics}{英语}{ying1 yu3}[][HSK 2]
    \definition{s.}{inglês, língua inglesa}
  \end{Phonetics}
\end{Entry}

\begin{Entry}{英雄}{8,12}{⾋、⾫}
  \begin{Phonetics}{英雄}{ying1xiong2}[][HSK 6]
    \definition{adj.}{heróico}
    \definition[名,个,位]{s.}{herói; uma pessoa cujas habilidades e coragem superam as das pessoas comuns | herói; aqueles que não têm medo das dificuldades, dos perigos ou da morte e que lutam bravamente pelos interesses do povo, mesmo ao custo das suas próprias vidas}
  \end{Phonetics}
\end{Entry}

%%%%%%%%%% 苹 %%%%%%%%%%
\subsection*{苹}\addcontentsline{loh}{figure}{苹}

\begin{Entry}{苹}{8}{⾋}
  \begin{Phonetics}{苹}{ping2}
    \definition[个]{s.}{uma espécie de artemísia | maçã | lentilha-d'água}
  \end{Phonetics}
\end{Entry}

\begin{Entry}{苹果}{8,8}{⾋、⽊}
  \begin{Phonetics}{苹果}{ping2guo3}[][HSK 3]
    \definition[个,斤,筐,箱,棵,种]{s.}{maçã}
  \end{Phonetics}
\end{Entry}

%%%%%%%%%% 茂 %%%%%%%%%%
\subsection*{茂}\addcontentsline{loh}{figure}{茂}

\begin{Entry}{茂}{8}{⾋}
  \begin{Phonetics}{茂}{mao4}
    \definition*{s.}{Sobrenome: Mao}
    \definition{adj.}{luxuriante; exuberante; profuso | rico e esplêndido; rico e requintado}
    \definition[种]{s.}{ciclopentadieno; composto orgânico, fórmula molecular $C_5H_6$, líquido incolor, usado na fabricação de pesticidas, plásticos, etc.}
  \end{Phonetics}
\end{Entry}

\begin{Entry}{茂密}{8,11}{⾋、⼧}
  \begin{Phonetics}{茂密}{mao4mi4}[][HSK 7-9]
    \definition{adj.}{(grama ou árvores) denso; espesso; exuberante e denso}
  \end{Phonetics}
\end{Entry}

\begin{Entry}{茂盛}{8,11}{⾋、⽫}
  \begin{Phonetics}{茂盛}{mao4sheng4}[][HSK 7-9]
    \definition{adj.}{exuberante; viçoso; abundante; florescente | exuberante; florescente; próspero}
  \end{Phonetics}
\end{Entry}

%%%%%%%%%% 茄 %%%%%%%%%%
\subsection*{茄}\addcontentsline{loh}{figure}{茄}

\begin{Entry}{茄}{8}{⾋}
  \begin{Phonetics}{茄}{jia1}
    \definition{s.}{caracter fonético usado em empréstimos linguísticos para o som "jia", embora 夹 seja mais comum}
  \seealsoref{夹}{jia1}
  \end{Phonetics}
  \begin{Phonetics}{茄}{qie2}
    \definition[只]{s.}{berinjela}
  \end{Phonetics}
\end{Entry}

\begin{Entry}{茄子}{8,3}{⾋、⼦}
  \begin{Phonetics}{茄子}{qie2 zi5}[][HSK 6]
    \definition{interj.}{Onomatopéia: ``xis'' fonético (ao ser fotografado), equivale ao ``diga xis''}
    \definition[个,根]{s.}{berinjela (fruto e planta)}
  \end{Phonetics}
\end{Entry}

%%%%%%%%%% 茅 %%%%%%%%%%
\subsection*{茅}\addcontentsline{loh}{figure}{茅}

\begin{Entry}{茅}{8}{⾋}
  \begin{Phonetics}{茅}{mao2}
    \definition*{s.}{Sobrenome: Mao}
    \definition[座]{s.}{capim-cogon | planta semelhante ao capim-cogon (como palha)}
  \end{Phonetics}
\end{Entry}

\begin{Entry}{茅台}{8,5}{⾋、⼝}
  \begin{Phonetics}{茅台}{mao2tai2}
    \definition*{s.}{Moutai (bebida alcoólica)}
  \seealsoref{茅台酒}{mao2tai2 jiu3}
  \end{Phonetics}
\end{Entry}

\begin{Entry}{茅台酒}{8,5,10}{⾋、⼝、⾣}
  \begin{Phonetics}{茅台酒}{mao2tai2 jiu3}[][HSK 7-9]
    \definition*[瓶,斤,箱,口,杯]{s.}{Maotai; Moutai (espírito forte) | Maotai (um licor chinês); Mao Tai}
  \end{Phonetics}
\end{Entry}

\begin{Entry}{茅厕}{8,8}{⾋、⼚}
  \begin{Phonetics}{茅厕}{mao2ce4}
    \definition{s.}{(dialeto) latrina}
  \end{Phonetics}
\end{Entry}

%%%%%%%%%% 茎 %%%%%%%%%%
\subsection*{茎}\addcontentsline{loh}{figure}{茎}

\begin{Entry}{茎}{8}{⾋}
  \begin{Phonetics}{茎}{jing1}[][HSK 7-9]
    \definition{clas.}{Literário: utilizado como classificador para indicar um objeto em forma de tira}[几茎小草。===Algumas folhas de grama.]
    \definition[根]{s.}{caule (de uma planta); talo; tronco | algo como um caule ou haste}
  \end{Phonetics}
\end{Entry}

%%%%%%%%%% 范 %%%%%%%%%%
\subsection*{范}\addcontentsline{loh}{figure}{范}

\begin{Entry}{范}{9}{⾋}
  \begin{Phonetics}{范}{fan4}
    \definition*{s.}{Sobrenome: Fan}
    \definition{s.}{padrão; molde; matriz | modelo; exemplo; modelo a seguir | limites; escopo | restrição; limite}
  \end{Phonetics}
\end{Entry}

\begin{Entry}{范成大}{9,6,3}{⾋、⼽、⼤}
  \begin{Phonetics}{范成大}{fan4 cheng2da4}
    \definition*{s.}{Fan Chengda (1126–1193), de nome de cortesia Zhineng 致能 e também Youyuan 幼元, autodenominou-se Cishan Jushi 此山 居士em seus primeiros anos e Shihu Jushi 石湖 居士em seus últimos anos; natural do Condado de Wu, Suzhou (atual Cidade de Suzhou, Província de Jiangsu), foi um oficial, poeta e escritor durante a Dinastia Song do Sul; seu nome póstumo foi Wenmu 文穆}
  \end{Phonetics}
\end{Entry}

\begin{Entry}{范围}{9,7}{⾋、⼞}
  \begin{Phonetics}{范围}{fan4wei2}[][HSK 3]
    \definition[个]{s.}{escopo; limite; alcance}
    \definition{v.}{estabelecer limites para; limitar o escopo de}
  \end{Phonetics}
\end{Entry}

\begin{Entry}{范畴}{9,12}{⾋、⽥}
  \begin{Phonetics}{范畴}{fan4chou2}[][HSK 7-9]
    \definition{s.}{tipo; domínio; escopo; alcance; categoria}
  \end{Phonetics}
\end{Entry}

%%%%%%%%%% 茫 %%%%%%%%%%
\subsection*{茫}\addcontentsline{loh}{figure}{茫}

\begin{Entry}{茫}{9}{⾋}
  \begin{Phonetics}{茫}{mang2}
    \definition{adj.}{ilimitado e indistinto | ignorante; no escuro | disseminado e pouco claro; descreve a água ou outras coisas como ilimitadas e incertas}
  \end{Phonetics}
\end{Entry}

\begin{Entry}{茫然}{9,12}{⾋、⽕}
  \begin{Phonetics}{茫然}{mang2ran2}[][HSK 7-9]
    \definition{adj.}{vazio; vago; ignorante; no escuro; descreve um estado de completa ignorância ou perplexidade | frustrado; decepcionado; descreve uma aparência atordoada ou distraída devido à decepção}
  \end{Phonetics}
\end{Entry}

%%%%%%%%%% 茶 %%%%%%%%%%
\subsection*{茶}\addcontentsline{loh}{figure}{茶}

\begin{Entry}{茶}{9}{⾋}
  \begin{Phonetics}{茶}{cha2}[][HSK 1]
    \definition{adj.}{moreno; fulvo; amarelo-acastanhado}
    \definition[杯,壶]{s.}{chá (a bebida); bebida feita com folhas de chá | chá (a planta) | certos tipos de bebidas ou alimentos líquidos | árvore de chá-de-óleo | camélia}
  \end{Phonetics}
\end{Entry}

\begin{Entry}{茶叶}{9,5}{⾋、⼝}
  \begin{Phonetics}{茶叶}{cha2 ye4}[][HSK 4]
    \definition[包,袋,盒,斤,把,种]{s.}{chá; folhas de chá; as folhas jovens da planta do chá que são processadas para produzir bebidas}
  \end{Phonetics}
\end{Entry}

\begin{Entry}{茶馆儿}{9,11,2}{⾋、⾷、⼉}
  \begin{Phonetics}{茶馆儿}{cha2guan3r5}[][HSK 7-9]
    \definition[家,个,间]{s.}{casa de chá}
  \end{Phonetics}
\end{Entry}

\begin{Entry}{茶道}{9,12}{⾋、⾡}
  \begin{Phonetics}{茶道}{cha2dao4}[][HSK 7-9]
    \definition{s.}{cerimônia do chá ; cerimônia japonesa do chá; Sadō}
  \end{Phonetics}
\end{Entry}

%%%%%%%%%% 荆 %%%%%%%%%%
\subsection*{荆}\addcontentsline{loh}{figure}{荆}

\begin{Entry}{荆}{9}{⾋}
  \begin{Phonetics}{荆}{jing1}
    \definition*{s.}{Sobrenome: Jing}
    \definition{s.}{árvore-da-castidade; vitex | uma vara para açoitar; varas de punição antigas feitas de galhos de espinho | a própria esposa; uma forma humilde de se referir à esposa antigamente}
  \end{Phonetics}
\end{Entry}

\begin{Entry}{荆棘}{9,12}{⾋、⽊}
  \begin{Phonetics}{荆棘}{jing1ji2}[][HSK 7-9]
    \definition{s.}{silvas; cardos e espinhos; vegetação rasteira espinhosa; geralmente se refere a arbustos espinhosos que crescem nas montanhas e nos campos}
  \end{Phonetics}
\end{Entry}

%%%%%%%%%% 草 %%%%%%%%%%
\subsection*{草}\addcontentsline{loh}{figure}{草}

\begin{Entry}{草}{9}{⾋}
  \begin{Phonetics}{草}{cao3}[][HSK 2]
    \definition*{s.}{Sobrenome: Cao}
    \definition{adj.}{descuidado; rude | rascunho; inicial | femea; na linguagem coloquial, refere-se a animais domésticos e aves fêmeas | precipitado; pouco cuidadoso | rascunho; não definitivo; preliminar; informal}
    \definition[种,棵,撮,株,根]{s.}{grama; gramado | palha | campo; zona rural; área selvagem | letra cursiva | letra cursiva (ou caligráfica) de um alfabeto fonético | rascunho | caligrafia cursiva; um tipo de escrita chinesa}
    \definition{v.}{esboçar; redigir}
  \end{Phonetics}
\end{Entry}

\begin{Entry}{草地}{9,6}{⾋、⼟}
  \begin{Phonetics}{草地}{cao3 di4}[][HSK 2]
    \definition[片,块]{s.}{prado; gramado; campo; pastagem ou grande área de terra plantada com pastagem | gramado; relvado; local com grama alta ou gramado}
  \end{Phonetics}
\end{Entry}

\begin{Entry}{草纸}{9,7}{⾋、⽷}
  \begin{Phonetics}{草纸}{cao3zhi3}
    \definition{s.}{papel pardo | pergaminho | papel de palha áspero | papel higiênico}
  \end{Phonetics}
\end{Entry}

\begin{Entry}{草坪}{9,8}{⾋、⼟}
  \begin{Phonetics}{草坪}{cao3ping2}[][HSK 7-9]
    \definition{s.}{gramado; prado plano; agora se refere principalmente a um pasto plano inteiro cultivado artificialmente}
  \end{Phonetics}
\end{Entry}

\begin{Entry}{草原}{9,10}{⾋、⼚}
  \begin{Phonetics}{草原}{cao3 yuan2}[][HSK 5]
    \definition[片,个]{s.}{estepe; pradaria; grandes áreas de terra coberta de vegetação em áreas semiáridas, intercaladas com árvores tolerantes à seca}
  \end{Phonetics}
\end{Entry}

\begin{Entry}{草案}{9,10}{⾋、⽊}
  \begin{Phonetics}{草案}{cao3'an4}[][HSK 7-9]
    \definition[个,项,份,部]{s.}{rascunho (de um plano, lei, etc.); leis e regulamentos que foram escritos, mas ainda não foram finalizados pelos departamentos relevantes ou ainda estão sendo testados}
  \end{Phonetics}
\end{Entry}

\begin{Entry}{草莓}{9,10}{⾋、⾋}
  \begin{Phonetics}{草莓}{cao3mei2}
    \definition[颗]{s.}{morango}
  \end{Phonetics}
\end{Entry}

%%%%%%%%%% 荒 %%%%%%%%%%
\subsection*{荒}\addcontentsline{loh}{figure}{荒}

\begin{Entry}{荒}{9}{⾋}
  \begin{Phonetics}{荒}{huang1}[][HSK 7-9]
    \definition*{s.}{Sobrenome: Huang}
    \definition{adj.}{(terra) não utilizada; não cultivada | desolado; estéril | irracional; delirante; fantástico; absurdo | incerto; duvidoso | dissoluto; autoindulgente | grosseiramente processado; bruto}
    \definition[片,块]{s.}{terra devastada; terra inculta; deserto | fome; quebra de safra | escassez | lixo; restos | terra selvagem (floresta)}
    \definition{v.}{(coloquial) negligenciar; estar fora de prática}
  \end{Phonetics}
\end{Entry}

\begin{Entry}{荒芜}{9,7}{⾋、⾋}
  \begin{Phonetics}{荒芜}{huang1wu2}
    \definition{adj.}{estéril}
  \end{Phonetics}
\end{Entry}

\begin{Entry}{荒诞}{9,8}{⾋、⾔}
  \begin{Phonetics}{荒诞}{huang1dan4}[][HSK 7-9]
    \definition{adj.}{fantástico; absurdo; incrível; inacreditável}
  \end{Phonetics}
\end{Entry}

\begin{Entry}{荒凉}{9,10}{⾋、⼎}
  \begin{Phonetics}{荒凉}{huang1liang2}[][HSK 7-9]
    \definition{adj.}{selvagem; sombrio e desolado; escassamente povoado; deserto}
  \end{Phonetics}
\end{Entry}

\begin{Entry}{荒唐}{9,10}{⾋、⼝}
  \begin{Phonetics}{荒唐}{huang1tang2}
    \definition{adj.}{absurdo; fantástico; grosseiramente exagerado; descreve pensamentos, palavras ou comportamentos anormais, fazendo as pessoas se sentirem estranhas ou ridículas | dissipado; dissoluto; descreve pessoas que não controlam seus desejos, não são limitadas por restrições e fazem as coisas casualmente}
  \end{Phonetics}
\end{Entry}

\begin{Entry}{荒谬}{9,13}{⾋、⾔}
  \begin{Phonetics}{荒谬}{huang1miu4}[][HSK 7-9]
    \definition{adj.}{absurdo; ridículo; extremamente errado; extremamente irracional}
  \end{Phonetics}
\end{Entry}

%%%%%%%%%% 荔 %%%%%%%%%%
\subsection*{荔}\addcontentsline{loh}{figure}{荔}

\begin{Entry}{荔}{9}{⾋}
  \begin{Phonetics}{荔}{li4}
    \definition[颗]{s.}{lichia | (arcaico) uma espécie de grama semelhante à taboa}
  \end{Phonetics}
\end{Entry}

\begin{Entry}{荔枝}{9,8}{⾋、⽊}
  \begin{Phonetics}{荔枝}{li4zhi1}
    \definition{s.}{lichia}
  \end{Phonetics}
\end{Entry}

%%%%%%%%%% 荡 %%%%%%%%%%
\subsection*{荡}\addcontentsline{loh}{figure}{荡}

\begin{Entry}{荡}{9}{⾋}
  \begin{Phonetics}{荡}{dang4}
    \definition*{s.}{Sobrenome: Dang}
    \definition{adj.}{indiferente às restrições morais; devasso, libertino, depravado, desregrado | vasto, amplo e nivelado}
    \definition{s.}{lago raso; pântano}
    \definition{v.}{oscilar; gingar; ondular | vadiar; vagabundear | enxaguar | limpar; varrer | vagar; vaguear; andar por aí; passear por aí}
  \end{Phonetics}
\end{Entry}

\begin{Entry}{荡漾}{9,14}{⾋、⽔}
  \begin{Phonetics}{荡漾}{dang4yang4}[][HSK 7-9]
    \definition{v.}{ondular; agitar; ser agitado}
  \end{Phonetics}
\end{Entry}

%%%%%%%%%% 荣 %%%%%%%%%%
\subsection*{荣}\addcontentsline{loh}{figure}{荣}

\begin{Entry}{荣}{9}{⾋}
  \begin{Phonetics}{荣}{rong2}
    \definition*{s.}{Sobrenome: Rong}
    \definition{adj.}{próspero; florescente | exuberante | glorioso}
    \definition{s.}{honra; glória (oposto a 辱) | guarda-sol chinês | flor; flor de planta herbácea | beirais virados para cima}
    \definition{v.}{glorificar; luxuriar; crescer abundantemente; florescer | florescer | lançar}
  \seealsoref{辱}{ru3}
  \end{Phonetics}
\end{Entry}

%%%%%%%%%% 荤 %%%%%%%%%%
\subsection*{荤}\addcontentsline{loh}{figure}{荤}

\begin{Entry}{荤}{9}{⾋}[Kangxi 9]
  \begin{Phonetics}{荤}{hun1}
    \definition{adj.}{obsceno; lascivo; vulgar}
    \definition{s.}{carne ou peixe (oposto a 素) | Budismo: vegetais picantes proibidos aos vegetarianos budistas, como cebola, alho-poró, alho, etc. | alimentos não vegetarianos (carne, peixe etc.) | vegetais com cheiro forte (alho etc.)}
  \seealsoref{素}{su4}
  \end{Phonetics}
\end{Entry}

%%%%%%%%%% 药 %%%%%%%%%%
\subsection*{药}\addcontentsline{loh}{figure}{药}

\begin{Entry}{药}{9}{⾋}
  \begin{Phonetics}{药}{yao4}[][HSK 2]
    \definition*{s.}{Sobrenome: Yao}
    \definition[片,粒,颗,瓶,服]{s.}{droga; loção; remédio; medicamento; substâncias que podem prevenir e tratar doenças, pragas ou melhorar funções corporais | certos produtos químicos com efeitos específicos}
    \definition{v.}{curar com remédios; tomar remédios para tratar doenças | matar com veneno; envenenar}
  \end{Phonetics}
\end{Entry}

\begin{Entry}{药丸}{9,3}{⾋、⼂}
  \begin{Phonetics}{药丸}{yao4wan2}
    \definition[粒]{s.}{pílula}
  \end{Phonetics}
\end{Entry}

\begin{Entry}{药水}{9,4}{⾋、⽔}
  \begin{Phonetics}{药水}{yao4 shui3}[][HSK 2]
    \definition*{s.}{Yaksu na Coreia do Norte, perto da fronteira com Liaoning e a província de Jilin}
    \definition{s.}{medicamento líquido; líquido medicinal | loção | remédio engarrafado | medicamento em forma líquida}
  \end{Phonetics}
\end{Entry}

\begin{Entry}{药片}{9,4}{⾋、⽚}
  \begin{Phonetics}{药片}{yao4 pian4}[][HSK 2]
    \definition[颗,片]{s.}{pílula; comprimido; preparações em comprimidos}
  \end{Phonetics}
\end{Entry}

\begin{Entry}{药补}{9,7}{⾋、⾐}
  \begin{Phonetics}{药补}{yao4bu3}
    \definition{s.}{suplemento dietético medicinal que ajuda a melhorar a saúde}
  \end{Phonetics}
\end{Entry}

\begin{Entry}{药典}{9,8}{⾋、⼋}
  \begin{Phonetics}{药典}{yao4dian3}
    \definition{s.}{farmacopéia}
  \end{Phonetics}
\end{Entry}

\begin{Entry}{药店}{9,8}{⾋、⼴}
  \begin{Phonetics}{药店}{yao4 dian4}[][HSK 2]
    \definition[家]{s.}{farmácia; drogaria; lojas que vendem medicamentos}
  \end{Phonetics}
\end{Entry}

\begin{Entry}{药房}{9,8}{⾋、⼾}
  \begin{Phonetics}{药房}{yao4fang2}
    \definition{s.}{farmácia | drogaria}
  \end{Phonetics}
\end{Entry}

\begin{Entry}{药物}{9,8}{⾋、⽜}
  \begin{Phonetics}{药物}{yao4 wu4}[][HSK 4]
    \definition[种]{s.}{droga; medicamento; remédio; substâncias que controlam doenças, pragas, etc.}
  \end{Phonetics}
\end{Entry}

\begin{Entry}{药品}{9,9}{⾋、⼝}
  \begin{Phonetics}{药品}{yao4pin3}[][HSK 6]
    \definition[个,些,种,类,批]{s.}{medicamentos e reagentes químicos; um termo geral para vários medicamentos e reagentes químicos}
  \end{Phonetics}
\end{Entry}

\begin{Entry}{药签}{9,13}{⾋、⽵}
  \begin{Phonetics}{药签}{yao4qian1}
    \definition{s.}{cotonete médico}
  \end{Phonetics}
\end{Entry}

\begin{Entry}{药膳}{9,16}{⾋、⾁}
  \begin{Phonetics}{药膳}{yao4shan4}
    \definition{s.}{alimentos medicamentosos; alimentos cozidos com ervas medicinais | cozinha medicinal}
  \end{Phonetics}
\end{Entry}

\begin{Entry}{药罐}{9,23}{⾋、⽸}
  \begin{Phonetics}{药罐}{yao4guan4}
    \definition{s.}{frasco de remédio; pote de remédio}
  \end{Phonetics}
\end{Entry}

%%%%%%%%%% 荷 %%%%%%%%%%
\subsection*{荷}\addcontentsline{loh}{figure}{荷}

\begin{Entry}{荷}{10}{⾋}
  \begin{Phonetics}{荷}{he2}
    \definition*{s.}{Países Baixos; Holanda, abreviação de 荷兰 | Sobrenome: He}
    \definition{s.}{lótus}
  \seealsoref{荷兰}{he2lan2}
  \end{Phonetics}
  \begin{Phonetics}{荷}{he4}
    \definition{s.}{fardo; responsabilidade}
    \definition{v.}{carregar no ombro ou nas costas | aceitar um favor, frequentemente usado em cartas para expressar cortesia}
  \end{Phonetics}
\end{Entry}

\begin{Entry}{荷兰}{10,5}{⾋、⼋}
  \begin{Phonetics}{荷兰}{he2lan2}
    \definition*{s.}{Países Baixos; Holanda}
  \end{Phonetics}
\end{Entry}

\begin{Entry}{荷花}{10,7}{⾋、⾋}
  \begin{Phonetics}{荷花}{he2hua1}[][HSK 7-9]
    \definition[朵,枝,片]{s.}{lótus; flor de lótus}
  \end{Phonetics}
\end{Entry}

%%%%%%%%%% 莎 %%%%%%%%%%
\subsection*{莎}\addcontentsline{loh}{figure}{莎}

\begin{Entry}{莎}{10}{⾋}
  \begin{Phonetics}{莎}{sha1}
    \definition{s.}{em nomes pessoais e de lugares | cigarra | fonético "sha" usado na transliteração}
  \end{Phonetics}
  \begin{Phonetics}{莎}{suo1}
  \end{Phonetics}
\end{Entry}

\begin{Entry}{莎莎舞}{10,10,14}{⾋、⾋、⾇}
  \begin{Phonetics}{莎莎舞}{sha1sha1wu3}
    \definition{s.}{salsa (dança)}
  \end{Phonetics}
\end{Entry}

%%%%%%%%%% 莫 %%%%%%%%%%
\subsection*{莫}\addcontentsline{loh}{figure}{莫}

\begin{Entry}{莫}{10}{⾋}
  \begin{Phonetics}{莫}{mo4}
    \definition*{s.}{Sobrenome: Mo}
    \definition{adv.}{não, frequentemente usado em frases imperativas | não; não pode | pode ser que; não pode ser que; é possível que}
    \definition{pron.}{nenhum; nada; ninguém; significa 没有谁 ou 没有哪一种东西}
  \seealsoref{没有哪一种东西}{mei2you3 na3 yi4 zhong3 dong1xi1}
  \seealsoref{没有谁}{mei2you3 shei2}
  \end{Phonetics}
\end{Entry}

\begin{Entry}{莫名其妙}{10,6,8,7}{⾋、⼝、⼋、⼥}
  \begin{Phonetics}{莫名其妙}{mo4ming2qi2miao4}[][HSK 7-9]
    \definition{adj.}{desconcertante; enigmático; bizarro; sem lógica; inexplicável; ninguém consegue explicar seu mistério (razão), indicando que as coisas são estranhas e incompreensíveis; o caractere 名 também é escrito como 明}
  \end{Phonetics}
\end{Entry}

\begin{Entry}{莫过于}{10,6,3}{⾋、⾡、⼆}
  \begin{Phonetics}{莫过于}{mo4guo4yu2}[][HSK 7-9]
    \definition{conj.}{nada é mais\dots do que; nada é melhor do que; nada pode superar}
  \end{Phonetics}
\end{Entry}

\begin{Entry}{莫非}{10,8}{⾋、⾮}
  \begin{Phonetics}{莫非}{mo4fei1}[][HSK 7-9]
    \definition{adv.}{pode ser que; é possível que; será que; frequentemente usado em conjunto com 不成}
  \seealsoref{不成}{bu4 cheng2}
  \end{Phonetics}
\end{Entry}

%%%%%%%%%% 莲 %%%%%%%%%%
\subsection*{莲}\addcontentsline{loh}{figure}{莲}

\begin{Entry}{莲}{10}{⾋}
  \begin{Phonetics}{莲}{lian2}
    \definition*{s.}{Sobrenome: Lian}
    \definition[粒]{s.}{lótus}
  \end{Phonetics}
\end{Entry}

\begin{Entry}{莲子}{10,3}{⾋、⼦}
  \begin{Phonetics}{莲子}{lian2zi3}[][HSK 7-9]
    \definition[颗,个]{s.}{semente de lótus; a semente de lótus tem formato oval, com um miolo verde no centro e polpa branco-leitosa; pode ser consumida e usada na medicina}
  \end{Phonetics}
\end{Entry}

\begin{Entry}{莲花}{10,7}{⾋、⾋}
  \begin{Phonetics}{莲花}{lian2hua1}
    \definition{s.}{flor de lótus | lírio aquático}
  \end{Phonetics}
\end{Entry}

\begin{Entry}{莲藕}{10,18}{⾋、⾋}
  \begin{Phonetics}{莲藕}{lian2'ou3}
    \definition{s.}{raiz de Lotus}
  \end{Phonetics}
\end{Entry}

%%%%%%%%%% 获 %%%%%%%%%%
\subsection*{获}\addcontentsline{loh}{figure}{获}

\begin{Entry}{获}{10}{⾋}
  \begin{Phonetics}{获}{huo4}[][HSK 4]
    \definition*{s.}{Sobrenome: Huo}
    \definition{v.}{capturar; pegar | obter; ganhar; colher | colher; ceifar}
  \end{Phonetics}
\end{Entry}

\begin{Entry}{获取}{10,8}{⾋、⼜}
  \begin{Phonetics}{获取}{huo4 qu3}[][HSK 4]
    \definition{v.}{adquirir; obter; ganhar; colher}
  \end{Phonetics}
\end{Entry}

\begin{Entry}{获奖}{10,9}{⾋、⼤}
  \begin{Phonetics}{获奖}{huo4 jiang3}[][HSK 4]
    \definition{v.}{ganhar prêmio; ser recompensado; ganhar um prêmio; receber um prêmio}
  \end{Phonetics}
\end{Entry}

\begin{Entry}{获胜}{10,9}{⾋、⾁}
  \begin{Phonetics}{获胜}{huo4sheng4}[][HSK 7-9]
    \definition{v.}{vencer; ser vitorioso; triunfar; alcançar a vitória}
  \end{Phonetics}
\end{Entry}

\begin{Entry}{获得}{10,11}{⾋、⼻}
  \begin{Phonetics}{获得}{huo4de2}[][HSK 4]
    \definition{v.}{adquirir; ganhar; obter; alcançar}
  \end{Phonetics}
\end{Entry}

\begin{Entry}{获悉}{10,11}{⾋、⼼}
  \begin{Phonetics}{获悉}{huo4xi1}[][HSK 7-9]
    \definition{v.}{saber (de um evento); receber notícias; ser informado}
  \end{Phonetics}
\end{Entry}

%%%%%%%%%% 菊 %%%%%%%%%%
\subsection*{菊}\addcontentsline{loh}{figure}{菊}

\begin{Entry}{菊}{11}{⾋}
  \begin{Phonetics}{菊}{ju2}
    \definition*{s.}{Sobrenome: Ju}
    \definition[朵]{s.}{crisântemo}
  \end{Phonetics}
\end{Entry}

\begin{Entry}{菊花}{11,7}{⾋、⾋}
  \begin{Phonetics}{菊花}{ju2hua1}[][HSK 7-9]
    \definition[朵,枝,把,束,棵,株]{s.}{crisântemo | Gíria: ânus; uma metáfora para o ânus (ou reto) humano}
  \end{Phonetics}
\end{Entry}

%%%%%%%%%% 菜 %%%%%%%%%%
\subsection*{菜}\addcontentsline{loh}{figure}{菜}

\begin{Entry}{菜}{11}{⾋}
  \begin{Phonetics}{菜}{cai4}[][HSK 1]
    \definition*{s.}{Sobrenome: Cai}
    \definition{adj.}{pouca habilidade; baixo nível; baixa capacidade}
    \definition[棵,个,道]{s.}{legumes; verduras; plantas que podem ser usadas como alimentos complementares | óleo de canola | prato; item ou prato do cardápio (seja de carne ou de vegetais)}
  \end{Phonetics}
\end{Entry}

\begin{Entry}{菜刀}{11,2}{⾋、⼑}
  \begin{Phonetics}{菜刀}{cai4dao1}
    \definition[把]{s.}{faca de vegetais | faca de cozinha | cutelo}
  \end{Phonetics}
\end{Entry}

\begin{Entry}{菜市场}{11,5,6}{⾋、⼱、⼟}
  \begin{Phonetics}{菜市场}{cai4shi4chang3}[][HSK 7-9]
    \definition[个,家]{s.}{mercado de alimentos; mercearia verde; mercado de produtos agrícolas; mercado de vegetais; um mercado em uma cidade ou município que vende vegetais, carne, ovos e outros alimentos não básicos}
  \end{Phonetics}
\end{Entry}

\begin{Entry}{菜单}{11,8}{⾋、⼗}
  \begin{Phonetics}{菜单}{cai4dan1}[][HSK 2]
    \definition[个,分,张]{s.}{menu; lista de pratos | menu (para computadores); lista utilizada para selecionar várias operações diferentes}
  \end{Phonetics}
\end{Entry}

%%%%%%%%%% 菠 %%%%%%%%%%
\subsection*{菠}\addcontentsline{loh}{figure}{菠}

\begin{Entry}{菠}{11}{⾋}
  \begin{Phonetics}{菠}{bo1}
    \definition{s.}{espinafre}
  \end{Phonetics}
\end{Entry}

\begin{Entry}{菠菜}{11,11}{⾋、⾋}
  \begin{Phonetics}{菠菜}{bo1cai4}
    \definition[棵]{s.}{espinafre}
  \end{Phonetics}
\end{Entry}

%%%%%%%%%% 菩 %%%%%%%%%%
\subsection*{菩}\addcontentsline{loh}{figure}{菩}

\begin{Entry}{菩}{11}{⾋}
  \begin{Phonetics}{菩}{pu2}
    \definition*[个]{s.}{Buda; Bodhisattva | Estátua de Bodhisattva}
  \end{Phonetics}
\end{Entry}

\begin{Entry}{菩萨}{11,11}{⾋、⾋}
  \begin{Phonetics}{菩萨}{pu2sa4}[][HSK 7-9]
    \definition*{s.}{Bodhisattva Guanyin; Buda; Deus}
    \definition[个]{s.}{divindade; deus | Figurativo: pessoa bondosa | ídolo budista}
  \end{Phonetics}
\end{Entry}

%%%%%%%%%% 菱 %%%%%%%%%%
\subsection*{菱}\addcontentsline{loh}{figure}{菱}

\begin{Entry}{菱}{11}{⾋}
  \begin{Phonetics}{菱}{ling2}
    \definition{s.}{maruca; caltrop aquático; castanha d'água}
  \end{Phonetics}
\end{Entry}

\begin{Entry}{菱角}{11,7}{⾋、⾓}
  \begin{Phonetics}{菱角}{ling2jiao5}
    \definition{s.}{castanha d'água}
  \end{Phonetics}
\end{Entry}

%%%%%%%%%% 萌 %%%%%%%%%%
\subsection*{萌}\addcontentsline{loh}{figure}{萌}

\begin{Entry}{萌}{11}{⾋}
  \begin{Phonetics}{萌}{meng2}
    \definition*{s.}{Sobrenome: Meng}
    \definition{s.}{broto; rebento | Arcaico: o povo comum}
    \definition{v.}{(plantas) brotar; surgir; brotar; germinar | começar; surgir; ocorrer; emergir}
  \end{Phonetics}
\end{Entry}

\begin{Entry}{萌发}{11,5}{⾋、⼜}
  \begin{Phonetics}{萌发}{meng2fa1}[][HSK 7-9]
    \definition{v.}{brotar; germinar; germinação de sementes ou esporos | emergir; vir à tona; isso se refere metaforicamente ao início de um determinado pensamento, ideia ou sentimento}
  \end{Phonetics}
\end{Entry}

\begin{Entry}{萌芽}{11,7}{⾋、⾋}
  \begin{Phonetics}{萌芽}{meng2ya2}[][HSK 7-9]
    \definition{s.}{semente; germe; rudimento; uma metáfora para algo novo e ainda não totalmente desenvolvido}
    \definition{v.}{brotar; germinar; desabrochar; o brotar de plantas é uma metáfora para algo que está apenas começando a acontecer}
  \end{Phonetics}
\end{Entry}

%%%%%%%%%% 萍 %%%%%%%%%%
\subsection*{萍}\addcontentsline{loh}{figure}{萍}

\begin{Entry}{萍}{11}{⾋}
  \begin{Phonetics}{萍}{ping2}
    \definition{s.}{lentilha-d'água}
  \end{Phonetics}
\end{Entry}

\begin{Entry}{萍水相逢}{11,4,9,10}{⾋、⽔、⽬、⾡}
  \begin{Phonetics}{萍水相逢}{ping2shui3-xiang1feng2}[][HSK 7-9]
    \definition{expr.}{(estranhos) encontram-se por acaso como tufos de lentilha-d'água à deriva; um conhecido casual; realizar uma reunião informal e temporária}
  \end{Phonetics}
\end{Entry}

%%%%%%%%%% 萝 %%%%%%%%%%
\subsection*{萝}\addcontentsline{loh}{figure}{萝}

\begin{Entry}{萝}{11}{⾋}
  \begin{Phonetics}{萝}{luo2}
    \definition[个]{s.}{plantas trepadeiras; videira}
  \end{Phonetics}
\end{Entry}

\begin{Entry}{萝卜}{11,2}{⾋、⼘}
  \begin{Phonetics}{萝卜}{luo2bo5}[][HSK 7-9]
    \definition[个,根]{s.}{rabanete; nabo | planta de rabanete}
  \end{Phonetics}
\end{Entry}

%%%%%%%%%% 营 %%%%%%%%%%
\subsection*{营}\addcontentsline{loh}{figure}{营}

\begin{Entry}{营}{11}{⾋}
  \begin{Phonetics}{营}{ying2}
    \definition*{s.}{Sobrenome: Ying}
    \definition{s.}{acampamento; quartel; onde o exército está estacionado | batalhão; unidades militares}
    \definition{v.}{procurar | operar; executar; gerenciar}
  \end{Phonetics}
\end{Entry}

\begin{Entry}{营业}{11,5}{⾋、⼀}
  \begin{Phonetics}{营业}{ying2ye4}[][HSK 4]
    \definition{v.}{fazer negócios; estar aberto para negócios}
  \end{Phonetics}
\end{Entry}

\begin{Entry}{营养}{11,9}{⾋、⼋}
  \begin{Phonetics}{营养}{ying2yang3}[][HSK 3]
    \definition[种]{s.}{nutrição; alimentação; a função do organismo de absorver as substâncias necessárias do meio externo para manter atividades vitais, como crescimento e desenvolvimento | nutrição; alimentação; ato ou processo de fornecer nutrição}
  \end{Phonetics}
\end{Entry}

%%%%%%%%%% 落 %%%%%%%%%%
\subsection*{落}\addcontentsline{loh}{figure}{落}

\begin{Entry}{落}{12}{⾋}
  \begin{Phonetics}{落}{la4}[][HSK 5]
    \definition{v.}{deixar de fora; estar ausente | deixar para trás; esquecer de trazer; deixar algo em algum lugar e esquecer de levar| ficar para trás (ou cair); não conseguir acompanhar}
  \end{Phonetics}
  \begin{Phonetics}{落}{lao4}
    \definition{v.}{cair; cair de uma altura elevada | se abaixar; descer; ir para baixo | permanecer; fazer uma parada; deixar para trás | obter; ter; receber}
  \end{Phonetics}
  \begin{Phonetics}{落}{luo4}[][HSK 4]
    \definition*{s.}{Sobrenome: Luo}
    \definition{s.}{paradeiro; lugar para ficar; local de descanso | assentamento; local de reunião | parte curta; área pequena; refere-se a um pequeno lugar ou área}
    \definition{v.}{cair; cair de uma altura elevada | se abaixar; descer; ir para baixo | abaixar; deixar cair (ou descer); fazer descer | afundar; declinar; descer | ficar para trás; ficar para trás ou ficar de fora | permanecer; fazer uma parada; deixar para trás | cair sobre; repousar com | obter; ter; receber | anotar; escrever no papel | cair em; entrar em; ficar preso}
  \end{Phonetics}
\end{Entry}

\begin{Entry}{落下}{12,3}{⾋、⼀}
  \begin{Phonetics}{落下}{luo4xia4}[][HSK 7-9]
    \definition{v.}{soltar; um objeto cai de uma altura para um lugar baixo | cair; algo é movido de um lugar alto para um lugar baixo | atingir o solo (um projétil)}
  \end{Phonetics}
\end{Entry}

\begin{Entry}{落户}{12,4}{⾋、⼾}
  \begin{Phonetics}{落户}{luo4/hu4}[][HSK 7-9]
    \definition{v.+compl.}{estabelecer-se | registrar-se; solicitar uma autorização de residência}
  \end{Phonetics}
\end{Entry}

\begin{Entry}{落日}{12,4}{⾋、⽇}
  \begin{Phonetics}{落日}{luo4ri4}
    \definition{s.}{pôr do sol}
  \end{Phonetics}
\end{Entry}

\begin{Entry}{落后}{12,6}{⾋、⼝}
  \begin{Phonetics}{落后}{luo4hou4}[][HSK 3]
    \definition{adj.}{atrasado; trabalho em atraso, nível de desenvolvimento ou grau de reconhecimento (em oposição a 进步)}
    \definition{v.}{ficar para trás; ficar atrasado; ficar para trás em relação aos outros durante o avanço ou o progresso do trabalho}
  \seealsoref{进步}{jin4bu4}
  \end{Phonetics}
\end{Entry}

\begin{Entry}{落地}{12,6}{⾋、⼟}
  \begin{Phonetics}{落地}{luo4/di4}[][HSK 7-9]
    \definition{v.+compl.}{cair no chão; (objeto) cai de uma altura até o chão | pôr em prática; isso se refere metaforicamente ao fim de uma conversa ou ao início da implementação de um plano ou política | nascer, o nascimento de um bebê}
  \end{Phonetics}
\end{Entry}

\begin{Entry}{落汤鸡}{12,6,7}{⾋、⽔、⿃}
  \begin{Phonetics}{落汤鸡}{luo4tang1ji1}
    \definition{s.}{uma pessoa que parece encharcada e acamada| sofrimento profundo}
  \end{Phonetics}
\end{Entry}

\begin{Entry}{落花生}{12,7,5}{⾋、⾋、⽣}
  \begin{Phonetics}{落花生}{luo4 hua1 sheng1}
    \definition{s.}{amendoim | noz de macaco}
  \end{Phonetics}
\end{Entry}

\begin{Entry}{落实}{12,8}{⾋、⼧}
  \begin{Phonetics}{落实}{luo4shi2}[][HSK 5]
    \definition{adj.}{sentimento de tranquilidade; (humor) estável; seguro}
    \definition{v.}{implementar; ser praticável; tornar os planos, políticas, medidas, etc. específicos e compreensíveis, de modo a que possam ser realizados | implementar; colocar em prática; pôr em prática significa que os planos, políticas e medidas são específicos e claros, e podem ser realizados}
  \end{Phonetics}
\end{Entry}

\begin{Entry}{落差}{12,9}{⾋、⼯}
  \begin{Phonetics}{落差}{luo4cha1}[][HSK 7-9]
    \definition{s.}{queda de elevação; desnível; a diferença no nível da água causada por mudanças na elevação do leito do rio; por exemplo, se a elevação da superfície da água no local A for de 20 metros e no local B for de 18 metros, a diferença de elevação será de 2 metros | lacuna; disparidade; discrepância; comparação metafórica de lacunas ou diferenças}
  \end{Phonetics}
\end{Entry}

\begin{Entry}{落魄}{12,14}{⾋、⿁}
  \begin{Phonetics}{落魄}{luo4bo2}
    \definition{adj.}{pobre, indigente; miserável; empobrecidos e desamparados, sem meios de subsistência | de espírito livre; natural; relaxado; de mente aberta}
  \end{Phonetics}
  \begin{Phonetics}{落魄}{luo4po4}
    \definition{adj.}{desencorajado; abatido; desanimado; sem rumo | desinibido; de espírito livre; tranquilo; ousado e desinibido | em pânico; assustado; petrificado; aterrorizado}
  \end{Phonetics}
\end{Entry}

%%%%%%%%%% 葡 %%%%%%%%%%
\subsection*{葡}\addcontentsline{loh}{figure}{葡}

\begin{Entry}{葡}{12}{⾋}
  \begin{Phonetics}{葡}{pu2}
    \definition*{s.}{Portugal, abreviação de 葡萄牙}
  \seealsoref{葡萄牙}{pu2tao2ya2}
  \end{Phonetics}
\end{Entry}

\begin{Entry}{葡文}{12,4}{⾋、⽂}
  \begin{Phonetics}{葡文}{pu2wen2}
    \definition{s.}{português, língua portuguesa}
  \seealsoref{葡萄牙文}{pu2tao2ya2wen2}
  \end{Phonetics}
\end{Entry}

\begin{Entry}{葡汉词典}{12,5,7,8}{⾋、⽔、⾔、⼋}
  \begin{Phonetics}{葡汉词典}{pu2-han4 ci2dian3}
    \definition{s.}{dicionário português-chinês}
  \seealsoref{汉葡词典}{han4-pu2 ci2dian3}
  \end{Phonetics}
\end{Entry}

\begin{Entry}{葡语}{12,9}{⾋、⾔}
  \begin{Phonetics}{葡语}{pu2yu3}
    \definition{s.}{português, língua portuguesa}
  \seealsoref{葡萄牙语}{pu2tao2ya2yu3}
  \end{Phonetics}
\end{Entry}

\begin{Entry}{葡萄}{12,11}{⾋、⾋}
  \begin{Phonetics}{葡萄}{pu2tao5}[][HSK 5]
    \definition[串,颗,粒,棵,种]{s.}{parreira | uva}
  \end{Phonetics}
\end{Entry}

\begin{Entry}{葡萄牙}{12,11,4}{⾋、⾋、⽛}
  \begin{Phonetics}{葡萄牙}{pu2tao2ya2}
    \definition{s.}{Portugal}
  \end{Phonetics}
\end{Entry}

\begin{Entry}{葡萄牙文}{12,11,4,4}{⾋、⾋、⽛、⽂}
  \begin{Phonetics}{葡萄牙文}{pu2tao2ya2wen2}
    \definition{s.}{português, língua portuguesa}
  \seealsoref{葡文}{pu2wen2}
  \end{Phonetics}
\end{Entry}

\begin{Entry}{葡萄牙语}{12,11,4,9}{⾋、⾋、⽛、⾔}
  \begin{Phonetics}{葡萄牙语}{pu2tao2ya2yu3}
    \definition{s.}{português, língua portuguesa}
  \seealsoref{葡语}{pu2yu3}
  \end{Phonetics}
\end{Entry}

\begin{Entry}{葡萄酒}{12,11,10}{⾋、⾋、⾣}
  \begin{Phonetics}{葡萄酒}{pu2 tao2 jiu3}[][HSK 5]
    \definition[瓶,杯,口,桶]{s.}{vinho (de uva)}
  \end{Phonetics}
\end{Entry}

%%%%%%%%%% 董 %%%%%%%%%%
\subsection*{董}\addcontentsline{loh}{figure}{董}

\begin{Entry}{董}{12}{⾋}
  \begin{Phonetics}{董}{dong3}
    \definition*{s.}{Sobrenome: Dong}
    \definition{s.}{diretor; administrador}
    \definition{v.}{Literário: dirigir; supervisionar; supervisionar}
  \end{Phonetics}
\end{Entry}

\begin{Entry}{董事}{12,8}{⾋、⼅}
  \begin{Phonetics}{董事}{dong3shi4}[][HSK 7-9]
    \definition[个,位,名]{s.}{administrador; diretor}
  \end{Phonetics}
\end{Entry}

\begin{Entry}{董事长}{12,8,4}{⾋、⼅、⾧}
  \begin{Phonetics}{董事长}{dong3shi4zhang3}[][HSK 7-9]
    \definition[个,位,名]{s.}{presidente; presidente do conselho; \emph{chairman}; o principal responsável pelo conselho de administração}
  \end{Phonetics}
\end{Entry}

\begin{Entry}{董事会}{12,8,6}{⾋、⼅、⼈}
  \begin{Phonetics}{董事会}{dong3shi4hui4}[][HSK 7-9]
    \definition[届]{s.}{conselho de administração (numa empresa); conselho de curadores (numa instituição de ensino); órgão decisório de uma sociedade anônima, escola ou grupo}
  \end{Phonetics}
\end{Entry}

%%%%%%%%%% 葫 %%%%%%%%%%
\subsection*{葫}\addcontentsline{loh}{figure}{葫}

\begin{Entry}{葫}{12}{⾋}
  \begin{Phonetics}{葫}{hu2}
    \definition{s.}{cabaça}
  \end{Phonetics}
\end{Entry}

\begin{Entry}{葫芦}{12,7}{⾋、⾋}
  \begin{Phonetics}{葫芦}{hu2lu5}
    \definition{adj.}{confuso}
    \definition{s.}{cabaça | termo genérico para bloco e equipamento (ou partes dele)}
  \end{Phonetics}
\end{Entry}

%%%%%%%%%% 葬 %%%%%%%%%%
\subsection*{葬}\addcontentsline{loh}{figure}{葬}

\begin{Entry}{葬}{12}{⾋}
  \begin{Phonetics}{葬}{zang4}
    \definition{v.}{enterrar (os mortos) | sepultar}
  \end{Phonetics}
\end{Entry}

%%%%%%%%%% 葱 %%%%%%%%%%
\subsection*{葱}\addcontentsline{loh}{figure}{葱}

\begin{Entry}{葱}{12}{⾋}
  \begin{Phonetics}{葱}{cong1}[][HSK 7-9]
    \definition{adj.}{verde; turquesa}
    \definition[根,把,捆]{s.}{cebola; cebolinha}
  \end{Phonetics}
\end{Entry}

%%%%%%%%%% 葵 %%%%%%%%%%
\subsection*{葵}\addcontentsline{loh}{figure}{葵}

\begin{Entry}{葵}{12}{⾋}
  \begin{Phonetics}{葵}{kui2}
    \definition*{s.}{Sobrenome: Kui}
    \definition[朵]{s.}{certas ervas de flores grandes}
  \end{Phonetics}
\end{Entry}

\begin{Entry}{葵花}{12,7}{⾋、⾋}
  \begin{Phonetics}{葵花}{kui2hua1}
    \definition{s.}{girassol (flor)}
  \end{Phonetics}
\end{Entry}

%%%%%%%%%% 蒙 %%%%%%%%%%
\subsection*{蒙}\addcontentsline{loh}{figure}{蒙}

\begin{Entry}{蒙}{13}{⾋}
  \begin{Phonetics}{蒙}{meng1}[][HSK 6]
    \definition{adj.}{inconsciente; sem sentido;  em coma | confuso; perplexo}
    \definition{v.}{enganar; enganar; trapacear; iludir; trair | fazer um palpite ousado; dar um palpite ousado; arriscar-se}
  \end{Phonetics}
  \begin{Phonetics}{蒙}{meng2}[][HSK 6]
    \definition*{s.}{Sobrenome: Meng}
    \definition{adj.}{ignorância; analfabetismo; falta de instrução | nebuloso; aparência pequena e pouco clara, como chuva ou neblina}
    \definition{s.}{aberto; inicial}
    \definition{v.}{cobrir; espalhar | receber apoio | receber; encontrar-se com; encontrar-se; palavras respeitosas; expressam os benefícios recebidos de outros | sofrer; incorrer}
  \end{Phonetics}
  \begin{Phonetics}{蒙}{meng3}
    \definition{s.}{grupo étnico mongol; mongol}
  \end{Phonetics}
\end{Entry}

\begin{Entry}{蒙面}{13,9}{⾋、⾯}
  \begin{Phonetics}{蒙面}{meng2mian4}
    \definition{adj.}{descarado | desavergonhado | mascarado}
    \definition{v.}{cobrir o rosto | usar uma máscara}
  \end{Phonetics}
\end{Entry}

%%%%%%%%%% 蓝 %%%%%%%%%%
\subsection*{蓝}\addcontentsline{loh}{figure}{蓝}

\begin{Entry}{蓝}{13}{⾋}
  \begin{Phonetics}{蓝}{lan2}[][HSK 2]
    \definition*{s.}{Sobrenome: Lan}
    \definition{adj.}{azul}
    \definition{s.}{planta índigo; anil | plantas azuis; refere-se a certas plantas que podem ser usadas como corante azul ou certas plantas cujas folhas são azul-esverdeadas}
  \end{Phonetics}
\end{Entry}

\begin{Entry}{蓝色}{13,6}{⾋、⾊}
  \begin{Phonetics}{蓝色}{lan2 se4}[][HSK 2]
    \definition[抹,片,缕,团,块]{s.}{cor azul}
  \end{Phonetics}
\end{Entry}

\begin{Entry}{蓝图}{13,8}{⾋、⼞}
  \begin{Phonetics}{蓝图}{lan2tu2}[][HSK 7-9]
    \definition[张,幅]{s.}{\emph{blueprint}; projeto; desenhos impressos em papel fotossensível; como aparecem com fundo azul e linhas brancas ou fundo branco com linhas roxas após a exposição à luz, são chamados de ``\emph{blueprint}'' | plano; esquema; metáfora para planejamento ou esquema}
  \end{Phonetics}
\end{Entry}

\begin{Entry}{蓝领}{13,11}{⾋、⾴}
  \begin{Phonetics}{蓝领}{lan2 ling3}[][HSK 6]
    \definition[名,位,个]{s.}{trabalhador braçal}
  \end{Phonetics}
\end{Entry}

%%%%%%%%%% 蓬 %%%%%%%%%%
\subsection*{蓬}\addcontentsline{loh}{figure}{蓬}

\begin{Entry}{蓬}{13}{⾋}
  \begin{Phonetics}{蓬}{peng2}
    \definition*{s.}{Sobrenome: Peng}
    \definition{adj.}{fofo; desgrenhado}
    \definition{clas.}{touceira; aglomerado; utilizado para flores e plantas exuberantes, etc.}
    \definition{s.}{pulga amarga; erva-de-pulga; absinto selvagem; erva-de-santa-luzia}
  \end{Phonetics}
\end{Entry}

\begin{Entry}{蓬勃}{13,9}{⾋、⼒}
  \begin{Phonetics}{蓬勃}{peng2bo2}[][HSK 7-9]
    \definition{adj.}{vigoroso; próspero; florescente}
  \end{Phonetics}
\end{Entry}

%%%%%%%%%% 蔓 %%%%%%%%%%
\subsection*{蔓}\addcontentsline{loh}{figure}{蔓}

\begin{Entry}{蔓}{14}{⾋}
  \begin{Phonetics}{蔓}{man2}
    \definition{s.}{couve-chinesa | nabo}
  \end{Phonetics}
  \begin{Phonetics}{蔓}{man4}
    \definition{s.}{uma videira com gavinhas; caule fino que não consegue ficar em pé}
    \definition{v.}{rastejar; espalhar; estender}
  \end{Phonetics}
  \begin{Phonetics}{蔓}{wan4}
    \definition*{s.}{Sobrenome: Wan}
    \definition{s.}{uma videira com gavinhas; caule fino que não consegue ficar em pé}
  \end{Phonetics}
\end{Entry}

\begin{Entry}{蔓延}{14,6}{⾋、⼵}
  \begin{Phonetics}{蔓延}{man4yan2}[][HSK 7-9]
    \definition{v.}{espalhar; esticar; estender | infestar; espalhar; essa metáfora descreve coisas que se estendem e se expandem para fora como trepadeiras rastejantes}
  \end{Phonetics}
\end{Entry}

\begin{Entry}{蔓草}{14,9}{⾋、⾋}
  \begin{Phonetics}{蔓草}{man4cao3}
    \definition{s.}{videira | trepadeira}
  \end{Phonetics}
\end{Entry}

%%%%%%%%%% 箱 %%%%%%%%%%
\subsection*{箱}\addcontentsline{loh}{figure}{箱}

\begin{Entry}{箱}{15}{⾋}
  \begin{Phonetics}{箱}{xiang1}[][HSK 4]
    \definition{s.}{caixa; estojo; baú | qualquer coisa no formato de caixa}
  \end{Phonetics}
\end{Entry}

\begin{Entry}{箱子}{15,3}{⾋、⼦}
  \begin{Phonetics}{箱子}{xiang1 zi5}[][HSK 4]
    \definition[个,只]{s.}{baú; caixa; estojo; maleta; pasta executiva}
  \end{Phonetics}
\end{Entry}

%%%%%%%%%% 蔬 %%%%%%%%%%
\subsection*{蔬}\addcontentsline{loh}{figure}{蔬}

\begin{Entry}{蔬}{15}{⾋}
  \begin{Phonetics}{蔬}{shu1}
    \definition{s.}{vegetais}
  \end{Phonetics}
\end{Entry}

\begin{Entry}{蔬菜}{15,11}{⾋、⾋}
  \begin{Phonetics}{蔬菜}{shu1cai4}[][HSK 5]
    \definition[样,种]{s.}{verduras; legumes; vegetais; ervas que podem ser usadas na culinária}
  \end{Phonetics}
\end{Entry}

%%%%%%%%%% 蕃 %%%%%%%%%%
\subsection*{蕃}\addcontentsline{loh}{figure}{蕃}

\begin{Entry}{蕃}{15}{⾋}
  \begin{Phonetics}{蕃}{bo1}
    \definition[种]{s.}{estrangeiros}
  \end{Phonetics}
  \begin{Phonetics}{蕃}{fan1}
    \definition[种]{s.}{estrangeiros; aborígenes}
  \end{Phonetics}
  \begin{Phonetics}{蕃}{fan2}
    \definition{adj.}{exuberante; próspero}
    \definition{v.}{multiplicar; proliferar}
  \end{Phonetics}
\end{Entry}

\begin{Entry}{蕃茄}{15,8}{⾋、⾋}
  \begin{Phonetics}{蕃茄}{fan1 qie2}
    \variantof{番茄}
  \end{Phonetics}
\end{Entry}

%%%%%%%%%% 蕹 %%%%%%%%%%
\subsection*{蕹}\addcontentsline{loh}{figure}{蕹}

\begin{Entry}{蕹}{16}{⾋}
  \begin{Phonetics}{蕹}{weng4}
    \definition{s.}{espinafre-d’água ou \emph{ong choy}, usado como vegetal no sul da China e no sudeste da Ásia}
  \end{Phonetics}
\end{Entry}

\begin{Entry}{蕹菜}{16,11}{⾋、⾋}
  \begin{Phonetics}{蕹菜}{weng4cai4}
    \definition{s.}{espinafre aquático | \emph{ong choy} | repolho do pântano | convolvulus aquático | glória-da-manhã aquática}
  \seealsoref{空心菜}{kong1xin1cai4}
  \end{Phonetics}
\end{Entry}

%%%%%%%%%% 薄 %%%%%%%%%%
\subsection*{薄}\addcontentsline{loh}{figure}{薄}

\begin{Entry}{薄}{16}{⾋}
  \begin{Phonetics}{薄}{bao2}[][HSK 4]
    \definition{adj.}{fino; frágil | frio; indiferente; carente de calor | leve; fraco | pobre; infértil}
  \end{Phonetics}
  \begin{Phonetics}{薄}{bo2}
    \definition*{s.}{Sobrenome: Bo}
    \definition{adj.}{pequeno; leve; magro | mau; cruel; mesquinho | frívolo; fútil; não solene | fraco; frágil}
    \definition{v.}{desprezar; tratar com desprezo; menosprezar | aproximar-se}
  \end{Phonetics}
  \begin{Phonetics}{薄}{bo4}
    \definition{s.}{menta; uma erva perene com aroma refrescante nos caules e folhas}
  \end{Phonetics}
\end{Entry}

\begin{Entry}{薄弱}{16,10}{⾋、⼸}
  \begin{Phonetics}{薄弱}{bo2ruo4}[][HSK 5]
    \definition{adj.}{fraco; frágil; não é firme; não é sólido}
  \end{Phonetics}
\end{Entry}

%%%%%%%%%% 薛 %%%%%%%%%%
\subsection*{薛}\addcontentsline{loh}{figure}{薛}

\begin{Entry}{薛}{16}{⾋}
  \begin{Phonetics}{薛}{xue1}
    \definition*{s.}{Estado vassalo durante a Dinastia Zhou (1046-256 a.C.) | Sobrenome: Xue}
    \definition{s.}{erva semelhante ao absinto (clássico)}
  \end{Phonetics}
\end{Entry}

\begin{Entry}{薛稷}{16,15}{⾋、⽲}
  \begin{Phonetics}{薛稷}{xue1 ji4}
    \definition*{s.}{Xue Ji (649-713), um dos quatro grandes calígrafos do início da dinastia Tang, 唐初四大家}
  \seealsoref{唐初四大家}{tang2 chu1 si4 da4jia1}
  \end{Phonetics}
\end{Entry}

%%%%%%%%%% 薪 %%%%%%%%%%
\subsection*{薪}\addcontentsline{loh}{figure}{薪}

\begin{Entry}{薪}{16}{⾋}
  \begin{Phonetics}{薪}{xin1}
    \definition{s.}{lenha; combustível | salário; ordenado; pagamento}
  \end{Phonetics}
\end{Entry}

\begin{Entry}{薪水}{16,4}{⾋、⽔}
  \begin{Phonetics}{薪水}{xin1shui3}[][HSK 6]
    \definition[份,笔]{s.}{pagamento; salário; ordenados; dinheiro ou bens pagos regularmente aos trabalhadores como compensação pelo seu trabalho}
  \end{Phonetics}
\end{Entry}

%%%%%%%%%% 薯 %%%%%%%%%%
\subsection*{薯}\addcontentsline{loh}{figure}{薯}

\begin{Entry}{薯}{16}{⾋}
  \begin{Phonetics}{薯}{shu3}
    \definition{s.}{batata | inhame}
  \end{Phonetics}
\end{Entry}

\begin{Entry}{薯片}{16,4}{⾋、⽚}
  \begin{Phonetics}{薯片}{shu3 pian4}[][HSK 6]
    \definition{s.}{batatas fritas (\emph{chips}); batatas fritas crocantes ; flocos finos feitos de batatas}
  \end{Phonetics}
\end{Entry}

\begin{Entry}{薯条}{16,7}{⾋、⽊}
  \begin{Phonetics}{薯条}{shu3 tiao2}[][HSK 6]
    \definition{s.}{batatas fritas (palito)}
  \end{Phonetics}
\end{Entry}

%%%%%%%%%% 藏 %%%%%%%%%%
\subsection*{藏}\addcontentsline{loh}{figure}{藏}

\begin{Entry}{藏}{17}{⾋}
  \begin{Phonetics}{藏}{cang2}[][HSK 6]
    \definition*{s.}{Sobrenome: Cang}
    \definition{v.}{esconder; ocultar; esconder da vista | armazenar; coletar; colocar de lado}
  \end{Phonetics}
  \begin{Phonetics}{藏}{zang4}
    \definition*{s.}{Escrituras budistas ou taoístas; um termo geral para clássicos budistas ou taoístas | Região Autônoma do Tibete, 西藏}
    \definition{s.}{depósito; local de armazenamento; armazém; local onde grandes quantidades de coisas são armazenadas | os tibetanos, 藏族; grupo étnico Zang (ou tibetano)}
  \seealsoref{西藏}{xi1zang4}
  \seealsoref{藏族}{zang4zu2}
  \end{Phonetics}
\end{Entry}

\begin{Entry}{藏身}{17,7}{⾋、⾝}
  \begin{Phonetics}{藏身}{cang2shen1}[][HSK 7-9]
    \definition{v.}{esconder-se; esconder | abrigar-se; estabelecer-se}
  \end{Phonetics}
\end{Entry}

\begin{Entry}{藏品}{17,9}{⾋、⼝}
  \begin{Phonetics}{藏品}{cang2pin3}[][HSK 7-9]
    \definition[件]{s.}{artigos coletados; coleção | item de colecionador | peça de museu | objeto precioso}
  \end{Phonetics}
\end{Entry}

\begin{Entry}{藏匿}{17,10}{⾋、⼖}
  \begin{Phonetics}{藏匿}{cang2ni4}[][HSK 7-9]
    \definition{v.}{esconder; esconder-se | abrigar}
  \end{Phonetics}
\end{Entry}

\begin{Entry}{藏族}{17,11}{⾋、⽅}
  \begin{Phonetics}{藏族}{zang4zu2}
    \definition*{s.}{Etnia Zang (ou tibetana); Os Zangs (ou tibetanos) , distribuídos pela Região Autônoma do Tibete e pelas províncias de Qinghai, Sichuan, Gansu e Yunnan}
  \end{Phonetics}
\end{Entry}

%%%%%%%%%% 蘑 %%%%%%%%%%
\subsection*{蘑}\addcontentsline{loh}{figure}{蘑}

\begin{Entry}{蘑}{19}{⾋}
  \begin{Phonetics}{蘑}{mo2}
    \definition{s.}{cogumelo}
  \end{Phonetics}
\end{Entry}

\begin{Entry}{蘑菇}{19,11}{⾋、⾋}
  \begin{Phonetics}{蘑菇}{mo2gu5}[][HSK 7-9]
    \definition[个,朵,斤,种]{s.}{cogumelo; termo genérico para fungos em forma de guarda-chuva; referindo-se especificamente a cogumelos champignon ou cogumelos shiitake}
    \definition{v.}{afligir; importunar; insistir em | demorar; enrolar; movimentar-se lenta e arrastadamente}
  \end{Phonetics}
\end{Entry}

%%%%% EOF %%%%%

