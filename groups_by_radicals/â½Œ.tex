%%%
%%% Radical "⽌"
%%%

\section*{Radical 77: ``⽌''}\addcontentsline{toc}{section}{Radical 77: ⽌}

\begin{entry}{止}{4}{⽌}
  \begin{phonetics}{止}{zhi3}[][HSK 6]
    \definition{adv.}{somente; apenas}
    \definition{v.}{parar; cortar; bloquear | finalizar; fechar (até o prazo final) | não ser permitido}
  \end{phonetics}
\end{entry}

\begin{entry}{正}{5}{⽌}
  \begin{phonetics}{正}{zheng1}
    \definition{s.}{o primeiro mês do ano lunar; a primeira lua}
  \end{phonetics}
  \begin{phonetics}{正}{zheng4}[][HSK 1,3]
    \definition*{s.}{Sobrenome Zheng}
    \definition{adj.}{reto; ereto; vertical | principal; posicionado no meio | direito; anverso | honesto; íntegro; justo | puro; sem mistura (de cor ou sabor) | regular; padronizado; de acordo com a lei; correto | chefe; comandante; diretor | regular; as laterais e os ângulos do gráfico têm comprimentos e tamanhos iguais | positivo; (matemática), significa maior que zero; (física) significa perda de elétrons (oposto de 负) | exato; preciso; usado para indicar tempo, refere-se ao momento exato ou ao ponto médio de um período}
    \definition{adv.}{apenas; certo; exatamente; precisamente | agora mesmo; neste momento; indica a continuidade de uma ação ou a permanência de um estado}
    \definition{v.}{definir (colocar) corretamente; alinhar; endireitar | ajustar; corrigir; retificar}
  \seealsoref{负}{fu4}
  \end{phonetics}
\end{entry}

\begin{entry}{正义}{5,3}{⽌、⼂}
  \begin{phonetics}{正义}{zheng4yi4}[][HSK 5]
    \definition{adj.}{justo; íntegro}
    \definition{s.}{justiça; o que é certo; o que é benéfico para o povo | (frequentemente em títulos de livros) interpretação ortodoxa ou retificada (de textos antigos)}
  \end{phonetics}
\end{entry}

\begin{entry}{正正}{5,5}{⽌、⽌}
  \begin{phonetics}{正正}{zheng4zheng4}
    \definition{adv.}{na hora certa | ordenadamente}
  \end{phonetics}
\end{entry}

\begin{entry}{正在}{5,6}{⽌、⼟}
  \begin{phonetics}{正在}{zheng4zai4}[][HSK 1]
    \definition{adv.}{em processo de; em andamento; indica que uma ação está em andamento ou que uma situação está em curso.}
    \definition{v.}{estar a + {v.inf.} | estar + {v.ger.}}
  \end{phonetics}
\end{entry}

\begin{entry}{正好}{5,6}{⽌、⼥}
  \begin{phonetics}{正好}{zheng4hao3}[][HSK 2]
    \definition{adj.}{na hora certa; na hora certa; o suficiente}
    \definition{adv.}{acontecer com; chance de; como acontece}
  \end{phonetics}
\end{entry}

\begin{entry}{正如}{5,6}{⽌、⼥}
  \begin{phonetics}{正如}{zheng4 ru2}[][HSK 5]
    \definition{adv.}{exatamente como; assim como}
  \end{phonetics}
\end{entry}

\begin{entry}{正式}{5,6}{⽌、⼷}
  \begin{phonetics}{正式}{zheng4shi4}[][HSK 3]
    \definition{adj.}{formal; oficial; descreve uma atmosfera séria, atitudes ou comportamentos que não são fáceis ou descontraídos | formal; oficial; descreve o cumprimento de determinados trâmites e procedimentos}
  \end{phonetics}
\end{entry}

\begin{entry}{正宗}{5,8}{⽌、⼧}
  \begin{phonetics}{正宗}{zheng4zong1}
    \definition{adj.}{autêntico | genuíno | \emph{old school} | (fig.) tradicional}
  \end{phonetics}
\end{entry}

\begin{entry}{正版}{5,8}{⽌、⽚}
  \begin{phonetics}{正版}{zheng4 ban3}[][HSK 5]
    \definition{s.}{versão genuína; versão autorizada; versão publicada e distribuída oficialmente por uma editora legal (em contraste com a 盗版)}
  \seealsoref{盗版}{dao4ban3}
  \end{phonetics}
\end{entry}

\begin{entry}{正规}{5,8}{⽌、⾒}
  \begin{phonetics}{正规}{zheng4gui1}[][HSK 5]
    \definition{adj.}{normal; regular; padrão; está em conformidade com padrões formalmente definidos ou geralmente reconhecidos}
  \end{phonetics}
\end{entry}

\begin{entry}{正是}{5,9}{⽌、⽇}
  \begin{phonetics}{正是}{zheng4 shi4}[][HSK 2]
    \definition{v.}{ser precisamente; ser exatamente}
  \end{phonetics}
\end{entry}

\begin{entry}{正常}{5,11}{⽌、⼱}
  \begin{phonetics}{正常}{zheng4chang2}[][HSK 2]
    \definition{adj.}{normal; regular; conforma-se com regras ou circunstâncias gerais}
  \end{phonetics}
\end{entry}

\begin{entry}{正确}{5,12}{⽌、⽯}
  \begin{phonetics}{正确}{zheng4que4}[][HSK 2]
    \definition{adj.}{correto; certo; próprio; conforma-se com fatos, razão ou algum padrão geralmente aceito}
  \end{phonetics}
\end{entry}

\begin{entry}{此}{6}{⽌}
  \begin{phonetics}{此}{ci3}[][HSK 4]
    \definition*{s.}{Sobrenome Ci}
    \definition{pron.}{esse; essa; isso; este; esta; isto; indica ou se refere a uma pessoa ou coisa que está mais próxima, equivalente a 这 ou 这个 (em oposição a 彼) | aqui e agora; refere-se a um tempo ou lugar recente, equivalente a 这会儿 ou 这里}
  \seealsoref{彼}{bi3}
  \seealsoref{这}{zhe4}
  \seealsoref{这会儿}{zhe4 hui4r5}
  \seealsoref{这里}{zhe4 li3}
  \seealsoref{这个}{zhe4ge5}
  \end{phonetics}
\end{entry}

\begin{entry}{此处}{6,5}{⽌、⼡}
  \begin{phonetics}{此处}{ci3 chu4}[][HSK 6]
    \definition{pron.}{este lugar; aqui (literário)}
  \end{phonetics}
\end{entry}

\begin{entry}{此外}{6,5}{⽌、⼣}
  \begin{phonetics}{此外}{ci3wai4}[][HSK 4]
    \definition{conj.}{além disso; em adição; além das coisas ou situações mencionadas acima}
  \end{phonetics}
\end{entry}

\begin{entry}{此后}{6,6}{⽌、⼝}
  \begin{phonetics}{此后}{ci3 hou4}[][HSK 5]
    \definition{s.}{daqui em diante; doravante; depois disso; após isso; de agora em diante}
  \end{phonetics}
\end{entry}

\begin{entry}{此次}{6,6}{⽌、⽋}
  \begin{phonetics}{此次}{ci3 ci4}[][HSK 6]
    \definition{adv.}{desta vez; refere-se a um ponto específico no tempo ou período de tempo}
  \end{phonetics}
\end{entry}

\begin{entry}{此时}{6,7}{⽌、⽇}
  \begin{phonetics}{此时}{ci3 shi2}[][HSK 5]
    \definition{s.}{agora; no presente; agora mesmo; neste momento; por enquanto}
  \end{phonetics}
\end{entry}

\begin{entry}{此事}{6,8}{⽌、⼅}
  \begin{phonetics}{此事}{ci3 shi4}[][HSK 6]
    \definition{s.}{matéria; assunto}
  \end{phonetics}
\end{entry}

\begin{entry}{此刻}{6,8}{⽌、⼑}
  \begin{phonetics}{此刻}{ci3 ke4}[][HSK 5]
    \definition{s.}{agora; no momento; exatamente agora; neste momento}
  \end{phonetics}
\end{entry}

\begin{entry}{此前}{6,9}{⽌、⼑}
  \begin{phonetics}{此前}{ci3 qian2}[][HSK 6]
    \definition{adv.}{literário: antes; anteriormente | literário: antes disso}
  \end{phonetics}
\end{entry}

\begin{entry}{此致}{6,10}{⽌、⾄}
  \begin{phonetics}{此致}{ci3 zhi4}[][HSK 6]
    \definition{expr.}{Atenciosamente; Sinceramente; Com os melhores votos; usada no final de uma carta ou correspondência oficial}
  \end{phonetics}
\end{entry}

\begin{entry}{步}{7}{⽌}
  \begin{phonetics}{步}{bu4}[][HSK 3]
    \definition*{s.}{Geralmente em nomes de lugares | Sobrenome Bu}[盐步___Yanbu, na província de Guangdong]
    \definition{clas.}{uma unidade antiga para medida de comprimento, equivalente a cinco 尺}
    \definition{s.}{passo; ritmo | etapa; passo | condição; situação; estado | cais; píer | porto; cidade portuária | (geralmente em nomes de lugares)}
    \definition{v.}{caminhar; ir a pé | seguir os passos de alguém | (dialeto) medir com passos | seguir; acompanhar | medir a distância com os passos}
  \seealsoref{尺}{chi3}
  \end{phonetics}
\end{entry}

\begin{entry}{步行}{7,6}{⽌、⾏}
  \begin{phonetics}{步行}{bu4 xing2}[][HSK 4]
    \definition{v.}{caminhar; ir a pé; andar a pé (diferente de andar de carro, a cavalo, etc.)}
  \end{phonetics}
\end{entry}

\begin{entry}{武}{8}{⽌}
  \begin{phonetics}{武}{wu3}
    \definition*{s.}{Sobrenome Wu}
    \definition{s.}{arte marcial}
  \end{phonetics}
\end{entry}

\begin{entry}{武力}{8,2}{⽌、⼒}
  \begin{phonetics}{武力}{wu3li4}
    \definition{s.}{forças armadas | militares}
  \end{phonetics}
\end{entry}

\begin{entry}{武士}{8,3}{⽌、⼠}
  \begin{phonetics}{武士}{wu3shi4}
    \definition{s.}{samurai | guerreiro}
  \end{phonetics}
\end{entry}

\begin{entry}{武大戏}{8,3,6}{⽌、⼤、⼽}
  \begin{phonetics}{武大戏}{wu3 da4xi4}
    \definition*{s.}{Drama de Luta Acrobática | Drama Wu}
  \end{phonetics}
\end{entry}

\begin{entry}{武艺}{8,4}{⽌、⾋}
  \begin{phonetics}{武艺}{wu3yi4}
    \definition{s.}{arte marcial | habilidade militar}
  \end{phonetics}
\end{entry}

\begin{entry}{武术}{8,5}{⽌、⽊}
  \begin{phonetics}{武术}{wu3shu4}[][HSK 3]
    \definition[种,套,门]{s.}{arte marcial; autodefesa; \emph{wushu}; um esporte tradicional chinês que utiliza técnicas com os punhos, pernas, pés ou armas como facas e espadas}
  \end{phonetics}
\end{entry}

\begin{entry}{武官}{8,8}{⽌、⼧}
  \begin{phonetics}{武官}{wu3guan1}
    \definition{s.}{oficial militar}
  \end{phonetics}
\end{entry}

\begin{entry}{武断}{8,11}{⽌、⽄}
  \begin{phonetics}{武断}{wu3duan4}
    \definition{adj.}{arbitrário | dogmático | subjetivo}
  \end{phonetics}
\end{entry}

\begin{entry}{武装}{8,12}{⽌、⾐}
  \begin{phonetics}{武装}{wu3zhuang1}
    \definition{s.}{forças armadas | militar | arma}
    \definition{v.}{armar}
  \end{phonetics}
\end{entry}

\begin{entry}{武器}{8,16}{⽌、⼝}
  \begin{phonetics}{武器}{wu3qi4}[][HSK 3]
    \definition[批,个,件,种]{s.}{arma; equipamentos e dispositivos utilizados diretamente para matar inimigos ou destruir suas instalações defensivas e ofensivas | armas; armamento; metáfora usada como ferramenta de luta}
  \end{phonetics}
\end{entry}

\begin{entry}{歪}{9}{⽌}
  \begin{phonetics}{歪}{wai1}
    \definition{adj.}{torto | tortuoso | nocivo}
  \end{phonetics}
\end{entry}

\begin{entry}{歪果仁}{9,8,4}{⽌、⽊、⼈}
  \begin{phonetics}{歪果仁}{wai1 guo3 ren2}
    \definition{s.}{gíria na \emph{Internet} para estrangeiro (外国人)}
  \seealsoref{外国人}{wai4 guo2 ren2}
  \end{phonetics}
\end{entry}

%%%%% EOF %%%%%

