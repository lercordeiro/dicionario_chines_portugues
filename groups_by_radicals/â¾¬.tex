%%%
%%% Radical "⾬"
%%%

\section*{Radical 173: ``⾬''}\addcontentsline{toc}{section}{Radical 173: ⾬}

\begin{entry}{雨}{8}{⾬}
  \begin{phonetics}{雨}{yu3}[][HSK 1]
    \definition[阵,场]{s.}{chuva}
  \end{phonetics}
  \begin{phonetics}{雨}{yu4}
    \definition{v.}{cair (chuva, neve, etc.) | precipitar | chover | molhar}
  \end{phonetics}
\end{entry}

\begin{entry}{雨水}{8,4}{⾬、⽔}
  \begin{phonetics}{雨水}{yu3 shui3}[][HSK 5]
    \definition{s.}{água da chuva; precipitação; chuva; água proveniente da chuva}
  \end{phonetics}
\end{entry}

\begin{entry}{雨伞}{8,6}{⾬、⼈}
  \begin{phonetics}{雨伞}{yu3san3}
    \definition[把]{s.}{guarda-chuva}
  \end{phonetics}
\end{entry}

\begin{entry}{雨衣}{8,6}{⾬、⾐}
  \begin{phonetics}{雨衣}{yu3yi1}
    \definition[件]{s.}{impermeável}
  \end{phonetics}
\end{entry}

\begin{entry}{雨蚀}{8,9}{⾬、⾷}
  \begin{phonetics}{雨蚀}{yu3shi2}
    \definition{s.}{erosão da chuva}
  \end{phonetics}
\end{entry}

\begin{entry}{雨靴}{8,13}{⾬、⾰}
  \begin{phonetics}{雨靴}{yu3xue1}
    \definition[双]{s.}{botas de chuva}
  \end{phonetics}
\end{entry}

\begin{entry}{雪}{11}{⾬}
  \begin{phonetics}{雪}{xue3}[][HSK 2]
    \definition*{s.}{sobrenome Xue}
    \definition[场]{s.}{neve}
  \end{phonetics}
\end{entry}

\begin{entry}{雪人}{11,2}{⾬、⼈}
  \begin{phonetics}{雪人}{xue3ren2}
    \definition{s.}{boneco de neve | \emph{Yeti}}
  \end{phonetics}
\end{entry}

\begin{entry}{雪山}{11,3}{⾬、⼭}
  \begin{phonetics}{雪山}{xue3shan1}
    \definition{s.}{montanha coberta de neve}
  \end{phonetics}
\end{entry}

\begin{entry}{雪花}{11,7}{⾬、⾋}
  \begin{phonetics}{雪花}{xue3hua1}
    \definition{s.}{floco de neve}
  \end{phonetics}
\end{entry}

\begin{entry}{雪板}{11,8}{⾬、⽊}
  \begin{phonetics}{雪板}{xue3ban3}
    \definition{s.}{prancha de \emph{snowboard}}
    \definition{v.}{praticar \textit{snowboard}}
  \end{phonetics}
\end{entry}

\begin{entry}{雪葩}{11,12}{⾬、⾋}
  \begin{phonetics}{雪葩}{xue3pa1}
    \definition{s.}{sorvete}
  \end{phonetics}
\end{entry}

\begin{entry}{雪鞋}{11,15}{⾬、⾰}
  \begin{phonetics}{雪鞋}{xue3xie2}
    \definition[双]{s.}{sapatos de neve}
  \end{phonetics}
\end{entry}

\begin{entry}{雪糕}{11,16}{⾬、⽶}
  \begin{phonetics}{雪糕}{xue3gao1}
    \definition{s.}{picolé}
  \end{phonetics}
\end{entry}

\begin{entry}{零下}{13,3}{⾬、⼀}
  \begin{phonetics}{零下}{ling2 xia4}[][HSK 2]
    \definition{s.}{abaixo de zero}
  \end{phonetics}
\end{entry}

\begin{entry}{零食}{13,9}{⾬、⾷}
  \begin{phonetics}{零食}{ling2shi2}[][HSK 4]
    \definition[包,袋,盒,箱,堆]{s.}{lanches; refrescos; petiscos entre as refeições; alimentação esporádica, além das refeições normais}
  \end{phonetics}
\end{entry}

\begin{entry}{零/〇}{13,13}{⾬、⾬}
  \begin{phonetics}{零/〇}{ling2 ling2}[][HSK 1]
    \definition{adj.}{extra}
    \definition{num.}{zero; 0}
    \definition{s.}{(matemática) resto (após a divisão) | fração | nada}
  \end{phonetics}
\end{entry}

\begin{entry}{雷电}{13,5}{⾬、⽥}
  \begin{phonetics}{雷电}{lei2dian4}
    \definition{s.}{trovão e relâmpago; raio}
  \end{phonetics}
\end{entry}

\begin{entry}{雷亚尔}{13,6,5}{⾬、⼆、⼩}
  \begin{phonetics}{雷亚尔}{lei2ya4'er3}
    \definition*{s.}{Real Brasileiro}
  \end{phonetics}
\end{entry}

\begin{entry}{雾气}{13,4}{⾬、⽓}
  \begin{phonetics}{雾气}{wu4qi4}
    \definition{s.}{nevoeiro | névoa | vapor}
  \end{phonetics}
\end{entry}

\begin{entry}{需求}{14,7}{⾬、⽔}
  \begin{phonetics}{需求}{xu1qiu2}[][HSK 3]
    \definition{s.}{necessidades; demanda; requisito; requerimento; exigência | solicitações decorrentes de necessidades}
  \end{phonetics}
\end{entry}

\begin{entry}{需要}{14,9}{⾬、⾑}
  \begin{phonetics}{需要}{xu1yao4}[][HSK 3]
    \definition{s.}{necessidade | desejo ou solicitação de algo}
    \definition{v.}{precisar; querer; requerer; demandar}
  \end{phonetics}
\end{entry}

\begin{entry}{震惊}{15,11}{⾬、⼼}
  \begin{phonetics}{震惊}{zhen4jing1}[][HSK 5]
    \definition{adj.}{cado; atordoado; espantado; atônito}
    \definition{v.}{chocar; surpreender; espantar}
  \end{phonetics}
\end{entry}

\begin{entry}{震撼}{15,16}{⾬、⼿}
  \begin{phonetics}{震撼}{zhen4han4}
    \definition{v.}{sacudir | chocar | atordoar}
  \end{phonetics}
\end{entry}

\begin{entry}{霜}{17}{⾬}
  \begin{phonetics}{霜}{shuang1}
    \definition{s.}{geada | pó branco ou creme espalhado por uma superfície | glacê | creme de pele}
  \end{phonetics}
\end{entry}

\begin{entry}{露珠}{21,10}{⾬、⽟}
  \begin{phonetics}{露珠}{lu4zhu1}
    \definition{s.}{orvalho}
  \end{phonetics}
\end{entry}

\begin{entry}{霸权}{21,6}{⾬、⽊}
  \begin{phonetics}{霸权}{ba4quan2}
    \definition{s.}{hegemonia | supremacia}
  \end{phonetics}
\end{entry}

%%%%% EOF %%%%%

