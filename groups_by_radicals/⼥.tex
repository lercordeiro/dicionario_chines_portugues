%%%
%%% Radical "⼥"
%%%

\section*{Radical 38: ``⼥''}\addcontentsline{toc}{section}{Radical 38: ⼥}

\begin{Entry}{女}{3}{⼥}[Kangxi 38]
  \begin{Phonetics}{女}{nv3}[][HSK 1]
    \definition{adj.}{mulher; feminino (em oposição a 男) | fêmea (de certos animais)}
    \definition{s.}{menina; filha | nü, uma das mansões lunares | mulher}
  \seealsoref{男}{nan2}
  \end{Phonetics}
\end{Entry}

\begin{Entry}{女人}{3,2}{⼥、⼈}
  \begin{Phonetics}{女人}{nv3 ren2}[][HSK 1]
    \definition[个,位]{s.}{mulher adulta}
  \end{Phonetics}
\end{Entry}

\begin{Entry}{女儿}{3,2}{⼥、⼉}
  \begin{Phonetics}{女儿}{nv3'er2}[][HSK 1]
    \definition[个]{s.}{menina; filha}
  \seealsoref{儿子}{er2zi5}
  \end{Phonetics}
\end{Entry}

\begin{Entry}{女士}{3,3}{⼥、⼠}
  \begin{Phonetics}{女士}{nv3shi4}[][HSK 4]
    \definition{pron.}{Sra.; Senhorita; Senhora; título honorífico para mulheres (agora usado em contextos diplomáticos)}
    \definition[位,名,个,些]{s.}{senhora; madame}
  \end{Phonetics}
\end{Entry}

\begin{Entry}{女子}{3,3}{⼥、⼦}
  \begin{Phonetics}{女子}{nv3 zi3}[][HSK 3]
    \definition[位,名,个]{s.}{mulher; feminino; pessoa do sexo feminino}
  \end{Phonetics}
\end{Entry}

\begin{Entry}{女王}{3,4}{⼥、⽟}
  \begin{Phonetics}{女王}{nv3wang2}
    \definition{s.}{rainha}
  \end{Phonetics}
\end{Entry}

\begin{Entry}{女生}{3,5}{⼥、⽣}
  \begin{Phonetics}{女生}{nv3 sheng1}[][HSK 1]
    \definition[个]{s.}{estudante; aluna; estudante do sexo feminino | menina; jovem mulher}
  \end{Phonetics}
\end{Entry}

\begin{Entry}{女性}{3,8}{⼥、⼼}
  \begin{Phonetics}{女性}{nv3 xing4}[][HSK 5]
    \definition[个,位,名]{s.}{mulher; feminino; feminilidade; em oposição a 男性}
  \seealsoref{男性}{nan2 xing4}
  \end{Phonetics}
\end{Entry}

\begin{Entry}{女朋友}{3,8,4}{⼥、⽉、⼜}
  \begin{Phonetics}{女朋友}{nv3 peng2 you5}[][HSK 1]
    \definition{s.}{namorada}
  \end{Phonetics}
\end{Entry}

\begin{Entry}{女孩}{3,9}{⼥、⼦}
  \begin{Phonetics}{女孩}{nv3hai2}
    \definition{s.}{menina | garota}
  \end{Phonetics}
\end{Entry}

\begin{Entry}{女孩儿}{3,9,2}{⼥、⼦、⼉}
  \begin{Phonetics}{女孩儿}{nv3 hai2r5}[][HSK 1]
    \definition{s.}{garota; menina; atualmente também se refere a mulher adolescente | filha}
  \end{Phonetics}
\end{Entry}

\begin{Entry}{女婿}{3,12}{⼥、⼥}
  \begin{Phonetics}{女婿}{nv3xu5}
    \definition{s.}{marido da filha}
  \end{Phonetics}
\end{Entry}

\begin{Entry}{奶}{5}{⼥}
  \begin{Phonetics}{奶}{nai3}[][HSK 1]
    \definition{adj.}{bebê; infância; infantil}
    \definition[杯,滴,瓶,只,桶]{s.}{seios; mama | leite; produtos lácteos}
    \definition{v.}{amamentar; mamar}
  \end{Phonetics}
\end{Entry}

\begin{Entry}{奶牛}{5,4}{⼥、⽜}
  \begin{Phonetics}{奶牛}{nai3 niu2}[][HSK 6]
    \definition{s.}{vaca leiteira (ou leiteira); vaca}
  \end{Phonetics}
\end{Entry}

\begin{Entry}{奶奶}{5,5}{⼥、⼥}
  \begin{Phonetics}{奶奶}{nai3nai5}[][HSK 1]
    \definition[位]{s.}{avó (paterna) | vovó; avó; mulheres mais velhas | jovem senhora da casa}
  \end{Phonetics}
\end{Entry}

\begin{Entry}{奶茶}{5,9}{⼥、⾋}
  \begin{Phonetics}{奶茶}{nai3 cha2}[][HSK 3]
    \definition[杯]{s.}{chá com leite; chá com leite de vaca ou de ovelha}
  \end{Phonetics}
\end{Entry}

\begin{Entry}{奶粉}{5,10}{⼥、⽶}
  \begin{Phonetics}{奶粉}{nai3 fen3}[][HSK 6]
    \definition[袋,桶,罐,勺]{s.}{leite em pó}
  \end{Phonetics}
\end{Entry}

\begin{Entry}{奸}{6}{⼥}
  \begin{Phonetics}{奸}{jian1}
    \definition{adj.}{perverso; maligno; traiçoeiro; malicioso}
    \definition{s.}{traidor; espião | pessoa perversa; pessoa traiçoeira | relações sexuais ilícitas; comportamento sexual impróprio}
    \definition{v.}{ter relações sexuais ilícitas}
  \end{Phonetics}
\end{Entry}

\begin{Entry}{奸夫}{6,4}{⼥、⼤}
  \begin{Phonetics}{奸夫}{jian1fu1}
    \definition{s.}{homem adúltero}
  \end{Phonetics}
\end{Entry}

\begin{Entry}{她}{6}{⼥}
  \begin{Phonetics}{她}{ta1}[][HSK 1]
    \definition{pron.}{ela | ela; referir-se a coisas que se ama ou aprecia, como a pátria, a bandeira nacional, etc.}
  \end{Phonetics}
\end{Entry}

\begin{Entry}{她们}{6,5}{⼥、⼈}
  \begin{Phonetics}{她们}{ta1men5}[][HSK 1]
    \definition{pron.}{elas; referindo-se a várias mulheres: em textos escritos, use 她们 quando todas as pessoas forem mulheres e 他们 quando houver homens e mulheres}
  \seealsoref{他们}{ta1men5}
  \end{Phonetics}
\end{Entry}

\begin{Entry}{她们的}{6,5,8}{⼥、⼈、⽩}
  \begin{Phonetics}{她们的}{ta1men5 de5}
    \definition{pron.}{delas}
  \end{Phonetics}
\end{Entry}

\begin{Entry}{她的}{6,8}{⼥、⽩}
  \begin{Phonetics}{她的}{ta1 de5}
    \definition{pron.}{dela}
  \end{Phonetics}
\end{Entry}

\begin{Entry}{好}{6}{⼥}
  \begin{Phonetics}{好}{hao3}[][HSK 1,2,4]
    \definition{adj.}{bom; ótimo; agradável; vantajoso; satisfatório | amigável; gentil; amistoso; amável | saudável; bem | pronto; concluído; usado após um verbo para indicar conclusão ou perfeição | fácil (de fazer); conveniente; responsável (por)}
    \definition{adv.}{muito; bastante; tão; usado na frente de uma palavra de quantidade ou uma palavra de tempo para indicar muito ou por muito tempo | em que medida; como; usado antes de adjetivos e verbos para indicar profundidade e com exclamação}
    \definition{interj.}{O.K.; tudo bem; aprovação, acordo ou encerramento | (no início de uma frase ou oração) expressa concordância (ou desaprovação, surpresa, etc.)}
    \definition{prep.}{de modo a; para que}
    \definition{s.}{referindo-se a palavras de elogio ou aplauso | saudações; cumprimentos}
    \definition{suf.}{sufixo que indica conclusão ou prontidão | depois de um pronome significa ``olá''}
    \definition{v.}{deve; precisa; tem que; deveria | apaixonar-se}
  \end{Phonetics}
  \begin{Phonetics}{好}{hao4}
    \definition*{s.}{Sobrenome Hao}
    \definition{adv.}{algo que acontece com frequência, que é fácil de acontecer}
    \definition{v.}{gostar; amar; ter afeição por}
  \end{Phonetics}
\end{Entry}

\begin{Entry}{好人}{6,2}{⼥、⼈}
  \begin{Phonetics}{好人}{hao3 ren2}[][HSK 2]
    \definition[个,位,名]{s.}{pessoa boa (ou excelente) (oposto de 坏人) | pessoa saudável | pessoa gentil que tenta se dar bem com todos (muitas vezes em detrimento dos princípios)}
  \seealsoref{坏人}{huai4 ren2}
  \end{Phonetics}
\end{Entry}

\begin{Entry}{好久}{6,3}{⼥、⼃}
  \begin{Phonetics}{好久}{hao3jiu3}[][HSK 2]
    \definition{adv.}{por muito tempo | por eras (no passado)}
  \end{Phonetics}
\end{Entry}

\begin{Entry}{好(不)容易}{6,4,10,8}{⼥、⼀、⼧、⽇}
  \begin{Phonetics}{好(不)容易}{hao3 bu4 rong2 yi4}[][HSK 6]
    \definition{adv.}{com grande dificuldade; muito difícil}
    \definition{v.}{ter dificuldade (em fazer algo)}
  \end{Phonetics}
\end{Entry}

\begin{Entry}{好友}{6,4}{⼥、⼜}
  \begin{Phonetics}{好友}{hao3you3}[][HSK 4]
    \definition[位,名,个,些]{s.}{bom amigo; amigo próximo}
  \end{Phonetics}
\end{Entry}

\begin{Entry}{好心}{6,4}{⼥、⼼}
  \begin{Phonetics}{好心}{hao3xin1}
    \definition{s.}{bondade | boas intenções}
  \end{Phonetics}
\end{Entry}

\begin{Entry}{好处}{6,5}{⼥、⼡}
  \begin{Phonetics}{好处}{hao3chu4}[][HSK 2]
    \definition[个]{s.}{bom; benefício; vantagem; fatores favoráveis a pessoas ou coisas | ganho; lucro; algo que não se deveria receber, dado por outra pessoa ou obtido através de uma oportunidade; geralmente tem conotação pejorativa}
  \end{Phonetics}
\end{Entry}

\begin{Entry}{好汉}{6,5}{⼥、⽔}
  \begin{Phonetics}{好汉}{hao3han4}
    \definition[条]{s.}{herói | pessoa forte e corajosa}
  \end{Phonetics}
\end{Entry}

\begin{Entry}{好生}{6,5}{⼥、⽣}
  \begin{Phonetics}{好生}{hao3sheng1}
    \definition{adv.}{bastante; extremamente | cuidadosamente; apropriadamente}
  \end{Phonetics}
\end{Entry}

\begin{Entry}{好用}{6,5}{⼥、⽤}
  \begin{Phonetics}{好用}{hao3yong4}
    \definition{adj.}{fácil de usar | adequado ao uso}
  \end{Phonetics}
\end{Entry}

\begin{Entry}{好似}{6,6}{⼥、⼈}
  \begin{Phonetics}{好似}{hao3 si4}[][HSK 6]
    \definition{v.}{parecer; ser como}
  \end{Phonetics}
\end{Entry}

\begin{Entry}{好吃}{6,6}{⼥、⼝}
  \begin{Phonetics}{好吃}{hao3chi1}[][HSK 1]
    \definition{adj.}{bom; saboroso; delicioso; descreve o sabor agradável de algo, que as pessoas gostam de comer}
  \end{Phonetics}
  \begin{Phonetics}{好吃}{hao4chi1}
    \definition{v.}{ser guloso; gostar de comer boa comida}
  \end{Phonetics}
\end{Entry}

\begin{Entry}{好多}{6,6}{⼥、⼣}
  \begin{Phonetics}{好多}{hao3 duo1}[][HSK 2]
    \definition{adj.}{muitos; uma boa quantidade; uma grande quantidade; uma quantidade enorme}
    \definition{pron.}{quantos?; quanto?; frequentemente usado para perguntar sobre quantidade}
  \end{Phonetics}
\end{Entry}

\begin{Entry}{好好}{6,6}{⼥、⼥}
  \begin{Phonetics}{好好}{hao3 hao3}[][HSK 3]
    \definition{adj.}{realmente bom/bem; em perfeitas condições; quando tudo está bem}
    \definition{adv.}{diretamente; seriamente; cuidadosamente; com todo o empenho; ao máximo}
  \end{Phonetics}
\end{Entry}

\begin{Entry}{好听}{6,7}{⼥、⼝}
  \begin{Phonetics}{好听}{hao3 ting1}[][HSK 1]
    \definition{adj.}{agradável de ouvir (de som ou voz) | bom; palatável; satisfatório (de palavras)  | decente; honrado (de ações, etc.); descreve uma coisa que parece prestigiosa | interessante; descreve palavras, histórias e outras coisas interessantes}
  \end{Phonetics}
\end{Entry}

\begin{Entry}{好运}{6,7}{⼥、⾡}
  \begin{Phonetics}{好运}{hao3 yun4}[][HSK 5]
    \definition{s.}{boa sorte, fortuna ou oportunidade}
  \end{Phonetics}
\end{Entry}

\begin{Entry}{好事}{6,8}{⼥、⼅}
  \begin{Phonetics}{好事}{hao3 shi4}[][HSK 2]
    \definition[个,件]{s.}{boa ação; gentileza | (antigo) obra de caridade | acontecimento feliz; evento festivo}
  \end{Phonetics}
  \begin{Phonetics}{好事}{hao4 shi4}
    \definition[个,件]{s.}{intrometido; gostar de se meter na vida dos outros}
  \end{Phonetics}
\end{Entry}

\begin{Entry}{好奇}{6,8}{⼥、⼤}
  \begin{Phonetics}{好奇}{hao4qi2}[][HSK 3]
    \definition{adj.}{curioso; curiosidade e interesse por coisas não conhecidas}
    \definition{s.}{curiosidade}
    \definition{v.}{ser ou estar curioso}
  \end{Phonetics}
\end{Entry}

\begin{Entry}{好学}{6,8}{⼥、⼦}
  \begin{Phonetics}{好学}{hao3xue2}
    \definition{adj.}{fácil de aprender}
  \end{Phonetics}
  \begin{Phonetics}{好学}{hao4xue2}[][HSK 6]
    \definition[个]{s.}{apaixonado para aprender; estudioso; erudito}
  \end{Phonetics}
\end{Entry}

\begin{Entry}{好玩儿}{6,8,2}{⼥、⽟、⼉}
  \begin{Phonetics}{好玩儿}{hao3 wan2r5}[][HSK 1]
    \definition{adj.}{divertido; interessante; capaz de despertar interesse}
  \end{Phonetics}
\end{Entry}

\begin{Entry}{好转}{6,8}{⼥、⾞}
  \begin{Phonetics}{好转}{hao3 zhuan3}[][HSK 6]
    \definition{v.}{melhorar; dar uma guinada para melhor; tomar um rumo favorável}
  \end{Phonetics}
\end{Entry}

\begin{Entry}{好看}{6,9}{⼥、⽬}
  \begin{Phonetics}{好看}{hao3 kan4}[][HSK 1]
    \definition{adj.}{de boa aparência; agradável; bonito | interessante; descreve o enredo ou conteúdo de filmes, romances, performances, etc., como sendo cativante, agradável ou apreciável}
  \end{Phonetics}
\end{Entry}

\begin{Entry}{好象}{6,11}{⼥、⾗}
  \begin{Phonetics}{好象}{hao3xiang4}
    \variantof{好像}
  \end{Phonetics}
\end{Entry}

\begin{Entry}{好像}{6,13}{⼥、⼈}
  \begin{Phonetics}{好像}{hao3xiang4}[][HSK 2]
    \definition{adv.}{como se; um pouco parecido; como se fosse}
    \definition{v.}{parecer; ser como; parecer-se com}
  \end{Phonetics}
\end{Entry}

\begin{Entry}{如}{6}{⼥}
  \begin{Phonetics}{如}{ru2}[][HSK 6]
    \definition{adv.}{por exemplo; tal como; como}
    \definition{conj.}{se; no caso (de); no caso de; como se; como}
    \definition{prep.}{em conformidade com; de acordo com}
    \definition{v.}{estar em conformidade (ou de acordo) com | (geralmente no negativo) pode ser comparado com; ser comparável a; ser tão bom quanto | superar; exceder | (literário) ir para}
  \end{Phonetics}
\end{Entry}

\begin{Entry}{如一}{6,1}{⼥、⼀}
  \begin{Phonetics}{如一}{ru2 yi1}[][HSK 6]
    \definition{adj.}{consistente; coerente}
  \end{Phonetics}
\end{Entry}

\begin{Entry}{如下}{6,3}{⼥、⼀}
  \begin{Phonetics}{如下}{ru2 xia4}[][HSK 5]
    \definition{adv.}{como descrito ou listado abaixo; conforme segue; conforme abaixo}
  \end{Phonetics}
\end{Entry}

\begin{Entry}{如今}{6,4}{⼥、⼈}
  \begin{Phonetics}{如今}{ru2jin1}[][HSK 4]
    \definition{s.}{agora; hoje em dia; atualmente; no presente}
  \end{Phonetics}
\end{Entry}

\begin{Entry}{如同}{6,6}{⼥、⼝}
  \begin{Phonetics}{如同}{ru2 tong2}[][HSK 5]
    \definition{v.}{parecer que; usado principalmente em metáforas}
  \end{Phonetics}
\end{Entry}

\begin{Entry}{如此}{6,6}{⼥、⽌}
  \begin{Phonetics}{如此}{ru2 ci3}[][HSK 5]
    \definition{adv.}{assim; tal; dessa forma; dessa maneira; refere-se a uma situação mencionada anteriormente, equivalente a 这样}
  \seealsoref{这样}{zhe4 yang4}
  \end{Phonetics}
\end{Entry}

\begin{Entry}{如何}{6,7}{⼥、⼈}
  \begin{Phonetics}{如何}{ru2he2}[][HSK 3]
    \definition{pron.}{como?; o que?; usado para perguntar como resolver um problema | como?; o que?; usado para perguntar sobre a situação ou obter a opinião de outras pessoas}
  \end{Phonetics}
\end{Entry}

\begin{Entry}{如果}{6,8}{⼥、⽊}
  \begin{Phonetics}{如果}{ru2guo3}[][HSK 2]
    \definition{conj.}{se; no caso de; na eventualidade de; supondo que; para expressar suposições, pode-se usar 要是 na linguagem falada.}
  \seealsoref{要是}{yao4shi5}
  \end{Phonetics}
\end{Entry}

\begin{Entry}{如画}{6,8}{⼥、⽥}
  \begin{Phonetics}{如画}{ru2hua4}
    \definition{adj.}{pitoresco}
  \end{Phonetics}
\end{Entry}

\begin{Entry}{妆}{6}{⼥}
  \begin{Phonetics}{妆}{zhuang1}
    \definition{s.}{adornos femininos | enxoval; dote | adornos pessoais femininos; maquiagem e figurino de palco; costumava se referir às maquiagens em mulheres, mas agora se refere às maquiagens em atores}
    \definition{v.}{aplicar maquiagem; maquiar | arrumar-se; maquiar-se}
  \end{Phonetics}
\end{Entry}

\begin{Entry}{妆扮}{6,7}{⼥、⼿}
  \begin{Phonetics}{妆扮}{zhuang1ban4}
    \variantof{装扮}
  \end{Phonetics}
\end{Entry}

\begin{Entry}{妇}{6}{⼥}
  \begin{Phonetics}{妇}{fu4}
    \definition{s.}{mulher | mulher casada | esposa}
  \end{Phonetics}
\end{Entry}

\begin{Entry}{妇女}{6,3}{⼥、⼥}
  \begin{Phonetics}{妇女}{fu4nv3}[][HSK 6]
    \definition[个,位,群,名,帮]{s.}{mulher; mulheres; um termo geral para mulheres adultas}
  \end{Phonetics}
\end{Entry}

\begin{Entry}{妈}{6}{⼥}
  \begin{Phonetics}{妈}{ma1}[][HSK 1]
    \definition[个,位]{s.}{mãe; mamãe | uma forma de tratamento para uma mulher casada uma geração mais velha | (antigo) uma forma de tratamento para uma empregada doméstica de meia-idade ou velha}
  \seealsoref{妈妈}{ma1 ma5}
  \end{Phonetics}
\end{Entry}

\begin{Entry}{妈妈}{6,6}{⼥、⼥}
  \begin{Phonetics}{妈妈}{ma1 ma5}[][HSK 1]
    \definition[个,位]{s.}{mamãe; mãe | uma forma de chamar uma mulher de meia-idade; títulos de respeito para mulheres mais velhas}
  \end{Phonetics}
\end{Entry}

\begin{Entry}{妖}{7}{⼥}
  \begin{Phonetics}{妖}{yao1}
    \definition{adj.}{maligno e fraudulento | sedutor; encantador | paquerador}
    \definition[个,只]{s.}{\emph{goblin}; demônio; espírito maligno}
  \end{Phonetics}
\end{Entry}

\begin{Entry}{妙}{7}{⼥}
  \begin{Phonetics}{妙}{miao4}[][HSK 6]
    \definition*{s.}{Sobrenome Miao}
    \definition{adj.}{maravilhoso; excelente; bom | engenhoso; esperto; sutil | extraordinário | requintado; mágico; engenhoso; misterioso}
  \end{Phonetics}
\end{Entry}

\begin{Entry}{妙招}{7,8}{⼥、⼿}
  \begin{Phonetics}{妙招}{miao4zhao1}
    \definition{adj.}{escorregadio}
    \definition{s.}{movimento inteligente | maneira inteligente de fazer algo}
  \end{Phonetics}
\end{Entry}

\begin{Entry}{妹}{8}{⼥}
  \begin{Phonetics}{妹}{mei4}[][HSK 1]
    \definition*{s.}{Sobrenome Mei}
    \definition[个]{s.}{irmã mais nova | parente do sexo feminino da mesma geração | jovem garota; jovem mulher ou menina}
  \seealsoref{妹妹}{mei4 mei5}
  \end{Phonetics}
\end{Entry}

\begin{Entry}{妹夫}{8,4}{⼥、⼤}
  \begin{Phonetics}{妹夫}{mei4fu5}
    \definition{s.}{marido da irmã mais nova}
  \end{Phonetics}
\end{Entry}

\begin{Entry}{妹妹}{8,8}{⼥、⼥}
  \begin{Phonetics}{妹妹}{mei4 mei5}[][HSK 1]
    \definition[个]{s.}{irmã mais nova}
  \end{Phonetics}
\end{Entry}

\begin{Entry}{妻}{8}{⼥}
  \begin{Phonetics}{妻}{qi1}
    \definition{s.}{esposa}
  \end{Phonetics}
  \begin{Phonetics}{妻}{qi4}
    \definition{v.}{casar uma mulher com (alguém)}
  \end{Phonetics}
\end{Entry}

\begin{Entry}{妻子}{8,3}{⼥、⼦}
  \begin{Phonetics}{妻子}{qi1zi3}
    \definition[个]{s.}{esposa e filhos; (chinês antigo) refere-se a esposas, filhos e filhas}
  \end{Phonetics}
  \begin{Phonetics}{妻子}{qi1zi5}[][HSK 4]
    \definition[个]{s.}{esposa (não é usado como um termo carinhoso)}
  \end{Phonetics}
\end{Entry}

\begin{Entry}{始}{8}{⼥}
  \begin{Phonetics}{始}{shi3}
    \definition*{s.}{Sobrenome Shi}
    \definition{adv.}{somente então; não\dots até}
    \definition{s.}{começo; início}
    \definition{v.}{começar; iniciar}
  \end{Phonetics}
\end{Entry}

\begin{Entry}{始终}{8,8}{⼥、⽷}
  \begin{Phonetics}{始终}{shi3zhong1}[][HSK 3]
    \definition{adv.}{sempre; o tempo todo; durante todo; do começo ao fim; indica continuidade do início ao fim}
    \definition{s.}{todo o processo do começo ao fim}
  \end{Phonetics}
\end{Entry}

\begin{Entry}{姐}{8}{⼥}
  \begin{Phonetics}{姐}{jie3}[][HSK 1]
    \definition[个,位]{s.}{irmã mais velha; irmã | termo genérico para mulheres jovens | mulheres da mesma geração que são mais velhas do que você (geralmente não inclui aquelas que podem ser chamadas de cunhadas) | um título respeitoso para mulheres jovens profissionais em determinados cargos}
  \seealsoref{姐姐}{jie3 jie5}
  \end{Phonetics}
\end{Entry}

\begin{Entry}{姐夫}{8,4}{⼥、⼤}
  \begin{Phonetics}{姐夫}{jie3fu5}
    \definition{s.}{marido da irmã mais velha}
  \end{Phonetics}
\end{Entry}

\begin{Entry}{姐妹}{8,8}{⼥、⼥}
  \begin{Phonetics}{姐妹}{jie3 mei4}[][HSK 4]
    \definition[个]{s.}{irmãs}
  \end{Phonetics}
\end{Entry}

\begin{Entry}{姐姐}{8,8}{⼥、⼥}
  \begin{Phonetics}{姐姐}{jie3 jie5}[][HSK 1]
    \definition[个]{s.}{irmã mais velha}
  \end{Phonetics}
\end{Entry}

\begin{Entry}{姑}{8}{⼥}
  \begin{Phonetics}{姑}{gu1}
    \definition{adv.}{provisoriamente; por enquanto}
    \definition[个,位,名,些]{s.}{irmã do pai; tia | irmã do marido; cunhada | mãe do marido; sogra | freira; mulher que exerce uma ocupação religiosa | a irmã do pai de alguém | mulheres jovens (no campo)}
  \end{Phonetics}
\end{Entry}

\begin{Entry}{姑且}{8,5}{⼥、⼀}
  \begin{Phonetics}{姑且}{gu1qie3}
    \definition{adv.}{provisoriamente | por enquanto}
  \end{Phonetics}
\end{Entry}

\begin{Entry}{姑姑}{8,8}{⼥、⼥}
  \begin{Phonetics}{姑姑}{gu1gu5}[][HSK 6]
    \definition[个,位,名]{s.}{tia; tia paterna}
  \end{Phonetics}
\end{Entry}

\begin{Entry}{姑娘}{8,10}{⼥、⼥}
  \begin{Phonetics}{姑娘}{gu1niang5}[][HSK 3]
    \definition[位,名,个,些]{s.}{menina; jovem senhora; mulher solteira | filha}
  \end{Phonetics}
\end{Entry}

\begin{Entry}{姓}{8}{⼥}
  \begin{Phonetics}{姓}{xing4}[][HSK 2]
    \definition[个]{s.}{sobrenome; nome de família; um caractere que representa um sistema familiar, os chineses colocam o sobrenome em primeiro lugar e o nome em segundo}
    \definition{v.}{ter como sobrenome; tratar um ou mais caracteres como sobrenome}
  \end{Phonetics}
\end{Entry}

\begin{Entry}{姓氏}{8,4}{⼥、⽒}
  \begin{Phonetics}{姓氏}{xing4shi4}
    \definition{s.}{sobrenome}
  \end{Phonetics}
\end{Entry}

\begin{Entry}{姓名}{8,6}{⼥、⼝}
  \begin{Phonetics}{姓名}{xing4ming2}[][HSK 2]
    \definition{s.}{nome; nome completo; sobrenome e nome próprio}
  \end{Phonetics}
\end{Entry}

\begin{Entry}{委}{8}{⼥}
  \begin{Phonetics}{委}{wei1}
    \definition{adj./adv.}{o mesmo que 逶 em 逶迤 sinuoso, curvo}
  \seealsoref{逶}{wei1}
  \seealsoref{逶迤}{wei1yi2}
  \end{Phonetics}
  \begin{Phonetics}{委}{wei3}
    \definition*{s.}{Sobrenome Wei}
    \definition{adj.}{indireto; desviado | apático; abatido | sinuoso; tortuoso | desanimado; apático; sem inspiração}
    \definition{adv.}{realmente; certamente; na verdade}
    \definition{s.}{membro do comitê | comitê; comissão; conselho}
    \definition{v.}{confiar; nomear |  jogar fora; deixar de lado | culpar os outros | confiar | descartar; abandonar | mudar; empurrar | acumular}
  \end{Phonetics}
\end{Entry}

\begin{Entry}{委内瑞拉}{8,4,13,8}{⼥、⼌、⽟、⼿}
  \begin{Phonetics}{委内瑞拉}{wei3nei4rui4la1}
    \definition*{s.}{Venezuela}
  \end{Phonetics}
\end{Entry}

\begin{Entry}{委托}{8,6}{⼥、⼿}
  \begin{Phonetics}{委托}{wei3tuo1}[][HSK 5]
    \definition{v.}{confiar; confiar uma tarefa a outra pessoa ou instituição (para que seja realizada)}
  \end{Phonetics}
\end{Entry}

\begin{Entry}{姜}{9}{⼥}
  \begin{Phonetics}{姜}{jiang1}
    \definition*{s.}{Sobrenome Jiang}
    \definition[磅,斤,两]{s.}{gengibre; rizoma de gengibre}
  \end{Phonetics}
\end{Entry}

\begin{Entry}{威}{9}{⼥}
  \begin{Phonetics}{威}{wei1}
    \definition*{s.}{Sobrenome Wei}
    \definition{adj.}{forte; poderoso}
    \definition{s.}{força impressionante; poder; força}
    \definition{v.}{ameaçar pela força; intimidar com força}
  \end{Phonetics}
\end{Entry}

\begin{Entry}{威胁}{9,8}{⼥、⾁}
  \begin{Phonetics}{威胁}{wei1xie2}[][HSK 6]
    \definition{v.}{pôr em perigo; ameaçar; intimidar}
  \end{Phonetics}
\end{Entry}

\begin{Entry}{娃}{9}{⼥}
  \begin{Phonetics}{娃}{wa2}
    \definition[个,名,位,只]{s.}{bebê; criança | filho ou filha; criança | Dialeto: animal recém-nascido | Literário: menina; jovem mulher | Literário: menina bonita}
  \end{Phonetics}
\end{Entry}

\begin{Entry}{娃娃}{9,9}{⼥、⼥}
  \begin{Phonetics}{娃娃}{wa2wa5}[][HSK 6]
    \definition[个,名,位]{s.}{bebê; criança; criança pequena | boneca; brinquedos em forma de crianças}
  \end{Phonetics}
\end{Entry}

\begin{Entry}{娱}{10}{⼥}
  \begin{Phonetics}{娱}{yu2}
    \definition{s.}{alegria; prazer; diversão; felicidade}
    \definition{v.}{dar prazer a; divertir; fazer feliz}
  \end{Phonetics}
\end{Entry}

\begin{Entry}{娱乐}{10,5}{⼥、⼃}
  \begin{Phonetics}{娱乐}{yu2le4}[][HSK 6]
    \definition[项]{s.}{entretenimento; diversão; recreação; passa-tempo; atividades recreativas, prazerosas e divertidas}
    \definition{v.}{recrear; divertir; distrair; entreter; passar o tempo}
  \end{Phonetics}
\end{Entry}

\begin{Entry}{孬}{10}{⼥}
  \begin{Phonetics}{孬}{nao1}
    \definition{adj.}{ruim | covarde | Dialeto: não (é) bom (contração de 不 + 好)}
  \seealsoref{不}{bu4}
  \seealsoref{好}{hao3}
  \end{Phonetics}
\end{Entry}

\begin{Entry}{婚}{11}{⼥}
  \begin{Phonetics}{婚}{hun1}
    \definition{s.}{casamento}
    \definition{v.}{casar}
  \end{Phonetics}
\end{Entry}

\begin{Entry}{婚礼}{11,5}{⼥、⽰}
  \begin{Phonetics}{婚礼}{hun1li3}[][HSK 4]
    \definition[场]{s.}{casamento; núpcias; cerimônia de casamento}
  \end{Phonetics}
\end{Entry}

\begin{Entry}{媒}{12}{⼥}
  \begin{Phonetics}{媒}{mei2}
    \definition{s.}{casamenteiro; intermediário | intermediário; médio}
    \definition{v.}{fazer uma combinação}
  \end{Phonetics}
\end{Entry}

\begin{Entry}{媒体}{12,7}{⼥、⼈}
  \begin{Phonetics}{媒体}{mei2ti3}[][HSK 3]
    \definition[家,个,种]{s.}{mídia; mídia de massa; vários meios de comunicação e transmissão de informações, como televisão, rádio, jornais, etc.}
  \end{Phonetics}
\end{Entry}

\begin{Entry}{嫂}{12}{⼥}
  \begin{Phonetics}{嫂}{sao3}
    \definition[个,位,名,些]{s.}{esposa do irmão mais velho; cunhada | irmã (uma forma de tratamento para uma mulher casada, mais ou menos da mesma idade)}
  \end{Phonetics}
\end{Entry}

\begin{Entry}{嫂子}{12,3}{⼥、⼦}
  \begin{Phonetics}{嫂子}{sao3zi5}
    \definition{s.}{esposa do irmão mais velho}
  \end{Phonetics}
\end{Entry}

\begin{Entry}{嫉}{13}{⼥}
  \begin{Phonetics}{嫉}{ji2}
    \definition{v.}{invejar | odiar | ter ciúmes; ter inveja}
  \end{Phonetics}
\end{Entry}

\begin{Entry}{嫉妒}{13,7}{⼥、⼥}
  \begin{Phonetics}{嫉妒}{ji2du4}
    \definition{v.}{estar com ciúmes de | invejar}
  \end{Phonetics}
\end{Entry}

\begin{Entry}{嫌}{13}{⼥}
  \begin{Phonetics}{嫌}{xian2}[][HSK 6]
    \definition{s.}{suspeita; suspeição; cisma | inimizade; rancor; má vontade; ressentimento}
    \definition{v.}{importar-se com; não gostar e evitar; reclamar de}
  \end{Phonetics}
\end{Entry}

\begin{Entry}{嫦}{14}{⼥}
  \begin{Phonetics}{嫦}{chang2}
    \definition{s.}{uma beleza lendária que voou para a lua | a dama da lua}
  \end{Phonetics}
\end{Entry}

\begin{Entry}{嫦娥}{14,10}{⼥、⼥}
  \begin{Phonetics}{嫦娥}{chang2'e2}[][HSK 7-9]
    \definition*{s.}{Chang'e, a dama da lua (mitologia chinesa); uma fada que voou do mundo humano para o Palácio da Lua na mitologia}
  \end{Phonetics}
\end{Entry}

%%%%% EOF %%%%%

