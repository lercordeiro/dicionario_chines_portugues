%%%
%%% Radical "⽲"
%%%

\section*{Radical 115: ``⽲''}\addcontentsline{toc}{section}{Radical 115: ⽲}

\begin{Entry}{私}{7}{⽲}
  \begin{Phonetics}{私}{si1}
    \definition*{s.}{Sobrenome Si}
    \definition{adj.}{pessoal; privado (oposição a 公) | egoísta (oposto a 公) | secreto; privado | ilícito; ilegal}
    \definition{s.}{interesse privado (ou egoísta); motivo (ou ideia) egoísta (oposição a 公) | contrabando; mercadorias contrabandeadas | propriedade privada | interesses privados; ganho pessoal}
  \seealsoref{公}{gong1}
  \end{Phonetics}
\end{Entry}

\begin{Entry}{私人}{7,2}{⽲、⼈}
  \begin{Phonetics}{私人}{si1ren2}[][HSK 5]
    \definition{adj.}{privado; pertencente a um indivíduo ou exercido a título individual; não público | pessoal; entre indivíduos}
    \definition[个]{s.}{algo privado; pessoas que se aproximam de você por motivos pessoais ou interesses próprios}
  \end{Phonetics}
\end{Entry}

\begin{Entry}{私人诊所}{7,2,7,8}{⽲、⼈、⾔、⼾}
  \begin{Phonetics}{私人诊所}{si1ren2 zhen3suo3}
    \definition[些]{s.}{clínica privada}
  \end{Phonetics}
\end{Entry}

\begin{Entry}{私人信件}{7,2,9,6}{⽲、⼈、⼈、⼈}
  \begin{Phonetics}{私人信件}{si1ren2 xin4jian4}
    \definition{s.}{carta pessoal}
  \end{Phonetics}
\end{Entry}

\begin{Entry}{私人钥匙}{7,2,9,11}{⽲、⼈、⾦、⼔}
  \begin{Phonetics}{私人钥匙}{si1ren2yao4shi5}
    \definition{s.}{(criptografia) chave privada}
  \end{Phonetics}
\end{Entry}

\begin{Entry}{私生活}{7,5,9}{⽲、⽣、⽔}
  \begin{Phonetics}{私生活}{si1sheng1huo2}
    \definition{s.}{vida privada}
  \end{Phonetics}
\end{Entry}

\begin{Entry}{私自}{7,6}{⽲、⾃}
  \begin{Phonetics}{私自}{si1zi4}
    \definition{adj.}{privado | pessoal}
    \definition{adv.}{secretamente | sem aprovação explícita}
  \end{Phonetics}
\end{Entry}

\begin{Entry}{秋}{9}{⽲}
  \begin{Phonetics}{秋}{qiu1}
    \definition*{s.}{Sobrenome Qiu}
    \definition{s.}{outono | época da colheita; a estação em que as colheitas amadurecem; colheitas maduras no outono | ano; refere-se a um ano | um período de tempo (geralmente conturbado)}
  \end{Phonetics}
\end{Entry}

\begin{Entry}{秋天}{9,4}{⽲、⼤}
  \begin{Phonetics}{秋天}{qiu1 tian1}[][HSK 2]
    \definition[个,段,季,番]{s.}{outono}
  \end{Phonetics}
\end{Entry}

\begin{Entry}{秋季}{9,8}{⽲、⼦}
  \begin{Phonetics}{秋季}{qiu1 ji4}[][HSK 4]
    \definition[个]{s.}{outono; terceiro trimestre do ano, segundo o costume chinês, refere-se ao período de três meses entre o outono e o inverno, também se refere aos sétimo, oitavo e nono meses do calendário lunar}
  \end{Phonetics}
\end{Entry}

\begin{Entry}{种}{9}{⽲}
  \begin{Phonetics}{种}{zhong3}[][HSK 3,4]
    \definition{clas.}{indica tipo, usado para pessoas e qualquer coisa}
    \definition{s.}{espécie | etnia | semente; estirpe; linhagem; material para reprodução biológica em cadeia | coragem; determinação; garra; força de caráter; refere-se à coragem ou determinação}
  \end{Phonetics}
  \begin{Phonetics}{种}{zhong4}
    \definition{v.}{semear; cultivar; plantar}
  \end{Phonetics}
\end{Entry}

\begin{Entry}{种子}{9,3}{⽲、⼦}
  \begin{Phonetics}{种子}{zhong3zi5}[][HSK 3]
    \definition[颗,粒,个,号]{s.}{semente; um órgão exclusivo de certas plantas, geralmente composto de três partes: tegumento, embrião e endosperma, as sementes podem germinar e se tornar novas plantas sob certas condições | jogador classificado; durante a competição, nas eliminatórias, os jogadores mais fortes de cada equipe são escalados}
  \end{Phonetics}
\end{Entry}

\begin{Entry}{种地}{9,6}{⽲、⼟}
  \begin{Phonetics}{种地}{zhong4di4}
    \definition{v.}{cultivar | trabalhar a terra}
  \end{Phonetics}
\end{Entry}

\begin{Entry}{种种}{9,9}{⽲、⽲}
  \begin{Phonetics}{种种}{zhong3 zhong3}[][HSK 6]
    \definition{adj.}{todos os tipos de; uma variedade de}
  \end{Phonetics}
\end{Entry}

\begin{Entry}{种类}{9,9}{⽲、⽶}
  \begin{Phonetics}{种类}{zhong3lei4}[][HSK 4]
    \definition[个]{s.}{espécie; classe; tipo; variedade; categoria; classificação de alguma coisa de acordo com sua natureza e características}
  \end{Phonetics}
\end{Entry}

\begin{Entry}{种族灭绝}{9,11,5,9}{⽲、⽅、⽕、⽷}
  \begin{Phonetics}{种族灭绝}{zhong3zu2mie4jue2}
    \definition{s.}{genocídio | extinção étnica}
  \end{Phonetics}
\end{Entry}

\begin{Entry}{种麻}{9,11}{⽲、⿇}
  \begin{Phonetics}{种麻}{zhong3ma2}
    \definition{s.}{planta de cânhamo (feminina)}
  \end{Phonetics}
\end{Entry}

\begin{Entry}{种植}{9,12}{⽲、⽊}
  \begin{Phonetics}{种植}{zhong4zhi2}[][HSK 4]
    \definition{v.}{plantar; crescer; cultivar; enterrar as sementes de uma planta no solo; plantar as mudas de uma planta no solo}
  \end{Phonetics}
\end{Entry}

\begin{Entry}{种薯}{9,16}{⽲、⾋}
  \begin{Phonetics}{种薯}{zhong3shu3}
    \definition{s.}{tubérculo semente}
  \end{Phonetics}
\end{Entry}

\begin{Entry}{科}{9}{⽲}
  \begin{Phonetics}{科}{ke1}[][HSK 2]
    \definition*{s.}{Sobrenome Ke}
    \definition{s.}{um ramo de estudo acadêmico ou profissional |uma divisão ou subdivisão de uma unidade administrativa | família | instruções de palco no drama chinês clássico; nos roteiros de peças clássicas, termos usados para indicar as ações dos personagens | nível; classificação; categoria | sessão de exames; refere-se às disciplinas, notas e anos das provas para a seleção de candidatos a cargos públicos militares e civis na antiguidade | tecnológico | assunto | lei; regulamento; decreto | penalidade; pena; punição | treinamento profissional ou formal; curso profissionalizante}
    \definition{v.}{proferir uma sentença (penal)}
  \end{Phonetics}
\end{Entry}

\begin{Entry}{科技}{9,7}{⽲、⼿}
  \begin{Phonetics}{科技}{ke1 ji4}[][HSK 3]
    \definition{s.}{ciência e tecnologia}
  \end{Phonetics}
\end{Entry}

\begin{Entry}{科学}{9,8}{⽲、⼦}
  \begin{Phonetics}{科学}{ke1xue2}[][HSK 2]
    \definition{adj.}{científico; em conformidade com as leis da ciência}
    \definition[门,个,种]{s.}{ciência; um conjunto de conhecimentos que reflete as leis objetivas da natureza, da sociedade, do pensamento, etc.}
  \end{Phonetics}
\end{Entry}

\begin{Entry}{科学家}{9,8,10}{⽲、⼦、⼧}
  \begin{Phonetics}{科学家}{ke1xue2jia1}
    \definition[位,名,个]{s.}{cientista; pessoas com realizações significativas no campo da pesquisa científica}
  \end{Phonetics}
\end{Entry}

\begin{Entry}{科研}{9,9}{⽲、⽯}
  \begin{Phonetics}{科研}{ke1 yan2}[][HSK 6]
    \definition{s.}{pesquisa científica}
    \definition{v.}{envolver-se em pesquisa científica}
  \end{Phonetics}
\end{Entry}

\begin{Entry}{秒}{9}{⽲}
  \begin{Phonetics}{秒}{miao3}[][HSK 5]
    \definition{adv.}{instantaneamente}
    \definition{s.}{segundo (unidade de tempo) | segundo (unidade de medida angular)}
  \end{Phonetics}
\end{Entry}

\begin{Entry}{乘}{10}{⽲}
  \begin{Phonetics}{乘}{cheng2}[][HSK 5]
    \definition*{s.}{Sobrenome Cheng}
    \definition{s.}{uma divisão principal das escolas budistas; uma seita ou doutrina do budismo}
    \definition{v.}{cavalgar; andar a cavalo; utilizar um veículo ou animal em vez de caminhar | aproveitar-se de; valer-se de; tirar vantagem de; tirar proveito de | multiplicar; realizar multiplicação | perseguir; caçar}
  \end{Phonetics}
  \begin{Phonetics}{乘}{sheng4}
    \definition{clas.}{usado para carruagens de guerra puxada por quatro cavalos}
    \definition{s.}{obras históricas; livros de história geral | antigamente, uma carruagem puxada por quatro cavalos}
  \end{Phonetics}
\end{Entry}

\begin{Entry}{乘车}{10,4}{⽲、⾞}
  \begin{Phonetics}{乘车}{cheng2 che1}[][HSK 5]
    \definition{v.}{montar; dirigir; conduzir; andar a cavalo, de moto, de bicicleta, etc.}
  \end{Phonetics}
\end{Entry}

\begin{Entry}{乘坐}{10,7}{⽲、⼟}
  \begin{Phonetics}{乘坐}{cheng2zuo4}[][HSK 5]
    \definition{v.}{pegar (um trem, ônibus, etc.); andar de (bicicleta, moto, etc.)}
  \end{Phonetics}
\end{Entry}

\begin{Entry}{乘客}{10,9}{⽲、⼧}
  \begin{Phonetics}{乘客}{cheng2 ke4}[][HSK 5]
    \definition[个,位,名]{s.}{passageiro; pessoas viajando de carro, navio ou avião}
  \end{Phonetics}
\end{Entry}

\begin{Entry}{乘客数}{10,9,13}{⽲、⼧、⽁}
  \begin{Phonetics}{乘客数}{cheng2ke4 shu4}
    \definition{s.}{número de passageiros}
  \end{Phonetics}
\end{Entry}

\begin{Entry}{乘积}{10,10}{⽲、⽲}
  \begin{Phonetics}{乘积}{cheng2ji1}
    \definition{s.}{(matemática) produto (resultado da multiplicação)}
  \end{Phonetics}
\end{Entry}

\begin{Entry}{秘}{10}{⽲}
  \begin{Phonetics}{秘}{bi4}
    \definition*{s.}{Abreviação de Peru, 秘鲁 | Sobrenome Bi}
  \seealsoref{秘鲁}{bi4lu3}
  \end{Phonetics}
  \begin{Phonetics}{秘}{mi4}
    \definition{adj.}{secreto; misterioso | raro; raramente visto; estranho}
    \definition{adv.}{secretamente; privadamente}
    \definition{s.}{secretário}
    \definition{v.}{manter algo em segredo; esconder algo; guardar segredos | bloquear; obstruir; ter dificuldade para defecar}
  \end{Phonetics}
\end{Entry}

\begin{Entry}{秘书}{10,4}{⽲、⼄}
  \begin{Phonetics}{秘书}{mi4shu1}[][HSK 4]
    \definition[个,位,名]{s.}{o cargo de secretário; funções de secretariado | secretário; pessoas encarregadas da correspondência e que auxiliam o chefe do órgão ou departamento na condução diária de seu trabalho}
  \end{Phonetics}
\end{Entry}

\begin{Entry}{秘书长}{10,4,4}{⽲、⼄、⾧}
  \begin{Phonetics}{秘书长}{mi4 shu1 zhang3}[][HSK 6]
    \definition{s.}{secretário-geral}
  \end{Phonetics}
\end{Entry}

\begin{Entry}{秘密}{10,11}{⽲、⼧}
  \begin{Phonetics}{秘密}{mi4mi4}[][HSK 4]
    \definition{adj.}{secreto}
    \definition[个,条,些]{s.}{segredo; algo secreto; coisas que você não quer que as pessoas saibam}
  \end{Phonetics}
\end{Entry}

\begin{Entry}{秘鲁}{10,12}{⽲、⿂}
  \begin{Phonetics}{秘鲁}{bi4lu3}
    \definition*{s.}{Peru}
  \end{Phonetics}
\end{Entry}

\begin{Entry}{租}{10}{⽲}
  \begin{Phonetics}{租}{zu1}[][HSK 2]
    \definition{s.}{aluguel | imposto sobre a terra; tributação; (antigo) refere-se ao imposto predial}
    \definition{v.}{contratar; alugar; fretar | alugar; arrendar}
  \end{Phonetics}
\end{Entry}

\begin{Entry}{租用}{10,5}{⽲、⽤}
  \begin{Phonetics}{租用}{zu1yong4}
    \definition{v.}{contratar | alugar | alugar (algo de alguém)}
  \end{Phonetics}
\end{Entry}

\begin{Entry}{租让}{10,5}{⽲、⾔}
  \begin{Phonetics}{租让}{zu1rang4}
    \definition{v.}{alugar | alugar (a propriedade de alguém para outra pessoa)}
  \end{Phonetics}
\end{Entry}

\begin{Entry}{租约}{10,6}{⽲、⽷}
  \begin{Phonetics}{租约}{zu1yue1}
    \definition{s.}{aluguel}
  \end{Phonetics}
\end{Entry}

\begin{Entry}{租房}{10,8}{⽲、⼾}
  \begin{Phonetics}{租房}{zu1fang2}
    \definition{v.}{alugar um apartamento}
  \end{Phonetics}
\end{Entry}

\begin{Entry}{租金}{10,8}{⽲、⾦}
  \begin{Phonetics}{租金}{zu1 jin1}[][HSK 6]
    \definition[笔]{s.}{aluguel; aluguer; o custo do aluguel de terras, casas ou itens; a renda do aluguel de terras, casas ou itens}
  \seealsoref{租钱}{zu1qian5}
  \end{Phonetics}
\end{Entry}

\begin{Entry}{租赁}{10,10}{⽲、⾙}
  \begin{Phonetics}{租赁}{zu1lin4}
    \definition{v.}{contratar | alugar}
  \end{Phonetics}
\end{Entry}

\begin{Entry}{租钱}{10,10}{⽲、⾦}
  \begin{Phonetics}{租钱}{zu1qian5}
    \definition{s.}{aluguel}
  \seealsoref{租金}{zu1 jin1}
  \end{Phonetics}
\end{Entry}

\begin{Entry}{租船}{10,11}{⽲、⾈}
  \begin{Phonetics}{租船}{zu1chuan2}
    \definition{v.}{fretar um navio | alugar um navio}
  \end{Phonetics}
\end{Entry}

\begin{Entry}{积}{10}{⽲}
  \begin{Phonetics}{积}{ji1}
    \definition{adj.}{de longa data; pendente há muito tempo | antiquíssimo; acumulado ao longo de um longo período de tempo}
    \definition{s.}{(medicina chinesa) indigestão (em bebês e crianças) | (matemática)  abreviação de produto, 乘积}
    \definition{v.}{acumular; juntar; amontoar; reunir; coletar}
  \seealsoref{乘积}{cheng2ji1}
  \end{Phonetics}
\end{Entry}

\begin{Entry}{积木}{10,4}{⽲、⽊}
  \begin{Phonetics}{积木}{ji1mu4}
    \definition{s.}{blocos de montar (brinquedo)}
  \end{Phonetics}
\end{Entry}

\begin{Entry}{积极}{10,7}{⽲、⽊}
  \begin{Phonetics}{积极}{ji1ji2}[][HSK 3]
    \definition{adj.}{ativo; descreve uma atitude proativa e esforçada | positivo; que tem um efeito positivo e ajuda no desenvolvimento das coisas}
  \end{Phonetics}
\end{Entry}

\begin{Entry}{积累}{10,11}{⽲、⽷}
  \begin{Phonetics}{积累}{ji1lei3}[][HSK 4]
    \definition{s.}{acúmulo; acumulação}
    \definition{v.}{acumular}
  \end{Phonetics}
\end{Entry}

\begin{Entry}{称}{10}{⽲}
  \begin{Phonetics}{称}{chen4}
    \definition{adj.}{ajustado; encaixado; adequado}
    \definition{v.}{ajustar; adequar; combinar; estar em conformidade com; ser adequado para | ter; possuir}
  \end{Phonetics}
  \begin{Phonetics}{称}{cheng1}[][HSK 2,5]
    \definition*{s.}{Sobrenome Cheng}
    \definition{s.}{nome}
    \definition{v.}{chamar; ser chamado | dizer; declarar | elogiar; louvar; expressar afirmação ou elogio a pessoas ou coisas por meio de palavras | pesar; medir o peso | elevar; levantar; erguer | aplaudir; concordar; expressar suas opiniões ou sentimentos por meio de palavras ou ações | declarar-se como; declarar que é; reivindicar ser alguém em virtude do próprio poder}
  \end{Phonetics}
\end{Entry}

\begin{Entry}{称为}{10,4}{⽲、⼂}
  \begin{Phonetics}{称为}{cheng1 wei2}[][HSK 3]
    \definition{v.}{ser chamado de; ser conhecido como; denominar}
  \end{Phonetics}
\end{Entry}

\begin{Entry}{称号}{10,5}{⽲、⼝}
  \begin{Phonetics}{称号}{cheng1hao4}[][HSK 5]
    \definition{s.}{título; nome; designação; nome dado a alguém, a uma organização ou a alguma coisa (geralmente usado de forma honrosa)}
  \end{Phonetics}
\end{Entry}

\begin{Entry}{称赞}{10,16}{⽲、⾙}
  \begin{Phonetics}{称赞}{cheng1zan4}[][HSK 4]
    \definition[句,声,番,次]{s.}{elogio; aclamação; louvor; avaliação positiva de um desempenho ou conquista}
    \definition{v.}{elogiar; aclamar; louvar; usar palavras para expressar um carinho pelas virtudes de uma pessoa ou coisa}
  \end{Phonetics}
\end{Entry}

\begin{Entry}{移}{11}{⽲}
  \begin{Phonetics}{移}{yi2}[][HSK 4]
    \definition*{s.}{Sobrenome Yi}
    \definition{v.}{mover; remover; deslocar; mudar | mudar; alterar}
  \end{Phonetics}
\end{Entry}

\begin{Entry}{移民}{11,5}{⽲、⽒}
  \begin{Phonetics}{移民}{yi2min2}[][HSK 4]
    \definition[个,批]{s.}{emigrante; migrantes; aqueles que se mudam para um país ou estado estrangeiro para se estabelecer}
    \definition{v.}{migrar; imigrar}
  \end{Phonetics}
\end{Entry}

\begin{Entry}{移动}{11,6}{⽲、⼒}
  \begin{Phonetics}{移动}{yi2dong4}[][HSK 4]
    \definition{v.}{deslocar; mover; mudar}
  \end{Phonetics}
\end{Entry}

\begin{Entry}{程}{12}{⽲}
  \begin{Phonetics}{程}{cheng2}
    \definition{s.}{regra; regulamento; lei | ordem; procedimento | jornada; etapa de uma jornada; estrada; um trecho de estrada | distância percorrida ou movida por um objeto | programação | medição; termo geral para pesos e medidas}
  \end{Phonetics}
\end{Entry}

\begin{Entry}{程序}{12,7}{⽲、⼴}
  \begin{Phonetics}{程序}{cheng2xu4}[][HSK 4]
    \definition[个,套,种]{s.}{ordem; curso; sequência; procedimento; ordem em que algo é feito; também, um determinado número de etapas em um trabalho | programa; conjunto de instruções de computador projetado em sequência para permitir que um computador execute uma ou mais operações}
  \end{Phonetics}
\end{Entry}

\begin{Entry}{程序设计}{12,7,6,4}{⽲、⼴、⾔、⾔}
  \begin{Phonetics}{程序设计}{cheng2xu4she4ji4}
    \definition{s.}{programação de computadores}
  \end{Phonetics}
\end{Entry}

\begin{Entry}{程序库}{12,7,7}{⽲、⼴、⼴}
  \begin{Phonetics}{程序库}{cheng2xu4ku4}
    \definition{s.}{biblioteca de funções e procedimentos para programas de computador}
  \end{Phonetics}
\end{Entry}

\begin{Entry}{程度}{12,9}{⽲、⼴}
  \begin{Phonetics}{程度}{cheng2du4}[][HSK 3]
    \definition[种]{s.}{nível; grau (de cultura, educação, aprendizagem, etc.) | extensão; grau; a situação, o nível ou o estágio em que as coisas mudam}
  \end{Phonetics}
\end{Entry}

\begin{Entry}{程控}{12,11}{⽲、⼿}
  \begin{Phonetics}{程控}{cheng2kong4}
    \definition{s.}{programado | sob controle automático}
  \end{Phonetics}
\end{Entry}

\begin{Entry}{稍}{12}{⽲}
  \begin{Phonetics}{稍}{shao1}[][HSK 5]
    \definition{adv.}{ligeiramente; um pouco; um pouquinho}
  \end{Phonetics}
\end{Entry}

\begin{Entry}{稍微}{12,13}{⽲、⼻}
  \begin{Phonetics}{稍微}{shao1wei1}[][HSK 5]
    \definition{adv.}{um pouco; um pouquinho; uma ninharia; indica que a quantidade é pequena ou o grau é superficial}
  \end{Phonetics}
\end{Entry}

\begin{Entry}{税}{12}{⽲}
  \begin{Phonetics}{税}{shui4}[][HSK 6]
    \definition*{s.}{Sobrenome Shui}
    \definition{s.}{imposto; taxa; tarifa}
  \end{Phonetics}
\end{Entry}

\begin{Entry}{稳}{14}{⽲}
  \begin{Phonetics}{稳}{wen3}[][HSK 4]
    \definition{adj.}{constante; estável; firme | estável; estático; sedado | seguro; confiável; certo}
    \definition{adv.}{certamente; com certeza; seguramente; sem dúvida}
    \definition{v.}{estabilizar, manter estável; acalmar}
  \end{Phonetics}
\end{Entry}

\begin{Entry}{稳定}{14,8}{⽲、⼧}
  \begin{Phonetics}{稳定}{wen3ding4}[][HSK 4]
    \definition{adj.}{estável; firme; descreve uma natureza, um estado, etc. relativamente fixo; não muda significativamente}
    \definition{v.}{manter estável; estabilizar}
  \end{Phonetics}
\end{Entry}

\begin{Entry}{稿}{15}{⽲}
  \begin{Phonetics}{稿}{gao3}
    \definition[篇]{s.}{(significado original) talo de grão; palha | rascunho; esboço; manuscrito | texto original}
  \end{Phonetics}
\end{Entry}

\begin{Entry}{稿子}{15,3}{⽲、⼦}
  \begin{Phonetics}{稿子}{gao3 zi5}[][HSK 6]
    \definition[篇,份,堆,叠]{s.}{rascunho; esboço; rascunhos de poemas, ensaios, desenhos, etc. | rascunho; manuscrito; poemas escritos | ideia; plano; plano preliminar ou conceito de trabalho}
  \end{Phonetics}
\end{Entry}

\begin{Entry}{稿纸}{15,7}{⽲、⽷}
  \begin{Phonetics}{稿纸}{gao3zhi3}
    \definition{s.}{rascunho | manuscrito}
  \end{Phonetics}
\end{Entry}

%%%%% EOF %%%%%

