%%%
%%% Radical "⿊"
%%%

\section*{Radical 203: ``⿊''}\addcontentsline{toc}{section}{Radical 203: ⿊}

\begin{entry}{黑}{12}{⿊}[Kangxi 203]
  \begin{phonetics}{黑}{hei1}[][HSK 2]
    \definition*{s.}{Província de Heilongjiang, abreviação de 黑龙江 | Sobrenome Hei}
    \definition{adj.}{preto; cor semelhante à do carvão | escuro | obscuro; secreto | perverso; sinistro; ruim; cruel | reacionário}
    \definition{s.}{noite}
    \definition{v.}{fazer algo ilegalmente ou de forma desonesta; enganar; desviar dinheiro ilegalmente | invadir (uma rede, sites, computador, etc.)}
  \seealsoref{黑龙江}{hei1long2jiang1}
  \end{phonetics}
\end{entry}

\begin{entry}{黑龙江}{12,5,6}{⿊、⿓、⽔}
  \begin{phonetics}{黑龙江}{hei1long2jiang1}
    \definition*{s.}{Província de Heilongjiang | Rio Heilong Jiang;  Rio Amur (na Rússia)}
  \end{phonetics}
\end{entry}

\begin{entry}{黑色}{12,6}{⿊、⾊}
  \begin{phonetics}{黑色}{hei1 se4}[][HSK 2]
    \definition{adj.}{metafórico: suspeito, ilegal}
    \definition{s.}{cor preta}
  \end{phonetics}
\end{entry}

\begin{entry}{黑夜}{12,8}{⿊、⼣}
  \begin{phonetics}{黑夜}{hei1 ye4}[][HSK 6]
    \definition[个]{s.}{noite ; uma noite muito escura sem luz}
  \end{phonetics}
\end{entry}

\begin{entry}{黑板}{12,8}{⿊、⽊}
  \begin{phonetics}{黑板}{hei1ban3}[][HSK 2]
    \definition[块,个]{s.}{quadro negro; quadro de giz; uma placa, na qual se pode escrever com giz}
  \end{phonetics}
\end{entry}

\begin{entry}{黑客}{12,9}{⿊、⼧}
  \begin{phonetics}{黑客}{hei1ke4}
    \definition{s.}{(empréstimo linguístico) (computação) \emph{hacker}}
  \end{phonetics}
\end{entry}

\begin{entry}{黑桃}{12,10}{⿊、⽊}
  \begin{phonetics}{黑桃}{hei1 tao2}
    \definition{s.}{espadas ♠ (em jogos de cartas)}
  \seealsoref{方片}{fang1 pian4}
  \seealsoref{红心}{hong2 xin1}
  \seealsoref{梅花}{mei2 hua1}
  \end{phonetics}
\end{entry}

\begin{entry}{黑暗}{12,13}{⿊、⽇}
  \begin{phonetics}{黑暗}{hei1 an4}[][HSK 4]
    \definition{adj.}{escuro; sombrio; sem luz | maligno; corrupto; reacionário}
  \end{phonetics}
\end{entry}

\begin{entry}{墨}{15}{⿊}
  \begin{phonetics}{墨}{mo4}
    \definition*{s.}{Escola Moísta; Moísmo | México, abreviação de 墨西哥}
    \definition{adj.}{preto; escuro como breu | corrupto | escuro}
    \definition{s.}{tinta chinesa; bastão de tinta | pigmento; tinta | caligrafia ou pintura | aprendizagem; alfabetização | marcador de linha de carpinteiro; marcador de tinta | tatuar o rosto (um castigo); uma punição na China antiga | corrupção; peculato; fraude}
  \seealsoref{墨西哥}{mo4xi1ge1}
  \end{phonetics}
\end{entry}

\begin{entry}{墨水}{15,4}{⿊、⽔}
  \begin{phonetics}{墨水}{mo4 shui3}[][HSK 6]
    \definition[瓶]{s.}{tinta chinesa preparada; tinta (para caneta-tinteiro) | aprendizagem; alfabetização; uma metáfora para o conhecimento ou a capacidade de ler e escrever}
  \end{phonetics}
\end{entry}

\begin{entry}{墨西哥}{15,6,10}{⿊、⾑、⼝}
  \begin{phonetics}{墨西哥}{mo4xi1ge1}
    \definition*{s.}{México; Planalto no México}
  \end{phonetics}
\end{entry}

\begin{entry}{墨镜}{15,16}{⿊、⾦}
  \begin{phonetics}{墨镜}{mo4jing4}
    \definition[只,双,副]{s.}{óculos escuros}
  \end{phonetics}
\end{entry}

\begin{entry}{默}{16}{⿊}
  \begin{phonetics}{默}{mo4}
    \definition*{s.}{Sobrenome Mo}
    \definition{adj.}{taciturno; reservado | silencioso}
    \definition{v.}{escrever de memória}
  \end{phonetics}
\end{entry}

\begin{entry}{默契}{16,9}{⿊、⼤}
  \begin{phonetics}{默契}{mo4qi4}
    \definition{adj.}{(de membros da equipe) bem coordenados}
    \definition{s.}{entendimento tácito | entendimento mútuo | conectado em um nível mútuo profundo | (de membros da equipe) bem coordenados}
  \end{phonetics}
\end{entry}

\begin{entry}{默默}{16,16}{⿊、⿊}
  \begin{phonetics}{默默}{mo4mo4}[][HSK 4]
    \definition{adj.}{mudo; silencioso}
    \definition{adv.}{silenciosamente}
  \end{phonetics}
\end{entry}

%%%%% EOF %%%%%

