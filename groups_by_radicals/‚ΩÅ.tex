%%%
%%% Radical "⽁"
%%%

\section*{Radical 66: ``⽁'' (攵)}\addcontentsline{toc}{section}{Radical 66: ⽁、攵}

\begin{entry}{收}{6}{⽁}
  \begin{phonetics}{收}{shou1}[][HSK 2]
    \definition{expr.}{aos cuidados de (usado na linha de endereço após o nome)}
    \definition{v.}{receber | aceitar | coletar | colher | guardar}
  \end{phonetics}
\end{entry}

\begin{entry}{收入}{6,2}{⽁、⼊}
  \begin{phonetics}{收入}{shou1ru4}[][HSK 2]
    \definition[笔,个]{s.}{renda | salário}
    \definition{v.}{receber dinheiro |  coletar | receber}
  \end{phonetics}
\end{entry}

\begin{entry}{收买}{6,6}{⽁、⼄}
  \begin{phonetics}{收买}{shou1mai3}
    \definition{v.}{subornar | comprar}
  \end{phonetics}
\end{entry}

\begin{entry}{收回}{6,6}{⽁、⼞}
  \begin{phonetics}{收回}{shou1 hui2}[][HSK 4]
    \definition{v.}{retomar; recuperar; relembrar; recordar; receber de volta o que foi enviado ou emprestado, ou o dinheiro que foi emprestado ou usado | sacar; retirar; recolher; rescindir; cancelar (uma opinião, ordem, etc.)}
  \end{phonetics}
\end{entry}

\begin{entry}{收听}{6,7}{⽁、⼝}
  \begin{phonetics}{收听}{shou1 ting1}[][HSK 3]
    \definition{v.}{ouvir; escutar}
  \end{phonetics}
\end{entry}

\begin{entry}{收到}{6,8}{⽁、⼑}
  \begin{phonetics}{收到}{shou1 dao4}[][HSK 2]
    \definition{v.}{receber}
  \end{phonetics}
\end{entry}

\begin{entry}{收看}{6,9}{⽁、⽬}
  \begin{phonetics}{收看}{shou1 kan4}[][HSK 3]
    \definition{v.}{assistir (a um programa de TV)}
  \end{phonetics}
\end{entry}

\begin{entry}{收费}{6,9}{⽁、⾙}
  \begin{phonetics}{收费}{shou1 fei4}[][HSK 3]
    \definition{v.}{cobrar; cobrar taxas}
  \end{phonetics}
\end{entry}

\begin{entry}{收音机}{6,9,6}{⽁、⾳、⽊}
  \begin{phonetics}{收音机}{shou1yin1ji1}[][HSK 3]
    \definition[部,台]{s.}{rádio; sem fio}
  \end{phonetics}
\end{entry}

\begin{entry}{收益}{6,10}{⽁、⽫}
  \begin{phonetics}{收益}{shou1yi4}[][HSK 4]
    \definition{s.}{lucro; renda; benefício; ganhos; vantagens ou benefícios obtidos}
  \end{phonetics}
\end{entry}

\begin{entry}{收获}{6,10}{⽁、⾋}
  \begin{phonetics}{收获}{shou1huo4}[][HSK 4]
    \definition[次,番,份]{s.}{resultados; ganhos; metaforicamente falando, conhecimento, experiência, etc. obtidos em estudo ou trabalho; os resultados obtidos por meio de trabalho árduo | colheita; colheita de safras}
    \definition{v.}{colher; juntar as colheitas;}
  \end{phonetics}
\end{entry}

\begin{entry}{收据}{6,11}{⽁、⼿}
  \begin{phonetics}{收据}{shou1ju4}
    \definition[张]{s.}{recibo | \emph{voucher}}
  \end{phonetics}
\end{entry}

\begin{entry}{收敛}{6,11}{⽁、⽁}
  \begin{phonetics}{收敛}{shou1lian3}
    \definition{v.}{diminuir | desaparecer | fazer desaparecer | exercer restrição | conter (alegria, arrogância, etc.) | constringir | (matemática) convergir}
  \end{phonetics}
\end{entry}

\begin{entry}{改}{7}{⽁}
  \begin{phonetics}{改}{gai3}[][HSK 2]
    \definition*{s.}{sobrenome Gai}
    \definition{v.}{mudar | transformar | revisar | alterar | modificar | retificar | corrigir | mudar para (fazer outra coisa)}
  \end{phonetics}
\end{entry}

\begin{entry}{改正}{7,5}{⽁、⽌}
  \begin{phonetics}{改正}{gai3 zheng4}[][HSK 4]
    \definition{v.}{corrigir; emendar; mudar o errado para o correto}
  \end{phonetics}
\end{entry}

\begin{entry}{改良}{7,7}{⽁、⾉}
  \begin{phonetics}{改良}{gai3liang2}
    \definition{v.}{melhorar (algo) | reformar (um sistema)}
  \end{phonetics}
\end{entry}

\begin{entry}{改进}{7,7}{⽁、⾡}
  \begin{phonetics}{改进}{gai3jin4}[][HSK 3]
    \definition[个]{s.}{melhoria}
    \definition{v.}{aprimorar; aperfeiçoar; melhorar; tornar melhor
modificar}
  \end{phonetics}
\end{entry}

\begin{entry}{改变}{7,8}{⽁、⼜}
  \begin{phonetics}{改变}{gai3bian4}[][HSK 2]
    \definition{v.}{mudar | alterar | transformar | virar | converter | moldar | modificar}
  \end{phonetics}
\end{entry}

\begin{entry}{改革}{7,9}{⽁、⾰}
  \begin{phonetics}{改革}{gai3ge2}[][HSK 5]
    \definition[项,次,种]{s.}{reforma; reformação; iniciativas para aprimorar a inovação}
    \definition{v.}{reformar; transformar as antigas partes irracionais das coisas em novas que possam ser adaptadas à situação objetiva}
  \end{phonetics}
\end{entry}

\begin{entry}{改造}{7,10}{⽁、⾡}
  \begin{phonetics}{改造}{gai3 zao4}[][HSK 3]
    \definition{v.}{transformar; renovar | remodelar}
  \end{phonetics}
\end{entry}

\begin{entry}{改善}{7,12}{⽁、⼝}
  \begin{phonetics}{改善}{gai3shan4}[][HSK 4]
    \definition{v.}{melhorar; amenizar; mudar a situação original para torná-la melhor}
  \end{phonetics}
\end{entry}

\begin{entry}{改善关系}{7,12,6,7}{⽁、⼝、⼋、⽷}
  \begin{phonetics}{改善关系}{gai3shan4guan1xi5}
    \definition{v.}{melhorar a relação}
  \end{phonetics}
\end{entry}

\begin{entry}{改善通讯}{7,12,10,5}{⽁、⼝、⾡、⾔}
  \begin{phonetics}{改善通讯}{gai3shan4tong1xun4}
    \definition{v.}{melhorar a comunicação}
  \end{phonetics}
\end{entry}

\begin{entry}{政纲}{9,7}{⽁、⽷}
  \begin{phonetics}{政纲}{zheng4gang1}
    \definition{s.}{programa ou plataforma política}
  \end{phonetics}
\end{entry}

\begin{entry}{政府}{9,8}{⽁、⼴}
  \begin{phonetics}{政府}{zheng4fu3}[][HSK 4]
    \definition[个]{s.}{governo;  órgãos executivos do poder do Estado, ou seja, órgãos administrativos do Estado, como o Conselho de Estado (Governo Popular Central) e os governos populares locais em todos os níveis na China}
  \end{phonetics}
\end{entry}

\begin{entry}{政治}{9,8}{⽁、⽔}
  \begin{phonetics}{政治}{zheng4zhi4}[][HSK 4]
    \definition{s.}{política; assuntos políticos; questões políticas}
  \end{phonetics}
\end{entry}

\begin{entry}{政治局}{9,8,7}{⽁、⽔、⼫}
  \begin{phonetics}{政治局}{zheng4zhi4ju2}
    \definition{s.}{o principal comitê de políticas de um partido comunista}
  \end{phonetics}
\end{entry}

\begin{entry}{故}{9}{⽁}
  \begin{phonetics}{故}{gu4}
    \definition{conj.}{por isso | portanto | então}
  \end{phonetics}
\end{entry}

\begin{entry}{故乡}{9,3}{⽁、⼄}
  \begin{phonetics}{故乡}{gu4xiang1}[][HSK 3]
    \definition[个]{s.}{cidade natal; terra natal}
  \end{phonetics}
\end{entry}

\begin{entry}{故事}{9,8}{⽁、⼅}
  \begin{phonetics}{故事}{gu4shi4}
    \definition{s.}{prática antiga}
  \end{phonetics}
  \begin{phonetics}{故事}{gu4shi5}[][HSK 2]
    \definition{s.}{narrativa | história | conto}
  \end{phonetics}
\end{entry}

\begin{entry}{故宫}{9,9}{⽁、⼧}
  \begin{phonetics}{故宫}{gu4gong1}
    \definition*{s.}{Palácio Imperial | Cidade Proibida}
  \end{phonetics}
\end{entry}

\begin{entry}{故意}{9,13}{⽁、⼼}
  \begin{phonetics}{故意}{gu4yi4}[][HSK 2]
    \definition{adv.}{intencionalmente | deliberadamente | propositalmente}
  \end{phonetics}
\end{entry}

\begin{entry}{效果}{10,8}{⽁、⽊}
  \begin{phonetics}{效果}{xiao4guo3}[][HSK 3]
    \definition[种,个]{s.}{efeito; resultado | efeitos sonoros; vários sons ou fenômenos naturais criados para combinar com o enredo em dramas e filmes, como vento e chuva, tiros, fogo, neve, etc.}
  \end{phonetics}
\end{entry}

\begin{entry}{效率}{10,11}{⽁、⽞}
  \begin{phonetics}{效率}{xiao4lv4}[][HSK 4]
    \definition[种]{s.}{eficiência; produtividade}
  \end{phonetics}
\end{entry}

\begin{entry}{敎}{11}{⽁}
  \begin{phonetics}{敎}{jiao4}
    \variantof{教}
  \end{phonetics}
\end{entry}

\begin{entry}{救}{11}{⽁}
  \begin{phonetics}{救}{jiu4}[][HSK 3]
    \definition*{s.}{sobrenome Jiu}
    \definition{v.}{resgatar; salvar | ajudar; aliviar; socorrer}
  \end{phonetics}
\end{entry}

\begin{entry}{救出}{11,5}{⽁、⼐}
  \begin{phonetics}{救出}{jiu4chu1}
    \definition{v.}{resgatar | tirar do perigo}
  \end{phonetics}
\end{entry}

\begin{entry}{救护车}{11,7,4}{⽁、⼿、⾞}
  \begin{phonetics}{救护车}{jiu4hu4che1}
    \definition[辆]{s.}{ambulância}
  \end{phonetics}
\end{entry}

\begin{entry}{救命}{11,8}{⽁、⼝}
  \begin{phonetics}{救命}{jiu4ming4}
    \definition{interj.}{Socorro! | Salve-me!}
    \definition{v.+compl.}{salvar a vida de alguém}
  \end{phonetics}
\end{entry}

\begin{entry}{教}{11}{⽁}
  \begin{phonetics}{教}{jiao1}[][HSK 1]
    \definition{v.}{ensinar | lecionar}
  \end{phonetics}
  \begin{phonetics}{教}{jiao4}
    \definition*{s.}{sobrenome Jiao}
    \definition{s.}{religião | ensinamento}
    \definition{v.}{causar | como fazer algo | contar (explicar como fazer algo)}
  \end{phonetics}
\end{entry}

\begin{entry}{教长}{11,4}{⽁、⾧}
  \begin{phonetics}{教长}{jiao4zhang3}
    \definition{s.}{imã (Islã) | mulá}
  \end{phonetics}
\end{entry}

\begin{entry}{教训}{11,5}{⽁、⾔}
  \begin{phonetics}{教训}{jiao4xun4}[][HSK 4]
    \definition{s.}{moral; lição}
    \definition{v.}{repreender; educar; ensinar uma lição a alguém; dar uma bronca em alguém; dar um sermão em alguém (por ter cometido um erro, etc.)}
  \end{phonetics}
\end{entry}

\begin{entry}{教会}{11,6}{⽁、⼈}
  \begin{phonetics}{教会}{jiao1hui4}
    \definition{v.}{mostrar | ensinar}
  \end{phonetics}
  \begin{phonetics}{教会}{jiao4hui4}
    \definition{s.}{igreja cristã}
  \end{phonetics}
\end{entry}

\begin{entry}{教导}{11,6}{⽁、⼨}
  \begin{phonetics}{教导}{jiao4dao3}
    \definition{s.}{instrução | orientação | ensino}
    \definition{v.}{instruir | orientar | ensinar}
  \end{phonetics}
\end{entry}

\begin{entry}{教师}{11,6}{⽁、⼱}
  \begin{phonetics}{教师}{jiao4 shi1}[][HSK 2]
    \definition[个]{s.}{professor | mestre}
  \end{phonetics}
\end{entry}

\begin{entry}{教材}{11,7}{⽁、⽊}
  \begin{phonetics}{教材}{jiao4cai2}[][HSK 3]
    \definition[本,套]{s.}{livro didático; material didático}
  \end{phonetics}
\end{entry}

\begin{entry}{教学}{11,8}{⽁、⼦}
  \begin{phonetics}{教学}{jiao1xue2}
    \definition{v.}{ensinar (como um professor)}
  \end{phonetics}
  \begin{phonetics}{教学}{jiao4 xue2}[][HSK 2]
    \definition[次]{s.}{ensino | instrução}
  \end{phonetics}
\end{entry}

\begin{entry}{教学楼}{11,8,13}{⽁、⼦、⽊}
  \begin{phonetics}{教学楼}{jiao4 xue2 lou2}[][HSK 1]
    \definition{s.}{edifício de salas de aula}
  \end{phonetics}
\end{entry}

\begin{entry}{教官}{11,8}{⽁、⼧}
  \begin{phonetics}{教官}{jiao4guan1}
    \definition{s.}{instrutor militar}
  \end{phonetics}
\end{entry}

\begin{entry}{教练}{11,8}{⽁、⽷}
  \begin{phonetics}{教练}{jiao4lian4}[][HSK 3]
    \definition[个,位,名]{s.}{instrutor; treinador (esportes)}
    \definition{v.}{treinar}
  \end{phonetics}
\end{entry}

\begin{entry}{教育}{11,8}{⽁、⾁}
  \begin{phonetics}{教育}{jiao4yu4}[][HSK 2]
    \definition{s.}{educação}
    \definition{v.}{ensinar | educar}
  \end{phonetics}
\end{entry}

\begin{entry}{教室}{11,9}{⽁、⼧}
  \begin{phonetics}{教室}{jiao4shi4}[][HSK 2]
    \definition[间]{s.}{sala de aula}
  \end{phonetics}
\end{entry}

\begin{entry}{教堂}{11,11}{⽁、⼟}
  \begin{phonetics}{教堂}{jiao4tang2}
    \definition[间]{s.}{igreja | capela}
  \end{phonetics}
\end{entry}

\begin{entry}{教授}{11,11}{⽁、⼿}
  \begin{phonetics}{教授}{jiao4shou4}[][HSK 4]
    \definition[个,位]{s.}{professor (universitário)}
    \definition{v.}{ensinar; instruir; dar aulas; dar palestras}
  \end{phonetics}
\end{entry}

\begin{entry}{敢}{11}{⽁}
  \begin{phonetics}{敢}{gan3}[][HSK 3]
    \definition{adj.}{ousado; corajoso; bravo}
    \definition{adv.}{talvez; provavelmente}
    \definition{v.}{ousar; aventurar-se | ter a confiança de; ter certeza | tornar ousado; aventurar-se}
  \end{phonetics}
\end{entry}

\begin{entry}{敢情}{11,11}{⽁、⼼}
  \begin{phonetics}{敢情}{gan3qing5}
    \definition{adv.}{claro | acontece que\dots}
  \end{phonetics}
\end{entry}

\begin{entry}{散}{12}{⽁}
  \begin{phonetics}{散}{san3}
    \definition{adj.}{disperso; fragmentado; não integrado}
    \definition{s.}{medicamento em forma de pó}
    \definition{v.}{divergir; espalhar-se; separar-se; soltar-se; não se manter unido;  desintegrar}
  \end{phonetics}
  \begin{phonetics}{散}{san4}
    \definition{v.}{quebrar; fragmentar; dispersar | dar; distribuir; disseminar; divulgar | dissipar; deixar sai  | terminar um acordo ou contrato; demitir}
  \end{phonetics}
\end{entry}

\begin{entry}{散心}{12,4}{⽁、⼼}
  \begin{phonetics}{散心}{san4xin1}
    \definition{v.+compl.}{aliviar o tédio | desfrutar de uma diversão | estar despreocupado}
  \end{phonetics}
\end{entry}

\begin{entry}{散步}{12,7}{⽁、⽌}
  \begin{phonetics}{散步}{san4bu4}[][HSK 3]
    \definition{v.+compl.}{dar uma volta; passear; dar uma caminhada}
  \end{phonetics}
\end{entry}

\begin{entry}{敬礼}{12,5}{⽁、⽰}
  \begin{phonetics}{敬礼}{jing4li3}
    \definition{s.}{saudação}
    \definition{v.}{saudar}
  \end{phonetics}
\end{entry}

\begin{entry}{数}{13}{⽁}
  \begin{phonetics}{数}{shu3}[][HSK 2]
    \definition{v.}{contar
ser considerado excepcionalmente (bom, ruim, etc.)
enumerar; listar}
  \end{phonetics}
  \begin{phonetics}{数}{shu4}
    \definition{num.}{vários | alguns}
    \definition{s.}{número | figura | destino}
  \end{phonetics}
  \begin{phonetics}{数}{shuo4}
    \definition{adv.}{frequentemente | repetidamente}
  \end{phonetics}
\end{entry}

\begin{entry}{数字}{13,6}{⽁、⼦}
  \begin{phonetics}{数字}{shu4zi4}[][HSK 2]
    \definition{adj.}{digital}
    \definition[个]{s.}{dígito | figura | número | numeral | quantidade | montante}
  \end{phonetics}
\end{entry}

\begin{entry}{数学}{13,8}{⽁、⼦}
  \begin{phonetics}{数学}{shu4xue2}
    \definition{s.}{matemática (disciplina)}
  \end{phonetics}
\end{entry}

\begin{entry}{数码}{13,8}{⽁、⽯}
  \begin{phonetics}{数码}{shu4ma3}[][HSK 4]
    \definition{s.}{dígito; numeral; algarismo | número; quantidade (usado principalmente na linguagem falada)}
    \definition{v.}{digitalizar}
  \end{phonetics}
\end{entry}

\begin{entry}{数据}{13,11}{⽁、⼿}
  \begin{phonetics}{数据}{shu4ju4}[][HSK 4]
    \definition[些,个]{s.}{dados; valores com base nos quais são realizadas estatísticas, cálculos, pesquisas científicas ou projetos técnicos}
  \end{phonetics}
\end{entry}

\begin{entry}{数量}{13,12}{⽁、⾥}
  \begin{phonetics}{数量}{shu4liang4}[][HSK 3]
    \definition[个]{s.}{quantidade; quantum; quantia; magnitude; número}
  \end{phonetics}
\end{entry}

\begin{entry}{整}{16}{⽁}
  \begin{phonetics}{整}{zheng3}[][HSK 3]
    \definition*{s.}{sobrenome Zheng}
    \definition{adj.}{cheio; integral; inteiro; completo; sem defeitos | limpo; arrumado; em boa ordem | redondo (não é uma fração)}
    \definition{s.}{inteiro (número)}
    \definition{v.}{retificar; pôr em ordem | consertar; renovar; reparar |consertar; punir; fazer alguém sofrer | fazer; produzir; trabalhar}
  \end{phonetics}
\end{entry}

\begin{entry}{整个}{16,3}{⽁、⼈}
  \begin{phonetics}{整个}{zheng3ge4}[][HSK 3]
    \definition{adj.}{total; inteiro; completo}
  \end{phonetics}
\end{entry}

\begin{entry}{整天}{16,4}{⽁、⼤}
  \begin{phonetics}{整天}{zheng3 tian1}[][HSK 3]
    \definition{s.}{o dia todo; de manhã até a noite}
  \end{phonetics}
\end{entry}

\begin{entry}{整齐}{16,6}{⽁、⿑}
  \begin{phonetics}{整齐}{zheng3qi2}[][HSK 3]
    \definition{adj.}{limpo; arrumado; em boa ordem | uniforme; regular; relativamente consistente em tamanho, comprimento, grau, etc. | usado para descrever que as coisas necessárias estão todas prontas}
    \definition{v.}{estar em boas condições; deixar as coisas organizadas e arrumadas}
  \end{phonetics}
\end{entry}

\begin{entry}{整体}{16,7}{⽁、⼈}
  \begin{phonetics}{整体}{zheng3ti3}[][HSK 3]
    \definition[个]{s.}{todo; totalidade}
  \end{phonetics}
\end{entry}

\begin{entry}{整理}{16,11}{⽁、⽟}
  \begin{phonetics}{整理}{zheng3li3}[][HSK 3]
    \definition{v.}{organizar; reorganizar; classificar; ordenar; colocar em ordem}
  \end{phonetics}
\end{entry}

\begin{entry}{整整}{16,16}{⽁、⽁}
  \begin{phonetics}{整整}{zheng3 zheng3}[][HSK 3]
    \definition{adv.}{inteiramente; completamente; solidamente; continuamente}
  \end{phonetics}
\end{entry}

%%%%% EOF %%%%%

