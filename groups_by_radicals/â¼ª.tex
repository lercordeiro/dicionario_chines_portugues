%%%
%%% Radical "⼪"
%%%

\section*{Radical 43: ``⼪'' (尣)}\addcontentsline{toc}{section}{Radical 43: ⼪、尣}

\begin{entry}{尤}{4}{⼪}
  \begin{phonetics}{尤}{you2}
    \definition*{s.}{Sobrenome You}
    \definition{adj.}{excelente; peculiar; notável}
    \definition{adv.}{particularmente; especialmente}
    \definition{s.}{falha; erro | irregularidade}
    \definition{v.}{ter rancor contra; culpar}
  \end{phonetics}
\end{entry}

\begin{entry}{尤其}{4,8}{⼪、⼋}
  \begin{phonetics}{尤其}{you2qi2}[][HSK 5]
    \definition{adv.}{especialmente; particularmente; indica um grau mais avançado, equivalente a 更加}
  \seealsoref{更加}{geng4 jia1}
  \end{phonetics}
\end{entry}

\begin{entry}{尧}{6}{⼪}
  \begin{phonetics}{尧}{yao2}
    \definition*{s.}{Yao, um monarca lendário da China antiga | Sobrenome Yao}
  \end{phonetics}
\end{entry}

\begin{entry}{就}{12}{⼪}
  \begin{phonetics}{就}{jiu4}[][HSK 1]
    \definition{adv.}{de imediato; imediatamente; indica que algo ocorrerá em breve | tão cedo quanto; já; há muito tempo; indica que a ação ocorreu há muito tempo | assim que; logo depois; indica que os eventos se sucedem imediatamente | nesse caso; então; indica que, sob determinadas condições, ocorre naturalmente um determinado resultado | exatamente; precisamente; indica que é exatamente assim | apenas; meramente; somente | tantos quanto; enfatiza a quantidade | apenas; simplesmente; reforço da afirmação | colocado entre dois componentes idênticos, significa tolerância ou indiferença}
    \definition{prep.}{tirar proveito de alguém (algo); expressa condições, oportunidades, etc., equivalente a 趁 | quando se trata de alguém (algo); relativo a; com relação a; sobre; objeto ou escopo da introdução da ação |no local; introduz o local próximo ao qual a ação ocorreu}
    \definition{v.}{ser comido com; ir com; pratos, frutas, etc., acompanhados de alimentos básicos ou bebidas alcoólicas | aproximar-se; mover-se em direção a | ir para; assumir; empreender; envolver-se em; entrar em | realizar; fazer | tirar proveito de; acomodar-se a; adequar-se; encaixar-se | assumir; começar a entrar ou a exercer | seguir; acompanhar}
  \seealsoref{趁}{chen4}
  \end{phonetics}
\end{entry}

\begin{entry}{就业}{12,5}{⼪、⼀}
  \begin{phonetics}{就业}{jiu4ye4}[][HSK 3]
    \definition{v.+compl.}{conseguir um emprego; obter emprego; assumir uma ocupação; começar a trabalhar}
  \end{phonetics}
\end{entry}

\begin{entry}{就是}{12,9}{⼪、⽇}
  \begin{phonetics}{就是}{jiu4 shi4}[][HSK 3]
    \definition{adv.}{exatamente; precisamente; expressar concordância com a afirmação da outra pessoa ou confirmar que a afirmação da outra pessoa está correta | apenas; simplesmente; expressa afirmação, determinação ou ênfase, o significado específico deve ser determinado com base no contexto anterior ou posterior | usado para indicar escolha}
    \definition{conj.}{ainda que; mesmo que se reconheça que essa situação é verdadeira, a situação posterior não mudará}
    \definition{part.}{usado no final de uma frase para expressar afirmação}
  \end{phonetics}
\end{entry}

\begin{entry}{就是说}{12,9,9}{⼪、⽇、⾔}
  \begin{phonetics}{就是说}{jiu4 shi4 shuo1}[][HSK 6]
    \definition{interj.}{ou seja; isto é; em outras palavras; é frequentemente usado como uma interjeição em uma frase para indicar que as palavras seguintes são uma explicação ou esclarecimento das anteriores}
  \end{phonetics}
\end{entry}

\begin{entry}{就要}{12,9}{⼪、⾑}
  \begin{phonetics}{就要}{jiu4 yao4}[][HSK 2]
    \definition{adv.}{estar prestes a; estar indo para; estar no ponto de}
  \end{phonetics}
\end{entry}

\begin{entry}{就职}{12,11}{⼪、⽿}
  \begin{phonetics}{就职}{jiu4zhi2}
    \definition{v.}{assumir o cargo | assumir um posto}
  \end{phonetics}
\end{entry}

\begin{entry}{就算}{12,14}{⼪、⽵}
  \begin{phonetics}{就算}{jiu4 suan4}[][HSK 6]
    \definition{conj.}{mesmo que; concedido que; expressam uma relação hipotética e concessiva, frequentemente usadas com 也, equivalente a 即使}
  \seealsoref{即使}{ji2shi3}
  \seealsoref{也}{ye3}
  \end{phonetics}
\end{entry}

%%%%% EOF %%%%%

