%%%
%%% Radical "⼻"
%%%

\section*{Radical 60: ``⼻''}\addcontentsline{toc}{section}{Radical 60: ⼻}

\begin{Entry}{彻}{7}{⼻}
  \begin{Phonetics}{彻}{che4}
    \definition{adj.}{minucioso; completo; penetrante}
    \definition{adv.}{minuciosamente; profundamente}
  \end{Phonetics}
\end{Entry}

\begin{Entry}{彻夜}{7,8}{⼻、⼣}
  \begin{Phonetics}{彻夜}{che4ye4}[][HSK 7-9]
    \definition{adv.}{a noite toda; durante toda a noite; do anoitecer ao amanhecer}
  \end{Phonetics}
\end{Entry}

\begin{Entry}{彻底}{7,8}{⼻、⼴}
  \begin{Phonetics}{彻底}{che4di3}[][HSK 4]
    \definition{adj.}{minucioso; completo; exaustivo; profundo e completo; nada é deixado de fora}
  \end{Phonetics}
\end{Entry}

\begin{Entry}{彼}{8}{⼻}
  \begin{Phonetics}{彼}{bi3}
    \definition{s.}{aquele; aquilo (oposto a 此) ; outro | a outra parte}
  \seealsoref{此}{ci3}
  \end{Phonetics}
\end{Entry}

\begin{Entry}{彼此}{8,6}{⼻、⽌}
  \begin{Phonetics}{彼此}{bi3ci3}[][HSK 5]
    \definition{pron.}{um ao outro; uns com os outros; este e aquele têm algum tipo de relacionamento; ambas as partes}
  \end{Phonetics}
\end{Entry}

\begin{Entry}{往}{8}{⼻}
  \begin{Phonetics}{往}{wang3}[][HSK 2]
    \definition{adj.}{passado; anterior}
    \definition{prep.}{para; em direção a; na direção de}
    \definition{v.}{ir}
  \end{Phonetics}
\end{Entry}

\begin{Entry}{往日}{8,4}{⼻、⽇}
  \begin{Phonetics}{往日}{wang3ri4}
    \definition{adv.}{dias passados}
    \definition{s.}{o passado}
  \end{Phonetics}
\end{Entry}

\begin{Entry}{往生}{8,5}{⼻、⽣}
  \begin{Phonetics}{往生}{wang3sheng1}
    \definition{v.}{renascer | morrer | (Budismo) viver no paraíso}
  \end{Phonetics}
\end{Entry}

\begin{Entry}{往后}{8,6}{⼻、⼝}
  \begin{Phonetics}{往后}{wang3 hou4}[][HSK 6]
    \definition{s.}{de agora em diante; mais tarde; no futuro | na parte traseira; na parte de trás | para trás; depois; à ré}
  \end{Phonetics}
\end{Entry}

\begin{Entry}{往年}{8,6}{⼻、⼲}
  \begin{Phonetics}{往年}{wang3 nian2}[][HSK 6]
    \definition{s.}{(em) anos anteriores}
  \end{Phonetics}
\end{Entry}

\begin{Entry}{往来}{8,7}{⼻、⽊}
  \begin{Phonetics}{往来}{wang3 lai2}[][HSK 6]
    \definition{s.}{contatos comerciais; relações comerciais; relações diplomáticas | negociações; visitas mútuas; comunicação}
    \definition{v.}{ir e vir | contatar; ter relações}
  \end{Phonetics}
\end{Entry}

\begin{Entry}{往返}{8,7}{⼻、⾡}
  \begin{Phonetics}{往返}{wang3fan3}
    \definition{s.}{ida e volta}
    \definition{v.}{ir e voltar | ir e vir}
  \end{Phonetics}
\end{Entry}

\begin{Entry}{往事}{8,8}{⼻、⼅}
  \begin{Phonetics}{往事}{wang3shi4}
    \definition{s.}{acontecimentos anteriores | eventos passados}
  \end{Phonetics}
\end{Entry}

\begin{Entry}{往例}{8,8}{⼻、⼈}
  \begin{Phonetics}{往例}{wang3li4}
    \definition{s.}{prática (habitual) do passado | precedente}
  \end{Phonetics}
\end{Entry}

\begin{Entry}{往往}{8,8}{⼻、⼻}
  \begin{Phonetics}{往往}{wang3wang3}[][HSK 3]
    \definition{adv.}{frequentemente; muitas vezes; mais frequentemente do que não; indica que uma situação existe ou ocorre com frequência}
  \end{Phonetics}
\end{Entry}

\begin{Entry}{往昔}{8,8}{⼻、⽇}
  \begin{Phonetics}{往昔}{wang3xi1}
    \definition{s.}{o passado}
  \end{Phonetics}
\end{Entry}

\begin{Entry}{往复}{8,9}{⼻、⼢}
  \begin{Phonetics}{往复}{wang3fu4}
    \definition{s.}{para trás e para frente (por exemplo, da ação do pistão ou da bomba)}
    \definition{v.}{ir e voltar | fazer uma viagem de volta}
  \end{Phonetics}
\end{Entry}

\begin{Entry}{往迹}{8,9}{⼻、⾡}
  \begin{Phonetics}{往迹}{wang3ji4}
    \definition{s.}{Literário: eventos passados; coisa do passado; tempos antigos}
  \end{Phonetics}
\end{Entry}

\begin{Entry}{往程}{8,12}{⼻、⽲}
  \begin{Phonetics}{往程}{wang3cheng2}
    \definition{s.}{saída (de uma viagem de ônibus ou trem, etc.)}
  \end{Phonetics}
\end{Entry}

\begin{Entry}{征}{8}{⼻}
  \begin{Phonetics}{征}{zheng1}
    \definition{s.}{prova; evidência | sinal; símbolo; presságio; sinais de manifestação; fenômeno}
    \definition{v.}{viajar; fazer uma jornada; pegar o caminho mais longo | iniciar uma campanha; fazer uma expedição punitiva | convocar; selecionar; recrutar | cobrar; impor; coletar | solicitar; pedir; procurar}
  \end{Phonetics}
\end{Entry}

\begin{Entry}{征求}{8,7}{⼻、⽔}
  \begin{Phonetics}{征求}{zheng1qiu2}[][HSK 4]
    \definition{v.}{procurar; buscar; solicitar; pedir abertamente opiniões, pontos de vista, etc.}
  \end{Phonetics}
\end{Entry}

\begin{Entry}{征服}{8,8}{⼻、⽉}
  \begin{Phonetics}{征服}{zheng1fu2}[][HSK 4]
    \definition{v.}{conquistar; cativar; usar a força para fazer a outra parte se submeter | subjugar; dominar; convencer as pessoas com poder infeccioso}
  \end{Phonetics}
\end{Entry}

\begin{Entry}{待}{9}{⼻}
  \begin{Phonetics}{待}{dai1}[][HSK 5]
    \definition{v.}{ficar; permanecer | ir além (de um período de tempo)}
  \end{Phonetics}
  \begin{Phonetics}{待}{dai4}[][HSK 7-9]
    \definition*{s.}{Sobrenome Dai}
    \definition{v.}{tratar; lidar com | entreter; receber (convidados) | aguardar; esperar por | precisar; necessitar | desejar; pretender; querer}
  \end{Phonetics}
\end{Entry}

\begin{Entry}{待会儿}{9,6,2}{⼻、⼈、⼉}
  \begin{Phonetics}{待会儿}{dai1 hui4r5}[][HSK 6]
    \definition{adv.}{em um momento; depois de um tempo | mais tarde; depois}
  \end{Phonetics}
\end{Entry}

\begin{Entry}{待遇}{9,12}{⼻、⾡}
  \begin{Phonetics}{待遇}{dai4yu4}[][HSK 4]
    \definition[种,项,份]{s.}{tratamento; refere-se a direitos, status social, etc. | salário; ordenado; remuneração}
  \end{Phonetics}
\end{Entry}

\begin{Entry}{很}{9}{⼻}
  \begin{Phonetics}{很}{hen3}[][HSK 1]
    \definition{adv.}{muito; bastante; terrivelmente; indica um grau bastante elevado; definitivo; o mais alto}
  \end{Phonetics}
\end{Entry}

\begin{Entry}{很难说}{9,10,9}{⼻、⾫、⾔}
  \begin{Phonetics}{很难说}{hen3 nan2 shuo1}[][HSK 6]
    \definition{adj.}{difícil dizer}
  \end{Phonetics}
\end{Entry}

\begin{Entry}{律}{9}{⼻}
  \begin{Phonetics}{律}{lv4}
    \definition*{s.}{Sobrenome Lü}
    \definition{s.}{lei; regra; estatuto; regulamento}
    \definition{v.}{restringir; disciplinar; manter sob controle}
  \end{Phonetics}
\end{Entry}

\begin{Entry}{律师}{9,6}{⼻、⼱}
  \begin{Phonetics}{律师}{lv4shi1}[][HSK 4]
    \definition[名,个,位]{s.}{advogado; procurador; profissionais encarregados pelas partes ou nomeados pelo tribunal para auxiliar as partes no litígio, para comparecer ao tribunal para defesa e para tratar de assuntos jurídicos relacionados, de acordo com a lei}
  \end{Phonetics}
\end{Entry}

\begin{Entry}{徐}{10}{⼻}
  \begin{Phonetics}{徐}{xu2}
    \definition*{s.}{Sobrenome Xu}
    \definition{adv.}{lentamente; suavemente}
  \end{Phonetics}
\end{Entry}

\begin{Entry}{徐徐}{10,10}{⼻、⼻}
  \begin{Phonetics}{徐徐}{xu2xu2}
    \definition{adv.}{lentamente; suavemente}
  \end{Phonetics}
\end{Entry}

\begin{Entry}{徒}{10}{⼻}
  \begin{Phonetics}{徒}{tu2}
    \definition{adj.}{vazio; nu}
    \definition{adv.}{somente; meramente; apenas | a pé | em vão; sem sucesso; sem sucesso}
    \definition{s.}{aprendiz; aluno | seguidor; crente | (pejorativo) pessoas da mesma facção | (pejorativo) pessoa; companheiro | (prisão) pena; prisão; sentença | estudante}
    \definition{v.}{estar a pé | andar}
  \end{Phonetics}
\end{Entry}

\begin{Entry}{徒手}{10,4}{⼻、⼿}
  \begin{Phonetics}{徒手}{tu2shou3}
    \definition{adj.}{com as mãos vazias | desarmado | mão livre (desenho) | lutando mão-a-mão}
  \end{Phonetics}
\end{Entry}

\begin{Entry}{徒弟}{10,7}{⼻、⼸}
  \begin{Phonetics}{徒弟}{tu2di4}[][HSK 6]
    \definition[位,名,个]{s.}{discípulo; aprendiz; uma pessoa que aprende com um mestre; geralmente se refere a uma pessoa que aprende com um especialista}[他是我的徒弟。===Ele é meu aprendiz.]
  \end{Phonetics}
\end{Entry}

\begin{Entry}{得}{11}{⼻}
  \begin{Phonetics}{得}{de2}[][HSK 2]
    \definition{adj.}{adequado; apropriado | satisfeito; complacente; orgulhoso de si mesmo}
    \definition{interj.}{usado para encerrar uma conversa para indicar concordância ou proibição | usado quando a situação não é a esperada, para expressar impotência}
    \definition{v.}{obter (em oposição a 失); conseguir; ganhar |  (de um cálculo) igual; resultar em | estar pronto; estar acabado | pegar; apanhar; contrair uma doença}
    \definition{v.aux.}{usado antes de outros verbos para expressar permissão | usado antes de outros verbos para indicar que é possível (usado principalmente na forma negativa) | usado em conversas para indicar que não há necessidade de dizer mais nada}
  \seealsoref{失}{shi1}
  \end{Phonetics}
  \begin{Phonetics}{得}{de5}[][HSK 2]
    \definition{part.}{depois de um verbo ou adjetivo para expressar possibilidade ou capacidade | entre um verbo e seu complemento para expressar possibilidade | ligando um verbo ou um adjetivo a um complemento que descreve a maneira ou o grau}
  \end{Phonetics}
  \begin{Phonetics}{得}{dei3}[][HSK 4]
    \definition{v.}{precisar; expressa uma necessidade lógica, factual ou subjetiva; deve; é necessário | ter de; ser obrigado a; indica uma necessidade de vontade ou de fato | certamente irá; expressa a inevitabilidade da especulação}
  \end{Phonetics}
\end{Entry}

\begin{Entry}{得了}{11,2}{⼻、⼅}
  \begin{Phonetics}{得了}{de2le5}[][HSK 5]
    \definition{expr.}{Tudo bem!; É o bastante!}
  \end{Phonetics}
  \begin{Phonetics}{得了}{de2liao3}
    \definition{adj.}{(enfaticamente, em perguntas retóricas) possível; indica que a situação é séria (usado principalmente em perguntas retóricas ou formas negativas)}
  \end{Phonetics}
\end{Entry}

\begin{Entry}{得力}{11,2}{⼻、⼒}
  \begin{Phonetics}{得力}{de2li4}[][HSK 7-9]
    \definition{adj.}{capaz; competente; capaz de fazer coisas | eficiente; poderoso}
    \definition{v.}{beneficiar-se de; obter ajuda de; beneficiar}
  \end{Phonetics}
\end{Entry}

\begin{Entry}{得不偿失}{11,4,11,5}{⼻、⼀、⼈、⼤}
  \begin{Phonetics}{得不偿失}{de2bu4chang2shi1}[][HSK 7-9]
    \definition{expr.}{``A perda supera o ganho.''; ``Os ganhos não compensam as perdas.''; perder mais do que ganhar; ``O jogo não vale a pena.''; ``O que é ganho não compensa o que é perdido.''}
  \end{Phonetics}
\end{Entry}

\begin{Entry}{得以}{11,4}{⼻、⼈}
  \begin{Phonetics}{得以}{de2 yi3}[][HSK 5]
    \definition{v.}{ser capaz de; para que\dots possa (ou possa)\dots}
  \end{Phonetics}
\end{Entry}

\begin{Entry}{得分}{11,4}{⼻、⼑}
  \begin{Phonetics}{得分}{de2 fen1}[][HSK 3]
    \definition{s.}{pontuação; classificação; nota; pontuação obtida em jogos ou competições}
    \definition{v.}{fazer pontos; pontuar}
  \end{Phonetics}
\end{Entry}

\begin{Entry}{得天独厚}{11,4,9,9}{⼻、⼤、⽝、⼚}
  \begin{Phonetics}{得天独厚}{de2tian1du2hou4}[][HSK 7-9]
    \definition{expr.}{ser ricamente dotado pela natureza; abundar em dádivas da natureza; desfrutar de vantagens excepcionais | abençoado pelo céu | desfrutar de vantagens excepcionais | favorecido pela natureza}
  \end{Phonetics}
\end{Entry}

\begin{Entry}{得手}{11,4}{⼻、⼿}
  \begin{Phonetics}{得手}{de2shou3}[][HSK 7-9]
    \definition{adj.}{Coloquial: prático; conveniente e fácil de usar}
    \definition{v.}{fazer algo suavemente; ter sucesso; atingir seu objetivo | ir suavemente; sair; fazer bem; fazer as coisas sem problemas}
  \end{Phonetics}
\end{Entry}

\begin{Entry}{得出}{11,5}{⼻、⼐}
  \begin{Phonetics}{得出}{de2 chu1}[][HSK 2]
    \definition{v.}{chegar (a uma conclusão); obter (a um resultado); deduzir ou calcular (conclusão ou resultado)}
  \end{Phonetics}
\end{Entry}

\begin{Entry}{得失}{11,5}{⼻、⼤}
  \begin{Phonetics}{得失}{de2shi1}[][HSK 7-9]
    \definition{s.}{ganho e perda; sucesso e fracasso | méritos e deméritos; vantagens e desvantagens; prós e contras}
  \end{Phonetics}
\end{Entry}

\begin{Entry}{得当}{11,6}{⼻、⼹}
  \begin{Phonetics}{得当}{de2dang4}[][HSK 7-9]
    \definition{adj.}{apropriado; próprio; adequado | apto}
  \end{Phonetics}
\end{Entry}

\begin{Entry}{得体}{11,7}{⼻、⼈}
  \begin{Phonetics}{得体}{de2ti3}[][HSK 7-9]
    \definition{adj.}{(fala, comportamento, etc.) apropriado; moderado}
  \end{Phonetics}
\end{Entry}

\begin{Entry}{得到}{11,8}{⼻、⼑}
  \begin{Phonetics}{得到}{de2 dao4}[][HSK 1]
    \definition{v.}{obter; conseguir; ganhar; receber; possuir algo; adquirir}
  \end{Phonetics}
\end{Entry}

\begin{Entry}{得知}{11,8}{⼻、⽮}
  \begin{Phonetics}{得知}{de2zhi1}[][HSK 7-9]
    \definition{v.}{saber; ser informado de; aprender}
  \end{Phonetics}
\end{Entry}

\begin{Entry}{得益于}{11,10,3}{⼻、⽫、⼆}
  \begin{Phonetics}{得益于}{de2yi4 yu2}[][HSK 7-9]
    \definition{s.}{correlação positiva; benefício}
  \end{Phonetics}
\end{Entry}

\begin{Entry}{得意}{11,13}{⼻、⼼}
  \begin{Phonetics}{得意}{de2yi4}[][HSK 4]
    \definition{adj.}{complacente; orgulhoso de si mesmo; satisfeito consigo mesmo}
  \end{Phonetics}
\end{Entry}

\begin{Entry}{得意扬扬}{11,13,6,6}{⼻、⼼、⼿、⼿}
  \begin{Phonetics}{得意扬扬}{de2yi4-yang2yang2}[][HSK 7-9]
    \definition{expr.}{orgulhoso e complacente | estar imensamente orgulhoso; parecer triunfante}
  \end{Phonetics}
\end{Entry}

\begin{Entry}{得罪}{11,13}{⼻、⽹}
  \begin{Phonetics}{得罪}{de2zui4}[][HSK 7-9]
    \definition{v.}{ofender; desagradar; causar desprazer ou ressentimento}
  \end{Phonetics}
\end{Entry}

\begin{Entry}{循}{12}{⼻}
  \begin{Phonetics}{循}{xun2}
    \definition{v.}{seguir; cumprir; cumprir com}
  \end{Phonetics}
\end{Entry}

\begin{Entry}{循环}{12,8}{⼻、⽟}
  \begin{Phonetics}{循环}{xun2huan2}[][HSK 6]
    \definition{s.}{ciclo; circulação}
    \definition{v.}{circular; as coisas se movem ou mudam em um ciclo}
  \end{Phonetics}
\end{Entry}

\begin{Entry}{微}{13}{⼻}
  \begin{Phonetics}{微}{wei1}
    \definition{adj.}{minúsculo; leve | profundo; abstruso | humilde; tendo pouca influência; baixo \emph{status}}
    \definition{adv.}{pouco; ligeiramente; indica um grau menor, equivalente a 稍 ou 略}
    \definition{num.}{um milionésimo de uma determinada unidade de medida}
    \definition{suf.}{micro-}
  \seealsoref{略}{lve4}
  \seealsoref{稍}{shao1}
  \end{Phonetics}
\end{Entry}

\begin{Entry}{微风}{13,4}{⼻、⾵}
  \begin{Phonetics}{微风}{wei1feng1}
    \definition{s.}{brisa | vento leve}
  \end{Phonetics}
\end{Entry}

\begin{Entry}{微波炉}{13,8,8}{⼻、⽔、⽕}
  \begin{Phonetics}{微波炉}{wei1 bo1 lu2}[][HSK 6]
    \definition[台,个]{s.}{forno de micro-ondas}
  \end{Phonetics}
\end{Entry}

\begin{Entry}{微软}{13,8}{⼻、⾞}
  \begin{Phonetics}{微软}{wei1ruan3}
    \definition*{s.}{Microsoft Corporation}
  \end{Phonetics}
\end{Entry}

\begin{Entry}{微信}{13,9}{⼻、⼈}
  \begin{Phonetics}{微信}{wei1 xin4}[][HSK 4]
    \definition*[个,条]{s.}{WeChat, aplicativo gratuito lançado pela Tencent em 21 de janeiro de 2011 para fornecer serviços de mensagens instantâneas para terminais inteligentes}
  \end{Phonetics}
\end{Entry}

\begin{Entry}{微型}{13,9}{⼻、⼟}
  \begin{Phonetics}{微型}{wei1xing2}
    \definition{adj.}{minúsculo}
    \definition{pref.}{micro-; mini-}
    \definition{s.}{miniatura; microescala}
  \end{Phonetics}
\end{Entry}

\begin{Entry}{微型博客}{13,9,12,9}{⼻、⼟、⼗、⼧}
  \begin{Phonetics}{微型博客}{wei1xing2 bo2ke4}
    \definition{s.}{\emph{microblog}}
  \end{Phonetics}
\end{Entry}

\begin{Entry}{微笑}{13,10}{⼻、⽵}
  \begin{Phonetics}{微笑}{wei1xiao4}[][HSK 4]
    \definition[个,丝]{s.}{sorriso; sorriso sutil}
    \definition{v.}{sorrir; rir baixinho e sutilmente}
  \end{Phonetics}
\end{Entry}

\begin{Entry}{微博}{13,12}{⼻、⼗}
  \begin{Phonetics}{微博}{wei1 bo2}[][HSK 5]
    \definition*{s.}{Weibo (um aplicativo de mídia social chinês)}
    \definition[条]{s.}{\emph{microblog}; abreviação de 微型博客}
  \seealsoref{微型博客}{wei1xing2 bo2ke4}
  \end{Phonetics}
\end{Entry}

\begin{Entry}{德}{15}{⼻}
  \begin{Phonetics}{德}{de2}[][HSK 7-9]
    \definition*{s.}{Alemanha, abreviação de 德国 | Sobrenome De}
    \definition{s.}{virtude; moral; caráter moral; moralidade; conduta; qualidades políticas | coração; mente; pensamentos | bondade; favor; graça}
  \seealsoref{德国}{de2guo2}
  \end{Phonetics}
\end{Entry}

\begin{Entry}{德国}{15,8}{⼻、⼞}
  \begin{Phonetics}{德国}{de2guo2}
    \definition*{s.}{Alemanha}
  \end{Phonetics}
\end{Entry}

\begin{Entry}{德国人}{15,8,2}{⼻、⼞、⼈}
  \begin{Phonetics}{德国人}{de2guo2ren2}
    \definition{s.}{alemão | pessoa ou povo da Alemanha}
  \end{Phonetics}
\end{Entry}

%%%%% EOF %%%%%

