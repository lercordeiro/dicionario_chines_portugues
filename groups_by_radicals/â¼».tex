%%%
%%% Radical "⼻"
%%%

\section*{Radical 60: ``⼻''}\addcontentsline{toc}{section}{Radical 60: ⼻}

\begin{entry}{彻底}{7,8}{⼻、⼴}
  \begin{phonetics}{彻底}{che4di3}[][HSK 4]
    \definition{adj.}{minucioso; completo; exaustivo; profundo e completo; nada é deixado de fora}
  \end{phonetics}
\end{entry}

\begin{entry}{彼此}{8,6}{⼻、⽌}
  \begin{phonetics}{彼此}{bi3ci3}[][HSK 5]
    \definition{pron.}{um ao outro; uns com os outros; este e aquele têm algum tipo de relacionamento; ambas as partes}
  \end{phonetics}
\end{entry}

\begin{entry}{往}{8}{⼻}
  \begin{phonetics}{往}{wang3}[][HSK 2]
    \definition{prep.}{para | em direção a}
  \end{phonetics}
\end{entry}

\begin{entry}{往日}{8,4}{⼻、⽇}
  \begin{phonetics}{往日}{wang3ri4}
    \definition{adv.}{dias passados}
    \definition{s.}{o passado}
  \end{phonetics}
\end{entry}

\begin{entry}{往生}{8,5}{⼻、⽣}
  \begin{phonetics}{往生}{wang3sheng1}
    \definition{v.}{renascer | morrer | (Budismo) viver no paraíso}
  \end{phonetics}
\end{entry}

\begin{entry}{往来}{8,7}{⼻、⽊}
  \begin{phonetics}{往来}{wang3lai2}
    \definition{s.}{contatos | negociações}
  \end{phonetics}
\end{entry}

\begin{entry}{往返}{8,7}{⼻、⾡}
  \begin{phonetics}{往返}{wang3fan3}
    \definition{s.}{ida e volta}
    \definition{v.}{ir e voltar | ir e vir}
  \end{phonetics}
\end{entry}

\begin{entry}{往事}{8,8}{⼻、⼅}
  \begin{phonetics}{往事}{wang3shi4}
    \definition{s.}{acontecimentos anteriores | eventos passados}
  \end{phonetics}
\end{entry}

\begin{entry}{往例}{8,8}{⼻、⼈}
  \begin{phonetics}{往例}{wang3li4}
    \definition{s.}{prática (habitual) do passado | precedente}
  \end{phonetics}
\end{entry}

\begin{entry}{往往}{8,8}{⼻、⼻}
  \begin{phonetics}{往往}{wang3wang3}[][HSK 3]
    \definition{adv.}{frequentemente; muitas vezes; mais frequentemente do que não}
  \end{phonetics}
\end{entry}

\begin{entry}{往昔}{8,8}{⼻、⽇}
  \begin{phonetics}{往昔}{wang3xi1}
    \definition{s.}{o passado}
  \end{phonetics}
\end{entry}

\begin{entry}{往复}{8,9}{⼻、⼢}
  \begin{phonetics}{往复}{wang3fu4}
    \definition{s.}{para trás e para frente (por exemplo, da ação do pistão ou da bomba)}
    \definition{v.}{ir e voltar | fazer uma viagem de volta}
  \end{phonetics}
\end{entry}

\begin{entry}{往迹}{8,9}{⼻、⾡}
  \begin{phonetics}{往迹}{wang3ji4}
    \definition{s.}{eventos passados}
  \end{phonetics}
\end{entry}

\begin{entry}{往程}{8,12}{⼻、⽲}
  \begin{phonetics}{往程}{wang3cheng2}
    \definition{s.}{saída (de uma viagem de ônibus ou trem, etc.)}
  \end{phonetics}
\end{entry}

\begin{entry}{征求}{8,7}{⼻、⽔}
  \begin{phonetics}{征求}{zheng1qiu2}[][HSK 4]
    \definition{v.}{procurar; buscar; solicitar; pedir abertamente opiniões, pontos de vista, etc.}
  \end{phonetics}
\end{entry}

\begin{entry}{征服}{8,8}{⼻、⽉}
  \begin{phonetics}{征服}{zheng1fu2}[][HSK 4]
    \definition{v.}{conquistar; cativar | subjugar; dominar}
  \end{phonetics}
\end{entry}

\begin{entry}{待}{9}{⼻}
  \begin{phonetics}{待}{dai1}[][HSK 5]
    \definition{v.}{ficar; permanecer | ir além (de um período de tempo)}
  \end{phonetics}
  \begin{phonetics}{待}{dai4}
    \definition{s.}{sobrenome Dai}
    \definition{v.}{tratar; lidar com | entreter; receber (convidados) | aguardar; esperar por | precisar; necessitar | desejar; pretender; querer}
  \end{phonetics}
\end{entry}

\begin{entry}{待遇}{9,12}{⼻、⾡}
  \begin{phonetics}{待遇}{dai4yu4}[][HSK 4]
    \definition[种,项,份]{s.}{tratamento; refere-se a direitos, status social, etc. | salário; ordenado; remuneração}
  \end{phonetics}
\end{entry}

\begin{entry}{很}{9}{⼻}
  \begin{phonetics}{很}{hen3}[][HSK 1]
    \definition{adv.}{bastante | muito | terrivelmente | advérbio de grau}
  \end{phonetics}
\end{entry}

\begin{entry}{律师}{9,6}{⼻、⼱}
  \begin{phonetics}{律师}{lv4shi1}[][HSK 4]
    \definition[名,个,位]{s.}{advogado; procurador; profissionais encarregados pelas partes ou nomeados pelo tribunal para auxiliar as partes no litígio, para comparecer ao tribunal para defesa e para tratar de assuntos jurídicos relacionados, de acordo com a lei}
  \end{phonetics}
\end{entry}

\begin{entry}{徒手}{10,4}{⼻、⼿}
  \begin{phonetics}{徒手}{tu2shou3}
    \definition{adj.}{com as mãos vazias | desarmado | mão livre (desenho) | lutando mão-a-mão}
  \end{phonetics}
\end{entry}

\begin{entry}{得}{11}{⼻}
  \begin{phonetics}{得}{de2}[][HSK 2]
    \definition{adj.}{adequado; apropriado | satisfeito; complacente; orgulhoso de si mesmo}
    \definition{interj.}{usado para encerrar uma conversa para indicar concordância ou proibição | usado quando a situação não é a esperada, para expressar impotência}
    \definition{v.}{obter (em vez de ``perder''); conseguir; ganhar | (de um cálculo) igual; resultar em; efetuar cálculos para produzir resultados | estar terminado; estar pronto; cumprir | contrair uma doença}
    \definition{v.aux.}{usado antes de outros verbos para expressar permissão | usado antes de outros verbos para indicar que é possível (usado principalmente na forma negativa) | usado em conversas para indicar que não há necessidade de dizer mais nada}
  \end{phonetics}
  \begin{phonetics}{得}{de5}[][HSK 2]
    \definition{part.}{depois de um verbo ou adjetivo para expressar possibilidade ou capacidade | entre um verbo e seu complemento para expressar possibilidade | ligando um verbo ou um adjetivo a um complemento que descreve a maneira ou o grau}
  \end{phonetics}
  \begin{phonetics}{得}{dei3}[][HSK 4]
    \definition{v.}{precisar; dever; expressa uma necessidade racional, factual ou subjetiva | dever; ter que; necessidade expressa, volitiva ou factual}
  \end{phonetics}
\end{entry}

\begin{entry}{得了}{11,2}{⼻、⼅}
  \begin{phonetics}{得了}{de2le5}[][HSK 5]
    \definition{expr.}{Tudo bem!; É o bastante!}
  \end{phonetics}
  \begin{phonetics}{得了}{de2liao3}
    \definition{adj.}{(enfaticamente, em perguntas retóricas) possível; indica que a situação é séria (usado principalmente em perguntas retóricas ou formas negativas)}
  \end{phonetics}
\end{entry}

\begin{entry}{得以}{11,4}{⼻、⼈}
  \begin{phonetics}{得以}{de2 yi3}[][HSK 5]
    \definition{v.}{ser capaz de; para que\dots possa (ou possa)\dots}
  \end{phonetics}
\end{entry}

\begin{entry}{得分}{11,4}{⼻、⼑}
  \begin{phonetics}{得分}{de2 fen1}[][HSK 3]
    \definition{v.}{fazer pontos; pontuar}
  \end{phonetics}
\end{entry}

\begin{entry}{得出}{11,5}{⼻、⼐}
  \begin{phonetics}{得出}{de2 chu1}[][HSK 2]
    \definition{v.}{chegar (a uma conclusão) | obter (a um resultado)}
  \end{phonetics}
\end{entry}

\begin{entry}{得到}{11,8}{⼻、⼑}
  \begin{phonetics}{得到}{de2 dao4}[][HSK 1]
    \definition{v.}{obter | receber}
  \end{phonetics}
\end{entry}

\begin{entry}{得意}{11,13}{⼻、⼼}
  \begin{phonetics}{得意}{de2yi4}[][HSK 4]
    \definition{adj.}{complacente; orgulhoso de si mesmo; satisfeito consigo mesmo}
    \definition{v.+compl.}{orgulhar-se de si mesmo; ter satisfação consigo mesmo; ser complacente}
  \end{phonetics}
\end{entry}

\begin{entry}{微风}{13,4}{⼻、⾵}
  \begin{phonetics}{微风}{wei1feng1}
    \definition{s.}{brisa | vento leve}
  \end{phonetics}
\end{entry}

\begin{entry}{微软}{13,8}{⼻、⾞}
  \begin{phonetics}{微软}{wei1ruan3}
    \definition*{s.}{\emph{Microsoft Corporation}}
  \end{phonetics}
\end{entry}

\begin{entry}{微信}{13,9}{⼻、⼈}
  \begin{phonetics}{微信}{wei1 xin4}[][HSK 4]
    \definition*{s.}{\emph{WeChat}; aplicativo gratuito lançado pela Tencent em 21 de janeiro de 2011 para fornecer serviços de mensagens instantâneas para terminais inteligentes}
  \end{phonetics}
\end{entry}

\begin{entry}{微型}{13,9}{⼻、⼟}
  \begin{phonetics}{微型}{wei1xing2}
    \definition{pref.}{micro-}
    \definition{s.}{miniatura}
  \end{phonetics}
\end{entry}

\begin{entry}{微笑}{13,10}{⼻、⽵}
  \begin{phonetics}{微笑}{wei1xiao4}[][HSK 4]
    \definition[个,丝]{s.}{sorriso;}
    \definition{v.}{sorrir}
  \end{phonetics}
\end{entry}

\begin{entry}{微博}{13,12}{⼻、⼗}
  \begin{phonetics}{微博}{wei1 bo2}[][HSK 5]
    \definition*{s.}{Weibo (um aplicativo de mídia social chinês)}
    \definition[条]{s.}{\emph{microblog}}
  \end{phonetics}
\end{entry}

\begin{entry}{德}{15}{⼻}
  \begin{phonetics}{德}{de2}
    \definition*{s.}{Alemanha, abreviação de 德国}
    \definition{s.}{virtude | bondade | moralidade | ética | personagem | tipo}
  \seealsoref{德国}{de2guo2}
  \end{phonetics}
\end{entry}

\begin{entry}{德国}{15,8}{⼻、⼞}
  \begin{phonetics}{德国}{de2guo2}
    \definition*{s.}{Alemanha}
  \end{phonetics}
\end{entry}

\begin{entry}{德国人}{15,8,2}{⼻、⼞、⼈}
  \begin{phonetics}{德国人}{de2guo2ren2}
    \definition{s.}{alemão | pessoa ou povo da Alemanha}
  \end{phonetics}
\end{entry}

%%%%% EOF %%%%%

