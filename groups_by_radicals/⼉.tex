%%%
%%% Radical "⼉"
%%%

\section*{Radical 10: ``⼉''}\addcontentsline{toc}{section}{Radical 10: ⼉}

\begin{entry}{儿}{2}{⼉}
  \begin{phonetics}{儿}{er2}
    \definition{s.}{criança | filho}
  \end{phonetics}
  \begin{phonetics}{儿}{r5}
    \definition{suf.}{sufixo diminutivo não silábico | final retroflexo}
  \end{phonetics}
  \begin{phonetics}{儿}{ren2}
    \definition{s.}{pessoa, radical em caracteres chineses}
    \variantof{人}
  \end{phonetics}
\end{entry}

\begin{entry}{儿子}{2,3}{⼉、⼦}
  \begin{phonetics}{儿子}{er2zi5}
    \definition{s.}{filho}
  \seealsoref{女儿}{nv3'er2}
  \end{phonetics}
\end{entry}

\begin{entry}{儿童}{2,12}{⼉、⽴}
  \begin{phonetics}{儿童}{er2tong2}[][HSK 4]
    \definition[个,群]{s.}{criança; menor de idade (mais jovem do que ``少年'')}
  \seealsoref{少年}{shao4 nian2}
  \end{phonetics}
\end{entry}

\begin{entry}{儿媳}{2,13}{⼉、⼥}
  \begin{phonetics}{儿媳}{er2xi2}
    \definition{s.}{esposa do filho}
  \end{phonetics}
\end{entry}

\begin{entry}{元}{4}{⼉}
  \begin{phonetics}{元}{yuan2}[][HSK 1]
    \definition*{s.}{sobrenome Yuan | Dinastia Yuan (1279-1368)}
    \definition{clas.}{unidade monetária da China}
  \end{phonetics}
\end{entry}

\begin{entry}{元气}{4,4}{⼉、⽓}
  \begin{phonetics}{元气}{yuan2qi4}
    \definition{s.}{força | vigor | vitalidade | energial vital}
  \end{phonetics}
\end{entry}

\begin{entry}{元旦}{4,5}{⼉、⽇}
  \begin{phonetics}{元旦}{yuan2dan4}
    \definition*{s.}{Dia de Ano Novo (1 de janeiro)}
  \end{phonetics}
\end{entry}

\begin{entry}{元来}{4,7}{⼉、⽊}
  \begin{phonetics}{元来}{yuan2lai2}
    \variantof{原来}
  \end{phonetics}
\end{entry}

\begin{entry}{元夜}{4,8}{⼉、⼣}
  \begin{phonetics}{元夜}{yuan2ye4}
    \definition*{s.}{Festival das Lanternas}
  \seealsoref{元宵}{yuan2xiao1}
  \seealsoref{元宵节}{yuan2xiao1jie2}
  \end{phonetics}
\end{entry}

\begin{entry}{元宵}{4,10}{⼉、⼧}
  \begin{phonetics}{元宵}{yuan2xiao1}
    \definition*{s.}{Festival das Lanternas}
  \seealsoref{元宵节}{yuan2xiao1jie2}
  \seealsoref{元夜}{yuan2ye4}
  \end{phonetics}
\end{entry}

\begin{entry}{元宵节}{4,10,5}{⼉、⼧、⾋}
  \begin{phonetics}{元宵节}{yuan2xiao1jie2}
    \definition*{s.}{Festival das Lanternas (15º~dia do primeiro mês lunar)}
  \seealsoref{元宵}{yuan2xiao1}
  \seealsoref{元夜}{yuan2ye4}
  \end{phonetics}
\end{entry}

\begin{entry}{兄弟}{5,7}{⼉、⼸}
  \begin{phonetics}{兄弟}{xiong1di4}[][HSK 4]
    \definition{adj.}{fraternal}
    \definition{pron.}{eu, me (termo de uso humilde por homens em discurso público)}
    \definition[个,对]{s.}{irmãos; irmão}
  \end{phonetics}
\end{entry}

\begin{entry}{充分}{6,4}{⼉、⼑}
  \begin{phonetics}{充分}{chong1fen4}[][HSK 4]
    \definition{adj.}{cheio; amplo; abundante; suficiente; adequado}
    \definition{adv.}{totalmente; até o fim}
  \end{phonetics}
\end{entry}

\begin{entry}{充电}{6,5}{⼉、⽥}
  \begin{phonetics}{充电}{chong1 dian4}[][HSK 4]
    \definition{v.}{carregar (uma bateria); conectar uma fonte de alimentação CC aos terminais da bateria para recarregar a bateria | relaxar; passar o tempo livre; ``recarregar as baterias''; estudar para adquirir mais conhecimento; reabastecer (ou ampliar) o conhecimento; metaforicamente falando, para reabastecer a força física e a energia por meio do descanso e da recreação; também metaforicamente falando, para reabastecer novos conhecimentos e desenvolver novas habilidades por meio do reaprendizado}
  \end{phonetics}
\end{entry}

\begin{entry}{充电器}{6,5,16}{⼉、⽥、⼝}
  \begin{phonetics}{充电器}{chong1dian4qi4}[][HSK 4]
    \definition{s.}{carregador de bateria; dispositivo para alimentar uma bateria com energia, forçando uma corrente através dela}
  \end{phonetics}
\end{entry}

\begin{entry}{充满}{6,13}{⼉、⽔}
  \begin{phonetics}{充满}{chong1man3}[][HSK 3]
    \definition{v.}{preencher | encher-se de; transbordar de; permear-se de}
  \end{phonetics}
\end{entry}

\begin{entry}{兆}{6}{⼉}
  \begin{phonetics}{兆}{zhao4}
    \definition{num.}{trilhão}
  \end{phonetics}
\end{entry}

\begin{entry}{先}{6}{⼉}
  \begin{phonetics}{先}{xian1}[][HSK 1]
    \definition{adv.}{em primeiro lugar | primeiramente | antes do tempo | de antemão}
  \end{phonetics}
\end{entry}

\begin{entry}{先不先}{6,4,6}{⼉、⼀、⼉}
  \begin{phonetics}{先不先}{xian1bu4xian1}
    \definition{adv.}{(dialeto) antes de tudo | em primeiro lugar}
  \end{phonetics}
\end{entry}

\begin{entry}{先天}{6,4}{⼉、⼤}
  \begin{phonetics}{先天}{xian1tian1}
    \definition{adj.}{congênito | inato | natural}
    \definition{s.}{período embrionário}
  \end{phonetics}
\end{entry}

\begin{entry}{先生}{6,5}{⼉、⽣}
  \begin{phonetics}{先生}{xian1sheng5}[][HSK 1]
    \definition[位]{s.}{senhor | marido | professor | (dialeto) doutor}
  \end{phonetics}
\end{entry}

\begin{entry}{先有}{6,6}{⼉、⽉}
  \begin{phonetics}{先有}{xian1you3}
    \definition{adj.}{preexistente | anterior}
  \end{phonetics}
\end{entry}

\begin{entry}{先进}{6,7}{⼉、⾡}
  \begin{phonetics}{先进}{xian1jin4}[][HSK 3]
    \definition{adj.}{avançado}
    \definition{s.}{indivíduo avançado; grupo avançado}
  \end{phonetics}
\end{entry}

\begin{entry}{先到先得}{6,8,6,11}{⼉、⼑、⼉、⼻}
  \begin{phonetics}{先到先得}{xian1dao4xian1de2}
    \definition{expr.}{primeiro a chegar | primeiro a ser servido}
  \end{phonetics}
\end{entry}

\begin{entry}{先烈}{6,10}{⼉、⽕}
  \begin{phonetics}{先烈}{xian1lie4}
    \definition{s.}{mártir}
  \end{phonetics}
\end{entry}

\begin{entry}{先验}{6,10}{⼉、⾺}
  \begin{phonetics}{先验}{xian1yan4}
    \definition{adj.}{(filosofia) a priori}
  \end{phonetics}
\end{entry}

\begin{entry}{先期}{6,12}{⼉、⽉}
  \begin{phonetics}{先期}{xian1qi1}
    \definition{adv.}{antecipadamente}
    \definition{s.}{prematuro | \emph{front-end}}
  \end{phonetics}
\end{entry}

\begin{entry}{光}{6}{⼉}
  \begin{phonetics}{光}{guang1}[][HSK 3]
    \definition*{s.}{sobrenome Guang}
    \definition{adj.}{suave; brilhante | nu; despido; descoberto | esgotado; sem nada sobrando | glorioso; gracioso | brilhante}
    \definition{adv.}{somente; sozinho; meramente}
    \definition{s.}{luz; raio | cenário | honra; glória; brilho | claridade | favor; graça | momento | corpo celeste}
    \definition{v.}{glorificar; recuperar; reconquistar | estar nu | brilhar}
  \end{phonetics}
\end{entry}

\begin{entry}{光污染}{6,6,9}{⼉、⽔、⽊}
  \begin{phonetics}{光污染}{guang1 wu1ran3}
    \definition{s.}{poluição luminosa}
  \end{phonetics}
\end{entry}

\begin{entry}{光明}{6,8}{⼉、⽇}
  \begin{phonetics}{光明}{guang1ming2}[][HSK 3]
    \definition{adj.}{brilhante | ingênuo | justo; honesto}
    \definition{s.}{luz}
  \end{phonetics}
\end{entry}

\begin{entry}{光临}{6,9}{⼉、⼁}
  \begin{phonetics}{光临}{guang1lin2}[][HSK 4]
    \definition{v.}{honrar com sua presença, uma palavra de honra, usada para dizer que um convidado chegou}
  \end{phonetics}
\end{entry}

\begin{entry}{光盘}{6,11}{⼉、⽫}
  \begin{phonetics}{光盘}{guang1pan2}[][HSK 4]
    \definition[片,张]{s.}{CD; disco compacto; um disco circular feito de plástico rígido composto que usa um laser para registrar e ler informações}
  \end{phonetics}
\end{entry}

\begin{entry}{光槃}{6,14}{⼉、⽊}
  \begin{phonetics}{光槃}{guang1pan2}
    \variantof{光盘}
  \end{phonetics}
\end{entry}

\begin{entry}{免费}{7,9}{⼉、⾙}
  \begin{phonetics}{免费}{mian3fei4}[][HSK 4]
    \definition{s.}{gratuito; sem custo}
    \definition{v.+compl.}{isentar de taxas; tonar grátis}
  \end{phonetics}
\end{entry}

\begin{entry}{免得}{7,11}{⼉、⼻}
  \begin{phonetics}{免得}{mian3de5}
    \definition{conj.}{de modo a não | para evitar | para que não}
  \end{phonetics}
\end{entry}

\begin{entry}{免税}{7,12}{⼉、⽲}
  \begin{phonetics}{免税}{mian3shui4}
    \definition{adj.}{isento de impostos (tributação)}
    \definition{s.}{livre de impostos | isenção de impostos}
    \definition{v.+compl.}{isentar impostos}
  \end{phonetics}
\end{entry}

\begin{entry}{兔子}{8,3}{⼉、⼦}
  \begin{phonetics}{兔子}{tu4zi5}
    \definition[只]{s.}{coelho | lebre}
  \end{phonetics}
\end{entry}

%%%%% EOF %%%%%

