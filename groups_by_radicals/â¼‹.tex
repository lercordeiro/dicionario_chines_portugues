%%%
%%% Radical "⼋"
%%%

\section*{Radical 12: ``⼋'' (丷)}\addcontentsline{toc}{section}{Radical 12: ⼋、丷}

\begin{Entry}{八}{2}{⼋}[Kangxi 12]
  \begin{Phonetics}{八}{ba1}[][HSK 1]
    \definition{num.}{oito; 8}
  \end{Phonetics}
\end{Entry}

\begin{Entry}{八八六}{2,2,4}{⼋、⼋、⼋}
  \begin{Phonetics}{八八六}{ba1 ba1 liu4}
    \definition{expr.}{\emph{Bye bye!}, em salas de bate-papo e mensagens de texto}
  \end{Phonetics}
\end{Entry}

\begin{Entry}{公}{4}{⼋}
  \begin{Phonetics}{公}{gong1}[][HSK 6]
    \definition*{s.}{Sobrenome Gong}
    \definition{adj.}{público; estatal; coletivo (oposto a 私) | comum; geral | do mundo; internacional; universal; métrico | imparcial; justo; equitativo |  (de um animal) masculino (oposto a 母)}
    \definition{s.}{assuntos públicos; negócios oficiais (ou deveres) (oposto a 私) | autoridade; coletivo | duque | títulos respeitosos para homens idosos; uma saudação respeitosa | marido}
    \definition{v.}{tornar público; divulgar; abrir a todos; exibir}
  \seealsoref{母}{mu3}
  \seealsoref{私}{si1}
  \end{Phonetics}
\end{Entry}

\begin{Entry}{公元}{4,4}{⼋、⼉}
  \begin{Phonetics}{公元}{gong1yuan2}[][HSK 4]
    \definition{s.}{D.C. (Depois de~Cristo); a era cristã; um método internacionalmente aceito de registro de datas, o ano lendário do nascimento de Jesus é 1 d.C., também conhecido como o primeiro ano da Era Comum, e é denotado por D.C.}
  \seealsoref{前}{qian2}
  \end{Phonetics}
\end{Entry}

\begin{Entry}{公开}{4,4}{⼋、⼶}
  \begin{Phonetics}{公开}{gong1kai1}[][HSK 3]
    \definition{adj.}{aberto; público; não oculto; exposto ao público}
    \definition{v.}{tornar público}
  \end{Phonetics}
\end{Entry}

\begin{Entry}{公斤}{4,4}{⼋、⽄}
  \begin{Phonetics}{公斤}{gong1jin1}[][HSK 2]
    \definition{clas.}{quilograma (kg)}
  \end{Phonetics}
\end{Entry}

\begin{Entry}{公认}{4,4}{⼋、⾔}
  \begin{Phonetics}{公认}{gong1ren4}[][HSK 5]
    \definition{v.}{(geralmente) reconhecer; (universalmente) aceitar}
  \end{Phonetics}
\end{Entry}

\begin{Entry}{公车}{4,4}{⼋、⾞}
  \begin{Phonetics}{公车}{gong1che1}
    \definition{s.}{ônibus, abreviação de公共汽车 | carro pertencente a uma organização e usado por seus membros (carro do governo, carro de polícia, carro da empresa etc.), abreviação de 公务用车}
  \seealsoref{公共}{gong1 gong4}
  \seealsoref{公共汽车}{gong1gong4qi4che1}
  \seealsoref{公务用车}{gong1wu4yong4che1}
  \end{Phonetics}
\end{Entry}

\begin{Entry}{公主}{4,5}{⼋、⼂}
  \begin{Phonetics}{公主}{gong1zhu3}[][HSK 6]
    \definition[个,位,名,些]{s.}{princesa; a filha do monarca}
  \end{Phonetics}
\end{Entry}

\begin{Entry}{公务用车}{4,5,5,4}{⼋、⼒、⽤、⾞}
  \begin{Phonetics}{公务用车}{gong1wu4yong4che1}
    \definition{s.}{veículos oficiais}
  \end{Phonetics}
\end{Entry}

\begin{Entry}{公务员}{4,5,7}{⼋、⼒、⼝}
  \begin{Phonetics}{公务员}{gong1 wu4 yuan2}[][HSK 3]
    \definition[个,位,名,些]{s.}{funcionário público; funcionário de órgãos governamentais}
  \end{Phonetics}
\end{Entry}

\begin{Entry}{公司}{4,5}{⼋、⼝}
  \begin{Phonetics}{公司}{gong1si1}[][HSK 2]
    \definition[个,家]{s.}{empresa; companhia; corporação; uma organização industrial e comercial que opera na produção de produtos, circulação de mercadorias ou certos empreendimentos de construção, etc.}
  \end{Phonetics}
\end{Entry}

\begin{Entry}{公司治理}{4,5,8,11}{⼋、⼝、⽔、⽟}
  \begin{Phonetics}{公司治理}{gong1si1zhi4li3}
    \definition{s.}{governança corporativa}
  \end{Phonetics}
\end{Entry}

\begin{Entry}{公布}{4,5}{⼋、⼱}
  \begin{Phonetics}{公布}{gong1bu4}[][HSK 3]
    \definition{v.}{(leis, decretos, comunicados e avisos de órgãos governamentais) promulgar; anunciar; publicar; tornar público; divulgar publicamente}
  \end{Phonetics}
\end{Entry}

\begin{Entry}{公平}{4,5}{⼋、⼲}
  \begin{Phonetics}{公平}{gong1ping2}[][HSK 2]
    \definition{adj.}{justo; imparcial; equitativo; equidade}
  \end{Phonetics}
\end{Entry}

\begin{Entry}{公正}{4,5}{⼋、⽌}
  \begin{Phonetics}{公正}{gong1zheng4}[][HSK 5]
    \definition{adj.}{justo; equitativo; imparcial; de mente justa; equidade e integridade sem favoritismo}
  \end{Phonetics}
\end{Entry}

\begin{Entry}{公民}{4,5}{⼋、⽒}
  \begin{Phonetics}{公民}{gong1min2}[][HSK 3]
    \definition[个,位]{s.}{cidadão; civil; pessoa que possui a nacionalidade de um país, goza dos direitos e cumpre as obrigações previstos na Constituição e nas demais leis desse país}
  \end{Phonetics}
\end{Entry}

\begin{Entry}{公用电话}{4,5,5,8}{⼋、⽤、⽥、⾔}
  \begin{Phonetics}{公用电话}{gong1yong4dian4hua4}
    \definition[部]{s.}{telefone público}
  \end{Phonetics}
\end{Entry}

\begin{Entry}{公交车}{4,6,4}{⼋、⼇、⾞}
  \begin{Phonetics}{公交车}{gong1 jiao1 che1}[][HSK 2]
    \definition[辆]{s.}{ônibus urbano; veículo de transporte público}
  \end{Phonetics}
\end{Entry}

\begin{Entry}{公众}{4,6}{⼋、⼈}
  \begin{Phonetics}{公众}{gong1 zhong4}[][HSK 6]
    \definition[对]{s.}{o público; as massas; refere-se à maioria das pessoas na sociedade}
  \end{Phonetics}
\end{Entry}

\begin{Entry}{公共}{4,6}{⼋、⼋}
  \begin{Phonetics}{公共}{gong1 gong4}[][HSK 3]
    \definition{adj.}{público; comum; comunal; comunitário; pertencente à sociedade}
    \definition[辆]{s.}{ônibus}
  \seealsoref{公车}{gong1che1}
  \seealsoref{公共汽车}{gong1gong4qi4che1}
  \end{Phonetics}
\end{Entry}

\begin{Entry}{公共汽车}{4,6,7,4}{⼋、⼋、⽔、⾞}
  \begin{Phonetics}{公共汽车}{gong1gong4qi4che1}[][HSK 2]
    \definition[辆,个]{s.}{ônibus}
  \seealsoref{公共}{gong1 gong4}
  \seealsoref{公车}{gong1che1}
  \end{Phonetics}
\end{Entry}

\begin{Entry}{公安}{4,6}{⼋、⼧}
  \begin{Phonetics}{公安}{gong1 an1}[][HSK 6]
    \definition[名,位]{s.}{segurança pública; a segurança e estabilidade dos direitos dos cidadãos, da propriedade da segurança pública e da ordem social | agente de segurança pública; pessoal que mantém a segurança pública}
  \end{Phonetics}
\end{Entry}

\begin{Entry}{公式}{4,6}{⼋、⼷}
  \begin{Phonetics}{公式}{gong1shi4}[][HSK 5]
    \definition[个,些,种]{s.}{fórmula; expressão}
  \end{Phonetics}
\end{Entry}

\begin{Entry}{公克}{4,7}{⼋、⼗}
  \begin{Phonetics}{公克}{gong1ke4}
    \definition{s.}{grama (medida de peso)}
  \end{Phonetics}
\end{Entry}

\begin{Entry}{公告}{4,7}{⼋、⼝}
  \begin{Phonetics}{公告}{gong1gao4}[][HSK 5]
    \definition[张,份,项]{s.}{anúncio; notificação de assuntos importantes ao público em geral pelo governo ou por um órgão importante}
    \definition{v.}{anunciar; o governo ou órgão governamental informa publicamente às pessoas algo importante}
  \end{Phonetics}
\end{Entry}

\begin{Entry}{公园}{4,7}{⼋、⼞}
  \begin{Phonetics}{公园}{gong1yuan2}[][HSK 2]
    \definition[个,座]{s.}{parque; jardim público; os jardins abertos ao público para passeios e descanso geralmente ficam nas cidades, têm muitas flores, árvores e, em alguns casos, lagos}
  \end{Phonetics}
\end{Entry}

\begin{Entry}{公里}{4,7}{⼋、⾥}
  \begin{Phonetics}{公里}{gong1li3}[][HSK 2]
    \definition{s.}{quilômetro (km)}
  \end{Phonetics}
\end{Entry}

\begin{Entry}{公鸡}{4,7}{⼋、⿃}
  \begin{Phonetics}{公鸡}{gong1 ji1}[][HSK 6]
    \definition{s.}{galo; frango macho}
  \end{Phonetics}
\end{Entry}

\begin{Entry}{公寓}{4,12}{⼋、⼧}
  \begin{Phonetics}{公寓}{gong1yu4}
    \definition[套]{s.}{prédio de apartamentos | pensão}
  \end{Phonetics}
\end{Entry}

\begin{Entry}{公路}{4,13}{⼋、⾜}
  \begin{Phonetics}{公路}{gong1 lu4}[][HSK 2]
    \definition[条,段]{s.}{rodovia; via de acesso; via de tráfego; estrada; estrada principal;}
  \end{Phonetics}
\end{Entry}

\begin{Entry}{六}{4}{⼋}
  \begin{Phonetics}{六}{liu4}[][HSK 1]
    \definition*{s.}{Sobrenome Liu}
    \definition{num.}{seis; 6}
    \definition{s.}{símbolo musical utilizado na partitura da música tradicional chinesa, representando o primeiro grau da escala musical, equivalente ao ``5'' da notação musical simplificada}
  \end{Phonetics}
\end{Entry}

\begin{Entry}{兰}{5}{⼋}
  \begin{Phonetics}{兰}{lan2}
    \definition*{s.}{Sobrenome Lan}
    \definition{s.}{orquídea | lírio magnólia}
  \end{Phonetics}
\end{Entry}

\begin{Entry}{兰州}{5,6}{⼋、⼮}
  \begin{Phonetics}{兰州}{lan2zhou1}
    \definition*{s.}{Lanzhou. capital da província de Gansu, 甘肃}
  \seealsoref{甘肃}{gan1su4}
  \end{Phonetics}
\end{Entry}

\begin{Entry}{兰花}{5,7}{⼋、⾋}
  \begin{Phonetics}{兰花}{lan2hua1}
    \definition{s.}{orquídea}
  \end{Phonetics}
\end{Entry}

\begin{Entry}{共}{6}{⼋}
  \begin{Phonetics}{共}{gong4}[][HSK 4]
    \definition*{s.}{Partido Comunista, abreviação de 共产党 | Sobrenome Gong}
    \definition{adj.}{conjunto; mútuo; geral; comum; o mesmo para todos}
    \definition{adv.}{juntos; juntamente; conjuntamente | em sua totalidade; em todos}
    \definition{v.}{compartilhar com; empreender ou realizar em conjunto}
  \seealsoref{共产党}{gong4chan3dang3}
  \end{Phonetics}
\end{Entry}

\begin{Entry}{共计}{6,4}{⼋、⾔}
  \begin{Phonetics}{共计}{gong4ji4}[][HSK 5]
    \definition{s.}{total; total geral; agregado; montante}
    \definition{v.}{contar até; somar até; totalizar}
  \end{Phonetics}
\end{Entry}

\begin{Entry}{共产}{6,6}{⼋、⼇}
  \begin{Phonetics}{共产}{gong4chan3}
    \definition{adj.}{comunista}
    \definition{s.}{comunismo}
  \end{Phonetics}
\end{Entry}

\begin{Entry}{共产党}{6,6,10}{⼋、⼇、⼉}
  \begin{Phonetics}{共产党}{gong4chan3dang3}
    \definition*{s.}{Partido Comunista}
  \end{Phonetics}
\end{Entry}

\begin{Entry}{共同}{6,6}{⼋、⼝}
  \begin{Phonetics}{共同}{gong4tong2}[][HSK 3]
    \definition{adj.}{comum; compartilhado; colaborativo; todos têm}
    \definition{adv.}{juntos; conjuntamente; todos juntos (fazemos)}
  \end{Phonetics}
\end{Entry}

\begin{Entry}{共同体}{6,6,7}{⼋、⼝、⼈}
  \begin{Phonetics}{共同体}{gong4tong2ti3}
    \definition{s.}{comunidade}
  \end{Phonetics}
\end{Entry}

\begin{Entry}{共有}{6,6}{⼋、⽉}
  \begin{Phonetics}{共有}{gong4 you3}[][HSK 3]
    \definition{v.}{compartilhar; possuir (por todos); possuir ou desfrutar em conjunto}
  \end{Phonetics}
\end{Entry}

\begin{Entry}{共享}{6,8}{⼋、⼇}
  \begin{Phonetics}{共享}{gong4 xiang3}[][HSK 5]
    \definition{v.}{compartilhar; desfrutar juntos; aproveitar as coisas boas juntos}
  \end{Phonetics}
\end{Entry}

\begin{Entry}{兲}{6}{⼋}
  \begin{Phonetics}{兲}{tian1}
    \variantof{天}
  \end{Phonetics}
\end{Entry}

\begin{Entry}{关}{6}{⼋}
  \begin{Phonetics}{关}{guan1}[][HSK 1,4]
    \definition*{s.}{Sobrenome Guan}
    \definition{s.}{passagem; ponto de controle | alfândega; escritórios de cobrança de impostos para exportação e importação de mercadorias | ponto de inflexão ou barreira; ponto de virada ou dificuldade | momento crítico; mecanismo}
    \definition{v.}{fechar; encerrar; amarrar algo | fechar; trancar | encerrar; sair do mercado; falir | conceder ou sacar o pagamento de um salário | desligar | envolver; preocupar-se; conectar-se}
  \end{Phonetics}
\end{Entry}

\begin{Entry}{关上}{6,3}{⼋、⼀}
  \begin{Phonetics}{关上}{guan1 shang4}[][HSK 1]
    \definition{v.}{fechar (uma porta); fechar um objeto | desligar (luz, equipamento elétrico etc.); parar ou encerrar (uma atividade, situação, etc.)}
  \end{Phonetics}
\end{Entry}

\begin{Entry}{关于}{6,3}{⼋、⼆}
  \begin{Phonetics}{关于}{guan1yu2}[][HSK 4]
    \definition{prep.}{sobre; relativo a; pertencente a; uma questão de; com relação a}
  \end{Phonetics}
\end{Entry}

\begin{Entry}{关心}{6,4}{⼋、⼼}
  \begin{Phonetics}{关心}{guan1xin1}[][HSK 2]
    \definition{v.}{cuidar; preocupar-se com; manifestar interesse por; demonstrar solicitude por; (colocar uma pessoa ou coisa) sempre no coração; valorizar e cuidar}
  \end{Phonetics}
\end{Entry}

\begin{Entry}{关机}{6,6}{⼋、⽊}
  \begin{Phonetics}{关机}{guan1 ji1}[][HSK 2]
    \definition{v.}{encerrar; terminar; refere-se especificamente à conclusão das filmagens de um filme ou série de TV | desligar; desligar a fonte de alimentação; parar o funcionamento da máquina}
  \end{Phonetics}
\end{Entry}

\begin{Entry}{关闭}{6,6}{⼋、⾨}
  \begin{Phonetics}{关闭}{guan1bi4}[][HSK 4]
    \definition{v.}{fechar | (empresa) falir}
  \end{Phonetics}
\end{Entry}

\begin{Entry}{关怀}{6,7}{⼋、⼼}
  \begin{Phonetics}{关怀}{guan1huai2}[][HSK 5]
    \definition{v.}{mostrar cuidado amoroso por; mostrar solicitude por; cuidar, amar, apoiar ou ajudar os fracos ou grupos em dificuldade | geralmente usado para superiores para subordinados, anciãos para juniores ou organizações para indivíduos}
  \end{Phonetics}
\end{Entry}

\begin{Entry}{关系}{6,7}{⼋、⽷}
  \begin{Phonetics}{关系}{guan1xi5}[][HSK 3]
    \definition[个,种]{s.}{relações; conexões; relacionamento; a interligação entre pessoas ou coisas | consequência; impacto; significado a influência ou importância de algo; algo digno de nota (geralmente usado com 没有, 有). | causa; razão (geralmente usado com 由于 ou 因为); refere-se genericamente a causas, condições, etc. | credenciais que mostram filiação a uma organização; documento que comprova a existência de algum tipo de relação organizacional}
    \definition{v.}{preocupar; afetar; ter influência sobre; ter a ver com}
  \seealsoref{没有}{mei2 you3}
  \seealsoref{因为}{yin1wei4}
  \seealsoref{由于}{you2yu2}
  \seealsoref{有}{you3}
  \end{Phonetics}
\end{Entry}

\begin{Entry}{关注}{6,8}{⼋、⽔}
  \begin{Phonetics}{关注}{guan1 zhu4}[][HSK 3]
    \definition{v.}{prestar atenção em; seguir algo de perto; seguir (nas redes sociais)}
  \end{Phonetics}
\end{Entry}

\begin{Entry}{关爱}{6,10}{⼋、⽖}
  \begin{Phonetics}{关爱}{guan1 ai4}[][HSK 6]
    \definition{v.}{cuidar; cuidar e amar}
  \end{Phonetics}
\end{Entry}

\begin{Entry}{关联}{6,12}{⼋、⽿}
  \begin{Phonetics}{关联}{guan1 lian2}[][HSK 6]
    \definition{s.}{conexão; inter-relação; a conexão entre as coisas}
    \definition{v.}{estar relacionado; estar conectado; as coisas estão envolvidas e influenciam umas às outras}
  \end{Phonetics}
\end{Entry}

\begin{Entry}{关键}{6,13}{⼋、⾦}
  \begin{Phonetics}{关键}{guan1jian4}[][HSK 5]
    \definition{adj.}{crucial; decisivo; importante; que pode determinar o curso e o resultado dos eventos}
    \definition[个,点,些]{s.}{chave; ponto crucial; aspectos ou condições mais importantes que determinam o desenvolvimento e o resultado de algo}
  \end{Phonetics}
\end{Entry}

\begin{Entry}{兴}{6}{⼋}
  \begin{Phonetics}{兴}{xing1}
    \definition*{s.}{Sobrenome Xing}
    \definition{adj.}{próspero; florescente}
    \definition{adv.}{Dialeto: talvez}
    \definition{v.}{ascender; prosperar; prevalecer; tornar-se popular | promover; encorajar; fazer prevalecer | começar; iniciar; lançar; mobilizar | erguer-se; levantar-se | (usualmente no negativo) permitir; deixar}
  \end{Phonetics}
  \begin{Phonetics}{兴}{xing4}
    \definition{s.}{sentimento ou desejo de fazer algo | interesse em algo | excitação}
  \end{Phonetics}
\end{Entry}

\begin{Entry}{兴奋}{6,8}{⼋、⼤}
  \begin{Phonetics}{兴奋}{xing1fen4}[][HSK 4]
    \definition{adj.}{animado; excitante; empolgante;}
    \definition{s.}{excitação; empolgação}
    \definition{v.}{excitar; intoxicar}
  \end{Phonetics}
\end{Entry}

\begin{Entry}{兴旺}{6,8}{⼋、⽇}
  \begin{Phonetics}{兴旺}{xing1wang4}[][HSK 6]
    \definition{adj.}{próspero; propício; favorável; auspicioso}
  \end{Phonetics}
\end{Entry}

\begin{Entry}{兴趣}{6,15}{⼋、⾛}
  \begin{Phonetics}{兴趣}{xing4 qu4}[][HSK 4]
    \definition[个,种,点,股,份]{s.}{interesse (desejo de conhecer sobre alguma coisa ou coisa no qual está interessado) | \emph{hobby}}
  \end{Phonetics}
\end{Entry}

\begin{Entry}{兵}{7}{⼋}
  \begin{Phonetics}{兵}{bing1}[][HSK 4]
    \definition[个,种]{s.}{armas; armamentos | soldado; pessoal militar | exército; tropas | soldado raso; membro mais jovem do exército | assuntos militares (estratégia) | peão, uma das peças do xadrez chinês}
  \end{Phonetics}
\end{Entry}

\begin{Entry}{兵器}{7,16}{⼋、⼝}
  \begin{Phonetics}{兵器}{bing1qi4}
    \definition{s.}{armas | armamento}
  \end{Phonetics}
\end{Entry}

\begin{Entry}{其}{8}{⼋}
  \begin{Phonetics}{其}{qi2}[][HSK 5]
    \definition*{s.}{Sobrenome Qi}
    \definition{adv.}{fazer uma suposição ou uma réplica | expressar comando, ordem}
    \definition{pron.}{dele (dela, deles, delas) | ele, ela, isso, eles; elas | isso; tal | isso (referindo-se a nenhuma pessoa ou coisa específica)}
    \definition{suf.}{sufixo de palavra, anexado ao advérbio}
  \end{Phonetics}
\end{Entry}

\begin{Entry}{其中}{8,4}{⼋、⼁}
  \begin{Phonetics}{其中}{qi2zhong1}[][HSK 2]
    \definition{pron.}{dentro; entre (os quais, eles, etc.); em (o qual, ele, etc.); nas pessoas ou coisas mencionadas anteriormente}
  \end{Phonetics}
\end{Entry}

\begin{Entry}{其他}{8,5}{⼋、⼈}
  \begin{Phonetics}{其他}{qi2ta1}[][HSK 2]
    \definition{pron.}{outra pessoa/outra coisa | outras coisas; outras pessoas; em substituição de outras pessoas ou coisas}
  \end{Phonetics}
\end{Entry}

\begin{Entry}{其次}{8,6}{⼋、⽋}
  \begin{Phonetics}{其次}{qi2ci4}[][HSK 3]
    \definition{adj.}{secundário}
    \definition{conj.}{próximo; então; em segundo lugar; mais tarde na ordem}
  \end{Phonetics}
\end{Entry}

\begin{Entry}{其余}{8,7}{⼋、⼈}
  \begin{Phonetics}{其余}{qi2yu2}[][HSK 4]
    \definition{pron.}{o resto; os outros; o restante}
  \end{Phonetics}
\end{Entry}

\begin{Entry}{其实}{8,8}{⼋、⼧}
  \begin{Phonetics}{其实}{qi2shi2}[][HSK 3]
    \definition{adv.}{na verdade; na realidade; a primeira parte é a situação aparente, e 其实 é usado para introduzir a situação real}
  \end{Phonetics}
\end{Entry}

\begin{Entry}{具}{8}{⼋}
  \begin{Phonetics}{具}{ju4}
    \definition*{s.}{Sobrenome Ju}
    \definition{clas.}{(literário) usado para caixões, cadáveres e certos objetos}
    \definition{s.}{utensílio; ferramenta; implemento | capacidade; habilidade}
    \definition{v.}{possuir; ter | fornecer; prover | declarar; enumerar}
  \end{Phonetics}
\end{Entry}

\begin{Entry}{具有}{8,6}{⼋、⽉}
  \begin{Phonetics}{具有}{ju4 you3}[][HSK 3]
    \definition{v.}{ter; possuir; ser provido de}
  \end{Phonetics}
\end{Entry}

\begin{Entry}{具体}{8,7}{⼋、⼈}
  \begin{Phonetics}{具体}{ju4ti3}[][HSK 3]
    \definition{adj.}{específico; particular | concreto; específico; mais detalhado; muito detalhado; muito claro | concreto; real; não é abstrato, tem uma forma definida; pode ser visto ou sentido}
    \definition{v.}{incorporar; objetivar; combinar teorias, princípios, padrões, etc. com pessoas ou coisas específicas}
  \end{Phonetics}
\end{Entry}

\begin{Entry}{具备}{8,8}{⼋、⼡}
  \begin{Phonetics}{具备}{ju4bei4}[][HSK 4]
    \definition{v.}{ter; possuir; ser provido de}
  \end{Phonetics}
\end{Entry}

\begin{Entry}{典}{8}{⼋}
  \begin{Phonetics}{典}{dian3}
    \definition{s.}{lei; cânone; padrão; sistema; regulamentos | trabalho padrão de bolsa de estudos; livros que podem servir como padrões ou especificações | alusão; citação literária | cerimônia; uma grande cerimônia (nos tempos antigos, a etiqueta era um dos sistemas importantes do estado) | modelo; normas; regras}
    \definition{v.}{estar no comando de | hipotecar; usar imóveis ou casas como garantia ao pedir dinheiro emprestado}
  \end{Phonetics}
\end{Entry}

\begin{Entry}{典礼}{8,5}{⼋、⽰}
  \begin{Phonetics}{典礼}{dian3li3}[][HSK 5]
    \definition[个,次,场]{s.}{cerimônia; celebração; comemoração}
  \end{Phonetics}
\end{Entry}

\begin{Entry}{典型}{8,9}{⼋、⼟}
  \begin{Phonetics}{典型}{dian3xing2}[][HSK 4]
    \definition{adj.}{típico; representativo}
    \definition[个,种]{s.}{modelo; caso típico; indivíduo ou evento representativo | personagens típicos; personalidades modelo (em obras literárias); personagens na literatura e na arte que refletem a natureza de uma determinada sociedade e têm uma personalidade distinta}
  \end{Phonetics}
\end{Entry}

\begin{Entry}{养}{9}{⼋}
  \begin{Phonetics}{养}{yang3}[][HSK 2]
    \definition*{s.}{Sobrenome Yang}
    \definition{adj.}{adotivo; órfão; adotado; não biológico}
    \definition{s.}{qualidade; (caráter moral, desempenho acadêmico, etc.) boas qualidades}
    \definition{v.}{apoiar; prover; fornecer dinheiro e materiais necessários para viver | aumentar; manter; crescer; alimentar os animais e cuidar de suas vidas para que possam crescer | dar à luz | formar; adquirir; cultivar | descansar; curar; convalescer; recuperar a saúde | manter; manter em bom estado | deixar (o cabelo) crescer | ajudar; apoiar | cultivar (plantações ou flores)}
  \end{Phonetics}
\end{Entry}

\begin{Entry}{养分}{9,4}{⼋、⼑}
  \begin{Phonetics}{养分}{yang3fen4}
    \definition{s.}{nutriente}
  \end{Phonetics}
\end{Entry}

\begin{Entry}{养成}{9,6}{⼋、⼽}
  \begin{Phonetics}{养成}{yang3cheng2}[][HSK 4]
    \definition{v.}{cultivar; desenvolver; cultivar para formar; nutrir para crescer}
  \end{Phonetics}
\end{Entry}

\begin{Entry}{养老}{9,6}{⼋、⽼}
  \begin{Phonetics}{养老}{yang3 lao3}[][HSK 6]
    \definition{v.}{prover assistência aos idosos (geralmente os pais) | viver a vida na aposentadoria; refere-se ao idoso que descansa em casa}
  \end{Phonetics}
\end{Entry}

\begin{Entry}{养料}{9,10}{⼋、⽃}
  \begin{Phonetics}{养料}{yang3liao4}
    \definition{s.}{nutrição}
  \end{Phonetics}
\end{Entry}

\begin{Entry}{兼}{10}{⼋}
  \begin{Phonetics}{兼}{jian1}
    \definition{conj.}{e (ocupando dois ou mais cargos (oficiais) ao mesmo tempo)}
  \end{Phonetics}
\end{Entry}

\begin{Entry}{兽}{11}{⼋}
  \begin{Phonetics}{兽}{shou4}
    \definition{adj.}{bestial; brutal}
    \definition{s.}{besta; animal}
  \end{Phonetics}
\end{Entry}

\begin{Entry}{兽力车}{11,2,4}{⼋、⼒、⾞}
  \begin{Phonetics}{兽力车}{shou4 li4 che1}
    \definition{s.}{veículo puxado por animais  (oposto a 人力车) | carruagem; carroça}
  \seealsoref{人力车}{ren2 li4 che1}
  \end{Phonetics}
\end{Entry}

\begin{Entry}{兽行}{11,6}{⼋、⾏}
  \begin{Phonetics}{兽行}{shou4xing2}
    \definition{s.}{ato brutal; brutalidade | bestialidade}
  \end{Phonetics}
\end{Entry}

%%%%% EOF %%%%%

