%%%
%%% Radical "⼋"
%%%

\section*{Radical 12: ``⼋'' (丷)}\addcontentsline{toc}{section}{Radical 12: ⼋、丷}

\begin{entry}{八}{2}{⼋}[Kangxi 12]
  \begin{phonetics}{八}{ba1}[][HSK 1]
    \definition{num.}{oito; 8}
  \end{phonetics}
\end{entry}

\begin{entry}{八八六}{2,2,4}{⼋、⼋、⼋}
  \begin{phonetics}{八八六}{ba1 ba1 liu4}
    \definition{expr.}{\emph{Bye bye!} (em salas de bate-papo e mensagens de texto)}
  \end{phonetics}
\end{entry}

\begin{entry}{公元}{4,4}{⼋、⼉}
  \begin{phonetics}{公元}{gong1yuan2}[][HSK 4]
    \definition{s.}{D.C. (Depois de~Cristo); a era cristã; um método internacionalmente aceito de registro de datas, o ano lendário do nascimento de Jesus é 1 d.C., também conhecido como o primeiro ano da Era Comum, e é denotado por D.C.}
  \seealsoref{前}{qian2}
  \end{phonetics}
\end{entry}

\begin{entry}{公开}{4,4}{⼋、⼶}
  \begin{phonetics}{公开}{gong1kai1}[][HSK 3]
    \definition{adj.}{aberto; público}
    \definition{v.}{tornar público}
  \end{phonetics}
\end{entry}

\begin{entry}{公斤}{4,4}{⼋、⽄}
  \begin{phonetics}{公斤}{gong1jin1}[][HSK 2]
    \definition{clas.}{quilograma (kg)}
  \end{phonetics}
\end{entry}

\begin{entry}{公认}{4,4}{⼋、⾔}
  \begin{phonetics}{公认}{gong1ren4}[][HSK 5]
    \definition{v.}{(geralmente) reconhecer; (universalmente) aceitar}
  \end{phonetics}
\end{entry}

\begin{entry}{公车}{4,4}{⼋、⾞}
  \begin{phonetics}{公车}{gong1che1}
    \definition{s.}{abreviação de~公共汽车, ônibus}
  \seealsoref{公共}{gong1 gong4}
  \seealsoref{公共汽车}{gong1gong4qi4che1}
  \end{phonetics}
\end{entry}

\begin{entry}{公务员}{4,5,7}{⼋、⼒、⼝}
  \begin{phonetics}{公务员}{gong1 wu4 yuan2}[][HSK 3]
    \definition[个,名]{s.}{funcionário público}
  \end{phonetics}
\end{entry}

\begin{entry}{公司}{4,5}{⼋、⼝}
  \begin{phonetics}{公司}{gong1si1}[][HSK 2]
    \definition[个,家]{s.}{empresa; companhia; corporação; uma organização industrial e comercial que opera na produção de produtos, circulação de mercadorias ou certos empreendimentos de construção, etc.}
  \end{phonetics}
\end{entry}

\begin{entry}{公司治理}{4,5,8,11}{⼋、⼝、⽔、⽟}
  \begin{phonetics}{公司治理}{gong1si1zhi4li3}
    \definition{s.}{governança corporativa}
  \end{phonetics}
\end{entry}

\begin{entry}{公布}{4,5}{⼋、⼱}
  \begin{phonetics}{公布}{gong1bu4}[][HSK 3]
    \definition{v.}{promulgar; anunciar; publicar; tornar público}
  \end{phonetics}
\end{entry}

\begin{entry}{公平}{4,5}{⼋、⼲}
  \begin{phonetics}{公平}{gong1ping2}[][HSK 2]
    \definition{adj.}{justo; imparcial; equitativo; equidade}
  \end{phonetics}
\end{entry}

\begin{entry}{公正}{4,5}{⼋、⽌}
  \begin{phonetics}{公正}{gong1zheng4}[][HSK 5]
    \definition{adj.}{justo; equitativo; imparcial; de mente justa; equidade e integridade sem favoritismo}
  \end{phonetics}
\end{entry}

\begin{entry}{公民}{4,5}{⼋、⽒}
  \begin{phonetics}{公民}{gong1min2}[][HSK 3]
    \definition{s.}{cidadão; civil}
  \end{phonetics}
\end{entry}

\begin{entry}{公用电话}{4,5,5,8}{⼋、⽤、⽥、⾔}
  \begin{phonetics}{公用电话}{gong1yong4dian4hua4}
    \definition[部]{s.}{telefone público}
  \end{phonetics}
\end{entry}

\begin{entry}{公交车}{4,6,4}{⼋、⼇、⾞}
  \begin{phonetics}{公交车}{gong1 jiao1 che1}[][HSK 2]
    \definition[辆]{s.}{ônibus urbano; veículo de transporte público}
  \end{phonetics}
\end{entry}

\begin{entry}{公共}{4,6}{⼋、⼋}
  \begin{phonetics}{公共}{gong1 gong4}[][HSK 3]
    \definition{adj.}{público; comum; comunal}
    \definition{s.}{ônibus}
  \seealsoref{公车}{gong1che1}
  \seealsoref{公共汽车}{gong1gong4qi4che1}
  \end{phonetics}
\end{entry}

\begin{entry}{公共汽车}{4,6,7,4}{⼋、⼋、⽔、⾞}
  \begin{phonetics}{公共汽车}{gong1gong4qi4che1}[][HSK 2]
    \definition[辆,个]{s.}{ônibus}
  \seealsoref{公共}{gong1 gong4}
  \seealsoref{公车}{gong1che1}
  \end{phonetics}
\end{entry}

\begin{entry}{公式}{4,6}{⼋、⼷}
  \begin{phonetics}{公式}{gong1shi4}[][HSK 5]
    \definition[个]{s.}{fórmula; expressão}
  \end{phonetics}
\end{entry}

\begin{entry}{公克}{4,7}{⼋、⼗}
  \begin{phonetics}{公克}{gong1ke4}
    \definition{s.}{grama (medida de peso)}
  \end{phonetics}
\end{entry}

\begin{entry}{公告}{4,7}{⼋、⼝}
  \begin{phonetics}{公告}{gong1gao4}[][HSK 5]
    \definition{s.}{anúncio; notificação de assuntos importantes ao público em geral pelo governo ou por um órgão importante}
    \definition{v.}{anunciar; o governo ou órgão governamental informa publicamente às pessoas algo importante}
  \end{phonetics}
\end{entry}

\begin{entry}{公园}{4,7}{⼋、⼞}
  \begin{phonetics}{公园}{gong1yuan2}[][HSK 2]
    \definition[个,座]{s.}{parque; jardim público; os jardins abertos ao público para passeios e descanso geralmente ficam nas cidades, têm muitas flores, árvores e, em alguns casos, lagos}
  \end{phonetics}
\end{entry}

\begin{entry}{公里}{4,7}{⼋、⾥}
  \begin{phonetics}{公里}{gong1li3}[][HSK 2]
    \definition{s.}{quilômetro (km)}
  \end{phonetics}
\end{entry}

\begin{entry}{公寓}{4,12}{⼋、⼧}
  \begin{phonetics}{公寓}{gong1yu4}
    \definition[套]{s.}{prédio de apartamentos | pensão}
  \end{phonetics}
\end{entry}

\begin{entry}{公路}{4,13}{⼋、⾜}
  \begin{phonetics}{公路}{gong1 lu4}[][HSK 2]
    \definition[条,段]{s.}{rodovia; via de acesso; via de tráfego; estrada; estrada principal;}
  \end{phonetics}
\end{entry}

\begin{entry}{六}{4}{⼋}
  \begin{phonetics}{六}{liu4}[][HSK 1]
    \definition*{s.}{sobrenome Liu}
    \definition{num.}{seis; 6}
    \definition{s.}{símbolo musical utilizado na partitura da música tradicional chinesa, representando o primeiro grau da escala musical, equivalente ao ``5'' da notação musical simplificada}
  \end{phonetics}
\end{entry}

\begin{entry}{兰花}{5,7}{⼋、⾋}
  \begin{phonetics}{兰花}{lan2hua1}
    \definition{s.}{orquídea}
  \end{phonetics}
\end{entry}

\begin{entry}{共}{6}{⼋}
  \begin{phonetics}{共}{gong4}[][HSK 4]
    \definition*{s.}{sobrenome Gong}
    \definition*{s.}{Abreviação de Partido Comunista, 共产党}
    \definition{adj.}{conjunto; mútuo; geral; comum; o mesmo para todos}
    \definition{adv.}{juntos; juntamente; conjuntamente | em sua totalidade; em todos}
    \definition{v.}{compartilhar com; empreender ou realizar em conjunto}
  \seealsoref{共产党}{gong4chan3dang3}
  \end{phonetics}
\end{entry}

\begin{entry}{共计}{6,4}{⼋、⾔}
  \begin{phonetics}{共计}{gong4ji4}[][HSK 5]
    \definition{s.}{total; total geral; agregado; montante}
    \definition{v.}{contar até; somar até; totalizar}
  \end{phonetics}
\end{entry}

\begin{entry}{共产}{6,6}{⼋、⼇}
  \begin{phonetics}{共产}{gong4chan3}
    \definition{adj.}{comunista}
    \definition{s.}{comunismo}
  \end{phonetics}
\end{entry}

\begin{entry}{共产党}{6,6,10}{⼋、⼇、⼉}
  \begin{phonetics}{共产党}{gong4chan3dang3}
    \definition*{s.}{Partido Comunista}
  \end{phonetics}
\end{entry}

\begin{entry}{共同}{6,6}{⼋、⼝}
  \begin{phonetics}{共同}{gong4tong2}[][HSK 3]
    \definition{adj.}{comum; compartilhado; colaborativo}
    \definition{adv.}{juntos; conjuntamente}
  \end{phonetics}
\end{entry}

\begin{entry}{共同体}{6,6,7}{⼋、⼝、⼈}
  \begin{phonetics}{共同体}{gong4tong2ti3}
    \definition{s.}{comunidade}
  \end{phonetics}
\end{entry}

\begin{entry}{共有}{6,6}{⼋、⽉}
  \begin{phonetics}{共有}{gong4 you3}[][HSK 3]
    \definition{v.}{ter completamente; compartilhar; possuir (por todos)}
  \end{phonetics}
\end{entry}

\begin{entry}{共享}{6,8}{⼋、⼇}
  \begin{phonetics}{共享}{gong4 xiang3}[][HSK 5]
    \definition{v.}{compartilhar; desfrutar juntos; aproveitar as coisas boas juntos}
  \end{phonetics}
\end{entry}

\begin{entry}{兲}{6}{⼋}
  \begin{phonetics}{兲}{tian1}
    \variantof{天}
  \end{phonetics}
\end{entry}

\begin{entry}{关}{6}{⼋}
  \begin{phonetics}{关}{guan1}[][HSK 1,4]
    \definition*{s.}{sobrenome Guan}
    \definition{s.}{passagem; ponto de controle | alfândega; escritórios de cobrança de impostos para exportação e importação de mercadorias | ponto de inflexão ou barreira; ponto de virada ou dificuldade | momento crítico; mecanismo}
    \definition{v.}{fechar; encerrar; amarrar algo | fechar; trancar | encerrar; sair do mercado; falir | conceder ou sacar o pagamento de um salário | desligar | envolver; preocupar-se; conectar-se}
  \end{phonetics}
\end{entry}

\begin{entry}{关上}{6,3}{⼋、⼀}
  \begin{phonetics}{关上}{guan1 shang4}[][HSK 1]
    \definition{v.}{fechar (uma porta); fechar um objeto | desligar (luz, equipamento elétrico etc.); parar ou encerrar (uma atividade, situação, etc.)}
  \end{phonetics}
\end{entry}

\begin{entry}{关于}{6,3}{⼋、⼆}
  \begin{phonetics}{关于}{guan1yu2}[][HSK 4]
    \definition{prep.}{sobre; relativo a; pertencente a; uma questão de; com relação a}
  \end{phonetics}
\end{entry}

\begin{entry}{关心}{6,4}{⼋、⼼}
  \begin{phonetics}{关心}{guan1xin1}[][HSK 2]
    \definition{v.}{cuidar; preocupar-se com; manifestar interesse por; demonstrar solicitude por; (colocar uma pessoa ou coisa) sempre no coração; valorizar e cuidar}
  \end{phonetics}
\end{entry}

\begin{entry}{关机}{6,6}{⼋、⽊}
  \begin{phonetics}{关机}{guan1 ji1}[][HSK 2]
    \definition{v.}{encerrar; terminar; refere-se especificamente à conclusão das filmagens de um filme ou série de TV | desligar; desligar a fonte de alimentação; parar o funcionamento da máquina}
  \end{phonetics}
\end{entry}

\begin{entry}{关闭}{6,6}{⼋、⾨}
  \begin{phonetics}{关闭}{guan1bi4}[][HSK 4]
    \definition{v.}{fechar | (empresa) falir}
  \end{phonetics}
\end{entry}

\begin{entry}{关怀}{6,7}{⼋、⼼}
  \begin{phonetics}{关怀}{guan1huai2}[][HSK 5]
    \definition{v.}{mostrar cuidado amoroso por; mostrar solicitude por; cuidar, amar, apoiar ou ajudar os fracos ou grupos em dificuldade | geralmente usado para superiores para subordinados, anciãos para juniores ou organizações para indivíduos}
  \end{phonetics}
\end{entry}

\begin{entry}{关系}{6,7}{⼋、⽷}
  \begin{phonetics}{关系}{guan1xi5}[][HSK 3]
    \definition[个,种]{s.}{relações; conexões; relacionamento | consequência; impacto; significado | causa; razão (geralmente usado com 由于 ou 因为) | credenciais que mostram filiação a uma organização}
    \definition{v.}{preocupar; afetar; ter influência sobre; ter a ver com}
  \seealsoref{因为}{yin1wei4}
  \seealsoref{由于}{you2yu2}
  \end{phonetics}
\end{entry}

\begin{entry}{关注}{6,8}{⼋、⽔}
  \begin{phonetics}{关注}{guan1 zhu4}[][HSK 3]
    \definition{s.}{preocupação; interesse; atenção}
    \definition{v.}{prestar atenção em; seguir algo de perto; seguir (nas redes sociais)}
  \end{phonetics}
\end{entry}

\begin{entry}{关键}{6,13}{⼋、⾦}
  \begin{phonetics}{关键}{guan1jian4}[][HSK 5]
    \definition{adj.}{crucial; decisivo; importante; que pode determinar o curso e o resultado dos eventos}
    \definition[个]{s.}{chave; ponto crucial; aspectos ou condições mais importantes que determinam o desenvolvimento e o resultado de algo}
  \end{phonetics}
\end{entry}

\begin{entry}{兴}{6}{⼋}
  \begin{phonetics}{兴}{xing1}
    \definition*{s.}{sobrenome Xing}
    \definition{adv.}{talvez (dialeto)}
    \definition{v.}{subir | florescer | tornar-se popular | começar | encorajar | levantar-se | (frequentemente usado em negativas) permitir (dialeto)}
  \end{phonetics}
  \begin{phonetics}{兴}{xing4}
    \definition{s.}{sentimento ou desejo de fazer algo | interesse em algo | excitação}
  \end{phonetics}
\end{entry}

\begin{entry}{兴奋}{6,8}{⼋、⼤}
  \begin{phonetics}{兴奋}{xing1fen4}[][HSK 4]
    \definition{adj.}{animado; excitante; empolgante;}
    \definition{s.}{excitação; empolgação}
    \definition{v.}{excitar; intoxicar}
  \end{phonetics}
\end{entry}

\begin{entry}{兴趣}{6,15}{⼋、⾛}
  \begin{phonetics}{兴趣}{xing4 qu4}[][HSK 4]
    \definition[个]{s.}{interesse (desejo de conhecer sobre alguma coisa ou coisa no qual está interessado) | \emph{hobby}}
  \end{phonetics}
\end{entry}

\begin{entry}{兵}{7}{⼋}
  \begin{phonetics}{兵}{bing1}[][HSK 4]
    \definition[名]{s.}{armas; armamentos | soldado; pessoal militar | exército; tropas | soldado raso; membro mais jovem do exército | assuntos militares (estratégia) | peão, uma das peças do xadrez chinês}
  \end{phonetics}
\end{entry}

\begin{entry}{兵器}{7,16}{⼋、⼝}
  \begin{phonetics}{兵器}{bing1qi4}
    \definition{s.}{armas | armamento}
  \end{phonetics}
\end{entry}

\begin{entry}{其}{8}{⼋}
  \begin{phonetics}{其}{qi2}[][HSK 5]
    \definition*{s.}{sobrenome Qi}
    \definition{adv.}{fazer uma suposição ou uma réplica | expressar comando, ordem}
    \definition{pron.}{dele (dela, deles, delas) | ele, ela, isso, eles; elas | isso; tal | isso (referindo-se a nenhuma pessoa ou coisa específica)}
    \definition{suf.}{sufixo de palavra, anexado ao advérbio}
  \end{phonetics}
\end{entry}

\begin{entry}{其中}{8,4}{⼋、⼁}
  \begin{phonetics}{其中}{qi2zhong1}[][HSK 2]
    \definition{pron.}{dentro | entre (o qual, eles, etc.) | em (o qual, isso, etc.)}
  \end{phonetics}
\end{entry}

\begin{entry}{其他}{8,5}{⼋、⼈}
  \begin{phonetics}{其他}{qi2ta1}[][HSK 2]
    \definition{pron.}{todos os outro(s) | o resto}
  \end{phonetics}
\end{entry}

\begin{entry}{其次}{8,6}{⼋、⽋}
  \begin{phonetics}{其次}{qi2ci4}[][HSK 3]
    \definition{adj.}{secundário}
    \definition{conj.}{próximo; então; em segundo lugar}
  \end{phonetics}
\end{entry}

\begin{entry}{其余}{8,7}{⼋、⼈}
  \begin{phonetics}{其余}{qi2yu2}[][HSK 4]
    \definition{pron.}{o restante; os outros}
  \end{phonetics}
\end{entry}

\begin{entry}{其实}{8,8}{⼋、⼧}
  \begin{phonetics}{其实}{qi2shi2}[][HSK 3]
    \definition{adv.}{na verdade; na realidade; de fato}
  \end{phonetics}
\end{entry}

\begin{entry}{具有}{8,6}{⼋、⽉}
  \begin{phonetics}{具有}{ju4 you3}[][HSK 3]
    \definition{v.}{ter; possuir; ser provido de}
  \end{phonetics}
\end{entry}

\begin{entry}{具体}{8,7}{⼋、⼈}
  \begin{phonetics}{具体}{ju4ti3}[][HSK 3]
    \definition{adj.}{específico; particular | concreto; específico | concreto; real}
    \definition{v.}{incorporar; objetivar}
  \end{phonetics}
\end{entry}

\begin{entry}{具备}{8,8}{⼋、⼡}
  \begin{phonetics}{具备}{ju4bei4}[][HSK 4]
    \definition{v.}{ter; possuir; ser provido de}
  \end{phonetics}
\end{entry}

\begin{entry}{典礼}{8,5}{⼋、⽰}
  \begin{phonetics}{典礼}{dian3li3}[][HSK 5]
    \definition[个,次,场]{s.}{cerimônia; celebração; comemoração}
  \end{phonetics}
\end{entry}

\begin{entry}{典型}{8,9}{⼋、⼟}
  \begin{phonetics}{典型}{dian3xing2}[][HSK 4]
    \definition{adj.}{típico; representativo}
    \definition[个]{s.}{modelo; caso típico; indivíduo ou evento representativo | personagens típicos; personalidades modelo (em obras literárias); personagens na literatura e na arte que refletem a natureza de uma determinada sociedade e têm uma personalidade distinta}
  \end{phonetics}
\end{entry}

\begin{entry}{养}{9}{⼋}
  \begin{phonetics}{养}{yang3}[][HSK 2]
    \definition{v.}{criar (animais ou filhos), plantar (flores), etc. | dar a luz}
  \end{phonetics}
\end{entry}

\begin{entry}{养分}{9,4}{⼋、⼑}
  \begin{phonetics}{养分}{yang3fen4}
    \definition{s.}{nutriente}
  \end{phonetics}
\end{entry}

\begin{entry}{养成}{9,6}{⼋、⼽}
  \begin{phonetics}{养成}{yang3cheng2}[][HSK 4]
    \definition{v.}{cultivar; desenvolver; cultivar para formar; nutrir para crescer}
  \end{phonetics}
\end{entry}

\begin{entry}{养料}{9,10}{⼋、⽃}
  \begin{phonetics}{养料}{yang3liao4}
    \definition{s.}{nutrição}
  \end{phonetics}
\end{entry}

\begin{entry}{兼}{10}{⼋}
  \begin{phonetics}{兼}{jian1}
    \definition{conj.}{e (ocupando dois ou mais cargos (oficiais) ao mesmo tempo)}
  \end{phonetics}
\end{entry}

%%%%% EOF %%%%%

