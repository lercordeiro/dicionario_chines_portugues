%%%
%%% Radical "⾞"
%%%

\section*{Radical 159: ``⾞'' (车)}\addcontentsline{toc}{section}{Radical 159: ⾞、车}

\begin{entry}{车}{4}{⾞}[Kangxi 159]
  \begin{phonetics}{车}{che1}[][HSK 1]
    \definition*{s.}{sobrenome Che}
    \definition[辆]{s.}{carro | veículo | viatura}
  \end{phonetics}
  \begin{phonetics}{车}{ju1}
    \definition{s.}{(arcaico) carruagem de guerra | torre (no xadrez)}
  \end{phonetics}
\end{entry}

\begin{entry}{车上}{4,3}{⾞、⼀}
  \begin{phonetics}{车上}{che1 shang4}[][HSK 1]
    \definition{adv.}{no carro | dentro do veículo}
  \end{phonetics}
\end{entry}

\begin{entry}{车子}{4,3}{⾞、⼦}
  \begin{phonetics}{车子}{che1zi5}
    \definition{s.}{qualquer veículo (carro, bicicleta, caminhão, etc)}
  \end{phonetics}
\end{entry}

\begin{entry}{车水马龙}{4,4,3,5}{⾞、⽔、⾺、⿓}
  \begin{phonetics}{车水马龙}{che1shui3-ma3long2}
    \definition{expr.}{tráfego engarrafado | engarrafamento | (literalmente) ``fluxo interminável de cavalos e carruagens''}
  \end{phonetics}
\end{entry}

\begin{entry}{车主}{4,5}{⾞、⼂}
  \begin{phonetics}{车主}{che1 zhu3}[][HSK 5]
    \definition{s.}{proprietário do carro; uma pessoa física ou família que possui um veículo motorizado}
  \end{phonetics}
\end{entry}

\begin{entry}{车次}{4,6}{⾞、⽋}
  \begin{phonetics}{车次}{che1ci4}
    \definition{s.}{número do trem}
  \end{phonetics}
\end{entry}

\begin{entry}{车库}{4,7}{⾞、⼴}
  \begin{phonetics}{车库}{che1ku4}
    \definition{s.}{garagem}
  \end{phonetics}
\end{entry}

\begin{entry}{车站}{4,10}{⾞、⽴}
  \begin{phonetics}{车站}{che1 zhan4}[][HSK 1]
    \definition[处,个]{s.}{estação | ponto de ônibus}
  \end{phonetics}
\end{entry}

\begin{entry}{车票}{4,11}{⾞、⽰}
  \begin{phonetics}{车票}{che1 piao4}[][HSK 1]
    \definition{s.}{bilhete (de ônibus, trem, metrô)}
  \end{phonetics}
\end{entry}

\begin{entry}{车辆}{4,11}{⾞、⾞}
  \begin{phonetics}{车辆}{che1 liang4}[][HSK 2]
    \definition{s.}{veículo | carro}
  \end{phonetics}
\end{entry}

\begin{entry}{车牌}{4,12}{⾞、⽚}
  \begin{phonetics}{车牌}{che1pai2}
    \definition{s.}{matrícula | placa de carro}
  \end{phonetics}
\end{entry}

\begin{entry}{转}{8}{⾞}
  \begin{phonetics}{转}{zhuai3}
  \end{phonetics}
  \begin{phonetics}{转}{zhuan3}
    \definition{v.}{mudar; deslocar; transferir; virar; mudar de direção, posição, situação, circunstâncias, etc. | transmitir; transferir; passar adiante}
  \end{phonetics}
  \begin{phonetics}{转}{zhuan4}[][HSK 3]
    \definition{clas.}{para ações repetidas | para rotações (por minuto, etc.): RPM}
    \definition{v.}{rodar; girar; virar; dar voltas | passear; andar por aí}
  \end{phonetics}
\end{entry}

\begin{entry}{转化}{8,4}{⾞、⼔}
  \begin{phonetics}{转化}{zhuan3 hua4}[][HSK 5]
    \definition{v.}{mudar; transformar | inverter; converter}
  \end{phonetics}
\end{entry}

\begin{entry}{转让}{8,5}{⾞、⾔}
  \begin{phonetics}{转让}{zhuan3rang4}[][HSK 5]
    \definition{v.}{ceder; fazer a entrega; transferir a posse de; ceder seus bens ou direitos a outra pessoa}
  \end{phonetics}
\end{entry}

\begin{entry}{转产}{8,6}{⾞、⼇}
  \begin{phonetics}{转产}{zhuan3chan3}
    \definition{v.}{mudar a produção | mudar para novos produtos}
  \end{phonetics}
\end{entry}

\begin{entry}{转动}{8,6}{⾞、⼒}
  \begin{phonetics}{转动}{zhuan3 dong4}[][HSK 4]
    \definition{v.}{girar; rodar; dar voltas; torcer | dar a volta em algo}
  \end{phonetics}
  \begin{phonetics}{转动}{zhuan4 dong4}[][HSK 4]
    \definition{v.}{girar; correr; rolar; revolver; rotacionar; torcer}
  \end{phonetics}
\end{entry}

\begin{entry}{转向}{8,6}{⾞、⼝}
  \begin{phonetics}{转向}{zhuan3 xiang4}[][HSK 5]
    \definition{v.}{desviar; desviar-se; mudar a direção do avanço | mudar a posição política de alguém | mudar de direção; virar-se para (a outra parte)}
  \end{phonetics}
  \begin{phonetics}{转向}{zhuan4 xiang4}
    \definition{v.+compl.}{perder-se; perder o rumo; não consiguir distinguir a direção; estar perdido}
  \end{phonetics}
\end{entry}

\begin{entry}{转告}{8,7}{⾞、⼝}
  \begin{phonetics}{转告}{zhuan3gao4}[][HSK 4]
    \definition{v.}{passar adiante; comunicar; transmitir; ser instruído a dizer a outra parte o que uma pessoa diz, o que está acontecendo, etc.}
  \end{phonetics}
\end{entry}

\begin{entry}{转身}{8,7}{⾞、⾝}
  \begin{phonetics}{转身}{zhuan3 shen1}[][HSK 4]
    \definition{adv.}{em um instante; em um piscar de olhos}
    \definition{v.}{dar a volta; dar meia-volta; dar a volta por cima | virar; girar; refere-se a uma mudança de direção, localização, natureza, etc.}
  \end{phonetics}
\end{entry}

\begin{entry}{转变}{8,8}{⾞、⼜}
  \begin{phonetics}{转变}{zhuan3bian4}[][HSK 3]
    \definition{v.}{mudar; converter; transformar}
  \end{phonetics}
\end{entry}

\begin{entry}{转念}{8,8}{⾞、⼼}
  \begin{phonetics}{转念}{zhuan3nian4}
    \definition{v.}{ter dúvidas sobre algo | pensar melhor}
  \end{phonetics}
\end{entry}

\begin{entry}{转账}{8,8}{⾞、⾙}
  \begin{phonetics}{转账}{zhuan3zhang4}
    \definition{v.+compl.}{transferir entre contas | trazer à frente | incluir uma soma de dinheiro do balanço anterior no seguinte}
  \end{phonetics}
\end{entry}

\begin{entry}{转弯}{8,9}{⾞、⼸}
  \begin{phonetics}{转弯}{zhuan4 wan1}[][HSK 4]
    \definition{v.}{rodar; desviar; virar uma esquina; fazer uma curva; fazer uma curva de 180º}
  \end{phonetics}
\end{entry}

\begin{entry}{转换}{8,10}{⾞、⼿}
  \begin{phonetics}{转换}{zhuan3 huan4}[][HSK 5]
    \definition{v.}{mudar; trocar; converter; transformar; alterar}
  \end{phonetics}
\end{entry}

\begin{entry}{转递}{8,10}{⾞、⾡}
  \begin{phonetics}{转递}{zhuan3di4}
    \definition{v.}{passar | retransmitir}
  \end{phonetics}
\end{entry}

\begin{entry}{转悠}{8,11}{⾞、⼼}
  \begin{phonetics}{转悠}{zhuan4you5}
    \definition{v.}{aparecer repetidamente | rolar | passear por aí}
  \end{phonetics}
\end{entry}

\begin{entry}{转移}{8,11}{⾞、⽲}
  \begin{phonetics}{转移}{zhuan3yi2}[][HSK 4]
    \definition{v.}{deslocar; desviar; transferir; redirecionar; reposicionar; reorientar | mudar; transformar}
  \end{phonetics}
\end{entry}

\begin{entry}{转游}{8,12}{⾞、⽔}
  \begin{phonetics}{转游}{zhuan4you5}
    \variantof{转悠}
  \end{phonetics}
\end{entry}

\begin{entry}{轮}{8}{⾞}
  \begin{phonetics}{轮}{lun2}[][HSK 4]
    \definition{clas.}{para sol vermelho, lua brilhante, etc. | para rodadas | doze anos de idade (os doze ramos terrestres são usados para lembrar o gênero humano e cada doze anos de idade é um ciclo)}
    \definition{s.}{roda | anel; disco; objeto semelhante a uma roda | navio a vapor; barco a vapor}
    \definition{v.}{revezar; substituir um ao outro em sequência (para fazer algo)}
  \end{phonetics}
\end{entry}

\begin{entry}{轮子}{8,3}{⾞、⼦}
  \begin{phonetics}{轮子}{lun2 zi5}[][HSK 4]
    \definition[个]{s.}{roda; peças circulares de veículos ou máquinas com capacidade de rotação}
  \end{phonetics}
\end{entry}

\begin{entry}{轮回}{8,6}{⾞、⼞}
  \begin{phonetics}{轮回}{lun2hui2}
    \definition[个]{s.}{reencarnação (Budismo) | ciclo}
    \definition{v.}{reencarnar}
  \end{phonetics}
\end{entry}

\begin{entry}{轮船}{8,11}{⾞、⾈}
  \begin{phonetics}{轮船}{lun2chuan2}[][HSK 4]
    \definition[艘]{s.}{navio}
  \end{phonetics}
\end{entry}

\begin{entry}{轮椅}{8,12}{⾞、⽊}
  \begin{phonetics}{轮椅}{lun2 yi3}[][HSK 4]
    \definition{s.}{cadeira de rodas; dispositivo de assento especialmente projetado com rodas para pessoas com dificuldade de locomoção, que pode ser acionado por um disco de roda ou manivela operados manualmente}
  \end{phonetics}
\end{entry}

\begin{entry}{软}{8}{⾞}
  \begin{phonetics}{软}{ruan3}[][HSK 5]
    \definition*{s.}{sobrenome Ruan}
    \definition{adj.}{macio; flexível; maleável; maleável (oposto de ``硬'') | suave; brando; delicado | fraco; débil | de baixa qualidade, capacidade, etc. | facilmente movido (ou influenciado) | de maneira suave (ou gentil) | indulgente; tolerante | maleável; flexível | fácil de se emocionar ou abalar}
  \seealsoref{硬}{ying4}
  \end{phonetics}
\end{entry}

\begin{entry}{软件}{8,6}{⾞、⼈}
  \begin{phonetics}{软件}{ruan3jian4}[][HSK 5]
    \definition[款,个]{v.}{(computador) \emph{software}; programas de computador, procedimentos, regras e quaisquer arquivos, documentos e dados relacionados à operação do sistema de computador}
  \end{phonetics}
\end{entry}

\begin{entry}{轰鸣}{8,8}{⾞、⿃}
  \begin{phonetics}{轰鸣}{hong1ming2}
    \definition{s.}{bum (som de explosão) | estrondo}
  \end{phonetics}
\end{entry}

\begin{entry}{轰炸机}{8,9,6}{⾞、⽕、⽊}
  \begin{phonetics}{轰炸机}{hong1zha4ji1}
    \definition{s.}{bombardeiro (aeronave)}
  \end{phonetics}
\end{entry}

\begin{entry}{轴承}{9,8}{⾞、⼿}
  \begin{phonetics}{轴承}{zhou2cheng2}
    \definition{s.}{(mecânico) rolamento}
  \end{phonetics}
\end{entry}

\begin{entry}{轻}{9}{⾞}
  \begin{phonetics}{轻}{qing1}[][HSK 2]
    \definition{adj.}{leve | pequeno em número, grau, etc. | não importante | relaxado}
    \definition{adv.}{suavemente | levemente | precipitadamente}
    \definition{v.}{menosprezar}
  \end{phonetics}
\end{entry}

\begin{entry}{轻易}{9,8}{⾞、⽇}
  \begin{phonetics}{轻易}{qing1yi4}[][HSK 4]
    \definition{adj.}{fácil; simples}
    \definition{adv.}{facilmente; prontamente | facilmente; precipitadamente; indica que uma ação é realizada casualmente, geralmente usado em frases negativas}
  \end{phonetics}
\end{entry}

\begin{entry}{轻松}{9,8}{⾞、⽊}
  \begin{phonetics}{轻松}{qing1song1}[][HSK 4]
    \definition{adj.}{leve; relaxado; livre de fardos; não se sentir nervoso ou cansado}
    \definition{v.}{relaxar; levar as coisas menos a sério}
  \end{phonetics}
\end{entry}

\begin{entry}{较}{10}{⾞}
  \begin{phonetics}{较}{jiao4}[][HSK 3]
    \definition{adj.}{claro; óbvio; marcado}
    \definition{adv.}{comparativamente; relativamente; razoavelmente; bastante; bastante}
    \definition{prep.}{usado para comparar características e graus}
    \definition{v.}{comparar | disputar}
  \end{phonetics}
\end{entry}

\begin{entry}{辅助}{11,7}{⾞、⼒}
  \begin{phonetics}{辅助}{fu3zhu4}[][HSK 5]
    \definition{adj.}{auxiliar; suplementar; complementar;}
    \definition{v.}{auxiliar; ajudar; colocar os outros em primeiro lugar e dar-lhes alguma ajuda externa}
  \end{phonetics}
\end{entry}

\begin{entry}{辆}{11}{⾞}
  \begin{phonetics}{辆}{liang4}[][HSK 2]
    \definition{clas.}{para automóveis, veículos, etc.}
  \end{phonetics}
\end{entry}

\begin{entry}{辈}{12}{⾞}
  \begin{phonetics}{辈}{bei4}[][HSK 5]
    \definition{s.}{geração da família | semelhante; círculo familiar; pessoas de um determinado tipo | vida útil; tempo de vida}
  \end{phonetics}
\end{entry}

\begin{entry}{输}{13}{⾞}
  \begin{phonetics}{输}{shu1}[][HSK 3]
    \definition{v.}{transportar; transmitir | contribuir com dinheiro; doar | perder; ser batido; ser derrotado}
  \end{phonetics}
\end{entry}

\begin{entry}{输入}{13,2}{⾞、⼊}
  \begin{phonetics}{输入}{shu1ru4}[][HSK 3]
    \definition{v.}{introduzir; importar  (de fora para dentro) | inserir informações, programas, dados, sinais, etc. em uma máquina}
  \end{phonetics}
\end{entry}

\begin{entry}{输出}{13,5}{⾞、⼐}
  \begin{phonetics}{输出}{shu1 chu1}[][HSK 5]
    \definition{v.}{exportar (de dentro para fora); transportar (de dentro) para fora | exportar; vender ou distribuir no exterior ou fora do país | emitir informações, programas, dados, sinais, etc. a partir de uma máquina; enviar por uma determinada instituição ou dispositivo (energia, sinal, etc.)}
  \end{phonetics}
\end{entry}

%%%%% EOF %%%%%

