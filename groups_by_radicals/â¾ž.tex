%%%
%%% Radical "⾞"
%%%

\section*{Radical 159: ``⾞'' (车)}\addcontentsline{toc}{section}{Radical 159: ⾞、车}

\begin{Entry}{车}{4}{⾞}[Kangxi 159]
  \begin{Phonetics}{车}{che1}[][HSK 1]
    \definition*{s.}{Sobrenome Che}
    \definition[辆]{s.}{veículo; meios de transporte terrestres sobre rodas | máquina ou instrumento com rodas; ferramentas com eixo giratório | máquina}
    \definition{v.}{tornear; usinar com torno mecânico | elevar água por meio de uma roda d'água; usar caminhão-pipa para coletar água | girar, geralmente se refere ao corpo}
  \end{Phonetics}
  \begin{Phonetics}{车}{ju1}
    \definition{s.}{torre; castelo; carruagem, uma das peças do xadrez chinês}
  \end{Phonetics}
\end{Entry}

\begin{Entry}{车上}{4,3}{⾞、⼀}
  \begin{Phonetics}{车上}{che1 shang4}[][HSK 1]
    \definition{adv.}{no carro; no interior do veículo}
  \end{Phonetics}
\end{Entry}

\begin{Entry}{车子}{4,3}{⾞、⼦}
  \begin{Phonetics}{车子}{che1zi5}
    \definition{s.}{qualquer veículo (carro, bicicleta, caminhão, etc)}
  \end{Phonetics}
\end{Entry}

\begin{Entry}{车水马龙}{4,4,3,5}{⾞、⽔、⾺、⿓}
  \begin{Phonetics}{车水马龙}{che1shui3-ma3long2}
    \definition{expr.}{tráfego engarrafado | engarrafamento | (literalmente) ``fluxo interminável de cavalos e carruagens''}
  \end{Phonetics}
\end{Entry}

\begin{Entry}{车主}{4,5}{⾞、⼂}
  \begin{Phonetics}{车主}{che1 zhu3}[][HSK 5]
    \definition[位,个]{s.}{proprietário do carro; uma pessoa física ou família que possui um veículo motorizado}
  \end{Phonetics}
\end{Entry}

\begin{Entry}{车号}{4,5}{⾞、⼝}
  \begin{Phonetics}{车号}{che1 hao4}[][HSK 6]
    \definition[个]{s.}{número da licença (de um veículo) | número do carro}
  \end{Phonetics}
\end{Entry}

\begin{Entry}{车次}{4,6}{⾞、⽋}
  \begin{Phonetics}{车次}{che1ci4}
    \definition{s.}{número do trem}
  \end{Phonetics}
\end{Entry}

\begin{Entry}{车位}{4,7}{⾞、⼈}
  \begin{Phonetics}{车位}{che1wei4}[][HSK 7-9]
    \definition[个]{s.}{vaga de estacionamento | vaga de garagem | ponto de táxi | ponto de descarga}
  \end{Phonetics}
\end{Entry}

\begin{Entry}{车库}{4,7}{⾞、⼴}
  \begin{Phonetics}{车库}{che1ku4}
    \definition{s.}{garagem}
  \end{Phonetics}
\end{Entry}

\begin{Entry}{车间}{4,7}{⾞、⾨}
  \begin{Phonetics}{车间}{che1jian1}[][HSK 7-9]
    \definition[个,栋]{s.}{fábrica; oficina; uma unidade dentro de uma empresa que conclui certos processos no processo de produção ou produz certos produtos de forma independente}
  \end{Phonetics}
\end{Entry}

\begin{Entry}{车轮}{4,8}{⾞、⾞}
  \begin{Phonetics}{车轮}{che1lun2}[][HSK 7-9]
    \definition[个]{s.}{roda (de um veículo); a parte redonda e giratória de um carro; também chamada de 车轮子}
  \seealsoref{车轮子}{che1lun2zi5}
  \end{Phonetics}
\end{Entry}

\begin{Entry}{车轮子}{4,8,3}{⾞、⾞、⼦}
  \begin{Phonetics}{车轮子}{che1lun2zi5}
    \definition{s.}{roda}
  \end{Phonetics}
\end{Entry}

\begin{Entry}{车型}{4,9}{⾞、⼟}
  \begin{Phonetics}{车型}{che1xing2}[][HSK 7-9]
    \definition{s.}{marca e modelo do carro; tipo de motocicleta | modelo (de um carro)}
  \end{Phonetics}
\end{Entry}

\begin{Entry}{车轴}{4,9}{⾞、⾞}
  \begin{Phonetics}{车轴}{che1zhou2}[][HSK 7-9]
    \definition[根,条,个]{s.}{eixo (de um veículo)}[这辆货车一共有六个轮子,所以有三条车轴。===Este caminhão tem seis rodas, então ele tem três eixos.]
  \end{Phonetics}
\end{Entry}

\begin{Entry}{车展}{4,10}{⾞、⼫}
  \begin{Phonetics}{车展}{che1 zhan3}[][HSK 6]
    \definition{s.}{salão do automóvel; exposição de carros}
  \end{Phonetics}
\end{Entry}

\begin{Entry}{车站}{4,10}{⾞、⽴}
  \begin{Phonetics}{车站}{che1 zhan4}[][HSK 1]
    \definition[个,处]{s.}{estação; estação ferroviária; parada; pontos de parada estabelecidos nas linhas de transporte rodoviário são locais para embarque e desembarque de passageiros ou carga e descarga de mercadorias}
  \end{Phonetics}
\end{Entry}

\begin{Entry}{车速}{4,10}{⾞、⾡}
  \begin{Phonetics}{车速}{che1su4}[][HSK 7-9]
    \definition{s.}{velocidade de um veículo | velocidade de um torno}
  \end{Phonetics}
\end{Entry}

\begin{Entry}{车厢}{4,11}{⾞、⼚}
  \begin{Phonetics}{车厢}{che1xiang1}[][HSK 7-9]
    \definition[节,个,号,辆]{s.}{compartimento; vagão ferroviário; trem de vagões; a parte de um carro ou outro veículo usado para transportar pessoas ou coisas}
  \end{Phonetics}
\end{Entry}

\begin{Entry}{车票}{4,11}{⾞、⽰}
  \begin{Phonetics}{车票}{che1 piao4}[][HSK 1]
    \definition{s.}{passagem de trem ou ônibus; bilhete; bilhete de transporte público}
  \end{Phonetics}
\end{Entry}

\begin{Entry}{车祸}{4,11}{⾞、⽰}
  \begin{Phonetics}{车祸}{che1huo4}[][HSK 7-9]
    \definition[场,起]{s.}{acidente de trânsito; acidente automobilístico; acidentes envolvendo vítimas durante a condução (principalmente carros)}
  \end{Phonetics}
\end{Entry}

\begin{Entry}{车辆}{4,11}{⾞、⾞}
  \begin{Phonetics}{车辆}{che1 liang4}[][HSK 2]
    \definition{s.}{veículo; carro; termo genérico para todos os tipos de veículos}
  \end{Phonetics}
\end{Entry}

\begin{Entry}{车牌}{4,12}{⾞、⽚}
  \begin{Phonetics}{车牌}{che1 pai2}[][HSK 6]
    \definition[个,块]{s.}{placa de licença; placa instalada no veículo}
  \end{Phonetics}
\end{Entry}

\begin{Entry}{车道}{4,12}{⾞、⾡}
  \begin{Phonetics}{车道}{che1dao4}[][HSK 7-9]
    \definition{s.}{(tráfego) pista; faixa; estrada}[三车道公路。===Rodovia de três pistas.]
  \end{Phonetics}
\end{Entry}

\begin{Entry}{轨}{6}{⾞}
  \begin{Phonetics}{轨}{gui3}
    \definition{s.}{trilho; pista | curso; caminho | ordem; regulamento; regra | rotina; metaforicamente falando, métodos, regras, ordem, etc.}
    \definition{v.}{seguir | Literário: cumprir; aderir a}
  \end{Phonetics}
\end{Entry}

\begin{Entry}{轨道}{6,12}{⾞、⾡}
  \begin{Phonetics}{轨道}{gui3dao4}[][HSK 6]
    \definition[条]{s.}{trilha; uma rota pavimentada com trilhos de aço para trens, bondes, etc. | órbita; trajetória; corpos celestes e objetos têm trajetórias de movimento regulares | caminho; curso; maneira adequada de fazer as coisas; curso adequado; uma metáfora para o desenvolvimento normal das coisas ou as normas e procedimentos que as pessoas devem seguir}
  \end{Phonetics}
\end{Entry}

\begin{Entry}{转}{8}{⾞}
  \begin{Phonetics}{转}{zhuai3}
  \end{Phonetics}
  \begin{Phonetics}{转}{zhuan3}
    \definition{v.}{mudar; deslocar; transferir; virar; mudar de direção, posição, situação, circunstâncias, etc. | transmitir; transferir; passar adiante}
  \end{Phonetics}
  \begin{Phonetics}{转}{zhuan4}[][HSK 3,6]
    \definition{clas.}{usado para rotações (por minuto, por segundo, etc.): RPM}
    \definition{v.}{girar; rodar; revolver; movimento em torno de um centro | passear; dar uma volta}
  \end{Phonetics}
\end{Entry}

\begin{Entry}{转化}{8,4}{⾞、⼔}
  \begin{Phonetics}{转化}{zhuan3 hua4}[][HSK 5]
    \definition{v.}{mudar; transformar | inverter; converter}
  \end{Phonetics}
\end{Entry}

\begin{Entry}{转让}{8,5}{⾞、⾔}
  \begin{Phonetics}{转让}{zhuan3rang4}[][HSK 5]
    \definition{v.}{ceder; fazer a entrega; transferir a posse de; ceder seus bens ou direitos a outra pessoa}
  \end{Phonetics}
\end{Entry}

\begin{Entry}{转产}{8,6}{⾞、⼇}
  \begin{Phonetics}{转产}{zhuan3chan3}
    \definition{v.}{mudar a produção | mudar para novos produtos}
  \end{Phonetics}
\end{Entry}

\begin{Entry}{转动}{8,6}{⾞、⼒}
  \begin{Phonetics}{转动}{zhuan3 dong4}[][HSK 4]
    \definition{v.}{girar; rodar; dar voltas; torcer | dar a volta em algo}
  \end{Phonetics}
  \begin{Phonetics}{转动}{zhuan4 dong4}[][HSK 6]
    \definition{s.}{tambor; roda}
    \definition{v.}{girar; correr; rolar; revolver; rotacionar; torcer}
  \end{Phonetics}
\end{Entry}

\begin{Entry}{转向}{8,6}{⾞、⼝}
  \begin{Phonetics}{转向}{zhuan3 xiang4}[][HSK 5]
    \definition{v.}{desviar; desviar-se; mudar a direção do avanço | mudar a posição política de alguém | mudar de direção; virar-se para (a outra parte)}
  \end{Phonetics}
  \begin{Phonetics}{转向}{zhuan4/xiang4}
    \definition{v.+compl.}{perder-se; perder o rumo; não consiguir distinguir a direção; estar perdido}
  \end{Phonetics}
\end{Entry}

\begin{Entry}{转告}{8,7}{⾞、⼝}
  \begin{Phonetics}{转告}{zhuan3gao4}[][HSK 4]
    \definition{v.}{passar adiante; comunicar; transmitir; ser instruído a dizer a outra parte o que uma pessoa diz, o que está acontecendo, etc.}
  \end{Phonetics}
\end{Entry}

\begin{Entry}{转身}{8,7}{⾞、⾝}
  \begin{Phonetics}{转身}{zhuan3 shen1}[][HSK 4]
    \definition{adv.}{em um instante; em um piscar de olhos}
    \definition{v.}{dar a volta; dar meia-volta; dar a volta por cima | virar; girar; refere-se a uma mudança de direção, localização, natureza, etc.}
  \end{Phonetics}
\end{Entry}

\begin{Entry}{转变}{8,8}{⾞、⼜}
  \begin{Phonetics}{转变}{zhuan3bian4}[][HSK 3]
    \definition{v.}{mudar; converter; transformar}
  \end{Phonetics}
\end{Entry}

\begin{Entry}{转念}{8,8}{⾞、⼼}
  \begin{Phonetics}{转念}{zhuan3nian4}
    \definition{v.}{ter dúvidas sobre algo | pensar melhor}
  \end{Phonetics}
\end{Entry}

\begin{Entry}{转账}{8,8}{⾞、⾙}
  \begin{Phonetics}{转账}{zhuan3/zhang4}
    \definition{v.+compl.}{transferir entre contas | trazer à frente | incluir uma soma de dinheiro do balanço anterior no seguinte}
  \end{Phonetics}
\end{Entry}

\begin{Entry}{转弯}{8,9}{⾞、⼸}
  \begin{Phonetics}{转弯}{zhuan4 wan1}[][HSK 4]
    \definition{s.}{esquina; curva}[小心急转弯。===Tenha cuidado em curvas fechadas.]
    \definition{v.}{rodar; desviar; virar uma esquina; fazer uma curva; fazer uma curva de 180º}
  \end{Phonetics}
\end{Entry}

\begin{Entry}{转换}{8,10}{⾞、⼿}
  \begin{Phonetics}{转换}{zhuan3 huan4}[][HSK 5]
    \definition{v.}{mudar; trocar; converter; transformar; alterar}
  \end{Phonetics}
\end{Entry}

\begin{Entry}{转递}{8,10}{⾞、⾡}
  \begin{Phonetics}{转递}{zhuan3di4}
    \definition{v.}{passar | retransmitir}
  \end{Phonetics}
\end{Entry}

\begin{Entry}{转悠}{8,11}{⾞、⼼}
  \begin{Phonetics}{转悠}{zhuan4you5}
    \definition{v.}{aparecer repetidamente | rolar | passear por aí}
  \end{Phonetics}
\end{Entry}

\begin{Entry}{转移}{8,11}{⾞、⽲}
  \begin{Phonetics}{转移}{zhuan3yi2}[][HSK 4]
    \definition{v.}{deslocar; desviar; transferir; redirecionar; reposicionar; reorientar | mudar; transformar}
  \end{Phonetics}
\end{Entry}

\begin{Entry}{转游}{8,12}{⾞、⽔}
  \begin{Phonetics}{转游}{zhuan4you5}
  \seealsoref{转悠}{zhuan4you5}
  \end{Phonetics}
\end{Entry}

\begin{Entry}{轮}{8}{⾞}
  \begin{Phonetics}{轮}{lun2}[][HSK 4]
    \definition{clas.}{usado para sol vermelho, lua brilhante, etc. | usado para rodadas | doze anos de idade (os doze ramos terrestres são usados para lembrar o gênero humano e cada doze anos de idade é um ciclo)}
    \definition{s.}{roda | anel; disco; objeto semelhante a uma roda | navio a vapor; barco a vapor}
    \definition{v.}{revezar; substituir um ao outro em sequência (para fazer algo)}
  \end{Phonetics}
\end{Entry}

\begin{Entry}{轮子}{8,3}{⾞、⼦}
  \begin{Phonetics}{轮子}{lun2 zi5}[][HSK 4]
    \definition[个,只]{s.}{roda; peças circulares de veículos ou máquinas com capacidade de rotação}
  \end{Phonetics}
\end{Entry}

\begin{Entry}{轮回}{8,6}{⾞、⼞}
  \begin{Phonetics}{轮回}{lun2hui2}
    \definition[个]{s.}{reencarnação (Budismo) | ciclo}
    \definition{v.}{reencarnar}
  \end{Phonetics}
\end{Entry}

\begin{Entry}{轮船}{8,11}{⾞、⾈}
  \begin{Phonetics}{轮船}{lun2chuan2}[][HSK 4]
    \definition[艘,班]{s.}{vapor; navio a vapor; barco a vapor}
  \end{Phonetics}
\end{Entry}

\begin{Entry}{轮椅}{8,12}{⾞、⽊}
  \begin{Phonetics}{轮椅}{lun2 yi3}[][HSK 4]
    \definition{s.}{cadeira de rodas; dispositivo de assento especialmente projetado com rodas para pessoas com dificuldade de locomoção, que pode ser acionado por um disco de roda ou manivela operados manualmente}
  \end{Phonetics}
\end{Entry}

\begin{Entry}{软}{8}{⾞}
  \begin{Phonetics}{软}{ruan3}[][HSK 5]
    \definition*{s.}{Sobrenome Ruan}
    \definition{adj.}{macio; flexível; maleável; maleável (oposto de 硬) | suave; brando; delicado | fraco; débil | de baixa qualidade, capacidade, etc. | facilmente movido (ou influenciado) | de maneira suave (ou gentil) | indulgente; tolerante | maleável; flexível | fácil de se emocionar ou abalar}
  \seealsoref{硬}{ying4}
  \end{Phonetics}
\end{Entry}

\begin{Entry}{软件}{8,6}{⾞、⼈}
  \begin{Phonetics}{软件}{ruan3jian4}[][HSK 5]
    \definition[款,个]{s.}{\emph{software}; programas de computador, procedimentos, regras e quaisquer arquivos, documentos e dados relacionados à operação do sistema de computador}
  \end{Phonetics}
\end{Entry}

\begin{Entry}{轰}{8}{⾞}
  \begin{Phonetics}{轰}{hong1}
    \definition{interj.}{(onomatopéia) Bum!; estrondo; refere-se aos ruídos altos feitos por trovões, fogo de artilharia, etc.}
    \definition{v.}{retumbar; bombardear; explodir | espantar; expulsar}
  \end{Phonetics}
\end{Entry}

\begin{Entry}{轰鸣}{8,8}{⾞、⿃}
  \begin{Phonetics}{轰鸣}{hong1ming2}
    \definition{s.}{bum (som de explosão) | estrondo}
  \end{Phonetics}
\end{Entry}

\begin{Entry}{轰炸机}{8,9,6}{⾞、⽕、⽊}
  \begin{Phonetics}{轰炸机}{hong1zha4ji1}
    \definition{s.}{bombardeiro (aeronave)}
  \end{Phonetics}
\end{Entry}

\begin{Entry}{轴}{9}{⾞}
  \begin{Phonetics}{轴}{zhou2}
    \definition{adj.}{(movimento) inflexível; rígido; desajeitado | direto; franco; decidido | em pergaminho}
    \definition{clas.}{usado para as linhas enroladas ao redor do eixo e as pinturas montadas no eixo}
    \definition{s.}{eixo | carretel; haste | rolo; pergaminho; objeto de enrolamento cilíndrico}
  \end{Phonetics}
  \begin{Phonetics}{轴}{zhou4}
    \definition{s.}{a parte final da performance; a última e central peça de uma peça dramática}
  \end{Phonetics}
\end{Entry}

\begin{Entry}{轴承}{9,8}{⾞、⼿}
  \begin{Phonetics}{轴承}{zhou2cheng2}
    \definition{s.}{(mecânico) rolamento}
  \end{Phonetics}
\end{Entry}

\begin{Entry}{轻}{9}{⾞}
  \begin{Phonetics}{轻}{qing1}[][HSK 2]
    \definition{adj.}{de pouco peso; leve (oposto de 重) | (de carga, equipamento, etc.) pequeno; simples | pequeno em número, grau, etc. | não sério; relaxante; leve | sem importância | suave; delicado | levianos, crédulos | leve; peso leve; densidade baixa | leve; descontraído; fácil | imprudente; descuidado | inconstante; frívolo}
    \definition{v.}{menosprezar; subestimar}
  \seealsoref{重}{zhong4}
  \end{Phonetics}
\end{Entry}

\begin{Entry}{轻易}{9,8}{⾞、⽇}
  \begin{Phonetics}{轻易}{qing1yi4}[][HSK 4]
    \definition{adv.}{facilmente; prontamente | facilmente; precipitadamente; indica que uma ação é realizada casualmente, geralmente usado em frases negativas}
  \end{Phonetics}
\end{Entry}

\begin{Entry}{轻松}{9,8}{⾞、⽊}
  \begin{Phonetics}{轻松}{qing1song1}[][HSK 4]
    \definition{adj.}{leve; relaxado; livre de fardos; não nervoso; não cansado}
    \definition{v.}{sentir-se livre de fardos; não se sentir nervoso ou cansado}
  \end{Phonetics}
\end{Entry}

\begin{Entry}{较}{10}{⾞}
  \begin{Phonetics}{较}{jiao4}[][HSK 3]
    \definition{adj.}{claro; óbvio; evidente}
    \definition{adv.}{comparativamente; relativamente; razoavelmente; bastante; bastante}
    \definition{prep.}{usado para comparar características e graus; introduzir o objeto de comparação; equivalente a 比}
    \definition{v.}{comparar | disputar}
  \seealsoref{比}{bi3}
  \end{Phonetics}
\end{Entry}

\begin{Entry}{辅}{11}{⾞}
  \begin{Phonetics}{辅}{fu3}
    \definition*{s.}{Sobrenome Fu}
    \definition{adj.}{subsidiário}
    \definition{s.}{barras laterais do carrinho atuando como proteção da roda; duas barras retas de madeira são adicionadas na parte externa da roda para prender o cubo | maçã do rosto | assistente oficial; títulos oficiais antigos | (literário) território que circunda a capital}
    \definition{v.}{auxiliar; complementar; suplementar | ajudar}
  \end{Phonetics}
\end{Entry}

\begin{Entry}{辅助}{11,7}{⾞、⼒}
  \begin{Phonetics}{辅助}{fu3zhu4}[][HSK 5]
    \definition{adj.}{auxiliar; suplementar; complementar}
    \definition{v.}{auxiliar; ajudar; colocar os outros em primeiro lugar e dar-lhes alguma ajuda externa}
  \end{Phonetics}
\end{Entry}

\begin{Entry}{辆}{11}{⾞}
  \begin{Phonetics}{辆}{liang4}[][HSK 2]
    \definition{clas.}{usado para automóveis, veículos, etc.}
  \end{Phonetics}
\end{Entry}

\begin{Entry}{辈}{12}{⾞}
  \begin{Phonetics}{辈}{bei4}[][HSK 5]
    \definition*{s.}{Sobrenome Bei}
    \definition{s.}{pessoas de um certo tipo; semelhantes | geração; geração na família | duração da vida | círculo familiar}
  \end{Phonetics}
\end{Entry}

\begin{Entry}{输}{13}{⾞}
  \begin{Phonetics}{输}{shu1}[][HSK 3]
    \definition{v.}{transportar; entregar | contribuir com dinheiro; doar | perder; falhar; ser batido; ser derrotado}
  \end{Phonetics}
\end{Entry}

\begin{Entry}{输入}{13,2}{⾞、⼊}
  \begin{Phonetics}{输入}{shu1ru4}[][HSK 3]
    \definition{v.}{introduzir; importar; comprar bens, introduzir tecnologia, contratar mão de obra, introduzir capital, etc. | inserir informações, programas, dados, sinais, etc. em uma máquina}
  \end{Phonetics}
\end{Entry}

\begin{Entry}{输出}{13,5}{⾞、⼐}
  \begin{Phonetics}{输出}{shu1 chu1}[][HSK 5]
    \definition{v.}{exportar (de dentro para fora); transportar (de dentro) para fora | exportar; vender ou distribuir no exterior ou fora do país | emitir informações, programas, dados, sinais, etc. a partir de uma máquina; enviar por uma determinada instituição ou dispositivo (energia, sinal, etc.)}
  \end{Phonetics}
\end{Entry}

\begin{Entry}{辗}{14}{⾞}
  \begin{Phonetics}{辗}{zhan3}
    \definition{v.}{(arcaico) virar | (arcaico) rolar para o lado | (arcaico) virar a metade}
  \end{Phonetics}
\end{Entry}

%%%%% EOF %%%%%

