%%%
%%% Radical "⾩"
%%%

\section*{Radical 170: ``⾩'' (⻖)}\addcontentsline{toc}{section}{Radical 170: ⾩、⻖}

\begin{entry}{队}{4}{⾩}
  \begin{phonetics}{队}{dui4}[][HSK 2]
    \definition*{s.}{Jovens Pioneiros; refere-se especificamente à Patrulha de Jovens Pioneiros}
    \definition[条,个,支]{s.}{fila de pessoas | equipe; grupo;}
  \end{phonetics}
\end{entry}

\begin{entry}{队友}{4,4}{⾩、⼜}
  \begin{phonetics}{队友}{dui4you3}
    \definition{s.}{companheiro de equipe}
  \end{phonetics}
\end{entry}

\begin{entry}{队长}{4,4}{⾩、⾧}
  \begin{phonetics}{队长}{dui4 zhang3}[][HSK 2]
    \definition[个,位,名]{s.}{capitão (de equipe); capitães | líder de equipe}
  \end{phonetics}
\end{entry}

\begin{entry}{队员}{4,7}{⾩、⼝}
  \begin{phonetics}{队员}{dui4 yuan2}[][HSK 3]
    \definition{s.}{membro da equipe}
  \end{phonetics}
\end{entry}

\begin{entry}{防}{6}{⾩}
  \begin{phonetics}{防}{fang2}[][HSK 3]
    \definition*{s.}{sobrenome Fang}
    \definition{s.}{defesa | barragem; dique; aterro}
    \definition{v.}{prover contra; defender contra; proteger contra}
  \end{phonetics}
\end{entry}

\begin{entry}{防止}{6,4}{⾩、⽌}
  \begin{phonetics}{防止}{fang2zhi3}[][HSK 3]
    \definition{v.}{evitar; prevenir; prevenir; proteger contra}
  \end{phonetics}
\end{entry}

\begin{entry}{防护}{6,7}{⾩、⼿}
  \begin{phonetics}{防护}{fang2hu4}
    \definition{v.}{defender | proteger}
  \end{phonetics}
\end{entry}

\begin{entry}{防治}{6,8}{⾩、⽔}
  \begin{phonetics}{防治}{fang2zhi4}[][HSK 5]
    \definition{s.}{tratamento preventivo; prevenção e cura; profilaxia e tratamento}
  \end{phonetics}
\end{entry}

\begin{entry}{防晒}{6,10}{⾩、⽇}
  \begin{phonetics}{防晒}{fang2shai4}
    \definition{s.}{protetor solar}
  \end{phonetics}
\end{entry}

\begin{entry}{阳}{6}{⾩}
  \begin{phonetics}{阳}{yang2}
    \definition*{s.}{Yang (o princípio positivo de Yin e Yang)}
    \definition{s.}{positivo (eletricidade) | sol}
  \seealsoref{阴}{yin1}
  \seealsoref{阴阳}{yin1yang2}
  \end{phonetics}
\end{entry}

\begin{entry}{阳台}{6,5}{⾩、⼝}
  \begin{phonetics}{阳台}{yang2tai2}[][HSK 4]
    \definition{s.}{varanda; terraço; sacada; pequeno terraço do edifício com grades para se refrescar, tomar sol ou olhar o horizonte}
  \end{phonetics}
\end{entry}

\begin{entry}{阳光}{6,6}{⾩、⼉}
  \begin{phonetics}{阳光}{yang2guang1}[][HSK 3]
    \definition{adj.}{alegre; otimista; personalidade positiva e alegre; cheia de vitalidade juvenil | aberto; transparente; público; conduzido sob supervisão pública}
    \definition[缕,束,道]{s.}{luz do sol; brilho do sol; raio de sol}
  \end{phonetics}
\end{entry}

\begin{entry}{阴}{6}{⾩}
  \begin{phonetics}{阴}{yin1}[][HSK 2]
    \definition*{s.}{sobrenome Yin}
    \definition*{s.}{Yin (o princípio negativo de Yin e Yang)}
    \definition{adj.}{nublado | sombrio | escondido | implícito}
    \definition{s.}{negativo (eletricidade) | lua}
  \seealsoref{阳}{yang2}
  \seealsoref{阴阳}{yin1yang2}
  \end{phonetics}
\end{entry}

\begin{entry}{阴天}{6,4}{⾩、⼤}
  \begin{phonetics}{阴天}{yin1 tian1}[][HSK 2]
    \definition{adj.}{céu nublado | céu cinzento}
  \end{phonetics}
\end{entry}

\begin{entry}{阴阳}{6,6}{⾩、⾩}
  \begin{phonetics}{阴阳}{yin1yang2}
    \definition*{s.}{Yin e Yang}
  \seealsoref{阳}{yang2}
  \seealsoref{阴}{yin1}
  \end{phonetics}
\end{entry}

\begin{entry}{阵}{6}{⾩}
  \begin{phonetics}{阵}{zhen4}[][HSK 4]
    \definition{clas.}{passagens que expressam a passagem de eventos ou ações}
    \definition{s.}{matriz de batalha (formação); termo tático antigo para as fileiras ou formações de uma equipe de combate | \emph{front}; frente de batalha; posição | um período de tempo}
  \end{phonetics}
\end{entry}

\begin{entry}{阵地}{6,6}{⾩、⼟}
  \begin{phonetics}{阵地}{zhen4di4}
    \definition{s.}{posição (militar) | frente de batalha | \emph{front}}
  \end{phonetics}
\end{entry}

\begin{entry}{阶段}{6,9}{⾩、⽎}
  \begin{phonetics}{阶段}{jie1duan4}[][HSK 4]
    \definition{s.}{estágio; fase; período; bancada; gradação}
  \end{phonetics}
\end{entry}

\begin{entry}{阻止}{7,4}{⾩、⽌}
  \begin{phonetics}{阻止}{zu3zhi3}[][HSK 4]
    \definition{v.}{parar; reter; conter; interromper; impedir o avanço; impedir o movimento; obstruir}
  \end{phonetics}
\end{entry}

\begin{entry}{阻击}{7,5}{⾩、⼐}
  \begin{phonetics}{阻击}{zu3ji1}
    \definition{v.}{verificar | parar}
  \end{phonetics}
\end{entry}

\begin{entry}{阻碍}{7,13}{⾩、⽯}
  \begin{phonetics}{阻碍}{zu3'ai4}[][HSK 5]
    \definition{s.}{obstáculo; impedimento; barreira}
    \definition{v.}{bloquear; impedir; obstruir; impedir o bom andamento ou desenvolvimento}
  \end{phonetics}
\end{entry}

\begin{entry}{阿}{7}{⾩}
  \begin{phonetics}{阿}{a1}
    \definition{pref.}{em dialetos do sul para formar termos carinhosos, antes de nomes de animais de estimação, sobrenomes monossilábicos ou números que denotam ordem de antiguidade em uma; anexado a 大, 二, 三\dots\ para indicar classificação (e, às vezes, intimidade) | antes dos termos de parentesco; na frente de um sobrenome, de um nome próprio ou de um determinado título, com uma conotação de intimidade | em alguns contextos, pode soar infantil ou muito informal (por exemplo, chamar um colega de trabalho por ``阿 + Nome'' sem intimidade)}[阿妈 (mamãe) | 阿明 (forma carinhosa de chamar alguém chamado Ming)]
  \end{phonetics}
  \begin{phonetics}{阿}{e1}
    \definition*{s.}{sobrenome E}
    \definition*{s.}{Dong'e (um condado na província de Shandong)}
    \definition{s.}{grande monte (ou colina) | um lugar sinuoso (montanha, água, etc.)}
    \definition{v.}{bajular; satisfazer}
  \end{phonetics}
\end{entry}

\begin{entry}{阿姨}{7,9}{⾩、⼥}
  \begin{phonetics}{阿姨}{a1yi2}[][HSK 4]
    \definition[个,位]{s.}{tia; uma forma de tratamento para uma mulher da geração dos pais; dirigir-se a uma mulher que tem aproximadamente a mesma idade da sua mãe, geralmente não é parente | babá em uma família; professora em um jardim de infância | tia; irmã da mãe (mais comum no sul da China)}[阿姨,生日快乐!(Tia, feliz aniversário!) | 阿姨,这个苹果多少钱一斤?(Tia/Senhora, quanto custa o quilo dessas maçãs?) | 阿姨,我想喝水。(Tia/Babá, eu quero beber água.)]
  \end{phonetics}
\end{entry}

\begin{entry}{阿哥}{7,10}{⾩、⼝}
  \begin{phonetics}{阿哥}{a1ge1}
    \definition{s.}{irmão mais velho (afetivo)}[阿哥,帮我拿一下书包!(Irmão, ajude-me com minha mochila escolar!)]
  \end{phonetics}
\end{entry}

\begin{entry}{附件}{7,6}{⾩、⼈}
  \begin{phonetics}{附件}{fu4jian4}[][HSK 5]
    \definition*{s.}{\emph{Adnexa Uteri} ; refere-se à genitália interna feminina que não seja o útero, as trompas de falópio e os ovários}
    \definition{s.}{apêndice; documentos que acompanham o documento principal | acessório; anexo; peças ou sobressalentes que não sejam peças principais de máquinas e equipamentos | anexo; documentos ou itens relevantes emitidos com o documento}
  \end{phonetics}
\end{entry}

\begin{entry}{附近}{7,7}{⾩、⾡}
  \begin{phonetics}{附近}{fu4jin4}[][HSK 4]
    \definition{adj.}{perto; vizinho}
    \definition{s.}{vizinhança; bairro}
  \end{phonetics}
\end{entry}

\begin{entry}{陆地}{7,6}{⾩、⼟}
  \begin{phonetics}{陆地}{lu4di4}[][HSK 4]
    \definition[块,片]{s.}{terra; terra seca (em oposição ao mar); superfície da Terra, excluindo os oceanos (e, às vezes, rios e lagos)}
  \end{phonetics}
\end{entry}

\begin{entry}{陆续}{7,11}{⾩、⽷}
  \begin{phonetics}{陆续}{lu4xu4}[][HSK 4]
    \definition{adv.}{sucessivamente; um após o outro; intermitentemente}
  \end{phonetics}
\end{entry}

\begin{entry}{陆路}{7,13}{⾩、⾜}
  \begin{phonetics}{陆路}{lu4lu4}
    \definition{s.}{rota terrestre}
  \end{phonetics}
\end{entry}

\begin{entry}{降}{8}{⾩}
  \begin{phonetics}{降}{jiang4}[][HSK 4]
    \definition*{s.}{sobrenome Jiang}
    \definition{v.}{cair; descer | diminuir; reduzir | nascer}
  \end{phonetics}
\end{entry}

\begin{entry}{降价}{8,6}{⾩、⼈}
  \begin{phonetics}{降价}{jiang4 jia4}[][HSK 4]
    \definition{v.}{ficar mais barato; cortar o preço; reduzir o preço}
  \end{phonetics}
\end{entry}

\begin{entry}{降低}{8,7}{⾩、⼈}
  \begin{phonetics}{降低}{jiang4di1}[][HSK 4]
    \definition{v.}{reduzir; cortar; diminuir; rebaixar; cair; abaixar}
  \end{phonetics}
\end{entry}

\begin{entry}{降温}{8,12}{⾩、⽔}
  \begin{phonetics}{降温}{jiang4 wen1}[][HSK 4]
    \definition{v.}{baixar a temperatura (como em uma oficina);  recusar | cair a temperatura | esfriar; resfriar; metáfora para um declínio no entusiasmo ou uma diminuição no ímpeto de algo}
  \end{phonetics}
\end{entry}

\begin{entry}{降落}{8,12}{⾩、⾋}
  \begin{phonetics}{降落}{jiang4luo4}[][HSK 4]
    \definition{v.}{aterrissar; descer; descer do céu}
  \end{phonetics}
\end{entry}

\begin{entry}{限制}{8,8}{⾩、⼑}
  \begin{phonetics}{限制}{xian4zhi4}[][HSK 4]
    \definition{s.}{limite; restrição; confinamento}
    \definition{v.}{limitar; adstringir; restringir; confinar; fechar em (sobre)}
  \end{phonetics}
\end{entry}

\begin{entry}{院}{9}{⾩}
  \begin{phonetics}{院}{yuan4}[][HSK 2]
    \definition[个]{s.}{pátio | instituição}
  \end{phonetics}
\end{entry}

\begin{entry}{院子}{9,3}{⾩、⼦}
  \begin{phonetics}{院子}{yuan4zi5}[][HSK 2]
    \definition[个]{s.}{pátio | jardim | quintal}
  \end{phonetics}
\end{entry}

\begin{entry}{院长}{9,4}{⾩、⾧}
  \begin{phonetics}{院长}{yuan4zhang3}[][HSK 2]
    \definition[个]{s.}{presidente de um conselho | reitor | chefe de departamento | primeiro-ministro da República da China | presidente de uma universidade}
  \end{phonetics}
\end{entry}

\begin{entry}{除了}{9,2}{⾩、⼅}
  \begin{phonetics}{除了}{chu2le5}[][HSK 3]
    \definition{prep.}{exceto; à parte | além disso; além de | ou \dots ou \dots}
  \end{phonetics}
\end{entry}

\begin{entry}{除夕}{9,3}{⾩、⼣}
  \begin{phonetics}{除夕}{chu2xi1}[][HSK 5]
    \definition*{s.}{Véspera de Ano Novo Lunar; a noite do último dia do ano, também se refere ao último dia do ano}
  \end{phonetics}
\end{entry}

\begin{entry}{除非}{9,8}{⾩、⾮}
  \begin{phonetics}{除非}{chu2fei1}[][HSK 5]
    \definition{conj.}{a menos que; somente se; indica a única condição, equivalente a 只有, frequentemente combinada com 才, 否则, 不然, etc.}
  \seealsoref{不然}{bu4ran2}
  \seealsoref{才}{cai2}
  \seealsoref{否则}{fou3ze2}
  \seealsoref{只有}{zhi3 you3}
  \end{phonetics}
\end{entry}

\begin{entry}{陪}{10}{⾩}
  \begin{phonetics}{陪}{pei2}[][HSK 5]
    \definition{v.}{servir; acompanhar; cuidar; fazer companhia a alguém | auxiliar; ajudar}
  \end{phonetics}
\end{entry}

\begin{entry}{陵园}{10,7}{⾩、⼞}
  \begin{phonetics}{陵园}{ling2yuan2}
    \definition{s.}{cemitério}
  \end{phonetics}
\end{entry}

\begin{entry}{陷入}{10,2}{⾩、⼊}
  \begin{phonetics}{陷入}{xian4ru4}
    \definition{v.}{afundar | ser pego em | pousar (em uma situação)}
  \end{phonetics}
\end{entry}

\begin{entry}{随}{11}{⾩}
  \begin{phonetics}{随}{sui2}[][HSK 3]
    \definition*{s.}{sobrenome Sui}
    \definition{v.}{seguir | cumprir com; adaptar-se a | deixar (alguém fazer o que ele gosta) | parecer-se com; assemelhar-se a}
  \end{phonetics}
\end{entry}

\begin{entry}{随手}{11,4}{⾩、⼿}
  \begin{phonetics}{随手}{sui2shou3}[][HSK 4]
    \definition{adv.}{convenientemente; sem problemas adicionais; casualmente}
  \end{phonetics}
\end{entry}

\begin{entry}{随处}{11,5}{⾩、⼡}
  \begin{phonetics}{随处}{sui2chu4}
    \definition{adv.}{em qualquer lugar}
  \end{phonetics}
\end{entry}

\begin{entry}{随后}{11,6}{⾩、⼝}
  \begin{phonetics}{随后}{sui2 hou4}[][HSK 5]
    \definition{adv.}{logo em seguida; logo depois; indica que segue imediatamente após a ação ou situação anterior (geralmente usado em conjunto com 就)}
  \seealsoref{就}{jiu4}
  \end{phonetics}
\end{entry}

\begin{entry}{随地}{11,6}{⾩、⼟}
  \begin{phonetics}{随地}{sui2di4}
    \definition{adv.}{qualquer lugar | todo lugar}
  \end{phonetics}
\end{entry}

\begin{entry}{随机存取记忆体}{11,6,6,8,5,4,7}{⾩、⽊、⼦、⼜、⾔、⼼、⼈}
  \begin{phonetics}{随机存取记忆体}{sui2ji1cun2qu3ji4yi4ti3}
    \definition{s.}{RAM (\emph{random access memory})}
  \seealsoref{内存}{nei4cun2}
  \seealsoref{随机存取存储器}{sui2ji1cun2qu3cun2chu3qi4}
  \end{phonetics}
\end{entry}

\begin{entry}{随机存取存储器}{11,6,6,8,6,12,16}{⾩、⽊、⼦、⼜、⼦、⼈、⼝}
  \begin{phonetics}{随机存取存储器}{sui2ji1cun2qu3cun2chu3qi4}
    \definition{s.}{RAM (\emph{random access memory})}
  \seealsoref{内存}{nei4cun2}
  \seealsoref{随机存取记忆体}{sui2ji1cun2qu3ji4yi4ti3}
  \end{phonetics}
\end{entry}

\begin{entry}{随时}{11,7}{⾩、⽇}
  \begin{phonetics}{随时}{sui2shi2}[][HSK 2]
    \definition{adv.}{a qualquer momento | sempre que necessário}
  \end{phonetics}
\end{entry}

\begin{entry}{随便}{11,9}{⾩、⼈}
  \begin{phonetics}{随便}{sui2bian4}[][HSK 2]
    \definition{adj.}{à vontade | como queira | como desejar | casual | negligente | devasso}
    \definition{adv.}{aleatoriamente}
  \end{phonetics}
\end{entry}

\begin{entry}{随着}{11,11}{⾩、⽬}
  \begin{phonetics}{随着}{sui2zhe5}[][HSK 5]
    \definition{prep.}{junto com; na esteira de; em sintonia com; usado no início da frase ou antes do verbo, indica as condições necessárias para que uma ação, comportamento ou evento ocorra}
  \end{phonetics}
\end{entry}

\begin{entry}{随意}{11,13}{⾩、⼼}
  \begin{phonetics}{随意}{sui2yi4}[][HSK 5]
    \definition{adj.}{aleatório; casual; à vontade; como se deseja}
  \end{phonetics}
\end{entry}

\begin{entry}{隔}{12}{⾩}
  \begin{phonetics}{隔}{ge2}[][HSK 4]
    \definition{adj.}{seguinte; vizinho}
    \definition{v.}{dividir; separar; bloquear; obstruir | estar a uma distância de, após ou em um intervalo de}
  \end{phonetics}
\end{entry}

\begin{entry}{隔开}{12,4}{⾩、⼶}
  \begin{phonetics}{隔开}{ge2 kai1}[][HSK 4]
    \definition{v.}{separar; manter separado; barricar; separar completamente duas pessoas (ou coisas) ou duas partes de uma coisa que estão intimamente unidas}
  \end{phonetics}
\end{entry}

\begin{entry}{隔壁}{12,16}{⾩、⼟}
  \begin{phonetics}{隔壁}{ge2bi4}[][HSK 5]
    \definition{s.}{vizinho; casas ou pessoas vizinhas | septo; distante (socialmente distante) | anteparo; partição}
  \end{phonetics}
\end{entry}

\begin{entry}{隧道}{14,12}{⾩、⾡}
  \begin{phonetics}{隧道}{sui4dao4}
    \definition{s.}{túnel}
  \end{phonetics}
\end{entry}

%%%%% EOF %%%%%

