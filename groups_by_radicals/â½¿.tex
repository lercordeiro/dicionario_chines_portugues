%%%
%%% Radical "⽿"
%%%

\section*{Radical 128: ``⽿''}\addcontentsline{toc}{section}{Radical 128: ⽿}

\begin{Entry}{耳}{6}{⽿}[Kangxi 128]
  \begin{Phonetics}{耳}{er3}
    \definition*{s.}{Sobrenome Er}
    \definition{part.}{(clássico) somente; apenas}
    \definition{s.}{orelha | coisa parecida com uma orelha | em ambos os lados; lado | orelha de um utensílio}
  \end{Phonetics}
\end{Entry}

\begin{Entry}{耳目一新}{6,5,1,13}{⽿、⽬、⼀、⽄}
  \begin{Phonetics}{耳目一新}{er3mu4-yi4xin1}[][HSK 7-9]
    \definition{expr.}{encontrar tudo fresco e novo; encontrar-se em um mundo inteiramente novo; apresentar uma nova aparência (de um lugar); uma mudança agradável de atmosfera; ``Tudo o que ouço e vejo mudou e parece novo.''}
  \end{Phonetics}
\end{Entry}

\begin{Entry}{耳光}{6,6}{⽿、⼉}
  \begin{Phonetics}{耳光}{er3guang1}[][HSK 7-9]
    \definition[个,记]{s.}{uma bofetada na orelha; um tapa na cara; (bater) no rosto em frente à orelha; a ação de bater no rosto}
  \end{Phonetics}
\end{Entry}

\begin{Entry}{耳朵}{6,6}{⽿、⽊}
  \begin{Phonetics}{耳朵}{er3duo5}[][HSK 5]
    \definition[双,只,个,对]{s.}{orelha; ouvido; órgão da audição e do equilíbrio}
  \end{Phonetics}
\end{Entry}

\begin{Entry}{耳机}{6,6}{⽿、⽊}
  \begin{Phonetics}{耳机}{er3 ji1}[][HSK 4]
    \definition[副,个,对]{s.}{fone de ouvido; receptor (de telefone); dispositivos que permitem que uma pessoa ouça sons sozinha, como ouvir música, histórias, chamadas telefônicas etc., usados na cabeça ou inseridos nos ouvidos}
  \end{Phonetics}
\end{Entry}

\begin{Entry}{耳闻目睹}{6,9,5,13}{⽿、⾨、⽬、⽬}
  \begin{Phonetics}{耳闻目睹}{er3wen2-mu4du3}[][HSK 7-9]
    \definition{expr.}{testemunhar pessoalmente; ver e ouvir pessoalmente; o que se vê e se ouve}
  \end{Phonetics}
\end{Entry}

\begin{Entry}{耳熟能详}{6,15,10,8}{⽿、⽕、⾁、⾔}
  \begin{Phonetics}{耳熟能详}{er3shu2-neng2xiang2}[][HSK 7-9]
    \definition{expr.}{o que é ouvido com frequência pode ser repetido em detalhes; já ouvi isso muitas vezes e estou familiarizado o suficiente para falar sobre isso em detalhes}
  \end{Phonetics}
\end{Entry}

\begin{Entry}{耻}{10}{⽿}
  \begin{Phonetics}{耻}{chi3}
    \definition{s.}{vergonha; desgraça; humilhação}
    \definition{v.}{estar envergonhado de; considerar vergonhoso}
  \end{Phonetics}
\end{Entry}

\begin{Entry}{耻笑}{10,10}{⽿、⽵}
  \begin{Phonetics}{耻笑}{chi3xiao4}[][HSK 7-9]
    \definition{v.}{ridicularizar alguém; zombar; zombar de; rir de}
  \end{Phonetics}
\end{Entry}

\begin{Entry}{耻辱}{10,10}{⽿、⾠}
  \begin{Phonetics}{耻辱}{chi3ru3}[][HSK 7-9]
    \definition{s.}{vergonha; desgraça; humilhação; danos à reputação; incidente vergonhoso}
  \end{Phonetics}
\end{Entry}

\begin{Entry}{耽}{10}{⽿}
  \begin{Phonetics}{耽}{dan1}
    \definition*{s.}{Sobrenome Dan}
    \definition{v.}{atrasar | (literário) abandonar-se a; entregar-se a}
  \end{Phonetics}
\end{Entry}

\begin{Entry}{耽心}{10,4}{⽿、⼼}
  \begin{Phonetics}{耽心}{dan1xin1}
    \variantof{担心}
  \end{Phonetics}
\end{Entry}

\begin{Entry}{耽误}{10,9}{⽿、⾔}
  \begin{Phonetics}{耽误}{dan1wu5}[][HSK 7-9]
    \definition{v.}{atrasar; segurar; perder algo devido a atraso ou oportunidade perdida; perder (oportunidade)}
  \end{Phonetics}
\end{Entry}

\begin{Entry}{耽搁}{10,12}{⽿、⼿}
  \begin{Phonetics}{耽搁}{dan1ge5}[][HSK 7-9]
    \definition{v.}{ficar; fazer uma parada | atrasar | perder (uma oportunidade, um prazo)}
  \end{Phonetics}
\end{Entry}

\begin{Entry}{耿}{10}{⽿}
  \begin{Phonetics}{耿}{geng3}
    \definition{adj.}{Literário: brilhante | honesto e justo; correto; íntegro | dedicado; leal}
    \definition{s.}{Sobrenome Geng}
  \end{Phonetics}
\end{Entry}

\begin{Entry}{耿直}{10,8}{⽿、⽬}
  \begin{Phonetics}{耿直}{geng3zhi2}[][HSK 7-9]
    \definition{adj.}{íntregro; franco; correto; honesto e franco}
  \end{Phonetics}
\end{Entry}

\begin{Entry}{聊}{11}{⽿}
  \begin{Phonetics}{聊}{liao2}[][HSK 6]
    \definition*{s.}{Sobrenome Liao}
    \definition{adv.}{apenas; meramente; provisoriamente; por enquanto | um pouco; ligeiramente}
    \definition{v.}{tagarelar; conversar; bater papo | confiar (ou depender, recorrer) a}
  \end{Phonetics}
\end{Entry}

\begin{Entry}{聊天}{11,4}{⽿、⼤}
  \begin{Phonetics}{聊天}{liao2/tian1}
    \definition{v.+compl.}{papear | bater papo}
  \end{Phonetics}
\end{Entry}

\begin{Entry}{聊天儿}{11,4,2}{⽿、⼤、⼉}
  \begin{Phonetics}{聊天儿}{liao2/tian1r5}[][HSK 6]
    \definition{v.+compl.}{conversar; fofocar; bater papo; duas ou mais pessoas conversando sem um tópico ou propósito específico}
  \end{Phonetics}
\end{Entry}

\begin{Entry}{职}{11}{⽿}
  \begin{Phonetics}{职}{zhi2}
    \definition*{s.}{Sobrenome Zhi}
    \definition{prep.}{para; devido a; por causa de}
    \definition{prep.}{(datado) Eu (em relatórios oficiais aos superiores)}
    \definition{s.}{dever; trabalho | cargo; posto; função; responsabilidades; posição}
    \definition{v.}{gerenciar; dirigir | administrar}
  \end{Phonetics}
\end{Entry}

\begin{Entry}{职工}{11,3}{⽿、⼯}
  \begin{Phonetics}{职工}{zhi2 gong1}[][HSK 3]
    \definition[个,位,名,些]{s.}{pessoal; trabalhadores e funcionários administrativos}
  \end{Phonetics}
\end{Entry}

\begin{Entry}{职业}{11,5}{⽿、⼀}
  \begin{Phonetics}{职业}{zhi2ye4}[][HSK 3]
    \definition{adj.}{profissional; não amador}
    \definition[种,份,个]{s.}{ocupação; profissão; vocação; o trabalho que um indivíduo realiza na sociedade como sua principal fonte de subsistência}
  \end{Phonetics}
\end{Entry}

\begin{Entry}{职务}{11,5}{⽿、⼒}
  \begin{Phonetics}{职务}{zhi2wu4}[][HSK 5]
    \definition{s.}{cargo; posto; deveres; função; funções que devem ser desempenhadas de acordo com as especificações do cargo}
  \end{Phonetics}
\end{Entry}

\begin{Entry}{职位}{11,7}{⽿、⼈}
  \begin{Phonetics}{职位}{zhi2wei4}[][HSK 5]
    \definition[个]{s.}{posto; posição; cargo que exerce determinadas funções em órgãos ou entidades}
  \end{Phonetics}
\end{Entry}

\begin{Entry}{职员}{11,7}{⽿、⼝}
  \begin{Phonetics}{职员}{zhi2yuan2}
    \definition[个,位]{s.}{empregado | trabalhador de escritório | membro da equipe}
  \end{Phonetics}
\end{Entry}

\begin{Entry}{职责}{11,8}{⽿、⾙}
  \begin{Phonetics}{职责}{zhi2 ze2}[][HSK 6]
    \definition[种]{s.}{dever; obrigação; responsabilidade; coisas que você deve fazer por causa de sua profissão ou identidade}
  \end{Phonetics}
\end{Entry}

\begin{Entry}{职能}{11,10}{⽿、⾁}
  \begin{Phonetics}{职能}{zhi2neng2}[][HSK 5]
    \definition[种,项]{s.}{função; funções ou papéis que as organizações, instituições, etc. devem desempenhar}
  \end{Phonetics}
\end{Entry}

\begin{Entry}{联}{12}{⽿}
  \begin{Phonetics}{联}{lian2}
    \definition{s.}{dísticos (antitéticos)}
    \definition{v.}{aliar-se a; unir-se; juntar-se a}
  \end{Phonetics}
\end{Entry}

\begin{Entry}{联手}{12,4}{⽿、⼿}
  \begin{Phonetics}{联手}{lian2 shou3}[][HSK 6]
    \definition{v.}{dar as mãos; cooperar | Literário: dar as mãos | agir em conjunto}
  \end{Phonetics}
\end{Entry}

\begin{Entry}{联合}{12,6}{⽿、⼝}
  \begin{Phonetics}{联合}{lian2he2}[][HSK 3]
    \definition{adj.}{conjunto; unido; federal; combinado}
    \definition{s.}{aliado; união; aliança; conectar-se ou unir-se para agir em conjunto}
  \end{Phonetics}
\end{Entry}

\begin{Entry}{联合会}{12,6,6}{⽿、⼝、⼈}
  \begin{Phonetics}{联合会}{lian2he2hui4}
    \definition{s.}{federação}
  \end{Phonetics}
\end{Entry}

\begin{Entry}{联合国}{12,6,8}{⽿、⼝、⼞}
  \begin{Phonetics}{联合国}{lian2 he2 guo2}[][HSK 3]
    \definition*{s.}{Nações Unidas; Organização internacional fundada em 1945, após o fim da Segunda Guerra Mundial, com sede em Nova Iorque, Estados Unidos ; as suas principais instituições são a Assembleia Geral, o Conselho de Segurança, o Conselho Econômico e Social e o Secretariado; de acordo com a Carta das Nações Unidas, os seus principais objetivos são manter a paz e a segurança internacionais, desenvolver relações amigáveis entre os países e promover a cooperação internacional nas áreas econômica e cultural}
  \end{Phonetics}
\end{Entry}

\begin{Entry}{联系}{12,7}{⽿、⽷}
  \begin{Phonetics}{联系}{lian2xi4}[][HSK 3]
    \definition[个,种,层]{s.}{relacionamento; relacionamento entre duas coisas}
    \definition{v.}{entrar em contato; contatar; comunicar-se com alguém por telefone, e-mail ou carta | agendar; entrar em contato com; estabelecer algum tipo de relação com a outra parte | relacionar; combinar; integrar}
  \end{Phonetics}
\end{Entry}

\begin{Entry}{联络}{12,9}{⽿、⽷}
  \begin{Phonetics}{联络}{lian2luo4}[][HSK 5]
    \definition{v.}{entrar em contato; comunicar-se; entrar em contato com}
  \end{Phonetics}
\end{Entry}

\begin{Entry}{联想}{12,13}{⽿、⼼}
  \begin{Phonetics}{联想}{lian2xiang3}[][HSK 5]
    \definition*{s.}{Lenovo (empresa)}
    \definition{v.}{associar-se a; estabelecer uma conexão mental; lembrar-se de algo; lembrar-se de outras pessoas ou coisas relacionadas devido a alguém ou algo; evocar outros conceitos relacionados devido a um determinado conceito}
  \end{Phonetics}
\end{Entry}

\begin{Entry}{联盟}{12,13}{⽿、⽫}
  \begin{Phonetics}{联盟}{lian2meng2}[][HSK 6]
    \definition{s.}{aliança; coalizão; liga; união}
  \end{Phonetics}
\end{Entry}

\begin{Entry}{联赛}{12,14}{⽿、⾙}
  \begin{Phonetics}{联赛}{lian2 sai4}[][HSK 6]
    \definition{s.}{jogos da liga | liga (esportiva) | torneio da liga}
  \end{Phonetics}
\end{Entry}

\begin{Entry}{聘}{13}{⽿}
  \begin{Phonetics}{聘}{pin4}
    \definition{v.}{contratar | noivar | (de uma menina) casar ou ser casada}
  \end{Phonetics}
\end{Entry}

\begin{Entry}{聘请}{13,10}{⽿、⾔}
  \begin{Phonetics}{聘请}{pin4 qing3}[][HSK 6]
    \definition{v.}{convidar; empregar; envolver; chamar; contratar alguém para assumir uma posição}
  \end{Phonetics}
\end{Entry}

\begin{Entry}{聚}{14}{⽿}
  \begin{Phonetics}{聚}{ju4}[][HSK 4]
    \definition*{s.}{Sobrenome Ju}
    \definition{v.}{reunir-se; juntar-se}
  \end{Phonetics}
\end{Entry}

\begin{Entry}{聚会}{14,6}{⽿、⼈}
  \begin{Phonetics}{聚会}{ju4hui4}[][HSK 4]
    \definition[个,次]{s.}{reunião; encontro; confraternização; festa}
    \definition{v.}{encontrar-se; reunir-se}
  \end{Phonetics}
\end{Entry}

\begin{Entry}{聚散}{14,12}{⽿、⽁}
  \begin{Phonetics}{聚散}{ju4san4}
    \definition{s.}{juntos e separados | agregação e dissipação}
  \end{Phonetics}
\end{Entry}

\begin{Entry}{聪}{15}{⽿}
  \begin{Phonetics}{聪}{cong1}
    \definition{adj.}{audição aguçada | brilhante; inteligente; esperto | perspicaz}
    \definition{s.}{(literário) faculdades auditivas}
  \end{Phonetics}
\end{Entry}

\begin{Entry}{聪明}{15,8}{⽿、⽇}
  \begin{Phonetics}{聪明}{cong1ming5}[][HSK 5]
    \definition{adj.}{brilhante; esperto; inteligente; intelecto bem desenvolvido com boa memória e capacidade de compreensão}
  \end{Phonetics}
\end{Entry}

\begin{Entry}{聪慧}{15,15}{⽿、⼼}
  \begin{Phonetics}{聪慧}{cong1hui4}
    \definition{adj.}{inteligente | brilhante}
  \end{Phonetics}
\end{Entry}

%%%%% EOF %%%%%

