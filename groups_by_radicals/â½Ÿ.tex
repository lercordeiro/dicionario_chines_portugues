%%%
%%% Radical "⽟"
%%%

\section*{Radical 96: ``⽟'' (王、玊)}\addcontentsline{toc}{section}{Radical 96: ⽟、王、玊}

\begin{entry}{王}{4}{⽟}
  \begin{phonetics}{王}{wang2}[][HSK 4]
    \definition*{s.}{Sobrenome Wang}
    \definition{adj.}{grande; ótimo; honoríficos antigos para avós}
    \definition{s.}{rei; monarca; imperador; governante supremo de uma monarquia | cabeça; chefe; líder | o primeiro, maior ou mais forte de seu tipo | duque; príncipe; o título mais alto da sociedade feudal após a dinastia Han}
  \end{phonetics}
  \begin{phonetics}{王}{wang4}
    \definition{v.}{reger; governar; reinar; dominar}
  \end{phonetics}
\end{entry}

\begin{entry}{王五}{4,4}{⽟、⼆}
  \begin{phonetics}{王五}{wang2wu3}
    \definition{s.}{Wang Wu | Zé Ninguém | nome para uma pessoa não especificada, 3 de 3}
  \seealsoref{李四}{li3si4}
  \seealsoref{张三}{zhang1san1}
  \end{phonetics}
\end{entry}

\begin{entry}{王朝}{4,12}{⽟、⽉}
  \begin{phonetics}{王朝}{wang2chao2}
    \definition{s.}{dinastia}
  \end{phonetics}
\end{entry}

\begin{entry}{玉}{5}{⽟}[Kangxi 96]
  \begin{phonetics}{玉}{yu4}[][HSK 4]
    \definition*{s.}{Sobrenome Yu}
    \definition{adj.}{(pessoa, especialmente uma mulher) pura; justa; bonita; bela | cristalino, branco e belo como o jade | (vida) rica; luxuosa}
    \definition{pron.}{seu; um termo de respeito, usado para honrar o corpo, as ações ou as coisas associadas à outra pessoa}
    \definition[块]{s.}{jade}
  \end{phonetics}
\end{entry}

\begin{entry}{玉米}{5,6}{⽟、⽶}
  \begin{phonetics}{玉米}{yu4mi3}[][HSK 4]
    \definition[个,株,粒]{s.}{milho}
  \end{phonetics}
\end{entry}

\begin{entry}{玉米片}{5,6,4}{⽟、⽶、⽚}
  \begin{phonetics}{玉米片}{yu4mi3pian4}
    \definition{s.}{flocos de milho | chips de tortilha}
  \end{phonetics}
\end{entry}

\begin{entry}{玉米花}{5,6,7}{⽟、⽶、⾋}
  \begin{phonetics}{玉米花}{yu4mi3hua1}
    \definition{s.}{pipoca}
  \end{phonetics}
\end{entry}

\begin{entry}{玉米面}{5,6,9}{⽟、⽶、⾯}
  \begin{phonetics}{玉米面}{yu4mi3mian4}
    \definition{s.}{fubá | farinha de milho}
  \end{phonetics}
\end{entry}

\begin{entry}{玉米饼}{5,6,9}{⽟、⽶、⾷}
  \begin{phonetics}{玉米饼}{yu4mi3bing3}
    \definition{s.}{tortilha mexicana | bolo de milho}
  \end{phonetics}
\end{entry}

\begin{entry}{玉米笋}{5,6,10}{⽟、⽶、⽵}
  \begin{phonetics}{玉米笋}{yu4mi3sun3}
    \definition{s.}{broto de milho}
  \end{phonetics}
\end{entry}

\begin{entry}{玉米粉}{5,6,10}{⽟、⽶、⽶}
  \begin{phonetics}{玉米粉}{yu4mi3fen3}
    \definition{s.}{amido de milho | farinha de milho}
  \end{phonetics}
\end{entry}

\begin{entry}{玉米糁}{5,6,14}{⽟、⽶、⽶}
  \begin{phonetics}{玉米糁}{yu4mi3san3}
    \definition{s.}{grãos de milho}
  \end{phonetics}
\end{entry}

\begin{entry}{玉米糕}{5,6,16}{⽟、⽶、⽶}
  \begin{phonetics}{玉米糕}{yu4mi3gao1}
    \definition{s.}{bolo de milho | polenta}
  \end{phonetics}
\end{entry}

\begin{entry}{玩}{8}{⽟}
  \begin{phonetics}{玩}{wan2}
    \definition{s.}{brinquedo | algo usado para diversão}
    \definition{v.}{divertir-se | manter algo para entretenimento | brincar com}
  \end{phonetics}
\end{entry}

\begin{entry}{玩儿}{8,2}{⽟、⼉}
  \begin{phonetics}{玩儿}{wan2r5}[][HSK 1]
    \definition{v.}{divertir-se; (entretenimento) relaxar ou experimentar alguma atividade}
  \end{phonetics}
\end{entry}

\begin{entry}{玩艺}{8,4}{⽟、⾋}
  \begin{phonetics}{玩艺}{wan2yi4}
    \variantof{玩意}
  \end{phonetics}
\end{entry}

\begin{entry}{玩伴}{8,7}{⽟、⼈}
  \begin{phonetics}{玩伴}{wan2ban4}
    \definition{s.}{parceiro de brincadeira}
  \end{phonetics}
\end{entry}

\begin{entry}{玩具}{8,8}{⽟、⼋}
  \begin{phonetics}{玩具}{wan2ju4}[][HSK 3]
    \definition[个,件,套]{s.}{brinquedo; coisas para brincar}
  \end{phonetics}
\end{entry}

\begin{entry}{玩具厂}{8,8,2}{⽟、⼋、⼚}
  \begin{phonetics}{玩具厂}{wan2ju4chang3}
    \definition{s.}{fábrica de brinquedos}
  \end{phonetics}
\end{entry}

\begin{entry}{玩具车}{8,8,4}{⽟、⼋、⾞}
  \begin{phonetics}{玩具车}{wan2ju4 che1}
    \definition{s.}{carrinho de brinquedo}
  \end{phonetics}
\end{entry}

\begin{entry}{玩味}{8,8}{⽟、⼝}
  \begin{phonetics}{玩味}{wan2wei4}
    \definition{v.}{ponderar sutilezas | ruminar (pensamentos)}
  \end{phonetics}
\end{entry}

\begin{entry}{玩者}{8,8}{⽟、⽼}
  \begin{phonetics}{玩者}{wan2zhe3}
    \definition{s.}{jogador}
  \end{phonetics}
\end{entry}

\begin{entry}{玩耍}{8,9}{⽟、⽽}
  \begin{phonetics}{玩耍}{wan2shua3}
    \definition{v.}{divertir-me | brincar (como as crianças fazem)}
  \end{phonetics}
\end{entry}

\begin{entry}{玩家}{8,10}{⽟、⼧}
  \begin{phonetics}{玩家}{wan2jia1}
    \definition{s.}{entusiasta (áudio, modelos de aviões, etc.) | jogador (de um jogo)}
  \end{phonetics}
\end{entry}

\begin{entry}{玩偶}{8,11}{⽟、⼈}
  \begin{phonetics}{玩偶}{wan2'ou3}
    \definition{s.}{estatueta de brinquedo | boneco de ação | bicho de pelúcia | boneca}
  \end{phonetics}
\end{entry}

\begin{entry}{玩遍}{8,12}{⽟、⾡}
  \begin{phonetics}{玩遍}{wan2bian4}
    \definition{v.}{passear (todo o país, toda a cidade, etc.) | visitar (um grande número de lugares)}
  \end{phonetics}
\end{entry}

\begin{entry}{玩意}{8,13}{⽟、⼼}
  \begin{phonetics}{玩意}{wan2yi4}
    \definition{s.}{ato | brinquedo | coisa | truque (em uma performance, show de palco, acrobacias, etc.)}
  \end{phonetics}
\end{entry}

\begin{entry}{环}{8}{⽟}
  \begin{phonetics}{环}{huan2}[][HSK 3]
    \definition*{s.}{Sobrenome Huan}
    \definition{clas.}{usado para anéis}
    \definition[个,串]{s.}{anel; arco | elo; \emph{link}; passo; etapa | anel; objeto em forma de círculo | arredores}
    \definition{v.}{cercar; rodear; circular; circundar}
  \end{phonetics}
\end{entry}

\begin{entry}{环卫}{8,3}{⽟、⼙}
  \begin{phonetics}{环卫}{huan2wei4}
    \definition{s.}{limpeza pública | saneamento urbano | saneamento ambiental | abreviação de 环境卫生}
  \seealsoref{环境卫生}{huan2jing4wei4sheng1}
  \end{phonetics}
\end{entry}

\begin{entry}{环节}{8,5}{⽟、⾋}
  \begin{phonetics}{环节}{huan2jie2}[][HSK 5]
    \definition{s.}{\emph{link}; ligação; vínculo; uma das muitas coisas que estão inter-relacionadas | segmento; estrutura anelar de alguns animais inferiores}
  \end{phonetics}
\end{entry}

\begin{entry}{环保}{8,9}{⽟、⼈}
  \begin{phonetics}{环保}{huan2 bao3}[][HSK 3]
    \definition{adj.}{ecológico; benefício para o meio ambiente; não prejudica o meio ambiente}
    \definition{s.}{proteção ambiental}
  \end{phonetics}
\end{entry}

\begin{entry}{环境}{8,14}{⽟、⼟}
  \begin{phonetics}{环境}{huan2jing4}[][HSK 3]
    \definition[个]{s.}{ambiente; os arredores | arredores; circunstâncias; condições políticas, econômicas, culturais, etc., dentro de um determinado âmbito}
  \end{phonetics}
\end{entry}

\begin{entry}{环境卫生}{8,14,3,5}{⽟、⼟、⼙、⽣}
  \begin{phonetics}{环境卫生}{huan2jing4wei4sheng1}
    \definition{s.}{saneamento ambiental}
  \seealsoref{环卫}{huan2wei4}
  \end{phonetics}
\end{entry}

\begin{entry}{玻}{9}{⽟}
  \begin{phonetics}{玻}{bo1}
    \definition{s.}{vidro}
  \end{phonetics}
\end{entry}

\begin{entry}{玻璃}{9,14}{⽟、⽟}
  \begin{phonetics}{玻璃}{bo1li5}[][HSK 5]
    \definition[张,块]{s.}{vidro; corpo duro, quebradiço e transparente, geralmente feito de areia, calcário, carbonato de sódio, etc. | \emph{nylon}; plástico; refere-se a determinados plásticos que se assemelham ao vidro.}
  \end{phonetics}
\end{entry}

\begin{entry}{珍}{9}{⽟}
  \begin{phonetics}{珍}{zhen1}
    \definition{adj.}{precioso; valioso; raro | inestimável}
    \definition{s.}{tesouro | objetos de valor}
    \definition{v.}{valorizar muito; estimar}
  \end{phonetics}
\end{entry}

\begin{entry}{珍贵}{9,9}{⽟、⾙}
  \begin{phonetics}{珍贵}{zhen1gui4}[][HSK 5]
    \definition{adj.}{raro; valioso; precioso; de grande valor; profundo significado}
  \end{phonetics}
\end{entry}

\begin{entry}{珍珠}{9,10}{⽟、⽟}
  \begin{phonetics}{珍珠}{zhen1zhu1}[][HSK 5]
    \definition[颗,串]{s.}{pérola; grânulos redondos produzidos nas conchas de certos animais aquáticos, de cor branca, rosa, etc., bonitos e brilhantes, frequentemente usados como adornos}
  \end{phonetics}
\end{entry}

\begin{entry}{珍惜}{9,11}{⽟、⼼}
  \begin{phonetics}{珍惜}{zhen1xi1}[][HSK 5]
    \definition{v.}{valorizar; estimar; valorizar e evitar o desperdício}
  \end{phonetics}
\end{entry}

\begin{entry}{珠}{10}{⽟}
  \begin{phonetics}{珠}{zhu1}
    \definition[粒,颗]{s.}{pérola | conta (de colar, ábaco, etc.) | coisa parecida com uma bola (como um globo ocular)}
  \end{phonetics}
\end{entry}

\begin{entry}{珠子}{10,3}{⽟、⼦}
  \begin{phonetics}{珠子}{zhu1zi5}
    \definition[粒,颗]{s.}{pérola | contas}
  \end{phonetics}
\end{entry}

\begin{entry}{班}{10}{⽟}
  \begin{phonetics}{班}{ban1}[][HSK 1]
    \definition*{s.}{Sobrenome Ban}
    \definition{adj.}{regular; programado; executado regularmente; com horários fixos (meios de transporte)}
    \definition{clas.}{um grupo de; uma classe de; usado para pessoas | meios de transporte com horários fixos}
    \definition[个]{s.}{equipe; turma; organização estruturada | dever; turno; período de trabalho dentro de um dia | equipe; esquadrão; unidade básica das forças armadas | nome usado antigamente para designar uma companhia teatral}
    \definition{v.}{mover; implantar; implementar}
  \end{phonetics}
\end{entry}

\begin{entry}{班长}{10,4}{⽟、⾧}
  \begin{phonetics}{班长}{ban1 zhang3}[][HSK 2]
    \definition[个,位,名]{s.}{monitor de turma; líder de equipe; alunos responsáveis nas turmas da escola | líder de esquadrão; responsável por uma turma de soldados, geralmente com patente de sargento}
  \end{phonetics}
\end{entry}

\begin{entry}{班级}{10,6}{⽟、⽷}
  \begin{phonetics}{班级}{ban1 ji2}[][HSK 3]
    \definition[个]{s.}{classe; série (na escola); o nome geral para as séries e turmas da escola}
  \end{phonetics}
\end{entry}

\begin{entry}{球}{11}{⽟}
  \begin{phonetics}{球}{qiu2}[][HSK 1]
    \definition[个,颗,筐]{s.}{esfera; globo; equipamento de jogo antigo, objeto tridimensional circular, feito de couro, recheado com penas, para ser chutado com os pés ou batido com um bastão | qualquer coisa com formato de bola; algo esférico ou quase esférico | bola; refere-se a certos artigos esportivos (geralmente redondos e tridimensionais) | jogo; partida; referência a esportes com bola | o Globo; a Terra; referindo-se especificamente à Terra}
  \end{phonetics}
\end{entry}

\begin{entry}{球队}{11,4}{⽟、⾩}
  \begin{phonetics}{球队}{qiu2 dui4}[][HSK 2]
    \definition[个,支]{s.}{equipe (basquete, futebol, etc.); equipe de atletas formada para competições esportivas com bola, como times de basquete, futebol, etc.}
  \end{phonetics}
\end{entry}

\begin{entry}{球场}{11,6}{⽟、⼟}
  \begin{phonetics}{球场}{qiu2 chang3}[][HSK 2]
    \definition[个,座]{s.}{quadra; campo; terreno para jogos com bola; campos para a prática de esportes com bola, como basquete, futebol, tênis e vôlei, cuja forma, tamanho e equipamentos variam de acordo com as exigências de cada esporte}
  \end{phonetics}
\end{entry}

\begin{entry}{球拍}{11,8}{⽟、⼿}
  \begin{phonetics}{球拍}{qiu2pai1}
    \definition{s.}{raquete}
  \end{phonetics}
\end{entry}

\begin{entry}{球迷}{11,9}{⽟、⾡}
  \begin{phonetics}{球迷}{qiu2mi2}[][HSK 3]
    \definition[个,位,名,些]{s.}{fã (de esportes de bola); pessoas obcecadas por jogar ou assistir jogos de bola}
  \end{phonetics}
\end{entry}

\begin{entry}{球鞋}{11,15}{⽟、⾰}
  \begin{phonetics}{球鞋}{qiu2 xie2}[][HSK 2]
    \definition[双,只,款]{s.}{tênis de ginástica; tênis de tênis; tênis esportivos}
  \end{phonetics}
\end{entry}

\begin{entry}{理}{11}{⽟}
  \begin{phonetics}{理}{li3}[][HSK 6]
    \definition*{s.}{Sobrenome Li}
    \definition{s.}{textura; grão (em madeira, pele, etc.) | ordem; sequência | razão; lógica; verdade | ciências naturais (especialmente física)}
    \definition{v.}{gerenciar; executar | colocar em ordem; arrumar | (geralmente no negativo) prestar atenção a; fazer um gesto ou falar com | tratar | colocar em ordem; limpar | tomar conhecimento de; prestar atenção a; expressar uma atitude; expressar uma opinião}
  \end{phonetics}
\end{entry}

\begin{entry}{理发}{11,5}{⽟、⼜}
  \begin{phonetics}{理发}{li3fa4}[][HSK 3]
    \definition{v.+compl.}{cortar e aparar o cabelo; ter (dar) um corte de cabelo}
  \end{phonetics}
\end{entry}

\begin{entry}{理由}{11,5}{⽟、⽥}
  \begin{phonetics}{理由}{li3you2}[][HSK 3]
    \definition[个,条,种,堆]{s.}{razão; justificativa; fundamento; a razão pela qual as coisas são feitas desta ou daquela maneira}
  \end{phonetics}
\end{entry}

\begin{entry}{理论}{11,6}{⽟、⾔}
  \begin{phonetics}{理论}{li3lun4}[][HSK 3]
    \definition[套,个]{s.}{teoria; uma série de conclusões tiradas pelas pessoas sobre atividades naturais ou sociais}
    \definition{v.}{argumentar; raciocinar com alguém; discutir com outras pessoas sobre quem está certo ou errado}
  \end{phonetics}
\end{entry}

\begin{entry}{理想}{11,13}{⽟、⼼}
  \begin{phonetics}{理想}{li3xiang3}[][HSK 2]
    \definition{adj.}{ideal; perfeito | conforme o desejado; satisfatório}
    \definition{adv.}{idealmente}
    \definition[个,种]{s.}{ideal; sonho; aspiração}
  \end{phonetics}
\end{entry}

\begin{entry}{理解}{11,13}{⽟、⾓}
  \begin{phonetics}{理解}{li3jie3}[][HSK 3]
    \definition{v.}{entender; compreender; compreender o significado por trás de algo através da reflexão e do aprendizado | entender com empatia; achar que os outros não conseguem fazer determinada coisa e demonstrar compaixão, perdão e não crítica}
  \end{phonetics}
\end{entry}

\begin{entry}{琴}{12}{⽟}
  \begin{phonetics}{琴}{qin2}[][HSK 5]
    \definition*{s.}{Sobrenome Qin}
    \definition[架,台]{s.}{cítara; qin; guqin (um instrumento de cordas dedilhadas com sete cordas, em alguns aspectos semelhante à cítara)  | nome genérico para certos instrumentos musicais}
  \end{phonetics}
\end{entry}

\begin{entry}{琴键}{12,13}{⽟、⾦}
  \begin{phonetics}{琴键}{qin2jian4}
    \definition{s.}{tecla de piano}
  \end{phonetics}
\end{entry}

\begin{entry}{瑜}{13}{⽟}
  \begin{phonetics}{瑜}{yu2}
    \definition{s.}{(arcaico) jade fino; gema | (literário) brilho das gemas — virtudes; pontos positivos | excelência}
  \end{phonetics}
\end{entry}

\begin{entry}{瑜伽}{13,7}{⽟、⼈}
  \begin{phonetics}{瑜伽}{yu2jia1}
    \definition*{s.}{Ioga}
  \end{phonetics}
\end{entry}

\begin{entry}{瑜珈}{13,9}{⽟、⽟}
  \begin{phonetics}{瑜珈}{yu2jia1}
    \variantof{瑜伽}
  \end{phonetics}
\end{entry}

%%%%% EOF %%%%%

