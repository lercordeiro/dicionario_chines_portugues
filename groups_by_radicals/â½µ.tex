%%%
%%% Radical "⽵"
%%%

\section*{Radical 118: ``⽵'' (⺮)}\addcontentsline{toc}{section}{Radical 118: ⽵、⺮}

\begin{Entry}{竹}{6}{⽵}[Kangxi 118]
  \begin{Phonetics}{竹}{zhu2}
    \definition[根]{s.}{bambu | instrumento de sopro | tira de bambu}
  \end{Phonetics}
\end{Entry}

\begin{Entry}{竹子}{6,3}{⽵、⼦}
  \begin{Phonetics}{竹子}{zhu2zi5}[][HSK 5]
    \definition[根,棵,丛,支]{s.}{bambu; nome genérico para os tipos de bambu}
  \end{Phonetics}
\end{Entry}

\begin{Entry}{竹马}{6,3}{⽵、⾺}
  \begin{Phonetics}{竹马}{zhu2ma3}
    \definition{s.}{cavalo de bambu | vara de bambu usada como cavalo de brinquedo}
  \end{Phonetics}
\end{Entry}

\begin{Entry}{竹排}{6,11}{⽵、⼿}
  \begin{Phonetics}{竹排}{zhu2pai2}
    \definition{s.}{jangada de bambu}
  \end{Phonetics}
\end{Entry}

\begin{Entry}{竹编}{6,12}{⽵、⽷}
  \begin{Phonetics}{竹编}{zhu2bian1}
    \definition{s.}{vime | tecelagem de bambu}
  \end{Phonetics}
\end{Entry}

\begin{Entry}{笋}{10}{⽵}
  \begin{Phonetics}{笋}{sun3}
    \definition{s.}{broto de bambu}
  \end{Phonetics}
\end{Entry}

\begin{Entry}{笑}{10}{⽵}
  \begin{Phonetics}{笑}{xiao4}[][HSK 1]
    \definition{adj.}{ridículo; engraçado; risível; hilário}
    \definition{v.}{sorrir; rir; mostrar expressão de alegria; emitir sons de alegria | ridicularizar; rir de; zombar}
  \end{Phonetics}
\end{Entry}

\begin{Entry}{笑声}{10,7}{⽵、⼠}
  \begin{Phonetics}{笑声}{xiao4 sheng1}[][HSK 6]
    \definition{s.}{riso; risada}
  \end{Phonetics}
\end{Entry}

\begin{Entry}{笑话}{10,8}{⽵、⾔}
  \begin{Phonetics}{笑话}{xiao4hua5}[][HSK 2]
    \definition[个]{s.}{piada; brincadeira; uma conversa ou história que faz as pessoas rirem; algo que as pessoas usam como piada}
    \definition{v.}{ridicularizar; zombar; rir de;}
  \end{Phonetics}
\end{Entry}

\begin{Entry}{笑话儿}{10,8,2}{⽵、⾔、⼉}
  \begin{Phonetics}{笑话儿}{xiao4 hua4r5}[][HSK 2]
    \definition{s.}{piada; brincadeira; gracejo}
  \end{Phonetics}
\end{Entry}

\begin{Entry}{笑容}{10,10}{⽵、⼧}
  \begin{Phonetics}{笑容}{xiao4 rong2}[][HSK 6]
    \definition[丝,抹,个]{s.}{sorriso; expressão sorridente; o olhar no rosto de alguém ao sorrir}
  \end{Phonetics}
\end{Entry}

\begin{Entry}{笑脸}{10,11}{⽵、⾁}
  \begin{Phonetics}{笑脸}{xiao4 lian3}[][HSK 6]
    \definition{s.}{\emph{smiley}; rosto sorridente (emoji)}
  \end{Phonetics}
\end{Entry}

\begin{Entry}{笔}{10}{⽵}
  \begin{Phonetics}{笔}{bi3}[][HSK 2]
    \definition{clas.}{usado para grandes quantias de dinheiro, compras, negócios, propriedades, etc. | usado em caligrafia e pintura, etc.}
    \definition[支,枝]{s.}{caneta; lápis; pincel para escrever; ferramentas para escrever ou desenhar | técnica de escrita; caligrafia ou desenho | traço}
    \definition{v.}{escrever à mão}
  \end{Phonetics}
\end{Entry}

\begin{Entry}{笔记}{10,5}{⽵、⾔}
  \begin{Phonetics}{笔记}{bi3 ji4}[][HSK 2]
    \definition[篇,本,个]{s.}{notas; anotações feitas durante aulas, palestras e leituras | ensaios; esboços}
    \definition{v.}{tomar nota (por escrito)}
  \end{Phonetics}
\end{Entry}

\begin{Entry}{笔记本}{10,5,5}{⽵、⾔、⽊}
  \begin{Phonetics}{笔记本}{bi3ji4ben3}[][HSK 2]
    \definition[个,本]{s.}{caderno para anotações | \emph{laptop}; refere-se a um computador portátil}
    \definition{s.}{\emph{laptop}}
  \end{Phonetics}
\end{Entry}

\begin{Entry}{笔试}{10,8}{⽵、⾔}
  \begin{Phonetics}{笔试}{bi3 shi4}[][HSK 6]
    \definition{s.}{exame escrito; um tipo de exame que exige respostas escritas; diferente de 口试}
  \seealsoref{口试}{kou3 shi4}
  \end{Phonetics}
\end{Entry}

\begin{Entry}{笛}{11}{⽵}
  \begin{Phonetics}{笛}{di2}
    \definition[只]{s.}{flauta de bambu | sirene; apito; buzina}
  \end{Phonetics}
\end{Entry}

\begin{Entry}{符}{11}{⽵}
  \begin{Phonetics}{符}{fu2}
    \definition*{s.}{Sobrenome Fu}
    \definition[个]{s.}{registro emitido por um governante para generais, enviados, etc., como credenciais na China antiga | símbolo; emblema | figuras mágicas desenhadas por sacerdotes taoístas para invocar ou expulsar espíritos e trazer boa ou má sorte | marca; sinal}
    \definition{v.}{(usado com 相 xiāng ou 不) coincidir com; concordar com | encaixar bem; combinar com; em conformidade com}
  \seealsoref{不}{bu4}
  \seealsoref{相}{xiang1}
  \end{Phonetics}
\end{Entry}

\begin{Entry}{符号}{11,5}{⽵、⼝}
  \begin{Phonetics}{符号}{fu2hao4}[][HSK 4]
    \definition[个]{s.}{marca; símbolo; sinais que marcam as coisas | insígnia; emblema; um símbolo usado no corpo para indicar posição, \emph{status}, etc.}
  \end{Phonetics}
\end{Entry}

\begin{Entry}{符合}{11,6}{⽵、⼝}
  \begin{Phonetics}{符合}{fu2he2}[][HSK 4]
    \definition{v.}{conformar-se com, estar de acordo com, estar em conformidade com}
  \end{Phonetics}
\end{Entry}

\begin{Entry}{笨}{11}{⽵}
  \begin{Phonetics}{笨}{ben4}[][HSK 4]
    \definition{adj.}{estúpido; sem graça; tolo; de pouca habilidade; sem inteligência | desajeitado; tosco; inflexível | incômodo; pesado; desajeitado; difícil de manejar; trabalhoso}
  \end{Phonetics}
\end{Entry}

\begin{Entry}{笨重}{11,9}{⽵、⾥}
  \begin{Phonetics}{笨重}{ben4zhong4}[][HSK 7-9]
    \definition{adj.}{pesado; desajeitado; incômodo; grande e pesado, inconveniente de usar | pesado; difícil de manejar; pesado e trabalhoso}
  \end{Phonetics}
\end{Entry}

\begin{Entry}{笨蛋}{11,11}{⽵、⾍}
  \begin{Phonetics}{笨蛋}{ben4dan4}[][HSK 7-9]
    \definition[个]{s.}{tolo; idiota; (depreciativo) refere-se a uma pessoa muito estúpida ou sem cérebro; geralmente usado para insultar pessoas}
  \end{Phonetics}
\end{Entry}

\begin{Entry}{第}{11}{⽵}
  \begin{Phonetics}{第}{di4}[][HSK 1]
    \definition*{s.}{Sobrenome Di}
    \definition{adv.}{mas, apenas, somente; Indica que a ação não está sujeita a restrições ou condições; equivalente a 只管}
    \definition{conj.}{mas; contudo; orações de conexão; indicando uma relação de transição; equivalente a 但是}
    \definition{pref.}{palavra auxiliar para números ordinais; usado antes de números inteiros, indica ordem}
    \definition{s.}{diferentes notas dos candidatos aprovados nos exames imperiais | a residência de um alto funcionário; grandes residências dos burocratas da era feudal}
  \seealsoref{但是}{dan4 shi4}
  \seealsoref{只管}{zhi3 guan3}
  \end{Phonetics}
\end{Entry}

\begin{Entry}{笼}{11}{⽵}
  \begin{Phonetics}{笼}{long2}
    \definition{s.}{armação fechada de bambu, arame, etc. | jaula | gaiola}
  \end{Phonetics}
  \begin{Phonetics}{笼}{long3}
    \definition{v.}{envolver | cobrir}
  \end{Phonetics}
\end{Entry}

\begin{Entry}{笼子}{11,3}{⽵、⼦}
  \begin{Phonetics}{笼子}{long2zi5}
    \definition{s.}{jaula | cesta | gaiola | recipiente}
  \end{Phonetics}
  \begin{Phonetics}{笼子}{long3zi5}
    \definition{s.}{caixa grande | porta-malas}
  \end{Phonetics}
\end{Entry}

\begin{Entry}{等}{12}{⽵}
  \begin{Phonetics}{等}{deng3}[][HSK 1,2]
    \definition*{s.}{Sobrenome Deng}
    \definition{adj.}{igual; na mesma medida ou quantidade}
    \definition{clas.}{usado para classe, grau, classificação | usado para tipo}
    \definition{part.}{e assim por diante; etc.; indica que a enumeração não está completa (pode ser usada repetidamente) | indica o fim de uma enumeração; após a enumeração, é usado para encerrar; geralmente é seguido pelo total dos itens anteriores}
    \definition{pron.}{usado após pronomes pessoais ou substantivos que se referem a pessoas; indica plural}
    \definition{s.}{classe; série; posição | equilíbrio; balança para pesar pequenas quantidades de objetos valiosos e ervas medicinais; atualmente, geralmente escrita como 戥}
    \definition{v.}{esperar; aguardar | esperar até}
  \end{Phonetics}
\end{Entry}

\begin{Entry}{等于}{12,3}{⽵、⼆}
  \begin{Phonetics}{等于}{deng3yu2}[][HSK 2]
    \definition{adv.}{igual a | equivalente a}
    \definition{v.}{equivaler a; ser equivalente a; ser quase igual a; não ter diferença}
  \end{Phonetics}
\end{Entry}

\begin{Entry}{等级}{12,6}{⽵、⽷}
  \begin{Phonetics}{等级}{deng3ji2}[][HSK 5]
    \definition[个]{s.}{grau; classificação; posição; distinções por qualidade, grau, status, etc. | estado social; estrato social; ordem e grau; grupos sociais desiguais em termos de status social e legal}
  \end{Phonetics}
\end{Entry}

\begin{Entry}{等到}{12,8}{⽵、⼑}
  \begin{Phonetics}{等到}{deng3 dao4}[][HSK 2]
    \definition{prep.}{na hora; quando; expressão de condições temporais | esperar até; aguardar até}
  \end{Phonetics}
\end{Entry}

\begin{Entry}{等待}{12,9}{⽵、⼻}
  \begin{Phonetics}{等待}{deng3dai4}[][HSK 3]
    \definition{v.}{esperar; aguardar; não agir até que a pessoa, coisa ou situação desejada apareça}
  \end{Phonetics}
\end{Entry}

\begin{Entry}{等候}{12,10}{⽵、⼈}
  \begin{Phonetics}{等候}{deng3hou4}[][HSK 5]
    \definition{v.}{esperar; aguardar; expectar; usado principalmente para objetos específicos}
  \end{Phonetics}
\end{Entry}

\begin{Entry}{等等}{12,12}{⽵、⽵}
  \begin{Phonetics}{等等}{deng3 deng3}
    \definition{part.}{etc.; e assim por diante; usada depois de duas ou mais palavras paralelas para indicar que a lista não está completa}
  \end{Phonetics}
\end{Entry}

\begin{Entry}{筏}{12}{⽵}
  \begin{Phonetics}{筏}{fa2}
    \definition[条]{s.}{jangada (de troncos, bambus, etc.)}
  \end{Phonetics}
\end{Entry}

\begin{Entry}{筒}{12}{⽵}
  \begin{Phonetics}{筒}{tong3}
    \definition[个]{s.}{seção de bambu grosso; tubo grosso de bambu | objeto em forma de tubo largo | a parte em forma de tubo das roupas etc.}
  \end{Phonetics}
\end{Entry}

\begin{Entry}{答}{12}{⽵}
  \begin{Phonetics}{答}{da1}[][HSK 5]
    \definition{v.}{concordar; responder | responder; prestar atenção}
  \end{Phonetics}
  \begin{Phonetics}{答}{da2}[][HSK 5]
    \definition{v.}{responder; dar resposta a; responder a | retribuir; devolver (uma visita, etc.); retribuir um favor feito a alguém por outro; fazer o bem}
  \end{Phonetics}
\end{Entry}

\begin{Entry}{答应}{12,7}{⽵、⼴}
  \begin{Phonetics}{答应}{da1ying5}[][HSK 2]
    \definition{v.}{responder; retribuir; reagir; retrucar | concordar; prometer; cumprir}
  \end{Phonetics}
\end{Entry}

\begin{Entry}{答复}{12,9}{⽵、⼢}
  \begin{Phonetics}{答复}{da2fu4}[][HSK 5]
    \definition[个]{s.}{resposta; respostas a perguntas ou solicitações}
    \definition{v.}{responder; dar uma resposta}
  \end{Phonetics}
\end{Entry}

\begin{Entry}{答案}{12,10}{⽵、⽊}
  \begin{Phonetics}{答案}{da2'an4}[][HSK 4]
    \definition[个,条,种,些]{s.}{chave; resposta; solução}
  \end{Phonetics}
\end{Entry}

\begin{Entry}{答辩}{12,16}{⽵、⾟}
  \begin{Phonetics}{答辩}{da2bian4}[][HSK 7-9]
    \definition{v.}{responder a perguntas, acusações, etc. de outras pessoas; defender as próprias opiniões ou ações}
  \end{Phonetics}
\end{Entry}

\begin{Entry}{策}{12}{⽵}
  \begin{Phonetics}{策}{ce4}
    \definition*{s.}{Sobrenome Ce}
    \definition[个,项,根]{s.}{plano; esquema | tiras de bambu ou madeira usadas para escrever na China antiga | questões sobre atualidades definidas para os exames imperiais | chicote de montaria antigo | um tipo de ensaio na China antiga; um estilo de escrita para exames antigos | estratégia; método}
    \definition{v.}{chicotear (um cavalo) com um chicote de montaria | incitar com um chicote de cavalo, espora}
  \end{Phonetics}
\end{Entry}

\begin{Entry}{策划}{12,6}{⽵、⼑}
  \begin{Phonetics}{策划}{ce4hua4}[][HSK 6]
    \definition{v.}{planejar; traçar; esquematizar; pensar repetidamente para elaborar um plano}
  \end{Phonetics}
\end{Entry}

\begin{Entry}{策略}{12,11}{⽵、⽥}
  \begin{Phonetics}{策略}{ce4lve4}[][HSK 6]
    \definition{adj.}{diplomático; (métodos) flexíveis sem sacrificar princípios}
    \definition[种,个,条,套]{s.}{tática; estratégia; política; para atingir determinadas tarefas estratégicas, o curso de ação e os métodos de luta são formulados de acordo com o desenvolvimento da situação}
  \end{Phonetics}
\end{Entry}

\begin{Entry}{筷}{13}{⽵}
  \begin{Phonetics}{筷}{kuai4}
    \definition[双,根,个]{s.}{pauzinhos para comer}
  \end{Phonetics}
\end{Entry}

\begin{Entry}{筷子}{13,3}{⽵、⼦}
  \begin{Phonetics}{筷子}{kuai4zi5}[][HSK 2]
    \definition[根,双,副,把,对]{s.}{pauzinhos; \emph{chopsticks}; dois bastôes finos feitos de bambu, madeira, metal ou outro material, usados para segurar comida ou outros objetos}
  \end{Phonetics}
\end{Entry}

\begin{Entry}{筹}{13}{⽵}
  \begin{Phonetics}{筹}{chou2}[][HSK 7-9]
    \definition{clas.}{usado para pessoas, visto principalmente no vernáculo antigo}
    \definition{s.}{estratégia; estratagema; meios; método | ficha; contador; um pequeno pedaço de bambu ou madeira; frequentemente usado para contagem ou como um voucher}
    \definition{v.}{preparar; planejar; levantar}
  \end{Phonetics}
\end{Entry}

\begin{Entry}{筹办}{13,4}{⽵、⼒}
  \begin{Phonetics}{筹办}{chou2ban4}[][HSK 7-9]
    \definition{v.}{fazer preparativos; fazer arranjos}
  \end{Phonetics}
\end{Entry}

\begin{Entry}{筹划}{13,6}{⽵、⼑}
  \begin{Phonetics}{筹划}{chou2hua4}[][HSK 7-9]
    \definition{v.}{planejar e preparar; encontrar um caminho; fazer um plano}
  \end{Phonetics}
\end{Entry}

\begin{Entry}{筹备}{13,8}{⽵、⼡}
  \begin{Phonetics}{筹备}{chou2bei4}[][HSK 7-9]
    \definition{v.}{preparar; organizar; planejar}
  \end{Phonetics}
\end{Entry}

\begin{Entry}{筹码}{13,8}{⽵、⽯}
  \begin{Phonetics}{筹码}{chou2ma3}[][HSK 7-9]
    \definition{s.}{ficha; contador (usado para cálculos, frequentemente em jogos de azar como substituto de moeda)}
  \end{Phonetics}
\end{Entry}

\begin{Entry}{筹措}{13,11}{⽵、⼿}
  \begin{Phonetics}{筹措}{chou2cuo4}[][HSK 7-9]
    \definition{v.}{arrecadar (dinheiro) | coletar (fundos)}
  \end{Phonetics}
\end{Entry}

\begin{Entry}{筹集}{13,12}{⽵、⾫}
  \begin{Phonetics}{筹集}{chou2ji2}[][HSK 7-9]
    \definition{v.}{arrecadar (dinheiro)}
  \end{Phonetics}
\end{Entry}

\begin{Entry}{签}{13}{⽵}
  \begin{Phonetics}{签}{qian1}[][HSK 5]
    \definition[个,根,支]{s.}{tiras de bambu usadas para adivinhação ou sorteio; pPequenas tiras de bambu ou varas finas com caracteres e símbolos gravados, usadas para adivinhação, jogos de azar ou como fichas para contagem, etc. | etiqueta; adesivo; pequena tira usada como marca | um pedaço fino e pontiagudo de bambu ou madeira; pequeno bastão pontiagudo}
    \definition{v.}{assinar; autografar; escrever o nome, palavras ou fazer marcas em documentos ou recibos | fazer comentários breves em um documento; escrever brevemente (pontos principais ou opiniões) | (em costura) alinhavar; costura grosseira}
  \end{Phonetics}
\end{Entry}

\begin{Entry}{签订}{13,4}{⽵、⾔}
  \begin{Phonetics}{签订}{qian1 ding4}[][HSK 5]
    \definition{v.}{concluir e assinar (um tratado, etc.)}
  \end{Phonetics}
\end{Entry}

\begin{Entry}{签名}{13,6}{⽵、⼝}
  \begin{Phonetics}{签名}{qian1/ming2}[][HSK 5]
    \definition[个,次]{s.}{assinatura; autógrafo}
    \definition{v.+compl.}{assinar o próprio nome; autografar; escrever seu nome para indicar concordância, apoio ou homenagem, etc.}
  \end{Phonetics}
\end{Entry}

\begin{Entry}{签字}{13,6}{⽵、⼦}
  \begin{Phonetics}{签字}{qian1 zi4}[][HSK 5]
    \definition{v.}{assinar; colocar a assinatura; escrever seu nome à mão em documentos, recibos, etc., para demonstrar responsabilidade}
  \end{Phonetics}
\end{Entry}

\begin{Entry}{签约}{13,6}{⽵、⽷}
  \begin{Phonetics}{签约}{qian1 yue1}[][HSK 5]
    \definition{v.}{assinar um contrato; assinar contratos e tratados, frequentemente utilizado no trabalho e em cooperações comerciais}
  \end{Phonetics}
\end{Entry}

\begin{Entry}{签证}{13,7}{⽵、⾔}
  \begin{Phonetics}{签证}{qian1zheng4}[][HSK 5]
    \definition[张,个,份]{s.}{visto; visto de entrada em um país}
  \end{Phonetics}
\end{Entry}

\begin{Entry}{简}{13}{⽵}
  \begin{Phonetics}{简}{jian3}
    \definition*{s.}{Sobrenome Jian}
    \definition{adj.}{simples; simplificado; breve (oposto a 繁) | breve; em resumo; em poucas palavras}
    \definition{s.}{Arcaico: tiras de bambu (para escrever) | carta; correspondência}
    \definition{v.}{simplificar | (literário) selecionar; escolher}
  \seealsoref{繁}{fan2}
  \end{Phonetics}
\end{Entry}

\begin{Entry}{简介}{13,4}{⽵、⼈}
  \begin{Phonetics}{简介}{jian3 jie4}[][HSK 6]
    \definition{s.}{breve introdução; sinopse; relato resumido}
    \definition{v.}{fazer um breve relato de (algo)}
  \end{Phonetics}
\end{Entry}

\begin{Entry}{简历}{13,4}{⽵、⼚}
  \begin{Phonetics}{简历}{jian3li4}[][HSK 4]
    \definition[个,份]{s.}{currículo; \emph{curriculum vitae} (CV); notas biográficas}
  \end{Phonetics}
\end{Entry}

\begin{Entry}{简单}{13,8}{⽵、⼗}
  \begin{Phonetics}{简单}{jian3dan1}[][HSK 3]
    \definition{adj.}{simples; descomplicado; estrutura simples; poucas complicações; fácil de entender, usar ou lidar | comum; lugar-comum; (experiência, capacidade, etc.) comum (usado principalmente em frases negativas) | casual; simplificado; precipitado; pouco cuidadoso}
  \end{Phonetics}
\end{Entry}

\begin{Entry}{简直}{13,8}{⽵、⽬}
  \begin{Phonetics}{简直}{jian3zhi2}[][HSK 3]
    \definition{adv.}{simplesmente; de forma alguma; praticamente; significa “exatamente assim” (tom exagerado)}
  \end{Phonetics}
\end{Entry}

\begin{Entry}{算}{14}{⽵}
  \begin{Phonetics}{算}{suan4}[][HSK 2]
    \definition{adv.}{finalmente; por fim; no final; significa que, após um longo período de tempo ou muitas dificuldades, finalmente se alcançou o objetivo, equivalente a 总算}
    \definition{v.}{calcular; estimar; computar | contar; incluir | planejar; calcular; projetar | pensar; supor; especular | considerar; considerar como; contar como; reconhecer como | (aritmética) contar; ter peso | deixe estar; deixe passar; seguido por 了: desistir, não se importar mais}
  \seealsoref{了}{le5}
  \seealsoref{总算}{zong3suan4}
  \end{Phonetics}
\end{Entry}

\begin{Entry}{算了}{14,2}{⽵、⼅}
  \begin{Phonetics}{算了}{suan4 le5}[][HSK 6]
    \definition{part.}{deixe estar; deixe passar; usado no final de uma frase para expressar imperativo, término, etc.}
    \definition{v.}{deixar;  deixe estar; deixe passar; esquecer isso; não querer continuar; é usado para persuadir os outros ou para expressar que posso aceitar a situação atual, para encerrar o assunto ou assunto atual, ou para dizer "esqueça"}
  \end{Phonetics}
\end{Entry}

\begin{Entry}{算命}{14,8}{⽵、⼝}
  \begin{Phonetics}{算命}{suan4ming4}
    \definition{s.}{cartomante}
    \definition{v.}{ler a sorte | fazer advinhações}
  \end{Phonetics}
\end{Entry}

\begin{Entry}{算是}{14,9}{⽵、⽇}
  \begin{Phonetics}{算是}{suan4 shi4}[][HSK 6]
    \definition{adv.}{finalmente; por fim; depois de muito tempo, o objetivo foi finalmente alcançado}
    \definition{v.}{contar como; pensar que; ser considerado}
  \end{Phonetics}
\end{Entry}

\begin{Entry}{管}{14}{⽵}
  \begin{Phonetics}{管}{guan3}[][HSK 3]
    \definition*{s.}{Guan, um estado da dinastia Zhou | Sobrenome Guan}
    \definition{adj.}{estreito; restrito; limitado; pequeno}
    \definition{clas.}{usado para objetos cilíndricos longos e finos}
    \definition{conj.}{não importa (quem, o quê, como, etc.)}
    \definition{prep.}{função semelhante a 把, usada especificamente em conjunto com 叫}
    \definition[根,条,排]{s.}{cano; tubo | instrumento musical de sopro | válvula; tubo | duto; canal; vasos}
    \definition{v.}{administrar; dirigir; controlar; cuidar; ser responsável por | ter jurisdição sobre; administrar | disciplinar (crianças ou alunos) | preocupar-se com; importar-se com; incomodar-se com; intervir | fornecer; garantir | supervisionar | governar | submeter alguém a disciplina | assumir; arcar com | incomodar; interferir | assegurar; garantir}
  \seealsoref{把}{ba3}
  \seealsoref{叫}{jiao4}
  \end{Phonetics}
\end{Entry}

\begin{Entry}{管……叫……}{14,5}{⽵、⼝}
  \begin{Phonetics}{管……叫……}{guan3 jiao4}
    \definition{expr.}{chamar alguém (ou algo) de alguém (ou algo)}
  \end{Phonetics}
\end{Entry}

\begin{Entry}{管家}{14,10}{⽵、⼧}
  \begin{Phonetics}{管家}{guan3jia1}
    \definition{s.}{mordomo | governanta}
    \definition{v.}{administrar uma casa}
  \end{Phonetics}
\end{Entry}

\begin{Entry}{管理}{14,11}{⽵、⽟}
  \begin{Phonetics}{管理}{guan3li3}[][HSK 3]
    \definition{v.}{gerenciar; executar; administrar; governar; estar encarregado de; responsável por garantir o bom andamento de uma determinada tarefa | controlar; gerenciar; fazer com que pessoas e animais obedeçam ou se comportem de maneira ordeira | cuidar; zelar por; proteger; cuidar, organizar coisas}
  \end{Phonetics}
\end{Entry}

\begin{Entry}{管道}{14,12}{⽵、⾡}
  \begin{Phonetics}{管道}{guan3 dao4}[][HSK 6]
    \definition[根,千米,公里]{s.}{oleoduto; canal; túnel; tubulação; um tubo feito de metal ou outro material usado para transportar ou descarregar fluidos (como vapor, gás, óleo, água, etc.) | caminho; canal; abordagem}
  \end{Phonetics}
\end{Entry}

\begin{Entry}{箭}{15}{⽵}
  \begin{Phonetics}{箭}{jian4}[][HSK 6]
    \definition[支]{s.}{seta | distância percorrida por uma flecha}
  \end{Phonetics}
\end{Entry}

\begin{Entry}{篇}{15}{⽵}
  \begin{Phonetics}{篇}{pian1}[][HSK 2]
    \definition*{s.}{Sobrenome Pian}
    \definition{clas.}{usado para folhas de papel, páginas de livros, artigos, etc.}
    \definition{s.}{um pedaço de escrita | folha (de papel, etc.) | (para papel, folhas de livros, artigos, etc.) folha; página; pedaço}
  \end{Phonetics}
\end{Entry}

\begin{Entry}{篮}{16}{⽵}
  \begin{Phonetics}{篮}{lan2}
    \definition[个]{s.}{cesto | o anel de ferro e a rede na cesta de basquete}
  \end{Phonetics}
\end{Entry}

\begin{Entry}{篮球}{16,11}{⽵、⽟}
  \begin{Phonetics}{篮球}{lan2qiu2}[][HSK 2]
    \definition[个,只]{s.}{basquetebol | bola de basquete; refere-se à bola utilizada no basquetebol}
  \end{Phonetics}
\end{Entry}

\begin{Entry}{簇}{17}{⽵}
  \begin{Phonetics}{簇}{cu4}
    \definition{clas.}{aglomerado; grupo; usado para pessoas ou coisas que se reúnem em grupos ou pilhas}
    \definition{s.}{pilha; aglomerado; buquê}
    \definition{v.}{aglomerar-se; formar um aglomerado; empilhar}
  \end{Phonetics}
\end{Entry}

\begin{Entry}{簇拥}{17,8}{⽵、⼿}
  \begin{Phonetics}{簇拥}{cu4yong1}[][HSK 7-9]
    \definition{v.}{aglomerar-se em volta de  | escoltar}
  \end{Phonetics}
\end{Entry}

%%%%% EOF %%%%%

