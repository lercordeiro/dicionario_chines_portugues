%%%
%%% Radical "⽵"
%%%

\section*{Radical 118: ``⽵'' (⺮)}\addcontentsline{toc}{section}{Radical 118: ⽵、⺮}

\begin{entry}{竹子}{6,3}{⽵、⼦}
  \begin{phonetics}{竹子}{zhu2zi5}[][HSK 5]
    \definition[块,株]{s.}{bambu; nome genérico para os tipos de bambu}
  \end{phonetics}
\end{entry}

\begin{entry}{竹马}{6,3}{⽵、⾺}
  \begin{phonetics}{竹马}{zhu2ma3}
    \definition{s.}{cavalo de bambu | vara de bambu usada como cavalo de brinquedo}
  \end{phonetics}
\end{entry}

\begin{entry}{竹排}{6,11}{⽵、⼿}
  \begin{phonetics}{竹排}{zhu2pai2}
    \definition{s.}{jangada de bambu}
  \end{phonetics}
\end{entry}

\begin{entry}{竹编}{6,12}{⽵、⽷}
  \begin{phonetics}{竹编}{zhu2bian1}
    \definition{s.}{vime | tecelagem de bambu}
  \end{phonetics}
\end{entry}

\begin{entry}{笋}{10}{⽵}
  \begin{phonetics}{笋}{sun3}
    \definition{s.}{broto de bambu}
  \end{phonetics}
\end{entry}

\begin{entry}{笑}{10}{⽵}
  \begin{phonetics}{笑}{xiao4}[][HSK 1]
    \definition{adj.}{ridículo; engraçado; risível; hilário}
    \definition{v.}{sorrir; rir; mostrar expressão de alegria; emitir sons de alegria | ridicularizar; rir de; zombar}
  \end{phonetics}
\end{entry}

\begin{entry}{笑话}{10,8}{⽵、⾔}
  \begin{phonetics}{笑话}{xiao4hua5}[][HSK 2]
    \definition{adj.}{absurdo | ridículo}
    \definition[个]{s.}{piada | brincadeira}
    \definition{v.}{rir de algo | zombar | ridicularizar}
  \end{phonetics}
\end{entry}

\begin{entry}{笑话儿}{10,8,2}{⽵、⾔、⼉}
  \begin{phonetics}{笑话儿}{xiao4 hua4r5}[][HSK 2]
    \definition{s.}{piada | gracejo}
  \end{phonetics}
\end{entry}

\begin{entry}{笑容}{10,10}{⽵、⼧}
  \begin{phonetics}{笑容}{xiao4rong2}
    \definition[副]{s.}{sorriso | expressão sorridente}
  \end{phonetics}
\end{entry}

\begin{entry}{笔}{10}{⽵}
  \begin{phonetics}{笔}{bi3}[][HSK 2]
    \definition{clas.}{usado para grandes quantias de dinheiro, compras, negócios, propriedades, etc. | usado em caligrafia e pintura, etc.}
    \definition[支,枝]{s.}{caneta; lápis; pincel para escrever; ferramentas para escrever ou desenhar |
técnica de escrita; caligrafia ou desenho | traço}
    \definition[支,枝]{v.}{escrever à mão}
  \end{phonetics}
\end{entry}

\begin{entry}{笔记}{10,5}{⽵、⾔}
  \begin{phonetics}{笔记}{bi3 ji4}[][HSK 2]
    \definition[篇,本,个]{s.}{notas; anotações feitas durante aulas, palestras e leituras | ensaios; esboços}
    \definition{v.}{tomar nota (por escrito)}
  \end{phonetics}
\end{entry}

\begin{entry}{笔记本}{10,5,5}{⽵、⾔、⽊}
  \begin{phonetics}{笔记本}{bi3ji4ben3}[][HSK 2]
    \definition[个,本]{s.}{caderno para anotações | \emph{laptop}; refere-se a um computador portátil}
    \definition{s.}{\emph{laptop}}
  \end{phonetics}
\end{entry}

\begin{entry}{笛}{11}{⽵}
  \begin{phonetics}{笛}{di2}
    \definition{s.}{flauta}
  \end{phonetics}
\end{entry}

\begin{entry}{符号}{11,5}{⽵、⼝}
  \begin{phonetics}{符号}{fu2hao4}[][HSK 4]
    \definition[个]{s.}{marca; símbolo; sinais que marcam as coisas
insígnia; emblema; um símbolo usado no corpo para indicar posição, \emph{status}, etc.}
  \end{phonetics}
\end{entry}

\begin{entry}{符合}{11,6}{⽵、⼝}
  \begin{phonetics}{符合}{fu2he2}[][HSK 4]
    \definition{conj.}{de acordo com; concordando com; contando com; alinhado com}
    \definition{v.}{concordar com; estar em conformidade com; corresponder com | gerenciar; lidar}
  \end{phonetics}
\end{entry}

\begin{entry}{笨}{11}{⽵}
  \begin{phonetics}{笨}{ben4}[][HSK 4]
    \definition{adj.}{estúpido; sem graça; tolo; de pouca habilidade; sem inteligência | desajeitado; tosco; inflexível | incômodo; pesado; desajeitado; difícil de manejar; trabalhoso}
  \end{phonetics}
\end{entry}

\begin{entry}{笨蛋}{11,11}{⽵、⾍}
  \begin{phonetics}{笨蛋}{ben4dan4}
    \definition{s.}{bobalhão | cabeça-oca | cabeça-dura}
    \definition{v.}{iludir | enganar}
  \end{phonetics}
\end{entry}

\begin{entry}{第}{11}{⽵}
  \begin{phonetics}{第}{di4}[][HSK 1]
    \definition*{s.}{sobrenome Di}
    \definition{adv.}{mas, apenas, somente; Indica que a ação não está sujeita a restrições ou condições; equivalente a 只管}
    \definition{conj.}{mas; contudo; orações de conexão; indicando uma relação de transição; equivalente a 但是}
    \definition{pref.}{palavra auxiliar para números ordinais; usado antes de números inteiros, indica ordem}
    \definition{s.}{diferentes notas dos candidatos aprovados nos exames imperiais | a residência de um alto funcionário; grandes residências dos burocratas da era feudal}
  \seealsoref{但是}{dan4 shi4}
  \seealsoref{只管}{zhi3 guan3}
  \end{phonetics}
\end{entry}

\begin{entry}{笼}{11}{⽵}
  \begin{phonetics}{笼}{long2}
    \definition{s.}{armação fechada de bambu, arame, etc. | jaula | gaiola}
  \end{phonetics}
  \begin{phonetics}{笼}{long3}
    \definition{v.}{envolver | cobrir}
  \end{phonetics}
\end{entry}

\begin{entry}{笼子}{11,3}{⽵、⼦}
  \begin{phonetics}{笼子}{long2zi5}
    \definition{s.}{jaula | cesta | gaiola | recipiente}
  \end{phonetics}
  \begin{phonetics}{笼子}{long3zi5}
    \definition{s.}{caixa grande | porta-malas}
  \end{phonetics}
\end{entry}

\begin{entry}{等}{12}{⽵}
  \begin{phonetics}{等}{deng3}[][HSK 1,2]
    \definition*{s.}{sobrenome Deng}
    \definition{adj.}{igual; na mesma medida ou quantidade}
    \definition{clas.}{usado para classe, grau, classificação | usado para tipo}
    \definition{part.}{e assim por diante; etc.; indica que a enumeração não está completa (pode ser usada repetidamente) | indica o fim de uma enumeração; após a enumeração, é usado para encerrar; geralmente é seguido pelo total dos itens anteriores}
    \definition{pron.}{usado após pronomes pessoais ou substantivos que se referem a pessoas; indica plural}
    \definition{s.}{classe; série; posição | equilíbrio; balança para pesar pequenas quantidades de objetos valiosos e ervas medicinais; atualmente, geralmente escrita como 戥}
    \definition{v.}{esperar; aguardar | esperar até}
  \end{phonetics}
\end{entry}

\begin{entry}{等于}{12,3}{⽵、⼆}
  \begin{phonetics}{等于}{deng3yu2}[][HSK 2]
    \definition{adv.}{igual a | equivalente a}
    \definition{v.}{equivaler a; ser equivalente a; ser quase igual a; não ter diferença}
  \end{phonetics}
\end{entry}

\begin{entry}{等级}{12,6}{⽵、⽷}
  \begin{phonetics}{等级}{deng3ji2}[][HSK 5]
    \definition{s.}{grau; classificação; posição; distinções por qualidade, grau, status, etc. | estado social; estrato social; ordem e grau; grupos sociais desiguais em termos de status social e legal}
  \end{phonetics}
\end{entry}

\begin{entry}{等到}{12,8}{⽵、⼑}
  \begin{phonetics}{等到}{deng3 dao4}[][HSK 2]
    \definition{prep.}{na hora; quando; expressão de condições temporais | esperar até; aguardar até}
  \end{phonetics}
\end{entry}

\begin{entry}{等待}{12,9}{⽵、⼻}
  \begin{phonetics}{等待}{deng3dai4}[][HSK 3]
    \definition{v.}{esperar; aguardar}
  \end{phonetics}
\end{entry}

\begin{entry}{等候}{12,10}{⽵、⼈}
  \begin{phonetics}{等候}{deng3hou4}[][HSK 5]
    \definition{v.}{esperar; aguardar; expectar; usado principalmente para objetos específicos}
  \end{phonetics}
\end{entry}

\begin{entry}{筏}{12}{⽵}
  \begin{phonetics}{筏}{fa2}
    \definition{s.}{jangada (de troncos, bambus, etc.)}
  \end{phonetics}
\end{entry}

\begin{entry}{答}{12}{⽵}
  \begin{phonetics}{答}{da1}[][HSK 5]
    \definition{v.}{concordar; responder | responder; prestar atenção}
  \end{phonetics}
  \begin{phonetics}{答}{da2}[][HSK 5]
    \definition{v.}{responder; dar resposta a; responder a | retribuir; devolver (uma visita, etc.); retribuir um favor feito a alguém por outro; fazer o bem}
  \end{phonetics}
\end{entry}

\begin{entry}{答应}{12,7}{⽵、⼴}
  \begin{phonetics}{答应}{da1ying5}[][HSK 2]
    \definition{v.}{responder; retribuir; reagir; retrucar | concordar; prometer; cumprir}
  \end{phonetics}
\end{entry}

\begin{entry}{答复}{12,9}{⽵、⼢}
  \begin{phonetics}{答复}{da2fu4}[][HSK 5]
    \definition[个]{s.}{resposta; respostas a perguntas ou solicitações}
    \definition{v.}{responder; dar uma resposta}
  \end{phonetics}
\end{entry}

\begin{entry}{答案}{12,10}{⽵、⽊}
  \begin{phonetics}{答案}{da2'an4}[][HSK 4]
    \definition[个]{s.}{chave; resposta; solução}
  \end{phonetics}
\end{entry}

\begin{entry}{策划}{12,6}{⽵、⼑}
  \begin{phonetics}{策划}{ce4hua4}
    \definition{s.}{planejador | produtor | plano}
    \definition{v.}{esquematizar | engenhar | planejar}
  \end{phonetics}
\end{entry}

\begin{entry}{筷子}{13,3}{⽵、⼦}
  \begin{phonetics}{筷子}{kuai4zi5}[][HSK 2]
    \definition[根,双,副,把,对]{s.}{pauzinhos; \emph{chopsticks}; dois bastôes finos feitos de bambu, madeira, metal ou outro material, usados para segurar comida ou outros objetos}
  \end{phonetics}
\end{entry}

\begin{entry}{签}{13}{⽵}
  \begin{phonetics}{签}{qian1}[][HSK 5]
    \definition{s.}{tiras de bambu usadas para adivinhação ou sorteio; pPequenas tiras de bambu ou varas finas com caracteres e símbolos gravados, usadas para adivinhação, jogos de azar ou como fichas para contagem, etc. | etiqueta; adesivo; pequena tira usada como marca | um pedaço fino e pontiagudo de bambu ou madeira; pequeno bastão pontiagudo}
    \definition{v.}{assinar; autografar; escrever o nome, palavras ou fazer marcas em documentos ou recibos | fazer comentários breves em um documento; escrever brevemente (pontos principais ou opiniões) | (em costura) alinhavar; costura grosseira}
  \end{phonetics}
\end{entry}

\begin{entry}{签订}{13,4}{⽵、⾔}
  \begin{phonetics}{签订}{qian1 ding4}[][HSK 5]
    \definition{v.}{concluir e assinar (um tratado, etc.)}
  \end{phonetics}
\end{entry}

\begin{entry}{签名}{13,6}{⽵、⼝}
  \begin{phonetics}{签名}{qian1 ming2}[][HSK 5]
    \definition[个,次]{s.}{assinatura; autógrafo}
    \definition{v.+compl.}{assinar o próprio nome; autografar; escrever seu nome para indicar concordância, apoio ou homenagem, etc.}
  \end{phonetics}
\end{entry}

\begin{entry}{签字}{13,6}{⽵、⼦}
  \begin{phonetics}{签字}{qian1 zi4}[][HSK 5]
    \definition{v.}{assinar; colocar a assinatura; escrever seu nome à mão em documentos, recibos, etc., para demonstrar responsabilidade}
  \end{phonetics}
\end{entry}

\begin{entry}{签约}{13,6}{⽵、⽷}
  \begin{phonetics}{签约}{qian1 yue1}[][HSK 5]
    \definition{v.}{assinar um contrato; assinar contratos e tratados, frequentemente utilizado no trabalho e em cooperações comerciais}
  \end{phonetics}
\end{entry}

\begin{entry}{签证}{13,7}{⽵、⾔}
  \begin{phonetics}{签证}{qian1zheng4}[][HSK 5]
    \definition[张,个]{s.}{visto; visto de entrada em um país}
  \end{phonetics}
\end{entry}

\begin{entry}{简历}{13,4}{⽵、⼚}
  \begin{phonetics}{简历}{jian3li4}[][HSK 4]
    \definition[个,份]{s.}{currículo; \emph{curriculum vitae} (CV); notas biográficas}
  \end{phonetics}
\end{entry}

\begin{entry}{简单}{13,8}{⽵、⼗}
  \begin{phonetics}{简单}{jian3dan1}[][HSK 3]
    \definition{adj.}{simples; descomplicado | comum; lugar-comum | casual; simplificado}
  \end{phonetics}
\end{entry}

\begin{entry}{简直}{13,8}{⽵、⽬}
  \begin{phonetics}{简直}{jian3zhi2}[][HSK 3]
    \definition{adv.}{simplesmente; em tudo; virtualmente}
  \end{phonetics}
\end{entry}

\begin{entry}{算}{14}{⽵}
  \begin{phonetics}{算}{suan4}[][HSK 2]
    \definition{adv.}{finalmente; no fim; finalmente}
    \definition{v.}{calcular | contar | computar | figurar | incluir | planejar | calcular | pensar | supor | considerar | considerar como | contar como | carregar peso | deixar estar | deixar passar}
  \end{phonetics}
\end{entry}

\begin{entry}{算了}{14,2}{⽵、⼅}
  \begin{phonetics}{算了}{suan4le5}
    \definition{v.}{deixar | deixe estar | deixe passar | esqueça isso}
  \end{phonetics}
\end{entry}

\begin{entry}{算命}{14,8}{⽵、⼝}
  \begin{phonetics}{算命}{suan4ming4}
    \definition{s.}{cartomante}
    \definition{v.}{ler a sorte | fazer advinhações}
  \end{phonetics}
\end{entry}

\begin{entry}{管}{14}{⽵}
  \begin{phonetics}{管}{guan3}[][HSK 3]
    \definition*{s.}{sobrenome Guan}
    \definition{adj.}{estreito; restrito; limitado; pequeno}
    \definition{clas.}{para objetos cilíndricos finos}
    \definition{conj.}{não importa (o que, como, etc.)}
    \definition{prep.}{a função é semelhante a 把, usada especificamente em conjunto com 叫}
    \definition[根,条,排]{s.}{cano; tubo | instrumento musical de sopro | válvula; tubo | duto; canal; vasos}
    \definition{v.}{estar encarregado de; gerenciar; executar; supervisionar | administrar; governar | sujeitar alguém à disciplina | assumir; arcar | interferir; incomodar | garantir; assegurar; fornecer}
  \end{phonetics}
\end{entry}

\begin{entry}{管家}{14,10}{⽵、⼧}
  \begin{phonetics}{管家}{guan3jia1}
    \definition{s.}{mordomo | governanta}
    \definition{v.}{administrar uma casa}
  \end{phonetics}
\end{entry}

\begin{entry}{管理}{14,11}{⽵、⽟}
  \begin{phonetics}{管理}{guan3li3}[][HSK 3]
    \definition{v.}{gerenciar; executar; administrar; governar; estar encarregado de | controlar; gerenciar | cuidar de}
  \end{phonetics}
\end{entry}

\begin{entry}{篇}{15}{⽵}
  \begin{phonetics}{篇}{pian1}[][HSK 2]
    \definition*{s.}{sobrenome Pian}
    \definition{clas.}{para pedaços, folhas}
    \definition{s.}{um pedaço de escrita
folha (de papel, etc.)}
  \end{phonetics}
\end{entry}

\begin{entry}{篮球}{16,11}{⽵、⽟}
  \begin{phonetics}{篮球}{lan2qiu2}[][HSK 2]
    \definition[个,只]{s.}{basquetebol | bola de basquete; refere-se à bola utilizada no basquetebol}
  \end{phonetics}
\end{entry}

%%%%% EOF %%%%%

