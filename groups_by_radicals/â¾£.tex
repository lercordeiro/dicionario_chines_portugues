%%%
%%% Radical "⾣"
%%%

\section*{Radical 164: ``⾣''}\addcontentsline{toc}{section}{Radical 164: ⾣}

\begin{entry}{配}{10}{⾣}
  \begin{phonetics}{配}{pei4}[][HSK 3]
    \definition{s.}{esposa}
    \definition{v.}{unir-se em matrimônio | acasalar (animais) | compor; combinar; mesclar; amalgamar; misturar |distribuir de acordo com o plano; repartir | encontrar algo para encaixar ou substituir outra coisa | corresponder; combinar; equiparar | merecer; ser digno de; ser qualificado}
  \end{phonetics}
\end{entry}

\begin{entry}{配合}{10,6}{⾣、⼝}
  \begin{phonetics}{配合}{pei4he2}[][HSK 3]
    \definition{s.}{coordenação}
    \definition{v.}{cooperar; coordenar}
  \end{phonetics}
\end{entry}

\begin{entry}{配备}{10,8}{⾣、⼡}
  \begin{phonetics}{配备}{pei4bei4}[][HSK 5]
    \definition{s.}{equipamento; material; conjunto completo de utensílios, etc.}
    \definition{v.}{fornecer; alocar; equipar; distribuir conforme necessário | posicionar; dispor (tropas, etc.)}
  \end{phonetics}
\end{entry}

\begin{entry}{配套}{10,10}{⾣、⼤}
  \begin{phonetics}{配套}{pei4tao4}[][HSK 5]
    \definition{v.+compl.}{formar um conjunto ou sistema completo; combinar vários elementos relacionados em um conjunto completo}
  \end{phonetics}
\end{entry}

\begin{entry}{酒}{10}{⾣}
  \begin{phonetics}{酒}{jiu3}[][HSK 2]
    \definition[杯,瓶,罐,桶,缸]{s.}{bebida alcoólica | vinho (especialmente vinho de arroz) | aguardente | licor | espíritos}
  \end{phonetics}
\end{entry}

\begin{entry}{酒吧}{10,7}{⾣、⼝}
  \begin{phonetics}{酒吧}{jiu3ba1}[][HSK 4]
    \definition[家,个]{s.}{bar; \emph{pub}; um local onde são vendidas bebidas alcoólicas e onde as pessoas podem beber e conversar, referindo-se principalmente a um restaurante ou hotel de estilo ocidental especializado na venda de bebidas alcoólicas.}
  \end{phonetics}
\end{entry}

\begin{entry}{酒店}{10,8}{⾣、⼴}
  \begin{phonetics}{酒店}{jiu3 dian4}[][HSK 2]
    \definition[家]{s.}{hotel | restaurante}
  \end{phonetics}
\end{entry}

\begin{entry}{酒鬼}{10,9}{⾣、⿁}
  \begin{phonetics}{酒鬼}{jiu3gui3}[][HSK 5]
    \definition{s.}{bebedor de vinho; beberrão; ébrio | alcoólatra}
  \end{phonetics}
\end{entry}

\begin{entry}{酒馆}{10,11}{⾣、⾷}
  \begin{phonetics}{酒馆}{jiu3guan3}
    \definition{s.}{bar | taverna | adega}
  \end{phonetics}
\end{entry}

\begin{entry}{酢}{12}{⾣}
  \begin{phonetics}{酢}{cu4}
    \variantof{醋}
  \end{phonetics}
  \begin{phonetics}{酢}{zuo4}
    \definition{v.}{brindar o anfitrião com vinho}
  \end{phonetics}
\end{entry}

\begin{entry}{酬劳}{13,7}{⾣、⼒}
  \begin{phonetics}{酬劳}{chou2lao2}
    \definition{s.}{recompensa}
  \end{phonetics}
\end{entry}

\begin{entry}{酱}{13}{⾣}
  \begin{phonetics}{酱}{jiang4}
    \definition{s.}{pasta grossa de soja fermentada | marinada em pasta de soja | pasta | geléia}
  \end{phonetics}
\end{entry}

\begin{entry}{酷}{14}{⾣}
  \begin{phonetics}{酷}{ku4}
    \definition{adj.}{impiedoso | forte (por exemplo, vinho) | (empréstimo linguístico) legal, \emph{cool}}
  \end{phonetics}
\end{entry}

\begin{entry}{酷斯拉}{14,12,8}{⾣、⽄、⼿}
  \begin{phonetics}{酷斯拉}{ku4si1la1}
    \definition*{s.}{Godzilla (Japonês ゴジラ Gojira)}
  \seealsoref{哥斯拉}{ge1si1la1}
  \end{phonetics}
\end{entry}

\begin{entry}{酸}{14}{⾣}
  \begin{phonetics}{酸}{suan1}[][HSK 4]
    \definition{adj.}{azedo; ácido | aflito; angustiado; doente do coração | pedante; descreve uma pessoa que finge ser culta e também descreve uma pessoa que é muito inflexível com suas próprias ideias e não está disposta a mudá-las para atender às exigências da época, é usado principalmente para satirizar intelectuais que fingem ser capazes de escrever poemas e artigos | ciumento; invejoso; sentimentos desconfortáveis porque outra pessoa é melhor do que você e, em geral, também apresenta comportamento hostil}
    \definition{s.}{ácido; produto químico que tem um sabor ácido quando misturado com água}
    \definition{v.}{estar dolorido (devido à fadiga ou doença); descreve a sensação de não ter força muscular e um pouco de dor por estar doente ou muito cansado}
  \end{phonetics}
\end{entry}

\begin{entry}{酸奶}{14,5}{⾣、⼥}
  \begin{phonetics}{酸奶}{suan1 nai3}[][HSK 4]
    \definition[瓶,杯,盒,袋]{s.}{iogurte; produto lácteo fermentado por bactérias de ácido láctico}
  \end{phonetics}
\end{entry}

\begin{entry}{酸甜苦辣}{14,11,8,14}{⾣、⽢、⾋、⾟}
  \begin{phonetics}{酸甜苦辣}{suan1 tian2 ku3 la4}[][HSK 5]
    \definition{expr.}{os altos e baixos da vida; as experiências agridoces da vida; os aspectos doces, azedos, amargos e picantes da vida; refere-se a todos os tipos de sabores, como metáfora para experiências diversas, como felicidade, sofrimento, etc. | azedo, doce, amargo, picante — alegrias e tristezas da vida}
  \end{phonetics}
\end{entry}

\begin{entry}{酸辣汤}{14,14,6}{⾣、⾟、⽔}
  \begin{phonetics}{酸辣汤}{suan1la4tang1}
    \definition{s.}{sopa avinagrada e picante (prato)}
  \end{phonetics}
\end{entry}

\begin{entry}{醉}{15}{⾣}
  \begin{phonetics}{醉}{zui4}
    \definition{v.}{embriagar-se | ficar bêbado}
  \end{phonetics}
\end{entry}

\begin{entry}{醋}{15}{⾣}
  \begin{phonetics}{醋}{cu4}
    \definition{s.}{vinagre}
  \end{phonetics}
\end{entry}

\begin{entry}{醒}{16}{⾣}
  \begin{phonetics}{醒}{xing3}[][HSK 4]
    \definition{adj.}{impressionante; notável; admirável; atraente; chamativo}
    \definition{v.}{ficar sóbrio; voltar a si; recuperar a consciência; retornar à normalidade após intoxicação, anestesia ou coma | despertar; estar acordado | ter a mente clara; mover a consciência da confusão para a compreensão | vir a entender; tornar-se ciente de; tomar consciência de}
  \end{phonetics}
\end{entry}

%%%%% EOF %%%%%

