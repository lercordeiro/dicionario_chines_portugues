%%%
%%% Radical "⽬"
%%%

\section*{Radical 109: ``⽬'' (⺫)}\addcontentsline{toc}{section}{Radical 109: ⽬、⺫}

\begin{entry}{目光}{5,6}{⽬、⼉}
  \begin{phonetics}{目光}{mu4guang1}[][HSK 5]
    \definition[道,束,种]{s.}{olhar fixo; a expressão e atitude reveladas pelos olhos | visão; vista; percepção visual; a linha imaginária formada entre os olhos e o objeto quando se olha para ele | perspicácia (capacidade de observar e reconhecer coisas); conhecimento adquirido através do contato com as coisas, capacidade de observar as coisas}
  \end{phonetics}
\end{entry}

\begin{entry}{目的}{5,8}{⽬、⽩}
  \begin{phonetics}{目的}{mu4di4}[][HSK 2]
    \definition[个]{s.}{objetivo | meta | alvo | propósito}
  \end{phonetics}
\end{entry}

\begin{entry}{目前}{5,9}{⽬、⼑}
  \begin{phonetics}{目前}{mu4qian2}[][HSK 3]
    \definition{adv.}{agora; recentemente; no momento; no presente}
  \end{phonetics}
\end{entry}

\begin{entry}{目标}{5,9}{⽬、⽊}
  \begin{phonetics}{目标}{mu4biao1}[][HSK 3]
    \definition[个]{s.}{alvo; objetivo | objetivo; destino}
  \end{phonetics}
\end{entry}

\begin{entry}{盲目}{8,5}{⽬、⽬}
  \begin{phonetics}{盲目}{mang2mu4}
    \definition{adj.}{ignorante | sem compreensão}
    \definition{adv.}{cegamente}
    \definition{s.}{cego}
  \end{phonetics}
\end{entry}

\begin{entry}{直}{8}{⽬}
  \begin{phonetics}{直}{zhi2}[][HSK 3]
    \definition*{s.}{sobrenome Zhi}
    \definition{adj.}{reto; rígido | ereto; vertical; perpendicular; vertical ao solo; de cima para baixo; da frente para trás | justo; honesto; correto | franco; direto}
    \definition{adv.}{diretamente; sempre; reto | continuamente; constantemente | apenas; simplesmente; de ​​fato}
    \definition[条]{s.}{traço vertical (em caracteres chineses, ``竖'')}
    \definition{v.}{endireitar; esticar}
  \end{phonetics}
\end{entry}

\begin{entry}{直译}{8,7}{⽬、⾔}
  \begin{phonetics}{直译}{zhi2yi4}
    \definition{s.}{tradução literal}
  \seealsoref{意译}{yi4yi4}
  \end{phonetics}
\end{entry}

\begin{entry}{直译器}{8,7,16}{⽬、⾔、⼝}
  \begin{phonetics}{直译器}{zhi2yi4qi4}
    \definition{s.}{(computação) interpretador}
  \end{phonetics}
\end{entry}

\begin{entry}{直到}{8,8}{⽬、⼑}
  \begin{phonetics}{直到}{zhi2 dao4}[][HSK 3]
    \definition{adv.}{até (na maior parte do tempo)}
  \end{phonetics}
\end{entry}

\begin{entry}{直接}{8,11}{⽬、⼿}
  \begin{phonetics}{直接}{zhi2jie1}[][HSK 2]
    \definition{adj.}{direto (oposto: indireto 间接) | imediato}
    \seeref{间接}{jian4jie1}
  \end{phonetics}
\end{entry}

\begin{entry}{直播}{8,15}{⽬、⼿}
  \begin{phonetics}{直播}{zhi2bo1}[][HSK 3]
    \definition{s.}{transmissão ao vivo; transmissão sem gravação por estações de rádio ou sem gravação por canais de televisão | (agricultura) semeadura direta}
    \definition[次]{v.}{(TV, rádio, Internet) transmitir ao vivo}
  \end{phonetics}
\end{entry}

\begin{entry}{相互}{9,4}{⽬、⼆}
  \begin{phonetics}{相互}{xiang1 hu4}[][HSK 3]
    \definition{adj.}{mútuo; recíproco}
    \definition{adv.}{mutuamente; um ao outro}
  \end{phonetics}
\end{entry}

\begin{entry}{相反}{9,4}{⽬、⼜}
  \begin{phonetics}{相反}{xiang1fan3}[][HSK 4]
    \definition{adj.}{oposto; contrário; dois aspectos das coisas são contraditórios e mutuamente exclusivos}
    \definition{conj.}{pelo contrário; usado no início ou no meio de uma frase para indicar uma contradição de significado com o que foi dito anteriormente.}
  \end{phonetics}
\end{entry}

\begin{entry}{相比}{9,4}{⽬、⽐}
  \begin{phonetics}{相比}{xiang1 bi3}[][HSK 3]
    \definition{v.}{combinar; comparar com |comparar uma coisa com outra, usar uma coisa como padrão para ver as características de outra coisa ou para obter um ponto de vista}
  \end{phonetics}
\end{entry}

\begin{entry}{相片}{9,4}{⽬、⽚}
  \begin{phonetics}{相片}{xiang4 pian4}[][HSK 4]
    \definition[张]{s.}{foto; fotografia; uma imagem de uma pessoa ou objeto feita pela exposição de papel fotográfico a um negativo fotográfico e, em seguida, revelando e fixando a imagem.}
  \end{phonetics}
\end{entry}

\begin{entry}{相处}{9,5}{⽬、⼡}
  \begin{phonetics}{相处}{xiang1chu3}[][HSK 4]
    \definition{v.}{dar-se bem; viver juntos; dar-se bem (uns com os outros); viver uns com os outros; entrar em contato uns com os outros, tratar uns aos outros}
  \end{phonetics}
\end{entry}

\begin{entry}{相似}{9,6}{⽬、⼈}
  \begin{phonetics}{相似}{xiang1si4}[][HSK 3]
    \definition{v.}{assemelhar-se; ser semelhante; ser igual}
  \end{phonetics}
\end{entry}

\begin{entry}{相关}{9,6}{⽬、⼋}
  \begin{phonetics}{相关}{xiang1guan1}[][HSK 3]
    \definition{v.}{mutuamente relacionados; inter-relacionados}
  \end{phonetics}
\end{entry}

\begin{entry}{相同}{9,6}{⽬、⼝}
  \begin{phonetics}{相同}{xiang1tong2}[][HSK 2]
    \definition{adj.}{igual | idêntico | o mesmo}
  \end{phonetics}
\end{entry}

\begin{entry}{相当}{9,6}{⽬、⼹}
  \begin{phonetics}{相当}{xiang1dang1}[][HSK 3]
    \definition{adj.}{adequado; ajustado; apropriado}
    \definition{adv.}{bastante; razoavelmente; consideravelmente}
    \definition{v.}{combinar; equilibrar; corresponder a; ser aproximadamente igual a; ser compatível com}
  \end{phonetics}
\end{entry}

\begin{entry}{相机}{9,6}{⽬、⽊}
  \begin{phonetics}{相机}{xiang4 ji1}[][HSK 2]
    \definition[台,个]{s.}{câmera | máquina fotográfica}
    \definition{v.}{ficar atento a uma oportunidade}
  \end{phonetics}
\end{entry}

\begin{entry}{相宜}{9,8}{⽬、⼧}
  \begin{phonetics}{相宜}{xiang1yi2}
    \definition{adj.}{adequado | apropriado}
    \definition{v.}{ser adequado ou apropriado}
  \end{phonetics}
\end{entry}

\begin{entry}{相亲}{9,9}{⽬、⼇}
  \begin{phonetics}{相亲}{xiang1qin1}
    \definition{s.}{encontro às cegas | entrevista arranjada para avaliar a proposta de um parceiro de casamento | apegar-se profundamente um ao outro}
  \end{phonetics}
\end{entry}

\begin{entry}{相信}{9,9}{⽬、⼈}
  \begin{phonetics}{相信}{xiang1xin4}[][HSK 2]
    \definition{v.}{acreditar | estar convencido | aceitar como verdadeiro}
  \end{phonetics}
\end{entry}

\begin{entry}{相思病}{9,9,10}{⽬、⼼、⽧}
  \begin{phonetics}{相思病}{xiang1si1bing4}
    \definition{s.}{saudade de amor}
  \end{phonetics}
\end{entry}

\begin{entry}{相遇}{9,12}{⽬、⾡}
  \begin{phonetics}{相遇}{xiang1yu4}
    \definition{v.}{encontrar (reunião, encontro, etc.)}
  \end{phonetics}
\end{entry}

\begin{entry}{相聚}{9,14}{⽬、⽿}
  \begin{phonetics}{相聚}{xiang1ju4}
    \definition{v.}{reunir-se | montar}
  \end{phonetics}
\end{entry}

\begin{entry}{省}{9}{⽬}
  \begin{phonetics}{省}{sheng3}[][HSK 2]
    \definition{s.}{província | capital provincial}
    \definition{v.}{economizar | guardar | ser frugal | omitir | excluir | deixar de fora}
  \end{phonetics}
  \begin{phonetics}{省}{xing3}
    \definition[个]{s.}{governadoria}
    \definition{v.}{examinar minuciosamente | refletir (sobre a conduta de alguém) | realizar | fazer uma visita (aos pais ou idosos)}
  \end{phonetics}
\end{entry}

\begin{entry}{省力}{9,2}{⽬、⼒}
  \begin{phonetics}{省力}{sheng3li4}
    \definition{v.}{economizar esforço ou trabalho}
  \end{phonetics}
\end{entry}

\begin{entry}{省心}{9,4}{⽬、⼼}
  \begin{phonetics}{省心}{sheng3xin1}
    \definition{adj.}{despreocupado}
    \definition{v.}{ser poupado de preocupações | despreocupar-se}
  \end{phonetics}
\end{entry}

\begin{entry}{省长}{9,4}{⽬、⾧}
  \begin{phonetics}{省长}{sheng3zhang3}
    \definition*{s.}{Governador | governador de uma província}
  \end{phonetics}
\end{entry}

\begin{entry}{省会}{9,6}{⽬、⼈}
  \begin{phonetics}{省会}{sheng3hui4}
    \definition{s.}{capital da província}
  \end{phonetics}
\end{entry}

\begin{entry}{省却}{9,7}{⽬、⼙}
  \begin{phonetics}{省却}{sheng3que4}
    \definition{v.}{livrar-se (para economizar espaço) | salvar}
  \end{phonetics}
\end{entry}

\begin{entry}{省俭}{9,9}{⽬、⼈}
  \begin{phonetics}{省俭}{sheng3jian3}
    \definition{s.}{econômico | frugal}
    \definition{v.}{economizar}
  \end{phonetics}
\end{entry}

\begin{entry}{省城}{9,9}{⽬、⼟}
  \begin{phonetics}{省城}{sheng3cheng2}
    \definition{s.}{capital da província}
  \end{phonetics}
\end{entry}

\begin{entry}{省悟}{9,10}{⽬、⼼}
  \begin{phonetics}{省悟}{xing3wu4}
    \definition{v.}{voltar a si | constatar | ver a verdade | acordar para a realidade}
  \end{phonetics}
\end{entry}

\begin{entry}{省钱}{9,10}{⽬、⾦}
  \begin{phonetics}{省钱}{sheng3qian2}
    \definition{v.}{economizar dinheiro}
  \end{phonetics}
\end{entry}

\begin{entry}{眉}{9}{⽬}
  \begin{phonetics}{眉}{mei2}
    \definition{s.}{sobrancelha | margem superior}
  \end{phonetics}
\end{entry}

\begin{entry}{眉毛}{9,4}{⽬、⽑}
  \begin{phonetics}{眉毛}{mei2mao5}
    \definition[根]{s.}{sobrancelha}
  \end{phonetics}
\end{entry}

\begin{entry}{眉头}{9,5}{⽬、⼤}
  \begin{phonetics}{眉头}{mei2tou2}
    \definition{s.}{testa}
  \end{phonetics}
\end{entry}

\begin{entry}{看}{9}{⽬}
  \begin{phonetics}{看}{kan1}
    \definition{v.}{cuidar | vigiar}
  \end{phonetics}
  \begin{phonetics}{看}{kan4}[][HSK 1]
    \definition{interj.}{Cuidado! (para um perigo)}
    \definition{part.}{(depois de um verbo) tentar}
    \definition{v.}{olhar | ver | assistir | ler | visitar (pessoas)}
  \end{phonetics}
\end{entry}

\begin{entry}{看上去}{9,3,5}{⽬、⼀、⼛}
  \begin{phonetics}{看上去}{kan4 shang4 qu4}[][HSK 3]
    \definition{adv.}{parece que}
  \end{phonetics}
\end{entry}

\begin{entry}{看不起}{9,4,10}{⽬、⼀、⾛}
  \begin{phonetics}{看不起}{kan4bu5qi3}[][HSK 4]
    \definition{v.}{desprezar; desdenhar; menosprezar; ter desprezo; olhar de cima para baixo}
  \end{phonetics}
\end{entry}

\begin{entry}{看见}{9,4}{⽬、⾒}
  \begin{phonetics}{看见}{kan4 jian4}[][HSK 1]
    \definition{v.}{encontrar | enxergar | ver | avistar}
  \end{phonetics}
\end{entry}

\begin{entry}{看出}{9,5}{⽬、⼐}
  \begin{phonetics}{看出}{kan4 chu1}[][HSK 5]
    \definition{v.}{perceber; descobrir; estar ciente de; ver}
  \end{phonetics}
\end{entry}

\begin{entry}{看成}{9,6}{⽬、⼽}
  \begin{phonetics}{看成}{kan4 cheng2}[][HSK 5]
    \definition{v.}{olhar como; considerar como; tratar como; pensar como; ter como}
  \end{phonetics}
\end{entry}

\begin{entry}{看来}{9,7}{⽬、⽊}
  \begin{phonetics}{看来}{kan4 lai2}[][HSK 4]
    \definition{adv.}{parecer; parecer como se (ou embora); refere-se a um julgamento aproximado; expressa um julgamento por observação}
    \definition{v.}{ser considerado; na visão de alguém; na opinião de alguém; expressar a ideia aproximada que o locutor tem da situação}
  \end{phonetics}
\end{entry}

\begin{entry}{看到}{9,8}{⽬、⼑}
  \begin{phonetics}{看到}{kan4 dao4}[][HSK 1]
    \definition{v.}{ver}
  \end{phonetics}
\end{entry}

\begin{entry}{看法}{9,8}{⽬、⽔}
  \begin{phonetics}{看法}{kan4fa3}[][HSK 2]
    \definition[个]{s.}{modo de olhar alguma coisa | ponto de vista | opinião}
  \end{phonetics}
\end{entry}

\begin{entry}{看待}{9,9}{⽬、⼻}
  \begin{phonetics}{看待}{kan4dai4}[][HSK 5]
    \definition{v.}{tratar; considerar; olhar com atenção; ter uma certa atitude ou visão em relação a alguém ou alguma coisa}
  \end{phonetics}
\end{entry}

\begin{entry}{看病}{9,10}{⽬、⽧}
  \begin{phonetics}{看病}{kan4 bing4}[][HSK 1]
    \definition{v.+compl.}{(médico) ver um paciente | (paciente) consultar (ver) um médico}
  \end{phonetics}
\end{entry}

\begin{entry}{看起来}{9,10,7}{⽬、⾛、⽊}
  \begin{phonetics}{看起来}{kan4 qi3 lai5}[][HSK 3]
    \definition{v.}{parecer; parecer com}
  \end{phonetics}
\end{entry}

\begin{entry}{看望}{9,11}{⽬、⽉}
  \begin{phonetics}{看望}{kan4wang4}[][HSK 4]
    \definition{v.}{ver; visitar; ligar; dar uma olhada; ir até os pais, idosos, professores ou amigos para cumprimentá-los}
  \end{phonetics}
\end{entry}

\begin{entry}{看淡}{9,11}{⽬、⽔}
  \begin{phonetics}{看淡}{kan4dan4}
    \definition{v.}{considerar sem importância | ser indiferente a (fama, riqueza, etc.) | (de uma economia ou mercado) enfraquecer, ficar mais lento, diminuir a velocidade}
  \end{phonetics}
\end{entry}

\begin{entry}{眞}{10}{⽬}
  \begin{phonetics}{眞}{zhen1}
    \variantof{真}
  \end{phonetics}
\end{entry}

\begin{entry}{眯}{11}{⽬}
  \begin{phonetics}{眯}{mi1}
    \definition{v.}{estreitar os olhos | esmagar | (dialeto) tirar uma soneca}
  \end{phonetics}
  \begin{phonetics}{眯}{mi2}
    \definition{v.}{cegar (como com poeira)}
  \end{phonetics}
\end{entry}

\begin{entry}{眼}{11}{⽬}
  \begin{phonetics}{眼}{yan3}[][HSK 2]
    \definition{clas.}{para grandes coisas ocas: poços, fogões, panelas, etc.}
    \definition[只,双]{s.}{ponto crucial (de um assunto) | olho | pequeno buraco}
  \end{phonetics}
\end{entry}

\begin{entry}{眼花缭乱}{11,7,15,7}{⽬、⾋、⽷、⼄}
  \begin{phonetics}{眼花缭乱}{yan3hua1liao2luan4}
    \definition{v.}{ficar deslumbrado | deslumbrar}
  \end{phonetics}
\end{entry}

\begin{entry}{眼证}{11,7}{⽬、⾔}
  \begin{phonetics}{眼证}{yan3zheng4}
    \definition{s.}{testemunha ocular}
  \end{phonetics}
\end{entry}

\begin{entry}{眼里}{11,7}{⽬、⾥}
  \begin{phonetics}{眼里}{yan3 li3}[][HSK 4]
    \definition{s.}{nos olhos de uma pessoa; dentro de sua visão}
  \end{phonetics}
\end{entry}

\begin{entry}{眼泪}{11,8}{⽬、⽔}
  \begin{phonetics}{眼泪}{yan3 lei4}[][HSK 4]
    \definition[滴,行]{s.}{lágrimas; termo genérico para lágrimas; fluido incolor e transparente secretado pelas glândulas lacrimais no olho, que serve para proteger o olho}
  \end{phonetics}
\end{entry}

\begin{entry}{眼前}{11,9}{⽬、⼑}
  \begin{phonetics}{眼前}{yan3 qian2}[][HSK 3]
    \definition{adv.}{agora; (no) momento}
    \definition{s.}{diante dos olhos; diante de; momento}
  \end{phonetics}
\end{entry}

\begin{entry}{眼柄}{11,9}{⽬、⽊}
  \begin{phonetics}{眼柄}{yan3bing3}
    \definition{s.}{pedúnculo ocular (de crustáceo, etc.)}
  \end{phonetics}
\end{entry}

\begin{entry}{眼袋}{11,11}{⽬、⾐}
  \begin{phonetics}{眼袋}{yan3dai4}
    \definition{s.}{inchaço sob os olhos}
  \end{phonetics}
\end{entry}

\begin{entry}{眼睛}{11,13}{⽬、⽬}
  \begin{phonetics}{眼睛}{yan3jing5}[][HSK 2]
    \definition[只,双]{s.}{olho(s)}
  \end{phonetics}
\end{entry}

\begin{entry}{眼镜}{11,16}{⽬、⾦}
  \begin{phonetics}{眼镜}{yan3jing4}[][HSK 4]
    \definition[副]{s.}{óculos; óculos de grau}
  \end{phonetics}
\end{entry}

\begin{entry}{着}{11}{⽬}
  \begin{phonetics}{着}{zhao1}
    \definition{interj.}{Tudo bem!}
    \definition{s.}{movimento (xadrez) | truque}
  \end{phonetics}
  \begin{phonetics}{着}{zhao2}
    \definition{v.}{ser afetado por | queimar | pegar fogo | entrar em contato com | sentir | tocar}
  \end{phonetics}
  \begin{phonetics}{着}{zhe5}[][HSK 1,4]
    \definition{interj.}{O.K.!; Tudo bem!; Tudo certo!}
    \definition{part.}{indicando ação em andamento ou estado em andamento}
    \definition{s.}{um movimento no xadrez |movimento; estratégia; estratagema}
    \definition{v.}{colocar; deixar de lado}
  \end{phonetics}
  \begin{phonetics}{着}{zhuo2}
    \definition{v.}{aplicar | contactar | usar | vestir (roupas)}
  \end{phonetics}
\end{entry}

\begin{entry}{着手}{11,4}{⽬、⼿}
  \begin{phonetics}{着手}{zhuo2shou3}
    \definition{v.}{colocar a mão nisso | estabelecer | começar uma tarefa}
  \end{phonetics}
\end{entry}

\begin{entry}{着火}{11,4}{⽬、⽕}
  \begin{phonetics}{着火}{zhao2huo3}[][HSK 4]
    \definition{v.}{pegar fogo; estar em chamas}
  \end{phonetics}
\end{entry}

\begin{entry}{着地}{11,6}{⽬、⼟}
  \begin{phonetics}{着地}{zhao2di4}
    \definition{v.}{pousar | tocar o chão}
  \end{phonetics}
\end{entry}

\begin{entry}{着花}{11,7}{⽬、⾋}
  \begin{phonetics}{着花}{zhao2hua1}
    \definition{v.}{florescer}
  \end{phonetics}
  \begin{phonetics}{着花}{zhuo2hua1}
    \definition{s.}{floração}
    \definition{v.}{florescer}
  \end{phonetics}
\end{entry}

\begin{entry}{着急}{11,9}{⽬、⼼}
  \begin{phonetics}{着急}{zhao2ji2}[][HSK 4]
    \definition{adj.}{ansioso; preocupado |}
    \definition{s.}{preocupação; ansiedade}
    \definition{v.+compl.}{preocupar-se | sentir-se ansioso | sentir uma sensação de urgência}
  \end{phonetics}
\end{entry}

\begin{entry}{着凉}{11,10}{⽬、⼎}
  \begin{phonetics}{着凉}{zhao2liang2}
    \definition{v.}{pegar um resfriado}
  \end{phonetics}
\end{entry}

\begin{entry}{着眼}{11,11}{⽬、⽬}
  \begin{phonetics}{着眼}{zhuo2yan3}
    \definition{v.}{ter seus olhos em (um objetivo) | ter algo em mente | concentrar-se}
  \end{phonetics}
\end{entry}

\begin{entry}{着装}{11,12}{⽬、⾐}
  \begin{phonetics}{着装}{zhuo2zhuang1}
    \definition{s.}{roupa | vestimenta}
    \definition{v.}{vestir}
  \end{phonetics}
\end{entry}

\begin{entry}{着想}{11,13}{⽬、⼼}
  \begin{phonetics}{着想}{zhuo2xiang3}
    \definition{v.}{considerar (as necessidades de outras pessoas) | pensar (para os outros)}
  \end{phonetics}
\end{entry}

\begin{entry}{着数}{11,13}{⽬、⽁}
  \begin{phonetics}{着数}{zhao1shu4}
    \definition{s.}{estratégia | movimento (no xadrez, no palco, nas artes marciais) | esquema | truque}
  \end{phonetics}
\end{entry}

\begin{entry}{著名}{11,6}{⽬、⼝}
  \begin{phonetics}{著名}{zhu4ming2}[][HSK 4]
    \definition{adj.}{famoso; bem conhecido; célebre}
  \end{phonetics}
\end{entry}

\begin{entry}{著作}{11,7}{⽬、⼈}
  \begin{phonetics}{著作}{zhu4zuo4}[][HSK 4]
    \definition[部]{s.}{obra; livro; escritos}
    \definition{v.}{escrever; usar palavras para expressar opiniões, conhecimentos, ideias, sentimentos, etc.}
  \end{phonetics}
\end{entry}

\begin{entry}{睡}{13}{⽬}
  \begin{phonetics}{睡}{shui4}[][HSK 1]
    \definition{v.}{dormir}
  \end{phonetics}
\end{entry}

\begin{entry}{睡衣}{13,6}{⽬、⾐}
  \begin{phonetics}{睡衣}{shui4yi1}
    \definition{s.}{pijamas | roupas de dormir}
  \end{phonetics}
\end{entry}

\begin{entry}{睡觉}{13,9}{⽬、⾒}
  \begin{phonetics}{睡觉}{shui4jiao4}[][HSK 1]
    \definition{v.+compl.}{ir para a cama | dormir | deitar-se}
  \end{phonetics}
\end{entry}

\begin{entry}{睡眠}{13,10}{⽬、⽬}
  \begin{phonetics}{睡眠}{shui4 mian2}[][HSK 5]
    \definition{s.}{sono; \emph{somnus}; sonolência}
  \end{phonetics}
\end{entry}

\begin{entry}{睡着}{13,11}{⽬、⽬}
  \begin{phonetics}{睡着}{shui4 zhao2}[][HSK 4]
    \definition{v.}{dormir; adormecer; cair no sono}
  \end{phonetics}
\end{entry}

\begin{entry}{睡懒觉}{13,16,9}{⽬、⼼、⾒}
  \begin{phonetics}{睡懒觉}{shui4lan3jiao4}
    \definition{v.}{levantar-se tarde | passar o tempo a dormir}
  \end{phonetics}
\end{entry}

\begin{entry}{瞧}{17}{⽬}
  \begin{phonetics}{瞧}{qiao2}[][HSK 5]
    \definition{v.}{ver; olhar | tratar; diagnosticar e tratar | ver; visitar; fazer uma visita}
  \end{phonetics}
\end{entry}

%%%%% EOF %%%%%

