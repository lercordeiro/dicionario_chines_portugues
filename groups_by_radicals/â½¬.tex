%%%
%%% Radical "⽬"
%%%

\section*{Radical 109: ``⽬'' (⺫)}\addcontentsline{toc}{section}{Radical 109: ⽬、⺫}

\begin{entry}{目}{5}{⽬}[Kangxi 109]
  \begin{phonetics}{目}{mu4}
    \definition*{s.}{Sobrenome Mu}
    \definition{s.}{olho | item | (biologia) ordem | lista de coisas; catálogo; sumário | buraco em uma rede; malha (abertura)  | (de documentos, teses, etc.) nome; título | ponto; ponto de território, um termo do Go; refere-se à intersecção das linhas verticais e horizontais no tabuleiro, uma intersecção é chamada de 一目, \dpy{yi2 mu4}}
    \definition{v.}{(literário) olhar; considerar}
  \end{phonetics}
\end{entry}

\begin{entry}{目光}{5,6}{⽬、⼉}
  \begin{phonetics}{目光}{mu4guang1}[][HSK 5]
    \definition[道,束,种]{s.}{olhar fixo; a expressão e atitude reveladas pelos olhos | visão; vista; percepção visual; a linha imaginária formada entre os olhos e o objeto quando se olha para ele | perspicácia (capacidade de observar e reconhecer coisas); conhecimento adquirido através do contato com as coisas, capacidade de observar as coisas}
  \end{phonetics}
\end{entry}

\begin{entry}{目的}{5,8}{⽬、⽩}
  \begin{phonetics}{目的}{mu4di4}[][HSK 2]
    \definition[个,些,种]{s.}{objetivo; meta; alvo; finalidade; propósito; o lugar ou situação que se deseja alcançar; o resultado que se deseja obter; o centro do alvo}
  \end{phonetics}
\end{entry}

\begin{entry}{目前}{5,9}{⽬、⼑}
  \begin{phonetics}{目前}{mu4qian2}[][HSK 3]
    \definition{adv.}{agora; recentemente; no momento; no presente}
  \end{phonetics}
\end{entry}

\begin{entry}{目标}{5,9}{⽬、⽊}
  \begin{phonetics}{目标}{mu4biao1}[][HSK 3]
    \definition[个,项]{s.}{alvo; objetivo; objeto de tiro, ataque ou busca| objetivo; meta; destino; a situação ou padrão que se deseja alcançar}
  \end{phonetics}
\end{entry}

\begin{entry}{盲}{8}{⽬}
  \begin{phonetics}{盲}{mang2}
    \definition{adj.}{cego | incapaz de distinguir coisas | totalmente incompetente}
    \definition{adv.}{cegamente}
  \end{phonetics}
\end{entry}

\begin{entry}{盲人}{8,2}{⽬、⼈}
  \begin{phonetics}{盲人}{mang2 ren2}[][HSK 6]
    \definition[个,位,名]{s.}{cego; pessoa cega; pessoas com deficiência visual}
  \end{phonetics}
\end{entry}

\begin{entry}{盲目}{8,5}{⽬、⽬}
  \begin{phonetics}{盲目}{mang2mu4}
    \definition{adj.}{ignorante | sem compreensão}
    \definition{adv.}{cegamente}
    \definition{s.}{cego}
  \end{phonetics}
\end{entry}

\begin{entry}{直}{8}{⽬}
  \begin{phonetics}{直}{zhi2}[][HSK 3]
    \definition*{s.}{Sobrenome Zhi}
    \definition{adj.}{reto (oposto a 弯,曲) | vertical; perpendicular (oposto a 横) | justo; íntegro; imparcial | franco; sincero; direto ao ponto | rígido; entorpecido | direto; em linha reta; rígido | ereto; perpendicular ao solo}
    \definition{adv.}{diretamente; sempre; reto | continuamente; constantemente | apenas; simplesmente | de ​​fato}
    \definition[条]{s.}{traço vertical (em caracteres chineses, 竖)}
    \definition{v.}{endireitar; tornar reto | alongar}
  \seealsoref{横}{heng2}
  \seealsoref{曲}{qu1}
  \seealsoref{竖}{shu4}
  \seealsoref{弯}{wan1}
  \end{phonetics}
\end{entry}

\begin{entry}{直译}{8,7}{⽬、⾔}
  \begin{phonetics}{直译}{zhi2yi4}
    \definition{s.}{tradução literal}
  \seealsoref{意译}{yi4yi4}
  \end{phonetics}
\end{entry}

\begin{entry}{直译器}{8,7,16}{⽬、⾔、⼝}
  \begin{phonetics}{直译器}{zhi2yi4qi4}
    \definition{s.}{(computação) interpretador}
  \end{phonetics}
\end{entry}

\begin{entry}{直到}{8,8}{⽬、⼑}
  \begin{phonetics}{直到}{zhi2 dao4}[][HSK 3]
    \definition{adv.}{até (geralmente se refere ao tempo); até que}
  \end{phonetics}
\end{entry}

\begin{entry}{直线}{8,8}{⽬、⽷}
  \begin{phonetics}{直线}{zhi2 xian4}[][HSK 5]
    \definition{adj.}{direto; retilíneo | íngreme; acentuada (subida ou descida)}
    \definition[条]{s.}{linha reta}
  \end{phonetics}
\end{entry}

\begin{entry}{直接}{8,11}{⽬、⼿}
  \begin{phonetics}{直接}{zhi2jie1}[][HSK 2]
    \definition{adj.}{direto (oposto: indireto 间接) | imediato}
  \seealsoref{间接}{jian4jie1}
  \end{phonetics}
\end{entry}

\begin{entry}{直播}{8,15}{⽬、⼿}
  \begin{phonetics}{直播}{zhi2bo1}[][HSK 3]
    \definition{v.}{transmissão ao vivo; transmitir diretamente, sem gravar}
  \end{phonetics}
\end{entry}

\begin{entry}{相}{9}{⽬}
  \begin{phonetics}{相}{xiang1}
    \definition*{s.}{Sobrenome Xiang}
    \definition{adv.}{uns aos outros; mutuamente | (para uma ação realizada por uma pessoa em relação a outra) | indica a ação de uma parte em relação à outra parte}
    \definition{s.}{qualidade; substância}
    \definition{v.}{ver por si mesmo (se algo ou algo é do seu agrado)}
  \end{phonetics}
  \begin{phonetics}{相}{xiang4}
    \definition*{s.}{Sobrenome Xiang}
    \definition{s.}{aparência | postura; porte; postura sentada, em pé, etc. | (física) fase; refere-se a uma parte homogênea de uma substância com a mesma composição e as mesmas propriedades físicas e químicas | fotografia | primeiro-ministro (na China antiga) | ministro; títulos oficiais de certos países | fácies marinha (carvão) | elefante, uma das peças do xadrez chinês | recepcionista (pessoa que ajuda o anfitrião a receber o hóspede); antigamente, referia-se a alguém que ajudava o anfitrião a receber convidados}
    \definition{v.}{olhar e avaliar; observe a aparência das coisas; julgar sua qualidade | assistir; ajudar; auxiliar}
  \end{phonetics}
\end{entry}

\begin{entry}{相互}{9,4}{⽬、⼆}
  \begin{phonetics}{相互}{xiang1 hu4}[][HSK 3]
    \definition{adj.}{mútuo; recíproco; entre duas pessoas ou coisas}
    \definition{adv.}{mutuamente; um ao outro; tratamento recíproco}
  \end{phonetics}
\end{entry}

\begin{entry}{相反}{9,4}{⽬、⼜}
  \begin{phonetics}{相反}{xiang1fan3}[][HSK 4]
    \definition{adj.}{oposto; contrário; dois aspectos das coisas são contraditórios e mutuamente exclusivos}
    \definition{conj.}{pelo contrário; usado no início ou no meio de uma frase para indicar uma contradição de significado com o que foi dito anteriormente.}
  \end{phonetics}
\end{entry}

\begin{entry}{相比}{9,4}{⽬、⽐}
  \begin{phonetics}{相比}{xiang1 bi3}[][HSK 3]
    \definition{v.}{combinar; comparar com | comparar mutuamente, usar uma coisa como padrão, perceber as características de outra coisa ou obter uma opinião}
  \end{phonetics}
\end{entry}

\begin{entry}{相片}{9,4}{⽬、⽚}
  \begin{phonetics}{相片}{xiang4 pian4}[][HSK 4]
    \definition[张]{s.}{foto; fotografia; uma imagem de uma pessoa ou objeto feita pela exposição de papel fotográfico a um negativo fotográfico e, em seguida, revelando e fixando a imagem.}
  \end{phonetics}
\end{entry}

\begin{entry}{相处}{9,5}{⽬、⼡}
  \begin{phonetics}{相处}{xiang1chu3}[][HSK 4]
    \definition{v.}{dar-se bem; viver juntos; dar-se bem (uns com os outros); viver uns com os outros; entrar em contato uns com os outros, tratar uns aos outros}
  \end{phonetics}
\end{entry}

\begin{entry}{相似}{9,6}{⽬、⼈}
  \begin{phonetics}{相似}{xiang1si4}[][HSK 3]
    \definition{v.}{assemelhar-se; ser semelhante; ser parecido}
  \end{phonetics}
\end{entry}

\begin{entry}{相关}{9,6}{⽬、⼋}
  \begin{phonetics}{相关}{xiang1guan1}[][HSK 3]
    \definition{v.}{estar mutuamente relacionado; estar intimamente relacionado; estar inter-relacionado}
  \end{phonetics}
\end{entry}

\begin{entry}{相同}{9,6}{⽬、⼝}
  \begin{phonetics}{相同}{xiang1tong2}[][HSK 2]
    \definition{adj.}{semelhante; similar; igual; idêntico; o mesmo; consistentes entre si, sem diferença}
  \end{phonetics}
\end{entry}

\begin{entry}{相当}{9,6}{⽬、⼹}
  \begin{phonetics}{相当}{xiang1dang1}[][HSK 3]
    \definition{adj.}{adequado; apropriado}
    \definition{adv.}{bastante; razoavelmente; consideravelmente; indica um grau relativamente alto e profundo}
    \definition{v.}{combinar; equilibrar; corresponder a; ser aproximadamente igual a; ser proporcional a}
  \end{phonetics}
\end{entry}

\begin{entry}{相机}{9,6}{⽬、⽊}
  \begin{phonetics}{相机}{xiang4 ji1}[][HSK 2]
    \definition[台,部,架,个]{s.}{câmera; máquina fotográfica}
    \definition{v.}{ficar atento a uma oportunidade; procurar oportunidades}
  \end{phonetics}
\end{entry}

\begin{entry}{相声}{9,7}{⽬、⼠}
  \begin{phonetics}{相声}{xiang4sheng5}[][HSK 5]
    \definition[个,段]{s.}{conversa cruzada; diálogo cômico; forma de performance humorística, em que os atores usam piadas, canções e imitações para satirizar e elogiar}
  \end{phonetics}
\end{entry}

\begin{entry}{相应}{9,7}{⽬、⼴}
  \begin{phonetics}{相应}{xiang1ying4}[][HSK 5]
    \definition{v.}{corresponder}
  \end{phonetics}
\end{entry}

\begin{entry}{相宜}{9,8}{⽬、⼧}
  \begin{phonetics}{相宜}{xiang1yi2}
    \definition{adj.}{adequado | apropriado}
    \definition{v.}{ser adequado ou apropriado}
  \end{phonetics}
\end{entry}

\begin{entry}{相亲}{9,9}{⽬、⼇}
  \begin{phonetics}{相亲}{xiang1qin1}
    \definition{s.}{encontro às cegas | entrevista arranjada para avaliar a proposta de um parceiro de casamento | apegar-se profundamente um ao outro}
  \end{phonetics}
\end{entry}

\begin{entry}{相信}{9,9}{⽬、⼈}
  \begin{phonetics}{相信}{xiang1xin4}[][HSK 2]
    \definition{v.}{acreditar em; estar convencido de; ter fé em; acreditar que algo é certo ou verdadeiro sem dúvida}
  \end{phonetics}
\end{entry}

\begin{entry}{相思病}{9,9,10}{⽬、⼼、⽧}
  \begin{phonetics}{相思病}{xiang1si1bing4}
    \definition{s.}{saudade de amor}
  \end{phonetics}
\end{entry}

\begin{entry}{相等}{9,12}{⽬、⽵}
  \begin{phonetics}{相等}{xiang1deng3}[][HSK 5]
    \definition{v.}{ser igual a; possuir a mesma quantidade, peso, tamanho e grau}
  \end{phonetics}
\end{entry}

\begin{entry}{相遇}{9,12}{⽬、⾡}
  \begin{phonetics}{相遇}{xiang1yu4}
    \definition{v.}{encontrar (reunião, encontro, etc.)}
  \end{phonetics}
\end{entry}

\begin{entry}{相聚}{9,14}{⽬、⽿}
  \begin{phonetics}{相聚}{xiang1ju4}
    \definition{v.}{reunir-se | montar}
  \end{phonetics}
\end{entry}

\begin{entry}{盼}{9}{⽬}
  \begin{phonetics}{盼}{pan4}
    \definition*{s.}{Sobrenome Pan}
    \definition{adj.}{(de olhos) com preto e branco fortemente contrastados; olhos claros}
    \definition{v.}{olhar | esperar por; ansiar por | sentir falta de ; continuar pensando sobre}
  \end{phonetics}
\end{entry}

\begin{entry}{盼望}{9,11}{⽬、⽉}
  \begin{phonetics}{盼望}{pan4wang4}[][HSK 6]
    \definition{v.}{esperar por; ansiar por; esperar que algo aconteça em breve}
  \end{phonetics}
\end{entry}

\begin{entry}{省}{9}{⽬}
  \begin{phonetics}{省}{sheng3}[][HSK 2]
    \definition*{s.}{Sobrenome Sheng}
    \definition{s.}{província; unidade administrativa, subordinada diretamente ao governo central | capital provincial; refere-se à capital da província, localização da administração provincial | abreviação (de palavras)}
    \definition{v.}{economizar; poupar; reduzir o consumo (em oposição a 费) | omitir; deixar de fora}
  \seealsoref{费}{fei4}
  \end{phonetics}
  \begin{phonetics}{省}{xing3}
    \definition{v.}{examinar-se criticamente; verificar (os próprios pensamentos, palavras e ações) | visitar (especialmente os pais ou pessoas mais velhas) | estar ciente; tornar-se consciente; compreender; tomar consciência | examinar minuciosamente; inspecionar; escrutinar}
  \end{phonetics}
\end{entry}

\begin{entry}{省力}{9,2}{⽬、⼒}
  \begin{phonetics}{省力}{sheng3li4}
    \definition{v.}{economizar esforço ou trabalho}
  \end{phonetics}
\end{entry}

\begin{entry}{省心}{9,4}{⽬、⼼}
  \begin{phonetics}{省心}{sheng3xin1}
    \definition{adj.}{despreocupado}
    \definition{v.}{ser poupado de preocupações | despreocupar-se}
  \end{phonetics}
\end{entry}

\begin{entry}{省长}{9,4}{⽬、⾧}
  \begin{phonetics}{省长}{sheng3zhang3}
    \definition{s.}{governador; governador de uma província}
  \end{phonetics}
\end{entry}

\begin{entry}{省会}{9,6}{⽬、⼈}
  \begin{phonetics}{省会}{sheng3hui4}
    \definition{s.}{capital da província}
  \end{phonetics}
\end{entry}

\begin{entry}{省却}{9,7}{⽬、⼙}
  \begin{phonetics}{省却}{sheng3que4}
    \definition{v.}{livrar-se (para economizar espaço) | salvar}
  \end{phonetics}
\end{entry}

\begin{entry}{省俭}{9,9}{⽬、⼈}
  \begin{phonetics}{省俭}{sheng3jian3}
    \definition{s.}{econômico | frugal}
    \definition{v.}{economizar}
  \end{phonetics}
\end{entry}

\begin{entry}{省城}{9,9}{⽬、⼟}
  \begin{phonetics}{省城}{sheng3cheng2}
    \definition{s.}{capital da província}
  \end{phonetics}
\end{entry}

\begin{entry}{省悟}{9,10}{⽬、⼼}
  \begin{phonetics}{省悟}{xing3wu4}
    \definition{v.}{voltar a si | constatar | ver a verdade | acordar para a realidade}
  \end{phonetics}
\end{entry}

\begin{entry}{省钱}{9,10}{⽬、⾦}
  \begin{phonetics}{省钱}{sheng3 qian2}[][HSK 6]
    \definition{adj.}{barato; não caro}
    \definition{v.}{economizar dinheiro}
  \end{phonetics}
\end{entry}

\begin{entry}{眉}{9}{⽬}
  \begin{phonetics}{眉}{mei2}
    \definition{s.}{sobrancelha | margem superior}
  \end{phonetics}
\end{entry}

\begin{entry}{眉毛}{9,4}{⽬、⽑}
  \begin{phonetics}{眉毛}{mei2mao5}
    \definition[根]{s.}{sobrancelha}
  \end{phonetics}
\end{entry}

\begin{entry}{眉头}{9,5}{⽬、⼤}
  \begin{phonetics}{眉头}{mei2tou2}
    \definition{s.}{testa}
  \end{phonetics}
\end{entry}

\begin{entry}{看}{9}{⽬}
  \begin{phonetics}{看}{kan1}
    \definition{v.}{cuidar de; tomar conta de; cuidar de; proteger | manter sob vigilância}
  \end{phonetics}
  \begin{phonetics}{看}{kan4}[][HSK 1,6]
    \definition{interj.}{Cuidado! (para um perigo)}
    \definition{part.}{tentar, usado depois de outros verbos}
    \definition{v.}{ver; olhar para; observar; fazer contato visual com pessoas ou objetos | pensar; considerar; observar; julgar; observar e analisar | visitar; ver; fazer uma visita | olhar para; considerar; tratar | tratar (um paciente ou uma doença) | cuidar | ficar atento; ficar de olho | depender de; ser dependente de | ler}
  \end{phonetics}
\end{entry}

\begin{entry}{看上去}{9,3,5}{⽬、⼀、⼛}
  \begin{phonetics}{看上去}{kan4 shang4 qu4}[][HSK 3]
    \definition{adv.}{parece que}
  \end{phonetics}
\end{entry}

\begin{entry}{看不起}{9,4,10}{⽬、⼀、⾛}
  \begin{phonetics}{看不起}{kan4bu5qi3}[][HSK 4]
    \definition{v.}{desprezar; desdenhar; menosprezar; ter desprezo; olhar de cima para baixo}
  \end{phonetics}
\end{entry}

\begin{entry}{看见}{9,4}{⽬、⾒}
  \begin{phonetics}{看见}{kan4jian4}[][HSK 1]
    \definition{v.}{ver; avistar; ao olhar, descobrir alguém ou algo}
  \end{phonetics}
\end{entry}

\begin{entry}{看出}{9,5}{⽬、⼐}
  \begin{phonetics}{看出}{kan4 chu1}[][HSK 5]
    \definition{v.}{perceber; descobrir; estar ciente de; ver}
  \end{phonetics}
\end{entry}

\begin{entry}{看好}{9,6}{⽬、⼥}
  \begin{phonetics}{看好}{kan4 hao3}[][HSK 6]
    \definition{v.}{elogiar; apreciar; encorajar; acreditar que pessoas ou coisas terão uma boa tendência | estar prestes a surgir uma boa tendência}
  \end{phonetics}
\end{entry}

\begin{entry}{看成}{9,6}{⽬、⼽}
  \begin{phonetics}{看成}{kan4 cheng2}[][HSK 5]
    \definition{v.}{olhar como; considerar como; tratar como; pensar como; ter como}
  \end{phonetics}
\end{entry}

\begin{entry}{看作}{9,7}{⽬、⼈}
  \begin{phonetics}{看作}{kan4 zuo4}[][HSK 6]
    \definition{v.}{considerar como; olhar como}
  \end{phonetics}
\end{entry}

\begin{entry}{看来}{9,7}{⽬、⽊}
  \begin{phonetics}{看来}{kan4 lai2}[][HSK 4]
    \definition{adv.}{parecer; parecer como se (ou embora); refere-se a um julgamento aproximado; expressa um julgamento por observação}
    \definition{v.}{ser considerado; na visão de alguém; na opinião de alguém; expressar a ideia aproximada que o locutor tem da situação}
  \end{phonetics}
\end{entry}

\begin{entry}{看到}{9,8}{⽬、⼑}
  \begin{phonetics}{看到}{kan4 dao4}[][HSK 1]
    \definition{v.}{ver; avistar}
  \end{phonetics}
\end{entry}

\begin{entry}{看法}{9,8}{⽬、⽔}
  \begin{phonetics}{看法}{kan4fa3}[][HSK 2]
    \definition[个,种,点]{s.}{opinião; perspectiva; (ponto de) vista; uma maneira de ver uma coisa | opinião desfavorável (ou crítica) sobre alguém}
  \end{phonetics}
\end{entry}

\begin{entry}{看待}{9,9}{⽬、⼻}
  \begin{phonetics}{看待}{kan4dai4}[][HSK 5]
    \definition{v.}{tratar; considerar; olhar com atenção; ter uma certa atitude ou visão em relação a alguém ou alguma coisa}
  \end{phonetics}
\end{entry}

\begin{entry}{看病}{9,10}{⽬、⽧}
  \begin{phonetics}{看病}{kan4 bing4}[][HSK 1]
    \definition{v.+compl.}{(de um médico) ver um paciente | (de um paciente) ver (consultar) um médico}
  \end{phonetics}
\end{entry}

\begin{entry}{看起来}{9,10,7}{⽬、⾛、⽊}
  \begin{phonetics}{看起来}{kan4 qi3 lai5}[][HSK 3]
    \definition{v.}{parecer; aparentar; dar a impressão de (ou como se)}
  \end{phonetics}
\end{entry}

\begin{entry}{看得见}{9,11,4}{⽬、⼻、⾒}
  \begin{phonetics}{看得见}{kan4 de5 jian4}[][HSK 6]
    \definition{adj.}{perceptível; visível; tangível}
  \end{phonetics}
\end{entry}

\begin{entry}{看得起}{9,11,10}{⽬、⼻、⾛}
  \begin{phonetics}{看得起}{kan4 de5 qi3}[][HSK 6]
    \definition{v.}{ter uma boa opinião sobre; pensar muito (ou muito) sobre}
  \end{phonetics}
\end{entry}

\begin{entry}{看望}{9,11}{⽬、⽉}
  \begin{phonetics}{看望}{kan4wang4}[][HSK 4]
    \definition{v.}{ver; visitar; ligar; dar uma olhada; ir até os pais, idosos, professores ou amigos para cumprimentá-los}
  \end{phonetics}
\end{entry}

\begin{entry}{看淡}{9,11}{⽬、⽔}
  \begin{phonetics}{看淡}{kan4dan4}
    \definition{v.}{considerar sem importância | ser indiferente a (fama, riqueza, etc.) | (de uma economia ou mercado) enfraquecer, ficar mais lento, diminuir a velocidade}
  \end{phonetics}
\end{entry}

\begin{entry}{看管}{9,14}{⽬、⽵}
  \begin{phonetics}{看管}{kan1 guan3}[][HSK 6]
    \definition{v.}{cuidar; atender | guardar; vigiar; ficar de olho em | assumir o comando; estar no comando}
  \end{phonetics}
\end{entry}

\begin{entry}{眞}{10}{⽬}
  \begin{phonetics}{眞}{zhen1}
    \variantof{真}
  \end{phonetics}
\end{entry}

\begin{entry}{眯}{11}{⽬}
  \begin{phonetics}{眯}{mi1}
    \definition{v.}{estreitar os olhos | esmagar | (dialeto) tirar uma soneca}
  \end{phonetics}
  \begin{phonetics}{眯}{mi2}
    \definition{v.}{cegar (como com poeira)}
  \end{phonetics}
\end{entry}

\begin{entry}{眼}{11}{⽬}
  \begin{phonetics}{眼}{yan3}[][HSK 2]
    \definition{clas.}{usado para grandes coisas ocas: poços, fogões, panelas, etc.}
    \definition[双,只]{s.}{olho; o órgão visual dos humanos ou animais | abertura; pequeno furo; pequeno buraco | ponto-chave; refere-se ao ponto-chave das coisas | armadilha; um termo do jogo Go que se refere a um espaço vazio cercado pelas peças de um jogador, onde o outro jogador não pode colocar uma peça, a menos que haja circunstâncias especiais | uma batida sem acento na música tradicional chinesa}
  \end{phonetics}
\end{entry}

\begin{entry}{眼光}{11,6}{⽬、⼉}
  \begin{phonetics}{眼光}{yan3guang1}[][HSK 5]
    \definition{s.}{olho; visão | visão; percepção; previsão; capacidade de observar e identificar coisas | vista; ponto de vista}
  \end{phonetics}
\end{entry}

\begin{entry}{眼花缭乱}{11,7,15,7}{⽬、⾋、⽷、⼄}
  \begin{phonetics}{眼花缭乱}{yan3hua1liao2luan4}
    \definition{v.}{ficar deslumbrado | deslumbrar}
  \end{phonetics}
\end{entry}

\begin{entry}{眼证}{11,7}{⽬、⾔}
  \begin{phonetics}{眼证}{yan3zheng4}
    \definition{s.}{testemunha ocular}
  \end{phonetics}
\end{entry}

\begin{entry}{眼里}{11,7}{⽬、⾥}
  \begin{phonetics}{眼里}{yan3 li3}[][HSK 4]
    \definition{s.}{nos olhos de uma pessoa; dentro de sua visão}
  \end{phonetics}
\end{entry}

\begin{entry}{眼泪}{11,8}{⽬、⽔}
  \begin{phonetics}{眼泪}{yan3 lei4}[][HSK 4]
    \definition[滴,行]{s.}{lágrimas; termo genérico para lágrimas; fluido incolor e transparente secretado pelas glândulas lacrimais no olho, que serve para proteger o olho}
  \end{phonetics}
\end{entry}

\begin{entry}{眼前}{11,9}{⽬、⼑}
  \begin{phonetics}{眼前}{yan3 qian2}[][HSK 3]
    \definition{adv.}{agora; (no) momento}
    \definition{s.}{diante dos olhos; diante de | agora; (no) momento}
  \end{phonetics}
\end{entry}

\begin{entry}{眼柄}{11,9}{⽬、⽊}
  \begin{phonetics}{眼柄}{yan3bing3}
    \definition{s.}{pedúnculo ocular (de crustáceo, etc.)}
  \end{phonetics}
\end{entry}

\begin{entry}{眼袋}{11,11}{⽬、⾐}
  \begin{phonetics}{眼袋}{yan3dai4}
    \definition{s.}{inchaço sob os olhos}
  \end{phonetics}
\end{entry}

\begin{entry}{眼睛}{11,13}{⽬、⽬}
  \begin{phonetics}{眼睛}{yan3jing5}[][HSK 2]
    \definition[双,只]{s.}{olho(s)}
  \end{phonetics}
\end{entry}

\begin{entry}{眼镜}{11,16}{⽬、⾦}
  \begin{phonetics}{眼镜}{yan3jing4}[][HSK 4]
    \definition[副]{s.}{óculos; óculos de grau}
  \end{phonetics}
\end{entry}

\begin{entry}{着}{11}{⽬}
  \begin{phonetics}{着}{zhao1}
    \definition{interj.}{tudo bem; tudo certo; \emph{O.K.}}
    \definition{s.}{uma jogada no xadrez | truque; meio; artifício; manobra; estratégia}
    \definition{v.}{colocar dentro; guardar}
  \end{phonetics}
  \begin{phonetics}{着}{zhao2}
    \definition{v.}{tocar (contato físico) | sentir; ser afetado por | queimar; acender | adormecer; cair no sono | acertar em cheio; ter sucesso em; usado após o verbo, indica que o objetivo foi alcançado ou que houve um resultado}
  \end{phonetics}
  \begin{phonetics}{着}{zhe5}[][HSK 1,4]
    \definition{part.}{adicionar a um verbo ou adjetivo para indicar uma ação ou estado contínuo | em frases que começam com uma palavra que indica um lugar, acrescente ao verbo para indicar um estado resultante | em frases imperativas, usado após verbos ou adjetivos para dar ênfase | adicionado após certos verbos, transforma-se em preposição}
    \definition{s.}{um movimento no xadrez |movimento; estratégia; estratagema}
  \end{phonetics}
  \begin{phonetics}{着}{zhuo2}
    \definition{v.}{vestir (roupas); vestir-se | tocar; entrar em contato com; aproximar-se de; (contato físico) | enviar; despachar | expressão usada em documentos oficiais antigos, indicando um tom de ordem | aplicar; usar; adicionar; anexar}
  \end{phonetics}
\end{entry}

\begin{entry}{着手}{11,4}{⽬、⼿}
  \begin{phonetics}{着手}{zhuo2shou3}
    \definition{v.}{colocar a mão nisso | estabelecer | começar uma tarefa}
  \end{phonetics}
\end{entry}

\begin{entry}{着火}{11,4}{⽬、⽕}
  \begin{phonetics}{着火}{zhao2huo3}[][HSK 4]
    \definition{v.}{pegar fogo; estar em chamas}
  \end{phonetics}
\end{entry}

\begin{entry}{着地}{11,6}{⽬、⼟}
  \begin{phonetics}{着地}{zhao2di4}
    \definition{v.}{pousar | tocar o chão}
  \end{phonetics}
\end{entry}

\begin{entry}{着花}{11,7}{⽬、⾋}
  \begin{phonetics}{着花}{zhao2hua1}
    \definition{v.}{florescer}
  \end{phonetics}
  \begin{phonetics}{着花}{zhuo2hua1}
    \definition{s.}{floração}
    \definition{v.}{florescer}
  \end{phonetics}
\end{entry}

\begin{entry}{着急}{11,9}{⽬、⼼}
  \begin{phonetics}{着急}{zhao2ji2}[][HSK 4]
    \definition{adj.}{ansioso; preocupado |}
    \definition{s.}{preocupação; ansiedade}
    \definition{v.+compl.}{preocupar-se | sentir-se ansioso | sentir uma sensação de urgência}
  \end{phonetics}
\end{entry}

\begin{entry}{着凉}{11,10}{⽬、⼎}
  \begin{phonetics}{着凉}{zhao2liang2}
    \definition{v.}{pegar um resfriado}
  \end{phonetics}
\end{entry}

\begin{entry}{着眼}{11,11}{⽬、⽬}
  \begin{phonetics}{着眼}{zhuo2yan3}
    \definition{v.}{ter seus olhos em (um objetivo) | ter algo em mente | concentrar-se}
  \end{phonetics}
\end{entry}

\begin{entry}{着装}{11,12}{⽬、⾐}
  \begin{phonetics}{着装}{zhuo2zhuang1}
    \definition{s.}{roupa | vestimenta}
    \definition{v.}{vestir}
  \end{phonetics}
\end{entry}

\begin{entry}{着想}{11,13}{⽬、⼼}
  \begin{phonetics}{着想}{zhuo2xiang3}
    \definition{v.}{considerar (as necessidades de outras pessoas) | pensar (para os outros)}
  \end{phonetics}
\end{entry}

\begin{entry}{着数}{11,13}{⽬、⽁}
  \begin{phonetics}{着数}{zhao1shu4}
    \definition{s.}{estratégia | movimento (no xadrez, no palco, nas artes marciais) | esquema | truque}
  \end{phonetics}
\end{entry}

\begin{entry}{著}{11}{⽬}
  \begin{phonetics}{著}{zhu4}
    \definition{adj.}{marcado; excelente; óbvio}
    \definition{s.}{livro; trabalho | nativo; pessoa/povo indígena; refere-se a pessoas que se estabeleceram em um lugar por gerações}
    \definition{v.}{mostrar; provar; revelar | escrever}
  \end{phonetics}
\end{entry}

\begin{entry}{著名}{11,6}{⽬、⼝}
  \begin{phonetics}{著名}{zhu4ming2}[][HSK 4]
    \definition{adj.}{famoso; bem conhecido; célebre}
  \end{phonetics}
\end{entry}

\begin{entry}{著作}{11,7}{⽬、⼈}
  \begin{phonetics}{著作}{zhu4zuo4}[][HSK 4]
    \definition[部]{s.}{obra; livro; escritos}
    \definition{v.}{escrever; usar palavras para expressar opiniões, conhecimentos, ideias, sentimentos, etc.}
  \end{phonetics}
\end{entry}

\begin{entry}{睡}{13}{⽬}
  \begin{phonetics}{睡}{shui4}[][HSK 1]
    \definition{v.}{dormir | deitar-se}
  \end{phonetics}
\end{entry}

\begin{entry}{睡衣}{13,6}{⽬、⾐}
  \begin{phonetics}{睡衣}{shui4yi1}
    \definition{s.}{pijamas | roupas de dormir}
  \end{phonetics}
\end{entry}

\begin{entry}{睡觉}{13,9}{⽬、⾒}
  \begin{phonetics}{睡觉}{shui4jiao4}[][HSK 1]
    \definition{v.+compl.}{dormir; ir para a cama; entrar em estado de sono}
  \end{phonetics}
\end{entry}

\begin{entry}{睡眠}{13,10}{⽬、⽬}
  \begin{phonetics}{睡眠}{shui4 mian2}[][HSK 5]
    \definition{s.}{sono; \emph{somnus}; sonolência}
  \end{phonetics}
\end{entry}

\begin{entry}{睡着}{13,11}{⽬、⽬}
  \begin{phonetics}{睡着}{shui4 zhao2}[][HSK 4]
    \definition{v.}{dormir; adormecer; cair no sono}
  \end{phonetics}
\end{entry}

\begin{entry}{睡懒觉}{13,16,9}{⽬、⼼、⾒}
  \begin{phonetics}{睡懒觉}{shui4lan3jiao4}
    \definition{v.}{levantar-se tarde | passar o tempo a dormir}
  \end{phonetics}
\end{entry}

\begin{entry}{瞧}{17}{⽬}
  \begin{phonetics}{瞧}{qiao2}[][HSK 5]
    \definition{v.}{ver; olhar | tratar; diagnosticar e tratar | ver; visitar; fazer uma visita}
  \end{phonetics}
\end{entry}

%%%%% EOF %%%%%

