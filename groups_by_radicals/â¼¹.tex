%%%
%%% Radical "⼹"
%%%

\section*{Radical 58: ``⼹'' (彑)}\addcontentsline{toc}{section}{Radical 58: ⼹、彑}

\begin{entry}{归}{5}{⼹}
  \begin{phonetics}{归}{gui1}[][HSK 4]
    \definition*{s.}{sobrenome Gui}
    \definition{s.}{divisão no ábaco com divisor de um dígito}
    \definition{v.}{retornar; voltar para; voltar (ou ir) | devolver algo a; dar de volta a | convergir; juntar-se | encarregar alguém de algo | atribuir a; pertencer a}
    \definition{v.aux.}{usado entre dois verbos idênticos, indicando que a ação não levou ao resultado correspondente}
  \end{phonetics}
\end{entry}

\begin{entry}{当}{6}{⼹}
  \begin{phonetics}{当}{dang1}[][HSK 2]
    \definition*{s.}{sobrenome Dang}
    \definition{adj.}{igual; adequado; compatível}
    \definition{interj.}{(onomatopéia) barulho metálico, som de um gongo ou sino}
    \definition{prep.}{na presença de alguém; na cara de alguém | exatamente em (um momento ou lugar); em algum momento, em algum lugar | na frente de alguém}
    \definition{s.}{topo; cume |uma lacuna no espaço ou no tempo; refere-se a um espaço ou intervalo de tempo}
    \definition{v.}{dever; ter que; dever ser | trabalhar como; servir como; ser; assumir; desempenhar a função de | suportar; aceitar; merecer | dirigir; gerenciar; estar no comando; ser responsável por;  presidir | conter; bloquear; segurar; reter; resistir}
  \end{phonetics}
  \begin{phonetics}{当}{dang4}
    \definition{adj.}{adequado; correto; apropriado | igual; o mesmo}
    \definition{pron.}{naquele mesmo (dia, etc.); refere-se ao momento em que algo aconteceu}
    \definition{s.}{algo penhorado; penhor; garantia; objetos físicos penhorados em casas de penhores}
    \definition{v.}{corresponder; ser igual a; combinar | tratar como; considerar como; tomar como | pensar que; achar que | penhorar; empréstimo com garantia real em uma loja de penhores}
  \end{phonetics}
\end{entry}

\begin{entry}{当中}{6,4}{⼹、⼁}
  \begin{phonetics}{当中}{dang1 zhong1}[][HSK 3]
    \definition{prep.}{no meio; no centro | entre; dentro}
  \end{phonetics}
\end{entry}

\begin{entry}{当代}{6,5}{⼹、⼈}
  \begin{phonetics}{当代}{dang1dai4}[][HSK 5]
    \definition{s.}{a era atual; a era contemporânea}
  \end{phonetics}
\end{entry}

\begin{entry}{当地}{6,6}{⼹、⼟}
  \begin{phonetics}{当地}{dang1di4}
    \definition{s.}{local; o lugar onde as pessoas e as coisas estão ou onde as coisas acontecem}
  \end{phonetics}
\end{entry}

\begin{entry}{当场}{6,6}{⼹、⼟}
  \begin{phonetics}{当场}{dang1chang3}[][HSK 5]
    \definition{adv.}{na hora; de imediato; na mesma hora}
  \end{phonetics}
\end{entry}

\begin{entry}{当年}{6,6}{⼹、⼲}
  \begin{phonetics}{当年}{dang1 nian2}[][HSK 5]
    \definition{adv.}{durante esse período; durante esse tempo | naquela época; naqueles dias | naqueles anos | nessa ocasião}
  \end{phonetics}
  \begin{phonetics}{当年}{dang4 nian2}
    \definition{s.}{no mesmo ano; naquele mesmo ano}
  \end{phonetics}
\end{entry}

\begin{entry}{当初}{6,7}{⼹、⾐}
  \begin{phonetics}{当初}{dang1chu1}[][HSK 3]
    \definition{s.}{no começo; originalmente; no início; em primeiro lugar; refere-se a algo que aconteceu no passado, seja em geral ou especificamente}
  \end{phonetics}
\end{entry}

\begin{entry}{当时}{6,7}{⼹、⽇}
  \begin{phonetics}{当时}{dang1shi2}[][HSK 2]
    \definition{s.}{naquela época; aquela ocasião; aquela vez; refere-se a algo que aconteceu no passado}
    \definition{v.}{ser o momento adequado; acontecer no momento certo}
  \end{phonetics}
  \begin{phonetics}{当时}{dang4shi2}
    \definition{adv.}{(depois de fazer algo ou algo acontecer) imediatamente; de imediato; agora mesmo}
  \end{phonetics}
\end{entry}

\begin{entry}{当前}{6,9}{⼹、⼑}
  \begin{phonetics}{当前}{dang1qian2}[][HSK 5]
    \definition{s.}{presente; atual}
    \definition{v.}{estar diante de alguém; estar frente a frente com alguém; na frente de, geralmente refere-se a uma situação perigosa}
  \end{phonetics}
\end{entry}

\begin{entry}{当选}{6,9}{⼹、⾡}
  \begin{phonetics}{当选}{dang1xuan3}[][HSK 5]
    \definition{v.}{ser eleito}
  \end{phonetics}
\end{entry}

\begin{entry}{当然}{6,12}{⼹、⽕}
  \begin{phonetics}{当然}{dang1ran2}[][HSK 3]
    \definition{adj.}{natural; verdadeiro; espontâneo}
    \definition{adv.}{sem dúvida; certamente; claro}
  \end{phonetics}
\end{entry}

\begin{entry}{录}{8}{⼹}
  \begin{phonetics}{录}{lu4}[][HSK 3]
    \definition{s.}{registro; cadastro; coleção; seleções}
    \definition{v.}{copiar; gravar; escrever; copiar; registrar | contratar; selecionar; empregar; adotar ou nomear | gravar em fita magnética}
  \end{phonetics}
\end{entry}

\begin{entry}{录取}{8,8}{⼹、⼜}
  \begin{phonetics}{录取}{lu4qu3}[][HSK 4]
    \definition{v.}{aceitar; admitir; recrutar; entrar; matricular (os aprovados no exame)}
  \end{phonetics}
\end{entry}

\begin{entry}{录音}{8,9}{⼹、⾳}
  \begin{phonetics}{录音}{lu4yin1}[][HSK 3]
    \definition[段,个]{s.}{gravação de som; som gravado com equipamento especializado}
    \definition{v.+compl.}{gravar; converter o som em sinal elétrico e, em seguida, gravá-lo por meios mecânicos, ópticos ou eletromagnéticos}
  \end{phonetics}
\end{entry}

\begin{entry}{录音机}{8,9,6}{⼹、⾳、⽊}
  \begin{phonetics}{录音机}{lu4yin1ji1}
    \definition[台]{s.}{gravador de áudio}
  \end{phonetics}
\end{entry}

\begin{entry}{录像机}{8,13,6}{⼹、⼈、⽊}
  \begin{phonetics}{录像机}{lu4xiang4ji1}
    \definition[台]{s.}{gravador de vídeo | VCR}
  \end{phonetics}
\end{entry}

\begin{entry}{录像带}{8,13,9}{⼹、⼈、⼱}
  \begin{phonetics}{录像带}{lu4xiang4dai4}
    \definition[盘]{s.}{video-cassete}
  \end{phonetics}
\end{entry}

%%%%% EOF %%%%%

