%%%
%%% Radical "⽰"
%%%

\section*{Radical 113: ``⽰'' (礻)}\addcontentsline{toc}{section}{Radical 113: ⽰、礻}

\begin{entry}{示}{5}{⽰}
  \begin{phonetics}{示}{shi4}
    \definition*{s.}{sobrenome Shi}
    \definition{s.}{(sua) carta  | missiva; instruções; palavras ou escritos para subordinados ou gerações mais jovens}
    \definition{v.}{mostrar; notificar; instruir | indicar; significar; mostrar ou apontar, fazer conhecido}
  \end{phonetics}
\end{entry}

\begin{entry}{示范}{5,9}{⽰、⾋}
  \begin{phonetics}{示范}{shi4fan4}[][HSK 5]
    \definition{v.}{demonstrar; dar o exemplo; criar um modelo que todos possam aprender}
  \end{phonetics}
\end{entry}

\begin{entry}{礼}{5}{⽰}
  \begin{phonetics}{礼}{li3}[][HSK 5]
    \definition*{s.}{sobrenome Li}
    \definition[份]{s.}{observâncias cerimoniais em geral; cerimônia; rito | cortesia; etiqueta; boas maneiras | presente; oferta}
  \end{phonetics}
\end{entry}

\begin{entry}{礼节}{5,5}{⽰、⾋}
  \begin{phonetics}{礼节}{li3jie2}
    \definition{s.}{protocolo | cerimônia | etiqueta}
  \end{phonetics}
\end{entry}

\begin{entry}{礼让}{5,5}{⽰、⾔}
  \begin{phonetics}{礼让}{li3rang4}
    \definition{s.}{cortesia}
    \definition{v.}{mostrar consideração por (outros) | ceder a (outro veículo, etc.)}
  \end{phonetics}
\end{entry}

\begin{entry}{礼物}{5,8}{⽰、⽜}
  \begin{phonetics}{礼物}{li3wu4}[][HSK 2]
    \definition[份,件,个,分,些]{s.}{presente; lembrança; itens oferecidos como forma de respeito ou celebração, referindo-se de maneira geral a itens oferecidos como presente}
  \end{phonetics}
\end{entry}

\begin{entry}{礼拜}{5,9}{⽰、⼿}
  \begin{phonetics}{礼拜}{li3 bai4}[][HSK 5]
    \definition[个]{s.}{dia da semana; usado em conjunto com 一, 二, 三, 四, 五, 六, 日(或天, indica um dia específico da semana | semana; referência à semana | domingo}
    \definition{v.}{prestar homenagem aos deuses que veneram; rezar; orar}
  \end{phonetics}
\end{entry}

\begin{entry}{礼貌}{5,14}{⽰、⾘}
  \begin{phonetics}{礼貌}{li3mao4}[][HSK 5]
    \definition{adj.}{educado; descreve uma pessoa que fala e age respeitando os outros, sem arrogância, de acordo com as exigências das relações sociais}
    \definition{s.}{cortesia; educação; boas maneiras}
  \end{phonetics}
\end{entry}

\begin{entry}{社}{7}{⽰}
  \begin{phonetics}{社}{she4}[][HSK 5]
    \definition[个]{s.}{agência; sociedade; órgão organizado; organização; comunidade | comuna popular | o deus da terra, sacrifícios a ele ou altares para tais sacrifícios; na antiguidade, o deus da terra, o local onde ele era venerado, o dia da veneração e o ritual eram chamados de 社 | agência de notícias |  imprensa}
  \end{phonetics}
\end{entry}

\begin{entry}{社区}{7,4}{⽰、⼖}
  \begin{phonetics}{社区}{she4qu1}[][HSK 5]
    \definition{s.}{bairro; comunidade residencial; bairros da cidade, divididos de acordo com a localização geográfica | distrito; comunidade (para pessoas da mesma classe social, etc.) ; lugar onde pessoas com características comuns, como classe social, vivem juntas}
  \end{phonetics}
\end{entry}

\begin{entry}{社会}{7,6}{⽰、⼈}
  \begin{phonetics}{社会}{she4hui4}[][HSK 3]
    \definition[个,种]{s.}{sociedade; em um determinado estágio do desenvolvimento histórico, a relação geral entre as pessoas nas atividades de produção | comunidade; geralmente se refere a um grupo de pessoas que estão conectadas por atividades comuns}
  \end{phonetics}
\end{entry}

\begin{entry}{祅}{8}{⽰}
  \begin{phonetics}{祅}{yao1}
    \definition{s.}{espírito maligno | \emph{goblin} | bruxaria}
    \variantof{妖}
  \end{phonetics}
\end{entry}

\begin{entry}{祖国}{9,8}{⽰、⼞}
  \begin{phonetics}{祖国}{zu3guo2}
    \definition{s.}{pátria | terra natal}
  \end{phonetics}
\end{entry}

\begin{entry}{祝}{9}{⽰}
  \begin{phonetics}{祝}{zhu4}[][HSK 3]
    \definition*{s.}{sobrenome Zhu}
    \definition{v.}{expressar bons votos; desejar; abençoar | rezar aos deuses ou espíritos para obter bênçãos}
  \end{phonetics}
\end{entry}

\begin{entry}{祝好}{9,6}{⽰、⼥}
  \begin{phonetics}{祝好}{zhu4hao3}
    \definition{expr.}{desejo-lhe tudo de melhor! (ao encerrar uma correspondência)}
  \end{phonetics}
\end{entry}

\begin{entry}{祝寿}{9,7}{⽰、⼨}
  \begin{phonetics}{祝寿}{zhu4shou4}
    \definition{v.}{dar parabéns pelo aniversário (a uma pessoa idosa)}
  \end{phonetics}
\end{entry}

\begin{entry}{祝贺}{9,9}{⽰、⾙}
  \begin{phonetics}{祝贺}{zhu4he4}[][HSK 5]
    \definition[个]{s.}{congratulações; felicitações}
    \definition{v.}{congratular; felicitar; parabenizar}
  \end{phonetics}
\end{entry}

\begin{entry}{祝酒}{9,10}{⽰、⾣}
  \begin{phonetics}{祝酒}{zhu4jiu3}
    \definition{v.}{parabenizar e fazer um brinde | brindar}
  \end{phonetics}
\end{entry}

\begin{entry}{祝颂}{9,10}{⽰、⾴}
  \begin{phonetics}{祝颂}{zhu4song4}
    \definition{v.}{expressar bons desejos}
  \end{phonetics}
\end{entry}

\begin{entry}{祝祷}{9,11}{⽰、⽰}
  \begin{phonetics}{祝祷}{zhu4dao3}
    \definition{v.}{rezar | orar}
  \end{phonetics}
\end{entry}

\begin{entry}{祝谢}{9,12}{⽰、⾔}
  \begin{phonetics}{祝谢}{zhu4xie4}
    \definition{v.}{agradecer | dar parabéns}
  \end{phonetics}
\end{entry}

\begin{entry}{祝福}{9,13}{⽰、⽰}
  \begin{phonetics}{祝福}{zhu4fu2}[][HSK 4]
    \definition[个]{s.}{bênção; benzedura; benzimento; originalmente, referia-se à oração para obter as bênçãos de Deus, mas, mais tarde, refere-se a desejar paz e felicidade às pessoas}
    \definition{v.}{desejar boa sorte a alguém}
  \end{phonetics}
\end{entry}

\begin{entry}{祝愿}{9,14}{⽰、⽕}
  \begin{phonetics}{祝愿}{zhu4yuan4}
    \definition{v.}{desejar}
  \end{phonetics}
\end{entry}

\begin{entry}{神}{9}{⽰}
  \begin{phonetics}{神}{shen2}[][HSK 5]
    \definition*{s.}{sobrenome Shen}
    \definition*{s.}{Deus}
    \definition{adj.}{inteligente; esperto | mágico; sobrenatural}
    \definition[个,位,尊]{s.}{divindade; deidade | espírito; mente; refere-se ao espírito, energia ou atenção de uma pessoa | olhar; expressão; expressões que refletem o estado interior}
  \end{phonetics}
\end{entry}

\begin{entry}{神奇}{9,8}{⽰、⼤}
  \begin{phonetics}{神奇}{shen2qi2}[][HSK 5]
    \definition{adj.}{mágico; peculiar; místico; milagroso; algo que parece muito novo; algo que ninguém imaginaria, mas que geralmente traz bons resultados}
    \definition{s.}{mágica; milagre}
  \end{phonetics}
\end{entry}

\begin{entry}{神明}{9,8}{⽰、⽇}
  \begin{phonetics}{神明}{shen2ming2}
    \definition{s.}{divindades | deuses}
  \end{phonetics}
\end{entry}

\begin{entry}{神经}{9,8}{⽰、⽷}
  \begin{phonetics}{神经}{shen2jing1}[][HSK 5]
    \definition{adj.}{excêntrico; estranho; peculiar; descreve anormalidade neurológica}
    \definition{s.}{nervo; um tipo de tecido presente no corpo humano ou animal que conecta o cérebro aos órgãos, transmitindo as sensações ao cérebro e as informações do cérebro aos órgãos}
  \end{phonetics}
\end{entry}

\begin{entry}{神经病学}{9,8,10,8}{⽰、⽷、⽧、⼦}
  \begin{phonetics}{神经病学}{shen2jing1bing4xue2}
    \definition{s.}{neurologia}
  \end{phonetics}
\end{entry}

\begin{entry}{神经病的}{9,8,10,8}{⽰、⽷、⽧、⽩}
  \begin{phonetics}{神经病的}{shen2jing1bing4de5}
    \definition{adj.}{neurótico}
  \end{phonetics}
\end{entry}

\begin{entry}{神话}{9,8}{⽰、⾔}
  \begin{phonetics}{神话}{shen2hua4}[][HSK 4]
    \definition[个]{s.}{mito; mitologia; conto de fadas; refere-se a deuses e deusas lendários e histórias de heróis antigos deificados | lorota; refere-se a alegações ridículas e infundadas}
  \end{phonetics}
\end{entry}

\begin{entry}{神秘}{9,10}{⽰、⽲}
  \begin{phonetics}{神秘}{shen2mi4}[][HSK 4]
    \definition{adj.}{místico; misterioso}
  \end{phonetics}
\end{entry}

\begin{entry}{神兽}{9,11}{⽰、⼋}
  \begin{phonetics}{神兽}{shen2shou4}
    \definition{s.}{animal mitológico | fera}
  \end{phonetics}
\end{entry}

\begin{entry}{神情}{9,11}{⽰、⼼}
  \begin{phonetics}{神情}{shen2 qing2}[][HSK 5]
    \definition{s.}{aparência; expressão; atividades internas reveladas no rosto das pessoas}
  \end{phonetics}
\end{entry}

\begin{entry}{神器}{9,16}{⽰、⼝}
  \begin{phonetics}{神器}{shen2qi4}
    \definition{s.}{objeto mágico | objeto simbólico do poder imperial | arma fina | ferramenta muito útil}
  \end{phonetics}
\end{entry}

\begin{entry}{票}{11}{⽰}
  \begin{phonetics}{票}{piao4}[][HSK 1]
    \definition{clas.}{para grupos, lotes, transações comerciais}
    \definition[张]{s.}{bilhete; passagem; ingresso | cédula | nota bancária; conta | pessoa mantida em cativeiro por sequestradores para obter resgate; refém | apresentação amadora (de ópera de Pequim, etc.); peças teatrais amadoras}
    \definition{v.}{atuar como amador (na ópera de Pequim)}
  \end{phonetics}
\end{entry}

\begin{entry}{票价}{11,6}{⽰、⼈}
  \begin{phonetics}{票价}{piao4 jia4}[][HSK 3]
    \definition[个]{s.}{o preço de um ingresso; taxa de admissão; taxa de entrada}
  \end{phonetics}
\end{entry}

\begin{entry}{祸}{11}{⽰}
  \begin{phonetics}{祸}{huo4}
    \definition[场]{s.}{infortúnio; desastre; calamidade (oposto de 福) | desgraça; catástrofe}
    \definition{v.}{trazer desastre; arruinar | causar problemas}
  \seealsoref{福}{fu2}
  \end{phonetics}
\end{entry}

\begin{entry}{禅}{12}{⽰}
  \begin{phonetics}{禅}{chan2}
    \definition{s.}{contemplação prolongada e intensa; meditação profunda | budista; refere-se geralmente a coisas relacionadas ao budismo}
  \end{phonetics}
  \begin{phonetics}{禅}{shan4}
    \definition{v.}{abdicar e entregar a coroa a outra pessoa}
  \end{phonetics}
\end{entry}

\begin{entry}{禁}{13}{⽰}
  \begin{phonetics}{禁}{jin4}
    \definition*{s.}{sobrenome Jin}
    \definition{s.}{um tabu; assuntos não permitidos por lei ou costume | área proibida | residência real; o lugar onde o imperador viveu nos tempos antigos}
    \definition{v.}{proibir; banir | aprisionar; deter}
  \end{phonetics}
\end{entry}

\begin{entry}{禁止}{13,4}{⽰、⽌}
  \begin{phonetics}{禁止}{jin4zhi3}[][HSK 4]
    \definition{v.}{banir; proibir; interditar}
  \end{phonetics}
\end{entry}

\begin{entry}{福}{13}{⽰}
  \begin{phonetics}{福}{fu2}[][HSK 3]
    \definition*{s.}{sobrenome Fu}
    \definition*{s.}{Província Fujian}
    \definition{s.}{benção; felicidade; boa sorte; boa fortuna; sorte (oposto de 祸)}
    \definition{v.}{(de uma mulher) fazer uma reverência; antigamente, as mulheres faziam a reverência 万福 (colocando as duas mãos na cintura do mesmo lado e dobrando ligeiramente os joelhos)}
  \seealsoref{祸}{huo4}
  \seealsoref{万福}{wan4fu2}
  \end{phonetics}
\end{entry}

\begin{entry}{福克斯}{13,7,12}{⽰、⼗、⽄}
  \begin{phonetics}{福克斯}{fu2ke4si1}
    \definition*{s.}{Fox (empresa de mídia) | Focus (automóvel fabricado pela Ford)}
  \end{phonetics}
\end{entry}

\begin{entry}{福利}{13,7}{⽰、⼑}
  \begin{phonetics}{福利}{fu2li4}[][HSK 5]
    \definition{s.}{bem-estar; benefícios materiais}
    \definition{v.}{melhorar suas condições de vida; facilitar a vida}
  \end{phonetics}
\end{entry}

\begin{entry}{福泽}{13,8}{⽰、⽔}
  \begin{phonetics}{福泽}{fu2ze2}
    \definition{s.}{boa sorte}
  \end{phonetics}
\end{entry}

%%%%% EOF %%%%%

