%%%
%%% Radical "⼍"
%%%

\section*{Radical 14: ``⼍''}\addcontentsline{toc}{section}{Radical 14: ⼍}

\begin{Entry}{写}{5}{⼍}
  \begin{Phonetics}{写}{xie3}[][HSK 1]
    \definition{v.}{escrever | compor; escrever (como autor, repórter, etc.) | descrever; retratar | pintar; desenhar | expressar a imagem das coisas através da linguagem e da escrita | desenhar (pintura)}
  \end{Phonetics}
\end{Entry}

\begin{Entry}{写字}{5,6}{⼍、⼦}
  \begin{Phonetics}{写字}{xie3zi4}
    \definition{v.}{escrever (à mão) | praticar caligrafia}
  \end{Phonetics}
\end{Entry}

\begin{Entry}{写字台}{5,6,5}{⼍、⼦、⼝}
  \begin{Phonetics}{写字台}{xie3 zi4 tai2}[][HSK 6]
    \definition[个,张]{s.}{escrivaninha; secretária; escrivaninha de escrever; uma mesa retangular usada para escrever e trabalhar, com gavetas e algumas com pequenos armários}
  \end{Phonetics}
\end{Entry}

\begin{Entry}{写字匠}{5,6,6}{⼍、⼦、⼕}
  \begin{Phonetics}{写字匠}{xie3zi4 jiang4}
    \definition{s.}{calígrafo}
  \end{Phonetics}
\end{Entry}

\begin{Entry}{写字楼}{5,6,13}{⼍、⼦、⽊}
  \begin{Phonetics}{写字楼}{xie3 zi4 lou2}[][HSK 6]
    \definition{s.}{prédio de escritórios}
  \end{Phonetics}
\end{Entry}

\begin{Entry}{写作}{5,7}{⼍、⼈}
  \begin{Phonetics}{写作}{xie3zuo4}[][HSK 3]
    \definition{v.}{escrever artigos; escrever livros, etc.; também se refere especificamente à criação de obras literárias}
  \end{Phonetics}
\end{Entry}

\begin{Entry}{写真}{5,10}{⼍、⼗}
  \begin{Phonetics}{写真}{xie3zhen1}
    \definition{s.}{retrato}
    \definition{v.}{descrever algo com precisão}
  \end{Phonetics}
\end{Entry}

\begin{Entry}{写意}{5,13}{⼍、⼼}
  \begin{Phonetics}{写意}{xie3yi4}
    \definition{s.}{estilo de pintura chinesa à mão livre, caracterizado por traços ousados em vez de detalhes precisos}
    \definition{v.}{sugerir (em vez de descrever em detalhes)}
  \end{Phonetics}
  \begin{Phonetics}{写意}{xie4yi4}
    \definition{adj.}{confortável | agradável | relaxado}
  \end{Phonetics}
\end{Entry}

\begin{Entry}{写照}{5,13}{⼍、⽕}
  \begin{Phonetics}{写照}{xie3zhao4}
    \definition{s.}{retrato}
  \end{Phonetics}
\end{Entry}

\begin{Entry}{军}{6}{⼍}
  \begin{Phonetics}{军}{jun1}
    \definition*{s.}{Sobrenome Jun}
    \definition{s.}{forças armadas; exército; tropas | exército; contingente; muitas pessoas participando de uma atividade | exército; unidades militares}
  \end{Phonetics}
\end{Entry}

\begin{Entry}{军人}{6,2}{⼍、⼈}
  \begin{Phonetics}{军人}{jun1 ren2}[][HSK 5]
    \definition[名,位,个]{s.}{soldado; militar; pessoal militar; pessoas com status militar; pessoas servindo nas forças armadas}
  \end{Phonetics}
\end{Entry}

\begin{Entry}{军队}{6,4}{⼍、⾩}
  \begin{Phonetics}{军队}{jun1dui4}[][HSK 6]
    \definition[支,个]{s.}{forças armadas; exército; tropas}
  \end{Phonetics}
\end{Entry}

\begin{Entry}{军事}{6,8}{⼍、⼅}
  \begin{Phonetics}{军事}{jun1shi4}[][HSK 6]
    \definition{s.}{militar; assuntos militares; assuntos relativos aos militares e à guerra}
  \end{Phonetics}
\end{Entry}

\begin{Entry}{军舰}{6,10}{⼍、⾈}
  \begin{Phonetics}{军舰}{jun1 jian4}[][HSK 6]
    \definition[艘,只]{s.}{navio de guerra; embarcação naval | \emph{warcraft}; um termo geral para embarcações militares equipadas com armas e equipamentos que podem executar missões de combate, incluindo principalmente navios de guerra, cruzadores, contratorpedeiros, porta-aviões, submarinos, torpedeiros, etc.}
  \end{Phonetics}
\end{Entry}

\begin{Entry}{军装}{6,12}{⼍、⾐}
  \begin{Phonetics}{军装}{jun1zhuang1}
    \definition{s.}{uniforme militar}
  \end{Phonetics}
\end{Entry}

\begin{Entry}{农}{6}{⼍}
  \begin{Phonetics}{农}{nong2}
    \definition*{s.}{Sobrenome Nong}
    \definition{s.}{agricultura; criação de animais | camponês; fazendeiro}
  \end{Phonetics}
\end{Entry}

\begin{Entry}{农业}{6,5}{⼍、⼀}
  \begin{Phonetics}{农业}{nong2ye4}[][HSK 3]
    \definition{s.}{agricultura}
  \end{Phonetics}
\end{Entry}

\begin{Entry}{农民}{6,5}{⼍、⽒}
  \begin{Phonetics}{农民}{nong2min2}[][HSK 3]
    \definition[个,位,名,些]{s.}{fazendeiro; camponês; campesinato; trabalhadores que participam da produção agrícola há muito tempo}
  \end{Phonetics}
\end{Entry}

\begin{Entry}{农产品}{6,6,9}{⼍、⼇、⼝}
  \begin{Phonetics}{农产品}{nong2 chan3 pin3}[][HSK 5]
    \definition[批]{s.}{produtos agrícolas}
  \end{Phonetics}
\end{Entry}

\begin{Entry}{农村}{6,7}{⼍、⽊}
  \begin{Phonetics}{农村}{nong2cun1}[][HSK 3]
    \definition{s.}{aldeia; campo; área rural; locais onde vivem os trabalhadores principalmente dedicados à produção agrícola}
  \end{Phonetics}
\end{Entry}

\begin{Entry}{冠}{9}{⼍}
  \begin{Phonetics}{冠}{guan1}
    \definition{s.}{chapéu | corona; coroa; copa | crista}
  \end{Phonetics}
  \begin{Phonetics}{冠}{guan4}
    \definition*{s.}{Sobrenome Guan}
    \definition{s.}{primeiro lugar; o melhor; classificado em primeiro lugar}
    \definition{v.}{colocar um chapéu (boné) | preceder com (por); coroar com; adicionar um nome ou texto na frente}
  \end{Phonetics}
\end{Entry}

\begin{Entry}{冠军}{9,6}{⼍、⼍}
  \begin{Phonetics}{冠军}{guan4jun1}[][HSK 5]
    \definition[位,名,项,个]{s.}{campeão; medalhista de ouro; primeiro lugar em esportes e outras competições}
  \end{Phonetics}
\end{Entry}

%%%%% EOF %%%%%

