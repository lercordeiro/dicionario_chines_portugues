%%%
%%% Radical "⼍"
%%%

\section*{Radical 14: ``⼍''}\addcontentsline{toc}{section}{Radical 14: ⼍}

\begin{entry}{写}{5}{⼍}
  \begin{phonetics}{写}{xie3}[][HSK 1]
    \definition{v.}{escrever | compor; escrever (como autor, repórter, etc.) | descrever; retratar | pintar; desenhar | expressar a imagem das coisas através da linguagem e da escrita | desenhar (pintura)}
  \end{phonetics}
\end{entry}

\begin{entry}{写字}{5,6}{⼍、⼦}
  \begin{phonetics}{写字}{xie3zi4}
    \definition{v.}{escrever (à mão) | praticar caligrafia}
  \end{phonetics}
\end{entry}

\begin{entry}{写字匠}{5,6,6}{⼍、⼦、⼕}
  \begin{phonetics}{写字匠}{xie3zi4 jiang4}
    \definition{s.}{calígrafo}
  \end{phonetics}
\end{entry}

\begin{entry}{写作}{5,7}{⼍、⼈}
  \begin{phonetics}{写作}{xie3zuo4}[][HSK 3]
    \definition{s.}{escrita; redação; composição}
    \definition{v.}{escrever artigos; escrever livros, etc.; também se refere especificamente à criação de obras literárias}
  \end{phonetics}
\end{entry}

\begin{entry}{写真}{5,10}{⼍、⼗}
  \begin{phonetics}{写真}{xie3zhen1}
    \definition{s.}{retrato}
    \definition{v.}{descrever algo com precisão}
  \end{phonetics}
\end{entry}

\begin{entry}{写意}{5,13}{⼍、⼼}
  \begin{phonetics}{写意}{xie3yi4}
    \definition{s.}{estilo de pintura chinesa à mão livre, caracterizado por traços ousados em vez de detalhes precisos}
    \definition{v.}{sugerir (em vez de descrever em detalhes)}
  \end{phonetics}
  \begin{phonetics}{写意}{xie4yi4}
    \definition{adj.}{confortável | agradável | relaxado}
  \end{phonetics}
\end{entry}

\begin{entry}{写照}{5,13}{⼍、⽕}
  \begin{phonetics}{写照}{xie3zhao4}
    \definition{s.}{retrato}
  \end{phonetics}
\end{entry}

\begin{entry}{军人}{6,2}{⼍、⼈}
  \begin{phonetics}{军人}{jun1 ren2}[][HSK 5]
    \definition{s.}{soldado; militar; pessoal militar; pessoas com status militar; pessoas servindo nas forças armadas}
  \end{phonetics}
\end{entry}

\begin{entry}{军装}{6,12}{⼍、⾐}
  \begin{phonetics}{军装}{jun1zhuang1}
    \definition{s.}{uniforme militar}
  \end{phonetics}
\end{entry}

\begin{entry}{农业}{6,5}{⼍、⼀}
  \begin{phonetics}{农业}{nong2ye4}[][HSK 3]
    \definition{s.}{agricultura; lavoura}
  \end{phonetics}
\end{entry}

\begin{entry}{农民}{6,5}{⼍、⽒}
  \begin{phonetics}{农民}{nong2min2}[][HSK 3]
    \definition[个,位]{s.}{fazendeiro; camponês; campesinato}
  \end{phonetics}
\end{entry}

\begin{entry}{农产品}{6,6,9}{⼍、⼇、⼝}
  \begin{phonetics}{农产品}{nong2 chan3 pin3}[][HSK 5]
    \definition{s.}{produtos agrícolas}
  \end{phonetics}
\end{entry}

\begin{entry}{农村}{6,7}{⼍、⽊}
  \begin{phonetics}{农村}{nong2cun1}[][HSK 3]
    \definition[个]{s.}{aldeia; campo; área rural}
  \end{phonetics}
\end{entry}

\begin{entry}{冠}{9}{⼍}
  \begin{phonetics}{冠}{guan1}
    \definition{s.}{chapéu | coroa | brasão | boné}
  \end{phonetics}
  \begin{phonetics}{冠}{guan4}
    \definition*{s.}{sobrenome Guan}
    \definition{v.}{colocar um chapéu | ser o primeiro | dublar}
  \end{phonetics}
\end{entry}

\begin{entry}{冠军}{9,6}{⼍、⼍}
  \begin{phonetics}{冠军}{guan4jun1}[][HSK 5]
    \definition[个]{s.}{campeão; medalhista de ouro; primeiro lugar em esportes e outras competições}
  \end{phonetics}
\end{entry}

%%%%% EOF %%%%%

