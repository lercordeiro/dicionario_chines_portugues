%%%
%%% Radical "⾧"
%%%

\section*{Radical 168: ``⾧'' (长、镸)}\addcontentsline{toc}{section}{Radical 168: ⾧、长、镸}

\begin{Entry}{长}{4}{⾧}[Kangxi 168]
  \begin{Phonetics}{长}{chang2}[][HSK 2]
    \definition*{s.}{Sobrenome Chang}
    \definition{adj.}{longo; comprido; a distância entre uma extremidade e outra é grande (em oposição a 短 ) | para sempre; duradouro; a distância entre o ponto inicial e o ponto final de um determinado período é grande | excesso; o que sobra; o que é desnecessário}
    \definition{s.}{comprimento; na direção horizontal, a distância percorrida por um objeto de uma extremidade à outra | ponto forte; especialidade; especialização em determinada área; vantagem}
    \definition{v.}{ser bom em; ser forte em; ser muito bom em algo; ser especialista em algo}
  \seealsoref{短}{duan3}
  \end{Phonetics}
  \begin{Phonetics}{长}{zhang3}[][HSK 2,6]
    \definition{adj.}{mais velho; ancião; sênior; mais antigo; o primeiro da fila}
    \definition{s.}{ancião; pessoas mais velhas ou de posição social mais elevada | chefe; líder; dirigente; responsável}
    \definition{v.}{crescer; desenvolver-se | surgir; começar a crescer; formar-se; nascer em pessoas, animais, plantas ou objetos (algo) | adquirir; melhorar; aumentar; aumentar conhecimento, capacidade, etc.; tornar-se cada vez mais ou cada vez melhor}
  \end{Phonetics}
\end{Entry}

\begin{Entry}{长久}{4,3}{⾧、⼃}
  \begin{Phonetics}{长久}{chang2 jiu3}[][HSK 6]
    \definition{adj.}{longo; permanente; duradouro; por muito tempo; já faz muito tempo}
  \end{Phonetics}
\end{Entry}

\begin{Entry}{长大}{4,3}{⾧、⼤}
  \begin{Phonetics}{长大}{zhang3 da4}[][HSK 2]
    \definition{v.}{crescer; ser criado; um estado de maturidade física ou mental completa}
  \end{Phonetics}
\end{Entry}

\begin{Entry}{长处}{4,5}{⾧、⼡}
  \begin{Phonetics}{长处}{chang2 chu4}[][HSK 3]
    \definition[个]{s.}{forte; boas qualidades; pontos fortes; especialização em determinada área}
  \end{Phonetics}
\end{Entry}

\begin{Entry}{长寿}{4,7}{⾧、⼨}
  \begin{Phonetics}{长寿}{chang2 shou4}[][HSK 5]
    \definition{adj.}{vida longa; longevidade}
  \end{Phonetics}
\end{Entry}

\begin{Entry}{长远}{4,7}{⾧、⾡}
  \begin{Phonetics}{长远}{chang2 yuan3}[][HSK 6]
    \definition{adj.}{longo prazo; longo alcance; duradouro; há muito tempo (referindo-se ao futuro); crônico}
  \end{Phonetics}
\end{Entry}

\begin{Entry}{长城}{4,9}{⾧、⼟}
  \begin{Phonetics}{长城}{chang2cheng2}[][HSK 3]
    \definition*[段,座,条]{s.}{A Grande Muralha da China; também é usado como metáfora para uma força poderosa e inabalável, um obstáculo intransponível, etc.}
  \end{Phonetics}
\end{Entry}

\begin{Entry}{长度}{4,9}{⾧、⼴}
  \begin{Phonetics}{长度}{chang2 du4}[][HSK 5]
    \definition{s.}{comprimento; extensão; distância entre dois pontos}
  \end{Phonetics}
\end{Entry}

\begin{Entry}{长途}{4,10}{⾧、⾡}
  \begin{Phonetics}{长途}{chang2tu2}[][HSK 4]
    \definition{adj.}{de longa distância; longe}
    \definition[段,次,程]{s.}{longa-distância; referindo-se especificamente a chamadas telefônicas de longa distância ou ônibus de longa distância}
  \end{Phonetics}
\end{Entry}

\begin{Entry}{长假}{4,11}{⾧、⼈}
  \begin{Phonetics}{长假}{chang2 jia4}[][HSK 6]
    \definition{s.}{longa licença de ausência | feriado prolongado | (datado) renúncia | férias longas | refere-se a um feriado nacional de uma semana na China, começando em 1º de maio e 1º de outubro}
  \end{Phonetics}
\end{Entry}

\begin{Entry}{长颈鹿}{4,11,11}{⾧、⾴、⿅}
  \begin{Phonetics}{长颈鹿}{chang2jing3lu4}
    \definition[只]{s.}{girafa}
  \end{Phonetics}
\end{Entry}

\begin{Entry}{长期}{4,12}{⾧、⽉}
  \begin{Phonetics}{长期}{chang2 qi1}[][HSK 3]
    \definition{adj.}{secular; longo prazo; longo alcance; durante um longo período de tempo}
    \definition{s.}{longo prazo; por muito tempo}
  \end{Phonetics}
\end{Entry}

\begin{Entry}{长短}{4,12}{⾧、⽮}
  \begin{Phonetics}{长短}{chang2 duan3}[][HSK 6]
    \definition{s.}{comprimento | acidente; infortúnio | certo e errado; pontos fortes e fracos}
  \end{Phonetics}
\end{Entry}

\begin{Entry}{长跑}{4,12}{⾧、⾜}
  \begin{Phonetics}{长跑}{chang2 pao3}[][HSK 6]
    \definition{s.}{corrida de longa distância (oposto a 短跑)}
  \seealsoref{短跑}{duan3 pao3}
  \end{Phonetics}
\end{Entry}

%%%%% EOF %%%%%

