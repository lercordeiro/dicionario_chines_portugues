%%%
%%% Radical "⼤"
%%%
\section*{Radical 37: ``⼤''}\addcontentsline{toc}{section}{Radical 37: ⼤}\addcontentsline{loh}{figure}{\#\#\#\# 37: ⼤}

%%%%%%%%%% 大 %%%%%%%%%%
\subsection*{大}\addcontentsline{loh}{figure}{大}

\begin{Entry}{大}{3}{⼤}[Kangxi 37]
  \begin{Phonetics}{大}{da4}[][HSK 1]
    \definition*{s.}{Sobrenome: Da}
    \definition{adj.}{grande; amplo; grande em volume, área, etc. | mais velho; em primeiro lugar no ranking | tamanho; descreve o grau de grandeza | usado em certas épocas do ano, condições climáticas, feriados ou antes de um determinado momento, para enfatizar | o tempo mais distante; há muito tempo}
    \definition{adv.}{grandemente; totalmente; expressa um grau muito profundo | não muito; não frequentemente; usado após 不, indica um grau baixo ou poucas vezes}
    \definition{s.}{adulto; crescido; pessoas idosas | pai | irmão do pai de alguém; tio}
  \seealsoref{不}{bu4}
  \end{Phonetics}
  \begin{Phonetics}{大}{dai4}
    \definition{s.}{usado em 大夫: médico, doutor | usado em 大王: grande rei}
  \seealsoref{大夫}{dai4fu5}
  \seealsoref{大王}{dai4wang5}
  \end{Phonetics}
\end{Entry}

\begin{Entry}{大人}{3,2}{⼤、⼈}
  \begin{Phonetics}{大人}{da4 ren2}[][HSK 2]
    \definition[个,位]{s.}{senhor; ilustre; sua excelência; antigo título honorífico para funcionários públicos | adulto; crescido; maduro;}
  \end{Phonetics}
\end{Entry}

\begin{Entry}{大力}{3,2}{⼤、⼒}
  \begin{Phonetics}{大力}{da4 li4}[][HSK 6]
    \definition{adv.}{energicamente; vigorosamente; indica uso de grande força}
    \definition{s.}{grande força, poder}
  \end{Phonetics}
\end{Entry}

\begin{Entry}{大于}{3,3}{⼤、⼆}
  \begin{Phonetics}{大于}{da4 yu2}[][HSK 5]
    \definition{v.}{ser maior, mais numeroso, mais importante, etc. do que}
  \end{Phonetics}
\end{Entry}

\begin{Entry}{大口}{3,3}{⼤、⼝}
  \begin{Phonetics}{大口}{da4kou3}
    \definition{s.}{grande bocado (de comida, bebida, fumo, etc.)}
  \end{Phonetics}
\end{Entry}

\begin{Entry}{大大}{3,3}{⼤、⼤}
  \begin{Phonetics}{大大}{da4 da4}[][HSK 2]
    \definition{adv.}{grandemente; enormemente; enfatizar grande quantidade ou grau profundo}
  \end{Phonetics}
\end{Entry}

\begin{Entry}{大大咧咧}{3,3,9,9}{⼤、⼤、⼝、⼝}
  \begin{Phonetics}{大大咧咧}{da4da5lie1lie1}[][HSK 7-9]
    \definition{expr.}{despreocupado; descuidado; casual}
  \end{Phonetics}
\end{Entry}

\begin{Entry}{大小}{3,3}{⼤、⼩}
  \begin{Phonetics}{大小}{da4 xiao3}[][HSK 2]
    \definition{adv.}{no mínimo; grande ou pequeno (geralmente pequeno), significa que ainda pode ser considerado}
    \definition[家]{s.}{tamanho; o grau de tamanho | ordem de senioridade; hierarquia | adultos e crianças | grande ou pequeno}
  \end{Phonetics}
\end{Entry}

\begin{Entry}{大门}{3,3}{⼤、⾨}
  \begin{Phonetics}{大门}{da4 men2}[][HSK 2]
    \definition{s.}{portão; entrada; portão grande, referindo-se especificamente ao portão principal de um edifício (como uma casa, pátio ou parque) que dá para a rua (em contraste com o segundo portão e as portas das várias divisões)}
  \end{Phonetics}
\end{Entry}

\begin{Entry}{大马}{3,3}{⼤、⾺}
  \begin{Phonetics}{大马}{da4ma3}
    \definition*{s.}{Malásia}
  \end{Phonetics}
\end{Entry}

\begin{Entry}{大公无私}{3,4,4,7}{⼤、⼋、⽆、⽲}
  \begin{Phonetics}{大公无私}{da4gong1-wu2si1}[][HSK 7-9]
    \definition{expr.}{altruísta; generoso; desinteressado | perfeitamente imparcial | nenhuma consideração pessoal (egoísta); não pensar em si mesmo; justo e altruísta; grande imparcialidade exclui consideração de si mesmo; desinteresse e altruísmo}
  \end{Phonetics}
\end{Entry}

\begin{Entry}{大厅}{3,4}{⼤、⼚}
  \begin{Phonetics}{大厅}{da4 ting1}[][HSK 5]
    \definition{s.}{\emph{hall}; saguão, uma sala grande para reuniões ou atividades em um edifício de grande porte}
  \end{Phonetics}
\end{Entry}

\begin{Entry}{大夫}{3,4}{⼤、⼤}
  \begin{Phonetics}{大夫}{da4fu1}
    \definition[个,位,名]{s.}{oficial sênior (na China Imperial)}
  \end{Phonetics}
  \begin{Phonetics}{大夫}{dai4fu5}[][HSK 3]
    \definition[个,位,名]{s.}{médico, doutor}
  \end{Phonetics}
\end{Entry}

\begin{Entry}{大巴}{3,4}{⼤、⼰}
  \begin{Phonetics}{大巴}{da4 ba1}[][HSK 4]
    \definition{s.}{ônibus}
  \end{Phonetics}
\end{Entry}

\begin{Entry}{大方}{3,4}{⼤、⽅}
  \begin{Phonetics}{大方}{da4fang1}
    \definition{s.}{generosidades; liberalidades | estudioso; pessoas com conhecimento especializado | um tipo de chá verde, produzido principalmente no Condado de Shexian, Província de Anhui, Condado de Chun'an, Província de Zhejiang, etc.}
  \end{Phonetics}
  \begin{Phonetics}{大方}{da4fang5}[][HSK 4]
    \definition{adj.}{generoso | não afetado; natural e equilibrado |  de bom gosto}
  \end{Phonetics}
\end{Entry}

\begin{Entry}{大气}{3,4}{⼤、⽓}
  \begin{Phonetics}{大气}{da4qi4}[][HSK 7-9]
    \definition{adj.}{generoso; magnânimo; refere-se a um porte extraordinário, não convencional}
    \definition{s.}{ar; atmosfera | respiração pesada}
  \end{Phonetics}
\end{Entry}

\begin{Entry}{大片}{3,4}{⼤、⽚}
  \begin{Phonetics}{大片}{da4pian4}[][HSK 7-9]
    \definition{adj.}{grande; enorme; imenso}
    \definition{s.}{um filme de grande orçamento; um sucesso de bilheteria; \emph{block-buster}; refere-se a um filme caro e bem feito | grande área; vasta extensão; ampla extensão}
  \end{Phonetics}
\end{Entry}

\begin{Entry}{大王}{3,4}{⼤、⽟}
  \begin{Phonetics}{大王}{da4wang2}
    \definition{s.}{rei; magnata | pessoa da mais alta classe ou habilidade em algo; ás | barões | pessoa com habilidade especializada em algo}
  \end{Phonetics}
  \begin{Phonetics}{大王}{dai4wang5}
    \definition{s.}{magnata; barões | barão ladrão (em ópera, histórias antigas)}
  \end{Phonetics}
\end{Entry}

\begin{Entry}{大队}{3,4}{⼤、⾩}
  \begin{Phonetics}{大队}{da4dui4}[][HSK 7-9]
    \definition{adj.}{grande (contingente de tropas, corpo de manifestantes, número de pessoas, etc.)}
    \definition{s.}{unidade militar correspondente ao batalhão ou regimento; grupo | História: brigada de produção (de uma comuna popular rural); brigada}
  \end{Phonetics}
\end{Entry}

\begin{Entry}{大包大揽}{3,5,3,12}{⼤、⼓、⼤、⼿}
  \begin{Phonetics}{大包大揽}{da4bao1-da4lan3}[][HSK 7-9]
    \definition{v.}{assumir o controle total}
  \end{Phonetics}
\end{Entry}

\begin{Entry}{大众}{3,6}{⼤、⼈}
  \begin{Phonetics}{大众}{da4 zhong4}[][HSK 4]
    \definition{s.}{massas; população; pessoas comuns; público em geral}
  \end{Phonetics}
\end{Entry}

\begin{Entry}{大伙儿}{3,6,2}{⼤、⼈、⼉}
  \begin{Phonetics}{大伙儿}{da4huo3r5}[][HSK 5]
    \definition{pron.}{todos nós; todos vocês; todo mundo; todos; equivalente a 大家}
  \seealsoref{大家}{da4jia1}
  \end{Phonetics}
\end{Entry}

\begin{Entry}{大会}{3,6}{⼤、⼈}
  \begin{Phonetics}{大会}{da4 hui4}[][HSK 4]
    \definition[场,次,个,届]{s.}{sessão plenária; reunião geral de membros; reuniões convocadas por partidos políticos socialistas | reunião de massa; comício de massa}
  \end{Phonetics}
\end{Entry}

\begin{Entry}{大全}{3,6}{⼤、⼊}
  \begin{Phonetics}{大全}{da4quan2}
    \definition{s.}{coleção abrangente}
  \end{Phonetics}
\end{Entry}

\begin{Entry}{大吃一惊}{3,6,1,11}{⼤、⼝、⼀、⼼}
  \begin{Phonetics}{大吃一惊}{da4chi1-yi1jing1}[][HSK 7-9]
    \definition{expr.}{ficar assustado com; ficar espantado com; ficar completamente surpreso; ficar muito surpreso; ficar completamente chocado; ser pego de surpresa; levar um choque; ter (levar) um susto; levar um susto com; ficar pasmo}
  \end{Phonetics}
\end{Entry}

\begin{Entry}{大同小异}{3,6,3,6}{⼤、⼝、⼩、⼶}
  \begin{Phonetics}{大同小异}{da4tong2-xiao3yi4}[][HSK 7-9]
    \definition{expr.}{muito parecidos, mas com pequenas diferenças; semelhante, exceto por pequenas diferenças; ser o mesmo em aspectos essenciais, embora diferindo em pontos menores; uma semelhança geral com pequenas (menores) diferenças; ser semelhante em todos os aspectos essenciais importantes, sendo as diferenças de natureza menor; ser em grande parte idêntico com apenas pequenas diferenças; diferir apenas em (em) pequenos pontos; essencialmente o mesmo, embora diferindo em pontos menores; na maior parte, eles são os mesmos; idênticos em questões importantes, embora com pequenas diferenças; principalmente semelhantes, exceto (por) pequenas diferenças; virtualmente os mesmos}
  \end{Phonetics}
\end{Entry}

\begin{Entry}{大名鼎鼎}{3,6,12,12}{⼤、⼝、⿍、⿍}
  \begin{Phonetics}{大名鼎鼎}{da4ming2-ding3ding3}[][HSK 7-9]
    \definition{expr.}{``O nome de alguém é conhecido em toda parte.''; ser muito famoso; ser amplamente conhecido; celebrado; desfrutar de um grande nome;; bem conhecido}
  \end{Phonetics}
\end{Entry}

\begin{Entry}{大后天}{3,6,4}{⼤、⼝、⼤}
  \begin{Phonetics}{大后天}{da4 hou4 tian1}
    \definition{s.}{daqui a três dias}
  \end{Phonetics}
\end{Entry}

\begin{Entry}{大地}{3,6}{⼤、⼟}
  \begin{Phonetics}{大地}{da4di4}[][HSK 7-9]
    \definition*{s.}{A Terra}[大地正在遭受污染。===A Terra está sendo poluída.]
    \definition[片,块,方]{s.}{terreno espaçoso; terra}
  \end{Phonetics}
\end{Entry}

\begin{Entry}{大多}{3,6}{⼤、⼣}
  \begin{Phonetics}{大多}{da4 duo1}[][HSK 4]
    \definition{adv.}{majoritariamente; em sua maior parte; em sua maioria; em grande parte}
  \end{Phonetics}
\end{Entry}

\begin{Entry}{大多数}{3,6,13}{⼤、⼣、⽁}
  \begin{Phonetics}{大多数}{da4 duo1 shu4}[][HSK 2]
    \definition{s.}{grande maioria; vasta maioria; a maior parte; mais da metade, um número significativo}
  \end{Phonetics}
\end{Entry}

\begin{Entry}{大妈}{3,6}{⼤、⼥}
  \begin{Phonetics}{大妈}{da4 ma1}[][HSK 4]
    \definition[个,位]{s.}{tia; esposa do irmão mais velho do pai | tratamento respeitoso às mulheres idosas}
  \end{Phonetics}
\end{Entry}

\begin{Entry}{大师}{3,6}{⼤、⼱}
  \begin{Phonetics}{大师}{da4 shi1}[][HSK 6]
    \definition*{s.}{Grande Mestre, título de cortesia usado para se dirigir a um monge budista}
    \definition{s.}{grande mestre; mestre; maestro; uma pessoa com realizações profundas}
  \end{Phonetics}
\end{Entry}

\begin{Entry}{大戏}{3,6}{⼤、⼽}
  \begin{Phonetics}{大戏}{da4xi4}
    \definition*{s.}{Drama, Ópera Chinesa}
  \end{Phonetics}
\end{Entry}

\begin{Entry}{大有可为}{3,6,5,4}{⼤、⽉、⼝、⼂}
  \begin{Phonetics}{大有可为}{da4you3-ke3wei2}[][HSK 7-9]
    \definition{expr.}{ter um futuro brilhante; valer a pena; poder realizar grandes coisas; ter perspectivas brilhantes; pode ser bem administrado; há amplo escopo para nossas habilidades em\dots; valer a pena fazer; ter perspectivas brilhantes}
  \end{Phonetics}
\end{Entry}

\begin{Entry}{大爷}{3,6}{⼤、⽗}
  \begin{Phonetics}{大爷}{da4 ye2}
    \definition[个,位]{s.}{Coloquial: irmão mais velho do pai; tio | tratamento respeitoso para um homem mais velho}
  \end{Phonetics}
  \begin{Phonetics}{大爷}{da4 ye5}[][HSK 4]
    \definition[个,位]{s.}{irmão mais velho do pai; tio | tio (homenagem aos homens mais velhos)}
  \end{Phonetics}
\end{Entry}

\begin{Entry}{大米}{3,6}{⼤、⽶}
  \begin{Phonetics}{大米}{da4 mi3}[][HSK 6]
    \definition[颗,粒,斤,包,袋]{s.}{arroz; arroz descascado; arroz bom}
  \end{Phonetics}
\end{Entry}

\begin{Entry}{大约}{3,6}{⼤、⽷}
  \begin{Phonetics}{大约}{da4yue1}[][HSK 3]
    \definition{adv.}{aproximadamente; sobre; estimativa não muito precisa| provavelmente; expressar suposições sobre a situação}
  \end{Phonetics}
\end{Entry}

\begin{Entry}{大臣}{3,6}{⼤、⾂}
  \begin{Phonetics}{大臣}{da4chen2}[][HSK 7-9]
    \definition[位,个]{s.}{secretário; ministro (de uma monarquia); altos funcionários da monarquia}
  \end{Phonetics}
\end{Entry}

\begin{Entry}{大自然}{3,6,12}{⼤、⾃、⽕}
  \begin{Phonetics}{大自然}{da4 zi4 ran2}[][HSK 2]
    \definition{s.}{natureza}
  \end{Phonetics}
\end{Entry}

\begin{Entry}{大衣}{3,6}{⼤、⾐}
  \begin{Phonetics}{大衣}{da4 yi1}[][HSK 2]
    \definition[件,个]{s.}{sobretudo; casaco; casaco ocidental mais comprido}
  \end{Phonetics}
\end{Entry}

\begin{Entry}{大伯子}{3,7,3}{⼤、⼈、⼦}
  \begin{Phonetics}{大伯子}{da4 bai3zi5}
    \definition{s.}{Coloquial: irmão mais velho do marido; cunhado}
  \end{Phonetics}
\end{Entry}

\begin{Entry}{大体}{3,7}{⼤、⼈}
  \begin{Phonetics}{大体}{da4ti3}[][HSK 7-9]
    \definition{adv.}{aproximadamente; em geral; mais ou menos}
    \definition{s.}{princípio primordial; interesse geral; um princípio basilar baseado na situação geral; um princípio relativo à situação geral}
  \end{Phonetics}
\end{Entry}

\begin{Entry}{大体上}{3,7,3}{⼤、⼈、⼀}
  \begin{Phonetics}{大体上}{da4ti3 shang4}[][HSK 7-9]
    \definition{adv.}{em geral; grosso modo; de um modo geral}
  \end{Phonetics}
\end{Entry}

\begin{Entry}{大声}{3,7}{⼤、⼠}
  \begin{Phonetics}{大声}{da4 sheng1}[][HSK 2]
    \definition{adj.}{alto; volume alto; em voz alta}
  \end{Phonetics}
\end{Entry}

\begin{Entry}{大局}{3,7}{⼤、⼫}
  \begin{Phonetics}{大局}{da4ju2}[][HSK 7-9]
    \definition{s.}{situação geral (ou inteira); toda a situação}
  \end{Phonetics}
\end{Entry}

\begin{Entry}{大批}{3,7}{⼤、⼿}
  \begin{Phonetics}{大批}{da4 pi1}[][HSK 6]
    \definition{num.}{grandes quantidades de; exércitos; inundações}[大批书籍被印刷出来。===Grandes quantidades de livros foram impressas.]
  \end{Phonetics}
\end{Entry}

\begin{Entry}{大纲}{3,7}{⼤、⽷}
  \begin{Phonetics}{大纲}{da4 gang1}[][HSK 5]
    \definition{s.}{esboço; compêndio; programa de estudos; resumo; fundamentos da organização sistemática de conteúdos (livros, discursos, programas, etc.)}
  \end{Phonetics}
\end{Entry}

\begin{Entry}{大豆}{3,7}{⼤、⾖}
  \begin{Phonetics}{大豆}{da4dou4}
    \definition{s.}{soja}
  \end{Phonetics}
\end{Entry}

\begin{Entry}{大陆}{3,7}{⼤、⾩}
  \begin{Phonetics}{大陆}{da4 lu4}[][HSK 4]
    \definition*{s.}{China continental; refere-se especificamente à vasta porção terrestre do território da China}
    \definition[个,块]{s.}{terra firme; continente; vasta extensão de terra}
  \end{Phonetics}
\end{Entry}

\begin{Entry}{大事}{3,8}{⼤、⼅}
  \begin{Phonetics}{大事}{da4 shi4}[][HSK 5]
    \definition{adv.}{em grande escala; em grande estilo; em grande parte}
    \definition[件,桩]{s.}{grande evento; grande acontecimento; assunto importante; grande questão; algo importante | situação geral}
  \end{Phonetics}
\end{Entry}

\begin{Entry}{大使}{3,8}{⼤、⼈}
  \begin{Phonetics}{大使}{da4 shi3}[][HSK 6]
    \definition[位,任]{s.}{embaixador; o representante diplomático de mais alto nível enviado por um país a outro país}
  \seealsoref{全称特命全权大使}{quan2cheng1 te4ming4 quan2quan2 da4shi3}
  \end{Phonetics}
\end{Entry}

\begin{Entry}{大使馆}{3,8,11}{⼤、⼈、⾷}
  \begin{Phonetics}{大使馆}{da4shi3guan3}[][HSK 3]
    \definition[座,个]{s.}{embaixada; uma representação diplomática de um país em outro país, chefiada por um embaixador}
  \end{Phonetics}
\end{Entry}

\begin{Entry}{大姐}{3,8}{⼤、⼥}
  \begin{Phonetics}{大姐}{da4 jie3}[][HSK 4]
    \definition[个,位]{s.}{irmã mais velha (também um termo educado para se dirigir a uma garota ou mulher um pouco mais velha do que a pessoa que fala)}
  \end{Phonetics}
\end{Entry}

\begin{Entry}{大学}{3,8}{⼤、⼦}
  \begin{Phonetics}{大学}{da4 xue2}[][HSK 1]
    \definition[所,座]{s.}{universidade; faculdade; tipo de instituição de ensino superior que, na China, geralmente se refere a uma universidade abrangente}
  \end{Phonetics}
\end{Entry}

\begin{Entry}{大学生}{3,8,5}{⼤、⼦、⽣}
  \begin{Phonetics}{大学生}{da4 xue2 sheng1}[][HSK 1]
    \definition[名,个]{s.}{estudante universitário; estudante de faculdade; estudantes de graduação ou cursos técnicos em instituições de ensino superior}
  \end{Phonetics}
\end{Entry}

\begin{Entry}{大宗}{3,8}{⼤、⼧}
  \begin{Phonetics}{大宗}{da4zong1}[][HSK 7-9]
    \definition{adj.}{grande; volumoso}
    \definition{s.}{itens principais; produtos ou bens básicos}
  \end{Phonetics}
\end{Entry}

\begin{Entry}{大抵}{3,8}{⼤、⼿}
  \begin{Phonetics}{大抵}{da4di3}
    \definition{adv.}{no geral; de um modo geral; provavelmente; principalmente}
  \end{Phonetics}
\end{Entry}

\begin{Entry}{大规模}{3,8,14}{⼤、⾒、⽊}
  \begin{Phonetics}{大规模}{da4 gui1 mo2}[][HSK 4]
    \definition{adj.}{em larga escala; extensivo; maciço; massivo}
    \definition{adv.}{em larga escala; extensivo; maciço; massa}
  \end{Phonetics}
\end{Entry}

\begin{Entry}{大雨}{3,8}{⼤、⾬}
  \begin{Phonetics}{大雨}{da4yu3}
    \definition[场]{s.}{chuva pesada, forte}
  \end{Phonetics}
\end{Entry}

\begin{Entry}{大前天}{3,9,4}{⼤、⼑、⼤}
  \begin{Phonetics}{大前天}{da4qian2tian1}
    \definition{adv.}{três dias atrás}
  \end{Phonetics}
\end{Entry}

\begin{Entry}{大型}{3,9}{⼤、⼟}
  \begin{Phonetics}{大型}{da4xing2}[][HSK 4]
    \definition{adj.}{grande; em larga escala; tamanho e volume grandes | larga escala (importante e influente)}
  \end{Phonetics}
\end{Entry}

\begin{Entry}{大城}{3,9}{⼤、⼟}
  \begin{Phonetics}{大城}{da4cheng2}
    \definition*[个,座]{s.}{Condado de Dacheng em Langfang 廊坊, Hebei | Município de Tacheng no condado de Changhua | Condado de Changhua, Taiwan}
  \seealsoref{廊坊}{lang2fang2}
  \end{Phonetics}
\end{Entry}

\begin{Entry}{大奖赛}{3,9,14}{⼤、⼤、⾙}
  \begin{Phonetics}{大奖赛}{da4 jiang3 sai4}[][HSK 5]
    \definition{s.}{grande competição; grande prêmio; \emph{grand prix}}
  \end{Phonetics}
\end{Entry}

\begin{Entry}{大战}{3,9}{⼤、⼽}
  \begin{Phonetics}{大战}{da4zhan4}
    \definition{s.}{guerra}
    \definition{v.}{guerrear | lutar em uma guerra}
  \end{Phonetics}
\end{Entry}

\begin{Entry}{大洋洲}{3,9,9}{⼤、⽔、⽔}
  \begin{Phonetics}{大洋洲}{da4yang2zhou1}
    \definition*{s.}{Oceania}
  \end{Phonetics}
\end{Entry}

\begin{Entry}{大神}{3,9}{⼤、⽰}
  \begin{Phonetics}{大神}{da4shen2}
    \definition{s.}{deidade | (gíria da Internet) guru | \emph{expert} | gênio}
  \end{Phonetics}
\end{Entry}

\begin{Entry}{大胆}{3,9}{⼤、⾁}
  \begin{Phonetics}{大胆}{da4 dan3}[][HSK 5]
    \definition{adj.}{ousado; atrevido; audacioso; corajoso; destemido}
  \end{Phonetics}
\end{Entry}

\begin{Entry}{大选}{3,9}{⼤、⾡}
  \begin{Phonetics}{大选}{da4xuan3}[][HSK 7-9]
    \definition[次]{s.}{eleição geral; refere-se à eleição de membros do parlamento ou presidente em alguns países}
  \end{Phonetics}
\end{Entry}

\begin{Entry}{大面积}{3,9,10}{⼤、⾯、⽲}
  \begin{Phonetics}{大面积}{da4 mian4ji1}[][HSK 7-9]
    \definition{s.}{grande área}
  \end{Phonetics}
\end{Entry}

\begin{Entry}{大哥}{3,10}{⼤、⼝}
  \begin{Phonetics}{大哥}{da4 ge1}[][HSK 4]
    \definition{s.}{irmão mais velho | \emph{big brother}; tratamento educado para um homem da mesma idade que você | líder de gangue; pessoa mais poderosa em uma organização que realiza atividades ilegais na sociedade}
  \end{Phonetics}
\end{Entry}

\begin{Entry}{大家}{3,10}{⼤、⼧}
  \begin{Phonetics}{大家}{da4jia1}[][HSK 2]
    \definition{pron.}{todos; toda a gente; refere-se a todas as pessoas dentro de um determinado âmbito}
    \definition{s.}{grande mestre; autoridade; especialista renomado | família nobre; família rica e influente; família tradicional}
  \end{Phonetics}
\end{Entry}

\begin{Entry}{大家庭}{3,10,9}{⼤、⼧、⼴}
  \begin{Phonetics}{大家庭}{da4jia1ting2}[][HSK 7-9]
    \definition[个,些]{s.}{grande família; comunidade | uma grande família é frequentemente usada para descrever um grupo com muitos membros e harmonia interna}
  \end{Phonetics}
\end{Entry}

\begin{Entry}{大海}{3,10}{⼤、⽔}
  \begin{Phonetics}{大海}{da4 hai3}[][HSK 2]
    \definition{s.}{o mar; o oceano; o mar aberto, ou seja, a parte do oceano que não está fechada entre cabos nem incluída em estreitos}
  \end{Phonetics}
\end{Entry}

\begin{Entry}{大笔}{3,10}{⼤、⽵}
  \begin{Phonetics}{大笔}{da4bi3}[][HSK 7-9]
    \definition{adj.}{grande quantidade (de dinheiro) | grande soma (de capital, dinheiro, etc.)}
    \definition{s.}{caneta | Cortês: sua escrita; sua caligrafia}
  \end{Phonetics}
\end{Entry}

\begin{Entry}{大脑}{3,10}{⼤、⾁}
  \begin{Phonetics}{大脑}{da4 nao3}[][HSK 5]
    \definition{s.}{cérebro; encéfalo}
  \end{Phonetics}
\end{Entry}

\begin{Entry}{大致}{3,10}{⼤、⾄}
  \begin{Phonetics}{大致}{da4zhi4}[][HSK 5]
    \definition{adj.}{geral; no todo}
    \definition{adv.}{grosso modo; aproximadamente; mais ou menos; indica uma estimativa aproximada da situação}
  \end{Phonetics}
\end{Entry}

\begin{Entry}{大部分}{3,10,4}{⼤、⾢、⼑}
  \begin{Phonetics}{大部分}{da4 bu4 fen4}[][HSK 2]
    \definition[把]{s.}{a maioria; a maior parte; em grande parte; refere-se a uma quantidade superior a metade do total}
  \end{Phonetics}
\end{Entry}

\begin{Entry}{大都}{3,10}{⼤、⾢}
  \begin{Phonetics}{大都}{da4 dou1}[][HSK 5]
  \end{Phonetics}
  \begin{Phonetics}{大都}{da4 du1}
    \definition*{s.}{Dadu, capital da China durante a Dinastia Yuan (1280-1368), atual Pequim}
    \definition{adv.}{em sua maior parte; na maior parte; indica que a maioria das pessoas ou coisas em um determinado intervalo tem a mesma natureza e características; também pronunciado como \dpy{da4dou1} na língua falada}
  \end{Phonetics}
\end{Entry}

\begin{Entry}{大惊小怪}{3,11,3,8}{⼤、⼼、⼩、⼼}
  \begin{Phonetics}{大惊小怪}{da4jing1-xiao3guai4}[][HSK 7-9]
    \definition{expr.}{ficar animado com uma coisa pequena; uma tempestade em um copo d'água; ficar surpreso com algo perfeitamente normal; ficar alarmado com algo perfeitamente normal; sentir-se surpreso; ficar alarmado com; ficar todo agitado por nada; grande alarme com um pequeno fantasma; fazer um alarido; fazer um alarido por nada; fazer um grande alarido sobre (algo); fazer um raro alarido sobre (algo); fazer muito alarido; fazer barulho por nada}
  \end{Phonetics}
\end{Entry}

\begin{Entry}{大象}{3,11}{⼤、⾗}
  \begin{Phonetics}{大象}{da4xiang4}[][HSK 5]
    \definition[只,头,群,个]{s.}{elefante}
  \end{Phonetics}
\end{Entry}

\begin{Entry}{大黄}{3,11}{⼤、⿈}
  \begin{Phonetics}{大黄}{da4huang2}
    \definition{s.}{ruibarbo chinês}
  \end{Phonetics}
\end{Entry}

\begin{Entry}{大厦}{3,12}{⼤、⼚}
  \begin{Phonetics}{大厦}{da4sha4}[][HSK 7-9]
    \definition[栋,座,撞]{s.}{prédio; edifício grande (frequentemente usado em nomes de grandes edifícios)}
  \end{Phonetics}
\end{Entry}

\begin{Entry}{大幅度}{3,12,9}{⼤、⼱、⼴}
  \begin{Phonetics}{大幅度}{da4 fu2du4}[][HSK 7-9]
    \definition{adv.}{drasticamente; substancial; por uma ampla margem; a extensão da mudança ou alteração é significativa}
  \end{Phonetics}
\end{Entry}

\begin{Entry}{大棚}{3,12}{⼤、⽊}
  \begin{Phonetics}{大棚}{da4peng2}[][HSK 7-9]
    \definition{s.}{grande abrigo em arco coberto com lona plástica usada na agricultura; grande estufa | estufa}
  \end{Phonetics}
\end{Entry}

\begin{Entry}{大款}{3,12}{⼤、⽋}
  \begin{Phonetics}{大款}{da4kuan3}[][HSK 7-9]
    \definition[个]{s.}{o rico; pessoa muito rica}
  \end{Phonetics}
\end{Entry}

\begin{Entry}{大猩猩}{3,12,12}{⼤、⽝、⽝}
  \begin{Phonetics}{大猩猩}{da4xing1xing5}
    \definition{s.}{gorila}
  \end{Phonetics}
\end{Entry}

\begin{Entry}{大腕儿}{3,12,2}{⼤、⾁、⼉}
  \begin{Phonetics}{大腕儿}{da4wan4r5}[][HSK 7-9]
    \definition[位]{s.}{grande nome, figurão; uma pessoa que tem grande habilidade, fama e influência em um determinado setor ou aspecto}
  \end{Phonetics}
\end{Entry}

\begin{Entry}{大街}{3,12}{⼤、⾏}
  \begin{Phonetics}{大街}{da4 jie1}[][HSK 6]
    \definition[条,个]{s.}{avenida; rua; rua principal}
  \end{Phonetics}
\end{Entry}

\begin{Entry}{大街小巷}{3,12,3,9}{⼤、⾏、⼩、⼰}
  \begin{Phonetics}{大街小巷}{da4jie1-xiao3xiang4}[][HSK 7-9]
    \definition{expr.}{grandes ruas e pequenos becos; em todos os lugares da cidade}
  \end{Phonetics}
\end{Entry}

\begin{Entry}{大道}{3,12}{⼤、⾡}
  \begin{Phonetics}{大道}{da4 dao4}[][HSK 6]
    \definition*{s.}{O Grande Tao; O Grande Caminho}
    \definition[条]{s.}{estrada principal | o caminho da justiça | avenida | rua principal}
  \end{Phonetics}
\end{Entry}

\begin{Entry}{大量}{3,12}{⼤、⾥}
  \begin{Phonetics}{大量}{da4 liang4}[][HSK 2]
    \definition{adj.}{numeroso; em grande quantidade; grande em número ou quantidade | generoso; magnânimo; descreve uma pessoa que não fica zangada quando os outros cometem erros e que costuma perdoar os outros}
  \end{Phonetics}
\end{Entry}

\begin{Entry}{大雁}{3,12}{⼤、⾫}
  \begin{Phonetics}{大雁}{da4yan4}[][HSK 7-9]
    \definition[只,群,行]{s.}{ganso selvagem}
  \end{Phonetics}
\end{Entry}

\begin{Entry}{大意}{3,13}{⼤、⼼}
  \begin{Phonetics}{大意}{da4yi4}[][HSK 7-9]
    \definition{s.}{essência; teor; ideia geral; pontos principais; o significado principal}
  \end{Phonetics}
  \begin{Phonetics}{大意}{da4yi5}[][HSK 7-9]
    \definition{adj.}{descuidado; negligente; desatento}
  \end{Phonetics}
\end{Entry}

\begin{Entry}{大数据}{3,13,11}{⼤、⽁、⼿}
  \begin{Phonetics}{大数据}{da4shu4ju4}[][HSK 7-9]
    \definition*{s.}{\emph{Big Data}}
  \end{Phonetics}
\end{Entry}

\begin{Entry}{大楼}{3,13}{⼤、⽊}
  \begin{Phonetics}{大楼}{da4 lou2}[][HSK 4]
    \definition[座,幢]{s.}{edifício; mansão; edifício de vários andares disponível para uso residencial e comercial}
  \end{Phonetics}
\end{Entry}

\begin{Entry}{大概}{3,13}{⼤、⽊}
  \begin{Phonetics}{大概}{da4gai4}[][HSK 3]
    \definition{adj.}{geral; grosseiro; aproximado; não é muito preciso ou muito detalhado}
    \definition{adv.}{sobre; provavelmente; estimativas ou suposições imprecisas sobre eventos, quantidades, tempo, localização, etc.| geralmente; brevemente; não muito seriamente, casualmente; não muito cuidadosamente}
    \definition{s.}{ideia geral; esboço geral; conteúdo geral ou situação}
  \end{Phonetics}
\end{Entry}

\begin{Entry}{大肆}{3,13}{⼤、⾀}
  \begin{Phonetics}{大肆}{da4si4}[][HSK 7-9]
    \definition{adv.}{desenfreadamente; violentamente; imprudentemente; sem restrições; sem escrúpulos (geralmente referindo-se a fazer coisas ruins)}
  \end{Phonetics}
\end{Entry}

\begin{Entry}{大腿}{3,13}{⼤、⾁}
  \begin{Phonetics}{大腿}{da4tui3}
    \definition{s.}{coxa}
  \end{Phonetics}
\end{Entry}

\begin{Entry}{大蒜}{3,13}{⼤、⾋}
  \begin{Phonetics}{大蒜}{da4suan4}
    \definition[瓣,头]{s.}{alho}
  \end{Phonetics}
\end{Entry}

\begin{Entry}{大模大样}{3,14,3,10}{⼤、⽊、⼤、⽊}
  \begin{Phonetics}{大模大样}{da4mu2-da4yang4}[][HSK 7-9]
    \definition{expr.}{pomposamente; corajosamente; ostensivamente; equilibrado; autoconfiante}
  \end{Phonetics}
\end{Entry}

\begin{Entry}{大熊猫}{3,14,11}{⼤、⽕、⽝}
  \begin{Phonetics}{大熊猫}{da4 xiong2 mao1}[][HSK 5]
    \definition{s.}{panda gigante}
  \end{Phonetics}
\end{Entry}

\begin{Entry}{大赛}{3,14}{⼤、⾙}
  \begin{Phonetics}{大赛}{da4 sai4}[][HSK 6]
    \definition{s.}{grande torneio; competição importante; um evento de grande porte e alto nível; um grande evento}
  \end{Phonetics}
\end{Entry}

%%%%%%%%%% 天 %%%%%%%%%%
\subsection*{天}\addcontentsline{loh}{figure}{天}

\begin{Entry}{天}{4}{⼤}
  \begin{Phonetics}{天}{tian1}[][HSK 1]
    \definition*{s.}{Sobrenome: Tian}
    \definition{adj.}{localizado no topo; suspenso no ar | inato; natural}
    \definition{clas.}{usado para contar dias}
    \definition{s.}{céu; paraíso; espaço onde se encontram o sol, a lua e as estrelas | dia; as 24 horas do dia, às vezes referindo-se especificamente ao período diurno | um período de tempo em um dia; em algum momento do dia | temporada; estação do ano | clima | natureza | Deus; céu; o criador | paraíso; refere-se ao local onde residem os deuses, budas e imortais}
  \end{Phonetics}
\end{Entry}

\begin{Entry}{天上}{4,3}{⼤、⼀}
  \begin{Phonetics}{天上}{tian1 shang4}[][HSK 2]
    \definition[期]{s.}{o céu; o paraíso}
  \end{Phonetics}
\end{Entry}

\begin{Entry}{天下}{4,3}{⼤、⼀}
  \begin{Phonetics}{天下}{tian1 xia4}[][HSK 6]
    \definition[期]{s.}{China ou o mundo; refere-se à China ou ao mundo | dominação; o poder dominante de um país | situação; um determinado campo; uma metáfora para uma determinada área ou situação}
  \end{Phonetics}
\end{Entry}

\begin{Entry}{天才}{4,3}{⼤、⼿}
  \begin{Phonetics}{天才}{tian1cai2}[][HSK 5]
    \definition{adj.}{talentoso | superdotado | genial}
    \definition[个,位,名]{s.}{dom; genialidade; talento natural; inteligência e sabedoria acima da média}
  \end{Phonetics}
\end{Entry}

\begin{Entry}{天公}{4,4}{⼤、⼋}
  \begin{Phonetics}{天公}{tian1gong1}
    \definition{s.}{céu, paraíso | senhor do céu}
  \end{Phonetics}
\end{Entry}

\begin{Entry}{天天}{4,4}{⼤、⼤}
  \begin{Phonetics}{天天}{tian1tian1}
    \definition{adv.}{todo dia}
  \end{Phonetics}
\end{Entry}

\begin{Entry}{天文}{4,4}{⼤、⽂}
  \begin{Phonetics}{天文}{tian1wen2}[][HSK 5]
    \definition[对]{s.}{astronomia; a distribuição e o movimento dos corpos celestes, como o sol, a lua e as estrelas, no universo}
  \end{Phonetics}
\end{Entry}

\begin{Entry}{天气}{4,4}{⼤、⽓}
  \begin{Phonetics}{天气}{tian1qi4}[][HSK 1]
    \definition{s.}{clima, tempo; mudanças meteorológicas que ocorrem na atmosfera em uma determinada área e durante um determinado período de tempo, tais como temperatura, umidade, pressão atmosférica, precipitação, vento, nuvens, etc.}
  \end{Phonetics}
\end{Entry}

\begin{Entry}{天花板}{4,7,8}{⼤、⾋、⽊}
  \begin{Phonetics}{天花板}{tian1hua1ban3}
    \definition{s.}{teto}
  \end{Phonetics}
\end{Entry}

\begin{Entry}{天使}{4,8}{⼤、⼈}
  \begin{Phonetics}{天使}{tian1shi3}
    \definition{s.}{anjo}
  \end{Phonetics}
\end{Entry}

\begin{Entry}{天择}{4,8}{⼤、⼿}
  \begin{Phonetics}{天择}{tian1ze2}
    \definition{s.}{seleção natural}
  \end{Phonetics}
\end{Entry}

\begin{Entry}{天空}{4,8}{⼤、⽳}
  \begin{Phonetics}{天空}{tian1kong1}[][HSK 3]
    \definition{s.}{o céu; o firmamento}
  \end{Phonetics}
\end{Entry}

\begin{Entry}{天柱}{4,9}{⼤、⽊}
  \begin{Phonetics}{天柱}{tian1zhu4}
    \definition{s.}{pilar celestial, que sustenta o céu}
  \end{Phonetics}
\end{Entry}

\begin{Entry}{天津}{4,9}{⼤、⽔}
  \begin{Phonetics}{天津}{tian1jin1}
    \definition*{s.}{Tianjin, um município no nordeste da China}
  \end{Phonetics}
\end{Entry}

\begin{Entry}{天真}{4,10}{⼤、⼗}
  \begin{Phonetics}{天真}{tian1zhen1}[][HSK 4]
    \definition{adj.}{ingênuo; inocente; ignorante; simples de coração, direto por natureza, livre de fingimento e hipocrisia}
  \end{Phonetics}
\end{Entry}

\begin{Entry}{天堂}{4,11}{⼤、⼟}
  \begin{Phonetics}{天堂}{tian1tang2}[][HSK 6]
    \definition[间]{s.}{paraíso, céu; em algumas religiões, refere-se ao paraíso para onde as almas das pessoas boas retornam após a morte (diferente do 地狱) | lugar perfeito; ambiente de vida extremamente feliz e bonito; uma metáfora para um ambiente de vida feliz e bonito}
  \seealsoref{地狱}{di4yu4}
  \end{Phonetics}
\end{Entry}

\begin{Entry}{天然}{4,12}{⼤、⽕}
  \begin{Phonetics}{天然}{tian1 ran2}[][HSK 6]
    \definition{adj.}{natural; produzido ou ocorrido narturalmente}
  \end{Phonetics}
\end{Entry}

\begin{Entry}{天然气}{4,12,4}{⼤、⽕、⽓}
  \begin{Phonetics}{天然气}{tian1ran2qi4}[][HSK 5]
    \definition{s.}{gás; gás natural; gás combustível produzido em campos petrolíferos, carboníferos e pântanos}
  \end{Phonetics}
\end{Entry}

\begin{Entry}{天鹅}{4,12}{⼤、⿃}
  \begin{Phonetics}{天鹅}{tian1'e2}
    \definition{s.}{cisne}
  \end{Phonetics}
\end{Entry}

%%%%%%%%%% 太 %%%%%%%%%%
\subsection*{太}\addcontentsline{loh}{figure}{太}

\begin{Entry}{太}{4}{⼤}
  \begin{Phonetics}{太}{tai4}[][HSK 1]
    \definition*{s.}{Sobrenome: Tai}
    \definition{adj.}{mais alto; maior; mais distante | maior; extremo | bisavô; mais velho ou mais antigo; o de maior posição social ou hierarquia}
    \definition{adv.}{demais; expressa um grau excessivo (usado principalmente para coisas indesejáveis) | muito; extremamente; excessivamente; indica um grau extremamente elevado | muito; usado após o advérbio negativo 不, enfraquece o grau de negação e contém um tom diplomático}
  \end{Phonetics}
\end{Entry}

\begin{Entry}{太太}{4,4}{⼤、⼤}
  \begin{Phonetics}{太太}{tai4tai5}[][HSK 2]
    \definition[位,名,个,些]{s.}{senhora; madame; títulos para mulheres casadas | esposa; senhora; madame; referir-se à própria esposa ou à esposa de outra pessoa}
  \end{Phonetics}
\end{Entry}

\begin{Entry}{太平洋}{4,5,9}{⼤、⼲、⽔}
  \begin{Phonetics}{太平洋}{tai4ping2 yang2}
    \definition*{s.}{Oceano Pacífico}
  \end{Phonetics}
\end{Entry}

\begin{Entry}{太阳}{4,6}{⼤、⾩}
  \begin{Phonetics}{太阳}{tai4yang5}[][HSK 2]
    \definition[个,轮,枚,颗,盏]{s.}{o Sol | luz do sol; luz solar}
  \end{Phonetics}
\end{Entry}

\begin{Entry}{太阳日}{4,6,4}{⼤、⾩、⽇}
  \begin{Phonetics}{太阳日}{tai4yang2ri4}
    \definition{s.}{dia solar}
  \end{Phonetics}
\end{Entry}

\begin{Entry}{太阳风}{4,6,4}{⼤、⾩、⾵}
  \begin{Phonetics}{太阳风}{tai4yang2feng1}
    \definition{s.}{vento solar}
  \end{Phonetics}
\end{Entry}

\begin{Entry}{太阳穴}{4,6,5}{⼤、⾩、⽳}
  \begin{Phonetics}{太阳穴}{tai4yang2xue2}
    \definition{s.}{têmpora (nas laterais da cabeça humana)}
  \end{Phonetics}
\end{Entry}

\begin{Entry}{太阳灯}{4,6,6}{⼤、⾩、⽕}
  \begin{Phonetics}{太阳灯}{tai4yang2deng1}
    \definition{s.}{lâmpada solar (com células fotovoltaicas)}
  \end{Phonetics}
\end{Entry}

\begin{Entry}{太阳雨}{4,6,8}{⼤、⾩、⾬}
  \begin{Phonetics}{太阳雨}{tai4yang2yu3}
    \definition{s.}{banho de sol}
  \end{Phonetics}
\end{Entry}

\begin{Entry}{太阳能}{4,6,10}{⼤、⾩、⾁}
  \begin{Phonetics}{太阳能}{tai4 yang2 neng2}[][HSK 6]
    \definition{s.}{energia solar; a energia de radiação emitida pelo Sol é a energia solar produzida pela reação de fusão dos núcleos de hidrogênio no Sol, é a fonte de luz e calor na Terra}
  \end{Phonetics}
\end{Entry}

\begin{Entry}{太阳窗}{4,6,12}{⼤、⾩、⽳}
  \begin{Phonetics}{太阳窗}{tai4yang2chuang1}
    \definition{s.}{teto solar (de veículos)}
  \end{Phonetics}
\end{Entry}

\begin{Entry}{太阳镜}{4,6,16}{⼤、⾩、⾦}
  \begin{Phonetics}{太阳镜}{tai4yang2jing4}
    \definition{s.}{óculos de sol}
  \end{Phonetics}
\end{Entry}

\begin{Entry}{太阳翼}{4,6,17}{⼤、⾩、⽻}
  \begin{Phonetics}{太阳翼}{tai4yang2yi4}
    \definition{s.}{painel solar}
  \end{Phonetics}
\end{Entry}

\begin{Entry}{太极拳}{4,7,10}{⼤、⽊、⼿}
  \begin{Phonetics}{太极拳}{tai4ji2quan2}
    \definition*{s.}{Tai Chi Chuan, Taiji, T'aichi ou T'aichichuan; forma tradicional de exercício físico ou relaxamento}
  \end{Phonetics}
\end{Entry}

\begin{Entry}{太空}{4,8}{⼤、⽳}
  \begin{Phonetics}{太空}{tai4kong1}[][HSK 5]
    \definition[把]{s.}{firmamento; espaço sideral; espaço além da atmosfera terrestre; o céu vasto e infinito}
  \end{Phonetics}
\end{Entry}

%%%%%%%%%% 夫 %%%%%%%%%%
\subsection*{夫}\addcontentsline{loh}{figure}{夫}

\begin{Entry}{夫}{4}{⼤}
  \begin{Phonetics}{夫}{fu1}
    \definition{s.}{marido | homem | (velho) alguém que faz algum tipo de trabalho manual | (velho) uma pessoa que serviu em trabalho forçado}
  \end{Phonetics}
  \begin{Phonetics}{夫}{fu2}
    \definition{part.}{usado no início de uma frase | usado no final de uma frase ou em uma pausa no meio de uma frase para expressar uma exclamação}
    \definition{pron.}{isto; isso; aqueles; estes | ele}
  \end{Phonetics}
\end{Entry}

\begin{Entry}{夫人}{4,2}{⼤、⼈}
  \begin{Phonetics}{夫人}{fu1ren2}[][HSK 4]
    \definition[位,名,个]{s.}{senhora; \emph{lady}; madame; na antiguidade, as esposas dos senhores feudais eram chamadas de ``madame'' e, nas dinastias Ming e Qing, as esposas dos oficiais de primeiro e segundo escalão eram chamadas de ``madame'', que mais tarde foi usada para homenagear as esposas das pessoas em geral e agora é usada principalmente em ocasiões diplomáticas}
  \end{Phonetics}
\end{Entry}

\begin{Entry}{夫妇}{4,6}{⼤、⼥}
  \begin{Phonetics}{夫妇}{fu1fu4}[][HSK 4]
    \definition[对]{s.}{casal; marido e mulher}
  \end{Phonetics}
\end{Entry}

\begin{Entry}{夫妻}{4,8}{⼤、⼥}
  \begin{Phonetics}{夫妻}{fu1qi1}[][HSK 4]
    \definition[对]{s.}{casal; marido e mulher}
  \end{Phonetics}
\end{Entry}

%%%%%%%%%% 厺 %%%%%%%%%%
\subsection*{厺}\addcontentsline{loh}{figure}{厺}

\begin{Entry}{厺}{5}{⼤}
  \begin{Phonetics}{厺}{qu4}
    \variantof{去}
  \end{Phonetics}
\end{Entry}

%%%%%%%%%% 失 %%%%%%%%%%
\subsection*{失}\addcontentsline{loh}{figure}{失}

\begin{Entry}{失}{5}{⼤}
  \begin{Phonetics}{失}{shi1}
    \definition{s.}{deslize; erro; defeito; acidente}
    \definition{v.}{perder (oposto de 得) | perder; deixar escapar | não agir de acordo com; negligenciar; violar | perder o controle de | errar; cometer um deslize; apresentar defeito em | não consiguir encontrar | não conseguir atingir o objetivo | desviar-se do normal | quebrar (uma promessa); voltar atrás (na palavra dada) | não conseguir obter | se perder}
  \seealsoref{得}{de2}
  \end{Phonetics}
\end{Entry}

\begin{Entry}{失业}{5,5}{⼤、⼀}
  \begin{Phonetics}{失业}{shi1ye4}[][HSK 4]
    \definition{v.}{não ter emprego; estar desempregado; estar sem trabalho; refere-se àqueles que estão dentro da idade legal para trabalhar, têm capacidade para trabalhar, estão desempregados e querem encontrar um emprego, mas não conseguem; embora se envolvam em certos trabalhos sociais, sua remuneração é menor do que o padrão mínimo de vida urbano local e são considerados desempregados}
  \end{Phonetics}
\end{Entry}

\begin{Entry}{失去}{5,5}{⼤、⼛}
  \begin{Phonetics}{失去}{shi1qu4}[][HSK 3]
    \definition{v.}{perder}
  \end{Phonetics}
\end{Entry}

\begin{Entry}{失败}{5,8}{⼤、⾒}
  \begin{Phonetics}{失败}{shi1bai4}[][HSK 4]
    \definition{adj.}{insatisfatório; a maneira como as coisas aconteceram deixou muito a desejar; o resultado final deixou muito a desejar}
    \definition{v.}{perder; ser derrotado; não vencer em uma guerra ou competição | falhar; fracassar; não dar em nada; falhar em atingir um objetivo ou meta desejada (trabalho, carreira, etc.)}
  \end{Phonetics}
\end{Entry}

\begin{Entry}{失误}{5,9}{⼤、⾔}
  \begin{Phonetics}{失误}{shi1wu4}[][HSK 5]
    \definition[个]{s.}{erro; engano; equívoco; erros causados por negligência ou medidas inadequadas}
    \definition{v.}{cometer um erro; cometer um equívoco}
  \end{Phonetics}
\end{Entry}

\begin{Entry}{失眠}{5,10}{⼤、⽬}
  \begin{Phonetics}{失眠}{shi1mian2}
    \definition{s.}{insônia}
    \definition{v.}{ter insônia}
  \end{Phonetics}
\end{Entry}

\begin{Entry}{失望}{5,11}{⼤、⽉}
  \begin{Phonetics}{失望}{shi1wang4}[][HSK 4]
    \definition{adj.}{desapontado; decepcionado}
    \definition{v.}{ficar desapontado; ficar decepcionado; estar desapontado; sentir-se sem esperança; perder a confiança}
  \end{Phonetics}
\end{Entry}

\begin{Entry}{失落}{5,12}{⼤、⾋}
  \begin{Phonetics}{失落}{shi1luo4}
    \definition{s.}{frustração | decepção | perda}
    \definition{v.}{perder (algo) | cair (algo) | sentir uma sensação de perda}
  \end{Phonetics}
\end{Entry}

\begin{Entry}{失意}{5,13}{⼤、⼼}
  \begin{Phonetics}{失意}{shi1yi4}
    \definition{adj.}{desapontado | frustrado}
  \end{Phonetics}
\end{Entry}

%%%%%%%%%% 头 %%%%%%%%%%
\subsection*{头}\addcontentsline{loh}{figure}{头}

\begin{Entry}{头}{5}{⼤}
  \begin{Phonetics}{头}{tou2}[][HSK 2,3]
    \definition{adj.}{(antes de um numeral) primeiro | (antes de 年 ou 天) último; anterior}
    \definition{clas.}{usado para suínos ou gado (animais de criação) | usado para cabeças de alho ou coisas com formato de cabeça}
    \definition{num.}{primeiro}
    \definition{prep.}{antes de; perto de; introduz o tempo de uma ação, equivalente a  在……之前 ou 临近 | (entre dois algarismos, indicando um número aproximado) cerca de}
    \definition[个,颗]{s.}{cabeça; a parte do corpo humano ou animal que possui órgãos como boca, nariz, olhos e ouvidos | cabelo ou penteado | topo; fim; a parte superior ou final de um objeto | começo ou fim; o ponto inicial ou final de algo | fim; remanescente; os restos de algo | cabeça; chefe; líder | lado; aspecto}
  \seealsoref{临近}{lin2jin4}
  \seealsoref{年}{nian2}
  \seealsoref{天}{tian1}
  \seealsoref{在}{zai4}
  \seealsoref{之前}{zhi1 qian2}
  \end{Phonetics}
  \begin{Phonetics}{头}{tou5}
    \definition{suf.}{adicionado após componentes nominais comuns | adicionado após o componente verbal, forma um substantivo abstrato, geralmente indicando que vale a pena realizar essa ação | adicionado após um componente adjetival, forma um substantivo, geralmente indicando algo abstrato | adicionado após o componente substantivo que indica a direção}
  \end{Phonetics}
\end{Entry}

\begin{Entry}{头发}{5,5}{⼤、⼜}
  \begin{Phonetics}{头发}{tou2fa5}[][HSK 2]
    \definition[根,缕,头]{s.}{cabelo}
  \end{Phonetics}
\end{Entry}

\begin{Entry}{头号}{5,5}{⼤、⼝}
  \begin{Phonetics}{头号}{tou2hao4}
    \definition{adj.}{primeira classe | número um | \emph{top rank}}
  \end{Phonetics}
\end{Entry}

\begin{Entry}{头头}{5,5}{⼤、⼤}
  \begin{Phonetics}{头头}{tou2tou2}
    \definition{s.}{chefe | o cabeça}
  \end{Phonetics}
\end{Entry}

\begin{Entry}{头疼}{5,10}{⼤、⽧}
  \begin{Phonetics}{头疼}{tou2 teng2}[][HSK 6]
    \definition{s.}{dor de cabeça}
    \definition{v.}{estar preocupado ou incomodado por alguém ou algo}
  \end{Phonetics}
\end{Entry}

\begin{Entry}{头脑}{5,10}{⼤、⾁}
  \begin{Phonetics}{头脑}{tou2 nao3}[][HSK 3]
    \definition{s.}{inteligência; mente | pista; tópicos principais | chefe; líder; capitão}
  \end{Phonetics}
\end{Entry}

\begin{Entry}{头脑风暴}{5,10,4,15}{⼤、⾁、⾵、⽇}
  \begin{Phonetics}{头脑风暴}{tou2nao3feng1bao4}
    \definition{s.}{\emph{brainstorm}}
  \end{Phonetics}
\end{Entry}

\begin{Entry}{头像}{5,13}{⼤、⼈}
  \begin{Phonetics}{头像}{tou2xiang4}
    \definition{s.}{retrato | busto | avatar | imagem de perfil (computação)}
  \end{Phonetics}
\end{Entry}

%%%%%%%%%% 夸 %%%%%%%%%%
\subsection*{夸}\addcontentsline{loh}{figure}{夸}

\begin{Entry}{夸}{6}{⼤}
  \begin{Phonetics}{夸}{kua1}[][HSK 7-9]
    \definition{v.}{exagerar; superestimar; vangloriar-se | elogiar; destacar as qualidades positivas de uma pessoa ou coisa}
  \end{Phonetics}
\end{Entry}

\begin{Entry}{夸大}{6,3}{⼤、⼤}
  \begin{Phonetics}{夸大}{kua1da4}[][HSK 7-9]
    \definition{v.}{exagerar; superestimar; magnificar; engrandecer}
  \end{Phonetics}
\end{Entry}

\begin{Entry}{夸夸其谈}{6,6,8,10}{⼤、⼤、⼋、⾔}
  \begin{Phonetics}{夸夸其谈}{kua1kua1-qi2tan2}[][HSK 7-9]
    \definition{expr.}{entregar-se à verborragia; conversa empolgante e jactanciosa sem muito significado; cheio de ar quente; um longo discurso cheio de pompa; tagarelice; falar demais; uma grande conversa; discursar; conversa pomposa; conversa ou escrita pomposa, mas sem sentido; discurso pomposo; exagerar; desabafar; exagerar (ou usar verborragia); tagarelar; falar pelos cotovelos}
  \end{Phonetics}
\end{Entry}

\begin{Entry}{夸张}{6,7}{⼤、⼸}
  \begin{Phonetics}{夸张}{kua1zhang1}[][HSK 7-9]
    \definition{adj.}{exagerado; superestimado; pomposo}
    \definition{s.}{hipérbole; uma figura de linguagem que utiliza palavras exageradas para descrever coisas}
  \end{Phonetics}
\end{Entry}

\begin{Entry}{夸奖}{6,9}{⼤、⼤}
  \begin{Phonetics}{夸奖}{kua1jiang3}[][HSK 7-9]
    \definition{v.}{louvar; elogiar; cumprimentar; demonstrar apreço e incentivar alguém por suas qualidades ou boas ações}
  \end{Phonetics}
\end{Entry}

\begin{Entry}{夸耀}{6,20}{⼤、⽻}
  \begin{Phonetics}{夸耀}{kua1yao4}[][HSK 7-9]
    \definition{v.}{ostentar; gabar-se de; vangloriar-se de; exibir (as próprias habilidades, conquistas, status e poder)}
  \end{Phonetics}
\end{Entry}

%%%%%%%%%% 夹 %%%%%%%%%%
\subsection*{夹}\addcontentsline{loh}{figure}{夹}

\begin{Entry}{夹}{6}{⼤}
  \begin{Phonetics}{夹}{ga1}
    \definition{s.}{axila; sovaco; atualmente, costuma-se escrever 胳肢窝}
  \seealsoref{胳肢窝}{ga1 zhi1 wo1}
  \end{Phonetics}
  \begin{Phonetics}{夹}{jia1}[][HSK 5]
    \definition{s.}{clipe, grampo, pasta, etc.}
    \definition{v.}{colocar no meio; pressionar de ambos os lados; aplicar força ou ação ao mesmo objeto de ambos os lados ao mesmo tempo | misturar; mesclar; intercalar}
  \end{Phonetics}
  \begin{Phonetics}{夹}{jia2}
    \definition{adj.}{forrado; com camada dupla; duas camadas (roupas, colchas, etc.) | pinçado; voz deliberadamente engraçada}
  \end{Phonetics}
\end{Entry}

\begin{Entry}{夹子}{6,3}{⼤、⼦}
  \begin{Phonetics}{夹子}{jia1 zi5}
    \definition[个,堆,盒]{s.}{pasta; carteira; algo para guardar dinheiro, papel, etc. | clipe; grampo; pasta; pinça; ferramentas para prender coisas}
  \end{Phonetics}
\end{Entry}

\begin{Entry}{夹杂}{6,6}{⼤、⽊}
  \begin{Phonetics}{夹杂}{jia1 za2}
    \definition{v.}{ser misturado com; estar carregado de; adicionar (algo mais)}
  \end{Phonetics}
\end{Entry}

\begin{Entry}{夹肢窝}{6,8,12}{⼤、⾁、⽳}
  \begin{Phonetics}{夹肢窝}{jia1 zhi1 wo1}
    \definition{s.}{axila; sovaco; também escrito como 胳肢窝}
  \seealsoref{胳肢窝}{ga1 zhi1 wo1}
  \end{Phonetics}
\end{Entry}

%%%%%%%%%% 夺 %%%%%%%%%%
\subsection*{夺}\addcontentsline{loh}{figure}{夺}

\begin{Entry}{夺}{6}{⼤}
  \begin{Phonetics}{夺}{duo2}[][HSK 6]
    \definition{v.}{tomar à força; apreender; arrancar; roubar | forçar a passagem; empurrar para abrir | lutar por; competir por; esforçar-se por; obter primeiro | privar; perder | perder; tirar | decidir; tomar uma decisão | omitir (palavra em um texto)}
  \end{Phonetics}
\end{Entry}

\begin{Entry}{夺取}{6,8}{⼤、⼜}
  \begin{Phonetics}{夺取}{duo2 qu3}[][HSK 6]
    \definition{v.}{capturar; apreender; arrancar; tomar à força | esforçar-se para; alcançar}
  \end{Phonetics}
\end{Entry}

\begin{Entry}{夺冠}{6,9}{⼤、⼍}
  \begin{Phonetics}{夺冠}{duo2/guan4}[][HSK 7-9]
    \definition{v.+compl.}{chegar em primeiro lugar; ficar em primeiro lugar; ganhar um campeonato}
  \end{Phonetics}
\end{Entry}

\begin{Entry}{夺魁}{6,13}{⼤、⿁}
  \begin{Phonetics}{夺魁}{duo2/kui2}[][HSK 7-9]
    \definition{v.+compl.}{Literário: ganhar o primeiro prêmio; ganhar o campeonato; conquistar; vencer}
  \end{Phonetics}
\end{Entry}

%%%%%%%%%% 奇 %%%%%%%%%%
\subsection*{奇}\addcontentsline{loh}{figure}{奇}

\begin{Entry}{奇}{8}{⼤}
  \begin{Phonetics}{奇}{qi2}
    \definition{adj.}{ímpar (número); singular; solteiro; não em pares (ao contrário de 偶)}
    \definition{s.}{lotes ímpares; quantidade fracionária (acima daquela mencionada em um número redondo)}
  \seealsoref{偶}{ou3}
  \end{Phonetics}
\end{Entry}

\begin{Entry}{奇妙}{8,7}{⼤、⼥}
  \begin{Phonetics}{奇妙}{qi2miao4}[][HSK 6]
    \definition{adj.}{maravilhoso; milagroso; intrigante; muito inteligente e engenhoso (usado principalmente para descrever coisas interessantes e novas)}
  \end{Phonetics}
\end{Entry}

\begin{Entry}{奇花异草}{8,7,6,9}{⼤、⾋、⼶、⾋}
  \begin{Phonetics}{奇花异草}{qi2hua1-yi4cao3}[][HSK 7-9]
    \definition{expr.}{``Flores exóticas e ervas raras.''; vista espetacular do mundo botânico; muito raramente visto, incomum}
  \end{Phonetics}
\end{Entry}

\begin{Entry}{奇怪}{8,8}{⼤、⼼}
  \begin{Phonetics}{奇怪}{qi2guai4}[][HSK 3]
    \definition{adj.}{estranho; diferente do habitual; raramente visto, até um pouco irracional | estranho; esquisito; a descrição é diferente do imaginado e é difícil de entender}
    \definition{v.}{ficar perplexo; maravilhar-se; sentir-se surpreso; sentir-se estranho; sentir-se incompreensível}
  \end{Phonetics}
\end{Entry}

\begin{Entry}{奇迹}{8,9}{⼤、⾡}
  \begin{Phonetics}{奇迹}{qi2ji4}[][HSK 7-9]
    \definition[个,种]{s.}{milagre; maravilha; coisas extraordinárias inimagináveis}
  \end{Phonetics}
\end{Entry}

\begin{Entry}{奇特}{8,10}{⼤、⽜}
  \begin{Phonetics}{奇特}{qi2te4}[][HSK 7-9]
    \definition{adj.}{estranho; peculiar; singular; incomum; extraordinário}
  \end{Phonetics}
\end{Entry}

%%%%%%%%%% 奉 %%%%%%%%%%
\subsection*{奉}\addcontentsline{loh}{figure}{奉}

\begin{Entry}{奉}{8}{⼤}
  \begin{Phonetics}{奉}{feng4}
    \definition*{s.}{Sobrenome: Feng}
    \definition{v.}{Literário: dedicar ou presentear com respeito | receber (pedidos, instruções, etc.) | Literário: estimar; reverenciar | Litrário: acreditar em  | esperar; atender; servir}
  \end{Phonetics}
\end{Entry}

\begin{Entry}{奉献}{8,13}{⼤、⽝}
  \begin{Phonetics}{奉献}{feng4xian4}[][HSK 6]
    \definition{v.}{dedicar; oferecer como tributo; apresentar com todo respeito; entregar respeitosamente}
  \end{Phonetics}
\end{Entry}

%%%%%%%%%% 奋 %%%%%%%%%%
\subsection*{奋}\addcontentsline{loh}{figure}{奋}

\begin{Entry}{奋}{8}{⼤}
  \begin{Phonetics}{奋}{fen4}
    \definition{adv.}{energicamente; com força e espírito}
    \definition{v.}{esforçar-se; agir vigorosamente; preparar-se | levantar | aplicar energia; resolver; animar-se | acenar; sacudir; levantar}
  \end{Phonetics}
\end{Entry}

\begin{Entry}{奋力}{8,2}{⼤、⼒}
  \begin{Phonetics}{奋力}{fen4li4}[][HSK 7-9]
    \definition{v.}{fazer tudo o que puder; não poupar esforços}
  \end{Phonetics}
\end{Entry}

\begin{Entry}{奋斗}{8,4}{⼤、⽃}
  \begin{Phonetics}{奋斗}{fen4dou4}[][HSK 4]
    \definition{v.}{lutar; esforçar-se; batalhar; trabalhar duro para atingir um determinado objetivo}
  \end{Phonetics}
\end{Entry}

\begin{Entry}{奋勇}{8,9}{⼤、⼒}
  \begin{Phonetics}{奋勇}{fen4yong3}[][HSK 7-9]
    \definition{v.}{reunir toda a coragem e energia; criar coragem}
  \end{Phonetics}
\end{Entry}

\begin{Entry}{奋战}{8,9}{⼤、⼽}
  \begin{Phonetics}{奋战}{fen4zhan4}
    \definition{v.}{lutar bravamente | trabalhar duro}
  \end{Phonetics}
\end{Entry}

%%%%%%%%%% 奔 %%%%%%%%%%
\subsection*{奔}\addcontentsline{loh}{figure}{奔}

\begin{Entry}{奔}{8}{⼤}
  \begin{Phonetics}{奔}{ben1}
    \definition{v.}{correr rápido; correr com pressa | apressar | fugir; escapar | galopar | fugir; termo antigo para uma mulher que foge com um homem}
  \end{Phonetics}
  \begin{Phonetics}{奔}{ben4}[][HSK 7-9]
    \definition{prep.}{em direção a}
    \definition{v.}{ir direto em direção a; seguir em direção a; ir direto para o seu destino | aproximar-se; estar prestes a | estar ocupado correndo por aí; correr por algo}
  \end{Phonetics}
\end{Entry}

\begin{Entry}{奔驰}{8,6}{⼤、⾺}
  \begin{Phonetics}{奔驰}{ben1chi2}
    \definition*{s.}{Benz de Mercedes-Benz}
    \definition{v.}{acelerar; galopar; (carro, cavalo, etc.) mover-se ou correr rapidamente}
  \seealsoref{梅赛德斯-奔驰}{mei2sai4de2si1-ben1chi2}
  \end{Phonetics}
\end{Entry}

\begin{Entry}{奔波}{8,8}{⼤、⽔}
  \begin{Phonetics}{奔波}{ben1bo1}[][HSK 7-9]
    \definition{v.}{correr; estar ocupado correndo; correr para frente e para trás, com dificuldade e ocupado}
  \end{Phonetics}
\end{Entry}

\begin{Entry}{奔赴}{8,9}{⼤、⾛}
  \begin{Phonetics}{奔赴}{ben1fu4}[][HSK 7-9]
    \definition{v.}{correr para; apressar-se para; correr em direção a (um certo destino)}
  \end{Phonetics}
\end{Entry}

\begin{Entry}{奔跑}{8,12}{⼤、⾜}
  \begin{Phonetics}{奔跑}{ben1 pao3}[][HSK 6]
    \definition{v.}{correr; correr muito rápido, com uma gama de aplicações mais ampla do que 奔驰, usado principalmente na linguagem falada}
  \seealsoref{奔驰}{ben1chi2}
  \end{Phonetics}
\end{Entry}

%%%%%%%%%% 奏 %%%%%%%%%%
\subsection*{奏}\addcontentsline{loh}{figure}{奏}

\begin{Entry}{奏}{9}{⼤}
  \begin{Phonetics}{奏}{zou4}[][HSK 6]
    \definition{v.}{tocar (música); executar (em um instrumento musical)  | alcançar; produzir; alcançar ou estabelecer (desempenho ou realização) | (antigo) apresentar um memorial a um imperador; fazer uma petição}
  \end{Phonetics}
\end{Entry}

\begin{Entry}{奏效}{9,10}{⼤、⽁}
  \begin{Phonetics}{奏效}{zou4xiao4}
    \definition{v.}{mostrar resultados | ser eficaz}
  \end{Phonetics}
\end{Entry}

%%%%%%%%%% 契 %%%%%%%%%%
\subsection*{契}\addcontentsline{loh}{figure}{契}

\begin{Entry}{契}{9}{⼤}
  \begin{Phonetics}{契}{qi4}
    \definition{s.}{contrato; escritura | Arcaico: personagens esculpidos}
    \definition{v.}{Literário: gravar; esculpir | concordar; dar-se bem}
  \end{Phonetics}
  \begin{Phonetics}{契}{xie4}
    \definition*{s.}{Xie (ancestral da dinastia Yin, considerado ministro do Imperador Shun)}
  \end{Phonetics}
\end{Entry}

\begin{Entry}{契机}{9,6}{⼤、⽊}
  \begin{Phonetics}{契机}{qi4ji1}[][HSK 7-9]
    \definition{s.}{oportunidade; ponto de virada; a chave para mudar as coisas numa direção favorável}
  \end{Phonetics}
\end{Entry}

\begin{Entry}{契约}{9,6}{⼤、⽷}
  \begin{Phonetics}{契约}{qi4yue1}[][HSK 7-9]
    \definition{s.}{escritura; carta; contrato; documentos comprovativos de vendas, hipotecas, arrendamentos, etc.}
  \end{Phonetics}
\end{Entry}

%%%%%%%%%% 奖 %%%%%%%%%%
\subsection*{奖}\addcontentsline{loh}{figure}{奖}

\begin{Entry}{奖}{9}{⼤}
  \begin{Phonetics}{奖}{jiang3}[][HSK 4]
    \definition[个,次]{s.}{prêmio; recompensa | elogio; loa}
    \definition{v.}{elogiar; recompensar; recomendar; incentivar}
  \end{Phonetics}
\end{Entry}

\begin{Entry}{奖励}{9,7}{⼤、⼒}
  \begin{Phonetics}{奖励}{jiang3li4}[][HSK 5]
    \definition{s.}{prêmio; recompensa; dinheiro ou honras dadas em troca de elogios ou incentivos}
    \definition{v.}{recompensar; incentivar; encorajar}
  \end{Phonetics}
\end{Entry}

\begin{Entry}{奖学金}{9,8,8}{⼤、⼦、⾦}
  \begin{Phonetics}{奖学金}{jiang3 xue2 jin1}[][HSK 4]
    \definition[笔]{s.}{bolsa de estudos; exposição; prêmios concedidos por escolas, organizações ou indivíduos a alunos com bom desempenho acadêmico}
  \end{Phonetics}
\end{Entry}

\begin{Entry}{奖杯}{9,8}{⼤、⽊}
  \begin{Phonetics}{奖杯}{jiang3bei1}[][HSK 7-9]
    \definition[个,座]{s.}{taça (como prêmio); copa; troféu; os prêmios em formato de taça, concedidos aos vencedores em competições esportivas, geralmente são feitos de ouro ou prata}
  \end{Phonetics}
\end{Entry}

\begin{Entry}{奖金}{9,8}{⼤、⾦}
  \begin{Phonetics}{奖金}{jiang3jin1}[][HSK 4]
    \definition[个,笔]{s.}{bônus; recompensa; prêmio; prêmio em dinheiro; dinheiro de recompensa, dinheiro dado às pessoas para incentivá-las ou elogiá-las por terem se saído bem em alguma coisa}
  \end{Phonetics}
\end{Entry}

\begin{Entry}{奖品}{9,9}{⼤、⼝}
  \begin{Phonetics}{奖品}{jiang3pin3}[][HSK 7-9]
    \definition[个,些,份]{s.}{prêmio; itens para recompensa}
  \end{Phonetics}
\end{Entry}

\begin{Entry}{奖项}{9,9}{⼤、⾴}
  \begin{Phonetics}{奖项}{jiang3xiang4}[][HSK 7-9]
    \definition[项]{s.}{prêmio; projetos premiados}
  \end{Phonetics}
\end{Entry}

\begin{Entry}{奖牌}{9,12}{⼤、⽚}
  \begin{Phonetics}{奖牌}{jiang3pai2}[][HSK 7-9]
    \definition{s.}{medalha (concedida como prêmio); as medalhas são divididas em ouro, prata e bronze; os vencedores em competições esportivas recebem prêmios de acordo com esses três níveis}
  \end{Phonetics}
\end{Entry}

%%%%%%%%%% 套 %%%%%%%%%%
\subsection*{套}\addcontentsline{loh}{figure}{套}

\begin{Entry}{套}{10}{⼤}
  \begin{Phonetics}{套}{tao4}[][HSK 2]
    \definition{clas.}{usado para coisas agrupadas: conjuntos, coleções, séries, etc.}
    \definition{s.}{estojo; capa; bainha | local onde o rio ou a cordilheira faz uma curva (usado principalmente em nomes de lugares) | enchimento de algodão em roupas e edredons | arreios; corda para amarrar animais | nó; laço; um objeto circular feito com corda ou algo semelhante | cortersia; convenção; conversa fiada; métodos repetitivos | armadilha; truque; conspiração}
    \definition{v.}{sobrepor; interligar | deslizar sobre; cobrir por fora | atrelar; engatar; usar um cinto de segurança | copiar; imitar; seguir o modelo de | extrair; induzir a falar; persuadir alguém a revelar um segredo; induzir; provocar | tentar vencer; aproximar-se de; aproximar-se intencionalmente de outras pessoas para algum propósito | fazer a rosca de um parafuso; usar um macho de rosca ou uma chave de rosca para fazer roscas}
  \end{Phonetics}
\end{Entry}

\begin{Entry}{套问}{10,6}{⼤、⾨}
  \begin{Phonetics}{套问}{tao4wen4}
    \definition{s.}{retórica}
    \definition{v.}{descobrir por meio de questionamento indireto diplomático}
  \end{Phonetics}
\end{Entry}

\begin{Entry}{套餐}{10,16}{⼤、⾷}
  \begin{Phonetics}{套餐}{tao4 can1}[][HSK 4]
    \definition{s.}{combo; pacote de produtos; pacote de serviços; metaforicamente, bens ou projetos que são combinados e levados ao mercado | refeição preparada; pacotes de refeições completos}
  \end{Phonetics}
\end{Entry}

%%%%%%%%%% 奠 %%%%%%%%%%
\subsection*{奠}\addcontentsline{loh}{figure}{奠}

\begin{Entry}{奠}{12}{⼤}
  \begin{Phonetics}{奠}{dian4}
    \definition{v.}{estabelecer; construir; fundar; lançar a pedra fundamental | fazer oferendas aos espíritos dos mortos | consertar}
  \end{Phonetics}
\end{Entry}

\begin{Entry}{奠定}{12,8}{⼤、⼧}
  \begin{Phonetics}{奠定}{dian4ding4}[][HSK 7-9]
    \definition{v.}{estabelecer; estabelecer de forma estável; tornar estável; estabelecer a base}
  \end{Phonetics}
\end{Entry}

%%%%%%%%%% 奥 %%%%%%%%%%
\subsection*{奥}\addcontentsline{loh}{figure}{奥}

\begin{Entry}{奥}{12}{⼤}
  \begin{Phonetics}{奥}{ao4}
    \definition*{s.}{Oersted, a unidade eletromagnética de intensidade do campo magnético; abreviação de 奥斯特 | Sobrenome: Ao}
    \definition{adj.}{profundo e difícil de entender; abstruso | significado profundo, não é fácil de entender}
    \definition{s.}{canto secreto da casa; antigamente, referia-se ao canto sudoeste de uma casa e também, de modo geral, à profundidade de uma casa}
  \seealsoref{奥斯特}{ao4 si1 te4}
  \end{Phonetics}
\end{Entry}

\begin{Entry}{奥运}{12,7}{⼤、⾡}
  \begin{Phonetics}{奥运}{ao4yun4}
    \definition*{s.}{Jogos Olímpicos, Olimpíadas; Abreviação de 奥林匹克运动会}
  \seealsoref{奥林匹克运动会}{ao4lin2pi3ke4 yun4dong4hui4}
  \end{Phonetics}
\end{Entry}

\begin{Entry}{奥运会}{12,7,6}{⼤、⾡、⼈}
  \begin{Phonetics}{奥运会}{ao4yun4hui4}[][HSK 7-9]
    \definition*[届,次]{s.}{Jogos Olímpicos, Olimpíadas; Abreviação de 奥林匹克运动会}
  \seealsoref{奥林匹克运动会}{ao4lin2pi3ke4 yun4dong4hui4}
  \end{Phonetics}
\end{Entry}

\begin{Entry}{奥林匹克运动会}{12,8,4,7,7,6,6}{⼤、⽊、⼖、⼗、⾡、⼒、⼈}
  \begin{Phonetics}{奥林匹克运动会}{ao4lin2pi3ke4 yun4dong4hui4}
    \definition*{s.}{Jogos Olímpicos, Olimpíadas}
  \end{Phonetics}
\end{Entry}

\begin{Entry}{奥特曼}{12,10,11}{⼤、⽜、⽈}
  \begin{Phonetics}{奥特曼}{ao4te4man4}
    \definition*{s.}{Ultraman,  super-herói de ficção científica japonesa}
  \end{Phonetics}
\end{Entry}

\begin{Entry}{奥秘}{12,10}{⼤、⽲}
  \begin{Phonetics}{奥秘}{ao4mi4}[][HSK 7-9]
    \definition[个]{s.}{enigma; mistério profundo; fenômenos ou princípios profundos e misteriosos}
  \end{Phonetics}
\end{Entry}

\begin{Entry}{奥斯特}{12,12,10}{⼤、⽄、⽜}
  \begin{Phonetics}{奥斯特}{ao4 si1 te4}
    \definition{s.}{Oersted}
  \end{Phonetics}
\end{Entry}

%%%%% EOF %%%%%

