%%%
%%% Radical "⼤"
%%%

\section*{Radical 37: ``⼤''}\addcontentsline{toc}{section}{Radical 37: ⼤}

\begin{entry}{大}{3}{⼤}[Kangxi 37]
  \begin{phonetics}{大}{da4}[][HSK 1]
    \definition{adj.}{grande | enorme | maior | largo | profundo | mais velho (que) | mais antigo | mais velho | muito}
    \definition{s.}{(dialeto) pai | irmão mais velho ou mais novo do pai}
  \end{phonetics}
  \begin{phonetics}{大}{dai4}
    \definition{s.}{usado em 大夫 \dpy{dai4fu5}: médico, doutor}
    \seeref{大夫}{dai4fu5}
  \end{phonetics}
\end{entry}

\begin{entry}{大人}{3,2}{⼤、⼈}
  \begin{phonetics}{大人}{da4 ren2}[][HSK 2]
    \definition{s.}{adulto}
  \end{phonetics}
\end{entry}

\begin{entry}{大于}{3,3}{⼤、⼆}
  \begin{phonetics}{大于}{da4 yu2}[][HSK 5]
    \definition{v.}{ser maior, mais numeroso, mais importante, etc. do que}
  \end{phonetics}
\end{entry}

\begin{entry}{大口}{3,3}{⼤、⼝}
  \begin{phonetics}{大口}{da4kou3}
    \definition{s.}{grande bocado (de comida, bebida, fumo, etc.)}
  \end{phonetics}
\end{entry}

\begin{entry}{大大}{3,3}{⼤、⼤}
  \begin{phonetics}{大大}{da4 da4}[][HSK 2]
    \definition{adv.}{muito; enormemente}
  \end{phonetics}
\end{entry}

\begin{entry}{大小}{3,3}{⼤、⼩}
  \begin{phonetics}{大小}{da4 xiao3}[][HSK 2]
    \definition{adv.}{no mínimo}
    \definition[家]{s.}{tamanho | grau de antiguidade | adultos e crianças | grande ou pequeno}
  \end{phonetics}
\end{entry}

\begin{entry}{大门}{3,3}{⼤、⾨}
  \begin{phonetics}{大门}{da4 men2}[][HSK 2]
    \definition{s.}{portão | entrada}
  \end{phonetics}
\end{entry}

\begin{entry}{大马}{3,3}{⼤、⾺}
  \begin{phonetics}{大马}{da4ma3}
    \definition*{s.}{Malásia}
  \end{phonetics}
\end{entry}

\begin{entry}{大厅}{3,4}{⼤、⼚}
  \begin{phonetics}{大厅}{da4 ting1}[][HSK 5]
    \definition{s.}{\emph{hall}; saguão, uma sala grande para reuniões ou atividades em um edifício de grande porte}
  \end{phonetics}
\end{entry}

\begin{entry}{大夫}{3,4}{⼤、⼤}
  \begin{phonetics}{大夫}{da4fu1}
    \definition{s.}{oficial sênior (na China Imperial)}
  \end{phonetics}
  \begin{phonetics}{大夫}{dai4fu5}[][HSK 3]
    \definition{s.}{médico, doutor}
  \end{phonetics}
\end{entry}

\begin{entry}{大巴}{3,4}{⼤、⼰}
  \begin{phonetics}{大巴}{da4 ba1}[][HSK 4]
    \definition{s.}{ônibus}
  \end{phonetics}
\end{entry}

\begin{entry}{大方}{3,4}{⼤、⽅}
  \begin{phonetics}{大方}{da4fang5}[][HSK 4]
    \definition{adj.}{generoso | não afetado; natural e equilibrado |  de bom gosto}
  \end{phonetics}
\end{entry}

\begin{entry}{大众}{3,6}{⼤、⼈}
  \begin{phonetics}{大众}{da4 zhong4}[][HSK 4]
    \definition{s.}{massas; população; pessoas comuns; público em geral}
  \end{phonetics}
\end{entry}

\begin{entry}{大伙儿}{3,6,2}{⼤、⼈、⼉}
  \begin{phonetics}{大伙儿}{da4huo3r5}[][HSK 5]
    \definition{pron.}{todos nós; todos vocês; todo mundo; todos | equivalente a ``大家''}
  \seealsoref{大家}{da4jia1}
  \end{phonetics}
\end{entry}

\begin{entry}{大会}{3,6}{⼤、⼈}
  \begin{phonetics}{大会}{da4 hui4}[][HSK 4]
    \definition{s.}{sessão plenária; reunião geral de membros; reuniões convocadas por partidos políticos socialistas | reunião de massa; comício de massa}
  \end{phonetics}
\end{entry}

\begin{entry}{大全}{3,6}{⼤、⼊}
  \begin{phonetics}{大全}{da4quan2}
    \definition{s.}{coleção abrangente}
  \end{phonetics}
\end{entry}

\begin{entry}{大后天}{3,6,4}{⼤、⼝、⼤}
  \begin{phonetics}{大后天}{da4 hou4 tian1}
    \definition{adv.}{daqui a três dias}
  \end{phonetics}
\end{entry}

\begin{entry}{大多}{3,6}{⼤、⼣}
  \begin{phonetics}{大多}{da4 duo1}[][HSK 4]
    \definition{adv.}{majoritariamente; em sua maior parte; em sua maioria; em grande parte}
  \end{phonetics}
\end{entry}

\begin{entry}{大多数}{3,6,13}{⼤、⼣、⽁}
  \begin{phonetics}{大多数}{da4 duo1 shu4}[][HSK 2]
    \definition{s.}{a grande maioria | a vasta maioria | a maior parte}
  \end{phonetics}
\end{entry}

\begin{entry}{大妈}{3,6}{⼤、⼥}
  \begin{phonetics}{大妈}{da4 ma1}[][HSK 4]
    \definition{s.}{tia; esposa do irmão mais velho do pai | tia (homenagem às mulheres idosas)}
  \end{phonetics}
\end{entry}

\begin{entry}{大戏}{3,6}{⼤、⼽}
  \begin{phonetics}{大戏}{da4xi4}
    \definition*{s.}{Drama, Ópera Chinesa}
  \end{phonetics}
\end{entry}

\begin{entry}{大爷}{3,6}{⼤、⽗}
  \begin{phonetics}{大爷}{da4 ye5}[][HSK 4]
    \definition{s.}{irmão mais velho do pai; tio | tio (homenagem aos homens mais velhos)}
  \end{phonetics}
\end{entry}

\begin{entry}{大约}{3,6}{⼤、⽷}
  \begin{phonetics}{大约}{da4yue1}[][HSK 3]
    \definition{adv.}{aproximadamente; sobre | provavelmente}
  \end{phonetics}
\end{entry}

\begin{entry}{大自然}{3,6,12}{⼤、⾃、⽕}
  \begin{phonetics}{大自然}{da4 zi4 ran2}[][HSK 2]
    \definition{s.}{natureza}
  \end{phonetics}
\end{entry}

\begin{entry}{大衣}{3,6}{⼤、⾐}
  \begin{phonetics}{大衣}{da4 yi1}[][HSK 2]
    \definition{s.}{sobretudo}
  \end{phonetics}
\end{entry}

\begin{entry}{大声}{3,7}{⼤、⼠}
  \begin{phonetics}{大声}{da4 sheng1}[][HSK 2]
    \definition{adj.}{alto volume | em voz alta}
  \end{phonetics}
\end{entry}

\begin{entry}{大纲}{3,7}{⼤、⽷}
  \begin{phonetics}{大纲}{da4 gang1}[][HSK 5]
    \definition{s.}{esboço; compêndio; programa de estudos; resumo; fundamentos da organização sistemática de conteúdos (livros, discursos, programas, etc.)}
  \end{phonetics}
\end{entry}

\begin{entry}{大豆}{3,7}{⼤、⾖}
  \begin{phonetics}{大豆}{da4dou4}
    \definition{s.}{soja}
  \end{phonetics}
\end{entry}

\begin{entry}{大陆}{3,7}{⼤、⾩}
  \begin{phonetics}{大陆}{da4 lu4}[][HSK 4]
    \definition*{s.}{China continental; refere-se especificamente à vasta porção terrestre do território da China}
    \definition[个,块]{s.}{terra firme; continente; vasta extensão de terra}
  \end{phonetics}
\end{entry}

\begin{entry}{大事}{3,8}{⼤、⼅}
  \begin{phonetics}{大事}{da4 shi4}[][HSK 5]
    \definition[件,桩]{s.}{grande evento; grande acontecimento; assunto importante; grande questão; algo importante |
situação geral | em grande escala; em grande estilo; em grande parte}
  \end{phonetics}
\end{entry}

\begin{entry}{大使馆}{3,8,11}{⼤、⼈、⾷}
  \begin{phonetics}{大使馆}{da4shi3guan3}[][HSK 3]
    \definition[座,个]{s.}{embaixada}
  \end{phonetics}
\end{entry}

\begin{entry}{大姐}{3,8}{⼤、⼥}
  \begin{phonetics}{大姐}{da4 jie3}[][HSK 4]
    \definition[个]{s.}{irmã mais velha (também um termo educado para se dirigir a uma garota ou mulher um pouco mais velha do que a pessoa que fala)}
  \end{phonetics}
\end{entry}

\begin{entry}{大学}{3,8}{⼤、⼦}
  \begin{phonetics}{大学}{da4 xue2}[][HSK 1]
    \definition[所]{s.}{faculdade | universidade}
  \end{phonetics}
\end{entry}

\begin{entry}{大学生}{3,8,5}{⼤、⼦、⽣}
  \begin{phonetics}{大学生}{da4 xue2 sheng1}[][HSK 1]
    \definition{s.}{estudante universitário}
  \end{phonetics}
\end{entry}

\begin{entry}{大规模}{3,8,14}{⼤、⾒、⽊}
  \begin{phonetics}{大规模}{da4 gui1 mo2}[][HSK 4]
    \definition{adj.}{em larga escala; extensivo; maciço; massa}
    \definition{adj.}{em larga escala; extensivo; maciço; massivo}
  \end{phonetics}
\end{entry}

\begin{entry}{大雨}{3,8}{⼤、⾬}
  \begin{phonetics}{大雨}{da4yu3}
    \definition[场]{s.}{chuva pesada, forte}
  \end{phonetics}
\end{entry}

\begin{entry}{大前天}{3,9,4}{⼤、⼑、⼤}
  \begin{phonetics}{大前天}{da4qian2tian1}
    \definition{adv.}{três dias atrás}
  \end{phonetics}
\end{entry}

\begin{entry}{大型}{3,9}{⼤、⼟}
  \begin{phonetics}{大型}{da4xing2}[][HSK 4]
    \definition{adj.}{grande; em larga escala; tamanho e volume grandes | larga escala (importante e influente)}
  \end{phonetics}
\end{entry}

\begin{entry}{大奖赛}{3,9,14}{⼤、⼤、⾙}
  \begin{phonetics}{大奖赛}{da4 jiang3 sai4}[][HSK 5]
    \definition{s.}{grande competição; grande prêmio; \emph{grand prix}}
  \end{phonetics}
\end{entry}

\begin{entry}{大战}{3,9}{⼤、⼽}
  \begin{phonetics}{大战}{da4zhan4}
    \definition{s.}{guerra}
    \definition{v.}{guerrear | lutar em uma guerra}
  \end{phonetics}
\end{entry}

\begin{entry}{大洋洲}{3,9,9}{⼤、⽔、⽔}
  \begin{phonetics}{大洋洲}{da4yang2zhou1}
    \definition*{s.}{Oceania}
  \end{phonetics}
\end{entry}

\begin{entry}{大神}{3,9}{⼤、⽰}
  \begin{phonetics}{大神}{da4shen2}
    \definition{s.}{deidade | (gíria da Internet) guru | \emph{expert} | gênio}
  \end{phonetics}
\end{entry}

\begin{entry}{大胆}{3,9}{⼤、⾁}
  \begin{phonetics}{大胆}{da4 dan3}[][HSK 5]
    \definition{adj.}{ousado; atrevido; audacioso; corajoso; destemido}
  \end{phonetics}
\end{entry}

\begin{entry}{大哥}{3,10}{⼤、⼝}
  \begin{phonetics}{大哥}{da4 ge1}[][HSK 4]
    \definition{s.}{irmão mais velho | \emph{big brother} (endereço educado para um homem da mesma idade que você) | líder de gangue; pessoa mais poderosa em uma organização que realiza atividades ilegais na sociedade}
  \end{phonetics}
\end{entry}

\begin{entry}{大家}{3,10}{⼤、⼧}
  \begin{phonetics}{大家}{da4jia1}[][HSK 2]
    \definition{pron.}{todos}
  \end{phonetics}
\end{entry}

\begin{entry}{大海}{3,10}{⼤、⽔}
  \begin{phonetics}{大海}{da4 hai3}[][HSK 2]
    \definition{s.}{mar | oceano}
  \end{phonetics}
\end{entry}

\begin{entry}{大脑}{3,10}{⼤、⾁}
  \begin{phonetics}{大脑}{da4 nao3}[][HSK 5]
    \definition{s.}{cérebro; encéfalo}
  \end{phonetics}
\end{entry}

\begin{entry}{大致}{3,10}{⼤、⾄}
  \begin{phonetics}{大致}{da4zhi4}[][HSK 5]
    \definition{adj.}{geral; no todo}
    \definition{adv.}{grosso modo; aproximadamente; mais ou menos; indica uma estimativa aproximada da situação}
  \end{phonetics}
\end{entry}

\begin{entry}{大部分}{3,10,4}{⼤、⾢、⼑}
  \begin{phonetics}{大部分}{da4 bu4 fen4}[][HSK 2]
    \definition{s.}{a maioria | a maior parte}
  \end{phonetics}
\end{entry}

\begin{entry}{大都}{3,10}{⼤、⾢}
  \begin{phonetics}{大都}{da4 dou1}[][HSK 5]
  \end{phonetics}
  \begin{phonetics}{大都}{da4 du1}
    \definition{adv.}{em sua maior parte; na maior parte; indica que a maioria das pessoas ou coisas em um determinado intervalo tem a mesma natureza e características | também pronunciado como \dpy{da4dou1} na língua falada}
  \end{phonetics}
\end{entry}

\begin{entry}{大象}{3,11}{⼤、⾗}
  \begin{phonetics}{大象}{da4xiang4}[][HSK 5]
    \definition[只,头,群,个]{s.}{elefante}
  \end{phonetics}
\end{entry}

\begin{entry}{大猩猩}{3,12,12}{⼤、⽝、⽝}
  \begin{phonetics}{大猩猩}{da4xing1xing5}
    \definition{s.}{gorila}
  \end{phonetics}
\end{entry}

\begin{entry}{大量}{3,12}{⼤、⾥}
  \begin{phonetics}{大量}{da4 liang4}[][HSK 2]
    \definition{adj.}{numeroso | em massa | grande em número ou quantidade | generoso | magnânimo}
  \end{phonetics}
\end{entry}

\begin{entry}{大楼}{3,13}{⼤、⽊}
  \begin{phonetics}{大楼}{da4 lou2}[][HSK 4]
    \definition[座,幢]{s.}{edifício; mansão; edifício de vários andares disponível para uso residencial e comercial}
  \end{phonetics}
\end{entry}

\begin{entry}{大概}{3,13}{⼤、⽊}
  \begin{phonetics}{大概}{da4gai4}[][HSK 3]
    \definition{adj.}{geral; grosseiro; aproximado}
    \definition{adv.}{sobre; provavelmente
geralmente; brevemente}
    \definition{s.}{ideia geral; esboço geral}
  \end{phonetics}
\end{entry}

\begin{entry}{大腿}{3,13}{⼤、⾁}
  \begin{phonetics}{大腿}{da4tui3}
    \definition{s.}{coxa}
  \end{phonetics}
\end{entry}

\begin{entry}{大蒜}{3,13}{⼤、⾋}
  \begin{phonetics}{大蒜}{da4suan4}
    \definition[瓣,头]{s.}{alho}
  \end{phonetics}
\end{entry}

\begin{entry}{大熊猫}{3,14,11}{⼤、⽕、⽝}
  \begin{phonetics}{大熊猫}{da4 xiong2 mao1}[][HSK 5]
    \definition{s.}{panda gigante}
  \end{phonetics}
\end{entry}

\begin{entry}{大赛}{3,14}{⼤、⾙}
  \begin{phonetics}{大赛}{da4sai4}
    \definition{s.}{grande concurso, competição}
  \end{phonetics}
\end{entry}

\begin{entry}{天}{4}{⼤}
  \begin{phonetics}{天}{tian1}[][HSK 1]
    \definition{s.}{dia | céu | paraíso}
  \end{phonetics}
\end{entry}

\begin{entry}{天上}{4,3}{⼤、⼀}
  \begin{phonetics}{天上}{tian1 shang4}[][HSK 2]
    \definition{s.}{o céu | paraíso}
  \end{phonetics}
\end{entry}

\begin{entry}{天下}{4,3}{⼤、⼀}
  \begin{phonetics}{天下}{tian1xia4}
    \definition{s.}{terra sob o céu | o mundo todo | toda a China | reino}
  \end{phonetics}
\end{entry}

\begin{entry}{天才}{4,3}{⼤、⼿}
  \begin{phonetics}{天才}{tian1cai2}
    \definition{adj.}{talentoso | superdotado | genial}
    \definition{s.}{talento | dom | gênio}
  \end{phonetics}
\end{entry}

\begin{entry}{天公}{4,4}{⼤、⼋}
  \begin{phonetics}{天公}{tian1gong1}
    \definition{s.}{céu, paraíso | senhor do céu}
  \end{phonetics}
\end{entry}

\begin{entry}{天天}{4,4}{⼤、⼤}
  \begin{phonetics}{天天}{tian1tian1}
    \definition{adv.}{todo dia}
  \end{phonetics}
\end{entry}

\begin{entry}{天气}{4,4}{⼤、⽓}
  \begin{phonetics}{天气}{tian1qi4}[][HSK 1]
    \definition{s.}{clima, tempo}
  \end{phonetics}
\end{entry}

\begin{entry}{天花板}{4,7,8}{⼤、⾋、⽊}
  \begin{phonetics}{天花板}{tian1hua1ban3}
    \definition{s.}{teto}
  \end{phonetics}
\end{entry}

\begin{entry}{天使}{4,8}{⼤、⼈}
  \begin{phonetics}{天使}{tian1shi3}
    \definition{s.}{anjo}
  \end{phonetics}
\end{entry}

\begin{entry}{天择}{4,8}{⼤、⼿}
  \begin{phonetics}{天择}{tian1ze2}
    \definition{s.}{seleção natural}
  \end{phonetics}
\end{entry}

\begin{entry}{天空}{4,8}{⼤、⽳}
  \begin{phonetics}{天空}{tian1kong1}[][HSK 3]
    \definition{s.}{o céu; o firmamento}
  \end{phonetics}
\end{entry}

\begin{entry}{天柱}{4,9}{⼤、⽊}
  \begin{phonetics}{天柱}{tian1zhu4}
    \definition{s.}{pilar celestial, que sustenta o céu}
  \end{phonetics}
\end{entry}

\begin{entry}{天真}{4,10}{⼤、⼗}
  \begin{phonetics}{天真}{tian1zhen1}[][HSK 4]
    \definition{adj.}{ingênuo; inocente; ignorante; simples de coração, direto por natureza, livre de fingimento e hipocrisia}
  \end{phonetics}
\end{entry}

\begin{entry}{天堂}{4,11}{⼤、⼟}
  \begin{phonetics}{天堂}{tian1tang2}
    \definition{s.}{paraíso, céu}
  \end{phonetics}
\end{entry}

\begin{entry}{天然}{4,12}{⼤、⽕}
  \begin{phonetics}{天然}{tian1ran2}
    \definition{adj.}{natural}
  \end{phonetics}
\end{entry}

\begin{entry}{天鹅}{4,12}{⼤、⿃}
  \begin{phonetics}{天鹅}{tian1'e2}
    \definition{s.}{cisne}
  \end{phonetics}
\end{entry}

\begin{entry}{太}{4}{⼤}
  \begin{phonetics}{太}{tai4}[][HSK 1]
    \definition{adv.}{excessivamente | demais | muito}
  \end{phonetics}
\end{entry}

\begin{entry}{太太}{4,4}{⼤、⼤}
  \begin{phonetics}{太太}{tai4tai5}[][HSK 2]
    \definition[个,位]{s.}{esposa | madame| mulher casada}
  \end{phonetics}
\end{entry}

\begin{entry}{太平洋}{4,5,9}{⼤、⼲、⽔}
  \begin{phonetics}{太平洋}{tai4ping2 yang2}
    \definition*{s.}{Oceano Pacífico}
  \end{phonetics}
\end{entry}

\begin{entry}{太阳}{4,6}{⼤、⾩}
  \begin{phonetics}{太阳}{tai4yang5}[][HSK 2]
    \definition[个]{s.}{sol | abreviação de 太阳穴}
    \seeref{太阳穴}{tai4yang2xue2}
  \end{phonetics}
\end{entry}

\begin{entry}{太阳日}{4,6,4}{⼤、⾩、⽇}
  \begin{phonetics}{太阳日}{tai4yang2ri4}
    \definition{s.}{dia solar}
  \end{phonetics}
\end{entry}

\begin{entry}{太阳风}{4,6,4}{⼤、⾩、⾵}
  \begin{phonetics}{太阳风}{tai4yang2feng1}
    \definition{s.}{vento solar}
  \end{phonetics}
\end{entry}

\begin{entry}{太阳穴}{4,6,5}{⼤、⾩、⽳}
  \begin{phonetics}{太阳穴}{tai4yang2xue2}
    \definition{s.}{têmpora (nas laterais da cabeça humana)}
  \end{phonetics}
\end{entry}

\begin{entry}{太阳灯}{4,6,6}{⼤、⾩、⽕}
  \begin{phonetics}{太阳灯}{tai4yang2deng1}
    \definition{s.}{lâmpada solar (com células fotovoltaicas)}
  \end{phonetics}
\end{entry}

\begin{entry}{太阳雨}{4,6,8}{⼤、⾩、⾬}
  \begin{phonetics}{太阳雨}{tai4yang2yu3}
    \definition{s.}{banho de sol}
  \end{phonetics}
\end{entry}

\begin{entry}{太阳窗}{4,6,12}{⼤、⾩、⽳}
  \begin{phonetics}{太阳窗}{tai4yang2chuang1}
    \definition{s.}{teto solar (de veículos)}
  \end{phonetics}
\end{entry}

\begin{entry}{太阳镜}{4,6,16}{⼤、⾩、⾦}
  \begin{phonetics}{太阳镜}{tai4yang2jing4}
    \definition{s.}{óculos de sol}
  \end{phonetics}
\end{entry}

\begin{entry}{太阳翼}{4,6,17}{⼤、⾩、⽻}
  \begin{phonetics}{太阳翼}{tai4yang2yi4}
    \definition{s.}{painel solar}
  \end{phonetics}
\end{entry}

\begin{entry}{太极拳}{4,7,10}{⼤、⽊、⼿}
  \begin{phonetics}{太极拳}{tai4ji2quan2}
    \definition*{s.}{Tai Chi Chuan, Taiji, T'aichi ou T'aichichuan; forma tradicional de exercício físico ou relaxamento}
  \end{phonetics}
\end{entry}

\begin{entry}{太空}{4,8}{⼤、⽳}
  \begin{phonetics}{太空}{tai4kong1}
    \definition{s.}{espaço sideral | espaço exterior}
  \end{phonetics}
\end{entry}

\begin{entry}{夫人}{4,2}{⼤、⼈}
  \begin{phonetics}{夫人}{fu1ren2}[][HSK 4]
    \definition[位]{s.}{senhora; \emph{lady}; madame; na antiguidade, as esposas dos senhores feudais eram chamadas de ``madame'' e, nas dinastias Ming e Qing, as esposas dos oficiais de primeiro e segundo escalão eram chamadas de ``madame'', que mais tarde foi usada para homenagear as esposas das pessoas em geral e agora é usada principalmente em ocasiões diplomáticas}
  \end{phonetics}
\end{entry}

\begin{entry}{夫妇}{4,6}{⼤、⼥}
  \begin{phonetics}{夫妇}{fu1fu4}[][HSK 4]
    \definition[对]{s.}{casal; marido e mulher}
  \end{phonetics}
\end{entry}

\begin{entry}{夫妻}{4,8}{⼤、⼥}
  \begin{phonetics}{夫妻}{fu1qi1}[][HSK 4]
    \definition[对]{s.}{casal; marido e mulher}
  \end{phonetics}
\end{entry}

\begin{entry}{厺}{5}{⼤}
  \begin{phonetics}{厺}{qu4}
    \variantof{去}
  \end{phonetics}
\end{entry}

\begin{entry}{失业}{5,5}{⼤、⼀}
  \begin{phonetics}{失业}{shi1ye4}[][HSK 4]
    \definition{v.}{não ter emprego; estar desempregado; estar sem trabalho}
  \end{phonetics}
\end{entry}

\begin{entry}{失去}{5,5}{⼤、⼛}
  \begin{phonetics}{失去}{shi1qu4}[][HSK 3]
    \definition{v.}{perder}
  \end{phonetics}
\end{entry}

\begin{entry}{失败}{5,8}{⼤、⾒}
  \begin{phonetics}{失败}{shi1bai4}[][HSK 4]
    \definition{adj.}{insatisfatório; a maneira como as coisas aconteceram deixou muito a desejar; o resultado final deixou muito a desejar}
    \definition{v.}{perder; ser derrotado; não vencer em uma guerra ou competição | falhar; fracassar; não dar em nada; falhar em atingir um objetivo ou meta desejada (trabalho, carreira, etc.)}
  \end{phonetics}
\end{entry}

\begin{entry}{失眠}{5,10}{⼤、⽬}
  \begin{phonetics}{失眠}{shi1mian2}
    \definition{s.}{insônia}
    \definition{v.}{ter insônia}
  \end{phonetics}
\end{entry}

\begin{entry}{失望}{5,11}{⼤、⽉}
  \begin{phonetics}{失望}{shi1wang4}[][HSK 4]
    \definition{adj.}{desapontado; decepcionado}
    \definition{v.}{ficar desapontado; ficar decepcionado; estar desapontado;}
  \end{phonetics}
\end{entry}

\begin{entry}{失落}{5,12}{⼤、⾋}
  \begin{phonetics}{失落}{shi1luo4}
    \definition{s.}{frustração | decepção | perda}
    \definition{v.}{perder (algo) | cair (algo) | sentir uma sensação de perda}
  \end{phonetics}
\end{entry}

\begin{entry}{失意}{5,13}{⼤、⼼}
  \begin{phonetics}{失意}{shi1yi4}
    \definition{adj.}{desapontado | frustrado}
  \end{phonetics}
\end{entry}

\begin{entry}{头}{5}{⼤}
  \begin{phonetics}{头}{tou2}[][HSK 2,3]
    \definition{adj.}{(antes de um numeral) primeiro | (antes de ``年'' ou ``天'') último; anterior}
    \definition{clas.}{para suínos ou gado (animais de criação) | para alho}
    \definition{num.}{primeiro}
    \definition{prep.}{antes de; perto de | (entre dois algarismos, indicando um número aproximado) cerca de}
    \definition[个]{s.}{cabeça | cabelo ou penteado | topo; fim | começo ou fim | fim; remanescente |cabeça; chefe; líder |lado; aspecto}
  \seealsoref{年}{nian2}
  \seealsoref{天}{tian1}
  \end{phonetics}
  \begin{phonetics}{头}{tou5}
    \definition{suf.}{sufixo para nomes}
  \end{phonetics}
\end{entry}

\begin{entry}{头发}{5,5}{⼤、⼜}
  \begin{phonetics}{头发}{tou2fa5}[][HSK 2]
    \definition{s.}{cabelo}
  \end{phonetics}
\end{entry}

\begin{entry}{头号}{5,5}{⼤、⼝}
  \begin{phonetics}{头号}{tou2hao4}
    \definition{adj.}{primeira classe | número um | \emph{top rank}}
  \end{phonetics}
\end{entry}

\begin{entry}{头头}{5,5}{⼤、⼤}
  \begin{phonetics}{头头}{tou2tou2}
    \definition{s.}{chefe | o cabeça}
  \end{phonetics}
\end{entry}

\begin{entry}{头脑}{5,10}{⼤、⾁}
  \begin{phonetics}{头脑}{tou2 nao3}[][HSK 3]
    \definition{s.}{inteligência; mente | pista; tópicos principais | chefe; líder; capitão}
  \end{phonetics}
\end{entry}

\begin{entry}{头脑风暴}{5,10,4,15}{⼤、⾁、⾵、⽇}
  \begin{phonetics}{头脑风暴}{tou2nao3feng1bao4}
    \definition{s.}{\emph{brainstorm}}
  \end{phonetics}
\end{entry}

\begin{entry}{头像}{5,13}{⼤、⼈}
  \begin{phonetics}{头像}{tou2xiang4}
    \definition{s.}{retrato | busto | avatar | imagem de perfil (computação)}
  \end{phonetics}
\end{entry}

\begin{entry}{夺冠}{6,9}{⼤、⼍}
  \begin{phonetics}{夺冠}{duo2guan4}
    \definition{v.}{apoderar-se da coroa | (fig.) ganhar um campeonato | ganhar a medalha de ouro}
  \end{phonetics}
\end{entry}

\begin{entry}{奇怪}{8,8}{⼤、⼼}
  \begin{phonetics}{奇怪}{qi2guai4}[][HSK 3]
    \definition{adj.}{estranho; esquisito}
    \definition{v.}{ficar perplexo; maravilhar-se; sentir-se surpreso}
  \end{phonetics}
\end{entry}

\begin{entry}{奇迹}{8,9}{⼤、⾡}
  \begin{phonetics}{奇迹}{qi2ji4}
    \definition{adj.}{milagroso}
    \definition{s.}{milagre}
  \end{phonetics}
\end{entry}

\begin{entry}{奋斗}{8,4}{⼤、⽃}
  \begin{phonetics}{奋斗}{fen4dou4}[][HSK 4]
    \definition{v.}{lutar; esforçar-se; batalhar; trabalhar duro para atingir um determinado objetivo}
  \end{phonetics}
\end{entry}

\begin{entry}{奋战}{8,9}{⼤、⼽}
  \begin{phonetics}{奋战}{fen4zhan4}
    \definition{v.}{lutar bravamente | trabalhar duro}
  \end{phonetics}
\end{entry}

\begin{entry}{奏效}{9,10}{⼤、⽁}
  \begin{phonetics}{奏效}{zou4xiao4}
    \definition{v.}{mostrar resultados | ser eficaz}
  \end{phonetics}
\end{entry}

\begin{entry}{奖}{9}{⼤}
  \begin{phonetics}{奖}{jiang3}[][HSK 4]
    \definition[个,次]{s.}{prêmio; recompensa | elogio; loa}
    \definition{v.}{elogiar; recompensar; recomendar; incentivar}
  \end{phonetics}
\end{entry}

\begin{entry}{奖学金}{9,8,8}{⼤、⼦、⾦}
  \begin{phonetics}{奖学金}{jiang3 xue2 jin1}[][HSK 4]
    \definition[笔]{s.}{bolsa de estudos; exposição; prêmios concedidos por escolas, organizações ou indivíduos a alunos com bom desempenho acadêmico}
  \end{phonetics}
\end{entry}

\begin{entry}{奖金}{9,8}{⼤、⾦}
  \begin{phonetics}{奖金}{jiang3jin1}[][HSK 4]
    \definition[个,笔]{s.}{bônus; recompensa; prêmio; prêmio em dinheiro; dinheiro de recompensa, dinheiro dado às pessoas para incentivá-las ou elogiá-las por terem se saído bem em alguma coisa}
  \end{phonetics}
\end{entry}

\begin{entry}{套}{10}{⼤}
  \begin{phonetics}{套}{tao4}[][HSK 2]
    \definition{clas.}{para conjuntos, coleções}
    \definition{s.}{cobertura | fórmula | laço de corda}
    \definition{v.}{cobrir | envolver | intercalar | sobrepor}
  \end{phonetics}
\end{entry}

\begin{entry}{套问}{10,6}{⼤、⾨}
  \begin{phonetics}{套问}{tao4wen4}
    \definition{s.}{retórica}
    \definition{v.}{descobrir por meio de questionamento indireto diplomático}
  \end{phonetics}
\end{entry}

\begin{entry}{套餐}{10,16}{⼤、⾷}
  \begin{phonetics}{套餐}{tao4 can1}[][HSK 4]
    \definition{s.}{combo; pacote de produtos; pacote de serviços; metaforicamente, bens ou projetos que são combinados e levados ao mercado | refeição preparada; pacotes de refeições completos}
  \end{phonetics}
\end{entry}

\begin{entry}{奥}{12}{⼤}
  \begin{phonetics}{奥}{ao4}
    \definition{adj.}{obscuro | misterioso}
  \end{phonetics}
\end{entry}

\begin{entry}{奥运}{12,7}{⼤、⾡}
  \begin{phonetics}{奥运}{ao4yun4}
    \definition*{s.}{Jogos Olímpicos, Olimpíadas, abreviação de 奥林匹克运动会}
  \seealsoref{奥林匹克运动会}{ao4lin2pi3ke4 yun4dong4hui4}
  \end{phonetics}
\end{entry}

\begin{entry}{奥运会}{12,7,6}{⼤、⾡、⼈}
  \begin{phonetics}{奥运会}{ao4yun4hui4}
    \definition*{s.}{Jogos Olímpicos, Olimpíadas, abreviação de 奥林匹克运动会}
  \seealsoref{奥林匹克运动会}{ao4lin2pi3ke4 yun4dong4hui4}
  \end{phonetics}
\end{entry}

\begin{entry}{奥林匹克运动会}{12,8,4,7,7,6,6}{⼤、⽊、⼖、⼗、⾡、⼒、⼈}
  \begin{phonetics}{奥林匹克运动会}{ao4lin2pi3ke4 yun4dong4hui4}
    \definition*{s.}{Jogos Olímpicos, Olimpíadas}
  \end{phonetics}
\end{entry}

\begin{entry}{奥特曼}{12,10,11}{⼤、⽜、⽈}
  \begin{phonetics}{奥特曼}{ao4te4man4}
    \definition*{s.}{\emph{Ultraman},  super-herói de ficção científica japonesa}
  \end{phonetics}
\end{entry}

%%%%% EOF %%%%%

