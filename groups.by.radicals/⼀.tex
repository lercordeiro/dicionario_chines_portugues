%%%
%%% Radical "⼀"
%%%
\section*{Radical 1: ``⼀''}\addcontentsline{toc}{section}{Radical 1: ⼀}\addcontentsline{loh}{figure}{\#\#\#\# 1: ⼀}

%%%%%%%%%% 一 %%%%%%%%%%
\subsection*{一}\addcontentsline{loh}{figure}{一}

\begin{Entry}{一}{1}{⼀}[Kangxi 1]
  \begin{Phonetics}{一}{yi1}[(\dpy{yao1})][HSK 1]
    \definition{adv.}{uma vez; assim que; indica que duas ações ocorreram em um intervalo de tempo muito curto, uma terminando e a outra começando imediatamente em seguida | indica que primeiro se realiza uma ação e, em seguida, o resultado dessa ação  | indica uma ação única, indicando que a ação é muito curta ou apenas uma tentativa}
    \definition{num.}{um; 1 | pronunciado como \dpy{yao1} quando dito número a número | igual; refere"-se ao mesmo ou igual | inteiro; todo; por toda parte | exclusivo ou único | refere"-se a algo específico | também; caso contrário; referindo"-se a outro ou mais um}
    \definition{part.}{antes de certas palavras para dar ênfase}
    \definition{prep.}{cada; por; toda vez}
    \definition{s.}{uma nota da escala em Gongchepu (工尺谱), correspondente ao 17 na notação musical numerada}
  \seealsoref{工尺谱}{gong1 che3 pu3}
  \antonymref{多次}{duo1ci4}
  \end{Phonetics}
  \begin{Phonetics}{一}{yi2}[(一 $+$ 4º tom)]
    \definition{num.}{um; 1 | um (artigo)}
  \end{Phonetics}
  \begin{Phonetics}{一}{yi4}[(一 $+$ 1º, 2º e 3º tom)]
    \definition{adv.}{uma vez | assim que | ao}
    \definition{num.}{um; 1 | um (artigo)}
  \end{Phonetics}
\end{Entry}

\begin{Entry}{一下}{1,3}{⼀,⼀}
  \begin{Phonetics}{一下}{yi2xia4}
    \definition{adv.}{em um curto tempo | rapidamente}
  \synonymref{统计}{tong3ji4}
  \end{Phonetics}
\end{Entry}

\begin{Entry}{一下儿}{1,3,2}{⼀,⼀,⼉}
  \begin{Phonetics}{一下儿}{yi2xia4r5}[][HSK 1,5]
    \definition{s.}{um tempo; um momento}
  \end{Phonetics}
\end{Entry}

\begin{Entry}{一下子}{1,3,3}{⼀,⼀,⼦}
  \begin{Phonetics}{一下子}{yi2xia4zi5}[][HSK 5]
    \definition{adv.}{tudo de uma vez; de repente; em pouco tempo; em um curto espaço de tempo}
  \synonymref{一会儿}{yi2hui4r5}
  \end{Phonetics}
\end{Entry}

\begin{Entry}{一个样}{1,3,10}{⼀,⼈,⽊}
  \begin{Phonetics}{一个样}{yi2ge5yang4}
    \definition{s.}{o mesmo}
  \seealsoref{一样}{yi2yang4}
  \end{Phonetics}
\end{Entry}

\begin{Entry}{一口气}{1,3,4}{⼀,⼝,⽓}
  \begin{Phonetics}{一口气}{yi4kou3qi4}[][HSK 5]
    \definition{adv.}{em um só fôlego; sem pausa; fazer algo continuamente}
  \synonymref{连续}{lian2xu4}
  \end{Phonetics}
\end{Entry}

\begin{Entry}{一切}{1,4}{⼀,⼑}
  \begin{Phonetics}{一切}{yi2qie4}[][HSK 3]
    \definition{pron.}{tudo; todo; todas as coisas}
  \synonymref{一共}{yi2gong4}
  \end{Phonetics}
\end{Entry}

\begin{Entry}{一方面}{1,4,9}{⼀,⽅,⾯}
  \begin{Phonetics}{一方面}{yi4fang1mian4}[][HSK 3]
    \definition{s.}{um lado; um dos dois aspectos opostos ou um lado de algo que está relacionado a outro}
  \seealsoref{一方面……,一方面……}{yi4fang1mian4 yi4fang1mian4}
  \synonymref{一部分}{yi2bu4fen5}
  \end{Phonetics}
\end{Entry}

\begin{Entry}{一方面……,一方面……}{1,4,9,1,4,9}{⼀,⽅,⾯,⼀,⽅,⾯}
  \begin{Phonetics}{一方面……,一方面……}{yi4fang1mian4 yi4fang1mian4}
    \definition{conj.}{por um lado\dots, por outro lado\dots; conecta duas orações paralelas (devem ser usadas juntas)}[\underline{一方面}觉得兴奋,\underline{一方面}又害怕。===Por um lado, sinto"-me entusiasmado, mas, por outro, também sinto medo.]
  \end{Phonetics}
\end{Entry}

\begin{Entry}{一代}{1,5}{⼀,⼈}
  \begin{Phonetics}{一代}{yi2dai4}[][HSK 6]
    \definition{s.}{uma dinastia | era; época atual | vida; geração; toda a vida de uma pessoa}
  \end{Phonetics}
\end{Entry}

\begin{Entry}{一半}{1,5}{⼀,⼗}
  \begin{Phonetics}{一半}{yi2ban4}[][HSK 1]
    \definition{num.}{metade; em parte; uma metade}
  \antonymref{全部}{quan2bu4}
  \end{Phonetics}
\end{Entry}

\begin{Entry}{一句话}{1,5,8}{⼀,⼝,⾔}
  \begin{Phonetics}{一句话}{yi2ju4hua4}[][HSK 5]
    \definition{s.}{em resumo; em uma palavra; expressar um conteúdo complexo de forma sucinta | trabalho fácil; fácil de fazer; descrever uma tarefa ou trabalho como muito simples e fácil de realizar}
  \end{Phonetics}
\end{Entry}

\begin{Entry}{一旦}{1,5}{⼀,⽇}
  \begin{Phonetics}{一旦}{yi2dan4}[][HSK 5]
    \definition{adv.}{uma vez; no caso; agora que | de repente; uma vez}
    \definition{s.}{em um único dia; em um tempo muito curto;}
  \synonymref{万一}{wan4yi1}
  \end{Phonetics}
\end{Entry}

\begin{Entry}{一生}{1,5}{⼀,⽣}
  \begin{Phonetics}{一生}{yi4sheng1}[][HSK 2]
    \definition{s.}{vida inteira; toda a vida; ao longo da vida; todo o tempo desde o nascimento até a morte; às vezes exagerado para indicar um longo período de tempo no curso da vida}
  \synonymref{一辈子}{yi2bei4zi5}
  \synonymref{终身}{zhong1shen1}
  \antonymref{瞬间}{shun4jian1}
  \end{Phonetics}
\end{Entry}

\begin{Entry}{一边}{1,5}{⼀,⾡}
  \begin{Phonetics}{一边}{yi4bian1}[][HSK 1]
    \definition{adj.}{igual; idêntico; da mesma forma}
    \definition{adv.}{enquanto; ao mesmo tempo; simultaneamente; indica que uma ação ocorre simultaneamente a outra ação}
    \definition{s.}{lado; um lado; um aspecto | ambos os lados; ao lado de}
  \synonymref{侧面}{ce4mian4}
  \antonymref{四处}{si4chu4}
  \end{Phonetics}
\end{Entry}

\begin{Entry}{一会儿}{1,6,2}{⼀,⼈,⼉}
  \begin{Phonetics}{一会儿}{yi2hui4r5}[][HSK 1,2]
    \definition{adv.}{agora\dots agora\dots; um momento\dots o próximo\dots; usado antes de dois antônimos, indica a alternância de situações}
    \definition{s.}{um pouquinho de tempo; muito pouco tempo}
  \synonymref{不一会儿}{bu4 yi2hui4r5}
  \synonymref{紧接着}{jin3 jie1zhe5}
  \synonymref{一下子}{yi2xia4zi5}
  \end{Phonetics}
\end{Entry}

\begin{Entry}{一会儿……一会儿……}{1,6,2,1,6,2}{⼀,⼈,⼉,⼀,⼈,⼉}
  \begin{Phonetics}{一会儿……一会儿……}{yi1hui4r5 yi1hui4r5}
    \definition{adv.}{um tempo\dots um tempo\dots}
  \end{Phonetics}
\end{Entry}

\begin{Entry}{一共}{1,6}{⼀,⼋}
  \begin{Phonetics}{一共}{yi2gong4}[][HSK 2]
    \definition{adv.}{completamente; em tudo; no todo}
  \synonymref{全部}{quan2bu4}
  \synonymref{全面}{quan2mian4}
  \synonymref{全体}{quan2ti3}
  \synonymref{所有}{suo3you3}
  \synonymref{统统}{tong3tong3}
  \synonymref{一切}{yi2qie4}
  \synonymref{整个}{zheng3ge4}
  \synonymref{总共}{zong3gong4}
  \end{Phonetics}
\end{Entry}

\begin{Entry}{一再}{1,6}{⼀,⼌}
  \begin{Phonetics}{一再}{yi2zai4}[][HSK 4]
    \definition{adv.}{repetidamente; de novo e de novo; repetidas vezes; uma e outra vez}
  \synonymref{常常}{chang2chang2}
  \synonymref{反复}{fan3fu4}
  \synonymref{屡次}{lv3ci4}
  \synonymref{频繁}{pin2fan2}
  \synonymref{频频}{pin2pin2}
  \synonymref{再三}{zai4san1}
  \antonymref{不再}{bu2zai4}
  \end{Phonetics}
\end{Entry}

\begin{Entry}{一同}{1,6}{⼀,⼝}
  \begin{Phonetics}{一同}{yi4tong2}[][HSK 6]
    \definition{adv.}{juntos; ao mesmo tempo e lugar}
  \synonymref{一块}{yi2kuai4}
  \synonymref{一路}{yi2lu4}
  \synonymref{一齐}{yi4qi2}
  \synonymref{一起}{yi4qi3}
  \antonymref{分开}{fen1/kai1}
  \antonymref{混合}{hun4he2}
  \end{Phonetics}
\end{Entry}

\begin{Entry}{一向}{1,6}{⼀,⼝}
  \begin{Phonetics}{一向}{yi2xiang4}[][HSK 5]
    \definition{adv.}{desde o início; indica do passado até o presente}
  \synonymref{不断}{bu2duan4}
  \synonymref{从来}{cong2lai2}
  \synonymref{历来}{li4lai2}
  \synonymref{一贯}{yi2guan4}
  \synonymref{一直}{yi4zhi2}
  \antonymref{偶尔}{ou3'er3}
  \antonymref{有时}{you3shi2}
  \end{Phonetics}
\end{Entry}

\begin{Entry}{一次性}{1,6,8}{⼀,⽋,⼼}
  \begin{Phonetics}{一次性}{yi2ci4xing4}[][HSK 6]
    \definition{adj.}{único; uso único; descartável (produtos); apenas uma vez, sem necessidade ou necessidade de fazer novamente}
  \end{Phonetics}
\end{Entry}

\begin{Entry}{一行}{1,6}{⼀,⾏}
  \begin{Phonetics}{一行}{yi4xing2}[][HSK 6]
    \definition{s.}{delegação; um grupo viajando junto; festa}
  \end{Phonetics}
\end{Entry}

\begin{Entry}{一齐}{1,6}{⼀,⿑}
  \begin{Phonetics}{一齐}{yi4qi2}[][HSK 6]
    \definition{adv.}{juntos; em uníssono; simultaneamente; ao mesmo tempo; indica que diferentes sujeitos emitem simultaneamente o mesmo comportamento ou o mesmo sujeito emite vários comportamentos diferentes ao mesmo tempo}
  \synonymref{全部}{quan2bu4}
  \synonymref{所有}{suo3you3}
  \synonymref{一块}{yi2kuai4}
  \synonymref{一路}{yi2lu4}
  \synonymref{一切}{yi2qie4}
  \synonymref{一起}{yi4qi3}
  \synonymref{一同}{yi4tong2}
  \antonymref{分散}{fen1san4}
  \antonymref{先后}{xian1hou4}
  \end{Phonetics}
\end{Entry}

\begin{Entry}{一块}{1,7}{⼀,⼟}
  \begin{Phonetics}{一块}{yi2kuai4}
    \definition{adv.}{(principalmente mandarim) juntos}
  \synonymref{一路}{yi2lu4}
  \synonymref{一齐}{yi4qi2}
  \synonymref{一起}{yi4qi3}
  \synonymref{一同}{yi4tong2}
  \antonymref{各自}{ge4zi4}
  \end{Phonetics}
\end{Entry}

\begin{Entry}{一块儿}{1,7,2}{⼀,⼟,⼉}
  \begin{Phonetics}{一块儿}{yi2kuai4r5}[][HSK 1]
    \definition{adv.}{juntos; em conjunto}
    \definition{s.}{no mesmo lugar; no mesmo local}
  \synonymref{一起}{yi4qi3}
  \end{Phonetics}
\end{Entry}

\begin{Entry}{一时}{1,7}{⼀,⽇}
  \begin{Phonetics}{一时}{yi4shi2}[][HSK 6]
    \definition{adv.}{por um curto período; temporário | (usado em pares) agora\dots, agora\dots; este momento\dots, e o próximo\dots; o mesmo que 时而}
    \definition{s.}{um período de tempo | um momento; um breve momento; um tempo muito curto}
  \seealsoref{时而}{shi2'er2}
  \seealsoref{一时……,一时……}{yi4shi2 yi4shi2}
  \synonymref{间或}{jian4huo4}
  \synonymref{临时}{lin2shi2}
  \synonymref{偶尔}{ou3'er3}
  \synonymref{有时}{you3shi2}
  \synonymref{暂时}{zan4shi2}
  \antonymref{常常}{chang2chang2}
  \antonymref{持久}{chi2jiu3}
  \antonymref{时常}{shi2chang2}
  \end{Phonetics}
\end{Entry}

\begin{Entry}{一时……,一时……}{1,7,1,7}{⼀,⽇,⼀,⽇}
  \begin{Phonetics}{一时……,一时……}{yi4shi2 yi4shi2}
    \definition{adv.}{por um tempo\dots, por um tempo\dots}
  \seealsoref{一时}{yi4shi2}
  \end{Phonetics}
\end{Entry}

\begin{Entry}{一身}{1,7}{⼀,⾝}
  \begin{Phonetics}{一身}{yi4shen1}[][HSK 5]
    \definition{s.}{o corpo inteiro; em todo o corpo | um terno; (um conjunto completo de) roupas | sozinho; uma única pessoa; relativo a uma única pessoa}
  \end{Phonetics}
\end{Entry}

\begin{Entry}{一些}{1,8}{⼀,⼆}
  \begin{Phonetics}{一些}{yi4xie1}[][HSK 1]
    \definition{clas.}{alguns; um número de; quantidade indeterminada | um pouco; uma pequena quantidade | mais de um; mais de uma vez; indica mais de um ou mais de uma vez, etc. | uma ligeira mudança no grau, intensidade; usado após certos verbos, adjetivos, etc., para indicar uma quantidade muito pequena}
    \definition{pron.}{uns; alguns}
  \synonymref{一般}{yi4ban1}
  \synonymref{有些}{you3xie1}
  \end{Phonetics}
\end{Entry}

\begin{Entry}{一定}{1,8}{⼀,⼧}
  \begin{Phonetics}{一定}{yi2ding4}[][HSK 2]
    \definition{adj.}{certo; particular; tendo um certo nível de especificidade; (objeto, situação) determinado em um ou mais | devido; certo; sempre foi assim, não vai mudar | fixo; especificado; há requisitos claros quanto à maneira, método, quantidade, etc.}
    \definition{adv.}{certamente; necessariamente; expressando determinação ou certeza | certamente; indica especulação ou avaliação de que um evento ou situação definitivamente acontecerá ou realmente existirá}
  \synonymref{必定}{bi4ding4}
  \synonymref{必然}{bi4ran2}
  \synonymref{必需}{bi4xu1}
  \synonymref{肯定}{ken3ding4}
  \synonymref{确定}{que4ding4}
  \synonymref{特定}{te4ding4}
  \synonymref{指定}{zhi3ding4}
  \antonymref{不必}{bu2bi4}
  \antonymref{不定}{bu2ding4}
  \antonymref{大概}{da4gai4}
  \antonymref{大致}{da4zhi4}
  \antonymref{或许}{huo4xu3}
  \antonymref{可能}{ke3neng2}
  \antonymref{莫非}{mo4fei1}
  \antonymref{倘若}{tang3ruo4}
  \antonymref{未必}{wei4bi4}
  \antonymref{也许}{ye3xu3}
  \end{Phonetics}
\end{Entry}

\begin{Entry}{一直}{1,8}{⼀,⽬}
  \begin{Phonetics}{一直}{yi4zhi2}[][HSK 2]
    \definition{adv.}{direto; indica que permanece inalterado em uma direção | sempre; continuamente; o tempo todo; o tempo todo; indica que a ação é sempre ininterrupta ou o estado é sempre inalterado | de um ponto a outro sem enfatizar nenhuma exceção}
  \synonymref{不断}{bu2duan4}
  \synonymref{不停}{bu4ting2}
  \synonymref{继续}{ji4xu4}
  \synonymref{连续}{lian2xu4}
  \synonymref{始终}{shi3zhong1}
  \synonymref{一贯}{yi2guan4}
  \synonymref{一向}{yi2xiang4}
  \antonymref{间断}{jian4duan4}
  \antonymref{中断}{zhong1duan4}
  \end{Phonetics}
\end{Entry}

\begin{Entry}{一贯}{1,8}{⼀,⾙}
  \begin{Phonetics}{一贯}{yi2guan4}[][HSK 6]
    \definition{adj./adv.}{do começo ao fim; inabalável; consistente; persistente; o tempo todo}
  \synonymref{不断}{bu2duan4}
  \synonymref{从来}{cong2lai2}
  \synonymref{通常}{tong1chang2}
  \synonymref{一向}{yi2xiang4}
  \synonymref{一直}{yi4zhi2}
  \antonymref{偶尔}{ou3'er3}
  \antonymref{偶然}{ou3ran2}
  \end{Phonetics}
\end{Entry}

\begin{Entry}{一带}{1,9}{⼀,⼱}
  \begin{Phonetics}{一带}{yi2dai4}[][HSK 5]
    \definition{s.}{a área em torno de um determinado local; refere"-se a um determinado local e suas proximidades}
  \end{Phonetics}
\end{Entry}

\begin{Entry}{一律}{1,9}{⼀,⼻}
  \begin{Phonetics}{一律}{yi2lv4}[][HSK 4]
    \definition{adj.}{igual; semelhante; uniforme; parecido; idêntico}
    \definition{adv.}{todos; tudo; sem exceção; enfatiza que todos devem ser assim, sem exceção, e é usado principalmente em regulamentos ou requisitos}
  \synonymref{绝对}{jue2dui4}
  \synonymref{同等}{tong2deng3}
  \synonymref{完全}{wan2quan2}
  \synonymref{一样}{yi2yang4}
  \synonymref{一致}{yi2zhi4}
  \synonymref{整齐}{zheng3qi2}
  \antonymref{不同}{bu4tong2}
  \antonymref{例外}{li4wai4}
  \end{Phonetics}
\end{Entry}

\begin{Entry}{一战}{1,9}{⼀,⼽}
  \begin{Phonetics}{一战}{yi2zhan4}
    \definition*{s.}{Primeira Guerra Mundial}
  \end{Phonetics}
\end{Entry}

\begin{Entry}{一点儿}{1,9,2}{⼀,⽕,⼉}
  \begin{Phonetics}{一点儿}{yi4dian3r5}[][HSK 1]
    \definition{adv.}{um pouco; uma pitada; uma gota; uma amostra; uma pequena quantidade; ({adj.} + (一)点儿, 一点儿 + {s.} ou 有 + (一)点儿 + {s.})}
  \synonymref{一点点}{yi4dian3dian3}
  \end{Phonetics}
\end{Entry}

\begin{Entry}{一点点}{1,9,9}{⼀,⽕,⽕}
  \begin{Phonetics}{一点点}{yi4dian3dian3}[][HSK 2]
    \definition{adj.}{um pouco; muito pouco ou um pouquinho}
  \end{Phonetics}
\end{Entry}

\begin{Entry}{一样}{1,10}{⼀,⽊}
  \begin{Phonetics}{一样}{yi2yang4}[][HSK 1]
    \definition{adj.}{o mesmo; tão\dots quanto\dots; igualmente; semelhante}
    \definition{part.}{na mesma medida; anexado a verbos ou palavras nominais, indica uma comparação ou semelhança, equivalente a 似的}
  \seealsoref{似的}{shi4de5}
  \synonymref{雷同}{lei2tong2}
  \synonymref{通常}{tong1chang2}
  \synonymref{同样}{tong2yang4}
  \synonymref{相似}{xiang1si4}
  \synonymref{相同}{xiang1tong2}
  \synonymref{一律}{yi2lv4}
  \antonymref{不同}{bu4tong2}
  \end{Phonetics}
\end{Entry}

\begin{Entry}{一流}{1,10}{⼀,⽔}
  \begin{Phonetics}{一流}{yi4liu2}[][HSK 5]
    \definition{adj.}{clássico; de primeira linha; de primeira classe; o melhor}
    \definition[些]{s.}{tipo; mesmo tipo; da mesma classe; da mesma categoria; uma categoria}
  \synonymref{顶尖}{ding3jian1}
  \synonymref{优秀}{you1xiu4}
  \end{Phonetics}
\end{Entry}

\begin{Entry}{一致}{1,10}{⼀,⾄}
  \begin{Phonetics}{一致}{yi2zhi4}[][HSK 4]
    \definition{adj.}{equado; idêntico; uniforme; unânime; nenhuma diferença (de opinião ou ação)}
    \definition{adv.}{juntos; em conjunto}
  \synonymref{一律}{yi2lv4}
  \antonymref{分歧}{fen1qi2}
  \antonymref{相反}{xiang1fan3}
  \end{Phonetics}
\end{Entry}

\begin{Entry}{一般}{1,10}{⼀,⾈}
  \begin{Phonetics}{一般}{yi4ban1}[][HSK 2]
    \definition{adj.}{o mesmo que; exatamente como | geral; ordinário; comum | médio; medíocre; o grau ou nível não é muito alto}
    \definition{adv.}{frequentemente; geralmente}
  \synonymref{凡是}{fan2shi4}
  \synonymref{平常}{ping2chang2}
  \synonymref{普遍}{pu3bian4}
  \synonymref{普通}{pu3tong1}
  \synonymref{日常}{ri4chang2}
  \synonymref{通常}{tong1chang2}
  \synonymref{一些}{yi4xie1}
  \antonymref{出色}{chu1se4}
  \antonymref{非常}{fei1chang2}
  \antonymref{格外}{ge2wai4}
  \antonymref{个别}{ge4bie2}
  \antonymref{特别}{te4bie2}
  \antonymref{特殊}{te4shu1}
  \antonymref{突出}{tu1/chu1}
  \antonymref{无比}{wu2bi3}
  \antonymref{尤其}{you2qi2}
  \end{Phonetics}
\end{Entry}

\begin{Entry}{一般来说}{1,10,7,9}{⼀,⾈,⽊,⾔}
  \begin{Phonetics}{一般来说}{yi4ban1lai2shuo1}[][HSK 4]
    \definition{expr.}{de modo geral; na média; no caso usual; a declaração usual}
  \end{Phonetics}
\end{Entry}

\begin{Entry}{一起}{1,10}{⼀,⾛}
  \begin{Phonetics}{一起}{yi4qi3}[][HSK 1]
    \definition{adv.}{juntos; em companhia; indica o mesmo local, ao mesmo tempo que se faz algo | no total; em todos; no conjunto}
    \definition{s.}{no mesmo lugar}
  \synonymref{全部}{quan2bu4}
  \synonymref{所有}{suo3you3}
  \synonymref{同步}{tong2bu4}
  \synonymref{一道}{yi2dao4}
  \synonymref{一块}{yi2kuai4}
  \synonymref{一路}{yi2lu4}
  \synonymref{一切}{yi2qie4}
  \synonymref{一齐}{yi4qi2}
  \synonymref{一同}{yi4tong2}
  \antonymref{独自}{du2zi4}
  \end{Phonetics}
\end{Entry}

\begin{Entry}{一部分}{1,10,4}{⼀,⾢,⼑}
  \begin{Phonetics}{一部分}{yi2bu4fen5}[][HSK 2]
    \definition{adj.}{parcial}
    \definition{adv.}{parcialmente}
    \definition{num.}{parte; porção; seção; fração}
  \synonymref{个别}{ge4bie2}
  \synonymref{局部}{ju2bu4}
  \synonymref{局限}{ju2xian4}
  \synonymref{一方面}{yi4fang1mian4}
  \end{Phonetics}
\end{Entry}

\begin{Entry}{一……就……}{1,12}{⼀,⼪}
  \begin{Phonetics}{一……就……}{yi1 jiu4}
    \definition{expr.}{logo que |  uma vez que}
  \end{Phonetics}
\end{Entry}

\begin{Entry}{一番}{1,12}{⼀,⽥}
  \begin{Phonetics}{一番}{yi4 fan1}[][HSK 6]
    \definition{adv.}{uma demonstração de, uma dose de, um pedaço de (conversa, investigação, pensamento)}
  \antonymref{多次}{duo1ci4}
  \end{Phonetics}
\end{Entry}

\begin{Entry}{一辈子}{1,12,3}{⼀,⾞,⼦}
  \begin{Phonetics}{一辈子}{yi2bei4zi5}[][HSK 5]
    \definition{s.}{uma vida inteira; vida inteira; toda a vida; durante toda a vida; enquanto se vive; todo o tempo entre o nascimento e a morte}
  \synonymref{一生}{yi4sheng1}
  \end{Phonetics}
\end{Entry}

\begin{Entry}{一道}{1,12}{⼀,⾡}
  \begin{Phonetics}{一道}{yi2dao4}[][HSK 6]
    \definition{adv.}{juntos; lado a lado; junto com}
  \synonymref{一起}{yi4qi3}
  \end{Phonetics}
\end{Entry}

\begin{Entry}{一路}{1,13}{⼀,⾜}
  \begin{Phonetics}{一路}{yi2lu4}[][HSK 5]
    \definition{adv.}{o tempo todo; persistentemente; continuamente | juntos; sem parar; continuamente}
    \definition{s.}{o mesmo caminho; a mesma rota; ao longo de toda a viagem, ao longo do caminho | do mesmo tipo; da mesma categoria}
  \end{Phonetics}
\end{Entry}

\begin{Entry}{一路上}{1,13,3}{⼀,⾜,⼀}
  \begin{Phonetics}{一路上}{yi2lu4shang4}[][HSK 6]
    \definition{s.}{ao longo do caminho; todo o caminho}
  \end{Phonetics}
\end{Entry}

\begin{Entry}{一路平安}{1,13,5,6}{⼀,⾜,⼲,⼧}
  \begin{Phonetics}{一路平安}{yi2lu4-ping2'an1}[][HSK 2]
    \definition{expr.}{Boa viagem!; Tenha uma boa viagem!}
    \definition{v.}{ter uma viagem agradável}
  \synonymref{一路顺风}{yi2lu4shun4feng1}
  \end{Phonetics}
\end{Entry}

\begin{Entry}{一路顺风}{1,13,9,4}{⼀,⾜,⾴,⾵}
  \begin{Phonetics}{一路顺风}{yi2lu4shun4feng1}[][HSK 2]
    \definition{expr.}{ter uma viagem agradável; toda a viagem foi segura e tranquila; é uma metáfora para cada etapa do processo de lidar com algo que ocorre sem problemas | Tenha uma boa viagem!; Boa viagem!}
  \synonymref{一路平安}{yi2lu4-ping2'an1}
  \end{Phonetics}
\end{Entry}

\begin{Entry}{一模一样}{1,14,1,10}{⼀,⽊,⼀,⽊}
  \begin{Phonetics}{一模一样}{yi4mu2-yi2yang4}[][HSK 6]
    \definition{expr.}{tão parecidos quanto duas ervilhas; ser exatamente iguais; muito parecido, a mesma aparência}
  \synonymref{大同小异}{da4tong2-xiao3yi4}
  \antonymref{截然不同}{jie2ran2-bu4tong2}
  \end{Phonetics}
\end{Entry}

%%%%%%%%%% 丁 %%%%%%%%%%
\subsection*{丁}\addcontentsline{loh}{figure}{丁}

\begin{Entry}{丁}{2}{⼀}
  \begin{Phonetics}{丁}{ding1}[][HSK 7-9]
    \definition*{s.}{O quarto dos Dez Troncos Celestiais | Sobrenome: Ding}
    \definition{s.}{homem; homem adulto | população; membro de uma família | uma pessoa envolvida em uma determinada ocupação; pessoas em certas profissões | cubos; pequenos cubos de carne ou vegetais}
    \definition{v.}{encontrar; encontrar"-se com; esbarrar em}
  \end{Phonetics}
  \begin{Phonetics}{丁}{zheng1}
    \definition{s.}{Onomatopéia: som agudo e metálico (como o de cortar madeira, jogar xadrez ou tocar instrumentos musicais)}
  \end{Phonetics}
\end{Entry}

%%%%%%%%%% 七 %%%%%%%%%%
\subsection*{七}\addcontentsline{loh}{figure}{七}

\begin{Entry}{七}{2}{⼀}
  \begin{Phonetics}{七}{qi1}[][HSK 1]
    \definition*{s.}{Sobrenome: Qi}
    \definition{num.}{sete; 7}
    \definition{s.}{antigamente, os mortos eram homenageados a cada sete dias, chamados de 七, até o quadragésimo nono dia, num total de sete 七}
  \end{Phonetics}
\end{Entry}

\begin{Entry}{七夕}{2,3}{⼀,⼣}
  \begin{Phonetics}{七夕}{qi1xi1}
    \definition*{s.}{Dia dos Namorados Chinês, quando o vaqueiro e a tecelã (牛郎织女) têm permissão para se reunirem anualmente | Festival das Meninas | Festival Duplo Sete, noite do sétimo mês lunar}
  \seealsoref{牛郎织女}{niu2 lang2 zhi1nv3}
  \end{Phonetics}
\end{Entry}

\begin{Entry}{七嘴八舌}{2,16,2,6}{⼀,⼝,⼋,⾆}
  \begin{Phonetics}{七嘴八舌}{qi1zui3-ba1she2}[][HSK 7-9]
    \definition{expr.}{``Uma cacofonia de vozes.''; todos falando ao mesmo tempo; falando uns por cima dos outros; isso descreve uma situação em que muitas pessoas estão falando ao mesmo tempo, com opiniões conflitantes; também descreve alguém que é falante e fofoqueiro}
  \synonymref{沸沸扬扬}{fei4fei4yang2yang2}
  \end{Phonetics}
\end{Entry}

%%%%%%%%%% 万 %%%%%%%%%%
\subsection*{万}\addcontentsline{loh}{figure}{万}

\begin{Entry}{万}{3}{⼀}
  \begin{Phonetics}{万}{wan4}[][HSK 2]
    \definition*{s.}{Sobrenome: Wan}
    \definition{adv.}{absolutamente; indica um grau extremamente alto, equivalente a 完全, 绝对 e 极}
    \definition{num.}{dez mil; 10.000; 1.0000 | miríade; um número muito grande}
  \seealsoref{极}{ji2}
  \seealsoref{绝对}{jue2dui4}
  \seealsoref{完全}{wan2quan2}
  \end{Phonetics}
\end{Entry}

\begin{Entry}{万一}{3,1}{⼀,⼀}
  \begin{Phonetics}{万一}{wan4yi1}[][HSK 4]
    \definition{conj.}{por via das dúvidas; se por acaso; só por precaução; expressa uma suposição muito improvável (usado para coisas desagradáveis)}
    \definition{num.}{um décimo milionésimo; uma porcentagem muito pequena}
    \definition{s.}{contingência; eventualidade; contingências muito improváveis}
  \synonymref{如果}{ru2guo3}
  \synonymref{要是}{yao4shi5}
  \synonymref{一旦}{yi2dan4}
  \end{Phonetics}
\end{Entry}

\begin{Entry}{万万}{3,3}{⼀,⼀}
  \begin{Phonetics}{万万}{wan4wan4}[][HSK 7-9]
    \definition{adv.}{totalmente; absolutamente; em qualquer caso; não importa o que aconteça}
    \definition{num.}{cem milhões; 100.000.000; 1.0000.0000}
  \synonymref{绝对}{jue2dui4}
  \synonymref{千万}{qian1wan4}
  \synonymref{完全}{wan2quan2}
  \end{Phonetics}
\end{Entry}

\begin{Entry}{万分}{3,4}{⼀,⼑}
  \begin{Phonetics}{万分}{wan4fen1}[][HSK 7-9]
    \definition{adv.}{extremamente; muito}
  \synonymref{非常}{fei1chang2}
  \synonymref{极度}{ji2du4}
  \synonymref{极端}{ji2duan1}
  \synonymref{十分}{shi2fen1}
  \synonymref{特别}{te4bie2}
  \synonymref{异常}{yi4chang2}
  \end{Phonetics}
\end{Entry}

\begin{Entry}{万无一失}{3,4,1,5}{⼀,⽆,⼀,⼤}
  \begin{Phonetics}{万无一失}{wan4wu2-yi1shi1}[][HSK 7-9]
    \definition{expr.}{não há perigo de algo dar errado; esteja do lado seguro; \dots não pode falhar em nenhuma circunstância; garantir sucesso total; nenhum risco; não há chance de erro; (faça com que seja mais do que provável que) nada dê errado; perfeitamente seguro; infalível; as chances são de mil para uma, não falharemos}
  \end{Phonetics}
\end{Entry}

\begin{Entry}{万古长青}{3,5,4,8}{⼀,⼝,⾧,⾭}
  \begin{Phonetics}{万古长青}{wan4gu3-chang2qing1}[][HSK 7-9]
    \definition{expr.}{seja perene; seja sempre verde; perene e eterno; sempre vivo; florescer para sempre; durar para sempre; permanecer fresco para sempre; sempre será verde como pinheiros e ciprestes por milhares de gerações; uma metáfora para um espírito nobre ou uma amizade profunda que nunca desaparecerá}
  \end{Phonetics}
\end{Entry}

\begin{Entry}{万圣节}{3,5,5}{⼀,⼟,⾋}
  \begin{Phonetics}{万圣节}{wan4 sheng4 jie2}
    \definition*{s.}{Dia de Todos os Santos}
  \seealsoref{万圣节前夕}{wan4sheng4 jie2 qian2xi1}
  \end{Phonetics}
\end{Entry}

\begin{Entry}{万圣节前夕}{3,5,5,9,3}{⼀,⼟,⾋,⼑,⼣}
  \begin{Phonetics}{万圣节前夕}{wan4sheng4 jie2 qian2xi1}
    \definition*{s.}{Véspera do Dia de Todos os Santos | Halloween}
  \seealsoref{万圣节}{wan4 sheng4 jie2}
  \end{Phonetics}
\end{Entry}

\begin{Entry}{万物}{3,8}{⼀,⽜}
  \begin{Phonetics}{万物}{wan4wu4}
    \definition{s.}{toda a criação; todas as coisas na Terra; todos os seres vivos; tudo no universo}
  \synonymref{生物}{sheng1wu4}
  \end{Phonetics}
\end{Entry}

\begin{Entry}{万能}{3,10}{⼀,⾁}
  \begin{Phonetics}{万能}{wan4neng2}[][HSK 7-9]
    \definition{adj.}{onipotente; todo"-poderoso | universal; multifacetado; de múltiplos usos}
  \synonymref{全能}{quan2neng2}
  \end{Phonetics}
\end{Entry}

\begin{Entry}{万福}{3,13}{⼀,⽰}
  \begin{Phonetics}{万福}{wan4fu2}
    \definition{s.}{(antigo) reverência feminina; reverência}
  \end{Phonetics}
\end{Entry}

%%%%%%%%%% 丈 %%%%%%%%%%
\subsection*{丈}\addcontentsline{loh}{figure}{丈}

\begin{Entry}{丈}{3}{⼀}
  \begin{Phonetics}{丈}{zhang4}
    \definition{clas.}{zhang, uma unidade tradicional de comprimento, igual a 10 市尺 e equivalente a 3,333 metros ou 3,65 jardas}
    \definition{s.}{zhang, uma unidade de comprimento (= 3,333\dots metros)}
    \definition{s.}{sênior; ancião | marido (em certos termos de parentesco) | tratamento respeitoso ao idoso na China antiga; um título respeitoso para homens idosos nos tempos antigos | uma forma de tratamento para certos parentes do sexo masculino por casamento}
  \seealsoref{市尺}{shi4 chi3}
  \end{Phonetics}
\end{Entry}

\begin{Entry}{丈夫}{3,4}{⼀,⼤}
  \begin{Phonetics}{丈夫}{zhang4fu5}[][HSK 4]
    \definition[个,位,名]{s.}{marido; esposo}
  \end{Phonetics}
\end{Entry}

%%%%%%%%%% 三 %%%%%%%%%%
\subsection*{三}\addcontentsline{loh}{figure}{三}

\begin{Entry}{三}{3}{⼀}
  \begin{Phonetics}{三}{san1}[][HSK 1]
    \definition*{s.}{Sobrenome: San}
    \definition{num.}{três; 3 | muitos; vários; mais de dois; referindo"-se a muitos ou à maioria | alguns; poucos; menos; não muitos}
  \end{Phonetics}
\end{Entry}

\begin{Entry}{三角}{3,7}{⼀,⾓}
  \begin{Phonetics}{三角}{san1jiao3}[][HSK 7-9]
    \definition{adj.}{tripartido; que constitui uma relação tripartite}
    \definition[个,些]{s.}{triângulo; coisas triangulares | trigonometria, abreviação de 三角学}
  \seealsoref{三角学}{san1jiao3 xue2}
  \end{Phonetics}
\end{Entry}

\begin{Entry}{三角学}{3,7,8}{⼀,⾓,⼦}
  \begin{Phonetics}{三角学}{san1jiao3 xue2}
    \definition{s.}{trigonometria; um ramo da matemática que estuda principalmente as funções trigonométricas e suas propriedades, bem como suas aplicações em geometria}[我三角学学得很好。===Sou muito bom em trigonometria.]
  \end{Phonetics}
\end{Entry}

\begin{Entry}{三角恋爱}{3,7,10,10}{⼀,⾓,⼼,⽖}
  \begin{Phonetics}{三角恋爱}{san1jiao3lian4'ai4}
    \definition[场]{s.}{triângulo amoroso | triângulo eterno}
  \end{Phonetics}
\end{Entry}

\begin{Entry}{三明治}{3,8,8}{⼀,⽇,⽔}
  \begin{Phonetics}{三明治}{san1ming2zhi4}[][HSK 6]
    \definition[个,些,块]{s.}{Empréstimo linguístico: sanduíche, \emph{sandwich}}
  \end{Phonetics}
\end{Entry}

\begin{Entry}{三轮车}{3,8,4}{⼀,⾞,⾞}
  \begin{Phonetics}{三轮车}{san1lun2che1}
    \definition{s.}{triciclo}
  \end{Phonetics}
\end{Entry}

\begin{Entry}{三维}{3,11}{⼀,⽷}
  \begin{Phonetics}{三维}{san1wei2}[][HSK 7-9]
    \definition{s.}{três dimensões; 3D; tridimensional}[我们生活在三维空间。===Vivemos em um espaço tridimensional.]
  \end{Phonetics}
\end{Entry}

\begin{Entry}{三番五次}{3,12,4,6}{⼀,⽥,⼆,⽋}
  \begin{Phonetics}{三番五次}{san1fan1-wu3ci4}[][HSK 7-9]
    \definition{expr.}{repetidamente; de novo e de novo; várias e várias vezes; diversas vezes}
  \synonymref{翻来覆去}{fan1lai2-fu4qu4}
  \synonymref{接二连三}{jie1'er4-lian2san1}
  \end{Phonetics}
\end{Entry}

%%%%%%%%%% 上 %%%%%%%%%%
\subsection*{上}\addcontentsline{loh}{figure}{上}

\begin{Entry}{上}{3}{⼀}
  \begin{Phonetics}{上}{shang3}
    \definition{s.}{tom descendente-ascendente; significa o segundo tom dos quatro tons do mandarim, e também se refere ao terceiro tom do mandarim padrão}
  \end{Phonetics}
  \begin{Phonetics}{上}{shang4}[][HSK 1]
    \definition{adj.}{mais recente; último; anterior; tempo ou a sequência anterior | superior; mais alto; melhor; indica uma posição elevada em termos de qualidade, nível, etc. | lugar elevado; posição superior}
    \definition{s.}{superior; acima; para cima; um lugar alto ou mais alto do que um determinado local | na superfície de um objeto; usado após um substantivo, indica a superfície de um objeto | indica estar dentro do escopo de algo; usado após um substantivo, indica que algo está dentro do âmbito de determinada coisa | indica um aspecto específico | antigamente, referia"-se ao imperador | usado após palavras que indicam idade, equivale a ``\dots 的时候'' | o primeiro nível da escala da música folclórica chinesa, usado como um símbolo de nota na notação musical, equivalente ao ``1'' na notação simplificada.}
    \definition{v.}{subir; montar; embarcar; entrar | ir para; partir para | estar ocupado (com trabalho, estudos, etc.) em um horário fixo; começar a trabalhar ou estudar na hora marcada, etc. | seguir em frente; prosseguir | encher; abastecer; servir; melhorar; aumentar | aparecer no palco; entrar | colocar algo em posição; ajustar; fixar; montar as duas partes de algo | aplicar; pintar; espalhar | ser registrado; ser publicado (em uma publicação) | atingir; ser suficiente (uma determinada quantidade ou grau) | submeter; enviar; apresentar; submeter à aprovação superior | ventilar; apertar; torcer | trazer; servir; colocar comida, pratos, chá e outras coisas na mesa para os convidados | indicar que uma ação tem um resultado | pesquisar na \emph{Internet} | emaranhar"-se; ficar emaranhado; enredar"-se}
    \definition{v.aux.}{usado após um verbo para indicar início e continuidade}
  \seealsoref{的时候}{de5 shi2hou4}
  \antonymref{下}{xia4}
  \end{Phonetics}
\end{Entry}

\begin{Entry}{上下}{3,3}{⼀,⼀}
  \begin{Phonetics}{上下}{shang4xia4}[][HSK 5]
    \definition{adv.}{para cima e para baixo}
    \definition[顶]{s.}{alto e baixo | de cima para baixo; para cima e para baixo | superioridade ou inferioridade relativa | (após números redondos) aproximadamente; mais ou menos; por aí | velhos e jovens; hierarquia em termos de cargo e posição social}
    \definition{v.}{subir ou descer | subir e descer; da alta para a baixa ou da baixa para a alta}
  \synonymref{高低}{gao1di1}
  \antonymref{左右}{zuo3you4}
  \end{Phonetics}
\end{Entry}

\begin{Entry}{上个月}{3,3,4}{⼀,⼈,⽉}
  \begin{Phonetics}{上个月}{shang4ge4yue4}[][HSK 4]
    \definition{s.}{mês passado; refere"-se à hora de um mês atrás, ou seja, o mês passado mais próximo da hora atual}
  \end{Phonetics}
\end{Entry}

\begin{Entry}{上门}{3,3}{⼀,⾨}
  \begin{Phonetics}{上门}{shang4 men2}[][HSK 4]
    \definition{v.}{chamar; visitar; aparecer; ir ou vir para ver alguém; ir até a porta; ir até a casa de alguém | trancar a porta; fechar a porta durante a noite | casar"-se e morar com a família da noiva}
  \synonymref{拜访}{bai4fang3}
  \end{Phonetics}
\end{Entry}

\begin{Entry}{上升}{3,4}{⼀,⼗}
  \begin{Phonetics}{上升}{shang4sheng1}[][HSK 3]
    \definition{v.}{elevar; subir; mover"-se para cima; mover de baixo para cima; aumentar em nível, grau, quantidade, etc.}
  \synonymref{上涨}{shang4zhang3}
  \antonymref{回升}{hui2sheng1}
  \antonymref{降落}{jiang4luo4}
  \antonymref{落下}{luo4xia4}
  \antonymref{下降}{xia4jiang4}
  \end{Phonetics}
\end{Entry}

\begin{Entry}{上午}{3,4}{⼀,⼗}
  \begin{Phonetics}{上午}{shang4wu3}[][HSK 1]
    \definition[个]{s.}{manhã; \emph{ante meridiem} (a.m.); geralmente refere"-se ao período entre a manhã e o meio"-dia}
  \antonymref{下午}{xia4wu3}
  \end{Phonetics}
\end{Entry}

\begin{Entry}{上方}{3,4}{⼀,⽅}
  \begin{Phonetics}{上方}{shang4fang1}[][HSK 7-9]
    \definition{s.}{acima; sobre; em cima de | superjacente}
  \seealsoref{下方}{xia4fang1}
  \end{Phonetics}
\end{Entry}

\begin{Entry}{上火}{3,4}{⼀,⽕}
  \begin{Phonetics}{上火}{shang4/huo3}[][HSK 7-9]
    \definition{v.+compl.}{ter dor de garganta | ter excesso de calor interno; na medicina tradicional chinesa, sintomas como prisão de ventre ou inflamação da mucosa nasal, da mucosa oral ou da conjuntiva são classificados como ``calor interno'' | ficar com raiva}
  \seealsoref{上火儿}{shang4huo3r5}
  \end{Phonetics}
\end{Entry}

\begin{Entry}{上火儿}{3,4,2}{⼀,⽕,⼉}
  \begin{Phonetics}{上火儿}{shang4huo3r5}
    \definition{v.}{Dialeto: ficar com raiva; explodir}
  \end{Phonetics}
\end{Entry}

\begin{Entry}{上车}{3,4}{⼀,⾞}
  \begin{Phonetics}{上车}{shang4che1}[][HSK 1]
    \definition{v.}{entrar; subir (em um ônibus, trem, carro etc.)}
  \antonymref{下车}{xia4che1}
  \end{Phonetics}
\end{Entry}

\begin{Entry}{上去}{3,5}{⼀,⼛}
  \begin{Phonetics}{上去}{shang4 qu5}[][HSK 3]
    \definition{v.}{subir (a partir da minha localização) | ascender a um lugar (ou estado) considerado mais elevado (ou acima); usado depois de um verbo para indicar movimento, de baixo para cima ou de perto para longe}
  \synonymref{上来}{shang4 lai5}
  \antonymref{下来}{xia4 lai5}
  \end{Phonetics}
\end{Entry}

\begin{Entry}{上古}{3,5}{⼀,⼝}
  \begin{Phonetics}{上古}{shang4gu3}
    \definition{s.}{tempos antigos; eras remotas | antiguidade | tempos históricos antigos | o passado distante}
  \antonymref{现代}{xian4dai4}
  \end{Phonetics}
\end{Entry}

\begin{Entry}{上台}{3,5}{⼀,⼝}
  \begin{Phonetics}{上台}{shang4 tai2}[][HSK 6]
    \definition{v.}{aparecer no palco; subir na plataforma; ir para o palco ou pódio | assumir o poder; chegar (subir) ao poder; começar a assumir papéis de liderança ou a ganhar algum tipo de poder}
  \synonymref{上任}{shang4/ren4}
  \end{Phonetics}
\end{Entry}

\begin{Entry}{上司}{3,5}{⼀,⼝}
  \begin{Phonetics}{上司}{shang4si5}[][HSK 7-9]
    \definition[位,名,个]{s.}{chefe; superior}
  \synonymref{上级}{shang4ji2}
  \antonymref{部属}{bu4shu3}
  \antonymref{部下}{bu4xia4}
  \end{Phonetics}
\end{Entry}

\begin{Entry}{上头}{3,5}{⼀,⼤}
  \begin{Phonetics}{上头}{shang4tou2}
    \definition{v.}{(álcool, amor etc.) subir à cabeça; (uma ideia, uma música etc.) entrar na cabeça de alguém; capturar a atenção de alguém | Obsoleto: (uma garota no dia do seu casamento) começar a usar o cabelo preso em um coque (em vez de uma trança)}
  \synonymref{方面}{fang1mian4}
  \synonymref{上面}{shang4mian5}
  \end{Phonetics}
  \begin{Phonetics}{上头}{shang4tou5}[][HSK 7-9]
    \definition{s.}{acima; em cima de; na superfície de; superior}
  \end{Phonetics}
\end{Entry}

\begin{Entry}{上市}{3,5}{⼀,⼱}
  \begin{Phonetics}{上市}{shang4 shi4}[][HSK 6]
    \definition{v.}{listar; abrir o capital; ser listado (na bolsa de valores) | estar na estação; estar (aparecer) no mercado | ir ao mercado (para fazer compras)}
  \synonymref{问世}{wen4shi4}
  \end{Phonetics}
\end{Entry}

\begin{Entry}{上边}{3,5}{⼀,⾡}
  \begin{Phonetics}{上边}{shang4bian5}[][HSK 1]
    \definition{s.}{topo; acima; sobre; superior}
  \synonymref{上方}{shang4fang1}
  \antonymref{下边}{xia4bian5}
  \end{Phonetics}
\end{Entry}

\begin{Entry}{上任}{3,6}{⼀,⼈}
  \begin{Phonetics}{上任}{shang4/ren4}[][HSK 7-9]
    \definition[班]{s.}{predecessor; ex"-funcionário}
    \definition{v.+compl.}{assumir o cargo; ocupar um cargo oficial; refere"-se à posse de autoridades}
  \synonymref{出任}{chu1ren4}
  \synonymref{就职}{jiu4/zhi2}
  \synonymref{就任}{jiu4ren4}
  \synonymref{上台}{shang4 tai2}
  \antonymref{辞职}{ci2/zhi2}
  \antonymref{离职}{li2/zhi2}
  \end{Phonetics}
\end{Entry}

\begin{Entry}{上场}{3,6}{⼀,⼟}
  \begin{Phonetics}{上场}{shang4/chang3}[][HSK 7-9]
    \definition{v.+compl.}{Esporte: entrar na quadra (ou campo); participar de uma competição | aparecer no palco; subir no palco; entrar em cena}
  \antonymref{退场}{tui4chang3}
  \end{Phonetics}
\end{Entry}

\begin{Entry}{上当}{3,6}{⼀,⼹}
  \begin{Phonetics}{上当}{shang4/dang4}[][HSK 6]
    \definition{v.+compl.}{ser enganado; ser ludibriado; morder a isca; cair nas mãos de alguém}
  \synonymref{受骗}{shou4/pian4}
  \antonymref{精明}{jing1ming2}
  \antonymref{警惕}{jing3ti4}
  \end{Phonetics}
\end{Entry}

\begin{Entry}{上旬}{3,6}{⼀,⽇}
  \begin{Phonetics}{上旬}{shang4xun2}[][HSK 7-9]
    \definition{s.}{primeiro terço do mês; os dez dias do dia 1 ao dia 10 de cada mês}
  \seealsoref{中旬}{zhong1xun2}
  \antonymref{下旬}{xia4xun2}
  \end{Phonetics}
\end{Entry}

\begin{Entry}{上次}{3,6}{⼀,⽋}
  \begin{Phonetics}{上次}{shang4ci4}[][HSK 1]
    \definition{adv.}{última vez}
  \end{Phonetics}
\end{Entry}

\begin{Entry}{上级}{3,6}{⼀,⽷}
  \begin{Phonetics}{上级}{shang4ji2}[][HSK 5]
    \definition[个,位]{s.}{nível superior; organização ou pessoa em nível superior; organizações ou pessoas de nível superior dentro do mesmo sistema organizacional}
  \synonymref{部属}{bu4shu3}
  \synonymref{上司}{shang4si5}
  \antonymref{部下}{bu4xia4}
  \end{Phonetics}
\end{Entry}

\begin{Entry}{上网}{3,6}{⼀,⽹}
  \begin{Phonetics}{上网}{shang4/wang3}[][HSK 1]
    \definition{v.+compl.}{conectar"-se à \emph{Internet}; acessar a \emph{Internet}; entrar na \emph{Internet}; acessar a rede; refere"-se especificamente ao computador do usuário conectado à \emph{Internet} para pesquisar e consultar informações, etc.}
  \end{Phonetics}
\end{Entry}

\begin{Entry}{上衣}{3,6}{⼀,⾐}
  \begin{Phonetics}{上衣}{shang4yi1}[][HSK 3]
    \definition[件]{s.}{jaqueta; roupas para a parte superior do corpo}
  \end{Phonetics}
\end{Entry}

\begin{Entry}{上访}{3,6}{⼀,⾔}
  \begin{Phonetics}{上访}{shang4fang3}
    \definition{v.}{buscar uma audiência com superiores (especialmente funcionários do governo) para fazer uma petição por algo}
  \end{Phonetics}
\end{Entry}

\begin{Entry}{上声}{3,7}{⼀,⼠}
  \begin{Phonetics}{上声}{shang3sheng1}
    \definition{s.}{tom descendente e ascendente | terceiro tom no mandarim moderno}
  \end{Phonetics}
\end{Entry}

\begin{Entry}{上岗}{3,7}{⼀,⼭}
  \begin{Phonetics}{上岗}{shang4/gang3}[][HSK 7-9]
    \definition{v.+compl.}{estar em período probatório | começar a trabalhar; assumir um cargo}
  \end{Phonetics}
\end{Entry}

\begin{Entry}{上报}{3,7}{⼀,⼿}
  \begin{Phonetics}{上报}{shang4bao4}[][HSK 7-9]
    \definition{v.}{aparecer nos jornais; ser publicado | reportar a um órgão superior; reportar à liderança; reportar"-se aos superiores}
  \synonymref{报到}{bao4/dao4}
  \end{Phonetics}
\end{Entry}

\begin{Entry}{上来}{3,7}{⼀,⽊}
  \begin{Phonetics}{上来}{shang4 lai5}[][HSK 3]
    \definition{v.}{subir (para a minha localização) | estar no começo; começar; iniciar | surgir; de um lugar baixo para um lugar alto (o interlocutor está em um lugar alto) | usado após o verbo, indica que algo foi concluído com sucesso}
  \synonymref{上去}{shang4 qu5}
  \antonymref{下去}{xia4 qu5}
  \end{Phonetics}
\end{Entry}

\begin{Entry}{上诉}{3,7}{⼀,⾔}
  \begin{Phonetics}{上诉}{shang4su4}[][HSK 7-9]
    \definition{s.}{apelação (para um tribunal superior)}
    \definition{v.}{apresentar um recurso; instaurar um recurso}
  \end{Phonetics}
\end{Entry}

\begin{Entry}{上周}{3,8}{⼀,⼝}
  \begin{Phonetics}{上周}{shang4zhou1}[][HSK 2]
    \definition{s.}{semana passada}
  \antonymref{下周}{xia4zhou1}
  \end{Phonetics}
\end{Entry}

\begin{Entry}{上坡路}{3,8,13}{⼀,⼟,⾜}
  \begin{Phonetics}{上坡路}{shang4po1lu4}
    \definition{s.}{aclive | progresso | (fig.) tendência ascendente}
  \end{Phonetics}
\end{Entry}

\begin{Entry}{上学}{3,8}{⼀,⼦}
  \begin{Phonetics}{上学}{shang4 xue2}[][HSK 1]
    \definition{v.}{ir à escola; frequentar a escola; estar na escola; ir à escola para estudar | começar a escola; começar a estudar no ensino fundamental}
  \synonymref{入学}{ru4/xue2}
  \synonymref{上课}{shang4/ke4}
  \antonymref{放学}{fang4/xue2}
  \end{Phonetics}
\end{Entry}

\begin{Entry}{上空}{3,8}{⼀,⽳}
  \begin{Phonetics}{上空}{shang4kong1}[][HSK 7-9]
    \definition{s.}{no céu; acima da cabeça; no alto, no ar}
  \end{Phonetics}
\end{Entry}

\begin{Entry}{上述}{3,8}{⼀,⾡}
  \begin{Phonetics}{上述}{shang4shu4}[][HSK 7-9]
    \definition{adj.}{mencionado anteriormente; supracitado; acima citado; conforme dito ou narrado acima}
  \synonymref{描述}{miao2shu4}
  \synonymref{摸索}{mo1suo3}
  \synonymref{试验}{shi4yan4}
  \end{Phonetics}
\end{Entry}

\begin{Entry}{上限}{3,8}{⼀,⾩}
  \begin{Phonetics}{上限}{shang4xian4}[][HSK 7-9]
    \definition[个]{s.}{teto; limite superior; refere"-se ao primeiro ou mais alto limite dentro de um determinado conjunto de limites}
  \antonymref{下限}{xia4xian4}
  \end{Phonetics}
\end{Entry}

\begin{Entry}{上帝}{3,9}{⼀,⼱}
  \begin{Phonetics}{上帝}{shang4di4}[][HSK 6]
    \definition*{s.}{Deus; O Deus Supremo no Cristianismo | O Imperador do Céu; um deus na antiga crença chinesa que pode controlar tudo no mundo}
    \definition[个]{s.}{(figurado) cliente; metáfora para consumidores}
  \end{Phonetics}
\end{Entry}

\begin{Entry}{上映}{3,9}{⼀,⽇}
  \begin{Phonetics}{上映}{shang4ying4}[][HSK 7-9]
    \definition{v.}{executar; exibir; mostrar (um filme); (novo filme) lançar para exibição}
  \synonymref{放映}{fang4ying4}
  \end{Phonetics}
\end{Entry}

\begin{Entry}{上面}{3,9}{⼀,⾯}
  \begin{Phonetics}{上面}{shang4mian5}[][HSK 3]
    \definition{s.}{uma posição mais alta que algo; uma posição acima/acima de algo | superfície do objeto | aspecto | a parte acima mencionada; a parte que vem primeiro na ordem; a parte de um artigo ou discurso que vem antes da presente | autoridades superiores | os mais velhos; a geração mais velha da família}
  \synonymref{方面}{fang1mian4}
  \synonymref{上方}{shang4fang1}
  \synonymref{上头}{shang4tou5}
  \antonymref{底下}{di3xia5}
  \antonymref{下面}{xia4mian5}
  \end{Phonetics}
\end{Entry}

\begin{Entry}{上流}{3,10}{⼀,⽔}
  \begin{Phonetics}{上流}{shang4liu2}[][HSK 7-9]
    \definition{adj.}{da classe alta; refinado | pertencente aos círculos superiores; anteriormente, referia"-se a pessoas de alto \emph{status} social}
    \definition{s.}{trecho superior (de um rio); a montante}
  \synonymref{崇高}{chong2gao1}
  \synonymref{高超}{gao1chao1}
  \synonymref{高贵}{gao1gui4}
  \synonymref{高尚}{gao1shang4}
  \end{Phonetics}
\end{Entry}

\begin{Entry}{上海}{3,10}{⼀,⽔}
  \begin{Phonetics}{上海}{shang4hai3}
    \definition*{s.}{Município de Xangai (Shanghai), centro-leste da China}
  \end{Phonetics}
\end{Entry}

\begin{Entry}{上涨}{3,10}{⼀,⽔}
  \begin{Phonetics}{上涨}{shang4zhang3}[][HSK 5]
    \definition{v.}{subir; ir para cima; ascender}
  \synonymref{高潮}{gao1chao2}
  \synonymref{高涨}{gao1zhang3}
  \synonymref{上升}{shang4sheng1}
  \antonymref{回落}{hui2luo4}
  \antonymref{下降}{xia4jiang4}
  \end{Phonetics}
\end{Entry}

\begin{Entry}{上班}{3,10}{⼀,⽟}
  \begin{Phonetics}{上班}{shang4/ban1}[][HSK 1]
    \definition{v.+compl.}{ir trabalhar; começar a trabalhar; estar de plantão; ir trabalhar no local de trabalho regular no horário especificado}
  \synonymref{加班}{jia1/ban1}
  \antonymref{下班}{xia4/ban1}
  \end{Phonetics}
\end{Entry}

\begin{Entry}{上班族}{3,10,11}{⼀,⽟,⽅}
  \begin{Phonetics}{上班族}{shang4 ban1 zu2}
    \definition[本]{s.}{trabalhadores de escritório (como grupo social)}
  \end{Phonetics}
\end{Entry}

\begin{Entry}{上课}{3,10}{⼀,⾔}
  \begin{Phonetics}{上课}{shang4/ke4}[][HSK 1]
    \definition{v.+compl.}{frequentar aulas; ir às aulas; dar uma aula}
  \synonymref{上学}{shang4 xue2}
  \antonymref{下课}{xia4/ke4}
  \end{Phonetics}
\end{Entry}

\begin{Entry}{上调}{3,10}{⼀,⾔}
  \begin{Phonetics}{上调}{shang4tiao2}[][HSK 7-9]
    \definition{v.}{transferir (alguém) para um cargo de nível superior | transferir bens, fundos, etc. para uma unidade de nível superior | ajustar para cima | aumentar (os preços)}
  \antonymref{下调}{xia4tiao2}
  \end{Phonetics}
\end{Entry}

\begin{Entry}{上期}{3,12}{⼀,⽉}
  \begin{Phonetics}{上期}{shang4 qi1}[][HSK 7-9]
    \definition{s.}{período anterior}
  \end{Phonetics}
\end{Entry}

\begin{Entry}{上游}{3,12}{⼀,⽔}
  \begin{Phonetics}{上游}{shang4you2}[][HSK 7-9]
    \definition{s.}{a montante; trecho superior de um rio; o trecho de um rio próximo à sua nascente; também se refere à área por onde esse trecho flui | posição avançada; metaforicamente, refere"-se a um \emph{status} ou nível avançado}
  \synonymref{领先}{ling3/xian1}
  \synonymref{先进}{xian1jin4}
  \antonymref{下游}{xia4you2}
  \end{Phonetics}
\end{Entry}

\begin{Entry}{上楼}{3,13}{⼀,⽊}
  \begin{Phonetics}{上楼}{shang4lou2}[][HSK 4]
    \definition{v.}{subir as escadas; ir para o andar de cima}
  \end{Phonetics}
\end{Entry}

\begin{Entry}{上演}{3,14}{⼀,⽔}
  \begin{Phonetics}{上演}{shang4yan3}[][HSK 6]
    \definition{s.}{exibição | encenação}
    \definition{v.}{exibir (um filme); encenar (uma peça); atuar; colocar no palco}
  \synonymref{演出}{yan3chu1}
  \end{Phonetics}
\end{Entry}

\begin{Entry}{上瘾}{3,16}{⼀,⽧}
  \begin{Phonetics}{上瘾}{shang4/yin3}[][HSK 7-9]
    \definition{v.+compl.}{ser viciado (em algo); adquirir o hábito (de fazer algo); gostar muito de algo, a ponto de não conseguir viver sem}
  \synonymref{沉迷}{chen2mi2}
  \synonymref{痴迷}{chi1mi2}
  \synonymref{陶醉}{tao2zui4}
  \end{Phonetics}
\end{Entry}

%%%%%%%%%% 下 %%%%%%%%%%
\subsection*{下}\addcontentsline{loh}{figure}{下}

\begin{Entry}{下}{3}{⼀}
  \begin{Phonetics}{下}{xia4}[][HSK 1,2]
    \definition{clas.}{número de vezes usado para a ação | volume de um contêiner; quantidade de objetos que cabem em um utensílio | usado depois de 两 e 几 para expressar habilidade, capacidade, destreza}
    \definition{s.}{abaixo | próximo; último; segundo; referindo"-se ao que está por vir ou ao que vem em seguida | mais baixo; inferior; de baixo nível ou grau | próximo; último; segundo; em ordem ou em ordem cronológica | indica pertencer a uma determinada faixa, situação, condição, etc. | indica uma determinada época ou estação | usado após um número para indicar posição ou direção | para baixo (após uma preposição) | sob (depois de um substantivo) | para baixo (antes de um verbo)}
    \definition{v.}{desembarcar; descer; sair | cair (chuva, neve, etc.) | enviar; emitir; entregar | ir para | sair; partir; retirar"-se | lançar; colocar | descarregar; desmontar; tirar (fora) | formar (uma opinião, ideia, etc.); tomar decisões, fazer julgamentos, etc. | usar; aplicar | dar à luz (animais) | tomar; capturar; conquistar | ceder | terminar; deixar de lado; terminar o trabalho ou os estudos diários na hora prevista | para negação; ser inferior a; ser menor que}
  \seealsoref{几}{ji3}
  \seealsoref{两}{liang3}
  \antonymref{高}{gao1}
  \antonymref{上}{shang4}
  \end{Phonetics}
\end{Entry}

\begin{Entry}{下个月}{3,3,4}{⼀,⼈,⽉}
  \begin{Phonetics}{下个月}{xia4ge4yue4}[][HSK 4]
    \definition{s.}{próximo mês; mês que vem; refere"-se ao próximo mês do mês atual}
  \end{Phonetics}
\end{Entry}

\begin{Entry}{下山}{3,3}{⼀,⼭}
  \begin{Phonetics}{下山}{xia4/shan1}[][HSK 7-9]
    \definition{v.+compl.}{descer uma colina ou montanha | (Sol) pôr"-se; desaparecer abaixo do horizonte}
  \end{Phonetics}
\end{Entry}

\begin{Entry}{下午}{3,4}{⼀,⼗}
  \begin{Phonetics}{下午}{xia4wu3}[][HSK 1]
    \definition[个]{s.}{tarde; \emph{post meridiem} (p.m.); refere"-se ao período entre o meio"-dia e o pôr do sol}
  \antonymref{上午}{shang4wu3}
  \end{Phonetics}
\end{Entry}

\begin{Entry}{下午茶}{3,4,9}{⼀,⼗,⾋}
  \begin{Phonetics}{下午茶}{xia4wu3cha2}
    \definition[杯]{s.}{chá da tarde (normalmente chás com doces)}
  \end{Phonetics}
\end{Entry}

\begin{Entry}{下巴}{3,4}{⼀,⼰}
  \begin{Phonetics}{下巴}{xia4ba5}
    \definition[个]{s.}{queixo | mandíbula inferior}
  \end{Phonetics}
\end{Entry}

\begin{Entry}{下手}{3,4}{⼀,⼿}
  \begin{Phonetics}{下手}{xia4/shou3}[][HSK 7-9]
    \definition{v.+compl.}{colocar a mão na massa (em uma tarefa, etc.); começar a fazer algo; empreender | fazer algo ruim a alguém}
  \synonymref{出手}{chu1/shou3}
  \synonymref{动手}{dong4/shou3}
  \synonymref{开始}{kai1shi3}
  \synonymref{入手}{ru4shou3}
  \synonymref{着手}{zhuo2shou3}
  \end{Phonetics}
\end{Entry}

\begin{Entry}{下方}{3,4}{⼀,⽅}
  \begin{Phonetics}{下方}{xia4fang1}
    \definition{s.}{parte inferior | abaixo | embaixo | mundo dos mortais}
    \definition{v.}{descer ao mundo dos mortais (deuses)}
  \seealsoref{上方}{shang4fang1}
  \antonymref{上方}{shang4fang1}
  \end{Phonetics}
\end{Entry}

\begin{Entry}{下水道}{3,4,12}{⼀,⽔,⾡}
  \begin{Phonetics}{下水道}{xia4shui3dao4}
    \definition{s.}{esgoto}
  \end{Phonetics}
\end{Entry}

\begin{Entry}{下车}{3,4}{⼀,⾞}
  \begin{Phonetics}{下车}{xia4che1}[][HSK 1]
    \definition{v.}{descer ou sair de (um ônibus, trem, carro etc.)}
  \end{Phonetics}
\end{Entry}

\begin{Entry}{下令}{3,5}{⼀,⼈}
  \begin{Phonetics}{下令}{xia4/ling4}[][HSK 7-9]
    \definition{v.+compl.}{dar ordens; ordenar}
  \synonymref{命令}{ming4ling4}
  \end{Phonetics}
\end{Entry}

\begin{Entry}{下功夫}{3,5,4}{⼀,⼒,⼤}
  \begin{Phonetics}{下功夫}{xia4 gong1fu5}[][HSK 7-9]
    \definition{v.}{concentrar esforços; investir tempo e energia em algo para alcançar bons resultados}
  \end{Phonetics}
\end{Entry}

\begin{Entry}{下去}{3,5}{⼀,⼛}
  \begin{Phonetics}{下去}{xia4 qu5}[][HSK 3]
    \definition{part.}{usado depois de verbos para indicar de alto a baixo | usado depois de um verbo para indicar continuação}
    \definition{v.}{descer; baixar (a partir da minha localização) | (após um verbo) continuar (fazendo algo); prosseguir | usado após o verbo, indica uma descida de um ponto alto para um ponto baixo | usado após o verbo, indica continuidade | usado após um adjetivo, indica que o grau continua aumentando}
  \synonymref{出来}{chu1 lai5}
  \synonymref{下来}{xia4 lai5}
  \antonymref{上来}{shang4 lai5}
  \end{Phonetics}
\end{Entry}

\begin{Entry}{下边}{3,5}{⼀,⾡}
  \begin{Phonetics}{下边}{xia4bian5}[][HSK 1]
    \definition{s.}{abaixo; sob; por baixo | próximo em ordem; seguinte | nível inferior; subordinado | a parte inferior}
  \antonymref{上边}{shang4bian5}
  \end{Phonetics}
\end{Entry}

\begin{Entry}{下决心}{3,6,4}{⼀,⼎,⼼}
  \begin{Phonetics}{下决心}{xia4 jue2xin1}[][HSK 7-9]
    \definition{v.}{resolver; determinar; tomar uma decisão}
  \end{Phonetics}
\end{Entry}

\begin{Entry}{下场}{3,6}{⼀,⼟}
  \begin{Phonetics}{下场}{xia4chang3}[][HSK 7-9]
    \definition[个,种]{s.}{final ruim; estado lamentável; destino}
    \definition{v.}{sair de cena; encerrar a noite}
  \seealsoref{结局}{jie2ju2}
  \synonymref{结局}{jie2ju2}
  \synonymref{结束}{jie2shu4}
  \antonymref{上场}{shang4/chang3}
  \end{Phonetics}
\end{Entry}

\begin{Entry}{下旬}{3,6}{⼀,⽇}
  \begin{Phonetics}{下旬}{xia4xun2}
    \definition[月]{s.}{última dezena do mês; último período de dez dias de um mês; do dia 21 até o final de cada mês}
  \seealsoref{中旬}{zhong1xun2}
  \antonymref{上旬}{shang4xun2}
  \end{Phonetics}
\end{Entry}

\begin{Entry}{下次}{3,6}{⼀,⽋}
  \begin{Phonetics}{下次}{xia4ci4}[][HSK 1]
    \definition{s.}{na próxima vez; na próxima oportunidade ou no próximo evento}
  \synonymref{再次}{zai4ci4}
  \end{Phonetics}
\end{Entry}

\begin{Entry}{下级}{3,6}{⼀,⽷}
  \begin{Phonetics}{下级}{xia4ji2}[][HSK 7-9]
    \definition{s.}{classificação inferior | subordinado | nível inferior (posição)}
  \synonymref{部下}{bu4xia4}
  \antonymref{上级}{shang4ji2}
  \antonymref{上司}{shang4si5}
  \end{Phonetics}
\end{Entry}

\begin{Entry}{下岗}{3,7}{⼀,⼭}
  \begin{Phonetics}{下岗}{xia4/gang3}[][HSK 7-9]
    \definition{v.+compl.}{ser despedido; perder o emprego; (funcionários) que deixam seus cargos devido à falência da empresa, redução de pessoal ou outros motivos | sair de serviço}
  \antonymref{全职}{quan2zhi2}
  \end{Phonetics}
\end{Entry}

\begin{Entry}{下来}{3,7}{⼀,⽊}
  \begin{Phonetics}{下来}{xia4 lai5}[][HSK 3]
    \definition{part.}{usado após o verbo, indica que a ação ou o comportamento se dirige para a posição do falante ou que a ação é contínua ou concluída | usado após um adjetivo, indica que uma determinada situação começou a ocorrer e continuará a se desenvolver}
    \definition{v.}{descer (para a minha localização) | (colheitas/frutas/vegetais, etc.) ser colhido; estar maduro o suficiente para ser colhido | (período de tempo) acabar; passar; chegar ao fim; indicar o fim de um período de tempo}
  \synonymref{下去}{xia4 qu5}
  \antonymref{上去}{shang4 qu5}
  \end{Phonetics}
\end{Entry}

\begin{Entry}{下周}{3,8}{⼀,⼝}
  \begin{Phonetics}{下周}{xia4zhou1}[][HSK 2]
    \definition{s.}{próxima semana}
  \antonymref{上周}{shang4zhou1}
  \end{Phonetics}
\end{Entry}

\begin{Entry}{下线}{3,8}{⼀,⽷}
  \begin{Phonetics}{下线}{xia4xian4}
    \definition{v.}{sair da sessão; desconectar-se da \emph{Internet}; refere"-se à suspensão temporária das atividades de comunicação online, geralmente significando interromper temporariamente o bate"-papo online ou os jogos online | sair da linha de produção; isso se refere a automóveis, eletrodomésticos, etc., que foram montados na linha de produção e estão prontos para sair da fábrica}
  \synonymref{离开}{li2/kai1}
  \antonymref{登陆}{deng1/lu4}
  \antonymref{问世}{wen4shi4}
  \antonymref{在线}{zai4xian4}
  \end{Phonetics}
\end{Entry}

\begin{Entry}{下降}{3,8}{⼀,⾩}
  \begin{Phonetics}{下降}{xia4jiang4}[][HSK 4]
    \definition{v.}{cair; despencar; declinar; descer; diminuir; ir para baixo}
  \synonymref{低落}{di1luo4}
  \synonymref{降低}{jiang4di1}
  \synonymref{降落}{jiang4luo4}
  \antonymref{回升}{hui2sheng1}
  \antonymref{爬升}{pa2sheng1}
  \antonymref{起飞}{qi3fei1}
  \antonymref{上升}{shang4sheng1}
  \antonymref{上涨}{shang4zhang3}
  \antonymref{增加}{zeng1jia1}
  \antonymref{增长}{zeng1zhang3}
  \end{Phonetics}
\end{Entry}

\begin{Entry}{下限}{3,8}{⼀,⾩}
  \begin{Phonetics}{下限}{xia4xian4}
    \definition{s.}{limite mínimo ou mais recente permitido; limite inferior; limiar; mínimo prescrito; piso (nível)}
  \antonymref{上限}{shang4xian4}
  \end{Phonetics}
\end{Entry}

\begin{Entry}{下雨}{3,8}{⼀,⾬}
  \begin{Phonetics}{下雨}{xia4/yu3}[][HSK 1]
    \definition{v.+compl.}{chover}
  \synonymref{暴雨}{bao4yu3}
  \synonymref{大雨}{da4yu3}
  \end{Phonetics}
\end{Entry}

\begin{Entry}{下面}{3,9}{⼀,⾯}
  \begin{Phonetics}{下面}{xia4mian5}[][HSK 3]
    \definition{s.}{em baixo; abaixo; parte de baixo | próximo; seguinte; a parte posterior; a parte posterior de um artigo ou discurso em relação ao que está sendo narrado no momento | subordinado; o nível inferior; homens nos níveis inferiores | por baixo}
  \synonymref{底下}{di3xia5}
  \synonymref{上方}{shang4fang1}
  \antonymref{上面}{shang4mian5}
  \end{Phonetics}
\end{Entry}

\begin{Entry}{下海}{3,10}{⼀,⽔}
  \begin{Phonetics}{下海}{xia4/hai3}[][HSK 7-9]
    \definition{v.+compl.}{ir pescar no mar; pescar para ganhar a vida; ir para o mar | tornar"-se profissional; isso se refere a atores de ópera amadores que se tornam atores profissionais | deixar o emprego original e abrir o próprio negócio; refere"-se a pessoas que não eram originalmente donas de negócios, mas que passaram a empreender}
  \antonymref{登陆}{deng1/lu4}
  \end{Phonetics}
\end{Entry}

\begin{Entry}{下班}{3,10}{⼀,⽟}
  \begin{Phonetics}{下班}{xia4/ban1}[][HSK 1]
    \definition{v.+compl.}{sair do trabalho; bater ponto; terminar o trabalho na hora prevista e sair do local de trabalho}
  \antonymref{开工}{kai1/gong1}
  \antonymref{上班}{shang4/ban1}
  \end{Phonetics}
\end{Entry}

\begin{Entry}{下课}{3,10}{⼀,⾔}
  \begin{Phonetics}{下课}{xia4/ke4}[][HSK 1]
    \definition{v.+compl.}{terminar a aula; sair da aula}
  \antonymref{上课}{shang4/ke4}
  \end{Phonetics}
\end{Entry}

\begin{Entry}{下调}{3,10}{⼀,⾔}
  \begin{Phonetics}{下调}{xia4diao4}
    \definition{v.}{rebaixar; diminuir a regulamentação | passar para uma unidade inferior}
  \end{Phonetics}
  \begin{Phonetics}{下调}{xia4tiao2}
    \definition{v.}{regular para baixo; ajustar para baixo}
  \end{Phonetics}
\end{Entry}

\begin{Entry}{下载}{3,10}{⼀,⾞}
  \begin{Phonetics}{下载}{xia4zai3}[][HSK 4]
    \definition{v.}{\emph{download}; baixar; salvar informações da \emph{Web} em um dispositivo, como um computador}
  \antonymref{传输}{chuan2shu1}
  \end{Phonetics}
\end{Entry}

\begin{Entry}{下蛋}{3,11}{⼀,⾍}
  \begin{Phonetics}{下蛋}{xia4dan4}
    \definition{v.}{botar ovos}
  \end{Phonetics}
\end{Entry}

\begin{Entry}{下雪}{3,11}{⼀,⾬}
  \begin{Phonetics}{下雪}{xia4/xue3}[][HSK 2]
    \definition{v.+compl.}{nevar}
  \end{Phonetics}
\end{Entry}

\begin{Entry}{下属}{3,12}{⼀,⼫}
  \begin{Phonetics}{下属}{xia4shu3}[][HSK 7-9]
    \definition[个,位,名]{s.}{subordinado; subalterno}
  \synonymref{部属}{bu4shu3}
  \synonymref{部下}{bu4xia4}
  \synonymref{下级}{xia4ji2}
  \antonymref{上司}{shang4si5}
  \end{Phonetics}
\end{Entry}

\begin{Entry}{下崽}{3,12}{⼀,⼭}
  \begin{Phonetics}{下崽}{xia4zai3}
    \definition{v.}{dar à luz à filhotes; parir}
  \end{Phonetics}
\end{Entry}

\begin{Entry}{下期}{3,12}{⼀,⽉}
  \begin{Phonetics}{下期}{xia4qi1}[][HSK 7-9]
    \definition{s.}{da próxima vez; o período ou intervalo de tempo do próximo evento ou atividade.}
  \end{Phonetics}
\end{Entry}

\begin{Entry}{下棋}{3,12}{⼀,⽊}
  \begin{Phonetics}{下棋}{xia4/qi2}[][HSK 7-9]
    \definition{v.+compl.}{jogar xadrez; ter uma partida de xadrez}
  \end{Phonetics}
\end{Entry}

\begin{Entry}{下游}{3,12}{⼀,⽔}
  \begin{Phonetics}{下游}{xia4you2}
    \definition{s.}{a jusante; rio abaixo; trechos inferiores; o trecho do rio próximo à sua foz | para trás; a posição inferior; referindo"-se metaforicamente a uma posição invertida}
  \antonymref{上游}{shang4you2}
  \end{Phonetics}
\end{Entry}

\begin{Entry}{下落}{3,12}{⼀,⾋}
  \begin{Phonetics}{下落}{xia4luo4}[][HSK 7-9]
    \definition{s.}{localização da pessoa ou coisa que está sendo procurada}
    \definition{v.}{cair; deixar cair}
  \synonymref{落下}{luo4xia4}
  \synonymref{下跌}{xia4die1}
  \synonymref{下降}{xia4jiang4}
  \antonymref{爬升}{pa2sheng1}
  \antonymref{上升}{shang4sheng1}
  \antonymref{上涨}{shang4zhang3}
  \end{Phonetics}
\end{Entry}

\begin{Entry}{下跌}{3,12}{⼀,⾜}
  \begin{Phonetics}{下跌}{xia4die1}[][HSK 7-9]
    \definition{v.}{cair; despencar; descer}
  \synonymref{下降}{xia4jiang4}
  \antonymref{上涨}{shang4zhang3}
  \end{Phonetics}
\end{Entry}

\begin{Entry}{下楼}{3,13}{⼀,⽊}
  \begin{Phonetics}{下楼}{xia4lou2}[][HSK 4]
    \definition{v.}{descer as escadas}
  \end{Phonetics}
\end{Entry}

%%%%%%%%%% 与 %%%%%%%%%%
\subsection*{与}\addcontentsline{loh}{figure}{与}

\begin{Entry}{与}{3}{⼀}
  \begin{Phonetics}{与}{yu2}
    \definition{part.}{não; o quê; hein}
    \variantof{欤}
  \end{Phonetics}
  \begin{Phonetics}{与}{yu3}[][HSK 6]
    \definition*{s.}{Sobrenome: Yu}
    \definition{conj.}{e; junto com}
    \definition{prep.}{com}
    \definition{v.}{dar; oferecer; conceder | conviver com; estar em bons termos com; socializar; ser amigável | ajudar; apoiar; patrocinar | Literário: esperar}
  \synonymref{给}{gei3}
  \synonymref{跟}{gen1}
  \synonymref{和}{he2}
  \antonymref{取}{qu3}
  \end{Phonetics}
  \begin{Phonetics}{与}{yu4}
    \definition{v.}{participar de; tomar parte em}
  \synonymref{参加}{can1jia1}
  \synonymref{出席}{chu1/xi2}
  \synonymref{加入}{jia1ru4}
  \synonymref{介入}{jie4ru4}
  \antonymref{旁观}{pang2guan1}
  \end{Phonetics}
\end{Entry}

\begin{Entry}{与其}{3,8}{⼀,⼋}
  \begin{Phonetics}{与其}{yu3qi2}
    \definition{conj.}{em vez de\dots; orações de ligação, indicando uma decisão tomada após comparação; 与其 é usado para o aspecto a ser abandonado, enquanto o aspecto a ser escolhido é frequentemente expresso por frases como 不如 ou 宁可}
  \seealsoref{不如}{bu4ru2}
  \seealsoref{宁可}{ning4ke3}
  \synonymref{如果}{ru2guo3}
  \synonymref{因为}{yin1wei5}
  \end{Phonetics}
\end{Entry}

\begin{Entry}{与其……不如……}{3,8,4,6}{⼀,⼋,⼀,⼥}
  \begin{Phonetics}{与其……不如……}{yu3qi2 bu4ru2}
    \definition{conj.}{em vez de\dots é melhor\dots}
  \end{Phonetics}
\end{Entry}

\begin{Entry}{与其……宁可……}{3,8,5,5}{⼀,⼋,⼧,⼝}
  \begin{Phonetics}{与其……宁可……}{yu3qi2 ning4ke3}
    \definition{conj.}{em vez de\dots eu prefereria\dots}
  \end{Phonetics}
\end{Entry}

%%%%%%%%%% 不 %%%%%%%%%%
\subsection*{不}\addcontentsline{loh}{figure}{不}

\begin{Entry}{不}{4}{⼀}
  \begin{Phonetics}{不}{bu2}[(不 $+$ 4º tom)][HSK 1]
    \definition{adv.}{不 muda seu tom para o segundo tom \dpy{bu2} quando vem antes de uma sílaba com o quarto tom}
  \synonymref{没}{mei2}
  \end{Phonetics}
  \begin{Phonetics}{不}{bu4}[][HSK 1]
    \definition{adv.}{(antes de verbos, adjetivos e outros advérbios; nunca antes do verbo 有) não; não vai; não quer | em algumas expressões educadas, significa que não é necessário fazer isso, o que equivale a 不用 ou 不要 | (entre um verbo e seu complemento) não pode | usado com 就 para indicar escolha}
    \definition{part.}{no final da frase para indicar uma pergunta; (usar sozinho ou com uma partícula nas respostas) não}
    \definition{pref.}{(antes de certos substantivos para formar um adjetivo) un-; in-}
  \seealsoref{不要}{bu2yao4}
  \seealsoref{不用}{bu2yong4}
  \seealsoref{就}{jiu4}
  \seealsoref{有}{you3}
  \synonymref{没有}{mei2you5}
  \end{Phonetics}
  \begin{Phonetics}{不}{bu5}
    \definition{adv.}{não (em expressões \{v.\} + 不 + \{v.\})}
  \end{Phonetics}
\end{Entry}

\begin{Entry}{不一会儿}{4,1,6,2}{⼀,⼀,⼈,⼉}
  \begin{Phonetics}{不一会儿}{bu4 yi2hui4r5}[][HSK 2]
    \definition{expr.}{em um momento; em pouco tempo; em breve; depois de algum tempo}
  \synonymref{一会儿}{yi2hui4r5}
  \end{Phonetics}
\end{Entry}

\begin{Entry}{不一定}{4,1,8}{⼀,⼀,⼧}
  \begin{Phonetics}{不一定}{bu4yi2ding4}[][HSK 2]
    \definition{adv.}{talvez; incerto; não tenho certeza; não necessariamente assim; refere"-se a algo que não pode ser determinado}
  \synonymref{差不多}{cha4bu5duo1}
  \synonymref{未必}{wei4bi4}
  \antonymref{肯定}{ken3ding4}
  \antonymref{确定}{que4ding4}
  \antonymref{一定}{yi2ding4}
  \end{Phonetics}
\end{Entry}

\begin{Entry}{不了了之}{4,2,2,3}{⼀,⼅,⼅,⼂}
  \begin{Phonetics}{不了了之}{bu4liao3-liao3zhi1}[][HSK 7-9]
    \definition{expr.}{deixar um assunto sem solução; acabar sem nada definido; pontas soltas | resolver um assunto deixando-o sem solução; abafar; deixando-o sem solução; concluir sem uma conclusão; acabar sem nada definido; concluir sem resultado concreto (decisão); deixá-lo sem solução; deixar (um assunto) seguir seu próprio curso | deixe-o inquieto}
  \end{Phonetics}
\end{Entry}

\begin{Entry}{不久}{4,3}{⼀,⼃}
  \begin{Phonetics}{不久}{bu4jiu3}[][HSK 2]
    \definition{adv.}{em breve; dentro em breve; num futuro próximo | logo depois; pouco tempo depois | não muito tempo (antes ou depois de algo)}
  \synonymref{刚才}{gang1cai2}
  \antonymref{多久}{duo1jiu3}
  \antonymref{好久}{hao3jiu3}
  \end{Phonetics}
\end{Entry}

\begin{Entry}{不大}{4,3}{⼀,⼤}
  \begin{Phonetics}{不大}{bu2da4}[][HSK 1]
    \definition{adv.}{não muito (indicando um grau baixo); não demasiado | não com frequência; raramente; dificilmente}
  \end{Phonetics}
\end{Entry}

\begin{Entry}{不大离}{4,3,10}{⼀,⼤,⼇}
  \begin{Phonetics}{不大离}{bu2da4li2}
    \definition{expr.}{quase perfeito; nada mal; bem próximo; igual a 差不多}
  \seealsoref{差不多}{cha4bu5duo1}
  \end{Phonetics}
\end{Entry}

\begin{Entry}{不已}{4,3}{⼀,⼰}
  \begin{Phonetics}{不已}{bu4yi3}[][HSK 7-9]
    \definition{v.aux.}{ser infinito; usado depois de um verbo para indicar que ele continua sem parar}
  \end{Phonetics}
\end{Entry}

\begin{Entry}{不为人知}{4,4,2,8}{⼀,⼂,⼈,⽮}
  \begin{Phonetics}{不为人知}{bu4wei2ren2zhi1}[][HSK 7-9]
    \definition{expr.}{não conhecido por ninguém | segredo | desconhecido}
  \antonymref{举世闻名}{ju3shi4-wen2ming2}
  \end{Phonetics}
\end{Entry}

\begin{Entry}{不予}{4,4}{⼀,⼅}
  \begin{Phonetics}{不予}{bu4yu3}[][HSK 7-9]
    \definition{v.}{Literário: não dar; negar; recusar; não conceder}
  \end{Phonetics}
\end{Entry}

\begin{Entry}{不仅}{4,4}{⼀,⼈}
  \begin{Phonetics}{不仅}{bu4jin3}[][HSK 3]
    \definition{adv.}{não apenas (em número, quantidade ou extensão); costuma"-se dizer 不仅仅}
    \definition{conj.}{não somente}
  \seealsoref{不仅仅}{bu4jin3jin3}
  \synonymref{不但}{bu2dan4}
  \antonymref{只得}{zhi3de2}
  \end{Phonetics}
\end{Entry}

\begin{Entry}{不仅仅}{4,4,4}{⼀,⼈,⼈}
  \begin{Phonetics}{不仅仅}{bu4jin3jin3}[][HSK 6]
    \definition{adv.}{não só; não apenas}
  \end{Phonetics}
\end{Entry}

\begin{Entry}{不以为然}{4,4,4,12}{⼀,⼈,⼂,⽕}
  \begin{Phonetics}{不以为然}{bu4yi3wei2ran2}[][HSK 7-9]
    \definition{expr.}{não aceitar como correto; objetar | desaprovar | ter exceção a}
  \antonymref{理所当然}{li3suo3dang1ran2}
  \antonymref{五体投地}{wu3ti3tou2di4}
  \end{Phonetics}
\end{Entry}

\begin{Entry}{不公}{4,4}{⼀,⼋}
  \begin{Phonetics}{不公}{bu4gong1}
    \definition{adj.}{injusto; desleal}
  \synonymref{不平}{bu4ping2}
  \antonymref{公平}{gong1ping2}
  \end{Phonetics}
\end{Entry}

\begin{Entry}{不太}{4,4}{⼀,⼤}
  \begin{Phonetics}{不太}{bu2 tai4}[][HSK 2]
    \definition{adv.}{não exatamente | não muito bom}
  \end{Phonetics}
\end{Entry}

\begin{Entry}{不少}{4,4}{⼀,⼩}
  \begin{Phonetics}{不少}{bu4shao3}[][HSK 2]
    \definition{adj.}{muitos; bastante; não poucos; indica uma quantidade considerável, equivalente a muitos ou bastante}
  \synonymref{大量}{da4liang4}
  \synonymref{许多}{xu3duo1}
  \synonymref{最多}{zui4duo1}
  \end{Phonetics}
\end{Entry}

\begin{Entry}{不日}{4,4}{⼀,⽇}
  \begin{Phonetics}{不日}{bu2ri4}
    \definition{adv.}{nos próximos dias; em breve; daqui a alguns dias; não vai demorar muito | Literário: nos próximos dias; daqui a alguns dias}
  \synonymref{近日}{jin4ri4}
  \end{Phonetics}
\end{Entry}

\begin{Entry}{不止}{4,4}{⼀,⽌}
  \begin{Phonetics}{不止}{bu4zhi3}[][HSK 5]
    \definition{adv.}{mais do que; não limitado a; indica mais do que esse valor ou intervalo}
    \definition{v.}{exceder; superar; não ser possível interromper a ação}
  \synonymref{不光}{bu4guang1}
  \synonymref{不仅}{bu4jin3}
  \antonymref{只有}{zhi3you3}
  \end{Phonetics}
\end{Entry}

\begin{Entry}{不见}{4,4}{⼀,⾒}
  \begin{Phonetics}{不见}{bu2jian4}[][HSK 6]
    \definition{v.}{não ver; não conhecer; não encontrar | estar desaparecido; desaparecer; não consiguir encontrar algo}
  \end{Phonetics}
\end{Entry}

\begin{Entry}{不见得}{4,4,11}{⼀,⾒,⼻}
  \begin{Phonetics}{不见得}{bu2jian4de2}[][HSK 7-9]
    \definition{adv.}{não pode; não é provável; não necessariamente; é improvável que}
  \synonymref{不一定}{bu4yi2ding4}
  \antonymref{必定}{bi4ding4}
  \antonymref{肯定}{ken3ding4}
  \antonymref{一定}{yi2ding4}
  \end{Phonetics}
\end{Entry}

\begin{Entry}{不计其数}{4,4,8,13}{⼀,⾔,⼋,⽁}
  \begin{Phonetics}{不计其数}{bu2 ji4 qi2 shu4}
    \definition{expr.}{seu número não pode ser contado; incontáveis; inumeráveis}
  \synonymref{成千上万}{cheng2qian1-shang4wan4}
  \antonymref{仅此而已}{jin3ci3'er2yi3}
  \antonymref{寥寥无几}{liao2liao2-wu2ji3}
  \end{Phonetics}
\end{Entry}

\begin{Entry}{不可思议}{4,5,9,5}{⼀,⼝,⼼,⾔}
  \begin{Phonetics}{不可思议}{bu4ke3-si1yi4}[][HSK 7-9]
    \definition{expr.}{incrível; inacreditável; inimaginável; inconcebível; algo que é difícil de entender ou imaginar.}
  \end{Phonetics}
\end{Entry}

\begin{Entry}{不可避免}{4,5,16,7}{⼀,⼝,⾌,⼉}
  \begin{Phonetics}{不可避免}{bu4ke3-bi4mian3}[][HSK 7-9]
    \definition{expr.}{inevitavelmente; inescapabilidade | inevitável}
  \end{Phonetics}
\end{Entry}

\begin{Entry}{不对}{4,5}{⼀,⼨}
  \begin{Phonetics}{不对}{bu2 dui4}[][HSK 1]
    \definition{adj.}{incorreto; errado | anormal; anômalo; estranho | desarmonia; incompatibilidade; discórdia}
  \synonymref{不易}{bu2yi4}
  \synonymref{差错}{cha1cuo4}
  \synonymref{错误}{cuo4wu4}
  \synonymref{过错}{guo4cuo4}
  \synonymref{过失}{guo4shi1}
  \synonymref{荒谬}{huang1miu4}
  \antonymref{正确}{zheng4que4}
  \end{Phonetics}
\end{Entry}

\begin{Entry}{不平}{4,5}{⼀,⼲}
  \begin{Phonetics}{不平}{bu4ping2}[][HSK 7-9]
    \definition{adj.}{irregular; não nivelado | injusto}
    \definition[些,点]{s.}{injustiça; deslealdade | ressentimento; queixa}
    \definition{v.}{estar indignado; estar ressentido | estar irregular; não estar nivelado; não estar liso | ser injusto}
  \end{Phonetics}
\end{Entry}

\begin{Entry}{不必}{4,5}{⼀,⼼}
  \begin{Phonetics}{不必}{bu2bi4}[][HSK 3]
    \definition{adv.}{não precisa; não tem que; indica que não é necessário em termos de razão ou emoção}
  \synonymref{不用}{bu2yong4}
  \synonymref{无须}{wu2xu1}
  \antonymref{必须}{bi4xu1}
  \antonymref{必要}{bi4yao4}
  \end{Phonetics}
\end{Entry}

\begin{Entry}{不正之风}{4,5,3,4}{⼀,⽌,⼂,⾵}
  \begin{Phonetics}{不正之风}{bu2zheng4zhi1feng1}[][HSK 7-9]
    \definition[种]{expr.}{tendências prejudiciais; tendência doentia; práticas inadequadas; maus estilos de trabalho; ventos malignos; tendências doentias | tendência doentia; negligência médica}
  \end{Phonetics}
\end{Entry}

\begin{Entry}{不用}{4,5}{⼀,⽤}
  \begin{Phonetics}{不用}{bu2yong4}[][HSK 1]
    \definition{v.}{não precisar; não ter necessidade; indicar que, na verdade, não é necessário}
  \synonymref{不必}{bu2bi4}
  \synonymref{抽奖}{chou1 jiang3}
  \synonymref{无须}{wu2xu1}
  \antonymref{必须}{bi4xu1}
  \end{Phonetics}
\end{Entry}

\begin{Entry}{不用说}{4,5,9}{⼀,⽤,⾔}
  \begin{Phonetics}{不用说}{bu2yong4shuo1}[][HSK 7-9]
    \definition{v.}{desnecessário dizer; ser evidente}
  \end{Phonetics}
\end{Entry}

\begin{Entry}{不由自主}{4,5,6,5}{⼀,⽥,⾃,⼂}
  \begin{Phonetics}{不由自主}{bu4you2zi4zhu3}[][HSK 7-9]
    \definition{expr.}{não pode ajudar; involuntariamente | além do controle de alguém; apesar de si mesmo; incapaz de se controlar | não posso evitar}
  \synonymref{情不自禁}{qing2bu2zi4jin1}
  \synonymref{身不由己}{shen1bu4you2ji3}
  \antonymref{独立自主}{du2li4-zi4zhu3}
  \end{Phonetics}
\end{Entry}

\begin{Entry}{不由得}{4,5,11}{⼀,⽥,⼻}
  \begin{Phonetics}{不由得}{bu4you2de5}[][HSK 7-9]
    \definition{adv.}{involuntariamente; como uma consequência necessária}
    \definition{v.}{não pode deixar de; não pode se abster de; não pode evitar (fazer algo); não permitir que certos resultados ocorram em determinadas circunstâncias}
  \synonymref{禁不住}{jin1bu5zhu4}
  \synonymref{忍不住}{ren3bu5zhu4}
  \end{Phonetics}
\end{Entry}

\begin{Entry}{不亚于}{4,6,3}{⼀,⼆,⼆}
  \begin{Phonetics}{不亚于}{bu2ya4yu2}[][HSK 7-9]
    \definition{v.}{ser tão bom quanto; não ser inferior a}
  \end{Phonetics}
\end{Entry}

\begin{Entry}{不亦乐乎}{4,6,5,5}{⼀,⼇,⼃,⼃}
  \begin{Phonetics}{不亦乐乎}{bu2yi4le4hu1}[][HSK 7-9]
    \definition{expr.}{terrivelmente; extremamente; ``Não é um grande prazer?''; ``Que delícia seria se\dots''; o significado original é ``Não é também muito feliz?'' (visto em ``Os Analectos de Confúcio Aprendizado''), e agora é frequentemente usado para expressar atingir o extremo | (como um complemento após 忙得) extremamente; terrivelmente | não é um grande prazer}
  \seealsoref{忙得}{mang2de2}
  \end{Phonetics}
\end{Entry}

\begin{Entry}{不光}{4,6}{⼀,⼉}
  \begin{Phonetics}{不光}{bu4guang1}[][HSK 3]
    \definition{adv.}{não é o único; não apenas; não só; indica que excede uma determinada quantidade ou faixa}
    \definition{conj.}{não somente; não só}
  \synonymref{不仅}{bu4jin3}
  \end{Phonetics}
\end{Entry}

\begin{Entry}{不再}{4,6}{⼀,⼌}
  \begin{Phonetics}{不再}{bu2zai4}[][HSK 6]
    \definition{adv.}{não mais; não repita uma segunda vez}
    \definition{v.}{ter ido embora; não retornar; não aparecer; não existir mais}
  \synonymref{再也}{zai4ye3}
  \antonymref{还是}{hai2shi5}
  \antonymref{仍旧}{reng2jiu4}
  \antonymref{仍然}{reng2ran2}
  \antonymref{一再}{yi2zai4}
  \antonymref{再三}{zai4san1}
  \end{Phonetics}
\end{Entry}

\begin{Entry}{不同}{4,6}{⼀,⼝}
  \begin{Phonetics}{不同}{bu4tong2}[][HSK 2]
    \definition{adj.}{diferente; distinto; não semelhante;}
  \synonymref{差别}{cha1bie2}
  \synonymref{差异}{cha1yi4}
  \synonymref{分别}{fen1bie2}
  \synonymref{分歧}{fen1qi2}
  \synonymref{例外}{li4wai4}
  \synonymref{区别}{qu1bie2}
  \antonymref{雷同}{lei2tong2}
  \antonymref{一律}{yi2lv4}
  \antonymref{一样}{yi2yang4}
  \end{Phonetics}
\end{Entry}

\begin{Entry}{不同寻常}{4,6,6,11}{⼀,⼝,⼨,⼱}
  \begin{Phonetics}{不同寻常}{bu4tong2-xun2chang2}[][HSK 7-9]
    \definition{adj.}{extraordinário; incomum}
  \end{Phonetics}
\end{Entry}

\begin{Entry}{不在乎}{4,6,5}{⼀,⼟,⼃}
  \begin{Phonetics}{不在乎}{bu2 zai4hu5}[][HSK 4]
    \definition{v.}{não se importar; não dar a mínima; não dar atenção}
  \synonymref{无所谓}{wu2suo3wei4}
  \end{Phonetics}
\end{Entry}

\begin{Entry}{不好意思}{4,6,13,9}{⼀,⼥,⼼,⼼}
  \begin{Phonetics}{不好意思}{bu4 hao3yi4si5}[][HSK 2]
    \definition{adj.}{envergonhado; desconfortável; constrangido; sem jeito}
    \definition{interj.}{com licença; peço desculpas; desculpe"-me}
    \definition{v.}{achar constrangedor (fazer algo) | pedir desculpas (por incomodar alguém) | sentir"-se envergonhado | achar algo embaraçoso}
  \end{Phonetics}
\end{Entry}

\begin{Entry}{不如}{4,6}{⼀,⼥}
  \begin{Phonetics}{不如}{bu4ru2}[][HSK 2]
    \definition{conj.}{em vez de; melhor do que; seria melhor; preferiria; seria melhor; usado no início da segunda parte da frase, indica uma escolha feita após comparação (geralmente em correspondência com o termo 与其 no texto anterior)}
    \definition{v.}{ser inferior a; não ser igual a; não ser tão bom quanto;  não poder fazer melhor que}
  \seealsoref{与其}{yu3qi2}
  \synonymref{干脆}{gan1cui4}
  \synonymref{究竟}{jiu1jing4}
  \antonymref{媲美}{pi4mei3}
  \end{Phonetics}
\end{Entry}

\begin{Entry}{不如说}{4,6,9}{⼀,⼥,⾔}
  \begin{Phonetics}{不如说}{bu4ru2 shuo1}[][HSK 7-9]
    \definition{expr.}{é melhor dizer\dots;  usada para indicar que uma afirmação é mais apropriada, precisa ou preferível a outra; é frequentemente usada em situações de comparação e escolha para enfatizar uma afirmação ou ação superior}
  \end{Phonetics}
\end{Entry}

\begin{Entry}{不安}{4,6}{⼀,⼧}
  \begin{Phonetics}{不安}{bu4'an1}[][HSK 3]
    \definition{adj.}{(humor) inquieto; (ambiente, etc.)  instável; intranquilo; perturbado; sem paz | desculpe; frases de cortesia, expressões de desculpas, equivalentes a 不好意思}
  \seealsoref{不好意思}{bu4 hao3yi4si5}
  \antonymref{安定}{an1ding4}
  \antonymref{安心}{an1xin1}
  \antonymref{放心}{fang4/xin1}
  \end{Phonetics}
\end{Entry}

\begin{Entry}{不成}{4,6}{⼀,⼽}
  \begin{Phonetics}{不成}{bu4cheng2}[][HSK 6]
    \definition{adj.}{não é bom; não funciona; impraticável}
    \definition{part.}{usada no final de uma frase para expressar especulação ou tom contraintuitivo, geralmente precedido por palavras como 嘛 ou 莫非}
    \definition{v.}{não ser permitido; não ser permissível; ser impossível}
  \seealsoref{莫非}{mo4fei1}
  \seealsoref{难道}{nan2dao4}
  \synonymref{不行}{bu4 xing2}
  \end{Phonetics}
\end{Entry}

\begin{Entry}{不成话}{4,6,8}{⼀,⼽,⾔}
  \begin{Phonetics}{不成话}{bu4cheng2hua4}
    \definition{expr.}{irracional | chocante; ultrajante; inapropriado}
  \seealsoref{不是话}{bu2shi4hua4}
  \seealsoref{不像话}{bu2xiang4hua4}
  \end{Phonetics}
\end{Entry}

\begin{Entry}{不约而同}{4,6,6,6}{⼀,⽷,⽽,⼝}
  \begin{Phonetics}{不约而同}{bu4yue1'er2tong2}[][HSK 7-9]
    \definition{expr.}{por coincidência; involuntariamente o mesmo; sem acordo prévio, mas de acordo; acontecer de coincidir; opiniões ou ações unânimes sem consulta prévia | fazer ou pensar a mesma coisa sem consulta prévia; acontecer de coincidir}
  \antonymref{见仁见智}{jian4ren2-jian4zhi4}
  \end{Phonetics}
\end{Entry}

\begin{Entry}{不至于}{4,6,3}{⼀,⾄,⼆}
  \begin{Phonetics}{不至于}{bu2zhi4yu2}[][HSK 6]
    \definition{adv.}{não pode ir tão longe a ponto de; não tanto\dots. a ponto de\dots; não a ponto de}
  \end{Phonetics}
\end{Entry}

\begin{Entry}{不行}{4,6}{⼀,⾏}
  \begin{Phonetics}{不行}{bu4 xing2}[][HSK 2]
    \definition{adj.}{não funciona; não é bom; falta de capacidade e habilidade; nível baixo}
    \definition{adv.}{profundamente; terrivelmente; extremamente; expressa um grau muito profundo; incrível (usado após o caractere 得 como complemento)}
    \definition{v.}{não servir; não ser permitido; estar fora de questão | estar à beira da morte}
  \seealsoref{得}{de5}
  \synonymref{不要}{bu2yao4}
  \synonymref{不成}{bu4cheng2}
  \antonymref{可行}{ke3xing2}
  \end{Phonetics}
\end{Entry}

\begin{Entry}{不许}{4,6}{⼀,⾔}
  \begin{Phonetics}{不许}{bu4xu3}[][HSK 5]
    \definition{v.}{não permitir; ser proibido; proibir firmemente | não pode (usado em perguntas retóricas)}
  \antonymref{可以}{ke3yi3}
  \end{Phonetics}
\end{Entry}

\begin{Entry}{不论}{4,6}{⼀,⾔}
  \begin{Phonetics}{不论}{bu2lun4}[][HSK 3]
    \definition{conj.}{não importa (o que, quem, como, etc.); se \dots ou \dots; significa que as condições ou situações são diferentes, mas os resultados permanecem os mesmos; geralmente é seguido por palavras paralelas ou pronomes interrogativos; geralmente é seguido por advérbios como 都 e 总}
    \definition{v.}{não discutir nem argumentar; não discutir; não debater}
  \seealsoref{都}{dou1}
  \seealsoref{总}{zong3}
  \synonymref{不管}{bu4guan3}
  \synonymref{无论}{wu2lun4}
  \end{Phonetics}
\end{Entry}

\begin{Entry}{不论……也……}{4,6,3}{⼀,⾔,⼄}
  \begin{Phonetics}{不论……也……}{bu2lun4 ye3}
    \definition{conj.}{independentemente de\dots e\dots}
  \end{Phonetics}
\end{Entry}

\begin{Entry}{不论……都……}{4,6,10}{⼀,⾔,⾢}
  \begin{Phonetics}{不论……都……}{bu2lun4 dou1}
    \definition{conj.}{independentemente de\dots e\dots}
  \end{Phonetics}
\end{Entry}

\begin{Entry}{不过}{4,6}{⼀,⾡}
  \begin{Phonetics}{不过}{bu2guo4}[][HSK 2]
    \definition{adv.}{apenas; meramente; nada mais do que; indica que não excede um determinado limite, equivalente a 仅 ou 只 | como intensificador após certos adjetivos}
    \definition{conj.}{mas; no entanto; apenas; usado no início da segunda parte da frase, indica o contrário do sentido anterior e modifica ou complementa o significado anterior}
  \synonymref{但是}{dan4shi4}
  \synonymref{可是}{ke3shi4}
  \synonymref{然而}{ran2'er2}
  \synonymref{只是}{zhi3shi4}
  \end{Phonetics}
\end{Entry}

\begin{Entry}{不但}{4,7}{⼀,⼈}
  \begin{Phonetics}{不但}{bu2dan4}[][HSK 2]
    \definition{conj.}{não só\dots mas também; usado na primeira parte de uma frase composta que expressa progressão, a segunda parte geralmente contém conjunções como 而且,  并且 ou advérbios como 也, 还 que correspondem à primeira parte}
  \seealsoref{并且}{bing4qie3}
  \seealsoref{而且}{er2qie3}
  \seealsoref{还}{hai2}
  \seealsoref{也}{ye3}
  \synonymref{不仅}{bu4jin3}
  \end{Phonetics}
\end{Entry}

\begin{Entry}{不但……而且……}{4,7,6,5}{⼀,⼈,⽽,⼀}
  \begin{Phonetics}{不但……而且……}{bu2 dan4 er2qie3}
    \definition{conj.}{não só\dots mas também\dots}
  \end{Phonetics}
\end{Entry}

\begin{Entry}{不免}{4,7}{⼀,⼉}
  \begin{Phonetics}{不免}{bu4mian3}[][HSK 5]
    \definition{adv.}{inevitavelmente; inexoravelmente}
  \synonymref{难免}{nan2mian3}
  \synonymref{未免}{wei4mian3}
  \end{Phonetics}
\end{Entry}

\begin{Entry}{不利}{4,7}{⼀,⼑}
  \begin{Phonetics}{不利}{bu2li4}[][HSK 5]
    \definition{adj.}{desfavorável; desvantajoso; nocivo; prejudicial | malsucedido}
  \synonymref{不良}{bu4liang2}
  \antonymref{有利}{you3li4}
  \end{Phonetics}
\end{Entry}

\begin{Entry}{不利于}{4,7,3}{⼀,⼑,⼆}
  \begin{Phonetics}{不利于}{bu2li4 yu2}[][HSK 7-9]
    \definition{v.}{ser prejudicial a}
  \end{Phonetics}
\end{Entry}

\begin{Entry}{不妨}{4,7}{⼀,⼥}
  \begin{Phonetics}{不妨}{bu4fang2}[][HSK 7-9]
    \definition{adv.}{pode muito bem; não há mal nenhum em; significa que você pode fazer isso, não há problema}
  \synonymref{可能}{ke3neng2}
  \synonymref{没关系}{mei2guan1xi5}
  \antonymref{妨碍}{fang2'ai4}
  \end{Phonetics}
\end{Entry}

\begin{Entry}{不时}{4,7}{⼀,⽇}
  \begin{Phonetics}{不时}{bu4shi2}[][HSK 5]
    \definition{adv.}{frequentemente; de tempos em tempos | a qualquer momento}
  \synonymref{不断}{bu2duan4}
  \synonymref{常常}{chang2chang2}
  \synonymref{时常}{shi2chang2}
  \synonymref{时时}{shi2shi2}
  \synonymref{往往}{wang3wang3}
  \antonymref{偶尔}{ou3'er3}
  \end{Phonetics}
\end{Entry}

\begin{Entry}{不良}{4,7}{⼀,⾉}
  \begin{Phonetics}{不良}{bu4liang2}[][HSK 5]
    \definition{adj.}{ruim; prejudicial; nocivo; insalubre}
  \synonymref{不利}{bu2li4}
  \synonymref{恶性}{e4xing4}
  \synonymref{有害}{you3hai4}
  \antonymref{良好}{liang2hao3}
  \end{Phonetics}
\end{Entry}

\begin{Entry}{不足}{4,7}{⼀,⾜}
  \begin{Phonetics}{不足}{bu4zu2}[][HSK 5]
    \definition{adj.}{não o bastante; inadequado; insuficiente}
    \definition{s.}{deficiência; inadequação; desvantagens, não é bom o suficiente}
    \definition{v.}{não exceder um determinado número | não valer a pena; ser inferior; não merecer | não pode; não deveria}
  \synonymref{不够}{bu2gou4}
  \synonymref{亏损}{kui1sun3}
  \synonymref{缺乏}{que1fa2}
  \synonymref{缺失}{que1shi1}
  \antonymref{充分}{chong1fen4}
  \antonymref{充足}{chong1zu2}
  \antonymref{多余}{duo1yu2}
  \antonymref{过剩}{guo4sheng4}
  \antonymref{满足}{man3zu2}
  \antonymref{先进}{xian1jin4}
  \end{Phonetics}
\end{Entry}

\begin{Entry}{不到}{4,8}{⼀,⼑}
  \begin{Phonetics}{不到}{bu2dao4}
    \definition{adj.}{insuficiente | desconsiderado; descuidado; desatento}
    \definition{adv.}{menos que; abaixo de}
    \definition{v.}{não chegar a; não ir a algum lugar; não comparer ou estar ausente}
  \end{Phonetics}
\end{Entry}

\begin{Entry}{不定}{4,8}{⼀,⼧}
  \begin{Phonetics}{不定}{bu2ding4}[][HSK 7-9]
    \definition{adj.}{indeterminado; indica pensamentos ou ações instáveis; incerteza; às vezes de uma maneira, às vezes de outra}
    \definition{adv.}{não é certo; não necessariamente; não sei, não tenho certeza do que vai acontecer}
  \synonymref{未必}{wei4bi4}
  \antonymref{决定}{jue2ding4}
  \antonymref{一定}{yi2ding4}
  \end{Phonetics}
\end{Entry}

\begin{Entry}{不宜}{4,8}{⼀,⼧}
  \begin{Phonetics}{不宜}{bu4yi2}[][HSK 7-9]
    \definition{adv.}{não adequado; desaconselhável; inapropriado}
  \antonymref{适宜}{shi4yi2}
  \end{Phonetics}
\end{Entry}

\begin{Entry}{不幸}{4,8}{⼀,⼲}
  \begin{Phonetics}{不幸}{bu2xing4}[][HSK 5]
    \definition{adj.}{triste; infeliz; lamentável; azarado | infeliz; indica o mais indesejável (que aconteceu)}
    \definition[些]{s.}{morte; desastre; infortúnio; adversidade; calamidade}
  \synonymref{倒霉}{dao3/mei2}
  \synonymref{厄运}{e4yun4}
  \synonymref{可怜}{ke3lian2}
  \antonymref{幸福}{xing4fu2}
  \antonymref{幸运}{xing4yun4}
  \end{Phonetics}
\end{Entry}

\begin{Entry}{不易}{4,8}{⼀,⽇}
  \begin{Phonetics}{不易}{bu2yi4}[][HSK 5]
    \definition{adj.}{não é fácil; difícil | imutável}
    \definition{v.}{não é fácil fazer algo | não pode ser alterado}
  \synonymref{不对}{bu2 dui4}
  \synonymref{艰难}{jian1nan2}
  \end{Phonetics}
\end{Entry}

\begin{Entry}{不服}{4,8}{⼀,⽉}
  \begin{Phonetics}{不服}{bu4fu2}[][HSK 7-9]
    \definition{v.}{recusar"-se a obedecer (ou cumprir); recusar"-se a aceitar como final; permanecer não convencido por; não ceder a | não estar acostumado a | recusar"-se a obedecer; recalcitrar}
  \synonymref{不平}{bu4ping2}
  \end{Phonetics}
\end{Entry}

\begin{Entry}{不服气}{4,8,4}{⼀,⽉,⽓}
  \begin{Phonetics}{不服气}{bu4 fu2qi4}[][HSK 7-9]
    \definition{v.}{recusar"-se a ser convencido; não confiar}
  \end{Phonetics}
\end{Entry}

\begin{Entry}{不注意}{4,8,13}{⼀,⽔,⼼}
  \begin{Phonetics}{不注意}{bu2zhu4yi4}
    \definition{adj.}{impensado}
    \definition{s.}{desatenção; falta de atenção}
    \definition{v.}{não prestar atenção}
  \end{Phonetics}
\end{Entry}

\begin{Entry}{不知}{4,8}{⼀,⽮}
  \begin{Phonetics}{不知}{bu4zhi1}[][HSK 7-9]
    \definition{v.}{não saber; ser ignorante de; não saber nada sobre; não ter ideia de; não estar ciente de; não ouvir falar de; não ter a mínima ideia}
  \antonymref{知道}{zhi1dao5}
  \end{Phonetics}
\end{Entry}

\begin{Entry}{不知不觉}{4,8,4,9}{⼀,⽮,⼀,⾒}
  \begin{Phonetics}{不知不觉}{bu4zhi1-bu4jue2}[][HSK 7-9]
    \definition{expr.}{imperceptivelmente; inconscientemente; involuntariamente; sem saber; sem querer}
  \end{Phonetics}
\end{Entry}

\begin{Entry}{不知所措}{4,8,8,11}{⼀,⽮,⼾,⼿}
  \begin{Phonetics}{不知所措}{bu4zhi1-suo3cuo4}
    \definition{expr.}{sem saída; estar perdido; estar confuso; não saber o que fazer}
  \end{Phonetics}
\end{Entry}

\begin{Entry}{不经意}{4,8,13}{⼀,⽷,⼼}
  \begin{Phonetics}{不经意}{bu4jing1yi4}[][HSK 7-9]
    \definition{v.}{não tomar cuidado; ser descuidado; ser desatento}
  \end{Phonetics}
\end{Entry}

\begin{Entry}{不肯}{4,8}{⼀,⾁}
  \begin{Phonetics}{不肯}{bu4 ken3}[][HSK 7-9]
    \definition{v.aux.}{não irá; não iria; usado como verbo auxiliar negativo para expressar recusa}
  \antonymref{愿意}{yuan4yi5}
  \end{Phonetics}
\end{Entry}

\begin{Entry}{不便}{4,9}{⼀,⼈}
  \begin{Phonetics}{不便}{bu2bian4}[][HSK 6]
    \definition{adj.}{inconveniente; inapropriado | não ter dinheiro em mãos; estar com pouco dinheiro}
    \definition{v.}{inadequado fazer algo; indica que fazer algo é inapropriado ou inconveniente}
  \antonymref{便利}{bian4li4}
  \antonymref{方便}{fang1bian4}
  \end{Phonetics}
\end{Entry}

\begin{Entry}{不客气}{4,9,4}{⼀,⼧,⽓}
  \begin{Phonetics}{不客气}{bu2 ke4qi4}[][HSK 1]
    \definition{adj.}{rude; indelicado; duro | franco; sincero; direto}
    \definition{expr.}{de modo algum; não mencione isso; de nada}
    \definition{v.}{dizer palavras ou fazer gestos indelicados}
  \end{Phonetics}
\end{Entry}

\begin{Entry}{不怎么}{4,9,3}{⼀,⼼,⼃}
  \begin{Phonetics}{不怎么}{bu4zen3me5}[][HSK 6]
    \definition{adv.}{não muito; não particularmente; não exatamente}
  \end{Phonetics}
\end{Entry}

\begin{Entry}{不怎么样}{4,9,3,10}{⼀,⼼,⼃,⽊}
  \begin{Phonetics}{不怎么样}{bu4 zen3me5yang4}[][HSK 6]
    \definition{adj.}{não muito bom; não particularmente bom | muito indiferente; mais ou menos}
  \end{Phonetics}
\end{Entry}

\begin{Entry}{不是……而是}{4,9,6,9}{⼀,⽇,⽽,⽇}
  \begin{Phonetics}{不是……而是}{bu4shi4 er2shi4}
    \definition{conj.}{não somente\dots mas também\dots, expressam um relacionamento mais profundo e avançado em significado, mas as orações antes e depois são consistentes em expressar significados negativos e afirmativos, entretanto, a primeira metade da frase expressa negação, e a segunda metade expressa afirmação, e o significado das orações anteriores e seguintes não pode ser de um nível mais alto}
  \end{Phonetics}
\end{Entry}

\begin{Entry}{不是话}{4,9,8}{⼀,⽇,⾔}
  \begin{Phonetics}{不是话}{bu2shi4hua4}
    \definition{expr.}{inacreditável; além das palavras; (palavras) não fazem sentido}
  \seealsoref{不像话}{bu2xiang4hua4}
  \seealsoref{不成话}{bu4cheng2hua4}
  \end{Phonetics}
\end{Entry}

\begin{Entry}{不相上下}{4,9,3,3}{⼀,⽬,⼀,⼀}
  \begin{Phonetics}{不相上下}{bu4xiang1-shang4xia4}[][HSK 7-9]
    \definition{expr.}{igualmente semelhante; quase igual; quase no mesmo nível | ser aproximadamente o mesmo; quase igual; ser igual a\dots; estar no mesmo nível; estar um pouco no mesmo nível de\dots; mais ou menos de força igual; aumentar o tamanho para; há pouca (não há muito) para escolher entre os dois; sem muita diferença}
  \end{Phonetics}
\end{Entry}

\begin{Entry}{不耐烦}{4,9,10}{⼀,⽽,⽕}
  \begin{Phonetics}{不耐烦}{bu2 nai4fan2}[][HSK 5]
    \definition{adj.}{impaciente; significa não ser capaz de suportar coisas tediosas ou que causam distração}
  \end{Phonetics}
\end{Entry}

\begin{Entry}{不要}{4,9}{⼀,⾑}
  \begin{Phonetics}{不要}{bu2yao4}[][HSK 2]
    \definition{adv.}{nada de (pedir a alguém para não fazer) | não; expressa proibição e dissuasão}
  \synonymref{不行}{bu4 xing2}
  \synonymref{没用}{mei2yong4}
  \synonymref{推辞}{tui1ci2}
  \antonymref{必要}{bi4yao4}
  \end{Phonetics}
\end{Entry}

\begin{Entry}{不要紧}{4,9,10}{⼀,⾑,⽷}
  \begin{Phonetics}{不要紧}{bu2 yao4jin3}[][HSK 4]
    \definition{adj.}{não é sério; não importa; não importa; não é um problema; nenhum obstáculo; nenhum problema; parece estar tudo bem; à primeira vista, não parece haver nenhum obstáculo}
  \synonymref{没关系}{mei2guan1xi5}
  \synonymref{没什么}{mei2shen2me5}
  \end{Phonetics}
\end{Entry}

\begin{Entry}{不适}{4,9}{⼀,⾡}
  \begin{Phonetics}{不适}{bu2shi4}[][HSK 7-9]
    \definition{adj.}{Literário: desconfortável; indisposto; mal"-estar; sentir"-se mal | Literário: não adequado; inadequado}
  \antonymref{合适}{he2shi4}
  \antonymref{适宜}{shi4yi2}
  \antonymref{适应}{shi4ying4}
  \end{Phonetics}
\end{Entry}

\begin{Entry}{不值}{4,10}{⼀,⼈}
  \begin{Phonetics}{不值}{bu4zhi2}[][HSK 6]
    \definition{v.}{não valer a pena}
  \antonymref{值得}{zhi2/de5}
  \end{Phonetics}
\end{Entry}

\begin{Entry}{不准}{4,10}{⼀,⼎}
  \begin{Phonetics}{不准}{bu4 zhun3}[][HSK 7-9]
    \definition{v.}{proibir; impedir; vedar; não permitir}
  \synonymref{反对}{fan3dui4}
  \synonymref{禁止}{jin4zhi3}
  \synonymref{阻止}{zu3zhi3}
  \antonymref{答应}{da1ying5}
  \antonymref{允许}{yun3xu3}
  \end{Phonetics}
\end{Entry}

\begin{Entry}{不容}{4,10}{⼀,⼧}
  \begin{Phonetics}{不容}{bu4rong2}[][HSK 7-9]
    \definition{s.}{pontos de acupuntura}
    \definition{v.}{não tolerar; não permitir; não poder deixar de}
  \antonymref{容许}{rong2xu3}
  \end{Phonetics}
\end{Entry}

\begin{Entry}{不屑}{4,10}{⼀,⼫}
  \begin{Phonetics}{不屑}{bu2xie4}[][HSK 7-9]
    \definition{adj.}{desdenhoso; zombador}
    \definition{v.}{pensar que algo não vale a pena fazer; sentir que está abaixo da dignidade de alguém fazer algo; ignorar}
  \synonymref{轻蔑}{qing1mie4}
  \antonymref{观赏}{guan1shang3}
  \antonymref{看中}{kan4/zhong4}
  \antonymref{重视}{zhong4shi4}
  \antonymref{注重}{zhu4zhong4}
  \end{Phonetics}
\end{Entry}

\begin{Entry}{不料}{4,10}{⼀,⽃}
  \begin{Phonetics}{不料}{bu2liao4}[][HSK 6]
    \definition{conj.}{inesperadamente; para surpresa de alguém}
  \antonymref{果然}{guo3ran2}
  \antonymref{果真}{guo3zhen1}
  \antonymref{意外}{yi4wai4}
  \end{Phonetics}
\end{Entry}

\begin{Entry}{不耻下问}{4,10,3,6}{⼀,⽿,⼀,⾨}
  \begin{Phonetics}{不耻下问}{bu4chi3-xia4wen4}[][HSK 7-9]
    \definition{expr.}{não ter vergonha de perguntar e aprender com os subordinados; Os Analectos de Confúcio: Gongye Chang: ``Seja rápido para aprender e não tenha vergonha de fazer perguntas.'' significa não ter vergonha de pedir conselhos a pessoas de status inferior ou menos informadas do que você}
  \end{Phonetics}
\end{Entry}

\begin{Entry}{不能不}{4,10,4}{⼀,⾁,⼀}
  \begin{Phonetics}{不能不}{bu4neng2bu4}[][HSK 5]
    \definition{adv.}{tem que; não pode, mas; necessariamente; definitivamente}
  \synonymref{不得不}{bu4de2bu4}
  \end{Phonetics}
\end{Entry}

\begin{Entry}{不起眼}{4,10,11}{⼀,⾛,⽬}
  \begin{Phonetics}{不起眼}{bu4qi3yan3}[][HSK 7-9]
    \definition{adj.}{imperceptível; discreto; não chamativo; despercebido; não apreciado}
  \end{Phonetics}
\end{Entry}

\begin{Entry}{不通}{4,10}{⼀,⾡}
  \begin{Phonetics}{不通}{bu4tong1}[][HSK 6]
    \definition{adj.}{sem sentido; ilógico; agramatical | usado para se referir a coisas abstratas}
    \definition{v.}{obstruir; bloquear; estar obstruído; estar bloqueado; ser intransitável | não saber; não entender; não poder aceitar}
  \end{Phonetics}
\end{Entry}

\begin{Entry}{不难}{4,10}{⼀,⾫}
  \begin{Phonetics}{不难}{bu4 nan2}[][HSK 7-9]
    \definition{v.}{não ser difícil}[这台电脑修理起来不难。===Este computador não é difícil de consertar.]
  \end{Phonetics}
\end{Entry}

\begin{Entry}{不顾}{4,10}{⼀,⾴}
  \begin{Phonetics}{不顾}{bu2gu4}[][HSK 5]
    \definition{v.}{não considerar; desconsiderar | desconsiderar; não levar em consideração; ignorar; não se preocupar com}
  \synonymref{不管}{bu4guan3}
  \end{Phonetics}
\end{Entry}

\begin{Entry}{不假思索}{4,11,9,10}{⼀,⼈,⼼,⽷}
  \begin{Phonetics}{不假思索}{bu4jia3-si1suo3}[][HSK 7-9]
    \definition{expr.}{(agir, responder, etc.) sem pensar; sem hesitação; prontamente; de improviso; reagir instantaneamente}
  \synonymref{毫不犹豫}{hao2 bu4 you2yu4}
  \synonymref{脱口而出}{tuo1kou3'er2chu1}
  \end{Phonetics}
\end{Entry}

\begin{Entry}{不停}{4,11}{⼀,⼈}
  \begin{Phonetics}{不停}{bu4ting2}[][HSK 5]
    \definition{adv.}{sem parar; sem interrupção; continuamente}
  \synonymref{不断}{bu2duan4}
  \synonymref{后续}{hou4xu4}
  \synonymref{继续}{ji4xu4}
  \synonymref{连续}{lian2xu4}
  \synonymref{一直}{yi4zhi2}
  \antonymref{间断}{jian4duan4}
  \antonymref{停止}{ting2zhi3}
  \antonymref{中断}{zhong1duan4}
  \end{Phonetics}
\end{Entry}

\begin{Entry}{不够}{4,11}{⼀,⼣}
  \begin{Phonetics}{不够}{bu2gou4}[][HSK 2]
    \definition{adv.}{insuficiente; indica que não atingiu o nível esperado}
    \definition{v.}{não ser suficiente; indica que é inferior ao exigido em quantidade ou grau}
  \synonymref{不足}{bu4zu2}
  \end{Phonetics}
\end{Entry}

\begin{Entry}{不得了}{4,11,2}{⼀,⼻,⼅}
  \begin{Phonetics}{不得了}{bu4 de2liao3}[][HSK 5]
    \definition{adj.}{terrível; horrível; extremamente sério; indica uma situação grave}
    \definition{adv.}{muito; extremamente; excessivamente; indica um grau profundo}
  \synonymref{了不起}{liao3bu5qi3}
  \end{Phonetics}
\end{Entry}

\begin{Entry}{不得已}{4,11,3}{⼀,⼻,⼰}
  \begin{Phonetics}{不得已}{bu4de2yi3}[][HSK 7-9]
    \definition{adj.}{agir contra a própria vontade; não ter alternativa senão; não há alternativa; tem que ser assim}
  \end{Phonetics}
\end{Entry}

\begin{Entry}{不得不}{4,11,4}{⼀,⼻,⼀}
  \begin{Phonetics}{不得不}{bu4de2bu4}[][HSK 3]
    \definition{adv.}{ter que; não ter outra escolha a não ser; como obrigação ou necessidade}
  \synonymref{不得已}{bu4de2yi3}
  \synonymref{不能不}{bu4neng2bu4}
  \antonymref{不要}{bu2yao4}
  \antonymref{不用}{bu2yong4}
  \end{Phonetics}
\end{Entry}

\begin{Entry}{不得而知}{4,11,6,8}{⼀,⼻,⽽,⽮}
  \begin{Phonetics}{不得而知}{bu4de2'er2zhi1}[][HSK 7-9]
    \definition{expr.}{desconhecido; incapaz de descobrir}
  \end{Phonetics}
\end{Entry}

\begin{Entry}{不惜}{4,11}{⼀,⼼}
  \begin{Phonetics}{不惜}{bu4xi1}[][HSK 7-9]
    \definition{v.}{não hesitar (em fazer algo)}
  \synonymref{浪费}{lang4fei4}
  \end{Phonetics}
\end{Entry}

\begin{Entry}{不敢当}{4,11,6}{⼀,⽁,⼹}
  \begin{Phonetics}{不敢当}{bu4 gan3dang1}[][HSK 5]
    \definition{expr.}{Eu realmente não mereço isso.; Eu não sou digno de tais elogios.; Não estou à altura da honra.; Você me lisonjeia.; palavra de humildade, para mostrar que você não pode pagar (hospitalidade, elogios, etc.)}
  \end{Phonetics}
\end{Entry}

\begin{Entry}{不断}{4,11}{⼀,⽄}
  \begin{Phonetics}{不断}{bu2duan4}[][HSK 3]
    \definition{adv.}{incessantemente; ininterruptamente; continuamente; constantemente}
    \definition{v.}{continuar; enfatiza a continuação da ação}
  \synonymref{不时}{bu4shi2}
  \synonymref{不停}{bu4ting2}
  \synonymref{持续}{chi2xu4}
  \synonymref{继续}{ji4xu4}
  \synonymref{连接}{lian2jie1}
  \synonymref{连续}{lian2xu4}
  \synonymref{陆续}{lu4xu4}
  \synonymref{延续}{yan2xu4}
  \synonymref{一向}{yi2xiang4}
  \synonymref{一直}{yi4zhi2}
  \antonymref{间断}{jian4duan4}
  \antonymref{停止}{ting2zhi3}
  \end{Phonetics}
\end{Entry}

\begin{Entry}{不理}{4,11}{⼀,⽟}
  \begin{Phonetics}{不理}{bu4 li3}[][HSK 7-9]
    \definition{v.}{ignorar; desconsiderar; recusar"-se a reconhecer; não prestar atenção a; não tomar conhecimento de; não responder | não fazer; não lidar com}
  \antonymref{理睬}{li3cai3}
  \end{Phonetics}
\end{Entry}

\begin{Entry}{不堪}{4,12}{⼀,⼟}
  \begin{Phonetics}{不堪}{bu4kan1}[][HSK 7-9]
    \definition{adj.}{extremamente indesejável; usado após adjetivos negativos | (após palavras com conotação negativa) insuportável; muito ruim}
    \definition{v.}{não poder suportar; não poder ficar de pé; ser incapaz de suportar | (geralmente referindo"-se a algo indesejável) não poder; ser incapaz de}
  \end{Phonetics}
\end{Entry}

\begin{Entry}{不景气}{4,12,4}{⼀,⽇,⽓}
  \begin{Phonetics}{不景气}{bu4jing3qi4}[][HSK 7-9]
    \definition{adj.}{frouxo; lento; em estado de depressão | Economia: em depressão; em recessão; em crise; estagnada | estado depressivo; não próspero}
  \end{Phonetics}
\end{Entry}

\begin{Entry}{不曾}{4,12}{⼀,⽈}
  \begin{Phonetics}{不曾}{bu4ceng2}[][HSK 5]
    \definition{adv.}{nunca (ter feito algo); indica que não aconteceu (negação de 曾经)}
  \seealsoref{曾经}{ceng2jing1}
  \antonymref{曾经}{ceng2jing1}
  \end{Phonetics}
\end{Entry}

\begin{Entry}{不然}{4,12}{⼀,⽕}
  \begin{Phonetics}{不然}{bu4ran2}[][HSK 4]
    \definition{adj.}{não é assim; não é o caso}
    \definition{conj.}{se não; caso contrário; indica outra consequência ou circunstância que teria ocorrido se não fosse}
  \synonymref{除非}{chu2fei1}
  \synonymref{否则}{fou3ze2}
  \end{Phonetics}
\end{Entry}

\begin{Entry}{不像话}{4,13,8}{⼀,⼈,⾔}
  \begin{Phonetics}{不像话}{bu2xiang4hua4}[][HSK 7-9]
    \definition{expr.}{absurdo; sem sentido; irracional; uma determinada prática ou afirmação não está de acordo com o senso comum ou a razão e parece irracional | chocante; ultrajante; uma ação, palavra ou situação tão extrema que não pode ser aceita ou tolerada}
  \seealsoref{不是话}{bu2shi4hua4}
  \seealsoref{不成话}{bu4cheng2hua4}
  \end{Phonetics}
\end{Entry}

\begin{Entry}{不慎}{4,13}{⼀,⼼}
  \begin{Phonetics}{不慎}{bu2shen4}[][HSK 7-9]
    \definition{adj.}{descuidado; desavisado}
  \end{Phonetics}
\end{Entry}

\begin{Entry}{不满}{4,13}{⼀,⽔}
  \begin{Phonetics}{不满}{bu4man3}[][HSK 2]
    \definition{adj.}{ressentido; insatisfeito; descontente}
    \definition{v.}{estar descontente com; insatisfação ou descontentamento com alguém ou alguma coisa | ser menor que; quantidade ou tempo insuficientes ou inadequados}
  \synonymref{生气}{sheng1/qi4}
  \antonymref{满意}{man3yi4}
  \end{Phonetics}
\end{Entry}

\begin{Entry}{不禁}{4,13}{⼀,⽰}
  \begin{Phonetics}{不禁}{bu4jin1}[][HSK 6]
    \definition{adv.}{não pode evitar (fazer algo); não pode se abster de; incapaz de conter (produzir certas emoções, realizar certas ações)}
  \synonymref{情不自禁}{qing2bu2zi4jin1}
  \end{Phonetics}
\end{Entry}

\begin{Entry}{不解}{4,13}{⼀,⾓}
  \begin{Phonetics}{不解}{bu4jie3}[][HSK 7-9]
    \definition{adj.}{inexplicável; inseparável}
    \definition{v.}{não entender; deixar de compreender}
  \synonymref{迷惑}{mi2huo4}
  \antonymref{解读}{jie3du2}
  \antonymref{明白}{ming2bai5}
  \end{Phonetics}
\end{Entry}

\begin{Entry}{不辞而别}{4,13,6,7}{⼀,⾟,⽽,⼑}
  \begin{Phonetics}{不辞而别}{bu4ci2'er2bie2}[][HSK 7-9]
    \definition{expr.}{ir embora sem se despedir; sair sem se despedir}
    \definition{v.}{ir embora sem dizer adeus}
  \end{Phonetics}
\end{Entry}

\begin{Entry}{不错}{4,13}{⼀,⾦}
  \begin{Phonetics}{不错}{bu2cuo4}[][HSK 2]
    \definition{adj.}{certo; correto | nada mal; muito bom}
  \synonymref{可以}{ke3yi3}
  \end{Phonetics}
\end{Entry}

\begin{Entry}{不算}{4,14}{⼀,⽵}
  \begin{Phonetics}{不算}{bu2 suan4}[][HSK 7-9]
    \definition{adv.}{não realmente; não em vão; não é grande coisa}
  \end{Phonetics}
\end{Entry}

\begin{Entry}{不管}{4,14}{⼀,⽵}
  \begin{Phonetics}{不管}{bu4guan3}[][HSK 4]
    \definition{conj.}{não importa (o que, como, etc.); independentemente de; indica que, embora as condições ou circunstâncias tenham mudado, o resultado permanece o mesmo; 不管 deve ser seguido por algo incerto}
  \seealsoref{不管……都……}{bu4guan3 dou1}
  \seealsoref{不管……也……}{bu4guan3 ye3}
  \synonymref{不顾}{bu2gu4}
  \synonymref{不论}{bu2lun4}
  \synonymref{无论}{wu2lun4}
  \antonymref{料理}{liao4li3}
  \end{Phonetics}
\end{Entry}

\begin{Entry}{不管……也……}{4,14,3}{⼀,⽵,⼄}
  \begin{Phonetics}{不管……也……}{bu4guan3 ye3}
    \definition{conj.}{independentemente de\dots e\dots}
  \end{Phonetics}
\end{Entry}

\begin{Entry}{不管……都……}{4,14,10}{⼀,⽵,⾢}
  \begin{Phonetics}{不管……都……}{bu4guan3 dou1}
    \definition{conj.}{independentemente de\dots tudo\dots}
  \end{Phonetics}
\end{Entry}

\begin{Entry}{不懈}{4,16}{⼀,⼼}
  \begin{Phonetics}{不懈}{bu2xie4}[][HSK 7-9]
    \definition{adj.}{incansável; incessante; infatigável}
  \synonymref{长远}{chang2yuan3}
  \synonymref{坚持}{jian1chi2}
  \end{Phonetics}
\end{Entry}

\begin{Entry}{不翼而飞}{4,17,6,3}{⼀,⽻,⽽,⾶}
  \begin{Phonetics}{不翼而飞}{bu2yi4'er2fei1}[][HSK 7-9]
    \definition{expr.}{(um objeto) desaparecer sem deixar vestígios; desaparecer no ar; ``voar sem asas'', desaparecer sem deixar rastros; ganhar asas; desaparecer de repente; desaparecer repentinamente | espalhar"-se rapidamente como se tivesse asas; espalhar"-se como fogo}
  \end{Phonetics}
\end{Entry}

%%%%%%%%%% 丑 %%%%%%%%%%
\subsection*{丑}\addcontentsline{loh}{figure}{丑}

\begin{Entry}{丑}{4}{⼀}
  \begin{Phonetics}{丑}{chou3}[][HSK 5]
    \definition*{s.}{Sobrenome: Chou}
    \definition{adj.}{feio, sem atrativos | vergonhoso; desavergonhado; escandaloso; censurável; questionável}
    \definition{s.}{palhaço na ópera de Pequim, etc. | o segundo dos Doze Ramos Terrestres}
  \synonymref{恶}{wu4}
  \antonymref{俊}{jun4}
  \antonymref{美}{mei3}
  \end{Phonetics}
\end{Entry}

\begin{Entry}{丑陋}{4,8}{⼀,⾩}
  \begin{Phonetics}{丑陋}{chou3lou4}[][HSK 7-9]
    \definition{adj.}{feio; de aparência muito feia | desgraçado; inglório; refere"-se a pensamentos e comportamentos desprezíveis}
  \synonymref{丑恶}{chou3'e4}
  \synonymref{难看}{nan2kan4}
  \antonymref{标志}{biao1zhi4}
  \antonymref{好看}{hao3kan4}
  \antonymref{美好}{mei3hao3}
  \antonymref{美丽}{mei3li4}
  \antonymref{美妙}{mei3miao4}
  \antonymref{漂亮}{piao4liang5}
  \antonymref{鲜艳}{xian1yan4}
  \antonymref{优美}{you1mei3}
  \end{Phonetics}
\end{Entry}

\begin{Entry}{丑闻}{4,9}{⼀,⾨}
  \begin{Phonetics}{丑闻}{chou3wen2}[][HSK 7-9]
    \definition[个,件,些]{s.}{escândalo; rumores ou notícias sobre escândalos}
  \synonymref{绯闻}{fei1wen2}
  \end{Phonetics}
\end{Entry}

\begin{Entry}{丑恶}{4,10}{⼀,⼼}
  \begin{Phonetics}{丑恶}{chou3'e4}[][HSK 7-9]
    \definition{adj./s.}{feio; repulsivo; hediondo}
  \synonymref{丑陋}{chou3lou4}
  \antonymref{好看}{hao3kan4}
  \antonymref{俊俏}{jun4qiao4}
  \antonymref{美好}{mei3hao3}
  \antonymref{美妙}{mei3miao4}
  \end{Phonetics}
\end{Entry}

%%%%%%%%%% 专 %%%%%%%%%%
\subsection*{专}\addcontentsline{loh}{figure}{专}

\begin{Entry}{专}{4}{⼀}
  \begin{Phonetics}{专}{zhuan1}
    \definition{adj.}{específico para; dedicado a um uso específico; dedicado a; especial}
    \definition{adv.}{especialmente; especificamente}
    \definition{v.}{monopolizar}
  \antonymref{博}{bo2}
  \end{Phonetics}
\end{Entry}

\begin{Entry}{专门}{4,3}{⼀,⾨}
  \begin{Phonetics}{专门}{zhuan1men2}[][HSK 3]
    \definition{adj.}{especializado; dedicar"-se exclusivamente a uma determinada tarefa; expressa ênfase em fazer frequentemente um determinado tipo de coisa}
    \definition{adv.}{especialmente}
  \synonymref{特地}{te4di4}
  \synonymref{特意}{te4yi4}
  \synonymref{专业}{zhuan1ye4}
  \antonymref{附带}{fu4dai4}
  \antonymref{顺便}{shun4bian4}
  \end{Phonetics}
\end{Entry}

\begin{Entry}{专心}{4,4}{⼀,⼼}
  \begin{Phonetics}{专心}{zhuan1xin1}[][HSK 4]
    \definition{adj.}{absorto; concentrado}
  \synonymref{用心}{yong4 xin1}
  \end{Phonetics}
\end{Entry}

\begin{Entry}{专业}{4,5}{⼀,⼀}
  \begin{Phonetics}{专业}{zhuan1ye4}[][HSK 3]
    \definition{adj.}{profissional; descreve uma pessoa que possui um alto nível ou conhecimento profundo em determinada área}
    \definition[个,门]{s.}{profissão; área específica; comércio especializado; departamentos operacionais da divisão de produção | especialidade; disciplina; matéria especializada; área de estudo específica; em um departamento de uma instituição de ensino superior ou em uma escola profissionalizante de nível médio}
  \synonymref{那些}{na4xie1}
  \synonymref{专门}{zhuan1men2}
  \antonymref{业余}{ye4yu2}
  \end{Phonetics}
\end{Entry}

\begin{Entry}{专业人士}{4,5,2,3}{⼀,⼀,⼈,⼠}
  \begin{Phonetics}{专业人士}{zhuan1ye4ren2shi4}
    \definition{s.}{profissional}
  \end{Phonetics}
\end{Entry}

\begin{Entry}{专业人才}{4,5,2,3}{⼀,⼀,⼈,⼿}
  \begin{Phonetics}{专业人才}{zhuan1ye4ren2cai2}
    \definition{s.}{especialista (em uma área)}
  \end{Phonetics}
\end{Entry}

\begin{Entry}{专业化}{4,5,4}{⼀,⼀,⼔}
  \begin{Phonetics}{专业化}{zhuan1ye4hua4}
    \definition{s.}{especialização}
  \end{Phonetics}
\end{Entry}

\begin{Entry}{专业户}{4,5,4}{⼀,⼀,⼾}
  \begin{Phonetics}{专业户}{zhuan1ye4hu4}
    \definition{s.}{indústria caseira | empresa familiar produzindo um produto especial}
  \end{Phonetics}
\end{Entry}

\begin{Entry}{专业性}{4,5,8}{⼀,⼀,⼼}
  \begin{Phonetics}{专业性}{zhuan1ye4xing4}
    \definition{s.}{profissionalismo | expertise}
  \end{Phonetics}
\end{Entry}

\begin{Entry}{专业教育}{4,5,11,8}{⼀,⼀,⽁,⾁}
  \begin{Phonetics}{专业教育}{zhuan1ye4jiao4yu4}
    \definition{s.}{educação especializada | escola técnica}
  \end{Phonetics}
\end{Entry}

\begin{Entry}{专用}{4,5}{⼀,⽤}
  \begin{Phonetics}{专用}{zhuan1yong4}[][HSK 6]
    \definition{adj.}{(reservado para) uso especial; para um propósito especial; dedicado a uma determinada necessidade ou a uma determinada pessoa}[他需要一个专用的工作空间。===Ele precisava de um espaço de trabalho dedicado.]
  \synonymref{专利}{zhuan1li4}
  \antonymref{公用}{gong1yong4}
  \antonymref{通用}{tong1yong4}
  \end{Phonetics}
\end{Entry}

\begin{Entry}{专利}{4,7}{⼀,⼑}
  \begin{Phonetics}{专利}{zhuan1li4}[][HSK 5]
    \definition[个,项,些]{s.}{patente; a garantia de que os criadores e inventores desfrutem exclusivamente dos benefícios decorrentes de suas criações e invenções durante um determinado período | direitos de patente; referência à patente}
  \synonymref{专用}{zhuan1yong4}
  \antonymref{公共}{gong1gong4}
  \antonymref{公用}{gong1yong4}
  \end{Phonetics}
\end{Entry}

\begin{Entry}{专家}{4,10}{⼀,⼧}
  \begin{Phonetics}{专家}{zhuan1jia1}[][HSK 3]
    \definition[个,位]{s.}{perito; especialista; profissional; pessoa que se dedica ao estudo aprofundado de uma determinada disciplina; pessoa especializada em uma determinada técnica}
  \synonymref{大家}{da4jia1}
  \synonymref{大师}{da4shi1}
  \synonymref{行家}{hang2jia5}
  \synonymref{内行}{nei4hang2}
  \antonymref{外行}{wai4hang2}
  \end{Phonetics}
\end{Entry}

\begin{Entry}{专辑}{4,13}{⼀,⾞}
  \begin{Phonetics}{专辑}{zhuan1ji2}[][HSK 5]
    \definition[张]{s.}{álbum (música) | registro (música) | coleção especial de material impresso ou transmitido}
  \end{Phonetics}
\end{Entry}

\begin{Entry}{专题}{4,15}{⼀,⾴}
  \begin{Phonetics}{专题}{zhuan1ti2}[][HSK 3]
    \definition[个,些,种]{s.}{assunto especial; tópico especial; questões específicas}
  \synonymref{单元}{dan1yuan2}
  \end{Phonetics}
\end{Entry}

%%%%%%%%%% 且 %%%%%%%%%%
\subsection*{且}\addcontentsline{loh}{figure}{且}

\begin{Entry}{且}{5}{⼀}
  \begin{Phonetics}{且}{qie3}[][HSK 7-9]
    \definition*{s.}{Sobrenome: Qie}
    \definition{adv.}{apenas; por enquanto | por um longo tempo; indica algo duradouro e resistente}
    \definition{conj.}{mesmo; até; até mesmo; usado na primeira cláusula de uma frase complexa para expressar concessão, equivalente a 尚且 | ambos\dots e\dots; conecta adjetivos ou verbos para expressar relacionamento paralelo, equivalente a 而且 e 又…又…}
  \seealsoref{而且}{er2qie3}
  \seealsoref{尚且}{shang4 qie3}
  \seealsoref{又……又……}{you4 you4}
  \end{Phonetics}
\end{Entry}

%%%%%%%%%% 世 %%%%%%%%%%
\subsection*{世}\addcontentsline{loh}{figure}{世}

\begin{Entry}{世}{5}{⼀}
  \begin{Phonetics}{世}{shi4}
    \definition*{s.}{Sobrenome: Shi}
    \definition{s.}{vida; tempo de vida; vida humana | geração; geração após geração | idade; era | o mundo; sociedade | Geologia: época, abaixo de período}
  \end{Phonetics}
\end{Entry}

\begin{Entry}{世代}{5,5}{⼀,⼈}
  \begin{Phonetics}{世代}{shi4dai4}[][HSK 7-9]
    \definition{s.}{anos; idades | por gerações; de uma geração para a seguinte; geração após geração | de geração em geração | uma época ou era | geração}
  \end{Phonetics}
\end{Entry}

\begin{Entry}{世纪}{5,6}{⼀,⽷}
  \begin{Phonetics}{世纪}{shi4ji4}[][HSK 3]
    \definition[个,段]{s.}{século; uma unidade para calcular anos, cem anos é um século}
  \synonymref{时代}{shi2dai4}
  \end{Phonetics}
\end{Entry}

\begin{Entry}{世故}{5,9}{⼀,⽁}
  \begin{Phonetics}{世故}{shi4gu4}
    \definition{s.}{os caminhos do mundo; a arte de lidar com as pessoas; experiência de vida}
  \end{Phonetics}
  \begin{Phonetics}{世故}{shi4gu5}[][HSK 7-9]
    \definition{adj.}{sofisticado; experiente; (ao lidar com pessoas) ser diplomático e evitar ofender alguém}
  \end{Phonetics}
\end{Entry}

\begin{Entry}{世界}{5,9}{⼀,⽥}
  \begin{Phonetics}{世界}{shi4jie4}[][HSK 3]
    \definition[个,片,种]{s.}{mundo; todos os lugares da Terra | a soma da natureza e da sociedade humana; refere"-se à soma de toda a existência objetiva na natureza e na sociedade humana | campo; refere"-se a uma determinada área ou campo | o universo sem limites; costumava ser um termo budista, mas agora também se refere ao mundo natural ilimitado e à sociedade humana | situação social; a situação ou atmosfera social de um determinado período}
  \synonymref{地球}{di4qiu2}
  \synonymref{全国}{quan2guo2}
  \synonymref{天下}{tian1xia4}
  \synonymref{宇宙}{yu3zhou4}
  \end{Phonetics}
\end{Entry}

\begin{Entry}{世界级}{5,9,6}{⼀,⽥,⽷}
  \begin{Phonetics}{世界级}{shi4jie4 ji2}[][HSK 7-9]
    \definition{adj.}{de classe mundial}
    \definition{s.}{classe mundial}
  \end{Phonetics}
\end{Entry}

\begin{Entry}{世界杯}{5,9,8}{⼀,⽥,⽊}
  \begin{Phonetics}{世界杯}{shi4jie4bei1}[][HSK 3]
    \definition*{s.}{Copa do Mundo; Troféu da Copa do Mundo}
  \end{Phonetics}
\end{Entry}

\begin{Entry}{世袭}{5,11}{⼀,⾐}
  \begin{Phonetics}{世袭}{shi4xi2}[][HSK 7-9]
    \definition{v.}{obter por herança | herdar; isso se refere à sucessão hereditária de tronos imperiais, títulos de nobreza, etc.}
  \antonymref{选举}{xuan3ju3}
  \end{Phonetics}
\end{Entry}

\begin{Entry}{世锦赛}{5,13,14}{⼀,⾦,⾙}
  \begin{Phonetics}{世锦赛}{shi4jin3sai4}
    \definition*{s.}{Campeonato Mundial}
  \end{Phonetics}
\end{Entry}

%%%%%%%%%% 丘 %%%%%%%%%%
\subsection*{丘}\addcontentsline{loh}{figure}{丘}

\begin{Entry}{丘}{5}{⼀}
  \begin{Phonetics}{丘}{qiu1}
    \definition*{s.}{Sobrenome: Qiu}
    \definition[个]{s.}{monte; outeiro | (literário) sepultura}
  \end{Phonetics}
\end{Entry}

\begin{Entry}{丘陵}{5,10}{⼀,⾩}
  \begin{Phonetics}{丘陵}{qiu1ling2}[][HSK 7-9]
    \definition[个,片]{s.}{colinas; colinas baixas contínuas}
  \end{Phonetics}
\end{Entry}

%%%%%%%%%% 丙 %%%%%%%%%%
\subsection*{丙}\addcontentsline{loh}{figure}{丙}

\begin{Entry}{丙}{5}{⼀}
  \begin{Phonetics}{丙}{bing3}[][HSK 7-9]
    \definition*{s.}{o terceiro dos Dez Troncos Celestiais}
    \definition{s.}{terceiro | Literário: fogo}
  \end{Phonetics}
\end{Entry}

%%%%%%%%%% 业 %%%%%%%%%%
\subsection*{业}\addcontentsline{loh}{figure}{业}

\begin{Entry}{业}{5}{⼀}
  \begin{Phonetics}{业}{ye4}
    \definition*{s.}{Sobrenome: Ye}
    \definition{adv.}{já; indica que a ação foi concluída, equivalente a 已经}
    \definition{s.}{comércio; indústria; ramo de negócios | emprego; ocupação; profissão | curso de estudo | causa; empreendimento | propriedade | carma; o budismo se refere a todas as ações, palavras e pensamentos humanos como carma, que são chamados de carma corporal, carma da fala e carma mental; o carma inclui aspectos bons e ruins, geralmente referindo"-se ao destino ou ao pecado}
    \definition{v.}{envolver"-se em; exercer uma determinada profissão}
  \seealsoref{已经}{yi3jing1}
  \end{Phonetics}
\end{Entry}

\begin{Entry}{业务}{5,5}{⼀,⼒}
  \begin{Phonetics}{业务}{ye4wu4}[][HSK 5]
    \definition[项,笔,个,类,种]{s.}{negócios; trabalho vocacional; trabalho profissional}
  \synonymref{交易}{jiao1yi4}
  \synonymref{生意}{sheng1yi5}
  \synonymref{营业}{ying2ye4}
  \end{Phonetics}
\end{Entry}

\begin{Entry}{业余}{5,7}{⼀,⼈}
  \begin{Phonetics}{业余}{ye4yu2}[][HSK 4]
    \definition{adj.}{tempo livre; depois do expediente; fora do horário de trabalho | amador; não profissional}
  \antonymref{职业}{zhi2ye4}
  \antonymref{专业}{zhuan1ye4}
  \end{Phonetics}
\end{Entry}

%%%%%%%%%% 丛 %%%%%%%%%%
\subsection*{丛}\addcontentsline{loh}{figure}{丛}

\begin{Entry}{丛}{5}{⼀}
  \begin{Phonetics}{丛}{cong2}
    \definition*{s.}{Sobrenome: Cong}
    \definition{s.}{aglomerado; matagal; bosque | coleção; multidão}
  \end{Phonetics}
\end{Entry}

\begin{Entry}{丛林}{5,8}{⼀,⽊}
  \begin{Phonetics}{丛林}{cong2lin2}[][HSK 7-9]
    \definition[片]{s.}{selva; floresta | mosteiro budista}
  \synonymref{森林}{sen1lin2}
  \end{Phonetics}
\end{Entry}

%%%%%%%%%% 东 %%%%%%%%%%
\subsection*{东}\addcontentsline{loh}{figure}{东}

\begin{Entry}{东}{5}{⼀}
  \begin{Phonetics}{东}{dong1}[][HSK 1]
    \definition*{s.}{Sobrenome: Dong}
    \definition{s.}{leste; uma das quatro direções básicas, o lado onde o sol nasce | proprietário; dono | anfitrião (antigamente, o anfitrião ficava a leste e os convidados a oeste)}
  \antonymref{西}{xi1}
  \end{Phonetics}
\end{Entry}

\begin{Entry}{东方}{5,4}{⼀,⽅}
  \begin{Phonetics}{东方}{dong1fang1}[][HSK 2]
    \definition*{s.}{Sobrenome: Dongfang}
    \definition{s.}{leste | oriente; o leste; o Oriente}
  \antonymref{西方}{xi1fang1}
  \end{Phonetics}
\end{Entry}

\begin{Entry}{东方学院}{5,4,8,9}{⼀,⽅,⼦,⾩}
  \begin{Phonetics}{东方学院}{dong1fang1 xue2yuan4}
    \definition*{s.}{Instituto Oriental}
  \end{Phonetics}
\end{Entry}

\begin{Entry}{东北}{5,5}{⼀,⼔}
  \begin{Phonetics}{东北}{dong1bei3}[][HSK 2]
    \definition*{s.}{Nordeste da China; O Nordeste | Manchúria}
    \definition{s.}{nordeste}
  \antonymref{西南}{xi1nan2}
  \end{Phonetics}
\end{Entry}

\begin{Entry}{东半球}{5,5,11}{⼀,⼗,⽟}
  \begin{Phonetics}{东半球}{dong1ban4qiu2}
    \definition*{s.}{Hemisfério Oriental}
  \antonymref{西半球}{xi1ban4qiu2}
  \end{Phonetics}
\end{Entry}

\begin{Entry}{东边}{5,5}{⼀,⾡}
  \begin{Phonetics}{东边}{dong1bian1}[][HSK 1]
    \definition{s.}{leste; o lado leste; refere"-se à fronteira oriental}
  \synonymref{东方}{dong1fang1}
  \antonymref{西边}{xi1bian5}
  \end{Phonetics}
\end{Entry}

\begin{Entry}{东西}{5,6}{⼀,⾑}
  \begin{Phonetics}{东西}{dong1xi1}
    \definition{s.}{leste e oeste | de leste a oeste; a distância de um lugar de leste a oeste}
  \end{Phonetics}
  \begin{Phonetics}{东西}{dong1xi5}[][HSK 1]
    \definition[个,件]{s.}{coisa; refere"-se a todos os tipos de coisas concretas ou abstratas | coisa; criatura; refere"-se especificamente a pessoas ou coisas que causam repulsa ou simpatia}
  \synonymref{对象}{dui4xiang4}
  \synonymref{工具}{gong1ju4}
  \synonymref{家伙}{jia1huo5}
  \synonymref{事物}{shi4wu4}
  \synonymref{物体}{wu4ti3}
  \end{Phonetics}
\end{Entry}

\begin{Entry}{东张西望}{5,7,6,11}{⼀,⼸,⾑,⽉}
  \begin{Phonetics}{东张西望}{dong1zhang1-xi1wang4}[][HSK 7-9]
    \definition{expr.}{olhar (ou espreitar) ao redor; olhar para um lado e para o outro; olhar em todas as direções | olhar (olhar) para um lado e para o outro; olhar (espreitar) ao redor; olhar para a direita e para a esquerda; olhar para todos os lados; olhar para leste e oeste; olhar em todas as direções | olhar (espiar) ao redor}
  \end{Phonetics}
\end{Entry}

\begin{Entry}{东奔西走}{5,8,6,7}{⼀,⼤,⾑,⾛}
  \begin{Phonetics}{东奔西走}{dong1ben1-xi1zou3}[][HSK 7-9]
    \definition{expr.}{correr de um lado para o outro; correr atarefadamente; ir em todas as direções em busca de algo; correr ou se apressar aqui e ali (procurando emprego, ganhando a vida, tentando a sorte em algo ou outro, etc.)}
  \end{Phonetics}
\end{Entry}

\begin{Entry}{东南}{5,9}{⼀,⼗}
  \begin{Phonetics}{东南}{dong1nan2}[][HSK 2]
    \definition*{s.}{Sudeste da China; O Sudeste; refere"-se à região costeira sudeste da China, incluindo as províncias e cidades de Xangai, Jiangsu, Zhejiang, Fujian, Taiwan, etc.}
    \definition{s.}{sudeste}
  \antonymref{西北}{xi1bei3}
  \end{Phonetics}
\end{Entry}

\begin{Entry}{东面}{5,9}{⼀,⾯}
  \begin{Phonetics}{东面}{dong1mian4}
    \definition{s.}{lado leste (de algo)}
  \end{Phonetics}
\end{Entry}

\begin{Entry}{东部}{5,10}{⼀,⾢}
  \begin{Phonetics}{东部}{dong1bu4}[][HSK 3]
    \definition{s.}{o leste; parte oriental; a parte oriental de uma determinada região}
  \antonymref{西部}{xi1bu4}
  \end{Phonetics}
\end{Entry}

\begin{Entry}{东道主}{5,12,5}{⼀,⾡,⼂}
  \begin{Phonetics}{东道主}{dong1dao4zhu3}[][HSK 7-9]
    \definition[个,些,位]{s.}{anfitrião; aquele que paga por uma refeição; o anfitrião do banquete}
  \end{Phonetics}
\end{Entry}

%%%%%%%%%% 丝 %%%%%%%%%%
\subsection*{丝}\addcontentsline{loh}{figure}{丝}

\begin{Entry}{丝}{5}{⼀}
  \begin{Phonetics}{丝}{si1}[][HSK 7-9]
    \definition{clas.}{si, uma unidade de peso (=0,0005 gramas) | usado para expressar a aparência ou expressão de uma pessoa | um décimo de milésimo de certas unidades de medida (medida de comprimento) | usado para representar coisas abstratas}
    \definition[些,种,类,跟,缕]{s.}{seda | uma coisa semelhante a um fio; itens semelhantes à seda | cordas; instrumentos de corda}
  \end{Phonetics}
\end{Entry}

\begin{Entry}{丝毫}{5,11}{⼀,⽊}
  \begin{Phonetics}{丝毫}{si1hao2}[][HSK 7-9]
    \definition{adj.}{um pouco; muito pouco; uma ninharia; um pedacinho; uma partícula; a menor quantidade ou grau}
  \antonymref{全部}{quan2bu4}
  \end{Phonetics}
\end{Entry}

\begin{Entry}{丝绸}{5,11}{⼀,⽷}
  \begin{Phonetics}{丝绸}{si1chou2}[][HSK 7-9]
    \definition[段,米,面]{s.}{seda; tecido de seda; termo genérico para tecidos feitos de seda}
  \end{Phonetics}
\end{Entry}

%%%%%%%%%% 两 %%%%%%%%%%
\subsection*{两}\addcontentsline{loh}{figure}{两}

\begin{Entry}{两}{7}{⼀}
  \begin{Phonetics}{两}{liang3}[][HSK 1,2]
    \definition*{s.}{Sobrenome: Liang}
    \definition{clas.}{liang, uma unidade de peso (=50 gramas)}
    \definition{num.}{dois (sempre usado antes de classificadores) | poucos; alguns; indica um número indeterminado}
    \definition{s.}{ambos (lados); qualquer (lado)}
  \synonymref{二}{er4}
  \end{Phonetics}
\end{Entry}

\begin{Entry}{两口子}{7,3,3}{⼀,⼝,⼦}
  \begin{Phonetics}{两口子}{liang3kou3zi5}[][HSK 7-9]
    \definition{s.}{Coloquial: marido e mulher; casal}
  \end{Phonetics}
\end{Entry}

\begin{Entry}{两手}{7,4}{⼀,⼿}
  \begin{Phonetics}{两手}{liang3shou3}[][HSK 6]
    \definition{s.}{ambas as mãos | ambos os aspectos; táticas duplas | Coloquial: habilidade; capacidade}
  \synonymref{双手}{shuang1shou3}
  \end{Phonetics}
\end{Entry}

\begin{Entry}{两边}{7,5}{⼀,⾡}
  \begin{Phonetics}{两边}{liang3bian1}[][HSK 4]
    \definition{s.}{ambos os lados; ambas as direções; ambos os lugares | ambas as partes; ambos os lados}
  \synonymref{双方}{shuang1fang1}
  \end{Phonetics}
\end{Entry}

\begin{Entry}{两侧}{7,8}{⼀,⼈}
  \begin{Phonetics}{两侧}{liang3ce4}[][HSK 6]
    \definition{s.}{dois flancos; dois (ambos) lados; ambos}
  \end{Phonetics}
\end{Entry}

\begin{Entry}{两岸}{7,8}{⼀,⼭}
  \begin{Phonetics}{两岸}{liang3'an4}[][HSK 5]
    \definition{s.}{ambos os lados; ambas as margens; ambas as costas; entre os dois lados do estreito; bilateral}
  \synonymref{美丽}{mei3li4}
  \end{Phonetics}
\end{Entry}

\begin{Entry}{两码事}{7,8,8}{⼀,⽯,⼅}
  \begin{Phonetics}{两码事}{liang3ma3shi4}
    \definition{expr.}{duas coisas completamente diferentes; dois assuntos diferentes}
  \end{Phonetics}
\end{Entry}

\begin{Entry}{两栖}{7,10}{⼀,⽊}
  \begin{Phonetics}{两栖}{liang3qi1}[][HSK 7-9]
    \definition[类]{adj.}{anfíbio | habilidade em duas formas de arte performática | capaz de trabalhar em duas linhas diferentes | com dupla aptidão}
  \end{Phonetics}
\end{Entry}

%%%%%%%%%% 严 %%%%%%%%%%
\subsection*{严}\addcontentsline{loh}{figure}{严}

\begin{Entry}{严}{7}{⼀}
  \begin{Phonetics}{严}{yan2}[][HSK 4]
    \definition*{s.}{Sobrenome: Yan}
    \definition{adj.}{apertado; próximo | rigoroso; severo; duro; áspero; rigoroso; austero | severo; extremo; difícil}
    \definition{s.}{pai; refere"-se ao pai}
  \antonymref{宽}{kuan1}
  \end{Phonetics}
\end{Entry}

\begin{Entry}{严厉}{7,5}{⼀,⼚}
  \begin{Phonetics}{严厉}{yan2li4}[][HSK 5]
    \definition{adj.}{severo; rigoroso; as palavras e atitudes de crítica ou punição são muito sérias e severas}
  \synonymref{强势}{qiang2shi4}
  \synonymref{严格}{yan2ge2}
  \synonymref{严肃}{yan2su4}
  \antonymref{慈祥}{ci2xiang2}
  \antonymref{和蔼}{he2'ai3}
  \antonymref{宽容}{kuan1rong2}
  \antonymref{平和}{ping2he2}
  \antonymref{温和}{wen1he2}
  \end{Phonetics}
\end{Entry}

\begin{Entry}{严肃}{7,8}{⼀,⾀}
  \begin{Phonetics}{严肃}{yan2su4}[][HSK 5]
    \definition{adj.}{sério; solene; sincero; (expressão, atmosfera, etc.) faz as pessoas se sentirem admiradas e desconfortáveis | sóbrio; grave; sério; sincero}
    \definition{v.}{aplicar rigorosamente; fazer algo sério}
  \synonymref{古板}{gu3ban3}
  \synonymref{稳重}{wen3zhong4}
  \synonymref{严格}{yan2ge2}
  \synonymref{严厉}{yan2li4}
  \antonymref{好笑}{hao3xiao4}
  \antonymref{胡闹}{hu2nao4}
  \antonymref{活泼}{huo2po5}
  \antonymref{轻松}{qing1song1}
  \antonymref{随便}{sui2/bian4}
  \antonymref{随和}{sui2he5}
  \antonymref{戏谑}{xi4xue4}
  \end{Phonetics}
\end{Entry}

\begin{Entry}{严重}{7,9}{⼀,⾥}
  \begin{Phonetics}{严重}{yan2zhong4}[][HSK 4]
    \definition{adj.}{sério; grave; crítico; severo}
  \synonymref{惨重}{can3zhong4}
  \synonymref{恶化}{e4hua4}
  \synonymref{首要}{shou3yao4}
  \synonymref{危机}{wei1ji1}
  \synonymref{危急}{wei1ji2}
  \synonymref{重要}{zhong4yao4}
  \synonymref{主要}{zhu3yao4}
  \antonymref{平常}{ping2chang2}
  \antonymref{轻微}{qing1wei1}
  \antonymref{稍微}{shao1wei1}
  \end{Phonetics}
\end{Entry}

\begin{Entry}{严重打伤}{7,9,5,6}{⼀,⾥,⼿,⼈}
  \begin{Phonetics}{严重打伤}{yan2zhong4 da3 shang1}
    \definition{s.}{gravemente ferido}
  \end{Phonetics}
\end{Entry}

\begin{Entry}{严重伤害}{7,9,6,10}{⼀,⾥,⼈,⼧}
  \begin{Phonetics}{严重伤害}{yan2zhong4 shang1hai4}
    \definition{s.}{ferimento grave; lesão grave}
  \end{Phonetics}
\end{Entry}

\begin{Entry}{严重关切}{7,9,6,4}{⼀,⾥,⼋,⼑}
  \begin{Phonetics}{严重关切}{yan2zhong4guan1qie4}
    \definition{s.}{preocupação séria}
  \end{Phonetics}
\end{Entry}

\begin{Entry}{严重危害}{7,9,6,10}{⼀,⾥,⼙,⼧}
  \begin{Phonetics}{严重危害}{yan2zhong4 wei1hai4}
    \definition{s.}{perigo crítico | dano grave}
  \end{Phonetics}
\end{Entry}

\begin{Entry}{严重后果}{7,9,6,8}{⼀,⾥,⼝,⽊}
  \begin{Phonetics}{严重后果}{yan2zhong4hou4guo3}
    \definition{s.}{consequências sérias | repercursões graves}
  \end{Phonetics}
\end{Entry}

\begin{Entry}{严重地}{7,9,6}{⼀,⾥,⼟}
  \begin{Phonetics}{严重地}{yan2zhong4 di4}
    \definition{adv.}{seriamente | gravemente}
  \end{Phonetics}
\end{Entry}

\begin{Entry}{严重问题}{7,9,6,15}{⼀,⾥,⾨,⾴}
  \begin{Phonetics}{严重问题}{yan2zhong4 wen4ti2}
    \definition{s.}{problema sério}
  \end{Phonetics}
\end{Entry}

\begin{Entry}{严重性}{7,9,8}{⼀,⾥,⼼}
  \begin{Phonetics}{严重性}{yan2zhong4xing4}
    \definition{s.}{seriedade | gravidade}
  \end{Phonetics}
\end{Entry}

\begin{Entry}{严重破坏}{7,9,10,7}{⼀,⾥,⽯,⼟}
  \begin{Phonetics}{严重破坏}{yan2zhong4 po4huai4}
    \definition{s.}{destruição grave}
  \end{Phonetics}
\end{Entry}

\begin{Entry}{严格}{7,10}{⼀,⽊}
  \begin{Phonetics}{严格}{yan2ge2}[][HSK 4]
    \definition{adj.}{rígido; estrito; rigoroso; muito consciente e meticuloso na implementação de sistemas e no domínio de padrões}
    \definition{v.}{tornar (sistemas, provisões, etc.) rigorosos}
  \synonymref{严厉}{yan2li4}
  \synonymref{严肃}{yan2su4}
  \synonymref{用心}{yong4 xin1}
  \antonymref{放松}{fang4song1}
  \antonymref{宽容}{kuan1rong2}
  \antonymref{马虎}{ma3hu5}
  \antonymref{松弛}{song1chi2}
  \end{Phonetics}
\end{Entry}

%%%%%%%%%% 靣 %%%%%%%%%%
\subsection*{靣}\addcontentsline{loh}{figure}{靣}

\begin{Entry}{靣}{8}{⼀}[Kangxi 176]
  \begin{Phonetics}{靣}{mian4}
    \variantof{面}
  \end{Phonetics}
\end{Entry}

%%%%% EOF %%%%%

