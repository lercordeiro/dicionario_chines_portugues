%%%
%%% Radical "⽜"
%%%
\section*{Radical 93: ``⽜'' (牜、⺧)}\addcontentsline{toc}{section}{Radical 93: ⽜,牜、⺧}\addcontentsline{loh}{figure}{\#\#\#\# 93: ⽜}

%%%%%%%%%% 牛 %%%%%%%%%%
\subsection*{牛}\addcontentsline{loh}{figure}{牛}

\begin{Entry}{牛}{4}{⽜}[Kangxi 93]
  \begin{Phonetics}{牛}{niu2}[][HSK 3,5]
    \definition*{s.}{Sobrenome: Niu}
    \definition{adj.}{muito capaz ou bom; descreve pessoas ou coisas como sendo muito capazes, muito competentes | teimoso; arrogante; descreve uma pessoa que é muito orgulhosa ou muito insistente em suas opiniões, difícil de mudar}
    \definition{clas.}{Newton (medida física de força)}
    \definition[头]{s.}{gado; boi | niu (nona das vinte e oito constelações em que a esfera celeste foi dividida, consistindo de seis estrelas, três em Áries e três em Sagitário)}
  \end{Phonetics}
\end{Entry}

\begin{Entry}{牛人}{4,2}{⽜,⼈}
  \begin{Phonetics}{牛人}{niu2ren2}
    \definition{s.}{(coloquial) o cara | verdadeiro especialista | \emph{badass}}
  \end{Phonetics}
\end{Entry}

\begin{Entry}{牛仔裤}{4,5,12}{⽜,⼈,⾐}
  \begin{Phonetics}{牛仔裤}{niu2zai3ku4}[][HSK 5]
    \definition[条]{s.}{calças jeans; calças geralmente feitas de tecido jeans azul grosso}
  \end{Phonetics}
\end{Entry}

\begin{Entry}{牛奶}{4,5}{⽜,⼥}
  \begin{Phonetics}{牛奶}{niu2nai3}[][HSK 1]
    \definition[杯,袋,瓶,盒,箱,桶]{s.}{leite}
  \end{Phonetics}
\end{Entry}

\begin{Entry}{牛肉}{4,6}{⽜,⾁}
  \begin{Phonetics}{牛肉}{niu2rou4}
    \definition{s.}{carne de vaca | bife}
  \end{Phonetics}
\end{Entry}

\begin{Entry}{牛郎织女}{4,8,8,3}{⽜,⾢,⽷,⼥}
  \begin{Phonetics}{牛郎织女}{niu2 lang2 zhi1nv3}
    \definition*{s.}{Vaqueiro e Tecelã (personagens de contos folclóricos) | Altair e Vega (estrelas)}[我们这些牛郎织女都恨透了那条无情的``天河''。===Nós, o Vaqueiro e a Tecelã, odiamos a implacável ``Via Láctea''.]
    \definition{s.}{marido e mulher que vivem longe um do outro}
  \end{Phonetics}
\end{Entry}

\begin{Entry}{牛顿}{4,10}{⽜,⾴}
  \begin{Phonetics}{牛顿}{niu2dun4}
    \definition*{s.}{Newton (nome) | N; Newton, unidade de força do SI}
  \end{Phonetics}
\end{Entry}

%%%%%%%%%% 牝 %%%%%%%%%%
\subsection*{牝}\addcontentsline{loh}{figure}{牝}

\begin{Entry}{牝}{6}{⽜}
  \begin{Phonetics}{牝}{pin4}
    \definition{adj.}{(de certas aves e animais) fêmea}
    \definition{s.}{fêmea (de algumas aves e animais)}
  \antonymref{牡}{mu3}
  \end{Phonetics}
\end{Entry}

%%%%%%%%%% 牡 %%%%%%%%%%
\subsection*{牡}\addcontentsline{loh}{figure}{牡}

\begin{Entry}{牡}{7}{⽜}
  \begin{Phonetics}{牡}{mu3}
    \definition[些]{s.}{macho (de certas aves e animais) | colinas | peônia}
  \antonymref{牝}{pin4}
  \end{Phonetics}
\end{Entry}

\begin{Entry}{牡丹}{7,4}{⽜,⼂}
  \begin{Phonetics}{牡丹}{mu3dan5}[][HSK 7-9]
    \definition{s.}{peônia; peônia arbórea}
  \end{Phonetics}
\end{Entry}

\begin{Entry}{牡丹江}{7,4,6}{⽜,⼂,⽔}
  \begin{Phonetics}{牡丹江}{mu3dan1jiang1}
    \definition*{s.}{Cidade de Mudanjiang na província de Heilongjiang, 黑龙江 no nordeste da China}
  \seealsoref{黑龙江}{hei1long2jiang1}
  \end{Phonetics}
\end{Entry}

%%%%%%%%%% 牦 %%%%%%%%%%
\subsection*{牦}\addcontentsline{loh}{figure}{牦}

\begin{Entry}{牦}{8}{⽜}
  \begin{Phonetics}{牦}{mao2}
    \definition[头]{s.}{iaque; boi da Tartária}
  \end{Phonetics}
\end{Entry}

\begin{Entry}{牦牛}{8,4}{⽜,⽜}
  \begin{Phonetics}{牦牛}{mao2niu2}
    \definition{s.}{iaque}
  \end{Phonetics}
\end{Entry}

%%%%%%%%%% 牧 %%%%%%%%%%
\subsection*{牧}\addcontentsline{loh}{figure}{牧}

\begin{Entry}{牧}{8}{⽜}
  \begin{Phonetics}{牧}{mu4}
    \definition*{s.}{Sobrenome: Mu}
    \definition{v.}{cuidar (de ovelhas, gado, etc.); pastorear}
  \end{Phonetics}
\end{Entry}

\begin{Entry}{牧民}{8,5}{⽜,⽒}
  \begin{Phonetics}{牧民}{mu4min2}[][HSK 7-9]
    \definition[个]{s.}{pastor; pessoas em áreas pastoris que ganham a vida criando gado}
  \end{Phonetics}
\end{Entry}

\begin{Entry}{牧场}{8,6}{⽜,⼟}
  \begin{Phonetics}{牧场}{mu4chang3}[][HSK 7-9]
    \definition[个,片]{s.}{pastagem; campo de pastoreio; área de pastagem; pastos}
  \end{Phonetics}
\end{Entry}

%%%%%%%%%% 物 %%%%%%%%%%
\subsection*{物}\addcontentsline{loh}{figure}{物}

\begin{Entry}{物}{8}{⽜}
  \begin{Phonetics}{物}{wu4}
    \definition{s.}{coisa; matéria; objeto | mundo exterior distinto de si mesmo; outras pessoas; refere"-se a outras pessoas além de si mesmo ou ao ambiente em relação a si mesmo | essência; conteúdo; substância | criatura; criação}
  \end{Phonetics}
\end{Entry}

\begin{Entry}{物业}{8,5}{⽜,⼀}
  \begin{Phonetics}{物业}{wu4ye4}[][HSK 5]
    \definition[处]{s.}{propriedade; gestão de propriedades; gestão patrimonial; administração de imóveis | empresa de administração de imóveis; empresa de gestão imobiliária; empresa de administração de bens imóveis}
  \end{Phonetics}
\end{Entry}

\begin{Entry}{物价}{8,6}{⽜,⼈}
  \begin{Phonetics}{物价}{wu4jia4}[][HSK 5]
    \definition[个]{s.}{preços das commodities; preços das matérias-primas; preço das mercadorias}
  \end{Phonetics}
\end{Entry}

\begin{Entry}{物质}{8,8}{⽜,⾙}
  \begin{Phonetics}{物质}{wu4zhi4}[][HSK 5]
    \definition[种,类,个]{s.}{matéria; substância; algo que existe além do espírito, que pode ser visto, tocado, cheirado ou detectado por instrumentos científicos | material; meios de subsistência; coisas que permitem às pessoas viver ou viver melhor, como comida, roupas, casas, dinheiro, etc.}
  \end{Phonetics}
\end{Entry}

\begin{Entry}{物品}{8,9}{⽜,⼝}
  \begin{Phonetics}{物品}{wu4pin3}[][HSK 6]
    \definition[件,个]{s.}{artigos; itens; bens}
  \end{Phonetics}
\end{Entry}

\begin{Entry}{物理}{8,11}{⽜,⽟}
  \begin{Phonetics}{物理}{wu4li3}
    \definition{s.}{física (disciplina)}
  \end{Phonetics}
\end{Entry}

%%%%%%%%%% 牲 %%%%%%%%%%
\subsection*{牲}\addcontentsline{loh}{figure}{牲}

\begin{Entry}{牲}{9}{⽜}
  \begin{Phonetics}{牲}{sheng1}
    \definition[头]{s.}{gado (para sacrifício) | sacrifício de animais | animal doméstico}
  \end{Phonetics}
\end{Entry}

\begin{Entry}{牲畜}{9,10}{⽜,⽥}
  \begin{Phonetics}{牲畜}{sheng1chu4}[][HSK 7-9]
    \definition[种,群]{s.}{gado; animais domésticos}
  \end{Phonetics}
\end{Entry}

%%%%%%%%%% 牵 %%%%%%%%%%
\subsection*{牵}\addcontentsline{loh}{figure}{牵}

\begin{Entry}{牵}{9}{⽜}
  \begin{Phonetics}{牵}{qian1}[][HSK 6]
    \definition{v.}{conduzir (segurando a mão, o cabresto, etc.); puxar | envolver-se | sentir falta; preocupar-se com | controlar; restringir; ser retido; ser constrangido}
  \end{Phonetics}
\end{Entry}

\begin{Entry}{牵头}{9,5}{⽜,⼤}
  \begin{Phonetics}{牵头}{qian1/tou2}[][HSK 7-9]
    \definition{v.+compl.}{intermediar (por exemplo: casamenteiro) | coordenar (uma operação combinada) | conduzir (um animal pela cabeça) | mediar | assumir a liderança}
  \end{Phonetics}
\end{Entry}

\begin{Entry}{牵扯}{9,7}{⽜,⼿}
  \begin{Phonetics}{牵扯}{qian1che3}[][HSK 7-9]
    \definition{v.}{envolver; arrastar para}
  \end{Phonetics}
\end{Entry}

\begin{Entry}{牵制}{9,8}{⽜,⼑}
  \begin{Phonetics}{牵制}{qian1zhi4}[][HSK 7-9]
    \definition{v.}{conter; imobilizar; amarrar; restringir ou impedir a livre circulação (frequentemente usado em contextos militares)}
  \end{Phonetics}
\end{Entry}

\begin{Entry}{牵挂}{9,9}{⽜,⼿}
  \begin{Phonetics}{牵挂}{qian1gua4}[][HSK 7-9]
    \definition{v.}{preocupar-se; estar preocupado; perder}
  \end{Phonetics}
\end{Entry}

\begin{Entry}{牵涉}{9,10}{⽜,⽔}
  \begin{Phonetics}{牵涉}{qian1she4}[][HSK 7-9]
    \definition{v.}{preocupar-se com; envolver; arrastar para; uma coisa está relacionada a outras coisas ou pessoas}
  \end{Phonetics}
\end{Entry}

%%%%%%%%%% 特 %%%%%%%%%%
\subsection*{特}\addcontentsline{loh}{figure}{特}

\begin{Entry}{特}{10}{⽜}
  \begin{Phonetics}{特}{te4}[][HSK 6]
    \definition{adj.}{especial; incomum; particular; excepcional; diferente do geral | especial; solteiro; solitário}
    \definition{adv.}{muito; extremamente | especialmente; para um propósito especial |mas; somente}
    \definition{clas.}{TEX; abreviação para unidades de medida como TEX; a unidade de medida TEX indica a espessura de um fio têxtil através do seu peso}
    \definition{s.}{espião; agente secreto}
  \end{Phonetics}
\end{Entry}

\begin{Entry}{特大}{10,3}{⽜,⼤}
  \begin{Phonetics}{特大}{te4da4}[][HSK 6]
    \definition{adj.}{especialmente (excepcionalmente) grande; o mais}
  \end{Phonetics}
\end{Entry}

\begin{Entry}{特价}{10,6}{⽜,⼈}
  \begin{Phonetics}{特价}{te4jia4}[][HSK 4]
    \definition{s.}{oferta especial; preço de barganha; preço especial reduzido}
  \end{Phonetics}
\end{Entry}

\begin{Entry}{特地}{10,6}{⽜,⼟}
  \begin{Phonetics}{特地}{te4di4}[][HSK 6]
    \definition{adv.}{especialmente; propositalmente; para um propósito especial}
  \end{Phonetics}
\end{Entry}

\begin{Entry}{特有}{10,6}{⽜,⽉}
  \begin{Phonetics}{特有}{te4you3}[][HSK 5]
    \definition{adj.}{específico; peculiar; característico; único; exclusivo; especial}
  \end{Phonetics}
\end{Entry}

\begin{Entry}{特色}{10,6}{⽜,⾊}
  \begin{Phonetics}{特色}{te4se4}[][HSK 3]
    \definition{s.}{característica; característica distintiva; a cor única, estilo, etc. de um objeto}
  \end{Phonetics}
\end{Entry}

\begin{Entry}{特别}{10,7}{⽜,⼑}
  \begin{Phonetics}{特别}{te4bie2}[][HSK 2]
    \definition{adj.}{especial; particular; fora do comum; diferente dos outros, com características próprias}
    \definition{adv.}{especialmente; particularmente | ainda mais; em particular; frequentemente usado com 是 | especialmente; deliberadamente; para um propósito específico}
  \seealsoref{是}{shi4}
  \end{Phonetics}
\end{Entry}

\begin{Entry}{特别快车}{10,7,7,4}{⽜,⼑,⼼,⾞}
  \begin{Phonetics}{特别快车}{te4bie2 kuai4che1}
    \definition{s.}{trem expresso; expresso; expresso especial; refere"-se a trens de passageiros que param em menos estações e têm menor tempo de viagem do que trens expressos diretos}
  \end{Phonetics}
\end{Entry}

\begin{Entry}{特快}{10,7}{⽜,⼼}
  \begin{Phonetics}{特快}{te4kuai4}[][HSK 6]
    \definition{adj.}{expresso (trem, entrega etc.)}
    \definition{s.}{trem expresso; abreviação de 特别快车}
  \seealsoref{特别快车}{te4bie2 kuai4che1}
  \antonymref{普快}{pu3 kuai4}
  \end{Phonetics}
\end{Entry}

\begin{Entry}{特技}{10,7}{⽜,⼿}
  \begin{Phonetics}{特技}{te4ji4}
    \definition{s.}{efeito especial | dublê}
  \end{Phonetics}
\end{Entry}

\begin{Entry}{特定}{10,8}{⽜,⼧}
  \begin{Phonetics}{特定}{te4ding4}[][HSK 5]
    \definition{adj.}{particular; específico; especialmente designado | dado; especificado; específico; uma pessoa específica, um determinado momento, lugar, ambiente, etc.}
  \end{Phonetics}
\end{Entry}

\begin{Entry}{特征}{10,8}{⽜,⼻}
  \begin{Phonetics}{特征}{te4zheng1}[][HSK 4]
    \definition[个,种]{s.}{característica; aparência ou o fenômeno característico de uma pessoa ou coisa que pode ser visto de fora}
  \end{Phonetics}
\end{Entry}

\begin{Entry}{特性}{10,8}{⽜,⼼}
  \begin{Phonetics}{特性}{te4xing4}[][HSK 5]
    \definition[种,个]{s.}{propriedade específica (ou característica) | característica; sabores | propriedade}
  \end{Phonetics}
\end{Entry}

\begin{Entry}{特点}{10,9}{⽜,⽕}
  \begin{Phonetics}{特点}{te4dian3}[][HSK 2]
    \definition[个,大]{s.}{característica; peculiaridade; traço distintivo; a singularidade de uma pessoa ou coisa}
  \end{Phonetics}
\end{Entry}

\begin{Entry}{特殊}{10,10}{⽜,⽍}
  \begin{Phonetics}{特殊}{te4shu1}[][HSK 4]
    \definition{adj.}{especial; particular; peculiar; excepcional; incomum}
  \end{Phonetics}
\end{Entry}

\begin{Entry}{特意}{10,13}{⽜,⼼}
  \begin{Phonetics}{特意}{te4yi4}[][HSK 6]
    \definition{adv.}{especialmente; para um propósito especial}
  \end{Phonetics}
\end{Entry}

%%%%%%%%%% 牺 %%%%%%%%%%
\subsection*{牺}\addcontentsline{loh}{figure}{牺}

\begin{Entry}{牺}{10}{⽜}
  \begin{Phonetics}{牺}{xi1}
    \definition{s.}{um animal de cor uniforme para sacrifício; sacrifício; gado com pelagem pura usado para sacrifício}
  \end{Phonetics}
\end{Entry}

\begin{Entry}{牺牲}{10,9}{⽜,⽜}
  \begin{Phonetics}{牺牲}{xi1sheng1}[][HSK 6]
    \definition[份]{s.}{sacrifício; um animal abatido para sacrifício; refere"-se ao sacrifício da própria vida ou dos próprios interesses por um propósito justo, ou refere"-se ao preço pago por um determinado propósito}
    \definition{v.}{sacrificar"-se; morrer como mártir; dar a própria vida; sacrificar sua vida pela justiça | sacrificar; desistir; fazer algo às custas de; geralmente se refere a pagar um preço ou sofrer danos por alguém ou algo}
  \end{Phonetics}
\end{Entry}

%%%%%%%%%% 犟 %%%%%%%%%%
\subsection*{犟}\addcontentsline{loh}{figure}{犟}

\begin{Entry}{犟}{16}{⽜}
  \begin{Phonetics}{犟}{jiang4}
    \variantof{强}
  \end{Phonetics}
\end{Entry}

%%%%% EOF %%%%%

