%%%
%%% Radical "⾲"
%%%
\section*{Radical 179: ``⾲''}\addcontentsline{toc}{section}{Radical 179: ⾲}\addcontentsline{loh}{figure}{\#\#\#\# 179: ⾲}

%%%%%%%%%% 韭 %%%%%%%%%%
\subsection*{韭}\addcontentsline{loh}{figure}{韭}

\begin{Entry}{韭}{9}{⾲}[Kangxi 179]
  \begin{Phonetics}{韭}{jiu3}
    \definition{s.}{alho de flor perfumada; cebolinha chinesa}
  \end{Phonetics}
\end{Entry}

\begin{Entry}{韭菜}{9,11}{⾲,⾋}
  \begin{Phonetics}{韭菜}{jiu3cai4}
    \definition{s.}{cebolinha-de-alho; cebolinha chinesa | Coloquial: pessoas ingênuas ou facilmente exploráveis, especialmente pequenos investidores ou consumidores, comparadas à cebolinha, que pode ser colhida repetidamente para obter lucro}
  \end{Phonetics}
\end{Entry}

%%%%% EOF %%%%%

