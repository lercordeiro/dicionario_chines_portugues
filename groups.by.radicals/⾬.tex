%%%
%%% Radical "⾬"
%%%
\section*{Radical 173: ``⾬''}\addcontentsline{toc}{section}{Radical 173: ⾬}\addcontentsline{loh}{figure}{\#\#\#\# 173: ⾬}

%%%%%%%%%% 雨 %%%%%%%%%%
\subsection*{雨}\addcontentsline{loh}{figure}{雨}

\begin{Entry}{雨}{8}{⾬}[Kangxi 173]
  \begin{Phonetics}{雨}{yu3}[][HSK 1]
    \definition*{s.}{Sobrenome: Yu}
    \definition[场,阵,滴]{s.}{chuva; água que cai das nuvens para o solo}
  \end{Phonetics}
  \begin{Phonetics}{雨}{yu4}
    \definition{v.}{cair (chuva, neve, etc.) | precipitar | chover | molhar}
  \end{Phonetics}
\end{Entry}

\begin{Entry}{雨水}{8,4}{⾬,⽔}
  \begin{Phonetics}{雨水}{yu3 shui3}[][HSK 5]
    \definition{s.}{água da chuva; precipitação; chuva; água proveniente da chuva}
  \end{Phonetics}
\end{Entry}

\begin{Entry}{雨伞}{8,6}{⾬,⼈}
  \begin{Phonetics}{雨伞}{yu3san3}
    \definition[把]{s.}{guarda-chuva}
  \end{Phonetics}
\end{Entry}

\begin{Entry}{雨衣}{8,6}{⾬,⾐}
  \begin{Phonetics}{雨衣}{yu3 yi1}[][HSK 6]
    \definition[件,个]{s.}{capa de chuva; jaqueta impermeável; roupas impermeáveis}
  \end{Phonetics}
\end{Entry}

\begin{Entry}{雨蚀}{8,9}{⾬,⾷}
  \begin{Phonetics}{雨蚀}{yu3shi2}
    \definition{s.}{erosão da chuva}
  \end{Phonetics}
\end{Entry}

\begin{Entry}{雨靴}{8,13}{⾬,⾰}
  \begin{Phonetics}{雨靴}{yu3xue1}
    \definition[双]{s.}{botas de chuva}
  \end{Phonetics}
\end{Entry}

%%%%%%%%%% 雪 %%%%%%%%%%
\subsection*{雪}\addcontentsline{loh}{figure}{雪}

\begin{Entry}{雪}{11}{⾬}
  \begin{Phonetics}{雪}{xue3}[][HSK 2]
    \definition*{s.}{Sobrenome: Xue}
    \definition[场,层]{s.}{neve | algo parecido com neve}
    \definition{v.}{limpar; enxugar; remover}
  \end{Phonetics}
\end{Entry}

\begin{Entry}{雪人}{11,2}{⾬,⼈}
  \begin{Phonetics}{雪人}{xue3ren2}
    \definition{s.}{boneco de neve | \emph{Yeti}}
  \end{Phonetics}
\end{Entry}

\begin{Entry}{雪山}{11,3}{⾬,⼭}
  \begin{Phonetics}{雪山}{xue3shan1}
    \definition{s.}{montanha coberta de neve}
  \end{Phonetics}
\end{Entry}

\begin{Entry}{雪花}{11,7}{⾬,⾋}
  \begin{Phonetics}{雪花}{xue3hua1}
    \definition{s.}{floco de neve}
  \end{Phonetics}
\end{Entry}

\begin{Entry}{雪板}{11,8}{⾬,⽊}
  \begin{Phonetics}{雪板}{xue3ban3}
    \definition{s.}{prancha de \emph{snowboard}}
    \definition{v.}{praticar \textit{snowboard}}
  \end{Phonetics}
\end{Entry}

\begin{Entry}{雪葩}{11,12}{⾬,⾋}
  \begin{Phonetics}{雪葩}{xue3pa1}
    \definition{s.}{sorvete}
  \end{Phonetics}
\end{Entry}

\begin{Entry}{雪鞋}{11,15}{⾬,⾰}
  \begin{Phonetics}{雪鞋}{xue3xie2}
    \definition[双]{s.}{sapatos de neve}
  \end{Phonetics}
\end{Entry}

\begin{Entry}{雪糕}{11,16}{⾬,⽶}
  \begin{Phonetics}{雪糕}{xue3gao1}
    \definition{s.}{picolé}
  \end{Phonetics}
\end{Entry}

%%%%%%%%%% 零 %%%%%%%%%%
\subsection*{零}\addcontentsline{loh}{figure}{零}

\begin{Entry}{零}{13}{⾬}
  \begin{Phonetics}{零}{ling2}[][HSK 1]
    \definition*{s.}{Sobrenome: Ling}
    \definition{adj.}{ímpar; dispersos; fragmentados (em oposição a 整)}
    \definition{num.}{zero; 0; também grafado como 〇; representa um número menor que qualquer número positivo e maior que qualquer número negativo; representa a ausência de quantidade | zero grau no termômetro | usado para indicar qualidade, comprimento, tempo, idade, etc. Entre dois dígitos, indica que a quantidade da unidade mais alta é acompanhada pela quantidade da unidade mais baixa | sinal de zero (0); nulo; espaço em branco para indicar números em caracteres chineses maiúsculos}
    \definition{s.}{fragmento; fração; lote ímpar; um número fracionário que não é suficiente para uma determinada unidade; um ponto decimal diferente de um inteiro}
    \definition{v.}{(de chuva, lágrimas, etc.) cair | murchar e cair}
  \seealsoref{整}{zheng3}
  \end{Phonetics}
\end{Entry}

\begin{Entry}{零下}{13,3}{⾬,⼀}
  \begin{Phonetics}{零下}{ling2 xia4}[][HSK 2]
    \definition{s.}{abaixo de zero; negativo}
  \end{Phonetics}
\end{Entry}

\begin{Entry}{零件}{13,6}{⾬,⼈}
  \begin{Phonetics}{零件}{ling2jian4}[][HSK 7-9]
    \definition[种,个]{s.}{parte; detalhe; elemento; componente; componentes individuais que podem ser montados em máquinas, ferramentas, etc.}
  \end{Phonetics}
\end{Entry}

\begin{Entry}{零花钱}{13,7,10}{⾬,⾋,⾦}
  \begin{Phonetics}{零花钱}{ling2hua1qian2}[][HSK 7-9]
    \definition[些]{s.}{mesada; dinheiro de bolso; pequenas quantias em dinheiro para despesas pessoais diversas}
  \end{Phonetics}
\end{Entry}

\begin{Entry}{零食}{13,9}{⾬,⾷}
  \begin{Phonetics}{零食}{ling2shi2}[][HSK 4]
    \definition[包,袋,盒,箱,堆]{s.}{lanches; refrescos; petiscos entre as refeições; alimentação esporádica, além das refeições normais}
  \end{Phonetics}
\end{Entry}

\begin{Entry}{零钱}{13,10}{⾬,⾦}
  \begin{Phonetics}{零钱}{ling2qian2}[][HSK 7-9]
    \definition[点,些]{s.}{troco; troco pequeno; moedas de pequeno valor, como 角 (centavos) e 分 (fen) | mesada; pequenas quantias de dinheiro gastas no dia a dia}
  \seealsoref{分}{fen1}
  \seealsoref{角}{jiao3}
  \end{Phonetics}
\end{Entry}

\begin{Entry}{零售}{13,11}{⾬,⼝}
  \begin{Phonetics}{零售}{ling2shou4}[][HSK 7-9]
    \definition{v.}{varejo; vender no varejo; venda direta ao consumidor em pequenas quantidades (em oposição à 批发)}
  \seealsoref{批发}{pi1fa1}
  \end{Phonetics}
\end{Entry}

\begin{Entry}{零散}{13,12}{⾬,⽁}
  \begin{Phonetics}{零散}{ling2san3}
    \definition{adj.}{espalhado; disperso}
  \end{Phonetics}
\end{Entry}

%%%%%%%%%% 雷 %%%%%%%%%%
\subsection*{雷}\addcontentsline{loh}{figure}{雷}

\begin{Entry}{雷}{13}{⾬}
  \begin{Phonetics}{雷}{lei2}
    \definition*{s.}{Sobrenome: Lei}
    \definition[声,个,颗]{s.}{trovão | (militar) mina}
  \end{Phonetics}
\end{Entry}

\begin{Entry}{雷电}{13,5}{⾬,⽥}
  \begin{Phonetics}{雷电}{lei2dian4}
    \definition{s.}{trovão e relâmpago; raio}
  \end{Phonetics}
\end{Entry}

\begin{Entry}{雷亚尔}{13,6,5}{⾬,⼆,⼩}
  \begin{Phonetics}{雷亚尔}{lei2ya4'er3}
    \definition*{s.}{Real Brasileiro}
  \end{Phonetics}
\end{Entry}

\begin{Entry}{雷同}{13,6}{⾬,⼝}
  \begin{Phonetics}{雷同}{lei2tong2}[][HSK 7-9]
    \definition{adj.}{duplicado; idêntico; atualmente, é frequentemente usado como metáfora para coisas que não deveriam ser iguais, mas são}
    \definition{v.}{ecoar o que outros disseram; os povos antigos acreditavam que todas as coisas respondiam ao trovão simultaneamente, mais tarde, isso se tornou uma metáfora para a expressão de opiniões alheias}
  \end{Phonetics}
\end{Entry}

%%%%%%%%%% 雾 %%%%%%%%%%
\subsection*{雾}\addcontentsline{loh}{figure}{雾}

\begin{Entry}{雾}{13}{⾬}
  \begin{Phonetics}{雾}{wu4}
    \definition[层,场,阵]{s.}{neblina; pequenas gotas de água condensadas do vapor de água | pulverização fina; como muitas pequenas gotas de água na neblina}
  \end{Phonetics}
\end{Entry}

\begin{Entry}{雾气}{13,4}{⾬,⽓}
  \begin{Phonetics}{雾气}{wu4qi4}
    \definition{s.}{nevoeiro | névoa | vapor}
  \end{Phonetics}
\end{Entry}

%%%%%%%%%% 需 %%%%%%%%%%
\subsection*{需}\addcontentsline{loh}{figure}{需}

\begin{Entry}{需}{14}{⾬}
  \begin{Phonetics}{需}{xu1}
    \definition*{s.}{Sobrenome: Xu}
    \definition{s.}{necessidades; bens de primeira necessidade}
    \definition{v.}{precisar; querer; exigir}
  \end{Phonetics}
\end{Entry}

\begin{Entry}{需求}{14,7}{⾬,⽔}
  \begin{Phonetics}{需求}{xu1qiu2}[][HSK 3]
    \definition[种]{s.}{necessidades; demanda; exigência; solicitações decorrentes de necessidades}
  \end{Phonetics}
\end{Entry}

\begin{Entry}{需要}{14,9}{⾬,⾑}
  \begin{Phonetics}{需要}{xu1yao4}[][HSK 3]
    \definition[种]{s.}{necessidade; desejo ou exigência em relação a algo}
    \definition{v.}{precisar; querer; exigir; demandar; solicitar}
  \end{Phonetics}
\end{Entry}

%%%%%%%%%% 震 %%%%%%%%%%
\subsection*{震}\addcontentsline{loh}{figure}{震}

\begin{Entry}{震}{15}{⾬}
  \begin{Phonetics}{震}{zhen4}
    \definition*{s.}{Zhen, um dos Oito Trigramas que representa o trovão | Sobrenome: Zhen}
    \definition{adj.}{(coloquial) muito animado; profundamente surpreso; chocado}
    \definition{s.}{vibração; trepidação; tremor; abalo | terremoto; refere-se especificamente a terremotos | trovão; relâmpago}
    \definition{v.}{sacudir; chocar; vibrar; estremecer | ficar muito animado; ficar profundamente surpreso; ficar chocado | superar; vencer}
  \end{Phonetics}
\end{Entry}

\begin{Entry}{震惊}{15,11}{⾬,⼼}
  \begin{Phonetics}{震惊}{zhen4jing1}[][HSK 5]
    \definition{adj.}{chocado; atordoado; espantado; atônito}
    \definition{v.}{chocar; surpreender; espantar}
  \end{Phonetics}
\end{Entry}

\begin{Entry}{震撼}{15,16}{⾬,⼿}
  \begin{Phonetics}{震撼}{zhen4han4}
    \definition{v.}{sacudir | chocar | atordoar}
  \end{Phonetics}
\end{Entry}

%%%%%%%%%% 霍 %%%%%%%%%%
\subsection*{霍}\addcontentsline{loh}{figure}{霍}

\begin{Entry}{霍}{16}{⾬}
  \begin{Phonetics}{霍}{huo4}
    \definition*{s.}{Sobrenome: Huo}
    \definition{adv.}{Literário: de repente; rapidamente}
  \end{Phonetics}
\end{Entry}

\begin{Entry}{霍乱}{16,7}{⾬,⼄}
  \begin{Phonetics}{霍乱}{huo4luan4}[][HSK 7-9]
    \definition{s.}{cólera; uma doença altamente contagiosa causada pelo Vibrio Cholerae | gastroenterite aguda (geralmente se refere a sintomas como vômitos intensos, diarreia, dor abdominal e cólicas)}
  \end{Phonetics}
\end{Entry}

%%%%%%%%%% 霜 %%%%%%%%%%
\subsection*{霜}\addcontentsline{loh}{figure}{霜}

\begin{Entry}{霜}{17}{⾬}
  \begin{Phonetics}{霜}{shuang1}
    \definition{s.}{geada | pó branco ou creme espalhado por uma superfície | glacê | creme de pele}
  \end{Phonetics}
\end{Entry}

%%%%%%%%%% 露 %%%%%%%%%%
\subsection*{露}\addcontentsline{loh}{figure}{露}

\begin{Entry}{露}{21}{⾬}
  \begin{Phonetics}{露}{lou4}[][HSK 6]
    \definition{v.}{mostrar; apresentar (uma certa emoção ou olhar no rosto) | mostrar; aparentar; fazer algo visível; as pessoas podem ver}
  \end{Phonetics}
  \begin{Phonetics}{露}{lu4}[][HSK 6]
    \definition{adj.}{fora de uma casa, tenda, etc., sem cobertura}
    \definition{s.}{orvalho; gotas de água condensadas | xarope; suco de fruta; bebida destilada de flores, folhas ou frutos}
    \definition{v.}{revelar; expor; mostrar; trair}
  \end{Phonetics}
\end{Entry}

\begin{Entry}{露天}{21,4}{⾬,⼤}
  \begin{Phonetics}{露天}{lu4tian1}[][HSK 7-9]
    \definition{adj.}{ao ar livre; aberto; descoberto; desembalado}
    \definition{s.}{o ar livre; o exterior; refere-se à parte externa da casa}
  \end{Phonetics}
\end{Entry}

\begin{Entry}{露面}{21,9}{⾬,⾯}
  \begin{Phonetics}{露面}{lou4/mian4}[][HSK 7-9]
    \definition{v.+compl.}{apresentar-se; fazer-se notar; comparecer em ocasiões públicas; aparecer em público}
  \end{Phonetics}
\end{Entry}

\begin{Entry}{露珠}{21,10}{⾬,⽟}
  \begin{Phonetics}{露珠}{lu4zhu1}
    \definition{s.}{orvalho}
  \end{Phonetics}
\end{Entry}

%%%%%%%%%% 霸 %%%%%%%%%%
\subsection*{霸}\addcontentsline{loh}{figure}{霸}

\begin{Entry}{霸}{21}{⾬}
  \begin{Phonetics}{霸}{ba4}
    \definition*{s.}{Sobrenome: Ba}
    \definition{adj.}{arrogante; dominador; tirânico}
    \definition{s.}{líder dos senhores feudais; suserano | tirano; déspota; valentão; \emph{bully} | poder hegemônico; hegemonismo; hegemonia | chefe dos príncipes feudais; líder da antiga aliança feudal}
    \definition{v.}{dominar; tiranizar; governar (ocupar) pela força}
  \end{Phonetics}
\end{Entry}

\begin{Entry}{霸占}{21,5}{⾬,⼘}
  \begin{Phonetics}{霸占}{ba4zhan4}[][HSK 7-9]
    \definition{v.}{ocupar à força; apreender ilegalmente | apreender; usurpar}
  \end{Phonetics}
\end{Entry}

\begin{Entry}{霸权}{21,6}{⾬,⽊}
  \begin{Phonetics}{霸权}{ba4quan2}
    \definition{s.}{hegemonia; supremacia}
  \end{Phonetics}
\end{Entry}

%%%%% EOF %%%%%

