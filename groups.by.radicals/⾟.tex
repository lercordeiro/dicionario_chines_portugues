%%%
%%% Radical "⾟"
%%%
\section*{Radical 160: ``⾟''}\addcontentsline{toc}{section}{Radical 160: ⾟}\addcontentsline{loh}{figure}{\#\#\#\# 160: ⾟}

%%%%%%%%%% 辛 %%%%%%%%%%
\subsection*{辛}\addcontentsline{loh}{figure}{辛}

\begin{Entry}{辛}{7}{⾟}[Kangxi 160]
  \begin{Phonetics}{辛}{xin1}
    \definition*{s.}{Sobrenome: Xin}
    \definition{adj.}{quente (no sabor, etc.); pungente | difícil; trabalhoso | ponto da bússola chinesa antiga: 285° | oitavo na ordem}
    \definition{pref.}{octa-}
    \definition{s.}{sofrimento}
    \definition{s.}{oitavo dos dez caules celestiais}
  \end{Phonetics}
\end{Entry}

\begin{Entry}{辛苦}{7,8}{⾟,⾋}
  \begin{Phonetics}{辛苦}{xin1ku3}[][HSK 5]
    \definition{adj.}{difícil; trabalhoso; árduo; descreve muito trabalho, alta intensidade e pouco descanso}
    \definition{s.}{dificuldades}
    \definition{v.}{trabalhar duro; passar por grandes dificuldades; passar por dificuldades}
  \end{Phonetics}
\end{Entry}

%%%%%%%%%% 辜 %%%%%%%%%%
\subsection*{辜}\addcontentsline{loh}{figure}{辜}

\begin{Entry}{辜}{12}{⾟}
  \begin{Phonetics}{辜}{gu1}
    \definition*{s.}{Sobrenome: Gu}
    \definition{s.}{culpa; crime}
  \end{Phonetics}
\end{Entry}

\begin{Entry}{辜负}{12,6}{⾟,⾙}
  \begin{Phonetics}{辜负}{gu1fu4}[][HSK 7-9]
    \definition{v.}{desapontar; decepcionar; ser indigno de; não corresponder a}
  \end{Phonetics}
\end{Entry}

%%%%%%%%%% 辞 %%%%%%%%%%
\subsection*{辞}\addcontentsline{loh}{figure}{辞}

\begin{Entry}{辞}{13}{⾟}
  \begin{Phonetics}{辞}{ci2}[][HSK 7-9]
    \definition[首]{s.}{dicção; fraseologia | um tipo de literatura clássica chinesa; um gênero da literatura clássica | uma forma de poesia clássica}
    \definition{v.}{despedir"-se | declinar | renunciar | dispensar; demitir | fugir; evitar}
  \end{Phonetics}
\end{Entry}

\begin{Entry}{辞去}{13,5}{⾟,⼛}
  \begin{Phonetics}{辞去}{ci2qu4}[][HSK 7-9]
    \definition{v.}{desistir | renunciar}
  \end{Phonetics}
\end{Entry}

\begin{Entry}{辞呈}{13,7}{⾟,⼝}
  \begin{Phonetics}{辞呈}{ci2cheng2}[][HSK 7-9]
    \definition{s.}{renúncia (por escrito)}
  \end{Phonetics}
\end{Entry}

\begin{Entry}{辞典}{13,8}{⾟,⼋}
  \begin{Phonetics}{辞典}{ci2 dian3}[][HSK 5]
    \definition[本,部]{s.}{dicionário; coleção de termos especializados ou enciclopédicos, organizados em uma determinada ordem e explicados, para fins de referência}
    \variantof{词典}
  \end{Phonetics}
\end{Entry}

\begin{Entry}{辞退}{13,9}{⾟,⾡}
  \begin{Phonetics}{辞退}{ci2tui4}[][HSK 7-9]
    \definition{v.}{dispensar; demitir; a unidade ou instituição toma a iniciativa de encerrar a relação de trabalho com o funcionário | recusar; retornar educadamente}
  \end{Phonetics}
\end{Entry}

\begin{Entry}{辞职}{13,11}{⾟,⽿}
  \begin{Phonetics}{辞职}{ci2/zhi2}[][HSK 5]
    \definition{v.+compl.}{renunciar; deixar o cargo; entregar a renúncia; pedir para ser dispensado de suas funções}
  \end{Phonetics}
\end{Entry}

%%%%%%%%%% 辣 %%%%%%%%%%
\subsection*{辣}\addcontentsline{loh}{figure}{辣}

\begin{Entry}{辣}{14}{⾟}
  \begin{Phonetics}{辣}{la4}[][HSK 4]
    \definition{adj.}{apimentado; picante; pungente; quente | cruel; implacável; venenoso; vicioso}
    \definition{v.}{queimar; picar; formigar; ter uma irritação picante (boca, nariz ou olhos)}
  \end{Phonetics}
\end{Entry}

\begin{Entry}{辣椒}{14,12}{⾟,⽊}
  \begin{Phonetics}{辣椒}{la4jiao1}[][HSK 7-9]
    \definition[颗,把,袋,种]{s.}{pimenta; pimenta-malagueta; pimenta caiena; pimentão; pimenta vermelha}
  \end{Phonetics}
\end{Entry}

%%%%%%%%%% 辨 %%%%%%%%%%
\subsection*{辨}\addcontentsline{loh}{figure}{辨}

\begin{Entry}{辨}{16}{⾟}
  \begin{Phonetics}{辨}{bian4}
    \definition{v.}{diferenciar; distinguir; discriminar | reconhecer; distinguir; identificar; discernir}
  \end{Phonetics}
\end{Entry}

\begin{Entry}{辨认}{16,4}{⾟,⾔}
  \begin{Phonetics}{辨认}{bian4ren4}[][HSK 7-9]
    \definition{v.}{identificar; reconhecer; identificar e julgar com base em características para encontrar ou identificar um objeto}
  \end{Phonetics}
\end{Entry}

\begin{Entry}{辨别}{16,7}{⾟,⼑}
  \begin{Phonetics}{辨别}{bian4bie2}[][HSK 7-9]
    \definition{v.}{diferenciar; distinguir; discriminar; encontrar características de diferentes coisas e diferenciá"-las}
  \end{Phonetics}
\end{Entry}

%%%%%%%%%% 辩 %%%%%%%%%%
\subsection*{辩}\addcontentsline{loh}{figure}{辩}

\begin{Entry}{辩}{16}{⾟}
  \begin{Phonetics}{辩}{bian4}
    \definition{v.}{argumentar; disputar; debater}
  \end{Phonetics}
\end{Entry}

\begin{Entry}{辩论}{16,6}{⾟,⾔}
  \begin{Phonetics}{辩论}{bian4lun4}[][HSK 4]
    \definition{v.}{debater; obter um entendimento unificado ou correto, ambos os lados usam linguagem, palavras etc. para explicar seus pontos de vista, apontar os erros ou as contradições do outro lado}
  \end{Phonetics}
\end{Entry}

\begin{Entry}{辩护}{16,7}{⾟,⼿}
  \begin{Phonetics}{辩护}{bian4hu4}[][HSK 7-9]
    \definition{v.}{pleitear; defender | defender; argumentar em favor de}
  \end{Phonetics}
\end{Entry}

\begin{Entry}{辩解}{16,13}{⾟,⾓}
  \begin{Phonetics}{辩解}{bian4jie3}[][HSK 7-9]
    \definition{v.}{fornecer uma explicação; tentar se defender; explicar uma visão ou comportamento criticado; eliminar a crítica ou reduzir sua gravidade}
  \end{Phonetics}
\end{Entry}

%%%%%%%%%% 辫 %%%%%%%%%%
\subsection*{辫}\addcontentsline{loh}{figure}{辫}

\begin{Entry}{辫}{17}{⾟}
  \begin{Phonetics}{辫}{bian4}
    \definition{s.}{trança; rabo de cavalo | para coisas como uma trança}
  \end{Phonetics}
\end{Entry}

\begin{Entry}{辫子}{17,3}{⾟,⼦}
  \begin{Phonetics}{辫子}{bian4zi5}[][HSK 7-9]
    \definition[条,根,种]{s.}{trança; rabo de cavalo; uma mecha de cabelo presa reta ou trançada em seções | Metafórico: erro; deficiência; fraqueza | coisas parecidas com tranças}
  \end{Phonetics}
\end{Entry}

%%%%% EOF %%%%%

