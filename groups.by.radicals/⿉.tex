%%%
%%% Radical "⿉"
%%%
\section*{Radical 202: ``⿉''}\addcontentsline{toc}{section}{Radical 202: ⿉}\addcontentsline{loh}{figure}{\#\#\#\# 202: ⿉}

%%%%%%%%%% 黍 %%%%%%%%%%
\subsection*{黍}\addcontentsline{loh}{figure}{黍}

\begin{Entry}{黍}{12}{⿉}[Kangxi 202]
  \begin{Phonetics}{黍}{shu3}
    \definition{s.}{painço}
  \end{Phonetics}
\end{Entry}

%%%%%%%%%% 黎 %%%%%%%%%%
\subsection*{黎}\addcontentsline{loh}{figure}{黎}

\begin{Entry}{黎}{15}{⿉}
  \begin{Phonetics}{黎}{li2}
    \definition*{s.}{Etnia Li, uma das minorias nacionais da província de Hainan | Sobrenome: Li}
    \definition{adj.}{Literário: preto; escuro | Literário: numeroso}
    \definition{s.}{multidão; as massas; a população}
  \end{Phonetics}
\end{Entry}

\begin{Entry}{黎明}{15,8}{⿉,⽇}
  \begin{Phonetics}{黎明}{li2ming2}[][HSK 7-9]
    \definition[个]{s.}{amanhecer; alvorecer; quando está prestes a amanhecer ou logo após o amanhecer}
  \end{Phonetics}
\end{Entry}

%%%%%%%%%% 黏 %%%%%%%%%%
\subsection*{黏}\addcontentsline{loh}{figure}{黏}

\begin{Entry}{黏}{17}{⿉}
  \begin{Phonetics}{黏}{nian2}[][HSK 7-9]
    \definition{adj.}{pegajoso; adesivo; glutinoso}
  \end{Phonetics}
\end{Entry}

%%%%% EOF %%%%%

