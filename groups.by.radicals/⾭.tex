%%%
%%% Radical "⾭"
%%%
\section*{Radical 174: ``⾭'' (青)}\addcontentsline{toc}{section}{Radical 174: ⾭、青}\addcontentsline{loh}{figure}{\#\#\#\# 174: ⾭}

%%%%%%%%%% 青 %%%%%%%%%%
\subsection*{青}\addcontentsline{loh}{figure}{青}

\begin{Entry}{青}{8}{⾭}[Kangxi 174]
  \begin{Phonetics}{青}{qing1}[][HSK 5]
    \definition*{s.}{Província de Qinghai, abreviação de 青海 | Sobrenome: Qing}
    \definition{adj.}{azul ou verde | preto | jovens (pessoas)}
    \definition{s.}{grama verde | colheitas jovens (não maduras) | tiras de bambu verde}
  \seealsoref{青海}{qing1hai3}
  \end{Phonetics}
\end{Entry}

\begin{Entry}{青天}{8,4}{⾭,⼤}
  \begin{Phonetics}{青天}{qing1tian1}
    \definition{s.}{céu claro, limpo ou azul}
  \end{Phonetics}
\end{Entry}

\begin{Entry}{青少年}{8,4,6}{⾭,⼩,⼲}
  \begin{Phonetics}{青少年}{qing1shao4nian2}[][HSK 2]
    \definition[位,名,个,些]{s.}{adolescentes}
  \end{Phonetics}
\end{Entry}

\begin{Entry}{青玉米}{8,5,6}{⾭,⽟,⽶}
  \begin{Phonetics}{青玉米}{qing1yu4mi3}
    \definition{s.}{milho verde}
  \end{Phonetics}
\end{Entry}

\begin{Entry}{青年}{8,6}{⾭,⼲}
  \begin{Phonetics}{青年}{qing1nian2}[][HSK 2]
    \definition[个,位,名,些]{s.}{juventude; jovem; refere"-se ao período entre os 15 e os 30 anos de idade}
  \end{Phonetics}
\end{Entry}

\begin{Entry}{青年节}{8,6,5}{⾭,⼲,⾋}
  \begin{Phonetics}{青年节}{qing1nian2jie2}
    \definition*{s.}{Dia da Juventude (4 de maio)}
  \end{Phonetics}
\end{Entry}

\begin{Entry}{青春}{8,9}{⾭,⽇}
  \begin{Phonetics}{青春}{qing1chun1}[][HSK 4]
    \definition[个]{s.}{juventude; jovialidade}
  \end{Phonetics}
\end{Entry}

\begin{Entry}{青春期}{8,9,12}{⾭,⽇,⽉}
  \begin{Phonetics}{青春期}{qing1chun1qi1}[][HSK 7-9]
    \definition{s.}{puberdade; adolescência; refere"-se ao período em que os órgãos sexuais masculinos e femininos se desenvolvem rapidamente até a maturidade completa, tipicamente entre os 14 e 16 anos para os meninos e entre os 13 e 14 anos para as meninas}
  \end{Phonetics}
\end{Entry}

\begin{Entry}{青海}{8,10}{⾭,⽔}
  \begin{Phonetics}{青海}{qing1hai3}
    \definition*{s.}{Província de Qinghai}
  \end{Phonetics}
\end{Entry}

\begin{Entry}{青菜}{8,11}{⾭,⾋}
  \begin{Phonetics}{青菜}{qing1cai4}
    \definition{s.}{verduras}
  \end{Phonetics}
\end{Entry}

\begin{Entry}{青铜}{8,11}{⾭,⾦}
  \begin{Phonetics}{青铜}{qing1tong2}
    \definition{s.}{bronze (liga de cobre, 銅, e estanho, 锡)}
  \end{Phonetics}
\end{Entry}

\begin{Entry}{青椒}{8,12}{⾭,⽊}
  \begin{Phonetics}{青椒}{qing1jiao1}
    \definition{s.}{pimenta verde}
  \end{Phonetics}
\end{Entry}

\begin{Entry}{青蛙}{8,12}{⾭,⾍}
  \begin{Phonetics}{青蛙}{qing1wa1}[][HSK 7-9]
    \definition[只]{s.}{sapo}
  \end{Phonetics}
\end{Entry}

%%%%%%%%%% 靓 %%%%%%%%%%
\subsection*{靓}\addcontentsline{loh}{figure}{靓}

\begin{Entry}{靓}{12}{⾭}
  \begin{Phonetics}{靓}{jing4}
    \definition{v.}{(referindo"-se a vestimenta de alguém) ficar bonito | vestir"-se | maquiar (o rosto)}
  \end{Phonetics}
  \begin{Phonetics}{靓}{liang4}
    \definition{v.}{Literário: vestir-se bem; maquiar-se}
  \end{Phonetics}
\end{Entry}

\begin{Entry}{靓丽}{12,7}{⾭,⼀}
  \begin{Phonetics}{靓丽}{liang4li4}
    \definition{adj.}{lindo; bonito}
  \end{Phonetics}
\end{Entry}

%%%%%%%%%% 静 %%%%%%%%%%
\subsection*{静}\addcontentsline{loh}{figure}{静}

\begin{Entry}{静}{14}{⾭}
  \begin{Phonetics}{静}{jing4}[][HSK 3]
    \definition*{s.}{Sobrenome: Jing}
    \definition{adj.}{tranquilo;  sossegado; calmo; imóvel | silencioso; quieto; sem emitir nenhum som | calmo, sereno; serenidade; (interior) paz}
    \definition{v.}{acalmar-se; aquietar-se; tranquilizar (o coração)}
  \end{Phonetics}
\end{Entry}

\begin{Entry}{静止}{14,4}{⾭,⽌}
  \begin{Phonetics}{静止}{jing4zhi3}[][HSK 7-9]
    \definition{adj.}{estático; imóvel; parado; estacionário}
  \end{Phonetics}
\end{Entry}

%%%%% EOF %%%%%

