%%%
%%% Radical "⼋"
%%%
\section*{Radical 12: ``⼋'' (丷)}\addcontentsline{toc}{section}{Radical 12: ⼋,丷}\addcontentsline{loh}{figure}{\#\#\#\# 12: ⼋}

%%%%%%%%%% 八 %%%%%%%%%%
\subsection*{八}\addcontentsline{loh}{figure}{八}

\begin{Entry}{八}{2}{⼋}[Kangxi 12]
  \begin{Phonetics}{八}{ba1}[][HSK 1]
    \definition{num.}{oito; 8}
  \end{Phonetics}
\end{Entry}

\begin{Entry}{八八六}{2,2,4}{⼋,⼋,⼋}
  \begin{Phonetics}{八八六}{ba1 ba1 liu4}
    \definition{expr.}{\emph{``Bye bye!''}, em salas de bate-papo e mensagens de texto}
  \end{Phonetics}
\end{Entry}

\begin{Entry}{八卦}{2,8}{⼋,⼘}
  \begin{Phonetics}{八卦}{ba1gua4}[][HSK 7-9]
    \definition*{s.}{Oito Trigramas; na China antiga, havia um conjunto de símbolos, os oito trigramas, que são chamados de Bagua: \TrigramHeaven\  (céu, paraíso), \TrigramLake\ (lago), \TrigramFire\ (fogo), \TrigramThunder\ (trovão), \TrigramWind\ (vento), \TrigramWater\ (água), \TrigramMountain\ (montanha) e \TrigramEarth\ (terra) onde \Yang\ representa Yang (阳) e \Yin\ representa Yin (阴); mais tarde, foi usado para prever sucesso ou fracasso, sorte ou infortúnio, etc.}
    \definition[个,条]{adj.}{fofoca}
    \definition{adj.}{fofoqueiro}
  \seealsoref{阳}{yang2}
  \seealsoref{阴}{yin1}
  \end{Phonetics}
\end{Entry}

%%%%%%%%%% 公 %%%%%%%%%%
\subsection*{公}\addcontentsline{loh}{figure}{公}

\begin{Entry}{公}{4}{⼋}
  \begin{Phonetics}{公}{gong1}[][HSK 6]
    \definition*{s.}{Sobrenome: Gong}
    \definition{adj.}{público; estatal; coletivo | comum; geral | do mundo; internacional; universal; métrico | imparcial; justo; equitativo |  (de um animal) masculino}
    \definition{s.}{assuntos públicos; negócios oficiais (ou deveres) | autoridade; coletivo | duque | títulos respeitosos para homens idosos; uma saudação respeitosa | marido}
    \definition{v.}{tornar público; divulgar; abrir a todos; exibir}
  \antonymref{母}{mu3}
  \antonymref{私}{si1}
  \end{Phonetics}
\end{Entry}

\begin{Entry}{公仆}{4,4}{⼋,⼈}
  \begin{Phonetics}{公仆}{gong1pu2}[][HSK 7-9]
    \definition[个,位,名,些]{s.}{servidor público; oficial | funcionário público; pessoas que servem o público}
  \end{Phonetics}
\end{Entry}

\begin{Entry}{公元}{4,4}{⼋,⼉}
  \begin{Phonetics}{公元}{gong1yuan2}[][HSK 4]
    \definition{s.}{D.C. (Depois de~Cristo); a era cristã; um método internacionalmente aceito de registro de datas, o ano lendário do nascimento de Jesus é 1 d.C., também conhecido como o primeiro ano da Era Comum, e é denotado por D.C.}
  \seealsoref{前}{qian2}
  \end{Phonetics}
\end{Entry}

\begin{Entry}{公办}{4,4}{⼋,⼒}
  \begin{Phonetics}{公办}{gong1ban4}
    \definition{adj.}{público; estatal; administrado pelo governo}
  \end{Phonetics}
\end{Entry}

\begin{Entry}{公开}{4,4}{⼋,⼶}
  \begin{Phonetics}{公开}{gong1kai1}[][HSK 3]
    \definition{adj.}{aberto; público; não oculto; exposto ao público}
    \definition{v.}{tornar público}
  \end{Phonetics}
\end{Entry}

\begin{Entry}{公开信}{4,4,9}{⼋,⼶,⼈}
  \begin{Phonetics}{公开信}{gong1kai1xin4}[][HSK 7-9]
    \definition{s.}{carta aberta; cartas endereçadas a indivíduos ou grupos que o autor acredita serem necessárias para divulgação pública}
  \end{Phonetics}
\end{Entry}

\begin{Entry}{公斤}{4,4}{⼋,⽄}
  \begin{Phonetics}{公斤}{gong1jin1}[][HSK 2]
    \definition{clas.}{quilograma (kg)}
  \end{Phonetics}
\end{Entry}

\begin{Entry}{公认}{4,4}{⼋,⾔}
  \begin{Phonetics}{公认}{gong1ren4}[][HSK 5]
    \definition{v.}{(geralmente) reconhecer; (universalmente) aceitar}
  \end{Phonetics}
\end{Entry}

\begin{Entry}{公车}{4,4}{⼋,⾞}
  \begin{Phonetics}{公车}{gong1che1}[][HSK 7-9]
    \definition[辆]{s.}{ônibus, abreviação de 公共汽车 | carro pertencente a uma organização e usado por seus membros (carro do governo, carro de polícia, carro da empresa etc.), abreviação de 公务用车}
  \seealsoref{公共}{gong1gong4}
  \seealsoref{公共汽车}{gong1gong4 qi4che1}
  \seealsoref{公务用车}{gong1wu4yong4che1}
  \end{Phonetics}
\end{Entry}

\begin{Entry}{公主}{4,5}{⼋,⼂}
  \begin{Phonetics}{公主}{gong1zhu3}[][HSK 6]
    \definition[个,位,名,些]{s.}{princesa; a filha do monarca}
  \end{Phonetics}
\end{Entry}

\begin{Entry}{公务}{4,5}{⼋,⼒}
  \begin{Phonetics}{公务}{gong1wu4}[][HSK 7-9]
    \definition{s.}{assuntos públicos; negócios oficiais; em relação a assuntos nacionais ou coletivos}
  \end{Phonetics}
\end{Entry}

\begin{Entry}{公务用车}{4,5,5,4}{⼋,⼒,⽤,⾞}
  \begin{Phonetics}{公务用车}{gong1wu4yong4che1}
    \definition{s.}{veículos oficiais}
  \end{Phonetics}
\end{Entry}

\begin{Entry}{公务员}{4,5,7}{⼋,⼒,⼝}
  \begin{Phonetics}{公务员}{gong1wu4yuan2}[][HSK 3]
    \definition[个,位,名,些]{s.}{funcionário público; funcionário de órgãos governamentais}
  \end{Phonetics}
\end{Entry}

\begin{Entry}{公司}{4,5}{⼋,⼝}
  \begin{Phonetics}{公司}{gong1si1}[][HSK 2]
    \definition[个,家]{s.}{empresa; companhia; corporação; uma organização industrial e comercial que opera na produção de produtos, circulação de mercadorias ou certos empreendimentos de construção, etc.}
  \end{Phonetics}
\end{Entry}

\begin{Entry}{公司治理}{4,5,8,11}{⼋,⼝,⽔,⽟}
  \begin{Phonetics}{公司治理}{gong1si1zhi4li3}
    \definition{s.}{governança corporativa}
  \end{Phonetics}
\end{Entry}

\begin{Entry}{公布}{4,5}{⼋,⼱}
  \begin{Phonetics}{公布}{gong1bu4}[][HSK 3]
    \definition{v.}{(leis, decretos, comunicados e avisos de órgãos governamentais) promulgar; anunciar; publicar; tornar público; divulgar publicamente}
  \end{Phonetics}
\end{Entry}

\begin{Entry}{公平}{4,5}{⼋,⼲}
  \begin{Phonetics}{公平}{gong1ping2}[][HSK 2]
    \definition{adj.}{justo; imparcial; equitativo; equidade}
  \end{Phonetics}
\end{Entry}

\begin{Entry}{公正}{4,5}{⼋,⽌}
  \begin{Phonetics}{公正}{gong1zheng4}[][HSK 5]
    \definition{adj.}{justo; equitativo; imparcial; de mente justa; equidade e integridade sem favoritismo}
  \end{Phonetics}
\end{Entry}

\begin{Entry}{公民}{4,5}{⼋,⽒}
  \begin{Phonetics}{公民}{gong1min2}[][HSK 3]
    \definition[个,位]{s.}{cidadão; civil; pessoa que possui a nacionalidade de um país, goza dos direitos e cumpre as obrigações previstos na Constituição e nas demais leis desse país}
  \end{Phonetics}
\end{Entry}

\begin{Entry}{公用}{4,5}{⼋,⽤}
  \begin{Phonetics}{公用}{gong1yong4}[][HSK 7-9]
    \definition{adj.}{público; comunitário; para uso público, uso comum}
  \end{Phonetics}
\end{Entry}

\begin{Entry}{公用电话}{4,5,5,8}{⼋,⽤,⽥,⾔}
  \begin{Phonetics}{公用电话}{gong1yong4dian4hua4}
    \definition[部]{s.}{telefone público}
  \end{Phonetics}
\end{Entry}

\begin{Entry}{公示}{4,5}{⼋,⽰}
  \begin{Phonetics}{公示}{gong1shi4}[][HSK 7-9]
    \definition{v.}{dar a conhecer ao público e pedir opiniões}
  \end{Phonetics}
\end{Entry}

\begin{Entry}{公立}{4,5}{⼋,⽴}
  \begin{Phonetics}{公立}{gong1li4}[][HSK 7-9]
    \definition{adj.}{estabelecido e mantido pelo governo; público | financiado publicamente e administrado pelo governo; estabelecido pelo governo e operado com fundos governamentais para fornecer serviços ao público}
  \antonymref{私立}{si1li4}
  \end{Phonetics}
\end{Entry}

\begin{Entry}{公交车}{4,6,4}{⼋,⼇,⾞}
  \begin{Phonetics}{公交车}{gong1jiao1che1}[][HSK 2]
    \definition[辆]{s.}{ônibus urbano; veículo de transporte público}
  \end{Phonetics}
\end{Entry}

\begin{Entry}{公众}{4,6}{⼋,⼈}
  \begin{Phonetics}{公众}{gong1zhong4}[][HSK 6]
    \definition[对]{s.}{o público; as massas; refere"-se à maioria das pessoas na sociedade}
  \end{Phonetics}
\end{Entry}

\begin{Entry}{公共}{4,6}{⼋,⼋}
  \begin{Phonetics}{公共}{gong1gong4}[][HSK 3]
    \definition{adj.}{público; comum; comunal; comunitário; pertencente à sociedade}
    \definition[辆]{s.}{ônibus}
  \seealsoref{公车}{gong1che1}
  \seealsoref{公共汽车}{gong1gong4 qi4che1}
  \end{Phonetics}
\end{Entry}

\begin{Entry}{公共关系}{4,6,6,7}{⼋,⼋,⼋,⽷}
  \begin{Phonetics}{公共关系}{gong1gong4 guan1xi4}
    \definition{s.}{relações públicas; refere"-se à relação entre grupos, empresas ou indivíduos em atividades sociais, denominada relações públicas}
  \end{Phonetics}
\end{Entry}

\begin{Entry}{公共场所}{4,6,6,8}{⼋,⼋,⼟,⼾}
  \begin{Phonetics}{公共场所}{gong1gong4 chang3suo3}[][HSK 7-9]
    \definition{s.}{lugares públicos; locais onde o público pode ir}
  \end{Phonetics}
\end{Entry}

\begin{Entry}{公共汽车}{4,6,7,4}{⼋,⼋,⽔,⾞}
  \begin{Phonetics}{公共汽车}{gong1gong4 qi4che1}[][HSK 2]
    \definition[辆,个]{s.}{ônibus}
  \seealsoref{公车}{gong1che1}
  \seealsoref{公共}{gong1gong4}
  \end{Phonetics}
\end{Entry}

\begin{Entry}{公关}{4,6}{⼋,⼋}
  \begin{Phonetics}{公关}{gong1guan1}[][HSK 7-9]
    \definition{s.}{relações públicas, abreviação de 公共关系 | pessoa que trabalha em relações públicas}
  \seealsoref{公共关系}{gong1gong4 guan1xi4}
  \end{Phonetics}
\end{Entry}

\begin{Entry}{公安}{4,6}{⼋,⼧}
  \begin{Phonetics}{公安}{gong1'an1}[][HSK 6]
    \definition[名,位]{s.}{segurança pública; a segurança e estabilidade dos direitos dos cidadãos, da propriedade da segurança pública e da ordem social | agente de segurança pública; pessoal que mantém a segurança pública}
  \end{Phonetics}
\end{Entry}

\begin{Entry}{公安局}{4,6,7}{⼋,⼧,⼫}
  \begin{Phonetics}{公安局}{gong1'an1ju2}[][HSK 7-9]
    \definition{s.}{departamento de polícia; departamento de segurança pública; o departamento responsável pelo trabalho de segurança pública do Governo Popular}
  \end{Phonetics}
\end{Entry}

\begin{Entry}{公式}{4,6}{⼋,⼷}
  \begin{Phonetics}{公式}{gong1shi4}[][HSK 5]
    \definition[个,些,种]{s.}{fórmula; expressão}
  \end{Phonetics}
\end{Entry}

\begin{Entry}{公约}{4,6}{⼋,⽷}
  \begin{Phonetics}{公约}{gong1yue1}[][HSK 7-9]
    \definition[项,条]{s.}{convenção; pacto; aliança; tratado | compromisso conjunto; regulamentos acordados coletivamente dentro de uma organização | regulamentos acordados coletivamente dentro de uma unidade de trabalho}
  \end{Phonetics}
\end{Entry}

\begin{Entry}{公克}{4,7}{⼋,⼗}
  \begin{Phonetics}{公克}{gong1ke4}
    \definition{s.}{grama (medida de peso)}
  \end{Phonetics}
\end{Entry}

\begin{Entry}{公告}{4,7}{⼋,⼝}
  \begin{Phonetics}{公告}{gong1gao4}[][HSK 5]
    \definition[张,份,项]{s.}{anúncio; notificação de assuntos importantes ao público em geral pelo governo ou por um órgão importante}
    \definition{v.}{anunciar; o governo ou órgão governamental informa publicamente às pessoas algo importante}
  \end{Phonetics}
\end{Entry}

\begin{Entry}{公园}{4,7}{⼋,⼞}
  \begin{Phonetics}{公园}{gong1yuan2}[][HSK 2]
    \definition[个,座]{s.}{parque; jardim público; os jardins abertos ao público para passeios e descanso geralmente ficam nas cidades, têm muitas flores, árvores e, em alguns casos, lagos}
  \end{Phonetics}
\end{Entry}

\begin{Entry}{公证}{4,7}{⼋,⾔}
  \begin{Phonetics}{公证}{gong1zheng4}[][HSK 7-9]
    \definition{adj.}{autenticado em cartório}
    \definition{s.}{reconhecimento de firma}
    \definition{v.}{autenticar}
  \end{Phonetics}
\end{Entry}

\begin{Entry}{公里}{4,7}{⼋,⾥}
  \begin{Phonetics}{公里}{gong1li3}[][HSK 2]
    \definition{s.}{quilômetro (km)}
  \end{Phonetics}
\end{Entry}

\begin{Entry}{公鸡}{4,7}{⼋,⿃}
  \begin{Phonetics}{公鸡}{gong1ji1}[][HSK 6]
    \definition{s.}{galo; frango macho}
  \end{Phonetics}
\end{Entry}

\begin{Entry}{公事}{4,8}{⼋,⼅}
  \begin{Phonetics}{公事}{gong1shi4}[][HSK 7-9]
    \definition{s.}{assuntos públicos; negócios oficiais (ou deveres) | Coloquial: documento oficial}
  \antonymref{私事}{si1shi4}
  \end{Phonetics}
\end{Entry}

\begin{Entry}{公函}{4,8}{⼋,⼐}
  \begin{Phonetics}{公函}{gong1han2}[][HSK 7-9]
    \definition{s.}{carta oficial; correspondência oficial entre departamentos paralelos e não relacionados}
  \antonymref{便函}{bian4han2}
  \antonymref{私函}{si1han2}
  \end{Phonetics}
\end{Entry}

\begin{Entry}{公顷}{4,8}{⼋,⾴}
  \begin{Phonetics}{公顷}{gong1qing3}[][HSK 7-9]
    \definition{clas.}{hectare; é uma unidade de área terrestre no sistema métrico;  equivale a 10.000 metros quadrados, ou 15 市亩}
  \seealsoref{市亩}{shi4mu3}
  \end{Phonetics}
\end{Entry}

\begin{Entry}{公费}{4,9}{⼋,⾙}
  \begin{Phonetics}{公费}{gong1fei4}[][HSK 7-9]
    \definition{s.}{despesa pública; despesas fornecidas pelo estado ou grupo}
  \end{Phonetics}
\end{Entry}

\begin{Entry}{公益}{4,10}{⼋,⽫}
  \begin{Phonetics}{公益}{gong1yi4}[][HSK 7-9]
    \definition{s.}{bem público; comunitário; bem"-estar; interesse público (referindo"-se principalmente ao bem"-estar público, como saúde e assistência)}
  \end{Phonetics}
\end{Entry}

\begin{Entry}{公益性}{4,10,8}{⼋,⽫,⼼}
  \begin{Phonetics}{公益性}{gong1yi4xing4}[][HSK 7-9]
    \definition{s.}{bem"-estar público}
  \end{Phonetics}
\end{Entry}

\begin{Entry}{公积金}{4,10,8}{⼋,⽲,⾦}
  \begin{Phonetics}{公积金}{gong1ji1jin1}[][HSK 7-9]
    \definition{s.}{fundo de acumulação (comum); fundo de reserva pública; fundo de reserva comum | fundo acumulado | reservas oficiais}
  \end{Phonetics}
\end{Entry}

\begin{Entry}{公职}{4,11}{⼋,⽿}
  \begin{Phonetics}{公职}{gong1zhi2}[][HSK 7-9]
    \definition{s.}{cargo público; emprego público; cargos ou patentes oficiais}
  \end{Phonetics}
\end{Entry}

\begin{Entry}{公募}{4,12}{⼋,⼒}
  \begin{Phonetics}{公募}{gong1mu4}
    \definition{s.}{financiamento público; arrecadação pública de fundos (investimento)}
  \end{Phonetics}
\end{Entry}

\begin{Entry}{公寓}{4,12}{⼋,⼧}
  \begin{Phonetics}{公寓}{gong1yu4}[][HSK 7-9]
    \definition[套,座,栋,间]{s.}{apartamentos; moradias; colônias de férias; prédio de apartamentos; construções que podem acomodar várias famílias}
  \end{Phonetics}
\end{Entry}

\begin{Entry}{公款}{4,12}{⼋,⽋}
  \begin{Phonetics}{公款}{gong1kuan3}[][HSK 7-9]
    \definition{s.}{dinheiro público (ou fundo); despesa do governo | fundos públicos}
  \end{Phonetics}
\end{Entry}

\begin{Entry}{公然}{4,12}{⼋,⽕}
  \begin{Phonetics}{公然}{gong1ran2}[][HSK 7-9]
    \definition{adv.}{Pejorativo: abertamente; publicamente; sem disfarces; descaradamente; flagrantemente}
  \antonymref{暗自}{an4zi4}
  \end{Phonetics}
\end{Entry}

\begin{Entry}{公道}{4,12}{⼋,⾡}
  \begin{Phonetics}{公道}{gong1dao4}
    \definition{s.}{justiça; o princípio da justiça}
  \end{Phonetics}
  \begin{Phonetics}{公道}{gong1dao5}[][HSK 7-9]
    \definition{adj.}{equitativo | justo}
  \end{Phonetics}
\end{Entry}

\begin{Entry}{公墓}{4,13}{⼋,⼟}
  \begin{Phonetics}{公墓}{gong1mu4}[][HSK 7-9]
    \definition[顿]{s.}{cemitério; cemitério público | Arcaico: túmulos ou cemitérios reais ou aristocráticos | parque memorial}
  \end{Phonetics}
\end{Entry}

\begin{Entry}{公路}{4,13}{⼋,⾜}
  \begin{Phonetics}{公路}{gong1lu4}[][HSK 2]
    \definition[条,段]{s.}{rodovia; via de acesso; via de tráfego; estrada; estrada principal;}
  \end{Phonetics}
\end{Entry}

%%%%%%%%%% 六 %%%%%%%%%%
\subsection*{六}\addcontentsline{loh}{figure}{六}

\begin{Entry}{六}{4}{⼋}
  \begin{Phonetics}{六}{liu4}[][HSK 1]
    \definition*{s.}{Sobrenome: Liu}
    \definition{num.}{seis; 6}
    \definition{s.}{símbolo musical utilizado na partitura da música tradicional chinesa, representando o primeiro grau da escala musical, equivalente ao ``5'' da notação musical simplificada}
  \end{Phonetics}
\end{Entry}

%%%%%%%%%% 兰 %%%%%%%%%%
\subsection*{兰}\addcontentsline{loh}{figure}{兰}

\begin{Entry}{兰}{5}{⼋}
  \begin{Phonetics}{兰}{lan2}
    \definition*{s.}{Sobrenome: Lan}
    \definition{s.}{orquídea | lírio magnólia}
  \end{Phonetics}
\end{Entry}

\begin{Entry}{兰州}{5,6}{⼋,⼮}
  \begin{Phonetics}{兰州}{lan2zhou1}
    \definition*{s.}{Lanzhou. capital da província de Gansu, 甘肃}
  \seealsoref{甘肃}{gan1su4}
  \end{Phonetics}
\end{Entry}

\begin{Entry}{兰花}{5,7}{⼋,⾋}
  \begin{Phonetics}{兰花}{lan2hua1}
    \definition{s.}{orquídea}
  \end{Phonetics}
\end{Entry}

%%%%%%%%%% 共 %%%%%%%%%%
\subsection*{共}\addcontentsline{loh}{figure}{共}

\begin{Entry}{共}{6}{⼋}
  \begin{Phonetics}{共}{gong4}[][HSK 4]
    \definition*{s.}{Partido Comunista, abreviação de 共产党 | Sobrenome: Gong}
    \definition{adj.}{conjunto; mútuo; geral; comum; o mesmo para todos}
    \definition{adv.}{juntos; juntamente; conjuntamente | em sua totalidade; em todos}
    \definition{v.}{compartilhar com; empreender ou realizar em conjunto}
  \seealsoref{共产党}{gong4chan3dang3}
  \end{Phonetics}
\end{Entry}

\begin{Entry}{共计}{6,4}{⼋,⾔}
  \begin{Phonetics}{共计}{gong4ji4}[][HSK 5]
    \definition{s.}{total; total geral; agregado; montante}
    \definition{v.}{contar até; somar até; totalizar}
  \end{Phonetics}
\end{Entry}

\begin{Entry}{共产}{6,6}{⼋,⼇}
  \begin{Phonetics}{共产}{gong4chan3}
    \definition{adj.}{comunista}
    \definition{s.}{comunismo}
  \end{Phonetics}
\end{Entry}

\begin{Entry}{共产主义}{6,6,5,3}{⼋,⼇,⼂,⼂}
  \begin{Phonetics}{共产主义}{gong4chan3 zhu3yi4}
    \definition*{s.}{Comunismo}
  \end{Phonetics}
\end{Entry}

\begin{Entry}{共产党}{6,6,10}{⼋,⼇,⼉}
  \begin{Phonetics}{共产党}{gong4chan3dang3}
    \definition*{s.}{Partido Comunista}
  \end{Phonetics}
\end{Entry}

\begin{Entry}{共同}{6,6}{⼋,⼝}
  \begin{Phonetics}{共同}{gong4tong2}[][HSK 3]
    \definition{adj.}{comum; compartilhado; colaborativo; todos têm}
    \definition{adv.}{juntos; conjuntamente; todos juntos (fazemos)}
  \end{Phonetics}
\end{Entry}

\begin{Entry}{共同体}{6,6,7}{⼋,⼝,⼈}
  \begin{Phonetics}{共同体}{gong4tong2ti3}[][HSK 7-9]
    \definition[个]{s.}{comunidade}[欧洲经济共同体===Comunidade Econômica Europeia]
  \end{Phonetics}
\end{Entry}

\begin{Entry}{共有}{6,6}{⼋,⽉}
  \begin{Phonetics}{共有}{gong4you3}[][HSK 3]
    \definition{v.}{compartilhar; possuir (por todos); possuir ou desfrutar em conjunto}
  \end{Phonetics}
\end{Entry}

\begin{Entry}{共时}{6,7}{⼋,⽇}
  \begin{Phonetics}{共时}{gong4shi2}
    \definition{adj.}{sincrônico; simultâneo}
  \antonymref{历时}{li4shi2}
  \end{Phonetics}
\end{Entry}

\begin{Entry}{共识}{6,7}{⼋,⾔}
  \begin{Phonetics}{共识}{gong4shi2}[][HSK 7-9]
    \definition{s.}{consenso; entendimento comum}
  \end{Phonetics}
\end{Entry}

\begin{Entry}{共享}{6,8}{⼋,⼇}
  \begin{Phonetics}{共享}{gong4xiang3}[][HSK 5]
    \definition{v.}{compartilhar; desfrutar juntos; aproveitar as coisas boas juntos}
  \end{Phonetics}
\end{Entry}

\begin{Entry}{共性}{6,8}{⼋,⼼}
  \begin{Phonetics}{共性}{gong4xing4}[][HSK 7-9]
    \definition{s.}{caráter geral (comum); natureza comum; generalidade; semelhança; universalidade}
  \end{Phonetics}
\end{Entry}

\begin{Entry}{共鸣}{6,8}{⼋,⿃}
  \begin{Phonetics}{共鸣}{gong4ming2}[][HSK 7-9]
    \definition{s.}{ressonância; fenômeno que ocorre quando um objeto ressoa, por exemplo, quando dois diapasões com a mesma frequência são colocados próximos um do outro, quando um vibra e emite um som, o outro também emite um som | resposta simpática; uma metáfora para ter as mesmas emoções que outra pessoa}
  \end{Phonetics}
\end{Entry}

%%%%%%%%%% 兲 %%%%%%%%%%
\subsection*{兲}\addcontentsline{loh}{figure}{兲}

\begin{Entry}{兲}{6}{⼋}
  \begin{Phonetics}{兲}{tian1}
    \variantof{天}
  \end{Phonetics}
\end{Entry}

%%%%%%%%%% 关 %%%%%%%%%%
\subsection*{关}\addcontentsline{loh}{figure}{关}

\begin{Entry}{关}{6}{⼋}
  \begin{Phonetics}{关}{guan1}[][HSK 1,4]
    \definition*{s.}{Sobrenome: Guan}
    \definition{s.}{passagem; ponto de controle | alfândega; escritórios de cobrança de impostos para exportação e importação de mercadorias | ponto de inflexão ou barreira; ponto de virada ou dificuldade | momento crítico; mecanismo}
    \definition{v.}{fechar; encerrar; amarrar algo | fechar; trancar | encerrar; sair do mercado; falir | conceder ou sacar o pagamento de um salário | desligar | envolver; preocupar"-se; conectar"-se}
  \end{Phonetics}
\end{Entry}

\begin{Entry}{关上}{6,3}{⼋,⼀}
  \begin{Phonetics}{关上}{guan1shang4}[][HSK 1]
    \definition{v.}{fechar (uma porta); fechar um objeto | desligar (luz, equipamento elétrico etc.); parar ou encerrar (uma atividade, situação, etc.)}
  \end{Phonetics}
\end{Entry}

\begin{Entry}{关于}{6,3}{⼋,⼆}
  \begin{Phonetics}{关于}{guan1yu2}[][HSK 4]
    \definition{prep.}{sobre; relativo a; pertencente a; uma questão de; com relação a}
  \end{Phonetics}
\end{Entry}

\begin{Entry}{关心}{6,4}{⼋,⼼}
  \begin{Phonetics}{关心}{guan1xin1}[][HSK 2]
    \definition{v.}{cuidar; preocupar"-se com; manifestar interesse por; demonstrar solicitude por; (colocar uma pessoa ou coisa) sempre no coração; valorizar e cuidar}
  \end{Phonetics}
\end{Entry}

\begin{Entry}{关头}{6,5}{⼋,⼤}
  \begin{Phonetics}{关头}{guan1tou2}[][HSK 7-9]
    \definition{s.}{conjuntura; momento; um momento decisivo ou ponto de virada}
  \end{Phonetics}
\end{Entry}

\begin{Entry}{关节}{6,5}{⼋,⾋}
  \begin{Phonetics}{关节}{guan1jie2}[][HSK 7-9]
    \definition{s.}{articulação; as partes onde os ossos se conectam e que possibilitam o movimento | suborno; relacionamentos que podem ajudar as pessoas a obter benefícios por meios impróprios | elo (ou ponto) chave (ou crucial)}
  \end{Phonetics}
\end{Entry}

\begin{Entry}{关机}{6,6}{⼋,⽊}
  \begin{Phonetics}{关机}{guan1 ji1}[][HSK 2]
    \definition{v.}{encerrar; terminar; refere"-se especificamente à conclusão das filmagens de um filme ou série de TV | desligar; desligar a fonte de alimentação; parar o funcionamento da máquina}
  \end{Phonetics}
\end{Entry}

\begin{Entry}{关闭}{6,6}{⼋,⾨}
  \begin{Phonetics}{关闭}{guan1bi4}[][HSK 4]
    \definition{v.}{fechar | (empresa) falir}
  \end{Phonetics}
\end{Entry}

\begin{Entry}{关张}{6,7}{⼋,⼸}
  \begin{Phonetics}{关张}{guan1zhang1}
    \definition{v.}{Dialeto: (uma loja) fechar as portas; falir}
  \end{Phonetics}
\end{Entry}

\begin{Entry}{关怀}{6,7}{⼋,⼼}
  \begin{Phonetics}{关怀}{guan1huai2}[][HSK 5]
    \definition{v.}{mostrar cuidado amoroso por; mostrar solicitude por; cuidar, amar, apoiar ou ajudar os fracos ou grupos em dificuldade | geralmente usado para superiores para subordinados, anciãos para juniores ou organizações para indivíduos}
  \end{Phonetics}
\end{Entry}

\begin{Entry}{关系}{6,7}{⼋,⽷}
  \begin{Phonetics}{关系}{guan1xi5}[][HSK 3]
    \definition[个,种]{s.}{relações; conexões; relacionamento; a interligação entre pessoas ou coisas | consequência; impacto; significado a influência ou importância de algo; algo digno de nota (geralmente usado com 没有, 有). | causa; razão (geralmente usado com 由于 ou 因为); refere"-se genericamente a causas, condições, etc. | credenciais que mostram filiação a uma organização; documento que comprova a existência de algum tipo de relação organizacional}
    \definition{v.}{preocupar; afetar; ter influência sobre; ter a ver com}
  \seealsoref{没有}{mei2you5}
  \seealsoref{因为}{yin1wei5}
  \seealsoref{由于}{you2yu2}
  \seealsoref{有}{you3}
  \end{Phonetics}
\end{Entry}

\begin{Entry}{关注}{6,8}{⼋,⽔}
  \begin{Phonetics}{关注}{guan1zhu4}[][HSK 3]
    \definition{v.}{prestar atenção em; seguir algo de perto; seguir (nas redes sociais)}
  \end{Phonetics}
\end{Entry}

\begin{Entry}{关爱}{6,10}{⼋,⽖}
  \begin{Phonetics}{关爱}{guan1'ai4}[][HSK 6]
    \definition{v.}{cuidar; cuidar e amar}
  \end{Phonetics}
\end{Entry}

\begin{Entry}{关掉}{6,11}{⼋,⼿}
  \begin{Phonetics}{关掉}{guan1diao4}[][HSK 7-9]
    \definition{v.}{desligar}
  \end{Phonetics}
\end{Entry}

\begin{Entry}{关税}{6,12}{⼋,⽲}
  \begin{Phonetics}{关税}{guan1shui4}[][HSK 7-9]
    \definition{s.}{tarifa; taxa aduaneira; impostos cobrados pelo estado sobre mercadorias importadas e exportadas}
  \end{Phonetics}
\end{Entry}

\begin{Entry}{关联}{6,12}{⼋,⽿}
  \begin{Phonetics}{关联}{guan1lian2}[][HSK 6]
    \definition{s.}{conexão; inter"-relação; a conexão entre as coisas}
    \definition{v.}{estar relacionado; estar conectado; as coisas estão envolvidas e influenciam umas às outras}
  \end{Phonetics}
\end{Entry}

\begin{Entry}{关照}{6,13}{⼋,⽕}
  \begin{Phonetics}{关照}{guan1zhao4}[][HSK 7-9]
    \definition{v.}{cuidar de; ficar de olho em; preocupar"-se e cuidar de alguém e tomar a iniciativa de ajudar quando perceber que essa pessoa está com problemas | contar; notificar de boca em boca; notificação verbal para que as pessoas saibam ou se lembrem de algo}
  \end{Phonetics}
\end{Entry}

\begin{Entry}{关键}{6,13}{⼋,⾦}
  \begin{Phonetics}{关键}{guan1jian4}[][HSK 5]
    \definition{adj.}{crucial; decisivo; importante; que pode determinar o curso e o resultado dos eventos}
    \definition[个,点,些]{s.}{chave; ponto crucial; aspectos ou condições mais importantes que determinam o desenvolvimento e o resultado de algo}
  \end{Phonetics}
\end{Entry}

%%%%%%%%%% 兴 %%%%%%%%%%
\subsection*{兴}\addcontentsline{loh}{figure}{兴}

\begin{Entry}{兴}{6}{⼋}
  \begin{Phonetics}{兴}{xing1}
    \definition*{s.}{Sobrenome: Xing}
    \definition{adj.}{próspero; florescente}
    \definition{adv.}{Dialeto: talvez}
    \definition{v.}{ascender; prosperar; prevalecer; tornar"-se popular | promover; encorajar; fazer prevalecer | começar; iniciar; lançar; mobilizar | erguer"-se; levantar"-se | (usualmente no negativo) permitir; deixar}
  \end{Phonetics}
  \begin{Phonetics}{兴}{xing4}
    \definition{s.}{sentimento ou desejo de fazer algo | interesse em algo | excitação}
  \end{Phonetics}
\end{Entry}

\begin{Entry}{兴奋}{6,8}{⼋,⼤}
  \begin{Phonetics}{兴奋}{xing1fen4}[][HSK 4]
    \definition{adj.}{animado; excitante; empolgante;}
    \definition{s.}{excitação; empolgação}
    \definition{v.}{excitar; intoxicar}
  \end{Phonetics}
\end{Entry}

\begin{Entry}{兴旺}{6,8}{⼋,⽇}
  \begin{Phonetics}{兴旺}{xing1wang4}[][HSK 6]
    \definition{adj.}{próspero; propício; favorável; auspicioso}
  \end{Phonetics}
\end{Entry}

\begin{Entry}{兴趣}{6,15}{⼋,⾛}
  \begin{Phonetics}{兴趣}{xing4qu4}[][HSK 4]
    \definition[个,种,点,股,份]{s.}{interesse (desejo de conhecer sobre alguma coisa ou coisa no qual está interessado) | \emph{hobby}}
  \end{Phonetics}
\end{Entry}

%%%%%%%%%% 兵 %%%%%%%%%%
\subsection*{兵}\addcontentsline{loh}{figure}{兵}

\begin{Entry}{兵}{7}{⼋}
  \begin{Phonetics}{兵}{bing1}[][HSK 4]
    \definition[个,种]{s.}{armas; armamentos | soldado; pessoal militar | exército; tropas | soldado raso; membro mais jovem do exército | assuntos militares (estratégia) | peão, uma das peças do xadrez chinês}
  \end{Phonetics}
\end{Entry}

\begin{Entry}{兵器}{7,16}{⼋,⼝}
  \begin{Phonetics}{兵器}{bing1qi4}
    \definition{s.}{armas | armamento}
  \end{Phonetics}
\end{Entry}

%%%%%%%%%% 其 %%%%%%%%%%
\subsection*{其}\addcontentsline{loh}{figure}{其}

\begin{Entry}{其}{8}{⼋}
  \begin{Phonetics}{其}{qi2}[][HSK 5]
    \definition*{s.}{Sobrenome: Qi}
    \definition{adv.}{fazer uma suposição ou uma réplica | expressar comando, ordem}
    \definition{pron.}{dele (dela, deles, delas) | ele, ela, isso, eles; elas | isso; tal | isso (referindo"-se a nenhuma pessoa ou coisa específica)}
    \definition{suf.}{sufixo de palavra, anexado ao advérbio}
  \end{Phonetics}
\end{Entry}

\begin{Entry}{其中}{8,4}{⼋,⼁}
  \begin{Phonetics}{其中}{qi2zhong1}[][HSK 2]
    \definition{pron.}{dentro; entre (os quais, eles, etc.); em (o qual, ele, etc.); nas pessoas ou coisas mencionadas anteriormente}
  \end{Phonetics}
\end{Entry}

\begin{Entry}{其他}{8,5}{⼋,⼈}
  \begin{Phonetics}{其他}{qi2ta1}[][HSK 2]
    \definition{pron.}{outra pessoa/outra coisa | outras coisas; outras pessoas; em substituição de outras pessoas ou coisas}
  \end{Phonetics}
\end{Entry}

\begin{Entry}{其后}{8,6}{⼋,⼝}
  \begin{Phonetics}{其后}{qi2hou4}[][HSK 7-9]
    \definition{adv.}{mais tarde; depois; posteriormente | depois disso | próximo}
  \end{Phonetics}
\end{Entry}

\begin{Entry}{其次}{8,6}{⼋,⽋}
  \begin{Phonetics}{其次}{qi2ci4}[][HSK 3]
    \definition{adj.}{secundário}
    \definition{conj.}{próximo; então; em segundo lugar; mais tarde na ordem}
  \end{Phonetics}
\end{Entry}

\begin{Entry}{其余}{8,7}{⼋,⼈}
  \begin{Phonetics}{其余}{qi2yu2}[][HSK 4]
    \definition{pron.}{o resto; os outros; o restante}
  \end{Phonetics}
\end{Entry}

\begin{Entry}{其间}{8,7}{⼋,⾨}
  \begin{Phonetics}{其间}{qi2jian1}[][HSK 7-9]
    \definition{s.}{nele; deles; entre eles; no meio | durante este (ou aquele) período; dentro de um determinado período de tempo}
  \end{Phonetics}
\end{Entry}

\begin{Entry}{其实}{8,8}{⼋,⼧}
  \begin{Phonetics}{其实}{qi2shi2}[][HSK 3]
    \definition{adv.}{na verdade; na realidade; a primeira parte é a situação aparente, e 其实 é usado para introduzir a situação real}
  \end{Phonetics}
\end{Entry}

%%%%%%%%%% 具 %%%%%%%%%%
\subsection*{具}\addcontentsline{loh}{figure}{具}

\begin{Entry}{具}{8}{⼋}
  \begin{Phonetics}{具}{ju4}
    \definition*{s.}{Sobrenome: Ju}
    \definition{clas.}{(literário) usado para caixões, cadáveres e certos objetos}
    \definition{s.}{utensílio; ferramenta; implemento | capacidade; habilidade}
    \definition{v.}{possuir; ter | fornecer; prover | declarar; enumerar}
  \end{Phonetics}
\end{Entry}

\begin{Entry}{具有}{8,6}{⼋,⽉}
  \begin{Phonetics}{具有}{ju4you3}[][HSK 3]
    \definition{v.}{ter; possuir; ser provido de}
  \end{Phonetics}
\end{Entry}

\begin{Entry}{具体}{8,7}{⼋,⼈}
  \begin{Phonetics}{具体}{ju4ti3}[][HSK 3]
    \definition{adj.}{específico; particular | concreto; específico; mais detalhado; muito detalhado; muito claro | concreto; real; não é abstrato, tem uma forma definida; pode ser visto ou sentido}
    \definition{v.}{incorporar; objetivar; combinar teorias, princípios, padrões, etc. com pessoas ou coisas específicas}
  \end{Phonetics}
\end{Entry}

\begin{Entry}{具备}{8,8}{⼋,⼡}
  \begin{Phonetics}{具备}{ju4bei4}[][HSK 4]
    \definition{v.}{ter; possuir; ser provido de}
  \end{Phonetics}
\end{Entry}

%%%%%%%%%% 典 %%%%%%%%%%
\subsection*{典}\addcontentsline{loh}{figure}{典}

\begin{Entry}{典}{8}{⼋}
  \begin{Phonetics}{典}{dian3}
    \definition{s.}{lei; cânone; padrão; sistema; regulamentos | trabalho padrão de bolsa de estudos; livros que podem servir como padrões ou especificações | alusão; citação literária | cerimônia; uma grande cerimônia (nos tempos antigos, a etiqueta era um dos sistemas importantes do estado) | modelo; normas; regras}
    \definition{v.}{estar no comando de | hipotecar; usar imóveis ou casas como garantia ao pedir dinheiro emprestado}
  \end{Phonetics}
\end{Entry}

\begin{Entry}{典礼}{8,5}{⼋,⽰}
  \begin{Phonetics}{典礼}{dian3li3}[][HSK 5]
    \definition[个,次,场]{s.}{cerimônia; celebração; comemoração}
  \end{Phonetics}
\end{Entry}

\begin{Entry}{典型}{8,9}{⼋,⼟}
  \begin{Phonetics}{典型}{dian3xing2}[][HSK 4]
    \definition{adj.}{típico; representativo}
    \definition[个,种]{s.}{modelo; caso típico; indivíduo ou evento representativo | personagens típicos; personalidades modelo (em obras literárias); personagens na literatura e na arte que refletem a natureza de uma determinada sociedade e têm uma personalidade distinta}
  \end{Phonetics}
\end{Entry}

\begin{Entry}{典范}{8,9}{⼋,⾋}
  \begin{Phonetics}{典范}{dian3fan4}[][HSK 7-9]
    \definition{s.}{modelo; exemplo; paradigma; uma pessoa ou coisa que pode ser usada como padrão para aprendizagem ou emulação}
  \end{Phonetics}
\end{Entry}

%%%%%%%%%% 养 %%%%%%%%%%
\subsection*{养}\addcontentsline{loh}{figure}{养}

\begin{Entry}{养}{9}{⼋}
  \begin{Phonetics}{养}{yang3}[][HSK 2]
    \definition*{s.}{Sobrenome: Yang}
    \definition{adj.}{adotivo; órfão; adotado; não biológico}
    \definition{s.}{qualidade; (caráter moral, desempenho acadêmico, etc.) boas qualidades}
    \definition{v.}{apoiar; prover; fornecer dinheiro e materiais necessários para viver | aumentar; manter; crescer; alimentar os animais e cuidar de suas vidas para que possam crescer | dar à luz | formar; adquirir; cultivar | descansar; curar; convalescer; recuperar a saúde | manter; manter em bom estado | deixar (o cabelo) crescer | ajudar; apoiar | cultivar (plantações ou flores)}
  \end{Phonetics}
\end{Entry}

\begin{Entry}{养分}{9,4}{⼋,⼑}
  \begin{Phonetics}{养分}{yang3fen4}
    \definition{s.}{nutriente}
  \end{Phonetics}
\end{Entry}

\begin{Entry}{养成}{9,6}{⼋,⼽}
  \begin{Phonetics}{养成}{yang3cheng2}[][HSK 4]
    \definition{v.}{cultivar; desenvolver; cultivar para formar; nutrir para crescer}
  \end{Phonetics}
\end{Entry}

\begin{Entry}{养老}{9,6}{⼋,⽼}
  \begin{Phonetics}{养老}{yang3 lao3}[][HSK 6]
    \definition{v.}{prover assistência aos idosos (geralmente os pais) | viver a vida na aposentadoria; refere"-se ao idoso que descansa em casa}
  \end{Phonetics}
\end{Entry}

\begin{Entry}{养料}{9,10}{⼋,⽃}
  \begin{Phonetics}{养料}{yang3liao4}
    \definition{s.}{nutrição}
  \end{Phonetics}
\end{Entry}

%%%%%%%%%% 兼 %%%%%%%%%%
\subsection*{兼}\addcontentsline{loh}{figure}{兼}

\begin{Entry}{兼}{10}{⼋}
  \begin{Phonetics}{兼}{jian1}[][HSK 7-9]
    \definition*{s.}{Sobrenome: Jian}
    \definition{adj.}{duplo; dobrado; duplicado | simultâneo; concomitante}
    \definition{adv.}{simultaneamente; concomitivamente; envolve várias coisas ao mesmo tempo.}
    \definition{v.}{ocupar um cargo simultâneo | ter dois ou mais empregos simultaneamente; fazer várias coisas ao mesmo tempo ou possuir várias coisas | Literário: reunir; unir em um só; anexar}
  \end{Phonetics}
\end{Entry}

\begin{Entry}{兼任}{10,6}{⼋,⼈}
  \begin{Phonetics}{兼任}{jian1ren4}[][HSK 7-9]
    \definition{v.}{ocupar um cargo simultâneo; ter vários empregos ao mesmo tempo | realizar algo em tempo parcial; trabalhar em tempo parcial}
  \end{Phonetics}
\end{Entry}

\begin{Entry}{兼容}{10,10}{⼋,⼧}
  \begin{Phonetics}{兼容}{jian1rong2}[][HSK 7-9]
    \definition{v.}{abranger a todos; ser compatível; aceitar e acomodar simultaneamente coisas ou aspectos diferentes.}
  \end{Phonetics}
\end{Entry}

\begin{Entry}{兼顾}{10,10}{⼋,⾴}
  \begin{Phonetics}{兼顾}{jian1gu4}[][HSK 7-9]
    \definition{v.}{levar em consideração duas ou mais coisas; dar atenção a duas ou mais coisas}
  \end{Phonetics}
\end{Entry}

\begin{Entry}{兼职}{10,11}{⼋,⽿}
  \begin{Phonetics}{兼职}{jian1zhi2}[][HSK 7-9]
    \definition[份]{pron.}{vaga simultânea; emprego de meio período; cargos ocupados fora da função principal de emprego}
    \definition{v.}{ocupar dois ou mais cargos simultaneamente; exercer outras funções além do trabalho principal}
  \end{Phonetics}
\end{Entry}

%%%%%%%%%% 兽 %%%%%%%%%%
\subsection*{兽}\addcontentsline{loh}{figure}{兽}

\begin{Entry}{兽}{11}{⼋}
  \begin{Phonetics}{兽}{shou4}
    \definition{adj.}{bestial; brutal}
    \definition{s.}{besta; animal}
  \end{Phonetics}
\end{Entry}

\begin{Entry}{兽力车}{11,2,4}{⼋,⼒,⾞}
  \begin{Phonetics}{兽力车}{shou4li4che1}
    \definition{s.}{veículo puxado por animais | carruagem; carroça}
  \antonymref{人力车}{ren2li4che1}
  \end{Phonetics}
\end{Entry}

\begin{Entry}{兽行}{11,6}{⼋,⾏}
  \begin{Phonetics}{兽行}{shou4xing2}
    \definition{s.}{ato brutal; brutalidade | bestialidade}
  \end{Phonetics}
\end{Entry}

%%%%% EOF %%%%%

