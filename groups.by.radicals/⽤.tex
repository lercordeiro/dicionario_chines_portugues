%%%
%%% Radical "⽤"
%%%
\section*{Radical 101: ``⽤''}\addcontentsline{toc}{section}{Radical 101: ⽤}\addcontentsline{loh}{figure}{\#\#\#\# 101: ⽤}

%%%%%%%%%% 用 %%%%%%%%%%
\subsection*{用}\addcontentsline{loh}{figure}{用}

\begin{Entry}{用}{5}{⽤}[Kangxi 101]
  \begin{Phonetics}{用}{yong4}[][HSK 1]
    \definition*{s.}{Sobrenome: Yong}
    \definition{conj.}{portanto; por isso; assim sendo; razões para a introdução, equivalentes a 因}
    \definition{prep.}{com; ação de introduzir ferramentas, meios, etc. utilizados ou empregados}
    \definition{s.}{despesas; gastos; custos | uso; utilidade; eficácia}
    \definition{v.}{usar; aplicar; empregar | necessitar (normalmente na forma negativa) | respeitosamente: comer; beber}
  \seealsoref{因}{yin1}
  \end{Phonetics}
\end{Entry}

\begin{Entry}{用于}{5,3}{⽤,⼆}
  \begin{Phonetics}{用于}{yong4yu2}[][HSK 5]
    \definition{v.}{usar para; ser usado para; usar em}
  \end{Phonetics}
\end{Entry}

\begin{Entry}{用不着}{5,4,11}{⽤,⼀,⽬}
  \begin{Phonetics}{用不着}{yong4bu4zhao2}[][HSK 5]
    \definition{v.}{não precisar; não ter utilidade para; não haver necessidade de}
  \end{Phonetics}
\end{Entry}

\begin{Entry}{用心}{5,4}{⽤,⼼}
  \begin{Phonetics}{用心}{yong4 xin1}[][HSK 6]
    \definition{adj.}{diligente; atento; com atenção concentrada}
    \definition{s.}{motivo; intenção; o verdadeiro propósito ou razão para fazer algo}
  \end{Phonetics}
\end{Entry}

\begin{Entry}{用户}{5,4}{⽤,⼾}
  \begin{Phonetics}{用户}{yong4hu4}[][HSK 5]
    \definition[个,位,名]{s.}{usuário; consumidor; entidades e indivíduos que utilizam determinados equipamentos públicos ou bens de consumo}
  \end{Phonetics}
\end{Entry}

\begin{Entry}{用处}{5,5}{⽤,⼡}
  \begin{Phonetics}{用处}{yong4chu3}[][HSK 6]
    \definition[个]{s.}{uso; usabilidade; utilidade}
  \end{Phonetics}
\end{Entry}

\begin{Entry}{用来}{5,7}{⽤,⽊}
  \begin{Phonetics}{用来}{yong4lai2}[][HSK 5]
    \definition{v.}{ser usado para; depender (dele) ou usar (ele) para atingir algum objetivo}
  \end{Phonetics}
\end{Entry}

\begin{Entry}{用法}{5,8}{⽤,⽔}
  \begin{Phonetics}{用法}{yong4fa3}[][HSK 6]
    \definition[种,个]{s.}{uso; emprego; a maneira de usar}
  \end{Phonetics}
\end{Entry}

\begin{Entry}{用品}{5,9}{⽤,⼝}
  \begin{Phonetics}{用品}{yong4pin3}[][HSK 6]
    \definition[批,件,种]{s.}{suprimentos; artigos para uso; itens para usar}
  \end{Phonetics}
\end{Entry}

\begin{Entry}{用料}{5,10}{⽤,⽃}
  \begin{Phonetics}{用料}{yong4liao4}
    \definition{s.}{ingredientes | materiais}
  \end{Phonetics}
\end{Entry}

\begin{Entry}{用途}{5,10}{⽤,⾡}
  \begin{Phonetics}{用途}{yong4tu2}[][HSK 4]
    \definition[个,种]{s.}{uso; aplicação; aspectos ou escopo da aplicação}
  \end{Phonetics}
\end{Entry}

\begin{Entry}{用得着}{5,11,11}{⽤,⼻,⽬}
  \begin{Phonetics}{用得着}{yong4de5zhao2}[][HSK 6]
    \definition{adj.}{útil; necessário}
    \definition{v.}{precisar; achar algo útil | ter necessidade de;  ser necessário; valer a pena}
  \end{Phonetics}
\end{Entry}

%%%%%%%%%% 甭 %%%%%%%%%%
\subsection*{甭}\addcontentsline{loh}{figure}{甭}

\begin{Entry}{甭}{9}{⽤}
  \begin{Phonetics}{甭}{beng2}
    \definition{adv.}{não; não precisa; não tem que; contração de 不用}
  \seealsoref{不用}{bu2yong4}
  \end{Phonetics}
\end{Entry}

%%%%% EOF %%%%%

