%%%
%%% Radical "⿊"
%%%
\section*{Radical 203: ``⿊''}\addcontentsline{toc}{section}{Radical 203: ⿊}\addcontentsline{loh}{figure}{\#\#\#\# 203: ⿊}

%%%%%%%%%% 黑 %%%%%%%%%%
\subsection*{黑}\addcontentsline{loh}{figure}{黑}

\begin{Entry}{黑}{12}{⿊}[Kangxi 203]
  \begin{Phonetics}{黑}{hei1}[][HSK 2]
    \definition*{s.}{Província de Heilongjiang, abreviação de 黑龙江 | Sobrenome: Hei}
    \definition{adj.}{preto; cor semelhante à do carvão | escuro | obscuro; secreto | perverso; sinistro; ruim; cruel | reacionário}
    \definition{s.}{noite}
    \definition{v.}{fazer algo ilegalmente ou de forma desonesta; enganar; desviar dinheiro ilegalmente | invadir (uma rede, sites, computador, etc.)}
  \seealsoref{黑龙江}{hei1long2jiang1}
  \end{Phonetics}
\end{Entry}

\begin{Entry}{黑马}{12,3}{⿊,⾺}
  \begin{Phonetics}{黑马}{hei1ma3}[][HSK 7-9]
    \definition[匹,群]{s.}{azarão (cavalo preto) | Figurativo: pessoa pouco conhecida que alcança sucesso inesperado}
  \end{Phonetics}
\end{Entry}

\begin{Entry}{黑心}{12,4}{⿊,⼼}
  \begin{Phonetics}{黑心}{hei1xin1}[][HSK 7-9]
    \definition{adj.}{malvado; perverso | ganancioso; avarento | (certos bens) de má qualidade | implacável e sem consciência | de mente viciosa cheia de ódio e ciúme}
    \definition{s.}{coração negro; mente maligna | núcleo preto (falha na cerâmica)}
  \end{Phonetics}
\end{Entry}

\begin{Entry}{黑手}{12,4}{⿊,⼿}
  \begin{Phonetics}{黑手}{hei1shou3}[][HSK 7-9]
    \definition{s.}{mão negra; manipulador maligno dos bastidores | uma pessoa cruel manipulando alguém ou algo nos bastidores; uma metáfora para pessoas ou forças que secretamente realizam atividades de conspiração}
  \end{Phonetics}
\end{Entry}

\begin{Entry}{黑白}{12,5}{⿊,⽩}
  \begin{Phonetics}{黑白}{hei1bai2}[][HSK 7-9]
    \definition[只]{s.}{preto e branco | certo e errado; metáfora para o certo e o errado, o bem e o mal}
  \end{Phonetics}
\end{Entry}

\begin{Entry}{黑龙江}{12,5,6}{⿊,⿓,⽔}
  \begin{Phonetics}{黑龙江}{hei1long2jiang1}
    \definition*{s.}{Província de Heilongjiang | Rio Heilong Jiang;  Rio Amur (na Rússia)}
  \end{Phonetics}
\end{Entry}

\begin{Entry}{黑色}{12,6}{⿊,⾊}
  \begin{Phonetics}{黑色}{hei1se4}[][HSK 2]
    \definition{adj.}{metafórico: suspeito, ilegal}
    \definition{s.}{cor preta}
  \end{Phonetics}
\end{Entry}

\begin{Entry}{黑夜}{12,8}{⿊,⼣}
  \begin{Phonetics}{黑夜}{hei1ye4}[][HSK 6]
    \definition[个]{s.}{noite ; uma noite muito escura sem luz}
  \end{Phonetics}
\end{Entry}

\begin{Entry}{黑板}{12,8}{⿊,⽊}
  \begin{Phonetics}{黑板}{hei1ban3}[][HSK 2]
    \definition[块,个]{s.}{quadro negro; quadro de giz; uma placa, na qual se pode escrever com giz}
  \end{Phonetics}
\end{Entry}

\begin{Entry}{黑客}{12,9}{⿊,⼧}
  \begin{Phonetics}{黑客}{hei1ke4}[][HSK 7-9]
    \definition[个,些,位,名]{s.}{Empréstimo linguístico: \emph{hacker}; \emph{cracker}; intruso cibernético; gênio da computação; originalmente se refere a pessoas que não são profissionais de informática, mas são muito proficientes em tecnologia de computadores; agora se refere especificamente a pessoas que podem escrever programas de descriptografia para invadir ilegalmente redes de computadores de outras pessoas para interferir ou destruí-las}
  \end{Phonetics}
\end{Entry}

\begin{Entry}{黑桃}{12,10}{⿊,⽊}
  \begin{Phonetics}{黑桃}{hei1tao2}
    \definition{s.}{espadas ♠ (em jogos de cartas)}
  \seealsoref{方片}{fang1 pian4}
  \seealsoref{红心}{hong2xin1}
  \seealsoref{梅花}{mei2hua1}
  \end{Phonetics}
\end{Entry}

\begin{Entry}{黑暗}{12,13}{⿊,⽇}
  \begin{Phonetics}{黑暗}{hei1'an4}[][HSK 4]
    \definition{adj.}{escuro; sombrio; sem luz | maligno; corrupto; reacionário}
  \end{Phonetics}
\end{Entry}

%%%%%%%%%% 墨 %%%%%%%%%%
\subsection*{墨}\addcontentsline{loh}{figure}{墨}

\begin{Entry}{墨}{15}{⿊}
  \begin{Phonetics}{墨}{mo4}[][HSK 7-9]
    \definition*{s.}{Escola Moísta; Moísmo | México, abreviação de 墨西哥}
    \definition{adj.}{preto; escuro como breu | corrupto | escuro}
    \definition{s.}{tinta chinesa; bastão de tinta | pigmento; tinta | caligrafia ou pintura | aprendizagem; alfabetização | marcador de linha de carpinteiro; marcador de tinta | tatuar o rosto (um castigo); uma punição na China antiga | corrupção; peculato; fraude}
  \seealsoref{墨西哥}{mo4xi1ge1}
  \end{Phonetics}
\end{Entry}

\begin{Entry}{墨水}{15,4}{⿊,⽔}
  \begin{Phonetics}{墨水}{mo4shui3}[][HSK 6]
    \definition[瓶]{s.}{tinta chinesa preparada; tinta (para caneta-tinteiro) | aprendizagem; alfabetização; uma metáfora para o conhecimento ou a capacidade de ler e escrever}
  \end{Phonetics}
\end{Entry}

\begin{Entry}{墨西哥}{15,6,10}{⿊,⾑,⼝}
  \begin{Phonetics}{墨西哥}{mo4xi1ge1}
    \definition*{s.}{México; Planalto no México}
  \end{Phonetics}
\end{Entry}

\begin{Entry}{墨镜}{15,16}{⿊,⾦}
  \begin{Phonetics}{墨镜}{mo4jing4}
    \definition[只,双,副]{s.}{óculos escuros}
  \end{Phonetics}
\end{Entry}

%%%%%%%%%% 默 %%%%%%%%%%
\subsection*{默}\addcontentsline{loh}{figure}{默}

\begin{Entry}{默}{16}{⿊}
  \begin{Phonetics}{默}{mo4}
    \definition*{s.}{Sobrenome: Mo}
    \definition{adj.}{taciturno; reservado | silencioso}
    \definition{v.}{escrever de memória}
  \end{Phonetics}
\end{Entry}

\begin{Entry}{默契}{16,9}{⿊,⼤}
  \begin{Phonetics}{默契}{mo4qi4}[][HSK 7-9]
    \definition{adj.}{bem coordenado; mutuamente e tacitamente compreendido/acordado; descreve uma conexão profunda entre duas pessoas que transcende as palavras}
    \definition[些,种,份,点]{s.}{acordo ou contrato secreto; entendimento tácito}
  \end{Phonetics}
\end{Entry}

\begin{Entry}{默读}{16,10}{⿊,⾔}
  \begin{Phonetics}{默读}{mo4du2}[][HSK 7-9]
    \definition{v.}{ler em silêncio | subvocalizar}
  \antonymref{朗读}{lang3du2}
  \end{Phonetics}
\end{Entry}

\begin{Entry}{默默}{16,16}{⿊,⿊}
  \begin{Phonetics}{默默}{mo4mo4}[][HSK 4]
    \definition{adj.}{mudo; quieto; silencioso}
    \definition{adv.}{silenciosamente}
  \end{Phonetics}
\end{Entry}

\begin{Entry}{默默无闻}{16,16,4,9}{⿊,⿊,⽆,⾨}
  \begin{Phonetics}{默默无闻}{mo4mo4-wu2wen2}[][HSK 7-9]
    \definition{expr.}{obscuro; quieto e desconhecido; desconhecido}
  \end{Phonetics}
\end{Entry}

%%%%% EOF %%%%%

