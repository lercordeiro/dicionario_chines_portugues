%%%
%%% Radical "⽚"
%%%
\section*{Radical 91: ``⽚''}\addcontentsline{toc}{section}{Radical 91: ⽚}\addcontentsline{loh}{figure}{\#\#\#\# 91: ⽚}

%%%%%%%%%% 片 %%%%%%%%%%
\subsection*{片}\addcontentsline{loh}{figure}{片}

\begin{Entry}{片}{4}{⽚}[Kangxi 91]
  \begin{Phonetics}{片}{pian1}
    \definition{s.}{película; filme; refere"-se a filmes com imagens, paisagens ou imagens gravadas com som}
  \end{Phonetics}
  \begin{Phonetics}{片}{pian4}[][HSK 2]
    \definition*{s.}{Sobrenome: Pian}
    \definition{adj.}{breve; parcial; incompleto; fragmentário; esporádico; breve | unilateral}
    \definition{clas.}{usado para coisas em forma de lâminas | usado para terrenos ou superfícies aquáticas com a mesma paisagem e que estão conectados entre si | usado para paisagens, clima, sons, linguagem, intenções, etc. (usado em conjunto com o numeral 一)}
    \definition{s.}{plano, fatia; floco; pedaço fino; algo plano e fino | seção; parte de uma grande área; uma pequena parte do todo ou uma área menor dividida dentro de uma área maior | filme; peça de TV; referência ao filme}
    \definition{v.}{fatiar; cortar em fatias; cortar em fatias finas com uma faca | abrir; cortar; separar}
  \seealsoref{一}{yi1}
  \end{Phonetics}
\end{Entry}

\begin{Entry}{片儿}{4,2}{⽚,⼉}
  \begin{Phonetics}{片儿}{pian1r5}
    \definition{s.}{folha | película; filme}
  \end{Phonetics}
\end{Entry}

\begin{Entry}{片子}{4,3}{⽚,⼦}
  \begin{Phonetics}{片子}{pian1zi5}[][HSK 7-9]
    \definition{s.}{imagem de raio-X; negativos fotográficos de raios X | rolo de filme | filme; cinema | disco; disco de gramofone}
  \end{Phonetics}
  \begin{Phonetics}{片子}{pian4zi5}[][HSK 7-9]
    \definition{s.}{fatia; lasca; pedaço; fatia fina e plana; coisas planas e finas | cartão de visita}
  \end{Phonetics}
\end{Entry}

\begin{Entry}{片段}{4,9}{⽚,⽎}
  \begin{Phonetics}{片段}{pian4duan4}[][HSK 7-9]
    \definition{s.}{parte; trecho; fragmento; excerto; segmento; episódio; seção; um segmento de um todo (geralmente referindo-se a um artigo, romance, peça de teatro, experiência de vida, etc.)}
  \end{Phonetics}
\end{Entry}

\begin{Entry}{片面}{4,9}{⽚,⾯}
  \begin{Phonetics}{片面}{pian4mian4}[][HSK 4]
    \definition{adj.}{unilateral; tendencioso para um lado}
  \antonymref{全面}{quan2mian4}
  \end{Phonetics}
\end{Entry}

%%%%%%%%%% 版 %%%%%%%%%%
\subsection*{版}\addcontentsline{loh}{figure}{版}

\begin{Entry}{版}{8}{⽚}
  \begin{Phonetics}{版}{ban3}[][HSK 5]
    \definition{clas.}{usado como uma palavra de medida para materiais impressos (por exemplo, livros, jornais, edições)}
    \definition{s.}{chapa, placa ou bloco de impressão | edição (livros impressos) | página (de um jornal) | moldes ou fromas de construção}
  \end{Phonetics}
\end{Entry}

%%%%%%%%%% 牌 %%%%%%%%%%
\subsection*{牌}\addcontentsline{loh}{figure}{牌}

\begin{Entry}{牌}{12}{⽚}
  \begin{Phonetics}{牌}{pai2}[][HSK 4]
    \definition[块,副,张,个,种]{s.}{placa; tabuleta; quadro; placar | marca; marca registrada; marca comercial; \emph{trademark} | cartas, dominó, etc. | a tonalidade de uma música}
  \end{Phonetics}
\end{Entry}

\begin{Entry}{牌子}{12,3}{⽚,⼦}
  \begin{Phonetics}{牌子}{pai2 zi5}[][HSK 3]
    \definition[个,种,块]{s.}{sinal; placa; placas feitas de madeira ou outros materiais, geralmente com texto nelas | marca; marca registrada; um nome especial dado por uma empresa ao seu próprio produto}
  \end{Phonetics}
\end{Entry}

\begin{Entry}{牌照}{12,13}{⽚,⽕}
  \begin{Phonetics}{牌照}{pai2zhao4}[][HSK 7-9]
    \definition{s.}{placa de matrícula; certificado de licenciamento; certificado de registro de veículo ou licença comercial emitida pelo departamento administrativo competente}
  \end{Phonetics}
\end{Entry}

%%%%% EOF %%%%%

