%%%
%%% Radical "⼶"
%%%
\section*{Radical 55: ``⼶''}\addcontentsline{toc}{section}{Radical 55: ⼶}\addcontentsline{loh}{figure}{\#\#\#\# 55: ⼶}

%%%%%%%%%% 廿 %%%%%%%%%%
\subsection*{廿}\addcontentsline{loh}{figure}{廿}

\begin{Entry}{廿}{4}{⼶}
  \begin{Phonetics}{廿}{nian4}
    \definition{num.}{(dialeto) vinte; 20}
  \end{Phonetics}
\end{Entry}

%%%%%%%%%% 开 %%%%%%%%%%
\subsection*{开}\addcontentsline{loh}{figure}{开}

\begin{Entry}{开}{4}{⼶}
  \begin{Phonetics}{开}{kai1}[][HSK 1]
    \definition*{s.}{Sobrenome: Kai}
    \definition{clas.}{divisão do papel de impressão de tamanho padrão (uma parte da folha inteira) | quilate; unidade de cálculo da quantidade de ouro puro contida no ouro}
    \definition{s.}{porcentagem; percentual}
    \definition{v.}{abrir; estar ligado; ligar | recuperar; abrir; fazer uma abertura; escavar; abrir caminho; desbravar | abrir para fora; soltar-se | descongelar (rios); tornar-se navegável | levantar; libertar | iniciar; operar; manobrar | mover; estabelecer | executar; configurar | começar; iniciar | manter | escrever; fazer uma lista de | pagamento (salários, tarifas, etc.) | ferver}
    \definition{v.aux.}{usado após um verbo, indica ampliação ou expansão | usado após um verbo, indica o início e a continuidade}
  \seealsoref{开尔文}{kai1'er3wen2}
  \end{Phonetics}
\end{Entry}

\begin{Entry}{开口}{4,3}{⼶,⼝}
  \begin{Phonetics}{开口}{kai1/kou3}[][HSK 7-9]
    \definition{s.}{abertura; um corte ou rachadura}
    \definition{v.+compl.}{abrir a boca; começar a falar; abrir a boca e falar | pedir algo a alguém; fazer um pedido ou uma exigência |  afiar (uma faca nova ou uma tesoura nova) | cavar uma brecha; encontrar uma brecha; abrir uma fenda}
  \end{Phonetics}
\end{Entry}

\begin{Entry}{开工}{4,3}{⼶,⼯}
  \begin{Phonetics}{开工}{kai1/gong1}[][HSK 7-9]
    \definition{v.+compl.}{(fábrica, etc.) entrar em operação; iniciar a construção}
  \end{Phonetics}
\end{Entry}

\begin{Entry}{开办}{4,4}{⼶,⼒}
  \begin{Phonetics}{开办}{kai1ban4}[][HSK 7-9]
    \definition{v.}{abrir; iniciar; criar; configurar; estabelecer (uma empresa ou entidade comercial)}
  \end{Phonetics}
\end{Entry}

\begin{Entry}{开天辟地}{4,4,13,6}{⼶,⼤,⾟,⼟}
  \begin{Phonetics}{开天辟地}{kai1tian1-pi4di4}[][HSK 7-9]
    \definition{expr.}{``Criação do Céu e da Terra.''; a criação do céu e da terra; quando o céu se separou da terra; a criação do mundo; no princípio do céu e da terra (Gênesis) | desde o alvorecer da história; desde o início da história | que marca época; inovador, revolucionário, que inaugura uma nova era}
  \end{Phonetics}
\end{Entry}

\begin{Entry}{开心}{4,4}{⼶,⼼}
  \begin{Phonetics}{开心}{kai1/xin1}[][HSK 2]
    \definition{adj.}{feliz; alegre; exultante; encantado}
    \definition{v.+compl.}{provocar; brincar; tirar sarro de alguém; zombar; divertir-se}
  \end{Phonetics}
\end{Entry}

\begin{Entry}{开支}{4,4}{⼶,⽀}
  \begin{Phonetics}{开支}{kai1zhi1}[][HSK 7-9]
    \definition{v.}{pagar (despesas); gastar | pagar salários; receber o pagamento}
  \end{Phonetics}
\end{Entry}

\begin{Entry}{开水}{4,4}{⼶,⽔}
  \begin{Phonetics}{开水}{kai1shui3}[][HSK 4]
    \definition[杯,瓶]{s.}{água fervida; água fervente}
  \end{Phonetics}
\end{Entry}

\begin{Entry}{开车}{4,4}{⼶,⾞}
  \begin{Phonetics}{开车}{kai1/che1}[][HSK 1]
    \definition{v.+compl.}{dirigir um carro, trem, etc. | colocar uma máquina em funcionamento | (de um trem, etc.) partida | dirigir veículos motorizados}
  \end{Phonetics}
\end{Entry}

\begin{Entry}{开业}{4,5}{⼶,⼀}
  \begin{Phonetics}{开业}{kai1 ye4}[][HSK 3]
    \definition[场]{v.}{iniciar um negócio; abrir para negócios | abrir um consultório particular}
  \end{Phonetics}
\end{Entry}

\begin{Entry}{开发}{4,5}{⼶,⼜}
  \begin{Phonetics}{开发}{kai1fa5}[][HSK 3]
    \definition{v.}{explorar; trabalhar com recursos naturais como terras baldias, minas, florestas e energia hidráulica para fins de aproveitamento | tornar acessível; descobrir ou explorar talentos, tecnologias, etc. para aproveitamento}
  \end{Phonetics}
\end{Entry}

\begin{Entry}{开发区}{4,5,4}{⼶,⼜,⼖}
  \begin{Phonetics}{开发区}{kai1fa1qu1}[][HSK 7-9]
    \definition*{s.}{Zona Econômica Aberta; Zona Econômica Especial}
    \definition{s.}{zona de desenvolvimento}
  \end{Phonetics}
\end{Entry}

\begin{Entry}{开发商}{4,5,11}{⼶,⼜,⼝}
  \begin{Phonetics}{开发商}{kai1fa1shang1}[][HSK 7-9]
    \definition{s.}{incorporador (de imóveis, de um produto comercial, etc.)}
  \end{Phonetics}
\end{Entry}

\begin{Entry}{开头}{4,5}{⼶,⼤}
  \begin{Phonetics}{开头}{kai1/tou2}[][HSK 6]
    \definition{s.}{início; começo; o momento ou estágio do início; antecedente no tempo}
    \definition{v.+compl.}{começar, iniciar; a primeira ocorrência de um evento, ação, fenômeno, etc. | pôr-se a pé; começar}
  \end{Phonetics}
\end{Entry}

\begin{Entry}{开尔文}{4,5,4}{⼶,⼩,⽂}
  \begin{Phonetics}{开尔文}{kai1'er3wen2}
    \definition{s.}{Kelvin, temperatura absoluta | K, escala de temperatura}
  \end{Phonetics}
\end{Entry}

\begin{Entry}{开会}{4,6}{⼶,⼈}
  \begin{Phonetics}{开会}{kai1/hui4}[][HSK 1]
    \definition{v.+compl.}{realizar uma reunião; ter uma reunião; participar de uma reunião (conferência)}
  \end{Phonetics}
\end{Entry}

\begin{Entry}{开关}{4,6}{⼶,⼋}
  \begin{Phonetics}{开关}{kai1guan1}[][HSK 6]
    \definition[个,种,些]{s.}{interruptor; um dispositivo que conecta e desconecta o circuito de um dispositivo elétrico | registro; um dispositivo instalado em uma tubulação de fluido para controlar o fluxo}
  \end{Phonetics}
\end{Entry}

\begin{Entry}{开创}{4,6}{⼶,⼑}
  \begin{Phonetics}{开创}{kai1chuang4}[][HSK 6]
    \definition{v.}{começar; iniciar; fundar; ser pioneiro; estabelecer; criar}
  \end{Phonetics}
\end{Entry}

\begin{Entry}{开动}{4,6}{⼶,⼒}
  \begin{Phonetics}{开动}{kai1dong4}[][HSK 7-9]
    \definition{v.}{iniciar; pôr em movimento; pôr em funcionamento; iniciar operação | mover-se; marchar; estar em movimento; zarpar}
  \end{Phonetics}
\end{Entry}

\begin{Entry}{开场}{4,6}{⼶,⼟}
  \begin{Phonetics}{开场}{kai1/chang3}[][HSK 7-9]
    \definition{v.+compl.}{(apresentação, etc.) começar; iniciar; estrear; o início de uma apresentação teatral ou de uma apresentação cultural em geral também pode ser usado metaforicamente para descrever o início de uma atividade geral}
  \end{Phonetics}
\end{Entry}

\begin{Entry}{开场白}{4,6,5}{⼶,⼟,⽩}
  \begin{Phonetics}{开场白}{kai1chang3bai2}[][HSK 7-9]
    \definition{s.}{prólogo (de uma peça); introdução; discurso de abertura; comentários iniciais (ou introdutórios); as linhas iniciais de uma peça teatral que introduzem o tema; metaforicamente, a seção inicial de um artigo ou discurso que apresenta a ideia principal}
  \end{Phonetics}
\end{Entry}

\begin{Entry}{开机}{4,6}{⼶,⽊}
  \begin{Phonetics}{开机}{kai1 ji1}[][HSK 2]
    \definition{v.}{começar a filmar um filme ou programa de TV; refere"-se ao início das filmagens (de filmes, séries de TV, etc.) | ligar uma máquina}
  \end{Phonetics}
\end{Entry}

\begin{Entry}{开设}{4,6}{⼶,⾔}
  \begin{Phonetics}{开设}{kai1she4}[][HSK 6]
    \definition{v.}{montar; estabelecer; abrir (uma loja, fábrica, etc.); estabelecer novas instituições ou campos | oferecer (um curso na faculdade, etc.)}
  \end{Phonetics}
\end{Entry}

\begin{Entry}{开启}{4,7}{⼶,⼝}
  \begin{Phonetics}{开启}{kai1qi3}[][HSK 7-9]
    \definition{v.}{abrir | iniciar | Computação: ativar}
  \end{Phonetics}
\end{Entry}

\begin{Entry}{开张}{4,7}{⼶,⼸}
  \begin{Phonetics}{开张}{kai1/zhang1}[][HSK 7-9]
    \definition{v.+compl.}{estrear; abrir um negócio; começar a fazer negócios; lojas, hotéis, etc., recém-construídos, começam a abrir as portas | fazer a primeira transação do dia; para empresários, isso se refere à primeira transação do dia}
  \antonymref{关张}{guan1zhang1}
  \end{Phonetics}
\end{Entry}

\begin{Entry}{开花}{4,7}{⼶,⾋}
  \begin{Phonetics}{开花}{kai1/hua1}[][HSK 4]
    \definition{v.+compl.}{florescer; desabrochar; estar em flor; entrar em flor;  metáfora para um coração feliz ou um rosto sorridente | explodir; quebrar; dividir | sentir-se feliz ou sorrir alegremente | (experiência) espalhar-se; (empreendimento) surgir; surgir | (cabeça) ser ferido e sangrar profusamente}
  \end{Phonetics}
\end{Entry}

\begin{Entry}{开夜车}{4,8,4}{⼶,⼣,⾞}
  \begin{Phonetics}{开夜车}{kai1/ye4che1}[][HSK 6]
    \definition{v.+compl.}{``Dirigir à noite.''; ``Conduzir carro à noite.''; trabalhar até tarde da noite; ficar acordado até tarde da noite estudando ou trabalhando para cumprir prazos}
  \end{Phonetics}
\end{Entry}

\begin{Entry}{开始}{4,8}{⼶,⼥}
  \begin{Phonetics}{开始}{kai1shi3}[][HSK 3]
    \definition[个]{s.}{começo; início; estágio inicial}
    \definition{v.}{começar; iniciar; começar a fazer algo}
  \end{Phonetics}
\end{Entry}

\begin{Entry}{开学}{4,8}{⼶,⼦}
  \begin{Phonetics}{开学}{kai1 xue2}[][HSK 2]
    \definition{v.}{iniciar as aulas; iniciar o semestre; começar as aulas}
  \end{Phonetics}
\end{Entry}

\begin{Entry}{开拓}{4,8}{⼶,⼿}
  \begin{Phonetics}{开拓}{kai1tuo4}[][HSK 7-9]
    \definition{v.}{desenvolver; ser pioneiro; abrir caminho; expandir; abrir}
  \end{Phonetics}
\end{Entry}

\begin{Entry}{开放}{4,8}{⼶,⽅}
  \begin{Phonetics}{开放}{kai1fang4}[][HSK 3]
    \definition{adj.}{de mente aberta; sem restrições por convenções; pensamento e ambiente não conservadores, disposto a aceitar coisas novas e novas ideias; personalidade alegre}
    \definition{v.}{florescer | abrir (para o público); levantar bloqueios, proibições, restrições, etc. | diminuir uma proibição, restrição, etc. (de política); (economia) reduzir as restrições políticas, com justificativas específicas}
  \end{Phonetics}
\end{Entry}

\begin{Entry}{开枪}{4,8}{⼶,⽊}
  \begin{Phonetics}{开枪}{kai1 qiang1}[][HSK 7-9]
    \definition{v.}{disparar com um rifle, pistola, etc.; disparar um tiro; atirar}
  \end{Phonetics}
\end{Entry}

\begin{Entry}{开玩笑}{4,8,10}{⼶,⽟,⽵}
  \begin{Phonetics}{开玩笑}{kai1 wan2xiao4}[][HSK 1]
    \definition{v.}{fazer (ou brincar, fazer) uma piada; gracejar; zombar de; provocar; fazer uma brincadeira; zombar de alguém | tratar casualmente; dar pouca importância a; considerar como um assunto insignificante; insignificante | fazer uma brincadeira; pregar uma peça; brincar; em tom de brincadeira}
  \end{Phonetics}
\end{Entry}

\begin{Entry}{开采}{4,8}{⼶,⾤}
  \begin{Phonetics}{开采}{kai1cai3}[][HSK 7-9]
    \definition{v.}{minerar; extrair; explorar; recuperar}
  \end{Phonetics}
\end{Entry}

\begin{Entry}{开垦}{4,9}{⼶,⼟}
  \begin{Phonetics}{开垦}{kai1ken3}[][HSK 7-9]
    \definition{v.}{abrir (ou recuperar) terrenos baldios; trazer para o cultivo; abrir terrenos baldios para a agricultura; preparar o terreno; cultivar}
  \end{Phonetics}
\end{Entry}

\begin{Entry}{开除}{4,9}{⼶,⾩}
  \begin{Phonetics}{开除}{kai1chu2}[][HSK 7-9]
    \definition{v.}{expulsar; demitir; dispensar; despedir}
  \end{Phonetics}
\end{Entry}

\begin{Entry}{开展}{4,10}{⼶,⼫}
  \begin{Phonetics}{开展}{kai1zhan3}[][HSK 3]
    \definition{v.}{lançar; desenvolver | abrir; inaugurar}
  \end{Phonetics}
\end{Entry}

\begin{Entry}{开朗}{4,10}{⼶,⽉}
  \begin{Phonetics}{开朗}{kai1lang3}[][HSK 7-9]
    \definition{adj.}{otimista; alegre; despreocupado; (pensamentos, personalidade e mentalidade) otimista, alegre e não melancólico ou deprimido | espaçoso; bem iluminado; aberto e desimpedido; aberto e luminoso}
  \end{Phonetics}
\end{Entry}

\begin{Entry}{开通}{4,10}{⼶,⾡}
  \begin{Phonetics}{开通}{kai1tong1}[][HSK 6]
    \definition{v.}{limpar; dragar; remover obstáculos de; abrir o canal; desbloquear}
  \end{Phonetics}
  \begin{Phonetics}{开通}{kai1tong5}
    \definition{adj.}{liberal; mente aberta; mente moderna; mente liberal; sábio e sensato; não conservador ou teimoso}
  \end{Phonetics}
\end{Entry}

\begin{Entry}{开销}{4,12}{⼶,⾦}
  \begin{Phonetics}{开销}{kai1xiao1}[][HSK 7-9]
    \definition[笔,项]{s.}{despesas; taxas pagas}
    \definition{v.}{pagar despesas (taxas)}
  \end{Phonetics}
\end{Entry}

\begin{Entry}{开锁}{4,12}{⼶,⾦}
  \begin{Phonetics}{开锁}{kai1suo3}
    \definition{v.}{desbloquear | destravar}
  \end{Phonetics}
\end{Entry}

\begin{Entry}{开阔}{4,12}{⼶,⾨}
  \begin{Phonetics}{开阔}{kai1kuo4}[][HSK 7-9]
    \definition{adj.}{aberto; amplo; largo | de mente aberta}
    \definition{v.}{ampliar; abrir}
  \end{Phonetics}
\end{Entry}

\begin{Entry}{开幕}{4,13}{⼶,⼱}
  \begin{Phonetics}{开幕}{kai1 mu4}[][HSK 5]
    \definition{v.}{começar a apresentação; iniciar o espetáculo; levantar das cortinas | abrir; inaugurar; iniciar (uma conferência, exposição, etc.)}
  \end{Phonetics}
\end{Entry}

\begin{Entry}{开幕式}{4,13,6}{⼶,⼱,⼷}
  \begin{Phonetics}{开幕式}{kai1mu4shi4}[][HSK 5]
    \definition[场,次,届]{s.}{cerimônia de abertura; cerimônias e apresentações antes de eventos esportivos ou grandes eventos}
  \end{Phonetics}
\end{Entry}

\begin{Entry}{开辟}{4,13}{⼶,⾟}
  \begin{Phonetics}{开辟}{kai1pi4}[][HSK 7-9]
    \definition{v.}{abrir; talhar; romper; abrir passagem | abrir; iniciar; desenvolver; explorar; ampliar; expandir | abrir; iniciar; fundar; estabelecer; criar}
  \end{Phonetics}
\end{Entry}

%%%%%%%%%% 异 %%%%%%%%%%
\subsection*{异}\addcontentsline{loh}{figure}{异}

\begin{Entry}{异}{6}{⼶}
  \begin{Phonetics}{异}{yi4}
    \definition{adj.}{diferente | estranho; incomum; extraordinário; especial | outro}
    \definition{v.}{surpreender | separar; divorciar-se}
  \end{Phonetics}
\end{Entry}

\begin{Entry}{异常}{6,11}{⼶,⼱}
  \begin{Phonetics}{异常}{yi4chang2}[][HSK 6]
    \definition{adj.}{incomum; anormal; descreve uma situação diferente do normal}
    \definition{adv.}{extremamente; particularmente; excepcionalmente; descreve uma situação que atingiu um nível extremamente alto}
  \end{Phonetics}
\end{Entry}

%%%%%%%%%% 弄 %%%%%%%%%%
\subsection*{弄}\addcontentsline{loh}{figure}{弄}

\begin{Entry}{弄}{7}{⼶}
  \begin{Phonetics}{弄}{long4}
    \definition{s.}{rua estreita; beco; viela; travessa}
  \end{Phonetics}
  \begin{Phonetics}{弄}{nong4}[][HSK 2]
    \definition{v.}{fazer, realizar; tratar; organizar | obter; buscar; tentar conseguir; encontrar uma maneira de conseguir | brincar com; enganar | pregar uma peça; brincar; manipular | mexer com; perturbar}
  \end{Phonetics}
\end{Entry}

\begin{Entry}{弄虚作假}{7,11,7,11}{⼶,⾌,⼈,⼈}
  \begin{Phonetics}{弄虚作假}{nong4xu1-zuo4jia3}[][HSK 7-9]
    \definition{expr.}{recorrer ao engano; empregar artimanhas; usar de artifícios; pregar peças; praticar fraude; humilhar-se; fazer truques e enganar as pessoas}
  \end{Phonetics}
\end{Entry}

%%%%%%%%%% 弊 %%%%%%%%%%
\subsection*{弊}\addcontentsline{loh}{figure}{弊}

\begin{Entry}{弊}{14}{⼶}
  \begin{Phonetics}{弊}{bi4}
    \definition{s.}{fraude; abuso; negligência médica | desvantagem; falha; defeito; dano | trapaça; fraude, engano e falsificação}
  \antonymref{利}{li4}
  \end{Phonetics}
\end{Entry}

\begin{Entry}{弊病}{14,10}{⼶,⽧}
  \begin{Phonetics}{弊病}{bi4bing4}[][HSK 7-9]
    \definition{s.}{doença; mal; negligência | incinveniente; desvantagem; problemas com coisas}
  \end{Phonetics}
\end{Entry}

\begin{Entry}{弊端}{14,14}{⼶,⽴}
  \begin{Phonetics}{弊端}{bi4duan1}[][HSK 7-9]
    \definition{s.}{abuso; negligência; prática corrupta; danos ao interesse público devido a uma lacuna no trabalho}
  \end{Phonetics}
\end{Entry}

%%%%% EOF %%%%%

