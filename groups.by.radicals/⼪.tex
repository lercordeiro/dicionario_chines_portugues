%%%
%%% Radical "⼪"
%%%
\section*{Radical 43: ``⼪'' (尣)}\addcontentsline{toc}{section}{Radical 43: ⼪、尣}\addcontentsline{loh}{figure}{\#\#\#\# 43: ⼪}

%%%%%%%%%% 尤 %%%%%%%%%%
\subsection*{尤}\addcontentsline{loh}{figure}{尤}

\begin{Entry}{尤}{4}{⼪}
  \begin{Phonetics}{尤}{you2}
    \definition*{s.}{Sobrenome: You}
    \definition{adj.}{excelente; peculiar; notável}
    \definition{adv.}{particularmente; especialmente}
    \definition{s.}{falha; erro | irregularidade}
    \definition{v.}{ter rancor contra; culpar}
  \end{Phonetics}
\end{Entry}

\begin{Entry}{尤其}{4,8}{⼪,⼋}
  \begin{Phonetics}{尤其}{you2qi2}[][HSK 5]
    \definition{adv.}{especialmente; particularmente; indica um grau mais avançado, equivalente a 更加}
  \seealsoref{更加}{geng4 jia1}
  \end{Phonetics}
\end{Entry}

%%%%%%%%%% 尧 %%%%%%%%%%
\subsection*{尧}\addcontentsline{loh}{figure}{尧}

\begin{Entry}{尧}{6}{⼪}
  \begin{Phonetics}{尧}{yao2}
    \definition*{s.}{Yao, um monarca lendário da China antiga | Sobrenome: Yao}
  \end{Phonetics}
\end{Entry}

%%%%%%%%%% 就 %%%%%%%%%%
\subsection*{就}\addcontentsline{loh}{figure}{就}

\begin{Entry}{就}{12}{⼪}
  \begin{Phonetics}{就}{jiu4}[][HSK 1]
    \definition{adv.}{de imediato; imediatamente; indica que algo ocorrerá em breve | tão cedo quanto; já; há muito tempo; indica que a ação ocorreu há muito tempo | assim que; logo depois; indica que os eventos se sucedem imediatamente | nesse caso; então; indica que, sob determinadas condições, ocorre naturalmente um determinado resultado | exatamente; precisamente; indica que é exatamente assim | apenas; meramente; somente | tantos quanto; enfatiza a quantidade | apenas; simplesmente; reforço da afirmação | colocado entre dois componentes idênticos, significa tolerância ou indiferença}
    \definition{prep.}{tirar proveito de alguém (algo); expressa condições, oportunidades, etc., equivalente a 趁 | quando se trata de alguém (algo); relativo a; com relação a; sobre; objeto ou escopo da introdução da ação |no local; introduz o local próximo ao qual a ação ocorreu}
    \definition{v.}{ser comido com; ir com; pratos, frutas, etc., acompanhados de alimentos básicos ou bebidas alcoólicas | aproximar-se; mover-se em direção a | ir para; assumir; empreender; envolver-se em; entrar em | realizar; fazer | tirar proveito de; acomodar-se a; adequar-se; encaixar-se | assumir; começar a entrar ou a exercer | seguir; acompanhar}
  \seealsoref{趁}{chen4}
  \end{Phonetics}
\end{Entry}

\begin{Entry}{就业}{12,5}{⼪,⼀}
  \begin{Phonetics}{就业}{jiu4/ye4}[][HSK 3]
    \definition{v.+compl.}{conseguir um emprego; obter emprego; assumir uma ocupação; começar a trabalhar}
  \end{Phonetics}
\end{Entry}

\begin{Entry}{就可以了}{12,5,4,2}{⼪,⼝,⼈,⼅}
  \begin{Phonetics}{就可以了}{jiu4 ke3yi3le5}
    \definition{expr.}{é isso; é o suficiente}
  \end{Phonetics}
\end{Entry}

\begin{Entry}{就任}{12,6}{⼪,⼈}
  \begin{Phonetics}{就任}{jiu4ren4}[][HSK 7-9]
    \definition{v.}{assumir o cargo; tomar posse | assumir o próprio cargo}
  \end{Phonetics}
\end{Entry}

\begin{Entry}{就地}{12,6}{⼪,⼟}
  \begin{Phonetics}{就地}{jiu4di4}[][HSK 7-9]
    \definition{adv.}{no local; no próprio local | localmente}
  \end{Phonetics}
\end{Entry}

\begin{Entry}{就医}{12,7}{⼪,⼖}
  \begin{Phonetics}{就医}{jiu4/yi1}[][HSK 7-9]
    \definition{v.+compl.}{consultar um médico; ir ao médico; buscar aconselhamento médico}
  \end{Phonetics}
\end{Entry}

\begin{Entry}{就坐}{12,7}{⼪,⼟}
  \begin{Phonetics}{就坐}{jiu4zuo4}
    \definition{v.}{sentar-se; estar sentado}
  \end{Phonetics}
\end{Entry}

\begin{Entry}{就诊}{12,7}{⼪,⾔}
  \begin{Phonetics}{就诊}{jiu4/zhen3}[][HSK 7-9]
    \definition{v.+compl.}{consultar um médico; procurar aconselhamento médico}
  \end{Phonetics}
\end{Entry}

\begin{Entry}{就近}{12,7}{⼪,⾡}
  \begin{Phonetics}{就近}{jiu4jin4}[][HSK 7-9]
    \definition{adv.}{(fazer ou obter algo) nas proximidades; na vizinhança; sem ter que ir longe; significa que está por perto}
  \end{Phonetics}
\end{Entry}

\begin{Entry}{就是}{12,9}{⼪,⽇}
  \begin{Phonetics}{就是}{jiu4 shi4}[][HSK 3]
    \definition{adv.}{exatamente; precisamente; expressar concordância com a afirmação da outra pessoa ou confirmar que a afirmação da outra pessoa está correta | apenas; simplesmente; expressa afirmação, determinação ou ênfase, o significado específico deve ser determinado com base no contexto anterior ou posterior | usado para indicar escolha}
    \definition{conj.}{ainda que; mesmo que se reconheça que essa situação é verdadeira, a situação posterior não mudará}
    \definition{part.}{usado no final de uma frase para expressar afirmação}
  \end{Phonetics}
\end{Entry}

\begin{Entry}{就是说}{12,9,9}{⼪,⽇,⾔}
  \begin{Phonetics}{就是说}{jiu4 shi4 shuo1}[][HSK 6]
    \definition{expr.}{ou seja; isto é; em outras palavras; é frequentemente usado como uma interjeição em uma frase para indicar que as palavras seguintes são uma explicação ou esclarecimento das anteriores}
  \end{Phonetics}
\end{Entry}

\begin{Entry}{就要}{12,9}{⼪,⾑}
  \begin{Phonetics}{就要}{jiu4 yao4}[][HSK 2]
    \definition{adv.}{estar prestes a; estar indo para; estar no ponto de}
  \end{Phonetics}
\end{Entry}

\begin{Entry}{就座}{12,10}{⼪,⼴}
  \begin{Phonetics}{就座}{jiu4/zuo4}[][HSK 7-9]
    \definition{v.+compl.}{sentar-se; estar sentado | ocupar o próprio lugar (assento)}
  \seealsoref{就坐}{jiu4zuo4}
  \end{Phonetics}
\end{Entry}

\begin{Entry}{就读}{12,10}{⼪,⾔}
  \begin{Phonetics}{就读}{jiu4du2}[][HSK 7-9]
    \definition{v.}{fazer um curso; frequentar a escola; ir à escola; ingressar na escola}
  \end{Phonetics}
\end{Entry}

\begin{Entry}{就职}{12,11}{⼪,⽿}
  \begin{Phonetics}{就职}{jiu4/zhi2}
    \definition{v.+compl.}{assumir o cargo; assumir oficialmente um cargo (geralmente referindo"-se a uma posição de maior hierarquia)}
  \end{Phonetics}
\end{Entry}

\begin{Entry}{就算}{12,14}{⼪,⽵}
  \begin{Phonetics}{就算}{jiu4 suan4}[][HSK 6]
    \definition{conj.}{mesmo que; concedido que; expressam uma relação hipotética e concessiva, frequentemente usadas com 也, equivalente a 即使}
  \seealsoref{即使}{ji2shi3}
  \seealsoref{也}{ye3}
  \end{Phonetics}
\end{Entry}

\begin{Entry}{就餐}{12,16}{⼪,⾷}
  \begin{Phonetics}{就餐}{jiu4can1}[][HSK 7-9]
    \definition{v.}{comer; jantar; fazer uma refeição; ir comer}
  \end{Phonetics}
\end{Entry}

%%%%%%%%%% 尴 %%%%%%%%%%
\subsection*{尴}\addcontentsline{loh}{figure}{尴}

\begin{Entry}{尴}{13}{⼪}
  \begin{Phonetics}{尴}{gan1}
    \definition{adj.}{envergonhado; uma situação ou assunto difícil de lidar | pouco à vontade; expressão não natural; envergonhado}
  \end{Phonetics}
\end{Entry}

\begin{Entry}{尴尬}{13,7}{⼪,⼪}
  \begin{Phonetics}{尴尬}{gan1ga4}[][HSK 7-9]
    \definition{adj.}{estranho; envergonhado; quando você se depara com algo difícil de lidar ou algo que o deixa envergonhado}
  \end{Phonetics}
\end{Entry}

%%%%% EOF %%%%%

