%%%
%%% Radical "⼊"
%%%
\section*{Radical 11: ``⼊''}\addcontentsline{toc}{section}{Radical 11: ⼊}\addcontentsline{loh}{figure}{\#\#\#\# 11: ⼊}

%%%%%%%%%% 入 %%%%%%%%%%
\subsection*{入}\addcontentsline{loh}{figure}{入}

\begin{Entry}{入}{2}{⼊}[Kangxi 11]
  \begin{Phonetics}{入}{ru4}[][HSK 6]
    \definition{s.}{renda | tom de entrada}
    \definition{v.}{entrar; entrar | juntar"-se; ser admitido em; tornar"-se membro de | conformar"-se com; concordar com | alcançar; atingir; entrar em (um certo nível ou estado) | fazer entrar; fazer algo entrar; fazer entrada}
  \antonymref{出}{chu1}
  \end{Phonetics}
\end{Entry}

\begin{Entry}{入乡随俗}{2,3,11,9}{⼊,⼄,⾩,⼈}
  \begin{Phonetics}{入乡随俗}{ru4xiang1-sui2su2}
    \definition{expr.}{``Em roma, faça como os romanos!''}
  \end{Phonetics}
\end{Entry}

\begin{Entry}{入口}{2,3}{⼊,⼝}
  \begin{Phonetics}{入口}{ru4/kou3}[][HSK 2]
    \definition[个]{s.}{entrada; entrada em locais, edifícios, estradas, etc., através de portões ou portas}
    \definition{v.+compl.}{entrar na boca | importar; mercadorias estrangeiras importadas, às vezes também se refere a mercadorias de outras regiões importadas para esta região}
  \end{Phonetics}
\end{Entry}

\begin{Entry}{入门}{2,3}{⼊,⾨}
  \begin{Phonetics}{入门}{ru4/men2}[][HSK 5]
    \definition{s.}{(geralmente em títulos de livros) curso básico; manual introdutório | ABC; guia; refere"-se a leituras básicas; conhecimentos básicos}
    \definition{v.+compl.}{ultrapassar o limiar; aprender os rudimentos de um assunto | aprender o ABC de; ser introduzido a um assunto; aprender o básico}
  \end{Phonetics}
\end{Entry}

\begin{Entry}{入手}{2,4}{⼊,⼿}
  \begin{Phonetics}{入手}{ru4shou3}
    \definition{v.}{começar com; proceder a partir de; tomar como ponto de partida | obter; apoderar"-se  | começar; para dar início}
  \antonymref{出手}{chu1/shou3}
  \end{Phonetics}
\end{Entry}

\begin{Entry}{入场}{2,6}{⼊,⼟}
  \begin{Phonetics}{入场}{ru4/chang3}[][HSK 7-9]
    \definition{v.+compl.}{entrar; ser admitido; entrar no local}
  \end{Phonetics}
\end{Entry}

\begin{Entry}{入场券}{2,6,8}{⼊,⼟,⼑}
  \begin{Phonetics}{入场券}{ru4chang3quan4}[][HSK 7-9]
    \definition{s.}{bilhete (de entrada) | pré-requisito para atingir um objetivo; qualificação para entrar em uma partida | ingresso; admissões}
  \end{Phonetics}
\end{Entry}

\begin{Entry}{入学}{2,8}{⼊,⼦}
  \begin{Phonetics}{入学}{ru4/xue2}[][HSK 6]
    \definition{v.+compl.}{(uma criança) começar a escola; começar a escola primária | entrar em uma escola; matricular-se em uma escola}
  \end{Phonetics}
\end{Entry}

\begin{Entry}{入侵}{2,9}{⼊,⼈}
  \begin{Phonetics}{入侵}{ru4qin1}[][HSK 7-9]
    \definition{v.}{invadir; intrometer"-se; fazer uma incursão; abrir caminho}
  \end{Phonetics}
\end{Entry}

\begin{Entry}{入选}{2,9}{⼊,⾡}
  \begin{Phonetics}{入选}{ru4xuan3}[][HSK 7-9]
    \definition{v.}{ser escolhido; ser selecionado}
  \end{Phonetics}
\end{Entry}

\begin{Entry}{入党}{2,10}{⼊,⼉}
  \begin{Phonetics}{入党}{ru4dang3}
    \definition{v.}{ingressar em um partido político (especialmente o partido comunista)}
  \end{Phonetics}
\end{Entry}

\begin{Entry}{入境}{2,14}{⼊,⼟}
  \begin{Phonetics}{入境}{ru4/jing4}[][HSK 7-9]
    \definition{v.+compl.}{entrar em um país; imigrar}
  \end{Phonetics}
\end{Entry}

%%%%%%%%%% 全 %%%%%%%%%%
\subsection*{全}\addcontentsline{loh}{figure}{全}

\begin{Entry}{全}{6}{⼊}
  \begin{Phonetics}{全}{quan2}[][HSK 2]
    \definition*{s.}{Sobrenome: Quan}
    \definition{adj.}{completo; total; inteiro}
    \definition{adv.}{inteiramente; totalmente; completamente; significa 100\%; equivalente a 完全 ou 全然}
    \definition{v.}{manter intacto; tornar perfeito ou completo; completar}
  \seealsoref{全然}{quan2ran2}
  \seealsoref{完全}{wan2quan2}
  \end{Phonetics}
\end{Entry}

\begin{Entry}{全力}{6,2}{⼊,⼒}
  \begin{Phonetics}{全力}{quan2li4}[][HSK 6]
    \definition{s.}{exercendo todos os seus esforços; energia ou força total; toda força ou energia}
  \end{Phonetics}
\end{Entry}

\begin{Entry}{全力以赴}{6,2,4,9}{⼊,⼒,⼈,⾛}
  \begin{Phonetics}{全力以赴}{quan2li4yi3fu4}[][HSK 7-9]
    \definition{expr.}{``Dê tudo de si.''; fazer a todo custo; dar o máximo de si; prosseguir; dedicar todas as suas forças a algo}
  \end{Phonetics}
\end{Entry}

\begin{Entry}{全心全意}{6,4,6,13}{⼊,⼼,⼊,⼼}
  \begin{Phonetics}{全心全意}{quan2xin1-quan2yi4}[][HSK 7-9]
    \definition{expr.}{``De todo o coração.''; dedicar"-se de corpo e alma a; de corpo e alma; com todo o coração}
  \end{Phonetics}
\end{Entry}

\begin{Entry}{全文}{6,4}{⼊,⽂}
  \begin{Phonetics}{全文}{quan2wen2}[][HSK 7-9]
    \definition{s.}{texto completo}
  \end{Phonetics}
\end{Entry}

\begin{Entry}{全方位}{6,4,7}{⼊,⽅,⼈}
  \begin{Phonetics}{全方位}{quan2fang1wei4}[][HSK 7-9]
    \definition{adj.}{versátil | em todo o redor | completo | abrangente | holístico | omnidirecional}
  \end{Phonetics}
\end{Entry}

\begin{Entry}{全长}{6,4}{⼊,⾧}
  \begin{Phonetics}{全长}{quan2chang2}[][HSK 7-9]
    \definition{s.}{comprimento total | extensão; alcance}
  \end{Phonetics}
\end{Entry}

\begin{Entry}{全世界}{6,5,9}{⼊,⼀,⽥}
  \begin{Phonetics}{全世界}{quan2shi4jie4}[][HSK 5]
    \definition[种]{s.}{mundo inteiro; mundo todo | em todo o mundo}
  \end{Phonetics}
\end{Entry}

\begin{Entry}{全场}{6,6}{⼊,⼟}
  \begin{Phonetics}{全场}{quan2chang3}[][HSK 3]
    \definition{s.}{toda a audiência; todos os presentes; todo o público}
  \end{Phonetics}
\end{Entry}

\begin{Entry}{全年}{6,6}{⼊,⼲}
  \begin{Phonetics}{全年}{quan2nian2}[][HSK 2]
    \definition{s.}{ano inteiro | anual; todo ano}
  \end{Phonetics}
\end{Entry}

\begin{Entry}{全体}{6,7}{⼊,⼈}
  \begin{Phonetics}{全体}{quan2ti3}[][HSK 2]
    \definition{s.}{(frequentemente referido a pessoas) todos; número total; todos | por todo o corpo | todos; inteiro; a soma de todas as partes; a soma de todos os indivíduos (geralmente se refere a pessoas)}
  \end{Phonetics}
\end{Entry}

\begin{Entry}{全局}{6,7}{⼊,⼫}
  \begin{Phonetics}{全局}{quan2ju2}[][HSK 7-9]
    \definition{s.}{situação geral; situação como um todo}
  \end{Phonetics}
\end{Entry}

\begin{Entry}{全身}{6,7}{⼊,⾝}
  \begin{Phonetics}{全身}{quan2shen1}[][HSK 2]
    \definition{s.}{corpo inteiro; por todo o corpo; todo o corpo}
  \end{Phonetics}
\end{Entry}

\begin{Entry}{全国}{6,8}{⼊,⼞}
  \begin{Phonetics}{全国}{quan2guo2}[][HSK 2]
    \definition{s.}{toda a nação (ou país); em todo o país; em todo o território nacional | toda a nação; todo o país}
  \end{Phonetics}
\end{Entry}

\begin{Entry}{全面}{6,9}{⼊,⾯}
  \begin{Phonetics}{全面}{quan2mian4}[][HSK 3]
    \definition{adj.}{geral; completo; abrangente; onipotente}
    \definition{s.}{todos os aspectos; cada aspecto}
  \seealsoref{片面}{pian4mian4}
  \end{Phonetics}
\end{Entry}

\begin{Entry}{全家}{6,10}{⼊,⼧}
  \begin{Phonetics}{全家}{quan2jia1}[][HSK 2]
    \definition{s.}{toda a família; a família inteira}
  \end{Phonetics}
\end{Entry}

\begin{Entry}{全称特命全权大使}{6,10,10,8,6,6,3,8}{⼊,⽲,⽜,⼝,⼊,⽊,⼤,⼈}
  \begin{Phonetics}{全称特命全权大使}{quan2cheng1 te4ming4 quan2quan2 da4shi3}
    \definition*{s.}{Embaixador Extraordinário e Plenipotenciário}
  \end{Phonetics}
\end{Entry}

\begin{Entry}{全能}{6,10}{⼊,⾁}
  \begin{Phonetics}{全能}{quan2neng2}[][HSK 7-9]
    \definition{adj.}{todo"-poderoso; onipotente | Esporte: versátil | pluripotente}
  \end{Phonetics}
\end{Entry}

\begin{Entry}{全部}{6,10}{⼊,⾢}
  \begin{Phonetics}{全部}{quan2bu4}[][HSK 2]
    \definition{adv.}{tudo; total; inteiro; completo; aplica"-se a todos, sem exceção}
    \definition{s.}{totalidade; total; integridade; a soma de todas as partes; o todo}
  \end{Phonetics}
\end{Entry}

\begin{Entry}{全都}{6,10}{⼊,⾢}
  \begin{Phonetics}{全都}{quan2dou1}[][HSK 5]
    \definition{adv.}{tudo; todos; sem exceção}
  \end{Phonetics}
\end{Entry}

\begin{Entry}{全都不}{6,10,4}{⼊,⾢,⼀}
  \begin{Phonetics}{全都不}{quan2dou1 bu4}
    \definition{adj.}{nada; nenhum; nenhum deles; nada disso}
  \end{Phonetics}
\end{Entry}

\begin{Entry}{全球}{6,11}{⼊,⽟}
  \begin{Phonetics}{全球}{quan2qiu2}[][HSK 3]
    \definition[门]{s.}{o mundo inteiro; a Terra inteira}
  \end{Phonetics}
\end{Entry}

\begin{Entry}{全职}{6,11}{⼊,⽿}
  \begin{Phonetics}{全职}{quan2zhi2}
    \definition{s.}{período integral | tempo inteiro | (trabalho) \emph{full-time}}
  \end{Phonetics}
\end{Entry}

\begin{Entry}{全然}{6,12}{⼊,⽕}
  \begin{Phonetics}{全然}{quan2ran2}
    \definition{adv.}{completamente; inteiramente}
  \end{Phonetics}
\end{Entry}

\begin{Entry}{全程}{6,12}{⼊,⽲}
  \begin{Phonetics}{全程}{quan2cheng2}[][HSK 7-9]
    \definition{s.}{toda a jornada; todo o percurso}
  \end{Phonetics}
\end{Entry}

\begin{Entry}{全新}{6,13}{⼊,⽄}
  \begin{Phonetics}{全新}{quan2xin1}[][HSK 6]
    \definition{adj.}{totalmente novo; inteiramente/completamente novo; refere"-se a algo completamente novo, especialmente algo que não foi usado}
  \end{Phonetics}
\end{Entry}

%%%%%%%%%% 肏 %%%%%%%%%%
\subsection*{肏}\addcontentsline{loh}{figure}{肏}

\begin{Entry}{肏}{8}{⼊}
  \begin{Phonetics}{肏}{cao4}
    \definition{v.}{(vulgar) foder; palavras sujas usadas para insultar pessoas; refere"-se à relação sexual masculina}
  \end{Phonetics}
\end{Entry}

%%%%% EOF %%%%%

