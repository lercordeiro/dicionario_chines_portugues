%%%
%%% Radical "⽣"
%%%
\section*{Radical 100: ``⽣''}\addcontentsline{toc}{section}{Radical 100: ⽣}\addcontentsline{loh}{figure}{\#\#\#\# 100: ⽣}

%%%%%%%%%% 生 %%%%%%%%%%
\subsection*{生}\addcontentsline{loh}{figure}{生}

\begin{Entry}{生}{5}{⽣}[Kangxi 100]
  \begin{Phonetics}{生}{sheng1}[][HSK 2,3]
    \definition*{s.}{Sobrenome: Sheng}
    \definition{adj.}{vivo; vital | verde; não maduro | cru; não cozido; mal cozido | bruto; não refinado; não processado | estranho; desconhecido; não familiarizado | rígido; mecânico; forçado}
    \definition{adv.}{muito; usado antes de certas palavras que expressam emoções e sentimentos | verdadeiramente; realmente; forçosamente}
    \definition{s.}{vida | meio de subsistência | aluno; estudante | estudioso; antigamente chamados de eruditos | o tipo de personagem masculino na ópera de Pequim, etc.}
    \definition{suf.}{certos sufixos substantivos que se referem a pessoas (学生) | sufixos de certos advérbios (好生)}
    \definition{v.}{dar à luz; ter um filho | nascer | crescer; cultivar | viver; existir; sobreviver | favorecer; gerar; ocorrer | acender (uma fogueira); fazer o combustível queimar}
  \seealsoref{好生}{hao3sheng1}
  \seealsoref{学生}{xue2sheng5}
  \end{Phonetics}
\end{Entry}

\begin{Entry}{生日}{5,4}{⽣,⽇}
  \begin{Phonetics}{生日}{sheng1ri5}[][HSK 1]
    \definition[个,次]{s.}{aniversário; dia de nascimento, também se refere ao dia em que se completa um ano de idade a cada ano}
  \end{Phonetics}
\end{Entry}

\begin{Entry}{生气}{5,4}{⽣,⽓}
  \begin{Phonetics}{生气}{sheng1/qi4}[][HSK 1]
    \definition{s.}{vitalidade; vigor; energia da vida}
    \definition{v.+compl.}{ficar com raiva; ficar ofendido; ficar zangado; encontrar algo que não é do seu agrado e sentir-se descontente}
  \end{Phonetics}
\end{Entry}

\begin{Entry}{生长}{5,4}{⽣,⾧}
  \begin{Phonetics}{生长}{sheng1zhang3}[][HSK 3]
    \definition{v.}{cresçer; sob certas condições de vida, o volume e o peso dos organismos aumentam gradualmente | nascer e crescer}
  \end{Phonetics}
\end{Entry}

\begin{Entry}{生平}{5,5}{⽣,⼲}
  \begin{Phonetics}{生平}{sheng1ping2}[][HSK 7-9]
    \definition{s.}{biografia | toda a vida; vida inteira; todo o processo da vida de uma pessoa | desde o nascimento; por toda a minha vida}
  \synonymref{从来}{cong2lai2}
  \synonymref{一生}{yi4sheng1}
  \end{Phonetics}
\end{Entry}

\begin{Entry}{生产}{5,6}{⽣,⼇}
  \begin{Phonetics}{生产}{sheng1chan3}[][HSK 3]
    \definition{v.}{produzir; fabricar; utilizar ferramentas para mudar o objeto de trabalho e criar meios de produção e meios de subsistência | dar à luz uma criança; ter filhos}
  \end{Phonetics}
\end{Entry}

\begin{Entry}{生动}{5,6}{⽣,⼒}
  \begin{Phonetics}{生动}{sheng1dong4}[][HSK 3]
    \definition{adj.}{vívido; animado; descreve a linguagem e as formas de expressão como sendo ativas e em movimento}
  \end{Phonetics}
\end{Entry}

\begin{Entry}{生存}{5,6}{⽣,⼦}
  \begin{Phonetics}{生存}{sheng1cun2}[][HSK 3]
    \definition{v.}{viver; sobreviver; subsistir; manter a vida; estar vivo}
  \end{Phonetics}
\end{Entry}

\begin{Entry}{生成}{5,6}{⽣,⼽}
  \begin{Phonetics}{生成}{sheng1cheng2}[][HSK 5]
    \definition{v.}{formar; gerar; produzir | ter por natureza; nascer com}
  \end{Phonetics}
\end{Entry}

\begin{Entry}{生机}{5,6}{⽣,⽊}
  \begin{Phonetics}{生机}{sheng1ji1}[][HSK 7-9]
    \definition{s.}{chance de sobrevivência | vitalidade}
  \synonymref{活力}{huo2li4}
  \synonymref{期望}{qi1wang4}
  \synonymref{希望}{xi1wang4}
  \end{Phonetics}
\end{Entry}

\begin{Entry}{生死}{5,6}{⽣,⽍}
  \begin{Phonetics}{生死}{sheng1si3}[][HSK 7-9]
    \definition{s.}{vida e morte}
  \end{Phonetics}
\end{Entry}

\begin{Entry}{生词}{5,7}{⽣,⾔}
  \begin{Phonetics}{生词}{sheng1ci2}[][HSK 2]
    \definition[个,组,堆,条]{s.}{nova palavra; palavras que não aprendi, não conheço ou não entendo}
  \end{Phonetics}
\end{Entry}

\begin{Entry}{生命}{5,8}{⽣,⼝}
  \begin{Phonetics}{生命}{sheng1ming4}[][HSK 3]
    \definition{s.}{vida; não envolve apenas a existência e as atividades dos organismos, mas também inclui experiências de vida humana, valores e elementos-chave da sobrevivência e do desenvolvimento de várias coisas}
  \end{Phonetics}
\end{Entry}

\begin{Entry}{生命线}{5,8,8}{⽣,⼝,⽷}
  \begin{Phonetics}{生命线}{sheng1ming4xian4}[][HSK 7-9]
    \definition{s.}{linha da vida; força vital}
  \end{Phonetics}
\end{Entry}

\begin{Entry}{生态}{5,8}{⽣,⼼}
  \begin{Phonetics}{生态}{sheng1tai4}[][HSK 7-9]
    \definition[种]{s.}{ecologia; ecossistema; habitat do organismo; refere"-se às condições de vida e às inter"-relações de vários organismos em um determinado ambiente natural; refere"-se também às características fisiológicas e aos hábitos de vida dos organismos}
  \synonymref{天然}{tian1ran2}
  \synonymref{自然}{zi4ran5}
  \end{Phonetics}
\end{Entry}

\begin{Entry}{生怕}{5,8}{⽣,⼼}
  \begin{Phonetics}{生怕}{sheng1pa4}[][HSK 7-9]
    \definition{v.}{termer; recear; estar com medo de; estar com receio de; estar preocupado}
  \synonymref{恐怕}{kong3pa4}
  \end{Phonetics}
\end{Entry}

\begin{Entry}{生物}{5,8}{⽣,⽜}
  \begin{Phonetics}{生物}{sheng1wu4}[][HSK 7-9]
    \definition{adj.}{biológico}
    \definition[种,个]{s.}{organismo; ser vivo; todos os seres vivos na natureza, incluindo animais, plantas e microrganismos | biologia; a disciplina que estuda diversos organismos}
  \synonymref{动物}{dong4wu4}
  \synonymref{生命}{sheng1ming4}
  \end{Phonetics}
\end{Entry}

\begin{Entry}{生的}{5,8}{⽣,⽩}
  \begin{Phonetics}{生的}{sheng1de5}
    \definition{conj.}{para evitar isso | para que\dots não\dots}
  \end{Phonetics}
\end{Entry}

\begin{Entry}{生育}{5,8}{⽣,⾁}
  \begin{Phonetics}{生育}{sheng1yu4}[][HSK 7-9]
    \definition{v.}{dar à luz; ter um bebê; parir}
  \end{Phonetics}
\end{Entry}

\begin{Entry}{生鱼片}{5,8,4}{⽣,⿂,⽚}
  \begin{Phonetics}{生鱼片}{sheng1yu2pian4}
    \definition{s.}{fatias de peixe cru, \emph{sashimi}}
  \end{Phonetics}
\end{Entry}

\begin{Entry}{生前}{5,9}{⽣,⼑}
  \begin{Phonetics}{生前}{sheng1qian2}[][HSK 7-9]
    \definition[出]{s.}{antes da morte; durante a vida}
  \end{Phonetics}
\end{Entry}

\begin{Entry}{生活}{5,9}{⽣,⽔}
  \begin{Phonetics}{生活}{sheng1huo2}[][HSK 2]
    \definition[个,段,种]{s.}{vida; subsistência; as diversas atividades realizadas por pessoas ou seres vivos para sobreviver e se desenvolver | estilo de vida; condições de vida; situação em termos de vestuário, alimentação, habitação e transporte | trabalho (principalmente nas áreas industrial, agrícola e artesanal)}
    \definition{v.}{viver; realizar várias atividades | sobreviver}
  \end{Phonetics}
\end{Entry}

\begin{Entry}{生活垃圾}{5,9,8,6}{⽣,⽔,⼟,⼟}
  \begin{Phonetics}{生活垃圾}{sheng1huo2la1ji1}
    \definition{s.}{lixo doméstico}
  \end{Phonetics}
\end{Entry}

\begin{Entry}{生活型}{5,9,9}{⽣,⽔,⼟}
  \begin{Phonetics}{生活型}{sheng1huo2 xing2}
    \definition{s.}{forma de vida}
  \end{Phonetics}
\end{Entry}

\begin{Entry}{生活费}{5,9,9}{⽣,⽔,⾙}
  \begin{Phonetics}{生活费}{sheng1huo2fei4}[][HSK 6]
    \definition{s.}{subsídio; despesas de subsistência; despesas necessárias para manter a vida diária}
  \end{Phonetics}
\end{Entry}

\begin{Entry}{生效}{5,10}{⽣,⽁}
  \begin{Phonetics}{生效}{sheng1/xiao4}[][HSK 7-9]
    \definition{v.+compl.}{entrar em vigor; tornar-se efetivo; produzir efeitos}
  \end{Phonetics}
\end{Entry}

\begin{Entry}{生病}{5,10}{⽣,⽧}
  \begin{Phonetics}{生病}{sheng1/bing4}[][HSK 1]
    \definition{v.+compl.}{adoecer; ficar doente; ficar mal; contrair uma doença}
  \end{Phonetics}
\end{Entry}

\begin{Entry}{生涯}{5,11}{⽣,⽔}
  \begin{Phonetics}{生涯}{sheng1ya2}[][HSK 7-9]
    \definition{s.}{carreira; profissão; refere-se à vida dedicada a uma determinada atividade ou profissão}
  \synonymref{生活}{sheng1huo2}
  \end{Phonetics}
\end{Entry}

\begin{Entry}{生理}{5,11}{⽣,⽟}
  \begin{Phonetics}{生理}{sheng1li3}[][HSK 7-9]
    \definition{adj.}{fisiológico}
    \definition{s.}{fisiologia; as atividades vitais do corpo e as funções de seus diversos órgãos}
  \end{Phonetics}
\end{Entry}

\begin{Entry}{生菜}{5,11}{⽣,⾋}
  \begin{Phonetics}{生菜}{sheng1cai4}
    \definition{s.}{alface}
  \end{Phonetics}
\end{Entry}

\begin{Entry}{生疏}{5,12}{⽣,⽦}
  \begin{Phonetics}{生疏}{sheng1shu1}
    \definition{adj.}{inexperiência; desconhecimento de assuntos ou situações, falta de experiência suficiente | falta de prática; habilidades e ofícios enferrujam devido à falta de uso ao longo de um longo período de tempo | não tão perto como antes; as relações entre as pessoas tornaram-se menos íntimas e mais distantes do que antes}
  \end{Phonetics}
\end{Entry}

\begin{Entry}{生硬}{5,12}{⽣,⽯}
  \begin{Phonetics}{生硬}{sheng1ying4}[][HSK 7-9]
    \definition{adj.}{grosseiro; artificial; não suave (na escrita); foi feito com relutância; foi antinatural; não foi habilidoso | rombudo; rígido; grosseiro; duro; não é gentil; não é meticuloso}
  \synonymref{生疏}{sheng1shu1}
  \antonymref{流利}{liu2li4}
  \antonymref{柔和}{rou2he2}
  \antonymref{熟练}{shu2lian4}
  \antonymref{自然}{zi4ran5}
  \end{Phonetics}
\end{Entry}

\begin{Entry}{生意}{5,13}{⽣,⼼}
  \begin{Phonetics}{生意}{sheng1yi4}
    \definition[笔,种,次]{s.}{tendência a crescer; vitalidade; vigor; energia}
  \end{Phonetics}
  \begin{Phonetics}{生意}{sheng1yi5}[][HSK 3]
    \definition[笔,种,次]{s.}{comércio, compra e venda; negócios; indústria; colegas do mesmo setor}
  \end{Phonetics}
\end{Entry}

%%%%% EOF %%%%%

