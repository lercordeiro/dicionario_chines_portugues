%%%
%%% Radical "⾱"
%%%
\section*{Radical 178: ``⾱'' (韦)}\addcontentsline{toc}{section}{Radical 178: ⾱、韦}\addcontentsline{loh}{figure}{\#\#\#\# 178: ⾱}

%%%%%%%%%% 韧 %%%%%%%%%%
\subsection*{韧}\addcontentsline{loh}{figure}{韧}

\begin{Entry}{韧}{7}{⾱}
  \begin{Phonetics}{韧}{ren4}
    \definition{adj.}{flexível, mas forte; tenaz; resistente | resistente; macio e forte, não quebra facilmente}
  \antonymref{脆}{cui4}
  \end{Phonetics}
\end{Entry}

\begin{Entry}{韧性}{7,8}{⾱,⼼}
  \begin{Phonetics}{韧性}{ren4xing4}[][HSK 7-9]
    \definition{s.}{ductilidade; tenacidade; resistência; propriedades de um objeto: macio, porém resistente, e não quebra facilmente | tenacidade; refere"-se a um espírito de perseverança e tenacidade}
  \end{Phonetics}
\end{Entry}

%%%%%%%%%% 韩 %%%%%%%%%%
\subsection*{韩}\addcontentsline{loh}{figure}{韩}

\begin{Entry}{韩}{12}{⾱}
  \begin{Phonetics}{韩}{han2}
    \definition*{s.}{Um estado durante o Período dos Estados Combatentes nas atuais províncias centrais de Henan e sudeste de Shanxi | O nome de um estado feudal durante a dinastia Zhou, localizado no que hoje é o nordeste de Hejin, província de Shanxi | Coreia do Sul, abreviação de 韩国; República da Coreia (RC) | Sobrenome: Han}
  \seealsoref{韩国}{han2guo2}
  \end{Phonetics}
\end{Entry}

\begin{Entry}{韩国}{12,8}{⾱,⼞}
  \begin{Phonetics}{韩国}{han2guo2}
    \definition*{s.}{Coréia do Sul; República da Coreia}
  \end{Phonetics}
\end{Entry}

\begin{Entry}{韩国人}{12,8,2}{⾱,⼞,⼈}
  \begin{Phonetics}{韩国人}{han2guo2ren2}
    \definition{s.}{coreano | pessoa ou povo da Coréia}
  \end{Phonetics}
\end{Entry}

%%%%% EOF %%%%%

