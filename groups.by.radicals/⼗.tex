%%%
%%% Radical "⼗"
%%%
\section*{Radical 24: ``⼗''}\addcontentsline{toc}{section}{Radical 24: ⼗}\addcontentsline{loh}{figure}{\#\#\#\# 24: ⼗}

%%%%%%%%%% 十 %%%%%%%%%%
\subsection*{十}\addcontentsline{loh}{figure}{十}

\begin{Entry}{十}{2}{⼗}[Kangxi 24]
  \begin{Phonetics}{十}{shi2}[][HSK 1]
    \definition*{s.}{Sobrenome: Shi}
    \definition{num.}{dez; 10 | dezena | completo; no topo; máximo; referindo"-se a algo que atingiu o ápice da perfeição ou plenitude | um monte de; indica que há muitos}
  \end{Phonetics}
\end{Entry}

\begin{Entry}{十分}{2,4}{⼗,⼑}
  \begin{Phonetics}{十分}{shi2fen1}[][HSK 2]
    \definition{adv.}{muito; totalmente; completamente; extremamente; indica um nível muito alto}
  \end{Phonetics}
\end{Entry}

\begin{Entry}{十字路口}{2,6,13,3}{⼗,⼦,⾜,⼝}
  \begin{Phonetics}{十字路口}{shi2zi4 lu4kou3}[][HSK 7-9]
    \definition{expr.}{encruzilhada; cruzamento; interseção; ponto de virada; uma encruzilhada, um lugar onde duas estradas se cruzam, é uma metáfora para uma situação em que se deve escolher um caminho a seguir em uma questão crucial}
  \end{Phonetics}
\end{Entry}

\begin{Entry}{十足}{2,7}{⼗,⾜}
  \begin{Phonetics}{十足}{shi2zu2}[][HSK 5]
    \definition{adj.}{puro e simples; apenas este componente ou esta característica é muito evidente | 100\%; completo; total; muito satisfatório; muito adequado}
  \end{Phonetics}
\end{Entry}

%%%%%%%%%% 千 %%%%%%%%%%
\subsection*{千}\addcontentsline{loh}{figure}{千}

\begin{Entry}{千}{3}{⼗}
  \begin{Phonetics}{千}{qian1}[][HSK 2]
    \definition*{s.}{Sobrenome: Qian}
    \definition{num.}{mil; 1.000; 1000 | a grande quantidade de; um grande número de}
  \end{Phonetics}
\end{Entry}

\begin{Entry}{千万}{3,3}{⼗,⼀}
  \begin{Phonetics}{千万}{qian1wan4}[][HSK 3]
    \definition{adv.}{(usado para indicar desejos fortes) por todos os meios; sob quaisquer circunstâncias; expressa uma exortação sincera, equivalente a 务必}
    \definition{num.}{dez milhões; 10.000.000; 1000.0000; milhões e milhões; um número aproximado, indicando um grande número}
  \seealsoref{务必}{wu4bi4}
  \end{Phonetics}
\end{Entry}

\begin{Entry}{千千万万}{3,3,3,3}{⼗,⼗,⼀,⼀}
  \begin{Phonetics}{千千万万}{qian1qian1wan4wan4}
    \definition{expr.}{inumerável; números incontáveis; milhares e milhares}
  \end{Phonetics}
\end{Entry}

\begin{Entry}{千方百计}{3,4,6,4}{⼗,⽅,⽩,⾔}
  \begin{Phonetics}{千方百计}{qian1fang1-bai3ji4}[][HSK 7-9]
    \definition{expr.}{por todos os meios; fazer tudo o que for possível; descreve alguém que esgotou todos os meios ou métodos}
  \end{Phonetics}
\end{Entry}

\begin{Entry}{千古}{3,5}{⼗,⼝}
  \begin{Phonetics}{千古}{qian1gu3}
    \definition{adv.}{por toda a eternidade | em todas as idades}
    \definition{s.}{eternidade (usada em um dístico elegíaco, coroa de flores, etc., dedicada aos mortos)}
  \end{Phonetics}
\end{Entry}

\begin{Entry}{千军万马}{3,6,3,3}{⼗,⼍,⼀,⾺}
  \begin{Phonetics}{千军万马}{qian1jun1-wan4ma3}[][HSK 7-9]
    \definition{expr.}{``Milhares de soldados.''; milhares e milhares de homens e cavalos; um exército poderoso; uma força imensa; todos os cavalos do rei e todos os homens do rei; exército magnífico com milhares de homens e cavalos; demonstração impressionante de força humana}
  \end{Phonetics}
\end{Entry}

\begin{Entry}{千年}{3,6}{⼗,⼲}
  \begin{Phonetics}{千年}{qian1nian2}
    \definition{s.}{milênio}
  \end{Phonetics}
\end{Entry}

\begin{Entry}{千克}{3,7}{⼗,⼗}
  \begin{Phonetics}{千克}{qian1ke4}[][HSK 2]
    \definition{clas.}{kg; quilo; quilograma; 1 quilograma equivale a 1.000 gramas, ou 2 jin (斤)}
  \seealsoref{斤}{jin1}
  \end{Phonetics}
\end{Entry}

\begin{Entry}{千变万化}{3,8,3,4}{⼗,⼜,⼀,⼔}
  \begin{Phonetics}{千变万化}{qian1bian4-wan4hua4}[][HSK 7-9]
    \definition{expr.}{``Sempre em mudança.''; as miríades de mudanças; mudança caleidoscópica; mudanças intermináveis; em constante transformação; ser infinito em variedade; mudanças infinitas; em constante mudança}
  \end{Phonetics}
\end{Entry}

\begin{Entry}{千钧一发}{3,9,1,5}{⼗,⾦,⼀,⼜}
  \begin{Phonetics}{千钧一发}{qian1jun1-yi1fa4}[][HSK 7-9]
    \definition{expr.}{``Por pouco não deu certo.''; cem pesos pendurados por um fio; em perigo iminente; uma questão de vida ou morte}
  \end{Phonetics}
\end{Entry}

\begin{Entry}{千家万户}{3,10,3,4}{⼗,⼧,⼀,⼾}
  \begin{Phonetics}{千家万户}{qian1jia1-wan4hu4}[][HSK 7-9]
    \definition{expr.}{``Milhares de famílias.''; inúmeras famílias; todas as famílias}
  \end{Phonetics}
\end{Entry}

%%%%%%%%%% 升 %%%%%%%%%%
\subsection*{升}\addcontentsline{loh}{figure}{升}

\begin{Entry}{升}{4}{⼗}
  \begin{Phonetics}{升}{sheng1}[][HSK 3]
    \definition*{s.}{Sobrenome: Sheng}
    \definition{clas.}{litro (l)}
    \definition{s.}{sheng, uma unidade de medida seca para grãos (= 1 litro), um décimo de 斗}
    \definition{v.}{elevar; içar; subir; ascender; subir ou subir mais alto | promover; melhorar (nível)}
  \seealsoref{斗}{dou4}
  \antonymref{降}{jiang4}
  \end{Phonetics}
\end{Entry}

\begin{Entry}{升级}{4,6}{⼗,⽷}
  \begin{Phonetics}{升级}{sheng1/ji2}[][HSK 6]
    \definition{v.+compl.}{atualizar (software) | (guerra) escalar; (tensão) aprofundar | subir um ou mais níveis; passar de uma série ou classe inferior para uma série ou classe superior}
  \end{Phonetics}
\end{Entry}

\begin{Entry}{升学}{4,8}{⼗,⼦}
  \begin{Phonetics}{升学}{sheng1 xue2}[][HSK 6]
    \definition{v.}{ir para uma universidade, faculdade; entrar em uma universidade, faculdade}
  \end{Phonetics}
\end{Entry}

\begin{Entry}{升值}{4,10}{⼗,⼈}
  \begin{Phonetics}{升值}{sheng1zhi2}[][HSK 6]
    \definition{v.}{Economia: reavaliar; apreciar | Figurativo: aumento de valor | valorização; apreciação; aumentar o valor; aumentar os preços}
  \end{Phonetics}
\end{Entry}

\begin{Entry}{升起}{4,10}{⼗,⾛}
  \begin{Phonetics}{升起}{sheng1qi3}
    \definition{v.}{levantar | içar | subir}
  \end{Phonetics}
\end{Entry}

\begin{Entry}{升高}{4,10}{⼗,⾼}
  \begin{Phonetics}{升高}{sheng1gao1}[][HSK 5]
    \definition{v.}{subir; ascender | promover; elevar; intensificar; potencializar; melhorar}
  \end{Phonetics}
\end{Entry}

\begin{Entry}{升温}{4,12}{⼗,⽔}
  \begin{Phonetics}{升温}{sheng1wen1}[][HSK 7-9]
    \definition{v.}{aquecer; aumentar (temperatura); metaforicamente, isso também se refere a um aumento no nível de atenção que algo recebe}
  \synonymref{加热}{jia1 re4}
  \end{Phonetics}
\end{Entry}

%%%%%%%%%% 午 %%%%%%%%%%
\subsection*{午}\addcontentsline{loh}{figure}{午}

\begin{Entry}{午}{4}{⼗}
  \begin{Phonetics}{午}{wu3}
    \definition{s.}{meio-dia; período entre 11h00 e 13h00 | wu (sétimo dos doze Ramos Terrestres)}
  \end{Phonetics}
\end{Entry}

\begin{Entry}{午休}{4,6}{⼗,⼈}
  \begin{Phonetics}{午休}{wu3xiu1}
    \definition{s.}{pausa para almoço | cochilo na hora do almoço | intervalo do meio-dia}
  \end{Phonetics}
\end{Entry}

\begin{Entry}{午后}{4,6}{⼗,⼝}
  \begin{Phonetics}{午后}{wu3hou4}
    \definition{s.}{tarde | período da tarde}
  \end{Phonetics}
\end{Entry}

\begin{Entry}{午饭}{4,7}{⼗,⾷}
  \begin{Phonetics}{午饭}{wu3fan4}[][HSK 1]
    \definition[顿]{s.}{almoço}
  \seealsoref{午餐}{wu3can1}
  \end{Phonetics}
\end{Entry}

\begin{Entry}{午夜}{4,8}{⼗,⼣}
  \begin{Phonetics}{午夜}{wu3ye4}
    \definition{s.}{meia-noite}
  \end{Phonetics}
\end{Entry}

\begin{Entry}{午前}{4,9}{⼗,⼑}
  \begin{Phonetics}{午前}{wu3qian2}
    \definition{s.}{\emph{A.M.} | manhã | período da manhã}
  \end{Phonetics}
\end{Entry}

\begin{Entry}{午宴}{4,10}{⼗,⼧}
  \begin{Phonetics}{午宴}{wu3yan4}
    \definition{s.}{banquete de almoço}
  \end{Phonetics}
\end{Entry}

\begin{Entry}{午睡}{4,13}{⼗,⽬}
  \begin{Phonetics}{午睡}{wu3shui4}[][HSK 2]
    \definition{s.}{\emph{siesta}; cochilo da tarde; soneca do meio-dia}
    \definition{v.}{tirar uma soneca depois do almoço}
  \end{Phonetics}
\end{Entry}

\begin{Entry}{午餐}{4,16}{⼗,⾷}
  \begin{Phonetics}{午餐}{wu3can1}[][HSK 2]
    \definition[份,顿,次]{s.}{almoço}
  \seealsoref{午饭}{wu3fan4}
  \end{Phonetics}
\end{Entry}

%%%%%%%%%% 半 %%%%%%%%%%
\subsection*{半}\addcontentsline{loh}{figure}{半}

\begin{Entry}{半}{5}{⼗}
  \begin{Phonetics}{半}{ban4}[][HSK 1]
    \definition{adv.}{parcialmente; usado antes de verbos ou adjetivos para indicar incompletude}
    \definition{num.}{(depois de um número) ``e meio'' | meio; metade | na metade; no meio | muito pouco; o mínimo}
  \end{Phonetics}
\end{Entry}

\begin{Entry}{半天}{5,4}{⼗,⼤}
  \begin{Phonetics}{半天}{ban4tian1}[][HSK 1]
    \definition{s.}{metade do dia; metade do dia dividida pelo meio"-dia | um longo tempo; bastante tempo; refere"-se a um período de tempo relativamente longo (com um tom exagerado)}
  \end{Phonetics}
\end{Entry}

\begin{Entry}{半边天}{5,5,4}{⼗,⾡,⼤}
  \begin{Phonetics}{半边天}{ban4bian1tian1}[][HSK 7-9]
    \definition{s.}{metade do céu; parte do céu | mulheres modernas; mulheres (do ditado de Mao Zedong ``As mulheres podem sustentar metade do céu.'')}
  \end{Phonetics}
\end{Entry}

\begin{Entry}{半决赛}{5,6,14}{⼗,⼎,⾙}
  \begin{Phonetics}{半决赛}{ban4jue2sai4}[][HSK 6]
    \definition{s.}{semifinais}
  \end{Phonetics}
\end{Entry}

\begin{Entry}{半场}{5,6}{⼗,⼟}
  \begin{Phonetics}{半场}{ban4chang3}[][HSK 7-9]
    \definition{s.}{metade de um jogo ou competição (tempo) | meia quadra (no basquete)}
  \end{Phonetics}
\end{Entry}

\begin{Entry}{半年}{5,6}{⼗,⼲}
  \begin{Phonetics}{半年}{ban4nian2}[][HSK 1]
    \definition{s.}{meio ano}
  \end{Phonetics}
\end{Entry}

\begin{Entry}{半岛}{5,7}{⼗,⼭}
  \begin{Phonetics}{半岛}{ban4dao3}[][HSK 7-9]
    \definition[个]{s.}{península; terra que se estende até o mar ou lago, cercada por água em três lados e conectada à terra em um lado}
  \end{Phonetics}
\end{Entry}

\begin{Entry}{半夜}{5,8}{⼗,⼣}
  \begin{Phonetics}{半夜}{ban4ye4}[][HSK 2]
    \definition{s.}{no meio da noite; metade da noite | por volta da meia-noite, também se refere à madrugada}
  \end{Phonetics}
\end{Entry}

\begin{Entry}{半信半疑}{5,9,5,14}{⼗,⼈,⼗,⽦}
  \begin{Phonetics}{半信半疑}{ban4xin4-ban4yi2}[][HSK 7-9]
    \definition{expr.}{meio acreditar e meio duvidar; ser bastante duvidoso sobre (acerca de) uma coisa; estar incerto quanto ao que acreditar; meio seriamente e meio cético; não totalmente convencido; bastante desconfiado | meio acreditando, meio duvidando}
  \end{Phonetics}
\end{Entry}

\begin{Entry}{半音}{5,9}{⼗,⾳}
  \begin{Phonetics}{半音}{ban4yin1}
    \definition{s.}{semitom; na música, uma oitava é dividida em doze notas e o intervalo entre duas notas adjacentes é chamado de semitom}
  \end{Phonetics}
\end{Entry}

\begin{Entry}{半真半假}{5,10,5,11}{⼗,⼗,⼗,⼈}
  \begin{Phonetics}{半真半假}{ban4zhen1-ban4jia3}[][HSK 7-9]
    \definition{expr.}{meio verdadeiro e meio falso | meio genuíno, meio falso; parcialmente verdadeiro, parcialmente falso | meio de brincadeira, meio a sério; meio brincando}
  \end{Phonetics}
\end{Entry}

\begin{Entry}{半途而废}{5,10,6,8}{⼗,⾡,⽽,⼴}
  \begin{Phonetics}{半途而废}{ban4tu2'er2fei4}[][HSK 7-9]
    \definition{expr.}{desistir no meio do caminho; deixar inacabado; fazer algo pela metade; parar no meio do caminho; metaforicamente, parar antes de concluir uma tarefa; não terminar o que foi iniciado}
  \end{Phonetics}
\end{Entry}

\begin{Entry}{半球}{5,11}{⼗,⽟}
  \begin{Phonetics}{半球}{ban4qiu2}
    \definition{s.}{hemisfério}
  \end{Phonetics}
\end{Entry}

\begin{Entry}{半数}{5,13}{⼗,⽁}
  \begin{Phonetics}{半数}{ban4shu4}[][HSK 7-9]
    \definition{s.}{metade do total; metade}
  \end{Phonetics}
\end{Entry}

\begin{Entry}{半路}{5,13}{⼗,⾜}
  \begin{Phonetics}{半路}{ban4lu4}[][HSK 7-9]
    \definition{adv.}{a caminho | em andamento}
    \definition{s.}{na metade do caminho; no meio do caminho}
  \end{Phonetics}
\end{Entry}

%%%%%%%%%% 华 %%%%%%%%%%
\subsection*{华}\addcontentsline{loh}{figure}{华}

\begin{Entry}{华}{6}{⼗}
  \begin{Phonetics}{华}{hua2}
    \definition*{s.}{China; refere"-se à China (anteriormente conhecida como Huaxia, 华夏, mais tarde chamada de Zhonghua, 中华, ou simplesmente Hua, 华)}
    \definition{adj.}{esplêndido; magnífico | próspero; florescente | chamativo; extravagante; vaidoso | grisalho}
    \definition{s.}{corona; um halo colorido ao redor do sol ou da lua causado pela difração da luz através das nuvens | creme; melhor parte; a melhor parte das coisas | chinês; refere"-se à nacionalidade Han (língua e escrita) | vezes; anos; refere"-se a (bons) momentos | elixir; essência líquida; substâncias formadas pela sedimentação de minerais na água de nascente | Seu, palavra honorífica, usada para se referir a coisas relacionadas à outra pessoa}
  \seealsoref{华夏}{hua2xia4}
  \seealsoref{中华}{zhong1hua2}
  \end{Phonetics}
  \begin{Phonetics}{华}{hua4}
    \definition*{s.}{Huashan Mountain (na província de Shaanxi) | Sobrenome: Hua}
  \end{Phonetics}
\end{Entry}

\begin{Entry}{华人}{6,2}{⼗,⼈}
  \begin{Phonetics}{华人}{hua2ren2}[][HSK 3]
    \definition[名,位,个]{s.}{Chinês; chinês étnico | chineses no exterior; refere"-se a cidadãos estrangeiros de ascendência chinesa que obtiveram a nacionalidade do país em que residem}
  \end{Phonetics}
\end{Entry}

\begin{Entry}{华山}{6,3}{⼗,⼭}
  \begin{Phonetics}{华山}{hua4shan1}
    \definition{s.}{Monte Hua em Shaanxi, montanha ocidental das Cinco Montanhas Sagradas (五岳)}
  \seealsoref{五岳}{wu3yue4}
  \end{Phonetics}
\end{Entry}

\begin{Entry}{华氏}{6,4}{⼗,⽒}
  \begin{Phonetics}{华氏}{hua2shi4}
    \definition{s.}{graus Fahrenheit (°F)}
  \end{Phonetics}
\end{Entry}

\begin{Entry}{华丽}{6,7}{⼗,⼀}
  \begin{Phonetics}{华丽}{hua2li4}[][HSK 7-9]
    \definition{adj.}{magnífico; resplandecente; deslumbrante; lindo e radiante}
  \end{Phonetics}
\end{Entry}

\begin{Entry}{华侨}{6,8}{⼗,⼈}
  \begin{Phonetics}{华侨}{hua2qiao2}[][HSK 7-9]
    \definition[个,位,名]{s.}{chineses que vivem no exterior}
  \end{Phonetics}
\end{Entry}

\begin{Entry}{华语}{6,9}{⼗,⾔}
  \begin{Phonetics}{华语}{hua2yu3}[][HSK 5]
    \definition*{s.}{Chinês (idioma)}
  \end{Phonetics}
\end{Entry}

\begin{Entry}{华夏}{6,10}{⼗,⼢}
  \begin{Phonetics}{华夏}{hua2xia4}
    \definition*{s.}{Huaxia, nome antigo da China | Catai}
  \end{Phonetics}
\end{Entry}

\begin{Entry}{华盛顿}{6,11,10}{⼗,⽫,⾴}
  \begin{Phonetics}{华盛顿}{hua2sheng4dun4}
    \definition*{s.}{Washington}
  \end{Phonetics}
\end{Entry}

\begin{Entry}{华裔}{6,13}{⼗,⾐}
  \begin{Phonetics}{华裔}{hua2yi4}[][HSK 7-9]
    \definition[位,名,个]{s.}{etnia chinesa; crianças nascidas de chineses no exterior no país de residência e que adquiriram a nacionalidade do país de residência}
  \end{Phonetics}
\end{Entry}

%%%%%%%%%% 协 %%%%%%%%%%
\subsection*{协}\addcontentsline{loh}{figure}{协}

\begin{Entry}{协}{6}{⼗}
  \begin{Phonetics}{协}{xie2}
    \definition*{s.}{Sobrenome: Xie}
    \definition{adv.}{conjuntamente; coordenadamente; juntos}
    \definition{s.}{harmonioso}
    \definition{v.}{auxiliar; assistir; ajudar}
  \end{Phonetics}
\end{Entry}

\begin{Entry}{协议}{6,5}{⼗,⾔}
  \begin{Phonetics}{协议}{xie2yi4}[][HSK 5]
    \definition[份,项]{s.}{acordo; tratado; decisão conjunta alcançada através de negociação e consulta}
    \definition{v.}{concordar em}
  \end{Phonetics}
\end{Entry}

\begin{Entry}{协议书}{6,5,4}{⼗,⾔,⼄}
  \begin{Phonetics}{协议书}{xie2yi4shu1}[][HSK 5]
    \definition{s.}{contrato | protocolo}
  \end{Phonetics}
\end{Entry}

\begin{Entry}{协会}{6,6}{⼗,⼈}
  \begin{Phonetics}{协会}{xie2hui4}[][HSK 6]
    \definition[个]{s.}{sociedade; instituto; associação; uma organização de massa formada para promover uma causa comum}
  \end{Phonetics}
\end{Entry}

\begin{Entry}{协助}{6,7}{⼗,⼒}
  \begin{Phonetics}{协助}{xie2zhu4}[][HSK 6]
    \definition{v.}{ajudar; auxiliar; dar assistência; fornecer ajuda}
  \end{Phonetics}
\end{Entry}

\begin{Entry}{协调}{6,10}{⼗,⾔}
  \begin{Phonetics}{协调}{xie2tiao2}[][HSK 6]
    \definition{adj.}{coordenado; harmonioso; em sintonia}
    \definition{v.}{coordenar; concertar; integrar; harmonizar; fazer a harmonia apropriada}
  \end{Phonetics}
\end{Entry}

\begin{Entry}{协商}{6,11}{⼗,⼝}
  \begin{Phonetics}{协商}{xie2shang1}[][HSK 6]
    \definition{v.}{discutir; consultar; negociar; várias partes discutiram e decidiram em conjunto para chegar à mesma visão}
  \end{Phonetics}
\end{Entry}

%%%%%%%%%% 克 %%%%%%%%%%
\subsection*{克}\addcontentsline{loh}{figure}{克}

\begin{Entry}{克}{7}{⼗}
  \begin{Phonetics}{克}{ke4}[][HSK 2]
    \definition*{s.}{Sobrenome: Ke}
    \definition{clas.}{g, grama, unidade de peso | unidade tibetana de volume ou medida seca (com capacidade para cerca de 25 斤, de cevada) | unidade tibetana de área de terra equivalente a cerca de 1 亩}
    \definition{v.}{poder; ser capaz de | tolerar; conter; restringir; suprimir| subjugar; capturar; conquistar (uma cidade, etc.) | digerir (alimentos) | reduzir; diminuir | definir um limite de tempo}
  \seealsoref{斤}{jin1}
  \seealsoref{亩}{mu3}
  \end{Phonetics}
\end{Entry}

\begin{Entry}{克制}{7,8}{⼗,⼑}
  \begin{Phonetics}{克制}{ke4zhi4}[][HSK 7-9]
    \definition{v.}{restringir; exercer contenção; exercer autocontrole nas emoções, palavras e ações}
  \end{Phonetics}
\end{Entry}

\begin{Entry}{克服}{7,8}{⼗,⽉}
  \begin{Phonetics}{克服}{ke4fu2}[][HSK 3]
    \definition{v.}{sobrepujar; superar; conquistar; vencer com força de vontade e determinação (deficiências, erros, fenômenos negativos, condições desfavoráveis, etc.) | aguentar; suportar (dificuldades, inconveniências, etc.)}
  \end{Phonetics}
\end{Entry}

\begin{Entry}{克隆}{7,11}{⼗,⾩}
  \begin{Phonetics}{克隆}{ke4long2}[][HSK 7-9]
    \definition{v.}{clonar; geralmente, refere"-se à reprodução assexuada induzida artificialmente | copiar; imitar; metaforicamente, significa copiar (frequentemente usado em um sentido humorístico ou depreciativo)}
  \end{Phonetics}
\end{Entry}

%%%%%%%%%% 丧 %%%%%%%%%%
\subsection*{丧}\addcontentsline{loh}{figure}{丧}

\begin{Entry}{丧}{8}{⼗}
  \begin{Phonetics}{丧}{sang1}
    \definition{adj.}{decepcionado; deprimido; desanimado}
    \definition{v.}{perder | desanimar; frustrar}
  \end{Phonetics}
  \begin{Phonetics}{丧}{sang4}
    \definition{adj.}{decepcionado | desanimado}
    \definition{v.}{estar enlutado (do cônjuge etc.) | morrer}
  \end{Phonetics}
\end{Entry}

\begin{Entry}{丧失}{8,5}{⼗,⼤}
  \begin{Phonetics}{丧失}{sang4shi1}[][HSK 6]
    \definition{v.}{perder (algo que se tem)}
  \end{Phonetics}
\end{Entry}

\begin{Entry}{丧生}{8,5}{⼗,⽣}
  \begin{Phonetics}{丧生}{sang4/sheng1}[][HSK 7-9]
    \definition{v.+compl.}{morrer; encontrar a morte; perder a vida; ser morto}
  \seealsoref{丧身}{sang4shen1}
  \end{Phonetics}
\end{Entry}

\begin{Entry}{丧身}{8,7}{⼗,⾝}
  \begin{Phonetics}{丧身}{sang4shen1}
    \definition{v.}{morrer; perder a vida}
  \seealsoref{丧生}{sang4/sheng1}
  \end{Phonetics}
\end{Entry}

\begin{Entry}{丧钟}{8,9}{⼗,⾦}
  \begin{Phonetics}{丧钟}{sang1zhong1}
    \definition{s.}{sentença de morte}
  \end{Phonetics}
\end{Entry}

%%%%%%%%%% 卑 %%%%%%%%%%
\subsection*{卑}\addcontentsline{loh}{figure}{卑}

\begin{Entry}{卑}{8}{⼗}
  \begin{Phonetics}{卑}{bei1}
    \definition{adj.}{Literário: baixo | inferior; médio | Literário: modesto; humilde}
  \end{Phonetics}
\end{Entry}

\begin{Entry}{卑鄙}{8,13}{⼗,⾢}
  \begin{Phonetics}{卑鄙}{bei1bi3}[][HSK 7-9]
    \definition{adj.}{ruim; vulgar; vil; desprezível}
  \end{Phonetics}
\end{Entry}

%%%%%%%%%% 卒 %%%%%%%%%%
\subsection*{卒}\addcontentsline{loh}{figure}{卒}

\begin{Entry}{卒}{8}{⼗}
  \begin{Phonetics}{卒}{cu4}
    \variantof{猝}
  \end{Phonetics}
  \begin{Phonetics}{卒}{zu2}
    \definition{adv.}{finalmente; enfim}
    \definition{s.}{Obsoleto: soldado; recruta | Obsoleto: servo | peão (uma das peças do xadrez chinês)}
    \definition{v.}{Literário: terminar; finalizar | morrer}
  \end{Phonetics}
\end{Entry}

%%%%%%%%%% 单 %%%%%%%%%%
\subsection*{单}\addcontentsline{loh}{figure}{单}

\begin{Entry}{单}{8}{⼗}
  \begin{Phonetics}{单}{chan2}
    \definition{s.}{usado em 单于 \dpy{chan2yu2}}
  \seealsoref{单于}{chan2yu2}
  \end{Phonetics}
  \begin{Phonetics}{单}{dan1}[][HSK 4]
    \definition*{s.}{Sobrenome: Dan}
    \definition{adj.}{sozinho; único | ímpar; número ímpar | simples; poucos projetos e tipos; estrutura e ideias simples | fino; fraco; frágil}
    \definition{adv.}{isoladamente; sozinho; indica que uma ação ou coisa está dentro de um escopo limitado e não é combinada com outras; equivale a 只 ou 仅}
    \definition[个]{s.}{lençol; um único pedaço grande de pano usado para cobrir | conta; lista; pedaços de papel para anotações detalhadas (geralmente folhas soltas)}
  \seealsoref{仅}{jin3}
  \seealsoref{只}{zhi3}
  \antonymref{双}{shuang1}
  \end{Phonetics}
  \begin{Phonetics}{单}{shan4}
    \definition*{s.}{Sobrenome: Shan}
    \definition{s.}{material de tecido de largura simples (dupla) | número singular (plural)}
  \end{Phonetics}
\end{Entry}

\begin{Entry}{单一}{8,1}{⼗,⼀}
  \begin{Phonetics}{单一}{dan1yi1}[][HSK 5]
    \definition{adj.}{único; unitário; exclusivo}
  \end{Phonetics}
\end{Entry}

\begin{Entry}{单于}{8,3}{⼗,⼆}
  \begin{Phonetics}{单于}{chan2yu2}
    \definition{s.}{rei de Xiongnu (匈奴)}
  \seealsoref{匈奴}{xiong1nu2}
  \end{Phonetics}
\end{Entry}

\begin{Entry}{单元}{8,4}{⼗,⼉}
  \begin{Phonetics}{单元}{dan1yuan2}[][HSK 3]
    \definition[个,组,套]{s.}{unidade (de algo); um conjunto completo, com parágrafos e sistemas próprios, que forma uma unidade independente}
  \end{Phonetics}
\end{Entry}

\begin{Entry}{单方面}{8,4,9}{⼗,⽅,⾯}
  \begin{Phonetics}{单方面}{dan1fang1mian4}[][HSK 7-9]
    \definition{adj.}{unilateral}
    \definition{adv.}{unilateralmente}
  \end{Phonetics}
\end{Entry}

\begin{Entry}{单打}{8,5}{⼗,⼿}
  \begin{Phonetics}{单打}{dan1da3}[][HSK 6]
    \definition[场,局,次]{s.}{Esporte: simples; competição um contra um}
  \end{Phonetics}
\end{Entry}

\begin{Entry}{单边}{8,5}{⼗,⾡}
  \begin{Phonetics}{单边}{dan1bian1}[][HSK 7-9]
    \definition{adj.}{unilateral}
  \end{Phonetics}
\end{Entry}

\begin{Entry}{单向}{8,6}{⼗,⼝}
  \begin{Phonetics}{单向}{dan1xiang4}
    \definition{adj.}{de mão única; unidirecional (oposto de 双向)}
  \seealsoref{双向}{shuang1xiang4}
  \end{Phonetics}
\end{Entry}

\begin{Entry}{单位}{8,7}{⼗,⼈}
  \begin{Phonetics}{单位}{dan1wei4}[][HSK 2]
    \definition[个,家]{s.}{unidade (como padrão de medida) | unidade (como uma organização, departamento, divisão, seção, etc.) | unidade (grupo de pessoas como um todo) | unidade de trabalho (local de trabalho, especialmente na República Popular da China antes da reforma econômica)}
  \end{Phonetics}
\end{Entry}

\begin{Entry}{单纯}{8,7}{⼗,⽷}
  \begin{Phonetics}{单纯}{dan1chun2}[][HSK 4]
    \definition{adj.}{puro; simples; descomplicado}
    \definition{adv.}{sozinho; puramente; meramente}
  \end{Phonetics}
\end{Entry}

\begin{Entry}{单身}{8,7}{⼗,⾝}
  \begin{Phonetics}{单身}{dan1shen1}[][HSK 7-9]
    \definition{s.}{solteiro}
  \end{Phonetics}
\end{Entry}

\begin{Entry}{单单}{8,8}{⼗,⼗}
  \begin{Phonetics}{单单}{dan1dan1}
    \definition{adv.}{somente; sozinho; exceto; indica a identificação de um indivíduo dentro de um grupo geral de pessoas ou coisas}[别人都去了,单单她没去。===Todos os outros foram, somente ela não foi.]
  \end{Phonetics}
\end{Entry}

\begin{Entry}{单质}{8,8}{⼗,⾙}
  \begin{Phonetics}{单质}{dan1zhi4}
    \definition{s.}{substância simples (consistindo puramente de um elemento, como diamante, grafite, etc.)}
  \end{Phonetics}
\end{Entry}

\begin{Entry}{单独}{8,9}{⼗,⽝}
  \begin{Phonetics}{单独}{dan1du2}[][HSK 4]
    \definition{adv.}{solo; sozinho; por si mesmo; por conta própria}
  \end{Phonetics}
\end{Entry}

\begin{Entry}{单调}{8,10}{⼗,⾔}
  \begin{Phonetics}{单调}{dan1diao4}[][HSK 4]
    \definition{adj.}{maçante; monótono}
  \end{Phonetics}
\end{Entry}

\begin{Entry}{单脚滑行车}{8,11,12,6,4}{⼗,⾁,⽔,⾏,⾞}
  \begin{Phonetics}{单脚滑行车}{dan1jiao3hua2xing2che1}
    \definition{s.}{\emph{scooter}}
  \end{Phonetics}
\end{Entry}

\begin{Entry}{单薄}{8,16}{⼗,⾋}
  \begin{Phonetics}{单薄}{dan1bo2}[][HSK 7-9]
    \definition{adj.}{fino; pouco | frágil; magro e fraco | fino; frágil; insubstancial}
  \end{Phonetics}
\end{Entry}

%%%%%%%%%% 卖 %%%%%%%%%%
\subsection*{卖}\addcontentsline{loh}{figure}{卖}

\begin{Entry}{卖}{8}{⼗}
  \begin{Phonetics}{卖}{mai4}[][HSK 2]
    \definition*{s.}{Sobrenome: Mai}
    \definition{clas.}{um prato (nos tempos antigos); antigamente, os restaurantes chamavam cada prato vendido de 一卖 (uma porção)}
    \definition{v.}{vender | trair (o próprio país ou amigos); alcançar objetivos pessoais à custa dos interesses do país, da nação e dos outros | não poupar esforços; esforçar-se ao máximo; tentar fazer o máximo possível | mostrar-se intencionalmente; exibir-se | vender o próprio trabalho; trabalhar em troca de dinheiro}
  \antonymref{买}{mai3}
  \end{Phonetics}
\end{Entry}

\begin{Entry}{卖弄}{8,7}{⼗,⼶}
  \begin{Phonetics}{卖弄}{mai4nong5}[][HSK 7-9]
    \definition{v.}{exibir-se; desfilar; exibir ou ostentar intencionalmente (as próprias habilidades)}
  \end{Phonetics}
\end{Entry}

%%%%%%%%%% 南 %%%%%%%%%%
\subsection*{南}\addcontentsline{loh}{figure}{南}

\begin{Entry}{南}{9}{⼗}
  \begin{Phonetics}{南}{nan2}[][HSK 1]
    \definition*{s.}{Sobrenome: Nan}
    \definition{s.}{sul; uma das quatro direções básicas, o lado direito quando se está de frente para o sol pela manhã | especificamente no sul da China}
  \antonymref{北}{bei3}
  \end{Phonetics}
\end{Entry}

\begin{Entry}{南方}{9,4}{⼗,⽅}
  \begin{Phonetics}{南方}{nan2fang1}[][HSK 2]
    \definition{s.}{sul; indica a direção sul | o sul; a região sul}
  \end{Phonetics}
\end{Entry}

\begin{Entry}{南北}{9,5}{⼗,⼔}
  \begin{Phonetics}{南北}{nan2bei3}[][HSK 5]
    \definition{s.}{(território) norte e sul | (distância) de norte a sul}
  \end{Phonetics}
\end{Entry}

\begin{Entry}{南瓜}{9,5}{⼗,⽠}
  \begin{Phonetics}{南瓜}{nan2gua1}[][HSK 7-9]
    \definition[个,斤,块]{s.}{abóbora}
  \end{Phonetics}
\end{Entry}

\begin{Entry}{南边}{9,5}{⼗,⾡}
  \begin{Phonetics}{南边}{nan2bian5}[][HSK 1]
    \definition{s.}{sul; lado sul}
  \end{Phonetics}
\end{Entry}

\begin{Entry}{南极}{9,7}{⼗,⽊}
  \begin{Phonetics}{南极}{nan2ji2}[][HSK 5]
    \definition*{s.}{Polo Sul; Polo Antártico | Polo sul magnético}
    \definition{s.}{polo sul magnético}
  \end{Phonetics}
\end{Entry}

\begin{Entry}{南京}{9,8}{⼗,⼇}
  \begin{Phonetics}{南京}{nan2jing1}
    \definition*{s.}{Nanquim, capital da província de Jiangsu, 江苏}
  \seealsoref{江苏}{jiang1su1}
  \end{Phonetics}
\end{Entry}

\begin{Entry}{南面}{9,9}{⼗,⾯}
  \begin{Phonetics}{南面}{nan2mian4}
    \definition{s.}{sul | lado sul}
  \end{Phonetics}
\end{Entry}

\begin{Entry}{南部}{9,10}{⼗,⾢}
  \begin{Phonetics}{南部}{nan2bu4}[][HSK 3]
    \definition{s.}{parte sul; sul | a parte sul}
  \end{Phonetics}
\end{Entry}

%%%%%%%%%% 真 %%%%%%%%%%
\subsection*{真}\addcontentsline{loh}{figure}{真}

\begin{Entry}{真}{10}{⼗}
  \begin{Phonetics}{真}{zhen1}[][HSK 1]
    \definition*{s.}{Sobrenome: Zhen}
    \definition{adj.}{verdadeiro; real; genuíno | claro; inequívoco | genuíno; conforme os fatos objetivos | sincero}
    \definition{adv.}{realmente; verdadeiramente; de fato}
    \definition{s.}{escrita regular | retrato; imagem; cópia exata de algo | instintos naturais (ou caráter, disposição); natureza; qualidade inerente; origem | estado original; refere"-se à forma original das coisas}
  \antonymref{假}{jia4}
  \antonymref{伪}{wei3}
  \end{Phonetics}
\end{Entry}

\begin{Entry}{真切}{10,4}{⼗,⼑}
  \begin{Phonetics}{真切}{zhen1qie4}
    \definition{adj.}{claro | distinto | honesto | sincero | vívido}
  \end{Phonetics}
\end{Entry}

\begin{Entry}{真心}{10,4}{⼗,⼼}
  \begin{Phonetics}{真心}{zhen1xin1}
    \definition{adj.}{sincero}
    \definition[片]{s.}{sinceridade}
  \end{Phonetics}
\end{Entry}

\begin{Entry}{真牛}{10,4}{⼗,⽜}
  \begin{Phonetics}{真牛}{zhen1niu2}
    \definition{adj.}{(gíria) muito legal, incrível}
  \end{Phonetics}
\end{Entry}

\begin{Entry}{真正}{10,5}{⼗,⽌}
  \begin{Phonetics}{真正}{zhen1zheng4}[][HSK 2]
    \definition{adj.}{verdadeiro; real; genuíno}
    \definition{adv.}{realmente; de fato; expressa afirmação de uma ação ou situação, equivalente a 确实}
  \seealsoref{确实}{que4shi2}
  \end{Phonetics}
\end{Entry}

\begin{Entry}{真声}{10,7}{⼗,⼠}
  \begin{Phonetics}{真声}{zhen1sheng1}
    \definition{s.}{voz modal; voz natural; voz verdadeira}
  \antonymref{假声}{jia3sheng1}
  \end{Phonetics}
\end{Entry}

\begin{Entry}{真实}{10,8}{⼗,⼧}
  \begin{Phonetics}{真实}{zhen1shi2}[][HSK 3]
    \definition{adj.}{verdadeiro; real; autêntico; de acordo com fatos objetivos}
  \end{Phonetics}
\end{Entry}

\begin{Entry}{真的}{10,8}{⼗,⽩}
  \begin{Phonetics}{真的}{zhen1 de5}[][HSK 1]
    \definition{adv.}{realmente; salientar que a situação existe realmente | verdadeiramente; realmente; existente na realidade; consistente com os fatos objetivos}
  \end{Phonetics}
\end{Entry}

\begin{Entry}{真诚}{10,8}{⼗,⾔}
  \begin{Phonetics}{真诚}{zhen1cheng2}[][HSK 5]
    \definition{adj.}{verdadeiro; honesto; sério; sincero; genuíno; descreve uma pessoa que fala e age com sinceridade, de coração, fazendo com que os outros acreditem nela}
  \end{Phonetics}
\end{Entry}

\begin{Entry}{真相}{10,9}{⼗,⽬}
  \begin{Phonetics}{真相}{zhen1xiang4}[][HSK 5]
    \definition[个]{s.}{face; verdade; verdade nua e crua; a situação real; o estado real das coisas; a verdadeira situação}
  \end{Phonetics}
\end{Entry}

\begin{Entry}{真珠}{10,10}{⼗,⽟}
  \begin{Phonetics}{真珠}{zhen1zhu1}
    \variantof{珍珠}
  \end{Phonetics}
\end{Entry}

\begin{Entry}{真真}{10,10}{⼗,⼗}
  \begin{Phonetics}{真真}{zhen1zhen1}
    \definition{adv.}{genuinamente | realmente | escrupulosamente}
  \end{Phonetics}
\end{Entry}

\begin{Entry}{真理}{10,11}{⼗,⽟}
  \begin{Phonetics}{真理}{zhen1li3}[][HSK 5]
    \definition[条,个]{s.}{verdade; o reflexo correto das coisas objetivas e suas leis no cérebro humano}
  \end{Phonetics}
\end{Entry}

\begin{Entry}{真释}{10,12}{⼗,⾤}
  \begin{Phonetics}{真释}{zhen1shi4}
    \definition{s.}{razão genuína | explicação verdadeira}
  \end{Phonetics}
\end{Entry}

%%%%%%%%%% 博 %%%%%%%%%%
\subsection*{博}\addcontentsline{loh}{figure}{博}

\begin{Entry}{博}{12}{⼗}
  \begin{Phonetics}{博}{bo2}
    \definition*{s.}{Sobrenome: Bo}
    \definition{adj.}{rico; abundante | erudito; bem informado | solto; grande | grande}
    \definition{s.}{doutor em filosofia; doutorado}
    \definition{v.}{ter um amplo conhecimento de; ser bem lido | ganhar; vencer | jogar}
  \end{Phonetics}
\end{Entry}

\begin{Entry}{博士}{12,3}{⼗,⼠}
  \begin{Phonetics}{博士}{bo2shi4}[][HSK 5]
    \definition[位,名,个,些]{s.}{doutorado; grau de doutor; nível mais alto de um diploma; também, uma pessoa que obteve esse diploma | doutor; antigo título honorífico para uma pessoa que é habilidosa em um determinado ofício ou especializada em uma determinada ocupação | doutor; autoridades que ensinavam as escrituras na China nos tempos antigos}
  \end{Phonetics}
\end{Entry}

\begin{Entry}{博文}{12,4}{⼗,⽂}
  \begin{Phonetics}{博文}{bo2wen2}
    \definition{s.}{artigo em um blog}
    \definition{v.}{escrever um artigo em um blog}
  \end{Phonetics}
\end{Entry}

\begin{Entry}{博主}{12,5}{⼗,⼂}
  \begin{Phonetics}{博主}{bo2zhu3}
    \definition{s.}{blogueiro}
  \end{Phonetics}
\end{Entry}

\begin{Entry}{博物馆}{12,8,11}{⼗,⽜,⾷}
  \begin{Phonetics}{博物馆}{bo2wu4guan3}[][HSK 5]
    \definition[座,个]{s.}{museu; locais para coleta, armazenamento, pesquisa, exibição e exposição de relíquias culturais ou espécimes relacionados à história, cultura, arte, ciências naturais, ciência e tecnologia, etc.}
  \end{Phonetics}
\end{Entry}

\begin{Entry}{博客}{12,9}{⼗,⼧}
  \begin{Phonetics}{博客}{bo2ke4}[][HSK 5]
    \definition[个]{s.}{\emph{blog}; página da Web ou site gerenciado por um indivíduo, geralmente composto por postagens organizadas da mais recente para a mais antiga | blogueiro; \emph{blogger}; pessoas que possuem ou escrevem \emph{blogs}}
  \end{Phonetics}
\end{Entry}

\begin{Entry}{博览会}{12,9,6}{⼗,⾒,⼈}
  \begin{Phonetics}{博览会}{bo2lan3hui4}[][HSK 5]
    \definition[次,届]{s.}{exposição; feira internacional; exposições de produtos em grande escala}
  \end{Phonetics}
\end{Entry}

%%%%% EOF %%%%%

