%%%
%%% Radical "⽁"
%%%
\section*{Radical 66: ``⽁'' (攵)}\addcontentsline{toc}{section}{Radical 66: ⽁,攵}\addcontentsline{loh}{figure}{\#\#\#\# 66: ⽁}

%%%%%%%%%% 收 %%%%%%%%%%
\subsection*{收}\addcontentsline{loh}{figure}{收}

\begin{Entry}{收}{6}{⽁}
  \begin{Phonetics}{收}{shou1}[][HSK 2]
    \definition{expr.}{aos cuidados de (usado na linha de endereço após o nome)}
    \definition{v.}{recolocar; juntar; reunir e juntar coisas espalhadas ou dispersas | recolher; cobrar | ganhar; obter (benefícios econômicos) | colher; recolher; colher ou cortar frutas, legumes, cereais maduros, etc. | aceitar; receber; acolher | controlar; restringir; restringir, controlar os sentimentos ou ações, para voltar ao estado normal | finalizar; parar; concluir; encerrar | prender; deter; colocar sob custódia}
  \end{Phonetics}
\end{Entry}

\begin{Entry}{收入}{6,2}{⽁,⼊}
  \begin{Phonetics}{收入}{shou1ru4}[][HSK 2]
    \definition[笔,个]{s.}{renda; salário; dinheiro recebido}
    \definition{v.}{receber dinheiro | coletar; receber}
  \end{Phonetics}
\end{Entry}

\begin{Entry}{收支}{6,4}{⽁,⽀}
  \begin{Phonetics}{收支}{shou1zhi1}[][HSK 7-9]
    \definition{s.}{receitas e despesas; rendimentos e despesas}
  \synonymref{出入}{chu1ru4}
  \synonymref{进出}{jin4chu1}
  \end{Phonetics}
\end{Entry}

\begin{Entry}{收买}{6,6}{⽁,⼄}
  \begin{Phonetics}{收买}{shou1mai3}[][HSK 7-9]
    \definition{v.}{comprar; adquirir | comprar; subornar; conquistar corações e mentes}
  \synonymref{拉拢}{la1long3}
  \synonymref{收购}{shou1gou4}
  \antonymref{出卖}{chu1mai4}
  \antonymref{出售}{chu1shou4}
  \end{Phonetics}
\end{Entry}

\begin{Entry}{收回}{6,6}{⽁,⼞}
  \begin{Phonetics}{收回}{shou1 hui2}[][HSK 4]
    \definition{v.}{retomar; recuperar; relembrar; recordar; receber de volta o que foi enviado ou emprestado, ou o dinheiro que foi emprestado ou usado | sacar; retirar; recolher; rescindir; cancelar (uma opinião, ordem, etc.)}
  \end{Phonetics}
\end{Entry}

\begin{Entry}{收听}{6,7}{⽁,⼝}
  \begin{Phonetics}{收听}{shou1ting1}[][HSK 3]
    \definition{v.}{ouvir (rádio)}
  \end{Phonetics}
\end{Entry}

\begin{Entry}{收到}{6,8}{⽁,⼑}
  \begin{Phonetics}{收到}{shou1 dao4}[][HSK 2]
    \definition{v.}{conseguir; obter; receber; alcançar}
  \end{Phonetics}
\end{Entry}

\begin{Entry}{收取}{6,8}{⽁,⼜}
  \begin{Phonetics}{收取}{shou1qu3}[][HSK 6]
    \definition{v.}{obter; coletar; receber; aceitar o dinheiro pago pela outra parte}
  \end{Phonetics}
\end{Entry}

\begin{Entry}{收视率}{6,8,11}{⽁,⾒,⽞}
  \begin{Phonetics}{收视率}{shou1shi4lv4}[][HSK 7-9]
    \definition{s.}{classificações (de um programa de TV); a audiência televisiva refere-se à porcentagem de pessoas (ou domicílios) que assistem a um determinado canal de televisão (ou programa de televisão) durante um período específico, em relação ao número total de telespectadores (ou domicílios)}
  \end{Phonetics}
\end{Entry}

\begin{Entry}{收购}{6,8}{⽁,⾙}
  \begin{Phonetics}{收购}{shou1gou4}[][HSK 5]
    \definition{v.}{comprar; adquirir; comprar muito em vários lugares | adquirir uma empresa; obter o controle efetivo de uma empresa por meio de dinheiro, transações de ações, etc.}
  \end{Phonetics}
\end{Entry}

\begin{Entry}{收养}{6,9}{⽁,⼋}
  \begin{Phonetics}{收养}{shou1yang3}[][HSK 6]
    \definition{v.}{acolher e criar; adotar; acolher os filhos dos outros e criá-los como se fossem da sua própria família}
  \end{Phonetics}
\end{Entry}

\begin{Entry}{收复}{6,9}{⽁,⼢}
  \begin{Phonetics}{收复}{shou1fu4}[][HSK 7-9]
    \definition{v.}{recuperar; recapturar | retomar; recuperar | recuperar o próprio território}
  \synonymref{复原}{fu4/yuan2}
  \synonymref{复兴}{fu4xing1}
  \synonymref{恢复}{hui1fu4}
  \end{Phonetics}
\end{Entry}

\begin{Entry}{收拾}{6,9}{⽁,⼿}
  \begin{Phonetics}{收拾}{shou1shi5}[][HSK 5]
    \definition{v.}{arrumar; empacotar; limpar; organizar, policiar, restaurar a normalidade em situações adversas | consertar; reparar; restaurar algo que está danificado ao seu estado ou função original |  punir; punir alguém, geralmente com medidas mais severas | matar}
  \end{Phonetics}
\end{Entry}

\begin{Entry}{收看}{6,9}{⽁,⽬}
  \begin{Phonetics}{收看}{shou1kan4}[][HSK 3]
    \definition{v.}{assistir (a um programa de TV)}
  \end{Phonetics}
\end{Entry}

\begin{Entry}{收费}{6,9}{⽁,⾙}
  \begin{Phonetics}{收费}{shou1 fei4}[][HSK 3]
    \definition{v.}{cobrar; cobrar taxas}
  \end{Phonetics}
\end{Entry}

\begin{Entry}{收音机}{6,9,6}{⽁,⾳,⽊}
  \begin{Phonetics}{收音机}{shou1yin1ji1}[][HSK 3]
    \definition[部,台]{s.}{rádio; sem fio; um termo geral para receptores de rádio}
  \end{Phonetics}
\end{Entry}

\begin{Entry}{收留}{6,10}{⽁,⽥}
  \begin{Phonetics}{收留}{shou1liu2}[][HSK 7-9]
    \definition{v.}{acolher alguém; ter alguém sob seus cuidados; oferecer abrigo}
  \synonymref{放走}{fang4zou3}
  \synonymref{收养}{shou1yang3}
  \antonymref{赶走}{gan3zou3}
  \antonymref{抛弃}{pao1qi4}
  \antonymref{驱逐}{qu1zhu2}
  \end{Phonetics}
\end{Entry}

\begin{Entry}{收益}{6,10}{⽁,⽫}
  \begin{Phonetics}{收益}{shou1yi4}[][HSK 4]
    \definition{s.}{lucro; renda; benefício; ganhos; vantagens ou benefícios obtidos}
  \end{Phonetics}
\end{Entry}

\begin{Entry}{收获}{6,10}{⽁,⾋}
  \begin{Phonetics}{收获}{shou1huo4}[][HSK 4]
    \definition[次,番,份]{s.}{resultados; ganhos; metaforicamente falando, conhecimento, experiência, etc. obtidos em estudo ou trabalho; os resultados obtidos por meio de trabalho árduo | colheita; colheita de safras}
    \definition{v.}{colher; juntar as colheitas}
  \end{Phonetics}
\end{Entry}

\begin{Entry}{收据}{6,11}{⽁,⼿}
  \begin{Phonetics}{收据}{shou1ju4}[][HSK 7-9]
    \definition[张]{s.}{recibo; quitação; uma declaração escrita entregue à outra parte como prova após o recebimento de dinheiro ou bens}
  \end{Phonetics}
\end{Entry}

\begin{Entry}{收敛}{6,11}{⽁,⽁}
  \begin{Phonetics}{收敛}{shou1lian3}[][HSK 7-9]
    \definition{v.}{1. enfraquecer; desaparecer; diminuir ou desaparecer (sorriso, luz, etc.) | conter"-se; restringir e controlar (fala e comportamento sem restrições) | adstringir; provocar a contração do corpo ou reduzir a secreção glandular}
  \synonymref{约束}{yue1shu4}
  \antonymref{放肆}{fang4si4}
  \antonymref{放纵}{fang4zong4}
  \antonymref{展开}{zhan3kai1}
  \end{Phonetics}
\end{Entry}

\begin{Entry}{收集}{6,12}{⽁,⾫}
  \begin{Phonetics}{收集}{shou1ji2}[][HSK 5]
    \definition{v.}{coletar; reunir; recolher}
  \end{Phonetics}
\end{Entry}

\begin{Entry}{收缩}{6,14}{⽁,⽷}
  \begin{Phonetics}{收缩}{shou1suo1}[][HSK 7-9]
    \definition{v.}{contrair; encolher; (objeto) mudar de grande para pequeno ou de comprido para curto | recuar; concentrar as forças; apertar; mudar de dispersão para concentração}
  \synonymref{关上}{guan1shang4}
  \synonymref{减弱}{jian3ruo4}
  \synonymref{减少}{jian3shao3}
  \synonymref{紧缩}{jin3suo1}
  \synonymref{缩短}{suo1/duan3}
  \synonymref{缩小}{suo1/xiao3}
  \synonymref{中断}{zhong1duan4}
  \antonymref{打开}{da3 kai1}
  \antonymref{发展}{fa1zhan3}
  \antonymref{开展}{kai1zhan3}
  \antonymref{扩大}{kuo4da4}
  \antonymref{扩散}{kuo4san4}
  \antonymref{扩张}{kuo4zhang1}
  \antonymref{膨胀}{peng2zhang4}
  \antonymref{展开}{zhan3kai1}
  \end{Phonetics}
\end{Entry}

\begin{Entry}{收藏}{6,17}{⽁,⾋}
  \begin{Phonetics}{收藏}{shou1cang2}[][HSK 6]
    \definition{v.}{coletar; armazenar; consagrar}
  \end{Phonetics}
\end{Entry}

%%%%%%%%%% 改 %%%%%%%%%%
\subsection*{改}\addcontentsline{loh}{figure}{改}

\begin{Entry}{改}{7}{⽁}
  \begin{Phonetics}{改}{gai3}[][HSK 2]
    \definition{v.}{mudar; converter; transformar; alterar; substituir | alterar; revisar; aperfeiçoar; modificar | corrigir; retificar; remediar; consertar}
  \end{Phonetics}
\end{Entry}

\begin{Entry}{改为}{7,4}{⽁,⼂}
  \begin{Phonetics}{改为}{gai3wei2}[][HSK 7-9]
    \definition{v.}{mudar para}[原计划改为明天开始。===O plano original foi mudado para começar amanhã.]
  \end{Phonetics}
\end{Entry}

\begin{Entry}{改日}{7,4}{⽁,⽇}
  \begin{Phonetics}{改日}{gai3ri4}[][HSK 7-9]
    \definition{adv.}{algum outro dia; outro dia}
  \end{Phonetics}
\end{Entry}

\begin{Entry}{改正}{7,5}{⽁,⽌}
  \begin{Phonetics}{改正}{gai3zheng4}[][HSK 4]
    \definition{v.}{corrigir; emendar; mudar o errado para o correto}
  \end{Phonetics}
\end{Entry}

\begin{Entry}{改动}{7,6}{⽁,⼒}
  \begin{Phonetics}{改动}{gai3dong4}[][HSK 7-9]
    \definition{v.}{mudar; alterar; modificar; polir; melhorar | alterar (texto, itens, ordem, etc.)}
  \end{Phonetics}
\end{Entry}

\begin{Entry}{改名}{7,6}{⽁,⼝}
  \begin{Phonetics}{改名}{gai3ming2}[][HSK 7-9]
    \definition{v.}{mudar o próprio nome; alterar nome}
  \end{Phonetics}
\end{Entry}

\begin{Entry}{改邪归正}{7,6,5,5}{⽁,⾢,⼹,⽌}
  \begin{Phonetics}{改邪归正}{gai3xie2-gui1zheng4}[][HSK 7-9]
    \definition{expr.}{abandonar os maus caminhos e retornar ao caminho certo; abandonar o vício e voltar-se para a virtude; virar uma nova página; retornar a um modo de vida cumpridor da lei}
  \end{Phonetics}
\end{Entry}

\begin{Entry}{改良}{7,7}{⽁,⾉}
  \begin{Phonetics}{改良}{gai3liang2}[][HSK 7-9]
    \definition{v.}{melhorar; amenizar; remover as deficiências individuais das coisas para torná-las mais adequadas às necessidades | reformar; melhorar | Metalurgia: modificar}
  \end{Phonetics}
\end{Entry}

\begin{Entry}{改进}{7,7}{⽁,⾡}
  \begin{Phonetics}{改进}{gai3jin4}[][HSK 3]
    \definition[个,些]{s.}{melhoria}
    \definition{v.}{aprimorar; aperfeiçoar; melhorar; tornar melhor; mudar a situação antiga para melhorar | modificar (mudança mecânica)}
  \end{Phonetics}
\end{Entry}

\begin{Entry}{改变}{7,8}{⽁,⼜}
  \begin{Phonetics}{改变}{gai3bian4}[][HSK 2]
    \definition{v.}{mudar; alterar; transformar; converter; moldar; modificar | causar mudanças; alterar}
  \end{Phonetics}
\end{Entry}

\begin{Entry}{改版}{7,8}{⽁,⽚}
  \begin{Phonetics}{改版}{gai3/ban3}[][HSK 7-9]
    \definition{s.}{(programas de rádio ou TV) reformulação; ajuste | edição revisada}
    \definition{v.+compl.}{alterar o layout de uma folha impressa | alterar ou corrigir uma página definida | revisar a edição atual}
  \end{Phonetics}
\end{Entry}

\begin{Entry}{改革}{7,9}{⽁,⾰}
  \begin{Phonetics}{改革}{gai3ge2}[][HSK 5]
    \definition[项,次,种]{s.}{reforma; reformação; iniciativas para aprimorar a inovação}
    \definition{v.}{reformar; transformar as antigas partes irracionais das coisas em novas que possam ser adaptadas à situação objetiva}
  \end{Phonetics}
\end{Entry}

\begin{Entry}{改革开放}{7,9,4,8}{⽁,⾰,⼶,⽅}
  \begin{Phonetics}{改革开放}{gai3ge2 kai1fang4}[][HSK 7-9]
    \definition{v.}{reformar e abrir"-se ao mundo exterior (refere"-se às políticas de Deng Xiaoping por volta de 1980)}
  \end{Phonetics}
\end{Entry}

\begin{Entry}{改造}{7,10}{⽁,⾡}
  \begin{Phonetics}{改造}{gai3zao4}[][HSK 3]
    \definition{v.}{transformar; renovar; modificar o original para melhor se adequar às necessidades; usado principalmente para coisas específicas | remodelar; mudar radicalmente o que é velho e ruim; criar algo novo e bom, para se adaptar às novas circunstâncias e necessidades; usado principalmente para coisas abstratas}
  \end{Phonetics}
\end{Entry}

\begin{Entry}{改善}{7,12}{⽁,⼝}
  \begin{Phonetics}{改善}{gai3shan4}[][HSK 4]
    \definition{v.}{melhorar; amenizar; mudar a situação original para torná-la melhor}
  \end{Phonetics}
\end{Entry}

\begin{Entry}{改善关系}{7,12,6,7}{⽁,⼝,⼋,⽷}
  \begin{Phonetics}{改善关系}{gai3shan4guan1xi5}
    \definition{v.}{melhorar a relação}
  \end{Phonetics}
\end{Entry}

\begin{Entry}{改善通讯}{7,12,10,5}{⽁,⼝,⾡,⾔}
  \begin{Phonetics}{改善通讯}{gai3shan4tong1xun4}
    \definition{v.}{melhorar a comunicação}
  \end{Phonetics}
\end{Entry}

\begin{Entry}{改编}{7,12}{⽁,⽷}
  \begin{Phonetics}{改编}{gai3bian1}[][HSK 7-9]
    \definition{v.}{adaptar; revisar; converter; reorganizar; transcrever; reescrever com base no trabalho original (geralmente em um gênero diferente) | reorganizar; redesignar; alterar a organização original (referindo"-se principalmente ao exército)}
  \end{Phonetics}
\end{Entry}

\begin{Entry}{改装}{7,12}{⽁,⾐}
  \begin{Phonetics}{改装}{gai3zhuang1}[][HSK 6]
    \definition{v.}{mudar de traje ou vestido | reembalar | reequipar; reaparelhar | modificar; alterar o dispositivo original}
  \end{Phonetics}
\end{Entry}

%%%%%%%%%% 攻 %%%%%%%%%%
\subsection*{攻}\addcontentsline{loh}{figure}{攻}

\begin{Entry}{攻}{7}{⽁}
  \begin{Phonetics}{攻}{gong1}[][HSK 7-9]
    \definition*{s.}{Sobrenome: Gong}
    \definition{v.}{atacar; assaltar; tomar a ofensiva | acusar; cobrar | estudar; trabalhar em; especializar-se em}
  \end{Phonetics}
\end{Entry}

\begin{Entry}{攻击}{7,5}{⽁,⼐}
  \begin{Phonetics}{攻击}{gong1ji1}[][HSK 6]
    \definition{v.}{atacar; assaltar; lançar uma ofensiva | difamar; caluniar; acusar; atacar (verbalmente)}
  \end{Phonetics}
\end{Entry}

\begin{Entry}{攻关}{7,6}{⽁,⼋}
  \begin{Phonetics}{攻关}{gong1guan1}[][HSK 7-9]
    \definition{v.}{ter que encarar; superar esseobstáculo; começar essa jornada; resolver esse problema; abordar os principais problemas}
  \end{Phonetics}
\end{Entry}

\begin{Entry}{攻读}{7,10}{⽁,⾔}
  \begin{Phonetics}{攻读}{gong1du2}[][HSK 7-9]
    \definition{v.}{especializar-se em; trabalhar arduamente em uma matéria para obter um diploma ou certificado nessa matéria; estudar bastante; estudar ou aprofundar-se em um assunto}
  \end{Phonetics}
\end{Entry}

%%%%%%%%%% 政 %%%%%%%%%%
\subsection*{政}\addcontentsline{loh}{figure}{政}

\begin{Entry}{政}{9}{⽁}
  \begin{Phonetics}{政}{zheng4}
    \definition*{s.}{Sobrenome: Zheng}
    \definition{s.}{política; assuntos políticos | certos aspectos administrativos do governo | assuntos de uma família ou de uma organização; refere"-se a assuntos familiares ou de grupo}
  \end{Phonetics}
\end{Entry}

\begin{Entry}{政权}{9,6}{⽁,⽊}
  \begin{Phonetics}{政权}{zheng4quan2}[][HSK 6]
    \definition{s.}{poder político ou estatal; regime}
  \end{Phonetics}
\end{Entry}

\begin{Entry}{政纲}{9,7}{⽁,⽷}
  \begin{Phonetics}{政纲}{zheng4gang1}
    \definition{s.}{programa ou plataforma política}
  \end{Phonetics}
\end{Entry}

\begin{Entry}{政府}{9,8}{⽁,⼴}
  \begin{Phonetics}{政府}{zheng4fu3}[][HSK 4]
    \definition{s.}{governo;  órgãos executivos do poder do Estado, ou seja, órgãos administrativos do Estado, como o Conselho de Estado (Governo Popular Central) e os governos populares locais em todos os níveis na China}
  \end{Phonetics}
\end{Entry}

\begin{Entry}{政治}{9,8}{⽁,⽔}
  \begin{Phonetics}{政治}{zheng4zhi4}[][HSK 4]
    \definition{s.}{política; assuntos políticos; questões políticas; as atividades de governos, partidos políticos, grupos sociais e indivíduos em assuntos internos e relações internacionais}
  \end{Phonetics}
\end{Entry}

\begin{Entry}{政治局}{9,8,7}{⽁,⽔,⼫}
  \begin{Phonetics}{政治局}{zheng4zhi4ju2}
    \definition{s.}{o principal comitê de políticas de um partido comunista}
  \end{Phonetics}
\end{Entry}

\begin{Entry}{政党}{9,10}{⽁,⼉}
  \begin{Phonetics}{政党}{zheng4dang3}[][HSK 6]
    \definition[个,些]{s.}{partido político; uma organização política que representa um determinado estágio, classe ou grupo e luta para concretizar seus interesses}
  \end{Phonetics}
\end{Entry}

\begin{Entry}{政策}{9,12}{⽁,⽵}
  \begin{Phonetics}{政策}{zheng4ce4}[][HSK 6]
    \definition[项,条,个]{s.}{política; um código de conduta formulado por um país ou partido político para alcançar sua política em um determinado período histórico}
  \end{Phonetics}
\end{Entry}

%%%%%%%%%% 故 %%%%%%%%%%
\subsection*{故}\addcontentsline{loh}{figure}{故}

\begin{Entry}{故}{9}{⽁}
  \begin{Phonetics}{故}{gu4}[][HSK 7-9]
    \definition*{s.}{Sobrenome: Gu}
    \definition{adj.}{velho; antigo; original}
    \definition{adv.}{propositalmente; intencionalmente; deliberadamente}
    \definition{conj.}{assim; portanto; consequentemente; pelo contrário}
    \definition{s.}{evento; incidente; acontecimento; acidente | causa; razão | amigo; conhecido | o velho; refere"-se a coisas antigas e passadas}
    \definition{v.}{morrer}
  \end{Phonetics}
\end{Entry}

\begin{Entry}{故乡}{9,3}{⽁,⼄}
  \begin{Phonetics}{故乡}{gu4xiang1}[][HSK 3]
    \definition[个]{s.}{cidade natal; terra natal; local de nascimento ou onde viveu por muito tempo}
  \end{Phonetics}
\end{Entry}

\begin{Entry}{故事}{9,8}{⽁,⼅}
  \begin{Phonetics}{故事}{gu4shi5}[][HSK 2]
    \definition[个,段,篇,则]{s.}{história; conto; coisas reais ou fictícias usadas como objeto de narrativa, com coerência, atraentes e capazes de emocionar as pessoas | enredo; trama; enredo que consegue mostrar a personalidade dos personagens e refletir a ideia central da obra literária}
  \end{Phonetics}
\end{Entry}

\begin{Entry}{故宫}{9,9}{⽁,⼧}
  \begin{Phonetics}{故宫}{gu4gong1}
    \definition*{s.}{O Palácio Imperial; O Museu do Palácio (em Pequim); A Cidade Proibida}
  \end{Phonetics}
\end{Entry}

\begin{Entry}{故意}{9,13}{⽁,⼼}
  \begin{Phonetics}{故意}{gu4yi4}[][HSK 2]
    \definition{adv.}{deliberadamente; intencionalmente; não é por descuido, mas sim conscientemente (geralmente coisas que não se devem fazer ou que não são necessárias)}
    \definition{s.}{intenção; um tipo de mentalidade, uma pessoa sabe claramente que seus atos podem causar danos a outras pessoas ou trazer consequências negativas para a sociedade, mas mesmo assim não faz nada para impedir isso}
  \end{Phonetics}
\end{Entry}

\begin{Entry}{故障}{9,13}{⽁,⾩}
  \begin{Phonetics}{故障}{gu4zhang4}[][HSK 6]
    \definition[出]{s.}{problema; falha; parada; mau funcionamento; avaria; situações em que máquinas, instrumentos, etc. não podem funcionar normalmente devido a problemas}
  \end{Phonetics}
\end{Entry}

%%%%%%%%%% 效 %%%%%%%%%%
\subsection*{效}\addcontentsline{loh}{figure}{效}

\begin{Entry}{效}{10}{⽁}
  \begin{Phonetics}{效}{xiao4}
    \definition{s.}{efeito; função | eficiência; resultado}
    \definition{v.}{imitar; seguir o exemplo de | dedicar (a energia ou a vida de alguém) a; prestar (um serviço)}
  \end{Phonetics}
\end{Entry}

\begin{Entry}{效果}{10,8}{⽁,⽊}
  \begin{Phonetics}{效果}{xiao4guo3}[][HSK 3]
    \definition[种,个]{s.}{efeito; resultado | efeitos sonoros; vários sons ou fenômenos naturais criados para combinar com o enredo em dramas e filmes, como vento e chuva, tiros, fogo, neve, etc.}
  \end{Phonetics}
\end{Entry}

\begin{Entry}{效率}{10,11}{⽁,⽞}
  \begin{Phonetics}{效率}{xiao4lv4}[][HSK 4]
    \definition[种]{s.}{eficiência; produtividade; a quantidade de trabalho concluído por unidade de tempo}
  \end{Phonetics}
\end{Entry}

%%%%%%%%%% 敎 %%%%%%%%%%
\subsection*{敎}\addcontentsline{loh}{figure}{敎}

\begin{Entry}{敎}{11}{⽁}
  \begin{Phonetics}{敎}{jiao4}
    \variantof{教}
  \end{Phonetics}
\end{Entry}

%%%%%%%%%% 敏 %%%%%%%%%%
\subsection*{敏}\addcontentsline{loh}{figure}{敏}

\begin{Entry}{敏}{11}{⽁}
  \begin{Phonetics}{敏}{min3}
    \definition*{s.}{Sobrenome: Min}
    \definition{adj.}{rápido; ágil | perspicaz; inteligente; rápido | inteligente; esperto}
  \end{Phonetics}
\end{Entry}

\begin{Entry}{敏捷}{11,11}{⽁,⼿}
  \begin{Phonetics}{敏捷}{min3jie2}[][HSK 7-9]
    \definition{adj.}{ágil; rápido; descreve reações rápidas em ações, pensamentos, etc.}
  \end{Phonetics}
\end{Entry}

\begin{Entry}{敏锐}{11,12}{⽁,⾦}
  \begin{Phonetics}{敏锐}{min3rui4}[][HSK 7-9]
    \definition{adj.}{agudo; perspicaz; aguçado; (pensamento) rápido de raciocínio, (intuição) aguçado}
  \end{Phonetics}
\end{Entry}

\begin{Entry}{敏感}{11,13}{⽁,⼼}
  \begin{Phonetics}{敏感}{min3gan3}[][HSK 5]
    \definition{adj.}{sensível; descreve pessoas ou animais que rapidamente percebem mudanças ou estímulos externos | reativo; sensível; fácil de causar reações intensas}
  \end{Phonetics}
\end{Entry}

%%%%%%%%%% 救 %%%%%%%%%%
\subsection*{救}\addcontentsline{loh}{figure}{救}

\begin{Entry}{救}{11}{⽁}
  \begin{Phonetics}{救}{jiu4}[][HSK 3]
    \definition*{s.}{Sobrenome: Jiu}
    \definition{v.}{resgatar; salvar | salvar de; aliviar (angústia, etc.) | resgatar; livrar alguém de um desastre ou perigo | ajudar; aliviar; socorrer; livrar pessoas e coisas de desastres e perigos}
  \end{Phonetics}
\end{Entry}

\begin{Entry}{救出}{11,5}{⽁,⼐}
  \begin{Phonetics}{救出}{jiu4chu1}
    \definition{v.}{resgatar | tirar do perigo}
  \end{Phonetics}
\end{Entry}

\begin{Entry}{救助}{11,7}{⽁,⼒}
  \begin{Phonetics}{救助}{jiu4zhu4}[][HSK 6]
    \definition{v.}{ajudar alguém em perigo ou dificuldade; socorrer; resgatar e ajudar}
  \end{Phonetics}
\end{Entry}

\begin{Entry}{救护车}{11,7,4}{⽁,⼿,⾞}
  \begin{Phonetics}{救护车}{jiu4hu4che1}[][HSK 7-9]
    \definition[辆]{s.}{ambulância; os veículos que transportam os feridos estão equipados com instalações que permitem à equipe médica prestar primeiros socorros temporários, cuidados médicos e serviços de enfermagem aos feridos}
  \end{Phonetics}
\end{Entry}

\begin{Entry}{救灾}{11,7}{⽁,⽕}
  \begin{Phonetics}{救灾}{jiu4 zai1}[][HSK 5]
    \definition{v.}{ajudar as vítimas de desastres, aliviar o desastre; resgatar pessoas afetadas por desastres; recuperar danos causados por desastres}
  \end{Phonetics}
\end{Entry}

\begin{Entry}{救命}{11,8}{⽁,⼝}
  \begin{Phonetics}{救命}{jiu4/ming4}[][HSK 6]
    \definition{interj.}{``Socorro!''; ``Salve-me!''}
    \definition{v.+compl.}{ajudar; salvar a vida de alguém}
  \end{Phonetics}
\end{Entry}

\begin{Entry}{救治}{11,8}{⽁,⽔}
  \begin{Phonetics}{救治}{jiu4zhi4}[][HSK 7-9]
    \definition{v.}{retirar um paciente do perigo; tratar e curar}
  \end{Phonetics}
\end{Entry}

\begin{Entry}{救济}{11,9}{⽁,⽔}
  \begin{Phonetics}{救济}{jiu4ji4}[][HSK 7-9]
    \definition{v.}{fornecer ajuda com dinheiro ou bens; usar dinheiro e bens para ajudar vítimas de desastres ou outras pessoas que vivem em situação de pobreza}
  \end{Phonetics}
\end{Entry}

\begin{Entry}{救援}{11,12}{⽁,⼿}
  \begin{Phonetics}{救援}{jiu4yuan2}[][HSK 6]
    \definition{v.}{resgatar; socorrer; vir em auxílio de alguém (resgate)}
  \end{Phonetics}
\end{Entry}

%%%%%%%%%% 教 %%%%%%%%%%
\subsection*{教}\addcontentsline{loh}{figure}{教}

\begin{Entry}{教}{11}{⽁}
  \begin{Phonetics}{教}{jiao1}
    \definition*{s.}{Sobrenome: Jiao}
    \definition{prep.}{em uma frase passiva para introduzir o executor da ação}
    \definition{s.}{religião | professor; referência à educação ou aos professores}
    \definition{v.}{ensinar; instruir |  pedir; ordenar; dizer | permitir; possibilitar}
  \synonymref{授}{shou4}
  \antonymref{学}{xue2}
  \end{Phonetics}
  \begin{Phonetics}{教}{jiao4}[][HSK 1]
    \definition*{s.}{Sobrenome: Jiao}
    \definition{prep.}{em uma frase passiva para apresentar o autor da ação}
    \definition{s.}{religião | educação; professor}
    \definition{v.}{ensinar; instruir | perguntar; ordenar; contar | permitir; permitir}
  \synonymref{授}{shou4}
  \antonymref{学}{xue2}
  \end{Phonetics}
\end{Entry}

\begin{Entry}{教长}{11,4}{⽁,⾧}
  \begin{Phonetics}{教长}{jiao4zhang3}
    \definition{s.}{imã (Islã) | mulá}
  \end{Phonetics}
\end{Entry}

\begin{Entry}{教训}{11,5}{⽁,⾔}
  \begin{Phonetics}{教训}{jiao4xun5}[][HSK 4]
    \definition[个,次,番,顿]{s.}{moral; lição}
    \definition{v.}{repreender; educar; ensinar uma lição a alguém; dar uma bronca em alguém; dar um sermão em alguém (por ter cometido um erro, etc.)}
  \end{Phonetics}
\end{Entry}

\begin{Entry}{教会}{11,6}{⽁,⼈}
  \begin{Phonetics}{教会}{jiao1hui4}
    \definition{v.}{mostrar | ensinar}
  \end{Phonetics}
  \begin{Phonetics}{教会}{jiao4hui4}
    \definition{s.}{igreja cristã}
  \end{Phonetics}
\end{Entry}

\begin{Entry}{教导}{11,6}{⽁,⼨}
  \begin{Phonetics}{教导}{jiao4dao3}
    \definition{s.}{instrução | orientação | ensino}
    \definition{v.}{instruir | orientar | ensinar}
  \end{Phonetics}
\end{Entry}

\begin{Entry}{教师}{11,6}{⽁,⼱}
  \begin{Phonetics}{教师}{jiao4shi1}[][HSK 2]
    \definition[个,位,名]{s.}{professor; professor de escola}
  \end{Phonetics}
\end{Entry}

\begin{Entry}{教材}{11,7}{⽁,⽊}
  \begin{Phonetics}{教材}{jiao4cai2}[][HSK 3]
    \definition[本,套]{s.}{livro didático; materiais didáticos, incluindo livros didáticos, apostilas, materiais de referência, vídeos, imagens, etc.}
  \end{Phonetics}
\end{Entry}

\begin{Entry}{教条}{11,7}{⽁,⽊}
  \begin{Phonetics}{教条}{jiao4tiao2}[][HSK 7-9]
    \definition{adj.}{dogmático; opinativo}
    \definition{s.}{dogma; doutrina; credo | princípio; chavão}
  \end{Phonetics}
\end{Entry}

\begin{Entry}{教学}{11,8}{⽁,⼦}
  \begin{Phonetics}{教学}{jiao4xue2}[][HSK 2]
    \definition[个,门]{s.}{ensino; educação; o processo de transmissão de conhecimentos e habilidades}
  \end{Phonetics}
\end{Entry}

\begin{Entry}{教学楼}{11,8,13}{⽁,⼦,⽊}
  \begin{Phonetics}{教学楼}{jiao4xue2lou2}[][HSK 1]
    \definition{s.}{prédio da escola; bloco de ensino; edifícios utilizados para atividades educacionais, geralmente incluindo salas de aula, laboratórios, auditórios, etc.}
  \end{Phonetics}
\end{Entry}

\begin{Entry}{教官}{11,8}{⽁,⼧}
  \begin{Phonetics}{教官}{jiao4guan1}
    \definition{s.}{instrutor militar; um oficial que serviu como treinador no antigo exército ou escola}
  \end{Phonetics}
\end{Entry}

\begin{Entry}{教练}{11,8}{⽁,⽷}
  \begin{Phonetics}{教练}{jiao4lian4}[][HSK 3]
    \definition[个,位,名]{s.}{instrutor; treinador (esportes); pessoas que trabalham como treinadores}
    \definition{v.}{treinar; treinar outras pessoas para dominarem uma determinada técnica (como esportes, dirigir carros, pilotar aviões, etc.)}
  \end{Phonetics}
\end{Entry}

\begin{Entry}{教育}{11,8}{⽁,⾁}
  \begin{Phonetics}{教育}{jiao4yu4}[][HSK 2]
    \definition{s.}{educação; refere"-se a atividades sociais cujo objetivo direto é influenciar o desenvolvimento físico e mental das pessoas; refere"-se principalmente ao processo de formação dos alunos nas escolas}
    \definition{v.}{ensinar; educar; inspirar, fazer compreender a razão}
  \end{Phonetics}
\end{Entry}

\begin{Entry}{教育部}{11,8,10}{⽁,⾁,⾢}
  \begin{Phonetics}{教育部}{jiao4yu4bu4}[][HSK 6]
    \definition*{s.}{Ministério da Educação}
  \end{Phonetics}
\end{Entry}

\begin{Entry}{教养}{11,9}{⽁,⼋}
  \begin{Phonetics}{教养}{jiao4yang3}[][HSK 7-9]
    \definition{s.}{criação; educação; formação}
  \end{Phonetics}
\end{Entry}

\begin{Entry}{教室}{11,9}{⽁,⼧}
  \begin{Phonetics}{教室}{jiao4shi4}[][HSK 2]
    \definition[间]{s.}{sala de aula}
  \end{Phonetics}
\end{Entry}

\begin{Entry}{教科书}{11,9,4}{⽁,⽲,⼄}
  \begin{Phonetics}{教科书}{jiao4ke1shu1}[][HSK 7-9]
    \definition[本]{s.}{livro didático; livro do aluno | livro escolar; um livro escrito especialmente para alunos usarem em sala de aula e para revisão}
  \end{Phonetics}
\end{Entry}

\begin{Entry}{教堂}{11,11}{⽁,⼟}
  \begin{Phonetics}{教堂}{jiao4tang2}[][HSK 6]
    \definition[座,所,间]{s.}{igreja; capela; catedral; casa de deus; um lugar onde os cristãos realizam cerimônias religiosas}
  \end{Phonetics}
\end{Entry}

\begin{Entry}{教授}{11,11}{⽁,⼿}
  \begin{Phonetics}{教授}{jiao4shou4}[][HSK 4]
    \definition[个,位,名]{s.}{professor (universitário); o professor com a classificação mais alta em uma universidade}
    \definition{v.}{ensinar; instruir; dar aulas; dar palestras}
  \end{Phonetics}
\end{Entry}

%%%%%%%%%% 敢 %%%%%%%%%%
\subsection*{敢}\addcontentsline{loh}{figure}{敢}

\begin{Entry}{敢}{11}{⽁}
  \begin{Phonetics}{敢}{gan3}[][HSK 3]
    \definition{adj.}{ousado; corajoso; audacioso; valente}
    \definition{adv.}{talvez; provavelmente}
    \definition{v.}{ser ousado o suficiente; atrever-se | ter confiança em; ter certeza; estar certo | aventurar-se; ter coragem de fazer algo | ser ousado; arriscar-se}
  \end{Phonetics}
\end{Entry}

\begin{Entry}{敢于}{11,3}{⽁,⼆}
  \begin{Phonetics}{敢于}{gan3yu2}[][HSK 6]
    \definition{v.}{ousar; ser ousado em; ter determinação; ter coragem (para fazer ou se esforçar para fazer)}
  \end{Phonetics}
\end{Entry}

\begin{Entry}{敢情}{11,11}{⽁,⼼}
  \begin{Phonetics}{敢情}{gan3qing5}[][HSK 7-9]
    \definition{adv.}{por que; então; eu digo; indica a descoberta de algo que não foi descoberto anteriormente | claro; de fato; realmente; isso significa que a razão é óbvia e não há necessidade de duvidar dela}
  \end{Phonetics}
\end{Entry}

%%%%%%%%%% 敞 %%%%%%%%%%
\subsection*{敞}\addcontentsline{loh}{figure}{敞}

\begin{Entry}{敞}{12}{⽁}
  \begin{Phonetics}{敞}{chang3}
    \definition{adj.}{espaçoso; aberto; desobstruído | Dialeto: (casa, pátio, etc.) espaçoso}
    \definition{v.}{abrir; descobrir}
  \end{Phonetics}
\end{Entry}

\begin{Entry}{敞开}{12,4}{⽁,⼶}
  \begin{Phonetics}{敞开}{chang3kai1}[][HSK 7-9]
    \definition{adj.}{aberto; irrestrito}
    \definition{adv.}{livremente; sem reservas; ilimitadamente; irrestritamente; totalmente aberto; infinitamente; sem limite}
    \definition{v.}{abrir o máximo possível}
  \end{Phonetics}
\end{Entry}

%%%%%%%%%% 散 %%%%%%%%%%
\subsection*{散}\addcontentsline{loh}{figure}{散}

\begin{Entry}{散}{12}{⽁}
  \begin{Phonetics}{散}{san3}[][HSK 5]
    \definition{adj.}{disperso; fragmentado; não integrado}
    \definition{s.}{medicamento em forma de pó}
    \definition{v.}{divergir; espalhar-se; separar-se; soltar-se; não se manter unido;  desintegrar}
  \end{Phonetics}
  \begin{Phonetics}{散}{san4}[][HSK 4]
    \definition{v.}{quebrar; fragmentar; dispersar | dar; distribuir; disseminar; divulgar | dissipar; deixar sai  | terminar um acordo ou contrato; demitir}
  \end{Phonetics}
\end{Entry}

\begin{Entry}{散心}{12,4}{⽁,⼼}
  \begin{Phonetics}{散心}{san4/xin1}
    \definition{v.+compl.}{aliviar o tédio | desfrutar de uma diversão | estar despreocupado}
  \end{Phonetics}
\end{Entry}

\begin{Entry}{散文}{12,4}{⽁,⽂}
  \begin{Phonetics}{散文}{san3wen2}[][HSK 5]
    \definition[篇,种]{s.}{ensaio; prosa; gênero literário, na antiguidade, referia"-se a textos em prosa, em oposição à poesia e à prosa paralela; atualmente, refere"-se a obras literárias que não sejam poesia, teatro ou romance, incluindo ensaios, contos, crônicas, relatos de viagem, etc.}
  \end{Phonetics}
\end{Entry}

\begin{Entry}{散发}{12,5}{⽁,⼜}
  \begin{Phonetics}{散发}{san4fa4}[][HSK 7-9]
    \definition{v.}{emitir; difundir; enviar; divulgar | emitir; distribuir; dar}
  \end{Phonetics}
\end{Entry}

\begin{Entry}{散布}{12,5}{⽁,⼱}
  \begin{Phonetics}{散布}{san4bu4}[][HSK 7-9]
    \definition{v.}{espalhar; distribuir; disseminar | espalhar; propagar}
  \end{Phonetics}
\end{Entry}

\begin{Entry}{散步}{12,7}{⽁,⽌}
  \begin{Phonetics}{散步}{san4/bu4}[][HSK 3]
    \definition{v.+compl.}{dar uma volta; dar um passeio; dar uma caminhada}
  \end{Phonetics}
\end{Entry}

%%%%%%%%%% 敦 %%%%%%%%%%
\subsection*{敦}\addcontentsline{loh}{figure}{敦}

\begin{Entry}{敦}{12}{⽁}
  \begin{Phonetics}{敦}{dui4}
    \definition{s.}{um recipiente tradicional para armazenar arroz ou grãos; utensílios antigos para guardar painço}
  \end{Phonetics}
  \begin{Phonetics}{敦}{dun1}
    \definition*{s.}{Sobrenome: Dun}
    \definition{adj.}{honesto; sincero}
  \end{Phonetics}
\end{Entry}

\begin{Entry}{敦促}{12,9}{⽁,⼈}
  \begin{Phonetics}{敦促}{dun1cu4}[][HSK 7-9]
    \definition{v.}{instar; pressionar; urgir}
  \end{Phonetics}
\end{Entry}

\begin{Entry}{敦厚}{12,9}{⽁,⼚}
  \begin{Phonetics}{敦厚}{dun1hou4}[][HSK 7-9]
    \definition{adj.}{genuíno | honesto e sincero}
  \end{Phonetics}
\end{Entry}

%%%%%%%%%% 敬 %%%%%%%%%%
\subsection*{敬}\addcontentsline{loh}{figure}{敬}

\begin{Entry}{敬}{12}{⽁}
  \begin{Phonetics}{敬}{jing4}[][HSK 7-9]
    \definition*{s.}{Sobrenome: Jing}
    \definition{adj.}{respeitoso; reverente}
    \definition{adv.}{respeitosamente}
    \definition{v.}{respeitar; honrar; estimar | oferecer educadamente | envolver-se em; dedicar-se a}
  \end{Phonetics}
\end{Entry}

\begin{Entry}{敬业}{12,5}{⽁,⼀}
  \begin{Phonetics}{敬业}{jing4ye4}[][HSK 7-9]
    \definition{v.}{ser dedicado ao próprio trabalho; refere"-se a um espírito louvável, dedicado aos estudos ou ao trabalho}
  \end{Phonetics}
\end{Entry}

\begin{Entry}{敬礼}{12,5}{⽁,⽰}
  \begin{Phonetics}{敬礼}{jing4/li3}[][HSK 7-9]
    \definition{s.}{honoríficos, usados no final de uma carta}[结尾都是``此致敬礼''。===Todas elas terminam com ``Atenciosamente''.]
    \definition{v.+compl.}{saudar; prestar continência; demonstrar respeito através de gestos como olhar para alguém, levantar a mão, ficar em posição de sentido e fazer uma reverência}
  \end{Phonetics}
\end{Entry}

\begin{Entry}{敬而远之}{12,6,7,3}{⽁,⽽,⾡,⼂}
  \begin{Phonetics}{敬而远之}{jing4'er2yuan3zhi1}[][HSK 7-9]
    \definition{expr.}{``Respeite, mas mantenha distância.''; mantenha uma distância respeitosa de alguém; afaste-se de; demonstrar respeito à distância}
  \end{Phonetics}
\end{Entry}

\begin{Entry}{敬佩}{12,8}{⽁,⼈}
  \begin{Phonetics}{敬佩}{jing4pei4}[][HSK 7-9]
    \definition{v.}{estimar; admirar; respeitar e admirar}
  \end{Phonetics}
\end{Entry}

\begin{Entry}{敬重}{12,9}{⽁,⾥}
  \begin{Phonetics}{敬重}{jing4zhong4}[][HSK 7-9]
    \definition{v.}{reverenciar; honrar; estimar; respeitar profundamente}
  \end{Phonetics}
\end{Entry}

\begin{Entry}{敬爱}{12,10}{⽁,⽖}
  \begin{Phonetics}{敬爱}{jing4'ai4}[][HSK 7-9]
    \definition{v.}{amar; estimar; acarinhar; respeitar e amar}
  \end{Phonetics}
\end{Entry}

\begin{Entry}{敬请}{12,10}{⽁,⾔}
  \begin{Phonetics}{敬请}{jing4qing3}[][HSK 7-9]
    \definition{v.}{solicitar; convidar respeitosamente; termos educados usados para convidar ou pedir (que alguém faça algo)}
  \end{Phonetics}
\end{Entry}

\begin{Entry}{敬酒}{12,10}{⽁,⾣}
  \begin{Phonetics}{敬酒}{jing4/jiu3}[][HSK 7-9]
    \definition{v.+compl.}{brindar; propor um brinde; levantar seu copo respeitosamente para convidar a outra pessoa a beber}
  \end{Phonetics}
\end{Entry}

\begin{Entry}{敬意}{12,13}{⽁,⼼}
  \begin{Phonetics}{敬意}{jing4yi4}[][HSK 7-9]
    \definition{s.}{respeito; tributo; homenagem}
  \end{Phonetics}
\end{Entry}

%%%%%%%%%% 数 %%%%%%%%%%
\subsection*{数}\addcontentsline{loh}{figure}{数}

\begin{Entry}{数}{13}{⽁}
  \begin{Phonetics}{数}{shu3}[][HSK 2]
    \definition{v.}{contar (número); contar (número) um a um | ser considerado excepcionalmente (bom, ruim, etc.) | enumerar; listar}
  \end{Phonetics}
  \begin{Phonetics}{数}{shu4}
    \definition{num.}{vários; alguns}
    \definition{s.}{número; cifra; figura | número (conceito matemático básico que representa a quantidade de coisas) | número; indica a quantidade de coisas a que se referem os substantivos ou pronomes | destino; sorte}
  \end{Phonetics}
  \begin{Phonetics}{数}{shuo4}
    \definition{adv.}{com frequência; repetidamente; indica uma ação frequente, equivalente a 屡次}
  \seealsoref{屡次}{lv3ci4}
  \end{Phonetics}
\end{Entry}

\begin{Entry}{数目}{13,5}{⽁,⽬}
  \begin{Phonetics}{数目}{shu4mu4}[][HSK 5]
    \definition{s.}{número; quantidade; quantidade de algo expressa em uma determinada medida padrão (como unidades de medida, etc.)}
  \end{Phonetics}
\end{Entry}

\begin{Entry}{数字}{13,6}{⽁,⼦}
  \begin{Phonetics}{数字}{shu4zi4}[][HSK 2]
    \definition{adj.}{digital; usando tecnologia digital}
    \definition[个,串]{s.}{dígito; número; um caractere que representa um número | numeral; símbolos que representam números, como algarismos arábicos, algarismos romanos, etc. | quantidade; montante}
  \end{Phonetics}
\end{Entry}

\begin{Entry}{数学}{13,8}{⽁,⼦}
  \begin{Phonetics}{数学}{shu4xue2}
    \definition{s.}{matemática; a ciência que estuda as formas espaciais e as relações quantitativas do mundo real, incluindo matemática elementar e matemática superior}
  \end{Phonetics}
\end{Entry}

\begin{Entry}{数码}{13,8}{⽁,⽯}
  \begin{Phonetics}{数码}{shu4ma3}[][HSK 4]
    \definition{s.}{dígito; numeral; algarismo | número; quantidade (usado principalmente na linguagem falada)}
    \definition{v.}{digitalizar}
  \end{Phonetics}
\end{Entry}

\begin{Entry}{数据}{13,11}{⽁,⼿}
  \begin{Phonetics}{数据}{shu4ju4}[][HSK 4]
    \definition[组,个,条]{s.}{dados; valores com base nos quais são realizadas estatísticas, cálculos, pesquisas científicas ou projetos técnicos}
  \end{Phonetics}
\end{Entry}

\begin{Entry}{数据库}{13,11,7}{⽁,⼿,⼴}
  \begin{Phonetics}{数据库}{shu4ju4ku4}[][HSK 7-9]
    \definition{s.}{banco de dados}
  \end{Phonetics}
\end{Entry}

\begin{Entry}{数量}{13,12}{⽁,⾥}
  \begin{Phonetics}{数量}{shu4liang4}[][HSK 3]
    \definition[个,种]{s.}{quantidade; quantum; quantia; magnitude; número}
  \end{Phonetics}
\end{Entry}

\begin{Entry}{数额}{13,15}{⽁,⾴}
  \begin{Phonetics}{数额}{shu4'e2}[][HSK 7-9]
    \definition{s.}{cota; número; quantidade; um certo número}
  \end{Phonetics}
\end{Entry}

%%%%%%%%%% 敲 %%%%%%%%%%
\subsection*{敲}\addcontentsline{loh}{figure}{敲}

\begin{Entry}{敲}{14}{⽁}
  \begin{Phonetics}{敲}{qiao1}[][HSK 5]
    \definition{v.}{bater; dar uma pancada; golpear | explorar alguém; cobrar a mais; extorquir; chantagear | lembrar; criticar; alertar; advertir}
  \end{Phonetics}
\end{Entry}

\begin{Entry}{敲门}{14,3}{⽁,⾨}
  \begin{Phonetics}{敲门}{qiao1 men2}[][HSK 5]
    \definition{v.}{bater na porta}
  \end{Phonetics}
\end{Entry}

\begin{Entry}{敲边鼓}{14,5,13}{⽁,⾡,⿎}
  \begin{Phonetics}{敲边鼓}{qiao1 bian1gu3}[][HSK 7-9]
    \definition{v.}{``Tocar trompa.'' | Coloquial: falar ou agir para ajudar alguém à margem; apoiar alguém; apoiar alguém em uma discussão}
  \end{Phonetics}
\end{Entry}

\begin{Entry}{敲诈}{14,7}{⽁,⾔}
  \begin{Phonetics}{敲诈}{qiao1zha4}[][HSK 7-9]
    \definition{v.}{extorquir; chantagear; usar o poder, a intimidação e as ameaças para extorquir dinheiro}
  \end{Phonetics}
\end{Entry}

%%%%%%%%%% 敷 %%%%%%%%%%
\subsection*{敷}\addcontentsline{loh}{figure}{敷}

\begin{Entry}{敷}{15}{⽁}
  \begin{Phonetics}{敷}{fu1}[][HSK 7-9]
    \definition*{s.}{Sobrenome: Fu}
    \definition{v.}{aplicar (pó, pomada, etc.) | espalhar; dispor | ser suficiente para | espalhar-se}
  \end{Phonetics}
\end{Entry}

%%%%%%%%%% 整 %%%%%%%%%%
\subsection*{整}\addcontentsline{loh}{figure}{整}

\begin{Entry}{整}{16}{⽁}
  \begin{Phonetics}{整}{zheng3}[][HSK 3]
    \definition*{s.}{Sobrenome: Zheng}
    \definition{adj.}{cheio; integral; inteiro; completo; sem defeitos | limpo; arrumado; organizado; em boa ordem | redondo (não é uma fração)}
    \definition{s.}{número inteiro (não fracionário)}
    \definition{v.}{retificar; corrigir; pôr em ordem | consertar; renovar; reparar | corrigir; punir; causar sofrimento;  fazer alguém sofrer | fazer; realizar; trabalhar; em algumas regiões, significa 做, 搞}
  \seealsoref{搞}{gao3}
  \seealsoref{做}{zuo4}
  \end{Phonetics}
\end{Entry}

\begin{Entry}{整个}{16,3}{⽁,⼈}
  \begin{Phonetics}{整个}{zheng3ge4}[][HSK 3]
    \definition{adj.}{total; inteiro; completo}
  \end{Phonetics}
\end{Entry}

\begin{Entry}{整天}{16,4}{⽁,⼤}
  \begin{Phonetics}{整天}{zheng3tian1}[][HSK 3]
    \definition{s.}{o dia inteiro; o dia todo; durante todo o dia; de manhã à noite}
  \end{Phonetics}
\end{Entry}

\begin{Entry}{整齐}{16,6}{⽁,⿑}
  \begin{Phonetics}{整齐}{zheng3qi2}[][HSK 3]
    \definition{adj.}{arrumado; organizado; em boa ordem | uniforme; regular; tamanho, comprimento, grau, etc. são relativamente consistentes | usado para descrever que todas as coisas necessárias estão prontas}
    \definition{v.}{estar em boas condições; manter a ordem e a organização}
  \end{Phonetics}
\end{Entry}

\begin{Entry}{整体}{16,7}{⽁,⼈}
  \begin{Phonetics}{整体}{zheng3ti3}[][HSK 3]
    \definition[个]{s.}{um todo; totalidade}
  \end{Phonetics}
\end{Entry}

\begin{Entry}{整治}{16,8}{⽁,⽔}
  \begin{Phonetics}{整治}{zheng3zhi4}[][HSK 6]
    \definition{v.}{renovar; consertar; arrumar; dragar (um rio, etc.) | punir; fazer alguém sofrer}
  \end{Phonetics}
\end{Entry}

\begin{Entry}{整顿}{16,10}{⽁,⾴}
  \begin{Phonetics}{整顿}{zheng3dun4}[][HSK 6]
    \definition{v.}{retificar; consolidar; reorganizar; tornar fenômenos, disciplinas e estilos desordenados e irracionais, ordenados e razoáveis}
  \end{Phonetics}
\end{Entry}

\begin{Entry}{整理}{16,11}{⽁,⽟}
  \begin{Phonetics}{整理}{zheng3li3}[][HSK 3]
    \definition{v.}{organizar; reorganizar; classificar; ordenar; colocar em ordem}
  \end{Phonetics}
\end{Entry}

\begin{Entry}{整整}{16,16}{⽁,⽁}
  \begin{Phonetics}{整整}{zheng3zheng3}[][HSK 3]
    \definition{adv.}{inteiramente; completamente; solidamente; continuamente}
  \end{Phonetics}
\end{Entry}

%%%%% EOF %%%%%

