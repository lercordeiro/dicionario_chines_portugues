%%%
%%% Radical "⼴"
%%%
\section*{Radical 53: ``⼴''}\addcontentsline{toc}{section}{Radical 53: ⼴}\addcontentsline{loh}{figure}{\#\#\#\# 53: ⼴}

%%%%%%%%%% 广 %%%%%%%%%%
\subsection*{广}\addcontentsline{loh}{figure}{广}

\begin{Entry}{广}{3}{⼴}[Kangxi 53]
  \begin{Phonetics}{广}{an1}
    \definition{s.}{mais comum em nomes de pessoas; o mesmo que 庵}[广安是我的朋友。===An'an é meu amigo.]
  \seealsoref{庵}{an1}
  \end{Phonetics}
  \begin{Phonetics}{广}{guang3}[][HSK 5]
    \definition*{s.}{Sobrenome: Guang}
    \definition{adj.}{largo; vasto; amplo; extenso | numeroso | comum; universal}
    \definition{s.}{Guangdong, 广东, e Guangxi, 广州}
    \definition{v.}{expandir; espalhar; ampliar}
  \seealsoref{广东}{guang3dong1}
  \seealsoref{广州}{guang3zhou1}
  \antonymref{狭}{xia2}
  \end{Phonetics}
  \begin{Phonetics}{广}{yan3}
    \definition[家]{s.}{casa ou edifício construído contra ou ao longo da encosta de uma montanha ou penhasco}
  \end{Phonetics}
\end{Entry}

\begin{Entry}{广义}{3,3}{⼴,⼂}
  \begin{Phonetics}{广义}{guang3yi4}[][HSK 7-9]
    \definition*{s.}{Província de Quang Ngai; nome de lugar vietnamita, uma das províncias do Vietnã Central}
    \definition{s.}{sentido amplo; sentido geral; definição mais ampla}
  \end{Phonetics}
\end{Entry}

\begin{Entry}{广大}{3,3}{⼴,⼤}
  \begin{Phonetics}{广大}{guang3da4}[][HSK 3]
    \definition{adj.}{muito difundido; enorme (alcance, escala) | (uma área ou espaço) vasto; extenso; em grande escala; amplo (área, espaço) | numeroso; muitos (número de pessoas)}
  \end{Phonetics}
\end{Entry}

\begin{Entry}{广东}{3,5}{⼴,⼀}
  \begin{Phonetics}{广东}{guang3dong1}
    \definition*{s.}{Província de Guangdong}
  \seealsoref{粤}{yue4}
  \end{Phonetics}
\end{Entry}

\begin{Entry}{广场}{3,6}{⼴,⼟}
  \begin{Phonetics}{广场}{guang3chang3}[][HSK 2]
    \definition{s.}{praça; praça pública; esplanada; área ampla, especificamente uma área ampla na cidade}
  \end{Phonetics}
\end{Entry}

\begin{Entry}{广场舞}{3,6,14}{⼴,⼟,⾇}
  \begin{Phonetics}{广场舞}{guang3chang3wu3}
    \definition{s.}{quadrilha, uma rotina de exercícios tocada com música em quadrados públicos, parques e praças, popular especialmente entre mulheres de meia-idade e aposentados na China}
  \end{Phonetics}
\end{Entry}

\begin{Entry}{广州}{3,6}{⼴,⼮}
  \begin{Phonetics}{广州}{guang3zhou1}
    \definition*{s.}{Guangzhou, antigamente Cantão; Capital da Província de Guangdong}
  \end{Phonetics}
\end{Entry}

\begin{Entry}{广西}{3,6}{⼴,⾑}
  \begin{Phonetics}{广西}{guang3xi1}
    \definition*{s.}{Guangxi (Região Autônoma de Zhuang)}
  \seealsoref{壮}{zhuang4}
  \end{Phonetics}
\end{Entry}

\begin{Entry}{广告}{3,7}{⼴,⼝}
  \begin{Phonetics}{广告}{guang3gao4}[][HSK 2]
    \definition[则,条,段,项,个]{s.}{anúncio; propaganda; uma forma de divulgação ao público de produtos, serviços ou programas culturais e esportivos, geralmente realizada por meio de jornais, televisão, rádio, cartazes, etc.}
    \definition{v.}{anunciar; a ação ou ato de promover ou divulgar algo}
  \end{Phonetics}
\end{Entry}

\begin{Entry}{广泛}{3,7}{⼴,⽔}
  \begin{Phonetics}{广泛}{guang3fan4}[][HSK 5]
    \definition{adj.}{amplo; extenso; de grande alcance; disseminado; escopo e cobertura amplos}
  \end{Phonetics}
\end{Entry}

\begin{Entry}{广阔}{3,12}{⼴,⾨}
  \begin{Phonetics}{广阔}{guang3kuo4}[][HSK 6]
    \definition{adj.}{vasto; largo; amplo}
  \end{Phonetics}
\end{Entry}

\begin{Entry}{广播}{3,15}{⼴,⼿}
  \begin{Phonetics}{广播}{guang3bo1}[][HSK 3]
    \definition[个,次,段,则,条]{s.}{programa de rádio; transmissão (de rádio); refere"-se a programas transmitidos por estações de rádio ou televisão a cabo}
    \definition{v.}{transmitir; estar no ar | espalhar"-se amplamente; ser conhecido em toda parte; divulgar amplamente}
  \end{Phonetics}
\end{Entry}

%%%%%%%%%% 庆 %%%%%%%%%%
\subsection*{庆}\addcontentsline{loh}{figure}{庆}

\begin{Entry}{庆}{6}{⼴}
  \begin{Phonetics}{庆}{qing4}
    \definition*{s.}{Sobrenome: Qing}
    \definition{s.}{celebração | ocasião para celebração; um aniversário que vale a pena comemorar}
    \definition{v.}{celebrar; felicitar; comemorar}
  \end{Phonetics}
\end{Entry}

\begin{Entry}{庆典}{6,8}{⼴,⼋}
  \begin{Phonetics}{庆典}{qing4dian3}[][HSK 7-9]
    \definition{s.}{cerimônia; celebração; uma cerimônia de celebração muito grandiosa}
  \end{Phonetics}
\end{Entry}

\begin{Entry}{庆幸}{6,8}{⼴,⼲}
  \begin{Phonetics}{庆幸}{qing4xing4}[][HSK 7-9]
    \definition{v.}{alegrar-se; ficar contente; ficar feliz por uma situação inesperadamente boa}
  \end{Phonetics}
\end{Entry}

\begin{Entry}{庆祝}{6,9}{⼴,⽰}
  \begin{Phonetics}{庆祝}{qing4zhu4}[][HSK 3]
    \definition{v.}{celebrar; comemorar; festejar; realizar atividades para comemorar ou celebrar festivais comuns e eventos felizes}
  \end{Phonetics}
\end{Entry}

\begin{Entry}{庆贺}{6,9}{⼴,⾙}
  \begin{Phonetics}{庆贺}{qing4he4}[][HSK 7-9]
    \definition{v.}{parabenizar; celebrar; celebrar uma ocasião alegre compartilhada ou parabenizar alguém que está recebendo boas notícias}
  \end{Phonetics}
\end{Entry}

%%%%%%%%%% 床 %%%%%%%%%%
\subsection*{床}\addcontentsline{loh}{figure}{床}

\begin{Entry}{床}{7}{⼴}
  \begin{Phonetics}{床}{chuang2}[][HSK 1]
    \definition{clas.}{usado para colchas, roupas de cama, etc.}
    \definition[张]{s.}{cama; sofá; móveis para dormir | algo com o formato de uma cama}
  \end{Phonetics}
\end{Entry}

\begin{Entry}{床位}{7,7}{⼴,⼈}
  \begin{Phonetics}{床位}{chuang2wei4}[][HSK 7-9]
    \definition{s.}{beliche; cama; camas para pacientes, viajantes e hóspedes em hospitais, navios e dormitórios}
  \end{Phonetics}
\end{Entry}

%%%%%%%%%% 库 %%%%%%%%%%
\subsection*{库}\addcontentsline{loh}{figure}{库}

\begin{Entry}{库}{7}{⼴}
  \begin{Phonetics}{库}{ku4}[][HSK 5]
    \definition{s.}{depósito; tesouraria; armazém; almoxarifado; edifícios e equipamentos para armazenamento de mercadorias | Computação: banco de dados}
  \end{Phonetics}
\end{Entry}

%%%%%%%%%% 应 %%%%%%%%%%
\subsection*{应}\addcontentsline{loh}{figure}{应}

\begin{Entry}{应}{7}{⼴}
  \begin{Phonetics}{应}{ying1}[][HSK 4,5]
    \definition{v.}{ecoar; responder; responder a; responder às chamadas, saudações, perguntas, etc. de outras pessoas | conceder; cumprir | adequar; adaptar; responder a | lidar com; enfrentar; abordar | tornar-se realidade; ser cumprido}
  \end{Phonetics}
\end{Entry}

\begin{Entry}{应对}{7,5}{⼴,⼨}
  \begin{Phonetics}{应对}{ying4 dui4}[][HSK 6]
    \definition{v.}{reagir; responder; lidar com; dar uma resposta; tomar medidas e contramedidas para lidar com a situação}
  \end{Phonetics}
\end{Entry}

\begin{Entry}{应用}{7,5}{⼴,⽤}
  \begin{Phonetics}{应用}{ying4yong4}[][HSK 3]
    \definition{adj.}{aplicado (na vida ou na produção); usado diretamente na vida ou na produção}
    \definition{v.}{usar; aplicar}
  \end{Phonetics}
\end{Entry}

\begin{Entry}{应用程序}{7,5,12,7}{⼴,⽤,⽲,⼴}
  \begin{Phonetics}{应用程序}{ying4yong4 cheng2xu4}
    \definition{s.}{programa aplicativo; principais categorias de \emph{software}}
  \end{Phonetics}
\end{Entry}

\begin{Entry}{应用程序接口}{7,5,12,7,11,3}{⼴,⽤,⽲,⼴,⼿,⼝}
  \begin{Phonetics}{应用程序接口}{ying4yong4 cheng2xu4 jie1kou3}
    \definition{s.}{API (\emph{application programming interface})}
  \seealsoref{应用程序编程接口}{ying4yong4 cheng2xu4 bian1cheng2 jie1kou3}
  \end{Phonetics}
\end{Entry}

\begin{Entry}{应用程序编程接口}{7,5,12,7,12,12,11,3}{⼴,⽤,⽲,⼴,⽷,⽲,⼿,⼝}
  \begin{Phonetics}{应用程序编程接口}{ying4yong4 cheng2xu4 bian1cheng2 jie1kou3}
    \definition{s.}{API (\emph{application programming interface})}
  \seealsoref{应用程序接口}{ying4yong4 cheng2xu4 jie1kou3}
  \end{Phonetics}
\end{Entry}

\begin{Entry}{应当}{7,6}{⼴,⼹}
  \begin{Phonetics}{应当}{ying1 dang1}[][HSK 3]
    \definition{v.}{dever}[学生们应当努力学习。===Os alunos devem se esforçar nos estudos.]
  \end{Phonetics}
\end{Entry}

\begin{Entry}{应该}{7,8}{⼴,⾔}
  \begin{Phonetics}{应该}{ying1gai1}[][HSK 2]
    \definition{v.}{deveria; deve ser assim | deveria; acho que deve ser esse o caso}
  \end{Phonetics}
\end{Entry}

\begin{Entry}{应急}{7,9}{⼴,⼼}
  \begin{Phonetics}{应急}{ying4 ji2}[][HSK 6]
    \definition{v.}{atender a uma necessidade urgente (emergência, contingência, etc.)}
  \end{Phonetics}
\end{Entry}

%%%%%%%%%% 底 %%%%%%%%%%
\subsection*{底}\addcontentsline{loh}{figure}{底}

\begin{Entry}{底}{8}{⼴}
  \begin{Phonetics}{底}{de5}
    \definition{part.}{usada após uma palavra ou frase que é usada como determinante para indicar subordinação à palavra central}
  \end{Phonetics}
  \begin{Phonetics}{底}{di3}[][HSK 4]
    \definition*{s.}{Sobrenome: Di}
    \definition{pron.}{o que? |  isto; isso; aqui | assim; tal}
    \definition{s.}{base; fundo; parte inferior de um objeto | detalhes; o cerne da questão; base, fonte ou contexto de uma coisa | rascunho; cópia mantida como registro; rascunho que pode ser usado como base | final de um ano ou mês | chão; fundo; fundação | a última parte de algo}
  \end{Phonetics}
\end{Entry}

\begin{Entry}{底下}{8,3}{⼴,⼀}
  \begin{Phonetics}{底下}{di3 xia4}[][HSK 3]
    \definition{adv.}{em baixo; abaixo; sob | próximo; mais tarde; depois; daqui para a frente}
  \end{Phonetics}
\end{Entry}

\begin{Entry}{底子}{8,3}{⼴,⼦}
  \begin{Phonetics}{底子}{di3zi5}[][HSK 7-9]
    \definition{s.}{fundo; base; a parte mais baixa de um objeto | solo; base; fundo; fundação | rascunho ou esboço; um rascunho para servir de base | cópia mantida como registro; cópia de arquivo | remanescente | detalhes; prós e contras | Literário: configuração (o padrão base)}
  \end{Phonetics}
\end{Entry}

\begin{Entry}{底气}{8,4}{⼴,⽓}
  \begin{Phonetics}{底气}{di3qi4}
    \definition{s.}{capacidade pulmonar | ousadia | confiança | autoconfiança | vigor}
  \end{Phonetics}
\end{Entry}

\begin{Entry}{底层}{8,7}{⼴,⼫}
  \begin{Phonetics}{底层}{di3ceng2}[][HSK 7-9]
    \definition[个]{s.}{andar térreo | fundo; o degrau mais baixo; classe social mais baixa | porão | subcamada; camada de base; subcapa; substrato}
  \end{Phonetics}
\end{Entry}

\begin{Entry}{底线}{8,8}{⼴,⽷}
  \begin{Phonetics}{底线}{di3xian4}[][HSK 7-9]
    \definition{s.}{linha de base (em esportes); limites em ambas as extremidades de campos esportivos como futebol, basquete, vôlei e badminton | um mínimo; o limite mais baixo; um limite mínimo; a menor quantidade possível; refere"-se às condições mínimas | um fantoche; um informante; um agente infiltrado; uma pessoa que se esconde dentro do inimigo para reunir informações ou conduzir outras atividades; um \emph{insider}}
  \end{Phonetics}
\end{Entry}

\begin{Entry}{底蕴}{8,15}{⼴,⾋}
  \begin{Phonetics}{底蕴}{di3yun4}[][HSK 7-9]
    \definition{s.}{detalhes; informações privilegiadas; história interna}
  \end{Phonetics}
\end{Entry}

%%%%%%%%%% 店 %%%%%%%%%%
\subsection*{店}\addcontentsline{loh}{figure}{店}

\begin{Entry}{店}{8}{⼴}
  \begin{Phonetics}{店}{dian4}[][HSK 2]
    \definition[家,间,个]{s.}{loja; armazém; loja de venda de mercadorias | pousada; pequena pousada com instalações simples | usado para nomes de lugares}
  \end{Phonetics}
\end{Entry}

\begin{Entry}{店主}{8,5}{⼴,⼂}
  \begin{Phonetics}{店主}{dian4zhu3}
    \definition{s.}{lojista | dono de loja}
  \end{Phonetics}
\end{Entry}

\begin{Entry}{店员}{8,7}{⼴,⼝}
  \begin{Phonetics}{店员}{dian4yuan2}
    \definition{s.}{assistente de loja | balconista | vendedor}
  \end{Phonetics}
\end{Entry}

%%%%%%%%%% 庙 %%%%%%%%%%
\subsection*{庙}\addcontentsline{loh}{figure}{庙}

\begin{Entry}{庙}{8}{⼴}
  \begin{Phonetics}{庙}{miao4}[][HSK 7-9]
    \definition[座,个,间]{s.}{templo; santuário | feira do templo | Literário: corte imperial; corte real | Literário: imperador falecido | casa de incenso; locais onde, no passado, foram consagradas tábuas ancestrais, divindades ou figuras históricas}
  \end{Phonetics}
\end{Entry}

\begin{Entry}{庙会}{8,6}{⼴,⼈}
  \begin{Phonetics}{庙会}{miao4hui4}[][HSK 7-9]
    \definition{s.}{feira; feira do templo; festival feira do templo; mercados montados em templos ou próximos a eles; geralmente realizados em festivais ou dias específicos}
  \end{Phonetics}
\end{Entry}

%%%%%%%%%% 庞 %%%%%%%%%%
\subsection*{庞}\addcontentsline{loh}{figure}{庞}

\begin{Entry}{庞}{8}{⼴}
  \begin{Phonetics}{庞}{pang2}
    \definition*{s.}{Sobrenome: Pang}
    \definition{adj.}{enorme | inúmeros e desordenados; numerosos e desorganizados}
    \definition{s.}{molde do rosto de alguém | rosto; placa frontal}
  \end{Phonetics}
\end{Entry}

\begin{Entry}{庞大}{8,3}{⼴,⼤}
  \begin{Phonetics}{庞大}{pang2da4}[][HSK 7-9]
    \definition{adj.}{enorme; colossal; gigantesco; imenso; (em termos de forma, estrutura, quantidade, etc.) é muito grande; excessivamente grande}
  \end{Phonetics}
\end{Entry}

%%%%%%%%%% 废 %%%%%%%%%%
\subsection*{废}\addcontentsline{loh}{figure}{废}

\begin{Entry}{废}{8}{⼴}
  \begin{Phonetics}{废}{fei4}[][HSK 7-9]
    \definition{adj.}{desperdíçado; inútil; fora de uso; inválido; tendo perdido sua função original | Literário: incapacitado; mutilado; aleijado; desabilitado}
    \definition{v.}{desistir; abandonar; abolir; revogar  | Coloquial: punir; bater em alguém | descartar; abandonar}
  \end{Phonetics}
\end{Entry}

\begin{Entry}{废物}{8,8}{⼴,⽜}
  \begin{Phonetics}{废物}{fei4wu4}[][HSK 7-9]
    \definition{s.}{lixo; material residual; coisas que perderam seu valor de uso original}
  \end{Phonetics}
  \begin{Phonetics}{废物}{fei4wu5}
    \definition{s.}{pessoa inútil; imprestável (insulto); uma metáfora para uma pessoa inútil (palavrão)}
  \end{Phonetics}
\end{Entry}

\begin{Entry}{废话}{8,8}{⼴,⾔}
  \begin{Phonetics}{废话}{fei4hua4}[][HSK 7-9]
    \definition{s.}{lixo; absurdo; palavras supérfluas; palavras redundantes e inúteis}
    \definition{v.}{falar bobagens; conversa fiada}
  \end{Phonetics}
\end{Entry}

\begin{Entry}{废品}{8,9}{⼴,⼝}
  \begin{Phonetics}{废品}{fei4pin3}[][HSK 7-9]
    \definition[件,吨,批,堆]{s.}{produto residual; rejeito; descarte; produtos não qualificados; produto descartado; sucata; refugo; material rejeitado}
  \end{Phonetics}
\end{Entry}

\begin{Entry}{废除}{8,9}{⼴,⾩}
  \begin{Phonetics}{废除}{fei4chu2}[][HSK 7-9]
    \definition{v.}{revogar; anular; cancelar; abolir (uma lei, sistema, tratado, etc.)}
  \end{Phonetics}
\end{Entry}

\begin{Entry}{废寝忘食}{8,13,7,9}{⼴,⼧,⼼,⾷}
  \begin{Phonetics}{废寝忘食}{fei4qin3-wang4shi2}[][HSK 7-9]
    \definition{expr.}{esquecer de comer e dormir; estar totalmente absorvido em}
  \end{Phonetics}
\end{Entry}

\begin{Entry}{废墟}{8,14}{⼴,⼟}
  \begin{Phonetics}{废墟}{fei4xu1}[][HSK 7-9]
    \definition[片,堆,个]{s.}{ruínas; terreno baldio; um lugar como uma cidade ou vila que ficou deserta e desolada após ser destruída ou sofrer um desastre natural}
  \end{Phonetics}
\end{Entry}

%%%%%%%%%% 度 %%%%%%%%%%
\subsection*{度}\addcontentsline{loh}{figure}{度}

\begin{Entry}{度}{9}{⼴}
  \begin{Phonetics}{度}{du4}[][HSK 2]
    \definition*{s.}{Sobrenome: Du}
    \definition{clas.}{grau; unidade de medida para ângulos, temperatura, etc. | quilowatt"-hora (kWh) | usado para indicar a quantidade de álcool presente no vinho | usado para arcos e ângulos | usado para indicar o grau de curvatura da lente dos óculos ou o grau de miopia | tempo; número de vezes | usado para longitude e latitude, localização geográfica}
    \definition{s.}{medida linear; padrões e instrumentos para medir comprimentos | grau de intensidade; refere"-se especificamente ao grau alcançado por uma determinada propriedade de uma coisa | limite; extensão; grau; quota | regras; código de conduta; diretrizes | tolerância; magnanimidade; refere"-se especificamente ao grau de tolerância | maneira; temperamento; disposição; a personalidade ou aparência de uma pessoa | indicador de grau, nível alcançado por algo | tempo ou espaço limitado; um determinado período de tempo ou espaço}
    \definition{v.}{passar; atravessar; passar por cima | (em termos de tempo) passar; passar por | (de monges ou monjas budistas, ou sacerdotes taoístas) pregar; converter; proselitar}
  \end{Phonetics}
  \begin{Phonetics}{度}{duo2}
    \definition{v.}{supor; estimar; especular}
  \end{Phonetics}
\end{Entry}

\begin{Entry}{度过}{9,6}{⼴,⾡}
  \begin{Phonetics}{度过}{du4guo4}[][HSK 4]
    \definition{s.}{passar o tempo; fazer o tempo desaparecer no trabalho, na vida, no lazer e no descanso}
  \end{Phonetics}
\end{Entry}

\begin{Entry}{度知名度}{9,8,6,9}{⼴,⽮,⼝,⼴}
  \begin{Phonetics}{度知名度}{du4 zhi1ming2du4}
    \definition{s.}{popularidade}
  \end{Phonetics}
\end{Entry}

\begin{Entry}{度假}{9,11}{⼴,⼈}
  \begin{Phonetics}{度假}{du4jia4}[][HSK 7-9]
    \definition{v.}{sair de férias; passar as férias}
  \end{Phonetics}
\end{Entry}

%%%%%%%%%% 座 %%%%%%%%%%
\subsection*{座}\addcontentsline{loh}{figure}{座}

\begin{Entry}{座}{10}{⼴}
  \begin{Phonetics}{座}{zuo4}[][HSK 2]
    \definition{clas.}{usado para montanhas, edifícios e objetos imóveis semelhantes}
    \definition{s.}{assento; lugar | suporte; pedestal; base | Astronomia: constelação | Arcaico: forma de tratamento a altos funcionários}
  \end{Phonetics}
\end{Entry}

\begin{Entry}{座子}{10,3}{⼴,⼦}
  \begin{Phonetics}{座子}{zuo4zi5}
    \definition{s.}{suporte; pedestal; base | selim (de bicicleta, motocicleta, etc.)}
  \end{Phonetics}
\end{Entry}

\begin{Entry}{座位}{10,7}{⼴,⼈}
  \begin{Phonetics}{座位}{zuo4wei4}[][HSK 2]
    \definition[个,排]{s.}{assento; lugar}
  \end{Phonetics}
\end{Entry}

\begin{Entry}{座标}{10,9}{⼴,⽊}
  \begin{Phonetics}{座标}{zuo4biao1}
    \variantof{坐标}
  \end{Phonetics}
\end{Entry}

\begin{Entry}{座谈会}{10,10,6}{⼴,⾔,⼈}
  \begin{Phonetics}{座谈会}{zuo4 tan2 hui4}[][HSK 6]
    \definition{s.}{fórum; simpósio; discussão informal | conferência | sessão de rap}
  \end{Phonetics}
\end{Entry}

%%%%%%%%%% 庵 %%%%%%%%%%
\subsection*{庵}\addcontentsline{loh}{figure}{庵}

\begin{Entry}{庵}{11}{⼴}
  \begin{Phonetics}{庵}{an1}
    \definition*{s.}{Sobrenome: An}
    \definition[个,座]{s.}{cabana | convento de freiras; templos budistas, principalmente onde vivem as freiras}
  \end{Phonetics}
\end{Entry}

%%%%%%%%%% 庶 %%%%%%%%%%
\subsection*{庶}\addcontentsline{loh}{figure}{庶}

\begin{Entry}{庶}{11}{⼴}
  \begin{Phonetics}{庶}{shu4}
    \definition*{s.}{Sobrenome: Shu}
    \definition{adj.}{multitudinário; numeroso}
    \definition{conj.}{para que; de modo a}
    \definition{s.}{da ou pela concubina (diferentemente da esposa legal); no sistema patriarcal, refere"-se ao ramo lateral da família}
  \end{Phonetics}
\end{Entry}

\begin{Entry}{庶民}{11,5}{⼴,⽒}
  \begin{Phonetics}{庶民}{shu4min2}
    \definition{s.}{(antigo) pessoas comuns | (antigo) plebeu; plebeus | (antigo) a multidão de pessoas comuns (na literatura erudita)}
  \end{Phonetics}
\end{Entry}

%%%%%%%%%% 康 %%%%%%%%%%
\subsection*{康}\addcontentsline{loh}{figure}{康}

\begin{Entry}{康}{11}{⼴}
  \begin{Phonetics}{康}{kang1}
    \definition*{s.}{Sobrenome: Kang}
    \definition{adj.}{saudável |  fácil; pacífico; abundante | amplo; largo | Dialeto: de baixa qualidade; inferior}
    \definition{s.}{bem-estar; saúde | palha; farelo; casca}
    \definition{v.}{(normalmente de um rabanete) tornar-se esponjoso}
  \end{Phonetics}
\end{Entry}

\begin{Entry}{康复}{11,9}{⼴,⼢}
  \begin{Phonetics}{康复}{kang1 fu4}[][HSK 6]
    \definition{v.}{Saúde: estaurar; recuperar; reabilitar}
  \end{Phonetics}
\end{Entry}

%%%%%%%%%% 廊 %%%%%%%%%%
\subsection*{廊}\addcontentsline{loh}{figure}{廊}

\begin{Entry}{廊}{11}{⼴}
  \begin{Phonetics}{廊}{lang2}
    \definition[个]{s.}{varanda; corredor}
  \end{Phonetics}
\end{Entry}

\begin{Entry}{廊坊}{11,7}{⼴,⼟}
  \begin{Phonetics}{廊坊}{lang2fang2}
    \definition*{s.}{Cidade de Langfang em Hebei}
  \end{Phonetics}
\end{Entry}

%%%%%%%%%% 廉 %%%%%%%%%%
\subsection*{廉}\addcontentsline{loh}{figure}{廉}

\begin{Entry}{廉}{13}{⼴}
  \begin{Phonetics}{廉}{lian2}
    \definition*{s.}{Sobrenome: Lian}
    \definition{adj.}{honesto e limpo | de baixo preço; barato; econômico}
    \definition{s.}{integridade; honestidade; limpeza}
  \antonymref{贪}{tan1}
  \end{Phonetics}
\end{Entry}

\begin{Entry}{廉正}{13,5}{⼴,⽌}
  \begin{Phonetics}{廉正}{lian2zheng4}[][HSK 7-9]
    \definition{adj.}{íntegro e honesto}
    \definition{s.}{integridade}
  \seealsoref{连直}{lian2zhi2}
  \seealsoref{廉直}{lian2zhi2}
  \end{Phonetics}
\end{Entry}

\begin{Entry}{廉价}{13,6}{⼴,⼈}
  \begin{Phonetics}{廉价}{lian2jia4}[][HSK 7-9]
    \definition{adj.}{barato; de baixo preço; a um preço baixo; o preço está mais baixo que o normal | sem valor; sem utilidade}
  \end{Phonetics}
\end{Entry}

\begin{Entry}{廉直}{13,8}{⼴,⽬}
  \begin{Phonetics}{廉直}{lian2zhi2}
    \definition{adj.}{incorruptível | impecavelmente limpo | íntegro e honesto}
  \end{Phonetics}
\end{Entry}

\begin{Entry}{廉政}{13,9}{⼴,⽁}
  \begin{Phonetics}{廉政}{lian2zheng4}[][HSK 7-9]
    \definition{s.}{governo incorrupto}
    \definition{v.}{limpar o governo (ou a política); administrar de forma honesta e transparente}
  \end{Phonetics}
\end{Entry}

\begin{Entry}{廉洁}{13,9}{⼴,⽔}
  \begin{Phonetics}{廉洁}{lian2jie2}[][HSK 7-9]
    \definition{adj.}{honesto e íntegro; incorruptível | honesto; com as mãos limpas; não se enriqueça às custas do público; não desvie fundos}
  \end{Phonetics}
\end{Entry}

%%%%% EOF %%%%%

