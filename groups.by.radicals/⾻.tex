%%%
%%% Radical "⾻"
%%%
\section*{Radical 188: ``⾻'' ( ⻣)}\addcontentsline{toc}{section}{Radical 188: ⾻, ⻣}\addcontentsline{loh}{figure}{\#\#\#\# 188: ⾻}

%%%%%%%%%% 骨 %%%%%%%%%%
\subsection*{骨}\addcontentsline{loh}{figure}{骨}

\begin{Entry}{骨}{9}{⾻}[Kangxi 188]
  \begin{Phonetics}{骨}{gu3}
    \definition*{s.}{Sobrenome: Gu}
    \definition[根,块]{s.}{osso | esqueleto; estrutura | caráter; espírito | cadáver; corpo}
  \end{Phonetics}
\end{Entry}

\begin{Entry}{骨干}{9,3}{⾻,⼲}
  \begin{Phonetics}{骨干}{gu3gan4}[][HSK 7-9]
    \definition[名,个,位]{s.}{diáfise; a parte central de um osso longo, conectada à epífise em ambas as extremidades, contém uma cavidade | espinha dorsal; esteio; metaforicamente falando, uma pessoa ou coisa que desempenha um papel importante}
  \end{Phonetics}
\end{Entry}

\begin{Entry}{骨气}{9,4}{⾻,⽓}
  \begin{Phonetics}{骨气}{gu3qi4}[][HSK 7-9]
    \definition[些,种]{s.}{espinha dorsal; integridade moral; força de caráter | vigor dos traços caligráficos; refere"-se ao impulso forte e vertical expresso pela caligrafia}
  \end{Phonetics}
\end{Entry}

\begin{Entry}{骨头}{9,5}{⾻,⼤}
  \begin{Phonetics}{骨头}{gu3tou5}[][HSK 4]
    \definition[根,块]{s.}{osso; tecidos mais duros no corpo de uma pessoa ou de alguns animais que sustentam o corpo ou protegem os órgãos do corpo | caráter de uma pessoa; refere"-se à qualidade do caráter de uma pessoa}
  \end{Phonetics}
\end{Entry}

\begin{Entry}{骨折}{9,7}{⾻,⼿}
  \begin{Phonetics}{骨折}{gu3zhe2}[][HSK 7-9]
    \definition{v.}{sofrer uma fratura; quebrar (um osso)}
  \end{Phonetics}
\end{Entry}

%%%%% EOF %%%%%

