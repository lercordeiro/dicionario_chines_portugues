%%%
%%% Radical "⾌"
%%%
\section*{Radical 141: ``⾌''}\addcontentsline{toc}{section}{Radical 141: ⾌}\addcontentsline{loh}{figure}{\#\#\#\# 141: ⾌}

%%%%%%%%%% 虎 %%%%%%%%%%
\subsection*{虎}\addcontentsline{loh}{figure}{虎}

\begin{Entry}{虎}{8}{⾌}
  \begin{Phonetics}{虎}{hu3}[][HSK 5]
    \definition*{s.}{Sobrenome: Hu}
    \definition{adj.}{corajoso; bravo; valente; vigoroso}
    \definition[只]{s.}{tigre}
    \definition{v.}{blefar; o mesmo que 唬 | parecer feroz; mostrar a aparência feroz de alguém}
  \seealsoref{唬}{hu3}
  \seealsoref{老虎}{lao3hu3}
  \end{Phonetics}
\end{Entry}

\begin{Entry}{虎口}{8,3}{⾌,⼝}
  \begin{Phonetics}{虎口}{hu3kou3}
    \definition{s.}{lugar perigoso | cova do tigre}
  \end{Phonetics}
\end{Entry}

\begin{Entry}{虎虎}{8,8}{⾌,⾌}
  \begin{Phonetics}{虎虎}{hu3hu3}
    \definition{adj.}{formidável | forte | vigoroso}
  \end{Phonetics}
\end{Entry}

\begin{Entry}{虎鼬}{8,18}{⾌,⿏}
  \begin{Phonetics}{虎鼬}{hu3you4}
    \definition{s.}{doninha}
  \end{Phonetics}
\end{Entry}

%%%%%%%%%% 虐 %%%%%%%%%%
\subsection*{虐}\addcontentsline{loh}{figure}{虐}

\begin{Entry}{虐}{9}{⾌}
  \begin{Phonetics}{虐}{nve4}
    \definition{adj.}{cruel; tirânico; brutal e cruel}
  \end{Phonetics}
\end{Entry}

\begin{Entry}{虐待}{9,9}{⾌,⼻}
  \begin{Phonetics}{虐待}{nve4dai4}[][HSK 7-9]
    \definition{v.}{maltratar; tiranizar; tratar com métodos cruéis e impiedosos}
  \end{Phonetics}
\end{Entry}

%%%%%%%%%% 虔 %%%%%%%%%%
\subsection*{虔}\addcontentsline{loh}{figure}{虔}

\begin{Entry}{虔}{10}{⾌}
  \begin{Phonetics}{虔}{qian2}
    \definition*{s.}{Sobrenome: Qian}
    \definition{adj.}{piedoso; sincero}
  \end{Phonetics}
\end{Entry}

\begin{Entry}{虔诚}{10,8}{⾌,⾔}
  \begin{Phonetics}{虔诚}{qian2cheng2}[][HSK 7-9]
    \definition{adj.}{piedoso; devoto; devotado; respeitoso e sincero (frequentemente referindo-se à religião ou à fé)}
  \end{Phonetics}
\end{Entry}

%%%%%%%%%% 彪 %%%%%%%%%%
\subsection*{彪}\addcontentsline{loh}{figure}{彪}

\begin{Entry}{彪}{11}{⾌}
  \begin{Phonetics}{彪}{biao1}
    \definition*{s.}{Sobrenome: Biao}
    \definition{adj.}{semelhante a um tigre (metáfora para estatura alta)}
    \definition{s.}{tigre jovem}
  \end{Phonetics}
\end{Entry}

%%%%%%%%%% 虚 %%%%%%%%%%
\subsection*{虚}\addcontentsline{loh}{figure}{虚}

\begin{Entry}{虚}{11}{⾌}
  \begin{Phonetics}{虚}{xu1}
    \definition*{s.}{Xu, a décima primeira das vinte e oito constelações em que a esfera celeste foi dividida, consistindo de duas estrelas em linha reta, uma em Aquário e a outra em Equuleus | Xu, uma das mansões lunares | Sobrenome: Xu}
    \definition{adj.}{vazio; oco; desocupado | desconfiado; tímido | falso; nominal (oposto a 实) | humilde; modesto | fraco; com saúde debilitada | (física) virtual}
    \definition{adv.}{em vão}
    \definition{s.}{vazio; nulidade; anulação | resumo; teoria; princípios orientadores; ideologia política e outros aspectos}
    \definition{v.}{reservar espaço}
  \seealsoref{实}{shi2}
  \end{Phonetics}
\end{Entry}

\begin{Entry}{虚心}{11,4}{⾌,⼼}
  \begin{Phonetics}{虚心}{xu1xin1}[][HSK 5]
    \definition{adj.}{modesto; humilde; de mente aberta; não ser presunçoso, ser capaz de aceitar as opiniões dos outros}
  \end{Phonetics}
\end{Entry}

\begin{Entry}{虚伪}{11,6}{⾌,⼈}
  \begin{Phonetics}{虚伪}{xu1wei3}
    \definition{adj.}{falso | hipócrita | artificial}
  \end{Phonetics}
\end{Entry}

%%%%%%%%%% 虞 %%%%%%%%%%
\subsection*{虞}\addcontentsline{loh}{figure}{虞}

\begin{Entry}{虞}{13}{⾌}
  \begin{Phonetics}{虞}{yu2}
    \definition*{s.}{Reino Yu, uma dinastia lendária fundada por Shun 舜 |Yu (um estado da Dinastia Zhou 周) | Sobrenome: Yu}
    \definition{s.}{Literário: suposição; previsão | Literário: ansiedade; preocupação}
    \definition{v.}{Literário: enganar; trapacear; fazer de bobo}
  \seealsoref{舜}{shun4}
  \seealsoref{周}{zhou1}
  \end{Phonetics}
\end{Entry}

\begin{Entry}{虞世南}{13,5,9}{⾌,⼀,⼗}
  \begin{Phonetics}{虞世南}{yu2 shi4'nan2}
    \definition*{s.}{Yu Shi'nan (558-638), político dos períodos Sui e Tang inicial, poeta e calígrafo, um dos Quatro Grandes Calígrafos do início da Dinastia Tang, 唐初四大家}
  \seealsoref{唐初四大家}{tang2 chu1 si4 da4jia1}
  \end{Phonetics}
\end{Entry}

%%%%%%%%%% 避 %%%%%%%%%%
\subsection*{避}\addcontentsline{loh}{figure}{避}

\begin{Entry}{避}{16}{⾌}
  \begin{Phonetics}{避}{bi4}[][HSK 4]
    \definition{v.}{evitar; evadir; esquivar-se; buscar abrigo; fugir | impedir; manter afastado; repelir; previnir}
  \end{Phonetics}
\end{Entry}

\begin{Entry}{避免}{16,7}{⾌,⼉}
  \begin{Phonetics}{避免}{bi4mian3}[][HSK 4]
    \definition{v.}{evitar; desviar; abster-se de; tentar não fazer com que algo aconteça; prevenir; tentar impedir (que algo ruim aconteça) com antecedência}
  \end{Phonetics}
\end{Entry}

\begin{Entry}{避难}{16,10}{⾌,⾫}
  \begin{Phonetics}{避难}{bi4/nan4}[][HSK 7-9]
    \definition{s.}{refúgio}
    \definition{v.+compl.}{refugiar-se; buscar asilo (político etc.)}
  \end{Phonetics}
\end{Entry}

\begin{Entry}{避暑}{16,12}{⾌,⽇}
  \begin{Phonetics}{避暑}{bi4/shu3}[][HSK 7-9]
    \definition{v.+compl.}{ir de férias em um resort de verão; ir para um lugar fresco para evitar o calor do verão | prevenir insolação}
  \end{Phonetics}
\end{Entry}

%%%%% EOF %%%%%

