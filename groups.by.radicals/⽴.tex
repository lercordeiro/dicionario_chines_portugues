%%%
%%% Radical "⽴"
%%%
\section*{Radical 117: ``⽴''}\addcontentsline{toc}{section}{Radical 117: ⽴}\addcontentsline{loh}{figure}{\#\#\#\# 117: ⽴}

%%%%%%%%%% 立 %%%%%%%%%%
\subsection*{立}\addcontentsline{loh}{figure}{立}

\begin{Entry}{立}{5}{⽴}[Kangxi 117]
  \begin{Phonetics}{立}{li4}[][HSK 5]
    \definition{adj.}{ereto; vertical; na vertical}
    \definition{adv.}{imediatamente; instantaneamente}
    \definition{v.}{ficar em pé, com os pés no chão ou apoiados em algum objeto; o objeto deve estar na vertical | erguer; colocar (ou levantar) algo; colocar em pé | encontrar; criar; elaborar; formular; estabelecer | configurar; fundar; estabelecer | viver; existir | ascender ao trono; antigamente, referia-se à ascensão ao trono de um monarca | nomear; designar; antigamente, significava estabelecer uma determinada posição ou status}
  \end{Phonetics}
\end{Entry}

\begin{Entry}{立方}{5,4}{⽴,⽅}
  \begin{Phonetics}{立方}{li4fang1}[][HSK 7-9]
    \definition[的]{s.}{Matemática: cubo ($x^3$) | cubo, abreviação de 立方体 | metro cúbico, abreviação de 立方米}
  \seealsoref{立方米}{li4fang1mi3}
  \seealsoref{立方体}{li4fang1ti3}
  \end{Phonetics}
\end{Entry}

\begin{Entry}{立方米}{5,4,6}{⽴,⽅,⽶}
  \begin{Phonetics}{立方米}{li4fang1mi3}[][HSK 7-9]
    \definition{clas.}{metro cúbico (unidade de volume)}
  \seealsoref{立米}{li4mi3}
  \end{Phonetics}
\end{Entry}

\begin{Entry}{立方体}{5,4,7}{⽴,⽅,⼈}
  \begin{Phonetics}{立方体}{li4fang1ti3}
    \definition{adj.}{cúbico}
    \definition{s.}{cubo | sólidos}
  \end{Phonetics}
\end{Entry}

\begin{Entry}{立功}{5,5}{⽴,⼒}
  \begin{Phonetics}{立功}{li4/gong1}[][HSK 7-9]
    \definition{v.+compl.}{prestar serviço meritório; realizar um feito meritório; conquistar honra; fazer contribuições; estabelecer mérito}
  \end{Phonetics}
\end{Entry}

\begin{Entry}{立交桥}{5,6,10}{⽴,⼇,⽊}
  \begin{Phonetics}{立交桥}{li4jiao1qiao2}[][HSK 7-9]
    \definition[座]{s.}{viaduto; passagem elevada; ponte (ou cruzamento) em trevo}
  \end{Phonetics}
\end{Entry}

\begin{Entry}{立场}{5,6}{⽴,⼟}
  \begin{Phonetics}{立场}{li4chang3}[][HSK 5]
    \definition[个]{s.}{posição; postura; a posição e a atitude adotadas ao reconhecer e lidar com os problemas | ponto de vista; refere-se especificamente à atitude de reconhecer e lidar com questões a partir dos interesses de uma determinada classe, ou seja, a posição de classe}
  \end{Phonetics}
\end{Entry}

\begin{Entry}{立米}{5,6}{⽴,⽶}
  \begin{Phonetics}{立米}{li4mi3}
    \definition{clas.}{metro cúbico}
  \seealsoref{立方米}{li4fang1mi3}
  \end{Phonetics}
\end{Entry}

\begin{Entry}{立体}{5,7}{⽴,⼈}
  \begin{Phonetics}{立体}{li4ti3}[][HSK 7-9]
    \definition{adj.}{tridimensional; estereoscópico, isso proporciona às pessoas uma sensação tridimensional | multinível; multicamadas; multifacetado}
    \definition{s.}{corpo geométrico; figura sólida; forma geométrica; forma tridimensional}
  \end{Phonetics}
\end{Entry}

\begin{Entry}{立即}{5,7}{⽴,⼙}
  \begin{Phonetics}{立即}{li4ji2}[][HSK 4]
    \definition{adv.}{prontamente; imediatamente; de imediato}
  \end{Phonetics}
\end{Entry}

\begin{Entry}{立足}{5,7}{⽴,⾜}
  \begin{Phonetics}{立足}{li4zu2}[][HSK 7-9]
    \definition{v.}{manter uma posição; ter um ponto de apoio em algum lugar | basear-se em; localizar-se em (um determinado local ou região)}
  \end{Phonetics}
\end{Entry}

\begin{Entry}{立刻}{5,8}{⽴,⼑}
  \begin{Phonetics}{立刻}{li4ke4}[][HSK 3]
    \definition{adv.}{imediatamente; de ​​uma vez; indica que algo acontecerá imediatamente após um determinado momento}
  \end{Phonetics}
\end{Entry}

\begin{Entry}{立法}{5,8}{⽴,⽔}
  \begin{Phonetics}{立法}{li4fa3}
    \definition{s.}{legislação}
    \definition{v.}{promulgar leis | legislar}
  \end{Phonetics}
\end{Entry}

%%%%%%%%%% 竖 %%%%%%%%%%
\subsection*{竖}\addcontentsline{loh}{figure}{竖}

\begin{Entry}{竖}{9}{⽴}
  \begin{Phonetics}{竖}{shu4}
    \definition*{s.}{Sobrenome: Shu}
    \definition{adj.}{vertical; ereto; perpendicular ao solo}
    \definition{s.}{traço vertical (em caracteres chineses) | empregados domésticos; jovens criados}
    \definition{v.}{colocar em pé; erguer; ficar de pé; colocar o objeto perpendicular ao solo}
  \end{Phonetics}
\end{Entry}

\begin{Entry}{竖向}{9,6}{⽴,⼝}
  \begin{Phonetics}{竖向}{shu4xiang4}
    \definition{adj.}{vertical}
  \end{Phonetics}
\end{Entry}

%%%%%%%%%% 站 %%%%%%%%%%
\subsection*{站}\addcontentsline{loh}{figure}{站}

\begin{Entry}{站}{10}{⽴}
  \begin{Phonetics}{站}{zhan4}[][HSK 1,2]
    \definition*{s.}{Sobrenome: Zhan}
    \definition{s.}{parada; estação; ponto de parada | central; estação; instituição criada para um determinado tipo de atividade | filial de uma empresa ou organização; local de trabalho criado para realizar uma determinada tarefa | \emph{website}; na rede de computadores, refere-se a um \emph{site}}
    \definition{v.}{ficar em pé; estar em pé | parar; interromper; fazer uma pausa}
  \end{Phonetics}
\end{Entry}

\begin{Entry}{站长}{10,4}{⽴,⾧}
  \begin{Phonetics}{站长}{zhan4zhang3}
    \definition{s.}{pessoa responsável pela estação de trem | chefe da estação | \emph{webmaster} | gerente de centro de voluntariado}
  \end{Phonetics}
\end{Entry}

\begin{Entry}{站台}{10,5}{⽴,⼝}
  \begin{Phonetics}{站台}{zhan4 tai2}[][HSK 6]
    \definition{s.}{plataforma (em uma estação ferroviária)}
  \end{Phonetics}
\end{Entry}

\begin{Entry}{站住}{10,7}{⽴,⼈}
  \begin{Phonetics}{站住}{zhan4 zhu4}[][HSK 2]
    \definition{v.}{parar; deter; parar enquanto se move | ficar firme nos pés; manter os pés; permanecer firme | manter-se firme; consolidar a posição de alguém; estabelecer-se em uma determinada unidade ou lugar | sustentar a opinião}
  \end{Phonetics}
\end{Entry}

\begin{Entry}{站姿}{10,9}{⽴,⼥}
  \begin{Phonetics}{站姿}{zhan4zi1}
    \definition{s.}{postura}
  \end{Phonetics}
\end{Entry}

\begin{Entry}{站点}{10,9}{⽴,⽕}
  \begin{Phonetics}{站点}{zhan4dian3}
    \definition{s.}{\emph{website}}
  \end{Phonetics}
\end{Entry}

%%%%%%%%%% 竞 %%%%%%%%%%
\subsection*{竞}\addcontentsline{loh}{figure}{竞}

\begin{Entry}{竞}{10}{⽴}
  \begin{Phonetics}{竞}{jing4}
    \definition{adj.}{forte; poderoso}
    \definition{v.}{competir; contender; disputar | contestar}
  \end{Phonetics}
\end{Entry}

\begin{Entry}{竞争}{10,6}{⽴,⼑}
  \begin{Phonetics}{竞争}{jing4zheng1}[][HSK 5]
    \definition{v.}{competir; disputar; lutar; entre duas ou mais partes; em prol de seus próprios interesses; lutar pela vitória por meio de uma disputa de sua própria força contra outra}
  \end{Phonetics}
\end{Entry}

\begin{Entry}{竞技}{10,7}{⽴,⼿}
  \begin{Phonetics}{竞技}{jing4ji4}[][HSK 7-9]
    \definition{s.}{atletismo; provas de atletismo; esportes; pista e campo}
    \definition{v.}{competir; desafiar; geralmente referindo-se a competições atléticas}
  \end{Phonetics}
\end{Entry}

\begin{Entry}{竞相}{10,9}{⽴,⽬}
  \begin{Phonetics}{竞相}{jing4xiang1}[][HSK 7-9]
    \definition{adv.}{ansiosamente}
    \definition{s.}{competição}
    \definition{v.}{competir; disputar}
  \end{Phonetics}
\end{Entry}

\begin{Entry}{竞选}{10,9}{⽴,⾡}
  \begin{Phonetics}{竞选}{jing4xuan3}[][HSK 7-9]
    \definition{s.}{eleição; campanha eleitoral}
    \definition{v.}{participar de uma disputa eleitoral; fazer campanha para (um cargo); candidatar-se a}
  \end{Phonetics}
\end{Entry}

\begin{Entry}{竞赛}{10,14}{⽴,⾙}
  \begin{Phonetics}{竞赛}{jing4sai4}[][HSK 5]
    \definition[个]{s.}{concurso; competição; partida; corrida}
    \definition{v.}{correr; competir; competir uns com os outros por superioridade; em esportes, produção e outras atividades, para comparar competência, habilidade etc., usado principalmente na linguagem falada}
  \end{Phonetics}
\end{Entry}

%%%%%%%%%% 竣 %%%%%%%%%%
\subsection*{竣}\addcontentsline{loh}{figure}{竣}

\begin{Entry}{竣}{12}{⽴}
  \begin{Phonetics}{竣}{jun4}
    \definition{v.}{concluir; terminar; finalizar}
  \end{Phonetics}
\end{Entry}

\begin{Entry}{竣工}{12,3}{⽴,⼯}
  \begin{Phonetics}{竣工}{jun4gong1}[][HSK 7-9]
    \definition{v.}{ser concluído, finalizado (projetos)}
  \end{Phonetics}
\end{Entry}

%%%%%%%%%% 童 %%%%%%%%%%
\subsection*{童}\addcontentsline{loh}{figure}{童}

\begin{Entry}{童}{12}{⽴}
  \begin{Phonetics}{童}{tong2}
    \definition*{s.}{Sobrenome: Tong}
    \definition{adj.}{virgem; solteira | nu; careca | árido; estéril}
    \definition{s.}{criança | jovem servo; antigamente, referia-se a um servo menor de idade.}
  \end{Phonetics}
\end{Entry}

\begin{Entry}{童年}{12,6}{⽴,⼲}
  \begin{Phonetics}{童年}{tong2 nian2}[][HSK 4]
    \definition[对]{s.}{infância; primeiros anos de vida}
  \end{Phonetics}
\end{Entry}

\begin{Entry}{童话}{12,8}{⽴,⾔}
  \begin{Phonetics}{童话}{tong2hua4}[][HSK 4]
    \definition[个,部]{s.}{conto de fadas; gênero de literatura infantil no qual as histórias adequadas para a diversão das crianças são escritas com muita imaginação, fantasia e exagero}
  \end{Phonetics}
\end{Entry}

%%%%%%%%%% 竭 %%%%%%%%%%
\subsection*{竭}\addcontentsline{loh}{figure}{竭}

\begin{Entry}{竭}{14}{⽴}
  \begin{Phonetics}{竭}{jie2}
    \definition*{s.}{Sobrenome: Jie}
    \definition{v.}{esgotar; consumir | Literário: secar; drenar}
  \end{Phonetics}
\end{Entry}

\begin{Entry}{竭力}{14,2}{⽴,⼒}
  \begin{Phonetics}{竭力}{jie2li4}[][HSK 7-9]
    \definition{v.}{fazer o máximo; fazer o máximo; não poupar esforços; tentar por todos os meios possíveis; dar o melhor de si; usar todos os esforços do corpo e da mente para\dots; usar cada grama de sua energia}
  \end{Phonetics}
\end{Entry}

\begin{Entry}{竭尽全力}{14,6,6,2}{⽴,⼫,⼊,⼒}
  \begin{Phonetics}{竭尽全力}{jie2jin4-quan2li4}[][HSK 7-9]
    \definition{expr.}{``Dê o seu melhor.''; não poupar esforços; fazer o máximo possível; com todas as forças; usar todas as suas forças para descrever o ato de fazer o máximo esforço; fazer o máximo possível; fazer tudo o que estiver ao seu alcance}
  \end{Phonetics}
\end{Entry}

%%%%%%%%%% 端 %%%%%%%%%%
\subsection*{端}\addcontentsline{loh}{figure}{端}

\begin{Entry}{端}{14}{⽴}
  \begin{Phonetics}{端}{duan1}[][HSK 6]
    \definition*{s.}{Sobrenome: Duan}
    \definition{adj.}{adequado; próprio | reto; correto}
    \definition{s.}{fim; extremidade | começo | item; ponto; pista, projeto ou aspecto | causa; razão | problema; incidente; coisas (geralmente se refere a coisas ruins, como acidentes, disputas, etc.)}
    \definition{v.}{carregar; segurar algo nivelado com ambas as mãos; segurar algo horizontalmente | erradicar; eliminar; acabar com; remover completamente; varrer | dar ares de superioridade | revelar}
  \end{Phonetics}
\end{Entry}

\begin{Entry}{端午节}{14,4,5}{⽴,⼗,⾋}
  \begin{Phonetics}{端午节}{duan1wu3jie2}[][HSK 6]
    \definition*[个]{s.}{Festa do Duplo Cinco, Festival dos Barcos-Dragão (5º~dia do quinto mês lunar)}
  \end{Phonetics}
\end{Entry}

\begin{Entry}{端正}{14,5}{⽴,⽌}
  \begin{Phonetics}{端正}{duan1zheng4}[][HSK 7-9]
    \definition{adj.}{apropriado; correto; não torto ou inclinado | ereto; integridade; decência}
    \definition{v.}{corrigir; fazer o certo}
  \end{Phonetics}
\end{Entry}

%%%%% EOF %%%%%

