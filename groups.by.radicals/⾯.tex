%%%
%%% Radical "⾯"
%%%
\section*{Radical 176: ``⾯'' (靣)}\addcontentsline{toc}{section}{Radical 176: ⾯、靣}\addcontentsline{loh}{figure}{\#\#\#\# 176: ⾯}

%%%%%%%%%% 面 %%%%%%%%%%
\subsection*{面}\addcontentsline{loh}{figure}{面}

\begin{Entry}{面}{9}{⾯}[Kangxi 176]
  \begin{Phonetics}{面}{mian4}[][HSK 2]
    \definition*{s.}{Sobrenome: Mian}
    \definition{adj.}{macio e farinhento; descreve algo que é muito macio ao comer | superficial}
    \definition{adv.}{diretamente; pessoalmente; na frente de alguém; cara a cara}
    \definition{clas.}{usado para objetos planos | usado para indicar o número de vezes que as pessoas se encontram}
    \definition[斤,两,碗]{s.}{face; parte frontal da cabeça; rosto | topo; superfície | capa; exterior; a parte externa de um objeto ou a face frontal de um tecido | Matemática: superfície | cara; sentimento; emoção | geral; área total; abrangente; toda a região | lado; aspecto | escopo; escala; extensão; alcance; âmbito | farinha; farinha de trigo | pó; algo em pó | macarrão; \emph{noodle}}
    \definition{suf.}{sufixo para localização ou direção; anexado ao final de palavras que indicam localização, equivalente a 边}
    \definition{v.}{encarar algo | encontrar; revelar-se}
  \seealsoref{边}{bian1}
  \antonymref{里}{li3}
  \end{Phonetics}
\end{Entry}

\begin{Entry}{面子}{9,3}{⾯,⼦}
  \begin{Phonetics}{面子}{mian4zi5}[][HSK 5]
    \definition{s.}{face; exterior; parte externa; superfície do objeto | imagem; reputação; prestígio; decência; vaidade superficial | sentimentos; sensibilidades | pó}
  \end{Phonetics}
\end{Entry}

\begin{Entry}{面包}{9,5}{⾯,⼓}
  \begin{Phonetics}{面包}{mian4bao1}[][HSK 1]
    \definition[个,片,袋,块]{s.}{pão}[我买八个面包了。===Comprei oito pães. | 他在吃两片面包。===Ele está comendo duas fatias de pão. | 我在家里带了一袋面包。===Trouxe um saco de pão para casa. | 我拿了一块面包。===Peguei um pedaço de pão.]
  \end{Phonetics}
\end{Entry}

\begin{Entry}{面对}{9,5}{⾯,⼨}
  \begin{Phonetics}{面对}{mian4dui4}[][HSK 3]
    \definition{v.}{enfrentar; defrontar; olhar para (uma pessoa ou um objeto específico) | confrontar (problema); problemas, dificuldades e outras questões que precisam ser resolvidas e que merecem atenção}
  \end{Phonetics}
\end{Entry}

\begin{Entry}{面对面}{9,5,9}{⾯,⼨,⾯}
  \begin{Phonetics}{面对面}{mian4dui4mian4}[][HSK 6]
    \definition{adj./expr.}{frente a frente; cara a cara; vis"-à"-vis}
  \end{Phonetics}
\end{Entry}

\begin{Entry}{面对面吃面}{9,5,9,6,9}{⾯,⼨,⾯,⼝,⾯}
  \begin{Phonetics}{面对面吃面}{mian4dui4mian4 chi1 mian4}
    \definition{expr.}{Comer macarrão cara a cara; indica que o seu estado atual, ou algumas das posições em que você está, ou algumas das coisas que você fez são muito claras}
  \end{Phonetics}
\end{Entry}

\begin{Entry}{面目全非}{9,5,6,8}{⾯,⽬,⼊,⾮}
  \begin{Phonetics}{面目全非}{mian4mu4-quan2fei1}[][HSK 7-9]
    \definition{expr.}{perder a própria identidade; uma mudança completa; tudo parece errado ou diferente; ser alterado (distorcido) a ponto de ficar irreconhecível; não ser mais como era antes; ser muito diferente do original; os originais não existem mais; a aparência das coisas mudou drasticamente (frequentemente com uma conotação negativa); mudança além do reconhecimento (frequentemente com uma conotação pejorativa)}
  \end{Phonetics}
\end{Entry}

\begin{Entry}{面向}{9,6}{⾯,⼝}
  \begin{Phonetics}{面向}{mian4xiang4}[][HSK 6]
    \definition{v.}{virar o rosto para; virar na direção de; defrontar; voltado para algum lugar | estar orientado para as necessidades de; atender a; principalmente para um certo tipo de pessoas}
  \end{Phonetics}
\end{Entry}

\begin{Entry}{面团}{9,6}{⾯,⼞}
  \begin{Phonetics}{面团}{mian4tuan2}
    \definition{s.}{massa | pasta}
  \end{Phonetics}
\end{Entry}

\begin{Entry}{面红耳赤}{9,6,6,7}{⾯,⽷,⽿,⾚}
  \begin{Phonetics}{面红耳赤}{mian4hong2-er3chi4}[][HSK 7-9]
    \definition{expr.}{ficar corado; ficar ruborizado de raiva; descreve um rosto corado devido à impaciência ou timidez}
  \end{Phonetics}
\end{Entry}

\begin{Entry}{面条}{9,7}{⾯,⽊}
  \begin{Phonetics}{面条}{mian4tiao2}
    \definition{s.}{macarrão | espaguete}
  \end{Phonetics}
\end{Entry}

\begin{Entry}{面条儿}{9,7,2}{⾯,⽊,⼉}
  \begin{Phonetics}{面条儿}{mian4tiao2r5}[][HSK 1]
    \definition{s.}{macarrão; \emph{noodles}}
  \end{Phonetics}
\end{Entry}

\begin{Entry}{面试}{9,8}{⾯,⾔}
  \begin{Phonetics}{面试}{mian4shi4}[][HSK 4]
    \definition{v.}{entrevistar (é realizado na forma de perguntas e respostas orais presenciais)}
  \end{Phonetics}
\end{Entry}

\begin{Entry}{面临}{9,9}{⾯,⼁}
  \begin{Phonetics}{面临}{mian4lin2}[][HSK 4]
    \definition{v.}{ser confrontado com; encontrar (uma situação) na frente de}
  \end{Phonetics}
\end{Entry}

\begin{Entry}{面前}{9,9}{⾯,⼑}
  \begin{Phonetics}{面前}{mian4qian2}[][HSK 2]
    \definition{s.}{antes; na frente de; diante de}
  \end{Phonetics}
\end{Entry}

\begin{Entry}{面面俱到}{9,9,10,8}{⾯,⾯,⼈,⼑}
  \begin{Phonetics}{面面俱到}{mian4mian4-ju4dao4}[][HSK 7-9]
    \definition{expr.}{cuidar de tudo; resolver tudo; contemplar todos os aspectos e não deixa nada de fora; dar atenção a todos os aspectos de uma questão}
  \end{Phonetics}
\end{Entry}

\begin{Entry}{面积}{9,10}{⾯,⽲}
  \begin{Phonetics}{面积}{mian4ji1}[][HSK 3]
    \definition{s.}{área (de um andar, pedaço de terreno, etc.); área de uma superfície; o tamanho de uma superfície plana ou da superfície de um objeto}
  \end{Phonetics}
\end{Entry}

\begin{Entry}{面粉}{9,10}{⾯,⽶}
  \begin{Phonetics}{面粉}{mian4fen3}[][HSK 7-9]
    \definition[份]{s.}{farinha; farinha de trigo}
  \end{Phonetics}
\end{Entry}

\begin{Entry}{面部}{9,10}{⾯,⾢}
  \begin{Phonetics}{面部}{mian4bu4}[][HSK 7-9]
    \definition{s.}{rosto; face}
  \end{Phonetics}
\end{Entry}

\begin{Entry}{面貌}{9,14}{⾯,⾘}
  \begin{Phonetics}{面貌}{mian4mao4}[][HSK 5]
    \definition[种,个]{s.}{rosto; traços faciais; formato do rosto; aparência | aparência; aspecto; aparência (das coisas)}
  \end{Phonetics}
\end{Entry}

%%%%% EOF %%%%%

