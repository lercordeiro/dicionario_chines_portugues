%%%
%%% Radical "⼿"
%%%
\section*{Radical 64: ``⼿'' (扌、龵)}\addcontentsline{toc}{section}{Radical 64: ⼿、扌、龵}\addcontentsline{loh}{figure}{\#\#\#\# 64: ⼿}

%%%%%%%%%% 才 %%%%%%%%%%
\subsection*{才}\addcontentsline{loh}{figure}{才}

\begin{Entry}{才}{3}{⼿}
  \begin{Phonetics}{才}{cai2}[][HSK 2,4]
    \definition*{s.}{Sobrenome: Cai}
    \definition{adv.}{há pouco; agora mesmo | (precedido por uma expressão de tempo) não até | (precedido por uma expressão de razão ou condição) não a menos que; não até que; então e somente então; por nenhuma outra razão | (seguido por uma expressão numérica) apenas; indica um intervalo pequeno ou uma quantidade reduzida, equivalente a 仅仅 ou 只 | (em uma afirmação ou negação, enfatizando o que vem antes de 才, geralmente com 呢 no final da frase) na verdade; realmente | dica que algo acontece tarde ou termina tarde | (precedido por uma expressão de tempo) não até; indicando que não era assim, mas agora surgiu uma nova situação | (precedido por uma expressão de razão ou condição) a menos que; indica que só em determinadas condições e, em seguida, como | (expressa ênfase )}
    \definition{s.}{habilidade; talento; dom | pessoa competente | pessoas de um determinado tipo (frequentemente usado como sufixo) | dotação; talento; habilidade}
  \seealsoref{呢}{ne5}
  \end{Phonetics}
\end{Entry}

\begin{Entry}{才华}{3,6}{⼿,⼗}
  \begin{Phonetics}{才华}{cai2hua2}[][HSK 7-9]
    \definition[份]{s.}{talento literário; talento artístico; talentos que são exibidos externamente (principalmente nas artes)}
  \end{Phonetics}
\end{Entry}

\begin{Entry}{才能}{3,10}{⼿,⾁}
  \begin{Phonetics}{才能}{cai2 neng2}[][HSK 3]
    \definition[间]{s.}{talento; habilidade; dom; capacidade; inteligência e habilidade}
  \end{Phonetics}
\end{Entry}

\begin{Entry}{才略}{3,11}{⼿,⽥}
  \begin{Phonetics}{才略}{cai2lve4}
    \definition{s.}{habilidade e sagacidade}
  \end{Phonetics}
\end{Entry}

%%%%%%%%%% 手 %%%%%%%%%%
\subsection*{手}\addcontentsline{loh}{figure}{手}

\begin{Entry}{手}{4}{⼿}[Kangxi 64]
  \begin{Phonetics}{手}{shou3}[][HSK 1]
    \definition{adj.}{prático; conveniente}
    \definition{adv.}{pessoalmente | para habilidade ou destreza}
    \definition{clas.}{usado para habilidades e competências | usado para indicar o número de vezes em que algo foi feito}
    \definition[双,只]{s.}{mão | pessoa proficiente em determinada atividade | habilidade; meios; referência a habilidades, técnicas ou meios | uma pessoa que faz ou é boa em determinado trabalho}
    \definition{v.}{ter na mão; segurar}
  \end{Phonetics}
\end{Entry}

\begin{Entry}{手工}{4,3}{⼿,⼯}
  \begin{Phonetics}{手工}{shou3gong1}[][HSK 4]
    \definition{s.}{trabalho manual; trabalho feito à mão | método de operação manual; método manual, sem máquina | remuneração por trabalho manual, braçal; custo de mão de obra braçal}
  \end{Phonetics}
\end{Entry}

\begin{Entry}{手工艺人}{4,3,4,2}{⼿,⼯,⾋,⼈}
  \begin{Phonetics}{手工艺人}{shou3gong1 yi4ren2}
    \definition{s.}{artesão}
  \end{Phonetics}
\end{Entry}

\begin{Entry}{手术}{4,5}{⼿,⽊}
  \begin{Phonetics}{手术}{shou3shu4}[][HSK 4]
    \definition[个,次]{s.}{cirurgia; operação (cirúrgica); método de tratamento no qual o médico usa uma faca, tesoura etc. para fazer uma incisão em uma parte do corpo do paciente}
    \definition{v.}{realizar uma cirurgia}
  \end{Phonetics}
\end{Entry}

\begin{Entry}{手边}{4,5}{⼿,⾡}
  \begin{Phonetics}{手边}{shou3bian1}
    \definition{adv.}{à mão | na mão}
  \end{Phonetics}
\end{Entry}

\begin{Entry}{手机}{4,6}{⼿,⽊}
  \begin{Phonetics}{手机}{shou3ji1}[][HSK 1]
    \definition[部,台,个]{s.}{celular; telefone celular; telefone móvel}
  \end{Phonetics}
\end{Entry}

\begin{Entry}{手里}{4,7}{⼿,⾥}
  \begin{Phonetics}{手里}{shou3 li3}[][HSK 4]
    \definition[个]{s.}{(uma situação está) nas mãos de alguém | em mãos}
  \end{Phonetics}
\end{Entry}

\begin{Entry}{手刹}{4,8}{⼿,⼑}
  \begin{Phonetics}{手刹}{shou3sha1}
    \definition{s.}{freio de mão}
  \end{Phonetics}
\end{Entry}

\begin{Entry}{手法}{4,8}{⼿,⽔}
  \begin{Phonetics}{手法}{shou3fa3}[][HSK 5]
    \definition[种,个]{s.}{habilidade; técnica; técnicas de criação (de obras literárias e artísticas) | truque; artifício; artimanha; refere-se a métodos inadequados usados para lidar com as pessoas}
  \end{Phonetics}
\end{Entry}

\begin{Entry}{手表}{4,8}{⼿,⾐}
  \begin{Phonetics}{手表}{shou3biao3}[][HSK 2]
    \definition[块,只,个]{s.}{relógio de pulso}
  \end{Phonetics}
\end{Entry}

\begin{Entry}{手指}{4,9}{⼿,⼿}
  \begin{Phonetics}{手指}{shou3zhi3}[][HSK 3]
    \definition[个,根,只]{s.}{dedo da mão}
  \end{Phonetics}
\end{Entry}

\begin{Entry}{手段}{4,9}{⼿,⽎}
  \begin{Phonetics}{手段}{shou3 duan4}[][HSK 5]
    \definition[种,个]{s.}{meios; meio; medida; método; métodos e técnicas utilizados para atingir um determinado objetivo | truque; artifício; métodos inadequados de lidar com as pessoas | habilidade; capacidade; delicadeza; sutileza; técnica}
  \end{Phonetics}
\end{Entry}

\begin{Entry}{手套}{4,10}{⼿,⼤}
  \begin{Phonetics}{手套}{shou3tao4}[][HSK 4]
    \definition[副,套,双,种]{s.}{luvas; itens usados ​​nas mãos, feitos de algodão, lã, couro, etc., para proteger as mãos ou manter o frio longe}
  \end{Phonetics}
\end{Entry}

\begin{Entry}{手续}{4,11}{⼿,⽷}
  \begin{Phonetics}{手续}{shou3xu4}[][HSK 3]
    \definition[项]{s.}{processo; formalidade; procedimento; procedimentos realizados de acordo com os regulamentos}
  \end{Phonetics}
\end{Entry}

\begin{Entry}{手续费}{4,11,9}{⼿,⽷,⾙}
  \begin{Phonetics}{手续费}{shou3 xu4 fei4}[][HSK 6]
    \definition{s.}{comissão; corretagem; taxa de serviço; taxas a pagar pelos procedimentos de manuseio}
  \end{Phonetics}
\end{Entry}

\begin{Entry}{手臂}{4,17}{⼿,⾁}
  \begin{Phonetics}{手臂}{shou3bi4}
    \definition{s.}{braço}
  \end{Phonetics}
\end{Entry}

%%%%%%%%%% 扎 %%%%%%%%%%
\subsection*{扎}\addcontentsline{loh}{figure}{扎}

\begin{Entry}{扎}{4}{⼿}
  \begin{Phonetics}{扎}{za1}
    \definition{clas.}{usado para pacotes, feixes, maços, etc.}
    \definition{v.}{atar; amarrar; embrulhar}
  \end{Phonetics}
  \begin{Phonetics}{扎}{zha1}[][HSK 6]
    \definition{s.}{chope; cerveja de pressão | caneca para chope}
    \definition{v.}{furar; esfaquear; enfiar (uma agulha, etc.) em | estacionar; aquartelar | entrar em; mergulhar em | trapacear; defraudar}
  \end{Phonetics}
  \begin{Phonetics}{扎}{zha2}
    \definition{v.}{lutar}
  \end{Phonetics}
\end{Entry}

\begin{Entry}{扎实}{4,8}{⼿,⼧}
  \begin{Phonetics}{扎实}{zha1shi5}[][HSK 6]
    \definition{adj.}{robusto; forte; sólido; confiável | sólido; pé no chão; prático}
  \end{Phonetics}
\end{Entry}

%%%%%%%%%% 扑 %%%%%%%%%%
\subsection*{扑}\addcontentsline{loh}{figure}{扑}

\begin{Entry}{扑}{5}{⼿}
  \begin{Phonetics}{扑}{pu1}[][HSK 6]
    \definition{s.}{sopro; refere-se a gases, fragrâncias, cinzas, areia, etc. que se apresentam | espanador}
    \definition{v.}{atacar; lançar-se sobre; correr para frente com toda a sua força e, de repente, jogar todo o seu corpo em um objeto | dedicar; dedicar todas as energias a uma causa; colocar toda a sua energia em (trabalho, carreira, etc.) | bater asas; esvoaçar | inclinar-se}
  \end{Phonetics}
\end{Entry}

\begin{Entry}{扑克}{5,7}{⼿,⼗}
  \begin{Phonetics}{扑克}{pu1ke4}[][HSK 7-9]
    \definition[副,张]{s.}{Eempréstimo linguístico: \emph{poker}  | baralho; cartas de baralho}
  \end{Phonetics}
\end{Entry}

\begin{Entry}{扑面而来}{5,9,6,7}{⼿,⾯,⽽,⽊}
  \begin{Phonetics}{扑面而来}{pu1mian4-er2lai2}[][HSK 7-9]
    \definition{expr.}{``Uma rajada de ar.''; vir de frente; bem na sua cara; está vindo direto na sua direção; indica uma situação em que uma ação, coisa, respiração ou palavra ocorre e chega diretamente}
  \end{Phonetics}
\end{Entry}

%%%%%%%%%% 扒 %%%%%%%%%%
\subsection*{扒}\addcontentsline{loh}{figure}{扒}

\begin{Entry}{扒}{5}{⼿}
  \begin{Phonetics}{扒}{ba1}[][HSK 7-9]
    \definition{v.}{segurar; agarrar-se a | cavar; varrer; puxar para baixo | empurrar para o lado | despir-se; tirar}
  \end{Phonetics}
  \begin{Phonetics}{扒}{pa2}
    \definition{v.}{reunir; juntar; reunir ou espalhar coisas com as mãos ou com um ancinho | roubar; furtar | arranhar; coçar com as mãos | cozinhar; refogar; cozinhar os alimentos em fogo baixo}
  \end{Phonetics}
\end{Entry}

\begin{Entry}{扒犁}{5,11}{⼿,⽜}
  \begin{Phonetics}{扒犁}{pa2li2}
    \definition{s.}{Dialeto: trenó; arado}
  \seealsoref{爬犁}{pa2li2}
  \end{Phonetics}
\end{Entry}

%%%%%%%%%% 打 %%%%%%%%%%
\subsection*{打}\addcontentsline{loh}{figure}{打}

\begin{Entry}{打}{5}{⼿}
  \begin{Phonetics}{打}{da2}
    \definition{clas./s.}{(empréstimo linguístico) dúzia}
  \end{Phonetics}
  \begin{Phonetics}{打}{da3}[][HSK 1,4,5]
    \definition{prep.}{de; desde; ponto de partida que indica lugar, tempo ou extensão; indica rotas e locais percorridos | devido a; origem da introdução de coisas novas}
    \definition{v.}{golpear; acertar; bater | quebrar; esmagar | lutar; atacar; espancar | entrar com uma ação judicial; negociar; fazer representações | construir; edificar | fabricar (em uma ferraria); forjar | misturar; mexer; bater | amarrar; embalar | tricotar; tecer | desenhar; pintar; deixar uma marca; imprimir | abrir; perfurar; cavar | içar; levantar | enviar; despachar; projetar | emitir ou receber (um certificado, etc.) | remover; livrar-se de | colher; tirar; retirar | comprar | capturar; caçar | reunir; coletar; colher; recolher através de ações como cortar e podar | estimar; calcular; contar; determinar | fazer; envolver-se em | jogar algum tipo de jogo | expressar certos movimentos corporais | adotar; usar; adotar uma determinada abordagem | pegar (um táxi) | indicar a melhora de seu estado mental; melhorar o estado mental}
  \end{Phonetics}
\end{Entry}

\begin{Entry}{打工}{5,3}{⼿,⼯}
  \begin{Phonetics}{打工}{da3gong1}[][HSK 2]
    \definition{v.}{contratar para trabalhar; trabalhar em tempo parcial; realizar trabalho manual (para alguém, geralmente temporariamente)}
  \end{Phonetics}
\end{Entry}

\begin{Entry}{打工人}{5,3,2}{⼿,⼯,⼈}
  \begin{Phonetics}{打工人}{da3gong1ren2}
    \definition{s.}{trabalhador}
  \end{Phonetics}
\end{Entry}

\begin{Entry}{打开}{5,4}{⼿,⼶}
  \begin{Phonetics}{打开}{da3 kai1}[][HSK 1]
    \definition{v.}{abrir; desdobrar; desenrolar | descobrir; revelar; desvendar | ativar; ligar; ligar o circuito | romper | abrir-se; espalhar-se; expandir; ampliar | abrir; iniciar o funcionamento do software, etc.}
  \end{Phonetics}
\end{Entry}

\begin{Entry}{打车}{5,4}{⼿,⾞}
  \begin{Phonetics}{打车}{da3 che1}[][HSK 1]
    \definition{v.}{pegar um táxi; chamar um táxi; dar sinal para um táxi}
  \end{Phonetics}
\end{Entry}

\begin{Entry}{打仗}{5,5}{⼿,⼈}
  \begin{Phonetics}{打仗}{da3/zhang4}[][HSK 7-9]
    \definition{v.+compl.}{lutar; ir à guerra; fazer guerra}
  \end{Phonetics}
\end{Entry}

\begin{Entry}{打击}{5,5}{⼿,⼐}
  \begin{Phonetics}{打击}{da3ji1}[][HSK 5]
    \definition{v.}{golpear; atacar; reprimir; atacar para frustrar; machucar | bater; bater (em um tambor, etc.); golpear ou bater em algo}
  \end{Phonetics}
\end{Entry}

\begin{Entry}{打包}{5,5}{⼿,⼓}
  \begin{Phonetics}{打包}{da3bao1}[][HSK 5]
    \definition{v.}{levar a comida embora; levar para viagem; refere-se especificamente a comer em um restaurante e levar as sobras em uma caixa, sacola ou outro recipiente | embalar; empacotar | desembalar; desempacotar}
  \end{Phonetics}
\end{Entry}

\begin{Entry}{打印}{5,5}{⼿,⼙}
  \begin{Phonetics}{打印}{da3yin4}[][HSK 2]
    \definition{v.}{imprimir; imprimir em papel ou outro suporte de gravação, como uma impressora}
  \end{Phonetics}
\end{Entry}

\begin{Entry}{打印机}{5,5,6}{⼿,⼙,⽊}
  \begin{Phonetics}{打印机}{da3 yin4 ji1}[][HSK 6]
    \definition[个,部,台]{s.}{impressora; uma máquina de escrever controlada por um microcomputador, sem teclado, que converte códigos de caracteres em caracteres e os imprime}
  \end{Phonetics}
\end{Entry}

\begin{Entry}{打发}{5,5}{⼿,⼜}
  \begin{Phonetics}{打发}{da3 fa5}[][HSK 6]
    \definition{v.}{enviar; despachar | dispensar; mandar embora | passar o tempo; matar o tempo}
  \end{Phonetics}
\end{Entry}

\begin{Entry}{打电话}{5,5,8}{⼿,⽥,⾔}
  \begin{Phonetics}{打电话}{da3 dian4 hua4}[][HSK 1]
    \definition{v.}{telefonar; fazer uma chamada telefônica; dar um telefonema}
  \seealsoref{给…打电话}{gei3 da3 dian4 hua4}
  \end{Phonetics}
\end{Entry}

\begin{Entry}{打交道}{5,6,12}{⼿,⼇,⾡}
  \begin{Phonetics}{打交道}{da3 jiao1dao5}[][HSK 7-9]
    \definition{v.}{mediar; formar equipe; entrar em contato com; fazer contato com; ter relações com}
  \end{Phonetics}
\end{Entry}

\begin{Entry}{打动}{5,6}{⼿,⼒}
  \begin{Phonetics}{打动}{da3 dong4}[][HSK 6]
    \definition{v.}{mover; tocar}[这番话打动了她的心。===Essas palavras tocaram seu coração.]
  \end{Phonetics}
\end{Entry}

\begin{Entry}{打压}{5,6}{⼿,⼚}
  \begin{Phonetics}{打压}{da3ya1}
    \definition{v.}{reprimir | derrotar}
  \end{Phonetics}
\end{Entry}

\begin{Entry}{打扫}{5,6}{⼿,⼿}
  \begin{Phonetics}{打扫}{da3sao3}[][HSK 4]
    \definition{v.}{varrer; limpar; varrer para limpar}
  \end{Phonetics}
\end{Entry}

\begin{Entry}{打听}{5,7}{⼿,⼝}
  \begin{Phonetics}{打听}{da3ting5}[][HSK 3]
    \definition{v.}{perguntar sobre; indagar sobre; obter uma linha sobre}
  \end{Phonetics}
\end{Entry}

\begin{Entry}{打屁股}{5,7,8}{⼿,⼫,⾁}
  \begin{Phonetics}{打屁股}{da3pi4gu5}
    \definition{v.}{dar um tapa no bumbum de alguém}
  \end{Phonetics}
\end{Entry}

\begin{Entry}{打岔}{5,7}{⼿,⼭}
  \begin{Phonetics}{打岔}{da3/cha4}[][HSK 7-9]
    \definition{s.}{interrupção}
    \definition{v.+compl.}{interromper; cortar | mudar de assunto | interromper (especialmente a fala)}
  \end{Phonetics}
\end{Entry}

\begin{Entry}{打扮}{5,7}{⼿,⼿}
  \begin{Phonetics}{打扮}{da3ban5}[][HSK 5]
    \definition{s.}{estilo de se vestir; o modo de se vestir; as roupas que se usa}
    \definition{v.}{vestir-se bem; maquiar-se; dar uma boa aparência e vestir-se bem; adornar}
  \end{Phonetics}
\end{Entry}

\begin{Entry}{打扰}{5,7}{⼿,⼿}
  \begin{Phonetics}{打扰}{da3rao3}[][HSK 5]
    \definition{v.}{perturbar; incomodar; interferir no trabalho normal, na vida ou no que as outras pessoas estão fazendo, etc. | usado para expressar um pedido de desculpas por ajuda; gratidão por ajuda; hospitalidade recebida}
  \end{Phonetics}
\end{Entry}

\begin{Entry}{打折}{5,7}{⼿,⼿}
  \begin{Phonetics}{打折}{da3/zhe2}[][HSK 4]
    \definition{v.+compl.}{dar desconto; dar um desconto; vender produtos a um preço reduzido em uma determinada porcentagem do preço original; metáfora para não cumprir 100\% do que foi originalmente padronizado ou prometido}
  \end{Phonetics}
\end{Entry}

\begin{Entry}{打针}{5,7}{⼿,⾦}
  \begin{Phonetics}{打针}{da3/zhen1}[][HSK 4]
    \definition{v.+compl.}{dar ou receber uma injeção; injetar um medicamento líquido em um organismo com uma seringa}
  \end{Phonetics}
\end{Entry}

\begin{Entry}{打官司}{5,8,5}{⼿,⼧,⼝}
  \begin{Phonetics}{打官司}{da3/guan1si5}[][HSK 6]
    \definition{v.+compl.}{ir ao tribunal (ou à lei); envolver-se em um processo judicial}
  \end{Phonetics}
\end{Entry}

\begin{Entry}{打招呼}{5,8,8}{⼿,⼿,⼝}
  \begin{Phonetics}{打招呼}{da3 zhao1hu5}[][HSK 7-9]
    \definition{v.}{cumprimentar alguém; dizer olá; tirar o chapéu; saudações por meio de palavras ou gestos | avisar; lembrar; informar; notificar com antecedência; dar aviso prévio}
  \end{Phonetics}
\end{Entry}

\begin{Entry}{打的}{5,8}{⼿,⽩}
  \begin{Phonetics}{打的}{da3/di1}
    \definition{v.+compl.}{(coloquial) pegar um táxi | ir de táxi}
  \end{Phonetics}
\end{Entry}

\begin{Entry}{打败}{5,8}{⼿,⾒}
  \begin{Phonetics}{打败}{da3 bai4}[][HSK 4]
    \definition{v.}{derrotar; vencer; piorar | sofrer uma derrota; ser derrotado}
  \end{Phonetics}
\end{Entry}

\begin{Entry}{打架}{5,9}{⼿,⽊}
  \begin{Phonetics}{打架}{da3/jia4}[][HSK 5]
    \definition{v.+compl.}{brigar; discutir; entrar em conflito | contradizer; conflitar; ser inconsistente}
  \end{Phonetics}
\end{Entry}

\begin{Entry}{打盹儿}{5,9,2}{⼿,⽬,⼉}
  \begin{Phonetics}{打盹儿}{da3/dun3r5}[][HSK 7-9]
    \definition{v.+compl.}{cochilar; tirar uma soneca}
  \end{Phonetics}
\end{Entry}

\begin{Entry}{打结}{5,9}{⼿,⽷}
  \begin{Phonetics}{打结}{da3jie2}
    \definition{v.}{dar um nó | amarrar}
  \end{Phonetics}
\end{Entry}

\begin{Entry}{打骂}{5,9}{⼿,⾺}
  \begin{Phonetics}{打骂}{da3ma4}
    \definition{v.}{bater e repreender}
  \end{Phonetics}
\end{Entry}

\begin{Entry}{打倒}{5,10}{⼿,⼈}
  \begin{Phonetics}{打倒}{da3/dao3}[][HSK 7-9]
    \definition{v.+compl.}{atacar e derrubar no chão; cair | derrubar; tombar}[打倒法西斯政权。===Derrubar o regime fascista.]
  \end{Phonetics}
\end{Entry}

\begin{Entry}{打捞}{5,10}{⼿,⼿}
  \begin{Phonetics}{打捞}{da3lao1}[][HSK 7-9]
    \definition{s.}{salvamento; resgate}
    \definition{v.}{sair da água; resgatar | encontrar e recuperar objetos que afundaram na água}
  \end{Phonetics}
\end{Entry}

\begin{Entry}{打破}{5,10}{⼿,⽯}
  \begin{Phonetics}{打破}{da3 po4}[][HSK 3]
    \definition{v.}{quebrar; esmagar; quebrar recordes, regras ou restrições existentes, etc.}
  \end{Phonetics}
\end{Entry}

\begin{Entry}{打通}{5,10}{⼿,⾡}
  \begin{Phonetics}{打通}{da3/tong1}[][HSK 7-9]
    \definition{v.+compl.}{passar; abrir-se | estabelecer contato | abrir o acesso | passar (uma conexão telefônica) | remover um bloco}
  \end{Phonetics}
\end{Entry}

\begin{Entry}{打造}{5,10}{⼿,⾡}
  \begin{Phonetics}{打造}{da3 zao4}[][HSK 6]
    \definition{v.}{forjar (trabalhar em metal); fabricar (principalmente objetos de metal) | fazer; criar; construir; desenvolver}
  \end{Phonetics}
\end{Entry}

\begin{Entry}{打断}{5,11}{⼿,⽄}
  \begin{Phonetics}{打断}{da3 duan4}[][HSK 6]
    \definition{v.}{interromper uma atividade (fala; pensamento ou ação) | fraturar (osso do corpo)  com força | arrombar; bater com força para quebrar}
  \end{Phonetics}
\end{Entry}

\begin{Entry}{打猎}{5,11}{⼿,⽝}
  \begin{Phonetics}{打猎}{da3/lie4}[][HSK 7-9]
    \definition{v.+compl.}{ir caçar}
  \end{Phonetics}
\end{Entry}

\begin{Entry}{打球}{5,11}{⼿,⽟}
  \begin{Phonetics}{打球}{da3 qiu2}[][HSK 1]
    \definition{v.}{jogar bola (com as mãos) | jogar (basquetebol, handbol, etc.) | jogar um jogo de bola}
  \end{Phonetics}
\end{Entry}

\begin{Entry}{打搅}{5,12}{⼿,⼿}
  \begin{Phonetics}{打搅}{da3jiao3}[][HSK 7-9]
    \definition{v.}{perturbar; incomodar | interromper}
  \end{Phonetics}
\end{Entry}

\begin{Entry}{打牌}{5,12}{⼿,⽚}
  \begin{Phonetics}{打牌}{da3 pai2}[][HSK 6]
    \definition{v.}{jogar cartas, usar cartas para entretenimento ou jogos de azar}
  \end{Phonetics}
\end{Entry}

\begin{Entry}{打量}{5,12}{⼿,⾥}
  \begin{Phonetics}{打量}{da3liang5}[][HSK 7-9]
    \definition{v.}{aumentar o tamanho; medir com o olho; olhar de cima a baixo; examinar; observar | pensar; calcular; supor; estimar}
  \end{Phonetics}
\end{Entry}

\begin{Entry}{打雷}{5,13}{⼿,⾬}
  \begin{Phonetics}{打雷}{da3 lei2}[][HSK 4]
    \definition{v.}{trovejar; produzir ruídos altos quando as nuvens descarregam eletricidade}
  \end{Phonetics}
\end{Entry}

\begin{Entry}{打算}{5,14}{⼿,⽵}
  \begin{Phonetics}{打算}{da3suan4}[][HSK 2]
    \definition[个,项]{s.}{plano; intenção; consideração; cálculo; ideias sobre a direção e os métodos da ação; pensamentos}
    \definition{v.}{pretender; planejar; calcular; considerar com antecedência}
  \end{Phonetics}
\end{Entry}

\begin{Entry}{打瞌睡}{5,15,13}{⼿,⽬,⽬}
  \begin{Phonetics}{打瞌睡}{da3ke1shui4}
    \definition{v.}{cochilar}
  \end{Phonetics}
\end{Entry}

\begin{Entry}{打磨}{5,16}{⼿,⽯}
  \begin{Phonetics}{打磨}{da3mo2}[][HSK 7-9]
    \definition{v.}{polir; dar brilho; fazer brilhar; esfregar a superfície de um objeto para torná-lo liso e delicado}
  \end{Phonetics}
\end{Entry}

%%%%%%%%%% 扔 %%%%%%%%%%
\subsection*{扔}\addcontentsline{loh}{figure}{扔}

\begin{Entry}{扔}{5}{⼿}
  \begin{Phonetics}{扔}{reng1}[][HSK 5]
    \definition{v.}{arremessar; lançar; atirar; jogar | esquecer; jogar fora; descartar | colocar casualmente; deixar as pessoas ou as coisas de lado, não se importar}
  \end{Phonetics}
\end{Entry}

\begin{Entry}{扔下}{5,3}{⼿,⼀}
  \begin{Phonetics}{扔下}{reng1xia4}
    \definition{v.}{lançar (uma bomba) | derrubar}
  \end{Phonetics}
\end{Entry}

\begin{Entry}{扔弃}{5,7}{⼿,⼶}
  \begin{Phonetics}{扔弃}{reng1qi4}
    \definition{v.}{abandonar | descartar | jogar fora}
  \end{Phonetics}
\end{Entry}

\begin{Entry}{扔掉}{5,11}{⼿,⼿}
  \begin{Phonetics}{扔掉}{reng1diao4}
    \definition{v.}{jogar fora}
  \end{Phonetics}
\end{Entry}

%%%%%%%%%% 托 %%%%%%%%%%
\subsection*{托}\addcontentsline{loh}{figure}{托}

\begin{Entry}{托}{6}{⼿}
  \begin{Phonetics}{托}{tuo1}[][HSK 6]
    \definition{clas.}{torr, uma unidade de pressão, 1 torr é igual à pressão de 1 mmHg, ou 133,322 Pa}
    \definition{s.}{algo servindo como suporte | fantoche; cúmplice; pessoas que ajudam golpistas a enganar outras pessoas}
    \definition{v.}{segurar na palma; apoiar com a mão ou palma; suportar (um objeto) com um objeto ou com a palma da mão | destacar; servir como contraste | pedir; confiar | implorar; dar como pretexto | dever a; confiar em}
  \end{Phonetics}
\end{Entry}

%%%%%%%%%% 扛 %%%%%%%%%%
\subsection*{扛}\addcontentsline{loh}{figure}{扛}

\begin{Entry}{扛}{6}{⼿}
  \begin{Phonetics}{扛}{gang1}
    \definition{v.}{levantar com as duas mãos | carregar alguma coisa juntos (duas ou mais pessoas)}
  \end{Phonetics}
  \begin{Phonetics}{扛}{kang2}[][HSK 7-9]
    \definition{v.}{carregar objetos nos ombros |  suportar; aguentar | lidar; assumir}
  \end{Phonetics}
\end{Entry}

%%%%%%%%%% 扣 %%%%%%%%%%
\subsection*{扣}\addcontentsline{loh}{figure}{扣}

\begin{Entry}{扣}{6}{⼿}
  \begin{Phonetics}{扣}{kou4}[][HSK 6]
    \definition*{s.}{Sobrenome: Kou}
    \definition{clas.}{giro; volta; uma volta de uma rosca}
    \definition[个,颗,粒]{s.}{nó | fivela; botão | círculo de rosca (em um parafuso)}
    \definition{v.}{fivela; abotoar; amarrar ou prender com um laço ou anel | colocar uma xícara, tigela etc. de cabeça para baixo; cobrir com uma xícara, tigela etc. invertida; colocar a boca do recipiente para baixo | deter; prender; levar sob custódia | cravar; esmagar (a bola); arremessar ou bater (em uma bola) com força de cima para baixo | atracar; deduzir; descontar; subtrair uma parte do valor original | puxar; pressionar | impor; marcar sem fundamento; acusar injustamente; impor ou atribuir (um crime ou má fama) a alguém}
  \end{Phonetics}
\end{Entry}

\begin{Entry}{扣人心弦}{6,2,4,8}{⼿,⼈,⼼,⼸}
  \begin{Phonetics}{扣人心弦}{kou4ren2xin1xian2}[][HSK 7-9]
    \definition{expr.}{emocionante; cativar alguém; conquistar o coração de; comovente; tocar os sentimentos de alguém; tocar o coração de alguém; tocar profundamente; muito tocante; descrever poesia, prosa, performances, etc., como tendo uma qualidade contagiante que desperta emoções; eletrizante; de ​​tirar o fôlego}
  \end{Phonetics}
\end{Entry}

\begin{Entry}{扣押}{6,8}{⼿,⼿}
  \begin{Phonetics}{扣押}{kou4ya1}[][HSK 7-9]
    \definition{s.}{detenção}
    \definition{v.}{deter; apreender; manter sob custódia | apreender; confiscar; embargar | sequestrar (ou tomar, reter) alguém como refém}
  \end{Phonetics}
\end{Entry}

\begin{Entry}{扣除}{6,9}{⼿,⾩}
  \begin{Phonetics}{扣除}{kou4chu2}[][HSK 7-9]
    \definition{v.}{deduzir; tirar; subtrair do total}
  \end{Phonetics}
\end{Entry}

\begin{Entry}{扣留}{6,10}{⼿,⽥}
  \begin{Phonetics}{扣留}{kou4liu2}[][HSK 7-9]
    \definition{s.}{apreensão; detenção}
    \definition{v.}{deter; manter sob custódia; pôr em prisão domiciliar; prender}
  \end{Phonetics}
\end{Entry}

%%%%%%%%%% 执 %%%%%%%%%%
\subsection*{执}\addcontentsline{loh}{figure}{执}

\begin{Entry}{执}{6}{⼿}
  \begin{Phonetics}{执}{zhi2}
    \definition*{s.}{Sobrenome: Zhi}
    \definition[期]{s.}{reconhecimento por escrito | (literário) amigo íntimo (ou do peito)}
    \definition{v.}{segurar; agarrar; pegar; capturar | assumir o comando de; dirigir; gerenciar; controlar; administrar; exercer | manter (os próprios pontos de vista, etc.); persistir; persistir em; manter-se em; insistir em | realizar; executar; implementar}
  \end{Phonetics}
\end{Entry}

\begin{Entry}{执行}{6,6}{⼿,⾏}
  \begin{Phonetics}{执行}{zhi2xing2}[][HSK 5]
    \definition{v.}{executar; implementar; realizar}
  \end{Phonetics}
\end{Entry}

\begin{Entry}{执着}{6,11}{⼿,⽬}
  \begin{Phonetics}{执着}{zhi2zhuo2}
    \definition{s.}{(budismo) apego}
    \definition{v.}{estar fortemente apegado a | ser dedicado | apegar-se a}
  \end{Phonetics}
\end{Entry}

%%%%%%%%%% 扩 %%%%%%%%%%
\subsection*{扩}\addcontentsline{loh}{figure}{扩}

\begin{Entry}{扩}{6}{⼿}
  \begin{Phonetics}{扩}{kuo4}[][HSK 7-9]
    \definition{v.}{expandir; ampliar; estender; alargar}
  \end{Phonetics}
\end{Entry}

\begin{Entry}{扩大}{6,3}{⼿,⼤}
  \begin{Phonetics}{扩大}{kuo4da4}[][HSK 4]
    \definition{v.}{ampliar; expandir; estender; alargar}
  \end{Phonetics}
\end{Entry}

\begin{Entry}{扩张}{6,7}{⼿,⼸}
  \begin{Phonetics}{扩张}{kuo4zhang1}[][HSK 7-9]
    \definition{v.}{expandir; aumentar; estender; espalhar; engrandecer | dilatar (dilatação vascular)}
  \end{Phonetics}
\end{Entry}

\begin{Entry}{扩建}{6,8}{⼿,⼵}
  \begin{Phonetics}{扩建}{kuo4jian4}[][HSK 7-9]
    \definition{v.}{expandir; ampliar; ampliar a escala original do edifício ou a escala da área}
  \end{Phonetics}
\end{Entry}

\begin{Entry}{扩展}{6,10}{⼿,⼫}
  \begin{Phonetics}{扩展}{kuo4 zhan3}[][HSK 4]
    \definition{v.}{esticar; expandir; estender; espalhar}
  \end{Phonetics}
\end{Entry}

\begin{Entry}{扩散}{6,12}{⼿,⽁}
  \begin{Phonetics}{扩散}{kuo4san4}[][HSK 7-9]
    \definition{v.}{espalhar; difundir; dispersar; proliferar}
  \end{Phonetics}
\end{Entry}

%%%%%%%%%% 扫 %%%%%%%%%%
\subsection*{扫}\addcontentsline{loh}{figure}{扫}

\begin{Entry}{扫}{6}{⼿}
  \begin{Phonetics}{扫}{sao3}[][HSK 4]
    \definition{v.}{varrer; limpar | passar rapidamente ao longo ou sobre; varrer | juntar tudo | Computação: scanear}
  \end{Phonetics}
  \begin{Phonetics}{扫}{sao4}
    \definition{s.}{elemento formadore de palavra}
  \seealsoref{扫帚}{sao4zhou5}
  \end{Phonetics}
\end{Entry}

\begin{Entry}{扫兴}{6,6}{⼿,⼋}
  \begin{Phonetics}{扫兴}{sao3/xing4}[][HSK 7-9]
    \definition{v.+compl.}{sentir-se desapontado; ter o ânimo abalado; quando você está se sentindo feliz, algo desagradável pode abalar seu ânimo}
  \end{Phonetics}
\end{Entry}

\begin{Entry}{扫帚}{6,8}{⼿,⼱}
  \begin{Phonetics}{扫帚}{sao4zhou5}
    \definition[把,个]{s.}{vassoura; ferramenta de varredura feita de varas de bambu, etc., maior que uma vassora}
  \end{Phonetics}
\end{Entry}

\begin{Entry}{扫除}{6,9}{⼿,⾩}
  \begin{Phonetics}{扫除}{sao3chu2}[][HSK 7-9]
    \definition{s.}{limpeza; arrumação}
    \definition{v.}{limpar; arrumar | limpar; remover; eliminar}
  \end{Phonetics}
\end{Entry}

\begin{Entry}{扫描}{6,11}{⼿,⼿}
  \begin{Phonetics}{扫描}{sao3miao2}[][HSK 7-9]
    \definition{v.}{digitalizar | dar uma olhada rápida; percorrer (o olhar, etc.) | utilizar software especializado para inspecionar e pesquisar (dados, vírus, etc. em computadores)}
  \end{Phonetics}
\end{Entry}

\begin{Entry}{扫墓}{6,13}{⼿,⼟}
  \begin{Phonetics}{扫墓}{sao3/mu4}[][HSK 7-9]
    \definition{v.+compl.}{limpar sepulturas e prestar homenagens aos mortos; também se refere à realização de atividades comemorativas nos túmulos dos mártires}
  \end{Phonetics}
\end{Entry}

%%%%%%%%%% 扬 %%%%%%%%%%
\subsection*{扬}\addcontentsline{loh}{figure}{扬}

\begin{Entry}{扬}{6}{⼿}
  \begin{Phonetics}{扬}{yang2}
    \definition*{s.}{Yangzhou, abreviação de 扬州 | Sobrenome: Yang}
    \definition{v.}{levantar | separar e espalhar; peneirar | espalhar; fazer conhecido}
  \seealsoref{扬州}{yang2zhou1}
  \end{Phonetics}
\end{Entry}

\begin{Entry}{扬州}{6,6}{⼿,⼮}
  \begin{Phonetics}{扬州}{yang2zhou1}
    \definition*{s.}{Yangzhou, uma cidade na província de Jiangsu}
  \end{Phonetics}
\end{Entry}

\begin{Entry}{扬雄}{6,12}{⼿,⾫}
  \begin{Phonetics}{扬雄}{yang2xiong2}
    \definition*{s.}{Yang Xiong (53 AC-18 DC), estudioso, poeta e lexicógrafo, autor do primeiro dicionário de dialeto chinês 方言}
  \seealsoref{方言}{fang1yan2}
  \end{Phonetics}
\end{Entry}

%%%%%%%%%% 扭 %%%%%%%%%%
\subsection*{扭}\addcontentsline{loh}{figure}{扭}

\begin{Entry}{扭}{7}{⼿}
  \begin{Phonetics}{扭}{niu3}[][HSK 6]
    \definition{v.}{virar-se; girar | torcer; girar | torcer; luxar | rolar; balançar (ao caminhar) | agarrar; pegar;  lutar com}
  \end{Phonetics}
\end{Entry}

\begin{Entry}{扭头}{7,5}{⼿,⼤}
  \begin{Phonetics}{扭头}{niu3/tou2}[][HSK 7-9]
    \definition{v.+compl.}{desviar o olhar; virar as costas | virar (em volta) | dar meia-volta | virar a cabeça}
  \end{Phonetics}
\end{Entry}

\begin{Entry}{扭曲}{7,6}{⼿,⽈}
  \begin{Phonetics}{扭曲}{niu3qu1}[][HSK 7-9]
    \definition{v.}{torcer; distorcer; deformação torsional | deformar; distorcer; deturpar; a distorção causa deformação e distorção}
  \end{Phonetics}
\end{Entry}

\begin{Entry}{扭转}{7,8}{⼿,⾞}
  \begin{Phonetics}{扭转}{niu3zhuan3}[][HSK 7-9]
    \definition{v.}{virar-se; inverter a marcha | dar meia-volta; reverter; corrigir situações anormais ou alterar circunstâncias desfavoráveis}
  \end{Phonetics}
\end{Entry}

%%%%%%%%%% 扮 %%%%%%%%%%
\subsection*{扮}\addcontentsline{loh}{figure}{扮}

\begin{Entry}{扮}{7}{⼿}
  \begin{Phonetics}{扮}{ban4}[][HSK 7-9]
    \definition{v.}{vestir-se como; desempenhar o papel de | maquiar-se; disfarçar-se como | (expressão facial) fazer cara de}
  \end{Phonetics}
\end{Entry}

\begin{Entry}{扮演}{7,14}{⼿,⽔}
  \begin{Phonetics}{扮演}{ban4yan3}[][HSK 5]
    \definition{v.}{desempenhar o papel de; ter um papel (em uma peça, etc.); atuar}
  \end{Phonetics}
\end{Entry}

%%%%%%%%%% 扯 %%%%%%%%%%
\subsection*{扯}\addcontentsline{loh}{figure}{扯}

\begin{Entry}{扯}{7}{⼿}
  \begin{Phonetics}{扯}{che3}[][HSK 7-9]
    \definition{v.}{puxar | rasgar; arrancar | comprar (tecido, linha, etc.) | conversar; fofocar; bater papo}
  \end{Phonetics}
\end{Entry}

%%%%%%%%%% 扰 %%%%%%%%%%
\subsection*{扰}\addcontentsline{loh}{figure}{扰}

\begin{Entry}{扰}{7}{⼿}
  \begin{Phonetics}{扰}{rao3}
    \definition*{s.}{Sobrenome: Rao}
    \definition{adj.}{desordenado; bagunçado}
    \definition{v.}{perturbar; importunar; causar problemas | abusar da hospitalidade de alguém}
  \end{Phonetics}
\end{Entry}

\begin{Entry}{扰乱}{7,7}{⼿,⼄}
  \begin{Phonetics}{扰乱}{rao3luan4}[][HSK 7-9]
    \definition{v.}{importunar; perturbar; causar confusão; usar palavras ou ações para interromper ou causar caos em um processo em andamento}
  \end{Phonetics}
\end{Entry}

%%%%%%%%%% 扳 %%%%%%%%%%
\subsection*{扳}\addcontentsline{loh}{figure}{扳}

\begin{Entry}{扳}{7}{⼿}
  \begin{Phonetics}{扳}{ban1}[][HSK 7-9]
    \definition{v.}{puxar; virar | reconquistar; compensar | reconquistar; virar o jogo; virar-se (uma situação de perda)}
  \end{Phonetics}
  \begin{Phonetics}{扳}{pan1}
    \definition{v.}{segurar; agarrar; puxar; escalar | confiar em; buscar ajuda; associar-se a pessoas de status superior; refere-se a formar um relacionamento ou estabelecer um relacionamento com alguém de alto \emph{status} | envolver; relacionar-se com}
  \end{Phonetics}
\end{Entry}

%%%%%%%%%% 扶 %%%%%%%%%%
\subsection*{扶}\addcontentsline{loh}{figure}{扶}

\begin{Entry}{扶}{7}{⼿}
  \begin{Phonetics}{扶}{fu2}[][HSK 5]
    \definition*{s.}{Sobrenome: Fu}
    \definition{v.}{segurar; apoiar com a mão; segurar algo com o apoio das mãos para que ninguém, objeto ou pessoa caia | dar apoio a; ajudar uma pessoa deitada ou caída a se levantar com as mãos; endireitar um objeto caído com as mãos | ajudar; tirar de baixo}
  \end{Phonetics}
\end{Entry}

\begin{Entry}{扶持}{7,9}{⼿,⼿}
  \begin{Phonetics}{扶持}{fu2chi2}[][HSK 7-9]
    \definition{v.}{apoiar com a mão; colocar uma mão em alguém para apoio; apoiar | apoiar; dar ajuda a; ajudar a sustentar}
  \end{Phonetics}
\end{Entry}

\begin{Entry}{扶梯}{7,11}{⼿,⽊}
  \begin{Phonetics}{扶梯}{fu2ti1}
    \definition{s.}{escada rolante}
  \end{Phonetics}
\end{Entry}

%%%%%%%%%% 批 %%%%%%%%%%
\subsection*{批}\addcontentsline{loh}{figure}{批}

\begin{Entry}{批}{7}{⼿}
  \begin{Phonetics}{批}{pi1}[][HSK 4]
    \definition{adj.}{(compra ou venda) atacado; a granel; em grandes quantidades}
    \definition{clas.}{usado para mercadorias a granel, grande número de pessoas}
    \definition{s.}{fibras de algodão, linho, etc., prontas para serem estiradas e torcidas | anotação; comentário}
    \definition{v.}{escrever comentários ou críticas sobre documentos subordinados, textos de outras pessoas, tarefas etc. | refutar; criticar | dar um tapa}
  \end{Phonetics}
\end{Entry}

\begin{Entry}{批发}{7,5}{⼿,⼜}
  \begin{Phonetics}{批发}{pi1fa1}[][HSK 7-9]
    \definition{v.}{verder no atacado; vender mercadorias a granel; comprar e vender mercadorias a granel}
  \end{Phonetics}
\end{Entry}

\begin{Entry}{批判}{7,7}{⼿,⼑}
  \begin{Phonetics}{批判}{pi1pan4}[][HSK 7-9]
    \definition[个]{s.}{crítica}[批判性地思考问题。===Analise os problemas de forma crítica.]
    \definition{v.}{criticar; repudiar; analisar e refutar sistematicamente pensamentos, afirmações ou ações errôneas}
  \end{Phonetics}
\end{Entry}

\begin{Entry}{批评}{7,7}{⼿,⾔}
  \begin{Phonetics}{批评}{pi1ping2}[][HSK 3]
    \definition{v.}{criticar; comentar sobre deficiências e erros | criticar; apontar vantagens e desvantagens; comentar sobre o que é bom e o que é ruim}
  \end{Phonetics}
\end{Entry}

\begin{Entry}{批准}{7,10}{⼿,⼎}
  \begin{Phonetics}{批准}{pi1zhun3}[][HSK 3]
    \definition{v.}{aprovar}
  \end{Phonetics}
\end{Entry}

%%%%%%%%%% 找 %%%%%%%%%%
\subsection*{找}\addcontentsline{loh}{figure}{找}

\begin{Entry}{找}{7}{⼿}
  \begin{Phonetics}{找}{zhao3}[][HSK 1]
    \definition{v.}{procurar; tentar encontrar; buscar | querer ver; visitar; abordar; solicitar | dar troco | descobrir; esforçar-se para ver ou obter a pessoa ou coisa desejada | examinar; investigar; completar as partes que faltam | causar intencionalmente (um resultado indesejável, negativo)}
  \end{Phonetics}
\end{Entry}

\begin{Entry}{找见}{7,4}{⼿,⾒}
  \begin{Phonetics}{找见}{zhao3jian4}
    \definition{v.}{encontrar (algo que está procurando)}
  \end{Phonetics}
\end{Entry}

\begin{Entry}{找出}{7,5}{⼿,⼐}
  \begin{Phonetics}{找出}{zhao3 chu1}[][HSK 2]
    \definition{v.}{encontrar | procurar}
  \end{Phonetics}
\end{Entry}

\begin{Entry}{找回}{7,6}{⼿,⼞}
  \begin{Phonetics}{找回}{zhao3hui2}
    \definition{v.}{recuperar algo}
  \end{Phonetics}
\end{Entry}

\begin{Entry}{找寻}{7,6}{⼿,⼨}
  \begin{Phonetics}{找寻}{zhao3xun2}
    \definition{v.}{encontrar falhas | procurar | buscar}
  \end{Phonetics}
\end{Entry}

\begin{Entry}{找事}{7,8}{⼿,⼅}
  \begin{Phonetics}{找事}{zhao3shi4}
    \definition{v.}{procurar emprego | começar uma briga}
  \end{Phonetics}
\end{Entry}

\begin{Entry}{找到}{7,8}{⼿,⼑}
  \begin{Phonetics}{找到}{zhao3 dao4}[][HSK 1]
    \definition{v.}{encontrar; procurar; achar; encontar através de pesquisa, exploração, etc.;  ver ou encontrar coisas ou padrões que os antepassados não viram}
  \end{Phonetics}
\end{Entry}

\begin{Entry}{找钱}{7,10}{⼿,⾦}
  \begin{Phonetics}{找钱}{zhao3qian2}
    \definition{v.}{dar troco}
  \end{Phonetics}
\end{Entry}

\begin{Entry}{找着}{7,11}{⼿,⽬}
  \begin{Phonetics}{找着}{zhao3zhao2}
    \definition{v.}{encontrar}
  \end{Phonetics}
\end{Entry}

\begin{Entry}{找遍}{7,12}{⼿,⾡}
  \begin{Phonetics}{找遍}{zhao3bian4}
    \definition{v.}{pentear | pesquisar em todos os lugares}
  \end{Phonetics}
\end{Entry}

\begin{Entry}{找零}{7,13}{⼿,⾬}
  \begin{Phonetics}{找零}{zhao3ling2}
    \definition{v.}{trocar dinheiro | dar troco}
  \end{Phonetics}
\end{Entry}

\begin{Entry}{找辙}{7,16}{⼿,⾞}
  \begin{Phonetics}{找辙}{zhao3zhe2}
    \definition{v.}{procurar um pretexto}
  \end{Phonetics}
\end{Entry}

%%%%%%%%%% 技 %%%%%%%%%%
\subsection*{技}\addcontentsline{loh}{figure}{技}

\begin{Entry}{技}{7}{⼿}
  \begin{Phonetics}{技}{ji4}
    \definition[门,项]{s.}{destreza; habilidade; estratagema | técnica; tecnologia}
  \end{Phonetics}
\end{Entry}

\begin{Entry}{技艺}{7,4}{⼿,⾋}
  \begin{Phonetics}{技艺}{ji4yi4}[][HSK 7-9]
    \definition{s.}{habilidade; arte; artes cênicas ou artesanato habilidosos}
  \end{Phonetics}
\end{Entry}

\begin{Entry}{技巧}{7,5}{⼿,⼯}
  \begin{Phonetics}{技巧}{ji4qiao3}[][HSK 4]
    \definition[个,些]{s.}{habilidade; técnica; habilidades engenhosas expressas em artes, artesanato, esportes, etc.}
  \end{Phonetics}
\end{Entry}

\begin{Entry}{技术}{7,5}{⼿,⽊}
  \begin{Phonetics}{技术}{ji4shu4}[][HSK 3]
    \definition[种,门,项]{s.}{habilidade; técnica; tecnologia; a experiência e o conhecimento acumulados pelo ser humano no processo de utilização e transformação da natureza, e refletidos no trabalho produtivo, também se referem, de maneira geral, a outras habilidades operacionais}
  \end{Phonetics}
\end{Entry}

\begin{Entry}{技俩}{7,9}{⼿,⼈}
  \begin{Phonetics}{技俩}{ji4liang3}
    \definition{s.}{truque | estratagema | ardil | esquema | estratégia | tática}
  \end{Phonetics}
\end{Entry}

\begin{Entry}{技能}{7,10}{⼿,⾁}
  \begin{Phonetics}{技能}{ji4 neng2}[][HSK 5]
    \definition[种,项]{s.}{habilidade técnica; domínio de uma habilidade ou técnica; capacidade de adquirir e aplicar conhecimento}
  \end{Phonetics}
\end{Entry}

%%%%%%%%%% 抄 %%%%%%%%%%
\subsection*{抄}\addcontentsline{loh}{figure}{抄}

\begin{Entry}{抄}{7}{⼿}
  \begin{Phonetics}{抄}{chao1}[][HSK 4]
    \definition*{s.}{Sobrenome: Chao}
    \definition{v.}{copiar; transcrever | plagiar | registrar as leituras de um medidor | revistar e confiscar; fazer uma incursão em  | pegar um atalho | dobrar (os braços) | agarrar; pegar | ir (andar) embora com}
  \end{Phonetics}
\end{Entry}

\begin{Entry}{抄写}{7,5}{⼿,⼍}
  \begin{Phonetics}{抄写}{chao1 xie3}[][HSK 4]
    \definition{v.}{copiar; transcrever}
  \end{Phonetics}
\end{Entry}

\begin{Entry}{抄表}{7,8}{⼿,⾐}
  \begin{Phonetics}{抄表}{chao1 biao3}
    \definition{s.}{leitura do medidor}
  \end{Phonetics}
\end{Entry}

\begin{Entry}{抄袭}{7,11}{⼿,⾐}
  \begin{Phonetics}{抄袭}{chao1xi2}[][HSK 7-9]
    \definition{v.}{plagiar | emular indiscriminadamente; copiar | Militar: lançar um ataque surpresa ao inimigo fazendo um desvio; atacar a retaguarda ou os flancos do inimigo | Militar: tomar emprestado indiscriminadamente da experiência de outras pessoas; copiar a experiência e os métodos de outras pessoas}
  \end{Phonetics}
\end{Entry}

%%%%%%%%%% 把 %%%%%%%%%%
\subsection*{把}\addcontentsline{loh}{figure}{把}

\begin{Entry}{把}{7}{⼿}
  \begin{Phonetics}{把}{ba3}[][HSK 3]
    \definition{adj.}{referindo-se à relação de irmandade}
    \definition{clas.}{usado antes de objetos com alças ou coisas para segurar | um punhado de; a quantidade que se pode pegar com uma mão | usado antes de coisas abstratas | usado em coisas feitas com as mãos | número de ações, coisas}
    \definition{part.}{adicionado após quantificadores como 百, 千, 万 e 里, 斤, 个, indica que a quantidade é próxima dessa unidade (não pode ser adicionado outro quantificador antes)}
    \definition{prep.}{fazer uma determinada alteração em um objeto; causar uma determinada mudança em um objeto | fazer com que os outros façam/sintam algo}
    \definition{s.}{alça; punho; a parte que se segura | feixe; molho; algo que se segura com as mãos ou se amarra em pequenos feixes}
    \definition{v.}{agarrar; segurar | segurar (um bebê enquanto ele urina) | controlar; dominar; monopolizar | encostar-se; apoiar-se | vigiar (locais importantes); observar; guardar | dar | usar algo como; considerar como; tratar como; conter o significado de 拿 | acorrentar; trancar}
  \seealsoref{百}{bai3}
  \seealsoref{个}{ge4}
  \seealsoref{斤}{jin1}
  \seealsoref{里}{li3}
  \seealsoref{拿}{na2}
  \seealsoref{千}{qian1}
  \seealsoref{万}{wan4}
  \end{Phonetics}
  \begin{Phonetics}{把}{ba4}
    \definition{s.}{punho; alça; empunhadura; parte do utensílio que é fácil de segurar com a mão |haste (de uma folha, flor ou fruto) | motivo de ridículo; alvo; comportamentos e declarações que servem de assunto para piadas}
  \end{Phonetics}
\end{Entry}

\begin{Entry}{把手}{7,4}{⼿,⼿}
  \begin{Phonetics}{把手}{ba3shou5}[][HSK 7-9]
    \definition[个]{s.}{pega; botão; alça; as partes de portas, janelas, móveis, etc. que são fáceis de segurar com as mãos}
  \end{Phonetics}
\end{Entry}

\begin{Entry}{把风}{7,4}{⼿,⾵}
  \begin{Phonetics}{把风}{ba3feng1}
    \definition{v.}{vigiar (em uma atividade clandestina) | estar atento}
  \end{Phonetics}
\end{Entry}

\begin{Entry}{把关}{7,6}{⼿,⼋}
  \begin{Phonetics}{把关}{ba3/guan1}[][HSK 7-9]
    \definition{v.+compl.}{verificar rigorosamente; examinar cuidadosamente para ver se algo está sendo feito de acordo com o padrão fixo; fazer a verificação final | proteger uma fronteira, passagem, portões, etc.}
  \end{Phonetics}
\end{Entry}

\begin{Entry}{把守}{7,6}{⼿,⼧}
  \begin{Phonetics}{把守}{ba3shou3}
    \definition{v.}{vigiar | guardar}
  \end{Phonetics}
\end{Entry}

\begin{Entry}{把式}{7,6}{⼿,⼷}
  \begin{Phonetics}{把式}{ba3shi4}
    \definition{s.}{pessoa qualificada em um comércio}
  \end{Phonetics}
\end{Entry}

\begin{Entry}{把戏}{7,6}{⼿,⼽}
  \begin{Phonetics}{把戏}{ba3xi4}
    \definition{s.}{acrobacia | malabarismo | truque barato}
  \end{Phonetics}
\end{Entry}

\begin{Entry}{把玩}{7,8}{⼿,⽟}
  \begin{Phonetics}{把玩}{ba3wan2}
    \definition{v.}{brincar com | mexer com | girar nas mãos}
  \end{Phonetics}
\end{Entry}

\begin{Entry}{把持}{7,9}{⼿,⼿}
  \begin{Phonetics}{把持}{ba3chi2}
    \definition{v.}{(frequentemente pejorativo) dominar; monopolizar | controlar (os próprios sentimentos, etc.) | manter sob controle}
  \end{Phonetics}
\end{Entry}

\begin{Entry}{把柄}{7,9}{⼿,⽊}
  \begin{Phonetics}{把柄}{ba3bing3}[][HSK 7-9]
    \definition[个]{s.}{alça; a parte de um objeto que é fácil de segurar com as mãos | evidências que podem ser obtidas em ações judiciais ou argumentos; uma metáfora para um erro ou falha que pode ser usada para chantagear alguém}
  \end{Phonetics}
\end{Entry}

\begin{Entry}{把脉}{7,9}{⼿,⾁}
  \begin{Phonetics}{把脉}{ba3mai4}
    \definition{v.}{resolver problemas por meio de investigação e estudo | sentir o pulso | para tomar o pulso de alguém}
  \end{Phonetics}
\end{Entry}

\begin{Entry}{把握}{7,12}{⼿,⼿}
  \begin{Phonetics}{把握}{ba3wo4}[][HSK 3]
    \definition[的]{s.}{seguro; garantia; certeza; confiabilidade do sucesso}
    \definition{v.}{agarrar; segurar; apreender |  (algo abstrato) agarrar; segurar}
  \end{Phonetics}
\end{Entry}

\begin{Entry}{把稳}{7,14}{⼿,⽲}
  \begin{Phonetics}{把稳}{ba3wen3}
    \definition{adj.}{confiável}
    \definition{v.}{ter certeza de; ser firme; manter-se firme}
  \end{Phonetics}
\end{Entry}

%%%%%%%%%% 抓 %%%%%%%%%%
\subsection*{抓}\addcontentsline{loh}{figure}{抓}

\begin{Entry}{抓}{7}{⼿}
  \begin{Phonetics}{抓}{zhua1}[][HSK 3]
    \definition{v.}{agarrar; segurar; obter; apreender; juntar os dedos para segurar o objeto na mão | riscar; arranhar; usar as unhas, objetos com dentes ou garras de animais para riscar a superfície de um objeto | apanhar; capturar; controlar pessoas ou animais; fazer com que pessoas ou animais caiam nas mãos de alguém | compreender; saber onde está o ponto principal ou a chave de uma questão ou problema | concentrar-se em algo; reforçar a força para fazer (alguma coisa), controlar (algum aspecto) | chamar a atenção de alguém; atrair a atenção}
  \end{Phonetics}
\end{Entry}

\begin{Entry}{抓住}{7,7}{⼿,⼈}
  \begin{Phonetics}{抓住}{zhua1 zhu4}[][HSK 3]
    \definition{v.}{prender; deter; capturar (pessoas ou animais) e ter sucesso | segurar; agarrar; apreender; agarrar algo para que não se mova}
  \end{Phonetics}
\end{Entry}

\begin{Entry}{抓紧}{7,10}{⼿,⽷}
  \begin{Phonetics}{抓紧}{zhua1jin3}[][HSK 4]
    \definition{v.}{agarrar com firmeza; segurar firme e não soltar | prestar muita atenção a}
  \end{Phonetics}
\end{Entry}

%%%%%%%%%% 投 %%%%%%%%%%
\subsection*{投}\addcontentsline{loh}{figure}{投}

\begin{Entry}{投}{7}{⼿}
  \begin{Phonetics}{投}{tou2}[][HSK 4]
    \definition*{s.}{Sobrenome: Tou}
    \definition{pron.}{para; indica tempo, equivalente a 到, 临 | para; em direção a; indica orientação, direção, equivalente a 朝 ou 向}
    \definition{s.}{um jogo durante uma festa em que o vencedor era decidido pelo número de flechas lançadas em um pote distante | jogo de dados}
    \definition{v.}{lançar; arremessar; atirar | deixar cair; colocar em; lançar | mergulhar em; lançar-se em; pular dentro | lançar; projetar; sombrear | entregar; postar; enviar | ir até; ir para; buscar; juntar-se | sentir-se atraído por; adaptar-se a; concordar com; atender a}
  \seealsoref{朝}{chao2}
  \seealsoref{到}{dao4}
  \seealsoref{临}{lin2}
  \seealsoref{向}{xiang4}
  \end{Phonetics}
\end{Entry}

\begin{Entry}{投入}{7,2}{⼿,⼊}
  \begin{Phonetics}{投入}{tou2ru4}[][HSK 4]
    \definition{adj.}{sisudo; dedicado; devotado; absorto}
    \definition{s.}{investimento; insumo; refere-se à aplicação de recursos}
    \definition{v.}{lançar em; colocar em; jogar em; por em | entrar em uma situação; participar de | aplicar; investir; colocar fundos em}
  \end{Phonetics}
\end{Entry}

\begin{Entry}{投诉}{7,7}{⼿,⾔}
  \begin{Phonetics}{投诉}{tou2su4}[][HSK 4]
    \definition{v.}{reclamar; queixar-se; reclamar às autoridades ou pessoas envolvidas}
  \end{Phonetics}
\end{Entry}

\begin{Entry}{投资}{7,10}{⼿,⾙}
  \begin{Phonetics}{投资}{tou2zi1}[][HSK 4]
    \definition[笔]{s.}{investimento}
    \definition{v.}{investir; aplicar dinheiro; investir dinheiro em negócios}
  \end{Phonetics}
\end{Entry}

\begin{Entry}{投资人}{7,10,2}{⼿,⾙,⼈}
  \begin{Phonetics}{投资人}{tou2zi1ren2}
    \definition{s.}{investidor}
  \seealsoref{投资家}{tou2zi1jia1}
  \seealsoref{投资者}{tou2zi1zhe3}
  \end{Phonetics}
\end{Entry}

\begin{Entry}{投资风险}{7,10,4,9}{⼿,⾙,⾵,⾩}
  \begin{Phonetics}{投资风险}{tou2zi1 feng1xian3}
    \definition*{s.}{risco de investimento}
  \end{Phonetics}
\end{Entry}

\begin{Entry}{投资回报率}{7,10,6,7,11}{⼿,⾙,⼞,⼿,⽞}
  \begin{Phonetics}{投资回报率}{tou2zi1 hui2bao4 lv4}
    \definition{s.}{retorno sobre o investimento (ROI)}
  \end{Phonetics}
\end{Entry}

\begin{Entry}{投资者}{7,10,8}{⼿,⾙,⽼}
  \begin{Phonetics}{投资者}{tou2zi1zhe3}
    \definition{s.}{investidor}
  \seealsoref{投资家}{tou2zi1jia1}
  \seealsoref{投资人}{tou2zi1ren2}
  \end{Phonetics}
\end{Entry}

\begin{Entry}{投资家}{7,10,10}{⼿,⾙,⼧}
  \begin{Phonetics}{投资家}{tou2zi1jia1}
    \definition{s.}{investidor}
  \seealsoref{投资人}{tou2zi1ren2}
  \seealsoref{投资者}{tou2zi1zhe3}
  \end{Phonetics}
\end{Entry}

\begin{Entry}{投递}{7,10}{⼿,⾡}
  \begin{Phonetics}{投递}{tou2di4}
    \definition{v.}{despachar | enviar}
  \end{Phonetics}
\end{Entry}

\begin{Entry}{投票}{7,11}{⼿,⽰}
  \begin{Phonetics}{投票}{tou2/piao4}[][HSK 6]
    \definition{v.+compl.}{votar; dar um voto; um método de eleição no qual os eleitores escrevem o nome da pessoa que querem eleger na cédula, ou marcam a cédula com o nome do candidato impresso e depois a colocam na urna para votar na resolução}
  \end{Phonetics}
\end{Entry}

%%%%%%%%%% 抖 %%%%%%%%%%
\subsection*{抖}\addcontentsline{loh}{figure}{抖}

\begin{Entry}{抖}{7}{⼿}
  \begin{Phonetics}{抖}{dou3}[][HSK 7-9]
    \definition{v.}{tremer; estremecer; arrepiar | sacudir | despertar; agitar | Coloquial geralmente sarcástico: (usualmente com 起来) progredir no mundo; ganhar fama (ou fortuna)}
  \seealsoref{起来}{qi3lai5}
  \end{Phonetics}
\end{Entry}

%%%%%%%%%% 抗 %%%%%%%%%%
\subsection*{抗}\addcontentsline{loh}{figure}{抗}

\begin{Entry}{抗}{7}{⼿}
  \begin{Phonetics}{抗}{kang4}[][HSK 6]
    \definition*{s.}{Sobrenome: Kang}
    \definition{pref.}{anti-}
    \definition{v.}{resistir; combater; lutar | recusar; desafiar}
  \end{Phonetics}
\end{Entry}

\begin{Entry}{抗生素}{7,5,10}{⼿,⽣,⽷}
  \begin{Phonetics}{抗生素}{kang4sheng1su4}[][HSK 7-9]
    \definition{s.}{antibiótico}
  \end{Phonetics}
\end{Entry}

\begin{Entry}{抗议}{7,5}{⼿,⾔}
  \begin{Phonetics}{抗议}{kang4yi4}[][HSK 6]
    \definition{v.}{protestar; reconsiderar; levantar objeções fortes}
  \end{Phonetics}
\end{Entry}

\begin{Entry}{抗争}{7,6}{⼿,⼑}
  \begin{Phonetics}{抗争}{kang4zheng1}[][HSK 7-9]
    \definition{v.}{resistir; opor-se; posicionar-se contra; confrontar; lutar}
  \end{Phonetics}
\end{Entry}

\begin{Entry}{抗拒}{7,7}{⼿,⼿}
  \begin{Phonetics}{抗拒}{kang4ju4}[][HSK 7-9]
    \definition{v.}{resistir; suportar}
  \end{Phonetics}
\end{Entry}

\begin{Entry}{抗衡}{7,16}{⼿,⾏}
  \begin{Phonetics}{抗衡}{kang4heng2}[][HSK 7-9]
    \definition{v.}{competir com; desafiar; rivalizar | servir de contrapeso a; competir com; rivalizar com; estar em pé de igualdade com}
  \end{Phonetics}
\end{Entry}

%%%%%%%%%% 折 %%%%%%%%%%
\subsection*{折}\addcontentsline{loh}{figure}{折}

\begin{Entry}{折}{7}{⼿}
  \begin{Phonetics}{折}{she2}
    \definition{clas.}{um ato de zaju | um parágrafo em um drama da Dinastia Yuan, aproximadamente equivalente a uma cena ou ato em uma ópera moderna}
    \definition[张,个,些]{s.}{abatimento; desconto | os traços dos caracteres chineses têm o formato de 𠃍 e 乚 | pasta; livreto}
    \definition{v.}{estalar; quebrar; fazer quebrar | perder; sofrer a perda de | dobrar; torcer; curvar-se | voltar; mudar de direção; retornar | estar convencido; estar cheio de admiração | equivaler a; converter em}
  \end{Phonetics}
  \begin{Phonetics}{折}{zhe1}
    \definition{v.}{rolar; virar | despejar algo de um recipiente em outro; ficar despejando algo entre dois recipientes}
  \end{Phonetics}
  \begin{Phonetics}{折}{zhe2}[][HSK 4]
    \definition*{s.}{Sobrenome: Zhe}
    \definition{clas.}{uma passagem em um roteiro de ópera miscelânea de Yuan, aproximadamente equivalente a uma cena ou ato em uma ópera moderna}
    \definition[张,个,些]{s.}{fratura; quebra | abatimento; desconto | traços dos caracteres chineses que têm o formato de ``𠃍'' e ``乚'', etc. | pasta; livreto; \emph{folder}}
    \definition{v.}{estalar; quebrar; fazer quebrar | perder; sofrer a perda de | voltar para trás; mudar de direção; retornar |ser convencido; estar cheio de admiração | equivaler a; converter em | dobrar}
  \end{Phonetics}
\end{Entry}

\begin{Entry}{折转}{7,8}{⼿,⾞}
  \begin{Phonetics}{折转}{zhe2zhuan3}
    \definition{s.}{reflexo (ângulo)}
    \definition{v.}{voltar atrás}
  \end{Phonetics}
\end{Entry}

%%%%%%%%%% 抚 %%%%%%%%%%
\subsection*{抚}\addcontentsline{loh}{figure}{抚}

\begin{Entry}{抚}{7}{⼿}
  \begin{Phonetics}{抚}{fu3}
    \definition{v.}{confortar; consolar | nutrir; fomentar | Literário: acariciar | proteger; promover; criar | o mesmo que 拊}
  \seealsoref{拊}{fu3}
  \end{Phonetics}
\end{Entry}

\begin{Entry}{抚养}{7,9}{⼿,⼋}
  \begin{Phonetics}{抚养}{fu3yang3}[][HSK 7-9]
    \definition{v.}{criar; cuidar; proporcionar às crianças as condições de vida necessárias para que possam crescer com saúde}
  \end{Phonetics}
\end{Entry}

\begin{Entry}{抚养费}{7,9,9}{⼿,⼋,⾙}
  \begin{Phonetics}{抚养费}{fu3yang3fei4}[][HSK 7-9]
    \definition{s.}{pensão alimentícia (após o divórcio) | pagamento pela educação dos filhos (como após o divórcio)}
  \end{Phonetics}
\end{Entry}

\begin{Entry}{抚恤}{7,9}{⼿,⼼}
  \begin{Phonetics}{抚恤}{fu3xu4}[][HSK 7-9]
    \definition{v.}{(estado ou organização) fornecer conforto e assistência material às famílias de pessoal ferido ou incapacitado no cumprimento do dever, ou daqueles que morreram de doença ou morreram no cumprimento do dever}
  \end{Phonetics}
\end{Entry}

\begin{Entry}{抚摸}{7,13}{⼿,⼿}
  \begin{Phonetics}{抚摸}{fu3mo1}[][HSK 7-9]
    \definition{v.}{acariciar; afagar; amimar}
  \end{Phonetics}
\end{Entry}

%%%%%%%%%% 抛 %%%%%%%%%%
\subsection*{抛}\addcontentsline{loh}{figure}{抛}

\begin{Entry}{抛}{7}{⼿}
  \begin{Phonetics}{抛}{pao1}[][HSK 7-9]
    \definition{v.}{atirar; lançar; arremessar | deixar para trás; descartar; abandonar | vender por um preço inferior ao real; vender em excesso; vender em grande quantidade | mostrar; expor}
  \end{Phonetics}
\end{Entry}

\begin{Entry}{抛开}{7,4}{⼿,⼶}
  \begin{Phonetics}{抛开}{pao1kai1}[][HSK 7-9]
    \definition{v.}{livrar-se de; revogar; afastar-se | jogar fora}
  \end{Phonetics}
\end{Entry}

\begin{Entry}{抛弃}{7,7}{⼿,⼶}
  \begin{Phonetics}{抛弃}{pao1qi4}[][HSK 7-9]
    \definition{v.}{abandonar; deixar de lado; renunciar; jogar fora}
  \end{Phonetics}
\end{Entry}

%%%%%%%%%% 抠 %%%%%%%%%%
\subsection*{抠}\addcontentsline{loh}{figure}{抠}

\begin{Entry}{抠}{7}{⼿}
  \begin{Phonetics}{抠}{kou1}[][HSK 7-9]
    \definition{adj.}{Dialeto: mesquinho; avarento}
    \definition{v.}{escolher; cavar ou desenterrar com o dedo ou algo pontiagudo; arranhar | esculpir; cortar | aprofundar-se em; estudar meticulosamente; ir desnecessariamente ao âmago de}
  \end{Phonetics}
\end{Entry}

%%%%%%%%%% 抡 %%%%%%%%%%
\subsection*{抡}\addcontentsline{loh}{figure}{抡}

\begin{Entry}{抡}{7}{⼿}
  \begin{Phonetics}{抡}{lun1}[][HSK 7-9]
    \definition{v.}{brandir (uma espada, os punhos); balançar (os braços, um objeto pesado) | lançar; espalhar | Figurativo: gastar livremente; atirar (dinheiro) | Dialeto: repreender}
  \end{Phonetics}
  \begin{Phonetics}{抡}{lun2}
    \definition{v.}{Literário: selecionar; escolher}
  \end{Phonetics}
\end{Entry}

%%%%%%%%%% 抢 %%%%%%%%%%
\subsection*{抢}\addcontentsline{loh}{figure}{抢}

\begin{Entry}{抢}{7}{⼿}
  \begin{Phonetics}{抢}{qiang1}
    \definition{prep.}{contra; direção relativa inversa}
    \definition{v.}{bater; tocar}
  \end{Phonetics}
  \begin{Phonetics}{抢}{qiang3}[][HSK 5]
    \definition{v.}{roubar; saquear | agarrar; apanhar; arrebatar | disputar; lutar por; ser o primeiro; competir para ser o primeiro | correr; apressar-se; fazer uma incursão | raspar; arranhar; raspar ou esfregar uma camada da superfície de um objeto}
  \end{Phonetics}
\end{Entry}

\begin{Entry}{抢夺}{7,6}{⼿,⼤}
  \begin{Phonetics}{抢夺}{qiang3duo2}[][HSK 7-9]
    \definition{v.}{agarrar; arrebatar; tomar}
  \end{Phonetics}
\end{Entry}

\begin{Entry}{抢劫}{7,7}{⼿,⼒}
  \begin{Phonetics}{抢劫}{qiang3jie2}[][HSK 7-9]
    \definition{v.}{roubar; saquear; pilhar; usar violência ilegalmente para se apropriar da propriedade alheia}
  \end{Phonetics}
\end{Entry}

\begin{Entry}{抢掠}{7,11}{⼿,⼿}
  \begin{Phonetics}{抢掠}{qiang3lve4}
    \definition{s.}{saque | pilhagem}
    \definition{v.}{saquear | pilhar}
  \end{Phonetics}
\end{Entry}

\begin{Entry}{抢救}{7,11}{⼿,⽁}
  \begin{Phonetics}{抢救}{qiang3jiu4}[][HSK 5]
    \definition{v.}{salvar; resgatar; prestar de socorro ou assistência rápidos em situações de emergência | salvar; tomar medidas rápidas para evitar ou minimizar perdas iminentes.}
  \end{Phonetics}
\end{Entry}

\begin{Entry}{抢眼}{7,11}{⼿,⽬}
  \begin{Phonetics}{抢眼}{qiang3yan3}[][HSK 7-9]
    \definition{adj.}{chamativo; significa ser muito chamativo e atrair a atenção do público}
  \end{Phonetics}
\end{Entry}

%%%%%%%%%% 护 %%%%%%%%%%
\subsection*{护}\addcontentsline{loh}{figure}{护}

\begin{Entry}{护}{7}{⼿}
  \begin{Phonetics}{护}{hu4}[][HSK 6]
    \definition{v.}{proteger; defender | blindar; ser parcial; proteger-se da censura}
  \end{Phonetics}
\end{Entry}

\begin{Entry}{护士}{7,3}{⼿,⼠}
  \begin{Phonetics}{护士}{hu4shi5}[][HSK 4]
    \definition[名,位]{s.}{enfermeiro; pessoas especializadas em enfermagem em hospitais ou instituições epidemiológicas}
  \end{Phonetics}
\end{Entry}

\begin{Entry}{护理}{7,11}{⼿,⽟}
  \begin{Phonetics}{护理}{hu4li3}[][HSK 7-9]
    \definition{v.}{cuidar; cooperar com médicos para tratar e cuidar de pacientes, idosos e deficientes | cuidar e proteger; proteger e gerenciar para permitir a vida normal ou o crescimento}
  \end{Phonetics}
\end{Entry}

\begin{Entry}{护照}{7,13}{⼿,⽕}
  \begin{Phonetics}{护照}{hu4zhao4}[][HSK 2]
    \definition[本,个]{s.}{passaporte; documento emitido pela autoridade competente do país para comprovar a nacionalidade e a identidade dos cidadãos que viajam para o exterior}
  \end{Phonetics}
\end{Entry}

%%%%%%%%%% 报 %%%%%%%%%%
\subsection*{报}\addcontentsline{loh}{figure}{报}

\begin{Entry}{报}{7}{⼿}
  \begin{Phonetics}{报}{bao4}[][HSK 3,7-9]
    \definition[份,张]{s.}{jornal | revista; periódico; referência a uma publicação específica | relatório; boletim; algo que transmite alguma informação | telegrama | julgamento; retribuição}
    \definition{v.}{relatar; declarar; anunciar; informar; comunicar | responder; retribuir; revidar | retribuir; recompensar | vingar-se; retaliar | relatar; condenar de acordo com a lei e reportar às autoridades superiores | enviar; submeter; especificamente, relatar ao superior}
  \end{Phonetics}
\end{Entry}

\begin{Entry}{报仇}{7,4}{⼿,⼈}
  \begin{Phonetics}{报仇}{bao4/chou2}[][HSK 7-9]
    \definition{v.+compl.}{vingar; vingar-se}
  \end{Phonetics}
\end{Entry}

\begin{Entry}{报刊}{7,5}{⼿,⼑}
  \begin{Phonetics}{报刊}{bao4 kan1}[][HSK 6]
    \definition{s.}{a imprensa; jornais e periódicos}
  \end{Phonetics}
\end{Entry}

\begin{Entry}{报名}{7,6}{⼿,⼝}
  \begin{Phonetics}{报名}{bao4/ming2}[][HSK 2]
    \definition{v.+compl.}{inscrever-se; alistar-se; registrar seu nome; cadastrar-se; matricular-se; informar seu nome à pessoa responsável, órgão, grupo etc., indicando que você deseja participar de alguma atividade ou organização}
  \end{Phonetics}
\end{Entry}

\begin{Entry}{报考}{7,6}{⼿,⽼}
  \begin{Phonetics}{报考}{bao4 kao3}[][HSK 6]
    \definition{v.}{inscrever-se para um exame}
  \end{Phonetics}
\end{Entry}

\begin{Entry}{报告}{7,7}{⼿,⼝}
  \begin{Phonetics}{报告}{bao4gao4}[][HSK 3]
    \definition[份,篇]{s.}{relatório; discurso; palestra; consultivo; declaração formal feita a superiores ou ao público}
    \definition{v.}{relatar; divulgar; informar; informar formalmente sobre um assunto ou opinião aos superiores ou ao público em geral}
  \end{Phonetics}
\end{Entry}

\begin{Entry}{报社}{7,7}{⼿,⽰}
  \begin{Phonetics}{报社}{bao4she4}[][HSK 7-9]
    \definition[家,个]{s.}{escritório de jornal; escritório geral de um jornal; uma organização que edita e publica jornais}
  \end{Phonetics}
\end{Entry}

\begin{Entry}{报纸}{7,7}{⼿,⽷}
  \begin{Phonetics}{报纸}{bao4zhi3}[][HSK 2]
    \definition[分,期,张]{s.}{jornal; publicações periódicas cujo conteúdo principal é notícias, geralmente referem-se a jornais diários | papel jornal; um tipo de papel usado para imprimir jornais ou publicações em geral}
  \end{Phonetics}
\end{Entry}

\begin{Entry}{报到}{7,8}{⼿,⼑}
  \begin{Phonetics}{报到}{bao4/dao4}[][HSK 3]
    \definition{v.+compl.}{apresentar-se ao serviço; fazer o check-in; registrar-se; assinar o livro de presença; informar à organização que você já chegou}
  \end{Phonetics}
\end{Entry}

\begin{Entry}{报废}{7,8}{⼿,⼴}
  \begin{Phonetics}{报废}{bao4/fei4}[][HSK 7-9]
    \definition{v.+compl.}{sucatear; rejeitar; descartar como inútil}
  \end{Phonetics}
\end{Entry}

\begin{Entry}{报复}{7,9}{⼿,⼢}
  \begin{Phonetics}{报复}{bao4fu4}[][HSK 7-9]
    \definition{v.}{retaliar; fazer represálias; agir de forma muito cruel com alguém que criticou ou prejudicou seus interesses}
  \end{Phonetics}
\end{Entry}

\begin{Entry}{报答}{7,12}{⼿,⽵}
  \begin{Phonetics}{报答}{bao4da2}[][HSK 5]
    \definition{v.}{reembolsar; devolver; retribuir; pagar de volta; mostrar seu apreço de forma tangível}
  \end{Phonetics}
\end{Entry}

\begin{Entry}{报道}{7,12}{⼿,⾡}
  \begin{Phonetics}{报道}{bao4dao4}[][HSK 3]
    \definition[个,篇,分]{s.}{história; reportagem; comunicado de imprensa publicado por escrito ou transmitido pela rádio}
    \definition{v.}{cobrir; reportar (notícias); divulgar notícias ao público através de jornais, rádio, etc.}
  \end{Phonetics}
\end{Entry}

\begin{Entry}{报销}{7,12}{⼿,⾦}
  \begin{Phonetics}{报销}{bao4xiao1}[][HSK 7-9]
    \definition{v.}{reembolsar; enviar uma fatura; elaborar a relação dos valores recebidos ou das contas de receitas e despesas e comunicar ao superior para verificação | anular; acabar; consumir; matar alguém; fazer algo perder sua utilidade; comer toda a comida}
  \end{Phonetics}
\end{Entry}

\begin{Entry}{报酬}{7,13}{⼿,⾣}
  \begin{Phonetics}{报酬}{bao4chou5}[][HSK 7-9]
    \definition[笔,个]{s.}{pagamento; recompensa; remuneração; dinheiro ou bens pagos a outros pelo uso de seu trabalho, objetos, etc.}
  \end{Phonetics}
\end{Entry}

\begin{Entry}{报警}{7,19}{⼿,⾔}
  \begin{Phonetics}{报警}{bao4jing3}[][HSK 5]
    \definition{v.}{relatar (um incidente) à polícia; relatar uma situação crítica ou sinalizar uma emergência às autoridades competentes}
  \end{Phonetics}
\end{Entry}

%%%%%%%%%% 拒 %%%%%%%%%%
\subsection*{拒}\addcontentsline{loh}{figure}{拒}

\begin{Entry}{拒}{7}{⼿}
  \begin{Phonetics}{拒}{ju4}
    \definition{v.}{resistir; repelir | recusar; rejeitar}
  \end{Phonetics}
\end{Entry}

\begin{Entry}{拒绝}{7,9}{⼿,⽷}
  \begin{Phonetics}{拒绝}{ju4jue2}[][HSK 5]
    \definition{v.}{recusar; rejeitar; declinar; não aceitar (pedidos, sugestões ou presentes)}
  \end{Phonetics}
\end{Entry}

%%%%%%%%%% 拟 %%%%%%%%%%
\subsection*{拟}\addcontentsline{loh}{figure}{拟}

\begin{Entry}{拟}{7}{⼿}
  \begin{Phonetics}{拟}{ni3}[][HSK 7-9]
    \definition{v.}{elaborar; projetar; conceber; rascunhar | Literário: pretender; planejar | imitar; reproduzir; simular | comparar; traçar um paralelo; com a intenção de ser ilícito | conjecturar; imaginar}
  \end{Phonetics}
\end{Entry}

\begin{Entry}{拟定}{7,8}{⼿,⼧}
  \begin{Phonetics}{拟定}{ni3ding4}[][HSK 7-9]
    \definition{v.}{elaborar; redigir; planejar; formular | adivinhar; conjecturar; especular}
  \end{Phonetics}
\end{Entry}

%%%%%%%%%% 承 %%%%%%%%%%
\subsection*{承}\addcontentsline{loh}{figure}{承}

\begin{Entry}{承}{8}{⼿}
  \begin{Phonetics}{承}{cheng2}
    \definition*{s.}{Sobrenome: Cheng}
    \definition{v.}{suportar; segurar; carregar; sustentar | empreender; contratar (para fazer um trabalho) | estar em dívida (com alguém por uma gentileza); receber um favor | continuar; prosseguir | receber de cima (instruções, mandato)}
  \end{Phonetics}
\end{Entry}

\begin{Entry}{承办}{8,4}{⼿,⼒}
  \begin{Phonetics}{承办}{cheng2ban4}[][HSK 5]
    \definition{v.}{ocupar-se de; encarregar-se de; (pessoas, organizações, instituições) aceitar (atividades, reuniões, negócios, etc.)}
  \end{Phonetics}
\end{Entry}

\begin{Entry}{承认}{8,4}{⼿,⾔}
  \begin{Phonetics}{承认}{cheng2ren4}[][HSK 4]
    \definition{s.}{reconhecimento (diplomático, artístico, etc.)}
    \definition{v.}{admitir; reconhecer | dar reconhecimento diplomático; reconhecer}
  \end{Phonetics}
\end{Entry}

\begin{Entry}{承包}{8,5}{⼿,⼓}
  \begin{Phonetics}{承包}{cheng2bao1}[][HSK 7-9]
    \definition{v.}{contratar (com; para); aceitar projetos ou pedidos em massa, etc. e ser responsável por concluí-los}[他承包了这个工程。===Ele foi contratado para esse projeto.]
  \end{Phonetics}
\end{Entry}

\begin{Entry}{承受}{8,8}{⼿,⼜}
  \begin{Phonetics}{承受}{cheng2shou4}[][HSK 4]
    \definition{v.}{suportar; resistir; realizar (tarefas, dificuldades, pressões, etc.); submeter-se a (testes, etc.) | herdar}
  \end{Phonetics}
\end{Entry}

\begin{Entry}{承担}{8,8}{⼿,⼿}
  \begin{Phonetics}{承担}{cheng2dan1}[][HSK 4]
    \definition{v.}{suportar; empreender; assumir; tomar conta de algo}
  \end{Phonetics}
\end{Entry}

\begin{Entry}{承诺}{8,10}{⼿,⾔}
  \begin{Phonetics}{承诺}{cheng2nuo4}[][HSK 6]
    \definition[个,句,份]{s.}{juramento; promessa; compromisso}
    \definition{v.}{prometer fazer algo; prometer empreender; comprometer-se a fazer algo}
  \end{Phonetics}
\end{Entry}

\begin{Entry}{承载}{8,10}{⼿,⾞}
  \begin{Phonetics}{承载}{cheng2zai4}[][HSK 7-9]
    \definition{v.}{suportar o peso; segurar o objeto e suportar seu peso}[桥梁承载着巨大的重量。===A ponte suporta uma carga pesada.]
  \end{Phonetics}
\end{Entry}

%%%%%%%%%% 抨 %%%%%%%%%%
\subsection*{抨}\addcontentsline{loh}{figure}{抨}

\begin{Entry}{抨}{8}{⼿}
  \begin{Phonetics}{抨}{peng1}
    \definition{s.}{Literário: \emph{impeachment}; censura}
    \definition{v.}{atacar; criticar; destituir; censurar}
  \end{Phonetics}
\end{Entry}

\begin{Entry}{抨击}{8,5}{⼿,⼐}
  \begin{Phonetics}{抨击}{peng1ji1}[][HSK 7-9]
    \definition{v.}{atacar (falando ou escrevendo); bombardear (com palavras); criticar}[他把电视采访作为一个机会,向反对党进行猛烈抨击。===Ele aproveitou a entrevista na televisão para lançar um ataque feroz contra o partido da oposição.]
  \end{Phonetics}
\end{Entry}

%%%%%%%%%% 披 %%%%%%%%%%
\subsection*{披}\addcontentsline{loh}{figure}{披}

\begin{Entry}{披}{8}{⼿}
  \begin{Phonetics}{披}{pi1}[][HSK 5]
    \definition{v.}{colocar sobre os ombros; enrolar em volta; cobrir ou colocar sobre os ombros | abrir; desenrolar; espalhar | abrir-se; rachar}
  \end{Phonetics}
\end{Entry}

\begin{Entry}{披露}{8,21}{⼿,⾬}
  \begin{Phonetics}{披露}{pi1lu4}[][HSK 7-9]
    \definition{v.}{publicar; anunciar; tornar público | revelar; mostrar; divulgar}
  \end{Phonetics}
\end{Entry}

%%%%%%%%%% 抬 %%%%%%%%%%
\subsection*{抬}\addcontentsline{loh}{figure}{抬}

\begin{Entry}{抬}{8}{⼿}
  \begin{Phonetics}{抬}{tai2}[][HSK 5]
    \definition{clas.}{usado para objetos que precisam ser carregados por pessoas quando transportados (por exemplo, móveis)}
    \definition{v.}{levantar; elevar; puxar para cima | (por duas ou mais pessoas) carregar; transportar; duas ou mais pessoas carregando algo com as mãos ou nos ombros | discutir, debater (geralmente sem sentido ou sem importância)}
  \end{Phonetics}
\end{Entry}

\begin{Entry}{抬头}{8,5}{⼿,⼤}
  \begin{Phonetics}{抬头}{tai2 tou2}[][HSK 5]
    \definition{s.}{(em recibos, contas, etc.) nome do comprador ou beneficiário, o local no documento onde o nome do beneficiário ou destinatário é escrito}
    \definition{v.}{levantar a cabeça}
  \end{Phonetics}
\end{Entry}

\begin{Entry}{抬杠}{8,7}{⼿,⽊}
  \begin{Phonetics}{抬杠}{tai2/gang4}
    \definition{v.+compl.}{brigar; discutir; discutir por discutir; discutir sobre o certo e o errado (geralmente sem princípios) | Arcaico: carregar um caixão em barras resistentes}
  \end{Phonetics}
\end{Entry}

%%%%%%%%%% 抱 %%%%%%%%%%
\subsection*{抱}\addcontentsline{loh}{figure}{抱}

\begin{Entry}{抱}{8}{⼿}
  \begin{Phonetics}{抱}{bao4}[][HSK 4]
    \definition*{s.}{Sobrenome: Bao}
    \definition{clas.}{braçada; medida dos dois braços}
    \definition{v.}{carregar no peito; segurar com ambos os braços; abraçar | ter o primeiro filho ou neto | adotar um bebê ou criança | ficar juntos, unidos | encaixar ou servir perfeitamente (roupas e sapatos do tamanho certo) | estimar; nutrir; abrigar; ter em mente | continuar; sobrecarregar com | chocar ovos}
  \end{Phonetics}
\end{Entry}

\begin{Entry}{抱负}{8,6}{⼿,⾙}
  \begin{Phonetics}{抱负}{bao4fu4}[][HSK 7-9]
    \definition{s.}{aspiração; ambição; objetivo elevado; grandes intenções e determinação, frequentemente usados na linguagem escrita}
  \end{Phonetics}
\end{Entry}

\begin{Entry}{抱怨}{8,9}{⼿,⼼}
  \begin{Phonetics}{抱怨}{bao4yuan4}[][HSK 5]
    \definition{v.}{reclamar ou expressar descontentamento ou insatisfação; falar com os outros sobre pessoas ou coisas com as quais você não está satisfeito}
  \end{Phonetics}
\end{Entry}

\begin{Entry}{抱歉}{8,14}{⼿,⽋}
  \begin{Phonetics}{抱歉}{bao4qian4}[][HSK 6]
    \definition{adj.}{pesaroso; arrependido; sentir pena de alguém porque você causou perda, inconveniência ou não atendeu às suas necessidades}
  \end{Phonetics}
\end{Entry}

%%%%%%%%%% 抵 %%%%%%%%%%
\subsection*{抵}\addcontentsline{loh}{figure}{抵}

\begin{Entry}{抵}{8}{⼿}
  \begin{Phonetics}{抵}{di3}
    \definition{v.}{apoiar; sustentar | resistir; suportar | compensar; fazer o bem | hipotecar; dar como garantia; garantir | equilibrar; cancelar; compensar | ser igual a; corresponder | alcançar; chegar a | colidir; dar cabeçada (por animais com chifres)}
  \end{Phonetics}
\end{Entry}

\begin{Entry}{抵达}{8,6}{⼿,⾡}
  \begin{Phonetics}{抵达}{di3da2}[][HSK 6]
    \definition{v.}{chegar; alcançar}
  \end{Phonetics}
\end{Entry}

\begin{Entry}{抵抗}{8,7}{⼿,⼿}
  \begin{Phonetics}{抵抗}{di3kang4}[][HSK 6]
    \definition{s.}{resistência}
    \definition{v.}{resistir; usar ação para resistir ou parar o ataque da outra parte}
  \end{Phonetics}
\end{Entry}

\begin{Entry}{抵制}{8,8}{⼿,⼑}
  \begin{Phonetics}{抵制}{di3zhi4}[][HSK 7-9]
    \definition{v.}{resistir; boicotar; bloquear, prevenir e impedir que forças externas invadam ou causem danos}
  \end{Phonetics}
\end{Entry}

\begin{Entry}{抵押}{8,8}{⼿,⼿}
  \begin{Phonetics}{抵押}{di3ya1}[][HSK 7-9]
    \definition{s.}{hipoteca; segurança; garantia}
    \definition{v.}{hipotecar; manter em penhor; penhorar}
  \end{Phonetics}
\end{Entry}

\begin{Entry}{抵挡}{8,9}{⼿,⼿}
  \begin{Phonetics}{抵挡}{di3dang3}[][HSK 7-9]
    \definition{v.}{resistir; suportar; bloquear}
  \end{Phonetics}
\end{Entry}

\begin{Entry}{抵御}{8,12}{⼿,⼻}
  \begin{Phonetics}{抵御}{di3yu4}[][HSK 7-9]
    \definition{v.}{resistir; suportar; afastar}[我们要抵御外敌的侵略。===Devemos resistir à invasão estrangeira.]
  \end{Phonetics}
\end{Entry}

\begin{Entry}{抵销}{8,12}{⼿,⾦}
  \begin{Phonetics}{抵销}{di3xiao1}[][HSK 7-9]
    \definition{v.}{compensar}[三笔债务可以抵销。===As três dívidas podem ser compensadas.]
  \end{Phonetics}
\end{Entry}

\begin{Entry}{抵触}{8,13}{⼿,⾓}
  \begin{Phonetics}{抵触}{di3chu4}[][HSK 7-9]
    \definition{adj.}{conflitante; contraditório}
    \definition{s.}{conflito}
    \definition{v.}{entrar em conflito; contradizer}
  \end{Phonetics}
\end{Entry}

%%%%%%%%%% 抹 %%%%%%%%%%
\subsection*{抹}\addcontentsline{loh}{figure}{抹}

\begin{Entry}{抹}{8}{⼿}
  \begin{Phonetics}{抹}{ma1}
    \definition{v.}{esfregar; limpar | deslizar algo para fora; tirar}
  \end{Phonetics}
  \begin{Phonetics}{抹}{mo3}[][HSK 7-9]
    \definition{v.}{colocar; aplicar; untar; engessar | limpar | anular; apagar | (para nuvem, etc.) irradiar; raiar; riscar; traçar | riscar; cancelar; marcar; remover; excluir}
  \end{Phonetics}
  \begin{Phonetics}{抹}{mo4}
    \definition{v.}{rebocar; engessar; alisar a massa ou o gesso com uma espátula | virar; contornar; dar uma volta de perto}
  \end{Phonetics}
\end{Entry}

\begin{Entry}{抹泪}{8,8}{⼿,⽔}
  \begin{Phonetics}{抹泪}{mo3lei4}
    \definition{v.}{limpar as lágrimas | Figurativo: derramar lágrimas}
  \end{Phonetics}
\end{Entry}

%%%%%%%%%% 押 %%%%%%%%%%
\subsection*{押}\addcontentsline{loh}{figure}{押}

\begin{Entry}{押}{8}{⼿}
  \begin{Phonetics}{押}{ya1}
    \definition*{s.}{Sobrenome: Ya}
    \definition{s.}{assinatura; marca em vez de assinatura; nome assinado ou símbolo desenhado}
    \definition{v.}{dar como garantia; hipotecar; penhorar | deter; levar sob custódia | escoltar | assinar (um documento, contrato, etc.); colocar sua assinatura (ou marcar no lugar da assinatura)}
  \end{Phonetics}
\end{Entry}

\begin{Entry}{押后}{8,6}{⼿,⼝}
  \begin{Phonetics}{押后}{ya1hou4}
    \definition{v.}{encerrar | adiar}
  \end{Phonetics}
\end{Entry}

\begin{Entry}{押运}{8,7}{⼿,⾡}
  \begin{Phonetics}{押运}{ya1yun4}
    \definition{v.}{escoltar sob guarda | escoltar (bens ou fundos)}
  \end{Phonetics}
\end{Entry}

\begin{Entry}{押注}{8,8}{⼿,⽔}
  \begin{Phonetics}{押注}{ya1zhu4}
    \definition{v.}{apostar}
  \end{Phonetics}
\end{Entry}

\begin{Entry}{押金}{8,8}{⼿,⾦}
  \begin{Phonetics}{押金}{ya1jin1}[][HSK 5]
    \definition[笔,份,些]{s.}{caução; sinal; depósito; dinheiro como garantia}
  \end{Phonetics}
\end{Entry}

\begin{Entry}{押送}{8,9}{⼿,⾡}
  \begin{Phonetics}{押送}{ya1song4}
    \definition{v.}{enviar sob escolta | transportar um detido}
  \end{Phonetics}
\end{Entry}

\begin{Entry}{押租}{8,10}{⼿,⽲}
  \begin{Phonetics}{押租}{ya1zu1}
    \definition{s.}{depósito de aluguel}
  \end{Phonetics}
\end{Entry}

\begin{Entry}{押韵}{8,13}{⼿,⾳}
  \begin{Phonetics}{押韵}{ya1yun4}
    \definition{v.}{rimar}
  \end{Phonetics}
\end{Entry}

%%%%%%%%%% 抽 %%%%%%%%%%
\subsection*{抽}\addcontentsline{loh}{figure}{抽}

\begin{Entry}{抽}{8}{⼿}
  \begin{Phonetics}{抽}{chou1}[][HSK 4]
    \definition{v.}{retirar; tirar (do meio); retirar, puxar ou arrancar algo que está preso ou emaranhado em outra coisa | tirar, retirar (uma parte de um todo) | (certas plantas) começar a crescer, produzir | bombear | encolher; contrair | chicotear; açoitar; surrar | dirigir; conduzir | encontrar tempo; libertar-se; sair de alguma coisa}
  \end{Phonetics}
\end{Entry}

\begin{Entry}{抽屉}{8,8}{⼿,⼫}
  \begin{Phonetics}{抽屉}{chou1ti4}[][HSK 7-9]
    \definition[个,层,组]{s.}{gaveta}
  \end{Phonetics}
\end{Entry}

\begin{Entry}{抽奖}{8,9}{⼿,⼤}
  \begin{Phonetics}{抽奖}{chou1 jiang3}[][HSK 4]
    \definition{s.}{loteria; sorteio de loteria}
  \end{Phonetics}
\end{Entry}

\begin{Entry}{抽烟}{8,10}{⼿,⽕}
  \begin{Phonetics}{抽烟}{chou1/yan1}[][HSK 4]
    \definition{v.+compl.}{fumar (um cigarro ou um cachimbo)}
  \end{Phonetics}
\end{Entry}

\begin{Entry}{抽象}{8,11}{⼿,⾗}
  \begin{Phonetics}{抽象}{chou1xiang4}[][HSK 7-9]
    \definition{adj.}{abstrato}[抽象的艺术需要想象力。===A arte abstrata requer imaginação.]
    \definition{v.}{abstrair}[这个理论很难抽象。===Essa teoria é difícil de abstrair.]
  \end{Phonetics}
\end{Entry}

\begin{Entry}{抽签}{8,13}{⼿,⽵}
  \begin{Phonetics}{抽签}{chou1/qian1}[][HSK 7-9]
    \definition{v.+compl.}{tirar/lançar sorte; realizar/fazer um sorteio}[他们抽签决定胜者。===Eles fizeram um sorteio para decidir o vencedor.]
  \end{Phonetics}
\end{Entry}

%%%%%%%%%% 担 %%%%%%%%%%
\subsection*{担}\addcontentsline{loh}{figure}{担}

\begin{Entry}{担}{8}{⼿}
  \begin{Phonetics}{担}{dan1}[][HSK 7-9]
    \definition{v.}{carregar em uma vara de ombro e baldes; carregar nos ombros | assumir; empreender; não ter medo de correr riscos}
  \end{Phonetics}
  \begin{Phonetics}{担}{dan4}[][HSK 7-9]
    \definition{clas.}{dan, uma unidade de peso (=50 quilogramas) ; 100 jin = 1 dan | usado em coisas usadas para transportar cargas}
    \definition{s.}{carga; fardo; cargas de mercadorias transportadas em uma vara de ombro por um mascate itinerante}
  \end{Phonetics}
\end{Entry}

\begin{Entry}{担子}{8,3}{⼿,⼦}
  \begin{Phonetics}{担子}{dan4zi5}[][HSK 7-9]
    \definition[副,个]{s.}{vara de transporte (ou de ombro) e as cargas sob ela; canga; carga; fardo | tarefa}
  \end{Phonetics}
\end{Entry}

\begin{Entry}{担心}{8,4}{⼿,⼼}
  \begin{Phonetics}{担心}{dan1xin1}[][HSK 4]
    \definition{v.}{preocupar-se; ficar ansioso; sentir-se desconfortável com algo}
  \end{Phonetics}
\end{Entry}

\begin{Entry}{担任}{8,6}{⼿,⼈}
  \begin{Phonetics}{担任}{dan1ren4}[][HSK 4]
    \definition{v.}{servir como; assumir o cargo de; ocupar o posto de; ocupar um determinado cargo ou emprego}
  \end{Phonetics}
\end{Entry}

\begin{Entry}{担当}{8,6}{⼿,⼹}
  \begin{Phonetics}{担当}{dan1dang1}[][HSK 7-9]
    \definition{v.}{aceitar e assumir responsabilidade; empreender (responsabilidade, trabalho, despesas)}
  \end{Phonetics}
\end{Entry}

\begin{Entry}{担负}{8,6}{⼿,⾙}
  \begin{Phonetics}{担负}{dan1fu4}[][HSK 7-9]
    \definition{v.}{suportar; carregar; assumir; ser encarregado de}
  \end{Phonetics}
\end{Entry}

\begin{Entry}{担忧}{8,7}{⼿,⼼}
  \begin{Phonetics}{担忧}{dan1 you1}[][HSK 6]
    \definition[项,条,套,种]{v.}{preocupar-se; estar ansioso}
  \end{Phonetics}
\end{Entry}

\begin{Entry}{担保}{8,9}{⼿,⼈}
  \begin{Phonetics}{担保}{dan1bao3}[][HSK 4]
    \definition{v.}{garantir; atestar; expressar responsabilidade e garantir que não haverá problemas ou que eles serão resolvidos}
  \end{Phonetics}
\end{Entry}

%%%%%%%%%% 拆 %%%%%%%%%%
\subsection*{拆}\addcontentsline{loh}{figure}{拆}

\begin{Entry}{拆}{8}{⼿}
  \begin{Phonetics}{拆}{chai1}[][HSK 5]
    \definition{v.}{rasgar; desmontar; separar o que está unido | derrubar; desmantelar; demolir; refere-se especificamente à demolição de edifícios}
  \end{Phonetics}
\end{Entry}

\begin{Entry}{拆迁}{8,6}{⼿,⾡}
  \begin{Phonetics}{拆迁}{chai1 qian1}[][HSK 6]
    \definition{v.}{demolir uma casa velha e realocar seus ocupantes em outro lugar; devido às necessidades de construção, unidades ou casas residenciais são demolidas e realocadas em outros lugares}
  \end{Phonetics}
\end{Entry}

\begin{Entry}{拆除}{8,9}{⼿,⾩}
  \begin{Phonetics}{拆除}{chai1 chu2}[][HSK 5]
    \definition{v.}{desmantelar; demolir; derrubar; remover (um edifício, etc.)}
  \end{Phonetics}
\end{Entry}

%%%%%%%%%% 拉 %%%%%%%%%%
\subsection*{拉}\addcontentsline{loh}{figure}{拉}

\begin{Entry}{拉}{8}{⼿}
  \begin{Phonetics}{拉}{la1}[][HSK 2]
    \definition{s.}{abreviação de América Latina, 拉丁美洲}
    \definition{v.}{puxar; arrastar; rebocar | transportar por veículo; rebocar | arrastar (ou puxar) para fora | mover (tropas para um lugar) | dar uma mãozinha; ajudar | arrastar para dentro; implicar; envolver | criar (criança) | atrair; conquistar; solicitar; angariar votos | bater-papo | organizar; preparar | ter dívidas; estar endividado | pressionar; recrutar à força | (no tênis, tênis de mesa, etc.) levantar (a bola) | tocar (certos instrumentos musicais); puxar uma parte do instrumento para que ele emita som | prolongar; espaçar | envolver-se em | (coloquial) esvaziar os intestinos | levantar, uma das técnicas do tênis de mesa | destruir; esmagar; quebrar}
  \seealsoref{拉丁美洲}{la1ding1 mei3zhou1}
  \end{Phonetics}
  \begin{Phonetics}{拉}{la4}
    \definition{s.}{usado em 拉拉蛄}
  \seealsoref{拉拉蛄}{la4la4gu3}
  \end{Phonetics}
\end{Entry}

\begin{Entry}{拉丁美洲}{8,2,9,9}{⼿,⼀,⽺,⽔}
  \begin{Phonetics}{拉丁美洲}{la1ding1 mei3zhou1}
    \definition*{s.}{América Latina, nome coletivo dos países da América Central e do Sul, devido ao fato de a maioria de seus habitantes ser descendente de povos latinos e de a língua falada ser do grupo latino}
  \end{Phonetics}
\end{Entry}

\begin{Entry}{拉开}{8,4}{⼿,⼶}
  \begin{Phonetics}{拉开}{la1 kai1}[][HSK 4]
    \definition{v.}{puxar para abrir; recuar | ampliar; espaçar; distanciar; afastar; separar}
  \end{Phonetics}
\end{Entry}

\begin{Entry}{拉布布}{8,5,5}{⼿,⼱,⼱}
  \begin{Phonetics}{拉布布}{la1bu4bu4}
    \definition*{s.}{Labubu}
  \end{Phonetics}
\end{Entry}

\begin{Entry}{拉动}{8,6}{⼿,⼒}
  \begin{Phonetics}{拉动}{la1dong4}[][HSK 7-9]
    \definition{v.}{estimular; impulsionar; promover; estimular ou promover o desenvolvimento de um setor específico (geralmente relacionado à economia)}
  \end{Phonetics}
\end{Entry}

\begin{Entry}{拉拉队}{8,8,4}{⼿,⼿,⾩}
  \begin{Phonetics}{拉拉队}{la1la1dui4}
    \definition{s.}{torcida organizada; torcedores | também escrito como 啦啦队 | equipe de líderes de torcida}
  \seealsoref{啦啦队}{la1la1dui4}
  \end{Phonetics}
\end{Entry}

\begin{Entry}{拉拉蛄}{8,8,11}{⼿,⼿,⾍}
  \begin{Phonetics}{拉拉蛄}{la4la4gu3}
    \variantof{蝲蝲蛄}
  \end{Phonetics}
\end{Entry}

\begin{Entry}{拉拢}{8,8}{⼿,⼿}
  \begin{Phonetics}{拉拢}{la1long3}[][HSK 7-9]
    \definition{v.}{atrair alguém para o seu lado; conquistar; envolver (oposto de 排挤) | aconchegar-se; envolver-se; utilizar-se de métodos para atrair outros para o próprio lado para benefício próprio}
  \seealsoref{排挤}{pai2ji3}
  \end{Phonetics}
\end{Entry}

\begin{Entry}{拉萨}{8,11}{⼿,⾋}
  \begin{Phonetics}{拉萨}{la1sa4}
    \definition*{s.}{Lhasa, capital da Região Autônoma do Tibete, 西藏自治区}
  \seealsoref{西藏自治区}{xi1zang4 zi4zhi4qu1}
  \end{Phonetics}
\end{Entry}

\begin{Entry}{拉锁}{8,12}{⼿,⾦}
  \begin{Phonetics}{拉锁}{la1suo3}[][HSK 7-9]
    \definition{s.}{zíper; um acessório de metal ou plástico em forma de corrente que pode ser separado e travado, usado para ser costurado em roupas, bolsos ou bolsas}
  \end{Phonetics}
\end{Entry}

%%%%%%%%%% 拊 %%%%%%%%%%
\subsection*{拊}\addcontentsline{loh}{figure}{拊}

\begin{Entry}{拊}{8}{⼿}
  \begin{Phonetics}{拊}{fu3}
    \definition{v.}{Literário: bater palmas; esbofetear; golpear}
  \end{Phonetics}
\end{Entry}

%%%%%%%%%% 拌 %%%%%%%%%%
\subsection*{拌}\addcontentsline{loh}{figure}{拌}

\begin{Entry}{拌}{8}{⼿}
  \begin{Phonetics}{拌}{ban4}[][HSK 7-9]
    \definition{v.}{misturar | mexer e misturar | discutir; brigar; ter uma discussão}
  \end{Phonetics}
\end{Entry}

%%%%%%%%%% 拍 %%%%%%%%%%
\subsection*{拍}\addcontentsline{loh}{figure}{拍}

\begin{Entry}{拍}{8}{⼿}
  \begin{Phonetics}{拍}{pai1}[][HSK 3]
    \definition[个,副,对]{s.}{bastão; raquete | batida; tempo; (música) uma unidade para medir a duração de uma nota musical}
    \definition{v.}{tirar (uma foto); usar uma câmera para capturar imagens de pessoas e objetos em filme | dar um tapinha; bater suavemente com as mãos ou ferramentas | bater asas | bater (ondas do mar) | enviar (um telegrama, etc.) | bajular}
  \end{Phonetics}
\end{Entry}

\begin{Entry}{拍马}{8,3}{⼿,⾺}
  \begin{Phonetics}{拍马}{pai1ma3}
    \definition{v.}{instigar um cavalo dando tapinhas em seu traseiro | lisonjear | bajular}
  \seealsoref{拍马屁}{pai1ma3pi4}
  \end{Phonetics}
\end{Entry}

\begin{Entry}{拍马屁}{8,3,7}{⼿,⾺,⼫}
  \begin{Phonetics}{拍马屁}{pai1ma3pi4}
    \definition{s.}{puxa-saco | bajulador}
    \definition{v.}{puxar o saco | bajular}
  \seealsoref{拍马}{pai1ma3}
  \end{Phonetics}
\end{Entry}

\begin{Entry}{拍戏}{8,6}{⼿,⼽}
  \begin{Phonetics}{拍戏}{pai1/xi4}[][HSK 7-9]
    \definition{v.+compl.}{fazer um filme ou peça de televisão; filmar uma cena | filmar}
  \end{Phonetics}
\end{Entry}

\begin{Entry}{拍卖}{8,8}{⼿,⼗}
  \begin{Phonetics}{拍卖}{pai1mai4}[][HSK 7-9]
    \definition{s.}{leilão; uma forma pública de venda de produtos onde todos oferecem publicamente o seu preço, e o item é vendido para quem oferecer o preço mais alto}
    \definition{v.}{leiloar; realizar atividades de leilão | vender mercadorias a preços reduzidos; baixar o preço para vender as mercadorias rapidamente}
  \end{Phonetics}
\end{Entry}

\begin{Entry}{拍板}{8,8}{⼿,⽊}
  \begin{Phonetics}{拍板}{pai1/ban3}[][HSK 7-9]
    \definition{s.}{aplausos}
    \definition{v.+compl.}{marcar o tempo com palmas | bater o martelo | ter a palavra final; dar o veredicto final; tomar a decisão final}
  \end{Phonetics}
\end{Entry}

\begin{Entry}{拍摄}{8,13}{⼿,⼿}
  \begin{Phonetics}{拍摄}{pai1 she4}[][HSK 5]
    \definition{s.}{fotografar; tirar (uma foto); usar uma câmera fotográfica para capturar imagens de pessoas e objetos}
  \end{Phonetics}
\end{Entry}

\begin{Entry}{拍照}{8,13}{⼿,⽕}
  \begin{Phonetics}{拍照}{pai1/zhao4}[][HSK 4]
    \definition{v.+compl.}{fotografar; tirar uma foto}
  \end{Phonetics}
\end{Entry}

%%%%%%%%%% 拎 %%%%%%%%%%
\subsection*{拎}\addcontentsline{loh}{figure}{拎}

\begin{Entry}{拎}{8}{⼿}
  \begin{Phonetics}{拎}{lin1}[][HSK 7-9]
    \definition{v.}{levantar; carregar}
  \end{Phonetics}
\end{Entry}

%%%%%%%%%% 拐 %%%%%%%%%%
\subsection*{拐}\addcontentsline{loh}{figure}{拐}

\begin{Entry}{拐}{8}{⼿}
  \begin{Phonetics}{拐}{guai3}[][HSK 6]
    \definition[支,根,副]{s.}{muleta; bengala; uma bengala com uma barra horizontal na parte superior, usada por pessoas com doenças ou deficiências nos membros inferiores para ajudá-las a caminhar |  sete; forma falada do numeral 七 | esquina; curva; canto}
    \definition{v.}{virar; girar; mudar de direção enquanto se move | enganar | mudar; transformar | mancar}
  \seealsoref{七}{qi1}
  \end{Phonetics}
\end{Entry}

\begin{Entry}{拐杖}{8,7}{⼿,⽊}
  \begin{Phonetics}{拐杖}{guai3zhang4}[][HSK 7-9]
    \definition[个,根,支,副]{s.}{muleta; bengala}
  \end{Phonetics}
\end{Entry}

\begin{Entry}{拐弯}{8,9}{⼿,⼸}
  \begin{Phonetics}{拐弯}{guai3/wan1}[][HSK 7-9]
    \definition[个]{s.}{esquina; curva; canto}
    \definition{v.}{virar; virar uma esquina; indica mudança de direção da viagem | dar meia-volta; seguir um novo curso; indica mudança de ideias, linguagem, etc.}
  \end{Phonetics}
\end{Entry}

%%%%%%%%%% 拔 %%%%%%%%%%
\subsection*{拔}\addcontentsline{loh}{figure}{拔}

\begin{Entry}{拔}{8}{⼿}
  \begin{Phonetics}{拔}{ba2}[][HSK 5]
    \definition{v.aux.}{puxar para cima; puxar para fora; arrastar para fora | extrair; sugar | escolher; selecionar | superar; destacar-se entre | apreender; capturar | esfriar na água; mergulhar algo em água fria para que esfrie}
  \end{Phonetics}
\end{Entry}

\begin{Entry}{拔尖}{8,6}{⼿,⼩}
  \begin{Phonetics}{拔尖}{ba2/jian1}
    \definition{adj.}{topo de linha | fora do comum | o melhor}
    \definition{v.+compl.}{empurrar-se para a frente | sentir que é superior aos outros}
  \end{Phonetics}
\end{Entry}

%%%%%%%%%% 拖 %%%%%%%%%%
\subsection*{拖}\addcontentsline{loh}{figure}{拖}

\begin{Entry}{拖}{8}{⼿}
  \begin{Phonetics}{拖}{tuo1}[][HSK 6]
    \definition{v.}{puxar; arrastar; transportar; puxar um objeto para movê-lo contra o solo ou outra superfície | esfregar; limpar o chão com uma ferramenta especial para esfregar | atrasar; prolongar; procrastinar; arrastar; coisas que deveriam ser feitas nunca são iniciadas ou concluídas; uma certa nota é prolongada por um longo tempo | atrasar; conter; segurar; restringir}
  \end{Phonetics}
\end{Entry}

\begin{Entry}{拖拉机}{8,8,6}{⼿,⼿,⽊}
  \begin{Phonetics}{拖拉机}{tuo1la1ji1}
    \definition[台]{s.}{trator}
  \end{Phonetics}
\end{Entry}

\begin{Entry}{拖鞋}{8,15}{⼿,⾰}
  \begin{Phonetics}{拖鞋}{tuo1 xie2}[][HSK 6]
    \definition[双,只]{s.}{chinelos; samdálias; babouche; sapatos sem cabedal geralmente são usados ​​em ambientes fechados}
  \end{Phonetics}
\end{Entry}

%%%%%%%%%% 拘 %%%%%%%%%%
\subsection*{拘}\addcontentsline{loh}{figure}{拘}

\begin{Entry}{拘}{8}{⼿}
  \begin{Phonetics}{拘}{ju1}
    \definition{adj.}{inflexível; sem flexibilidade}
    \definition{v.}{prender; deter | restringir; limitar; constranger | aderir rigidamente; ser inflexível |limitar}
  \end{Phonetics}
\end{Entry}

\begin{Entry}{拘束}{8,7}{⼿,⽊}
  \begin{Phonetics}{拘束}{ju1shu4}[][HSK 7-9]
    \definition{adj.}{desajeitado; desconfortável; constrangido; reservado; não natural}
    \definition{v.}{restringir; limitar; restringir excessivamente as palavras e ações de outras pessoas}
  \end{Phonetics}
\end{Entry}

\begin{Entry}{拘留}{8,10}{⼿,⽥}
  \begin{Phonetics}{拘留}{ju1liu2}[][HSK 7-9]
    \definition{v.}{deter; manter sob custódia; colocar em prisão provisória; restringir a liberdade pessoal}
  \end{Phonetics}
\end{Entry}

%%%%%%%%%% 招 %%%%%%%%%%
\subsection*{招}\addcontentsline{loh}{figure}{招}

\begin{Entry}{招}{8}{⼿}
  \begin{Phonetics}{招}{zhao1}[][HSK 6]
    \definition*{s.}{Sobrenome: Zhao}
    \definition{s.}{\emph{banner}; Faixas e outros itens costumavam ser pendurados nas entradas de hotéis, restaurantes ou lojas para atrair clientes | movimento; estratagema; artifício; meios ou táticas | movimentos de artes marciais}
    \definition{v.}{acenar; gestuar para alguém ver | alistar; inscrever; recrutar | incorrer; cortejar; atrair; provocar (um certo resultado ou reação) | provocar; tocar ou provocar a outra pessoa com palavras ou ações | confessar (culpa); assumir (culpa) | infectar; ser contagioso}
  \end{Phonetics}
\end{Entry}

\begin{Entry}{招手}{8,4}{⼿,⼿}
  \begin{Phonetics}{招手}{zhao1/shou3}[][HSK 5]
    \definition{v.+compl.}{acenar; chamar a atenção; levantar a mão e acenar com a palma, para indicar que a outra pessoa se aproxime ou para cumprimentá-la}
  \end{Phonetics}
\end{Entry}

\begin{Entry}{招生}{8,5}{⼿,⽣}
  \begin{Phonetics}{招生}{zhao1/sheng1}[][HSK 5]
    \definition{v.+compl.}{conseguir alunos; matricular novos alunos; recrutar novos alunos}
  \end{Phonetics}
\end{Entry}

\begin{Entry}{招呼}{8,8}{⼿,⼝}
  \begin{Phonetics}{招呼}{zhao1 hu5}[][HSK 4]
    \definition{v.}{chamar; chamar a atenção com palavras ou gestos | cumprimentar; saudar; cumprimentar ou despedir-se das pessoas com palavras ou gestos | pedir a alguém para fazer algo; fazer solicitações, pedir ajuda ou fazer coisas | receber e dar boas-vindas aos convidados}
  \end{Phonetics}
\end{Entry}

\begin{Entry}{招数}{8,13}{⼿,⽁}
  \begin{Phonetics}{招数}{zhao1shu4}
    \definition{s.}{estratégia | movimento (no xadrez, no palco, nas artes marciais) | esquema | truque}
  \end{Phonetics}
\end{Entry}

\begin{Entry}{招聘}{8,13}{⼿,⽿}
  \begin{Phonetics}{招聘}{zhao1pin4}[][HSK 6]
    \definition{v.}{contratar; procurar; recrutar; convidar candidatos para um emprego}
  \end{Phonetics}
\end{Entry}

%%%%%%%%%% 拣 %%%%%%%%%%
\subsection*{拣}\addcontentsline{loh}{figure}{拣}

\begin{Entry}{拣}{8}{⼿}
  \begin{Phonetics}{拣}{jian3}[][HSK 7-9]
    \definition{v.}{escolher; selecionar | pegar; coletar; reunir | o mesmo que 捡}
  \seealsoref{捡}{jian3}
  \end{Phonetics}
\end{Entry}

%%%%%%%%%% 拥 %%%%%%%%%%
\subsection*{拥}\addcontentsline{loh}{figure}{拥}

\begin{Entry}{拥}{8}{⼿}
  \begin{Phonetics}{拥}{yong1}
    \definition{v.}{segurar nos braços; abraçar | reunir em volta; envolver em volta | aglomerar-se; enxamear | para apoiar | (literário) ter; possuir}
  \end{Phonetics}
\end{Entry}

\begin{Entry}{拥有}{8,6}{⼿,⽉}
  \begin{Phonetics}{拥有}{yong1you3}[][HSK 5]
    \definition{v.}{possuir; deter; ter (grande quantidade de terras, população, bens, etc.)}
  \end{Phonetics}
\end{Entry}

\begin{Entry}{拥抱}{8,8}{⼿,⼿}
  \begin{Phonetics}{拥抱}{yong1bao4}[][HSK 5]
    \definition[个,次]{s.}{abraço}
    \definition{v.}{abraçar; segurar em seus braços; abraçar para demonstrar afeto}
  \end{Phonetics}
\end{Entry}

%%%%%%%%%% 拦 %%%%%%%%%%
\subsection*{拦}\addcontentsline{loh}{figure}{拦}

\begin{Entry}{拦}{8}{⼿}
  \begin{Phonetics}{拦}{lan2}[][HSK 7-9]
    \definition{v.}{barrar; bloquear a passagem; dificultar; obstruir; impedir}
  \end{Phonetics}
\end{Entry}

%%%%%%%%%% 拧 %%%%%%%%%%
\subsection*{拧}\addcontentsline{loh}{figure}{拧}

\begin{Entry}{拧}{8}{⼿}
  \begin{Phonetics}{拧}{ning2}[][HSK 7-9]
    \definition{v.}{torcer | beliscar; torcer a pele com os dedos e virá-la com força}
  \end{Phonetics}
  \begin{Phonetics}{拧}{ning3}[][HSK 7-9]
    \definition{adj.}{errado; equivocado; de cabeça para baixo; oposto}
    \definition{v.}{torcer; parafusar | divergir; discordar; estar em desacordo}
  \end{Phonetics}
  \begin{Phonetics}{拧}{ning4}
    \definition{adj.}{teimoso}
  \end{Phonetics}
\end{Entry}

\begin{Entry}{拧开}{8,4}{⼿,⼶}
  \begin{Phonetics}{拧开}{ning3kai1}
    \definition{v.}{ligar ou desligar (girando um botão) | girar (a maçaneta de uma porta) | abrir (uma torneira) | desenroscar (uma tampa) | desaparafusar | arrancar à força}
  \end{Phonetics}
\end{Entry}

%%%%%%%%%% 拨 %%%%%%%%%%
\subsection*{拨}\addcontentsline{loh}{figure}{拨}

\begin{Entry}{拨}{8}{⼿}
  \begin{Phonetics}{拨}{bo1}[][HSK 7-9]
    \definition{clas.}{usado para agrupar pessoas; grupo; lote}
    \definition{v.}{mover (mexer) com a mão, o pé, o bastão, etc.; usar as mãos, os pés ou os bastões para mover objetos | atribuir; alocar; reservar | virar-se; inverter a marcha | dedilhar (uma corda de violão) com os dedos ou com um instrumento | chamar (alguém)}
  \end{Phonetics}
\end{Entry}

\begin{Entry}{拨及}{8,3}{⼿,⼃}
  \begin{Phonetics}{拨及}{bo1ji2}[][HSK 7-9]
    \definition{v.}{espalhar para; envolver; afetar}
  \end{Phonetics}
\end{Entry}

\begin{Entry}{拨打}{8,5}{⼿,⼿}
  \begin{Phonetics}{拨打}{bo1 da3}[][HSK 6]
    \definition{v.}{ligar; discar; de acordo com o número da chamada, discar o número no telefone ou pressionar as teclas numéricas para fazer uma chamada}
  \end{Phonetics}
\end{Entry}

\begin{Entry}{拨转}{8,8}{⼿,⾞}
  \begin{Phonetics}{拨转}{bo1zhuan3}
    \definition{v.}{transferir (fundos, etc.) | virar | dar a volta}
  \end{Phonetics}
\end{Entry}

\begin{Entry}{拨通}{8,10}{⼿,⾡}
  \begin{Phonetics}{拨通}{bo1/tong1}[][HSK 7-9]
    \definition{v.+compl.}{discar (os números de um telefone, etc.)}
  \end{Phonetics}
\end{Entry}

\begin{Entry}{拨款}{8,12}{⼿,⽋}
  \begin{Phonetics}{拨款}{bo1kuan3}[][HSK 7-9]
    \definition[项,笔]{s.}{dinheiro apropriado; apropriação; subsídio financeiro do estado; alocação de fundos; financiamento alocado}
    \definition{v.}{apropriar-se de dinheiro; alocar fundos}
  \end{Phonetics}
\end{Entry}

%%%%%%%%%% 拜 %%%%%%%%%%
\subsection*{拜}\addcontentsline{loh}{figure}{拜}

\begin{Entry}{拜}{9}{⼿}
  \begin{Phonetics}{拜}{bai4}
    \definition*{s.}{Sobrenome: Bai}
    \definition{adv.}{respeitosamente (usado na comunicação interpessoal)}
    \definition{v.}{fazer uma visita de cortesia | adorar; prestar homenagem | fazer uma chamada cerimonial | ligar; fazer uma visita | intitular alguém com cerimônia; conceder uma posição oficial ou um determinado título com certa etiqueta | estabelecer ou jurar formalmente relacionamentos}
  \end{Phonetics}
\end{Entry}

\begin{Entry}{拜见}{9,4}{⼿,⾒}
  \begin{Phonetics}{拜见}{bai4jian4}[][HSK 7-9]
    \definition{v.}{fazer uma visita formal; ligar para prestar homenagens | encontrar-se com alguém superior ou senior}
  \end{Phonetics}
\end{Entry}

\begin{Entry}{拜会}{9,6}{⼿,⼈}
  \begin{Phonetics}{拜会}{bai4hui4}[][HSK 7-9]
    \definition{v.}{fazer uma visita oficial; fazer uma visita de cortesia; visitar; visitar e conhecer (agora usado principalmente para visitas diplomáticas oficiais)}
  \end{Phonetics}
\end{Entry}

\begin{Entry}{拜年}{9,6}{⼿,⼲}
  \begin{Phonetics}{拜年}{bai4/nian2}[][HSK 7-9]
    \definition{v.+compl.}{fazer uma visita de Ano Novo; desejar a alguém um Feliz Ano Novo; fazer uma visita cerimonial no Ano Novo}
  \end{Phonetics}
\end{Entry}

\begin{Entry}{拜托}{9,6}{⼿,⼿}
  \begin{Phonetics}{拜托}{bai4tuo1}[][HSK 7-9]
    \definition{v.}{pedir a alguém para fazer algo; pedir para outra pessoa fazer coisas para você}
  \end{Phonetics}
\end{Entry}

\begin{Entry}{拜访}{9,6}{⼿,⾔}
  \begin{Phonetics}{拜访}{bai4fang3}[][HSK 5]
    \definition{v.}{visitar; fazer uma visita (respeitosamente)}
  \end{Phonetics}
\end{Entry}

%%%%%%%%%% 括 %%%%%%%%%%
\subsection*{括}\addcontentsline{loh}{figure}{括}

\begin{Entry}{括}{9}{⼿}
  \begin{Phonetics}{括}{kuo4}
    \definition{v.}{unir (músculos, etc.); contrair | incluir | adicionar colchetes a | amarrar; empacotar}
  \end{Phonetics}
\end{Entry}

\begin{Entry}{括号}{9,5}{⼿,⼝}
  \begin{Phonetics}{括号}{kuo4 hao4}[][HSK 4]
    \definition{s.}{chaves, colchetes e parênteses (em fórmulas aritméticas ou algébricas, os símbolos que indicam a combinação e a ordem de vários números ou termos) | colchetes e parênteses usados como um tipo de sinal de pontuação para mostrar a parte explicativa de uma passagem em um texto}
  \end{Phonetics}
\end{Entry}

\begin{Entry}{括弧}{9,8}{⼿,⼸}
  \begin{Phonetics}{括弧}{kuo4hu2}[][HSK 7-9]
    \definition{s.}{parênteses; também podem se referir a indicadores}
  \end{Phonetics}
\end{Entry}

%%%%%%%%%% 拮 %%%%%%%%%%
\subsection*{拮}\addcontentsline{loh}{figure}{拮}

\begin{Entry}{拮}{9}{⼿}
  \begin{Phonetics}{拮}{jie2}
    \definition{adj.}{trabalhoso | sem dinheiro | antagônico | trabalhando duro | pressionado}
  \end{Phonetics}
\end{Entry}

\begin{Entry}{拮据}{9,11}{⼿,⼿}
  \begin{Phonetics}{拮据}{jie2ju1}
    \definition{adj.}{em circunstâncias difíceis; sem dinheiro; em dificuldades}
  \end{Phonetics}
\end{Entry}

%%%%%%%%%% 拱 %%%%%%%%%%
\subsection*{拱}\addcontentsline{loh}{figure}{拱}

\begin{Entry}{拱}{9}{⼿}
  \begin{Phonetics}{拱}{gong3}[][HSK 7-9]
    \definition*{s.}{Sobrenome: Gong}
    \definition{s.}{Arquitetura: arco}[游客们在拱门前留影。===Turistas tiram fotos em frente ao arco.]
    \definition{v.}{colocar uma mão na outra em frente ao peito (em saudação) | cercar | arquear-se | empurrar sem usar as mãos; bater em um objeto com seu corpo | (porcos, etc.) cavar a terra com o focinho; (minhocas, etc.) contorcer-se na terra | brotar através da terra}
  \end{Phonetics}
\end{Entry}

%%%%%%%%%% 拷 %%%%%%%%%%
\subsection*{拷}\addcontentsline{loh}{figure}{拷}

\begin{Entry}{拷}{9}{⼿}
  \begin{Phonetics}{拷}{kao3}[][HSK 7-9]
    \definition{v.}{açoitar; bater; torturar | copiar | Dialeto, Empréstimo linguístico: \emph{call}, ligar | vencer | interrogar sob tortura}
  \end{Phonetics}
\end{Entry}

%%%%%%%%%% 拼 %%%%%%%%%%
\subsection*{拼}\addcontentsline{loh}{figure}{拼}

\begin{Entry}{拼}{9}{⼿}
  \begin{Phonetics}{拼}{pin1}[][HSK 5]
    \definition{v.}{montar; juntar as peças | dar tudo de si no trabalho; estar disposto a arriscar a vida (em lutas, no trabalho, etc.); fazer tudo o que for preciso; arriscar tudo}
  \end{Phonetics}
\end{Entry}

\begin{Entry}{拼命}{9,8}{⼿,⼝}
  \begin{Phonetics}{拼命}{pin1/ming4}[][HSK 7-9]
    \definition{adv.}{desesperadamente; arriscar a vida; com todas as forças; por favor, faça algo com o máximo empenho e energia}
    \definition{v.+compl.}{arriscar a própria vida; esforçar-se ao máximo; brigar com alguém ou fazer coisas sem considerar a segurança}
  \end{Phonetics}
\end{Entry}

\begin{Entry}{拼音}{9,9}{⼿,⾳}
  \begin{Phonetics}{拼音}{pin1yin1}
    \definition{s.}{escrita fonética | pinyin (romanização chinesa)}
  \end{Phonetics}
\end{Entry}

\begin{Entry}{拼搏}{9,13}{⼿,⼿}
  \begin{Phonetics}{拼搏}{pin1bo2}[][HSK 7-9]
    \definition{v.}{dar tudo de si; encarar (um desafio) de frente; utilizar todas as suas forças para obter algo ou atingir um objetivo}
  \end{Phonetics}
\end{Entry}

%%%%%%%%%% 拾 %%%%%%%%%%
\subsection*{拾}\addcontentsline{loh}{figure}{拾}

\begin{Entry}{拾}{9}{⼿}
  \begin{Phonetics}{拾}{shi2}[][HSK 5]
    \definition{num.}{dez (usado no lugar do numeral 十 em cheques, notas bancárias, etc., para evitar erros ou alterações)}
    \definition{v.}{pegar (do chão); recolher}
  \end{Phonetics}
\end{Entry}

%%%%%%%%%% 持 %%%%%%%%%%
\subsection*{持}\addcontentsline{loh}{figure}{持}

\begin{Entry}{持}{9}{⼿}
  \begin{Phonetics}{持}{chi2}[][HSK 7-9]
    \definition{v.}{segurar; agarrar | opor; confrontar | apoiar; manter | gerenciar; supervisionar | sequestrar; agarrar (controlar; forçar)}
  \end{Phonetics}
\end{Entry}

\begin{Entry}{持久}{9,3}{⼿,⼃}
  \begin{Phonetics}{持久}{chi2jiu3}[][HSK 7-9]
    \definition{adj.}{duradouro; prolongado; persistente; permanente}
  \end{Phonetics}
\end{Entry}

\begin{Entry}{持之以恒}{9,3,4,9}{⼿,⼂,⼈,⼼}
  \begin{Phonetics}{持之以恒}{chi2zhi1-yi3heng2}[][HSK 7-9]
    \definition{expr.}{perseguir incessantemente | de forma persistente | perseverar em (fazer algo)}
  \end{Phonetics}
\end{Entry}

\begin{Entry}{持有}{9,6}{⼿,⽉}
  \begin{Phonetics}{持有}{chi2 you3}[][HSK 6]
    \definition{v.}{segurar; possuir | segurar; ter; abrigar; ter em mente (ideias, opiniões, etc.)}
  \end{Phonetics}
\end{Entry}

\begin{Entry}{持续}{9,11}{⼿,⽷}
  \begin{Phonetics}{持续}{chi2xu4}[][HSK 3]
    \definition{v.}{durar; continuar; sustentar; manter a situação ou as condições como estão, sem alterações}
  \end{Phonetics}
\end{Entry}

%%%%%%%%%% 挂 %%%%%%%%%%
\subsection*{挂}\addcontentsline{loh}{figure}{挂}

\begin{Entry}{挂}{9}{⼿}
  \begin{Phonetics}{挂}{gua4}[][HSK 3]
    \definition{clas.}{usado principalmente para coisas que vêm em conjuntos ou séries}
    \definition{v.}{pendurar; colocar; suspender; usando cordas, ganchos, pregos e outros itens para prender objetos em um ou mais pontos específicos | interromper chamada (telefônica) | colocar alguém em contato com; ligar; telefonar; refere-se a ligar o telefone, bem como a fazer uma chamada | falhar; fracassar | colocar em registro; registrarpegar carona; ser pego | preocupar-se com | ser revestido com; ser coberto com | estar pendente; deixar algo sem solução}
  \end{Phonetics}
\end{Entry}

\begin{Entry}{挂号}{9,5}{⼿,⼝}
  \begin{Phonetics}{挂号}{gua4/hao4}[][HSK 7-9]
    \definition{v.+compl.}{registrar-se (em um hospital, etc.) | enviar através de carta registrada}
  \end{Phonetics}
\end{Entry}

\begin{Entry}{挂号信}{9,5,9}{⼿,⼝,⼈}
  \begin{Phonetics}{挂号信}{gua4hao4xin4}[][HSK 5]
    \definition{s.}{carta registrada}
  \end{Phonetics}
\end{Entry}

\begin{Entry}{挂失}{9,5}{⼿,⼤}
  \begin{Phonetics}{挂失}{gua4/shi1}[][HSK 7-9]
    \definition{v.+compl.}{relatar a perda de algo; se perder uma nota ou certificado, você deve registrá-lo junto à autoridade emissora ou declará-lo inválido}
  \end{Phonetics}
\end{Entry}

\begin{Entry}{挂念}{9,8}{⼿,⼼}
  \begin{Phonetics}{挂念}{gua4nian4}[][HSK 7-9]
    \definition{v.}{sentir falta; preocupar-se com alguém que está ausente}
  \end{Phonetics}
\end{Entry}

\begin{Entry}{挂钩}{9,9}{⼿,⾦}
  \begin{Phonetics}{挂钩}{gua4gou1}[][HSK 7-9]
    \definition[个,种]{s.}{(vagões ferroviários) acoplamento; manilha; engate | gancho}
    \definition{v.}{acoplar (dois vagões ferroviários); articular | conectar-se com; estabelecer contato com; entrar em contato com; vincular-se a}
  \end{Phonetics}
\end{Entry}

%%%%%%%%%% 指 %%%%%%%%%%
\subsection*{指}\addcontentsline{loh}{figure}{指}

\begin{Entry}{指}{9}{⼿}
  \begin{Phonetics}{指}{zhi3}[][HSK 3]
    \definition*{s.}{Sobrenome: Zhi}
    \definition{clas.}{dígito; largura do dedo; a largura de um dedo é chamada de 一指, que é usado para medir profundidade, largura, etc.}
    \definition{s.}{dedo}
    \definition{v.}{apontar para; indicar; usar o dedo ou a ponta de um objeto para apontar | (pelo) eriçar;  (cabelo) ficar em pé | indicar; mostrar; apontar; demonstrar | referir-se a; dirigir-se a | confiar em; contar com; depender de | criticar; repreender}
  \end{Phonetics}
\end{Entry}

\begin{Entry}{指出}{9,5}{⼿,⼐}
  \begin{Phonetics}{指出}{zhi3 chu1}[][HSK 3]
    \definition{v.}{apontar; indicar}
  \end{Phonetics}
\end{Entry}

\begin{Entry}{指头}{9,5}{⼿,⼤}
  \begin{Phonetics}{指头}{zhi3 tou5}[][HSK 6]
    \definition[个,根,只]{s.}{dedo da mão ou do pé}
  \end{Phonetics}
\end{Entry}

\begin{Entry}{指甲}{9,5}{⼿,⽥}
  \begin{Phonetics}{指甲}{zhi3jia5}[][HSK 5]
    \definition[个,种]{s.}{unha; unha de agulha; unha de dedo; camada córnea na ponta dos dedos}
  \end{Phonetics}
\end{Entry}

\begin{Entry}{指示}{9,5}{⼿,⽰}
  \begin{Phonetics}{指示}{zhi3shi4}[][HSK 5]
    \definition[点,条,项,个]{s.}{diretriz; instruções; para orientar o trabalho, os superiores emitem opiniões verbais ou escritas aos subordinados}
    \definition{v.}{indicar; apontar; apontar para alguém | instruir; superiores emitem opiniões verbais ou escritas para orientar o trabalho dos subordinados}
  \end{Phonetics}
\end{Entry}

\begin{Entry}{指导}{9,6}{⼿,⼨}
  \begin{Phonetics}{指导}{zhi3dao3}[][HSK 3]
    \definition[位]{s.}{guia; diretor; pessoa que dá orientações}
    \definition{v.}{orientar; dirigir; instruir}
  \end{Phonetics}
\end{Entry}

\begin{Entry}{指定}{9,8}{⼿,⼧}
  \begin{Phonetics}{指定}{zhi3ding4}[][HSK 6]
    \definition{adv.}{certamente; com certeza; reforça o tom de palpite e estimativa}
    \definition{v.}{nomear; atribuir; determinar a pessoa, evento, lugar, conteúdo, etc. que faz algo}
  \end{Phonetics}
\end{Entry}

\begin{Entry}{指责}{9,8}{⼿,⾙}
  \begin{Phonetics}{指责}{zhi3ze2}[][HSK 5]
    \definition{v.}{censurar; criticar; encontrar falhas; repreender}
  \end{Phonetics}
\end{Entry}

\begin{Entry}{指南针}{9,9,7}{⼿,⼗,⾦}
  \begin{Phonetics}{指南针}{zhi3nan2zhen1}
    \definition{s.}{bússola}
  \end{Phonetics}
\end{Entry}

\begin{Entry}{指挥}{9,9}{⼿,⼿}
  \begin{Phonetics}{指挥}{zhi3hui1}[][HSK 4]
    \definition[个,位,名]{s.}{diretor; comandante; despachante; operador | maestro; condutor; pessoa na frente de uma orquestra ou coro que dá instruções sobre como tocar ou cantar}
    \definition{v.}{dirigir; conduzir; comandar; direcionar; emitir ordens de agendamento}
  \end{Phonetics}
\end{Entry}

\begin{Entry}{指标}{9,9}{⼿,⽊}
  \begin{Phonetics}{指标}{zhi3biao1}[][HSK 5]
    \definition[个,种]{s.}{meta; cota; norma; índice; objetivos a serem alcançados | alvo; índice; refletir os requisitos de desenvolvimento em determinados aspectos através de números absolutos ou percentagens de aumento ou diminuição, inclui indicadores quantitativos e qualitativos}
  \end{Phonetics}
\end{Entry}

\begin{Entry}{指着}{9,11}{⼿,⽬}
  \begin{Phonetics}{指着}{zhi3 zhe5}[][HSK 6]
    \definition{v.}{apontar}
  \end{Phonetics}
\end{Entry}

\begin{Entry}{指数}{9,13}{⼿,⽁}
  \begin{Phonetics}{指数}{zhi3 shu4}[][HSK 6]
    \definition{s.}{Matemática: expoente; refere-se ao número de vezes que um número é multiplicado por si mesmo, o que é registrado no canto superior direito do número | Estatística: índice, refere-se à razão entre o valor de um fenômeno econômico em um determinado período e o valor de outro período usado como padrão de comparação, geralmente expresso como uma porcentagem, como o índice de preços ao consumidor}
  \end{Phonetics}
\end{Entry}

%%%%%%%%%% 按 %%%%%%%%%%
\subsection*{按}\addcontentsline{loh}{figure}{按}

\begin{Entry}{按}{9}{⼿}
  \begin{Phonetics}{按}{an4}[][HSK 3]
    \definition{prep.}{de acordo com; à luz de; com base em; em conformidade com}
    \definition{v.}{pressionar; empurrar para baixo; pressionar ou apertar com a mão ou os dedos | pôr de parte; deixar de lado; deixar para mais tarde | restringir; controlar; inibir; parar | verificar; consultar | comentar ou anotar (por um editor ou autor)}
  \end{Phonetics}
\end{Entry}

\begin{Entry}{按时}{9,7}{⼿,⽇}
  \begin{Phonetics}{按时}{an4shi2}[][HSK 4]
    \definition{adv.}{na hora; no horário; pontualmente; de acordo com o tempo estipulado}
  \end{Phonetics}
\end{Entry}

\begin{Entry}{按说}{9,9}{⼿,⾔}
  \begin{Phonetics}{按说}{an4shuo1}[][HSK 7-9]
    \definition{adv.}{no curso normal dos eventos; ordinariamente; normalmente | de acordo com o fato (senso comum); refere"-se a falar de acordo com fatos ou senso comum; como uma questão de razão; expressões semelhantes incluem 按理 e 按理说}
  \seealsoref{按理}{an4li3}
  \seealsoref{按理说}{an4li3 shuo1}
  \end{Phonetics}
\end{Entry}

\begin{Entry}{按理}{9,11}{⼿,⽟}
  \begin{Phonetics}{按理}{an4li3}
    \definition{adv.}{de acordo com o princípio ou a razão; no curso normal dos eventos; normalmente | de acordo com a razão; de acordo com a prática comum; por (bons) direitos}
  \end{Phonetics}
\end{Entry}

\begin{Entry}{按理说}{9,11,9}{⼿,⽟,⾔}
  \begin{Phonetics}{按理说}{an4li3 shuo1}[][HSK 7-9]
    \definition{adv.}{de acordo com o princípio ou a razão; no curso normal dos eventos; normalmente | é razoável dizer que\dots}
  \end{Phonetics}
\end{Entry}

\begin{Entry}{按照}{9,13}{⼿,⽕}
  \begin{Phonetics}{按照}{an4zhao4}[][HSK 3]
    \definition{prep.}{de acordo com; em conformidade com; à luz de; com base em; apresentar os fundamentos ou critérios de julgamento para fazer algo}
  \end{Phonetics}
\end{Entry}

\begin{Entry}{按键}{9,13}{⼿,⾦}
  \begin{Phonetics}{按键}{an4jian4}[][HSK 7-9]
    \definition[个]{s.}{tecla; botão; teclas pressionadas manualmente}[键盘上的按键非常灵敏。===As teclas do teclado são muito responsivas.]
  \end{Phonetics}
\end{Entry}

\begin{Entry}{按摩}{9,15}{⼿,⼿}
  \begin{Phonetics}{按摩}{an4mo2}[][HSK 5]
    \definition{s.}{massagem; empurrar, pressionar, beliscar e amassar o corpo de uma pessoa com as mãos para promover a circulação sanguínea, aumentar a resistência da pele e regular a função dos nervos}
  \end{Phonetics}
\end{Entry}

%%%%%%%%%% 挎 %%%%%%%%%%
\subsection*{挎}\addcontentsline{loh}{figure}{挎}

\begin{Entry}{挎}{9}{⼿}
  \begin{Phonetics}{挎}{kua4}[][HSK 7-9]
    \definition{v.}{carregar no braço | carregar algo sobre o ombro, ao redor do pescoço ou ao lado do corpo | pendurar coisas no ombro, pescoço ou cintura}
  \end{Phonetics}
\end{Entry}

%%%%%%%%%% 挑 %%%%%%%%%%
\subsection*{挑}\addcontentsline{loh}{figure}{挑}

\begin{Entry}{挑}{9}{⼿}
  \begin{Phonetics}{挑}{tiao1}[][HSK 4]
    \definition{clas.}{usado para coisas que são escolhidas ou selecionadas | usado para coisas que podem ser usadas como palhetas}
    \definition{s.}{vara comprida com algo pendurado nas pontas; haste de transporte}
    \definition{v.}{escolher; selecionar | fazer picuinhas; ser hipercrítico; ser meticuloso; ser excessivamente rigoroso nos detalhes | carregar com uma haste de transporte; carregar no ombro; pendurar coisas nas pontas de varas longas e carregá-las em seus ombros}
  \end{Phonetics}
  \begin{Phonetics}{挑}{tiao3}[][HSK 4]
    \definition{s.}{um dos traços dos caracteres chineses; inclinado para cima da esquerda para a direita}
    \definition{v.}{levantar; elevar; erguer | levantar ou apoiar com uma extremidade de uma vara ou objeto semelhante; segurar ou apoiar com a ponta de uma vara etc. | causar conflitos deliberadamente; provocar deliberadamente um conflito | (método de bordado) usar uma agulha para levantar os fios de urdidura ou trama, com a agulha e a linha passando por baixo para formar padrões e desenhos}
  \end{Phonetics}
\end{Entry}

\begin{Entry}{挑战}{9,9}{⼿,⼽}
  \begin{Phonetics}{挑战}{tiao3zhan4}[][HSK 4]
    \definition{v.}{desafiar; deixar um oponente deliberadamente irritado e sair para lutar ou lutar consigo mesmo; estimular um oponente a lutar consigo mesmo}
  \end{Phonetics}
\end{Entry}

\begin{Entry}{挑选}{9,9}{⼿,⾡}
  \begin{Phonetics}{挑选}{tiao1 xuan3}[][HSK 4]
    \definition{v.}{escolher; optar; selecionar; escolher a pessoa ou coisa certa para o trabalho}
  \end{Phonetics}
\end{Entry}

\begin{Entry}{挑衅}{9,11}{⼿,⾎}
  \begin{Phonetics}{挑衅}{tiao3xin4}
    \definition{s.}{provocação}
    \definition{v.}{provocar; causar problemas; tentar causar conflito ou guerra}
  \end{Phonetics}
\end{Entry}

%%%%%%%%%% 挖 %%%%%%%%%%
\subsection*{挖}\addcontentsline{loh}{figure}{挖}

\begin{Entry}{挖}{9}{⼿}
  \begin{Phonetics}{挖}{wa1}[][HSK 6]
    \definition{v.}{cavar; escavar; arrancar | explorar; sondar | (dialeto) arranhar | escavar a superfície de um objeto com ferramentas ou mãos}
  \end{Phonetics}
\end{Entry}

\begin{Entry}{挖掘机}{9,11,6}{⼿,⼿,⽊}
  \begin{Phonetics}{挖掘机}{wa1jue2ji1}
    \definition{s.}{escavadeira | escavador | escavadora | pá mecânica}
  \end{Phonetics}
\end{Entry}

%%%%%%%%%% 挠 %%%%%%%%%%
\subsection*{挠}\addcontentsline{loh}{figure}{挠}

\begin{Entry}{挠}{9}{⼿}
  \begin{Phonetics}{挠}{nao2}[][HSK 7-9]
    \definition{v.}{coçar; (usar os dedos) para segurar delicadamente | dificultar; obstruir; impedir que os outros façam as coisas sem problemas | recuar; ceder; dar a mão; dobrar, metaforicamente significando ceder}
  \end{Phonetics}
\end{Entry}

%%%%%%%%%% 挡 %%%%%%%%%%
\subsection*{挡}\addcontentsline{loh}{figure}{挡}

\begin{Entry}{挡}{9}{⼿}
  \begin{Phonetics}{挡}{dang3}[][HSK 5]
    \definition{s.}{persiana; veneziana; paralama; coisas para cobrir ou bloquear | caixa de câmbio (automóvel)}
    \definition{v.}{bloquear; resistir; manter afastado; afastar | cobrir; bloquear; atrapalhar}
  \end{Phonetics}
  \begin{Phonetics}{挡}{dang4}
    \definition{v.}{organizar}
  \end{Phonetics}
\end{Entry}

\begin{Entry}{挡风玻璃}{9,4,9,14}{⼿,⾵,⽟,⽟}
  \begin{Phonetics}{挡风玻璃}{dang3feng1bo1li5}
    \definition{s.}{parabrisa}
  \end{Phonetics}
\end{Entry}

%%%%%%%%%% 挣 %%%%%%%%%%
\subsection*{挣}\addcontentsline{loh}{figure}{挣}

\begin{Entry}{挣}{9}{⼿}
  \begin{Phonetics}{挣}{zheng4}[][HSK 5]
    \definition{v.}{empurrar e puxar; tentar se livrar; lutar para se libertar; esforçar-se para se libertar das amarras | ganhar; fazer; trabalhar para; trocar trabalho por}
  \end{Phonetics}
\end{Entry}

\begin{Entry}{挣扎}{9,4}{⼿,⼿}
  \begin{Phonetics}{挣扎}{zheng1zha2}
    \definition{v.}{lutar}
  \end{Phonetics}
\end{Entry}

\begin{Entry}{挣钱}{9,10}{⼿,⾦}
  \begin{Phonetics}{挣钱}{zheng4/qian2}[][HSK 5]
    \definition{v.+compl.}{ganhar dinheiro; fazer dinheiro; lucrar; trabalhar para ganhar dinheiro}
  \end{Phonetics}
\end{Entry}

\begin{Entry}{挣得}{9,11}{⼿,⼻}
  \begin{Phonetics}{挣得}{zheng4de2}
    \definition{v.}{ganhar renda ou dinheiro}
  \end{Phonetics}
\end{Entry}

%%%%%%%%%% 挤 %%%%%%%%%%
\subsection*{挤}\addcontentsline{loh}{figure}{挤}

\begin{Entry}{挤}{9}{⼿}
  \begin{Phonetics}{挤}{ji3}[][HSK 5]
    \definition{adj.}{lotado; congestionado; descreve um grande número de pessoas ou coisas e muito pouco espaço}
    \definition{v.}{empacotar; amontoar; aglomerar | sacudir; empurrar contra; empurrar alguém ou algo para longe com seu corpo com toda a força que puder| pressionar; apertar; expulsar por pressão}
  \end{Phonetics}
\end{Entry}

\begin{Entry}{挤压}{9,6}{⼿,⼚}
  \begin{Phonetics}{挤压}{ji3ya1}[][HSK 7-9]
    \definition{v.}{pressionar; espremer; esmagar; aperta | Metalurgia: extrudar}
  \end{Phonetics}
\end{Entry}

%%%%%%%%%% 挥 %%%%%%%%%%
\subsection*{挥}\addcontentsline{loh}{figure}{挥}

\begin{Entry}{挥}{9}{⼿}
  \begin{Phonetics}{挥}{hui1}[][HSK 7-9]
    \definition{v.}{acenar; empunhar; socar | limpar lágrimas, suor, etc. com as mãos | comandar (um exército) | espalhar; dispersar | afastar-se; livrar-se de}
  \end{Phonetics}
\end{Entry}

\begin{Entry}{挥汗如雨}{9,6,6,8}{⼿,⽔,⼥,⾬}
  \begin{Phonetics}{挥汗如雨}{hui1han4ru2yu3}
    \definition{s.}{suor derramado}
    \definition{v.}{pingar com suor}
  \end{Phonetics}
\end{Entry}

%%%%%%%%%% 挪 %%%%%%%%%%
\subsection*{挪}\addcontentsline{loh}{figure}{挪}

\begin{Entry}{挪}{9}{⼿}
  \begin{Phonetics}{挪}{nuo2}[][HSK 7-9]
    \definition{v.}{mover; deslocar; transportar}
  \end{Phonetics}
\end{Entry}

%%%%%%%%%% 挺 %%%%%%%%%%
\subsection*{挺}\addcontentsline{loh}{figure}{挺}

\begin{Entry}{挺}{9}{⼿}
  \begin{Phonetics}{挺}{ting3}[][HSK 2,4]
    \definition{adj.}{rígido; ereto; vertical; reto | notável; destacado; distinto}
    \definition{adv.}{muito; bastante}
    \definition{clas.}{usado para metralhadoras}
    \definition{v.}{sobressair; endireitar-se; protrudir (protuberância ou saliência) | suportar; aguentar; resistir; perseverar}
  \end{Phonetics}
\end{Entry}

\begin{Entry}{挺尸}{9,3}{⼿,⼫}
  \begin{Phonetics}{挺尸}{ting3shi1}
    \definition{v.}{(coloquial) dormir | (literalmente) ficar deitado duro como um cadáver}
  \end{Phonetics}
\end{Entry}

\begin{Entry}{挺立}{9,5}{⼿,⽴}
  \begin{Phonetics}{挺立}{ting3li4}
    \definition{v.}{ficar ereto | ficar de pé}
  \end{Phonetics}
\end{Entry}

\begin{Entry}{挺好}{9,6}{⼿,⼥}
  \begin{Phonetics}{挺好}{ting3 hao3}[][HSK 2]
    \definition{adj.}{nada mal; surpreendentemente bom}
  \end{Phonetics}
\end{Entry}

\begin{Entry}{挺过}{9,6}{⼿,⾡}
  \begin{Phonetics}{挺过}{ting3guo4}
    \definition{s.}{sobreviver}
  \end{Phonetics}
\end{Entry}

\begin{Entry}{挺住}{9,7}{⼿,⼈}
  \begin{Phonetics}{挺住}{ting3zhu4}
    \definition{v.}{permanecer firme | manter-se firme (diante da adversidade ou da dor)}
  \end{Phonetics}
\end{Entry}

\begin{Entry}{挺杆}{9,7}{⼿,⽊}
  \begin{Phonetics}{挺杆}{ting3gan3}
    \definition{s.}{tucho (peça de máquina)}
  \end{Phonetics}
\end{Entry}

\begin{Entry}{挺身}{9,7}{⼿,⾝}
  \begin{Phonetics}{挺身}{ting3shen1}
    \definition{v.}{endireitar as costas}
  \end{Phonetics}
\end{Entry}

\begin{Entry}{挺进}{9,7}{⼿,⾡}
  \begin{Phonetics}{挺进}{ting3jin4}
    \definition{s.}{progresso | avanço}
    \definition{v.}{progredir | avançar}
  \end{Phonetics}
\end{Entry}

\begin{Entry}{挺拔}{9,8}{⼿,⼿}
  \begin{Phonetics}{挺拔}{ting3ba2}
    \definition{adj.}{alto e reto}
  \end{Phonetics}
\end{Entry}

\begin{Entry}{挺腰}{9,13}{⼿,⾁}
  \begin{Phonetics}{挺腰}{ting3yao1}
    \definition{v.}{arquear as costas | endireitar as costas}
  \end{Phonetics}
\end{Entry}

%%%%%%%%%% 拳 %%%%%%%%%%
\subsection*{拳}\addcontentsline{loh}{figure}{拳}

\begin{Entry}{拳}{10}{⼿}
  \begin{Phonetics}{拳}{quan2}[][HSK 7-9]
    \definition*{s.}{Sobrenome: Quan}
    \definition[个,记,套]{s.}{punho | boxe; pugilismo}
    \definition{v.}{enrolar}
  \end{Phonetics}
\end{Entry}

\begin{Entry}{拳王}{10,4}{⼿,⽟}
  \begin{Phonetics}{拳王}{quan2wang2}
    \definition{s.}{pugilista | boxeador}
  \end{Phonetics}
\end{Entry}

\begin{Entry}{拳头}{10,5}{⼿,⼤}
  \begin{Phonetics}{拳头}{quan2tou2}[][HSK 7-9]
    \definition{adj.}{nocaute; (de produtos) de boa qualidade e competitividade; uma metáfora para ter uma vantagem e uma forte competitividade}
    \definition[个]{s.}{punho; mãos com os dedos dobrados para dentro e entrelaçados}
  \end{Phonetics}
\end{Entry}

\begin{Entry}{拳法}{10,8}{⼿,⽔}
  \begin{Phonetics}{拳法}{quan2fa3}
    \definition{s.}{boxe | luta}
  \end{Phonetics}
\end{Entry}

%%%%%%%%%% 拿 %%%%%%%%%%
\subsection*{拿}\addcontentsline{loh}{figure}{拿}

\begin{Entry}{拿}{10}{⼿}
  \begin{Phonetics}{拿}{na2}[][HSK 1]
    \definition{part.}{usado da mesma forma que 把: para marcar o seguinte substantivo seguinte como objeto direto}
    \definition{prep.}{ferramentas, materiais, métodos, etc. utilizados para a introdução | os objetos que estão sendo manipulados para introdução}
    \definition{v.}{segurar; pegar; pegar ou mover objetos com as mãos ou de outra forma | apreender; capturar; prender; usar força bruta para capturar | ter certeza de; ser capaz de fazer; ter uma compreensão firme de | tornar as coisas difíceis para alguém; colocar alguém em uma situação difícil; obstruir; chantagear; coagir; causar dificuldades intencionalmente | fingir ou fazer (algum tipo de postura ou aparência) | ter certeza de; tomar uma decisão | obter; ganhar; receber}
  \end{Phonetics}
\end{Entry}

\begin{Entry}{拿手}{10,4}{⼿,⼿}
  \begin{Phonetics}{拿手}{na2shou3}[][HSK 7-9]
    \definition{adj.}{hábil; especialista; bom em; proficiente em determinada tecnologia}
  \end{Phonetics}
\end{Entry}

\begin{Entry}{拿出}{10,5}{⼿,⼐}
  \begin{Phonetics}{拿出}{na2 chu1}[][HSK 2]
    \definition{v.}{apresentar (evidências) | fornecer | apresentar (uma proposta) | oferecer; servir | retirar; tirar}
  \end{Phonetics}
\end{Entry}

\begin{Entry}{拿走}{10,7}{⼿,⾛}
  \begin{Phonetics}{拿走}{na2 zou3}[][HSK 6]
    \definition{v.}{tirar; remover}
  \end{Phonetics}
\end{Entry}

\begin{Entry}{拿到}{10,8}{⼿,⼑}
  \begin{Phonetics}{拿到}{na2 dao4}[][HSK 2]
    \definition{v.}{pegar; obter, conseguir}
  \end{Phonetics}
\end{Entry}

%%%%%%%%%% 挨 %%%%%%%%%%
\subsection*{挨}\addcontentsline{loh}{figure}{挨}

\begin{Entry}{挨}{10}{⼿}
  \begin{Phonetics}{挨}{ai1}
    \definition{prep.}{por turnos; em sequência; indica sequencialmente}
    \definition{v.}{estar próximo de; estar (chegar) perto de; abordar}
  \end{Phonetics}
  \begin{Phonetics}{挨}{ai2}[][HSK 6]
    \definition{v.}{sofrer; suportar | prolongar; lutar para superar (tempos difíceis); superar (tempo) com dificuldade | atrasar; adiar; procrastinar}
  \end{Phonetics}
\end{Entry}

\begin{Entry}{挨打}{10,5}{⼿,⼿}
  \begin{Phonetics}{挨打}{ai2/da3}[][HSK 6]
    \definition{v.+compl.}{levar uma surra; ser atacado; ser espancado}
  \end{Phonetics}
\end{Entry}

\begin{Entry}{挨家挨户}{10,10,10,4}{⼿,⼧,⼿,⼾}
  \begin{Phonetics}{挨家挨户}{ai1jia1-ai1hu4}[][HSK 7-9]
    \definition{expr.}{ir de casa em casa, de porta em porta; um após o outro}
  \end{Phonetics}
\end{Entry}

\begin{Entry}{挨着}{10,11}{⼿,⽬}
  \begin{Phonetics}{挨着}{ai1 zhe5}[][HSK 6]
    \definition{adv.}{ao lado de; perto de; imediatamente depois}
  \end{Phonetics}
\end{Entry}

%%%%%%%%%% 挫 %%%%%%%%%%
\subsection*{挫}\addcontentsline{loh}{figure}{挫}

\begin{Entry}{挫}{10}{⼿}
  \begin{Phonetics}{挫}{cuo4}
    \definition{v.}{frustrar | diminuir; embotar; desinflar | pressionar para baixo; abaixar}
  \end{Phonetics}
\end{Entry}

\begin{Entry}{挫折}{10,7}{⼿,⼿}
  \begin{Phonetics}{挫折}{cuo4zhe2}[][HSK 7-9]
    \definition[个,次]{s.}{retrocesso; reversão; frustração | derrota, fracasso, insucesso}
    \definition{v.}{falhar; derrotar; fracassar}
  \end{Phonetics}
\end{Entry}

%%%%%%%%%% 振 %%%%%%%%%%
\subsection*{振}\addcontentsline{loh}{figure}{振}

\begin{Entry}{振}{10}{⼿}
  \begin{Phonetics}{振}{zhen4}
    \definition{v.}{sacudir; acenar; bater as asas; empunhar | vibrar | recompor-se; levantar-se; animar}
  \end{Phonetics}
\end{Entry}

\begin{Entry}{振动}{10,6}{⼿,⼒}
  \begin{Phonetics}{振动}{zhen4dong4}[][HSK 5]
    \definition{s.}{vibração}
    \definition{v.}{sacudir; balançar; tremer; roncar; tagarelar; vibrar; oscilar; a física se refere ao movimento contínuo de um objeto em torno de um determinado ponto no espaço, como o movimento de um pêndulo, um diapasão ou uma corda de violão}
  \end{Phonetics}
\end{Entry}

%%%%%%%%%% 捆 %%%%%%%%%%
\subsection*{捆}\addcontentsline{loh}{figure}{捆}

\begin{Entry}{捆}{10}{⼿}
  \begin{Phonetics}{捆}{kun3}[][HSK 7-9]
    \definition{clas.}{feixe; maço; materiais usados ​​para amarrar}
    \definition{s.}{coisas que estão agrupadas}
    \definition{v.}{amarrar; prender; agrupar | amarrar; acorrentar; algemar | agrupar; enfardar}
  \seealsoref{捆儿}{kun3r5}
  \end{Phonetics}
\end{Entry}

\begin{Entry}{捆儿}{10,2}{⼿,⼉}
  \begin{Phonetics}{捆儿}{kun3r5}
    \definition{s.}{coisas que estão agrupadas}
  \seealsoref{捆}{kun3}
  \end{Phonetics}
\end{Entry}

%%%%%%%%%% 捉 %%%%%%%%%%
\subsection*{捉}\addcontentsline{loh}{figure}{捉}

\begin{Entry}{捉}{10}{⼿}
  \begin{Phonetics}{捉}{zhuo1}[][HSK 6]
    \definition{v.}{agarrar; segurar; apreender | pegar; capturar; aprisionar}
  \end{Phonetics}
\end{Entry}

%%%%%%%%%% 捍 %%%%%%%%%%
\subsection*{捍}\addcontentsline{loh}{figure}{捍}

\begin{Entry}{捍}{10}{⼿}
  \begin{Phonetics}{捍}{han4}
    \definition{v.}{defender; guardar | defender-se | afastar (um golpe) | resistir}
  \end{Phonetics}
\end{Entry}

\begin{Entry}{捍卫}{10,3}{⼿,⼙}
  \begin{Phonetics}{捍卫}{han4wei4}[][HSK 7-9]
    \definition{v.}{defender; guardar; proteger; defender-se pela força ou outros meios de ser violado ou prejudicado}
  \end{Phonetics}
\end{Entry}

%%%%%%%%%% 捎 %%%%%%%%%%
\subsection*{捎}\addcontentsline{loh}{figure}{捎}

\begin{Entry}{捎}{10}{⼿}
  \begin{Phonetics}{捎}{shao1}[][HSK 7-9]
    \definition{v.}{trazer para alguém; levar algo para alguém ou em nome de alguém; levar consigo}
  \end{Phonetics}
  \begin{Phonetics}{捎}{shao4}
    \definition{adj.}{2. desbotado (cor)}
    \definition{v.}{recuar; puxar para trás (cavalo, burro); mover-se ligeiramente para trás (geralmente referindo-se a mulas, cavalos, etc.) | desbotar (cor)}
  \end{Phonetics}
\end{Entry}

%%%%%%%%%% 捏 %%%%%%%%%%
\subsection*{捏}\addcontentsline{loh}{figure}{捏}

\begin{Entry}{捏}{10}{⼿}
  \begin{Phonetics}{捏}{nie1}[][HSK 7-9]
    \definition{v.}{beliscar; segurar entre os dedos; usar o polegar e os outros dedos para pinçar | moldar; amassar com os dedos; usar os dedos para moldar o objeto macio | fabricar; compor; apresentar deliberadamente uma declaração falsa como se fosse um fato | juntar; unir; unir duas coisas ou duas pessoas | beliscar; usar as mãos para empurrar, pressionar, beliscar e amassar o corpo pode promover a circulação sanguínea, aumentar a resistência da pele e regular a função nervosa; pressionar firmemente com a palma da mão}
  \end{Phonetics}
\end{Entry}

%%%%%%%%%% 捐 %%%%%%%%%%
\subsection*{捐}\addcontentsline{loh}{figure}{捐}

\begin{Entry}{捐}{10}{⼿}
  \begin{Phonetics}{捐}{juan1}[][HSK 6]
    \definition{s.}{imposto}
    \definition{v.}{renunciar; abandonar | contribuir; doar; assinar}
  \end{Phonetics}
\end{Entry}

\begin{Entry}{捐助}{10,7}{⼿,⼒}
  \begin{Phonetics}{捐助}{juan1 zhu4}[][HSK 6]
    \definition{v.}{oferecer (assistência financeira ou material); contribuir; doar}
  \end{Phonetics}
\end{Entry}

\begin{Entry}{捐款}{10,12}{⼿,⽋}
  \begin{Phonetics}{捐款}{juan1/kuan3}[][HSK 6]
    \definition[笔]{s.}{doação; contribuição (de dinheiro); valor doado}
    \definition{v.+compl.}{doar; contribuir com dinheiro}
  \end{Phonetics}
\end{Entry}

\begin{Entry}{捐献}{10,13}{⼿,⽝}
  \begin{Phonetics}{捐献}{juan1xian4}[][HSK 7-9]
    \definition{v.}{doar; apresentar; contribuir (para uma organização); doar bens ao (estado, a uma cooperativa, etc.)}
  \end{Phonetics}
\end{Entry}

\begin{Entry}{捐赠}{10,16}{⼿,⾙}
  \begin{Phonetics}{捐赠}{juan1 zeng4}[][HSK 6]
    \definition{v.}{apresentar; contribuir (como um presente); doar (itens para um país ou grupo)}
  \end{Phonetics}
\end{Entry}

%%%%%%%%%% 捕 %%%%%%%%%%
\subsection*{捕}\addcontentsline{loh}{figure}{捕}

\begin{Entry}{捕}{10}{⼿}
  \begin{Phonetics}{捕}{bu3}[][HSK 6]
    \definition{v.}{pegar; apreender; prender}
  \end{Phonetics}
\end{Entry}

\begin{Entry}{捕捉}{10,10}{⼿,⼿}
  \begin{Phonetics}{捕捉}{bu3zhuo1}[][HSK 7-9]
    \definition{v.}{caçar; perseguir; pegar; capturar; apreender; pegar; fazer uma pessoa ou animal cair nas mãos; pode ser usado tanto para pessoas quanto para coisas; tem uma ampla gama de aplicações; usado tanto na linguagem falada quanto na escrita}
  \end{Phonetics}
\end{Entry}

%%%%%%%%%% 捞 %%%%%%%%%%
\subsection*{捞}\addcontentsline{loh}{figure}{捞}

\begin{Entry}{捞}{10}{⼿}
  \begin{Phonetics}{捞}{lao1}[][HSK 7-9]
    \definition{v.}{arrastar para; pescar para; recolher; dragar (para fora); retirar algo da água ou de outros líquidos | ganhar; obter por meios ilícitos | sair andando com alguma coisa; puxar ou pegar casualmente}
  \end{Phonetics}
\end{Entry}

%%%%%%%%%% 损 %%%%%%%%%%
\subsection*{损}\addcontentsline{loh}{figure}{损}

\begin{Entry}{损}{10}{⼿}
  \begin{Phonetics}{损}{sun3}
    \definition{adj.}{sarcástico; cortante; de ​​língua afiada; maldoso; mau; cruel}
    \definition{v.}{diminuir; perder; reduzir | prejudicar; danificar; degradar; destruir; arruinar; destruir o estado original ou fazê-lo perder sua eficácia original | ser sarcástico; ser cáustico; ser cortante; ferir; insultar; usar palavras duras para zombar de alguém}
  \end{Phonetics}
\end{Entry}

\begin{Entry}{损失}{10,5}{⼿,⼤}
  \begin{Phonetics}{损失}{sun3shi1}[][HSK 5]
    \definition{s.}{perda; desperdício; algo que se consome ou se perde sem custo algum}
    \definition{v.}{perder; consumir ou perder}
  \end{Phonetics}
\end{Entry}

\begin{Entry}{损害}{10,10}{⼿,⼧}
  \begin{Phonetics}{损害}{sun3 hai4}[][HSK 5]
    \definition{v.}{prejudicar; danificar; ferir; causar danos; causar perdas}
  \end{Phonetics}
\end{Entry}

%%%%%%%%%% 捡 %%%%%%%%%%
\subsection*{捡}\addcontentsline{loh}{figure}{捡}

\begin{Entry}{捡}{10}{⼿}
  \begin{Phonetics}{捡}{jian3}[][HSK 6]
    \definition{v.}{coletar; reunir; apanhar; pegar coisas do chão}
  \end{Phonetics}
\end{Entry}

%%%%%%%%%% 换 %%%%%%%%%%
\subsection*{换}\addcontentsline{loh}{figure}{换}

\begin{Entry}{换}{10}{⼿}
  \begin{Phonetics}{换}{huan4}[][HSK 2]
    \definition{v.}{negociar; trocar; permutar; dar algo a alguém e, ao mesmo tempo, obter algo dele em troca | mudar; transformar; substituir | trocar dinheiro (câmbio) | transfundir (sangue) | transplantar (um órgão)}
  \end{Phonetics}
\end{Entry}

\begin{Entry}{换成}{10,6}{⼿,⼽}
  \begin{Phonetics}{换成}{huan4cheng2}[][HSK 7-9]
    \definition{v.}{trocar (algo) por (outro); indica a substituição de um objeto, estado ou situação por outro}
  \end{Phonetics}
\end{Entry}

\begin{Entry}{换位}{10,7}{⼿,⼈}
  \begin{Phonetics}{换位}{huan4wei4}[][HSK 7-9]
    \definition{v.}{trocar posições; transpor | mudar de posição}
  \end{Phonetics}
\end{Entry}

\begin{Entry}{换言之}{10,7,3}{⼿,⾔,⼂}
  \begin{Phonetics}{换言之}{huan4yan2zhi1}[][HSK 7-9]
    \definition{adv.}{em outras palavras}
  \end{Phonetics}
\end{Entry}

\begin{Entry}{换取}{10,8}{⼿,⼜}
  \begin{Phonetics}{换取}{huan4qu3}[][HSK 7-9]
    \definition{v.}{trocar (ou escambo) algo por; obter em troca | trocar algo por; obter por troca}
  \end{Phonetics}
\end{Entry}

\begin{Entry}{换钱}{10,10}{⼿,⾦}
  \begin{Phonetics}{换钱}{huan4/qian2}
    \definition{v.+compl.}{trocar dinheiro (em pequenas valores ou em outra moeda) | trocar (mercadorias) por dinheiro | vender}
  \end{Phonetics}
\end{Entry}

%%%%%%%%%% 捣 %%%%%%%%%%
\subsection*{捣}\addcontentsline{loh}{figure}{捣}

\begin{Entry}{捣}{10}{⼿}
  \begin{Phonetics}{捣}{dao3}
    \definition{v.}{bater com um pilão, etc.; bater; esmagar | assediar; perturbar | bater com um pedaço de pau}
  \end{Phonetics}
\end{Entry}

\begin{Entry}{捣乱}{10,7}{⼿,⼄}
  \begin{Phonetics}{捣乱}{dao3/luan4}[][HSK 7-9]
    \definition{v.+compl.}{causar problemas; criar uma perturbação; causar intencionalmente problemas para os outros; interromper | perturbar; interferir com; causar problemas intencionalmente}
  \end{Phonetics}
\end{Entry}

%%%%%%%%%% 捧 %%%%%%%%%%
\subsection*{捧}\addcontentsline{loh}{figure}{捧}

\begin{Entry}{捧}{11}{⼿}
  \begin{Phonetics}{捧}{peng3}[][HSK 7-9]
    \definition{clas.}{utilizado para coisas que podem ser seguradas}
    \definition{v.}{segurar ou carregar com ambas as mãos; apoiar com ambas as mãos | impulsionar; elogiar; exaltar; lisonjear; vangloriar}
  \end{Phonetics}
\end{Entry}

\begin{Entry}{捧场}{11,6}{⼿,⼟}
  \begin{Phonetics}{捧场}{peng3/chang3}[][HSK 7-9]
    \definition{v.+compl.}{aplaudir; comparecer e apoiar (uma reunião, apresentação, etc.) | promover; elogiar; bajular | ser membro de uma claque | ser membro de um grupo de apoio; elogiar profusamente; prestar homenagem pública a alguém; promover alguém no programa; gabar-se para os outros}
  \end{Phonetics}
\end{Entry}

%%%%%%%%%% 据 %%%%%%%%%%
\subsection*{据}\addcontentsline{loh}{figure}{据}

\begin{Entry}{据}{11}{⼿}
  \begin{Phonetics}{据}{ju1}
    \definition{part.}{elemento formador de palavras}
  \seealsoref{拮据}{jie2ju1}
  \end{Phonetics}
  \begin{Phonetics}{据}{ju4}[][HSK 6]
    \definition*{s.}{Sobrenome: Ju}
    \definition{prep.}{de acordo com; com base em}
    \definition{s.}{evidência; certificado; prova}
    \definition{v.}{ocupar; apreender | confiar em; depender de}
  \end{Phonetics}
\end{Entry}

\begin{Entry}{据此}{11,6}{⼿,⽌}
  \begin{Phonetics}{据此}{ju4ci3}[][HSK 7-9]
    \definition{v.}{se basear nesses fundamentos; ter em vista o exposto acima; fazer algo em conformidade; se basear nas circunstâncias ou razões já mencionadas}
  \end{Phonetics}
\end{Entry}

\begin{Entry}{据说}{11,9}{⼿,⾔}
  \begin{Phonetics}{据说}{ju4shuo1}[][HSK 3]
    \definition{v.}{é o que dizem; é o que se diz}
  \end{Phonetics}
\end{Entry}

\begin{Entry}{据悉}{11,11}{⼿,⼼}
  \begin{Phonetics}{据悉}{ju4xi1}[][HSK 7-9]
    \definition{adv.}{é relatado (que); de acordo com o que aprendi}
  \end{Phonetics}
\end{Entry}

%%%%%%%%%% 捶 %%%%%%%%%%
\subsection*{捶}\addcontentsline{loh}{figure}{捶}

\begin{Entry}{捶}{11}{⼿}
  \begin{Phonetics}{捶}{chui2}[][HSK 7-9]
    \definition{v.}{bater (com um pedaço de pau, martelo ou punho)}
  \end{Phonetics}
\end{Entry}

\begin{Entry}{捶子}{11,3}{⼿,⼦}
  \begin{Phonetics}{捶子}{chui2zi5}[][HSK 7-9]
    \definition[把]{s.}{martelo}
  \end{Phonetics}
\end{Entry}

%%%%%%%%%% 捷 %%%%%%%%%%
\subsection*{捷}\addcontentsline{loh}{figure}{捷}

\begin{Entry}{捷}{11}{⼿}
  \begin{Phonetics}{捷}{jie2}
    \definition*{s.}{Sobrenome: Jie}
    \definition{adj.}{rápido; ágil}
    \definition{s.}{vitória; triunfo; sucesso}
  \end{Phonetics}
\end{Entry}

\begin{Entry}{捷径}{11,8}{⼿,⼻}
  \begin{Phonetics}{捷径}{jie2jing4}
    \definition{s.}{atalho}
  \end{Phonetics}
\end{Entry}

%%%%%%%%%% 掉 %%%%%%%%%%
\subsection*{掉}\addcontentsline{loh}{figure}{掉}

\begin{Entry}{掉}{11}{⼿}
  \begin{Phonetics}{掉}{diao4}[][HSK 2]
    \definition{v.}{cair; soltar-se; desprender-se | ficar para trás | perder; desaparecer; omitir | diminuir; reduzir | balançar; abanar; oscilar | virar; voltar; retornar | alterar; trocar; intercambiar}
    \definition{v.aux.}{usado após certos verbos para indicar a conclusão de uma ação}
  \end{Phonetics}
\end{Entry}

\begin{Entry}{掉队}{11,4}{⼿,⾩}
  \begin{Phonetics}{掉队}{diao4/dui4}[][HSK 7-9]
    \definition{v.+compl.}{abandonar (ou sair); ficar para trás; cair fora}
  \end{Phonetics}
\end{Entry}

\begin{Entry}{掉包}{11,5}{⼿,⼓}
  \begin{Phonetics}{掉包}{diao4bao1}
    \definition{v.}{vender uma falsificação pelo artigo genuíno | roubar o item valioso de alguém e substituí-lo por um item de aparência semelhante, mas sem valor}
  \end{Phonetics}
\end{Entry}

\begin{Entry}{掉头}{11,5}{⼿,⼤}
  \begin{Phonetics}{掉头}{diao4/tou2}[][HSK 7-9]
    \definition{v.+compl.}{virar-se; afastar-se (das pessoas) | dar meia-volta (carro, barco, etc.); (carro, barco, etc.) virar na direção oposta}
  \end{Phonetics}
\end{Entry}

\begin{Entry}{掉线}{11,8}{⼿,⽷}
  \begin{Phonetics}{掉线}{diao4xian4}
    \definition{v.}{desconectar-se (da \emph{Internet})}
  \end{Phonetics}
\end{Entry}

\begin{Entry}{掉转}{11,8}{⼿,⾞}
  \begin{Phonetics}{掉转}{diao4zhuan3}
    \definition{v.}{dar a volta}
  \end{Phonetics}
\end{Entry}

\begin{Entry}{掉落}{11,12}{⼿,⾋}
  \begin{Phonetics}{掉落}{diao4luo4}
    \definition{v.}{derrubar}
  \end{Phonetics}
\end{Entry}

\begin{Entry}{掉膘}{11,15}{⼿,⾁}
  \begin{Phonetics}{掉膘}{diao4biao1}
    \definition{v.}{perder peso (gado)}
  \end{Phonetics}
\end{Entry}

%%%%%%%%%% 掏 %%%%%%%%%%
\subsection*{掏}\addcontentsline{loh}{figure}{掏}

\begin{Entry}{掏}{11}{⼿}
  \begin{Phonetics}{掏}{tao1}[][HSK 6]
    \definition{v.}{extrair; retirar; pescar | cavar (um buraco, etc.); escavar; retirar | (coloquial) roubar do bolso de alguém | tirar}
  \end{Phonetics}
\end{Entry}

%%%%%%%%%% 掐 %%%%%%%%%%
\subsection*{掐}\addcontentsline{loh}{figure}{掐}

\begin{Entry}{掐}{11}{⼿}
  \begin{Phonetics}{掐}{qia1}[][HSK 7-9]
    \definition{s.}{Dialeto: um punhado, maço, pitada, etc. de}
    \definition{v.}{beliscar; dar uma mordidinha | agarrar}
  \seealsoref{掐儿}{qia1r5}
  \end{Phonetics}
\end{Entry}

\begin{Entry}{掐儿}{11,2}{⼿,⼉}
  \begin{Phonetics}{掐儿}{qia1r5}
    \definition{s.}{Dialeto: um punhado, maço, pitada, etc. de}
  \end{Phonetics}
\end{Entry}

%%%%%%%%%% 排 %%%%%%%%%%
\subsection*{排}\addcontentsline{loh}{figure}{排}

\begin{Entry}{排}{11}{⼿}
  \begin{Phonetics}{排}{pai2}[][HSK 2,3]
    \definition{clas.}{usado para linhas, filas; coisas usadas para formar filas}
    \definition{s.}{linha; fileira; fileiras horizontais | pelotão; unidade militar, abaixo do nível de companhia, acima do nível de pelotão | jangada; balsa; um meio de transporte aquático feito de bambu e madeira unidos lado a lado; também se refere a bambu e madeira amarrados em fileiras para facilitar o transporte aquático | torta; bolo de carne; bolinho assado; comida cozida no vapor}
    \definition{v.}{organizar; alinhar; colocar em ordem; posicionar ou organizar em uma determinada ordem; ordenar | ensaiar | ejetar; excluir; dispensar; remover; eliminar | empurrar o obstáculo para fora do caminho}
  \end{Phonetics}
\end{Entry}

\begin{Entry}{排水}{11,4}{⼿,⽔}
  \begin{Phonetics}{排水}{pai2shui3}
    \definition{v.}{drenar}
  \end{Phonetics}
\end{Entry}

\begin{Entry}{排队}{11,4}{⼿,⾩}
  \begin{Phonetics}{排队}{pai2/dui4}[][HSK 2]
    \definition{v.+compl.}{formar uma fila; alinhar-se; enfileirar-se; organizar em sequência | listar; classificar}
  \end{Phonetics}
\end{Entry}

\begin{Entry}{排斥}{11,5}{⼿,⽄}
  \begin{Phonetics}{排斥}{pai2chi4}[][HSK 7-9]
    \definition{v.}{repelir; rejeitar; excluir; fazer com que (uma pessoa ou coisa) se afaste do seu próprio grupo}
  \end{Phonetics}
\end{Entry}

\begin{Entry}{排列}{11,6}{⼿,⼑}
  \begin{Phonetics}{排列}{pai2lie4}[][HSK 4]
    \definition{v.}{classificar; colocar; variar; organizar; pôr em ordem}
  \end{Phonetics}
\end{Entry}

\begin{Entry}{排名}{11,6}{⼿,⼝}
  \begin{Phonetics}{排名}{pai2 ming2}[][HSK 3]
    \definition{s.}{classificação; resultado; organizado de acordo com determinados critérios}
  \end{Phonetics}
\end{Entry}

\begin{Entry}{排行榜}{11,6,14}{⼿,⾏,⽊}
  \begin{Phonetics}{排行榜}{pai2 hang2 bang3}[][HSK 6]
    \definition{s.}{lista; classificação; lista de classificação; (de registros) os gráficos; uma lista em uma determinada ordem publicada com base em certos resultados estatísticos}
  \end{Phonetics}
\end{Entry}

\begin{Entry}{排放}{11,8}{⼿,⽅}
  \begin{Phonetics}{排放}{pai2fang4}[][HSK 7-9]
    \definition{v.}{colocar (as coisas) em ordem adequada | emitir; descarregar (gases de escape, águas residuais, etc.); deixar sair; drenar}
  \end{Phonetics}
\end{Entry}

\begin{Entry}{排练}{11,8}{⼿,⽷}
  \begin{Phonetics}{排练}{pai2lian4}[][HSK 7-9]
    \definition{v.}{ensaiar; ensaiar ou praticar uma determinada cerimônia ou apresentação}
  \end{Phonetics}
\end{Entry}

\begin{Entry}{排挤}{11,9}{⼿,⼿}
  \begin{Phonetics}{排挤}{pai2ji3}
    \definition{v.}{ostracizar; afastar; expulsar; espremer; excluir; marginalizar; usar o poder ou os meios para fazer com que aqueles que lhe são desfavoráveis ​​percam seu status ou seus interesses}
  \end{Phonetics}
\end{Entry}

\begin{Entry}{排除}{11,9}{⼿,⾩}
  \begin{Phonetics}{排除}{pai2chu2}[][HSK 5]
    \definition{v.}{remover; superar; excluir; eliminar; livrar-se de}
  \end{Phonetics}
\end{Entry}

\begin{Entry}{排球}{11,11}{⼿,⽟}
  \begin{Phonetics}{排球}{pai2 qiu2}[][HSK 2]
    \definition[场,只,个]{s.}{voleibol; bola de voleibol}
  \end{Phonetics}
\end{Entry}

%%%%%%%%%% 掠 %%%%%%%%%%
\subsection*{掠}\addcontentsline{loh}{figure}{掠}

\begin{Entry}{掠}{11}{⼿}
  \begin{Phonetics}{掠}{lve3}
    \definition{v.}{agarrar; tomar | copiar casualmente; copiar}
  \end{Phonetics}
  \begin{Phonetics}{掠}{lve4}
    \definition{v.}{pilhar; saquear; roubar | passar por cima; roçar; raspar; deslizar sobre; passar rapidamente; limpar ou escovar suavemente | Literário: bater ou açoitar (com um bastão ou chicote)}
  \end{Phonetics}
\end{Entry}

\begin{Entry}{掠夺}{11,6}{⼿,⼤}
  \begin{Phonetics}{掠夺}{lve4duo2}[][HSK 7-9]
    \definition{v.}{roubar; pilhar; saquear; usar força ou violência para roubar coisas}
  \end{Phonetics}
\end{Entry}

%%%%%%%%%% 探 %%%%%%%%%%
\subsection*{探}\addcontentsline{loh}{figure}{探}

\begin{Entry}{探}{11}{⼿}
  \begin{Phonetics}{探}{tan4}
    \definition[个,位,名]{s.}{batedor; espião; detetive}
    \definition{v.}{tentar descobrir; explorar; soar | explorar; espionar | visitar; fazer uma visita em | se destacar | preocupar-se com; envolver-se em | ver; invocar}
  \end{Phonetics}
\end{Entry}

\begin{Entry}{探讨}{11,5}{⼿,⾔}
  \begin{Phonetics}{探讨}{tan4tao3}[][HSK 6]
    \definition{v.}{examinar; indagar; investigar; discutir}
  \end{Phonetics}
\end{Entry}

\begin{Entry}{探亲}{11,9}{⼿,⼇}
  \begin{Phonetics}{探亲}{tan4/qin1}
    \definition{v.+compl.}{ir para casa para visitar a família}
  \end{Phonetics}
\end{Entry}

\begin{Entry}{探索}{11,10}{⼿,⽷}
  \begin{Phonetics}{探索}{tan4suo3}[][HSK 6]
    \definition{v.}{sondar; explorar; procurar respostas de várias fontes para resolver dúvidas}
  \end{Phonetics}
\end{Entry}

%%%%%%%%%% 接 %%%%%%%%%%
\subsection*{接}\addcontentsline{loh}{figure}{接}

\begin{Entry}{接}{11}{⼿}
  \begin{Phonetics}{接}{jie1}[][HSK 2]
    \definition*{s.}{Sobrenome: Jie}
    \definition{v.}{entrar em contato com; aproximar-se de | conectar; unir; juntar | continuar; prosseguir | assumir o controle; assumir o trabalho de outra pessoa e continuar a fazê-lo | pegar; agarrar; segurar ou sustentar com as mãos | receber; aceitar | encontrar; dar as boas-vindas}
  \end{Phonetics}
\end{Entry}

\begin{Entry}{接二连三}{11,2,7,3}{⼿,⼆,⾡,⼀}
  \begin{Phonetics}{接二连三}{jie1'er4-lian2san1}[][HSK 7-9]
    \definition{expr.}{um após o outro; em rápida sucessão}
  \end{Phonetics}
\end{Entry}

\begin{Entry}{接力}{11,2}{⼿,⼒}
  \begin{Phonetics}{接力}{jie1li4}[][HSK 7-9]
    \definition{s.}{relé; trabalho por revezamento; revezamento}
  \end{Phonetics}
\end{Entry}

\begin{Entry}{接下来}{11,3,7}{⼿,⼀,⽊}
  \begin{Phonetics}{接下来}{jie1 xia4 lai2}[][HSK 2]
    \definition{expr.}{próximo; seguinte; indica uma sequência temporal subsequente}
  \end{Phonetics}
\end{Entry}

\begin{Entry}{接手}{11,4}{⼿,⼿}
  \begin{Phonetics}{接手}{jie1shou3}[][HSK 7-9]
    \definition{v.}{assumir (responsabilidades, etc.); assumir problemas; assumir o trabalho de outra pessoa.}
  \end{Phonetics}
\end{Entry}

\begin{Entry}{接见}{11,4}{⼿,⾒}
  \begin{Phonetics}{接见}{jie1jian4}[][HSK 7-9]
    \definition{v.}{receber alguém; conceder uma entrevista a; reunir-se com as pessoas que vieram}
  \end{Phonetics}
\end{Entry}

\begin{Entry}{接(电话)}{11,5,8}{⼿,⽥,⾔}
  \begin{Phonetics}{接(电话)}{jie1(dian4hua4)}
    \definition{v.}{atender (o telefone) | receber (uma ligação telefônica)}
  \end{Phonetics}
\end{Entry}

\begin{Entry}{接收}{11,6}{⼿,⽁}
  \begin{Phonetics}{接收}{jie1 shou1}[][HSK 6]
    \definition{v.}{aceitar; receber | assumir; expropriar; tomar posse (de uma instituição, propriedade, etc.) de acordo com a lei | admitir; aceitar; absorver}
  \end{Phonetics}
\end{Entry}

\begin{Entry}{接轨}{11,6}{⼿,⾞}
  \begin{Phonetics}{接轨}{jie1/gui3}[][HSK 7-9]
    \definition{s.}{junção; integração; ligação}
    \definition{v.+compl.}{ligar; juntar; conectar os trilhos | integrar; juntar-se a; mudar para; entrar na onda; alinhar-se; alinhar a; essa metáfora descreve como sistemas e métodos podem ser interconectados e consistentes}
  \end{Phonetics}
\end{Entry}

\begin{Entry}{接听}{11,7}{⼿,⼝}
  \begin{Phonetics}{接听}{jie1ting1}[][HSK 7-9]
    \definition{v.}{atender (o telefone)}
  \end{Phonetics}
\end{Entry}

\begin{Entry}{接纳}{11,7}{⼿,⽷}
  \begin{Phonetics}{接纳}{jie1na4}[][HSK 7-9]
    \definition{v.}{ser admitido (em uma organização); aceitar (como membro); incluir (indivíduos ou grupos que ingressam na organização) | adotar; aceitar; tomar}
  \end{Phonetics}
\end{Entry}

\begin{Entry}{接近}{11,7}{⼿,⾡}
  \begin{Phonetics}{接近}{jie1jin4}[][HSK 3]
    \definition{adj.}{perto; próximo; a diferença entre os dois é mínima}
    \definition{v.}{estar perto de; aproximar; aproximar-se}
  \end{Phonetics}
\end{Entry}

\begin{Entry}{接连}{11,7}{⼿,⾡}
  \begin{Phonetics}{接连}{jie1lian2}[][HSK 5]
    \definition{adv.}{no final; em sucessão; em uma fileira; um após o outro; seguindo o anterior; continuando}
  \end{Phonetics}
\end{Entry}

\begin{Entry}{接到}{11,8}{⼿,⼑}
  \begin{Phonetics}{接到}{jie1 dao4}[][HSK 2]
    \definition{v.}{receber (carta, etc.)}
  \end{Phonetics}
\end{Entry}

\begin{Entry}{接受}{11,8}{⼿,⼜}
  \begin{Phonetics}{接受}{jie1shou4}[][HSK 2]
    \definition{v.}{aceitar; não recusar (o que os outros oferecem) | concordar; não recusar (opiniões/sugestões/críticas/convites de outras pessoas, etc.)}
  \end{Phonetics}
\end{Entry}

\begin{Entry}{接待}{11,9}{⼿,⼻}
  \begin{Phonetics}{接待}{jie1dai4}[][HSK 3]
    \definition{v.}{receber (alguém); acolher; recepcionar; receber com cordialidade e generosidade}
  \end{Phonetics}
\end{Entry}

\begin{Entry}{接济}{11,9}{⼿,⽔}
  \begin{Phonetics}{接济}{jie1ji4}[][HSK 7-9]
    \definition{v.}{prestar assistência material a; dar ajuda financeira a; prestar auxílio material a}
  \end{Phonetics}
\end{Entry}

\begin{Entry}{接送}{11,9}{⼿,⾡}
  \begin{Phonetics}{接送}{jie1song4}[][HSK 7-9]
    \definition{v.}{buscar e levar}
  \end{Phonetics}
\end{Entry}

\begin{Entry}{接班}{11,10}{⼿,⽟}
  \begin{Phonetics}{接班}{jie1/ban1}[][HSK 7-9]
    \definition{v.}{assumir o turno de alguém; substituir alguém; assumir o lugar de; dar continuidade a; (sucessor) Assumir o trabalho do turno anterior | ter sucesso; dar continuidade a algo iniciado por seu antecessor}
  \seealsoref{接班儿}{jie1ban1r5}
  \end{Phonetics}
\end{Entry}

\begin{Entry}{接班人}{11,10,2}{⼿,⽟,⼈}
  \begin{Phonetics}{接班人}{jie1ban1ren2}[][HSK 7-9]
    \definition{s.}{sucessor; a pessoa que assume o trabalho do turno anterior é frequentemente usada metaforicamente}
  \end{Phonetics}
\end{Entry}

\begin{Entry}{接班儿}{11,10,2}{⼿,⽟,⼉}
  \begin{Phonetics}{接班儿}{jie1ban1r5}
    \definition{v.}{assumir o turno de alguém; substituir alguém | ter sucesso; dar continuidade a algo iniciado por seu antecessor}
  \seealsoref{接班}{jie1/ban1}
  \end{Phonetics}
\end{Entry}

\begin{Entry}{接通}{11,10}{⼿,⾡}
  \begin{Phonetics}{接通}{jie1tong1}[][HSK 7-9]
    \definition{v.}{transmitir; fazer ligação telefônica | conectar; completar a ligação; conseguir passar | fechar; encerrar; interromper; inserir; ativar; ligar; completar}
  \end{Phonetics}
\end{Entry}

\begin{Entry}{接着}{11,11}{⼿,⽬}
  \begin{Phonetics}{接着}{jie1zhe5}[][HSK 2]
    \definition{adv.}{por sua vez; um após o outro; sucessivamente; conectado (à frase anterior); imediatamente após (a ação anterior)}
    \definition{v.}{seguir; prosseguir; continuar; seguir em frente; ficar ao lado | pegar com as mãos; apanhar}
  \end{Phonetics}
\end{Entry}

\begin{Entry}{接替}{11,12}{⼿,⽈}
  \begin{Phonetics}{接替}{jie1ti4}[][HSK 7-9]
    \definition{v.}{assumir o controle; substituir | suceder; ocupar o lugar de}
  \end{Phonetics}
\end{Entry}

\begin{Entry}{接触}{11,13}{⼿,⾓}
  \begin{Phonetics}{接触}{jie1chu4}[][HSK 5]
    \definition{v.}{entrar em contato com | entrar em contato; tocar; interagir | engajar; o termo militar refere-se a fogo cruzado}
  \end{Phonetics}
\end{Entry}

%%%%%%%%%% 控 %%%%%%%%%%
\subsection*{控}\addcontentsline{loh}{figure}{控}

\begin{Entry}{控}{11}{⼿}
  \begin{Phonetics}{控}{kong4}
    \definition{v.}{acusar; cobrar | controlar; dominar | manter (parte do corpo em uma determinada posição) sem apoio | virar (um recipiente) de cabeça para baixo para deixar o líquido escorrer}
  \end{Phonetics}
\end{Entry}

\begin{Entry}{控告}{11,7}{⼿,⼝}
  \begin{Phonetics}{控告}{kong4gao4}[][HSK 7-9]
    \definition{v.}{acusar; denunciar; incriminar; indiciar; processar alguém; apresentar uma queixa legal contra alguém}
  \end{Phonetics}
\end{Entry}

\begin{Entry}{控制}{11,8}{⼿,⼑}
  \begin{Phonetics}{控制}{kong4zhi4}[][HSK 5]
    \definition{v.}{controlar; restringir; dominar; fazer com que não ultrapasse um determinado limite | controlar; dominar; comandar; ocupar, fazer com que não se perca}
  \end{Phonetics}
\end{Entry}

%%%%%%%%%% 推 %%%%%%%%%%
\subsection*{推}\addcontentsline{loh}{figure}{推}

\begin{Entry}{推}{11}{⼿}
  \begin{Phonetics}{推}{tui1}[][HSK 2]
    \definition{v.}{empurrar; dar um encontrão | girar um moinho ou uma pedra de amolar; moer | cortar; aparar | impulsionar; promover; avançar | inferir; deduzir | afastar; fugir; deslocar | adiar | eleger; escolher | ter em alta estima; elogiar muito | declinar | selecionar | elogiar muito}
  \end{Phonetics}
\end{Entry}

\begin{Entry}{推广}{11,3}{⼿,⼴}
  \begin{Phonetics}{推广}{tui1guang3}[][HSK 3]
    \definition{v.}{espalhar; estender; promover; popularizar; expandir o escopo de uso ou função de algo}
  \end{Phonetics}
\end{Entry}

\begin{Entry}{推介}{11,4}{⼿,⼈}
  \begin{Phonetics}{推介}{tui1jie4}
    \definition{s.}{promoção}
    \definition{v.}{promover | introduzir e recomendar}
  \end{Phonetics}
\end{Entry}

\begin{Entry}{推开}{11,4}{⼿,⼶}
  \begin{Phonetics}{推开}{tui1 kai1}[][HSK 3]
    \definition{v.}{declinar; rejeitar | empurrar para longe; aplicar força em uma determinada direção para mover uma pessoa ou objeto para longe de seu lugar original | empurrar para abrir (um portão, etc.); empurrar para fora para abrir algo que está fechado | estender; popularizar; promover para um alcance mais amplo e realizar em uma escala mais ampla}
  \end{Phonetics}
\end{Entry}

\begin{Entry}{推出}{11,5}{⼿,⼐}
  \begin{Phonetics}{推出}{tui1 chu1}[][HSK 6]
    \definition{v.}{lançar; apresentar; fazer com que apareça diante do público | deduzir; tirar conclusões da análise}
  \end{Phonetics}
\end{Entry}

\begin{Entry}{推动}{11,6}{⼿,⼒}
  \begin{Phonetics}{推动}{tui1 dong4}[][HSK 3]
    \definition{v.}{promover; atuar; impulsionar; empurrar para a frente; dar ímpeto a; começar ou avançar algo (com alguma força); começar a trabalhar}
  \end{Phonetics}
\end{Entry}

\begin{Entry}{推行}{11,6}{⼿,⾏}
  \begin{Phonetics}{推行}{tui1 xing2}[][HSK 5]
    \definition{v.}{realizar; prosseguir; praticar | implementar; praticar; implementação generalizada; divulgar (experiências, métodos, etc.)}
  \end{Phonetics}
\end{Entry}

\begin{Entry}{推进}{11,7}{⼿,⾡}
  \begin{Phonetics}{推进}{tui1 jin4}[][HSK 3]
    \definition{v.}{avançar; empurrar; levar adiante; dar ímpeto a; promover o trabalho e fazê-lo avançar | empurrar; dirigir; avançar; seguir em frente; seguir em frente}
  \end{Phonetics}
\end{Entry}

\begin{Entry}{推迟}{11,7}{⼿,⾡}
  \begin{Phonetics}{推迟}{tui1chi2}[][HSK 4]
    \definition{v.}{adiar; postergar; tardar; deixar para mais tarde}
  \end{Phonetics}
\end{Entry}

\begin{Entry}{推销}{11,12}{⼿,⾦}
  \begin{Phonetics}{推销}{tui1xiao1}[][HSK 4]
    \definition{v.}{vender; comercializar; promover vendas; promover a comercialização de mercadorias}
  \end{Phonetics}
\end{Entry}

%%%%%%%%%% 措 %%%%%%%%%%
\subsection*{措}\addcontentsline{loh}{figure}{措}

\begin{Entry}{措}{11}{⼿}
  \begin{Phonetics}{措}{cuo4}
    \definition{s.}{iniciativa; solução; medida}
    \definition{v.}{organizar; gerenciar; lidar | fazer planos; administrar; organizar}
  \end{Phonetics}
\end{Entry}

\begin{Entry}{措手不及}{11,4,4,3}{⼿,⼿,⼀,⼃}
  \begin{Phonetics}{措手不及}{cuo4shou3-bu4ji2}[][HSK 7-9]
    \definition{expr.}{ser pego de surpresa; ser pego de surpresa (despreparado); ser tarde demais para fazer algo a respeito; ficar surpreso demais para se defender; não conseguir fazer uma defesa adequada; não conseguir pensar a tempo em uma maneira de se defender; não ter tempo para colocar em prática; surpreender alguém; pegar alguém desprevenido | ser pego desprevenido; ser pego de surpresa}
  \end{Phonetics}
\end{Entry}

\begin{Entry}{措施}{11,9}{⼿,⽅}
  \begin{Phonetics}{措施}{cuo4shi1}[][HSK 4]
    \definition[项,个]{s.}{medida; etapa; passo; abordagem adotada para lidar com as coisas}
  \end{Phonetics}
\end{Entry}

%%%%%%%%%% 掺 %%%%%%%%%%
\subsection*{掺}\addcontentsline{loh}{figure}{掺}

\begin{Entry}{掺}{11}{⼿}
  \begin{Phonetics}{掺}{can4}
    \definition{s.}{um estilo antigo de tocar bateria; uma antiga canção de tambor}
  \end{Phonetics}
  \begin{Phonetics}{掺}{chan1}[][HSK 7-9]
    \definition{v.}{misturar; mesclar; adicionar}
  \end{Phonetics}
  \begin{Phonetics}{掺}{shan3}
    \definition{v.}{misturar; mesclar | conter; reter}
  \end{Phonetics}
\end{Entry}

%%%%%%%%%% 描 %%%%%%%%%%
\subsection*{描}\addcontentsline{loh}{figure}{描}

\begin{Entry}{描}{11}{⼿}
  \begin{Phonetics}{描}{miao2}
    \definition{v.}{traçar; copiar | retocar; retocar | traçar um desenho | retratar | esboçar}
  \end{Phonetics}
\end{Entry}

\begin{Entry}{描写}{11,5}{⼿,⼍}
  \begin{Phonetics}{描写}{miao2xie3}[][HSK 4]
    \definition{v.}{representar; retratar; descrever; usar a linguagem e as palavras para transmitir uma imagem concreta de uma pessoa, evento ou situação}
  \end{Phonetics}
\end{Entry}

\begin{Entry}{描述}{11,8}{⼿,⾡}
  \begin{Phonetics}{描述}{miao2 shu4}[][HSK 4]
    \definition[段,种]{s.}{descrição; trecho que descreve um evento ou uma cena}
    \definition{v.}{descrever; representar}
  \end{Phonetics}
\end{Entry}

\begin{Entry}{描绘}{11,9}{⼿,⽷}
  \begin{Phonetics}{描绘}{miao2hui4}[][HSK 7-9]
    \definition{v.}{descrever; retratar; representar; desenhar}
  \end{Phonetics}
\end{Entry}

%%%%%%%%%% 掌 %%%%%%%%%%
\subsection*{掌}\addcontentsline{loh}{figure}{掌}

\begin{Entry}{掌}{12}{⼿}
  \begin{Phonetics}{掌}{zhang3}
    \definition{s.}{palma da mão | sola do pé | pata | ferradura}
    \definition{v.}{dar um tapa | segurar na mão | empunhar}
  \end{Phonetics}
\end{Entry}

\begin{Entry}{掌声}{12,7}{⼿,⼠}
  \begin{Phonetics}{掌声}{zhang3 sheng1}[][HSK 6]
    \definition[阵]{s.}{aplausos; palmas; o som dos aplausos}
  \end{Phonetics}
\end{Entry}

\begin{Entry}{掌握}{12,12}{⼿,⼿}
  \begin{Phonetics}{掌握}{zhang3wo4}[][HSK 5]
    \definition{v.}{compreender; dominar; conhecer bem; compreender as coisas; ser capaz de dominar ou utilizar plenamente | segurar; controlar; ter em mãos; tomar nas mãos}
  \end{Phonetics}
\end{Entry}

%%%%%%%%%% 掰 %%%%%%%%%%
\subsection*{掰}\addcontentsline{loh}{figure}{掰}

\begin{Entry}{掰}{12}{⼿}
  \begin{Phonetics}{掰}{bai1}[][HSK 7-9]
    \definition{v.}{separar ou quebrar coisas com as mãos | Dialeto: romper (relacionamento); cortar | Dialeto: analisar; estudar; examinar}
  \end{Phonetics}
\end{Entry}

%%%%%%%%%% 掱 %%%%%%%%%%
\subsection*{掱}\addcontentsline{loh}{figure}{掱}

\begin{Entry}{掱}{12}{⼿}
  \begin{Phonetics}{掱}{shou3}
    \variantof{手}
  \end{Phonetics}
\end{Entry}

%%%%%%%%%% 揉 %%%%%%%%%%
\subsection*{揉}\addcontentsline{loh}{figure}{揉}

\begin{Entry}{揉}{12}{⼿}
  \begin{Phonetics}{揉}{rou2}[][HSK 7-9]
    \definition{v.}{esfregar; esfregar ou friccionar com as mãos | amassar; enrolar | Literário: dobrar; torcer}
  \end{Phonetics}
\end{Entry}

\begin{Entry}{揉碎}{12,13}{⼿,⽯}
  \begin{Phonetics}{揉碎}{rou2sui4}
    \definition{v.}{desfazer-se em pedaços | esmagar}
  \end{Phonetics}
\end{Entry}

%%%%%%%%%% 提 %%%%%%%%%%
\subsection*{提}\addcontentsline{loh}{figure}{提}

\begin{Entry}{提}{12}{⼿}
  \begin{Phonetics}{提}{ti2}[][HSK 2]
    \definition*{s.}{Sobrenome: Ti}
    \definition{s.}{concha; utensílio para servir óleo ou vinho | traço ascendente (em caracteres chineses)}
    \definition{v.}{carregar (na mão, com o braço para baixo) ; segurar com as mãos para baixo | elevar; levantar; promover | avançar; antecipar uma data; mudar para uma data anterior; adiar o prazo previsto | levantar; apresentar; indicar ou citar | extrair; retirar (tirar) | (prisioneiros) trazer; entregar | mencionar; referir-se a; abordar}
  \end{Phonetics}
\end{Entry}

\begin{Entry}{提及}{12,3}{⼿,⼃}
  \begin{Phonetics}{提及}{ti2ji2}
    \definition{v.}{mencionar | levantar (um assunto) | chamar a atenção de alguém}
  \end{Phonetics}
\end{Entry}

\begin{Entry}{提升}{12,4}{⼿,⼗}
  \begin{Phonetics}{提升}{ti2 sheng1}[][HSK 6]
    \definition{v.}{promover; avançar; melhorar (posição, grau, qualidade, etc.) | içar; elevar; transportar (minerais, materiais, etc.) para um local mais alto usando um guincho, etc.}
  \end{Phonetics}
\end{Entry}

\begin{Entry}{提出}{12,5}{⼿,⼐}
  \begin{Phonetics}{提出}{ti2 chu1}[][HSK 2]
    \definition{v.}{levantar; propor; apresentar; expressar seus desejos, ideias, sugestões, etc. por meio de palavras ou textos}
  \end{Phonetics}
\end{Entry}

\begin{Entry}{提示}{12,5}{⼿,⽰}
  \begin{Phonetics}{提示}{ti2shi4}[][HSK 5]
    \definition[个]{s.}{dica; lembrete; pistas ou informações fornecidas para chamar a atenção, fazer com que a outra pessoa pense ou compreenda}
    \definition{v.}{solicitar; lembrar; indicar; alertar; levantar questões que o outro não tenha pensado ou não tenha imaginado, para chamar a atenção dele}
  \end{Phonetics}
\end{Entry}

\begin{Entry}{提交}{12,6}{⼿,⼇}
  \begin{Phonetics}{提交}{ti2 jiao1}[][HSK 6]
    \definition{v.}{referir-se a; submeter (um problema, etc.) a; enviar questões que precisam ser discutidas, decididas ou tratadas para agências ou reuniões relevantes}
  \end{Phonetics}
\end{Entry}

\begin{Entry}{提问}{12,6}{⼿,⾨}
  \begin{Phonetics}{提问}{ti2wen4}[][HSK 3]
    \definition{v.}{\emph{quiz}; fazer uma pergunta; colocar questões para}
  \end{Phonetics}
\end{Entry}

\begin{Entry}{提防}{12,6}{⼿,⾩}
  \begin{Phonetics}{提防}{di1fang5}[][HSK 7-9]
    \definition{v.}{proteger-se contra; ter cuidado com; tomar precauções contra; tomar cuidado}
  \end{Phonetics}
\end{Entry}

\begin{Entry}{提供}{12,8}{⼿,⼈}
  \begin{Phonetics}{提供}{ti2gong1}[][HSK 4]
    \definition{v.}{oferecer; fornecer; suprir; prover; proporcionar}
  \end{Phonetics}
\end{Entry}

\begin{Entry}{提到}{12,8}{⼿,⼑}
  \begin{Phonetics}{提到}{ti2 dao4}[][HSK 2]
    \definition{v.}{mencionar; referir-se a; levantar (assunto)}
  \end{Phonetics}
\end{Entry}

\begin{Entry}{提前}{12,9}{⼿,⼑}
  \begin{Phonetics}{提前}{ti2qian2}[][HSK 3]
    \definition{adv.}{antecipadamente; faça uma coisa antes de fazer outra}
    \definition{v.}{avançar; adiantar; mudar para uma data anterior; trazer para frente}
  \end{Phonetics}
\end{Entry}

\begin{Entry}{提倡}{12,10}{⼿,⼈}
  \begin{Phonetics}{提倡}{ti2chang4}[][HSK 5]
    \definition{v.}{promover; incentivar; recomendar; apresentar as vantagens de algo para incentivar as pessoas a usá-lo ou implementá-lo}
  \end{Phonetics}
\end{Entry}

\begin{Entry}{提起}{12,10}{⼿,⾛}
  \begin{Phonetics}{提起}{ti2 qi3}[][HSK 5]
    \definition{v.}{mencionar; falar sobre; abordar | levantar; despertar; estimular; revigorar; alegrar/animar | iniciar; instituir; propor | levantar; pegar}
  \end{Phonetics}
\end{Entry}

\begin{Entry}{提高}{12,10}{⼿,⾼}
  \begin{Phonetics}{提高}{ti2gao1}[][HSK 2]
    \definition{v.}{elevar; aprimorar; aumentar; melhorar a posição, o nível, a quantidade, a qualidade e outros aspectos em relação ao estado original}
  \end{Phonetics}
\end{Entry}

\begin{Entry}{提醒}{12,16}{⼿,⾣}
  \begin{Phonetics}{提醒}{ti2/xing3}[][HSK 4]
    \definition{v.+compl.}{alertar; avisar; advertir; lembrar; apontar para ou chamar a atenção para}
  \end{Phonetics}
\end{Entry}

%%%%%%%%%% 插 %%%%%%%%%%
\subsection*{插}\addcontentsline{loh}{figure}{插}

\begin{Entry}{插}{12}{⼿}
  \begin{Phonetics}{插}{cha1}[][HSK 5]
    \definition{v.}{perfurar; inserir | interpor; inserir; colocar no meio}
  \end{Phonetics}
\end{Entry}

\begin{Entry}{插手}{12,4}{⼿,⼿}
  \begin{Phonetics}{插手}{cha1/shou3}[][HSK 7-9]
    \definition{v.+compl.}{participar; dar uma mão | meter a mão em; meter o nariz em; intrometer-se | ter (tomar) uma mão em}
  \end{Phonetics}
\end{Entry}

\begin{Entry}{插图}{12,8}{⼿,⼞}
  \begin{Phonetics}{插图}{cha1tu2}[][HSK 7-9]
    \definition[张,幅]{s.}{ilustração (artística ou científica) | ilustração; figura; mapa; demonstração; inserção}
  \end{Phonetics}
\end{Entry}

\begin{Entry}{插话}{12,8}{⼿,⾔}
  \begin{Phonetics}{插话}{cha1/hua4}
    \definition{s.}{interrupção | digressão}
    \definition{v.+compl.}{interromper (a fala de alguém)}
  \end{Phonetics}
\end{Entry}

\begin{Entry}{插嘴}{12,16}{⼿,⼝}
  \begin{Phonetics}{插嘴}{cha1/zui3}[][HSK 7-9]
    \definition{v.+compl.}{interromper; intrometer-se; participar da conversa (geralmente de forma inadequada)}
  \end{Phonetics}
\end{Entry}

%%%%%%%%%% 握 %%%%%%%%%%
\subsection*{握}\addcontentsline{loh}{figure}{握}

\begin{Entry}{握}{12}{⼿}
  \begin{Phonetics}{握}{wo4}[][HSK 5]
    \definition{v.}{segurar; agarrar | agarrar; segurar; empunhar; controlar | pegar pela mão}
  \end{Phonetics}
\end{Entry}

\begin{Entry}{握手}{12,4}{⼿,⼿}
  \begin{Phonetics}{握手}{wo4/shou3}[][HSK 3]
    \definition{v.+compl.}{apertar as mãos; dar um aperto de mão; estender a mão e apertar a mão do outro é uma forma de saudação ao se encontrar ou se despedir, e também é usado para expressar felicitações ou condolências}
  \end{Phonetics}
\end{Entry}

%%%%%%%%%% 揣 %%%%%%%%%%
\subsection*{揣}\addcontentsline{loh}{figure}{揣}

\begin{Entry}{揣}{12}{⼿}
  \begin{Phonetics}{揣}{chuai1}[][HSK 7-9]
    \definition{v.}{esconder (ou carregar) nas roupas | Dialeto: encher-se de comida; comer demais; encher alguém de comida; alimentar em excesso}
  \end{Phonetics}
  \begin{Phonetics}{揣}{chuai3}
    \definition*{s.}{Sobrenome: Chuai}
    \definition{v.}{contar; calcular; medir | estimar; palpitar; conjecturar}
  \end{Phonetics}
\end{Entry}

\begin{Entry}{揣测}{12,9}{⼿,⽔}
  \begin{Phonetics}{揣测}{chuai3ce4}[][HSK 7-9]
    \definition{v.}{adivinhar; conjecturar | supor; calcular; especular}
  \end{Phonetics}
\end{Entry}

\begin{Entry}{揣摩}{12,15}{⼿,⼿}
  \begin{Phonetics}{揣摩}{chuai3mo2}[][HSK 7-9]
    \definition{v.}{tentar compreender; tentar descobrir; obter algo por meio de estudo cuidadoso; pesar e considerar}
  \end{Phonetics}
\end{Entry}

%%%%%%%%%% 揪 %%%%%%%%%%
\subsection*{揪}\addcontentsline{loh}{figure}{揪}

\begin{Entry}{揪}{12}{⼿}
  \begin{Phonetics}{揪}{jiu1}[][HSK 7-9]
    \definition{v.}{segurar com firmeza; agarrar e puxar}
  \end{Phonetics}
\end{Entry}

%%%%%%%%%% 揭 %%%%%%%%%%
\subsection*{揭}\addcontentsline{loh}{figure}{揭}

\begin{Entry}{揭}{12}{⼿}
  \begin{Phonetics}{揭}{jie1}[][HSK 6]
    \definition*{s.}{Sobrenome: Jie}
    \definition{v.}{rasgar; arrancar; tirar | descobrir; levantar (a tampa, etc.) | expor; mostrar; trazer à luz | (literário) levantar; içar}
  \end{Phonetics}
\end{Entry}

\begin{Entry}{揭发}{12,5}{⼿,⼜}
  \begin{Phonetics}{揭发}{jie1fa1}[][HSK 7-9]
    \definition{v.}{expor; desmascarar; trazer à luz; expor e denunciar (pessoas más e más ações)}
  \end{Phonetics}
\end{Entry}

\begin{Entry}{揭示}{12,5}{⼿,⽰}
  \begin{Phonetics}{揭示}{jie1shi4}[][HSK 7-9]
    \definition{v.}{anunciar; promulgar; exibir publicamente | revelar; desvendar; trazer à luz; apontar ou esclarecer a essência de coisas que não são facilmente visíveis}
  \end{Phonetics}
\end{Entry}

\begin{Entry}{揭晓}{12,10}{⼿,⽇}
  \begin{Phonetics}{揭晓}{jie1xiao3}[][HSK 7-9]
    \definition{v.}{revelar; anunciar; tornar conhecido; divulgar publicamente os resultados da investigação para que todos fiquem cientes}
  \end{Phonetics}
\end{Entry}

\begin{Entry}{揭露}{12,21}{⼿,⾬}
  \begin{Phonetics}{揭露}{jie1lu4}[][HSK 7-9]
    \definition{v.}{expor; desmascarar; descobrir; revelar o que estava oculto}
  \end{Phonetics}
\end{Entry}

%%%%%%%%%% 援 %%%%%%%%%%
\subsection*{援}\addcontentsline{loh}{figure}{援}

\begin{Entry}{援}{12}{⼿}
  \begin{Phonetics}{援}{yuan2}
    \definition*{s.}{Sobrenome: Yuan}
    \definition{v.}{puxar com a mão; segurar | citar; referenciar | ajudar; auxiliar; resgatar}
  \end{Phonetics}
\end{Entry}

\begin{Entry}{援助}{12,7}{⼿,⼒}
  \begin{Phonetics}{援助}{yuan2 zhu4}[][HSK 6]
    \definition{s.}{ajuda; assistência; auxílio}
    \definition{v.}{ajudar; apoiar; auxiliar}
  \end{Phonetics}
\end{Entry}

%%%%%%%%%% 揽 %%%%%%%%%%
\subsection*{揽}\addcontentsline{loh}{figure}{揽}

\begin{Entry}{揽}{12}{⼿}
  \begin{Phonetics}{揽}{lan3}[][HSK 7-9]
    \definition{v.}{puxar (ou tomar) para os braços; puxar alguém para os braços; abraçar; envolver alguém em seus braços | prender com uma corda, etc.; reunir os materiais soltos com cordas ou meios semelhantes | assumir; tomar a iniciativa; fazer campanha | agarrar; monopolizar; agarre-se a; controlar}
  \end{Phonetics}
\end{Entry}

%%%%%%%%%% 搀 %%%%%%%%%%
\subsection*{搀}\addcontentsline{loh}{figure}{搀}

\begin{Entry}{搀}{12}{⼿}
  \begin{Phonetics}{搀}{chan1}[][HSK 7-9]
    \definition{v.}{apoiar alguém pelo braço; apoiar alguém com a mão; apoiar | misturar}
  \end{Phonetics}
\end{Entry}

%%%%%%%%%% 搁 %%%%%%%%%%
\subsection*{搁}\addcontentsline{loh}{figure}{搁}

\begin{Entry}{搁}{12}{⼿}
  \begin{Phonetics}{搁}{ge1}[][HSK 7-9]
    \definition{v.}{pôr; colocar | colocar à parte; deixar para trás; deixar para mais tarde| deixar de lado}
  \end{Phonetics}
  \begin{Phonetics}{搁}{ge2}
    \definition{v.}{suportar; resistir}
  \end{Phonetics}
\end{Entry}

\begin{Entry}{搁浅}{12,8}{⼿,⽔}
  \begin{Phonetics}{搁浅}{ge1/qian3}[][HSK 7-9]
    \definition{v.+compl.}{ficar encalhado (navio); encalhar | ser retido; chegar a um impasse; metaforicamente, algo está bloqueado e não pode prosseguir}
  \end{Phonetics}
\end{Entry}

\begin{Entry}{搁置}{12,13}{⼿,⽹}
  \begin{Phonetics}{搁置}{ge1zhi4}[][HSK 7-9]
    \definition{v.}{arquivar; deixar de lado; suspender; classificar; deitar; adiar; colocar na prateleira}
  \end{Phonetics}
\end{Entry}

%%%%%%%%%% 搂 %%%%%%%%%%
\subsection*{搂}\addcontentsline{loh}{figure}{搂}

\begin{Entry}{搂}{12}{⼿}
  \begin{Phonetics}{搂}{lou3}[][HSK 7-9]
    \definition{v.}{juntar; ajuntar; reunir | segurar; pegar; aconchegar; usar as mãos para juntar e levantar o objeto | fazer  (dinheiro); extorquir | puxar; virar | verificar; analisar}
  \end{Phonetics}
\end{Entry}

%%%%%%%%%% 搅 %%%%%%%%%%
\subsection*{搅}\addcontentsline{loh}{figure}{搅}

\begin{Entry}{搅}{12}{⼿}
  \begin{Phonetics}{搅}{jiao3}[][HSK 7-9]
    \definition{v.}{mexer; misturar | perturbar; incomodar; interromper}
  \end{Phonetics}
\end{Entry}

\begin{Entry}{搅拌}{12,8}{⼿,⼿}
  \begin{Phonetics}{搅拌}{jiao3ban4}[][HSK 7-9]
    \definition{v.}{misturar; mexer; agitar; usar uma colher, um palito ou um utensílio semelhante para girar a mistura e homogeneizá-la}
  \end{Phonetics}
\end{Entry}

%%%%%%%%%% 搓 %%%%%%%%%%
\subsection*{搓}\addcontentsline{loh}{figure}{搓}

\begin{Entry}{搓}{12}{⼿}
  \begin{Phonetics}{搓}{cuo1}[][HSK 7-9]
    \definition{s.}{torção}
    \definition{v.}{esfregar ou rolar entre as mãos ou dedos |  (no tênis, tênis de mesa, críquete, etc.) cortar | (roupa, etc.) torcer}
  \end{Phonetics}
\end{Entry}

%%%%%%%%%% 搜 %%%%%%%%%%
\subsection*{搜}\addcontentsline{loh}{figure}{搜}

\begin{Entry}{搜}{12}{⼿}
  \begin{Phonetics}{搜}{sou1}[][HSK 5]
    \definition{v.}{procurar | pesquisar | coletar; reunir | procurar ou revistar um lugar de forma completa e desordenada}
  \end{Phonetics}
\end{Entry}

\begin{Entry}{搜索}{12,10}{⼿,⽷}
  \begin{Phonetics}{搜索}{sou1suo3}[][HSK 5]
    \definition{v.}{procurar; caçar; explorar; pesquisar cuidadosamente; refere-se especificamente à busca militar para identificar situações suspeitas em determinada região, área marítima ou aérea}
  \end{Phonetics}
\end{Entry}

%%%%%%%%%% 搭 %%%%%%%%%%
\subsection*{搭}\addcontentsline{loh}{figure}{搭}

\begin{Entry}{搭}{12}{⼿}
  \begin{Phonetics}{搭}{da1}[][HSK 6]
    \definition{v.}{colocar em prática; construir | ficar pendurado; colocar para cima | entrar em contato; juntar-se | adicionar (mais pessoas, dinheiro, etc.) | levantar algo junto |
pegar (um navio, avião, etc.); viajar (ou ir) por}
    \variantof{褡}
  \end{Phonetics}
\end{Entry}

\begin{Entry}{搭讪}{12,5}{⼿,⾔}
  \begin{Phonetics}{搭讪}{da1shan4}
    \definition{v.}{bater em alguém | incitar uma conversa | começar a conversar para acabar com um silêncio constrangedor ou uma situação embaraçosa}
  \end{Phonetics}
\end{Entry}

\begin{Entry}{搭建}{12,8}{⼿,⼵}
  \begin{Phonetics}{搭建}{da1jian4}[][HSK 7-9]
    \definition{v.}{montar (um galpão, abrigo temporário, etc.) | criar (uma organização) | construir (especialmente com materiais simples) | juntar (um galpão temporário) | armar}
  \end{Phonetics}
\end{Entry}

\begin{Entry}{搭乘}{12,10}{⼿,⽲}
  \begin{Phonetics}{搭乘}{da1cheng2}[][HSK 7-9]
    \definition{v.}{viajar de (carro, barco, avião etc.)}
  \end{Phonetics}
\end{Entry}

\begin{Entry}{搭档}{12,10}{⼿,⽊}
  \begin{Phonetics}{搭档}{da1dang4}[][HSK 6]
    \definition[个,名,位]{s.}{parceiro; colega de trabalho}
    \definition{v.}{cooperar; trabalhar em conjunto; formar pares; colaborar; formar uma parceria}
  \end{Phonetics}
\end{Entry}

\begin{Entry}{搭配}{12,10}{⼿,⾣}
  \begin{Phonetics}{搭配}{da1pei4}[][HSK 6]
    \definition{v.}{emparelhar; organizar em pares ou grupos; organizar a distribuição de acordo com certos requisitos | encaixar; combinar}
  \end{Phonetics}
\end{Entry}

%%%%%%%%%% 搏 %%%%%%%%%%
\subsection*{搏}\addcontentsline{loh}{figure}{搏}

\begin{Entry}{搏}{13}{⼿}
  \begin{Phonetics}{搏}{bo2}
    \definition{v.}{brigar; lutar; combater | atacar | bater; pulsar (coração)}
  \end{Phonetics}
\end{Entry}

\begin{Entry}{搏斗}{13,4}{⼿,⽃}
  \begin{Phonetics}{搏斗}{bo2dou4}[][HSK 7-9]
    \definition{v.}{brigar; lutar; combater | envolver-se em combate corpo a corpo}
  \end{Phonetics}
\end{Entry}

%%%%%%%%%% 搞 %%%%%%%%%%
\subsection*{搞}\addcontentsline{loh}{figure}{搞}

\begin{Entry}{搞}{13}{⼿}
  \begin{Phonetics}{搞}{gao3}[][HSK 5]
    \definition{v.}{fazer; realizar; estar envolvido em; engajar-se em um estudo, fazer algo em relação a, etc. | fazer; produzir; gerar; trabalhar | iniciar; estabelecer; organizar; configurar | consertar (mudar) alguém; fazer alguém sofrer | obter; assegurar; agarrar |  (seguido de um complemento) fazer com que se torne; produzir um determinado efeito ou resultado}
  \end{Phonetics}
\end{Entry}

\begin{Entry}{搞好}{13,6}{⼿,⼥}
  \begin{Phonetics}{搞好}{gao3 hao3}[][HSK 5]
    \definition{v.}{fazer um bom trabalho; fazer bem; suar; tornar submisso, tornar útil, por meio de solicitações e presentes amigáveis; amolecer}
  \end{Phonetics}
\end{Entry}

\begin{Entry}{搞乱}{13,7}{⼿,⼄}
  \begin{Phonetics}{搞乱}{gao3luan4}
    \definition{v.}{estragar | confundir | bagunçar}
  \end{Phonetics}
\end{Entry}

\begin{Entry}{搞定}{13,8}{⼿,⼧}
  \begin{Phonetics}{搞定}{gao3ding4}
    \definition{v.}{consertar | resolver}
  \end{Phonetics}
\end{Entry}

\begin{Entry}{搞鬼}{13,9}{⼿,⿁}
  \begin{Phonetics}{搞鬼}{gao3/gui3}[][HSK 7-9]
    \definition{v.+compl.}{Coloquial: pregar peças; planejar em segredo; fazer alguma travessura}
  \end{Phonetics}
\end{Entry}

\begin{Entry}{搞笑}{13,10}{⼿,⽵}
  \begin{Phonetics}{搞笑}{gao3xiao4}[][HSK 7-9]
    \definition{adj.}{engraçado; divertido; descreve um estado ou qualidade que é interessante, engraçado ou faz as pessoas rirem}
    \definition{v.}{fazer palhaçadas para provocar risos; fazer as pessoas rirem deliberadamente; criar piadas e fazer as pessoas rirem}
  \end{Phonetics}
\end{Entry}

\begin{Entry}{搞通}{13,10}{⼿,⾡}
  \begin{Phonetics}{搞通}{gao3tong1}
    \definition{v.}{entender algo}
  \end{Phonetics}
\end{Entry}

\begin{Entry}{搞钱}{13,10}{⼿,⾦}
  \begin{Phonetics}{搞钱}{gao3qian2}
    \definition{v.}{fazer dinheiro | acumular dinheiro}
  \end{Phonetics}
\end{Entry}

\begin{Entry}{搞混}{13,11}{⼿,⽔}
  \begin{Phonetics}{搞混}{gao3hun4}
    \definition{v.}{confundir; embaralhar}
  \end{Phonetics}
\end{Entry}

\begin{Entry}{搞错}{13,13}{⼿,⾦}
  \begin{Phonetics}{搞错}{gao3cuo4}
    \definition{v.}{cometer um erro}
  \end{Phonetics}
\end{Entry}

%%%%%%%%%% 搬 %%%%%%%%%%
\subsection*{搬}\addcontentsline{loh}{figure}{搬}

\begin{Entry}{搬}{13}{⼿}
  \begin{Phonetics}{搬}{ban1}[][HSK 3]
    \definition{v.}{tirar; mover; remover | mudar-se (de casa) | aplicar indiscriminadamente; copiar mecanicamente}
  \end{Phonetics}
\end{Entry}

\begin{Entry}{搬口}{13,3}{⼿,⼝}
  \begin{Phonetics}{搬口}{ban1kou3}
    \definition{v.}{tagarelar | (idioma) transmitir histórias;  semear dissensão | contar histórias}
  \end{Phonetics}
\end{Entry}

\begin{Entry}{搬动}{13,6}{⼿,⼒}
  \begin{Phonetics}{搬动}{ban1dong4}
    \definition{v.}{mover (algo ao redor) | mudar de casa}
  \end{Phonetics}
\end{Entry}

\begin{Entry}{搬迁}{13,6}{⼿,⾡}
  \begin{Phonetics}{搬迁}{ban1qian1}[][HSK 7-9]
    \definition{v.}{mover; transferir; realocar}
  \end{Phonetics}
\end{Entry}

\begin{Entry}{搬弄}{13,7}{⼿,⼶}
  \begin{Phonetics}{搬弄}{ban1nong4}
    \definition{v.}{causar problemas | mexer com alguém | mostrar (o que se pode fazer)}
  \end{Phonetics}
\end{Entry}

\begin{Entry}{搬走}{13,7}{⼿,⾛}
  \begin{Phonetics}{搬走}{ban1zou3}
    \definition{v.}{carregar}
  \end{Phonetics}
\end{Entry}

\begin{Entry}{搬运}{13,7}{⼿,⾡}
  \begin{Phonetics}{搬运}{ban1yun4}
    \definition{v.}{carregar; transportar}
  \end{Phonetics}
\end{Entry}

\begin{Entry}{搬家}{13,10}{⼿,⼧}
  \begin{Phonetics}{搬家}{ban1/jia1}[][HSK 3]
    \definition{v.+compl.}{mudar de casa; mudar-se para outro lugar}
  \end{Phonetics}
\end{Entry}

%%%%%%%%%% 摄 %%%%%%%%%%
\subsection*{摄}\addcontentsline{loh}{figure}{摄}

\begin{Entry}{摄}{13}{⼿}
  \begin{Phonetics}{摄}{she4}
    \definition*{s.}{Sobrenome: She}
    \definition{v.}{absorver; assimilar | tirar uma fotografia de; fotografar | conservar (a saúde) | atuar}
  \end{Phonetics}
\end{Entry}

\begin{Entry}{摄氏}{13,4}{⼿,⽒}
  \begin{Phonetics}{摄氏}{she4shi4}
    \definition{s.}{graus Celsius (°C), centígrado}
  \end{Phonetics}
\end{Entry}

\begin{Entry}{摄氏度}{13,4,9}{⼿,⽒,⼴}
  \begin{Phonetics}{摄氏度}{she4shi4du4}[][HSK 7-9]
    \definition{s.}{centígrado; grau Celsius}[水在100摄氏度时沸腾。===A água ferve a 100 graus Celsius.]
  \end{Phonetics}
\end{Entry}

\begin{Entry}{摄像}{13,13}{⼿,⼈}
  \begin{Phonetics}{摄像}{she4 xiang4}[][HSK 5]
    \definition{v.}{gravar; filmar; filmar com câmera; fazer uma gravação de vídeo (com uma câmera de vídeo ou TV)}
  \end{Phonetics}
\end{Entry}

\begin{Entry}{摄像机}{13,13,6}{⼿,⼈,⽊}
  \begin{Phonetics}{摄像机}{she4 xiang4 ji1}[][HSK 5]
    \definition[个,部,台]{s.}{câmera de vídeo; dispositivo que pode ser usado para converter imagens captadas em sinais de imagem de televisão}
  \end{Phonetics}
\end{Entry}

\begin{Entry}{摄影}{13,15}{⼿,⼺}
  \begin{Phonetics}{摄影}{she4ying3}[][HSK 5]
    \definition{v.}{fotografar; tirar uma foto; tirar fotos ou filmar}
  \end{Phonetics}
\end{Entry}

\begin{Entry}{摄影师}{13,15,6}{⼿,⼺,⼱}
  \begin{Phonetics}{摄影师}{she4 ying3 shi1}[][HSK 5]
    \definition[个,名,位]{s.}{fotógrafo; cinegrafista; operador de câmera; técnico de fotografia em estúdio fotográfico}
  \end{Phonetics}
\end{Entry}

%%%%%%%%%% 摆 %%%%%%%%%%
\subsection*{摆}\addcontentsline{loh}{figure}{摆}

\begin{Entry}{摆}{13}{⼿}
  \begin{Phonetics}{摆}{bai3}[][HSK 4]
    \definition*{s.}{Festival de Ganbai; uma reunião realizada nas áreas Dai durante festivais religiosos, para celebrar uma boa colheita ou para trocar materiais; geralmente se refere a uma reunião em massa | Sobrenome: Bai}
    \definition{s.}{pêndulo; dispositivo mecânico que controla a frequência de oscilação em relógios e instrumentos |  a bainha inferior de um vestido, jaqueta ou saia}
    \definition{v.}{colocar; posicionar; organizar | assumir; mostrar intencionalmente | balançar; ondular; balançar para frente e para trás | revelar; listar; afirmar claramente | dizer; falar; declarar | libertar-se; livrar-se}
  \end{Phonetics}
\end{Entry}

\begin{Entry}{摆手}{13,4}{⼿,⼿}
  \begin{Phonetics}{摆手}{bai3/shou3}
    \definition{v.+compl.}{gesticular com a mão (acenando, acenando adeus, etc.) | balançar os braços | acenar com as mãos}
  \end{Phonetics}
\end{Entry}

\begin{Entry}{摆平}{13,5}{⼿,⼲}
  \begin{Phonetics}{摆平}{bai3/ping2}[][HSK 7-9]
    \definition{v.+compl.}{ser justo com; ser imparcial com; tratar com justiça | Dialeto: punir}
  \end{Phonetics}
\end{Entry}

\begin{Entry}{摆动}{13,6}{⼿,⼒}
  \begin{Phonetics}{摆动}{bai3 dong4}[][HSK 4]
    \definition{v.}{balançar; balançar para frente e para trás; oscilar; vibrar}
  \end{Phonetics}
\end{Entry}

\begin{Entry}{摆设}{13,6}{⼿,⾔}
  \begin{Phonetics}{摆设}{bai3she4}[][HSK 7-9]
    \definition{v.}{mobiliar e decorar (um cômodo)}
  \end{Phonetics}
\end{Entry}

\begin{Entry}{摆放}{13,8}{⼿,⽅}
  \begin{Phonetics}{摆放}{bai3fang4}[][HSK 7-9]
    \definition{v.}{colocar; posicionar; arranjar; organizar}
  \end{Phonetics}
\end{Entry}

\begin{Entry}{摆烂}{13,9}{⼿,⽕}
  \begin{Phonetics}{摆烂}{bai3lan4}
    \definition{v.}{(neologismo, gíria) parar de lutar (especialmente quando se sabe que não pode ter sucesso) | deixar tudo ir para o inferno}
  \end{Phonetics}
\end{Entry}

\begin{Entry}{摆脱}{13,11}{⼿,⾁}
  \begin{Phonetics}{摆脱}{bai3tuo1}[][HSK 4]
    \definition{v.}{sacudir; rejeitar; romper com; libertar-se (ou desembaraçar-se) de; livrar-se de dificuldades, escravidão, controle, etc.}
  \end{Phonetics}
\end{Entry}

%%%%%%%%%% 摇 %%%%%%%%%%
\subsection*{摇}\addcontentsline{loh}{figure}{摇}

\begin{Entry}{摇}{13}{⼿}
  \begin{Phonetics}{摇}{yao2}[][HSK 4]
    \definition{v.}{chacoalhar; ondular; balançar; fazer com que um objeto se mova para frente e para trás | agitar algo | sacudir; chacoalhar; agitar algo para que se mova}
  \end{Phonetics}
\end{Entry}

\begin{Entry}{摇头}{13,5}{⼿,⼤}
  \begin{Phonetics}{摇头}{yao2/tou2}[][HSK 5]
    \definition{v.+compl.}{sacudir; balançar a cabeça; balançar a cabeça para a esquerda e para a direita, indicando negação, desacordo ou impedimento}
  \end{Phonetics}
\end{Entry}

\begin{Entry}{摇晃}{13,10}{⼿,⽇}
  \begin{Phonetics}{摇晃}{yao2huang4}
    \definition{v.}{sacudir | agitar | balançar | chacoalhar}
  \end{Phonetics}
\end{Entry}

%%%%%%%%%% 摸 %%%%%%%%%%
\subsection*{摸}\addcontentsline{loh}{figure}{摸}

\begin{Entry}{摸}{13}{⼿}
  \begin{Phonetics}{摸}{mo1}[][HSK 4]
    \definition{v.}{sentir; acariciar; tocar; tocar (um objeto) levemente com a mão e depois removê-lo ou mover a mão suavemente sobre a superfície do objeto | sentir para; tatear para; sentir algo com as mãos | descobrir; sentir; sondar; explorar; tentar fazer ou entender | sentir o caminho; tatear no escuro; andar por estradas que você não consegue reconhecer | furtar; roubar}
  \end{Phonetics}
\end{Entry}

\begin{Entry}{摸索}{13,10}{⼿,⽷}
  \begin{Phonetics}{摸索}{mo1suo3}[][HSK 7-9]
    \definition{v.}{tatear; apalpar; explorar; sentir; procurar | investigar; estudar; explorar; descobrir; procurar; buscar; pesquisar}
  \end{Phonetics}
\end{Entry}

%%%%%%%%%% 摔 %%%%%%%%%%
\subsection*{摔}\addcontentsline{loh}{figure}{摔}

\begin{Entry}{摔}{14}{⼿}
  \begin{Phonetics}{摔}{shuai1}[][HSK 5]
    \definition{v.}{cair; tropeçar; perder o equilíbrio | mergulhar; precipitar-se; cair de uma altura elevada | quebrar; fazer cair e quebrar | lançar; atirar; arremessar; joguar coisas com força e para baixo | bater; golpear; bater com força para que o que está grudado cair}
  \end{Phonetics}
\end{Entry}

\begin{Entry}{摔倒}{14,10}{⼿,⼈}
  \begin{Phonetics}{摔倒}{shuai1dao3}[][HSK 5]
    \definition{v.}{cair; tropeçar; perder o equilíbrio e cair}
  \end{Phonetics}
\end{Entry}

%%%%%%%%%% 摘 %%%%%%%%%%
\subsection*{摘}\addcontentsline{loh}{figure}{摘}

\begin{Entry}{摘}{14}{⼿}
  \begin{Phonetics}{摘}{zhai1}[][HSK 5]
    \definition{v.}{pegar; arrancar; tirar; colher (flores, frutos, folhas de plantas); retirar (coisas que estão sendo usadas ou penduradas) | selecionar; fazer extrações de | pedir dinheiro emprestado em caso de necessidade urgente | vencer; ganhar; alcançar; obter}
  \end{Phonetics}
\end{Entry}

%%%%%%%%%% 摧 %%%%%%%%%%
\subsection*{摧}\addcontentsline{loh}{figure}{摧}

\begin{Entry}{摧}{14}{⼿}
  \begin{Phonetics}{摧}{cui1}
    \definition{v.}{quebrar; destruir}
  \end{Phonetics}
\end{Entry}

\begin{Entry}{摧毁}{14,13}{⼿,⽎}
  \begin{Phonetics}{摧毁}{cui1hui3}[][HSK 7-9]
    \definition{v.}{destruir; esmagar; nocautear; destruir com grande força}
  \end{Phonetics}
\end{Entry}

%%%%%%%%%% 撇 %%%%%%%%%%
\subsection*{撇}\addcontentsline{loh}{figure}{撇}

\begin{Entry}{撇}{14}{⼿}
  \begin{Phonetics}{撇}{pie1}
    \definition{v.}{descartar; jogar ao mar; abandonar | desnatar; retirar delicadamente o líquido da superfície}
  \end{Phonetics}
  \begin{Phonetics}{撇}{pie3}[][HSK 7-9]
    \definition{clas.}{utilizado para coisas sobrancelhas e barbas}
    \definition{s.}{traço descendente à esquerda 丿(em caracteres chineses)}
    \definition{v.}{atirar; arremessar; lançar}
  \end{Phonetics}
\end{Entry}

%%%%%%%%%% 摩 %%%%%%%%%%
\subsection*{摩}\addcontentsline{loh}{figure}{摩}

\begin{Entry}{摩}{15}{⼿}
  \begin{Phonetics}{摩}{mo2}
    \definition{v.}{esfregar; raspar; tocar | refletir; estudar | afagar}
  \end{Phonetics}
\end{Entry}

\begin{Entry}{摩托}{15,6}{⼿,⼿}
  \begin{Phonetics}{摩托}{mo2 tuo1}[][HSK 5]
    \definition[辆]{s.}{Empréstimo linguístico: motor; motor de combustão interna | Empréstimo linguístico: motocicleta, abreviação de 摩托车}
  \seealsoref{摩托车}{mo2tuo1che1}
  \end{Phonetics}
\end{Entry}

\begin{Entry}{摩托车}{15,6,4}{⼿,⼿,⾞}
  \begin{Phonetics}{摩托车}{mo2tuo1che1}
    \definition[辆,部]{s.}{(empréstimo linguístico) motocicleta}
  \end{Phonetics}
\end{Entry}

\begin{Entry}{摩擦}{15,17}{⼿,⼿}
  \begin{Phonetics}{摩擦}{mo2ca1}[][HSK 5]
    \definition{s.}{atrito; desacordo; conflito (entre duas partes); a ação de impedir o movimento relativo entre dois objetos em contato, produzida na superfície de contato | atrito; metáfora para o conflito entre as duas partes}
    \definition{v.}{esfregar}
  \end{Phonetics}
\end{Entry}

%%%%%%%%%% 撑 %%%%%%%%%%
\subsection*{撑}\addcontentsline{loh}{figure}{撑}

\begin{Entry}{撑}{15}{⼿}
  \begin{Phonetics}{撑}{cheng1}[][HSK 6]
    \definition{s.}{suporte; escora;  apoio; esteio}
    \definition{v.}{sustentar; apoiar; resistir a | empurrar (ou mover) com uma vara; usar um mastro para empurrar a margem ou o leito do rio para fazer o barco avançar | manter; manter-se atualizado | abrir; desdobrar; expandir (um objeto contraído) | encher até estourar (inchaço devido a excesso de comida ou alimentação excessiva)}
  \end{Phonetics}
\end{Entry}

%%%%%%%%%% 撒 %%%%%%%%%%
\subsection*{撒}\addcontentsline{loh}{figure}{撒}

\begin{Entry}{撒}{15}{⼿}
  \begin{Phonetics}{撒}{sa1}[][HSK 7-9]
    \definition{v.}{lançar; soltar; deixar escapar; liberar | abandonar todas as restrições; deixar-se levar; tentar usá-lo ou exibi-lo o máximo possível}
  \end{Phonetics}
  \begin{Phonetics}{撒}{sa3}
    \definition*{s.}{Sobrenome: Sa}
    \definition{v.}{espalhar; polvilhar; difundir; lançar | derramar; deixar cair}
  \end{Phonetics}
\end{Entry}

\begin{Entry}{撒旦}{15,5}{⼿,⽇}
  \begin{Phonetics}{撒旦}{sa1dan4}
    \definition*{s.}{Satanás, que significa 抵挡, sinônimo do diabo nas histórias bíblicas; um termo cristão para alguém que se opõe especificamente a Deus e é um inimigo Dele | Satã; Diabo}
  \seealsoref{抵挡}{di3dang3}
  \end{Phonetics}
\end{Entry}

\begin{Entry}{撒旦主义}{15,5,5,3}{⼿,⽇,⼂,⼂}
  \begin{Phonetics}{撒旦主义}{sa1dan4 zhu3yi4}
    \definition*{s.}{Satanismo}
  \end{Phonetics}
\end{Entry}

\begin{Entry}{撒但}{15,7}{⼿,⼈}
  \begin{Phonetics}{撒但}{sa1dan4}
    \variantof{撒旦}
  \seealsoref{撒旦}{sa1dan4}
  \end{Phonetics}
\end{Entry}

\begin{Entry}{撒谎}{15,11}{⼿,⾔}
  \begin{Phonetics}{撒谎}{sa1/huang3}[][HSK 7-9]
    \definition{v.+compl.}{mentir; contar uma mentira}
  \end{Phonetics}
\end{Entry}

%%%%%%%%%% 撞 %%%%%%%%%%
\subsection*{撞}\addcontentsline{loh}{figure}{撞}

\begin{Entry}{撞}{15}{⼿}
  \begin{Phonetics}{撞}{zhuang4}[][HSK 5]
    \definition{v.}{chocar-se contra; chocar-se com; bater; colidir | encontrar-se por acaso; esbarrar em; deparar-se com | apressar; correr; empurrar | aproveitar a chance | esbarrar de repente em |  encontrar | confiar em; tentar | agir precipitadamente; invadir}
  \end{Phonetics}
\end{Entry}

\begin{Entry}{撞车}{15,4}{⼿,⾞}
  \begin{Phonetics}{撞车}{zhuang4/che1}
    \definition{v.+compl.}{(figurativo) colidir (opiniões, cronogramas, etc.) | ser o mesmo (assunto) | colidir (com outro veículo)}
  \end{Phonetics}
\end{Entry}

\begin{Entry}{撞运气}{15,7,4}{⼿,⾡,⽓}
  \begin{Phonetics}{撞运气}{zhuang4yun4qi5}
    \definition{v.}{confiar no destino | tentar a sorte}
  \end{Phonetics}
\end{Entry}

%%%%%%%%%% 撤 %%%%%%%%%%
\subsection*{撤}\addcontentsline{loh}{figure}{撤}

\begin{Entry}{撤}{15}{⼿}
  \begin{Phonetics}{撤}{che4}[][HSK 7-9]
    \definition{v.}{remover, tirar | demitir; liberar | retirar-se; evacuar}
  \end{Phonetics}
\end{Entry}

\begin{Entry}{撤换}{15,10}{⼿,⼿}
  \begin{Phonetics}{撤换}{che4huan4}[][HSK 7-9]
    \definition{v.}{demitir e substituir (alguém); revogar; substituir (alguém ou alguma coisa)}
  \end{Phonetics}
\end{Entry}

\begin{Entry}{撤离}{15,10}{⼿,⼇}
  \begin{Phonetics}{撤离}{che4 li2}[][HSK 6]
    \definition{v.}{retirar-se de; deixar; evacuar}
  \end{Phonetics}
\end{Entry}

\begin{Entry}{撤销}{15,12}{⼿,⾦}
  \begin{Phonetics}{撤销}{che4xiao1}[][HSK 6]
    \definition{v.}{cancelar; rescindir; revogar; remover}
  \end{Phonetics}
\end{Entry}

%%%%%%%%%% 撬 %%%%%%%%%%
\subsection*{撬}\addcontentsline{loh}{figure}{撬}

\begin{Entry}{撬}{15}{⼿}
  \begin{Phonetics}{撬}{qiao4}[][HSK 7-9]
    \definition{v.}{arrombar; arrancar; forçar com alavanca}[钥匙丢了,他只好把门撬开。===Ele perdeu a chave, então teve que arrombar a porta.]
  \end{Phonetics}
\end{Entry}

%%%%%%%%%% 播 %%%%%%%%%%
\subsection*{播}\addcontentsline{loh}{figure}{播}

\begin{Entry}{播}{15}{⼿}
  \begin{Phonetics}{播}{bo1}[][HSK 6]
    \definition{v.}{espalhar; transmitir | semear | mover-se; migrar; ir para o exílio}
  \end{Phonetics}
\end{Entry}

\begin{Entry}{播出}{15,5}{⼿,⼐}
  \begin{Phonetics}{播出}{bo1 chu1}[][HSK 3]
    \definition{v.}{radiodifundir; transmitir; estar no ar; transmitir via rádio e televisão}
  \end{Phonetics}
\end{Entry}

\begin{Entry}{播放}{15,8}{⼿,⽅}
  \begin{Phonetics}{播放}{bo1fang4}[][HSK 3]
    \definition{v.}{ir ao ar; transmitir por rádio | mostrar; exibir; transmitir (um programa de TV)}
  \end{Phonetics}
\end{Entry}

\begin{Entry}{播音}{15,9}{⼿,⾳}
  \begin{Phonetics}{播音}{bo1/yin1}
    \definition{s.}{transmissão}
    \definition{v.+compl.}{transmitir}
  \end{Phonetics}
\end{Entry}

%%%%%%%%%% 擒 %%%%%%%%%%
\subsection*{擒}\addcontentsline{loh}{figure}{擒}

\begin{Entry}{擒}{15}{⼿}
  \begin{Phonetics}{擒}{qin2}
    \definition{v.}{capturar; pegar; apreender}
  \end{Phonetics}
\end{Entry}

\begin{Entry}{擒获}{15,10}{⼿,⾋}
  \begin{Phonetics}{擒获}{qin2huo4}
    \definition{v.}{apreender | capturar}
  \end{Phonetics}
\end{Entry}

%%%%%%%%%% 撼 %%%%%%%%%%
\subsection*{撼}\addcontentsline{loh}{figure}{撼}

\begin{Entry}{撼}{16}{⼿}
  \begin{Phonetics}{撼}{han4}
    \definition{v.}{agitar; sacudir}
  \end{Phonetics}
\end{Entry}

%%%%%%%%%% 擅 %%%%%%%%%%
\subsection*{擅}\addcontentsline{loh}{figure}{擅}

\begin{Entry}{擅}{16}{⼿}
  \begin{Phonetics}{擅}{shan4}
    \definition{adv.}{sem autorização; arbitrariamente | fazer algo por conta própria}
    \definition{v.}{ser bom em; ser especialista em | arrogar-se a si mesmo; fazer algo por conta própria | reivindicar arbitrariamente; ir além do escopo e ajir arbitrariamente}
  \end{Phonetics}
\end{Entry}

\begin{Entry}{擅长}{16,4}{⼿,⾧}
  \begin{Phonetics}{擅长}{shan4chang2}[][HSK 7-9]
    \definition{v.}{ser bom em; ser especialista em; ser habilidoso em; ter um talento especial em determinada área}
  \end{Phonetics}
\end{Entry}

\begin{Entry}{擅自}{16,6}{⼿,⾃}
  \begin{Phonetics}{擅自}{shan4zi4}[][HSK 7-9]
    \definition{adv.}{arbitrariamente; sem permissão ou autorização; agir por iniciativa própria em assuntos que estão fora da sua alçada}
  \end{Phonetics}
\end{Entry}

%%%%%%%%%% 操 %%%%%%%%%%
\subsection*{操}\addcontentsline{loh}{figure}{操}

\begin{Entry}{操}{16}{⼿}
  \begin{Phonetics}{操}{cao1}
    \definition*{s.}{Sobrenome: Cao}
    \definition[节,套]{s.}{exercício; ginástica | conduta; comportamento; moralidade, a moral e o código de conduta que as pessoas seguem}
    \definition{v.}{segurar; agarrar; segurar na mão | fazer algo; envolver-se em | falar (uma língua ou dialeto) | treinar (tropas); exercitar (corpo); praticar ou treinar de acordo com uma determinada forma ou postura | dirigir; manusear}
  \end{Phonetics}
\end{Entry}

\begin{Entry}{操心}{16,4}{⼿,⼼}
  \begin{Phonetics}{操心}{cao1/xin1}[][HSK 7-9]
    \definition{v.+compl.}{se esforçar; preocupar-se com; incomodar-se com}
  \end{Phonetics}
\end{Entry}

\begin{Entry}{操场}{16,6}{⼿,⼟}
  \begin{Phonetics}{操场}{cao1chang3}[][HSK 4]
    \definition[个,片,座,处]{s.}{\emph{playground}; campo esportivo; locais para exercícios físicos ou exercícios militares}
  \end{Phonetics}
\end{Entry}

\begin{Entry}{操作}{16,7}{⼿,⼈}
  \begin{Phonetics}{操作}{cao1zuo4}[][HSK 4]
    \definition{v.}{operar; seguir os requisitos e procedimentos prescritos | implementar; realizar; executar; refere-se à implementação concreta (planos, medidas, etc.)}
  \end{Phonetics}
\end{Entry}

\begin{Entry}{操劳}{16,7}{⼿,⼒}
  \begin{Phonetics}{操劳}{cao1lao2}[][HSK 7-9]
    \definition{v.}{trabalhar duro | cuidar; cuidar de}
  \end{Phonetics}
\end{Entry}

\begin{Entry}{操纵}{16,7}{⼿,⽷}
  \begin{Phonetics}{操纵}{cao1zong4}[][HSK 6]
    \definition{v.}{operar; controlar (uma máquina, instrumento, etc.) | manipular; controlar secretamente; assumir o controle de (uma pessoa, organização, situação, etc.)}
  \end{Phonetics}
\end{Entry}

\begin{Entry}{操控}{16,11}{⼿,⼿}
  \begin{Phonetics}{操控}{cao1kong4}[][HSK 7-9]
    \definition{v.}{controlar; manipular}
  \end{Phonetics}
\end{Entry}

%%%%%%%%%% 擦 %%%%%%%%%%
\subsection*{擦}\addcontentsline{loh}{figure}{擦}

\begin{Entry}{擦}{17}{⼿}
  \begin{Phonetics}{擦}{ca1}[][HSK 4]
    \definition{v.}{enxugar; esfregar; apagar; limpar; limpar esfregando com um pano, toalha de mão, etc. | espalhar sobre; colocar sobre | passar raspando | ralar (em pedaços); ralar frutas em um ralador para fazer fios finos}
  \end{Phonetics}
\end{Entry}

\begin{Entry}{擦拭}{17,9}{⼿,⼿}
  \begin{Phonetics}{擦拭}{ca1shi4}
    \definition{v.}{limpar com um pano}
  \end{Phonetics}
\end{Entry}

%%%%%%%%%% 攀 %%%%%%%%%%
\subsection*{攀}\addcontentsline{loh}{figure}{攀}

\begin{Entry}{攀}{19}{⼿}
  \begin{Phonetics}{攀}{pan1}[][HSK 7-9]
    \definition{v.}{escalar; escalar | buscar conexões em altos cargos | envolver; implicar | agarrar; agarrar-se; segurar-se a}
  \end{Phonetics}
\end{Entry}

\begin{Entry}{攀升}{19,4}{⼿,⼗}
  \begin{Phonetics}{攀升}{pan1sheng1}[][HSK 7-9]
    \definition{v.}{subir para um ponto mais alto | (preços, quantidade, etc.) subir; aumentar; escalar | subir}
  \end{Phonetics}
\end{Entry}

\begin{Entry}{攀岩}{19,8}{⼿,⼭}
  \begin{Phonetics}{攀岩}{pan1yan2}
    \definition{s.}{escalada em rocha; isso se refere a esse tipo de esporte}
    \definition{v.}{escalar uma parede rochosa íngreme com equipamento mínimo}
  \end{Phonetics}
\end{Entry}

\begin{Entry}{攀爬}{19,8}{⼿,⽖}
  \begin{Phonetics}{攀爬}{pan1pa2}
    \definition{v.}{escalar; escalada em rocha; refere-se ao movimento em uma determinada direção usando apenas as mãos e os pés, com o mínimo uso de ferramentas}
  \end{Phonetics}
\end{Entry}

%%%%% EOF %%%%%

