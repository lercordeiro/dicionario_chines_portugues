%%%
%%% Radical "⼂"
%%%
\section*{Radical 3: ``⼂'' (乀,乁)}\addcontentsline{toc}{section}{Radical 3: ⼂,乀,乁}\addcontentsline{loh}{figure}{\#\#\#\# 3: ⼂}

%%%%%%%%%% 丸 %%%%%%%%%%
\subsection*{丸}\addcontentsline{loh}{figure}{丸}

\begin{Entry}{丸}{3}{⼂}
  \begin{Phonetics}{丸}{wan2}[][HSK 7-9]
    \definition{clas.}{utilizado para medicamentos em comprimido}
    \definition[个]{s.}{bola; grânulo | comprimido; bolo (como em bolo alimentar); pílula}
  \seealsoref{丸儿}{wan2r5}
  \end{Phonetics}
\end{Entry}

\begin{Entry}{丸儿}{3,2}{⼂,⼉}
  \begin{Phonetics}{丸儿}{wan2r5}
    \definition{s.}{bola; grânulo}
  \end{Phonetics}
\end{Entry}

%%%%%%%%%% 义 %%%%%%%%%%
\subsection*{义}\addcontentsline{loh}{figure}{义}

\begin{Entry}{义}{3}{⼂}
  \begin{Phonetics}{义}{yi4}
    \definition*{s.}{Sobrenome: Yi}
    \definition{adj.}{justo; equitativo | adotado; adotivo | juramentado | artificial; falso}
    \definition[个]{s.}{justiça; retidão | laços humanos; relacionamento | significado; importância}
  \end{Phonetics}
\end{Entry}

\begin{Entry}{义务}{3,5}{⼂,⼒}
  \begin{Phonetics}{义务}{yi4wu4}[][HSK 4]
    \definition{adj.}{voluntário; fornecer serviços ou ajuda a outros gratuitamente}
    \definition[项]{s.}{dever; obrigação; responsabilidades perante a lei | obrigação moral; responsabilidade moral}
  \synonymref{负担}{fu4dan1}
  \synonymref{任务}{ren4wu5}
  \synonymref{无偿}{wu2chang2}
  \synonymref{仔肩}{zi1jian1}
  \antonymref{权利}{quan2li4}
  \antonymref{权力}{quan2li4}
  \antonymref{权益}{quan2yi4}
  \antonymref{责任}{ze2ren4}
  \end{Phonetics}
\end{Entry}

%%%%%%%%%% 之 %%%%%%%%%%
\subsection*{之}\addcontentsline{loh}{figure}{之}

\begin{Entry}{之}{3}{⼂}
  \begin{Phonetics}{之}{zhi1}
    \definition*{s.}{Sobrenome: Zhi}
    \definition{part.}{entre um atributivo e a palavra que ele modifica; equivalente a 的 | usado entre o sujeito e o predicado, em estruturas sujeito"-predicado, de modo a torná"-lo nominalizado}
    \definition{pron.}{substituto de uma pessoa ou coisa, limitado a ser usado como um objeto; substituir a pessoa ou coisa mencionada anteriormente | isto; isso; não substitui uma pessoa ou coisa específica, mas serve apenas para complementar sílabas}
    \definition{v.}{ir; deixar}
  \seealsoref{的}{de5}
  \end{Phonetics}
\end{Entry}

\begin{Entry}{之一}{3,1}{⼂,⼀}
  \begin{Phonetics}{之一}{zhi1yi1}[][HSK 4]
    \definition[分]{s.}{um de (algo); pertence a um ou a todo um grupo de coisas com as mesmas características}
  \synonymref{之中}{zhi1zhong1}
  \end{Phonetics}
\end{Entry}

\begin{Entry}{之下}{3,3}{⼂,⼀}
  \begin{Phonetics}{之下}{zhi1xia4}[][HSK 5]
    \definition{s.}{usado para indicar algo abaixo de um determinado intervalo, posição, grau, etc.; indica um aspecto inferior em termos de alcance, posição, status, nível, Chengdu, etc. | usado para indicar as condições sob as quais algo acontece | usado para indicar o humor, estado em que alguém faz algo; expressa um determinado comportamento em um determinado estado de espírito ou situação}
  \synonymref{以下}{yi3xia4}
  \end{Phonetics}
\end{Entry}

\begin{Entry}{之中}{3,4}{⼂,⼁}
  \begin{Phonetics}{之中}{zhi1zhong1}[][HSK 5]
    \definition{prep.}{em; no meio de; entre}
  \synonymref{之一}{zhi1yi1}
  \end{Phonetics}
\end{Entry}

\begin{Entry}{之内}{3,4}{⼂,⼌}
  \begin{Phonetics}{之内}{zhi1nei4}[][HSK 5]
    \definition{adv.}{em; dentro de; indica dentro de um determinado intervalo, limite ou período de tempo, etc.}
  \synonymref{以内}{yi3nei4}
  \end{Phonetics}
\end{Entry}

\begin{Entry}{之外}{3,5}{⼂,⼣}
  \begin{Phonetics}{之外}{zhi1wai4}[][HSK 5]
    \definition{adv.}{lado de fora; exceto; além de; além disso; refere"-se a algo que excede um determinado limite}
  \synonymref{除外}{chu2wai4}
  \synonymref{以外}{yi3wai4}
  \end{Phonetics}
\end{Entry}

\begin{Entry}{之后}{3,6}{⼂,⼝}
  \begin{Phonetics}{之后}{zhi1hou4}[][HSK 4]
    \definition{s.}{mais tarde; posteriormente; depois; desde então; para indicar que é depois de um determinado tempo ou de uma determinada coisa, 以后 é usado com frequência na linguagem falada; às vezes, também pode indicar que é depois de um determinado lugar ou local,  后面 é usado com frequência na linguagem falada}
  \seealsoref{后面}{hou4mian5}
  \seealsoref{以后}{yi3hou4}
  \synonymref{后来}{hou4lai2}
  \synonymref{继而}{ji4'er2}
  \synonymref{接着}{jie1zhe5}
  \synonymref{其后}{qi2hou4}
  \synonymref{事后}{shi4hou4}
  \synonymref{以后}{yi3hou4}
  \antonymref{之前}{zhi1qian2}
  \end{Phonetics}
\end{Entry}

\begin{Entry}{之间}{3,7}{⼂,⾨}
  \begin{Phonetics}{之间}{zhi1jian1}[][HSK 4]
    \definition{s.}{(depois de um substantivo) entre; dentro de duas delimitações de tempo, local ou quantitativas | colocado após certos verbos ou advérbios de duas sílabas para indicar um curto período de tempo}
  \end{Phonetics}
\end{Entry}

\begin{Entry}{之前}{3,9}{⼂,⼑}
  \begin{Phonetics}{之前}{zhi1qian2}[][HSK 4]
    \definition{adv.}{(referindo"-se ao tempo) antes, antes de, atrás | (referindo"-se ao local físico) na frente de | (usado independentemente) no passado, antes disso}
  \synonymref{曾经}{ceng2jing1}
  \synonymref{此前}{ci3qian2}
  \synonymref{以前}{yi3qian2}
  \antonymref{正在}{zheng4zai4}
  \antonymref{之后}{zhi1hou4}
  \end{Phonetics}
\end{Entry}

\begin{Entry}{之类}{3,9}{⼂,⽶}
  \begin{Phonetics}{之类}{zhi1lei4}[][HSK 6]
    \definition{s.}{usado para dar exemplos (coisas do tipo, desse tipo, assim); uma categoria de pessoas ou coisas que compartilham as mesmas características das pessoas ou coisas mencionadas anteriormente}[我喜欢香蕉、苹果之类的水果。===Eu gosto de frutas como bananas e maçãs.]
  \end{Phonetics}
\end{Entry}

%%%%%%%%%% 为 %%%%%%%%%%
\subsection*{为}\addcontentsline{loh}{figure}{为}

\begin{Entry}{为}{4}{⼂}
  \begin{Phonetics}{为}{wei2}[][HSK 3]
    \definition*{s.}{Sobrenome: Wei}
    \definition{part.}{frequentemente usado com 何 em uma pergunta retórica}
    \definition{prep.}{por; usado em frases passivas para introduzir o agente da ação, equivalente a 被 (frequentemente usado com 所)}
    \definition{suf.}{é anexado a alguns adjetivos ou advérbios monossilábicos para formar advérbios dissilábicos que expressam grau ou amplitude, geralmente modificando adjetivos ou verbos dissilábicos}
    \definition{v.}{fazer; agir | tornar"-se; transformar"-se em | ser; significar | servir como; agir como; desempenhar o papel de | fazer; trabalhar; indica certas ações e comportamentos, incluindo os significados de governança, engajamento, cenário e pesquisa}
  \seealsoref{被}{bei4}
  \seealsoref{何}{he2}
  \seealsoref{所}{suo3}
  \synonymref{替}{ti4}
  \end{Phonetics}
  \begin{Phonetics}{为}{wei4}[][HSK 2]
    \definition*{s.}{Sobrenome: Wei}
    \definition{part.}{com 何 em uma pergunta retórica para expressar dúvida}
    \definition{prep.}{por; usado em frases passivas para introduzir o agente da ação, equivalente a 被 (frequentemente usado com 所)}
    \definition{suf.}{é anexado a alguns adjetivos ou advérbios monossilábicos para formar advérbios dissilábicos que expressam grau ou amplitude, geralmente modificando adjetivos ou verbos dissilábicos}
    \definition{v.}{fazer; agir | tornar"-se; transformar-se em | ser;  significar | servir como; agir como; desempenhar o papel de | fazer; trabalhar; indica certas ações e comportamentos, incluindo os significados de governança, engajamento, cenário e pesquisa}
  \seealsoref{被}{bei4}
  \seealsoref{何}{he2}
  \seealsoref{所}{suo3}
  \synonymref{替}{ti4}
  \end{Phonetics}
\end{Entry}

\begin{Entry}{为了}{4,2}{⼂,⼅}
  \begin{Phonetics}{为了}{wei4le5}[][HSK 3]
    \definition{prep.}{para; por causa de; a fim de; o objetivo da introdução de ações comportamentais}
  \synonymref{使得}{shi3de5}
  \synonymref{为}{wei4}
  \synonymref{因为}{yin1wei5}
  \synonymref{着想}{zhuo2xiang3}
  \end{Phonetics}
\end{Entry}

\begin{Entry}{为人}{4,2}{⼂,⼈}
  \begin{Phonetics}{为人}{wei2ren2}[][HSK 7-9]
    \definition[方]{s.}{comportamento; conduta; atitude em relação aos outros e às coisas}
    \definition{v.}{comportar"-se; conduzir"-se}
  \synonymref{君子}{jun1zi3}
  \synonymref{人品}{ren2pin3}
  \end{Phonetics}
\end{Entry}

\begin{Entry}{为什么}{4,4,3}{⼂,⼈,⼃}
  \begin{Phonetics}{为什么}{wei4shen2me5}[][HSK 2]
    \definition{adv.}{por que?; por que é que?; como é que?;  nota: 为什么不 geralmente tem o significado de conselho, o mesmo que 何不}
  \seealsoref{何不}{he2bu4}
  \end{Phonetics}
\end{Entry}

\begin{Entry}{为止}{4,4}{⼂,⽌}
  \begin{Phonetics}{为止}{wei2zhi3}[][HSK 5]
    \definition{adv.}{até; até um determinado momento}
  \synonymref{截止}{jie2zhi3}
  \synonymref{截至}{jie2zhi4}
  \end{Phonetics}
\end{Entry}

\begin{Entry}{为主}{4,5}{⼂,⼂}
  \begin{Phonetics}{为主}{wei2zhu3}[][HSK 5]
    \definition{v.}{dar prioridade a; dar preferência a; dar importância a}
  \end{Phonetics}
\end{Entry}

\begin{Entry}{为此}{4,6}{⼂,⽌}
  \begin{Phonetics}{为此}{wei4ci3}[][HSK 6]
    \definition{conj.}{portanto; para este fim; por esta razão; para este propósito; nesta conexão; contexto de conexão, indicando que o comportamento descrito é devido aos motivos mencionados anteriormente}
  \synonymref{所以}{suo3yi3}
  \synonymref{因而}{yin1'er2}
  \synonymref{因此}{yin1ci3}
  \end{Phonetics}
\end{Entry}

\begin{Entry}{为何}{4,7}{⼂,⼈}
  \begin{Phonetics}{为何}{wei4he2}[][HSK 6]
    \definition{adv.}{por que?; por qual razão?}
  \seealsoref{为什么}{wei4shen2me5}
  \synonymref{何故}{he2gu4}
  \end{Phonetics}
\end{Entry}

\begin{Entry}{为难}{4,10}{⼂,⾫}
  \begin{Phonetics}{为难}{wei2nan2}[][HSK 5]
    \definition{adj.}{envergonhado; sentir"-se constrangido; sentir"-se sobrecarregado; sentir"-se incapaz de lidar com algo}
    \definition{v.}{dificultar as coisas para; dificultar; contrariar}
  \synonymref{刁难}{diao1nan4}
  \synonymref{尴尬}{gan1ga4}
  \antonymref{乐意}{le4yi4}
  \antonymref{愿意}{yuan4yi5}
  \end{Phonetics}
\end{Entry}

\begin{Entry}{为期}{4,12}{⼂,⽉}
  \begin{Phonetics}{为期}{wei2qi1}[][HSK 5]
    \definition{s.}{Literário: tempo restante}
    \definition{v.}{Literário: a ser concluído (até uma data definida, por um determinado período de tempo)}
  \end{Phonetics}
\end{Entry}

%%%%%%%%%% 主 %%%%%%%%%%
\subsection*{主}\addcontentsline{loh}{figure}{主}

\begin{Entry}{主}{5}{⼂}
  \begin{Phonetics}{主}{zhu3}
    \definition*{s.}{Deus; Senhor; o nome do Deus em que se acredita o cristianismo, o judaísmo, etc.}
    \definition{adj.}{principal; primário; o mais básico; o mais importante | de si mesmo; por vontade própria; próprio; do próprio}
    \definition[位,名,个]{s.}{anfitrião; alguém que convida e recebe convidados | mestre; dono; uma pessoa que possui poder ou propriedade; uma pessoa em posição dominante | pessoa ou parte interessada | decisão; opinião; visão definitiva | placa espiritual (ou memorial)}
    \definition{v.}{dirigir; administrar; assumir o comando de; presidir; assumir a responsabilidade primária | decidir; reivindicar | significar; indicar; prever um certo resultado}
  \antonymref{宾}{bin1}
  \antonymref{次}{ci4}
  \antonymref{从}{cong2}
  \antonymref{副}{fu4}
  \antonymref{客}{ke4}
  \antonymref{奴}{nu2}
  \end{Phonetics}
\end{Entry}

\begin{Entry}{主人}{5,2}{⼂,⼈}
  \begin{Phonetics}{主人}{zhu3ren2}[][HSK 2]
    \definition[个,位]{s.}{mestre; uma pessoa que empregava tutores, contadores, etc. antigamente; uma pessoa que empregava empregados domésticos | anfitrião; Aaguém que entretém convidados | proprietário; uma pessoa que possui um certo tipo de bens ou poder}
  \synonymref{老板}{lao3ban3}
  \antonymref{客人}{ke4ren5}
  \end{Phonetics}
\end{Entry}

\begin{Entry}{主义}{5,3}{⼂,⼂}
  \begin{Phonetics}{主义}{zhu3yi4}
    \definition[种]{s.}{doutrina; um determinado sistema social ou sistema político e econômico | estilo de pensamento; um certo ponto de vista ou estilo | ideologia; teorias e doutrinas sistemáticas sobre a natureza, a sociedade humana, etc.}
    \definition{suf.}{-ismo}
  \synonymref{办法}{ban4fa3}
  \synonymref{方针}{fang1zhen1}
  \synonymref{目标}{mu4biao1}
  \synonymref{目的}{mu4di4}
  \synonymref{想法}{xiang3fa5}
  \synonymref{主张}{zhu3zhang1}
  \end{Phonetics}
\end{Entry}

\begin{Entry}{主办}{5,4}{⼂,⼒}
  \begin{Phonetics}{主办}{zhu3ban4}[][HSK 5]
    \definition{v.}{manter; hospedar; dirigir; patrocinar}
  \synonymref{主持}{zhu3chi2}
  \end{Phonetics}
\end{Entry}

\begin{Entry}{主任}{5,6}{⼂,⼈}
  \begin{Phonetics}{主任}{zhu3ren4}[][HSK 3]
    \definition[个,位,名]{s.}{chefe; diretor; presidente; o principal responsável por um departamento ou instituição}
  \synonymref{主管}{zhu3guan3}
  \antonymref{职员}{zhi2yuan2}
  \end{Phonetics}
\end{Entry}

\begin{Entry}{主动}{5,6}{⼂,⼒}
  \begin{Phonetics}{主动}{zhu3dong4}[][HSK 3]
    \definition{adj.}{ativo; positivo; agir sem esperar por um impulso externo | iniciativo; capaz de impulsionar as coisas por vontade própria; capaz de criar uma situação favorável e fazer as coisas acontecerem de acordo com suas próprias intenções}
  \synonymref{积极}{ji1ji2}
  \synonymref{自动}{zi4dong4}
  \antonymref{被动}{bei4dong4}
  \end{Phonetics}
\end{Entry}

\begin{Entry}{主导}{5,6}{⼂,⼨}
  \begin{Phonetics}{主导}{zhu3dao3}[][HSK 5]
    \definition{adj.}{líder; dominante; guiado; principais e guias para que as coisas se desenvolvam em uma determinada direção}
    \definition{s.}{fator principal (ou orientador)}
  \synonymref{核心}{he2xin1}
  \synonymref{掌握}{zhang3wo4}
  \synonymref{支配}{zhi1pei4}
  \synonymref{主管}{zhu3guan3}
  \end{Phonetics}
\end{Entry}

\begin{Entry}{主观}{5,6}{⼂,⾒}
  \begin{Phonetics}{主观}{zhu3guan1}[][HSK 5]
    \definition{adj.}{subjetivo; não com base nas condições reais, mas com base nos próprios desejos | subjetivo; filosoficamente, refere"-se à consciência e aos aspectos espirituais dos seres humanos}
  \antonymref{客观}{ke4guan1}
  \end{Phonetics}
\end{Entry}

\begin{Entry}{主体}{5,7}{⼂,⼈}
  \begin{Phonetics}{主体}{zhu3ti3}[][HSK 5]
    \definition[个,些,种,群]{s.}{corpo principal; parte principal; parte principal; esteio; a parte principal das coisas | Filosofia: sujeito}
  \synonymref{本身}{ben3shen1}
  \synonymref{首要}{shou3yao4}
  \end{Phonetics}
\end{Entry}

\begin{Entry}{主张}{5,7}{⼂,⼸}
  \begin{Phonetics}{主张}{zhu3zhang1}[][HSK 3]
    \definition[个,项,些,种]{s.}{vista; posição; proposição}
    \definition{v.}{defender; apoiar; manter; representar; ter uma opinião sobre como agir, fazer uma sugestão}
  \synonymref{办法}{ban4fa3}
  \synonymref{倡导}{chang4dao3}
  \synonymref{观点}{guan1dian3}
  \synonymref{见解}{jian4jie3}
  \synonymref{看法}{kan4fa5}
  \synonymref{想法}{xiang3fa5}
  \synonymref{意见}{yi4jian5}
  \synonymref{主义}{zhu3yi4}
  \synonymref{主意}{zhu3yi5}
  \antonymref{禁止}{jin4zhi3}
  \end{Phonetics}
\end{Entry}

\begin{Entry}{主角}{5,7}{⼂,⾓}
  \begin{Phonetics}{主角}{zhu3jue2}[][HSK 6]
    \definition[个,位,名]{s.}{liderança; papel principal; protagonista; um papel importante em uma peça, filme, etc.; um ator que desempenha um papel importante | Figurativo: algo que tem grande influência em uma determinada área; refere"-se ao personagem principal}
  \synonymref{主人}{zhu3ren2}
  \end{Phonetics}
\end{Entry}

\begin{Entry}{主持}{5,9}{⼂,⼿}
  \begin{Phonetics}{主持}{zhu3chi2}[][HSK 3]
    \definition[位,名]{s.}{anfitrião; a pessoa responsável por administrar e lidar com uma determinada atividade}
    \definition{v.}{dirigir; administrar; assumir o comando; encarregar-se de; ser responsável por gerenciar, organizar uma determinada atividade ou lidar com um determinado assunto | defender; apoiar; preservar; manter}
  \synonymref{把持}{ba3chi2}
  \synonymref{主办}{zhu3ban4}
  \synonymref{主席}{zhu3xi2}
  \synonymref{总裁}{zong3cai2}
  \end{Phonetics}
\end{Entry}

\begin{Entry}{主持人}{5,9,2}{⼂,⼿,⼈}
  \begin{Phonetics}{主持人}{zhu3chi2ren2}[][HSK 6]
    \definition[个,位]{s.}{anfitrião; âncora; apresentador}
  \end{Phonetics}
\end{Entry}

\begin{Entry}{主要}{5,9}{⼂,⾑}
  \begin{Phonetics}{主要}{zhu3yao4}[][HSK 2]
    \definition{adj.}{principal; chefe; o mais importante na questão; o decisivo | principal; núcleo; a raiz ou parte mais importante de algo}
  \synonymref{重点}{chong2dian3}
  \synonymref{关键}{guan1jian4}
  \synonymref{首要}{shou3yao4}
  \synonymref{严重}{yan2zhong4}
  \synonymref{重点}{zhong4dian3}
  \synonymref{重要}{zhong4yao4}
  \antonymref{辅助}{fu3zhu4}
  \end{Phonetics}
\end{Entry}

\begin{Entry}{主席}{5,10}{⼂,⼱}
  \begin{Phonetics}{主席}{zhu3xi2}[][HSK 4]
    \definition*{s.}{Presidente (da China)}
    \definition[个,位,名]{s.}{presidente, \emph{chairman} (de uma reunião) | chefe; presidente (de uma organização ou estado)}
  \synonymref{领导}{ling3dao3}
  \synonymref{主持}{zhu3chi2}
  \synonymref{总理}{zong3li3}
  \synonymref{总统}{zong3tong3}
  \end{Phonetics}
\end{Entry}

\begin{Entry}{主席台}{5,10,5}{⼂,⼱,⼝}
  \begin{Phonetics}{主席台}{zhu3xi2tai2}
    \definition[个]{s.}{plataforma | tribuna}
  \end{Phonetics}
\end{Entry}

\begin{Entry}{主席团}{5,10,6}{⼂,⼱,⼞}
  \begin{Phonetics}{主席团}{zhu3xi2tuan2}
    \definition{s.}{presídio}
  \end{Phonetics}
\end{Entry}

\begin{Entry}{主流}{5,10}{⼂,⽔}
  \begin{Phonetics}{主流}{zhu3liu2}[][HSK 6]
    \definition{s.}{corrente principal; corrente mãe; convencional | tendência principal; aspecto essencial ou principal; falando metaforicamente, os principais aspectos do desenvolvimento das coisas}
  \end{Phonetics}
\end{Entry}

\begin{Entry}{主意}{5,13}{⼂,⼼}
  \begin{Phonetics}{主意}{zhu3yi5}[][HSK 3]
    \definition[个,种]{s.}{ideia; plano; decisão; método}
  \synonymref{办法}{ban4fa3}
  \synonymref{点子}{dian3zi5}
  \synonymref{方法}{fang1fa3}
  \synonymref{方针}{fang1zhen1}
  \synonymref{目的}{mu4di4}
  \synonymref{想法}{xiang3fa5}
  \synonymref{主张}{zhu3zhang1}
  \end{Phonetics}
\end{Entry}

\begin{Entry}{主管}{5,14}{⼂,⽵}
  \begin{Phonetics}{主管}{zhu3guan3}[][HSK 5]
    \definition[位,名,个,些]{s.}{pessoa responsável, como supervisor, gerente, diretor, etc.}
    \definition{v.}{estar encarregado de; ser responsável por; ser o principal responsável pela gestão de um trabalho; assumir a responsabilidade primária pela gestão (um certo aspecto)}
  \synonymref{主导}{zhu3dao3}
  \synonymref{总裁}{zong3cai2}
  \end{Phonetics}
\end{Entry}

\begin{Entry}{主题}{5,15}{⼂,⾴}
  \begin{Phonetics}{主题}{zhu3ti2}[][HSK 4]
    \definition[个]{s.}{tema; assunto; motivo; lema; ideias básicas expressas em toda a obra de literatura e arte por meio de imagens artísticas concretas | pontos/conteúdos principais; referência geral ao conteúdo principal de artigos, discursos, conferências, etc.}
  \synonymref{草坪}{cao3ping2}
  \synonymref{核心}{he2xin1}
  \synonymref{焦点}{jiao1dian3}
  \synonymref{题材}{ti2cai2}
  \synonymref{中心}{zhong1xin1}
  \synonymref{中央}{zhong1yang1}
  \end{Phonetics}
\end{Entry}

%%%%%%%%%% 举 %%%%%%%%%%
\subsection*{举}\addcontentsline{loh}{figure}{举}

\begin{Entry}{举}{9}{⼂}
  \begin{Phonetics}{举}{ju3}[][HSK 2]
    \definition*{s.}{Sobrenome: Ju}
    \definition{adj.}{inteiro; completo}
    \definition{s.}{ato; ação; movimento; comportamento | (nas dinastias Ming e Qing) candidato aprovado nos exames imperiais a nível provincial}
    \definition{v.}{levantar; erguer; sustentar | começar; iniciar; surgir | eleger; escolher; recomendar; selecionar | citar; enumerar; propor; revelar}
  \end{Phonetics}
\end{Entry}

\begin{Entry}{举一反三}{9,1,4,3}{⼂,⼀,⼜,⼀}
  \begin{Phonetics}{举一反三}{ju3yi1-fan3san1}[][HSK 7-9]
    \definition{expr.}{aprender por analogia; inferir outras coisas a partir de um fato; aprender muitas coisas por analogia a partir de uma única coisa}
  \end{Phonetics}
\end{Entry}

\begin{Entry}{举办}{9,4}{⼂,⼒}
  \begin{Phonetics}{举办}{ju3ban4}[][HSK 3]
    \definition{v.}{conduzir; organizar; realizar}
  \antonymref{进行}{jin4xing2}
  \antonymref{举行}{ju3xing2}
  \antonymref{召开}{zhao4kai1}
  \end{Phonetics}
\end{Entry}

\begin{Entry}{举手}{9,4}{⼂,⼿}
  \begin{Phonetics}{举手}{ju3 shou3}[][HSK 2]
    \definition{v.}{levantar a mão ou as mãos; levantar a mão para sinalizar ou responder a uma pergunta}
  \end{Phonetics}
\end{Entry}

\begin{Entry}{举止}{9,4}{⼂,⽌}
  \begin{Phonetics}{举止}{ju3zhi3}[][HSK 7-9]
    \definition{s.}{maneira; comportamento; porte; postura; refere"-se à postura e ao comportamento}
  \synonymref{举动}{ju3dong4}
  \synonymref{行动}{xing2dong4}
  \synonymref{行为}{xing2wei2}
  \end{Phonetics}
\end{Entry}

\begin{Entry}{举世无双}{9,5,4,4}{⼂,⼀,⽆,⼜}
  \begin{Phonetics}{举世无双}{ju3shi4-wu2shuang1}[][HSK 7-9]
    \definition{expr.}{``Inigualável no mundo.''; inigualável; incomparável; sem igual; único; número um do mundo}
  \synonymref{独一无二}{du2yi1-wu2'er4}
  \end{Phonetics}
\end{Entry}

\begin{Entry}{举世闻名}{9,5,9,6}{⼂,⼀,⾨,⼝}
  \begin{Phonetics}{举世闻名}{ju3shi4-wen2ming2}[][HSK 7-9]
    \definition{expr.}{``De renome mundial.''; mundialmente famoso}
  \synonymref{大名鼎鼎}{da4ming2-ding3ding3}
  \synonymref{举世瞩目}{ju3shi4-zhu3mu4}
  \antonymref{不为人知}{bu4wei2ren2zhi1}
  \antonymref{默默无闻}{mo4mo4-wu2wen2}
  \end{Phonetics}
\end{Entry}

\begin{Entry}{举世瞩目}{9,5,17,5}{⼂,⼀,⽬,⽬}
  \begin{Phonetics}{举世瞩目}{ju3shi4-zhu3mu4}[][HSK 7-9]
    \definition{expr.}{``De renome mundial.''; atrair a atenção mundial; tornar"-se o centro das atenções mundiais; o mundo inteiro está assistindo}
  \synonymref{举世闻名}{ju3shi4-wen2ming2}
  \antonymref{默默无闻}{mo4mo4-wu2wen2}
  \end{Phonetics}
\end{Entry}

\begin{Entry}{举动}{9,6}{⼂,⼒}
  \begin{Phonetics}{举动}{ju3dong4}[][HSK 5]
    \definition{s.}{ato; atividade; movimento; ação}
  \synonymref{举措}{ju3cuo4}
  \synonymref{举止}{ju3zhi3}
  \synonymref{行为}{xing2wei2}
  \synonymref{作为}{zuo4wei2}
  \antonymref{内心}{nei4xin1}
  \end{Phonetics}
\end{Entry}

\begin{Entry}{举行}{9,6}{⼂,⾏}
  \begin{Phonetics}{举行}{ju3xing2}[][HSK 2]
    \definition{v.}{realizar (uma reunião, cerimônia, etc.); realizar (atividades formais ou solenes)}
  \synonymref{进行}{jin4xing2}
  \synonymref{举办}{ju3ban4}
  \synonymref{实行}{shi2xing2}
  \synonymref{召开}{zhao4kai1}
  \antonymref{保留}{bao3liu2}
  \antonymref{取消}{qu3xiao1}
  \end{Phonetics}
\end{Entry}

\begin{Entry}{举报}{9,7}{⼂,⼿}
  \begin{Phonetics}{举报}{ju3bao4}[][HSK 7-9]
    \definition{v.}{relatar; denunciar}[我决定举报不法行为。===Decidi denunciar as atividades ilegais.]
  \synonymref{揭发}{jie1fa1}
  \synonymref{投诉}{tou2su4}
  \end{Phonetics}
\end{Entry}

\begin{Entry}{举例}{9,8}{⼂,⼈}
  \begin{Phonetics}{举例}{ju3/li4}[][HSK 7-9]
    \definition{v.+compl.}{dar um exemplo; citar um caso}
  \synonymref{例如}{li4ru2}
  \end{Phonetics}
\end{Entry}

\begin{Entry}{举重}{9,9}{⼂,⾥}
  \begin{Phonetics}{举重}{ju3zhong4}[][HSK 7-9]
    \definition{s.}{levantamento de peso}
    \definition{v.}{levantar pesos}
  \end{Phonetics}
\end{Entry}

\begin{Entry}{举措}{9,11}{⼂,⼿}
  \begin{Phonetics}{举措}{ju3cuo4}[][HSK 7-9]
    \definition{v.}{mover; agir; medir}
  \synonymref{办法}{ban4fa3}
  \synonymref{步骤}{bu4zhou4}
  \synonymref{措施}{cuo4shi1}
  \synonymref{动作}{dong4zuo4}
  \synonymref{方法}{fang1fa3}
  \synonymref{举动}{ju3dong4}
  \synonymref{设施}{she4shi1}
  \synonymref{行动}{xing2dong4}
  \end{Phonetics}
\end{Entry}

%%%%% EOF %%%%%

