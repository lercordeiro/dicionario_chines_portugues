%%%
%%% Radical "⽭"
%%%
\section*{Radical 110: ``⽭''}\addcontentsline{toc}{section}{Radical 110: ⽭}\addcontentsline{loh}{figure}{\#\#\#\# 110: ⽭}

%%%%%%%%%% 矛 %%%%%%%%%%
\subsection*{矛}\addcontentsline{loh}{figure}{矛}

\begin{Entry}{矛}{5}{⽭}[Kangxi 110]
  \begin{Phonetics}{矛}{mao2}
    \definition{s.}{Arcaico: lança; lanceta}
  \end{Phonetics}
\end{Entry}

\begin{Entry}{矛头}{5,5}{⽭,⼤}
  \begin{Phonetics}{矛头}{mao2tou2}[][HSK 7-9]
    \definition{s.}{ponta de lança; lança}
  \end{Phonetics}
\end{Entry}

\begin{Entry}{矛盾}{5,9}{⽭,⽬}
  \begin{Phonetics}{矛盾}{mao2dun4}[][HSK 5]
    \definition{adj.}{contraditório; descreve pessoas ou coisas que se opõem ou se repelem mutuamente}
    \definition[对,个,种]{s.}{problema; contradição; discrepância; inconsistência | disputas e conflitos; relacionamento de oposição entre as duas partes devido a diferenças de opinião ou abordagem}
    \definition{v.}{opor"-se; entrar em conflito; contradizer; nesta situação, apenas uma das opções está correta ou é verdadeira; não é possível que ambas estejam corretas ao mesmo tempo}
  \end{Phonetics}
\end{Entry}

%%%%%%%%%% 矜 %%%%%%%%%%
\subsection*{矜}\addcontentsline{loh}{figure}{矜}

\begin{Entry}{矜}{9}{⽭}
  \begin{Phonetics}{矜}{jin1}
    \definition{adj.}{presunçoso; vaidoso | contido; reservado; determinado}
    \definition{v.}{ter pena; simpatizar com; compadecer-se}
  \end{Phonetics}
\end{Entry}

%%%%% EOF %%%%%

