%%%
%%% Radical "⼹"
%%%
\section*{Radical 58: ``⼹'' (彑)}\addcontentsline{toc}{section}{Radical 58: ⼹,彑}\addcontentsline{loh}{figure}{\#\#\#\# 58: ⼹}

%%%%%%%%%% 归 %%%%%%%%%%
\subsection*{归}\addcontentsline{loh}{figure}{归}

\begin{Entry}{归}{5}{⼹}
  \begin{Phonetics}{归}{gui1}[][HSK 4]
    \definition*{s.}{Sobrenome: Gui}
    \definition{s.}{divisão no ábaco com divisor de um dígito}
    \definition{v.}{retornar; voltar para; voltar (ou ir) | devolver algo a; dar de volta a | convergir; juntar-se | encarregar alguém de algo | atribuir a; pertencer a}
    \definition{v.aux.}{usado entre dois verbos idênticos, indicando que a ação não levou ao resultado correspondente}
  \end{Phonetics}
\end{Entry}

\begin{Entry}{归来}{5,7}{⼹,⽊}
  \begin{Phonetics}{归来}{gui1lai2}[][HSK 7-9]
    \definition{v.}{retornar; voltar ou estar de volta; retornar ao local de onde você começou ou partiu de outro lugar}
  \end{Phonetics}
\end{Entry}

\begin{Entry}{归纳}{5,7}{⼹,⽷}
  \begin{Phonetics}{归纳}{gui1na4}[][HSK 7-9]
    \definition{s.}{indução; método indutivo}
    \definition{v.}{induzir; concluir; mesclar e classificar; resumir (usado principalmente para coisas abstratas)}
  \end{Phonetics}
\end{Entry}

\begin{Entry}{归还}{5,7}{⼹,⾡}
  \begin{Phonetics}{归还}{gui1huan2}[][HSK 7-9]
    \definition{v.}{retornar; reverter; devolver dinheiro ou itens emprestados ao proprietário original}
  \antonymref{借用}{jie4yong4}
  \end{Phonetics}
\end{Entry}

\begin{Entry}{归结}{5,9}{⼹,⽷}
  \begin{Phonetics}{归结}{gui1jie2}[][HSK 7-9]
    \definition{s.}{fim; final (de uma história, etc.) | resolução}
    \definition{v.}{chegar a uma conclusão; resumir; colocar em poucas palavras}
  \end{Phonetics}
\end{Entry}

\begin{Entry}{归根到底}{5,10,8,8}{⼹,⽊,⼑,⼴}
  \begin{Phonetics}{归根到底}{gui1gen1-dao4di3}[][HSK 7-9]
    \definition{expr.}{``Em última análise.'', significa que, no final, as coisas acabarão de uma certa maneira; ela vem de 《何典》, de 张南庄, da Dinastia Qing (清); na análise final (última); no longo prazo; afinal; na análise final; em essência; fundamentalmente}
  \seealsoref{何典}{he2 dian3}
  \seealsoref{清}{qing1}
  \seealsoref{张南庄}{zhang1 nan2zhuang1}
  \end{Phonetics}
\end{Entry}

\begin{Entry}{归宿}{5,11}{⼹,⼧}
  \begin{Phonetics}{归宿}{gui1su4}[][HSK 7-9]
    \definition{s.}{um lar para retornar; destino final; fim}
  \end{Phonetics}
\end{Entry}

\begin{Entry}{归属}{5,12}{⼹,⼫}
  \begin{Phonetics}{归属}{gui1shu3}[][HSK 7-9]
    \definition{v.}{pertencer a; estar sob a jurisdição de; definir afiliação}
  \end{Phonetics}
\end{Entry}

%%%%%%%%%% 当 %%%%%%%%%%
\subsection*{当}\addcontentsline{loh}{figure}{当}

\begin{Entry}{当}{6}{⼹}
  \begin{Phonetics}{当}{dang1}[][HSK 2]
    \definition*{s.}{Sobrenome: Dang}
    \definition{adj.}{igual; adequado; compatível}
    \definition{prep.}{na presença de alguém; na cara de alguém | exatamente em (um momento ou lugar); em algum momento, em algum lugar | na frente de alguém}
    \definition{s.}{Onomatopéia: barulho metálico, som de um gongo ou sino}
    \definition{s.}{topo; cume | uma lacuna no espaço ou no tempo; refere"-se a um espaço ou intervalo de tempo}
    \definition{v.}{dever; ter que; dever ser | trabalhar como; servir como; ser; assumir; desempenhar a função de | suportar; aceitar; merecer | dirigir; gerenciar; estar no comando; ser responsável por; presidir | conter; bloquear; segurar; reter; resistir}
  \end{Phonetics}
  \begin{Phonetics}{当}{dang4}[][HSK 6]
    \definition{adj.}{adequado; correto; apropriado | igual; o mesmo}
    \definition{pron.}{naquele mesmo (dia, etc.); refere"-se ao momento em que algo aconteceu}
    \definition{s.}{algo penhorado; penhor; garantia; objetos físicos penhorados em casas de penhores}
    \definition{v.}{corresponder; ser igual a; combinar | tratar como; considerar como; tomar como | pensar que; achar que | penhorar; empréstimo com garantia real em uma loja de penhores}
  \end{Phonetics}
\end{Entry}

\begin{Entry}{当下}{6,3}{⼹,⼀}
  \begin{Phonetics}{当下}{dang1xia4}[][HSK 7-9]
    \definition{adv.}{instantaneamente; imediatamente; de uma vez}
    \definition{s.}{o tempo presente}
  \end{Phonetics}
\end{Entry}

\begin{Entry}{当之无愧}{6,3,4,12}{⼹,⼂,⽆,⼼}
  \begin{Phonetics}{当之无愧}{dang1zhi1wu2kui4}[][HSK 7-9]
    \definition{expr.}{merecer plenamente (um título, uma honra, etc.); merecer a recompensa; ser merecedor | ser digno de; ser digno do nome}
  \end{Phonetics}
\end{Entry}

\begin{Entry}{当中}{6,4}{⼹,⼁}
  \begin{Phonetics}{当中}{dang1zhong1}[][HSK 3]
    \definition{prep.}{no meio; no centro | entre; dentro}
  \end{Phonetics}
\end{Entry}

\begin{Entry}{当今}{6,4}{⼹,⼈}
  \begin{Phonetics}{当今}{dang1jin1}[][HSK 7-9]
    \definition{s.}{o presente; hoje | Arcaico: imperador no trono; imperador reinante | agora; no presente; hoje em dia}
  \end{Phonetics}
\end{Entry}

\begin{Entry}{当天}{6,4}{⼹,⼤}
  \begin{Phonetics}{当天}{dang1tian1}[][HSK 6]
    \definition{s.}{no mesmo dia; naquele mesmo dia; refere"-se ao dia em que algo aconteceu no passado}
  \end{Phonetics}
\end{Entry}

\begin{Entry}{当心}{6,4}{⼹,⼼}
  \begin{Phonetics}{当心}{dang1xin1}[][HSK 7-9]
    \definition{s.}{centro; o centro do peito}
    \definition{v.}{ter cuidado com; ter cuidado}
  \end{Phonetics}
\end{Entry}

\begin{Entry}{当日}{6,4}{⼹,⽇}
  \begin{Phonetics}{当日}{dang1ri4}[][HSK 7-9]
    \definition[点]{s.}{nessa ocasião; naquela época; no mesmo dia; naquele mesmo dia}
  \end{Phonetics}
  \begin{Phonetics}{当日}{dang4ri4}
    \definition[点]{s.}{mesmo dia; naquele mesmo dia; refere"-se ao mesmo dia em que algo aconteceu; (neste) dia}
  \end{Phonetics}
\end{Entry}

\begin{Entry}{当代}{6,5}{⼹,⼈}
  \begin{Phonetics}{当代}{dang1dai4}[][HSK 5]
    \definition{s.}{a era atual; a era contemporânea}
  \end{Phonetics}
\end{Entry}

\begin{Entry}{当务之急}{6,5,3,9}{⼹,⼒,⼂,⼼}
  \begin{Phonetics}{当务之急}{dang1wu4zhi1ji2}[][HSK 7-9]
    \definition{expr.}{assunto mais urgente do momento; uma tarefa de alta prioridade; assunto urgente | assunto de vital importância; preocupações | trabalho de alta prioridade}
  \end{Phonetics}
\end{Entry}

\begin{Entry}{当众}{6,6}{⼹,⼈}
  \begin{Phonetics}{当众}{dang1zhong4}[][HSK 7-9]
    \definition{adv.}{abertamente; publicamente; em público; diante do público; na presença de todos; na frente de todos; de frente para a multidão}
  \end{Phonetics}
\end{Entry}

\begin{Entry}{当地}{6,6}{⼹,⼟}
  \begin{Phonetics}{当地}{dang1di4}[][HSK 3]
    \definition{s.}{local; o lugar onde as pessoas e as coisas estão ou onde as coisas acontecem}
  \end{Phonetics}
\end{Entry}

\begin{Entry}{当场}{6,6}{⼹,⼟}
  \begin{Phonetics}{当场}{dang1chang3}[][HSK 5]
    \definition{adv.}{na hora; de imediato; na mesma hora}
  \end{Phonetics}
\end{Entry}

\begin{Entry}{当年}{6,6}{⼹,⼲}
  \begin{Phonetics}{当年}{dang1nian2}[][HSK 5]
    \definition{s.}{aqueles anos (ou dias) | naqueles anos (ou dias) | durante esse tempo}
    \definition{v.}{estar no auge da vida}
  \end{Phonetics}
  \begin{Phonetics}{当年}{dang4nian2}
    \definition{s.}{no mesmo ano; naquele mesmo ano}
  \end{Phonetics}
\end{Entry}

\begin{Entry}{当成}{6,6}{⼹,⼽}
  \begin{Phonetics}{当成}{dang4cheng2}[][HSK 6]
    \definition{v.}{considerar como; tratar como; tomar por}
  \end{Phonetics}
\end{Entry}

\begin{Entry}{当作}{6,7}{⼹,⼈}
  \begin{Phonetics}{当作}{dang4zuo4}[][HSK 6]
    \definition{v.}{tratar como; considerar como}
  \end{Phonetics}
\end{Entry}

\begin{Entry}{当初}{6,7}{⼹,⾐}
  \begin{Phonetics}{当初}{dang1chu1}[][HSK 3]
    \definition{s.}{no começo; originalmente; no início; em primeiro lugar; refere"-se a algo que aconteceu no passado, seja em geral ou especificamente}
  \end{Phonetics}
\end{Entry}

\begin{Entry}{当即}{6,7}{⼹,⼙}
  \begin{Phonetics}{当即}{dang1ji2}[][HSK 7-9]
    \definition{adv.}{imediatamente}
  \end{Phonetics}
\end{Entry}

\begin{Entry}{当时}{6,7}{⼹,⽇}
  \begin{Phonetics}{当时}{dang1shi2}[][HSK 2]
    \definition{s.}{naquela época; aquela ocasião; aquela vez; refere"-se a algo que aconteceu no passado}
    \definition{v.}{ser o momento adequado; acontecer no momento certo}
  \end{Phonetics}
  \begin{Phonetics}{当时}{dang4shi2}
    \definition{adv.}{(depois de fazer algo ou algo acontecer) imediatamente; de imediato; agora mesmo}
  \end{Phonetics}
\end{Entry}

\begin{Entry}{当事人}{6,8,2}{⼹,⼅,⼈}
  \begin{Phonetics}{当事人}{dang1shi4ren2}[][HSK 7-9]
    \definition{s.}{litigante; parte (em um processo judicial); refere"-se especificamente a pessoas que têm uma relação direta com os fatos do caso, como a vítima, o promotor particular, o réu, etc. em um processo criminal | partes interessadas; pessoa (ou parte) envolvida; alguém que tem uma relação direta com algo}
  \end{Phonetics}
\end{Entry}

\begin{Entry}{当前}{6,9}{⼹,⼑}
  \begin{Phonetics}{当前}{dang1qian2}[][HSK 5]
    \definition{s.}{presente; atual}
    \definition{v.}{estar diante de alguém; estar frente a frente com alguém; na frente de, geralmente refere"-se a uma situação perigosa}
  \end{Phonetics}
\end{Entry}

\begin{Entry}{当选}{6,9}{⼹,⾡}
  \begin{Phonetics}{当选}{dang1xuan3}[][HSK 5]
    \definition{v.}{ser eleito}
  \end{Phonetics}
\end{Entry}

\begin{Entry}{当面}{6,9}{⼹,⾯}
  \begin{Phonetics}{当面}{dang1mian4}[][HSK 7-9]
    \definition{adv.}{na cara de alguém; na presença de alguém; cara a cara}
  \end{Phonetics}
\end{Entry}

\begin{Entry}{当真}{6,10}{⼹,⼗}
  \begin{Phonetics}{当真}{dang4zhen1}[][HSK 7-9]
    \definition{adj.}{verdadeiro; real; confiável}
    \definition{adv.}{realmente; verdadeiramente}
    \definition{v.}{levar a sério; acreditar}
  \end{Phonetics}
\end{Entry}

\begin{Entry}{当晚}{6,11}{⼹,⽇}
  \begin{Phonetics}{当晚}{dang1wan3}
    \definition{s.}{naquela noite; esta noite; a mesma noite}
  \end{Phonetics}
  \begin{Phonetics}{当晚}{dang4wan3}[][HSK 7-9]
    \definition{s.}{na mesma noite; esta noite}
  \end{Phonetics}
\end{Entry}

\begin{Entry}{当着}{6,11}{⼹,⽬}
  \begin{Phonetics}{当着}{dang1zhe5}[][HSK 7-9]
    \definition{prep.}{na frente de | na presença de}
  \end{Phonetics}
\end{Entry}

\begin{Entry}{当然}{6,12}{⼹,⽕}
  \begin{Phonetics}{当然}{dang1ran2}[][HSK 3]
    \definition{adj.}{natural; verdadeiro; espontâneo}
    \definition{adv.}{sem dúvida; certamente; claro}
  \end{Phonetics}
\end{Entry}

%%%%%%%%%% 录 %%%%%%%%%%
\subsection*{录}\addcontentsline{loh}{figure}{录}

\begin{Entry}{录}{8}{⼹}
  \begin{Phonetics}{录}{lu4}[][HSK 3]
    \definition{s.}{registro; cadastro; coleção; seleções}
    \definition{v.}{copiar; gravar; escrever; copiar; registrar | contratar; selecionar; empregar; adotar ou nomear | gravar em fita magnética}
  \end{Phonetics}
\end{Entry}

\begin{Entry}{录制}{8,8}{⼹,⼑}
  \begin{Phonetics}{录制}{lu4zhi4}[][HSK 7-9]
    \definition{v.}{gravar som ou imagem usando um gravador de fita ou de vídeo; processar e criar uma obra de arte}
  \end{Phonetics}
\end{Entry}

\begin{Entry}{录取}{8,8}{⼹,⼜}
  \begin{Phonetics}{录取}{lu4qu3}[][HSK 4]
    \definition{v.}{aceitar; admitir; recrutar; entrar; matricular (os aprovados no exame)}
  \end{Phonetics}
\end{Entry}

\begin{Entry}{录音}{8,9}{⼹,⾳}
  \begin{Phonetics}{录音}{lu4/yin1}[][HSK 3]
    \definition[段,个]{s.}{gravação de som; som gravado com equipamento especializado}
    \definition{v.+compl.}{gravar; converter o som em sinal elétrico e, em seguida, gravá-lo por meios mecânicos, ópticos ou eletromagnéticos}
  \end{Phonetics}
\end{Entry}

\begin{Entry}{录音机}{8,9,6}{⼹,⾳,⽊}
  \begin{Phonetics}{录音机}{lu4yin1ji1}[][HSK 6]
    \definition[台]{s.}{gravador de som; máquina de gravação (de fita)}
  \end{Phonetics}
\end{Entry}

\begin{Entry}{录像}{8,13}{⼹,⼈}
  \begin{Phonetics}{录像}{lu4/xiang4}[][HSK 6]
    \definition[段,个,些,盘]{s.}{vídeo; gravação; fita de vídeo; imagens gravadas com celulares, câmeras, etc.}
    \definition{v.+compl.}{gravar bídeo; gravar em fita de vídeo | usar celulares, câmeras e outros dispositivos para salvar registros de vídeo}
  \end{Phonetics}
\end{Entry}

\begin{Entry}{录像机}{8,13,6}{⼹,⼈,⽊}
  \begin{Phonetics}{录像机}{lu4xiang4ji1}
    \definition[台]{s.}{gravador de vídeo | VCR}
  \end{Phonetics}
\end{Entry}

\begin{Entry}{录像带}{8,13,9}{⼹,⼈,⼱}
  \begin{Phonetics}{录像带}{lu4xiang4dai4}
    \definition[盘]{s.}{video-cassete}
  \end{Phonetics}
\end{Entry}

%%%%% EOF %%%%%

