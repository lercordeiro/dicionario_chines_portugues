%%%
%%% Radical "⾈"
%%%
\section*{Radical 137: ``⾈''}\addcontentsline{toc}{section}{Radical 137: ⾈}\addcontentsline{loh}{figure}{\#\#\#\# 137: ⾈}

%%%%%%%%%% 航 %%%%%%%%%%
\subsection*{航}\addcontentsline{loh}{figure}{航}

\begin{Entry}{航}{10}{⾈}
  \begin{Phonetics}{航}{hang2}
    \definition*{s.}{Sobrenome: Hang}
    \definition[趟]{s.}{barco; navio}
    \definition{v.}{navegar (por água ou ar) | velejar}
  \end{Phonetics}
\end{Entry}

\begin{Entry}{航天}{10,4}{⾈,⼤}
  \begin{Phonetics}{航天}{hang2tian1}[][HSK 7-9]
    \definition{s.}{voo espacial; astronáutica}
    \definition{v.}{voar ou viajar no espaço}
  \end{Phonetics}
\end{Entry}

\begin{Entry}{航天员}{10,4,7}{⾈,⼤,⼝}
  \begin{Phonetics}{航天员}{hang2tian1yuan2}[][HSK 7-9]
    \definition[名,位,个]{s.}{astronauta}
  \end{Phonetics}
\end{Entry}

\begin{Entry}{航行}{10,6}{⾈,⾏}
  \begin{Phonetics}{航行}{hang2xing2}[][HSK 7-9]
    \definition{v.}{velejar; voar; navegar pela água, pelo ar}
  \end{Phonetics}
\end{Entry}

\begin{Entry}{航运}{10,7}{⾈,⾡}
  \begin{Phonetics}{航运}{hang2yun4}[][HSK 7-9]
    \definition{s.}{transporte hidroviário; transporte marítimo}
  \end{Phonetics}
\end{Entry}

\begin{Entry}{航空}{10,8}{⾈,⽳}
  \begin{Phonetics}{航空}{hang2kong1}[][HSK 4]
    \definition{s.}{viagem; aviação; refere"-se ao voo de uma aeronave no ar}
  \end{Phonetics}
\end{Entry}

\begin{Entry}{航海}{10,10}{⾈,⽔}
  \begin{Phonetics}{航海}{hang2hai3}[][HSK 7-9]
    \definition{v.}{velejar; navegar}
  \end{Phonetics}
\end{Entry}

\begin{Entry}{航班}{10,10}{⾈,⽟}
  \begin{Phonetics}{航班}{hang2ban1}[][HSK 4]
    \definition[个,次]{s.}{número do voo; voo programado; o horário de um navio ou avião de passageiros}
  \end{Phonetics}
\end{Entry}

%%%%%%%%%% 般 %%%%%%%%%%
\subsection*{般}\addcontentsline{loh}{figure}{般}

\begin{Entry}{般}{10}{⾈}
  \begin{Phonetics}{般}{ban1}
    \definition{clas.}{tipo; classe; gênero; amostra}
    \definition{part.}{(o mesmo) que; como; semelhante}
  \end{Phonetics}
  \begin{Phonetics}{般}{bo1}
    \definition{s.}{utilizado em 般若}
  \seealsoref{般若}{bo1re3}
  \end{Phonetics}
  \begin{Phonetics}{般}{pan2}
    \definition{adj.}{feliz; bem-aventurado}
  \end{Phonetics}
\end{Entry}

\begin{Entry}{般乐}{10,5}{⾈,⼃}
  \begin{Phonetics}{般乐}{pan2le4}
    \definition{v.}{jogar | divertir-se}
  \end{Phonetics}
\end{Entry}

\begin{Entry}{般若}{10,8}{⾈,⾋}
  \begin{Phonetics}{般若}{bo1re3}
    \definition*{s.}{Prajña (sânscrito), \emph{insight} sobre a verdadeira natureza da realidade}
    \definition{s.}{budismo: sabedoria}
  \end{Phonetics}
\end{Entry}

%%%%%%%%%% 舱 %%%%%%%%%%
\subsection*{舱}\addcontentsline{loh}{figure}{舱}

\begin{Entry}{舱}{10}{⾈}
  \begin{Phonetics}{舱}{cang1}[][HSK 7-9]
    \definition{s.}{cabine (de um avião ou navio) | módulo (de uma nave espacial) | espaço em um navio ou aeronave para transportar pessoas, carga ou máquinas}
  \end{Phonetics}
\end{Entry}

%%%%%%%%%% 舵 %%%%%%%%%%
\subsection*{舵}\addcontentsline{loh}{figure}{舵}

\begin{Entry}{舵}{11}{⾈}
  \begin{Phonetics}{舵}{duo4}
    \definition{s.}{leme; dispositivos para controlar a direção de navios, aeronaves, etc.}
  \seealsoref{柁}{tuo2}
  \end{Phonetics}
\end{Entry}

\begin{Entry}{舵手}{11,4}{⾈,⼿}
  \begin{Phonetics}{舵手}{duo4shou3}[][HSK 7-9]
    \definition{s.}{timoneiro}
  \end{Phonetics}
\end{Entry}

%%%%%%%%%% 船 %%%%%%%%%%
\subsection*{船}\addcontentsline{loh}{figure}{船}

\begin{Entry}{船}{11}{⾈}
  \begin{Phonetics}{船}{chuan2}[][HSK 2]
    \definition*{s.}{Sobrenome: Chuan}
    \definition[条,艘,叶,只]{s.}{barco; navio | embarcação; meio de transporte aquático, nome genérico para embarcações}
  \end{Phonetics}
\end{Entry}

\begin{Entry}{船长}{11,4}{⾈,⾧}
  \begin{Phonetics}{船长}{chuan2zhang3}[][HSK 6]
    \definition{s.}{capitão do navio; mestre; marinheiro; comandante; o oficial chefe a bordo}
  \end{Phonetics}
\end{Entry}

\begin{Entry}{船只}{11,5}{⾈,⼝}
  \begin{Phonetics}{船只}{chuan2zhi1}[][HSK 6]
    \definition[艘,条]{s.}{transporte marítimo; embarcação | navio; veleiro}
  \end{Phonetics}
\end{Entry}

\begin{Entry}{船员}{11,7}{⾈,⼝}
  \begin{Phonetics}{船员}{chuan2yuan2}[][HSK 6]
    \definition[名,位,个]{s.}{tripulação (do navio) | membro da tripulação (do navio); marinheiro; marujo; barqueiro; velejador}
  \end{Phonetics}
\end{Entry}

\begin{Entry}{船桨}{11,10}{⾈,⽊}
  \begin{Phonetics}{船桨}{chuan2jiang3}[][HSK 7-9]
    \definition{s.}{remo}
  \end{Phonetics}
\end{Entry}

\begin{Entry}{船舶}{11,11}{⾈,⾈}
  \begin{Phonetics}{船舶}{chuan2bo2}[][HSK 7-9]
    \definition[艘,条,只]{s.}{transporte marítimo; barcos e navios; refere"-se a vários navios}
  \end{Phonetics}
\end{Entry}

%%%%%%%%%% 艁 %%%%%%%%%%
\subsection*{艁}\addcontentsline{loh}{figure}{艁}

\begin{Entry}{艁}{13}{⾈}
  \begin{Phonetics}{艁}{zao4}
    \variantof{造}
  \end{Phonetics}
\end{Entry}

%%%%%%%%%% 艘 %%%%%%%%%%
\subsection*{艘}\addcontentsline{loh}{figure}{艘}

\begin{Entry}{艘}{15}{⾈}
  \begin{Phonetics}{艘}{sou1}[][HSK 7-9]
    \definition{clas.}{utilizado para barcos grandes, navios}
  \end{Phonetics}
\end{Entry}

%%%%% EOF %%%%%

