%%%
%%% Radical "⾑"
%%%
\section*{Radical 146: ``⾑'' (西、覀)}\addcontentsline{toc}{section}{Radical 146: ⾑、西、覀}\addcontentsline{loh}{figure}{\#\#\#\# 146: ⾑}

%%%%%%%%%% 西 %%%%%%%%%%
\subsection*{西}\addcontentsline{loh}{figure}{西}

\begin{Entry}{西}{6}{⾑}[Kangxi 146]
  \begin{Phonetics}{西}{xi1}[][HSK 1]
    \definition*{s.}{Espanha, abreviatura de 西班牙 | Paraíso Ocidental | Sobrenome: Xi}
    \definition{s.}{oeste; uma das quatro direções básicas, o lado onde o sol se põe (oposto ao 东) | ocidental; refere-se ao Ocidente (principalmente aos países europeus e americanos) | aqui e ali; em contraposição a 东, significa 到处 ou 零散, 没有次序}
  \seealsoref{到处}{dao4chu4}
  \seealsoref{东}{dong1}
  \seealsoref{零散}{ling2san3}
  \seealsoref{没有次序}{mei2you3 ci4xu4}
  \seealsoref{西班牙}{xi1ban1ya2}
  \end{Phonetics}
\end{Entry}

\begin{Entry}{西文}{6,4}{⾑,⽂}
  \begin{Phonetics}{西文}{xi1wen2}
    \definition{s.}{espanhol | língua espanhola}
  \seealsoref{西班牙文}{xi1ban1ya2wen2}
  \end{Phonetics}
\end{Entry}

\begin{Entry}{西方}{6,4}{⾑,⽅}
  \begin{Phonetics}{西方}{xi1 fang1}[][HSK 2]
    \definition{s.}{oeste | o Ocidente; o Oeste; países europeus e americanos | Paraíso Ocidental, termo budista}
  \end{Phonetics}
\end{Entry}

\begin{Entry}{西兰花}{6,5,7}{⾑,⼋,⾋}
  \begin{Phonetics}{西兰花}{xi1lan2hua1}
    \definition{s.}{brócolis}
  \end{Phonetics}
\end{Entry}

\begin{Entry}{西北}{6,5}{⾑,⼔}
  \begin{Phonetics}{西北}{xi1 bei3}[][HSK 2]
    \definition{s.}{noroeste | noroeste da China; o Noroeste}
  \end{Phonetics}
\end{Entry}

\begin{Entry}{西半球}{6,5,11}{⾑,⼗,⽟}
  \begin{Phonetics}{西半球}{xi1ban4qiu2}
    \definition{s.}{hemisfério oeste}
  \end{Phonetics}
\end{Entry}

\begin{Entry}{西瓜}{6,5}{⾑,⽠}
  \begin{Phonetics}{西瓜}{xi1gua1}[][HSK 4]
    \definition[个,颗,粒]{s.}{melancia; fruto que é uma baga de formato grande, globular ou oval, com muita polpa aguada e doce}
  \end{Phonetics}
\end{Entry}

\begin{Entry}{西边}{6,5}{⾑,⾡}
  \begin{Phonetics}{西边}{xi1bian1}[][HSK 1]
    \definition{s.}{lado oeste; (oeste) Uma das quatro direções principais; uma das direções cardeais, oposta ao 东方}
  \seealsoref{东方}{dong1 fang1}
  \end{Phonetics}
\end{Entry}

\begin{Entry}{西安}{6,6}{⾑,⼧}
  \begin{Phonetics}{西安}{xi1'an1}
    \definition*{s.}{Xi'an, Capital da Província de Shaanxi}
  \end{Phonetics}
\end{Entry}

\begin{Entry}{西红柿}{6,6,9}{⾑,⽷,⽊}
  \begin{Phonetics}{西红柿}{xi1hong2shi4}[][HSK 5]
    \definition[种,只,株]{s.}{tomate}
  \end{Phonetics}
\end{Entry}

\begin{Entry}{西西}{6,6}{⾑,⾑}
  \begin{Phonetics}{西西}{xi1xi1}
    \definition{num.}{centímetro cúbico}
  \end{Phonetics}
\end{Entry}

\begin{Entry}{西医}{6,7}{⾑,⼖}
  \begin{Phonetics}{西医}{xi1 yi1}[][HSK 2]
    \definition[名,位]{s.}{medicina ocidental; medicina introduzida na China a partir da Europa e da América | um médico treinado em medicina ocidental}
  \end{Phonetics}
\end{Entry}

\begin{Entry}{西南}{6,9}{⾑,⼗}
  \begin{Phonetics}{西南}{xi1 nan2}[][HSK 2]
    \definition{s.}{sudoeste | o Sudoeste; Sudoeste da China}
  \end{Phonetics}
\end{Entry}

\begin{Entry}{西药}{6,9}{⾑,⾋}
  \begin{Phonetics}{西药}{xi1 yao4}
    \definition[片,粒]{s.}{medicina ocidental; refere-se aos medicamentos usados ​​na medicina ocidental, geralmente feitos por métodos sintéticos ou extraídos de produtos naturais, como comprimidos anti-inflamatórios, aspirina, tintura de iodo, penicilina, etc.}
  \end{Phonetics}
\end{Entry}

\begin{Entry}{西语}{6,9}{⾑,⾔}
  \begin{Phonetics}{西语}{xi1yu3}
    \definition{s.}{línguas ocidentais | espanhol | língua espanhola}
  \seealsoref{西班牙语}{xi1 ban1 ya2 yu3}
  \end{Phonetics}
\end{Entry}

\begin{Entry}{西面}{6,9}{⾑,⾯}
  \begin{Phonetics}{西面}{xi1mian4}
    \definition{s.}{oeste | lado oeste}
  \end{Phonetics}
\end{Entry}

\begin{Entry}{西班牙}{6,10,4}{⾑,⽟,⽛}
  \begin{Phonetics}{西班牙}{xi1ban1ya2}
    \definition*{s.}{Espanha}
  \end{Phonetics}
\end{Entry}

\begin{Entry}{西班牙文}{6,10,4,4}{⾑,⽟,⽛,⽂}
  \begin{Phonetics}{西班牙文}{xi1ban1ya2wen2}
    \definition{s.}{espanhol, língua espanhola}
  \seealsoref{西文}{xi1wen2}
  \end{Phonetics}
\end{Entry}

\begin{Entry}{西班牙语}{6,10,4,9}{⾑,⽟,⽛,⾔}
  \begin{Phonetics}{西班牙语}{xi1 ban1 ya2 yu3}[][HSK 6]
    \definition[句]{s.}{espanhol | língua espanhola}
  \seealsoref{西语}{xi1yu3}
  \end{Phonetics}
\end{Entry}

\begin{Entry}{西部}{6,10}{⾑,⾢}
  \begin{Phonetics}{西部}{xi1 bu4}[][HSK 3]
    \definition{s.}{(EUA) filme de faroeste; filme de \emph{cowboys}; um faroeste | filme da região ocidental (China) | parte ocidental; região oeste da China}
  \end{Phonetics}
\end{Entry}

\begin{Entry}{西装}{6,12}{⾑,⾐}
  \begin{Phonetics}{西装}{xi1 zhuang1}[][HSK 5]
    \definition[件,套,个]{s.}{terno; roupas de estilo ocidental; roupas ocidentais, divididas em masculinas e femininas}
  \end{Phonetics}
\end{Entry}

\begin{Entry}{西蓝花}{6,13,7}{⾑,⾋,⾋}
  \begin{Phonetics}{西蓝花}{xi1lan2hua1}
    \variantof{西兰花}
  \end{Phonetics}
\end{Entry}

\begin{Entry}{西餐}{6,16}{⾑,⾷}
  \begin{Phonetics}{西餐}{xi1 can1}[][HSK 2]
    \definition[份,顿,桌]{s.}{comida ocidental; comida de estilo ocidental, comida com garfo e faca (diferente da 中餐)}
  \seealsoref{中餐}{zhong1 can1}
  \end{Phonetics}
\end{Entry}

\begin{Entry}{西藏}{6,17}{⾑,⾋}
  \begin{Phonetics}{西藏}{xi1zang4}
    \definition*{s.}{Xizang; Região Autônoma do Tibete, 西藏自治区}
  \seealsoref{西藏自治区}{xi1zang4 zi4zhi4qu1}
  \end{Phonetics}
\end{Entry}

\begin{Entry}{西藏自治区}{6,17,6,8,4}{⾑,⾋,⾃,⽔,⼖}
  \begin{Phonetics}{西藏自治区}{xi1zang4 zi4zhi4qu1}
    \definition*{s.}{Região Autônoma do Tibete}
  \end{Phonetics}
\end{Entry}

%%%%%%%%%% 要 %%%%%%%%%%
\subsection*{要}\addcontentsline{loh}{figure}{要}

\begin{Entry}{要}{9}{⾑}
  \begin{Phonetics}{要}{yao1}[][HSK 1]
    \definition*{s.}{Sobrenome: Yao}
    \definition{v.}{exigir; pedir; requerer; solicitar; buscar; insistir com base em algo em que se apoia | forçar; coagir; ameaçar}
  \end{Phonetics}
  \begin{Phonetics}{要}{yao4}[][HSK 4]
    \definition{adj.}{importante; essencial}
    \definition{conj.}{suponha; no caso; se, indicando um relacionamento hipotético | ou; ou\dots ou\dots}
    \definition{s.}{ponto principal; manchete; conteúdo importante}
    \definition{v.}{querer; desejar; pensar | querer; pedir; deseja; querer obter; querer manter | recuperar algo; dizer a alguém para guardar algo para você ou devolver | pedir (ou querer) que alguém faça algo; pedir a alguém para fazer algo, quando usado para conseguir que alguém faça algo, tem um tom de comando e pode ser indelicado | precisar; tomar; pegar | deve; deveria; é necessário (imperativo, essencial) que\dots | estar indo para | querer; ter um desejo por; expressar determinação ou desejo de fazer algo | poder; dever;  indica uma estimativa, usada para comparação}
  \seealsoref{要是}{yao4shi5}
  \end{Phonetics}
\end{Entry}

\begin{Entry}{要么}{9,3}{⾑,⼃}
  \begin{Phonetics}{要么}{yao4 me5}[][HSK 6]
    \definition{conj.}{ou; ou\dots ou\dots; indica uma escolha entre duas situações ou dois desejos}
  \seealsoref{要么…要么…}{yao4 me5 yao4 me5}
  \end{Phonetics}
\end{Entry}

\begin{Entry}{要么…要么…}{9,3,9,3}{⾑,⼃,⾑,⼃}
  \begin{Phonetics}{要么…要么…}{yao4 me5 yao4 me5}[][HSK 6]
    \definition{conj.}{ou\dots ou\dots}
  \seealsoref{要么}{yao4 me5}
  \end{Phonetics}
\end{Entry}

\begin{Entry}{要义}{9,3}{⾑,⼂}
  \begin{Phonetics}{要义}{yao4yi4}
    \definition{s.}{resumo | o essencial}
  \end{Phonetics}
\end{Entry}

\begin{Entry}{要不}{9,4}{⾑,⼀}
  \begin{Phonetics}{要不}{yao4 bu4}
    \definition{conj.}{ou então; caso contrário; se você não fizer isso (haverá um resultado ruim) | usado para propor educadamente; usado para fazer uma sugestão educadamente | ou; se você não fizer isso, faça aquilo}
  \end{Phonetics}
\end{Entry}

\begin{Entry}{要不然}{9,4,12}{⾑,⼀,⽕}
  \begin{Phonetics}{要不然}{yao4 bu4 ran2}[][HSK 6]
    \definition{conj.}{caso contrário; ou então; se você não fizer isso (haverá um resultado ruim) | ou então; usado entre duas frases em um relacionamento de escolha; significa escolher uma entre as duas; equivalente a 要不}
  \seealsoref{要不}{yao4 bu4}
  \end{Phonetics}
\end{Entry}

\begin{Entry}{要好}{9,6}{⾑,⼥}
  \begin{Phonetics}{要好}{yao4 hao3}[][HSK 6]
    \definition{adj.}{estar em bons termos; ser amigos próximos; relacionamento harmonioso | estar ansioso para melhorar a si mesmo; esforçar-se para progredir | ansioso para melhorar a si mesmo; esforçar-se para progredir}
  \end{Phonetics}
\end{Entry}

\begin{Entry}{要死}{9,6}{⾑,⽍}
  \begin{Phonetics}{要死}{yao4si3}
    \definition{adv.}{extremamente | muito}
  \end{Phonetics}
\end{Entry}

\begin{Entry}{要求}{9,7}{⾑,⽔}
  \begin{Phonetics}{要求}{yao1qiu2}[][HSK 2]
    \definition[个,点]{s.}{exigência; demanda; reivindicação; desejos ou condições específicas propostas}
    \definition{v.}{pedir; exigir; exigir; reivindicar; apresentar desejos ou condições específicas, esperando que sejam satisfeitos ou realizados}
  \end{Phonetics}
\end{Entry}

\begin{Entry}{要挟}{9,9}{⾑,⼿}
  \begin{Phonetics}{要挟}{yao1xie2}
    \definition{v.}{chantagear | ameaçar}
  \end{Phonetics}
\end{Entry}

\begin{Entry}{要是}{9,9}{⾑,⽇}
  \begin{Phonetics}{要是}{yao4shi5}[][HSK 3]
    \definition{conj.}{se; suponha; no caso de; conecta frases, expressa uma relação hipotética, equivalente a 如果, e pode ser usado em conjunto com 的话}
  \seealsoref{的话}{de5 hua4}
  \seealsoref{如果}{ru2guo3}
  \end{Phonetics}
\end{Entry}

\begin{Entry}{要是…的话}{9,9,8,8}{⾑,⽇,⽩,⾔}
  \begin{Phonetics}{要是…的话}{yao4shi5 de5hua4}
    \definition{conj.}{se for assim\dots}
  \end{Phonetics}
\end{Entry}

\begin{Entry}{要点}{9,9}{⾑,⽕}
  \begin{Phonetics}{要点}{yao4dian3}
    \definition{s.}{pontos principais | essencial}
  \end{Phonetics}
\end{Entry}

\begin{Entry}{要素}{9,10}{⾑,⽷}
  \begin{Phonetics}{要素}{yao4su4}[][HSK 6]
    \definition[个]{s.}{fator essencial; elemento-chave; os elementos essenciais que compõem as coisas}
  \end{Phonetics}
\end{Entry}

\begin{Entry}{要谎}{9,11}{⾑,⾔}
  \begin{Phonetics}{要谎}{yao4huang3}
    \definition{v.}{pedir um preço enorme (como primeiro passo de negociação)}
  \end{Phonetics}
\end{Entry}

\begin{Entry}{要强}{9,12}{⾑,⼸}
  \begin{Phonetics}{要强}{yao4qiang2}
    \definition{adj.}{ansioso para se destacar | ansioso para progredir na vida | obstinado}
  \end{Phonetics}
\end{Entry}

%%%%%%%%%% 覆 %%%%%%%%%%
\subsection*{覆}\addcontentsline{loh}{figure}{覆}

\begin{Entry}{覆}{18}{⾑}
  \begin{Phonetics}{覆}{fu4}
    \definition{v.}{cobrir; encapar | derrubar; perturbar; virar de cabeça para baixo}
  \end{Phonetics}
\end{Entry}

\begin{Entry}{覆盆子}{18,9,3}{⾑,⽫,⼦}
  \begin{Phonetics}{覆盆子}{fu4pen2zi5}
    \definition{s.}{framboesa}
  \end{Phonetics}
\end{Entry}

\begin{Entry}{覆盖}{18,11}{⾑,⽫}
  \begin{Phonetics}{覆盖}{fu4gai4}[][HSK 7-9]
    \definition{s.}{vegetação; cobertura vegetal; refere-se às plantas que cobrem o solo}
    \definition{v.}{cobrir}
  \end{Phonetics}
\end{Entry}

%%%%% EOF %%%%%

