%%%
%%% Radical "⽳"
%%%
\section*{Radical 116: ``⽳''}\addcontentsline{toc}{section}{Radical 116: ⽳}\addcontentsline{loh}{figure}{\#\#\#\# 116: ⽳}

%%%%%%%%%% 究 %%%%%%%%%%
\subsection*{究}\addcontentsline{loh}{figure}{究}

\begin{Entry}{究}{7}{⽳}
  \begin{Phonetics}{究}{jiu1}
    \definition{adv.}{na verdade; realmente; afinal}
    \definition{v.}{estudar cuidadosamente; aprofundar; investigar; rastrear}
  \end{Phonetics}
\end{Entry}

\begin{Entry}{究竟}{7,11}{⽳,⾳}
  \begin{Phonetics}{究竟}{jiu1jing4}[][HSK 4]
    \definition{adv.}{de fato; exatamente; usado em frases interrogativas para buscar | afinal de contas, no final; ênfase em fatos ou motivos}
    \definition{s.}{resultado; desfecho; a causa, o efeito ou a história completa do que aconteceu}
  \end{Phonetics}
\end{Entry}

%%%%%%%%%% 穷 %%%%%%%%%%
\subsection*{穷}\addcontentsline{loh}{figure}{穷}

\begin{Entry}{穷}{7}{⽳}
  \begin{Phonetics}{穷}{qiong2}[][HSK 4]
    \definition{adj.}{remoto; isolado; de difícil acesso | pobre; atingido pela pobreza | situação difícil, sem saída}
    \definition{adv.}{completamente | extremamente}
    \definition{v.}{exaurir; esgotar; consmir | ir até o fim; perseguir completamente perseguido; sondar profundamente | gastar}
  \end{Phonetics}
\end{Entry}

\begin{Entry}{穷人}{7,2}{⽳,⼈}
  \begin{Phonetics}{穷人}{qiong2ren2}[][HSK 4]
    \definition[个]{s.}{os pobres; pessoas pobres}
  \end{Phonetics}
\end{Entry}

%%%%%%%%%% 空 %%%%%%%%%%
\subsection*{空}\addcontentsline{loh}{figure}{空}

\begin{Entry}{空}{8}{⽳}
  \begin{Phonetics}{空}{kong1}[][HSK 3]
    \definition*{s.}{Sobrenome: Kong}
    \definition{adj.}{vazio; oco; nulo; não inclui nada; não contém nada ou não tem conteúdo; irrealista}
    \definition{adv.}{por nada; em vão; sem efeito}
    \definition{s.}{céu; ar | vazio; vazio do mundo dos sentidos}
  \end{Phonetics}
  \begin{Phonetics}{空}{kong4}[][HSK 4]
    \definition*{s.}{Sobrenome: Kong}
    \definition{adj.}{vazio; oco; nulo; que não contém nada; que não tem nada ou nenhum conteúdo; impraticável}
    \definition{adv.}{para nada; em vão; sem efeito}
    \definition{s.}{céu; ar | vazio; ausência do mundo dos sentidos}
  \end{Phonetics}
\end{Entry}

\begin{Entry}{空儿}{8,2}{⽳,⼉}
  \begin{Phonetics}{空儿}{kong4r5}[][HSK 3]
    \definition[个]{s.}{tempo livre; sem horário específico | sala; espaço (não utilizado); área ainda não utilizada}
    \definition{v.}{ter tempo livre}
  \end{Phonetics}
\end{Entry}

\begin{Entry}{空中}{8,4}{⽳,⼁}
  \begin{Phonetics}{空中}{kong1zhong1}[][HSK 5]
    \definition{adj.}{aéreo; aerotransportado; refere"-se à transmissão de sinais de rádio}
    \definition{s.}{no céu; no ar}
  \end{Phonetics}
\end{Entry}

\begin{Entry}{空中小姐}{8,4,3,8}{⽳,⼁,⼩,⼥}
  \begin{Phonetics}{空中小姐}{kong1zhong1xiao3jie3}
    \definition{s.}{aeromoça}
  \end{Phonetics}
\end{Entry}

\begin{Entry}{空心菜}{8,4,11}{⽳,⼼,⾋}
  \begin{Phonetics}{空心菜}{kong1xin1cai4}
    \definition{s.}{espinafre aquático | \emph{ong choy} | repolho do pântano | convolvulus aquático | glória-da-manhã aquática}
  \seealsoref{蕹菜}{weng4cai4}
  \end{Phonetics}
\end{Entry}

\begin{Entry}{空气}{8,4}{⽳,⽓}
  \begin{Phonetics}{空气}{kong1qi4}[][HSK 2]
    \definition[缕,股,份,阵]{s.}{ar; gases que compõe a atmosfera terrestre | atmosfera}
  \end{Phonetics}
\end{Entry}

\begin{Entry}{空白}{8,5}{⽳,⽩}
  \begin{Phonetics}{空白}{kong4bai2}[][HSK 7-9]
    \definition[块,片,个]{s.}{espaço; margem; espaço em branco; (na diagramação da página, páginas do livro, ilustrações, etc.) partes vazias, não preenchidas ou não utilizadas}
  \end{Phonetics}
\end{Entry}

\begin{Entry}{空军}{8,6}{⽳,⼍}
  \begin{Phonetics}{空军}{kong1jun1}[][HSK 6]
    \definition[名,位,个,支]{s.}{força aérea; um exército que luta no ar, geralmente composto por várias unidades de aviação e unidades terrestres da força aérea}
  \end{Phonetics}
\end{Entry}

\begin{Entry}{空地}{8,6}{⽳,⼟}
  \begin{Phonetics}{空地}{kong1di4}
    \definition{s.}{abertura; espaço vazio; área; gramado; terreno baldio; espaço aberto}
  \end{Phonetics}
  \begin{Phonetics}{空地}{kong4di4}[][HSK 7-9]
    \definition{s.}{abertura; espaço vazio; área; gramado; terreno baldio; espaço aberto}
  \end{Phonetics}
\end{Entry}

\begin{Entry}{空间}{8,7}{⽳,⾨}
  \begin{Phonetics}{空间}{kong1jian1}[][HSK 4]
    \definition[个]{s.}{espaço; recinto; cômodo; espaço em branco; interespaço}
  \end{Phonetics}
\end{Entry}

\begin{Entry}{空间站}{8,7,10}{⽳,⾨,⽴}
  \begin{Phonetics}{空间站}{kong1jian1zhan4}
    \definition{s.}{estação espacial}
  \end{Phonetics}
\end{Entry}

\begin{Entry}{空姐}{8,8}{⽳,⼥}
  \begin{Phonetics}{空姐}{kong1jie3}
    \definition[名,位,个]{s.}{aeromoça; comissária de bordo; abreviação de 空中小姐}
  \seealsoref{空中小姐}{kong1zhong1xiao3jie3}
  \end{Phonetics}
\end{Entry}

\begin{Entry}{空前}{8,9}{⽳,⼑}
  \begin{Phonetics}{空前}{kong1qian2}[][HSK 7-9]
    \definition{adj.}{sem precedentes; nunca antes}
  \end{Phonetics}
\end{Entry}

\begin{Entry}{空荡荡}{8,9,9}{⽳,⾋,⾋}
  \begin{Phonetics}{空荡荡}{kong1dang4dang4}[][HSK 7-9]
    \definition{adj.}{vazio; deserto; descreve uma casa, terreno, etc., como estando muito vazio | vazio; desolado; descreve um estado de vazio espiritual e falta de plenitude}
  \end{Phonetics}
\end{Entry}

\begin{Entry}{空调}{8,10}{⽳,⾔}
  \begin{Phonetics}{空调}{kong1tiao2}[][HSK 3]
    \definition[台,个]{s.}{ar-condicionado;  condicionador de ar}
  \end{Phonetics}
\end{Entry}

\begin{Entry}{空难}{8,10}{⽳,⾫}
  \begin{Phonetics}{空难}{kong1nan4}[][HSK 7-9]
    \definition{s.}{desastre aéreo; acidente aéreo; incidente aéreo; acidente de aviação}
  \end{Phonetics}
\end{Entry}

\begin{Entry}{空虚}{8,11}{⽳,⾌}
  \begin{Phonetics}{空虚}{kong1xu1}[][HSK 7-9]
    \definition{adj.}{vazio; oco; não contém nada de substancial; não é substancial}
  \end{Phonetics}
\end{Entry}

\begin{Entry}{空隙}{8,12}{⽳,⾩}
  \begin{Phonetics}{空隙}{kong4xi4}[][HSK 7-9]
    \definition{s.}{lacuna; vazio; espaço; folga; o espaço vazio no meio | intervalo; interstício; tempo livre não utilizado | chance; ocasião; oportunidade; lacunas; oportunidades a explorar}
  \end{Phonetics}
\end{Entry}

\begin{Entry}{空想}{8,13}{⽳,⼼}
  \begin{Phonetics}{空想}{kong1xiang3}[][HSK 7-9]
    \definition{s.}{pensamento irrealista; fantasia; devaneio | fantasia; sonho vão; esperança vã}
    \definition{v.}{entregar"-se à fantasia; sonhar acordado}
  \end{Phonetics}
\end{Entry}

%%%%%%%%%% 穿 %%%%%%%%%%
\subsection*{穿}\addcontentsline{loh}{figure}{穿}

\begin{Entry}{穿}{9}{⽳}
  \begin{Phonetics}{穿}{chuan1}[][HSK 1]
    \definition{adj.}{direto; através; usado após certos verbos, indica um estado de revelação completa}
    \definition{s.}{vestuário; roupas; refere"-se a roupas, sapatos, meias, etc.}
    \definition{v.}{usar; vestir; estar vestido; ter\dots vestido;  vestir roupas, sapatos, meias, etc. | perfurar através de; penetrar; formar orifícios por meio de cinzéis, brocas ou pontas afiadas | enfiar; amarrar; usar cordas e fios para ligar coisas | passar por; atravessar; passar por; através de (buracos, fendas, espaços vazios, etc.)}
  \end{Phonetics}
\end{Entry}

\begin{Entry}{穿上}{9,3}{⽳,⼀}
  \begin{Phonetics}{穿上}{chuan1shang5}[][HSK 4]
    \definition{v.}{vestir (roupas, etc.); colocar roupas}
  \end{Phonetics}
\end{Entry}

\begin{Entry}{穿小鞋}{9,3,15}{⽳,⼩,⾰}
  \begin{Phonetics}{穿小鞋}{chuan1 xiao3xie2}[][HSK 7-9]
    \definition{v.}{dar a alguém sapatos apertados para usar; dificultar as coisas para alguém abusando do seu poder; metáfora para dificultar secretamente as coisas para alguém ou impor restrições ou obstáculos a essa pessoa}
  \end{Phonetics}
\end{Entry}

\begin{Entry}{穿过}{9,6}{⽳,⾡}
  \begin{Phonetics}{穿过}{chuan1guo4}[][HSK 7-9]
    \definition{v.}{atravessar; penetrar; passar; passar por cima}
  \end{Phonetics}
\end{Entry}

\begin{Entry}{穿着}{9,11}{⽳,⽬}
  \begin{Phonetics}{穿着}{chuan1zhuo2}[][HSK 7-9]
    \definition[次]{s.}{vestido; vestuário; o que alguém veste; o efeito geral das roupas e decorações que as pessoas usam}
  \end{Phonetics}
\end{Entry}

\begin{Entry}{穿越}{9,12}{⽳,⾛}
  \begin{Phonetics}{穿越}{chuan1yue4}[][HSK 7-9]
    \definition{v.}{passar através de; cortar através de; atravessar; passar por cima}
  \end{Phonetics}
\end{Entry}

%%%%%%%%%% 突 %%%%%%%%%%
\subsection*{突}\addcontentsline{loh}{figure}{突}

\begin{Entry}{突}{9}{⽳}
  \begin{Phonetics}{突}{tu1}
    \definition{adv.}{de repente; abruptamente; inesperadamente}
    \definition{s.}{chaminé}
    \definition{v.}{avançar rapidamente; atacar | projetar; destacar-se | romper | projetar-se; inchar; fazer bojo}
  \end{Phonetics}
\end{Entry}

\begin{Entry}{突出}{9,5}{⽳,⼐}
  \begin{Phonetics}{突出}{tu1/chu1}[][HSK 3]
    \definition{adj.}{proeminente; excelente; mais que a média}
    \definition{v.+compl.}{romper | enfatizar; destacar; dar destaque a | sobressair; projetar-se; destacar-se}
  \end{Phonetics}
\end{Entry}

\begin{Entry}{突击}{9,5}{⽳,⼐}
  \begin{Phonetics}{突击}{tu1ji1}[][HSK 7-9]
    \definition[方]{s.}{ataque repentino e violento; agressão | Figurativo: trabalho apressado; esforço concentrado para terminar um trabalho rapidamente}
    \definition{v.}{fazer um ataque repentino e violento; agredir | fazer um esforço concentrado para terminar um trabalho rapidamente; fazer um trabalho às pressas}
  \end{Phonetics}
\end{Entry}

\begin{Entry}{突发}{9,5}{⽳,⼜}
  \begin{Phonetics}{突发}{tu1fa1}[][HSK 7-9]
    \definition{v.}{surgir de repente; aparecer inesperadamente | explodir repentinamente}
  \end{Phonetics}
\end{Entry}

\begin{Entry}{突如其来}{9,6,8,7}{⽳,⼥,⼋,⽊}
  \begin{Phonetics}{突如其来}{tu1ru2-qi2lai2}[][HSK 7-9]
    \definition{expr.}{surgir repentinamente; aparecer de repente; surgir do nada; aconteceu de forma inesperada e repentina}
  \end{Phonetics}
\end{Entry}

\begin{Entry}{突破}{9,10}{⽳,⽯}
  \begin{Phonetics}{突破}{tu1/po4}[][HSK 5]
    \definition{v.+compl.}{romper; fazer uma descoberta revolucionária; concentrar esforços em um único ponto de ataque, reunir o sucesso | quebrar (limite); superar (dificuldade); superar dificuldades; ultrapassar os números ou limites anteriores, superar recordes anteriores, etc.; romper com as limitações e restrições anteriores}
  \end{Phonetics}
\end{Entry}

\begin{Entry}{突破口}{9,10,3}{⽳,⽯,⼝}
  \begin{Phonetics}{突破口}{tu1po4kou3}[][HSK 7-9]
    \definition{s.}{ponto de virada; avanço; solução inovadora | brecha; lacuna}
  \end{Phonetics}
\end{Entry}

\begin{Entry}{突然}{9,12}{⽳,⽕}
  \begin{Phonetics}{突然}{tu1ran2}[][HSK 3]
    \definition{adj.}{repentino; abrupto; inesperado}
    \definition{adv.}{de repente; abruptamente; inesperadamente; subitamente}
  \end{Phonetics}
\end{Entry}

%%%%%%%%%% 窃 %%%%%%%%%%
\subsection*{窃}\addcontentsline{loh}{figure}{窃}

\begin{Entry}{窃}{9}{⽳}
  \begin{Phonetics}{窃}{qie4}
    \definition{adv.}{secretamente; sorrateiramente; furtivamente | usado antes de verbos para demonstrar modéstia, frequentemente significando ``Eu humildemente penso\dots'' ou ``Eu acredito em particular\dots''}[臣窃谓此策虽妙,实难实行。===Acredito humildemente que, embora essa estratégia seja engenhosa, na realidade é difícil de implementar.]
    \definition{pron.}{Literário: (referindo"-se às próprias opiniões) meu; minha}
    \definition{v.}{roubar; furtar | apoderar"-se ou ocupar ilegitimamente; tomar posse sem direito}
  \end{Phonetics}
\end{Entry}

\begin{Entry}{窃取}{9,8}{⽳,⼜}
  \begin{Phonetics}{窃取}{qie4qu3}[][HSK 7-9]
    \definition{v.}{usurpar; apoderar"-se; roubar (frequentemente usado metaforicamente)}
  \end{Phonetics}
\end{Entry}

%%%%%%%%%% 窄 %%%%%%%%%%
\subsection*{窄}\addcontentsline{loh}{figure}{窄}

\begin{Entry}{窄}{10}{⽳}
  \begin{Phonetics}{窄}{zhai3}
    \definition{adj.}{estreito; pequena distância horizontal | mesquinho; estreito; (mente) não alegre; (capacidade) pequena | difícil; mal; falta de; (vida) não bem de vida}
  \end{Phonetics}
\end{Entry}

%%%%%%%%%% 窍 %%%%%%%%%%
\subsection*{窍}\addcontentsline{loh}{figure}{窍}

\begin{Entry}{窍}{10}{⽳}
  \begin{Phonetics}{窍}{qiao4}
    \definition{s.}{abertura; furo | chave para algo; a chave para a questão}
  \end{Phonetics}
\end{Entry}

\begin{Entry}{窍门}{10,3}{⽳,⾨}
  \begin{Phonetics}{窍门}{qiao4men2}[][HSK 7-9]
    \definition[个]{s.}{habilidade; chave para um problema; segredo para fazer algo; um método inteligente que resolve o problema e é simples e fácil de implementar}
  \end{Phonetics}
\end{Entry}

%%%%%%%%%% 窗 %%%%%%%%%%
\subsection*{窗}\addcontentsline{loh}{figure}{窗}

\begin{Entry}{窗}{12}{⽳}
  \begin{Phonetics}{窗}{chuang1}
    \definition[扇,个]{s.}{janela}
  \end{Phonetics}
\end{Entry}

\begin{Entry}{窗口}{12,3}{⽳,⼝}
  \begin{Phonetics}{窗口}{chuang1kou3}[][HSK 6]
    \definition[个,号]{s.}{janela; em frente à janela; perto da janela | janela; postigo; refere"-se a uma abertura especial em forma de janela | janela; meio; intermediário; peça de exibição; campo de testes; uma metáfora para um lugar com muitas interações com o mundo exterior e através do qual o entendimento mútuo é alcançado |  janela; uma metáfora para um lugar que pode refletir ou exibir a totalidade ou parte de algo |  caixa de diálogo; uma caixa de operação quadrada para aplicativos ou arquivos que aparece na tela do computador}
  \end{Phonetics}
\end{Entry}

\begin{Entry}{窗子}{12,3}{⽳,⼦}
  \begin{Phonetics}{窗子}{chuang1zi5}[][HSK 4]
    \definition[扇,个]{s.}{janela}
  \end{Phonetics}
\end{Entry}

\begin{Entry}{窗户}{12,4}{⽳,⼾}
  \begin{Phonetics}{窗户}{chuang1hu5}[][HSK 4]
    \definition[个,扇,面,排]{s.}{janela; dispositivo de ventilação e transmissão de luz nas paredes}
  \end{Phonetics}
\end{Entry}

\begin{Entry}{窗台}{12,5}{⽳,⼝}
  \begin{Phonetics}{窗台}{chuang1tai2}[][HSK 4]
    \definition{s.}{parapeito da janela; peitoril; parte plana de uma janela que segura a moldura}
  \end{Phonetics}
\end{Entry}

\begin{Entry}{窗帘}{12,8}{⽳,⼱}
  \begin{Phonetics}{窗帘}{chuang1lian2}[][HSK 5]
    \definition[个,套,片,对]{s.}{cortinas para janelas}
  \end{Phonetics}
\end{Entry}

%%%%%%%%%% 窘 %%%%%%%%%%
\subsection*{窘}\addcontentsline{loh}{figure}{窘}

\begin{Entry}{窘}{12}{⽳}
  \begin{Phonetics}{窘}{jiong3}
    \definition{adj.}{em situação financeira precária; sem dinheiro; pobre | desajeitado; constrangido; desconfortável; difícil}
    \definition{v.}{constranger; desconcertar; dificultar as coisas}
  \end{Phonetics}
\end{Entry}

\begin{Entry}{窘迫}{12,8}{⽳,⾡}
  \begin{Phonetics}{窘迫}{jiong3po4}[][HSK 7-9]
    \definition{adj.}{muito pobre; miserável; extremamente ruim | envergonhado; pressionado; em apuros; descreve uma situação que causa constrangimento}
  \end{Phonetics}
\end{Entry}

%%%%%%%%%% 窜 %%%%%%%%%%
\subsection*{窜}\addcontentsline{loh}{figure}{窜}

\begin{Entry}{窜}{12}{⽳}
  \begin{Phonetics}{窜}{cuan4}[][HSK 7-9]
    \definition{v.}{fugir; correr (usado para bandidos, tropas inimigas e animais) | Literário: exilar; expulsar | Obsoleto: mudar (a redação de um texto, manuscrito, etc.); alterar}
  \end{Phonetics}
\end{Entry}

%%%%%%%%%% 窝 %%%%%%%%%%
\subsection*{窝}\addcontentsline{loh}{figure}{窝}

\begin{Entry}{窝}{12}{⽳}
  \begin{Phonetics}{窝}{wo1}[][HSK 7-9]
    \definition{clas.}{ninhada; cria; usado para animais}
    \definition[个,只,块]{s.}{ninho; habitat de aves, animais e insetos | toca; covil; ninho; uma metáfora para um lugar onde pessoas más se reúnem | lugar; metaforicamente, a posição ocupada por um corpo humano ou objeto | poço; cavidade; área rebaixada}
    \definition{v.}{abrigar; dar abrigo | reprimir; suprimir; reter (voltar); emoções reprimidas não podem ser expressas ou manifestadas | dobrar; flexionar; torcer}
  \end{Phonetics}
\end{Entry}

%%%%%%%%%% 窟 %%%%%%%%%%
\subsection*{窟}\addcontentsline{loh}{figure}{窟}

\begin{Entry}{窟}{13}{⽳}
  \begin{Phonetics}{窟}{ku1}
    \definition{s.}{buraco; caverna; cavidade; toca; gruta}
  \end{Phonetics}
\end{Entry}

\begin{Entry}{窟窿}{13,16}{⽳,⽳}
  \begin{Phonetics}{窟窿}{ku1long5}[][HSK 7-9]
    \definition[个]{s.}{furo; cavidade; buraco | déficit (figurativo); débito; dívida; essa metáfora se refere a déficits financeiros ou brechas no trabalho}
  \end{Phonetics}
\end{Entry}

%%%%%%%%%% 窾 %%%%%%%%%%
\subsection*{窾}\addcontentsline{loh}{figure}{窾}

\begin{Entry}{窾}{17}{⽳}
  \begin{Phonetics}{窾}{cuan4}
    \definition{adj.}{vazio | seco | destituído; pobre}
    \definition{s.}{buraco | lei}
    \definition{v.}{esconder}
  \end{Phonetics}
  \begin{Phonetics}{窾}{kuan3}
    \definition{adj.}{oco}
    \definition{s.}{rachadura; cavidade | (onomatopéia) água batendo na rocha}
    \definition{v.}{escavar um buraco}
  \end{Phonetics}
\end{Entry}

%%%%% EOF %%%%%

