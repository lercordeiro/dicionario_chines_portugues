%%%
%%% Radical "⽮"
%%%
\section*{Radical 111: ``⽮''}\addcontentsline{toc}{section}{Radical 111: ⽮}\addcontentsline{loh}{figure}{\#\#\#\# 111: ⽮}

%%%%%%%%%% 知 %%%%%%%%%%
\subsection*{知}\addcontentsline{loh}{figure}{知}

\begin{Entry}{知}{8}{⽮}
  \begin{Phonetics}{知}{zhi1}
    \definition{s.}{conhecimento | amigo íntimo; refere-se a um confidente}
    \definition{v.}{saber; entender; perceber; estar ciente de | contar; informar; notificar; tornar conhecido | administrar; estar encarregado de; supervisionar}
  \end{Phonetics}
\end{Entry}

\begin{Entry}{知了}{8,2}{⽮,⼅}
  \begin{Phonetics}{知了}{zhi1liao3}
    \definition[通]{s.}{cigarra}
  \end{Phonetics}
\end{Entry}

\begin{Entry}{知名}{8,6}{⽮,⼝}
  \begin{Phonetics}{知名}{zhi1 ming2}[][HSK 6]
    \definition{adj.}{notável; famoso; celebrado; bem conhecido}
  \end{Phonetics}
\end{Entry}

\begin{Entry}{知识}{8,7}{⽮,⾔}
  \begin{Phonetics}{知识}{zhi1shi5}[][HSK 1]
    \definition[个,门,种]{s.}{conhecimento; conjunto de conhecimentos e experiências adquiridos pelas pessoas na prática de transformar o mundo | intelectual; refere-se à cultura acadêmica}
  \end{Phonetics}
\end{Entry}

\begin{Entry}{知道}{8,12}{⽮,⾡}
  \begin{Phonetics}{知道}{zhi1dao4}[][HSK 1]
    \definition{v.}{saber; perceber; estar ciente de; ter conhecimento dos fatos ou da razão; ser sensato}
  \end{Phonetics}
\end{Entry}

\begin{Entry}{知道了}{8,12,2}{⽮,⾡,⼅}
  \begin{Phonetics}{知道了}{zhi1dao4le5}
    \definition{interj.}{Entendi! | OK!}
  \end{Phonetics}
\end{Entry}

%%%%%%%%%% 矫 %%%%%%%%%%
\subsection*{矫}\addcontentsline{loh}{figure}{矫}

\begin{Entry}{矫}{11}{⽮}
  \begin{Phonetics}{矫}{jiao2}
    \definition{s.}{usado em 矫情}
  \seealsoref{矫情}{jiao2qing5}
  \end{Phonetics}
  \begin{Phonetics}{矫}{jiao3}
    \definition*{s.}{Sobrenome: Jiao}
    \definition{adj.}{forte; corajoso}
    \definition{v.}{retificar; corrigir; resolver | fingir; simular; dissimular}
  \end{Phonetics}
\end{Entry}

\begin{Entry}{矫正}{11,5}{⽮,⽌}
  \begin{Phonetics}{矫正}{jiao3zheng4}[][HSK 7-9]
    \definition{v.}{corrigir; retificar}
  \end{Phonetics}
\end{Entry}

\begin{Entry}{矫情}{11,11}{⽮,⼼}
  \begin{Phonetics}{矫情}{jiao2qing2}
    \definition{v.}{ser afetadamente não convencional; fingir ser incomum; ir deliberadamente contra o senso comum para demonstrar superioridade ou ser diferente dos outros}
  \end{Phonetics}
  \begin{Phonetics}{矫情}{jiao2qing5}
    \definition{adj.}{briguento; contencioso; irracional; isso se refere a apresentar argumentos descabidos e causar problemas.}
  \end{Phonetics}
\end{Entry}

%%%%%%%%%% 短 %%%%%%%%%%
\subsection*{短}\addcontentsline{loh}{figure}{短}

\begin{Entry}{短}{12}{⽮}
  \begin{Phonetics}{短}{duan3}[][HSK 2]
    \definition{adj.}{curto; comprimento pequeno de uma extremidade à outra (em oposição a 长) | curto; breve; a distância entre o ponto inicial e o ponto final de um determinado período é pequena | raso; superficial}
    \definition{s.}{falha; defeito; ponto fraco; desvantagens | tonelada curta (EUA)}
    \definition{v.}{dever; carecer}
  \seealsoref{长}{zhang3}
  \end{Phonetics}
\end{Entry}

\begin{Entry}{短少}{12,4}{⽮,⼩}
  \begin{Phonetics}{短少}{duan3shao3}
    \definition{v.}{estar aquém do valor total}
  \end{Phonetics}
\end{Entry}

\begin{Entry}{短片}{12,4}{⽮,⽚}
  \begin{Phonetics}{短片}{duan3 pian4}[][HSK 6]
    \definition{s.}{curta-metragem; curtas-metragens documentais ou educativos exibidos individualmente ou em série}
  \end{Phonetics}
\end{Entry}

\begin{Entry}{短处}{12,5}{⽮,⼡}
  \begin{Phonetics}{短处}{duan3 chu4}[][HSK 3]
    \definition[个]{s.}{deficiência; ponto fraco; defeito; fraqueza}
  \end{Phonetics}
\end{Entry}

\begin{Entry}{短视}{12,8}{⽮,⾒}
  \begin{Phonetics}{短视}{duan3shi4}
    \definition{adj.}{míope}
  \end{Phonetics}
\end{Entry}

\begin{Entry}{短促}{12,9}{⽮,⼈}
  \begin{Phonetics}{短促}{duan3cu4}
    \definition{adj.}{curto (tom de voz) | fugaz | ofegante (respiração) | curto no tempo}
  \end{Phonetics}
\end{Entry}

\begin{Entry}{短信}{12,9}{⽮,⼈}
  \begin{Phonetics}{短信}{duan3xin4}[][HSK 2]
    \definition[条,个,封]{s.}{mensagem de texto; refere-se especificamente a mensagens de texto curtas, imagens, etc., enviadas ou recebidas por celular}
  \end{Phonetics}
\end{Entry}

\begin{Entry}{短缺}{12,10}{⽮,⽸}
  \begin{Phonetics}{短缺}{duan3que1}[][HSK 7-9]
    \definition{s.}{falta; déficit; escassez; insuficiência}
  \end{Phonetics}
\end{Entry}

\begin{Entry}{短暂}{12,12}{⽮,⽇}
  \begin{Phonetics}{短暂}{duan3zan4}[][HSK 7-9]
    \definition{adj.}{breve; transitório; momentâneo; de curta duração}
  \end{Phonetics}
\end{Entry}

\begin{Entry}{短期}{12,12}{⽮,⽉}
  \begin{Phonetics}{短期}{duan3 qi1}[][HSK 3]
    \definition{adj.}{de curta duração; de prazo curto}
    \definition[个]{s.}{curto prazo}
  \end{Phonetics}
\end{Entry}

\begin{Entry}{短裤}{12,12}{⽮,⾐}
  \begin{Phonetics}{短裤}{duan3 ku4}[][HSK 3]
    \definition[条]{s.}{calças curtas; calção; \emph{shorts}; calças com bainha acima do joelho}
  \end{Phonetics}
\end{Entry}

\begin{Entry}{短跑}{12,12}{⽮,⾜}
  \begin{Phonetics}{短跑}{duan3 pao3}
    \definition{s.}{corrida de curta distância; corrida rápida (oposto a 长跑)}
  \seealsoref{长跑}{chang2 pao3}
  \end{Phonetics}
\end{Entry}

%%%%%%%%%% 矮 %%%%%%%%%%
\subsection*{矮}\addcontentsline{loh}{figure}{矮}

\begin{Entry}{矮}{13}{⽮}
  \begin{Phonetics}{矮}{ai3}[][HSK 4]
    \definition{adj.}{baixo; pequeno | baixo; refere-se a \emph{status}, posição, classificação, nível, etc.}[他比我矮。===Ele é mais baixo que eu. | 这栋楼很矮,只有三层。===Esse prédio é baixo, tem só três andares. | 她虽然矮,但是跑得很快!===Ela pode ser baixinha, mas corre muito rápido!]
  \end{Phonetics}
\end{Entry}

\begin{Entry}{矮人}{13,2}{⽮,⼈}
  \begin{Phonetics}{矮人}{ai3ren2}
    \definition{s.}{anão; pessoa de baixa estatura (indivíduo) | homúnculo; figuras criadas artificialmente pelos alquimistas em frascos de destilação | nanismo}[他虽然是矮人,但很有力气。===Embora ele seja baixo, é muito forte. | 北欧神话中的矮人是技艺高超的工匠。===Na mitologia nórdica, os anões são artesãos habilidosos. | 他因为身高被嘲笑为‘矮人’,这让他很伤心。===Ele foi zombado por ser chamado de ‘anão’ devido à sua altura, o que o magoou.]
  \end{Phonetics}
\end{Entry}

\begin{Entry}{矮子}{13,3}{⽮,⼦}
  \begin{Phonetics}{矮子}{ai3zi5}
    \definition{s.}{pessoa baixa; anão; baixinho}[白雪公主和七个小矮子住在森林里。===Branca de Neve e os sete anões vivem na floresta. | 用``矮子''称呼他人是不礼貌的。===Chamar alguém de ``baixinho'' é falta de educação.]
  \end{Phonetics}
\end{Entry}

\begin{Entry}{矮小}{13,3}{⽮,⼩}
  \begin{Phonetics}{矮小}{ai3 xiao3}[][HSK 4]
    \definition{adj.}{subdimensionado; curto e pequeno; baixo e pequeno | quando usado para pessoas, pode soar depreciativo se não for em contexto neutro ou afetuoso}[这位矮小的老人是村里的智者。===Este idoso baixinho é o sábio da vila. | 这种矮小的灌木适合盆栽。===Este tipo de arbusto pequeno é ideal para vasos. | 山脚下有一片矮小的房屋,显得格外宁静。===Ao pé da montanha, havia casas baixas que transmitiam uma tranquilidade única.]
  \end{Phonetics}
\end{Entry}

\begin{Entry}{矮林}{13,8}{⽮,⽊}
  \begin{Phonetics}{矮林}{ai3lin2}
    \definition{s.}{mata rasteira | bosque baixo}[这片矮林里有很多野兔和鸟类。===Neste bosque baixo há muitos coelhos selvagens e pássaros. | 山坡上长满了矮林,远看像绿色的地毯。===A encosta está coberta de mata rasteira, que de longe parece um tapete verde.]
  \end{Phonetics}
\end{Entry}

\begin{Entry}{矮星}{13,9}{⽮,⽇}
  \begin{Phonetics}{矮星}{ai3xing1}
    \definition{s.}{estrela anã}[白矮星是恒星演化的最终阶段之一。===Anãs brancas são um dos estágios finais da evolução estelar.]
  \end{Phonetics}
\end{Entry}

\begin{Entry}{矮树}{13,9}{⽮,⽊}
  \begin{Phonetics}{矮树}{ai3shu4}
    \definition[棵]{s.}{arbusto | árvore pequena, baixa}[矮树比高树更容易修剪。===Árvores baixas são mais fáceis de podar do que árvores altas. | 我们种了些矮树作为花园的边界。===Plantamos alguns arbustos como cerca natural do jardim.]
  \end{Phonetics}
\end{Entry}

\begin{Entry}{矮胖}{13,9}{⽮,⾁}
  \begin{Phonetics}{矮胖}{ai3pang4}
    \definition{adj.}{atarracado; gorducho; rechonchudo; roliço; baixo e robusto | chamar alguém diretamente de 矮胖 pode ser ofensivo}[我家猫矮胖矮胖的,像个毛球。===Meu gato é baixinho e gordinho, parece uma bolinha de pelo.]
  \end{Phonetics}
\end{Entry}

\begin{Entry}{矮凳}{13,14}{⽮,⼏}
  \begin{Phonetics}{矮凳}{ai3deng4}
    \definition{s.}{banquinho baixo | banqueta}[这个矮凳是木制的,很结实。===Este banquinho é de madeira e bem resistente.]
  \end{Phonetics}
\end{Entry}

%%%%% EOF %%%%%

