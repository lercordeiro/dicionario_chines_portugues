%%%
%%% Radical "⾨"
%%%
\section*{Radical 169: ``⾨'' (门)}\addcontentsline{toc}{section}{Radical 169: ⾨、门}\addcontentsline{loh}{figure}{\#\#\#\# 169: ⾨}

%%%%%%%%%% 门 %%%%%%%%%%
\subsection*{门}\addcontentsline{loh}{figure}{门}

\begin{Entry}{门}{3}{⾨}[Kangxi 169]
  \begin{Phonetics}{门}{men2}[][HSK 1]
    \definition*{s.}{Sobrenome: Men}
    \definition{clas.}{para equipamentos de artilharia (por exemplo: canhões) | para trabalhos escolares, ciência e tecnologia, etc. | para idiomas | para casamentos | para parentes}
    \definition[个,把,道,扇]{s.}{entradas e saídas de edifícios, veículos, navios, aviões, etc. | válvula; interruptor; algo que funciona como um interruptor ou como uma porta | habilidade; método; acesso; maneira de fazer algo | família; ramo de uma família ou clã | seita (religiosa); escola (de pensamento); faculdades acadêmicas, ideológicas ou religiosas | classe; categoria; ramo de estudo; refere-se à categoria geral de coisas | filo; segundo nível da classificação biológica | (computador) \emph{gate}; porta (lógica) | porta; portão; entrada; refere-se a uma porta que pode ser aberta e fechada, instalada na entrada e saída | qualquer abertura; partes de objetos que podem ser abertas e fechadas | orifício no corpo humano; refere-se especificamente aos orifícios do corpo humano | estudar com o mesmo professor; refere-se especificamente ao professor ou mestre | posição em um jogo de apostas (em relação ao local onde se senta ou onde se faz uma aposta)}
  \end{Phonetics}
\end{Entry}

\begin{Entry}{门口}{3,3}{⾨,⼝}
  \begin{Phonetics}{门口}{men2 kou3}[][HSK 1]
    \definition[个]{s.}{porta; portão; entrada; porta de entrada}
  \end{Phonetics}
\end{Entry}

\begin{Entry}{门当户对}{3,6,4,5}{⾨,⼹,⼾,⼨}
  \begin{Phonetics}{门当户对}{men2dang1-hu4dui4}[][HSK 7-9]
    \definition{expr.}{``Casar com alguém de posição social equivalente.''; compatibilidade social e econômica adequada (para casamento); (um possível parceiro para casamento) uma combinação adequada; as famílias são bem compatíveis em termos de status social}
  \end{Phonetics}
\end{Entry}

\begin{Entry}{门诊}{3,7}{⾨,⾔}
  \begin{Phonetics}{门诊}{men2 zhen3}[][HSK 5]
    \definition{s.}{(no hospital) clínica ambulatorial; seção para pacientes ambulatoriais; local onde os médicos atendem pacientes que não estão internados no hospital}
  \end{Phonetics}
\end{Entry}

\begin{Entry}{门铃}{3,10}{⾨,⾦}
  \begin{Phonetics}{门铃}{men2ling2}[][HSK 7-9]
    \definition{s.}{campainha; uma campainha ou campainha elétrica para bater à porta}
  \end{Phonetics}
\end{Entry}

\begin{Entry}{门票}{3,11}{⾨,⽰}
  \begin{Phonetics}{门票}{men2 piao4}[][HSK 1]
    \definition{s.}{bilhete de entrada; bilhete de admissão; ingressos para locais de turismo, entretenimento, etc.}
  \end{Phonetics}
\end{Entry}

\begin{Entry}{门道}{3,12}{⾨,⾡}
  \begin{Phonetics}{门道}{men2dao4}
    \definition{s.}{porta | portal}
  \end{Phonetics}
  \begin{Phonetics}{门道}{men2dao5}
    \definition{s.}{talento | a maneira de fazer algo}
  \end{Phonetics}
\end{Entry}

\begin{Entry}{门路}{3,13}{⾨,⾜}
  \begin{Phonetics}{门路}{men2lu5}[][HSK 7-9]
    \definition{s.}{maneira de fazer algo; jeito | conexões sociais (para conseguir empregos, etc.) | jeito; saber fazer; truque do ofício; o segredo para fazer as coisas acontecerem}
  \seealsoref{门道}{men2dao5}
  \end{Phonetics}
\end{Entry}

\begin{Entry}{门槛}{3,14}{⾨,⽊}
  \begin{Phonetics}{门槛}{men2kan3}[][HSK 7-9]
    \definition{s.}{soleira; batente da porta; a viga horizontal ou faixa de pedra na parte inferior do batente da porta, próxima ao chão, etc. | o requisito para entrar em um determinado campo; metaforicamente, refere-se aos padrões ou condições para entrar em um determinado intervalo}
  \end{Phonetics}
\end{Entry}

%%%%%%%%%% 闪 %%%%%%%%%%
\subsection*{闪}\addcontentsline{loh}{figure}{闪}

\begin{Entry}{闪}{5}{⾨}
  \begin{Phonetics}{闪}{shan3}[][HSK 4]
    \definition*{s.}{Sobrenome: Shan}
    \definition{s.}{relâmpago}
    \definition{v.}{esquivar-se; desviar; sair do caminho | torcer; distender | surgir de repente | cintilar; brilhar | deixar para trás; abandonar | (corpo) oscilar dramaticamente}
  \end{Phonetics}
\end{Entry}

\begin{Entry}{闪电}{5,5}{⾨,⽥}
  \begin{Phonetics}{闪电}{shan3dian4}[][HSK 4]
    \definition[道]{s.}{relâmpago; descargas elétricas entre nuvens ou entre nuvens e o solo}
  \seealsoref{雷电}{lei2dian4}
  \end{Phonetics}
\end{Entry}

\begin{Entry}{闪存盘}{5,6,11}{⾨,⼦,⽫}
  \begin{Phonetics}{闪存盘}{shan3cun2pan2}
    \definition{s.}{unidade de memória \emph{USB} | cartão de memória}
  \seealsoref{优盘}{you1pan2}
  \end{Phonetics}
\end{Entry}

\begin{Entry}{闪烁}{5,9}{⾨,⽕}
  \begin{Phonetics}{闪烁}{shan3shuo4}[][HSK 7-9]
    \definition{adj.}{vago; evasivo; não comprometido}
    \definition{v.}{cintilar; brilhar; reluzir; tremeluzir}
  \end{Phonetics}
\end{Entry}

\begin{Entry}{闪耀}{5,20}{⾨,⽻}
  \begin{Phonetics}{闪耀}{shan3yao4}[][HSK 7-9]
    \definition{v.}{brilhar; cintilar; resplandecer; irradiar; a luz oscila, às vezes brilhante, às vezes fraca | brilhar; irradiar luz deslumbrante}
  \end{Phonetics}
\end{Entry}

%%%%%%%%%% 闭 %%%%%%%%%%
\subsection*{闭}\addcontentsline{loh}{figure}{闭}

\begin{Entry}{闭}{6}{⾨}
  \begin{Phonetics}{闭}{bi4}[][HSK 6]
    \definition*{s.}{Sobrenome: Bi}
    \definition{v.}{fechar; encerrar | bloquear; obstruir; parar}
  \end{Phonetics}
\end{Entry}

\begin{Entry}{闭幕}{6,13}{⾨,⼱}
  \begin{Phonetics}{闭幕}{bi4/mu4}[][HSK 5]
    \definition{v.+compl.}{fechar; concluir; (conferência, exposição, etc.) terminar | cair a cortina; abaixar a cortina; terminar a apresentação e a cortina se fechar em frente ao palco}
  \end{Phonetics}
\end{Entry}

\begin{Entry}{闭幕式}{6,13,6}{⾨,⼱,⼷}
  \begin{Phonetics}{闭幕式}{bi4 mu4 shi4}[][HSK 5]
    \definition{s.}{cerimônia de encerramento; cerimônia formal realizada no final de uma conferência ou exposição}
  \end{Phonetics}
\end{Entry}

\begin{Entry}{闭嘴}{6,16}{⾨,⼝}
  \begin{Phonetics}{闭嘴}{bi4zui3}
    \definition{expr.}{Cale-se!; Pare de falar!}
  \end{Phonetics}
\end{Entry}

%%%%%%%%%% 问 %%%%%%%%%%
\subsection*{问}\addcontentsline{loh}{figure}{问}

\begin{Entry}{问}{6}{⾨}
  \begin{Phonetics}{问}{wen4}[][HSK 1]
    \definition*{s.}{Sobrenome: Wen}
    \definition{prep.}{de; introduzir o objeto da ação, equivalente a 向 e 跟}
    \definition{v.}{perguntar; indagar; fazer com que as pessoas respondam ou esclareçam coisas que não sabem ou não têm certeza | perguntar (ou indagar) sobre | examinar; interrogar | intervir; responsabilizar; investigar | cuidar; preocupar-se; gerenciar; interferir}
  \seealsoref{跟}{gen1}
  \seealsoref{向}{xiang4}
  \end{Phonetics}
\end{Entry}

\begin{Entry}{问市}{6,5}{⾨,⼱}
  \begin{Phonetics}{问市}{wen4shi4}
    \definition{v.}{chegar ao mercado | bater o mercado | atingir o mercado}
  \end{Phonetics}
\end{Entry}

\begin{Entry}{问安}{6,6}{⾨,⼧}
  \begin{Phonetics}{问安}{wen4'an1}
    \definition{s.}{saudações}
    \definition{v.}{dar cumprimentos a | prestar homenagem}
  \end{Phonetics}
\end{Entry}

\begin{Entry}{问卷}{6,8}{⾨,⼙}
  \begin{Phonetics}{问卷}{wen4juan4}
    \definition[份]{s.}{questionário}
  \end{Phonetics}
\end{Entry}

\begin{Entry}{问候}{6,10}{⾨,⼈}
  \begin{Phonetics}{问候}{wen4hou4}[][HSK 4]
    \definition{v.}{prestar homenagem; enviar uma saudação;  dar os respeitos (cumprimentos) a alguém}
  \end{Phonetics}
\end{Entry}

\begin{Entry}{问鼎}{6,12}{⾨,⿍}
  \begin{Phonetics}{问鼎}{wen4ding3}
    \definition{v.}{visar (o primeiro lugar, etc.) | aspirar ao trono}
  \end{Phonetics}
\end{Entry}

\begin{Entry}{问路}{6,13}{⾨,⾜}
  \begin{Phonetics}{问路}{wen4 lu4}[][HSK 2]
    \definition{v.}{perguntar o caminho; pedir direções}
  \end{Phonetics}
\end{Entry}

\begin{Entry}{问题}{6,15}{⾨,⾴}
  \begin{Phonetics}{问题}{wen4ti2}[][HSK 2]
    \definition{adj.}{desqualificado; indesejável; anormal, não atende aos requisitos}
    \definition[个,种,类,串]{s.}{pergunta; problema; perguntas a serem respondidas | problema; questão; contradições que precisam ser estudadas e resolvidas | problema; acidente; incidente | chave; ponto crucial; pontos importantes}
  \end{Phonetics}
\end{Entry}

%%%%%%%%%% 闯 %%%%%%%%%%
\subsection*{闯}\addcontentsline{loh}{figure}{闯}

\begin{Entry}{闯}{6}{⾨}
  \begin{Phonetics}{闯}{chuang3}[][HSK 5]
    \definition*{s.}{Sobrenome: Chuang}
    \definition{v.}{apressar-se; correr | moderar a si mesmo (lutando contra dificuldades e perigos); aventurar-se no mundo | incorrer; causar (um desastre, etc.)}
  \end{Phonetics}
\end{Entry}

%%%%%%%%%% 闲 %%%%%%%%%%
\subsection*{闲}\addcontentsline{loh}{figure}{闲}

\begin{Entry}{闲}{7}{⾨}
  \begin{Phonetics}{闲}{xian2}[][HSK 5]
    \definition{adj.}{ocioso; não ocupado; desocupado; sem coisas para fazer; sem atividades; tempo livre | desocupado; (casa, objeto, etc.) não em uso; ocioso | não oficial; não sério; não relacionado ao negócio}
    \definition{s.}{lazer; tempo livre}
  \end{Phonetics}
\end{Entry}

%%%%%%%%%% 间 %%%%%%%%%%
\subsection*{间}\addcontentsline{loh}{figure}{间}

\begin{Entry}{间}{7}{⾨}
  \begin{Phonetics}{间}{jian1}[][HSK 1]
    \definition{clas.}{a menor unidade de uma casa; a menor unidade habitacional; cômodo}
    \definition{s.}{espaço entre duas partes  | (em um) tempo ou espaço definido | sala; quarto | uma seção de uma sala ou o espaço lateral entre dois pares de pilares | com um tempo ou espaço definido}
  \end{Phonetics}
  \begin{Phonetics}{间}{jian4}
    \definition{s.}{espaço entre as duas partes; abertura; lacuna}
    \definition{v.}{separar | semear a discórdia | desbastar (mudas); podar; remover ou arrancar as mudas em excesso}
  \end{Phonetics}
\end{Entry}

\begin{Entry}{间或}{7,8}{⾨,⼽}
  \begin{Phonetics}{间或}{jian4huo4}
    \definition{adv.}{às vezes | ocasionalmente | de vez em quando}
  \end{Phonetics}
\end{Entry}

\begin{Entry}{间接}{7,11}{⾨,⼿}
  \begin{Phonetics}{间接}{jian4jie1}[][HSK 5]
    \definition{adj.}{indireto; de segunda mão; em oposição a 直接}
  \seealsoref{直接}{zhi2jie1}
  \end{Phonetics}
\end{Entry}

\begin{Entry}{间断}{7,11}{⾨,⽄}
  \begin{Phonetics}{间断}{jian4duan4}[][HSK 7-9]
    \definition{s.}{intervalo; salto temporal; disjunção; hiato; interrupção; parada; pulsação; lacuna}
    \definition{v.}{ser desconectado; ser interrompido}
  \end{Phonetics}
\end{Entry}

\begin{Entry}{间谍}{7,11}{⾨,⾔}
  \begin{Phonetics}{间谍}{jian4die2}[][HSK 7-9]
    \definition[个,名]{s.}{espião; agentes enviados ou recrutados por potências inimigas ou estrangeiras para espionar informações militares, segredos de Estado ou realizar atividades subversivas}
  \end{Phonetics}
\end{Entry}

\begin{Entry}{间隔}{7,12}{⾨,⾩}
  \begin{Phonetics}{间隔}{jian4ge2}[][HSK 7-9]
    \definition{s.}{intervalo; distância das coisas no espaço ou no tempo}
  \end{Phonetics}
\end{Entry}

\begin{Entry}{间隙}{7,12}{⾨,⾩}
  \begin{Phonetics}{间隙}{jian4xi4}[][HSK 7-9]
    \definition{s.}{lacuna; espaço; intervalo; tempo ou espaço não utilizado}
  \end{Phonetics}
\end{Entry}

%%%%%%%%%% 闷 %%%%%%%%%%
\subsection*{闷}\addcontentsline{loh}{figure}{闷}

\begin{Entry}{闷}{7}{⾨}
  \begin{Phonetics}{闷}{men1}[][HSK 7-9]
    \definition{adj.}{abafado; fechado; sufocante; baixa pressão de ar ou má circulação de ar | abafado; som baixo ou opaco}
    \definition{v.}{cubrir bem; fazer algo hermético | ficar sem fala; parar de falar | fechar a si mesmo ou alguém dentro de casa; ficar em casa e não sair}
  \end{Phonetics}
  \begin{Phonetics}{闷}{men4}[][HSK 7-9]
    \definition{adj.}{entediado; deprimido; irritado; desanimado | hermeticamente fechado; selado | triste e silencioso; chateado | hermético}
    \definition{s.}{desânimo}
  \end{Phonetics}
\end{Entry}

\begin{Entry}{闷热}{7,10}{⾨,⽕}
  \begin{Phonetics}{闷热}{men1re4}
    \definition{adj.}{abafado | quente e abafado | sufocantemente quente | quente e sensual}
  \end{Phonetics}
\end{Entry}

%%%%%%%%%% 闸 %%%%%%%%%%
\subsection*{闸}\addcontentsline{loh}{figure}{闸}

\begin{Entry}{闸}{8}{⾨}
  \begin{Phonetics}{闸}{zha2}
    \definition[个,道]{s.}{comporta; comporta | freio | (coloquial) interruptor}
    \definition{v.}{represar um córrego, rio, etc. | represar a água; parar a água}
  \end{Phonetics}
\end{Entry}

\begin{Entry}{闸门}{8,3}{⾨,⾨}
  \begin{Phonetics}{闸门}{zha2men2}
    \definition{s.}{eclusa | comporta}
  \end{Phonetics}
\end{Entry}

%%%%%%%%%% 闺 %%%%%%%%%%
\subsection*{闺}\addcontentsline{loh}{figure}{闺}

\begin{Entry}{闺}{9}{⾨}
  \begin{Phonetics}{闺}{gui1}
    \definition{s.}{Arcaico: (em uma casa) porta pequena; porta com arco | quarto da senhora; \emph{boudoir} | Literário: um pequeno portão; parte superior redonda e porta pequena na parte inferior}
  \end{Phonetics}
\end{Entry}

\begin{Entry}{闺女}{9,3}{⾨,⼥}
  \begin{Phonetics}{闺女}{gui1nv5}[][HSK 7-9]
    \definition[个]{s.}{menina; donzela; mulher solteira | filha}
  \end{Phonetics}
\end{Entry}

%%%%%%%%%% 闻 %%%%%%%%%%
\subsection*{闻}\addcontentsline{loh}{figure}{闻}

\begin{Entry}{闻}{9}{⾨}
  \begin{Phonetics}{闻}{wen2}[][HSK 2]
    \definition*{s.}{Sobrenome: Wen}
    \definition{adj.}{bem conhecido; famoso}
    \definition{s.}{notícia; história | reputação | boato; rumor}
    \definition{v.}{cheirar | ouvir}
  \end{Phonetics}
\end{Entry}

%%%%%%%%%% 阀 %%%%%%%%%%
\subsection*{阀}\addcontentsline{loh}{figure}{阀}

\begin{Entry}{阀}{9}{⾨}
  \begin{Phonetics}{阀}{fa2}
    \definition[个]{s.}{casa estabelecida ou grupo de poder; uma pessoa ou família poderosa; refere-se a uma pessoa ou família que tem uma influência dominante em uma determinada área | válvula (mecânica)}
  \end{Phonetics}
\end{Entry}

\begin{Entry}{阀门}{9,3}{⾨,⾨}
  \begin{Phonetics}{阀门}{fa2men2}[][HSK 7-9]
    \definition{s.}{válvula (mecânica); dispositivos para controlar o fluxo de água e ar em máquinas e tubulações}
  \end{Phonetics}
\end{Entry}

%%%%%%%%%% 阁 %%%%%%%%%%
\subsection*{阁}\addcontentsline{loh}{figure}{阁}

\begin{Entry}{阁}{9}{⾨}
  \begin{Phonetics}{阁}{ge2}
    \definition{s.}{pavilhão (geralmente de dois andares) | gabinete (de um governo) | Obsoleto: quarto da mulher; \emph{boudoir} | prateleira}
  \end{Phonetics}
\end{Entry}

\begin{Entry}{阁下}{9,3}{⾨,⼀}
  \begin{Phonetics}{阁下}{ge2xia4}
    \definition{pron.}{Sua Excelência | Sua Majestade | \emph{Sire}}
  \end{Phonetics}
\end{Entry}

%%%%%%%%%% 阅 %%%%%%%%%%
\subsection*{阅}\addcontentsline{loh}{figure}{阅}

\begin{Entry}{阅}{10}{⾨}
  \begin{Phonetics}{阅}{yue4}
    \definition{v.}{ler; repassar; examinar | revisar; inspecionar | experimentar; passar por}
  \end{Phonetics}
\end{Entry}

\begin{Entry}{阅兵式}{10,7,6}{⾨,⼋,⼷}
  \begin{Phonetics}{阅兵式}{yue4bing1shi4}
    \definition{s.}{parada militar; desfile militar}
  \end{Phonetics}
\end{Entry}

\begin{Entry}{阅览室}{10,9,9}{⾨,⾒,⼧}
  \begin{Phonetics}{阅览室}{yue4 lan3 shi4}[][HSK 5]
    \definition[个,间]{s.}{sala de leitura; a biblioteca dispõe de salas para leitura e pesquisa, equipadas com mesas e cadeiras adequadas, livros, jornais, revistas, etc.}
  \end{Phonetics}
\end{Entry}

\begin{Entry}{阅读}{10,10}{⾨,⾔}
  \begin{Phonetics}{阅读}{yue4du2}[][HSK 4]
    \definition{v.}{ler; examinar; olhar (livros, jornais, etc.) e entender seu conteúdo}
  \end{Phonetics}
\end{Entry}

\begin{Entry}{阅读广度}{10,10,3,9}{⾨,⾔,⼴,⼴}
  \begin{Phonetics}{阅读广度}{yue4du2guang3du4}
    \definition{s.}{intervalo de leitura}
  \end{Phonetics}
\end{Entry}

\begin{Entry}{阅读时间}{10,10,7,7}{⾨,⾔,⽇,⾨}
  \begin{Phonetics}{阅读时间}{yue4 du2 shi2 jian1}
    \definition{s.}{tempo de leitura}
  \end{Phonetics}
\end{Entry}

\begin{Entry}{阅读理解}{10,10,11,13}{⾨,⾔,⽟,⾓}
  \begin{Phonetics}{阅读理解}{yue4du2li3jie3}
    \definition{s.}{compreensão de leitura}
  \end{Phonetics}
\end{Entry}

\begin{Entry}{阅读装置}{10,10,12,13}{⾨,⾔,⾐,⽹}
  \begin{Phonetics}{阅读装置}{yue4du2zhuang1zhi4}
    \definition{s.}{dispositivo de leitura (por exemplo, para códigos de barras, etiquetas RFID, etc.)}
  \end{Phonetics}
\end{Entry}

\begin{Entry}{阅读障碍}{10,10,13,13}{⾨,⾔,⾩,⽯}
  \begin{Phonetics}{阅读障碍}{yue4du2zhang4ai4}
    \definition{s.}{dislexia}
  \end{Phonetics}
\end{Entry}

\begin{Entry}{阅读器}{10,10,16}{⾨,⾔,⼝}
  \begin{Phonetics}{阅读器}{yue4du2qi4}
    \definition{s.}{leitor (\emph{software})}
  \end{Phonetics}
\end{Entry}

%%%%%%%%%% 阐 %%%%%%%%%%
\subsection*{阐}\addcontentsline{loh}{figure}{阐}

\begin{Entry}{阐}{11}{⾨}
  \begin{Phonetics}{阐}{chan3}
    \definition{v.}{explicar; expor; expressar; divulgar; esclarecer; elucidar}
  \end{Phonetics}
\end{Entry}

\begin{Entry}{阐述}{11,8}{⾨,⾡}
  \begin{Phonetics}{阐述}{chan3shu4}[][HSK 7-9]
    \definition{v.}{explicar; expor; elaborar; discutir}
  \end{Phonetics}
\end{Entry}

%%%%%%%%%% 阔 %%%%%%%%%%
\subsection*{阔}\addcontentsline{loh}{figure}{阔}

\begin{Entry}{阔}{12}{⾨}
  \begin{Phonetics}{阔}{kuo4}[][HSK 6]
    \definition{adj.}{amplo; amplo; vasto | rico | longo, no sentido de ``há muito tempo'' | vazio; impraticável}
  \end{Phonetics}
\end{Entry}

\begin{Entry}{阔绰}{12,11}{⾨,⽷}
  \begin{Phonetics}{阔绰}{kuo4chuo4}[][HSK 7-9]
    \definition{adj.}{ostentoso; generoso com dinheiro; extravagante; luxuoso}
  \end{Phonetics}
\end{Entry}

%%%%% EOF %%%%%

