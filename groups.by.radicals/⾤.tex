%%%
%%% Radical "⾤"
%%%
\section*{Radical 165: ``⾤''}\addcontentsline{toc}{section}{Radical 165: ⾤}\addcontentsline{loh}{figure}{\#\#\#\# 165: ⾤}

%%%%%%%%%% 采 %%%%%%%%%%
\subsection*{采}\addcontentsline{loh}{figure}{采}

\begin{Entry}{采}{8}{⾤}
  \begin{Phonetics}{采}{cai3}[][HSK 7-9]
    \definition*{s.}{Sobrenome: Cai}
    \definition{s.}{espírito; tez; cor e expressão facial | cores}
    \definition{v.}{escolher; arrancar; reunir; colher (flores, folhas, frutas) | minerar; extrair | reunir; coletar | adotar; pegar; selecionar}
  \end{Phonetics}
  \begin{Phonetics}{采}{cai4}
    \definition{s.}{atribuição a um nobre feudal; a terra (incluindo os escravos que cultivavam a terra) concedida pelos antigos príncipes aos nobres; também chamada de feudo}
  \end{Phonetics}
\end{Entry}

\begin{Entry}{采用}{8,5}{⾤,⽤}
  \begin{Phonetics}{采用}{cai3yong4}[][HSK 3]
    \definition{v.}{selecionar e usar; adotar; considerar adequado e utilizar}
  \end{Phonetics}
\end{Entry}

\begin{Entry}{采访}{8,6}{⾤,⾔}
  \begin{Phonetics}{采访}{cai3fang3}[][HSK 4]
    \definition{s.}{cobertura; entrevista; coleta de notícias; entrevistas, pesquisas, gravações de áudio e vídeo, etc., com o objetivo de coletar os materiais necessários}
    \definition{v.}{cobrir; entrevistar; reunir novas informações}
  \end{Phonetics}
\end{Entry}

\begin{Entry}{采纳}{8,7}{⾤,⽷}
  \begin{Phonetics}{采纳}{cai3na4}[][HSK 6]
    \definition{v.}{aceitar; adotar; tomar (opiniões, sugestões, solicitações, etc.)}
  \end{Phonetics}
\end{Entry}

\begin{Entry}{采取}{8,8}{⾤,⼜}
  \begin{Phonetics}{采取}{cai3qu3}[][HSK 3]
    \definition{v.}{adotar; escolha da implementação (diretrizes, políticas, métodos, ações, etc.) | reunir; coletar; tomar; assumir}
  \end{Phonetics}
\end{Entry}

\begin{Entry}{采矿}{8,8}{⾤,⽯}
  \begin{Phonetics}{采矿}{cai3/kuang4}[][HSK 7-9]
    \definition{s.}{mina}
    \definition{v.+compl.}{minerar; extrair minerais}
  \end{Phonetics}
\end{Entry}

\begin{Entry}{采购}{8,8}{⾤,⾙}
  \begin{Phonetics}{采购}{cai3gou4}[][HSK 5]
    \definition[名]{s.}{comprador; responsável pelas compras}
    \definition{v.}{adquirir; comprar; fazer compras para uma organização; fazer compras para uma empresa}
  \end{Phonetics}
\end{Entry}

\begin{Entry}{采集}{8,12}{⾤,⾫}
  \begin{Phonetics}{采集}{cai3ji2}[][HSK 7-9]
    \definition{v.}{reunir; coletar}
  \end{Phonetics}
\end{Entry}

%%%%%%%%%% 释 %%%%%%%%%%
\subsection*{释}\addcontentsline{loh}{figure}{释}

\begin{Entry}{释}{12}{⾤}
  \begin{Phonetics}{释}{shi4}
    \definition*{s.}{Sakyamuni; refere"-se a Siddhartha Gautama, o fundador do budismo; também se refere ao próprio budismo}
    \definition{s.}{budismo}
    \definition{v.}{explicar; elucidar | esclarecer; dissipar; deixar ir; aliviar | soltar; ser aliviado de; aliviar; deixar ir; colocar no chão | libertar; pôr em liberdade}
  \end{Phonetics}
\end{Entry}

\begin{Entry}{释放}{12,8}{⾤,⽅}
  \begin{Phonetics}{释放}{shi4fang4}[][HSK 7-9]
    \definition{v.}{libertar; pôr em liberdade; restaurar a liberdade pessoal dos detidos | libertar; deixar sair; liberar algo interno, como matéria ou energia; metaforicamente, liberar as emoções internas}
  \synonymref{放走}{fang4zou3}
  \synonymref{排放}{pai2fang4}
  \antonymref{捕捉}{bu3zhuo1}
  \antonymref{逮捕}{dai4bu3}
  \antonymref{拘留}{ju1liu2}
  \antonymref{扣押}{kou4ya1}
  \antonymref{收集}{shou1ji2}
  \antonymref{吸收}{xi1shou1}
  \end{Phonetics}
\end{Entry}

%%%%% EOF %%%%%

