%%%
%%% Radical "⽉"
%%%
\section*{Radical 74: ``⽉''}\addcontentsline{toc}{section}{Radical 74: ⽉}\addcontentsline{loh}{figure}{\#\#\#\# 74: ⽉}

%%%%%%%%%% 月 %%%%%%%%%%
\subsection*{月}\addcontentsline{loh}{figure}{月}

\begin{Entry}{月}{4}{⽉}[Kangxi 74]
  \begin{Phonetics}{月}{yue4}[][HSK 1]
    \definition*{s.}{Sobrenome: Yue}
    \definition[个]{s.}{mês | por mês | lua | redondo; em forma de lua cheia}
  \end{Phonetics}
\end{Entry}

\begin{Entry}{月月}{4,4}{⽉,⽉}
  \begin{Phonetics}{月月}{yue4yue4}
    \definition{adv.}{todo mês}
  \end{Phonetics}
\end{Entry}

\begin{Entry}{月份}{4,6}{⽉,⼈}
  \begin{Phonetics}{月份}{yue4fen4}[][HSK 2]
    \definition[个]{s.}{mês; refere"-se a um determinado mês}
  \end{Phonetics}
\end{Entry}

\begin{Entry}{月底}{4,8}{⽉,⼴}
  \begin{Phonetics}{月底}{yue4di3}[][HSK 4]
    \definition[个]{s.}{final do mês; últimos dias do mês}
  \end{Phonetics}
\end{Entry}

\begin{Entry}{月径}{4,8}{⽉,⼻}
  \begin{Phonetics}{月径}{yue4jing4}
    \definition{s.}{diâmetro da lua | diâmetro da órbita da lua | caminho iluminado pela lua}
  \end{Phonetics}
\end{Entry}

\begin{Entry}{月亮}{4,9}{⽉,⼇}
  \begin{Phonetics}{月亮}{yue4liang5}[][HSK 2]
    \definition[个,轮,挂,快,页]{s.}{Lua; Lua é o nome comum do satélite da Terra}
  \end{Phonetics}
\end{Entry}

\begin{Entry}{月相}{4,9}{⽉,⽬}
  \begin{Phonetics}{月相}{yue4xiang4}
    \definition{s.}{fases da lua, a saber: lua nova 朔, lua crescente 上弦, lua cheia 望 e lua minguante 下弦}
  \end{Phonetics}
\end{Entry}

\begin{Entry}{月饼}{4,9}{⽉,⾷}
  \begin{Phonetics}{月饼}{yue4bing5}[][HSK 5]
    \definition[个,块,盒,筒]{s.}{bolinho da lua; comida típica do Festival do Meio Outono; redonda e recheada; simboliza a reunião familiar}
  \end{Phonetics}
\end{Entry}

\begin{Entry}{月球}{4,11}{⽉,⽟}
  \begin{Phonetics}{月球}{yue4qiu2}[][HSK 5]
    \definition*[颗,个]{s.}{Lua}
    \definition{pref.}{seleno-; seleni-}
  \end{Phonetics}
\end{Entry}

\begin{Entry}{月壤}{4,20}{⽉,⼟}
  \begin{Phonetics}{月壤}{yue4rang3}
    \definition{s.}{solo lunar}
  \end{Phonetics}
\end{Entry}

%%%%%%%%%% 有 %%%%%%%%%%
\subsection*{有}\addcontentsline{loh}{figure}{有}

\begin{Entry}{有}{6}{⽉}
  \begin{Phonetics}{有}{you3}[][HSK 1]
    \definition*{s.}{Sobrenome: You}
    \definition{pref.}{usado antes do nome de certas dinastias ou etnias}
    \definition{v.}{ter; possuir; indica posse ou propriedade | existe; há; indica que certas coisas existem em certos lugares | fazer uma estimativa ou uma comparação; expressar estimativa ou comparação | indicar ação; indica que algo aconteceu ou ocorreu | usado antes de substantivos abstratos, indica quantidade ou grandeza | em termos gerais, semelhante a 某; refere"-se de maneira geral a algo semelhante | usado antes de pessoa, hora e lugar, indica a existência parcial | usado antes de certos verbos para formar uma expressão idiomática, indicando cortesia, polidez}
  \seealsoref{某}{mou3}
  \end{Phonetics}
\end{Entry}

\begin{Entry}{有(一)些}{6,1,8}{⽉,⼀,⼆}
  \begin{Phonetics}{有(一)些}{you3 (yi4) xie1}
    \definition{adv.}{em vez disso; em vez de; de certa forma}
    \definition{pron.}{de certa forma}
  \seealsoref{有些}{you3xie1}
  \end{Phonetics}
\end{Entry}

\begin{Entry}{有(一)点儿}{6,1,9,2}{⽉,⼀,⽕,⼉}
  \begin{Phonetics}{有(一)点儿}{you3 yi4 dian3r5}
    \definition{adv.}{um pouco (有点儿 + {s.} ou {v. mental})}
  \seealsoref{有点儿}{you3dian3r5}
  \end{Phonetics}
\end{Entry}

\begin{Entry}{有人}{6,2}{⽉,⼈}
  \begin{Phonetics}{有人}{you3ren2}[][HSK 2]
    \definition{adj.}{ocupado (como no banheiro)}
    \definition{pron.}{qualquer um; alguém}
    \definition[所]{s.}{pessoas}
    \definition{v.}{ter alguém ali}
  \end{Phonetics}
\end{Entry}

\begin{Entry}{有力}{6,2}{⽉,⼒}
  \begin{Phonetics}{有力}{you3li4}[][HSK 5]
    \definition{adj.}{forte; vigoroso; poderoso; energético}
  \end{Phonetics}
\end{Entry}

\begin{Entry}{有用}{6,5}{⽉,⽤}
  \begin{Phonetics}{有用}{you3yong4}[][HSK 1]
    \definition{adj.}{útil; prático; funcional}
  \end{Phonetics}
\end{Entry}

\begin{Entry}{有关}{6,6}{⽉,⼋}
  \begin{Phonetics}{有关}{you3guan1}[][HSK 6]
    \definition{prep.}{no caminho de; sobre}
    \definition{v.}{preocupar-se com; relacionar-se com; ter algo a ver com; existir algum tipo de relacionamento}
  \end{Phonetics}
\end{Entry}

\begin{Entry}{有名}{6,6}{⽉,⼝}
  \begin{Phonetics}{有名}{you3 ming2}[][HSK 1]
    \definition{adj.}{conhecido; famoso; célebre; nome conhecido por todos}
  \end{Phonetics}
\end{Entry}

\begin{Entry}{有名无实}{6,6,4,8}{⽉,⼝,⽆,⼧}
  \begin{Phonetics}{有名无实}{you3ming2wu2shi2}
    \definition{v.}{(literal) tem um nome, mas não tem realidade | existe apenas no nome}
  \end{Phonetics}
\end{Entry}

\begin{Entry}{有利}{6,7}{⽉,⼑}
  \begin{Phonetics}{有利}{you3li4}[][HSK 3]
    \definition{adj.}{benéfico; favorável; vantajoso}
  \end{Phonetics}
\end{Entry}

\begin{Entry}{有利于}{6,7,3}{⽉,⼑,⼆}
  \begin{Phonetics}{有利于}{you3li4yu2}[][HSK 5]
    \definition{prep.}{disponível; é benéfico para alguém ou alguma coisa e pode ajudar e promovê-lo}
  \end{Phonetics}
\end{Entry}

\begin{Entry}{有劲儿}{6,7,2}{⽉,⼒,⼉}
  \begin{Phonetics}{有劲儿}{you3jin4er5}[][HSK 4]
    \definition{adj.}{interessante; divertido; estimulante | energético}
    \definition{v.}{ter força}
  \end{Phonetics}
\end{Entry}

\begin{Entry}{有劳}{6,7}{⽉,⼒}
  \begin{Phonetics}{有劳}{you3lao2}
    \definition{v.}{posso incomodá-lo; desculpe incomodá-lo | (educado) obrigado pelo seu trabalho (usado ao pedir um favor ou após ter recebido um)}
  \end{Phonetics}
\end{Entry}

\begin{Entry}{有时}{6,7}{⽉,⽇}
  \begin{Phonetics}{有时}{you3shi2}
    \definition{expr.}{às vezes; ocasionalmente; de vez em quando}
  \seealsoref{有的时候}{you3de5shi2hou4}
  \seealsoref{有时候}{you3shi2hou5}
  \end{Phonetics}
\end{Entry}

\begin{Entry}{有时……有时……}{6,7,6,7}{⽉,⽇,⽉,⽇}
  \begin{Phonetics}{有时……有时……}{you3shi2 you3shi2}
    \definition{adv.}{às vezes\dots às vezes\dots}
  \end{Phonetics}
\end{Entry}

\begin{Entry}{有时候}{6,7,10}{⽉,⽇,⼈}
  \begin{Phonetics}{有时候}{you3shi2hou5}[][HSK 1]
    \definition{adv.}{às vezes; indica um momento incerto, mas não frequente}
  \seealsoref{有的时候}{you3de5shi2hou4}
  \seealsoref{有时}{you3shi2}
  \end{Phonetics}
\end{Entry}

\begin{Entry}{有没有}{6,7,6}{⽉,⽔,⽉}
  \begin{Phonetics}{有没有}{you3mei2you3}[][HSK 6]
    \definition{adv.}{Você tem\dots?; Você já\dots? ; Existe algum\dots?}
  \end{Phonetics}
\end{Entry}

\begin{Entry}{有事}{6,8}{⽉,⼅}
  \begin{Phonetics}{有事}{you3shi4}[][HSK 6]
    \definition{v.}{estar ocupado; estar envolvido | ter algo acontecendo; sofrer um acidente; se meter em encrenca | (com 心里) ter algo em mente; estar ansioso; preocupar-se | ter um emprego; estar empregado}[看他这几天愁眉苦脸的, 心里一定有事。===Vendo como ele parece triste ultimamente, deve haver algo em sua mente.]
  \seealsoref{心里}{xin1li3}
  \end{Phonetics}
\end{Entry}

\begin{Entry}{有些}{6,8}{⽉,⼆}
  \begin{Phonetics}{有些}{you3xie1}[][HSK 1]
    \definition{adv.}{um pouco; bastante; ligeiramente}
    \definition{pron.}{uma parte; alguns}
    \definition{v.}{usado para indicar que há alguns, mas não muitos;}
  \seealsoref{有(一)些}{you3 (yi4) xie1}
  \end{Phonetics}
\end{Entry}

\begin{Entry}{有的}{6,8}{⽉,⽩}
  \begin{Phonetics}{有的}{you3de5}[][HSK 1]
    \definition{pron.}{algum, alguns}
  \end{Phonetics}
\end{Entry}

\begin{Entry}{有的时候}{6,8,7,10}{⽉,⽩,⽇,⼈}
  \begin{Phonetics}{有的时候}{you3de5shi2hou4}
    \definition{adv.}{às vezes; ocasionalmente}
  \seealsoref{有时}{you3shi2}
  \seealsoref{有时候}{you3shi2hou5}
  \end{Phonetics}
\end{Entry}

\begin{Entry}{有的是}{6,8,9}{⽉,⽩,⽇}
  \begin{Phonetics}{有的是}{you3de5shi4}[][HSK 3]
    \definition{expr.}{ter em abundância; não faltar; enfatizar que há muitos}
  \end{Phonetics}
\end{Entry}

\begin{Entry}{有空儿}{6,8,2}{⽉,⽳,⼉}
  \begin{Phonetics}{有空儿}{you3kong4r5}[][HSK 2]
    \definition{v.}{estar livre; ter tempo livre}
  \end{Phonetics}
\end{Entry}

\begin{Entry}{有限}{6,8}{⽉,⾩}
  \begin{Phonetics}{有限}{you3xian4}[][HSK 4]
    \definition{adj.}{finito; limitado; restrito | indica baixo grau; indica pouco número; número baixo; nível baixo}
  \end{Phonetics}
\end{Entry}

\begin{Entry}{有限公司}{6,8,4,5}{⽉,⾩,⼋,⼝}
  \begin{Phonetics}{有限公司}{you3xian4gong1si1}
    \definition{s.}{companhia limitada | corporação}
  \end{Phonetics}
\end{Entry}

\begin{Entry}{有毒}{6,9}{⽉,⽏}
  \begin{Phonetics}{有毒}{you3du2}[][HSK 5]
    \definition{adj.}{venenoso; tóxico; nocivo; geralmente é usada para descrever as propriedades nocivas à saúde de produtos químicos, plantas ou animais.}
  \end{Phonetics}
\end{Entry}

\begin{Entry}{有点儿}{6,9,2}{⽉,⽕,⼉}
  \begin{Phonetics}{有点儿}{you3dian3r5}[][HSK 2]
    \definition{adv.}{um pouco; indica um grau inferior, equivalente a 稍微 (usado principalmente para coisas que são insatisfatórias)}
    \definition{v.}{há um pouco; tem (ou ser de) algum; existem alguns}
  \seealsoref{稍微}{shao1wei1}
  \seealsoref{有(一)点儿}{you3 yi4 dian3r5}
  \end{Phonetics}
\end{Entry}

\begin{Entry}{有害}{6,10}{⽉,⼧}
  \begin{Phonetics}{有害}{you3hai4}[][HSK 5]
    \definition{adj.}{prejudicial; nocivo; danoso; que pode causar danos ou prejuízos a algo}
  \end{Phonetics}
\end{Entry}

\begin{Entry}{有效}{6,10}{⽉,⽁}
  \begin{Phonetics}{有效}{you3xiao4}[][HSK 3]
    \definition{adj.}{válido; eficiente; eficaz; capaz de alcançar os objetivos esperados}
  \end{Phonetics}
\end{Entry}

\begin{Entry}{有着}{6,11}{⽉,⽬}
  \begin{Phonetics}{有着}{you3zhe5}[][HSK 5]
    \definition{v.}{ter; possuir; haver; existir}
  \end{Phonetics}
\end{Entry}

\begin{Entry}{有道理}{6,12,11}{⽉,⾡,⽟}
  \begin{Phonetics}{有道理}{you3dao4li5}
    \definition{v.}{fazer sentido; ser bem fundamentado; haver verdade em}
  \end{Phonetics}
\end{Entry}

\begin{Entry}{有意思}{6,13,9}{⽉,⼼,⼼}
  \begin{Phonetics}{有意思}{you3yi4si5}[][HSK 2]
    \definition{adj.}{significativo; significativo e intrigante | interessante; agradável}
    \definition{v.}{ter interesse por; ser atraído sexualmente}
  \end{Phonetics}
\end{Entry}

\begin{Entry}{有趣}{6,15}{⽉,⾛}
  \begin{Phonetics}{有趣}{you3qu4}[][HSK 4]
    \definition{adj.}{interessante; fascinante; divertido; excitante; estimulante}
  \end{Phonetics}
\end{Entry}

%%%%%%%%%% 朋 %%%%%%%%%%
\subsection*{朋}\addcontentsline{loh}{figure}{朋}

\begin{Entry}{朋}{8}{⽉}
  \begin{Phonetics}{朋}{peng2}
    \definition*{s.}{Sobrenome: Peng}
    \definition{s.}{amigo}
    \definition{v.}{(literário) rivalizar; igualar; comparar | (literário) reunir-se em grupo; juntar-se em grupo}
  \end{Phonetics}
\end{Entry}

\begin{Entry}{朋友}{8,4}{⽉,⼜}
  \begin{Phonetics}{朋友}{peng2you5}[][HSK 1]
    \definition[个,位,帮,群]{s.}{amigo; pessoas que têm um bom relacionamento, uma boa relação, se entendem e se ajudam mutuamente | namorado; namorada}
  \end{Phonetics}
\end{Entry}

%%%%%%%%%% 服 %%%%%%%%%%
\subsection*{服}\addcontentsline{loh}{figure}{服}

\begin{Entry}{服}{8}{⽉}
  \begin{Phonetics}{服}{fu2}[][HSK 6]
    \definition*{s.}{Sobrenome: Fu}
    \definition{s.}{roupas | vestuário de luto; refere"-se a roupas de luto}
    \definition{v.}{vestir (roupas) | tomar (remédio) | envolver"-se em; servir | obedecer; ser convencido | convencer; persuadir | adaptar"-se; acostumar"-se a}
  \end{Phonetics}
  \begin{Phonetics}{服}{fu4}
    \definition{clas.}{usado para remédio: dose; usado na medicina tradicional chinesa}
  \end{Phonetics}
\end{Entry}

\begin{Entry}{服从}{8,4}{⽉,⼈}
  \begin{Phonetics}{服从}{fu2cong2}[][HSK 5]
    \definition{v.}{obedecer; submeter"-se a; estar subordinado a}
  \end{Phonetics}
\end{Entry}

\begin{Entry}{服务}{8,5}{⽉,⼒}
  \begin{Phonetics}{服务}{fu2wu4}[][HSK 2]
    \definition{v.}{prestar serviço a; estar a serviço de; servir; trabalhar para o benefício coletivo (ou de outras pessoas) ou para uma causa específica | trabalhar; servir}
  \end{Phonetics}
\end{Entry}

\begin{Entry}{服务员}{8,5,7}{⽉,⼒,⼝}
  \begin{Phonetics}{服务员}{fu2wu4yuan2}
    \definition{s.}{atendente | garçom | garçonete | pessoal de atendimento ao cliente}
  \end{Phonetics}
\end{Entry}

\begin{Entry}{服务器}{8,5,16}{⽉,⼒,⼝}
  \begin{Phonetics}{服务器}{fu2wu4qi4}[][HSK 7-9]
    \definition[个,台]{s.}{Computção: servidor; um dispositivo dedicado que fornece serviços aos usuários em uma rede eletrônica de computadores}
  \end{Phonetics}
\end{Entry}

\begin{Entry}{服用}{8,5}{⽉,⽤}
  \begin{Phonetics}{服用}{fu2yong4}[][HSK 7-9]
    \definition{v.}{tomar (remédio)}[他已开始服用这种药。===Ele começou a tomar o remédio.]
  \end{Phonetics}
\end{Entry}

\begin{Entry}{服饰}{8,8}{⽉,⾷}
  \begin{Phonetics}{服饰}{fu2shi4}[][HSK 7-9]
    \definition[套]{s.}{roupa; vestido; traje; vestimenta e adorno pessoal}
  \end{Phonetics}
\end{Entry}

\begin{Entry}{服装}{8,12}{⽉,⾐}
  \begin{Phonetics}{服装}{fu2zhuang1}[][HSK 3]
    \definition[套,件,身]{s.}{roupas; vestuário; trajes; termo genérico para roupas, sapatos e chapéus, geralmente referido especificamente a roupas}
  \end{Phonetics}
\end{Entry}

%%%%%%%%%% 朗 %%%%%%%%%%
\subsection*{朗}\addcontentsline{loh}{figure}{朗}

\begin{Entry}{朗}{10}{⽉}
  \begin{Phonetics}{朗}{lang3}
    \definition*{s.}{Sobrenome: Lang}
    \definition{adj.}{claro; brilhante | alto e claro (som)}
  \end{Phonetics}
\end{Entry}

\begin{Entry}{朗诵}{10,9}{⽉,⾔}
  \begin{Phonetics}{朗诵}{lang3song4}[][HSK 7-9]
    \definition{v.}{recitar; ler em voz alta com expressividade; ler poemas ou prosa em voz alta para expressar as emoções transmitidas pela obra}
  \end{Phonetics}
\end{Entry}

\begin{Entry}{朗读}{10,10}{⽉,⾔}
  \begin{Phonetics}{朗读}{lang3du2}[][HSK 5]
    \definition{v.}{ler em voz alta; recitar com voz clara e alta}
  \end{Phonetics}
\end{Entry}

%%%%%%%%%% 望 %%%%%%%%%%
\subsection*{望}\addcontentsline{loh}{figure}{望}

\begin{Entry}{望}{11}{⽉}
  \begin{Phonetics}{望}{wang4}
    \definition*{s.}{Sobrenome: Wang}
    \definition{prep.}{para; em direção a; em ``olhando para frente (望前看)'', ``caminhando para o leste (望东走)'', etc.; 望 é frequentemente escrito como 往}
    \definition{s.}{prestígio; reputação; fama | lua cheia | o 15º dia de um mês lunar}
    \definition{v.}{olhar por cima; olhar para a distância; olhar para longe na distância | visitar; ligar para | ter esperança; esperar | odiar; ressentir-se | pensar em atingir um determinado objetivo ou uma determinada situação em mente}
  \seealsoref{往}{wang3}
  \end{Phonetics}
\end{Entry}

\begin{Entry}{望见}{11,4}{⽉,⾒}
  \begin{Phonetics}{望见}{wang4jian4}[][HSK 6]
    \definition{v.}{espiar; ver; pôr os olhos em | detectar}
  \end{Phonetics}
\end{Entry}

%%%%%%%%%% 朝 %%%%%%%%%%
\subsection*{朝}\addcontentsline{loh}{figure}{朝}

\begin{Entry}{朝}{12}{⽉}
  \begin{Phonetics}{朝}{chao2}[][HSK 3]
    \definition*{s.}{Sobrenome: Chao}
    \definition{prep.}{para; em direção a; a direção ou o objeto da ação introduzida, equivalente a 向 ou 对}
    \definition[个]{s.}{corte real; governo; assembleia realizada por um soberano; também se refere à posição no poder | dinastia, todo o período de governo transmitido de geração em geração por um determinado sobrenome imperial | reinado (de um soberano); o período de reinado de um determinado monarca}
    \definition{v.}{fazer uma peregrinação para; ter uma audiência com (um rei, um imperador, etc.) | estar voltado para; estar em frente a}
  \seealsoref{对}{dui4}
  \seealsoref{向}{xiang4}
  \antonymref{野}{ye3}
  \end{Phonetics}
  \begin{Phonetics}{朝}{zhao1}
    \definition{s.}{manhã cedo; manhã | dia}
  \end{Phonetics}
\end{Entry}

\begin{Entry}{朝代}{12,5}{⽉,⼈}
  \begin{Phonetics}{朝代}{chao2dai4}[][HSK 7-9]
    \definition[个]{s.}{dinastia; refere"-se a um período ou era histórica}
  \end{Phonetics}
\end{Entry}

\begin{Entry}{朝廷}{12,6}{⽉,⼵}
  \begin{Phonetics}{朝廷}{chao2ting2}
    \definition{s.}{corte imperial | dinastia}
  \end{Phonetics}
\end{Entry}

\begin{Entry}{朝着}{12,11}{⽉,⽬}
  \begin{Phonetics}{朝着}{chao2zhe5}[][HSK 7-9]
    \definition{prep.}{voltado para; em direção a; pessoas ou coisas voltadas para uma direção}
  \end{Phonetics}
\end{Entry}

\begin{Entry}{朝鲜}{12,14}{⽉,⿂}
  \begin{Phonetics}{朝鲜}{chao2xian3}
    \definition*{s.}{Coréia do Norte}
  \end{Phonetics}
\end{Entry}

%%%%%%%%%% 期 %%%%%%%%%%
\subsection*{期}\addcontentsline{loh}{figure}{期}

\begin{Entry}{期}{12}{⽉}
  \begin{Phonetics}{期}{qi1}[][HSK 3]
    \definition{clas.}{questão; número; termo; coisas usadas para parcelamento}
    \definition{s.}{um período de tempo; fase; estágio | horário agendado; data agendada | tempo designado (programado)}
    \definition{v.}{marcar uma consulta | esperar; aguardar | esperar; ter esperança}
  \end{Phonetics}
\end{Entry}

\begin{Entry}{期中}{12,4}{⽉,⼁}
  \begin{Phonetics}{期中}{qi1zhong1}[][HSK 4]
    \definition{adj.}{provisório; interino; intermediário}
  \end{Phonetics}
\end{Entry}

\begin{Entry}{期末}{12,5}{⽉,⽊}
  \begin{Phonetics}{期末}{qi1mo4}[][HSK 4]
    \definition{s.}{terminal; final do prazo; fim do período}
  \end{Phonetics}
\end{Entry}

\begin{Entry}{期间}{12,7}{⽉,⾨}
  \begin{Phonetics}{期间}{qi1jian1}[][HSK 4]
    \definition{s.}{prazo; tempo; período}
  \end{Phonetics}
\end{Entry}

\begin{Entry}{期限}{12,8}{⽉,⾩}
  \begin{Phonetics}{期限}{qi1xian4}[][HSK 4]
    \definition{s.}{prazo; limite de tempo; tempo alocado; período de tempo limitado, também o limite final do limite de tempo; \emph{deadline}}
  \end{Phonetics}
\end{Entry}

\begin{Entry}{期待}{12,9}{⽉,⼻}
  \begin{Phonetics}{期待}{qi1dai4}[][HSK 4]
    \definition{v.}{aguardar; esperar; aguardar ansiosamente; ter em mente a realização de um determinado fim ou a ocorrência de uma determinada situação}
  \end{Phonetics}
\end{Entry}

\begin{Entry}{期盼}{12,9}{⽉,⽬}
  \begin{Phonetics}{期盼}{qi1pan4}[][HSK 7-9]
    \definition{v.}{esperar; aguardar}
  \end{Phonetics}
\end{Entry}

\begin{Entry}{期望}{12,11}{⽉,⽉}
  \begin{Phonetics}{期望}{qi1wang4}[][HSK 5]
    \definition{s.}{esperança; expectativa}
    \definition{v.}{esperar; ter esperança}
  \end{Phonetics}
\end{Entry}

%%%%%%%%%% 朦 %%%%%%%%%%
\subsection*{朦}\addcontentsline{loh}{figure}{朦}

\begin{Entry}{朦}{17}{⽉}
  \begin{Phonetics}{朦}{meng2}
    \definition{adj.}{indistinto | pouco claro}
    \definition{v.}{enganar}
  \end{Phonetics}
\end{Entry}

\begin{Entry}{朦胧}{17,9}{⽉,⾁}
  \begin{Phonetics}{朦胧}{meng2long2}[][HSK 7-9]
    \definition{adj.}{fraco; nebuloso; obscuro; pouco claro; vago}
  \end{Phonetics}
\end{Entry}

%%%%% EOF %%%%%

