%%%
%%% Radical "⾾"
%%%
\section*{Radical 191: ``⾾''}\addcontentsline{toc}{section}{Radical 191: ⾾}\addcontentsline{loh}{figure}{\#\#\#\# 191: ⾾}

%%%%%%%%%% 闹 %%%%%%%%%%
\subsection*{闹}\addcontentsline{loh}{figure}{闹}

\begin{Entry}{闹}{8}{⾾}
  \begin{Phonetics}{闹}{nao4}[][HSK 4]
    \definition{adj.}{barulhento}
    \definition{v.}{fazer barulho; provocar problemas | dar vazão (à sua raiva, ressentimento, etc.) | sofrer de; ser incomodado por; ocorrer (um desastre ou coisa ruim) | fazer;  entrar em ação | agitar; perturbar | brincar; fazer bagunça}
  \end{Phonetics}
\end{Entry}

\begin{Entry}{闹事}{8,8}{⾾,⼅}
  \begin{Phonetics}{闹事}{nao4/shi4}[][HSK 7-9]
    \definition{v.+compl.}{criar perturbação; causar problemas}
  \end{Phonetics}
\end{Entry}

\begin{Entry}{闹钟}{8,9}{⾾,⾦}
  \begin{Phonetics}{闹钟}{nao4 zhong1}[][HSK 4]
    \definition[个,台,只,款]{s.}{despertador; relógios capazes de tocar alarmes em horários predeterminados}
  \end{Phonetics}
\end{Entry}

\begin{Entry}{闹着玩儿}{8,11,8,2}{⾾,⽬,⽟,⼉}
  \begin{Phonetics}{闹着玩儿}{nao4zhe5wan2r5}[][HSK 7-9]
    \definition{expr.}{``Estou brincando.''; piada; brincar; fazer algo por diversão; estar brincando}
  \end{Phonetics}
\end{Entry}

%%%%% EOF %%%%%

