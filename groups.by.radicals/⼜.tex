%%%
%%% Radical "⼜"
%%%
\section*{Radical 29: ``⼜''}\addcontentsline{toc}{section}{Radical 29: ⼜}\addcontentsline{loh}{figure}{\#\#\#\# 29: ⼜}

%%%%%%%%%% 又 %%%%%%%%%%
\subsection*{又}\addcontentsline{loh}{figure}{又}

\begin{Entry}{又}{2}{⼜}[Kangxi 29]
  \begin{Phonetics}{又}{you4}[][HSK 2]
    \definition{adv.}{indica repetição ou continuação | indica que várias situações ou propriedades existem simultaneamente | indica um nível mais profundo de significado | indica adicionar zero a números inteiros | indica duas coisas contraditórias | indica um ponto de virada, significando 可是 | usado em frases negativas ou perguntas retóricas para fortalecer o tom | além disso; indica informações adicionais ou suplementares}
  \seealsoref{可是}{ke3shi4}
  \end{Phonetics}
\end{Entry}

\begin{Entry}{又一次}{2,1,6}{⼜,⼀,⽋}
  \begin{Phonetics}{又一次}{you4yi2ci4}
    \definition{adv.}{outra vez | mais uma vez | de novo}
  \end{Phonetics}
\end{Entry}

\begin{Entry}{又……又……}{2,2}{⼜,⼜}
  \begin{Phonetics}{又……又……}{you4 you4}
    \definition{conj.}{\dots e\dots; tanto\dots como\dots; as duas palavras usadas depois de 又 não devem ter nenhuma conotação de contraste, ambas devem ser positivas ou negativas}[这件毛衣挺不错的,\underline{又}便宜\underline{又}漂亮。===Este suéter é muito bom, barato e bonito.]
  \end{Phonetics}
\end{Entry}

\begin{Entry}{又及}{2,3}{⼜,⼃}
  \begin{Phonetics}{又及}{you4ji2}
    \definition{s.}{P.S., \emph{postscript}}
  \end{Phonetics}
\end{Entry}

\begin{Entry}{又名}{2,6}{⼜,⼝}
  \begin{Phonetics}{又名}{you4ming2}
    \definition{s.}{também conhecido como | nome alternativo}
  \end{Phonetics}
\end{Entry}

\begin{Entry}{又称}{2,10}{⼜,⽲}
  \begin{Phonetics}{又称}{you4cheng1}
    \definition{s.}{também conhecido como}
  \end{Phonetics}
\end{Entry}

%%%%%%%%%% 叉 %%%%%%%%%%
\subsection*{叉}\addcontentsline{loh}{figure}{叉}

\begin{Entry}{叉}{3}{⼜}
  \begin{Phonetics}{叉}{cha1}[][HSK 5]
    \definition{s.}{garfo; forquilha | símbolo de cruz, ``×''}
    \definition{v.}{trabalhar com um garfo; garfar; pegar coisas com um garfo}
  \end{Phonetics}
  \begin{Phonetics}{叉}{cha2}
    \definition{v.}{bloquear; emperrar; congestionar}
  \end{Phonetics}
  \begin{Phonetics}{叉}{cha3}
    \definition{v.}{separar de modo a formar uma bifurcação; bifurcar}
  \end{Phonetics}
\end{Entry}

\begin{Entry}{叉子}{3,3}{⼜,⼦}
  \begin{Phonetics}{叉子}{cha1zi5}[][HSK 5]
    \definition[把,个]{s.}{garfo; ferramenta com mais de duas pontas em uma extremidade | tridente; forquilha; ferramentas de agricultura antigas}
  \end{Phonetics}
\end{Entry}

%%%%%%%%%% 友 %%%%%%%%%%
\subsection*{友}\addcontentsline{loh}{figure}{友}

\begin{Entry}{友}{4}{⼜}
  \begin{Phonetics}{友}{you3}
    \definition{adj.}{amigável; bom relacionamento; próximo | de relações amigáveis}
    \definition{s.}{amigo; pessoas intimamente relacionadas}
  \end{Phonetics}
\end{Entry}

\begin{Entry}{友好}{4,6}{⼜,⼥}
  \begin{Phonetics}{友好}{you3hao3}[][HSK 2]
    \definition{adj.}{amigável; amistoso; muito próximo, relacionamento muito bom; como bons amigos}
    \definition{s.}{amigo próximo, íntimo; em ocasiões formais referem"-se a bons amigos}
  \end{Phonetics}
\end{Entry}

\begin{Entry}{友谊}{4,10}{⼜,⾔}
  \begin{Phonetics}{友谊}{you3yi4}[][HSK 5]
    \definition[段,份]{s.}{amizade; amizade entre amigos}
  \end{Phonetics}
\end{Entry}

%%%%%%%%%% 双 %%%%%%%%%%
\subsection*{双}\addcontentsline{loh}{figure}{双}

\begin{Entry}{双}{4}{⼜}
  \begin{Phonetics}{双}{shuang1}[][HSK 3]
    \definition*{s.}{Sobrenome: Shuang}
    \definition{adj.}{dois; gêmeo; par; dual | números pares | duplo; dobro}
    \definition{clas.}{usado para certos membros, órgãos ou coisas pareadas que são bilateralmente simétricas, por exemplo, sapatos, meias, pauzinhos, etc.}
  \antonymref{单}{dan1}
  \end{Phonetics}
\end{Entry}

\begin{Entry}{双手}{4,4}{⼜,⼿}
  \begin{Phonetics}{双手}{shuang1shou3}[][HSK 5]
    \definition{s.}{com as duas mãos; ambas as mãos; par de mãos}
  \end{Phonetics}
\end{Entry}

\begin{Entry}{双方}{4,4}{⼜,⽅}
  \begin{Phonetics}{双方}{shuang1fang1}[][HSK 3]
    \definition{s.}{ambos os lados; as duas partes; duas pessoas ou dois grupos frente a frente em um determinado relacionamento ou situação}
  \end{Phonetics}
\end{Entry}

\begin{Entry}{双方同意}{4,4,6,13}{⼜,⽅,⼝,⼼}
  \begin{Phonetics}{双方同意}{shuang1fang1tong2yi4}
    \definition{s.}{acordo bilateral}
  \end{Phonetics}
\end{Entry}

\begin{Entry}{双打}{4,5}{⼜,⼿}
  \begin{Phonetics}{双打}{shuang1da3}[][HSK 6]
    \definition[场,局,次]{s.}{duplas (em esportes)}
  \end{Phonetics}
\end{Entry}

\begin{Entry}{双边}{4,5}{⼜,⾡}
  \begin{Phonetics}{双边}{shuang1bian1}[][HSK 7-9]
    \definition{adj.}{bilateral; participação de ambas as partes; especificamente, participação de dois países | duplicado}
  \end{Phonetics}
\end{Entry}

\begin{Entry}{双向}{4,6}{⼜,⼝}
  \begin{Phonetics}{双向}{shuang1xiang4}[][HSK 7-9]
    \definition{s.}{mão"-dupla; bidirecional (oposto de 单向) | interativo}
  \seealsoref{单向}{dan1xiang4}
  \end{Phonetics}
\end{Entry}

\begin{Entry}{双层床}{4,7,7}{⼜,⼫,⼴}
  \begin{Phonetics}{双层床}{shuang1ceng2chuang2}
    \definition{s.}{beliche}
  \end{Phonetics}
\end{Entry}

\begin{Entry}{双胞胎}{4,9,9}{⼜,⾁,⾁}
  \begin{Phonetics}{双胞胎}{shuang1bao1tai1}[][HSK 7-9]
    \definition[对]{s.}{gêmeos; dois bebês na mesma gravidez; duas pessoas nascidas na mesma gravidez}
  \end{Phonetics}
\end{Entry}

\begin{Entry}{双重}{4,9}{⼜,⾥}
  \begin{Phonetics}{双重}{shuang1chong2}[][HSK 7-9]
    \definition{adj.}{dobro; dual; duplo; duas camadas; dois aspectos (frequentemente usado para conceitos abstratos)}
  \end{Phonetics}
\end{Entry}

\begin{Entry}{双赢}{4,17}{⼜,⾙}
  \begin{Phonetics}{双赢}{shuang1ying2}[][HSK 7-9]
    \definition{v.}{ganhar\-ganhar; alcançar um resultado em uma situação em que todos ganham; ser mutuamente vantajoso}
  \synonymref{互利}{hu4li4}
  \end{Phonetics}
\end{Entry}

%%%%%%%%%% 反 %%%%%%%%%%
\subsection*{反}\addcontentsline{loh}{figure}{反}

\begin{Entry}{反}{4}{⼜}
  \begin{Phonetics}{反}{fan3}[][HSK 4]
    \definition{adj.}{oposto; contrário; invertido}
    \definition{adv.}{pelo contrário; inversamente}
    \definition{v.}{inverter o lado; de cabeça para baixo; na direção oposta | virar; converter | retornar | opor"-se; combater; voltar"-se contra | rebelar"-se; revoltar"-se | inferir; deduzir; raciocinar por analogia}
  \end{Phonetics}
\end{Entry}

\begin{Entry}{反击}{4,5}{⼜,⼐}
  \begin{Phonetics}{反击}{fan3ji1}[][HSK 7-9]
    \definition{v.}{revidar; responder fogo; contra"-atacar}
  \end{Phonetics}
\end{Entry}

\begin{Entry}{反对}{4,5}{⼜,⼨}
  \begin{Phonetics}{反对}{fan3dui4}[][HSK 3]
    \definition{v.}{lutar; opor"-se; objetar a; ser contra; discordar}
  \end{Phonetics}
\end{Entry}

\begin{Entry}{反对派}{4,5,9}{⼜,⼨,⽔}
  \begin{Phonetics}{反对派}{fan3dui4pai4}
    \definition{s.}{facção de oposição}
  \end{Phonetics}
\end{Entry}

\begin{Entry}{反对党}{4,5,10}{⼜,⼨,⼉}
  \begin{Phonetics}{反对党}{fan3dui4dang3}
    \definition{s.}{partido de oposição}
  \end{Phonetics}
\end{Entry}

\begin{Entry}{反对票}{4,5,11}{⼜,⼨,⽰}
  \begin{Phonetics}{反对票}{fan3dui4piao4}
    \definition{s.}{voto dissidente}
  \end{Phonetics}
\end{Entry}

\begin{Entry}{反正}{4,5}{⼜,⽌}
  \begin{Phonetics}{反正}{fan3zheng5}[][HSK 3]
    \definition{adv.}{de qualquer forma; de qualquer maneira; embora as circunstâncias sejam diferentes, o resultado é o mesmo | tudo igual; em qualquer caso; tom de voz que expressa afirmação categórica}
  \end{Phonetics}
\end{Entry}

\begin{Entry}{反而}{4,6}{⼜,⽽}
  \begin{Phonetics}{反而}{fan3'er2}[][HSK 4]
    \definition{adv.}{em vez disso; ao contrário de; contrário ao significado da frase anterior ou inesperado, desempenha o papel de uma reviravolta em uma frase}
  \end{Phonetics}
\end{Entry}

\begin{Entry}{反过来}{4,6,7}{⼜,⾡,⽊}
  \begin{Phonetics}{反过来}{fan3 guo4lai2}[][HSK 7-9]
    \definition{adv.}{inversamente; na ordem inversa; em direção oposta; indica uma reversão ou mudança de uma ação ou estado}
    \definition{conj.}{vice"-versa; inversamente; na ordem inversa; ao contrário; usado para orientar relacionamentos de transição, como condições e causas}
    \definition{v.}{virar; virar para frente e para trás}
  \end{Phonetics}
\end{Entry}

\begin{Entry}{反问}{4,6}{⼜,⾨}
  \begin{Phonetics}{反问}{fan3wen4}[][HSK 6]
    \definition{v.}{fazer uma pergunta em resposta; responder a uma pergunta com outra pergunta | fazer uma pergunta retórica (uma pergunta com significado negativo)}
  \end{Phonetics}
\end{Entry}

\begin{Entry}{反应}{4,7}{⼜,⼴}
  \begin{Phonetics}{反应}{fan3ying4}[][HSK 3]
    \definition[个]{s.}{reação; resposta; opiniões, atitudes ou ações causadas pelo acontecimento}
    \definition{v.}{reagir; responder; atividade correspondente causada pela estimulação do organismo}
  \end{Phonetics}
\end{Entry}

\begin{Entry}{反抗}{4,7}{⼜,⼿}
  \begin{Phonetics}{反抗}{fan3kang4}[][HSK 6]
    \definition{s.}{resistência}
    \definition{v.}{revoltar"-se; resistir; opor"-se com ação}
  \end{Phonetics}
\end{Entry}

\begin{Entry}{反驳}{4,7}{⼜,⾺}
  \begin{Phonetics}{反驳}{fan3bo2}[][HSK 7-9]
    \definition{v.}{refutar; replicar; apresentar suas próprias razões para refutar teorias ou opiniões de outras pessoas que diferem das suas}
  \end{Phonetics}
\end{Entry}

\begin{Entry}{反响}{4,9}{⼜,⼝}
  \begin{Phonetics}{反响}{fan3xiang3}[][HSK 6]
    \definition{s.}{eco; reverberação; repercusão}
  \end{Phonetics}
\end{Entry}

\begin{Entry}{反复}{4,9}{⼜,⼢}
  \begin{Phonetics}{反复}{fan3fu4}[][HSK 3]
    \definition{adv.}{repetidamente; de novo e de novo; várias vezes}
    \definition{s.}{reversão; recaída; a situação anterior se repetiu}
    \definition{v.}{recuar; cortar e mudar; virar de cabeça para baixo; arrepender"-se; aparecer várias vezes (usado principalmente em situações ruins)}
  \end{Phonetics}
\end{Entry}

\begin{Entry}{反差}{4,9}{⼜,⼯}
  \begin{Phonetics}{反差}{fan3cha1}[][HSK 7-9]
    \definition{s.}{contraste; o contraste de cores da foto ou do cenário é muito diferente, como o contraste entre o preto e o branco | discrepância; o contraste entre o bem e o mal, o alto e o baixo, etc. de pessoas ou coisas}
  \end{Phonetics}
\end{Entry}

\begin{Entry}{反思}{4,9}{⼜,⼼}
  \begin{Phonetics}{反思}{fan3si1}[][HSK 7-9]
    \definition{v.}{refletir; introspectar; refletir sobre o passado e tirar lições dele}
  \end{Phonetics}
\end{Entry}

\begin{Entry}{反映}{4,9}{⼜,⽇}
  \begin{Phonetics}{反映}{fan3ying4}[][HSK 4]
    \definition{s.}{reflexão; opiniões sobre pessoas ou situações}
    \definition{v.}{refletir; espelhar; figurativamente, trazer à tona a essência de uma questão objetiva (expressão idiomática); expressar a essência de algo objetivamente | relatar; tornar conhecido; informar às autoridades superiores | refletir; espelhar; a imagem de um objeto aparece invertida em outro objeto}
  \end{Phonetics}
\end{Entry}

\begin{Entry}{反省}{4,9}{⼜,⽬}
  \begin{Phonetics}{反省}{fan3xing3}[][HSK 7-9]
    \definition{v.}{refletir sobre si mesmo; envolver"-se em introspecção e autoexame; refletir sobre seus pensamentos e ações e examinar quaisquer erros; examinar a consciência; questionar"-se; sondar a alma}
  \end{Phonetics}
\end{Entry}

\begin{Entry}{反面}{4,9}{⼜,⾯}
  \begin{Phonetics}{反面}{fan3mian4}[][HSK 7-9]
    \definition{adj.}{oposto; negativo; ruim}
    \definition{s.}{costas; lado reverso; lado avesso; o lado de um objeto oposto à frente | o reverso de um estado de coisas, um problema, etc.; o outro lado de uma questão, problema, etc.}
  \end{Phonetics}
\end{Entry}

\begin{Entry}{反倒}{4,10}{⼜,⼈}
  \begin{Phonetics}{反倒}{fan3dao4}[][HSK 7-9]
    \definition{adv.}{em vez disso; pelo contrário}
    \definition{conj.}{em vez disso; pelo contrário; frequentemente acompanhadas por várias palavras que expressam negação}
  \end{Phonetics}
\end{Entry}

\begin{Entry}{反常}{4,11}{⼜,⼱}
  \begin{Phonetics}{反常}{fan3chang2}[][HSK 7-9]
    \definition{adj.}{incomum; anormal; diferente da situação normal}
  \end{Phonetics}
\end{Entry}

\begin{Entry}{反弹}{4,11}{⼜,⼸}
  \begin{Phonetics}{反弹}{fan3tan2}[][HSK 7-9]
    \definition{s.}{rebote}
    \definition{v.}{recuperar; um objeto elástico retorna à sua forma original após ser deformado por uma força externa | rebater; saltar de volta; ressurgir; metáfora para recuperação de preço ou mercado | rebotar; quicar; ricochetear; um objeto em movimento salta na direção oposta quando encontra um obstáculo}
  \end{Phonetics}
\end{Entry}

\begin{Entry}{反馈}{4,12}{⼜,⾶}
  \begin{Phonetics}{反馈}{fan3kui4}[][HSK 7-9]
    \definition{s.}{\emph{feedback}; resposta; uma resposta ou reação a algo, informação, etc.}
    \definition{v.}{dar \emph{feedback}; enviar informações de volta; (informações, \emph{feedback}, etc.) retornar ao local de onde foi enviado}
  \end{Phonetics}
\end{Entry}

\begin{Entry}{反感}{4,13}{⼜,⼼}
  \begin{Phonetics}{反感}{fan3gan3}[][HSK 7-9]
    \definition{adj.}{avesso; enojado; desgostoso; insatisfeito}
    \definition{s.}{antipatia; aversão; oposição ou insatisfação}
  \end{Phonetics}
\end{Entry}

%%%%%%%%%% 发 %%%%%%%%%%
\subsection*{发}\addcontentsline{loh}{figure}{发}

\begin{Entry}{发}{5}{⼜}
  \begin{Phonetics}{发}{fa1}[][HSK 2]
    \definition*{s.}{Sobrenome: Fa}
    \definition{clas.}{bala, usada para munições e cartuchos}
    \definition{v.}{distribuir; enviar; entregar | emitir; disparar; lançar; descarregar | produzir; gerar; criar (dar origem a) | proferir; emitir; expressar | expandir; desenvolver | prosperar; prosperidade graças à aquisição de bens materiais | crescer ou expandir quando fermentado ou embebido | difundir; dispersar; espalhar | expor; descobrir; revelar | transformar"-se; tornar"-se; entrar em um determinado estado | demonstrar seus sentimentos; expressar (sentimentos) | sentir; ter um sentimento | começar; estabelecer | fazer com que se faça; iniciar um empreendimento; começar a agir; provocar uma ação}
  \end{Phonetics}
  \begin{Phonetics}{发}{fa4}
    \definition*{s.}{Sobrenome: Fa}
    \definition[件]{s.}{cabelo}
  \end{Phonetics}
\end{Entry}

\begin{Entry}{发火}{5,4}{⼜,⽕}
  \begin{Phonetics}{发火}{fa1/huo3}[][HSK 7-9]
    \definition{v.+compl.}{ficar com raiva; explodir; ficar furioso; perder a paciência | detonar; explodir | inflamar; pegar fogo; acender; começar a queimar}
  \end{Phonetics}
\end{Entry}

\begin{Entry}{发出}{5,5}{⼜,⼐}
  \begin{Phonetics}{发出}{fa1 chu1}[][HSK 3]
    \definition{v.}{fazer; produzir; deixar sair; ocorrer (som, dúvida, etc.) | emitir; anunciar; publicar; divulgar (ordens, instruções) | enviar (mercadorias, cartas, etc.); partir (veículos, etc.) | emitir; exalar (cheiro, calor, etc.)}
  \end{Phonetics}
\end{Entry}

\begin{Entry}{发布}{5,5}{⼜,⼱}
  \begin{Phonetics}{发布}{fa1bu4}[][HSK 5]
    \definition{v.}{emitir; publicar; liberar; anunciar; fazer ordens públicas, anúncios, notícias, etc.}
  \end{Phonetics}
\end{Entry}

\begin{Entry}{发布会}{5,5,6}{⼜,⼱,⼈}
  \begin{Phonetics}{发布会}{fa1bu4hui4}[][HSK 7-9]
    \definition[次]{s.}{coletiva de imprensa; um formato de conferência usado para divulgar notícias ou responder perguntas da mídia e do público | \emph{briefing}; atividades de exposição para promover novos produtos, etc.}
  \end{Phonetics}
\end{Entry}

\begin{Entry}{发生}{5,5}{⼜,⽣}
  \begin{Phonetics}{发生}{fa1sheng1}[][HSK 3]
    \definition{v.}{ocorrer; acontecer; tomar lugar; surgir algo que não existia antes}
  \end{Phonetics}
\end{Entry}

\begin{Entry}{发电}{5,5}{⼜,⽥}
  \begin{Phonetics}{发电}{fa1 dian4}[][HSK 6]
    \definition{s.}{geração de energia elétrica; produção de eletricidade; fornecimento de energia}
    \definition{v.}{gerar eletricidade (ou energia elétrica) | enviar um telegrama}
  \end{Phonetics}
\end{Entry}

\begin{Entry}{发电机}{5,5,6}{⼜,⽥,⽊}
  \begin{Phonetics}{发电机}{fa1dian4ji1}[][HSK 7-9]
    \definition{s.}{gerador; dínamo | alternador; gerador elétrico}
  \end{Phonetics}
\end{Entry}

\begin{Entry}{发光}{5,6}{⼜,⼉}
  \begin{Phonetics}{发光}{fa1/guang1}[][HSK 7-9]
    \definition{s.}{luminescência}
    \definition{v.+compl.}{emitir luz; brilhar; ser luminoso; cintilar}
  \end{Phonetics}
\end{Entry}

\begin{Entry}{发动}{5,6}{⼜,⼒}
  \begin{Phonetics}{发动}{fa1dong4}[][HSK 3]
    \definition{v.}{iniciar; começar; lançar | chamar à ação; mobilizar; despertar | ligar o motor; dar a partida; dar o pontapé inicial (motor de combustão interna) | estimular; colocar em ação}
  \end{Phonetics}
\end{Entry}

\begin{Entry}{发动机}{5,6,6}{⼜,⼒,⽊}
  \begin{Phonetics}{发动机}{fa1dong4ji1}
    \definition[台]{s.}{motor}
  \end{Phonetics}
\end{Entry}

\begin{Entry}{发扬}{5,6}{⼜,⼿}
  \begin{Phonetics}{发扬}{fa1yang2}[][HSK 7-9]
    \definition{v.}{desenvolver; continuar; levar adiante; desenvolver e promover (boas práticas, tradições, etc.) | aproveitar ao máximo; fazer uso total de; exercer ou mostrar (algum poder, habilidade, etc.) tanto quanto possível}
  \end{Phonetics}
\end{Entry}

\begin{Entry}{发扬光大}{5,6,6,3}{⼜,⼿,⼉,⼤}
  \begin{Phonetics}{发扬光大}{fa1yang2-guang1da4}[][HSK 7-9]
    \definition{expr.}{levar adiante; desenvolver; aprimorar; fomentar e aprimorar; dar pleno uso a; dar maior escopo a; levar a um maior nível de desenvolvimento; desenvolver para um estágio mais alto; espalhar e florescer; ``O desenvolvimento e a promoção tornam"-no cada vez mais grandioso.''}
  \end{Phonetics}
\end{Entry}

\begin{Entry}{发行}{5,6}{⼜,⾏}
  \begin{Phonetics}{发行}{fa1xing2}[][HSK 5]
    \definition{v.}{emitir; liberar; publicar; emitir ou vender de publicações recém-impressas, moeda, selos, etc.}
  \end{Phonetics}
\end{Entry}

\begin{Entry}{发达}{5,6}{⼜,⾡}
  \begin{Phonetics}{发达}{fa1da2}[][HSK 3]
    \definition{adj.}{desenvolvido; florescente; (coisas) Já estão bem desenvolvidas; (negócios) prosperam}
    \definition{v.}{desenvolver; promover; florescer; a pessoa tem um bom desempenho profissional e é muito bem-sucedida}
  \end{Phonetics}
\end{Entry}

\begin{Entry}{发作}{5,7}{⼜,⼈}
  \begin{Phonetics}{发作}{fa1zuo4}[][HSK 7-9]
    \definition{v.}{sair; mostrar efeito; a doença no corpo se manifesta repentinamente ou o álcool ou as drogas fazem efeito | explodir; ter um ataque de raiva; perder a paciência porque está muito zangado ou insatisfeito}
  \end{Phonetics}
\end{Entry}

\begin{Entry}{发抖}{5,7}{⼜,⼿}
  \begin{Phonetics}{发抖}{fa1dou3}[][HSK 7-9]
    \definition{v.}{tremer; sacudir; estremecer; tremer devido ao medo, raiva ou frio}
  \end{Phonetics}
\end{Entry}

\begin{Entry}{发言}{5,7}{⼜,⾔}
  \begin{Phonetics}{发言}{fa1/yan2}[][HSK 3]
    \definition[个]{s.}{discurso; declaração; palestra; opiniões publicadas}
    \definition{v.+compl.}{falar; fazer uma declaração (discurso); expressar opinião (geralmente em reuniões)}
  \end{Phonetics}
\end{Entry}

\begin{Entry}{发言人}{5,7,2}{⼜,⾔,⼈}
  \begin{Phonetics}{发言人}{fa1yan2ren2}[][HSK 6]
    \definition{s.}{porta"-voz}
  \end{Phonetics}
\end{Entry}

\begin{Entry}{发财}{5,7}{⼜,⾙}
  \begin{Phonetics}{发财}{fa1/cai2}[][HSK 7-9]
    \definition{v.+compl.}{ficar rico; fazer fortuna; acumular fortuna; ganhar muito dinheiro ou propriedades | trabalhar; conseguir um emprego; uma maneira educada de perguntar a alguém onde ele trabalha é dizer onde ele fez fortuna}
  \end{Phonetics}
\end{Entry}

\begin{Entry}{发奋图强}{5,8,8,12}{⼜,⼤,⼞,⼸}
  \begin{Phonetics}{发奋图强}{fa1fen4-tu2qiang2}
    \definition{expr.}{fazer um esforço para se tornar forte (expressão idiomática); determinado a fazer melhor | arregaçar as mangas}
  \end{Phonetics}
\end{Entry}

\begin{Entry}{发放}{5,8}{⼜,⽅}
  \begin{Phonetics}{发放}{fa1fang4}[][HSK 6]
    \definition{v.}{conceder; estender; fornecer; (governo, organização) distribuir dinheiro ou suprimentos para os necessitados | emitir; enviar}
  \end{Phonetics}
\end{Entry}

\begin{Entry}{发明}{5,8}{⼜,⽇}
  \begin{Phonetics}{发明}{fa1ming2}[][HSK 3]
    \definition[个,项,种]{s.}{invenção; novos produtos ou métodos inventados}
    \definition{v.}{inventar; pesquisa que cria (novos produtos ou novos métodos) | expor; explicar; explicação criativa}
  \end{Phonetics}
\end{Entry}

\begin{Entry}{发明者}{5,8,8}{⼜,⽇,⽼}
  \begin{Phonetics}{发明者}{fa1ming2zhe3}
    \definition{s.}{inventor}
  \end{Phonetics}
\end{Entry}

\begin{Entry}{发泄}{5,8}{⼜,⽔}
  \begin{Phonetics}{发泄}{fa1xie4}[][HSK 7-9]
    \definition{v.}{soltar; abreviar; dar vazão a; desabafar emoções ou desejos}
  \end{Phonetics}
\end{Entry}

\begin{Entry}{发炎}{5,8}{⼜,⽕}
  \begin{Phonetics}{发炎}{fa1/yan2}[][HSK 6]
    \definition{s.}{inflamação}
    \definition{v.+compl.}{irritar; inflamar; reação complexa de organismos a fatores patogênicos, como microrganismos, substâncias químicas e estímulos físicos; os sintomas sistêmicos incluem aumento da temperatura corporal, alterações na composição do sangue, vermelhidão local, inchaço, febre, dor, etc.}
  \end{Phonetics}
\end{Entry}

\begin{Entry}{发现}{5,8}{⼜,⾒}
  \begin{Phonetics}{发现}{fa1xian4}[][HSK 2]
    \definition[个,项]{s.}{descoberta; achado}
    \definition{v.}{encontrar; descobrir; detectar; identificar; através de pesquisa, exploração, etc., ver ou encontrar coisas ou leis que os antepassados não viram | descobrir; perceber; perceber; notar; estar ciente de}
  \end{Phonetics}
\end{Entry}

\begin{Entry}{发现者}{5,8,8}{⼜,⾒,⽼}
  \begin{Phonetics}{发现者}{fa1xian4 zhe3}
    \definition{s.}{descobridor}
  \end{Phonetics}
\end{Entry}

\begin{Entry}{发育}{5,8}{⼜,⾁}
  \begin{Phonetics}{发育}{fa1yu4}[][HSK 7-9]
    \definition{s.}{crescimento}
    \definition{v.}{crescer; desenvolver; a estrutura e a função dos organismos evoluem do simples para o complexo ou do imaturo para o maduro}
  \end{Phonetics}
\end{Entry}

\begin{Entry}{发表}{5,8}{⼜,⾐}
  \begin{Phonetics}{发表}{fa1biao3}[][HSK 3]
    \definition{v.}{publicar; entregar; emitir; expressar; anunciar; expressar (opiniões) ou divulgar (assuntos) ao público, verbalmente ou por escrito | publicar em jornais (artigos, etc.)}
  \end{Phonetics}
\end{Entry}

\begin{Entry}{发型}{5,9}{⼜,⼟}
  \begin{Phonetics}{发型}{fa4xing2}[][HSK 7-9]
    \definition[个,种]{s.}{penteado}
  \end{Phonetics}
\end{Entry}

\begin{Entry}{发怒}{5,9}{⼜,⼼}
  \begin{Phonetics}{发怒}{fa1 nu4}[][HSK 6]
    \definition{v.}{ficar com raiva; explodir; perder a paciência | entrar em fúria | entrar em fúria (paixão)}
  \end{Phonetics}
\end{Entry}

\begin{Entry}{发挥}{5,9}{⼜,⼿}
  \begin{Phonetics}{发挥}{fa1hui1}[][HSK 4]
    \definition{v.}{colocar em jogo; dar jogo a; dar espaço a; dar rédea solta a; revelar a natureza ou a capacidade interior | expressar; desenvolver (uma ideia, um tema, etc.); elaborar; fazer valer o ponto ou o motivo}
  \end{Phonetics}
\end{Entry}

\begin{Entry}{发觉}{5,9}{⼜,⾒}
  \begin{Phonetics}{发觉}{fa1jue2}[][HSK 5]
    \definition{v.}{vir a saber; estar ciente (de); perceber; tornar"-se consciente | encontrar; detectar; perceber; descobrir}
  \end{Phonetics}
\end{Entry}

\begin{Entry}{发送}{5,9}{⼜,⾡}
  \begin{Phonetics}{发送}{fa1song5}[][HSK 3]
    \definition{v.}{enviar; despachar | transmitir (rádio)}
  \end{Phonetics}
\end{Entry}

\begin{Entry}{发音}{5,9}{⼜,⾳}
  \begin{Phonetics}{发音}{fa1yin1}
    \definition{s.}{pronúncia}
    \definition{v.}{pronunciar}
  \end{Phonetics}
\end{Entry}

\begin{Entry}{发射}{5,10}{⼜,⼨}
  \begin{Phonetics}{发射}{fa1she4}[][HSK 5]
    \definition{v.}{subir; disparar; lançar; irradiar; projetar; descarregar; enviar algo (como uma bala, um projétil, um satélite, etc.) de um dispositivo em uma velocidade muito alta}
  \end{Phonetics}
\end{Entry}

\begin{Entry}{发展}{5,10}{⼜,⼫}
  \begin{Phonetics}{发展}{fa1zhan3}[][HSK 3]
    \definition{v.}{crescer; expandir; avançar; desenvolver; a mudança das coisas de pequeno para grande, de simples para complexo, de inferior para superior | recrutar; admitir expandir (organização, escala, etc.)}
  \end{Phonetics}
\end{Entry}

\begin{Entry}{发烧}{5,10}{⼜,⽕}
  \begin{Phonetics}{发烧}{fa1shao1}[][HSK 4]
    \definition{v.}{ter febre; a temperatura corporal normal de uma pessoa é de cerca de 37ºC; se exceder 37,5ºC, é febre}
  \end{Phonetics}
\end{Entry}

\begin{Entry}{发热}{5,10}{⼜,⽕}
  \begin{Phonetics}{发热}{fa1/re4}[][HSK 7-9]
    \definition{pref.}{piro-}
    \definition{s.}{ebulição; febre; calor; pirexia}
    \definition{v.+compl.}{emitir calor; gerar calor; aquecer; esquentar | ter febre | ser cabeça quente}
  \end{Phonetics}
\end{Entry}

\begin{Entry}{发病}{5,10}{⼜,⽧}
  \begin{Phonetics}{发病}{fa1 bing4}[][HSK 6]
    \definition{v.}{(de uma doença) avanço | patogênese; morbidade | surto (de uma doença)}
  \end{Phonetics}
\end{Entry}

\begin{Entry}{发起}{5,10}{⼜,⾛}
  \begin{Phonetics}{发起}{fa1qi3}[][HSK 6]
    \definition{s.}{iniciador; patrocinador}
    \definition{v.}{iniciar; patrocinar; começar; lançar}
  \end{Phonetics}
\end{Entry}

\begin{Entry}{发起人}{5,10,2}{⼜,⾛,⼈}
  \begin{Phonetics}{发起人}{fa1qi3ren2}[][HSK 7-9]
    \definition{s.}{iniciador; patrocinador | membro fundador; originadores; autores | propositor}
  \end{Phonetics}
\end{Entry}

\begin{Entry}{发掘}{5,11}{⼜,⼿}
  \begin{Phonetics}{发掘}{fa1jue2}[][HSK 7-9]
    \definition{v.}{explorar; escavar; desenterrar}
  \end{Phonetics}
\end{Entry}

\begin{Entry}{发票}{5,11}{⼜,⽰}
  \begin{Phonetics}{发票}{fa1piao4}[][HSK 4]
    \definition[张,种]{s.}{conta; recibo; fatura; recibos emitidos por lojas ou outros escritórios de cobrança}
  \end{Phonetics}
\end{Entry}

\begin{Entry}{发愣}{5,12}{⼜,⼼}
  \begin{Phonetics}{发愣}{fa1/leng4}[][HSK 7-9]
    \definition{v.+compl.}{olhar fixamente; estar em transe (ou atordoado)}
  \end{Phonetics}
\end{Entry}

\begin{Entry}{发愤图强}{5,12,8,12}{⼜,⼼,⼞,⼸}
  \begin{Phonetics}{发愤图强}{fa1fen4-tu2qiang2}[][HSK 7-9]
    \definition{expr.}{busque a excelência; estar fortemente determinado a ter sucesso; trabalhar com a vontade de fortalecer o país; buscar a perfeição}
  \synonymref{发奋图强}{fa1fen4-tu2qiang2}
  \synonymref{艰苦奋斗}{jian1ku3-fen4dou4}
  \end{Phonetics}
\end{Entry}

\begin{Entry}{发脾气}{5,12,4}{⼜,⾁,⽓}
  \begin{Phonetics}{发脾气}{fa1 pi2qi5}[][HSK 7-9]
    \definition{v.}{ficar com raiva; perder a paciência; ficar furioso; fazer barulho ou xingar porque as coisas não saem do seu jeito}
  \end{Phonetics}
\end{Entry}

\begin{Entry}{发愁}{5,13}{⼜,⼼}
  \begin{Phonetics}{发愁}{fa1/chou2}[][HSK 7-9]
    \definition{v.+compl.}{preocupar"-se; ficar ansioso; ficar triste; sentir"-se deprimido por não ter ideias ou soluções}
  \end{Phonetics}
\end{Entry}

\begin{Entry}{发源地}{5,13,6}{⼜,⽔,⼟}
  \begin{Phonetics}{发源地}{fa1yuan2di4}[][HSK 7-9]
    \definition{s.}{fonte; berço; lar; terra natal; lugar de origem; terra de origem | local de nascimento}[青藏高原是藏族文化的发源地。===O Planalto Qinghai-Tibete é o berço da cultura tibetana.]
  \end{Phonetics}
\end{Entry}

\begin{Entry}{发誓}{5,14}{⼜,⾔}
  \begin{Phonetics}{发誓}{fa1/shi4}[][HSK 7-9]
    \definition{v.+compl.}{jurar; prometer; fazer um juramento; expressar solenemente a resolução e a promessa de fazer o que foi acordado ou dito}
  \end{Phonetics}
\end{Entry}

\begin{Entry}{发酵}{5,14}{⼜,⾣}
  \begin{Phonetics}{发酵}{fa1/jiao4}[][HSK 7-9]
    \definition{s.}{fermentação; zimólise; compostos orgânicos complexos são decompostos em substâncias mais simples sob a ação de microrganismos}
    \definition{v.+compl.}{fermentar}
  \end{Phonetics}
\end{Entry}

\begin{Entry}{发簪}{5,18}{⼜,⽵}
  \begin{Phonetics}{发簪}{fa4zan1}
    \definition{s.}{grampo de cabelo}
  \end{Phonetics}
\end{Entry}

%%%%%%%%%% 叔 %%%%%%%%%%
\subsection*{叔}\addcontentsline{loh}{figure}{叔}

\begin{Entry}{叔}{8}{⼜}
  \begin{Phonetics}{叔}{shu1}
    \definition*{s.}{Sobrenome: Shu}
    \definition{s.}{irmão mais novo do pai; tio (por parte de pai)| irmão mais novo do marido | terceiro entre irmãos | tio | uma forma de tratamento para um homem um pouco mais jovem que o pai; tio | terceiro tio (de quatro irmãos) | primo mais novo da mãe}
  \end{Phonetics}
\end{Entry}

\begin{Entry}{叔叔}{8,8}{⼜,⼜}
  \begin{Phonetics}{叔叔}{shu1shu5}[][HSK 4]
    \definition[个,位,名]{s.}{tio; irmão mais novo do pai | tio, dirigindo"-se a um homem da mesma geração que o pai e mais jovem em idade}
  \end{Phonetics}
\end{Entry}

%%%%%%%%%% 取 %%%%%%%%%%
\subsection*{取}\addcontentsline{loh}{figure}{取}

\begin{Entry}{取}{8}{⼜}
  \begin{Phonetics}{取}{qu3}[][HSK 2]
    \definition{v.}{pegar; obter; buscar; pegar de um lugar; pegar nas mãos | visar; procurar; obter; provocar | adotar; assumir; escolher; selecionar}
  \end{Phonetics}
\end{Entry}

\begin{Entry}{取水}{8,4}{⼜,⽔}
  \begin{Phonetics}{取水}{qu3shui3}
    \definition{v.}{obter água (de um poço, etc.)}
  \end{Phonetics}
\end{Entry}

\begin{Entry}{取代}{8,5}{⼜,⼈}
  \begin{Phonetics}{取代}{qu3dai4}[][HSK 7-9]
    \definition{v.}{deslocar; substituir; suplantar; substituir por; assumir o controle; tomar o lugar de}
  \end{Phonetics}
\end{Entry}

\begin{Entry}{取决于}{8,6,3}{⼜,⼎,⼆}
  \begin{Phonetics}{取决于}{qu3jue2 yu2}[][HSK 7-9]
    \definition{v.}{depender de; ser determinado por (algo)}
  \end{Phonetics}
\end{Entry}

\begin{Entry}{取而代之}{8,6,5,3}{⼜,⽽,⼈,⼂}
  \begin{Phonetics}{取而代之}{qu3'er2dai4zhi1}[][HSK 7-9]
    \definition{expr.}{substituir alguém; suplantar alguém; tomar o lugar de alguém ou de algo; assumir o controle}
  \end{Phonetics}
\end{Entry}

\begin{Entry}{取现}{8,8}{⼜,⾒}
  \begin{Phonetics}{取现}{qu3xian4}
    \definition{v.}{sacar dinheiro}
  \end{Phonetics}
\end{Entry}

\begin{Entry}{取经}{8,8}{⼜,⽷}
  \begin{Phonetics}{取经}{qu3/jing1}[][HSK 7-9]
    \definition{v.+compl.}{fazer uma peregrinação em busca de escrituras budistas | buscar experiência; aprender com a experiência de outra pessoa}
  \end{Phonetics}
\end{Entry}

\begin{Entry}{取胜}{8,9}{⼜,⾁}
  \begin{Phonetics}{取胜}{qu3sheng4}[][HSK 7-9]
    \definition{v.}{obter a vitória; alcançar o sucesso; alcançar a vitória}
  \end{Phonetics}
\end{Entry}

\begin{Entry}{取悦}{8,10}{⼜,⼼}
  \begin{Phonetics}{取悦}{qu3yue4}
    \definition{v.}{tentar agradar}
  \end{Phonetics}
\end{Entry}

\begin{Entry}{取消}{8,10}{⼜,⽔}
  \begin{Phonetics}{取消}{qu3xiao1}[][HSK 3]
    \definition{v.}{cancelar; suspender; anular; abolir; revogar; rescindir; tornar o sistema original, regulamentos, qualificações, direitos, etc. inválidos}
  \end{Phonetics}
\end{Entry}

\begin{Entry}{取笑}{8,10}{⼜,⽵}
  \begin{Phonetics}{取笑}{qu3xiao4}[][HSK 7-9]
    \definition{v.}{ridicularizar; zombar de; fazer alarde de; buscar diversão}
  \end{Phonetics}
\end{Entry}

\begin{Entry}{取得}{8,11}{⼜,⼻}
  \begin{Phonetics}{取得}{qu3de2}[][HSK 2]
    \definition{v.}{ganhar; adquirir; obter; ser o primeiro a conseguir}
  \end{Phonetics}
\end{Entry}

\begin{Entry}{取款}{8,12}{⼜,⽋}
  \begin{Phonetics}{取款}{qu3/kuan3}[][HSK 6]
    \definition{v.+compl.}{sacar dinheiro (de um banco); retirar o dinheiro que você depositou (geralmente se refere a retirar dinheiro do banco)}
  \end{Phonetics}
\end{Entry}

\begin{Entry}{取款机}{8,12,6}{⼜,⽋,⽊}
  \begin{Phonetics}{取款机}{qu3kuan3ji1}[][HSK 6]
    \definition{s.}{ATM; caixa eletrônico; um caixa eletrônico é uma máquina que pode concluir automaticamente operações bancárias, como saques e consultas de saldo}
  \end{Phonetics}
\end{Entry}

\begin{Entry}{取缔}{8,12}{⼜,⽷}
  \begin{Phonetics}{取缔}{qu3di4}[][HSK 7-9]
    \definition{v.}{proibir; criminalizar; suprimir; cancelar, encerrar ou proibir explicitamente}
  \end{Phonetics}
\end{Entry}

\begin{Entry}{取暖}{8,13}{⼜,⽇}
  \begin{Phonetics}{取暖}{qu3nuan3}[][HSK 7-9]
    \definition{v.}{aquecer"-se; utilizar a energia térmica para aquecer o corpo}
  \end{Phonetics}
\end{Entry}

%%%%%%%%%% 受 %%%%%%%%%%
\subsection*{受}\addcontentsline{loh}{figure}{受}

\begin{Entry}{受}{8}{⼜}
  \begin{Phonetics}{受}{shou4}[][HSK 3]
    \definition{v.}{receber; aceitar | sofrer; ser submetido a | aguentar; suportar; tolerar | ser agradável}
  \end{Phonetics}
\end{Entry}

\begin{Entry}{受不了}{8,4,2}{⼜,⼀,⼅}
  \begin{Phonetics}{受不了}{shou4bu5liao3}[][HSK 4]
    \definition{v.}{ser insuportável; não poder suportar algo; não suportar algo}
  \end{Phonetics}
\end{Entry}

\begin{Entry}{受伤}{8,6}{⼜,⼈}
  \begin{Phonetics}{受伤}{shou4/shang1}[][HSK 3]
    \definition{v.+compl.}{ser ferido; sofrer uma lesão}
  \end{Phonetics}
\end{Entry}

\begin{Entry}{受过}{8,6}{⼜,⾡}
  \begin{Phonetics}{受过}{shou4/guo4}[][HSK 7-9]
    \definition{v.+compl.}{assumir a culpa (por outra pessoa) | sofrer}
  \end{Phonetics}
\end{Entry}

\begin{Entry}{受灾}{8,7}{⼜,⽕}
  \begin{Phonetics}{受灾}{shou4 zai1}[][HSK 5]
    \definition{v.}{ser atingido por um desastre natural (ou calamidade) | ser atingido por uma adversidade natural}
  \end{Phonetics}
\end{Entry}

\begin{Entry}{受到}{8,8}{⼜,⼑}
  \begin{Phonetics}{受到}{shou4dao4}[][HSK 2]
    \definition{v.}{receber; receber itens, mensagens, instruções, etc. fornecidos por outras pessoas}
  \end{Phonetics}
\end{Entry}

\begin{Entry}{受苦}{8,8}{⼜,⾋}
  \begin{Phonetics}{受苦}{shou4/ku3}[][HSK 7-9]
    \definition{v.+compl.}{sofrer (dificuldades); passar por momentos difíceis}
  \synonymref{吃苦}{chi1/ku3}
  \synonymref{刻苦}{ke4ku3}
  \antonymref{舒服}{shu1fu5}
  \end{Phonetics}
\end{Entry}

\begin{Entry}{受限}{8,8}{⼜,⾩}
  \begin{Phonetics}{受限}{shou4xian4}
    \definition{v.}{ser limitado | ser restrito | ser constrangido}
  \end{Phonetics}
\end{Entry}

\begin{Entry}{受害}{8,10}{⼜,⼧}
  \begin{Phonetics}{受害}{shou4/hai4}[][HSK 7-9]
    \definition{v.+compl.}{sofrer lesão; ser vítima de queda; ser afetado por | ser afetado; ser afligido}
  \synonymref{受伤}{shou4/shang1}
  \antonymref{受益}{shou4yi4}
  \end{Phonetics}
\end{Entry}

\begin{Entry}{受害人}{8,10,2}{⼜,⼧,⼈}
  \begin{Phonetics}{受害人}{shou4hai4ren2}[][HSK 7-9]
    \definition{s.}{vítima | sofredor}
  \end{Phonetics}
\end{Entry}

\begin{Entry}{受益}{8,10}{⼜,⽫}
  \begin{Phonetics}{受益}{shou4yi4}[][HSK 7-9]
    \definition{v.}{lucrar com; beneficiar"-se de; ser beneficiado; receber benefícios; obter vantagens}
  \synonymref{利益}{li4yi4}
  \synonymref{收益}{shou1yi4}
  \antonymref{受害}{shou4/hai4}
  \end{Phonetics}
\end{Entry}

\begin{Entry}{受贿}{8,10}{⼜,⾙}
  \begin{Phonetics}{受贿}{shou4/hui4}[][HSK 7-9]
    \definition{v.+compl.}{aceitar (ou receber) subornos}
  \end{Phonetics}
\end{Entry}

\begin{Entry}{受得了}{8,11,2}{⼜,⼻,⼅}
  \begin{Phonetics}{受得了}{shou4de5liao3}
    \definition{v.}{suportar | aguentar}
  \end{Phonetics}
\end{Entry}

\begin{Entry}{受惊}{8,11}{⼜,⼼}
  \begin{Phonetics}{受惊}{shou4/jing1}[][HSK 7-9]
    \definition{adj.}{assustado; sobressaltado; medo causado por estímulos ou ameaças repentinas}
    \definition{v.+compl.}{ficar assustado; levar um susto}
  \synonymref{吃惊}{chi1/jing1}
  \antonymref{坦然}{tan3ran2}
  \end{Phonetics}
\end{Entry}

\begin{Entry}{受理}{8,11}{⼜,⽟}
  \begin{Phonetics}{受理}{shou4li3}[][HSK 7-9]
    \definition{v.}{aceitar e tratar um caso (autoridades judiciais) | aceitar e lidar com; aceitar e processar}
  \synonymref{办理}{ban4li3}
  \synonymref{承办}{cheng2ban4}
  \synonymref{处理}{chu3li3}
  \synonymref{处置}{chu3zhi4}
  \antonymref{驳回}{bo2hui2}
  \end{Phonetics}
\end{Entry}

\begin{Entry}{受骗}{8,12}{⼜,⾺}
  \begin{Phonetics}{受骗}{shou4/pian4}[][HSK 7-9]
    \definition{v.+compl.}{ser enganado; ser iludido; ser ludibriado}
  \synonymref{上当}{shang4/dang4}
  \end{Phonetics}
\end{Entry}

%%%%%%%%%% 变 %%%%%%%%%%
\subsection*{变}\addcontentsline{loh}{figure}{变}

\begin{Entry}{变}{8}{⼜}
  \begin{Phonetics}{变}{bian4}[][HSK 2]
    \definition{adj.}{alterado; mutável; que pode mudar; que está mudando ou já mudou}
    \definition{s.}{uma reviravolta inesperada nos acontecimentos; mudanças significativas repentinas}
    \definition{v.}{mudar; tornar"-se diferente; fazer mudanças | tornar"-se; transformar"-se; natureza, estado ou situação diferentes dos originais | alterar; mudar; transformar}
  \end{Phonetics}
\end{Entry}

\begin{Entry}{变为}{8,4}{⼜,⼂}
  \begin{Phonetics}{变为}{bian4wei2}[][HSK 3]
    \definition{v.}{transformar"-se em; tornar"-se | mudar para}
  \end{Phonetics}
\end{Entry}

\begin{Entry}{变化}{8,4}{⼜,⼔}
  \begin{Phonetics}{变化}{bian4hua4}[][HSK 3]
    \definition[个]{s.}{mudança; variação; a nova situação após uma mudança em pessoas ou coisas}
    \definition{v.}{mudar;  variar}
  \end{Phonetics}
\end{Entry}

\begin{Entry}{变幻莫测}{8,4,10,9}{⼜,⼳,⾋,⽔}
  \begin{Phonetics}{变幻莫测}{bian4huan4-mo4ce4}[][HSK 7-9]
    \definition{expr.}{mutável; imprevisível | errático | mudar imprevisivelmente | traiçoeiro}
  \end{Phonetics}
\end{Entry}

\begin{Entry}{变心}{8,4}{⼜,⼼}
  \begin{Phonetics}{变心}{bian4/xin1}
    \definition{v.+compl.}{deixar de ser fiel}
  \end{Phonetics}
\end{Entry}

\begin{Entry}{变节}{8,5}{⼜,⾋}
  \begin{Phonetics}{变节}{bian4jie2}
    \definition{s.}{traição | deserção | vira"-casaca}
    \definition{v.}{retratar"-se e submeter"-se; renunciar e render"-se | mudar de lado politicamente}
  \end{Phonetics}
\end{Entry}

\begin{Entry}{变动}{8,6}{⼜,⼒}
  \begin{Phonetics}{变动}{bian4dong4}[][HSK 5]
    \definition{s.}{mudança; alteração; oscilação; modificação; variação}
    \definition{v.}{mudar; alterar; oscilar; modificar; variar}
  \end{Phonetics}
\end{Entry}

\begin{Entry}{变异}{8,6}{⼜,⼶}
  \begin{Phonetics}{变异}{bian4yi4}[][HSK 7-9]
    \definition{s.}{variação; mutação; muta; diferenças nas características morfológicas e fisiológicas entre gerações da mesma espécie ou entre indivíduos da mesma geração}
    \definition{v.}{variar; mudar}
  \end{Phonetics}
\end{Entry}

\begin{Entry}{变成}{8,6}{⼜,⼽}
  \begin{Phonetics}{变成}{bian4 cheng2}[][HSK 2]
    \definition{v.}{crescer; tornar"-se; fazer; desenvolver"-se; revelar"-se; resultar; acontecer; passar a ser; passar para; acumular"-se; converter"-se; transformar"-se; transformar"-se em; mudar"-se em; transformação da situação ou condição anterior para a situação ou condição atual}
  \end{Phonetics}
\end{Entry}

\begin{Entry}{变迁}{8,6}{⼜,⾡}
  \begin{Phonetics}{变迁}{bian4qian1}[][HSK 7-9]
    \definition{s.}{mudanças; transição; vicissitudes; mudança em tendências ou condições; mudança de situação ou estágio}
  \end{Phonetics}
\end{Entry}

\begin{Entry}{变形}{8,7}{⼜,⼺}
  \begin{Phonetics}{变形}{bian4/xing2}[][HSK 6]
    \definition{v.+compl.}{deformar; ficar fora de forma | transformar; transformar"-se em outras formas}
  \end{Phonetics}
\end{Entry}

\begin{Entry}{变更}{8,7}{⼜,⽈}
  \begin{Phonetics}{变更}{bian4geng1}[][HSK 6]
    \definition{v.}{alterar; mudar; modificar}
  \end{Phonetics}
\end{Entry}

\begin{Entry}{变性}{8,8}{⼜,⼼}
  \begin{Phonetics}{变性}{bian4xing4}
    \definition{s.}{desnaturação | transexual}
    \definition{v.}{desnaturar | mudar de sexo}
  \end{Phonetics}
\end{Entry}

\begin{Entry}{变质}{8,8}{⼜,⾙}
  \begin{Phonetics}{变质}{bian4/zhi4}[][HSK 7-9]
    \definition{v.+compl.}{deteriorar"-se; estragar"-se | tornar"-se moralmente degenerado}
  \end{Phonetics}
\end{Entry}

\begin{Entry}{变革}{8,9}{⼜,⾰}
  \begin{Phonetics}{变革}{bian4ge2}[][HSK 7-9]
    \definition{s.}{mudança; transformação; a natureza das coisas foi reformada}
    \definition{v.}{transformar; mudar (de sistemas sociais, políticas, etc.); mudar o antigo e inovar; mudar a essência das coisas (referindo"-se principalmente aos sistemas sociais)}
  \end{Phonetics}
\end{Entry}

\begin{Entry}{变换}{8,10}{⼜,⼿}
  \begin{Phonetics}{变换}{bian4huan4}[][HSK 6]
    \definition{v.}{variar; alternar; mudar a forma ou o conteúdo de algo de uma coisa para outra}
  \end{Phonetics}
\end{Entry}

\begin{Entry}{变装}{8,12}{⼜,⾐}
  \begin{Phonetics}{变装}{bian4zhuang1}
    \definition{v.}{trocar de roupa | vestir"-se | vestir uma fantasia | disfarçar"-se ou fantasiar"-se de personagem real ou ficcional, \emph{cosplay} | travestir"-se}
  \end{Phonetics}
\end{Entry}

\begin{Entry}{变数}{8,13}{⼜,⽁}
  \begin{Phonetics}{变数}{bian4shu4}
    \definition{s.}{(matemática) variável | fatores variáveis}
  \end{Phonetics}
\end{Entry}

%%%%%%%%%% 叛 %%%%%%%%%%
\subsection*{叛}\addcontentsline{loh}{figure}{叛}

\begin{Entry}{叛}{9}{⼜}
  \begin{Phonetics}{叛}{pan4}
    \definition{adj.}{rebelde}
    \definition{s.}{rebelião}
    \definition{v.}{trair | rebelar"-se | revoltar"-se}
  \end{Phonetics}
\end{Entry}

\begin{Entry}{叛逆}{9,9}{⼜,⾡}
  \begin{Phonetics}{叛逆}{pan4ni4}[][HSK 7-9]
    \definition{s.}{rebelde; pessoas que traem}
    \definition{v.}{rebelar"-se/revoltar"-se contra; trair}
  \end{Phonetics}
\end{Entry}

%%%%%%%%%% 叠 %%%%%%%%%%
\subsection*{叠}\addcontentsline{loh}{figure}{叠}

\begin{Entry}{叠}{13}{⼜}
  \begin{Phonetics}{叠}{die2}[][HSK 7-9]
    \definition{clas.}{maço; pacote; pilha}
    \definition{v.}{acumular; empilhar | dobrar}
  \end{Phonetics}
\end{Entry}

%%%%% EOF %%%%%

